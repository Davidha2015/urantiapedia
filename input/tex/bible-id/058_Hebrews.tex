\begin{document}

\title{Ibrani}


\chapter{1}

\par 1 Pada zaman dahulu banyak kali Allah berbicara kepada nenek moyang kita melalui nabi-nabi dengan memakai bermacam-macam cara.
\par 2 Tetapi pada zaman akhir ini Ia berbicara kepada kita dengan perantaraan Anak-Nya. Melalui Anak-Nya inilah Allah menciptakan alam semesta. Dan Allah sudah menentukan bahwa Anak-Nya inilah juga yang berhak memiliki segala sesuatu.
\par 3 Dialah yang memancarkan keagungan Allah yang gilang-gemilang; Dialah gambar yang nyata dari diri Allah sendiri. Dialah juga yang memelihara keutuhan alam semesta ini dengan sabda-Nya yang sangat berkuasa. Sesudah Ia memungkinkan manusia untuk dibebaskan dari dosa-dosa mereka, Ia menduduki takhta pemerintahan di surga bersama-sama dengan Allah, Penguasa yang tertinggi.
\par 4 Anak itu mendapat kedudukan yang jauh lebih tinggi dari malaikat, dan nama yang diberikan Allah kepada-Nya juga jauh lebih terhormat daripada nama yang diberikan kepada malaikat.
\par 5 Tidak pernah Allah berkata begini kepada seorang malaikat pun, "Engkaulah Anak-Ku; pada hari ini Aku menjadi Bapa-Mu." Tidak pernah pula Allah berkata begini mengenai malaikat yang manapun juga, "Aku akan menjadi Bapa-Nya, dan Ia akan menjadi Anak-Ku."
\par 6 Tetapi pada saat Allah mengutus Anak-Nya yang sulung ke dunia ini, Allah berkata begini, "Semua malaikat Allah wajib menyembah Anak itu."
\par 7 Mengenai malaikat-malaikat, Allah berkata begini, "Allah membuat malaikat-malaikat-Nya menjadi angin, dan pelayan-pelayan-Nya menjadi nyala api."
\par 8 Tetapi kepada Anak itu, Allah berkata, "Takhta-Mu, ya Allah, akan kekal selama-lamanya! Pemerintahan-Mu adalah pemerintahan yang adil.
\par 9 Engkau suka akan keadilan, dan benci akan kecurangan; itulah sebabnya Allah, Allah-Mu, memilih Engkau dan memberi kepada-Mu kehormatan yang mendatangkan sukacita, melebihi teman-teman-Mu."
\par 10 Allah berkata juga, "Engkau, Tuhan, pada mulanya menciptakan bumi, dan Engkau sendiri membuat langit.
\par 11 Semuanya itu akan lenyap, dan menjadi tua seperti pakaian; tetapi Engkau tidak akan berubah.
\par 12 Alam semesta ini akan Kaulipat seperti baju, dan akan Kauganti dengan yang lain. Tetapi Engkau tidak pernah akan berubah, dan hidup-Mu tidak akan berakhir."
\par 13 Allah tidak pernah berkata begini kepada seorang malaikat pun, "Duduklah di sebelah kanan-Ku, sampai musuh-musuh-Mu Kutaklukkan kepada-Mu."
\par 14 Kalau begitu, malaikat-malaikat itu apa sebenarnya? Mereka adalah roh-roh yang melayani Allah, dan yang disuruh Allah untuk menolong orang-orang yang akan menerima keselamatan.

\chapter{2}

\par 1 Itulah sebabnya kita harus lebih sungguh-sungguh berpegang pada ajaran-ajaran yang sudah kita dengar, supaya kita jangan meninggalkan kepercayaan kita.
\par 2 Pesan-pesan yang disampaikan oleh para malaikat kepada nenek moyang kita ternyata benar, dan orang yang tidak menuruti atau mentaatinya menerima hukuman yang setimpal.
\par 3 Apalagi dengan keselamatan yang lebih hebat dari pesan-pesan itu! Kalau kita tidak menaruh perhatian terhadap keselamatan itu, kita tentu tidak akan terlepas dari hukuman. Sebab Tuhan sendirilah yang pertama-tama mengumumkan keselamatan itu, dan orang-orang yang mula-mula mendengarnya telah membuktikan kebenarannya kepada kita.
\par 4 Di samping itu, Allah turut menguatkan kesaksian orang-orang itu dengan mengadakan segala macam keajaiban dan hal-hal luar biasa serta membagi-bagikan pemberian-pemberian dari Roh Allah menurut kemauan-Nya sendiri.
\par 5 Malaikat-malaikat tidak mendapat kuasa dari Allah untuk memerintah dunia baru yang akan datang, yaitu dunia yang sedang kita bicarakan ini.
\par 6 Sebaliknya, pada suatu bagian di dalam Alkitab tertulis begini, "Manusia itu apa, ya Allah, sehingga Engkau mau mengingatnya? Manusia hanya manusia saja, namun Engkau memperhatikannya!
\par 7 Untuk waktu yang singkat Engkau menjadikan dia sedikit lebih rendah daripada malaikat. Engkau memberikan kepadanya kedudukan yang mulia dan terhormat
\par 8 serta menjadikan dia penguasa atas segala sesuatu." Nah, kalau dikatakan bahwa Allah menjadikan manusia "penguasa atas segala sesuatu", itu berarti bahwa tidak ada sesuatu pun yang tidak di bawah kekuasaan manusia. Meskipun begitu, kita tidak melihat sekarang manusia berkuasa atas segala sesuatu.
\par 9 Tetapi kita sudah melihat Yesus berkuasa! Ia dijadikan sedikit lebih rendah daripada malaikat untuk waktu yang singkat, supaya atas kebaikan hati Allah Ia dapat mati untuk seluruh umat manusia. Kita melihat Dia sekarang diberikan kedudukan yang mulia dan terhormat karena Ia sudah menderita sampai mati.
\par 10 Memang sudah sepatutnya Allah--yang menciptakan segala sesuatu untuk maksud-Nya sendiri--membuat Yesus penyelamat yang sempurna melalui penderitaan. Dengan itu Allah dapat mengajak banyak orang untuk turut diagungkan bersama Yesus. Sebab Dialah pembuka jalan bagi mereka untuk bisa diselamatkan.
\par 11 Yesus membersihkan manusia dari dosa-dosa mereka; dan Dia yang membersihkan, serta mereka yang dibersihkan itu, sama-sama mempunyai satu Bapa. Itulah sebabnya Yesus tidak malu mengaku mereka itu sebagai saudara-saudara-Nya.
\par 12 Yesus berkata kepada Allah, "Aku akan memberitakan kepada saudara-saudara-Ku tentang Engkau. Aku akan memuji Engkau di dalam pertemuan mereka."
\par 13 Yesus berkata juga, "Aku akan menaruh harapan-Ku kepada Allah." Dan Ia berkata juga, "Inilah Aku, bersama anak-anak yang sudah diberikan Allah kepada-Ku."
\par 14 Oleh sebab orang-orang yang Ia sebut anak itu, adalah makhluk manusia yang dapat mati, maka Yesus sendiri menjadi sama dengan mereka dan hidup dalam keadaan manusia. Ia berbuat begitu, supaya dengan kematian-Nya Ia dapat menghancurkan Iblis yang menguasai kematian.
\par 15 Dengan cara itu Ia membebaskan orang-orang yang seumur hidup diperbudak karena takut kepada kematian.
\par 16 Nyatalah bahwa bukan malaikat yang ditolong-Nya, melainkan keturunan Abraham.
\par 17 Ini berarti bahwa Ia harus menjadi sama dengan saudara-saudara-Nya dalam segala hal. Dan dengan itu Ia dapat menjadi Imam Agung yang setia dan berbelaskasihan. Dengan pelayanan-Nya itu dosa manusia dapat diampuni.
\par 18 Dan karena Ia sendiri pernah menderita dan dicobai, Ia dapat menolong orang-orang yang terkena cobaan, sebab Ia sendiri pernah dicobai dan menderita.

\chapter{3}

\par 1 Saudara-saudara sesama Kristen yang sudah dipanggil juga oleh Allah! Coba pikirkan dalam-dalam mengenai Yesus ini! Allah mengutus Dia khusus untuk menjadi Imam Agung dalam agama yang kita anut.
\par 2 Ia setia kepada Allah yang sudah memilih Dia untuk pekerjaan itu, seperti juga Musa dahulu setia menunaikan tugasnya di Rumah Allah.
\par 3 Tetapi Yesus patut mendapat kehormatan yang lebih besar daripada Musa. Sebab orang yang membangun rumah harus mendapat kehormatan lebih besar daripada rumah itu sendiri.
\par 4 Memang setiap rumah ada yang membangunnya, tetapi yang membangun segala sesuatu adalah Allah sendiri.
\par 5 Musa setia sebagai pelayan di dalam Rumah Allah dan menyampaikan hal-hal yang akan diberitahukan oleh Allah pada masa yang akan datang.
\par 6 Tetapi Kristus setia sebagai Anak yang bertanggung jawab atas Rumah Allah. Dan kita inilah Rumah Allah, kalau kita tetap bersemangat dan tetap yakin untuk mendapat apa yang kita harapkan.
\par 7 Sebab itu, seperti kata Roh Allah, "Kalau pada hari ini kamu mendengar suara Allah,
\par 8 janganlah kamu berkeras kepala, seperti leluhurmu, ketika mereka memberontak terhadap Allah, dan menguji Allah di padang pasir.
\par 9 'Di sana leluhurmu mencobai Aku, dan menguji Aku,' kata Allah, 'padahal mereka sudah melihat perbuatan-Ku empat puluh tahun lamanya.'
\par 10 Itulah sebabnya Aku murka terhadap mereka dan Aku berkata, 'Mereka selalu tidak setia, dan enggan mentaati perintah-perintah-Ku.'
\par 11 Aku marah dan bersumpah, 'Mereka tak akan masuk ke negeri itu untuk mendapat istirahat bersama Aku.'"
\par 12 Saudara-saudara, hati-hatilah jangan sampai ada di antaramu seorang yang hatinya begitu jahat dan tidak beriman, sehingga ia berbalik dan menjauhi Allah yang hidup!
\par 13 Jadi, supaya tidak ada seorang pun dari antaramu yang tertipu oleh dosa sehingga menentang Allah, hendaklah kalian saling menasihati setiap hari selama kita masih hidup dalam zaman yang disebut "Hari Ini" dalam Alkitab.
\par 14 Kita semua adalah teman seperjuangan dengan Kristus, asal kita sampai akhir memegang teguh keyakinan yang kita miliki pada mulanya.
\par 15 Inilah yang tertulis dalam Alkitab, "Kalau pada hari ini kamu mendengar suara Allah, janganlah kamu berkeras kepala, seperti leluhurmu, ketika mereka memberontak terhadap Allah."
\par 16 Nah, siapakah orang-orang yang mendengar suara Allah lalu memberontak terhadap-Nya? Itulah semua orang yang dipimpin oleh Musa keluar dari Mesir, bukan?
\par 17 Dan terhadap siapakah Allah marah empat puluh tahun lamanya? Terhadap mereka yang berdosa, yang jatuh mati di padang gurun, bukan?
\par 18 Ketika Allah bersumpah begini, "Mereka tidak akan masuk ke negeri itu untuk mendapat istirahat bersama-Ku" --siapakah pula yang dimaksudkan oleh Allah? Yang dimaksudkan ialah orang-orang yang memberontak terhadap Allah, bukan?
\par 19 Sekarang kita mengerti bahwa mereka tidak dapat masuk, karena mereka tidak percaya kepada Allah.

\chapter{4}

\par 1 Nah, janji Allah masih berlaku, dan kita akan diberi istirahat yang dijanjikan-Nya. Jadi, hendaklah kita menjaga supaya jangan ada seorang pun dari antara kalian yang ternyata tidak menikmati istirahat yang dijanjikan itu.
\par 2 Sebab Kabar Baik itu sudah diberitakan kepada kita sama seperti kepada mereka. Namun kepada mereka, berita itu tidak ada gunanya. Sebab ketika mereka mendengarnya, mereka tidak percaya.
\par 3 Tetapi kita yang percaya menerima istirahat yang dijanjikan Allah itu. Itu cocok dengan apa yang dikatakan Allah, begini, "Aku marah dan bersumpah, 'Mereka tak akan masuk untuk mendapat istirahat bersama Aku.'" Allah berkata begitu meskipun pekerjaan-Nya sudah selesai semenjak Ia menciptakan dunia ini.
\par 4 Sebab tentang hari ketujuh, ada tertulis dalam Alkitab begini, "Pada hari ketujuh Allah beristirahat dari semua pekerjaan-Nya."
\par 5 Mengenai hal itu ditulis lagi begini, "Mereka tidak akan masuk ke negeri itu untuk mendapat istirahat bersama Aku."
\par 6 Jadi mereka yang pertama-tama mendengar berita tentang Kabar Baik itu, tidak menerima istirahat itu karena mereka tidak percaya. Kalau begitu sudah jelas bahwa ada orang-orang lain yang boleh menerima istirahat itu.
\par 7 Sebab Allah sudah menentukan suatu hari yang lain lagi, yang disebut "Hari Ini". Bertahun-tahun kemudian Allah berbicara mengenai hal itu melalui Daud, dalam ayat yang dikutip tadi, "Kalau pada hari ini kamu mendengar suara Allah, janganlah kamu berkeras kepala."
\par 8 Kalau seandainya Yosua sudah memberi orang-orang itu istirahat yang dijanjikan Allah, maka Allah tidak akan berbicara lagi tentang suatu hari yang lain.
\par 9 Jadi, bagi umat Allah masih ada janji untuk beristirahat seperti Allah beristirahat pada hari yang ketujuh itu.
\par 10 Karena orang yang menerima istirahat yang dijanjikan Allah kepadanya itu, akan beristirahat juga dari semua pekerjaannya, sama seperti Allah.
\par 11 Sebab itu, marilah kita berusaha sungguh-sungguh untuk menerima istirahat yang dijanjikan Allah itu. Jangan sampai seorang pun dari kita gagal seperti mereka dahulu, karena tidak percaya kepada Allah.
\par 12 Perkataan Allah adalah perkataan yang hidup dan kuat; lebih tajam dari pedang bermata dua. Perkataan itu menusuk sampai ke batas antara jiwa dan roh; sampai ke batas antara sendi-sendi dan tulang sumsum, sehingga mengetahui sedalam-dalamnya pikiran dan niat hati manusia.
\par 13 Tidak ada satu makhluk pun yang tersembunyi dari pandangan Allah. Segala sesuatu telanjang dan terbuka di depan-Nya. Dan kita harus memberi pertanggungjawaban kepada-Nya.
\par 14 Itulah sebabnya kita harus berpegang teguh pada pengakuan kepercayaan kita. Sebab kita mempunyai Imam Agung yang besar, yang sudah masuk sampai ke depan Allah--Dialah Yesus Anak Allah.
\par 15 Imam Agung kita itu bukanlah imam yang tidak dapat turut merasakan kelemahan-kelemahan kita. Sebaliknya, Ia sudah dicobai dalam segala hal, sama seperti kita sendiri; hanya Ia tidak berbuat dosa!
\par 16 Sebab itu, marilah kita dengan penuh keberanian menghadap Allah yang memerintah dengan baik hati. Allah akan mengasihani kita dan memberkati kita supaya kita mendapat pertolongan tepat pada waktunya.

\chapter{5}

\par 1 Setiap imam agung dipilih dari antara umat, dan diangkat untuk melayani Allah sebagai wakil mereka. Tugasnya ialah mempersembahkan kepada Allah pemberian-pemberian dan kurban-kurban untuk pengampunan dosa.
\par 2 Imam agung itu sendiri lemah dalam banyak hal, dan karena itu ia dapat berlaku lemah lembut terhadap orang-orang yang tidak tahu apa-apa dan yang sesat jalannya.
\par 3 Dan karena ia sendiri lemah, maka ia harus mempersembahkan kurban, bukan saja karena dosa-dosa umat, tetapi juga karena dosa-dosanya sendiri.
\par 4 Tidak ada seorang pun yang mengangkat dirinya sendiri menjadi imam agung. Orang menjadi imam agung, kalau Allah memanggil dia untuk itu--sama seperti Harun.
\par 5 Begitu juga Kristus. Ia tidak mengangkat diri sendiri menjadi Imam Agung. Allah sendirilah yang mengangkat Dia. Allah berkata kepada-Nya, "Engkaulah Anak-Ku; pada hari ini Aku menjadi Bapa-Mu."
\par 6 Di tempat lain Allah berkata juga, "Engkau adalah Imam selama-lamanya, seperti Imam Melkisedek."
\par 7 Pada masa Yesus hidup di dunia ini, Ia berdoa dan memohon dengan teriakan dan tangis kepada Allah, yang sanggup menyelamatkan-Nya dari kematian. Dan karena Ia tunduk kepada Allah dengan penuh hormat, maka Ia didengarkan.
\par 8 Yesus adalah Anak Allah, tetapi meskipun begitu, Ia belajar menjadi taat melalui penderitaan-Nya.
\par 9 Maka sesudah Ia dijadikan penyelamat yang sempurna, Ia menjadi sumber keselamatan yang kekal bagi semua orang yang taat kepada-Nya,
\par 10 dan Allah pun menyatakan Dia sebagai Imam Agung, seperti Imam Melkisedek!
\par 11 Mengenai kedudukan Yesus sebagai Imam Agung, ada banyak yang perlu kami beritahukan kepadamu, tetapi sukar untuk menerangkannya sebab kalian lambat sekali mengerti.
\par 12 Sebenarnya sudah waktunya bagimu untuk menjadi guru, tetapi nyatanya kalian masih perlu belajar dari orang lain tentang asas-asas pertama ajaran Allah. Kalian belum dapat menerima makanan yang keras; kalian masih harus minum susu saja.
\par 13 Orang yang masih minum susu, berarti ia masih bayi; ia belum punya pengalaman tentang apa yang benar dan apa yang salah.
\par 14 Makanan yang keras adalah untuk orang dewasa yang karena pengalaman, sudah dapat membeda-bedakan mana yang baik dan mana yang jahat.

\chapter{6}

\par 1 Sebab itu, marilah kita maju ke pelajaran-pelajaran yang lebih lanjut tentang kedewasaan kehidupan Kristen, dan jangan hanya memperhatikan asas-asas pertama ajaran agama kita. Jangan kita mengulangi lagi pelajaran dasar bahwa orang harus berhenti melakukan hal-hal yang tidak berguna dan harus percaya kepada Allah,
\par 2 atau pelajaran dasar mengenai pembaptisan atau mengenai meletakkan tangan atas orang, atau mengenai hidup kembali sesudah mati atau hukuman yang kekal.
\par 3 Marilah kita, jika Allah mengizinkan, maju ke pelajaran-pelajaran yang lebih lanjut!
\par 4 Sebab, orang-orang yang sudah murtad, tidak mungkin dibimbing kembali. Mereka dahulu sudah berada di dalam terang dari Allah, dan sudah menikmati pemberian-Nya. Mereka sudah juga turut dikuasai oleh Roh Allah.
\par 5 Mereka tahu dari pengalaman mereka bahwa perkataan Allah itu baik, dan mereka sudah merasai karunia-karunia yang penuh kuasa dari zaman yang akan datang.
\par 6 Tetapi sesudah itu mereka murtad! Mana mungkin membimbing mereka kembali untuk bertobat lagi dari dosa-dosa mereka. Sebab mereka sedang menyalibkan lagi Anak Allah dan membuat Dia dihina di depan umum.
\par 7 Tanah yang menghisap air hujan yang sering turun ke atasnya, lalu menghasilkan tanaman-tanaman yang berguna untuk orang yang menggarapnya, tanah itu diberkati oleh Allah.
\par 8 Tetapi kalau tanah itu menghasilkan alang-alang dan tumbuhan berduri, maka tanah itu tidak ada gunanya dan nanti akan dikutuk oleh Allah dan akhirnya dibakar habis.
\par 9 Tetapi Saudara-saudara yang tercinta, sekalipun kami berkata begitu, namun kami yakin mengenai kalian. Kami yakin bahwa kalian telah menerima hal-hal yang lebih baik, yaitu berkat-berkat yang merupakan bagian dari keselamatanmu.
\par 10 Allah bukannya Allah yang tidak adil. Ia tidak melupakan apa yang kalian kerjakan bagi-Nya, dan kasih yang kalian tunjukkan kepada-Nya sewaktu menolong saudara-saudara seiman, dahulu dan sekarang.
\par 11 Kami ingin sekali supaya kalian masing-masing terus bersemangat sampai akhir, sehingga kalian menerima apa yang kalian harap-harapkan itu.
\par 12 Kami tidak mau kalian menjadi malas. Tetapi kami ingin supaya kalian hidup seperti orang-orang yang menerima apa yang dijanjikan Allah, karena percaya kepada-Nya dan karena menunggu dengan sabar.
\par 13 Ketika Allah berjanji kepada Abraham, Ia bersumpah bahwa Ia akan melakukan apa yang telah dijanjikan-Nya. Dan karena tidak ada yang lebih tinggi daripada Allah, maka Ia bersumpah atas nama-Nya sendiri.
\par 14 Ia berkata, "Aku berjanji akan memberkati engkau dan membuat keturunanmu menjadi banyak."
\par 15 Abraham menunggu dengan sabar, maka ia menerima apa yang dijanjikan Allah kepadanya.
\par 16 Kalau orang bersumpah, ia bersumpah atas nama orang lain yang lebih tinggi daripadanya, maka sumpah itu mengakhiri segala bantahan.
\par 17 Allah mau menegaskan kepada orang-orang yang menerima janji-Nya, bahwa Ia tidak akan merubah rencana-Nya. Itulah sebabnya Ia menambah sumpah pada janji-Nya itu.
\par 18 Dua hal itu tidak dapat berubah: Allah tidak mungkin berdusta mengenai janji dan sumpah-Nya. Sebab itu, kita yang sudah berlindung pada Allah, diberi dorongan kuat untuk berpegang teguh pada harapan yang terbentang di depan kita.
\par 19 Harapan kita itu seperti jangkar yang tertanam sangat dalam dan merupakan pegangan yang kuat dan aman bagi hidup kita. Harapan itu menembus gorden Ruang Mahasuci di Rumah Tuhan di surga.
\par 20 Yesus sudah merintis jalan ke tempat itu untuk kita, dan sudah masuk ke sana menjadi Imam Agung kita untuk selama-lamanya, seperti Imam Melkisedek.

\chapter{7}

\par 1 Melkisedek ini adalah raja dari Salem dan imam Allah Yang Mahatinggi. Waktu Abraham kembali dari pertempuran mengalahkan raja-raja, Melkisedek datang menyambut dia dan memberkatinya.
\par 2 Maka Abraham memberikan kepada Melkisedek sepersepuluh dari segala apa yang direbutnya. (Nama Melkisedek berarti, pertama-tama, "Raja Keadilan"; dan karena ia raja dari Salem, maka namanya berarti juga "Raja Sejahtera".)
\par 3 Mengenai Melkisedek ini tidak ada keterangan di mana pun bahwa ia mempunyai bapak atau ibu atau nenek moyang; tidak ada juga keterangan tentang kelahirannya, ataupun kematiannya. Ia sama seperti Anak Allah; ia adalah imam yang abadi.
\par 4 Jadi, kita lihat di sini betapa besarnya Melkisedek ini; ia begitu besar sehingga Abraham, bapak leluhur bangsa kita, memberikan kepadanya sepersepuluh dari segala sesuatu yang didapatnya dari pertempuran itu.
\par 5 Di dalam hukum agama Yahudi ditentukan bahwa imam-imam keturunan Lewi harus memungut sepersepuluh dari pendapatan umat Israel, yakni saudara-saudara mereka sendiri, walaupun mereka sama-sama keturunan Abraham.
\par 6 Melkisedek bukan keturunan Lewi, namun ia menerima juga sepersepuluh dari segala yang direbut Abraham, dan memberkati pula Abraham yang justru sudah menerima janji-janji dari Allah.
\par 7 Memang tidak bisa dibantah bahwa orang yang diberkati adalah lebih rendah daripada orang yang memberkatinya.
\par 8 Imam-imam yang menerima pungutan sepersepuluh bagian itu adalah orang yang bisa mati. Tetapi menurut Alkitab, Melkisedek yang menerima pemberian sepersepuluh bagian itu adalah orang yang tetap hidup.
\par 9 Boleh dikatakan, bahwa Lewi--yang keturunannya memungut sepersepuluh bagian dari umat Israel--membayar juga bagian itu lewat Abraham, pada waktu Abraham membayarnya kepada Melkisedek.
\par 10 Walaupun Lewi belum lahir, tetapi boleh dikatakan ia sudah berada di dalam tubuh Abraham nenek moyangnya, pada waktu Melkisedek bertemu dengan Abraham.
\par 11 Di bawah pimpinan imam-imam Lewi, hukum agama Yahudi diberikan kepada umat Israel. Karena imam-imam Lewi dahulu itu tidak dapat melakukan dengan sempurna apa yang harus dikerjakannya, maka harus ada imam lain, yaitu yang seperti Imam Melkisedek, dan bukan dari golongan Imam Harun lagi!
\par 12 Nah, kalau kedudukan imam-imam berubah, hukum agama Yahudi pun harus berubah juga.
\par 13 Yang dimaksudkan di sini ialah Tuhan kita: Ia dari suku lain, dan belum pernah seorang pun dari suku-Nya bertugas sebagai imam.
\par 14 Semua orang tahu bahwa Tuhan kita berasal dari suku Yehuda; dan Musa tidak pernah menyebut suku itu sewaktu ia berbicara tentang imam-imam.
\par 15 Hal yang dibicarakan ini menjadi lebih jelas lagi, dengan munculnya seorang imam lain yang seperti Melkisedek.
\par 16 Ia diangkat menjadi imam, bukan berdasarkan peraturan-peraturan manusia, melainkan berdasarkan hidup-Nya yang berkuasa dan yang tidak ada akhirnya.
\par 17 Sebab di dalam Alkitab dikatakan, "Engkau adalah imam selama-lamanya, seperti Imam Melkisedek."
\par 18 Jadi, sekarang peraturan yang lama sudah dikesampingkan, sebab tidak kuat dan tidak berguna.
\par 19 Hukum agama Yahudi tidak dapat membuat sesuatu pun menjadi sempurna. Itu sebabnya kita sekarang diberikan suatu harapan yang lebih baik; dengan itu kita dapat mendekati Allah.
\par 20 Lagipula, semuanya itu disertai dengan sumpah dari Allah. Tidak ada sumpah seperti itu pada waktu imam-imam yang lain itu diangkat.
\par 21 Tetapi Yesus diangkat menjadi imam pakai sumpah, pada waktu Allah berkata kepada-Nya, "Tuhan sudah bersumpah, dan Ia tidak akan merubah pendirian-Nya, 'Engkau adalah imam selama-lamanya.'"
\par 22 Dengan ini pula Yesus menjadi jaminan untuk suatu perjanjian yang lebih baik.
\par 23 Ada lagi satu hal: imam-imam yang lain itu ada banyak, sebab mereka masing-masing mati sehingga tidak dapat meneruskan jabatannya.
\par 24 Tetapi Yesus hidup selama-lamanya, jadi jabatan-Nya sebagai imam tidak berpindah kepada orang lain.
\par 25 Oleh karena itu untuk selama-lamanya Yesus dapat menyelamatkan orang-orang yang datang kepada Allah melalui Dia, sebab Ia hidup selama-lamanya untuk mengajukan permohonan kepada Allah bagi orang-orang itu.
\par 26 Jadi, Yesus itulah Imam Agung yang kita perlukan. Ia suci; pada-Nya tidak terdapat kesalahan atau dosa apa pun. Ia dipisahkan dari orang-orang berdosa, dan dinaikkan sampai ke tempat yang lebih tinggi dari segala langit.
\par 27 Ia tidak seperti imam-imam agung yang lain, yang setiap hari harus mempersembahkan kurban karena dosa-dosanya sendiri dahulu, baru karena dosa-dosa umat. Yesus mempersembahkan kurban hanya sekali saja untuk selama-lamanya, ketika Ia mempersembahkan diri-Nya sendiri sebagai kurban.
\par 28 Hukum agama Yahudi mengangkat orang yang tidak sempurna menjadi imam agung. Tetapi kemudian dari hukum itu, Allah membuat janji dengan sumpah bahwa Ia mengangkat sebagai Imam Agung, Anak yang sudah dijadikan Penyelamat yang sempurna untuk selama-lamanya.

\chapter{8}

\par 1 Pokok dari seluruh pembicaraan ini ialah: Kita mempunyai Imam Agung yang seperti itu, yang duduk memerintah bersama Allah Mahabesar di surga.
\par 2 Ia mengerjakan tugas sebagai Imam Agung di Ruang Mahasuci, yaitu di dalam Kemah Tuhan yang sejati, yang didirikan oleh Tuhan, bukan oleh manusia.
\par 3 Setiap imam agung diangkat untuk mempersembahkan kurban atau persembahan kepada Allah. Begitu jugalah Imam Agung kita; Ia harus mempunyai sesuatu untuk dipersembahkan.
\par 4 Andaikata Ia berada di dunia ini, Ia tidak akan menjadi imam, sebab sudah ada imam yang mempersembahkan kurban yang dituntut dalam hukum agama Yahudi.
\par 5 Pekerjaan yang mereka lakukan sebagai imam itu sebenarnya hanya suatu gambaran dan bayangan dari apa yang ada di surga. Itu sudah diberitahukan oleh Allah kepada Musa dahulu, ketika Musa mau mendirikan Kemah Tuhan. Allah berkata begini kepada Musa, "Ingat! Semuanya itu haruslah kaubuat menurut pola yang ditunjukkan kepadamu di atas gunung."
\par 6 Tetapi sekarang Yesus mendapat tugas imam yang jauh lebih mulia daripada yang dikerjakan oleh imam-imam itu. Sebab perjanjian yang diadakan-Nya antara Allah dan manusia, adalah perjanjian yang lebih baik karena berdasarkan janji untuk hal-hal yang lebih baik.
\par 7 Andaikata tidak ada kekurangan pada perjanjian yang pertama, maka tidak perlu diadakan perjanjian yang kedua.
\par 8 Tetapi Allah sudah menemukan kesalahan pada umat-Nya, sehingga Allah berkata, "Ingat, akan datang waktunya, Aku akan mengadakan perjanjian yang baru dengan bangsa Israel, dan dengan bangsa Yehuda.
\par 9 Bukan seperti perjanjian yang Kubuat dengan leluhur mereka pada waktu Aku menuntun mereka keluar dari negeri Mesir. Mereka tidak setia kepada perjanjian yang Aku buat dengan mereka; itulah sebabnya Aku tidak mempedulikan mereka.
\par 10 Tetapi sekarang, inilah perjanjian yang akan Kubuat dengan umat Israel pada hari-hari yang akan datang, kata Tuhan: Aku akan menaruh hukum-Ku ke dalam pikiran mereka, dan menulisnya pada hati mereka. Aku akan menjadi Allah mereka, dan mereka akan menjadi umat-Ku.
\par 11 Mereka tidak perlu mengajar sesama warganya, atau memberitahu kepada saudaranya, 'Kenallah Tuhan.' Sebab mereka semua, besar kecil, akan mengenal Aku.
\par 12 Aku akan mengampuni kesalahan-kesalahan mereka, dan tidak mengingat lagi dosa-dosa mereka."
\par 13 Dengan mengemukakan suatu perjanjian yang baru, Allah membuat perjanjian yang pertama itu menjadi tua dan usang; dan apa yang sudah tua, akan segera pula lenyap.

\chapter{9}

\par 1 Perjanjian yang pertama mempunyai peraturan-peraturan ibadat, dan mempunyai juga tempat ibadat buatan manusia.
\par 2 Sebuah kemah didirikan yang bagian depannya dinamakan Ruang Suci. Di situ ada standar untuk pelita, dan ada juga meja dengan roti yang dipersembahkan kepada Allah.
\par 3 Di bagian dalamnya, yaitu di belakang gorden yang kedua, ada ruangan yang dinamakan Ruang Mahasuci.
\par 4 Di dalam ruangan itu ada mezbah yang dibuat dari emas untuk membakar dupa, dan ada juga Peti Perjanjian yang seluruhnya dilapisi dengan emas. Di dalam Peti itu terdapat belanga emas berisi manna, tongkat Harun yang telah bertunas, dan dua lempeng batu tulis yang di atasnya tertulis sepuluh perintah dari Allah.
\par 5 Di atas Peti itu terdapat dua Kerub, yaitu makhluk bersayap yang melambangkan kehadiran Allah. Sayap dari kedua makhluk itu terkembang di atas tutup Peti, yaitu tempat pengampunan dosa. Tetapi semuanya itu tidak dapat diterangkan sekarang secara terperinci.
\par 6 Begitulah semuanya diatur. Tiap-tiap hari imam-imam masuk ke dalam bagian depan kemah itu untuk menjalankan tugas mereka.
\par 7 Yang masuk ke bagian paling dalam dari kemah itu hanyalah imam agung saja. Ia melakukan itu cuma sekali setahun. Itu dilakukannya dengan membawa darah untuk dipersembahkan kepada Allah karena dirinya sendiri dan karena dosa-dosa yang dilakukan tanpa sadar oleh umat-Nya.
\par 8 Dengan aturan tersebut, Roh Allah menunjukkan dengan jelas, bahwa selama kemah bagian depan itu masih berdiri, jalan masuk ke dalam Ruang Mahasuci itu belum terbuka.
\par 9 Ini melambangkan zaman sekarang; berarti bahwa persembahan-persembahan dan kurban-kurban binatang yang dipersembahkan kepada Allah, tidak dapat menyempurnakan hati nurani orang yang membawa persembahan.
\par 10 Sebab upacara-upacara itu hanya berkenaan dengan makanan, minuman, dan bermacam-macam upacara penyucian. Semuanya cuma peraturan-peratura lahir yang berlaku hanya sampai saatnya Allah mengadakan pembaharuan.
\par 11 Tetapi Kristus sudah datang sebagai Imam Agung dari hal-hal yang baik yang sudah ada. Kemah Tuhan di mana Ia mengerjakan tugas-Nya sebagai Imam Agung adalah kemah yang lebih agung dan lebih sempurna. Itu tidak dibuat oleh manusia; artinya bukan berasal dari dunia yang diciptakan ini.
\par 12 Kristus memasuki Ruang Mahasuci di dalam kemah itu hanya sekali saja untuk selama-lamanya. Pada waktu itu Ia tidak membawa darah kambing jantan atau darah anak lembu untuk dipersembahkan; Ia membawa darah-Nya sendiri, dan dengan itu Ia membebaskan kita untuk selama-lamanya.
\par 13 Darah dari kambing dan sapi jantan serta abu dari kurban anak sapi, dipakai untuk memerciki orang-orang yang najis menurut peraturan agama supaya mereka menjadi bersih.
\par 14 Nah, kalau darah dan abu itu dapat membersihkan kenajisan orang-orang itu, apalagi darah Kristus! Melalui Roh yang abadi, Kristus mempersembahkan diri-Nya sendiri kepada Allah sebagai kurban yang sempurna. Darah-Nya membersihkan hati nurani kita dari upacara agama yang tidak berguna, supaya kita dapat melayani Allah yang hidup.
\par 15 Itulah sebabnya Kristus menjadi Pengantara untuk suatu perjanjian yang baru, supaya orang yang sudah dipanggil oleh Allah dapat menerima berkat-berkat abadi yang telah dijanjikan oleh Allah. Semuanya itu dapat terjadi karena sudah ada yang mati, yaitu Kristus; dan kematian-Nya itu membebaskan orang dari kesalahan-kesalahan yang mereka lakukan pada waktu perjanjian yang pertama masih berlaku.
\par 16 Kalau ada surat warisan, harus juga ada buktinya bahwa orang yang membuat surat itu sudah meninggal.
\par 17 Sebab surat warisan tidak berlaku selama orang yang membuatnya masih hidup. Surat itu berlaku hanya setelah orang itu mati.
\par 18 Karena itu perjanjian yang pertama pun harus disahkan dengan darah.
\par 19 Mula-mula Musa menyampaikan semua perintah hukum Allah kepada bangsa Israel. Sesudah itu Musa mengambil darah anak sapi dan darah kambing jantan, lalu mencampurkannya dengan air, kemudian memercikkannya pada Kitab Hukum-hukum dari Allah dan pada seluruh bangsa Israel dengan memakai rerumput hisop dan bulu domba berwarna merah tua.
\par 20 Sambil melakukan itu Musa berkata, "Inilah darah yang mensahkan perjanjian dari Allah yang harus kalian taati."
\par 21 Kemudian dengan cara yang sama, Musa memercikkan darah itu juga pada Kemah Tuhan dan pada semua alat-alat untuk ibadah.
\par 22 Memang menurut hukum agama Yahudi, hampir segala sesuatu disucikan dengan darah; dan dosa hanya bisa diampuni kalau ada penumpahan darah.
\par 23 Dengan cara seperti itulah barang-barang yang melambangkan hal-hal yang di surga, perlu disucikan. Tetapi untuk hal-hal yang di surga itu sendiri diperlukan kurban yang jauh lebih baik.
\par 24 Sebab Kristus tidak masuk ke Ruang Suci buatan manusia, yang hanya melambangkan Ruang Suci yang sebenarnya. Kristus masuk ke surga sendiri; di sana Ia sekarang menghadap Allah untuk kepentingan kita.
\par 25 Imam agung Yahudi tiap-tiap tahun masuk ke Ruang Mahasuci di dalam Rumah Tuhan dengan membawa darah seekor binatang. Tetapi Kristus tidak masuk untuk mempersembahkan diri-Nya berulang-ulang.
\par 26 Sebab kalau demikian, itu berarti Ia sudah berulang-ulang menderita sejak dunia ini diciptakan. Tetapi nyatanya, sekarang pada zaman akhir ini, Ia datang satu kali saja untuk menghapus dosa dengan mengurbankan diri-Nya sendiri.
\par 27 Allah sudah menetapkan bahwa manusia mati satu kali saja dan setelah itu diadili oleh Allah.
\par 28 Begitu juga Kristus satu kali saja dipersembahkan sebagai kurban untuk menghapus dosa banyak orang. Ia akan datang lagi pada kedua kalinya, bukan untuk menyelesaikan persoalan dosa, tetapi untuk menyelamatkan orang-orang yang menantikan kedatangan-Nya.

\chapter{10}

\par 1 Hukum agama Yahudi hanya memberikan gambaran yang samar-samar tentang hal-hal yang baik yang akan datang, dan bukan gambaran yang sebenarnya dari hal-hal itu. Tidak mungkin hukum itu dapat menyempurnakan orang yang datang menyembah Allah dengan membawa persembahan, walaupun tiap tahun terus dipersembahkan kurban-kurban yang sama.
\par 2 Andaikata orang-orang yang menyembah Allah itu benar-benar sudah dibersihkan dari dosa, mereka tidak lagi akan mempunyai perasaan berdosa, dan kurban tidak akan dipersembahkan lagi.
\par 3 Tetapi nyatanya kurban-kurban yang dipersembahkan setiap tahun itu justru memperingatkan orang akan dosa-dosa mereka,
\par 4 sebab memang darah sapi dan darah kambing jantan tidak dapat menghapuskan dosa.
\par 5 Itulah sebabnya pada waktu Kristus masuk ke dunia, Ia berkata kepada Allah, "Engkau tidak menghendaki kurban dan persembahan; sebaliknya Engkau sudah menyediakan tubuh bagi-Ku.
\par 6 Engkau tidak berkenan akan kurban bakaran atau kurban untuk pengampunan dosa.
\par 7 Lalu Aku berkata, 'Inilah Aku, ya Allah! Aku datang untuk melakukan kehendak-Mu, seperti yang tersurat tentang diri-Ku di dalam Alkitab.'"
\par 8 Mula-mula Kristus berkata, "Engkau tidak menghendaki kurban dan persembahan; Engkau tidak berkenan akan kurban binatang yang dibakar sebagai persembahan dan akan kurban untuk pengampunan dosa." Kristus berkata begitu, sekalipun segala kurban itu dipersembahkan menurut hukum agama Yahudi.
\par 9 Sesudah itu Kristus berkata, "Inilah Aku, ya Allah! Aku datang untuk melakukan kehendak-Mu." Jadi Allah menghapuskan segala kurban yang lama itu, dan menggantikannya dengan kurban Kristus.
\par 10 Yesus Kristus sudah melakukan apa yang dikehendaki Allah dan mempersembahkan diri-Nya sebagai kurban. Dengan persembahan itu, yang dilakukan-Nya hanya sekali saja untuk selama-lamanya, kita semua dibersihkan dari dosa.
\par 11 Setiap imam Yahudi menjalankan tugasnya sebagai imam tiap-tiap hari, dan berulang kali ia mempersembahkan kurban-kurban yang sama. Tetapi kurban-kurban itu sama sekali tidak dapat menghapuskan dosa.
\par 12 Sebaliknya, Kristus mempersembahkan hanya satu kurban untuk pengampunan dosa, dan kurban itu berlaku untuk selama-lamanya. Sesudah mempersembahkan kurban itu, Kristus duduk di sebelah kanan Allah dan memerintah bersama-sama dengan Dia.
\par 13 Dan sekarang Kristus menunggu sampai Allah membuat musuh-musuh-Nya takluk kepada-Nya.
\par 14 Jadi dengan satu kurban, Kristus menyempurnakan orang-orang yang sudah dibersihkan dari dosa.
\par 15 Mengenai hal itu Roh Allah memberikan juga kesaksian-Nya kepada kita. Roh Allah berkata,
\par 16 "'Inilah perjanjian yang akan Kubuat dengan mereka pada hari-hari yang akan datang,' kata Tuhan, 'Aku akan menaruh hukum-hukum-Ku ke dalam hati mereka, dan menulisnya ke dalam pikiran mereka.'"
\par 17 Allah juga berkata, "Aku akan melupakan dosa-dosa dan kejahatan-kejahatan mereka."
\par 18 Jadi, dosa-dosa dan kejahatan-kejahatan itu sudah diampuni, maka tidak perlu lagi dipersembahkan kurban untuk pengampunan dosa.
\par 19 Nah, Saudara-saudara, oleh kematian Yesus itu kita sekarang berani memasuki Ruang Mahasuci.
\par 20 Yesus sudah membuka suatu jalan yang baru untuk kita, yaitu jalan yang memberi kehidupan. Jalan itu melalui gorden, yaitu tubuh Yesus sendiri.
\par 21 Dan kita sekarang mempunyai seorang imam yang agung, yang bertanggung jawab atas Rumah Allah.
\par 22 Sebab itu, marilah kita mendekati Allah dengan hati yang tulus dan iman yang teguh; dengan hati yang sudah disucikan dari perasaan bersalah, dan dengan tubuh yang sudah dibersihkan dengan air yang murni.
\par 23 Hendaklah kita berpegang teguh pada harapan yang kita akui, sebab Allah bisa dipercayai dan Ia akan menepati janji-Nya.
\par 24 Dan hendaklah kita saling memperhatikan, supaya kita dapat saling memberi dorongan untuk mengasihi sesama dan melakukan hal-hal yang baik.
\par 25 Hendaklah kita tetap berkumpul bersama-sama, dan janganlah lalai seperti orang lain. Kita justru harus lebih setia saling menguatkan, sebab kita tahu bahwa tidak lama lagi Tuhan akan datang.
\par 26 Sebab kalau berita yang benar dari Allah sudah disampaikan kepada kita, tetapi kita terus saja berbuat dosa dengan sengaja, maka tidak ada lagi kurban untuk menghapus dosa kita.
\par 27 Satu-satunya yang ada untuk kita ialah menghadapi pengadilan Allah dan api kemarahan-Nya yang akan membakar habis orang-orang yang melawan Dia!
\par 28 Orang yang tidak mentaati hukum yang diberi oleh Musa, dihukum mati tanpa ampun, kalau atas kesaksian dua tiga orang, ia terbukti bersalah.
\par 29 Betapa lebih berat hukuman yang harus dijatuhkan atas orang yang menginjak-injak Anak Allah. Orang itu menganggap hina darah perjanjian Allah, yakni kematian Kristus yang membersihkannya dari dosa. Ia menghina Roh pemberi rahmat.
\par 30 Kita tahu siapa Dia yang berkata, "Aku akan membalas! Aku akan menghukum!" dan yang berkata, "Tuhan akan menghakimi umat-Nya."
\par 31 Alangkah ngerinya kalau jatuh ke tangan Allah Yang Hidup!
\par 32 Ingatlah bagaimana keadaan Saudara-saudara pada waktu yang lalu. Pada waktu itu, setelah cahaya Allah menyinarimu, kalian banyak menderita; namun kalian tetap berjuang dengan gigih.
\par 33 Ada kalanya kalian dihina dan diperlakukan dengan tidak baik di depan umum. Ada kalanya juga kalian turut menderita dengan mereka yang diperlakukan demikian,
\par 34 dan kalian turut merasakan kesedihan orang yang dipenjarakan. Dan ketika semua yang kalian miliki dirampas, kalian menerima itu dengan senang hati, sebab kalian tahu bahwa kalian masih mempunyai sesuatu yang lebih baik, yang akan tahan selama-lamanya.
\par 35 Oleh karena itu, janganlah putus asa, sebab kalau kalian tetap percaya, maka ada upah yang besar untuk itu!
\par 36 Kalian perlu bersabar, supaya kalian dapat melakukan kehendak Allah dan dengan demikian menerima apa yang dijanjikan-Nya.
\par 37 Sebab dalam Alkitab tertulis, "Hanya sebentar saja lagi, maka Ia yang akan datang itu, akan segera datang; Ia tidak akan menunda-nunda kedatangan-Nya.
\par 38 Dan umat-Ku yang benar akan percaya dan hidup; tetapi kalau ada di antara mereka yang mundur, maka Aku tidak akan senang kepadanya."
\par 39 Kita ini bukanlah umat yang mundur dan sesat. Sebaliknya, kita adalah umat yang percaya kepada Allah dan yang diselamatkan.

\chapter{11}

\par 1 Beriman berarti yakin sungguh-sungguh akan hal-hal yang diharapkan, berarti mempunyai kepastian akan hal-hal yang tidak dilihat.
\par 2 Karena beriman, maka orang-orang zaman lampau disenangi oleh Allah.
\par 3 Karena beriman, maka kita mengerti bahwa alam ini diciptakan oleh sabda Allah; jadi, apa yang dapat dilihat, terjadi dari apa yang tidak dapat dilihat.
\par 4 Karena beriman, maka Habel mempersembahkan kepada Allah kurban yang lebih baik daripada kurban Kain. Karena imannya itu, Habel diterima oleh Allah sebagai orang yang baik, sebab nyatalah bahwa Allah menerima persembahannya. Habel sudah meninggal, tetapi karena imannya itu, maka ia masih berbicara sampai sekarang.
\par 5 Karena beriman, maka Henokh tidak mati, melainkan diangkat dan dibawa kepada Allah. Tidak seorang pun dapat menemukan dia, sebab dia sudah diangkat oleh Allah. Dalam Alkitab tertulis bahwa sebelum Henokh diangkat, ia menyenangkan hati Allah.
\par 6 Tanpa beriman, tidak seorang pun dapat menyenangkan hati Allah. Sebab orang yang datang kepada Allah harus percaya bahwa Allah ada, dan bahwa Allah memberi balasan kepada orang yang mencari-Nya.
\par 7 Karena beriman, maka Nuh diberitahu oleh Allah tentang hal-hal yang akan terjadi kemudian, yang tidak dapat dilihat olehnya. Nuh mentaati Allah sehingga ia membuat sebuah kapal yang kemudian ternyata menyelamatkan dirinya bersama keluarganya. Dengan demikian dunia dihukum, sedangkan Nuh sendiri karena imannya dinyatakan oleh Allah sebagai orang yang baik.
\par 8 Karena beriman, maka Abraham mentaati Allah ketika Allah memanggilnya dan menyuruhnya pergi ke negeri yang Allah janjikan kepadanya. Lalu Abraham berangkat dengan tidak tahu ke mana akan pergi.
\par 9 Dengan beriman, Abraham tinggal sebagai orang asing di negeri yang dijanjikan Allah kepadanya itu. Abraham tinggal di situ di dalam kemah. Begitu pula Ishak dan Yakub, yang menerima janji yang sama dari Allah.
\par 10 Sebab Abraham sedang menanti-nantikan kota yang direncanakan dan dibangun oleh Allah dengan pondasi yang kuat.
\par 11 Karena beriman, maka Abraham bisa mendapat keturunan dari Sara, sekalipun Abraham sudah terlalu tua, dan Sara sendiri mandul. Abraham yakin bahwa Allah akan menepati janji-Nya.
\par 12 Meskipun Abraham pada masa itu seperti orang yang sudah mati tubuhnya, namun ia memperoleh banyak keturunan, yang banyaknya tidak terhitung--seperti banyaknya bintang di langit dan pasir di tepi pantai.
\par 13 Semua orang itu tetap beriman sampai mati. Mereka tidak menerima hal-hal yang dijanjikan oleh Allah, tetapi hanya melihat dan menyambutnya dari jauh. Dan dengan itu mereka menyatakan bahwa mereka hanyalah orang asing dan perantau di bumi ini.
\par 14 Orang yang mengatakan demikian menunjukkan dengan jelas bahwa mereka sedang mencari negeri yang akan menjadi tanah air mereka.
\par 15 Bukan negeri yang sudah mereka tinggalkan itu yang mereka pikir-pikirkan. Sebab kalau demikian, maka sudah banyak kesempatan bagi mereka untuk kembali ke negeri itu.
\par 16 Tetapi nyatanya, mereka merindukan sebuah negeri yang lebih baik, yaitu negeri yang di surga. Itulah sebabnya Allah tidak malu kalau mereka menyebut Dia Allah mereka, sebab Allah sudah menyediakan sebuah kota untuk mereka.
\par 17 Karena beriman juga, maka Abraham mempersembahkan Ishak sebagai kurban ketika ia diuji Allah. Kepada Abrahamlah Allah memberikan janji-Nya, namun Abraham rela menyerahkan anaknya yang satu-satunya itu.
\par 18 Allah telah berkata kepada Abraham, "Melalui Ishak inilah engkau akan mendapat keturunan yang Aku janjikan kepadamu."
\par 19 Abraham yakin bahwa Allah sanggup menghidupkan kembali Ishak dari kematian--jadi, boleh dikatakan, Abraham sudah menerima kembali Ishak dari kematian.
\par 20 Karena beriman, maka Ishak menjanjikan berkat-berkat kepada Yakub dan Esau untuk masa depan.
\par 21 Karena beriman, maka sebelum Yakub meninggal, ia memberi berkatnya kepada anak-anak Yusuf--dengan bersandar pada kepala tongkatnya dan menyembah Allah.
\par 22 Karena beriman, maka Yusuf--ketika hampir meninggal dunia--berbicara tentang keluarnya umat Israel dari Mesir, dan meninggalkan pesan tentang apa yang harus dilakukan terhadap jenazahnya.
\par 23 Karena beriman, maka orang tua Musa menyembunyikannya tiga bulan lamanya setelah kelahirannya. Mereka melihat bahwa ia seorang anak yang bagus, dan mereka tidak takut melawan perintah raja.
\par 24 Karena beriman, maka Musa sesudah besar, tidak mau disebut anak dari putri raja Mesir.
\par 25 Ia lebih suka menderita bersama-sama dengan umat Allah daripada untuk sementara waktu menikmati kesenangan dari hidup yang berdosa.
\par 26 Musa merasa bahwa jauh lebih berharga untuk mendapat penghinaan demi Raja Penyelamat yang dijanjikan Allah itu daripada mendapat segala harta negeri Mesir, sebab Musa mengharapkan upah di hari kemudian.
\par 27 Karena beriman, maka Musa meninggalkan Mesir tanpa merasa takut terhadap kemarahan raja. Musa maju menuju tujuannya seolah-olah ia sudah melihat Allah yang tidak kelihatan itu.
\par 28 Karena beriman, maka Musa mengadakan Paskah dan memerintahkan agar dipercikkan darah pada pintu rumah orang Israel supaya Malaikat Kematian jangan membunuh anak-anak sulung mereka.
\par 29 Karena beriman, maka orang-orang Israel dapat menyeberangi Laut Merah, seolah-olah mereka berjalan di atas tanah yang kering, sedangkan orang-orang Mesir ditelan oleh laut itu, ketika mereka mencoba menyeberang juga.
\par 30 Karena beriman, orang-orang Israel membuat tembok-tembok Yerikho runtuh setelah mereka mengelilinginya selama tujuh hari.
\par 31 Karena beriman juga, maka Rahab, wanita pelacur itu, tidak turut terbunuh bersama-sama dengan orang-orang yang melawan Allah; sebab ia menerima dengan ramah pengintai-pengintai Israel.
\par 32 Nah, saya bisa saja terus-menerus berbicara, tetapi waktu tidak cukup untuk saya. Sebab saya belum lagi menyebut Gideon, Barak, Simson, Yefta, Daud, Samuel, dan nabi-nabi.
\par 33 Karena beriman, maka mereka sudah mengalahkan kerajaan-kerajaan. Mereka melakukan apa yang benar, sehingga menerima apa yang dijanjikan Allah. Mereka menutup mulut-mulut singa,
\par 34 memadamkan api yang hebat, terhindar dari tikaman pedang. Mereka lemah, tetapi menjadi kuat; mereka perkasa dalam peperangan sehingga mengalahkan pasukan-pasukan bangsa asing.
\par 35 Karena beriman, maka wanita-wanita mendapat kembali orang-orangnya yang telah mati. Ada pula yang rela disiksa sampai mati, dan menolak untuk dibebaskan, karena mau dihidupkan kembali bagi suatu kehidupan yang lebih baik.
\par 36 Ada yang diolok-olok, dicambuk, diikat dengan rantai dan yang dimasukkan ke dalam penjara.
\par 37 Mereka dilempari batu sampai mati, ada yang dipotong dengan gergaji, dan yang dibunuh dengan pedang. Mereka mengembara dengan pakaian dari kulit domba atau kulit kambing; mereka miskin, dianiaya dan disiksa.
\par 38 Dunia ini bukan tempat yang layak bagi mereka. Mereka mengembara di padang gurun dan di bukit-bukit, serta tinggal di dalam gua-gua dan dalam lubang-lubang tanah.
\par 39 Alangkah besarnya kemenangan yang dicapai oleh semua orang itu karena iman mereka! Namun mereka tidak menerima apa yang dijanjikan Allah,
\par 40 sebab Allah mempunyai rencana yang lebih baik untuk kita. Rencana-Nya ialah bahwa hanya bersama dengan kita, mereka akan menjadi sempurna.

\chapter{12}

\par 1 Nah, mengenai kita sendiri, di sekeliling kita ada banyak sekali saksi! Sebab itu, marilah kita membuang semua yang memberatkan kita dan dosa yang terus melekat pada kita. Dan marilah kita dengan tekun menempuh perlombaan yang ada di depan kita.
\par 2 Hendaklah pandangan kita tertuju kepada Yesus, sebab Dialah yang membangkitkan iman kita dan memeliharanya dari permulaan sampai akhir. Yesus tahan menderita di kayu salib! Ia tidak peduli bahwa mati di kayu salib itu adalah suatu hal yang memalukan. Ia hanya ingat akan kegembiraan yang akan dirasakan-Nya kemudian. Sekarang Ia duduk di sebelah kanan takhta Allah dan memerintah bersama dengan Dia.
\par 3 Coba pikirkan bagaimana sengsaranya Yesus menghadapi orang-orang berdosa yang melawan-Nya dengan begitu sengit! Sebab itu janganlah berkecil hati dan putus asa.
\par 4 Sebab dalam perjuanganmu melawan dosa, kalian belum pernah berjuang sampai harus menumpahkan darah.
\par 5 Dan janganlah melupakan nasihat Allah ini, yang diberikan kepadamu sebagai anak-anak-Nya: "Anak-Ku, perhatikanlah baik-baik ajaran Tuhan, dan janganlah berkecil hati kalau Ia memarahimu.
\par 6 Sebab Tuhan menghajar setiap orang yang dikasihi-Nya, dan Ia mencambuk setiap orang yang diakui-Nya sebagai anak-Nya."
\par 7 Hendaklah kalian menerima cambukan dari Allah sebagai suatu hajaran dari seorang bapak. Sebab apakah pernah seorang anak tidak dihukum oleh bapaknya?
\par 8 Kalau kalian tidak turut dihukum seperti semua anaknya yang lain ini berarti kalian bukan anak sah, melainkan anak yang tidak sah.
\par 9 Kita mempunyai bapak di dunia. Ia mengajar kita, dan kita menghormatinya. Nah, apalagi terhadap Bapa rohani kita yang di surga, tentu kita harus lebih lagi tunduk kepada-Nya supaya kita hidup.
\par 10 Orang tua kita yang di dunia mengajar kita hanya dalam waktu yang terbatas, menurut apa yang mereka merasa baik. Tetapi Allah mengajar kita untuk kebaikan kita sendiri, supaya kita dapat menjadi suci bersama-sama dengan Dia.
\par 11 Memang pada waktu kita diajar, hukuman itu tidak menyenangkan hati kita, melainkan hanya menyedihkan saja. Tetapi kemudian dari itu, bagi kita yang sudah diajar, hukuman itu menyebabkan kita hidup menurut kemauan Allah, dan menghasilkan perasaan sejahtera pada kita.
\par 12 Sebab itu, kuatkanlah tanganmu yang lemah dan lututmu yang gemetar itu!
\par 13 Berjalanlah selalu pada jalan yang rata, supaya kakimu yang timpang itu tidak terkilir, tetapi malah menjadi sembuh.
\par 14 Berusahalah untuk hidup rukun dengan semua orang. Berusahalah juga untuk hidup suci, khusus untuk Tuhan. Sebab tidak seorang pun dapat melihat Tuhan kalau ia tidak hidup seperti itu.
\par 15 Jagalah jangan sampai ada seorang pun yang keluar dari lingkungan kebaikan hati Allah, supaya jangan ada yang menjadi seperti tumbuhan beracun di tengah-tengah kalian sehingga menimbulkan kesukaran dan merusak banyak orang dengan racunnya.
\par 16 Jagalah supaya jangan ada yang hidup cabul atau tidak menghargai hal-hal rohani, seperti yang dilakukan oleh Esau. Ia menjual haknya sebagai anak sulung, hanya untuk satu mangkuk makanan.
\par 17 Kalian tahu bahwa kemudian Esau ingin mendapat berkat itu dari bapaknya, tetapi ia ditolak. Sebab sekalipun dengan tangis ia mencari jalan untuk memperbaiki kesalahannya, kesempatan untuk itu tidak ada lagi.
\par 18 Saudara-saudara tidak datang menghadapi sesuatu seperti yang dihadapi oleh bangsa Israel dahulu. Mereka menghadapi sesuatu yang bisa diraba, yaitu Gunung Sinai dengan apinya yang bernyala-nyala; mereka menghadapi kegelapan, kekelaman dan angin ribut;
\par 19 mereka menghadapi bunyi trompet, dan bunyi suara yang hebat. Ketika orang-orang Israel mendengar suara itu, mereka meminta dengan sangat supaya suara itu tidak berbicara lagi kepada mereka.
\par 20 Sebab mereka tidak tahan mendengar perintah yang disampaikan oleh suara itu. Karena suara itu berkata, "Semua yang menyentuh gunung ini, tidak peduli apakah itu binatang atau siapapun juga, harus dilempari dengan batu sampai mati."
\par 21 Apa yang dilihat oleh orang-orang Israel itu begitu hebat sampai Musa berkata, "Saya takut dan gemetar!"
\par 22 Sebaliknya, kalian telah datang ke Bukit Sion dan kota Allah yang hidup, yaitu Yerusalem yang di surga dengan beribu-ribu malaikatnya.
\par 23 Kalian mengikuti suatu pertemuan yang meriah--pertemuan anak-anak sulung Allah, yang nama-namanya terdaftar di dalam surga. Kalian datang menghadap Allah, Hakim seluruh umat manusia. Kalian menghadapi roh-roh orang-orang baik, yang sudah dijadikan sempurna.
\par 24 Kalian datang menghadap Yesus, Pengantara untuk perjanjian yang baru itu; kalian menghadapi darah percikan yang menjamin hal-hal yang jauh lebih baik daripada yang dijamin oleh darah Habel.
\par 25 Sebab itu, berhati-hatilah jangan sampai kalian tidak mau mendengarkan Dia yang berbicara itu. Mereka yang tidak mau mendengarkan Dia yang datang ke bumi dan menyampaikan berita dari Allah, tidak bisa melarikan diri. Apalagi kita ini yang mendengarkan Dia yang berbicara dari surga! Kalau kita tidak mau mendengarkan-Nya, mana mungkin kita bisa luput!
\par 26 Pada waktu itu suara-Nya menggemparkan bumi. Tetapi sekarang Ia berjanji, "Sekali lagi, Aku akan menggemparkan bukan saja bumi tetapi langit juga."
\par 27 Perkataan "sekali lagi" menunjukkan bahwa seluruh dunia yang sudah diciptakan akan digoncangkan dan disingkirkan, supaya yang tertinggal hanyalah yang tidak dapat bergoncang.
\par 28 Sebab itu, hendaklah kita mengucap terima kasih kepada Allah, karena kita menerima dari Dia suatu kerajaan yang tidak dapat bergoncang. Hendaklah kita berterima kasih dan beribadat kepada Allah dengan hormat dan takut, menurut cara yang diinginkan oleh-Nya sendiri.
\par 29 Sebab Allah kita seperti api yang menghanguskan.

\chapter{13}

\par 1 Hendaklah kalian tetap mengasihi satu sama lain sebagai orang-orang Kristen yang bersaudara.
\par 2 Jangan segan menerima di rumahmu orang-orang yang belum kalian kenal. Sebab dengan melakukan yang demikian, pernah orang, tanpa menyadarinya, telah menerima malaikat di rumahnya.
\par 3 Ingatlah orang-orang yang di dalam penjara, seolah-olah kalian juga berada di dalam penjara bersama mereka. Dan orang-orang yang diperlakukan sewenang-wenang, hendaklah kalian ingat kepada mereka seolah-olah kalian juga diperlakukan demikian.
\par 4 Semua orang harus menunjukkan sikap hormat terhadap perkawinan, itu sebabnya hendaklah suami-istri setia satu sama lain. Orang yang cabul dan orang yang berzinah akan diadili oleh Allah.
\par 5 Janganlah hidupmu dikuasai oleh cinta akan uang, tetapi hendaklah kalian puas dengan apa yang ada padamu. Sebab Allah sudah berkata, "Aku tidak akan membiarkan atau akan meninggalkan engkau."
\par 6 Sebab itu kita berani berkata, "Tuhan adalah Penolongku, aku tidak takut. Apa yang dapat manusia lakukan terhadapku?"
\par 7 Janganlah lupa kepada pemimpin-pemimpinmu yang menyampaikan pesan Allah kepadamu. Perhatikanlah bagaimana mereka hidup dan bagaimana mereka mati, dan contohilah iman mereka.
\par 8 Yesus Kristus tetap sama, baik dahulu, sekarang, dan sampai selama-lamanya.
\par 9 Janganlah membiarkan segala macam ajaran yang aneh-aneh menyesatkan kalian. Hati kita harus dikuatkan oleh kebaikan hati Allah, itulah yang baik; jangan oleh peraturan-peraturan tentang makanan. Orang-orang yang dahulu mentaati peraturan-peraturan itu tidak mendapat faedahnya.
\par 10 Kita mempunyai sebuah mezbah untuk mempersembahkan kurban kepada Allah. Dan imam-imam yang melayani di dalam Kemah Tuhan, tidak berhak makan kurban yang ada di atas mezbah itu.
\par 11 Darah binatang yang dipersembahkan sebagai kurban untuk pengampunan dosa, dibawa oleh imam agung ke Ruang Mahasuci, tetapi bangkai binatang itu sendiri dibakar di luar perkemahan.
\par 12 Itulah sebabnya Yesus juga mati di luar pintu gerbang kota untuk membersihkan umat-Nya dari dosa dengan darah-Nya sendiri.
\par 13 Karena itu, marilah kita pergi kepada-Nya di luar perkemahan dan turut dihina bersama Dia.
\par 14 Sebab di bumi tidak ada tempat tinggal yang kekal untuk kita; kita mencari tempat tinggal yang akan datang.
\par 15 Sebab itu, dengan perantaraan Yesus, hendaklah kita selalu memuji-muji Allah; itu merupakan kurban syukur kita kepada-Nya, yang kita persembahkan melalui ucapan bibir untuk memuliakan nama-Nya.
\par 16 Jangan lupa berbuat baik dan saling menolong, sebab inilah kurban-kurban yang menyenangkan hati Allah.
\par 17 Ikutlah perintah pemimpin-pemimpinmu, dan tunduklah kepada mereka. Sebab mereka selalu memperhatikan jiwamu, dan mereka harus bertanggung jawab kepada Allah. Kalau kalian taat kepada mereka, mereka dapat bekerja dengan senang hati; kalau tidak, maka mereka akan bekerja dengan sedih hati, dan itu tidak menguntungkan kalian.
\par 18 Teruslah berdoa untuk kami. Kami yakin hati nurani kami murni, sebab dalam segala hal kami selalu mau melakukan apa yang benar.
\par 19 Khususnya saya mohon dengan sangat, supaya kalian mendoakan saya agar Allah segera membawa saya kembali kepadamu.
\par 20 Allah sudah menghidupkan Tuhan kita Yesus dari kematian. Karena kematian-Nya itu Ia sekarang menjadi Gembala Agung bagi kita, yakni domba-domba-Nya. Kematian-Nya itu juga mensahkan perjanjian kekal yang dibuat Allah dengan kita.
\par 21 Semoga Allah sumber sejahtera itu melengkapi kalian dengan segala yang baik yang kalian perlukan untuk melakukan kehendak-Nya. Semoga dengan perantaraan Yesus Kristus, Allah mengerjakan di dalam kita, apa yang diinginkan-Nya. Hendaklah Kristus dipuji selama-lamanya! Amin.
\par 22 Saya mohon supaya kalian memperhatikan nasihat-nasihat saya ini dengan sabar, sebab surat saya ini tidak terlalu panjang.
\par 23 Hendaklah kalian tahu bahwa saudara kita Timotius sudah dibebaskan dari penjara. Kalau ia sudah ke tempat saya ini, saya akan membawanya waktu saya mengunjungi kalian nanti.
\par 24 Sampaikanlah salam kami kepada semua pemimpinmu dan kepada semua umat Allah. Terimalah juga salam dari saudara-saudara di Italia.
\par 25 Semoga Tuhan memberkati Saudara semuanya.


\end{document}