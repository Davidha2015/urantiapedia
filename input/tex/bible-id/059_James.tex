\begin{document}

\title{Yakobus}


\chapter{1}

\par 1 Saudara-saudara umat Allah semuanya yang tersebar di seluruh dunia! Salam dari saya, Yakobus, hamba Allah dan hamba Tuhan Yesus Kristus.
\par 2 Saudara-saudara! Kalau kalian mengalami bermacam-macam cobaan, hendaklah kalian merasa beruntung.
\par 3 Sebab kalian tahu, bahwa kalau kalian tetap percaya kepada Tuhan pada waktu mengalami cobaan, akibatnya ialah: kalian menjadi tabah.
\par 4 Jagalah supaya ketabahan hatimu itu terus berkembang sampai kalian menjadi sungguh-sungguh sempurna serta tidak berkekurangan dalam hal apa pun.
\par 5 Kalau ada seorang di antaramu yang kurang bijaksana, hendaklah ia memintanya dari Allah, maka Allah akan memberikan kebijaksanaan kepadanya; sebab kepada setiap orang, Allah memberi dengan murah hati dan dengan perasaan belas kasihan.
\par 6 Tetapi orang yang meminta, harus percaya; ia tidak boleh ragu-ragu. Sebab orang yang ragu-ragu adalah seperti ombak di laut yang ditiup angin ke sana ke mari.
\par 7 Orang yang seperti itu tidak tetap pikirannya; ia tidak bisa mengambil keputusan apa-apa dalam segala sesuatu yang dibuatnya. Karena itu, tidak usah juga ia mengharapkan untuk mendapat apa-apa dari Tuhan.
\par 9 Orang Kristen yang miskin hendaklah merasa gembira kalau Allah meninggikannya.
\par 10 Dan orang Kristen yang kaya hendaklah merasa gembira juga, kalau Allah merendahkannya. Sebab orang kaya akan lenyap seperti bunga rumput.
\par 11 Pada waktu matahari terbit dengan panasnya yang terik, maka rumput itu akan menjadi layu sehingga gugurlah bunganya dan hilanglah pula keindahannya. Begitulah juga dengan orang yang kaya; ia akan hancur pada waktu ia sedang menjalankan usahanya.
\par 12 Berbahagialah orang yang tabah pada waktu ia mengalami cobaan. Sebab sesudah ia berhasil bertahan dalam cobaan itu, ia akan menerima upahnya, yaitu kehidupan yang telah dijanjikan Allah kepada orang-orang yang mengasihi Allah.
\par 13 Kalau seseorang tergoda oleh cobaan yang semacam itu, janganlah ia berkata, "Godaan ini datangnya dari Allah," sebab Allah tidak dapat tergoda oleh kejahatan, dan tidak juga menggoda seorang pun.
\par 14 Tetapi orang tergoda kalau ia ditarik dan dipikat oleh keinginannya sendiri yang jahat.
\par 15 Kemudian, kalau keinginan yang jahat itu dituruti, maka lahirlah dosa; dan kalau dosa sudah matang, maka akibatnya ialah kematian.
\par 16 Janganlah kalian tertipu, Saudara-saudaraku yang tercinta!
\par 17 Setiap pemberian yang baik dan hadiah yang sempurna datangnya dari surga, diturunkan oleh Allah, Pencipta segala terang di langit. Ialah Allah yang tidak berubah, dan tidak pula menyebabkan kegelapan apa pun.
\par 18 Atas kemauan-Nya sendiri Ia menjadikan kita anak-anak-Nya melalui sabda-Nya yang benar. Ia melakukan itu supaya kita mendapat tempat yang utama di antara semua makhluk ciptaan-Nya.
\par 19 Perhatikanlah ini baik-baik, Saudara-saudara yang tercinta! Setiap orang harus cepat untuk mendengar, tetapi lambat untuk berbicara dan lambat untuk marah.
\par 20 Orang yang marah tidak dapat melakukan yang baik, yang menyenangkan hati Allah.
\par 21 Sebab itu, buanglah setiap kebiasaan yang kotor dan jahat. Terimalah dengan rendah hati perkataan yang ditanam oleh Allah di dalam hatimu, sebab perkataan itu mempunyai kekuatan untuk menyelamatkan kalian.
\par 22 Hendaklah kalian melakukan apa yang dikatakan oleh Allah, jangan hanya mendengarkan saja, sehingga dengan demikian kalian menipu diri sendiri.
\par 23 Orang yang mendengar perkataan Allah, tetapi tidak melakukannya adalah seperti orang yang sedang melihat mukanya yang sebenarnya di depan cermin.
\par 24 Sesudah ia memperhatikannya baik-baik, ia pun pergi dan langsung melupakan bagaimana rupa mukanya itu.
\par 25 Hukum Allah sempurna dan mempunyai kekuatan untuk memerdekakan manusia. Dan orang yang menyelidiki dan memperhatikan baik-baik serta melakukan hukum-hukum itu, dan bukannya mendengar saja lalu melupakannya, orang itu akan diberkati Allah dalam setiap hal yang dilakukannya.
\par 26 Kalau ada seseorang yang merasa dirinya seorang yang patuh beragama, tetapi ia tidak menjaga lidahnya, maka ia menipu dirinya sendiri; ibadatnya tidak ada gunanya.
\par 27 Ketaatan beragama yang murni dan sejati menurut pandangan Allah Bapa ialah: menolong anak-anak yatim piatu dan janda-janda yang menderita, dan menjaga diri sendiri supaya jangan dirusakkan oleh dunia ini.

\chapter{2}

\par 1 Saudara-saudaraku! Sebagai orang yang percaya kepada Tuhan Yesus Kristus, Tuhan Yang Mahamulia, janganlah kalian membeda-bedakan orang berdasarkan hal-hal lahir.
\par 2 Sebab kalau ada seorang kaya yang memakai cincin emas dan pakaian bagus datang ke pertemuanmu, lalu datang pula seorang miskin yang memakai pakaian compang-camping,
\par 3 maka kalian lebih menghormati orang yang berpakaian bagus itu. Kalian berkata kepadanya, "Silakan Tuan duduk di kursi yang terbaik ini." Tetapi kepada orang yang miskin kalian berkata, "Berdirilah di sana," atau "Duduklah di lantai di sini."
\par 4 Dengan berbuat demikian, kalian membuat perbedaan di antara sesamamu dan menilai orang berdasarkan pikiran yang jahat.
\par 5 Ingatlah, Saudara-saudara yang tercinta! Orang yang miskin di dunia ini, dipilih Allah untuk menjadi orang yang kaya dalam iman. Orang-orang itu akan menjadi anggota umat Allah seperti yang dijanjikan Allah kepada orang yang mengasihi-Nya.
\par 6 Tetapi kalian malah menghina orang miskin, padahal orang-orang kayalah yang menindas kalian dan menyeret kalian ke pengadilan!
\par 7 Merekalah yang menghina nama yang terhormat yang kalian terima dari Allah!
\par 8 Kalian melakukan yang benar, kalau kalian melaksanakan hukum Kerajaan yang terdapat dalam ayat Alkitab ini, "Cintailah sesamamu seperti kamu mencintai dirimu sendiri."
\par 9 Tetapi kalau kalian membeda-bedakan orang berdasarkan hal-hal lahir, kalian berbuat dosa, dan hukum Allah menyatakan bahwa kalian adalah pelanggar hukum.
\par 10 Orang yang melanggar salah satu dari hukum-hukum Allah, berarti melanggar seluruhnya.
\par 11 Sebab yang berkata, "Jangan berzinah," dialah juga yang berkata, "Jangan membunuh." Jadi, kalau kalian tidak berzinah, tetapi membunuh, maka kalian adalah pelanggar hukum juga.
\par 12 Berbicaralah dan bertindaklah sebagai orang yang akan diadili menurut hukum Allah yang memerdekakan manusia.
\par 13 Sebab Allah tidak akan menunjukkan belas kasihan kepada orang yang tidak mengenal belas kasihan. Tetapi belas kasihan lebih kuat daripada hukuman!
\par 14 Saudara-saudara! Apa gunanya orang berkata, "Saya orang yang percaya", kalau ia tidak menunjukkannya dengan perbuatannya? Dapatkah iman semacam itu menyelamatkannya?
\par 15 Seandainya seorang saudara atau saudari memerlukan pakaian dan tidak mempunyai cukup makanan untuk sehari-hari.
\par 16 Apa gunanya kalian berkata kepadanya, "Selamat memakai pakaian yang hangat dan selamat makan!" --kalau kalian tidak memberikan kepadanya apa yang diperlukannya untuk hidup?
\par 17 Begitulah juga dengan iman, jika tidak dinyatakan dengan perbuatan, maka iman itu tidak ada gunanya.
\par 18 Mungkin ada yang berkata, "Ada orang yang bersandar kepada imannya dan ada pula yang bersandar kepada perbuatannya." Saya akan menjawab, "Tunjukkanlah kepada saya bagaimana orang dapat mempunyai iman tanpa perbuatan dan saya akan menunjukkan dengan perbuatan bahwa saya mempunyai iman."
\par 19 Kalian percaya bahwa Allah hanya satu, bukan? Nah, roh-roh jahat pun percaya dan mereka gemetar ketakutan!
\par 20 Kalian bodoh sekali! Apakah perlu dibuktikan kepadamu bahwa tidak ada gunanya mempunyai iman tanpa perbuatan?
\par 21 Lihat saja Abraham nenek moyang kita. Ia diterima baik oleh Allah karena perbuatannya, yaitu pada waktu ia mempersembahkan Ishak, anaknya, kepada Allah di atas mezbah.
\par 22 Di sini jelaslah bahwa iman harus ditunjukkan dengan perbuatan supaya menjadi sempurna.
\par 23 Itu sesuai dengan ayat Alkitab ini, "Abraham percaya kepada Allah, dan karena imannya itu Allah menerimanya sebagai orang yang melakukan kehendak Allah." Itu sebabnya Abraham disebut, "Sahabat Allah".
\par 24 Jelaslah sekarang, bahwa orang diterima baik oleh Allah karena apa yang dilakukan oleh orang itu, dan bukan hanya karena imannya saja.
\par 25 Lihat juga pada Rahab, wanita pelacur itu. Ia diterima baik oleh Allah karena perbuatannya, yaitu ketika ia menerima di dalam rumahnya pengintai-pengintai bangsa Israel dan menolong mereka melarikan diri melalui jalan lain.
\par 26 Nah, sebagaimana tubuh tanpa roh adalah tubuh yang mati, begitu juga iman tanpa perbuatan adalah iman yang mati.

\chapter{3}

\par 1 Saudara-saudara! Janganlah banyak-banyak dari antaramu yang mau menjadi guru. Kalian tahu bahwa kita yang menjadi guru akan diadili dengan lebih keras daripada orang lain.
\par 2 Kita semua sering membuat kesalahan. Tetapi orang yang tidak pernah membuat kesalahan dengan kata-katanya, ia orang yang sempurna, yang dapat menguasai seluruh dirinya.
\par 3 Kalau kita memasang kekang pada mulut kuda supaya ia menuruti kemauan kita, maka kita dapat mengendalikan seluruh badan kuda itu.
\par 4 Ambillah juga kapal sebagai contoh. Meskipun kapal adalah sesuatu yang begitu besar dan dibawa oleh angin yang keras, namun ia dikendalikan oleh kemudi yang sangat kecil, menurut keinginan jurumudi.
\par 5 Begitu juga dengan lidah kita; meskipun lidah kita itu kecil, namun ia dapat menyombongkan diri tentang hal-hal yang besar-besar. Bayangkan betapa besarnya hutan dapat dibakar oleh api yang sangat kecil!
\par 6 Lidah sama dengan api. Di tubuh kita, ia merupakan sumber kejahatan yang menyebarkan kejahatan ke seluruh diri kita. Dengan api yang berasal dari neraka, ia menghanguskan seluruh hidup kita.
\par 7 Segala macam binatang buas, burung, binatang menjalar dan ikan dapat dijinakkan, dan sudah pula dijinakkan oleh manusia.
\par 8 Tetapi lidah manusia tidak dapat dijinakkan oleh seorang pun. Lidah itu jahat dan tidak dapat dikuasai; penuh dengan racun yang mematikan.
\par 9 Kita menggunakannya untuk mengucapkan terima kasih kepada Tuhan dan Bapa kita, tetapi juga untuk mengutuki sesama manusia, yang telah diciptakan menurut rupa Allah.
\par 10 Dari mulut yang sama keluar kata-kata terima kasih dan juga kata-kata kutukan. Seharusnya tidak demikian!
\par 11 Apakah ada mata air yang memancarkan air tawar dan air pahit dari sumber yang sama?
\par 12 Pohon ara, Saudara-saudaraku, tidak bisa menghasilkan buah zaitun, dan pohon anggur tidak bisa menghasilkan buah ara. Mata air yang asin tidak bisa juga mengeluarkan air tawar.
\par 13 Kalau di antaramu ada orang yang bijaksana dan berbudi, hendaklah ia menunjukkannya dengan hidup baik dan dengan melakukan hal-hal yang baik, yang disertai kerendahan hati dan kebijaksanaan.
\par 14 Tetapi kalau kalian cemburu, sakit hati, dan mementingkan diri sendiri, janganlah membanggakan kebijaksanaan itu, karena dengan itu kalian memutarbalikkan berita yang benar dari Allah.
\par 15 Kebijaksanaan semacam itu tidak berasal dari surga. Ia berasal dari dunia, dari nafsu manusia, dan dari setan!
\par 16 Di mana ada cemburu dan sifat mementingkan diri sendiri, di situ juga terdapat kerusuhan dan segala macam perbuatan yang jahat.
\par 17 Tetapi orang yang mempunyai kebijaksanaan yang berasal dari atas, ia pertama-tama sekali murni, kemudian suka berdamai, peramah, dan penurut. Ia penuh dengan belas kasihan dan menghasilkan perbuatan-perbuatan yang baik. Ia tidak memihak dan tidak berpura-pura.
\par 18 Memang kebaikan adalah hasil dari benih damai yang ditabur oleh orang yang cinta damai!

\chapter{4}

\par 1 Dari manakah asalnya segala perkelahian dan pertengkaran di antaramu? Bukankah itu berasal dari keinginan-keinginanmu yang terus saja berperang di dalam dirimu untuk mendapatkan kesenangan dunia!
\par 2 Kalian ingin, tetapi tidak mendapat, maka kalian mau membunuh! Kalian bersemangat, tetapi tidak mencapai apa yang kalian cari, maka kalian bertengkar dan berkelahi. Kalian tidak mendapat apa-apa, sebab kalian tidak minta kepada Allah.
\par 3 Dan kalaupun kalian sudah memintanya, kalian toh tidak mendapatnya, sebab tujuan permintaanmu salah; apa yang kalian minta adalah untuk kesenangan diri sendiri.
\par 4 Kalian adalah orang yang tidak setia! Tahukah kalian bahwa kalau kalian berkawan dengan dunia, maka kalian menjadi musuh Allah? Jadi barangsiapa hendak menjadi sahabat dunia ini, ia menjadikan dirinya musuh Allah.
\par 5 Jangan kira bahwa Alkitab tanpa alasan berkata, "Di dalam diri kita Allah menempatkan Roh yang keras keinginannya."
\par 6 Meskipun begitu, rahmat Allah yang diberikan kepada kita lebih kuat daripada keinginan roh kita itu. Itulah sebabnya di dalam Alkitab tertulis juga, "Allah menentang orang yang sombong, tetapi sebaliknya Ia mengasihi orang yang rendah hati."
\par 7 Sebab itu, tunduklah kepada Allah dan lawanlah Iblis, maka Iblis akan lari dari kalian.
\par 8 Dekatilah Allah, dan Allah pun akan mendekati kalian. Bersihkanlah tanganmu, kalian yang berdosa! Dan jernihkanlah hatimu, kalian yang bercabang hati!
\par 9 Hendaklah kalian sungguh-sungguh menyesal dan menangis serta meratap; hendaklah tertawamu menjadi tangisan dan kegembiraanmu menjadi kesedihan!
\par 10 Hendaklah kalian merendahkan diri di hadapan Tuhan, maka Tuhan akan meninggikan kalian.
\par 11 Saudara-saudara, janganlah saling mencela atau saling menyalahkan. Orang yang mencela atau menyalahkan saudaranya yang sama-sama Kristen, ia mencela dan menyalahkan hukum Allah. Dan kalau kalian menyalahkan hukum Allah, itu berarti kalian tidak menuruti hukum-hukum itu, melainkan menjadi hakimnya.
\par 12 Padahal hanya satu yang berhak memberi hukum kepada manusia dan mengadili manusia. Ialah Allah yang berkuasa menyelamatkan dan membinasakan. Jadi, siapakah kalian, sehingga kalian mau menyalahkan sesama manusia?
\par 13 Saudara-saudara yang berkata, "Hari ini atau besok kami akan berangkat ke kota anu dan tinggal di sana setahun lamanya untuk berdagang dan mencari uang," --dengarkanlah nasihat saya ini.
\par 14 Apa yang akan terjadi dengan kehidupanmu besok, kalian sendiri pun tidak mengetahuinya! Kalian hanya seperti asap yang sebentar saja kelihatan, kemudian lenyap.
\par 15 Seharusnya kalian berkata begini, "Kalau Tuhan memperkenankan, kami akan hidup dan melakukan ini atau itu."
\par 16 Tetapi sekarang kalian sombong dan membual. Semua bualan seperti itu salah.
\par 17 Sebab itu, orang yang tahu apa yang baik yang harus dilakukannya tetapi tidak melakukannya, orang itu berdosa.

\chapter{5}

\par 1 Saudara-saudara, orang-orang kaya! Dengarkanlah nasihat saya. Hendaklah kalian menangis dan meratap karena kalian akan menderita sengsara!
\par 2 Kekayaanmu sudah busuk. Pakaianmu sudah dimakan rayap.
\par 3 Emas dan perakmu sudah ditutupi karat; karatnya akan menjadi saksi melawan kalian dan akan memakan habis tubuhmu seperti api. Kalian sudah menimbun harta pada zaman menjelang akhir.
\par 4 Kalian tidak membayar upah orang-orang yang bekerja di ladangmu. Sekarang dengarkan keluhan mereka! Orang-orang yang mengumpulkan hasil ladangmu, berteriak minta tolong dan teriakan mereka sudah sampai ke telinga Allah, Tuhan Yang Mahakuasa.
\par 5 Kalian sudah hidup mewah dan menikmati kesenangan di dunia ini. Kalian seolah-olah sedang menggemukkan diri untuk hari penyembelihan.
\par 6 Kalian menghukum dan membunuh orang-orang yang tidak bersalah, dan mereka tidak melawan kalian.
\par 7 Sebab itu, sabarlah Saudara-saudaraku, sampai Tuhan datang. Lihatlah bagaimana sabarnya seorang petani menunggu sampai tanahnya memberikan hasil yang berharga kepadanya. Dengan sabar ia menunggu hujan musim gugur dan hujan musim bunga.
\par 8 Hendaklah kalian juga bersabar dan berbesar hati, sebab hari kedatangan Tuhan sudah dekat.
\par 9 Janganlah menggerutu dan saling menyalahkan, supaya kalian tidak dihukum oleh Allah. Ingat! Hakim sudah dekat, siap untuk datang.
\par 10 Saudara-saudara, ingatlah para nabi yang berbicara atas nama Tuhan. Mereka sabar dan tabah menderita. Jadi, ambillah mereka sebagai contoh.
\par 11 Mereka disebut berbahagia sebab mereka tabah. Kalian sudah mendengar tentang kesabaran Ayub, dan kalian tahu bagaimana Tuhan pada akhirnya memberkati dia. Sebab Tuhan sangat berbelaskasihan dan baik hati.
\par 12 Yang terutama sekali, ialah: Janganlah bersumpah, baik demi langit, maupun demi bumi, atau demi apa pun saja. Katakanlah saja "Ya" kalau maksudmu ya, dan "Tidak" kalau maksudmu tidak; supaya kalian jangan dijatuhi hukuman oleh Allah.
\par 13 Kalau di antaramu ada yang sedang susah, hendaklah ia berdoa. Kalau ada yang gembira, hendaklah ia menyanyi memuji Allah.
\par 14 Kalau ada yang sakit, hendaklah ia memanggil pemimpin-pemimpin jemaat. Dan hendaklah pemimpin-pemimpin itu mendoakan orang yang sakit itu dan mengolesnya dengan minyak atas nama Tuhan.
\par 15 Kalau doa mereka disampaikan dengan yakin, Tuhan akan menyembuhkan orang yang sakit itu, dan mengampuni dosa-dosa yang telah dibuatnya.
\par 16 Sebab itu, hendaklah kalian saling mengakui kesalahan dan saling mendoakan, supaya kalian disembuhkan. Doa orang yang menuruti kemauan Allah, mempunyai kuasa yang besar.
\par 17 Elia sama-sama manusia seperti kita. Ia berdoa dengan sungguh-sungguh supaya hujan tidak turun, maka hujan pun tidak turun selama tiga setengah tahun.
\par 18 Kemudian ia berdoa lagi, lalu langit menurunkan hujan sehingga tanah memberikan hasilnya pula.
\par 19 Saudara-saudaraku! Kalau di antaramu ada seseorang yang tidak mengikuti lagi ajaran dari Allah, lalu seorang yang lain membuat orang itu kembali kepada Allah,
\par 20 ingatlah ayat ini: Orang yang membuat orang berdosa berbalik dari jalannya yang salah, orang itu menyelamatkan jiwa orang berdosa itu dari kematian dan menyebabkan banyak dosa diampuni.


\end{document}