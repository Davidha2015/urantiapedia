\begin{document}

\title{1 Peter}

1Pe 1:1  Saudara-saudara umat pilihan Allah, yang tersebar di perantauan di daerah Pontus, Galatia, Kapadokia, Asia, dan Bitinia! Saya, Petrus, rasul Yesus Kristus, mengharap semoga Allah memberi berkat dan sejahtera kepadamu dengan berlimpah-limpah. Kalian dipilih menjadi umat Allah sesuai dengan rencana Allah Bapa. Dan Roh Allah sudah menjadikan kalian umat yang kudus, khusus untuk Allah, supaya kalian taat kepada Yesus Kristus dan disucikan oleh darah-Nya.
1Pe 1:3  Marilah kita bersyukur kepada Allah, Bapa Tuhan kita Yesus Kristus! Ia sangat mengasihani kita, itu sebabnya Ia memberikan kepada kita hidup yang baru, dengan menghidupkan kembali Yesus Kristus dari kematian. Ini memberikan kita harapan yang kokoh.
1Pe 1:4  Kita mengharap untuk memiliki berkat-berkat yang disediakan Allah bagi umat-Nya, yaitu berkat-berkat yang disimpan-Nya di surga, sehingga tidak dapat rusak, atau menjadi busuk ataupun luntur.
1Pe 1:5  Semuanya itu adalah untukmu, karena kalian percaya kepada Allah. Maka kalian dijaga dengan kuasa Allah supaya kalian menerima keselamatan yang siap dinyatakan pada akhir zaman nanti.
1Pe 1:6  Karena itu hendaklah kalian bersuka hati, meskipun sekarang untuk sementara waktu kalian harus menjadi sedih karena kalian mengalami bermacam-macam cobaan.
1Pe 1:7  Tujuannya ialah untuk membuktikan apakah kalian sungguh-sungguh percaya kepada Tuhan atau tidak. Emas yang dapat rusak pun, diuji dengan api. Nah, iman kalian adalah lebih berharga dari emas, jadi harus diuji juga supaya menjadi teguh. Dan dengan demikian kalian akan dipuji dan dihormati serta ditinggikan pada hari Yesus Kristus datang kembali.
1Pe 1:8  Kalian mengasihi-Nya, meskipun dahulu kalian tidak melihat-Nya. Dan kalian percaya kepada-Nya meskipun sekarang kalian tidak melihat-Nya. Itu sebabnya kalian bergembira, dan merasakan sukacita yang agung dan tak terkatakan.
1Pe 1:9  Sebab tujuan imanmu tercapai, yakni keselamatan jiwamu.
1Pe 1:10  Keselamatan inilah yang diteliti dan diselidiki oleh para nabi, dan mereka meramalkan tentang berkat yang diberikan Allah kepadamu.
1Pe 1:11  Mereka menyelidiki bagaimana dan kapan hal itu akan terjadi, yang dimaksudkan oleh Roh Kristus yang ada di dalam diri mereka. Pada waktu itu Roh-Nya menubuatkan kepada mereka tentang penderitaan-penderitaa yang harus dialami oleh Kristus dan tentang keagungan-Nya yang akan menyusul sesudah itu.
1Pe 1:12  Kepada nabi-nabi itu Allah memberitahukan bahwa apa yang mereka lakukan bukan untuk kepentingan mereka, melainkan untukmu; yakni hal-hal yang sekarang ini kalian dengar dengan jelas dari para pemberita Kabar Baik tentang Yesus Kristus. Mereka menyampaikan hal-hal itu kepadamu dengan kuasa Roh Allah yang dikirim-Nya dari surga. Bahkan malaikat-malaikat pun ingin juga mengetahui kabar yang mereka beritakan itu.
1Pe 1:13  Sebab itu, hendaklah kalian siap siaga. Waspadalah dan berharaplah sepenuhnya pada berkat yang akan diberikan kepadamu pada waktu Yesus Kristus datang nanti.
1Pe 1:14  Taatlah kepada Allah, dan janganlah hidup menurut keinginanmu yang dahulu, pada waktu kalian masih belum mengenal Allah.
1Pe 1:15  Sebaliknya, hendaklah kalian suci dalam segala sesuatu yang kalian lakukan, sama seperti Allah yang memanggil kalian itu suci.
1Pe 1:16  Dalam Alkitab tertulis, "Hendaklah kamu suci, sebab Aku suci."
1Pe 1:17  Kalian menyebut Allah itu Bapa pada waktu kalian berdoa kepada-Nya. Nah, Allah itulah yang menghakimi manusia setimpal dengan perbuatan masing-masing tanpa pandang bulu. Sebab itu selama kalian masih ada di dunia ini, hendaklah kalian mengagungkan Allah dalam hidupmu.
1Pe 1:18  Kalian tahu apa yang sudah dibayarkan untuk membebaskan kalian dari kehidupan yang sia-sia yang diwariskan oleh nenek moyangmu. Bayarannya bukanlah sesuatu yang bisa rusak seperti perak atau emas,
1Pe 1:19  melainkan sesuatu yang sangat berharga; yaitu diri Kristus sendiri, yang menjadi sebagai domba yang dikurbankan kepada Allah tanpa cacat atau cela.
1Pe 1:20  Kristus dipilih oleh Allah sebelum dunia diciptakan. Dan pada hari-hari menjelang akhir zaman ini, Ia datang ke dunia supaya kalian diselamatkan.
1Pe 1:21  Melalui Dialah kalian percaya kepada Allah yang sudah menghidupkan-Nya kembali dari kematian dan mengagungkan-Nya. Jadi, Allahlah yang kalian percayai dan kepada Allahlah juga kalian menaruh harapan.
1Pe 1:22  Karena kalian taat kepada ajaran dari Allah, maka kalian membersihkan diri dan kalian mengasihi orang-orang seiman secara ikhlas. Sebab itu, hendaklah kalian mengasihi satu sama lain dengan sepenuh hati.
1Pe 1:23  Sebab melalui sabda Allah yang hidup dan yang abadi itu, kalian sudah dijadikan manusia baru yang bukannya lahir dari manusia, melainkan dari Bapa yang abadi.
1Pe 1:24  Dalam Alkitab tertulis begini, "Seluruh umat manusia bagaikan rumput, dan segala kebesarannya seperti bunga rumput. Rumput layu, bunganya pun gugur;
1Pe 1:25  tetapi sabda Tuhan tetap untuk selama-lamanya." Sabda itu Kabar Baik yang sudah diberitakan kepadamu.
1Pe 2:1  Sebab itu, buanglah dari dirimu segala yang jahat; jangan lagi berdusta, dan jangan berpura-pura. Jangan iri hati, dan jangan menghina orang lain.
1Pe 2:2  Hendaklah kalian menjadi seperti bayi yang baru lahir, selalu haus akan susu rohani yang murni. Dengan demikian kalian akan tumbuh dan diselamatkan.
1Pe 2:3  Di dalam Alkitab tertulis begini, "Kamu sudah merasakan sendiri betapa baiknya Tuhan."
1Pe 2:4  Sebab itu, datanglah kepada Tuhan. Ia bagaikan batu yang hidup, batu yang dibuang oleh manusia karena dianggap tidak berguna; tetapi yang dipilih oleh Allah sebagai batu yang berharga.
1Pe 2:5  Kalian seperti batu-batu yang hidup. Sebab itu hendaklah kalian mau dipakai untuk membangun Rumah Allah yang rohani. Dengan demikian kalian menjadi imam-imam, yang hidup khusus untuk Allah, dan yang melalui Yesus Kristus mempersembahkan kepada Allah, kurban rohani yang berkenan di hati Allah.
1Pe 2:6  Karena di dalam Alkitab tertulis begini, "Aku telah memilih sebuah batu berharga, yang Kutempatkan di Sion sebagai batu utama; dan orang yang percaya kepada-Nya tidak akan dikecewakan."
1Pe 2:7  Batu itu sangat berharga untuk kalian yang percaya. Tetapi bagi orang-orang yang tidak percaya, berlakulah ayat-ayat Alkitab berikut ini, "Batu yang tidak terpakai oleh tukang-tukang bangunan itu ternyata menjadi batu yang terutama,"
1Pe 2:8  dan "Itulah batu yang membuat orang tersandung, batu yang membuat mereka jatuh." Mereka tersandung sebab mereka tidak percaya akan perkataan Allah. Begitulah sudah ditentukan Allah mengenai mereka.
1Pe 2:9  Tetapi kalian adalah bangsa yang terpilih, imam-imam yang melayani raja, bangsa yang kudus, khusus untuk Allah, umat Allah sendiri. Allah memilih kalian dan memanggil kalian keluar dari kegelapan untuk masuk ke dalam terang-Nya yang gemilang, dengan maksud supaya kalian menyebarkan berita tentang perbuatan-perbuatan-Nya yang luar biasa.
1Pe 2:10  Dahulu kalian bukan umat Allah, tetapi sekarang kalian adalah umat Allah. Dahulu kalian tidak dikasihani oleh Allah, tetapi sekarang kalian menerima belas kasihan-Nya.
1Pe 2:11  Saudara-saudara yang tercinta! Kalian adalah orang asing dan perantau di dunia ini. Sebab itu saya minta dengan sangat supaya kalian jangan menuruti hawa nafsu manusia yang selalu berperang melawan jiwa.
1Pe 2:12  Kelakuanmu di antara orang yang tidak mengenal Tuhan haruslah sangat baik, sehingga apabila mereka memfitnah kalian sebagai orang jahat, mereka toh harus mengakui perbuatanmu yang baik, sehingga mereka akan memuji Allah pada hari kedatangan-Nya.
1Pe 2:13  Demi Tuhan, hendaklah kalian tunduk kepada setiap penguasa manusia: baik kepada Kaisar yang menjadi penguasa yang terutama,
1Pe 2:14  maupun kepada gubernur yang ditunjuk oleh Kaisar untuk menghukum orang yang berbuat jahat dan untuk menghormati orang yang berbuat baik.
1Pe 2:15  Sebab Allah mau supaya dengan perbuatan-perbuatanmu yang baik kalian menutup mulut orang yang bercakap bodoh.
1Pe 2:16  Hendaklah kalian hidup sebagai orang merdeka, tetapi janganlah memakai kemerdekaanmu itu untuk menutupi kejahatan, melainkan hiduplah sebagai hamba Allah.
1Pe 2:17  Hargailah semua orang, kasihilah saudara-saudaramu sesama Kristen. Takutlah kepada Allah, dan hormatilah Kaisar.
1Pe 2:18  Saudara-saudara yang menjadi pelayan, tunduklah kepada majikanmu dengan sehormat-hormatnya; bukan hanya kepada mereka yang baik hati dan peramah, tetapi juga kepada mereka yang kejam.
1Pe 2:19  Allah akan memberkati kalian, kalau kalian karena sadar akan kemauan Allah, sabar menderita perlakuan yang tidak adil.
1Pe 2:20  Sebab apakah istimewanya kalau kalian sabar menderita hukuman yang seharusnya kalian tanggung karena bersalah? Tetapi kalau kalian dengan sabar menanggung penderitaan yang menimpamu karena berbuat yang benar, maka Allah akan memberkatimu.
1Pe 2:21  Untuk itulah Allah memanggilmu. Sebab Kristus sendiri sudah menderita untukmu, dan dengan itu Ia memberikan kepadamu suatu teladan, supaya kalian mengikuti jejak-Nya.
1Pe 2:22  Ia tidak pernah berbuat dosa, dan tidak pernah seorang pun mendengar Ia berdusta.
1Pe 2:23  Pada waktu Ia dicaci maki, Ia tidak membalas dengan caci maki. Sewaktu Ia menderita, Ia tidak mengancam; Ia hanya menyerahkan perkara-Nya kepada Allah, Hakim yang adil itu.
1Pe 2:24  Kristus sendiri memikul dosa-dosa kita pada diri-Nya di atas kayu salib, supaya kita bebas dari kekuasaan dosa, dan hidup menurut kemauan Allah. Luka-luka Kristuslah yang menyembuhkan kalian.
1Pe 2:25  Dahulu kalian seperti domba yang tersesat, tetapi sekarang kalian sudah dibawa kembali untuk mengikuti Gembala dan Pemelihara jiwamu.
1Pe 3:1  Begitu juga kalian, istri-istri, harus tunduk kepada suami supaya kalau di antara mereka ada yang tidak percaya kepada berita dari Allah, kelakuanmu dapat membuat mereka menjadi percaya. Dan tidak perlu kalian mengatakan apa-apa kepada mereka,
1Pe 3:2  sebab mereka melihat kelakuanmu yang murni dan saleh.
1Pe 3:3  Janganlah kecantikanmu hanya kecantikan luar, seperti misalnya menghias rambut atau memakai perhiasan, atau berpakaian yang mahal-mahal.
1Pe 3:4  Sebaliknya, hendaklah kecantikanmu timbul dari dalam batin, budi pekerti yang lemah lembut dan tenang; itulah kecantikan abadi yang sangat berharga menurut pandangan Allah.
1Pe 3:5  Dengan cara inilah pada zaman dahulu wanita-wanita beragama yang berharap kepada Allah mempercantik diri dengan tunduk kepada suami mereka.
1Pe 3:6  Sara pun begitu juga, ia taat kepada Abraham dan menyebut dia tuannya. Saudara sekarang adalah anak-anak Sara, kalau kalian melakukan hal-hal yang baik dan tidak takut kepada apa pun.
1Pe 3:7  Dan kalian juga, suami-suami, hendaklah hidup dengan penuh pengertian terhadap istrimu, dan dengan kesadaran bahwa mereka adalah kaum yang lemah. Perlakukanlah mereka dengan hormat, sebab mereka bersama-sama dengan kalian, akan menerima anugerah hidup yang sejati dari Allah. Lakukanlah ini, supaya tidak ada yang menghalangi doamu.
1Pe 3:8  Kesimpulannya ialah: hendaklah Saudara-saudara seia sekata dan seperasaan. Hendaklah kalian saling sayang-menyayangi seperti orang-orang yang bersaudara. Dan hendaklah kalian saling berbelaskasihan dan bersikap rendah hati.
1Pe 3:9  Janganlah membalas kejahatan dengan kejahatan, atau caci maki dengan caci maki; sebaliknya balaslah dengan memohonkan berkat dari Allah. Sebab Allah memanggil kalian justru supaya kalian menerima berkat daripada-Nya.
1Pe 3:10  Di dalam Alkitab tertulis begini, "Orang yang mau menikmati hidup dan mau mengalami hari-hari yang baik, harus menjaga mulutnya supaya tidak membicarakan hal-hal yang jahat dan tidak mengucapkan hal-hal yang dusta.
1Pe 3:11  Ia harus menjauhi yang jahat dan melakukan yang baik; hendaklah ia berjuang sungguh-sungguh untuk mendapatkan perdamaian.
1Pe 3:12  Sebab Tuhan selalu memperhatikan orang-orang yang menuruti kemauan-Nya, dan Tuhan selalu mendengar doa-doa mereka; tetapi Tuhan melawan orang-orang yang melakukan kejahatan."
1Pe 3:13  Siapakah yang mau berbuat jahat kepadamu, kalau kalian senang berbuat baik?
1Pe 3:14  Tetapi sekalipun kalian harus menderita karena melakukan hal-hal yang baik, kalian beruntung! Janganlah takut kepada siapa pun, dan jangan khawatir.
1Pe 3:15  Tetapi di dalam hatimu, hendaklah kalian memberikan kepada Kristus penghormatan yang khusus yang sesuai dengan kedudukan-Nya sebagai Tuhan. Dan hendaklah kalian selalu siap untuk memberi jawaban kepada setiap orang yang bertanya mengenai harapan yang kalian miliki.
1Pe 3:16  Tetapi lakukanlah itu dengan lemah lembut dan hormat. Dan hendaklah hati nuranimu murni, supaya kalau kalian difitnah karena kalian hidup dengan baik sebagai pengikut Kristus, maka orang yang memfitnah itu akan menjadi malu sendiri.
1Pe 3:17  Lebih baik menderita karena berbuat baik--kalau itu adalah kemauan Allah--daripada menderita karena melakukan yang jahat.
1Pe 3:18  Sebab Kristus sendiri mati hanya sekali saja, untuk selama-lamanya karena dosa manusia--seorang yang tidak bersalah, mati untuk orang yang bersalah. Kristus melakukan itu supaya Ia dapat membimbing kalian kepada Allah. Ia dibunuh secara jasmani, tetapi dihidupkan kembali secara rohani.
1Pe 3:19  Dalam keadaan roh Ia pergi mengabarkan berita dari Allah kepada roh-roh yang dipenjarakan:
1Pe 3:20  yaitu roh orang-orang yang tidak taat kepada Allah pada zaman Nuh. Pada waktu itu Allah menanti dengan sabar selama Nuh membuat kapalnya. Hanya orang-orang yang ada di kapal saja--semuanya delapan orang--yang diselamatkan melalui banjir besar itu.
1Pe 3:21  Nah, kejadian itu merupakan kiasan dari baptisan yang sekarang ini menyelamatkan kalian. Baptisan ini bukanlah suatu upacara membersihkan badan dari semua yang kotor-kotor, melainkan merupakan janjimu kepada Allah dari hati nurani yang baik. Baptisan itu menyelamatkan kalian karena Yesus Kristus sudah hidup kembali dari kematian,
1Pe 3:22  dan sudah naik ke surga. Sekarang Ia berkuasa bersama dengan Allah dan memerintah semua malaikat, semua penguasa serta semua kekuatan.
1Pe 4:1  Karena Kristus sudah menderita secara jasmani, kalian juga harus memperkuat diri dengan pendirian yang seperti itu. Sebab orang yang menderita secara badani, tidak lagi berbuat dosa.
1Pe 4:2  Sebab itu, hendaklah kalian hidup selanjutnya di dunia ini menurut kemauan Allah, dan bukan menurut keinginan manusia.
1Pe 4:3  Pada masa yang lampau, kalian sudah cukup lama melakukan apa yang suka dilakukan oleh orang-orang yang tidak mengenal Tuhan. Kalian hidup tidak baik. Kalian menuruti hawa nafsu, mabuk-mabuk, berpesta-pora, minum-minum dan menyembah berhala secara menjijikkan.
1Pe 4:4  Sekarang orang yang tidak mengenal Tuhan, heran bahwa kalian tidak ikut hidup seperti mereka, menikmati dosa tanpa batas. Itulah sebabnya mereka menghina kalian.
1Pe 4:5  Tetapi mereka nanti harus memberi pertanggungjawaban kepada Allah yang sudah siap untuk mengadili orang yang hidup dan yang mati.
1Pe 4:6  Itulah sebabnya Kabar Baik sudah diberitakan juga kepada orang-orang mati. Maksudnya supaya mereka, yang telah diadili dalam keadaan jasmani--seperti halnya semua orang akan diadili--dapat hidup secara rohani, menurut kehendak Allah.
1Pe 4:7  Segala sesuatu sudah mendekati kesudahannya, sebab itu hendaklah kalian menguasai diri dan waspada, supaya kalian dapat berdoa.
1Pe 4:8  Lebih daripada segala-galanya, hendaklah kalian sungguh-sungguh mengasihi satu sama lain, sebab dengan saling mengasihi kalian akan bersedia juga untuk saling mengampuni.
1Pe 4:9  Hendaklah kalian menerima satu sama lain di rumah masing-masing, tanpa mengeluh.
1Pe 4:10  Kalian masing-masing sudah menerima pemberian-pemberian yang berbeda-beda dari Allah. Sebab itu sebagai pengelola yang baik dari pemberian-pemberian Allah, hendaklah kalian menggunakan kemampuan itu untuk kepentingan bersama.
1Pe 4:11  Orang yang menyampaikan berita, haruslah menyampaikan berita dari Allah; orang yang melayani orang lain, haruslah melayani dengan kekuatan yang dari Allah, supaya dalam segala hal, Allah dapat diagungkan melalui Yesus Kristus. Dialah yang berkuasa dan patut diagungkan untuk selama-lamanya. Amin.
1Pe 4:12  Saudara-saudara yang tercinta! Kalian sekarang sangat menderita karena sedang diuji. Tetapi janganlah merasa heran, seolah-olah ada sesuatu yang luar biasa terjadi kepadamu.
1Pe 4:13  Malah, hendaklah kalian merasa senang karena turut menderita bersama-sama dengan Kristus. Maka nanti kalau keagungan Kristus diperlihatkan, kalian akan bersukacita.
1Pe 4:14  Kalian beruntung kalau kalian dihina sebab kalian pengikut Kristus. Itu berarti Roh yang mulia, yaitu Roh Allah, ada padamu.
1Pe 4:15  Janganlah seorang pun dari antara kalian menderita karena ia pembunuh, atau pencuri, atau penjahat, atau karena ia mencampuri urusan orang lain.
1Pe 4:16  Tetapi kalau kalian menderita sebab kalian orang Kristen, janganlah malu karena hal itu. Berterimakasihlah kepada Allah, bahwa kalian membawa nama Kristus.
1Pe 4:17  Sudah sampai waktunya Allah mengadili dunia. Dan umat Allah sendirilah yang akan diadili terlebih dahulu. Nah, kalau Allah akan mulai dengan kita, bagaimanakah jadinya nanti dengan orang-orang yang tidak percaya kepada Kabar Baik dari Allah itu?
1Pe 4:18  Dalam Alkitab tertulis begini, "Kalau orang-orang yang baik pun sudah sukar untuk diselamatkan, apa pula yang akan terjadi dengan orang-orang berdosa yang tidak mengenal Tuhan!"
1Pe 4:19  Sebab itu, kalau ada orang-orang yang menderita karena Allah menghendaki demikian, hendaklah orang-orang itu hidup dengan benar dan mempercayakan diri kepada Pencipta mereka. Ia akan selalu menepati janji-Nya.
1Pe 5:1  Saya minta perhatian para pemimpin jemaatmu. Saya menulis ini sebagai rekan pemimpin dan sebagai orang yang sudah menyaksikan sendiri penderitaan Kristus, dan yang akan turut juga diagungkan, apabila keagungan Kristus diperlihatkan kepada manusia. Saya minta dengan sangat
1Pe 5:2  supaya kalian menggembalakan kawanan domba yang diserahkan Allah kepadamu. Gembalakanlah mereka dengan senang hati sebagaimana yang diinginkan oleh Allah, dan janganlah dengan berat hati. Janganlah pula melakukan pekerjaanmu guna mendapat keuntungan, melainkan karena kalian sungguh-sungguh ingin melayani.
1Pe 5:3  Janganlah bertindak sebagai penguasa terhadap mereka yang dipercayakan kepadamu, melainkan jadilah teladan untuk mereka.
1Pe 5:4  Dan pada waktu Gembala Agung itu datang nanti, kalian akan menerima mahkota yang gemilang, yang tidak akan pudar kegemilangannya.
1Pe 5:5  Begitu juga Saudara, orang-orang muda. Tunduklah kepada orang-orang yang tua. Saudara semuanya harus merendahkan diri dan saling melayani dengan rendah hati. Sebab dalam Alkitab tertulis begini, "Allah menentang orang yang sombong, tetapi mengasihani orang yang rendah hati."
1Pe 5:6  Sebab itu, rendahkanlah dirimu di bawah tangan Allah yang kuat, supaya Ia meninggikan kalian kalau sudah waktunya.
1Pe 5:7  Serahkanlah segala kekhawatiranmu kepada Allah, sebab Ia mempedulikanmu.
1Pe 5:8  Hendaklah kalian waspada dan siap siaga! Sebab Iblis adalah musuhmu. Ia seperti singa berjalan ke sana kemari sambil mengaum mencari mangsanya.
1Pe 5:9  Lawanlah dia dengan iman yang teguh, sebab kalian tahu bahwa saudara-saudara sesama Kristen lainnya di seluruh dunia juga menderita seperti kalian.
1Pe 5:10  Tetapi sesudah kalian menderita sebentar, Allah sendiri akan membuat kalian menjadi sempurna. Ia akan menegakkan dan menguatkan kalian serta mengokohkan kalian. Sebab Ia adalah Allah yang sangat baik hati, yang sudah memanggilmu untuk turut merasakan keagungan-Nya yang abadi melalui Kristus.
1Pe 5:11  Ialah yang berkuasa sampai selama-lamanya! Amin.
1Pe 5:12  Surat saya yang singkat ini saya tulis kepadamu dengan bantuan Silas, yang saya anggap seorang saudara Kristen yang setia. Saya ingin memberi dorongan kepadamu dan meyakinkan kalian bahwa apa yang saya tulis ini berita yang benar mengenai rahmat Allah. Hendaklah kalian berpegang teguh pada rahmat itu.
1Pe 5:13  Teman-teman sejemaat di Babilon, yang juga terpilih oleh Allah, mengirim salam kepadamu; begitu juga anakku Markus yang tercinta.
1Pe 5:14  Bersalam-salamanlah satu sama lain dengan penuh kasih sebagai saudara Kristen. Semoga Tuhan memberi sejahtera kepada Saudara semuanya yang menjadi milik Kristus. Hormat kami, Petrus.


\end{document}