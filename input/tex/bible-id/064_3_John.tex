\begin{document}

\title{3 Yohanes}


\chapter{1}

\par 1 Saudara Gayus yang tercinta! Saya, pemimpin jemaat, yang sungguh-sungguh mengasihimu,
\par 2 berdoa semoga Saudara sehat-sehat dan semuanya baik-baik denganmu, sama seperti jiwamu baik-baik saja.
\par 3 Saya senang sekali ketika beberapa saudara Kristen tiba dan memberitahukan kepada saya betapa setianya Saudara terhadap Allah yang benar, sebab memang Saudara selalu hidup menurut ajaran Allah.
\par 4 Tidak ada yang dapat menjadikan saya lebih gembira, daripada mendengar bahwa anak-anak saya hidup menurut ajaran Allah.
\par 5 Saudaraku! Saudara begitu setia dalam pekerjaan yang Saudara lakukan bagi teman-teman sesama Kristen; bahkan orang Kristen yang belum Saudara kenal pun, Saudara layani.
\par 6 Mereka sudah memberitahukan kepada jemaat di tempat kami mengenai kasihmu. Tolonglah mereka supaya dapat meneruskan perjalanan mereka dan dengan demikian menyenangkan hati Allah,
\par 7 sebab dalam perjalanan untuk melayani Kristus, mereka tidak menerima bantuan apa pun dari orang-orang yang tidak mengenal Allah.
\par 8 Sebab itu, kita orang-orang Kristen harus menolong teman-teman sesama Kristen yang seperti itu, supaya kita dapat turut dalam usaha mereka untuk menyebarkan ajaran yang benar dari Allah.
\par 9 Saya sudah menulis surat yang pendek kepada jemaat, tetapi Diotrefes yang ingin menjadi pemimpin di dalam jemaat, tidak mau menuruti saya.
\par 10 Nanti kalau saya datang, saya akan membongkar segala sesuatu yang telah dilakukannya, yaitu tentang hal-hal jahat dan dusta yang ia ucapkan tentang kami! Tetapi semuanya itu belum cukup untuk dia. Pada waktu teman-teman sesama Kristen itu datang, ia tidak mau menerima mereka. Malah ia melarang orang-orang yang mau menerima saudara-saudara itu, dan ia menyuruh orang-orang itu keluar dari jemaat!
\par 11 Saudara yang tercinta, janganlah meniru apa yang buruk. Tirulah apa yang baik. Orang yang berbuat baik adalah milik Allah. Dan orang yang berbuat jahat belum mengenal Allah.
\par 12 Semua orang memuji Demetrius. Bahkan Allah pun memujinya. Kami juga memujinya, dan Saudara tahu bahwa apa yang kami sampaikan dapat dipercaya.
\par 13 Banyak sekali yang perlu saya katakan kepadamu, tetapi saya rasa lebih baik jangan melalui surat.
\par 14 Saya berharap tidak lama lagi akan berjumpa denganmu, dan kita akan berbicara langsung.
\par 15 Semoga Tuhan memberkati Saudara. Kawan-kawanmu di tempat saya ini mengirim salam kepadamu. Sampaikan salam saya kepada semua kawan-kawan kita satu persatu.


\end{document}