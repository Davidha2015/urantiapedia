\begin{document}

\title{Exodus}


\chapter{1}

\par 1 Or questi sono i nomi dei figliuoli d'Israele che vennero in Egitto. Essi ci vennero con Giacobbe, ciascuno con la sua famiglia:
\par 2 Ruben, Simeone, Levi e Giuda;
\par 3 Issacar, Zabulon e Beniamino;
\par 4 Dan e Neftali, Gad e Ascer.
\par 5 Tutte le persone discendenti da Giacobbe ammontavano a settanta. Giuseppe era già in Egitto.
\par 6 E Giuseppe morì, come moriron pure tutti i suoi fratelli e tutta quella generazione.
\par 7 E i figliuoli d'Israele furon fecondi, moltiplicarono copiosamente, diventarono numerosi e si fecero oltremodo potenti, e il paese ne fu ripieno.
\par 8 Or sorse sopra l'Egitto un nuovo re, che non avea conosciuto Giuseppe.
\par 9 Egli disse al suo popolo: 'Ecco, il popolo de' figliuoli d'Israele è più numeroso e più potente di noi.
\par 10 Orsù, usiamo prudenza con essi; che non abbiano a moltiplicare e, in caso di guerra, non abbiano a unirsi ai nostri nemici e combattere contro di noi e poi andarsene dal paese'.
\par 11 Stabilirono dunque sopra Israele de' soprastanti ai lavori, che l'opprimessero con le loro angherie. Ed esso edificò a Faraone le città di approvvigionamento, Pithom e Raamses.
\par 12 Ma più l'opprimevano, e più il popolo moltiplicava e s'estendeva; e gli Egiziani presero in avversione i figliuoli d'Israele,
\par 13 e fecero servire i figliuoli d'Israele con asprezza,
\par 14 e amareggiaron loro la vita con una dura servitù, adoprandoli nei lavori d'argilla e di mattoni, e in ogni sorta di lavori nei campi. E imponevano loro tutti questi lavori, con asprezza.
\par 15 Il re d'Egitto parlò anche alle levatrici degli Ebrei, delle quali l'una si chiamava Scifra e l'altra Pua. E disse:
\par 16 'Quando assisterete le donne ebree al tempo del parto, e le vedrete sulla seggiola, se è un maschio, uccidetelo; ma se è una femmina, lasciatela vivere'.
\par 17 Ma le levatrici temettero Iddio, e non fecero quello che il re d'Egitto aveva ordinato loro; lasciarono vivere i maschi.
\par 18 Allora il re d'Egitto chiamò le levatrici, e disse loro: 'Perché avete fatto questo, e avete lasciato vivere i maschi?'
\par 19 E le levatrici risposero a Faraone: 'Egli è che le donne ebree non sono come le egiziane; sono vigorose, e, prima che la levatrice arrivi da loro, hanno partorito'.
\par 20 E Dio fece del bene a quelle levatrici; e il popolo moltiplicò e divenne oltremodo potente.
\par 21 E perché quelle levatrici temettero Iddio, egli fece prosperare le loro case.
\par 22 Allora Faraone diede quest'ordine al suo popolo: 'Ogni maschio che nasce, gettatelo nel fiume; ma lasciate vivere tutte le femmine'.

\chapter{2}

\par 1 Or un uomo della casa di Levi andò e prese per moglie una figliuola di Levi.
\par 2 Questa donna concepì, e partorì un figliuolo; e vedendo com'egli era bello, lo tenne nascosto tre mesi.
\par 3 E quando non poté più tenerlo nascosto, prese un canestro fatto di giunchi, lo spalmò di bitume e di pece, vi pose dentro il bambino, e lo mise nel canneto sulla riva del fiume.
\par 4 E la sorella del bambino se ne stava a una certa distanza, per sapere quel che gli succederebbe.
\par 5 Or la figliuola di Faraone scese a fare le sue abluzioni sulla riva del fiume; e le sue donzelle passeggiavano lungo il fiume. Ella vide il canestro nel canneto, e mandò la sua cameriera a prenderlo.
\par 6 L'aprì, e vide il bimbo; ed ecco, il piccino piangeva; ed ella n'ebbe compassione, e disse: 'Questo è uno de' figliuoli degli Ebrei'.
\par 7 Allora la sorella del bambino disse alla figliuola di Faraone: 'Devo andare a chiamarti una balia tra le donne ebree che t'allatti questo bimbo?'
\par 8 La figliuola di Faraone le rispose: 'Va''. E la fanciulla andò a chiamare la madre del bambino.
\par 9 E la figliuola di Faraone le disse: 'Porta via questo bambino, allattamelo, e io ti darò il tuo salario'. E quella donna prese il bambino e l'allattò.
\par 10 E quando il bambino fu cresciuto, ella lo menò dalla figliuola di Faraone: esso fu per lei come un figliuolo, ed ella gli pose nome Mosè; 'Perché, disse, io l'ho tratto dall'acqua'.
\par 11 Or in que' giorni, quando Mosè era già diventato grande, avvenne ch'egli uscì a trovare i suoi fratelli, e notò i lavori di cui erano gravati; e vide un Egiziano, che percoteva uno degli Ebrei suoi fratelli.
\par 12 Egli volse lo sguardo di qua e di là; e, visto che non c'era nessuno, uccise l'Egiziano, e lo nascose nella sabbia.
\par 13 Il giorno seguente uscì, ed ecco due Ebrei che si litigavano; ed egli disse a quello che avea torto: 'Perché percuoti il tuo compagno?'
\par 14 E quegli rispose: 'Chi t'ha costituito principe e giudice sopra di noi? Vuoi tu uccider me come uccidesti l'Egiziano?' Allora Mosè ebbe paura, e disse: 'Certo, la cosa è nota'.
\par 15 E quando Faraone udì il fatto, cercò di uccidere Mosè; ma Mosè fuggì dal cospetto di Faraone, e si fermò nel paese di Madian; e si mise a sedere presso ad un pozzo.
\par 16 Or il sacerdote di Madian aveva sette figliuole; ed esse vennero ad attinger acqua, e a riempire gli abbeveratoi per abbeverare il gregge del padre loro.
\par 17 Ma sopraggiunsero i pastori, che le scacciarono. Allora Mosè si levò, prese la loro difesa, e abbeverò il loro gregge.
\par 18 E com'esse giunsero da Reuel loro padre, questi disse: 'Come mai siete tornate così presto oggi?'
\par 19 Ed esse risposero: 'Un Egiziano ci ha liberate dalle mani de' pastori, e di più ci ha attinto l'acqua, ed ha abbeverato il gregge'.
\par 20 Ed egli disse alle sue figliuole: 'E dov'è? Chiamatelo, che prenda qualche cibo'.
\par 21 E Mosè acconsentì a stare da quell'uomo; ed egli diede a Mosè Sefora, sua figliuola.
\par 22 Ed ella partorì un figliuolo ch'egli chiamò Ghershom; 'perché, disse, io soggiorno in terra straniera'.
\par 23 Or nel corso di quel tempo, che fu lungo, avvenne che il re d'Egitto morì; e i figliuoli d'Israele sospiravano a motivo della schiavitù, e alzavan delle grida; e le grida che il servaggio strappava loro salirono a Dio.
\par 24 E Dio udì i loro gemiti; e Dio si ricordò del suo patto con Abrahamo, con Isacco e con Giacobbe.
\par 25 E Dio vide i figliuoli d'Israele, e Dio ebbe riguardo alla loro condizione.

\chapter{3}

\par 1 Or Mosè pasceva il gregge di Jethro suo suocero, sacerdote di Madian; e guidando il gregge dietro al deserto, giunse alla montagna di Dio, a Horeb.
\par 2 E l'angelo dell'Eterno gli apparve in una fiamma di fuoco, di mezzo a un pruno. Mosè guardò, ed ecco il pruno era tutto in fiamme, ma non si consumava.
\par 3 E Mosè disse: 'Ora voglio andar da quella parte a vedere questa grande visione e come mai il pruno non si consuma!'
\par 4 E l'Eterno vide ch'egli s'era scostato per andare a vedere. E Dio lo chiamò di mezzo al pruno, e disse: 'Mosè! Mosè!' Ed egli rispose: 'Eccomi'.
\par 5 E Dio disse: 'Non t'avvicinar qua; togliti i calzari dai piedi, perché il luogo sul quale stai, è suolo sacro'.
\par 6 Poi aggiunse: 'Io sono l'Iddio di tuo padre, l'Iddio d'Abrahamo, l'Iddio d'Isacco e l'Iddio di Giacobbe'. E Mosè si nascose la faccia, perché avea paura di guardare Iddio.
\par 7 E l'Eterno disse: 'Ho veduto, ho veduto l'afflizione del mio popolo che è in Egitto, e ho udito il grido che gli strappano i suoi angariatori; perché conosco i suoi affanni;
\par 8 e sono sceso per liberarlo dalla mano degli Egiziani, e per farlo salire da quel paese in un paese buono e spazioso, in un paese ove scorre il latte e il miele, nel luogo dove sono i Cananei, gli Hittei, gli Amorei, i Ferezei, gli Hivvei e i Gebusei.
\par 9 Ed ora, ecco, le grida de' figliuoli d'Israele son giunte a me, ed ho anche veduto l'oppressione che gli Egiziani fanno loro soffrire.
\par 10 Or dunque vieni, e io ti manderò a Faraone perché tu faccia uscire il mio popolo, i figliuoli d'Israele, dall'Egitto'.
\par 11 E Mosè disse a Dio: 'Chi son io per andare da Faraone e per trarre i figliuoli d'Israele dall'Egitto?'
\par 12 E Dio disse: 'Va', perché io sarò teco; e questo sarà per te il segno che son io che t'ho mandato: quando avrai tratto il popolo dall'Egitto, voi servirete Iddio su questo monte'.
\par 13 E Mosè disse a Dio: 'Ecco, quando sarò andato dai figliuoli d'Israele e avrò detto loro: L'Iddio de' vostri padri m'ha mandato da voi, se essi mi dicono: Qual è il suo nome? che risponderò loro?'
\par 14 Iddio disse a Mosè: 'Io sono quegli che sono'. Poi disse: 'Dirai così ai figliuoli d'Israele: L'Io sono m'ha mandato da voi'.
\par 15 Iddio disse ancora a Mosè: 'Dirai così ai figliuoli d'Israele: L'Eterno, l'Iddio de' vostri padri, l'Iddio d'Abrahamo, l'Iddio d'Isacco e l'Iddio di Giacobbe mi ha mandato da voi. Tale è il mio nome in perpetuo, tale la mia designazione per tutte le generazioni.
\par 16 Va' e raduna gli anziani d'Israele, e di' loro: L'Eterno, l'Iddio de' vostri padri, l'Iddio d'Abrahamo, d'Isacco e di Giacobbe m'è apparso, dicendo: Certo, io vi ho visitati, e ho veduto quello che vi si fa in Egitto;
\par 17 e ho detto: Io vi trarrò dall'afflizione d'Egitto, e vi farò salire nel paese dei Cananei, degli Hittei, degli Amorei, de' Ferezei, degli Hivvei e de' Gebusei, in un paese ove scorre il latte e il miele.
\par 18 Ed essi ubbidiranno alla tua voce; e tu, con gli anziani d'Israele, andrai dal re d'Egitto, e gli direte: L'Eterno, l'Iddio degli Ebrei, ci è venuto incontro; or dunque, lasciaci andare tre giornate di cammino nel deserto per offrir sacrifizi all'Eterno, all'Iddio nostro.
\par 19 Or io so che il re d'Egitto non vi concederà d'andare, se non forzato da una potente mano.
\par 20 E io stenderò la mia mano e percoterò l'Egitto con tutti i miracoli che io farò in mezzo ad esso; e, dopo questo, vi lascerà andare.
\par 21 E farò sì che questo popolo trovi favore presso gli Egiziani; e avverrà che, quando ve ne andrete, non ve ne andrete a mani vuote;
\par 22 ma ogni donna domanderà alla sua vicina e alla sua casigliana degli oggetti d'argento, degli oggetti d'oro e dei vestiti; voi li metterete addosso ai vostri figliuoli e alle vostre figliuole, e così spoglierete gli Egiziani'.

\chapter{4}

\par 1 Mosè rispose e disse: 'Ma ecco, essi non mi crederanno e non ubbidiranno alla mia voce, perché diranno: L'Eterno non t'è apparso'.
\par 2 E l'Eterno gli disse: 'Che è quello che hai in mano?' Egli rispose: 'Un bastone'.
\par 3 E l'Eterno disse: 'Gettalo in terra'. Egli lo gettò in terra, ed esso diventò un serpente; e Mosè fuggì d'innanzi a quello.
\par 4 Allora l'Eterno disse a Mosè: 'Stendi la tua mano, e prendilo per la coda'. Egli stese la mano, e lo prese, ed esso ritornò un bastone nella sua mano.
\par 5 'Questo farai, disse l'Eterno, affinché credano che l'Eterno, l'Iddio dei loro padri, l'Iddio d'Abrahamo, l'Iddio d'Isacco e l'Iddio di Giacobbe t'è apparso'.
\par 6 L'Eterno gli disse ancora: 'Mettiti la mano in seno'. Ed egli si mise la mano in seno; poi, cavatala fuori, ecco che la mano era lebbrosa, bianca come neve.
\par 7 E l'Eterno gli disse: 'Rimettiti la mano in seno'. Egli si rimise la mano in seno; poi, cavatasela di seno, ecco ch'era ritornata come l'altra sua carne.
\par 8 'Or avverrà, disse l'Eterno, che, se non ti crederanno e non daranno ascolto alla voce del primo segno, crederanno alla voce del secondo segno;
\par 9 e se avverrà che non credano neppure a questi due segni e non ubbidiscano alla tua voce, tu prenderai dell'acqua del fiume, e la verserai sull'asciutto; e l'acqua che avrai presa dal fiume, diventerà sangue sull'asciutto'.
\par 10 E Mosè disse all'Eterno: 'Ahimè, Signore, io non sono un parlatore; non lo ero in passato, e non lo sono da quando tu hai parlato al tuo servo; giacché io sono tardo di parola e di lingua'.
\par 11 E l'Eterno gli disse: 'Chi ha fatto la bocca dell'uomo? o chi rende muto o sordo o veggente o cieco? non son io, l'Eterno?
\par 12 Or dunque va', e io sarò con la tua bocca, e t'insegnerò quello che dovrai dire'.
\par 13 E Mosè disse: 'Deh! Signore, manda il tuo messaggio per mezzo di chi vorrai!'
\par 14 Allora l'ira dell'Eterno s'accese contro Mosè, ed egli disse: 'Non c'è Aaronne tuo fratello, il Levita? Io so che parla bene. E per l'appunto, ecco ch'egli esce ad incontrarti; e, come ti vedrà, si rallegrerà in cuor suo.
\par 15 Tu gli parlerai, e gli metterai le parole in bocca; io sarò con la tua bocca e con la bocca sua, e v'insegnerò quello che dovrete fare.
\par 16 Egli parlerà per te al popolo; e così ti servirà di bocca, e tu sarai per lui come Dio.
\par 17 Or prendi in mano questo bastone col quale farai i prodigi'.
\par 18 Allora Mosè se ne andò, tornò da Jethro suo suocero, e gli disse: 'Deh, lascia ch'io me ne vada e torni dai miei fratelli che sono in Egitto, e vegga se sono ancor vivi'. E Jethro disse a Mosè: 'Va' in pace'.
\par 19 Or l'Eterno disse a Mosè in Madian: 'Va', tornatene in Egitto, perché tutti quelli che cercavano di toglierti la vita sono morti'.
\par 20 Mosè dunque prese la sua moglie e i suoi figliuoli, li pose su degli asini, e tornò nel paese d'Egitto; e Mosè prese nella sua mano il bastone di Dio.
\par 21 E l'Eterno disse a Mosè: 'Quando sarai tornato in Egitto, avrai cura di fare dinanzi a Faraone tutti i prodigi che t'ho dato potere di compiere; ma io gl'indurerò il cuore, ed egli non lascerà partire il popolo.
\par 22 E tu dirai a Faraone: Così dice l'Eterno: Israele è il mio figliuolo, il mio primogenito;
\par 23 e io ti dico: Lascia andare il mio figliuolo, affinché mi serva; e se tu ricusi di lasciarlo andare, ecco, io ucciderò il tuo figliuolo, il tuo primogenito'.
\par 24 Or avvenne che, essendo Mosè in viaggio, nel luogo dov'egli albergava, l'Eterno gli si fece incontro, e cercò di farlo morire.
\par 25 Allora Sefora prese una selce tagliente, recise il prepuzio del suo figliuolo, e lo gettò ai piedi di Mosè, dicendo: 'Sposo di sangue tu mi sei!'
\par 26 E l'Eterno lo lasciò. Allora ella disse: 'Sposo di sangue, per via della circoncisione'.
\par 27 L'Eterno disse ad Aaronne: 'Va' nel deserto incontro a Mosè'. Ed egli andò, lo incontrò al monte di Dio, e lo baciò.
\par 28 E Mosè riferì ad Aaronne tutte le parole che l'Eterno l'aveva incaricato di dire, e tutti i segni portentosi che gli aveva ordinato di fare.
\par 29 Mosè ed Aaronne dunque andarono, e radunarono tutti gli anziani dei figliuoli d'Israele.
\par 30 E Aaronne riferì tutte le parole che l'Eterno avea dette a Mosè, e fece i prodigi in presenza del popolo.
\par 31 Ed il popolo prestò loro fede. Essi intesero che l'Eterno aveva visitato i figliuoli d'Israele e aveva veduto la loro afflizione, e si inchinarono e adorarono.

\chapter{5}

\par 1 Dopo questo, Mosè ed Aaronne vennero a Faraone, e gli dissero: 'Così dice l'Eterno, l'Iddio d'Israele: Lascia andare il mio popolo, perché mi celebri una festa nel deserto'.
\par 2 Ma Faraone rispose: 'Chi è l'Eterno ch'io debba ubbidire alla sua voce e lasciar andare Israele? Io non conosco l'Eterno, e non lascerò affatto andare Israele'.
\par 3 Ed essi dissero: 'L'Iddio degli Ebrei si è presentato a noi; lasciaci andare tre giornate di cammino nel deserto per offrir sacrifizi all'Eterno, ch'è il nostro Dio, onde ei non abbia a colpirci con la peste o con la spada'.
\par 4 E il re d'Egitto disse loro: 'O Mosè e Aaronne, perché distraete il popolo dai suoi lavori? Andate a fare quello che vi è imposto!'
\par 5 E Faraone disse: 'Ecco, il popolo è ora numeroso nel paese, e voi gli fate interrompere i lavori che gli sono imposti'.
\par 6 E quello stesso giorno Faraone dette quest'ordine agli ispettori del popolo e ai suoi sorveglianti:
\par 7 'Voi non darete più, come prima, la paglia al popolo per fare i mattoni; vadano essi a raccogliersi della paglia!
\par 8 E imponete loro la stessa quantità di mattoni di prima, senza diminuzione alcuna; perché son de' pigri; e però gridano dicendo: Andiamo a offrir sacrifizi al nostro Dio!
\par 9 Sia questa gente caricata di lavoro; e si occupi di quello senza badare a parole di menzogna'.
\par 10 Allora gl'ispettori del popolo e i sorveglianti uscirono e dissero al popolo: 'Così dice Faraone: Io non vi darò più paglia.
\par 11 Andate voi a procurarvi della paglia dove ne potrete trovare, perché il vostro lavoro non sarà diminuito per nulla'.
\par 12 Così il popolo si sparse per tutto il paese d'Egitto, per raccogliere della stoppia invece di paglia.
\par 13 E gli ispettori li sollecitavano dicendo: 'Compite i vostri lavori giorno per giorno, come quando c'era la paglia!'
\par 14 E i sorveglianti de' figliuoli d'Israele stabiliti sopra loro dagli ispettori di Faraone, furon battuti; e fu loro detto: 'Perché non avete fornito, ieri e oggi come prima, la quantità di mattoni che v'è imposta?'
\par 15 Allora i sorveglianti dei figliuoli d'Israele vennero a lagnarsi da Faraone, dicendo: 'Perché tratti così i tuoi servitori?
\par 16 Non si dà più paglia ai tuoi servitori, e ci si dice: Fate de' mattoni! ed ecco che i tuoi servitori sono battuti, e il tuo popolo è considerato come colpevole!'
\par 17 Ed egli rispose: 'Siete dei pigri! siete dei pigri! Per questo dite: Andiamo a offrir sacrifizi all'Eterno.
\par 18 Or dunque andate a lavorare! non vi si darà più paglia e fornirete la quantità di mattoni prescritta'.
\par 19 I sorveglianti de' figliuoli d'Israele si videro ridotti a mal partito, perché si diceva loro: 'Non diminuite per nulla il numero de' mattoni impostovi giorno per giorno'.
\par 20 E, uscendo da Faraone, incontrarono Mosè e Aaronne, che stavano ad aspettarli,
\par 21 e dissero loro: 'L'Eterno volga il suo sguardo su voi, e giudichi! poiché ci avete messi in cattivo odore dinanzi a Faraone e dinanzi ai suoi servitori, e avete loro messa la spada in mano perché ci uccida'.
\par 22 Allora Mosè tornò dall'Eterno, e disse: 'Signore, perché hai fatto del male a questo popolo? Perché dunque mi hai mandato?
\par 23 Poiché, da quando sono andato da Faraone per parlargli in tuo nome, egli ha maltrattato questo popolo, e tu non hai affatto liberato il tuo popolo'.

\chapter{6}

\par 1 L'Eterno disse a Mosè: 'Ora vedrai quello che farò a Faraone; perché, forzato da una mano potente, li lascerà andare; anzi, forzato da una mano potente, li caccerà dal suo paese'.
\par 2 E Dio parlò a Mosè, e gli disse:
\par 3 'Io sono l'Eterno, e apparii ad Abrahamo, ad Isacco e a Giacobbe, come l'Iddio onnipotente; ma non fui conosciuto da loro sotto il mio nome di Eterno.
\par 4 Stabilii pure con loro il mio patto, promettendo di dar loro il paese di Canaan, il paese dei loro pellegrinaggi, nel quale soggiornavano.
\par 5 Ed ho anche udito i gemiti de' figliuoli d'Israele che gli Egiziani tengono in schiavitù, e mi son ricordato del mio patto.
\par 6 Perciò di' ai figliuoli d'Israele: Io sono l'Eterno, vi sottrarrò ai duri lavori di cui vi gravano gli Egiziani, vi emanciperò dalla loro schiavitù, e vi redimerò con braccio steso e con grandi giudizi.
\par 7 E vi prenderò per mio popolo, e sarò vostro Dio; e voi conoscerete che io sono l'Eterno, il vostro Dio, che vi sottrae ai duri lavori impostivi dagli Egiziani.
\par 8 E v'introdurrò nel paese, che giurai di dare ad Abrahamo, a Isacco e a Giacobbe; e ve lo darò come possesso ereditario: io sono l'Eterno'.
\par 9 E Mosè parlò a quel modo ai figliuoli d'Israele; ma essi non dettero ascolto a Mosè, a motivo dell'angoscia dello spirito loro e della loro dura schiavitù.
\par 10 E l'Eterno parlò a Mosè, dicendo:
\par 11 'Va', parla a Faraone re d'Egitto, ond'egli lasci uscire i figliuoli d'Israele dal suo paese'.
\par 12 Ma Mosè parlò nel cospetto dell'Eterno, e disse: 'Ecco, i figliuoli d'Israele non mi hanno dato ascolto; come dunque darebbe Faraone ascolto a me che sono incirconciso di labbra?'
\par 13 E l'Eterno parlò a Mosè e ad Aaronne, e comandò loro d'andare dai figliuoli d'Israele e da Faraone re d'Egitto, per trarre i figliuoli d'Israele dal paese d'Egitto.
\par 14 Questi sono i capi delle loro famiglie. Figliuoli di Ruben, primogenito d'Israele: Henoc e Pallu, Hetsron e Carmi. Questi sono i rami dei Rubeniti. -
\par 15 Figliuoli di Simeone: Jemuel, Jamin, Ohad, Jakin, Tsochar e Saul, figliuolo della Cananea. Questi sono i rami dei Simeoniti. -
\par 16 Questi sono i nomi dei figliuoli di Levi, secondo le loro generazioni: Gherson, Kehath e Merari. E gli anni della vita di Levi furono centotrentasette. -
\par 17 Figliuoli di Gherson: Libni e Scimei, con le loro diverse famiglie. -
\par 18 Figliuoli di Kehath: Amram, Jitshar, Hebron e Uziel. E gli anni della vita di Kehath furono centotrentatre. -
\par 19 Figliuoli di Merari: Mahli e Musci. Questi sono i rami dei Leviti, secondo le loro generazioni.
\par 20 Or Amram prese per moglie Iokebed, sua zia; ed ella gli partorì Aaronne e Mosè. E gli anni della vita di Amram furono centotrentasette. -
\par 21 Figliuoli di Jitshar: Kore, Nefeg e Zicri. -
\par 22 Figliuoli di Uziel: Mishael, Eltsafan e Sitri. -
\par 23 Aaronne prese per moglie Elisceba, figliuola di Amminadab, sorella di Nahashon; ed ella gli partorì Nadab, Abihu, Eleazar e Ithamar. -
\par 24 Figliuoli di Kore: Assir, Elkana e Abiasaf. Questi sono i rami dei Koriti. -
\par 25 Eleazar, figliuolo d'Aaronne, prese per moglie una delle figliuole di Putiel; ed ella gli partorì Fineas. Questi sono i capi delle famiglie dei Leviti nei loro diversi rami.
\par 26 E questo è quell'Aaronne e quel Mosè ai quali l'Eterno disse: 'Fate uscire i figliuoli d'Israele dal paese d'Egitto, spartiti nelle loro schiere'.
\par 27 Essi son quelli che parlarono a Faraone re d'Egitto, per trarre i figliuoli d'Israele dall'Egitto: sono quel Mosè e quell'Aaronne.
\par 28 Or avvenne, allorché l'Eterno parlò a Mosè nel paese d'Egitto,
\par 29 che l'Eterno disse a Mosè: 'Io sono l'Eterno: di' a Faraone, re d'Egitto, tutto quello che dico a te'.
\par 30 E Mosè rispose, nel cospetto dell'Eterno: 'Ecco, io sono incirconciso di labbra; come dunque Faraone mi porgerà egli ascolto?'

\chapter{7}

\par 1 L'Eterno disse a Mosè: 'Vedi, io ti ho stabilito come Dio per Faraone, e Aaronne tuo fratello sarà il tuo profeta.
\par 2 Tu dirai tutto quello che t'ordinerò, e Aaronne tuo fratello parlerà a Faraone, perché lasci partire i figliuoli d'Israele dal suo paese.
\par 3 E io indurerò il cuore di Faraone, e moltiplicherò i miei segni e i miei prodigi nel paese d'Egitto.
\par 4 E Faraone non vi darà ascolto; e io metterò la mia mano sull'Egitto, e farò uscire dal paese d'Egitto le mie schiere, il mio popolo, i figliuoli d'Israele, mediante grandi giudizi.
\par 5 E gli Egiziani conosceranno che io sono l'Eterno, quando avrò steso la mia mano sull'Egitto e avrò tratto di mezzo a loro i figliuoli d'Israele'.
\par 6 E Mosè e Aaronne fecero così; fecero come l'Eterno avea loro ordinato.
\par 7 Or Mosè aveva ottant'anni e Aaronne ottantatre, quando parlarono a Faraone.
\par 8 L'Eterno parlò a Mosè e ad Aaronne, dicendo:
\par 9 'Quando Faraone vi parlerà e vi dirà: Fate un prodigio! tu dirai ad Aaronne: Prendi il tuo bastone, gettalo davanti a Faraone, e diventerà un serpente'.
\par 10 Mosè ed Aaronne andaron dunque da Faraone, e fecero come l'Eterno aveva ordinato. Aaronne gettò il suo bastone davanti a Faraone e davanti ai suoi servitori, e quello diventò un serpente.
\par 11 Faraone a sua volta chiamò i savi e gl'incantatori; e i magi d'Egitto fecero anch'essi lo stesso, con le loro arti occulte.
\par 12 Ognun d'essi gettò il suo bastone, e i bastoni diventaron serpenti; ma il bastone d'Aaronne inghiottì i bastoni di quelli.
\par 13 E il cuore di Faraone s'indurò, ed egli non diè ascolto a Mosè e ad Aaronne, come l'Eterno avea detto.
\par 14 L'Eterno disse a Mosè: 'Il cuor di Faraone è ostinato;
\par 15 egli rifiuta di lasciar andare il popolo. Va' da Faraone domani mattina; ecco, egli uscirà per andare verso l'acqua; tu sta' ad aspettarlo sulla riva del fiume, e prendi in mano il bastone ch'è stato mutato in serpente.
\par 16 E digli: L'Eterno, l'Iddio degli Ebrei, m'ha mandato da te per dirti: Lascia andare il mio popolo, perché mi serva nel deserto; ed ecco, fino ad ora, tu non hai ubbidito.
\par 17 Così dice l'Eterno: Da questo conoscerai che io sono l'Eterno; ecco, io percoterò col bastone che ho in mia mano le acque che son nel fiume, ed esse saran mutate in sangue.
\par 18 E il pesce ch'è nel fiume morrà, e il fiume sarà ammorbato, e gli Egiziani avranno ripugnanza a bere l'acqua del fiume'.
\par 19 E l'Eterno disse a Mosè: 'Di' ad Aaronne: Prendi il tuo bastone, e stendi la tua mano sulle acque dell'Egitto, sui loro fiumi, sui loro rivi, sui loro stagni e sopra ogni raccolta d'acqua; essi diventeranno sangue, e vi sarà sangue per tutto il paese d'Egitto, perfino ne' recipienti di legno e ne' recipienti di pietra'.
\par 20 Mosè ed Aaronne fecero come l'Eterno aveva ordinato. Aaronne alzò il bastone, e in presenza di Faraone e in presenza dei suoi servitori percosse le acque ch'erano nel fiume; e tutte le acque ch'erano nel fiume furon cangiate in sangue.
\par 21 E il pesce ch'era nel fiume morì; e il fiume fu ammorbato, sì che gli Egiziani non potevan bere l'acqua del fiume; e vi fu sangue per tutto il paese d'Egitto.
\par 22 E i magi d'Egitto fecero lo stesso con le loro arti occulte; e il cuore di Faraone s'indurò ed egli non diè ascolto a Mosè e ad Aaronne, come l'Eterno avea detto.
\par 23 E Faraone, volte ad essi le spalle, se ne andò a casa sua, e neanche di questo fece alcun caso.
\par 24 E tutti gli Egiziani fecero degli scavi ne' pressi del fiume per trovare dell'acqua da bere, perché non potevan bere l'acqua del fiume.
\par 25 E passaron sette interi giorni, dopo che l'Eterno ebbe percosso il fiume.

\chapter{8}

\par 1 Poi l'Eterno disse a Mosè: 'Va' da Faraone, e digli: Così dice l'Eterno: Lascia andare il mio popolo perché mi serva.
\par 2 E se rifiuti di lasciarlo andare, ecco, io colpirò tutta l'estensione del tuo paese col flagello delle rane;
\par 3 e il fiume brulicherà di rane, che saliranno ed entreranno nella tua casa, nella camera ove dormi, sul tuo letto, nelle case de' tuoi servitori e fra il tuo popolo, ne' tuoi forni e nelle tue madie.
\par 4 E le rane assaliranno te, il tuo popolo e tutti i tuoi servitori'.
\par 5 E l'Eterno disse a Mosè: 'Di' ad Aaronne: Stendi la tua mano col tuo bastone sui fiumi, sui rivi e sugli stagni e fa salir le rane sul paese d'Egitto'.
\par 6 E Aaronne stese la sua mano sulle acque d'Egitto, e le rane salirono e coprirono il paese d'Egitto.
\par 7 E i magi fecero lo stesso con le loro arti occulte, e fecero salire le rane sul paese d'Egitto.
\par 8 Allora Faraone chiamò Mosè ed Aaronne e disse loro: 'Pregate l'Eterno che allontani le rane da me e dal mio popolo, e io lascerò andare il popolo, perché offra sacrifizi all'Eterno'.
\par 9 E Mosè disse a Faraone: 'Fammi l'onore di dirmi per quando io devo chiedere, nelle mie supplicazioni per te, per i tuoi servitori e per il tuo popolo, che l'Eterno distrugga le rane intorno a te e nelle tue case, e non ne rimanga se non nel fiume'.
\par 10 Egli rispose: 'Per domani'. E Mosè disse: 'Sarà fatto come tu dici, affinché tu sappia che non v'è alcuno pari all'Eterno, ch'è il nostro Dio.
\par 11 E le rane s'allontaneranno da te, dalle tue case, dai tuoi servitori e dal tuo popolo; non ne rimarrà che nel fiume'.
\par 12 Mosè ed Aaronne uscirono da Faraone; e Mosè implorò l'Eterno relativamente alle rane che aveva inflitte a Faraone.
\par 13 E l'Eterno fece quello che Mosè avea domandato, e le rane morirono nelle case, nei cortili e nei campi.
\par 14 Le radunarono a mucchi e il paese ne fu ammorbato.
\par 15 Ma quando Faraone vide che v'era un po' di respiro, si ostinò in cuor suo, e non diè ascolto a Mosè e ad Aaronne, come l'Eterno avea detto.
\par 16 E l'Eterno disse a Mosè: 'Di' ad Aaronne: Stendi il tuo bastone e percuoti la polvere della terra, ed essa diventerà zanzare per tutto il paese d'Egitto'.
\par 17 Ed essi fecero così. Aaronne stese la mano col suo bastone, percosse la polvere della terra, e ne vennero delle zanzare sugli uomini e sugli animali; tutta la polvere della terra diventò zanzare per tutto il paese d'Egitto.
\par 18 E i magi cercarono di far lo stesso coi loro incantesimi per produrre le zanzare, ma non poterono. Le zanzare furon dunque sugli uomini e sugli animali.
\par 19 Allora i magi dissero a Faraone: 'Questo è il dito di Dio'. Ma il cuore di Faraone s'indurò ed egli non diè ascolto a Mosè e ad Aaronne, come l'Eterno avea detto.
\par 20 Poi l'Eterno disse a Mosè: 'Alzati di buon mattino, e presentati a Faraone. Ecco, egli uscirà per andar verso l'acqua; e digli: Così dice l'Eterno: Lascia andare il mio popolo, perché mi serva.
\par 21 Se no, se non lasci andare il mio popolo, ecco io manderò su te, sui tuoi servitori, sul tuo popolo e nelle tue case, le mosche velenose; le case degli Egiziani saran piene di mosche velenose e il suolo su cui stanno ne sarà coperto.
\par 22 Ma in quel giorno io farò eccezione del paese di Goscen, dove abita il mio popolo; e quivi non ci saranno mosche, affinché tu sappia che io, l'Eterno, sono in mezzo al paese.
\par 23 E io farò una distinzione fra il mio popolo e il tuo popolo. Domani avverrà questo miracolo'.
\par 24 E l'Eterno fece così; e vennero grandi sciami di mosche velenose in casa di Faraone e nelle case dei suoi servitori; e in tutto il paese d'Egitto la terra fu guasta dalle mosche velenose.
\par 25 Faraone chiamò Mosè ed Aaronne e disse: 'Andate, offrite sacrifizi al vostro Dio nel paese'.
\par 26 Ma Mosè rispose: 'Non si può far così; poiché offriremmo all'Eterno, ch'è l'Iddio nostro, dei sacrifizi che sono un abominio per gli Egiziani. Ecco, se offrissimo sotto i loro occhi dei sacrifizi che sono un abominio per gli Egiziani, non ci lapiderebbero essi?
\par 27 Andremo tre giornate di cammino nel deserto, e offriremo sacrifizi all'Eterno, ch'è il nostro Dio, com'egli ci ordinerà'.
\par 28 E Faraone disse: 'Io vi lascerò andare, perché offriate sacrifizi all'Eterno, ch'è il vostro Dio, nel deserto; soltanto, non andate troppo lontano; pregate per me'.
\par 29 E Mosè disse: 'Ecco, io esco da te e pregherò l'Eterno, e domani le mosche s'allontaneranno da Faraone, dai suoi servitori e dal suo popolo; soltanto, Faraone non si faccia più beffe, impedendo al popolo d'andare a offrir sacrifizi all'Eterno'.
\par 30 E Mosè uscì dalla presenza di Faraone, e pregò l'Eterno.
\par 31 E l'Eterno fece quel che Mosè domandava, e allontanò le mosche velenose da Faraone, dai suoi servitori e dal suo popolo; non ne restò neppur una.
\par 32 Ma anche questa volta Faraone si ostinò in cuor suo, e non lasciò andare il popolo.

\chapter{9}

\par 1 Allora l'Eterno disse a Mosè: 'Va' da Faraone, e digli: Così dice l'Eterno, l'Iddio degli Ebrei: Lascia andare il mio popolo, perché mi serva;
\par 2 che se tu rifiuti di lasciarlo andare e lo trattieni ancora,
\par 3 ecco, la mano dell'Eterno sarà sul tuo bestiame ch'è nei campi, sui cavalli, sugli asini, sui cammelli, sui buoi e sulle pecore; ci sarà una tremenda mortalità.
\par 4 E l'Eterno farà distinzione fra il bestiame d'Israele ed il bestiame d'Egitto; e nulla morrà di tutto quello che appartiene ai figliuoli d'Israele'.
\par 5 E l'Eterno fissò un termine, dicendo: 'Domani, l'Eterno farà questo nel paese'.
\par 6 E l'indomani l'Eterno lo fece, e tutto il bestiame d'Egitto morì; ma del bestiame dei figliuoli d'Israele neppure un capo morì.
\par 7 Faraone mandò a vedere, ed ecco che neppure un capo del bestiame degl'Israeliti era morto. Ma il cuore di Faraone fu ostinato, ed ei non lasciò andare il popolo.
\par 8 E l'Eterno disse a Mosè e ad Aaronne: 'Prendete delle manate di cenere di fornace, e la sparga Mosè verso il cielo, sotto gli occhi di Faraone.
\par 9 Essa diventerà una polvere che coprirà tutto il paese d'Egitto, e produrrà delle ulceri germoglianti pustole sulle persone e sugli animali, per tutto il paese d'Egitto'.
\par 10 Ed essi presero della cenere di fornace, e si presentarono a Faraone; Mosè la sparse verso il cielo, ed essa produsse delle ulceri germoglianti pustole sulle persone e sugli animali.
\par 11 E i magi non poteron stare dinanzi a Mosè, a motivo delle ulceri, perché le ulceri erano addosso ai magi come addosso a tutti gli Egiziani.
\par 12 E l'Eterno indurò il cuor di Faraone, ed egli non diè ascolto a Mosè e ad Aaronne come l'Eterno avea detto a Mosè.
\par 13 Poi l'Eterno disse a Mosè: 'Levati di buon mattino, presentati a Faraone, e digli: Così dice l'Eterno, l'Iddio degli Ebrei: Lascia andare il mio popolo, perché mi serva;
\par 14 poiché questa volta manderò tutte le mie piaghe sul tuo cuore, sui tuoi servitori e sul tuo popolo, affinché tu conosca che non c'è nessuno simile a me su tutta la terra.
\par 15 Che se ora io avessi steso la mia mano e avessi percosso di peste te e il tuo popolo, tu saresti stato sterminato di sulla terra.
\par 16 Ma no; io t'ho lasciato sussistere per questo: per mostrarti la mia potenza, e perché il mio nome sia divulgato per tutta la terra.
\par 17 E ti opponi ancora al mio popolo per non lasciarlo andare?
\par 18 Ecco, domani, verso quest'ora, io farò cadere una grandine così forte, che non ce ne fu mai di simile in Egitto, da che fu fondato, fino al dì d'oggi.
\par 19 Or dunque manda a far mettere al sicuro il tuo bestiame e tutto quello che hai per i campi. La grandine cadrà su tutta la gente e su tutti gli animali che si troveranno per i campi e non saranno stati raccolti in casa, e morranno'.
\par 20 Fra i servitori di Faraone, quelli che temettero la parola dell'Eterno fecero rifugiare nelle case i loro servitori e il loro bestiame;
\par 21 ma quelli che non fecero conto della parola dell'Eterno, lasciarono i loro servitori e il loro bestiame per i campi.
\par 22 E l'Eterno disse a Mosè: 'Stendi la tua mano verso il cielo, e cada grandine in tutto il paese d'Egitto, sulla gente, sugli animali e sopra ogni erba dei campi, nel paese d'Egitto'.
\par 23 E Mosè stese il suo bastone verso il cielo; e l'Eterno mandò tuoni e grandine, e del fuoco s'avventò sulla terra; e l'Eterno fece piovere grandine sul paese d'Egitto.
\par 24 Così ci fu grandine e fuoco guizzante del continuo tra la grandine; e la grandine fu così forte, come non ce n'era stata di simile in tutto il paese d'Egitto, da che era diventato nazione.
\par 25 E la grandine percosse, in tutto il paese d'Egitto, tutto quello ch'era per i campi: uomini e bestie; e la grandine percosse ogni erba de' campi e fracassò ogni albero della campagna.
\par 26 Solamente nel paese di Goscen, dov'erano i figliuoli d'Israele, non cadde grandine.
\par 27 Allora Faraone mandò a chiamare Mosè ed Aaronne, e disse loro: 'Questa volta io ho peccato; l'Eterno è giusto, mentre io e il mio popolo siamo colpevoli.
\par 28 Pregate l'Eterno perché cessino questi grandi tuoni e la grandine: e io vi lascerò andare, e non sarete più trattenuti'.
\par 29 E Mosè gli disse: 'Come sarò uscito dalla città, protenderò le mani all'Eterno; i tuoni cesseranno e non ci sarà più grandine, affinché tu sappia che la terra è dell'Eterno.
\par 30 Ma quanto a te e ai tuoi servitori, io so che non avrete ancora timore dell'Eterno Iddio'.
\par 31 Ora il lino e l'orzo erano stati percossi, perché l'orzo era in spiga e il lino in fiore;
\par 32 ma il grano e la spelda non furon percossi, perché sono serotini.
\par 33 Mosè dunque, lasciato Faraone, uscì di città, protese le mani all'Eterno, e i tuoni e la grandine cessarono, e non cadde più pioggia sulla terra.
\par 34 E quando Faraone vide che la pioggia, la grandine e i tuoni eran cessati, continuò a peccare, e si ostinò in cuor suo: lui e i suoi servitori.
\par 35 E il cuor di Faraone s'indurò, ed egli non lasciò andare i figliuoli d'Israele, come l'Eterno avea detto per bocca di Mosè.

\chapter{10}

\par 1 E l'Eterno disse a Mosè: 'Va' da Faraone; poiché io ho reso ostinato il suo cuore e il cuore dei suoi servitori, per fare in mezzo a loro i segni che vedrai,
\par 2 e perché tu narri ai tuoi figliuoli e ai figliuoli dei tuoi figliuoli quello che ho operato in Egitto e i segni che ho fatto in mezzo a loro, onde sappiate che io sono l'Eterno'.
\par 3 Mosè ed Aaronne andaron dunque da Faraone, e gli dissero: 'Così dice l'Eterno, l'Iddio degli Ebrei: Fino a quando rifiuterai d'umiliarti dinanzi a me? Lascia andare il mio popolo, perché mi serva.
\par 4 Se tu rifiuti di lasciar andare il mio popolo, ecco, domani farò venire delle locuste in tutta l'estensione del tuo paese.
\par 5 Esse copriranno la faccia della terra, sì che non si potrà vedere il suolo; ed esse divoreranno il resto ch'è scampato, ciò che v'è rimasto dalla grandine, e divoreranno ogni albero che vi cresce ne' campi.
\par 6 Ed empiranno le tue case, le case di tutti i tuoi servitori e le case di tutti gli Egiziani, come né i tuoi padri né i padri de' tuoi padri videro mai, dal giorno che furono sulla terra, al dì d'oggi'. Detto questo, voltò le spalle, e uscì dalla presenza di Faraone.
\par 7 E i servitori di Faraone gli dissero: 'Fino a quando quest'uomo ci sarà come un laccio? Lascia andare questa gente, e che serva l'Eterno, l'Iddio suo! Non sai tu che l'Egitto è rovinato?'
\par 8 Allora Mosè ed Aaronne furon fatti tornare da Faraone; ed egli disse loro: 'Andate, servite l'Eterno, l'Iddio vostro; ma chi son quelli che andranno?' E Mosè disse:
\par 9 'Noi andremo coi nostri fanciulli e coi nostri vecchi, coi nostri figliuoli e con le nostre figliuole; andremo coi nostri greggi e coi nostri armenti, perché dobbiam celebrare una festa all'Eterno'.
\par 10 E Faraone disse loro: 'Così sia l'Eterno con voi, com'io lascerò andare voi e i vostri bambini! Badate bene, perché avete delle cattive intenzioni!
\par 11 No, no; andate voi uomini, e servite l'Eterno; poiché questo è quel che cercate'. E Faraone li cacciò dalla sua presenza.
\par 12 Allora l'Eterno disse a Mosè: 'Stendi la tua mano sul paese d'Egitto per farvi venire le locuste; e salgano esse sul paese d'Egitto e divorino tutta l'erba del paese, tutto quello che la grandine ha lasciato'.
\par 13 E Mosè stese il suo bastone sul paese d'Egitto; e l'Eterno fece levare un vento orientale sul paese, tutto quel giorno e tutta la notte; e, come venne la mattina, il vento orientale avea portato le locuste.
\par 14 E le locuste salirono su tutto il paese d'Egitto, e si posarono su tutta l'estensione dell'Egitto; erano in sì grande quantità, che prima non ce n'eran mai state tante, né mai più tante ce ne saranno.
\par 15 Esse coprirono la faccia di tutto il paese, in guisa che il paese ne rimase oscurato; e divorarono tutta l'erba del paese e tutti i frutti degli alberi, che la grandine avea lasciato; e nulla restò di verde negli alberi, e nell'erba della campagna, per tutto il paese d'Egitto.
\par 16 Allora Faraone chiamò in fretta Mosè ed Aaronne, e disse: 'Io ho peccato contro l'Eterno, l'Iddio vostro, e contro voi.
\par 17 Ma ora perdona, ti prego, il mio peccato, questa volta soltanto; e supplicate l'Eterno, l'Iddio vostro, perché almeno allontani da me questo flagello mortale'.
\par 18 E Mosè uscì da Faraone, e pregò l'Eterno.
\par 19 E l'Eterno fe' levare un vento contrario, un gagliardissimo vento di ponente, che portò via le locuste e le precipitò nel mar Rosso. Non ci rimase neppure una locusta in tutta l'estensione dell'Egitto.
\par 20 Ma l'Eterno indurò il cuor di Faraone, ed egli non lasciò andare i figliuoli d'Israele.
\par 21 E l'Eterno disse a Mosè: 'Stendi la tua mano verso il cielo, e sianvi tenebre nel paese d'Egitto: tali, che si possan palpare'.
\par 22 E Mosè stese la sua mano verso il cielo, e ci fu una fitta tenebrìa in tutto il paese d'Egitto per tre giorni.
\par 23 Uno non vedeva l'altro, e nessuno si mosse di dove stava, per tre giorni; ma tutti i figliuoli d'Israele aveano della luce nelle loro dimore.
\par 24 Allora Faraone chiamò Mosè e disse: 'Andate, servite l'Eterno; rimangano soltanto i vostri greggi e i vostri armenti; anche i vostri bambini potranno andare con voi'.
\par 25 E Mosè disse: 'Tu ci devi anche concedere di prendere di che fare de' sacrifizi e degli olocausti, perché possiamo offrire sacrifizi all'Eterno, ch'è l'Iddio nostro.
\par 26 Anche il nostro bestiame verrà con noi, senza che ne rimanga addietro neppure un'unghia; poiché di esso dobbiam prendere per servire l'Eterno Iddio nostro; e noi non sapremo con che dovremo servire l'Eterno, finché sarem giunti colà'.
\par 27 Ma l'Eterno indurò il cuore di Faraone, ed egli non volle lasciarli andare.
\par 28 E Faraone disse a Mosè: 'Vattene via da me! Guardati bene dal comparire più alla mia presenza! poiché il giorno che comparirai alla mia presenza, tu morrai!'
\par 29 E Mosè rispose: 'Hai detto bene; io non comparirò più alla tua presenza'.

\chapter{11}

\par 1 E l'Eterno disse a Mosè: 'Io farò venire ancora una piaga su Faraone e sull'Egitto; poi egli vi lascerà partire di qui. Quando vi lascerà partire, egli addirittura vi caccerà di qui.
\par 2 Or parla al popolo e digli che ciascuno domandi al suo vicino e ogni donna alla sua vicina degli oggetti d'argento e degli oggetti d'oro'.
\par 3 E l'Eterno fece entrare il popolo nelle buone grazie degli Egiziani; anche Mosè era personalmente in gran considerazione nel paese d'Egitto, agli occhi dei servitori di Faraone e agli occhi del popolo.
\par 4 E Mosè disse: 'Così dice l'Eterno: Verso mezzanotte, io passerò in mezzo all'Egitto;
\par 5 e ogni primogenito nel paese d'Egitto morrà: dal primogenito di Faraone che siede sul suo trono, al primogenito della serva che sta dietro la macina, e ad ogni primogenito del bestiame.
\par 6 E vi sarà per tutto il paese d'Egitto un gran grido, quale non ci fu mai prima, né ci sarà di poi.
\par 7 Ma fra tutti i figliuoli d'Israele, tanto fra gli uomini quanto fra gli animali, neppure un cane moverà la lingua, affinché conosciate la distinzione che l'Eterno fa tra gli Egiziani e Israele.
\par 8 E tutti questi tuoi servitori scenderanno da me, e s'inchineranno davanti a me, dicendo: Parti, tu e tutto il popolo ch'è al tuo seguito! E, dopo questo, io partirò'. E Mosè uscì dalla presenza di Faraone, acceso d'ira.
\par 9 E l'Eterno disse a Mosè: 'Faraone non vi darà ascolto, affinché i miei prodigi si moltiplichino nel paese d'Egitto'.
\par 10 E Mosè ed Aaronne fecero tutti questi prodigi dinanzi a Faraone; ma l'Eterno indurò il cuore di Faraone, ed egli non lasciò uscire i figliuoli d'Israele dal suo paese.

\chapter{12}

\par 1 L'Eterno parlò a Mosè e ad Aaronne nel paese d'Egitto, dicendo:
\par 2 'Questo mese sarà per voi il primo dei mesi: sarà per voi il primo dei mesi dell'anno.
\par 3 Parlate a tutta la raunanza d'Israele, e dite: Il decimo giorno di questo mese, prenda ognuno un agnello per famiglia, un agnello per casa;
\par 4 e se la casa è troppo poco numerosa per un agnello, se ne prenda uno in comune col vicino di casa più prossimo, tenendo conto del numero delle persone; voi conterete ogni persona secondo quel che può mangiare dell'agnello.
\par 5 Il vostro agnello sia senza difetto, maschio, dell'anno; potrete prendere un agnello o un capretto.
\par 6 Lo serberete fino al quattordicesimo giorno di questo mese, e tutta la raunanza d'Israele, congregata, lo immolerà sull'imbrunire.
\par 7 E si prenda del sangue d'esso, e si metta sui due stipiti e sull'architrave della porta delle case dove lo si mangerà.
\par 8 E se ne mangi la carne in quella notte; si mangi arrostita al fuoco, con pane senza lievito e con dell'erbe amare.
\par 9 Non ne mangiate niente di poco cotto o di lessato nell'acqua, ma sia arrostito al fuoco, con la testa, le gambe e le interiora.
\par 10 E non ne lasciate nulla di resto fino alla mattina; e quel che ne sarà rimasto fino alla mattina, bruciatelo col fuoco.
\par 11 E mangiatelo in questa maniera: coi vostri fianchi cinti, coi vostri calzari ai piedi e col vostro bastone in mano; e mangiatelo in fretta: è la Pasqua dell'Eterno.
\par 12 Quella notte io passerò per il paese d'Egitto, e percoterò ogni primogenito nel paese d'Egitto, tanto degli uomini quanto degli animali, e farò giustizia di tutti gli dèi d'Egitto. Io sono l'Eterno.
\par 13 E quel sangue vi servirà di segno sulle case dove sarete; e quand'io vedrò il sangue passerò oltre, e non vi sarà piaga su voi per distruggervi, quando percoterò il paese d'Egitto.
\par 14 Quel giorno sarà per voi un giorno di ricordanza, e lo celebrerete come una festa in onore dell'Eterno; lo celebrerete d'età in età come una festa d'istituzione perpetua.
\par 15 Per sette giorni mangerete pani azzimi. Fin dal primo giorno toglierete ogni lievito dalle vostre case; poiché, chiunque mangerà pane lievitato, dal primo giorno fino al settimo sarà reciso da Israele.
\par 16 E il primo giorno avrete una santa convocazione, e una santa convocazione il settimo giorno. Non si faccia alcun lavoro in que' giorni; si prepari soltanto quel ch'è necessario a ciascuno per mangiare, e non altro.
\par 17 Osservate dunque la festa degli azzimi; poiché in quel medesimo giorno io avrò tratto le vostre schiere dal paese d'Egitto; osservate dunque quel giorno d'età in età, come una istituzione perpetua.
\par 18 Mangiate pani azzimi dalla sera del quattordicesimo giorno del mese, fino alla sera del ventunesimo giorno.
\par 19 Per sette giorni non si trovi lievito nelle vostre case; perché chiunque mangerà qualcosa di lievitato, quel tale sarà reciso dalla raunanza d'Israele: sia egli forestiero o nativo del paese.
\par 20 Non mangiate nulla di lievitato; in tutte le vostre dimore mangiate pani azzimi'.
\par 21 Mosè dunque chiamò tutti gli anziani d'Israele, e disse loro: 'Sceglietevi e prendetevi degli agnelli per le vostre famiglie, e immolate la Pasqua.
\par 22 E prendete un mazzetto d'issopo, intingetelo nel sangue che sarà nel bacino, e spruzzate di quel sangue che sarà nel bacino, l'architrave e i due stipiti delle porte; e nessuno di voi varchi la porta di casa sua, fino al mattino.
\par 23 Poiché l'Eterno passerà per colpire gli Egiziani; e quando vedrà il sangue sull'architrave e sugli stipiti, l'Eterno passerà oltre la porta, e non permetterà al distruttore d'entrare nelle vostre case per colpirvi.
\par 24 Osservate dunque questo come una istituzione perpetua per voi e per i vostri figliuoli.
\par 25 E quando sarete entrati nel paese che l'Eterno vi darà, conforme ha promesso, osservate questo rito;
\par 26 e quando i vostri figliuoli vi diranno: Che significa per voi questo rito?
\par 27 risponderete: Questo è il sacrifizio della Pasqua in onore dell'Eterno, il quale passò oltre le case dei figliuoli d'Israele in Egitto, quando colpì gli Egiziani e salvò le nostre case'.
\par 28 E il popolo s'inchinò e adorò. E i figliuoli d'Israele andarono, e fecero così; fecero come l'Eterno aveva ordinato a Mosè e ad Aaronne.
\par 29 E avvenne che, alla mezzanotte, l'Eterno colpì tutti i primogeniti nel paese di Egitto, dal primogenito di Faraone che sedeva sul trono al primogenito del carcerato ch'era in prigione, e tutti i primogeniti del bestiame.
\par 30 E Faraone si alzò di notte: egli e tutti i suoi servitori e tutti gli Egiziani; e vi fu un gran grido in Egitto, perché non c'era casa dove non fosse un morto.
\par 31 Ed egli chiamò Mosè ed Aaronne, di notte, e disse: 'Levatevi, partite di mezzo al mio popolo, voi e i figliuoli d'Israele; e andate, servite l'Eterno, come avete detto.
\par 32 Prendete i vostri greggi e i vostri armenti, come avete detto; andatevene, e benedite anche me!'
\par 33 E gli Egiziani facevano forza al popolo per affrettarne la partenza dal paese, perché dicevano: 'Noi siamo tutti morti'.
\par 34 Il popolo portò via la sua pasta prima che fosse lievitata; avvolse le sue madie ne' suoi vestiti e se le mise sulle spalle.
\par 35 Or i figliuoli d'Israele fecero come Mosè avea detto: domandarono agli Egiziani degli oggetti d'argento, degli oggetti d'oro e de' vestiti;
\par 36 e l'Eterno fece entrare il popolo nelle buone grazie degli Egiziani, che gli dettero quel che domandava. Così spogliarono gli Egiziani.
\par 37 I figliuoli d'Israele partirono da Ramses per Succoth, in numero di circa seicentomila uomini a piedi, senza contare i fanciulli.
\par 38 E una folla di gente d'ogni specie salì anch'essa con loro; e avevano pure greggi, armenti, bestiame in grandissima quantità.
\par 39 E cossero la pasta che avean portata dall'Egitto, e ne fecero delle focacce azzime; poiché la pasta non era lievitata, essendo essi stati cacciati dall'Egitto senza poter indugiare e senza potersi prendere provvisioni di sorta.
\par 40 Or la dimora che i figliuoli d'Israele fecero in Egitto fu di quattrocentotrent'anni.
\par 41 E al termine di quattrocentotrent'anni, proprio il giorno che finiva, avvenne che tutte le schiere dell'Eterno uscirono dal paese d'Egitto.
\par 42 Questa è una notte da celebrarsi in onore dell'Eterno, perché ei li trasse dal paese d'Egitto; questa è una notte consacrata all'Eterno, per essere osservata da tutti i figliuoli d'Israele, d'età in età.
\par 43 E l'Eterno disse a Mosè e ad Aaronne: 'Questa è la norma della Pasqua: Nessuno straniero ne mangi;
\par 44 ma qualunque servo, comprato a prezzo di danaro, dopo che l'avrai circonciso, potrà mangiarne.
\par 45 L'avventizio e il mercenario non ne mangino.
\par 46 Si mangi ogni agnello in una medesima casa; non portate fuori nulla della carne d'esso, e non ne spezzate alcun osso.
\par 47 Tutta la raunanza d'Israele celebri la Pasqua.
\par 48 E quando uno straniero soggiornerà teco e vorrà far la Pasqua in onore dell'Eterno, siano circoncisi prima tutti i maschi della sua famiglia; e poi s'accosti pure per farla, e sia come un nativo del paese; ma nessun incirconciso ne mangi.
\par 49 Siavi un'unica legge per il nativo del paese e per lo straniero che soggiorna tra voi'.
\par 50 Tutti i figliuoli d'Israele fecero così; fecero come l'Eterno aveva ordinato a Mosè e ad Aaronne.
\par 51 E avvenne che in quel medesimo giorno l'Eterno trasse i figliuoli d'Israele dal paese d'Egitto, secondo le loro schiere.

\chapter{13}

\par 1 L'Eterno parlò a Mosè, dicendo: 'Consacrami ogni primogenito,
\par 2 tutto ciò che nasce primo tra i figliuoli d'Israele, tanto degli uomini quanto degli animali: esso mi appartiene'.
\par 3 E Mosè disse al popolo: 'Ricordatevi di questo giorno, nel quale siete usciti dall'Egitto, dalla casa di servitù; poiché l'Eterno vi ha tratti fuori di questo luogo, con mano potente; non si mangi pane lievitato.
\par 4 Voi uscite oggi, nel mese di Abib.
\par 5 Quando dunque l'Eterno ti avrà introdotto nel paese dei Cananei, degli Hittei, degli Amorei, degli Hivvei e dei Gebusei che giurò ai tuoi padri di darti, paese ove scorre il latte e il miele, osserva questo rito, in questo mese.
\par 6 Per sette giorni mangia pane senza lievito; e il settimo giorno si faccia una festa all'Eterno.
\par 7 Si mangi pane senza lievito per sette giorni; e non si vegga pan lievitato presso di te, né si vegga lievito presso di te, entro tutti i tuoi confini.
\par 8 E in quel giorno tu spiegherai la cosa al tuo figliuolo, dicendo: Si fa così, a motivo di quello che l'Eterno fece per me quand'uscii dall'Egitto.
\par 9 E ciò ti sarà come un segno sulla tua mano, come un ricordo fra i tuoi occhi, affinché la legge dell'Eterno sia nella tua bocca; poiché l'Eterno ti ha tratto fuori dall'Egitto con mano potente.
\par 10 Osserva dunque questa istituzione, al tempo fissato, d'anno in anno'.
\par 11 'Quando l'Eterno t'avrà introdotto nel paese dei Cananei, come giurò a te e ai tuoi padri, e te lo avrà dato,
\par 12 consacra all'Eterno ogni fanciullo primogenito e ogni primo parto del bestiame che t'appartiene: i maschi saranno dell'Eterno.
\par 13 Ma riscatta ogni primo parto dell'asino con un agnello; e se non lo vuoi riscattare, fiaccagli il collo; riscatta anche ogni primogenito dell'uomo fra i tuoi figliuoli.
\par 14 E quando, in avvenire, il tuo figliuolo t'interrogherà, dicendo: Che significa questo? gli risponderai: L'Eterno ci trasse fuori dall'Egitto, dalla casa di servitù, con mano potente;
\par 15 e avvenne che, quando Faraone s'ostinò a non lasciarci andare, l'Eterno uccise tutti i primogeniti nel paese d'Egitto, tanto i primogeniti degli uomini quanto i primogeniti degli animali; perciò io sacrifico all'Eterno tutti i primi parti maschi, ma riscatto ogni primogenito dei miei figliuoli.
\par 16 Ciò sarà come un segno sulla tua mano e come un frontale fra i tuoi occhi, poiché l'Eterno ci ha tratti dall'Egitto con mano potente'.
\par 17 Or quando Faraone ebbe lasciato andare il popolo, Iddio non lo condusse per la via del paese de' Filistei, perché troppo vicina; poiché Iddio disse: 'Bisogna evitare che il popolo, di fronte a una guerra, si penta e torni in Egitto';
\par 18 ma Iddio fece fare al popolo un giro, per la via del deserto, verso il mar Rosso. E i figliuoli d'Israele salirono armati dal paese d'Egitto.
\par 19 E Mosè prese seco le ossa di Giuseppe; perché questi aveva espressamente fatto giurare i figliuoli d'Israele, dicendo: 'Iddio, certo, vi visiterà; allora, trasportate di qui le mie ossa con voi'.
\par 20 E gl'Israeliti, partiti da Succoth, si accamparono a Etham, all'estremità del deserto.
\par 21 E l'Eterno andava davanti a loro: di giorno, in una colonna di nuvola per guidarli per il loro cammino; e di notte, in una colonna di fuoco per illuminarli, onde potessero camminare giorno e notte.
\par 22 La colonna di nuvola non si ritirava mai di davanti al popolo di giorno, né la colonna di fuoco di notte.

\chapter{14}

\par 1 E l'Eterno parlò a Mosè, dicendo:
\par 2 'Di' ai figliuoli d'Israele che tornino indietro e s'accampino dirimpetto a Pi-Hahiroth, fra Migdol e il mare, di fronte a Baal-Tsefon; accampatevi di faccia a quel luogo presso il mare.
\par 3 E Faraone dirà de' figliuoli d'Israele: Si sono smarriti nel paese; il deserto li tiene rinchiusi.
\par 4 E io indurerò il cuor di Faraone, ed egli li inseguirà; ma io trarrò gloria da Faraone e da tutto il suo esercito, e gli Egiziani sapranno che io sono l'Eterno'. Ed essi fecero così.
\par 5 Or fu riferito al re d'Egitto che il popolo era fuggito; e il cuore di Faraone e de' suoi servitori mutò sentimento verso il popolo, e quelli dissero: 'Che abbiam fatto a lasciar andare Israele, sì che non ci serviranno più?'
\par 6 E Faraone fece attaccare il suo carro, e prese il suo popolo seco.
\par 7 Prese seicento carri scelti e tutti i carri d'Egitto; e su tutti c'eran de' guerrieri.
\par 8 E l'Eterno indurò il cuor di Faraone, re d'Egitto, ed egli inseguì i figliuoli d'Israele, che uscivano pieni di baldanza.
\par 9 Gli Egiziani dunque li inseguirono; e tutti i cavalli, i carri di Faraone, i suoi cavalieri e il suo esercito li raggiunsero mentr'essi erano accampati presso il mare, vicino a Pi-Hahiroth, di fronte a Baal-Tsefon.
\par 10 E quando Faraone si fu avvicinato, i figliuoli d'Israele alzarono gli occhi: ed ecco, gli Egiziani marciavano alle loro spalle; ond'ebbero una gran paura, e gridarono all'Eterno.
\par 11 E dissero a Mosè: 'Mancavan forse sepolture in Egitto, che ci hai menati a morire nel deserto? Perché ci hai fatto quest'azione, di farci uscire dall'Egitto?
\par 12 Non è egli questo che ti dicevamo in Egitto: Lasciaci stare, che serviamo gli Egiziani? Poiché meglio era per noi servire gli Egiziani che morire nel deserto'.
\par 13 E Mosè disse al popolo: 'Non temete, state fermi, e mirate la liberazione che l'Eterno compirà oggi per voi; poiché gli Egiziani che avete veduti quest'oggi, non li vedrete mai più in perpetuo.
\par 14 L'Eterno combatterà per voi, e voi ve ne starete queti'.
\par 15 E l'Eterno disse a Mosè: 'Perché gridi a me? Di' ai figliuoli d'Israele che si mettano in marcia.
\par 16 E tu alza il tuo bastone, stendi la tua mano sul mare, e dividilo; e i figliuoli d'Israele entreranno in mezzo al mare a piedi asciutti.
\par 17 E quanto a me, ecco, io indurerò il cuore degli Egiziani, ed essi v'entreranno, dietro a loro; ed io trarrò gloria da Faraone, da tutto il suo esercito, dai suoi carri e dai suoi cavalieri.
\par 18 E gli Egiziani sapranno che io sono l'Eterno, quando avrò tratto gloria da Faraone, dai suoi carri e dai suoi cavalieri'.
\par 19 Allora l'angelo di Dio, che precedeva il campo d'Israele, si mosse e andò a porsi alle loro spalle; parimente la colonna di nuvola si mosse dal loro fronte e si fermò alle loro spalle;
\par 20 e venne a mettersi fra il campo dell'Egitto e il campo d'Israele; e la nube era tenebrosa per gli uni, mentre rischiarava gli altri nella notte. E l'un campo non si accostò all'altro per tutta la notte.
\par 21 Or Mosè stese la sua mano sul mare; e l'Eterno fece ritirare il mare mediante un gagliardo vento orientale durato tutta la notte, e ridusse il mare in terra asciutta; e le acque si divisero.
\par 22 E i figliuoli d'Israele entrarono in mezzo al mare sull'asciutto; e le acque formavano come un muro alla loro destra e alla loro sinistra.
\par 23 E gli Egiziani li inseguirono; e tutti i cavalli di Faraone, i suoi carri e i suoi cavalieri entrarono dietro a loro in mezzo al mare.
\par 24 E avvenne verso la vigilia del mattino, che l'Eterno, dalla colonna di fuoco e dalla nuvola, guardò verso il campo degli Egiziani, e lo mise in rotta.
\par 25 E tolse le ruote dei loro carri, e ne rese l'avanzata pesante; in guisa che gli Egiziani dissero: 'Fuggiamo d'innanzi ad Israele, perché l'Eterno combatte per loro contro gli Egiziani'.
\par 26 E l'Eterno disse a Mosè: 'Stendi la tua mano sul mare, e le acque ritorneranno sugli Egiziani, sui loro carri e sui loro cavalieri'.
\par 27 E Mosè stese la sua mano sul mare; e, sul far della mattina, il mare riprese la sua forza; e gli Egiziani, fuggendo, gli andavano incontro; e l'Eterno precipitò gli Egiziani in mezzo al mare.
\par 28 Le acque tornarono e coprirono i carri, i cavalieri, tutto l'esercito di Faraone ch'erano entrati nel mare dietro agl'Israeliti; e non ne scampò neppur uno.
\par 29 Ma i figliuoli d'Israele camminarono sull'asciutto in mezzo al mare, e le acque formavano come un muro alla loro destra e alla loro sinistra.
\par 30 Così, in quel giorno, l'Eterno salvò Israele dalle mani degli Egiziani, e Israele vide sul lido del mare gli Egiziani morti.
\par 31 E Israele vide la gran potenza che l'Eterno avea spiegata contro gli Egiziani; onde il popolo temé l'Eterno e credette nell'Eterno e in Mosè suo servo.

\chapter{15}

\par 1 Allora Mosè e i figliuoli d'Israele cantarono questo cantico all'Eterno, e dissero così: "Io canterò all'Eterno, perché si è sommamente esaltato; ha precipitato in mare cavallo e cavaliere.
\par 2 L'Eterno è la mia forza e l'oggetto del mio cantico; egli è stato la mia salvezza. Questo è il mio Dio, io lo glorificherò; è l'Iddio di mio padre, io lo esalterò.
\par 3 L'Eterno, è un guerriero, il suo nome è l'Eterno.
\par 4 Egli ha gettato in mare i carri di Faraone e il suo esercito, e i migliori suoi condottieri sono stati sommersi nel mar Rosso.
\par 5 Gli abissi li coprono; sono andati a fondo come una pietra.
\par 6 La tua destra, o Eterno, è mirabile per la sua forza, la tua destra, o Eterno, schiaccia i nemici.
\par 7 Con la grandezza della tua maestà, tu rovesci i tuoi avversari; tu scateni la tua ira, essa li consuma come stoppia.
\par 8 Al soffio delle tue nari le acque si sono ammontate, le onde si son drizzate come un muro, i flutti si sono assodati nel cuore del mare.
\par 9 Il nemico diceva: 'Inseguirò, raggiungerò, dividerò le spoglie, la mia brama si sazierà su loro; sguainerò la mia spada, la mia mano li sterminerà';
\par 10 ma tu hai mandato fuori il tuo soffio; e il mare li ha ricoperti; sono affondati come piombo nelle acque potenti.
\par 11 Chi è pari a te fra gli dèi, o Eterno? Chi è pari a te, mirabile nella tua santità, tremendo anche a chi ti loda, operator di prodigi?
\par 12 Tu hai steso la destra, la terra li ha ingoiati.
\par 13 Tu hai condotto con la tua benignità il popolo che hai riscattato; l'hai guidato con la tua forza verso la tua santa dimora.
\par 14 I popoli l'hanno udito, e tremano. L'angoscia ha colto gli abitanti della Filistia.
\par 15 Già sono smarriti i capi di Edom, il tremito prende i potenti di Moab, tutti gli abitanti di Canaan vengono meno.
\par 16 Spavento e terrore piomberà su loro. Per la forza del tuo braccio diventeran muti come una pietra, finché il tuo popolo, o Eterno, sia passato, finché sia passato il popolo che ti sei acquistato.
\par 17 Tu li introdurrai e li pianterai sul monte del tuo retaggio, nel luogo che hai preparato, o Eterno, per tua dimora, nel santuario che le tue mani, o Signore, hanno stabilito.
\par 18 L'Eterno regnerà per sempre, in perpetuo".
\par 19 Questo cantarono gl'Israeliti perché i cavalli di Faraone coi suoi carri e i suoi cavalieri erano entrati nel mare, e l'Eterno avea fatto ritornar su loro le acque del mare, ma i figliuoli d'Israele aveano camminato in mezzo al mare, sull'asciutto.
\par 20 E Maria, la profetessa, sorella d'Aaronne, prese in mano il timpano, e tutte le donne usciron dietro a lei con de' timpani, e danzando.
\par 21 E Maria rispondeva ai figliuoli d'Israele: 'Cantate all'Eterno, perché si è sommamente esaltato; ha precipitato in mare cavallo e cavaliere'.
\par 22 Poi Mosè fece partire gl'Israeliti dal Mar Rosso, ed essi si diressero verso il deserto di Shur; camminarono tre giorni nel deserto, e non trovarono acqua.
\par 23 E quando giunsero a Mara, non poteron bevere le acque di Mara, perché erano amare; perciò quel luogo fu chiamato Mara.
\par 24 E il popolo mormorò contro Mosè, dicendo: 'Che berremo?'
\par 25 Ed egli gridò all'Eterno; e l'Eterno gli mostrò un legno ch'egli gettò nelle acque, e le acque divennero dolci. Quivi l'Eterno dette al popolo una legge e una prescrizione, e lo mise alla prova, e disse:
\par 26 'Se ascolti attentamente la voce dell'Eterno, ch'è il tuo Dio, e fai ciò ch'è giusto agli occhi suoi e porgi orecchio ai suoi comandamenti e osservi tutte le sue leggi, io non ti manderò addosso alcuna delle malattie che ho mandato addosso agli Egiziani, perché io sono l'Eterno che ti guarisco'.
\par 27 Poi giunsero ad Elim, dov'erano dodici sorgenti d'acqua e settanta palme; e si accamparono quivi presso le acque.

\chapter{16}

\par 1 E tutta la raunanza de' figliuoli d'Israele partì da Elim e giunse al deserto di Sin, ch'è fra Elim e Sinai, il quindicesimo giorno del secondo mese dopo la loro partenza dal paese d'Egitto.
\par 2 E tutta la raunanza de' figliuoli d'Israele mormorò contro Mosè e contro Aaronne nel deserto.
\par 3 I figliuoli d'Israele dissero loro: 'Oh, fossimo pur morti per mano dell'Eterno nel paese d'Egitto, quando sedevamo presso le pignatte della carne e mangiavamo del pane a sazietà! Poiché voi ci avete menati in questo deserto per far morir di fame tutta questa raunanza'.
\par 4 E l'Eterno disse a Mosè: 'Ecco, io vi farò piovere del pane dal cielo; e il popolo uscirà e ne raccoglierà giorno per giorno quanto gliene abbisognerà per la giornata, ond'io lo metta alla prova per vedere se camminerà o no secondo la mia legge.
\par 5 Ma il sesto giorno, quando prepareranno quello che avran portato a casa, esso sarà il doppio di quello che avranno raccolto ogni altro giorno'.
\par 6 E Mosè ed Aaronne dissero a tutti i figliuoli d'Israele: 'Questa sera voi conoscerete che l'Eterno è quegli che vi ha tratto fuori dal paese d'Egitto;
\par 7 e domattina vedrete la gloria dell'Eterno; poich'egli ha udito le vostre mormorazioni contro l'Eterno; quanto a noi, che cosa siamo perché mormoriate contro di noi?'
\par 8 E Mosè disse: 'Vedrete la gloria dell'Eterno quando stasera egli vi darà della carne da mangiare e domattina del pane a sazietà; giacché l'Eterno ha udito le vostre mormorazioni che proferite contro di lui; quanto a noi, che cosa siamo? le vostre mormorazioni non sono contro di noi, ma contro l'Eterno'.
\par 9 Poi Mosè disse ad Aaronne: 'Di' a tutta la raunanza de' figliuoli d'Israele: Avvicinatevi alla presenza dell'Eterno, perch'egli ha udito le vostre mormorazioni'.
\par 10 E come Aaronne parlava a tutta la raunanza de' figliuoli d'Israele, questi volsero gli occhi verso il deserto; ed ecco che la gloria dell'Eterno apparve nella nuvola.
\par 11 E l'Eterno parlò a Mosè, dicendo:
\par 12 'Io ho udito le mormorazioni dei figliuoli d'Israele; parla loro, dicendo: Sull'imbrunire mangerete della carne, e domattina sarete saziati di pane; e conoscerete che io sono l'Eterno, l'Iddio vostro'.
\par 13 E avvenne, verso sera, che saliron delle quaglie, che ricopersero il campo; e, la mattina, c'era uno strato di rugiada intorno al campo.
\par 14 E quando lo strato di rugiada fu sparito, ecco sulla faccia del deserto una cosa minuta, tonda, minuta come brina sulla terra.
\par 15 E i figliuoli d'Israele, veduta che l'ebbero, dissero l'uno all'altro: 'Che cos'è?' perché non sapevan che cosa fosse. E Mosè disse loro: 'Questo è il pane che l'Eterno vi dà a mangiare.
\par 16 Ecco quel che l'Eterno ha comandato: Ne raccolga ognuno quanto gli basta per il suo nutrimento: un omer a testa, secondo il numero delle vostre persone; ognuno ne pigli per quelli che sono nella sua tenda'.
\par 17 I figliuoli d'Israele fecero così, e ne raccolsero gli uni più e gli altri meno.
\par 18 Lo misurarono con l'omer, e chi ne aveva raccolto molto non n'ebbe di soverchio; e chi ne aveva raccolto poco non n'ebbe penuria. Ognuno ne raccolse quanto gliene abbisognava per il suo nutrimento.
\par 19 E Mosè disse loro: 'Nessuno ne serbi fino a domattina'.
\par 20 Ma alcuni non ubbidirono a Mosè, e ne serbarono fino all'indomani; e quello inverminì e mandò fetore; e Mosè s'adirò contro costoro.
\par 21 Così lo raccoglievano tutte le mattine: ciascuno nella misura che bastava al suo nutrimento; e quando il sole si faceva caldo, quello si struggeva.
\par 22 E il sesto giorno raccolsero di quel pane il doppio: due omer per ciascuno. E tutti i capi della raunanza lo vennero a dire a Mosè.
\par 23 Ed egli disse loro: 'Questo è quello che ha detto l'Eterno: Domani è un giorno solenne di riposo: un sabato sacro all'Eterno; fate cuocere oggi quel che avete da cuocere e fate bollire quel che avete da bollire; e tutto quel che vi avanza, riponetelo e serbatelo fino a domani'.
\par 24 Essi dunque lo riposero fino all'indomani, come Mosè aveva ordinato: e quello non diè fetore e non inverminì.
\par 25 Mosè disse: 'Mangiatelo oggi, perché oggi è il sabato sacro all'Eterno; oggi non ne troverete per i campi.
\par 26 Raccoglietene durante sei giorni; ma il settimo giorno è il sabato; in quel giorno non ve ne sarà'.
\par 27 Or nel settimo giorno avvenne che alcuni del popolo uscirono per raccoglierne, e non ne trovarono.
\par 28 E l'Eterno disse a Mosè: 'Fino a quando rifiuterete d'osservare i miei comandamenti e le mie leggi?
\par 29 Riflettete che l'Eterno vi ha dato il sabato; per questo, nel sesto giorno egli vi dà del pane per due giorni; ognuno stia dov'è; nessuno esca dalla sua tenda il settimo giorno'.
\par 30 Così il popolo si riposò il settimo giorno.
\par 31 E la casa d'Israele chiamò quel pane Manna; esso era simile al seme di coriandolo; era bianco, e aveva il gusto di schiacciata fatta col miele.
\par 32 E Mosè disse: 'Questo è quello che l'Eterno ha ordinato: Empi un omer di manna, perché sia conservato per i vostri discendenti, onde veggano il pane col quale vi ho nutriti nel deserto, quando vi ho tratti fuori dal paese d'Egitto'.
\par 33 E Mosè disse ad Aaronne: 'Prendi un vaso, mettivi dentro un intero omer di manna, e deponilo davanti all'Eterno, perché sia conservato per i vostri discendenti'.
\par 34 Secondo l'ordine che l'Eterno avea dato a Mosè, Aaronne lo depose dinanzi alla Testimonianza, perché fosse conservato.
\par 35 E i figliuoli d'Israele mangiarono la manna per quarant'anni, finché arrivarono in paese abitato; mangiarono la manna finché giunsero ai confini del paese di Canaan.
\par 36 Or l'omer è la decima parte dell'efa.

\chapter{17}

\par 1 Poi tutta la raunanza de' figliuoli d'Israele partì dal deserto di Sin, marciando a tappe secondo gli ordini dell'Eterno, e si accampò a Refidim; e non c'era acqua da bere per il popolo.
\par 2 Allora il popolo contese con Mosè, e disse: 'Dateci dell'acqua da bere'. E Mosè rispose loro: 'Perché contendete con me? perché tentate l'Eterno?'
\par 3 Il popolo dunque patì quivi la sete, e mormorò contro Mosè, dicendo: 'Perché ci hai fatti salire dall'Egitto per farci morire di sete noi, i nostri figliuoli e il nostro bestiame?'
\par 4 E Mosè gridò all'Eterno, dicendo: 'Che farò io per questo popolo? Non andrà molto che mi lapiderà'.
\par 5 E l'Eterno disse a Mosè: 'Passa oltre in fronte al popolo, e prendi teco degli anziani d'Israele; piglia anche in mano il bastone col quale percotesti il fiume, e va'.
\par 6 Ecco, io starò là dinanzi a te, sulla roccia ch'è in Horeb; tu percoterai la roccia, e ne scaturirà dell'acqua, ed il popolo berrà'. Mosè fece così in presenza degli anziani d'Israele.
\par 7 E pose nome a quel luogo Massah e Meribah a motivo della contesa de' figliuoli d'Israele e perché aveano tentato l'Eterno, dicendo: 'L'Eterno è egli in mezzo a noi, sì o no?'
\par 8 Allora venne Amalek a dar battaglia a Israele a Refidim.
\par 9 E Mosè disse a Giosuè: 'Facci una scelta d'uomini ed esci a combattere contro Amalek; domani io starò sulla vetta del colle col bastone di Dio in mano'.
\par 10 Giosuè fece come Mosè gli aveva detto e combatté contro Amalek; e Mosè, Aaronne e Hur salirono sulla vetta del colle.
\par 11 E avvenne che, quando Mosè teneva la mano alzata, Israele vinceva; e quando la lasciava cadere, vinceva Amalek.
\par 12 Or siccome le mani di Mosè s'eran fatte stanche, essi presero una pietra, gliela posero sotto, ed egli vi si mise a sedere; e Aaronne e Hur gli sostenevano le mani: l'uno da una parte, l'altro dall'altra; così le sue mani rimasero immobili fino al tramonto del sole.
\par 13 E Giosuè sconfisse Amalek e la sua gente, mettendoli a fil di spada.
\par 14 E l'Eterno disse a Mosè: 'Scrivi questo fatto in un libro, perché se ne conservi il ricordo, e fa' sapere a Giosuè che io cancellerò interamente di sotto al cielo la memoria di Amalek'.
\par 15 E Mosè edificò un altare, al quale pose nome: 'L'Eterno è la mia bandiera'; e disse:
\par 16 'La mano è stata alzata contro il trono dell'Eterno, e l'Eterno farà guerra ad Amalek d'età in età'.

\chapter{18}

\par 1 Or Jethro, sacerdote di Madian, suocero di Mosè, udì tutto quello che Dio avea fatto a favor di Mosè e d'Israele suo popolo: come l'Eterno avea tratto Israele fuor dall'Egitto.
\par 2 E Jethro, suocero di Mosè, prese Sefora, moglie di Mosè,
\par 3 che questi avea rimandata, e i due figliuoli di lei che si chiamavano: l'uno, Ghershom, perché Mosè avea detto: 'Ho soggiornato in terra straniera';
\par 4 e l'altro Eliezer, perché avea detto: 'L'Iddio del padre mio è stato il mio aiuto, e mi ha liberato dalla spada di Faraone'.
\par 5 Jethro dunque, suocero di Mosè, venne a Mosè, coi figliuoli e la moglie di lui, nel deserto dov'egli era accampato, al monte di Dio;
\par 6 e mandò a dire a Mosè: 'Io, Jethro, tuo suocero, vengo da te con la tua moglie e i due suoi figliuoli con lei'.
\par 7 E Mosè uscì a incontrare il suo suocero, gli s'inchinò, e lo baciò; s'informarono scambievolmente della loro salute, poi entrarono nella tenda.
\par 8 Allora Mosè raccontò al suo suocero tutto quello che l'Eterno avea fatto a Faraone e agli Egiziani per amor d'Israele, tutte le sofferenze patite durante il viaggio, e come l'Eterno li avea liberati.
\par 9 E Jethro si rallegrò di tutto il bene che l'Eterno avea fatto a Israele, liberandolo dalla mano degli Egiziani.
\par 10 E Jethro disse: 'Benedetto sia l'Eterno, che vi ha liberati dalla mano degli Egiziani e dalla mano di Faraone, e ha liberato il popolo dal giogo degli Egiziani!
\par 11 Ora riconosco che l'Eterno è più grande di tutti gli dèi; tale s'è mostrato, quando gli Egiziani hanno agito orgogliosamente contro Israele'.
\par 12 E Jethro, suocero di Mosè, prese un olocausto e dei sacrifizi per offrirli a Dio; e Aaronne e tutti gli anziani d'Israele vennero a mangiare col suocero di Mosè in presenza di Dio.
\par 13 Il giorno seguente, Mosè si assise per render ragione al popolo; e il popolo stette intorno a Mosè dal mattino fino alla sera.
\par 14 E quando il suocero di Mosè vide tutto quello ch'egli faceva per il popolo, disse: 'Che è questo che tu fai col popolo? Perché siedi solo, e tutto il popolo ti sta attorno dal mattino fino alla sera?'
\par 15 E Mosè rispose al suo suocero: 'Perché il popolo viene da me per consultare Dio.
\par 16 Quand'essi hanno qualche affare, vengono da me, e io giudico fra l'uno e l'altro, e fo loro conoscere gli ordini di Dio e le sue leggi'.
\par 17 Ma il suocero di Mosè gli disse: 'Questo che tu fai non va bene.
\par 18 Tu ti esaurirai certamente: tu e questo popolo ch'è teco; poiché quest'affare è troppo grave per te; tu non puoi bastarvi da te solo.
\par 19 Or ascolta la mia voce; io ti darò un consiglio, e Dio sia teco: Sii tu il rappresentante del popolo dinanzi a Dio, e porta a Dio le loro cause.
\par 20 Insegna loro gli ordini e le leggi, e mostra loro la via per la quale han da camminare e quello che devon fare;
\par 21 ma scegli fra tutto il popolo degli uomini capaci che temano Dio: degli uomini fidati, che detestino il lucro iniquo; e stabilisci sul popolo come capi di migliaia, capi di centinaia, capi di cinquantine e capi di diecine;
\par 22 e rendano essi ragione al popolo in ogni tempo; e riferiscano a te ogni affare di grande importanza, ma ogni piccolo affare lo decidano loro. Allevia così il peso che grava su te, e lo portino essi teco.
\par 23 Se tu fai questo, e se Dio te l'ordina, potrai durare; e anche tutto questo popolo arriverà felicemente al luogo che gli è destinato'.
\par 24 Mosè acconsentì al dire del suo suocero, e fece tutto quello ch'egli avea detto.
\par 25 E Mosè scelse fra tutto Israele degli uomini capaci, e li stabilì capi del popolo: capi di migliaia, capi di centinaia, capi di cinquantine e capi di diecine.
\par 26 E quelli rendevano ragione al popolo in ogni tempo; le cause difficili le portavano a Mosè, ma ogni piccolo affare lo decidevano loro.
\par 27 Poi Mosè accomiatò il suo suocero, il quale se ne tornò al suo paese.

\chapter{19}

\par 1 Nel primo giorno del terzo mese da che furono usciti dal paese d'Egitto, i figliuoli d'Israele giunsero al deserto di Sinai.
\par 2 Essendo partiti da Refidim, giunsero al deserto di Sinai e si accamparono nel deserto; quivi si accampò Israele, dirimpetto al monte.
\par 3 E Mosè salì verso Dio; e l'Eterno lo chiamò dal monte, dicendo: 'Di' così alla casa di Giacobbe, e annunzia questo ai figliuoli d'Israele:
\par 4 Voi avete veduto quello che ho fatto agli Egiziani, e come io v'ho portato sopra ali d'aquila e v'ho menato a me.
\par 5 Or dunque, se ubbidite davvero alla mia voce e osservate il mio patto, sarete fra tutti i popoli il mio tesoro particolare;
\par 6 poiché tutta la terra è mia; e mi sarete un regno di sacerdoti e una nazione santa. Queste sono le parole che dirai ai figliuoli d'Israele'.
\par 7 E Mosè venne, chiamò gli anziani del popolo, ed espose loro tutte queste parole che l'Eterno gli aveva ordinato di dire.
\par 8 E tutto il popolo rispose concordemente e disse: 'Noi faremo tutto quello che l'Eterno ha detto'. E Mosè riferì all'Eterno le parole del popolo.
\par 9 E l'Eterno disse a Mosè: 'Ecco, io verrò a te in una folta nuvola, affinché il popolo oda quand'io parlerò con te, e ti presti fede per sempre'. E Mosè riferì all'Eterno le parole del popolo.
\par 10 Allora l'Eterno disse a Mosè: 'Va' dal popolo, santificalo oggi e domani, e fa' che si lavi le vesti.
\par 11 E siano pronti per il terzo giorno; perché il terzo giorno l'Eterno scenderà in presenza di tutto il popolo sul monte Sinai.
\par 12 E tu fisserai attorno attorno de' limiti al popolo, e dirai: Guardatevi dal salire sul monte o dal toccarne il lembo. Chiunque toccherà il monte sarà messo a morte.
\par 13 Nessuna mano tocchi quel tale; ma sia lapidato o trafitto di frecce; animale o uomo che sia, non sia lasciato vivere! Quando il corno sonerà a distesa allora salgano pure sul monte'.
\par 14 E Mosè scese dal monte verso il popolo; santificò il popolo, e quelli si lavarono le vesti.
\par 15 Ed egli disse al popolo: 'Siate pronti fra tre giorni; non v'accostate a donna'.
\par 16 Il terzo giorno, come fu mattino, cominciaron de' tuoni, de' lampi, apparve una folta nuvola sul monte, e s'udì un fortissimo suon di tromba; e tutto il popolo ch'era nel campo, tremò.
\par 17 E Mosè fece uscire il popolo dal campo per menarlo incontro a Dio; e si fermarono appiè del monte.
\par 18 Or il monte Sinai era tutto fumante, perché l'Eterno v'era disceso in mezzo al fuoco; e il fumo ne saliva come il fumo d'una fornace, e tutto il monte tremava forte.
\par 19 Il suon della tromba s'andava facendo sempre più forte; Mosè parlava, e Dio gli rispondeva con una voce.
\par 20 L'Eterno dunque scese sul monte Sinai, in vetta al monte; e l'Eterno chiamò Mosè in vetta al monte, e Mosè vi salì.
\par 21 E l'Eterno disse a Mosè: 'Scendi, avverti solennemente il popolo onde non faccia irruzione verso l'Eterno per guardare, e non n'abbiano a perire molti.
\par 22 E anche i sacerdoti che si appressano all'Eterno, si santifichino, affinché l'Eterno non si avventi contro a loro'.
\par 23 Mosè disse all'Eterno: 'Il popolo non può salire sul monte Sinai, perché tu ce l'hai divietato dicendo: Poni de' limiti attorno al monte, e santificalo'.
\par 24 Ma l'Eterno gli disse: 'Va', scendi abbasso; poi salirai tu, e Aaronne teco; ma i sacerdoti e il popolo non facciano irruzione per salire verso l'Eterno, onde non s'avventi contro a loro'.
\par 25 Mosè discese al popolo e glielo disse.

\chapter{20}

\par 1 Allora Iddio pronunziò tutte queste parole, dicendo:
\par 2 'Io sono l'Eterno, l'Iddio tuo, che ti ho tratto dal paese d'Egitto, dalla casa di servitù.
\par 3 Non avere altri dii nel mio cospetto.
\par 4 Non ti fare scultura alcuna né immagine alcuna delle cose che sono lassù ne' cieli o quaggiù sulla terra o nelle acque sotto la terra;
\par 5 non ti prostrare dinanzi a tali cose e non servir loro, perché io, l'Eterno, l'Iddio tuo, sono un Dio geloso che punisco l'iniquità dei padri sui figliuoli fino alla terza e alla quarta generazione di quelli che mi odiano,
\par 6 e uso benignità, fino alla millesima generazione, verso quelli che m'amano e osservano i miei comandamenti.
\par 7 Non usare il nome dell'Eterno, ch'è l'Iddio tuo, in vano; perché l'Eterno non terrà per innocente chi avrà usato il suo nome in vano.
\par 8 Ricordati del giorno del riposo per santificarlo.
\par 9 Lavora sei giorni e fa' in essi ogni opera tua;
\par 10 ma il settimo è giorno di riposo, sacro all'Eterno, ch'è l'Iddio tuo; non fare in esso lavoro alcuno, né tu, né il tuo figliuolo, né la tua figliuola, né il tuo servo, né la tua serva, né il tuo bestiame, né il forestiero ch'è dentro alle tue porte;
\par 11 poiché in sei giorni l'Eterno fece i cieli, la terra, il mare e tutto ciò ch'è in essi, e si riposò il settimo giorno; perciò l'Eterno ha benedetto il giorno del riposo e l'ha santificato.
\par 12 Onora tuo padre e tua madre, affinché i tuoi giorni siano prolungati sulla terra che l'Eterno, l'Iddio tuo, ti dà.
\par 13 Non uccidere.
\par 14 Non commettere adulterio.
\par 15 Non rubare.
\par 16 Non attestare il falso contro il tuo prossimo.
\par 17 Non concupire la casa del tuo prossimo; non concupire la moglie del tuo prossimo, né il suo servo, né la sua serva, né il suo bue, né il suo asino, né cosa alcuna che sia del tuo prossimo'.
\par 18 Or tutto il popolo udiva i tuoni, il suon della tromba e vedeva i lampi e il monte fumante. A tal vista, tremava e se ne stava da lungi.
\par 19 E disse a Mosè: 'Parla tu con noi, e noi t'ascolteremo; ma non ci parli Iddio, che non abbiamo a morire'.
\par 20 E Mosè disse al popolo: 'Non temete, poiché Dio è venuto per mettervi alla prova, e affinché il suo timore vi stia dinanzi, e così non pecchiate'.
\par 21 Il popolo dunque se ne stava da lungi; ma Mosè s'avvicinò alla caligine dov'era Dio.
\par 22 E l'Eterno disse a Mosè: 'Di' così ai figliuoli d'Israele: Voi stessi avete visto ch'io v'ho parlato dai cieli.
\par 23 Non fate altri dii accanto a me; non vi fate dii d'argento, né dii d'oro.
\par 24 Fammi un altare di terra; e su questo offri i tuoi olocausti, i tuoi sacrifizi di azioni di grazie, le tue pecore e i tuoi buoi; in qualunque luogo dove farò che il mio nome sia ricordato, io verrò a te e ti benedirò.
\par 25 E se mi fai un altare di pietre, non lo costruire di pietre tagliate; perché, se tu alzassi su di esse lo scalpello, tu le contamineresti.
\par 26 E non salire al mio altare per gradini, affinché la tua nudità non si scopra sovr'esso.

\chapter{21}

\par 1 Or queste sono le leggi che tu porrai dinanzi a loro:
\par 2 Se compri un servo ebreo, egli ti servirà per sei anni; ma il settimo se ne andrà libero, senza pagar nulla.
\par 3 Se è venuto solo, se ne andrà solo; se aveva moglie, la moglie se ne andrà con lui.
\par 4 Se il suo padrone gli dà moglie e questa gli partorisce figliuoli e figliuole, la moglie e i figliuoli di lei saranno del padrone, ed egli se ne andrà solo.
\par 5 Ma se il servo fa questa dichiarazione: - 'Io amo il mio padrone, mia moglie e i miei figliuoli; io non voglio andarmene libero' -
\par 6 allora il suo padrone lo farà comparire davanti a Dio, e lo farà accostare alla porta o allo stipite, e il suo padrone gli forerà l'orecchio con una lesina; ed egli lo servirà per sempre.
\par 7 Se uno vende la propria figliuola per esser serva, ella non se ne andrà come se ne vanno i servi.
\par 8 S'ella dispiace al suo padrone, che se l'era presa per moglie, egli la farà riscattare; ma non avrà il diritto di venderla a gente straniera, dopo esserle stato infedele.
\par 9 E se la dà in isposa al suo figliuolo, la tratterà secondo il diritto delle fanciulle.
\par 10 Se prende un'altra moglie, non toglierà alla prima né il vitto, né il vestire, né la coabitazione.
\par 11 Se non le fa queste tre cose, ella se ne andrà senza pagamento di prezzo.
\par 12 Chi percuote un uomo sì ch'egli muoia, dev'essere messo a morte.
\par 13 Se non gli ha teso agguato, ma Dio gliel'ha fatto cader sotto mano, io ti stabilirò un luogo dov'ei si possa rifugiare.
\par 14 Se alcuno con premeditazione uccide il suo prossimo mediante insidia, tu lo strapperai anche dal mio altare, per farlo morire.
\par 15 Chi percuote suo padre o sua madre dev'esser messo a morte.
\par 16 Chi ruba un uomo - sia che l'abbia venduto o che gli sia trovato nelle mani - dev'esser messo a morte.
\par 17 Chi maledice suo padre o sua madre dev'esser messo a morte.
\par 18 Se degli uomini vengono a rissa, e uno percuote l'altro con una pietra o col pugno, e quello non muoia, ma debba mettersi a letto,
\par 19 se si rileva e può camminar fuori appoggiato al suo bastone, colui che lo percosse sarà assolto; soltanto, lo indennizzerà del tempo che ha perduto e lo farà curare fino a guarigione compiuta.
\par 20 Se uno percuote il suo servo o la sua serva col bastone sì che gli muoiano fra le mani, il padrone dev'esser punito;
\par 21 ma se sopravvivono un giorno o due, non sarà punito, perché son danaro suo.
\par 22 Se alcuni vengono a rissa e percuotono una donna incinta sì ch'ella si sgravi, ma senza che ne segua altro danno, il percotitore sarà condannato all'ammenda che il marito della donna gl'imporrà; e la pagherà come determineranno i giudici;
\par 23 ma se ne seguono danno,
\par 24 darai vita per vita, occhio per occhio, dente per dente, mano per mano,
\par 25 piede per piede, scottatura per scottatura, ferita per ferita, contusione per contusione.
\par 26 Se uno colpisce l'occhio del suo servo o l'occhio della sua serva e glielo fa perdere, li lascerà andar liberi in compenso dell'occhio perduto.
\par 27 E se fa cadere un dente al suo servo o un dente alla sua serva, li lascerà andar liberi in compenso del dente perduto.
\par 28 Se un bue cozza un uomo o una donna sì che muoia, il bue dovrà esser lapidato e non se ne mangerà la carne; ma il padrone del bue sarà assolto.
\par 29 Però, se il bue era già da tempo uso cozzare, e il padrone n'è stato avvertito, ma non l'ha tenuto rinchiuso, e il bue ha ucciso un uomo o una donna, il bue sarà lapidato, e il suo padrone pure sarà messo a morte.
\par 30 Ove sia imposto al padrone un prezzo di riscatto, egli pagherà per il riscatto della propria vita tutto quello che gli sarà imposto.
\par 31 Se il bue cozza un figliuolo o una figliuola, gli si applicherà questa medesima legge.
\par 32 Se il bue cozza un servo o una serva, il padrone del bue pagherà al padrone del servo trenta sicli d'argento, e il bue sarà lapidato.
\par 33 Se uno apre una fossa, o se uno scava una fossa e non la copre, e un bue o un asino vi cade dentro,
\par 34 il padron della fossa rifarà il danno: pagherà in danaro il valore della bestia al padrone, e la bestia morta sarà sua.
\par 35 Se il bue d'un uomo ferisce il bue d'un altro sì ch'esso muoia, si venderà il bue vivo e se ne dividerà il prezzo; e anche il bue morto sarà diviso fra loro.
\par 36 Se poi è noto che quel bue era già da tempo uso cozzare, e il suo padrone non l'ha tenuto rinchiuso, questi dovrà pagare bue per bue, e la bestia morta sarà sua.

\chapter{22}

\par 1 Se uno ruba un bue o una pecora e li ammazza o li vende, restituirà cinque buoi per il bue e quattro pecore per la pecora.
\par 2 Se il ladro, còlto nell'atto di fare uno scasso, è percosso e muore, non v'è delitto d'omicidio.
\par 3 Se il sole era levato quand'avvenne il fatto, vi sarà delitto d'omicidio. Il ladro dovrà risarcire il danno; se non ha di che risarcirlo, sarà venduto per ciò che ha rubato.
\par 4 Se il furto, bue o asino o pecora che sia, gli è trovato vivo nelle mani, restituirà il doppio.
\par 5 Se uno arrecherà de' danni a un campo o ad una vigna, lasciando andare le sue bestie a pascere nel campo altrui, risarcirà il danno col meglio del suo campo e col meglio della sua vigna.
\par 6 Se divampa un fuoco e s'attacca alle spine sì che ne sia distrutto il grano in covoni o il grano in piedi o il campo, chi avrà acceso il fuoco dovrà risarcire il danno.
\par 7 Se uno affida al suo vicino del danaro o degli oggetti da custodire, e questi siano rubati dalla casa di quest'ultimo, se il ladro si trova, restituirà il doppio.
\par 8 Se il ladro non si trova, il padrone della casa comparirà davanti a Dio per giurare che non ha messo la mano sulla roba del suo vicino.
\par 9 In ogni caso di delitto, sia che si tratti d'un bue o d'un asino o d'una pecora o d'un vestito o di qualunque oggetto perduto del quale uno dica: 'È questo qui!' la causa d'ambedue le parti verrà davanti a Dio; colui che Dio condannerà, restituirà il doppio al suo prossimo.
\par 10 Se uno dà in custodia al suo vicino un asino o un bue o una pecora o qualunque altra bestia, ed essa muore o resta stroppiata o è portata via senza che ci sian testimoni,
\par 11 interverrà fra le due parti il giuramento dell'Eterno per sapere se colui che avea la bestia in custodia non ha messo la mano sulla roba del suo vicino. Il padrone della bestia si contenterà del giuramento, e l'altro non sarà tenuto a rifacimento di danni.
\par 12 Ma se la bestia gli è stata rubata, egli dovrà risarcire del danno il padrone d'essa.
\par 13 Se la bestia è stata sbranata, la produrrà come prova, e non sarà tenuto a risarcimento per la bestia sbranata.
\par 14 Se uno prende in prestito dal suo vicino una bestia, e questa resti stroppiata o muoia essendo assente il padrone d'essa, egli dovrà rifare il danno.
\par 15 Se il padrone è presente, non v'è luogo a rifacimento di danni; se la bestia è stata presa a nolo, essa è compresa nel prezzo del nolo.
\par 16 Se uno seduce una fanciulla non ancora fidanzata e si giace con lei, dovrà pagare la sua dote e prenderla per moglie.
\par 17 Se il padre di lei rifiuta del tutto di dargliela, paghi la somma che si suol dare per le fanciulle.
\par 18 Non lascerai vivere la strega.
\par 19 Chi s'accoppia con una bestia dovrà esser messo a morte.
\par 20 Chi offre sacrifizi ad altri dèi, fuori che all'Eterno solo, sarà sterminato come anatema.
\par 21 Non maltratterai lo straniero e non l'opprimerai; perché anche voi foste stranieri nel paese d'Egitto.
\par 22 Non affliggerete alcuna vedova, né alcun orfano.
\par 23 Se in qualche modo li affliggi, ed essi gridano a me, io udrò senza dubbio il loro grido;
\par 24 la mia ira s'accenderà, e io vi ucciderò con la spada; e le vostre mogli saranno vedove, e i vostri figliuoli orfani.
\par 25 Se tu presti del danaro a qualcuno del mio popolo, al povero ch'è teco, non lo tratterai da usuraio; non gl'imporrai interesse.
\par 26 Se prendi in pegno il vestito del tuo prossimo, glielo renderai prima che tramonti il sole;
\par 27 perché esso è l'unica sua coperta, è la veste con cui si avvolge il corpo. Su che dormirebb'egli? E se avverrà ch'egli gridi a me, io l'udrò; perché sono misericordioso.
\par 28 Non bestemmierai contro Dio, e non maledirai il principe del tuo popolo.
\par 29 Non indugerai a offrirmi il tributo dell'abbondanza delle tue raccolte e di ciò che cola dai tuoi strettoi. Mi darai il primogenito de' tuoi figliuoli.
\par 30 Lo stesso farai del tuo grosso e del tuo minuto bestiame: il loro primo parto rimarrà sette giorni presso la madre; l'ottavo giorno, me lo darai.
\par 31 Voi mi sarete degli uomini santi; non mangerete carne di bestia trovata sbranata nei campi; gettatela ai cani.

\chapter{23}

\par 1 Non spargere alcuna voce calunniosa e non tener di mano all'empio nell'attestare il falso.
\par 2 Non andar dietro alla folla per fare il male; e non deporre in giudizio schierandoti dalla parte dei più per pervertire la giustizia.
\par 3 Parimente non favorire il povero nel suo processo.
\par 4 Se incontri il bue del tuo nemico o il suo asino smarrito, non mancare di ricondurglielo.
\par 5 Se vedi l'asino di colui che t'odia steso a terra sotto il carico, guardati bene dall'abbandonarlo, ma aiuta il suo padrone a scaricarlo.
\par 6 Non violare il diritto del povero del tuo popolo nel suo processo.
\par 7 Rifuggi da ogni parola bugiarda; e non far morire l'innocente e il giusto; perché io non assolverò il malvagio.
\par 8 Non accettar presenti; perché il presente acceca quelli che ci veggon chiaro, e perverte le parole dei giusti.
\par 9 Non opprimere lo straniero; voi lo conoscete l'animo dello straniero, giacché siete stati stranieri nel paese d'Egitto.
\par 10 Per sei anni seminerai la tua terra e ne raccoglierai i frutti;
\par 11 ma il settimo anno la lascerai riposare e rimanere incolta; i poveri del tuo popolo ne godranno, e le bestie della campagna mangeranno quel che rimarrà. Lo stesso farai della tua vigna e de' tuoi ulivi.
\par 12 Per sei giorni farai il tuo lavoro; ma il settimo giorno ti riposerai, affinché il tuo bue e il tuo asino possano riposarsi, e il figliuolo della tua serva e il forestiero possano riprender fiato.
\par 13 Porrete ben mente a tutte le cose che io vi ho dette, e non pronunzierete il nome di dèi stranieri: non lo si oda uscire dalla vostra bocca.
\par 14 Tre volte all'anno mi celebrerai una festa.
\par 15 Osserverai la festa degli azzimi. Per sette giorni mangerai pane senza lievito, come te l'ho ordinato, al tempo stabilito del mese di Abib, perché in quel mese tu uscisti dal paese d'Egitto; e nessuno comparirà dinanzi a me a mani vuote.
\par 16 Osserverai la festa della mietitura, delle primizie del tuo lavoro, di quello che avrai seminato nei campi; e la festa della raccolta, alla fine dell'anno, quando avrai raccolto dai campi i frutti del tuo lavoro.
\par 17 Tre volte all'anno tutti i maschi compariranno davanti al Signore, l'Eterno.
\par 18 Non offrirai il sangue della mia vittima insieme con pane lievitato; e il grasso dei sacrifizi della mia festa non sarà serbato durante la notte fino al mattino.
\par 19 Porterai alla casa dell'Eterno, ch'è il tuo Dio, le primizie de' primi frutti della terra. Non farai cuocere il capretto nel latte di sua madre.
\par 20 Ecco, io mando un angelo davanti a te per proteggerti per via, e per introdurti nel luogo che ho preparato.
\par 21 Sii guardingo in sua presenza, e ubbidisci alla sua voce; non ti ribellare a lui, perch'egli non perdonerà le vostre trasgressioni; poiché il mio nome è in lui.
\par 22 Ma se ubbidisci fedelmente alla sua voce e fai tutto quello che ti dirò, io sarò il nemico de' tuoi nemici, l'avversario dei tuoi avversari;
\par 23 poiché il mio angelo andrà innanzi a te e t'introdurrà nel paese degli Amorei, degli Hittei, dei Ferezei, dei Cananei, degli Hivvei e dei Gebusei, e li sterminerò.
\par 24 Tu non ti prostrerai davanti ai loro dèi, e non servirai loro. Non farai quello ch'essi fanno; ma distruggerai interamente quegli dèi e spezzerai le loro colonne.
\par 25 Servirete all'Eterno, ch'è il vostro Dio, ed egli benedirà il tuo pane e la tua acqua; ed io allontanerò la malattia di mezzo a te.
\par 26 Nel tuo paese non ci sarà donna che abortisca, né donna sterile. Io farò completo il numero de' tuoi giorni.
\par 27 Io manderò davanti a te il mio terrore, e metterò in rotta ogni popolo presso il quale arriverai, e farò voltar le spalle dinanzi a te a tutti i tuoi nemici.
\par 28 E manderò davanti a te i calabroni, che scacceranno gli Hivvei, i Cananei e gli Hittei dal tuo cospetto.
\par 29 Non li scaccerò dal tuo cospetto in un anno, affinché il paese non diventi un deserto, e le bestie de' campi non si moltiplichino contro di te.
\par 30 Li scaccerò dal tuo cospetto a poco a poco, finché tu cresca di numero e possa prender possesso del paese.
\par 31 E fisserò i tuoi confini dal mar Rosso al mar de' Filistei, e dal deserto sino al fiume; poiché io vi darò nelle mani gli abitanti del paese; e tu li scaccerai d'innanzi a te.
\par 32 Non farai alleanza di sorta con loro, né coi loro dèi.
\par 33 Non dovranno abitare nel tuo paese, perché non t'inducano a peccare contro di me: tu serviresti ai loro dèi, e questo ti sarebbe un laccio'.

\chapter{24}

\par 1 Poi Dio disse a Mosè: 'Sali all'Eterno tu ed Aaronne, Nadab e Abihu e settanta degli anziani d'Israele, e adorate da lungi;
\par 2 poi Mosè solo s'accosterà all'Eterno; ma gli altri non s'accosteranno, né salirà il popolo con lui'.
\par 3 E Mosè venne e riferì al popolo tutte le parole dell'Eterno e tutte le leggi. E tutto il popolo rispose ad una voce e disse: 'Noi faremo tutte le cose che l'Eterno ha dette'.
\par 4 Poi Mosè scrisse tutte le parole dell'Eterno; e, levatosi di buon'ora la mattina, eresse appiè del monte un altare e dodici pietre per le dodici tribù d'Israele.
\par 5 E mandò dei giovani tra i figliuoli d'Israele a offrire olocausti e a immolare giovenchi come sacrifizi di azioni di grazie all'Eterno.
\par 6 E Mosè prese la metà del sangue e lo mise in bacini; e l'altra metà la sparse sull'altare.
\par 7 Poi prese il libro del patto e lo lesse in presenza del popolo, il quale disse: 'Noi faremo tutto quello che l'Eterno ha detto, e ubbidiremo'.
\par 8 Allora Mosè prese il sangue, ne asperse il popolo e disse: 'Ecco il sangue del patto che l'Eterno ha fatto con voi sul fondamento di tutte queste parole'.
\par 9 Poi Mosè ed Aaronne, Nadab e Abihu e settanta degli anziani d'Israele salirono,
\par 10 e videro l'Iddio d'Israele. Sotto i suoi piedi c'era come un pavimento lavorato in trasparente zaffiro, e simile, per limpidezza, al cielo stesso.
\par 11 Ed egli non mise la mano addosso a quegli eletti tra i figliuoli d'Israele; ma essi videro Iddio, e mangiarono e bevvero.
\par 12 E l'Eterno disse a Mosè: 'Sali da me sul monte, e fermati quivi; e io ti darò delle tavole di pietra, la legge e i comandamenti che ho scritti, perché siano insegnati ai figliuoli d'Israele'.
\par 13 Mosè dunque si levò con Giosuè suo ministro; e Mosè salì sul monte di Dio.
\par 14 E disse agli anziani: 'Aspettateci qui, finché torniamo a voi. Ecco, Aaronne e Hur sono con voi; chiunque abbia qualche affare si rivolga a loro'.
\par 15 Mosè dunque salì sul monte, e la nuvola ricoperse il monte.
\par 16 E la gloria dell'Eterno rimase sul monte Sinai e la nuvola lo coperse per sei giorni; e il settimo giorno l'Eterno chiamò Mosè di mezzo alla nuvola.
\par 17 E l'aspetto della gloria dell'Eterno era agli occhi de' figliuoli d'Israele come un fuoco divorante sulla cima del monte.
\par 18 E Mosè entrò in mezzo alla nuvola e salì sul monte; e Mosè rimase sul monte quaranta giorni e quaranta notti.

\chapter{25}

\par 1 L'Eterno parlò a Mosè dicendo: 'Di' ai figliuoli d'Israele che mi facciano un'offerta;
\par 2 accetterete l'offerta da ogni uomo che sarà disposto a farmela di cuore.
\par 3 E questa è l'offerta che accetterete da loro: oro, argento e rame;
\par 4 stoffe di color violaceo, porporino, scarlatto;
\par 5 lino fino e pel di capra; pelli di montone tinte in rosso, pelli di delfino e legno d'acacia;
\par 6 olio per il candelabro, aromi per l'olio della unzione e per il profumo odoroso;
\par 7 pietre di ònice e pietre da incastonare per l'efod e il pettorale.
\par 8 E mi facciano un santuario perch'io abiti in mezzo a loro.
\par 9 Me lo farete in tutto e per tutto secondo il modello del tabernacolo e secondo il modello di tutti i suoi arredi, che io sto per mostrarti.
\par 10 Faranno dunque un'arca di legno d'acacia; la sua lunghezza sarà di due cubiti e mezzo, la sua larghezza di un cubito e mezzo, e la sua altezza di un cubito e mezzo.
\par 11 La rivestirai d'oro puro; la rivestirai così di dentro e di fuori; e le farai al di sopra una ghirlanda d'oro, che giri intorno.
\par 12 Fonderai per essa quattro anelli d'oro, che metterai ai suoi quattro piedi: due anelli da un lato e due anelli dall'altro lato.
\par 13 Farai anche delle stanghe di legno d'acacia, e le rivestirai d'oro.
\par 14 E farai passare le stanghe per gli anelli ai lati dell'arca, perché servano a portarla.
\par 15 Le stanghe rimarranno negli anelli dell'arca; non ne saranno tratte fuori.
\par 16 E metterai nell'arca la testimonianza che ti darò.
\par 17 Farai anche un propiziatorio d'oro puro; la sua lunghezza sarà di due cubiti e mezzo, e la sua larghezza di un cubito e mezzo.
\par 18 E farai due cherubini d'oro; li farai lavorati al martello, alle due estremità del propiziatorio;
\par 19 fa' un cherubino a una delle estremità, e un cherubino all'altra; farete che questi cherubini escano dal propiziatorio alle due estremità.
\par 20 E i cherubini avranno le ali spiegate in alto, in modo da coprire il propiziatorio con le loro ali; avranno la faccia vòlta l'uno verso l'altro; le facce dei cherubini saranno vòlte verso il propiziatorio.
\par 21 E metterai il propiziatorio in alto, sopra l'arca; e nell'arca metterai la testimonianza che ti darò.
\par 22 Quivi io m'incontrerò teco; e di sul propiziatorio, di fra i due cherubini che sono sull'arca della testimonianza, ti comunicherò tutti gli ordini che avrò da darti per i figliuoli d'Israele.
\par 23 Farai anche una tavola di legno d'acacia; la sua lunghezza sarà di due cubiti; la sua larghezza di un cubito, e la sua altezza di un cubito e mezzo.
\par 24 La rivestirai d'oro puro, e le farai una ghirlanda d'oro che le giri attorno.
\par 25 Le farai all'intorno una cornice alta quattro dita; e a questa cornice farai tutt'intorno una ghirlanda d'oro.
\par 26 Le farai pure quattro anelli d'oro, e metterai gli anelli ai quattro canti, ai quattro piedi della tavola.
\par 27 Gli anelli saranno vicinissimi alla cornice per farvi passare le stanghe destinate a portar la tavola.
\par 28 E le stanghe le farai di legno d'acacia, le rivestirai d'oro, e serviranno a portare la tavola.
\par 29 Farai pure i suoi piatti, le sue coppe, i suoi calici e le sue tazze da servire per le libazioni; li farai d'oro puro.
\par 30 E metterai sulla tavola il pane della presentazione, che starà del continuo nel mio cospetto.
\par 31 Farai anche un candelabro d'oro puro; il candelabro, il suo piede e il suo tronco saranno lavorati al martello; i suoi calici, i suoi pomi e i suoi fiori saranno tutti d'un pezzo col candelabro.
\par 32 Gli usciranno sei bracci dai lati: tre bracci del candelabro da un lato e tre bracci del candelabro dall'altro;
\par 33 su l'uno dei bracci saranno tre calici in forma di mandorla, con un pomo e un fiore; e sull'altro braccio, tre calici in forma di mandorla, con un pomo e un fiore. Lo stesso per i sei bracci uscenti dal candelabro.
\par 34 Nel tronco del candelabro ci saranno poi quattro calici in forma di mandorla, coi loro pomi e i loro fiori.
\par 35 Ci sarà un pomo sotto i due primi bracci che partono dal candelabro; un pomo sotto i due seguenti bracci, e un pomo sotto i due ultimi bracci che partono dal candelabro: così per i sei bracci uscenti dal candelabro.
\par 36 Questi pomi e questi bracci saranno tutti d'un pezzo col candelabro; il tutto sarà d'oro fino lavorato al martello.
\par 37 Farai pure le sue lampade, in numero di sette; e le sue lampade si accenderanno in modo che la luce rischiari il davanti del candelabro.
\par 38 E i suoi smoccolatoi e i suoi porta smoccolature saranno d'oro puro.
\par 39 Per fare il candelabro con tutti questi suoi utensili s'impiegherà un talento d'oro puro.
\par 40 E vedi di fare ogni cosa secondo il modello che t'è stato mostrato sul monte.

\chapter{26}

\par 1 Farai poi il tabernacolo di dieci teli di lino fino ritorto, di filo color violaceo, porporino e scarlatto, con dei cherubini artisticamente lavorati.
\par 2 La lunghezza d'ogni telo sarà di ventotto cubiti, e la larghezza d'ogni telo di quattro cubiti; tutti i teli saranno d'una stessa misura.
\par 3 Cinque teli saranno uniti assieme, e gli altri cinque teli saran pure uniti assieme.
\par 4 Farai de' nastri di color violaceo all'orlo del telo ch'è all'estremità della prima serie; e lo stesso farai all'orlo del telo ch'è all'estremità della seconda serie.
\par 5 Metterai cinquanta nastri al primo telo, e metterai cinquanta nastri all'orlo del telo ch'è all'estremità della seconda serie di teli: i nastri si corrisponderanno l'uno all'altro.
\par 6 E farai cinquanta fermagli d'oro, e unirai i teli l'uno all'altro mediante i fermagli, perché il tabernacolo formi un tutto.
\par 7 Farai pure dei teli, di pel di capra, per servir da tenda per coprire il tabernacolo: di questi teli ne farai undici.
\par 8 La lunghezza d'ogni telo sarà di trenta cubiti, e la larghezza d'ogni telo, di quattro cubiti; gli undici teli avranno la stessa misura.
\par 9 Unirai assieme, da sé, cinque di questi teli, e unirai da sé gli altri sei, e addoppierai il sesto sulla parte anteriore della tenda.
\par 10 E metterai cinquanta nastri all'orlo del telo ch'è all'estremità della prima serie, e cinquanta nastri all'orlo del telo ch'è all'estremità della seconda serie di teli.
\par 11 E farai cinquanta fermagli di rame, e farai entrare i fermagli nei nastri e unirai così la tenda, in modo che formi un tutto.
\par 12 Quanto alla parte che rimane di soprappiù dei teli della tenda, la metà del telo di soprappiù ricadrà sulla parte posteriore del tabernacolo;
\par 13 e il cubito da una parte e il cubito dall'altra parte che saranno di soprappiù nella lunghezza dei teli della tenda, ricadranno sui due lati del tabernacolo, di qua e di là, per coprirlo.
\par 14 Farai pure per la tenda una coperta di pelli di montone tinte di rosso, e sopra questa un'altra coperta di pelli di delfino.
\par 15 Farai per il tabernacolo delle assi di legno d'acacia, messe per ritto.
\par 16 La lunghezza d'un'asse sarà di dieci cubiti, e la larghezza d'un'asse, di un cubito e mezzo.
\par 17 Ogni asse avrà due incastri paralleli; farai così per tutte le assi del tabernacolo.
\par 18 Farai dunque le assi per il tabernacolo: venti assi dal lato meridionale, verso il sud.
\par 19 Metterai quaranta basi d'argento sotto le venti assi: due basi sotto ciascun'asse per i suoi due incastri.
\par 20 E farai venti assi per il secondo lato del tabernacolo, il lato di nord,
\par 21 e le loro quaranta basi d'argento: due basi sotto ciascun'asse.
\par 22 E per la parte posteriore del tabernacolo, verso occidente, farai sei assi.
\par 23 Farai pure due assi per gli angoli del tabernacolo, dalla parte posteriore.
\par 24 Queste saranno doppie dal basso in su, e al tempo stesso formeranno un tutto fino in cima, fino al primo anello. Così sarà per ambedue le assi, che saranno ai due angoli.
\par 25 Vi saranno dunque otto assi, con le loro basi d'argento: sedici basi: due basi sotto ciascun'asse.
\par 26 Farai anche delle traverse di legno d'acacia: cinque, per le assi di un lato del tabernacolo;
\par 27 cinque traverse per le assi dell'altro lato del tabernacolo, e cinque traverse per le assi della parte posteriore del tabernacolo, a occidente.
\par 28 La traversa di mezzo, in mezzo alle assi, passerà da una parte all'altra.
\par 29 E rivestirai d'oro le assi, e farai d'oro i loro anelli per i quali passeranno le traverse, e rivestirai d'oro le traverse.
\par 30 Erigerai il tabernacolo secondo la forma esatta che te n'è stata mostrata sul monte.
\par 31 Farai un velo di filo violaceo, porporino, scarlatto, e di lino fino ritorto con de' cherubini artisticamente lavorati,
\par 32 e lo sospenderai a quattro colonne di acacia, rivestite d'oro, che avranno i chiodi d'oro e poseranno su basi d'argento.
\par 33 Metterai il velo sotto i fermagli; e quivi, al di là del velo, introdurrai l'arca della testimonianza; quel velo sarà per voi la separazione del luogo santo dal santissimo.
\par 34 E metterai il propiziatorio sull'arca della testimonianza nel luogo santissimo.
\par 35 E metterai la tavola fuori del velo, e il candelabro dirimpetto alla tavola dal lato meridionale del tabernacolo; e metterai la tavola dal lato di settentrione.
\par 36 Farai pure per l'ingresso della tenda una portiera di filo violaceo, porporino, scarlatto, e di lino fino ritorto, in lavoro di ricamo.
\par 37 E farai cinque colonne di acacia per sospendervi la portiera; le rivestirai d'oro, e avranno i chiodi d'oro e tu fonderai per esse cinque basi di rame.

\chapter{27}

\par 1 Farai anche un altare di legno d'acacia, lungo cinque cubiti e largo cinque cubiti; l'altare sarà quadrato, e avrà tre cubiti d'altezza.
\par 2 Farai ai quattro angoli dei corni che spuntino dall'altare, il quale rivestirai di rame.
\par 3 Farai pure i suoi vasi per raccoglier le ceneri, le sue palette, i suoi bacini, i suoi forchettoni e i suoi bracieri; tutti i suoi utensili li farai di rame.
\par 4 E gli farai una gratella di rame in forma di rete; e sopra la rete, ai suoi quattro canti, farai quattro anelli di rame;
\par 5 e la porrai sotto la cornice dell'altare, nella parte inferiore, in modo che la rete raggiunga la metà dell'altezza dell'altare.
\par 6 Farai anche delle stanghe per l'altare: delle stanghe di legno d'acacia, e le rivestirai di rame.
\par 7 E si faran passare le stanghe per gli anelli; e le stanghe saranno ai due lati dell'altare, quando lo si dovrà portare.
\par 8 Lo farai di tavole, vuoto; dovrà esser fatto, conforme ti è stato mostrato sul monte.
\par 9 Farai anche il cortile del tabernacolo; dal lato meridionale, ci saranno, per formare il cortile, delle cortine di lino fino ritorto, per una lunghezza di cento cubiti, per un lato.
\par 10 Questo lato avrà venti colonne con le loro venti basi di rame; i chiodi e le aste delle colonne saranno d'argento.
\par 11 Così pure per il lato di settentrione, per lungo, ci saranno delle cortine lunghe cento cubiti, con venti colonne e le loro venti basi di rame; i chiodi e le aste delle colonne saranno d'argento.
\par 12 E per largo, dal lato d'occidente, il cortile avrà cinquanta cubiti di cortine, con dieci colonne e le loro dieci basi.
\par 13 E per largo, sul davanti, dal lato orientale il cortile avrà cinquanta cubiti.
\par 14 Da uno dei lati dell'ingresso ci saranno quindici cubiti di cortine, con tre colonne e le loro tre basi;
\par 15 e dall'altro lato pure ci saranno quindici cubiti di cortine, con tre colonne e le loro tre basi.
\par 16 Per l'ingresso del cortile ci sarà una portiera di venti cubiti, di filo violaceo, porporino, scarlatto, e di lino fino ritorto, in lavoro di ricamo, con quattro colonne e le loro quattro basi.
\par 17 Tutte le colonne attorno al cortile saran congiunte con delle aste d'argento; i loro chiodi saranno d'argento, e le loro basi di rame.
\par 18 La lunghezza del cortile sarà di cento cubiti; la larghezza, di cinquanta da ciascun lato; e l'altezza, di cinque cubiti; le cortine saranno di lino fino ritorto, e le basi delle colonne, di rame.
\par 19 Tutti gli utensili destinati al servizio del tabernacolo, tutti i suoi piuoli e tutti i piuoli del cortile saranno di rame.
\par 20 Ordinerai ai figliuoli d'Israele che ti portino dell'olio d'uliva puro, vergine, per il candelabro, per tener le lampade continuamente accese.
\par 21 Nella tenda di convegno, fuori del velo che sta davanti alla testimonianza, Aaronne e i suoi figliuoli lo prepareranno perché le lampade ardano dalla sera al mattino davanti all'Eterno. Questa sarà una regola perpetua per i loro discendenti, da essere osservata dai figliuoli d'Israele.

\chapter{28}

\par 1 E tu fa' accostare a te, di tra i figliuoli d'Israele, Aaronne tuo fratello e i suoi figliuoli con lui perché mi esercitino l'ufficio di sacerdoti: Aaronne, Nadab, Abihu, Eleazar e Ithamar, figliuoli d'Aaronne.
\par 2 E farai ad Aaronne, tuo fratello, dei paramenti sacri, come insegne della loro dignità e come ornamento.
\par 3 Parlerai a tutti gli uomini intelligenti, i quali io ho ripieni di spirito di sapienza, ed essi faranno i paramenti d'Aaronne per consacrarlo, onde mi eserciti l'ufficio di sacerdote.
\par 4 E questi sono i paramenti che faranno: un pettorale, un efod, un manto, una tunica lavorata a maglia, una mitra e una cintura. Faranno dunque de' paramenti sacri per Aaronne tuo fratello e per i suoi figliuoli, affinché mi esercitino l'ufficio di sacerdoti;
\par 5 e si serviranno d'oro, di filo violaceo, porporino, scarlatto, e di lino fino.
\par 6 Faranno l'efod d'oro, di filo violaceo, porporino, scarlatto e di lino fino ritorto, lavorato artisticamente.
\par 7 Esso avrà alle due estremità due spallette, che si uniranno, in guisa ch'esso si terrà bene insieme.
\par 8 E la cintura artistica che è sull'efod per fissarlo, sarà del medesimo lavoro dell'efod, e tutto d'un pezzo con esso; sarà d'oro, di filo color violaceo, porporino, scarlatto, e di lino fino ritorto.
\par 9 E prenderai due pietre d'ònice e v'inciderai su i nomi dei figliuoli d'Israele:
\par 10 sei de' loro nomi sopra una pietra, e gli altri sei nomi sopra la seconda pietra, secondo il loro ordine di nascita.
\par 11 Inciderai su queste due pietre i nomi de' figliuoli d'Israele come fa il lapidario, come s'incide un sigillo; le farai incastrare in castoni d'oro.
\par 12 Metterai le due pietre sulle spallette dell'efod, come pietre di ricordanza per i figliuoli d'Israele; e Aaronne porterà i loro nomi davanti all'Eterno sulle sue due spalle, per ricordanza.
\par 13 E farai de' castoni d'oro,
\par 14 e due catenelle d'oro puro che intreccerai a mo' di cordone, e metterai ne' castoni le catenelle così intrecciate.
\par 15 Farai pure il pettorale del giudizio, artisticamente lavorato; lo farai come il lavoro dell'efod: d'oro, di filo violaceo, porporino, scarlatto, e di lino fino ritorto.
\par 16 Sarà quadrato e doppio; avrà la lunghezza d'una spanna, e una spanna di larghezza.
\par 17 E v'incastonerai un fornimento di pietre: quattro ordini di pietre; nel primo ordine sarà un sardonio, un topazio e uno smeraldo;
\par 18 nel secondo ordine, un rubino, uno zaffiro, un calcedonio;
\par 19 nel terzo ordine, un'opale, un'agata, un'ametista;
\par 20 nel quarto ordine, un grisolito, un'ònice e un diaspro. Queste pietre saranno incastrate nei loro castoni d'oro.
\par 21 E le pietre corrisponderanno ai nomi dei figliuoli d'Israele, e saranno dodici, secondo i loro nomi; saranno incise come de' sigilli, ciascuna col nome d'una delle tribù d'Israele.
\par 22 Farai pure sul pettorale delle catenelle d'oro puro, intrecciate a mo' di cordoni.
\par 23 Poi farai sul pettorale due anelli d'oro, e metterai i due anelli alle due estremità del pettorale.
\par 24 Fisserai i due cordoni d'oro ai due anelli alle estremità del pettorale;
\par 25 e attaccherai gli altri due capi dei due cordoni ai due castoni, e li metterai sulle due spallette dell'efod, sul davanti.
\par 26 E farai due anelli d'oro, e li metterai alle altre due estremità del pettorale, sull'orlo interiore vòlto verso l'efod.
\par 27 Farai due altri anelli d'oro, e li metterai alle due spallette dell'efod, in basso, sul davanti, vicino al punto dove avviene la giuntura, al disopra della cintura artistica dell'efod.
\par 28 E si fisserà il pettorale mediante i suoi anelli agli anelli dell'efod con un cordone violaceo, affinché il pettorale sia al di sopra della cintura artistica dell'efod, e non si possa staccare dall'efod.
\par 29 Così Aaronne porterà i nomi de' figliuoli d'Israele incisi nel pettorale del giudizio, sul suo cuore, quando entrerà nel santuario, per conservarne del continuo la ricordanza dinanzi all'Eterno.
\par 30 Metterai sul pettorale del giudizio l'Urim e il Thummim; e staranno sul cuore d'Aaronne quand'egli si presenterà davanti all'Eterno. Così Aaronne porterà il giudizio de' figliuoli d'Israele sul suo cuore, davanti all'Eterno, del continuo.
\par 31 Farai anche il manto dell'efod, tutto di color violaceo.
\par 32 Esso avrà, in mezzo, un'apertura per passarvi il capo; e l'apertura avrà all'intorno un'orlatura tessuta, come l'apertura d'una corazza, perché non si strappi.
\par 33 All'orlo inferiore del manto, tutt'all'intorno, farai delle melagrane di color violaceo, porporino e scarlatto; e in mezzo ad esse, d'ogn'intorno, porrai de' sonagli d'oro:
\par 34 un sonaglio d'oro e una melagrana, un sonaglio d'oro e una melagrana, sull'orlatura del manto, tutt'all'intorno.
\par 35 Aaronne se lo metterà per fare il servizio; quand'egli entrerà nel luogo santo dinanzi all'Eterno e quando ne uscirà, s'udrà il suono, ed egli non morrà.
\par 36 Farai anche una lamina d'oro puro, e sovr'essa inciderai, come s'incide sopra un sigillo: SANTO ALL'ETERNO.
\par 37 La fisserai ad un nastro violaceo sulla mitra, e starà sul davanti della mitra.
\par 38 Starà sulla fronte d'Aaronne, e Aaronne porterà le iniquità commesse dai figliuoli d'Israele nelle cose sante che consacreranno, in ogni genere di sante offerte; ed essa starà continuamente sulla fronte di lui, per renderli graditi nel cospetto dell'Eterno.
\par 39 Farai pure la tunica di lino fino, lavorata a maglia; farai una mitra di lino fino, e farai una cintura in lavoro di ricamo.
\par 40 E per i figliuoli d'Aaronne farai delle tuniche, farai delle cinture, e farai delle tiare, come insegne della loro dignità e come ornamento.
\par 41 E ne vestirai Aaronne, tuo fratello, e i suoi figliuoli con lui; e li ungerai, li consacrerai e li santificherai perché mi esercitino l'ufficio di sacerdoti.
\par 42 Farai anche loro delle brache di lino per coprire la loro nudità; esse andranno dai fianchi fino alle cosce.
\par 43 Aaronne e i suoi figliuoli le porteranno quando entreranno nella tenda di convegno, o quando s'accosteranno all'altare per fare il servizio nel luogo santo, affinché non si rendano colpevoli e non muoiano. Questa è una regola perpetua per lui e per la sua progenie dopo di lui.

\chapter{29}

\par 1 Questo è quello che farai per consacrarli perché mi esercitino l'ufficio di sacerdoti.
\par 2 Prendi un giovenco e due montoni senza difetto, de' pani senza lievito, delle focacce senza lievito impastate con olio, e delle gallette senza lievito unte d'olio; tutte queste cose farai di fior di farina di grano.
\par 3 Le metterai in un paniere, e le offrirai nel paniere al tempo stesso del giovenco e de' due montoni.
\par 4 Farai avvicinare Aaronne e i suoi figliuoli all'ingresso della tenda di convegno, e li laverai con acqua.
\par 5 Poi prenderai i paramenti, e vestirai Aaronne della tunica, del manto dell'efod, dell'efod e del pettorale, e lo cingerai della cintura artistica dell'efod.
\par 6 Gli porrai in capo la mitra, e metterai sulla mitra il santo diadema.
\par 7 Poi prenderai l'olio dell'unzione, glielo spanderai sul capo, e l'ungerai.
\par 8 Farai quindi accostare i suoi figliuoli, e li vestirai delle tuniche.
\par 9 Cingerai Aaronne e i suoi figliuoli con delle cinture, e assicurerai sul loro capo delle tiare; e il sacerdozio apparterrà loro per legge perpetua. Così consacrerai Aaronne e i suoi figliuoli.
\par 10 Poi farai accostare il giovenco davanti alla tenda di convegno; e Aaronne e i suoi figliuoli poseranno le mani sul capo del giovenco.
\par 11 E scannerai il giovenco davanti all'Eterno, all'ingresso della tenda di convegno.
\par 12 E prenderai del sangue del giovenco, e ne metterai col dito sui corni dell'altare, e spanderai tutto il sangue appiè dell'altare.
\par 13 Prenderai pure tutto il grasso che copre le interiora, la rete ch'è sopra il fegato, i due arnioni e il grasso che v'è sopra, e farai fumar tutto sull'altare.
\par 14 Ma la carne del giovenco, la sua pelle e i suoi escrementi li brucerai col fuoco fuori del campo: è un sacrifizio per il peccato.
\par 15 Poi prenderai uno de' montoni; e Aaronne e i suoi figliuoli poseranno le loro mani sul capo del montone.
\par 16 E scannerai il montone, ne prenderai il sangue, e lo spanderai sull'altare, tutto all'intorno.
\par 17 Poi farai a pezzi il montone, laverai le sue interiora e le sue gambe, e le metterai sui pezzi e sulla sua testa.
\par 18 E farai fumare tutto il montone sull'altare: è un olocausto all'Eterno; è un sacrifizio di soave odore fatto mediante il fuoco all'Eterno.
\par 19 Poi prenderai l'altro montone, e Aaronne e i suoi figliuoli poseranno le loro mani sul capo del montone.
\par 20 Scannerai il montone, prenderai del suo sangue e lo metterai sull'estremità dell'orecchio destro d'Aaronne e sull'estremità dell'orecchio destro de' suoi figliuoli, e sul pollice della loro man destra e sul dito grosso del loro piè destro, e spanderai il sangue sull'altare, tutto all'intorno.
\par 21 E prenderai del sangue che è sull'altare, e dell'olio dell'unzione, e ne aspergerai Aaronne e i suoi paramenti, e i suoi figliuoli e i paramenti de' suoi figliuoli con lui. Così saranno consacrati lui, i suoi paramenti, i suoi figliuoli e i loro paramenti con lui.
\par 22 Prenderai pure il grasso del montone, la coda, il grasso che copre le interiora, la rete del fegato, i due arnioni e il grasso che v'è sopra e la coscia destra, perché è un montone di consacrazione;
\par 23 prenderai anche un pane, una focaccia oliata e una galletta dal paniere degli azzimi che è davanti all'Eterno;
\par 24 e porrai tutte queste cose sulle palme delle mani d'Aaronne e sulle palme delle mani de' suoi figliuoli, e le agiterai come offerta agitata davanti all'Eterno.
\par 25 Poi le prenderai dalle loro mani e le farai fumare sull'altare sopra l'olocausto, come un profumo soave davanti all'Eterno; è un sacrifizio fatto mediante il fuoco all'Eterno.
\par 26 E prenderai il petto del montone che avrà servito alla consacrazione d'Aaronne, e lo agiterai come offerta agitata davanti all'Eterno; e questa sarà la tua parte.
\par 27 E consacrerai, di ciò che spetta ad Aaronne e ai suoi figliuoli, il petto dell'offerta agitata e la coscia dell'offerta elevata: vale a dire, ciò che del montone della consacrazione sarà stato agitato ed elevato;
\par 28 esso apparterrà ad Aaronne e ai suoi figliuoli, per legge perpetua da osservarsi dai figliuoli d'Israele: poiché è un'offerta fatta per elevazione. Sarà un'offerta fatta per elevazione dai figliuoli d'Israele nei loro sacrifizi di azioni di grazie: la loro offerta per elevazione sarà per l'Eterno.
\par 29 E i paramenti sacri di Aaronne saranno, dopo di lui, per i suoi figliuoli, che se li metteranno all'atto della loro unzione e della loro consacrazione.
\par 30 Quello de' suoi figliuoli che gli succederà nel sacerdozio, li indosserà per sette giorni quando entrerà nella tenda di convegno per fare il servizio nel luogo santo.
\par 31 Poi prenderai il montone della consacrazione, e ne farai cuocere la carne in un luogo santo;
\par 32 e Aaronne, e i suoi figliuoli mangeranno, all'ingresso della tenda di convegno, la carne del montone e il pane che sarà nel paniere.
\par 33 Mangeranno le cose che avranno servito a fare l'espiazione per consacrarli e santificarli; ma nessun estraneo ne mangerà, perché son cose sante.
\par 34 E se rimarrà della carne della consacrazione o del pane fino alla mattina dopo, brucerai quel resto col fuoco; non lo si mangerà, perché è cosa santa.
\par 35 Eseguirai dunque, riguardo ad Aaronne e ai suoi figliuoli, tutto quello che ti ho ordinato: li consacrerai durante sette giorni.
\par 36 E ogni giorno offrirai un giovenco, come sacrifizio per il peccato, per fare l'espiazione; purificherai l'altare mediante questa tua espiazione, e l'ungerai per consacrarlo.
\par 37 Per sette giorni farai l'espiazione dell'altare, e lo santificherai; e l'altare sarà santissimo: tutto ciò che toccherà l'altare sarà santo.
\par 38 Or questo è ciò che offrirai sull'altare: due agnelli d'un anno, ogni giorno, del continuo.
\par 39 Uno degli agnelli l'offrirai la mattina; e l'altro l'offrirai sull'imbrunire.
\par 40 Col primo agnello offrirai la decima parte di un efa di fior di farina impastata con la quarta parte di un hin d'olio vergine, e una libazione di un quarto di hin di vino.
\par 41 Il secondo agnello l'offrirai sull'imbrunire; l'accompagnerai con la stessa oblazione e con la stessa libazione della mattina; è un sacrifizio di profumo soave offerto mediante il fuoco all'Eterno.
\par 42 Sarà un olocausto perpetuo offerto dai vostri discendenti, all'ingresso della tenda di convegno, davanti all'Eterno, dove io v'incontrerò per parlar quivi con te.
\par 43 E là io mi troverò coi figliuoli d'Israele; e la tenda sarà santificata dalla mia gloria.
\par 44 E santificherò la tenda di convegno e l'altare; anche Aaronne e i suoi figliuoli santificherò, perché mi esercitino l'ufficio di sacerdoti.
\par 45 E dimorerò in mezzo ai figliuoli d'Israele e sarò il loro Dio.
\par 46 Ed essi conosceranno che io sono l'Eterno, l'Iddio loro, che li ho tratti dal paese d'Egitto per dimorare tra loro. Io sono l'Eterno, l'Iddio loro.

\chapter{30}

\par 1 Farai pure un altare per bruciarvi su il profumo: lo farai di legno d'acacia.
\par 2 La sua lunghezza sarà di un cubito; e la sua larghezza, di un cubito; sarà quadro, e avrà un'altezza di due cubiti; i suoi corni saranno tutti d'un pezzo con esso.
\par 3 Lo rivestirai d'oro puro: il disopra, i suoi lati tutt'intorno, i suoi corni; e gli farai una ghirlanda d'oro che gli giri attorno.
\par 4 E gli farai due anelli d'oro, sotto la ghirlanda, ai suoi due lati; li metterai ai suoi due lati, per passarvi le stanghe che serviranno a portarlo.
\par 5 Farai le stanghe di legno d'acacia, e le rivestirai d'oro.
\par 6 E collocherai l'altare davanti al velo ch'è dinanzi all'arca della testimonianza, di faccia al propiziatorio che sta sopra la testimonianza, dove io mi ritroverò con te.
\par 7 E Aaronne vi brucerà su del profumo fragrante; lo brucerà ogni mattina, quando acconcerà le lampade;
\par 8 e quando Aaronne accenderà le lampade sull'imbrunire, lo farà bruciare come un profumo perpetuo davanti all'Eterno, di generazione in generazione.
\par 9 Non offrirete sovr'esso né profumo straniero, né olocausto, né oblazione; e non vi farete libazione.
\par 10 E Aaronne farà una volta all'anno l'espiazione sui corni d'esso; col sangue del sacrifizio d'espiazione per il peccato vi farà l'espiazione una volta l'anno, di generazione in generazione. Sarà cosa santissima, sacra all'Eterno'.
\par 11 L'Eterno parlò ancora a Mosè, dicendo:
\par 12 'Quando farai il conto de' figliuoli d'Israele, facendone il censimento, ognun d'essi darà all'Eterno il riscatto della propria persona, quando saranno contati; onde non siano colpiti da qualche piaga, allorché farai il loro censimento.
\par 13 Daranno questo: chiunque sarà compreso nel censimento darà un mezzo siclo, secondo il siclo del santuario, che è di venti ghere: un mezzo siclo sarà l'offerta da fare all'Eterno.
\par 14 Ognuno che sarà compreso nel censimento, dai venti anni in su, darà quest'offerta all'Eterno.
\par 15 Il ricco non darà di più, né il povero darà meno del mezzo siclo, quando si farà quest'offerta all'Eterno per il riscatto delle vostre persone.
\par 16 Prenderai dunque dai figliuoli d'Israele questo danaro del riscatto e lo adoprerai per il servizio della tenda di convegno: sarà per i figliuoli d'Israele una ricordanza dinanzi all'Eterno per fare il riscatto delle vostre persone'.
\par 17 L'Eterno parlò ancora a Mosè, dicendo:
\par 18 'Farai pure una conca di rame, con la sua base di rame, per le abluzioni; la porrai fra la tenda di convegno e l'altare, e ci metterai dell'acqua.
\par 19 E Aaronne e i suoi figliuoli vi si laveranno le mani e i piedi.
\par 20 Quando entreranno nella tenda di convegno, si laveranno con acqua, onde non abbiano a morire; così pure quando si accosteranno all'altare per fare il servizio, per far fumare un'offerta fatta all'Eterno mediante il fuoco.
\par 21 Si laveranno le mani e i piedi, onde non abbiano a morire. Questa sarà una norma perpetua per loro, per Aaronne e per la sua progenie, di generazione in generazione'.
\par 22 L'Eterno parlò ancora a Mosè, dicendo:
\par 23 'Prenditi anche de' migliori aromi: di mirra vergine, cinquecento sicli; di cinnamomo aromatico, la metà, cioè duecentocinquanta; di canna aromatica, pure duecentocinquanta:
\par 24 di cassia, cinquecento, secondo il siclo del santuario; e un hin d'olio d'oliva.
\par 25 E ne farai un olio per l'unzione sacra, un profumo composto con arte di profumiere: sarà l'olio per l'unzione sacra.
\par 26 E con esso ungerai la tenda di convegno e l'arca della testimonianza,
\par 27 la tavola e tutti i suoi utensili, il candelabro e i suoi utensili, l'altare dei profumi,
\par 28 l'altare degli olocausti e tutti i suoi utensili, la conca e la sua base.
\par 29 Consacrerai così queste cose, e saranno santissime; tutto quello che le toccherà, sarà santo.
\par 30 E ungerai Aaronne e i suoi figliuoli, e li consacrerai perché mi esercitino l'ufficio di sacerdoti.
\par 31 E parlerai ai figliuoli d'Israele, dicendo: Quest'olio mi sarà un olio di sacra unzione, di generazione in generazione.
\par 32 Non lo si spanderà su carne d'uomo, e non ne farete altro di simile, della stessa composizione; esso è cosa santa, e sarà per voi cosa santa.
\par 33 Chiunque ne comporrà di simile, o chiunque ne metterà sopra un estraneo, sarà sterminato di fra il suo popolo'.
\par 34 L'Eterno disse ancora a Mosè: 'Prenditi degli aromi, della resina, della conchiglia odorosa, del galbano, degli aromi con incenso puro, in dosi uguali;
\par 35 e ne farai un profumo composto secondo l'arte del profumiere, salato, puro, santo;
\par 36 ne ridurrai una parte in minutissima polvere, e ne porrai davanti alla testimonianza nella tenda di convegno, dove io m'incontrerò con te: esso vi sarà cosa santissima.
\par 37 E del profumo che farai, non ne farete della stessa composizione per uso vostro; ti sarà cosa santa, consacrata all'Eterno.
\par 38 Chiunque ne farà di simile per odorarlo, sarà sterminato di fra il suo popolo'.

\chapter{31}

\par 1 L'Eterno parlò ancora a Mosè, dicendo: 'Vedi, io ho chiamato per nome Betsaleel,
\par 2 figliuolo di Uri, figliuolo di Hur, della tribù di Giuda;
\par 3 e l'ho ripieno dello spirito di Dio, di abilità, d'intelligenza e di sapere per ogni sorta di lavori,
\par 4 per concepire opere d'arte, per lavorar l'oro, l'argento e il rame,
\par 5 per incidere pietre da incastonare, per scolpire il legno, per eseguire ogni sorta di lavori.
\par 6 Ed ecco, gli ho dato per compagno Oholiab, figliuolo di Ahisamac, della tribù di Dan; e ho messo sapienza nella mente di tutti gli uomini abili, perché possan fare tutto quello che t'ho ordinato:
\par 7 la tenda di convegno, l'arca per la testimonianza, il propiziatorio che vi dovrà esser sopra, e tutti gli arredi della tenda; la tavola e i suoi utensili,
\par 8 il candelabro d'oro puro e tutti i suoi utensili,
\par 9 l'altare dei profumi, l'altare degli olocausti e tutti i suoi utensili, la conca e la sua base,
\par 10 i paramenti per le cerimonie, i paramenti sacri per il sacerdote Aaronne e i paramenti dei suoi figliuoli per esercitare il sacerdozio,
\par 11 l'olio dell'unzione e il profumo fragrante per il luogo santo. Faranno tutto conformemente a quello che ho ordinato'.
\par 12 L'Eterno parlò ancora a Mosè, dicendo:
\par 13 'Quanto a te, parla ai figliuoli d'Israele e di' loro: Badate bene d'osservare i miei sabati, perché il sabato è un segno fra me e voi per tutte le vostre generazioni, affinché conosciate che io sono l'Eterno che vi santifica.
\par 14 Osserverete dunque il sabato, perché è per voi un giorno santo; chi lo profanerà dovrà esser messo a morte; chiunque farà in esso qualche lavoro sarà sterminato di fra il suo popolo.
\par 15 Si lavorerà sei giorni; ma il settimo giorno è un sabato di solenne riposo, sacro all'Eterno; chiunque farà qualche lavoro nel giorno del sabato dovrà esser messo a morte.
\par 16 I figliuoli d'Israele quindi osserveranno il sabato, celebrandolo di generazione in generazione come un patto perpetuo.
\par 17 Esso è un segno perpetuo fra me e i figliuoli d'Israele; poiché in sei giorni l'Eterno fece i cieli e la terra, e il settimo giorno cessò di lavorare, e si riposò'.
\par 18 Quando l'Eterno ebbe finito di parlare con Mosè sul monte Sinai, gli dette le due tavole della testimonianza, tavole di pietra, scritte col dito di Dio.

\chapter{32}

\par 1 Or il popolo, vedendo che Mosè tardava a scender dal monte, si radunò intorno ad Aaronne e gli disse: 'Orsù, facci un dio, che ci vada dinanzi; poiché, quanto a Mosè, a quest'uomo che ci ha tratto dal paese d'Egitto, non sappiamo che ne sia stato'.
\par 2 E Aaronne rispose loro: 'Staccate gli anelli d'oro che sono agli orecchi delle vostre mogli, dei vostri figliuoli e delle vostre figliuole, e portatemeli'.
\par 3 E tutto il popolo si staccò dagli orecchi gli anelli d'oro e li portò ad Aaronne,
\par 4 il quale li prese dalle loro mani, e, dopo averne cesellato il modello, ne fece un vitello di getto. E quelli dissero: 'O Israele, questo è il tuo dio che ti ha tratto dal paese d'Egitto!'
\par 5 Quando Aaronne vide questo, eresse un altare davanti ad esso, e fece un bando che diceva: 'Domani sarà festa in onore dell'Eterno!'
\par 6 E l'indomani, quelli si levarono di buon'ora, offrirono olocausti e recarono de' sacrifizi di azioni di grazie; e il popolo si adagiò per mangiare e bere, e poi si alzò per divertirsi.
\par 7 E l'Eterno disse a Mosè: 'Va', scendi; perché il tuo popolo che hai tratto dal paese d'Egitto, s'è corrotto;
\par 8 si son presto sviati dalla strada ch'io avevo loro ordinato di seguire; si son fatti un vitello di getto, l'hanno adorato, gli hanno offerto sacrifizi, e hanno detto: O Israele, questo è il tuo dio che ti ha tratto dal paese d'Egitto'.
\par 9 L'Eterno disse ancora a Mosè: 'Ho considerato bene questo popolo; ecco, è un popolo di collo duro.
\par 10 Or dunque, lascia che la mia ira s'infiammi contro a loro, e ch'io li consumi! ma di te io farò una grande nazione'.
\par 11 Allora Mosè supplicò l'Eterno, il suo Dio, e disse: 'Perché, o Eterno, l'ira tua s'infiammerebbe contro il tuo popolo che hai tratto dal paese d'Egitto con gran potenza e con mano forte?
\par 12 Perché direbbero gli Egiziani: Egli li ha tratti fuori per far loro del male, per ucciderli su per le montagne e per sterminarli di sulla faccia della terra? Calma l'ardore della tua ira e pèntiti del male di cui minacci il tuo popolo.
\par 13 Ricordati d'Abrahamo, d'Isacco e d'Israele, tuoi servi, ai quali giurasti per te stesso, dicendo loro: Io moltiplicherò la vostra progenie come le stelle de' cieli; darò alla vostra progenie tutto questo paese di cui vi ho parlato, ed essa lo possederà in perpetuo'.
\par 14 E l'Eterno si pentì del male che avea detto di fare al suo popolo.
\par 15 Allora Mosè si voltò e scese dal monte con le due tavole della testimonianza nelle mani: tavole scritte d'ambo i lati, di qua e di là.
\par 16 Le tavole erano opera di Dio, e la scrittura era scrittura di Dio, incisa sulle tavole.
\par 17 Or Giosuè, udendo il clamore del popolo che gridava, disse a Mosè: 'S'ode un fragore di battaglia nel campo'.
\par 18 E Mosè rispose: 'Questo non è né grido di vittoria, né grido di vinti; il clamore ch'io odo è di gente che canta'.
\par 19 E come fu vicino al campo, vide il vitello e le danze; e l'ira di Mosè s'infiammò, ed egli gettò dalle mani le tavole e le spezzò appiè del monte.
\par 20 Poi prese il vitello che quelli avean fatto, lo bruciò col fuoco, lo ridusse in polvere, sparse la polvere sull'acqua, e la fece bere ai figliuoli d'Israele.
\par 21 E Mosè disse ad Aaronne: 'Che t'ha fatto questo popolo, che gli hai tirato addosso un sì gran peccato?'
\par 22 Aaronne rispose: 'L'ira del mio signore non s'infiammi; tu conosci questo popolo, e sai ch'è inclinato al male.
\par 23 Essi m'hanno detto: Facci un dio che ci vada dinanzi; poiché, quanto a Mosè, a quest'uomo che ci ha tratti dal paese d'Egitto, non sappiamo che ne sia stato.
\par 24 E io ho detto loro: Chi ha dell'oro se lo levi di dosso! Essi me l'hanno dato; io l'ho buttato nel fuoco, e n'è venuto fuori questo vitello'.
\par 25 Quando Mosè vide che il popolo era senza freno e che Aaronne lo aveva lasciato sfrenarsi esponendolo all'obbrobrio de' suoi nemici,
\par 26 si fermò all'ingresso del campo, e disse: 'Chiunque è per l'Eterno, venga a me!' E tutti i figliuoli di Levi si radunarono presso a lui.
\par 27 Ed egli disse loro: 'Così dice l'Eterno, l'Iddio d'Israele: Ognuno di voi si metta la spada al fianco; passate e ripassate nel campo, da una porta all'altra d'esso, e ciascuno uccida il fratello, ciascuno l'amico, ciascuno il vicino!'
\par 28 I figliuoli di Levi eseguirono l'ordine di Mosè, e in quel giorno caddero circa tremila uomini.
\par 29 Or Mosè avea detto: 'Consacratevi oggi all'Eterno, anzi ciascuno si consacri a prezzo del proprio figliuolo e del proprio fratello, onde l'Eterno v'impartisca una benedizione'.
\par 30 L'indomani Mosè disse al popolo: 'Voi avete commesso un gran peccato; ma ora io salirò all'Eterno; forse otterrò che il vostro peccato vi sia perdonato'.
\par 31 Mosè dunque tornò all'Eterno e disse: 'Ahimè, questo popolo ha commesso un gran peccato, e s'è fatto un dio d'oro;
\par 32 nondimeno, perdona ora il loro peccato! Se no, deh, cancellami dal tuo libro che hai scritto!'
\par 33 E l'Eterno rispose a Mosè: 'Colui che ha peccato contro di me, quello cancellerò dal mio libro!
\par 34 Or va', conduci il popolo dove t'ho detto. Ecco, il mio angelo andrà dinanzi a te; ma nel giorno che verrò a punire, io li punirò del loro peccato'.
\par 35 E l'Eterno percosse il popolo, perch'esso era l'autore del vitello che Aaronne avea fatto.

\chapter{33}

\par 1 L'Eterno disse a Mosè: 'Va' sali di qui, tu col popolo che hai tratto dal paese d'Egitto, verso il paese che promisi con giuramento ad Abrahamo ad Isacco e a Giacobbe, dicendo: Io lo darò alla tua progenie.
\par 2 Io manderò un angelo dinanzi a te, e caccerò i Cananei, gli Amorei, gli Hittei, i Ferezei, gli Hivvei e i Gebusei.
\par 3 Esso vi condurrà in un paese ove scorre il latte e il miele; poiché io non salirò in mezzo a te, perché sei un popolo di collo duro, ond'io non abbia a sterminarti per via'.
\par 4 Quando il popolo udì queste sinistre parole, fece cordoglio, e nessuno si mise i propri ornamenti.
\par 5 Infatti l'Eterno avea detto a Mosè: 'Di' ai figliuoli d'Israele: Voi siete un popolo di collo duro; s'io salissi per un momento solo in mezzo a te, ti consumerei! Or dunque, togliti i tuoi ornamenti, e vedrò com'io ti debba trattare'.
\par 6 E i figliuoli d'Israele si spogliarono de' loro ornamenti, dalla partenza dal monte Horeb in poi.
\par 7 E Mosè prese la tenda, e la piantò per sé fuori del campo, a una certa distanza dal campo, e la chiamò la tenda di convegno; e chiunque cercava l'Eterno, usciva verso la tenda di convegno, ch'era fuori del campo.
\par 8 Quando Mosè usciva per recarsi alla tenda, tutto il popolo si alzava, e ognuno se ne stava ritto all'ingresso della propria tenda, e seguiva con lo sguardo Mosè, finch'egli fosse entrato nella tenda.
\par 9 E come Mosè era entrato nella tenda, la colonna di nuvola scendeva, si fermava all'ingresso della tenda, e l'Eterno parlava con Mosè.
\par 10 Tutto il popolo vedeva la colonna di nuvola ferma all'ingresso della tenda; e tutto il popolo si alzava, e ciascuno si prostrava all'ingresso della propria tenda.
\par 11 Or l'Eterno parlava con Mosè faccia a faccia, come un uomo parla col proprio amico; poi Mosè tornava al campo; ma Giosuè, figliuolo di Nun, suo giovane ministro, non si dipartiva dalla tenda.
\par 12 E Mosè disse all'Eterno: 'Vedi, tu mi dici: Fa' salire questo popolo! e non mi fai conoscere chi manderai meco. Eppure hai detto: Io ti conosco personalmente ed anche hai trovato grazia agli occhi miei.
\par 13 Or dunque, se ho trovato grazia agli occhi tuoi, deh, fammi conoscere le tue vie, ond'io ti conosca e possa trovar grazia agli occhi tuoi. E considera che questa nazione è popolo tuo'.
\par 14 E l'Eterno rispose: 'La mia presenza andrà teco, e io ti darò riposo'.
\par 15 E Mosè gli disse: 'Se la tua presenza non vien meco, non ci far partire di qui.
\par 16 Poiché, come si farà ora a conoscere che io e il tuo popolo abbiam trovato grazia agli occhi tuoi? Non sarà egli dal fatto che tu vieni con noi? Questo distinguerà me e il tuo popolo da tutti i popoli che sono sulla faccia della terra'.
\par 17 E l'Eterno disse a Mosè: 'Farò anche questo che tu chiedi, poiché tu hai trovato grazia agli occhi miei, e ti conosco personalmente'.
\par 18 Mosè disse: 'Deh, fammi vedere la tua gloria!'
\par 19 E l'Eterno gli rispose: 'Io farò passare davanti a te tutta la mia bontà, e proclamerò il nome dell'Eterno davanti a te; e farò grazia a chi vorrò far grazia, e avrò pietà di chi vorrò aver pietà'.
\par 20 Disse ancora: 'Tu non puoi veder la mia faccia, perché l'uomo non mi può vedere e vivere'.
\par 21 E l'Eterno disse: 'Ecco qui un luogo presso a me; tu starai su quel masso;
\par 22 e mentre passerà la mia gloria, io ti metterò in una buca del masso, e ti coprirò con la mia mano, finché io sia passato;
\par 23 poi ritirerò la mano, e mi vedrai per di dietro; ma la mia faccia non si può vedere'.

\chapter{34}

\par 1 L'Eterno disse a Mosè: 'Tagliati due tavole di pietra come le prime; e io scriverò sulle tavole le parole che erano sulle prime che spezzasti.
\par 2 E sii pronto domattina, e sali al mattino sul monte Sinai, e presentati quivi a me in vetta al monte.
\par 3 Nessuno salga con te, e non si vegga alcuno per tutto il monte; e greggi ed armenti non pascolino nei pressi di questo monte'.
\par 4 Mosè dunque tagliò due tavole di pietra, come le prime; si alzò la mattina di buon'ora, e salì sul monte Sinai come l'Eterno gli avea comandato, e prese in mano le due tavole di pietra.
\par 5 E l'Eterno discese nella nuvola, si fermò quivi con lui e proclamò il nome dell'Eterno.
\par 6 E l'Eterno passò davanti a lui, e gridò: 'L'Eterno! l'Eterno! l'Iddio misericordioso e pietoso, lento all'ira, ricco in benignità e fedeltà,
\par 7 che conserva la sua benignità fino alla millesima generazione, che perdona l'iniquità, la trasgressione e il peccato ma non terrà il colpevole per innocente, e che punisce l'iniquità dei padri sopra i figliuoli e sopra i figliuoli de' figliuoli, fino alla terza e alla quarta generazione!'
\par 8 E Mosè subito s'inchinò fino a terra, e adorò.
\par 9 Poi disse: 'Deh, Signore, se ho trovato grazia agli occhi tuoi, venga il Signore in mezzo a noi, perché questo è un popolo di collo duro; perdona la nostra iniquità e il nostro peccato, e prendici come tuo possesso'.
\par 10 E l'Eterno rispose: 'Ecco, io faccio un patto: farò dinanzi a tutto il tuo popolo maraviglie, quali non si son mai fatte su tutta la terra né in alcuna nazione; e tutto il popolo in mezzo al quale ti trovi vedrà l'opera dell'Eterno, perché tremendo è quello ch'io sono per fare per mezzo di te.
\par 11 Osserva quello che oggi ti comando: Ecco, io caccerò dinanzi a te gli Amorei, i Cananei, gli Hittei, i Ferezei, gli Hivvei e i Gebusei.
\par 12 Guardati dal far lega con gli abitanti del paese nel quale stai per andare, onde non abbiano a diventare, in mezzo a te, un laccio;
\par 13 ma demolite i loro altari, frantumate le loro colonne, abbattete i loro idoli;
\par 14 poiché tu non adorerai altro dio, perché l'Eterno, che si chiama 'il Geloso', è un Dio geloso.
\par 15 Guardati dal far lega con gli abitanti del paese, affinché, quando quelli si prostituiranno ai loro dèi e offriranno sacrifizi ai loro dèi, non avvenga ch'essi t'invitino, e tu mangi dei loro sacrifizi,
\par 16 e prenda delle loro figliuole per i tuoi figliuoli, e le loro figliuole si prostituiscano ai loro dèi, e inducano i tuoi figliuoli a prostituirsi ai loro dèi.
\par 17 Non ti farai dèi di getto.
\par 18 Osserverai la festa degli azzimi. Sette giorni, al tempo fissato del mese di Abib, mangerai pane senza lievito, come t'ho ordinato; poiché nel mese di Abib tu sei uscito dall'Egitto.
\par 19 Ogni primogenito è mio; e mio è ogni primo parto maschio di tutto il tuo bestiame: del bestiame grosso e minuto.
\par 20 Ma riscatterai con un agnello il primo nato dell'asino; e, se non lo vorrai riscattare, gli fiaccherai il collo. Riscatterai ogni primogenito de' tuoi figliuoli. E nessuno comparirà davanti a me a mani vuote.
\par 21 Lavorerai sei giorni; ma il settimo giorno ti riposerai: ti riposerai anche al tempo dell'aratura e della mietitura.
\par 22 Celebrerai la festa delle settimane: cioè delle primizie della mietitura del frumento, e la festa della raccolta alla fine dell'anno.
\par 23 Tre volte all'anno comparirà ogni vostro maschio nel cospetto del Signore, dell'Eterno, ch'è l'Iddio d'Israele.
\par 24 Poiché io caccerò dinanzi a te delle nazioni, e allargherò i tuoi confini; né alcuno agognerà il tuo paese, quando salirai, tre volte all'anno, per comparire nel cospetto dell'Eterno, ch'è l'Iddio tuo.
\par 25 Non offrirai con pane lievitato il sangue della vittima immolata a me; e il sacrifizio della festa di Pasqua non sarà serbato fino al mattino.
\par 26 Porterai alla casa dell'Eterno Iddio tuo le primizie de' primi frutti della tua terra. Non cuocerai il capretto nel latte di sua madre'.
\par 27 Poi l'Eterno disse a Mosè: 'Scrivi queste parole; perché sul fondamento di queste parole io ho contratto alleanza con te e con Israele'.
\par 28 E Mosè rimase quivi con l'Eterno quaranta giorni e quaranta notti; non mangiò pane e non bevve acqua. E l'Eterno scrisse sulle tavole le parole del patto, le dieci parole.
\par 29 Or Mosè, quando scese dal monte Sinai - scendendo dal monte Mosè aveva in mano le due tavole della testimonianza - non sapeva che la pelle del suo viso era diventata tutta raggiante mentr'egli parlava con l'Eterno;
\par 30 e quando Aaronne e tutti i figliuoli d'Israele videro Mosè, ecco che la pelle del suo viso era tutta raggiante, ed essi temettero d'accostarsi a lui.
\par 31 Ma Mosè li chiamò, ed Aaronne e tutti i capi della raunanza tornarono a lui, e Mosè parlò loro.
\par 32 Dopo questo, tutti i figliuoli d'Israele si accostarono, ed egli ordinò loro tutto quello che l'Eterno gli avea detto sul monte Sinai.
\par 33 E quando Mosè ebbe finito di parlar con loro, si mise un velo sulla faccia.
\par 34 Ma quando Mosè entrava al cospetto dell'Eterno per parlare con lui, si toglieva il velo, finché non tornasse fuori; tornava fuori, e diceva ai figliuoli d'Israele quello che gli era stato comandato.
\par 35 I figliuoli d'Israele, guardando la faccia di Mosè, ne vedeano la pelle tutta raggiante; e Mosè si rimetteva il velo sulla faccia, finché non entrasse a parlare con l'Eterno.

\chapter{35}

\par 1 Mosè convocò tutta la raunanza de' figliuoli d'Israele, e disse loro: 'Queste son le cose che l'Eterno ha ordinato di fare.
\par 2 Sei giorni si dovrà lavorare, ma il settimo giorno sarà per voi un giorno santo, un sabato di solenne riposo, consacrato all'Eterno. Chiunque farà qualche lavoro in esso sarà messo a morte.
\par 3 Non accenderete fuoco in alcuna delle vostre abitazioni il giorno del sabato'.
\par 4 Poi Mosè parlò a tutta la raunanza de' figliuoli d'Israele, e disse: 'Questo è quello che l'Eterno ha ordinato:
\par 5 Prelevate da quello che avete, un'offerta all'Eterno; chiunque è di cuor volenteroso recherà un'offerta all'Eterno: oro, argento, rame;
\par 6 stoffe di color violaceo, porporino, scarlatto, lino fino, pel di capra,
\par 7 pelli di montone tinte in rosso, pelli di delfino, legno d'acacia,
\par 8 olio per il candelabro, aromi per l'olio dell'unzione e per il profumo fragrante,
\par 9 pietre d'ònice, pietre da incastonare per l'efod e per il pettorale.
\par 10 Chiunque tra voi ha dell'abilità venga ed eseguisca tutto quello che l'Eterno ha ordinato:
\par 11 il tabernacolo, la sua tenda e la sua coperta, i suoi fermagli, le sue assi, le sue traverse, le sue colonne e le sue basi,
\par 12 l'arca, le sue stanghe, il propiziatorio e il velo da stender davanti all'arca, la tavola e le sue stanghe,
\par 13 tutti i suoi utensili, e il pane della presentazione;
\par 14 il candelabro per la luce e i suoi utensili, le sue lampade e l'olio per il candelabro;
\par 15 l'altare dei profumi e le sue stanghe, l'olio dell'unzione e il profumo fragrante, la portiera dell'ingresso per l'entrata del tabernacolo,
\par 16 l'altare degli olocausti con la sua gratella di rame, le sue stanghe e tutti i suoi utensili, la conca e la sua base,
\par 17 le cortine del cortile, le sue colonne e le loro basi e la portiera all'ingresso del cortile;
\par 18 i piuoli del tabernacolo e i piuoli del cortile e le loro funi;
\par 19 i paramenti per le cerimonie per fare il servizio nel luogo santo, i paramenti sacri per il sacerdote Aaronne, e i paramenti dei suoi figliuoli per esercitare il sacerdozio'.
\par 20 Allora tutta la raunanza dei figliuoli d'Israele si partì dalla presenza di Mosè.
\par 21 E tutti quelli che il loro cuore spingeva e tutti quelli che il loro spirito rendea volenterosi, vennero a portare l'offerta all'Eterno per l'opera della tenda di convegno, per tutto il suo servizio e per i paramenti sacri.
\par 22 Vennero uomini e donne; quanti erano di cuor volenteroso portarono fermagli, orecchini, anelli da sigillare e braccialetti, ogni sorta di gioielli d'oro; ognuno portò qualche offerta d'oro all'Eterno.
\par 23 E chiunque aveva delle stoffe tinte in violaceo, porporino, scarlatto, o lino fino, o pel di capra, o pelli di montone tinte in rosso, o pelli di delfino, portava ogni cosa.
\par 24 Chiunque prelevò un'offerta d'argento e di rame, portò l'offerta consacrata all'Eterno; e chiunque aveva del legno d'acacia per qualunque lavoro destinato al servizio, lo portò.
\par 25 E tutte le donne abili filarono con le proprie mani e portarono i loro filati in color violaceo, porporino, scarlatto, e del lino fino.
\par 26 E tutte le donne che il cuore spinse ad usare la loro abilità, filarono del pel di capra.
\par 27 E i capi del popolo portarono pietre d'ònice e pietre da incastonare per l'efod e per il pettorale,
\par 28 aromi e olio per il candelabro, per l'olio dell'unzione e per il profumo fragrante.
\par 29 Tutti i figliuoli d'Israele, uomini e donne, che il cuore mosse a portare volenterosamente il necessario per tutta l'opera che l'Eterno aveva ordinata per mezzo di Mosè, recarono all'Eterno delle offerte volontarie.
\par 30 Mosè disse ai figliuoli d'Israele: 'Vedete, l'Eterno ha chiamato per nome Betsaleel figliuolo di Uri, figliuolo di Hur, della tribù di Giuda;
\par 31 e lo ha riempito dello spirito di Dio, di abilità, d'intelligenza e di sapere per ogni sorta di lavori,
\par 32 per concepire opere d'arte, per lavorar l'oro, l'argento e il rame,
\par 33 per incidere pietre da incastonare, per scolpire il legno, per eseguire ogni sorta di lavori d'arte.
\par 34 E gli ha comunicato il dono d'insegnare: a lui ed a Oholiab, figliuolo di Ahisamac, della tribù di Dan.
\par 35 Li ha ripieni d'intelligenza per eseguire ogni sorta di lavori d'artigiano e di disegnatore, di ricamatore e di tessitore in colori svariati: violaceo, porporino, scarlatto, e di lino fino, per eseguire qualunque lavoro e per concepire lavori d'arte.

\chapter{36}

\par 1 E Betsaleel e Oholiab e tutti gli uomini abili, nei quali l'Eterno ha messo sapienza e intelligenza per saper eseguire tutti i lavori per il servizio del santuario, faranno ogni cosa secondo che l'Eterno ha ordinato'.
\par 2 Mosè chiamò dunque Betsaleel e Oholiab e tutti gli uomini abili nei quali l'Eterno avea messo intelligenza, tutti quelli che il cuore moveva ad applicarsi al lavoro per eseguirlo;
\par 3 ed essi presero in presenza di Mosè tutte le offerte recate dai figliuoli d'Israele per i lavori destinati al servizio del santuario, affin di eseguirli. Ma ogni mattina i figliuoli d'Israele continuavano a portare a Mosè delle offerte volontarie.
\par 4 Allora tutti gli uomini abili ch'erano occupati a tutti i lavori del santuario, lasciato ognuno il lavoro che faceva, vennero a dire a Mosè:
\par 5 'Il popolo porta molto più di quel che bisogna per eseguire i lavori che l'Eterno ha comandato di fare'.
\par 6 Allora Mosè dette quest'ordine, che fu bandito per il campo: 'Né uomo né donna faccia più alcun lavoro come offerta per il santuario'. Così s'impedì che il popolo portasse altro.
\par 7 Poiché la roba già pronta bastava a fare tutto il lavoro, e ve n'era d'avanzo.
\par 8 Tutti gli uomini abili, fra quelli che eseguivano il lavoro, fecero dunque il tabernacolo di dieci teli, di lino fino ritorto, e di filo color violaceo, porporino e scarlatto, con dei cherubini artisticamente lavorati.
\par 9 La lunghezza d'un telo era di ventotto cubiti; e la larghezza, di quattro cubiti; tutti i teli erano d'una stessa misura.
\par 10 Cinque teli furono uniti assieme, e gli altri cinque furon pure uniti assieme.
\par 11 Si fecero de' nastri di color violaceo all'orlo del telo ch'era all'estremità della prima serie di teli; e lo stesso si fece all'orlo del telo ch'era all'estremità della seconda serie.
\par 12 Si misero cinquanta nastri al primo telo, e parimente cinquanta nastri all'orlo del telo ch'era all'estremità della seconda serie: i nastri si corrispondevano l'uno all'altro.
\par 13 Si fecero pure cinquanta fermagli d'oro, e si unirono i teli l'uno all'altro mediante i fermagli; e così il tabernacolo formò un tutto.
\par 14 Si fecero inoltre dei teli di pel di capra, per servir da tenda per coprire il tabernacolo: di questi teli se ne fecero undici.
\par 15 La lunghezza d'ogni telo era di trenta cubiti; e la larghezza, di quattro cubiti; gli undici teli aveano la stessa misura.
\par 16 E si unirono insieme, da una parte, cinque teli, e si uniron insieme, dall'altra parte, gli altri sei.
\par 17 E si misero cinquanta nastri all'orlo del telo ch'era all'estremità della prima serie di teli, e cinquanta nastri all'orlo del telo ch'era all'estremità della seconda serie.
\par 18 E si fecero cinquanta fermagli di rame per unire assieme la tenda, in modo che formasse un tutto.
\par 19 Si fece pure per la tenda una coperta di pelli di montone tinte di rosso, e, sopra questa, un'altra di pelli di delfino.
\par 20 Poi si fecero per il tabernacolo le assi di legno d'acacia, messe per ritto.
\par 21 La lunghezza d'un'asse era di dieci cubiti, e la larghezza d'un'asse, di un cubito e mezzo.
\par 22 Ogni asse aveva due incastri paralleli; così fu fatto per tutte le assi del tabernacolo.
\par 23 Si fecero dunque le assi per il tabernacolo: venti assi dal lato meridionale, verso il sud;
\par 24 e si fecero quaranta basi d'argento sotto le venti assi: due basi sotto ciascun'asse per i suoi due incastri.
\par 25 E per il secondo lato del tabernacolo, il lato di nord,
\par 26 si fecero venti assi, con le loro quaranta basi d'argento: due basi sotto ciascun'asse.
\par 27 E per la parte posteriore del tabernacolo, verso occidente, si fecero sei assi.
\par 28 Si fecero pure due assi per gli angoli del tabernacolo, dalla parte posteriore.
\par 29 E queste erano doppie dal basso in su, e al tempo stesso formavano un tutto fino in cima, fino al primo anello. Così fu fatto per ambedue le assi, ch'erano ai due angoli.
\par 30 V'erano dunque otto assi, con le loro basi d'argento: sedici basi: due basi sotto ciascun'asse.
\par 31 E si fecero delle traverse di legno d'acacia: cinque, per le assi di un lato del tabernacolo;
\par 32 cinque traverse per le assi dell'altro lato del tabernacolo, e cinque traverse per le assi della parte posteriore del tabernacolo, a occidente.
\par 33 E si fece la traversa di mezzo, in mezzo alle assi, per farla passare da una parte all'altra.
\par 34 E le assi furon rivestite d'oro, e furon fatti d'oro i loro anelli per i quali dovean passare le traverse, e le traverse furon rivestite d'oro.
\par 35 Fu fatto pure il velo, di filo violaceo, porporino, scarlatto, e di lino fino ritorto con de' cherubini artisticamente lavorati;
\par 36 e si fecero per esso quattro colonne di acacia e si rivestirono d'oro; i loro chiodi erano d'oro; e per le colonne si fusero quattro basi d'argento.
\par 37 Si fece anche per l'ingresso della tenda una portiera, di filo violaceo, porporino, scarlatto, e di lino fino ritorto, in lavoro di ricamo.
\par 38 E si fecero le sue cinque colonne coi loro chiodi; si rivestiron d'oro i loro capitelli e le loro aste; e le loro cinque basi eran di rame.

\chapter{37}

\par 1 Poi Betsaleel fece l'arca di legno d'acacia; la sua lunghezza era di due cubiti e mezzo, la sua larghezza di un cubito e mezzo, e la sua altezza di un cubito e mezzo.
\par 2 E la rivestì d'oro puro di dentro e di fuori, e le fece una ghirlanda d'oro che le girava attorno.
\par 3 E fuse per essa quattro anelli d'oro, che mise ai suoi quattro piedi: due anelli da un lato e due anelli dall'altro lato.
\par 4 Fece anche delle stanghe di legno d'acacia, e le rivestì d'oro.
\par 5 E fece passare le stanghe per gli anelli ai lati dell'arca per portar l'arca.
\par 6 Fece anche un propiziatorio d'oro puro; la sua lunghezza era di due cubiti e mezzo, e la sua larghezza di un cubito e mezzo.
\par 7 E fece due cherubini d'oro; li fece lavorati al martello, alle due estremità del propiziatorio:
\par 8 un cherubino a una delle estremità, e un cherubino all'altra; fece che questi cherubini uscissero dal propiziatorio alle due estremità.
\par 9 E i cherubini aveano le ali spiegate in alto, in modo da coprire il propiziatorio con le ali; aveano la faccia vòlta l'uno verso l'altro; le facce dei cherubini erano volte verso il propiziatorio.
\par 10 Fece anche la tavola di legno d'acacia; la sua lunghezza era di due cubiti, la sua larghezza di un cubito, e la sua altezza di un cubito e mezzo.
\par 11 La rivestì d'oro puro e le fece una ghirlanda d'oro che le girava attorno.
\par 12 E le fece attorno una cornice alta quattro dita; e a questa cornice fece tutt'intorno una ghirlanda d'oro.
\par 13 E fuse per essa quattro anelli d'oro; e mise gli anelli ai quattro canti, ai quattro piedi della tavola.
\par 14 Gli anelli erano vicinissimi alla cornice per farvi passare le stanghe destinate a portar la tavola.
\par 15 E fece le stanghe di legno d'acacia, e le rivestì d'oro; esse dovean servire a portar la tavola.
\par 16 Fece anche, d'oro puro, gli utensili da mettere sulla tavola: i suoi piatti, le sue coppe, le sue tazze e i suoi calici da servire per le libazioni.
\par 17 Fece anche il candelabro d'oro puro; fece il candelabro lavorato al martello, col suo piede e il suo tronco; i suoi calici, i suoi pomi e i suoi fiori erano tutti d'un pezzo col candelabro.
\par 18 Gli uscivano sei bracci dai lati: tre bracci del candelabro da un lato e tre bracci del candelabro dall'altro;
\par 19 sull'uno de' bracci erano tre calici in forma di mandorla, con un pomo e un fiore; e sull'altro braccio, tre calici in forma di mandorla, con un pomo e un fiore. Lo stesso per i sei bracci uscenti dal candelabro.
\par 20 E nel tronco del candelabro v'erano quattro calici in forma di mandorla, coi loro pomi e i loro fiori.
\par 21 E c'era un pomo sotto i due primi bracci che partivano dal candelabro; un pomo sotto i due seguenti bracci che partivano dal candelabro, e un pomo sotto i due ultimi bracci che partivano dal candelabro; così per i sei rami uscenti dal candelabro.
\par 22 Questi pomi e questi bracci erano tutti d'un pezzo col candelabro; il tutto era d'oro puro lavorato al martello.
\par 23 Fece pure le sue lampade, in numero di sette, i suoi smoccolatoi e i suoi porta smoccolature, d'oro puro.
\par 24 Per fare il candelabro con tutti i suoi utensili impiegò un talento d'oro puro.
\par 25 Poi fece l'altare dei profumi, di legno d'acacia; la sua lunghezza era di un cubito; e la sua larghezza di un cubito; era quadro, e aveva un'altezza di due cubiti; i suoi corni erano tutti d'un pezzo con esso.
\par 26 E lo rivestì d'oro puro: il disopra, i suoi lati tutt'intorno, i suoi corni; e gli fece una ghirlanda d'oro che gli girava attorno.
\par 27 Gli fece pure due anelli d'oro, sotto la ghirlanda, ai suoi due lati; li mise ai suoi due lati per passarvi le stanghe che servivano a portarlo.
\par 28 E fece le stanghe di legno d'acacia, e le rivestì d'oro.
\par 29 Poi fece l'olio santo per l'unzione e il profumo fragrante, puro, secondo l'arte del profumiere.

\chapter{38}

\par 1 Poi fece l'altare degli olocausti, di legno d'acacia; la sua lunghezza era di cinque cubiti; e la sua larghezza di cinque cubiti; era quadro, e avea un'altezza di tre cubiti.
\par 2 E ai quattro angoli gli fece dei corni, che spuntavano da esso, e lo rivestì di rame.
\par 3 Fece pure tutti gli utensili dell'altare: i vasi per le ceneri, le palette, i bacini, i forchettoni, i bracieri; tutti i suoi utensili fece di rame.
\par 4 E fece per l'altare una gratella di rame in forma di rete, sotto la cornice nella parte inferiore; in modo che la rete raggiungeva la metà dell'altezza dell'altare.
\par 5 E fuse quattro anelli per i quattro angoli della gratella di rame, per farvi passare le stanghe.
\par 6 Poi fece le stanghe di legno d'acacia, e lo rivestì di rame.
\par 7 E fece passare le stanghe per gli anelli, ai lati dell'altare, le quali dovean servire a portarlo; e lo fece di tavole, vuoto.
\par 8 Poi fece la conca di rame, e la sua base di rame, servendosi degli specchi delle donne che venivano a gruppi a fare il servizio all'ingresso della tenda di convegno.
\par 9 Poi fece il cortile; dal lato meridionale, c'erano, per formare il cortile, cento cubiti di cortine di lino fino ritorto,
\par 10 con le loro venti colonne e le loro venti basi di rame; i chiodi e le aste delle colonne erano d'argento.
\par 11 Dal lato di settentrione, c'erano cento cubiti di cortine con le loro venti colonne e le loro venti basi di rame; i chiodi e le aste delle colonne erano d'argento.
\par 12 Dal lato d'occidente, c'erano cinquanta cubiti di cortine con le loro dieci colonne e le loro dieci basi; i chiodi e le aste delle colonne erano d'argento.
\par 13 E sul davanti, dal lato orientale, c'erano cinquanta cubiti:
\par 14 da uno dei lati dell'ingresso c'erano quindici cubiti di cortine, con tre colonne e le loro tre basi;
\par 15 e dall'altro lato (tanto di qua quanto di là dall'ingresso del cortile) c'erano quindici cubiti di cortine, con le loro tre colonne e le loro tre basi.
\par 16 Tutte le cortine formanti il recinto del cortile erano di lino fino ritorto;
\par 17 e le basi per le colonne eran di rame; i chiodi e le aste delle colonne erano d'argento, e i capitelli delle colonne eran rivestiti d'argento, e tutte le colonne del cortile eran congiunte con delle aste d'argento.
\par 18 La portiera per l'ingresso del cortile era in lavoro di ricamo, di filo violaceo, porporino, scarlatto, e di lino fino ritorto; aveva una lunghezza di venti cubiti, un'altezza di cinque cubiti, corrispondente alla larghezza delle cortine del cortile.
\par 19 Le colonne erano quattro, e quattro le loro basi, di rame; i loro chiodi eran d'argento, e i loro capitelli e le loro aste eran rivestiti d'argento.
\par 20 Tutti i piuoli del tabernacolo e del recinto del cortile erano di rame.
\par 21 Questi sono i conti del tabernacolo, del tabernacolo della testimonianza, che furon fatti per ordine di Mosè, per cura dei Leviti, sotto la direzione d'Ithamar, figliuolo del sacerdote Aaronne.
\par 22 Betsaleel, figliuolo d'Uri, figliuolo di Hur della tribù di Giuda, fece tutto quello che l'Eterno aveva ordinato a Mosè,
\par 23 avendo con sé Oholiab, figliuolo di Ahisamac, della tribù di Dan, scultore, disegnatore, e ricamatore di stoffe violacee, porporine, scarlatte e di lino fino.
\par 24 Tutto l'oro che fu impiegato nell'opera per tutti i lavori del santuario, oro delle offerte, fu ventinove talenti e settecentotrenta sicli, secondo il siclo del santuario.
\par 25 E l'argento di quelli della raunanza de' quali si fece il censimento, fu cento talenti e millesettecentosettantacinque sicli, secondo il siclo del santuario:
\par 26 un beka a testa, vale a dire un mezzo siclo, secondo il siclo del santuario, per ogni uomo compreso nel censimento, dall'età di venti anni in su: cioè, per seicentotremilacinquecentocinquanta uomini.
\par 27 I cento talenti d'argento servirono a fondere le basi del santuario e le basi del velo: cento basi per i cento talenti, un talento per base.
\par 28 E coi millesettecentosettantacinque sicli si fecero dei chiodi per le colonne, si rivestirono i capitelli, e si fecero le aste delle colonne.
\par 29 Il rame delle offerte ammontava a settanta talenti e a duemilaquattrocento sicli.
\par 30 E con questi si fecero le basi dell'ingresso della tenda di convegno, l'altare di rame con la sua gratella di rame, e tutti gli utensili dell'altare,
\par 31 le basi del cortile tutt'all'intorno, le basi dell'ingresso del cortile, tutti i piuoli del tabernacolo e tutti i piuoli del recinto del cortile.

\chapter{39}

\par 1 Poi, con le stoffe tinte in violaceo, porporino e scarlatto, fecero de' paramenti cerimoniali ben lavorati per le funzioni nel santuario, e fecero i paramenti sacri per Aaronne, come l'Eterno aveva ordinato a Mosè.
\par 2 Si fece l'efod, d'oro, di filo violaceo, porporino, scarlatto, e di lino fino ritorto.
\par 3 E batteron l'oro in lamine e lo tagliarono in fili, per intesserlo nella stoffa violacea, porporina, scarlatta, e nel lino fino, e farne un lavoro artistico.
\par 4 Gli fecero delle spallette, unite assieme; in guisa che l'efod era tenuto assieme mediante le sue due estremità.
\par 5 E la cintura artistica che era sull'efod per fissarlo, era tutta d'un pezzo con l'efod, e del medesimo lavoro d'esso: cioè, d'oro, di filo violaceo, porporino, scarlatto, e di lino fino ritorto, come l'Eterno aveva ordinato a Mosè.
\par 6 Poi lavorarono le pietre d'ònice, incastrate in castoni d'oro, sulle quali incisero i nomi dei figliuoli d'Israele, come s'incidono i sigilli.
\par 7 E le misero sulle spallette dell'efod, come pietre di ricordanza per i figliuoli d'Israele, nel modo che l'Eterno aveva ordinato a Mosè.
\par 8 Poi si fece il pettorale, artisticamente lavorato come il lavoro dell'efod: d'oro, di filo violaceo, porporino, scarlatto, e di lino fino ritorto.
\par 9 Il pettorale era quadrato; e lo fecero doppio; aveva la lunghezza d'una spanna e una spanna di larghezza; era doppio.
\par 10 E v'incastonarono quattro ordini di pietre; nel primo ordine v'era un sardonio, un topazio e uno smeraldo;
\par 11 nel secondo ordine, un rubino, uno zaffiro, un calcedonio;
\par 12 nel terzo ordine, un'opale, un'agata, un'ametista;
\par 13 nel quarto ordine, un grisolito, un'ònice e un diaspro. Queste pietre erano incastrate nei loro castoni d'oro.
\par 14 E le pietre corrispondevano ai nomi dei figliuoli d'Israele, ed erano dodici, secondo i loro nomi; erano incise come de' sigilli, ciascuna col nome d'una delle dodici tribù.
\par 15 Fecero pure sul pettorale delle catenelle d'oro puro, intrecciate a mo' di cordoni.
\par 16 E fecero due castoni d'oro e due anelli d'oro, e misero i due anelli alle due estremità del pettorale.
\par 17 E fissarono i due cordoni d'oro ai due anelli alle estremità del pettorale;
\par 18 e attaccarono gli altri due capi dei due cordoni d'oro ai due castoni, e li misero sulle due spallette dell'efod, sul davanti.
\par 19 Fecero anche due anelli d'oro e li misero alle altre due estremità del pettorale, sull'orlo interiore vòlto verso l'efod.
\par 20 E fecero due altri anelli d'oro, e li misero alle due spallette dell'efod, in basso, sul davanti, vicino al punto dove avveniva la giuntura, al disopra della cintura artistica dell'efod.
\par 21 E attaccarono il pettorale mediante i suoi anelli agli anelli dell'efod con un cordone violaceo, affinché il pettorale fosse al disopra della banda artisticamente lavorata dell'efod, e non si potesse staccare dall'efod; come l'Eterno aveva ordinato a Mosè.
\par 22 Si fece pure il manto dell'efod, di lavoro tessuto tutto di color violaceo,
\par 23 e l'apertura, in mezzo al manto, per passarvi il capo: apertura, come quella d'una corazza, con all'intorno un'orlatura tessuta, perché non si strappasse.
\par 24 E all'orlo inferiore del manto fecero delle melagrane di color violaceo, porporino e scarlatto, di filo ritorto.
\par 25 E fecero de' sonagli d'oro puro; e posero i sonagli in mezzo alle melagrane all'orlo inferiore del manto, tutt'all'intorno, fra le melagrane:
\par 26 un sonaglio e una melagrana, un sonaglio e una melagrana, sull'orlatura del manto, tutt'all'intorno, per fare il servizio, come l'Eterno aveva ordinato a Mosè.
\par 27 Si fecero pure le tuniche di lino fino, di lavoro tessuto, per Aaronne e per i suoi figliuoli,
\par 28 e la mitra di lino fino e le tiare di lino fino da servir come ornamento e le brache di lino fino ritorto,
\par 29 e la cintura di lino fino ritorto, di color violaceo, porporino, scarlatto, in lavoro di ricamo, come l'Eterno aveva ordinato a Mosè.
\par 30 E fecero d'oro puro la lamina del sacro diadema, e v'incisero, come s'incide sopra un sigillo: SANTO ALL'ETERNO.
\par 31 E v'attaccarono un nastro violaceo per fermarla sulla mitra, in alto, come l'Eterno aveva ordinato a Mosè.
\par 32 Così fu finito tutto il lavoro del tabernacolo e della tenda di convegno. I figliuoli d'Israele fecero interamente come l'Eterno aveva ordinato a Mosè; fecero a quel modo.
\par 33 Poi portarono a Mosè il tabernacolo, la tenda e tutti i suoi utensili, i suoi fermagli, le sue tavole, le sue traverse, le sue colonne, le sue basi;
\par 34 la coperta di pelli di montone tinte in rosso, la coperta di pelli di delfino, e il velo di separazione;
\par 35 l'arca della testimonianza con le sue stanghe, e il propiziatorio;
\par 36 la tavola con tutti i suoi utensili e il pane della presentazione;
\par 37 il candelabro d'oro puro con le sue lampade, le lampade disposte in ordine, tutti i suoi utensili, e l'olio per il candelabro;
\par 38 l'altare d'oro, l'olio dell'unzione, il profumo fragrante, e la portiera per l'ingresso della tenda;
\par 39 l'altare di rame, la sua gratella di rame, le sue stanghe e tutti i suoi utensili, la conca con la sua base;
\par 40 le cortine del cortile, le sue colonne con le sue basi, la portiera per l'ingresso del cortile, i cordami del cortile, i suoi piuoli e tutti gli utensili per il servizio del tabernacolo, per la tenda di convegno;
\par 41 i paramenti cerimoniali per le funzioni nel santuario, i paramenti sacri per il sacerdote Aaronne e i paramenti de' suoi figliuoli per esercitare il sacerdozio.
\par 42 I figliuoli d'Israele eseguirono tutto il lavoro, secondo che l'Eterno aveva ordinato a Mosè.
\par 43 E Mosè vide tutto il lavoro; ed ecco, essi l'avevano eseguito come l'Eterno aveva ordinato; l'avevano eseguito a quel modo. E Mosè li benedisse.

\chapter{40}

\par 1 L'Eterno parlò a Mosè, dicendo:
\par 2 'Il primo giorno del primo mese erigerai il tabernacolo, la tenda di convegno.
\par 3 Vi porrai l'arca della testimonianza, e stenderai il velo dinanzi all'arca.
\par 4 Vi porterai dentro la tavola, e disporrai in ordine le cose che vi son sopra; vi porterai pure il candelabro e accenderai le sue lampade.
\par 5 Porrai l'altare d'oro per i profumi davanti all'arca della testimonianza e metterai la portiera all'ingresso del tabernacolo.
\par 6 Porrai l'altare degli olocausti davanti all'ingresso del tabernacolo, della tenda di convegno.
\par 7 Metterai la conca fra la tenda di convegno e l'altare, e vi metterai dentro dell'acqua.
\par 8 Stabilirai il cortile tutt'intorno, e attaccherai la portiera all'ingresso del cortile.
\par 9 Poi prenderai l'olio dell'unzione e ungerai il tabernacolo e tutto ciò che v'è dentro, lo consacrerai con tutti i suoi utensili, e sarà santo.
\par 10 Ungerai pure l'altare degli olocausti e tutti i suoi utensili, consacrerai l'altare, e l'altare sarà santissimo.
\par 11 Ungerai anche la conca con la sua base, e la consacrerai.
\par 12 Poi farai accostare Aaronne e i suoi figliuoli all'ingresso della tenda di convegno, e li laverai con acqua.
\par 13 Rivestirai Aaronne de' paramenti sacri, e lo ungerai e lo consacrerai, perché mi eserciti l'ufficio di sacerdote.
\par 14 Farai pure accostare i suoi figliuoli, li rivestirai di tuniche,
\par 15 e li ungerai come avrai unto il loro padre, perché mi esercitino l'ufficio di sacerdoti; e la loro unzione conferirà loro un sacerdozio perpetuo, di generazione in generazione'.
\par 16 E Mosè fece così: fece interamente come l'Eterno gli aveva ordinato.
\par 17 E il primo giorno del primo mese del secondo anno, il tabernacolo fu eretto.
\par 18 Mosè eresse il tabernacolo, ne pose le basi, ne collocò le assi, ne mise le traverse e ne rizzò le colonne.
\par 19 Stese la tenda sul tabernacolo, e sopra la tenda pose la coperta d'essa, come l'Eterno aveva ordinato a Mosè.
\par 20 Poi prese la testimonianza e la pose dentro l'arca, mise le stanghe all'arca, e collocò il propiziatorio sull'arca;
\par 21 portò l'arca nel tabernacolo, sospese il velo di separazione e coprì con esso l'arca della testimonianza, come l'Eterno aveva ordinato a Mosè.
\par 22 Pose pure la tavola nella tenda di convegno, dal lato settentrionale del tabernacolo, fuori del velo.
\par 23 Vi dispose sopra in ordine il pane, davanti all'Eterno, come l'Eterno aveva ordinato a Mosè.
\par 24 Poi mise il candelabro nella tenda di convegno, dirimpetto alla tavola, dal lato meridionale del tabernacolo;
\par 25 e accese le lampade davanti all'Eterno, come l'Eterno aveva ordinato a Mosè.
\par 26 Poi mise l'altare d'oro nella tenda di convegno, davanti al velo,
\par 27 e vi bruciò su il profumo fragrante, come l'Eterno aveva ordinato a Mosè.
\par 28 Mise pure la portiera all'ingresso del tabernacolo.
\par 29 Poi collocò l'altare degli olocausti all'ingresso del tabernacolo della tenda di convegno, e v'offrì sopra l'olocausto e l'oblazione, come l'Eterno aveva ordinato a Mosè.
\par 30 E pose la conca fra la tenda di convegno e l'altare, e vi pose dentro dell'acqua per le abluzioni.
\par 31 E Mosè ed Aaronne e i suoi figliuoli vi si lavarono le mani e i piedi;
\par 32 quando entravano nella tenda di convegno e quando s'accostavano all'altare, si lavavano, come l'Eterno aveva ordinato a Mosè.
\par 33 Eresse pure il cortile attorno al tabernacolo e all'altare, e sospese la portiera all'ingresso del cortile. Così Mosè compié l'opera.
\par 34 Allora la nuvola coprì la tenda di convegno, e la gloria dell'Eterno riempì il tabernacolo.
\par 35 E Mosè non poté entrare nella tenda di convegno perché la nuvola vi s'era posata sopra, e la gloria dell'Eterno riempiva il tabernacolo.
\par 36 Or durante tutti i loro viaggi, quando la nuvola s'alzava di sul tabernacolo, i figliuoli d'Israele partivano;
\par 37 ma se la nuvola non s'alzava, non partivano fino al giorno che s'alzasse.
\par 38 Poiché la nuvola dell'Eterno stava sul tabernacolo durante il giorno; e di notte vi stava un fuoco, a vista di tutta la casa d'Israele durante tutti i loro viaggi.


\end{document}