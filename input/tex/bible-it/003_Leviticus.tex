\begin{document}

\title{Leviticus}


\chapter{1}

\par 1 L'Eterno chiamò Mosè e gli parlò dalla tenda di convegno, dicendo:
\par 2 'Parla ai figliuoli d'Israele e di' loro: Quando qualcuno tra voi recherà un'offerta all'Eterno, l'offerta che recherete sarà di bestiame: di capi d'armento o di capi di gregge.
\par 3 Se la sua offerta è un olocausto di capi d'armento, offrirà un maschio senza difetto; l'offrirà all'ingresso della tenda di convegno, per ottenere il favore dell'Eterno.
\par 4 E poserà la mano sulla testa dell'olocausto, il quale sarà accetto all'Eterno, per fare espiazione per lui.
\par 5 Poi scannerà il vitello davanti all'Eterno; e i sacerdoti, figliuoli d'Aaronne, offriranno il sangue, e lo spargeranno tutt'intorno sull'altare, che è all'ingresso della tenda di convegno.
\par 6 Si trarrà quindi la pelle all'olocausto, e lo si taglierà a pezzi.
\par 7 E i figliuoli del sacerdote Aaronne metteranno del fuoco sull'altare, e accomoderanno delle legna sul fuoco.
\par 8 Poi i sacerdoti, figliuoli d'Aaronne, disporranno que' pezzi, la testa e il grasso, sulle legna messe sul fuoco sopra l'altare;
\par 9 ma le interiora e le gambe si laveranno con acqua, e il sacerdote farà fumare ogni cosa sull'altare, come un olocausto, un sacrifizio di soave odore, fatto mediante il fuoco all'Eterno.
\par 10 Se la sua offerta è un olocausto di capi di gregge, di pecore o di capre, offrirà un maschio senza difetto.
\par 11 Lo scannerà dal lato settentrionale dell'altare, davanti all'Eterno; e i sacerdoti, figliuoli d'Aaronne, ne spargeranno il sangue sull'altare, tutt'intorno.
\par 12 Poi lo si taglierà a pezzi, che, insieme colla testa e col grasso, il sacerdote disporrà sulle legna messe sul fuoco sopra l'altare;
\par 13 ma le interiora e le gambe si laveranno con acqua, e il sacerdote offrirà ogni cosa e la farà fumare sull'altare. Questo è un olocausto, un sacrifizio di soave odore, fatto mediante il fuoco all'Eterno.
\par 14 Se la sua offerta all'Eterno è un olocausto d'uccelli, offrirà delle tortore o de' giovani piccioni.
\par 15 Il sacerdote offrirà in sacrifizio l'uccello sull'altare, gli spiccherà la testa, la farà fumare sull'altare, e il sangue d'esso sarà fatto scorrere sopra uno de' lati dell'altare.
\par 16 Poi gli toglierà il gozzo con quel che contiene, e getterà tutto allato all'altare, verso oriente, nel luogo delle ceneri.
\par 17 Spaccherà quindi l'uccello per le ali, senza però dividerlo in due, e il sacerdote lo farà fumare sull'altare, sulle legna messe sopra il fuoco. Questo è un olocausto, un sacrifizio di soave odore, fatto mediante il fuoco all'Eterno.

\chapter{2}

\par 1 Quando qualcuno presenterà all'Eterno come offerta una oblazione, la sua offerta sarà di fior di farina; vi verserà sopra dell'olio e v'aggiungerà dell'incenso.
\par 2 E la porterà ai sacerdoti figliuoli d'Aaronne; e il sacerdote prenderà una manata piena del fior di farina spruzzata d'olio, con tutto l'incenso, e farà fumare ogni cosa sull'altare, come ricordanza. Questo è un sacrifizio di soave odore, fatto mediante il fuoco all'Eterno.
\par 3 Ciò che rimarrà dell'oblazione sarà per Aaronne e per i suoi figliuoli; è cosa santissima tra i sacrifizi fatti mediante il fuoco all'Eterno.
\par 4 E quando offrirai un'oblazione di cosa cotta in forno, ti servirai di focacce non lievitate di fior di farina impastata con olio, e di gallette senza lievito unte d'olio.
\par 5 E se la tua offerta è un'oblazione cotta sulla gratella, sarà di fior di farina, impastata con olio, senza lievito.
\par 6 La farai a pezzi, e vi verserai su dell'olio; è un'oblazione.
\par 7 E se la tua offerta è un'oblazione cotta in padella, sarà fatta di fior di farina con olio.
\par 8 Porterai all'Eterno l'oblazione fatta di queste cose; sarà presentata al sacerdote, che la porterà sull'altare.
\par 9 Il sacerdote preleverà dall'oblazione la parte che dev'essere offerta come ricordanza, e la farà fumare sull'altare. È un sacrifizio di soave odore, fatto mediante il fuoco all'Eterno.
\par 10 Ciò che rimarrà dell'oblazione sarà per Aaronne e per i suoi figliuoli; è cosa santissima tra i sacrifizi fatti mediante il fuoco all'Eterno.
\par 11 Qualunque oblazione offrirete all'Eterno sarà senza lievito; poiché non farete fumar nulla che contenga lievito o miele, come sacrifizio fatto mediante il fuoco all'Eterno.
\par 12 Potrete offrirne all'Eterno come oblazione di primizie; ma queste offerte non saranno poste sull'altare come offerte di soave odore.
\par 13 E ogni oblazione che offrirai la condirai con sale, e non lascerai la tua oblazione mancar di sale, segno del patto del tuo Dio. Su tutte le tue offerte offrirai del sale.
\par 14 E se offri all'Eterno un'oblazione di primizie, offrirai, come oblazione delle tue primizie, delle spighe tostate al fuoco, chicchi di grano nuovo, tritati.
\par 15 E vi porrai su dell'olio e v'aggiungerai dell'incenso: è un'oblazione.
\par 16 E il sacerdote farà fumare come ricordanza una parte del grano tritato e dell'olio, con tutto l'incenso. È un sacrifizio fatto mediante il fuoco all'Eterno.

\chapter{3}

\par 1 Quand'uno offrirà un sacrifizio di azioni di grazie, se offre capi d'armenti, un maschio o una femmina, l'offrirà senza difetto davanti all'Eterno.
\par 2 Poserà la mano sulla testa della sua offerta, e la sgozzerà all'ingresso della tenda di convegno; e i sacerdoti, figliuoli d'Aaronne, spargeranno il sangue sull'altare tutt'intorno.
\par 3 E di questo sacrifizio di azioni di grazie offrirà, come sacrifizio mediante il fuoco all'Eterno, il grasso che copre le interiora e tutto il grasso che aderisce alle interiora,
\par 4 i due arnioni e il grasso che v'è sopra e che copre i fianchi, e la rete del fegato, che staccherà vicino agli arnioni.
\par 5 E i figliuoli d'Aaronne faranno fumare tutto questo sull'altare sopra l'olocausto, che è sulle legna messe sul fuoco. Questo è un sacrifizio di soave odore, fatto mediante il fuoco all'Eterno.
\par 6 Se l'offerta ch'egli fa come sacrifizio di azioni di grazie all'Eterno è di capi di gregge, un maschio o una femmina, l'offrirà senza difetto.
\par 7 Se presenta come offerta un agnello, l'offrirà davanti all'Eterno.
\par 8 Poserà la mano sulla testa della sua offerta, e la sgozzerà all'ingresso della tenda di convegno; e i figliuoli d'Aaronne ne spargeranno il sangue sull'altare tutt'intorno.
\par 9 E di questo sacrifizio di azioni di grazie offrirà, come sacrifizio mediante il fuoco all'Eterno, il grasso, tutta la coda ch'egli staccherà presso l'estremità della spina, il grasso che copre le interiora e tutto il grasso che aderisce alle interiora,
\par 10 i due arnioni e il grasso che v'è sopra e che copre i fianchi, e la rete del fegato, che staccherà vicino agli arnioni.
\par 11 E il sacerdote farà fumare tutto questo sull'altare. È un cibo offerto mediante il fuoco all'Eterno.
\par 12 Se la sua offerta è una capra, l'offrirà davanti all'Eterno.
\par 13 Poserà la mano sulla testa della vittima, e la sgozzerà all'ingresso della tenda di convegno; e i figliuoli d'Aaronne ne spargeranno il sangue sull'altare tutt'intorno.
\par 14 E della vittima offrirà, come sacrifizio mediante il fuoco all'Eterno, il grasso che copre le interiora e tutto il grasso che aderisce alle interiora,
\par 15 i due arnioni e il grasso che v'è sopra e che copre i fianchi, e la rete del fegato, che staccherà vicino agli arnioni.
\par 16 E il sacerdote farà fumare tutto questo sull'altare. È un cibo di soave odore, offerto mediante il fuoco. Tutto il grasso appartiene all'Eterno.
\par 17 Questa è una legge perpetua, per tutte le vostre generazioni, e in tutti i luoghi dove abiterete: non mangerete né grasso né sangue'.

\chapter{4}

\par 1 L'Eterno parlò ancora a Mosè, dicendo:
\par 2 'Parla ai figliuoli d'Israele e di' loro: Quando qualcuno avrà peccato per errore e avrà fatto alcuna delle cose che l'Eterno ha vietato di fare,
\par 3 se il sacerdote che ha ricevuto l'unzione è quegli che ha peccato, rendendo per tal modo colpevole il popolo, offrirà all'Eterno, per il peccato commesso, un giovenco senza difetto, come sacrifizio per il peccato.
\par 4 Menerà il giovenco all'ingresso della tenda di convegno, davanti all'Eterno; poserà la mano sulla testa del giovenco, e sgozzerà il giovenco davanti all'Eterno.
\par 5 Poi il sacerdote che ha ricevuto l'unzione prenderà del sangue del giovenco e lo porterà entro la tenda di convegno;
\par 6 e il sacerdote intingerà il suo dito nel sangue, e farà aspersione di quel sangue sette volte davanti all'Eterno, di fronte al velo del santuario.
\par 7 Il sacerdote quindi metterà di quel sangue sui corni dell'altare del profumo fragrante, altare che è davanti all'Eterno, nella tenda di convegno; e spanderà tutto il sangue del giovenco appiè dell'altare degli olocausti, che è all'ingresso della tenda di convegno.
\par 8 E torrà dal giovenco del sacrifizio per il peccato tutto il grasso: il grasso che copre le interiora e tutto il grasso che aderisce alle interiora,
\par 9 i due arnioni e il grasso che v'è sopra e che copre i fianchi,
\par 10 e la rete del fegato, che staccherà vicino agli arnioni, nello stesso modo che queste parti si tolgono dal bue del sacrifizio di azioni di grazie; e il sacerdote le farà fumare sull'altare degli olocausti.
\par 11 Ma la pelle del giovenco e tutta la sua carne, con la sua testa, le sue gambe, le sue interiora e i suoi escrementi,
\par 12 il giovenco intero, lo porterà fuori del campo, in un luogo puro, dove si gettan le ceneri; e lo brucerà col fuoco, su delle legna; sarà bruciato sul mucchio delle ceneri.
\par 13 Se tutta la raunanza d'Israele ha peccato per errore, senz'accorgersene, e ha fatto alcuna delle cose che l'Eterno ha vietato di fare, e si è così resa colpevole,
\par 14 quando il peccato che ha commesso venga ad esser conosciuto, la raunanza offrirà, come sacrifizio per il peccato, un giovenco, e lo menerà davanti alla tenda di convegno.
\par 15 Gli anziani della raunanza poseranno le mani sulla testa del giovenco davanti all'Eterno; e il giovenco sarà sgozzato davanti all'Eterno.
\par 16 Poi il sacerdote che ha ricevuto l'unzione porterà del sangue del giovenco entro la tenda di convegno;
\par 17 e il sacerdote intingerà il dito nel sangue e ne farà aspersione sette volte davanti all'Eterno, di fronte al velo.
\par 18 E metterà di quel sangue sui corni dell'altare che è davanti all'Eterno, nella tenda di convegno; e spanderà tutto il sangue appiè dell'altare dell'olocausto, che è all'ingresso della tenda di convegno.
\par 19 E torrà dal giovenco tutto il grasso, e lo farà fumare sull'altare.
\par 20 Farà di questo giovenco, come ha fatto del giovenco offerto per il peccato. Così il sacerdote farà l'espiazione per la raunanza, e le sarà perdonato.
\par 21 Poi porterà il giovenco fuori del campo, e lo brucerà come ha bruciato il primo giovenco. Questo è il sacrifizio per il peccato della raunanza.
\par 22 Se uno dei capi ha peccato, e ha fatto per errore alcuna di tutte le cose che l'Eterno Iddio suo ha vietato di fare, e si è così reso colpevole,
\par 23 quando il peccato che ha commesso gli sarà fatto conoscere, menerà, come sua offerta, un becco, un maschio fra le capre, senza difetto.
\par 24 Poserà la mano sulla testa del becco, e lo scannerà nel luogo dove si scannano gli olocausti, davanti all'Eterno. È un sacrifizio per il peccato.
\par 25 Poi il sacerdote prenderà col suo dito del sangue del sacrifizio per il peccato, e lo metterà sui corni dell'altare degli olocausti, e spanderà il sangue del becco appiè dell'altare dell'olocausto;
\par 26 e farà fumare tutto il grasso del becco sull'altare, come ha fatto del grasso del sacrifizio di azioni di grazie. Così il sacerdote farà l'espiazione del peccato di lui, e gli sarà perdonato.
\par 27 Se qualcuno del popolo del paese peccherà per errore e farà alcuna delle cose che l'Eterno ha vietato di fare, rendendosi così colpevole,
\par 28 quando il peccato che ha commesso gli sarà fatto conoscere, dovrà menare, come sua offerta, una capra, una femmina senza difetto, per il peccato che ha commesso.
\par 29 Poserà la mano sulla testa del sacrifizio per il peccato, e sgozzerà il sacrifizio per il peccato nel luogo ove si sgozzano gli olocausti.
\par 30 Poi il sacerdote prenderà col suo dito del sangue della capra e lo metterà sui corni dell'altare dell'olocausto, e spanderà tutto il sangue della capra appiè dell'altare.
\par 31 E torrà tutto il grasso dalla capra, come ha tolto il grasso dal sacrifizio di azioni di grazie; e il sacerdote lo farà fumare sull'altare come un soave odore all'Eterno. Così il sacerdote farà l'espiazione per quel tale, e gli sarà perdonato.
\par 32 E se colui menerà un agnello come suo sacrifizio per il peccato, dovrà menare una femmina senza difetto.
\par 33 Poserà la mano sulla testa del sacrifizio per il peccato, e lo sgozzerà come sacrifizio per il peccato nel luogo ove si sgozzano gli olocausti.
\par 34 Poi il sacerdote prenderà col suo dito del sangue del sacrifizio per il peccato, e lo metterà sui corni dell'altare dell'olocausto, e spanderà tutto il sangue della vittima appiè dell'altare;
\par 35 e torrà dalla vittima tutto il grasso, come si toglie il grasso dall'agnello del sacrifizio di azioni di grazie; e il sacerdote lo farà fumare sull'altare, sui sacrifizi fatti mediante il fuoco all'Eterno. Così il sacerdote farà per quel tale l'espiazione del peccato che ha commesso, e gli sarà perdonato.

\chapter{5}

\par 1 Quando una persona, dopo aver udito dal giudice la formula del giuramento, nella sua qualità di testimonio pecca non dichiarando ciò che ha veduto o altrimenti conosciuto, porterà la pena della sua iniquità.
\par 2 O quand'uno, senza saperlo, avrà toccato qualcosa d'impuro, come il cadavere d'una bestia selvatica impura, o il cadavere d'un animale domestico impuro, o quello d'un rettile impuro, rimarrà egli stesso impuro e colpevole.
\par 3 O quando, senza saperlo, toccherà una impurità umana - una qualunque delle cose per le quali l'uomo diviene impuro - allorché viene a saperlo, è colpevole.
\par 4 O quand'uno, senza badarvi, parlando leggermente con le labbra, avrà giurato, con uno di quei giuramenti che gli uomini sogliono proferire alla leggera, di fare qualcosa di male o di bene, allorché viene ad accorgersene, è colpevole.
\par 5 Quand'uno dunque si sarà reso colpevole d'una di queste cose, confesserà il peccato che ha commesso;
\par 6 recherà all'Eterno, come sacrifizio della sua colpa, per il peccato che ha commesso, una femmina del gregge, una pecora o una capra, come sacrifizio per il peccato; e il sacerdote farà per lui l'espiazione del suo peccato.
\par 7 Se non ha mezzi da procurarsi una pecora o una capra, porterà all'Eterno, come sacrifizio della sua colpa, per il suo peccato, due tortore o due giovani piccioni: uno come sacrifizio per il peccato, l'altro come olocausto.
\par 8 E li porterà al sacerdote, il quale offrirà prima quello per il peccato; gli spiccherà la testa vicino alla nuca, ma senza staccarla del tutto;
\par 9 poi spargerà del sangue del sacrifizio per il peccato sopra uno dei lati dell'altare, e il resto del sangue sarà spremuto appiè dell'altare. Questo è un sacrifizio per il peccato.
\par 10 Dell'altro uccello farà un olocausto, secondo le norme stabilite. Così il sacerdote farà per quel tale l'espiazione del peccato che ha commesso, e gli sarà perdonato.
\par 11 Ma se non ha mezzi da procurarsi due tortore o due giovani piccioni, porterà, come sua offerta per il peccato che ha commesso, la decima parte di un efa di fior di farina, come sacrifizio per il peccato; non vi metterà su né olio né incenso, perché è un sacrifizio per il peccato.
\par 12 Porterà la farina al sacerdote, e il sacerdote ne prenderà una manata piena come ricordanza, e la farà fumare sull'altare sopra i sacrifizi fatti mediante il fuoco all'Eterno. È un sacrifizio per il peccato.
\par 13 Così il sacerdote farà per quel tale l'espiazione del peccato che ha commesso in uno di quei casi, e gli sarà perdonato. Il resto della farina sarà per il sacerdote come si fa nell'oblazione'.
\par 14 L'Eterno parlò ancora a Mosè, dicendo:
\par 15 'Quand'uno commetterà una infedeltà e peccherà per errore relativamente a ciò che dev'esser consacrato all'Eterno, porterà all'Eterno, come sacrifizio di riparazione, un montone senza difetto, preso dal gregge, secondo la tua stima in sicli d'argento a siclo di santuario, come sacrifizio di riparazione.
\par 16 E risarcirà il danno fatto al santuario, aggiungendovi un quinto in più e lo darà al sacerdote; e il sacerdote farà per lui l'espiazione col montone offerto come sacrifizio di riparazione, e gli sarà perdonato.
\par 17 E quand'uno peccherà facendo, senza saperlo, qualcuna delle cose che l'Eterno ha vietato di fare, sarà colpevole e porterà la pena della sua iniquità.
\par 18 Presenterà al sacerdote, come sacrifizio di riparazione, un montone senza difetto, preso dal gregge, secondo la tua stima; e il sacerdote farà per lui l'espiazione dell'errore commesso per ignoranza, e gli sarà perdonato.
\par 19 Questo è un sacrifizio di riparazione; quel tale si è realmente reso colpevole verso l'Eterno'.

\chapter{6}

\par 1 E l'Eterno parlò a Mosè dicendo:
\par 2 'Quand'uno peccherà e commetterà una infedeltà verso l'Eterno, negando al suo prossimo un deposito da lui ricevuto, o un pegno messo nelle sue mani, o una cosa che ha rubata o estorta con frode al prossimo,
\par 3 o una cosa perduta che ha trovata, e mentendo a questo proposito e giurando il falso circa una delle cose nelle quali l'uomo può peccare,
\par 4 quando avrà così peccato e si sarà reso colpevole, restituirà la cosa rubata o estorta con frode, o il deposito che gli era stato confidato, o l'oggetto perduto che ha trovato,
\par 5 o qualunque cosa circa la quale abbia giurato il falso. Ne farà la restituzione per intero e v'aggiungerà un quinto in più, consegnandola al proprietario il giorno stesso che offrirà il suo sacrifizio di riparazione.
\par 6 E porterà al sacerdote il suo sacrifizio di riparazione all'Eterno: un montone senza difetto, preso dal gregge, secondo la tua stima, come sacrifizio di riparazione.
\par 7 E il sacerdote farà l'espiazione per lui davanti all'Eterno, e gli sarà perdonato qualunque sia la cosa di cui si è reso colpevole'.
\par 8 L'Eterno parlò ancora a Mosè, dicendo:
\par 9 'Da' quest'ordine ad Aaronne e ai suoi figliuoli, e di' loro: Questa è la legge dell'olocausto. L'olocausto rimarrà sulle legna accese sopra l'altare tutta la notte, fino al mattino; e il fuoco dell'altare sarà tenuto acceso.
\par 10 Il sacerdote si vestirà della sua tunica di lino e si metterà sulla carne le brache; leverà la cenere fatta dal fuoco che avrà consumato l'olocausto sull'altare e la porrà allato all'altare.
\par 11 Poi si spoglierà delle vesti e ne indosserà delle altre, e porterà la cenere fuori del campo, in un luogo puro.
\par 12 Il fuoco sarà mantenuto acceso sull'altare e non si lascerà spegnere; e il sacerdote vi brucerà su delle legna ogni mattina, vi disporrà sopra l'olocausto, e vi farà fumar sopra il grasso dei sacrifizi di azioni di grazie.
\par 13 Il fuoco dev'esser del continuo mantenuto acceso sull'altare, e non si lascerà spegnere.
\par 14 Questa è la legge dell'oblazione. I figliuoli d'Aaronne l'offriranno davanti all'Eterno, dinanzi all'altare.
\par 15 Si leverà una manata di fior di farina con il suo olio e tutto l'incenso che è sull'oblazione, e si farà fumare ogni cosa sull'altare in sacrifizio di soave odore, come una ricordanza per l'Eterno.
\par 16 Aaronne e i suoi figliuoli mangeranno quel che rimarrà dell'oblazione; la si mangerà senza lievito, in luogo santo; la mangeranno nel cortile della tenda di convegno.
\par 17 Non la si cocerà con lievito; è la parte che ho data loro de' miei sacrifizi fatti mediante il fuoco. È cosa santissima, come il sacrifizio per il peccato e come il sacrifizio di riparazione.
\par 18 Ogni maschio tra i figliuoli d'Aaronne ne potrà mangiare. È una parte perpetua, assegnatavi di generazione in generazione, sui sacrifizi fatti mediante il fuoco all'Eterno. Chiunque toccherà quelle cose dovrà esser santo'.
\par 19 L'Eterno parlò ancora a Mosè, dicendo:
\par 20 'Questa è l'offerta che Aaronne e i suoi figliuoli faranno all'Eterno il giorno che riceveranno l'unzione: un decimo d'efa di fior di farina, come oblazione perpetua, metà la mattina e metà la sera.
\par 21 Essa sarà preparata con olio, sulla gratella; la porterai quando sarà fritta; l'offrirai in pezzi, come offerta divisa di soave odore all'Eterno;
\par 22 e il sacerdote che, tra i figliuoli d'Aaronne, sarà unto per succedergli, farà anch'egli quest'offerta; è la parte assegnata in perpetuo all'Eterno; sarà fatta fumare per intero.
\par 23 Ogni oblazione del sacerdote sarà fatta fumare per intero; non sarà mangiata'.
\par 24 L'Eterno parlò ancora a Mosè, dicendo:
\par 25 'Parla ad Aaronne e ai suoi figliuoli, e di' loro: Questa è la legge del sacrifizio per il peccato. Nel luogo dove si sgozza l'olocausto, sarà sgozzata, davanti all'Eterno, la vittima per il peccato. È cosa santissima.
\par 26 Il sacerdote che l'offrirà per il peccato, la mangerà; dovrà esser mangiata in luogo santo, nel cortile della tenda di convegno.
\par 27 Chiunque ne toccherà la carne dovrà esser santo; e se ne schizza del sangue sopra una veste, il posto ove sarà schizzato il sangue lo laverai in luogo santo.
\par 28 Ma il vaso di terra che avrà servito a cuocerla, sarà spezzato; e se è stata cotta in un vaso di rame, questo si strofini bene e si sciacqui con acqua.
\par 29 Ogni maschio, fra i sacerdoti, ne potrà mangiare; è cosa santissima.
\par 30 Ma non si mangerà alcuna vittima per il peccato, quando si deve portare del sangue d'essa nella tenda di convegno per fare l'espiazione nel santuario. Essa sarà bruciata col fuoco.

\chapter{7}

\par 1 Questa è la legge del sacrifizio di riparazione; è cosa santissima.
\par 2 Nel luogo ove si scanna l'olocausto, si scannerà la vittima di riparazione; e se ne spanderà il sangue sull'altare tutt'intorno;
\par 3 e se ne offrirà tutto il grasso, la coda, il grasso che copre le interiora,
\par 4 i due arnioni, il grasso che v'è sopra e che copre i fianchi, e la rete del fegato, che si staccherà vicino agli arnioni.
\par 5 Il sacerdote farà fumare tutto questo sull'altare, come un sacrifizio fatto mediante il fuoco all'Eterno. Questo è un sacrifizio di riparazione.
\par 6 Ogni maschio tra i sacerdoti ne potrà mangiare; lo si mangerà in luogo santo; è cosa santissima.
\par 7 Il sacrifizio di riparazione è come il sacrifizio per il peccato; la stessa legge vale per ambedue; la vittima sarà del sacerdote che farà l'espiazione.
\par 8 E il sacerdote che offrirà l'olocausto per qualcuno avrà per sé la pelle dell'olocausto che avrà offerto.
\par 9 Così pure ogni oblazione cotta in forno, o preparata in padella, o sulla gratella, sarà del sacerdote che l'ha offerta.
\par 10 E ogni oblazione impastata con olio, o asciutta, sarà per tutti i figliuoli d'Aaronne: per l'uno come per l'altro.
\par 11 Questa è la legge del sacrifizio di azioni di grazie, che si offrirà all'Eterno.
\par 12 Se uno l'offre per riconoscenza, offrirà, col sacrifizio di azioni di grazie, delle focacce senza lievito intrise con olio, delle gallette senza lievito unte con olio, e del fior di farina cotto, in forma di focacce intrise con olio.
\par 13 Presenterà anche, per sua offerta, oltre quelle focacce, delle focacce di pan lievitato, insieme col suo sacrifizio di riconoscenza e di azioni di grazie.
\par 14 D'ognuna di queste offerte si presenterà una parte come oblazione elevata all'Eterno; essa sarà del sacerdote che avrà fatto l'aspersione del sangue del sacrifizio di azioni di grazie.
\par 15 E la carne del sacrifizio di riconoscenza e di azioni di grazie sarà mangiata il giorno stesso ch'esso è offerto; non se ne lascerà nulla fino alla mattina.
\par 16 Ma se il sacrifizio che uno offre è votivo o volontario, la vittima sarà mangiata il giorno ch'ei l'offrirà, e quel che ne rimane dovrà esser mangiato l'indomani;
\par 17 ma quel che sarà rimasto della carne del sacrifizio fino al terzo giorno, dovrà bruciarsi col fuoco.
\par 18 Che se uno mangia della carne del suo sacrifizio di azioni di grazie il terzo giorno, colui che l'ha offerto non sarà gradito; e dell'offerta non gli sarà tenuto conto; sarà cosa aborrita; e colui che ne avrà mangiato porterà la pena della sua iniquità.
\par 19 La carne che sarà stata in contatto di qualcosa d'impuro, non sarà mangiata; sarà bruciata col fuoco.
\par 20 Quanto alla carne che si mangia, chiunque è puro ne potrà mangiare; ma la persona che, essendo impura, mangerà della carne del sacrifizio di azioni di grazie che appartiene all'Eterno, sarà sterminata di fra il suo popolo.
\par 21 E se uno toccherà qualcosa d'impuro, una impurità umana, un animale impuro o qualsivoglia cosa abominevole, immonda, e mangerà della carne del sacrifizio di azioni di grazie che appartiene all'Eterno, quel tale sarà sterminato di fra il suo popolo'.
\par 22 L'Eterno parlò ancora a Mosè, dicendo:
\par 23 'Parla ai figliuoli d'Israele, e di' loro: Non mangerete alcun grasso, né di bue, né di pecora, né di capra.
\par 24 Il grasso di una bestia morta da sé, o il grasso d'una bestia sbranata potrà servire per qualunque altro uso; ma non ne mangerete affatto;
\par 25 perché chiunque mangerà del grasso degli animali che si offrono in sacrifizio mediante il fuoco all'Eterno, quel tale sarà sterminato di fra il suo popolo.
\par 26 E non mangerete affatto alcun sangue, né di uccelli né di quadrupedi, in tutti i luoghi dove abiterete.
\par 27 Chiunque mangerà sangue di qualunque specie, sarà sterminato di fra il suo popolo'.
\par 28 L'Eterno parlò ancora a Mosè, dicendo:
\par 29 'Parla ai figliuoli d'Israele, e di' loro: Colui che offrirà all'Eterno il suo sacrifizio di azioni di grazie porterà la sua offerta all'Eterno, prelevandola dal suo sacrifizio di azioni di grazie.
\par 30 Porterà con le proprie mani ciò che dev'essere offerto all'Eterno mediante il fuoco; porterà il grasso insieme col petto, il petto per agitarlo come offerta agitata davanti all'Eterno.
\par 31 Il sacerdote farà fumare il grasso sull'altare; e il petto sarà d'Aaronne e de' suoi figliuoli.
\par 32 Darete pure al sacerdote, come offerta elevata, la coscia destra dei vostri sacrifizi d'azioni di grazie.
\par 33 Colui de' figliuoli d'Aaronne che offrirà il sangue e il grasso dei sacrifizi di azioni di grazie avrà, come sua parte, la coscia destra.
\par 34 Poiché, dai sacrifizi di azioni di grazie offerti dai figliuoli d'Israele, io prendo il petto dell'offerta agitata e la coscia dell'offerta elevata, e li do al sacerdote Aaronne e ai suoi figliuoli per legge perpetua, da osservarsi dai figliuoli d'Israele.
\par 35 Questa è la parte consacrata ad Aaronne e consacrata ai suoi figliuoli, dei sacrifizi fatti mediante il fuoco all'Eterno, dal giorno in cui saranno presentati per esercitare il sacerdozio dell'Eterno.
\par 36 Questo l'Eterno ha ordinato ai figliuoli d'Israele di dar loro dal giorno della loro unzione. È una parte ch'è loro dovuta in perpetuo, di generazione in generazione'.
\par 37 Questa è la legge dell'olocausto, dell'oblazione, del sacrifizio per il peccato, del sacrifizio di riparazione, della consacrazione e del sacrifizio di azioni di grazie:
\par 38 legge che l'Eterno dette a Mosè sul monte Sinai il giorno che ordinò ai figliuoli d'Israele di presentare le loro offerte all'Eterno nel deserto di Sinai.

\chapter{8}

\par 1 L'Eterno parlò ancora a Mosè, dicendo:
\par 2 'Prendi Aaronne e i suoi figliuoli con lui, i paramenti, l'olio dell'unzione, il giovenco del sacrifizio per il peccato, i due montoni e il paniere dei pani azzimi;
\par 3 e convoca tutta la raunanza all'ingresso della tenda di convegno'.
\par 4 E Mosè fece come l'Eterno gli aveva ordinato, e la raunanza fu convocata all'ingresso della tenda di convegno.
\par 5 E Mosè disse alla raunanza: 'Questo è quello che l'Eterno ha ordinato di fare'.
\par 6 E Mosè fece accostare Aaronne e i suoi figliuoli, e li lavò con acqua.
\par 7 Poi rivestì Aaronne della tunica, lo cinse della cintura, gli pose addosso il manto, gli mise l'efod, e lo cinse della cintura artistica dell'efod, con la quale gli fissò l'efod addosso.
\par 8 Gli mise pure il pettorale, e sul pettorale pose l'Urim e il Thummim.
\par 9 Poi gli mise in capo la mitra, e sul davanti della mitra pose la lamina d'oro, il santo diadema, come l'Eterno aveva ordinato a Mosè.
\par 10 Poi Mosè prese l'olio dell'unzione, unse il tabernacolo e tutte le cose che vi si trovavano, e le consacrò.
\par 11 Ne fece sette volte l'aspersione sull'altare, unse l'altare e tutti i suoi utensili, e la conca e la sua base, per consacrarli.
\par 12 E versò dell'olio dell'unzione sul capo d'Aaronne, e unse Aaronne, per consacrarlo.
\par 13 Poi Mosè fece accostare i figliuoli d'Aaronne, li vestì di tuniche, li cinse di cinture, e assicurò sul loro capo delle tiare, come l'Eterno aveva ordinato a Mosè.
\par 14 Fece quindi accostare il giovenco del sacrifizio per il peccato, e Aaronne e i suoi figliuoli posarono le loro mani sulla testa del giovenco del sacrifizio per il peccato.
\par 15 Mosè lo scannò, ne prese del sangue, lo mise col dito sui corni dell'altare tutto all'intorno e purificò l'altare; poi sparse il resto del sangue appiè dell'altare, e lo consacrò per farvi su l'espiazione.
\par 16 Poi prese tutto il grasso ch'era sulle interiora, la rete del fegato, i due arnioni col loro grasso, e Mosè fece fumar tutto sull'altare.
\par 17 Ma il giovenco, la sua pelle, la sua carne e i suoi escrementi, li bruciò col fuoco fuori del campo, come l'Eterno aveva ordinato a Mosè.
\par 18 Fece quindi accostare il montone dell'olocausto, e Aaronne e i suoi figliuoli posarono le mani sulla testa del montone.
\par 19 E Mosè lo scannò, e ne sparse il sangue sull'altare tutto all'intorno.
\par 20 Poi fece a pezzi il montone, e Mosè fece fumare la testa, i pezzi e il grasso.
\par 21 E quando n'ebbe lavato le interiora e le gambe con acqua, Mosè fece fumare tutto il montone sull'altare. Fu un olocausto di soave odore, un sacrifizio fatto mediante il fuoco all'Eterno, come l'Eterno aveva ordinato a Mosè.
\par 22 Poi fece accostare il secondo montone, il montone della consacrazione; e Aaronne e i suoi figliuoli posarono le mani sulla testa del montone.
\par 23 E Mosè lo scannò, e ne prese del sangue e lo mise sull'estremità dell'orecchio destro d'Aaronne e sul pollice della sua man destra e sul dito grosso del suo piede destro.
\par 24 Poi Mosè fece accostare i figliuoli d'Aaronne, e pose di quel sangue sull'estremità del loro orecchio destro, sul pollice della loro man destra e sul dito grosso del loro piè destro; e sparse il resto del sangue sull'altare tutto all'intorno.
\par 25 Poi prese il grasso, la coda, tutto il grasso che copriva le interiora, la rete del fegato, i due arnioni, il loro grasso, e la coscia destra;
\par 26 e dal paniere dei pani azzimi, ch'era davanti all'Eterno, prese una focaccia senza lievito, una focaccia di pasta oliata e una galletta, e le pose sui grassi e sulla coscia destra.
\par 27 Poi mise tutte queste cose sulle palme delle mani d'Aaronne e sulle palme delle mani dei suoi figliuoli, e le agitò come offerta agitata davanti all'Eterno.
\par 28 Mosè quindi le prese dalle loro mani, e le fece fumare sull'altare sopra l'olocausto. Fu un sacrifizio di consacrazione, di soave odore: un sacrifizio fatto mediante il fuoco all'Eterno.
\par 29 Poi Mosè prese il petto del montone e lo agitò come offerta agitata davanti all'Eterno; questa fu la parte del montone della consacrazione che toccò a Mosè, come l'Eterno aveva ordinato a Mosè.
\par 30 Mosè prese quindi dell'olio dell'unzione e del sangue ch'era sopra l'altare, e ne asperse Aaronne e i suoi paramenti, i figliuoli di lui e i loro paramenti; e consacrò Aaronne e i suoi paramenti, i figliuoli di lui e i loro paramenti con lui.
\par 31 Poi Mosè disse ad Aaronne e ai suoi figliuoli: 'Fate cuocere la carne all'ingresso della tenda di convegno; e quivi la mangerete col pane che è nel paniere della consacrazione, come ho ordinato, dicendo: Aaronne e i suoi figliuoli la mangeranno.
\par 32 E quel che rimane della carne e del pane lo brucerete col fuoco.
\par 33 E per sette giorni non vi dipartirete dall'ingresso della tenda di convegno, finché non siano compiuti i giorni delle vostre consacrazioni; poiché la vostra consacrazione durerà sette giorni.
\par 34 Come s'è fatto oggi, così l'Eterno ha ordinato che si faccia, per fare espiazione per voi.
\par 35 Rimarrete dunque sette giorni all'ingresso della tenda di convegno, giorno e notte, e osserverete il comandamento dell'Eterno, affinché non muoiate; poiché così m'è stato ordinato'.
\par 36 E Aaronne e i suoi figliuoli fecero tutte le cose che l'Eterno aveva ordinate per mezzo di Mosè.

\chapter{9}

\par 1 L'ottavo giorno, Mosè chiamò Aaronne, i suoi figliuoli e gli anziani d'Israele,
\par 2 e disse ad Aaronne: 'Prendi un giovine vitello per un sacrifizio per il peccato, e un montone per un olocausto: ambedue senza difetto, e offrili all'Eterno.
\par 3 E dirai così ai figliuoli d'Israele: Prendete un capro per un sacrifizio per il peccato, e un vitello e un agnello, ambedue d'un anno, senza difetto, per un olocausto;
\par 4 e un bue e un montone per un sacrifizio di azioni di grazie, per sacrificarli davanti all'Eterno; e un'oblazione intrisa con olio; perché oggi l'Eterno vi apparirà'.
\par 5 Essi dunque menarono davanti alla tenda di convegno le cose che Mosè aveva ordinate; e tutta la raunanza si accostò, e si tenne in piè davanti all'Eterno.
\par 6 E Mosè disse: 'Questo è quello che l'Eterno vi ha ordinato; fatelo, e la gloria dell'Eterno vi apparirà'.
\par 7 E Mosè disse ad Aaronne: 'Accostati all'altare; offri il tuo sacrifizio per il peccato e il tuo olocausto, e fa' l'espiazione per te e per il popolo; presenta anche l'offerta del popolo e fa' l'espiazione per esso, come l'Eterno ha ordinato'.
\par 8 Aaronne dunque s'accostò all'altare e scannò il vitello del sacrifizio per il peccato, ch'era per sé.
\par 9 E i suoi figliuoli gli porsero il sangue, ed egli intinse il dito nel sangue, ne mise sui corni dell'altare, e sparse il resto del sangue appiè dell'altare;
\par 10 ma il grasso, gli arnioni e la rete del fegato della vittima per il peccato, li fece fumare sull'altare, come l'Eterno aveva ordinato a Mosè.
\par 11 E la carne e la pelle, le bruciò col fuoco fuori del campo.
\par 12 Poi scannò l'olocausto; e i figliuoli d'Aaronne gli porsero il sangue ed egli lo sparse sull'altare tutto all'intorno.
\par 13 Gli porsero pure l'olocausto fatto a pezzi, e la testa; ed egli li fece fumare sull'altare.
\par 14 E lavò le interiora e le gambe, e le fece fumare sull'olocausto, sopra l'altare.
\par 15 Poi presentò l'offerta del popolo. Prese il capro destinato al sacrifizio per il peccato del popolo, lo scannò e l'offrì per il peccato, come la prima volta.
\par 16 Poi offrì l'olocausto, e lo fece secondo la regola stabilita.
\par 17 Presentò quindi l'oblazione; ne prese una manata piena, e la fece fumare sull'altare, oltre l'olocausto della mattina.
\par 18 E scannò il bue e il montone, come sacrifizio di azioni di grazie per il popolo. I figliuoli d'Aaronne gli porsero il sangue, ed egli lo sparse sull'altare, tutto all'intorno.
\par 19 Gli porsero i grassi del bue, del montone, la coda, il grasso che copre le interiora, gli arnioni e la rete del fegato;
\par 20 misero i grassi sui petti, ed egli fece fumare i grassi sull'altare;
\par 21 e i petti e la coscia destra, Aaronne li agitò davanti all'Eterno come offerta agitata, nel modo che Mosè aveva ordinato.
\par 22 Poi Aaronne alzò le mani verso il popolo, e lo benedisse; e, dopo aver fatto il sacrifizio per il peccato, l'olocausto e i sacrifizi di azioni di grazie, scese giù dall'altare.
\par 23 E Mosè ed Aaronne entrarono nella tenda di convegno; poi uscirono e benedissero il popolo; e la gloria dell'Eterno apparve a tutto il popolo.
\par 24 Un fuoco uscì dalla presenza dell'Eterno e consumò sull'altare l'olocausto e i grassi; e tutto il popolo lo vide, diè in grida d'esultanza, e si prostrò colla faccia a terra.

\chapter{10}

\par 1 Or Nabad ed Abihu, figliuoli d'Aaronne, presero ciascuno il suo turibolo, vi misero dentro del fuoco, vi posero su del profumo, e offrirono davanti all'Eterno del fuoco estraneo: il che egli non aveva loro ordinato.
\par 2 E un fuoco uscì dalla presenza dell'Eterno, e li divorò; e morirono davanti all'Eterno.
\par 3 Allora Mosè disse ad Aaronne: 'Questo è quello di cui l'Eterno ha parlato, quando ha detto: Io sarò santificato per mezzo di quelli che mi stanno vicino, e sarò glorificato in presenza di tutto il popolo'. E Aaronne si tacque.
\par 4 E Mosè chiamò Mishael ed Eltsafan, figliuoli di Uziel, zio d'Aaronne, e disse loro: 'Accostatevi, portate via i vostri fratelli di davanti al santuario, fuori del campo'.
\par 5 Ed essi si accostarono, e li portaron via nelle loro tuniche, fuori del campo, come Mosè avea detto.
\par 6 E Mosè disse ad Aaronne, ad Eleazar e ad Ithamar, suoi figliuoli: 'Non andate a capo scoperto, e non vi stracciate le vesti, affinché non muoiate, e l'Eterno non s'adiri contro tutta la raunanza; ma i vostri fratelli, tutta quanta la casa d'Israele, menino duolo, a motivo dell'arsione che l'Eterno ha fatto.
\par 7 E non vi dipartite dall'ingresso della tenda di convegno, onde non abbiate a perire; perché l'olio dell'unzione dell'Eterno è su voi'. Ed essi fecero come Mosè avea detto.
\par 8 L'Eterno parlò ad Aaronne, dicendo:
\par 9 'Non bevete vino né bevande alcooliche tu e i tuoi figliuoli quando entrerete nella tenda di convegno, affinché non muoiate; sarà una legge perpetua, di generazione in generazione;
\par 10 e questo, perché possiate discernere ciò ch'è santo da ciò che è profano e ciò che è impuro da ciò ch'è puro,
\par 11 e possiate insegnare ai figliuoli d'Israele tutte le leggi che l'Eterno ha dato loro per mezzo di Mosè'.
\par 12 Poi Mosè disse ad Aaronne, ad Eleazar e ad Ithamar, i due figliuoli che restavano ad Aaronne: 'Prendete quel che rimane dell'oblazione dei sacrifizi fatti mediante il fuoco all'Eterno, e mangiatelo senza lievito, presso l'altare; perché è cosa santissima.
\par 13 Lo mangerete in luogo santo, perché è la parte che spetta a te e ai tuoi figliuoli, de' sacrifizi fatti mediante il fuoco all'Eterno; poiché così mi è stato ordinato.
\par 14 E il petto dell'offerta agitata e la coscia dell'offerta elevata li mangerete tu, i tuoi figliuoli e le tue figliuole con te, in luogo puro; perché vi sono stati dati come parte spettante a te ed ai tuoi figliuoli, dei sacrifizi di azioni di grazie de' figliuoli d'Israele.
\par 15 Oltre ai grassi da ardere si porteranno la coscia dell'offerta elevata e il petto dell'offerta agitata, per esser agitati davanti all'Eterno come offerta agitata; anche questo apparterrà a te e ai tuoi figliuoli con te, per diritto perpetuo, come l'Eterno ha ordinato'.
\par 16 Or Mosè cercò e ricercò il capro del sacrifizio per il peccato; ed ecco, era stato bruciato; ond'egli s'adirò gravemente contro Eleazar e contro Ithamar, i figliuoli ch'eran rimasti ad Aaronne, dicendo:
\par 17 'Perché non avete mangiato il sacrifizio per il peccato nel luogo santo? giacché è cosa santissima, e l'Eterno ve l'ha dato perché portiate l'iniquità della raunanza, perché ne facciate l'espiazione davanti all'Eterno.
\par 18 Ecco, il sangue della vittima non è stato portato dentro il santuario; voi avreste dovuto mangiarla nel santuario, come io avevo ordinato'.
\par 19 Ed Aaronne disse a Mosè: 'Ecco, oggi essi hanno offerto il loro sacrifizio per il peccato e il loro olocausto davanti all'Eterno; e, dopo le cose che mi son successe, se oggi avessi mangiato la vittima del sacrifizio per il peccato, sarebbe ciò piaciuto all'Eterno?'
\par 20 Quando Mosè udì questo, rimase soddisfatto.

\chapter{11}

\par 1 Poi l'Eterno parlò a Mosè e ad Aaronne, dicendo loro:
\par 2 'Parlate così ai figliuoli d'Israele: Questi sono gli animali che potrete mangiare fra tutte le bestie che sono sulla terra.
\par 3 Mangerete d'ogni animale che ha l'unghia spartita e ha il piè forcuto, e che rumina.
\par 4 Ma di fra quelli che ruminano e di fra quelli che hanno l'unghia spartita, non mangerete questi: il cammello, perché rumina, ma non ha l'unghia spartita; lo considererete come impuro;
\par 5 il coniglio, perché rumina, ma non ha l'unghia spartita; lo considererete come impuro;
\par 6 la lepre, perché rumina, ma non ha l'unghia spartita; la considererete come impura;
\par 7 il porco, perché ha l'unghia spartita e il piè forcuto, ma non rumina; lo considererete come impuro.
\par 8 Non mangerete della loro carne e non toccherete i loro corpi morti; li considererete come impuri.
\par 9 Questi sono gli animali che potrete mangiare fra tutti quelli che sono nell'acqua. Mangerete tutto ciò che ha pinne e scaglie nelle acque, tanto ne' mari quanto ne' fiumi.
\par 10 Ma tutto ciò che non ha né pinne né scaglie, tanto ne' mari quanto ne' fiumi, fra tutto ciò che si muove nelle acque e tutto ciò che vive nelle acque, l'avrete in abominio.
\par 11 Essi vi saranno in abominio; non mangerete della loro carne, e avrete in abominio i loro corpi morti.
\par 12 Tutto ciò che non ha né pinne né scaglie nelle acque vi sarà in abominio.
\par 13 E fra gli uccelli avrete in abominio questi: non se ne mangi; sono un abominio: l'aquila, l'ossifraga e l'aquila di mare;
\par 14 il nibbio e ogni specie di falco;
\par 15 ogni specie di corvo;
\par 16 lo struzzo, il barbagianni, il gabbiano e ogni specie di sparviere;
\par 17 il gufo, lo smergo, l'ibi;
\par 18 il cigno, il pellicano, l'avvoltoio;
\par 19 la cicogna, ogni specie di airone, l'upupa e il pipistrello.
\par 20 Vi sarà pure in abominio ogni insetto alato che cammina su quattro piedi.
\par 21 Però, fra tutti gl'insetti alati che camminano su quattro piedi, mangerete quelli che hanno gambe al disopra de' piedi per saltare sulla terra.
\par 22 Di questi potrete mangiare: ogni specie di cavalletta, ogni specie di solam, ogni specie di hargol e ogni specie di hagab.
\par 23 Ogni altro insetto alato che ha quattro piedi vi sarà in abominio.
\par 24 Questi animali vi renderanno impuri; chiunque toccherà il loro corpo morto sarà impuro fino alla sera.
\par 25 E chiunque porterà i loro corpi morti si laverà le vesti, e sarà impuro fino alla sera.
\par 26 Considererete come impuro ogni animale che ha l'unghia spartita, ma non ha il piè forcuto, e che non rumina; chiunque lo toccherà sarà impuro.
\par 27 Considererete come impuri tutti i quadrupedi che camminano sulla pianta de' piedi; chiunque toccherà il loro corpo morto sarà impuro fino alla sera.
\par 28 E chiunque porterà i loro corpi si laverà le vesti, e sarà immondo fino alla sera. Questi animali considererete come impuri.
\par 29 E fra i piccoli animali che strisciano sulla terra, considererete come impuri questi: la talpa, il topo e ogni specie di lucertola, il toporagno,
\par 30 la rana, la tartaruga, la lumaca, il camaleonte.
\par 31 Questi animali, fra tutto ciò che striscia, saranno impuri per voi; chiunque li toccherà morti, sarà impuro fino alla sera.
\par 32 Ogni oggetto sul quale cadrà qualcun d'essi quando sarà morto, sarà immondo: siano utensili di legno, o veste, o pelle, o sacco, o qualunque altro oggetto di cui si faccia uso; sarà messo nell'acqua, e sarà impuro fino alla sera; poi sarà puro.
\par 33 E se ne cade qualcuno in un vaso di terra, tutto quello che vi si troverà dentro sarà impuro, e spezzerete il vaso.
\par 34 Ogni cibo che serve al nutrimento, sul quale sarà caduta di quell'acqua, sarà impuro; e ogni bevanda di cui si fa uso, qualunque sia il vaso che la contiene, sarà impura.
\par 35 Ogni oggetto sul quale cadrà qualcosa del loro corpo morto, sarà impuro; il forno o il fornello sarà spezzato; sono impuri e li considererete come impuri.
\par 36 Però, una fonte o una cisterna, dov'è una raccolta d'acqua, sarà pura; ma chi toccherà i loro corpi morti sarà impuro.
\par 37 E se qualcosa de' loro corpi morti cade su qualche seme che dev'esser seminato, questo sarà puro;
\par 38 ma se è stata messa dell'acqua sul seme, e vi cade su qualcosa de' loro corpi morti, lo considererai come impuro.
\par 39 Se muore un animale di quelli che vi servono per nutrimento, colui che ne toccherà il corpo morto sarà impuro fino alla sera.
\par 40 Colui che mangerà di quel corpo morto si laverà le vesti, e sarà impuro fino alla sera; parimente colui che porterà quel corpo morto si laverà le vesti, e sarà impuro fino alla sera.
\par 41 Ogni cosa che brulica sulla terra è un abominio; non se ne mangerà.
\par 42 Di tutti gli animali che brulicano sulla terra non ne mangerete alcuno che strisci sul ventre o cammini con quattro piedi o con molti piedi, poiché sono un abominio.
\par 43 Non rendete le vostre persone abominevoli mediante alcuno di questi animali che strisciano; e non vi rendete impuri per loro mezzo, in guisa da rimaner così contaminati.
\par 44 Poiché io sono l'Eterno, l'Iddio vostro; santificatevi dunque e siate santi, perché io son santo; e non contaminate le vostre persone mediante alcuno di questi animali che strisciano sulla terra.
\par 45 Poiché io sono l'Eterno che vi ho fatti salire dal paese d'Egitto, per essere il vostro Dio; siate dunque santi, perché io son santo.
\par 46 Questa è la legge concernente i quadrupedi, gli uccelli, ogni essere vivente che si muove nelle acque e ogni essere che striscia sulla terra,
\par 47 affinché sappiate discernere ciò ch'è impuro da ciò ch'è puro, l'animale che si può mangiare da quello che non si deve mangiare'.

\chapter{12}

\par 1 L'Eterno parlò ancora a Mosè, dicendo: 'Parla così ai figliuoli d'Israele:
\par 2 Quando una donna sarà rimasta incinta e partorirà un maschio, sarà impura sette giorni; sarà impura come nel tempo de' suoi corsi mensuali.
\par 3 L'ottavo giorno si circonciderà la carne del prepuzio del bambino.
\par 4 Poi, ella resterà ancora trentatre giorni a purificarsi del suo sangue; non toccherà alcuna cosa santa, e non entrerà nel santuario finché non siano compiuti i giorni della sua purificazione.
\par 5 Ma, se partorisce una bambina, sarà impura due settimane come al tempo de' suoi corsi mensuali; e resterà sessantasei giorni a purificarsi del suo sangue.
\par 6 E quando i giorni della sua purificazione, per un figliuolo o per una figliuola, saranno compiuti, porterà al sacerdote, all'ingresso della tenda di convegno, un agnello d'un anno come olocausto, e un giovane piccione o una tortora come sacrifizio per il peccato;
\par 7 e il sacerdote li offrirà davanti all'Eterno e farà l'espiazione per lei; ed ella sarà purificata del flusso del suo sangue. Questa è la legge relativa alla donna che partorisce un maschio o una femmina.
\par 8 E se non ha mezzi da offrire un agnello, prenderà due tortore o due giovani piccioni: uno per l'olocausto, e l'altro per il sacrifizio per il peccato. Il sacerdote farà l'espiazione per lei, ed ella sarà pura'.

\chapter{13}

\par 1 L'Eterno parlò ancora a Mosè e ad Aaronne, dicendo:
\par 2 'Quand'uno avrà sulla pelle del suo corpo un tumore o una pustola o una macchia lucida che sia sintomo di piaga di lebbra sulla pelle del suo corpo, quel tale sarà menato al sacerdote Aaronne o ad uno de' suoi figliuoli sacerdoti.
\par 3 Il sacerdote esaminerà la piaga sulla pelle del corpo; e se il pelo della piaga è diventato bianco, e la piaga appare più profonda della pelle del corpo, è piaga di lebbra; e il sacerdote che l'avrà esaminato, dichiarerà quell'uomo impuro.
\par 4 Ma se la macchia lucida sulla pelle del corpo è bianca, e non appare esser più profonda della pelle, e il suo pelo non è diventato bianco, il sacerdote rinchiuderà per sette giorni colui che ha la piaga.
\par 5 Il sacerdote, il settimo giorno, l'esaminerà; e se gli parrà che la piaga si sia fermata e non si sia allargata sulla pelle, il sacerdote lo rinchiuderà altri sette giorni.
\par 6 Il sacerdote, il settimo giorno, lo esaminerà di nuovo; e se vedrà che la piaga non è più lucida e non s'è allargata sulla pelle, il sacerdote dichiarerà quell'uomo puro: è una pustola. Quel tale laverà le sue vesti, e sarà puro.
\par 7 Ma se la pustola s'è allargata sulla pelle dopo ch'egli s'è mostrato al sacerdote per esser dichiarato puro, si farà esaminare per la seconda volta dal sacerdote;
\par 8 il sacerdote l'esaminerà; e se vedrà che la pustola si è allargata sulla pelle, il sacerdote lo dichiarerà impuro; è lebbra.
\par 9 Quand'uno avrà addosso una piaga di lebbra, sarà menato al sacerdote.
\par 10 Il sacerdote lo esaminerà; e se vedrà che sulla pelle c'è un tumor bianco, che questo tumore ha fatto imbiancare il pelo e che v'è nel tumore della carne viva,
\par 11 è lebbra inveterata nella pelle del corpo di colui, e il sacerdote lo dichiarerà impuro; non lo rinchiuderà, perché è impuro.
\par 12 E se la lebbra produce delle efflorescenze sulla pelle in modo da coprire tutta la pelle di colui che ha la piaga, dal capo ai piedi, dovunque il sacerdote guardi,
\par 13 il sacerdote lo esaminerà; e quando avrà veduto che la lebbra copre tutto il corpo, dichiarerà puro colui che ha la piaga. Egli è divenuto tutto quanto bianco, quindi è puro.
\par 14 Ma dal momento che apparirà in lui della carne viva, sarà dichiarato impuro.
\par 15 Quando il sacerdote avrà visto la carne viva, dichiarerà quell'uomo impuro; la carne viva è impura; è lebbra.
\par 16 Ma se la carne viva ridiventa bianca, vada colui al sacerdote, e il sacerdote lo esaminerà;
\par 17 e se vedrà che la piaga è ridiventata bianca, il sacerdote dichiarerà puro colui che ha la piaga: è puro.
\par 18 Quand'uno avrà avuto sulla pelle della carne un'ulcera che sia guarita,
\par 19 e poi, sul luogo dell'ulcera apparirà un tumor bianco o una macchia lucida, bianca, tendente al rosso, quel tale si mostrerà al sacerdote.
\par 20 Il sacerdote l'esaminerà; e se vedrà che la macchia apparisce più profonda della pelle e che il pelo n'è diventato bianco, il sacerdote lo dichiarerà impuro; è piaga di lebbra che è scoppiata nell'ulcera.
\par 21 Ma se il sacerdote, esaminandola, vede che nella macchia non ci sono peli bianchi e che non è più profonda della pelle e non è più lucida, il sacerdote lo rinchiuderà sette giorni.
\par 22 E se la macchia s'allarga sulla pelle, il sacerdote lo dichiarerà impuro; è piaga di lebbra.
\par 23 Ma se la macchia è rimasta allo stesso punto e non si è allargata, è la cicatrice dell'ulcera, e il sacerdote lo dichiarerà puro.
\par 24 Quand'uno avrà sulla pelle del suo corpo una bruciatura cagionata dal fuoco, e su questa bruciatura apparirà una macchia lucida, bianca, tendente al rosso o soltanto bianca,
\par 25 il sacerdote l'esaminerà; e se vedrà che il pelo della macchia è diventato bianco e la macchia appare più profonda della pelle, è lebbra scoppiata nella bruciatura. Il sacerdote dichiarerà quel tale impuro; è piaga di lebbra.
\par 26 Ma se il sacerdote, esaminandola, vede che non c'è pelo bianco nella macchia, e ch'essa non è più profonda della pelle e non è più lucida, il sacerdote lo rinchiuderà sette giorni.
\par 27 Il sacerdote, il settimo giorno, l'esaminerà; e se la macchia s'è allargata sulla pelle, il sacerdote dichiarerà quel tale impuro; è piaga di lebbra.
\par 28 E se la macchia è rimasta ferma nello stesso luogo, e non si è allargata sulla pelle, e non è più lucida, è il tumore della bruciatura; il sacerdote dichiarerà quel tale puro, perch'è la cicatrice della bruciatura.
\par 29 Quand'un uomo o una donna avrà una piaga sul capo o nella barba,
\par 30 il sacerdote esaminerà la piaga; e se vedrà ch'essa appare più profonda della pelle, e che v'è del pelo gialliccio e sottile, il sacerdote li dichiarerà impuri; è tigna, è lebbra del capo o della barba.
\par 31 E se il sacerdote, esaminando la piaga della tigna, vedrà che non appare più profonda della pelle e che non v'è pelo nero, il sacerdote rinchiuderà sette giorni colui che ha la piaga della tigna.
\par 32 E se il sacerdote, esaminando il settimo giorno la piaga, vedrà che la tigna non s'è allargata, e che non v'è pelo giallo, e che la tigna non appare più profonda della pelle,
\par 33 quel tale si raderà, ma non raderà il luogo dov'è la tigna; e il sacerdote rinchiuderà altri sette giorni colui che ha la tigna.
\par 34 Il sacerdote, il settimo giorno, esaminerà la tigna; e se vedrà che la tigna non s'è allargata sulla pelle e non appare più profonda della pelle, il sacerdote dichiarerà quel tale puro; colui si laverà le vesti, e sarà puro.
\par 35 Ma se, dopo ch'egli è stato dichiarato puro, la tigna s'è allargata sulla pelle,
\par 36 il sacerdote l'esaminerà; e se vedrà che la tigna s'è allargata sulla pelle, il sacerdote non cercherà se v'è del pelo giallo; quel tale è impuro.
\par 37 Ma se vedrà che la tigna s'è fermata e che v'è cresciuto del pelo nero, la tigna è guarita; quel tale è puro, e il sacerdote lo dichiarerà puro.
\par 38 Quand'un uomo o una donna avrà sulla pelle del suo corpo delle macchie lucide, delle macchie bianche,
\par 39 il sacerdote l'esaminerà; e se vedrà che le macchie sulla pelle del loro corpo sono di un bianco pallido, è una eruzione cutanea; quel tale è puro.
\par 40 Colui al quale son cascati i capelli del capo è calvo, ma è puro.
\par 41 Se i capelli gli son cascati dalla parte della faccia, è calvo di fronte, ma è puro.
\par 42 Ma se sulla parte calva del di dietro o del davanti del capo appare una piaga bianca tendente al rosso, è lebbra, scoppiata nella parte calva del di dietro o del davanti del capo.
\par 43 Il sacerdote lo esaminerà; e se vedrà che il tumore della piaga nella parte calva del di dietro o del davanti del capo è bianco tendente al rosso, simile alla lebbra della pelle del corpo,
\par 44 quel tale è un lebbroso; è impuro, e il sacerdote lo dovrà dichiarare impuro; egli ha la sua piaga sul capo.
\par 45 Il lebbroso, affetto da questa piaga, porterà le vesti strappate e il capo scoperto; si coprirà la barba, e andrà gridando: Impuro! impuro!
\par 46 Sarà impuro tutto il tempo che avrà la piaga; è impuro; se ne starà solo; abiterà fuori del campo.
\par 47 Quando apparirà una piaga di lebbra sopra una veste, sia veste di lana o veste di lino,
\par 48 un tessuto o un lavoro a maglia, di lino o di lana, un oggetto di pelle o qualunque altra cosa fatta di pelle,
\par 49 se la piaga sarà verdastra o rossastra sulla veste o sulla pelle, sul tessuto, o sulla maglia, o su qualunque cosa fatta di pelle, è piaga di lebbra, e sarà mostrata al sacerdote.
\par 50 Il sacerdote esaminerà la piaga, e rinchiuderà sette giorni colui che ha la piaga.
\par 51 Il settimo giorno esaminerà la piaga; e se la piaga si sarà allargata sulla veste o sul tessuto o sulla maglia o sulla pelle o sull'oggetto fatto di pelle per un uso qualunque, è una piaga di lebbra maligna; è cosa impura.
\par 52 Egli brucerà quella veste o il tessuto o la maglia di lana o di lino o qualunque oggetto fatto di pelle, sul quale è la piaga; perché è lebbra maligna; saran bruciati col fuoco.
\par 53 E se il sacerdote, esaminandola, vedrà che la piaga non s'è allargata sulla veste o sul tessuto o sulla maglia o sull'oggetto qualunque di pelle,
\par 54 il sacerdote ordinerà che si lavi l'oggetto su cui è la piaga, e lo rinchiuderà altri sette giorni.
\par 55 Il sacerdote esaminerà la piaga, dopo che sarà stata lavata; e se vedrà che la piaga non ha mutato colore, benché non si sia allargata, è un oggetto immondo; lo brucerai col fuoco; v'è corrosione, sia che la parte corrosa si trovi sul diritto o sul rovescio dell'oggetto.
\par 56 E se il sacerdote, esaminandola, vede che la piaga, dopo essere stata lavata, è diventata pallida, la strapperà dalla veste o dalla pelle o dal tessuto o dalla maglia.
\par 57 E se apparisce ancora sulla veste o sul tessuto o sulla maglia o sull'oggetto qualunque fatto di pelle, è una eruzione lebbrosa; brucerai col fuoco l'oggetto su cui è la piaga.
\par 58 La veste o il tessuto o la maglia o qualunque oggetto fatto di pelle che avrai lavato e dal quale la piaga sarà scomparsa, si laverà una seconda volta, e sarà puro.
\par 59 Questa è la legge relativa alla piaga di lebbra sopra una veste di lana o di lino, sul tessuto o sulla maglia o su qualunque oggetto fatto di pelle, per dichiararli puri o impuri'.

\chapter{14}

\par 1 L'Eterno parlò ancora a Mosè, dicendo:
\par 2 'Questa è la legge relativa al lebbroso per il giorno della sua purificazione. Egli sarà menato al sacerdote.
\par 3 Il sacerdote uscirà dal campo, e l'esaminerà; e se vedrà che la piaga della lebbra è guarita nel lebbroso,
\par 4 il sacerdote ordinerà che si prendano, per colui che dev'esser purificato, due uccelli vivi, puri, del legno di cedro, dello scarlatto e dell'issopo.
\par 5 Il sacerdote ordinerà che si sgozzi uno degli uccelli in un vaso di terra su dell'acqua viva.
\par 6 Poi prenderà l'uccello vivo, il legno di cedro, lo scarlatto e l'issopo, e l'immergerà, con l'uccello vivo, nel sangue dell'uccello sgozzato sopra l'acqua viva.
\par 7 Ne aspergerà sette volte colui che dev'esser purificato dalla lebbra; lo dichiarerà puro, e lascerà andar libero per i campi l'uccello vivo.
\par 8 Colui che si purifica si laverà le vesti, si raderà tutti i peli, si laverà nell'acqua, e sarà puro. Dopo questo potrà entrar nel campo, ma resterà sette giorni fuori della sua tenda.
\par 9 Il settimo giorno si raderà tutti i peli, il capo, la barba, le ciglia: si raderà insomma tutti i peli, si laverà le vesti e si laverà il corpo nell'acqua, e sarà puro.
\par 10 L'ottavo giorno prenderà due agnelli senza difetto, un'agnella d'un anno senza difetto, tre decimi d'un efa di fior di farina, una oblazione, intrisa con olio, e un log d'olio;
\par 11 e il sacerdote che fa la purificazione, presenterà colui che si purifica e quelle cose davanti all'Eterno, all'ingresso della tenda di convegno.
\par 12 Poi il sacerdote prenderà uno degli agnelli e l'offrirà come sacrifizio di riparazione, con il log d'olio, e li agiterà come offerta agitata davanti all'Eterno.
\par 13 Poi scannerà l'agnello nel luogo dove si scannano i sacrifizi per il peccato e gli olocausti: vale a dire, nel luogo sacro; poiché il sacrifizio di riparazione appartiene al sacerdote, come quello per il peccato; è cosa santissima.
\par 14 E il sacerdote prenderà del sangue del sacrifizio di riparazione e lo metterà sull'estremità dell'orecchio destro di colui che si purifica, sul pollice della sua man destra e sul dito grosso del suo piede destro.
\par 15 Poi il sacerdote prenderà dell'olio del log, e lo verserà nella palma della sua mano sinistra;
\par 16 quindi il sacerdote intingerà il dito della sua destra nell'olio che avrà nella sinistra, e col dito farà sette volte aspersione di quell'olio davanti all'Eterno.
\par 17 E del rimanente dell'olio che avrà in mano, il sacerdote ne metterà sull'estremità dell'orecchio destro di colui che si purifica, sul pollice della sua man destra e sul dito grosso del suo piede destro, oltre al sangue del sacrifizio di riparazione.
\par 18 Il resto dell'olio che avrà in mano, il sacerdote lo metterà sul capo di colui che si purifica; e il sacerdote farà per lui l'espiazione davanti all'Eterno.
\par 19 Poi il sacerdote offrirà il sacrifizio per il peccato, e farà l'espiazione per colui che si purifica della sua impurità; quindi, scannerà l'olocausto.
\par 20 Il sacerdote offrirà l'olocausto e l'oblazione sull'altare; farà per quel tale l'espiazione, ed egli sarà puro.
\par 21 Se colui è povero e non può procurarsi quel tanto, prenderà un solo agnello da offrire in sacrifizio di riparazione come offerta agitata, per fare l'espiazione per lui, e un solo decimo d'un efa di fior di farina intrisa con olio, come oblazione, e un log d'olio.
\par 22 Prenderà pure due tortore o due giovani piccioni, secondo i suoi mezzi; uno sarà per il sacrifizio per il peccato, e l'altro per l'olocausto.
\par 23 L'ottavo giorno porterà, per la sua purificazione, queste cose al sacerdote, all'ingresso della tenda di convegno, davanti all'Eterno.
\par 24 E il sacerdote prenderà l'agnello del sacrifizio di riparazione e il log d'olio, e li agiterà come offerta agitata davanti all'Eterno.
\par 25 Poi scannerà l'agnello del sacrifizio di riparazione. Il sacerdote prenderà del sangue del sacrifizio di riparazione, e lo metterà sull'estremità dell'orecchio destro di colui che si purifica, e sul pollice della sua man destra e sul dito grosso del suo piede destro.
\par 26 Il sacerdote verserà di quell'olio sulla palma della sua mano sinistra.
\par 27 E col dito della sua man destra il sacerdote farà aspersione dell'olio che avrà nella mano sinistra, sette volte davanti all'Eterno.
\par 28 Poi il sacerdote metterà dell'olio che avrà in mano, sull'estremità dell'orecchio destro di colui che si purifica, sul pollice della sua man destra e sul dito grosso del suo piede destro, nel luogo dove ha messo del sangue del sacrifizio di riparazione.
\par 29 Il resto dell'olio che avrà in mano, il sacerdote lo metterà sul capo di colui che si purifica, per fare espiazione per lui davanti all'Eterno.
\par 30 Poi sacrificherà una delle tortore o uno dei due giovani piccioni, secondo che ha potuto procurarsi;
\par 31 delle vittime che ha potuto procurarsi, una offrirà come sacrifizio per il peccato, e l'altra come olocausto, insieme con l'oblazione; e il sacerdote farà l'espiazione davanti all'Eterno per colui che si purifica.
\par 32 Questa è la legge relativa a colui ch'è affetto da piaga di lebbra, e non ha mezzi da procurarsi ciò ch'è richiesto per la sua purificazione'.
\par 33 L'Eterno parlò ancora a Mosè e ad Aaronne, dicendo:
\par 34 'Quando sarete entrati nel paese di Canaan che io vi do come vostro possesso, se mando la piaga della lebbra in una casa del paese che sarà vostro possesso,
\par 35 il padrone della casa andrà a dichiararlo al sacerdote, dicendo: Mi pare che in casa mia ci sia qualcosa di simile alla lebbra.
\par 36 Allora il sacerdote ordinerà che si sgomberi la casa prima ch'egli v'entri per esaminare la piaga, affinché tutto quello che è nella casa non diventi impuro. Dopo questo, il sacerdote entrerà per esaminar la casa.
\par 37 Ed esaminerà la piaga; e se vedrà che la piaga che è sui muri della casa consiste in fossette verdastre o rossastre che appaiono più profonde della superficie della parete,
\par 38 il sacerdote uscirà dalla casa; e, giunto alla porta, farà chiudere la casa per sette giorni.
\par 39 Il settimo giorno, il sacerdote vi tornerà; e se, esaminandola, vedrà che la piaga s'è allargata sulle pareti della casa,
\par 40 il sacerdote ordinerà che se ne smurino le pietre sulle quali è la piaga, e che si gettino in luogo immondo, fuori di città.
\par 41 Farà raschiare tutto l'interno della casa, e butteranno i calcinacci raschiati fuor di città, in luogo impuro.
\par 42 Poi si prenderanno delle altre pietre e si metteranno al posto delle prime, e si prenderà dell'altra calcina per intonacare la casa.
\par 43 E se la piaga torna ed erompe nella casa dopo averne smurate le pietre e dopo che la casa è stata raschiata e rintonacata,
\par 44 il sacerdote entrerà ad esaminare la casa; e se vedrà che la piaga vi s'è allargata, nella casa c'è della lebbra maligna; la casa è impura.
\par 45 Perciò si demolirà la casa; e se ne porteranno le pietre, il legname e i calcinacci fuori della città, in luogo impuro.
\par 46 Inoltre, chiunque sarà entrato in quella casa durante tutto il tempo che è stata chiusa, sarà impuro fino alla sera.
\par 47 Chi avrà dormito in quella casa, si laverà le vesti; e chi avrà mangiato in quella casa, si laverà le vesti.
\par 48 E se il sacerdote che è entrato nella casa e l'ha esaminata vede che la piaga non s'è allargata nella casa dopo che la casa è stata rintonacata, il sacerdote dichiarerà la casa pura, perché la piaga è guarita.
\par 49 Poi, per purificare la casa, prenderà due uccelli, del legno di cedro, dello scarlatto e dell'issopo;
\par 50 sgozzerà uno degli uccelli in un vaso di terra su dell'acqua viva;
\par 51 e prenderà il legno di cedro, l'issopo, lo scarlatto e l'uccello vivo, e l'immergerà nel sangue dell'uccello sgozzato e nell'acqua viva, e ne aspergerà sette volte la casa.
\par 52 E purificherà la casa col sangue dell'uccello, dell'acqua viva, dell'uccello vivo, col legno di cedro, con l'issopo e con lo scarlatto;
\par 53 ma lascerà andar libero l'uccello vivo, fuor di città, per i campi; e così farà l'espiazione per la casa ed essa sarà pura.
\par 54 Questa è la legge relativa a ogni sorta di piaga di lebbra e alla tigna,
\par 55 alla lebbra delle vesti e della casa,
\par 56 ai tumori, alle pustole e alle macchie lucide,
\par 57 per insegnare quando una cosa è impura e quando è pura. Questa è la legge relativa alla lebbra'.

\chapter{15}

\par 1 L'Eterno parlò ancora a Mosè e ad Aaronne, dicendo:
\par 2 'Parlate ai figliuoli d'Israele e dite loro: Chiunque ha una gonorrea, a motivo della sua gonorrea è impuro.
\par 3 La sua impurità sta nella sua gonorrea; sia la sua gonorrea, continua o intermittente, la impurità esiste.
\par 4 Ogni letto sul quale si coricherà colui che ha la gonorrea, sarà impuro; e ogni oggetto sul quale si sederà sarà impuro.
\par 5 Chi toccherà il letto di colui si laverà le vesti, laverà se stesso nell'acqua, e sarà impuro fino alla sera.
\par 6 Chi si sederà sopra un oggetto qualunque sul quale si sia seduto colui che ha la gonorrea, si laverà le vesti, laverà se stesso nell'acqua, e sarà impuro fino alla sera.
\par 7 Chi toccherà il corpo di colui che ha la gonorrea, si laverà le vesti, laverà se stesso nell'acqua, e sarà impuro fino alla sera.
\par 8 Se colui che ha la gonorrea sputerà sopra uno che è puro, questi si laverà le vesti, laverà se stesso nell'acqua, e sarà impuro fino alla sera.
\par 9 Ogni sella su cui sarà salito chi ha la gonorrea, sarà impura.
\par 10 Chiunque toccherà qualsivoglia cosa che sia stata sotto quel tale, sarà impuro fino alla sera. E chi porterà cotali oggetti si laverà le vesti, laverà se stesso nell'acqua, e sarà impuro fino alla sera.
\par 11 Chiunque sarà toccato da colui che ha la gonorrea, se questi non s'era lavato le mani, dovrà lavarsi le vesti, lavare se stesso nell'acqua, e sarà immondo fino alla sera.
\par 12 Il vaso di terra toccato da colui che ha la gonorrea, sarà spezzato; e ogni vaso di legno sarà lavato nell'acqua.
\par 13 Quando colui che ha la gonorrea sarà purificato della sua gonorrea, conterà sette giorni per la sua purificazione; poi si laverà le vesti, laverà il suo corpo nell'acqua viva, e sarà puro.
\par 14 L'ottavo giorno prenderà due tortore o due giovani piccioni, verrà davanti all'Eterno all'ingresso della tenda di convegno, e li darà al sacerdote.
\par 15 E il sacerdote li offrirà: uno come sacrifizio per il peccato, l'altro come olocausto; e il sacerdote farà l'espiazione per lui davanti all'Eterno, a motivo della sua gonorrea.
\par 16 L'uomo da cui sarà uscito seme genitale si laverà tutto il corpo nell'acqua, e sarà impuro fino alla sera.
\par 17 Ogni veste e ogni pelle su cui sarà seme genitale, si laveranno nell'acqua e saranno impuri fino alla sera.
\par 18 La donna e l'uomo che giaceranno insieme carnalmente, si laveranno ambedue nell'acqua e saranno impuri fino alla sera.
\par 19 Quando una donna avrà i suoi corsi e il sangue le fluirà dalla carne, la sua impurità durerà sette giorni; e chiunque la toccherà sarà impuro fino alla sera.
\par 20 Ogni letto sul quale si sarà messa a dormire durante la sua impurità, sarà impuro; e ogni mobile sul quale si sarà messa a sedere, sarà impuro.
\par 21 Chiunque toccherà il letto di colei si laverà le vesti, laverà se stesso nell'acqua, e sarà impuro fino alla sera.
\par 22 E chiunque toccherà qualsivoglia mobile sul quale ella si sarà seduta, si laverà le vesti, laverà se stesso nell'acqua, e sarà impuro fino alla sera.
\par 23 E se l'uomo si trovava sul letto o sul mobile dov'ella sedeva quand'è avvenuto il contatto, egli sarà impuro fino alla sera.
\par 24 E se un uomo giace con essa, e avvien che lo tocchi la impurità di lei, egli sarà impuro sette giorni; e ogni letto sul quale si coricherà, sarà impuro.
\par 25 La donna che avrà un flusso di sangue, per parecchi giorni, fuori del tempo de' suoi corsi, o che avrà questo flusso oltre il tempo de' suoi corsi, sarà impura per tutto il tempo del flusso, com'è al tempo dei suoi corsi.
\par 26 Ogni letto sul quale si coricherà durante tutto il tempo del suo flusso, sarà per lei come il letto sul quale si corica quando ha i suoi corsi; e ogni mobile sul quale si sederà sarà impuro, com'è impuro quand'ella ha i suoi corsi.
\par 27 E chiunque toccherà quelle cose sarà immondo; si laverà le vesti, laverà se stesso nell'acqua, e sarà impuro fino alla sera.
\par 28 E quand'ella sarà purificata del suo flusso, conterà sette giorni, e poi sarà pura.
\par 29 L'ottavo giorno prenderà due tortore o due giovani piccioni, e li porterà al sacerdote all'ingresso della tenda di convegno.
\par 30 E il sacerdote ne offrirà uno come sacrifizio per il peccato e l'altro come olocausto; il sacerdote farà per lei l'espiazione, davanti all'Eterno, del flusso che la rendeva impura.
\par 31 Così terrete lontani i figliuoli d'Israele da ciò che potrebbe contaminarli, affinché non muoiano a motivo della loro impurità, contaminando il mio tabernacolo ch'è in mezzo a loro.
\par 32 Questa è la legge relativa a colui che ha una gonorrea e a colui dal quale è uscito seme genitale che lo rende immondo,
\par 33 e la legge relativa a colei che è indisposta a motivo de' suoi corsi, all'uomo o alla donna che ha un flusso, e all'uomo che si corica con donna impura'.

\chapter{16}

\par 1 L'Eterno parlò a Mosè dopo la morte dei due figliuoli d'Aaronne, i quali morirono quando si presentarono davanti all'Eterno.
\par 2 L'Eterno disse a Mosè: 'Parla ad Aaronne, tuo fratello, e digli di non entrare in ogni tempo nel santuario, di là dal velo, davanti al propiziatorio che è sull'arca, onde non abbia a morire; poiché io apparirò nella nuvola sul propiziatorio.
\par 3 Aaronne entrerà nel santuario in questo modo: prenderà un giovenco per un sacrifizio per il peccato, e un montone per un olocausto.
\par 4 Si metterà la tunica sacra di lino, e porterà sulla carne le brache di lino; si cingerà della cintura di lino, e si porrà in capo la mitra di lino. Questi sono i paramenti sacri; egli l'indosserà dopo essersi lavato il corpo nell'acqua.
\par 5 Dalla raunanza de' figliuoli d'Israele prenderà due capri per un sacrifizio per il peccato, e un montone per un olocausto.
\par 6 Aaronne offrirà il giovenco del sacrifizio per il peccato, che è per sé, e farà l'espiazione per sé e per la sua casa.
\par 7 Poi prenderà i due capri, e li presenterà davanti all'Eterno all'ingresso della tenda di convegno.
\par 8 E Aaronne trarrà le sorti per vedere qual de' due debba essere dell'Eterno e quale di Azazel.
\par 9 E Aaronne farà accostare il capro ch'è toccato in sorte all'Eterno, e l'offrirà come sacrifizio per il peccato;
\par 10 ma il capro ch'è toccato in sorte ad Azazel sarà posto vivo davanti all'Eterno, perché serva a fare l'espiazione e per mandarlo poi ad Azazel nel deserto.
\par 11 Aaronne offrirà dunque il giovenco del sacrifizio per il peccato per sé, e farà l'espiazione per sé e per la sua casa; e scannerà il giovenco del sacrifizio per il peccato per sé.
\par 12 Poi prenderà un turibolo pieno di carboni accesi tolti di sopra all'altare davanti all'Eterno, e due manate piene di profumo fragrante polverizzato; e porterà ogni cosa di là dal velo.
\par 13 Metterà il profumo sul fuoco davanti all'Eterno, affinché il nuvolo del profumo copra il propiziatorio che è sulla testimonianza, e non morrà.
\par 14 Poi prenderà del sangue del giovenco, e ne aspergerà col dito il propiziatorio dal lato d'oriente, e farà sette volte l'aspersione del sangue col dito, davanti al propiziatorio.
\par 15 Poi scannerà il capro del sacrifizio per il peccato, che è per il popolo, e ne porterà il sangue di là dal velo; e farà di questo sangue quello che ha fatto del sangue del giovenco; ne farà l'aspersione sul propiziatorio e davanti al propiziatorio.
\par 16 Così farà l'espiazione per il santuario, a motivo delle impurità dei figliuoli d'Israele, delle loro trasgressioni e di tutti i loro peccati. Lo stesso farà per la tenda di convegno ch'è stabilita fra loro, in mezzo alle loro impurità.
\par 17 E nella tenda di convegno, quand'egli entrerà nel santuario per farvi l'espiazione, non ci sarà alcuno, finch'egli non sia uscito e non abbia fatto l'espiazione per sé, per la sua casa e per tutta la raunanza d'Israele.
\par 18 Egli uscirà verso l'altare ch'è davanti all'Eterno, e farà l'espiazione per esso; prenderà del sangue del giovenco e del sangue del capro, e lo metterà sui corni dell'altare tutto all'intorno.
\par 19 E farà sette volte l'aspersione del sangue col dito, sopra l'altare, e così lo purificherà e lo santificherà a motivo delle impurità dei figliuoli d'Israele.
\par 20 E quando avrà finito di fare l'espiazione per il santuario, per la tenda di convegno e per l'altare, farà accostare il capro vivo.
\par 21 Aaronne poserà ambedue le mani sul capo del capro vivo, confesserà sopra esso tutte le iniquità dei figliuoli d'Israele, tutte le loro trasgressioni, tutti i loro peccati, e li metterà sulla testa del capro; poi, per mano di un uomo incaricato di questo, lo manderà via nel deserto.
\par 22 E quel capro porterà su di sé tutte le loro iniquità in terra solitaria, e sarà lasciato andare nel deserto.
\par 23 Poi Aaronne entrerà nella tenda di convegno, si spoglierà delle vesti di lino che aveva indossate per entrar nel santuario, e le deporrà quivi.
\par 24 Si laverà il corpo nell'acqua in un luogo santo, si metterà i suoi paramenti, e uscirà ad offrire il suo olocausto e l'olocausto del popolo, e farà l'espiazione per sé e per il popolo.
\par 25 E farà fumare sull'altare il grasso del sacrifizio per il peccato.
\par 26 Colui che avrà lasciato andare il capro destinato ad Azazel si laverà le vesti, laverà il suo corpo nell'acqua, e dopo questo rientrerà nel campo.
\par 27 E si porterà fuori del campo il giovenco del sacrifizio per il peccato e il capro del sacrifizio per il peccato, il cui sangue sarà stato portato nel santuario, per farvi l'espiazione; e se ne bruceranno nel fuoco le pelli, la carne e gli escrementi.
\par 28 Poi colui che li avrà bruciati si laverà le vesti e laverà il suo corpo nell'acqua; dopo questo, rientrerà nel campo.
\par 29 Questa sarà per voi una legge perpetua: nel settimo mese, il decimo giorno del mese, umilierete le anime vostre, non farete lavoro di sorta, né colui che è nativo del paese, né il forestiero che soggiorna fra voi.
\par 30 Poiché in quel giorno si farà l'espiazione per voi, affin di purificarvi; voi sarete purificati da tutti i vostri peccati, davanti all'Eterno.
\par 31 È per voi un sabato di riposo solenne, e voi umilierete le anime vostre; è una legge perpetua.
\par 32 E il sacerdote che ha ricevuto l'unzione ed è stato consacrato per esercitare il sacerdozio al posto di suo padre, farà l'espiazione; si vestirà delle vesti di lino, de' paramenti sacri.
\par 33 E farà l'espiazione per il santuario sacro; farà l'espiazione per la tenda di convegno e per l'altare; farà l'espiazione per i sacerdoti e per tutto il popolo della raunanza.
\par 34 Questa sarà per voi una legge perpetua, per fare una volta all'anno, per i figliuoli d'Israele, l'espiazione di tutti i loro peccati'. E si fece come l'Eterno aveva ordinato a Mosè.

\chapter{17}

\par 1 L'Eterno parlò ancora a Mosè dicendo:
\par 2 'Parla ad Aaronne, ai suoi figliuoli e a tutti i figliuoli d'Israele e di' loro: Questo è quello che l'Eterno ha ordinato, dicendo:
\par 3 Se un uomo qualunque della casa d'Israele scanna un bue o un agnello o una capra entro il campo, o fuori del campo,
\par 4 e non lo mena all'ingresso della tenda di convegno per presentarlo come offerta all'Eterno davanti al tabernacolo dell'Eterno, sarà considerato come colpevole di delitto di sangue; ha sparso del sangue, e cotest'uomo sarà sterminato di fra il suo popolo,
\par 5 affinché i figliuoli d'Israele, invece d'immolare, come fanno, i loro sacrifizi nei campi, li rechino all'Eterno presentandoli al sacerdote, all'ingresso della tenda di convegno, e li offrano all'Eterno come sacrifizi di azioni di grazie.
\par 6 Il sacerdote ne spanderà il sangue sull'altare dell'Eterno, all'ingresso della tenda di convegno, e farà fumare il grasso come un profumo soave all'Eterno.
\par 7 Ed essi non offriranno più i loro sacrifizi ai demoni, ai quali sogliono prostituirsi. Questa sarà per loro una legge perpetua, di generazione in generazione.
\par 8 Di' loro ancora: Se un uomo qualunque della casa d'Israele o degli stranieri che soggiornano fra loro offrirà un olocausto o un sacrifizio,
\par 9 e non lo porterà all'ingresso della tenda di convegno per immolarlo all'Eterno, cotest'uomo sarà sterminato di fra il suo popolo.
\par 10 Se un uomo qualunque della casa d'Israele o degli stranieri che soggiornano fra loro mangia di qualsivoglia specie di sangue, io volgerò la mia faccia contro la persona che avrà mangiato del sangue, e la sterminerò di fra il suo popolo.
\par 11 Poiché la vita della carne è nel sangue. Per questo vi ho ordinato di porlo sull'altare per far l'espiazione per le vostre persone; perché il sangue è quello che fa l'espiazione, mediante la vita.
\par 12 Perciò ho detto ai figliuoli d'Israele: Nessuno tra voi mangerà del sangue; neppure lo straniero che soggiorna fra voi mangerà del sangue.
\par 13 E se uno qualunque de' figliuoli d'Israele o degli stranieri che soggiornano fra loro prende alla caccia un quadrupede o un uccello che si può mangiare, ne spargerà il sangue e lo coprirà di polvere;
\par 14 perché la vita d'ogni carne è il sangue; nel sangue suo sta la vita; perciò ho detto ai figliuoli d'Israele: Non mangerete sangue d'alcuna specie di carne, poiché il sangue è la vita d'ogni carne; chiunque ne mangerà sarà sterminato.
\par 15 E qualunque persona, sia essa nativa del paese o straniera, che mangerà carne di bestia morta da sé o sbranata, si laverà le vesti, laverà se stesso nell'acqua, e sarà impuro fino alla sera; poi sarà puro.
\par 16 Ma se non si lava le vesti e se non lava il suo corpo, porterà la pena della sua iniquità'.

\chapter{18}

\par 1 L'Eterno parlò ancora a Mosè, dicendo: 'Parla ai figliuoli d'Israele, e di' loro:
\par 2 Io sono l'Eterno, l'Iddio vostro.
\par 3 Non farete quel che si fa nel paese d'Egitto dove avete abitato, e non farete quel che si fa nel paese di Canaan dove io vi conduco, e non seguirete i loro costumi.
\par 4 Metterete in pratica le mie prescrizioni e osserverete le mie leggi, per conformarvi ad esse. Io sono l'Eterno, l'Iddio vostro.
\par 5 Osserverete le mie leggi e le mie prescrizioni, mediante le quali chiunque le metterà in pratica, vivrà. Io sono l'Eterno.
\par 6 Nessuno si accosterà ad alcuna sua parente carnale per scoprire la sua nudità. Io sono l'Eterno.
\par 7 Non scoprirai la nudità di tuo padre, né la nudità di tua madre: è tua madre; non scoprirai la sua nudità.
\par 8 Non scoprirai la nudità della moglie di tuo padre: è la nudità di tuo padre.
\par 9 Non scoprirai la nudità della tua sorella, figliuola di tuo padre o figliuola di tua madre, sia essa nata in casa o nata fuori.
\par 10 Non scoprirai la nudità della figliuola del tuo figliuolo o della figliuola della tua figliuola, poiché è la tua propria nudità.
\par 11 Non scoprirai la nudità della figliuola della moglie di tuo padre, generata da tuo padre: è tua sorella.
\par 12 Non scoprirai la nudità della sorella di tuo padre; è parente stretta di tuo padre.
\par 13 Non scoprirai la nudità della sorella di tua madre, perch'è parente stretta di tua madre.
\par 14 Non scoprirai la nudità del fratello di tuo padre, e non t'accosterai alla sua moglie: è tua zia.
\par 15 Non scoprirai la nudità della tua nuora: è la moglie del tuo figliuolo; non scoprire la sua nudità.
\par 16 Non scoprirai la nudità della moglie di tuo fratello: è la nudità di tuo fratello.
\par 17 Non scoprirai la nudità di una donna e della sua figliuola; non prenderai la figliuola del figliuolo di lei, né la figliuola della figliuola di lei per scoprirne la nudità: sono parenti stretti: è un delitto.
\par 18 Non prenderai la sorella di tua moglie per farne una rivale, scoprendo la sua nudità insieme con quella di tua moglie, mentre questa è in vita.
\par 19 Non t'accosterai a donna per scoprir la sua nudità mentre è impura a motivo dei suoi corsi.
\par 20 Non avrai relazioni carnali con la moglie del tuo prossimo per contaminarti con lei.
\par 21 Non darai de' tuoi figliuoli ad essere immolati a Moloc; e non profanerai il nome del tuo Dio. Io sono l'Eterno.
\par 22 Non avrai con un uomo relazioni carnali come si hanno con una donna: è cosa abominevole.
\par 23 Non t'accoppierai con alcuna bestia per contaminarti con essa; e la donna non si prostituirà ad una bestia: è una mostruosità.
\par 24 Non vi contaminate con alcuna di queste cose; poiché con tutte queste cose si son contaminate le nazioni ch'io sto per cacciare dinanzi a voi.
\par 25 Il paese n'è stato contaminato; ond'io punirò la sua iniquità; il paese vomiterà i suoi abitanti.
\par 26 Voi dunque osserverete le mie leggi e le mie prescrizioni, e non commetterete alcuna di queste cose abominevoli; né colui ch'è nativo del paese, né il forestiero che soggiorna fra voi.
\par 27 Poiché tutte queste cose abominevoli le ha commesse la gente che v'era prima di voi, e il paese n'è stato contaminato.
\par 28 Badate che, se lo contaminate, il paese non vi vomiti come vomiterà la gente che vi stava prima di voi.
\par 29 Poiché tutti quelli che commetteranno alcuna di queste cose abominevoli saranno sterminati di fra il loro popolo.
\par 30 Osserverete dunque i miei ordini, e non seguirete alcuno di quei costumi abominevoli che sono stati seguiti prima di voi, e non vi contaminerete con essi. Io sono l'Eterno, l'Iddio vostro'.

\chapter{19}

\par 1 L'Eterno parlò ancora a Mosè, dicendo:
\par 2 'Parla a tutta la raunanza de' figliuoli d'Israele, e di' loro: Siate santi, perché io, l'Eterno, l'Iddio vostro, son santo.
\par 3 Rispetti ciascuno sua madre e suo padre, e osservate i miei sabati. Io sono l'Eterno, l'Iddio vostro.
\par 4 Non vi rivolgete agl'idoli, e non vi fate degli dèi di getto. Io sono l'Eterno, l'Iddio vostro.
\par 5 E quando offrirete un sacrifizio di azioni di grazie all'Eterno, l'offrirete in modo da esser graditi.
\par 6 Lo si mangerà il giorno stesso che l'avrete immolato, e il giorno seguente; e se ne rimarrà qualcosa fino al terzo giorno, lo brucerete col fuoco.
\par 7 Se se ne mangerà il terzo giorno, sarà cosa abominevole; il sacrifizio non sarà gradito.
\par 8 E chiunque ne mangerà porterà la pena della sua iniquità, perché avrà profanato ciò ch'è sacro all'Eterno; e quel tale sarà sterminato di fra il suo popolo.
\par 9 Quando mieterete la raccolta della vostra terra, non mieterai fino all'ultimo canto il tuo campo, e non raccoglierai ciò che resta da spigolare della tua raccolta;
\par 10 e nella tua vigna non coglierai i raspolli, né raccoglierai i granelli caduti; li lascerai per il povero e per il forestiero. Io sono l'Eterno, l'Iddio vostro.
\par 11 Non ruberete, e non userete inganno né menzogna gli uni a danno degli altri.
\par 12 Non giurerete il falso, usando il mio nome; ché profaneresti il nome del tuo Dio. Io sono l'Eterno.
\par 13 Non opprimerai il tuo prossimo, e non gli rapirai ciò ch'è suo; il salario dell'operaio al tuo servizio non ti resti in mano la notte fino al mattino.
\par 14 Non maledirai il sordo, e non porrai inciampo davanti al cieco, ma temerai il tuo Dio. Io sono l'Eterno.
\par 15 Non commetterete iniquità, nel giudicare; non avrai riguardo alla persona del povero, né tributerai speciale onore alla persona del potente; ma giudicherai il tuo prossimo con giustizia.
\par 16 Non andrai qua e là facendo il diffamatore fra il tuo popolo, né ti presenterai ad attestare il falso a danno della vita del tuo prossimo. Io sono l'Eterno.
\par 17 Non odierai il tuo fratello in cuor tuo; riprendi pure il tuo prossimo, ma non ti caricare d'un peccato a cagion di lui.
\par 18 Non ti vendicherai, e non serberai rancore contro i figliuoli del tuo popolo, ma amerai il prossimo tuo come te stesso. Io sono l'Eterno. Osserverete le mie leggi.
\par 19 Non accoppierai bestie di specie differenti; non seminerai il tuo campo con due sorta di seme, né porterai veste tessuta di due diverse materie.
\par 20 Se uno si giace carnalmente con donna che sia schiava promessa a un uomo, ma non riscattata o affrancata, saranno ambedue puniti; ma non saranno messi a morte, perché colei non era libera.
\par 21 L'uomo menerà all'Eterno, all'ingresso della tenda di convegno, come sacrifizio di riparazione, un montone;
\par 22 e il sacerdote farà per lui l'espiazione davanti all'Eterno, col montone del sacrifizio di riparazione, per il peccato che colui ha commesso, e il peccato che ha commesso gli sarà perdonato.
\par 23 Quando sarete entrati nel paese e vi avrete piantato ogni sorta d'alberi fruttiferi, ne considererete i frutti come incirconcisi; per tre anni saranno per voi come incirconcisi; non si dovranno mangiare.
\par 24 Ma il quarto anno tutti i loro frutti saranno consacrati all'Eterno, per dargli lode.
\par 25 E il quinto anno mangerete il frutto di quegli alberi, affinché essi vi aumentino il loro prodotto. Io sono l'Eterno, l'Iddio vostro.
\par 26 Non mangerete nulla che contenga sangue. Non praticherete alcuna sorta di divinazione o di magia.
\par 27 Non vi taglierete in tondo i capelli ai lati del capo, né toglierai i canti alla tua barba.
\par 28 Non vi farete incisioni nella carne per un morto, né vi stamperete segni addosso. Io sono l'Eterno.
\par 29 Non profanare la tua figliuola, prostituendola, affinché il paese non si dia alla prostituzione e non si riempia di scelleratezze.
\par 30 Osserverete i miei sabati, e porterete rispetto al mio santuario. Io sono l'Eterno.
\par 31 Non vi rivolgete agli spiriti, né agl'indovini; non li consultate, per non contaminarvi per mezzo loro. Io sono l'Eterno, l'Iddio vostro.
\par 32 Alzati dinanzi al capo canuto, onora la persona del vecchio, e temi il tuo Dio. Io sono l'Eterno.
\par 33 Quando qualche forestiero soggiornerà con voi nel vostro paese, non gli farete torto.
\par 34 Il forestiero che soggiorna fra voi, lo tratterete come colui ch'è nato fra voi; tu l'amerai come te stesso; poiché anche voi foste forestieri nel paese d'Egitto. Io sono l'Eterno, l'Iddio vostro.
\par 35 Non commettere ingiustizie nei giudizi, né con le misure di lunghezza, né coi pesi, né con le misure di capacità.
\par 36 Avrete stadere giuste, pesi giusti, efa giusto, hin giusto. Io sono l'Eterno, l'Iddio vostro, che v'ho tratto dal paese d'Egitto.
\par 37 Osserverete dunque tutte le mie leggi e tutte le mie prescrizioni, e le metterete in pratica. Io sono l'Eterno'.

\chapter{20}

\par 1 L'Eterno parlò ancora a Mosè, dicendo: 'Dirai ai figliuoli d'Israele:
\par 2 Chiunque de' figliuoli d'Israele o de' forestieri che soggiornano in Israele darà de' suoi figliuoli a Moloc, dovrà esser messo a morte; il popolo del paese lo lapiderà.
\par 3 E anch'io volgerò la mia faccia contro quell'uomo, e lo sterminerò di fra il suo popolo, perché avrà dato de' suoi figliuoli a Moloc per contaminare il mio santuario e profanare il mio santo nome.
\par 4 E se il popolo del paese chiude gli occhi quando quell'uomo dà dei suoi figliuoli a Moloc, e non lo mette a morte,
\par 5 io volgerò la mia faccia contro quell'uomo e contro la sua famiglia, e sterminerò di fra il suo popolo lui con tutti quelli che si prostituiscono come lui, prostituendosi a Moloc.
\par 6 E se qualche persona si volge agli spiriti e agl'indovini per prostituirsi dietro a loro, io volgerò la mia faccia contro quella persona, e lo sterminerò di fra il suo popolo.
\par 7 Santificatevi dunque e siate santi, perché io sono l'Eterno, l'Iddio vostro.
\par 8 E osservate le mie leggi, e mettetele in pratica. Io sono l'Eterno che vi santifica.
\par 9 Chiunque maledice suo padre o sua madre dovrà esser messo a morte; ha maledetto suo padre o sua madre; il suo sangue ricadrà su lui.
\par 10 Se uno commette adulterio con la moglie d'un altro, se commette adulterio con la moglie del suo prossimo, l'adultero e l'adultera dovranno esser messi a morte.
\par 11 Se uno si giace con la moglie di suo padre, egli scopre la nudità di suo padre; ambedue dovranno esser messi a morte; il loro sangue ricadrà su loro.
\par 12 Se uno si giace con la sua nuora, ambedue dovranno esser messi a morte; hanno commesso una cosa abominevole; il loro sangue ricadrà su loro.
\par 13 Se uno ha con un uomo relazioni carnali come si hanno con una donna, ambedue hanno commesso cosa abominevole; dovranno esser messi a morte; il loro sangue ricadrà su loro.
\par 14 Se uno prende per moglie la figlia e la madre, è un delitto; si bruceranno col fuoco lui e loro, affinché non si trovi fra voi alcun delitto.
\par 15 L'uomo che s'accoppia con una bestia, dovrà esser messo a morte; e ucciderete la bestia.
\par 16 E se una donna s'accosta a una bestia per prostituirsi ad essa, ucciderai la donna e la bestia; ambedue dovranno esser messe a morte; il loro sangue ricadrà su loro.
\par 17 Se uno prende la propria sorella, figliuola di suo padre o figliuola di sua madre, e vede la nudità di lei ed ella vede la nudità di lui è una infamia; ambedue saranno sterminati in presenza de' figliuoli del loro popolo; quel tale ha scoperto la nudità della propria sorella; porterà la pena della sua iniquità.
\par 18 Se uno si giace con una donna che ha i suoi corsi, e scopre la nudità di lei, quel tale ha scoperto il flusso di quella donna, ed ella ha scoperto il flusso del proprio sangue; perciò ambedue saranno sterminati di fra il loro popolo.
\par 19 Non scoprirai la nudità della sorella di tua madre o della sorella di tuo padre; chi lo fa scopre la sua stretta parente; ambedue porteranno la pena della loro iniquità.
\par 20 Se uno si giace con la moglie di suo zio, scopre la nudità di suo zio; ambedue porteranno la pena del loro peccato; morranno senza figliuoli.
\par 21 Se uno prende la moglie di suo fratello, è una impurità, egli ha scoperto la nudità di suo fratello; non avranno figliuoli.
\par 22 Osserverete dunque tutte le mie leggi e le mie prescrizioni, e le metterete in pratica, affinché il paese dove io vi conduco per abitarvi non vi vomiti fuori.
\par 23 E non adotterete i costumi delle nazioni che io sto per cacciare d'innanzi a voi; esse hanno fatto tutte quelle cose, e perciò le ho avute in abominio;
\par 24 e vi ho detto: Sarete voi quelli che possederete il loro paese; ve lo darò come vostra proprietà; è un paese ove scorre il latte e il miele. Io sono l'Eterno, l'Iddio vostro, che vi ho separato dagli altri popoli.
\par 25 Farete dunque distinzione fra gli animali puri e quelli impuri, fra gli uccelli impuri e quelli puri, e non renderete le vostre persone abominevoli, mangiando animali, uccelli, o cosa alcuna strisciante sulla terra, e che io v'ho fatto distinguere come impuri.
\par 26 E mi sarete santi, perché io, l'Eterno, son santo, e v'ho separati dagli altri popoli perché foste miei.
\par 27 Se un uomo o una donna ha uno spirito o indovina, dovranno esser messi a morte; saranno lapidati; il loro sangue ricadrà su loro'.

\chapter{21}

\par 1 L'Eterno disse ancora a Mosè: 'Parla ai sacerdoti, figliuoli d'Aaronne, e di' loro: Un sacerdote non si esporrà a divenire impuro in mezzo al suo popolo per il contatto con un morto,
\par 2 a meno che si tratti d'uno de' suoi parenti più stretti: di sua madre, di suo padre, del suo figliuolo, della sua figliuola,
\par 3 del suo fratello e della sua sorella ancora vergine che vive con lui, non essendo ancora maritata; per questa può esporsi alla impurità.
\par 4 Capo com'è in mezzo al suo popolo, non si contaminerà, profanando se stesso.
\par 5 I sacerdoti non si faranno tonsure sul capo, non si raderanno i canti della barba, e non si faranno incisioni nella carne.
\par 6 Saranno santi al loro Dio e non profaneranno il nome del loro Dio, poiché offrono all'Eterno i sacrifizi fatti mediante il fuoco, il pane del loro Dio; perciò saran santi.
\par 7 Non prenderanno una prostituta, né una donna disonorata; non prenderanno una donna ripudiata dal suo marito, perché sono santi al loro Dio.
\par 8 Tu considererai dunque il sacerdote come santo, perch'egli offre il pane del tuo Dio: ei ti sarà santo, perché io, l'Eterno che vi santifico, son santo.
\par 9 Se la figliuola di un sacerdote si disonora prostituendosi, ella disonora suo padre; sarà arsa col fuoco.
\par 10 Il sommo sacerdote che sta al disopra de' suoi fratelli, sul capo del quale è stato sparso l'olio dell'unzione e che è stato consacrato per rivestire i paramenti sacri, non si scoprirà il capo e non si straccerà le vesti.
\par 11 Non si avvicinerà ad alcun cadavere; non si renderà impuro neppure per suo padre e per sua madre.
\par 12 Non uscirà dal santuario, e non profanerà il santuario del suo Dio, perché l'olio dell'unzione del suo Dio è su lui come un diadema. Io sono l'Eterno.
\par 13 Sposerà una vergine.
\par 14 Non sposerà né una vedova, né una divorziata, né una disonorata, né una meretrice; ma prenderà per moglie una vergine del suo popolo.
\par 15 Non disonorerà la sua progenie in mezzo al suo popolo; poiché io sono l'Eterno che lo santifico'.
\par 16 L'Eterno parlò ancora a Mosè, dicendo:
\par 17 'Parla ad Aaronne e digli: Nelle generazioni a venire nessun uomo della tua stirpe che abbia qualche deformità s'accosterà per offrire il pane del suo Dio;
\par 18 perché nessun uomo che abbia qualche deformità potrà accostarsi: né il cieco, né lo zoppo, né colui che ha una deformità per difetto o per eccesso,
\par 19 o una frattura al piede o alla mano,
\par 20 né il gobbo, né il nano, né colui che ha una macchia nell'occhio, o ha la rogna o un erpete o i testicoli infranti.
\par 21 Nessun uomo della stirpe del sacerdote Aaronne, che abbia qualche deformità, si accosterà per offrire i sacrifizi fatti mediante il fuoco all'Eterno. Ha un difetto: non s'accosti quindi per offrire il pane del suo Dio.
\par 22 Egli potrà mangiare del pane del suo Dio, delle cose santissime e delle cose sante;
\par 23 ma non si avvicinerà al velo, e non s'accosterà all'altare, perché ha una deformità. Non profanerà i miei luoghi santi, perché io sono l'Eterno che li santifico'.
\par 24 Così parlò Mosè ad Aaronne, ai figliuoli di lui e a tutti i figliuoli d'Israele.

\chapter{22}

\par 1 L'Eterno parlò ancora a Mosè, dicendo:
\par 2 'Di' ad Aaronne e ai suoi figliuoli che si astengano dalle cose sante che mi son consacrate dai figliuoli d'Israele, e non profanino il mio santo nome. Io sono l'Eterno.
\par 3 Di' loro: Qualunque uomo della vostra stirpe che nelle vostre future generazioni, trovandosi in stato d'impurità, s'accosterà alle cose sante che i figliuoli d'Israele consacrano all'Eterno, sarà sterminato dal mio cospetto. Io sono l'Eterno.
\par 4 Qualunque uomo della stirpe d'Aaronne che sia lebbroso o abbia la gonorrea, non mangerà delle cose sante, finché non sia puro. E così sarà di chi avrà toccato una persona impura per contatto con un morto, o avrà avuto una perdita di seme genitale,
\par 5 o di chi avrà toccato un rettile che l'abbia reso impuro, o un uomo che gli abbia comunicato una impurità di qualsivoglia specie.
\par 6 La persona che avrà avuto di tali contatti sarà impura fino alla sera, e non mangerà delle cose sante prima d'essersi lavato il corpo nell'acqua;
\par 7 dopo il tramonto del sole sarà pura, e potrà poi mangiare delle cose sante, perché sono il suo pane.
\par 8 Il sacerdote non mangerà carne di bestia morta da sé o sbranata, per non rendersi impuro. Io sono l'Eterno.
\par 9 Osserveranno dunque ciò che ho comandato, onde non portino la pena del loro peccato, e muoiano per aver profanato le cose sante. Io sono l'Eterno che li santifico.
\par 10 Nessun estraneo al sacerdozio mangerà delle cose sante: chi sta da un sacerdote o lavora da lui per un salario non mangerà delle cose sante.
\par 11 Ma una persona che il sacerdote avrà comprata coi suoi danari, ne potrà mangiare; così pure colui che gli è nato in casa: questi potranno mangiare del pane di lui.
\par 12 La figliuola di un sacerdote maritata a un estraneo non mangerà delle cose sante offerte per elevazione.
\par 13 Ma se la figliuola del sacerdote è vedova, o ripudiata, senza figliuoli, e torna a stare da suo padre come quand'era giovine, potrà mangiare del pane del padre; ma nessun estraneo al sacerdozio ne mangerà.
\par 14 E se uno mangia per sbaglio di una cosa santa, darà al sacerdote il valore della cosa santa, aggiungendovi un quinto.
\par 15 I sacerdoti non profaneranno dunque le cose sante dei figliuoli d'Israele, ch'essi offrono per elevazione all'Eterno,
\par 16 e non faranno loro portare la pena del peccato di cui si renderebbero colpevoli, mangiando delle loro cose sante; poiché io sono l'Eterno che li santifico'.
\par 17 L'Eterno parlò ancora a Mosè, dicendo:
\par 18 'Parla ad Aaronne, ai suoi figliuoli, a tutti i figliuoli d'Israele, e di' loro: Chiunque sia della casa d'Israele o de' forestieri in Israele che presenti in olocausto all'Eterno un'offerta per qualche voto o per qualche dono volontario per essere gradito,
\par 19 dovrà offrire un maschio, senza difetto, di fra i buoi, di fra le pecore o di fra le capre.
\par 20 Non offrirete nulla che abbia qualche difetto, perché non sarebbe gradito.
\par 21 Quand'uno offrirà all'Eterno un sacrifizio di azioni di grazie, di buoi o di pecore, sia per sciogliere un voto, sia come offerta volontaria, la vittima, perché sia gradita, dovrà esser perfetta: non dovrà aver difetti.
\par 22 Non offrirete all'Eterno una vittima che sia cieca, o storpia, o mutilata, o che abbia delle ulceri, o la rogna, o la scabbia; e non ne farete sull'altare un sacrifizio mediante il fuoco all'Eterno.
\par 23 Potrai presentare come offerta volontaria un bue o una pecora che abbia un membro troppo lungo o troppo corto; ma, come offerta per qualche voto, non sarebbe gradito.
\par 24 Non offrirete all'Eterno un animale che abbia i testicoli ammaccati o schiacciati o strappati o tagliati; e di queste operazioni non ne farete nel vostro paese.
\par 25 Non accetterete dallo straniero alcuna di queste vittime per offrirla come pane del vostro Dio; siccome sono mutilate, difettose, non sarebbero gradite per il vostro bene'.
\par 26 L'Eterno parlò ancora a Mosè, dicendo:
\par 27 'Quando sarà nato un vitello, o un agnello, o un capretto, starà sette giorni sotto la madre; dall'ottavo giorno in poi, sarà gradito come sacrifizio fatto mediante il fuoco all'Eterno.
\par 28 Sia vacca, sia pecora, non la scannerete lo stesso giorno col suo parto.
\par 29 Quando offrirete all'Eterno un sacrifizio di azioni di grazie, l'offrirete in modo da esser graditi.
\par 30 La vittima sarà mangiata il giorno stesso; non ne lascerete nulla fino al mattino. Io sono l'Eterno.
\par 31 Osserverete dunque i miei comandamenti, e li metterete in pratica. Io sono l'Eterno.
\par 32 Non profanerete il mio santo nome, ond'io sia santificato in mezzo ai figliuoli d'Israele. Io sono l'Eterno che vi santifico,
\par 33 che vi ho tratto dal paese d'Egitto per esser vostro Dio. Io sono l'Eterno'.

\chapter{23}

\par 1 L'Eterno parlò ancora a Mosè, dicendo:
\par 2 'Parla ai figliuoli d'Israele e di' loro: Ecco le solennità dell'Eterno, che voi bandirete come sante convocazioni. Le mie solennità son queste.
\par 3 Durante sei giorni si attenderà al lavoro; ma il settimo giorno è sabato, giorno di completo riposo e di santa convocazione. Non farete in esso lavoro alcuno; è un riposo consacrato all'Eterno in tutti i luoghi dove abiterete.
\par 4 Queste sono le solennità dell'Eterno, le sante convocazioni che bandirete ai tempi stabiliti.
\par 5 Il primo mese, il quattordicesimo giorno del mese, sull'imbrunire, sarà la Pasqua dell'Eterno;
\par 6 e il quindicesimo giorno dello stesso mese sarà la festa dei pani azzimi in onore dell'Eterno; per sette giorni mangerete pane senza lievito.
\par 7 Il primo giorno avrete una santa convocazione; non farete in esso alcuna opera servile;
\par 8 e per sette giorni offrirete all'Eterno de' sacrifizi mediante il fuoco. Il settimo giorno si avrà una santa convocazione, non farete alcuna opera servile'.
\par 9 L'Eterno parlò ancora a Mosè, dicendo:
\par 10 'Parla ai figliuoli d'Israele, e di' loro: Quando sarete entrati nel paese che io vi do e ne mieterete la raccolta, porterete al sacerdote una mannella, come primizia della vostra raccolta;
\par 11 e il sacerdote agiterà la mannella davanti all'Eterno, perché sia gradita per il vostro bene; il sacerdote l'agiterà il giorno dopo il sabato.
\par 12 E il giorno che agiterete la mannella, offrirete un agnello di un anno, che sia senza difetto, come olocausto all'Eterno.
\par 13 L'oblazione che l'accompagna sarà di due decimi di un efa di fior di farina intrisa con olio, come sacrifizio mediante il fuoco, di soave odore all'Eterno; la libazione sarà d'un quarto di un hin di vino.
\par 14 Non mangerete pane, né grano arrostito, né spighe fresche, fino a quel giorno, fino a che abbiate portata l'offerta al vostro Dio. È una legge perpetua, di generazione in generazione, in tutti i luoghi dove abiterete.
\par 15 Dall'indomani del sabato, dal giorno che avrete portato la mannella dell'offerta agitata, conterete sette settimane intere.
\par 16 Conterete cinquanta giorni fino all'indomani del settimo sabato, e offrirete all'Eterno una nuova oblazione.
\par 17 Porterete dai luoghi dove abiterete due pani per un'offerta agitata, i quali saranno di due decimi di un efa di fior di farina e cotti con del lievito; sono le primizie offerte all'Eterno.
\par 18 E con que' pani offrirete sette agnelli dell'anno, senza difetto, un giovenco e due montoni, che saranno un olocausto all'Eterno assieme alla loro oblazione e alle loro libazioni; sarà un sacrifizio di soave odore fatto mediante il fuoco all'Eterno.
\par 19 E offrirete un capro come sacrifizio per il peccato, e due agnelli dell'anno come sacrifizio di azioni di grazie.
\par 20 Il sacerdote agiterà gli agnelli col pane delle primizie, come offerta agitata davanti all'Eterno; e tanto i pani quanto i due agnelli consacrati all'Eterno apparterranno al sacerdote.
\par 21 In quel medesimo giorno bandirete la festa, e avrete una santa convocazione. Non farete alcuna opera servile. È una legge perpetua, di generazione in generazione, in tutti i luoghi dove abiterete.
\par 22 Quando mieterete la raccolta della vostra terra, non mieterai fino all'ultimo canto il tuo campo, e non raccoglierai ciò che resta da spigolare della tua raccolta; lo lascerai per il povero e per il forestiero. Io sono l'Eterno, l'Iddio vostro'.
\par 23 L'Eterno parlò ancora a Mosè, dicendo:
\par 24 'Parla ai figliuoli d'Israele, e di' loro: Il settimo mese, il primo giorno del mese avrete un riposo solenne, una commemorazione fatta a suon di tromba, una santa convocazione.
\par 25 Non farete alcun'opera servile, e offrirete all'Eterno dei sacrifizi mediante il fuoco'.
\par 26 L'Eterno parlò ancora a Mosè, dicendo:
\par 27 'Il decimo giorno di questo settimo mese sarà il giorno delle espiazioni; avrete una santa convocazione, umilierete le anime vostre e offrirete all'Eterno de' sacrifizi mediante il fuoco.
\par 28 In quel giorno non farete alcun lavoro; poiché è un giorno d'espiazione, destinato a fare espiazione per voi davanti all'Eterno, ch'è l'Iddio vostro.
\par 29 Poiché, ogni persona che non si umilierà in quel giorno, sarà sterminata di fra il suo popolo.
\par 30 E ogni persona che farà in quel giorno qualsivoglia lavoro, io la distruggerò di fra il suo popolo.
\par 31 Non farete alcun lavoro. È una legge perpetua, di generazione in generazione, in tutti i luoghi dove abiterete.
\par 32 Sarà per voi un sabato di completo riposo, e umilierete le anime vostre; il nono giorno del mese, dalla sera alla sera seguente, celebrerete il vostro sabato'.
\par 33 L'Eterno parlò ancora a Mosè, dicendo:
\par 34 'Parla ai figliuoli d'Israele, e di' loro: Il quindicesimo giorno di questo settimo mese sarà la festa delle Capanne, durante sette giorni, in onore dell'Eterno.
\par 35 Il primo giorno vi sarà una santa convocazione; non farete alcuna opera servile.
\par 36 Per sette giorni offrirete all'Eterno dei sacrifizi mediante il fuoco. L'ottavo giorno avrete una santa convocazione, e offrirete all'Eterno dei sacrifizi mediante il fuoco. È giorno di solenne raunanza; non farete alcuna opera servile.
\par 37 Queste sono le solennità dell'Eterno che voi bandirete come sante convocazioni, perché si offrano all'Eterno sacrifizi mediante il fuoco, olocausti e oblazioni, vittime e libazioni, ogni cosa al giorno stabilito,
\par 38 oltre i sabati dell'Eterno, oltre i vostri doni, oltre tutti i vostri voti e tutte le offerte volontarie che presenterete all'Eterno.
\par 39 Or il quindicesimo giorno del settimo mese, quando avrete raccolto i frutti della terra, celebrerete una festa all'Eterno, durante sette giorni; il primo giorno sarà di completo riposo; e l'ottavo, di completo riposo.
\par 40 Il primo giorno prenderete del frutto di alberi d'ornamento: rami di palma, rami dalla verzura folta e salci de' torrenti, e vi rallegrerete dinanzi all'Eterno, ch'è l'Iddio vostro, durante sette giorni.
\par 41 Celebrerete questa festa in onore dell'Eterno per sette giorni, ogni anno. È una legge perpetua, di generazione in generazione. La celebrerete il settimo mese.
\par 42 Dimorerete in capanne durante sette giorni; tutti quelli che saranno nativi d'Israele dimoreranno in capanne,
\par 43 affinché i vostri discendenti sappiano che io feci dimorare in capanne i figliuoli d'Israele, quando li trassi fuori dal paese d'Egitto. Io sono l'Eterno, l'Iddio vostro'.
\par 44 Così Mosè dette ai figliuoli d'Israele le istruzioni relative alle solennità dell'Eterno.

\chapter{24}

\par 1 L'Eterno parlò ancora a Mosè, dicendo:
\par 2 'Ordina ai figliuoli d'Israele che ti portino dell'olio di uliva puro, vergine, per il candelabro, per tener le lampade continuamente accese.
\par 3 Aaronne lo preparerà nella tenda di convegno, fuori del velo che sta davanti alla testimonianza, perché le lampade ardano del continuo, dalla sera al mattino, davanti all'Eterno. È una legge perpetua, di generazione in generazione.
\par 4 Egli le disporrà sul candelabro d'oro puro, perché ardano del continuo davanti all'Eterno.
\par 5 Prenderai pure del fior di farina, e ne farai cuocere dodici focacce; ogni focaccia sarà di due decimi d'efa.
\par 6 Le metterai in due file, sei per fila, sulla tavola d'oro puro davanti all'Eterno.
\par 7 E porrai dell'incenso puro sopra ogni fila, e sarà sul pane come una ricordanza, come un sacrifizio fatto mediante il fuoco all'Eterno.
\par 8 Ogni giorno di sabato si disporranno i pani davanti all'Eterno, del continuo; saranno forniti dai figliuoli d'Israele; è un patto perpetuo.
\par 9 I pani apparterranno ad Aaronne e ai suoi figliuoli, ed essi li mangeranno in luogo santo; poiché saranno per loro cosa santissima tra i sacrifizi fatti mediante il fuoco all'Eterno. È una legge perpetua'.
\par 10 Or il figliuolo di una donna israelita e di un Egiziano uscì tra i figliuoli d'Israele: e fra questo figliuolo della donna israelita e un Israelita nacque una lite.
\par 11 Il figliuolo della Israelita bestemmiò il nome dell'Eterno, e lo maledisse; onde fu condotto a Mosè. La madre di quel tale si chiamava Shelomith, figliuola di Dibri, della tribù di Dan.
\par 12 Lo misero in prigione, finché fosse deciso che cosa fare per ordine dell'Eterno.
\par 13 E l'Eterno parlò a Mosè, dicendo:
\par 14 'Mena quel bestemmiatore fuori del campo; e tutti quelli che l'hanno udito posino le mani sul suo capo, e tutta la raunanza lo lapidi.
\par 15 E parla ai figliuoli d'Israele, e di' loro: Chiunque maledirà il suo Dio porterà la pena del suo peccato.
\par 16 E chi bestemmia il nome dell'Eterno dovrà esser messo a morte; tutta la raunanza lo dovrà lapidare. Sia straniero o nativo del paese, quando bestemmi il nome dell'Eterno, sarà messo a morte.
\par 17 Chi percuote mortalmente un uomo qualsivoglia, dovrà esser messo a morte.
\par 18 Chi percuote a morte un capo di bestiame, lo pagherà: vita per vita.
\par 19 Quand'uno avrà fatto una lesione al suo prossimo, gli sarà fatto com'egli ha fatto:
\par 20 frattura per frattura, occhio per occhio, dente per dente; gli si farà la stessa lesione ch'egli ha fatta all'altro.
\par 21 Chi uccide un capo di bestiame, lo pagherà; ma chi uccide un uomo sarà messo a morte.
\par 22 Avrete una stessa legge tanto per il forestiero quanto per il nativo del paese; poiché io sono l'Eterno, l'Iddio vostro'.
\par 23 E Mosè parlò ai figliuoli d'Israele, i quali trassero quel bestemmiatore fuori del campo, e lo lapidarono. Così i figliuoli d'Israele fecero quello che l'Eterno aveva ordinato a Mosè.

\chapter{25}

\par 1 L'Eterno parlò ancora a Mosè sul monte Sinai, dicendo:
\par 2 'Parla ai figliuoli d'Israele e di' loro: Quando sarete entrati nel paese che io vi do, la terra dovrà avere il suo tempo di riposo consacrato all'Eterno.
\par 3 Per sei anni seminerai il tuo campo, per sei anni poterai la tua vigna e ne raccoglierai i frutti;
\par 4 ma il settimo anno sarà un sabato, un riposo completo per la terra, un sabato in onore dell'Eterno; non seminerai il tuo campo, né poterai la tua vigna.
\par 5 Non mieterai quello che nascerà da sé dal seme caduto nella tua raccolta precedente, e non vendemmierai l'uva della vigna che non avrai potata; sarà un anno di completo riposo per la terra.
\par 6 Ciò che la terra produrrà durante il suo riposo, servirà di nutrimento a te, al tuo servo, alla tua serva, al tuo operaio e al tuo forestiero che stanno da te,
\par 7 al tuo bestiame e agli animali che sono nel tuo paese; tutto il suo prodotto servirà loro di nutrimento.
\par 8 Conterai pure sette settimane d'anni: sette volte sette anni; e queste sette settimane d'anni ti faranno un periodo di quarantanove anni.
\par 9 Poi, il decimo giorno del settimo mese farai squillar la tromba; il giorno delle espiazioni farete squillar la tromba per tutto il paese.
\par 10 E santificherete il cinquantesimo anno, e proclamerete l'affrancamento nel paese per tutti i suoi abitanti. Sarà per voi un giubileo; ognun di voi tornerà nella sua proprietà, e ognun di voi tornerà nella sua famiglia.
\par 11 Il cinquantesimo anno sarà per voi un giubileo; non seminerete e non raccoglierete quello che i campi produrranno da sé, e non vendemmierete le vigne non potate.
\par 12 Poiché è il giubileo; esso vi sarà sacro; mangerete il prodotto che vi verrà dai campi.
\par 13 In quest'anno del giubileo ciascuno tornerà in possesso del suo.
\par 14 Se vendete qualcosa al vostro prossimo o se comprate qualcosa dal vostro prossimo, nessuno faccia torto al suo fratello.
\par 15 Regolerai la compra che farai dal tuo prossimo, sul numero degli anni passati dall'ultimo giubileo, e quegli venderà a te in ragione degli anni di rendita.
\par 16 Quanti più anni resteranno, tanto più aumenterai il prezzo; e quanto minore sarà il tempo, tanto calerai il prezzo; poiché quegli ti vende il numero delle raccolte.
\par 17 Nessun di voi danneggi il suo fratello, ma temerai il tuo Dio; poiché io sono l'Eterno, l'Iddio vostro.
\par 18 Voi metterete in pratica le mie leggi, e osserverete le mie prescrizioni e le adempirete, e abiterete il paese in sicurtà.
\par 19 La terra produrrà i suoi frutti, voi ne mangerete a sazietà e abiterete in essa in sicurtà.
\par 20 E se dite: - Che mangeremo il settimo anno, giacché non semineremo e non faremo la nostra raccolta? -
\par 21 Io disporrò che la mia benedizione venga su voi il sesto anno, ed esso vi darà una raccolta per tre anni.
\par 22 E l'ottavo anno seminerete e mangerete della vecchia raccolta fino al nono anno; mangerete della raccolta vecchia finché sia venuta la nuova.
\par 23 Le terre non si venderanno per sempre; perché la terra è mia, e voi state da me come forestieri e avventizi.
\par 24 Perciò, in tutto il paese che sarà vostro possesso, concederete il diritto di riscatto del suolo.
\par 25 Se il tuo fratello diventa povero e vende una parte della sua proprietà, colui che ha il diritto di riscatto, il suo parente più prossimo, verrà e riscatterà ciò che il suo fratello ha venduto.
\par 26 E se uno non ha chi possa fare il riscatto, ma giunge a procurarsi da sé la somma necessaria al riscatto,
\par 27 conterà le annate scorse dalla vendita, renderà il soprappiù al compratore, e rientrerà così nel suo.
\par 28 Ma se non trova da sé la somma sufficiente a rimborsarlo, ciò che ha venduto rimarrà in mano del compratore fino all'anno del giubileo; al giubileo sarà cosa franca, ed egli rientrerà nel suo possesso.
\par 29 Se uno vende una casa da abitare in una città murata, avrà il diritto di riscattarla fino al compimento di un anno dalla vendita; il suo diritto di riscatto durerà un anno intero.
\par 30 Ma se quella casa posta in una città murata non è riscattata prima del compimento d'un anno intero, rimarrà in perpetuo proprietà del compratore e dei suoi discendenti; non sarà franca al giubileo.
\par 31 Però, le case de' villaggi non attorniati da mura saranno considerate come parte dei fondi di terreno; potranno essere riscattate, e al giubileo saranno franche.
\par 32 Quanto alle città de' Leviti e alle case ch'essi vi possederanno, i Leviti avranno il diritto perpetuo di riscatto.
\par 33 E se anche uno de' Leviti ha fatto il riscatto, la casa venduta, con la città dove si trova, sarà franca al giubileo, perché le case delle città dei Leviti sono loro proprietà, in mezzo ai figliuoli d'Israele.
\par 34 I campi situati ne' dintorni delle città dei Leviti non si potranno vendere, perché sono loro proprietà perpetua.
\par 35 Se il tuo fratello ch'è presso di te è impoverito e i suoi mezzi vengon meno, tu lo sosterrai, anche se forestiero e avventizio, onde possa vivere presso di te.
\par 36 Non trarre da lui interesse, né utile; ma temi il tuo Dio, e viva il tuo fratello presso di te.
\par 37 Non gli darai il tuo danaro a interesse, né gli darai i tuoi viveri per ricavarne un utile.
\par 38 Io sono l'Eterno, il vostro Dio, che vi ho tratto dal paese d'Egitto per darvi il paese di Canaan, per essere il vostro Dio.
\par 39 Se il tuo fratello ch'è presso di te è impoverito e si vende a te, non lo farai servire come uno schiavo;
\par 40 starà da te come un lavorante, come un avventizio. Ti servirà fino all'anno del giubileo;
\par 41 allora se ne andrà da te insieme coi suoi figliuoli, tornerà nella sua famiglia, e rientrerà nella proprietà de' suoi padri.
\par 42 Poiché essi sono miei servi, ch'io trassi dal paese d'Egitto; non debbon esser venduti come si vendono gli schiavi.
\par 43 Non lo dominerai con asprezza, ma temerai il tuo Dio.
\par 44 Quanto allo schiavo e alla schiava che potrete avere in proprio, li prenderete dalle nazioni che vi circondano; da queste comprerete lo schiavo e la schiava.
\par 45 Potrete anche comprarne tra i figliuoli degli stranieri stabiliti fra voi e fra le loro famiglie che si troveranno fra voi, tra i figliuoli ch'essi avranno generato nel vostro paese; e saranno vostra proprietà.
\par 46 E li potrete lasciare in eredità ai vostri figliuoli dopo di voi, come loro proprietà; vi servirete di loro come di schiavi in perpetuo; ma quanto ai vostri fratelli, i figliuoli d'Israele, nessun di voi dominerà l'altro con asprezza.
\par 47 Se un forestiero stabilito presso di te arricchisce, e il tuo fratello divien povero presso di lui e si vende al forestiero stabilito presso di te o a qualcuno della famiglia del forestiero,
\par 48 dopo che si sarà venduto, potrà essere riscattato; lo potrà riscattare uno de' suoi fratelli;
\par 49 lo potrà riscattare suo zio, o il figliuolo del suo zio; lo potrà riscattare uno dei parenti dello stesso suo sangue, o, se ha i mezzi di farlo, potrà riscattarsi da sé.
\par 50 Farà il conto, col suo compratore, dall'anno che gli si è venduto all'anno del giubileo; e il prezzo da pagare si regolerà secondo il numero degli anni, valutando le sue giornate come quelle di un lavorante.
\par 51 Se vi sono ancora molti anni per arrivare al giubileo, pagherà il suo riscatto in ragione di questi anni, e in proporzione del prezzo per il quale fu comprato:
\par 52 se rimangon pochi anni per arrivare al giubileo, farà il conto col suo compratore, e pagherà il prezzo del suo riscatto in ragione di quegli anni.
\par 53 Starà da lui come un lavorante fissato annualmente; il padrone non lo dominerà con asprezza sotto i tuoi occhi.
\par 54 E se non è riscattato in alcuno di quei modi, se ne uscirà libero l'anno del giubileo: egli, coi suoi figliuoli.
\par 55 Poiché i figliuoli d'Israele son servi miei; sono miei servi, che ho tratto dal paese d'Egitto. Io sono l'Eterno, l'Iddio vostro.

\chapter{26}

\par 1 Non vi farete idoli, non vi eleverete immagini scolpite né statue, e non collocherete nel vostro paese alcuna pietra ornata di figure per prostrarvi davanti ad essa; poiché io sono l'Eterno, l'Iddio vostro.
\par 2 Osserverete i miei sabati, e porterete rispetto al mio santuario. Io sono l'Eterno.
\par 3 Se vi conducete secondo le mie leggi, se osservate i miei comandamenti e li mettete in pratica,
\par 4 io vi darò le piogge nella loro stagione, la terra darà i suoi prodotti, e gli alberi della campagna daranno i loro frutti.
\par 5 La trebbiatura vi durerà fino alla vendemmia, e la vendemmia vi durerà fino alla sementa; mangerete a sazietà il vostro pane, e abiterete in sicurtà il vostro paese.
\par 6 Io farò che la pace regni nel paese; voi vi coricherete, e non ci sarà chi vi spaventi; farò sparire dal paese le bestie nocive, e la spada non passerà per il vostro paese.
\par 7 Voi inseguirete i vostri nemici, ed essi cadranno dinanzi a voi per la spada.
\par 8 Cinque di voi ne inseguiranno cento, cento di voi ne inseguiranno diecimila, e i vostri nemici cadranno dinanzi a voi per la spada.
\par 9 E io mi volgerò verso voi, vi renderò fecondi e vi moltiplicherò, e raffermerò il mio patto con voi.
\par 10 E voi mangerete delle raccolte vecchie, serbate a lungo, e trarrete fuori la raccolta vecchia per far posto alla nuova.
\par 11 Io stabilirò la mia dimora in mezzo a voi, e l'anima mia non vi aborrirà.
\par 12 Camminerò tra voi, sarò vostro Dio, e voi sarete mio popolo.
\par 13 Io sono l'Eterno, l'Iddio vostro, che vi ho tratto dal paese d'Egitto affinché non vi foste più schiavi; ho spezzato il vostro giogo, e v'ho fatto camminare a test'alta.
\par 14 Ma se non mi date ascolto e se non mettete in pratica tutti questi comandamenti,
\par 15 se disprezzate le mie leggi e l'anima vostra disdegna le mie prescrizioni in guisa che non mettiate in pratica tutti i miei comandamenti e rompiate il mio patto,
\par 16 ecco quel che vi farò a mia volta: manderò contro voi il terrore, la consunzione e la febbre, che vi faranno venir meno gli occhi e languir l'anima, e seminerete invano la vostra sementa: la mangeranno i vostri nemici.
\par 17 Volgerò la mia faccia contro di voi, e voi sarete sconfitti dai vostri nemici; quelli che vi odiano vi domineranno, e vi darete alla fuga senza che alcuno v'insegua.
\par 18 E se nemmeno dopo questo vorrete darmi ascolto, io vi castigherò sette volte di più per i vostri peccati.
\par 19 Spezzerò la superbia della vostra forza, farò che il vostro cielo sia come di ferro, e la vostra terra come di rame.
\par 20 La vostra forza si consumerà invano, poiché la vostra terra non darà i suoi prodotti, e gli alberi della campagna non daranno i loro frutti.
\par 21 E se mi resistete con la vostra condotta e non volete darmi ascolto, io vi colpirò sette volte di più, secondo i vostri peccati.
\par 22 Manderò contro di voi le fiere della campagna, che vi rapiranno i figliuoli, stermineranno il vostro bestiame, vi ridurranno a un piccol numero, e le vostre strade diverranno deserte.
\par 23 E se, nonostante questi castighi, non volete correggervi per tornare a me, ma con la vostra condotta mi resistete, anch'io vi resisterò,
\par 24 e vi colpirò sette volte di più per i vostri peccati.
\par 25 E farò venir contro di voi la spada, vindice del mio patto; voi vi raccoglierete nelle vostre città, ma io manderò in mezzo a voi la peste, e sarete dati in man del nemico.
\par 26 Quando vi toglierò il pane che sostiene, dieci donne coceranno il vostro pane in uno stesso forno, vi distribuiranno il vostro pane a peso, e mangerete, ma non vi sazierete.
\par 27 E se, nonostante tutto questo, non volete darmi ascolto ma con la vostra condotta mi resistete,
\par 28 anch'io vi resisterò con furore, e vi castigherò sette volte di più per i vostri peccati.
\par 29 Mangerete la carne dei vostri figliuoli, e mangerete la carne delle vostre figliuole.
\par 30 Io devasterò i vostri alti luoghi, distruggerò le vostre statue consacrate al sole, metterò i vostri cadaveri sui cadaveri dei vostri idoli, e l'anima mia vi aborrirà.
\par 31 E ridurrò le vostre città in deserti, desolerò i vostri santuari, e non aspirerò più il soave odore dei vostri profumi.
\par 32 Desolerò il paese; e i vostri nemici che vi abiteranno, ne saranno stupefatti.
\par 33 E, quanto a voi, io vi disperderò fra le nazioni, e vi darò dietro a spada tratta; il vostro paese sarà desolato, e le vostre città saranno deserte.
\par 34 Allora la terra si godrà i suoi sabati per tutto il tempo che rimarrà desolata e che voi sarete nel paese dei vostri nemici; allora la terra si riposerà e si godrà i suoi sabati.
\par 35 Per tutto il tempo che rimarrà desolata avrà il riposo che non ebbe nei vostri sabati, quando voi l'abitavate.
\par 36 Quanto ai superstiti fra voi, io renderò pusillanime il loro cuore nel paese dei loro nemici: il rumore d'una foglia agitata li metterà in fuga; fuggiranno come si fugge dinanzi alla spada, e cadranno senza che alcuno l'insegua.
\par 37 Precipiteranno l'uno sopra l'altro come davanti alla spada, senza che alcuno l'insegua, e voi non potrete resistere dinanzi ai vostri nemici.
\par 38 E perirete fra le nazioni, e il paese de' vostri nemici vi divorerà.
\par 39 I superstiti fra voi si struggeranno nei paesi de' loro nemici, a motivo delle proprie iniquità; e si struggeranno pure a motivo delle iniquità dei loro padri.
\par 40 E confesseranno la loro iniquità e l'iniquità dei loro padri: l'iniquità delle trasgressioni commesse contro di me e della resistenza oppostami,
\par 41 peccati per i quali anch'io avrò dovuto resister loro, e menarli nel paese de' loro nemici. Ma se allora il cuor loro incirconciso si umilierà, e se accetteranno la punizione della loro iniquità,
\par 42 io mi ricorderò del mio patto con Giacobbe, mi ricorderò del mio patto con Isacco e del mio patto con Abrahamo, e mi ricorderò del paese;
\par 43 poiché il paese sarà abbandonato da loro, e si godrà i suoi sabati mentre rimarrà desolato, senza di loro, ed essi accetteranno la punizione della loro iniquità per aver disprezzato le mie prescrizioni e aver avuto in avversione le mie leggi.
\par 44 E, nonostante tutto questo, quando saranno nel paese dei loro nemici, io non li disprezzerò e non li prenderò in avversione fino al punto d'annientarli del tutto e di rompere il mio patto con loro; poiché io sono l'Eterno, il loro Dio;
\par 45 ma per amor d'essi mi ricorderò del patto stretto coi loro antenati, i quali trassi dal paese d'Egitto, nel cospetto delle nazioni, per essere il loro Dio. Io sono l'Eterno'.
\par 46 Tali sono gli statuti, le prescrizioni e le leggi che l'Eterno stabilì fra sé e i figliuoli d'Israele, sul monte Sinai, per mezzo di Mosè.

\chapter{27}

\par 1 L'Eterno parlò ancora a Mosè, dicendo:
\par 2 'Parla ai figliuoli d'Israele e di' loro: Quand'uno farà un voto concernente delle persone, queste persone apparterranno all'Eterno secondo la valutazione che ne farai.
\par 3 E la tua stima sarà, per un maschio dai venti ai sessant'anni, cinquanta sicli d'argento, secondo il siclo del santuario;
\par 4 se si tratta di una donna, la tua stima sarà di trenta sicli.
\par 5 Dai cinque ai vent'anni, la tua stima sarà di venti sicli per un maschio, e di dieci sicli per una femmina.
\par 6 Da un mese a cinque anni, la tua stima sarà di cinque sicli d'argento per un maschio, e di tre sicli d'argento per una femmina.
\par 7 Dai sessant'anni in su, la tua stima sarà di quindici sicli per un maschio e di dieci sicli per una femmina.
\par 8 E se colui che ha fatto il voto è troppo povero per pagare la somma fissata da te, lo si farà presentare al sacerdote, il quale lo tasserà. Il sacerdote farà una stima, in proporzione dei mezzi di colui che ha fatto il voto.
\par 9 Se si tratta di animali che possono essere presentati come offerta all'Eterno, ogni animale che si darà all'Eterno sarà cosa santa.
\par 10 Non lo si dovrà cambiare; non se ne metterà uno buono al posto di uno cattivo, o uno cattivo al posto di uno buono; e se pure uno sostituisce un animale all'altro, ambedue gli animali saranno cosa sacra.
\par 11 E se si tratta di animali impuri di cui non si può fare offerta all'Eterno, l'animale sarà presentato davanti al sacerdote;
\par 12 e il sacerdote ne farà la stima, secondo che l'animale sarà buono o cattivo; e uno se ne starà alla stima fattane dal sacerdote.
\par 13 Ma se uno lo vuol riscattare, aggiungerà un quinto alla tua stima.
\par 14 Se uno consacra la sua casa per esser cosa santa all'Eterno, il sacerdote ne farà la stima secondo ch'essa sarà buona o cattiva; e uno se ne starà alla stima fattane dal sacerdote.
\par 15 E se colui che ha consacrato la sua casa la vuol riscattare, aggiungerà un quinto al prezzo della stima, e sarà sua.
\par 16 Se uno consacra all'Eterno un pezzo di terra della sua proprietà, ne farai la stima in ragione della sementa: cinquanta sicli d'argento per un omer di seme d'orzo.
\par 17 Se consacra la sua terra dall'anno del giubileo, il prezzo ne resterà fissato secondo la tua stima;
\par 18 ma se la consacra dopo il giubileo, il sacerdote ne valuterà il prezzo in ragione del numero degli anni che rimangono fino al giubileo, e si farà una detrazione dalla tua stima.
\par 19 E se colui che ha consacrato il pezzo di terra lo vuol riscattare, aggiungerà un quinto al prezzo della tua stima, e resterà suo.
\par 20 Ma se non riscatta il pezzo di terra e lo vende ad un altro, non lo si potrà più riscattare;
\par 21 ma quel pezzo di terra, quando rimarrà franco al giubileo, sarà consacrato all'Eterno come una terra interdetta, e diventerà proprietà del sacerdote.
\par 22 Se uno consacra all'Eterno un pezzo di terra ch'egli ha comprato e che non fa parte della sua proprietà,
\par 23 il sacerdote ne valuterà il prezzo secondo la stima fino all'anno del giubileo; e quel tale pagherà il giorno stesso il prezzo fissato, giacché è cosa consacrata all'Eterno.
\par 24 L'anno del giubileo la terra tornerà a colui da cui fu comprata, e del cui patrimonio faceva parte.
\par 25 Tutte le tue stime si faranno in sicli del santuario; il siclo è di venti ghere.
\par 26 Però, nessuno potrà consacrare i primogeniti del bestiame, i quali appartengono già all'Eterno, perché primogeniti: sia un bue, sia un agnello, appartiene all'Eterno.
\par 27 E se si tratta di un animale impuro, lo si riscatterà al prezzo della tua stima, aggiungendovi un quinto; se non è riscattato, sarà venduto al prezzo della tua stima.
\par 28 Nondimeno, tutto ciò che uno avrà consacrato all'Eterno per voto d'interdetto, di fra le cose che gli appartengono, sia che si tratti di una persona, di un animale o di un pezzo di terra del suo patrimonio, non potrà esser né venduto, né riscattato; ogni interdetto è cosa interamente consacrata all'Eterno.
\par 29 Nessuna persona consacrata per voto d'interdetto potrà essere riscattata; dovrà essere messa a morte.
\par 30 Ogni decima della terra, sia delle raccolte del suolo sia dei frutti degli alberi, appartiene all'Eterno; è cosa consacrata all'Eterno.
\par 31 Se uno vuol riscattare una parte della sua decima, vi aggiungerà il quinto.
\par 32 E ogni decima dell'armento o del gregge, il decimo capo di tutto ciò che passa sotto la verga del pastore, sarà consacrata all'Eterno.
\par 33 Non si farà distinzione fra animale buono e cattivo, e non si faranno sostituzioni; e se si sostituisce un animale all'altro, ambedue saranno cosa sacra; non si potranno riscattare'.
\par 34 Questi sono i comandamenti che l'Eterno diede a Mosè per i figliuoli d'Israele, sul monte Sinai.


\end{document}