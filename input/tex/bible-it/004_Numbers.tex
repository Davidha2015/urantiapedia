\begin{document}

\title{Numbers}


\chapter{1}

\par 1 L'Eterno parlò ancora a Mosè, nel deserto di Sinai, nella tenda di convegno, il primo giorno del secondo mese, il secondo anno dell'uscita de' figliuoli d'Israele dal paese d'Egitto, e disse:
\par 2 'Fate la somma di tutta la raunanza de' figliuoli d'Israele secondo le loro famiglie, secondo le case dei loro padri, contando i nomi di tutti i maschi, uno per uno,
\par 3 dall'età di venti anni in su, tutti quelli che in Israele possono andare alla guerra; tu ed Aaronne ne farete il censimento, secondo le loro schiere.
\par 4 E con voi ci sarà un uomo per tribù, il capo della casa de' suoi padri.
\par 5 Questi sono i nomi degli uomini che staranno con voi. Di Ruben: Elitsur, figliuolo di Scedeur;
\par 6 di Simeone: Scelumiel, figliuolo di Tsurishaddai;
\par 7 di Giuda: Nahshon, figliuolo di Aminadab;
\par 8 di Issacar: Nethaneel, figliuolo di Tsuar;
\par 9 di Zabulon: Eliab, figliuolo di Helon;
\par 10 de' figliuoli di Giuseppe: di Efraim: Elishama, figliuolo di Ammihud; di Manasse: Gamaliel, figliuolo di Pedahtsur;
\par 11 di Beniamino: Abidan, figliuolo di Ghideoni;
\par 12 di Dan: Ahizer, figliuolo di Ammishaddai;
\par 13 di Ascer: Paghiel, figliuolo di Ocran;
\par 14 di Gad: Eliasaf, figliuolo di Deuel;
\par 15 di Neftali: Ahira, figliuolo di Enan'.
\par 16 Questi furono i chiamati dal seno della raunanza, i principi delle tribù de' loro padri, i capi delle migliaia d'Israele.
\par 17 Mosè ed Aaronne presero dunque questi uomini ch'erano stati designati per nome,
\par 18 e convocarono tutta la raunanza, il primo giorno del secondo mese; e il popolo fu inscritto secondo le famiglie, secondo le case de' padri, contando il numero delle persone dai venti anni in su, uno per uno.
\par 19 Come l'Eterno gli aveva ordinato, Mosè ne fece il censimento nel deserto di Sinai.
\par 20 Figliuoli di Ruben, primogenito d'Israele, loro discendenti secondo le loro famiglie, secondo le case dei loro padri, contando i nomi di tutti i maschi, uno per uno, dall'età di vent'anni in su, tutti quelli che potevano andare alla guerra:
\par 21 il censimento della tribù di Ruben dette la cifra di quarantaseimilacinquecento.
\par 22 Figliuoli di Simeone, loro discendenti secondo le loro famiglie, secondo le case dei loro padri, inscritti contando i nomi di tutti i maschi, uno per uno, dall'età di vent'anni in su, tutti quelli che potevano andare alla guerra:
\par 23 il censimento della tribù di Simeone dette la cifra di cinquantanovemilatrecento.
\par 24 Figliuoli di Gad, loro discendenti secondo le loro famiglie, secondo le case dei loro padri, contando i nomi dall'età di vent'anni in su, tutti quelli che potevano andare alla guerra:
\par 25 il censimento della tribù di Gad dette la cifra di quarantacinquemilaseicentocinquanta.
\par 26 Figliuoli di Giuda, loro discendenti secondo le loro famiglie, secondo le case dei loro padri, contando i nomi dall'età di vent'anni in su, tutti quelli che potevano andare alla guerra:
\par 27 il censimento della tribù di Giuda dette la cifra di settantaquattromilaseicento.
\par 28 Figliuoli di Issacar, loro discendenti secondo le loro famiglie, secondo le case dei loro padri, contando i nomi dall'età di vent'anni in su, tutti quelli che potevano andare alla guerra:
\par 29 il censimento della tribù di Issacar dette la cifra di cinquantaquattromilaquattrocento.
\par 30 Figliuoli di Zabulon, loro discendenti secondo le loro famiglie, secondo le case dei loro padri, contando i nomi dall'età di vent'anni in su, tutti quelli che potevano andare alla guerra:
\par 31 il censimento della tribù di Zabulon dette la cifra di cinquantasettemilaquattrocento.
\par 32 Figliuoli di Giuseppe: Figliuoli d'Efraim, loro discendenti secondo le loro famiglie, secondo le case dei loro padri, contando i nomi dall'età di vent'anni in su, tutti quelli che potevano andare alla guerra:
\par 33 il censimento della tribù di Efraim dette la cifra di quarantamilacinquecento.
\par 34 Figliuoli di Manasse, loro discendenti secondo le loro famiglie, secondo le case dei loro padri, contando i nomi dall'età di vent'anni in su, tutti quelli che potevano andare alla guerra:
\par 35 il censimento della tribù di Manasse dette la cifra di trentaduemiladuecento.
\par 36 Figliuoli di Beniamino, loro discendenti secondo le loro famiglie, secondo le case dei loro padri, contando i nomi dall'età di vent'anni in su, tutti quelli che potevano andare alla guerra:
\par 37 il censimento della tribù di Beniamino dette la cifra di trentacinquemilaquattrocento.
\par 38 Figliuoli di Dan, loro discendenti secondo le loro famiglie, secondo le case dei loro padri, contando i nomi dall'età di vent'anni in su, tutti quelli che potevano andare alla guerra:
\par 39 il censimento della tribù di Dan dette la cifra di sessantaduemilasettecento.
\par 40 Figliuoli di Ascer, loro discendenti secondo le loro famiglie, secondo le case dei loro padri, contando i nomi dall'età di vent'anni in su, tutti quelli che potevano andare alla guerra:
\par 41 il censimento della tribù di Ascer dette la cifra di quarantunmilacinquecento.
\par 42 Figliuoli di Neftali, loro discendenti secondo le loro famiglie, secondo le case dei loro padri, contando i nomi dall'età di vent'anni in su, tutti quelli che potevano andare alla guerra:
\par 43 il censimento della tribù di Neftali dette la cifra di cinquantatremilaquattrocento.
\par 44 Questi son quelli di cui Mosè ed Aaronne fecero il censimento, coi dodici uomini, principi d'Israele: ce n'era uno per ognuna delle case de' loro padri.
\par 45 Così tutti i figliuoli d'Israele dei quali fu fatto il censimento secondo le case dei loro padri, dall'età di vent'anni in su, cioè tutti gli uomini che in Israele potevano andare alla guerra,
\par 46 tutti quelli dei quali fu fatto il censimento, furono seicentotremilacinquecentocinquanta.
\par 47 Ma i Leviti, come tribù dei loro padri, non furon compresi nel censimento con gli altri;
\par 48 poiché l'Eterno avea parlato a Mosè, dicendo:
\par 49 'Soltanto della tribù di Levi non farai il censimento, e non ne unirai l'ammontare a quello de' figliuoli d'Israele;
\par 50 ma affida ai Leviti la cura del tabernacolo della testimonianza, di tutti i suoi utensili e di tutto ciò che gli appartiene. Essi porteranno il tabernacolo e tutti i suoi utensili, ne faranno il servizio, e staranno accampati attorno al tabernacolo.
\par 51 Quando il tabernacolo dovrà partire, i Leviti lo smonteranno; quando il tabernacolo dovrà accamparsi in qualche luogo, i Leviti lo rizzeranno; e l'estraneo che gli si avvicinerà sarà messo a morte.
\par 52 I figliuoli d'Israele pianteranno le loro tende ognuno nel suo campo, ognuno vicino alla sua bandiera, secondo le loro schiere.
\par 53 Ma i Leviti pianteranno le loro attorno al tabernacolo della testimonianza, affinché non si accenda l'ira mia contro la raunanza dei figliuoli d'Israele; e i Leviti avranno la cura del tabernacolo della testimonianza'.
\par 54 I figliuoli d'Israele si conformarono in tutto agli ordini che l'Eterno avea dato a Mosè; fecero così.

\chapter{2}

\par 1 L'Eterno parlò ancora a Mosè e ad Aaronne, dicendo:
\par 2 'I figliuoli d'Israele s'accamperanno ciascuno vicino alla sua bandiera sotto le insegne delle case dei loro padri; si accamperanno di faccia e tutt'intorno alla tenda di convegno.
\par 3 Sul davanti, verso oriente, s'accamperà la bandiera del campo di Giuda con le sue schiere:
\par 4 il principe de' figliuoli di Giuda è Nahshon, figliuolo di Aminadab, e il suo corpo, secondo il censimento, è di settantaquattromilaseicento uomini.
\par 5 Accanto a lui s'accamperà la tribù di Issacar; il principe dei figliuoli di Issacar è Nethaneel, figliuolo di Tsuar,
\par 6 e il suo corpo, secondo il censimento, è di cinquantaquattromilaquattrocento uomini.
\par 7 Poi la tribù di Zabulon; il principe dei figliuoli di Zabulon è Eliab, figliuolo di Helon, e il suo corpo,
\par 8 secondo il censimento, è di cinquantasettemilaquattrocento uomini.
\par 9 Il totale del censimento del campo di Giuda è dunque centottantaseimilaquattrocento uomini, secondo le loro schiere. Si metteranno in marcia i primi.
\par 10 A mezzogiorno starà la bandiera del campo di Ruben con le sue schiere; il principe de' figliuoli di Ruben è Elitsur, figliuolo di Scedeur,
\par 11 e il suo corpo, secondo il censimento, è di quarantaseimilacinquecento uomini.
\par 12 Accanto a lui s'accamperà la tribù di Simeone; il principe de' figliuoli di Simeone è Scelumiel, figliuolo di Tsurishaddai,
\par 13 e il suo corpo, secondo il censimento, è di cinquantanovemilatrecento uomini.
\par 14 Poi la tribù di Gad; il principe de' figliuoli di Gad è Eliasaf, figliuolo di Reuel,
\par 15 e il suo corpo, secondo il censimento, è di quarantacinquemilaseicentocinquanta uomini.
\par 16 Il totale del censimento del campo di Ruben è dunque centocinquantunmila e quattrocentocinquanta uomini, secondo le loro schiere. Si metteranno in marcia in seconda linea.
\par 17 Poi si metterà in marcia la tenda di convegno col campo dei Leviti in mezzo agli altri campi. Seguiranno nella marcia l'ordine nel quale erano accampati, ciascuno al suo posto, con la sua bandiera.
\par 18 Ad occidente starà la bandiera del campo di Efraim con le sue schiere; il principe de' figliuoli di Efraim è Elishama,
\par 19 figliuolo di Ammihud, e il suo corpo, secondo il censimento, è di quarantamilacinquecento uomini.
\par 20 Accanto a lui s'accamperà la tribù di Manasse; il principe de' figliuoli di Manasse è Gamaliel, figliuolo di Pedahtsur,
\par 21 e il suo corpo, secondo il censimento, è di trentaduemiladuecento uomini.
\par 22 Poi la tribù di Beniamino; il principe dei figliuoli di Beniamino è Abidan, figliuolo di Ghideoni,
\par 23 e il suo corpo, secondo il censimento, è di trentacinquemilaquattrocento uomini.
\par 24 Il totale del censimento del campo d'Efraim è dunque centottantamilacento uomini, secondo le loro schiere. Si metteranno in marcia in terza linea.
\par 25 A settentrione starà il campo di Dan con le sue schiere; il principe de' figliuoli di Dan è Ahiezer, figliuolo di Ammishaddai,
\par 26 e il suo campo, secondo il censimento, è di sessantaduemilasettecento uomini.
\par 27 Accanto a lui s'accamperà la tribù di Ascer; il principe de' figliuoli di Ascer è Paghiel, figliuolo d'Ocran,
\par 28 e il suo campo, secondo il censimento, è di quarantunmilacinquecento uomini.
\par 29 Poi la tribù di Neftali; il principe de' figliuoli di Neftali è Ahira, figliuolo di Enan,
\par 30 e il suo campo, secondo il censimento, è di cinquantatremilaquattrocento uomini.
\par 31 Il totale del censimento del campo di Dan è dunque centocinquantasettemilaseicento. Si metteranno in marcia gli ultimi, secondo le loro bandiere'.
\par 32 Questi furono i figliuoli d'Israele de' quali si fece il censimento secondo le case dei loro padri. Tutti gli uomini de' quali si fece il censimento, e che formarono i campi, secondo i loro corpi, furono seicentotremilacinquecentocinquanta.
\par 33 Ma i Leviti, secondo l'ordine che l'Eterno avea dato a Mosè, non furono compresi nel censimento coi figliuoli d'Israele.
\par 34 E i figliuoli d'Israele si conformarono in tutto agli ordini che l'Eterno avea dati a Mosè: così s'accampavano secondo le loro bandiere, e così si mettevano in marcia, ciascuno secondo la sua famiglia, secondo la casa dei suoi padri.

\chapter{3}

\par 1 Or questi sono i discendenti di Aaronne e di Mosè nel tempo in cui l'Eterno parlò a Mosè sul monte Sinai.
\par 2 Questi sono i nomi dei figliuoli di Aaronne: Nadab, il primogenito, Abihu, Eleazar e Ithamar.
\par 3 Tali i nomi dei figliuoli d'Aaronne, che ricevettero l'unzione come sacerdoti e furon consacrati per esercitare il sacerdozio.
\par 4 Nadab e Abihu morirono davanti all'Eterno quand'offrirono fuoco straniero davanti all'Eterno, nel deserto di Sinai. Essi non aveano figliuoli, ed Eleazar e Ithamar esercitarono il sacerdozio in presenza d'Aaronne, loro padre.
\par 5 E l'Eterno parlò a Mosè, dicendo:
\par 6 'Fa' avvicinare la tribù de' Leviti e ponila davanti al sacerdote Aaronne, affinché sia al suo servizio.
\par 7 Essi avranno la cura di tutto ciò che è affidato a lui e a tutta la raunanza davanti alla tenda di convegno e faranno così il servizio del tabernacolo.
\par 8 Avranno cura di tutti gli utensili della tenda di convegno e di quanto è affidato ai figliuoli d'Israele, e faranno così il servizio del tabernacolo.
\par 9 Tu darai i Leviti ad Aaronne e ai suoi figliuoli; essi gli sono interamente dati di tra i figliuoli d'Israele.
\par 10 Tu stabilirai Aaronne e i suoi figliuoli, perché esercitino le funzioni del loro sacerdozio; lo straniero che s'accosterà all'altare sarà messo a morte'.
\par 11 E l'Eterno parlò a Mosè, dicendo:
\par 12 'Ecco, io ho preso i Leviti di tra i figliuoli d'Israele in luogo d'ogni primogenito che apre il seno materno fra i figliuoli d'Israele; e i Leviti saranno miei;
\par 13 poiché ogni primogenito è mio; il giorno ch'io colpii tutti i primogeniti nel paese d'Egitto, io mi consacrai tutti i primi parti in Israele, tanto degli uomini quanto degli animali; saranno miei: io sono l'Eterno'.
\par 14 E l'Eterno parlò a Mosè nel deserto di Sinai, dicendo:
\par 15 'Fa' il censimento de' figliuoli di Levi secondo le case de' loro padri, secondo le loro famiglie; farai il censimento di tutti i maschi dall'età d'un mese in su'.
\par 16 E Mosè ne fece il censimento secondo l'ordine dell'Eterno, come gli era stato comandato di fare.
\par 17 Questi sono i figliuoli di Levi, secondo i loro nomi: Gherson, Kehath e Merari.
\par 18 Questi i nomi dei figliuoli di Gherson, secondo le loro famiglie: Libni e Scimei.
\par 19 E i figliuoli di Kehath, secondo le loro famiglie: Amram, Jitsehar, Hebron e Uzziel.
\par 20 E i figliuoli di Merari secondo le loro famiglie: Mahli e Musci. Queste sono le famiglie dei Leviti, secondo le case de' loro padri.
\par 21 Da Gherson discendono la famiglia dei Libniti e la famiglia dei Scimeiti, che formano le famiglie dei Ghersoniti.
\par 22 Quelli dei quali fu fatto il censimento, contando tutti i maschi dall'età di un mese in su, furono settemilacinquecento.
\par 23 Le famiglie dei Ghersoniti avevano il campo dietro il tabernacolo, a occidente.
\par 24 Il principe della casa de' padri dei Ghersoniti era Eliasaf, figliuolo di Lael.
\par 25 Per quel che concerne la tenda di convegno, i figliuoli di Gherson doveano aver cura del tabernacolo e della tenda, della sua coperta, della portiera all'ingresso della tenda di convegno,
\par 26 delle tele del cortile e della portiera dell'ingresso del cortile, tutt'intorno al tabernacolo e all'altare, e dei suoi cordami per tutto il servizio del tabernacolo.
\par 27 Da Kehath discendono la famiglia degli Amramiti, la famiglia degli Hitsehariti, la famiglia degli Hebroniti e la famiglia degli Uzzieliti, che formano le famiglie dei Kehathiti.
\par 28 Contando tutti i maschi dall'età di un mese in su, furono ottomilaseicento, incaricati della cura del santuario.
\par 29 Le famiglie dei figliuoli di Kehath avevano il campo al lato meridionale del tabernacolo.
\par 30 Il principe della casa dei padri dei Kehathiti era Elitsafan, figliuolo di Uzziel.
\par 31 Alle loro cure erano affidati l'arca, la tavola, il candelabro, gli altari e gli utensili del santuario coi quali si fa il servizio, il velo e tutto ciò che si riferisce al servizio del santuario.
\par 32 Il principe dei principi dei Leviti era Eleazar, figliuolo del sacerdote Aaronne; egli aveva la sorveglianza di quelli ch'erano incaricati della cura del santuario.
\par 33 Da Merari discendono la famiglia dei Mahliti e la famiglia dei Musciti, che formano le famiglie di Merari.
\par 34 Quelli di cui si fece il censimento, contando tutti i maschi dall'età di un mese in su, furono seimiladuecento.
\par 35 Il principe della casa de' padri delle famiglie di Merari era Tsuriel, figliuolo di Abihail. Essi aveano il campo dal lato settentrionale del tabernacolo.
\par 36 Alle cure dei figliuoli di Merari furono affidati le tavole del tabernacolo, le sue traverse, le sue colonne e le loro basi, tutti i suoi utensili e tutto ciò che si riferisce al servizio del tabernacolo,
\par 37 le colonne del cortile tutt'intorno, le loro basi, i loro piuoli e il loro cordame.
\par 38 Sul davanti del tabernacolo, a oriente, di faccia alla tenda di convegno, verso il sol levante, avevano il campo Mosè, Aaronne e i suoi figliuoli; essi aveano la cura del santuario in luogo de' figliuoli d'Israele; lo straniero che vi si fosse accostato sarebbe stato messo a morte.
\par 39 Tutti i Leviti di cui Mosè ed Aaronne fecero il censimento secondo le loro famiglie per ordine dell'Eterno, tutti i maschi dall'età di un mese in su, furono ventiduemila.
\par 40 E l'Eterno disse a Mosè: 'Fa' il censimento di tutti i primogeniti maschi tra i figliuoli d'Israele dall'età di un mese in su e fa' il conto dei loro nomi.
\par 41 Prenderai i Leviti per me - io sono l'Eterno - invece di tutti i primogeniti de' figliuoli d'Israele, e il bestiame dei Leviti in luogo dei primi parti del bestiame de' figliuoli d'Israele'.
\par 42 E Mosè fece il censimento di tutti i primogeniti tra i figliuoli d'Israele, secondo l'ordine che l'Eterno gli avea dato.
\par 43 Tutti i primogeniti maschi di cui si fece il censimento, contando i nomi dall'età di un mese in su, furono ventiduemiladuecentosettantatre.
\par 44 E l'Eterno parlò a Mosè, dicendo:
\par 45 'Prendi i Leviti in luogo di tutti i primogeniti dei figliuoli d'Israele, e il bestiame dei Leviti in luogo del loro bestiame; e i Leviti saranno miei. Io sono l'Eterno.
\par 46 Per il riscatto dei duecentosettantatre primogeniti dei figliuoli d'Israele che oltrepassano il numero dei Leviti,
\par 47 prenderai cinque sicli a testa, li prenderai secondo il siclo del santuario, che è di venti ghere.
\par 48 Darai il danaro ad Aaronne e ai suoi figliuoli per il riscatto di quelli che oltrepassano il numero dei Leviti'.
\par 49 E Mosè prese il danaro per il riscatto di quelli che oltrepassavano il numero dei primogeniti riscattati dai Leviti;
\par 50 prese il danaro dai primogeniti dei figliuoli d'Israele: milletrecentosessantacinque sicli, secondo il siclo del santuario.
\par 51 E Mosè dette il danaro del riscatto ad Aaronne e ai suoi figliuoli, secondo l'ordine dell'Eterno, come l'Eterno aveva ordinato a Mosè.

\chapter{4}

\par 1 L'Eterno parlò ancora a Mosè e ad Aaronne, dicendo:
\par 2 'Fate il conto dei figliuoli di Kehath, tra i figliuoli di Levi, secondo le loro famiglie, secondo le case dei loro padri,
\par 3 dall'età di trent'anni in su fino all'età di cinquant'anni, di tutti quelli che possono assumere un ufficio per far l'opera nella tenda di convegno.
\par 4 Questo è il servizio che i figliuoli di Kehath avranno a fare nella tenda di convegno, e che concerne le cose santissime.
\par 5 Quando il campo si moverà, Aaronne e i suoi figliuoli verranno a smontare il velo di separazione, e copriranno con esso l'arca della testimonianza;
\par 6 poi porranno sull'arca una coperta di pelli di delfino, vi stenderanno sopra un panno tutto di stoffa violacea e vi metteranno al posto le stanghe.
\par 7 Poi stenderanno un panno violaceo sulla tavola dei pani della presentazione, e vi metteranno su i piatti, le coppe, i bacini, i calici per le libazioni; e vi sarà su anche il pane perpetuo;
\par 8 e su queste cose stenderanno un panno scarlatto, e sopra questo una coperta di pelli di delfino, e metteranno le stanghe alla tavola.
\par 9 Poi prenderanno un panno violaceo, col quale copriranno il candelabro, le sue lampade, le sue forbici, i suoi smoccolatoi e tutti i suoi vasi dell'olio destinati al servizio del candelabro;
\par 10 metteranno il candelabro con tutti i suoi utensili in una coperta di pelli di delfino, e lo porranno sopra un paio di stanghe.
\par 11 Poi stenderanno sull'altare d'oro un panno violaceo, e sopra questo una coperta di pelli di delfino; e metteranno le stanghe all'altare.
\par 12 E prenderanno tutti gli utensili di cui si fa uso per il servizio nel santuario, li metteranno in un panno violaceo, li avvolgeranno in una coperta di pelli di delfino e li porranno sopra un paio di stanghe.
\par 13 Poi toglieranno le ceneri dall'altare, e stenderanno sull'altare un panno scarlatto;
\par 14 vi metteranno su tutti gli utensili destinati al suo servizio, i bracieri, i forchettoni, le palette, i bacini, tutti gli utensili dell'altare, e vi stenderanno su una coperta di pelli di delfino; poi porranno le stanghe all'altare.
\par 15 E dopo che Aaronne e i suoi figliuoli avranno finito di coprire il santuario e tutti gli arredi del santuario, quando il campo si moverà, i figliuoli di Kehath verranno per portar quelle cose; ma non toccheranno le cose sante, che non abbiano a morire. Queste sono le incombenze de' figliuoli di Kehath nella tenda di convegno.
\par 16 Ed Eleazar, figliuolo del sacerdote Aaronne, avrà l'incarico dell'olio per il candelabro, del profumo fragrante, dell'offerta perpetua e dell'olio dell'unzione, e l'incarico di tutto il tabernacolo e di tutto ciò che contiene, del santuario e dei suoi arredi'.
\par 17 Poi l'Eterno parlò a Mosè e ad Aaronne dicendo:
\par 18 'Badate che la tribù delle famiglie dei Kehathiti non abbia ad essere sterminata di tra i Leviti;
\par 19 ma fate questo per loro, affinché vivano e non muoiano quando si accosteranno al luogo santissimo: Aaronne e i suoi figliuoli vengano e assegnino a ciascun d'essi il proprio servizio e il proprio incarico.
\par 20 E non entrino quelli a guardare anche per un istante le cose sante, onde non muoiano'.
\par 21 L'Eterno parlò ancora a Mosè, dicendo:
\par 22 'Fa' il conto anche dei figliuoli di Gherson, secondo le case dei loro padri, secondo le loro famiglie.
\par 23 Farai il censimento, dall'età di trent'anni in su fino all'età di cinquant'anni, di tutti quelli che possono assumere un ufficio per far l'opera nella tenda di convegno.
\par 24 Questo è il servizio delle famiglie dei Ghersoniti: quel che debbono fare e quello che debbono portare:
\par 25 porteranno i teli del tabernacolo e la tenda di convegno, la sua coperta, la coperta di pelli di delfino che v'è sopra, e la portiera all'ingresso della tenda di convegno;
\par 26 le cortine del cortile con la portiera dell'ingresso del cortile, cortine che stanno tutt'intorno al tabernacolo e all'altare, i loro cordami e tutti gli utensili destinati al loro servizio; faranno tutto il servizio che si riferisce a queste cose.
\par 27 Tutto il servizio dei figliuoli dei Ghersoniti sarà sotto gli ordini di Aaronne e dei suoi figliuoli per tutto quello che dovranno portare e per tutto quello che dovranno fare; voi affiderete alla loro cura tutto quello che debbon portare.
\par 28 Tale è il servizio delle famiglie dei figliuoli dei Ghersoniti nella tenda di convegno; e l'incarico loro sarà eseguito agli ordini di Ithamar figliuolo del sacerdote Aaronne.
\par 29 Farai il censimento dei figliuoli di Merari secondo le loro famiglie, secondo le case dei loro padri;
\par 30 farai il censimento, dall'età di trent'anni in su fino all'età di cinquant'anni, di tutti quelli che possono assumere un ufficio per far l'opera nella tenda di convegno.
\par 31 Questo è quanto è affidato alle loro cure e quello che debbono portare, in conformità di tutto il loro servizio nella tenda di convegno: le assi del tabernacolo, le sue traverse, le sue colonne, le sue basi;
\par 32 le colonne che sono intorno al cortile, le loro basi, i loro piuoli, i loro cordami, tutti i loro utensili e tutto il servizio che vi si riferisce. Farete l'inventario nominativo degli oggetti affidati alle loro cure e ch'essi dovranno portare.
\par 33 Tale è il servizio delle famiglie dei figliuoli di Merari, tutto il loro servizio nella tenda di convegno, sotto gli ordini di Ithamar, figliuolo del sacerdote Aaronne'.
\par 34 Mosè, Aaronne e i principi della raunanza fecero dunque il censimento dei figliuoli dei Kehathiti secondo le loro famiglie e secondo le case dei loro padri,
\par 35 di tutti quelli che dall'età di trent'anni in su fino all'età di cinquant'anni potevano assumere un ufficio per far l'opera nella tenda di convegno.
\par 36 E quelli di cui si fece il censimento secondo le loro famiglie, furono duemilasettecentocinquanta.
\par 37 Questi son quelli delle famiglie dei Kehathiti dei quali si fece il censimento: tutti quelli che esercitavano un qualche ufficio nella tenda di convegno; Mosè ed Aaronne ne fecero il censimento secondo l'ordine che l'Eterno avea dato per mezzo di Mosè.
\par 38 I figliuoli di Gherson, di cui si fece il censimento secondo le loro famiglie e secondo le case dei loro padri,
\par 39 dall'età di trent'anni in su fino all'età di cinquant'anni, tutti quelli che potevano assumere un ufficio per far l'opera nella tenda di convegno,
\par 40 quelli di cui si fece il censimento secondo le loro famiglie, secondo le case dei loro padri, furono duemilaseicentotrenta.
\par 41 Questi son quelli delle famiglie dei figliuoli di Gherson, di cui si fece il censimento: tutti quelli che esercitavano un qualche ufficio nella tenda di convegno; Mosè ed Aaronne ne fecero il censimento secondo l'ordine dell'Eterno.
\par 42 Quelli delle famiglie dei figliuoli di Merari dei quali si fece il censimento secondo le loro famiglie, secondo le famiglie dei loro padri,
\par 43 dall'età di trent'anni in su fino all'età di cinquant'anni, tutti quelli che potevano assumere un ufficio per far l'opera nella tenda di convegno
\par 44 quelli di cui si fece il censimento secondo le loro famiglie, furono tremiladuecento.
\par 45 Questi son quelli delle famiglie dei figliuoli di Merari, di cui si fece il censimento; Mosè ed Aaronne ne fecero il censimento secondo l'ordine che l'Eterno avea dato per mezzo di Mosè.
\par 46 Tutti i Leviti dei quali Mosè, Aaronne e i principi d'Israele fecero il censimento secondo le loro famiglie e secondo le case dei loro padri,
\par 47 dall'età di trent'anni in su fino all'età di cinquant'anni, tutti quelli che potevano assumere l'ufficio di servitori e l'ufficio di portatori nella tenda di convegno,
\par 48 tutti quelli di cui si fece il censimento, furono ottomilacinquecentottanta.
\par 49 Ne fu fatto il censimento secondo l'ordine che l'Eterno avea dato per mezzo di Mosè, assegnando a ciascuno il servizio che dovea fare e quello che dovea portare. Così ne fu fatto il censimento come l'Eterno aveva ordinato a Mosè.

\chapter{5}

\par 1 Poi l'Eterno parlò a Mosè, dicendo:
\par 2 'Ordina ai figliuoli d'Israele che mandino fuori del campo ogni lebbroso, chiunque ha la gonorrea o è impuro per il contatto con un morto.
\par 3 Maschi o femmine che siano, li manderete fuori; li manderete fuori del campo perché non contaminino il loro campo in mezzo al quale io abito'.
\par 4 I figliuoli d'Israele fecero così, e li mandarono fuori del campo. Come l'Eterno avea detto a Mosè, così fecero i figliuoli d'Israele.
\par 5 L'Eterno parlò ancora a Mosè, dicendo:
\par 6 'Di' ai figliuoli d'Israele: Quando un uomo o una donna avrà fatto un torto a qualcuno commettendo una infedeltà rispetto all'Eterno, e questa persona si sarà così resa colpevole,
\par 7 ella confesserà il peccato commesso, restituirà per intero il corpo del delitto, aggiungendovi in più un quinto, e lo darà a colui verso il quale si è resa colpevole.
\par 8 Ma se questi non ha prossimo parente a cui si possa restituire il corpo del delitto, questo corpo del delitto restituito spetterà all'Eterno, cioè al sacerdote, oltre al montone espiatorio, mediante il quale si farà l'espiazione per il colpevole.
\par 9 Ogni offerta elevata di tutte le cose consacrate che i figliuoli d'Israele presenteranno al sacerdote, sarà del sacerdote;
\par 10 le cose che uno consacrerà saranno del sacerdote; ciò che uno darà al sacerdote, apparterrà a lui'.
\par 11 L'Eterno parlò ancora a Mosè, dicendo:
\par 12 'Parla ai figliuoli d'Israele, e di' loro: Se una donna si svia dal marito e commette una infedeltà contro di lui;
\par 13 se uno ha relazioni carnali con lei e la cosa è nascosta agli occhi del marito; s'ella si è contaminata in segreto senza che vi sian testimoni contro di lei o ch'ella sia stata colta sul fatto,
\par 14 ove lo spirito di gelosia s'impossessi del marito e questi diventi geloso della moglie che si è contaminata, ovvero lo spirito di gelosia s'impossessi di lui e questi diventi geloso della moglie che non si è contaminata,
\par 15 quell'uomo menerà la moglie al sacerdote, e porterà un'offerta per lei: un decimo d'efa di farina d'orzo; non vi spanderà sopra olio né vi metterà sopra incenso, perché è un'oblazione di gelosia, un'oblazione commemorativa, destinata a ricordare una iniquità.
\par 16 Il sacerdote farà avvicinare la donna, e la farà stare in piè davanti all'Eterno.
\par 17 Poi il sacerdote prenderà dell'acqua santa in un vaso di terra; prenderà pure della polvere ch'è sul suolo del tabernacolo, e la metterà nell'acqua.
\par 18 Il sacerdote farà quindi stare la donna in piè davanti all'Eterno, le scoprirà il capo e porrà in mano di lei l'oblazione commemorativa, ch'è l'oblazione di gelosia; e il sacerdote avrà in mano l'acqua amara che arreca maledizione.
\par 19 Il sacerdote farà giurare quella donna, e le dirà: Se nessun uomo ha dormito teco, e se non ti sei sviata per contaminarti ricevendo un altro invece del tuo marito, quest'acqua amara che arreca maledizione, non ti faccia danno!
\par 20 Ma se tu ti sei sviata ricevendo un altro invece del tuo marito e ti sei contaminata, e altri che il tuo marito ha dormito teco... -
\par 21 allora il sacerdote farà giurare la donna con un giuramento d'imprecazione e le dirà: L'Eterno faccia di te un oggetto di maledizione e di esecrazione fra il tuo popolo, facendoti smagrire i fianchi e gonfiare il ventre;
\par 22 e quest'acqua che arreca maledizione, t'entri nelle viscere per farti gonfiare il ventre e smagrire i fianchi! E la donna dirà: Amen! amen!
\par 23 Poi il sacerdote scriverà queste imprecazioni in un rotolo, e le cancellerà con l'acqua amara.
\par 24 Farà bere alla donna quell'acqua amara che arreca maledizione, e l'acqua che arreca maledizione entrerà in lei per produrle amarezza;
\par 25 e il sacerdote prenderà dalle mani della donna l'oblazione di gelosia, agiterà l'oblazione davanti all'Eterno, e l'offrirà sull'altare;
\par 26 e il sacerdote prenderà una manata di quell'oblazione come ricordanza, e la farà fumare sull'altare; poi farà bere l'acqua alla donna.
\par 27 E quando le avrà fatto bere l'acqua, avverrà che, s'ella si è contaminata ed ha commesso una infedeltà contro il marito, l'acqua che arreca maledizione entrerà in lei per produrre amarezza; il ventre le si gonfierà, i suoi fianchi smagriranno, e quella donna diventerà un oggetto di maledizione in mezzo al suo popolo.
\par 28 Ma se la donna non si è contaminata ed è pura, sarà riconosciuta innocente, ed avrà de' figliuoli.
\par 29 Questa è la legge relativa alla gelosia, per il caso in cui la moglie di uno si svii ricevendo un altro invece del suo marito, e si contamini,
\par 30 e per il caso in cui lo spirito di gelosia s'impossessi del marito, e questi diventi geloso della moglie; egli farà comparire sua moglie davanti all'Eterno, e il sacerdote le applicherà questa legge integralmente.
\par 31 Il marito sarà immune da colpa, ma la donna porterà la pena della sua iniquità'.

\chapter{6}

\par 1 L'Eterno parlò ancora a Mosè, dicendo:
\par 2 'Parla ai figliuoli d'Israele e di' loro: Quando un uomo o una donna avrà fatto un voto speciale, il voto di nazireato,
\par 3 per consacrarsi all'Eterno, si asterrà dal vino e dalle bevande alcooliche; non berrà aceto fatto di vino, né aceto fatto di bevanda alcoolica; non berrà liquori tratti dall'uva, e non mangerà uva, né fresca né secca.
\par 4 Tutto il tempo del suo nazireato non mangerà alcun prodotto della vigna, dagli acini alla buccia.
\par 5 Tutto il tempo del suo voto di nazireato il rasoio non passerà sul suo capo; fino a che non sian compiuti i giorni per i quali ei s'è consacrato all'Eterno, sarà santo; si lascerà crescer liberamente i capelli sul capo.
\par 6 Tutto il tempo ch'ei s'è consacrato all'Eterno, non si accosterà a corpo morto;
\par 7 si trattasse anche di suo padre, di sua madre, di suo fratello e della sua sorella, non si contaminerà per loro alla loro morte, perché porta sul capo il segno della sua consacrazione a Dio.
\par 8 Tutto il tempo del suo nazireato egli è consacrato all'Eterno.
\par 9 E se uno gli muore accanto improvvisamente, e il suo capo consacrato rimane così contaminato, si raderà il capo il giorno della sua purificazione; se lo raderà il settimo giorno;
\par 10 l'ottavo giorno porterà due tortore o due giovani piccioni al sacerdote, all'ingresso della tenda di convegno.
\par 11 E il sacerdote ne offrirà uno come sacrifizio per il peccato e l'altro come olocausto, e farà per lui l'espiazione del peccato che ha commesso a cagion di quel morto; e, in quel giorno stesso, il nazireo consacrerà così il suo capo.
\par 12 Consacrerà di nuovo all'Eterno i giorni del suo nazireato, e offrirà un agnello dell'anno come sacrifizio di riparazione; i giorni precedenti non saranno contati, perché il suo nazireato è stato contaminato.
\par 13 Questa è la legge del nazireato: quando i giorni del suo nazireato saranno compiuti, lo si farà venire all'ingresso della tenda di convegno;
\par 14 ed egli presenterà la sua offerta all'Eterno: un agnello dell'anno, senza difetto, per l'olocausto; una pecora dell'anno, senza difetto, per il sacrifizio per il peccato, e un montone senza difetto, per il sacrifizio di azioni di grazie;
\par 15 un paniere di pani azzimi fatti con fior di farina, di focacce intrise con olio, di gallette senza lievito unte d'olio, insieme con l'oblazione e le libazioni relative.
\par 16 Il sacerdote presenterà quelle cose davanti all'Eterno, e offrirà il suo sacrifizio per il peccato e il suo olocausto;
\par 17 offrirà il montone come sacrifizio di azioni di grazie all'Eterno, col paniere dei pani azzimi; il sacerdote offrirà pure l'oblazione e la libazione.
\par 18 Il nazireo raderà, all'ingresso della tenda di convegno, il suo capo consacrato; prenderà i capelli del suo capo consacrato e li metterà sul fuoco che è sotto il sacrifizio di azioni di grazie.
\par 19 Il sacerdote prenderà la spalla del montone, quando sarà cotta, una focaccia non lievitata del paniere, una galletta senza lievito, e le porrà nelle mani del nazireo, dopo che questi avrà raso il suo capo consacrato.
\par 20 Il sacerdote le agiterà, come offerta agitata, davanti all'Eterno; è cosa santa che appartiene al sacerdote, assieme al petto dell'offerta agitata e alla spalla dell'offerta elevata. Dopo questo, il nazireo potrà bere del vino.
\par 21 Tale è la legge relativa a colui che ha fatto voto di nazireato, tale è la sua offerta all'Eterno per il suo nazireato, oltre quello che i suoi mezzi gli permetteranno di fare. Egli agirà secondo il voto che avrà fatto, conformemente alla legge del suo nazireato'.
\par 22 L'Eterno parlò ancora a Mosè, dicendo:
\par 23 'Parla ad Aaronne e ai suoi figliuoli, e di' loro: Voi benedirete così i figliuoli d'Israele; direte loro:
\par 24 L'Eterno ti benedica e ti guardi!
\par 25 L'Eterno faccia risplendere il suo volto su te e ti sia propizio!
\par 26 L'Eterno volga verso te il suo volto, e ti dia la pace!
\par 27 Così metteranno il mio nome sui figliuoli d'Israele, e io li benedirò'.

\chapter{7}

\par 1 Il giorno che Mosè ebbe finito di rizzare il tabernacolo e l'ebbe unto e consacrato con tutti i suoi utensili, quando ebbe rizzato l'altare con tutti i suoi utensili, e li ebbe unti e consacrati,
\par 2 i principi d'Israele, capi delle case de' loro padri, che erano i principi delle tribù ed aveano presieduto al censimento, presentarono un'offerta
\par 3 e la portarono davanti all'Eterno: sei carri-lettiga e dodici buoi; vale a dire un carro per due principi e un bove per ogni principe; e li offrirono davanti al tabernacolo.
\par 4 E l'Eterno parlò a Mosè, dicendo:
\par 5 'Prendili da loro per impiegarli al servizio della tenda di convegno, e dalli ai Leviti; a ciascuno secondo le sue funzioni'.
\par 6 Mosè prese dunque i carri e i buoi, e li dette ai Leviti.
\par 7 Dette due carri e quattro buoi ai figliuoli di Gherson, secondo le loro funzioni,
\par 8 dette quattro carri e otto buoi ai figliuoli di Merari, secondo le loro funzioni, sotto la sorveglianza d'Ithamar, figliuolo del sacerdote Aaronne;
\par 9 ma ai figliuoli di Kehath non ne diede punti, perché avevano il servizio degli oggetti sacri e doveano portarli sulle spalle.
\par 10 E i principi presentarono la loro offerta per la dedicazione dell'altare, il giorno ch'esso fu unto; i principi presentarono la loro offerta davanti all'altare.
\par 11 E l'Eterno disse a Mosè: 'I principi presenteranno la loro offerta uno per giorno, per la dedicazione dell'altare'.
\par 12 Colui che presentò la sua offerta il primo giorno fu Nahshon, figliuolo d'Amminadab della tribù di Giuda;
\par 13 e la sua offerta fu un piatto d'argento del peso di centotrenta sicli, un bacino d'argento di settanta sicli, secondo il siclo del santuario, ambedue pieni di fior di farina intrisa con olio, per l'oblazione;
\par 14 una coppa d'oro di dieci sicli piena di profumo,
\par 15 un giovenco, un montone,
\par 16 un agnello dell'anno per l'olocausto, un capro per il sacrifizio per il peccato,
\par 17 e, per il sacrifizio di azioni di grazie, due buoi, cinque montoni, cinque capri, cinque agnelli dell'anno. Tale fu l'offerta di Nahshon, figliuolo d'Amminadab.
\par 18 Il secondo giorno, Nethaneel, figliuolo di Tsuar, principe d'Issacar, presentò la sua offerta.
\par 19 Offrì un piatto d'argento del peso di centotrenta sicli, un bacino d'argento di settanta sicli, secondo il siclo del santuario, ambedue pieni di fior di farina intrisa con olio, per l'oblazione;
\par 20 una coppa d'oro di dieci sicli piena di profumo,
\par 21 un giovenco, un montone, un agnello dell'anno per l'olocausto,
\par 22 un capro per il sacrifizio per il peccato,
\par 23 e, per il sacrifizio di azioni di grazie, due buoi, cinque montoni, cinque capri, cinque agnelli dell'anno. Tale fu l'offerta di Nethaneel, figliuolo di Tsuar.
\par 24 Il terzo giorno fu Eliab, figliuolo di Helon, principe dei figliuoli di Zabulon.
\par 25 La sua offerta fu un piatto d'argento del peso di centotrenta sicli, un bacino d'argento di settanta sicli, secondo il siclo del santuario, ambedue pieni di fior di farina intrisa con olio, per l'oblazione;
\par 26 una coppa d'oro di dieci sicli piena di profumo,
\par 27 un giovenco, un montone, un agnello dell'anno per l'olocausto,
\par 28 un capro per il sacrifizio per il peccato,
\par 29 e, per sacrifizio da render grazie, due buoi, cinque montoni, cinque capri, cinque agnelli dell'anno. Tale fu l'offerta di Eliab, figliuolo di Helon.
\par 30 Il quarto giorno fu Elitsur, figliuolo di Scedeur, principe dei figliuoli di Ruben.
\par 31 La sua offerta fu un piatto d'argento del peso di centotrenta sicli, un bacino d'argento di settanta sicli, secondo il siclo del santuario, ambedue pieni di fior di farina intrisa con olio, per l'oblazione;
\par 32 una coppa d'oro di dieci sicli piena di profumo,
\par 33 un giovenco, un montone, un agnello dell'anno per l'olocausto,
\par 34 un capro per il sacrifizio per il peccato,
\par 35 e, per sacrifizio di azioni di grazie, due buoi, cinque montoni, cinque capri, cinque agnelli dell'anno. Tale fu l'offerta di Elitsur, figliuolo di Scedeur.
\par 36 Il quinto giorno fu Scelumiel, figliuolo di Tsurishaddai, principe dei figliuoli di Simeone.
\par 37 La sua offerta fu un piatto d'argento del peso di centotrenta sicli, un bacino d'argento di settanta sicli, secondo il siclo del santuario, ambedue pieni di fior di farina intrisa con olio, per l'oblazione;
\par 38 una coppa d'oro di dieci sicli piena di profumo,
\par 39 un giovenco, un montone, un agnello dell'anno per l'olocausto,
\par 40 un capro per il sacrifizio per il peccato,
\par 41 e, per sacrifizio di azioni di grazie, due buoi, cinque montoni, cinque capri, cinque agnelli dell'anno. Tale fu l'offerta di Scelumiel, figliuolo di Tsurishaddai.
\par 42 Il sesto giorno fu Eliasaf, figliuolo di Deuel, principe dei figliuoli di Gad.
\par 43 La sua offerta fu un piatto d'argento del peso di centotrenta sicli, un bacino d'argento di settanta sicli, secondo il siclo del santuario, ambedue pieni di fior di farina intrisa con olio, per l'oblazione;
\par 44 una coppa d'oro di dieci sicli piena di profumo,
\par 45 un giovenco, un montone, un agnello dell'anno per l'olocausto,
\par 46 un capro per il sacrifizio per il peccato,
\par 47 e, per sacrifizio di azioni di grazie, due buoi, cinque montoni, cinque capri, cinque agnelli dell'anno. Tale fu l'offerta di Eliasaf, figliuolo di Deuel.
\par 48 Il settimo giorno fu Elishama, figliuolo di Ammihud, principe dei figliuoli d'Efraim.
\par 49 La sua offerta fu un piatto d'argento del peso di centotrenta sicli, un bacino d'argento di settanta sicli, secondo il siclo del santuario, ambedue pieni di fior di farina intrisa con olio, per l'oblazione;
\par 50 una coppa d'oro di dieci sicli piena di profumo,
\par 51 un giovenco, un montone, un agnello dell'anno per l'olocausto,
\par 52 un capro per il sacrifizio per il peccato,
\par 53 e, per il sacrifizio di azioni di grazie, due buoi, cinque montoni, cinque capri, cinque agnelli dell'anno. Tale fu l'offerta di Elishama, figliuolo di Ammihud.
\par 54 L'ottavo giorno fu Gamaliel, figliuolo di Pedahtsur, principe dei figliuoli di Manasse.
\par 55 La sua offerta fu un piatto d'argento del peso di centotrenta sicli, un bacino d'argento di settanta sicli, secondo il siclo del santuario, ambedue pieni di fior di farina intrisa con olio, per l'oblazione;
\par 56 una coppa d'oro di dieci sicli piena di profumo,
\par 57 un giovenco, un montone, un agnello dell'anno per l'olocausto,
\par 58 un capro per il sacrifizio per il peccato,
\par 59 e, per il sacrifizio di azioni di grazie, due buoi, cinque montoni, cinque capri, cinque agnelli dell'anno. Tale fu l'offerta di Gamaliel, figliuolo di Pedahtsur.
\par 60 Il nono giorno fu Abidan, figliuolo di Ghideoni, principe dei figliuoli di Beniamino.
\par 61 La sua offerta fu un piatto d'argento del peso di centotrenta sicli, un bacino d'argento di settanta sicli, secondo il siclo del santuario, ambedue pieni di fior di farina intrisa con olio, per l'oblazione;
\par 62 una coppa d'oro di dieci sicli piena di profumo,
\par 63 un giovenco, un montone, un agnello dell'anno per l'olocausto,
\par 64 un capro per il sacrifizio per il peccato,
\par 65 e, per il sacrifizio di azioni di grazie, due buoi, cinque montoni, cinque capri, cinque agnelli dell'anno. Tale fu l'offerta di Abidan, figliuolo di Ghideoni.
\par 66 Il decimo giorno fu Ahiezer, figliuolo di Ammishaddai, principe dei figliuoli di Dan.
\par 67 La sua offerta fu un piatto d'argento del peso di centotrenta sicli, un bacino d'argento di settanta sicli, secondo il siclo del santuario, ambedue pieni di fior di farina intrisa con olio, per l'oblazione;
\par 68 una coppa d'oro di dieci sicli piena di profumo,
\par 69 un giovenco, un montone, un agnello dell'anno per l'olocausto,
\par 70 un capro per il sacrifizio per il peccato,
\par 71 e, per il sacrifizio di azioni di grazie, due buoi, cinque montoni, cinque capri, cinque agnelli dell'anno. Tale fu l'offerta di Ahiezer, figliuolo di Ammishaddai.
\par 72 L'undecimo giorno fu Paghiel, figliuolo di Ocran, principe dei figliuoli di Ascer.
\par 73 La sua offerta fu un piatto d'argento del peso di centotrenta sicli, un bacino d'argento di settanta sicli, secondo il siclo del santuario, ambedue pieni di fior di farina intrisa con olio, per l'oblazione;
\par 74 una coppa d'oro di dieci sicli piena di profumo,
\par 75 un giovenco, un montone, un agnello dell'anno per l'olocausto,
\par 76 un capro per il sacrifizio per il peccato,
\par 77 e, per il sacrifizio di azioni di grazie, due buoi, cinque montoni, cinque capri, cinque agnelli dell'anno. Tale fu l'offerta di Paghiel, figliuolo di Ocran.
\par 78 Il dodicesimo giorno fu Ahira, figliuolo d'Enan, principe dei figliuoli di Neftali.
\par 79 La sua offerta fu un piatto d'argento del peso di centotrenta sicli, un bacino d'argento di settanta sicli, secondo il siclo del santuario, ambedue pieni di fior di farina intrisa con olio, per l'oblazione;
\par 80 una coppa d'oro di dieci sicli piena di profumo,
\par 81 un giovenco, un montone, un agnello dell'anno per l'olocausto,
\par 82 un capro per il sacrifizio per il peccato,
\par 83 e, per il sacrifizio di azioni di grazie, due buoi, cinque montoni, cinque capri, cinque agnelli dell'anno. Tale fu l'offerta di Ahira, figliuolo di Enan.
\par 84 Questi furono i doni per la dedicazione dell'altare, da parte dei principi d'Israele, il giorno in cui esso fu unto: dodici piatti d'argento, dodici bacini d'argento, dodici coppe d'oro;
\par 85 ogni piatto d'argento pesava centotrenta sicli e ogni bacino d'argento, settanta; il totale dell'argento dei vasi fu duemilaquattrocento sicli, secondo il siclo del santuario;
\par 86 dodici coppe d'oro piene di profumo, le quali, a dieci sicli per coppa, secondo il siclo del santuario, dettero, per l'oro delle coppe, un totale di centoventi sicli.
\par 87 Totale del bestiame per l'olocausto: dodici giovenchi, dodici montoni, dodici agnelli dell'anno con le oblazioni ordinarie, e dodici capri per il sacrifizio per il peccato.
\par 88 Totale del bestiame per il sacrifizio di azioni di grazie: ventiquattro giovenchi, sessanta montoni, sessanta capri, sessanta agnelli dell'anno. Tali furono i doni per la dedicazione dell'altare, dopo ch'esso fu unto.
\par 89 E quando Mosè entrava nella tenda di convegno per parlare con l'Eterno, udiva la voce che gli parlava dall'alto del propiziatorio che è sull'arca della testimonianza fra i due cherubini; e l'Eterno gli parlava.

\chapter{8}

\par 1 L'Eterno parlò ancora a Mosè, dicendo:
\par 2 'Parla ad Aaronne, e digli: Quando collocherai le lampade, le sette lampade dovranno proiettare la luce sul davanti del candelabro'.
\par 3 E Aaronne fece così; collocò le lampade in modo che facessero luce sul davanti del candelabro, come l'Eterno aveva ordinato a Mosè.
\par 4 Or il candelabro era fatto così: era d'oro battuto; tanto la sua base quanto i suoi fiori erano lavorati a martello. Mosè avea fatto il candelabro secondo il modello che l'Eterno gli aveva mostrato.
\par 5 E l'Eterno parlò a Mosè, dicendo:
\par 6 'Prendi i Leviti di tra i figliuoli d'Israele, e purificali.
\par 7 E, per purificarli, farai così: li aspergerai con l'acqua dell'espiazione, essi faranno passare il rasoio su tutto il loro corpo, laveranno le loro vesti e si purificheranno.
\par 8 Poi prenderanno un giovenco con l'oblazione ordinaria di fior di farina intrisa con olio, e tu prenderai un altro giovenco per il sacrifizio per il peccato.
\par 9 Farai avvicinare i Leviti dinanzi alla tenda di convegno, e convocherai tutta la raunanza de' figliuoli d'Israele.
\par 10 Farai avvicinare i Leviti dinanzi all'Eterno, e i figliuoli d'Israele poseranno le loro mani sui Leviti;
\par 11 e Aaronne presenterà i Leviti come offerta agitata davanti all'Eterno da parte dei figliuoli d'Israele, ed essi faranno il servizio dell'Eterno.
\par 12 Poi i Leviti poseranno le loro mani sulla testa dei giovenchi, e tu ne offrirai uno come sacrifizio per il peccato e l'altro come olocausto all'Eterno, per fare l'espiazione per i Leviti.
\par 13 E farai stare i Leviti in piè davanti ad Aaronne e davanti ai suoi figliuoli, e li presenterai come un'offerta agitata all'Eterno.
\par 14 Così separerai i Leviti di tra i figliuoli d'Israele, e i Leviti saranno miei.
\par 15 Dopo questo, i Leviti verranno a fare il servizio nella tenda di convegno; e tu li purificherai, e li presenterai come un'offerta agitata;
\par 16 poiché mi sono interamente dati di tra i figliuoli d'Israele; io li ho presi per me, invece di tutti quelli che aprono il seno materno, dei primogeniti di tutti i figliuoli d'Israele.
\par 17 Poiché tutti i primogeniti dei figliuoli d'Israele, tanto degli uomini quanto del bestiame, sono miei; io me li consacrai il giorno che percossi tutti i primogeniti nel paese d'Egitto.
\par 18 E ho preso i Leviti invece di tutti i primogeniti dei figliuoli d'Israele.
\par 19 E ho dato in dono ad Aaronne ed ai suoi figliuoli i Leviti di tra i figliuoli d'Israele, perché facciano il servizio dei figliuoli d'Israele nella tenda di convegno, e perché facciano l'espiazione per i figliuoli d'Israele, onde nessuna piaga scoppi tra i figliuoli d'Israele per il loro accostarsi al santuario'.
\par 20 Così fecero Mosè, Aaronne e tutta la raunanza dei figliuoli d'Israele rispetto ai Leviti; i figliuoli d'Israele fecero a loro riguardo tutto quello che l'Eterno avea ordinato a Mosè relativamente a loro.
\par 21 E i Leviti si purificarono e lavarono le loro vesti; e Aaronne li presentò come un'offerta agitata davanti all'Eterno, e fece l'espiazione per essi, per purificarli.
\par 22 Dopo questo, i Leviti vennero a fare il loro servizio nella tenda di convegno in presenza di Aaronne e dei suoi figliuoli. Si fece rispetto ai Leviti secondo l'ordine che l'Eterno avea dato a Mosè circa loro.
\par 23 E l'Eterno parlò a Mosè, dicendo:
\par 24 'Questo è quel che concerne i Leviti: da venticinque anni in su il Levita entrerà in servizio per esercitare un ufficio nella tenda di convegno;
\par 25 e dall'età di cinquant'anni si ritirerà dall'esercizio dell'ufficio, e non servirà più.
\par 26 Potrà assistere i suoi fratelli nella tenda di convegno, sorvegliando ciò che è affidato alle loro cure; ma non farà più servizio. Così farai, rispetto ai Leviti, per quel che concerne i loro uffici'.

\chapter{9}

\par 1 L'Eterno parlò ancora a Mosè, nel deserto di Sinai, il primo mese del secondo anno da che furono usciti dal paese d'Egitto, dicendo:
\par 2 'I figliuoli d'Israele celebreranno la pasqua nel tempo stabilito.
\par 3 La celebrerete nel tempo stabilito, il quattordicesimo giorno di questo mese, sull'imbrunire; la celebrerete secondo tutte le leggi e secondo tutte le prescrizioni che vi si riferiscono'.
\par 4 E Mosè parlò ai figliuoli d'Israele perché celebrassero la pasqua.
\par 5 Ed essi celebrarono la pasqua il quattordicesimo giorno del primo mese, sull'imbrunire, nel deserto di Sinai; i figliuoli d'Israele si conformarono a tutti gli ordini che l'Eterno avea dati a Mosè.
\par 6 Or v'erano degli uomini che, essendo impuri per aver toccato un morto, non potevan celebrare la pasqua in quel giorno. Si presentarono in quello stesso giorno davanti a Mosè e davanti ad Aaronne;
\par 7 e quegli uomini dissero a Mosè: 'Noi siamo impuri per aver toccato un morto; perché ci sarebb'egli tolto di poter presentare l'offerta dell'Eterno, al tempo stabilito, in mezzo ai figliuoli d'Israele?'
\par 8 E Mosè rispose loro: 'Aspettate, e sentirò quel che l'Eterno ordinerà a vostro riguardo'.
\par 9 E l'Eterno parlò a Mosè, dicendo:
\par 10 'Parla ai figliuoli d'Israele, e di' loro: Se uno di voi o de' vostri discendenti sarà impuro per il contatto con un morto o sarà lontano in viaggio, celebrerà lo stesso la pasqua in onore dell'Eterno.
\par 11 La celebreranno il quattordicesimo giorno del secondo mese, sull'imbrunire; la mangeranno con del pane senza lievito e con delle erbe amare;
\par 12 non ne lasceranno nulla di resto fino al mattino e non ne spezzeranno alcun osso. La celebreranno secondo tutte le leggi della pasqua.
\par 13 Ma colui ch'è puro e che non è in viaggio, se s'astiene dal celebrare la pasqua, quel tale sarà sterminato di fra il suo popolo; siccome non ha presentato l'offerta all'Eterno nel tempo stabilito, quel tale porterà la pena del suo peccato.
\par 14 E se uno straniero che soggiorna tra voi celebra la pasqua dell'Eterno, si conformerà alle leggi e alle prescrizioni della pasqua. Avrete un'unica legge, per lo straniero e per il nativo del paese'.
\par 15 Or il giorno in cui il tabernacolo fu eretto, la nuvola coprì il tabernacolo, la tenda della testimonianza; e, dalla sera fino alla mattina, aveva sul tabernacolo l'apparenza d'un fuoco.
\par 16 Così avveniva sempre: la nuvola copriva il tabernacolo, e di notte avea l'apparenza d'un fuoco.
\par 17 E tutte le volte che la nuvola s'alzava di sulla tenda, i figliuoli d'Israele si mettevano in cammino; e dove la nuvola si fermava, quivi i figliuoli d'Israele si accampavano.
\par 18 I figliuoli d'Israele si mettevano in cammino all'ordine dell'Eterno, e all'ordine dell'Eterno si accampavano; rimanevano accampati tutto il tempo che la nuvola restava sul tabernacolo.
\par 19 E quando la nuvola rimaneva per molti giorni sul tabernacolo, i figliuoli d'Israele osservavano la prescrizione dell'Eterno e non si movevano.
\par 20 E se avveniva che la nuvola rimanesse pochi giorni sul tabernacolo, all'ordine dell'Eterno rimanevano accampati, e all'ordine dell'Eterno si mettevano in cammino.
\par 21 E se la nuvola si fermava dalla sera alla mattina, e s'alzava la mattina, si mettevano in cammino; o se dopo un giorno e una notte la nuvola si alzava, si mettevano in cammino.
\par 22 Se la nuvola rimaneva ferma sul tabernacolo due giorni o un mese o un anno, i figliuoli d'Israele rimanevano accampati e non si moveano; ma, quando s'alzava, si mettevano in cammino.
\par 23 All'ordine dell'Eterno si accampavano, e all'ordine dell'Eterno si mettevano in cammino; osservavano le prescrizioni dell'Eterno, secondo l'ordine trasmesso dall'Eterno per mezzo di Mosè.

\chapter{10}

\par 1 L'Eterno parlò ancora a Mosè, dicendo:
\par 2 'Fatti due trombe d'argento; le farai d'argento battuto; ti serviranno per convocare la raunanza e per far muovere i campi.
\par 3 Al suon d'esse tutta la raunanza si raccoglierà presso di te, all'ingresso della tenda di convegno.
\par 4 Al suono d'una tromba sola, i principi, i capi delle migliaia d'Israele, si aduneranno presso di te.
\par 5 Quando sonerete a lunghi e forti squilli, i campi che sono a levante si metteranno in cammino.
\par 6 Quando sonerete una seconda volta a lunghi e forti squilli, i campi che si trovano a mezzogiorno si metteranno in cammino; si sonerà a lunghi e forti squilli quando dovranno mettersi in cammino.
\par 7 Quando dev'esser convocata la raunanza, sonerete, ma non a lunghi e forti squilli.
\par 8 E i sacerdoti figliuoli d'Aaronne soneranno le trombe; sarà una legge perpetua per voi e per i vostri discendenti.
\par 9 Quando nel vostro paese andrete alla guerra contro il nemico che vi attaccherà, sonerete a lunghi e forti squilli con le trombe, e sarete ricordati dinanzi all'Eterno, al vostro Dio, e sarete liberati dai vostri nemici.
\par 10 Così pure nei vostri giorni di gioia, nelle vostre solennità e al principio de' vostri mesi, sonerete con le trombe quand'offrirete i vostri olocausti e i vostri sacrifizi di azioni di grazie; ed esse vi faranno ricordare nel cospetto del vostro Dio. Io sono l'Eterno, il vostro Dio'.
\par 11 Or avvenne che, il secondo anno, il secondo mese, il ventesimo giorno del mese, la nuvola s'alzò di sopra il tabernacolo della testimonianza.
\par 12 E i figliuoli d'Israele partirono dal deserto di Sinai, secondo l'ordine fissato per le loro marce; e la nuvola si fermò nel deserto di Paran.
\par 13 Così si misero in cammino la prima volta, secondo l'ordine dell'Eterno trasmesso per mezzo di Mosè.
\par 14 La bandiera del campo de' figliuoli di Giuda, diviso secondo le loro schiere, si mosse la prima. Nahshon, figliuolo di Amminadab comandava l'esercito di Giuda.
\par 15 Nethaneel, figliuolo di Tsuar, comandava l'esercito della tribù de' figliuoli d'Issacar,
\par 16 ed Eliab, figliuolo di Helon, comandava l'esercito della tribù dei figliuoli di Zabulon.
\par 17 Il tabernacolo fu smontato, e i figliuoli di Gherson e i figliuoli di Merari si misero in cammino, portando il tabernacolo.
\par 18 Poi si mosse la bandiera del campo di Ruben, diviso secondo le sue schiere. Elitsur, figliuolo di Scedeur, comandava l'esercito di Ruben.
\par 19 Scelumiel, figliuolo di Tsurishaddai, comandava l'esercito della tribù de' figliuoli di Simeone,
\par 20 ed Eliasaf, figliuolo di Deuel, comandava l'esercito della tribù de' figliuoli di Gad.
\par 21 Poi si mossero i Kehathiti, portando gli oggetti sacri; e gli altri rizzavano il tabernacolo, prima che quelli arrivassero.
\par 22 Poi si mosse la bandiera del campo de' figliuoli di Efraim, diviso secondo le sue schiere. Elishama, figliuolo di Ammihud, comandava l'esercito di Efraim.
\par 23 Gamaliel, figliuolo di Pedahtsur, comandava l'esercito della tribù dei figliuoli di Manasse,
\par 24 e Abidan, figliuolo di Ghideoni, comandava l'esercito della tribù de' figliuoli di Beniamino.
\par 25 Poi si mosse la bandiera del campo de' figliuoli di Dan, diviso secondo le sue schiere, formando la retroguardia di tutti i campi. Ahiezer, figliuolo di Ammishaddai, comandava l'esercito di Dan.
\par 26 Paghiel, figliuolo di Ocran, comandava l'esercito della tribù de' figliuoli di Ascer,
\par 27 e Ahira, figliuolo di Enan, comandava l'esercito della tribù de' figliuoli di Neftali.
\par 28 Tale era l'ordine in cui i figliuoli d'Israele si misero in cammino, secondo le loro schiere. E così partirono.
\par 29 Or Mosè disse a Hobab, figliuolo di Reuel, Madianita, suocero di Mosè: 'Noi c'incamminiamo verso il luogo del quale l'Eterno ha detto: Io ve lo darò. Vieni con noi e ti faremo del bene, perché l'Eterno ha promesso di far del bene a Israele'.
\par 30 Hobab gli rispose: 'Io non verrò, ma andrò al mio paese e dai miei parenti'.
\par 31 E Mosè disse: 'Deh, non ci lasciare; poiché tu conosci i luoghi dove dovremo accamparci nel deserto, e sarai la nostra guida.
\par 32 E, se vieni con noi, qualunque bene l'Eterno farà a noi, noi lo faremo a te'.
\par 33 Così partirono dal monte dell'Eterno, e fecero tre giornate di cammino; e l'arca del patto dell'Eterno andava davanti a loro durante le tre giornate di cammino, per cercar loro un luogo di riposo.
\par 34 E la nuvola dell'Eterno era su loro, durante il giorno, quando partivano dal campo.
\par 35 Quando l'arca partiva, Mosè diceva: 'Lèvati, o Eterno, e siano dispersi i tuoi nemici, e fuggano dinanzi alla tua presenza quelli che t'odiano!'
\par 36 E quando si posava, diceva: 'Torna, o Eterno, alle miriadi delle schiere d'Israele!'

\chapter{11}

\par 1 Or il popolo fece giungere empi mormorii agli orecchi dell'Eterno; e come l'Eterno li udì, la sua ira si accese, il fuoco dell'Eterno divampò fra loro e divorò l'estremità del campo.
\par 2 E il popolo gridò a Mosè; Mosè pregò l'Eterno, e il fuoco si spense.
\par 3 E a quel luogo fu posto nome Taberah, perché il fuoco dell'Eterno avea divampato fra loro.
\par 4 E l'accozzaglia di gente raccogliticcia ch'era tra il popolo, fu presa da concupiscenza; e anche i figliuoli d'Israele ricominciarono a piagnucolare e a dire: 'Chi ci darà da mangiare della carne?
\par 5 Ci ricordiamo de' pesci che mangiavamo in Egitto per nulla, dei cocomeri, de' poponi, de' porri, delle cipolle e degli agli.
\par 6 E ora l'anima nostra è inaridita; non c'è più nulla! gli occhi nostri non vedono altro che questa manna'.
\par 7 Or la manna era simile al seme di coriandolo e avea l'aspetto del bdellio.
\par 8 Il popolo andava attorno a raccoglierla; poi la riduceva in farina con le macine o la pestava nel mortaio, la faceva cuocere in pentole o ne faceva delle focacce, e aveva il sapore d'una focaccia con l'olio.
\par 9 Quando la rugiada cadeva sul campo, la notte, vi cadeva anche la manna.
\par 10 E Mosè udì il popolo che piagnucolava, in tutte le famiglie, ognuno all'ingresso della propria tenda; l'ira dell'Eterno si accese gravemente e la cosa dispiacque anche a Mosè.
\par 11 E Mosè disse all'Eterno: 'Perché hai trattato così male il tuo servo? perché non ho io trovato grazia agli occhi tuoi, che tu m'abbia messo addosso il carico di tutto questo popolo?
\par 12 L'ho forse concepito io tutto questo popolo? o l'ho forse dato alla luce io, che tu mi dica: Portalo sul tuo seno, come il balio porta il bimbo lattante, fino al paese che tu hai promesso con giuramento ai suoi padri?
\par 13 Donde avrei io della carne da dare a tutto questo popolo? Poiché piagnucola dietro a me dicendo: Dacci da mangiare della carne!
\par 14 Io non posso, da me solo, portare tutto questo popolo; è un peso troppo grave per me.
\par 15 E se mi vuoi trattare così, uccidimi, ti prego; uccidimi, se ho trovato grazia agli occhi tuoi; e ch'io non vegga la mia sventura!'
\par 16 E l'Eterno disse a Mosè: 'Radunami settanta uomini degli anziani d'Israele, conosciuti da te come anziani del popolo e come aventi autorità sovr'esso; conducili alla tenda di convegno e vi si presentino con te.
\par 17 Io scenderò e parlerò quivi teco; prenderò dello spirito che è su te e lo metterò su loro, perché portino con te il carico del popolo, e tu non lo porti più da solo.
\par 18 E dirai al popolo: Santificatevi per domani, e mangerete della carne, poiché avete pianto agli orecchi dell'Eterno, dicendo: Chi ci farà mangiar della carne? Stavamo pur bene in Egitto! Ebbene, l'Eterno vi darà della carne, e voi ne mangerete.
\par 19 E ne mangerete, non per un giorno, non per due giorni, non per cinque giorni, non per dieci giorni, non per venti giorni, ma per un mese intero,
\par 20 finché vi esca per le narici e vi faccia nausea, poiché avete rigettato l'Eterno che è in mezzo a voi, e avete pianto davanti a lui, dicendo: Perché mai siamo usciti dall'Egitto?'
\par 21 E Mosè disse: 'Questo popolo, in mezzo al quale mi trovo, novera seicentomila adulti, e tu hai detto: Io darò loro della carne, e ne mangeranno per un mese intero!
\par 22 Si scanneranno per loro greggi ed armenti in modo che n'abbiano abbastanza? o si radunerà per loro tutto il pesce del mare in modo che n'abbiano abbastanza?'
\par 23 E l'Eterno rispose a Mosè: 'La mano dell'Eterno è forse raccorciata? Ora vedrai se la parola che t'ho detta s'adempia o no'.
\par 24 Mosè dunque uscì e riferì al popolo le parole dell'Eterno; e radunò settanta uomini degli anziani del popolo, e li pose intorno alla tenda.
\par 25 E l'Eterno scese nella nuvola e gli parlò; prese dello spirito ch'era su lui, e lo mise sui settanta anziani; e avvenne che, quando lo spirito si fu posato su loro, quelli profetizzarono, ma non continuarono.
\par 26 Intanto, due uomini, l'uno chiamato Eldad e l'altro Medad, erano rimasti nel campo, e lo spirito si posò su loro; erano fra gl'iscritti, ma non erano usciti per andare alla tenda; e profetizzarono nel campo.
\par 27 Un giovane corse a riferire la cosa a Mosè, e disse: 'Eldad e Medad profetizzano nel campo'.
\par 28 Allora Giosuè, figliuolo di Nun, servo di Mosè dalla sua giovinezza, prese a dire: 'Mosè, signor mio, non glielo permettere!'
\par 29 Ma Mosè gli rispose: 'Sei tu geloso per me? Oh! fossero pur tutti profeti nel popolo dell'Eterno, e volesse l'Eterno metter su loro lo spirito suo!'
\par 30 E Mosè si ritirò nel campo, insieme con gli anziani d'Israele.
\par 31 E un vento si levò, per ordine dell'Eterno, e portò delle quaglie dalla parte del mare, e le fe' cadere presso il campo, sulla distesa di circa una giornata di cammino da un lato e una giornata di cammino dall'altro intorno al campo, e a un'altezza di circa due cubiti sulla superficie del suolo.
\par 32 E il popolo si levò, e tutto quel giorno e tutta la notte e tutto il giorno seguente raccolse le quaglie. Chi ne raccolse meno n'ebbe dieci omer; e se le distesero tutt'intorno al campo.
\par 33 Ne avevano ancora la carne fra i denti e non l'avevano peranco masticata, quando l'ira dell'Eterno s'accese contro il popolo, e l'Eterno percosse il popolo con una gravissima piaga.
\par 34 E a quel luogo fu dato il nome di Kibroth-Hattaava, perché vi si seppellì la gente ch'era stata presa dalla concupiscenza.
\par 35 Da Kibroth-Hattaava il popolo partì per Hatseroth, e a Hatseroth si fermò.

\chapter{12}

\par 1 Maria ed Aaronne parlarono contro Mosè a cagione della moglie Cuscita che avea preso; poiché avea preso una moglie Cuscita.
\par 2 E dissero: 'L'Eterno ha egli parlato soltanto per mezzo di Mosè? non ha egli parlato anche per mezzo nostro?' E l'Eterno l'udì.
\par 3 Or Mosè era un uomo molto mansueto, più d'ogni altro uomo sulla faccia della terra.
\par 4 E l'Eterno disse a un tratto a Mosè, ad Aaronne e a Maria: 'Uscite voi tre, e andate alla tenda di convegno'. E uscirono tutti e tre.
\par 5 E l'Eterno scese in una colonna di nuvola, si fermò all'ingresso della tenda, e chiamò Aaronne e Maria; ambedue si fecero avanti.
\par 6 E l'Eterno disse: 'Ascoltate ora le mie parole; se v'è tra voi alcun profeta, io, l'Eterno, mi faccio conoscere a lui in visione, parlo con lui in sogno.
\par 7 Non così col mio servitore Mosè, che è fedele in tutta la mia casa.
\par 8 Con lui io parlo a tu per tu, facendomi vedere, e non per via d'enimmi; ed egli contempla la sembianza dell'Eterno. Perché dunque non avete temuto di parlar contro il mio servo, contro Mosè?'
\par 9 E l'ira dell'Eterno s'accese contro loro, ed egli se ne andò,
\par 10 e la nuvola si ritirò di sopra alla tenda; ed ecco che Maria era lebbrosa, bianca come neve; Aaronne guardò Maria, ed ecco era lebbrosa.
\par 11 E Aaronne disse a Mosè: 'Deh, signor mio, non ci far portare la pena di un peccato che abbiamo stoltamente commesso, e di cui siamo colpevoli.
\par 12 Deh, ch'ella non sia come il bimbo nato morto, la cui carne è già mezzo consumata quand'esce dal seno materno!'
\par 13 E Mosè gridò all'Eterno, dicendo: 'Guariscila, o Dio, te ne prego!'
\par 14 E l'Eterno rispose a Mosè: 'Se suo padre le avesse sputato in viso, non ne porterebbe ella la vergogna per sette giorni? Stia dunque rinchiusa fuori del campo sette giorni; poi, vi sarà di nuovo ammessa'.
\par 15 Maria dunque fu rinchiusa fuori del campo sette giorni; e il popolo non si mise in cammino finché Maria non fu riammessa al campo.
\par 16 Poi il popolo partì da Hatseroth, e si accampò nel deserto di Paran.

\chapter{13}

\par 1 L'Eterno parlò a Mosè, dicendo:
\par 2 'Manda degli uomini ad esplorare il paese di Canaan che io do ai figliuoli d'Israele. Mandate un uomo per ogni tribù de' loro padri; siano tutti dei loro principi'.
\par 3 E Mosè li mandò dal deserto di Paran, secondo l'ordine dell'Eterno; quegli uomini erano tutti capi de' figliuoli d'Israele.
\par 4 E questi erano i loro nomi: Per la tribù di Ruben: Shammua, figliuolo di Zaccur; per la tribù di Simeone:
\par 5 Shafat, figliuolo di Hori;
\par 6 per la tribù di Giuda: Caleb, figliuolo di Gefunne;
\par 7 per la tribù d'Issacar: Igal, figliuolo di Giuseppe;
\par 8 per la tribù di Efraim: Hoscea, figliuolo di Nun;
\par 9 per la tribù di Beniamino: Palti, figliuolo di Rafu;
\par 10 per la tribù di Zabulon: Gaddiel, figliuolo di Sodi;
\par 11 per la tribù di Giuseppe, cioè, per la tribù di Manasse: Gaddi figliuolo di Susi;
\par 12 per la tribù di Dan: Ammiel, figliuolo di Ghemalli;
\par 13 per la tribù di Ascer: Sethur, figliuolo di Micael;
\par 14 per la tribù di Neftali: Nahbi, figliuolo di Vofsi;
\par 15 per la tribù di Gad: Gheual, figliuolo di Machi.
\par 16 Tali i nomi degli uomini che Mosè mandò a esplorare il paese. E Mosè dette ad Hoscea, figliuolo di Nun, il nome di Giosuè.
\par 17 Mosè dunque li mandò ad esplorare il paese di Canaan, e disse loro: 'Andate su di qua per il mezzogiorno; poi salirete sui monti,
\par 18 e vedrete che paese sia, che popolo l'abiti, se forte o debole, se poco o molto numeroso;
\par 19 come sia il paese che abita, se buono o cattivo, e come siano le città dove abita, se siano degli accampamenti o dei luoghi fortificati;
\par 20 e come sia il terreno, se grasso o magro, se vi siano alberi o no. Abbiate coraggio, e portate de' frutti del paese'. Era il tempo che cominciava a maturar l'uva.
\par 21 Quelli dunque salirono ed esplorarono il paese dal deserto di Tsin fino a Rehob, sulla via di Hamath.
\par 22 Salirono per il mezzogiorno e andarono fino a Hebron, dov'erano Ahiman, Sceshai e Talmai, figliuoli di Anak. Or Hebron era stata edificata sette anni prima di Tsoan in Egitto.
\par 23 E giunsero fino alla valle d'Eshcol, dove tagliarono un tralcio con un grappolo d'uva, che portarono in due con una stanga, e presero anche delle melagrane e dei fichi.
\par 24 Quel luogo fu chiamato valle d'Eshcol a motivo del grappolo d'uva che i figliuoli d'Israele vi tagliarono.
\par 25 E alla fine di quaranta giorni tornarono dall'esplorazione del paese,
\par 26 e andarono a trovar Mosè ed Aaronne e tutta la raunanza de' figliuoli d'Israele nel deserto di Paran, a Kades; riferirono ogni cosa a loro e a tutta la raunanza, e mostrarono loro i frutti del paese.
\par 27 E fecero il loro racconto, dicendo: 'Noi arrivammo nel paese dove tu ci mandasti, ed è davvero un paese dove scorre il latte e il miele, ed ecco de' suoi frutti.
\par 28 Soltanto, il popolo che abita il paese è potente, le città sono fortificate e grandissime, e v'abbiamo anche veduto de' figliuoli di Anak.
\par 29 Gli Amalekiti abitano la parte meridionale del paese; gli Hittei, i Gebusei e gli Amorei, la regione montuosa; e i Cananei abitano presso il mare e lungo il Giordano'.
\par 30 E Caleb calmò il popolo che mormorava contro Mosè, e disse: 'Saliamo pure e conquistiamo il paese; poiché possiamo benissimo soggiogarlo'.
\par 31 Ma gli uomini che v'erano andati con lui, dissero: 'Noi non siam capaci di salire contro questo popolo; perché è più forte di noi'.
\par 32 E screditarono presso i figliuoli d'Israele il paese che aveano esplorato, dicendo: 'Il paese che abbiamo attraversato per esplorarlo, è un paese che divora i suoi abitanti; e tutta la gente che vi abbiam veduta, è gente d'alta statura;
\par 33 e v'abbiam visto i giganti, figliuoli di Anak, della razza de' giganti, appetto ai quali ci pareva d'esser locuste; e tali parevamo a loro'.

\chapter{14}

\par 1 Allora tutta la raunanza alzò la voce e diede in alte grida; e il popolo pianse tutta quella notte.
\par 2 E tutti i figliuoli d'Israele mormorarono contro Mosè e contro Aaronne, e tutta la raunanza disse loro: 'Fossimo pur morti nel paese d'Egitto! o fossimo pur morti in questo deserto!
\par 3 E perché ci mena l'Eterno in quel paese ove cadremo per la spada? Le nostre mogli e i nostri piccini vi saranno preda del nemico. Non sarebb'egli meglio per noi di tornare in Egitto?'
\par 4 E si dissero l'uno all'altro: 'Nominiamoci un capo, e torniamo in Egitto!'
\par 5 Allora Mosè ed Aaronne si prostrarono a terra dinanzi a tutta l'assemblea riunita de' figliuoli d'Israele.
\par 6 E Giosuè, figliuolo di Nun, e Caleb, figliuolo di Gefunne, ch'erano di quelli che aveano esplorato il paese, si stracciarono le vesti,
\par 7 e parlarono così a tutta la raunanza de' figliuoli d'Israele: 'Il paese che abbiamo attraversato per esplorarlo, è un paese buono, buonissimo.
\par 8 Se l'Eterno ci è favorevole, c'introdurrà in quel paese, e ce lo darà: è un paese dove scorre il latte e il miele.
\par 9 Soltanto, non vi ribellate all'Eterno, e non abbiate paura del popolo di quel paese; poiché ne faremo nostro pascolo; l'ombra che li copriva s'è ritirata, e l'Eterno è con noi; non ne abbiate paura'.
\par 10 Allora tutta la raunanza parlò di lapidarli; ma la gloria dell'Eterno apparve sulla tenda di convegno a tutti i figliuoli d'Israele.
\par 11 E l'Eterno disse a Mosè: 'Fino a quando mi disprezzerà questo popolo? e fino a quando non avranno fede in me dopo tutti i miracoli che ho fatto in mezzo a loro?
\par 12 Io lo colpirò con la peste, e lo distruggerò, ma farò di te una nazione più grande e più potente di lui'.
\par 13 E Mosè disse all'Eterno: 'Ma l'udranno gli Egiziani, di mezzo ai quali tu hai fatto salire questo popolo per la tua potenza,
\par 14 e la cosa sarà risaputa dagli abitanti di questo paese. Essi hanno udito che tu, o Eterno, sei nel mezzo di questo popolo, che apparisci loro faccia a faccia, che la tua nuvola si ferma sopra loro, e che cammini davanti a loro il giorno in una colonna di nuvola, e la notte in una colonna di fuoco;
\par 15 ora, se fai perire questo popolo come un sol uomo, le nazioni che hanno udito la tua fama, diranno:
\par 16 Siccome l'Eterno non è stato capace di far entrare questo popolo nel paese che avea giurato di dargli, li ha scannati nel deserto.
\par 17 E ora si mostri, ti prego, la potenza del Signore nella sua grandezza, come tu hai promesso dicendo:
\par 18 L'Eterno è lento all'ira e grande in benignità; egli perdona l'iniquità e il peccato, ma non lascia impunito il colpevole, e punisce l'iniquità dei padri sui figliuoli, fino alla terza e alla quarta generazione.
\par 19 Deh, perdona l'iniquità di questo popolo, secondo la grandezza della tua benignità, nel modo che hai perdonato a questo popolo dall'Egitto fin qui'.
\par 20 E l'Eterno disse: 'Io perdono, come tu hai chiesto;
\par 21 ma, com'è vero ch'io vivo, tutta la terra sarà ripiena della gloria dell'Eterno,
\par 22 e tutti quegli uomini che hanno veduto la mia gloria e i miracoli che ho fatto in Egitto e nel deserto, e nonostante m'hanno tentato già dieci volte e non hanno ubbidito alla mia voce,
\par 23 certo non vedranno il paese che promisi con giuramento ai loro padri. Nessuno di quelli che m'hanno disprezzato lo vedrà; ma il mio servo Caleb,
\par 24 siccome è stato animato da un altro spirito e m'ha seguito appieno, io lo introdurrò nel paese nel quale è andato; e la sua progenie lo possederà.
\par 25 Or gli Amalekiti e i Cananei abitano nella valle; domani tornate addietro, incamminatevi verso il deserto, in direzione del mar Rosso'.
\par 26 L'Eterno parlò ancora a Mosè e ad Aaronne, dicendo:
\par 27 'Fino a quando sopporterò io questa malvagia raunanza che mormora contro di me? Io ho udito i mormorii che i figliuoli d'Israele fanno contro di me.
\par 28 Di' loro: Com'è vero ch'io vivo, dice l'Eterno, io vi farò quello che ho sentito dire da voi.
\par 29 I vostri cadaveri cadranno in questo deserto; e voi tutti, quanti siete, di cui s'è fatto il censimento, dall'età di venti anni in su, e che avete mormorato contro di me,
\par 30 non entrerete di certo nel paese nel quale giurai di farvi abitare; salvo Caleb, figliuolo di Gefunne, e Giosuè, figliuolo di Nun.
\par 31 I vostri piccini, che avete detto sarebbero preda de' nemici, quelli vi farò entrare; ed essi conosceranno il paese che voi avete disdegnato.
\par 32 Ma quanto a voi, i vostri cadaveri cadranno in questo deserto.
\par 33 E i vostri figliuoli andran pascendo i greggi nel deserto per quarant'anni e porteranno la pena delle vostre infedeltà, finché i vostri cadaveri non siano consunti nel deserto.
\par 34 Come avete messo quaranta giorni a esplorare il paese, porterete la pena delle vostre iniquità quarant'anni; un anno per ogni giorno; e saprete che cosa sia incorrere nella mia disgrazia.
\par 35 Io, l'Eterno, ho parlato; certo, così farò a tutta questa malvagia raunanza, la quale s'è messa assieme contro di me; in questo deserto saranno consunti; quivi morranno'.
\par 36 E gli uomini che Mosè avea mandato ad esplorare il paese e che, tornati, avean fatto mormorare tutta la raunanza contro di lui screditando il paese,
\par 37 quegli uomini, dico, che aveano screditato il paese, morirono colpiti da una piaga, dinanzi all'Eterno.
\par 38 Ma Giosuè, figliuolo di Nun, e Caleb, figliuolo di Gefunne, rimasero vivi fra quelli ch'erano andati ad esplorare il paese.
\par 39 Or Mosè riferì quelle parole a tutti i figliuoli d'Israele; e il popolo ne fece gran cordoglio.
\par 40 E la mattina si levarono di buon'ora e salirono sulla cima del monte, dicendo: 'Eccoci qua; noi saliremo al luogo di cui ha parlato l'Eterno, poiché abbiamo peccato'.
\par 41 Ma Mosè disse: 'Perché trasgredite l'ordine dell'Eterno? La cosa non v'andrà bene.
\par 42 Non salite, perché l'Eterno non è in mezzo a voi; che non abbiate ad essere sconfitti dai vostri nemici!
\par 43 Poiché là, di fronte a voi, stanno gli Amalekiti e i Cananei, e voi cadrete per la spada; giacché vi siete sviati dall'Eterno, l'Eterno non sarà con voi'.
\par 44 Nondimeno, s'ostinarono a salire sulla cima del monte; ma l'arca del patto dell'Eterno e Mosè non si mossero di mezzo al campo.
\par 45 Allora gli Amalekiti e i Cananei che abitavano su quel monte scesero giù, li batterono, e li fecero a pezzi fino a Hormah.

\chapter{15}

\par 1 Poi l'Eterno parlò a Mosè, dicendo:
\par 2 'Parla ai figliuoli d'Israele e di' loro: Quando sarete entrati nel paese che dovrete abitare e che io vi do,
\par 3 e offrirete all'Eterno un sacrifizio fatto mediante il fuoco, olocausto o sacrifizio, per adempimento d'un voto o come offerta volontaria, o nelle vostre feste solenni, per fare un profumo soave all'Eterno col vostro grosso o minuto bestiame,
\par 4 colui che presenterà la sua offerta all'Eterno, offrirà come oblazione un decimo d'efa di fior di farina stemperata col quarto di un hin d'olio,
\par 5 e farai una libazione d'un quarto di hin di vino con l'olocausto o il sacrifizio, per ogni agnello.
\par 6 Se è per un montone, offrirai come oblazione due decimi d'efa di fior di farina stemperata col terzo di un hin d'olio,
\par 7 e farai una libazione d'un terzo di hin di vino come offerta di odor soave all'Eterno.
\par 8 E se offri un giovenco come olocausto o come sacrifizio, per adempimento d'un voto o come sacrifizio d'azioni di grazie all'Eterno,
\par 9 si offrirà, col giovenco, come oblazione, tre decimi d'efa di fior di farina stemperata con la metà di un hin d'olio,
\par 10 e farai una libazione di un mezzo hin di vino: è un sacrifizio fatto mediante il fuoco, di soave odore all'Eterno.
\par 11 Così si farà per ogni bue, per ogni montone, per ogni agnello o capretto.
\par 12 Qualunque sia il numero degli animali che immolerete, farete così per ciascuna vittima.
\par 13 Tutti quelli che sono nativi del paese faranno le cose così, quando offriranno un sacrifizio fatto mediante il fuoco, di soave odore all'Eterno.
\par 14 E se uno straniero che soggiorna da voi, o chiunque dimori fra voi nel futuro, offre un sacrifizio fatto mediante il fuoco, di soave odore all'Eterno, farà come fate voi.
\par 15 Vi sarà una sola legge per tutta l'assemblea, per voi e per lo straniero che soggiorna fra voi; sarà una legge perpetua, di generazione in generazione; come siete voi, così sarà lo straniero davanti all'Eterno.
\par 16 Ci sarà una stessa legge e uno stesso diritto per voi e per lo straniero che soggiorna da voi'.
\par 17 L'Eterno parlò ancora a Mosè, dicendo:
\par 18 'Parla ai figliuoli d'Israele e di' loro: Quando sarete arrivati nel paese dove io vi conduco,
\par 19 e mangerete del pane di quel paese, ne preleverete un'offerta da presentare all'Eterno.
\par 20 Delle primizie della vostra pasta metterete da parte una focaccia come offerta; la metterete da parte, come si mette da parte l'offerta dell'aia.
\par 21 Delle primizie della vostra pasta darete all'Eterno una parte come offerta, di generazione in generazione.
\par 22 Quando avrete errato e non avrete osservato tutti questi comandamenti che l'Eterno ha dati a Mosè,
\par 23 tutto quello che l'Eterno vi ha comandato per mezzo di Mosè, dal giorno che l'Eterno vi ha dato dei comandamenti e in appresso, nelle vostre successive generazioni,
\par 24 se il peccato è stato commesso per errore, senza che la raunanza se ne sia accorta, tutta la raunanza offrirà un giovenco come olocausto di soave odore all'Eterno, con la sua oblazione e la sua libazione secondo le norme stabilite, e un capro come sacrifizio per il peccato.
\par 25 E il sacerdote farà l'espiazione per tutta la raunanza dei figliuoli d'Israele, e sarà loro perdonato, perché è stato un peccato commesso per errore, ed essi hanno portato la loro offerta, un sacrifizio fatto all'Eterno mediante il fuoco, e il loro sacrifizio per il peccato dinanzi all'Eterno, a causa del loro errore.
\par 26 Sarà perdonato a tutta la raunanza de' figliuoli d'Israele e allo straniero che soggiorna in mezzo a loro, perché tutto il popolo ha peccato per errore.
\par 27 Se è una persona sola che pecca per errore, offre una capra d'un anno come sacrifizio per il peccato.
\par 28 E il sacerdote farà l'espiazione dinanzi all'Eterno per la persona che avrà mancato commettendo un peccato per errore; e quando avrà fatta l'espiazione per essa, le sarà perdonato.
\par 29 Sia che si tratti d'un nativo del paese tra i figliuoli d'Israele o d'uno straniero che soggiorna fra voi, avrete un'unica legge per colui che pecca per errore.
\par 30 Ma la persona che agisce con proposito deliberato, sia nativo del paese o straniero, oltraggia l'Eterno; quella persona sarà sterminata di fra il suo popolo.
\par 31 Siccome ha sprezzato la parola dell'Eterno e ha violato il suo comandamento, quella persona dovrà essere sterminata; porterà il peso della sua iniquità'.
\par 32 Or mentre i figliuoli d'Israele erano nel deserto, trovarono un uomo che raccoglieva delle legna in giorno di sabato.
\par 33 Quelli che l'aveano trovato a raccogliere le legna lo menarono a Mosè, ad Aaronne e a tutta la raunanza.
\par 34 E lo misero in prigione, perché non era ancora stato stabilito che cosa gli si dovesse fare.
\par 35 E l'Eterno disse a Mosè: 'Quell'uomo dev'esser messo a morte; tutta la raunanza lo lapiderà fuori del campo'.
\par 36 Tutta la raunanza lo menò fuori del campo e lo lapidò; e quello morì, secondo l'ordine che l'Eterno avea dato a Mosè.
\par 37 L'Eterno parlò ancora a Mosè, dicendo:
\par 38 'Parla ai figliuoli d'Israele e di' loro che si facciano, di generazione in generazione, delle nappe agli angoli delle loro vesti, e che mettano alla nappa d'ogni angolo un cordone violetto.
\par 39 Sarà questa una nappa d'ornamento, e quando la guarderete, vi ricorderete di tutti i comandamenti dell'Eterno per metterli in pratica; e non andrete vagando dietro ai desideri del vostro cuore e dei vostri occhi che vi trascinano alla infedeltà.
\par 40 Così vi ricorderete di tutti i miei comandamenti, li metterete in pratica, e sarete santi al vostro Dio.
\par 41 Io sono l'Eterno, il vostro Dio, che vi ho tratti dal paese d'Egitto per essere vostro Dio. Io sono l'Eterno, l'Iddio vostro'.

\chapter{16}

\par 1 Or Kore, figliuolo di Itshar, figliuolo di Kehath, figliuolo di Levi, insieme con Dathan e Abiram figliuolo di Eliab, e On, figliuolo di Peleth, tutti e tre figliuoli di Ruben,
\par 2 presero altra gente e si levaron su in presenza di Mosè, con duecentocinquanta uomini dei figliuoli d'Israele, principi della raunanza, membri del consiglio, uomini di grido;
\par 3 e, radunatisi contro Mosè e contro Aaronne, dissero loro: 'Basta! tutta la raunanza, tutti fino ad uno son santi, e l'Eterno è in mezzo a loro; perché dunque v'innalzate voi sopra la raunanza dell'Eterno?'
\par 4 Quando Mosè ebbe udito questo, si prostrò colla faccia a terra;
\par 5 poi parlò a Kore e a tutta la gente ch'era con lui, dicendo: 'Domattina l'Eterno farà conoscere chi è suo e chi è santo, e se lo farà avvicinare: farà avvicinare a sé colui ch'egli avrà scelto.
\par 6 Fate questo: prendete de' turiboli, tu, Kore, e tutta la gente che è con te;
\par 7 e domani mettetevi del fuoco, e ponetevi su del profumo dinanzi all'Eterno; e colui che l'Eterno avrà scelto sarà santo. Basta, figliuoli di Levi!'
\par 8 Mosè disse inoltre a Kore: 'Ora ascoltate, o figliuoli di Levi!
\par 9 È egli poco per voi che l'Iddio d'Israele v'abbia appartati dalla raunanza d'Israele e v'abbia fatto accostare a sé per fare il servizio del tabernacolo dell'Eterno e per tenervi davanti alla raunanza affin d'esercitare a pro suo il vostro ministerio?
\par 10 Egli vi fa accostare a sé, te e tutti i tuoi fratelli figliuoli di Levi con te, e cercate anche il sacerdozio?
\par 11 E per questo tu e tutta la gente che è teco vi siete radunati contro l'Eterno! poiché chi è Aaronne che vi mettiate a mormorare contro di lui?'
\par 12 E Mosè mandò a chiamare Dathan e Abiram, figliuoli di Eliab; ma essi dissero: 'Noi non saliremo.
\par 13 È egli poco per te l'averci tratti fuori da un paese ove scorre il latte e il miele, per farci morire nel deserto, che tu voglia anche farla da principe, sì, da principe su noi?
\par 14 E poi, non ci hai davvero condotti in un paese dove scorra il latte e il miele, e non ci hai dato possessi di campi e di vigne! Credi tu di potere render cieca questa gente? Noi non saliremo'.
\par 15 Allora Mosè si adirò forte e disse all'Eterno: 'Non gradire la loro oblazione; io non ho preso da costoro neppure un asino, e non ho fatto torto ad alcuno di loro'.
\par 16 Poi Mosè disse a Kore: 'Tu e tutta la tua gente trovatevi domani davanti all'Eterno: tu e loro, con Aaronne;
\par 17 e ciascun di voi prenda il suo turibolo, vi metta del profumo, e porti ciascuno il suo turibolo davanti all'Eterno: saranno duecentocinquanta turiboli. Anche tu ed Aaronne prenderete ciascuno il vostro turibolo'.
\par 18 Essi dunque presero ciascuno il suo turibolo, vi misero del fuoco, e vi posero su del profumo, e si fermarono all'ingresso della tenda di convegno; lo stesso fecero Mosè ed Aaronne.
\par 19 E Kore convocò tutta la raunanza contro Mosè ed Aaronne all'ingresso della tenda di convegno; e la gloria dell'Eterno apparve a tutta la raunanza.
\par 20 E l'Eterno parlò a Mosè e ad Aaronne, dicendo:
\par 21 'Separatevi da questa raunanza, e io li consumerò in un attimo'.
\par 22 Ma essi, prostratisi con la faccia a terra, dissero: 'O Dio, Dio degli spiriti d'ogni carne! Un uomo solo ha peccato, e ti adireresti tu contro tutta la raunanza?'
\par 23 E l'Eterno parlò a Mosè, dicendo:
\par 24 'Parla alla raunanza e dille: Ritiratevi d'intorno alla dimora di Kore, di Dathan e di Abiram'.
\par 25 Mosè si levò e andò da Dathan e da Abiram; e gli anziani d'Israele lo seguirono.
\par 26 Ed egli parlò alla raunanza, dicendo: 'Allontanatevi dalle tende di questi uomini malvagi, e non toccate nulla di ciò ch'è loro, affinché non abbiate a perire a cagione di tutti i loro peccati'.
\par 27 Così quelli si ritirarono d'intorno alla dimora di Kore, di Dathan e di Abiram. Dathan ed Abiram uscirono, e si fermarono all'ingresso delle loro tende con le loro mogli, i loro figliuoli e i loro piccini. E Mosè disse:
\par 28 'Da questo conoscerete che l'Eterno mi ha mandato per fare tutte queste cose, e che io non le ho fatte di mia testa.
\par 29 Se questa gente muore come muoion tutti gli uomini, se la loro sorte è la sorte comune a tutti gli uomini, l'Eterno non mi ha mandato;
\par 30 ma se l'Eterno fa una cosa nuova, se la terra apre la sua bocca e li ingoia con tutto quello che appartiene loro e s'essi scendono vivi nel soggiorno dei morti, allora riconoscerete che questi uomini hanno disprezzato l'Eterno'.
\par 31 E avvenne, com'egli ebbe finito di proferire tutte queste parole, che il suolo si spaccò sotto i piedi di coloro,
\par 32 la terra spalancò la sua bocca e li ingoiò: essi e le loro famiglie, con tutta la gente che apparteneva a Kore, e tutta la loro roba.
\par 33 E scesero vivi nel soggiorno de' morti; la terra si richiuse su loro, ed essi scomparvero di mezzo all'assemblea.
\par 34 Tutto Israele ch'era attorno ad essi fuggì alle loro grida; perché dicevano: 'Che la terra non inghiottisca noi pure!'
\par 35 E un fuoco uscì dalla presenza dell'Eterno e divorò i duecentocinquanta uomini che offrivano il profumo.
\par 36 Poi l'Eterno parlò a Mosè, dicendo:
\par 37 'Di' a Eleazar, figliuolo del sacerdote Aaronne, di trarre i turiboli di mezzo all'incendio e di disperdere qua e là il fuoco, perché quelli son sacri;
\par 38 e dei turiboli di quegli uomini che hanno peccato al prezzo della loro vita si facciano tante lamine battute per rivestirne l'altare, poiché sono stati presentati davanti all'Eterno e quindi son sacri; e serviranno di segno ai figliuoli d'Israele'.
\par 39 E il sacerdote Eleazar prese i turiboli di rame presentati dagli uomini ch'erano stati arsi; e furon tirati in lamine per rivestirne l'altare,
\par 40 affinché servissero di ricordanza ai figliuoli d'Israele, e niun estraneo che non sia della progenie d'Aaronne s'accosti ad arder profumo davanti all'Eterno ed abbia la sorte di Kore e di quelli ch'eran con lui. Eleazar fece come l'Eterno gli avea detto per mezzo di Mosè.
\par 41 Il giorno seguente, tutta la raunanza de' figliuoli d'Israele mormorò contro Mosè ed Aaronne dicendo: 'Voi avete fatto morire il popolo dell'Eterno'.
\par 42 E avvenne che, come la raunanza si faceva numerosa contro Mosè e contro Aaronne, i figliuoli d'Israele si volsero verso la tenda di convegno; ed ecco che la nuvola la ricoprì, e apparve la gloria dell'Eterno.
\par 43 Mosè ed Aaronne vennero davanti alla tenda di convegno.
\par 44 E l'Eterno parlò a Mosè, dicendo:
\par 45 'Toglietevi di mezzo a questa raunanza, e io li consumerò in un attimo'. Ed essi si prostrarono con la faccia a terra.
\par 46 E Mosè disse ad Aaronne: 'Prendi il turibolo, mettivi del fuoco di sull'altare, ponvi su del profumo, e portalo presto in mezzo alla raunanza e fa' espiazione per essi; poiché l'ira dell'Eterno è scoppiata, la piaga è già cominciata'.
\par 47 E Aaronne prese il turibolo, come Mosè avea detto; corse in mezzo all'assemblea, ed ecco che la piaga era già cominciata fra il popolo; mise il profumo nel turibolo e fece l'espiazione per il popolo.
\par 48 E si fermò tra i morti e i vivi, e la piaga fu arrestata.
\par 49 Or quelli che morirono di quella piaga furono quattordicimilasettecento, oltre quelli che morirono per il fatto di Kore.
\par 50 Aaronne tornò a Mosè all'ingresso della tenda di convegno e la piaga fu arrestata.

\chapter{17}

\par 1 Poi l'Eterno parlò a Mosè, dicendo:
\par 2 'Parla ai figliuoli d'Israele, e fatti dare da loro delle verghe: una per ogni casa dei loro padri: cioè, dodici verghe da parte di tutti i loro principi secondo le case dei loro padri; scriverai il nome d'ognuno sulla sua verga;
\par 3 e scriverai il nome d'Aaronne sulla verga di Levi; poiché ci sarà una verga per ogni capo delle case dei loro padri.
\par 4 E riporrai quelle verghe nella tenda di convegno, davanti alla testimonianza, dove io mi ritrovo con voi.
\par 5 E avverrà che l'uomo che io avrò scelto sarà quello la cui verga fiorirà; e farò cessare davanti a me i mormorii che i figliuoli d'Israele fanno contro di voi'.
\par 6 E Mosè parlò ai figliuoli d'Israele, e tutti i loro principi gli dettero una verga per uno, secondo le case dei loro padri: cioè, dodici verghe; e la verga d'Aaronne era in mezzo alle verghe loro.
\par 7 E Mosè ripose quelle verghe davanti all'Eterno nella tenda della testimonianza.
\par 8 E avvenne, l'indomani, che Mosè entrò nella tenda della testimonianza; ed ecco che la verga d'Aaronne per la casa di Levi aveva fiorito, gettato dei bottoni, sbocciato dei fiori e maturato delle mandorle.
\par 9 Allora Mosè tolse tutte le verghe di davanti all'Eterno e le portò a tutti i figliuoli d'Israele; ed essi le videro e presero ciascuno la sua verga.
\par 10 E l'Eterno disse a Mosè: 'Riporta la verga d'Aaronne davanti alla testimonianza, perché sia conservata come un segno ai ribelli; onde sia messo fine ai loro mormorii contro di me, ed essi non muoiano'.
\par 11 Mosè fece così; fece come l'Eterno gli avea comandato.
\par 12 E i figliuoli d'Israele dissero a Mosè: 'Ecco, periamo! siam perduti! siam tutti perduti!
\par 13 Chiunque s'accosta, chiunque s'accosta al tabernacolo dell'Eterno, muore; dovrem perire tutti quanti?'

\chapter{18}

\par 1 E l'Eterno disse ad Aaronne: 'Tu, i tuoi figliuoli e la casa di tuo padre con te porterete il peso delle iniquità commesse nel santuario; e tu e i tuoi figliuoli porterete il peso delle iniquità commesse nell'esercizio del vostro sacerdozio.
\par 2 E anche i tuoi fratelli, la tribù di Levi, la tribù di tuo padre, farai accostare a te, affinché ti siano aggiunti e ti servano quando tu e i tuoi figliuoli con te sarete davanti alla tenda della testimonianza.
\par 3 Essi faranno il servizio sotto i tuoi ordini in tutto quel che concerne la tenda; soltanto non si accosteranno agli utensili del santuario né all'altare affinché non moriate e gli uni e gli altri.
\par 4 Essi ti saranno dunque aggiunti, e faranno il servizio della tenda di convegno in tutto ciò che la concerne, e nessun estraneo s'accosterà a voi.
\par 5 E voi farete il servizio del santuario e dell'altare affinché non vi sia più ira contro i figliuoli d'Israele.
\par 6 Quanto a me, ecco, io ho preso i vostri fratelli, i Leviti, di mezzo ai figliuoli d'Israele; dati all'Eterno, essi son rimessi in dono a voi per fare il servizio della tenda di convegno.
\par 7 E tu e i tuoi figliuoli con te eserciterete il vostro sacerdozio in tutto ciò che concerne l'altare e ciò ch'è di là dal velo; e farete il vostro servizio. Io vi do l'esercizio del sacerdozio come un dono; l'estraneo che si accosterà sarà messo a morte'.
\par 8 L'Eterno disse ancora ad Aaronne: 'Ecco, di tutte le cose consacrate dai figliuoli d'Israele io ti do quelle che mi sono offerte per elevazione: io te le do, a te e ai tuoi figliuoli, come diritto d'unzione, per legge perpetua.
\par 9 Questo ti apparterrà fra le cose santissime non consumate dal fuoco: tutte le loro offerte, vale a dire ogni oblazione, ogni sacrifizio per il peccato e ogni sacrifizio di riparazione che mi presenteranno; son tutte cose santissime che apparterranno a te ed ai tuoi figliuoli.
\par 10 Le mangerai in luogo santissimo; ne mangerà ogni maschio; ti saranno cose sante.
\par 11 Questo ancora ti apparterrà: i doni che i figliuoli d'Israele presenteranno per elevazione, e tutte le loro offerte agitate; io le do a te, ai tuoi figliuoli e alle tue figliuole con te, per legge perpetua. Chiunque sarà puro in casa tua ne potrà mangiare.
\par 12 Ti do pure tutte le primizie ch'essi offriranno all'Eterno: il meglio dell'olio e il meglio del mosto e del grano.
\par 13 Le primizie di tutto ciò che produrrà la loro terra e ch'essi presenteranno all'Eterno saranno tue. Chiunque sarà puro in casa tua ne potrà mangiare.
\par 14 Tutto ciò che sarà consacrato per voto d'interdetto in Israele sarà tuo.
\par 15 Ogni primogenito d'ogni carne ch'essi offriranno all'Eterno, così degli uomini come degli animali, sarà tuo; però, farai riscattare il primogenito dell'uomo, e farai parimente riscattare il primogenito d'un animale impuro.
\par 16 E quanto al riscatto, li farai riscattare dall'età di un mese, secondo la tua stima, per cinque sicli d'argento, a siclo di santuario, che è di venti ghere.
\par 17 Ma non farai riscattare il primogenito della vacca né il primogenito della pecora né il primogenito della capra; sono cosa sacra; spanderai il loro sangue sull'altare, e farai fumare il loro grasso come sacrifizio fatto mediante il fuoco, di soave odore all'Eterno.
\par 18 La loro carne sarà tua; sarà tua come il petto dell'offerta agitata e come la coscia destra.
\par 19 Io ti do, a te, ai tuoi figliuoli e alle tue figliuole con te, per legge perpetua, tutte le offerte di cose sante che i figliuoli d'Israele presenteranno all'Eterno per elevazione. È un patto inalterabile, perpetuo, dinanzi all'Eterno, per te e per la tua progenie con te'.
\par 20 L'Eterno disse ancora ad Aaronne: 'Tu non avrai alcun possesso nel loro paese, e non ci sarà parte per te in mezzo a loro; io sono la tua parte e il tuo possesso in mezzo ai figliuoli d'Israele.
\par 21 E ai figliuoli di Levi io do come possesso tutte le decime in Israele in contraccambio del servizio che fanno, il servizio della tenda di convegno.
\par 22 E i figliuoli d'Israele non s'accosteranno più alla tenda di convegno, per non caricarsi d'un peccato che li trarrebbe a morte.
\par 23 Ma il servizio della tenda di convegno lo faranno soltanto i Leviti; ed essi porteranno il peso delle proprie iniquità; sarà una legge perpetua, di generazione in generazione; e non possederanno nulla tra i figliuoli d'Israele;
\par 24 poiché io do come possesso ai Leviti le decime che i figliuoli d'Israele presenteranno all'Eterno come offerta elevata; per questo dico di loro: Non possederanno nulla tra i figliuoli d'Israele'.
\par 25 E l'Eterno parlò a Mosè, dicendo:
\par 26 'Parlerai inoltre ai Leviti e dirai loro: Quando riceverete dai figliuoli d'Israele le decime che io vi do per conto loro come vostro possesso, ne metterete da parte un'offerta da fare all'Eterno: una decima della decima;
\par 27 e l'offerta che avrete prelevata vi sarà contata come il grano che vien dall'aia e come il mosto che esce dallo strettoio.
\par 28 Così anche voi metterete da parte un'offerta per l'Eterno da tutte le decime che riceverete dai figliuoli d'Israele, e darete al sacerdote Aaronne l'offerta che avrete messa da parte per l'Eterno.
\par 29 Da tutte le cose che vi saranno donate metterete da parte tutte le offerte per l'Eterno; di tutto ciò che vi sarà di meglio metterete da parte quel tanto ch'è da consacrare.
\par 30 E dirai loro: Quando ne avrete messo da parte il meglio, quel che rimane sarà contato ai Leviti come il provento dell'aia e come il provento dello strettoio.
\par 31 E lo potrete mangiare in qualunque luogo, voi e le vostre famiglie, perché è la vostra mercede, in contraccambio del vostro servizio nella tenda di convegno.
\par 32 E così non vi caricherete d'alcun peccato, giacché ne avrete messo da parte il meglio; e non profanerete le cose sante de' figliuoli d'Israele, e non morrete'.

\chapter{19}

\par 1 L'Eterno parlò ancora a Mosè e ad Aaronne, dicendo:
\par 2 'Questo è l'ordine della legge che l'Eterno ha prescritta dicendo: Di' ai figliuoli d'Israele che ti menino una giovenca rossa, senza macchia, senza difetti, e che non abbia mai portato il giogo.
\par 3 E la darete al sacerdote Eleazar, che la condurrà fuori del campo e la farà scannare in sua presenza.
\par 4 Il sacerdote Eleazar prenderà col dito del sangue della giovenca e ne farà sette volte l'aspersione dal lato dell'ingresso della tenda di convegno;
\par 5 poi si brucerà la giovenca sotto gli occhi di lui; se ne brucerà la pelle, la carne e il sangue con i suoi escrementi.
\par 6 Il sacerdote prenderà quindi del legno di cedro, dell'issopo, della stoffa scarlatta, e getterà tutto in mezzo al fuoco che consuma la giovenca.
\par 7 Poi il sacerdote si laverà le vesti ed il corpo nell'acqua; dopo di che rientrerà nel campo, e il sacerdote sarà impuro fino alla sera.
\par 8 E colui che avrà bruciato la giovenca si laverà le vesti nell'acqua, farà un'abluzione del corpo nell'acqua, e sarà impuro fino alla sera.
\par 9 Un uomo puro raccoglierà le ceneri della giovenca e le depositerà fuori del campo in luogo puro, dove saranno conservate per la raunanza de' figliuoli d'Israele come acqua di purificazione: è un sacrifizio per il peccato.
\par 10 E colui che avrà raccolto le ceneri della giovenca si laverà le vesti e sarà impuro fino alla sera. E questa sarà una legge perpetua per i figliuoli d'Israele e per lo straniero che soggiornerà da loro:
\par 11 chi avrà toccato il cadavere di una persona umana sarà impuro sette giorni.
\par 12 Quand'uno si sarà purificato con quell'acqua il terzo e il settimo giorno, sarà puro; ma se non si purifica il terzo ed il settimo giorno, non sarà puro.
\par 13 Chiunque avrà toccato un morto, il corpo d'una persona umana che sia morta e non si sarà purificato, avrà contaminato la dimora dell'Eterno; e quel tale sarà sterminato di mezzo a Israele. Siccome l'acqua di purificazione non è stata spruzzata su lui, egli è impuro; ha ancora addosso la sua impurità.
\par 14 Questa è la legge: Quando un uomo sarà morto in una tenda, chiunque entrerà nella tenda e chiunque sarà nella tenda sarà impuro sette giorni.
\par 15 E ogni vaso scoperto sul quale non sia coperchio attaccato, sarà impuro.
\par 16 E chiunque, per i campi, avrà toccato un uomo ucciso per la spada o morto da sé, o un osso d'uomo, o un sepolcro, sarà impuro sette giorni.
\par 17 E per colui che sarà divenuto impuro si prenderà della cenere della vittima arsa per il peccato, e vi si verserà su dell'acqua viva, in un vaso:
\par 18 poi un uomo puro prenderà dell'issopo, lo intingerà nell'acqua, e ne spruzzerà la tenda, tutti gli utensili e tutte le persone che son quivi, e colui che ha toccato l'osso o l'ucciso o il morto da sé o il sepolcro.
\par 19 L'uomo puro spruzzerà l'impuro il terzo giorno e il settimo giorno, e lo purificherà il settimo giorno; poi colui ch'è stato immondo si laverà le vesti, laverà se stesso nell'acqua, e sarà puro la sera.
\par 20 Ma colui che divenuto impuro non si purificherà, sarà sterminato di mezzo alla raunanza, perché ha contaminato il santuario dell'Eterno; l'acqua della purificazione non è stata spruzzata su lui; è impuro.
\par 21 Sarà per loro una legge perpetua: Colui che avrà spruzzato l'acqua di purificazione si laverà le vesti; e chi avrà toccato l'acqua di purificazione sarà impuro fino alla sera.
\par 22 E tutto quello che l'impuro avrà toccato sarà impuro; e la persona che avrà toccato lui sarà impura fino alla sera'.

\chapter{20}

\par 1 Or tutta la raunanza dei figliuoli d'Israele arrivò al deserto di Tsin il primo mese, e il popolo si fermò a Kades. Quivi morì e fu sepolta Maria.
\par 2 E mancava l'acqua per la raunanza; onde ci fu assembramento contro Mosè e contro Aaronne.
\par 3 E il popolo contese con Mosè, dicendo: 'Fossimo pur morti quando morirono i nostri fratelli davanti all'Eterno!
\par 4 E perché avete menato la raunanza dell'Eterno in questo deserto per morirvi noi e il nostro bestiame?
\par 5 E perché ci avete fatti salire dall'Egitto per menarci in questo tristo luogo? Non è un luogo dove si possa seminare; non ci son fichi, non vigne, non melagrane, e non c'è acqua da bere'.
\par 6 Allora Mosè ed Aaronne s'allontanarono dalla raunanza per recarsi all'ingresso della tenda di convegno; si prostrarono con la faccia in terra, e la gloria dell'Eterno apparve loro.
\par 7 E l'Eterno parlò a Mosè, dicendo:
\par 8 'Prendi il bastone; e tu e tuo fratello Aaronne convocate la raunanza e parlate a quel sasso, in loro presenza, ed esso darà la sua acqua; e tu farai sgorgare per loro l'acqua dal sasso, e darai da bere alla raunanza e al suo bestiame'.
\par 9 Mosè dunque prese il bastone ch'era davanti all'Eterno, come l'Eterno gli aveva ordinato.
\par 10 E Mosè ed Aaronne convocarono la raunanza dirimpetto al sasso, e Mosè disse loro: 'Ora ascoltate, o ribelli; vi farem noi uscir dell'acqua da questo sasso?'
\par 11 E Mosè alzò la mano, percosse il sasso col suo bastone due volte, e ne uscì dell'acqua in abbondanza; e la raunanza e il suo bestiame bevvero.
\par 12 Poi l'Eterno disse a Mosè e ad Aaronne: 'Siccome non avete avuto fiducia in me per dar gloria al mio santo nome agli occhi dei figliuoli d'Israele, voi non introdurrete questa raunanza nel paese che io le do'.
\par 13 Queste sono le acque di Meriba dove i figliuoli d'Israele contesero con l'Eterno che si fece riconoscere come il Santo in mezzo a loro.
\par 14 Poi Mosè mandò da Kades degli ambasciatori al re di Edom per dirgli: 'Così dice Israele tuo fratello: Tu sai tutte le tribolazioni che ci sono avvenute:
\par 15 come i nostri padri scesero in Egitto e noi in Egitto dimorammo per lungo tempo e gli Egiziani maltrattaron noi e i nostri padri.
\par 16 E noi gridammo all'Eterno ed egli udì la nostra voce e mandò un angelo e ci fece uscire dall'Egitto; ed eccoci ora in Kades, che è città agli estremi tuoi confini.
\par 17 Deh, lasciaci passare per il tuo paese; noi non passeremo né per campi né per vigne, e non berremo l'acqua dei pozzi; seguiremo la strada pubblica senza deviare né a destra né a sinistra finché abbiamo oltrepassato i tuoi confini'.
\par 18 Ma Edom gli rispose: 'Tu non passerai sul mio territorio; altrimenti, ti verrò contro con la spada'.
\par 19 I figliuoli d'Israele gli dissero: 'Noi saliremo per la strada maestra; e se noi e il nostro bestiame berremo dell'acqua tua, te la pagheremo; lasciami semplicemente transitare a piedi'.
\par 20 Ma quello rispose: 'Non passerai!' E Edom mosse contro Israele con molta gente e con potente mano.
\par 21 Così Edom ricusò a Israele il transito per i suoi confini; onde Israele s'allontanò da lui.
\par 22 Tutta la raunanza de' figliuoli d'Israele si partì da Kades e arrivò al monte Hor.
\par 23 E l'Eterno parlò a Mosè e ad Aaronne al monte Hor sui confini del paese di Edom, dicendo:
\par 24 'Aaronne sta per esser raccolto presso il suo popolo, e non entrerà nel paese che ho dato ai figliuoli d'Israele, perché siete stati ribelli al mio comandamento alle acque di Meriba.
\par 25 Prendi Aaronne ed Eleazar suo figliuolo e falli salire sul monte Hor.
\par 26 Spoglia Aaronne de' suoi paramenti, e rivestine Eleazar suo figliuolo; quivi Aaronne sarà raccolto presso il suo popolo, e morrà'.
\par 27 E Mosè fece come l'Eterno aveva ordinato; ed essi salirono sul monte Hor, a vista di tutta la raunanza.
\par 28 Mosè spogliò Aaronne de' suoi paramenti, e ne rivestì Eleazar, figliuolo di lui; e Aaronne morì quivi sulla cima del monte. Poi Mosè ed Eleazar scesero dal monte.
\par 29 E quando tutta la raunanza vide che Aaronne era morto, tutta la casa d'Israele lo pianse per trenta giorni.

\chapter{21}

\par 1 Or il re cananeo di Arad, che abitava il mezzogiorno, avendo udito che Israele veniva per la via di Atharim, combatté contro Israele, e fece alcuni prigionieri.
\par 2 Allora Israele fece un voto all'Eterno, e disse: 'Se tu dài nelle mie mani questo popolo, le loro città saranno da me votate allo sterminio'.
\par 3 L'Eterno porse ascolto alla voce d'Israele e gli diede nelle mani i Cananei; e Israele votò allo sterminio i Cananei e le loro città, e a quel luogo fu posto nome Horma.
\par 4 Poi gl'Israeliti si partirono dal monte Hor, movendo verso il mar Rosso per fare il giro del paese di Edom; e il popolo si fe' impaziente nel viaggio.
\par 5 E il popolo parlò contro Dio e contro Mosè, dicendo: 'Perché ci avete fatti salire fuori d'Egitto per farci morire in questo deserto? Poiché qui non c'è né pane né acqua, e l'anima nostra è nauseata di questo cibo tanto leggero'.
\par 6 Allora l'Eterno mandò fra il popolo de' serpenti ardenti i quali mordevano la gente, e gran numero d'Israeliti morirono.
\par 7 Allora il popolo venne a Mosè e disse: 'Abbiamo peccato, perché abbiam parlato contro l'Eterno e contro te; prega l'Eterno che allontani da noi questi serpenti'. E Mosè pregò per il popolo.
\par 8 E l'Eterno disse a Mosè: 'Fatti un serpente ardente, e mettilo sopra un'antenna; e avverrà che chiunque sarà morso e lo guarderà, scamperà'.
\par 9 Mosè allora fece un serpente di rame e lo mise sopra un'antenna; e avveniva che, quando un serpente avea morso qualcuno, se questi guardava il serpente di rame, scampava.
\par 10 Poi i figliuoli d'Israele partirono e si accamparono a Oboth;
\par 11 e partitisi da Oboth, si accamparono a Ije-Abarim nel deserto ch'è dirimpetto a Moab dal lato dove sorge il sole.
\par 12 Di là si partirono e si accamparono nella valle di Zered.
\par 13 Poi si partirono di là e si accamparono dall'altro lato dell'Arnon, che scorre nel deserto e nasce sui confini degli Amorei; poiché l'Arnon è il confine di Moab, fra Moab e gli Amorei.
\par 14 Per questo è detto nel Libro delle Guerre dell'Eterno: "...Vaheb in Sufa, e le valli dell'Arnon
\par 15 e i declivi delle valli che si estendono verso le dimore di Ar, e s'appoggiano alla frontiera di Moab".
\par 16 E di là andarono a Beer, che è il pozzo a proposito del quale l'Eterno disse a Mosè: 'Raduna il popolo e io gli darò dell'acqua'.
\par 17 Fu in quell'occasione che Israele cantò questo cantico: "Scaturisci, o pozzo! Salutatelo con canti!
\par 18 Pozzo che i principi hanno scavato, che i nobili del popolo hanno aperto con lo scettro, coi loro bastoni!"
\par 19 Poi dal deserto andarono a Matthana; da Matthana a Nahaliel; da Nahaliel a Bamoth,
\par 20 e da Bamoth nella valle che è nella campagna di Moab, verso l'altura del Pisga che domina il deserto.
\par 21 Or Israele mandò ambasciatori a Sihon, re degli Amorei, per dirgli:
\par 22 'Lasciami passare per il tuo paese; noi non ci svieremo per i campi né per le vigne, non berremo l'acqua dei pozzi; seguiremo la strada pubblica finché abbiamo oltrepassato i tuoi confini'.
\par 23 Ma Sihon non permise a Israele di passare per i suoi confini; anzi radunò tutta la sua gente e uscì fuori contro Israele nel deserto; giunse a Jahats, e diè battaglia a Israele.
\par 24 Israele lo sconfisse passandolo a fil di spada, e conquistò il suo paese dall'Arnon fino al Jabbok, sino ai confini de' figliuoli di Ammon, poiché la frontiera dei figliuoli di Ammon era forte.
\par 25 E Israele prese tutte quelle città, e abitò in tutte le città degli Amorei: in Heshbon e in tutte le città del suo territorio;
\par 26 poiché Heshbon era la città di Sihon, re degli Amorei, il quale avea mosso guerra al precedente re di Moab, e gli avea tolto tutto il suo paese fino all'Arnon.
\par 27 Per questo dicono i poeti: "Venite a Heshbon! La città di Sihon sia ricostruita e fortificata!
\par 28 Poiché un fuoco è uscito da Heshbon, una fiamma dalla città di Sihon; essa ha divorato Ar di Moab,
\par 29 i padroni delle alture dell'Arnon. Guai a te, o Moab! Sei perduto, o popolo di Kemosh! Kemosh ha fatto de' suoi figliuoli tanti fuggiaschi, e ha dato le sue figliuole come schiave a Sihon, re degli Amorei.
\par 30 Noi abbiamo scagliato su loro le nostre frecce; Heshbon è distrutta fino a Dibon. Abbiam tutto devastato fino a Nofah, il fuoco è giunto fino a Medeba".
\par 31 Così Israele si stabilì nel paese degli Amorei.
\par 32 Poi Mosè mandò a esplorare Jaezer, e gl'Israeliti presero le città del suo territorio e ne cacciarono gli Amorei che vi si trovavano.
\par 33 E, mutata direzione, risalirono il paese in direzione di Bashan; e Og, re di Bashan, uscì contro loro con tutta la sua gente per dar loro battaglia a Edrei.
\par 34 Ma l'Eterno disse a Mosè: 'Non lo temere; poiché io lo do nelle tue mani: lui, tutta la sua gente e il suo paese; trattalo com'hai trattato Sihon, re degli Amorei che abitava a Heshbon'.
\par 35 E gli Israeliti batteron lui, coi suoi figliuoli e con tutto il suo popolo, in guisa che non gli rimase più anima viva; e s'impadronirono del suo paese.

\chapter{22}

\par 1 Poi i figliuoli d'Israele partirono e si accamparono nelle pianure di Moab, oltre il Giordano di Gerico.
\par 2 Or Balak, figliuolo di Tsippor, vide tutto quello che Israele avea fatto agli Amorei;
\par 3 e Moab ebbe grande paura di questo popolo, ch'era così numeroso; Moab fu preso d'angoscia a cagione de' figliuoli d'Israele.
\par 4 Onde Moab disse agli anziani di Madian: 'Ora questa moltitudine divorerà tutto ciò ch'è dintorno a noi, come il bue divora l'erba dei campi'. Or Balak, figliuolo di Tsippor era, in quel tempo, re di Moab.
\par 5 Egli mandò ambasciatori a Balaam, figliuolo di Beor, a Pethor che sta sul fiume, nel paese de' figliuoli del suo popolo per chiamarlo e dirgli: 'Ecco, un popolo è uscito d'Egitto; esso ricopre la faccia della terra, e si è stabilito dirimpetto a me;
\par 6 or dunque vieni, te ne prego, e maledicimi questo popolo; poiché è troppo potente per me; forse così riusciremo a sconfiggerlo, e potrò cacciarlo dal paese; poiché so che chi tu benedici è benedetto, e chi tu maledici è maledetto'.
\par 7 Gli anziani di Moab e gli anziani di Madian partirono portando in mano la mercede dell'indovino; e, arrivati da Balaam, gli riferirono le parole di Balak.
\par 8 E Balaam disse loro: 'Alloggiate qui stanotte; e vi darò la risposta secondo che mi dirà l'Eterno'. E i principi di Moab stettero da Balaam.
\par 9 Or Dio venne a Balaam e gli disse: 'Chi sono questi uomini che stanno da te?'
\par 10 E Balaam rispose a Dio: 'Balak, figliuolo di Tsippor, re di Moab, mi ha mandato a dire:
\par 11 Ecco, il popolo ch'è uscito d'Egitto ricopre la faccia della terra; or vieni a maledirmelo; forse riuscirò così a batterlo e potrò cacciarlo'.
\par 12 E Dio disse a Balaam: 'Tu non andrai con loro, non maledirai quel popolo, perché egli è benedetto'.
\par 13 Balaam si levò, la mattina, e disse ai principi di Balak: 'Andatevene al vostro paese, perché l'Eterno m'ha rifiutato il permesso di andare con voi'.
\par 14 E i principi di Moab si levarono, tornarono da Balak e dissero: 'Balaam ha rifiutato di venir con noi'.
\par 15 Allora Balak mandò di nuovo de' principi, in maggior numero e più ragguardevoli che que' di prima.
\par 16 I quali vennero da Balaam e gli dissero: 'Così dice Balak, figliuolo di Tsippor: Deh, nulla ti trattenga dal venire da me;
\par 17 poiché io ti ricolmerò di onori e farò tutto ciò che mi dirai; vieni dunque, te ne prego, e maledicimi questo popolo'.
\par 18 Ma Balaam rispose e disse ai servi di Balak: 'Quand'anche Balak mi desse la sua casa piena d'argento e d'oro, non potrei trasgredire l'ordine dell'Eterno, del mio Dio, per far cosa piccola o grande che fosse.
\par 19 Nondimeno, trattenetevi qui, anche voi, stanotte, ond'io sappia ciò che l'Eterno mi dirà ancora'.
\par 20 E Dio venne la notte a Balaam e gli disse: 'Se quegli uomini son venuti a chiamarti, lèvati e va' con loro; soltanto, farai ciò che io ti dirò'.
\par 21 Balaam quindi si levò la mattina, sellò la sua asina e se ne andò coi principi di Moab.
\par 22 Ma l'ira di Dio s'accese perché egli se n'era andato; e l'angelo dell'Eterno si pose sulla strada per fargli ostacolo. Or egli cavalcava la sua asina e avea seco due servitori.
\par 23 L'asina, vedendo l'angelo dell'Eterno che stava sulla strada con la sua spada sguainata in mano, uscì di via e cominciava ad andare per i campi. Balaam percosse l'asina per rimetterla sulla strada.
\par 24 Allora l'angelo dell'Eterno si fermò in un sentiero incavato che passava tra le vigne e aveva un muro di qua e un muro di là.
\par 25 L'asina vide l'angelo dell'Eterno; si serrò al muro e strinse il piede di Balaam al muro; e Balaam la percosse di nuovo.
\par 26 L'angelo dell'Eterno passò di nuovo oltre, e si fermò in un luogo stretto dove non c'era modo di volgersi né a destra né a sinistra.
\par 27 L'asina vide l'angelo dell'Eterno e si sdraiò sotto Balaam; l'ira di Balaam s'accese, ed egli percosse l'asina con un bastone.
\par 28 Allora l'Eterno aprì la bocca all'asina, che disse a Balaam: 'Che t'ho io fatto che tu mi percuoti già per la terza volta?'
\par 29 E Balaam rispose all'asina: 'Perché ti sei fatta beffe di me. Ah se avessi una spada in mano! t'ammazzerei sull'attimo'.
\par 30 L'asina disse a Balaam: 'Non son io la tua asina che hai sempre cavalcata fino a quest'oggi? Sono io solita farti così?' Ed egli rispose: 'No'.
\par 31 Allora l'Eterno aprì gli occhi a Balaam, ed egli vide l'angelo dell'Eterno che stava sulla strada, con la sua spada sguainata. Balaam s'inchinò e si prostrò con la faccia in terra.
\par 32 L'angelo dell'Eterno gli disse: 'Perché hai percosso la tua asina già tre volte? Ecco, io sono uscito per farti ostacolo, perché la via che batti è contraria al voler mio;
\par 33 e l'asina m'ha visto ed è uscita di strada davanti a me queste tre volte; se non fosse uscita di strada davanti a me, certo io avrei già ucciso te e lasciato in vita lei'.
\par 34 Allora Balaam disse all'angelo dell'Eterno: 'Io ho peccato, perché non sapevo che tu ti fossi posto contro di me sulla strada; e ora, se questo ti dispiace, io me ne ritornerò'.
\par 35 E l'angelo dell'Eterno disse a Balaam: 'Va' pure con quegli uomini; ma dirai soltanto quello che io ti dirò'. E Balaam se ne andò coi principi di Balak.
\par 36 Quando Balak udì che Balaam arrivava, gli andò incontro a Jr-Moab che è sul confine segnato dall'Arnon, alla frontiera estrema.
\par 37 E Balak disse a Balaam: 'Non t'ho io mandato con insistenza a chiamare? perché non sei venuto da me? non son io proprio in grado di farti onore?'
\par 38 E Balaam rispose a Balak: 'Ecco, son venuto da te, ma posso io adesso dire qualsiasi cosa? la parola che Dio mi metterà in bocca, quella dirò'.
\par 39 Balaam andò con Balak, e giunsero a Kiriath-Hutsoth.
\par 40 E Balak sacrificò buoi e pecore e mandò parte delle carni a Balaam e ai principi ch'eran con lui.
\par 41 La mattina Balak prese Balaam e lo fece salire a Bamoth Baal, donde Balaam vide l'estremità del campo d'Israele.

\chapter{23}

\par 1 Balaam disse a Balak: 'Edificami qui sette altari e preparami qui sette giovenchi e sette montoni'.
\par 2 Balak fece come Balaam avea detto, e Balak e Balaam offrirono un giovenco e un montone su ciascun altare.
\par 3 E Balaam disse a Balak: 'Stattene presso al tuo olocausto, e io andrò: forse l'Eterno mi verrà incontro; e quel che mi avrà fatto vedere, te lo riferirò'. E se ne andò sopra una nuda altura.
\par 4 E Dio si fece incontro a Balaam, e Balaam gli disse: 'Io ho preparato i sette altari, ed ho offerto un giovenco e un montone su ciascun altare'.
\par 5 Allora l'Eterno mise delle parole in bocca a Balaam e gli disse: 'Torna da Balak, e parla così'.
\par 6 Balaam tornò da Balak, ed ecco che questi stava presso al suo olocausto: egli con tutti i principi di Moab.
\par 7 Allora Balaam pronunziò il suo oracolo e disse: "Balak m'ha fatto venire da Aram, il re di Moab, dalle montagne d'Oriente. - 'Vieni', disse, 'maledicimi Giacobbe! Vieni, esècra Israele!'
\par 8 Come farò a maledire? Iddio non l'ha maledetto. Come farò ad esecrare? L'Eterno non l'ha esecrato.
\par 9 Io lo guardo dal sommo delle rupi e lo contemplo dall'alto dei colli; ecco, è un popolo che dimora solo, e non è contato nel novero delle nazioni.
\par 10 Chi può contar la polvere di Giacobbe o calcolare il quarto d'Israele? Possa io morire della morte dei giusti, e possa la mia fine esser simile alla loro!"
\par 11 Allora Balak disse a Balaam: 'Che m'hai tu fatto? T'ho preso per maledire i miei nemici, ed ecco, non hai fatto che benedirli'.
\par 12 L'altro gli rispose e disse: 'Non debbo io stare attento a dire soltanto ciò che l'Eterno mi mette in bocca?'
\par 13 E Balak gli disse: 'Deh, vieni meco in un altro luogo, donde tu lo potrai vedere; tu, di qui, non ne puoi vedere che una estremità; non lo puoi vedere tutto quanto; e di là me lo maledirai'.
\par 14 E lo condusse al campo di Tsofim, sulla cima del Pisga; edificò sette altari, e offrì un giovenco e un montone su ciascun altare.
\par 15 E Balaam disse a Balak: 'Stattene qui presso al tuo olocausto, e io andrò a incontrare l'Eterno'.
\par 16 E l'Eterno si fece incontro a Balaam, gli mise delle parole in bocca e gli disse: 'Torna da Balak, e parla così'.
\par 17 Balaam tornò da Balak, ed ecco che questi stava presso al suo olocausto, coi principi di Moab. E Balak gli disse: 'Che ha detto l'Eterno?'
\par 18 Allora Balaam pronunziò il suo oracolo e disse: "Lèvati, Balak, e ascolta! Porgimi orecchio, figliuolo di Tsippor!
\par 19 Iddio non è un uomo, perch'ei mentisca, né un figliuol d'uomo, perch'ei si penta. Quand'ha detto una cosa non la farà? o quando ha parlato non manterrà la parola?
\par 20 Ecco, ho ricevuto l'ordine di benedire; egli ha benedetto; io non revocherò la benedizione.
\par 21 Egli non scorge iniquità in Giacobbe, non vede perversità in Israele. L'Eterno, il suo Dio, è con lui, e Israele lo acclama come suo re.
\par 22 Iddio lo ha tratto dall'Egitto, e gli dà il vigore del bufalo.
\par 23 In Giacobbe non v'è magia, in Israele, non v'è divinazione; a suo tempo vien detto a Giacobbe e ad Israele qual è l'opera che Iddio compie.
\par 24 Ecco un popolo che si leva su come una leonessa, e si rizza come un leone; ei non si sdraia prima d'aver divorato la preda e bevuto il sangue di quelli che ha ucciso".
\par 25 Allora Balak disse a Balaam: 'Non lo maledire, ma anche non lo benedire'.
\par 26 Ma Balaam rispose e disse a Balak: 'Non t'ho io detto espressamente: Io farò tutto quello che l'Eterno dirà?'
\par 27 E Balak disse a Balaam: 'Deh, vieni, io ti condurrò in un altro luogo; forse piacerà a Dio che tu me lo maledica di là'.
\par 28 Balak dunque condusse Balaam in cima al Peor che domina il deserto.
\par 29 E Balaam disse a Balak: 'Edificami qui sette altari, e preparami qui sette giovenchi e sette montoni'.
\par 30 Balak fece come Balaam avea detto, e offrì un giovenco e un montone su ciascun altare.

\chapter{24}

\par 1 E Balaam, vedendo che piaceva all'Eterno di benedire Israele, non ricorse come le altre volte alla magia, ma voltò la faccia verso il deserto.
\par 2 E, alzati gli occhi, Balaam vide Israele accampato tribù per tribù; e lo spirito di Dio fu sopra lui.
\par 3 E Balaam pronunziò il suo oracolo e disse: "Così dice Balaam, figliuolo di Beor, così dice l'uomo che ha l'occhio aperto,
\par 4 così dice colui che ode le parole di Dio, colui che contempla la visione dell'Onnipotente, colui che si prostra e a cui s'aprono gli occhi:
\par 5 Come son belle le tue tende, o Giacobbe, le tue dimore, o Israele!
\par 6 Esse si estendono come valli, come giardini in riva ad un fiume, come aloe piantati dall'Eterno, come cedri vicini alle acque.
\par 7 L'acqua trabocca dalle sue secchie, la sua semenza è bene adacquata, il suo re sarà più in alto di Agag, e il suo regno sarà esaltato.
\par 8 Iddio che l'ha tratto d'Egitto, gli dà il vigore del bufalo. Egli divorerà i popoli che gli sono avversari, frantumerà loro le ossa, li trafiggerà con le sue frecce.
\par 9 Egli si china, s'accovaccia come un leone, come una leonessa: chi lo farà rizzare? Benedetto chiunque ti benedice, maledetto chiunque ti maledice!"
\par 10 Allora l'ira di Balak s'accese contro Balaam; e Balak, battendo le mani, disse a Balaam: 'Io t'ho chiamato per maledire i miei nemici, ed ecco che li hai benedetti già per la terza volta.
\par 11 Or dunque fuggitene a casa tua! Io avevo detto che ti colmerai di onori; ma, ecco, l'Eterno ti rifiuta gli onori'.
\par 12 E Balaam rispose a Balak: 'E non dissi io, fin da principio, agli ambasciatori che mi mandasti:
\par 13 Quand'anche Balak mi desse la sua casa piena d'argento e d'oro, non potrei trasgredire l'ordine dell'Eterno per far di mia iniziativa alcun che di bene o di male; ciò che l'Eterno dirà, quello dirò?
\par 14 Ed ora, ecco, io me ne vado al mio popolo; vieni, io t'annunzierò ciò che questo popolo farà al popolo tuo nei giorni avvenire'.
\par 15 Allora Balaam pronunziò il suo oracolo e disse: "Così dice Balaam, figliuolo di Beor; così dice l'uomo che ha l'occhio aperto,
\par 16 così dice colui che ode le parole di Dio, che conosce la scienza dell'Altissimo, che contempla la visione dell'Onnipotente, colui che si prostra e a cui s'aprono gli occhi:
\par 17 Lo vedo, ma non ora; lo contemplo, ma non vicino: un astro sorge da Giacobbe, e uno scettro s'eleva da Israele, che colpirà Moab da un capo all'altro e abbatterà tutta quella razza turbolenta.
\par 18 S'impadronirà di Edom, s'impadronirà di Seir, suo nemico; Israele farà prodezze.
\par 19 Da Giacobbe verrà un dominatore che sterminerà i superstiti delle città".
\par 20 Balaam vide anche Amalek, e pronunziò il suo oracolo, dicendo: "Amalek è la prima delle nazioni ma il suo avvenire fa capo alla rovina".
\par 21 Vide anche i Kenei, e pronunziò il suo oracolo, dicendo: "La tua dimora è solida e il tuo nido è posto nella roccia;
\par 22 nondimeno, il Keneo dovrà essere devastato, finché l'Assiro ti meni in cattività".
\par 23 Poi pronunziò di nuovo il suo oracolo e disse: "Ahimè! Chi sussisterà quando Iddio avrà stabilito colui?
\par 24 Ma delle navi verranno dalle parti di Kittim e umilieranno Assur, umilieranno Eber, ed egli pure finirà per esser distrutto".
\par 25 Poi Balaam si levò, partì e se ne tornò a casa; e Balak pure se ne andò per la sua strada.

\chapter{25}

\par 1 Or Israele era stanziato a Sittim, e il popolo cominciò a darsi alla impurità con le figliuole di Moab.
\par 2 Esse invitarono il popolo ai sacrifizi offerti ai loro dèi, e il popolo mangiò e si prostrò dinanzi agli dèi di quelle.
\par 3 Israele si unì a Baal-Peor, e l'ira dell'Eterno si accese contro Israele.
\par 4 E l'Eterno disse a Mosè: 'Prendi tutti i capi del popolo e falli appiccare davanti all'Eterno, in faccia al sole, affinché l'ardente ira dell'Eterno sia rimossa da Israele'.
\par 5 E Mosè disse ai giudici d'Israele: 'Ciascuno di voi uccida quelli de' suoi uomini che si sono uniti a Baal-Peor'.
\par 6 Ed ecco che uno dei figliuoli d'Israele venne e condusse ai suoi fratelli una donna Madianita, sotto gli occhi di Mosè e di tutta la raunanza dei figliuoli d'Israele, mentr'essi stavano piangendo all'ingresso della tenda di convegno.
\par 7 La qual cosa avendo veduta Fineas, figliuolo di Eleazar, figliuolo del sacerdote Aaronne, si alzò di mezzo alla raunanza e die' di piglio ad una lancia;
\par 8 andò dietro a quell'uomo d'Israele nella sua tenda, e li trafisse ambedue, l'uomo d'Israele e la donna, nel basso ventre. E il flagello cessò tra i figliuoli d'Israele.
\par 9 Di quel flagello morirono ventiquattromila persone.
\par 10 L'Eterno parlò a Mosè, dicendo:
\par 11 'Fineas, figliuolo di Eleazar, figliuolo del sacerdote Aaronne, ha rimossa l'ira mia dai figliuoli d'Israele, perch'egli è stato animato del mio zelo in mezzo ad essi; ed io, nella mia indignazione, non ho sterminato i figliuoli d'Israele.
\par 12 Perciò digli ch'io fermo con lui un patto di pace,
\par 13 che sarà per lui e per la sua progenie dopo di lui l'alleanza d'un sacerdozio perpetuo, perch'egli ha avuto zelo per il suo Dio, e ha fatta l'espiazione per i figliuoli d'Israele'.
\par 14 Or l'uomo d'Israele che fu ucciso con la donna Madianita, si chiamava Zimri, figliuolo di Salu, capo di una casa patriarcale dei Simeoniti.
\par 15 E la donna che fu uccisa, la Madianita, si chiamava Cozbi, figliuola di Tsur, capo della gente di una casa patriarcale in Madian.
\par 16 Poi l'Eterno parlò a Mosè, dicendo:
\par 17 'Trattate i Madianiti come nemici e uccideteli,
\par 18 poiché essi vi hanno trattati da nemici con gl'inganni mediante i quali v'hanno sedotti nell'affare di Peor e nell'affare di Cozbi, figliuola d'un principe di Madian, loro sorella, che fu uccisa il giorno della piaga causata dall'affare di Peor'.

\chapter{26}

\par 1 Or avvenne che, dopo quella piaga, l'Eterno disse a Mosè e ad Eleazar, figliuolo del sacerdote Aaronne:
\par 2 'Fate il conto di tutta la raunanza de' figliuoli d'Israele, dall'età di vent'anni in su, secondo le case de' loro padri, di tutti quelli che in Israele possono andare alla guerra'.
\par 3 E Mosè e il sacerdote Eleazar parlarono loro nelle pianure di Moab presso al Giordano di faccia a Gerico, dicendo:
\par 4 'Si faccia il censimento dall'età di venti anni in su, come l'Eterno ha ordinato a Mosè e ai figliuoli d'Israele, quando furono usciti dal paese d'Egitto'.
\par 5 Ruben, primogenito d'Israele. Figliuoli di Ruben: Hanoch, da cui discende la famiglia degli Hanochiti; Pallu, da cui discende la famiglia de' Palluiti;
\par 6 Hetsron, da cui discende la famiglia degli Hetsroniti; Carmi da cui discende la famiglia de' Carmiti.
\par 7 Tali sono le famiglie dei Rubeniti: e quelli dei quali si fece il censimento furono quarantatremilasettecentotrenta.
\par 8 Figliuoli di Pallu: Eliab.
\par 9 Figliuoli di Eliab: Nemuel, Dathan ed Abiram. Questi sono quel Dathan e quell'Abiram, membri del consiglio, che si sollevarono contro Mosè e contro Aaronne con la gente di Kore, quando si sollevarono contro l'Eterno;
\par 10 e la terra aprì la sua bocca e li inghiottì assieme con Kore, allorché quella gente perì, e il fuoco divorò duecentocinquanta uomini, che servirono d'esempio.
\par 11 Ma i figliuoli di Kore non perirono.
\par 12 Figliuoli di Simeone secondo le loro famiglie. Da Nemuel discende la famiglia dei Nemueliti; da Jamin, la famiglia degli Jaminiti; da Jakin, la famiglia degli Jakiniti; da Zerach, la famiglia de' Zerachiti;
\par 13 da Saul, la famiglia dei Sauliti.
\par 14 Tali sono le famiglie dei Simeoniti: ventiduemiladuecento.
\par 15 Figliuoli di Gad secondo le loro famiglie. Da Tsefon discende la famiglia dei Tsefoniti; da Hagghi, la famiglia degli Hagghiti; da Shuni, la famiglia degli Shuniti;
\par 16 da Ozni, la famiglia degli Ozniti; da Eri, la famiglia degli Eriti;
\par 17 da Arod, la famiglia degli Aroditi; da Areli, la famiglia degli Areliti.
\par 18 Tali sono le famiglie dei figliuoli di Gad secondo il loro censimento: quarantamilacinquecento.
\par 19 Figliuoli di Giuda: Er e Onan; ma Er e Onan morirono nel paese di Canaan.
\par 20 Ecco i figliuoli di Giuda secondo le loro famiglie: da Scelah discende la famiglia degli Shelaniti; da Perets, la famiglia dei Peretsiti; da Zerach, la famiglia dei Zerachiti.
\par 21 I figliuoli di Perets furono: Hetsron da cui discende la famiglia degli Hetsroniti; Hamul da cui discende la famiglia degli Hamuliti.
\par 22 Tali sono le famiglie di Giuda secondo il loro censimento: settantaseimilacinquecento.
\par 23 Figliuoli d'Issacar secondo le loro famiglie: da Thola discende la famiglia dei Tholaiti: da Puva, la famiglia dei Puviti;
\par 24 da Jashub, la famiglia degli Jashubiti; da Scimron, la famiglia dei Scimroniti.
\par 25 Tali sono le famiglie d'Issacar secondo il loro censimento: sessantaquattromilatrecento.
\par 26 Figliuoli di Zabulon secondo le loro famiglie: da Sered discende la famiglia dei Sarditi; da Elon, la famiglia degli Eloniti; da Jahleel, la famiglia degli Jahleeliti.
\par 27 Tali sono le famiglie degli Zabuloniti secondo il loro censimento: sessantamilacinquecento.
\par 28 Figliuoli di Giuseppe secondo le loro famiglie: Manasse ed Efraim.
\par 29 Figliuoli di Manasse: da Makir discende la famiglia dei Makiriti. Makir generò Galaad. Da Galaad discende la famiglia dei Galaaditi.
\par 30 Questi sono i figliuoli di Galaad: Jezer, da cui discende la famiglia degli Jezeriti; Helek, da cui discende la famiglia degli Helekiti;
\par 31 Asriel, da cui discende la famiglia degli Asrieliti; Sichem, da cui discende la famiglia dei Sichemiti;
\par 32 Scemida, da cui discende la famiglia dei Scemidaiti; Hefer, da cui discende la famiglia degli Heferiti.
\par 33 Or Tselofehad, figliuolo di Hefer, non ebbe maschi ma soltanto delle figliuole; e i nomi delle figliuole di Tselofehad furono: Mahlah, Noah, Hoglah, Milcah e Thirtsah.
\par 34 Tali sono le famiglie di Manasse; le persone censite furono cinquantaduemilasettecento.
\par 35 Ecco i figliuoli di Efraim secondo le loro famiglie: da Shuthelah discende la famiglia dei Shuthelahiti; da Beker, la famiglia dei Bakriti; da Tahan, la famiglia dei Tahaniti.
\par 36 Ed ecco i figliuoli di Shuthelah: da Eran è discesa la famiglia degli Eraniti.
\par 37 Tali sono le famiglie de' figliuoli d'Efraim secondo il loro censimento: trentaduemilacinquecento. Questi sono i figliuoli di Giuseppe secondo le loro famiglie.
\par 38 Figliuoli di Beniamino secondo le loro famiglie: da Bela discende la famiglia dei Belaiti; da Ashbel, la famiglia degli Ashbeliti; da Ahiram, la famiglia degli Ahiramiti;
\par 39 da Scefulam, la famiglia degli Shufamiti;
\par 40 da Hufam, la famiglia degli Hufamiti. I figliuoli di Bela furono: Ard e Naaman; da Ard discende la famiglia degli Arditi; da Naaman, la famiglia dei Naamiti.
\par 41 Tali sono i figliuoli di Beniamino secondo le loro famiglie. Le persone censite furono quarantacinquemilaseicento.
\par 42 Ecco i figliuoli di Dan secondo le loro famiglie: da Shuham discende la famiglia degli Shuhamiti. Sono queste le famiglie di Dan secondo le loro famiglie.
\par 43 Totale per le famiglie degli Shuhamiti secondo il loro censimento: sessantaquattromilaquattrocento.
\par 44 Figliuoli di Ascer secondo le loro famiglie: da Imna discende la famiglia degli Imniti; da Ishvi, la famiglia degli Ishviti; da Beriah, la famiglia de' Beriiti.
\par 45 Dai figliuoli di Beriah discendono: da Heber, la famiglia degli Hebriti; da Malkiel, la famiglia de' Malkieliti.
\par 46 Il nome della figliuola di Ascer era Serah.
\par 47 Tali sono le famiglie dei figliuoli di Ascer secondo il loro censimento: cinquantatremilaquattrocento.
\par 48 Figliuoli di Neftali secondo le loro famiglie: da Jahtseel discende la famiglia degli Jahtseeliti; da Guni, la famiglia dei Guniti;
\par 49 da Jetser, la famiglia degli Jetseriti; da Scillem la famiglia degli Scillemiti.
\par 50 Tali sono le famiglie di Neftali secondo le loro famiglie. Le persone censite furono quarantacinquemilaquattrocento.
\par 51 Tali sono i figliuoli d'Israele di cui si fece il censimento: seicentunmilasettecentotrenta.
\par 52 L'Eterno parlò a Mosè dicendo:
\par 53 'Il paese sarà diviso tra essi, per esser loro proprietà secondo il numero de' nomi.
\par 54 A quelli che sono in maggior numero darai in possesso una porzione maggiore; a quelli che sono in minor numero darai una porzione minore; si darà a ciascuno la sua porzione secondo il censimento.
\par 55 Ma la spartizione del paese sarà fatta a sorte; essi riceveranno la rispettiva proprietà secondo i nomi delle loro tribù paterne.
\par 56 La spartizione delle proprietà sarà fatta a sorte fra quelli che sono in maggior numero e quelli che sono in numero minore'.
\par 57 Ecco i Leviti dei quali si fece il censimento secondo le loro famiglie; da Gherson discende la famiglia dei Ghersoniti; da Kehath, la famiglia de' Kehathiti; da Merari, la famiglia de' Merariti.
\par 58 Ecco le famiglie di Levi: la famiglia de' Libniti, la famiglia degli Hebroniti, la famiglia dei Mahliti, la famiglia de' Mushiti, la famiglia de' Korahiti. E Kehath generò Amram.
\par 59 Il nome della moglie di Amram era Jokebed, figliuola di Levi che nacque a Levi in Egitto; ed essa partorì ad Amram Aaronne, Mosè e Maria loro sorella.
\par 60 E ad Aaronne nacquero Nadab e Abihu, Eleazar e Ithamar.
\par 61 Or Nadab e Abihu morirono quando presentarono all'Eterno fuoco estraneo.
\par 62 Quelli de' quali si fece il censimento furono ventitremila: tutti maschi, dell'età da un mese in su. Non furon compresi nel censimento dei figliuoli d'Israele perché non fu loro data alcuna proprietà tra i figliuoli d'Israele.
\par 63 Tali son quelli de' figliuoli d'Israele dei quali Mosè e il sacerdote Eleazar fecero il censimento nelle pianure di Moab presso al Giordano di Gerico.
\par 64 Fra questi non v'era alcuno di quei figliuoli d'Israele de' quali Mosè e il sacerdote Aaronne aveano fatto il censimento nel deserto di Sinai.
\par 65 Poiché l'Eterno avea detto di loro: 'Certo, morranno nel deserto!' E non ne rimase neppur uno, salvo Caleb, figliuolo di Gefunne, e Giosuè, figliuolo di Nun.

\chapter{27}

\par 1 Or le figliuole di Tselofehad figliuolo di Hefer, figliuolo di Galaad, figliuolo di Makir, figliuolo di Manasse, delle famiglie di Manasse, figliuolo di Giuseppe, che si chiamavano Mahlah, Noah, Hoglah, Milcah e Thirtsah,
\par 2 si accostarono e si presentarono davanti a Mosè, davanti al sacerdote Eleazar, davanti ai principi e a tutta la raunanza all'ingresso della tenda di convegno, e dissero:
\par 3 'Il padre nostro morì nel deserto, e non fu nella compagnia di quelli che si adunarono contro l'Eterno, non fu della gente di Kore, ma morì a motivo del suo peccato, e non ebbe figliuoli.
\par 4 Perché dovrebbe il nome del padre nostro scomparire di mezzo dalla sua famiglia s'egli non ebbe figliuoli? Dacci un possesso in mezzo ai fratelli di nostro padre'.
\par 5 E Mosè portò la loro causa davanti all'Eterno.
\par 6 E l'Eterno disse a Mosè:
\par 7 'Le figliuole di Tselofehad dicono bene. Sì, tu darai loro in eredità un possesso tra i fratelli del padre loro, e farai passare ad esse l'eredità del padre loro.
\par 8 Parlerai pure ai figliuoli d'Israele, e dirai: Quand'uno sarà morto senza lasciar figliuolo maschio, farete passare l'eredità sua alla sua figliuola.
\par 9 E, se non ha figliuola, darete la sua eredità ai suoi fratelli.
\par 10 E, se non ha fratelli, darete la sua eredità ai fratelli di suo padre.
\par 11 E, se non ci sono fratelli del padre, darete la sua eredità al parente più stretto nella sua famiglia; e quello la possederà. Questo sarà per i figliuoli d'Israele una norma di diritto, come l'Eterno ha ordinato a Mosè'.
\par 12 Poi l'Eterno disse a Mosè: 'Sali su questo monte di Abarim e contempla il paese che io do ai figliuoli d'Israele.
\par 13 E quando l'avrai veduto, anche tu sarai raccolto presso il tuo popolo, come fu raccolto Aaronne tuo fratello,
\par 14 perché vi ribellaste all'ordine che vi detti nel deserto di Tsin quando la raunanza si mise a contendere, e voi non mi santificaste agli occhi loro, a proposito di quelle acque'. - Sono le acque della contesa di Kades, nel deserto di Tsin. -
\par 15 E Mosè parlò all'Eterno, dicendo:
\par 16 'L'Eterno, l'Iddio degli spiriti d'ogni carne, costituisca su questa raunanza un uomo
\par 17 che esca davanti a loro ed entri davanti a loro, e li faccia uscire e li faccia entrare, affinché la raunanza dell'Eterno non sia come un gregge senza pastore'.
\par 18 E l'Eterno disse a Mosè: 'Prenditi Giosuè, figliuolo di Nun, uomo in cui è lo spirito; poserai la tua mano su lui,
\par 19 lo farai comparire davanti al sacerdote Eleazar e davanti a tutta la raunanza, gli darai i tuoi ordini in loro presenza,
\par 20 e lo farai partecipe della tua autorità, affinché tutta la raunanza de' figliuoli d'Israele gli obbedisca.
\par 21 Egli si presenterà davanti al sacerdote Eleazar, che consulterà per lui il giudizio dell'Urim davanti all'Eterno; egli e tutti i figliuoli d'Israele con lui e tutta la raunanza usciranno all'ordine di Eleazar ed entreranno all'ordine suo'.
\par 22 E Mosè fece come l'Eterno gli aveva ordinato; prese Giosuè e lo fece comparire davanti al sacerdote Eleazar e davanti a tutta la raunanza;
\par 23 posò su lui le sue mani e gli diede i suoi ordini, come l'Eterno aveva comandato per mezzo di Mosè.

\chapter{28}

\par 1 E l'Eterno parlò a Mosè, dicendo:
\par 2 'Da' quest'ordine ai figliuoli d'Israele, e di' loro: Avrete cura d'offrirmi al tempo stabilito la mia offerta, il cibo de' miei sacrifizi fatti mediante il fuoco, e che mi sono di soave odore.
\par 3 E dirai loro: Questo è il sacrifizio mediante il fuoco, che offrirete all'Eterno: degli agnelli dell'anno, senza difetti, due al giorno, come olocausto perpetuo.
\par 4 Uno degli agnelli offrirai la mattina, e l'altro agnello offrirai sull'imbrunire:
\par 5 e, come oblazione, un decimo d'efa di fior di farina, intrisa con un quarto di hin d'olio vergine.
\par 6 Tale è l'olocausto perpetuo, offerto sul monte Sinai: sacrifizio fatto mediante il fuoco, di soave odore all'Eterno.
\par 7 La libazione sarà di un quarto di hin per ciascun agnello; la libazione di vino puro all'Eterno la farai nel luogo santo.
\par 8 E l'altro agnello l'offrirai sull'imbrunire, con un'oblazione e una libazione simili a quelle della mattina: è un sacrifizio fatto mediante il fuoco, di soave odore all'Eterno.
\par 9 Nel giorno di sabato offrirete due agnelli dell'anno, senza difetti; e, come oblazione, due decimi di fior di farina intrisa con olio, con la sua libazione.
\par 10 È l'olocausto del sabato, per ogni sabato, oltre l'olocausto perpetuo e la sua libazione.
\par 11 Al principio de' vostri mesi offrirete come olocausto all'Eterno due giovenchi, un montone, sette agnelli dell'anno, senza difetti,
\par 12 e tre decimi di fior di farina intrisa con olio, come oblazione per ciascun giovenco; due decimi di fior di farina intrisa con olio, come oblazione per il montone,
\par 13 e un decimo di fior di farina intrisa con olio, come oblazione per ogni agnello. È un olocausto di soave odore, un sacrifizio fatto mediante il fuoco all'Eterno.
\par 14 Le libazioni saranno di un mezzo hin di vino per giovenco, d'un terzo di hin per il montone e di un quarto di hin per agnello. Tale è l'olocausto del mese, per tutti i mesi dell'anno.
\par 15 E s'offrirà all'Eterno un capro come sacrifizio per il peccato, oltre l'olocausto perpetuo e la sua libazione.
\par 16 Il primo mese, il quattordicesimo giorno del mese sarà la Pasqua in onore dell'Eterno.
\par 17 E il quindicesimo giorno di quel mese sarà giorno di festa. Per sette giorni si mangerà pane senza lievito.
\par 18 Il primo giorno vi sarà una santa convocazione; non farete alcuna opera servile,
\par 19 ma offrirete, come sacrifizio mediante il fuoco, un olocausto all'Eterno: due giovenchi, un montone e sette agnelli dell'anno che siano senza difetti;
\par 20 e, come oblazione, del fior di farina intrisa con olio; e ne offrirete tre decimi per giovenco e due per il montone;
\par 21 ne offrirai un decimo per ciascuno de' sette agnelli,
\par 22 e offrirai un capro come sacrifizio per il peccato, per fare l'espiazione per voi.
\par 23 Offrirete questi sacrifizi oltre l'olocausto della mattina, che è un olocausto perpetuo.
\par 24 L'offrirete ogni giorno, per sette giorni; è un cibo di sacrifizio fatto mediante il fuoco, di soave odore all'Eterno. Lo si offrirà oltre l'olocausto perpetuo con la sua libazione.
\par 25 E il settimo giorno avrete una santa convocazione, non farete alcuna opera servile.
\par 26 Il giorno delle primizie, quando presenterete all'Eterno una oblazione nuova, alla vostra festa delle settimane, avrete una santa convocazione; non farete alcuna opera servile.
\par 27 E offrirete, come olocausto di soave odore all'Eterno, due giovenchi, un montone e sette agnelli dell'anno;
\par 28 e, come oblazione, del fior di farina intrisa con olio: tre decimi per ciascun giovenco, due decimi per il montone,
\par 29 e un decimo per ciascuno dei sette agnelli;
\par 30 e offrirete un capro per fare l'espiazione per voi.
\par 31 Offrirete questi sacrifizi, oltre l'olocausto perpetuo e la sua oblazione. Sceglierete degli animali senza difetti e v'aggiungerete le relative libazioni.

\chapter{29}

\par 1 Il settimo mese, il primo giorno del mese avrete una santa convocazione; non farete alcuna opera servile; sarà per voi il giorno del suon delle trombe.
\par 2 Offrirete come olocausto di soave odore all'Eterno un giovenco, un montone, sette agnelli dell'anno senza difetti,
\par 3 e, come oblazione, del fior di farina intrisa con olio: tre decimi per il giovenco, due decimi per il montone,
\par 4 un decimo per ciascuno dei sette agnelli;
\par 5 e un capro, come sacrifizio per il peccato, per fare l'espiazione per voi,
\par 6 oltre l'olocausto del mese con la sua oblazione, e l'olocausto perpetuo con la sua oblazione, e le loro libazioni, secondo le regole stabilite. Sarà un sacrifizio, fatto mediante il fuoco, di soave odore all'Eterno.
\par 7 Il decimo giorno di questo settimo mese avrete una santa convocazione e umilierete le anime vostre; non farete lavoro di sorta,
\par 8 e offrirete, come olocausto di soave odore all'Eterno, un giovenco, un montone, sette agnelli dell'anno che siano senza difetti,
\par 9 e, come oblazione, del fior di farina intrisa con olio: tre decimi per il giovenco, due decimi per il montone,
\par 10 un decimo per ciascuno dei sette agnelli;
\par 11 e un capro come sacrifizio per il peccato, oltre il sacrifizio d'espiazione, l'olocausto perpetuo con la sua oblazione e le loro libazioni.
\par 12 Il quindicesimo giorno del settimo mese avrete una santa convocazione; non farete alcuna opera servile, e celebrerete una festa in onor dell'Eterno per sette giorni.
\par 13 E offrirete come olocausto, come sacrifizio fatto mediante il fuoco, di soave odore all'Eterno, tredici giovenchi, due montoni, quattordici agnelli dell'anno, che siano senza difetti,
\par 14 e, come oblazione, del fior di farina intrisa con olio: tre decimi per ciascuno dei tredici giovenchi, due decimi per ciascuno dei due montoni,
\par 15 un decimo per ciascuno dei quattordici agnelli,
\par 16 e un capro come sacrifizio per il peccato, oltre l'olocausto perpetuo, con la sua oblazione e la sua libazione.
\par 17 Il secondo giorno offrirete dodici giovenchi, due montoni, quattordici agnelli dell'anno, senza difetti,
\par 18 con le loro oblazioni e le loro libazioni per i giovenchi, i montoni e gli agnelli secondo il loro numero, seguendo le regole stabilite;
\par 19 e un capro come sacrifizio per il peccato, oltre l'olocausto perpetuo, la sua oblazione e la sua libazione.
\par 20 Il terzo giorno offrirete undici giovenchi, due montoni, quattordici agnelli dell'anno, senza difetti,
\par 21 con le loro oblazioni e le loro libazioni per i giovenchi, i montoni e gli agnelli, secondo il loro numero, seguendo le regole stabilite;
\par 22 e un capro come sacrifizio per il peccato, oltre l'olocausto perpetuo, la sua oblazione e la sua libazione.
\par 23 Il quarto giorno offrirete dieci giovenchi, due montoni e quattordici agnelli dell'anno, senza difetti,
\par 24 con le loro offerte e le loro libazioni per i giovenchi, i montoni e gli agnelli, secondo il loro numero e seguendo le regole stabilite;
\par 25 e un capro, come sacrifizio per il peccato, oltre l'olocausto perpetuo, la sua oblazione e la sua libazione.
\par 26 Il quinto giorno offrirete nove giovenchi, due montoni, quattordici agnelli dell'anno, senza difetti,
\par 27 con le loro oblazioni e le loro libazioni per i giovenchi, i montoni e gli agnelli, secondo il loro numero e seguendo le regole stabilite;
\par 28 e un capro, come sacrifizio per il peccato, oltre l'olocausto perpetuo, la sua oblazione e la sua libazione.
\par 29 Il sesto giorno offrirete otto giovenchi, due montoni, quattordici agnelli dell'anno, senza difetti,
\par 30 con le loro oblazioni e le loro libazioni per i giovenchi, i montoni e gli agnelli, secondo il loro numero e seguendo le regole stabilite;
\par 31 e un capro, come sacrifizio per il peccato, oltre l'olocausto perpetuo, la sua oblazione e la sua libazione.
\par 32 Il settimo giorno offrirete sette giovenchi, due montoni, quattordici agnelli dell'anno, senza difetti,
\par 33 con le loro oblazioni e le loro libazioni per i giovenchi, i montoni e gli agnelli, secondo il loro numero e seguendo le regole stabilite;
\par 34 e un capro, come sacrifizio per il peccato, oltre l'olocausto perpetuo, la sua oblazione e la sua libazione.
\par 35 L'ottavo giorno avrete una solenne raunanza; non farete alcuna opera servile,
\par 36 e offrirete come olocausto, come sacrifizio fatto mediante il fuoco, di soave odore all'Eterno, un giovenco, un montone, sette agnelli dell'anno, senza difetti,
\par 37 con le loro oblazioni e le loro libazioni per il giovenco, il montone e gli agnelli, secondo il loro numero, seguendo le regole stabilite;
\par 38 e un capro, come sacrifizio per il peccato, oltre l'olocausto perpetuo, la sua oblazione e la sua libazione.
\par 39 Tali sono i sacrifizi che offrirete all'Eterno nelle vostre solennità, oltre i vostri voti e le vostre offerte volontarie, sia che si tratti de' vostri olocausti o delle vostre oblazioni o delle vostre libazioni o de' vostri sacrifizi di azioni di grazie'.
\par 40 E Mosè riferì ai figliuoli d'Israele tutto quello che l'Eterno gli aveva ordinato.

\chapter{30}

\par 1 Mosè parlò ai capi delle tribù de' figliuoli d'Israele, dicendo: 'Questo è quel che l'Eterno ha ordinato:
\par 2 Quand'uno avrà fatto un voto all'Eterno od avrà con giuramento contratta una solenne obbligazione, non violerà la sua parola, ma metterà in esecuzione tutto quello che gli è uscito di bocca.
\par 3 Così pure quando una donna avrà fatto un voto all'Eterno e si sarà legata con un impegno essendo in casa del padre, durante la sua giovinezza,
\par 4 se il padre, avendo conoscenza del voto di lei e dell'impegno per il quale ella si è legata, non dice nulla a questo proposito, tutti i voti di lei saranno validi, e saranno validi tutti gli impegni per i quali ella si sarà legata.
\par 5 Ma se il padre, il giorno che ne viene a conoscenza, le fa opposizione, tutti i voti di lei e tutti gl'impegni per i quali si sarà legata, non saranno validi; e l'Eterno le perdonerà, perché il padre le ha fatto opposizione.
\par 6 E se viene a maritarsi essendo legata da voti o da una promessa fatta alla leggera con le labbra, per la quale si sia impegnata,
\par 7 se il marito ne ha conoscenza e il giorno che ne viene a conoscenza non dice nulla a questo proposito, i voti di lei saranno validi, e saranno validi gl'impegni per i quali ella si è legata.
\par 8 Ma se il marito, il giorno che ne viene a conoscenza, le fa opposizione, egli annullerà il voto ch'ella ha fatto e la promessa che ha proferito alla leggera per la quale s'è impegnata; e l'Eterno le perdonerà.
\par 9 Ma il voto di una vedova o di una donna ripudiata, qualunque sia l'impegno per il quale si sarà legata, rimarrà valido.
\par 10 Quando una donna, nella casa di suo marito, farà dei voti o si legherà con un giuramento,
\par 11 e il marito ne avrà conoscenza, se il marito non dice nulla a questo proposito e non le fa opposizione, tutti i voti di lei saranno validi, e saran validi tutti gl'impegni per i quali ella si sarà legata.
\par 12 Ma se il marito, il giorno che ne viene a conoscenza li annulla, tutto ciò che le sarà uscito dalle labbra, siano voti o impegni per cui s'è legata, non sarà valido; il marito lo ha annullato; e l'Eterno le perdonerà.
\par 13 Il marito può ratificare e il marito può annullare qualunque voto e qualunque giuramento, per il quale ella si sia impegnata a mortificare la sua persona.
\par 14 Ma se il marito, giorno dopo giorno, non dice nulla in proposito, egli ratifica così tutti i voti di lei e tutti gl'impegni per i quali ella si è legata; li ratifica, perché non ha detto nulla a questo proposito il giorno che ne ha avuto conoscenza.
\par 15 Ma se li annulla qualche tempo dopo averne avuto conoscenza, sarà responsabile del peccato della moglie'.
\par 16 Tali sono le leggi che l'Eterno prescrisse a Mosè, riguardo al marito e alla moglie, al padre e alla figliuola, quando questa è ancora fanciulla, in casa di suo padre.

\chapter{31}

\par 1 Poi l'Eterno parlò a Mosè, dicendo:
\par 2 'Vendica i figliuoli d'Israele dai Madianiti; poi sarai raccolto col tuo popolo'.
\par 3 E Mosè parlò al popolo, dicendo: 'Mobilitate fra voi uomini per la guerra, e marcino contro Madian per eseguire la vendetta dell'Eterno su Madian.
\par 4 Manderete alla guerra mille uomini per tribù, di tutte le tribù d'Israele'.
\par 5 Così furon forniti, fra le migliaia d'Israele, mille uomini per tribù: cioè dodicimila uomini, armati per la guerra.
\par 6 E Mosè mandò alla guerra que' mille uomini per tribù, e con loro Fineàs figliuolo del sacerdote Eleazar, il quale portava gli strumenti sacri ed aveva in mano le trombe d'allarme.
\par 7 Essi marciarono dunque contro Madian, come l'Eterno aveva ordinato a Mosè, e uccisero tutti i maschi.
\par 8 Uccisero pure, con tutti gli altri, i re di Madian Evi, Rekem, Tsur, Hur e Reba: cinque re di Madian; uccisero pure con la spada Balaam, figliuolo di Beor.
\par 9 E i figliuoli d'Israele presero prigioniere le donne di Madian e i loro fanciulli, e predarono tutto il loro bestiame, tutti i loro greggi e ogni loro bene;
\par 10 e appiccarono il fuoco a tutte le città che quelli abitavano, e a tutti i loro accampamenti,
\par 11 e presero tutte le spoglie e tutta la preda: gente e bestiame;
\par 12 e menarono i prigionieri, la preda e le spoglie a Mosè, al sacerdote Eleazar e alla raunanza dei figliuoli d'Israele, accampati nelle pianure di Moab, presso il Giordano, difaccia a Gerico.
\par 13 Mosè, il sacerdote Eleazar e tutti i principi della raunanza uscirono loro incontro fuori del campo.
\par 14 E Mosè si adirò contro i comandanti dell'esercito, capi di migliaia e capi di centinaia, che tornavano da quella spedizione di guerra.
\par 15 Mosè disse loro: 'Avete lasciato la vita a tutte le donne?
\par 16 Ecco, sono esse che, a suggestione di Balaam, trascinarono i figliuoli d'Israele alla infedeltà verso l'Eterno, nel fatto di Peor, onde la piaga scoppiò nella raunanza dell'Eterno.
\par 17 Or dunque uccidete ogni maschio tra i fanciulli, e uccidete ogni donna che ha avuto relazioni carnali con un uomo;
\par 18 ma tutte le fanciulle che non hanno avuto relazioni carnali con uomini, serbatele in vita per voi.
\par 19 E voi accampatevi per sette giorni fuori del campo; chiunque ha ucciso qualcuno e chiunque ha toccato una persona uccisa, si purifichi il terzo e il settimo giorno: e questo, tanto per voi quanto per i vostri prigionieri.
\par 20 Purificherete anche ogni veste, ogni oggetto di pelle, ogni lavoro di pel di capra e ogni utensile di legno'.
\par 21 E il sacerdote Eleazar disse ai soldati ch'erano andati alla guerra: 'Questo è l'ordine della legge che l'Eterno ha prescritta a Mosè:
\par 22 L'oro, l'argento, il rame, il ferro, lo stagno e il piombo,
\par 23 tutto ciò, insomma, che può reggere al fuoco, lo farete passare per il fuoco e sarà reso puro; nondimeno, sarà purificato anche con l'acqua di purificazione; e tutto ciò che non può reggere al fuoco lo farete passare per l'acqua.
\par 24 E vi laverete le vesti il settimo giorno, e sarete puri; poi potrete entrare nel campo'.
\par 25 L'Eterno parlò ancora a Mosè, dicendo:
\par 26 'Tu, col sacerdote Eleazar e con i capi delle case della raunanza, fa' il conto di tutta la preda ch'è stata fatta: della gente e del bestiame;
\par 27 e dividi la preda fra i combattenti che sono andati alla guerra e tutta la raunanza.
\par 28 Dalla parte spettante ai soldati che sono andati alla guerra preleverai un tributo per l'Eterno: cioè uno su cinquecento, tanto delle persone quanto de' buoi, degli asini e delle pecore.
\par 29 Lo prenderete sulla loro metà e lo darai al sacerdote Eleazar come un'offerta all'Eterno.
\par 30 E dalla metà che spetta ai figliuoli d'Israele prenderai uno su cinquanta, tanto delle persone quanto dei buoi, degli asini, delle pecore, di tutto il bestiame; e lo darai ai Leviti, che hanno l'incarico del tabernacolo dell'Eterno'.
\par 31 E Mosè e il sacerdote Eleazar fecero come l'Eterno aveva ordinato a Mosè.
\par 32 Or la preda, cioè quel che rimaneva del bottino fatto da quelli ch'erano stati alla guerra, consisteva in seicentosettantacinquemila pecore,
\par 33 settantaduemila buoi,
\par 34 sessantamila asini, e trentaduemila persone, ossia donne,
\par 35 che non avevano avuto relazioni carnali con uomini.
\par 36 La metà, cioè la parte di quelli ch'erano andati alla guerra, fu di trecentotrentasettemilacinquecento pecore,
\par 37 delle quali seicentosettantacinque per il tributo all'Eterno;
\par 38 trentaseimila bovi, dei quali settantadue per il tributo all'Eterno;
\par 39 trentamilacinquecento asini, dei quali sessantuno per il tributo all'Eterno;
\par 40 e sedicimila persone, delle quali trentadue per il tributo all'Eterno.
\par 41 E Mosè dette al sacerdote Eleazar il tributo prelevato per l'offerta all'Eterno, come l'Eterno gli aveva ordinato.
\par 42 La metà che spettava ai figliuoli d'Israele, dopo che Mosè ebbe fatta la spartizione con gli uomini andati alla guerra, la metà spettante alla raunanza,
\par 43 fu di trecentotrentasettemilacinquecento pecore,
\par 44 trentaseimila buoi,
\par 45 trentamilacinquecento asini e sedicimila persone.
\par 46 Da questa metà,
\par 47 che spettava ai figliuoli d'Israele, Mosè prese uno su cinquanta, tanto degli uomini quanto degli animali, e li dette ai Leviti che hanno l'incarico del tabernacolo dell'Eterno, come l'Eterno aveva ordinato a Mosè.
\par 48 I comandanti delle migliaia dell'esercito, capi di migliaia e capi di centinaia, s'avvicinarono a Mosè e gli dissero:
\par 49 'I tuoi servi hanno fatto il conto dei soldati che erano sotto i nostri ordini, e non ne manca neppur uno.
\par 50 E noi portiamo, come offerta all'Eterno, ciascuno quel che ha trovato di oggetti d'oro: catenelle, braccialetti, anelli, pendenti, collane, per fare l'espiazione per le nostre persone davanti all'Eterno'.
\par 51 E Mosè e il sacerdote Eleazar presero dalle loro mani tutto quell'oro in gioielli lavorati.
\par 52 Tutto l'oro dell'offerta ch'essi presentarono all'Eterno da parte de' capi di migliaia e de' capi di centinaia, pesava sedicimilasettecentocinquanta sicli.
\par 53 Or gli uomini dell'esercito si tennero il bottino che ognuno avea fatto per conto suo.
\par 54 E Mosè e il sacerdote Eleazar presero l'oro dei capi di migliaia e di centinaia e lo portarono nella tenda di convegno come ricordanza per i figliuoli d'Israele davanti all'Eterno.

\chapter{32}

\par 1 Or i figliuoli di Ruben e i figliuoli di Gad aveano del bestiame in grandissimo numero; e quando videro che il paese di Iazer e il paese di Galaad erano luoghi da bestiame,
\par 2 i figliuoli di Gad e i figliuoli di Ruben vennero a parlare a Mosè, al sacerdote Eleazar e ai principi della raunanza, e dissero:
\par 3 'Ataroth, Dibon, Iazer, Nimrah, Heshbon, Elealeh, Sebam, Nebo e Beon,
\par 4 terre che l'Eterno ha colpite dinanzi alla raunanza d'Israele, sono terre da bestiame, e i tuoi servi hanno del bestiame'.
\par 5 E dissero ancora: 'Se abbiam trovato grazia agli occhi tuoi, sia concesso ai tuoi servi il possesso di questo paese e non ci far passare il Giordano'.
\par 6 Ma Mosè rispose ai figliuoli di Gad e ai figliuoli di Ruben: 'Andrebbero eglino i vostri fratelli alla guerra e voi ve ne stareste qui?
\par 7 E perché volete scoraggiare i figliuoli d'Israele dal passare nel paese che l'Eterno ha loro dato?
\par 8 Così fecero i vostri padri, quando li mandai da Kades-Barnea per esplorare il paese.
\par 9 Salirono fino alla valle d'Eshcol; e dopo aver esplorato il paese, scoraggiarono i figliuoli d'Israele dall'entrare nel paese che l'Eterno avea loro dato.
\par 10 E l'ira dell'Eterno s'accese in quel giorno, ed egli giurò dicendo:
\par 11 Gli uomini che sono saliti dall'Egitto, dall'età di vent'anni in su non vedranno mai il paese che promisi con giuramento ad Abrahamo, a Isacco ed a Giacobbe, perché non m'hanno seguitato fedelmente,
\par 12 salvo Caleb, figliuolo di Gefunne, il Kenizeo, e Giosuè, figliuolo di Nun, che hanno seguitato l'Eterno fedelmente.
\par 13 E l'ira dell'Eterno s'accese contro Israele; ed ei lo fece andar vagando per il deserto durante quarant'anni, finché tutta la generazione che avea fatto quel male agli occhi dell'Eterno, fosse consumata.
\par 14 Ed ecco che voi sorgete al posto de' vostri padri, razza d'uomini peccatori, per rendere l'ira dell'Eterno anche più ardente contro Israele.
\par 15 Perché, se voi vi sviate da lui, egli continuerà a lasciare Israele nel deserto, e voi farete perire tutto questo popolo'.
\par 16 Ma quelli s'accostarono a Mosè e gli dissero: 'Noi edificheremo qui dei recinti per il nostro bestiame, e delle città per i nostri figliuoli;
\par 17 ma, quanto a noi, ci terremo pronti in armi, per marciare alla testa de' figliuoli d'Israele, finché li abbiam condotti al luogo destinato loro; intanto, i nostri figliuoli dimoreranno nelle città forti a cagione degli abitanti del paese.
\par 18 Non torneremo alle nostre case finché ciascuno dei figliuoli d'Israele non abbia preso possesso della sua eredità;
\par 19 e non possederemo nulla con loro al di là del Giordano e più oltre, giacché la nostra eredità ci è toccata da questa parte del Giordano, a oriente'.
\par 20 E Mosè disse loro: 'Se fate questo, se vi armate per andare a combattere davanti all'Eterno,
\par 21 se tutti quelli di voi che s'armeranno passeranno il Giordano davanti all'Eterno finch'egli abbia cacciato i suoi nemici dal suo cospetto,
\par 22 e se non tornate che quando il paese vi sarà sottomesso davanti all'Eterno, voi non sarete colpevoli di fronte all'Eterno e di fronte a Israele, e questo paese sarà vostra proprietà davanti all'Eterno.
\par 23 Ma, se non fate così, voi avrete peccato contro l'Eterno; e sappiate che il vostro peccato vi ritroverà.
\par 24 Edificatevi delle città per i vostri figliuoli e dei recinti per i vostri greggi, e fate quello che la vostra bocca ha proferito'.
\par 25 E i figliuoli di Gad e i figliuoli di Ruben parlarono a Mosè, dicendo: 'I tuoi servi faranno quello che il mio signore comanda.
\par 26 I nostri fanciulli, le nostre mogli, i nostri greggi e tutto il nostro bestiame rimarranno qui nelle città di Galaad;
\par 27 ma i tuoi servi, tutti quanti armati per la guerra, andranno a combattere davanti all'Eterno, come dice il mio signore'.
\par 28 Allora Mosè dette per loro degli ordini al sacerdote Eleazar, a Giosuè figliuolo di Nun e ai capi famiglia delle tribù de' figliuoli d'Israele.
\par 29 Mosè disse loro: 'Se i figliuoli di Gad e i figliuoli di Ruben passano con voi il Giordano tutti armati per combattere davanti all'Eterno, e se il paese sarà sottomesso davanti a voi, darete loro come proprietà il paese di Galaad.
\par 30 Ma se non passano armati con voi, avranno la loro proprietà tra voi nel paese di Canaan'.
\par 31 E i figliuoli di Gad e i figliuoli di Ruben risposero dicendo: 'Faremo come l'Eterno ha detto ai tuoi servi.
\par 32 Passeremo in armi, davanti all'Eterno, nel paese di Canaan; ma il possesso della nostra eredità resti per noi di qua dal Giordano'.
\par 33 Mosè dunque dette ai figliuoli di Gad, ai figliuoli di Ruben e alla metà della tribù di Manasse, figliuolo di Giuseppe, il regno di Sihon, re degli Amorei, e il regno di Og, re di Basan: il paese, le sue città e i territori delle città del paese all'intorno.
\par 34 E i figliuoli di Gad edificarono Dibon, Ataroth, Aroer, Atroth-Shofan,
\par 35 Iazer, Iogbehah,
\par 36 Beth-Nimra e Beth-Haran, città fortificate, e fecero de' recinti per i greggi.
\par 37 E i figliuoli di Ruben edificarono Heshbon, Elealeh, Kiriathaim, Nebo e Baal-Meon,
\par 38 i cui nomi furon mutati, e Sibmah, e dettero dei nomi alle città che edificarono.
\par 39 E i figliuoli di Makir, figliuolo di Manasse, andarono nel paese di Galaad, lo presero, e ne cacciarono gli Amorei che vi stavano.
\par 40 Mosè dunque dette Galaad a Makir, figliuolo di Manasse, che vi si stabilì.
\par 41 Iair, figliuolo di Manasse, andò anch'egli e prese i loro borghi, e li chiamò Havvoth-Iair.
\par 42 E Nobah andò e prese Kenath co' suoi villaggi, e le diede il suo nome di Nobah.

\chapter{33}

\par 1 Queste sono le tappe dei figliuoli d'Israele che uscirono dal paese d'Egitto, secondo le loro schiere, sotto la guida di Mosè e di Aaronne.
\par 2 Or Mosè mise in iscritto le loro marce, tappa per tappa, per ordine dell'Eterno; e queste sono le loro tappe nell'ordine delle loro marce.
\par 3 Partirono da Rameses il primo mese, il quindicesimo giorno del primo mese. Il giorno dopo la Pasqua i figliuoli d'Israele partirono a test'alta, a vista di tutti gli Egiziani,
\par 4 mentre gli Egiziani seppellivano quelli che l'Eterno avea colpiti fra loro, cioè tutti i primogeniti, allorché anche i loro dèi erano stati colpiti dal giudizio dell'Eterno.
\par 5 I figliuoli d'Israele partiron dunque da Rameses e si accamparono a Succoth.
\par 6 Partirono da Succoth e si accamparono a Etham che è all'estremità del deserto.
\par 7 Partirono da Etham e piegarono verso Pi-Hahiroth che è dirimpetto a Baal-Tsefon, e si accamparono davanti a Migdol.
\par 8 Partirono d'innanzi ad Hahiroth, attraversarono il mare in direzione del deserto, fecero tre giornate di marcia nel deserto di Etham e si accamparono a Mara.
\par 9 Partirono da Mara e giunsero ad Elim; ad Elim c'erano dodici sorgenti d'acqua e settanta palme; e quivi si accamparono.
\par 10 Partirono da Elim e si accamparono presso il mar Rosso.
\par 11 Partirono dal mar Rosso e si accamparono nel deserto di Sin.
\par 12 Partirono dal deserto di Sin e si accamparono a Dofka.
\par 13 Partirono da Dofka e si accamparono ad Alush.
\par 14 Partirono da Alush e si accamparono a Refidim dove non c'era acqua da bere per il popolo.
\par 15 Partirono da Refidim e si accamparono nel deserto di Sinai.
\par 16 Partirono dal deserto di Sinai e si accamparono a Kibroth-Hattaava.
\par 17 Partirono da Kibroth-Hattaava e si accamparono a Hatseroth.
\par 18 Partirono da Hatseroth e si accamparono a Rithma.
\par 19 Partirono da Rithma e si accamparono a Rimmon-Perets.
\par 20 Partirono da Rimmon-Perets e si accamparono a Libna.
\par 21 Partirono da Libna e si accamparono a Rissa.
\par 22 Partirono da Rissa e si accamparono a Kehelatha.
\par 23 Partirono da Kehelatha e si accamparono al monte di Scefer.
\par 24 Partirono dal monte di Scefer e si accamparono a Harada.
\par 25 Partirono da Harada e si accamparono a Makheloth.
\par 26 Partirono da Makheloth e si accamparono a Tahath.
\par 27 Partirono da Tahath e si accamparono a Tarach.
\par 28 Partirono da Tarach e si accamparono a Mithka.
\par 29 Partirono da Mithka e si accamparono a Hashmona.
\par 30 Partirono da Hashmona e si accamparono a Moseroth.
\par 31 Partirono da Moseroth e si accamparono a Bene-Jaakan.
\par 32 Partirono da Bene-Jaakan e si accamparono a Hor-Ghidgad.
\par 33 Partirono da Hor-Ghidgad e si accamparono a Jotbathah.
\par 34 Partirono da Jotbathah e si accamparono a Abrona.
\par 35 Partirono da Abrona e si accamparono a Etsion-Gheber.
\par 36 Partirono da Etsion-Gheber e si accamparono nel deserto di Tsin, cioè a Kades.
\par 37 Poi partirono da Kades e si accamparono al monte Hor all'estremità del paese di Edom.
\par 38 E il sacerdote Aaronne salì sul monte Hor per ordine dell'Eterno, e quivi morì il quarantesimo anno dopo l'uscita de' figliuoli d'Israele dal paese di Egitto, il quinto mese, il primo giorno del mese.
\par 39 Aaronne era in età di centoventitre anni quando morì sul monte Hor.
\par 40 E il Cananeo re di Arad, che abitava il mezzogiorno del paese di Canaan, udì che i figliuoli d'Israele arrivavano.
\par 41 E quelli partirono dal monte Hor e si accamparono a Tsalmona.
\par 42 Partirono da Tsalmona e si accamparono a Punon.
\par 43 Partirono da Punon e si accamparono a Oboth.
\par 44 Partirono da Oboth e si accamparono a Ije-Abarim sui confini di Moab.
\par 45 Partirono da Ijim e si accamparono a Dibon-Gad.
\par 46 Partirono da Dibon-Gad e si accamparono a Almon-Diblathaim.
\par 47 Partirono da Almon-Diblathaim e si accamparono ai monti d'Abarim dirimpetto a Nebo.
\par 48 Partirono dai monti d'Abarim e si accamparono nelle pianure di Moab, presso il Giordano di faccia a Gerico.
\par 49 E si accamparono presso al Giordano, da Beth-Jescimoth fino ad Abel-Sittim, nelle pianure di Moab.
\par 50 E l'Eterno parlò a Mosè, nelle pianure di Moab, presso al Giordano di faccia a Gerico, dicendo:
\par 51 'Parla ai figliuoli d'Israele, e di' loro: Quando avrete passato il Giordano e sarete entrati nel paese di Canaan,
\par 52 caccerete d'innanzi a voi tutti gli abitanti del paese, distruggerete tutte le loro immagini, distruggerete tutte le loro statue di getto e demolirete tutti i loro alti luoghi.
\par 53 Prenderete possesso del paese, e in esso vi stabilirete, perché io vi ho dato il paese affinché lo possediate.
\par 54 Dividerete il paese a sorte, secondo le vostre famiglie. A quelle che sono più numerose darete una porzione maggiore, e a quelle che sono meno numerose darete una porzione minore. Ognuno possederà quello che gli sarà toccato a sorte; vi spartirete il possesso secondo le tribù dei vostri padri.
\par 55 Ma se non cacciate d'innanzi a voi gli abitanti del paese, quelli di loro che vi avrete lasciato saranno per voi come spine negli occhi e pungoli ne' fianchi, e vi faranno tribolare nel paese che abiterete.
\par 56 E avverrà che io tratterò voi come mi ero proposto di trattar loro'.

\chapter{34}

\par 1 L'Eterno parlò ancora a Mosè, dicendo:
\par 2 'Da' quest'ordine ai figliuoli d'Israele, e di' loro: Quando entrerete nel paese di Canaan, questo sarà il paese che vi toccherà come eredità: il paese di Canaan, di cui ecco i confini:
\par 3 la vostra regione meridionale comincerà al deserto di Tsin, vicino a Edom; così la vostra frontiera meridionale partirà dalla estremità del mar Salato, verso oriente;
\par 4 e questa frontiera volgerà al sud della salita di Akrabbim, passerà per Tsin, e si estenderà a mezzogiorno di Kades-Barnea; poi continuerà verso Hatsar-Addar, e passerà per Atsmon.
\par 5 Da Atsmon la frontiera girerà fino al torrente d'Egitto, e finirà al mare.
\par 6 La vostra frontiera a occidente sarà il mar grande: quella sarà la vostra frontiera occidentale.
\par 7 E questa sarà la vostra frontiera settentrionale: partendo dal mar grande, la traccerete fino al monte Hor;
\par 8 dal monte Hor la traccerete fin là dove s'entra in Hamath, e l'estremità della frontiera sarà a Tsedad;
\par 9 la frontiera continuerà fino a Zifron, per finire a Hatsar-Enan: questa sarà la vostra frontiera settentrionale.
\par 10 Traccerete la vostra frontiera orientale da Hatsar-Enan a Scefam;
\par 11 la frontiera scenderà da Scefam verso Ribla, a oriente di Ain; poi la frontiera scenderà, e si estenderà lungo il mare di Kinnereth, a oriente;
\par 12 poi la frontiera scenderà verso il Giordano, e finirà al mar Salato. Tale sarà il vostro paese con le sue frontiere tutt'intorno'.
\par 13 E Mosè trasmise quest'ordine ai figliuoli d'Israele, e disse loro: 'Questo è il paese che vi distribuirete a sorte, e che l'Eterno ha ordinato si dia a nove tribù e mezzo;
\par 14 poiché la tribù dei figliuoli di Ruben, secondo le case dei loro padri, e la tribù dei figliuoli di Gad, secondo le case de' loro padri, e la mezza tribù di Manasse hanno ricevuto la loro porzione.
\par 15 Queste due tribù e mezzo hanno ricevuto la loro porzione di qua dal Giordano di Gerico, dal lato d'oriente'.
\par 16 E l'Eterno parlò a Mosè, dicendo:
\par 17 'Questi sono i nomi degli uomini che spartiranno il paese fra voi: il sacerdote Eleazar, e Giosuè, figliuolo di Nun.
\par 18 Prenderete anche un principe d'ogni tribù per fare la spartizione del paese.
\par 19 Ecco i nomi di questi uomini. Per la tribù di Giuda: Caleb, figliuolo di Gefunne.
\par 20 Per la tribù de' figliuoli di Simeone: Samuele, figliuolo di Ammihud.
\par 21 Per la tribù di Beniamino: Elidad, figliuolo di Kislon.
\par 22 Per la tribù de' figliuoli di Dan: il principe Buki, figliuolo di Iogli.
\par 23 Per i figliuoli di Giuseppe: per la tribù de' figliuoli di Manasse, il principe Hanniel, figliuolo d'Efod;
\par 24 e per la tribù de' figliuoli d'Efraim, il principe Kemuel, figliuolo di Sciftan.
\par 25 Per la tribù de' figliuoli di Zabulon: il principe Elitsafan, figliuolo di Parnac.
\par 26 Per la tribù de' figliuoli di Issacar: il principe Paltiel, figliuolo d'Azzan.
\par 27 Per la tribù de' figliuoli di Ascer: il principe Ahihud, figliuolo di Scelomi.
\par 28 E per la tribù de' figliuoli di Neftali: il principe Pedahel, figliuolo d'Ammihud'.
\par 29 Queste sono le persone alle quali l'Eterno ordinò di spartire il possesso del paese di Canaan tra i figliuoli d'Israele.

\chapter{35}

\par 1 L'Eterno parlò ancora a Mosè nelle pianure di Moab presso il Giordano, di faccia a Gerico, dicendo:
\par 2 'Ordina ai figliuoli d'Israele che, della eredità che possederanno diano ai Leviti delle città da abitare; darete pure ai Leviti il contado ch'è intorno alle città.
\par 3 Ed essi avranno le città per abitarvi; e il contado servirà per i loro bestiami, per i loro beni e per tutti i loro animali.
\par 4 Il contado delle città che darete ai Leviti si estenderà fuori per lo spazio di mille cubiti dalle mura della città, tutt'intorno.
\par 5 Misurerete dunque, fuori della città, duemila cubiti dal lato orientale, duemila cubiti dal lato meridionale, duemila cubiti dal lato occidentale e duemila cubiti dal lato settentrionale; la città sarà in mezzo. Tale sarà il contado di ciascuna delle loro città.
\par 6 Tra le città che darete ai Leviti ci saranno le sei città di rifugio, che voi designerete perché vi si rifugi l'omicida; e a queste aggiungerete altre quarantadue città.
\par 7 Tutte le città che darete ai Leviti saranno dunque quarantotto, col relativo contado.
\par 8 E di queste città che darete ai Leviti, prendendole dalla proprietà dei figliuoli d'Israele, ne prenderete di più da quelli che ne hanno di più, e di meno da quelli che ne hanno di meno; ognuno darà, delle sue città, ai Leviti, in proporzione della eredità che gli sarà toccata'.
\par 9 Poi l'Eterno parlò a Mosè, dicendo:
\par 10 'Parla ai figliuoli d'Israele e di' loro: Quando avrete passato il Giordano e sarete entrati nel paese di Canaan,
\par 11 designerete delle città che siano per voi delle città di rifugio, dove possa ricoverarsi l'omicida che avrà ucciso qualcuno involontariamente.
\par 12 Queste città vi serviranno di rifugio contro il vindice del sangue, affinché l'omicida non sia messo a morte prima d'esser comparso in giudizio dinanzi alla raunanza.
\par 13 Delle città che darete, sei saranno dunque per voi città di rifugio.
\par 14 Darete tre città di qua dal Giordano, e darete tre altre città nel paese di Canaan; e saranno città di rifugio.
\par 15 Queste sei città serviranno di rifugio ai figliuoli d'Israele, allo straniero e a colui che soggiornerà fra voi, affinché vi scampi chiunque abbia ucciso qualcuno involontariamente.
\par 16 Ma se uno colpisce un altro con uno stromento di ferro, sì che quello ne muoia, quel tale è un omicida; l'omicida dovrà esser punito di morte.
\par 17 E se lo colpisce con una pietra che aveva in mano, atta a causare la morte, e il colpito muore, quel tale è un omicida; l'omicida dovrà esser punito di morte.
\par 18 O se lo colpisce con uno stromento di legno che aveva in mano, atto a causare la morte, e il colpito muore, quel tale è un omicida; l'omicida dovrà esser punito di morte.
\par 19 Sarà il vindice del sangue quegli che metterà a morte l'omicida; quando lo incontrerà, l'ucciderà.
\par 20 Se uno dà a un altro una spinta per odio, o gli getta contro qualcosa con premeditazione, sì che quello ne muoia,
\par 21 o lo colpisce per inimicizia con la mano, sì che quello ne muoia, colui che ha colpito dovrà esser punito di morte; è un omicida; il vindice del sangue ucciderà l'omicida quando lo incontrerà.
\par 22 Ma se gli dà una spinta per caso e non per inimicizia, o gli getta contro qualcosa senza premeditazione,
\par 23 o se, senza vederlo, gli fa cadere addosso una pietra che possa causare la morte, e quello ne muore, senza che l'altro gli fosse nemico o gli volesse fare del male,
\par 24 allora ecco le norme secondo le quali la raunanza giudicherà fra colui che ha colpito e il vindice del sangue.
\par 25 La raunanza libererà l'omicida dalle mani del vindice del sangue e lo farà tornare alla città di rifugio dove s'era ricoverato. Quivi dimorerà, fino alla morte del sommo sacerdote che fu unto con l'olio santo.
\par 26 Ma se l'omicida esce dai confini della città di rifugio dove s'era ricoverato,
\par 27 e se il vindice del sangue trova l'omicida fuori de' confini della sua città di rifugio e l'uccide, il vindice del sangue non sarà responsabile del sangue versato.
\par 28 Poiché l'omicida deve stare nella sua città di rifugio fino alla morte del sommo sacerdote; ma, dopo la morte del sommo sacerdote, l'omicida potrà tornare nella terra di sua proprietà.
\par 29 Queste vi servano come norme di diritto, di generazione in generazione, dovunque dimorerete.
\par 30 Se uno uccide un altro, l'omicida sarà messo a morte in seguito a deposizione di testimoni; ma un unico testimone non basterà per far condannare una persona a morte.
\par 31 Non accetterete prezzo di riscatto per la vita d'un omicida colpevole e degno di morte, perché dovrà esser punito di morte.
\par 32 Non accetterete prezzo di riscatto che permetta a un omicida di ricoverarsi nella sua città di rifugio e di tornare ad abitare nel paese prima della morte del sacerdote.
\par 33 Non contaminerete il paese dove sarete, perché il sangue contamina il paese; e non si potrà fare per il paese alcuna espiazione del sangue che vi sarà stato sparso, se non mediante il sangue di colui che l'avrà sparso.
\par 34 Non contaminerete dunque il paese che andate ad abitare, e in mezzo al quale io dimorerò; poiché io sono l'Eterno che dimoro in mezzo ai figliuoli d'Israele'.

\chapter{36}

\par 1 Or i capi famiglia dei figliuoli di Galaad, figliuolo di Makir, figliuolo di Manasse, di tra le famiglie de' figliuoli di Giuseppe, si fecero avanti a parlare in presenza di Mosè e dei principi capi famiglia dei figliuoli d'Israele,
\par 2 e dissero: 'L'Eterno ha ordinato al mio signore di dare il paese in eredità ai figliuoli d'Israele, a sorte; e il mio signore ha pure ricevuto l'ordine dall'Eterno di dare l'eredità di Tselofehad, nostro fratello, alle figliuole di lui.
\par 3 Se queste si maritano a qualcuno de' figliuoli delle altre tribù de' figliuoli d'Israele, la loro eredità sarà detratta dall'eredità de' nostri padri, e aggiunta all'eredità della tribù nella quale esse saranno entrate; così sarà detratta dall'eredità che ci è toccata a sorte.
\par 4 E quando verrà il giubileo per i figliuoli d'Israele, la loro eredità sarà aggiunta a quella della tribù nella quale saranno entrate, e l'eredità loro sarà detratta dall'eredità della tribù de' nostri padri'.
\par 5 E Mosè trasmise ai figliuoli d'Israele questi ordini dell'Eterno, dicendo: 'La tribù dei figliuoli di Giuseppe dice bene.
\par 6 Questo è quel che l'Eterno ha ordinato riguardo alle figliuole di Tselofehad: si mariteranno a chi vorranno, purché si maritino in una famiglia della tribù de' loro padri.
\par 7 Cosicché, nessuna eredità, tra i figliuoli d'Israele, passerà da una tribù all'altra, poiché ciascuno dei figliuoli d'Israele si terrà stretto all'eredità della tribù dei suoi padri.
\par 8 E ogni fanciulla che possiede un'eredità in una delle tribù dei figliuoli d'Israele, si mariterà a qualcuno d'una famiglia della tribù di suo padre, affinché ognuno dei figliuoli d'Israele possegga l'eredità de' suoi padri.
\par 9 Così nessuna eredità passerà da una tribù all'altra, ma ognuna delle tribù de' figliuoli d'Israele si terrà stretta alla propria eredità'.
\par 10 Le figliuole di Tselofehad si conformarono all'ordine che l'Eterno aveva dato a Mosè.
\par 11 Mahlah, Thirtsah, Hoglah, Milcah e Noah, figliuole di Tselofehad, si maritarono coi figliuoli dei loro zii;
\par 12 si maritarono nelle famiglie de' figliuoli di Manasse, figliuolo di Giuseppe, e la loro eredità rimase nella tribù della famiglia del padre loro.
\par 13 Tali sono i comandamenti e le leggi che l'Eterno dette ai figliuoli d'Israele per mezzo di Mosè, nelle pianure di Moab, presso al Giordano, di faccia a Gerico.


\end{document}