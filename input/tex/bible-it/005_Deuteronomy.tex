\begin{document}

\title{Deuteronomio}


\chapter{1}

\par 1 Queste sono le parole che Mosè rivolse a Israele di là dal Giordano, nel deserto, nella pianura dirimpetto a Suf, fra Paran, Tofel, Laban, Hatseroth e Di-Zahab.
\par 2 (Vi sono undici giornate dallo Horeb, per la via del monte Seir, fino a Kades-Barnea).
\par 3 Il quarantesimo anno, l'undecimo mese, il primo giorno del mese, Mosè parlò ai figliuoli d'Israele, secondo tutto quello che l'Eterno gli aveva ordinato di dir loro.
\par 4 Questo avvenne dopo ch'egli ebbe sconfitto Sihon, re degli Amorei che abitava in Heshbon, e Og, re di Basan che abitava in Astaroth e in Edrei.
\par 5 Di là dal Giordano, nel paese di Moab, Mosè cominciò a spiegare questa legge, dicendo:
\par 6 L'Eterno, l'Iddio nostro, ci parlò in Horeb e ci disse: 'Voi avete dimorato abbastanza in queste montagne;
\par 7 voltatevi, partite, e andate nella contrada montuosa degli Amorei e in tutte le vicinanze, nella pianura, sui monti, nella regione bassa, nel mezzogiorno, sulla costa del mare, nel paese dei Cananei ed al Libano, fino al gran fiume, il fiume Eufrate.
\par 8 Ecco, io v'ho posto il paese dinanzi; entrate, prendete possesso del paese che l'Eterno giurò di dare ai vostri padri, Abrahamo, Isacco e Giacobbe, e alla loro progenie dopo di loro'.
\par 9 In quel tempo io vi parlai e vi dissi: 'Io non posso da solo sostenere il carico del popolo.
\par 10 L'Eterno, ch'è il vostro Dio, vi ha moltiplicati, ed ecco che oggi siete numerosi come le stelle del cielo.
\par 11 - L'Eterno, l'Iddio de' vostri padri, vi aumenti anche mille volte di più, e vi benedica come vi ha promesso di fare! -
\par 12 Ma come posso io, da solo, portare il vostro carico, il vostro peso e le vostre liti?
\par 13 Prendete nelle vostre tribù degli uomini savi, intelligenti e conosciuti, e io ve li stabilirò come capi'.
\par 14 E voi mi rispondeste, dicendo: 'È bene che facciamo quel che tu proponi'.
\par 15 Allora presi i capi delle vostre tribù, uomini savi e conosciuti, e li stabilii sopra voi come capi di migliaia, capi di centinaia, capi di cinquantine, capi di diecine, e come ufficiali nelle vostre tribù.
\par 16 E in quel tempo detti quest'ordine ai vostri giudici: 'Ascoltate le cause de' vostri fratelli, e giudicate con giustizia le questioni che uno può avere col fratello o con lo straniero che sta da lui.
\par 17 Nei vostri giudizi non avrete riguardi personali; darete ascolto al piccolo come al grande; non temerete alcun uomo, poiché il giudizio appartiene a Dio; e le cause troppo difficili per voi le recherete a me, e io le udirò'.
\par 18 Così, in quel tempo, io vi ordinai tutte le cose che dovevate fare.
\par 19 Poi partimmo dallo Horeb e attraversammo tutto quel grande e spaventevole deserto che avete veduto, dirigendoci verso la contrada montuosa degli Amorei, come l'Eterno, l'Iddio nostro, ci aveva ordinato di fare, e giungemmo a Kades-Barnea.
\par 20 Allora vi dissi: 'Siete arrivati alla contrada montuosa degli Amorei, che l'Eterno, l'Iddio nostro, ci dà.
\par 21 Ecco, l'Eterno, il tuo Dio, t'ha posto il paese dinanzi; sali, prendine possesso, come l'Eterno, l'Iddio de' tuoi padri, t'ha detto; non temere, e non ti spaventare'.
\par 22 E voi vi accostaste a me tutti quanti, e diceste: 'Mandiamo degli uomini davanti a noi, che ci esplorino il paese, e ci riferiscano qualcosa del cammino per il quale noi dovremo salire, e delle città alle quali dovremo arrivare'.
\par 23 La cosa mi piacque, e presi dodici uomini tra voi, uno per tribù.
\par 24 Quelli si incamminarono, salirono verso i monti, giunsero alla valle d'Eshcol, ed esplorarono il paese.
\par 25 Presero con le loro mani de' frutti del paese, ce li portarono, e ci fecero la loro relazione dicendo: 'Quello che l'Eterno, il nostro Dio, ci dà, è un buon paese'.
\par 26 Ma voi non voleste salirvi, e vi ribellaste all'ordine dell'Eterno, del vostro Dio;
\par 27 mormoraste nelle vostre tende, e diceste: 'L'Eterno ci odia, per questo ci ha fatti uscire dal paese d'Egitto per darci in mano agli Amorei e per distruggerci.
\par 28 Dove saliam noi? I nostri fratelli ci han fatto struggere il cuore, dicendo: Quella gente è più grande e più alta di noi; le città vi sono grandi e fortificate fino al cielo; e abbiam perfino visto colà de' figliuoli degli Anakim'.
\par 29 E io vi dissi: 'Non vi sgomentate, e non abbiate paura di loro.
\par 30 L'Eterno, l'Iddio vostro, che va davanti a voi, combatterà egli stesso per voi, come ha fatto tante volte sotto gli occhi vostri, in Egitto
\par 31 e nel deserto, dove hai veduto come l'Eterno, il tuo Dio, ti ha portato come un uomo porta il suo figliuolo, per tutto il cammino che avete fatto, finché siete arrivati a questo luogo'.
\par 32 Nonostante questo non aveste fiducia nell'Eterno, nell'Iddio vostro,
\par 33 che andava innanzi a voi nel cammino per cercarvi un luogo da piantar le tende: di notte, nel fuoco, per mostrarvi la via per la quale dovevate andare, e, di giorno, nella nuvola.
\par 34 E l'Eterno udì le vostre parole, si adirò gravemente, e giurò dicendo:
\par 35 'Certo, nessuno degli uomini di questa malvagia generazione vedrà il buon paese che ho giurato di dare ai vostri padri,
\par 36 salvo Caleb, figliuolo di Gefunne. Egli lo vedrà; e a lui e ai suoi figliuoli darò la terra che egli ha calcato, perché ha pienamente seguito l'Eterno'.
\par 37 Anche contro a me l'Eterno si adirò per via di voi, e disse: 'Neanche tu v'entrerai;
\par 38 Giosuè, figliuolo di Nun, che ti serve, v'entrerà; fortificalo, perch'egli metterà Israele in possesso di questo paese.
\par 39 E i vostri fanciulli, de' quali avete detto: Diventeranno tanta preda! e i vostri figliuoli, che oggi non conoscono né il bene né il male, sono quelli che v'entreranno; a loro lo darò, e saranno essi che lo possederanno.
\par 40 Ma voi, tornate indietro e avviatevi verso il deserto, in direzione del mar Rosso'.
\par 41 Allora voi rispondeste, dicendomi: 'Abbiam peccato contro l'Eterno; noi saliremo e combatteremo, interamente come l'Eterno, l'Iddio nostro, ci ha ordinato'. E ognun di voi cinse le armi, e vi metteste temerariamente a salire verso i monti.
\par 42 E l'Eterno mi disse: 'Di' loro: Non salite, e non combattete, perché io non sono in mezzo a voi; voi sareste sconfitti davanti ai vostri nemici'.
\par 43 Io ve lo dissi, ma voi non mi deste ascolto; anzi foste ribelli all'ordine dell'Eterno, foste presuntuosi, e vi metteste a salire verso i monti.
\par 44 Allora gli Amorei, che abitano quella contrada montuosa, uscirono contro a voi, v'inseguirono come fanno le api, e vi batterono in Seir fino a Horma.
\par 45 E voi tornaste e piangeste davanti all'Eterno; ma l'Eterno non dette ascolto alla vostra voce e non vi porse orecchio.
\par 46 Così rimaneste in Kades molti giorni; e ben sapete quanti giorni vi siete rimasti.

\chapter{2}

\par 1 Poi tornammo indietro e partimmo per il deserto in direzione del mar Rosso, come l'Eterno m'avea detto, e girammo attorno al monte Seir per lungo tempo.
\par 2 E l'Eterno mi parlò dicendo:
\par 3 'Avete girato abbastanza attorno a questo monte; volgetevi verso settentrione.
\par 4 E da' quest'ordine al popolo: Voi state per passare i confini de' figliuoli d'Esaù, vostri fratelli, che dimorano in Seir; ed essi avranno paura di voi; state quindi bene in guardia;
\par 5 non movete lor guerra, poiché del loro paese io non vi darò neppur quanto ne può calcare un piede; giacché ho dato il monte di Seir a Esaù, come sua proprietà.
\par 6 Comprerete da loro a danaro contante le vettovaglie che mangerete, e comprerete pure da loro con tanto danaro l'acqua che berrete.
\par 7 Poiché l'Eterno, il tuo Dio, ti ha benedetto in tutta l'opera delle tue mani, t'ha seguito nel tuo viaggio attraverso questo gran deserto; l'Eterno, il tuo Dio, è stato teco durante questi quarant'anni, e non t'è mancato nulla'.
\par 8 Così passammo, lasciando a distanza i figliuoli di Esaù, nostri fratelli, che abitano in Seir, ed evitando la via della pianura, come pure Elath ed Etsion-Gheber. Poi ci voltammo, e c'incamminammo verso il deserto di Moab.
\par 9 E l'Eterno mi disse: 'Non attaccare Moab e non gli muover guerra, poiché io non ti darò nulla da possedere nel suo paese, giacché ho dato Ar ai figliuoli di Lot, come loro proprietà.
\par 10 (Prima vi abitavano gli Emim: popolo grande, numeroso, alto di statura come gli Anakim.
\par 11 Erano anch'essi tenuti in conto di Refaim, come gli Anakim; ma i Moabiti li chiamavano Emim.
\par 12 Anche Seir era prima abitata dagli Horei; ma i figliuoli di Esaù li cacciarono, li distrussero e si stabilirono in luogo loro, come ha fatto Israele nel paese che possiede e che l'Eterno gli ha dato).
\par 13 Ora levatevi, e passate il torrente di Zered'. E noi passammo il torrente di Zered.
\par 14 Or il tempo che durarono le nostre marce, da Kades-Barnea al passaggio del torrente di Zered, fu di trentotto anni, finché tutta quella generazione degli uomini di guerra scomparve interamente dal campo, come l'Eterno l'avea loro giurato.
\par 15 E infatti la mano dell'Eterno fu contro a loro per sterminarli dal campo, finché fossero del tutto scomparsi.
\par 16 E quando la morte ebbe finito di consumare tutti quegli uomini di guerra,
\par 17 l'Eterno mi parlò dicendo:
\par 18 'Oggi tu stai per passare i confini di Moab, ad Ar, e ti avvicinerai ai figliuoli di Ammon.
\par 19 Non li attaccare e non muover loro guerra, perché io non ti darò nulla da possedere nel paese de' figliuoli di Ammon, giacché l'ho dato ai figliuoli di Lot, come loro proprietà.
\par 20 (Anche questo paese era reputato paese di Refaim: prima vi abitavano dei Refaim, e gli Ammoniti li chiamavano Zamzummim:
\par 21 popolo grande, numeroso, alto di statura come gli Anakim; ma l'Eterno li distrusse davanti agli Ammoniti, che li cacciarono e si stabilirono nel luogo loro.
\par 22 Così l'Eterno avea fatto per i figliuoli d'Esaù che abitano in Seir, quando distrusse gli Horei davanti a loro; essi li cacciarono e si stabilirono nel luogo loro, e vi son rimasti fino al dì d'oggi.
\par 23 E anche gli Avvei, che dimoravano in villaggi fino a Gaza, furon distrutti dai Caftorei, usciti da Caftor, i quali si stabilirono nel luogo loro).
\par 24 Levatevi, partite, e passate la valle dell'Arnon; ecco, io do in tuo potere Sihon, l'Amoreo, re di Heshbon, e il suo paese; comincia a prenderne possesso, e muovigli guerra.
\par 25 Oggi comincerò a ispirare paura e terrore di te ai popoli che sono sotto il cielo intero, sì che, all'udire la tua fama, tremeranno e saranno presi d'angoscia dinanzi a te'.
\par 26 Allora mandai ambasciatori dal deserto di Kedemoth a Sihon, re di Heshbon, con parole di pace, e gli feci dire:
\par 27 'Lasciami passare per il tuo paese; io camminerò per la strada maestra, senza volgermi né a destra né a sinistra.
\par 28 Tu mi venderai a danaro contante le vettovaglie che mangerò, e mi darai per danaro contante l'acqua che berrò; permettimi semplicemente il transito
\par 29 (come m'han fatto i figliuoli d'Esaù che abitano in Seir e i Moabiti che abitano in Ar), finché io abbia passato il Giordano per entrare nel paese che l'Eterno, il nostro Dio, ci dà'.
\par 30 Ma Sihon, re di Heshbon, non ci volle lasciar passare per il suo paese, perché l'Eterno, il tuo Dio, gli aveva indurato lo spirito e reso ostinato il cuore, per dartelo nelle mani, come difatti oggi si vede.
\par 31 E l'Eterno mi disse: 'Vedi, ho principiato a dare in tuo potere Sihon e il suo paese; comincia la conquista, impadronendoti del suo paese'.
\par 32 Allora Sihon uscì contro a noi con tutta la sua gente, per darci battaglia a Iahats.
\par 33 E l'Eterno, l'Iddio nostro, ce lo diè nelle mani, e noi ponemmo in rotta lui, i suoi figliuoli e tutta la sua gente.
\par 34 E in quel tempo prendemmo tutte le sue città e votammo allo sterminio ogni città, uomini, donne, bambini; non vi lasciammo anima viva.
\par 35 Ma riserbammo come nostra preda il bestiame e le spoglie delle città che avevamo prese.
\par 36 Da Aroer, che è sull'orlo della valle dell'Arnon e dalla città che è nella valle, fino a Galaad, non ci fu città che fosse troppo forte per noi: l'Eterno, l'Iddio nostro, le diè tutte in nostro potere.
\par 37 Ma non ti avvicinasti al paese de' figliuoli di Ammon, ad alcun posto toccato dal torrente di Iabbok, alle città del paese montuoso, a tutti i luoghi che l'Eterno il nostro Dio, ci avea proibito d'attaccare.

\chapter{3}

\par 1 Poi ci voltammo, e salimmo per la via di Basan; e Og, re di Basan, con tutta la sua gente, ci uscì contro per darci battaglia a Edrei.
\par 2 E l'Eterno mi disse: 'Non lo temere, poiché io ti do nelle mani lui, tutta la sua gente e il suo paese; e tu farai a lui quel che facesti a Sihon, re degli Amorei, che abitava a Heshbon'.
\par 3 Così l'Eterno, il nostro Dio, diede in poter nostro anche Og, re di Basan, con tutta la sua gente; e noi lo battemmo in guisa che non gli restò anima viva.
\par 4 Gli prendemmo in quel tempo tutte le sue città; non ci fu città che noi non prendessimo loro: sessanta città, tutta la contrada d'Argob, il regno di Og in Basan.
\par 5 Tutte queste città erano fortificate, con alte mura, porte e sbarre, senza contare le città aperte, ch'erano in grandissimo numero.
\par 6 Noi le votammo allo sterminio, come avevamo fatto di Sihon, re di Heshbon: votammo allo sterminio ogni città, uomini, donne, bambini.
\par 7 Ma riserbammo come nostra preda tutto il bestiame e le spoglie delle città.
\par 8 In quel tempo dunque prendemmo ai due re degli Amorei il paese ch'è al di là del Giordano, dalla valle dell'Arnon al monte Hermon
\par 9 (il quale Hermon i Sidonii chiamano Sirion, e gli Amorei Senir),
\par 10 tutte le città della pianura, tutto Galaad, tutto Basan fino a Salca e a Edrei, città del regno di Og, in Basan.
\par 11 (Poiché Og, re di Basan, era rimasto solo della stirpe dei Refaim. Ecco, il suo letto, un letto di ferro, non è esso a Rabbah degli Ammoniti? Ha nove cubiti di lunghezza e quattro cubiti di larghezza, a misura di cubito ordinario d'uomo).
\par 12 Fu allora che c'impossessammo di questo paese; io detti ai Rubeniti e ai Gaditi il territorio che si parte da Aroer, presso la valle dell'Arnon, e la metà della contrada montuosa di Galaad con le sue città;
\par 13 e detti alla mezza tribù di Manasse il resto di Galaad e tutto il regno di Og in Basan: tutta la regione di Argob con tutto Basan, che si chiamava il paese dei Refaim.
\par 14 Iair, figliuolo di Manasse, prese tutta la regione di Argob, sino ai confini dei Gheshuriti, e dei Mahacathiti; e chiamò col suo nome le borgate di Basan, che si nominano anche oggi Havvoth-Iair.
\par 15 E detti Galaad a Makir.
\par 16 E ai Rubeniti e ai Gaditi detti una parte di Galaad e il paese fino alla valle dell'Arnon, fino al mezzo della valle che serve di confine, e fino al torrente di Iabbok, frontiera dei figliuoli di Ammon,
\par 17 e la pianura col Giordano che ne segna il confine, da Kinnereth fino al mare della pianura, il mar Salato, appiè delle pendici del Pisga verso l'oriente.
\par 18 Or in quel tempo, io vi detti quest'ordine, dicendo: 'L'Eterno, il vostro Dio, vi ha dato questo paese perché lo possediate. Voi tutti, uomini di valore, marcerete armati alla testa de' figliuoli d'Israele, vostri fratelli.
\par 19 Ma le vostre mogli, i vostri fanciulli e il vostro bestiame (so che del bestiame ne avete molto) rimarranno nelle città che vi ho date,
\par 20 finché l'Eterno abbia dato riposo ai vostri fratelli come ha fatto a voi, e prendano anch'essi possesso del paese che l'Eterno Iddio vostro dà loro al di là del Giordano. Poi ciascuno tornerà nel possesso che io v'ho dato'.
\par 21 In quel tempo, detti anche a Giosuè quest'ordine, dicendo: 'I tuoi occhi hanno veduto tutto quello che l'Iddio vostro, l'Eterno, ha fatto a questi due re; lo stesso farà l'Eterno a tutti i regni nei quali tu stai per entrare.
\par 22 Non li temete, poiché l'Eterno, il vostro Dio, è quegli che combatte per voi'.
\par 23 In quel medesimo tempo, io supplicai l'Eterno, dicendo:
\par 24 'O Signore, o Eterno, tu hai cominciato a mostrare al tuo servo la tua grandezza e la tua mano potente; poiché qual è l'Iddio, in cielo o sulla terra, che possa fare delle opere e dei portenti pari a quelli che fai tu?
\par 25 Deh, lascia ch'io passi e vegga il bel paese ch'è oltre il Giordano e la bella contrada montuosa e il Libano!'
\par 26 Ma l'Eterno si adirò contro di me, per cagion vostra; e non mi esaudì. E l'Eterno mi disse: 'Basta così; non mi parlar più di questa cosa.
\par 27 Sali in vetta al Pisga, volgi lo sguardo a occidente, a settentrione, a mezzogiorno e ad oriente, e contempla il paese con gli occhi tuoi; poiché tu non passerai questo Giordano.
\par 28 Ma da' i tuoi ordini a Giosuè, fortificalo e incoraggialo, perché sarà lui che lo passerà alla testa di questo popolo, e metterà Israele in possesso del paese che vedrai'.
\par 29 Così ci fermammo nella valle dirimpetto a Beth-Peor.

\chapter{4}

\par 1 Ora, dunque, Israele, da' ascolto alle leggi e alle prescrizioni che io v'insegno perché le mettiate in pratica, affinché viviate ed entriate in possesso del paese che l'Eterno, l'Iddio de' vostri padri, vi dà.
\par 2 Non aggiungerete nulla a ciò che io vi comando, e non ne toglierete nulla; ma osserverete i comandamenti dell'Eterno Iddio vostro che io vi prescrivo.
\par 3 Gli occhi vostri videro ciò che l'Eterno fece nel caso di Baal-Peor: come l'Eterno, il tuo Dio, distrusse di mezzo a te tutti quelli ch'erano andati dietro a Baal-Peor:
\par 4 ma voi che vi teneste stretti all'Eterno, all'Iddio vostro, siete oggi tutti in vita.
\par 5 Ecco, io vi ho insegnato leggi e prescrizioni, come l'Eterno, l'Iddio mio, mi ha ordinato, affinché le mettiate in pratica nel paese nel quale state per entrare per prenderne possesso.
\par 6 Le osserverete dunque e le metterete in pratica; poiché quella sarà la vostra sapienza e la vostra intelligenza agli occhi dei popoli, i quali, udendo parlare di tutte queste leggi, diranno: 'Questa grande nazione è il solo popolo savio e intelligente!'
\par 7 Qual è difatti la gran nazione alla quale la divinità sia così vicina come l'Eterno, l'Iddio nostro, è vicino a noi, ogni volta che l'invochiamo?
\par 8 E qual è la gran nazione che abbia delle leggi e delle prescrizioni giuste com'è tutta questa legge ch'io vi espongo quest'oggi?
\par 9 Soltanto, bada bene a te stesso e veglia diligentemente sull'anima tua, onde non avvenga che tu dimentichi le cose che gli occhi tuoi hanno vedute, ed esse non t'escano dal cuore finché ti duri la vita. Falle anzi sapere ai tuoi figliuoli e ai figliuoli de' tuoi figliuoli.
\par 10 Ricordati del giorno che comparisti davanti all'Eterno, all'Iddio tuo, in Horeb, quando l'Eterno mi disse: 'Adunami il popolo, e io farò loro udire le mie parole, ond'essi imparino a temermi tutto il tempo che vivranno sulla terra, e le insegnino ai loro figliuoli'.
\par 11 E voi vi avvicinaste, e vi fermaste appiè del monte; e il monte era tutto in fiamme, che s'innalzavano fino al cielo; e v'eran tenebre, nuvole ed oscurità.
\par 12 E l'Eterno vi parlò di mezzo al fuoco; voi udiste il suono delle parole, ma non vedeste alcuna figura; non udiste che una voce.
\par 13 Ed egli vi promulgò il suo patto, che vi comandò di osservare, cioè le dieci parole; e le scrisse su due tavole di pietra.
\par 14 E a me, in quel tempo, l'Eterno ordinò d'insegnarvi leggi e prescrizioni, perché voi le metteste in pratica nel paese dove state per passare per prenderne possesso.
\par 15 Or dunque, siccome non vedeste alcuna figura il giorno che l'Eterno vi parlò in Horeb in mezzo al fuoco, vegliate diligentemente sulle anime vostre,
\par 16 affinché non vi corrompiate e vi facciate qualche immagine scolpita, la rappresentazione di qualche idolo, la figura d'un uomo o d'una donna,
\par 17 la figura di un animale tra quelli che son sulla terra, la figura d'un uccello che vola nei cieli,
\par 18 la figura d'una bestia che striscia sul suolo, la figura d'un pesce che vive nelle acque sotto la terra;
\par 19 ed anche affinché, alzando gli occhi al cielo e vedendo il sole, la luna, le stelle, tutto l'esercito celeste, tu non sia tratto a prostrarti davanti a quelle cose e ad offrir loro un culto. Quelle cose sono il retaggio che l'Eterno, l'Iddio tuo, ha assegnato a tutti i popoli che sono sotto tutti i cieli;
\par 20 ma voi l'Eterno vi ha presi, v'ha tratti fuori dalla fornace di ferro, dall'Egitto, perché foste un popolo che gli appartenesse in proprio, come oggi difatti siete.
\par 21 Or l'Eterno s'adirò contro di me per cagion vostra, e giurò ch'io non passerei il Giordano e non entrerei nel buon paese che l'Eterno, l'Iddio tuo, ti dà in eredità.
\par 22 Poiché, io dovrò morire in questo paese, senza passare il Giordano; ma voi lo passerete, e possederete quel buon paese.
\par 23 Guardatevi dal dimenticare il patto che l'Eterno, il vostro Dio, ha fermato con voi, e dal farvi alcuna immagine scolpita, o rappresentazione di qualsivoglia cosa che l'Eterno, l'Iddio tuo, t'abbia proibita.
\par 24 Poiché l'Eterno, il tuo Dio, è un fuoco consumante, un Dio geloso.
\par 25 Quando avrai dei figliuoli e de' figliuoli de' tuoi figliuoli e sarete stati lungo tempo nel paese, se vi corrompete, se vi fate delle immagini scolpite, delle rappresentazioni di qualsivoglia cosa, se fate ciò ch'è male agli occhi dell'Eterno, ch'è l'Iddio vostro, per irritarlo,
\par 26 io chiamo oggi in testimonio contro di voi il cielo e la terra, che voi ben presto perirete, scomparendo dal paese di cui andate a prender possesso di là dal Giordano. Voi non vi prolungherete i vostri giorni, ma sarete interamente distrutti.
\par 27 E l'Eterno vi disperderà fra i popoli e non resterete più che un piccolo numero fra le nazioni dove l'Eterno vi condurrà.
\par 28 E quivi servirete a dèi fatti da mano d'uomo, dèi di legno e di pietra, i quali non vedono, non odono, non mangiano, non fiutano.
\par 29 Ma di là cercherai l'Eterno, il tuo Dio; e lo troverai, se lo cercherai con tutto il tuo cuore e con tutta l'anima tua.
\par 30 Nell'angoscia tua, quando tutte queste cose ti saranno avvenute, negli ultimi tempi, tornerai all'Eterno, all'Iddio tuo, e darai ascolto alla sua voce;
\par 31 poiché l'Eterno, l'Iddio tuo, è un Dio pietoso; egli non ti abbandonerà e non ti distruggerà; non dimenticherà il patto che giurò ai tuoi padri.
\par 32 Interroga pure i tempi antichi, che furon prima di te, dal giorno che Dio creò l'uomo sulla terra, e da un'estremità de' cieli all'altra: Ci fu egli mai cosa così grande come questa, e s'udì egli mai cosa simile a questa?
\par 33 ci fu egli mai popolo che udisse la voce di Dio parlante di mezzo al fuoco come l'hai udita tu, e che rimanesse vivo?
\par 34 ci fu egli mai un dio che provasse di venire a prendersi una nazione di mezzo a un'altra nazione mediante prove, segni, miracoli e battaglie, con mano potente e con braccio steso e con grandi terrori, come fece per voi l'Eterno, l'Iddio vostro, in Egitto, sotto i vostri occhi?
\par 35 Tu sei stato testimone di queste cose affinché tu riconosca che l'Eterno è Dio, e che non ve n'è altri fuori di lui.
\par 36 Dal cielo t'ha fatto udire la sua voce per ammaestrarti; e sulla terra t'ha fatto vedere il suo gran fuoco, e tu hai udito le sue parole di mezzo al fuoco.
\par 37 E perch'egli ha amato i tuoi padri, ha scelto la loro progenie dopo loro, ed egli stesso, in persona, ti ha tratto dall'Egitto con la sua gran potenza,
\par 38 per cacciare d'innanzi a te nazioni più grandi e più potenti di te, per farti entrare nel loro paese e per dartene il possesso, come oggi si vede.
\par 39 Sappi dunque oggi e ritieni bene in cuor tuo che l'Eterno è Dio: lassù ne' cieli, e quaggiù sulla terra; e che non ve n'è alcun altro.
\par 40 Osserva dunque le sue leggi e i suoi comandamenti che oggi ti do, affinché sii felice tu e i tuoi figliuoli dopo di te, e affinché tu prolunghi in perpetuo i tuoi giorni nel paese che l'Eterno, l'Iddio tuo, ti dà.
\par 41 Allora Mosè appartò tre città di là dal Giordano, verso oriente,
\par 42 perché servissero di rifugio all'omicida che avesse ucciso il suo prossimo involontariamente, senz'averlo odiato per l'addietro, e perch'egli potesse aver salva la vita, ricoverandosi in una di quelle città.
\par 43 Esse furono Betser, nel deserto, nella regione piana, per i Rubeniti; Ramoth, in Galaad, per i Gaditi, e Golan, in Basan, per i Manassiti.
\par 44 Or questa è la legge che Mosè espose ai figliuoli d'Israele.
\par 45 Queste sono le istruzioni, le leggi e le prescrizioni che Mosè dette ai figliuoli d'Israele quando furono usciti dall'Egitto,
\par 46 di là dal Giordano, nella valle, dirimpetto a Beth-Peor, nel paese di Sihon, re degli Amorei che dimorava a Heshbon, e che Mosè e i figliuoli d'Israele sconfissero quando furono usciti dall'Egitto.
\par 47 Essi s'impossessarono del paese di lui e del paese di Og re di Basan - due re degli Amorei, che stavano di là dal Giordano, verso oriente, -
\par 48 da Aroer, che è sull'orlo della valle dell'Arnon, fino al monte Sion, che è lo Hermon,
\par 49 con tutta la pianura oltre il Giordano, verso oriente, fino al mare della pianura appiè delle pendici del Pisga.

\chapter{5}

\par 1 Mosè convocò tutto Israele, e disse loro: Ascolta, Israele, le leggi e le prescrizioni che oggi io proclamo dinanzi a voi; imparatele, e mettetele diligentemente in pratica.
\par 2 L'Eterno, l'Iddio nostro, fermò con noi un patto in Horeb.
\par 3 L'Eterno non fermò questo patto coi nostri padri, ma con noi, che siam qui oggi tutti quanti in vita.
\par 4 L'Eterno vi parlò faccia a faccia sul monte, di mezzo al fuoco.
\par 5 Io stavo allora fra l'Eterno e voi per riferirvi la parola dell'Eterno; poiché voi avevate paura di quel fuoco, e non saliste sul monte. - Egli disse:
\par 6 'Io sono l'Eterno, l'Iddio tuo, che ti ho tratto fuori dal paese d'Egitto, dalla casa di schiavitù.
\par 7 Non avere altri dèi nel mio cospetto.
\par 8 Non ti fare scultura alcuna né immagine alcuna delle cose che sono lassù nel cielo o quaggiù sulla terra o nelle acque sotto la terra.
\par 9 Non ti prostrare davanti a quelle cose e non servir loro, perché io, l'Eterno, il tuo Dio, sono un Dio geloso che punisco l'iniquità dei padri sopra i figliuoli fino alla terza e alla quarta generazione di quelli che m'odiano,
\par 10 ed uso benignità fino a mille generazioni verso quelli che mi amano e osservano i miei comandamenti.
\par 11 Non usare il nome dell'Eterno, dell'Iddio tuo, in vano, poiché l'Eterno non terrà per innocente chi avrà usato il suo nome in vano.
\par 12 Osserva il giorno del riposo per santificarlo, come l'Eterno, l'Iddio tuo, ti ha comandato.
\par 13 Lavora sei giorni, e fa' in essi tutta l'opera tua;
\par 14 ma il settimo giorno è giorno di riposo consacrato all'Eterno, al tuo Dio: non fare in esso lavoro alcuno, né tu, né il tuo figliuolo, né la tua figliuola, né il tuo servo, né la tua serva, né il tuo bue, né il tuo asino, né alcuna delle tue bestie, né il tuo forestiero che sta dentro le tue porte, affinché il tuo servo e la tua serva si riposino come tu.
\par 15 E ricordati che sei stato schiavo nel paese d'Egitto, e che l'Eterno, l'Iddio tuo, ti ha tratto di là con mano potente e con braccio steso; perciò l'Eterno, il tuo Dio, ti ordina d'osservare il giorno del riposo.
\par 16 Onora tuo padre e tua madre, come l'Eterno, l'Iddio tuo, ti ha comandato, affinché i tuoi giorni siano prolungati, e tu sii felice sulla terra che l'Eterno, l'Iddio tuo, ti dà.
\par 17 Non uccidere.
\par 18 Non commettere adulterio.
\par 19 Non rubare.
\par 20 Non attestare il falso contro il tuo prossimo.
\par 21 Non concupire la moglie del tuo prossimo, e non bramare la casa del tuo prossimo, né il suo campo, né il suo servo, né la sua serva, né il suo bue, né il suo asino, né cosa alcuna che sia del tuo prossimo'.
\par 22 Queste parole pronunziò l'Eterno parlando a tutta la vostra raunanza, sul monte, di mezzo al fuoco, alla nuvola, all'oscurità, con voce forte, e non aggiunse altro. Le scrisse su due tavole di pietra, e me le diede.
\par 23 Or come udiste la voce che usciva dalle tenebre mentre il monte era tutto in fiamme, i vostri capi tribù e i vostri anziani s'accostarono tutti a me, e diceste:
\par 24 'Ecco, l'Eterno, l'Iddio nostro, ci ha fatto vedere la sua gloria e la sua grandezza, e noi abbiamo udito la sua voce di mezzo al fuoco; oggi abbiam veduto che Dio ha parlato con l'uomo e l'uomo è rimasto vivo.
\par 25 Or dunque, perché morremmo noi? giacché questo gran fuoco ci consumerà; se continuiamo a udire ancora la voce dell'Eterno, dell'Iddio nostro, noi morremo.
\par 26 Poiché qual è il mortale, chiunque egli sia, che abbia udito come noi la voce dell'Iddio vivente parlare di mezzo al fuoco e sia rimasto vivo?
\par 27 Accòstati tu e ascolta tutto ciò che l'Eterno, il nostro Dio, dirà; e ci riferirai tutto ciò che l'Eterno, l'Iddio nostro, ti avrà detto, e noi lo ascolteremo e lo faremo'.
\par 28 E l'Eterno udì le vostre parole, mentre mi parlavate; e l'Eterno mi disse: 'Io ho udito le parole che questo popolo ti ha rivolte; tutto quello che hanno detto, sta bene.
\par 29 Oh avessero pur sempre un tal cuore, da temermi e da osservare tutti i miei comandamenti, per esser felici in perpetuo eglino ed i loro figliuoli!
\par 30 Va' e di' loro: Tornate alle vostre tende;
\par 31 ma tu resta qui meco, e io ti dirò tutti i comandamenti, tutte le leggi e le prescrizioni che insegnerai loro, perché le mettano in pratica nel paese di cui do loro il possesso'.
\par 32 Abbiate dunque cura di far ciò che l'Eterno, l'Iddio vostro, vi ha comandato; non ve ne sviate né a destra né a sinistra;
\par 33 camminate in tutto e per tutto per la via che l'Eterno, il vostro Dio, vi ha prescritta, affinché viviate e siate felici e prolunghiate i vostri giorni nel paese di cui avrete il possesso.

\chapter{6}

\par 1 Or questi sono i comandamenti, le leggi e le prescrizioni che l'Eterno, il vostro Dio, ha ordinato di insegnarvi, perché li mettiate in pratica nel paese nel quale state per passare per prenderne possesso;
\par 2 affinché tu tema l'Iddio tuo, l'Eterno, osservando, tutti i giorni della tua vita, tu, il tuo figliuolo e il figliuolo del tuo figliuolo, tutte le sue leggi e tutti i suoi comandamenti che io ti do, e affinché i tuoi giorni siano prolungati.
\par 3 Ascolta dunque, Israele, e abbi cura di metterli in pratica, affinché tu sii felice e moltiplichiate grandemente nel paese ove scorre il latte e il miele, come l'Eterno, l'Iddio de' tuoi padri, ti ha detto.
\par 4 Ascolta, Israele: l'Eterno, l'Iddio nostro, è l'unico Eterno.
\par 5 Tu amerai dunque l'Eterno, il tuo Dio, con tutto il cuore, con tutta l'anima tua e con tutte le tue forze.
\par 6 E questi comandamenti che oggi ti do ti staranno nel cuore;
\par 7 li inculcherai ai tuoi figliuoli, ne parlerai quando te ne starai seduto in casa tua, quando sarai per via, quando ti coricherai e quando ti alzerai.
\par 8 Te li legherai alla mano come un segnale, ti saranno come frontali tra gli occhi,
\par 9 e li scriverai sugli stipiti della tua casa e sulle tue porte.
\par 10 E quando l'Eterno, l'Iddio tuo, t'avrà fatto entrare nel paese che giurò ai tuoi padri, Abrahamo, Isacco e Giacobbe, di darti; quando t'avrà menato alle grandi e buone città che tu non hai edificate,
\par 11 alle case piene d'ogni bene che tu non hai riempite, alle cisterne scavate che tu non hai scavate, alle vigne e agli uliveti che tu non hai piantati, e quando mangerai e sarai satollo,
\par 12 guardati dal dimenticare l'Eterno che ti ha tratto dal paese d'Egitto, dalla casa di schiavitù.
\par 13 Temerai l'Eterno, l'Iddio tuo, lo servirai e giurerai per il suo nome.
\par 14 Non andrete dietro ad altri dèi, fra gli dèi dei popoli che vi staranno attorno,
\par 15 perché l'Iddio tuo, l'Eterno, che sta in mezzo a te, è un Dio geloso; l'ira dell'Eterno, dell'Iddio tuo, s'accenderebbe contro a te e ti sterminerebbe di sulla terra.
\par 16 Non tenterete l'Eterno, il vostro Dio, come lo tentaste a Massa.
\par 17 Osserverete diligentemente i comandamenti dell'Eterno, ch'è l'Iddio vostro, le sue istruzioni e le sue leggi che v'ha date.
\par 18 E farai ciò ch'è giusto e buono agli occhi dell'Eterno, affinché tu sii felice ed entri in possesso del buon paese che l'Eterno giurò ai tuoi padri di darti,
\par 19 dopo ch'egli avrà cacciati tutti i tuoi nemici d'innanzi a te, come l'Eterno ha promesso.
\par 20 Quando, in avvenire, il tuo figliuolo ti domanderà: 'Che significano queste istruzioni, queste leggi e queste prescrizioni che l'Eterno, l'Iddio nostro, vi ha date?'
\par 21 tu risponderai al tuo figliuolo: 'Eravamo schiavi di Faraone in Egitto, e l'Eterno ci trasse dall'Egitto con mano potente.
\par 22 E l'Eterno operò sotto i nostri occhi miracoli e prodigi grandi e disastrosi contro l'Egitto, contro Faraone e contro tutta la sua casa.
\par 23 E ci trasse di là per condurci nel paese che avea giurato ai nostri padri di darci.
\par 24 E l'Eterno ci ordinò di mettere in pratica tutte queste leggi, temendo l'Eterno, l'Iddio nostro, affinché fossimo sempre felici, ed egli ci conservasse in vita, come ha fatto finora.
\par 25 E questa sarà la nostra giustizia: l'aver cura di mettere in pratica tutti questi comandamenti nel cospetto dell'Eterno, dell'Iddio nostro, com'egli ci ha ordinato'.

\chapter{7}

\par 1 Quando l'Iddio tuo, l'Eterno, ti avrà introdotto nel paese dove vai per prenderne possesso, e ne avrà cacciate d'innanzi a te molte nazioni: gli Hittei, i Ghirgasei, gli Amorei, i Cananei, i Ferezei, gli Hivvei e i Gebusei, sette nazioni più grandi e più potenti di te,
\par 2 e quando l'Eterno, l'Iddio tuo, le avrà date in tuo potere e tu le avrai sconfitte, tu le voterai allo sterminio: non farai con esse alleanza, né farai loro grazia.
\par 3 Non t'imparenterai con loro, non darai le tue figliuole ai loro figliuoli e non prenderai le loro figliuole per i tuoi figliuoli,
\par 4 perché stornerebbero i tuoi figliuoli dal seguir me per farli servire a dèi stranieri, e l'ira dell'Eterno s'accenderebbe contro a voi, ed egli ben presto vi distruggerebbe.
\par 5 Ma farete loro così: demolirete i loro altari, spezzerete le loro statue, abbatterete i loro idoli e darete alle fiamme le loro immagini scolpite.
\par 6 Poiché tu sei un popolo consacrato all'Eterno, ch'è l'Iddio tuo; l'Eterno, l'Iddio tuo, ti ha scelto per essere il suo tesoro particolare fra tutti i popoli che sono sulla faccia della terra.
\par 7 L'Eterno ha riposto in voi la sua affezione e vi ha scelti, non perché foste più numerosi di tutti gli altri popoli, ché anzi siete meno numerosi d'ogni altro popolo;
\par 8 ma perché l'Eterno vi ama, perché ha voluto mantenere il giuramento fatto ai vostri padri, l'Eterno vi ha tratti fuori con mano potente e vi ha redenti dalla casa di schiavitù, dalla mano di Faraone, re d'Egitto.
\par 9 Riconosci dunque che l'Eterno, l'Iddio tuo, è Dio: l'Iddio fedele, che mantiene il suo patto e la sua benignità fino alla millesima generazione a quelli che l'amano e osservano i suoi comandamenti,
\par 10 ma rende immediatamente a quelli che l'odiano ciò che si meritano, distruggendoli; non differisce, ma rende immediatamente a chi l'odia ciò che si merita.
\par 11 Osserva dunque i comandamenti, le leggi e le prescrizioni che oggi ti do, mettendoli in pratica.
\par 12 E avverrà che, per aver voi dato ascolto a queste prescrizioni e per averle osservate e messe in pratica, il vostro Dio, l'Eterno, vi manterrà il patto e la benignità che promise con giuramento ai vostri padri.
\par 13 Egli t'amerà, ti benedirà, ti moltiplicherà, benedirà il frutto del tuo seno e il frutto del tuo suolo: il tuo frumento, il tuo mosto e il tuo olio, il figliare delle tue vacche e delle tue pecore, nel paese che giurò ai tuoi padri di darti.
\par 14 Tu sarai benedetto più di tutti i popoli, e non ci sarà in mezzo a te né uomo né donna sterile, né animale sterile fra il tuo bestiame.
\par 15 L'Eterno allontanerà da te ogni malattia, e non manderà su te alcuno di quei morbi funesti d'Egitto che ben conoscesti, ma li farà venire addosso a quelli che t'odiano.
\par 16 Sterminerai dunque tutti i popoli che l'Eterno, l'Iddio tuo, sta per dare in tuo potere; l'occhio tuo non n'abbia pietà; e non servire agli dèi loro, perché ciò ti sarebbe un laccio. Forse dirai in cuor tuo:
\par 17 'Queste nazioni sono più numerose di me; come potrò io cacciarle?'
\par 18 Non le temere; ricordati di quello che l'Eterno, il tuo Dio, fece a Faraone e a tutti gli Egiziani;
\par 19 ricordati delle grandi prove che vedesti con gli occhi tuoi, de' miracoli e dei prodigi, della mano potente e del braccio steso coi quali l'Eterno, l'Iddio tuo, ti trasse dall'Egitto; così farà l'Eterno, l'Iddio tuo, a tutti i popoli, dei quali hai timore.
\par 20 L'Eterno, il tuo Dio, manderà pure contro a loro i calabroni, finché quelli che saranno rimasti e quelli che si saranno nascosti per paura di te, siano periti.
\par 21 Non ti sgomentare per via di loro, poiché l'Iddio tuo, l'Eterno, è in mezzo a te, Dio grande e terribile.
\par 22 E l'Eterno, l'Iddio tuo, caccerà a poco a poco queste nazioni d'innanzi a te; tu non le potrai distruggere a un tratto, perché altrimenti le fiere della campagna moltiplicherebbero a tuo danno;
\par 23 ma il tuo Dio, l'Eterno, le darà in tuo potere, e le metterà interamente in rotta finché siano distrutte.
\par 24 Ti darà nelle mani i loro re, e tu farai scomparire i loro nomi di sotto ai cieli; nessuno potrà starti a fronte, finché tu le abbia distrutte.
\par 25 Darai alle fiamme le immagini scolpite dei loro dèi; non agognerai e non prenderai per te l'argento ch'è su quelle, onde tu non abbia a esserne preso come da un laccio; perché sono un'abominazione per l'Eterno, ch'è l'Iddio tuo;
\par 26 e non introdurrai cosa abominevole in casa tua, perché saresti maledetto, com'è quella cosa; la detesterai e l'abominerai assolutamente, perché è un interdetto.

\chapter{8}

\par 1 Abbiate cura di mettere in pratica tutti i comandamenti che oggi vi do, affinché viviate, moltiplichiate, ed entriate in possesso del paese che l'Eterno giurò di dare ai vostri padri.
\par 2 Ricordati di tutto il cammino che l'Eterno, l'Iddio tuo, ti ha fatto fare questi quarant'anni nel deserto per umiliarti e metterti alla prova, per sapere quello che avevi nel cuore, e se tu osserveresti o no i suoi comandamenti.
\par 3 Egli dunque t'ha umiliato, t'ha fatto provar la fame, poi t'ha nutrito di manna che tu non conoscevi e che i tuoi padri non avean mai conosciuta, per insegnarti che l'uomo non vive soltanto di pane, ma vive di tutto quello che la bocca dell'Eterno avrà ordinato.
\par 4 Il tuo vestito non ti s'è logorato addosso, e il tuo piè non s'è gonfiato durante questi quarant'anni.
\par 5 Riconosci dunque in cuor tuo che, come un uomo corregge il suo figliuolo, così l'Iddio tuo, l'Eterno, corregge te.
\par 6 E osserva i comandamenti dell'Eterno, dell'Iddio tuo, camminando nelle sue vie e temendolo;
\par 7 perché il tuo Dio, l'Eterno, sta per farti entrare in un buon paese: paese di corsi d'acqua, di laghi e di sorgenti che nascono nelle valli e nei monti;
\par 8 paese di frumento, d'orzo, di vigne, di fichi e di melagrani; paese d'ulivi da olio e di miele;
\par 9 paese dove mangerai del pane a volontà, dove non ti mancherà nulla; paese dove le pietre son ferro, e dai cui monti scaverai il rame.
\par 10 Mangerai dunque e ti sazierai, e benedirai l'Eterno, il tuo Dio, a motivo del buon paese che t'avrà dato.
\par 11 Guardati bene dal dimenticare il tuo Dio, l'Eterno, al punto da non osservare i suoi comandamenti, le sue prescrizioni e le sue leggi che oggi ti do;
\par 12 onde non avvenga, dopo che avrai mangiato a sazietà e avrai edificato e abitato delle belle case,
\par 13 dopo che avrai veduto il tuo grosso e il tuo minuto bestiame moltiplicare, accrescersi il tuo argento e il tuo oro, ed abbondare ogni cosa tua,
\par 14 che il tuo cuore s'innalzi, e tu dimentichi il tuo Dio, l'Eterno, che ti ha tratto dal paese d'Egitto, dalla casa di schiavitù;
\par 15 che t'ha condotto attraverso questo grande e terribile deserto, pieno di serpenti ardenti e di scorpioni, terra arida, senz'acqua; che ha fatto sgorgare per te dell'acqua dalla durissima rupe;
\par 16 che nel deserto t'ha nutrito di manna che i tuoi padri non avean mai conosciuta, per umiliarti e per provarti, per farti, alla fine, del bene.
\par 17 Guardati dunque dal dire in cuor tuo: 'La mia forza e la potenza della mia mano m'hanno acquistato queste ricchezze';
\par 18 ma ricordati dell'Eterno, dell'Iddio tuo; poiché egli ti dà la forza per acquistar ricchezze, affin di confermare, come fa oggi, il patto che giurò ai tuoi padri.
\par 19 Ma se avvenga che tu dimentichi il tuo Dio, l'Eterno, e vada dietro ad altri dèi e li serva e ti prostri davanti a loro, io vi dichiaro quest'oggi solennemente che certo perirete.
\par 20 Perirete come le nazioni che l'Eterno fa perire davanti a voi, perché non avrete dato ascolto alla voce dell'Eterno, dell'Iddio vostro.

\chapter{9}

\par 1 Ascolta, Israele! Oggi tu stai per passare il Giordano per andare a impadronirti di nazioni più grandi e più potenti di te, di città grandi e fortificate fino al cielo,
\par 2 di un popolo grande e alto di statura, de' figliuoli degli Anakim che tu conosci, e dei quali hai sentito dire: 'Chi mai può stare a fronte de' figliuoli di Anak?'
\par 3 Sappi dunque oggi che l'Eterno, il tuo Dio, è quegli che marcerà alla tua testa, come un fuoco divorante; ei li distruggerà e li abbatterà davanti a te; tu li scaccerai e li farai perire in un attimo, come l'Eterno ti ha detto.
\par 4 Quando l'Eterno, il tuo Dio, li avrà cacciati via d'innanzi a te, non dire nel tuo cuore: 'A cagione della mia giustizia l'Eterno mi ha fatto entrare in possesso di questo paese'; poiché l'Eterno caccia d'innanzi a te queste nazioni, per la loro malvagità.
\par 5 No, tu non entri in possesso del loro paese a motivo della tua giustizia, né a motivo della rettitudine del tuo cuore; ma l'Eterno, il tuo Dio, sta per cacciare quelle nazioni d'innanzi a te per la loro malvagità e per mantenere la parola giurata ai tuoi padri, ad Abrahamo, a Isacco e a Giacobbe.
\par 6 Sappi dunque che, non a motivo della tua giustizia l'Eterno, il tuo Dio, ti dà il possesso di questo buon paese; poiché tu sei un popolo di collo duro.
\par 7 Ricordati, non dimenticare come hai provocato ad ira l'Eterno, il tuo Dio, nel deserto. Dal giorno che uscisti dal paese d'Egitto, fino al vostro arrivo in questo luogo, siete stati ribelli all'Eterno.
\par 8 Anche ad Horeb provocaste ad ira l'Eterno; e l'Eterno si adirò contro di voi, al punto di volervi distruggere.
\par 9 Quand'io fui salito sul monte a prendere le tavole di pietra, le tavole del patto che l'Eterno avea fermato con voi, io rimasi sul monte quaranta giorni e quaranta notti, senza mangiar pane né bere acqua;
\par 10 e l'Eterno mi dette le due tavole di pietra, scritte col dito di Dio, sulle quali stavano tutte le parole che l'Eterno vi avea dette sul monte, di mezzo al fuoco, il giorno della raunanza.
\par 11 E fu alla fine dei quaranta giorni e delle quaranta notti che l'Eterno mi dette le due tavole di pietra, le tavole del patto.
\par 12 Poi l'Eterno mi disse: 'Lèvati, scendi prontamente di qui, perché il tuo popolo che hai tratto dall'Egitto si è corrotto; hanno ben presto lasciato la via che io avevo loro ordinato di seguire; si son fatti una immagine di getto'.
\par 13 L'Eterno mi parlò ancora, dicendo: 'Io l'ho visto questo popolo; ecco, esso è un popolo di collo duro;
\par 14 lasciami fare; io li distruggerò e cancellerò il loro nome di sotto i cieli, e farò di te una nazione più potente e più grande di loro'.
\par 15 Così io mi volsi e scesi dal monte, dal monte tutto in fiamme, tenendo nelle mie due mani le due tavole del patto.
\par 16 Guardai, ed ecco che avevate peccato contro l'Eterno, il vostro Dio; v'eravate fatto un vitello di getto; avevate ben presto lasciata la via che l'Eterno vi aveva ordinato di seguire.
\par 17 E afferrai le due tavole, le gettai dalle mie due mani, e le spezzai sotto i vostri occhi.
\par 18 Poi mi prostrai davanti all'Eterno, come avevo fatto la prima volta, per quaranta giorni e per quaranta notti; non mangiai pane né bevvi acqua, a cagione del gran peccato che avevate commesso, facendo ciò ch'è male agli occhi dell'Eterno, per irritarlo.
\par 19 Poiché io avevo paura, a veder l'ira e il furore da cui l'Eterno era invaso contro di voi, al punto di volervi distruggere. Ma l'Eterno m'esaudì anche questa volta.
\par 20 L'Eterno s'adirò anche fortemente contro Aaronne, al punto di volerlo far perire; e io pregai in quell'occasione anche per Aaronne.
\par 21 Poi presi il corpo del vostro delitto, il vitello che avevate fatto, lo detti alle fiamme, lo feci a pezzi, frantumandolo finché fosse ridotto in polvere, e buttai quella polvere nel torrente che scende dal monte.
\par 22 Anche a Tabeera, a Massa e a Kibroth-Hattaava voi irritaste l'Eterno.
\par 23 E quando l'Eterno vi volle far partire da Kades-Barnea, dicendo: 'Salite, e impossessatevi del paese che io vi do', voi vi ribellaste all'ordine dell'Eterno, del vostro Dio, non aveste fede in lui, e non ubbidiste alla sua voce.
\par 24 Siete stati ribelli all'Eterno, dal giorno che vi conobbi.
\par 25 Io stetti dunque così prostrato davanti all'Eterno quei quaranta giorni e quelle quaranta notti, perché l'Eterno avea detto di volervi distruggere.
\par 26 E pregai l'Eterno e dissi: 'O Signore, o Eterno, non distruggere il tuo popolo, la tua eredità, che hai redento nella tua grandezza, che hai tratto dall'Egitto con mano potente.
\par 27 Ricordati de' tuoi servi, Abrahamo, Isacco e Giacobbe; non guardare alla caparbietà di questo popolo, e alla sua malvagità, e al suo peccato,
\par 28 affinché il paese donde ci hai tratti non dica: Siccome l'Eterno non era capace d'introdurli nella terra che aveva loro promessa, e siccome li odiava, li ha fatti uscir di qui per farli morire nel deserto.
\par 29 E nondimeno, essi sono il tuo popolo, la tua eredità, che tu traesti dall'Egitto con la tua gran potenza e col tuo braccio steso'.

\chapter{10}

\par 1 In quel tempo, l'Eterno mi disse: 'Tagliati due tavole di pietra simili alle prime, e sali da me sul monte; fatti anche un'arca di legno;
\par 2 e io scriverò su quelle tavole le parole che erano sulle prime che tu spezzasti, e tu le metterai nell'arca'.
\par 3 Io feci allora un'arca di legno d'acacia, e tagliai due tavole di pietra simili alle prime; poi salii sul monte, tenendo le due tavole in mano.
\par 4 E l'Eterno scrisse su quelle due tavole ciò che era stato scritto la prima volta, cioè le dieci parole che l'Eterno avea pronunziate per voi sul monte di mezzo al fuoco, il giorno della raunanza. E l'Eterno me le diede.
\par 5 Allora mi volsi e scesi dal monte; misi le tavole nell'arca che avevo fatta, e quivi stanno, come l'Eterno mi aveva ordinato.
\par 6 (Or i figliuoli d'Israele partirono da Beeroth-Benè-Jaakan per Mosera. Quivi morì Aaronne, e quivi fu sepolto; ed Eleazar, suo figliuolo, divenne sacerdote al posto di lui.
\par 7 Di là partirono alla volta di Gudgoda; e da Gudgoda alla volta di Jotbatha, paese di corsi d'acqua.
\par 8 In quel tempo l'Eterno separò la tribù di Levi per portare l'arca del patto dell'Eterno, per stare davanti all'Eterno ed esser suoi ministri, e per dar la benedizione nel nome di lui, come ha fatto sino al dì d'oggi.
\par 9 Perciò Levi non ha parte né eredità coi suoi fratelli; l'Eterno è la sua eredità, come gli ha detto l'Eterno, l'Iddio tuo).
\par 10 Or io rimasi sul monte, come la prima volta, quaranta giorni e quaranta notti; e l'Eterno mi esaudì anche questa volta: l'Eterno non ti volle distruggere.
\par 11 E l'Eterno mi disse: 'Lèvati, mettiti in cammino alla testa del tuo popolo, ed entrino essi nel paese che giurai ai loro padri di dar loro, e ne prendano possesso'.
\par 12 Ed ora, Israele, che chiede da te l'Eterno, il tuo Dio, se non che tu tema l'Eterno, il tuo Dio, che tu cammini in tutte le sue vie, che tu l'ami e serva all'Eterno, ch'è il tuo Dio, con tutto il tuo cuore e con tutta l'anima tua,
\par 13 che tu osservi per il tuo bene i comandamenti dell'Eterno e le sue leggi che oggi ti do?
\par 14 Ecco, all'Eterno, al tuo Dio, appartengono i cieli, i cieli dei cieli, la terra e tutto quanto essa contiene;
\par 15 ma soltanto ne' tuoi padri l'Eterno pose affezione, e li amò; e, dopo loro, fra tutti i popoli, scelse la loro progenie, cioè voi, come oggi si vede.
\par 16 Circoncidete dunque il vostro cuore e non indurate più il vostro collo;
\par 17 poiché l'Eterno, il vostro Dio, è l'Iddio degli dèi, il Signor dei signori, l'Iddio grande, forte e tremendo, che non ha riguardi personali e non accetta presenti,
\par 18 che fa giustizia all'orfano e alla vedova che ama lo straniero e gli dà pane e vestito.
\par 19 Amate dunque lo straniero, poiché anche voi foste stranieri nel paese d'Egitto.
\par 20 Temi l'Eterno, il tuo Dio, a lui servi, tienti stretto a lui, e giura nel suo nome.
\par 21 Egli è l'oggetto delle tue lodi, egli è il tuo Dio, che ha fatto per te queste cose grandi e tremende che gli occhi tuoi hanno vedute.
\par 22 I tuoi padri scesero in Egitto in numero di settanta persone; e ora l'Eterno, il tuo Dio, ha fatto di te una moltitudine pari alle stelle de' cieli.

\chapter{11}

\par 1 Ama dunque l'Eterno, il tuo Dio, e osserva sempre quel che ti dice d'osservare, le sue leggi, le sue prescrizioni e i suoi comandamenti.
\par 2 E riconoscete oggi (poiché non parlo ai vostri figliuoli che non hanno conosciuto né hanno veduto le lezioni dell'Eterno, del vostro Dio), riconoscete la sua grandezza, la sua mano potente, il suo braccio steso,
\par 3 i suoi miracoli, le opere che fece in mezzo all'Egitto contro Faraone, re d'Egitto, e contro il suo paese;
\par 4 e quel che fece all'esercito d'Egitto, ai suoi cavalli e ai suoi carri, come fece rifluir su loro le acque del mar Rosso quand'essi v'inseguivano, e come li distrusse per sempre;
\par 5 e quel che ha fatto per voi nel deserto, fino al vostro arrivo in questo luogo;
\par 6 e quel che fece a Dathan e ad Abiram, figliuoli di Eliab, figliuolo di Ruben; come la terra spalancò la sua bocca e li inghiottì con le loro famiglie, le loro tende e tutti quelli ch'erano al loro seguito, in mezzo a tutto Israele.
\par 7 Poiché gli occhi vostri hanno veduto le grandi cose che l'Eterno ha fatte.
\par 8 Osservate dunque tutti i comandamenti che oggi vi do, affinché siate forti e possiate entrare in possesso del paese nel quale state per entrare per impadronirvene,
\par 9 e affinché prolunghiate i vostri giorni sul suolo che l'Eterno giurò di dare ai vostri padri e alla loro progenie: terra ove scorre il latte e il miele.
\par 10 Poiché il paese del quale state per entrare in possesso non è come il paese d'Egitto donde siete usciti, e nel quale gettavi la tua semenza e poi lo annaffiavi coi piedi, come si fa d'un orto;
\par 11 ma il paese di cui andate a prendere possesso è paese di monti e di valli, che beve l'acqua della pioggia che vien dal cielo:
\par 12 paese del quale l'Eterno, il tuo Dio, ha cura, e sul quale stanno del continuo gli occhi dell'Eterno, del tuo Dio, dal principio dell'anno sino alla fine.
\par 13 E se ubbidirete diligentemente ai miei comandamenti che oggi vi do, amando il vostro Dio, l'Eterno, e servendogli con tutto il vostro cuore e con tutta l'anima vostra,
\par 14 avverrà ch'io darò al vostro paese la pioggia a suo tempo: la pioggia d'autunno e di primavera, perché tu possa raccogliere il tuo grano, il tuo vino e il tuo olio;
\par 15 e farò pure crescere dell'erba ne' tuoi campi per il tuo bestiame, e tu mangerai e sarai saziato.
\par 16 Vegliate su voi stessi, onde il vostro cuore non sia sedotto e voi lasciate la retta via e serviate a dèi stranieri e vi prostriate dinanzi a loro,
\par 17 e si accenda contro di voi l'ira dell'Eterno, ed egli chiuda i cieli in guisa che non vi sia più pioggia, e la terra non dia più i suoi prodotti, e voi periate ben presto, scomparendo dal buon paese che l'Eterno vi dà.
\par 18 Vi metterete dunque nel cuore e nell'anima queste mie parole; ve le legherete alla mano come un segnale e vi saranno come frontali tra gli occhi;
\par 19 le insegnerete ai vostri figliuoli, parlandone quando te ne starai seduto in casa tua, quando sarai per viaggio, quando ti coricherai e quando ti alzerai;
\par 20 e le scriverai sugli stipiti della tua casa e sulle tue porte,
\par 21 affinché i vostri giorni e i giorni de' vostri figliuoli, nel paese che l'Eterno giurò ai vostri padri di dar loro, siano numerosi come i giorni de' cieli al disopra della terra.
\par 22 Poiché, se osservate diligentemente tutti questi comandamenti che vi do, e li mettete in pratica, amando l'Eterno, il vostro Dio, camminando in tutte le sue vie e tenendovi stretti a lui,
\par 23 l'Eterno caccerà d'innanzi a voi tutte quelle nazioni, e voi v'impadronirete di nazioni più grandi e più potenti di voi.
\par 24 Ogni luogo che la pianta del vostro piede calcherà, sarà vostro; i vostri confini si estenderanno dal deserto al Libano, dal fiume, il fiume Eufrate, al mare occidentale.
\par 25 Nessuno vi potrà stare a fronte; l'Eterno, il vostro Dio, come vi ha detto, spanderà la paura e il terrore di voi per tutto il paese dove camminerete.
\par 26 Guardate, io pongo oggi dinanzi a voi la benedizione e la maledizione:
\par 27 la benedizione, se ubbidite ai comandamenti dell'Eterno, del vostro Dio, i quali oggi vi do;
\par 28 la maledizione, se non ubbidite ai comandamenti dell'Eterno, dell'Iddio vostro, e se vi allontanate dalla via che oggi vi prescrivo, per andar dietro a dèi stranieri che voi non avete mai conosciuti.
\par 29 E quando l'Eterno, il tuo Dio, t'avrà introdotto nel paese nel quale vai per prenderne possesso, tu pronunzierai la benedizione sul monte Gherizim, e la maledizione sul monte Ebal.
\par 30 Questi monti non sono essi di là dal Giordano, dietro la via di ponente, nel paese dei Cananei che abitano nella pianura dirimpetto a Ghilgal presso la querce di Moreh?
\par 31 Poiché voi state per passare il Giordano per andare a prender possesso del paese, che l'Eterno, l'Iddio vostro, vi dà; voi lo possederete e vi abiterete.
\par 32 Abbiate dunque cura di mettere in pratica tutte le leggi e le prescrizioni, che oggi io pongo dinanzi a voi.

\chapter{12}

\par 1 Queste sono le leggi e le prescrizioni che avrete cura d'osservare nel paese che l'Eterno, l'Iddio dei tuoi padri, ti dà perché tu lo possegga, tutto il tempo che vivrete sulla terra.
\par 2 Distruggerete interamente tutti i luoghi dove le nazioni che state per cacciare servono i loro dèi; sugli alti monti, sui colli, e sotto qualunque albero verdeggiante.
\par 3 Demolirete i loro altari, spezzerete le loro statue, darete alle fiamme i loro idoli d'Astarte, abbatterete le immagini scolpite dei loro dèi, e farete sparire il loro nome da quei luoghi.
\par 4 Non così farete riguardo all'Eterno, all'Iddio vostro;
\par 5 ma lo cercherete nella sua dimora, nel luogo che l'Eterno, il vostro Dio, avrà scelto fra tutte le vostre tribù, per mettervi il suo nome; e quivi andrete;
\par 6 quivi recherete i vostri olocausti e i vostri sacrifizi, le vostre decime, quel che le vostre mani avranno prelevato, le vostre offerte votive e le vostre offerte volontarie, e i primogeniti de' vostri armenti e de' vostri greggi;
\par 7 e quivi mangerete davanti all'Eterno, ch'è il vostro Dio, e vi rallegrerete, voi e le vostre famiglie, godendo di tutto ciò a cui avrete messo mano, e in cui l'Eterno, il vostro Dio, vi avrà benedetti.
\par 8 Non farete come facciamo oggi qui, dove ognuno fa tutto quel che gli par bene,
\par 9 perché finora non siete giunti al riposo e all'eredità che l'Eterno, il vostro Dio, vi dà.
\par 10 Ma passerete il Giordano e abiterete il paese che l'Eterno, il vostro Dio, vi dà in eredità, e avrete requie da tutti i vostri nemici che vi circondano e sarete stanziati in sicurtà;
\par 11 e allora, recherete al luogo che l'Eterno, il vostro Dio, avrà scelto per dimora del suo nome, tutto quello che vi comando: i vostri olocausti e i vostri sacrifizi, le vostre decime, quel che le vostre mani avranno prelevato, e tutte le offerte scelte che avrete votate all'Eterno.
\par 12 E vi rallegrerete dinanzi all'Eterno, al vostro Dio, voi, i vostri figliuoli, le vostre figliuole, i vostri servi, le vostre serve e il Levita che sarà entro le vostre porte; poich'egli non ha né parte né possesso tra voi.
\par 13 Allora ti guarderai bene dall'offrire i tuoi olocausti in qualunque luogo vedrai;
\par 14 ma offrirai i tuoi olocausti nel luogo che l'Eterno avrà scelto in una delle tue tribù; e quivi farai tutto quello che ti comando.
\par 15 Però, potrai a tuo piacimento scannare animali e mangiarne la carne in tutte le tue città, secondo la benedizione che l'Eterno t'avrà largita; tanto colui che sarà impuro come colui che sarà puro ne potranno mangiare, come si fa della carne di gazzella e di cervo;
\par 16 ma non ne mangerete il sangue; lo spargerai per terra come acqua.
\par 17 Non potrai mangiare entro le tue porte le decime del tuo frumento, del tuo mosto, del tuo olio, né i primogeniti de' tuoi armenti e de' tuoi greggi, né ciò che avrai consacrato per voto, né le tue offerte volontarie, né quel che le tue mani avranno prelevato;
\par 18 tali cose mangerai dinanzi all'Eterno, ch'è il tuo Dio, nel luogo che l'Eterno, il tuo Dio, avrà scelto, tu, il tuo figliuolo, la tua figliuola, il tuo servo, la tua serva, e il Levita che sarà entro le tue porte; e ti rallegrerai dinanzi all'Eterno, ch'è il tuo Dio, d'ogni cosa a cui avrai messo mano.
\par 19 Guardati bene, tutto il tempo che vivrai nel tuo paese, dall'abbandonare il Levita.
\par 20 Quando l'Eterno, il tuo Dio, avrà ampliato i tuoi confini, come t'ha promesso, e tu, desiderando di mangiar della carne dirai: 'Vorrei mangiar della carne', potrai mangiar della carne a tuo piacimento.
\par 21 Se il luogo che l'Eterno, il tuo Dio, avrà scelto per porvi il suo nome sarà lontano da te, potrai ammazzare del grosso e del minuto bestiame che l'Eterno t'avrà dato, come t'ho prescritto; e potrai mangiarne entro le tue porte a tuo piacimento.
\par 22 Soltanto, ne mangerai come si mangia la carne di gazzella e di cervo; ne potrà mangiare tanto chi sarà impuro quanto chi sarà puro;
\par 23 ma guardati assolutamente dal mangiarne il sangue, perché il sangue è la vita; e tu non mangerai la vita insieme con la carne.
\par 24 Non lo mangerai; lo spargerai per terra come acqua.
\par 25 Non lo mangerai affinché sii felice tu e i tuoi figliuoli dopo di te, quando avrai fatto ciò ch'è retto agli occhi dell'Eterno.
\par 26 Ma quanto alle cose che avrai consacrate o promesse per voto, le prenderai e andrai al luogo che l'Eterno avrà scelto,
\par 27 e offrirai i tuoi olocausti, la carne e il sangue, sull'altare dell'Eterno, ch'è il tuo Dio; e il sangue delle altre tue vittime dovrà essere sparso sull'altare dell'Eterno, del tuo Dio, e tu ne mangerai la carne.
\par 28 Osserva e ascolta tutte queste cose che ti comando, affinché sii sempre felice tu e i tuoi figliuoli dopo di te, quando avrai fatto ciò ch'è bene e retto agli occhi dell'Eterno, ch'è il tuo Dio.
\par 29 Quando l'Eterno, l'Iddio tuo, avrà sterminate davanti a te le nazioni là dove tu stai per entrare a spodestarle, e quando le avrai spodestate e ti sarai stanziato nel loro paese,
\par 30 guardati bene dal cadere nel laccio, seguendo il loro esempio, dopo che saranno state distrutte davanti a te, e dall'informarti de' loro dèi, dicendo: 'Queste nazioni come servivano esse ai loro dèi? Anch'io vo' fare lo stesso'.
\par 31 Non così farai riguardo all'Eterno, all'Iddio tuo; poiché esse praticavano verso i loro dèi tutto ciò ch'è abominevole per l'Eterno e ch'egli detesta; davan perfino alle fiamme i loro figliuoli e le loro figliuole, in onore dei loro dèi.
\par 32 Avrete cura di mettere in pratica tutte le cose che vi comando; non vi aggiungerai nulla, e nulla ne toglierai.

\chapter{13}

\par 1 Quando sorgerà in mezzo a te un profeta o un sognatore che ti mostri un segno o un prodigio,
\par 2 e il segno o il prodigio di cui t'avrà parlato succeda, ed egli ti dica: 'Andiamo dietro a dèi stranieri (che tu non hai mai conosciuto) e ad essi serviamo',
\par 3 tu non darai retta alle parole di quel profeta o di quel sognatore; perché l'Eterno, il vostro Dio, vi mette alla prova per sapere se amate l'Eterno, il vostro Dio, con tutto il vostro cuore e con tutta l'anima vostra.
\par 4 Seguirete l'Eterno, l'Iddio vostro, temerete lui, osserverete i suoi comandamenti, ubbidirete alla sua voce, a lui servirete e vi terrete stretti.
\par 5 E quel profeta o quel sognatore sarà messo a morte, perché avrà predicato l'apostasia dall'Eterno, dal vostro Dio, che vi ha tratti dal paese d'Egitto e vi ha redenti dalla casa di schiavitù, per spingerti fuori della via per la quale l'Eterno, il tuo Dio, t'ha ordinato di camminare. Così toglierai il male di mezzo a te.
\par 6 Se il tuo fratello, figliuolo di tua madre, o il tuo figliuolo o la tua figliuola o la moglie che riposa sul tuo seno o l'amico che ti è come un altro te stesso t'inciterà in segreto, dicendo: 'Andiamo, serviamo ad altri dèi': dèi che né tu né i tuoi padri avete mai conosciuti,
\par 7 dèi de' popoli che vi circondano, vicini a te o da te lontani, da una estremità all'altra della terra,
\par 8 tu non acconsentire, non gli dar retta; l'occhio tuo non abbia pietà per lui; non lo risparmiare, non lo ricettare;
\par 9 anzi uccidilo senz'altro; la tua mano sia la prima a levarsi su lui, per metterlo a morte; poi venga la mano di tutto il popolo;
\par 10 lapidalo, e muoia, perché ha cercato di spingerti lungi dall'Eterno, dall'Iddio tuo, che ti trasse dal paese d'Egitto, dalla casa di schiavitù.
\par 11 E tutto Israele l'udrà e temerà e non commetterà più nel mezzo di te una simile azione malvagia.
\par 12 Se sentirai dire di una delle tue città che l'Eterno, il tuo Dio, ti dà per abitarle:
\par 13 'Degli uomini perversi sono usciti di mezzo a te e hanno sedotto gli abitanti della loro città dicendo: Andiamo, serviamo ad altri dèi' (che voi non avete mai conosciuti),
\par 14 tu farai delle ricerche, investigherai, interrogherai con cura; e, se troverai che sia vero, che il fatto sussiste e che una tale abominazione è stata realmente commessa in mezzo a te,
\par 15 allora metterai senz'altro a fil di spada gli abitanti di quella città, la voterai allo sterminio, con tutto quel che contiene, e passerai a fil di spada anche il suo bestiame.
\par 16 E radunerai tutto il bottino in mezzo alla piazza, e darai interamente alle fiamme la città con tutto il suo bottino, come sacrifizio arso interamente all'Eterno, ch'è il vostro Dio; essa sarà in perpetuo un mucchio di rovine, e non sarà mai più riedificata.
\par 17 E nulla di ciò che sarà così votato allo sterminio s'attaccherà alle tue mani, affinché l'Eterno si distolga dall'ardore della sua ira, ti faccia misericordia, abbia pietà di te e ti moltiplichi, come giurò di fare ai tuoi padri,
\par 18 quando tu obbedisca alla voce dell'Eterno, del tuo Dio, osservando tutti i suoi comandamenti che oggi ti do, e facendo ciò ch'è retto agli occhi dell'Eterno, ch'è il tuo Dio.

\chapter{14}

\par 1 Voi siete i figliuoli dell'Eterno, ch'è l'Iddio vostro; non vi fate incisioni addosso, e non vi radete i peli tra gli occhi per lutto d'un morto;
\par 2 poiché tu sei il popolo consacrato all'Eterno, all'Iddio tuo, e l'Eterno ti ha scelto perché tu gli fossi un popolo specialmente suo, fra tutti i popoli che sono sulla faccia della terra.
\par 3 Non mangerai cosa alcuna abominevole.
\par 4 Questi sono gli animali dei quali potrete mangiare: il bue, la pecora e la capra;
\par 5 il cervo, la gazzella, il daino, lo stambecco, l'antilope, il capriolo e il camoscio.
\par 6 Potrete mangiare d'ogni animale che ha l'unghia spartita, il piè forcuto, e che rumina.
\par 7 Ma non mangerete di quelli che ruminano soltanto, o che hanno soltanto l'unghia spartita o il piè forcuto; e sono: il cammello, la lepre, il coniglio, che ruminano ma non hanno l'unghia spartita; considerateli come impuri;
\par 8 e anche il porco, che ha l'unghia spartita ma non rumina; lo considererete come impuro. Non mangerete della loro carne, e non toccherete i loro corpi morti.
\par 9 Fra tutti gli animali che vivono nelle acque, potrete mangiare di tutti quelli che hanno pinne e squame;
\par 10 ma non mangerete di alcuno di quelli che non hanno pinne e squame; considerateli come impuri.
\par 11 Potrete mangiare di qualunque uccello puro;
\par 12 ma ecco quelli dei quali non dovete mangiare: l'aquila, l'ossifraga e l'aquila di mare;
\par 13 il nibbio, il falco e ogni specie d'avvoltoio;
\par 14 ogni specie di corvo;
\par 15 lo struzzo, il barbagianni, il gabbiano e ogni specie di sparviere;
\par 16 il gufo, l'ibi, il cigno;
\par 17 il pellicano, il tùffolo, lo smergo;
\par 18 la cicogna, ogni specie di airone, l'upupa e il pipistrello.
\par 19 E considererete come impuro ogni insetto alato; non se ne mangerà.
\par 20 Potrete mangiare d'ogni volatile puro.
\par 21 Non mangerete d'alcuna bestia morta da sé; la darai allo straniero che sarà entro le tue porte perché la mangi, o la venderai a qualche estraneo; poiché tu sei un popolo consacrato all'Eterno, ch'è il tuo Dio. Non farai cuocere il capretto nel latte di sua madre.
\par 22 Avrete cura di prelevare la decima da tutto quello che produrrà la tua semenza, da quello che ti frutterà il campo ogni anno.
\par 23 Mangerai, nel cospetto dell'Eterno, del tuo Dio, nel luogo ch'egli avrà scelto per dimora del suo nome, la decima del tuo frumento, del tuo mosto, del tuo olio, e i primi parti de' tuoi armenti e de' tuoi greggi, affinché tu impari a temer sempre l'Eterno, l'Iddio tuo.
\par 24 Ma se il cammino è troppo lungo per te, sì che tu non possa portar colà quelle decime, essendo il luogo che l'Eterno, il tuo Dio, avrà scelto per stabilirvi il suo nome troppo lontano da te (perché l'Eterno, il tuo Dio, t'avrà benedetto),
\par 25 allora le convertirai in danaro, terrai stretto in mano questo danaro, andrai al luogo che l'Eterno, il tuo Dio, avrà scelto,
\par 26 e impiegherai quel danaro a comprarti tutto quello che il cuor tuo desidererà: buoi, pecore, vino, bevande alcooliche, o qualunque cosa possa più piacerti; e quivi mangerai nel cospetto dell'Eterno, del tuo Dio, e ti rallegrerai: tu con la tua famiglia.
\par 27 E il Levita che abita entro le tue porte, non lo abbandonerai poiché non ha parte né eredità con te.
\par 28 Alla fine d'ogni triennio, metterai da parte tutte le decime delle tue entrate del terzo anno, e le riporrai entro le tue porte;
\par 29 e il Levita, che non ha parte né eredità con te, e lo straniero e l'orfano e la vedova che saranno entro le tue porte verranno, mangeranno e si sazieranno, affinché l'Eterno, il tuo Dio, ti benedica in ogni opera a cui porrai mano.

\chapter{15}

\par 1 Alla fine d'ogni settennio celebrerete l'anno di remissione.
\par 2 Ed ecco il modo di questa remissione: Ogni creditore sospenderà il suo diritto relativamente al prestito fatto al suo prossimo; non esigerà il pagamento dal suo prossimo, dal suo fratello, quando si sarà proclamato l'anno di remissione in onore dell'Eterno.
\par 3 Potrai esigerlo dallo straniero; ma quanto a ciò che il tuo fratello avrà del tuo, sospenderai il tuo diritto.
\par 4 Nondimeno, non vi sarà alcun bisognoso tra voi; poiché l'Eterno senza dubbio ti benedirà nel paese che l'Eterno, il tuo Dio, ti dà in eredità, perché tu lo possegga,
\par 5 purché però tu ubbidisca diligentemente alla voce dell'Eterno, ch'è il tuo Dio, avendo cura di mettere in pratica tutti questi comandamenti, che oggi ti do.
\par 6 Il tuo Dio, l'Eterno, ti benedirà come t'ha promesso, e tu farai dei prestiti a molte nazioni, e non prenderai nulla in prestito; dominerai su molte nazioni, ed esse non domineranno su te.
\par 7 Quando vi sarà in mezzo a te qualcuno de' tuoi fratelli che sia bisognoso in una delle tue città nel paese che l'Eterno, l'Iddio tuo, ti dà, non indurerai il cuor tuo, e non chiuderai la mano davanti al tuo fratello bisognoso;
\par 8 anzi gli aprirai largamente la mano e gli presterai quanto gli abbisognerà per la necessità nella quale si trova.
\par 9 Guardati dall'accogliere in cuor tuo un cattivo pensiero, che ti faccia dire: 'Il settimo anno, l'anno di remissione, è vicino!', e ti spinga ad essere spietato verso il tuo fratello bisognoso, sì da non dargli nulla; poich'egli griderebbe contro di te all'Eterno, e ci sarebbe del peccato in te.
\par 10 Dagli liberalmente; e quando gli darai, non te ne dolga il cuore; perché, a motivo di questo, l'Eterno, l'Iddio tuo, ti benedirà in ogni opera tua e in ogni cosa a cui porrai mano.
\par 11 Poiché i bisognosi non mancheranno mai nel paese; perciò io ti do questo comandamento, e ti dico: 'Apri liberalmente la tua mano al tuo fratello povero e bisognoso nel tuo paese'.
\par 12 Se un tuo fratello ebreo o una sorella ebrea si vende a te, ti servirà sei anni; ma il settimo, lo manderai via da te libero.
\par 13 E quando lo manderai via da te libero, non lo rimanderai a vuoto;
\par 14 lo fornirai liberalmente di doni tratti dal tuo gregge, dalla tua aia e dal tuo strettoio; gli farai parte delle benedizioni che l'Eterno, il tuo Dio, t'avrà largite;
\par 15 e ti ricorderai che sei stato schiavo nel paese d'Egitto, e che l'Eterno, il tuo Dio, ti ha redento; perciò io ti do oggi questo comandamento.
\par 16 Ma se avvenga ch'egli ti dica: 'Non voglio andarmene da te', perché ama te e la tua casa e sta bene da te,
\par 17 allora prenderai una lesina, gli forerai l'orecchio contro la porta, ed egli ti sarà schiavo per sempre. Lo stesso farai per la tua schiava.
\par 18 Non ti sia grave rimandarlo da te libero, poiché t'ha servito sei anni, e un mercenario ti sarebbe costato il doppio; e l'Eterno, il tuo Dio, ti benedirà in tutto ciò che farai.
\par 19 Consacrerai all'Eterno, il tuo Dio, ogni primogenito maschio che ti nascerà ne' tuoi armenti e ne' tuoi greggi. Non metterai al lavoro il primogenito della tua vacca, e non toserai il primogenito della tua pecora.
\par 20 Li mangerai ogni anno con la tua famiglia, in presenza dell'Eterno, dell'Iddio tuo, nel luogo che l'Eterno avrà scelto.
\par 21 E se l'animale ha qualche difetto, se è zoppo o cieco o ha qualche altro grave difetto, non lo sacrificherai all'Eterno, al tuo Dio;
\par 22 lo mangerai entro le tue porte; colui che sarà impuro e colui che sarà puro ne mangeranno senza distinzione, come si mangia della gazzella e del cervo.
\par 23 Però, non ne mangerai il sangue; lo spargerai per terra come acqua.

\chapter{16}

\par 1 Osserva il mese di Abib e celebra la Pasqua in onore dell'Eterno, del tuo Dio; poiché, nel mese di Abib, l'Eterno, il tuo Dio, ti trasse dall'Egitto, durante la notte.
\par 2 E immolerai la Pasqua all'Eterno, all'Iddio tuo, con vittime de' tuoi greggi e de' tuoi armenti, nel luogo che l'Eterno avrà scelto per dimora del suo nome.
\par 3 Non mangerai con queste offerte pane lievitato; per sette giorni mangerai con esse pane azzimo, pane d'afflizione (poiché uscisti in fretta dal paese d'Egitto); affinché tu ti ricordi del giorno che uscisti dal paese d'Egitto, tutto il tempo della tua vita.
\par 4 Non si vegga lievito presso di te, entro tutti i tuoi confini, per sette giorni; e della carne che avrai immolata la sera del primo giorno, nulla se ne serbi durante la notte fino al mattino.
\par 5 Non potrai immolare la Pasqua in una qualunque delle città che l'Eterno, il tuo Dio, ti dà;
\par 6 anzi, immolerai la Pasqua soltanto nel luogo che l'Eterno, il tuo Dio, avrà scelto per dimora del suo nome; la immolerai la sera, al tramontar del sole, nell'ora in cui uscisti dall'Egitto.
\par 7 Farai cuocere la vittima, e la mangerai nel luogo che l'Eterno, il tuo Dio, avrà scelto; e la mattina te ne potrai tornare e andartene alle tue tende.
\par 8 Per sei giorni mangerai pane senza lievito; e il settimo giorno vi sarà una solenne raunanza, in onore dell'Eterno, ch'è l'Iddio tuo; non farai lavoro di sorta.
\par 9 Conterai sette settimane; da quando si metterà la falce nella messe comincerai a contare sette settimane;
\par 10 poi celebrerai la festa delle settimane in onore dell'Eterno, del tuo Dio, mediante offerte volontarie, che presenterai nella misura delle benedizioni che avrai ricevute dall'Eterno, ch'è il tuo Dio.
\par 11 E ti rallegrerai in presenza dell'Eterno, del tuo Dio, tu, il tuo figliuolo e la tua figliuola, il tuo servo e la tua serva, il Levita che sarà entro le tue porte, e lo straniero, l'orfano e la vedova che saranno in mezzo a te, nel luogo che l'Eterno, il tuo Dio, avrà scelto per dimora del suo nome.
\par 12 Ti ricorderai che fosti schiavo in Egitto, e osserverai e metterai in pratica queste leggi.
\par 13 Celebrerai la festa delle Capanne per sette giorni, quando avrai raccolto il prodotto della tua aia e del tuo strettoio;
\par 14 e ti rallegrerai in questa tua festa, tu, il tuo figliuolo e la tua figliuola, il tuo servo e la tua serva, e il Levita, lo straniero, l'orfano e la vedova che saranno entro le tue porte.
\par 15 Celebrerai la festa per sette giorni in onore dell'Eterno, del tuo Dio, nel luogo che l'Eterno avrà scelto; poiché l'Eterno, il tuo Dio, ti benedirà in tutta la tua raccolta e in tutta l'opera delle tue mani, e tu ti darai interamente alla gioia.
\par 16 Tre volte all'anno ogni tuo maschio si presenterà davanti all'Eterno, al tuo Dio, nel luogo che questi avrà scelto: nella festa de' pani azzimi, nella festa delle settimane e nella festa delle Capanne; e nessuno si presenterà davanti all'Eterno a mani vuote.
\par 17 Ognuno darà ciò che potrà, secondo le benedizioni che l'Eterno, l'Iddio tuo, t'avrà date.
\par 18 Stabilisciti de' giudici e dei magistrati in tutte le città che l'Eterno, il tuo Dio, ti dà, tribù per tribù; ed essi giudicheranno il popolo con giusti giudizi.
\par 19 Non pervertirai il diritto, non avrai riguardi personali, e non accetterai donativi, perché il donativo acceca gli occhi de' savi e corrompe le parole de' giusti.
\par 20 La giustizia, solo la giustizia seguirai, affinché tu viva e possegga il paese che l'Eterno, il tuo Dio, ti dà.
\par 21 Non pianterai alcun idolo d'Astarte, di qualsivoglia specie di legno, allato all'altare che edificherai all'Eterno, ch'è il tuo Dio;
\par 22 e non erigerai alcuna statua; cosa, che l'Eterno, il tuo Dio, odia.

\chapter{17}

\par 1 Non immolerai all'Eterno, al tuo Dio, bue o pecora che abbia qualche difetto o qualche deformità, perché sarebbe cosa abominevole per l'Eterno, ch'è il tuo Dio.
\par 2 Se si troverà nel tuo mezzo, in una delle città che l'Eterno, il tuo Dio, ti dà, un uomo o una donna che faccia ciò che è male agli occhi dell'Eterno, del tuo Dio, trasgredendo il suo patto
\par 3 e che vada e serva ad altri dèi e si prostri dinanzi a loro, dinanzi al sole o alla luna o a tutto l'esercito celeste, cosa che io non ho comandata,
\par 4 quando ciò ti sia riferito e tu l'abbia saputo, informatene diligentemente; e se è vero, se il fatto sussiste, se una tale abominazione è stata realmente commessa in Israele,
\par 5 farai condurre alle porte della tua città quell'uomo o quella donna che avrà commesso quell'atto malvagio, e lapiderai quell'uomo o quella donna, sì che muoia.
\par 6 Colui che dovrà morire sarà messo a morte sulla deposizione di due o di tre testimoni; non sarà messo a morte sulla deposizione di un solo testimonio.
\par 7 La mano dei testimoni sarà la prima a levarsi contro di lui per farlo morire; poi, la mano di tutto il popolo; così torrai il male di mezzo a te.
\par 8 Quando il giudizio d'una causa sarà troppo difficile per te, sia che si tratti d'un omicidio o d'una contestazione o d'un ferimento, di materie da processo entro le tue porte, ti leverai e salirai al luogo che l'Eterno, il tuo Dio, avrà scelto;
\par 9 andrai dai sacerdoti levitici e dal giudice in carica a quel tempo; li consulterai, ed essi ti faranno conoscere ciò che dice il diritto;
\par 10 e tu ti conformerai a quello ch'essi ti dichiareranno nel luogo che l'Eterno avrà scelto, e avrai cura di fare tutto quello che t'avranno insegnato.
\par 11 Ti conformerai alla legge ch'essi t'avranno insegnata e al diritto come te l'avranno dichiarato; non devierai da quello che t'avranno insegnato, né a destra né a sinistra.
\par 12 E l'uomo che avrà la presunzione di non dare ascolto al sacerdote che sta là per servire l'Eterno, il tuo Dio, o al giudice, quell'uomo morrà; così torrai via il male da Israele;
\par 13 e tutto il popolo udrà la cosa, temerà, e non agirà più con presunzione.
\par 14 Quando sarai entrato nel paese che l'Eterno, il tuo Dio, ti dà e ne avrai preso possesso e l'abiterai, se dici: 'Voglio costituire su di me un re come tutte le nazioni che mi circondano',
\par 15 dovrai costituire su di te come re colui che l'Eterno, il tuo Dio, avrà scelto. Costituirai su di te come re uno de' tuoi fratelli; non potrai costituire su di te uno straniero che non sia tuo fratello.
\par 16 Però, non abbia egli gran numero di cavalli, e non riconduca il popolo in Egitto per procurarsi gran numero di cavalli, poiché l'Eterno vi ha detto: 'Non rifarete mai più quella via'.
\par 17 E neppure abbia gran numero di mogli; affinché il suo cuore non si svii; e neppure abbia gran quantità d'argento e d'oro.
\par 18 E quando s'insedierà sul suo trono reale, scriverà per suo uso in un libro, una copia di questa legge secondo l'esemplare dei sacerdoti levitici.
\par 19 E terrà il libro presso di sé, e vi leggerà dentro tutti i giorni della sua vita, per imparare a temere l'Eterno, il suo Dio, a mettere diligentemente in pratica tutte le parole di questa legge e tutte queste prescrizioni,
\par 20 affinché il cuor suo non si elevi al disopra de' suoi fratelli, ed egli non devii da questi comandamenti né a destra né a sinistra, e prolunghi così i suoi giorni nel suo regno, egli coi suoi figliuoli, in mezzo ad Israele.

\chapter{18}

\par 1 I sacerdoti levitici, tutta quanta la tribù di Levi, non avranno parte né eredità con Israele; vivranno dei sacrifizi fatti mediante il fuoco all'Eterno, e della eredità di lui.
\par 2 Non avranno, dico, alcuna eredità tra i loro fratelli; l'Eterno è la loro eredità, com'egli ha detto loro.
\par 3 Or questo sarà il diritto de' sacerdoti sul popolo, su quelli che offriranno come sacrifizio sia un bue sia una pecora: essi daranno al sacerdote la spalla, le mascelle e il ventricolo.
\par 4 Gli darai le primizie del tuo frumento, del tuo mosto e del tuo olio, e le primizie della tosatura delle tue pecore;
\par 5 poiché l'Eterno, il tuo Dio, l'ha scelto fra tutte le tue tribù, perché si presentino a fare il servizio nel nome dell'Eterno, egli e i suoi figliuoli, in perpetuo.
\par 6 E quando un Levita, partendo da una qualunque delle città dove soggiorna in Israele, verrà, seguendo il pieno desiderio del suo cuore, al luogo che l'Eterno avrà scelto,
\par 7 e farà il servizio nel nome dell'Eterno, del tuo Dio, come tutti i suoi fratelli Leviti che stanno quivi davanti all'Eterno,
\par 8 egli riceverà, per il suo mantenimento, una parte uguale a quella degli altri, oltre quello che gli può venire dalla vendita del suo patrimonio.
\par 9 Quando sarai entrato nel paese che l'Eterno, l'Iddio tuo, ti dà, non imparerai a imitare le abominazioni delle nazioni che son quivi.
\par 10 Non si trovi in mezzo a te chi faccia passare il suo figliuolo o la sua figliuola per il fuoco, né chi eserciti la divinazione, né pronosticatore, né augure, né mago,
\par 11 né incantatore, né chi consulti gli spiriti, né chi dica la buona fortuna, né negromante;
\par 12 perché chiunque fa queste cose è in abominio all'Eterno; e, a motivo di queste abominazioni, l'Eterno, il tuo Dio, sta per cacciare quelle nazioni d'innanzi a te.
\par 13 Tu sarai integro verso l'Eterno, l'Iddio tuo;
\par 14 poiché quelle nazioni, del cui paese tu vai ad impossessarti, danno ascolto ai pronosticatori e agli indovini; ma, quanto a te, l'Eterno, il tuo Dio, ha disposto altrimenti.
\par 15 L'Eterno, il tuo Dio, ti susciterà un profeta come me, in mezzo a te, d'infra i tuoi fratelli; a quello darete ascolto!
\par 16 Avrai così per l'appunto quello che chiedesti all'Eterno, al tuo Dio, in Horeb, il giorno della raunanza, quando dicesti: 'Ch'io non oda più la voce dell'Eterno, dell'Iddio mio, e non vegga più questo gran fuoco, ond'io non muoia'.
\par 17 E l'Eterno mi disse: 'Quello che han detto, sta bene;
\par 18 io susciterò loro un profeta come te, di mezzo ai loro fratelli, e porrò le mie parole nella sua bocca, ed egli dirà loro tutto quello che io gli comanderò.
\par 19 E avverrà che se qualcuno non darà ascolto alle mie parole ch'egli dirà in mio nome, io gliene domanderò conto.
\par 20 Ma il profeta che avrà la presunzione di dire in mio nome qualcosa ch'io non gli abbia comandato di dire o che parlerà in nome di altri dèi, quel profeta sarà punito di morte'.
\par 21 E se tu dici in cuor tuo: 'Come riconosceremo la parola che l'Eterno non ha detta?'
\par 22 Quando il profeta parlerà in nome dell'Eterno, e la cosa non succede e non si avvera, quella sarà una parola che l'Eterno non ha detta; il profeta l'ha detta per presunzione; tu non lo temere.

\chapter{19}

\par 1 Quando l'Eterno, il tuo Dio, avrà sterminato le nazioni delle quali l'Eterno, il tuo Dio, ti dà il paese, e tu succederai a loro e abiterai nelle loro città e nelle loro case,
\par 2 ti metterai da parte tre città, in mezzo al paese, del quale l'Eterno, il tuo Dio, ti dà il possesso.
\par 3 Preparerai delle strade, e dividerai in tre parti il territorio del paese che l'Eterno, il tuo Dio, ti dà come eredità, affinché qualsivoglia omicida si possa rifugiare in quelle città.
\par 4 Ed ecco in qual caso l'omicida che vi si rifugerà avrà salva la vita: chiunque avrà ucciso il suo prossimo involontariamente, senza che l'abbia odiato prima,
\par 5 - come se uno, ad esempio, va al bosco col suo compagno a tagliar delle legna e, mentre la mano avventa la scure per abbatter l'albero, il ferro gli sfugge dal manico e colpisce il compagno sì ch'egli ne muoia, - quel tale si rifugerà in una di queste città ed avrà salva la vita;
\par 6 altrimenti, il vindice del sangue, mentre l'ira gli arde in cuore, potrebbe inseguire l'omicida e, quando sia lungo il cammino da fare, raggiungerlo e colpirlo a morte, mentre non era degno di morte, in quanto che non aveva prima odiato il compagno.
\par 7 Perciò ti do quest'ordine: 'Mettiti da parte tre città'.
\par 8 E se l'Eterno, il tuo Dio, allarga i tuoi confini, come giurò ai tuoi padri di fare, e ti dà tutto il paese che promise di dare ai tuoi padri,
\par 9 qualora tu abbia cura d'osservare tutti questi comandamenti che oggi ti do, amando l'Eterno, il tuo Dio, e camminando sempre nelle sue vie, aggiungerai tre altre città a quelle prime tre,
\par 10 affinché non si sparga sangue innocente in mezzo al paese che l'Eterno, il tuo Dio, ti dà in eredità, e tu non ti renda colpevole di omicidio.
\par 11 Ma se un uomo odia il suo prossimo, gli tende insidie, l'assale, lo percuote in modo da cagionargli la morte, e poi si rifugia in una di quelle città,
\par 12 gli anziani della sua città lo manderanno a trarre di là, e lo daranno nelle mani del vindice del sangue, affinché sia messo a morte.
\par 13 L'occhio tuo non ne avrà pietà; torrai via da Israele il sangue innocente, e così sarai felice.
\par 14 Non sposterai i termini del tuo prossimo, posti dai tuoi antenati, nell'eredità che avrai nel paese di cui l'Eterno, il tuo Dio, ti dà il possesso.
\par 15 Un solo testimone non sarà sufficiente contro ad alcuno, qualunque sia il delitto o il peccato che questi abbia commesso; il fatto sarà stabilito sulla deposizione di due o di tre testimoni.
\par 16 Quando un testimonio iniquo si leverà contro qualcuno per accusarlo d'un delitto,
\par 17 i due uomini fra i quali ha luogo la contestazione compariranno davanti all'Eterno, davanti ai sacerdoti e ai giudici in carica in que' giorni.
\par 18 I giudici faranno una diligente inchiesta; e se quel testimonio risulta un testimonio falso, che ha deposto il falso contro il suo fratello,
\par 19 farete a lui quello ch'egli avea intenzione di fare al suo fratello. Così torrai via il male di mezzo a te.
\par 20 Gli altri l'udranno e temeranno, e d'allora in poi non si commetterà più in mezzo a te una simile malvagità.
\par 21 L'occhio tuo non avrà pietà: vita per vita, occhio per occhio, dente per dente, mano per mano, piede per piede.

\chapter{20}

\par 1 Quando andrai alla guerra contro i tuoi nemici e vedrai cavalli e carri e gente in maggior numero di te, non li temere, perché l'Eterno, il tuo Dio, che ti fece salire dal paese d'Egitto, è teco.
\par 2 E quando sarete sul punto di dar battaglia, il sacerdote si farà avanti, parlerà al popolo
\par 3 e gli dirà: 'Ascolta, Israele! Voi state oggi per impegnar battaglia coi vostri nemici; il vostro cuore non venga meno; non temete, non vi smarrite e non vi spaventate dinanzi a loro,
\par 4 perché l'Eterno, il vostro Dio, è colui che marcia con voi per combattere per voi contro i vostri nemici, e per salvarvi'.
\par 5 Poi gli ufficiali parleranno al popolo, dicendo: 'C'è qualcuno che abbia edificata una casa nuova e non l'abbia ancora inaugurata? Vada, torni a casa sua, onde non abbia a morire in battaglia, e un altro inauguri la casa.
\par 6 C'è qualcuno che abbia piantato una vigna e non ne abbia ancora goduto il frutto? Vada, torni a casa sua, onde non abbia a morire in battaglia, e un altro ne goda il frutto.
\par 7 C'è qualcuno che si sia fidanzato con una donna e non l'abbia ancora presa? Vada, torni a casa sua, onde non abbia a morire in battaglia, e un altro se la prenda'.
\par 8 E gli ufficiali parleranno ancora al popolo, dicendo: 'C'è qualcuno che abbia paura e senta venirgli meno il cuore? Vada, torni a casa sua, onde il cuore dei suoi fratelli non abbia ad avvilirsi come il suo'.
\par 9 E come gli ufficiali avranno finito di parlare al popolo, costituiranno i capi delle schiere alla testa del popolo.
\par 10 Quando ti avvicinerai a una città per attaccarla, le offrirai prima la pace.
\par 11 E se acconsente alla pace e t'apre le sue porte, tutto il popolo che vi si troverà ti sarà tributario e soggetto.
\par 12 Ma s'essa non vuol far pace teco e ti vuol far guerra, allora l'assedierai;
\par 13 e quando l'Eterno, il tuo Dio, te l'avrà data nelle mani, ne metterai a fil di spada tutti i maschi;
\par 14 ma le donne, i bambini, il bestiame e tutto ciò che sarà nella città, tutto quanto il suo bottino, te li prenderai come tua preda; e mangerai il bottino de' tuoi nemici, che l'Eterno, l'Iddio tuo, t'avrà dato.
\par 15 Così farai per tutte le città che sono molto lontane da te, e che non sono città di queste nazioni.
\par 16 Ma nelle città di questi popoli che l'Eterno, il tuo Dio, ti dà come eredità, non conserverai in vita nulla che respiri;
\par 17 ma voterai a completo sterminio gli Hittei, gli Amorei, i Cananei, i Ferezei, gli Hivvei e i Gebusei, come l'Eterno, il tuo Dio, ti ha comandato di fare;
\par 18 affinché essi non v'insegnino a imitare tutte le abominazioni che fanno per i loro dèi, e voi non pecchiate contro l'Eterno, ch'è il vostro Dio.
\par 19 Quando cingerai d'assedio una città per lungo tempo, attaccandola per prenderla, non ne distruggerai gli alberi a colpi di scure; ne mangerai il frutto, ma non li abbatterai; poiché l'albero della campagna è forse un uomo che tu l'abbia ad includere nell'assedio?
\par 20 Potrai però distruggere e abbattere gli alberi che saprai non esser alberi da frutto, e ne costruirai delle opere d'assedio contro la città che fa guerra teco, finch'essa cada.

\chapter{21}

\par 1 Quando nella terra di cui l'Eterno, il tuo Dio, ti dà il possesso si troverà un uomo ucciso, disteso in un campo, senza che sappiasi chi l'abbia ucciso,
\par 2 i tuoi anziani e i tuoi giudici usciranno e misureranno la distanza fra l'ucciso e le città dei dintorni.
\par 3 Poi gli anziani della città più vicina all'ucciso prenderanno una giovenca, che non abbia ancora lavorato né portato il giogo;
\par 4 e gli anziani di quella città faranno scendere la giovenca presso un torrente perenne in luogo dove non si lavora e non si semina, e quivi troncheranno il collo alla giovenca nel torrente.
\par 5 E i sacerdoti figliuoli di Levi, si avvicineranno; poiché l'Eterno, il tuo Dio, li ha scelti per servirlo e per dare la benedizione nel nome dell'Eterno, e la loro parola ha da decidere ogni controversia e ogni caso di lesione.
\par 6 Allora tutti gli anziani di quella città che sono i più vicini all'ucciso, si laveranno le mani sulla giovenca a cui si sarà troncato il collo nel torrente;
\par 7 e, prendendo la parola, diranno: 'Le nostre mani non hanno sparso questo sangue, e i nostri occhi non l'hanno visto spargere.
\par 8 O Eterno, perdona al tuo popolo Israele che tu hai riscattato, e non far responsabile il tuo popolo Israele del sangue innocente'. E quel sangue sparso sarà loro perdonato.
\par 9 Così tu torrai via di mezzo a te il sangue innocente, perché avrai fatto ciò ch'è giusto agli occhi dell'Eterno.
\par 10 Quando andrai alla guerra contro i tuoi nemici e l'Eterno, il tuo Dio, te li avrà dati nelle mani e tu avrai fatto de' prigionieri,
\par 11 se vedrai tra i prigionieri una donna bella d'aspetto, e le porrai affezione e vorrai prendertela per moglie, la menerai in casa tua;
\par 12 ella si raderà il capo, si taglierà le unghie,
\par 13 si leverà il vestito che portava quando fu presa, dimorerà in casa tua, e piangerà suo padre e sua madre per un mese intero; poi entrerai da lei, e tu sarai suo marito, ed ella tua moglie.
\par 14 E se avvenga che non ti piaccia più, la lascerai andare dove vorrà; ma non la potrai in alcun modo vendere per danaro né trattare da schiava, giacché l'hai umiliata.
\par 15 Quand'un uomo avrà due mogli, l'una amata e l'altra odiata, e tanto l'amata quanto l'odiata gli avrà dato de' figliuoli, se il primogenito è figliuolo dell'odiata,
\par 16 nel giorno ch'ei dividerà tra i suoi figliuoli i beni che possiede, non potrà far primogenito il figliuolo dell'amata, anteponendolo al figliuolo della odiata, che è il primogenito;
\par 17 ma riconoscerà come primogenito il figliuolo dell'odiata, dandogli una parte doppia di tutto quello che possiede; poich'egli è la primizia del suo vigore, e a lui appartiene il diritto di primogenitura.
\par 18 Quando un uomo avrà un figliuolo caparbio e ribelle che non ubbidisce alla voce né di suo padre né di sua madre, e benché l'abbian castigato non dà loro retta,
\par 19 suo padre e sua madre lo prenderanno e lo meneranno dagli anziani della sua città, alla porta del luogo dove abita,
\par 20 e diranno agli anziani della sua città: 'Questo nostro figliuolo è caparbio e ribelle; non vuol ubbidire alla nostra voce, è un ghiotto e un ubriacone';
\par 21 e tutti gli uomini della sua città lo lapideranno, sì che muoia; così toglierai via di mezzo a te il male, e tutto Israele lo saprà e temerà.
\par 22 E quand'uno avrà commesso un delitto degno di morte, e tu l'avrai fatto morire e appiccato a un albero,
\par 23 il suo cadavere non dovrà rimanere tutta la notte sull'albero, ma lo seppellirai senza fallo lo stesso giorno; perché l'appiccato è maledetto da Dio, e tu non contaminerai la terra che l'Eterno, il tuo Dio, ti dà come eredità.

\chapter{22}

\par 1 Se vedi smarriti il bue o la pecora del tuo fratello, tu non farai vista di non averli scorti, ma avrai cura di ricondurli al tuo fratello.
\par 2 E se il tuo fratello non abita vicino a te e non lo conosci, raccoglierai l'animale in casa tua, e rimarrà da te finché il tuo fratello non ne faccia ricerca; e allora glielo renderai.
\par 3 Lo stesso farai del suo asino, lo stesso della sua veste, lo stesso di qualunque altro oggetto che il tuo fratello abbia perduto e che tu trovi; tu non farai vista di non averli scorti.
\par 4 Se vedi l'asino del tuo fratello o il suo bue caduto nella strada, tu non farai vista di non averli scorti, ma dovrai aiutare il tuo fratello a rizzarlo.
\par 5 La donna non si vestirà da uomo, né l'uomo si vestirà da donna; poiché chiunque fa tali cose è in abominio all'Eterno, il tuo Dio.
\par 6 Quando, cammin facendo, t'avverrà di trovare sopra un albero o per terra un nido d'uccello con de' pulcini o delle uova e la madre che cova i pulcini o le uova, non prenderai la madre coi piccini;
\par 7 avrai cura di lasciar andare la madre, prendendo per te i piccini; e questo affinché tu sii felice e prolunghi i tuoi giorni.
\par 8 Quando edificherai una casa nuova, farai un parapetto intorno al tuo tetto, per non metter sangue sulla tua casa, nel caso che qualcuno avesse a cascare di lassù.
\par 9 Non seminerai nella tua vigna semi di specie diverse; perché altrimenti il prodotto di ciò che avrai seminato e la rendita della vigna saranno cosa consacrata.
\par 10 Non lavorerai con un bue ed un asino aggiogati assieme.
\par 11 Non porterai vestito di tessuto misto, fatto di lana e di lino.
\par 12 Metterai delle frange ai quattro canti del mantello con cui ti cuopri.
\par 13 Se un uomo sposa una donna, coabita con lei e poi la prende in odio,
\par 14 l'accusa di cose turpi e la diffama, dicendo: 'Ho preso questa donna, e quando mi sono accostato a lei non l'ho trovata vergine',
\par 15 il padre e la madre della giovane prenderanno i segni della verginità della giovane e li produrranno dinanzi agli anziani della città, alla porta;
\par 16 e il padre della giovane dirà agli anziani: 'Io ho dato la mia figliuola per moglie a quest'uomo; egli l'ha presa in odio,
\par 17 ed ecco che l'accusa di cose infami, dicendo: Non ho trovata vergine la tua figliuola; or ecco qua i segni della verginità della mia figliuola'. E spiegheranno il lenzuolo davanti agli anziani della città.
\par 18 Allora gli anziani di quella città prenderanno il marito e lo castigheranno;
\par 19 e siccome ha diffamato una vergine d'Israele, lo condanneranno a un'ammenda di cento sicli d'argento, che daranno al padre della giovane. Ella rimarrà sua moglie ed egli non potrà mandarla via per tutto il tempo della sua vita.
\par 20 Ma se la cosa è vera, se la giovane non è stata trovata vergine,
\par 21 allora si farà uscire quella giovane all'ingresso della casa di suo padre, e la gente della sua città la lapiderà, sì ch'ella muoia, perché ha commesso un atto infame in Israele, prostituendosi in casa di suo padre. Così torrai via il male di mezzo a te.
\par 22 Quando si troverà un uomo a giacere con una donna maritata, ambedue morranno: l'uomo che s'è giaciuto con la donna, e la donna. Così torrai via il male di mezzo ad Israele.
\par 23 Quando una fanciulla vergine è fidanzata, e un uomo, trovandola in città, si giace con lei,
\par 24 condurrete ambedue alla porta di quella città, e li lapiderete sì che muoiano: la fanciulla, perché essendo in città, non ha gridato; e l'uomo perché ha disonorato la donna del suo prossimo. Così torrai via il male di mezzo a te.
\par 25 Ma se l'uomo trova per i campi la fanciulla fidanzata e facendole violenza, si giace con lei, allora morrà soltanto l'uomo che s'è giaciuto con lei;
\par 26 ma non farai niente alla fanciulla; nella fanciulla non c'è colpa degna di morte; si tratta d'un caso come quello d'un uomo che si levi contro il suo prossimo, e l'uccida;
\par 27 poiché egli l'ha trovata per i campi; la fanciulla fidanzata ha gridato, ma non c'era nessuno per salvarla.
\par 28 Se un uomo trova una fanciulla vergine che non sia fidanzata, e l'afferra, e si giace con lei, e sono sorpresi,
\par 29 l'uomo che s'è giaciuto con lei darà al padre della fanciulla cinquanta sicli d'argento, ed ella sarà sua moglie, perché l'ha disonorata; e non potrà mandarla via per tutto il tempo della sua vita.
\par 30 Nessuno prenderà la moglie di suo padre né solleverà il lembo della coperta di suo padre.

\chapter{23}

\par 1 L'eunuco a cui sono state infrante o mutilate le parti, non entrerà nella raunanza dell'Eterno.
\par 2 Il bastardo non entrerà nella raunanza dell'Eterno; nessuno de' suoi, neppure alla decima generazione, entrerà nella raunanza dell'Eterno.
\par 3 L'Ammonita e il Moabita non entreranno nella raunanza dell'Eterno; nessuno dei loro discendenti, neppure alla decima generazione, entrerà nella raunanza dell'Eterno;
\par 4 non v'entreranno mai, perché non vi vennero incontro col pane e con l'acqua nel vostro viaggio, quand'usciste dall'Egitto, e perché salariarono a tuo danno Balaam, figliuolo di Beor, da Pethor in Mesopotamia, per maledirti.
\par 5 Ma l'Eterno, il tuo Dio, non volle ascoltar Balaam; ma l'Eterno, il tuo Dio, mutò per te la maledizione in benedizione perché l'Eterno, il tuo Dio, ti ama.
\par 6 Non cercherai né la loro pace né la loro prosperità, finché tu viva, in perpetuo.
\par 7 Non aborrirai l'Idumeo, poich'egli è tuo fratello; non aborrirai l'Egiziano, perché fosti straniero nel suo paese;
\par 8 i figliuoli che nasceranno loro potranno, alla terza generazione, entrare nella raunanza dell'Eterno.
\par 9 Quando uscirai e ti accamperai contro i tuoi nemici, guardati da ogni cosa malvagia.
\par 10 Se v'è qualcuno in mezzo a te che sia impuro a motivo d'un accidente notturno, uscirà dal campo, e non vi rientrerà;
\par 11 sulla sera si laverà con acqua, e dopo il tramonto del sole potrà rientrare nel campo.
\par 12 Avrai pure un luogo fuori del campo; e là fuori andrai per i tuoi bisogni;
\par 13 e fra i tuoi utensili avrai una pala, con la quale, quando vorrai andar fuori per i tuoi bisogni, scaverai la terra, e coprirai i tuoi escrementi.
\par 14 Poiché l'Eterno, il tuo Dio, cammina in mezzo al tuo campo per liberarti e per darti nelle mani i tuoi nemici; perciò il tuo campo dovrà esser santo; affinché l'Eterno non abbia a vedere in mezzo a te alcuna bruttura e a ritrarsi da te.
\par 15 Non consegnerai al suo padrone lo schiavo che, dopo averlo lasciato, si sarà rifugiato presso di te.
\par 16 Rimarrà da te, nel tuo paese, nel luogo che avrà scelto, in quella delle tue città che gli parrà meglio; e non lo molesterai.
\par 17 Non vi sarà alcuna meretrice tra le figliuole d'Israele, né vi sarà alcun uomo che si prostituisca tra i figliuoli d'Israele.
\par 18 Non porterai nella casa dell'Eterno, del tuo Dio, la mercede d'una meretrice né il prezzo della vendita d'un cane, per sciogliere qualsivoglia voto; poiché ambedue son cose abominevoli per l'Eterno, ch'è il tuo Dio.
\par 19 Non farai al tuo fratello prestiti a interesse, né di danaro, né di viveri, né di qualsivoglia cosa che si presta a interesse.
\par 20 Allo straniero potrai prestare a interesse, ma non al tuo fratello; affinché l'Eterno, il tuo Dio, ti benedica in tutto ciò a cui porrai mano, nel paese dove stai per entrare per prenderne possesso.
\par 21 Quando avrai fatto un voto all'Eterno, al tuo Dio, non tarderai ad adempirlo; poiché l'Eterno, il tuo Dio, te ne domanderebbe certamente conto, e tu saresti colpevole;
\par 22 ma se ti astieni dal far voti, non commetti peccato.
\par 23 Mantieni e compi la parola uscita dalle tue labbra; fa' secondo il voto che avrai fatto volontariamente all'Eterno, al tuo Dio, e che la tua bocca avrà pronunziato.
\par 24 Quando entrerai nella vigna del tuo prossimo, potrai a tuo piacere mangiare dell'uva a sazietà, ma non ne metterai nel tuo paniere.
\par 25 Quando entrerai nelle biade del tuo prossimo, potrai coglierne delle spighe con la mano; ma non metterai la falce nelle biade del tuo prossimo.

\chapter{24}

\par 1 Quand'uno avrà preso una donna e sarà divenuto suo marito, se avvenga ch'ella poi non gli sia più gradita perché ha trovato in lei qualcosa di vergognoso, e scriva per lei un libello di ripudio e glielo consegni in mano e la mandi via di casa sua,
\par 2 s'ella, uscita di casa di colui, va e divien moglie d'un altro marito,
\par 3 e quest'altro marito la prende in odio, scrive per lei un libello di ripudio, glielo consegna in mano e la manda via di casa sua, o se quest'altro marito che l'avea presa per moglie viene a morire,
\par 4 il primo marito che l'avea mandata via non potrà riprenderla per moglie dopo ch'ella è stata contaminata; poiché sarebbe un'abominazione agli occhi dell'Eterno; e tu non macchierai di peccato il paese che l'Eterno, il tuo Dio, ti dà come eredità.
\par 5 Quando un uomo si sarà sposato di fresco, non andrà alla guerra, e non gli sarà imposto alcun incarico; sarà libero per un anno di starsene a casa e farà lieta la moglie che ha sposata.
\par 6 Nessuno prenderà in pegno sia le due macine, sia la macina superiore, perché sarebbe come prendere in pegno la vita.
\par 7 Quando si troverà un uomo che abbia rubato qualcuno dei suoi fratelli di tra i figliuoli d'Israele, ne abbia fatto un suo schiavo e l'abbia venduto, quel ladro sarà messo a morte; così torrai via il male di mezzo a te.
\par 8 State in guardia contro la piaga della lebbra, per osservare diligentemente e fare tutto quello che i sacerdoti levitici v'insegneranno; avrete cura di fare come io ho loro ordinato.
\par 9 Ricordati di quello che l'Eterno, il tuo Dio, fece a Maria, durante il viaggio, dopo che foste usciti dall'Egitto.
\par 10 Quando presterai qualsivoglia cosa al tuo prossimo, non entrerai in casa sua per prendere il suo pegno;
\par 11 te ne starai di fuori, e l'uomo a cui avrai fatto il prestito, ti porterà il pegno fuori.
\par 12 E se quell'uomo è povero, non ti coricherai, avendo ancora il suo pegno.
\par 13 Non mancherai di restituirgli il pegno, al tramonto del sole, affinché egli possa dormire nel suo mantello, e benedirti; e questo ti sarà contato come un atto di giustizia agli occhi dell'Eterno, ch'è il tuo Dio.
\par 14 Non defrauderai il mercenario povero e bisognoso, sia egli uno dei tuoi fratelli o uno degli stranieri che stanno nel tuo paese, entro le tue porte;
\par 15 gli darai il suo salario il giorno stesso, prima che tramonti il sole; poich'egli è povero, e l'aspetta con impazienza; così egli non griderà contro di te all'Eterno, e tu non commetterai un peccato.
\par 16 Non si metteranno a morte i padri per i figliuoli, né si metteranno a morte i figliuoli per i padri; ognuno sarà messo a morte per il proprio peccato.
\par 17 Non conculcherai il diritto dello straniero o dell'orfano, e non prenderai in pegno la veste della vedova;
\par 18 ma ti ricorderai che sei stato schiavo in Egitto, e che di là, ti ha redento l'Eterno, l'Iddio tuo; perciò io ti comando che tu faccia così.
\par 19 Allorché, facendo la mietitura nel tuo campo, vi avrai dimenticato qualche manipolo, non tornerai indietro a prenderlo; sarà per lo straniero, per l'orfano e per la vedova, affinché l'Eterno, il tuo Dio, ti benedica in tutta l'opera delle tue mani.
\par 20 Quando scoterai i tuoi ulivi, non starai a cercar le ulive rimaste sui rami; saranno per lo straniero, per l'orfano e per la vedova.
\par 21 Quando vendemmierai la tua vigna, non starai a coglierne i raspolli; saranno per lo straniero, per l'orfano e per la vedova.
\par 22 E ti ricorderai che sei stato schiavo nel paese d'Egitto; perciò ti comando che tu faccia così.

\chapter{25}

\par 1 Quando sorgerà una lite fra alcuni, e verranno in giudizio, i giudici che li giudicheranno assolveranno l'innocente e condanneranno il colpevole.
\par 2 E se il colpevole avrà meritato d'esser battuto, il giudice lo farà distendere per terra e battere in sua presenza, con un numero di colpi proporzionato alla gravità della sua colpa.
\par 3 Gli farà dare non più di quaranta colpi, per tema che il tuo fratello resti avvilito agli occhi tuoi, qualora si oltrepassasse di molto questo numero di colpi.
\par 4 Non metterai la musoliera al bue che trebbia il grano.
\par 5 Quando de' fratelli staranno assieme, e l'un d'essi morrà senza lasciar figliuoli, la moglie del defunto non si mariterà fuori, con uno straniero; il suo cognato verrà da lei e se la prenderà per moglie, compiendo così verso di lei il suo dovere di cognato;
\par 6 e il primogenito ch'ella partorirà, succederà al fratello defunto e ne porterà il nome, affinché questo nome non sia estinto in Israele.
\par 7 E se a quell'uomo non piaccia di prender la sua cognata, la cognata salirà alla porta degli anziani e dirà: 'Il mio cognato rifiuta di far rivivere in Israele il nome del suo fratello; ei non vuol compiere verso di me il suo dovere di cognato'.
\par 8 Allora gli anziani della sua città lo chiameranno e gli parleranno; e se egli persiste e dice: 'Non mi piace di prenderla',
\par 9 allora la sua cognata gli si avvicinerà in presenza degli anziani, gli leverà il calzare dal piede, gli sputerà in faccia, e dirà: 'Così sarà fatto all'uomo che non vuol edificare la casa del suo fratello'.
\par 10 E la casa di lui sarà chiamata in Israele 'la casa dello scalzato'.
\par 11 Quando alcuni verranno a contesa fra loro, e la moglie dell'uno s'accosterà per liberare suo marito dalle mani di colui che lo percuote, e stendendo la mano afferrerà quest'ultimo per le sue vergogne, tu le mozzerai la mano;
\par 12 l'occhio tuo non ne abbia pietà.
\par 13 Non avrai nella tua sacchetta due pesi, uno grande e uno piccolo.
\par 14 Non avrai in casa due misure, una grande e una piccola.
\par 15 Terrai pesi esatti e giusti, terrai misure esatte e giuste, affinché i tuoi giorni siano prolungati sulla terra che l'Eterno, l'Iddio tuo, ti dà.
\par 16 Poiché chiunque fa altrimenti, chiunque commette iniquità, è in abominio all'Eterno, al tuo Dio.
\par 17 Ricordati di ciò che ti fece Amalek, durante il viaggio, quando usciste dall'Egitto:
\par 18 com'egli ti attaccò per via, piombando per di dietro su tutti i deboli che ti seguivano, quand'eri già stanco e sfinito, e come non ebbe alcun timore di Dio.
\par 19 Quando dunque l'Eterno, il tuo Dio, t'avrà dato requie, liberandoti da tutti i tuoi nemici all'intorno nel paese che l'Eterno, il tuo Dio, ti dà come eredità perché tu lo possegga, cancellerai la memoria di Amalek di sotto al cielo: non te ne scordare!

\chapter{26}

\par 1 Or quando sarai entrato nel paese che l'Eterno, il tuo Dio, ti dà come eredità, e lo possederai e ti ci sarai stanziato,
\par 2 prenderai delle primizie di tutti i frutti del suolo da te raccolti nel paese che l'Eterno, il tuo Dio, ti dà, le metterai in un paniere, e andrai al luogo che l'Eterno, l'Iddio tuo, avrà scelto per dimora del suo nome.
\par 3 E ti presenterai al sacerdote in carica in que' giorni, e gli dirai: 'Io dichiaro oggi all'Eterno, all'Iddio tuo, che sono entrato nel paese che l'Eterno giurò ai nostri padri di darci'.
\par 4 Il sacerdote prenderà il paniere dalle tue mani, e lo deporrà davanti all'altare dell'Eterno, del tuo Dio,
\par 5 e tu pronunzierai queste parole davanti all'Eterno, ch'è il tuo Dio: 'Mio padre era un Arameo errante; scese in Egitto, vi stette come straniero con poca gente, e vi diventò una nazione grande, potente e numerosa.
\par 6 E gli Egiziani ci maltrattarono, ci umiliarono e c'imposero un duro servaggio.
\par 7 Allora gridammo all'Eterno, all'Iddio de' nostri padri, e l'Eterno udì la nostra voce, vide la nostra umiliazione, il nostro travaglio e la nostra oppressione,
\par 8 e l'Eterno ci trasse dall'Egitto con potente mano e con braccio disteso, con grandi terrori, con miracoli e con prodigi,
\par 9 e ci ha condotti in questo luogo e ci ha dato questo paese, paese ove scorre il latte e il miele.
\par 10 Ed ora, ecco, io reco le primizie de' frutti del suolo che tu, o Eterno, m'hai dato!' E le deporrai davanti all'Eterno, al tuo Dio, e ti prostrerai davanti all'Eterno, al tuo Dio;
\par 11 e ti rallegrerai, tu col Levita e con lo straniero che sarà in mezzo a te, di tutto il bene che l'Eterno, il tuo Dio, avrà dato a te e alla tua casa.
\par 12 Quando avrai finito di prelevare tutte le decime delle tue entrate, il terzo anno, l'anno delle decime, e le avrai date al Levita, allo straniero, all'orfano e alla vedova perché ne mangino entro le tue porte e siano saziati,
\par 13 dirai, dinanzi all'Eterno, al tuo Dio: 'Io ho tolto dalla mia casa ciò che era consacrato, e l'ho dato al Levita, allo straniero, all'orfano e alla vedova, interamente secondo gli ordini che mi hai dato; non ho trasgredito né dimenticato alcuno dei tuoi comandamenti.
\par 14 Non ho mangiato cose consacrate, durante il mio lutto; non ne ho tolto nulla quand'ero impuro, e non ne ho dato nulla in occasione di qualche morto; ho ubbidito alla voce dell'Eterno, dell'Iddio mio, ho fatto interamente come tu m'hai comandato.
\par 15 Volgi a noi lo sguardo dalla dimora della tua santità, dal cielo, e benedici il tuo popolo d'Israele e la terra che ci hai dato, come giurasti ai nostri padri, terra ove scorre il latte e il miele'.
\par 16 Oggi, l'Eterno, il tuo Dio, ti comanda di mettere in pratica queste leggi e queste prescrizioni; osservale dunque, mettile in pratica con tutto il tuo cuore, con tutta l'anima tua.
\par 17 Tu hai fatto dichiarare oggi all'Eterno ch'egli sarà il tuo Dio, purché tu cammini nelle sue vie e osservi le sue leggi, i suoi comandamenti, le sue prescrizioni, e tu ubbidisca alla sua voce.
\par 18 E l'Eterno t'ha fatto oggi dichiarare che gli sarai un popolo specialmente suo, com'egli t'ha detto, e che osserverai tutti i suoi comandamenti,
\par 19 ond'egli ti renda eccelso per gloria, rinomanza e splendore, su tutte le nazioni che ha fatte, e tu sia un popolo consacrato all'Eterno, al tuo Dio, com'egli t'ha detto.

\chapter{27}

\par 1 Or Mosè e gli anziani d'Israele dettero quest'ordine al popolo: 'Osservate tutti i comandamenti che oggi vi do.
\par 2 E quando avrete passato il Giordano per entrare nel paese che l'Eterno, l'Iddio vostro, vi dà, rizzerai delle grandi pietre, e le intonacherai di calcina.
\par 3 Poi vi scriverai sopra tutte le parole di questa legge, quand'avrai passato il Giordano per entrare nel paese che l'Eterno, il tuo Dio, ti dà: paese ove scorre il latte e il miele, come l'Eterno, l'Iddio de' tuoi padri, ti ha detto.
\par 4 Quando dunque avrete passato il Giordano, rizzerete sul monte Ebal queste pietre, come oggi vi comando, e le intonacherete di calcina.
\par 5 Quivi edificherai pure un altare all'Eterno, ch'è il tuo Dio: un altare di pietre, sulle quali non passerai ferro.
\par 6 Edificherai l'altare dell'Eterno, del tuo Dio, di pietre intatte, e su d'esso offrirai degli olocausti all'Eterno, al tuo Dio.
\par 7 E offrirai de' sacrifizi di azioni di grazie, e quivi mangerai e ti rallegrerai dinanzi all'Eterno, al tuo Dio.
\par 8 E scriverai su quelle pietre tutte le parole di questa legge, in modo che siano nitidamente scolpite'.
\par 9 E Mosè e i sacerdoti levitici parlarono a tutto Israele, dicendo: 'Fa' silenzio e ascolta, o Israele! Oggi sei divenuto il popolo dell'Eterno, del tuo Dio.
\par 10 Ubbidirai quindi alla voce dell'Eterno, del tuo Dio, e metterai in pratica i suoi comandamenti e le sue leggi che oggi ti do'.
\par 11 In quello stesso giorno Mosè diede pure quest'ordine al popolo:
\par 12 'Quando avrete passato il Giordano, ecco quelli che staranno sul monte Gherizim per benedire il popolo: Simeone, Levi, Giuda, Issacar, Giuseppe e Beniamino;
\par 13 ed ecco quelli che staranno sul monte Ebal, per pronunziare la maledizione: Ruben, Gad, Ascer, Zabulon, Dan e Neftali.
\par 14 I Leviti parleranno e diranno ad alta voce a tutti gli uomini d'Israele:
\par 15 Maledetto l'uomo che fa un'immagine scolpita o di getto, cosa abominevole per l'Eterno, opera di mano d'artefice, e la pone in luogo occulto! E tutto il popolo risponderà e dirà: Amen.
\par 16 Maledetto chi sprezza suo padre o sua madre! E tutto il popolo dirà: Amen.
\par 17 Maledetto chi sposta i termini del suo prossimo! E tutto il popolo dirà: Amen.
\par 18 Maledetto chi fa smarrire al cieco il suo cammino! E tutto il popolo dirà: Amen.
\par 19 Maledetto chi conculca il diritto dello straniero, dell'orfano e della vedova! E tutto il popolo dirà: Amen.
\par 20 Maledetto chi giace con la moglie di suo padre, perché ha sollevato il lembo della coperta di suo padre! E tutto il popolo dirà: Amen.
\par 21 Maledetto chi giace con qualsivoglia bestia! E tutto il popolo dirà: Amen.
\par 22 Maledetto chi giace con la propria sorella, figliuola di suo padre o figliuola di sua madre! E tutto il popolo dirà: Amen.
\par 23 Maledetto chi giace con la sua suocera! E tutto il popolo dirà: Amen.
\par 24 Maledetto chi uccide il suo prossimo in occulto! E tutto il popolo dirà: Amen.
\par 25 Maledetto chi accetta un donativo per condannare a morte un innocente! E tutto il popolo dirà: Amen.
\par 26 Maledetto chi non si attiene alle parole di questa legge, per metterle in pratica! E tutto il popolo dirà: Amen.

\chapter{28}

\par 1 Ora, se tu ubbidisci diligentemente alla voce dell'Eterno, del tuo Dio, avendo cura di mettere in pratica tutti i suoi comandamenti che oggi ti do, avverrà che l'Eterno, il tuo Dio, ti renderà eccelso sopra tutte le nazioni della terra;
\par 2 e tutte queste benedizioni verranno su te e si compiranno per te, se darai ascolto alla voce dell'Eterno, dell'Iddio tuo:
\par 3 sarai benedetto nelle città e sarai benedetto nella campagna.
\par 4 Benedetto sarà il frutto delle tue viscere, il frutto del tuo suolo e il frutto del tuo bestiame; benedetti i parti delle tue vacche e delle tue pecore.
\par 5 Benedetti saranno il tuo paniere e la tua madia.
\par 6 Sarai benedetto al tuo entrare e benedetto al tuo uscire.
\par 7 L'Eterno farà sì che i tuoi nemici, quando si leveranno contro di te, siano sconfitti dinanzi a te; usciranno contro a te per una via, e per sette vie fuggiranno d'innanzi a te.
\par 8 L'Eterno ordinerà alla benedizione d'esser teco ne' tuoi granai e in tutto ciò a cui metterai mano; e ti benedirà nel paese che l'Eterno, il tuo Dio, ti dà.
\par 9 L'Eterno ti stabilirà perché tu gli sia un popolo santo, come t'ha giurato, se osserverai i comandamenti dell'Eterno, ch'è il tuo Dio, e se camminerai nelle sue vie;
\par 10 e tutti i popoli della terra vedranno che tu porti il nome dell'Eterno, e ti temeranno.
\par 11 L'Eterno, il tuo Dio, ti colmerà di beni, moltiplicando il frutto delle tue viscere, il frutto del tuo bestiame e il frutto del tuo suolo, nel paese che l'Eterno giurò ai tuoi padri di darti.
\par 12 L'Eterno aprirà per te il suo buon tesoro, il cielo, per dare alla tua terra la pioggia a suo tempo, e per benedire tutta l'opera delle tue mani, e tu presterai a molte nazioni e non prenderai nulla in prestito.
\par 13 L'Eterno ti metterà alla testa e non alla coda, e sarai sempre in alto e mai in basso, se ubbidirai ai comandamenti dell'Eterno, del tuo Dio, i quali oggi ti do perché tu li osservi e li metta in pratica,
\par 14 e se non devierai né a destra né a sinistra da alcuna delle cose che oggi vi comando, per andar dietro ad altri dèi e per servirli.
\par 15 Ma se non ubbidisci alla voce dell'Eterno, del tuo Dio, se non hai cura di mettere in pratica tutti i suoi comandamenti e tutte le sue leggi che oggi ti do, avverrà che tutte queste maledizioni verranno su te e si compiranno per te:
\par 16 Sarai maledetto nella città e sarai maledetto nella campagna.
\par 17 Maledetti saranno il tuo paniere e la tua madia.
\par 18 Maledetto sarà il frutto delle tue viscere, il frutto del tuo suolo; maledetti i parti delle tue vacche e delle tue pecore.
\par 19 Sarai maledetto al tuo entrare e maledetto al tuo uscire.
\par 20 L'Eterno manderà contro di te la maledizione, lo spavento e la minaccia in ogni cosa a cui metterai mano e che farai, finché tu sia distrutto e tu perisca rapidamente, a motivo della malvagità delle tue azioni per la quale m'avrai abbandonato.
\par 21 L'Eterno farà sì che la peste s'attaccherà a te, finch'essa t'abbia consumato nel paese nel quale stai per entrare per prenderne possesso.
\par 22 L'Eterno ti colpirà di consunzione, di febbre, d'infiammazione, d'arsura, di aridità, di carbonchio e di ruggine, che ti perseguiteranno finché tu sia perito.
\par 23 Il tuo cielo sarà di rame sopra il tuo capo, e la terra sotto di te sarà di ferro.
\par 24 L'Eterno manderà sul tuo paese, invece di pioggia, sabbia e polvere, che cadranno su te dal cielo, finché tu sia distrutto.
\par 25 L'Eterno farà sì che sarai messo in rotta dinanzi ai tuoi nemici; uscirai contro a loro per una via e per sette vie fuggirai d'innanzi a loro, e nessuno dei regni della terra ti darà requie.
\par 26 I tuoi cadaveri saran pasto di tutti gli uccelli del cielo e delle bestie della terra, che nessuno scaccerà.
\par 27 L'Eterno ti colpirà con l'ulcera d'Egitto, con emorroidi, con la rogna e con la tigna, di cui non potrai guarire.
\par 28 L'Eterno ti colpirà di delirio, di cecità e di smarrimento di cuore;
\par 29 e andrai brancolando in pien mezzodì, come il cieco brancola nel buio; non prospererai nelle tue vie, sarai del continuo oppresso e spogliato, e non vi sarà alcuno che ti soccorra.
\par 30 Ti fidanzerai con una donna, e un altro si giacerà con lei; edificherai una casa, ma non vi abiterai; pianterai una vigna, e non ne godrai il frutto.
\par 31 Il tuo bue sarà ammazzato sotto i tuoi occhi, e tu non ne mangerai; il tuo asino sarà portato via in tua presenza, e non ti sarà reso; le tue pecore saranno date ai tuoi nemici, e non vi sarà chi ti soccorra.
\par 32 I tuoi figliuoli e le tue figliuole saran dati in balìa d'un altro popolo: i tuoi occhi lo vedranno e languiranno del continuo dal rimpianto di loro, e la tua mano sarà senza forza.
\par 33 Un popolo, che tu non avrai conosciuto, mangerà il frutto della tua terra e di tutta la tua fatica, e sarai del continuo oppresso e schiacciato.
\par 34 E sarai fuor di te per le cose che vedrai con gli occhi tuoi.
\par 35 L'Eterno ti colpirà sulle ginocchia e sulle cosce con un'ulcera maligna, della quale non potrai guarire; ti colpirà dalle piante dei piedi alla sommità del capo.
\par 36 L'Eterno farà andare te e il tuo re che avrai costituito sopra di te, verso una nazione che né tu né i padri tuoi avrete conosciuta; e quivi servirai a dèi stranieri, al legno e alla pietra;
\par 37 e diverrai lo stupore, il proverbio e la favola di tutti i popoli fra i quali l'Eterno t'avrà condotto.
\par 38 Porterai molta semenza al campo e raccoglierai poco, perché la locusta la divorerà.
\par 39 Pianterai vigne, le coltiverai, ma non berrai vino né coglierai uva, perché il verme le roderà.
\par 40 Avrai degli ulivi in tutto il tuo territorio ma non t'ungerai d'olio, perché i tuoi ulivi perderanno il loro frutto.
\par 41 Genererai figliuoli e figliuole, ma non saranno tuoi, perché andranno in schiavitù.
\par 42 Tutti i tuoi alberi e il frutto del tuo suolo saran preda alla locusta.
\par 43 Lo straniero che sarà in mezzo a te salirà sempre più in alto al disopra di te, e tu scenderai sempre più in basso.
\par 44 Egli presterà a te, e tu non presterai a lui; egli sarà alla testa, e tu in coda.
\par 45 Tutte queste maledizioni verranno su te, ti perseguiteranno e ti raggiungeranno, finché tu sia distrutto, perché non avrai ubbidito alla voce dell'Eterno, del tuo Dio, osservando i comandamenti e le leggi ch'egli t'ha dato.
\par 46 Esse saranno per te e per la tua progenie come un segno e come un prodigio, in perpetuo.
\par 47 E perché non avrai servito all'Eterno, al tuo Dio, con gioia e di buon cuore in mezzo all'abbondanza d'ogni cosa,
\par 48 servirai ai tuoi nemici che l'Eterno manderà contro di te, in mezzo alla fame, alla sete, alla nudità e alla mancanza d'ogni cosa; ed essi ti metteranno un giogo di ferro sul collo, finché t'abbiano distrutto.
\par 49 L'Eterno farà muover contro di te, da lontano, dalle estremità della terra, una nazione, pari all'aquila che vola: una nazione della quale non intenderai la lingua,
\par 50 una nazione dall'aspetto truce, che non avrà riguardo al vecchio e non avrà mercè del fanciullo;
\par 51 che mangerà il frutto del tuo bestiame e il frutto del tuo suolo, finché tu sia distrutto, e non ti lascerà di resto né frumento, né mosto, né olio, né parti delle tue vacche e delle tue pecore, finché t'abbia fatto perire.
\par 52 E t'assedierà in tutte le tue città, finché in tutto il tuo paese cadano le alte e forti mura nelle quali avrai riposto la tua fiducia. Essa ti assedierà in tutte le tue città, in tutto il paese che l'Eterno, il tuo Dio, t'avrà dato.
\par 53 E durante l'assedio e nella distretta alla quale ti ridurrà il tuo nemico, mangerai il frutto delle tue viscere, le carni de' tuoi figliuoli e delle tue figliuole, che l'Eterno, il tuo Dio, t'avrà dati.
\par 54 L'uomo più delicato e più molle tra voi guarderà di mal occhio il suo fratello, la donna che riposa sul suo seno, i figliuoli che ancora gli rimangono,
\par 55 non volendo dare ad alcun d'essi delle carni de' suoi figliuoli delle quali si ciberà, perché non gli sarà rimasto nulla in mezzo all'assedio e alla distretta alla quale i nemici t'avranno ridotto in tutte le tue città.
\par 56 La donna più delicata e più molle tra voi, che per mollezza e delicatezza non si sarebbe attentata a posare la pianta del piede in terra, guarderà di mal occhio il marito che le riposa sul seno, il suo figliuolo e la sua figliuola,
\par 57 per non dar loro nulla della placenta uscita dal suo seno e de' figliuoli che metterà al mondo, perché, mancando di tutto, se ne ciberà di nascosto, in mezzo all'assedio e alla penuria alla quale i nemici t'avranno ridotto in tutte le tue città.
\par 58 Se non hai cura di mettere in pratica tutte le parole di questa legge, scritte in questo libro, se non temi questo nome glorioso e tremendo dell'Eterno, dell'Iddio tuo,
\par 59 l'Eterno renderà straordinarie le piaghe con le quali colpirà te e la tua progenie: piaghe grandi e persistenti e malattie maligne e persistenti,
\par 60 e farà tornare su te tutte le malattie d'Egitto, dinanzi alle quali tu tremavi, e s'attaccheranno a te.
\par 61 Ed anche le molte malattie e le molte piaghe non menzionate nel libro di questa legge, l'Eterno le farà venir su te, finché tu sia distrutto.
\par 62 E voi rimarrete poca gente, dopo essere stati numerosi come le stelle del cielo, perché non avrai ubbidito alla voce dell'Eterno, ch'è il tuo Dio.
\par 63 E avverrà che come l'Eterno prendeva piacere a farvi del bene e moltiplicarvi, così l'Eterno prenderà piacere a farvi perire e a distruggervi; e sarete strappati dal paese del quale vai a prender possesso.
\par 64 L'Eterno ti disperderà fra tutti i popoli, da un'estremità della terra sino all'altra; e là servirai ad altri dèi, che né tu né i tuoi padri avete mai conosciuti: al legno e alla pietra.
\par 65 E fra quelle nazioni non avrai requie, e non vi sarà luogo di riposo per la pianta de' tuoi piedi; ma l'Eterno ti darà quivi un cuor tremante, degli occhi che si spegneranno e un'anima languente.
\par 66 La tua vita ti starà dinanzi come sospesa; tremerai notte e giorno, e non sarai sicuro della tua esistenza.
\par 67 La mattina dirai: 'Fosse pur sera!' e la sera dirai: 'Fosse pur mattina!' a motivo dello spavento ond'avrai pieno il cuore, e a motivo delle cose che vedrai con gli occhi tuoi.
\par 68 E l'Eterno ti farà tornare in Egitto su delle navi, per la via della quale t'avevo detto: 'Non la rivedrai mai più!' E là sarete offerti in vendita ai vostri nemici come schiavi e come schiave, e mancherà il compratore!

\chapter{29}

\par 1 Queste sono le parole del patto che l'Eterno comandò a Mosè di stabilire coi figliuoli d'Israele nel paese di Moab, oltre il patto che avea stabilito con essi a Horeb.
\par 2 Mosè convocò dunque tutto Israele, e disse loro: 'Voi avete veduto tutto quello che l'Eterno ha fatto sotto gli occhi vostri, nel paese d'Egitto, a Faraone, a tutti i suoi servitori e a tutto il suo paese;
\par 3 gli occhi tuoi han vedute le calamità grandi con le quali furono provati, quei miracoli, quei gran prodigi;
\par 4 ma, fino a questo giorno, l'Eterno non v'ha dato un cuore per comprendere, né occhi per vedere, né orecchi per udire.
\par 5 Io vi ho condotti quarant'anni nel deserto; le vostre vesti non vi si son logorate addosso, né i vostri calzari vi si son logorati ai piedi.
\par 6 Non avete mangiato pane, non avete bevuto vino né bevanda alcoolica, affinché conosceste che io sono l'Eterno, il vostro Dio.
\par 7 E quando siete arrivati a questo luogo, e Sihon re di Heshbon, e Og re di Basan sono usciti contro noi per combattere, noi li abbiamo sconfitti,
\par 8 abbiamo preso il loro paese, e l'abbiam dato come proprietà ai Rubeniti, ai Gaditi e alla mezza tribù di Manasse.
\par 9 Osservate dunque le parole di questo patto e mettetele in pratica, affinché prosperiate in tutto ciò che farete.
\par 10 Oggi voi comparite tutti davanti all'Eterno, al vostro Dio, i vostri capi, le vostre tribù, i vostri anziani, i vostri ufficiali, tutti gli uomini d'Israele,
\par 11 i vostri bambini, le vostre mogli, lo straniero ch'è in mezzo al tuo campo, da colui che ti spacca le legna a colui che ti attinge l'acqua,
\par 12 per entrare nel patto dell'Eterno, ch'è il tuo Dio: patto fermato con giuramento, e che l'Eterno, il tuo Dio, fa oggi con te,
\par 13 per stabilirti oggi come suo popolo, e per esser tuo Dio, come ti disse e come giurò ai tuoi padri, ad Abrahamo, ad Isacco e a Giacobbe.
\par 14 E non con voi soltanto fo io questo patto e questo giuramento,
\par 15 ma con quelli che stanno qui oggi con noi davanti all'Eterno, ch'è l'Iddio nostro, e con quelli che non son qui oggi con noi.
\par 16 Poiché voi sapete come abbiam dimorato nel paese d'Egitto, e come siam passati per mezzo alle nazioni, che avete attraversate;
\par 17 e avete vedute le loro abominazioni e gli idoli di legno, di pietra, d'argento e d'oro, che son fra quelle.
\par 18 Non siavi tra voi uomo o donna o famiglia o tribù che volga oggi il cuore lungi dall'Eterno, ch'è il nostro Dio, per andare a servire agli dèi di quelle nazioni; non siavi tra voi radice alcuna che produca veleno e assenzio;
\par 19 e non avvenga che alcuno, dopo aver udito le parole di questo giuramento, si lusinghi in cuor suo dicendo: 'Avrò pace, anche se camminerò secondo la caparbietà del mio cuore'; in guisa che chi ha bevuto largamente tragga a perdizione chi ha sete.
\par 20 L'Eterno non vorrà perdonargli; ma in tal caso l'ira dell'Eterno e la sua gelosia s'infiammeranno contro quell'uomo, tutte le maledizioni scritte in questo libro si poseranno su lui, e l'Eterno cancellerà il nome di lui di sotto al cielo;
\par 21 l'Eterno lo separerà, per sua sventura, da tutte le tribù d'Israele, secondo tutte le maledizioni del patto scritto in questo libro della legge.
\par 22 La generazione a venire, i vostri figliuoli che sorgeranno dopo di voi, e lo straniero che verrà da paese lontano, anzi tutte le nazioni, quando vedranno le piaghe di questo paese e le malattie onde l'Eterno l'avrà afflitto,
\par 23 e che tutto il suo suolo sarà zolfo, sale, arsura, e non vi sarà più sementa, né prodotto, né erba di sorta che vi cresca, come dopo la ruina di Sodoma, di Gomorra, di Adma e di Tseboim che l'Eterno distrusse nella sua ira e nel suo furore, diranno:
\par 24 'Perché l'Eterno ha egli trattato così questo paese? perché l'ardore di questa grand'ira?'
\par 25 E si risponderà: 'Perché hanno abbandonato il patto dell'Eterno, dell'Iddio dei loro padri: il patto che egli fermò con loro quando li ebbe tratti dal paese d'Egitto;
\par 26 perché sono andati a servire ad altri dèi e si son prostrati dinanzi a loro dèi, ch'essi non aveano conosciuti, e che l'Eterno non avea assegnati loro.
\par 27 Per questo s'è accesa l'ira dell'Eterno contro questo paese per far venire su di esso tutte le maledizioni scritte in questo libro;
\par 28 e l'Eterno li ha divelti dal loro suolo con ira, con furore, con grande indignazione, e li ha gettati in un altro paese, come oggi si vede'.
\par 29 Le cose occulte appartengono all'Eterno, al nostro Dio, ma le cose rivelate sono per noi e per i nostri figliuoli, in perpetuo, perché mettiamo in pratica tutte le parole di questa legge.

\chapter{30}

\par 1 Or quando tutte queste cose che io t'ho poste dinanzi, la benedizione e la maledizione, si saranno effettuate per te, e tu te le ridurrai a memoria fra tutte le nazioni dove l'Eterno, il tuo Dio, t'avrà sospinto,
\par 2 e ti convertirai all'Eterno, al tuo Dio, e ubbidirai alla sua voce, tu e i tuoi figliuoli, con tutto il tuo cuore e con tutta l'anima tua, secondo tutto ciò che oggi io ti comando,
\par 3 l'Eterno, il tuo Dio, farà ritornare i tuoi dalla schiavitù, avrà pietà di te, e ti raccoglierà di nuovo di fra tutti i popoli, fra i quali l'Eterno, il tuo Dio, t'aveva disperso.
\par 4 Quand'anche i tuoi esuli fossero all'estremità de' cieli, l'Eterno, il tuo Dio, ti raccoglierà di là, e di là ti prenderà.
\par 5 L'Eterno, il tuo Dio, ti ricondurrà nel paese che i tuoi padri avevano posseduto, e tu lo possederai; ed Egli ti farà del bene e ti moltiplicherà più dei tuoi padri.
\par 6 L'Eterno, il tuo Dio, circonciderà il tuo cuore e il cuore della tua progenie affinché tu ami l'Eterno, il tuo Dio, con tutto il tuo cuore e con tutta l'anima tua, e così tu viva.
\par 7 E l'Eterno, il tuo Dio, farà cadere tutte queste maledizioni sui tuoi nemici e su tutti quelli che t'avranno odiato e perseguitato.
\par 8 E tu ti convertirai, ubbidirai alla voce dell'Eterno, e metterai in pratica tutti questi comandamenti che oggi ti do.
\par 9 L'Eterno, il tuo Dio, ti colmerà di beni, facendo prosperare tutta l'opera delle tue mani, il frutto delle tue viscere, il frutto del tuo bestiame e il frutto del tuo suolo; poiché l'Eterno si compiacerà di nuovo nel farti del bene, come si compiacque nel farlo ai tuoi padri,
\par 10 perché ubbidirai alla voce dell'Eterno, ch'è il tuo Dio, osservando i suoi comandamenti e i suoi precetti scritti in questo libro della legge, perché ti sarai convertito all'Eterno, al tuo Dio, con tutto il tuo cuore e con tutta l'anima tua.
\par 11 Questo comandamento che oggi ti do, non è troppo alto per te, né troppo lontano da te.
\par 12 Non è nel cielo, perché tu dica: 'Chi salirà per noi nel cielo e ce lo recherà e ce lo farà udire perché lo mettiamo in pratica?'
\par 13 Non è di là dal mare, perché tu dica: 'Chi passerà per noi di là dal mare e ce lo recherà e ce lo farà udire perché lo mettiamo in pratica?'
\par 14 Invece questa parola è molto vicina a te; è nella tua bocca e nel tuo cuore, perché tu la metta in pratica.
\par 15 Vedi, io pongo oggi davanti a te la vita e il bene, la morte e il male;
\par 16 poiché io ti comando oggi d'amare l'Eterno, il tuo Dio, di camminare nelle sue vie, d'osservare i suoi comandamenti; le sue leggi e i suoi precetti affinché tu viva e ti moltiplichi, e l'Eterno, il tuo Dio, ti benedica nel paese dove stai per entrare per prenderne possesso.
\par 17 Ma se il tuo cuore si volge indietro, e se tu non ubbidisci, e ti lasci trascinare a prostrarti davanti ad altri dèi e a servir loro,
\par 18 io vi dichiaro oggi che certamente perirete, che non prolungherete i vostri giorni nel paese, per entrare in possesso del quale voi siete in procinto di passare il Giordano.
\par 19 Io prendo oggi a testimoni contro a voi il cielo e la terra, che io ti ho posto davanti la vita e la morte, la benedizione e la maledizione; scegli dunque la vita, onde tu viva, tu e la tua progenie,
\par 20 amando l'Eterno, il tuo Dio, ubbidendo alla sua voce e tenendoti stretto a lui (poich'egli è la tua vita e colui che prolunga i tuoi giorni), affinché tu possa abitare sul suolo che l'Eterno giurò di dare ai tuoi padri Abrahamo, Isacco e Giacobbe.

\chapter{31}

\par 1 Mosè andò e rivolse ancora queste parole a tutto Israele.
\par 2 Disse loro: 'Io sono oggi in età di centovent'anni; non posso più andare e venire, e l'Eterno m'ha detto: Tu non passerai questo Giordano.
\par 3 L'Eterno, il tuo Dio, sarà quegli che passerà davanti a te, che distruggerà d'innanzi a te quelle nazioni, e tu possederai il loro paese; e Giosuè passerà davanti a te, come l'Eterno ha detto.
\par 4 E l'Eterno tratterà quelle nazioni come trattò Sihon e Og, re degli Amorei, ch'egli distrusse col loro paese.
\par 5 L'Eterno le darà in vostro potere, e voi le tratterete secondo tutti gli ordini che v'ho dato.
\par 6 Siate forti, fatevi animo, non temete e non vi spaventate di loro, perché l'Eterno, il tuo Dio, è quegli che cammina teco; egli non ti lascerà e non ti abbandonerà'.
\par 7 Poi Mosè chiamò Giosuè, e gli disse in presenza di tutto Israele: 'Sii forte e fatti animo, perché tu entrerai con questo popolo nel paese che l'Eterno giurò ai loro padri di dar loro, e tu sarai quello che gliene darai il possesso.
\par 8 E l'Eterno cammina egli stesso davanti a te; egli sarà con te; non ti lascerà e non ti abbandonerà; non temere e non ti perdere d'animo'.
\par 9 E Mosè scrisse questa legge e la diede ai sacerdoti figliuoli di Levi che portano l'arca del patto dell'Eterno, e a tutti gli anziani d'Israele.
\par 10 Mosè diede loro quest'ordine: 'Alla fine d'ogni settennio, al tempo dell'anno di remissione, alla festa delle Capanne,
\par 11 quando tutto Israele verrà a presentarsi davanti all'Eterno, al tuo Dio, nel luogo ch'egli avrà scelto, leggerai questa legge dinanzi a tutto Israele, in guisa ch'egli l'oda.
\par 12 Radunerai il popolo, uomini, donne, bambini, con lo straniero che sarà entro le tue porte, affinché odano, imparino a temere l'Eterno, il vostro Dio, e abbiano cura di mettere in pratica tutte le parole di questa legge.
\par 13 E i loro figliuoli, che non ne avranno ancora avuto conoscenza, l'udranno e impareranno a temer l'Eterno, il vostro Dio, tutto il tempo che vivrete nel paese del quale voi andate a prender possesso, passando il Giordano'.
\par 14 E l'Eterno disse a Mosè: 'Ecco, il giorno della tua morte s'avvicina; chiama Giosuè, e presentatevi nella tenda di convegno perch'io gli dia i miei ordini'. Mosè e Giosuè dunque andarono e si presentarono nella tenda di convegno.
\par 15 L'Eterno apparve, nella tenda, in una colonna di nuvola; e la colonna di nuvola si fermò all'ingresso della tenda.
\par 16 E l'Eterno disse a Mosè: 'Ecco, tu stai per addormentarti coi tuoi padri; e questo popolo si leverà e si prostituirà, andando dietro agli dèi stranieri del paese nel quale va a stare; e mi abbandonerà, e violerà il mio patto che io ho fermato con lui.
\par 17 In quel giorno, l'ira mia s'infiammerà contro a lui; e io li abbandonerò, nasconderò loro la mia faccia, e saranno divorati, e molti mali e molte angosce cadranno loro addosso; talché in quel giorno diranno: Questi mali non ci son eglino caduti addosso perché il nostro Dio non è in mezzo a noi?
\par 18 E io, in quel giorno, nasconderò del tutto la mia faccia a cagione di tutto il male che avranno fatto, rivolgendosi ad altri dèi.
\par 19 Scrivetevi dunque questo cantico, e insegnatelo ai figliuoli d'Israele; mettetelo loro in bocca, affinché questo cantico mi serva di testimonio contro i figliuoli d'Israele.
\par 20 Quando li avrò introdotti nel paese che promisi ai padri loro con giuramento, paese ove scorre il latte e il miele, ed essi avranno mangiato, si saranno saziati e ingrassati, e si saranno rivolti ad altri dèi per servirli, e avranno sprezzato me e violato il mio patto,
\par 21 e quando molti mali e molte angosce saran piombati loro addosso, allora questo cantico leverà la sua voce contro di loro, come un testimonio; poiché esso non sarà dimenticato, e rimarrà sulle labbra dei loro posteri; giacché io conosco quali siano i pensieri ch'essi concepiscono, anche ora, prima ch'io li abbia introdotti nel paese che giurai di dar loro'.
\par 22 Così Mosè scrisse quel giorno questo cantico, e lo insegnò ai figliuoli d'Israele.
\par 23 Poi l'Eterno dette i suoi ordini a Giosuè, figliuolo di Nun, e gli disse: 'Sii forte e fatti animo, poiché tu sei quello che introdurrai i figliuoli d'Israele nel paese che giurai di dar loro; e io sarò teco'.
\par 24 E quando Mosè ebbe finito di scrivere in un libro tutte quante le parole di questa legge,
\par 25 diede quest'ordine ai Leviti che portavano l'arca del patto dell'Eterno:
\par 26 'Prendete questo libro della legge e mettetelo allato all'arca del patto dell'Eterno, ch'è il vostro Dio; e quivi rimanga come testimonio contro di te;
\par 27 perché io conosco il tuo spirito ribelle e la durezza del tuo collo. Ecco, oggi, mentre sono ancora vivente tra voi, siete stati ribelli contro l'Eterno; quanto più lo sarete dopo la mia morte!
\par 28 Radunate presso di me tutti gli anziani delle vostre tribù e i vostri ufficiali; io farò loro udire queste parole, e prenderò a testimoni contro di loro il cielo e la terra.
\par 29 Poiché io so che, dopo la mia morte, voi certamente vi corromperete e lascerete la via che v'ho prescritta; e la sventura v'incoglierà nei giorni a venire, perché avrete fatto ciò ch'è male agli occhi dell'Eterno, provocandolo a sdegno con l'opera delle vostre mani'.
\par 30 Mosè dunque pronunziò dal principio alla fine le parole di questo cantico, in presenza di tutta la raunanza d'Israele.

\chapter{32}

\par 1 "Porgete orecchio, o cieli, ed io parlerò, e ascolti la terra le parole della mia bocca.
\par 2 Si spanda il mio insegnamento come la pioggia, stilli la mia parola come la rugiada, come la pioggerella sopra la verdura, e come un acquazzone sopra l'erba,
\par 3 poiché io proclamerò il nome dell'Eterno. Magnificate il nostro Iddio!
\par 4 Quanto alla Ròcca, l'opera sua è perfetta, poiché tutte le sue vie sono giustizia. È un Dio fedele e senza iniquità; egli è giusto e retto.
\par 5 Ma essi si sono condotti male verso di lui; non sono suoi figliuoli, l'infamia è di loro, razza storta e perversa.
\par 6 È questa la ricompensa che date all'Eterno, o popolo insensato e privo di saviezza? Non è egli il padre tuo che t'ha creato? non è egli colui che t'ha fatto e ti ha stabilito?
\par 7 Ricordati de' giorni antichi, considera gli anni delle età passate, interroga tuo padre, ed egli te lo farà conoscere, i tuoi vecchi, ed essi te lo diranno.
\par 8 Quando l'Altissimo diede alle nazioni la loro eredità, quando separò i figliuoli degli uomini, egli fissò i confini dei popoli, tenendo conto del numero de' figliuoli d'Israele.
\par 9 Poiché la parte dell'Eterno è il suo popolo, Giacobbe è la porzione della sua eredità.
\par 10 Egli lo trovò in una terra deserta, in una solitudine piena d'urli e di desolazione. Egli lo circondò, ne prese cura, lo custodì come la pupilla dell'occhio suo.
\par 11 Pari all'aquila che desta la sua nidiata, si libra a volo sopra i suoi piccini, spiega le sue ali, li prende e li porta sulle penne,
\par 12 l'Eterno solo l'ha condotto, e nessun dio straniero era con lui.
\par 13 Egli l'ha fatto passare a cavallo sulle alture della terra, e Israele ha mangiato il prodotto de' campi; gli ha fatto succhiare il miele ch'esce dalla rupe, l'olio ch'esce dalle rocce più dure,
\par 14 la crema delle vacche e il latte delle pecore. Gli ha dato il grasso degli agnelli, de' montoni di Basan e de' capri, col fior di farina del frumento; e tu hai bevuto il vino generoso, il sangue dell'uva.
\par 15 Ma Ieshurun s'è fatto grasso ed ha ricalcitrato, - ti sei fatto grasso, grosso e pingue! - ha abbandonato l'Iddio che l'ha fatto, e ha sprezzato la Ròcca della sua salvezza.
\par 16 Essi l'han mosso a gelosia con divinità straniere, l'hanno irritato con abominazioni.
\par 17 Han sacrificato a dèmoni che non son Dio, a dèi che non avean conosciuti, dèi nuovi, apparsi di recente, dinanzi ai quali i vostri padri non avean tremato.
\par 18 Hai abbandonato la Ròcca che ti diè la vita, e hai obliato l'Iddio che ti mise al mondo.
\par 19 E l'Eterno l'ha veduto, e ha reietto i suoi figliuoli e le sue figliuole che l'aveano irritato;
\par 20 e ha detto: 'Io nasconderò loro la mia faccia, e starò a vedere quale ne sarà la fine; poiché sono una razza quanto mai perversa, figliuoli in cui non è fedeltà di sorta.
\par 21 Essi m'han mosso a gelosia con ciò che non è Dio, m'hanno irritato coi loro idoli vani; e io li moverò a gelosia con gente che non è un popolo, li irriterò con una nazione stolta.
\par 22 Poiché un fuoco s'è acceso, nella mia ira, e divamperà fino in fondo al soggiorno de' morti; divorerà la terra e i suoi prodotti, e infiammerà le fondamenta delle montagne.
\par 23 Io accumulerò su loro dei mali, esaurirò contro a loro tutti i miei strali.
\par 24 Essi saran consunti dalla fame, divorati dalla febbre, da mortifera pestilenza; lancerò contro a loro le zanne delle fiere, col veleno delle bestie che striscian nella polvere.
\par 25 Di fuori la spada, e di dentro il terrore spargeranno il lutto, mietendo giovani e fanciulle, lattanti e uomini canuti.
\par 26 Io direi: Li spazzerò via d'un soffio, farò sparire la loro memoria di fra gli uomini,
\par 27 se non temessi gl'insulti del nemico, e che i loro avversari, prendendo abbaglio, fosser tratti a dire: 'È stata la nostra potente mano e non l'Eterno, che ha fatto tutto questo!'
\par 28 Poiché è una nazione che ha perduto il senno, e non v'è in essi alcuna intelligenza.
\par 29 Se fosser savi, lo capirebbero, considererebbero la fine che li aspetta.
\par 30 Come potrebbe un solo inseguirne mille, e due metterne in fuga diecimila, se la Ròcca loro non li avesse venduti, se l'Eterno non li avesse dati in man del nemico?
\par 31 Poiché la ròcca loro non è come la nostra Ròcca; i nostri stessi nemici ne son giudici;
\par 32 ma la loro vigna vien dalla vigna di Sodoma e dalle campagne di Gomorra; le loro uve son uve avvelenate, i loro grappoli, amari;
\par 33 il loro vino è un tossico di serpenti, un crudel veleno d'aspidi.
\par 34 'Tutto questo non è egli tenuto in serbo presso di me, sigillato ne' miei tesori?
\par 35 A me la vendetta e la retribuzione, quando il loro piede vacillerà!' Poiché il giorno della loro calamità è vicino, e ciò che per loro è preparato, s'affretta a venire.
\par 36 Sì, l'Eterno giudicherà il suo popolo, ma avrà pietà de' suoi servi quando vedrà che la forza è sparita, e che non riman più tra loro né schiavo né libero.
\par 37 Allora egli dirà: 'Ove sono i loro dèi, la ròcca nella quale confidavano,
\par 38 gli dèi che mangiavano il grasso de' loro sacrifizi e beveano il vino delle loro libazioni? Si levino essi a soccorrervi, a coprirvi della loro protezione!
\par 39 Ora vedete che io solo son Dio, e che non v'è altro dio accanto a me. Io fo morire e fo vivere, ferisco e risano, e non v'è chi possa liberare dalla mia mano.
\par 40 Sì, io alzo la mia mano al cielo, e dico: Com'è vero ch'io vivo in perpetuo,
\par 41 quando aguzzerò la mia folgorante spada e metterò mano a giudicare, farò vendetta de' miei nemici e darò ciò che si meritano a quelli che m'odiano.
\par 42 Inebrierò di sangue le mie frecce, del sangue degli uccisi e dei prigionieri, la mia spada divorerà la carne, le teste dei condottieri nemici'.
\par 43 Nazioni, cantate le lodi del suo popolo! poiché l'Eterno vendica il sangue de' suoi servi, fa ricadere la sua vendetta sopra i suoi avversari, ma si mostra propizio alla sua terra, al suo popolo".
\par 44 E Mosè venne con Giosuè, figliuolo di Nun, e pronunziò in presenza del popolo tutte le parole di questo cantico.
\par 45 E quando Mosè ebbe finito di pronunziare tutte queste parole dinanzi a tutto Israele, disse loro:
\par 46 'Prendete a cuore tutte le parole con le quali testimonio oggi contro a voi. Le prescriverete ai vostri figliuoli, onde abbian cura di mettere in pratica tutte le parole di questa legge.
\par 47 Poiché questa non è una parola senza valore per voi: anzi, è la vostra vita; e per questa parola prolungherete i vostri giorni nel paese del quale andate a prender possesso, passando il Giordano'.
\par 48 E, in quello stesso giorno, l'Eterno parlò a Mosè, dicendo:
\par 49 'Sali su questo monte di Abarim, sul monte Nebo, ch'è nel paese di Moab, di faccia a Gerico, e mira il paese di Canaan, ch'io do a possedere ai figliuoli d'Israele.
\par 50 Tu morrai sul monte sul quale stai per salire, e sarai riunito al tuo popolo, come Aaronne tuo fratello è morto sul monte di Hor ed è stato riunito al suo popolo,
\par 51 perché commetteste un'infedeltà contro di me in mezzo ai figliuoli d'Israele, alle acque di Meriba a Kades, nel deserto di Tsin, e perché non mi santificaste in mezzo ai figliuoli d'Israele.
\par 52 Tu vedrai il paese davanti a te, ma là, nel paese che io do ai figliuoli d'Israele, non entrerai'.

\chapter{33}

\par 1 Or questa è la benedizione con la quale Mosè, uomo di Dio, benedisse i figliuoli d'Israele, prima di morire.
\par 2 Disse dunque: "L'Eterno è venuto dal Sinai, e s'è levato su loro da Seir; ha fatto splendere la sua luce dal monte di Paran, è giunto dal mezzo delle sante miriadi; dalla sua destra usciva per essi il fuoco della legge.
\par 3 Certo, l'Eterno ama i popoli; ma i suoi santi son tutti agli ordini suoi. Ed essi si tennero ai tuoi piedi, e raccolsero le tue parole.
\par 4 Mosè ci ha dato una legge, eredità della raunanza di Giacobbe;
\par 5 ed egli è stato re in Ieshurun, quando s'adunavano i capi del popolo e tutte assieme le tribù d'Israele.
\par 6 Viva Ruben! ch'egli non muoia; ma siano gli uomini suoi ridotti a pochi!"
\par 7 E questo è per Giuda. Egli disse: "Ascolta, o Eterno, la voce di Giuda, e riconducilo al suo popolo. Con tutte le sue forze egli lotta per esso; tu gli sarai d'aiuto contro i suoi nemici!"
\par 8 Poi disse di Levi: "I tuoi Thummim e i tuoi Urim appartengono all'uomo pio che ti sei scelto, che tu provasti a Massa, e col quale contendesti alle acque di Meriba.
\par 9 Egli dice di suo padre, e di sua madre: 'Io non li ho visti!' non riconosce i suoi fratelli, e nulla sa de' propri figliuoli; perché i Leviti osservano la tua parola e sono i custodi del tuo patto.
\par 10 Essi insegnano i tuoi statuti a Giacobbe e la tua legge a Israele; metton l'incenso sotto le tue nari, e l'olocausto sopra il tuo altare.
\par 11 O Eterno, benedici la sua forza, e gradisci l'opera delle sue mani. Trafiggi le reni a quelli che insorgono contro di lui, che gli sono nemici, sì che non possan risorgere".
\par 12 Di Beniamino disse: "L'amato dell'Eterno abiterà sicuro presso di lui. L'Eterno gli farà riparo del continuo, e abiterà fra le colline di lui".
\par 13 Poi disse di Giuseppe: "Il suo paese sarà benedetto dall'Eterno coi doni più preziosi del cielo, con la rugiada, con le acque dell'abisso che giace in basso,
\par 14 coi frutti più preziosi che il sole matura, con le cose più squisite che ogni luna arreca,
\par 15 coi migliori prodotti de' monti antichi, coi doni più preziosi de' colli eterni, coi doni più preziosi della terra e di quanto essa racchiude.
\par 16 Il favor di colui che stava nel pruno venga sul capo di Giuseppe, sulla fronte di colui ch'è principe tra i suoi fratelli!
\par 17 Del suo toro primogenito egli ha la maestà; le sue corna son corna di bufalo. Con esse darà di cozzo ne' popoli tutti quanti assieme, fino alle estremità della terra. Tali sono le miriadi d'Efraim, tali sono le migliaia di Manasse".
\par 18 Poi disse di Zabulon: "Rallegrati, o Zabulon, nel tuo uscire, e tu, Issacar, nelle tue tende!
\par 19 Essi chiameranno i popoli al monte, e quivi offriranno sacrifizi di giustizia; poich'essi succhieranno la dovizia del mare e i tesori nascosti nella rena".
\par 20 Poi disse di Gad: "Benedetto colui che mette Gad al largo! Egli sta nella sua dimora come una leonessa, e sbrana braccio e cranio.
\par 21 Ei s'è scelto le primizie del paese, poiché quivi è la parte riserbata al condottiero, ed egli v'è giunto alla testa del popolo, ha compiuto la giustizia dell'Eterno e i suoi decreti, insieme ad Israele".
\par 22 Poi disse di Dan: "Dan è un leoncello, che balza da Basan".
\par 23 Poi disse di Neftali: "O Neftali, sazio di favori e ricolmo di benedizioni dell'Eterno, prendi possesso dell'occidente e del mezzodì!"
\par 24 Poi disse di Ascer: "Benedetto sia Ascer tra i figliuoli d'Israele! Sia il favorito de' suoi fratelli, e tuffi il suo piè nell'olio!
\par 25 Sian le sue sbarre di ferro e di rame, e duri quanto i tuoi giorni la tua quiete!
\par 26 O Ieshurun, nessuno è pari a Dio che, sul carro dei cieli, corre in tuo aiuto, che, nella sua maestà, s'avanza sulle nubi:
\par 27 l'Iddio che ab antico è il tuo rifugio; e sotto a te stanno le braccia eterne. Egli scaccia d'innanzi a te il nemico, e ti dice: 'Distruggi!'
\par 28 Israele starà sicuro nella sua dimora; la sorgente di Giacobbe sgorgherà solitaria in un paese di frumento e di mosto, e dove il cielo stilla la rugiada.
\par 29 Te felice, o Israele! Chi è pari a te, un popolo salvato dall'Eterno, ch'è lo scudo che ti protegge, e la spada che ti fa trionfare? I tuoi nemici verranno a blandirti, e tu calpesterai le loro alture".

\chapter{34}

\par 1 Poi Mosè salì dalle pianure di Moab sul Monte Nebo, in vetta al Pisga, che è di faccia a Gerico. E l'Eterno gli fece vedere tutto il paese: Galaad fino a Dan,
\par 2 tutto Neftali, il paese di Efraim e di Manasse, tutto il paese di Giuda fino al mare occidentale,
\par 3 il mezzogiorno, il bacino del Giordano e la valle di Gerico, città delle palme, fino a Tsoar.
\par 4 L'Eterno gli disse: 'Questo è il paese riguardo al quale io feci ad Abrahamo, a Isacco ed a Giacobbe, questo giuramento: - Io lo darò alla tua progenie. - Io te l'ho fatto vedere con i tuoi occhi, ma tu non v'entrerai'.
\par 5 Mosè, servo dell'Eterno, morì quivi, nel paese di Moab, come l'Eterno avea comandato.
\par 6 E l'Eterno lo seppellì nella valle, nel paese di Moab, dirimpetto a Beth-Peor; e nessuno fino a questo giorno ha mai saputo dove fosse la sua tomba.
\par 7 Or Mosè avea centovent'anni quando morì; la vista non gli s'era indebolita e il vigore non gli era venuto meno.
\par 8 E i figliuoli d'Israele lo piansero nelle pianure di Moab per trenta giorni, e si compieron così i giorni del pianto, del lutto per Mosè.
\par 9 E Giosuè, figliuolo di Nun, fu riempito dello spirito di sapienza, perché Mosè gli aveva imposto le mani; e i figliuoli d'Israele gli ubbidirono e fecero quello che l'Eterno avea comandato a Mosè.
\par 10 Non è mai più sorto in Israele un profeta simile a Mosè, col quale l'Eterno abbia trattato faccia a faccia.
\par 11 Niuno è stato simile a lui in tutti quei segni e miracoli che Dio lo mandò a fare nel paese d'Egitto contro Faraone, contro tutti i suoi servi e contro tutto il suo paese;
\par 12 né simile a lui in quegli atti potenti e in tutte quelle gran cose tremende, che Mosè fece dinanzi agli occhi di tutto Israele.


\end{document}