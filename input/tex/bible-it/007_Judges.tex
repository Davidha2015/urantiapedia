\begin{document}

\title{Giudici}


\chapter{1}

\par 1 Dopo la morte di Giosuè, i figliuoli d'Israele consultarono l'Eterno, dicendo: 'Chi di noi salirà il primo contro i Cananei a muover loro guerra?'
\par 2 E l'Eterno rispose: 'Salirà Giuda; ecco, io ho dato il paese nelle sue mani'.
\par 3 Allora Giuda disse a Simeone suo fratello: 'Sali meco nel paese che m'è toccato a sorte, e combatteremo contro i Cananei; poi anch'io andrò teco in quello ch'è toccato a te'. E Simeone andò con lui.
\par 4 Giuda dunque salì, e l'Eterno diede nelle loro mani i Cananei e i Ferezei; e sconfissero a Bezek diecimila uomini.
\par 5 E, trovato Adoni-Bezek a Bezek, l'attaccarono, e sconfissero i Cananei e i Ferezei.
\par 6 Adoni-Bezek si diè alla fuga; ma essi lo inseguirono, lo presero, e gli tagliarono i pollici delle mani e dei piedi.
\par 7 E Adoni-Bezek disse: 'Settanta re, a cui erano stati tagliati i pollici delle mani e de' piedi raccoglievano gli avanzi del cibo sotto la mia mensa. Quello che ho fatto io, Iddio me lo rende'. E lo menarono a Gerusalemme, dove morì.
\par 8 I figliuoli di Giuda attaccarono Gerusalemme, e la presero; passarono gli abitanti a fil di spada e misero la città a fuoco e fiamma.
\par 9 Poi i figliuoli di Giuda scesero a combattere contro i Cananei che abitavano la contrada montuosa, il mezzogiorno e la regione bassa.
\par 10 Giuda marciò contro i Cananei che abitavano a Hebron, (il cui nome era prima Kiriath-Arba) e sconfisse Sceshai, Ahiman e Talmai.
\par 11 Di là marciò contro gli abitanti di Debir, che prima si chiamava Kiriath-Sefer.
\par 12 E Caleb disse: 'A chi batterà Kiriath-Sefer e la prenderà io darò in moglie Acsa, mia figliuola'.
\par 13 La prese Othniel, figliuolo di Kenaz, fratello minore di Caleb, e questi gli diede in moglie Acsa sua figliuola.
\par 14 E quand'ella venne a star con lui, lo persuase a chiedere un campo al padre di lei. Essa scese di sull'asino, e Caleb le disse: 'Che vuoi?'
\par 15 E quella rispose: 'Fammi un dono; giacché tu m'hai data una terra arida dammi anche delle sorgenti d'acqua'. Ed egli le donò le sorgenti superiori e le sorgenti sottostanti.
\par 16 Or i figliuoli del Keneo, suocero di Mosè, salirono dalla città delle palme, coi figliuoli di Giuda, nel deserto di Giuda, che è a mezzogiorno di Arad; andarono, e si stabilirono fra il popolo.
\par 17 Poi Giuda partì con Simeone suo fratello, e sconfissero i Cananei che abitavano in Tsefath; distrussero interamente la città, che fu chiamata Hormah.
\par 18 Giuda prese anche Gaza col suo territorio, Askalon col suo territorio ed Ekron col suo territorio.
\par 19 L'Eterno fu con Giuda, che cacciò gli abitanti della contrada montuosa, ma non poté cacciare gli abitanti della valle, perché aveano de' carri di ferro.
\par 20 E, come Mosè avea detto, Hebron fu data a Caleb, che ne scacciò i tre figliuoli di Anak.
\par 21 I figliuoli di Beniamino non cacciarono i Gebusei che abitavano Gerusalemme; e i Gebusei hanno abitato coi figliuoli di Beniamino in Gerusalemme fino al dì d'oggi.
\par 22 La casa di Giuseppe salì anch'essa contro Bethel, e l'Eterno fu con loro.
\par 23 La casa di Giuseppe mandò ad esplorare Bethel, città che prima si chiamava Luz.
\par 24 E gli esploratori videro un uomo che usciva dalla città, e gli dissero: 'Deh, insegnaci la via per entrare nella città, e noi ti tratteremo benignamente'.
\par 25 Egli insegnò loro la via per entrare nella città, ed essi passarono la città a fil di spada, ma lasciarono andare quell'uomo con tutta la sua famiglia.
\par 26 E quell'uomo andò nel paese degli Hittei e vi edificò una città, che chiamò Luz: nome, ch'essa porta anche al dì d'oggi.
\par 27 Manasse pure non cacciò gli abitanti di Beth-Scean e delle città del suo territorio, né quelli di Taanac e delle città del suo territorio, né quelli di Dor e delle città del suo territorio, né quelli d'Ibleam e delle città del suo territorio, né quelli di Meghiddo e delle città del suo territorio, essendo i Cananei decisi a restare in quel paese.
\par 28 Però, quando Israele si fu rinforzato, assoggettò i Cananei a servitù, ma non li cacciò del tutto.
\par 29 Efraim anch'esso non cacciò i Cananei che abitavano a Ghezer; e i Cananei abitarono in Ghezer in mezzo ad Efraim.
\par 30 Zabulon non cacciò gli abitanti di Kitron, né gli abitanti di Nahalol; e i Cananei abitarono in mezzo a Zabulon e furon soggetti a servitù.
\par 31 Ascer non cacciò gli abitanti di Acco, né gli abitanti di Sidone, né quelli di Ahlab, di Aczib, di Helba, di Afik, di Rehob;
\par 32 e i figliuoli di Ascer si stabilirono in mezzo ai Cananei che abitavano il paese, perché non li scacciarono.
\par 33 Neftali non cacciò gli abitanti di Beth-Scemesh, né gli abitanti di Beth-Anath, e si stabilì in mezzo ai Cananei che abitavano il paese; ma gli abitanti di Beth-Scemesh e di Beth-Anath furon da loro sottoposti a servitù.
\par 34 Gli Amorei respinsero i figliuoli di Dan nella contrada montuosa e non li lasciarono scendere nella valle.
\par 35 Gli Amorei si mostrarono decisi a restare a Har-Heres, ad Aialon ed a Shaalbim; ma la mano della casa di Giuseppe si aggravò su loro sì che furon soggetti a servitù.
\par 36 E il confine degli Amorei si estendeva dalla salita di Akrabbim, movendo da Sela, e su verso il nord.

\chapter{2}

\par 1 Or l'angelo dell'Eterno salì da Ghilgal a Bokim e disse: 'Io vi ho fatto salire dall'Egitto e vi ho condotto nel paese che avevo giurato ai vostri padri di darvi. Avevo anche detto: Io non romperò mai il mio patto con voi;
\par 2 e voi, dal canto vostro, non farete alleanza con gli abitanti di questo paese; demolirete i loro altari. Ma voi non avete ubbidito alla mia voce. Perché avete fatto questo?
\par 3 Perciò anch'io ho detto: Io non li caccerò d'innanzi a voi; ma essi saranno per voi tanti nemici, e i loro dèi vi saranno un'insidia'.
\par 4 Appena l'angelo dell'Eterno ebbe detto queste parole a tutti i figliuoli d'Israele, il popolo si mise a piangere ad alta voce.
\par 5 E posero nome a quel luogo Bokim e vi offrirono dei sacrifizi all'Eterno.
\par 6 Or Giosuè rimandò il popolo, e i figliuoli d'Israele se ne andarono, ciascuno nel suo territorio, a prender possesso del paese.
\par 7 E il popolo servì all'Eterno durante tutta la vita di Giosuè e durante tutta la vita degli anziani che sopravvissero a Giosuè, e che avean veduto tutte le grandi opere che l'Eterno avea fatte a pro d'Israele.
\par 8 Poi Giosuè, figliuolo di Nun, servo dell'Eterno, morì in età di centodieci anni;
\par 9 e fu sepolto nel territorio che gli era toccato a Timnath-Heres nella contrada montuosa di Efraim, al nord della montagna di Gaash.
\par 10 Anche tutta quella generazione fu riunita ai suoi padri; poi, dopo di quella, sorse un'altra generazione, che non conosceva l'Eterno, né le opere ch'egli avea compiute a pro d'Israele.
\par 11 I figliuoli d'Israele fecero ciò ch'è male agli occhi dell'Eterno, e servirono agl'idoli di Baal;
\par 12 abbandonarono l'Eterno, l'Iddio dei loro padri che li avea tratti dal paese d'Egitto, e andaron dietro ad altri dèi fra gli dèi dei popoli che li attorniavano; si prostrarono dinanzi a loro, e provocarono ad ira l'Eterno;
\par 13 abbandonarono l'Eterno, e servirono a Baal e agl'idoli d'Astarte.
\par 14 E l'ira dell'Eterno s'accese contro Israele ed ei li dette in mano di predoni, che li spogliarono; li vendé ai nemici che stavan loro intorno, in guisa che non poteron più tener fronte ai loro nemici.
\par 15 Dovunque andavano, la mano dell'Eterno era contro di loro a loro danno, come l'Eterno avea detto, come l'Eterno avea loro giurato: e furono oltremodo angustiati.
\par 16 E l'Eterno suscitava dei giudici, che li liberavano dalle mani di quelli che li spogliavano.
\par 17 Ma neppure ai loro giudici davano ascolto, poiché si prostituivano ad altri dèi, e si prostravan dinanzi a loro. E abbandonarono ben presto la via battuta dai loro padri, i quali aveano ubbidito ai comandamenti dell'Eterno; ma essi non fecero così.
\par 18 E quando l'Eterno suscitava loro de' giudici, l'Eterno era col giudice, e li liberava dalla mano de' loro nemici durante tutta la vita del giudice; poiché l'Eterno si pentiva a sentire i gemiti che mandavano a motivo di quelli che li opprimevano e li angariavano.
\par 19 Ma, quando il giudice moriva, tornavano a corrompersi più dei loro padri, andando dietro ad altri dèi per servirli e prostrarsi dinanzi a loro; non rinunziavano menomamente alle loro pratiche e alla loro caparbia condotta.
\par 20 Perciò l'ira dell'Eterno si accese contro Israele, ed egli disse: 'Giacché questa nazione ha violato il patto che avevo stabilito coi loro padri ed essi non hanno ubbidito alla mia voce,
\par 21 anch'io non caccerò più d'innanzi a loro alcuna delle nazioni che Giosuè lasciò quando morì;
\par 22 così, per mezzo d'esse, metterò alla prova Israele per vedere se si atterranno alla via dell'Eterno e cammineranno per essa come fecero i loro padri, o no'.
\par 23 E l'Eterno lasciò stare quelle nazioni senz'affrettarsi a cacciarle, e non le diede nelle mani di Giosuè.

\chapter{3}

\par 1 Or queste son le nazioni che l'Eterno lasciò stare affin di mettere per mezzo d'esse alla prova Israele, cioè tutti quelli che non avean visto le guerre di Canaan.
\par 2 (Egli volea soltanto che le nuove generazioni de' figliuoli d'Israele conoscessero e imparassero la guerra: quelli, per lo meno, che prima non l'avean mai vista):
\par 3 i cinque principi dei Filistei, tutti i Cananei, i Sidonî, e gli Hivvei, che abitavano la montagna del Libano, dal monte Baal-Hermon fino all'ingresso di Hamath.
\par 4 Queste nazioni servirono a mettere Israele alla prova, per vedere se Israele ubbidirebbe ai comandamenti che l'Eterno avea dati ai loro padri per mezzo di Mosè.
\par 5 Così i figliuoli d'Israele abitarono in mezzo ai Cananei, agli Hittei, agli Amorei, ai Ferezei, agli Hivvei ed ai Gebusei;
\par 6 sposarono le loro figliuole, maritaron le proprie figliuole coi loro figliuoli, e servirono ai loro dèi.
\par 7 I figliuoli d'Israele fecero ciò ch'è male agli occhi dell'Eterno; dimenticarono l'Eterno, il loro Dio, e servirono agl'idoli di Baal e d'Astarte.
\par 8 Perciò l'ira dell'Eterno si accese contro Israele, ed egli li diede nelle mani di Cushan-Rishathaim, re di Mesopotamia; e i figliuoli d'Israele furon servi di Cushan-Rishathaim per otto anni.
\par 9 Poi i figliuoli d'Israele gridarono all'Eterno, e l'Eterno suscitò loro un liberatore: Othniel, figliuolo di Kenaz, fratello minore di Caleb; ed egli li liberò.
\par 10 Lo spirito dell'Eterno fu sopra lui, ed egli fu giudice d'Israele; uscì a combattere, e l'Eterno gli diede nelle mani Cushan-Rishathaim, re di Mesopotamia; e la sua mano fu potente contro Cushan-Rishathaim.
\par 11 Il paese ebbe requie per quarant'anni; poi Othniel, figlio di Kenaz, morì.
\par 12 I figliuoli d'Israele continuarono a fare ciò ch'è male agli occhi dell'Eterno; e l'Eterno rese forte Eglon, re di Moab, contro Israele, perch'essi avean fatto ciò ch'è male agli occhi dell'Eterno.
\par 13 Ed Eglon radunò attorno a sé i figliuoli di Ammon e di Amalek, e andò e batté Israele e s'impadronì della città delle palme.
\par 14 E i figliuoli d'Israele furon servi di Eglon, re di Moab, per diciotto anni.
\par 15 Ma i figliuoli d'Israele gridarono all'Eterno, ed egli suscitò loro un liberatore: Ehud, figliuolo di Ghera, Beniaminita, che era mancino. I figliuoli d'Israele mandarono per mezzo di lui un regalo a Eglon, re di Moab.
\par 16 Ehud si fece una spada a due tagli, lunga un cubito; e se la cinse sotto la veste, al fianco destro.
\par 17 E offrì il regalo a Eglon, re di Moab, ch'era uomo molto grasso.
\par 18 E quand'ebbe finita la presentazione del regalo, rimandò la gente che l'avea portato.
\par 19 Ma egli, giunto alla cava di pietre ch'è presso a Ghilgal, tornò indietro, e disse: 'O re, io ho qualcosa da dirti in segreto'. E il re disse: 'Silenzio!' E tutti quelli che gli stavan dappresso, uscirono.
\par 20 Allora Ehud s'accostò al re, che stava seduto nella sala disopra, riservata a lui solo per prendervi il fresco, e gli disse: 'Ho una parola da dirti da parte di Dio'. Quegli s'alzò dal suo seggio:
\par 21 e Ehud, stesa la mano sinistra, trasse la spada dal suo fianco destro, e gliela piantò nel ventre.
\par 22 Anche l'elsa entrò dopo la lama; e il grasso si rinchiuse attorno alla lama; poich'egli non gli ritirò dal ventre la spada, che gli usciva per di dietro.
\par 23 Poi Ehud uscì nel portico, chiuse le porte della sala disopra, e mise i chiavistelli.
\par 24 Or quando fu uscito, vennero i servi, i quali guardarono, ed ecco che le porte della sala disopra eran chiuse a chiavistello; e dissero: 'Certo egli fa i suoi bisogni nello stanzino della sala fresca'.
\par 25 E tanto aspettarono, che ne furon confusi; e com'egli non apriva le porte della sala, quelli presero la chiave, aprirono, ed ecco che il loro signore era steso per terra, morto.
\par 26 Mentr'essi indugiavano, Ehud si diè alla fuga, passò oltre le cave di pietra, e si mise in salvo nella Seira.
\par 27 Arrivato che fu, suonò la tromba nella contrada montuosa di Efraim, e i figliuoli d'Israele scesero con lui dalla contrada montuosa, ed egli si mise alla loro testa.
\par 28 E disse loro: 'Seguitemi, perché l'Eterno v'ha dato nelle mani i Moabiti, vostri nemici'. E quelli scesero dietro a lui, s'impadronirono de' guadi del Giordano per impedirne il passo ai Moabiti, e non lasciaron passare alcuno.
\par 29 In quel tempo sconfissero circa diecimila Moabiti, tutti robusti e valorosi; e non ne scampò uno.
\par 30 Così, in quel giorno, Moab fu umiliato sotto la mano d'Israele, e il paese ebbe requie per ottant'anni.
\par 31 Dopo Ehud, venne Shamgar, figliuolo di Anath. Egli sconfisse seicento Filistei con un pungolo da buoi; e anch'egli liberò Israele.

\chapter{4}

\par 1 Morto che fu Ehud, i figliuoli d'Israele continuarono a fare ciò ch'è male agli occhi dell'Eterno.
\par 2 E l'Eterno li diede nelle mani di Iabin, re di Canaan, che regnava a Hatsor. Il capo del suo esercito era Sisera che abitava a Harosceth-Goim.
\par 3 E i figliuoli d'Israele gridarono all'Eterno, perché Iabin avea novecento carri di ferro, e già da venti anni opprimeva con violenza i figliuoli d'Israele.
\par 4 Or in quel tempo era giudice d'Israele una profetessa, Debora, moglie di Lappidoth.
\par 5 Essa sedeva sotto la palma di Debora, fra Rama e Bethel, nella contrada montuosa di Efraim, e i figliuoli d'Israele salivano a lei per farsi rendere giustizia.
\par 6 Or ella mandò a chiamare Barak, figliuolo di Abinoam, da Kades di Neftali, e gli disse: 'L'Eterno, l'Iddio d'Israele, non t'ha egli dato quest'ordine: Va', raduna sul monte Tabor e prendi teco diecimila uomini de' figliuoli di Neftali e de' figliuoli di Zabulon.
\par 7 E io attirerò verso te, al torrente Kison, Sisera, capo dell'esercito di Iabin, coi suoi carri e la sua numerosa gente, e io lo darò nelle tue mani'.
\par 8 Barak le rispose: 'Se vieni meco andrò; ma se non vieni meco, non andrò'.
\par 9 Ed ella disse: 'Certamente, verrò con te; soltanto, la via per cui ti metti non ridonderà ad onor tuo; poiché l'Eterno darà Sisera in man d'una donna'. E Debora si levò e andò con Barak a Kades.
\par 10 E Barak convocò Zabulon e Neftali a Kades; diecimila uomini si misero al suo seguito, e Debora salì con lui.
\par 11 Or Heber, il Keneo, s'era separato dai Kenei, discendenti di Hobab, suocero di Mosè, e avea piantate le sue tende fino al querceto di Tsaannaim, ch'è presso a Kades.
\par 12 Fu riferito a Sisera che Barak, figliuolo di Abinoam, era salito sul monte Tabor.
\par 13 E Sisera adunò tutti i suoi carri, novecento carri di ferro, e tutta la gente ch'era seco, da Harosceth-Goim fino al torrente Kison.
\par 14 E Debora disse a Barak: 'Lèvati, perché questo è il giorno in cui l'Eterno ha dato Sisera nelle tue mani. L'Eterno non va egli dinanzi a te?' Allora Barak scese dal monte Tabor, seguito da diecimila uomini.
\par 15 E l'Eterno mise in rotta, davanti a Barak, Sisera con tutti i suoi carri e con tutto il suo esercito, che fu passato a fil di spada; e Sisera, sceso dal carro, si diè alla fuga a piedi.
\par 16 Ma Barak inseguì i carri e l'esercito fino a Harosceth-Goim; e tutto l'esercito di Sisera cadde sotto i colpi della spada, e non ne scampò un uomo.
\par 17 Sisera fuggì a piedi verso la tenda di Jael, moglie di Heber, il Keneo, perché v'era pace fra Iabin, re di Hatsor, e la casa di Heber il Keneo.
\par 18 E Jael uscì incontro a Sisera e gli disse: 'Entra, signor mio, entra da me: non temere'. Ed egli entrò da lei nella sua tenda, ed essa lo coprì con una coperta.
\par 19 Ed egli le disse: 'Deh, dammi un po' d'acqua da bere perché ho sete'. E quella, aperto l'otre del latte, gli diè da bere, e lo coprì.
\par 20 Ed egli le disse: 'Stattene all'ingresso della tenda; e se qualcuno viene a interrogarti dicendo: C'è qualcuno qui dentro? di' di no'.
\par 21 Allora Jael, moglie di Heber, prese un piuolo della tenda; e, dato di piglio al martello, venne pian piano a lui, e gli piantò il piuolo nella tempia sì ch'esso penetrò in terra. Egli era profondamente addormentato e sfinito; e morì.
\par 22 Ed ecco che, come Barak inseguiva Sisera, Jael uscì ad incontrarlo, e gli disse: 'Vieni, e ti mostrerò l'uomo che cerchi'. Ed egli entrò da lei; ed ecco, Sisera era steso morto, col piuolo nella tempia.
\par 23 Così Dio umiliò quel giorno Iabin, re di Canaan, dinanzi ai figliuoli d'Israele.
\par 24 E la mano de' figliuoli d'Israele s'andò sempre più aggravando su Iabin, re di Canaan, finché ebbero sterminato Iabin, re di Canaan.

\chapter{5}

\par 1 In quel giorno, Debora cantò questo cantico con Barak, figliuolo di Abinoam:
\par 2 "Perché dei capi si son messi alla testa del popolo in Israele, perché il popolo s'è mostrato volenteroso, benedite l'Eterno!
\par 3 Ascoltate, o re! Porgete orecchio, o principi! All'Eterno, sì, io canterò, salmeggerò all'Eterno, all'Iddio d'Israele.
\par 4 O Eterno, quand'uscisti da Seir, quando venisti dai campi di Edom, la terra tremò, ed anche i cieli si sciolsero, anche le nubi si sciolsero in acqua.
\par 5 I monti furono scossi per la presenza dell'Eterno, anche il Sinai, là, fu scosso dinanzi all'Eterno, all'Iddio d'Israele.
\par 6 Ai giorni di Shamgar, figliuolo di Anath, ai giorni di Jael, le strade erano abbandonate, e i viandanti seguivan sentieri tortuosi.
\par 7 I capi mancavano in Israele; mancavano, finché non sorsi io, Debora, finché non sorsi io, come una madre in Israele.
\par 8 Si sceglievan de' nuovi dèi, e la guerra era alle porte. Si scorgeva forse uno scudo, una lancia, fra quarantamila uomini d'Israele?
\par 9 Il mio cuore va ai condottieri d'Israele! O voi che v'offriste volenterosi fra il popolo, benedite l'Eterno!
\par 10 Voi che montate asine bianche, voi che sedete su ricchi tappeti, e voi che camminate per le vie, cantate!
\par 11 Lungi dalle grida degli arcieri là tra gli abbeveratoi, si celebrino gli atti di giustizia dell'Eterno, gli atti di giustizia de' suoi capi in Israele! Allora il popolo dell'Eterno discese alle porte.
\par 12 Déstati, déstati, o Debora! déstati, déstati, sciogli un canto! Lèvati, o Barak, e prendi i tuoi prigionieri, o figlio d'Abinoam!
\par 13 Allora scese un residuo, alla voce dei nobili scese un popolo, l'Eterno scese con me fra i prodi.
\par 14 Da Efraim vennero quelli che stanno sul monte Amalek; al tuo seguito venne Beniamino fra le tue genti; da Makir scesero de' capi, e da Zabulon quelli che portano il bastone del comando.
\par 15 I principi d'Issacar furon con Debora; quale fu Barak, tale fu Issacar, si slanciò nella valle sulle orme di lui. Presso i rivi di Ruben, grandi furon le risoluzioni del cuore!
\par 16 Perché sei tu rimasto fra gli ovili ad ascoltare il flauto dei pastori? Presso i rivi di Ruben, grandi furon le deliberazioni del cuore!
\par 17 Galaad non ha lasciato la sua dimora di là dal Giordano; e Dan perché s'è tenuto sulle sue navi? Ascer è rimasto presso il lido del mare, e s'è riposato ne' suoi porti.
\par 18 Zabulon è un popolo che ha esposto la sua vita alla morte, e Neftali, anch'egli, sulle alture della campagna.
\par 19 I re vennero, pugnarono; allora pugnarono i re di Canaan a Taanac, presso le acque di Meghiddo; non ne riportarono un pezzo d'argento.
\par 20 Dai cieli si combatté: gli astri, nel loro corso, combatteron contro Sisera.
\par 21 Il torrente di Kison li travolse, l'antico torrente, il torrente di Kison. Anima mia, avanti, con forza!
\par 22 Allora gli zoccoli de' cavalli martellavano il suolo, al galoppo, al galoppo de' lor guerrieri in fuga.
\par 23 'Maledite Meroz', dice l'angelo dell'Eterno: 'maledite, maledite i suoi abitanti, perché non vennero in soccorso dell'Eterno, in soccorso dell'Eterno insieme coi prodi!'
\par 24 Benedetta sia fra le donne Jael, moglie di Heber, il Keneo! Fra le donne che stan sotto le tende, sia ella benedetta!
\par 25 Egli chiese dell'acqua, ed ella gli diè del latte; in una coppa d'onore gli offerse della crema.
\par 26 Con una mano, diè di piglio al piuolo; e, con la destra, al martello degli operai; colpì Sisera, gli spaccò la testa, gli fracassò, gli trapassò le tempie.
\par 27 Ai piedi d'essa ei si piegò, cadde, giacque disteso; a' piedi d'essa si piegò, e cadde; là dove si piegò, cadde esanime.
\par 28 La madre di Sisera guarda per la finestra, e grida a traverso l'inferriata: 'Perché il suo carro sta tanto a venire? perché son così lente le ruote de' suoi carri?'
\par 29 Le più savie delle sue dame le rispondono, ed ella pure replica a se stessa:
\par 30 'Non trovan bottino? non se lo dividono? Una fanciulla, due fanciulle per ognuno; a Sisera un bottino di vesti variopinte; un bottino di vesti variopinte e ricamate, di vesti variopinte e ricamate d'ambo i lati per le spalle del vincitore!'
\par 31 Così periscano tutti i tuoi nemici, o Eterno! E quei che t'amano sian come il sole quando si leva in tutta la sua forza!" Ed il paese ebbe requie per quarant'anni.

\chapter{6}

\par 1 Or i figliuoli d'Israele fecero ciò ch'è male agli occhi dell'Eterno, e l'Eterno li diede nelle mani di Madian per sette anni.
\par 2 La mano di Madian fu potente contro Israele; e, per la paura dei Madianiti, i figliuoli d'Israele si fecero quelle caverne che son nei monti, e delle spelonche e dei forti.
\par 3 Quando Israele avea seminato, i Madianiti con gli Amalekiti e coi figliuoli dell'oriente salivano contro di lui,
\par 4 s'accampavano contro gl'Israeliti, distruggevano tutti i prodotti del paese fin verso Gaza, e non lasciavano in Israele né viveri, né pecore, né buoi, né asini.
\par 5 Poiché salivano coi loro greggi e con le loro tende, e arrivavano come una moltitudine di locuste; essi e i loro cammelli erano innumerevoli, e venivano nel paese per devastarlo.
\par 6 Israele dunque fu ridotto in gran miseria a motivo di Madian, e i figliuoli d'Israele gridarono all'Eterno.
\par 7 E avvenne che, quando i figliuoli d'Israele ebbero gridato all'Eterno a motivo di Madian,
\par 8 l'Eterno mandò ai figliuoli d'Israele un profeta, che disse loro: 'Così dice l'Eterno, l'Iddio d'Israele: Io vi feci salire dall'Egitto e vi trassi dalla casa di schiavitù;
\par 9 vi liberai dalla mano degli Egiziani e dalla mano di tutti quelli che vi opprimevano; li cacciai d'innanzi a voi, vi detti il loro paese, e vi dissi:
\par 10 Io sono l'Eterno, il vostro Dio; non adorate gli dèi degli Amorei nel paese de' quali abitate; ma voi non avete dato ascolto alla mia voce'.
\par 11 Poi venne l'angelo dell'Eterno, e si assise sotto il terebinto d'Ofra, che apparteneva a Joas, Abiezerita; e Gedeone, figliuolo di Joas, batteva il grano nello strettoio, per metterlo al sicuro dai Madianiti.
\par 12 L'angelo dell'Eterno gli apparve e gli disse: 'L'Eterno è teco, o uomo forte e valoroso!'
\par 13 E Gedeone gli rispose: 'Ahimè, signor mio, se l'Eterno è con noi, perché ci è avvenuto tutto questo? e dove sono tutte quelle sue maraviglie che i nostri padri ci hanno narrate dicendo: - L'Eterno non ci trasse egli dall'Egitto? - Ma ora l'Eterno ci ha abbandonato e ci ha dato nelle mani di Madian'.
\par 14 Allora l'Eterno si volse a lui, e gli disse: 'Va' con cotesta tua forza, e salva Israele dalla mano di Madian; non son io che ti mando?'
\par 15 Ed egli a lui: 'Ah, signor mio, con che salverò io Israele? Ecco, il mio migliaio è il più povero di Manasse, e io sono il più piccolo nella casa di mio padre'.
\par 16 L'Eterno gli disse: 'Perché io sarò teco, tu sconfiggerai i Madianiti come se fossero un uomo solo'.
\par 17 E Gedeone a lui: 'Se ho trovato grazia agli occhi tuoi, dammi un segno che sei proprio tu che mi parli.
\par 18 Deh, non te ne andar di qui prima ch'io torni da te, ti rechi la mia offerta, e te la metta dinanzi'. E l'Eterno disse: 'Aspetterò finché tu ritorni'.
\par 19 Allora Gedeone entrò in casa, preparò un capretto, e con un efa di farina fece delle focacce azzime; mise la carne in un canestro, il brodo in una pentola, gli portò tutto sotto il terebinto, e gliel'offrì.
\par 20 E l'angelo di Dio gli disse: 'Prendi la carne e le focacce azzime, mettile su questa roccia, e versavi su il brodo'. Ed egli fece così.
\par 21 Allora l'angelo dell'Eterno stese la punta del bastone che aveva in mano e toccò la carne e le focacce azzime; e salì dalla roccia un fuoco, che consumò la carne e le focacce azzime; e l'angelo dell'Eterno scomparve dalla vista di lui.
\par 22 E Gedeone vide ch'era l'angelo dell'Eterno, e disse: 'Misero me, o Signore, o Eterno! giacché ho veduto l'angelo dell'Eterno a faccia a faccia!'
\par 23 E l'Eterno gli disse: 'Sta' in pace, non temere, non morrai!'
\par 24 Allora Gedeone edificò quivi un altare all'Eterno, e lo chiamò 'l'Eterno pace'. Esso esiste anche al dì d'oggi a Ofra degli Abiezeriti.
\par 25 In quella stessa notte, l'Eterno gli disse: 'Prendi il giovenco di tuo padre e il secondo toro di sette anni, demolisci l'altare di Baal che è di tuo padre, abbatti l'idolo che gli sta vicino,
\par 26 e costruisci un altare all'Eterno, al tuo Dio, in cima a questa roccia, disponendo ogni cosa con ordine; poi prendi il secondo toro, e offrilo in olocausto sulle legna dell'idolo che avrai abbattuto'.
\par 27 Allora Gedeone prese dieci uomini fra i suoi servitori e fece come l'Eterno gli avea detto; ma, non osando farlo di giorno, per paura della casa di suo padre e della gente della città, lo fece di notte.
\par 28 E quando la gente della città l'indomani mattina si levò, ecco che l'altare di Baal era stato demolito, che l'idolo postovi accanto era abbattuto, e che il secondo toro era offerto in olocausto sull'altare ch'era stato costruito.
\par 29 E dissero l'uno all'altro: 'Chi ha fatto questo?' Ed essendosi informati e avendo fatto delle ricerche, fu loro detto: 'Gedeone, figliuolo di Joas, ha fatto questo'.
\par 30 Allora la gente della città disse a Joas: 'Mena fuori il tuo figliuolo, e sia messo a morte, perché ha demolito l'altare di Baal ed ha abbattuto l'idolo che gli stava vicino'.
\par 31 E Joas rispose a tutti quelli che insorgevano contro a lui: 'Volete voi difender la causa di Baal? volete venirgli in soccorso? chi vorrà difendere la sua causa sarà messo a morte prima di domattina; s'esso è dio, difenda da sé la sua causa, giacché hanno demolito il suo altare'.
\par 32 Perciò quel giorno Gedeone fu chiamato Ierubbaal, perché si disse: 'Difenda Baal la sua causa contro a lui, giacché egli ha demolito il suo altare'.
\par 33 Or tutti i Madianiti, gli Amalekiti e i figliuoli dell'oriente si radunarono, passarono il Giordano, e si accamparono nella valle di Izreel.
\par 34 Ma lo spirito dell'Eterno s'impossessò di Gedeone, il quale sonò la tromba, e gli Abiezeriti furono convocati per seguirlo.
\par 35 Egli mandò anche dei messi in tutto Manasse, che fu pure convocato per seguirlo; mandò altresì de' messi nelle tribù di Ascer, di Zabulon e di Neftali, le quali salirono a incontrare gli altri.
\par 36 E Gedeone disse a Dio: 'Se vuoi salvare Israele per mia mano, come hai detto,
\par 37 ecco, io metterò un vello di lana sull'aia: se c'è della rugiada sul vello soltanto e tutto il terreno resta asciutto, io conoscerò che tu salverai Israele per mia mano come hai detto'.
\par 38 E così avvenne. La mattina dopo, Gedeone si levò per tempo, strizzò il vello e ne spremé la rugiada: una coppa piena d'acqua.
\par 39 E Gedeone disse a Dio: 'Non s'accenda l'ira tua contro di me; io non parlerò più che questa volta. Deh, ch'io faccia ancora un'altra prova sola col vello: resti asciutto soltanto il vello, e ci sia della rugiada su tutto il terreno'.
\par 40 E Dio fece così quella notte: il vello soltanto restò asciutto, e ci fu della rugiada su tutto il terreno.

\chapter{7}

\par 1 Ierubbaal dunque, vale a dire Gedeone, con tutta la gente ch'era con lui, levatosi la mattina di buon'ora, si accampò presso la sorgente di Harod. Il campo di Madian era al nord di quello di Gedeone, verso la collina di Moreh, nella valle.
\par 2 E l'Eterno disse a Gedeone: 'La gente che è teco è troppo numerosa perch'io dia Madian nelle sue mani; Israele potrebbe vantarsi di fronte a me, e dire: - La mia mano è quella che m'ha salvato. -
\par 3 Or dunque fa' proclamar questo, sì che il popolo l'oda: - Chiunque ha paura, e trema, se ne torni indietro e s'allontani dal monte di Galaad'. E tornarono indietro ventiduemila uomini del popolo, e ne rimasero diecimila.
\par 4 L'Eterno disse a Gedeone: 'La gente è ancora troppo numerosa; falla scendere all'acqua, e quivi io te ne farò la scelta. Quello del quale ti dirò: - Questo vada teco - andrà teco; e quello del quale ti dirò: - Questo non vada teco - non andrà'.
\par 5 Gedeone fece dunque scender la gente all'acqua; e l'Eterno gli disse: 'Tutti quelli che lambiranno l'acqua con la lingua, come la lambisce il cane, li porrai da parte; così pure tutti quelli che, per bere, si metteranno in ginocchio'.
\par 6 E il numero di quelli che lambirono l'acqua portandosela alla bocca nella mano, fu di trecento uomini; tutto il resto della gente si mise in ginocchio per bever l'acqua.
\par 7 Allora l'Eterno disse a Gedeone: 'Mediante questi trecento uomini che hanno lambito l'acqua io vi libererò e darò i Madianiti nelle tue mani. Tutto il resto della gente se ne vada, ognuno a casa sua'.
\par 8 I trecento presero i viveri del popolo e le sue trombe; e Gedeone, rimandati tutti gli altri uomini d'Israele, ciascuno alla sua tenda, ritenne questi con sé. Or il campo di Madian era sotto quello di lui, nella valle.
\par 9 In quella stessa notte, l'Eterno disse a Gedeone: 'Lèvati, piomba sul campo, perché io te l'ho dato nelle mani.
\par 10 Ma se hai paura di farlo, scendivi con Purah tuo servo,
\par 11 e udrai quello che dicono; e, dopo questo, le tue mani saranno fortificate per piombar sul campo'. Egli dunque scese con Purah, suo servo, fino agli avamposti del campo.
\par 12 Or i Madianiti, gli Amalekiti e tutti i figliuoli dell'oriente erano sparsi nella valle come una moltitudine di locuste, e i loro cammelli erano innumerevoli, come la rena ch'è sul lido del mare.
\par 13 E come Gedeone vi giunse, ecco che un uomo raccontava un sogno al suo compagno, e gli diceva: 'Io ho fatto un sogno; mi pareva che un pan tondo, d'orzo, rotolasse nel campo di Madian, giungesse alla tenda, la investisse, in modo da farla cadere, da rovesciarla, da lasciarla atterrata'.
\par 14 E il suo compagno gli rispose e gli disse: 'Questo non è altro che la spada di Gedeone, figliuolo di Joas, uomo d'Israele; nelle sue mani Iddio ha dato Madian e tutto il campo'.
\par 15 Quando Gedeone ebbe udito il racconto del sogno e la sua interpretazione, adorò Dio; poi tornò al campo d'Israele, e disse: 'Levatevi, perché l'Eterno ha dato nelle vostre mani il campo di Madian!'
\par 16 E divise i trecento uomini in tre schiere, consegnò a tutti quanti delle trombe e delle brocche vuote con delle fiaccole entro le brocche;
\par 17 e disse loro: 'Guardate me, e fate come farò io; quando sarò giunto all'estremità del campo, come farò io, così farete voi;
\par 18 e quando io con tutti quelli che son meco sonerò la tromba, anche voi darete nelle trombe intorno a tutto il campo, e direte: - Per l'Eterno e per Gedeone!'
\par 19 Gedeone e i cento uomini ch'eran con lui giunsero alla estremità del campo, al principio della vigilia di mezzanotte, nel mentre che si era appena data la muta alle sentinelle. Sonaron le trombe, e spezzaron le brocche che tenevano in mano.
\par 20 Allora le tre schiere dettero nelle trombe, spezzaron le brocche; con la sinistra presero le fiaccole, e con la destra le trombe per sonare, e si misero a gridare: 'La spada per l'Eterno e per Gedeone!'
\par 21 Ognuno di loro rimase al suo posto, intorno al campo; e tutto il campo si diè a correre, a gridare, a fuggire.
\par 22 E mentre quelli sonavan le trecento trombe, l'Eterno fece volger la spada di ciascuno contro il compagno, per tutto il campo. E il campo fuggì fino a Beth-Scittah, verso Tserera, sino all'orlo d'Abel-Meholah presso Tabbath.
\par 23 Gl'Israeliti di Neftali, di Ascer e di tutto Manasse si radunarono e inseguirono i Madianiti.
\par 24 E Gedeone mandò de' messi per tutta la contrada montuosa di Efraim a dire: 'Scendete incontro ai Madianiti, e tagliate loro il passo delle acque fino a Beth-Barah, e i guadi del Giordano'. Così tutti gli uomini di Efraim si radunarono e s'impadronirono dei passi delle acque fino a Beth-Barah e dei guadi del Giordano.
\par 25 E presero due principi di Madian, Oreb e Zeeb; uccisero Oreb al masso di Oreb, e Zeeb allo strettoio di Zeeb: inseguirono i Madianiti, e portarono le teste di Oreb e di Zeeb a Gedeone, dall'altro lato del Giordano.

\chapter{8}

\par 1 Gli uomini di Efraim dissero a Gedeone: 'Che azione è questa che tu ci hai fatto, non chiamandoci quando sei andato a combattere contro Madian?' Ed ebbero con lui una disputa violenta.
\par 2 Ed egli rispose loro: 'Che ho fatto io al paragon di voi? la racimolatura d'Efraim non vale essa più della vendemmia d'Abiezer?
\par 3 Iddio v'ha dato nelle mani i principi di Madian, Oreb e Zeeb! che dunque ho potuto far io al paragon di voi?' Quand'egli ebbe lor detto quella parola, la loro ira contro di lui si calmò.
\par 4 E Gedeone arrivò al Giordano, e lo passò con i trecento uomini ch'erano con lui; i quali, benché stanchi, continuavano a inseguire il nemico.
\par 5 E disse a quelli di Succoth: 'Date, vi prego, dei pani alla gente che mi segue, perché è stanca, ed io sto inseguendo Zebah e Tsalmunna, re di Madian'.
\par 6 Ma i capi di Succoth risposero: 'Tieni tu forse già nelle tue mani i polsi di Zebah e di Tsalmunna, che abbiamo a dar del pane al tuo esercito?'
\par 7 E Gedeone disse: 'Ebbene! quando l'Eterno mi avrà dato nelle mani Zebah e Tsalmunna, io vi lacererò le carni con delle spine del deserto e con de' triboli'.
\par 8 Di là salì a Penuel, e parlò a quei di Penuel nello stesso modo; ed essi gli risposero come avean fatto quei di Succoth.
\par 9 Ed egli disse anche a quei di Penuel: 'Quando tornerò in pace, abbatterò questa torre'.
\par 10 Or Zebah e Tsalmunna erano a Karkor col loro esercito di circa quindicimila uomini, ch'era tutto quel che rimaneva dell'intero esercito dei figli dell'oriente, poiché centoventimila uomini che portavano spada erano stati uccisi.
\par 11 Gedeone salì per la via di quelli che abitano sotto tende a oriente di Nobah e di Iogbeha, e sconfisse l'esercito che si credeva sicuro.
\par 12 E Zebah e Tsalmunna si diedero alla fuga; ma egli li inseguì, prese i due re di Madian, Zebah e Tsalmunna, e sbaragliò tutto l'esercito.
\par 13 Poi Gedeone, figliuolo di Joas, tornò dalla battaglia, per la salita di Heres.
\par 14 Mise le mani sopra un giovane della gente di Succoth, e lo interrogò; ed ei gli diè per iscritto i nomi dei capi e degli anziani di Succoth, ch'erano settantasette uomini.
\par 15 Poi venne alla gente di Succoth, e disse: 'Ecco Zebah e Tsalmunna, a proposito de' quali m'insultaste dicendo: Hai tu forse già nelle mani i polsi di Zebah e di Tsalmunna, che noi abbiamo da dar del pane alla tua gente stanca?'
\par 16 E prese gli anziani della città, e con delle spine del deserto e con de' triboli castigò gli uomini di Succoth.
\par 17 E abbatté la torre di Penuel e uccise la gente della città.
\par 18 Poi disse a Zebah e a Tsalmunna: 'Com'erano gli uomini che avete uccisi al Tabor?' Quelli risposero: 'Eran come te; ognun d'essi avea l'aspetto d'un figlio di re'.
\par 19 Ed egli riprese: 'Eran miei fratelli, figliuoli di mia madre; com'è vero che l'Eterno vive, se aveste risparmiato loro la vita, io non vi ucciderei!'
\par 20 Poi disse a Iether, suo primogenito: 'Lèvati, uccidili!' Ma il giovane non tirò la spada, perché avea paura, essendo ancora un giovinetto.
\par 21 E Zebah e Tsalmunna dissero: 'Lèvati tu stesso e dacci il colpo mortale; poiché qual è l'uomo tal è la sua forza'. E Gedeone si levò e uccise Zebah e Tsalmunna, e prese le mezzelune che i loro cammelli portavano al collo.
\par 22 Allora gli uomini d'Israele dissero a Gedeone: 'Regna su noi tu e il tuo figliuolo e il figliuolo del tuo figliuolo, giacché ci hai salvati dalla mano di Madian'.
\par 23 Ma Gedeone rispose loro: 'Io non regnerò su voi, né il mio figliuolo regnerà su voi; l'Eterno è quegli che regnerà su voi!'
\par 24 Poi Gedeone disse loro: 'Una cosa voglio chiedervi: che ciascun di voi mi dia gli anelli del suo bottino'. (I nemici aveano degli anelli d'oro perché erano Ismaeliti).
\par 25 Quelli risposero: 'Li daremo volentieri'. E stesero un mantello, sul quale ciascuno gettò gli anelli del suo bottino.
\par 26 Il peso degli anelli d'oro ch'egli avea chiesto fu di millesettecento sicli d'oro, oltre le mezzelune, i pendenti e le vesti di porpora che i re di Madian aveano addosso, e oltre i collari che i loro cammelli aveano al collo.
\par 27 E Gedeone ne fece un efod, che pose in Ofra, sua città; tutto Israele v'andò a prostituirsi, ed esso diventò un'insidia per Gedeone e per la sua casa.
\par 28 Così Madian fu umiliato davanti ai figliuoli d'Israele, e non alzò più il capo; e il paese ebbe requie per quarant'anni, durante la vita di Gedeone.
\par 29 Ierubbaal figliuolo di Joas, tornò a dimorare a casa sua.
\par 30 Or Gedeone ebbe settanta figliuoli, che gli nacquero dalle molte mogli che ebbe.
\par 31 E la sua concubina, che stava a Sichem, gli partorì anch'ella un figliuolo, al quale pose nome Abimelec.
\par 32 Poi Gedeone, figliuolo di Joas, morì in buona vecchiaia, e fu sepolto nella tomba di Joas suo padre, a Ofra degli Abiezeriti.
\par 33 Dopo che Gedeone fu morto, i figliuoli d'Israele ricominciarono a prostituirsi agl'idoli di Baal, e presero Baal-Berith come loro dio.
\par 34 I figliuoli d'Israele non si ricordarono dell'Eterno, del loro Dio, che li avea liberati dalle mani di tutti i loro nemici d'ogn'intorno;
\par 35 e non dimostrarono alcuna gratitudine alla casa di Ierubbaal, ossia di Gedeone, per tutto il bene ch'egli avea fatto a Israele.

\chapter{9}

\par 1 Or Abimelec, figliuolo di Ierubbaal, andò a Sichem dai fratelli di sua madre e parlò loro e a tutta la famiglia del padre di sua madre, dicendo:
\par 2 'Deh, dite ai Sichemiti, in modo che tutti odano: Qual cosa è migliore per voi, che settanta uomini, tutti figliuoli di Ierubbaal, regnino su voi, oppure che regni su voi uno solo? E ricordatevi ancora che io sono vostre ossa e vostra carne'.
\par 3 I fratelli di sua madre parlarono di lui, ripetendo a tutti i Sichemiti tutte quelle parole; e il cuor loro s'inchinò a favore di Abimelec, perché dissero: 'È nostro fratello'.
\par 4 E gli diedero settanta sicli d'argento, che tolsero dal tempio di Baal-Berith, coi quali Abimelec assoldò degli uomini da nulla e audaci che lo seguirono.
\par 5 Ed egli venne alla casa di suo padre, a Ofra, e uccise sopra una stessa pietra i suoi fratelli, figliuoli di Ierubbaal, settanta uomini; ma Jotham, figliuolo minore di Ierubbaal, scampò, perché s'era nascosto.
\par 6 Poi tutti i Sichemiti e tutta la casa di Millo si radunarono e andarono a proclamar re Abimelec, presso la quercia del monumento che si trova a Sichem.
\par 7 E Jotham, essendo stato informato della cosa, andò a porsi sulla sommità del monte Garizim, e alzando la voce gridò: 'Ascoltatemi, Sichemiti, e vi ascolti Iddio!
\par 8 Un giorno gli alberi si misero in cammino per ungere un re che regnasse su loro; e dissero all'ulivo: - Regna tu su noi.
\par 9 - Ma l'ulivo rispose loro: Rinunzierei io al mio olio che Dio e gli uomini onorano in me, per andare ad agitarmi al disopra degli alberi?
\par 10 Allora gli alberi dissero al fico: - Vieni tu a regnare su noi.
\par 11 - Ma il fico rispose loro: Rinunzierei io alla mia dolcezza e al mio frutto squisito per andare ad agitarmi al disopra degli alberi?
\par 12 Poi gli alberi dissero alla vite: - Vieni tu a regnare su noi.
\par 13 - Ma la vite rispose loro: Rinunzierei io al mio vino che rallegra Dio e gli uomini, per andare ad agitarmi al disopra degli alberi?
\par 14 Allora tutti gli alberi dissero al pruno: - Vieni tu a regnare su noi.
\par 15 - E il pruno rispose agli alberi: Se è proprio in buona fede che volete ungermi re per regnare su voi, venite a rifugiarvi sotto l'ombra mia; se no, esca un fuoco dal pruno e divori i cedri del Libano!
\par 16 E ora, se vi siete condotti con fedeltà e con integrità proclamando re Abimelec, se avete agito bene verso Ierubbaal e la sua casa, se avete ricompensato lui, mio padre, di quel che ha fatto per voi
\par 17 quando ha combattuto per voi, quando ha messo a repentaglio la sua vita e vi ha liberati dalle mani di Madian,
\par 18 mentre voi, oggi, siete insorti contro la casa di mio padre, avete ucciso i suoi figliuoli, settanta uomini, sopra una stessa pietra, e avete proclamato re dei Sichemiti Abimelec, figliuolo della sua serva, perché è vostro fratello,
\par 19 se, dico, avete oggi agito con fedeltà e con integrità verso Ierubbaal e la sua casa, godetevi Abimelec, e Abimelec si goda di voi!
\par 20 Se no, esca da Abimelec un fuoco, che divori i Sichemiti e la casa di Millo, ed esca dai Sichemiti e dalla casa di Millo un fuoco, che divori Abimelec!'
\par 21 Poi Jotham corse via, fuggì e andò a stare a Beer, per paura di Abimelec, suo fratello.
\par 22 E Abimelec signoreggiò sopra Israele tre anni.
\par 23 Poi Iddio mandò un cattivo spirito fra Abimelec e i Sichemiti, e i Sichemiti ruppero fede ad Abimelec,
\par 24 affinché la violenza fatta ai settanta figliuoli di Ierubbaal ricevesse il suo castigo, e il loro sangue ricadesse sopra Abimelec, loro fratello, che li aveva uccisi, e sopra i Sichemiti che gli avean prestato mano a uccidere i suoi fratelli.
\par 25 I Sichemiti posero in agguato contro di lui, sulla cima dei monti, della gente che svaligiava sulla strada chiunque le passasse vicino. E Abimelec fu informato della cosa.
\par 26 Poi Gaal, figliuolo di Ebed, e i suoi fratelli vennero e si stabilirono a Sichem, e i Sichemiti riposero in lui la loro fiducia.
\par 27 E, usciti alla campagna, vendemmiarono le loro vigne, pestarono le uve, e fecero festa. Poi entrarono nella casa del loro dio, mangiarono, bevvero, e maledissero Abimelec.
\par 28 E Gaal, figliuolo di Ebed, disse: 'Chi è Abimelec, e che cos'è Sichem, che abbiamo a servire ad Abimelec? non è egli figliuolo di Ierubbaal? e Zebul non è egli suo commissario? Servite agli uomini di Hamor, padre di Sichem! Ma noi perché serviremmo a costui?
\par 29 Ah, se avessi in poter mio questo popolo, io caccerei Abimelec!' Poi disse ad Abimelec: 'Rinforza il tuo esercito e fatti avanti!'
\par 30 Or Zebul, governatore della città, avendo udito le parole di Gaal, figliuolo di Ebed, s'accese d'ira,
\par 31 e mandò segretamente de' messi ad Abimelec per dirgli: 'Ecco, Gaal, figliuolo di Ebed, e i suoi fratelli son venuti a Sichem, e sollevano la città contro di te.
\par 32 Or dunque, lèvati di notte con la gente che è teco, e fa' un'imboscata nella campagna;
\par 33 e domattina, non appena spunterà il sole, ti leverai e piomberai sulla città. E quando Gaal con la gente che è con lui uscirà contro a te, tu gli farai quel che sarà necessario'.
\par 34 Abimelec e tutta la gente ch'era con lui si levaron di notte, e fecero un'imboscata contro a Sichem, divisi in quattro schiere.
\par 35 Intanto Gaal, figliuolo di Ebed, uscì, e si fermò all'ingresso della porta della città; e Abimelec uscì dall'imboscata con la gente ch'era con lui.
\par 36 Gaal, veduta quella gente, disse a Zebul: 'Ecco gente che scende dall'alto de' monti'. E Zebul gli rispose: 'Tu vedi l'ombra de' monti e la prendi per uomini'.
\par 37 E Gaal riprese a dire: 'Guarda, c'è gente che scende dalle alture del paese, e una schiera che giunge per la via della quercia degl'indovini'.
\par 38 Allora Zebul gli disse: 'Dov'è ora la tua millanteria di quando dicevi: - Chi è Abimelec, che abbiamo a servirgli? - Non è questo il popolo che disprezzavi? Orsù, fatti avanti e combatti contro di lui!'
\par 39 Allora Gaal uscì alla testa dei Sichemiti, e diè battaglia ad Abimelec.
\par 40 Ma Abimelec gli diè la caccia, ed egli fuggì d'innanzi a lui, e molti uomini caddero morti fino all'ingresso della porta.
\par 41 E Abimelec si fermò ad Aruma, e Zebul cacciò Gaal e i suoi fratelli, che non poteron più rimanere a Sichem.
\par 42 Il giorno seguente, il popolo di Sichem uscì alla campagna; e Abimelec ne fu informato.
\par 43 Egli prese allora la sua gente, la divise in tre schiere, e fece un'imboscata ne' campi; e quando vide che il popolo usciva dalla città, gli si levò contro e ne fece strage.
\par 44 Poi Abimelec e la gente che avea seco si slanciarono e vennero a porsi all'ingresso della porta della città, mentre le altre due schiere si gettarono su tutti quelli che erano nella campagna, e ne fecero strage.
\par 45 E Abimelec attaccò la città tutto quel giorno, la prese e uccise il popolo che vi si trovava; poi spianò la città e vi seminò del sale.
\par 46 Tutti gli abitanti della torre di Sichem, all'udir questo, si ritirarono nel torrione del tempio di El-Berith.
\par 47 E fu riferito ad Abimelec che tutti gli abitanti della torre di Sichem s'erano adunati quivi.
\par 48 Allora Abimelec salì sul monte Tsalmon con tutta la gente ch'era con lui; diè di piglio ad una scure, tagliò un ramo d'albero, lo sollevò e se lo mise sulla spalla; poi disse alla gente ch'era con lui: 'Quel che m'avete veduto fare fatelo presto anche voi!'
\par 49 Tutti tagliaron quindi anch'essi dei rami, ognuno il suo, e seguitarono Abimelec; posero i rami contro al torrione, e arsero il torrione con quelli che v'eran dentro. Così perì tutta la gente della torre di Sichem, circa mille persone, fra uomini e donne.
\par 50 Poi Abimelec andò a Thebets, la cinse d'assedio e la prese.
\par 51 Or in mezzo alla città c'era una forte torre, dove si rifugiarono tutti gli abitanti della città, uomini e donne; vi si rinchiusero dentro, e salirono sul tetto della torre.
\par 52 Abimelec, giunto alla torre, l'attaccò, e si accostò alla porta della torre per appiccarvi il fuoco.
\par 53 Ma una donna gettò giù un pezzo di macina sulla testa di Abimelec e gli spezzò il cranio.
\par 54 Ed egli chiamò tosto il giovane che gli portava le armi, e gli disse: 'Tira fuori la spada e uccidimi, affinché non si dica: L'ha ammazzato una donna!' Il suo giovane allora lo trafisse, ed egli morì.
\par 55 E quando gl'Israeliti ebbero veduto che Abimelec era morto, se ne andarono, ognuno a casa sua.
\par 56 Così Dio fece ricader sopra Abimelec il male ch'egli avea fatto contro suo padre, uccidendo settanta suoi fratelli.
\par 57 Iddio fece anche ricadere sul capo della gente di Sichem tutto il male ch'essa avea fatto; e su loro si compié la maledizione di Jotham, figliuolo di Ierubbaal.

\chapter{10}

\par 1 Or dopo Abimelec sorse, per liberare Israele, Thola, figliuolo di Puah, figliuolo di Dodo, uomo d'Issacar. Dimorava a Samir, nella contrada montuosa di Efraim;
\par 2 fu giudice d'Israele per ventitre anni; poi morì e fu sepolto a Samir.
\par 3 Dopo di lui sorse Jair, il Galaadita, che fu giudice d'Israele per ventidue anni;
\par 4 ebbe trenta figliuoli che cavalcavano trenta asinelli e aveano trenta città, che si chiamano anche oggi i borghi di Jair, e sono nel paese di Galaad.
\par 5 Poi Jair morì e fu sepolto a Kamon.
\par 6 E i figliuoli d'Israele continuarono a fare ciò ch'è male agli occhi dell'Eterno e servirono agl'idoli di Baal e di Astarte, agli dèi della Siria, agli dèi di Sidon, agli dèi di Moab, agli dèi de' figliuoli di Ammon e agli dèi de' Filistei; abbandonaron l'Eterno e non gli serviron più.
\par 7 L'ira dell'Eterno s'accese contro Israele, ed egli li diede nelle mani de' Filistei e nelle mani de' figliuoli di Ammon.
\par 8 E in quell'anno, questi angariarono ed oppressero i figliuoli d'Israele; per diciotto anni oppressero tutti i figliuoli d'Israele ch'erano di là dal Giordano, nel paese degli Amorei in Galaad.
\par 9 E i figliuoli di Ammon passarono il Giordano per combattere anche contro Giuda, contro Beniamino e contro la casa d'Efraim; e Israele fu in grande angustia.
\par 10 Allora i figliuoli d'Israele gridarono all'Eterno, dicendo: 'Abbiam peccato contro di te, perché abbiamo abbandonato il nostro Dio, e abbiam servito agl'idoli Baal'.
\par 11 E l'Eterno disse ai figliuoli d'Israele: 'Non vi ho io liberati dagli Egiziani, dagli Amorei, dai figliuoli di Ammon e dai Filistei?
\par 12 Quando i Sidonii, gli Amalekiti e i Maoniti vi opprimevano e voi gridaste a me, non vi liberai io dalle loro mani?
\par 13 Eppure, m'avete abbandonato e avete servito ad altri dèi; perciò io non vi libererò più.
\par 14 Andate a gridare agli dèi che avete scelto; vi salvino essi nel tempo della vostra angoscia!'
\par 15 E i figliuoli d'Israele dissero all'Eterno: 'Abbiamo peccato; facci tutto quello che a te piace; soltanto, te ne preghiamo, liberaci oggi!'
\par 16 Allora tolsero di mezzo a loro gli dèi stranieri e servirono all'Eterno, che si accorò per l'afflizione d'Israele.
\par 17 I figliuoli di Ammon s'adunarono e si accamparono in Galaad, e i figliuoli d'Israele s'adunaron pure, e si accamparono a Mitspa.
\par 18 Il popolo, i principi di Galaad, si dissero l'uno all'altro: 'Chi sarà l'uomo che comincerà l'attacco contro i figliuoli di Ammon? Quegli sarà il capo di tutti gli abitanti di Galaad'.

\chapter{11}

\par 1 Or Jefte, il Galaadita, era un uomo forte e valoroso, figliuolo di una meretrice, e avea Galaad per padre.
\par 2 La moglie di Galaad gli avea dato de' figliuoli; e quando questi figliuoli della moglie furono grandi, cacciarono Jefte e gli dissero: 'Tu non avrai eredità in casa di nostro padre, perché sei figliuolo d'un'altra donna'.
\par 3 E Jefte se ne fuggì lungi dai suoi fratelli e si stabilì nel paese di Tob. Degli uomini da nulla si raccolsero attorno a Jefte, e facevano delle incursioni con lui.
\par 4 Qualche tempo dopo avvenne che i figliuoli di Ammon mossero guerra a Israele.
\par 5 E come i figliuoli di Ammon movean guerra a Israele, gli anziani di Galaad andarono a cercare Jefte nel paese di Tob.
\par 6 E dissero a Jefte: 'Vieni, sii nostro capitano, e combatteremo contro i figliuoli di Ammon'.
\par 7 Ma Jefte rispose agli anziani di Galaad: 'Non m'avete voi odiato e cacciato dalla casa di mio padre? Perché venite da me ora che siete nell'angustia?'
\par 8 E gli anziani di Galaad dissero a Jefte: 'Appunto per questo torniamo ora da te, onde tu venga con noi e combatta contro i figliuoli di Ammon, e tu sia capo di noi tutti abitanti di Galaad'.
\par 9 Jefte rispose agli anziani di Galaad: 'Se mi riconducete da voi per combattere contro i figliuoli di Ammon, e l'Eterno li dà in mio potere, io sarò vostro capo'.
\par 10 E gli anziani di Galaad dissero a Jefte: 'L'Eterno sia testimone tra noi, e ci punisca se non facciamo quello che hai detto'.
\par 11 Jefte dunque andò con gli anziani di Galaad; il popolo lo costituì suo capo e condottiero, e Jefte ripeté davanti all'Eterno, a Mitspa, tutte le parole che avea detto prima.
\par 12 Poi Jefte inviò de' messi al re de' figliuoli di Ammon per dirgli: 'Che questione c'è fra me e te che tu venga contro di me per far guerra al mio paese?'
\par 13 E il re de' figliuoli di Ammon rispose ai messi di Jefte: 'Mi son mosso perché, quando Israele salì dall'Egitto, s'impadronì del mio paese, dall'Arnon fino allo Jabbok e al Giordano; rendimelo all'amichevole'.
\par 14 Jefte inviò di nuovo de' messi al re de' figliuoli di Ammon per dirgli:
\par 15 'Così dice Jefte: Israele non s'impadronì del paese di Moab, né del paese de' figliuoli di Ammon;
\par 16 ma, quando Israele salì dall'Egitto e attraversò il deserto fino al mar Rosso e giunse a Kades,
\par 17 inviò de' messi al re di Edom per dirgli: - Ti prego, lasciami passare per il tuo paese; - ma il re di Edom non acconsentì. Mandò anche al re di Moab, il quale pure rifiutò; e Israele rimase a Kades.
\par 18 Poi camminò per il deserto, fece il giro del paese di Edom e del paese di Moab, giunse a oriente del paese di Moab, e si accampò di là dall'Arnon, senza entrare nel territorio di Moab; perché l'Arnon segna il confine di Moab.
\par 19 E Israele inviò dei messi a Sihon, re degli Amorei, re di Heshbon, e gli fe' dire: - Ti preghiamo, lasciaci passare dal tuo paese, per arrivare al nostro. -
\par 20 Ma Sihon non si fidò d'Israele per permettergli di passare per il suo territorio; anzi Sihon radunò tutta la sua gente, s'accampò a Jahats, e combatté contro Israele.
\par 21 E l'Eterno, l'Iddio d'Israele, diede Sihon e tutta la sua gente nelle mani d'Israele, che li sconfisse; così Israele conquistò tutto il paese degli Amorei, che abitavano quella contrada;
\par 22 conquistò tutto il territorio degli Amorei, dall'Arnon allo Jabbok e dal deserto al Giordano.
\par 23 E ora che l'Eterno, l'Iddio d'Israele, ha cacciato gli Amorei d'innanzi a Israele, ch'è il suo popolo, dovresti tu possedere il loro paese?
\par 24 Non possiedi tu quello che Kemosh, il tuo dio, t'ha fatto possedere? Così anche noi possederemo il paese di quelli che l'Eterno ha cacciati d'innanzi a noi.
\par 25 Sei tu forse da più di Balak, figliuolo di Tsippor, re di Moab? Mosse egli querela ad Israele, o gli fece egli guerra?
\par 26 Son trecent'anni che Israele abita ad Heshbon e nelle città del suo territorio, ad Aroer e nelle città del suo territorio, e in tutte le città lungo l'Arnon; perché non gliele avete tolte durante questo tempo?
\par 27 E io non t'ho offeso, e tu agisci male verso di me, movendomi guerra. L'Eterno, il giudice, giudichi oggi tra i figliuoli d'Israele e i figliuoli di Ammon!'
\par 28 Ma il re de' figliuoli di Ammon non diede ascolto alle parole che Jefte gli avea fatto dire.
\par 29 Allora lo spirito dell'Eterno venne su Jefte, che attraversò Galaad e Manasse, passò a Mitspa di Galaad, e da Mitspa di Galaad mosse contro i figliuoli di Ammon.
\par 30 E Jefte fece un voto all'Eterno, e disse: 'Se tu mi dai nelle mani i figliuoli di Ammon,
\par 31 la persona che uscirà dalle porte di casa mia per venirmi incontro quando tornerò vittorioso dai figliuoli di Ammon, sarà dell'Eterno, e io l'offrirò in olocausto'.
\par 32 E Jefte marciò contro i figliuoli di Ammon per far loro guerra, e l'Eterno glieli diede nelle mani.
\par 33 Ed egli inflisse loro una grandissima sconfitta, da Aroer fin verso Minnith, prendendo loro venti città, e fino ad Abel-Keramim. Così i figliuoli di Ammon furono umiliati dinanzi ai figliuoli d'Israele.
\par 34 Or Jefte se ne tornò a Mitspa, a casa sua; ed ecco uscirgli incontro la sua figliuola, con timpani e danze. Era l'unica sua figlia: non aveva altri figliuoli né altre figliuole.
\par 35 E, come la vide, si stracciò le vesti, e disse: 'Ah, figlia mia! tu mi accasci, tu mi accasci; tu sei fra quelli che mi conturbano! poiché io ho dato parola all'Eterno, e non posso ritrarmene'.
\par 36 Ella gli disse: 'Padre mio, se hai dato parola all'Eterno, fa' di me secondo quel che hai proferito, giacché l'Eterno t'ha dato di far vendetta de' figliuoli di Ammon, tuoi nemici'.
\par 37 Poi disse a suo padre: 'Mi sia concesso questo: lasciami libera per due mesi, ond'io vada e scenda per i monti a piangere la mia verginità con le mie compagne'.
\par 38 Egli le rispose: 'Va'!' e la lasciò andare per due mesi. Ed ella se ne andò con le sue compagne, e pianse sui monti la sua verginità.
\par 39 Alla fine dei due mesi, ella tornò da suo padre; ed egli fece di lei quello che aveva promesso con voto. Ella non avea conosciuto uomo. Di qui venne in Israele
\par 40 l'usanza che le figliuole d'Israele vanno tutti gli anni a celebrar la figliuola di Jefte, il Galaadita, per quattro giorni.

\chapter{12}

\par 1 Or gli uomini di Efraim si radunarono, passarono a Tsafon, e dissero a Jefte: 'Perché sei andato a combattere contro i figliuoli di Ammon e non ci hai chiamati ad andar teco? Noi bruceremo la tua casa e te con essa'.
\par 2 Jefte rispose loro: 'Io e il mio popolo abbiamo avuto gran contesa coi figliuoli di Ammon; e quando v'ho chiamati in aiuto, non mi avete liberato dalle loro mani.
\par 3 E vedendo che voi non venivate in mio soccorso, ho posto a repentaglio la mia vita, ho marciato contro i figliuoli di Ammon, e l'Eterno me li ha dati nelle mani. Perché dunque siete saliti oggi contro di me per muovermi guerra?'
\par 4 Poi Jefte, radunati tutti gli uomini di Galaad, diè battaglia ad Efraim; e gli uomini di Galaad sconfissero gli Efraimiti, perché questi dicevano: 'Voi, Galaaditi, siete de' fuggiaschi d'Efraim, in mezzo ad Efraim e in mezzo a Manasse!'
\par 5 E i Galaaditi intercettarono i guadi del Giordano agli Efraimiti; e quando uno de' fuggiaschi d'Efraim diceva: 'Lasciatemi passare', gli uomini di Galaad gli chiedevano: 'Sei tu un Efraimita?' Se quello rispondeva: 'No', i Galaaditi gli dicevano:
\par 6 'Ebbene, di' Scibboleth'; e quello diceva 'Sibboleth', senza fare attenzione a pronunziar bene; allora lo pigliavano e lo scannavano presso i guadi del Giordano. E perirono in quel tempo quarantaduemila uomini d'Efraim.
\par 7 Jefte fu giudice d'Israele per sei anni. Poi Jefte, il Galaadita, morì e fu sepolto in una delle città di Galaad.
\par 8 Dopo di lui fu giudice d'Israele Ibtsan di Bethlehem,
\par 9 che ebbe trenta figliuoli, maritò fuori trenta figliuole, e condusse di fuori trenta fanciulle per i suoi figliuoli. Fu giudice d'Israele per sette anni.
\par 10 Poi Ibtsan morì e fu sepolto a Bethlehem.
\par 11 Dopo di lui fu giudice d'Israele Elon, lo Zabulonita; fu giudice d'Israele per dieci anni.
\par 12 Poi Elon, lo Zabulonita, morì e fu sepolto ad Aialon, nel paese di Zabulon.
\par 13 Dopo di lui fu giudice d'Israele Abdon, figliuolo di Hillel, il Pirathonita.
\par 14 Ebbe quaranta figliuoli e trenta nipoti, i quali cavalcarono settanta asinelli. Fu giudice d'Israele per otto anni.
\par 15 Poi Abdon, figliuolo di Hillel, il Pirathonita, morì e fu sepolto a Pirathon, nel paese di Efraim, sul monte Amalek.

\chapter{13}

\par 1 E i figliuoli d'Israele continuarono a fare quel ch'era male agli occhi dell'Eterno, e l'Eterno li diede nelle mani de' Filistei per quarant'anni.
\par 2 Or v'era un uomo di Tsorea, della famiglia dei Daniti, per nome Manoah; sua moglie era sterile e non avea figliuoli.
\par 3 E l'angelo dell'Eterno apparve a questa donna, e le disse: 'Ecco, tu sei sterile e non hai figliuoli; ma concepirai e partorirai un figliuolo.
\par 4 Or dunque, guardati bene dal bere vino o bevanda alcoolica, e dal mangiare alcun che d'impuro.
\par 5 Poiché ecco, tu concepirai e partorirai un figliuolo, sulla testa del quale non passerà rasoio, giacché il fanciullo sarà un Nazireo consacrato a Dio dal seno di sua madre, e sarà lui che comincerà a liberare Israele dalle mani de' Filistei'.
\par 6 E la donna andò a dire a suo marito: 'Un uomo di Dio è venuto da me; avea il sembiante d'un angelo di Dio: un sembiante terribile fuor di modo. Io non gli ho domandato donde fosse, ed egli non m'ha detto il suo nome;
\par 7 ma mi ha detto: Ecco, tu concepirai e partorirai un figliuolo; or dunque non bere vino né bevanda alcoolica, e non mangiare alcun che d'impuro, giacché il fanciullo sarà un Nazireo, consacrato a Dio dal seno di sua madre e fino al giorno della sua morte'.
\par 8 Allora Manoah supplicò l'Eterno, e disse: 'O Signore, ti prego che l'uomo di Dio mandato da te torni di nuovo a noi e c'insegni quello che dobbiam fare per il bambino che nascerà'.
\par 9 E Dio esaudì la preghiera di Manoah; e l'angelo di Dio tornò ancora dalla donna, che stava sedendo nel campo; ma Manoah, suo marito, non era con lei.
\par 10 La donna corse in fretta a informar suo marito del fatto, e gli disse: 'Ecco, quell'uomo che venne da me l'altro giorno, m'è apparito'.
\par 11 Manoah s'alzò, andò dietro a sua moglie, e giunto a quell'uomo, gli disse: 'Sei tu che parlasti a questa donna?' E quegli rispose: 'Son io'.
\par 12 E Manoah: 'Quando la tua parola si sarà verificata, qual norma s'avrà da seguire per il bambino? e che si dovrà fare per lui?'
\par 13 L'angelo dell'Eterno rispose a Manoah: 'Si astenga la donna da tutto quello che le ho detto.
\par 14 Non mangi di alcun prodotto della vigna, né beva vino o bevanda alcoolica, e non mangi alcun che d'impuro; osservi tutto quello che le ho comandato'.
\par 15 E Manoah disse all'angelo dell'Eterno: 'Deh, permettici di trattenerti, e di prepararti un capretto!'
\par 16 E l'angelo dell'Eterno rispose a Manoah: 'Anche se tu mi trattenessi, non mangerei del tuo cibo; ma, se vuoi fare un olocausto, offrilo all'Eterno'. Or Manoah non sapeva che quello fosse l'angelo dell'Eterno.
\par 17 Poi Manoah disse all'angelo dell'Eterno: 'Qual è il tuo nome, affinché, adempiute che siano le tue parole, noi ti rendiamo onore?'
\par 18 E l'angelo dell'Eterno gli rispose: 'Perché mi chiedi il mio nome? esso è maraviglioso'.
\par 19 E Manoah prese il capretto e l'oblazione e li offrì all'Eterno sul sasso. Allora avvenne una cosa prodigiosa, mentre Manoah e sua moglie stavano guardando:
\par 20 come la fiamma saliva dall'altare al cielo, l'angelo dell'Eterno salì con la fiamma dell'altare. E Manoah e sua moglie, vedendo questo, caddero con la faccia a terra.
\par 21 E l'angelo dell'Eterno non apparve più né a Manoah né a sua moglie. Allora Manoah riconobbe che quello era l'angelo dell'Eterno.
\par 22 E Manoah disse a sua moglie: 'Noi morremo sicuramente, perché abbiam veduto Dio'.
\par 23 Ma sua moglie gli disse: 'Se l'Eterno avesse voluto farci morire, non avrebbe accettato dalle nostre mani l'olocausto e l'oblazione; non ci avrebbe fatto vedere tutte queste cose, e non ci avrebbe fatto udire proprio ora delle cose come queste'.
\par 24 Poi la donna partorì un figliuolo, a cui pose nome Sansone. Il bambino crebbe, e l'Eterno lo benedisse.
\par 25 E lo spirito dell'Eterno cominciò ad agitarlo quand'esso era a Mahaneh-Dan, fra Tsorea ed Eshtaol.

\chapter{14}

\par 1 Sansone scese a Timnah, e vide quivi una donna tra le figliuole de' Filistei.
\par 2 Tornato a casa, ne parlò a suo padre e a sua madre, dicendo: 'Ho veduto a Timnah una donna tra le figliuole de' Filistei; or dunque, prendetemela per moglie'.
\par 3 Suo padre e sua madre gli dissero: 'Non v'è egli dunque tra le figliuole de' tuoi fratelli e in tutto il nostro popolo una donna per te, che tu vada a prenderti una moglie tra i Filistei incirconcisi?' E Sansone rispose a suo padre: 'Prendimi quella, poiché mi piace'.
\par 4 Or suo padre e sua madre non sapevano che questo veniva dall'Eterno, poiché Sansone cercava che i Filistei gli fornissero un'occasione di contesa. In quel tempo, i Filistei dominavano Israele.
\par 5 Poi Sansone scese con suo padre e con sua madre a Timnah; e come furon giunti alle vigne di Timnah, ecco un leoncello farglisi incontro, ruggendo.
\par 6 Lo spirito dell'Eterno investì Sansone, che, senz'aver niente in mano, squarciò il leone, come uno squarcerebbe un capretto; ma non disse nulla a suo padre né a sua madre di ciò che avea fatto.
\par 7 E scese, parlò alla donna, e questa gli piacque.
\par 8 Di lì a qualche tempo, tornò per prenderla, e uscì di strada per vedere il carcame del leone; ed ecco, nel corpo del leone c'era uno sciame d'api e del miele.
\par 9 Egli prese in mano di quel miele, e si mise a mangiarlo per istrada; e quando ebbe raggiunto suo padre e sua madre, ne diede loro, ed essi ne mangiarono; ma non disse loro che avea preso il miele dal corpo del leone.
\par 10 Suo padre scese a trovar quella donna, e Sansone fece quivi un convito; perché tale era il costume dei giovani.
\par 11 Non appena i parenti della sposa videro Sansone, invitarono trenta compagni perché stessero con lui.
\par 12 Sansone disse loro: 'Io vi proporrò un enimma; e se voi me lo spiegate entro i sette giorni del convito, e se l'indovinate, vi darò trenta tuniche e trenta mute di vesti;
\par 13 ma, se non me lo potete spiegare, darete trenta tuniche e trenta mute di vesti a me'.
\par 14 E quelli gli risposero: 'Proponi il tuo enimma, e noi l'udremo'. Ed egli disse loro: 'Dal mangiatore è uscito del cibo, e dal forte è uscito del dolce'. Per tre giorni quelli non poterono spiegar l'enimma.
\par 15 E il settimo giorno dissero alla moglie di Sansone: 'Induci il tuo marito a spiegarci l'enimma; se no, darem fuoco a te e alla casa di tuo padre. E che? ci avete invitati qui per spogliarci?'
\par 16 La moglie di Sansone si mise a piangere presso di lui, e a dirgli: 'Tu non hai per me che dell'odio, e non mi vuoi bene; hai proposto un enimma ai figliuoli del mio popolo, e non me l'hai spiegato!' Ed egli a lei: 'Ecco, non l'ho spiegato a mio padre né a mia madre, e lo spiegherei a te?'
\par 17 Ed ella pianse presso di lui, durante i sette giorni che durava il convito; e il settimo giorno Sansone glielo spiegò, perché lo tormentava; ed essa spiegò l'enimma ai figliuoli del suo popolo.
\par 18 E gli uomini della città, il settimo giorno, prima che tramontasse il sole, dissero a Sansone: 'Che v'è di più dolce del miele? e che v'è di più forte del leone?' Ed egli rispose loro: 'Se non aveste arato con la mia giovenca, non avreste indovinato il mio enimma'.
\par 19 E lo spirito dell'Eterno lo investì, ed egli scese ad Askalon, vi uccise trenta uomini dei loro, prese le loro spoglie, e dette le mute di vesti a quelli che aveano spiegato l'enimma. E, acceso d'ira, risalì a casa di suo padre.
\par 20 Ma la moglie di Sansone fu data al compagno di lui, ch'ei s'era scelto per amico.

\chapter{15}

\par 1 Di lì a qualche tempo, verso la mietitura del grano, Sansone andò a visitare sua moglie, le portò un capretto, e disse: 'Voglio entrare in camera da mia moglie'. Ma il padre di lei non gli permise d'entrare,
\par 2 e gli disse: 'Io credevo sicuramente che tu l'avessi presa in odio, e però l'ho data al tuo compagno; la sua sorella minore non è più bella di lei? Prendila dunque in sua vece'.
\par 3 Sansone rispose loro: 'Questa volta, non avrò colpa verso i Filistei, quando farò loro del male'.
\par 4 E Sansone se ne andò e acchiappò trecento sciacalli; prese pure delle fiaccole, vòlse coda contro coda, e mise una fiaccola in mezzo, fra le due code.
\par 5 Poi accese le fiaccole, dette la via agli sciacalli per i campi di grano de' Filistei, e bruciò i covoni ammassati, il grano tuttora in piedi, e perfino gli uliveti.
\par 6 E i Filistei chiesero: 'Chi ha fatto questo?' Fu risposto: 'Sansone, il genero del Thimneo, perché questi gli ha preso la moglie, e l'ha data al compagno di lui'. E i Filistei salirono e diedero alle fiamme lei e suo padre.
\par 7 E Sansone disse loro: 'Giacché agite a questo modo, siate certi che non avrò posa finché non mi sia vendicato di voi'.
\par 8 E li sbaragliò interamente, facendone un gran macello. Poi discese, e si ritirò nella caverna della roccia d'Etam.
\par 9 Allora i Filistei salirono, si accamparono in Giuda, e si distesero fino a Lehi.
\par 10 Gli uomini di Giuda dissero loro: 'Perché siete saliti contro di noi?' Quelli risposero: 'Siam saliti per legare Sansone; per fare a lui quello che ha fatto a noi'.
\par 11 E tremila uomini di Giuda scesero alla caverna della roccia d'Etam, e dissero a Sansone: 'Non sai tu che i Filistei sono nostri dominatori? Che è dunque questo che ci hai fatto?' Ed egli rispose loro: 'Quello che hanno fatto a me, l'ho fatto a loro'.
\par 12 E quelli a lui: 'Noi siam discesi per legarti e darti nelle mani de' Filistei'. Sansone replicò loro: 'Giuratemi che voi stessi non mi ucciderete'.
\par 13 Quelli risposero: 'No, ti legheremo soltanto, e ti daremo nelle loro mani; ma certamente non ti metteremo a morte'. E lo legarono con due funi nuove, e lo fecero uscire dalla caverna.
\par 14 Quando giunse a Lehi, i Filistei gli si fecero incontro con grida di gioia; ma lo spirito dell'Eterno lo investì, e le funi che aveva alle braccia divennero come fili di lino a cui si appicchi il fuoco; e i legami gli caddero dalle mani.
\par 15 E, trovata una mascella d'asino ancor fresca, stese la mano, l'afferrò, e uccise con essa mille uomini.
\par 16 E Sansone disse: 'Con una mascella d'asino, un mucchio! due mucchi! Con una mascella d'asino ho ucciso mille uomini!'
\par 17 Quand'ebbe finito di parlare, gettò via di mano la mascella, e chiamò quel luogo Ramath-Lehi.
\par 18 Poi ebbe gran sete; e invocò l'Eterno, dicendo: 'Tu hai concesso questa gran liberazione per mano del tuo servo; e ora, dovrò io morir di sete e cader nelle mani degli incirconcisi?'
\par 19 Allora Iddio fendé la roccia concava ch'è a Lehi, e ne uscì dell'acqua. Sansone bevve, il suo spirito si rianimò, ed egli riprese vita. Donde il nome di En-Hakkore dato a quella fonte, che esiste anche al dì d'oggi a Lehi.
\par 20 Sansone fu giudice d'Israele, al tempo de' Filistei, per vent'anni.

\chapter{16}

\par 1 E Sansone andò a Gaza, vide quivi una meretrice, ed entrò da lei.
\par 2 Fu detto a que' di Gaza: 'Sansone è venuto qua'. Ed essi lo circondarono, stettero in agguato tutta la notte presso la porta della città, e tutta quella notte se ne stettero queti dicendo: 'Allo spuntar del giorno l'uccideremo'.
\par 3 E Sansone si giacque fino a mezzanotte; e a mezzanotte si levò, diè di piglio ai battenti della porta della città e ai due stipiti, li divelse insieme con la sbarra, se li mise sulle spalle, e li portò in cima al monte ch'è dirimpetto a Hebron.
\par 4 Dopo questo, s'innamorò di una donna della valle di Sorek, che si chiamava Delila.
\par 5 E i principi de' Filistei salirono da lei e le dissero: 'Lusingalo, e vedi dove risieda quella sua gran forza, e come potremmo prevalere contro di lui per giungere a legarlo e a domarlo; e ti daremo ciascuno mille e cento sicli d'argento'.
\par 6 Delila dunque disse a Sansone: 'Dimmi, ti prego, dove risieda la tua gran forza, e in che modo ti si potrebbe legare per domarti'.
\par 7 Sansone le rispose: 'Se mi si legasse con sette corde d'arco fresche, non ancora secche, io diventerei debole e sarei come un uomo qualunque'.
\par 8 Allora i principi dei Filistei le portarono sette corde d'arco fresche, non ancora secche, ed ella lo legò con esse.
\par 9 Or c'era gente che stava in agguato, da lei, in una camera interna. Ed ella gli disse: 'Sansone, i Filistei ti sono addosso!' Ed egli ruppe le corde, come si rompe un fil di stoppa quando sente il fuoco. Così il segreto della sua forza restò sconosciuto.
\par 10 Poi Delila disse a Sansone: 'Ecco, tu m'hai beffata e m'hai detto delle bugie; or dunque, ti prego, dimmi con che ti si potrebbe legare'.
\par 11 Egli le rispose: 'Se mi si legasse con funi nuove che non fossero ancora state adoperate, io diventerei debole e sarei come un uomo qualunque'.
\par 12 Delila prese dunque delle funi nuove, lo legò, e gli disse: 'Sansone, i Filistei ti sono addosso'. L'agguato era posto nella camera interna. Ed egli ruppe, come un filo, le funi che aveva alle braccia.
\par 13 Delila disse a Sansone: 'Fino ad ora tu m'hai beffata e m'hai detto delle bugie; dimmi con che ti si potrebbe legare'. Ed egli le rispose: 'Non avresti che da tessere le sette trecce del mio capo col tuo ordìto'.
\par 14 Essa le fissò al subbio, poi gli disse: 'Sansone, i Filistei ti sono addosso'. Ma egli si svegliò dal sonno, e strappò via il subbio del telaio con l'ordìto.
\par 15 Ed ella gli disse: 'Come fai a dirmi: T'amo! mentre il tuo cuore non è con me? Già tre volte m'hai beffata, e non m'hai detto dove risiede la tua gran forza'.
\par 16 Or avvenne che, premendolo ella ogni giorno con le sue parole e tormentandolo, egli se ne accorò mortalmente,
\par 17 e le aperse tutto il cuor suo e le disse: 'Non è mai passato rasoio sulla mia testa, perché sono un Nazireo, consacrato a Dio, dal seno di mia madre; se fossi tosato, la mia forza se ne andrebbe, diventerei debole, e sarei come un uomo qualunque'.
\par 18 Delila, visto ch'egli le aveva aperto tutto il cuor suo, mandò a chiamare i principi de' Filistei, e fece dir loro: 'Venite su, questa volta, perché egli m'ha aperto tutto il suo cuore'. Allora i principi dei Filistei salirono da lei, e portaron seco il danaro.
\par 19 Ed ella lo addormentò sulle sue ginocchia, chiamò l'uomo fissato, e gli fece tosare le sette trecce della testa di Sansone; così giunse a domarlo; e la sua forza si partì da lui.
\par 20 Allora ella gli disse: 'Sansone, i Filistei ti sono addosso'. Ed egli, svegliatosi dal sonno, disse: 'Io ne uscirò come le altre volte, e mi svincolerò'. Ma non sapeva che l'Eterno s'era ritirato da lui.
\par 21 E i Filistei lo presero e gli cavaron gli occhi; lo fecero scendere a Gaza, e lo legarono con catene di rame. Ed egli girava la macina nella prigione.
\par 22 Intanto, la capigliatura che gli avean tosata, cominciava a ricrescergli.
\par 23 Or i principi dei Filistei si radunarono per offrire un gran sacrifizio a Dagon, loro dio, e per rallegrarsi. Dicevano: 'Il nostro dio ci ha dato nelle mani Sansone, nostro nemico'.
\par 24 E quando il popolo lo vide, cominciò a lodare il suo dio e a dire: 'Il nostro dio ci ha dato nelle mani il nostro nemico, colui che ci devastava il paese e che ha ucciso tanti di noi'.
\par 25 E nella gioia del cuor loro, dissero: 'Chiamate Sansone, che ci faccia divertire!' Fecero quindi uscir Sansone dalla prigione, ed egli si mise a fare il buffone in loro presenza. Lo posero fra le colonne;
\par 26 e Sansone disse al fanciullo che lo teneva per la mano: 'Lasciami, ch'io possa toccar le colonne sulle quali posa la casa, e m'appoggi ad esse'.
\par 27 Or la casa era piena d'uomini e di donne; e tutti i principi de' Filistei eran quivi; c'eran sul tetto circa tremila persone, fra uomini e donne, che stavano a guardare mentre Sansone faceva il buffone.
\par 28 Allora Sansone invocò l'Eterno, e disse: 'O Signore, o Eterno, ti prego, ricordati di me! Dammi forza per questa volta soltanto, o Dio, perch'io mi vendichi in un colpo solo de' Filistei, per la perdita de' miei due occhi'.
\par 29 E Sansone abbracciò le due colonne di mezzo, sulle quali posava la casa; s'appoggiò ad esse: all'una con la destra, all'altra con la sinistra, e disse:
\par 30 'Ch'io muoia insieme coi Filistei!' Si curvò con tutta la sua forza, e la casa rovinò addosso ai principi e a tutto il popolo che v'era dentro; talché più ne uccise egli morendo, che non ne avea uccisi da vivo.
\par 31 Poi i suoi fratelli e tutta la casa di suo padre scesero e lo portaron via; quindi risalirono, e lo seppellirono fra Tsorea ed Eshtaol nel sepolcro di Manoah suo padre. Egli era stato giudice d'Israele per venti anni.

\chapter{17}

\par 1 Or v'era un uomo nella contrada montuosa d'Efraim, che si chiamava Mica.
\par 2 Egli disse a sua madre: 'I millecento sicli d'argento che t'hanno rubato, e a proposito dei quali hai pronunziato una maledizione, e l'hai pronunziata in mia presenza, ecco, li ho io; quel denaro l'avevo preso io'. E sua madre disse: 'Benedetto sia dall'Eterno il mio figliuolo!'
\par 3 Egli restituì a sua madre i millecento sicli d'argento, e sua madre disse: 'Io consacro di mano mia quest'argento a pro del mio figliuolo, per farne un'immagine scolpita e un'immagine di getto; or dunque te lo rendo'.
\par 4 E quand'egli ebbe restituito l'argento a sua madre, questa prese dugento sicli e li diede al fonditore, il quale ne fece un'immagine scolpita e un'immagine di getto, che furon messe in casa di Mica.
\par 5 E quest'uomo, Mica, ebbe una casa di Dio; e fece un efod e degl'idoli, e consacrò uno de' suoi figliuoli, che li servì da sacerdote.
\par 6 In quel tempo non v'era re in Israele; ognuno faceva quel che gli pareva meglio.
\par 7 Or v'era un giovine di Bethlehem di Giuda, della famiglia di Giuda, il quale era un Levita, e abitava quivi.
\par 8 Quest'uomo si partì dalla città di Bethlehem di Giuda, per stabilirsi in luogo che trovasse adatto; e, cammin facendo, giunse nella contrada montuosa di Efraim, alla casa di Mica.
\par 9 Mica gli chiese: 'Donde vieni?' Quello gli rispose: 'Sono un Levita di Bethlehem di Giuda, e vado a stabilirmi dove troverò un luogo adatto'.
\par 10 Mica gli disse: 'Rimani con me, e siimi padre e sacerdote; ti darò dieci sicli d'argento all'anno, un vestito completo, e il vitto'. E il Levita entrò.
\par 11 Egli acconsentì a stare con quell'uomo, che trattò il giovine come uno de' suoi figliuoli.
\par 12 Mica consacrò quel Levita; il giovine gli servì da sacerdote, e si stabilì in casa di lui.
\par 13 E Mica disse: 'Ora so che l'Eterno mi farà del bene, perché ho un Levita come mio sacerdote'.

\chapter{18}

\par 1 In quel tempo, non v'era re in Israele; e in quel medesimo tempo, la tribù dei Daniti cercava un possesso per stabilirvisi, perché fino a quei giorni, non le era toccato alcuna eredità fra le tribù d'Israele.
\par 2 I figliuoli di Dan mandaron dunque da Tsorea e da Eshtaol cinque uomini della loro tribù, presi di fra loro tutti, uomini valorosi, per esplorare ed esaminare il paese; e dissero loro: 'Andate a esaminare il paese!' Quelli giunsero nella contrada montuosa di Efraim, alla casa di Mica, e pernottarono in quel luogo.
\par 3 Come furon presso alla casa di Mica, riconobbero la voce del giovine Levita; e, avvicinatisi, gli chiesero: 'Chi t'ha condotto qua? che fai in questo luogo? che hai tu qui?'
\par 4 Egli rispose loro: 'Mica mi ha fatto questo e questo: mi stipendia, e io gli servo da sacerdote'.
\par 5 E quelli gli dissero: 'Deh, consulta Iddio, affinché sappiamo se il viaggio che abbiamo intrapreso sarà prospero'.
\par 6 Il sacerdote rispose loro: 'Andate in pace; il viaggio che fate è sotto lo sguardo dell'Eterno'.
\par 7 I cinque uomini dunque partirono, giunsero a Lais, e videro che il popolo, il quale vi abitava, viveva in sicurtà, al modo de' Sidonii, tranquillo e fidente, poiché nel paese non c'era alcuno in autorità che potesse far loro il menomo torto, ed erano lontani dai Sidonii e non aveano relazione con alcuno.
\par 8 Poi tornarono ai loro fratelli a Tsorea ed a Eshtaol; e i fratelli chiesero loro: 'Che dite?'
\par 9 Quelli risposero: 'Leviamoci e saliamo contro quella gente; poiché abbiam visto il paese, ed ecco, è eccellente. E voi ve ne state là senza dir verbo? Non siate pigri a muovervi per andare a prender possesso del paese!
\par 10 Quando arriverete là troverete un popolo che se ne sta sicuro. Il paese è vasto, e Dio ve lo ha dato nelle mani: è un luogo dove non manca nulla di ciò che è sulla terra'.
\par 11 E seicento uomini della famiglia dei Daniti partirono da Tsorea e da Eshtaol, muniti d'armi.
\par 12 Salirono, e si accamparono a Kiriath-Jearim, in Giuda; perciò quel luogo, che è dietro a Kiriath-Jearim, fu chiamato e si chiama anche oggi Mahané-Dan.
\par 13 E di là passarono nella contrada montuosa di Efraim, e giunsero alla casa di Mica.
\par 14 Allora i cinque uomini che erano andati ad esplorare il paese di Lais, presero a dire ai loro fratelli: 'Sapete voi che in queste case c'è un efod, ci son degl'idoli, un'immagine scolpita e un'immagine di getto? Considerate ora quel che dovete fare'.
\par 15 Quelli si diressero da quella parte, giunsero alla casa del giovane Levita, alla casa di Mica, e gli chiesero notizie del suo bene stare.
\par 16 I seicento uomini de' figliuoli di Dan, muniti delle loro armi, si misero davanti alla porta.
\par 17 Ma i cinque uomini ch'erano andati ad esplorare il paese, salirono, entrarono in casa, presero l'immagine scolpita, l'efod, gl'idoli e l'immagine di getto, mentre il sacerdote stava davanti alla porta coi seicento uomini armati.
\par 18 E quando furono entrati in casa di Mica ed ebbero preso l'immagine scolpita, l'efod, gl'idoli e l'immagine di getto, il sacerdote disse loro: 'Che fate?'
\par 19 Quelli gli risposero: 'Taci, mettiti la mano sulla bocca, vieni con noi, e sarai per noi un padre e un sacerdote. Che è meglio per te, esser sacerdote in casa d'un uomo solo, ovvero esser sacerdote di una tribù e d'una famiglia in Israele?'
\par 20 Il sacerdote si rallegrò in cuor suo; prese l'efod, gl'idoli e l'immagine scolpita, e s'unì a quella gente.
\par 21 Così si rimisero in cammino, mettendo innanzi a loro i bambini, il bestiame e i bagagli.
\par 22 Com'erano già lungi dalla casa di Mica, la gente che abitava nelle case vicine a quella di Mica, si radunò e inseguì i figliuoli di Dan.
\par 23 E siccome gridava dietro ai figliuoli di Dan, questi, rivoltatisi indietro, dissero a Mica: 'Che cosa hai, che hai radunata cotesta gente?'
\par 24 Egli rispose: 'Avete portato via gli dèi che m'ero fatti e il sacerdote, e ve ne siete andati. Or che mi resta egli più? Come potete dunque dirmi: Che hai?'
\par 25 I figliuoli di Dan gli dissero: 'Fa' che non s'oda la tua voce dietro a noi, perché degli uomini irritati potrebbero scagliarsi su voi, e tu ci perderesti la vita tua e quella della tua famiglia!'
\par 26 I figliuoli di Dan continuarono il loro viaggio; e Mica, vedendo ch'essi eran più forti di lui se ne tornò indietro e venne a casa sua.
\par 27 Ed essi, dopo aver preso le cose che Mica avea fatte e il sacerdote che aveva al suo servizio, giunsero a Lais, a un popolo che se ne stava tranquillo e in sicurtà; lo passarono a fil di spada, e dettero la città alle fiamme.
\par 28 E non ci fu alcuno che la liberasse, perch'era lontana da Sidon, e i suoi abitanti non avean relazioni con altra gente. Essa era nella valle che si estende verso Beth-Rehob.
\par 29 Poi i Daniti ricostruirono la città e l'abitarono. E le posero nome Dan, dal nome di Dan loro padre, che fu figliuolo d'Israele; ma prima, il nome della città era Lais.
\par 30 Poi i figliuoli di Dan rizzarono per sé l'immagine scolpita; e Gionathan, figliuolo di Ghershom, figliuolo di Mosè, e i suoi figliuoli furono sacerdoti della tribù dei Daniti fino al giorno in cui gli abitanti del paese furon deportati.
\par 31 Così rizzarono per sé l'immagine scolpita che Mica avea fatta, durante tutto il tempo che la casa di Dio rimase a Sciloh.

\chapter{19}

\par 1 Or in quel tempo non v'era re in Israele; ed avvenne che un Levita, il quale dimorava nella parte più remota della contrada montuosa di Efraim, si prese per concubina una donna di Bethlehem di Giuda.
\par 2 Questa sua concubina gli fu infedele, e lo lasciò per andarsene a casa di suo padre a Bethlehem di Giuda, ove stette per lo spazio di quattro mesi.
\par 3 E suo marito si levò e andò da lei per parlare al suo cuore e ricondurla seco. Egli aveva preso con sé il suo servo e due asini. Essa lo menò in casa di suo padre; e come il padre della giovane lo vide, gli si fece incontro festosamente.
\par 4 Il suo suocero, il padre della giovane, lo trattenne, ed egli rimase con lui tre giorni; e mangiarono e bevvero e pernottarono quivi.
\par 5 Il quarto giorno si levarono di buon'ora, e il Levita si disponeva a partire; e il padre della giovane disse al suo genero: 'Prendi un boccon di pane per fortificarti il cuore; poi ve ne andrete'.
\par 6 E si posero ambedue a sedere e mangiarono e bevvero assieme. Poi il padre della giovane disse al marito: 'Ti prego, acconsenti a passar qui la notte, e il cuor tuo si rallegri'.
\par 7 Ma quell'uomo si alzò per andarsene; nondimeno, per le istanze del suocero, pernottò quivi di nuovo.
\par 8 Il quinto giorno egli si levò di buon'ora per andarsene; e il padre della giovane gli disse: 'Ti prego, fortificati il cuore, e aspettate finché declini il giorno'. E si misero a mangiare assieme.
\par 9 E quando quell'uomo si levò per andarsene con la sua concubina e col suo servo, il suocero, il padre della giovane, gli disse: 'Ecco, il giorno volge ora a sera; ti prego, trattienti qui questa notte; vedi, il giorno sta per finire; pernotta qui, e il cuor tuo si rallegri; e domani vi metterete di buon'ora in cammino e te ne andrai a casa'.
\par 10 Ma il marito non volle passar quivi la notte; si levò, partì, e giunse dirimpetto a Jebus, che è Gerusalemme, coi suoi due asini sellati e con la sua concubina.
\par 11 Quando furono vicini a Jebus, il giorno era molto calato; e il servo disse al suo padrone: 'Vieni, ti prego, e dirigiamo il cammino verso questa città de' Gebusei, e pernottiamo quivi'.
\par 12 Il padrone gli rispose: 'No, non dirigeremo il cammino verso una città di stranieri i cui abitanti non sono figliuoli d'Israele, ma andremo fino a Ghibea'.
\par 13 E disse ancora al suo servo: 'Andiamo, cerchiamo d'arrivare a uno di que' luoghi, e pernotteremo a Ghibea o a Rama'.
\par 14 Così passarono oltre, e continuarono il viaggio; e il sole tramontò loro com'eran presso a Ghibea, che appartiene a Beniamino. E volsero il cammino in quella direzione, per andare a pernottare a Ghibea.
\par 15 Il Levita entrò e si fermò sulla piazza della città; ma nessuno li accolse in casa per passar la notte.
\par 16 Quand'ecco un vecchio, che tornava la sera dai campi, dal suo lavoro; era un uomo della contrada montuosa d'Efraim, che abitava come forestiero in Ghibea, la gente del luogo essendo Beniaminita.
\par 17 Alzati gli occhi, vide quel viandante sulla piazza della città. E il vecchio gli disse: 'Dove vai, e donde vieni?'
\par 18 E quello gli rispose: 'Siam partiti da Bethlehem di Giuda, e andiamo nella parte più remota della contrada montuosa d'Efraim. Io sono di là, ed ero andato a Bethlehem di Giuda; ora mi reco alla casa dell'Eterno, e non v'è alcuno che m'accolga in casa sua.
\par 19 Eppure abbiamo della paglia e del foraggio per i nostri asini, e anche del pane e del vino per me, per la tua serva e per il garzone che è coi tuoi servi; a noi non manca nulla'.
\par 20 Il vecchio gli disse: 'La pace sia teco! Io m'incarico d'ogni tuo bisogno; ma non devi passar la notte sulla piazza'.
\par 21 Così lo menò in casa sua, e diè del foraggio agli asini; i viandanti si lavarono i piedi, e mangiarono e bevvero.
\par 22 Mentre stavano rallegrandosi, ecco gli uomini della città, gente perversa, circondare la casa, picchiare alla porta, e dire al vecchio, padron di casa: 'Mena fuori quell'uomo ch'è entrato in casa tua ché lo vogliam conoscere!'
\par 23 Ma il padron di casa, uscito fuori, disse loro: 'No, fratelli miei, vi prego, non fate una mala azione; giacché quest'uomo è venuto in casa mia, non commettete questa infamia!
\par 24 Ecco qua la mia figliuola ch'è vergine, e la concubina di quell'uomo; io ve le menerò fuori, e voi servitevene, e fatene quel che vi pare; ma non commettete contro quell'uomo una simile infamia!'
\par 25 Ma quegli uomini non vollero dargli ascolto. Allora l'uomo prese la sua concubina e la menò fuori a loro; ed essi la conobbero, e abusarono di lei tutta la notte fino al mattino; poi, allo spuntar dell'alba, la lasciaron andare.
\par 26 E quella donna, sul far del giorno, venne a cadere alla porta di casa dell'uomo presso il quale stava il suo marito, e quivi rimase finché fu giorno chiaro.
\par 27 Il suo marito, la mattina, si levò, aprì la porta di casa e uscì per continuare il suo viaggio, quand'ecco la donna, la sua concubina, giacer distesa alla porta di casa, con le mani sulla soglia.
\par 28 Egli le disse: 'Lèvati, andiamocene!' Ma non ebbe risposta. Allora il marito la caricò sull'asino, e partì per tornare alla sua dimora.
\par 29 E come fu giunto a casa, si munì d'un coltello, prese la sua concubina e la divise, membro per membro, in dodici pezzi, che mandò per tutto il territorio d'Israele.
\par 30 Di guisa che chiunque vide ciò, disse: 'Una cosa simile non è mai accaduta né s'è mai vista, da quando i figliuoli d'Israele salirono dal paese d'Egitto, fino al dì d'oggi! Prendete il fatto a cuore, consigliatevi e parlate'.

\chapter{20}

\par 1 Allora tutti i figliuoli d'Israele uscirono, da Dan fino a Beer-Sceba e al paese di Galaad, e la raunanza si raccolse come un sol uomo dinanzi all'Eterno, a Mitspa.
\par 2 I capi di tutto il popolo, e tutte le tribù d'Israele si presentarono nella raunanza del popolo di Dio, in numero di quattrocentomila fanti, atti a trar la spada.
\par 3 E i figliuoli di Beniamino udirono che i figliuoli d'Israele eran saliti a Mitspa. I figliuoli d'Israele dissero: 'Parlate! Com'è stato commesso questo delitto?'
\par 4 Allora il Levita, il marito della donna ch'era stata uccisa, rispose: 'Io ero giunto con la mia concubina a Ghibea di Beniamino per passarvi la notte.
\par 5 Ma gli abitanti di Ghibea si levarono contro di me e attorniarono di notte la casa dove stavo; aveano l'intenzione d'uccidermi; violentarono la mia concubina, ed ella morì.
\par 6 Io presi la mia concubina, la feci in pezzi, che mandai per tutto il territorio della eredità d'Israele, perché costoro han commesso un delitto e una infamia in Israele.
\par 7 Eccovi qui tutti, o figliuoli d'Israele; dite qui il vostro parere, e che consigliate di fare'.
\par 8 Tutto il popolo si levò come un sol uomo, dicendo: 'Nessun di noi tornerà alla sua tenda, nessun di noi rientrerà in casa sua.
\par 9 E ora ecco quel che faremo a Ghibea: l'assaliremo, traendo a sorte chi deve cominciare.
\par 10 Prenderemo in tutte le tribù d'Israele dieci uomini su cento, cento su mille e mille su diecimila, i quali andranno a cercar dei viveri per il popolo, affinché, al loro ritorno, Ghibea di Beniamino sia trattata secondo tutta l'infamia che ha commessa in Israele'.
\par 11 Così tutti gli uomini d'Israele si radunarono contro quella città, uniti come fossero un sol uomo.
\par 12 E le tribù d'Israele mandarono degli uomini in tutte le famiglie di Beniamino a dire: 'Che delitto è questo ch'è stato commesso fra voi?
\par 13 Or dunque consegnateci quegli uomini, quegli scellerati di Ghibea, perché li mettiamo a morte, e togliam via il male da Israele'. Ma i figliuoli di Beniamino non vollero dare ascolto alla voce dei loro fratelli, i figliuoli d'Israele.
\par 14 E i figliuoli di Beniamino uscirono dalle loro città, e si radunarono a Ghibea per andare a combattere contro i figliuoli d'Israele.
\par 15 Il censimento che in quel giorno si fece dei figliuoli di Beniamino usciti dalle città, fu di ventiseimila uomini atti a trar la spada, senza contare gli abitanti di Ghibea, che ascendevano al numero di settecento uomini scelti.
\par 16 Fra tutta questa gente c'erano settecento uomini scelti, ch'erano mancini. Tutti costoro poteano lanciare una pietra con la fionda ad un capello, senza fallire il colpo.
\par 17 Si fece pure il censimento degli uomini d'Israele, non compresi quelli di Beniamino; ed erano in numero di quattrocentomila uomini atti a trar la spada, tutta gente di guerra.
\par 18 E i figliuoli d'Israele si mossero, salirono a Bethel e consultarono Iddio, dicendo: 'Chi di noi salirà il primo a combattere contro i figliuoli di Beniamino?' L'Eterno rispose: 'Giuda salirà il primo'.
\par 19 E l'indomani mattina, i figliuoli d'Israele si misero in marcia e si accamparono presso Ghibea.
\par 20 E gli uomini d'Israele uscirono per combattere contro Beniamino, e si disposero in ordine di battaglia contro di loro, presso Ghibea.
\par 21 Allora i figliuoli di Beniamino s'avanzarono da Ghibea, e in quel giorno stesero morti al suolo ventiduemila uomini d'Israele.
\par 22 Il popolo, gli uomini d'Israele, ripresero animo, si disposero di nuovo in ordine di battaglia, nel luogo ove s'eran disposti il primo giorno.
\par 23 E i figliuoli d'Israele salirono e piansero davanti all'Eterno fino alla sera; consultarono l'Eterno, dicendo: 'Debbo io seguitare a combattere contro i figliuoli di Beniamino mio fratello?' L'Eterno rispose: 'Salite contro di loro'.
\par 24 I figliuoli d'Israele vennero a battaglia coi figliuoli di Beniamino una seconda volta.
\par 25 E i Beniaminiti una seconda volta usciron da Ghibea contro di loro, e stesero morti al suolo altri diciottomila uomini dei figliuoli d'Israele, tutti atti a trar la spada.
\par 26 Allora tutti i figliuoli d'Israele e tutto il popolo salirono a Bethel, e piansero, e rimasero quivi davanti all'Eterno, e digiunarono quel dì fino alla sera, e offriron olocausti e sacrifizi di azioni di grazie davanti all'Eterno.
\par 27 E i figliuoli d'Israele consultarono l'Eterno - l'arca del patto di Dio, in quel tempo, era quivi,
\par 28 e Fineas, figliuolo d'Eleazar, figliuolo d'Aaronne, ne faceva allora il servizio - e dissero: 'Debbo io seguitare ancora a combattere contro i figliuoli di Beniamino mio fratello, o debbo cessare?' E l'Eterno rispose: 'Salite, poiché domani ve li darò nelle mani'.
\par 29 E Israele pose un'imboscata tutt'intorno a Ghibea.
\par 30 I figliuoli d'Israele salirono per la terza volta contro i figliuoli di Beniamino, e si disposero in ordine di battaglia presso Ghibea come le altre volte.
\par 31 E i figliuoli di Beniamino, avendo fatto una sortita contro il popolo, si lasciarono attirare lungi dalla città, e cominciarono a colpire e ad uccidere, come le altre volte, alcuni del popolo d'Israele, per le strade, delle quali una sale a Bethel, e l'altra a Ghibea per la campagna: ne uccisero circa trenta.
\par 32 Allora i figliuoli di Beniamino dissero: 'Eccoli sconfitti davanti a noi come la prima volta!' Ma i figliuoli d'Israele dissero: 'Fuggiamo, e attiriamoli lungi dalla città sulle strade maestre!'
\par 33 E tutti gli uomini d'Israele abbandonarono la loro posizione e si disposero in ordine di battaglia a Baal-Thamar, e l'imboscata d'Israele si slanciò fuori dal luogo ove si trovava, da Maareh-Ghibea.
\par 34 Diecimila uomini scelti in tutto Israele giunsero davanti a Ghibea. Il combattimento fu aspro, e i Beniaminiti non si avvedevano del disastro che stava per colpirli.
\par 35 E l'Eterno sconfisse Beniamino davanti ad Israele; e i figliuoli d'Israele uccisero quel giorno venticinquemila e cento uomini di Beniamino, tutti atti a trar la spada.
\par 36 I figliuoli di Beniamino videro che gl'Israeliti eran battuti. Questi, infatti, avean ceduto terreno a Beniamino, perché confidavano nella imboscata che avean posta presso Ghibea.
\par 37 Quelli dell'imboscata si gettaron prontamente su Ghibea; e, avanzatisi, passarono a fil di spada l'intera città.
\par 38 Or v'era un segnale convenuto fra gli uomini d'Israele e quelli dell'imboscata: questi dovean far salire dalla città una gran fumata.
\par 39 Gli uomini d'Israele aveano dunque voltate le spalle nel combattimento; e que' di Beniamino avean cominciato a colpire e uccidere circa trenta uomini d'Israele. Essi dicevano: 'Per certo, eccoli sconfitti davanti a noi come nella prima battaglia!'
\par 40 Ma quando il segnale, la colonna di fumo, cominciò ad alzarsi dalla città, que' di Beniamino si volsero indietro, ed ecco che tutta la città saliva in fiamme verso il cielo.
\par 41 Allora gli uomini d'Israele fecero fronte indietro, e que' di Beniamino furono spaventati, vedendo il disastro che piombava loro addosso.
\par 42 E voltaron le spalle davanti agli uomini d'Israele, e presero la via del deserto; ma gli assalitori si misero alle loro calcagna, e stendevano morti sul posto quelli che uscivano dalle città.
\par 43 Circondarono i Beniaminiti, l'inseguirono, furon loro sopra dovunque si fermavano, fin dirimpetto a Ghibea dal lato del sol levante.
\par 44 Caddero, de' Beniaminiti, diciottomila uomini, tutta gente di valore.
\par 45 I Beniaminiti voltaron le spalle e fuggiron verso il deserto, in direzione del masso di Rimmon; e gl'Israeliti ne mieterono per le strade cinquemila, li inseguirono da presso fino a Ghideom, e ne colpirono altri duemila.
\par 46 Così, il numero totale de' Beniaminiti che caddero quel giorno fu di venticinquemila, atti a trar la spada, tutta gente di valore.
\par 47 Seicento uomini, che avean voltato le spalle ed eran fuggiti verso il deserto in direzione del masso di Rimmon, rimasero al masso di Rimmon quattro mesi.
\par 48 Poi gl'Israeliti tornarono contro i figliuoli di Beniamino, li sconfissero mettendoli a fil di spada, dagli abitanti delle città al bestiame, a tutto quel che capitava loro; e dettero alle fiamme tutte le città che trovarono.

\chapter{21}

\par 1 Or gli uomini d'Israele avean giurato a Mitspa, dicendo: 'Nessuno di noi darà la sua figliuola in moglie a un Beniaminita'.
\par 2 E il popolo venne a Bethel, dove rimase fino alla sera in presenza di Dio, e alzando la voce, pianse dirottamente, e disse:
\par 3 'O Eterno, o Dio d'Israele, perché mai è avvenuto questo in Israele, che oggi ci sia in Israele una tribù di meno?'
\par 4 Il giorno seguente, il popolo si levò di buon mattino, costruì quivi un altare, e offerse olocausti e sacrifizi di azioni di grazie.
\par 5 E i figliuoli d'Israele dissero: 'Chi è, fra tutte le tribù d'Israele, che non sia salito alla raunanza davanti all'Eterno?' - Poiché avean fatto questo giuramento solenne relativamente a chi non fosse salito in presenza dell'Eterno a Mitspa: 'Quel tale dovrà esser messo a morte'.
\par 6 I figliuoli d'Israele si pentivano di quel che avean fatto a Beniamino loro fratello, e dicevano: 'Oggi è stata soppressa una tribù d'Israele.
\par 7 Come faremo a procurar delle donne ai superstiti, giacché abbiam giurato nel nome dell'Eterno di non dar loro in moglie alcuna delle nostre figliuole?'
\par 8 - dissero dunque: 'Qual è fra le tribù d'Israele quella che non è salita in presenza dell'Eterno a Mitspa?' E ecco che nessuno di Jabes in Galaad era venuto al campo, alla raunanza;
\par 9 poiché, fatto il censimento del popolo, si trovò che quivi non v'era alcuno degli abitanti di Jabes in Galaad.
\par 10 Allora la raunanza mandò là dodicimila uomini dei più valorosi, e diede loro quest'ordine: 'Andate, e mettete a fil di spada gli abitanti di Jabes in Galaad, con le donne e i bambini.
\par 11 E farete questo: voterete allo sterminio ogni maschio e ogni donna che abbia avuto relazioni carnali con uomo'.
\par 12 E quelli trovarono, fra gli abitanti di Jabes in Galaad, quattrocento fanciulle che non aveano avuto relazioni carnali con uomo, e le menarono al campo, a Sciloh, che è nel paese di Canaan.
\par 13 Tutta la raunanza inviò de' messi per parlare ai figliuoli di Beniamino che erano al masso di Rimmon e per proclamar loro la pace.
\par 14 Allora i Beniaminiti tornarono e furon loro date le donne a cui era stata risparmiata la vita fra le donne di Jabes in Galaad; ma non ve ne fu abbastanza per tutti.
\par 15 Il popolo dunque si pentiva di quel che avea fatto a Beniamino, perché l'Eterno avea aperta una breccia fra le tribù d'Israele.
\par 16 E gli anziani della raunanza dissero: 'Come faremo a procurar delle donne ai superstiti, giacché le donne Beniaminite sono state distrutte?' Poi dissero:
\par 17 'Quelli che sono scampati posseggano ciò che apparteneva a Beniamino, affinché non sia soppressa una tribù in Israele.
\par 18 Ma noi non possiamo dar loro delle nostre figliuole in moglie'. Poiché i figliuoli d'Israele avean giurato, dicendo: 'Maledetto chi darà una moglie a Beniamino!'
\par 19 E dissero: 'Ecco, ogni anno si fa una festa in onore dell'Eterno a Sciloh, ch'è al nord di Bethel, a oriente della strada che sale da Bethel a Sichem, e al mezzogiorno di Lebna'.
\par 20 E diedero quest'ordine ai figliuoli di Beniamino: 'Andate, fate un'imboscata nelle vigne;
\par 21 state attenti, e quando le figliuole di Sciloh usciranno per danzare in coro, sbucherete dalle vigne, rapirete ciascuno una delle figliuole di Sciloh per farne vostra moglie, e ve ne andrete nel paese di Beniamino.
\par 22 E quando i loro padri o i loro fratelli verranno a querelarsi con noi, noi diremo loro: Datecele, per favore, giacché in questa guerra non abbiam preso una donna per uno; né siete voi che le avete date loro; nel qual caso, voi sareste colpevoli'.
\par 23 E i figliuoli di Beniamino fecero a quel modo: si presero delle mogli, secondo il loro numero, fra le danzatrici; le rapirono, poi partirono e tornarono nella loro eredità, riedificarono le città e vi stabilirono la loro dimora.
\par 24 In quel medesimo tempo, i figliuoli d'Israele se ne andarono di là, ciascuno nella sua tribù e nella sua famiglia, e ognuno tornò di là nella sua eredità.
\par 25 In quel tempo, non v'era re in Israele; ognun facea quel che gli pareva meglio.


\end{document}