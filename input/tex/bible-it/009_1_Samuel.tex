\begin{document}

\title{1 Samuel}


\chapter{1}

\par 1 V'era un uomo di Ramathaim-Tsofim, della contrada montuosa di Efraim, che si chiamava Elkana, figliuolo di Jeroham, figliuolo d'Elihu, figliuolo di Tohu, figliuolo di Tsuf, Efraimita.
\par 2 Aveva due mogli: una per nome Anna, e l'altra per nome Peninna. Peninna avea de' figliuoli, ma Anna non ne aveva.
\par 3 E quest'uomo, ogni anno, saliva dalla sua città per andare ad adorar l'Eterno degli eserciti e ad offrirgli de' sacrifizi a Sciloh; e quivi erano i due figliuoli di Eli, Hofni e Fineas, sacerdoti dell'Eterno.
\par 4 Quando venne il giorno, Elkana offerse il sacrifizio, e diede a Peninna, sua moglie, e a tutti i figliuoli e a tutte le figliuole di lei le loro parti;
\par 5 ma ad Anna diede una parte doppia, perché amava Anna, benché l'Eterno l'avesse fatta sterile.
\par 6 E la rivale mortificava continuamente Anna affin d'inasprirla perché l'Eterno l'avea fatta sterile.
\par 7 Così avveniva ogni anno; ogni volta che Anna saliva alla casa dell'Eterno, Peninna la mortificava a quel modo; ond'ella piangeva e non mangiava più.
\par 8 Elkana, suo marito, le diceva: 'Anna, perché piangi? Perché non mangi? Perché è triste il cuor tuo? Non ti valgo io più di dieci figliuoli?'
\par 9 E, dopo ch'ebbero mangiato e bevuto a Sciloh, Anna si levò (il sacerdote Eli stava in quell'ora seduto sulla sua sedia all'entrata del tempio dell'Eterno);
\par 10 ella avea l'anima piena di amarezza, e pregò l'Eterno piangendo dirottamente.
\par 11 E fece un voto, dicendo: 'O Eterno degli eserciti! se hai riguardo all'afflizione della tua serva, e ti ricordi di me, e non dimentichi la tua serva, e dai alla tua serva un figliuolo maschio, io lo consacrerò all'Eterno per tutti i giorni della sua vita, e il rasoio non passerà sulla sua testa'.
\par 12 E, com'ella prolungava la sua preghiera dinanzi all'Eterno, Eli stava osservando la bocca di lei.
\par 13 Anna parlava in cuor suo; e si movevano soltanto le sue labbra ma non si sentiva la sua voce; onde Eli credette ch'ella fosse ubriaca; e le disse:
\par 14 'Quanto durerà cotesta tua ebbrezza? Va' a smaltire il tuo vino!'
\par 15 Ma Anna, rispondendo, disse: 'No, signor mio, io sono una donna tribolata nello spirito, e non ho bevuto né vino né bevanda alcoolica, ma stavo spandendo l'anima mia dinanzi all'Eterno.
\par 16 Non prender la tua serva per una donna da nulla; perché l'eccesso del mio dolore e della tristezza mia m'ha fatto parlare fino adesso'.
\par 17 Ed Eli replicò: 'Va' in pace, e l'Iddio d'Israele esaudisca la preghiera che gli hai rivolta!'
\par 18 Ella rispose: 'Possa la tua serva trovar grazia agli occhi tuoi!' Così la donna se ne andò per la sua via, mangiò, e il suo sembiante non fu più quello di prima.
\par 19 L'indomani, ella e suo marito, alzatisi di buon'ora, si prostrarono dinanzi all'Eterno; poi partirono e ritornarono a casa loro a Rama. Elkana conobbe Anna, sua moglie, e l'Eterno si ricordò di lei.
\par 20 Nel corso dell'anno, Anna concepì e partorì un figliuolo, al quale pose nome Samuele, 'perché', disse, 'l'ho chiesto all'Eterno'.
\par 21 E quell'uomo, Elkana, salì con tutta la sua famiglia per andare a offrire all'Eterno il sacrifizio annuo e a sciogliere il suo voto.
\par 22 Ma Anna non salì, e disse a suo marito: 'Io non salirò finché il bambino non sia divezzato; allora lo condurrò, perché sia presentato dinanzi all'Eterno e quivi rimanga per sempre'.
\par 23 Elkana, suo marito, le rispose: 'Fa' come ti par bene; rimani finché tu l'abbia divezzato, purché l'Eterno adempia la sua parola!' Così la donna rimase a casa, e allattò il suo figliuolo fino al momento di divezzarlo.
\par 24 E quando l'ebbe divezzato, lo menò seco, e prese tre giovenchi, un efa di farina e un otre di vino; e lo menò nella casa dell'Eterno a Sciloh. Il fanciullo era ancora piccolino.
\par 25 Elkana ed Anna immolarono il giovenco, e menarono il fanciullo ad Eli.
\par 26 E Anna gli disse: 'Signor mio! Com'è vero che vive l'anima tua, o mio signore, io son quella donna che stava qui vicina a te, a pregare l'Eterno.
\par 27 Pregai per aver questo fanciullo; e l'Eterno mi ha concesso quel che io gli avevo domandato.
\par 28 E, dal canto mio, lo dono all'Eterno; e finché gli durerà la vita, egli sarà donato all'Eterno'. E si prostraron quivi dinanzi all'Eterno.

\chapter{2}

\par 1 Allora Anna pregò e disse: "Il mio cuore esulta nell'Eterno, l'Eterno mi ha dato una forza vittoriosa, la mia bocca s'apre contro i miei nemici perché gioisco per la liberazione che tu m'hai concessa.
\par 2 Non v'è alcuno che sia santo come l'Eterno, poiché non v'è altro Dio fuori di te; né v'è ròcca pari all'Iddio nostro.
\par 3 Non parlate più con tanto orgoglio; non esca più l'arroganza dalla vostra bocca; poiché l'Eterno è un Dio che sa tutto, e da lui son pesate le azioni dell'uomo.
\par 4 L'arco dei potenti è spezzato, e i deboli son cinti di forza.
\par 5 Quei ch'eran satolli s'allogano per aver del pane, e quei che pativan la fame non la patiscono più; perfin la sterile partorisce sette volte, mentre quella che avea molti figli diventa fiacca.
\par 6 L'Eterno fa morire e fa vivere; fa scendere nel soggiorno de' morti e ne fa risalire.
\par 7 L'Eterno fa impoverire ed arricchisce, egli abbassa ed anche innalza.
\par 8 Rileva il misero dalla polvere e trae su il povero dal letame, per farli sedere coi principi, per farli eredi di un trono di gloria; poiché le colonne della terra son dell'Eterno, e sopra queste Egli ha posato il mondo.
\par 9 Egli veglierà sui passi de' suoi fedeli, ma gli empi periranno nelle tenebre; poiché l'uomo non trionferà per la sua forza.
\par 10 Gli avversari dell'Eterno saran frantumati. Egli tonerà contr'essi dal cielo; l'Eterno giudicherà gli estremi confini della terra, darà forza al suo re, farà grande la potenza del suo unto".
\par 11 Elkana se ne andò a casa sua a Rama, e il fanciullo rimase a servire l'Eterno sotto gli occhi del sacerdote Eli.
\par 12 Or i figliuoli di Eli erano uomini scellerati; non conoscevano l'Eterno.
\par 13 Ed ecco qual era il modo d'agire di questi sacerdoti riguardo al popolo: quando qualcuno offriva un sacrifizio, il servo del sacerdote veniva, nel momento in cui si faceva cuocere la carne, avendo in mano una forchetta a tre punte;
\par 14 la piantava nella caldaia o nel paiuolo o nella pentola o nella marmitta; e tutto quello che la forchetta tirava su, il sacerdote lo pigliava per sé. Così facevano a tutti gl'Israeliti, che andavano là, a Sciloh.
\par 15 E anche prima che si fosse fatto fumare il grasso, il servo del sacerdote veniva, e diceva all'uomo che faceva il sacrifizio: 'Dammi della carne da fare arrostire, per il sacerdote; giacché egli non accetterà da te carne cotta, ma cruda'.
\par 16 E se quell'uomo gli diceva: 'Si faccia, prima di tutto, fumare il grasso; poi prenderai quel che vorrai', egli rispondeva: 'No, me la devi dare ora; altrimenti la prenderò per forza!'
\par 17 Il peccato dunque di que' giovani era grande oltremodo agli occhi dell'Eterno, perché la gente sprezzava le offerte fatte all'Eterno.
\par 18 Ma Samuele faceva il servizio nel cospetto dell'Eterno; era giovinetto, e cinto d'un efod di lino.
\par 19 Sua madre gli faceva ogni anno una piccola tonaca, e gliela portava quando saliva con suo marito ad offrire il sacrifizio annuale.
\par 20 Eli benedisse Elkana e sua moglie, dicendo: 'L'Eterno ti dia prole da questa donna, in luogo del dono ch'ella ha fatto all'Eterno!' E se ne tornarono a casa loro.
\par 21 E l'Eterno visitò Anna, la quale concepì e partorì tre figliuoli e due figliuole. E il giovinetto Samuele cresceva presso l'Eterno.
\par 22 Or Eli era molto vecchio e udì tutto quello che i suoi figliuoli facevano a tutto Israele, e come si giacevano con le donne che eran di servizio all'ingresso della tenda di convegno.
\par 23 E disse loro: 'Perché fate tali cose? poiché odo tutto il popolo parlare delle vostre malvage azioni.
\par 24 Non fate così, figliuoli miei, poiché quel che odo di voi non è buono; voi inducete a trasgressione il popolo di Dio.
\par 25 Se un uomo pecca contro un altr'uomo, Iddio lo giudica; ma, se pecca contro l'Eterno, chi intercederà per lui?' Quelli però non diedero ascolto alla voce del padre loro, perché l'Eterno li volea far morire.
\par 26 Intanto, il giovinetto Samuele continuava a crescere, ed era gradito così all'Eterno come agli uomini.
\par 27 Or un uomo di Dio venne da Eli e gli disse: 'Così parla l'Eterno: Non mi sono io forse rivelato alla casa di tuo padre, quand'essi erano in Egitto al servizio di Faraone?
\par 28 Non lo scelsi io forse, fra tutte le tribù d'Israele, perché fosse mio sacerdote, salisse al mio altare, bruciasse il profumo e portasse l'efod in mia presenza? E non diedi io forse alla casa di tuo padre tutti i sacrifizi dei figliuoli d'Israele, fatti mediante il fuoco?
\par 29 E allora perché calpestate i miei sacrifizi e le mie oblazioni che ho comandato mi siano offerti nella mia dimora? E come mai onori i tuoi figliuoli più di me, e v'ingrassate col meglio di tutte le oblazioni d'Israele, mio popolo?
\par 30 Perciò così dice l'Eterno, l'Iddio d'Israele: Io avevo dichiarato che la tua casa e la casa di tuo padre sarebbe al mio servizio, in perpetuo; ma ora l'Eterno dice: Lungi da me tal cosa! Poiché io onoro quelli che m'onorano, e quelli che mi sprezzano saranno avviliti.
\par 31 Ecco, i giorni vengono, quand'io troncherò il tuo braccio e il braccio della casa di tuo padre, in guisa che non vi sarà in casa tua alcun vecchio.
\par 32 E vedrai lo squallore nella mia dimora, mentre Israele sarà ricolmo di beni, e non vi sarà più mai alcun vecchio nella tua casa.
\par 33 E quello de' tuoi che lascerò sussistere presso il mio altare, rimarrà per consumarti gli occhi e illanguidirti il cuore; e tutti i nati e cresciuti in casa tua morranno nel fior degli anni.
\par 34 E ti servirà di segno quello che accadrà ai tuoi figliuoli, Hofni e Fineas: ambedue morranno in uno stesso giorno.
\par 35 Io mi susciterò un sacerdote fedele, che agirà secondo il mio cuore e secondo l'anima mia; io gli edificherò una casa stabile, ed egli sarà al servizio del mio unto per sempre.
\par 36 E chiunque rimarrà della tua casa verrà a prostrarsi davanti a lui per avere una moneta d'argento e un tozzo di pane, e dirà: - Ammettimi, ti prego, a fare alcuno de' servigi del sacerdozio perch'io abbia un boccon di pane da mangiare'. -

\chapter{3}

\par 1 Or il giovinetto Samuele serviva all'Eterno sotto gli occhi di Eli. La parola dell'Eterno era rara, a quei tempi, e le visioni non erano frequenti.
\par 2 In quel medesimo tempo, Eli, la cui vista cominciava a intorbidarsi in guisa ch'egli non ci poteva vedere, se ne stava un giorno coricato nel suo luogo consueto;
\par 3 la lampada di Dio non era ancora spenta, e Samuele era coricato nel tempio dell'Eterno dove si trovava l'arca di Dio.
\par 4 E l'Eterno chiamò Samuele, il quale rispose: 'Eccomi!'
\par 5 e corse da Eli e disse: 'Eccomi, poiché tu m'hai chiamato'. Eli rispose: 'Io non t'ho chiamato, torna a coricarti'. Ed egli se ne andò a coricarsi.
\par 6 L'Eterno chiamò di nuovo Samuele. E Samuele s'alzò, andò da Eli e disse: 'Eccomi, poiché tu m'hai chiamato'. E quegli rispose: 'Figliuol mio, io non t'ho chiamato; torna a coricarti'.
\par 7 Or Samuele non conosceva ancora l'Eterno, e la parola dell'Eterno non gli era ancora stata rivelata.
\par 8 L'Eterno chiamò di bel nuovo Samuele, per la terza volta. Ed egli s'alzò, andò da Eli e disse: 'Eccomi, poiché tu m'hai chiamato'. Allora Eli comprese che l'Eterno chiamava il giovinetto.
\par 9 Ed Eli disse a Samuele: 'Va' a coricarti; e, se sarai chiamato ancora, dirai: Parla, o Eterno, poiché il tuo servo ascolta'. Samuele andò dunque a coricarsi al suo posto.
\par 10 E l'Eterno venne, si tenne lì presso, e chiamò come le altre volte: 'Samuele, Samuele!' Samuele rispose: 'Parla, poiché il tuo servo ascolta'.
\par 11 Allora l'Eterno disse a Samuele: 'Ecco, io sto per fare in Israele una cosa tale che chi l'udrà ne avrà intronati ambedue gli orecchi.
\par 12 In quel giorno io metterò ad effetto contro ad Eli, dal principio fino alla fine, tutto ciò che ho detto circa la sua casa.
\par 13 Gli ho predetto che avrei esercitato i miei giudizi sulla casa di lui in perpetuo, a cagione della iniquità ch'egli ben conosce, poiché i suoi figli hanno attratto su di sé la maledizione, ed egli non li ha repressi.
\par 14 Perciò io giuro alla casa d'Eli che l'iniquità della casa d'Eli non sarà mai espiata né con sacrifizi né con oblazioni'.
\par 15 Samuele rimase coricato sino alla mattina, poi aprì le porte della casa dell'Eterno. Egli temeva di raccontare ad Eli la visione.
\par 16 Ma Eli chiamò Samuele e disse: 'Samuele, figliuol mio!' Egli rispose: 'Eccomi'.
\par 17 Ed Eli: 'Qual è la parola ch'Egli t'ha detta? Ti prego, non me la celare! Iddio ti tratti col massimo rigore, se mi nascondi qualcosa di tutto quello ch'Egli t'ha detto'.
\par 18 Samuele allora gli raccontò tutto, senza celargli nulla. Ed Eli disse: 'Egli è l'Eterno: faccia quello che gli parrà bene'.
\par 19 Samuele intanto cresceva, e l'Eterno era con lui e non lasciò cader a terra alcuna delle parole di lui.
\par 20 Tutto Israele, da Dan fino a Beer-Sceba, riconobbe che Samuele era stabilito profeta dell'Eterno.
\par 21 L'Eterno continuò ad apparire a Sciloh, poiché a Sciloh l'Eterno si rivelava a Samuele mediante la sua parola, e la parola di Samuele era rivolta a tutto Israele.

\chapter{4}

\par 1 Or Israele uscì contro i Filistei per dar battaglia, e si accampò presso Eben-Ezer; i Filistei erano accampati presso Afek.
\par 2 I Filistei si schierarono in battaglia in faccia ad Israele; e, impegnatosi il combattimento, Israele fu sconfitto dai Filistei, che uccisero sul campo di battaglia circa quattromila uomini.
\par 3 Quando il popolo fu tornato nell'accampamento, gli anziani d'Israele dissero: 'Perché l'Eterno ci ha egli oggi sconfitti davanti ai Filistei? Andiamo a prendere a Sciloh l'arca del patto dell'Eterno, e venga essa in mezzo a noi e ci salvi dalle mani de' nostri nemici!'
\par 4 Il popolo quindi mandò gente a Sciloh, e di là fu portata l'arca del patto dell'Eterno degli eserciti, il quale sta fra i cherubini; e i due figliuoli di Eli, Hofni e Fineas, erano là, con l'arca del patto di Dio.
\par 5 E quando l'arca del patto dell'Eterno entrò nel campo, tutto Israele diè in grandi grida di gioia, sì che ne rimbombò la terra.
\par 6 I Filistei, all'udire quelle alte grida, dissero: 'Che significano queste grandi grida nel campo degli Ebrei?' E seppero che l'arca dell'Eterno era arrivata nell'accampamento.
\par 7 E i Filistei ebbero paura, perché dicevano: 'Dio è venuto nell'accampamento'. Ed esclamarono: 'Guai a noi! poiché non era così nei giorni passati.
\par 8 Guai a noi! Chi ci salverà dalle mani di questi dèi potenti? Questi son gli dèi che colpiron gli Egiziani d'ogni sorta di piaghe nel deserto.
\par 9 Siate forti, Filistei, e comportatevi da uomini, onde non abbiate a diventare schiavi degli Ebrei, com'essi sono stati schiavi vostri! Conducetevi da uomini, e combattete!'
\par 10 I Filistei dunque combatterono, e Israele fu sconfitto, e ciascuno se ne fuggì nella sua tenda. La rotta fu enorme, e caddero, d'Israele, trentamila fanti.
\par 11 L'arca di Dio fu presa, e i due figliuoli d'Eli, Hofni e Fineas, morirono.
\par 12 Un uomo di Beniamino, fuggito dal campo di battaglia, giunse correndo a Sciloh quel medesimo giorno, con le vesti stracciate e la testa coperta di terra.
\par 13 Al suo arrivo, ecco che Eli stava sull'orlo della strada, seduto sul suo seggio, aspettando ansiosamente, perché gli tremava il cuore per l'arca di Dio. E come quell'uomo entrò nella città portando la nuova, un grido si levò da tutta la città.
\par 14 Ed Eli, udendo lo strepito delle grida, disse: 'Che significa il chiasso di questo tumulto?' E quell'uomo andò in fretta a portar la nuova ad Eli.
\par 15 Or Eli avea novantott'anni; la vista gli era venuta meno, sicché non potea vedere.
\par 16 Quell'uomo gli disse: 'Son io che vengo dal campo di battaglia e che ne son fuggito oggi'. Ed Eli disse: 'Com'è andata la cosa, figliuol mio?'
\par 17 E colui che portava la nuova, rispondendo, disse: 'Israele è fuggito d'innanzi ai Filistei; e v'è stata una grande strage fra il popolo; anche i tuoi due figliuoli, Hofni e Fineas, sono morti, e l'arca di Dio è stata presa'.
\par 18 E come ebbe mentovato l'arca di Dio, Eli cadde dal suo seggio all'indietro, allato alla porta, si ruppe la nuca, e morì, perché era un uomo vecchio e pesante. Era stato giudice d'Israele quarant'anni.
\par 19 La nuora di lui, moglie di Fineas, era incinta e prossima al parto; quando udì la nuova che l'arca di Dio era presa e che il suo suocero e il suo marito erano morti, si curvò e partorì, perché sorpresa a un tratto dai dolori.
\par 20 E nel punto che stava per morire, le donne che l'assistevano le dissero: 'Non temere, poiché hai partorito un figliuolo'. Ma ella non rispose e non ne fece caso.
\par 21 E al suo bambino pose nome Icabod, dicendo: 'La gloria ha esulato da Israele', perché l'arca di Dio era stata presa, e a motivo del suo suocero e del suo marito.
\par 22 E disse: 'La gloria ha esulato da Israele, perché l'arca di Dio è stata presa'.

\chapter{5}

\par 1 I Filistei, dunque, presero l'arca di Dio, e la trasportarono da Eben-Ezer a Asdod;
\par 2 presero l'arca di Dio, la portarono nella casa di Dagon, e la posarono allato a Dagon.
\par 3 E il giorno dopo, gli Asdodei alzatisi di buon'ora trovarono Dagon caduto con la faccia a terra, davanti all'arca dell'Eterno. Presero Dagon e lo rimisero al suo posto.
\par 4 Il giorno dopo, alzatisi di buon'ora, trovarono che Dagon era di nuovo caduto con la faccia a terra, davanti all'arca dell'Eterno; la testa e ambedue le mani di Dagon giacevano mozzate sulla soglia, e non gli restava più che il tronco.
\par 5 Perciò, fino al dì d'oggi, i sacerdoti di Dagon e tutti quelli che entrano nella casa di Dagon a Asdod non pongono il piede sulla soglia.
\par 6 Poi la mano dell'Eterno si aggravò su quei di Asdod, portò fra loro la desolazione, e li colpì di emorroidi, a Asdod e nel suo territorio.
\par 7 E quando quelli di Asdod videro che così avveniva, dissero: 'L'arca dell'Iddio d'Israele non rimarrà presso di noi, poiché la mano di lui è dura su noi e su Dagon, nostro dio'.
\par 8 Mandaron quindi a convocare presso di loro tutti i principi dei Filistei, e dissero: 'Che faremo dell'arca dell'Iddio d'Israele?' I principi risposero: 'Si trasporti l'arca dell'Iddio d'Israele a Gath'.
\par 9 E trasportaron quivi l'arca dell'Iddio d'Israele. E come l'ebbero trasportata, la mano dell'Eterno si volse contro la città, e vi fu una immensa costernazione. L'Eterno colpì gli uomini della città, piccoli e grandi, e un flagello d'emorroidi scoppiò fra loro.
\par 10 Allora mandarono l'arca di Dio a Ekron. E come l'arca di Dio giunse a Ekron, que' di Ekron cominciarono a gridare, dicendo: 'Hanno trasportato l'arca dell'Iddio d'Israele da noi, per far morire noi e il nostro popolo!'
\par 11 Mandarono quindi a convocare tutti i principi dei Filistei, e dissero: 'Rimandate l'arca dell'Iddio d'Israele; torni essa al suo posto, e non faccia morir noi e il nostro popolo!' Poiché tutta la città era in preda a un terrore di morte, e la mano di Dio s'aggravava grandemente su di essa.
\par 12 Quelli che non morivano eran colpiti d'emorroidi, e le grida della città salivano fino al cielo.

\chapter{6}

\par 1 L'arca dell'Eterno rimase nel paese dei Filistei sette mesi.
\par 2 Poi i Filistei chiamarono i sacerdoti e gl'indovini, e dissero: 'Che faremo dell'arca dell'Eterno? Insegnateci il modo di rimandarla al suo luogo'.
\par 3 E quelli risposero: 'Se rimandate l'arca dell'Iddio d'Israele, non la rimandate senza nulla, ma fategli ad ogni modo un'offerta di riparazione; allora guarirete, e così saprete perché la sua mano non abbia cessato d'aggravarsi su voi'.
\par 4 Essi chiesero: 'Quale offerta di riparazione gli offriremo noi?' Quelli risposero: 'Cinque emorroidi d'oro e cinque topi d'oro, secondo il numero dei principi dei Filistei; giacché una stessa piaga ha colpito voi e i vostri principi.
\par 5 Fate dunque delle figure delle vostre emorroidi e delle figure dei topi che vi devastano il paese, e date gloria all'Iddio d'Israele; forse egli cesserà d'aggravare la sua mano su voi, sui vostri dèi e sul vostro paese.
\par 6 E perché indurereste il cuor vostro come gli Egiziani e Faraone indurarono il cuor loro? Dopo ch'Egli ebbe spiegato contro ad essi la sua potenza, gli Egiziani non lasciarono essi partire gl'Israeliti, sì che questi poterono andarsene?
\par 7 Or dunque fatevi un carro nuovo, e prendete due vacche che allattino e che non abbian mai portato giogo; attaccate al carro le vacche, e riconducete nella stalla i loro vitelli.
\par 8 Poi prendete l'arca dell'Eterno e mettetela sul carro; e accanto ad essa ponete, in una cassetta, i lavori d'oro che presentate all'Eterno come offerta di riparazione; e lasciatela, sì che se ne vada.
\par 9 E state a vedere: se sale per la via che mena al suo paese, verso Beth-Scemesh, vuol dire che l'Eterno è quegli che ci ha fatto questo gran male; se no, sapremo che non la sua mano ci ha percossi, ma che questo ci è avvenuto per caso'.
\par 10 Quelli dunque fecero così; presero due vacche che allattavano, le attaccarono al carro, e chiusero nella stalla i vitelli.
\par 11 Poi misero sul carro l'arca dell'Eterno e la cassetta coi topi d'oro e le figure delle emorroidi.
\par 12 Le vacche presero direttamente la via che mena a Beth-Scemesh; seguirono sempre la medesima strada, muggendo mentre andavano e non piegarono né a destra né a sinistra. I principi dei Filistei tennero loro dietro, sino ai confini di Beth-Scemesh.
\par 13 Ora quei di Beth-Scemesh mietevano il grano nella valle; e alzando gli occhi videro l'arca, e si rallegrarono vedendola.
\par 14 Il carro, giunto al campo di Giosuè di Beth-Scemesh, vi si fermò. C'era quivi una gran pietra; essi spaccarono il legname del carro, e offrirono le vacche in olocausto all'Eterno.
\par 15 I Leviti deposero l'arca dell'Eterno e la cassetta che le stava accanto e conteneva gli oggetti d'oro, e misero ogni cosa sulla gran pietra; e, in quello stesso giorno, quei di Beth-Scemesh offrirono olocausti e presentarono sacrifizi all'Eterno.
\par 16 I cinque principi dei Filistei, veduto ciò, tornarono il medesimo giorno a Ekron.
\par 17 Questo è il numero delle emorroidi d'oro che i Filistei presentarono all'Eterno come offerta di riparazione; una per Asdod, una per Gaza, una per Askalon, una per Gath, una per Ekron.
\par 18 E de' topi d'oro ne offriron tanti quante eran le città dei Filistei appartenenti ai cinque principi, dalle città murate ai villaggi di campagna che si estendono fino alla gran pietra sulla quale fu posata l'arca dell'Eterno, e che sussiste anche al dì d'oggi nel campo di Giosuè, il Beth-scemita.
\par 19 L'Eterno colpì que' di Beth-Scemesh, perché aveano portato gli sguardi sull'arca dell'Eterno; colpì settanta uomini del popolo. Il popolo fece cordoglio, perché l'Eterno l'avea colpito d'una gran piaga.
\par 20 E quelli di Beth-Scemesh dissero: 'Chi può sussistere in presenza dell'Eterno, di questo Dio santo? E da chi salirà l'arca, partendo da noi?'
\par 21 E spedirono de' messi agli abitanti di Kiriath-Jearim per dir loro: 'I Filistei hanno ricondotto l'arca dell'Eterno; scendete e menatela su fra voi'.

\chapter{7}

\par 1 Quei di Kiriath-Jearim vennero, menarono su l'arca dell'Eterno, e la trasportarono in casa di Abinadab, sulla collina, e consacrarono il suo figliuolo Eleazar, perché custodisse l'arca dell'Eterno.
\par 2 Ora dal giorno che l'arca era stata collocata a Kiriath-Jearim era passato molto tempo; vent'anni erano trascorsi e tutta la casa d'Israele sospirava, anelando all'Eterno.
\par 3 Allora Samuele parlò a tutta la casa d'Israele dicendo: 'Se tornate all'Eterno con tutto il vostro cuore, togliete di mezzo a voi gli dèi stranieri e gl'idoli di Astarte, volgete risolutamente il cuor vostro verso l'Eterno, e servite a lui solo; ed egli vi libererà dalle mani dei Filistei'.
\par 4 E i figliuoli d'Israele tolsero via gl'idoli di Baal e di Astarte, e servirono all'Eterno soltanto.
\par 5 Poi Samuele disse: 'Radunate tutto Israele a Mitspa, e io pregherò l'Eterno per voi'.
\par 6 Ed essi si adunarono a Mitspa, attinsero dell'acqua e la sparsero davanti all'Eterno, e digiunarono quivi quel giorno, e dissero: 'Abbiamo peccato contro l'Eterno'. E Samuele fece la funzione di giudice d'Israele a Mitspa.
\par 7 Quando i Filistei seppero che i figliuoli d'Israele s'erano adunati a Mitspa, i principi loro salirono contro Israele. La qual cosa avendo udita i figliuoli d'Israele, ebbero paura dei Filistei,
\par 8 e dissero a Samuele: 'Non cessare di gridar per noi all'Eterno, all'Iddio nostro, affinché ci liberi dalle mani dei Filistei'.
\par 9 E Samuele prese un agnello di latte e l'offerse intero in olocausto all'Eterno; e gridò all'Eterno per Israele, e l'Eterno l'esaudì.
\par 10 Ora mentre Samuele offriva l'olocausto, i Filistei s'avvicinarono per assalire Israele; ma l'Eterno tuonò quel giorno con gran fracasso contro i Filistei, e li mise in rotta, talché furono sconfitti dinanzi a Israele.
\par 11 Gli uomini d'Israele uscirono da Mitspa, inseguirono i Filistei, e li batterono fin sotto Beth-Car.
\par 12 Allora Samuele prese una pietra, la pose tra Mitspa e Scen, e la chiamò Eben-Ezer dicendo: 'Fin qui l'Eterno ci ha soccorsi'.
\par 13 I Filistei furono umiliati, e non tornaron più ad invadere il territorio d'Israele; e la mano dell'Eterno fu contro i Filistei per tutto il tempo di Samuele.
\par 14 Le città che i Filistei aveano prese ad Israele, tornarono ad Israele, da Ekron fino a Gath. Israele liberò il loro territorio dalle mani dei Filistei. E vi fu pace fra Israele e gli Amorei.
\par 15 E Samuele fu giudice d'Israele per tutto il tempo della sua vita.
\par 16 Egli andava ogni anno a fare il giro di Bethel, di Ghilgal e di Mitspa, ed esercitava il suo ufficio di giudice d'Israele in tutti quei luoghi.
\par 17 Poi tornava a Rama, dove stava di casa; quivi fungeva da giudice d'Israele, e quivi edificò un altare all'Eterno.

\chapter{8}

\par 1 Or quando Samuele fu diventato vecchio costituì giudici d'Israele i suoi figliuoli.
\par 2 Il suo figliuolo primogenito si chiamava Joel, e il secondo Abia, e faceano le funzioni di giudici a Beer-Sceba.
\par 3 I suoi figliuoli però non seguivano le sue orme, ma si lasciavano sviare dalla cupidigia, accettavano regali e pervertivano la giustizia.
\par 4 Allora tutti gli anziani d'Israele si radunarono, vennero da Samuele a Rama, e gli dissero:
\par 5 'Ecco, tu sei oramai vecchio, e i tuoi figliuoli non seguono le tue orme; or dunque stabilisci su di noi un re che ci amministri la giustizia, come l'hanno tutte le nazioni'.
\par 6 A Samuele dispiacque questo lor dire: 'Dacci un re che amministri la giustizia fra noi'; e Samuele pregò l'Eterno.
\par 7 E l'Eterno disse a Samuele: 'Da' ascolto alla voce del popolo in tutto quello che ti dirà, poiché essi hanno rigettato non te, ma me, perch'io non regni su di loro.
\par 8 Agiscono con te come hanno sempre agito dal giorno che li feci salire dall'Egitto a oggi: m'hanno abbandonato per servire altri dèi.
\par 9 Ora dunque da' ascolto alla loro voce; abbi cura però di avvertirli solennemente e di far loro ben conoscere qual sarà il modo d'agire del re che regnerà su di loro'.
\par 10 Samuele riferì tutte le parole dell'Eterno al popolo che gli domandava un re.
\par 11 E disse: 'Questo sarà il modo d'agire del re che regnerà su di voi. Egli prenderà i vostri figliuoli e li metterà sui suoi carri e fra i suoi cavalieri, e dovranno correre davanti al suo carro;
\par 12 se ne farà de' capitani di migliaia e dei capitani di cinquantine; li metterà ad arare i suoi campi, a mieter le sue biade, a fabbricare i suoi ordigni di guerra e gli attrezzi de' suoi carri.
\par 13 Prenderà le vostre figliuole per farsene delle profumiere, delle cuoche, delle fornaie.
\par 14 Prenderà i vostri campi, le vostre vigne, i vostri migliori uliveti per darli ai suoi servitori.
\par 15 Prenderà la decima delle vostre semente e delle vostre vigne per darla ai suoi eunuchi e ai suoi servitori.
\par 16 Prenderà i vostri servi, le vostre serve, il fiore della vostra gioventù e i vostri asini per adoprarli nei suoi lavori.
\par 17 Prenderà la decima de' vostri greggi, e voi sarete suoi schiavi.
\par 18 E allora griderete per cagione del re che vi sarete scelto, ma in quel giorno l'Eterno non vi risponderà'.
\par 19 Il popolo rifiutò di dare ascolto alle parole di Samuele, e disse: 'No! ci sarà un re su di noi;
\par 20 e anche noi saremo come tutte le nazioni; il nostro re amministrerà la giustizia fra noi, marcerà alla nostra testa e condurrà le nostre guerre'.
\par 21 Samuele, udite tutte le parole del popolo, le riferì all'Eterno.
\par 22 E l'Eterno disse a Samuele: 'Da' ascolto alla loro voce, e stabilisci su di loro un re'. E Samuele disse agli uomini d'Israele: 'Ognuno se ne torni alla sua città'.

\chapter{9}

\par 1 Or v'era un uomo di Beniamino, per nome Kis, figliuolo d'Abiel, figliuolo di Tseror, figliuolo di Becorath, figliuolo d'Afiac, figliuolo d'un Beniaminita. Era un uomo forte e valoroso;
\par 2 aveva un figliuolo per nome Saul, giovine e bello; non ve n'era tra i figliuoli d'Israele uno più bello di lui; era più alto di tutta la gente dalle spalle in su.
\par 3 Or le asine di Kis, padre di Saul, s'erano smarrite; e Kis disse a Saul, suo figliuolo: 'Prendi teco uno dei servi, lèvati e va' in cerca delle asine'.
\par 4 Egli passò per la contrada montuosa di Efraim e attraversò il paese di Shalisha, senza trovarle; poi passarono per il paese di Shaalim, ma non vi erano; attraversarono il paese dei Beniaminiti, ma non le trovarono.
\par 5 Quando furon giunti nel paese di Tsuf, Saul disse al servo che era con lui: 'Vieni, torniamocene, ché altrimenti mio padre cesserebbe dal pensare alle asine e sarebbe in pena per noi'.
\par 6 Il servo gli disse: 'Ecco, v'è in questa città un uomo di Dio, ch'è tenuto in grande onore; tutto quello ch'egli dice, succede sicuramente; andiamoci; forse egli c'indicherà la via che dobbiamo seguire'.
\par 7 E Saul disse al suo servo: 'Ma, ecco, se v'andiamo, che porteremo noi all'uomo di Dio? Poiché non ci son più provvisioni nei nostri sacchi, e non abbiamo alcun presente da offrire all'uomo di Dio. Che abbiamo con noi?'
\par 8 Il servo replicò a Saul, dicendo: 'Ecco, io mi trovo in possesso del quarto d'un siclo d'argento; lo darò all'uomo di Dio, ed egli c'indicherà la via'.
\par 9 (Anticamente, in Israele, quand'uno andava a consultare Iddio, diceva: 'Venite, andiamo dal Veggente!' poiché colui che oggi si chiama Profeta, anticamente si chiamava Veggente).
\par 10 E Saul disse al suo servo: 'Dici bene; vieni, andiamo'. E andarono alla città dove stava l'uomo di Dio.
\par 11 Mentre facevano la salita che mena alla città, trovarono delle fanciulle che uscivano ad attingere acqua, e chiesero loro: 'È qui il veggente?'
\par 12 Quelle risposer loro, dicendo: 'Sì, c'è; è là dove sei diretto; ma va' presto, giacché è venuto oggi in città, perché oggi il popolo fa un sacrifizio sull'alto luogo.
\par 13 Quando sarete entrati in città, lo troverete di certo, prima ch'egli salga all'alto luogo a mangiare. Il popolo non mangerà prima ch'egli sia giunto, perché è lui che deve benedire il sacrifizio; dopo di che, i convitati mangeranno. Or dunque salite, perché proprio ora lo troverete'.
\par 14 Ed essi salirono alla città; e, come vi furono entrati, ecco Samuele che usciva loro incontro per salire all'alto luogo.
\par 15 Or un giorno prima dell'arrivo di Saul, l'Eterno aveva avvertito Samuele, dicendo:
\par 16 'Domani, a quest'ora, ti manderò un uomo del paese di Beniamino, e tu l'ungerai come capo del mio popolo d'Israele. Egli salverà il mio popolo dalle mani dei Filistei; poiché io ho rivolto lo sguardo verso il mio popolo, perché il suo grido è giunto fino a me'.
\par 17 E quando Samuele vide Saul, l'Eterno gli disse: 'Ecco l'uomo di cui t'ho parlato; egli è colui che signoreggerà sul mio popolo'.
\par 18 Saul s'avvicinò a Samuele entro la porta della città, e gli disse: 'Indicami, ti prego, dove sia la casa del veggente'.
\par 19 E Samuele rispose a Saul: 'Sono io il veggente. Sali davanti a me all'alto luogo, e mangerete oggi con me; poi domattina ti lascerò partire, e ti dirò tutto quello che hai nel cuore.
\par 20 E quanto alle asine smarrite tre giorni fa, non dartene pensiero, perché son trovate. E per chi è tutto quello che v'è di desiderabile in Israele? Non è esso per te e per tutta la casa di tuo padre?'
\par 21 Saul, rispondendo, disse: 'Non son io un Beniaminita? di una delle più piccole tribù d'Israele? La mia famiglia non è essa la più piccola fra tutte le famiglie della tribù di Beniamino? Perché dunque mi parli a questo modo?'
\par 22 Samuele prese Saul e il suo servo, li introdusse nella sala e li fe' sedere in capo di tavola fra i convitati, ch'eran circa trenta persone.
\par 23 E Samuele disse al cuoco: 'Porta qua la porzione che t'ho data, e della quale t'ho detto: Tienla in serbo presso di te'.
\par 24 Il cuoco allora prese la coscia e ciò che v'aderiva, e la mise davanti a Saul. E Samuele disse: 'Ecco ciò ch'è stato tenuto in serbo; mettitelo dinanzi e mangia, poiché è stato serbato apposta per te quand'ho invitato il popolo'. Così Saul, quel giorno, mangiò con Samuele.
\par 25 Poi scesero dall'alto luogo in città, e Samuele s'intrattenne con Saul sul terrazzo.
\par 26 L'indomani si alzarono presto; allo spuntar dell'alba, Samuele chiamò Saul sul terrazzo, e gli disse: 'Vieni, ch'io ti lasci partire'. Saul s'alzò, e uscirono fuori ambedue, egli e Samuele.
\par 27 Quando furon discesi all'estremità della città, Samuele disse a Saul: 'Di' al servo che passi, e vada innanzi a noi (e il servo passò); ma tu adesso fermati, ed io ti farò udire la parola di Dio'.

\chapter{10}

\par 1 Allora Samuele prese un vasetto d'olio, lo versò sul capo di lui, baciò Saul e disse: 'L'Eterno non t'ha egli unto perché tu sia il capo della sua eredità?
\par 2 Oggi, quando tu sarai partito da me, troverai due uomini presso al sepolcro di Rachele, ai confini di Beniamino, a Tseltsah, i quali ti diranno: Le asine delle quali andavi in cerca, sono trovate; ed ecco tuo padre non è più in pensiero per le asine, ma è in pena per voi, e va dicendo: Che farò io riguardo al mio figliuolo?
\par 3 E quando sarai passato più innanzi e sarai giunto alla quercia di Tabor, t'incontrerai con tre uomini che salgono ad adorare Iddio a Bethel, portando l'uno tre capretti, l'altro tre pani, e il terzo un otre di vino.
\par 4 Essi ti saluteranno, e ti daranno due pani, che riceverai dalla loro mano.
\par 5 Poi arriverai a Ghibea-Elohim, dov'è la guarnigione dei Filistei; e avverrà che, entrando in città, incontrerai una schiera di profeti che scenderanno dall'alto luogo, preceduti da saltèri, da timpani, da flauti, da cetre, e che profeteranno.
\par 6 E lo spirito dell'Eterno t'investirà e tu profeterai con loro, e sarai mutato in un altr'uomo.
\par 7 E quando questi segni ti saranno avvenuti, fa' quello che avrai occasione di fare, poiché Dio è teco.
\par 8 Poi scenderai prima di me a Ghilgal; ed ecco io scenderò verso te per offrire olocausti e sacrifizi di azioni di grazie. Tu aspetterai sette giorni, finch'io giunga da te e ti faccia sapere quello che devi fare'.
\par 9 E non appena egli ebbe voltate le spalle per partirsi da Samuele, Iddio gli mutò il cuore, e tutti quei segni si verificarono in quel medesimo giorno.
\par 10 E come giunsero a Ghibea, ecco che una schiera di profeti si fece incontro a Saul; allora lo spirito di Dio lo investì, ed egli si mise a profetare in mezzo a loro.
\par 11 Tutti quelli che l'avean conosciuto prima, lo videro che profetava coi profeti, e dicevano l'uno all'altro: 'Che è mai avvenuto al figliuolo di Kis? Saul è anch'egli tra i profeti?'
\par 12 E un uomo del luogo rispose, dicendo: 'E chi è il loro padre?' Di qui venne il proverbio: 'Saul è anch'egli tra i profeti?'
\par 13 E come Saul ebbe finito di profetare, si recò all'alto luogo.
\par 14 E lo zio di Saul disse a lui e al suo servo: 'Dove siete andati?' Saul rispose: 'A cercare le asine; ma vedendo che non le potevamo trovare, siamo andati da Samuele'.
\par 15 E lo zio di Saul disse: 'Raccontami, ti prego, quello che vi ha detto Samuele'.
\par 16 E Saul a suo zio: 'Egli ci ha dichiarato positivamente che le asine erano trovate'. Ma di quel che Samuele avea detto riguardo al regno non gli riferì nulla.
\par 17 Poi Samuele convocò il popolo dinanzi all'Eterno a Mitspa,
\par 18 e disse ai figliuoli d'Israele: 'Così dice l'Eterno, l'Iddio d'Israele: Io trassi Israele dall'Egitto, e vi liberai dalle mani degli Egiziani e dalle mani di tutti i regni che vi opprimevano.
\par 19 Ma oggi voi rigettate l'Iddio vostro che vi salvò da tutti i vostri mali e da tutte le vostre tribolazioni, e gli dite: stabilisci su di noi un re! Or dunque presentatevi nel cospetto dell'Eterno per tribù e per migliaia'.
\par 20 Poi Samuele fece accostare tutte le tribù d'Israele, e la tribù di Beniamino fu designata dalla sorte.
\par 21 Fece quindi accostare la tribù di Beniamino per famiglie, e la famiglia di Matri fu designata dalla sorte. Poi fu designato Saul, figliuolo di Kis; e lo cercarono, ma non fu trovato.
\par 22 Allora consultarono di nuovo l'Eterno: 'Quell'uomo è egli già venuto qua?' L'Eterno rispose: 'Guardate, ei s'è nascosto fra i bagagli'.
\par 23 Corsero a trarlo di là; e quand'egli si presentò in mezzo al popolo, era più alto di tutta la gente dalle spalle in su.
\par 24 E Samuele disse a tutto il popolo: 'Vedete colui che l'Eterno si è scelto? Non v'è alcuno in tutto il popolo che sia pari a lui'. E tutto il popolo diè in esclamazioni di gioia, gridando: 'Viva il re!'
\par 25 Allora Samuele espose al popolo la legge del regno, e la scrisse in un libro, che depose nel cospetto dell'Eterno. Poi Samuele rimandò tutto il popolo, ciascuno a casa sua.
\par 26 Saul se ne andò anch'egli a casa sua a Ghibea, e con lui andarono gli uomini valorosi a cui Dio avea toccato il cuore.
\par 27 Nondimeno, ci furono degli uomini da nulla che dissero: 'Come ci salverebbe costui?' E lo disprezzarono, e non gli portaron alcun dono. Ma egli fece vista di non udire.

\chapter{11}

\par 1 Or Nahas, l'Ammonita, salì e s'accampò contro Iabes di Galaad. E tutti quelli di Iabes dissero a Nahas: 'Fa' alleanza con noi, e noi ti serviremo'.
\par 2 E Nahas, l'Ammonita, rispose loro: 'Io farò alleanza con voi a questa condizione: ch'io vi cavi a tutti l'occhio destro, e getti così quest'obbrobrio su tutto Israele'.
\par 3 Gli anziani di Iabes gli dissero: 'Concedici sette giorni di tregua perché inviamo de' messi per tutto il territorio d'Israele; e se non vi sarà chi ci soccorra, ci arrenderemo a te'.
\par 4 I messi vennero dunque a Ghibea di Saul, riferirono queste parole in presenza del popolo, e tutto il popolo alzò la voce e pianse.
\par 5 Ed ecco Saul tornava dai campi, seguendo i bovi, e disse: 'Che ha egli il popolo, che piange?' E gli riferiron le parole di quei di Iabes.
\par 6 E com'egli ebbe udite quelle parole, lo spirito di Dio investì Saul, che s'infiammò d'ira;
\par 7 e prese un paio di buoi, li tagliò a pezzi, che mandò, per mano dei messi, per tutto il territorio d'Israele, dicendo: 'Così saranno trattati i buoi di chi non seguirà Saul e Samuele'. Il terrore dell'Eterno s'impadronì del popolo, e partirono come se fossero stati un uomo solo.
\par 8 Saul li passò in rassegna a Bezek, ed erano trecentomila figliuoli d'Israele e trentamila uomini di Giuda.
\par 9 E dissero a que' messi ch'eran venuti: 'Dite così a quei di Iabes di Galaad: Domani, quando il sole sarà in tutto il suo calore, sarete liberati'. E i messi andarono a riferire queste parole a quei di Iabes, i quali si rallegrarono.
\par 10 E quei di Iabes dissero agli Ammoniti: 'Domani verrem da voi, e farete di noi tutto quello che vi parrà'.
\par 11 Il giorno seguente, Saul divise il popolo in tre schiere, che penetrarono nel campo degli Ammoniti in su la vigilia del mattino, e li batterono fino alle ore calde del giorno. Quelli che scamparono furon dispersi in guisa che non ne rimasero due assieme.
\par 12 Il popolo disse a Samuele: 'Chi è che diceva: Saul regnerà egli su noi? Dateci quegli uomini e li metteremo a morte'.
\par 13 Ma Saul rispose: 'Nessuno sarà messo a morte in questo giorno, perché oggi l'Eterno ha operato una liberazione in Israele'.
\par 14 E Samuele disse al popolo: 'Venite, andiamo a Ghilgal, ed ivi confermiamo l'autorità reale'.
\par 15 E tutto il popolo andò a Ghilgal, e quivi, a Ghilgal, fecero Saul re davanti all'Eterno, e quivi offrirono nel cospetto dell'Eterno sacrifizi di azioni di grazie. E Saul e gli uomini tutti d'Israele fecero gran festa in quel luogo.

\chapter{12}

\par 1 Allora Samuele disse a tutto Israele: 'Ecco, io vi ho ubbidito in tutto quello che m'avete detto, ed ho costituito un re su di voi.
\par 2 Ed ora, ecco il re che andrà dinanzi a voi. Quanto a me, io son vecchio e canuto, e i miei figliuoli sono tra voi; io sono andato innanzi a voi dalla mia giovinezza fino a questo giorno.
\par 3 Eccomi qui; rendete la vostra testimonianza a mio carico, in presenza dell'Eterno e in presenza del suo unto: A chi ho preso il bue? A chi ho preso l'asino? Chi ho defraudato? A chi ho fatto violenza? Dalle mani di chi ho accettato doni per chiuder gli occhi a suo riguardo? Io vi restituirò ogni cosa!'
\par 4 Quelli risposero: 'Tu non ci hai defraudati, non ci hai fatto violenza, e non hai preso nulla dalle mani di chicchessia'.
\par 5 Ed egli a loro: 'Oggi l'Eterno è testimone contro di voi, e il suo unto pure è testimone, che voi non avete trovato nulla nelle mie mani'. Il popolo rispose: 'Egli è testimone!'
\par 6 Allora Samuele disse al popolo: 'Testimone è l'Eterno, che costituì Mosè ed Aaronne e fe' salire i padri vostri dal paese d'Egitto.
\par 7 Or dunque presentatevi, ond'io, dinanzi all'Eterno, dibatta con voi la causa relativa a tutte le opere di giustizia che l'Eterno ha compiute a beneficio vostro e dei vostri padri.
\par 8 Dopo che Giacobbe fu entrato in Egitto, i vostri padri gridarono all'Eterno, e l'Eterno mandò Mosè ed Aaronne, i quali trassero i padri vostri fuor dall'Egitto e li fecero abitare in questo luogo.
\par 9 Ma essi dimenticarono l'Eterno, il loro Dio, ed egli li diede in potere di Sisera, capo dell'esercito di Hatsor, e in potere dei Filistei e del re di Moab, i quali mossero loro guerra.
\par 10 Allora gridarono all'Eterno e dissero: 'Abbiam peccato, perché abbiamo abbandonato l'Eterno, e abbiam servito agl'idoli di Baal e d'Astarte; ma ora, liberaci dalle mani dei nostri nemici, e serviremo te'.
\par 11 E l'Eterno mandò Jerubbaal e Bedan e Jefte e Samuele, e vi liberò dalle mani de' nemici che vi circondavano, e viveste al sicuro.
\par 12 Ma quando udiste che Nahas, re de' figliuoli di Ammon, marciava contro di voi, mi diceste: 'No, deve regnar su noi un re', mentre l'Eterno, il vostro Dio, era il vostro re.
\par 13 Or dunque, ecco il re che vi siete scelto, che avete chiesto; ecco, l'Eterno ha costituito un re su di voi.
\par 14 Se temete l'Eterno, lo servite, e ubbidite alla sua voce, se non siete ribelli al comandamento dell'Eterno, e tanto voi quanto il re che regna su voi siete seguaci dell'Eterno, ch'è il vostro Dio, bene;
\par 15 ma, se non ubbidite alla voce dell'Eterno, se vi ribellate al comandamento dell'Eterno, la mano dell'Eterno sarà contro di voi, come fu contro i vostri padri.
\par 16 E anche ora, fermatevi e mirate questa cosa grande che l'Eterno sta per compiere dinanzi agli occhi vostri!
\par 17 Non siamo al tempo della messe del grano? Io invocherò l'Eterno, ed egli manderà tuoni e pioggia affinché sappiate e veggiate quanto è grande agli occhi dell'Eterno il male che avete fatto chiedendo per voi un re'.
\par 18 Allora Samuele invocò l'Eterno, e l'Eterno mandò quel giorno tuoni e pioggia e tutto il popolo ebbe gran timore dell'Eterno e di Samuele.
\par 19 E tutto il popolo disse a Samuele: 'Prega l'Eterno, il tuo Dio, per i tuoi servi, affinché non muoiano; poiché a tutti gli altri nostri peccati abbiamo aggiunto questo torto di chiedere per noi un re'.
\par 20 E Samuele rispose al popolo: 'Non temete; è vero, voi avete fatto tutto questo male; nondimeno, non vi ritraete dal seguir l'Eterno, ma servitelo con tutto il cuor vostro;
\par 21 non ve ne ritraete, perché andreste dietro a cose vane, che non posson giovare né liberare, perché son cose vane.
\par 22 Poiché l'Eterno, per amore del suo gran nome, non abbandonerà il suo popolo, giacché è piaciuto all'Eterno di far di voi il popolo suo.
\par 23 E, quanto a me, lungi da me il peccare contro l'Eterno cessando di pregare per voi! Anzi, io vi mostrerò la buona e diritta via.
\par 24 Solo temete l'Eterno, e servitelo fedelmente, con tutto il cuor vostro; poiché mirate le cose grandi ch'egli ha fatte per voi!
\par 25 Ma, se continuate ad agire malvagiamente, perirete e voi e il vostro re'.

\chapter{13}

\par 1 Saul aveva trent'anni quando cominciò a regnare; e regnò quarantadue anni sopra Israele.
\par 2 Saul si scelse tremila uomini d'Israele: duemila stavano con lui a Micmas e sul monte di Bethel, e mille con Gionathan a Ghibea di Beniamino; e rimandò il resto del popolo, ognuno alla sua tenda.
\par 3 Gionathan batté la guarnigione de' Filistei che stava a Gheba, e i Filistei lo seppero e Saul fe' sonar la tromba per tutto il paese, dicendo: 'Lo sappiano gli Ebrei!'
\par 4 E tutto Israele sentì dire: 'Saul ha battuto la guarnigione de' Filistei, e Israele è venuto in odio ai Filistei'. Così il popolo fu convocato a Ghilgal per seguir Saul.
\par 5 E i Filistei si radunarono per combattere contro Israele; aveano trentamila carri, seimila cavalieri, e gente numerosa come la rena ch'è sul lido del mare. Saliron dunque e si accamparono a Micmas, a oriente di Beth-Aven.
\par 6 Or gl'Israeliti, vedendosi ridotti a mal partito, perché il popolo era messo alle strette, si nascosero nelle caverne, nelle macchie, tra le rocce, nelle buche e nelle cisterne.
\par 7 Ci furon degli Ebrei che passarono il Giordano, per andare nel paese di Gad e di Galaad. Quanto a Saul, egli era ancora a Ghilgal, e tutto il popolo che lo seguiva, tremava.
\par 8 Egli aspettò sette giorni, secondo il termine fissato da Samuele; ma Samuele non giungeva a Ghilgal, e il popolo cominciò a disperdersi e ad abbandonarlo.
\par 9 Allora Saul disse: 'Menatemi l'olocausto e i sacrifizi di azioni di grazie'; e offerse l'olocausto.
\par 10 E come finiva d'offrir l'olocausto, ecco che arrivò Samuele; e Saul gli uscì incontro per salutarlo.
\par 11 Ma Samuele gli disse: 'Che hai tu fatto?' Saul rispose: 'Siccome vedevo che il popolo si disperdeva e m'abbandonava, che tu non giungevi nel giorno stabilito, e che i Filistei erano adunati a Micmas, mi son detto:
\par 12 Ora i Filistei mi piomberanno addosso a Ghilgal, e io non ho ancora implorato l'Eterno! Così, mi son fatto violenza, ed ho offerto l'olocausto'.
\par 13 Allora Samuele disse a Saul: 'Tu hai agito stoltamente; non hai osservato il comandamento che l'Eterno, il tuo Dio, ti avea dato. L'Eterno avrebbe stabilito il tuo regno sopra Israele in perpetuo;
\par 14 ma ora il tuo regno non durerà; l'Eterno s'è cercato un uomo secondo il cuor suo, e l'Eterno l'ha destinato ad esser principe del suo popolo, giacché tu non hai osservato quel che l'Eterno t'aveva ordinato'.
\par 15 Poi Samuele si levò e salì da Ghilgal a Ghibea di Beniamino, e Saul fece la rassegna del popolo che si trovava con lui; eran circa seicento uomini.
\par 16 Or Saul, Gionathan suo figliuolo, e la gente che si trovava con essi occupavano Ghibea di Beniamino, mentre i Filistei erano accampati a Micmas.
\par 17 Dal campo de' Filistei uscirono dei guastatori divisi in tre schiere; una prese la via d'Ofra, verso il paese di Shual;
\par 18 l'altra prese la via di Beth-Horon; la terza prese la via della frontiera che guarda la valle di Tseboim, verso il deserto.
\par 19 Or in tutto il paese d'Israele non si trovava un fabbro; poiché i Filistei avevan detto: 'Vediamo che gli Ebrei non si facciano spade o lance'.
\par 20 E tutti gl'Israeliti scendevano dai Filistei per farsi aguzzare chi il suo vomero, chi la sua zappa, chi la sua scure, chi la sua vanga.
\par 21 E il prezzo dell'arrotatura era di un pim per le vanghe, per le zappe, per i tridenti, per le scuri e per aggiustare i pungoli.
\par 22 Così avvenne che il dì della battaglia non si trovava in mano a tutta la gente, ch'era con Saul e con Gionathan, né spada né lancia; non se ne trovava che in man di Saul e di Gionathan suo figliuolo.
\par 23 E la guarnigione dei Filistei uscì ad occupare il passo di Micmas.

\chapter{14}

\par 1 Or avvenne che un giorno, Gionathan, figliuolo di Saul, disse al giovane suo scudiero: 'Vieni, andiamo verso la guarnigione de' Filistei, che è là dall'altra parte'. Ma non ne disse nulla a suo padre.
\par 2 Saul stava allora all'estremità di Ghibea sotto il melagrano di Migron, e la gente che avea seco noverava circa seicento uomini;
\par 3 e Ahia, figliuolo di Ahitub, fratello d'Icabod, figliuolo di Fineas, figliuolo d'Eli sacerdote dell'Eterno a Sciloh, portava l'efod. Il popolo non sapeva che Gionathan se ne fosse andato.
\par 4 Or fra i passi attraverso ai quali Gionathan cercava d'arrivare alla guarnigione de' Filistei, c'era una punta di rupe da una parte e una punta di rupe dall'altra parte: una si chiamava Botsets, e l'altra Seneh.
\par 5 Una di queste punte sorgeva al nord, dirimpetto a Micmas, e l'altra a mezzogiorno, dirimpetto a Ghibea.
\par 6 Gionathan disse al suo giovane scudiero: 'Vieni, andiamo verso la guarnigione di questi incirconcisi; forse l'Eterno agirà per noi, poiché nulla può impedire all'Eterno di salvare con molta o con poca gente'.
\par 7 Il suo scudiero gli rispose: 'Fa' tutto quello che ti sta nel cuore; va' pure; ecco, io son teco dove il cuor ti mena'.
\par 8 Allora Gionathan disse: 'Ecco, noi andremo verso quella gente, e ci mostreremo a loro.
\par 9 Se ci dicono: - Fermatevi finché veniam da voi -, ci fermeremo al nostro posto, e non saliremo fino a loro;
\par 10 ma se ci dicono - Venite su da noi -, saliremo, perché l'Eterno li avrà dati nelle nostre mani. Questo ci servirà di segno'.
\par 11 Così si mostrarono ambedue alla guarnigione de' Filistei; e i Filistei dissero: 'Ecco gli Ebrei che escon dalle grotte dove s'eran nascosti!'
\par 12 E gli uomini della guarnigione, rivolgendosi a Gionathan e al suo scudiero, dissero: 'Venite su da noi, e vi faremo saper qualcosa'. Gionathan disse al suo scudiero: 'Sali dietro a me, poiché l'Eterno li ha dati nelle mani d'Israele'.
\par 13 Gionathan salì, arrampicandosi con le mani e coi piedi, seguito dal suo scudiero. E i Filistei caddero dinanzi a Gionathan; e lo scudiero dietro a lui dava loro la morte.
\par 14 In questa prima disfatta, inflitta da Gionathan e dal suo scudiero, caddero circa venti uomini, sullo spazio di circa la metà di un iugero di terra.
\par 15 E lo spavento si sparse nell'accampamento, nella campagna e fra tutto il popolo; la guarnigione e i guastatori furono anch'essi spaventati; e il paese tremò; fu uno spavento di Dio.
\par 16 Le sentinelle di Saul a Ghibea di Beniamino guardarono ed ecco che la moltitudine si sbandava e fuggiva di qua e di là.
\par 17 Allora Saul disse alla gente ch'era con lui: 'Fate la rassegna, e vedete chi se n'è andato da noi'. E, fatta la rassegna, ecco che mancavano Gionathan e il suo scudiero.
\par 18 E Saul disse ad Ahia: 'Fa' accostare l'arca di Dio!' Poiché l'arca di Dio era allora coi figliuoli d'Israele.
\par 19 E mentre Saul parlava col sacerdote, il tumulto andava aumentando nel campo de' Filistei; e Saul disse al sacerdote: 'Ritira la mano!'
\par 20 Poi Saul e tutto il popolo ch'era con lui si radunarono e s'avanzarono fino al luogo della battaglia; ed ecco che la spada dell'uno era rivolta contro l'altro, e la confusione era grandissima.
\par 21 Or gli Ebrei, che già prima si trovavan coi Filistei ed eran saliti con essi al campo dal paese d'intorno, fecero voltafaccia e s'unirono anch'essi con gl'Israeliti ch'erano con Saul e con Gionathan.
\par 22 E parimenti tutti gl'Israeliti che s'eran nascosti nella contrada montuosa di Efraim, quand'udirono che i Filistei fuggivano, si misero anch'essi a inseguirli da presso, combattendo.
\par 23 In quel giorno l'Eterno salvò Israele, e la battaglia s'estese fin oltre Beth-Aven.
\par 24 Or gli uomini d'Israele, in quel giorno, erano sfiniti; ma Saul fece fare al popolo questo giuramento: 'Maledetto l'uomo che toccherà cibo prima di sera, prima ch'io mi sia vendicato de' miei nemici'. E nessuno del popolo toccò cibo.
\par 25 Or tutto il popolo giunse a una foresta, dove c'era del miele per terra.
\par 26 E come il popolo fu entrato nella foresta, vide il miele che colava; ma nessuno si portò la mano alla bocca, perché il popolo rispettava il giuramento.
\par 27 Ma Gionathan non avea sentito quando suo padre avea fatto giurare il popolo e stese la punta del bastone che teneva in mano, la intinse nel miele che colava, portò la mano alla bocca, e gli si rischiarò la vista.
\par 28 Uno del popolo, rivolgendosi a lui, gli disse: 'Tuo padre ha espressamente fatto fare al popolo questo giuramento: Maledetto l'uomo che toccherà oggi cibo; e il popolo è estenuato'.
\par 29 Allora Gionathan disse: 'Mio padre ha recato un danno al paese; vedete come l'aver gustato un po' di questo miele m'ha rischiarato la vista!
\par 30 Ah, se il popolo avesse oggi mangiato a sua voglia del bottino che ha trovato presso i nemici! Non si sarebb'egli fatto una più grande strage de' Filistei?'
\par 31 Essi dunque sconfissero quel giorno i Filistei da Micmas ad Ajalon; il popolo era estenuato, e si gettò sul bottino;
\par 32 prese pecore, buoi e vitelli, li scannò sul suolo, e li mangiò col sangue.
\par 33 E questo fu riferito a Saul e gli fu detto: 'Ecco, il popolo pecca contro l'Eterno, mangiando carne col sangue'. Ed egli disse: 'Voi avete commesso un'infedeltà; rotolate subito qua presso di me una gran pietra'.
\par 34 E Saul soggiunse: 'Andate attorno fra il popolo, e dite a ognuno di menarmi qua il suo bue e la sua pecora, e di scannarli qui; poi mangiate, e non peccate contro l'Eterno, mangiando carne con sangue!' E, quella notte, ognuno del popolo menò di propria mano il suo bue, e lo scannò quivi.
\par 35 E Saul edificò un altare all'Eterno; questo fu il primo altare ch'egli edificò all'Eterno.
\par 36 Poi Saul disse: 'Scendiamo nella notte a inseguire i Filistei; saccheggiamoli fino alla mattina, e facciamo che non ne scampi uno'. Il popolo rispose: 'Fa' tutto quello che ti par bene'. Allora disse il sacerdote: 'Accostiamoci qui a Dio'.
\par 37 E Saul consultò Dio, dicendo: 'Debbo io scendere a inseguire i Filistei? Li darai tu nelle mani d'Israele?' Ma questa volta Iddio non gli diede alcuna risposta.
\par 38 E Saul disse: 'Accostatevi qua, voi tutti capi del popolo, riconoscete e vedete in che consista il peccato commesso quest'oggi!
\par 39 Poiché, com'è vero che l'Eterno, il salvatore d'Israele, vive, quand'anche il reo fosse Gionathan mio figliuolo, egli dovrà morire'. Ma in tutto il popolo non ci fu alcuno che gli rispondesse.
\par 40 Allora egli disse a tutto Israele: 'Mettetevi da un lato, e io e Gionathan mio figliuolo staremo dall'altro'. E il popolo disse a Saul: 'Fa' quello che ti par bene'.
\par 41 Saul disse all'Eterno: 'Dio d'Israele, fa' conoscere la verità!' E Gionathan e Saul furon designati dalla sorte, e il popolo scampò.
\par 42 Poi Saul disse: 'Tirate a sorte fra me e Gionathan mio figliuolo'. E Gionathan fu designato.
\par 43 Allora Saul disse a Gionathan: 'Dimmi quello che hai fatto'. E Gionathan glielo confessò, e disse: 'Sì, io assaggiai un po' di miele, con la punta del bastone che avevo in mano; eccomi qui: morrò!'
\par 44 Saul disse: 'Mi tratti Iddio con tutto il suo rigore, se non andrai alla morte, o Gionathan!'
\par 45 E il popolo disse a Saul: 'Gionathan, che ha operato questa gran liberazione in Israele, dovrebb'egli morire? Non sarà mai! Com'è vero che l'Eterno vive, non cadrà in terra un capello del suo capo; poiché oggi egli ha operato con Dio!' Così il popolo salvò Gionathan, che non fu messo a morte.
\par 46 Poi Saul tornò dall'inseguimento de' Filistei, e i Filistei se ne tornarono al loro paese.
\par 47 Or Saul, quand'ebbe preso possesso del suo regno in Israele, mosse guerra a tutti i suoi nemici d'ogn'intorno: a Moab, ai figliuoli d'Ammon, a Edom, ai re di Tsoba e ai Filistei; e dovunque si volgeva, vinceva.
\par 48 Spiegò il suo valore, sconfisse gli Amalekiti, e liberò Israele dalle mani di quelli che lo predavano.
\par 49 I figliuoli di Saul erano: Gionathan, Ishvi e Malkishua; e delle sue due figliuole, la primogenita si chiamava Merab, e la minore Mical.
\par 50 Il nome della moglie di Saul era Ahinoam, figliuola di Ahimaaz, e il nome del capitano del suo esercito era Abner, figliuolo di Ner, zio di Saul.
\par 51 E Kis, padre di Saul, e Ner, padre d'Abner, erano figliuoli d'Abiel.
\par 52 Per tutto il tempo di Saul, vi fu guerra accanita contro i Filistei; e, come Saul scorgeva un uomo forte e valoroso, lo prendeva seco.

\chapter{15}

\par 1 Or Samuele disse a Saul: 'L'Eterno mi ha mandato per ungerti re del suo popolo, d'Israele; ascolta dunque quel che ti dice l'Eterno.
\par 2 Così parla l'Eterno degli eserciti: Io ricordo ciò che Amalek fece ad Israele quando gli s'oppose nel viaggio mentre saliva dall'Egitto.
\par 3 Ora va', sconfiggi Amalek, vota allo sterminio tutto ciò che gli appartiene; non lo risparmiare, ma uccidi uomini e donne, fanciulli e lattanti, buoi e pecore, cammelli ed asini'.
\par 4 Saul dunque convocò il popolo e ne fece la rassegna in Telaim: erano duecentomila fanti e diecimila uomini di Giuda.
\par 5 Saul giunse alla città di Amalek, pose un'imboscata nella valle,
\par 6 e disse ai Kenei: 'Andatevene, ritiratevi, scendete di mezzo agli Amalekiti, perch'io non vi distrugga insieme a loro, giacché usaste benignità verso tutti i figliuoli d'Israele quando salirono dall'Egitto'. Così i Kenei si ritirarono di mezzo agli Amalekiti.
\par 7 E Saul sconfisse gli Amalekiti da Havila fino a Shur, che sta dirimpetto all'Egitto.
\par 8 E prese vivo Agag, re degli Amalekiti, e votò allo sterminio tutto il popolo, passandolo a fil di spada.
\par 9 Ma Saul e il popolo risparmiarono Agag e il meglio delle pecore, de' buoi, gli animali della seconda figliatura, gli agnelli e tutto quel che v'era di buono; non vollero votarli allo sterminio, ma votarono allo sterminio tutto ciò che non avea valore ed era meschino.
\par 10 Allora la parola dell'Eterno fu rivolta a Samuele, dicendo:
\par 11 'Io mi pento d'aver stabilito re Saul, perché si è sviato da me, e non ha eseguito i miei ordini'. Samuele ne fu irritato, e gridò all'Eterno tutta la notte.
\par 12 Poi si levò la mattina di buon'ora e andò incontro a Saul; e vennero a dire a Samuele: 'Saul è andato a Carmel, ed ecco che vi s'è eretto un trofeo; poi se n'è ritornato e, passando più lungi, è sceso a Ghilgal'.
\par 13 Samuele si recò da Saul; e Saul gli disse: 'Benedetto sii tu dall'Eterno! Io ho eseguito l'ordine dell'Eterno'.
\par 14 E Samuele disse: 'Che è dunque questo belar di pecore che mi giunge agli orecchi, e questo muggir di buoi che sento?'
\par 15 Saul rispose: 'Son bestie menate dal paese degli Amalekiti; perché il popolo ha risparmiato il meglio delle pecore e de' buoi per farne de' sacrifizi all'Eterno, al tuo Dio; il resto, però, l'abbiam votato allo sterminio'.
\par 16 Allora Samuele disse a Saul: 'Basta! Io t'annunzierò quel che l'Eterno m'ha detto stanotte!' E Saul gli disse: 'Parla'.
\par 17 E Samuele disse: 'Non è egli vero che quando ti reputavi piccolo sei divenuto capo delle tribù d'Israele, e l'Eterno t'ha unto re d'Israele?
\par 18 L'Eterno t'avea dato una missione, dicendo: - Va', vota allo sterminio que' peccatori d'Amalekiti, e fa' loro guerra finché siano sterminati. -
\par 19 E perché dunque non hai ubbidito alla voce dell'Eterno? Perché ti sei gettato sul bottino, e hai fatto ciò ch'è male agli occhi dell'Eterno?'
\par 20 E Saul disse a Samuele: 'Ma io ho ubbidito alla voce dell'Eterno, ho compiuto la missione che l'Eterno m'aveva affidata, ho menato Agag, re di Amalek, e ho votato allo sterminio gli Amalekiti;
\par 21 ma il popolo ha preso, fra il bottino, delle pecore e de' buoi come primizie di ciò che doveva essere sterminato, per farne de' sacrifizi all'Eterno, al tuo Dio, a Ghilgal'.
\par 22 E Samuele disse: 'L'Eterno ha egli a grado gli olocausti e i sacrifizi come che si ubbidisca alla sua voce? Ecco, l'ubbidienza val meglio che il sacrifizio, e dare ascolto val meglio che il grasso dei montoni;
\par 23 poiché la ribellione è come il peccato della divinazione, e l'ostinatezza è come l'adorazione degli idoli e degli dèi domestici. Giacché tu hai rigettata la parola dell'Eterno, anch'egli ti rigetta come re'.
\par 24 Allora Saul disse a Samuele: 'Io ho peccato, poiché ho trasgredito il comandamento dell'Eterno e le tue parole; io ho temuto il popolo, e ho dato ascolto alla sua voce.
\par 25 Or dunque, ti prego, perdona il mio peccato, ritorna con me, e io mi prostrerò davanti all'Eterno'. E Samuele disse a Saul:
\par 26 'Io non ritornerò con te, poiché hai rigettato la parola dell'Eterno, e l'Eterno ha rigettato te perché tu non sia più re sopra Israele'.
\par 27 E come Samuele si voltava per andarsene, Saul lo prese per il lembo del mantello, che si strappò.
\par 28 Allora Samuele gli disse: 'L'Eterno strappa oggi d'addosso a te il regno d'Israele, e lo dà ad un altro, ch'è migliore di te.
\par 29 E colui ch'è la gloria d'Israele non mentirà e non si pentirà; poiché egli non è un uomo perché abbia da pentirsi'.
\par 30 Allora Saul disse: 'Ho peccato; ma tu adesso onorami, ti prego, in presenza degli anziani del mio popolo e in presenza d'Israele; ritorna con me, ed io mi prostrerò davanti all'Eterno, al tuo Dio'.
\par 31 Samuele dunque ritornò, seguendo Saul, e Saul si prostrò davanti all'Eterno.
\par 32 Poi Samuele disse: 'Menatemi qua Agag, re degli Amalekiti'. E Agag venne a lui incatenato. E Agag diceva: 'Certo, l'amarezza della morte è passata'.
\par 33 Samuele gli disse: 'Come la tua spada ha privato le donne di figliuoli, così la madre tua sarà privata di figliuoli fra le donne'. E Samuele fe' squartare Agag in presenza dell'Eterno a Ghilgal.
\par 34 Poi Samuele se ne andò a Rama, e Saul salì a casa sua, a Ghibea di Saul.
\par 35 E Samuele, finché visse, non andò più a vedere Saul, perché Samuele faceva cordoglio per Saul; e l'Eterno si pentiva d'aver fatto Saul re d'Israele.

\chapter{16}

\par 1 L'Eterno disse a Samuele: 'Fino a quando farai tu cordoglio per Saul, mentre io l'ho rigettato perché non regni più sopra Israele? Empi d'olio il tuo corno, e va'; io ti manderò da Isai di Bethlehem, perché mi son provveduto di un re tra i suoi figliuoli'.
\par 2 E Samuele rispose: 'Come andrò io? Saul lo verrà a sapere, e mi ucciderà'. L'Eterno disse: 'Prenderai teco una giovenca, e dirai: - Son venuto ad offrire un sacrifizio all'Eterno. -
\par 3 Inviterai Isai al sacrifizio; io ti farò sapere quello che dovrai fare, e mi ungerai colui che ti dirò'.
\par 4 Samuele dunque fece quello che l'Eterno gli avea detto; si recò a Bethlehem, e gli anziani della città gli si fecero incontro tutti turbati, e gli dissero: 'Porti tu pace?'
\par 5 Ed egli rispose: 'Porto pace; vengo ad offrire un sacrifizio all'Eterno; purificatevi, e venite meco al sacrifizio'. Fece anche purificare Isai e i suoi figliuoli, e li invitò al sacrifizio.
\par 6 Mentre entravano, egli scòrse Eliab, e disse: 'Certo, ecco l'unto dell'Eterno davanti a lui'.
\par 7 Ma l'Eterno disse a Samuele: 'Non badare al suo aspetto né all'altezza della sua statura, perché io l'ho scartato; giacché l'Eterno non guarda a quello a cui guarda l'uomo: l'uomo riguarda all'apparenza, ma l'Eterno riguarda al cuore'.
\par 8 Allora Isai chiamò Abinadab, e lo fece passare davanti a Samuele; ma Samuele disse: 'L'Eterno non s'è scelto neppure questo'.
\par 9 Isai fece passare Shamma, ma Samuele disse: 'L'Eterno non s'è scelto neppur questo'.
\par 10 Isai fece passar così sette de' suoi figliuoli davanti a Samuele; ma Samuele disse ad Isai: 'L'Eterno non s'è scelto questi'.
\par 11 Poi Samuele disse ad Isai: 'Sono questi tutti i tuoi figli?' Isai rispose: 'Resta ancora il più giovane, ma è a pascere le pecore'.
\par 12 E Samuele disse ad Isai: 'Mandalo a cercare, perché non ci metteremo a tavola prima che sia arrivato qua'. Isai dunque lo mandò a cercare, e lo fece venire. Or egli era biondo, avea dei begli occhi e un bell'aspetto. E l'Eterno disse a Samuele: 'Lèvati, ungilo, perch'egli è desso'.
\par 13 Allora Samuele prese il corno dell'olio, e l'unse in mezzo ai suoi fratelli; e, da quel giorno in poi, lo spirito dell'Eterno investì Davide. E Samuele si levò e se ne andò a Rama.
\par 14 Or lo spirito dell'Eterno s'era ritirato da Saul, ch'era turbato da un cattivo spirito suscitato dall'Eterno.
\par 15 I servitori di Saul gli dissero: 'Ecco, un cattivo spirito suscitato da Dio, ti turba.
\par 16 Ordini ora il nostro signore ai tuoi servi che ti stanno dinanzi, di cercare un uomo che sappia sonar l'arpa; e quando il cattivo spirito suscitato da Dio t'investirà, quegli si metterà a sonare, e tu ne sarai sollevato'.
\par 17 Saul disse ai suoi servitori: 'Trovatemi un uomo che suoni bene e conducetemelo'.
\par 18 Allora uno de' domestici prese a dire: 'Ecco io ho veduto un figliuolo di Isai, il Bethlehemita, che sa sonar bene; è un uomo forte, valoroso, un guerriero, parla bene, è di bell'aspetto, e l'Eterno è con lui'.
\par 19 Saul dunque inviò de' messi a Isai per dirgli: 'Mandami Davide, tuo figliuolo, che è col gregge'.
\par 20 Ed Isai prese un asino carico di pane, un otre di vino, un capretto, e mandò tutto a Saul per mezzo di Davide suo figliuolo.
\par 21 Davide arrivò da Saul e si presentò a lui; ed ei gli pose grande affetto e lo fece suo scudiero.
\par 22 E Saul mandò a dire ad Isai: 'Ti prego, lascia Davide al mio servizio, poich'egli ha trovato grazia agli occhi miei'.
\par 23 Or quando il cattivo spirito suscitato da Dio investiva Saul, Davide pigliava l'arpa e si metteva a sonare; Saul si sentiva sollevato, stava meglio, e il cattivo spirito se n'andava da lui.

\chapter{17}

\par 1 Or i Filistei misero insieme i loro eserciti per combattere, si radunarono a Soco, che appartiene a Giuda, e si accamparono fra Soco e Azeka, a Efes-Dammim.
\par 2 Saul e gli uomini d'Israele si radunarono anch'essi, si accamparono nella valle dei terebinti, e si schierarono in battaglia contro ai Filistei.
\par 3 I Filistei stavano sul monte da una parte, e Israele stava sul monte dall'altra parte; e fra loro c'era la valle.
\par 4 Or dal campo dei Filistei uscì come campione un guerriero per nome Goliath, di Gath, alto sei cubiti e un palmo.
\par 5 Aveva in testa un elmo di rame, era vestito d'una corazza a squame il cui peso era di cinquemila sicli di rame,
\par 6 portava delle gambiere di rame e, sospeso dietro le spalle, un giavellotto di rame.
\par 7 L'asta della sua lancia era come un subbio di tessitore; la punta della lancia pesava seicento sicli di ferro, e colui che portava la sua targa lo precedeva.
\par 8 Egli dunque si fermò; e, vòlto alle schiere d'Israele, gridò: 'Perché uscite a schierarvi in battaglia? Non sono io il Filisteo, e voi de' servi di Saul? Scegliete uno fra voi, e scenda contro a me.
\par 9 S'egli potrà lottare con me ed uccidermi, noi sarem vostri servi; ma se io sarò vincitore e l'ucciderò, voi sarete nostri sudditi e ci servirete'.
\par 10 E il Filisteo aggiunse: 'Io lancio oggi questa sfida a disonore delle schiere d'Israele: Datemi un uomo, e ci batteremo!'
\par 11 Quando Saul e tutto Israele udirono queste parole del Filisteo, rimasero sbigottiti e presi da gran paura.
\par 12 Or Davide era figliuolo di quell'Efrateo di Bethlehem di Giuda, per nome Isai, che aveva otto figliuoli e che al tempo di Saul, era vecchio, molto innanzi nell'età.
\par 13 I tre figliuoli maggiori d'Isai erano andati alla guerra con Saul; e i tre figliuoli ch'erano andati alla guerra, si chiamavano: Eliab, il primogenito; Abinadab il secondo, e Shamma il terzo.
\par 14 Davide era il più giovine; e quando i tre maggiori ebbero seguito Saul,
\par 15 Davide si partì da Saul, e tornò a Bethlehem a pascolar le pecore di suo padre.
\par 16 E il Filisteo si faceva avanti la mattina e la sera, e si presentò così per quaranta giorni.
\par 17 Or Isai disse a Davide, suo figliuolo: 'Prendi per i tuoi fratelli quest'efa di grano arrostito e questi dieci pani, e portali presto al campo ai tuoi fratelli.
\par 18 Porta anche queste dieci caciole al capitano del loro migliaio; vedi se i tuoi fratelli stanno bene, e riportami da loro un qualche contrassegno.
\par 19 Saul ed essi con tutti gli uomini d'Israele sono nella valle dei terebinti a combattere contro i Filistei'.
\par 20 L'indomani Davide s'alzò di buon mattino, lasciò le pecore a un guardiano, prese il suo carico, e partì come Isai gli aveva ordinato; e come giunse al parco dei carri, l'esercito usciva per schierarsi in battaglia e alzava gridi di guerra.
\par 21 Israeliti e Filistei s'erano schierati, esercito contro esercito.
\par 22 Davide, lasciate in mano del guardiano de' bagagli le cose che portava, corse alla linea di battaglia; e, giuntovi, chiese ai suoi fratelli come stavano.
\par 23 Or mentr'egli parlava con loro, ecco avanzarsi di tra le file de' Filistei quel campione, quel Filisteo di Gath, di nome Goliath, e ripetere le solite parole; e Davide le udì.
\par 24 E tutti gli uomini d'Israele, alla vista di quell'uomo, fuggiron d'innanzi a lui, presi da gran paura.
\par 25 Gli uomini d'Israele dicevano: 'Avete visto quell'uomo che s'avanza? Egli s'avanza per coprir d'obbrobrio Israele. Se qualcuno l'uccide, il re lo farà grandemente ricco, gli darà la sua propria figliuola, ed esenterà in Israele la casa del padre di lui da ogni aggravio'.
\par 26 E Davide, rivolgendosi a quelli che gli eran vicini, disse: 'Che si farà egli a quell'uomo che ucciderà questo Filisteo e torrà l'obbrobrio di dosso a Israele? E chi è dunque questo Filisteo, questo incirconciso, che osa insultare le schiere dell'Iddio vivente?'
\par 27 E la gente gli rispose con le stesse parole, dicendo: 'Si farà questo e questo a colui che lo ucciderà'.
\par 28 Eliab, suo fratello maggiore, avendo udito Davide parlare a quella gente, s'accese d'ira contro di lui, e disse: 'Perché sei sceso qua? E a chi hai lasciato quelle poche pecore nel deserto? Io conosco il tuo orgoglio e la malignità del tuo cuore; tu sei sceso qua per veder la battaglia?'
\par 29 Davide rispose: 'Che ho io fatto ora? Non era che una semplice domanda!'
\par 30 E, scostandosi da lui, si rivolse ad un altro, facendo la stessa domanda; e la gente gli diede la stessa risposta di prima.
\par 31 Or le parole che Davide avea dette essendo state sentite, furon riportate a Saul, che lo fece venire.
\par 32 E Davide disse a Saul: 'Nessuno si perda d'animo a motivo di costui! Il tuo servo andrà e si batterà con quel Filisteo'.
\par 33 Saul disse a Davide: 'Tu non puoi andare a batterti con questo Filisteo; poiché tu non sei che un giovanetto, ed egli è un guerriero fin dalla sua giovinezza'.
\par 34 E Davide rispose a Saul: 'Il tuo servo pascolava il gregge di suo padre; e quando un leone o un orso veniva a portar via una pecora di mezzo al gregge,
\par 35 io gli correvo dietro, lo colpivo, gli strappavo dalle fauci la preda; e se quello mi si rivoltava contro, io lo pigliavo per le ganasce, lo ferivo e l'ammazzavo.
\par 36 Sì, il tuo servo ha ucciso il leone e l'orso; e questo incirconciso Filisteo sarà come uno di quelli, perché ha coperto d'obbrobrio le schiere dell'Iddio vivente'.
\par 37 E Davide soggiunse: 'L'Eterno che mi liberò dalla zampa del leone e dalla zampa dell'orso, mi libererà anche dalla mano di questo Filisteo'. E Saul disse a Davide: 'Va', e l'Eterno sia teco'.
\par 38 Saul rivestì Davide della sua propria armatura, gli mise in capo un elmo di rame, e lo armò di corazza.
\par 39 Poi Davide cinse la spada di Saul sopra la sua armatura, e cercò di camminare, perché non aveva ancora provato; ma disse a Saul: 'Io non posso camminare con quest'armatura; non ci sono abituato'. E se la tolse di dosso.
\par 40 E prese in mano il suo bastone, si scelse nel torrente cinque pietre ben lisce, le pose nella sacchetta da pastore, che gli serviva di carniera, e con la fionda in mano mosse contro il Filisteo.
\par 41 Il Filisteo anch'egli si fe' innanzi, avvicinandosi sempre più a Davide, ed era preceduto dal suo scudiero.
\par 42 E quando il Filisteo ebbe scòrto Davide, lo disprezzò, perch'egli non era che un giovinetto, biondo e di bell'aspetto.
\par 43 Il Filisteo disse a Davide: 'Sono io un cane, che tu vieni contro a me col bastone?' E il Filisteo maledisse Davide in nome de' suoi dèi;
\par 44 e il Filisteo disse a Davide: 'Vieni qua ch'io dia la tua carne agli uccelli del cielo e alle bestie de' campi'.
\par 45 Allora Davide rispose al Filisteo: 'Tu vieni a me con la spada, con la lancia e col giavellotto; ma io vengo a te nel nome dell'Eterno degli eserciti, dell'Iddio delle schiere d'Israele che tu hai insultato.
\par 46 Oggi l'Eterno ti darà nelle mie mani, e io ti abbatterò, ti taglierò la testa, e darò oggi stesso i cadaveri dell'esercito de' Filistei agli uccelli del cielo e alle fiere della terra; e tutta la terra riconoscerà che v'è un Dio in Israele;
\par 47 e tutta questa moltitudine riconoscerà che l'Eterno non salva per mezzo di spada né per mezzo di lancia; poiché l'esito della battaglia dipende dall'Eterno, ed egli vi darà nelle nostre mani'.
\par 48 E come il Filisteo si mosse e si fe' innanzi per accostarsi a Davide, Davide anch'egli corse prestamente verso la linea di battaglia incontro al Filisteo;
\par 49 mise la mano nella sacchetta, ne cavò una pietra, la lanciò con la fionda, e colpì il Filisteo nella fronte; la pietra gli si conficcò nella fronte, ed ei cadde bocconi per terra.
\par 50 Così Davide, con una fionda e con una pietra, vinse il Filisteo; lo colpì e l'uccise, senz'aver spada alla mano.
\par 51 Poi Davide corse, si gettò sul Filisteo, gli prese la spada e, sguainatala, lo mise a morte e gli tagliò la testa. E i Filistei, vedendo che il loro eroe era morto, si diedero alla fuga.
\par 52 E gli uomini d'Israele e di Giuda sorsero, alzando gridi di guerra, e inseguirono i Filistei fino all'ingresso di Gath e alle porte di Ekron. I Filistei feriti a morte caddero sulla via di Shaaraim, fino a Gath e fino ad Ekron.
\par 53 E i figliuoli d'Israele, dopo aver dato la caccia ai Filistei, tornarono e predarono il loro campo.
\par 54 E Davide prese la testa del Filisteo, la portò a Gerusalemme, ma ripose l'armatura di lui nella sua tenda.
\par 55 Or quando Saul avea veduto Davide che andava contro il Filisteo, avea chiesto ad Abner, capo dell'esercito: 'Abner, di chi è figliuolo questo giovinetto?' E Abner avea risposto: 'Com'è vero che tu vivi, o re, io non lo so'.
\par 56 E il re avea detto: 'Informati di chi sia figliuolo questo ragazzo'.
\par 57 Or quando Davide, ucciso il Filisteo, fu di ritorno, Abner lo prese e lo menò alla presenza di Saul, avendo egli in mano la testa del Filisteo.
\par 58 E Saul gli disse: 'Giovinetto, di chi sei tu figliuolo?' Davide rispose: 'Sono figliuolo del tuo servo Isai di Bethlehem'.

\chapter{18}

\par 1 Come Davide ebbe finito di parlare con Saul, l'anima di Gionathan rimase così legata all'anima di lui, che Gionathan l'amò come l'anima sua.
\par 2 Da quel giorno Saul lo tenne presso di sé e non permise più ch'ei se ne tornasse a casa di suo padre.
\par 3 E Gionathan fece alleanza con Davide, perché lo amava come l'anima propria.
\par 4 Quindi Gionathan si tolse di dosso il mantello, e lo diede a Davide; e così fece delle sue vesti, fino alla sua spada, al suo arco e alla sua cintura.
\par 5 E Davide andava e riusciva bene dovunque Saul lo mandava: Saul lo mise a capo della gente di guerra, ed egli era gradito a tutto il popolo, anche ai servi di Saul.
\par 6 Or all'arrivo dell'esercito, quando Davide, ucciso il Filisteo, facea ritorno, le donne uscirono da tutte le città d'Israele incontro al re Saul, cantando e danzando al suon de' timpani e de' triangoli, e alzando grida di gioia;
\par 7 e le donne, danzando, si rispondevano a vicenda e dicevano: Saul ha ucciso i suoi mille, e Davide i suoi diecimila.
\par 8 Saul n'ebbe sdegno fortissimo; quelle parole gli dispiacquero, e disse: 'Ne dànno diecimila a Davide, e a me non ne dan che mille! Non gli manca più che il regno!'
\par 9 E Saul, da quel giorno in poi, guardò Davide di mal occhio.
\par 10 Il giorno dopo, un cattivo spirito, suscitato da Dio, s'impossessò di Saul che era come fuori di sé in mezzo alla casa, mentre Davide sonava l'arpa, come solea fare tutti i giorni. Saul aveva in mano la sua lancia;
\par 11 e la scagliò, dicendo: 'Inchioderò Davide al muro!' Ma Davide schivò il colpo per due volte.
\par 12 Saul avea paura di Davide, perché l'Eterno era con lui e s'era ritirato da Saul;
\par 13 perciò Saul lo allontanò da sé, e lo fece capitano di mille uomini; ed egli andava e veniva alla testa del popolo.
\par 14 Or Davide riusciva bene in tutte le sue imprese, e l'Eterno era con lui.
\par 15 E quando Saul vide ch'egli riusciva splendidamente, cominciò ad aver timore di lui;
\par 16 ma tutto Israele e Giuda amavano Davide, perché andava e veniva alla loro testa.
\par 17 Saul disse a Davide: 'Ecco Merab, la mia figliuola maggiore; io te la darò per moglie; solo siimi valente, e combatti le battaglie dell'Eterno'. Or Saul diceva tra sé: 'Non sia la mia mano che lo colpisca, ma sia la mano de' Filistei'.
\par 18 Ma Davide rispose a Saul: 'Chi son io, che è la vita mia, e che è la famiglia di mio padre in Israele, ch'io abbia ad essere genero del re?'
\par 19 Or avvenne che, quando Merab figliuola di Saul doveva esser data a Davide, fu invece sposata ad Adriel di Mehola.
\par 20 Ma Mical, figliuola di Saul, amava Davide; lo riferirono a Saul, e la cosa gli piacque.
\par 21 E Saul disse: 'Gliela darò, perché sia per lui un'insidia ed egli cada sotto la mano de' Filistei'. Saul dunque disse a Davide: 'Oggi, per la seconda volta, tu puoi diventar mio genero'.
\par 22 Poi Saul diede quest'ordine ai suoi servitori: 'Parlate in confidenza a Davide, e ditegli: Ecco, tu sei in grazia del re, e tutti i suoi servi ti amano, diventa dunque genero del re'.
\par 23 I servi di Saul ridissero queste parole a Davide. Ma Davide replicò: 'Sembra a voi cosa lieve il diventar genero del re? E io son povero e di basso stato'.
\par 24 I servi riferirono a Saul: 'Davide ha risposto così e così'.
\par 25 E Saul disse: 'Dite così a Davide: Il re non domanda dote; ma domanda cento prepuzi di Filistei, per trar vendetta de' suoi nemici'. Or Saul aveva in animo di far cadere Davide nelle mani de' Filistei.
\par 26 I servitori dunque riferirono quelle parole a Davide, e a Davide piacque di diventar in tal modo genero del re. E prima del termine fissato,
\par 27 Davide si levò, partì con la sua gente, uccise duecento uomini de' Filistei, portò i loro prepuzi e ne consegnò il numero preciso al re, per diventar suo genero.
\par 28 E Saul gli diede per moglie Mical, sua figliuola. E Saul vide e riconobbe che l'Eterno era con Davide; e Mical, figliuola di Saul, l'amava.
\par 29 E Saul continuò più che mai a temer Davide, e gli fu sempre nemico.
\par 30 Or i principi de' Filistei uscivano a combattere; e ogni volta che uscivano, Davide riusciva meglio di tutti i servi di Saul, in guisa che il suo nome divenne molto famoso.

\chapter{19}

\par 1 Saul parlò a Gionathan, suo figliuolo, e a tutti i suoi servi di far morire Davide. Ma Gionathan, figliuolo di Saul, che voleva gran bene a Davide,
\par 2 informò Davide della cosa e gli disse: 'Saul, mio padre, cerca di farti morire; or dunque, ti prego, sta' in guardia domattina, tienti in luogo segreto e nasconditi.
\par 3 Io uscirò, e mi terrò allato a mio padre, nel campo ove tu sarai; parlerò di te a mio padre, vedrò come vanno le cose, e te lo farò sapere'.
\par 4 Gionathan dunque parlò a Saul, suo padre, in favore di Davide, e gli disse: 'Non pecchi il re contro al suo servo, contro a Davide, giacché ei non ha peccato contro a te, e anzi l'opera sua t'è stata di grande utilità.
\par 5 Egli ha messo la propria vita a repentaglio, ha ucciso il Filisteo, e l'Eterno ha operato una grande liberazione a pro di tutto Israele. Tu l'hai veduto, e te ne sei rallegrato; perché dunque peccheresti tu contro il sangue innocente facendo morir Davide senza ragione?'
\par 6 Saul diè ascolto alla voce di Gionathan, e fece questo giuramento: 'Com'è vero che l'Eterno vive, egli non sarà fatto morire!'
\par 7 Allora Gionathan chiamò Davide e gli riferì tutto questo. Poi Gionathan ricondusse Davide da Saul, al servizio del quale egli rimase come prima.
\par 8 Ricominciò di nuovo la guerra; e Davide uscì a combattere contro i Filistei, inflisse loro una grave sconfitta, e quelli fuggirono d'innanzi a lui.
\par 9 E uno spirito cattivo, suscitato dall'Eterno, s'impossessò di Saul. Egli sedeva in casa sua avendo in mano una lancia; e Davide stava sonando l'arpa.
\par 10 E Saul cercò d'inchiodar Davide al muro con la lancia, ma Davide schivò il colpo, e la lancia diè nel muro. Davide fuggì e si mise in salvo in quella stessa notte.
\par 11 Saul inviò de' messi a casa di Davide per tenerlo d'occhio e farlo morire la mattina dipoi; ma Mical, moglie di Davide, lo informò della cosa, dicendo: 'Se in questa stessa notte non ti salvi la vita, domani sei morto'.
\par 12 E Mical calò Davide per una finestra; ed egli se ne andò, fuggì, e si mise in salvo.
\par 13 Poi Mical prese l'idolo domestico e lo pose nel letto; gli mise in capo un cappuccio di pelo di capra, e lo coperse d'un mantello.
\par 14 E quando Saul inviò de' messi a pigliar Davide, ella disse: 'È malato'.
\par 15 Allora Saul inviò di nuovo i messi perché vedessero Davide, e disse loro: 'Portatemelo nel letto, perch'io lo faccia morire'.
\par 16 E quando giunsero i messi, ecco che nel letto c'era l'idolo domestico con in capo un cappuccio di pel di capra.
\par 17 E Saul disse a Mical: 'Perché mi hai ingannato così e hai dato campo al mio nemico di fuggire?' E Mical rispose a Saul: 'È lui che mi ha detto: Lasciami andare; altrimenti, t'ammazzo!'
\par 18 Davide dunque fuggì, si pose in salvo, e venne da Samuele a Rama, e gli raccontò tutto quello che Saul gli avea fatto. Poi, egli e Samuele andarono a stare a Naioth.
\par 19 Questo fu riferito a Saul, dicendo: 'Ecco, Davide è a Naioth, presso Rama'.
\par 20 E Saul inviò dei messi per pigliar Davide; ma quando questi videro l'adunanza de' profeti che profetavano, con Samuele che tenea la presidenza, lo spirito di Dio investì i messi di Saul che si misero anch'essi a profetare.
\par 21 Ne informarono Saul, che inviò altri messi, i quali pure si misero a profetare. Saul ne mandò ancora per la terza volta, e anche questi si misero a profetare.
\par 22 Allora si recò egli stesso a Rama; e, giunto alla gran cisterna ch'è a Secu, chiese: 'Dove sono Samuele e Davide?' Gli fu risposto: 'Ecco, sono a Naioth, presso Rama'.
\par 23 Egli andò dunque là, a Naioth, presso Rama; e lo spirito di Dio investì anche lui; ed egli continuò il suo viaggio, profetando, finché giunse a Naioth, presso Rama.
\par 24 E anch'egli si spogliò delle sue vesti, anch'egli profetò in presenza di Samuele, e giacque nudo per terra tutto quel giorno e tutta quella notte. Donde il detto: 'Saul è anch'egli tra i profeti?'

\chapter{20}

\par 1 Davide fuggì a Naioth presso Rama, andò a trovare Gionathan, e gli disse: 'Che ho mai fatto? Qual è il mio delitto, qual è il mio peccato verso tuo padre, ch'egli vuole la mia vita?'
\par 2 Gionathan gli rispose: 'Tolga ciò Iddio! tu non morrai; ecco, mio padre non fa cosa alcuna o grande o piccola, senza farmene parte; e perché mi celerebbe egli questa? Non è possibile'.
\par 3 Ma Davide replicò, giurando: 'Tuo padre sa molto bene che io ho trovato grazia agli occhi tuoi; perciò avrà detto: Gionathan non sappia questo, affinché non ne abbia dispiacere; ma com'è vero che l'Eterno vive e che vive l'anima tua, fra me e la morte non v'ha che un passo'.
\par 4 Gionathan disse a Davide: 'Che desideri tu ch'io ti faccia?'
\par 5 Davide rispose a Gionathan: 'Ecco, domani è la luna nuova, e io dovrei sedermi a mensa col re; lasciami andare, e mi nasconderò per la campagna fino alla sera del terzo giorno.
\par 6 Se tuo padre nota la mia assenza, tu gli dirai: - Davide mi ha pregato istantemente di poter dare una corsa fino a Bethlehem, sua città, perché v'è il sacrifizio annuo per tutta la sua famiglia -.
\par 7 S'egli dice: - Sta bene - il tuo servo avrà pace; ma, se si adira, sappi che il male che mi vuol fare è deciso.
\par 8 Mostra dunque la tua bontà verso il tuo servo, giacché hai fatto entrare il tuo servo in un patto con te nel nome dell'Eterno; ma, se v'è in me qualche iniquità, dammi la morte tu; perché mi meneresti da tuo padre?'
\par 9 Gionathan disse: 'Lungi da te questo pensiero! S'io venissi a sapere che il male è deciso da parte di mio padre e sta per venirti addosso, non te lo farei io sapere?'
\par 10 Davide disse a Gionathan: 'Chi m'informerà, nel caso che tuo padre ti dia una risposta dura?'
\par 11 E Gionathan disse a Davide: 'Vieni, andiamo fuori alla campagna!' E andarono ambedue fuori alla campagna.
\par 12 E Gionathan disse a Davide: 'L'Eterno, l'Iddio d'Israele, mi sia testimonio! Quando domani o posdomani, a quest'ora, io avrò scandagliato mio padre, s'egli è ben disposto verso Davide, ed io non mando a fartelo sapere, l'Eterno tratti Gionathan con tutto il suo rigore!
\par 13 Nel caso poi che piaccia a mio padre di farti del male, te lo farò sapere, e ti lascerò partire perché tu te ne vada in pace; e l'Eterno sia teco, com'è stato con mio padre!
\par 14 E se sarò ancora in vita, non è egli vero? tu agirai verso di me con la bontà dell'Eterno, ond'io non sia messo a morte;
\par 15 e non cesserai mai d'esser buono verso la mia casa, neppur quando l'Eterno avrà sterminato di sulla faccia della terra fino all'ultimo i nemici di Davide'.
\par 16 Così Gionathan strinse alleanza con la casa di Davide, dicendo: 'L'Eterno faccia vendetta dei nemici di Davide!'
\par 17 E, per l'amore che gli portava, Gionathan fece di nuovo giurar Davide; perch'egli l'amava come l'anima propria.
\par 18 Poi Gionathan gli disse: 'Domani è la nuova luna, e la tua assenza sarà notata, perché il tuo posto sarà vuoto.
\par 19 Domani l'altro dunque tu scenderai giù fino al luogo dove ti nascondesti il giorno del fatto, e rimarrai presso la pietra di Ezel.
\par 20 Io tirerò tre frecce da quel lato, come se tirassi a segno.
\par 21 Poi subito manderò il mio ragazzo, dicendogli: - Va' a cercare le frecce. - Se dico al ragazzo: - Guarda, le frecce son di qua da te, prendile! - tu allora vieni, perché tutto va bene per te, e non hai nulla da temere, come l'Eterno vive!
\par 22 Ma se dico al giovanetto: - Guarda, le frecce son di là da te - allora vattene, perché l'Eterno vuol che tu parta.
\par 23 Quanto a quello che abbiam convenuto fra noi, fra me e te, ecco, l'Eterno n'è testimonio in perpetuo'.
\par 24 Davide dunque si nascose nella campagna; e quando venne il novilunio, il re si pose a sedere a mensa per il pasto.
\par 25 Il re, come al solito, si pose a sedere sulla sua sedia ch'era vicina al muro; Gionathan s'alzò per porsi di faccia, Abner si assise accanto a Saul, ma il posto di Davide rimase vuoto.
\par 26 Nondimeno Saul non disse nulla quel giorno, perché pensava: 'Gli è successo qualcosa; ei non dev'esser puro; per certo ei non è puro'.
\par 27 Ma l'indomani, secondo giorno della luna nuova, il posto di Davide era ancora vuoto; e Saul disse a Gionathan, suo figliuolo: 'Perché il figliuolo d'Isai non è venuto a mangiare né ieri né oggi?'
\par 28 Gionathan rispose a Saul: 'Davide m'ha chiesto istantemente di lasciarlo andare a Bethlehem;
\par 29 e ha detto: - Ti prego, lasciami andare, perché abbiamo in città un sacrifizio di famiglia, e il mio fratello mi ha raccomandato d'andarvi; ora dunque, se ho trovato grazia agli occhi tuoi, ti prego, lasciami dare una corsa per vedere i miei fratelli. - Per questa ragione egli non è venuto alla mensa del re'.
\par 30 Allora l'ira di Saul s'accese contro Gionathan, ed ei gli disse: 'Figliuolo perverso e ribelle, non lo so io forse che tieni le parti del figliuol d'Isai, a tua vergogna ed a vergogna del seno di tua madre?
\par 31 Poiché, fino a tanto che il figliuol d'Isai avrà vita sulla terra, non vi sarà stabilità né per te né per il tuo regno. Or dunque mandalo a cercare e fallo venire da me, perché deve morire'.
\par 32 Gionathan rispose a Saul suo padre e gli disse: 'Perché dovrebb'egli morire? Che ha fatto?'
\par 33 E Saul brandì la lancia contro a lui per colpirlo. Allora Gionathan riconobbe che suo padre avea deciso di far morire Davide.
\par 34 E, acceso d'ira, si levò da mensa, e non mangiò nulla il secondo giorno della luna nuova, addolorato com'era per l'onta che suo padre avea fatta a Davide.
\par 35 La mattina dopo, Gionathan uscì fuori alla campagna, al luogo fissato con Davide, ed avea seco un ragazzetto.
\par 36 E disse al suo ragazzo: 'Corri a cercare le frecce che tiro'. E, come il ragazzo correva, tirò una freccia che passò di là da lui.
\par 37 E quando il ragazzo fu giunto al luogo dov'era la freccia che Gionathan avea tirata, Gionathan gli gridò dietro: 'La freccia non è essa di là da te?'
\par 38 E Gionathan gridò ancora dietro al ragazzo: 'Via, fa' presto, non ti trattenere!' Il ragazzo di Gionathan raccolse le frecce, e tornò dal suo padrone.
\par 39 Or il ragazzo non sapeva nulla; Gionathan e Davide soli sapevano di che si trattasse.
\par 40 Gionathan diede le sue armi al suo ragazzo e gli disse: 'Va', portale alla città'.
\par 41 E come il ragazzo se ne fu andato, Davide si levò di dietro il mucchio di pietre, si gettò con la faccia a terra, e si prostrò tre volte; poi i due si baciarono l'un l'altro e piansero assieme; Davide soprattutto diè in pianto dirotto.
\par 42 E Gionathan disse a Davide: 'Va' in pace, ora che abbiam fatto ambedue questo giuramento nel nome dell'Eterno: L'Eterno sia testimonio fra me e te e fra la mia progenie e la progenie tua, in perpetuo'.
\par 43 Davide si levò e se ne andò, e Gionathan tornò in città.

\chapter{21}

\par 1 Davide andò a Nob dal sacerdote Ahimelec; e Ahimelec gli venne incontro tutto tremante, e gli disse: 'Perché sei solo e non hai alcuno teco?'
\par 2 Davide rispose al sacerdote Ahimelec: 'Il re m'ha dato un'incombenza, e m'ha detto: - Nessuno sappia nulla dell'affare per cui ti mando e dell'ordine che t'ho dato; - e quanto alla mia gente, le ho detto di trovarsi in un dato luogo.
\par 3 E ora che hai tu sotto mano? Dammi cinque pani o quel che si potrà trovare'.
\par 4 Il sacerdote rispose a Davide, dicendo: 'Non ho sotto mano del pane comune, ma c'è del pane consacrato; ma la tua gente s'è almeno astenuta da contatto con donne?'
\par 5 Davide rispose al sacerdote: 'Da che son partito, tre giorni fa, siamo rimasti senza donne; e quanto ai vasi della mia gente erano puri; e se anche la nostra incombenza è profana, essa sarà oggi santificata da quel che si porrà nei vasi'.
\par 6 Il sacerdote gli diè dunque del pane consacrato perché non v'era quivi altro pane tranne quello della presentazione, ch'era stato tolto d'innanzi all'Eterno, per mettervi invece del pan caldo nel momento in cui si toglieva l'altro.
\par 7 - Or quel giorno, un cert'uomo di tra i servi di Saul si trovava quivi, trattenuto in presenza dell'Eterno; si chiamava Doeg, era Edomita, e capo de' pastori di Saul.
\par 8 E Davide disse ad Ahimelec: 'Non hai tu qui disponibile una lancia o una spada? Perché io non ho preso meco né la mia spada né le mie armi, tanto premeva l'incombenza del re'.
\par 9 Il sacerdote rispose: 'C'è la spada di Goliath, il Filisteo, che tu uccidesti nella valle de' terebinti; è là involta in un panno dietro all'efod; se la vuoi prendere, prendila, perché qui non ve n'è altra fuori di questa'. E Davide disse: 'Nessuna è pari a quella; dammela!'
\par 10 Allora Davide si levò, e quel giorno fuggì per timore di Saul, e andò da Akis, re di Gath.
\par 11 E i servi del re dissero ad Akis: 'Non è questi Davide, il re del paese? Non è quegli colui del quale cantavan nelle loro danze: Saul ha uccisi i suoi mille, e Davide i suoi diecimila?'
\par 12 Davide si tenne in cuore queste parole, ed ebbe gran timore di Akis, re di Gath.
\par 13 Mutò il suo modo di fare in loro presenza, faceva il pazzo in mezzo a loro, tracciava de' segni sui battenti delle porte, e si lasciava scorrer la saliva sulla barba.
\par 14 E Akis disse ai suoi servi: 'Guardate, è un pazzo; perché me l'avete menato?
\par 15 Mi mancan forse de' pazzi, che m'avete condotto questo a fare il pazzo in mia presenza? Costui non entrerà in casa mia!'

\chapter{22}

\par 1 Or Davide si partì di là e si rifugiò nella spelonca di Adullam; e quando i suoi fratelli e tutta la famiglia di suo padre lo seppero, scesero quivi per unirsi a lui.
\par 2 E tutti quelli ch'erano in angustie, che avean dei debiti o che erano scontenti, si radunaron presso di lui, ed egli divenne loro capo, ed ebbe con sé circa quattrocento uomini.
\par 3 Di là Davide andò a Mitspa di Moab, e disse al re di Moab: 'Deh, permetti che mio padre e mia madre vengano a stare da voi, fino a tanto ch'io sappia quel che Iddio farà di me'.
\par 4 Egli dunque li condusse davanti al re di Moab, ed essi rimasero con lui tutto il tempo che Davide fu nella sua fortezza.
\par 5 E il profeta Gad disse a Davide: 'Non star più in questa fortezza; parti, e recati nel paese di Giuda'. Davide allora partì, e venne nella foresta di Hereth.
\par 6 Saul seppe che Davide e gli uomini ch'eran con lui erano stati scoperti. Saul si trovava allora a Ghibea, seduto sotto la tamerice, ch'è sull'altura; aveva in mano la lancia, e tutti i suoi servi gli stavano attorno.
\par 7 E Saul disse ai servi che gli stavano intorno: 'Ascoltate ora, Beniaminiti! Il figliuolo d'Isai vi darà egli forse a tutti dei campi e delle vigne? Farà egli di tutti voi de' capi di migliaia e de' capi di centinaia,
\par 8 che avete tutti congiurato contro di me, e non v'è alcuno che m'abbia informato dell'alleanza che il mio figliuolo ha fatta col figliuolo d'Isai, e non v'è alcuno di voi che mi compianga e m'informi che il mio figliuolo ha sollevato contro di me il mio servo perché mi tenda insidie come fa oggi?'
\par 9 E Doeg, l'Idumeo, il quale era preposto ai servi di Saul, rispose e disse: 'Io vidi il figliuolo d'Isai venire a Nob da Ahimelec, figliuolo di Ahitub,
\par 10 il quale consultò l'Eterno per lui, gli diede dei viveri, e gli diede la spada di Goliath il Filisteo'.
\par 11 Allora il re mandò a chiamare il sacerdote Ahimelec, figliuolo di Ahitub, e tutta la famiglia del padre di lui, vale a dire i sacerdoti ch'erano a Nob. E tutti vennero al re.
\par 12 E Saul disse: 'Ora ascolta, o figliuolo di Ahitub!' Ed egli rispose: 'Eccomi, signor mio!'
\par 13 E Saul gli disse: 'Perché tu e il figliuolo d'Isai avete congiurato contro di me? Perché gli hai dato del pane e una spada; e hai consultato Dio per lui affinché insorga contro di me e mi tenda insidie come fa oggi?'
\par 14 Allora Ahimelec rispose al re, dicendo: 'E chi v'è dunque, fra tutti i tuoi servi, fedele come Davide, genero del re, pronto al tuo comando e onorato nella tua casa?
\par 15 Ho io forse cominciato oggi a consultare Iddio per lui? Lungi da me il pensiero di tradirti! Non imputi il re nulla di simile al suo servo o a tutta la famiglia di mio padre; perché il tuo servo non sa cosa alcuna, piccola o grande, di tutto questo'.
\par 16 Il re disse: 'Tu morrai senz'altro, Ahimelec, tu con tutta la famiglia del padre tuo!'
\par 17 E il re disse alle guardie che gli stavano attorno: 'Volgetevi e uccidete i sacerdoti dell'Eterno, perché anch'essi son d'accordo con Davide; sapevano ch'egli era fuggito, e non me ne hanno informato'. Ma i servi del re non vollero metter le mani addosso ai sacerdoti dell'Eterno.
\par 18 E il re disse a Doeg: 'Volgiti tu, e gettati sui sacerdoti!' E Doeg, l'Idumeo, si volse, si avventò addosso ai sacerdoti, e uccise in quel giorno ottantacinque persone che portavano l'efod di lino.
\par 19 E Saul mise pure a fil di spada Nob, la città de' sacerdoti, uomini, donne, fanciulli, bambini di latte, buoi, asini e pecore: tutto mise a fil di spada.
\par 20 Nondimeno, uno de' figliuoli di Ahimelec, figliuolo di Ahitub, di nome Abiathar, scampò e si rifugiò presso Davide.
\par 21 Abiathar riferì a Davide che Saul aveva ucciso i sacerdoti dell'Eterno.
\par 22 E Davide disse ad Abiathar: 'Io sapevo bene, quel giorno che Doeg, l'Idumeo, era là, ch'egli avrebbe senza dubbio avvertito Saul; io son causa della morte di tutte le persone della famiglia di tuo padre.
\par 23 Resta con me, non temere; chi cerca la mia vita cerca la tua; con me sarai al sicuro'.

\chapter{23}

\par 1 Or vennero a dire a Davide: 'Ecco, i Filistei hanno attaccato Keila e saccheggiano le aie'.
\par 2 E Davide consultò l'Eterno, dicendo: 'Andrò io a sconfiggere questi Filistei?' L'Eterno rispose a Davide: 'Va', sconfiggi i Filistei, e salva Keila'.
\par 3 Ma la gente di Davide gli disse: 'Tu vedi che qui in Giuda abbiam paura; e che sarà se andiamo a Keila contro le schiere de' Filistei?'
\par 4 Davide consultò di nuovo l'Eterno, e l'Eterno gli rispose e gli disse: 'Lèvati, scendi a Keila, perché io darò i Filistei nelle tue mani'.
\par 5 Davide dunque andò con la sua gente a Keila, combatté contro i Filistei, portò via il loro bestiame, e inflisse loro una grande sconfitta. Così Davide liberò gli abitanti di Keila.
\par 6 - Quando Abiathar, figliuolo di Ahimelec, si rifugiò presso Davide a Keila, portò seco l'efod. -
\par 7 Saul fu informato che Davide era giunto a Keila. E Saul disse: 'Iddio lo dà nelle mie mani, poiché è venuto a rinchiudersi in una città che ha porte e sbarre'.
\par 8 Saul dunque convocò tutto il popolo per andare alla guerra, per scendere a Keila e cinger d'assedio Davide e la sua gente.
\par 9 Ma Davide, avuta conoscenza che Saul gli macchinava del male, disse al sacerdote Abiathar: 'Porta qua l'efod'.
\par 10 Poi disse: 'O Eterno, Dio d'Israele, il tuo servo ha sentito come cosa certa che Saul cerca di venire a Keila per distruggere la città per causa mia.
\par 11 Quei di Keila mi daranno essi nelle sue mani? Saul scenderà egli come il tuo servo ha sentito dire? O Eterno, Dio d'Israele, deh! fallo sapere al tuo servo!' L'Eterno rispose: 'Scenderà'. Davide chiese ancora:
\par 12 'Quei di Keila daranno essi me e la mia gente nelle mani di Saul?' L'Eterno rispose: 'Vi daranno nelle sue mani'.
\par 13 Allora Davide e la sua gente, circa seicento uomini, si levarono, uscirono da Keila e andaron qua e là a caso; e Saul informato che Davide era fuggito da Keila, rinunziò alla sua spedizione.
\par 14 Davide rimase nel deserto in luoghi forti; e se ne stette nella contrada montuosa del deserto di Zif. Saul lo cercava continuamente, ma Dio non glielo dette nelle mani.
\par 15 E Davide, sapendo che Saul s'era mosso per torgli la vita, restò nel deserto di Zif, nella foresta.
\par 16 Allora Gionathan, figliuolo di Saul, si levò, e si recò da Davide nella foresta. Egli fortificò la sua fiducia in Dio,
\par 17 e gli disse: 'Non temere, poiché Saul, mio padre, non riuscirà a metterti le mani addosso: tu regnerai sopra Israele, e io sarò il secondo dopo di te; e ben lo sa anche Saul mio padre'.
\par 18 E i due fecero alleanza in presenza dell'Eterno; poi Davide rimase nella foresta, e Gionathan se ne andò a casa sua.
\par 19 Or gli Zifei salirono da Saul a Ghibea e gli dissero: 'Davide non sta egli nascosto fra noi, ne' luoghi forti della foresta, sul colle di Hakila che è a mezzogiorno del deserto?
\par 20 Scendi dunque, o re, giacché tutto il desiderio dell'anima tua è di scendere, e penserem noi a darlo nelle mani del re'.
\par 21 Saul disse: 'Siate benedetti dall'Eterno, voi che avete pietà di me!
\par 22 Andate, vi prego, informatevi anche più sicuramente per sapere e scoprire il luogo dove suol fermarsi, e chi l'abbia quivi veduto; poiché mi si dice ch'egli è molto astuto.
\par 23 E vedete di conoscere tutti i nascondigli dov'ei si ritira; poi tornate da me con notizie sicure, e io andrò con voi. S'egli è nel paese, io lo cercherò fra tutte le migliaia di Giuda'.
\par 24 Quelli dunque si levarono e se n'andarono a Zif, innanzi a Saul; ma Davide e i suoi erano nel deserto di Maon, nella pianura a mezzogiorno del deserto.
\par 25 Saul con la sua gente partì in cerca di lui; ma Davide, che ne fu informato, scese dalla roccia e rimase nel deserto di Maon. E quando Saul lo seppe, andò in traccia di Davide nel deserto di Maon.
\par 26 Saul camminava da un lato del monte, e Davide con la sua gente dall'altro lato; e come Davide affrettava la marcia per sfuggire a Saul e Saul e la sua gente stavano per circondare Davide e i suoi per impadronirsene,
\par 27 arrivò a Saul un messo che disse: 'Affrettati a venire, perché i Filistei hanno invaso il paese'.
\par 28 Così Saul cessò d'inseguire Davide e andò a far fronte ai Filistei; perciò a quel luogo fu messo nome Sela-Hammahlekoth.

\chapter{24}

\par 1 Poi Davide si partì di là e si stabilì nei luoghi forti di En-Ghedi.
\par 2 E quando Saul fu tornato dall'inseguire i Filistei, gli vennero a dire: 'Ecco, Davide è nel deserto di En-Ghedi'.
\par 3 Allora Saul prese tremila uomini scelti fra tutto Israele, e andò in traccia di Davide e della sua gente fin sulle rocce delle capre salvatiche;
\par 4 e giunse ai parchi di pecore ch'eran presso la via; quivi era una spelonca, nella quale Saul entrò per fare i suoi bisogni. Or Davide e la sua gente se ne stavano in fondo alla spelonca.
\par 5 La gente di Davide gli disse: 'Ecco il giorno nel quale l'Eterno ti dice: Vedi, io ti do nelle mani il tuo nemico; fa' di lui quello che ti piacerà'. Allora Davide s'alzò, e senza farsi scorgere tagliò il lembo del mantello di Saul.
\par 6 Ma dopo, il cuore gli batté, per aver egli tagliato il lembo del mantello di Saul.
\par 7 E Davide disse alla sua gente: 'Mi guardi l'Eterno, dal commettere contro il mio signore, ch'è l'unto dell'Eterno, l'azione di mettergli le mani addosso; poich'egli è l'unto dell'Eterno'.
\par 8 E colle sue parole Davide raffrenò la sua gente, e non le permise di gettarsi su Saul. E Saul si levò, uscì dalla spelonca e continuò il suo cammino.
\par 9 Poi anche Davide si levò, uscì dalla spelonca, e gridò dietro a Saul, dicendo: 'O re, mio signore!' Saul si guardò dietro, e Davide s'inchinò con la faccia a terra e si prostrò.
\par 10 Davide disse a Saul: 'Perché dai tu retta alle parole della gente che dice: Davide cerca di farti del male?
\par 11 Ecco in quest'ora stessa tu vedi coi tuoi propri occhi che l'Eterno t'avea dato oggi nelle mie mani in quella spelonca; qualcuno mi disse di ucciderti, ma io t'ho risparmiato, e ho detto: Non metterò le mani addosso al mio signore, perch'egli è l'unto dell'Eterno.
\par 12 Ora guarda, padre mio, guarda qui nella mia mano il lembo del tuo mantello. Se io t'ho tagliato il lembo del mantello e non t'ho ucciso, puoi da questo veder chiaro che non v'è nella mia condotta né malvagità né ribellione, e che io non ho peccato contro di te, mentre tu mi tendi insidie per tòrmi la vita!
\par 13 L'Eterno sia giudice fra me e te, e l'Eterno mi vendichi di te; ma io non ti metterò le mani addosso.
\par 14 Dice il proverbio antico: - Il male vien dai malvagi; - io quindi non ti metterò le mani addosso.
\par 15 Contro chi è uscito il re d'Israele? Chi vai tu perseguitando? Un can morto, una pulce.
\par 16 Sia dunque arbitro l'Eterno, e giudichi fra me e te, e vegga e difenda la mia causa e mi renda giustizia, liberandomi dalle tue mani'.
\par 17 Quando Davide ebbe finito di dire queste parole a Saul, Saul disse: 'È questa la tua voce, figliuol mio Davide?' E Saul alzò la voce e pianse.
\par 18 E disse a Davide: 'Tu sei più giusto di me, poiché tu m'hai reso bene per male, mentre io t'ho reso male per bene.
\par 19 Tu hai mostrato oggi la bontà con la quale ti conduci verso di me; poiché l'Eterno m'avea dato nelle tue mani, e tu non m'hai ucciso.
\par 20 Se uno incontra il suo nemico, lo lascia egli andarsene in pace? Ti renda dunque l'Eterno il contraccambio del bene che m'hai fatto quest'oggi!
\par 21 Ora, ecco, io so che per certo tu regnerai, e che il regno d'Israele rimarrà stabile nelle tue mani.
\par 22 Or dunque giurami nel nome dell'Eterno che non distruggerai la mia progenie dopo di me, e che non estirperai il mio nome dalla casa di mio padre'.
\par 23 E Davide lo giurò a Saul. Poi Saul se ne andò a casa sua, e Davide e la sua gente risaliron al loro forte rifugio.

\chapter{25}

\par 1 Samuele morì, e tutto Israele si radunò e ne fece cordoglio; e lo seppellirono nella sua proprietà, a Rama. Allora Davide si levò, e scese verso il deserto di Paran.
\par 2 Or v'era un uomo a Maon, che aveva i suoi beni a Carmel; era molto ricco, avea tremila pecore e mille capre, e si trovava a Carmel per la tosatura delle sue pecore.
\par 3 Quest'uomo avea nome Nabal, e il nome di sua moglie era Abigail, donna di buon senso e di bell'aspetto; ma l'uomo era duro e malvagio nell'agir suo; discendeva da Caleb.
\par 4 Davide, avendo saputo nel deserto che Nabal tosava le sue pecore,
\par 5 gli mandò dieci giovani, ai quali disse: 'Salite a Carmel, andate da Nabal, salutatelo a nome mio,
\par 6 e dite così: Salute! pace a te, pace alla tua casa, e pace a tutto quello che t'appartiene!
\par 7 Ho saputo che tu hai i tosatori; ora, i tuoi pastori sono stati con noi, e noi non abbiam fatto loro alcun oltraggio, e nulla è stato loro portato via per tutto il tempo che sono stati a Carmel.
\par 8 Domandane ai tuoi servi, e te lo diranno. Trovin dunque questi giovani grazia agli occhi tuoi, giacché siam venuti in giorno di gioia; e da', ti prego, ai tuoi servi e al tuo figliuolo Davide ciò che avrai fra mano'.
\par 9 Quando i giovani di Davide arrivarono, ripeterono a Nabal tutte queste parole in nome di Davide, poi si tacquero.
\par 10 Ma Nabal rispose ai servi di Davide, dicendo: 'Chi è Davide? E chi è il figliuolo d'Isai? Sono molti, oggi, i servi che scappano dai loro padroni;
\par 11 e prenderei io il mio pane, la mia acqua e la carne che ho macellata pei miei tosatori, per darli a gente che non so donde venga?'
\par 12 I giovani ripresero la loro strada, tornarono, e andarono a riferire a Davide tutte queste parole.
\par 13 Allora Davide disse ai suoi uomini: 'Ognun di voi si cinga la sua spada'. Ognuno si cinse la sua spada, e Davide pure si cinse la sua, e saliron dietro a Davide circa quattrocento uomini; duecento rimasero presso il bagaglio.
\par 14 Or Abigail, moglie di Nabal, fu informata della cosa da uno de' suoi servi, che le disse: 'Ecco, Davide ha inviato dal deserto de' messi per salutare il nostro padrone, ed egli li ha trattati male.
\par 15 Eppure, quella gente è stata molto buona verso di noi; noi non ne abbiam ricevuto alcun oltraggio, e non ci han portato via nulla per tutto il tempo che siamo andati attorno con loro quand'eravamo per la campagna.
\par 16 Di giorno e di notte sono stati per noi come una muraglia, per tutto il tempo che siamo stati con loro pascendo i greggi.
\par 17 Or dunque rifletti, e vedi quel che tu debba fare; poiché un guaio è certo che avverrà al nostro padrone e a tutta la sua casa; ed egli è uomo così malvagio, che non gli si può parlare'.
\par 18 Allora Abigail prese in fretta duecento pani, due otri di vino, cinque montoni allestiti, cinque misure di grano arrostito, cento picce d'uva secca e duecento masse di fichi, e caricò ogni cosa su degli asini.
\par 19 Poi disse ai suoi servi: 'Andate innanzi a me; ecco, io vi seguirò'. Ma non disse nulla a Nabal suo marito.
\par 20 E com'ella, a cavallo al suo asino, scendeva il monte per un sentiero coperto, ecco Davide e i suoi uomini che scendevano di fronte a lei, sì ch'ella li incontrò.
\par 21 - Or Davide avea detto: 'Invano dunque ho io protetto tutto ciò che colui aveva nel deserto, in guisa che nulla è mancato di tutto ciò ch'ei possiede; ed egli m'ha reso male per bene.
\par 22 Così tratti Iddio i nemici di Davide col massimo rigore! Fra qui e lo spuntar del giorno, di tutto quel che gli appartiene io non lascerò in vita un sol uomo'.
\par 23 E quando Abigail ebbe veduto Davide, scese in fretta dall'asino e gettandosi con la faccia a terra, si prostrò dinanzi a lui.
\par 24 Poi, gettandosi ai suoi piedi, disse: 'O mio signore, la colpa è mia! Deh, lascia che la tua serva parli in tua presenza, e tu ascolta le parole della tua serva!
\par 25 Te ne prego, signor mio, non far caso di quell'uomo da nulla ch'è Nabal; poiché egli è quel che dice il suo nome; si chiama Nabal, e in lui non c'è che stoltezza; ma io, la tua serva, non vidi i giovani mandati dal mio signore.
\par 26 Or dunque, signor mio, com'è vero che vive l'Eterno e che l'anima tua vive, l'Eterno t'ha impedito di spargere il sangue e di farti giustizia con le tue proprie mani. Ed ora, i tuoi nemici e quelli che voglion fare del male al mio signore siano come Nabal!
\par 27 E adesso, ecco questo regalo che la tua serva reca al mio signore; sia dato ai giovani che seguono il mio signore.
\par 28 Deh, perdona il fallo della tua serva; poiché per certo l'Eterno renderà stabile la casa del mio signore, giacché il mio signore combatte le battaglie dell'Eterno, e in tutto il tempo della tua vita non s'è trovata malvagità in te.
\par 29 E se mai sorgesse alcuno a perseguitarti e ad attentare alla tua vita, l'anima del mio signore sarà custodita nello scrigno della vita presso l'Eterno, ch'è il tuo Dio; ma l'anima de' tuoi nemici l'Eterno la lancerà via, come dalla rete d'una frombola.
\par 30 E quando l'Eterno avrà fatto al mio signore tutto il bene che t'ha promesso e t'avrà stabilito come capo sopra Israele,
\par 31 il mio signore non avrà questo dolore e questo rimorso d'avere sparso del sangue senza motivo e d'essersi fatto giustizia da sé. E quando l'Eterno avrà fatto del bene al mio signore, ricordati della tua serva'.
\par 32 E Davide disse ad Abigail: 'Sia benedetto l'Eterno, l'Iddio d'Israele, che t'ha oggi mandata incontro a me!
\par 33 E sia benedetto il tuo senno, e benedetta sii tu che m'hai oggi impedito di spargere del sangue e di farmi giustizia con le mie proprie mani!
\par 34 Poiché certo, com'è vero che vive l'Eterno, l'Iddio d'Israele, che m'ha impedito di farti del male, se tu non ti fossi affrettata a venirmi incontro, fra qui e lo spuntar del giorno a Nabal non sarebbe rimasto un sol uomo'.
\par 35 Davide quindi ricevé dalle mani di lei quello ch'essa avea portato, e le disse: 'Risali in pace a casa tua; vedi, io ho dato ascolto alla tua voce, e ho avuto riguardo a te'.
\par 36 Ed Abigail venne da Nabal; ed ecco ch'egli faceva banchetto in casa sua; banchetto da re. Nabal aveva il cuore allegro, perch'era ebbro fuor di modo; ond'ella non gli fece sapere alcuna cosa, piccola o grande, fino allo spuntar del giorno.
\par 37 Ma la mattina, quando gli fu passata l'ebbrezza, la moglie raccontò a Nabal queste cose; allora gli si freddò il cuore, ed ei rimase come un sasso.
\par 38 E circa dieci giorni dopo, l'Eterno colpì Nabal, ed egli morì.
\par 39 Quando Davide seppe che Nabal era morto, disse: 'Sia benedetto l'Eterno, che m'ha reso giustizia dell'oltraggio fattomi da Nabal, e ha preservato il suo servo dal far del male! La malvagità di Nabal, l'Eterno l'ha fatta ricadere sul capo di lui!' Poi Davide mandò da Abigail a proporle di diventar sua moglie.
\par 40 E i servi di Davide vennero da Abigail a Carmel, e le parlarono così: 'Davide ci ha mandati da te, perché vuol prenderti in moglie'.
\par 41 Allora ella si levò, si prostrò con la faccia a terra, e disse: 'Ecco, la tua serva farà da schiava, per lavare i piedi ai servi del mio signore'.
\par 42 Poi Abigail si levò tosto, montò sopra un asino, e seguita da cinque fanciulle tenne dietro ai messi di Davide, e divenne sua moglie.
\par 43 Davide sposò anche Ahinoam di Izreel, e ambedue furono sue mogli.
\par 44 Or Saul avea dato Mical sua figliuola, moglie di Davide, a Palti, figliuolo di Laish, che era di Gallim.

\chapter{26}

\par 1 Or gli Zifei vennero da Saul a Ghibea e gli dissero: 'Davide non sta egli nascosto sulla collina di Hakila dirimpetto al deserto?'
\par 2 Allora Saul si levò: e scese nel deserto di Zif avendo seco tremila uomini scelti d'Israele, per cercar Davide nel deserto di Zif.
\par 3 E Saul si accampò sulla collina di Hakila ch'è dirimpetto al deserto, presso la strada. E Davide, che stava nel deserto, avendo inteso che Saul veniva nel deserto per cercarlo,
\par 4 mandò delle spie, e seppe con certezza che Saul era giunto.
\par 5 Allora Davide si levò, venne al luogo dove Saul stava accampato, e notò il luogo ov'eran coricati Saul ed Abner, il figliuolo di Ner, capo dell'esercito di lui. Saul stava coricato nel parco dei carri, e la sua gente era accampata intorno a lui.
\par 6 E Davide prese a dire ad Ahimelec, lo Hitteo, e ad Abishai, figliuolo di Tseruia, fratello di Joab: 'Chi scenderà con me verso Saul nel campo?' E Abishai rispose: 'Scenderò io con te'.
\par 7 Davide ed Abishai dunque pervennero di notte a quella gente; ed ecco che Saul giaceva addormentato nel parco dei carri, con la sua lancia fitta in terra, dalla parte del capo; ed Abner e la sua gente gli stavan coricati all'intorno.
\par 8 Allora Abishai disse a Davide: 'Oggi Iddio t'ha messo il tuo nemico nelle mani; or lascia, ti prego, ch'io lo colpisca con la lancia e lo inchiodi in terra con un sol colpo; e non ci sarà bisogno d'un secondo'.
\par 9 Ma Davide disse ad Abishai: 'Non lo ammazzare; che potrebbe metter le mani addosso all'unto dell'Eterno senza rendersi colpevole?'
\par 10 Poi Davide aggiunse: 'Com'è vero che l'Eterno vive, l'Eterno solo sarà quegli che lo colpirà, sia che venga il suo giorno e muoia, sia che scenda in campo di battaglia e vi perisca.
\par 11 Mi guardi l'Eterno dal metter le mani addosso all'unto dell'Eterno! Prendi ora soltanto, ti prego, la lancia ch'è presso al suo capo e la brocca dell'acqua, e andiamocene'.
\par 12 Davide dunque prese la lancia e la brocca dell'acqua che Saul avea presso al suo capo, e se ne andarono. Nessuno vide la cosa né s'accorse di nulla; e nessuno si svegliò; tutti dormivano, perché l'Eterno avea fatto cader su loro un sonno profondo.
\par 13 Poi Davide passò dalla parte opposta e si fermò in lontananza in vetta al monte, a gran distanza dal campo di Saul;
\par 14 e gridò alla gente di Saul e ad Abner, figliuolo di Ner: 'Non rispondi tu, Abner?' Abner rispose e disse: 'Chi sei tu che gridi al re?'
\par 15 E Davide disse ad Abner: 'Non sei tu un valoroso? E chi è pari a te in Israele? Perché dunque non hai tu fatto buona guardia al re tuo signore? Poiché uno del popolo è venuto per ammazzare il re tuo signore.
\par 16 Questo che tu hai fatto non sta bene. Com'è vero che l'Eterno vive, meritate la morte voi che non avete fatto buona guardia al vostro signore, all'unto dell'Eterno! Ed ora guarda dove sia la lancia del re e dove sia la brocca dell'acqua che stava presso il suo capo!'
\par 17 Saul riconobbe la voce di Davide e disse: 'È questa la tua voce, o figliuol mio Davide?' Davide rispose: 'È la mia voce, o re, mio signore!'
\par 18 Poi aggiunse: 'Perché il mio signore perseguita il suo servo? Che ho io fatto? Che delitto ho io commesso?
\par 19 Ora dunque, si degni il re, mio signore, d'ascoltare le parole del suo servo. Se è l'Eterno quegli che t'incita contro di me, accetti egli un'oblazione! Ma se son gli uomini, siano essi maledetti dinanzi all'Eterno, poiché m'hanno oggi cacciato per separarmi dall'eredità dell'Eterno, dicendomi: - Va' a servire a degli dèi stranieri!
\par 20 Or dunque non cada il mio sangue in terra lungi dalla presenza dell'Eterno! Poiché il re d'Israele è uscito per andar in traccia d'una pulce, come si va dietro a una pernice su per i monti'.
\par 21 Allora Saul disse: 'Ho peccato; torna, figliuol mio Davide; poiché io non ti farò più alcun male, giacché oggi la mia vita è stata preziosa agli occhi tuoi; ecco, io ho operato da stolto, e ho commesso un gran fallo'.
\par 22 Davide rispose: 'Ecco la lancia del re; passi qua uno dei tuoi giovani a prenderla.
\par 23 L'Eterno retribuirà ciascuno secondo la sua giustizia e la sua fedeltà; giacché l'Eterno t'avea dato oggi nelle mie mani, e io non ho voluto metter le mani addosso all'unto dell'Eterno.
\par 24 E come preziosa è stata oggi la tua vita agli occhi miei, così preziosa sarà la vita mia agli occhi dell'Eterno; ed egli mi libererà da ogni tribolazione'.
\par 25 E Saul disse a Davide: 'Sii tu benedetto, figliuol mio Davide. Tu agirai da forte, e riuscirai per certo vittorioso'. Davide continuò il suo cammino, e Saul tornò a casa sua.

\chapter{27}

\par 1 Or Davide disse in cuor suo: 'Un giorno o l'altro io perirò per le mani di Saul; non v'è nulla di meglio per me che rifugiarmi nel paese dei Filistei, in guisa che Saul, perduta ogni speranza, finisca di cercarmi per tutto il territorio d'Israele; così scamperò dalle sue mani'.
\par 2 Davide dunque si levò, e coi seicento uomini che avea seco, si recò da Akis, figlio di Maoc, re di Gath.
\par 3 E Davide dimorò con Akis a Gath, egli e la sua gente, ciascuno con la sua famiglia. Davide avea seco le sue due mogli: Ahinoam, la Izreelita, e Abigail, la Carmelita, ch'era stata moglie di Nabal.
\par 4 E Saul, informato che Davide era fuggito a Gath, non si diè più a cercarlo.
\par 5 Davide disse ad Akis: 'Se ho trovato grazia agli occhi tuoi, siami dato in una delle città della campagna un luogo dove io possa stabilirmi; e perché il tuo servo dimorerebb'egli con te nella città reale?'
\par 6 Ed Akis, in quel giorno, gli diede Tsiklag; perciò Tsiklag ha appartenuto ai re di Giuda fino al dì d'oggi.
\par 7 Or il tempo che Davide passò nel paese dei Filistei fu di un anno e quattro mesi.
\par 8 E Davide e la sua gente salivano e facevano delle scorrerie nel paese dei Gheshuriti, dei Ghirziti e degli Amalekiti; poiché queste popolazioni abitavano da tempi antichi il paese, dal lato di Shur fino al paese d'Egitto.
\par 9 E Davide devastava il paese, non vi lasciava in vita né uomo né donna, e pigliava pecore, buoi, asini, cammelli e indumenti; poi se ne tornava e andava da Akis.
\par 10 Akis domandava: 'Dove avete fatto la scorreria quest'oggi?' E Davide rispondeva: 'Verso il mezzogiorno di Giuda, verso il mezzogiorno degli Jerahmeeliti e verso il mezzogiorno dei Kenei'.
\par 11 E Davide non lasciava in vita né uomo né donna per menarli a Gath, poiché diceva: 'Potrebbero parlare contro di noi e dire: Così ha fatto Davide'. Questo fu il suo modo d'agire tutto il tempo che dimorò nel paese dei Filistei.
\par 12 Ed Akis avea fiducia in Davide e diceva: 'Egli si rende odioso a Israele, suo popolo; e così sarà mio servo per sempre'.

\chapter{28}

\par 1 Or avvenne in quei giorni che i Filistei radunarono i loro eserciti per muover guerra ad Israele. Ed Akis disse a Davide: 'Sappi per cosa certa che verrai meco alla guerra, tu e la tua gente'. Davide rispose ad Akis:
\par 2 'E tu vedrai quello che il tuo servo farà'. E Akis a Davide: 'E io t'affiderò per sempre la guardia della mia persona'.
\par 3 Or Samuele era morto; tutto Israele ne avea fatto cordoglio, e l'avean sepolto in Rama, nella sua città. E Saul avea cacciato dal paese gli evocatori di spiriti e gl'indovini.
\par 4 I Filistei si radunarono e vennero ad accamparsi a Sunem. Saul parimente radunò tutto Israele, e si accamparono a Ghilboa.
\par 5 Quando Saul vide l'accampamento dei Filistei ebbe paura e il cuore gli tremò forte.
\par 6 E Saul consultò l'Eterno, ma l'Eterno non gli rispose né per via di sogni, né mediante l'Urim, né per mezzo dei profeti.
\par 7 Allora Saul disse ai suoi servi: 'Cercatemi una donna che sappia evocar gli spiriti ed io anderò da lei a consultarla'. I servi gli dissero: 'Ecco, a En-Dor v'è una donna che evoca gli spiriti'.
\par 8 E Saul si contraffece, si mise altri abiti, e partì accompagnato da due uomini. Giunsero di notte presso la donna, e Saul le disse: 'Dimmi l'avvenire, ti prego, evocando uno spirito, e fammi salire colui che ti dirò'.
\par 9 La donna gli rispose: 'Ecco, tu sai quel che Saul ha fatto, com'egli ha sterminato dal paese gli evocatori di spiriti e gl'indovini; perché dunque tendi un'insidia alla mia vita per farmi morire?'
\par 10 E Saul le giurò per l'Eterno dicendo: 'Com'è vero che l'Eterno vive, nessuna punizione ti toccherà per questo!'
\par 11 Allora la donna gli disse: 'Chi debbo farti salire?' Ed egli: 'Fammi salire Samuele'.
\par 12 E quando la donna vide Samuele levò un gran grido e disse a Saul: 'Perché m'hai ingannata? Tu sei Saul!'
\par 13 Il re le disse: 'Non temere; ma che vedi?' E la donna a Saul: 'Vedo un essere sovrumano che esce di sotto terra'.
\par 14 Ed egli a lei: 'Che forma ha?' Ella rispose: 'È un vecchio che sale, ed è avvolto in un mantello'. Allora Saul comprese ch'era Samuele, si chinò con la faccia a terra e gli si prostrò dinanzi.
\par 15 E Samuele disse a Saul: 'Perché mi hai tu disturbato, facendomi salire?' Saul rispose: 'Io sono in grande angustia, poiché i Filistei mi fanno guerra, e Dio si è ritirato da me e non mi risponde più né mediante i profeti né per via di sogni; perciò t'ho chiamato perché tu mi faccia sapere quel che ho da fare'.
\par 16 Samuele disse: 'Perché consulti me, mentre l'Eterno si è ritirato da te e t'è divenuto avversario?
\par 17 L'Eterno ha agito come aveva annunziato per mio mezzo; l'Eterno ti strappa di mano il regno e lo dà al tuo prossimo, a Davide,
\par 18 perché non hai ubbidito alla voce dell'Eterno e non hai lasciato corso all'ardore della sua ira contro ad Amalek; perciò l'Eterno ti tratta così quest'oggi.
\par 19 E l'Eterno darà anche Israele con te nelle mani dei Filistei, e domani tu e i tuoi figliuoli sarete meco; l'Eterno darà pure il campo d'Israele nelle mani dei Filistei'.
\par 20 Allora Saul cadde subitamente lungo disteso per terra, perché spaventato dalle parole di Samuele; ed era inoltre senza forza, giacché non avea preso cibo tutto quel giorno e tutta quella notte.
\par 21 La donna s'avvicinò a Saul; e vedutolo tutto atterrito, gli disse: 'Ecco, la tua serva ha ubbidito alla tua voce; io ho messo a repentaglio la mia vita per ubbidire alle parole che m'hai dette.
\par 22 Or dunque anche tu porgi ascolto alla voce della tua serva, e lascia ch'io ti metta davanti un boccon di pane; e mangia per prender forza da rimetterti in viaggio'.
\par 23 Ma egli rifiutò e disse: 'Non mangerò'. I suoi servi, però, unitamente alla donna gli fecero violenza, ed egli s'arrese alle loro istanze; s'alzò da terra e si pose a sedere sul letto.
\par 24 Or la donna aveva in casa un vitello ingrassato, che ella s'affrettò ad ammazzare; poi prese della farina, la impastò e ne fece dei pani senza lievito;
\par 25 mise quei cibi davanti a Saul e ai suoi servi, e quelli mangiarono, poi si levarono e ripartirono quella stessa notte.

\chapter{29}

\par 1 I Filistei radunarono tutte le loro truppe ad Afek, e gl'Israeliti si accamparono presso la sorgente di Izreel.
\par 2 I principi dei Filistei marciavano alla testa delle loro centinaia e delle loro migliaia, e Davide e la sua gente marciavano alla retroguardia con Akis.
\par 3 Allora i capi dei Filistei dissero: 'Che fanno qui questi Ebrei?' E Akis rispose ai capi dei Filistei: 'Ma questi è Davide, servo di Saul re d'Israele, che è stato presso di me da giorni, anzi da anni, e contro il quale non ho avuto nulla da ridire dal giorno della sua defezione a oggi!'
\par 4 Ma i capi de' Filistei si adirarono contro di lui, e gli dissero: 'Rimanda costui e se ne ritorni al luogo che tu gli hai assegnato, e non scenda con noi alla battaglia, affinché non sia per noi un nemico durante la battaglia. Poiché come potrebbe costui riacquistar la grazia del signor suo, se non a prezzo delle teste di questi uomini nostri?
\par 5 Non è egli quel Davide di cui si cantava in mezzo alle danze: Saul ha ucciso i suoi mille, e Davide i suoi diecimila?'
\par 6 Allora Akis chiamò Davide e gli disse: 'Com'è vero che l'Eterno vive, tu sei un uomo retto, e vedo con piacere il tuo andare e venire con me nel campo, poiché non ho trovato in te nulla di male dal giorno che arrivasti da me fino ad oggi; ma tu non piaci ai principi.
\par 7 Or dunque, ritornatene e vattene in pace, per non disgustare i principi dei Filistei'.
\par 8 Davide disse ad Akis: 'Ma che ho mai fatto? e che hai tu trovato nel tuo servo, in tutto il tempo che sono stato presso di te fino al dì d'oggi, perch'io non debba andare a combattere contro i nemici del re, mio signore?'
\par 9 Akis rispose a Davide, dicendo: 'Lo so; tu sei caro agli occhi miei come un angelo di Dio; ma i principi dei Filistei hanno detto: - Egli non deve salire con noi alla battaglia!
\par 10 Or dunque, alzati domattina di buon'ora, coi servi del tuo signore che son venuti teco; alzatevi di buon mattino e appena farà giorno, andatevene'.
\par 11 Davide dunque con la sua gente si levò di buon'ora, per partire al mattino e tornare nel paese dei Filistei. E i Filistei salirono a Izreel.

\chapter{30}

\par 1 Tre giorni dopo, quando Davide e la sua gente furon giunti a Tsiklag, ecco che gli Amalekiti avean fatto una scorreria verso il mezzogiorno e verso Tsiklag; aveano presa Tsiklag e l'aveano incendiata;
\par 2 avean fatto prigionieri le donne e tutti quelli che vi si trovavano, piccoli e grandi; non avevano ucciso alcuno, ma aveano menato via tutti, e se n'eran tornati donde eran venuti.
\par 3 Quando Davide e la sua gente giunsero alla città, ecco ch'essa era distrutta dal fuoco, e le loro mogli, i loro figliuoli e le loro figliuole erano stati menati via prigionieri.
\par 4 Allora Davide e tutti quelli ch'eran con lui alzaron la voce e piansero, finché non ebbero più forza di piangere.
\par 5 Le due mogli di Davide, Ahinoam la Izreelita e Abigail la Carmelita ch'era stata moglie di Nabal, erano anch'esse prigioniere.
\par 6 E Davide fu grandemente angosciato perché la gente parlava di lapidarlo, essendo l'animo di tutti amareggiato a motivo dei lor figliuoli e delle loro figliuole; ma Davide si fortificò nell'Eterno, nel suo Dio.
\par 7 Davide disse al sacerdote Abiathar, figliuolo di Ahimelec: 'Ti prego, portami qua l'efod'. E Abiathar portò l'efod a Davide.
\par 8 E Davide consultò l'Eterno, dicendo: 'Debbo io dar dietro a questa banda di predoni? la raggiungerò io?' L'Eterno rispose: 'Dàlle dietro, poiché certamente la raggiungerai, e potrai ricuperare ogni cosa'.
\par 9 Davide dunque andò coi seicento uomini che avea seco, e giunsero al torrente Besor, dove quelli ch'erano rimasti indietro si fermarono:
\par 10 ma Davide continuò l'inseguimento con quattrocento uomini: duecento erano rimasti addietro, troppo stanchi per poter attraversare il torrente Besor.
\par 11 Trovarono per la campagna un Egiziano, e lo menarono a Davide. Gli diedero del pane, ch'egli mangiò, e dell'acqua da bere;
\par 12 e gli diedero un pezzo di schiacciata di fichi secchi e due picce d'uva. Quand'egli ebbe mangiato, si riebbe, perché non avea mangiato pane né bevuto acqua per tre giorni e tre notti.
\par 13 Davide gli chiese: 'A chi appartieni? e di dove sei?' Quegli rispose: 'Sono un giovine Egiziano, servo di un Amalekita; e il mio padrone m'ha abbandonato perché tre giorni fa caddi infermo.
\par 14 Abbiam fatto una scorreria nel mezzogiorno dei Kerethei, sul territorio di Giuda e nel mezzogiorno di Caleb, e abbiamo incendiato Tsiklag'.
\par 15 Davide gli disse: 'Vuoi tu condurmi giù dov'è quella banda?' Quegli rispose: 'Giurami per il nome di Dio che non mi ucciderai e non mi darai nelle mani del mio padrone, e io ti menerò giù dov'è quella banda'.
\par 16 E quand'ei l'ebbe menato là, ecco che gli Amalekiti erano sparsi dappertutto per la campagna, mangiando, bevendo, e facendo festa, a motivo del gran bottino che avean portato via dal paese dei Filistei e dal paese di Giuda.
\par 17 Davide diè loro addosso dalla sera di quel giorno fino alla sera dell'indomani; e non uno ne scampò, tranne quattrocento giovani, che montarono su dei cammelli e fuggirono.
\par 18 Davide ricuperò tutto quello che gli Amalekiti aveano portato via, e liberò anche le sue due mogli.
\par 19 E non vi mancò alcuno, né dei piccoli né dei grandi, né de' figliuoli né delle figliuole, né alcun che del bottino, né cosa alcuna che gli Amalekiti avessero presa. Davide ricondusse via tutto.
\par 20 Davide riprese anche tutti i greggi e tutti gli armenti; e quelli che menavano questo bestiame e camminavano alla sua testa, dicevano: 'Questo è il bottino di Davide!'
\par 21 Poi Davide tornò verso quei duecento uomini che per la grande stanchezza non gli avevano potuto tener dietro, e che egli avea fatti rimanere al torrente Besor. Quelli si fecero avanti incontro a Davide e alla gente ch'era con lui. E Davide, accostatosi a loro, li salutò.
\par 22 Allora tutti i tristi e i perversi fra gli uomini che erano andati con Davide, presero a dire: 'Giacché costoro non son venuti con noi, non darem loro nulla del bottino che abbiamo ricuperato; tranne a ciascun di loro la sua moglie e i suoi figliuoli; se li menino via, e se ne vadano!'
\par 23 Ma Davide disse: 'Non fate così, fratelli miei, riguardo alle cose che l'Eterno ci ha date: Egli che ci ha protetti, e ha dato nelle nostre mani la banda ch'era venuta contro di noi.
\par 24 E chi vi darebbe retta in quest'affare? Qual è la parte di chi scende alla battaglia, tale dev'essere la parte di colui che rimane presso il bagaglio; faranno tra loro le parti uguali'.
\par 25 E da quel giorno in poi si fece così; Davide ne fece in Israele una legge e una norma, che han durato fino al dì d'oggi.
\par 26 Quando Davide fu tornato a Tsiklag, mandò parte di quel bottino agli anziani di Giuda, suoi amici, dicendo: 'Eccovi un dono che viene dal bottino preso ai nemici dell'Eterno'.
\par 27 Ne mandò a quelli di Bethel, a quelli di Bamoth del mezzogiorno, a quelli di Jattir,
\par 28 a quelli d'Aroer, a quelli di Simoth, a quelli d'Eshtemoa,
\par 29 a quelli di Racal, a quelli delle città degli Ierahmeeliti, a quelli delle città dei Kenei,
\par 30 a quelli di Horma, a quelli di Cor Ashan, a quelli di Athac,
\par 31 a quelli di Hebron, e a quelli di tutti i luoghi che Davide e la sua gente aveano percorso.

\chapter{31}

\par 1 Or i Filistei vennero a battaglia con Israele, e gl'Israeliti fuggirono dinanzi ai Filistei, e caddero morti in gran numero sul monte Ghilboa.
\par 2 I Filistei inseguirono accanitamente Saul e i suoi figliuoli, e uccisero Gionathan, Abinadab e Malkishua, figliuoli di Saul.
\par 3 Il forte della battaglia si volse contro Saul; gli arcieri lo raggiunsero, ed egli si trovò in grande angoscia a motivo degli arcieri.
\par 4 Saul disse al suo scudiero: 'Sfodera la spada, e trafiggimi, affinché questi incirconcisi non vengano a trafiggermi ed a farmi oltraggio'. Ma lo scudiero non volle farlo, perch'era còlto da gran paura. Allora Saul prese la propria spada e vi si gettò sopra.
\par 5 Lo scudiero di Saul, vedendolo morto, si gettò anch'egli sulla propria spada, e morì con lui.
\par 6 Così, in quel giorno, morirono insieme Saul, i suoi tre figliuoli, il suo scudiero e tutta la sua gente.
\par 7 E quando gl'Israeliti che stavano di là dalla valle e di là dal Giordano videro che la gente d'Israele s'era data alla fuga e che Saul e i suoi figliuoli erano morti, abbandonarono le città, e fuggirono; e i Filistei andarono essi ad abitarle.
\par 8 L'indomani i Filistei vennero a spogliare i morti, e trovarono Saul e i suoi tre figliuoli caduti sul monte Ghilboa.
\par 9 Tagliarono la testa a Saul, lo spogliarono delle sue armi, e mandarono all'intorno per il paese de' Filistei ad annunziare la buona notizia nei tempî dei loro idoli ed al popolo;
\par 10 e collocarono le armi di lui nel tempio di Astarte, e appesero il suo cadavere alle mura di Beth-Shan.
\par 11 Ma quando gli abitanti di Jabes di Galaad udirono quello che i Filistei avean fatto a Saul,
\par 12 tutti gli uomini valorosi si levarono, camminarono tutta la notte, tolsero dalle mura di Beth-Shan il cadavere di Saul e i cadaveri dei suoi figliuoli, tornarono a Jabes, e quivi li bruciarono.
\par 13 Poi presero le loro ossa, le seppellirono sotto alla tamerice di Jabes, e digiunarono per sette giorni.


\end{document}