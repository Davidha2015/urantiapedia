\begin{document}

\title{1 Kings}


\chapter{1}

\par 1 Ora il re Davide era vecchio e molto attempato; e, per quanto lo coprissero di panni, non potea riscaldarsi.
\par 2 Perciò i suoi servi gli dissero: 'Si cerchi per il re nostro signore una fanciulla vergine, la quale stia al servizio del re, n'abbia cura, e dorma fra le sue braccia, sì che il re nostro signore possa riscaldarsi'.
\par 3 Cercaron dunque per tutto il paese d'Israele una bella fanciulla; trovarono Abishag, la Sunamita, e la menarono al re.
\par 4 La fanciulla era bellissima, avea cura del re, e lo serviva; ma il re non la conobbe.
\par 5 Or Adonija, figliuolo di Hagghith, mosso dall'ambizione, diceva: 'Sarò io il re!' E si preparò dei carri, de' cavalieri, e cinquanta uomini che corressero dinanzi a lui.
\par 6 Suo padre non gli avea mai fatto un rimprovero in vita sua, dicendogli: 'Perché fai così?' Adonija era anch'egli di bellissimo aspetto, ed era nato subito dopo Absalom.
\par 7 Egli si abboccò con Joab, figliuolo di Tseruia, e col sacerdote Abiathar, i quali seguirono il suo partito e lo favorirono.
\par 8 Ma il sacerdote Tsadok, Benaia figliuolo di Jehoiada, il profeta Nathan, Scimei, Rei e gli uomini prodi di Davide non erano per Adonija.
\par 9 Adonija immolò pecore, buoi e vitelli grassi vicino al masso di Zohelet che è accanto alla fontana di Roghel, e invitò tutti i suoi fratelli, figliuoli del re, e tutti gli uomini di Giuda ch'erano al servizio del re;
\par 10 ma non invitò il profeta Nathan, né Benaia, né gli uomini prodi, né Salomone suo fratello.
\par 11 Allora Nathan parlò a Bath-Sceba, madre di Salomone, e le disse: 'Non hai udito che Adonija, figliuolo di Hagghith, è diventato re senza che Davide nostro signore ne sappia nulla?
\par 12 Or dunque vieni, e permetti ch'io ti dia un consiglio, affinché tu salvi la vita tua e quella del tuo figliuolo Salomone.
\par 13 Va', entra dal re Davide, e digli: - O re, mio signore, non giurasti tu alla tua serva, dicendo: Salomone, tuo figliuolo, regnerà dopo di me e sederà sul mio trono? Perché dunque regna Adonija?
\par 14 Ed ecco che mentre tu starai ancora quivi parlando col re, io entrerò dopo di te, e confermerò le tue parole'.
\par 15 Bath-Sceba entrò dunque nella camera del re. - Il re era molto vecchio, e Abishag, la Sunamita, lo serviva. -
\par 16 Bath-Sceba s'inchinò e si prostrò davanti al re. E il re disse: 'Che vuoi?'
\par 17 Essa gli rispose: 'Signor mio, tu alla tua serva, giurasti per l'Eterno ch'è il tuo Dio, dicendo: - Salomone, tuo figliuolo, regnerà dopo di me e sederà sul mio trono; -
\par 18 e intanto, ecco che Adonija è diventato re senza che tu, o re mio signore, ne sappia nulla.
\par 19 Ed ha immolato buoi, vitelli grassi, e pecore in gran numero, ed ha invitato tutti i figliuoli del re e il sacerdote Abiathar e Joab, il capo dell'esercito, ma non ha invitato il tuo servo Salomone.
\par 20 Ora gli occhi di tutto Israele son rivolti verso di te, o re mio signore, perché tu gli dichiari chi debba sedere sul trono del re mio signore, dopo di lui.
\par 21 Altrimenti avverrà che, quando il re mio signore giacerà coi suoi padri, io e il mio figliuolo Salomone sarem trattati come colpevoli'.
\par 22 Mentr'ella parlava ancora col re, ecco arrivare il profeta Nathan.
\par 23 La cosa fu riferita al re, dicendo: 'Ecco il profeta Nathan!' E questi venne in presenza del re, e gli si prostrò dinanzi con la faccia a terra.
\par 24 Nathan disse: 'O re, mio signore, hai tu detto: - Adonija regnerà dopo di me e sederà sul mio trono? -
\par 25 Giacché oggi egli è sceso, ha immolato buoi, vitelli grassi, e pecore in gran numero, ed ha invitato tutti i figliuoli del re, i capi dell'esercito e il sacerdote Abiathar; ed ecco che mangiano e bevono davanti a lui, e dicono: - Viva il re Adonija! -
\par 26 Ma egli non ha invitato me, tuo servo, né il sacerdote Tsadok, né Benaia figliuolo di Jehoiada, né Salomone tuo servo.
\par 27 Questa cosa è ella proprio stata fatta dal re mio signore, senza che tu abbia dichiarato al tuo servo chi sia quegli che deve sedere sul trono del re mio signore dopo di lui?'
\par 28 Il re Davide, rispondendo, disse: 'Chiamatemi Bath-Sceba'. Ella entrò alla presenza del re, e si tenne in piedi davanti a lui.
\par 29 E il re giurò e disse: 'Com'è vero che vive l'Eterno il quale ha liberato l'anima mia da ogni distretta,
\par 30 io farò oggi quel che ti giurai per l'Eterno, per l'Iddio d'Israele, dicendo: - Salomone tuo figliuolo regnerà dopo di me e sederà sul mio trono in vece mia'.
\par 31 Bath-Sceba s'inchinò con la faccia a terra, si prostrò dinanzi al re, e disse: 'Possa il re Davide mio signore vivere in perpetuo!'
\par 32 Poi il re Davide disse: 'Chiamatemi il sacerdote Tsadok, il profeta Nathan e Benaia, figliuolo di Jehoiada'. Essi vennero in presenza del re, e il re disse loro:
\par 33 'Prendete con voi i servi del vostro signore, fate montare Salomone mio figliuolo sulla mia mula, e menatelo giù a Ghihon.
\par 34 E quivi il sacerdote Tsadok e il profeta Nathan lo ungano re d'Israele. Poi sonate la tromba e dite: Viva il re Salomone! -
\par 35 Voi risalirete al suo seguito, ed egli verrà, si porrà a sedere sul mio trono, e regnerà in mia vece. Io costituisco lui come principe d'Israele e di Giuda'.
\par 36 Benaia, figliuolo di Jehoiada, rispose al re: 'Amen! Così voglia l'Eterno, l'Iddio del re mio signore!
\par 37 Come l'Eterno è stato col re mio signore, così sia con Salomone, e innalzi il suo trono al di sopra del trono del re Davide, mio signore!'
\par 38 Allora il sacerdote Tsadok, il profeta Nathan, Benaia figliuolo di Jehoiada, i Kerethei e i Pelethei scesero, fecero montare Salomone sulla mula del re Davide, e lo menarono a Ghihon.
\par 39 Il sacerdote Tsadok prese il corno dell'olio dal tabernacolo e unse Salomone. Sonaron la tromba, e tutto il popolo disse: 'Viva il re Salomone!'
\par 40 E tutto il popolo risalì al suo seguito sonando flauti e abbandonandosi a una gran gioia, sì che la terra rimbombava delle loro grida.
\par 41 Adonija e tutti i suoi convitati, come stavano per finir di mangiare, udirono questo rumore; e quando Joab udì il suon della tromba, disse: 'Che vuol dire questo strepito della città in tumulto?'
\par 42 E mentre egli parlava ancora, ecco giungere Gionathan, figliuolo del sacerdote Abiathar. Adonija gli disse: 'Entra, poiché tu sei un uomo di valore, e devi recar buone novelle'.
\par 43 E Gionathan, rispondendo a Adonija, disse: 'Tutt'altro! Il re Davide, nostro signore, ha fatto re Salomone.
\par 44 Egli ha mandato con lui il sacerdote Tsadok, il profeta Nathan, Benaia figliuolo di Jehoiada, i Kerethei e i Pelethei, i quali l'hanno fatto montare sulla mula del re.
\par 45 Il sacerdote Tsadok e il profeta Nathan l'hanno unto re a Ghihon, e di là son risaliti abbandonandosi alla gioia, e la città n'è tutta sossopra. Questo è lo strepito che avete udito.
\par 46 E c'è di più: Salomone s'è posto a sedere sul trono reale.
\par 47 E i servi del re son venuti a benedire il re Davide signor nostro, dicendo: - Renda Iddio il nome di Salomone più glorioso del tuo, e innalzi il suo trono al di sopra del tuo! E il re si è prostrato sul suo letto, poi il re ha detto così:
\par 48 - Benedetto sia l'Eterno, l'Iddio d'Israele, che m'ha dato oggi uno che segga sul mio trono, e m'ha permesso di vederlo coi miei propri occhi!'
\par 49 Allora tutti i convitati di Adonija furono presi da spavento, si alzarono, e se ne andarono ciascuno per il suo cammino.
\par 50 E Adonija, avendo timore di Salomone, si levò e andò ad impugnare i corni dell'altare.
\par 51 E vennero a dire a Salomone: 'Ecco, Adonija ha timore del re Salomone, ed ha impugnato i corni dell'altare, dicendo: - Il re Salomone mi giuri oggi che non farà morir di spada il suo servo'.
\par 52 Salomone rispose: 'S'egli si addimostra uomo dabbene, non cadrà in terra neppure uno dei suoi capelli; ma, se sarà trovato in fallo, morrà'.
\par 53 E il re Salomone mandò gente a farlo scendere dall'altare. Ed egli venne a prostrarsi davanti al re Salomone; e Salomone gli disse: 'Vattene a casa tua'.

\chapter{2}

\par 1 Or avvicinandosi per Davide il giorno della morte, egli diede i suoi ordini a Salomone suo figliuolo, dicendo:
\par 2 'Io me ne vo per la via di tutti gli abitanti della terra; fortificati e portati da uomo!
\par 3 Osserva quello che l'Eterno, il tuo Dio, t'ha comandato d'osservare, camminando nelle sue vie e mettendo in pratica le sue leggi, i suoi comandamenti, i suoi precetti, i suoi insegnamenti, secondo che è scritto nella legge di Mosè, affinché tu riesca in tutto ciò che farai
\par 4 e dovunque tu ti volga, e affinché l'Eterno adempia la parola da lui pronunciata a mio riguardo quando disse: - Se i tuoi figliuoli veglieranno sulla loro condotta camminando nel mio cospetto con fedeltà, con tutto il loro cuore e con tutta l'anima loro, non ti mancherà mai qualcuno che segga sul trono d'Israele. -
\par 5 Sai anche tu quel che m'ha fatto Joab, figliuolo di Tseruia, quel che ha fatto ai due capi degli eserciti d'Israele, ad Abner figliuolo di Ner, e ad Amasa, figliuolo di Jether, i quali egli uccise, spargendo in tempo di pace sangue di guerra, e macchiando di sangue la cintura che portava ai fianchi e i calzari che portava ai piedi.
\par 6 Agisci dunque secondo la tua saviezza, e non lasciare la sua canizie scendere in pace nel soggiorno de' morti.
\par 7 Ma tratta con bontà i figliuoli di Barzillai il Galaadita e siano fra quelli che mangiano alla tua mensa; poiché così anch'essi mi trattarono quando vennero a me, allorch'io fuggivo d'innanzi ad Absalom tuo fratello.
\par 8 Ed ecco, tu hai vicino a te Scimei figliuolo di Ghera, il Beniaminita, di Bahurim, il quale proferì contro di me una maledizione atroce il giorno che andavo a Mahanaim. Ma egli scese ad incontrarmi verso il Giordano, e io gli giurai per l'Eterno che non lo farei morire di spada. -
\par 9 Ma ora non lo lasciare impunito; poiché sei savio per conoscere quel che tu debba fargli, e farai scendere tinta di sangue la sua canizie nel soggiorno de' morti'.
\par 10 E Davide s'addormentò coi suoi padri, e fu sepolto nella città di Davide.
\par 11 Il tempo che Davide regnò sopra Israele fu di quarant'anni: regnò sette anni a Hebron e trentatre anni a Gerusalemme.
\par 12 E Salomone si assise sul trono di Davide suo padre, e il suo regno fu saldamente stabilito.
\par 13 Or Adonija figliuolo di Hagghith, venne da Bath-Sceba, madre di Salomone. Questa gli disse: 'Vieni tu con intenzioni pacifiche?' Egli rispose: 'Sì, pacifiche'.
\par 14 Poi aggiunse: 'Ho da dirti una parola'. Quella rispose: 'Di' pure'.
\par 15 Ed egli disse: 'Tu sai che il regno mi apparteneva, e che tutto Israele mi considerava come suo futuro re; ma il regno è stato trasferito e fatto passare a mio fratello, perché glielo ha dato l'Eterno.
\par 16 Or dunque io ti domando una cosa; non me la rifiutare'. Ella rispose: 'Di' pure'.
\par 17 Ed egli disse: 'Ti prego, di' al re Salomone, il quale nulla ti negherà, che mi dia Abishag la Sunamita per moglie'.
\par 18 Bath-Sceba rispose: 'Sta bene, parlerò al re in tuo favore'.
\par 19 Bath-Sceba dunque si recò dal re Salomone per parlargli in favore di Adonija. Il re si alzò per andarle incontro, le s'inchinò, poi si pose a sedere sul suo trono, e fece mettere un altro trono per sua madre, la quale si assise alla sua destra.
\par 20 Ella gli disse: 'Ho una piccola cosa da chiederti; non me la negare'. Il re rispose: 'Chiedila pure, madre mia; io non te la negherò'.
\par 21 Ed ella: 'Diasi Abishag la Sunamita al tuo fratello Adonija per moglie'.
\par 22 Il re Salomone, rispondendo a sua madre, disse: 'E perché chiedi tu Abishag la Sunamita per Adonija? Chiedi piuttosto il regno per lui, giacché egli è mio fratello maggiore; chiedilo per lui, per il sacerdote Abiathar e per Joab, figliuolo di Tseruia!'
\par 23 Allora il re Salomone giurò per l'Eterno, dicendo: 'Iddio mi tratti con tutto il suo rigore, se Adonija non ha proferito questa parola a costo della sua vita!
\par 24 Ed ora, com'è vero che vive l'Eterno, il quale m'ha stabilito, m'ha fatto sedere sul trono di Davide mio padre, e m'ha fondato una casa come avea promesso, oggi Adonija sarà messo a morte!'
\par 25 E il re Salomone mandò Benaia, figliuolo di Jehoiada, il quale s'avventò addosso ad Adonija sì che morì.
\par 26 Poi il re disse al sacerdote Abiathar: 'Vattene ad Anatoth, nelle tue terre, poiché tu meriti la morte; ma io non ti farò morire oggi, perché portasti davanti a Davide mio padre l'arca del Signore, dell'Eterno, e perché partecipasti a tutte le sofferenze di mio padre'.
\par 27 Così Salomone depose Abiathar dalle funzioni di sacerdote dell'Eterno, adempiendo così la parola che l'Eterno avea pronunziata contro la casa di Eli a Sciloh.
\par 28 E la notizia ne giunse a Joab, il quale avea seguito il partito di Adonija, benché non avesse seguito quello di Absalom. Egli si rifugiò nel tabernacolo dell'Eterno, e impugnò i corni dell'altare.
\par 29 E fu detto al re Salomone: 'Joab s'è rifugiato nel tabernacolo dell'Eterno, e sta presso l'altare'. Allora Salomone mandò Benaia, figliuolo di Jehoiada, dicendogli: 'Va', avventati contro di lui!'
\par 30 Benaia entrò nel tabernacolo dell'Eterno, e disse a Joab: 'Così dice il re: Vieni fuori!' Quegli rispose: 'No! voglio morir qui!' E Benaia riferì la cosa al re, dicendo: 'Così ha parlato Joab e così m'ha risposto'.
\par 31 E il re gli disse: 'Fa' com'egli ha detto; avventati contro di lui e seppelliscilo; così toglierai d'addosso a me ed alla casa di mio padre il sangue che Joab sparse senza motivo.
\par 32 E l'Eterno farà ricadere sul capo di lui il sangue ch'egli sparse, quando s'avventò contro due uomini più giusti e migliori di lui, e li uccise di spada, senza che Davide mio padre ne sapesse nulla: Abner, figliuolo di Ner, capitano dell'esercito d'Israele, e Amasa, figliuolo di Jether, capitano dell'esercito di Giuda.
\par 33 Il loro sangue ricadrà sul capo di Joab e sul capo della sua progenie in perpetuo, ma vi sarà pace per sempre, da parte dell'Eterno, per Davide, per la sua progenie, per la sua casa e per il suo trono'.
\par 34 Allora Benaia, figliuolo di Jehoiada, salì, s'avventò contro a lui e lo mise a morte; e Joab fu sepolto in casa sua nel deserto.
\par 35 E in vece sua il re fece capo dell'esercito Benaia, figliuolo di Jehoiada, e mise il sacerdote Tsadok al posto di Abiathar.
\par 36 Poi il re mandò a chiamare Scimei e gli disse: 'Costruisciti una casa in Gerusalemme, prendivi dimora, e non ne uscire per andare qua o là;
\par 37 poiché il giorno che ne uscirai e passerai il torrente Kidron, sappi per certo che morrai; il tuo sangue ricadrà sul tuo capo'.
\par 38 Scimei rispose al re: 'Sta bene; il tuo servo farà come il re mio signore ha detto'. E Scimei dimorò lungo tempo a Gerusalemme.
\par 39 Di lì a tre anni avvenne che due servi di Scimei fuggirono presso Akis, figliuolo di Maaca, re di Gath. La cosa fu riferita a Scimei, e gli fu detto: 'Ecco i tuoi servi sono a Gath'.
\par 40 E Scimei si levò, sellò il suo asino, e andò a Gath, da Akis, in cerca dei suoi servi; andò, e rimenò via da Gath i suoi servi.
\par 41 E fu riferito a Salomone che Scimei era andato da Gerusalemme a Gath, ed era tornato.
\par 42 Il re mandò a chiamare Scimei, e gli disse: 'Non t'avevo io fatto giurare per l'Eterno, e non t'avevo solennemente avvertito, dicendoti: - Sappi per certo che il giorno che uscirai per andar qua o là, morrai? - E non mi rispondesti tu: - La parola che ho udita sta bene?
\par 43 E perché dunque non hai mantenuto il giuramento fatto all'Eterno e non hai osservato il comandamento che t'avevo dato?'
\par 44 Il re disse inoltre a Scimei: 'Tu sai tutto il male che facesti a Davide mio padre; il tuo cuore n'è consapevole; ora l'Eterno fa ricadere sul tuo capo la tua malvagità;
\par 45 ma il re Salomone sarà benedetto e il trono di Davide sarà reso stabile in perpetuo dinanzi all'Eterno'.
\par 46 E il re diede i suoi ordini a Benaia, figliuolo di Jehoiada, il quale uscì, s'avventò contro Scimei, che morì. Così rimase saldo il regno nelle mani di Salomone.

\chapter{3}

\par 1 Or Salomone s'imparentò con Faraone, re di Egitto. Sposò la figliuola di Faraone, e la menò nella città di Davide, finché avesse finito di edificare la sua casa, la casa dell'Eterno e le mura di cinta di Gerusalemme.
\par 2 Intanto il popolo non offriva sacrifizi che sugli alti luoghi, perché fino a quei giorni non era stata edificata casa al nome dell'Eterno.
\par 3 E Salomone amava l'Eterno e seguiva i precetti di Davide suo padre; soltanto offriva sacrifizi e profumi sugli alti luoghi.
\par 4 Il re si recò a Gabaon per offrirvi sacrifizi, perché quello era il principale fra gli alti luoghi; e su quell'altare Salomone offerse mille olocausti.
\par 5 A Gabaon, l'Eterno apparve di notte, in sogno, a Salomone. E Dio gli disse: 'Chiedi quello che vuoi ch'io ti dia'.
\par 6 Salomone rispose: 'Tu hai trattato con gran benevolenza il tuo servo Davide, mio padre, perch'egli camminava dinanzi a te con fedeltà, con giustizia, con rettitudine di cuore a tuo riguardo; tu gli hai conservata questa gran benevolenza, e gli hai dato un figliuolo che siede sul trono di lui, come oggi avviene.
\par 7 Ora, o Eterno, o mio Dio, tu hai fatto regnar me, tuo servo, in luogo di Davide mio padre, e io non sono che un giovanetto, e non so come condurmi;
\par 8 e il tuo servo è in mezzo al popolo che tu hai scelto, popolo numeroso, che non può essere contato né calcolato, tanto è grande.
\par 9 Da' dunque al tuo servo un cuore intelligente ond'egli possa amministrar la giustizia per il tuo popolo e discernere il bene dal male; poiché chi mai potrebbe amministrar la giustizia per questo tuo popolo che è così numeroso?'
\par 10 Piacque al Signore che Salomone gli avesse fatta una tale richiesta.
\par 11 E Dio gli disse: 'Giacché tu hai domandato questo, e non hai chiesto per te lunga vita, né ricchezze, né la morte de' tuoi nemici, ma hai chiesto intelligenza per poter discernere ciò ch'è giusto,
\par 12 ecco, io faccio secondo la tua parola; e ti do un cuor savio e intelligente, in guisa che nessuno è stato simile a te per lo innanzi, e nessuno sorgerà simile a te in appresso.
\par 13 E oltre a questo io ti do quello che non hai domandato: ricchezze e gloria; talmente, che non vi sarà durante tutta la tua vita alcuno fra i re che possa esserti paragonato.
\par 14 E se cammini nelle mie vie osservando le mie leggi e i miei comandamenti, come fece Davide tuo padre, io prolungherò i tuoi giorni'.
\par 15 Salomone si svegliò, ed ecco era un sogno; tornò a Gerusalemme, si presentò davanti all'arca del patto del Signore, e offerse olocausti, sacrifizi di azioni di grazie e fece un convito a tutti i suoi servi.
\par 16 Allora due meretrici vennero a presentarsi davanti al re.
\par 17 Una delle due disse: 'Permetti, Signor mio! Io e questa donna abitavamo nella medesima casa, e io partorii nella camera, dov'ella pure stava.
\par 18 E il terzo giorno dopo che ebbi partorito io, questa donna partorì anch'ella; noi stavamo insieme, e non v'era da noi alcun estraneo; non c'eravamo che noi due in casa.
\par 19 Ora, la notte passata, il bimbo di questa donna morì, perch'ella gli s'era coricata addosso.
\par 20 Ed essa, alzatasi nel cuor della notte, prese il mio figliuolo d'accanto a me, mentre la tua serva dormiva, e lo pose a giacere sul suo seno, e sul mio seno pose il suo figliuolo morto.
\par 21 E quando m'alzai la mattina per far poppare il mio figlio, ecco ch'era morto; ma, mirando meglio a giorno chiaro, m'accorsi che non era il mio figlio ch'io avevo partorito'.
\par 22 L'altra donna disse: 'No, il vivo è il figliuolo mio, e il morto è il tuo'. Ma la prima replicò: 'No, invece, il morto è il figliuolo tuo, e il vivo è il mio'. Così altercavano in presenza del re.
\par 23 Allora il re disse: 'Una dice: - Questo ch'è vivo è il figliuolo mio, e quello ch'è morto è il tuo; - e l'altra dice: - No, invece, il morto è il figliuolo tuo, e il vivo è il mio'. -
\par 24 Il re soggiunse: 'Portatemi una spada!' E portarono una spada davanti al re.
\par 25 E il re disse: 'Dividete il bambino vivo in due parti, e datene la metà all'una, e la metà all'altra'.
\par 26 Allora la donna di cui era il bambino vivo, sentendosi commuover le viscere per amore del suo figliuolo, disse al re: 'Deh! Signor mio, date a lei il bambino vivo, e non l'uccidete, no!' Ma l'altra diceva: 'Non sia né mio né tuo; si divida!'
\par 27 Allora il re, rispondendo, disse: 'Date a quella il bambino vivo, e non l'uccidete; la madre del bimbo è lei!'
\par 28 E tutto Israele udì parlare del giudizio che il re avea pronunziato, e temettero il re perché vedevano che la sapienza di Dio era in lui per amministrare la giustizia.

\chapter{4}

\par 1 Il re Salomone regnava su tutto Israele. E questi erano i suoi principali ufficiali:
\par 2 Azaria, figliuolo del sacerdote Tsadok,
\par 3 Elihoref ed Ahija, figliuoli di Scisa, erano segretari; Giosafat, figliuolo di Ahilud, era cancelliere;
\par 4 Benaia, figliuolo di Nathan, era capo dell'esercito; Tsadok e Abiathar erano sacerdoti;
\par 5 Azaria, figliuolo di Nathan, era capo degl'intendenti; Zabud, figliuolo di Nathan, era consigliere intimo del re.
\par 6 Ahishar era maggiordomo, e Adoniram, figliuolo di Abda, era preposto ai tributi.
\par 7 Salomone avea dodici intendenti su tutto Israele, i quali provvedevano al mantenimento del re e della sua casa; ciascuno d'essi dovea provvedervi per un mese all'anno.
\par 8 Questi erano i loro nomi: Ben Hur, nella contrada montuosa di Efraim;
\par 9 Ben-Deker, a Makats, a Shaalbim, a Beth-Scemesh, a Elon di Beth-Hanan;
\par 10 Ben-Hesed, ad Arubboth; aveva Soco e tutto il paese di Hefer;
\par 11 Ben-Abinadab, in tutta la regione di Dor; Tafath, figliuola di Salomone era sua moglie;
\par 12 Baana, figliuolo d'Ahilud, avea Taanac, Meghiddo e tutto Beth-Scean, che è presso a Tsarthan, sotto Jizreel, da Beth-Scean ad Abel-Mehola, e fino al di là di Iokmeam;
\par 13 Ben-Gheber, a Ramoth di Galaad; egli aveva i villaggi di Jair, figliuolo di Manasse, che sono in Galaad; aveva anche la regione di Argob ch'è in Basan, sessanta grandi città murate e munite di sbarre di rame;
\par 14 Ahinadab, figliuolo d'Iddo, a Mahanaim;
\par 15 Ahimaats, in Neftali; anche questi avea preso per moglie Basmath, figliuola di Salomone;
\par 16 Baana, figliuolo di Hushai, in Ascer e ad Aloth;
\par 17 Giosafat, figliuolo di Parna, in Issacar;
\par 18 Scimei, figliuolo di Ela, in Beniamino;
\par 19 Gheber, figliuolo di Uri, nel paese di Galaad, il paese di Sihon, re degli Amorei, e di Og, re di Basan. V'era un solo intendente per tutta questa regione.
\par 20 Giuda e Israele erano numerosissimi, come la rena ch'è sulla riva del mare. Essi mangiavano e bevevano allegramente.
\par 21 E Salomone dominava su tutti i regni di qua dal fiume, fino al paese dei Filistei e sino ai confini dell'Egitto. Essi gli recavano dei doni, e gli furon soggetti tutto il tempo ch'ei visse.
\par 22 Or la provvisione de' viveri di Salomone, per ogni giorno, consisteva in trenta cori di fior di farina e sessanta cori di farina ordinaria;
\par 23 in dieci bovi ingrassati, venti bovi di pastura e cento montoni, senza contare i cervi, le gazzelle, i daini e il pollame di stia.
\par 24 Egli dominava su tutto il paese di qua dal fiume, da Tifsa fino a Gaza, su tutti i re di qua dal fiume, ed era in pace con tutti i confinanti all'intorno.
\par 25 E Giuda ed Israele, da Dan fino a Beer-Sceba, vissero al sicuro ognuno all'ombra della sua vite e del suo fico, tutto il tempo che regnò Salomone.
\par 26 Salomone avea pure quarantamila greppie da cavalli per i suoi carri, e dodicimila cavalieri.
\par 27 E quegli intendenti, un mese all'anno per uno, provvedevano al mantenimento del re Salomone e di tutti quelli che si accostavano alla sua mensa; e non lasciavano mancar nulla.
\par 28 Facevano anche portar l'orzo e la paglia per i cavalli da tiro e da corsa nel luogo dove si trovava il re, ciascuno secondo gli ordini che avea ricevuti.
\par 29 E Dio diede a Salomone sapienza, una grandissima intelligenza e una mente vasta com'è la rena che sta sulla riva del mare.
\par 30 E la sapienza di Salomone superò la sapienza di tutti gli Orientali e tutta la sapienza degli Egiziani.
\par 31 Era più savio d'ogni altro uomo, più di Ethan l'Ezrahita, più di Heman, di Calcol e di Darda, figliuoli di Mahol; e la sua fama si sparse per tutte le nazioni circonvicine.
\par 32 Pronunziò tremila massime e i suoi inni furono in numero di mille e cinque.
\par 33 Parlò degli alberi, dal cedro del Libano all'issopo che spunta dalla muraglia; parlò pure degli animali, degli uccelli, dei rettili, dei pesci.
\par 34 Da tutti i popoli veniva gente per udire la sapienza di Salomone, da parte di tutti i re della terra che avean sentito parlare della sua sapienza.

\chapter{5}

\par 1 Or Hiram, re di Tiro, avendo udito che Salomone era stato unto re in luogo di suo padre, gli mandò i suoi servi; perché Hiram era stato sempre amico di Davide.
\par 2 E Salomone mandò a dire a Hiram:
\par 3 'Tu sai che Davide, mio padre, non poté edificare una casa al nome dell'Eterno, del suo Dio, a motivo delle guerre nelle quali fu impegnato da tutte le parti, finché l'Eterno non gli ebbe posti i suoi nemici sotto la pianta de' piedi.
\par 4 Ma ora l'Eterno, il mio Dio, m'ha dato riposo d'ogn'intorno; io non ho più avversari, né mi grava alcuna calamità.
\par 5 Ho quindi l'intenzione di costruire una casa al nome dell'Eterno, dell'Iddio mio, secondo la promessa che l'Eterno fece a Davide mio padre, quando gli disse: - Il tuo figliuolo ch'io metterò sul tuo trono in luogo di te, sarà quello che edificherà una casa al mio nome. -
\par 6 Or dunque da' ordine che mi si taglino dei cedri del Libano. I miei servi saranno insieme coi servi tuoi, e io ti pagherò pel salario de' tuoi servi tutto quello che domanderai; poiché tu sai che non v'è alcuno fra noi che sappia tagliare il legname, come quei di Sidone'.
\par 7 Quando Hiram ebbe udite le parole di Salomone, ne provò una gran gioia e disse: 'Benedetto sia oggi l'Eterno, che ha dato a Davide un figliuolo savio per regnare sopra questo gran popolo'.
\par 8 E Hiram mandò a dire a Salomone: 'Ho udito quello che m'hai fatto dire. Io farò tutto quello che desideri riguardo al legname di cedro e al legname di cipresso.
\par 9 I miei servi li porteranno dal Libano al mare, e io li spedirò per mare su zattere fino al luogo che tu m'indicherai; quindi li farò sciogliere, e tu li prenderai; e tu, dal canto tuo, farai quel che desidero io, fornendo di viveri la mia casa'.
\par 10 Così Hiram dette a Salomone del legname di cedro e del legname di cipresso, quanto ei ne volle.
\par 11 E Salomone dette a Hiram ventimila cori di grano per il mantenimento della sua casa, e venti cori d'olio vergine; Salomone dava tutto questo a Hiram, anno per anno.
\par 12 L'Eterno diede sapienza a Salomone, come gli aveva promesso; e vi fu pace tra Hiram e Salomone, e fecero tra di loro alleanza.
\par 13 Il re Salomone fece una comandata d'operai in tutto Israele, e furon comandati trentamila uomini.
\par 14 Li mandava al Libano, diecimila al mese, alternativamente; un mese stavano sul Libano, e due mesi a casa; e Adoniram era preposto a questa comandata.
\par 15 Salomone avea inoltre settantamila uomini che portavano i pesi, e ottantamila scalpellini sui monti,
\par 16 senza contare i capi, in numero di tremilatrecento, preposti da Salomone ai lavori, e incaricati di dirigere gli operai.
\par 17 Il re comandò che si scavassero delle pietre grandi, delle pietre di pregio, per fare i fondamenti della casa con pietre da taglio.
\par 18 E gli operai di Salomone e gli operai di Hiram e i Ghiblei tagliarono e prepararono il legname e le pietre per la costruzione della casa.

\chapter{6}

\par 1 Or il quattrocentottantesimo anno dopo l'uscita dei figliuoli d'Israele dal paese d'Egitto, nel quarto anno del suo regno sopra Israele, nel mese di Ziv, che è il secondo mese, Salomone cominciò a costruire la casa consacrata all'Eterno.
\par 2 La casa che il re Salomone costruì per l'Eterno, avea sessanta cubiti di lunghezza, venti di larghezza, trenta di altezza.
\par 3 Il portico sul davanti del luogo santo della casa avea venti cubiti di lunghezza rispondenti alla larghezza della casa, e dieci cubiti di larghezza sulla fronte della casa.
\par 4 E il re fece alla casa delle finestre a reticolato fisso.
\par 5 Egli costruì, a ridosso del muro della casa, tutt'intorno, de' piani che circondavano i muri della casa: del luogo santo e del luogo santissimo; e fece delle camere laterali, tutt'all'intorno.
\par 6 Il piano inferiore era largo cinque cubiti; quello di mezzo sei cubiti, e il terzo sette cubiti; perch'egli avea fatto delle sporgenze tutt'intorno ai muri esterni della casa, affinché le travi non fossero incastrate nei muri della casa.
\par 7 Per la costruzione della casa si servirono di pietre già approntate alla cava; in guisa che nella casa, durante la sua costruzione, non s'udì mai rumore di martello, d'ascia o d'altro strumento di ferro.
\par 8 L'ingresso del piano di mezzo si trovava al lato destro della casa; e per una scala a chiocciola si saliva al piano di mezzo, e dal piano di mezzo al terzo.
\par 9 Dopo aver finito di costruire la casa, Salomone la coperse di travi e di assi di legno di cedro.
\par 10 Fece i piani addossati a tutta la casa dando ad ognuno cinque cubiti d'altezza, e li collegò con la casa con delle travi di cedro.
\par 11 E la parola dell'Eterno fu rivolta a Salomone dicendo:
\par 12 'Quanto a questa casa che tu edifichi, se tu cammini secondo le mie leggi, se metti in pratica i miei precetti e osservi e segui tutti i miei comandamenti, io confermerò in tuo favore la promessa che feci a Davide tuo padre:
\par 13 abiterò in mezzo ai figliuoli d'Israele, e non abbandonerò il mio popolo Israele'.
\par 14 Quando Salomone ebbe finito di costruire la casa,
\par 15 ne rivestì le pareti interne di tavole di cedro, dal pavimento sino alla travatura del tetto; rivestì così di legno l'interno, e coperse il pavimento della casa di tavole di cipresso.
\par 16 Rivestì di tavole di cedro uno spazio di venti cubiti in fondo alla casa, dal pavimento al soffitto; e riserbò quello spazio interno per farne un santuario, il luogo santissimo.
\par 17 I quaranta cubiti sul davanti formavano la casa, vale a dire il tempio.
\par 18 Il legno di cedro, nell'interno della casa, presentava delle sculture di colloquintide e di fiori sbocciati; tutto era di cedro, non si vedeva pietra.
\par 19 Salomone stabilì il santuario nell'interno, in fondo alla casa, per collocarvi l'arca del patto dell'Eterno.
\par 20 Il santuario avea venti cubiti di lunghezza, venti cubiti di larghezza, e venti cubiti d'altezza. Salomone lo ricoprì d'oro finissimo; e davanti al santuario fece un altare di legno di cedro e lo ricoprì d'oro.
\par 21 Salomone ricoprì d'oro finissimo l'interno della casa, e fece passare un velo per mezzo di catenelle d'oro davanti al santuario, che ricoprì d'oro.
\par 22 Ricoprì d'oro tutta la casa, tutta quanta la casa, e ricoprì pur d'oro tutto l'altare che apparteneva al santuario.
\par 23 E fece nel santuario due cherubini di legno d'ulivo, dell'altezza di dieci cubiti ciascuno.
\par 24 L'una delle ali d'un cherubino misurava cinque cubiti, e l'altra, pure cinque cubiti; il che faceva dieci cubiti, dalla punta d'un'ala alla punta dell'altra.
\par 25 Il secondo cherubino era parimente di dieci cubiti; ambedue i cherubini erano delle stesse dimensioni e della stessa forma.
\par 26 L'altezza dell'uno dei cherubini era di dieci cubiti, e tale era l'altezza dell'altro.
\par 27 E Salomone pose i cherubini in mezzo alla casa, nell'interno. I cherubini aveano le ali spiegate, in guisa che l'ala del primo toccava una delle pareti, e l'ala del secondo toccava l'altra parete; le altre ali si toccavano l'una l'altra con le punte, in mezzo alla casa.
\par 28 Salomone ricoprì d'oro i cherubini.
\par 29 E fece ornare tutte le pareti della casa, all'intorno, tanto all'interno quanto all'esterno, di sculture di cherubini, di palme e di fiori sbocciati.
\par 30 E, tanto nella parte interiore quanto nella esteriore, ricoprì d'oro il pavimento della casa.
\par 31 All'ingresso del santuario fece una porta a due battenti, di legno d'ulivo; la sua inquadratura, con gli stipiti, occupava la quinta parte della parete.
\par 32 I due battenti erano di legno d'ulivo. Egli vi fece scolpire dei cherubini, delle palme e dei fiori sbocciati, e li ricoprì d'oro, stendendo l'oro sui cherubini e sulle palme.
\par 33 Fece pure, per la porta del tempio, degli stipiti di legno d'ulivo, che occupavano il quarto della larghezza del muro,
\par 34 e due battenti di legno di cipresso; ciascun battente si componeva di due pezzi mobili.
\par 35 Salomone vi fece scolpire dei cherubini, delle palme e de' fiori sbocciati e li ricoprì d'oro, che distese esattamente sulle sculture.
\par 36 E costruì il muro di cinta del cortile interno con tre ordini di pietre lavorate e un ordine di travatura di cedro.
\par 37 Il quarto anno, nel mese di Ziv, furono gettati i fondamenti della casa dell'Eterno;
\par 38 e l'undicesimo anno, nel mese di Bul, che è l'ottavo mese, la casa fu terminata in tutte le sue parti, secondo il disegno datone. Salomone mise sette anni a fabbricarla.

\chapter{7}

\par 1 Poi Salomone costruì la sua propria casa, e la compì interamente in tredici anni.
\par 2 Fabbricò prima di tutto la casa della 'Foresta del Libano', di cento cubiti di lunghezza, di cinquanta di larghezza e di trenta d'altezza. Era basata su quattro ordini di colonne di cedro, sulle quali poggiava una travatura di cedro.
\par 3 Un soffitto di cedro copriva le camere che poggiavano sulle colonne, e che erano in numero di quarantacinque, quindici per fila.
\par 4 E v'erano tre ordini di camere, le cui finestre si trovavano le une dirimpetto alle altre lungo tutti e tre gli ordini.
\par 5 E tutte le porte coi loro stipiti ed architravi erano quadrangolari, e le finestre dei tre ordini di camere si trovavano le une dirimpetto alle altre, in tutti e tre gli ordini.
\par 6 Fece pure il portico di colonne, avente cinquanta cubiti di lunghezza e trenta di larghezza, con un vestibolo davanti, delle colonne, e una scalinata in fronte.
\par 7 Poi fece il portico del trono dove amministrava la giustizia, e che si chiamò il 'Portico del giudizio'; e lo ricoprì di legno di cedro dal pavimento al soffitto.
\par 8 E la casa sua, dov'egli dimorava, fu costruita nello stesso modo, in un altro cortile, dietro il portico. E fece una casa dello stesso stile di questo portico per la figliuola di Faraone, ch'egli avea sposata.
\par 9 Tutte queste costruzioni erano di pietre scelte, tagliate a misura, segate con la sega, internamente ed esternamente, dai fondamenti ai cornicioni, e al di fuori fino al cortile maggiore.
\par 10 Anche i fondamenti erano di pietre scelte, grandi, di pietre di dieci cubiti, e di pietre di otto cubiti.
\par 11 E al di sopra c'erano delle pietre scelte, tagliate a misura, e del legname di cedro.
\par 12 Il gran cortile avea tutto all'intorno tre ordini di pietre lavorate e un ordine di travi di cedro, come il cortile interiore della casa dell'Eterno e come il portico della casa.
\par 13 Il re Salomone fece venire da Tiro Hiram,
\par 14 figliuolo d'una vedova della tribù di Neftali; suo padre era di Tiro. Egli lavorava in rame; era pieno di sapienza, d'intelletto e d'industria per eseguire qualunque lavoro in rame. Egli si recò dal re Salomone ed eseguì tutti i lavori da lui ordinati.
\par 15 Fece le due colonne di rame. La prima avea diciotto cubiti d'altezza, e una corda di dodici cubiti misurava la circonferenza della seconda.
\par 16 E fuse due capitelli di rame, per metterli in cima alle colonne; l'uno avea cinque cubiti d'altezza, e l'altro cinque cubiti d'altezza.
\par 17 Fece un graticolato, un lavoro d'intreccio, dei festoni a guisa di catenelle, per i capitelli ch'erano in cima alle colonne: sette per il primo capitello, e sette per il secondo.
\par 18 E fece due ordini di melagrane attorno all'uno di que' graticolati, per coprire il capitello ch'era in cima all'una delle colonne; e lo stesso fece per l'altro capitello.
\par 19 I capitelli che erano in cima alle colonne nel portico eran fatti a forma di giglio, ed erano di quattro cubiti.
\par 20 I capitelli posti sulle due colonne erano circondati da duecento melagrane, in alto, vicino alla convessità ch'era al di là del graticolato; c'eran duecento melagrane disposte attorno al primo, e duecento intorno al secondo capitello.
\par 21 Egli rizzò le colonne nel portico del tempio; rizzò la colonna a man destra, e la chiamò Jakin; poi rizzò la colonna a man sinistra, e la chiamò Boaz.
\par 22 In cima alle colonne c'era un lavoro fatto a forma di giglio. Così fu compiuto il lavoro delle colonne.
\par 23 Poi fece il mare di getto, che avea dieci cubiti da un orlo all'altro; era di forma perfettamente rotonda, avea cinque cubiti d'altezza, e una corda di trenta cubiti ne misurava la circonferenza.
\par 24 Sotto all'orlo lo circondavano delle colloquintide, dieci per cubito, facendo tutto il giro del mare; le colloquintide, disposte in due ordini, erano state fuse insieme col mare.
\par 25 Questo posava su dodici buoi, dei quali tre guardavano a settentrione, tre a occidente, tre a mezzogiorno, e tre ad oriente; il mare stava su di essi, e le parti posteriori de' buoi erano vòlte verso il di dentro.
\par 26 Esso avea lo spessore d'un palmo; il suo orlo, fatto come l'orlo d'una coppa, avea la forma d'un fior di giglio; il mare conteneva duemila bati.
\par 27 Fece pure le dieci basi di rame; ciascuna avea quattro cubiti di lunghezza, quattro cubiti di larghezza e tre cubiti d'altezza.
\par 28 E il lavoro delle basi consisteva in questo. Eran formate di riquadri, tenuti assieme per mezzo di sostegni.
\par 29 Sopra i riquadri, fra i sostegni, c'erano dei leoni, de' buoi e dei cherubini; lo stesso, sui sostegni superiori; ma sui sostegni inferiori, sotto i leoni ed i buoi, c'erano delle ghirlande a festoni.
\par 30 Ogni base avea quattro ruote di rame con le sale di rame; e ai quattro angoli c'erano delle mensole, sotto il bacino; queste mensole erano di getto; di faccia a ciascuna stavan delle ghirlande.
\par 31 Al coronamento della base, nell'interno, c'era un'apertura in cui s'adattava il bacino; essa avea un cubito d'altezza, era rotonda, della forma d'una base di colonna, e aveva un cubito e mezzo di diametro; anche lì v'erano delle sculture; i riquadri erano quadrati e non circolari.
\par 32 Le quattro ruote eran sotto i riquadri, le sale delle ruote eran fissate alla base, e l'altezza d'ogni ruota era di un cubito e mezzo.
\par 33 Le ruote eran fatte come quelle d'un carro. Le loro sale, i loro quarti, i loro razzi, i loro mòzzi eran di getto.
\par 34 Ai quattro angoli d'ogni base, c'eran quattro mensole d'un medesimo pezzo con la base.
\par 35 La parte superiore della base terminava con un cerchio di mezzo cubito d'altezza, ed aveva i suoi sostegni e i suoi riquadri tutti d'un pezzo con la base.
\par 36 Sulla parte liscia de' sostegni e sui riquadri, Hiram scolpì dei cherubini, de' leoni e delle palme, secondo gli spazi liberi, e delle ghirlande tutt'intorno.
\par 37 Così fece le dieci basi; la fusione, la misura e la forma eran le stesse per tutte.
\par 38 Poi fece le dieci conche di rame, ciascuna delle quali conteneva quaranta bati, ed era di quattro cubiti; e ogni conca posava sopra una delle dieci basi.
\par 39 Egli collocò le basi così: cinque al lato destro della casa, e cinque al lato sinistro; e pose il mare al lato destro della casa, verso sud-est.
\par 40 Hiram fece pure i vasi per le ceneri, le palette ed i bacini.
\par 41 Così Hiram compì tutta l'opera che il re Salomone gli fece fare per la casa dell'Eterno: le due colonne, le due palle dei capitelli in cima alle colonne, i due reticolati per coprire le due palle dei capitelli in cima alle colonne,
\par 42 le quattrocento melagrane per i due reticolati, a due ordini di melagrane per ogni reticolato che coprivano le due palle dei capitelli in cima alle colonne,
\par 43 le dieci basi, le dieci conche sulle basi,
\par 44 il mare, ch'era unico, e i dodici buoi sotto il mare;
\par 45 i vasi per le ceneri, le palette e i bacini. Tutti questi utensili che Salomone fece fare a Hiram per la casa dell'Eterno erano di rame tirato a pulimento.
\par 46 Il re li fece fondere nella pianura del Giordano, in un suolo argilloso, fra Succoth e Tsarthan.
\par 47 Salomone lasciò tutti questi utensili senza riscontrare il peso del rame, perché erano in grandissima quantità.
\par 48 Salomone fece fabbricare tutti gli arredi della casa dell'Eterno: l'altare d'oro, la tavola d'oro sulla quale si mettevano i pani della presentazione;
\par 49 i candelabri d'oro puro, cinque a destra e cinque a sinistra, davanti al santuario, con i fiori, le lampade e gli smoccolatoi, d'oro;
\par 50 le coppe, i coltelli, i bacini, i cucchiai e i bracieri, d'oro fino; e i cardini d'oro per la porta interna della casa all'ingresso del luogo santissimo, e per la porta della casa all'ingresso del tempio.
\par 51 Così fu compiuta tutta l'opera che il re Salomone fece eseguire per la casa dell'Eterno. Poi Salomone fece portare l'argento, l'oro e gli utensili che Davide suo padre avea consacrati, e li mise nei tesori della casa dell'Eterno.

\chapter{8}

\par 1 Allora Salomone radunò presso di sé a Gerusalemme gli anziani d'Israele e tutti i capi delle tribù, i principi delle famiglie de' figliuoli d'Israele, per portar su l'arca del patto dell'Eterno, dalla città di Davide, cioè da Sion.
\par 2 Tutti gli uomini d'Israele si radunarono presso il re Salomone nel mese di Ethanim, che è il settimo mese, durante la festa.
\par 3 Arrivati che furono gli anziani d'Israele, i sacerdoti presero l'arca,
\par 4 e portarono su l'arca dell'Eterno, la tenda di convegno, e tutti gli utensili sacri ch'erano nella tenda. I sacerdoti ed i Leviti eseguirono il trasporto.
\par 5 Il re Salomone e tutta la raunanza d'Israele convocata presso di lui si raccolsero davanti all'arca, e immolarono pecore e buoi in tal quantità da non potersi contare né calcolare.
\par 6 I sacerdoti portarono l'arca del patto dell'Eterno al luogo destinatole, nel santuario della casa, nel luogo santissimo, sotto le ali dei cherubini;
\par 7 poiché i cherubini aveano le ali spiegate sopra il sito dell'arca, e coprivano dall'alto l'arca e le sue stanghe.
\par 8 Le stanghe aveano una tale lunghezza che le loro estremità si vedevano dal luogo santo, davanti al santuario, ma non si vedevano dal di fuori. Esse son rimaste quivi fino al dì d'oggi.
\par 9 Nell'arca non v'era altro se non le due tavole di pietra che Mosè vi avea deposte sullo Horeb, quando l'Eterno fece patto coi figliuoli d'Israele, dopo che questi furono usciti dal paese d'Egitto.
\par 10 Or avvenne che, mentre i sacerdoti uscivano dal luogo santo, la nuvola riempì la casa dell'Eterno,
\par 11 e i sacerdoti non poterono rimanervi per farvi l'ufficio loro, a motivo della nuvola; poiché la gloria dell'Eterno riempiva la casa dell'Eterno.
\par 12 Allora Salomone disse: 'L'Eterno ha dichiarato che abiterebbe nella oscurità!
\par 13 Io t'ho costruito una casa per tua abitazione, un luogo ove tu dimorerai in perpetuo!'
\par 14 Poi il re voltò la faccia, e benedisse tutta la raunanza d'Israele; e tutta la raunanza d'Israele stava in piedi.
\par 15 E disse: 'Benedetto sia l'Eterno, l'Iddio d'Israele, il quale di sua propria bocca parlò a Davide mio padre, e con la sua potenza ha adempito quel che avea dichiarato dicendo:
\par 16 - Dal giorno che trassi il mio popolo d'Israele dall'Egitto, io non scelsi alcuna città, fra tutte le tribù d'Israele, per edificarvi una casa, ove il mio nome dimorasse; ma scelsi Davide per regnare sul mio popolo d'Israele. -
\par 17 Or Davide, mio padre, ebbe in cuore di costruire una casa al nome dell'Eterno, dell'Iddio d'Israele;
\par 18 ma l'Eterno disse a Davide mio padre: - Quanto all'aver tu avuto in cuore di costruire una casa al mio nome, hai fatto bene ad aver questo in cuore;
\par 19 però, non sarai tu che edificherai la casa; ma il tuo figliuolo che uscirà dalle tue viscere, sarà quegli che costruirà la casa al mio nome. -
\par 20 E l'Eterno ha adempita la parola che avea pronunziata; ed io son sorto in luogo di Davide mio padre, e mi sono assiso sul trono d'Israele, come l'Eterno aveva annunziato, ed ho costruita la casa al nome dell'Eterno, dell'Iddio d'Israele.
\par 21 E vi ho assegnato un posto all'arca, nella quale è il patto dell'Eterno: il patto ch'egli fermò coi nostri padri, quando li trasse fuori dal paese d'Egitto'.
\par 22 Poi Salomone si pose davanti all'altare dell'Eterno, in presenza di tutta la raunanza d'Israele, stese le mani verso il cielo, e disse:
\par 23 'O Eterno, Dio d'Israele! Non v'è Dio che sia simile a te né lassù in cielo, né quaggiù in terra! Tu mantieni il patto e la misericordia verso i tuoi servi che camminano in tua presenza con tutto il cuor loro.
\par 24 Tu hai mantenuta la promessa da te fatta al tuo servo Davide, mio padre; e ciò che dichiarasti con la tua propria bocca, la tua mano oggi l'adempie.
\par 25 Ora dunque, o Eterno, Dio d'Israele, mantieni al tuo servo Davide, mio padre, la promessa che gli facesti, dicendo: - Non ti mancherà mai qualcuno che segga nel mio cospetto sul trono d'Israele, purché i tuoi figliuoli veglino sulla loro condotta, e camminino in mia presenza, come tu hai camminato.
\par 26 Or dunque, o Dio d'Israele, s'avveri la parola che dicesti al tuo servo Davide mio padre!
\par 27 Ma è egli proprio vero che Dio abiti sulla terra? Ecco, i cieli e i cieli de' cieli non ti posson contenere; quanto meno questa casa che io ho costruita!
\par 28 Nondimeno, o Eterno, Dio mio, abbi riguardo alla preghiera del tuo servo e alla sua supplicazione, ascoltando il grido e la preghiera che il tuo servo ti rivolge quest'oggi.
\par 29 Siano gli occhi tuoi notte e giorno aperti su questa casa, sul luogo di cui dicesti: - Quivi sarà il mio nome! - Ascolta la preghiera che il tuo servo farà rivolto a questo luogo!
\par 30 Ascolta la supplicazione del tuo servo e del tuo popolo d'Israele quando pregheranno rivolti a questo luogo; ascoltali dal luogo della tua dimora nei cieli; ascolta e perdona!
\par 31 Se uno pecca contro il suo prossimo, e si esige da lui il giuramento per costringerlo a giurare, se quegli viene a giurare davanti al tuo altare in questa casa,
\par 32 tu ascoltalo dal cielo, agisci e giudica i tuoi servi; condanna il colpevole, facendo ricadere sul suo capo i suoi atti, e dichiara giusto l'innocente, trattandolo secondo la sua giustizia.
\par 33 Quando il tuo popolo Israele sarà sconfitto dal nemico per aver peccato contro di te, se torna a te, se dà gloria al tuo nome e ti rivolge preghiere e supplicazioni in questa casa,
\par 34 tu esaudiscilo dal cielo, perdona al tuo popolo d'Israele il suo peccato, e riconducilo nel paese che desti ai suoi padri.
\par 35 Quando il cielo sarà chiuso e non vi sarà più pioggia a motivo dei loro peccati contro di te, se essi pregano rivolti a questo luogo, se danno gloria al tuo nome e si convertono dai loro peccati perché li hai afflitti,
\par 36 tu esaudiscili dal cielo, perdona il loro peccato ai tuoi servi ed al tuo popolo d'Israele, ai quali mostrerai la buona strada per cui debbon camminare; e manda la pioggia sulla terra, che hai data come eredità al tuo popolo.
\par 37 Quando il paese sarà invaso dalla carestia o dalla peste, dalla ruggine o dal carbone, dalle locuste o dai bruchi, quando il nemico assedierà il tuo popolo, nel suo paese, nelle sue città, quando scoppierà qualsivoglia flagello o epidemia,
\par 38 ogni preghiera, ogni supplicazione che ti sarà rivolta da un individuo o dall'intero tuo popolo d'Israele, allorché ciascuno avrà riconosciuta la piaga del proprio cuore e stenderà le sue mani verso questa casa,
\par 39 tu esaudiscila dal cielo, dal luogo della tua dimora, e perdona; agisci e rendi a ciascuno secondo le sue vie, tu, che conosci il cuore d'ognuno; poiché tu solo conosci il cuore di tutti i figliuoli degli uomini;
\par 40 e fa' sì ch'essi ti temano tutto il tempo che vivranno nel paese che tu desti ai padri nostri.
\par 41 Anche lo straniero, che non è del tuo popolo d'Israele, quando verrà da un paese lontano a motivo del tuo nome,
\par 42 - perché si udrà parlare del tuo gran nome, della tua mano potente e del tuo braccio disteso - quando verrà a pregarti in questa casa,
\par 43 tu esaudiscilo dal cielo, dal luogo della tua dimora, e concedi a questo straniero tutto quello che ti domanderà, affinché tutti i popoli della terra conoscano il tuo nome per temerti, come fa il tuo popolo d'Israele, e sappiano che il tuo nome è invocato su questa casa che io ho costruita!
\par 44 Quando il tuo popolo partirà per muover guerra al suo nemico seguendo la via per la quale tu l'avrai mandato, se innalza preghiere all'Eterno rivolto alla città che tu hai scelta e alla casa che io ho costruita al tuo nome,
\par 45 esaudisci dal cielo le sue preghiere e le sue supplicazioni, e fagli ragione.
\par 46 Quando peccheranno contro di te - poiché non v'è uomo che non pecchi - e tu ti sarai mosso a sdegno contro di loro e li avrai abbandonati in balìa del nemico che li menerà in cattività in un paese ostile, lontano o vicino,
\par 47 se, nel paese dove saranno schiavi, rientrano in se stessi, se tornano a te e ti rivolgono supplicazioni nel paese di quelli che li hanno menati in cattività e dicono: - Abbiam peccato, abbiamo operato iniquamente, siamo stati malvagi,
\par 48 - se tornano a te con tutto il loro cuore e con tutta l'anima loro nel paese dei loro nemici che li hanno menati in cattività, e ti pregano rivolti al loro paese, il paese che tu desti ai loro padri, alla città che tu hai scelta e alla casa che io ho costruita al tuo nome,
\par 49 esaudisci dal cielo, dal luogo della tua dimora, le loro preghiere e le loro supplicazioni, e fa' loro ragione;
\par 50 perdona al tuo popolo che ha peccato contro di te, tutte le trasgressioni di cui si è reso colpevole verso di te, e muovi a pietà per essi quelli che li hanno menati in cattività, affinché abbiano compassione di loro;
\par 51 giacché essi sono il tuo popolo, la tua eredità, e tu li hai tratti fuor dall'Egitto, di mezzo a una fornace da ferro!
\par 52 Siano aperti gli occhi tuoi alle supplicazioni del tuo servo e alle supplicazioni del tuo popolo Israele, per esaudirli in tutto quello che ti chiederanno;
\par 53 poiché tu li hai appartati da tutti i popoli della terra per farne la tua eredità; come dichiarasti per mezzo del tuo servo Mosè, quando traesti dall'Egitto i padri nostri, o Signore, o Eterno!'
\par 54 Or quando Salomone ebbe finito di rivolgere all'Eterno tutta questa preghiera e questa supplicazione, s'alzò di davanti all'altare dell'Eterno dove stava inginocchiato tenendo le mani stese verso il cielo.
\par 55 E, levatosi in piè, benedisse tutta la raunanza d'Israele ad alta voce, dicendo:
\par 56 'Benedetto sia l'Eterno, che ha dato riposo al suo popolo Israele, secondo tutte le promesse che avea fatte; non una delle buone promesse da lui fatte per mezzo del suo servo Mosè, è rimasta inadempiuta.
\par 57 L'Eterno, il nostro Dio, sia con noi, come fu coi nostri padri; non ci lasci e non ci abbandoni,
\par 58 ma inchini i nostri cuori verso di lui, affinché camminiamo in tutte le sue vie, e osserviamo i suoi comandamenti, le sue leggi e i suoi precetti, ch'egli prescrisse ai nostri padri!
\par 59 E le parole di questa mia supplicazione all'Eterno siano giorno e notte presenti all'Eterno, all'Iddio nostro, ond'egli faccia ragione al suo servo e al suo popolo Israele, secondo occorrerà giorno per giorno,
\par 60 affinché tutti i popoli della terra riconoscano che l'Eterno è Dio e non ve n'è alcun altro.
\par 61 Sia dunque il cuor vostro dato interamente all'Eterno, al nostro Dio, per seguire le sue leggi e osservare i suoi comandamenti come fate oggi'.
\par 62 Poi il re e tutto Israele con lui offriron dei sacrifizi davanti all'Eterno.
\par 63 Salomone immolò, come sacrifizio di azioni di grazie offerto all'Eterno, ventiduemila buoi e centoventimila pecore. Così il re e tutti i figliuoli d'Israele dedicarono la casa dell'Eterno.
\par 64 In quel giorno il re consacrò la parte di mezzo del cortile, ch'è davanti alla casa dell'Eterno; poiché offrì quivi gli olocausti, le oblazioni e i grassi dei sacrifizi di azioni di grazie, giacché l'altare di rame, ch'è davanti all'Eterno, era troppo piccolo per contenere gli olocausti, le oblazioni e i grassi dei sacrifizi di azioni di grazie.
\par 65 E in quel tempo Salomone celebrò la festa, e tutto Israele con lui. Ci fu una grande raunanza di gente, venuta da tutto il paese: dai dintorni di Hamath fino al torrente d'Egitto, e raccolta dinanzi all'Eterno, al nostro Dio, per sette giorni e poi per altri sette, in tutto quattordici giorni.
\par 66 L'ottavo giorno licenziò il popolo; e quelli benedirono il re, e se n'andarono alle loro tende allegri e col cuore contento per tutto il bene che l'Eterno avea fatto a Davide, suo servo, e ad Israele, suo popolo.

\chapter{9}

\par 1 Dopo che Salomone ebbe finito di costruire la casa dell'Eterno, la casa del re e tutto quello ch'ebbe gusto e volontà di fare,
\par 2 l'Eterno gli apparve per la seconda volta, come gli era apparito a Gabaon,
\par 3 e gli disse: 'Io ho esaudita la tua preghiera e la supplicazione che hai fatta dinanzi a me; ho santificata questa casa che tu hai edificata per mettervi il mio nome in perpetuo; e gli occhi miei ed il mio cuore saran quivi sempre.
\par 4 E quanto a te, se tu cammini dinanzi a me come camminò Davide, tuo padre, con integrità di cuore e con rettitudine, facendo tutto quello che t'ho comandato, e se osservi le mie leggi e i miei precetti,
\par 5 io stabilirò il trono del tuo regno in Israele in perpetuo, come promisi a Davide tuo padre, dicendo: - Non ti mancherà mai qualcuno che segga sul trono d'Israele.
\par 6 - Ma se voi o i vostri figliuoli vi ritraete dal seguir me, se non osservate i miei comandamenti e le mie leggi che io vi ho posti dinanzi, e andate invece a servire altri dèi ed a prostrarvi dinanzi a loro,
\par 7 io sterminerò Israele d'in sulla faccia del paese che gli ho dato, rigetterò dal mio cospetto la casa che ho consacrata al mio nome, e Israele sarà la favola e lo zimbello di tutti i popoli.
\par 8 E questa casa, per quanto sia così in alto, sarà desolata; e chiunque le passerà vicino rimarrà stupefatto e si metterà a fischiare; e si dirà: - Perché l'Eterno ha egli trattato in tal guisa questo paese e questa casa? - e si risponderà:
\par 9 - Perché hanno abbandonato l'Eterno, l'Iddio loro, il quale trasse i loro padri dal paese d'Egitto, si sono invaghiti d'altri dèi, si sono prostrati dinanzi a loro e li hanno serviti; ecco perché l'Eterno ha fatto venire tutti questi mali su loro'.
\par 10 Or avvenne che, passati i venti anni nei quali Salomone costruì le due case, la casa dell'Eterno e la casa del re,
\par 11 siccome Hiram, re di Tiro, avea fornito a Salomone legname di cedro e di cipresso, e oro, a piacere di lui, il re Salomone diede a Hiram venti città nel paese di Galilea.
\par 12 Hiram uscì da Tiro per veder le città dategli da Salomone; ma non gli piacquero;
\par 13 e disse: 'Che città son queste che tu m'hai date, fratel mio?' E le chiamò 'terra di Kabul' nome ch'è rimasto loro fino al dì d'oggi.
\par 14 Hiram avea mandato al re centoventi talenti d'oro.
\par 15 Or ecco quel che concerne gli operai presi e comandati dal re Salomone per costruire la casa dell'Eterno e la sua propria casa, Millo e le mura di Gerusalemme, Hatsor; Meghiddo e Ghezer.
\par 16 - Faraone, re d'Egitto, era salito a impadronirsi di Ghezer, l'avea data alle fiamme, ed avea ucciso i Cananei che abitavano la città; poi l'aveva data per dote alla sua figliuola, moglie di Salomone.
\par 17 E Salomone ricostruì Ghezer, Beth-Horon inferiore,
\par 18 Baalath e Tadmor nella parte deserta del paese,
\par 19 tutte le città di rifornimento che gli appartenevano, le città per i suoi carri, le città per i suoi cavalieri, insomma tutto quello che gli piacque di costruire a Gerusalemme, al Libano e in tutto il paese del suo dominio.
\par 20 - Di tutta la popolazione ch'era rimasta degli Amorei, degli Hittei, dei Ferezei, degli Hivvei e dei Gebusei, che non erano de' figliuoli d'Israele,
\par 21 vale a dire dei loro discendenti ch'eran rimasti dopo di loro nel paese e che gl'Israeliti non avean potuto votare allo sterminio, Salomone fece tanti servi per le comandate; e tali son rimasti fino al dì d'oggi.
\par 22 Ma de' figliuoli d'Israele Salomone non impiegò alcuno come servo; essi furono la sua gente di guerra, i suoi ministri, i suoi principi, i suoi capitani, i comandanti dei suoi carri e de' suoi cavalieri.
\par 23 I capi, preposti da Salomone alla direzione dei suoi lavori, erano in numero di cinquecentocinquanta, incaricati di sorvegliare la gente che eseguiva i lavori.
\par 24 Non appena la figliuola di Faraone salì dalla città di Davide alla casa che Salomone le avea fatto costruire, questi si mise a costruire Millo.
\par 25 Tre volte all'anno Salomone offriva olocausti e sacrifizi di azioni di grazie sull'altare che egli aveva eretto all'Eterno, e offriva profumi su quello che era posto davanti all'Eterno. Così egli terminò definitivamente la casa.
\par 26 Il re Salomone costruì anche una flotta ad Etsion-Gheber, presso Eloth, sul lido del mar Rosso, nel paese di Edom.
\par 27 Hiram mandò su questa flotta, con la gente di Salomone, la sua propria gente: marinai, che conoscevano il mare.
\par 28 Essi andarono ad Ofir, vi presero dell'oro, quattrocentoventi talenti, e li portarono al re Salomone.

\chapter{10}

\par 1 Or la regina di Sceba avendo udito la fama che circondava Salomone a motivo del nome dell'Eterno, venne a metterlo alla prova con degli enimmi.
\par 2 Essa giunse a Gerusalemme con un numerosissimo seguito, con cammelli carichi di aromi, d'oro in gran quantità, e di pietre preziose; e, recatasi da Salomone, gli disse tutto quello che aveva in cuore.
\par 3 Salomone rispose a tutte le questioni propostegli da lei, e non ci fu cosa che fosse oscura per il re, e ch'ei non sapesse spiegare.
\par 4 E quando la regina di Sceba ebbe veduto tutta la sapienza di Salomone e la casa ch'egli aveva costruita
\par 5 e le vivande della sua mensa e gli alloggi de' suoi servi e l'ordine del servizio de' suoi ufficiali e le loro vesti e i suoi coppieri e gli olocausti ch'egli offriva nella casa dell'Eterno, rimase fuori di sé dalla maraviglia.
\par 6 E disse al re: 'Quello che avevo sentito dire nel mio paese dei fatti tuoi e della tua sapienza era dunque vero.
\par 7 Ma non ci ho creduto finché non son venuta io stessa, e non ho visto con gli occhi miei; ed ora, ecco, non me n'era stata riferita neppure la metà! La tua sapienza e la tua prosperità sorpassano la fama che me n'era giunta!
\par 8 Beata la tua gente, beati questi tuoi servi che stanno del continuo dinanzi a te, ed ascoltano la tua sapienza.
\par 9 Sia benedetto l'Eterno, il tuo Dio, il quale t'ha gradito, mettendoti sul trono d'Israele! L'Eterno ti ha stabilito re, per far ragione e giustizia, perch'egli nutre per Israele un amore perpetuo'.
\par 10 Poi ella donò al re centoventi talenti d'oro, grandissima quantità di aromi, e delle pietre preziose. Non furon mai più portati tanti aromi quanti ne diede la regina di Sceba al re Salomone.
\par 11 (La flotta di Hiram che portava oro da Ofir, portava anche da Ofir del legno di sandalo in grandissima quantità, e delle pietre preziose,
\par 12 e di questo legno di sandalo il re fece delle balaustrate per la casa dell'Eterno e per la casa reale, delle cetre e de' saltèri per i cantori. Di questo legno di sandalo non ne fu più portato, e non se n'è più visto fino al dì d'oggi).
\par 13 Il re Salomone diede alla regina di Sceba tutto quel che essa bramò e chiese, oltre a quello ch'ei le donò con la sua munificenza sovrana. Poi ella si rimise in cammino, e coi suoi servi se ne tornò al suo paese.
\par 14 Or il peso dell'oro che giungeva ogni anno a Salomone, era di seicentosessantasei talenti,
\par 15 oltre quello ch'ei percepiva dai mercanti, dal traffico dei negozianti, da tutti i re d'Arabia e dai governatori del paese.
\par 16 E il re Salomone fece fare duecento scudi grandi d'oro battuto, per ognuno dei quali impiegò seicento sicli d'oro,
\par 17 e trecento scudi d'oro battuto più piccoli, per ognuno dei quali impiegò tre mine d'oro; e il re li mise nella casa della 'Foresta del Libano'.
\par 18 Il re fece pure un gran trono d'avorio, che rivestì d'oro finissimo.
\par 19 Questo trono aveva sei gradini; la sommità del trono era rotonda dalla parte di dietro; il seggio avea due bracci, uno di qua e uno di là; presso i due bracci stavano due leoni,
\par 20 e dodici leoni stavano sui sei gradini, da una parte e dall'altra. Niente di simile era ancora stato fatto in verun altro regno.
\par 21 E tutte le coppe del re Salomone erano d'oro, e tutto il vasellame della casa della 'Foresta del Libano' era d'oro puro. Nulla era d'argento; dell'argento non si faceva alcun conto al tempo di Salomone.
\par 22 Poiché il re aveva in mare una flotta di Tarsis insieme con la flotta di Hiram; e la flotta di Tarsis, una volta ogni tre anni, veniva a portare oro, argento, avorio, scimmie e pavoni.
\par 23 Così il re Salomone fu il più grande di tutti i re della terra per ricchezze e per sapienza.
\par 24 E tutto il mondo cercava di veder Salomone per udir la sapienza che Dio gli avea messa in cuore.
\par 25 E ognuno gli portava il suo dono: vasi d'argento, vasi d'oro, vesti, armi, aromi, cavalli e muli; e questo avveniva ogni anno.
\par 26 Salomone radunò carri e cavalieri, ed ebbe millequattrocento carri e dodicimila cavalieri, che distribuì nelle città dove teneva i suoi carri, e in Gerusalemme presso di sé.
\par 27 E il re fece sì che l'argento era in Gerusalemme così comune come le pietre, e i cedri tanto abbondanti quanto i sicomori della pianura.
\par 28 I cavalli che Salomone aveva, gli venivan menati dall'Egitto; le carovane di mercanti del re li andavano a prendere a mandre, per un prezzo convenuto.
\par 29 Un equipaggio, uscito dall'Egitto e giunto a destinazione, veniva a costare seicento sicli d'argento; un cavallo, centocinquanta. Nello stesso modo, per mezzo di que' mercanti, se ne facean venire per tutti i re degli Hittei e per i re della Siria.

\chapter{11}

\par 1 Or il re Salomone, oltre la figliuola di Faraone, amò molte donne straniere: delle Moabite, delle Ammonite, delle Idumee, delle Sidonie, delle Hittee,
\par 2 donne appartenenti ai popoli dei quali l'Eterno avea detto ai figliuoli d'Israele: 'Non andate da loro e non vengano essi da voi; poiché essi certo pervertirebbero il vostro cuore per farvi seguire i loro dèi'. A tali donne s'unì Salomone ne' suoi amori.
\par 3 Ed ebbe settecento principesse per mogli e trecento concubine; e le sue mogli gli pervertirono il cuore;
\par 4 cosicché, al tempo della vecchiaia di Salomone, le sue mogli gl'inclinarono il cuore verso altri dèi; e il cuore di lui non appartenne tutto quanto all'Eterno, al suo Dio, come avea fatto il cuore di Davide suo padre.
\par 5 E Salomone seguì Astarte, divinità dei Sidonî, e Milcom, l'abominazione degli Ammoniti.
\par 6 Così Salomone fece ciò ch'è male agli occhi dell'Eterno e non seguì pienamente l'Eterno, come avea fatto Davide suo padre.
\par 7 Fu allora che Salomone costruì, sul monte che sta dirimpetto a Gerusalemme, un alto luogo per Kemosh, l'abominazione di Moab, e per Molec, l'abominazione dei figliuoli di Ammon.
\par 8 E fece così per tutte le sue donne straniere, le quali offrivano profumi e sacrifizi ai loro dèi.
\par 9 E l'Eterno s'indignò contro Salomone, perché il cuor di lui s'era alienato dall'Eterno, dall'Iddio d'Israele, che gli era apparito due volte,
\par 10 e gli aveva ordinato, a questo proposito, di non andar dietro ad altri dèi; ma egli non osservò l'ordine datogli dall'Eterno.
\par 11 E l'Eterno disse a Salomone: 'Giacché tu hai agito a questo modo, e non hai osservato il mio patto e le leggi che t'avevo date, io ti strapperò di dosso il reame, e lo darò al tuo servo.
\par 12 Nondimeno, per amor di Davide tuo padre, io non lo farò a te vivente, ma lo strapperò dalle mani del tuo figliuolo.
\par 13 Però, non gli strapperò tutto il reame, ma lascerò una tribù al tuo figliuolo, per amor di Davide mio servo, e per amor di Gerusalemme che io ho scelta'.
\par 14 L'Eterno suscitò un nemico a Salomone: Hadad, l'Idumeo, ch'era della stirpe reale di Edom.
\par 15 Quando Davide sconfisse Edom, e Joab, capo dell'esercito, salì per seppellire i morti, e uccise tutti i maschi che erano in Edom,
\par 16 (poiché Joab rimase in Edom sei mesi, con tutto Israele, finché v'ebbe sterminati tutti i maschi),
\par 17 questo Hadad fuggì con alcuni Idumei, servi di suo padre, per andare in Egitto. - Hadad era allora un giovinetto. -
\par 18 Quelli dunque partirono da Madian, andarono a Paran, presero seco degli uomini di Paran, e giunsero in Egitto da Faraone, re d'Egitto, il quale diede a Hadad una casa, provvide al suo mantenimento, e gli assegnò dei terreni.
\par 19 Hadad entrò talmente nelle grazie di Faraone, che questi gli diede per moglie la sorella della propria moglie, la sorella della regina Tahpenes.
\par 20 E la sorella di Tahpenes gli partorì un figliuolo, Ghenubath, che Tahpenes divezzò in casa di Faraone; e Ghenubath rimase in casa di Faraone tra i figliuoli di Faraone.
\par 21 Or quando Hadad ebbe sentito in Egitto che Davide s'era addormentato coi suoi padri e che Joab, capo dell'esercito, era morto, disse a Faraone: 'Dammi licenza ch'io me ne vada al mio paese'.
\par 22 E Faraone gli rispose: 'Che ti manca da me perché tu cerchi d'andartene al tuo paese?' E quegli replicò: 'Nulla; nondimeno, ti prego, lasciami partire'.
\par 23 Iddio suscitò un altro nemico a Salomone: Rezon, figliuolo d'Eliada, ch'era fuggito dal suo signore Hadadezer, re di Tsoba.
\par 24 Ed egli avea radunato gente intorno a sé ed era diventato capo banda, quando Davide massacrò i Sirî. Egli ed i suoi andarono a Damasco, vi si stabilirono, e regnarono in Damasco.
\par 25 E Rezon fu nemico d'Israele per tutto il tempo di Salomone; e questo, oltre il male già fatto da Hadad. Aborrì Israele e regnò sulla Siria.
\par 26 Anche Geroboamo, servo di Salomone, si ribellò contro il re. Egli era figlio di Nebat, Efrateo di Tsereda, e avea per madre una vedova che si chiamava Tserua.
\par 27 La causa per cui si ribellò contro il re, fu questa. Salomone costruiva Millo e chiudeva la breccia della città di Davide suo padre.
\par 28 Or Geroboamo era un uomo forte e valoroso; e Salomone, veduto come questo giovine lavorava, gli diede la sorveglianza di tutta la gente della casa di Giuseppe, comandata ai lavori.
\par 29 In quel tempo avvenne che Geroboamo, essendo uscito di Gerusalemme, s'imbatté per istrada nel profeta Ahija di Scilo, che portava un mantello nuovo; ed erano loro due soli nella campagna.
\par 30 Ahija prese il mantello nuovo che aveva addosso, lo stracciò in dodici pezzi,
\par 31 e disse a Geroboamo: 'Prendine per te dieci pezzi, perché l'Eterno, l'Iddio d'Israele, dice così: - Ecco, io strappo questo regno dalle mani di Salomone, e te ne darò dieci tribù,
\par 32 ma gli resterà una tribù per amor di Davide mio servo, e per amor di Gerusalemme, della città che ho scelta fra tutte le tribù d'Israele.
\par 33 E ciò, perché i figliuoli d'Israele m'hanno abbandonato, si sono prostrati davanti ad Astarte, divinità dei Sidonî, davanti a Kemosh, dio di Moab e davanti a Milcom, dio dei figliuoli d'Ammon, e non han camminato nelle mie vie per fare ciò ch'è giusto agli occhi miei e per osservare le mie leggi e i miei precetti, come fece Davide, padre di Salomone.
\par 34 Nondimeno non torrò dalle mani di lui tutto il regno, ma lo manterrò principe tutto il tempo della sua vita, per amor di Davide, mio servo, che io scelsi, e che osservò i miei comandamenti e le mie leggi;
\par 35 ma torrò il regno dalle mani del suo figliuolo, e te ne darò dieci tribù;
\par 36 e al suo figliuolo lascerò una tribù affinché Davide, mio servo, abbia sempre una lampada davanti a me in Gerusalemme, nella città che ho scelta per mettervi il mio nome.
\par 37 Io prenderò dunque te, e tu regnerai su tutto quello che l'anima tua desidererà, e sarai re sopra Israele.
\par 38 E se tu ubbidisci a tutto quello che ti comanderò, e cammini nelle mie vie, e fai ciò ch'è giusto agli occhi miei, osservando le mie leggi e i miei comandamenti, come fece Davide mio servo, io sarò con te, ti edificherò una casa stabile, come ne edificai una a Davide, e ti darò Israele;
\par 39 e umilierò così la progenie di Davide, ma non per sempre'. -
\par 40 Perciò Salomone cercò di far morire Geroboamo; ma questi si levò e fuggì in Egitto presso Scishak, re d'Egitto, e rimase in Egitto fino alla morte di Salomone.
\par 41 Or il rimanente delle gesta di Salomone, tutto quello che fece, e la sua sapienza sta scritto nel libro delle gesta di Salomone.
\par 42 Salomone regnò a Gerusalemme, su tutto Israele, quarant'anni.
\par 43 Poi Salomone s'addormentò coi suoi padri, e fu sepolto nella città di Davide suo padre; e Roboamo suo figliuolo gli succedette nel regno.

\chapter{12}

\par 1 Roboamo andò a Sichem, perché tutto Israele era venuto a Sichem per farlo re.
\par 2 Quando Geroboamo, figliuolo di Nebat, ebbe di ciò notizia, si trovava ancora in Egitto, dov'era fuggito per scampare dal re Salomone; stava in Egitto,
\par 3 e quivi lo mandarono a chiamare. Allora Geroboamo e tutta la raunanza d'Israele vennero a parlare a Roboamo, e gli dissero:
\par 4 'Tuo padre ha reso duro il nostro giogo; ora rendi tu più lieve la dura servitù e il giogo pesante che tuo padre ci ha imposti, e noi ti serviremo'.
\par 5 Ed egli rispose loro: 'Andatevene, e tornate da me fra tre giorni'. E il popolo se ne andò.
\par 6 Il re Roboamo si consigliò coi vecchi ch'erano stati al servizio del re Salomone suo padre mentre era vivo, e disse: 'Che mi consigliate voi di rispondere a questo popolo?'
\par 7 E quelli gli parlarono così: 'Se oggi tu ti fai servo di questo popolo, se tu gli cedi, se gli rispondi e gli parli con bontà, ti sarà servo per sempre'.
\par 8 Ma Roboamo abbandonò il consiglio datogli dai vecchi e si consigliò coi giovani ch'eran cresciuti con lui ed erano al suo servizio,
\par 9 e disse loro: 'Come consigliate voi che rispondiamo a questo popolo che m'ha parlato dicendo: - Allevia il giogo che tuo padre ci ha imposto?'
\par 10 E i giovani ch'erano cresciuti con lui, gli parlarono così: 'Ecco quel che dirai a questo popolo che s'è rivolto a te dicendo: - Tuo padre ha reso pesante il nostro giogo, e tu ce lo allevia! - Gli risponderai così: - Il mio dito mignolo è più grosso del corpo di mio padre;
\par 11 ora, mio padre vi ha caricati d'un giogo pesante, ma io lo renderò più pesante ancora; mio padre vi ha castigati con la frusta, e io vi castigherò coi flagelli a punte'.
\par 12 Tre giorni dopo, Geroboamo e tutto il popolo vennero da Roboamo, come aveva ordinato il re dicendo: 'Tornate da me fra tre giorni'.
\par 13 E il re rispose aspramente, abbandonando il consiglio che i vecchi gli aveano dato;
\par 14 e parlò al popolo secondo il consiglio dei giovani, dicendo: 'Mio padre ha reso pesante il vostro giogo, ma io lo renderò più pesante ancora; mio padre vi ha castigati con la frusta, e io vi castigherò coi flagelli a punte'.
\par 15 Così il re non diede ascolto al popolo; perché questa cosa era diretta dall'Eterno, affinché si adempisse la parola da lui detta per mezzo di Ahija di Scilo a Geroboamo, figliuolo di Nebat.
\par 16 E quando tutto il popolo d'Israele vide che il re non gli dava ascolto, rispose al re, dicendo: 'Che abbiam noi da fare con Davide? Noi non abbiam nulla di comune col figliuolo d'Isai! Alle tue tende, o Israele! Provvedi ora tu alla tua casa, o Davide!' E Israele se ne andò alle sue tende.
\par 17 Ma sui figliuoli d'Israele che abitavano nelle città di Giuda, regnò Roboamo.
\par 18 E il re Roboamo mandò loro Adoram, preposto alle comandate; ma tutto Israele lo lapidò, ed egli morì. E il re Roboamo salì in fretta sopra un carro per fuggire a Gerusalemme.
\par 19 Così Israele si ribellò alla casa di Davide, ed è rimasto ribelle fino al dì d'oggi.
\par 20 E quando tutto Israele ebbe udito che Geroboamo era tornato, lo mandò a chiamare perché venisse nella raunanza, e lo fece re su tutto Israele. Nessuno seguitò la casa di Davide, tranne la sola tribù di Giuda.
\par 21 E Roboamo, giunto che fu a Gerusalemme, radunò tutta la casa di Giuda e la tribù di Beniamino, centottantamila uomini, guerrieri scelti, per combattere contro la casa d'Israele e restituire il regno a Roboamo, figliuolo di Salomone.
\par 22 Ma la parola di Dio fu così rivolta a Scemaia, uomo di Dio:
\par 23 'Parla a Roboamo, figliuolo di Salomone, re di Giuda, a tutta la casa di Giuda e di Beniamino e al resto del popolo, e di' loro:
\par 24 - Così parla l'Eterno: Non salite a combattere contro i vostri fratelli, i figliuoli d'Israele! Ognuno se ne torni a casa sua; perché questo è avvenuto per voler mio'. Quelli ubbidirono alla parola dell'Eterno, e se ne tornaron via secondo la parola dell'Eterno.
\par 25 Geroboamo edificò Sichem nella contrada montuosa di Efraim, e vi si stabilì; poi uscì di là, ed edificò Penuel.
\par 26 E Geroboamo disse in cuor suo: 'Ora il regno potrebbe benissimo tornare alla casa di Davide.
\par 27 Se questo popolo sale a Gerusalemme per offrir dei sacrifizi nella casa dell'Eterno, il suo cuore si volgerà verso il suo signore, verso Roboamo re di Giuda, e mi uccideranno, e torneranno a Roboamo re di Giuda'.
\par 28 Il re, quindi, dopo essersi consigliato, fece due vitelli d'oro e disse al popolo: 'Siete ormai saliti abbastanza a Gerusalemme! O Israele, ecco i tuoi dèi, che ti hanno tratto dal paese d'Egitto!'
\par 29 E ne mise uno a Bethel, e l'altro a Dan.
\par 30 Questo diventò un'occasione di peccato; perché il popolo andava fino a Dan per presentarsi davanti ad uno di que' vitelli.
\par 31 Egli fece anche delle case d'alti luoghi, e creò dei sacerdoti presi qua e là di fra il popolo, e che non erano de' figliuoli di Levi.
\par 32 Geroboamo istituì pure una solennità nell'ottavo mese, nel quindicesimo giorno del mese, simile alla solennità che si celebrava in Giuda, e offrì dei sacrifizi sull'altare. Così fece a Bethel perché si offrissero sacrifizi ai vitelli ch'egli avea fatti; e a Bethel stabilì i sacerdoti degli alti luoghi che aveva eretti.
\par 33 Il quindicesimo giorno dell'ottavo mese, mese che aveva scelto di sua testa, Geroboamo salì all'altare che aveva costruito a Bethel, fece una festa per i figliuoli d'Israele, e salì all'altare per offrire profumi.

\chapter{13}

\par 1 Ed ecco che un uomo di Dio giunse da Giuda a Bethel per ordine dell'Eterno, mentre Geroboamo stava presso l'altare per ardere il profumo;
\par 2 e per ordine dell'Eterno si mise a gridare contro l'altare e a dire: 'Altare, altare! così dice l'Eterno: - Ecco, nascerà alla casa di Davide un figliuolo, per nome Giosia, il quale immolerà su di te i sacerdoti degli alti luoghi che su di te ardono profumi, e s'arderanno su di te ossa umane'.
\par 3 E quello stesso giorno diede un segno miracoloso dicendo: 'Questo è il segno che l'Eterno ha parlato: ecco, l'altare si spaccherà, e la cenere che v'è sopra si spanderà'.
\par 4 Quando il re Geroboamo ebbe udita la parola che l'uomo di Dio aveva gridata contro l'altare di Bethel, stese la mano dall'alto dell'altare, e disse: 'Pigliatelo!' Ma la mano che Geroboamo avea stesa contro di lui si seccò, e non poté più ritirarla a sé.
\par 5 E l'altare si spaccò; e la cenere che v'era sopra si disperse, secondo il segno che l'uomo di Dio avea dato per ordine dell'Eterno.
\par 6 Allora il re si rivolse all'uomo di Dio, e gli disse: 'Deh, implora la grazia dell'Eterno, del tuo Dio, e prega per me affinché mi sia resa la mano'. E l'uomo di Dio implorò la grazia dell'Eterno, e il re riebbe la sua mano, che tornò com'era prima.
\par 7 E il re disse all'uomo di Dio: 'Vieni meco a casa; ti ristorerai, e io ti farò un regalo'.
\par 8 Ma l'uomo di Dio rispose al re: 'Quand'anche tu mi dessi la metà della tua casa, io non entrerò da te, e non mangerò pane né berrò acqua in questo luogo;
\par 9 poiché questo è l'ordine che m'è stato dato dall'Eterno: - Tu non vi mangerai pane né berrai acqua, e non tornerai per la strada che avrai fatta, andando'. -
\par 10 Così egli se ne andò per un'altra strada, e non tornò per quella che avea fatta, venendo a Bethel.
\par 11 Or v'era un vecchio profeta che abitava a Bethel; e uno de' suoi figliuoli venne a raccontargli tutte le cose che l'uomo di Dio avea fatte in quel giorno a Bethel, e le parole che avea dette al re. Il padre, udito ch'ebbe il racconto,
\par 12 disse ai suoi figliuoli: 'Per qual via se n'è egli andato?' Poiché i suoi figliuoli avean veduto la via per la quale se n'era andato l'uomo di Dio venuto da Giuda.
\par 13 Ed egli disse ai suoi figliuoli: 'Sellatemi l'asino'. Quelli gli sellarono l'asino; ed egli vi montò su,
\par 14 andò dietro all'uomo di Dio, e lo trovò a sedere sotto un terebinto, e gli disse: 'Sei tu l'uomo di Dio venuto da Giuda?' Quegli rispose: 'Son io'.
\par 15 Allora il vecchio profeta gli disse: 'Vieni meco a casa mia, e prendi un po' di cibo'.
\par 16 Ma quegli rispose: 'Io non posso tornare indietro teco, né entrare da te; e non mangerò pane né berrò acqua teco in questo luogo;
\par 17 poiché m'è stato detto, per ordine dell'Eterno: - Tu non mangerai quivi pane, né berrai acqua, e non tornerai per la strada che avrai fatta, andando'. -
\par 18 L'altro gli disse: 'Anch'io son profeta come sei tu; e un angelo mi ha parlato per ordine dell'Eterno, dicendo: - Rimenalo teco in casa tua, affinché mangi del pane e beva dell'acqua'. - Costui gli mentiva. -
\par 19 Così, l'uomo di Dio tornò indietro con l'altro, e mangiò del pane e bevve dell'acqua in casa di lui.
\par 20 Or mentre sedevano a mensa, la parola dell'Eterno fu rivolta al profeta che avea fatto tornare indietro l'altro;
\par 21 ed egli gridò all'uomo di Dio ch'era venuto da Giuda: 'Così parla l'Eterno: - Giacché tu ti sei ribellato all'ordine dell'Eterno, e non hai osservato il comandamento che l'Eterno, l'Iddio tuo, t'avea dato,
\par 22 e sei tornato indietro, e hai mangiato del pane e bevuto dell'acqua nel luogo del quale egli t'avea detto: Non vi mangiare del pane e non vi bere dell'acqua, il tuo cadavere non entrerà nel sepolcro de' tuoi padri'. -
\par 23 Quando l'uomo di Dio ebbe mangiato e bevuto, il vecchio profeta, che l'avea fatto tornare indietro, gli sellò l'asino.
\par 24 L'uomo di Dio se ne andò, e un leone lo incontrò per istrada, e l'uccise. Il suo cadavere restò disteso sulla strada; l'asino se ne stava presso di lui, e il leone pure presso il cadavere.
\par 25 Quand'ecco passarono degli uomini che videro il cadavere disteso sulla strada e il leone che stava dappresso al cadavere, e vennero a riferire la cosa nella città dove abitava il vecchio profeta.
\par 26 E quando il profeta che avea fatto tornare indietro l'uomo di Dio ebbe ciò udito, disse: 'È l'uomo di Dio, ch'è stato ribelle all'ordine dell'Eterno; perciò l'Eterno l'ha dato in balìa d'un leone, che l'ha sbranato e ucciso, secondo la parola che l'Eterno gli avea detta'.
\par 27 Poi si rivolse ai suoi figliuoli, e disse loro: 'Sellatemi l'asino'. E quelli glielo sellarono.
\par 28 E quegli andò, trovò il cadavere disteso sulla strada, e l'asino e il leone che stavano presso il cadavere; il leone non avea divorato il cadavere né sbranato l'asino.
\par 29 Il profeta prese il cadavere dell'uomo di Dio, lo pose sull'asino, e lo portò indietro; e il vecchio profeta rientrò in città per piangerlo, e per dargli sepoltura.
\par 30 E pose il cadavere nel proprio sepolcro; ed egli e i suoi figliuoli lo piansero, dicendo:
\par 31 'Ahi fratel mio!' E quando l'ebbe seppellito, il vecchio profeta disse ai suoi figliuoli: 'Quando sarò morto, seppellitemi nel sepolcro dov'è sepolto l'uomo di Dio; ponete le ossa mie accanto alle sue.
\par 32 Poiché la parola da lui gridata per ordine dell'Eterno contro l'altare di Bethel e contro tutte le case degli alti luoghi che sono nelle città di Samaria, si verificherà certamente'.
\par 33 Dopo questo fatto, Geroboamo non si distolse dalla sua mala via; creò anzi di nuovo de' sacerdoti degli alti luoghi, prendendoli qua e là di fra il popolo; chiunque voleva, era da lui consacrato, e diventava sacerdote degli alti luoghi.
\par 34 E quella fu, per la casa di Geroboamo, un'occasione di peccato, che attirò su lei la distruzione e lo sterminio di sulla faccia della terra.

\chapter{14}

\par 1 In quel tempo, Abija, figliuolo di Geroboamo, si ammalò.
\par 2 E Geroboamo disse a sua moglie: 'Lèvati, ti prego, e travèstiti, affinché non si conosca che tu sei moglie di Geroboamo, e va' a Sciloh. Ecco, quivi è il profeta Ahija, il quale predisse di me che sarei stato re di questo popolo.
\par 3 E prendi teco dieci pani, delle focacce, un vaso di miele, e va' da lui; egli ti dirà quello che avverrà di questo fanciullo'.
\par 4 La moglie di Geroboamo fece così; si levò, andò a Sciloh, e giunse a casa di Ahija. Ahija non potea vedere, poiché gli s'era offuscata la vista per la vecchiezza.
\par 5 - Or l'Eterno avea detto ad Ahija: 'Ecco, la moglie di Geroboamo sta per venire a consultarti riguardo al suo figliuolo, che è ammalato. Tu parlale così e così. Quando entrerà, fingerà d'essere un'altra'. -
\par 6 Come dunque Ahija udì il rumore de' piedi di lei che entrava per la porta, disse: 'Entra pure, moglie di Geroboamo; perché fingi d'essere un'altra? Io sono incaricato di dirti delle cose dure.
\par 7 Va' e di' a Geroboamo: - Così parla l'Eterno, l'Iddio d'Israele: Io t'ho innalzato di mezzo al popolo, t'ho fatto principe del mio popolo Israele,
\par 8 ed ho strappato il regno dalle mani della casa di Davide e l'ho dato a te, ma tu non sei stato come il mio servo Davide il quale osservò i miei comandamenti e mi seguì con tutto il suo cuore, non facendo se non ciò ch'è giusto agli occhi miei,
\par 9 e hai fatto peggio di tutti quelli che t'hanno preceduto, e sei andato a farti degli altri dèi e delle immagini fuse per provocarmi ad ira ed hai gettato me dietro alle tue spalle;
\par 10 per questo ecco ch'io faccio scender la sventura sulla casa di Geroboamo, e sterminerò dalla casa di Geroboamo fino all'ultimo uomo, tanto chi è schiavo come chi è libero in Israele, e spazzerò la casa di Geroboamo, come si spazza lo sterco finché sia tutto sparito.
\par 11 Quelli della casa di Geroboamo che morranno in città, saran divorati dai cani; e quelli che morranno per i campi, li divoreranno gli uccelli del cielo; poiché l'Eterno ha parlato.
\par 12 Quanto a te, lèvati, vattene a casa tua; e non appena avrai messo piede in città, il bambino morrà.
\par 13 E tutto Israele lo piangerà e gli darà sepoltura. Egli è il solo della casa di Geroboamo che sarà messo in un sepolcro, perché è il solo nella casa di Geroboamo in cui si sia trovato qualcosa di buono, rispetto all'Eterno, all'Iddio d'Israele.
\par 14 L'Eterno stabilirà sopra Israele un re, che in quel giorno sterminerà la casa di Geroboamo. E che dico? Non è forse quello che già succede?
\par 15 E l'Eterno colpirà Israele, che sarà come una canna agitata nell'acqua; sradicherà Israele da questa buona terra che avea data ai loro padri, e li disperderà oltre il fiume, perché si son fatti degl'idoli di Astarte provocando ad ira l'Eterno.
\par 16 E abbandonerà Israele a cagion dei peccati che Geroboamo ha commessi e fatti commettere a Israele'. -
\par 17 La moglie di Geroboamo si levò, partì, e giunse a Tirtsa; e com'ella metteva il piede sulla soglia di casa, il fanciullo morì;
\par 18 e lo seppellirono, e tutto Israele lo pianse, secondo la parola che l'Eterno avea pronunziata per bocca del profeta Ahija, suo servo.
\par 19 Il resto delle azioni di Geroboamo e le sue guerre e il modo come regnò, sono cose scritte nel libro delle Cronache dei re d'Israele.
\par 20 E la durata del regno di Geroboamo fu di ventidue anni; poi s'addormentò coi suoi padri, e Nadab suo figliuolo regnò in luogo suo.
\par 21 Roboamo, figliuolo di Salomone, regnò in Giuda. Avea quarantun anni quando cominciò a regnare, e regnò diciassette anni in Gerusalemme, nella città che l'Eterno s'era scelta fra tutte le tribù d'Israele per mettervi il suo nome. Sua madre si chiamava Naama, l'Ammonita.
\par 22 Que' di Giuda fecero ciò ch'è male agli occhi dell'Eterno; e coi peccati che commisero provocarono l'Eterno a gelosia più di quanto avesser fatto i loro padri.
\par 23 Si eressero anch'essi degli alti luoghi con delle statue e degl'idoli d'Astarte su tutte le alte colline e sotto ogni albero verdeggiante.
\par 24 V'erano anche nel paese degli uomini che si prostituivano. Essi praticarono tutti gli atti abominevoli delle nazioni che l'Eterno avea cacciate d'innanzi ai figliuoli d'Israele.
\par 25 L'anno quinto del regno di Roboamo, Scishak, re d'Egitto, salì contro Gerusalemme,
\par 26 e portò via i tesori della casa dell'Eterno e i tesori della casa del re; portò via ogni cosa; prese pure tutti gli scudi d'oro che Salomone avea fatti;
\par 27 invece de' quali Roboamo fece fare degli scudi di rame, e li affidò ai capitani della guardia che custodiva la porta della casa del re.
\par 28 E ogni volta che il re entrava nella casa dell'Eterno, quei della guardia li portavano; poi li riportavano nella sala della guardia.
\par 29 Il resto delle azioni di Roboamo e tutto quello ch'ei fece, sta scritto nel libro delle Cronache dei re di Giuda.
\par 30 Or vi fu guerra continua fra Roboamo e Geroboamo.
\par 31 E Roboamo s'addormentò coi suoi padri e con essi fu sepolto nella città di Davide. Sua madre si chiamava Naama, l'Ammonita. Ed Abijam, suo figliuolo, regnò in luogo suo.

\chapter{15}

\par 1 Il diciottesimo anno del regno di Geroboamo, figliuolo di Nebat, Abijam cominciò a regnare sopra Giuda.
\par 2 Regnò tre anni in Gerusalemme. Sua madre si chiamava Maaca, figliuola di Abishalom.
\par 3 Egli s'abbandonò a tutti i peccati che suo padre avea commessi prima di lui, e il suo cuore non fu tutto quanto per l'Eterno, l'Iddio suo, com'era stato il cuore di Davide suo padre.
\par 4 Nondimeno, per amor di Davide, l'Eterno, il suo Dio, gli lasciò una lampada a Gerusalemme, stabilendo dopo di lui il suo figliuolo, e lasciando sussistere Gerusalemme;
\par 5 perché Davide avea fatto ciò ch'è giusto agli occhi dell'Eterno, e non si era scostato in nulla dai suoi comandamenti per tutto il tempo della sua vita, salvo nel fatto di Uria, lo Hitteo.
\par 6 Or fra Roboamo e Geroboamo vi fu guerra, finché Roboamo visse.
\par 7 Il resto delle azioni di Abijam e tutto quello ch'ei fece, sta scritto nel libro delle Cronache dei re di Giuda. E vi fu guerra fra Abijam e Geroboamo.
\par 8 E Abijam s'addormentò coi suoi padri, e fu sepolto nella città di Davide; ed Asa, suo figliuolo, regnò in luogo suo.
\par 9 L'anno ventesimo del regno di Geroboamo, re d'Israele, Asa cominciò a regnare sopra Giuda.
\par 10 Regnò quarantun anni in Gerusalemme. Sua madre si chiamava Maaca, figliuola d'Abishalom.
\par 11 Asa fece ciò ch'è giusto agli occhi dell'Eterno, come avea fatto Davide suo padre,
\par 12 tolse via dal paese quelli che si prostituivano, fece sparire tutti gl'idoli che i suoi padri aveano fatti,
\par 13 e destituì pure dalla dignità di regina sua madre Maaca, perch'essa avea rizzato un'immagine ad Astarte; Asa abbatté l'immagine, e la bruciò presso al torrente Kidron.
\par 14 Nondimeno, gli alti luoghi non furono eliminati; quantunque il cuore d'Asa fosse tutto quanto per l'Eterno, durante l'intera sua vita.
\par 15 Egli fece portare nella casa dell'Eterno le cose che suo padre avea consacrate, e quelle che avea consacrate egli stesso: argento, oro, vasi.
\par 16 E ci fu guerra fra Asa e Baasa, re d'Israele, tutto il tempo della lor vita.
\par 17 Baasa, re d'Israele, salì contro Giuda, ed edificò Rama, per impedire che alcuno andasse e venisse dalla parte di Asa, re di Giuda.
\par 18 Allora Asa prese tutto l'argento e l'oro ch'era rimasto nei tesori della casa dell'Eterno, prese i tesori della casa del re, e mise tutto in mano dei suoi servi, che mandò a Ben-Hadad, figliuolo di Tabrimmon, figliuolo di Hezion, re di Siria, che abitava a Damasco, per dirgli:
\par 19 'Siavi alleanza fra me e te, come vi fu fra il padre mio e il padre tuo. Ecco, io ti mando in dono dell'argento e dell'oro; va', rompi la tua alleanza con Baasa, re d'Israele, ond'egli si ritiri da me'.
\par 20 Ben-Hadad diè ascolto al re Asa; mandò i capi del suo esercito contro le città d'Israele ed espugnò Ijon, Dan, Abel-Beth-Maaca, tutta la contrada di Kinneroth con tutto il paese di Neftali.
\par 21 E quando Baasa ebbe udito questo, cessò di edificare Rama, e rimase a Tirtsa.
\par 22 Allora il re Asa convocò tutti que' di Giuda, senza eccettuarne alcuno; e quelli portaron via le pietre e il legname di cui Baasa s'era servito per la costruzione di Rama; e con essi il re Asa edificò Gheba di Beniamino, e Mitspa.
\par 23 Il resto di tutte le azioni di Asa, tutte le sue prodezze, tutto quello ch'ei fece e le città che edificò, si trova scritto nel libro delle Cronache dei re di Giuda. Ma, nella sua vecchiaia, egli patì di male ai piedi.
\par 24 E Asa si addormentò coi suoi padri, e fu sepolto con essi nella città di Davide, suo padre; e Giosafat, suo figliuolo, regnò in luogo suo.
\par 25 Nadab, figliuolo di Geroboamo, cominciò a regnare sopra Israele il secondo anno di Asa, re di Giuda; e regnò sopra Israele due anni.
\par 26 E fece ciò ch'è male agli occhi dell'Eterno, e seguì le tracce di suo padre e il peccato nel quale aveva indotto Israele.
\par 27 Baasa, figliuolo di Ahija, della casa d'Issacar, cospirò contro di lui, e lo uccise a Ghibbethon, che apparteneva ai Filistei, mentre Nadab e tutto Israele assediavano Ghibbethon.
\par 28 Baasa lo uccise l'anno terzo di Asa, re di Giuda, e regnò in luogo suo.
\par 29 E, non appena fu re, sterminò tutta la casa di Geroboamo; non risparmiò anima viva di quella casa, ma la distrusse interamente, secondo la parola che l'Eterno avea pronunziata, per bocca del suo servo Ahija lo Scilonita,
\par 30 a motivo de' peccati che Geroboamo avea commessi e fatti commettere a Israele, quando avea provocato ad ira l'Iddio d'Israele.
\par 31 Il resto delle azioni di Nadab e tutto quello che fece, non sono cose scritte nel libro delle Cronache dei re d'Israele?
\par 32 E ci fu guerra fra Asa e Baasa, re d'Israele, tutto il tempo della loro vita.
\par 33 L'anno terzo di Asa, re di Giuda, Baasa, figliuolo di Ahija, cominciò a regnare su tutto Israele. Stava a Tirtsa, e regnò ventiquattro anni.
\par 34 Fece quel ch'è male agli occhi dell'Eterno; e seguì le vie di Geroboamo e il peccato che questi avea fatto commettere a Israele.

\chapter{16}

\par 1 E la parola dell'Eterno fu rivolta a Jehu, figliuolo di Hanani, contro Baasa, in questi termini:
\par 2 'Io t'ho innalzato dalla polvere e t'ho fatto principe del mio popolo Israele, ma tu hai battuto le vie di Geroboamo ed hai indotto il mio popolo Israele a peccare, in guisa da provocarmi a sdegno coi suoi peccati;
\par 3 perciò io spazzerò via Baasa e la sua casa, e farò della casa tua quel che ho fatto della casa di Geroboamo, figliuolo di Nebat.
\par 4 Quelli della famiglia di Baasa che morranno in città, saran divorati dai cani; e quelli che morranno per i campi, li mangeranno gli uccelli del cielo'.
\par 5 Le rimanenti azioni di Baasa, le sue gesta, e le sue prodezze, trovansi scritte nel libro delle Cronache dei re d'Israele.
\par 6 E Baasa si addormentò coi suoi padri, e fu sepolto in Tirtsa; ed Ela, suo figliuolo, regnò in luogo suo.
\par 7 La parola che l'Eterno avea pronunziata per bocca del profeta Jehu, figliuolo di Hanani, fu diretta contro Baasa e contro la casa di lui, non soltanto a motivo di tutto il male che Baasa avea fatto sotto gli occhi dell'Eterno, provocandolo ad ira con l'opera delle sue mani così da imitare la casa di Geroboamo, ma anche perché aveva sterminata quella casa.
\par 8 L'anno ventesimosesto di Asa, re di Giuda, Ela, figliuolo di Baasa, cominciò a regnare sopra Israele. Stava a Tirtsa, e regnò due anni.
\par 9 Zimri, suo servo, comandante della metà dei suoi carri, congiurò contro di lui. Ela era a Tirtsa, bevendo ed ubriacandosi in casa di Artsa, prefetto del palazzo di Tirtsa,
\par 10 quando Zimri entrò, lo colpì e l'uccise, l'anno ventisettesimo d'Asa, re di Giuda, e regnò in luogo suo.
\par 11 E quando fu re, non appena si fu assiso sul trono, distrusse tutta la casa di Baasa; non gli lasciò neppure un bimbo: né parenti, né amici.
\par 12 Così Zimri sterminò tutta la casa di Baasa, secondo la parola che l'Eterno avea pronunziata contro Baasa per bocca del profeta Jehu,
\par 13 a motivo di tutti i peccati che Baasa ed Ela, suo figliuolo, aveano commesso e fatto commettere ad Israele, provocando ad ira l'Eterno, l'Iddio d'Israele, con i loro idoli.
\par 14 Il resto delle azioni d'Ela e tutto quello ch'ei fece, trovasi scritto nel libro delle Cronache dei re d'Israele.
\par 15 L'anno ventisettesimo di Asa, re di Giuda, Zimri regnò per sette giorni in Tirtsa. Or il popolo era accampato contro Ghibbethon, città dei Filistei.
\par 16 Il popolo quivi accampato, sentì dire: 'Zimri ha fatto una congiura e ha perfino ucciso il re!' E quello stesso giorno, nell'accampamento, tutto Israele fece re d'Israele Omri, capo dell'esercito.
\par 17 Ed Omri con tutto Israele salì da Ghibbethon e assediò Tirtsa.
\par 18 Zimri, vedendo che la città era presa, si ritirò nella torre della casa del re, diè fuoco alla casa reale restando sotto alle rovine, e così morì,
\par 19 a motivo de' peccati che aveva commessi, facendo ciò ch'è male agli occhi dell'Eterno, battendo la via di Geroboamo e abbandonandosi al peccato che questi avea commesso, inducendo a peccare Israele.
\par 20 Il resto delle azioni di Zimri, la congiura ch'egli ordì, sono cose scritte nel libro delle Cronache dei re d'Israele.
\par 21 Allora il popolo d'Israele si divise in due parti: metà del popolo seguiva Tibni, figliuolo di Ghinath, per farlo re; l'altra metà seguiva Omri.
\par 22 Ma il popolo che seguiva Omri la vinse contro quello che seguiva Tibni, figliuolo di Ghinath. Tibni morì, e regnò Omri.
\par 23 Il trentunesimo anno d'Asa, re di Giuda, Omri cominciò a regnare sopra Israele, e regnò dodici anni. Regnò sei anni in Tirtsa,
\par 24 poi comprò da Scemer il monte di Samaria per due talenti d'argento; edificò su quel monte una città, e alla città che edificò diede il nome di Samaria dal nome di Scemer, padrone del monte.
\par 25 Omri fece ciò ch'è male agli occhi dell'Eterno, e fece peggio di tutti i suoi predecessori;
\par 26 batté in tutto la via di Geroboamo, figliuolo di Nebat, e s'abbandonò ai peccati che Geroboamo avea fatti commettere a Israele, provocando a sdegno l'Eterno, l'Iddio d'Israele, coi suoi idoli.
\par 27 Il resto delle azioni compiute da Omri e le prodezze da lui fatte, sta tutto scritto nel libro delle Cronache dei re d'Israele.
\par 28 Ed Omri s'addormentò coi suoi padri, e fu sepolto in Samaria; e Achab, suo figliuolo, regnò in luogo suo.
\par 29 Achab, figliuolo di Omri, cominciò a regnare sopra Israele l'anno trentottesimo di Asa, re di Giuda; e regnò in Samaria sopra Israele per ventidue anni.
\par 30 Achab, figliuolo di Omri, fece ciò ch'è male agli occhi dell'Eterno più di tutti quelli che l'aveano preceduto.
\par 31 E, come se fosse stata per lui poca cosa lo abbandonarsi ai peccati di Geroboamo figliuolo di Nebat, prese per moglie Izebel, figliuola di Ethbaal, re dei Sidonî, andò a servire Baal, a prostrarsi dinanzi a lui,
\par 32 ed eresse un altare a Baal, nel tempio di Baal, che edificò a Samaria.
\par 33 Achab fece anche l'idolo d'Astarte. Achab fece più, per provocare a sdegno l'Eterno, l'Iddio d'Israele, di quello che non avean fatto tutti i re d'Israele che l'avean preceduto.
\par 34 Al tempo di lui, Hiel di Bethel, riedificò Gerico; ne gettò le fondamenta su Abiram, suo primogenito, e ne rizzò le porte su Segub, il più giovane de' suoi figliuoli, secondo la parola che l'Eterno avea pronunziata per bocca di Giosuè, figliuolo di Nun.

\chapter{17}

\par 1 Elia, il Tishbita, uno di quelli che s'erano stabiliti in Galaad, disse ad Achab: 'Com'è vero che vive l'Eterno, l'Iddio d'Israele, di cui io son servo, non vi sarà né rugiada né pioggia in questi anni, se non alla mia parola'.
\par 2 E la parola dell'Eterno gli fu rivolta, in questi termini:
\par 3 'Pàrtiti di qua, vòlgiti verso oriente, e nasconditi presso al torrente Kerith, che è dirimpetto al Giordano.
\par 4 Tu berrai al torrente, ed io ho comandato ai corvi che ti dian quivi da mangiare'.
\par 5 Egli dunque partì, e fece secondo la parola dell'Eterno: andò, e si stabilì presso il torrente Kerith, che è dirimpetto al Giordano.
\par 6 E i corvi gli portavano del pane e della carne la mattina, e del pane e della carne la sera; e beveva al torrente.
\par 7 Ma di lì a qualche tempo il torrente rimase asciutto, perché non veniva pioggia sul paese.
\par 8 Allora la parola dell'Eterno gli fu rivolta in questi termini:
\par 9 'Lèvati, va' a Sarepta de' Sidonî, e fa' quivi la tua dimora; ecco, io ho ordinato colà ad una vedova che ti dia da mangiare'.
\par 10 Egli dunque si levò, e andò a Sarepta; e, come giunse alla porta della città, ecco quivi una donna vedova, che raccoglieva delle legna. Egli la chiamò, e le disse: 'Ti prego, vammi a cercare un po' d'acqua in un vaso, affinché io beva'.
\par 11 E mentr'ella andava a prenderne, egli le gridò dietro: 'Portami, ti prego, anche un pezzo di pane'.
\par 12 Ella rispose: 'Com'è vero che vive l'Eterno, il tuo Dio, del pane non ne ho, ma ho solo una manata di farina in un vaso, e un po' d'olio in un orciuolo; ed ecco, sto raccogliendo due stecchi, per andare a cuocerla per me e per il mio figliuolo; e la mangeremo, e poi morremo'.
\par 13 Elia le disse: 'Non temere; va' e fa' come tu hai detto; ma fanne prima una piccola stiacciata per me, e pòrtamela; poi ne farai per te e per il tuo figliuolo.
\par 14 Poiché così dice l'Eterno, l'Iddio d'Israele: - Il vaso della farina non si esaurirà e l'orciuolo dell'olio non calerà, fino al giorno che l'Eterno manderà la pioggia sulla terra'.
\par 15 Ed ella andò e fece come le avea detto Elia; ed essa, la sua famiglia ed Elia ebbero di che mangiare per molto tempo.
\par 16 Il vaso della farina non si esaurì, e l'orciuolo dell'olio non calò, secondo la parola che l'Eterno avea pronunziata per bocca d'Elia.
\par 17 Or dopo queste cose avvenne che il figliuolo di quella donna, ch'era la padrona di casa, si ammalò; e la sua malattia fu così grave, che non gli rimase più soffio di vita.
\par 18 Allora la donna disse ad Elia: 'Che ho io mai da far teco, o uomo di Dio? Sei tu venuto da me per rinnovar la memoria delle mie iniquità e far morire il mio figliuolo?'
\par 19 Ei le rispose: 'Dammi il tuo figliuolo'. E lo prese dal seno di lei, lo portò su nella camera dov'egli albergava, e lo coricò sul suo letto.
\par 20 Poi invocò l'Eterno, e disse: 'O Eterno, Iddio mio, colpisci tu di sventura anche questa vedova, della quale io sono ospite, facendole morire il figliuolo?'
\par 21 Si distese quindi tre volte sul fanciullo, e invocò l'Eterno, dicendo: 'O Eterno, Iddio mio, torni, ti prego, l'anima di questo fanciullo in lui!'
\par 22 E l'Eterno esaudì la voce d'Elia: l'anima del fanciullo tornò in lui, ed ei fu reso alla vita.
\par 23 Elia prese il fanciullo, lo portò giù dalla camera al pian terreno della casa, e lo rimise a sua madre, dicendole: 'Guarda! il tuo figliuolo è vivo'.
\par 24 Allora la donna disse ad Elia: 'Ora riconosco che tu sei un uomo di Dio, e che la parola dell'Eterno ch'è nella tua bocca è verità'.

\chapter{18}

\par 1 Molto tempo dopo, nel corso del terzo anno, la parola dell'Eterno fu rivolta ad Elia, in questi termini: 'Va', presentati ad Achab, e io manderò la pioggia sul paese'.
\par 2 Ed Elia andò a presentarsi ad Achab. Or la carestia era grave in Samaria.
\par 3 E Achab mandò a chiamare Abdia, ch'era il suo maggiordomo. - Or Abdia era molto timorato dell'Eterno;
\par 4 e quando Izebel sterminava i profeti dell'Eterno, Abdia avea preso cento profeti, li avea nascosti cinquanta in una e cinquanta in un'altra spelonca, e li avea sostentati con del pane e dell'acqua.
\par 5 E Achab disse ad Abdia: 'Va' per il paese, verso tutte le sorgenti e tutti i ruscelli; forse troveremo dell'erba e potremo conservare in vita i cavalli e i muli, e non avrem bisogno di uccidere parte del bestiame'.
\par 6 Si spartirono dunque il paese da percorrere; Achab andò da sé da una parte, e Abdia da sé dall'altra.
\par 7 E mentre Abdia era in viaggio, ecco farglisi incontro Elia; e Abdia, avendolo riconosciuto, si prostrò con la faccia a terra, e disse: 'Sei tu il mio signore Elia?'
\par 8 Quegli rispose: 'Son io; va' a dire al tuo signore: - Ecco qua Elia'. -
\par 9 Ma Abdia replicò: 'Che peccato ho io mai commesso, che tu dia il tuo servo nelle mani di Achab, perch'ei mi faccia morire?
\par 10 Com'è vero che l'Eterno, il tuo Dio, vive, non v'è nazione né regno dove il mio signore non abbia mandato a cercarti; e quando gli si diceva: - Ei non è qui, - faceva giurare il regno e la nazione, che proprio non t'avean trovato.
\par 11 E ora tu dici: - Va' a dire al tuo signore: Ecco qua Elia! -
\par 12 Succederà che, quand'io sarò partito da te, lo spirito dell'Eterno ti trasporterà non so dove; io andrò a fare l'ambasciata ad Achab, ed egli, non trovandoti, mi ucciderà. Eppure, il tuo servo teme l'Eterno fin dalla sua giovinezza!
\par 13 Non hanno riferito al mio signore quello ch'io feci quando Izebel uccideva i profeti dell'Eterno? Com'io nascosi cento uomini di que' profeti dell'Eterno, cinquanta in una e cinquanta in un'altra spelonca, e li sostentai con del pane e dell'acqua?
\par 14 E ora tu dici: - Va' a dire al tuo signore: Ecco qua Elia! Ma egli m'ucciderà!'
\par 15 Ed Elia rispose: 'Com'è vero che vive l'Eterno degli eserciti di cui son servo, oggi mi presenterò ad Achab'.
\par 16 Abdia dunque andò a trovare Achab, e gli fece l'ambasciata; e Achab andò incontro ad Elia.
\par 17 E, non appena Achab vide Elia, gli disse: 'Sei tu colui che mette sossopra Israele?'
\par 18 Elia rispose: 'Non io metto sossopra Israele, ma tu e la casa di tuo padre, perché avete abbandonati i comandamenti dell'Eterno, e tu sei andato dietro ai Baali.
\par 19 Manda ora a far raunare tutto Israele presso di me sul monte Carmel, insieme ai quattrocentocinquanta profeti di Baal ed ai quattrocento profeti d'Astarte che mangiano alla mensa di Izebel'.
\par 20 E Achab mandò a chiamare tutti i figliuoli d'Israele, e radunò que' profeti sul monte Carmel.
\par 21 Allora Elia s'accostò a tutto il popolo, e disse: 'Fino a quando zoppicherete voi dai due lati? Se l'Eterno è Dio, seguitelo; se poi lo è Baal, seguite lui'. Il popolo non gli rispose verbo.
\par 22 Allora Elia disse al popolo: 'Son rimasto io solo dei profeti dell'Eterno, mentre i profeti di Baal sono in quattrocentocinquanta.
\par 23 Ci sian dunque dati due giovenchi; quelli ne scelgano uno per loro, lo facciano a pezzi e lo mettano sulle legna, senz'appiccarvi il fuoco; io pure preparerò l'altro giovenco, lo metterò sulle legna, e non v'appiccherò il fuoco.
\par 24 Quindi invocate voi il nome del vostro dio, e io invocherò il nome dell'Eterno; e il dio che risponderà mediante il fuoco, egli sia Dio'. Tutto il popolo rispose e disse: 'Ben detto!'
\par 25 Allora Elia disse ai profeti di Baal: 'Sceglietevi uno de' giovenchi; preparatelo i primi, giacché siete i più numerosi; e invocate il vostro dio, ma non appiccate il fuoco'.
\par 26 E quelli presero il giovenco che fu dato loro, e lo prepararono; poi invocarono il nome di Baal dalla mattina fino al mezzodì, dicendo: 'O Baal, rispondici!' Ma non s'udì né voce né risposta; e saltavano intorno all'altare che aveano fatto.
\par 27 A mezzogiorno, Elia cominciò a beffarsi di loro, e a dire: 'Gridate forte; poich'egli è dio, ma sta meditando, o è andato in disparte, o è in viaggio; fors'anche dorme, e si risveglierà'.
\par 28 E quelli si misero a gridare a gran voce, e a farsi delle incisioni addosso, secondo il loro costume, con delle spade e delle picche, finché grondavan sangue.
\par 29 E passato che fu il mezzogiorno, quelli profetarono fino all'ora in cui si offriva l'oblazione, senza che s'udisse voce o risposta o ci fosse chi desse loro retta.
\par 30 Allora Elia disse a tutto il popolo: 'Accostatevi a me!' E tutto il popolo s'accostò a lui; ed Elia restaurò l'altare dell'Eterno ch'era stato demolito.
\par 31 Poi prese dodici pietre, secondo il numero delle tribù de' figliuoli di Giacobbe, al quale l'Eterno avea detto: 'Il tuo nome sarà Israele'.
\par 32 E con quelle pietre edificò un altare al nome dell'Eterno, e fece intorno all'altare un fosso, della capacità di due misure di grano.
\par 33 Poi vi accomodò le legna, fece a pezzi il giovenco, e lo pose sopra le legna.
\par 34 E disse: 'Empite quattro vasi d'acqua, e versatela sull'olocausto e sulle legna'. Di nuovo disse: 'Fatelo una seconda volta'. E quelli lo fecero una seconda volta. E disse ancora: 'Fatelo per la terza volta'. E quelli lo fecero per la terza volta.
\par 35 L'acqua correva attorno all'altare, ed egli empì d'acqua anche il fosso.
\par 36 E sull'ora in cui si offriva l'oblazione, il profeta Elia si avvicinò e disse: 'O Eterno, Dio d'Abrahamo, d'Isacco e d'Israele, fa' che oggi si conosca che tu sei Dio in Israele, che io sono tuo servo, e che ho fatte tutte queste cose per ordine tuo.
\par 37 Rispondimi, o Eterno, rispondimi, affinché questo popolo riconosca che tu, o Eterno, sei Dio, e che tu sei quegli che converte il cuor loro!'
\par 38 Allora cadde il fuoco dell'Eterno, e consumò l'olocausto, le legna, le pietre e la polvere, e prosciugò l'acqua ch'era nel fosso.
\par 39 Tutto il popolo, veduto ciò, si gettò con la faccia a terra, e disse: 'L'Eterno è Dio! L'Eterno è Dio!'
\par 40 Ed Elia disse loro: 'Pigliate i profeti di Baal; neppur uno ne scampi!' Quelli li pigliarono, ed Elia li fece scendere al torrente Kison, e quivi li scannò.
\par 41 Poi Elia disse ad Achab: 'Risali, mangia e bevi, poiché già s'ode rumor di gran pioggia'.
\par 42 Ed Achab risalì per mangiare e bere; ma Elia salì in vetta al Carmel; e, gettatosi a terra, si mise la faccia tra le ginocchia,
\par 43 e disse al suo servo: 'Or va' su, e guarda dalla parte del mare!' Quegli andò su, guardò, e disse: 'Non v'è nulla'. Elia gli disse: 'Ritornaci sette volte!'
\par 44 E la settima volta, il servo disse: 'Ecco una nuvoletta grossa come la palma della mano, che sale dal mare'. Ed Elia: 'Sali e di' ad Achab: - Attacca i cavalli al carro e scendi, che la pioggia non ti fermi'. -
\par 45 E in un momento il cielo s'oscurò di nubi, il vento si scatenò, e cadde una gran pioggia. Achab montò sul suo carro, e se n'andò a Izreel.
\par 46 E la mano dell'Eterno fu sopra Elia, il quale, cintosi i fianchi, corse innanzi ad Achab fino all'ingresso di Izreel.

\chapter{19}

\par 1 Or Achab raccontò a Izebel tutto quello che Elia avea fatto, e come avea ucciso di spada tutti i profeti.
\par 2 Allora Izebel spedì un messo ad Elia per dirgli: 'Gli dèi mi trattino con tutto il loro rigore, se domani a quest'ora non farò della vita tua quel che tu hai fatto della vita d'ognun di quelli'.
\par 3 Elia, vedendo questo, si levò, e se ne andò per salvarsi la vita; giunse a Beer-Sceba, che appartiene a Giuda, e vi lasciò il suo servo;
\par 4 ma egli s'inoltrò nel deserto una giornata di cammino, andò a sedersi sotto una ginestra, ed espresse il desiderio di morire, dicendo: 'Basta! Prendi ora, o Eterno, l'anima mia, poiché io non valgo meglio de' miei padri!'
\par 5 Poi si coricò, e si addormentò sotto la ginestra; quand'ecco che un angelo lo toccò, e gli disse: 'Alzati e mangia'.
\par 6 Egli guardò, e vide presso il suo capo una focaccia cotta su delle pietre calde, e una brocca d'acqua. Egli mangiò e bevve, poi si coricò di nuovo.
\par 7 E l'angelo dell'Eterno tornò la seconda volta, lo toccò, e disse: 'Alzati e mangia, poiché il cammino è troppo lungo per te'.
\par 8 Egli s'alzò, mangiò e bevve; e per la forza che quel cibo gli dette, camminò quaranta giorni e quaranta notti fino a Horeb, il monte di Dio.
\par 9 E quivi entrò in una spelonca, e vi passò la notte. Ed ecco, gli fu rivolta la parola dell'Eterno, in questi termini: 'Che fai tu qui, Elia?'
\par 10 Egli rispose: 'Io sono stato mosso da una gran gelosia per l'Eterno, per l'Iddio degli eserciti, perché i figliuoli d'Israele hanno abbandonato il tuo patto, han demolito i tuoi altari, e hanno ucciso colla spada i tuoi profeti; son rimasto io solo, e cercano di togliermi la vita'.
\par 11 Iddio gli disse: 'Esci fuori e fermati sul monte, dinanzi all'Eterno'. Ed ecco passava l'Eterno. Un vento forte, impetuoso, schiantava i monti e spezzava le rocce dinanzi all'Eterno, ma l'Eterno non era nel vento. E, dopo il vento, un terremoto; ma l'Eterno non era nel terremoto.
\par 12 E, dopo il terremoto, un fuoco; ma l'Eterno non era nel fuoco. E, dopo il fuoco, un suono dolce e sommesso.
\par 13 Come Elia l'ebbe udito, si coperse il volto col mantello, uscì fuori, e si fermò all'ingresso della spelonca; ed ecco che una voce giunse fino a lui, e disse: 'Che fai tu qui, Elia?'
\par 14 Ed egli rispose: 'Io sono stato mosso da una gran gelosia per l'Eterno, per l'Iddio degli eserciti, perché i figliuoli d'Israele hanno abbandonato il tuo patto, han demolito i tuoi altari, e hanno ucciso colla spada i tuoi profeti; son rimasto io solo, e cercano di togliermi la vita'.
\par 15 E l'Eterno gli disse: 'Va', rifa' la strada del deserto, fino a Damasco; e quando sarai giunto colà, ungerai Hazael come re di Siria;
\par 16 ungerai pure Jehu, figliuolo di Nimsci, come re d'Israele, e ungerai Eliseo, figliuolo di Shafat da Abel-Mehola, come profeta, in luogo tuo.
\par 17 E avverrà che chi sarà scampato dalla spada di Hazael, sarà ucciso da Jehu; e chi sarà scampato dalla spada di Jehu, sarà ucciso da Eliseo.
\par 18 Ma io lascerò in Israele un resto di settemila uomini, tutti quelli il cui ginocchio non s'è piegato dinanzi a Baal, e la cui bocca non l'ha baciato'.
\par 19 Elia si partì di là e trovò Eliseo, figliuolo di Shafat, il quale arava, avendo dodici paia di buoi davanti a sé; ed egli stesso guidava il dodicesimo paio. Elia, avvicinatosi a lui, gli gittò addosso il suo mantello.
\par 20 Ed Eliseo, lasciati i buoi, corse dietro ad Elia, e disse: 'Ti prego, lascia ch'io vada a dar un bacio a mio padre e a mia madre, e poi ti seguirò'. Elia gli rispose: 'Va' e torna; ma pensa a quel che t'ho fatto!'
\par 21 Dopo essersi allontanato da Elia, Eliseo tornò a prendere un paio di bovi, e li offrì in sacrifizio; con le legna degli arnesi de' buoi ne cosse le carni, e le diede alla gente, che le mangiò. Poi si levò, seguitò Elia, e si mise al suo servizio.

\chapter{20}

\par 1 Or Ben-Hadad, re di Siria, radunò tutto il suo esercito; avea seco trentadue re, cavalli e carri; poi salì, cinse d'assedio Samaria, e l'attaccò.
\par 2 E inviò de' messi nella città, che dicessero ad Achab, re d'Israele:
\par 3 'Così dice Ben-Hadad: - Il tuo argento ed il tuo oro sono miei; così pure le tue mogli ed i figliuoli tuoi più belli son cosa mia'. -
\par 4 Il re d'Israele rispose: 'Come dici tu, o re signor mio, io son tuo con tutte le cose mie'.
\par 5 I messi tornarono di nuovo e dissero: 'Così parla Ben-Hadad: - Io t'avevo mandato a dire che tu mi dessi il tuo argento ed il tuo oro, le tue mogli e i tuoi figliuoli;
\par 6 invece, domani, a quest'ora, manderò da te i miei servi, i quali rovisteranno la casa tua e le case dei tuoi servi, e metteran le mani su tutto quello che hai di più caro, e lo porteranno via'. -
\par 7 Allora il re d'Israele chiamò tutti gli anziani del paese, e disse: 'Guardate, vi prego, e vedete come quest'uomo cerca la nostra rovina; poiché mi ha mandato a chiedere le mie mogli, i miei figliuoli, il mio argento e il mio oro, ed io non gli ho rifiutato nulla'.
\par 8 E tutti gli anziani e tutto il popolo gli dissero: 'Non lo ascoltare e non gli condiscendere!'
\par 9 Achab dunque rispose ai messi di Ben-Hadad: 'Dite al re, mio signore: - Tutto quello che facesti dire al tuo servo, la prima volta, io lo farò; ma questo non lo posso fare'. - I messi se ne andarono e portaron la risposta a Ben-Hadad.
\par 10 E Ben-Hadad mandò a dire ad Achab: 'Gli dèi mi trattino con tutto il loro rigore, se la polvere di Samaria basterà ad empire il pugno di tutta la gente che mi segue!'
\par 11 Il re d'Israele rispose: 'Ditegli così: - Chi cinge l'armi non si glori come chi le depone'. -
\par 12 Quando Ben-Hadad ricevette quella risposta era a bere coi re sotto i frascati; e disse ai suoi servi: 'Disponetevi in ordine!' e quelli si disposero ad attaccar la città.
\par 13 Quand'ecco un profeta si accostò ad Achab, re d'Israele, e disse: 'Così dice l'Eterno: - Vedi tu questa gran moltitudine? Ecco, oggi io la darò in tuo potere, e tu saprai ch'io son l'Eterno'. -
\par 14 Achab disse: 'Per mezzo di chi?' E quegli rispose: 'Così dice l'Eterno: - Per mezzo dei servi dei capi delle province'. - Achab riprese: 'Chi comincerà la battaglia?' L'altro rispose: 'Tu'.
\par 15 Allora Achab fece la rassegna de' servi dei capi delle province, ed erano duecentotrentadue; e dopo questi fece la rassegna di tutto il popolo, di tutti i figliuoli d'Israele, ed erano settemila.
\par 16 E fecero una sortita sul mezzogiorno, mentre Ben-Hadad stava a bere e ad ubriacarsi sotto i frascati coi trentadue re, venuti in suo aiuto.
\par 17 I servi dei capi delle province usciron fuori i primi. Ben-Hadad mandò a vedere, e gli fu riferito: 'È uscita gente fuor di Samaria'.
\par 18 Il re disse: 'Se sono usciti per la pace, pigliateli vivi; se sono usciti per la guerra, e vivi pigliateli!'
\par 19 E quando que' servi dei capi delle province e l'esercito che li seguiva furono usciti dalla città,
\par 20 ciascuno di quelli uccise il suo uomo. I Sirî si diedero alla fuga, gl'Israeliti li inseguirono, e Ben-Hadad, re di Siria scampò a cavallo con alcuni cavalieri.
\par 21 Il re d'Israele uscì anch'egli, mise in rotta cavalli e carri, e fece una grande strage fra i Sirî.
\par 22 Allora il profeta si avvicinò al re d'Israele, e gli disse: 'Va', rinforzati; considera bene quel che dovrai fare; perché di qui ad un anno, il re di Siria salirà contro di te'.
\par 23 I servi del re di Siria gli dissero: 'Gli dèi d'Israele son dèi di montagna; per questo ci hanno vinti; ma diamo la battaglia in pianura, e li vinceremo di certo.
\par 24 E tu fa' questo: leva ognuno di quei re dal suo luogo, e metti al posto loro de' capitani;
\par 25 formati quindi un esercito pari a quello che hai perduto, con altrettanti cavalli e altrettanti carri; poi daremo battaglia a costoro in pianura e li vinceremo di certo'. Egli accettò il loro consiglio, e fece così.
\par 26 L'anno seguente Ben-Hadad fece la rassegna dei Sirî, e salì verso Afek per combattere con Israele.
\par 27 Anche i figliuoli d'Israele furon passati in rassegna e provveduti di viveri; quindi mossero contro i Sirî, e si accamparono dirimpetto a loro: parevano due minuscoli greggi di capre di fronte ai Sirî che inondavano il paese.
\par 28 Allora l'uomo di Dio si avvicinò al re d'Israele, e gli disse: 'Così dice l'Eterno: - Giacché i Sirî hanno detto: L'Eterno è Dio de' monti e non è Dio delle valli, io ti darò nelle mani tutta questa gran moltitudine; e voi conoscerete che io sono l'Eterno'. -
\par 29 E stettero accampati gli uni di fronte agli altri per sette giorni; il settimo giorno s'impegnò la battaglia, e i figliuoli d'Israele uccisero de' Sirî, in un giorno, centomila pedoni.
\par 30 Il rimanente si rifugiò nella città di Afek, dove le mura caddero sui ventisettemila uomini ch'erano restati. Anche Ben-Hadad fuggì e, giunto nella città, cercava rifugio di camera in camera.
\par 31 I suoi servi gli dissero: 'Ecco, abbiam sentito dire che i re della casa d'Israele sono dei re clementi; lascia dunque che ci mettiam de' sacchi sui fianchi e delle corde al collo e usciamo incontro al re d'Israele; forse egli ti salverà la vita'.
\par 32 Così essi si misero dei sacchi intorno ai fianchi e delle corde al collo, andarono dal re d'Israele, e dissero: 'Il tuo servo Ben-Hadad dice: - Ti prego, lasciami la vita!' - Achab rispose: 'È ancora vivo? egli è mio fratello'.
\par 33 La qual cosa presero quegli uomini per buon augurio, e subito vollero accertarsi se quello era proprio il suo sentimento, e gli dissero: 'Ben-Hadad è dunque tuo fratello!' Egli rispose: 'Andate, e conducetelo qua'. Ben-Hadad si recò da Achab, il quale lo fece salire sul suo carro.
\par 34 E Ben-Hadad gli disse: 'Io ti restituirò le città che mio padre tolse al padre tuo; e tu ti stabilirai delle vie in Damasco, come mio padre se n'era stabilite in Samaria'. 'Ed io', riprese Achab, 'con questo patto ti lascerò andare'; così Achab fermò il patto con lui, e lo lasciò andare.
\par 35 Allora uno de' figliuoli dei profeti disse per ordine dell'Eterno al suo compagno: 'Ti prego, percuotimi!' Ma quegli non volle percuoterlo.
\par 36 Allora il primo gli disse: 'Poiché tu non hai ubbidito alla voce dell'Eterno, ecco, non appena sarai partito da me, un leone ti ucciderà'. E, non appena quegli si fu partito da lui, un leone lo incontrò e lo uccise.
\par 37 Poi quel profeta trovò un altro uomo, e gli disse: 'Ti prego, percuotimi!' E quegli lo percosse e lo ferì.
\par 38 Allora il profeta andò ad aspettare il re sulla strada, e cangiò il suo aspetto mettendosi una benda sugli occhi.
\par 39 E come il re passava, egli si mise a gridare e disse al re: 'Il tuo servo si trovava in piena battaglia; quand'ecco uno s'avvicina, mi mena un uomo e mi dice: - Custodisci quest'uomo; se mai venisse a mancare, la tua vita pagherà per la sua, ovvero pagherai un talento d'argento. -
\par 40 E mentre il tuo servo era occupato qua e là quell'uomo sparì'. Il re d'Israele gli disse: 'Quella è la tua sentenza; l'hai pronunziata da te stesso'.
\par 41 Allora quegli si tolse immediatamente la benda dagli occhi e il re d'Israele lo riconobbe per uno dei profeti.
\par 42 E il profeta disse al re: 'Così dice l'Eterno: - Giacché ti sei lasciato sfuggir di mano l'uomo che io avevo votato allo sterminio, la tua vita pagherà per la sua, e il tuo popolo per il suo popolo'.
\par 43 E il re d'Israele se ne tornò a casa sua triste ed irritato, e si recò a Samaria.

\chapter{21}

\par 1 Or dopo queste cose avvenne che Naboth d'Izreel aveva in Izreel una vigna presso il palazzo di Achab, re di Samaria.
\par 2 Ed Achab parlò a Naboth, e gli disse: 'Dammi la tua vigna, di cui vo' farmi un orto di erbaggi, perché è contigua alla mia casa; e in sua vece ti darò una vigna migliore; o, se meglio ti conviene, te ne pagherò il valore in danaro'.
\par 3 Ma Naboth rispose ad Achab: 'Mi guardi l'Eterno dal darti l'eredità dei miei padri!'
\par 4 E Achab se ne tornò a casa sua triste ed irritato per quella parola dettagli da Naboth d'Izreel: 'Io non ti darò l'eredità dei miei padri!' Si gettò sul suo letto, voltò la faccia verso il muro, e non prese cibo.
\par 5 Allora Izebel, sua moglie, venne da lui e gli disse: 'Perché hai lo spirito così contristato, e non mangi?'
\par 6 Quegli le rispose: 'Perché ho parlato a Naboth d'Izreel e gli ho detto: Dammi la tua vigna pel danaro che vale; o, se più ti piace, ti darò un'altra vigna invece di quella; ed egli m'ha risposto: - Io non ti darò la mia vigna!' -
\par 7 E Izebel, sua moglie, gli disse: 'Sei tu, sì o no, che eserciti la sovranità sopra Israele? Alzati, prendi cibo, e sta' di buon animo; la vigna di Naboth d'Izreel te la farò aver io'.
\par 8 E scrisse delle lettere a nome di Achab, le sigillò col sigillo di lui, e le mandò agli anziani ed ai notabili della città di Naboth che abitavano insieme con lui.
\par 9 E in quelle lettere scrisse così: 'Bandite un digiuno, e fate sedere Naboth in prima fila davanti al popolo;
\par 10 e mettetegli a fronte due scellerati, i quali depongano contro di lui, dicendo: Tu hai maledetto Iddio ed il re; poi menatelo fuor di città, lapidatelo, e così muoia'.
\par 11 La gente della città di Naboth, gli anziani e i notabili che abitavano nella città, fecero come Izebel avea loro fatto dire, secondo ch'era scritto nelle lettere ch'ella avea loro mandate.
\par 12 Bandirono il digiuno, e fecero sedere Naboth davanti al popolo;
\par 13 i due scellerati vennero a metterglisi a fronte; e questi scellerati deposero così contro di lui, dinanzi al popolo: 'Naboth ha maledetto Iddio ed il re'. Per la qual cosa lo menarono fuori della città, lo lapidarono, sì ch'egli morì.
\par 14 Poi mandarono a dire a Izebel: 'Naboth è stato lapidato ed è morto'.
\par 15 Quando Izebel ebbe udito che Naboth era stato lapidato ed era morto, disse ad Achab: 'Lèvati, prendi possesso della vigna di Naboth d'Izreel, ch'egli rifiutò di darti per danaro; giacché Naboth non vive più, è morto'.
\par 16 E come Achab ebbe udito che Naboth era morto, si levò per scendere alla vigna di Naboth d'Izreel, e prenderne possesso.
\par 17 Allora la parola dell'Eterno fu rivolta ad Elia, il Tishbita, in questi termini:
\par 18 'Lèvati, scendi incontro ad Achab, re d'Israele, che sta in Samaria; ecco, egli è nella vigna di Naboth, dov'è sceso per prenderne possesso.
\par 19 E gli parlerai in questo modo: - Così dice l'Eterno: Dopo aver commesso un omicidio, vieni a prender possesso! - E gli dirai: - Così dice l'Eterno: Nello stesso luogo dove i cani hanno leccato il sangue di Naboth, i cani leccheranno pure il tuo proprio sangue'. -
\par 20 Achab disse ad Elia: 'M'hai tu trovato, nemico mio?' Elia rispose: 'Sì t'ho trovato, perché ti sei venduto a far ciò ch'è male agli occhi dell'Eterno.
\par 21 Ecco, io ti farò venire addosso la sciagura, ti spazzerò via, e sterminerò della casa di Achab ogni maschio, schiavo o libero che sia, in Israele;
\par 22 e ridurrò la tua casa come la casa di Geroboamo, figliuolo di Nebat, e come la casa di Baasa, figliuolo d'Ahija, perché tu m'hai provocato ad ira, ed hai fatto peccare Israele.
\par 23 Anche riguardo a Izebel l'Eterno parla e dice: I cani divoreranno Izebel sotto le mura d'Izreel. -
\par 24 Quei d'Achab che morranno in città saran divorati dai cani, e quei che morranno nei campi saran mangiati dagli uccelli del cielo'. -
\par 25 E veramente non v'è mai stato alcuno che, come Achab, si sia venduto a far ciò ch'è male agli occhi dell'Eterno, perché v'era istigato da sua moglie Izebel.
\par 26 E si condusse in modo abominevole, andando dietro agl'idoli, come avean fatto gli Amorei che l'Eterno avea cacciati d'innanzi ai figliuoli d'Israele. -
\par 27 Quando Achab ebbe udite queste parole, si stracciò le vesti, si coperse il corpo con un sacco, e digiunò; dormiva involto nel sacco, e camminava a passo lento.
\par 28 E la parola dell'Eterno fu rivolta ad Elia, il Tishbita, in questi termini:
\par 29 'Hai tu veduto come Achab s'è umiliato dinanzi a me? Poich'egli s'è umiliato dinanzi a me, io non farò venire la sciagura mentr'egli sarà vivo; ma manderò la sciagura sulla sua casa, durante la vita del suo figliuolo'.

\chapter{22}

\par 1 Passarono tre anni senza guerra tra la Siria e Israele.
\par 2 Ma il terzo anno Giosafat, re di Giuda, scese a trovare il re d'Israele.
\par 3 Or il re d'Israele avea detto ai suoi servi: 'Non sapete voi che Ramoth di Galaad è nostra, e noi ce ne stiam lì tranquilli senza levarla di mano al re di Siria?'
\par 4 E disse a Giosafat: 'Vuoi venire con me alla guerra contro Ramoth di Galaad?' Giosafat rispose al re d'Israele: 'Fa' conto di me come di te stesso, della mia gente come della tua, de' miei cavalli come dei tuoi'.
\par 5 E Giosafat disse al re d'Israele: 'Ti prego, consulta oggi la parola dell'Eterno'.
\par 6 Allora il re d'Israele radunò i profeti, in numero di circa quattrocento, e disse loro: 'Debbo io andare a far guerra a Ramoth di Galaad, o no?' Quelli risposero: 'Va', e il Signore la darà nelle mani del re'.
\par 7 Ma Giosafat disse: 'Non v'ha egli qui alcun altro profeta dell'Eterno da poter consultare?'
\par 8 Il re d'Israele rispose a Giosafat: 'V'è ancora un uomo per mezzo del quale si potrebbe consultare l'Eterno; ma io l'odio perché non mi predice mai nulla di buono, ma soltanto del male: è Micaiah, figliuolo d'Imla'. E Giosafat disse: 'Non dica così il re!'
\par 9 E allora il re d'Israele chiamò un eunuco, e gli disse: 'Fa' venir presto Micaiah, figliuolo d'Imla'.
\par 10 Or il re d'Israele e Giosafat, re di Giuda, sedevano ciascuno sul suo trono, vestiti de' loro abiti reali, nell'aia ch'è all'ingresso della porta di Samaria; e tutti i profeti profetavano dinanzi ad essi.
\par 11 Sedekia, figliuolo di Kenaana, s'era fatto delle corna di ferro, e disse: 'Così dice l'Eterno: - Con queste corna darai di cozzo nei Sirî finché tu li abbia completamente distrutti'.
\par 12 E tutti i profeti profetavano nello stesso modo, dicendo: 'Sali contro Ramoth di Galaad, e vincerai; l'Eterno la darà nelle mani del re'.
\par 13 Or il messo ch'era andato a chiamar Micaiah, gli parlò così: 'Ecco, i profeti tutti, ad una voce, predicono del bene al re; ti prego, sia il tuo parlare come il parlare d'ognun d'essi, e predici del bene!'
\par 14 Ma Micaiah rispose: 'Com'è vero che l'Eterno vive, io dirò quel che l'Eterno mi dirà'.
\par 15 E, come fu giunto dinanzi al re, il re gli disse: 'Micaiah, dobbiam noi andare a far guerra a Ramoth di Galaad, o no?' Quegli rispose: 'Va' pure, tu vincerai; l'Eterno la darà nelle mani del re'.
\par 16 E il re gli disse: 'Quante volte dovrò io scongiurarti di non dirmi se non la verità nel nome dell'Eterno?'
\par 17 Micaiah rispose: 'Ho veduto tutto Israele disperso su per i monti, come pecore che non hanno pastore; e l'Eterno ha detto: - Questa gente non ha padrone; se ne torni ciascuno in pace a casa sua'. -
\par 18 E il re d'Israele disse a Giosafat: 'Non te l'ho io detto che costui non mi predirebbe nulla di buono, ma soltanto del male?'
\par 19 E Micaiah replicò: 'Perciò ascolta la parola dell'Eterno. Io ho veduto l'Eterno che sedeva sul suo trono, e tutto l'esercito del cielo che gli stava dappresso a destra e a sinistra.
\par 20 E l'Eterno disse: - Chi sedurrà Achab affinché salga a Ramoth di Galaad e vi perisca? - E uno rispose in un modo e l'altro in un altro.
\par 21 Allora si fece avanti uno spirito, il quale si presentò dinanzi all'Eterno, e disse: - Lo sedurrò io. -
\par 22 L'Eterno gli disse: - E come? - Quegli rispose: - Io uscirò, e sarò spirito di menzogna in bocca a tutti i suoi profeti. - L'Eterno gli disse: - Sì, riuscirai a sedurlo; esci, e fa' così. -
\par 23 Ed ora ecco che l'Eterno ha posto uno spirito di menzogna in bocca a tutti questi tuoi profeti; ma l'Eterno ha pronunziato del male contro di te'.
\par 24 Allora Sedekia, figliuolo di Kenaana, si accostò, diede uno schiaffo a Micaiah, e disse: 'Per dove è passato lo spirito dell'Eterno quand'è uscito da me per parlare a te?'
\par 25 Micaiah rispose: 'Lo vedrai il giorno che andrai di camera in camera per nasconderti!'
\par 26 E il re d'Israele disse a uno dei suoi servi: 'Prendi Micaiah, menalo da Ammon, governatore della città, e da Joas, figliuolo del re, e di' loro:
\par 27 Così dice il re: Mettete costui in prigione, nutritelo di pan d'afflizione e d'acqua d'afflizione, finch'io ritorni sano e salvo'.
\par 28 E Micaiah disse: 'Se tu ritorni sano e salvo, non sarà l'Eterno quegli che avrà parlato per bocca mia'. E aggiunse: 'Udite questo, o voi popoli tutti!'
\par 29 Il re d'Israele e Giosafat, re di Giuda, saliron dunque contro Ramoth di Galaad.
\par 30 E il re d'Israele disse a Giosafat: 'Io mi travestirò per andare in battaglia; ma tu mettiti i tuoi abiti reali'. E il re d'Israele si travestì, e andò in battaglia.
\par 31 Or il re di Siria avea dato quest'ordine ai trentadue capitani dei suoi carri: 'Non combattete contro veruno, o piccolo o grande, ma contro il solo re d'Israele'.
\par 32 E quando i capitani dei carri scorsero Giosafat dissero: 'Certo, quello è il re d'Israele', e si volsero contro di lui per attaccarlo; ma Giosafat mandò un grido.
\par 33 E allorché i capitani s'accorsero ch'egli non era il re d'Israele, cessarono di dargli addosso.
\par 34 Or qualcuno scoccò a caso la freccia del suo arco, e ferì il re d'Israele tra la corazza e le falde; onde il re disse al suo cocchiere: 'Vòlta, menami fuori del campo, perché son ferito'.
\par 35 Ma la battaglia fu così accanita quel giorno, che il re fu trattenuto sul suo carro in faccia ai Sirî, e morì verso sera; il sangue della sua ferita era colato nel fondo del carro.
\par 36 E come il sole tramontava, un grido corse per tutto il campo: 'Ognuno alla sua città! Ognuno al suo paese!'
\par 37 Così il re morì, fu portato a Samaria, e in Samaria fu sepolto.
\par 38 E quando si lavò il carro presso allo stagno di Samaria - in quell'acqua si lavavano le prostitute - i cani leccarono il sangue di Achab, secondo la parola che l'Eterno avea pronunziata.
\par 39 Or il resto delle azioni di Achab, tutto quello che fece, la casa d'avorio che costruì e tutte le città che edificò, tutto questo sta scritto nel libro delle Cronache dei re d'Israele.
\par 40 Così Achab s'addormentò coi suoi padri, e Achazia, suo figliuolo, regnò in luogo suo.
\par 41 Giosafat, figliuolo di Asa, cominciò a regnare sopra Giuda, l'anno quarto di Achab, re d'Israele.
\par 42 Giosafat avea trentacinque anni quando cominciò a regnare, e regnò venticinque anni a Gerusalemme. Il nome di sua madre era Azuba, figliuola di Scilhi.
\par 43 Egli camminò in tutto per le vie di Asa suo padre, e non se ne allontanò, facendo ciò ch'è giusto agli occhi dell'Eterno.
\par 44 Nondimeno gli alti luoghi non scomparvero; il popolo offriva ancora sacrifizi e profumi sugli alti luoghi.
\par 45 E Giosafat visse in pace col re d'Israele.
\par 46 Or il resto delle azioni di Giosafat, le prodezze che fece e le sue guerre son cose scritte nel libro delle Cronache dei re di Giuda.
\par 47 Egli fece sparire dal paese gli avanzi degli uomini che si prostituivano, che v'eran rimasti dal tempo di Asa suo padre.
\par 48 Or a quel tempo non v'era re in Edom; un governatore fungeva da re.
\par 49 Giosafat costruì delle navi di Tarsis per andare a Ofir in cerca d'oro; ma poi non andò, perché le navi naufragarono a Etsion-Gheber.
\par 50 Allora Achazia, figliuolo d'Achab, disse a Giosafat: 'Lascia che i miei servi vadano coi servi tuoi sulle navi!' Ma Giosafat non volle.
\par 51 E Giosafat si addormentò coi suoi padri, e con essi fu sepolto nella città di Davide, suo padre; e Jehoram, suo figliuolo, regnò in luogo suo.
\par 52 Achazia, figliuolo di Achab, cominciò a regnare sopra Israele a Samaria l'anno diciassettesimo di Giosafat, re di Giuda, e regnò due anni sopra Israele.
\par 53 E fece ciò ch'è male agli occhi dell'Eterno, e camminò per la via di suo padre, per la via di sua madre, e per la via di Geroboamo, figliuolo di Nebat, che avea fatto peccare Israele.
\par 54 E servì a Baal, si prostrò dinanzi a lui, e provocò a sdegno l'Eterno, l'Iddio d'Israele, esattamente come avea fatto suo padre.


\end{document}