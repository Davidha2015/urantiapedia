\begin{document}

\title{II Re}


\chapter{1}

\par 1 Or dopo la morte di Achab, Moab si ribellò contro Israele.
\par 2 Achazia cadde dalla cancellata della sala superiore di un suo appartamento a Samaria, e ne restò ammalato; e spedì dei messi, dicendo loro: 'Andate a consultare Baal-Zebub, dio di Ekron, per sapere se mi riavrò di questa malattia'.
\par 3 Ma un angelo dell'Eterno disse ad Elia il Tishbita: 'Lèvati, sali incontro ai messi del re di Samaria, e di' loro: È forse perché non v'è Dio in Israele che voi andate a consultare Baal-Zebub, dio di Ekron?
\par 4 Perciò, così dice l'Eterno: - Tu non scenderai dal letto sul quale sei salito, ma per certo morrai'. - Ed Elia se ne andò.
\par 5 I messi tornarono ad Achazia, il quale disse loro: 'Perché siete tornati?'
\par 6 E quelli risposero: 'Un uomo ci è venuto incontro, e ci ha detto: Andate, tornate dal re che vi ha mandati, e ditegli: Così dice l'Eterno: - È forse perché non v'è alcun Dio in Israele che tu mandi a consultare Baal-Zebub, dio di Ekron? Perciò, non scenderai dal letto sul quale sei salito, ma per certo morrai'. -
\par 7 Ed Achazia chiese loro: 'Com'era l'uomo che vi è venuto incontro e vi ha detto coteste parole?'
\par 8 Quelli gli risposero: 'Era un uomo vestito di pelo, con una cintola di cuoio intorno ai fianchi'. E Achazia disse: 'È Elia il Tishbita!'
\par 9 Allora mandò un capitano di cinquanta uomini con la sua compagnia ad Elia; quegli salì e trovò Elia che stava seduto in cima al monte. Il capitano gli disse: 'O uomo di Dio, il re dice: - Scendi!' -
\par 10 Elia rispose e disse al capitano dei cinquanta: 'Se io sono un uomo di Dio, scenda del fuoco dal cielo, e consumi te e i tuoi cinquanta uomini!' E dal cielo scese del fuoco che consumò lui e i suoi cinquanta.
\par 11 Achazia mandò di nuovo un altro capitano di cinquanta uomini con la sua compagnia, il quale si rivolse ad Elia e gli disse: 'O uomo di Dio, il re dice così: Fa' presto, scendi!'
\par 12 Elia rispose e disse loro: 'Se io sono un uomo di Dio, scenda del fuoco dal cielo, e consumi te e i tuoi cinquanta uomini'. E dal cielo scese il fuoco di Dio che consumò lui e i suoi cinquanta.
\par 13 Achazia mandò di nuovo un terzo capitano di cinquanta uomini con la sua compagnia. Questo terzo capitano di cinquanta uomini salì da Elia; e, giunto presso a lui, gli si gittò davanti in ginocchio, e lo supplicò, dicendo: 'O uomo di Dio, ti prego, la mia vita e la vita di questi cinquanta tuoi servi sia preziosa agli occhi tuoi!
\par 14 Ecco che del fuoco è sceso dal cielo, e ha consumato i due primi capitani di cinquanta uomini con le loro compagnie; ma ora sia la vita mia preziosa agli occhi tuoi'.
\par 15 E l'angelo dell'Eterno disse ad Elia: 'Scendi con lui; non aver timore di lui'. Elia dunque si levò, scese col capitano, andò dal re, e gli disse:
\par 16 'Così dice l'Eterno: - Poiché tu hai spediti de' messi a consultar Baal-Zebub, dio d'Ekron, quasi che non ci fosse in Israele alcun Dio da poter consultare, perciò tu non scenderai dal letto sul quale sei salito, ma per certo morrai'. -
\par 17 E Achazia morì, secondo la parola dell'Eterno pronunziata da Elia; e Jehoram cominciò a regnare invece di lui l'anno secondo di Jehoram, figliuolo di Giosafat re di Giuda, perché Achazia non aveva figliuoli.
\par 18 Or il resto delle azioni compiute da Achazia sta scritto nel libro delle Cronache dei re d'Israele.

\chapter{2}

\par 1 Or quando l'Eterno volle rapire in cielo Elia in un turbine, Elia si partì da Ghilgal con Eliseo.
\par 2 Ed Elia disse ad Eliseo: 'Fermati qui, ti prego, poiché l'Eterno mi manda fino a Bethel'. Ma Eliseo rispose: 'Com'è vero che l'Eterno vive, e che vive l'anima tua, io non ti lascerò'. Così discesero a Bethel.
\par 3 I discepoli dei profeti ch'erano a Bethel andarono a trovare Eliseo, e gli dissero: 'Sai tu che l'Eterno quest'oggi rapirà in alto il tuo signore?' Quegli rispose: 'Sì, lo so; tacete!'
\par 4 Ed Elia gli disse: 'Eliseo, fermati qui, ti prego, poiché l'Eterno mi manda a Gerico'. Quegli rispose: 'Com'è vero che l'Eterno vive, e che vive l'anima tua, io non ti lascerò'. Così se ne vennero a Gerico.
\par 5 I discepoli dei profeti ch'erano a Gerico s'accostarono ad Eliseo, e gli dissero: 'Sai tu che l'Eterno quest'oggi rapirà in alto il tuo signore?' Quegli rispose: 'Sì, lo so; tacete!'
\par 6 Ed Elia gli disse: 'Fermati qui, ti prego, poiché l'Eterno mi manda al Giordano'. Quegli rispose: 'Com'è vero che l'Eterno vive, e che vive l'anima tua, io non ti lascerò'. E proseguirono il cammino assieme.
\par 7 E cinquanta uomini di tra i discepoli dei profeti andarono dietro a loro e si fermarono dirimpetto al Giordano, da lungi, mentre Elia ed Eliseo si fermarono sulla riva del Giordano.
\par 8 Allora Elia prese il suo mantello, lo rotolò, e percosse le acque, le quali si divisero di qua e di là, in guisa che passarono ambedue a piedi asciutti.
\par 9 E, passati che furono, Elia disse ad Eliseo: 'Chiedi quello che vuoi ch'io faccia per te, prima ch'io ti sia tolto'. Eliseo rispose: 'Ti prego, siami data una parte doppia del tuo spirito!'
\par 10 Elia disse: 'Tu domandi una cosa difficile; nondimeno, se tu mi vedi quando io ti sarò rapito, ti sarà dato quello che chiedi; ma se non mi vedi, non ti sarà dato'.
\par 11 E com'essi continuavano a camminare discorrendo assieme, ecco un carro di fuoco e de' cavalli di fuoco che li separarono l'uno dall'altro, ed Elia salì al cielo in un turbine.
\par 12 E Eliseo lo vide e si mise a gridare: 'Padre mio, padre mio! Carro d'Israele e sua cavalleria!' Poi non lo vide più. E, afferrate le proprie vesti, le strappò in due pezzi;
\par 13 e raccolse il mantello ch'era caduto di dosso ad Elia, tornò indietro, e si fermò sulla riva del Giordano.
\par 14 E, preso il mantello ch'era caduto di dosso ad Elia, percosse le acque, e disse: 'Dov'è l'Eterno, l'Iddio d'Elia?' E quando anch'egli ebbe percosse le acque, queste si divisero di qua e di là, ed Eliseo passò.
\par 15 Quando i discepoli dei profeti che stavano a Gerico di faccia al Giordano ebbero visto Eliseo, dissero: 'Lo spirito d'Elia s'è posato sopra Eliseo'. E gli si fecero incontro, s'inchinarono fino a terra davanti a lui,
\par 16 e gli dissero: 'Ecco qui fra i tuoi servi cinquanta uomini robusti: lascia che vadano in cerca del tuo signore, se mai lo spirito dell'Eterno l'avesse preso e gettato su qualche monte o in qualche valle'. Eliseo rispose: 'Non li mandate'.
\par 17 Ma insistettero tanto, presso di lui, ch'ei ne fu confuso, e disse: 'Mandateli'. Allora quelli mandarono cinquanta uomini, i quali cercarono Elia per tre giorni, e non lo trovarono.
\par 18 E quando furono tornati a lui, che s'era fermato a Gerico, egli disse loro: 'Non vi avevo io detto di non andare?'
\par 19 Or gli abitanti della città dissero ad Eliseo: 'Ecco, il soggiorno di questa città è gradevole, come vede il mio signore; ma le acque son cattive, e il paese è sterile'.
\par 20 Ed egli disse: 'Portatemi una scodella nuova, e mettetevi del sale'. Quelli gliela portarono.
\par 21 Ed egli si recò alla sorgente delle acque, vi gettò il sale, e disse: 'Così dice l'Eterno: - Io rendo sane queste acque, ed esse non saran più causa di morte né di sterilità'.
\par 22 Così le acque furon rese sane e tali son rimaste fino al dì d'oggi, secondo la parola che Eliseo aveva pronunziata.
\par 23 Poi di là Eliseo salì a Bethel; e, come saliva per la via, usciron dalla città dei piccoli ragazzi, i quali lo beffeggiavano, dicendo: 'Sali calvo! Sali calvo!'
\par 24 Egli si voltò, li vide, e li maledisse nel nome dell'Eterno; e due orse uscirono dal bosco, che sbranarono quarantadue di quei ragazzi.
\par 25 Di là Eliseo si recò sul monte Carmel, donde poi tornò a Samaria.

\chapter{3}

\par 1 Or Jehoram, figliuolo di Achab, cominciò a regnare sopra Israele a Samaria l'anno decimottavo di Giosafat, re di Giuda, e regnò dodici anni.
\par 2 Egli fece ciò ch'è male agli occhi dell'Eterno; ma non quanto suo padre e sua madre, perché tolse via la statua di Baal, che suo padre avea fatta.
\par 3 Nondimeno egli rimase attaccato ai peccati coi quali Geroboamo figliuolo di Nebat, aveva fatto peccare Israele e non se ne distolse.
\par 4 Or Mesha, re di Moab, allevava molto bestiame e pagava al re d'Israele un tributo di centomila agnelli e centomila montoni con le loro lane.
\par 5 Ma, morto che fu Achab, il re di Moab si ribellò al re d'Israele.
\par 6 Allora il re Jehoram uscì di Samaria e passò in rassegna tutto Israele;
\par 7 poi si mise in via, e mandò a dire a Giosafat, re di Giuda: 'Il re di Moab mi si è ribellato; vuoi tu venire con me alla guerra contro Moab?' Quegli rispose: 'Verrò; fa' conto di me come di te stesso, del mio popolo come del tuo, de' miei cavalli come dei tuoi'.
\par 8 E soggiunse: 'Per che via saliremo?' Jehoram rispose: 'Per la via del deserto di Edom'.
\par 9 Così il re d'Israele, il re di Giuda e il re di Edom si mossero; e dopo aver girato a mezzodì con una marcia di sette giorni, mancò l'acqua all'esercito e alle bestie che gli andavan dietro.
\par 10 Allora il re d'Israele disse: 'Ahimè, l'Eterno ha chiamati assieme questi tre re, per darli nelle mani di Moab!'
\par 11 Ma Giosafat chiese: 'Non v'ha egli qui alcun profeta dell'Eterno mediante il quale possiam consultare l'Eterno?' Uno dei servi del re d'Israele rispose: 'V'è qui Eliseo, figliuolo di Shafat, il quale versava l'acqua sulle mani d'Elia'. E Giosafat disse:
\par 12 'La parola dell'Eterno è con lui'. Così il re d'Israele, Giosafat e il re di Edom andarono a trovarlo.
\par 13 Eliseo disse al re d'Israele: 'Che ho io da far con te? Vattene ai profeti di tuo padre ed ai profeti di tua madre!' Il re d'Israele gli rispose: 'No, perché l'Eterno ha chiamati insieme questi tre re per darli nelle mani di Moab'.
\par 14 Allora Eliseo disse: 'Com'è vero che vive l'Eterno degli eserciti al quale io servo, se non avessi rispetto a Giosafat, re di Giuda, io non avrei badato a te né t'avrei degnato d'uno sguardo.
\par 15 Ma ora conducetemi qua un sonatore d'arpa'. E, mentre il sonatore arpeggiava, la mano dell'Eterno fu sopra Eliseo,
\par 16 che disse: 'Così parla l'Eterno: Fate in questa valle delle fosse, delle fosse.
\par 17 Poiché così dice l'Eterno: Voi non vedrete vento, non vedrete pioggia, e nondimeno questa valle si riempirà d'acqua; e berrete voi, il vostro bestiame e le vostre bestie da tiro.
\par 18 E questo è ancora poca cosa agli occhi dell'Eterno; perché egli darà anche Moab nelle vostre mani.
\par 19 E voi distruggerete tutte le città fortificate e tutte le città ragguardevoli, abbatterete tutti i buoni alberi, turerete tutte le sorgenti d'acqua, e guasterete con delle pietre ogni buon pezzo di terra'. -
\par 20 La mattina dopo, nell'ora in cui s'offre l'oblazione, ecco che l'acqua arrivò dal lato di Edom e il paese ne fu ripieno.
\par 21 Ora tutti i Moabiti, avendo udito che quei re eran saliti per muover loro guerra, avevan radunato tutti quelli ch'erano in età di portare le armi, e occupavano la frontiera.
\par 22 La mattina, come furono alzati, il sole splendeva sulle acque, e i Moabiti videro, là dirimpetto a loro, le acque rosse come sangue;
\par 23 e dissero: 'Quello è sangue! Quei re son di certo venuti alle mani fra loro e si son distrutti fra loro; or dunque, Moab, alla preda!'
\par 24 E si avanzarono verso il campo d'Israele; ma sorsero gl'Israeliti e sbaragliarono i Moabiti, che fuggirono d'innanzi a loro. Poi penetrarono nel paese, e continuarono a battere Moab.
\par 25 Distrussero le città; ogni buon pezzo di terra lo riempirono di pietre, ciascuno gettandovi la sua; turarono tutte le sorgenti d'acque e abbatterono tutti i buoni alberi. Non rimasero che le mura di Kir-Hareseth, e i frombolieri la circondarono e l'attaccarono.
\par 26 Il re di Moab, vedendo che l'attacco era troppo forte per lui, prese seco settecento uomini, per aprirsi, a spada tratta, un varco, fino al re di Edom; ma non poterono.
\par 27 Allora prese il suo figliuolo primogenito, che dovea succedergli nel regno, e l'offerse in olocausto sopra le mura. A questa vista, un profondo orrore s'impadronì degli Israeliti, che s'allontanarono dal re di Moab e se ne tornarono al loro paese.

\chapter{4}

\par 1 Or una donna di tra le mogli dei discepoli de' profeti esclamò e disse ad Eliseo: 'Il mio marito, tuo servo, è morto; e tu sai che il tuo servo temeva l'Eterno; e il suo creditore è venuto per prendersi i miei due figliuoli e farsene degli schiavi'.
\par 2 Eliseo le disse: 'Che debbo io fare, per te? Dimmi; che hai tu in casa?' Ella rispose: 'La tua serva non ha nulla in casa, tranne un vasetto d'olio'.
\par 3 Allora egli disse: 'Va' fuori, chiedi in prestito da tutti i tuoi vicini de' vasi vuoti; e non ne chieder pochi.
\par 4 Poi torna, serra l'uscio dietro a te ed ai tuoi figliuoli, e versa dell'olio in tutti que' vasi; e, man mano che saran pieni, falli mettere da parte'.
\par 5 Ella dunque si partì da lui, e si chiuse in casa coi suoi figliuoli; questi le portarono i vasi, ed ella vi versava l'olio.
\par 6 E quando i vasi furono pieni, ella disse al suo figliuolo: 'Portami ancora un vaso'. Quegli le rispose: 'Non ce n'è più dei vasi'. E l'olio si fermò.
\par 7 Allora ella andò e riferì tutto all'uomo di Dio, che le disse: 'Va' a vender l'olio, e paga il tuo debito; e di quel che resta sostentati tu ed i tuoi figliuoli'.
\par 8 Or avvenne che un giorno Eliseo passava per Shunem, e c'era quivi una donna ricca che lo trattenne con premura perché prendesse cibo da lei; e tutte le volte che passava di là, si recava da lei a mangiare.
\par 9 Ed ella disse a suo marito: 'Ecco, io son convinta che quest'uomo che passa sempre da noi, è un santo uomo di Dio.
\par 10 Ti prego, facciamogli costruire, di sopra, una piccola camera in muratura, e mettiamoci per lui un letto, un tavolino, una sedia e un candeliere, affinché, quando verrà da noi, egli possa ritirarvisi'.
\par 11 Così, un giorno ch'egli giunse a Shunem, si ritirò su in quella camera, e vi dormì.
\par 12 E disse a Ghehazi, suo servo: 'Chiama questa Shunamita'. Quegli la chiamò, ed ella si presentò davanti a lui.
\par 13 Ed Eliseo disse a Ghehazi: 'Or dille così: - Ecco, tu hai avuto per noi tutta questa premura; che si può fare per te? Hai bisogno che si parli per te al re o al capo dell'esercito?' - Ella rispose:
\par 14 'Io vivo in mezzo al mio popolo'. Ed Eliseo disse: 'Che si potrebbe fare per lei?' Ghehazi rispose: 'Ma! ella non ha figliuoli, e il suo marito è vecchio'.
\par 15 Eliseo gli disse: 'Chiamala!' Ghehazi la chiamò, ed ella si presentò alla porta.
\par 16 Ed Eliseo le disse: 'L'anno prossimo, in questo stesso tempo, tu abbraccerai un figliuolo'. Ella rispose: 'No, signor mio, tu che sei un uomo di Dio, non ingannare la tua serva!'
\par 17 E questa donna concepì e partorì un figliuolo, in quel medesimo tempo, l'anno dopo, come Eliseo le aveva detto.
\par 18 Il bambino si fe' grande; e, un giorno ch'era uscito per andare da suo padre presso i mietitori,
\par 19 disse a suo padre: 'Oh! la mia testa! la mia testa!' Il padre disse al suo servo: 'Portalo a sua madre!'
\par 20 Il servo lo portò via e lo recò a sua madre. Il fanciullo rimase sulle ginocchia di lei fino a mezzogiorno, poi si morì.
\par 21 Allora ella salì, lo adagiò sul letto dell'uomo di Dio, chiuse la porta, ed uscì.
\par 22 E, chiamato il suo marito, disse: 'Ti prego, mandami uno de' servi e un'asina, perché voglio correre dall'uomo di Dio, e tornare'.
\par 23 Il marito le chiese: 'Perché vuoi andar da lui quest'oggi? Non è il novilunio, e non è sabato'. Ella rispose: 'Lascia fare!'
\par 24 Poi fece sellar l'asina e disse al suo servo: 'Guidala, e tira via; non mi fermare per istrada, a meno ch'io tel dica'.
\par 25 Ella dunque partì, e giunse dall'uomo di Dio, sul monte Carmel. E come l'uomo di Dio l'ebbe scorta di lontano, disse a Ghehazi, suo servo: 'Ecco la Shunamita che viene!
\par 26 Ti prego, corri ad incontrarla, e dille: - Stai bene? Sta bene tuo marito? E il bimbo sta bene?' - Ella rispose: 'Stanno bene'.
\par 27 E come fu giunta dall'uomo di Dio, sul monte, gli abbracciò i piedi. Ghehazi si appressò per respingerla; ma l'uomo di Dio disse: 'Lasciala stare, poiché l'anima sua è in amarezza, e l'Eterno me l'ha nascosto, e non me l'ha rivelato'.
\par 28 La donna disse: 'Avevo io forse domandato al mio signore un figliuolo? Non ti diss'io: - Non m'ingannare? -'
\par 29 Allora Eliseo disse a Ghehazi: 'Cingiti i fianchi, prendi in mano il mio bastone, e parti. Se t'imbatti in qualcuno, non lo salutare; e se alcuno ti saluta, non gli rispondere; e poserai il mio bastone sulla faccia del fanciullo'.
\par 30 La madre del fanciullo disse ad Eliseo: 'Com'è vero che l'Eterno vive, e che vive l'anima tua, io non ti lascerò'. Ed Eliseo si levò e le andò appresso.
\par 31 Or Ghehazi, che li avea preceduti, pose il bastone sulla faccia del fanciullo, ma non ci fu né voce né segno alcuno di vita. Tornò quindi incontro ad Eliseo, e gli riferì la cosa, dicendo: 'Il fanciullo non s'è svegliato'.
\par 32 E quando Eliseo arrivò in casa, ecco che il fanciullo era morto e adagiato sul letto di lui.
\par 33 Egli entrò, si chiuse dentro col fanciullo, e pregò l'Eterno.
\par 34 Poi salì sul letto e si coricò sul fanciullo: pose la sua bocca sulla bocca di lui, i suoi occhi sugli occhi di lui, le sue mani sulle mani di lui; si distese sopra di lui, e le carni del fanciullo si riscaldarono.
\par 35 Poi Eliseo s'allontanò, andò qua e là per la casa; poi risalì e si ridistese sopra il fanciullo; e il fanciullo starnutì sette volte, ed aperse gli occhi.
\par 36 Allora Eliseo chiamò Ghehazi, e gli disse: 'Chiama questa Shunamita'. Egli la chiamò; e com'ella fu giunta da Eliseo, questi le disse: 'Prendi il tuo figliuolo'.
\par 37 Ed ella entrò, gli si gettò ai piedi, e si prostrò in terra; poi prese il suo figliuolo, ed uscì.
\par 38 Eliseo se ne tornò a Ghilgal, e v'era carestia nel paese. Or mentre i discepoli de' profeti stavan seduti davanti a lui, egli disse al suo servo: 'Metti il marmittone al fuoco, e cuoci una minestra per i discepoli dei profeti'.
\par 39 E uno di questi uscì fuori nei campi per coglier delle erbe; trovò una specie di vite salvatica, ne colse delle colloquintide, e se n'empì la veste; e, tornato che fu, le tagliò a pezzi nella marmitta dov'era la minestra; perché non si sapeva che cosa fossero.
\par 40 Poi versarono della minestra a quegli uomini perché mangiassero; ma com'essi l'ebbero gustata, esclamarono: 'C'è la morte, nella marmitta, o uomo di Dio!' E non ne poteron mangiare.
\par 41 Eliseo disse: 'Ebbene, portatemi della farina!' La gettò nella marmitta, e disse: 'Versatene a questa gente che mangi'. E non c'era più nulla di cattivo nella marmitta.
\par 42 Giunse poi un uomo da Baal-Shalisha, che portò all'uomo di Dio del pane delle primizie: venti pani d'orzo, e del grano nuovo nella sua bisaccia. Eliseo disse al suo servo: 'Danne alla gente che mangi'.
\par 43 Quegli rispose: 'Come fare a por questo davanti a cento persone?' Ma Eliseo disse: 'Danne alla gente che mangi; perché così dice l'Eterno: - Mangeranno, e ne avanzerà'. -
\par 44 Così egli pose quelle provviste davanti alla gente, che mangiò e ne lasciò d'avanzo, secondo la parola dell'Eterno.

\chapter{5}

\par 1 Or Naaman, capo dell'esercito del re di Siria, era un uomo in grande stima ed onore presso il suo signore, perché per mezzo di lui l'Eterno avea reso vittoriosa la Siria; ma quest'uomo forte e prode era lebbroso.
\par 2 Or alcune bande di Sirî, in una delle loro incursioni, avean condotta prigioniera dal paese d'Israele una piccola fanciulla, ch'era passata al servizio della moglie di Naaman.
\par 3 Ed ella disse alla sua padrona: 'Oh se il mio signore potesse presentarsi al profeta ch'è a Samaria! Questi lo libererebbe dalla sua lebbra!'
\par 4 Naaman andò dal suo signore, e gli riferì la cosa, dicendo: 'Quella fanciulla del paese d'Israele ha detto così e così'.
\par 5 Il re di Siria gli disse: 'Ebbene, va'; io manderò una lettera al re d'Israele'. Quegli dunque partì, prese seco dieci talenti d'argento, seimila sicli d'oro, e dieci mute di vestiti.
\par 6 E portò al re d'Israele la lettera, che diceva: 'Or quando questa lettera ti sarà giunta, saprai che ti mando Naaman mio servo, perché tu lo guarisca dalla sua lebbra'.
\par 7 Quando il re d'Israele ebbe letta la lettera, si stracciò le vesti, e disse: 'Son io forse Dio, col potere di far morire e vivere, che colui manda da me perch'io guarisca un uomo dalla sua lebbra? Tenete per cosa certa ed evidente ch'ei cerca pretesti contro di me'.
\par 8 Quando Eliseo, l'uomo di Dio, ebbe udito che il re s'era stracciato le vesti, gli mandò a dire: 'Perché ti sei stracciato le vesti? Venga pure colui da me, e vedrà che v'è un profeta in Israele'.
\par 9 Naaman dunque venne coi suoi cavalli ed i suoi carri, e si fermò alla porta della casa di Eliseo.
\par 10 Ed Eliseo gl'inviò un messo a dirgli: 'Va', lavati sette volte nel Giordano; la tua carne tornerà sana, e tu sarai puro'.
\par 11 Ma Naaman si adirò e se ne andò, dicendo: 'Ecco, io pensavo: Egli uscirà senza dubbio incontro a me, si fermerà là, invocherà il nome dell'Eterno, del suo Dio, agiterà la mano sulla parte malata, e guarirà il lebbroso.
\par 12 I fiumi di Damasco, l'Abanah e il Farpar, non son essi migliori di tutte le acque d'Israele? Non posso io lavarmi in quelli ed esser mondato?' E, voltatosi, se n'andava infuriato.
\par 13 Ma i suoi servi gli si accostarono per parlargli, e gli dissero: 'Padre mio, se il profeta t'avesse ordinato una qualche cosa difficile, non l'avresti tu fatta? Quanto più ora ch'egli t'ha detto: - Lavati, e sarai mondato'? -
\par 14 Allora egli scese e si tuffò sette volte nel Giordano, secondo la parola dell'uomo di Dio; e la sua carne tornò come la carne d'un piccolo fanciullo, e rimase puro.
\par 15 Poi tornò con tutto il suo séguito all'uomo di Dio, andò a presentarsi davanti a lui, e disse: 'Ecco, io riconosco adesso che non v'è alcun Dio in tutta la terra, fuorché in Israele. Ed ora, ti prego, accetta un regalo dal tuo servo'.
\par 16 Ma Eliseo rispose: 'Com'è vero che vive l'Eterno di cui sono servo, io non accetterò nulla'. Naaman lo pressava ad accettare, ma egli rifiutò.
\par 17 Allora Naaman disse: 'Poiché non vuoi, permetti almeno che sia data al tuo servo tanta terra quanta ne portano due muli; giacché il tuo servo non offrirà più olocausti e sacrifizi ad altri dèi, ma solo all'Eterno.
\par 18 Nondimeno, questa cosa voglia l'Eterno perdonare al tuo servo: quando il mio signore entra nella casa di Rimmon per quivi adorare, e s'appoggia al mio braccio, ed anch'io mi prostro nel tempio di Rimmon, voglia l'Eterno perdonare a me, tuo servo, quand'io mi prostrerò così nel tempio di Rimmon!'
\par 19 Eliseo gli disse: 'Va' in pace!' Ed egli si partì da lui e fece un buon tratto di strada.
\par 20 Ma Ghehazi, servo d'Eliseo, uomo di Dio, disse fra sé: 'Ecco, il mio signore è stato troppo generoso con Naaman, con questo Siro, non accettando dalla sua mano quel ch'egli avea portato; com'è vero che l'Eterno vive, io gli voglio correr dietro, e voglio aver da lui qualcosa'.
\par 21 Così Ghehazi corse dietro a Naaman; e quando Naaman vide che gli correva dietro, saltò giù dal carro per andargli incontro, e gli disse: 'Va egli tutto bene?'
\par 22 Quegli rispose: 'Tutto bene. Il mio signore mi manda a dirti: - Ecco, proprio ora mi sono arrivati dalla contrada montuosa d'Efraim due giovani de' discepoli dei profeti; ti prego, da' loro un talento d'argento e due mute di vestiti'. -
\par 23 Naaman disse: 'Piacciati accettare due talenti!' E gli fece premura; chiuse due talenti d'argento in due sacchi con due mute di vesti, e li caricò addosso a due de' suoi servi, che li portarono davanti a Ghehazi.
\par 24 E, giunto che fu alla collina, prese i sacchi dalle loro mani, li ripose nella casa, e licenziò quegli uomini, che se ne andarono.
\par 25 Poi andò a presentarsi davanti al suo signore. Eliseo gli disse: 'Donde vieni, Ghehazi?' Questi rispose: 'Il tuo servo non è andato in verun luogo'.
\par 26 Ma Eliseo gli disse: 'Il mio spirito non era egli là presente, quando quell'uomo si voltò e scese dal suo carro per venirti incontro? È forse questo il momento di prender danaro, di prender vesti, e uliveti e vigne, pecore e buoi, servi e serve?
\par 27 La lebbra di Naaman s'attaccherà perciò a te e alla tua progenie in perpetuo'. E Ghehazi uscì dalla presenza di Eliseo, tutto lebbroso, bianco come la neve.

\chapter{6}

\par 1 I discepoli dei profeti dissero ad Eliseo: 'Ecco, il luogo dove noi ci raduniamo in tua presenza è troppo angusto per noi.
\par 2 Lasciaci andare fino al Giordano; ciascun di noi prenderà là una trave, e ci farem quivi un luogo dove ci possiam radunare'. Eliseo rispose: 'Andate'.
\par 3 E un di loro disse: 'Abbi, ti prego, la compiacenza di venire anche tu coi tuoi servi'. Egli rispose: 'Verrò'.
\par 4 E così andò con loro. Giunti che furono al Giordano, si misero a tagliar legna.
\par 5 E come l'un d'essi abbatteva una trave, il ferro della scure gli cadde nell'acqua; ond'egli cominciò a gridare: - 'Ah, signor mio! e l'avevo presa ad imprestito!' -
\par 6 L'uomo di Dio disse: 'Dov'è caduta?' E colui gli additò il luogo. Allora Eliseo tagliò un pezzo di legno, lo gettò in quel medesimo luogo, fece venire a galla il ferro, e disse: 'Prendilo'.
\par 7 E quegli stese la mano e lo prese.
\par 8 Ora il re di Siria faceva guerra contro Israele; e in un consiglio che tenne coi suoi servi, disse: 'Io porrò il mio campo nel tale e tal luogo'.
\par 9 E l'uomo di Dio mandò a dire al re d'Israele: 'Guardati dal trascurare quel tal luogo, perché vi stan calando i Sirî'.
\par 10 E il re d'Israele mandò gente verso il luogo che l'uomo di Dio gli aveva detto, e circa il quale l'avea premunito; e quivi si mise in guardia. Il fatto avvenne non una né due ma più volte.
\par 11 Questa cosa turbò molto il cuore del re di Siria, che chiamò i suoi servi, e disse loro: 'Non mi farete dunque sapere chi dei nostri è per il re d'Israele?'
\par 12 Uno dei suoi servi rispose: 'Nessuno, o re, mio signore! ma Eliseo, il profeta ch'è in Israele, fa sapere al re d'Israele perfino le parole che tu dici nella camera ove dormi'.
\par 13 E il re disse: 'Andate, vedete dov'è, ed io lo manderò a pigliare'. Gli fu riferito ch'era a Dothan.
\par 14 Ed il re vi mandò cavalli, carri e gran numero di soldati, i quali giunsero di nottetempo, e circondarono la città.
\par 15 Il servitore dell'uomo di Dio, alzatosi di buon mattino, uscì fuori, ed ecco che un gran numero di soldati con cavalli e carri accerchiava la città. E il servo disse all'uomo di Dio: 'Ah, signor mio, come faremo?'
\par 16 Quegli rispose: 'Non temere, perché quelli che son con noi son più numerosi di quelli che son con loro'.
\par 17 Ed Eliseo pregò e disse: 'O Eterno, ti prego, aprigli gli occhi, affinché vegga!' E l'Eterno aperse gli occhi del servo, che vide a un tratto il monte pieno di cavalli e di carri di fuoco intorno ad Eliseo.
\par 18 E come i Sirî scendevano verso Eliseo, questi pregò l'Eterno e disse: 'Ti prego, acceca cotesta gente!' E l'Eterno l'accecò, secondo la parola d'Eliseo.
\par 19 Allora Eliseo disse loro: 'Non è questa la strada, e non è questa la città; venitemi appresso ed io vi condurrò all'uomo che voi cercate'. E li menò a Samaria.
\par 20 Quando furono entrati in Samaria, Eliseo disse: 'O Eterno, apri loro gli occhi, affinché veggano'. L'Eterno aperse loro gli occhi, e a un tratto videro che si trovavano nel mezzo di Samaria.
\par 21 E il re d'Israele, come li ebbe veduti, disse ad Eliseo: 'Padre mio, li debbo colpire? li debbo colpire?'
\par 22 Eliseo rispose: 'Non li colpire! Colpisci tu forse quelli che fai prigionieri con la tua spada e col tuo arco? Metti loro davanti del pane e dell'acqua, affinché mangino e bevano, e se ne tornino al loro signore'.
\par 23 Il re d'Israele preparò loro gran copia di cibi; e quand'ebbero mangiato e bevuto, li licenziò, e quelli tornarono al loro signore; e le bande dei Sirî non vennero più a fare incursioni sul territorio d'Israele.
\par 24 Or dopo queste cose avvenne che Ben-Hadad, re di Siria, radunato tutto il suo esercito, salì contro Samaria, e la cinse d'assedio.
\par 25 E vi fu una gran carestia in Samaria; e i Sirî la strinsero tanto dappresso che una testa d'asino vi si vendeva ottanta sicli d'argento, e il quarto d'un kab di sterco di colombi, cinque sicli d'argento.
\par 26 Or come il re d'Israele passava sulle mura, una donna gli gridò: 'Aiutami, o re, mio signore!'
\par 27 Il re le disse: 'Se non t'aiuta l'Eterno, come posso aiutarti io? Con quel che dà l'aia o con quel che dà lo strettoio?'
\par 28 Poi il re aggiunse: 'Che hai?' Ella rispose: 'Questa donna mi disse: - Da' qua il tuo figliuolo, che lo mangiamo oggi; domani mangeremo il mio. -
\par 29 Così cocemmo il mio figliuolo, e lo mangiammo. Il giorno seguente io le dissi: - Da' qua il tuo figliuolo, che lo mangiamo. - Ma essa ha nascosto il suo figliuolo'.
\par 30 Quando il re ebbe udite le parole della donna, si stracciò le vesti; e come passava sulle mura, il popolo vide ch'egli portava, sotto, un cilicio sulla carne.
\par 31 E il re disse: 'Mi tratti Iddio con tutto il suo rigore, se oggi la testa di Eliseo, figliuolo di Shafat, rimane ancora sulle sue spalle!'
\par 32 Or Eliseo se ne stava sedendo in casa sua, e con lui stavano a sedere gli anziani. Il re mandò innanzi un uomo; ma prima che questo messo giungesse, Eliseo disse agli anziani: 'Lo vedete voi che questo figliuol d'un assassino manda qualcuno a tagliarmi la testa? Badate bene; quand'arriva il messo, chiudete la porta, e tenetegliela ben chiusa in faccia. Non si sente già dietro a lui il rumore de' passi del suo signore?'
\par 33 Egli parlava ancora con essi, quand'ecco scendere verso di lui il messo. E il re disse: 'Ecco, questo male vien dall'Eterno; che ho io più da sperar dall'Eterno?'

\chapter{7}

\par 1 Allora Eliseo disse: 'Ascoltate la parola dell'Eterno! Così dice l'Eterno: - Domani, a quest'ora, alla porta di Samaria, la misura di fior di farina si avrà per un siclo, e le due misure d'orzo si avranno per un siclo'.
\par 2 Ma il capitano sul cui braccio il re s'appoggiava, rispose all'uomo di Dio: 'Ecco, anche se l'Eterno facesse delle finestre in cielo, potrebbe mai avvenire una cosa siffatta?' Eliseo rispose: 'Ebbene, lo vedrai con gli occhi tuoi, ma non ne mangerai'.
\par 3 Or v'erano quattro lebbrosi presso all'entrata della porta, i quali dissero tra di loro: 'Perché vogliam noi restar qui finché moriamo?
\par 4 Se diciamo: - Entriamo in città - in città c'è la fame, e noi vi morremo; se restiamo qui, morremo lo stesso. Or dunque venite, andiamoci a buttare nel campo dei Sirî; se ci lascian vivere, vivremo; se ci danno la morte, morremo'.
\par 5 E, sull'imbrunire, si mossero per andare al campo dei Sirî; e come furon giunti all'estremità del campo dei Sirî, ecco che non v'era alcuno.
\par 6 Il Signore avea fatto udire nel campo dei Sirî un rumor di carri, un rumor di cavalli, un rumor di grande esercito, sì che i Sirî avean detto fra di loro: 'Ecco, il re d'Israele ha assoldato contro di noi i re degli Hittei e i re degli Egiziani, perché vengano ad assalirci'.
\par 7 E s'eran levati, ed eran fuggiti sull'imbrunire, abbandonando le loro tende, i loro cavalli, i loro asini, e il campo così com'era; eran fuggiti per salvarsi la vita.
\par 8 Que' lebbrosi, giunti che furono all'estremità del campo, entrarono in una tenda, mangiarono, bevvero, e portaron via argento, oro, vesti, e andarono a nascondere ogni cosa. Poi tornarono, entrarono in un'altra tenda, e anche di là portaron via roba, che andarono a nascondere.
\par 9 Ma poi dissero fra di loro: 'Noi non facciamo bene; questo è giorno di buone novelle, e noi ci tacciamo! Se aspettiamo finché si faccia giorno, sarem tenuti per colpevoli. Or dunque venite, andiamo ad informare la casa del re'.
\par 10 Così partirono, chiamarono i guardiani della porta di città, e li informarono della cosa, dicendo: 'Siamo andati al campo dei Sirî, ed ecco che non v'è alcuno, né vi s'ode voce d'uomo; non vi son che i cavalli attaccati, gli asini attaccati, e le tende intatte'.
\par 11 Allora i guardiani chiamarono, e fecero sapere la cosa alla gente del re dentro il palazzo.
\par 12 E il re si levò nella notte, e disse ai suoi servi: 'Vi voglio dire io quel che ci hanno fatto i Sirî. Sanno che patiamo la fame; sono quindi usciti dal campo a nascondersi per la campagna, dicendo: - Come usciranno dalla città, li prenderemo vivi, ed entreremo nella città'.
\par 13 Uno de' suoi servi gli rispose: 'Ti prego, si prendan cinque de' cavalli che rimangono ancora nella città - guardate! son come tutta la moltitudine d'Israele che va in consunzione! - e mandiamo a vedere di che si tratta'.
\par 14 Presero dunque due carri coi loro cavalli, e il re mandò degli uomini in traccia dell'esercito dei Sirî, dicendo: 'Andate e vedete'.
\par 15 E quelli andarono in traccia de' Sirî, fino al Giordano; ed ecco, tutta la strada era piena di vesti e di oggetti, che i Sirî avean gettati via nella loro fuga precipitosa. E i messi tornarono e riferiron tutto al re.
\par 16 Allora il popolo uscì fuori, e saccheggiò il campo dei Sirî; e una misura di fior di farina si ebbe per un siclo, e due misure d'orzo per un siclo secondo la parola dell'Eterno.
\par 17 Il re aveva affidato la guardia della porta al capitano sul cui braccio s'appoggiava; ma questo capitano fu calpestato dalla folla presso la porta e morì, come avea detto l'uomo di Dio, quando avea parlato al re ch'era sceso a trovarlo.
\par 18 Difatti, quando l'uomo di Dio avea parlato al re dicendo: 'Domani, a quest'ora, alla porta di Samaria, due misure d'orzo s'avranno per un siclo e una misura di fior di farina per un siclo',
\par 19 quel capitano avea risposto all'uomo di Dio e gli avea detto: 'Ecco, anche se l'Eterno facesse delle finestre in cielo, potrebbe mai avvenire una cosa siffatta?' Ed Eliseo gli avea detto: 'Ebbene, lo vedrai con gli occhi tuoi, ma non ne mangerai'.
\par 20 E così gli avvenne: fu calpestato dalla folla presso la porta, e morì.

\chapter{8}

\par 1 Or Eliseo avea detto alla donna di cui avea risuscitato il figliuolo: 'Lèvati, vattene, tu con la tua famiglia, a soggiornare all'estero, dove potrai; perché l'Eterno ha chiamata la carestia, e difatti essa verrà nel paese per sette anni'.
\par 2 E la donna si levò, e fece come le avea detto l'uomo di Dio; se ne andò con la sua famiglia, e soggiornò per sette anni nel paese de' Filistei.
\par 3 Finiti i sette anni, quella donna tornò dal paese de' Filistei, e andò a ricorrere al re per riavere la sua casa e le sue terre.
\par 4 Or il re discorreva con Ghehazi, servo dell'uomo di Dio, e gli diceva: 'Ti prego raccontami tutte le cose grandi che ha fatte Eliseo'.
\par 5 E mentre appunto Ghehazi raccontava al re come Eliseo aveva risuscitato il morto, ecco che la donna, di cui era stato risuscitato il figliuolo, venne a ricorrere al re per riavere la sua casa e le sue terre. E Ghehazi disse: 'O re, mio signore, questa è quella donna, e questo è il suo figliuolo, che Eliseo ha risuscitato'.
\par 6 Il re interrogò la donna che gli raccontò tutto; e il re le dette un eunuco, al quale disse: 'Falle restituire tutto quello ch'è suo, e tutte le rendite delle terre, dal giorno in cui ella lasciò il paese, fino ad ora'.
\par 7 Or Eliseo si recò a Damasco; Ben-Hadad, re di Siria, era ammalato, e gli fu riferito che l'uomo di Dio era giunto colà.
\par 8 Allora il re disse ad Hazael: 'Prendi teco un regalo, va' incontro all'uomo di Dio, e consulta per mezzo di lui l'Eterno, per sapere se io guarirò da questa malattia'.
\par 9 Hazael dunque andò incontro ad Eliseo, portando seco in regalo tutto quello che v'era di meglio in Damasco: un carico di quaranta cammelli. Come fu giunto, si presentò ad Eliseo, e gli disse: 'Il tuo figliuolo Ben-Hadad, re di Siria, mi ha mandato a te per dirti: Guarirò io da questa malattia?'
\par 10 Eliseo gli rispose: 'Vagli a dire: - Guarirai di certo. - Ma l'Eterno m'ha fatto vedere che di sicuro morrà'.
\par 11 E l'uomo di Dio posò lo sguardo sopra Hazael, e lo fissò così a lungo, da farlo arrossire, poi si mise a piangere.
\par 12 Hazael disse: 'Perché piange il mio signore?' Eliseo rispose: 'Perché so il male che tu farai ai figliuoli d'Israele; tu darai alle fiamme le loro fortezze, ucciderai la loro gioventù con la spada, schiaccerai i loro bambini, e sventrerai le loro donne incinte'.
\par 13 Hazael disse: 'Ma che cos'è mai il tuo servo, questo cane, per fare delle cose sì grandi?' Eliseo rispose: 'L'Eterno m'ha fatto vedere che tu sarai re di Siria'.
\par 14 Hazael si partì da Eliseo e tornò dal suo signore, che gli chiese: 'Che t'ha detto Eliseo?' Quegli rispose: 'Mi ha detto che guarirai di certo'.
\par 15 Il giorno dopo, Hazael prese una coperta, la tuffò nell'acqua, e la distese sulla faccia di Ben-Hadad, che morì. E Hazael regnò in luogo suo.
\par 16 Or l'anno quinto di Joram, figliuolo di Achab, re d'Israele, Jehoram, figliuolo di Giosafat re di Giuda, cominciò a regnare su Giuda.
\par 17 Avea trentadue anni quando cominciò a regnare, e regnò otto anni in Gerusalemme.
\par 18 E camminò per la via dei re d'Israele, come avea fatto la casa di Achab; poiché avea per moglie una figliuola di Achab; e fece ciò ch'è male agli occhi dell'Eterno.
\par 19 Nondimeno l'Eterno non volle distrugger Giuda, per amor di Davide suo servo, conformemente alla promessa fattagli di lasciar sempre una lampada a lui ed ai suoi figliuoli.
\par 20 Ai tempi suoi, Edom si ribellò, sottraendosi al giogo di Giuda e si dette un re.
\par 21 Allora Joram passò a Tsair con tutti i suoi carri; e una notte si levò, e sconfisse gli Edomiti che lo aveano accerchiato e i capitani dei carri; e la gente di Joram poté fuggire alle proprie case.
\par 22 Così Edom si è ribellato e si è sottratto al giogo di Giuda fino al dì d'oggi. In quel medesimo tempo, anche Libna si ribellò.
\par 23 Il rimanente delle azioni di Joram e tutto quello che fece, si trova scritto nel libro delle Cronache dei re di Giuda.
\par 24 E Joram si addormentò coi suoi padri, e coi suoi padri fu sepolto nella città di Davide. E Achazia, suo figliuolo, regnò in luogo suo.
\par 25 L'anno dodicesimo di Joram, figliuolo di Achab, re d'Israele, Achazia, figliuolo di Jehoram re di Giuda, cominciò a regnare.
\par 26 Aveva ventidue anni quando cominciò a regnare, e regnò un anno in Gerusalemme. Sua madre si chiamava Athalia, nipote di Omri, re d'Israele.
\par 27 Egli camminò per la via della casa di Achab, e fece ciò ch'è male agli occhi dell'Eterno, come la casa di Achab, perché era imparentato con la casa di Achab.
\par 28 E andò con Joram, figliuolo di Achab, a combattere contro Hazael, re di Siria, a Ramoth di Galaad; e i Sirî ferirono Joram;
\par 29 e il re Joram tornò a Izreel per farsi curare delle ferite che avea ricevute dai Sirî a Ramah, quando combatteva contro Hazael, re di Siria. Ed Achazia, figliuolo di Jehoram re di Giuda, scese ad Izreel a vedere Joram, figliuolo di Achab, perché questi era ammalato.

\chapter{9}

\par 1 Allora il profeta Eliseo chiamò uno de' discepoli dei profeti, e gli disse: 'Cingiti i fianchi, prendi teco quest'ampolla d'olio, e va' a Ramoth di Galaad.
\par 2 Quando vi sarai arrivato, cerca di vedere Jehu, figliuolo di Jehoshafat, figliuolo di Nimsci; entra, fallo alzare di mezzo ai suoi fratelli, e menalo in una camera appartata.
\par 3 Poi prendi l'ampolla d'olio, versagliela sul capo, e digli: Così dice l'Eterno: - Io ti ungo re d'Israele. - Poi apri la porta, e fuggi senza indugiare'.
\par 4 Così quel giovine, il servo del profeta, partì per Ramoth di Galaad.
\par 5 E, come vi fu giunto, ecco che i capitani dell'esercito stavan seduti assieme; e disse: 'Capitano, ho da dirti una parola'. Jehu chiese: 'A chi di tutti noi?' Quegli rispose: 'A te, capitano'.
\par 6 Jehu si alzò, ed entrò in casa; e il giovane gli versò l'olio sul capo, dicendogli: 'Così dice l'Eterno, l'Iddio d'Israele: - Io ti ungo re del popolo dell'Eterno, re d'Israele. -
\par 7 E tu colpirai la casa di Achab, tuo signore, ed io farò vendetta del sangue de' profeti miei servi, e del sangue di tutti i servi dell'Eterno, sopra Izebel;
\par 8 e tutta la casa di Achab perirà, e io sterminerò dalla casa di Achab fino all'ultimo uomo, tanto chi è schiavo quanto chi è libero in Israele.
\par 9 E ridurrò la casa di Achab come la casa di Geroboamo, figliuolo di Nebat, e come la casa di Baasa, figliuolo di Ahija.
\par 10 E i cani divoreranno Izebel nel campo d'Izreel, e non vi sarà chi le dia sepoltura'. Poi il giovine aprì la porta, e fuggì.
\par 11 Quando Jehu uscì per raggiungere i servi del suo signore, gli dissero: 'Va tutto bene? Perché quel pazzo è egli venuto da te?' Egli rispose loro: 'Voi conoscete l'uomo e i suoi discorsi!'
\par 12 Ma quelli dissero: 'Non è vero! Orsù, diccelo!' Jehu rispose: 'Ei m'ha parlato così e così, e m'ha detto: - Così dice l'Eterno: Io t'ungo re d'Israele'.
\par 13 Allora ognun d'essi s'affrettò a togliersi il proprio mantello, e a stenderlo sotto Jehu su per i nudi gradini; poi suonarono la tromba, e dissero: 'Jehu è re!'
\par 14 E Jehu, figliuolo di Jehoshafat, figliuolo di Nimsci, fece una congiura contro Joram. - Or Joram, con tutto Israele, stava difendendo Ramoth di Galaad contro Hazael, re di Siria;
\par 15 ma il re Joram era tornato a Izreel per farsi curare delle ferite che avea ricevuto dai Sirî, combattendo contro Hazael, re di Siria. - E Jehu disse: 'Se così vi piace, nessuno esca e fugga dalla città per andare a portar la nuova a Izreel'.
\par 16 Poi Jehu montò sopra un carro e partì per Izreel, perché quivi si trovava Joram allettato; e Achazia, re di Giuda, v'era sceso per visitare Joram.
\par 17 Or la sentinella che stava sulla torre di Izreel, scòrse la schiera numerosa di Jehu che veniva, e disse: 'Vedo una schiera numerosa!' Joram disse: 'Prendi un cavaliere, e mandalo incontro a coloro a dire: Recate pace?'
\par 18 Un uomo a cavallo andò dunque incontro a Jehu, e gli disse: 'Così dice il re: - Recate pace?' - Jehu rispose: 'Che importa a te della pace? Passa dietro a me'. E la sentinella fece il suo rapporto, dicendo: 'Il messo è giunto fino a loro, ma non torna indietro'.
\par 19 Allora Joram mandò un secondo cavaliere che, giunto da coloro, disse: 'Così dice il re: - Recate pace?' Jehu rispose: 'Che importa a te della pace? Passa dietro a me'.
\par 20 E la sentinella fece il suo rapporto, dicendo: 'Il messo è giunto fino a loro, e non torna indietro. A vederlo guidare, si direbbe che è Jehu, figliuolo di Nimsci; perché va a precipizio'.
\par 21 Allora Joram disse: 'Allestite il carro!' E gli allestirono il carro. E Joram, re d'Israele, e Achazia, re di Giuda, uscirono ciascuno sul suo carro per andare incontro a Jehu, e lo trovarono nel campo di Naboth d'Izreel.
\par 22 E come Joram ebbe veduto Jehu, gli disse: 'Jehu rechi tu pace?' Jehu rispose: 'Che pace vi può egli essere finché duran le fornicazioni di Izebel, tua madre, e le tante sue stregonerie?'
\par 23 Allora Joram voltò indietro, e si diè alla fuga, dicendo ad Achazia: 'Siam traditi, Achazia!'
\par 24 Ma Jehu impugnò l'arco e colpì Joram fra le spalle, sì che la freccia gli uscì pel cuore, ed egli stramazzò nel suo carro.
\par 25 Poi Jehu disse a Bidkar, suo aiutante: 'Piglialo, e buttalo nel campo di Naboth d'Izreel; poiché, ricordalo, quando io e tu cavalcavamo assieme al seguito di Achab, suo padre, l'Eterno pronunciò contro di lui questa sentenza: -
\par 26 Com'è vero che ieri vidi il sangue di Naboth e il sangue dei suoi figliuoli, dice l'Eterno, io ti renderò il contraccambio qui in questo campo, dice l'Eterno! - Piglialo dunque e buttalo in cotesto campo, secondo la parola dell'Eterno'.
\par 27 Achazia, re di Giuda, veduto questo, prese la fuga per la strada della casa del giardino; ma Jehu gli tenne dietro, e disse: 'Tirate anche a lui sul carro!' E gli tirarono alla salita di Gur, ch'è vicino a Ibleam. E Achazia fuggì a Meghiddo, e quivi morì.
\par 28 I suoi servi lo trasportarono sopra un carro a Gerusalemme, e lo seppellirono nel suo sepolcro, coi suoi padri, nella città di Davide. -
\par 29 Achazia avea cominciato a regnare sopra Giuda l'undecimo anno di Joram, figliuolo di Achab.
\par 30 Poi Jehu giunse ad Izreel. Izebel, che lo seppe, si diede il belletto agli occhi, si acconciò il capo, e si mise alla finestra a guardare.
\par 31 E come Jehu entrava per la porta di città, ella gli disse: 'Rechi pace, novello Zimri, uccisore del tuo signore?'
\par 32 Jehu alzò gli occhi verso la finestra, e disse: 'Chi è per me? chi?' E due o tre eunuchi, affacciatisi, volsero lo sguardo verso di lui.
\par 33 Egli disse: 'Buttatela giù!' Quelli la buttarono; e il suo sangue schizzò contro il muro e contro i cavalli. Jehu le passò sopra, calpestandola;
\par 34 poi entrò, mangiò e bevve, quindi disse: 'Andate a vedere di quella maledetta donna e sotterratela, giacché è figliuola di re'.
\par 35 Andaron dunque per sotterrarla, ma non trovarono di lei altro che il cranio, i piedi e le palme delle mani.
\par 36 E tornarono a riferir la cosa a Jehu, il quale disse: 'Questa è la parola dell'Eterno pronunziata per mezzo del suo servo Elia il Tishbita, quando disse: I cani divoreranno la carne di Izebel nel campo d'Izreel;
\par 37 e il cadavere di Izebel sarà, nel campo d'Izreel, come letame sulla superficie del suolo, in guisa che non si potrà dire: - Questa è Izebel'.

\chapter{10}

\par 1 Or v'erano a Samaria settanta figliuoli d'Achab. Jehu scrisse delle lettere, e le mandò a Samaria ai capi della città, agli anziani, e agli educatori dei figliuoli d'Achab; in esse diceva:
\par 2 'Subito che avrete ricevuto questa lettera, giacché avete con voi i figliuoli del vostro signore e avete a vostra disposizione carri e cavalli, nonché una città fortificata e delle armi,
\par 3 scegliete il migliore e il più adatto tra i figliuoli del vostro signore, mettetelo sul trono di suo padre, e combattete per la casa del vostro signore'.
\par 4 Ma quelli ebbero gran paura, e dissero: 'Ecco, due re non gli han potuto resistere; come potremo resistergli noi?'
\par 5 E il prefetto del palazzo, il governatore della città, gli anziani e gli educatori dei figliuoli di Achab mandarono a dire a Jehu: 'Noi siamo tuoi servi, e faremo tutto quello che ci ordinerai; non eleggeremo alcuno come re; fa' tu quel che ti piace'.
\par 6 Allora Jehu scrisse loro una seconda lettera, nella quale diceva: 'Se voi siete per me e volete ubbidire alla mia voce, prendete le teste di quegli uomini, de' figliuoli del vostro signore, e venite da me, domani a quest'ora, a Izreel'. - Or i figliuoli del re, in numero di settanta, stavano dai magnati della città, che li educavano. -
\par 7 E come questi ebbero ricevuta la lettera, presero i figliuoli del re, li scannarono tutti e settanta; poi misero le loro teste in ceste, e le mandarono a Jehu a Izreel.
\par 8 E un messo venne a Jehu a recargli la notizia, dicendo: 'Hanno portato le teste dei figliuoli del re'. Jehu rispose: 'Mettetele in due mucchi all'entrata della porta, fino a domattina'.
\par 9 La mattina dopo, egli uscì fuori; e fermatosi, disse a tutto il popolo: 'Voi siete giusti; ecco, io congiurai contro il mio signore, e l'uccisi; ma chi ha uccisi tutti questi?
\par 10 Riconoscete dunque che non cade a terra una parola di quelle che l'Eterno pronunziò contro la casa di Achab; l'Eterno ha fatto quello che predisse per mezzo del suo servo Elia'.
\par 11 E Jehu fece morire tutti quelli ch'erano rimasti della casa di Achab a Izreel, tutti i suoi grandi, i suoi amici e i suoi consiglieri, senza che ne scampasse uno.
\par 12 Poi si levò, e partì per andare a Samaria. Cammin facendo, giunto che fu alla casa di ritrovo dei pastori,
\par 13 Jehu s'imbatté nei fratelli di Achazia, re di Giuda, e disse: 'Chi siete voi?' Quelli risposero: 'Siamo i fratelli di Achazia, e scendiamo a salutare i figliuoli del re e i figliuoli della regina'.
\par 14 Jehu disse ai suoi: 'Pigliateli vivi!' e quelli li presero vivi e li scannarono presso la cisterna della casa di ritrovo. Erano quarantadue, e non ne scampò uno.
\par 15 Partitosi di là, trovò Jehonadab, figliuolo di Recab, che gli veniva incontro; lo salutò, e gli disse: 'Il tuo cuore è egli retto verso il mio, come il mio verso il tuo?' Jehonadab rispose: 'Lo è'. 'Se è così', disse Jehu, 'dammi la mano'. Jehonadab gli dette la mano; Jehu se lo fe' salire vicino sul carro, e gli disse:
\par 16 'Vieni meco, e vedrai il mio zelo per l'Eterno!' E lo menò via nel suo carro.
\par 17 E, giunto che fu a Samaria, Jehu colpì tutti quelli che rimanevano della casa di Achab a Samaria, finché l'ebbe distrutta, secondo la parola che l'Eterno avea pronunziata per mezzo di Elia.
\par 18 Poi Jehu radunò tutto il popolo, e gli parlò così: 'Achab ha servito un poco Baal; Jehu lo servirà di molto.
\par 19 Or convocate presso di me tutti i profeti di Baal, tutti i suoi servi, tutti i suoi sacerdoti; che non ne manchi uno! poiché voglio fare un gran sacrifizio a Baal; chi mancherà non vivrà'. - Ma Jehu faceva questo con astuzia, per distruggere gli adoratori di Baal. -
\par 20 E disse: 'Bandite una festa solenne in onore di Baal!' E la festa fu bandita.
\par 21 Jehu inviò dei messi per tutto Israele; e tutti gli adoratori di Baal vennero, e neppur uno vi fu che mancasse di venire; entrarono nel tempio di Baal, e il tempio di Baal fu ripieno da un capo all'altro.
\par 22 E Jehu disse a colui che avea in custodia le vestimenta: 'Metti fuori le vesti per tutti gli adoratori di Baal'. E quegli mise loro fuori le vesti.
\par 23 Allora Jehu, con Jehonadab, figliuolo di Recab, entrò nel tempio di Baal, e disse agli adoratori di Baal: 'Cercate bene, e guardate che non ci sia qui con voi alcun servo dell'Eterno, ma ci sian soltanto degli adoratori di Baal'.
\par 24 E quelli entrarono per offrir dei sacrifizi e degli olocausti. Or Jehu aveva appostati fuori del tempio ottanta uomini, ai quali avea detto: 'Colui che lascerà fuggire qualcuno degli uomini ch'io metto in poter vostro, pagherà con la sua vita la vita di quello'.
\par 25 E, come fu finita l'offerta dell'olocausto, Jehu disse ai soldati e ai capitani: 'Entrate, uccideteli, e che non ne esca uno!' Ed essi li passarono a fil di spada; poi, soldati e capitani ne buttaron là i cadaveri, e penetrarono nell'edifizio del tempio di Baal;
\par 26 portaron fuori le statue del tempio di Baal, e le bruciarono;
\par 27 mandarono in frantumi la statua di Baal; e demolirono il tempio di Baal, e lo ridussero in un mondezzaio che sussiste anche oggi dì.
\par 28 Così Jehu estirpò Baal da Israele;
\par 29 nondimeno egli non si ritrasse dai peccati coi quali Geroboamo, figliuolo di Nebat, aveva fatto peccare Israele; non abbandonò cioè i vitelli d'oro ch'erano a Bethel e a Dan.
\par 30 E l'Eterno disse a Jehu: 'Perché tu hai eseguito puntualmente ciò ch'è giusto agli occhi miei, e hai fatto alla casa di Achab tutto quello che mi stava nel cuore, i tuoi figliuoli sederanno sul trono d'Israele fino alla quarta generazione'.
\par 31 Ma Jehu non si fe' premura di seguir con tutto il cuore la legge dell'Eterno, dell'Iddio d'Israele; non si dipartì dai peccati coi quali Geroboamo avea fatto peccare Israele.
\par 32 In quel tempo, l'Eterno cominciò a diminuire il territorio d'Israele; Hazael difatti sconfisse gl'Israeliti su tutta la loro frontiera:
\par 33 dal Giordano, verso oriente, soggiogò tutto il paese di Galaad, i Gaditi, i Rubeniti e i Manassiti, fino ad Aroer ch'è presso la valle dell'Arnon, vale a dire tutto il paese di Galaad e di Bashan.
\par 34 Il rimanente delle azioni di Jehu, tutto quello che fece e tutte le sue prodezze, si trova scritto nel libro delle Cronache dei re d'Israele.
\par 35 E Jehu s'addormentò coi suoi padri, e lo seppellirono a Samaria. E Jehoachaz, suo figliuolo, regnò in luogo suo.
\par 36 E il tempo che Jehu regnò sopra Israele a Samaria fu di ventott'anni.

\chapter{11}

\par 1 Or quando Athalia, madre di Achazia, vide che il suo figliuolo era morto, si levò e distrusse tutta la stirpe reale.
\par 2 Ma Jehosceba, figliuola del re Joram, sorella di Achazia, prese Joas, figliuolo di Achazia, lo trafugò di mezzo ai figliuoli del re ch'eran messi a morte, e lo pose con la sua balia nella camera dei letti; così fu nascosto alle ricerche d'Athalia, e non fu messo a morte.
\par 3 E rimase nascosto con Jehosceba per sei anni nella casa dell'Eterno; intanto Athalia regnava sul paese.
\par 4 Il settimo anno, Jehoiada mandò a chiamare i capi-centurie delle guardie del corpo e dei soldati, e li fece venire a sé nella casa dell'Eterno; fermò un patto con essi, fece loro pestar giuramento nella casa dell'Eterno, e mostrò loro il figliuolo del re.
\par 5 Poi diede loro i suoi ordini, dicendo: 'Ecco quello che voi farete: un terzo di quelli tra voi che entrano in servizio il giorno del sabato, starà di guardia alla casa del re;
\par 6 un altro terzo starà alla porta di Sur, e un altro terzo starà alla porta ch'è dietro alla caserma dei soldati. E farete la guardia alla casa, impedendo a tutti l'ingresso.
\par 7 E le altre due parti di voi, tutti quelli cioè che escon di servizio il giorno del sabato, staranno di guardia alla casa dell'Eterno, intorno al re.
\par 8 E circonderete bene il re, ognuno con le armi alla mano; e chiunque cercherà di penetrare nelle vostre file, sia messo a morte; e voi starete col re, quando uscirà e quando entrerà'.
\par 9 I capi-centurie eseguirono tutti gli ordini dati dal sacerdote Jehoiada; ognun d'essi prese i suoi uomini: quelli che entravano in servizio il giorno del sabato, e quelli che uscivan di servizio il giorno del sabato; e si recarono dal sacerdote Jehoiada.
\par 10 E il sacerdote diede ai capi-centurie le lance e gli scudi che avevano appartenuto al re Davide, e che erano nella casa dell'Eterno.
\par 11 I soldati, con le armi alla mano, presero posto dall'angolo meridionale della casa, fino all'angolo settentrionale della casa, fra l'altare e l'edifizio, in modo da proteggere il re da tutte le parti.
\par 12 Allora il sacerdote menò fuori il figliuolo del re, e gli pose in testa il diadema, e gli consegnò la legge. E lo proclamarono re, lo unsero, e, battendo le mani, esclamarono: 'Viva il re!'
\par 13 Or Athalia udì il rumore dei soldati e del popolo, e andò verso il popolo nella casa dell'Eterno.
\par 14 Guardò, ed ecco che il re stava in piedi sul palco, secondo l'uso; i capitani e i trombettieri erano accanto al re; tutto il popolo del paese era in festa, e dava nelle trombe. Allora Athalia si stracciò le vesti, e gridò: 'Congiura! Congiura!'
\par 15 Ma il sacerdote Jehoiada diede i suoi ordini ai capi-centurie che comandavano l'esercito, e disse loro: 'Fatela uscire di tra le file; e chiunque la seguirà sia ucciso di spada!' Poiché il sacerdote avea detto: 'Non sia messa a morte nella casa dell'Eterno'.
\par 16 Così quelli le fecero largo, ed ella giunse alla casa del re per la strada della porta dei cavalli; e quivi fu uccisa.
\par 17 E Jehoiada fermò tra l'Eterno, il re ed il popolo il patto, per il quale Israele doveva essere il popolo dell'Eterno; e fermò pure il patto fra il re ed il popolo.
\par 18 E tutto il popolo del paese entrò nel tempio di Baal, e lo demolì; fece interamente in pezzi i suoi altari e le sue immagini, e uccise dinanzi agli altari Mattan, sacerdote di Baal. Poi il sacerdote Jehoiada pose delle guardie alla casa dell'Eterno.
\par 19 E prese i capi-centurie, le guardie del corpo, i soldati e tutto il popolo del paese; e fecero scendere il re dalla casa dell'Eterno, e giunsero alla casa del re per la strada della porta dei soldati. E Joas si assise sul trono dei re.
\par 20 E tutto il popolo del paese fu in festa, e la città rimase tranquilla, quando Athalia fu uccisa di spada, nella casa del re.
\par 21 Joas avea sette anni quando cominciò a regnare.

\chapter{12}

\par 1 L'anno settimo di Jehu, Joas cominciò a regnare, e regnò quarant'anni a Gerusalemme. Sua madre si chiamava Tsibia di Beer-Sceba.
\par 2 Joas fece ciò ch'è giusto agli occhi dell'Eterno per tutto il tempo in cui fu diretto dal sacerdote Jehoiada.
\par 3 Nondimeno, gli alti luoghi non scomparvero; il popolo continuava ad offrir sacrifizi e profumi sugli alti luoghi.
\par 4 Joas disse ai sacerdoti: 'Tutto il danaro consacrato che sarà recato alla Casa dell'Eterno vale a dire il danaro versato da ogni Israelita censito, il danaro che paga per il suo riscatto personale secondo la stima fatta dal sacerdote, tutto il danaro che a qualunque persona venga in cuore di portare alla casa dell'Eterno,
\par 5 i sacerdoti lo ricevano, ognuno dalle mani dei suoi conoscenti, e se ne servano per fare i restauri alla casa, dovunque si troverà qualcosa da restaurare'.
\par 6 Ma fino al ventesimoterzo anno del re Joas i sacerdoti non aveano ancora eseguito i restauri alla casa.
\par 7 Allora il re Joas chiamò il sacerdote Jehoiada e gli altri sacerdoti, e disse loro: 'Perché non restaurate quel che c'è da restaurare nella casa? Da ora innanzi dunque non ricevete più danaro dalle mani dei vostri conoscenti, ma lasciatelo per i restauri della casa'.
\par 8 I sacerdoti acconsentirono a non ricever più danaro dalle mani del popolo, e a non aver più l'incarico dei restauri della casa.
\par 9 E il sacerdote Jehoiada prese una cassa, le fece un buco nel coperchio, e la collocò presso all'altare, a destra, entrando nella casa dell'Eterno; e i sacerdoti che custodivan la soglia vi mettevan tutto il danaro ch'era portato alla casa dell'Eterno.
\par 10 E quando vedevano che v'era molto danaro nella cassa, il segretario del re e il sommo sacerdote salivano a serrare in borse e contare il danaro che si trovava nella casa dell'Eterno.
\par 11 Poi rimettevano il danaro così pesato nelle mani dei direttori preposti ai lavori della casa dell'Eterno, i quali ne pagavano i legnaiuoli e i costruttori che lavoravano alla casa dell'Eterno,
\par 12 i muratori e gli scalpellini, compravano i legnami e le pietre da tagliare occorrenti per restaurare la casa dell'Eterno, e provvedevano a tutte le spese relative ai restauri della casa.
\par 13 Ma col danaro ch'era portato alla casa dell'Eterno non si fecero, per la casa dell'Eterno, né coppe d'argento, né smoccolatoi, né bacini, né trombe, né alcun altro utensile d'oro o d'argento;
\par 14 il danaro si dava a quelli che facevano l'opera, ed essi lo impiegavano a restaurare la casa dell'Eterno.
\par 15 E non si faceva render conto a quelli nelle cui mani si rimetteva il danaro per pagare chi eseguiva il lavoro; perché agivano con fedeltà.
\par 16 Il danaro dei sacrifizi di riparazione e quello dei sacrifizi per il peccato non si portava nella casa dell'Eterno; era per i sacerdoti.
\par 17 In quel tempo Hazael, re di Siria, salì a combattere contro Gath, e la prese; poi si dispose a salire contro Gerusalemme.
\par 18 Allora Joas, re di Giuda, prese tutte le cose sacre che i suoi padri Giosafat, Jehoram e Achazia, re di Giuda, aveano consacrato, quelle che avea consacrate egli stesso, e tutto l'oro che si trovava nei tesori della casa dell'Eterno e della casa del re, e mandò ogni cosa ad Hazael, re di Siria, il quale si ritirò da Gerusalemme.
\par 19 Il rimanente delle azioni di Joas e tutto quello che fece, si trova scritto nel libro delle Cronache dei re di Giuda.
\par 20 I servi di Joas si sollevarono, fecero una congiura, e lo colpirono nella casa di Millo, sulla discesa di Silla.
\par 21 Jozacar, figliuolo di Shimeath, e Jehozabad, figliuolo di Shomer, suoi servi, lo colpirono, ed egli morì e fu sepolto coi suoi padri nella città di Davide; e Amatsia, suo figliuolo, regnò in luogo suo.

\chapter{13}

\par 1 L'anno ventesimoterzo di Joas, figliuolo di Achazia, re di Giuda, Joachaz, figliuolo di Jehu, cominciò a regnare sopra Israele a Samaria; e regnò diciassette anni.
\par 2 Egli fece ciò ch'è male agli occhi dell'Eterno, imitò i peccati coi quali Geroboamo, figliuolo di Nebat, aveva fatto peccare Israele, e non se ne ritrasse.
\par 3 E l'ira dell'Eterno si accese contro gl'Israeliti, ed ei li diede nelle mani di Hazael, re di Siria, e nelle mani di Ben-Hadad, figliuolo di Hazael, per tutto quel tempo.
\par 4 Ma Joachaz implorò l'Eterno, e l'Eterno lo esaudì, perché vide l'oppressione sotto la quale il re di Siria teneva Israele.
\par 5 - E l'Eterno diede un liberatore agl'Israeliti, i quali riuscirono a sottrarsi al potere dei Sirî, in guisa che i figliuoli d'Israele poteron dimorare nelle loro tende, come per l'addietro.
\par 6 Ma non si ritrassero dai peccati coi quali la casa di Geroboamo aveva fatto peccare Israele; e continuarono a camminare per quella via; perfino l'idolo di Astarte rimase in piè a Samaria.
\par 7 Di tutta la sua gente, a Joachaz, l'Eterno non avea lasciato che cinquanta cavalieri, dieci carri, e diecimila fanti; perché il re di Siria li avea distrutti, e li avea ridotti come la polvere che si calpesta.
\par 8 Il rimanente delle azioni di Joachaz, e tutto quello che fece, e tutte le sue prodezze, sono cose scritte nel libro delle Cronache dei re d'Israele.
\par 9 Joachaz si addormentò coi suoi padri, e fu sepolto a Samaria; e Joas, suo figliuolo, regnò in luogo suo.
\par 10 L'anno trentasettesimo di Joas, re di Giuda, Joas, figliuolo di Joachaz, cominciò a regnare sopra Israele a Samaria, e regnò sedici anni.
\par 11 Egli fece ciò ch'è male agli occhi dell'Eterno, e non si ritrasse da alcuno de' peccati coi quali Geroboamo, figliuolo di Nebat, avea fatto peccare Israele, ma batté anch'egli la stessa strada.
\par 12 Il rimanente delle azioni di Joas, e tutto quello che fece, e il valore col quale combatté contro Amatsia re di Giuda, sono cose scritte nel libro delle Cronache dei re d'Israele.
\par 13 Joas si addormentò coi suoi padri, e Geroboamo salì sul trono di lui. E Joas fu sepolto a Samaria coi re d'Israele.
\par 14 Or Eliseo cadde malato di quella malattia che lo dovea condurre alla morte; e Joas, re d'Israele, scese a trovarlo, pianse su lui, e disse: 'Padre mio, padre mio! Carro d'Israele e sua cavalleria!'
\par 15 Ed Eliseo gli disse: 'Prendi un arco e delle frecce'; e Joas prese un arco e delle frecce.
\par 16 Eliseo disse al re d'Israele: 'Impugna l'arco'; e quegli impugnò l'arco; ed Eliseo posò le sue mani sulle mani del re,
\par 17 poi gli disse: 'Apri la finestra a levante!' E Joas l'aprì. Allora Eliseo disse: 'Tira!' E quegli tirò. Ed Eliseo disse: 'Questa è una freccia di vittoria da parte dell'Eterno: la freccia della vittoria contro la Siria. Tu sconfiggerai i Sirî in Afek fino a sterminarli'.
\par 18 Poi disse: 'Prendi le frecce!' Joas le prese, ed Eliseo disse al re d'Israele: 'Percuoti il suolo'; ed egli lo percosse tre volte, indi si fermò.
\par 19 L'uomo di Dio si adirò contro di lui, e disse: 'Avresti dovuto percuoterlo cinque o sei volte; allora tu avresti sconfitto i Sirî fino a sterminarli; mentre adesso non li sconfiggerai che tre volte'. Eliseo morì, e fu sepolto.
\par 20 L'anno seguente delle bande di Moabiti fecero una scorreria nel paese;
\par 21 e avvenne, mentre certuni stavano seppellendo un morto, che scòrsero una di quelle bande, e gettarono il morto nel sepolcro di Eliseo. Il morto, non appena ebbe toccate le ossa di Eliseo, risuscitò, e si levò in piedi.
\par 22 Or Hazael, re di Siria, aveva oppresso gl'Israeliti durante tutta la vita di Joachaz;
\par 23 ma l'Eterno fece loro grazia, ne ebbe compassione e fu loro favorevole per amor del suo patto con Abrahamo, con Isacco e con Giacobbe; e non li volle distruggere; e, fino ad ora, non li ha rigettati dalla sua presenza.
\par 24 Hazael, re di Siria, morì e Ben-Hadad, suo figliuolo, regnò in luogo suo.
\par 25 E Joas, figliuolo di Joachaz, ritolse di mano a Ben-Hadad, figliuolo di Hazael, le città che Hazael avea prese in guerra a Joachaz suo padre. Tre volte Joas lo sconfisse, e ricuperò così le città d'Israele.

\chapter{14}

\par 1 L'anno secondo di Joas, figliuolo di Joachaz, re d'Israele, cominciò a regnare Amatsia, figliuolo di Joas, re di Giuda.
\par 2 Avea venticinque anni quando cominciò a regnare, e regnò ventinove anni a Gerusalemme. Sua madre si chiamava Jehoaddan, ed era di Gerusalemme.
\par 3 Egli fece ciò ch'è giusto agli occhi dell'Eterno; non però come Davide suo padre; fece interamente come avea fatto Joas suo padre.
\par 4 Nondimeno gli alti luoghi non furon soppressi; il popolo continuava ad offrir sacrifizi e profumi sugli alti luoghi.
\par 5 E, non appena il potere reale fu assicurato nelle sue mani, egli fece morire quei servi suoi che avean ucciso il re suo padre;
\par 6 ma non fece morire i figliuoli degli uccisori, secondo ch'è scritto nel libro della legge di Mosè, dove l'Eterno ha dato questo comandamento: 'I padri non saranno messi a morte a cagione dei figliuoli, né i figliuoli saranno messi a morte a cagione dei padri; ma ciascuno sarà messo a morte a cagione del proprio peccato'.
\par 7 Egli uccise diecimila Idumei nella valle del Sale; e in questa guerra prese Sela e le dette il nome di Joktheel, che ha conservato fino al dì d'oggi.
\par 8 Allora Amatsia inviò dei messi a Joas, figliuolo di Joachaz, figliuolo di Jehu, re d'Israele, per dirgli: 'Vieni, mettiamoci a faccia a faccia!'
\par 9 E Joas, re d'Israele, fece dire ad Amatsia, re di Giuda: 'Lo spino del Libano mandò a dire al cedro del Libano: - Da' la tua figliuola per moglie al mio figliuolo. - E le bestie selvagge del Libano passarono, e calpestarono lo spino.
\par 10 Tu hai messo in rotta gl'Idumei, e il tuo cuore t'ha reso orgoglioso. Godi della tua gloria, e stattene a casa tua. Perché impegnarti in una disgraziata impresa che menerebbe alla ruina te e Giuda con te?'
\par 11 Ma Amatsia non gli volle dar retta. Così Joas, re d'Israele, salì contro Amatsia; ed egli ed Amatsia, re di Giuda, si trovarono a faccia a faccia a Beth-Scemesh, che apparteneva a Giuda.
\par 12 Giuda rimase sconfitto da Israele; e que' di Giuda fuggirono ognuno alla sua tenda.
\par 13 E Joas, re d'Israele, fece prigioniero a Beth-Scemesh Amatsia, re di Giuda, figliuolo di Joas, figliuolo di Achazia. Poi venne a Gerusalemme, e fece una breccia di quattrocento cubiti nelle mura di Gerusalemme, dalla porta di Efraim alla porta dell'angolo.
\par 14 E prese tutto l'oro e l'argento e tutti i vasi che si trovavano nella casa dell'Eterno e nei tesori della casa del re; prese anche degli ostaggi, e se ne tornò a Samaria.
\par 15 Il rimanente delle azioni compiute da Joas, e il suo valore, e come combatté contro Amatsia re di Giuda, sono cose scritte nel libro delle Cronache dei re d'Israele.
\par 16 Joas si addormentò coi suoi padri e fu sepolto a Samaria coi re d'Israele; e Geroboamo, suo figliuolo, regnò in luogo suo.
\par 17 Amatsia, figliuolo di Joas, re di Giuda, visse ancora quindici anni dopo la morte di Joas, figliuolo di Joachaz, re d'Israele.
\par 18 Il rimanente delle azioni di Amatsia si trova scritto nel libro delle Cronache dei re di Giuda.
\par 19 Fu ordita contro di lui una congiura a Gerusalemme; ed egli fuggì a Lakis; ma lo fecero inseguire fino a Lakis, e quivi fu messo a morte.
\par 20 Di là fu trasportato sopra cavalli, e quindi sepolto a Gerusalemme coi suoi padri nella città di Davide.
\par 21 E tutto il popolo di Giuda prese Azaria, che aveva allora sedici anni, e lo fece re in luogo di Amatsia suo padre.
\par 22 Egli riedificò Elath, e la riconquistò a Giuda, dopo che il re si fu addormentato coi suoi padri.
\par 23 L'anno quindicesimo di Amatsia, figliuolo di Joas, re di Giuda, cominciò a regnare a Samaria Geroboamo, figliuolo di Joas, re d'Israele; e regnò quarantun anni.
\par 24 Egli fece quello ch'è male agli occhi dell'Eterno; non si ritrasse da alcuno dei peccati coi quali Geroboamo, figliuolo di Nebat, avea fatto peccare Israele.
\par 25 Egli ristabilì i confini d'Israele dall'ingresso di Hamath al mare della pianura, secondo la parola che l'Eterno, l'Iddio d'Israele, avea pronunziata per mezzo del suo servitore il profeta Giona, figliuolo di Amittai, che era di Gath-Hefer.
\par 26 Poiché l'Eterno vide che l'afflizione d'Israele era amarissima, che schiavi e liberi eran ridotti all'estremo, e che non c'era più alcuno che soccorresse Israele.
\par 27 L'Eterno non avea parlato ancora di cancellare il nome d'Israele di disotto al cielo; quindi li salvò, per mano di Geroboamo, figliuolo di Joas.
\par 28 Il rimanente delle azioni di Geroboamo, e tutto quello che fece, e il suo valore in guerra, e come riconquistò a Israele Damasco e Hamath che aveano appartenuto a Giuda, si trova scritto nel libro delle Cronache dei re d'Israele.
\par 29 Geroboamo si addormentò coi suoi padri, i re d'Israele; e Zaccaria, suo figliuolo, regnò in luogo suo.

\chapter{15}

\par 1 L'anno ventisettesimo di Geroboamo, re d'Israele, cominciò a regnare Azaria, figliuolo di Amatsia, re di Giuda.
\par 2 Avea sedici anni quando cominciò a regnare, e regnò cinquantadue anni a Gerusalemme. Sua madre si chiamava Jecolia, ed era di Gerusalemme.
\par 3 Egli fece ciò ch'è giusto agli occhi dell'Eterno, interamente come avea fatto Amatsia suo padre.
\par 4 Nondimeno, gli alti luoghi non furon soppressi; il popolo continuava ad offrire sacrifizi e profumi sugli alti luoghi.
\par 5 E l'Eterno colpì il re, che fu lebbroso fino al giorno della sua morte e visse nell'infermeria; e Jotham, figliuolo del re, era a capo della casa reale e rendea giustizia al popolo del paese.
\par 6 Il rimanente delle azioni di Azaria, e tutto quello che fece, trovasi scritto nel libro delle Cronache dei re di Giuda.
\par 7 Azaria si addormentò coi suoi padri, e coi suoi padri lo seppellirono nella città di Davide; e Jotham, suo figliuolo, regnò in luogo suo.
\par 8 Il trentottesimo anno di Azaria, re di Giuda, Zaccaria, figliuolo di Geroboamo, cominciò a regnare sopra Israele a Samaria; e regnò sei mesi.
\par 9 Egli fece ciò ch'è male agli occhi dell'Eterno, come avean fatto i suoi padri; non si ritrasse dai peccati coi quali Geroboamo, figliuolo di Nebat, avea fatto peccare Israele.
\par 10 E Shallum, figliuolo di Jabesh, congiurò contro di lui; lo colpì in presenza del popolo, l'uccise, e regnò in sua vece.
\par 11 Il rimanente delle azioni di Zaccaria trovasi scritto nel libro delle Cronache dei re d'Israele.
\par 12 Così si avverò la parola che l'Eterno avea detta a Jehu: 'I tuoi figliuoli sederanno sul trono d'Israele fino alla quarta generazione'. E così avvenne.
\par 13 Shallum, figliuolo di Jabesh, cominciò a regnare l'anno trentanovesimo di Uzzia re di Giuda, e regnò un mese a Samaria.
\par 14 E Menahem, figliuolo di Gadi, salì da Tirtsa e venne a Samaria; colpì in Samaria Shallum, figliuolo di Jabesh, l'uccise, e regnò in luogo suo.
\par 15 Il rimanente delle azioni di Shallum, e la congiura ch'egli ordì, sono cose scritte nel libro delle Cronache dei re d'Israele.
\par 16 Allora Menahem, partito da Tirtsa, colpì Tifsah, tutto quello che ci si trovava, e il suo territorio; la colpì, perch'essa non gli aveva aperte le sue porte; e tutte le donne che ci si trovavano incinte, le fece sventrare.
\par 17 L'anno trentanovesimo del regno di Azaria, re di Giuda, Menahem, figliuolo di Gadi, cominciò a regnare sopra Israele; e regnò dieci anni a Samaria.
\par 18 Egli fece ciò ch'è male agli occhi dell'Eterno; non si ritrasse dai peccati coi quali Geroboamo, figliuolo di Nebat, aveva fatto peccare Israele.
\par 19 Ai suoi tempi Pul, re d'Assiria, fece invasione nel paese; e Menahem diede a Pul mille talenti d'argento affinché gli desse man forte per assicurare nelle sue mani il potere reale.
\par 20 E Menahem fece pagare quel danaro ad Israele, a tutti quelli ch'erano molto ricchi, per darlo al re d'Assiria; li tassò a ragione di cinquanta sicli d'argento a testa. Così il re d'Assiria se ne tornò via, e non si fermò nel paese.
\par 21 Il rimanente delle azioni di Menahem, e tutto quello che fece, si trova scritto nel libro delle Cronache dei re d'Israele.
\par 22 Menahem s'addormentò coi suoi padri, e Pekachia, suo figliuolo, regnò in luogo suo.
\par 23 Il cinquantesimo anno di Azaria re di Giuda, Pekachia, figliuolo di Menahem, cominciò a regnare sopra Israele a Samaria, e regnò due anni.
\par 24 Egli fece ciò ch'è male agli occhi dell'Eterno; non si ritrasse dai peccati coi quali Geroboamo, figliuolo di Nebat, avea fatto peccare Israele.
\par 25 E Pekah, figliuolo di Remalia, suo capitano, congiurò contro di lui, e lo colpì a Samaria, e con lui Argob e Arech, nella torre del palazzo reale. Avea seco cinquanta uomini di Galaad; uccise Pekachia, e regnò in luogo suo.
\par 26 Il rimanente delle azioni di Pekachia, tutto quello che fece, si trova scritto nel libro delle Cronache dei re d'Israele.
\par 27 L'anno cinquantesimosecondo di Azaria, re di Giuda, Pekah, figliuolo di Remalia, cominciò a regnare sopra Israele a Samaria, e regnò venti anni.
\par 28 Egli fece ciò ch'è male agli occhi dell'Eterno; non si ritrasse dai peccati coi quali Geroboamo, figliuolo di Nebat, avea fatto peccare Israele.
\par 29 Al tempo di Pekah, re d'Israele, venne Tiglath-Pileser, re di Assiria, e prese Ijon, Abel-Beth-Maaca, Janoah, Kedesh, Hatsor, Galaad, la Galilea, tutto il paese di Neftali, e ne menò gli abitanti in cattività in Assiria.
\par 30 Hosea, figliuolo di Ela, ordì una congiura contro Pekah, figliuolo di Remalia; lo colpì, l'uccise e regnò in luogo suo, l'anno ventesimo del regno di Jotham, figliuolo di Uzzia.
\par 31 Il rimanente delle azioni di Pekah, tutto quello che fece, si trova scritto nel libro delle Cronache dei re d'Israele.
\par 32 L'anno secondo del regno di Pekah, figliuolo di Remalia, re d'Israele, cominciò a regnare Jotham, figliuolo di Uzzia, re di Giuda.
\par 33 Aveva venticinque anni quando cominciò a regnare, e regnò sedici anni a Gerusalemme. Sua madre si chiamava Jerusha, figliuola di Tsadok.
\par 34 Egli fece ciò ch'è giusto agli occhi dell'Eterno, interamente come avea fatto Uzzia suo padre.
\par 35 Nondimeno, gli alti luoghi non furono soppressi; il popolo continuava ad offrir sacrifizi e profumi sugli alti luoghi. Jotham costruì la porta superiore della casa dell'Eterno.
\par 36 Il rimanente delle azioni di Jotham, tutto quello che fece, si trova scritto nel libro delle Cronache dei re di Giuda.
\par 37 In quel tempo l'Eterno cominciò a mandare contro Giuda Retsin, re di Siria, e Pekah, figliuolo di Remalia.
\par 38 Jotham s'addormentò coi suoi padri, e coi suoi padri fu sepolto nella città di Davide, suo padre. Ed Achaz, suo figliuolo, regnò in luogo suo.

\chapter{16}

\par 1 L'anno diciassettesimo di Pekah, figliuolo di Remalia, cominciò a regnare Achaz, figliuolo di Jotham, re di Giuda.
\par 2 Achaz avea venti anni quando cominciò a regnare, e regnò sedici anni a Gerusalemme. Egli non fece ciò ch'è giusto agli occhi dell'Eterno, il suo Dio, come avea fatto Davide suo padre;
\par 3 ma seguì la via dei re d'Israele, e fece perfino passare il suo figliuolo per il fuoco, seguendo le abominazioni delle genti che l'Eterno avea cacciate d'innanzi ai figliuoli d'Israele;
\par 4 e offriva sacrifizi e profumi sugli alti luoghi, sulle colline, e sotto ogni albero verdeggiante.
\par 5 Allora Retsin, re di Siria, e Pekah, figliuolo di Remalia, re d'Israele, salirono contro Gerusalemme per assalirla; e vi assediarono Achaz, ma non riuscirono a vincerlo.
\par 6 In quel tempo, Retsin, re di Siria, riconquistò Elath alla Siria, e cacciò i Giudei da Elath, e i Sirî entrarono in Elath dove sono rimasti fino al dì d'oggi.
\par 7 Achaz inviò dei messi a Tiglath-Pileser, re degli Assiri, per dirgli: 'Io son tuo servo e tuo figliuolo; sali qua e liberami dalle mani del re di Siria e dalle mani del re d'Israele, che sono sorti contro di me'.
\par 8 E Achaz prese l'argento e l'oro che si poté trovare nella casa dell'Eterno e nei tesori della casa reale, e li mandò in dono al re degli Assiri.
\par 9 Il re d'Assiria gli diè ascolto; salì contro Damasco, la prese, ne menò gli abitanti in cattività a Kir, e fece morire Retsin.
\par 10 E il re Achaz andò a Damasco, incontro a Tiglath-Pileser, re d'Assiria; e avendo veduto l'altare ch'era a Damasco, il re Achaz mandò al sacerdote Uria il disegno e il modello di quell'altare, in tutti i suoi particolari.
\par 11 E il sacerdote Uria costruì un altare, esattamente secondo il modello che il re Achaz gli avea mandato da Damasco; e il sacerdote Uria lo costruì prima del ritorno del re Achaz da Damasco.
\par 12 Al suo ritorno da Damasco, il re vide l'altare, vi s'accostò, vi salì,
\par 13 vi fece arder sopra il suo olocausto e la sua offerta, vi versò la sua libazione, e vi sparse il sangue dei suoi sacrifizi di azioni di grazie.
\par 14 L'altare di rame, ch'era dinanzi all'Eterno, perché non fosse fra il nuovo altare e la casa dell'Eterno, lo pose allato al nuovo altare, verso settentrione.
\par 15 E il re Achaz diede quest'ordine al sacerdote Uria: 'Fa' fumare sull'altare grande l'olocausto del mattino e l'oblazione della sera, l'olocausto del re e la sua oblazione, gli olocausti di tutto il popolo del paese e le sue oblazioni; versavi le loro libazioni, e spandivi tutto il sangue degli olocausti e tutto il sangue dei sacrifizi; quanto all'altare di rame toccherà a me a pensarvi'.
\par 16 E il sacerdote Uria fece tutto quello che il re Achaz gli aveva comandato.
\par 17 Il re Achaz spezzò anche i riquadri delle basi, e ne tolse le conche che v'eran sopra; trasse giù il mare di su i buoi di rame che lo reggevano, e lo posò sopra un pavimento di pietra.
\par 18 Mutò pure, nella casa dell'Eterno, a motivo del re d'Assiria, il portico del sabato ch'era stato edificato nella casa, e l'ingresso esterno riserbato al re.
\par 19 Il rimanente delle azioni compiute da Achaz si trova scritto nel libro delle Cronache dei re di Giuda.
\par 20 Achaz si addormentò coi suoi padri, e coi suoi padri fu sepolto nella città di Davide. Ed Ezechia, suo figliuolo, regnò in luogo suo.

\chapter{17}

\par 1 L'anno dodicesimo di Achaz, re di Giuda, Hosea, figliuolo di Elah, cominciò a regnare sopra Israele a Samaria, e regnò nove anni.
\par 2 Egli fece ciò ch'è male agli occhi dell'Eterno; non però come gli altri re d'Israele che l'aveano preceduto.
\par 3 Shalmaneser, re d'Assiria salì contro di lui; ed Hosea gli fu assoggettato e gli pagò tributo.
\par 4 Ma il re d'Assiria scoprì una congiura ordita da Hosea, il quale aveva inviato de' messi a So, re d'Egitto, e non pagava più il consueto annuo tributo al re d'Assiria; perciò il re d'Assiria lo fece imprigionare e mettere in catene.
\par 5 Poi il re d'Assiria invase tutto il paese, salì contro Samaria, e l'assediò per tre anni.
\par 6 L'anno nono di Hosea, il re d'Assiria prese Samaria, e trasportò gl'Israeliti in Assiria e li collocò in Halah, e sullo Habor, fiume di Gozan, e nelle città dei Medi.
\par 7 Questo avvenne perché i figliuoli d'Israele avevan peccato contro l'Eterno, il loro Dio, che li avea tratti dal paese d'Egitto, di sotto al potere di Faraone re d'Egitto; ed aveano riveriti altri dèi;
\par 8 essi aveano imitati i costumi delle nazioni che l'Eterno avea cacciate d'innanzi a loro, e quelli che i re d'Israele aveano introdotti.
\par 9 I figliuoli d'Israele aveano fatto, in segreto, contro l'Eterno, il loro Dio, delle cose non rette; s'erano costruiti degli alti luoghi in tutte le loro città, dalle torri de' guardiani alle città fortificate;
\par 10 aveano eretto colonne ed idoli sopra ogni colle elevato e sotto ogni albero verdeggiante;
\par 11 e quivi, su tutti gli alti luoghi, aveano offerto profumi, come le nazioni che l'Eterno avea cacciate d'innanzi a loro; aveano commesso azioni malvage, provocando ad ira l'Eterno;
\par 12 e avean servito gl'idoli, mentre l'Eterno avea lor detto: 'Non fate una tal cosa!'
\par 13 Eppure l'Eterno avea avvertito Israele e Giuda per mezzo di tutti i profeti e di tutti i veggenti, dicendo: 'Convertitevi dalle vostre vie malvage, e osservate i miei comandamenti e i miei precetti, seguendo in tutta la legge che io prescrissi ai vostri padri, e che ho mandata a voi per mezzo dei miei servi, i profeti';
\par 14 ma essi non vollero dargli ascolto, e indurarono la loro cervice, come aveano fatto i loro padri, i quali non ebbero fede nell'Eterno, nel loro Dio;
\par 15 e rigettarono le sue leggi e il patto ch'egli avea fermato coi loro padri e gli avvertimenti ch'egli avea loro dato; andaron dietro a cose vacue, diventando vacui essi stessi; e andaron dietro alle nazioni circonvicine, che l'Eterno avea loro proibito d'imitare;
\par 16 e abbandonarono tutti i comandamenti dell'Eterno, del loro Dio; si fecero due vitelli di getto, si fabbricarono degl'idoli d'Astarte, adorarono tutto l'esercito del cielo, servirono Baal;
\par 17 fecero passare per il fuoco i loro figliuoli e le loro figliuole, si applicarono alla divinazione e agli incantesimi, e si dettero a fare ciò ch'è male agli occhi dell'Eterno, provocandolo ad ira.
\par 18 Perciò l'Eterno si adirò fortemente contro Israele, e lo allontanò dalla sua presenza; non rimase altro che la sola tribù di Giuda.
\par 19 - E neppure Giuda osservò i comandamenti dell'Eterno, del suo Dio, ma seguì i costumi stabiliti da Israele.
\par 20 E l'Eterno rigettò tutta la stirpe d'Israele, la umiliò, e l'abbandonò in balìa di predoni, finché la cacciò dalla sua presenza.
\par 21 Poiché, quand'egli ebbe strappato Israele dalla casa di Davide e quelli ebbero proclamato re Geroboamo, figliuolo di Nebat, Geroboamo distolse Israele dal seguire l'Eterno, e gli fece commettere un gran peccato.
\par 22 E i figliuoli d'Israele s'abbandonarono a tutti i peccati che Geroboamo avea commessi, e non se ne ritrassero,
\par 23 fino a tanto che l'Eterno mandò via Israele dalla sua presenza, come l'avea predetto per bocca di tutti i profeti suoi servi; e Israele fu trasportato dal suo paese in Assiria, dov'è rimasto fino al dì d'oggi.
\par 24 E il re d'Assiria fece venir genti da Babilonia, da Cutha, da Avva, da Hamath e da Sefarvaim, e le stabilì nelle città della Samaria in luogo dei figliuoli d'Israele; e quelle presero possesso della Samaria, e dimorarono nelle sue città.
\par 25 E quando cominciarono a dimorarvi, non temevano l'Eterno; e l'Eterno mandò contro di loro dei leoni, che faceano strage fra loro.
\par 26 Fu quindi detto al re d'Assiria: 'Le genti che tu hai trasportate e stabilite nelle città della Samaria non conoscono il modo di servire l'Iddio del paese; perciò questi ha mandato contro di loro de' leoni, che ne fanno strage, perch'esse non conoscono il modo di servire l'iddio del paese'.
\par 27 Allora il re d'Assiria dette quest'ordine: 'Fate tornare colà uno dei sacerdoti che avete di là trasportati; ch'egli vada a stabilirsi quivi, e insegni loro il modo di servire l'iddio del paese'.
\par 28 Così uno dei sacerdoti ch'erano stati trasportati dalla Samaria venne a stabilirsi a Bethel, e insegnò loro come doveano temere l'Eterno.
\par 29 Nondimeno, ognuna di quelle genti si fece i propri dèi nelle città dove dimorava, e li mise nelle case degli alti luoghi che i Samaritani aveano costruito.
\par 30 Quei di Babilonia fecero Succoth-Benoth; quelli di Cuth fecero Nergal; quelli di Hamath fecero Ascima;
\par 31 quelli di Avva fecero Nibhaz e Tartak; e quelli di Sefarvaim bruciavano i loro figliuoli in onore di Adrammelec e di Anammelec, dèi di Sefarvaim.
\par 32 E temevano anche l'Eterno; e si fecero de' sacerdoti degli alti luoghi ch'essi prendevano di fra loro, e che offrivano per essi de' sacrifizi nelle case degli alti luoghi.
\par 33 Così temevano l'Eterno, e servivano al tempo stesso i loro dèi, secondo il costume delle genti di fra le quali erano stati trasportati in Samaria.
\par 34 Anche oggi continuano nell'antico costume: non temono l'Eterno, e non si conformano né alle loro leggi e ai loro precetti, né alla legge e ai comandamenti che l'Eterno prescrisse ai figliuoli di Giacobbe, da lui chiamato Israele,
\par 35 coi quali l'Eterno avea fermato un patto, dando loro quest'ordine: 'Non temete altri dèi, non vi prostrate dinanzi a loro, non li servite, né offrite loro sacrifizi;
\par 36 ma temete l'Eterno, che vi fe' salire dal paese d'Egitto per la sua gran potenza e col suo braccio disteso; dinanzi a lui prostratevi, a lui offrite sacrifizi;
\par 37 e abbiate cura di metter sempre in pratica i precetti, le regole, la legge e i comandamenti ch'egli scrisse per voi; e non temete altri dèi.
\par 38 Non dimenticate il patto ch'io fermai con voi, e non temete altri dèi;
\par 39 ma temete l'Eterno, il vostro Dio, ed egli vi libererà dalle mani di tutti i vostri nemici'.
\par 40 Ma quelli non ubbidirono, e continuarono invece a seguire l'antico loro costume.
\par 41 Così quelle genti temevano l'Eterno, e al tempo stesso servivano i loro idoli; e i loro figliuoli e i figliuoli dei loro figliuoli hanno continuato fino al dì d'oggi a fare quello che avean fatto i loro padri.

\chapter{18}

\par 1 Or l'anno terzo di Hosea, figliuolo d'Ela, re d'Israele, cominciò a regnare Ezechia, figliuolo di Achaz, re di Giuda.
\par 2 Avea venticinque anni quando cominciò a regnare, e regnò ventinove anni a Gerusalemme. Sua madre si chiamava Abi, figliuola di Zaccaria.
\par 3 Egli fece ciò ch'è giusto agli occhi dell'Eterno, interamente come avea fatto Davide suo padre.
\par 4 Soppresse gli alti luoghi, frantumò le statue, abbatté l'idolo d'Astarte, e fece a pezzi il serpente di rame che Mosè avea fatto; perché i figliuoli d'Israele gli aveano fino a quel tempo offerto profumi; ei lo chiamò Nehushtan.
\par 5 Egli ripose la sua fiducia nell'Eterno, nell'Iddio d'Israele; e fra tutti i re di Giuda che vennero dopo di lui o che lo precedettero non ve ne fu alcuno simile a lui.
\par 6 Si tenne unito all'Eterno, non cessò di seguirlo, e osservò i comandamenti che l'Eterno avea dati a Mosè.
\par 7 E l'Eterno fu con Ezechia, che riusciva in tutte le sue imprese. Si ribellò al re d'Assiria, e non gli fu più soggetto;
\par 8 sconfisse i Filistei fino a Gaza, e ne devastò il territorio, dalle torri dei guardiani alle città fortificate.
\par 9 Il quarto anno del re Ezechia, ch'era il settimo anno di Hosea, figliuolo d'Ela re d'Israele, Shalmaneser, re d'Assiria, salì contro Samaria e l'assediò.
\par 10 In capo a tre anni, la prese; il sesto anno d'Ezechia, ch'era il nono anno di Hosea, re d'Israele, Samaria fu presa.
\par 11 E il re d'Assiria trasportò gl'Israeliti in Assiria, e li collocò in Halah, e sullo Habor, fiume di Gozan, e nelle città dei Medi,
\par 12 perché non aveano ubbidito alla voce dell'Eterno, dell'Iddio loro, ed aveano trasgredito il suo patto, cioè tutto quello che Mosè, servo dell'Eterno, avea comandato; essi non l'aveano né ascoltato, né messo in pratica.
\par 13 Or il quattordicesimo anno del re Ezechia, Sennacherib, re d'Assiria, salì contro tutte le città fortificate di Giuda, e le prese.
\par 14 Ed Ezechia, re di Giuda, mandò a dire al re d'Assiria a Lakis: 'Ho mancato; ritirati da me, ed io mi sottometterò a tutto quello che m'imporrai'. E il re d'Assiria impose ad Ezechia, re di Giuda, trecento talenti d'argento e trenta talenti d'oro.
\par 15 Ezechia diede tutto l'argento che si trovava nella casa dell'Eterno, e nei tesori della casa del re.
\par 16 E fu allora che Ezechia, re di Giuda, staccò dalle porte del tempio dell'Eterno e dagli stipiti le lame d'oro di cui egli stesso li aveva ricoperti, e le diede al re d'Assiria.
\par 17 E il re d'Assiria mandò ad Ezechia da Lakis a Gerusalemme, Tartan, Rabsaris e Rabshaké con un grande esercito. Essi salirono e giunsero a Gerusalemme. E, come furon giunti, vennero a fermarsi presso l'acquedotto dello stagno superiore, che è sulla strada del campo del lavator di panni.
\par 18 Chiamarono il re; ed Eliakim, figliuolo di Hilkia, prefetto del palazzo, si recò da loro con Scebna, il segretario, e Joah figliuolo di Asaf, l'archivista.
\par 19 E Rabshaké disse loro: 'Andate a dire ad Ezechia: - Così parla il gran re, il re d'Assiria: Che fiducia è cotesta che tu hai?
\par 20 Tu dici che consiglio e forza per far la guerra non son che parole vane; ma in chi metti la tua fiducia per ardire di ribellarti a me?
\par 21 Ecco, tu t'appoggi sull'Egitto, su questo sostegno di canna rotta, che penetra nella mano di chi vi s'appoggia e gliela fora; tal è Faraone, re d'Egitto, per tutti quelli che confidano in lui.
\par 22 Forse mi direte: - Noi confidiamo nell'Eterno, nel nostro Dio. - Ma non è egli quello stesso di cui Ezechia ha soppresso gli alti luoghi e gli altari, dicendo a Giuda e a Gerusalemme: - Voi adorerete soltanto dinanzi a questo altare, a Gerusalemme?
\par 23 Or dunque fa' una scommessa col mio signore, il re d'Assiria! Io ti darò duemila cavalli, se tu puoi fornire altrettanti cavalieri da montarli.
\par 24 E come potresti tu far voltar le spalle a un solo capitano tra gl'infimi servi del mio signore? E confidi nell'Egitto, a motivo de' suoi carri e de' suoi cavalieri!
\par 25 E adesso sono io forse salito senza il volere dell'Eterno contro questo luogo per distruggerlo? L'Eterno m'ha detto: - Sali contro questo paese e distruggilo'. -
\par 26 Allora Eliakim, figliuolo di Hilkia, Scebna e Joah dissero a Rabshaké: 'Ti prego, parla ai tuoi servi in aramaico, perché noi lo intendiamo; e non ci parlare in lingua giudaica, in guisa che la gente che sta sulle mura oda'.
\par 27 Ma Rabshaké rispose loro: 'Forse che il mio signore m'ha mandato a dir queste cose al tuo signore e a te? Non m'ha egli mandato a dirle a quegli uomini che stan seduti sulle mura e saran quanto prima ridotti a mangiare il loro sterco e a bere la loro orina con voi?'
\par 28 Allora Rabshaké, stando in piè, gridò ad alta voce, e disse in lingua giudaica: 'Udite la parola del gran re, del re d'Assiria!
\par 29 Così parla il re: - Non v'inganni Ezechia; poich'egli non potrà liberarvi dalle mie mani;
\par 30 né v'induca Ezechia a confidarvi nell'Eterno, dicendo: L'Eterno ci libererà certamente, e questa città non sarà data nelle mani del re d'Assiria.
\par 31 Non date ascolto ad Ezechia, perché così dice il re d'Assiria: - Fate pace con me e arrendetevi a me, e ognuno di voi mangerà del frutto della sua vigna e del suo fico, e berrà dell'acqua della sua cisterna,
\par 32 finch'io venga e vi meni in un paese simile al vostro: paese di grano e di vino, paese di pane e di vigne, paese d'ulivi da olio e di miele; e voi vivrete, e non morrete. - Non date dunque ascolto ad Ezechia, quando cerca d'ingannarvi dicendo: L'Eterno ci libererà.
\par 33 Ha qualcuno degli dèi delle genti liberato il proprio paese dalle mani del re d'Assiria?
\par 34 Dove sono gli dèi di Hamath e d'Arpad? Dove sono gli dèi di Sefarvaim, di Hena e d'Ivva? Hanno essi liberata Samaria dalla mia mano?
\par 35 Quali sono, fra tutti gli dèi di quei paesi, quelli che abbiano liberato il paese loro dalla mia mano? L'Eterno avrebb'egli a liberar dalla mia mano Gerusalemme?'
\par 36 E il popolo si tacque, e non gli rispose nulla; poiché il re avea dato quest'ordine: 'Non gli rispondete!'
\par 37 Allora Eliakim, figliuolo di Hilkia, prefetto del palazzo, Scebna il segretario, e Joah figliuolo d'Asaf, l'archivista, vennero da Ezechia con le vesti stracciate, e gli riferirono le parole di Rabshaké.

\chapter{19}

\par 1 Quando il re Ezechia ebbe udito queste cose, si stracciò le vesti, si coprì d'un sacco, ed entrò nella casa dell'Eterno.
\par 2 E mandò Eliakim, prefetto del palazzo, Scebna il segretario, e i più vecchi tra i sacerdoti, coperti di sacchi, dal profeta Isaia, figliuolo di Amots.
\par 3 Essi gli dissero: 'Così parla Ezechia: - Questo è giorno d'angoscia, di castigo, d'obbrobrio; poiché i figliuoli stan per uscire dal seno materno, ma la forza manca per partorirli.
\par 4 Forse l'Eterno, il tuo Dio, ha udite tutte le parole di Rabshaké, che il re d'Assiria, suo signore, ha mandato ad oltraggiare l'Iddio vivente; e, forse, l'Eterno, il tuo Dio, punirà le parole che ha udite. Rivolgigli dunque una preghiera a pro del resto del popolo che sussiste ancora!' -
\par 5 I servi del re Ezechia si recaron dunque da Isaia.
\par 6 Ed Isaia disse loro: 'Ecco quel che direte al vostro signore: Così dice l'Eterno: Non ti spaventare per le parole che hai udite, con le quali i servi del re d'Assiria m'hanno oltraggiato.
\par 7 Ecco, io metterò in lui uno spirito tale che, all'udire una certa notizia, egli tornerà al suo paese; ed io lo farò cadere di spada nel suo paese'.
\par 8 Rabshaké tornò al re d'Assiria, e lo trovò che assediava Libna; poiché egli avea saputo che il suo signore era partito da Lakis.
\par 9 Or Sennacherib ricevette notizie di Tirhaka, re d'Etiopia, che dicevano: 'Ecco, egli s'è mosso per darti battaglia'; perciò inviò di nuovo dei messi ad Ezechia, dicendo loro:
\par 10 'Direte così ad Ezechia, re di Giuda: - Il tuo Dio, nel quale confidi, non t'inganni dicendo: Gerusalemme non sarà data nelle mani del re d'Assiria.
\par 11 Ecco, tu hai udito quello che i re d'Assiria hanno fatto a tutti i paesi, e come li hanno distrutti; e tu scamperesti?
\par 12 Gli dèi delle nazioni che i miei padri distrussero, gli dèi di Gozan, di Haran, di Retsef, dei figliuoli di Eden ch'erano a Telassar, valsero eglino a liberarle?
\par 13 Dov'è il re di Hamath, il re d'Arpad, e il re della città di Sefarvaim, di Hena e d'Ivva?'
\par 14 Ezechia, ricevuta la lettera per le mani dei messi, la lesse; poi salì alla casa dell'Eterno, e la spiegò davanti all'Eterno;
\par 15 e davanti all'Eterno pregò in questo modo: 'O Eterno, Dio d'Israele, che siedi sopra i cherubini, tu, tu solo sei l'Iddio di tutti i regni della terra; tu hai fatti i cieli e la terra.
\par 16 O Eterno, porgi l'orecchio tuo, e ascolta! o Eterno, apri gli occhi tuoi, e guarda! Ascolta le parole di Sennacherib, che ha mandato quest'uomo per insultare l'Iddio vivente!
\par 17 È vero, o Eterno: i re d'Assiria hanno desolato le nazioni e i loro paesi,
\par 18 e han gettati nel fuoco i loro dèi; perché quelli non erano dèi; erano opera delle mani degli uomini; eran legno e pietra; ed essi li hanno distrutti.
\par 19 Ma ora, o Eterno, o Dio nostro, salvaci, te ne supplico, dalle mani di costui, affinché tutti i regni della terra conoscano che tu solo, o Eterno, sei Dio!'
\par 20 Allora Isaia, figliuolo di Amots, mandò a dire ad Ezechia: 'Così parla l'Eterno, l'Iddio d'Israele: - Ho udito la preghiera che mi hai rivolta riguardo a Sennacherib, re d'Assiria.
\par 21 Questa è la parola che l'Eterno ha pronunziata contro di lui: "La vergine figliuola di Sion ti sprezza, si fa beffe di te; la figliuola di Gerusalemme scrolla il capo dietro a te.
\par 22 Chi hai tu insultato ed oltraggiato? Contro chi hai tu alzata la voce e levati in alto gli occhi tuoi? contro il Santo d'Israele!
\par 23 Per bocca de' tuoi messi tu hai insultato il Signore, e hai detto: - Con la moltitudine de' miei carri io son salito in vetta alle montagne, son penetrato nei recessi del Libano; io abbatterò i suoi cedri più alti, i suoi cipressi più belli, e arriverò al suo più remoto ricovero, alla sua più magnifica foresta.
\par 24 Io ho scavato e ho bevuto delle acque straniere; con la pianta dei miei piedi prosciugherò tutti i fiumi d'Egitto. -
\par 25 Non hai udito? Da lungo tempo ho preparato questo: dai tempi antichi ne ho formato il disegno; ed ora ho fatto sì che si compia: che tu riduca città forti in monti di ruine.
\par 26 I loro abitanti, privi di forza, sono spaventati e confusi; son come l'erba de' campi, come il verde tenero dei prati, come l'erbetta che nasce sui tetti, come grano riarso prima che formi la spiga.
\par 27 Ma io so quando ti siedi, quand'esci, quand'entri, e quando t'infurii contro di me.
\par 28 E per codesto tuo infuriare contro di me e perché la tua arroganza è giunta alle mie orecchie, io ti metterò il mio anello nelle narici, il mio morso in bocca, e ti rimenerò indietro per la via che hai fatta, venendo".
\par 29 E questo, o Ezechia, ti servirà di segno: Quest'anno si mangerà il frutto del grano caduto; il secondo anno, quello che crescerà da sé; ma il terzo anno, seminerete e mieterete; pianterete vigne, e ne mangerete il frutto.
\par 30 E ciò che resterà della casa di Giuda e scamperà, continuerà a mettere radici all'ingiù e a portar frutto in alto;
\par 31 poiché da Gerusalemme uscirà un residuo, e dal monte Sion uscirà quel che sarà scampato. Questo farà lo zelo ardente dell'Eterno degli eserciti!
\par 32 Perciò così parla l'Eterno riguardo al re d'Assiria: - Egli non entrerà in questa città, e non vi lancerà freccia; non le si farà innanzi con scudi, e non eleverà trincee contro ad essa.
\par 33 Ei se ne tornerà per la via ond'è venuto, e non entrerà in questa città, dice l'Eterno.
\par 34 Io proteggerò questa città affin di salvarla, per amor di me stesso, e per amor di Davide, mio servo'.
\par 35 E quella stessa notte avvenne che l'angelo dell'Eterno uscì e colpì nel campo degli Assiri cent'ottantacinquemila uomini; e quando la gente si levò la mattina, ecco, eran tutti cadaveri.
\par 36 Allora Sennacherib re d'Assiria levò il campo, partì e se ne tornò a Ninive, dove rimase.
\par 37 E avvenne che, mentr'egli stava adorando nella casa del suo dio Nisroc, i suoi figliuoli Adrammelec e Saretser lo uccisero a colpi di spada, e si rifugiarono nel paese di Ararat. Esarhaddon, suo figliuolo, regnò in luogo suo.

\chapter{20}

\par 1 In quel tempo, Ezechia fu malato a morte. Il profeta Isaia, figliuolo di Amots, si recò da lui, e gli disse: 'Così parla l'Eterno: - Metti ordine alle cose della tua casa; perché tu sei un uomo morto; non vivrai'. -
\par 2 Allora Ezechia volse la faccia verso il muro, e fece una preghiera all'Eterno, dicendo:
\par 3 'O Eterno, te ne supplico, ricordati come io ho camminato nel tuo cospetto con fedeltà e con integrità di cuore, e come ho fatto ciò ch'è bene agli occhi tuoi'. Ed Ezechia dette in un gran pianto.
\par 4 Isaia non era ancora giunto nel centro della città, quando la parola dell'Eterno gli fu rivolta in questi termini:
\par 5 'Torna indietro, e di' ad Ezechia, principe del mio popolo: - Così parla l'Eterno, l'Iddio di Davide tuo padre: Ho udita la tua preghiera, ho vedute le tue lacrime; ecco, io ti guarisco; fra tre giorni salirai alla casa dell'Eterno.
\par 6 Aggiungerò alla tua vita quindici anni, libererò te e questa città dalle mani del re d'Assiria, e proteggerò questa città per amor di me stesso, e per amor di Davide mio servo'.
\par 7 Ed Isaia disse: 'Prendete un impiastro di fichi secchi!' Lo presero, e lo misero sull'ulcera, e il re guarì.
\par 8 Or Ezechia avea detto ad Isaia: 'A che segno riconoscerò io che l'Eterno mi guarirà e che fra tre giorni salirò alla casa dell'Eterno?'
\par 9 E Isaia gli avea risposto: 'Eccoti da parte dell'Eterno il segno, dal quale riconoscerai che l'Eterno adempirà la parola che ha pronunziata: - Vuoi tu che l'ombra s'allunghi per dieci gradini ovvero retroceda di dieci gradini?' -
\par 10 Ezechia rispose: 'È cosa facile che l'ombra s'allunghi per dieci gradini; no; l'ombra retroceda piuttosto di dieci gradini'.
\par 11 E il profeta Isaia invocò l'Eterno, il quale fece retrocedere l'ombra di dieci gradini sui gradini d'Achaz, sui quali era discesa.
\par 12 In quel tempo, Berodac-Baladan, figliuolo di Baladan, re di Babilonia, mandò una lettera e un dono ad Ezechia, giacché avea sentito che Ezechia era stato infermo.
\par 13 Ezechia dette udienza agli ambasciatori, e mostrò loro la casa dov'erano tutte le sue cose preziose, l'argento, l'oro, gli aromi, gli olî finissimi, il suo arsenale, e tutto quello che si trovava nei suoi tesori. Non vi fu cosa nella sua casa e in tutti i suoi dominî, che Ezechia non mostrasse loro.
\par 14 Allora il profeta Isaia si recò dal re Ezechia, e gli disse: 'Che hanno detto quegli uomini? e donde son venuti a te?' Ezechia rispose: 'Son venuti da un paese lontano: da Babilonia'.
\par 15 Isaia disse: 'Che hanno veduto in casa tua?' Ezechia rispose: 'Hanno veduto tutto quello ch'è in casa mia; non v'è cosa nei miei tesori, ch'io non abbia mostrata loro'.
\par 16 Allora Isaia disse ad Ezechia: 'Ascolta la parola dell'Eterno:
\par 17 - Ecco, i giorni stanno per venire, quando tutto quello ch'è in casa tua e tutto quello che i tuoi padri hanno accumulato fino al dì d'oggi, sarà trasportato a Babilonia; e nulla ne rimarrà, dice l'Eterno.
\par 18 E de' tuoi figliuoli che saranno usciti da te, che tu avrai generati, ne saranno presi per farne degli eunuchi nel palazzo del re di Babilonia'.
\par 19 Ed Ezechia rispose ad Isaia: 'La parola dell'Eterno che tu hai pronunziata, è buona'. E aggiunse: 'Sì, se almeno vi sarà pace e sicurtà durante i giorni miei'.
\par 20 Il rimanente delle azioni di Ezechia, e tutte le sue prodezze, e com'egli fece il serbatoio e l'acquedotto e condusse le acque nella città, sono cose scritte nel libro delle Cronache dei re di Giuda.
\par 21 Ezechia s'addormentò coi suoi padri, e Manasse, suo figliuolo, regnò in luogo suo.

\chapter{21}

\par 1 Manasse avea dodici anni quando cominciò a regnare, e regnò cinquantacinque anni a Gerusalemme. Sua madre si chiamava Heftsiba.
\par 2 Egli fece ciò ch'è male agli occhi dell'Eterno, seguendo le abominazioni delle nazioni che l'Eterno avea cacciate d'innanzi ai figliuoli d'Israele.
\par 3 Egli riedificò gli alti luoghi che Ezechia suo padre avea distrutti, eresse altari a Baal, fece un idolo d'Astarte, come avea fatto Achab re d'Israele, e adorò tutto l'esercito del cielo e lo servì.
\par 4 Eresse pure degli altari ad altri dèi nella casa dell'Eterno, riguardo alla quale l'Eterno avea detto: 'In Gerusalemme io porrò il mio nome'.
\par 5 Eresse altari a tutto l'esercito del cielo nei due cortili della casa dell'Eterno.
\par 6 Fece passare pel fuoco il suo figliuolo, si dette alla magia e agl'incantesimi, e istituì di quelli che evocavano gli spiriti e predicevan l'avvenire; s'abbandonò interamente a fare ciò ch'è male agli occhi dell'Eterno, provocandolo ad ira.
\par 7 Mise l'idolo d'Astarte che avea fatto, nella casa riguardo alla quale l'Eterno avea detto a Davide e a Salomone suo figliuolo: 'In questa casa, e a Gerusalemme, che io ho scelta fra tutte le tribù d'Israele, porrò il mio nome in perpetuo;
\par 8 e non permetterò più che il piè d'Israele vada errando fuori del paese ch'io detti ai suoi padri, purché essi abbian cura di mettere in pratica tutto quello che ho loro comandato, e tutta la legge che il mio servo Mosè ha loro prescritta'.
\par 9 Ma essi non obbedirono, e Manasse li indusse a far peggio delle nazioni che l'Eterno avea distrutte dinanzi ai figliuoli d'Israele.
\par 10 E l'Eterno parlò per mezzo de' suoi servi, i profeti, in questi termini:
\par 11 'Giacché Manasse, re di Giuda, ha commesso queste abominazioni e ha fatto peggio di quanto fecer mai gli Amorei, prima di lui, e mediante i suoi idoli ha fatto peccare anche Giuda,
\par 12 così dice l'Eterno, l'Iddio d'Israele: - Ecco, io faccio venire su Gerusalemme e su Giuda tali sciagure, che chiunque ne udrà parlare n'avrà intronate le orecchie.
\par 13 E stenderò su Gerusalemme la cordella di Samaria e il livello della casa di Achab; e ripulirò Gerusalemme come si ripulisce un piatto, che, dopo ripulito, si volta sottosopra.
\par 14 E abbandonerò quel che resta della mia eredità; li darò nelle mani dei loro nemici, e diverranno preda e bottino di tutti i loro nemici,
\par 15 perché hanno fatto ciò ch'è male agli occhi miei, e m'hanno provocato ad ira dal giorno che i loro padri uscirono dall'Egitto, fino al dì d'oggi'. -
\par 16 Manasse sparse inoltre moltissimo sangue innocente: tanto, da empirne Gerusalemme da un capo all'altro; senza contare i peccati che fece commettere a Giuda, facendo ciò ch'è male agli occhi dell'Eterno.
\par 17 Il rimanente delle azioni di Manasse, e tutto quello che fece, e i peccati che commise, si trova scritto nel libro delle Cronache dei re di Giuda.
\par 18 Manasse s'addormentò coi suoi padri, e fu sepolto nel giardino della sua casa, nel giardino di Uzza; e Amon, suo figliuolo, regnò in luogo suo.
\par 19 Amon avea ventidue anni quando cominciò a regnare, e regnò due anni a Gerusalemme. Sua madre si chiamava Meshullemeth, figliuola di Haruts di Jotha.
\par 20 Egli fece ciò ch'è male agli occhi dell'Eterno, come avea fatto Manasse suo padre;
\par 21 seguì in tutto la via battuta dal padre suo, servì agl'idoli ai quali avea servito suo padre, e li adorò;
\par 22 abbandonò l'Eterno, l'Iddio dei suoi padri, e non camminò per la via dell'Eterno.
\par 23 Or i servi di Amon ordirono una congiura contro di lui, e uccisero il re in casa sua.
\par 24 Ma il popolo del paese mise a morte tutti quelli che avean congiurato contro il re Amon, e fece re, in sua vece, Giosia suo figliuolo.
\par 25 Il rimanente delle azioni compiute da Amon, si trova scritto nel libro delle Cronache dei re di Giuda.
\par 25 Il rimanente delle azioni compiute da Amon, si trova scritto nel libro delle Cronache dei re di Giuda.

\chapter{22}

\par 1 Giosia avea otto anni quando incominciò a regnare, e regnò trentun anni a Gerusalemme. Sua madre si chiamava Jedida, figliuola d'Adaia, da Botskath.
\par 2 Egli fece ciò ch'è giusto agli occhi dell'Eterno, e camminò in tutto e per tutto per la via di Davide suo padre, senza scostarsene né a destra né a sinistra.
\par 3 Or l'anno diciottesimo del re Giosia, il re mandò nella casa dell'Eterno Shafan, il segretario, figliuolo di Atsalia, figliuolo di Meshullam, e gli disse:
\par 4 'Sali da Hilkia, il sommo sacerdote, e digli che metta assieme il danaro ch'è stato portato nella casa dell'Eterno, e che i custodi della soglia hanno raccolto dalle mani del popolo;
\par 5 che lo si consegni ai direttori preposti ai lavori della casa dell'Eterno; e che questi lo diano agli operai addetti alle riparazioni della casa dell'Eterno:
\par 6 ai legnaiuoli, ai costruttori ed ai muratori, e se ne servano per comprare del legname e delle pietre da tagliare, per le riparazioni della casa.
\par 7 Ma non si farà render conto a quelli in mano ai quali sarà rimesso il danaro, perché agiscono con fedeltà'.
\par 8 Allora il sommo sacerdote Hilkia disse a Shafan, il segretario: 'Ho trovato nella casa dell'Eterno il libro della legge'. E Hilkia diede il libro a Shafan, che lo lesse.
\par 9 E Shafan, il segretario, andò a riferir la cosa al re, e gli disse: 'I tuoi servi hanno versato il danaro che s'è trovato nella casa, e l'hanno consegnato a quelli che son preposti ai lavori della casa dell'Eterno'.
\par 10 E Shafan, il segretario, disse ancora al re: 'Il sacerdote Hilkia mi ha dato un libro'. E Shafan lo lesse alla presenza del re.
\par 11 Quando il re ebbe udite le parole del libro della legge, si stracciò le vesti.
\par 12 Poi diede quest'ordine al sacerdote Hilkia, ad Ahikam, figliuolo di Shafan, ad Acbor, figliuolo di Micaia, a Shafan, il segretario, e ad Asaia, servo del re:
\par 13 'Andate a consultare l'Eterno per me, per il popolo e per tutto Giuda, riguardo alle parole di questo libro che s'è trovato; giacché grande è l'ira dell'Eterno che s'è accesa contro di noi, perché i nostri padri non hanno ubbidito alle parole di questo libro, e non hanno messo in pratica tutto quello che in esso ci è prescritto'.
\par 14 Il sacerdote Hilkia, Ahikam, Acbor, Shafan ed Asaia andarono dalla profetessa Hulda, moglie di Shallum, guardaroba, figliuolo di Tikva, figliuolo di Harhas. Essa dimorava a Gerusalemme, nel secondo quartiere; e quando ebbero parlato con lei, ella disse loro:
\par 15 'Così dice l'Eterno, l'Iddio d'Israele: Dite all'uomo che vi ha mandati da me:
\par 16 - Così dice l'Eterno: Ecco, io farò venire delle sciagure su questo luogo e sopra i suoi abitanti, conformemente a tutte le parole del libro che il re di Giuda ha letto.
\par 17 Essi m'hanno abbandonato ed hanno offerto profumi ad altri dèi per provocarmi ad ira con tutte le opere delle loro mani; perciò la mia ira s'è accesa contro questo luogo, e non si estinguerà.
\par 18 Quanto al re di Giuda che v'ha mandati a consultare l'Eterno, gli direte questo: Così dice l'Eterno, l'Iddio d'Israele, riguardo alle parole che tu hai udite:
\par 19 Giacché il tuo cuore è stato toccato, giacché ti sei umiliato dinanzi all'Eterno, udendo ciò che io ho detto contro questo luogo e contro i suoi abitanti, che saranno cioè abbandonati alla desolazione ed alla maledizione; giacché ti sei stracciate le vesti e hai pianto dinanzi a me, anch'io t'ho ascoltato, dice l'Eterno.
\par 20 Perciò, ecco, io ti riunirò coi tuoi padri, e te n'andrai in pace nel tuo sepolcro; e gli occhi tuoi non vedranno tutte le sciagure ch'io farò piombare su questo luogo'. - E quelli riferirono al re la risposta.

\chapter{23}

\par 1 Allora il re mandò a far raunare presso di sé tutti gli anziani di Giuda e di Gerusalemme.
\par 2 E il re salì alla casa dell'Eterno, con tutti gli uomini di Giuda, tutti gli abitanti di Gerusalemme, i sacerdoti, i profeti e tutto il popolo, piccoli e grandi, e lesse in loro presenza tutte le parole del libro del patto, ch'era stato trovato nella casa dell'Eterno.
\par 3 Il re, stando in piedi sul palco, stabilì un patto dinanzi all'Eterno, impegnandosi di seguire l'Eterno, d'osservare i suoi comandamenti, i suoi precetti e le sue leggi con tutto il cuore e con tutta l'anima, per mettere in pratica le parole di questo patto, scritte in quel libro. E tutto il popolo acconsentì al patto.
\par 4 E il re ordinò al sommo sacerdote Hilkia, ai sacerdoti del secondo ordine e ai custodi della soglia di trar fuori del tempio dell'Eterno tutti gli arredi che erano stati fatti per Baal, per Astarte e per tutto l'esercito celeste, e li arse fuori di Gerusalemme nei campi del Kidron, e ne portò le ceneri a Bethel.
\par 5 E destituì i sacerdoti idolatri che i re di Giuda aveano istituito per offrir profumi negli alti luoghi nelle città di Giuda e nei dintorni di Gerusalemme, e quelli pure che offrivan profumi a Baal, al sole, alla luna, ai segni dello zodiaco, e a tutto l'esercito del cielo.
\par 6 Trasse fuori dalla casa dell'Eterno l'idolo d'Astarte, che trasportò fuori di Gerusalemme verso il torrente Kidron; l'arse presso il torrente Kidron, lo ridusse in cenere, e ne gettò la cenere sui sepolcri della gente del popolo.
\par 7 Demolì le case di quelli che si prostituivano, le quali si trovavano nella casa dell'Eterno, e dove le donne tessevano delle tende per Astarte.
\par 8 Fece venire tutti i sacerdoti dalle città di Giuda, contaminò gli alti luoghi dove i sacerdoti aveano offerto profumi, da Gheba a Beer-Sceba, e abbatté gli alti luoghi delle porte: quello ch'era all'ingresso della porta di Giosuè, governatore della città, e quello ch'era a sinistra della porta della città.
\par 9 Or quei sacerdoti degli alti luoghi non salivano a sacrificare sull'altare dell'Eterno a Gerusalemme; mangiavan però pane azzimo in mezzo ai loro fratelli.
\par 10 Contaminò Tofeth, nella valle dei figliuoli di Hinnom, affinché nessuno facesse più passare per il fuoco il suo figliuolo o la sua figliuola in onore di Molec.
\par 11 Non permise più che i cavalli consacrati al sole dai re di Giuda entrassero nella casa dell'Eterno, nell'abitazione dell'eunuco Nethan-Melec, ch'era nel recinto del tempio; e diede alle fiamme i carri del sole.
\par 12 Il re demolì gli altari ch'erano sulla terrazza della camera superiore di Achaz, e che i re di Giuda aveano fatti, e gli altari che avea fatti Manasse nei due cortili della casa dell'Eterno; e, dopo averli fatti a pezzi e tolti di là, ne gettò la polvere nel torrente Kidron.
\par 13 E il re contaminò gli alti luoghi ch'erano dirimpetto a Gerusalemme, a destra del monte della perdizione, e che Salomone re d'Israele aveva eretti in onore di Astarte, l'abominazione dei Sidonî, di Kemosh, l'abominazione di Moab, e di Milcom, l'abominazione dei figliuoli d'Ammon.
\par 14 E spezzò le statue, abbatté gl'idoli d'Astarte, e riempì que' luoghi d'ossa umane.
\par 15 Abbatté pure l'altare che era a Bethel, e l'alto luogo, fatto da Geroboamo, figliuolo di Nebat, il quale avea fatto peccare Israele: arse l'alto luogo e lo ridusse in polvere, ed arse l'idolo d'Astarte.
\par 16 Or Giosia, voltatosi, scòrse i sepolcri ch'eran quivi sul monte, e mandò a trarre le ossa fuori da quei sepolcri, e le arse sull'altare, contaminandolo, secondo la parola dell'Eterno pronunziata dall'uomo di Dio, che aveva annunziate queste cose.
\par 17 Poi disse: 'Che monumento è quello ch'io vedo là?' La gente della città gli rispose: 'È il sepolcro dell'uomo di Dio che venne da Giuda, e che proclamò contro l'altare di Bethel queste cose che tu hai fatte'.
\par 18 Egli disse: 'Lasciatelo stare; nessuno muova le sue ossa!' Così le sue ossa furon conservate con le ossa del profeta ch'era venuto da Samaria.
\par 19 Giosia fece anche sparire tutte le case degli alti luoghi che erano nella città di Samaria e che i re d'Israele aveano fatte per provocare ad ira l'Eterno, e fece di essi esattamente quel che avea fatto di quei di Bethel.
\par 20 Immolò sugli altari tutti i sacerdoti degli alti luoghi che eran colà, e su quegli altari bruciò ossa umane. Poi tornò a Gerusalemme.
\par 21 Il re diede a tutto il popolo quest'ordine: 'Fate la Pasqua in onore dell'Eterno, del vostro Dio, secondo che sta scritto in questo libro del patto'.
\par 22 Poiché Pasqua simile non era stata fatta dal tempo de' giudici che avean governato Israele, e per tutto il tempo dei re d'Israele e dei re di Giuda;
\par 23 ma nel diciottesimo anno del re Giosia cotesta Pasqua fu fatta, in onor dell'Eterno, a Gerusalemme.
\par 24 Giosia fe' pure sparire quelli che evocavano gli spiriti e quelli che predicevano l'avvenire, le divinità familiari, gl'idoli e tutte le abominazioni che si vedevano nel paese di Giuda e a Gerusalemme, affin di mettere in pratica le parole della legge, scritte nel libro che il sacerdote Hilkia aveva trovato nella casa dell'Eterno.
\par 25 E prima di Giosia non c'è stato re che come lui si sia convertito all'Eterno con tutto il suo cuore, con tutta l'anima sua e con tutta la sua forza, seguendo in tutto la legge di Mosè; e, dopo di lui, non n'è sorto alcuno di simile.
\par 26 Tuttavia l'Eterno non desistette dall'ardore della grand'ira ond'era infiammato contro Giuda, a motivo di tutti gli oltraggi coi quali Manasse lo avea provocato ad ira.
\par 27 E l'Eterno disse: 'Anche Giuda io torrò d'innanzi al mio cospetto come n'ho tolto Israele; e rigetterò Gerusalemme, la città ch'io m'ero scelta, e la casa della quale avevo detto: - Là sarà il mio nome'. -
\par 28 Il rimanente delle azioni di Giosia, tutto quello che fece, si trova scritto nel libro delle Cronache dei re di Giuda.
\par 29 Al tempo suo, Faraone Neco, re d'Egitto, salì contro il re d'Assiria, verso il fiume Eufrate. Il re Giosia gli marciò contro, e Faraone, al primo incontro, l'uccise a Meghiddo.
\par 30 I suoi servi lo menaron via morto sopra un carro, e lo trasportarono da Meghiddo a Gerusalemme, dove lo seppellirono nel suo sepolcro. E il popolo del paese prese Joachaz, figliuolo di Giosia, lo unse, e lo fece re in luogo di suo padre.
\par 31 Joachaz avea ventitre anni quando cominciò a regnare, e regnò tre mesi a Gerusalemme. Il nome di sua madre era Hamutal, figliuola di Geremia da Libna.
\par 32 Egli fece ciò ch'è male agli occhi dell'Eterno, in tutto e per tutto come avean fatto i suoi padri.
\par 33 Faraone Neco lo mise in catene a Ribla, nel paese di Hamath, perché non regnasse più a Gerusalemme; e impose al paese un'indennità di cento talenti d'argento e di un talento d'oro.
\par 34 E Faraone Neco fece re Eliakim, figliuolo di Giosia, in luogo di Giosia suo padre, e gli mutò il nome in quello di Joiakim; e, preso Joachaz, lo menò in Egitto, dove morì.
\par 35 Joiakim diede a Faraone l'argento e l'oro; ma, per pagare quel danaro secondo l'ordine di Faraone, tassò il paese; e, imponendo a ciascuno una certa tassa, cavò dal popolo del paese l'argento e l'oro da dare a Faraone Neco.
\par 36 Joiakim avea venticinque anni quando cominciò a regnare, e regnò undici anni a Gerusalemme. Il nome di sua madre era Zebudda, figliuola di Pedaia da Ruma.
\par 37 Egli fece ciò ch'è male agli occhi dell'Eterno, in tutto e per tutto come aveano fatto i suoi padri.

\chapter{24}

\par 1 Al suo tempo, venne Nebucadnetsar re di Babilonia, e Joiakim gli fu assoggettato per tre anni; poi tornò a ribellarsi.
\par 2 E l'Eterno mandò contro Joiakim schiere di Caldei, di Sirî, schiere di Moabiti, schiere di Ammoniti, le mandò contro Giuda per distruggerlo, secondo la parola che l'Eterno avea pronunziata per mezzo dei profeti, suoi servi.
\par 3 Questo avvenne solo per ordine dell'Eterno, il quale voleva allontanare Giuda dalla sua presenza, a motivo di tutti i peccati che Manasse avea commessi,
\par 4 e a motivo pure del sangue innocente ch'egli avea sparso, e di cui avea riempito Gerusalemme. Per questo l'Eterno non volle perdonare.
\par 5 Il rimanente delle azioni di Joiakim, tutto quello che fece, si trova scritto nel libro delle Cronache dei re di Giuda.
\par 6 Joiakim s'addormentò coi suoi padri, e Joiakin, suo figliuolo, regnò in luogo suo.
\par 7 Or il re d'Egitto non uscì più dal suo paese, perché il re di Babilonia avea preso tutto quello che era stato del re d'Egitto, dal torrente d'Egitto al fiume Eufrate.
\par 8 Joiakin avea diciotto anni quando cominciò a regnare, e regnò a Gerusalemme tre mesi. Sua madre si chiamava Nehushta, figliuola di Elnathan da Gerusalemme.
\par 9 Egli fece ciò ch'è male agli occhi dell'Eterno, in tutto e per tutto come avea fatto suo padre.
\par 10 In quel tempo, i servi di Nebucadnetsar, re di Babilonia, salirono contro Gerusalemme, e la città fu cinta d'assedio.
\par 11 E Nebucadnetsar, re di Babilonia, giunse davanti alla città mentre la sua gente la stava assediando.
\par 12 Allora Joiakin, re di Giuda, si recò dal re di Babilonia, con sua madre, i suoi servi, i suoi capi ed i suoi eunuchi. E il re di Babilonia lo fece prigioniero, l'ottavo anno del suo regno.
\par 13 E, come l'Eterno avea predetto, portò via di là tutti i tesori della casa dell'Eterno e i tesori della casa del re, e spezzò tutti gli utensili d'oro che Salomone, re d'Israele, avea fatti per il tempio dell'Eterno.
\par 14 E menò in cattività tutta Gerusalemme, tutti i capi, tutti gli uomini valorosi, in numero di diecimila prigioni, e tutti i legnaiuoli e i fabbri; non vi rimase che la parte più povera della popolazione del paese.
\par 15 E deportò Joiakin a Babilonia; e menò in cattività da Gerusalemme a Babilonia la madre del re, le mogli del re, gli eunuchi di lui,
\par 16 i magnati del paese, tutti i guerrieri, in numero di settemila, i legnaiuoli e i fabbri, in numero di mille, tutta gente valorosa e atta alla guerra. Il re di Babilonia li menò in cattività a Babilonia.
\par 17 E il re di Babilonia fece re, in luogo di Joiakin, Mattania, zio di lui, al quale mutò il nome in quello di Sedekia.
\par 18 Sedekia avea ventun anni quando cominciò a regnare e regnò a Gerusalemme undici anni. Sua madre si chiamava Hamutal, figliuola di Geremia da Libna.
\par 19 Egli fece ciò ch'è male agli occhi dell'Eterno, in tutto e per tutto come avea fatto Joiakim.
\par 20 E a causa dell'ira dell'Eterno contro Gerusalemme e Giuda, le cose arrivarono al punto che l'Eterno li cacciò via dalla sua presenza.

\chapter{25}

\par 1 E Sedekia si ribellò al re di Babilonia. L'anno nono del regno di Sedekia, il decimo giorno del decimo mese, Nebucadnetsar, re di Babilonia, venne con tutto il suo esercito contro Gerusalemme; s'accampò contro di lei, e le costruì attorno delle trincee.
\par 2 E la città fu assediata fino all'undecimo anno del re Sedekia.
\par 3 Il nono giorno del quarto mese, la carestia era grave nella città; e non c'era più pane per il popolo del paese.
\par 4 Allora fu fatta una breccia alla città, e tutta la gente di guerra fuggì, di notte, per la via della porta fra le due mura, in prossimità del giardino del re, mentre i Caldei stringevano la città da ogni parte. E il re prese la via della pianura;
\par 5 ma l'esercito dei Caldei lo inseguì, lo raggiunse nelle pianure di Gerico, e tutto l'esercito di lui si disperse e l'abbandonò.
\par 6 Allora i Caldei presero il re, e lo condussero al re di Babilonia a Ribla, dove fu pronunziata sentenza contro di lui.
\par 7 I figliuoli di Sedekia furono scannati in sua presenza; poi cavaron gli occhi a Sedekia; lo incatenarono con una doppia catena di rame e lo menarono a Babilonia.
\par 8 Or il settimo giorno del quinto mese - era il diciannovesimo anno di Nebucadnetsar re di Babilonia - Nebuzaradan, capitano della guardia del corpo, servo del re di Babilonia, giunse a Gerusalemme,
\par 9 ed arse la casa dell'Eterno e la casa del re, e diede alle fiamme tutte le case di Gerusalemme, tutte le case della gente ragguardevole.
\par 10 E tutto l'esercito dei Caldei ch'era col capitano della guardia atterrò da tutte le parti le mura di Gerusalemme.
\par 11 Nebuzaradan, capitano della guardia, menò in cattività i superstiti ch'erano rimasti nella città, i fuggiaschi che s'erano arresi al re di Babilonia, e il resto della popolazione.
\par 12 Il capitano della guardia non lasciò che alcuni dei più poveri del paese a coltivar le vigne ed i campi.
\par 13 I Caldei spezzarono le colonne di rame ch'erano nella casa dell'Eterno, le basi, il mar di rame ch'era nella casa dell'Eterno, e ne portaron via il rame a Babilonia.
\par 14 Presero le pignatte, le palette, i coltelli, le coppe e tutti gli utensili di rame coi quali si faceva il servizio.
\par 15 Il capitano della guardia prese pure i bracieri, i bacini: l'oro di ciò ch'era d'oro, l'argento di ciò ch'era d'argento.
\par 16 Quanto alle due colonne, al mare e alle basi che Salomone avea fatti per la casa dell'Eterno, il rame di tutti questi oggetti avea un peso incalcolabile.
\par 17 L'altezza di una di queste colonne era di diciotto cubiti, e v'era su un capitello di rame alto tre cubiti; e attorno al capitello v'erano un reticolato e delle melagrane, ogni cosa di rame; lo stesso era della seconda colonna, munita pure di reticolato.
\par 18 Il capitano della guardia prese Seraia, il sommo sacerdote, Sofonia, il secondo sacerdote,
\par 19 e i tre custodi della soglia, e prese nella città un eunuco che comandava la gente di guerra, cinque uomini di fra i consiglieri intimi del re che furon trovati nella città, il segretario del capo dell'esercito che arrolava il popolo del paese, e sessanta privati che furono anch'essi trovati nella città.
\par 20 Nebuzaradan, capitano della guardia, li prese e li condusse al re di Babilonia a Ribla;
\par 21 e il re di Babilonia li fece colpire a morte a Ribla, nel paese di Hamath. Così Giuda fu menato in cattività lungi dal suo paese.
\par 22 Quanto al popolo che rimase nel paese di Giuda, lasciatovi da Nebucadnetsar, re di Babilonia, il re pose a governarli Ghedalia, figliuolo di Ahikam, figliuolo di Shafan.
\par 23 Quando tutti i capitani della gente di guerra e i loro uomini ebbero udito che il re di Babilonia avea fatto Ghedalia governatore, si recarono da Ghedalia a Mitspa: erano Ismael figliuolo di Nethania, Johanan figliuolo di Kareah, Seraia figliuolo di Tanhumet da Netofah, Jaazania figliuolo d'uno di Maacah, con la loro gente.
\par 24 Ghedalia fece ad essi e alla loro gente un giuramento, dicendo: 'Non v'incutano timore i servi dei Caldei; restate nel paese, servite al re di Babilonia, e ve ne troverete bene'.
\par 25 Ma il settimo mese, Ismael, figliuolo di Nethania, figliuolo di Elishama, di stirpe reale, venne accompagnato da dieci uomini e colpirono a morte Ghedalia insieme coi Giudei e coi Caldei ch'eran con lui a Mitspa.
\par 26 E tutto il popolo, piccoli e grandi, e i capitani della gente di guerra si levarono e se ne andarono in Egitto, perché avean paura dei Caldei.
\par 27 Il trentasettesimo anno della cattività di Joiakin, re di Giuda, il ventisettesimo giorno del dodicesimo mese, Evilmerodac, re di Babilonia, l'anno stesso che cominciò a regnare, fece grazia a Joiakin, re di Giuda, e lo trasse di prigione;
\par 28 gli parlò benignamente, e mise il trono d'esso più in alto di quello degli altri re ch'eran con lui a Babilonia.
\par 29 Gli fece mutare le vesti di prigione; e Joiakin mangiò sempre a tavola con lui per tutto il tempo ch'ei visse:
\par 30 il re provvide continuamente al suo mantenimento quotidiano, fintanto che visse.


\end{document}