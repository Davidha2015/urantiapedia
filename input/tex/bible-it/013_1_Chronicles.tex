\begin{document}

\title{1 Chronicles}


\chapter{1}

\par 1 Adamo, Seth, Enosh;
\par 2 Kenan, Mahalaleel, Jared;
\par 3 Enoc, Methushelah, Lamec;
\par 4 Noè, Sem, Cam, e Jafet.
\par 5 Figliuoli di Jafet: Gomer, Magog, Madai, Javan, Tubal, Mescec e Tiras.
\par 6 - Figliuoli di Gomer: Ashkenaz, Rifat e Togarma.
\par 7 - Figliuoli di Javan: Elisha, Tarsis, Kittim e Rodanim.
\par 8 Figliuoli di Cam: Cush, Mitsraim, Put e Canaan.
\par 9 - Figliuoli di Cush: Seba, Havila, Sabta, Raama e Sabteca. - Figliuoli di Raama: Sceba e Dedan.
\par 10 Cush generò Nimrod, che cominciò ad esser potente sulla terra.
\par 11 - Mitsraim generò i Ludim, gli Anamim, i Lehabim, i Naftuhim,
\par 12 i Pathrusim, i Casluhim (donde uscirono i Filistei) e i Caftorim.
\par 13 - Canaan generò Sidon, suo primogenito, e Heth,
\par 14 e i Gebusei, gli Amorei, i Ghirgasei,
\par 15 gli Hivvei, gli Archei, i Sinei,
\par 16 gli Arvadei, i Tsemarei e gli Hamathei.
\par 17 Figliuoli di Sem: Elam, Assur, Arpacshad, Lud e Aram; Uz, Hul, Ghether e Mescec.
\par 18 - Arpacshad generò Scelah, e Scelah generò Eber.
\par 19 Ad Eber nacquero due figliuoli: il nome dell'uno fu Peleg, perché ai suoi giorni la terra fu spartita; e il nome del suo fratello fu Joktan.
\par 20 Joktan generò Almodad, Scelef, Hatsarmaveth, Jerah,
\par 21 Hadoram, Uzal, Diklah,
\par 22 Ebal, Abimael, Sceba, Ofir, Havila e Jobab.
\par 23 Tutti questi furono figliuoli di Joktan.
\par 24 Sem, Arpacshad, Scelah,
\par 25 Eber, Peleg, Reu, Serug,
\par 26 Nahor, Terah,
\par 27 Abramo, che è Abrahamo.
\par 28 Figliuoli di Abrahamo: Isacco e Ismaele.
\par 29 Questi sono i loro discendenti: il primogenito d'Ismaele fu Nebaioth; poi, Kedar, Adbeel, Mibsam,
\par 30 Mishma, Duma, Massa, Hadad, Tema,
\par 31 Jetur, Nafish e Kedma. Questi furono i figliuoli d'Ismaele.
\par 32 Figliuoli di Ketura, concubina d'Abrahamo: essa partorì Zimran, Jokshan, Medan, Madian, Jishbak e Shuach. - Figliuoli di Jokshan: Sceba e Dedan.
\par 33 - Figliuoli di Madian: Efa, Efer, Hanoch, Abida ed Eldaa. Tutti questi furono i figliuoli di Ketura.
\par 34 Abrahamo generò Isacco. Figliuoli d'Isacco: Esaù e Israele.
\par 35 Figliuoli di Esaù: Elifaz, Reuel, Ieush, Ialam e Korah.
\par 36 - Figliuoli di Elifaz: Teman, Omar, Tsefi, Gatam, Kenaz, Timna ed Amalek. -
\par 37 Figliuoli di Reuel: Nahath, Zerach, Shammah e Mizza.
\par 38 Figliuoli di Seit: Lotan, Shobal, Tsibeon, Ana, Dishon, Etser e Dishan.
\par 39 - Figliuoli di Lotan: Hori e Homam; e la sorella di Lotan fu Timna.
\par 40 Figliuoli di Shobal: Alian, Manahath, Ebal, Scefi e Onam. - Figliuoli di Tsibeon: Aiah e Ana.
\par 41 - Figliuoli di Ana: Dishon. - Figliuoli di Dishon: Hamran, Eshban, Jthran e Keran.
\par 42 - Figliuoli di Etser: Bilhan, Zaavan, Jaakan. - Figliuoli di Dishon: Uts e Aran.
\par 43 Questi sono i re che regnarono nel paese di Edom prima che alcun re regnasse sui figliuoli d'Israele: Bela, figliuolo di Beor; e il nome della sua città fu Dinhaba.
\par 44 Bela morì e Jobab, figliuolo di Zerach, di Botsra, regnò in luogo suo.
\par 45 Jobab morì, e Husham, del paese de' Temaniti, regnò in luogo suo.
\par 46 Husham morì, e Hadad, figliuolo di Bedad, che sconfisse i Madianiti ne' campi di Moab, regnò in luogo suo; e il nome della sua città era Avith.
\par 47 Hadad morì, e Samla, di Masreka, regnò in luogo suo.
\par 48 Samla morì, e Saul di Rehoboth sul Fiume, regnò in luogo suo.
\par 49 Saul morì, e Baal-Hanan, figliuolo di Acbor, regnò in luogo suo.
\par 50 Baal-Hanan morì, e Hadad regnò in luogo suo. Il nome della sua città fu Pai, e il nome della sua moglie, Mehetabeel, figliuola di Matred, figliuola di Mezahab.
\par 51 E Hadad morì. I capi di Edom furono: il capo Timna, il capo Alva, il capo Ietheth,
\par 52 il capo Oholibama, il capo Ela, il capo Pinon,
\par 53 il capo Kenaz, il capo Teman, il capo Mibtsar,
\par 54 il capo Magdiel, il capo Iram. Questi sono i capi di Edom.

\chapter{2}

\par 1 Questi sono i figliuoli d'Israele: Ruben, Simeone, Levi, Giuda, Issacar e Zabulon;
\par 2 Dan, Giuseppe, Beniamino, Neftali, Gad e Ascer.
\par 3 Figliuoli di Giuda: Er, Onan e Scela; questi tre gli nacquero dalla figliuola di Shua, la Cananea. Er, primogenito di Giuda, era perverso agli occhi dell'Eterno, e l'Eterno lo fece morire.
\par 4 Tamar, nuora di Giuda, gli partorì Perets e Zerach. Totale dei figliuoli di Giuda: cinque.
\par 5 Figliuoli di Perets: Hetsron, Hamul. -
\par 6 Figliuoli di Zerach: Zimri, Ethan, Heman, Calcol e Dara: in tutto, cinque.
\par 7 Figliuoli di Carmi: Acan che conturbò Israele quando commise una infedeltà riguardo all'interdetto.
\par 8 Figliuoli di Ethan: Azaria.
\par 9 Figliuoli che nacquero a Hetsron: Jerahmeel, Ram e Kelubai.
\par 10 Ram generò Amminadab; Amminadab generò Nahshon, principe dei figliuoli di Giuda;
\par 11 e Nahshon generò Salma; e Salma generò Boaz.
\par 12 Boaz generò Obed. Obed generò Isai.
\par 13 Isai generò Eliab, suo primogenito, Abinadab il secondo, Scimea il terzo,
\par 14 Nethaneel il quarto, Raddai il quinto,
\par 15 Otsem il sesto, Davide il settimo.
\par 16 Le loro sorelle erano Tseruia ed Abigail. Figliuoli di Tseruia: Abishai, Joab ed Asael: tre.
\par 17 Abigail partorì Amasa, il cui padre fu Jether, l'Ismaelita.
\par 18 Caleb, figliuolo di Hetsron, ebbe dei figliuoli da Azuba sua moglie, e da Jerioth. Questi sono i figliuoli che ebbe da Azuba: Jescer, Shobab e Ardon.
\par 19 Azuba morì e Caleb sposò Efrath, che gli partorì Hur.
\par 20 Hur generò Uri, e Uri generò Betsaleel.
\par 21 - Poi Hetsron prese la figliuola di Makir, padre di Galaad; egli avea sessant'anni quando la sposò; ed essa gli partorì Segub.
\par 22 Segub generò Jair, che ebbe ventitre città nel paese di Galaad.
\par 23 I Gheshuriti e i Sirî presero loro le borgate di Jair, Kenath e i villaggi che ne dipendevano, sessanta città. Tutti cotesti erano figliuoli di Makir, padre di Galaad.
\par 24 Dopo la morte di Hetsron, avvenuta a Caleb-Efratha, Abiah, moglie di Hetsron, gli partorì Ashhur padre di Tekoa.
\par 25 I figliuoli di Jerahmeel, primogenito di Hetsron, furono: Ram, il primogenito, Buna, Oren ed Otsem, nati da Ahija.
\par 26 Jerahmeel ebbe un'altra moglie, di nome Atara, che fu madre di Onam.
\par 27 I figliuoli di Ram, primogenito di Jerahmeel, furono: Maats, Jamin ed Eker.
\par 28 - I figliuoli di Onam furono: Shammai e Jada. Figliuoli di Shammai: Nadab e Abishur.
\par 29 La moglie di Abishur si chiamava Abihail, che gli partorì Ahban e Molid.
\par 30 Figliuoli di Nadab: Seled e Appaim. Seled morì senza figliuoli.
\par 31 Figliuoli di Appaim: Jscei. Figliuoli di Jscei: Sceshan. Figliuoli di Sceshan: Ahlai.
\par 32 - Figliuoli di Jada, fratello di Shammai: Jether e Jonathan. Jether morì senza figliuoli.
\par 33 Figliuoli di Jonathan: Peleth e Zaza. Questi sono i figliuoli di Jerahmeel. -
\par 34 Sceshan non ebbe figliuoli, ma sì delle figlie. Sceshan aveva uno schiavo egiziano per nome Jarha.
\par 35 E Sceshan diede la sua figliuola per moglie a Jarha, suo schiavo; ed essa gli partorì Attai.
\par 36 Attai generò Nathan; Nathan generò Zabad;
\par 37 Zabad generò Efial; Efial generò Obed;
\par 38 Obed generò Jehu; Jehu generò Azaria;
\par 39 Azaria generò Helets; Helets generò Elasa;
\par 40 Elasa generò Sismai; Sismai generò Shallum;
\par 41 Shallum generò Jekamia e Jekamia generò Elishama.
\par 42 Figliuoli di Caleb, fratello di Jerahmeel: Mesha, suo primogenito che fu padre di Zif, e i figliuoli di Maresha, che fu padre di Hebron.
\par 43 Figliuoli di Hebron: Kora, Tappuah, Rekem e Scema.
\par 44 Scema generò Raham, padre di Jorkeam, Rekem generò Shammai.
\par 45 Il figliuolo di Shammai fu Maon; e Maon fu il padre di Beth-Tsur.
\par 46 Efa, concubina di Caleb, partorì Haran, Motsa e Gazez. Haran generò Gazez.
\par 47 Figliuoli di Jahdai: Reghem, Jotham, Gheshan, Pelet, Efa e Shaaf.
\par 48 Maaca, concubina di Caleb, partorì Sceber e Tirhana.
\par 49 Partorì anche Shaaf, padre di Madmanna, Sceva, padre di Macbena e padre di Ghibea. La figliuola di Caleb era Acsa.
\par 50 Questi furono i figliuoli di Caleb: Ben-Hur, primogenito di Efrata, Shobal, padre di Kiriath-Jearim;
\par 51 Salma, padre di Bethlehem; Haref, padre di Beth-Gader.
\par 52 Shobal, padre di Kiriath-Jearim, ebbe per discendenti: Haroe, e la metà di Menuhoth.
\par 53 Le famiglie di Kiriath-Jearim furono: gli Ithrei, i Puthei, gli Shumatei e i Mishraei; dalle quali famiglie derivarono gli Tsorathiti e gli Eshtaoliti.
\par 54 Figliuoli di Salma: Bethlehem e i Netofatei, Atroth-Beth-Joab, la metà dei Manahatei, gli Tsoriti.
\par 55 E le famiglie di scribi che abitavano a Jabets: i Tirathei, gli Scimeathei, i Sucathei. Questi sono i Kenei discesi da Hammath, padre della casa di Recab.

\chapter{3}

\par 1 Questi furono i figliuoli di Davide, che gli nacquero a Hebron: il primogenito fu Amnon, di Ahinoam, la Jzreelita; il secondo fu Daniel, da Abigail, la Carmelita;
\par 2 il terzo fu Absalom, figliuolo di Maaca, figliuola di Talmai, re di Gheshur; il quarto fu Adonija, figliuolo di Hagghith;
\par 3 il quinto fu Scefatia, di Abital; il sesto fu Jithream, di Egla, sua moglie.
\par 4 Questi sei figliuoli gli nacquero a Hebron. Quivi regnò sette anni e sei mesi, e in Gerusalemme regnò trentatre anni.
\par 5 E questi furono i figliuoli che gli nacquero a Gerusalemme: Scimea, Shobab, Nathan, Salomone: quattro figliuoli natigli da Bath-Shua, figliuola di Ammiel;
\par 6 poi Jibhar, Elishama,
\par 7 Elifelet, Noga, Nefeg, Jafia,
\par 8 Elishama, Eliada ed Elifelet, cioè nove figliuoli.
\par 9 Tutti questi furono i figliuoli di Davide, senza contare i figliuoli delle sue concubine. E Tamar era loro sorella.
\par 10 Figliuoli di Salomone: Roboamo, che ebbe per figliuolo Abija, che ebbe per figliuolo Asa, che ebbe per figliuolo Giosafat,
\par 11 che ebbe per figliuolo Joram, che ebbe per figliuolo Achazia, che ebbe per figliuolo Joas,
\par 12 che ebbe per figliuolo Amatsia, che ebbe per figliuolo Azaria, che ebbe per figliuolo Jotham,
\par 13 che ebbe per figliuolo Achaz, che ebbe per figliuolo Ezechia, che ebbe per figliuolo Manasse,
\par 14 che ebbe per figliuolo Amon, che ebbe per figliuolo Giosia.
\par 15 Figliuoli di Giosia: Johanan, il primogenito; Joiakim, il secondo; Sedekia, il terzo; Shallum, il quarto.
\par 16 Figliuoli di Johiakim: Jeconia, ch'ebbe per figliuolo Sedekia.
\par 17 Figliuoli di Jeconia, il prigioniero: il suo figliuolo Scealtiel,
\par 18 e Malkiram, Pedaia, Scenatsar, Jekamia, Hoshama e Nedabia.
\par 19 Figliuoli di Pedaia: Zorobabele e Scimei. Figliuoli di Zorobabele: Meshullam e Hanania, e Scelomith, loro sorella;
\par 20 poi Hashuba, Ohel, Berekia, Hasadia, Jushab-Hesed, cinque in tutto.
\par 21 Figliuoli di Hanania: Pelatia e Isaia, i figliuoli di Refaia, i figliuoli d'Arnan, i figliuoli di Abdia, i figliuoli di Scecania.
\par 22 Figliuoli di Scecania: Scemaia. Figliuoli di Scemaia: Hattush, Jgal, Bariah, Nearia e Shafath, sei in tutto.
\par 23 Figliuoli di Nearia: Elioenai, Ezechia e Azrikam, tre in tutto.
\par 24 Figliuoli di Elioenai: Hodavia, Eliascib, Pelaia, Akkub, Iohanan, Delaia e Anani, sette in tutto.

\chapter{4}

\par 1 Figliuoli di Giuda: Perets, Hetseron, Carmi, Hur e Shobal.
\par 2 Reaia, figliuolo di Shobal, generò Jahath; Jahath generò Ahumai e Lahad. Queste sono le famiglie degli Tsorathei.
\par 3 Questi furono i discendenti del padre di Etham: Jzreel, Jshma e Jdbash; la loro sorella si chiamava Hatselelponi.
\par 4 Penuel fu padre di Ghedor; ed Ezer, padre di Husha. Questi sono i figliuoli di Hur, primogenito di Efrata, padre di Bethlehem.
\par 5 Ashhur, padre di Tekoa, ebbe due mogli: Helea e Naara.
\par 6 Naara gli partorì Ahuzam, Hefer, Themeni ed Ahashtari.
\par 7 Questi sono i figliuoli di Naara. Figliuoli di Helea: Tsereth, Tsohar ed Ethnan.
\par 8 Kotz generò Anub, Hatsobeba, e le famiglie di Aharhel, figliuolo di Harum.
\par 9 Jabets fu più onorato dei suoi fratelli; sua madre gli aveva messo nome Jabets, perché, diceva, 'l'ho partorito con dolore'.
\par 10 Jabets invocò l'Iddio d'Israele, dicendo: 'Oh se tu mi benedicessi e allargassi i miei confini, e se la tua mano fosse meco e se tu mi preservassi dal male in guisa ch'io non avessi da soffrire!' E Dio gli concedette quello che avea chiesto.
\par 11 Kelub, fratello di Shuha generò Mehir, che fu padre di Eshton.
\par 12 Eshton generò Beth-Rafa, Paseah e Tehinna, padre di Ir-Nahash. Questa è la gente di Zeca.
\par 13 Figliuoli di Kenaz: Othniel e Seraia. Figliuoli di Othniel: Hathath.
\par 14 Meonothai generò Ofra. Seraia generò Joab, padre degli abitanti la valle degli artigiani, perché erano artigiani.
\par 15 Figliuoli di Caleb figliuolo di Gefunne: Iru, Ela e Naam, i figliuoli d'Ela e Kenaz.
\par 16 Figliuoli di Jehallelel: Zif, Zifa, Thiria ed Asareel.
\par 17 Figliuoli di Esdra: Jether, Mered, Efer e Jalon. La moglie di Mered partorì Miriam, Shammai ed Ishbah, padre di Eshtemoa.
\par 18 L'altra sua moglie, la Giudea, partorì Jered, padre di Ghedor, Heber, padre di Soco e Jekuthiel, padre di Zanoah. Quelli nominati prima eran figliuoli di Bithia, figliuola di Faraone che Mered avea presa per moglie.
\par 19 Figliuoli della moglie di Hodija, sorella di Naham: il padre di Kehila, il Garmeo, ed Eshtemoa, il Maacatheo.
\par 20 Figliuoli di Scimon: Ammon, Rinna, Benhanan e Tilon. Figliuoli di Isci: Zozeth e Ben-Zoeth.
\par 21 Figliuoli di Scela, figliuolo di Giuda: Er, padre di Leca, Lada, padre di Maresha, e le famiglie della casa dove si lavora il bisso di Beth-Ashbea e Jokim,
\par 22 e la gente di Cozeba, e Joas, e Saraf, che signoreggiarono su Moab, e Jashubi-Lehem. Ma queste son cose d'antica data.
\par 23 Erano de' vasai e stavano a Netaim e a Ghederah; stavano quivi presso al re per lavorare al suo servizio.
\par 24 Figliuoli di Simeone: Nemuel, Jamin, Jarib, Zerah, Saul,
\par 25 ch'ebbe per figliuolo Shallum, ch'ebbe per figliuolo Mibsam, ch'ebbe per figliuolo Mishma.
\par 26 Figliuoli di Mishma: Hammuel, ch'ebbe per figliuolo Zaccur, ch'ebbe per figliuolo Scimei.
\par 27 Scimei ebbe sedici figliuoli e sei figliuole; ma i suoi fratelli non ebbero molti figliuoli; e le loro famiglie non si moltiplicarono quanto quelle dei figliuoli di Giuda.
\par 28 Si stabilirono a Beer-Sceba, a Molada, ad Hatsar-Shual,
\par 29 a Bilha, ad Etsem, a Tolad,
\par 30 a Bethuel, ad Horma, a Tsiklag,
\par 31 a Beth-Marcaboth, ad Hatsar-Susim, a Beth-Biri ed a Shaaraim. Queste furono le loro città, fino al regno di Davide.
\par 32 Aveano pure i villaggi di Etam, Ain, Rimmon, Token ed Ashan: cinque terre,
\par 33 e tutti i villaggi ch'erano nei dintorni di quelle città, fino a Baal. Queste furono le loro dimore, ed essi aveano le loro genealogie.
\par 34 Meshobad, Jamlec, Josha, figliuolo di Amatsia,
\par 35 Joel, Jehu, figliuolo di Joscibia, figliuolo di Seraia, figliuolo di Asiel,
\par 36 Elioenai, Jaakoba, Jeshohaia, Asaia, Adiel, Jesimiel,
\par 37 Benaia, Ziza, figliuolo di Scifi, figliuolo di Allon, figliuolo di Jedaia, figliuolo di Scimri, figliuolo di Scemaia,
\par 38 questi uomini, enumerati per nome, erano principi nelle loro famiglie, e le loro case patriarcali si accrebbero grandemente.
\par 39 Andarono dal lato di Ghedor, fino ad oriente della valle, in cerca di pasture per i loro bestiami.
\par 40 Trovarono pasture grasse e buone, e un paese vasto, quieto e tranquillo; poiché quelli che lo abitavano prima erano discendenti di Cam.
\par 41 Questi uomini, ricordati più sopra per nome, giunsero, al tempo di Ezechia, re di Giuda, fecero man bassa sulle loro tende e sui Maoniti che si trovavan quivi, e li votarono allo sterminio, né sono risorti fino al dì d'oggi; poi si stabiliron colà in luogo di quelli, perché v'era pastura per i bestiami.
\par 42 E una parte di questi figliuoli di Simeone, cinquecento uomini, andarono verso il monte Seir, avendo alla loro testa Pelatia, Nearia, Refaia ed Uziel figliuoli di Isci;
\par 43 distrussero gli avanzi degli Amalekiti che avean potuto salvarsi, e si stabiliron quivi, dove son rimasti fino al dì d'oggi.

\chapter{5}

\par 1 Figliuoli di Ruben, primogenito d'Israele. - Poiché egli era il primogenito; ma siccome profanò il talamo di suo padre, la sua primogenitura fu data ai figliuoli di Giuseppe, figliuolo d'Israele. Nondimeno, Giuseppe non fu iscritto nelle genealogie come primogenito;
\par 2 Giuda ebbe, è vero, la prevalenza tra i suoi fratelli, e da lui è disceso il principe; ma il diritto di primogenitura appartiene a Giuseppe.
\par 3 - Figliuoli di Ruben, primogenito d'Israele: Hanoc, Pallu, Hetsron e Carmi.
\par 4 Figliuoli di Joel: Scemaia, ch'ebbe per figliuolo Gog, che ebbe per figliuolo Scimei,
\par 5 che ebbe per figliuolo Mica, ch'ebbe per figliuolo Reaia, ch'ebbe per figliuolo Baal,
\par 6 ch'ebbe per figliuolo Beera, che Tilgath-Pilneser, re di Assiria, menò in cattività. Esso era principe dei Rubeniti.
\par 7 Fratelli di Beera, secondo le loro famiglie, come sono iscritti nelle genealogie secondo le loro generazioni: il primo, Jeiel; poi Zaccaria,
\par 8 Bela, figliuolo di Azaz, figliuolo di Scema, figliuolo di Joel. Bela dimorava ad Aroer e si estendeva fino a Nebo ed a Baal-Meon;
\par 9 a oriente occupava il paese dal fiume Eufrate fino all'entrata del deserto, perché avea gran quantità di bestiame nel paese di Galaad.
\par 10 Al tempo di Saul, i discendenti di Bela mossero guerra agli Hagareni, che caddero nelle loro mani; e quelli si stabilirono nelle loro tende, su tutto il lato orientale di Galaad.
\par 11 I figliuoli di Gad dimoravano dirimpetto a loro nel paese di Bashan, fino a Salca.
\par 12 Joel fu il primo; Shafam, il secondo; poi Janai e Shafat in Bashan.
\par 13 I loro fratelli, secondo le loro case patriarcali, furono: Micael, Meshullam, Sceba, Jorai, Jacan, Zia ed Eber: sette in tutto.
\par 14 Essi erano figliuoli di Abihail, figliuolo di Huri, figliuolo di Jaroah, figliuolo di Galaad, figliuolo di Micael, figliuolo di Jeshishai, figliuolo di Jahdo, figliuolo di Buz;
\par 15 Ahi, figliuolo di Abdiel, figliuolo di Guni, era il capo della loro casa patriarcale.
\par 16 Abitavano nel paese di Galaad e di Bashan e nelle città che ne dipendevano, e in tutti i pascoli di Sharon fino ai loro estremi limiti.
\par 17 Tutti furono iscritti nelle genealogie al tempo di Jotham, re di Giuda, e al tempo di Geroboamo, re d'Israele.
\par 18 I figliuoli di Ruben, i Gaditi e la mezza tribù di Manasse, che aveano degli uomini prodi che portavano scudo e spada, tiravan d'arco ed erano addestrati alla guerra, in numero di quarantaquattromila settecentosessanta, atti a combattere,
\par 19 mossero guerra agli Hagareni, a Jetur, a Nafish e a Nodab.
\par 20 Furon soccorsi combattendo contro di loro, e gli Hagareni e tutti quelli ch'eran con essi furon dati loro nelle mani, perché durante il combattimento essi gridarono a Dio, che li esaudì, perché s'eran confidati in lui.
\par 21 Essi presero il bestiame dei vinti: cinquantamila cammelli, duecentocinquantamila pecore, duemila asini, e centomila persone;
\par 22 molti ne caddero morti, perché quella guerra procedeva da Dio. E si stabilirono nel luogo di quelli, fino alla cattività.
\par 23 I figliuoli della mezza tribù di Manasse abitarono anch'essi in quel paese, da Bashan fino a Baal-Hermon e a Senir e al monte Hermon. Erano numerosi,
\par 24 e questi sono i capi delle loro case patriarcali: Efer, Isci, Eliel, Azriel, Geremia, Hodavia, Jahdiel, uomini forti e valorosi, di gran rinomanza, capi delle loro case patriarcali.
\par 25 Ma furono infedeli all'Iddio dei loro padri, e si prostituirono andando dietro agli dèi dei popoli del paese, che Dio avea distrutti dinanzi a loro.
\par 26 E l'Iddio d'Israele eccitò lo spirito di Pul, re di Assiria, e lo spirito di Tilgath-Pilneser, re di Assiria; e Tilgath-Pilneser menò in cattività i Rubeniti, i Gaditi e la mezza tribù di Manasse, e li trasportò a Halah, ad Habor, ad Hara e presso al fiume di Gozan, dove son rimasti fino al dì d'oggi.

\chapter{6}

\par 1 Figliuoli di Levi: Ghershom, Kehath e Merari.
\par 2 Figliuoli di Kehath: Amram, Itsehar, Hebron ed Uziel.
\par 3 Figliuoli di Amram: Aaronne, Mosè e Maria. Figliuoli d'Aaronne: Nadab, Abihu, Eleazar ed Ithamar.
\par 4 Eleazar generò Fineas; Fineas generò Abishua;
\par 5 Abishua generò Bukki; Bukki generò Uzzi;
\par 6 Uzzi generò Zerahia; Zerahia generò Meraioth; Meraioth generò Amaria;
\par 7 Amaria generò Ahitub;
\par 8 Ahitub generò Tsadok; Tsadok generò Ahimaats;
\par 9 Ahimaats generò Azaria; Azaria generò Johanan;
\par 10 Johanan generò Azaria, che esercitò il sacerdozio nella casa che Salomone edificò a Gerusalemme.
\par 11 Azaria generò Amaria; Amaria generò Ahitub;
\par 12 Ahitub generò Tsadok; Tsadok generò Shallum;
\par 13 Shallum generò Hilkija;
\par 14 Hilkija generò Azaria; Azaria generò Seraia; Seraia generò Jehotsadak;
\par 15 Jehotsadak se n'andò in esilio quando l'Eterno fece menare in cattività Giuda e Gerusalemme da Nebucadnetsar.
\par 16 Figliuoli di Levi: Ghershom, Kehath e Merari.
\par 17 - Questi sono i nomi dei figliuoli di Ghershom: Libni e Scimei. -
\par 18 Figliuoli di Kehath: Amram, Jtsehar, Hebron e Uziel. -
\par 19 Figliuoli di Merari: Mahli e Musci. - Queste sono le famiglie di Levi, secondo le loro case patriarcali.
\par 20 Ghershom ebbe per figliuolo Libni, che ebbe per figliuolo Jahath, che ebbe per figliuolo Zimma,
\par 21 che ebbe per figliuolo Joah, ch'ebbe per figliuolo Iddo, ch'ebbe per figliuolo Zerah, ch'ebbe per figliuolo Jeathrai.
\par 22 - Figliuoli di Kehath: Amminadab, che ebbe per figliuolo Core, che ebbe per figliuolo Assir,
\par 23 che ebbe per figliuolo Elkana, che ebbe per figliuolo Ebiasaf, che ebbe per figliuolo Assir,
\par 24 che ebbe per figliuolo Tahath, che ebbe per figliuolo Uriel, che ebbe per figliuolo Uzzia, che ebbe per figliuolo Saul.
\par 25 Figliuoli di Elkana: Amasai ed Ahimoth,
\par 26 che ebbe per figliuolo Elkana, che ebbe per figliuolo Tsofai, che ebbe per figliuolo Nahath,
\par 27 che ebbe per figliuolo Eliab, che ebbe per figliuolo Jeroham, che ebbe per figliuolo Elkana.
\par 28 Figliuoli di Samuele: Vashni, il primogenito, ed Abia.
\par 29 - Figliuoli di Merari: Mahli, che ebbe per figliuolo Libni, che ebbe per figliuolo Scimei, che ebbe per figliuolo Uzza,
\par 30 che ebbe per figliuolo Scimea, che ebbe per figliuolo Hagghia, che ebbe per figliuolo Asaia.
\par 31 Questi son quelli che Davide stabilì per la direzione del canto nella casa dell'Eterno, dopo che l'arca ebbe un luogo di riposo.
\par 32 Essi esercitarono il loro ufficio di cantori davanti al tabernacolo, davanti la tenda di convegno, finché Salomone ebbe edificata la casa dell'Eterno a Gerusalemme; e facevano il loro servizio, secondo la regola loro prescritta.
\par 33 Questi sono quelli che facevano il loro servizio, e questi i loro figliuoli. - Dei figliuoli dei Kehathiti: Heman, il cantore, figliuolo di Joel, figliuolo di Samuele,
\par 34 figliuolo di Elkana, figliuolo di Jeroham, figliuolo di Eiel, figliuolo di Toah,
\par 35 figliuolo di Tsuf, figliuolo di Elkana, figliuolo di Mahath,
\par 36 figliuolo d'Amasai, figliuolo d'Elkana, figliuolo di Joel, figliuolo d'Azaria, figliuolo di Sofonia,
\par 37 figliuolo di Tahath, figliuolo d'Assir, figliuolo di Ebiasaf, figliuolo di Core, figliuolo di Jtsehar,
\par 38 figliuolo di Kehath, figliuolo di Levi, figliuolo d'Israele. -
\par 39 Poi v'era il suo fratello Asaf, che gli stava alla destra: Asaf, figliuolo di Berekia, figliuolo di Scimea,
\par 40 figliuolo di Micael, figliuolo di Baaseia, figliuolo di Malkija,
\par 41 figliuolo d'Ethni, figliuolo di Zerah, figliuolo d'Adaia,
\par 42 figliuolo d'Ethan, figliuolo di Zimma, figliuolo di Scimei,
\par 43 figliuolo di Jahath, figliuolo di Ghershom, figliuolo di Levi. -
\par 44 I figliuoli di Merari, loro fratelli, stavano a sinistra, ed erano: Ethan, figliuolo di Kisci, figliuolo d'Abdi, figliuolo di Malluc,
\par 45 figliuolo di Hashabia, figliuolo d'Amatsia, figliuolo di Hilkia,
\par 46 figliuolo d'Amtsi, figliuolo di Bani, figliuolo di Scemer,
\par 47 figliuolo di Mahli, figliuolo di Musci, figliuolo di Merari, figliuolo di Levi.
\par 48 I loro fratelli, i Leviti, erano incaricati di tutto il servizio del tabernacolo della casa di Dio.
\par 49 Ma Aaronne ed i suoi figliuoli offrivano i sacrifizi sull'altare degli olocausti e l'incenso sull'altare dei profumi, compiendo tutto il servizio nel luogo santissimo, e facendo l'espiazione per Israele, secondo tutto quello che Mosè, servo di Dio, aveva ordinato.
\par 50 Questi sono i figliuoli d'Aaronne: Eleazar, che ebbe per figliuolo Fineas, che ebbe per figliuolo Abishua,
\par 51 che ebbe per figliuolo Bukki, che ebbe per figliuolo Uzzi, che ebbe per figliuolo Zerahia,
\par 52 che ebbe per figliuolo Meraioth, che ebbe per figliuolo Amaria, che ebbe per figliuolo Ahitub,
\par 53 che ebbe per figliuolo Tsadok, che ebbe per figliuolo Ahimaats.
\par 54 Questi sono i luoghi delle loro dimore, secondo le loro circoscrizioni nei territori loro assegnati. Ai figliuoli d'Aaronne della famiglia dei Kehathiti, che furono i primi tirati a sorte,
\par 55 furono dati Hebron, nel paese di Giuda, e il contado all'intorno;
\par 56 ma il territorio della città e i suoi villaggi furon dati a Caleb, figliuolo di Gefunne.
\par 57 Ai figliuoli d'Aaronne fu data Hebron, città di rifugio, Libna col suo contado, Jattir, Eshtemoa col suo contado,
\par 58 Hilen col suo contado, Debir col suo contado,
\par 59 Hasban col suo contado, Beth-Scemesh col suo contado;
\par 60 e della tribù di Beniamino: Gheba e il suo contado, Allemeth col suo contado, Anatoth col suo contado. Le loro città erano in tutto in numero di tredici, pari al numero delle loro famiglie.
\par 61 Agli altri figliuoli di Kehath toccarono a sorte dieci città delle famiglie della tribù di Efraim, della tribù di Dan e della mezza tribù di Manasse.
\par 62 Ai figliuoli di Ghershom, secondo le loro famiglie, toccarono tredici città, della tribù d'Issacar, della tribù di Ascer, della tribù di Neftali e della tribù di Manasse in Bashan.
\par 63 Ai figliuoli di Merari, secondo le loro famiglie, toccarono a sorte dodici città della tribù di Ruben, della tribù di Gad e della tribù di Zabulon.
\par 64 I figliuoli d'Israele dettero ai Leviti quelle città coi loro contadi;
\par 65 dettero a sorte, della tribù dei figliuoli di Giuda, della tribù dei figliuoli di Simeone e della tribù dei figliuoli di Beniamino, le dette città che furono designate per nome.
\par 66 Quanto alle altre famiglie dei figliuoli di Kehath, le città del territorio assegnato loro appartenevano alla tribù di Efraim.
\par 67 Dettero loro Sichem, città di rifugio, col suo contado, nella contrada montuosa di Efraim, Ghezer col suo contado,
\par 68 Jokmeam col suo contado, Beth-Horon col suo contado,
\par 69 Ajalon col suo contado, Gath-Rimmon col suo contado; e della mezza tribù di Manasse, Aner col suo contado, Bileam col suo contado.
\par 70 Queste furon le città date alle famiglie degli altri figliuoli di Kehath.
\par 71 Ai figliuoli di Ghershom toccarono: della famiglia della mezza tribù di Manasse, Golan in Bashan col suo contado, e Ashtaroth col suo contado; della tribù d'Issacar,
\par 72 Kedesh col suo contado, Dobrath col suo contado;
\par 73 Ramoth col suo contado, ed Anem col suo contado;
\par 74 della tribù di Ascer: Mashal col suo contado, Abdon col suo contado,
\par 75 Hukok col suo contado, Rehob col suo contado;
\par 76 della tribù di Neftali: Kedesh in Galilea col suo contado, Hammon col suo contado, e Kiriathaim col suo contado.
\par 77 Al rimanente dei Leviti, ai figliuoli di Merari, toccarono: della tribù di Zabulon, Rimmon col suo contado e Tabor col suo contado;
\par 78 e di là dal Giordano di Gerico, all'oriente del Giordano: della tribù di Ruben, Betser, nel deserto, col suo contado; Jahtsa col suo contado,
\par 79 Kedemoth col suo contado, e Mefaath col suo contado;
\par 80 e della tribù di Gad: Ramoth in Galaad, col suo contado, Mahanaim col suo contado,
\par 81 Heshbon col suo contado, e Jaezer col suo contado.

\chapter{7}

\par 1 Figliuoli d'Issacar: Tola, Puah, Jashub e Scimron: quattro in tutto.
\par 2 Figliuoli di Tola: Uzzi, Refaia, Jeriel, Jahmai, Jbsam e Samuele, capi delle case patriarcali discese da Tola; ed erano uomini forti e valorosi nelle loro generazioni; il loro numero, al tempo di Davide, era di ventiduemilaseicento.
\par 3 Figliuoli d'Uzzi: Jzrahia. Figliuoli di Jzrahia: Micael, Abdia, Joel ed Jsshia: cinque in tutto, e tutti capi.
\par 4 Aveano con loro, secondo le loro genealogie, secondo le loro case patriarcali, trentaseimila uomini in schiere armate per la guerra; perché aveano molte mogli e molti figliuoli.
\par 5 I loro fratelli, contando tutte le famiglie d'Issacar, uomini forti e valorosi, formavano un totale di ottantasettemila, iscritti nelle genealogie.
\par 6 Figliuoli di Beniamino: Bela, Beker e Jediael: tre in tutto.
\par 7 Figliuoli di Bela: Etsbon, Uzzi, Uzziel, Jerimoth ed Iri: cinque capi di case patriarcali, uomini forti e valorosi, iscritti nelle genealogie in numero di ventiduemilatrentaquattro. Figliuoli di Beker:
\par 8 Zemira, Joash, Eliezer, Elioenai, Omri, Jeremoth, Abija, Anathoth ed Alemeth. Tutti questi erano figliuoli di Beker,
\par 9 e iscritti nelle genealogie, secondo le loro generazioni, come capi di case patriarcali, uomini forti e valorosi, in numero di ventimila duecento.
\par 10 Figliuoli di Jediael: Bilhan. Figliuoli di Bilhan: Jeush, Beniamino, Ehud, Kenaana, Zethan, Tarsis ed Ahishahar.
\par 11 Tutti questi erano figliuoli di Jediael, capi di case patriarcali, uomini forti e valorosi, in numero di diciassettemiladuecento pronti a partire per la guerra.
\par 12 Shuppim e Huppim, figliuoli d'Ir; Huscim, figliuolo d'un altro.
\par 13 Figliuoli di Neftali: Jahtsiel, Guni, Jetser, Shallum, figliuoli di Bilha.
\par 14 Figliuoli di Manasse: Asriel, che gli fu partorito dalla moglie. - La sua concubina Sira partorì Makir, padre di Galaad;
\par 15 Makir prese per moglie una donna di Huppim e di Shuppim, e la sorella di lui avea nome Maaca. - Il nome del suo secondo figliuolo era Tselofehad; e Tselofehad ebbe delle figliuole.
\par 16 Maaca, moglie di Makir, partorì un figliuolo, al quale pose nome Peresh; questi ebbe un fratello di nome Sceresh, i cui figliuoli furono Ulam e Rekem.
\par 17 Figliuoli di Ulam: Bedan. Questi furono i figliuoli di Galaad, figliuolo di Makir, figliuolo di Manasse.
\par 18 La sua sorella Hammoleketh partorì Ishod, Abiezer e Mahla.
\par 19 I figliuoli di Semida furono Ahian, Scekem, Likhi ed Aniam.
\par 20 Figliuoli di Efraim: Shutela, che ebbe per figliuolo Bered, che ebbe per figliuolo Tahath, che ebbe per figliuolo Eleada, che ebbe per figliuolo Tahath,
\par 21 che ebbe per figliuolo Zabad, che ebbe per figliuolo Shutelah, - Ezer ed Elead, i quali furono uccisi da quei di Gath, nativi del paese, perch'erano scesi a predare il loro bestiame.
\par 22 Efraim, loro padre, li pianse per molto tempo, e i suoi fratelli vennero a consolarlo.
\par 23 Poi entrò da sua moglie, la quale concepì e partorì un figliuolo; ed egli lo chiamò Beria, perché questo era avvenuto mentre avea l'afflizione in casa.
\par 24 Efraim ebbe per figliuola Sceera, che edificò Beth-Horon, la inferiore e la superiore, ed Uzzen-Sceera.
\par 25 Ebbe ancora per figliuoli: Refa e Resef; il qual Refa ebbe per figliuolo Telah, che ebbe per figliuolo Tahan,
\par 26 che ebbe per figliuolo Ladan, che ebbe per figliuolo Ammihud, che ebbe per figliuolo Elishama, che ebbe per figliuolo Nun,
\par 27 che ebbe per figliuolo Giosuè.
\par 28 Le loro possessioni e abitazioni furono Bethel e le città che ne dipendevano; dalla parte d'oriente, Naaran; da occidente, Ghezer con le città che ne dipendevano, Sichem con le città che ne dipendevano, fino a Gaza con le città che ne dipendevano.
\par 29 I figliuoli di Manasse possedevano: Beth-Scean e le città che ne dipendevano, Taanac e le città che ne dipendevano, Meghiddo e le città che ne dipendevano, Dor e le città che ne dipendevano. In queste città abitarono i figliuoli di Giuseppe, figliuolo d'Israele.
\par 30 Figliuoli di Ascer: Jmna, Ishva, Ishvi, Beria, e Serah, loro sorella.
\par 31 Figliuoli di Beria: Heber e Malkiel. Malkiel fu padre di Birzavith.
\par 32 Heber generò Jaflet, Shomer, Hotham e Shua, loro sorella.
\par 33 Figliuoli di Jaflet: Pasac, Bimhal ed Asvath. Questi sono i figliuoli di Jaflet.
\par 34 Figliuoli di Scemer: Ahi, Rohega, Hubba ed Aram.
\par 35 Figliuoli di Helem, suo fratello: Tsofah, Jmna, Scelesh ed Amal.
\par 36 Figliuoli di Tsofah: Suah, Harnefer, Shual, Beri, Jmra,
\par 37 Betser, Hod, Shamma, Scilsha, Jthran e Beera.
\par 38 Figliuoli di Jether: Jefunne, Pispa ed Ara.
\par 39 Figliuoli di Ulla: Arah, Hanniel e Ritsia.
\par 40 Tutti questi furon figliuoli di Ascer, capi di case patriarcali, uomini scelti, forti e valorosi capi tra i principi, iscritti per servizio di guerra in numero di ventiseimila uomini.

\chapter{8}

\par 1 Beniamino generò Bela, suo primogenito, Ashbel il secondo, Aharah il terzo,
\par 2 Nohah il quarto, e Rafa il quinto.
\par 3 I figliuoli di Bela furono: Addar, Ghera, Abihud,
\par 4 Abishua, Naaman, Ahoah,
\par 5 Ghera, Scefufan e Huram.
\par 6 Questi sono i figliuoli di Ehud, che erano capi delle famiglie che abitavano Gheba e che furon trasportati schiavi a Manahath.
\par 7 Egli generò Naaman, Ahija e Ghera, che li menò via schiavi; e generò Uzza ed Ahihud.
\par 8 Shaharaim ebbe de' figliuoli nella terra di Moab dopo che ebbe ripudiate le sue mogli Huscim e Baara.
\par 9 Da Hodesh sua moglie ebbe: Jobab, Tsibia, Mesha, Malcam,
\par 10 Jeuts, Sokia e Mirma. Questi furono i suoi figliuoli, capi di famiglie patriarcali.
\par 11 Da Huscim ebbe: Abitub ed Elpaal.
\par 12 Figliuoli di Elpaal: Eber, Misham, Scemed, che edificò Ono, Lod, e le città che ne dipendevano.
\par 13 Beria e Scema erano i capi delle famiglie che abitavano Ajalon, e misero in fuga gli abitanti di Gath.
\par 14 Ahio, Shashak, Jeremoth, Zebadia,
\par 15 Arad, Eder,
\par 16 Micael, Jishpa, Joha erano figliuoli di Beria.
\par 17 Zebadia, Meshullam, Hizki, Heber,
\par 18 Jshmerai, Jzlia e Jobab erano figliuoli di Elpaal.
\par 19 Jakim, Zicri, Zabdi,
\par 20 Elienai, Tsilletai, Eliel,
\par 21 Adaia, Beraia e Scimrath erano figliuoli di Scimei.
\par 22 Jshpan, Eber, Eliel,
\par 23 Abdon, Zicri, Hanan,
\par 24 Hanania, Elam, Anthotija,
\par 25 Jfdeia e Penuel erano figliuoli di Shashak.
\par 26 Shamscerai, Sceharia, Atalia,
\par 27 Jaaresia, Elija e Zicri erano figliuoli di Jeroham.
\par 28 Questi erano capi di famiglie patriarcali: capi secondo le loro generazioni; e abitavano a Gerusalemme.
\par 29 Il padre di Gabaon abitava a Gabaon, e sua moglie si chiamava Maaca.
\par 30 Il suo figliuolo primogenito fu Abdon; poi ebbe Tsur, Kish, Baal, Nadab, Ghedor, Ahio, Zeker.
\par 31 Mikloth generò Scimea.
\par 32 Anche questi abitarono dirimpetto ai loro fratelli a Gerusalemme coi loro fratelli.
\par 33 Ner generò Kis; Kis generò Saul; Saul generò Gionathan, Malkishua, Abinadab, Eshbaal.
\par 34 Figliuoli di Gionathan: Merib-Baal. Merib-Baal generò Mica.
\par 35 Figliuoli di Mica: Pithon, Melec, Taarea, Ahaz.
\par 36 Ahaz generò Jehoadda; Jehoadda generò Alemeth, Azmaveth e Zimri; Zimri generò Motsa;
\par 37 Motsa generò Binea, che ebbe per figliuolo Rafa, che ebbe per figliuolo Eleasa, che ebbe per figliuolo Atsel.
\par 38 Atsel ebbe sei figliuoli, dei quali questi sono i nomi: Azrikam, Bocru, Ishmael, Scearia, Obadia e Hanan. Tutti questi erano figliuoli di Atsel.
\par 39 Figliuoli di Escek suo fratello: Ulam, il suo primogenito; Jeush il secondo, ed Elifelet il terzo.
\par 39 Figliuoli di Escek suo fratello: Ulam, il suo primogenito; Jeush il secondo, ed Elifelet il terzo.

\chapter{9}

\par 1 Tutti gl'Israeliti furono registrati nelle genealogie, e si trovano iscritti nel libro dei re d'Israele. Giuda fu menato in cattività a Babilonia, a motivo delle sue infedeltà.
\par 2 Or i primi abitanti che si stabilirono nei loro possessi e nelle loro città, erano Israeliti, sacerdoti, Leviti e Nethinei.
\par 3 A Gerusalemme si stabilirono dei figliuoli di Giuda, dei figliuoli di Beniamino, e dei figliuoli di Efraim e di Manasse.
\par 4 Dei figliuoli di Perets, figliuolo di Giuda: Uthai, figliuolo di Ammihud, figliuolo di Omri, figliuolo di Imri, figliuolo di Bani.
\par 5 Dei Sciloniti: Asaia il primogenito, e i suoi figliuoli.
\par 6 Dei figliuoli di Zerah: Jeuel e i suoi fratelli: seicentonovanta in tutto.
\par 7 Dei figliuoli di Beniamino: Sallu, figliuolo di Meshullam, figliuolo di Hodavia, figliuolo di Hassena;
\par 8 Jbneia, figliuolo di Jeroham; Ela, figliuolo di Uzzi, figliuolo di Micri; Meshullam, figliuolo di Scefatia, figliuolo di Reuel, figliuolo d'Jbnia;
\par 9 e i loro fratelli, secondo le loro generazioni, novecentocinquantasei in tutto. Tutti questi erano capi delle rispettive case patriarcali.
\par 10 Dei sacerdoti: Jedaia, Jehoiarib, Jakin,
\par 11 Azaria, figliuolo di Hilkia, figliuolo di Meshullam, figliuolo di Tsadok, figliuolo di Meraioth, figliuolo di Ahitub, preposto alla casa di Dio,
\par 12 Adaia, figliuolo di Jeroham, figliuolo di Pashur, figliuolo di Malkija; Maesai, figliuolo di Adiel, figliuolo di Jahzera, figliuolo di Meshullam, figliuolo di Mescillemith, figliuolo di Immer;
\par 13 e i loro fratelli, capi delle rispettive case patriarcali: millesettecentosessanta, uomini valentissimi, occupati a compiere il servizio della casa di Dio.
\par 14 Dei Leviti: Scemaia, figliuolo di Hasshub, figliuolo di Azrikam, figliuolo di Hashabia, dei figliuoli di Merari;
\par 15 Bakbakkar, Heresh, Galal, Mattania, figliuolo di Mica, figliuolo di Zicri, figliuolo di Asaf;
\par 16 Obadia, figliuolo di Scemaia, figliuolo di Galal, figliuolo di Jeduthun; Berakia, figliuolo di Asa, figliuolo di Elkana, che abitava nei villaggi dei Netofatiti.
\par 17 Dei portinai: Shallum, Akkub, Talmon, Ahiman e i loro fratelli; Shallum era il capo;
\par 18 e tale è rimasto fino al dì d'oggi, alla porta del re che è ad oriente. Essi son quelli che furono i portieri del campo dei figliuoli di Levi.
\par 19 Shallum, figliuolo di Kore, figliuolo di Ebiasaf, figliuolo di Korah, e i suoi fratelli, i Korahiti, della casa di suo padre, erano preposti all'opera del servizio, custodendo le porte del tabernacolo; i loro padri erano stati preposti al campo dell'Eterno per custodirne l'entrata;
\par 20 e Fineas, figliuolo di Eleazaro, era stato anticamente loro capo; e l'Eterno era con lui.
\par 21 Zaccaria, figliuolo di Mescelemia, era portiere all'ingresso della tenda di convegno.
\par 22 Tutti questi che furono scelti per essere custodi alle porte erano in numero di duecentododici, ed erano iscritti nelle genealogie, secondo i loro villaggi. Davide e Samuele il veggente li aveano stabiliti nel loro ufficio.
\par 23 Essi e i loro figliuoli erano preposti alla custodia delle porte della casa dell'Eterno, cioè della casa del tabernacolo.
\par 24 V'erano dei portinai ai quattro lati: a oriente, a occidente, a settentrione e a mezzogiorno.
\par 25 I loro fratelli, che dimoravano nei loro villaggi, doveano di quando in quando venire a stare dagli altri, per sette giorni;
\par 26 poiché i quattro capi portinai, Leviti, erano sempre in funzione, ed avevano anche la sorveglianza delle camere e dei tesori della casa di Dio,
\par 27 e passavano la notte intorno alla casa di Dio, perché aveano l'incarico di custodirla e a loro spettava l'aprirla tutte le mattine.
\par 28 Alcuni d'essi dovean prender cura degli arredi del culto, ch'essi contavano quando si portavano nel tempio e quando si riportavan fuori.
\par 29 Altri aveano l'incarico di custodire gli utensili, tutti i vasi sacri, il fior di farina, il vino, l'olio, l'incenso e gli aromi.
\par 30 Quelli che preparavano i profumi aromatici erano figliuoli di sacerdoti.
\par 31 Mattithia, uno dei Leviti, primogenito di Shallum il Korahita, avea l'ufficio di badare alle cose che si dovean cuocere sulla gratella.
\par 32 E alcuni dei loro fratelli, tra i Kehathiti, erano incaricati di preparare per ogni sabato i pani della presentazione.
\par 33 Tali sono i cantori, capi delle famiglie leviti che dimoravano nelle camere del tempio ed erano esenti da ogni altro servizio, perché l'ufficio loro li teneva occupati giorno e notte.
\par 34 Tali sono i capi delle famiglie levitiche, capi secondo le loro generazioni; essi stavano a Gerusalemme.
\par 35 A Gabaon abitavano Jeiel, padre di Gabaon, la cui moglie si chiamava Maaca,
\par 36 Abdon, suo figliuolo primogenito, Tsur-Kis,
\par 37 Baal, Ner, Nadab, Ghedor, Ahio, Zaccaria e Mikloth. Mikloth generò Scimeam.
\par 38 Anch'essi dimoravano dirimpetto ai loro fratelli a Gerusalemme coi loro fratelli.
\par 39 Ner generò Kis; Kis generò Saul; Saul generò Gionathan, Malkishua, Abinadab ed Eshbaal.
\par 40 Il figliuolo di Gionathan fu Merib-Baal, e Merib-Baal generò Mica.
\par 41 Figliuoli di Mica: Pithon, Melec, Taharea ed Ahaz.
\par 42 Ahaz generò Jarah; Jarah generò Alemeth, Azmaveth e Zimri. Zimri generò Motsa.
\par 43 Motsa generò Binea, che ebbe per figliuolo Refaia, che ebbe per figliuolo Eleasa, che ebbe per figliuolo Atsel.
\par 44 Atsel ebbe sei figliuoli, dei quali questi sono i nomi: Azrikam, Bocru, Ismaele, Scearia, Obadia e Hanan. Questi sono i figliuoli di Atsel.

\chapter{10}

\par 1 Or i Filistei vennero a battaglia con Israele, e gl'Israeliti fuggirono dinanzi ai Filistei, e caddero morti in gran numero sul monte Ghilboa.
\par 2 I Filistei inseguirono accanitamente Saul e i suoi figliuoli, e uccisero Gionathan, Abinadab e Malkishua, figliuoli di Saul.
\par 3 Il forte della battaglia si volse contro Saul; gli arcieri lo raggiunsero, ed egli si trovò in grande angoscia a motivo degli arcieri.
\par 4 E Saul disse al suo scudiere: 'Sfodera la spada e trafiggimi, affinché questi incirconcisi non vengano a trafiggermi ed a farmi oltraggio'. Ma lo scudiere non volle farlo, perch'era còlto da gran paura. Allora Saul prese la propria spada e vi si gettò sopra.
\par 5 Lo scudiere di Saul, vedendolo morto, si gettò anch'egli sulla propria spada, e morì.
\par 6 Così morirono Saul e i suoi tre figliuoli; e tutta la sua casa perì nel medesimo tempo.
\par 7 E tutti gl'Israeliti che abitavano nella valle, quando videro che la gente d'Israele s'era data alla fuga e che Saul e i suoi figliuoli erano morti, abbandonarono le loro città, e fuggirono; e i Filistei andarono ad abitarle.
\par 8 L'indomani i Filistei vennero a spogliare gli uccisi, e trovarono Saul e i suoi figliuoli caduti sul monte Ghilboa.
\par 9 Spogliarono Saul, e portaron via la sua testa e le sue armi, e mandarono all'intorno per il paese de' Filistei ad annunziare la buona notizia ai loro idoli ed al popolo;
\par 10 e collocarono le armi di lui nella casa del loro dio, e inchiodarono il suo teschio nel tempio di Dagon.
\par 11 Tutta la gente di Jabes di Galaad udì tutto quello che i Filistei avean fatto a Saul,
\par 12 e tutti gli uomini valorosi si levarono, presero il cadavere di Saul e i cadaveri dei suoi figliuoli, e li portarono a Jabes; seppellirono le loro ossa sotto alla tamerice di Jabes, e digiunarono per sette giorni.
\par 13 Così morì Saul, a motivo della infedeltà ch'egli avea commessa contro l'Eterno col non aver osservato la parola dell'Eterno, ed anche perché aveva interrogato e consultato quelli che evocano gli spiriti,
\par 14 mentre non avea consultato l'Eterno. E l'Eterno lo fece morire, e trasferì il regno a Davide, figliuolo d'Isai.

\chapter{11}

\par 1 Allora tutto Israele si radunò presso Davide a Hebron, e gli disse: 'Ecco noi siamo tue ossa e tua carne.
\par 2 Anche in passato quando era re Saul, eri tu quel che guidavi e riconducevi Israele; e l'Eterno, il tuo Dio, t'ha detto: - Tu pascerai il mio popolo d'Israele, tu sarai il principe del mio popolo d'Israele'.
\par 3 Tutti gli anziani d'Israele vennero dunque dal re a Hebron, e Davide fece alleanza con loro a Hebron in presenza dell'Eterno; ed essi unsero Davide come re d'Israele, secondo la parola che l'Eterno avea pronunziata per mezzo di Samuele.
\par 4 Davide con tutto Israele si mosse contro Gerusalemme, che è Gebus. Quivi erano i Gebusei, abitanti del paese.
\par 5 E gli abitanti di Gebus dissero a Davide: 'Tu non entrerai qui'. Ma Davide prese la fortezza di Sion che è la città di Davide.
\par 6 Or Davide avea detto: 'Chiunque batterà per il primo i Gebusei, sarà capo e principe'. E Joab, figliuolo di Tseruia, salì, il primo, e fu fatto capo.
\par 7 E Davide abitò nella fortezza, e per questo essa fu chiamata 'la città di Davide'.
\par 8 Ed egli cinse la città di costruzioni, cominciando da Millo, e tutto all'intorno; e Joab riparò il resto della città.
\par 9 E Davide andava diventando sempre più grande, e l'Eterno degli eserciti era con lui.
\par 10 Questi sono i capi dei valorosi guerrieri che furono al servizio di Davide, e che l'aiutarono con tutto Israele ad assicurare il suo dominio per stabilirlo re, secondo la parola dell'Eterno riguardo ad Israele.
\par 11 Questa è la lista dei valorosi guerrieri che furono al servizio di Davide: Jashobeam, figliuolo di un Hakmonita, capo dei principali ufficiali; egli impugnò la lancia contro trecento uomini, che uccise in un solo scontro.
\par 12 Dopo di lui veniva Eleazar, figliuolo di Dodo, lo Ahohita, uno dei tre valorosi guerrieri.
\par 13 Egli era con Davide a Pas-Dammin, dove i Filistei s'erano raunati per combattere. V'era quivi un campo pieno d'orzo; e il popolo fuggiva dinanzi ai Filistei.
\par 14 Ma quelli si piantarono in mezzo al campo, lo difesero e sconfissero i Filistei; e l'Eterno diede una gran vittoria.
\par 15 Tre dei trenta capi scesero sulla roccia, presso Davide, nella spelonca di Adullam, mentre l'esercito dei Filistei era accampato nella valle di Refaim.
\par 16 Davide era allora nella fortezza, e c'era un posto di Filistei a Bethlehem.
\par 17 Davide ebbe un desiderio, e disse: 'Oh se qualcuno mi desse da bere dell'acqua del pozzo ch'è vicino alla porta di Bethlehem!'
\par 18 E quei tre s'aprirono un varco attraverso al campo filisteo, attinsero dell'acqua dal pozzo di Bethlehem, vicino alla porta; e, presala seco, la presentarono a Davide; il quale però non ne volle bere, ma la sparse davanti all'Eterno,
\par 19 dicendo: 'Mi guardi Iddio dal far tal cosa! Beverei io il sangue di questi uomini, che sono andati là a rischio della loro vita? Perché l'han portata a rischio della loro vita'. E non la volle bere. Questo fecero quei tre prodi.
\par 20 Abishai, fratello di Joab, fu il capo di altri tre. Egli impugnò la lancia contro trecento uomini, e li uccise; e fu famoso fra i tre.
\par 21 Fu il più illustre dei tre della seconda serie, e fu fatto loro capo; nondimeno non giunse ad eguagliare i primi tre.
\par 22 Poi veniva Benaia, figliuolo di Jehoiada, figliuolo di un uomo da Kabtseel, valoroso, e celebre per le sue prodezze. Egli uccise i due grandi eroi di Moab. Discese anche in mezzo a una cisterna, dove uccise un leone, un giorno di neve.
\par 23 Uccise pure un Egiziano di statura enorme, alto cinque cubiti, che teneva in mano una lancia grossa come un subbio da tessitore; ma Benaia gli scese contro con un bastone, strappò di mano all'Egiziano la lancia, e se ne servì per ucciderlo.
\par 24 Questo fece Benaia, figliuolo di Jehoiada; e fu famoso fra i tre prodi.
\par 25 Fu il più illustre dei trenta; nondimeno non giunse ad eguagliare i primi tre. E Davide lo ammise nel suo consiglio.
\par 26 Poi v'erano questi uomini, forti e valorosi: Asael, fratello di Joab; Elhanan, figliuolo di Dodo da Bethlehem;
\par 27 Shammoth da Haror; Helets da Palon;
\par 28 Ira, figliuolo di Ikkesh, da Tekoa; Abiezer da Anatoth;
\par 29 Sibbecai da Husha;
\par 30 Ilai da Ahoa; Maharai da Netofa; Heled, figliuolo di Baana, da Netofa;
\par 31 Ithai, figliuolo di Ribai, da Ghibea dei figliuoli di Beniamino; Benaia da Pirathon;
\par 32 Hurai da Nahale-Gaash; Abiel da Arbath;
\par 33 Azmaveth da Baharum; Eliahba da Shaalbon;
\par 34 Bene-Hascem da Ghizon; Jonathan, figliuolo di Shaghé, da Harar;
\par 35 Hahiam, figliuolo di Sacar, da Harar; Elifal, figliuolo di Ur;
\par 36 Hefer da Mekera; Ahija da Palon;
\par 37 Hetsro da Carmel; Naarai, figliuolo di Ezbai;
\par 38 Joel, fratello di Nathan; Mibhar, figliuolo di Hagri;
\par 39 Tselek, l'Ammonita; Naharai da Beroth, scudiero di Joab figliuolo di Tseruia.
\par 40 Ira da Jether; Gareb da Jether;
\par 41 Uria, lo Hitteo; Zabad, figliuolo di Ahlai;
\par 42 Adina, figliuolo di Sciza, il Rubenita, capo dei Rubeniti, e altri trenta con lui.
\par 43 Hanan, figliuolo di Maaca; Joshafat da Mithni;
\par 44 Uzzia da Ashtaroth; Shama e Jeiel, figliuoli di Hotham, da Aroer;
\par 45 Jediael, figliuolo di Scimri; Joha, suo fratello, il Titsita;
\par 46 Eliel da Mahavim; Jeribai e Joshavia, figliuoli di Elnaam; Jthma, il Moabita;
\par 46 Eliel da Mahavim; Jeribai e Joshavia, figliuoli di Elnaam; Jthma, il Moabita;

\chapter{12}

\par 1 Or questi son quelli che vennero a Davide a Tsiklag, mentr'egli era ancora fuggiasco per tema di Saul, figliuolo di Kis; essi facean parte dei prodi che gli prestarono aiuto durante la guerra.
\par 2 Erano armati d'arco, abili a scagliar sassi ed a tirar frecce tanto con la destra quanto con la sinistra; erano della tribù di Beniamino, de' fratelli di Saul.
\par 3 Il capo Ahiezer e Joas, figliuoli di Scemaa, da Ghibea; Jeziel e Pelet, figliuoli di Azmaveth; Beraca; Jehu da Anathoth;
\par 4 Jshmaia da Gabaon, valoroso fra i trenta e capo di trenta; Geremia; Jahaziel; Johanan; Jozabad da Ghedera;
\par 5 Eluzai; Jerimoth; Bealia; Scemaria; Scefatia da Harun;
\par 6 Elkana; Jscia; Azareel; Joezer e Jashobeam, Koraiti;
\par 7 Joela e Zebadia, figliuoli di Jeroham, da Ghedor.
\par 8 Fra i Gaditi degli uomini partirono per recarsi da Davide nella fortezza del deserto: erano uomini forti e valorosi, esercitati alla guerra, che sapevan maneggiare scudo e lancia: dalle facce leonine, e veloci come gazzelle sui monti.
\par 9 Ezer era il capo; Obadia, il secondo; Eliab, il terzo;
\par 10 Mishmanna, il quarto; Geremia, il quinto;
\par 11 Attai, il sesto; Eliel, il settimo;
\par 12 Johanan, l'ottavo; Elzabad, il nono;
\par 13 Geremia, il decimo; Macbannai, l'undecimo.
\par 14 Questi erano dei figliuoli di Gad, capi dell'esercito; il minimo tenea fronte a cento; il maggiore, a mille.
\par 15 Questi son quelli che passarono il Giordano il primo mese quand'era straripato da per tutto, e misero in fuga tutti gli abitanti delle valli, a oriente e ad occidente.
\par 16 Anche dei figliuoli di Beniamino e di Giuda vennero a Davide, nella fortezza.
\par 17 Davide uscì loro incontro, e si rivolse a loro, dicendo: 'Se venite da me con buon fine per soccorrermi, il mio cuore sarà unito col vostro; ma se venite per tradirmi e darmi nelle mani de' miei avversari, mentre io non commetto alcuna violenza, l'Iddio dei nostri padri lo vegga, e faccia egli giustizia!'
\par 18 Allora lo spirito investì Amasai, capo dei trenta, che esclamò: 'Noi siamo tuoi, o Davide; e siam con te, o figliuolo d'Isai! Pace, pace a te, e pace a quei che ti soccorrono, poiché il tuo Dio ti soccorre!' Allora Davide li accolse, e li fece capi delle sue schiere.
\par 19 Anche degli uomini di Manasse passarono a Davide, quando questi andò coi Filistei a combattere contro Saul; ma Davide e i suoi uomini non furono d'alcun aiuto ai Filistei; giacché i principi dei Filistei, dopo essersi consultati, rimandarono Davide, dicendo: 'Egli passerebbe dalla parte del suo signore Saul, a prezzo delle nostre teste'.
\par 20 Quand'egli tornò a Tsiklag, questi furono quelli di Manasse, che passarono a lui: Adna, Jozabad, Jediael, Micael, Jozabad, Elihu, Tsilletai, capi di migliaia nella tribù di Manasse.
\par 21 Questi uomini diedero aiuto a Davide contro le bande dei predoni, perché erano tutti uomini forti e valorosi; e furon fatti capi nell'esercito.
\par 22 E ogni giorno veniva gente a Davide per soccorrerlo: tanta, che se ne formò un esercito grande come un esercito di Dio.
\par 23 Questo è il numero degli uomini armati per la guerra, che si recarono da Davide a Hebron per trasferire a lui la potestà reale di Saul, secondo l'ordine dell'Eterno.
\par 24 Figliuoli di Giuda, che portavano scudo e lancia, seimila ottocento, armati per la guerra.
\par 25 De' figliuoli di Simeone, uomini forti e valorosi in guerra, settemila cento.
\par 26 Dei figliuoli di Levi, quattromila seicento;
\par 27 e Jehoiada, principe della famiglia d'Aaronne, e con lui tremila settecento uomini;
\par 28 e Tsadok, giovine forte e valoroso, e la sua casa patriarcale, che contava ventidue capi.
\par 29 Dei figliuoli di Beniamino, fratelli di Saul, tremila; poiché la maggior parte d'essi fino allora era rimasta fedele alla casa di Saul.
\par 30 Dei figliuoli d'Efraim, ventimila ottocento: uomini forti e valorosi, gente di gran nome, divisi secondo le loro case patriarcali.
\par 31 Della mezza tribù di Manasse, diciottomila che furono designati nominatamente, per andare a proclamare re Davide.
\par 32 Dei figliuoli d'Issacar, che intendevano i tempi, in modo da sapere quel che Israele dovea fare, duecento capi, e tutti i loro fratelli sotto i loro ordini.
\par 33 Di Zabulon, cinquantamila, atti a servire, forniti per il combattimento di tutte le armi da guerra, e pronti ad impegnar l'azione con cuore risoluto.
\par 34 Di Neftali, mille capi, e con essi trentasettemila uomini armati di scudo e lancia.
\par 35 Dei Daniti, armati per la guerra, ventottomila seicento.
\par 36 Di Ascer, atti a servire, e pronti a ordinarsi in battaglia, quarantamila.
\par 37 E di là dal Giordano, dei Rubeniti, dei Gaditi e della mezza tribù di Manasse, forniti per il combattimento di tutte le armi da guerra, centoventimila.
\par 38 Tutti questi uomini, gente di guerra, pronti a ordinarsi in battaglia, giunsero a Hebron, con sincerità di cuore, per proclamare Davide re sopra tutto Israele; e anche tutto il rimanente d'Israele era unanime per fare re Davide.
\par 39 Essi rimasero quivi tre giorni con Davide a mangiare e a bere, perché i loro fratelli avean preparato per essi dei viveri.
\par 40 E anche quelli ch'eran loro vicini, e perfino gente da Issacar, da Zabulon e da Neftali, portavan dei viveri sopra asini, sopra cammelli, sopra muli e su buoi: farina, fichi secchi, uva secca, vino, olio, buoi e pecore in abbondanza; perché v'era gioia in Israele.

\chapter{13}

\par 1 Davide tenne consiglio coi capi di migliaia e di centinaia, cioè con tutti i principi del popolo,
\par 2 poi disse a tutta la raunanza d'Israele: 'Se vi par bene, e se l'Eterno, il nostro Dio, l'approva, mandiamo da per tutto a dire ai nostri fratelli che son rimasti in tutte le regioni d'Israele, e così pure ai sacerdoti ed ai Leviti nelle loro città e nei loro contadi, che si uniscano a noi;
\par 3 e riconduciamo qui da noi l'arca del nostro Dio; poiché non ce ne siamo occupati ai tempi di Saul'.
\par 4 E tutta la raunanza rispose che si facesse così, giacché la cosa parve buona agli occhi di tutto il popolo.
\par 5 Davide dunque radunò tutto Israele dallo Scihor d'Egitto fino all'ingresso di Hamath, per ricondurre l'arca di Dio da Kiriath-Jearim.
\par 6 E Davide, con tutto Israele, salì verso Baala, cioè verso Kiriath-Jearim, che appartiene a Giuda, per trasferire di là l'arca di Dio, dinanzi alla quale è invocato il nome dell'Eterno, che siede sovr'essa fra i cherubini.
\par 7 E posero l'arca di Dio sopra un carro nuovo, levandola dalla casa di Abinadab; e Uzza ed Ahio conducevano il carro.
\par 8 Davide e tutto Israele danzavano dinanzi a Dio a tutto potere, cantando e sonando cetre, saltèri, timpani, cembali e trombe.
\par 9 Or come furon giunti all'aia di Kidon, Uzza stese la mano per reggere l'arca, perché i buoi la facevano piegare.
\par 10 E l'ira dell'Eterno s'accese contro Uzza, e l'Eterno lo colpì per avere stesa la mano sull'arca; e quivi Uzza morì dinanzi a Dio.
\par 11 Davide si attristò perché l'Eterno avea fatto una breccia nel popolo, colpendo Uzza; e quel luogo è stato chiamato Perets-Uzza fino al dì d'oggi.
\par 12 E Davide in quel giorno, ebbe paura di Dio, e disse: 'Come farò a portare a casa mia l'arca di Dio?'
\par 13 E Davide non ritirò l'arca presso di sé, nella città di Davide, ma la fece portare in casa di Obed-Edom di Gath.
\par 14 E l'arca di Dio rimase tre mesi dalla famiglia di Obed-Edom, in casa di lui; e l'Eterno benedisse la casa di Obed-Edom e tutto quello che gli apparteneva.

\chapter{14}

\par 1 Hiram, re di Tiro, inviò a Davide de' messi, del legname di cedro, dei muratori e dei legnaiuoli, per edificargli una casa.
\par 2 Allora Davide riconobbe che l'Eterno lo stabiliva saldamente come re d'Israele, giacché la sua dignità reale era grandemente esaltata per amore d'Israele, del popolo di Dio.
\par 3 Davide si prese ancora delle mogli a Gerusalemme, e generò ancora figliuoli e figliuole.
\par 4 Questi sono i nomi dei figliuoli che gli nacquero a Gerusalemme: Shammua, Shobab, Nathan, Salomone,
\par 5 Jbhar, Elishua, Elpelet,
\par 6 Noga, Nefeg, Jafia,
\par 7 Elishama, Beeliada ed Elifelet.
\par 8 Or quando i Filistei ebbero udito che Davide era stato unto re di tutto Israele, saliron tutti in cerca di lui; e Davide, saputolo, uscì loro incontro.
\par 9 I Filistei giunsero e si sparsero per la valle dei Refaim.
\par 10 Allora Davide consultò Dio, dicendo: 'Salirò io contro i Filistei? E me li darai tu nelle mani?' L'Eterno gli rispose: 'Sali, e io li darò nelle tue mani'.
\par 11 I Filistei dunque salirono a Baal-Peratsim, dove Davide li sconfisse, e disse: 'Iddio ha rotto i miei nemici per mano mia come quando le acque rompono le dighe'. Perciò fu dato a quel luogo il nome di Baal-Peratsim.
\par 12 I Filistei lasciaron quivi i loro dèi, che per ordine di Davide, furon dati alle fiamme.
\par 13 Di poi i Filistei tornarono a spargersi per quella valle.
\par 14 E Davide consultò di nuovo Dio; e Dio gli disse: 'Non salire dietro ad essi, allontanati e gira intorno a loro, e giungerai su di essi dal lato dei Gelsi.
\par 15 E quando udrai un rumor di passi tra le vette dei gelsi, esci subito all'attacco, perché Dio marcerà alla tua testa per sconfiggere l'esercito dei Filistei'.
\par 16 Davide fece come Dio gli avea comandato, e gl'Israeliti sconfissero l'esercito dei Filistei da Gabaon a Ghezer.
\par 17 E la fama di Davide si sparse per tutti i paesi, e l'Eterno fece sì ch'egli incutesse spavento a tutte le genti.

\chapter{15}

\par 1 Davide si costruì delle case nella città di Davide; preparò un luogo per l'arca di Dio, e drizzò una tenda per essa.
\par 2 Allora Davide disse: 'Nessuno deve portare l'arca di Dio tranne i Leviti; perché l'Eterno ha scelti loro per portare l'arca di Dio, e per esser suoi ministri in perpetuo'.
\par 3 E Davide convocò tutto Israele a Gerusalemme per trasportar l'arca dell'Eterno al luogo ch'egli le avea preparato.
\par 4 Davide radunò pure i figliuoli d'Aaronne ed i Leviti:
\par 5 dei figliuoli di Kehath, Uriel, il capo, e i suoi fratelli: centoventi;
\par 6 dei figliuoli di Merari, Asaia, il capo, e i suoi fratelli: duecentoventi;
\par 7 dei figliuoli di Ghershom, Joel, il capo, e i suoi fratelli: centotrenta;
\par 8 dei figliuoli di Elitsafan, Scemaia, il capo, e i suoi fratelli: duecento;
\par 9 dei figliuoli di Hebron, Eliel, il capo, e i suoi fratelli: ottanta;
\par 10 dei figliuoli di Uzziel, Amminadab, il capo, e i suoi fratelli: centododici.
\par 11 Poi Davide chiamò i sacerdoti Tsadok e Abiathar, e i Leviti Uriel, Asaia, Joel, Scemaia, Eliel e Amminadab,
\par 12 e disse loro: 'Voi siete i capi delle case patriarcali dei Leviti; santificatevi, voi e i vostri fratelli, affinché possiate trasportar l'arca dell'Eterno, dell'Iddio d'Israele, nel luogo che io le ho preparato.
\par 13 Siccome voi non c'eravate la prima volta, l'Eterno, il nostro Dio, fece una breccia fra noi, perché non lo cercammo secondo le regole stabilite'.
\par 14 I sacerdoti e i Leviti dunque si santificarono per trasportare l'arca dell'Eterno, dell'Iddio d'Israele.
\par 15 E i figliuoli dei Leviti portarono l'arca di Dio sulle loro spalle, per mezzo di stanghe, come Mosè aveva ordinato, secondo la parola dell'Eterno.
\par 16 E Davide ordinò ai capi dei Leviti che chiamassero i loro fratelli cantori a prestar servizio coi loro strumenti musicali, saltèri, cetre e cembali, da cui trarrebbero suoni vigorosi, in segno di gioia.
\par 17 I Leviti dunque chiamarono a prestar servizio Heman, figliuolo di Joel; e fra i suoi fratelli, Asaf, figliuolo di Berekia; tra i figliuoli di Merari, loro fratelli, Ethan, figliuolo di Kushaia.
\par 18 Con loro, furon chiamati i loro fratelli del secondo ordine: Zaccaria, Ben, Jaaziel, Scemiramoth, Jehiel, Unni, Eliab, Benaia, Maaseia, Mattithia, Elifalehu, Mikneia, Obed-Edom e Jeiel, i portinai.
\par 19 I cantori Heman, Asaf ed Ethan, aveano dei cembali di rame per sonare;
\par 20 Zaccaria, Aziel, Scemiramoth, Jehiel, Unni, Eliab, Maaseia e Benaia avean dei saltèri per accompagnare voci di fanciulle;
\par 21 Mattithia, Elifalehu, Mikneia, Obed-Edom, Jeiel ed Azazia sonavano con cetre all'ottava, per guidare il canto;
\par 22 Kenania, capo dei Leviti, era preposto al canto; dirigeva la musica, perché era competente in questo.
\par 23 Berekia ed Elkana erano portinai dell'arca.
\par 24 Scebania, Joshafat, Nethaneel, Amasai, Zaccaria, Benaia ed Eliezer, sacerdoti, sonavano la tromba davanti all'arca di Dio; e Obed-Edom e Jehija erano portinai dell'arca.
\par 25 Davide, gli anziani d'Israele e i capi di migliaia si misero in cammino per trasportare l'arca del patto dell'Eterno dalla casa di Obed-Edom, con gaudio.
\par 26 E poiché Dio prestò assistenza ai Leviti che portavan l'arca del patto dell'Eterno, fu offerto un sacrifizio di sette giovenchi e di sette montoni.
\par 27 Davide indossava un manto di lino fino, come anche tutti i Leviti che portavano l'arca, i cantori, e Kenania, capo musica fra i cantori; e Davide avea sul manto un efod di lino.
\par 28 Così tutto Israele portò su l'arca del patto dell'Eterno con grida di gioia, a suon di corni, di trombe, di cembali, di saltèri e d'arpe.
\par 29 E come l'arca del patto dell'Eterno giunse alla città di Davide, Mical, figliuola di Saul, guardava dalla finestra: e vedendo il re Davide che danzava e saltava, lo sprezzò in cuor suo.

\chapter{16}

\par 1 Portarono dunque l'arca di Dio e la collocarono in mezzo al padiglione che Davide aveva rizzato per lei; e si offrirono olocausti e sacrifizi di azioni di grazie dinanzi a Dio.
\par 2 E quando Davide ebbe finito d'offrire gli olocausti e i sacrifizi di azioni di grazie, benedisse il popolo nel nome dell'Eterno;
\par 3 e distribuì a tutti gl'Israeliti, uomini e donne, un pane per uno, una porzione di carne, e un dolce d'uva secca.
\par 4 Poi stabilì davanti all'arca dell'Eterno alcuni di fra i Leviti per fare il servizio per ringraziare, lodare e celebrare l'Eterno, l'Iddio d'Israele.
\par 5 Erano: Asaf, il capo; Zaccaria, il secondo dopo di lui; poi Jehiel, Scemiramoth, Jehiel, Mattithia, Eliab, Benaia, Obed-Edom ed Jeiel. Essi sonavano saltèri e cetre, e Asaf sonava i cembali;
\par 6 i sacerdoti Benaia e Jahaziel sonavano del continuo la tromba davanti all'arca del patto di Dio.
\par 7 Allora, in quel giorno, Davide diede per la prima volta ad Asaf e ai suoi fratelli l'incarico di cantare le lodi dell'Eterno:
\par 8 "Celebrate l'Eterno, invocate il suo nome; fate conoscere le sue gesta fra i popoli.
\par 9 Cantategli, salmeggiategli, meditate su tutte le sue maraviglie.
\par 10 Gloriatevi nel santo suo nome; si rallegri il cuore di quelli che cercano l'Eterno!
\par 11 Cercate l'Eterno e la sua forza, cercate del continuo la sua faccia!
\par 12 Ricordatevi delle maraviglie ch'egli ha fatte, de' suoi miracoli e dei giudizi della sua bocca,
\par 13 o voi, progenie d'Israele, suo servitore, figliuoli di Giacobbe, suoi eletti!
\par 14 Egli, l'Eterno, è l'Iddio nostro; i suoi giudizi s'esercitano su tutta la terra.
\par 15 Ricordatevi in perpetuo del suo patto, della parola da lui data per mille generazioni,
\par 16 del patto che fece con Abrahamo, che giurò ad Isacco,
\par 17 e che confermò a Giacobbe come uno statuto, ad Israele come un patto eterno,
\par 18 dicendo: 'Io ti darò il paese di Canaan per vostra parte di eredità'.
\par 19 Non erano allora che poca gente, pochissimi e stranieri nel paese,
\par 20 e andavano da una nazione all'altra, da un regno a un altro popolo.
\par 21 Egli non permise che alcuno li opprimesse; anzi, castigò dei re per amor loro,
\par 22 dicendo: 'Non toccate i miei unti, e non fate alcun male ai miei profeti'.
\par 23 Cantate all'Eterno, abitanti di tutta la terra, annunziate di giorno in giorno la sua salvezza!
\par 24 Raccontate la sua gloria fra le nazioni e le sue maraviglie fra tutti i popoli!
\par 25 Perché l'Eterno è grande e degno di sovrana lode; egli è tremendo sopra tutti gli dèi.
\par 26 Poiché tutti gli dèi dei popoli son idoli vani, ma l'Eterno ha fatto i cieli.
\par 27 Splendore e maestà stanno dinanzi a lui, forza e gioia sono nella sua dimora.
\par 28 Date all'Eterno, o famiglie dei popoli, date all'Eterno gloria e forza.
\par 29 Date all'Eterno la gloria dovuta al suo nome, portategli offerte e venite in sua presenza. Prostratevi dinanzi all'Eterno vestiti di sacri ornamenti,
\par 30 tremate dinanzi a lui, o abitanti di tutta la terra! Il mondo è stabile e non sarà smosso.
\par 31 Si rallegrino i cieli e gioisca la terra; dicasi fra le nazioni: 'L'Eterno regna'.
\par 32 Risuoni il mare e quel ch'esso contiene; festeggi la campagna e tutto quello ch'è in essa.
\par 33 Gli alberi delle foreste dian voci di gioia nel cospetto dell'Eterno, poich'egli viene a giudicare la terra.
\par 34 Celebrate l'Eterno, perch'egli è buono, perché la sua benignità dura in perpetuo.
\par 35 E dite: 'Salvaci, o Dio della nostra salvezza! Raccoglici di fra le nazioni e liberaci, affinché celebriamo il tuo santo nome e mettiamo la nostra gloria nel lodarti'.
\par 36 Benedetto sia l'Eterno, l'Iddio d'Israele, d'eternità in eternità!" E tutto il popolo disse: 'Amen', e lodò l'Eterno.
\par 37 Poi Davide lasciò quivi, davanti all'arca del patto dell'Eterno, Asaf e i suoi fratelli perché fossero del continuo di servizio davanti all'arca, secondo i bisogni d'ogni giorno.
\par 38 Lasciò Obed-Edom e Hosa e i loro fratelli, in numero di sessantotto: Obed-Edom, figliuolo di Jeduthun, e Hosa, come portieri.
\par 39 Lasciò pure il sacerdote Tsadok e i sacerdoti suoi fratelli davanti al tabernacolo dell'Eterno, sull'alto luogo che era a Gabaon,
\par 40 perché offrissero del continuo all'Eterno olocausti, mattina e sera, sull'altare degli olocausti, ed eseguissero tutto quello che sta scritto nella legge data dall'Eterno ad Israele.
\par 41 E con essi erano Heman, Jeduthun, e gli altri ch'erano stati scelti e designati nominatamente per lodare l'Eterno, perché la sua benignità dura in perpetuo.
\par 42 Heman e Jeduthun eran con essi, con trombe e cembali per i musici, e con degli strumenti per i cantici in lode di Dio. I figliuoli di Jeduthun erano addetti alla porta.
\par 43 Tutto il popolo se ne andò, ciascuno a casa sua, e Davide se ne ritornò per benedire la propria casa.

\chapter{17}

\par 1 Or avvenne che Davide quando si fu stabilito nella sua casa, disse al profeta Nathan: 'Ecco, io abito in una casa di cedro, e l'arca del patto dell'Eterno sta sotto una tenda'.
\par 2 Nathan rispose a Davide: 'Fa' tutto quello che hai in cuore di fare, poiché Dio è teco'.
\par 3 Ma quella stessa notte la parola di Dio fu diretta a Nathan in questi termini:
\par 4 'Va' e di' al mio servo Davide: Così dice l'Eterno: - Non sarai tu quegli che mi edificherà una casa perch'io vi dimori;
\par 5 poiché io non ho abitato in una casa, dal giorno che trassi Israele dall'Egitto, fino al dì d'oggi; ma sono andato di tenda in tenda, di dimora in dimora.
\par 6 Dovunque sono andato, or qua or là, in mezzo a tutto Israele, ho io mai fatto parola a qualcuno dei giudici d'Israele ai quali avevo comandato di pascere il mio popolo, dicendogli: Perché non mi edificate una casa di cedro?
\par 7 Ora dunque parlerai così al mio servo Davide: Così dice l'Eterno degli eserciti: Io ti presi dall'ovile, di dietro alle pecore, perché tu fossi il principe d'Israele, mio popolo;
\par 8 e sono stato teco dovunque sei andato, ho sterminato dinanzi a te tutti i tuoi nemici, e ho reso il tuo nome grande come quello dei grandi che son sulla terra;
\par 9 ho assegnato un posto ad Israele, mio popolo, e ve l'ho piantato perché abiti in casa sua e non sia più agitato, né seguitino gl'iniqui a farne scempio come prima,
\par 10 e fin dal tempo in cui avevo stabilito dei giudici sul mio popolo d'Israele. Io ho umiliato tutti i tuoi nemici; e t'annunzio che l'Eterno ti fonderà una casa.
\par 11 Quando i tuoi giorni saranno compiuti e tu te n'andrai a raggiungere i tuoi padri, io innalzerò al trono dopo di te la tua progenie, uno de' tuoi figliuoli, e stabilirò saldamente il suo regno.
\par 12 Egli mi edificherà una casa, ed io renderò stabile in perpetuo il suo trono.
\par 13 Io sarò per lui un padre, ed egli mi sarà figliuolo; e non gli ritirerò la mia grazia, come l'ho ritirata da colui che t'ha preceduto.
\par 14 Io lo renderò saldo per sempre nella mia casa e nel mio regno, e il suo trono sarà reso stabile in perpetuo'.
\par 15 Nathan parlò a Davide, secondo tutte queste parole e secondo tutta questa visione.
\par 16 Allora il re Davide andò a presentarsi davanti all'Eterno, e disse: 'Chi son io, o Eterno Iddio, e che è la mia casa, che tu m'abbia fatto arrivare fino a questo punto?
\par 17 E questo è parso ancora poca cosa agli occhi tuoi, o Dio; e tu hai parlato anche della casa del tuo servo per un lontano avvenire, e hai degnato considerar me come se fossi uomo d'alto grado, o Eterno Iddio.
\par 18 Che potrebbe Davide dirti di più riguardo all'onore ch'è fatto al tuo servo? Tu conosci il tuo servo.
\par 19 O Eterno, per amor del tuo servo e seguendo il cuor tuo, hai compiuto tutte queste grandi cose per rivelargli tutte le tue maraviglie.
\par 20 O Eterno, nessuno è pari a te, e non v'è altro Dio fuori di te, secondo tutto quello che abbiamo udito coi nostri orecchi.
\par 21 E qual popolo è come il tuo popolo d'Israele, l'unica nazione sulla terra che Dio sia venuto a redimere per formarne il suo popolo, per farti un nome e per compiere cose grandi e tremende, cacciando delle nazioni d'innanzi al tuo popolo che tu hai redento dall'Egitto?
\par 22 Tu hai fatto del tuo popolo d'Israele il popolo tuo speciale in perpetuo; e tu, o Eterno, sei divenuto il suo Dio.
\par 23 Or dunque, o Eterno, la parola che tu hai pronunziata riguardo al tuo servo ed alla sua casa rimanga stabile in perpetuo, e fa' come tu hai detto.
\par 24 Sì, rimanga stabile, affinché il tuo nome sia magnificato in perpetuo, e si dica: L'Eterno degli eserciti, l'Iddio d'Israele, è veramente un Dio per Israele; e la casa del tuo servo Davide sia stabile dinanzi a te!
\par 25 Poiché tu stesso, o mio Dio, hai rivelato al tuo servo di volergli fondare una casa. Perciò il tuo servo ha preso l'ardire di rivolgerti questa preghiera.
\par 26 Ed ora, o Eterno, tu sei Dio, e hai promesso questo bene al tuo servo;
\par 27 piacciati dunque benedire ora la casa del tuo servo, affinch'ella sussista in perpetuo dinanzi a te! Poiché ciò che tu benedici, o Eterno, è benedetto in perpetuo'.

\chapter{18}

\par 1 Dopo queste cose, Davide sconfisse i Filistei e li umiliò, e tolse di mano ai Filistei Gath e le città che ne dipendevano.
\par 2 Sconfisse pure i Moabiti; e i Moabiti divennero sudditi e tributari di Davide.
\par 3 Davide sconfisse anche Hadarezer, re di Tsoba, verso Hamath, mentr'egli andava a stabilire il suo dominio sul fiume Eufrate.
\par 4 Davide gli prese mille carri, settemila cavalieri e ventimila pedoni; tagliò i garetti a tutti i cavalli da tiro, ma riserbò de' cavalli per cento carri.
\par 5 E quando i Sirî di Damasco vennero per soccorrere Hadarezer, re di Tsoba, Davide ne uccise ventiduemila.
\par 6 Poi Davide mise delle guarnigioni nella Siria di Damasco, e i Sirî divennero sudditi e tributari di Davide; e l'Eterno lo rendea vittorioso dovunque egli andava.
\par 7 E Davide tolse ai servi di Hadarezer i loro scudi d'oro e li portò a Gerusalemme.
\par 8 Davide prese anche una grande quantità di rame a Tibhath e a Cun, città di Hadarezer. Salomone se ne servì per fare il mar di rame, le colonne e gli utensili di rame.
\par 9 Or quando Tou, re di Hamath, ebbe udito che Davide avea sconfitto tutto l'esercito di Hadarezer, re di Tsoba,
\par 10 mandò al re Davide Hadoram, suo figliuolo, per salutarlo e per benedirlo perché avea mosso guerra a Hadarezer e l'avea sconfitto (Hadarezer era sempre in guerra con Tou); e Hadoram portò seco ogni sorta di vasi d'oro, d'argento, e di rame.
\par 11 E il re Davide consacrò anche quelli all'Eterno, come avea già consacrato l'argento e l'oro che avea portato via a tutte le nazioni: agli Edomiti, ai Moabiti, agli Ammoniti, ai Filistei ed agli Amalekiti.
\par 12 Abishai, figliuolo di Tseruia, sconfisse pure diciottomila Edomiti nella valle del Sale.
\par 13 E pose delle guarnigioni in Idumea, e tutti gli Edomiti divennero sudditi di Davide; e l'Eterno rendea Davide vittorioso dovunque egli andava.
\par 14 Davide regnò su tutto Israele, facendo ragione e amministrando la giustizia a tutto il suo popolo.
\par 15 Joab, figliuolo di Tseruia, comandava l'esercito; Giosafat, figliuolo di Ahilud, era cancelliere;
\par 16 Tsadok, figliuolo di Ahitub, e Abimelec, figliuolo di Abiathar, erano sacerdoti; Shavsa era segretario;
\par 17 Benaia, figliuolo di Jehoiada, era capo dei Kerethei e dei Pelethei; e i figliuoli di Davide erano i primi al fianco del re.

\chapter{19}

\par 1 Or avvenne, dopo queste cose, che Nahash, re dei figliuoli di Ammon, morì, e il suo figliuolo regnò in luogo di lui.
\par 2 Davide disse: 'Io voglio usare benevolenza verso Hanun, figliuolo di Nahash, perché suo padre ne usò verso di me'. E Davide inviò dei messi a consolarlo della perdita del padre. Ma quando i servi di Davide furon giunti nel paese dei figliuoli di Ammon presso Hanun per consolarlo,
\par 3 i principi de' figliuoli di Ammon dissero ad Hanun: 'Credi tu che Davide t'abbia mandato dei consolatori per onorar tuo padre? I suoi servi non son eglino piuttosto venuti per esplorare la città e distruggerla e per spiare il paese?'
\par 4 Allora Hanun prese i servi di Davide, li fece radere e fece lor tagliare la metà delle vesti fino alle natiche, poi li rimandò.
\par 5 Intanto vennero alcuni ad informar Davide del modo con cui quegli uomini erano stati trattati; e Davide mandò gente ad incontrarli, perch'essi erano oltremodo confusi. E il re fece dir loro: 'Restate a Gerico finché vi sia ricresciuta la barba, poi tornerete'.
\par 6 I figliuoli di Ammon videro che s'erano attirati l'odio di Davide; e Hanun e gli Ammoniti mandarono mille talenti d'argento per prendere al loro soldo dei carri e dei cavalieri presso i Sirî di Mesopotamia e presso i Sirî di Maaca e di Tsoba.
\par 7 E presero al loro soldo trentaduemila carri e il re di Maaca col suo popolo, i quali vennero ad accamparsi dirimpetto a Medeba. E i figliuoli di Ammon si raunarono dalle loro città, per andare a combattere.
\par 8 Quando Davide udì questo, inviò contro di loro Joab e tutto l'esercito degli uomini di valore.
\par 9 I figliuoli di Ammon uscirono e si disposero in ordine di battaglia alla porta della città; e i re ch'erano venuti in loro soccorso stavano a parte nella campagna.
\par 10 Or come Joab vide che quelli eran pronti ad attaccarlo di fronte e alle spalle, scelse un corpo fra gli uomini migliori d'Israele, lo dispose in ordine di battaglia contro i Sirî,
\par 11 e mise il resto del popolo sotto gli ordini del suo fratello Abishai, che li dispose di fronte ai figliuoli di Ammon;
\par 12 e disse ad Abishai: 'Se i Sirî son più forti di me, tu mi darai soccorso; e se i figliuoli di Ammon son più forti di te, andrò io a soccorrerti.
\par 13 Abbi coraggio, e dimostriamoci forti per il nostro popolo e per le città del nostro Dio; e faccia l'Eterno quello che a lui piacerà'.
\par 14 Poi Joab, con la gente che avea seco, s'avanzò per attaccare i Sirî, i quali fuggirono d'innanzi a lui.
\par 15 E come i figliuoli di Ammon videro che i Sirî eran fuggiti, fuggirono anch'essi d'innanzi ad Abishai, fratello di Joab, e rientrarono nella città. Allora Joab se ne tornò a Gerusalemme.
\par 16 I Sirî, vedendosi sconfitti da Israele, inviarono de' messi e fecero venire i Sirî che abitavano di là dal fiume. Shofac, capo dell'esercito di Hadarezer, era alla loro testa.
\par 17 E la cosa fu riferita a Davide, che radunò tutto Israele, passò il Giordano, marciò contro di loro e si dispose in ordine di battaglia contro ad essi. E come Davide si fu disposto in ordine di battaglia contro i Sirî, questi impegnarono l'azione con lui.
\par 18 Ma i Sirî fuggirono d'innanzi a Israele; e Davide uccise ai Sirî gli uomini di settecento carri e quarantamila fanti, e uccise pure Shofac, capo dell'esercito.
\par 19 E quando i servi di Hadarezer si videro sconfitti da Israele, fecero pace con Davide, e furono a lui soggetti. E i Sirî non vollero più recar soccorso ai figliuoli di Ammon.

\chapter{20}

\par 1 Or avvenne che l'anno seguente, nel tempo in cui i re sogliono andare alla guerra, Joab, alla testa di un poderoso esercito, andò a devastare il paese dei figliuoli di Ammon e ad assediare Rabba; ma Davide rimase a Gerusalemme. E Joab batté Rabba e la distrusse.
\par 2 E Davide tolse dalla testa del loro re la corona, e trovò che pesava un talento d'oro e che avea delle pietre preziose; ed essa fu posta sulla testa di Davide. Egli riportò anche dalla città grandissima preda.
\par 3 Fece uscire gli abitanti ch'erano nella città, e li fece a pezzi con delle seghe, degli érpici di ferro e delle scuri. Così fece Davide a tutte le città dei figliuoli di Ammon. Poi Davide se ne tornò a Gerusalemme con tutto il popolo.
\par 4 Dopo queste cose, ci fu una battaglia coi Filistei, a Ghezer; allora Sibbecai di Hushah uccise Sippai, uno dei discendenti di Rafa; e i Filistei furono umiliati.
\par 5 Ci fu un'altra battaglia coi Filistei; ed Elhanan, figliuolo di Jair, uccise Lahmi, fratello di Goliath di Gath, di cui l'asta della lancia era come un subbio da tessitore.
\par 6 Ci fu ancora una battaglia a Gath, dove si trovò un uomo di grande statura, che avea sei dita a ciascuna mano e a ciascun piede, in tutto ventiquattro dita, e che era anch'esso dei discendenti di Rafa.
\par 7 Egli ingiuriò Israele; e Gionathan, figliuolo di Scimea, fratello di Davide, l'uccise.
\par 8 Questi quattro uomini erano nati a Gath, della stirpe di Rafa. Essi perirono per man di Davide e per mano della sua gente.

\chapter{21}

\par 1 Or Satana si levò contro Israele, e incitò Davide a fare il censimento d'Israele.
\par 2 E Davide disse a Joab e ai capi del popolo: 'Andate, fate il censimento degl'Israeliti da Beer-Sceba fino a Dan; e venite a riferirmene il risultato, perch'io ne sappia il numero'.
\par 3 Joab rispose: 'L'Eterno renda il suo popolo cento volte più numeroso di quello che è! Ma, o re, mio signore, non sono eglino tutti servi del mio signore? Perché il mio signore domanda egli questo? Perché render così Israele colpevole?'
\par 4 Ma l'ordine del re prevalse contro Joab. Joab dunque partì, percorse tutto Israele, poi tornò a Gerusalemme.
\par 5 E Joab rimise a Davide la cifra del censimento del popolo: c'erano in tutto Israele un milione e centomila uomini atti a portare le armi; e in Giuda quattrocentosettantamila uomini atti a portar le armi.
\par 6 Or Joab non avea fatto il censimento di Levi e di Beniamino come degli altri, perché l'ordine del re era per lui abominevole.
\par 7 Questa cosa dispiacque a Dio, che perciò colpì Israele.
\par 8 E Davide disse a Dio: 'Io ho gravemente peccato in questo che ho fatto; ma ora, ti prego, perdona l'iniquità del tuo servo, perché io ho agito con grande stoltezza'.
\par 9 E l'Eterno parlò così a Gad, il veggente di Davide:
\par 10 'Va', e parla a Davide in questo modo: - Così dice l'Eterno: Io ti propongo tre cose; sceglitene una, e quella ti farò'.
\par 11 Gad andò dunque da Davide, e gli disse: 'Così dice l'Eterno: Scegli quello che vuoi:
\par 12 o tre anni di carestia, o tre mesi durante i quali i tuoi avversari facciano scempio di te e ti raggiunga la spada dei tuoi nemici, ovvero tre giorni di spada dell'Eterno, ossia di peste nel paese, durante i quali l'angelo dell'Eterno porterà la distruzione in tutto il territorio d'Israele. Or dunque vedi che cosa io debba rispondere a colui che mi ha mandato'.
\par 13 E Davide disse a Gad: 'Io sono in una grande angoscia! Ebbene, ch'io cada nelle mani dell'Eterno, giacché le sue compassioni sono immense; ma ch'io non cada nelle mani degli uomini!'
\par 14 Così l'Eterno mandò la peste in Israele; e caddero settantamila persone d'Israele.
\par 15 E Dio mandò un angelo a Gerusalemme per distruggerla; e come questi si disponeva a distruggerla, l'Eterno gettò su di lei lo sguardo, si pentì della calamità che avea inflitta, e disse all'angelo distruttore: 'Basta; ritieni ora la tua mano!' Or l'angelo dell'Eterno si trovava presso l'aia di Ornan, il Gebuseo.
\par 16 E Davide, alzando gli occhi, vide l'angelo dell'Eterno che stava fra terra e cielo, avendo in mano una spada sguainata, vòlta contro Gerusalemme. Allora Davide e gli anziani, coperti di sacchi, si gettarono con la faccia a terra.
\par 17 E Davide disse a Dio: 'Non sono io quegli che ordinai il censimento del popolo? Son io che ho peccato, e che ho agito con tanta malvagità; ma queste pecore che hanno fatto? Ti prego, o Eterno, o mio Dio, si volga la tua mano contro di me e contro la casa di mio padre, ma non contro il tuo popolo, per colpirlo col flagello!'
\par 18 Allora l'angelo dell'Eterno ordinò a Gad di dire a Davide che salisse ad erigere un altare all'Eterno nell'aia di Ornan, il Gebuseo.
\par 19 E Davide salì, secondo la parola che Gad avea pronunziata nel nome dell'Eterno.
\par 20 Ornan, voltandosi, vide l'angelo; e i suoi quattro figliuoli ch'eran con lui si nascosero. Ornan stava battendo il grano.
\par 21 E come Davide giunse presso Ornan, Ornan guardò, e vide Davide; e, uscito dall'aia, si prostrò dinanzi a Davide, con la faccia a terra.
\par 22 Allora Davide disse ad Ornan: 'Dammi il sito di quest'aia, perch'io vi eriga un altare all'Eterno; dammelo per tutto il prezzo che vale, affinché la piaga cessi d'infierire sul popolo'.
\par 23 Ornan disse a Davide: 'Prenditelo; e il re, mio signore, faccia quello che par bene agli occhi suoi; guarda, io ti do i buoi per gli olocausti, le macchine da trebbiare per legna, e il grano per l'oblazione; tutto ti do'.
\par 24 Ma il re Davide disse ad Ornan: 'No, io comprerò da te queste cose per il loro intero prezzo; giacché io non prenderò per l'Eterno ciò ch'è tuo, né offrirò un olocausto che non mi costi nulla'.
\par 25 E Davide diede ad Ornan come prezzo del luogo il peso di seicento sicli d'oro;
\par 26 poi edificò quivi un altare all'Eterno, offrì olocausti e sacrifizi di azioni di grazie, e invocò l'Eterno, il quale gli rispose mediante il fuoco, che discese dal cielo sull'altare dell'olocausto.
\par 27 Poi l'Eterno comandò all'angelo di rimettere la spada nel fodero.
\par 28 In quel tempo Davide, vedendo che l'Eterno lo aveva esaudito nell'aia d'Ornan, il Gebuseo, vi offriva dei sacrifizi.
\par 29 - Il tabernacolo dell'Eterno che Mosè avea costruito nel deserto e l'altare degli olocausti si trovavano allora sull'alto luogo di Gabaon.
\par 30 E Davide non poteva andare davanti a quell'altare a cercare Iddio, per lo spavento che gli avea cagionato la spada dell'angelo dell'Eterno.

\chapter{22}

\par 1 - E Davide disse: 'Qui sarà la casa di Dio, dell'Eterno, e qui sarà l'altare degli olocausti per Israele'.
\par 2 Davide ordinò che si radunassero gli stranieri che erano nel paese d'Israele, e fissò degli scarpellini per lavorar le pietre da taglio per la costruzione della casa di Dio.
\par 3 Davide preparò pure del ferro in abbondanza per i chiodi per i battenti delle porte e per le commettiture; e una quantità di rame di peso incalcolabile
\par 4 e del legname di cedro da non potersi contare; perché i Sidonî e i Tirî aveano portato a Davide del legname di cedro in abbondanza.
\par 5 Davide diceva: 'Salomone, mio figliuolo, è giovine e di tenera età, e la casa che si deve edificare all'Eterno ha da essere talmente magnifica da salire in fama ed in gloria in tutti i paesi; io voglio dunque far dei preparativi per lui'. Così Davide preparò degli abbondanti materiali, prima di morire.
\par 6 Poi chiamò Salomone, suo figliuolo, e gli ordinò di edificare una casa all'Eterno, all'Iddio d'Israele.
\par 7 Davide disse a Salomone: 'Figliuol mio, io stesso avevo in cuore di edificare una casa al nome dell'Eterno, del mio Dio;
\par 8 ma la parola dell'Eterno mi fu rivolta, e mi fu detto: - Tu hai sparso molto sangue, e hai fatte di gran guerre; tu non edificherai una casa al mio nome, poiché hai sparso molto sangue sulla terra, dinanzi a me.
\par 9 Ma ecco, ti nascerà un figliuolo, che sarà uomo tranquillo, e io gli darò quiete, liberandolo da tutti i suoi nemici d'ogni intorno. Salomone sarà il suo nome; e io darò pace e tranquillità a Israele, durante la vita di lui.
\par 10 Egli edificherà una casa al mio nome; ei mi sarà figliuolo, ed io gli sarò padre; e renderò stabile il trono del suo regno sopra Israele in perpetuo. -
\par 11 Ora, figliuol mio, l'Eterno sia teco, onde tu prosperi, ed edifichi la casa dell'Eterno, del tuo Dio, secondo ch'egli ha detto di te.
\par 12 Sol diati l'Eterno senno e intelligenza, e ti costituisca re d'Israele, per osservare la legge dell'Eterno, del tuo Dio.
\par 13 Allora prospererai, se tu ti applichi a mettere in pratica le leggi e i precetti che l'Eterno prescrisse a Mosè per Israele. Sii forte e fatti animo; non temere e non ti sgomentare.
\par 14 Ora ecco io, colle mie fatiche, ho preparato per la casa dell'Eterno centomila talenti d'oro, un milione di talenti d'argento, e una quantità di rame e di ferro da non potersi pesare, tant'è abbondante; ho pur preparato del legname e delle pietre; e tu ve ne potrai aggiungere ancora.
\par 15 E tu hai presso di te degli operai in abbondanza: degli scarpellini, de' muratori, de' falegnami, e ogni sorta d'uomini esperti in qualunque specie di lavoro.
\par 16 Quanto all'oro, all'argento, al rame, al ferro, ve n'è una quantità incalcolabile. Lèvati dunque, mettiti all'opra, e l'Eterno sia teco!'
\par 17 Davide ordinò pure a tutti i capi d'Israele d'aiutare Salomone, suo figliuolo, e disse loro:
\par 18 'L'Eterno, l'Iddio vostro, non è egli con voi, e non v'ha egli dato quiete d'ogn'intorno? Infatti egli m'ha dato nelle mani gli abitanti del paese, e il paese è assoggettato all'Eterno ed al suo popolo.
\par 19 Disponete dunque il vostro cuore e l'anima vostra a cercare l'Eterno, ch'è il vostro Dio; poi levatevi, e costruite il santuario dell'Eterno Iddio, per trasferire l'arca del patto dell'Eterno e gli utensili consacrati a Dio, nella casa che dev'essere edificata al nome dell'Eterno'.

\chapter{23}

\par 1 Davide vecchio e sazio di giorni, stabilì Salomone, suo figliuolo, re d'Israele.
\par 2 E radunò tutti i capi d'Israele, i sacerdoti e i Leviti.
\par 3 Fu fatto un censimento dei Leviti dall'età di trent'anni in su; e, contati testa per testa, uomo per uomo, il loro numero risultò di trentottomila.
\par 4 E Davide disse: 'Ventiquattromila di questi siano addetti a dirigere l'opera della casa dell'Eterno; seimila siano magistrati e giudici;
\par 5 quattromila siano portinai, e quattromila celebrino l'Eterno con gli strumenti che io ho fatti per celebrarlo'.
\par 6 E Davide li divise in classi, secondo i figliuoli di Levi: Ghershon, Kehath e Merari.
\par 7 Dei Ghershoniti: Laedan e Scimei. -
\par 8 Figliuoli di Laedan: il capo Jehiel, Zetham, Joel; tre. -
\par 9 Figliuoli di Scimei: Scelomith, Haziel, Haran; tre. Questi sono i capi delle famiglie patriarcali di Laedan.
\par 10 - Figliuoli di Scimei: Jahath, Zina, Jeush e Beria. Questi sono i quattro figliuoli di Scimei.
\par 11 Jahath era il capo; Zina, il secondo; Jeush e Beria non ebbero molti figliuoli, e, nel censimento, formarono una sola casa patriarcale.
\par 12 Figliuoli di Kehath: Amram, Jtsehar, Hebron, Uzziel; quattro. -
\par 13 Figliuoli di Amram: Aaronne e Mosè. Aaronne fu appartato per esser consacrato come santissimo, egli coi suoi figliuoli, in perpetuo, per offrire i profumi dinanzi all'Eterno, per ministrargli, e per pronunziare in perpetuo la benedizione nel nome di lui.
\par 14 Quanto a Mosè, l'uomo di Dio, i suoi figliuoli furono contati nella tribù di Levi.
\par 15 Figliuoli di Mosè: Ghershom ed Eliezer.
\par 16 Figliuoli di Ghershom: Scebuel, il capo.
\par 17 E i figliuoli di Eliezer furono: Rehabia, il capo. Eliezer non ebbe altri figliuoli; ma i figliuoli di Rehabia furono numerosissimi.
\par 18 - Figliuoli di Jtsehar: Scelomith, il capo.
\par 19 - Figliuoli di Hebron: Jerija, il capo; Amaria, il secondo; Jahaziel, il terzo, e Jekameam, il quarto.
\par 20 - Figliuoli d'Uzziel: Mica, il capo, e Jscia, il secondo.
\par 21 Figliuoli di Merari: Mahli e Musci. - Figliuoli di Mahli: Eleazar e Kis;
\par 22 Eleazar morì e non ebbe figliuoli, ma solo delle figliuole; e le sposarono i figliuoli di Kis, loro parenti.
\par 23 - Figliuoli di Musci: Mahli, Eder e Jeremoth; tre.
\par 24 Questi sono i figliuoli di Levi secondo le loro case patriarcali, i capi famiglia secondo il censimento, fatto contando i nomi, testa per testa. Essi erano addetti a fare il servizio della casa dell'Eterno, dall'età di vent'anni in su,
\par 25 poiché Davide avea detto: 'L'Eterno, l'Iddio d'Israele, ha dato riposo al suo popolo, ed esso è venuto a stabilirsi a Gerusalemme per sempre;
\par 26 e anche i Leviti non avranno più bisogno di portare il tabernacolo e tutti gli utensili per il suo servizio'.
\par 27 Fu secondo le ultime disposizioni di Davide che il censimento dei figliuoli di Levi si fece dai venti anni in su.
\par 28 Posti presso i figliuoli d'Aaronne per il servizio della casa dell'Eterno, essi aveano l'incarico dei cortili, delle camere, della purificazione di tutte le cose sacre, dell'opera relativa al servizio della casa di Dio,
\par 29 dei pani della presentazione, del fior di farina per le offerte, delle focacce non lievitate, delle cose da cuocere sulla gratella, di quelle da friggere, e di tutte le misure di capacità e di lunghezza.
\par 30 Doveano presentarsi ogni mattina e ogni sera per lodare e celebrare l'Eterno,
\par 31 e per offrire del continuo davanti all'Eterno tutti gli olocausti, secondo il numero prescritto loro dalla legge, per i sabati, pei noviluni e per le feste solenni;
\par 32 e doveano prender cura della tenda di convegno, del santuario, e stare agli ordini dei figliuoli d'Aaronne loro fratelli, per il servizio della casa dell'Eterno.

\chapter{24}

\par 1 Le classi dei figliuoli d'Aaronne furono queste. Figliuoli d'Aaronne: Nadab, Abihu, Eleazar e Ithamar.
\par 2 Nadab e Abihu morirono prima del loro padre, e non ebbero figliuoli; Eleazar e Ithamar esercitarono il sacerdozio.
\par 3 Or Davide, con Tsadok de' figliuoli di Eleazar, e con Ahimelec de' figliuoli d'Ithamar, classificò i figliuoli d'Aaronne secondo il servizio che doveano fare.
\par 4 Tra i figliuoli di Eleazar si trovavano più capi di famiglie che tra i figliuoli d'Ithamar; e furon divisi così: per i figliuoli di Eleazar, sedici capi di famiglie patriarcali; per i figliuoli d'Ithamar, otto capi delle loro famiglie patriarcali.
\par 5 La classificazione fu fatta a sorte, tanto per gli uni quanto per gli altri; perché v'erano dei principi del santuario e de' principi di Dio tanto tra i figliuoli d'Eleazar quanto tra i figliuoli d'Ithamar.
\par 6 Scemaia, figliuolo di Nathaneel, il segretario, ch'era della tribù di Levi, li iscrisse in presenza del re e dei principi, in presenza del sacerdote Tsadok, di Ahimelec, figliuolo di Ebiathar, e in presenza dei capi delle famiglie patriarcali dei sacerdoti e dei Leviti. Si tirò a sorte una casa patriarcale per Eleazar, e, proporzionalmente, per Ithamar.
\par 7 Il primo, designato dalla sorte, fu Jehoiarib; il secondo, Jedaia;
\par 8 il terzo, Harim; il quarto, Seorim;
\par 9 il quinto, Malkija;
\par 10 il sesto, Mijamin; il settimo, Hakkots; l'ottavo, Abija;
\par 11 il nono, Jeshua; il decimo, Scecania;
\par 12 l'undecimo, Eliascib; il dodicesimo, Jakim;
\par 13 il tredicesimo, Huppa; il quattordicesimo, Jescebeab;
\par 14 il quindicesimo, Bilga; il sedicesimo, Immer;
\par 15 il diciassettesimo, Hezir; il diciottesimo, Happitsets;
\par 16 il diciannovesimo, Pethahia; il ventesimo, Ezechiele;
\par 17 il ventunesimo, Jakin; il ventiduesimo, Gamul;
\par 18 il ventitreesimo, Delaia; il ventiquattresimo, Maazia.
\par 19 Così furono classificati per il loro servizio, affinché entrassero nella casa dell'Eterno secondo la regola stabilita per loro da Aaronne loro padre, e che l'Eterno, l'Iddio d'Israele, gli aveva prescritta.
\par 20 Quanto al rimanente de' figliuoli di Levi, questi ne furono i capi. Dei figliuoli d'Amram: Shubael; de' figliuoli di Shubal; Jehdia.
\par 21 Di Rehabia, de' figliuoli di Rehabia: il capo Jscia.
\par 22 Degli Jtsehariti: Scelomoth; de' figliuoli di Scelomoth: Jahath.
\par 23 Figliuoli di Hebron: Jerija, Amaria il secondo, Jahaziel il terzo, Jekameam il quarto.
\par 24 Figliuoli di Uzziel: Mica; de' figliuoli di Mica: Shamir;
\par 25 fratello di Mica: Jscia; de' figliuoli d'Jscia: Zaccaria.
\par 26 Figliuoli di Merari: Mahli e Musci, e i figliuoli di Jaazia, suo figliuolo,
\par 27 vale a dire i figliuoli di Merari, per il tramite di Jaazia, suo figliuolo: Shoham, Zaccur e Ibri.
\par 28 Di Mahli: Eleazar, che non ebbe figliuoli.
\par 29 Di Kis: i figliuoli di Kis: Jerahmeel.
\par 30 Figliuoli di Musci: Mahli, Eder e Jerimoth. Questi sono i figliuoli dei Leviti secondo le loro case patriarcali.
\par 31 Anch'essi, come i figliuoli d'Aaronne, loro fratelli, tirarono a sorte in presenza del re Davide, di Tsadok, di Ahimelec e dei capi delle famiglie patriarcali dei sacerdoti e dei Leviti. Ogni capo di famiglia patriarcale tirò a sorte, nello stesso modo che il fratello, più giovane di lui.

\chapter{25}

\par 1 Poi Davide e i capi dell'esercito appartarono per il servizio quelli de' figliuoli di Asaf, di Heman e di Jeduthun che cantavano gl'inni sacri accompagnandosi con cetre, con saltèri e con cembali; e questo è il numero di quelli che furono incaricati di questo servizio.
\par 2 Dei figliuoli di Asaf: Zaccur, Josef, Nethania, Asarela, figliuoli di Asaf, sotto la direzione di Asaf, che cantava gl'inni sacri, seguendo le istruzioni del re.
\par 3 Di Jeduthun: i figliuoli di Jeduthun: Ghedalia, Tseri, Isaia, Hashabia, Mattithia e Scimei, sei, sotto la direzione del loro padre Jeduthun, che cantava gl'inni sacri con la cetra per lodare e celebrare l'Eterno.
\par 4 Di Heman: i figliuoli di Heman: Bukkija, Mattania, Uzziel, Scebuel, Jerimoth, Hanania, Hanani, Eliathak, Ghiddalthi, Romamti-Ezer, Joshbekasha, Mallothi, Hothir, Mahazioth.
\par 5 Tutti questi erano figliuoli di Heman, veggente del re, secondo la promessa di Dio di accrescer la potenza di Heman. Iddio infatti avea dato a Heman quattordici figliuoli e tre figliuole.
\par 6 Tutti questi erano sotto la direzione dei loro padri per il canto della casa dell'Eterno, ed aveano dei cembali, dei saltèri e delle cetre per il servizio della casa di Dio. Eran sotto la direzione del re, di Asaf, di Jeduthun e di Heman.
\par 7 Il loro numero, compresi i loro fratelli istruiti nel canto in onore dell'Eterno, tutti quelli cioè ch'erano esperti in questo, ascendeva a dugentottantotto.
\par 8 Tirarono a sorte il loro ordine di servizio, tanto i piccoli quanto i grandi, tanto i maestri quanto i discepoli.
\par 9 Il primo designato dalla sorte per Asaf fu Josef; il secondo, Ghedalia, coi suoi fratelli e i suoi figliuoli, dodici in tutto;
\par 10 il terzo fu Zaccur, coi suoi figliuoli e i suoi fratelli, dodici in tutto;
\par 11 il quarto fu Jtseri, coi suoi figliuoli e i suoi fratelli, dodici in tutto;
\par 12 il quinto fu Nethania, coi suoi figliuoli e i suoi fratelli, dodici in tutto;
\par 13 il sesto fu Bukkia, coi suoi figliuoli e i suoi fratelli, dodici in tutto;
\par 14 il settimo fu Jesarela, coi suoi figliuoli e i suoi fratelli, dodici in tutto;
\par 15 l'ottavo fu Isaia, coi suoi figliuoli e i suoi fratelli, dodici in tutto;
\par 16 il nono fu Mattania, coi suoi figliuoli e i suoi fratelli, dodici in tutto;
\par 17 il decimo fu Scimei, coi suoi figliuoli e i suoi fratelli, dodici in tutto;
\par 18 l'undecimo fu Azarel, coi suoi figliuoli e i suoi fratelli, dodici in tutto;
\par 19 il dodicesimo fu Hashabia, coi suoi figliuoli e i suoi fratelli, dodici in tutto;
\par 20 il tredicesimo fu Shubael, coi suoi figliuoli e i suoi fratelli, dodici in tutto;
\par 21 il quattordicesimo fu Mattithia, coi suoi figliuoli e i suoi fratelli, dodici in tutto;
\par 22 il quindicesimo fu Jeremoth, coi suoi figliuoli e i suoi fratelli, dodici in tutto;
\par 23 il sedicesimo fu Hanania, coi suoi figliuoli e i suoi fratelli, dodici in tutto;
\par 24 il diciassettesimo fu Joshbekasha, coi suoi figliuoli e i suoi fratelli, dodici in tutto;
\par 25 il diciottesimo fu Hanani, coi suoi figliuoli e i suoi fratelli, dodici in tutto;
\par 26 il diciannovesimo fu Mallothi, coi suoi figliuoli e i suoi fratelli, dodici in tutto;
\par 27 il ventesimo fu Eliatha, coi suoi figliuoli e i suoi fratelli, dodici in tutto;
\par 28 il ventunesimo fu Hothir, coi suoi figliuoli e i suoi fratelli, dodici in tutto;
\par 29 il ventiduesimo fu Ghiddalti, coi suoi figliuoli e i suoi fratelli, dodici in tutto;
\par 30 il ventesimoterzo fu Mahazioth, coi suoi figliuoli e i suoi fratelli, dodici in tutto;
\par 31 il ventesimoquarto, fu Romamti-Ezer, coi suoi figliuoli e i suoi fratelli, dodici in tutto.

\chapter{26}

\par 1 Quanto alle classi de' portinai, v'erano: dei Korahiti: Mescelemia, figliuolo di Kore, dei figliuoli d'Asaf.
\par 2 Figliuoli di Mescelemia: Zaccaria, il primogenito, Jediael il secondo, Zebadia il terzo, Jathniel il quarto,
\par 3 Elam il quinto, Johanan il sesto, Eliehoenai il settimo.
\par 4 Figliuoli di Obed-Edom: Scemaia, il primogenito, Jehozabad il secondo, Joah il terzo, Sacar il quarto, Nethanel il quinto,
\par 5 Ammiel il sesto, Issacar il settimo, Peullethai l'ottavo; poiché Dio l'aveva benedetto.
\par 6 E a Scemaia, suo figliuolo, nacquero dei figliuoli che signoreggiarono sulla casa del padre loro, perché erano uomini forti e valorosi.
\par 7 Figliuoli di Scemaia: Othni, Refael, Obed, Elzabad e i suoi fratelli, uomini valorosi, Elihu e Semachia.
\par 8 Tutti questi erano figliuoli di Obed-Edom; essi, i loro figliuoli e i loro fratelli erano uomini valenti e pieni di forza per il servizio: sessantadue di Obed-Edom.
\par 9 Mescelemia ebbe figliuoli e fratelli, uomini valenti, in numero di diciotto.
\par 10 Hosa, de' figliuoli di Merari, ebbe per figliuoli: Scimri il capo - che il padre avea fatto capo, quantunque non fosse il primogenito -
\par 11 Hilkia il secondo, Tebalia il terzo, Zaccaria il quarto. Tutti i figliuoli e i fratelli di Hosa erano in numero di tredici.
\par 12 A queste classi di portinai, ai capi di questi uomini, come anche ai loro fratelli, fu affidato l'incarico del servizio della casa dell'Eterno.
\par 13 E tirarono a sorte, per ciascuna porta: i più piccoli come i più grandi, nell'ordine delle loro case patriarcali.
\par 14 Per il lato d'oriente la sorte designò Scelemia. Si tirò poi a sorte per Zaccaria, suo figliuolo, ch'era un consigliere di senno; e la sorte designò lui per il lato di settentrione.
\par 15 Per il lato di mezzogiorno, la sorte designò Obed-Edom; e per i magazzini designò i suoi figliuoli.
\par 16 Per il lato d'occidente, con la porta Shalleketh, sulla via che sale, la sorte designò Shuppim e Hosa: erano due posti di guardia, uno dirimpetto all'altro.
\par 17 A oriente v'erano sei Leviti; al settentrione, quattro per giorno; a mezzodì, quattro per giorno, e quattro ai magazzini, due per ogni ingresso;
\par 18 al recinto del tempio, a occidente, ve n'erano addetti quattro per la strada, due per il recinto.
\par 19 Queste sono le classi dei portinai, scelti fra i figliuoli di Kore e i figliuoli di Merari.
\par 20 I Leviti, loro fratelli, erano preposti ai tesori della casa di Dio e ai tesori delle cose consacrate.
\par 21 I figliuoli di Laedan, i figliuoli dei Ghershoniti discesi da Laedan, i capi delle case patriarcali di Laedan il Ghershonita, cioè Jehieli;
\par 22 e i figliuoli di Jehieli: Zetham e Joel suo fratello, erano preposti ai tesori della casa dell'Eterno.
\par 23 Fra gli Amramiti, gli Jtsehariti, gli Hebroniti e gli Uzzieliti,
\par 24 Scebuel, figliuolo di Ghershom, figliuolo di Mosè, era sovrintendente dei tesori.
\par 25 Tra i suoi fratelli per il tramite di Eliezer, che ebbe per figliuolo Rehabia, ch'ebbe per figliuolo Isaia, ch'ebbe per figliuolo Joram, ch'ebbe per figliuolo Zicri, ch'ebbe per figliuolo Scelomith,
\par 26 questo Scelomith e i suoi fratelli erano preposti a tutti i tesori delle cose sacre, che il re Davide, i capi delle case patriarcali, i capi di migliaia e di centinaia e i capi dell'esercito aveano consacrate
\par 27 (prelevandole dal bottino di guerra per il mantenimento della casa dell'Eterno),
\par 28 e a tutto quello ch'era stato consacrato da Samuele, il veggente, da Saul, figliuolo di Kis, da Abner, figliuolo di Ner, e da Joab, figliuolo di Tseruia. Chiunque consacrava qualcosa l'affidava alle mani di Scelomith e de' suoi fratelli.
\par 29 Fra gli Jtshariti, Kenania e i suoi figliuoli erano addetti agli affari estranei al tempio, come magistrati e giudici in Israele.
\par 30 Fra gli Hebroniti, Hashabia e i suoi fratelli, uomini valorosi, in numero di millesettecento furono preposti alla sorveglianza d'Israele, di qua dal Giordano, a occidente, per tutti gli affari che concernevano l'Eterno, e per il servizio del re.
\par 31 Fra gli Hebroniti (circa gli Hebroniti, l'anno quarantesimo del regno di Davide si fecero delle ricerche relative alle loro genealogie, secondo le loro case patriarcali, e si trovaron fra loro degli uomini forti e valorosi a Jaezer in Galaad)
\par 32 v'erano il capo Ieria e i suoi fratelli, uomini valorosi, in numero di duemila settecento capi di case patriarcali; e il re Davide affidò loro la sorveglianza dei Rubeniti, dei Gaditi, della mezza tribù di Manasse, per tutte le cose concernenti Dio e per tutti gli affari del re.

\chapter{27}

\par 1 Ora ecco i figliuoli d'Israele, secondo il loro numero, i capi di famiglie patriarcali, i capi di migliaia e di centinaia e i loro ufficiali al servizio del re per tutto quello che concerneva le divisioni che entravano e uscivano di servizio, mese per mese, tutti i mesi dell'anno, ogni divisione essendo di ventiquattromila uomini.
\par 2 A capo della prima divisione per il primo mese, stava Jashobeam, figliuolo di Zabdiel, e la sua divisione era di ventiquattromila uomini.
\par 3 Egli era dei figliuoli di Perets, e capo di tutti gli ufficiali dell'esercito, per il primo mese.
\par 4 A capo della divisione del secondo mese stava Dodai, lo Ahohita, con la sua divisione; Mikloth era l'ufficiale superiore e la sua divisione era di ventiquattromila uomini.
\par 5 Il capo della terza divisione per il terzo mese era Benaia, figliuolo del sacerdote Jehoiada; era capo, e la sua divisione noverava ventiquattromila uomini.
\par 6 Questo Benaia era un prode fra i trenta, e a capo dei trenta; e Ammizabad, suo figliuolo, era l'ufficiale superiore della sua divisione.
\par 7 Il quarto, per il quarto mese, era Asael, fratello di Joab; e, dopo di lui, Zebadia, suo figliuolo; aveva una divisione di ventiquattromila uomini.
\par 8 Il quinto, per il quinto mese, era il capo Shamehuth, lo Jzrahita, e aveva una divisione di ventiquattromila uomini.
\par 9 Il sesto, per il sesto mese, era Ira, figliuolo di Ikkesh, il Tekoita, e aveva una divisione di ventiquattromila uomini.
\par 10 Il settimo, per il settimo mese, era Helets, il Pelonita, dei figliuoli d'Efraim, e aveva una divisione di ventiquattromila uomini.
\par 11 L'ottavo, per l'ottavo mese, era Sibbecai, lo Hushathita, della famiglia degli Zerahiti, e aveva una divisione di ventiquattromila uomini.
\par 12 Il nono, per il nono mese, era Abiezer da Anatoth, dei Beniaminiti, e aveva una divisione di ventiquattromila uomini.
\par 13 Il decimo, per il decimo mese, era Mahrai da Netofa, della famiglia degli Zerahiti, e aveva una divisione di ventiquattromila uomini.
\par 14 L'undecimo, per l'undecimo mese, era Benaia da Pirathon, de' figliuoli di Efraim, e aveva una divisione di ventiquattromila uomini.
\par 15 Il dodicesimo, per il dodicesimo mese, era Heldai da Netofa, della famiglia di Othniel, e aveva una divisione di ventiquattromila uomini.
\par 16 Questi erano i capi delle tribù d'Israele. Capo dei Rubeniti: Eliezer, figliuolo di Zicri. Dei Simeoniti: Scefatia, figliuolo di Maaca.
\par 17 Dei Leviti: Hashabia, figliuolo di Kemuel. Di Aaronne: Tsadok.
\par 18 Di Giuda: Elihu, dei fratelli di Davide. Di Issacar: Omri, figliuolo di Micael.
\par 19 Di Zabulon: Ishmaia, figliuolo di Obadia. Di Neftali: Jerimoth, figliuolo di Azriel.
\par 20 Dei figliuoli d'Efraim: Osea, figliuolo di Azazia. Della mezza tribù di Manasse: Ioel, figliuolo di Pedaia.
\par 21 Della mezza tribù di Manasse in Galaad: Iddo, figliuolo di Zaccaria. Di Beniamino: Jaaziel, figliuolo di Abner.
\par 22 Di Dan: Azareel, figliuolo di Jeroham. Questi erano i capi delle tribù d'Israele.
\par 23 Davide non fece il censimento di quei d'Israele ch'erano in età di vent'anni in giù, perché l'Eterno avea detto di moltiplicare Israele come le stelle del cielo.
\par 24 Joab, figliuolo di Tseruia, avea cominciato il censimento, ma non lo finì; e l'ira dell'Eterno piombò sopra Israele a motivo di questo censimento, che non fu iscritto fra gli altri nelle Cronache del re Davide.
\par 25 Azmaveth, figliuolo di Adiel, era preposto ai tesori del re; Gionathan, figliuolo di Uzzia, ai tesori ch'erano nella campagna, nelle città, nei villaggi e nelle torri;
\par 26 Ezri, figliuolo di Kelub, ai lavoratori della campagna per la cultura del suolo;
\par 27 Scimei da Rama, alle vigne; Zabdi da Sefam, al prodotto de' vigneti per fornire le cantine;
\par 28 Baal-Hanan da Gheder, agli uliveti ed ai sicomori nella pianura; Joash, alle cantine dell'olio;
\par 29 Scitrai da Sharon, al grosso bestiame che pasceva a Sharon; Shafat, figliuolo di Adlai, al grosso bestiame delle valli;
\par 30 Obil, l'Ishmaelita, ai cammelli; Jehdeia da Meronoth, agli asini;
\par 31 Jaziz, lo Hagarita, al minuto bestiame. Tutti questi erano amministratori dei beni del re Davide.
\par 32 E Gionathan, zio di Davide, era consigliere, uomo intelligente e istruito; Jehiel, figliuolo di Hacmoni, stava presso i figliuoli del re;
\par 33 Ahitofel era consigliere del re; Hushai, l'Arkita, era amico del re;
\par 34 dopo Ahitofel furono consiglieri Jehoiada, figliuolo di Benaia, e Abiathar; il capo dell'esercito del re era Joab.

\chapter{28}

\par 1 Or Davide convocò a Gerusalemme tutti i capi d'Israele, i capi delle tribù, i capi delle divisioni al servizio del re, i capi di migliaia, i capi di centinaia, gli amministratori di tutti i beni e del bestiame appartenente al re ed ai suoi figliuoli, insieme con gli ufficiali di corte, cogli uomini prodi e tutti i valorosi.
\par 2 Poi Davide, alzatosi e stando in piedi, disse: 'Ascoltatemi, fratelli miei e popolo mio! Io avevo in cuore di edificare una casa di riposo per l'arca del patto dell'Eterno e per lo sgabello de' piedi del nostro Dio, e avevo fatto dei preparativi per la fabbrica.
\par 3 Ma Dio mi disse: - Tu non edificherai una casa al mio nome, perché sei uomo di guerra e hai sparso del sangue. -
\par 4 L'Eterno, l'Iddio d'Israele, ha scelto me, in tutta la casa di mio padre, perché io fossi re d'Israele in perpetuo; poich'egli ha scelto Giuda, come principe; e, nella casa di Giuda, la casa di mio padre; e tra i figliuoli di mio padre gli è piaciuto di far me re di tutto Israele;
\par 5 e fra tutti i miei figliuoli - giacché l'Eterno mi ha dati molti figliuoli - egli ha scelto il figliuol mio Salomone, perché segga sul trono dell'Eterno, che regna sopra Israele.
\par 6 Egli m'ha detto: - Salomone, tuo figliuolo, sarà quegli che edificherà la mia casa e i miei cortili; poiché io l'ho scelto per mio figliuolo, ed io gli sarò padre.
\par 7 E stabilirò saldamente il suo regno in perpetuo, s'egli sarà perseverante nella pratica de' miei comandamenti e dei miei precetti, com'è oggi. -
\par 8 Or dunque, in presenza di tutto Israele, dell'assemblea dell'Eterno, e dinanzi al nostro Dio che ci ascolta, io v'esorto ad osservare e a prendere a cuore tutti i comandamenti dell'Eterno, ch'è il vostro Dio, affinché possiate rimanere in possesso di questo buon paese, e lasciarlo in eredità ai vostri figliuoli, dopo di voi, in perpetuo.
\par 9 E tu, Salomone, figliuol mio, riconosci l'Iddio di tuo padre, e servilo con cuore integro e con animo volenteroso; poiché l'Eterno scruta tutti i cuori, e penetra tutti i disegni e tutti i pensieri. Se tu lo cerchi, egli si lascerà trovare da te; ma, se lo abbandoni, egli ti rigetterà in perpetuo.
\par 10 Considera ora che l'Eterno ha scelto te per edificare una casa, che serva da santuario; sii forte, e mettiti all'opra!'
\par 11 Allora Davide diede a Salomone suo figliuolo il piano del portico del tempio e degli edifizi, delle stanze dei tesori, delle stanze superiori, delle camere interne e del luogo per il propiziatorio,
\par 12 e il piano di tutto quello che aveva in mente relativamente ai cortili della casa dell'Eterno, a tutte le camere all'intorno, ai tesori della casa di Dio, ai tesori delle cose consacrate,
\par 13 alle classi dei sacerdoti e dei Leviti, a tutto quello che concerneva il servizio della casa dell'Eterno, e a tutti gli utensili che dovean servire alla casa dell'Eterno.
\par 14 Gli diede il modello degli utensili d'oro, col relativo peso d'oro per tutti gli utensili d'ogni specie di servizi, e il modello di tutti gli utensili d'argento, col relativo peso d'argento per tutti gli utensili d'ogni specie di servizi.
\par 15 Gli diede l'indicazione del peso dei candelabri d'oro e delle loro lampade d'oro, col peso d'ogni candelabro e delle sue lampade, e l'indicazione del peso dei candelabri d'argento, col peso d'ogni candelabro e delle sue lampade, secondo l'uso al quale ogni candelabro era destinato.
\par 16 Gli diede l'indicazione del peso dell'oro necessario per ognuna delle tavole dei pani della presentazione, e del peso dell'argento per le tavole d'argento;
\par 17 gli diede ugualmente l'indicazione del peso dell'oro puro per i forchettoni, per i bacini e per i calici; e l'indicazione del peso dell'oro per ciascuna delle coppe d'oro e del peso dell'argento per ciascuna delle coppe d'argento;
\par 18 e l'indicazione del peso necessario d'oro purificato per l'altare dei profumi, e il modello del carro ossia dei cherubini d'oro che stendevano le ali e coprivano l'arca del patto dell'Eterno.
\par 19 'Tutto questo', disse Davide, 'tutto il piano da eseguire, te lo do per iscritto, giacché la mano dell'Eterno, che è stata sopra me, m'ha dato l'intelligenza necessaria'.
\par 20 Davide disse ancora a Salomone, suo figliuolo: 'Sii forte, fatti animo, e mettiti all'opra; non temere, non ti sgomentare; poiché l'Eterno Iddio, il mio Dio, sarà teco; egli non ti lascerà e non ti abbandonerà fino a tanto che tutta l'opera per il servizio della casa dell'Eterno sia compiuta.
\par 21 Ed ecco le classi dei sacerdoti e dei Leviti per tutto il servizio della casa di Dio; e tu hai presso di te, per ogni lavoro, ogni sorta di uomini di buona volontà e abili in ogni specie di servizio; e i capi e tutto il popolo sono pronti ad eseguire tutti i tuoi comandi'.

\chapter{29}

\par 1 Poi il re Davide disse a tutta la raunanza: 'Salomone, mio figliuolo, il solo che Dio abbia scelto, è ancora giovine e in tenera età, e l'opera è grande; poiché questo palazzo non è destinato a un uomo, ma a Dio, all'Eterno.
\par 2 Ora io ho impiegato tutte le mie forze a preparare per la casa del mio Dio dell'oro per ciò che dev'esser d'oro, dell'argento per ciò che dev'esser d'argento, del rame per ciò che dev'esser di rame, del ferro per ciò che dev'esser di ferro, e del legname per ciò che dev'esser di legno, delle pietre d'ònice e delle pietre da incastonare, delle pietre brillanti e di diversi colori, ogni specie di pietre preziose, e del marmo bianco in gran quantità.
\par 3 Di più, per l'affezione che porto alla casa del mio Dio, siccome io posseggo in proprio un tesoro d'oro e d'argento, io lo do alla casa del mio Dio, oltre a tutto quello che ho preparato per la casa del santuario:
\par 4 cioè tremila talenti d'oro, d'oro d'Ofir, e settemila talenti d'argento purissimo, per rivestirne le pareti delle sale:
\par 5 l'oro per ciò che dev'esser d'oro, l'argento per ciò che dev'esser d'argento, e per tutti i lavori da eseguirsi dagli artefici. Chi è disposto a fare oggi qualche offerta all'Eterno?'
\par 6 Allora i capi delle case patriarcali, i capi delle tribù d'Israele, i capi delle migliaia e delle centinaia e gli amministratori degli affari del re fecero delle offerte volontarie;
\par 7 e diedero per il servizio della casa di Dio cinquemila talenti d'oro, diecimila dariche, diecimila talenti d'argento, diciottomila talenti di rame, e centomila talenti di ferro.
\par 8 Quelli che possedevano delle pietre preziose, le consegnarono a Jehiel, il Ghershonita, perché fossero riposte nel tesoro della casa dell'Eterno.
\par 9 Il popolo si rallegrò di quelle loro offerte volontarie, perché avean fatte quelle offerte all'Eterno con tutto il cuore; e anche il re Davide se ne rallegrò grandemente.
\par 10 Davide benedisse l'Eterno in presenza di tutta la raunanza, e disse: 'Benedetto sii tu, o Eterno, Dio del padre nostro Israele, di secolo in secolo!
\par 11 A te, o Eterno, la grandezza, la potenza, la gloria, lo splendore, la maestà, poiché tutto quello che sta in cielo e sulla terra è tuo! A te, o Eterno, il regno; a te, che t'innalzi come sovrano al disopra di tutte le cose!
\par 12 Da te vengono la ricchezza e la gloria; tu signoreggi su tutto; in tua mano sono la forza e la potenza, e sta in tuo potere il far grande e il render forte ogni cosa.
\par 13 Or dunque, o Dio nostro, noi ti rendiamo grazie, e celebriamo il tuo nome glorioso.
\par 14 Poiché chi son io, e chi è il mio popolo, che siamo in grado di offrirti volenterosamente cotanto? Giacché tutto viene da te; e noi t'abbiam dato quello che dalla tua mano abbiam ricevuto.
\par 15 Noi siamo dinanzi a te dei forestieri e dei pellegrini, come furon tutti i nostri padri; i nostri giorni sulla terra son come un'ombra, e non v'è speranza.
\par 16 O Eterno, Dio nostro, tutta quest'abbondanza di cose che abbiam preparata per edificare una casa a te, al tuo santo nome, viene dalla tua mano, e tutta ti appartiene.
\par 17 Io so, o mio Dio, che tu scruti il cuore, e ti compiaci della rettitudine; perciò, nella rettitudine del cuor mio, t'ho fatte tutte queste offerte volontarie, e ho veduto ora con gioia il tuo popolo che si trova qui, farti volenterosamente le offerte sue.
\par 18 O Eterno, o Dio d'Abrahamo, d'Isacco e d'Israele nostri padri, mantieni in perpetuo nel cuore del tuo popolo queste disposizioni, questi pensieri, e rendi saldo il suo cuore in te;
\par 19 e da' a Salomone, mio figliuolo, un cuore integro, affinch'egli osservi i tuoi comandamenti, i tuoi precetti e le tue leggi, affinché eseguisca tutti questi miei piani, e costruisca il palazzo, per il quale ho fatto i preparativi'.
\par 20 Poi Davide disse a tutta la raunanza: 'Or benedite l'Eterno, il vostro Dio'. E tutta la raunanza benedì l'Eterno, l'Iddio de' loro padri; e s'inchinarono, e si prostrarono dinanzi all'Eterno e dinanzi al re.
\par 21 E il giorno seguente immolarono delle vittime in onore dell'Eterno, e gli offrirono degli olocausti: mille giovenchi, mille montoni, mille agnelli, con le relative libazioni, e altri sacrifizi in gran numero, per tutto Israele.
\par 22 E mangiarono e bevvero, in quel giorno, nel cospetto dell'Eterno, con gran gioia; proclamarono re, per la seconda volta, Salomone, figliuolo di Davide, e lo unsero, consacrandolo all'Eterno come conduttore del popolo, e unsero Tsadok come sacerdote.
\par 23 Salomone si assise dunque sul trono dell'Eterno come re, invece di Davide suo padre; prosperò, e tutto Israele gli ubbidì.
\par 24 E tutti i capi, gli uomini prodi e anche tutti i figliuoli del re Davide si sottomisero al re Salomone.
\par 25 E l'Eterno innalzò sommamente Salomone nel cospetto di tutto Israele, e gli diede un regale splendore, quale nessun re, prima di lui, ebbe mai in Israele.
\par 26 Davide, figliuolo d'Isai, regnò su tutto Israele.
\par 27 Il tempo che regnò sopra Israele fu quarant'anni: a Hebron regnò sette anni; e a Gerusalemme, trentatre.
\par 28 Morì in prospera vecchiezza, sazio di giorni, di ricchezze, e di gloria; e Salomone, suo figliuolo, regnò in luogo suo. Or le azioni di Davide,
\par 29 le prime e le ultime, sono scritte nel libro di Samuele, il veggente, nel libro di Nathan, il profeta, e nel libro di Gad, il veggente,
\par 30 con tutta la storia del suo regno, delle sue gesta, e di quel che avvenne ai suoi tempi tanto in Israele, quanto in tutti i regni degli altri paesi.


\end{document}