\begin{document}

\title{2 Chronicles}


\chapter{1}

\par 1 Salomone, figliuolo di Davide, si stabilì saldamente nel suo regno; l'Eterno, il suo Dio, fu con lui e lo elevò a somma grandezza.
\par 2 Salomone parlò a tutto Israele, ai capi delle migliaia e delle centinaia, ai giudici, a tutti i principi capi delle case patriarcali di tutto Israele;
\par 3 ed egli, con tutta la raunanza, si recò all'alto luogo, ch'era a Gabaon; quivi, infatti, si trovava la tenda di convegno di Dio, che Mosè, servo dell'Eterno, avea fatta nel deserto.
\par 4 Quanto all'arca di Dio, Davide l'avea trasportata da Kiriath-Jearim al luogo ch'ei le avea preparato; poiché egli avea rizzata per lei una tenda a Gerusalemme;
\par 5 e l'altare di rame, fatto da Betsaleel, figliuolo d'Uri, figliuolo di Hur, si trovava anch'esso a Gabaon, davanti al tabernacolo dell'Eterno. Salomone e l'assemblea vennero a ricercarvi l'Eterno.
\par 6 E quivi, sull'altare di rame ch'era davanti alla tenda di convegno, Salomone offerse in presenza dell'Eterno mille olocausti.
\par 7 In quella notte, Iddio apparve a Salomone, e gli disse: 'Chiedi quello che vuoi ch'io ti dia'.
\par 8 Salomone rispose a Dio: 'Tu hai trattato con gran benevolenza Davide, mio padre, e hai fatto regnar me in luogo suo.
\par 9 Ora, o Eterno Iddio, si avveri la promessa che hai fatta a Davide mio padre, poiché tu m'hai fatto re di un popolo numeroso come la polvere della terra!
\par 10 Dammi dunque saviezza e intelligenza, affinché io sappia come condurmi di fronte a questo popolo; poiché chi mai potrebbe amministrar la giustizia per questo tuo popolo che è così numeroso?'
\par 11 E Dio disse a Salomone: 'Giacché questo è ciò che hai nel cuore, e non hai chiesto ricchezze, né beni, né gloria, né la morte de' tuoi nemici, e nemmeno una lunga vita, ma hai chiesto per te saviezza e intelligenza per poter amministrare la giustizia per il mio popolo del quale io t'ho fatto re,
\par 12 la saviezza e l'intelligenza ti sono concesse; e, oltre a questo, ti darò ricchezze, beni e gloria, come non n'ebbero mai i re che t'han preceduto, e come non ne avrà mai alcuno dei tuoi successori'.
\par 13 E Salomone tornò dall'alto luogo ch'era a Gabaon, e dalla tenda di convegno, a Gerusalemme, e regnò sopra Israele.
\par 14 Salomone radunò carri e cavalieri, ed ebbe millequattrocento carri e dodicimila cavalieri, che stanziò nelle città dove teneva i carri, e presso il re a Gerusalemme.
\par 15 E il re fece sì che l'argento e l'oro erano a Gerusalemme così comuni come le pietre, e i cedri tanto abbondanti quanto i sicomori della pianura.
\par 16 I cavalli che Salomone aveva, gli venivan menati dall'Egitto; le carovane di mercanti del re li andavano a prendere a mandre, per un prezzo convenuto;
\par 17 e facevano uscire dall'Egitto e giungere a destinazione un equipaggio per il costo di seicento sicli d'argento; un cavallo per il costo di centocinquanta. Nello stesso modo, per mezzo di que' mercanti, se ne facean venire per tutti i re degli Hittei e per i re della Siria.

\chapter{2}

\par 1 Salomone decise di costruire una casa per il nome dell'Eterno, e una casa reale per sé.
\par 2 Salomone arruolò settantamila uomini per portar pesi, ottantamila per tagliar pietre nella montagna, e tremilaseicento per sorvegliarli.
\par 3 Poi Salomone mandò a dire a Huram, re di Tiro: 'Fa' con me come facesti con Davide mio padre, al quale mandasti de' cedri per edificarsi una casa di abitazione.
\par 4 Ecco, io sto per edificare una casa per il nome dell'Eterno, dell'Iddio mio, per consacrargliela, per bruciare dinanzi a lui il profumo fragrante, per esporvi permanentemente i pani della presentazione, e per offrirvi gli olocausti del mattino e della sera, dei sabati, dei noviluni, e delle feste dell'Eterno, dell'Iddio nostro. Questa è una legge perpetua per Israele.
\par 5 La casa ch'io sto per edificare sarà grande, perché l'Iddio nostro è grande sopra tutti gli dèi.
\par 6 Ma chi sarà da tanto da edificargli una casa, se i cieli e i cieli de' cieli non lo posson contenere? E chi son io per edificargli una casa, se non sia tutt'al più per bruciarvi dei profumi dinanzi a lui?
\par 7 Mandami dunque un uomo abile a lavorare l'oro, l'argento, il rame, il ferro, la porpora, lo scarlatto, il violaceo, che sappia fare ogni sorta di lavori d'intagli, collaborando con gli artisti che sono presso di me in Giuda e a Gerusalemme, e che Davide mio padre aveva approntati.
\par 8 Mandami anche dal Libano del legname di cedro, di cipresso e di sandalo; perché io so che i tuoi servi sono abili nel tagliare il legname del Libano; ed ecco, i miei servi saranno coi servi tuoi,
\par 9 per prepararmi del legname in abbondanza; giacché la casa ch'io sto per edificare, sarà grande e maravigliosa.
\par 10 E ai tuoi servi che abbatteranno e taglieranno il legname io darò ventimila cori di grano battuto, ventimila cori d'orzo, ventimila bati di vino e ventimila bati d'olio'.
\par 11 E Huram, re di Tiro, rispose così in una lettera, che mandò a Salomone: 'L'Eterno, perché ama il suo popolo, ti ha costituito re su di esso'.
\par 12 Huram aggiunse: 'Benedetto sia l'Eterno, l'Iddio d'Israele, che ha fatto i cieli e la terra, perché ha dato al re Davide un figliuolo savio, pieno di senno e d'intelligenza, il quale edificherà una casa per l'Eterno, e una casa reale per sé!
\par 13 Io ti mando dunque un uomo abile e intelligente, maestro Huram,
\par 14 figliuolo d'una donna della tribù di Dan e di padre Tiro, il quale è abile a lavorare l'oro, l'argento, il rame, il ferro, la pietra, il legno, la porpora, il violaceo, il bisso, lo scarlatto, e sa pur fare ogni sorta di lavori d'intaglio, ed eseguire qualsivoglia lavoro d'arte gli si affidi. Egli collaborerà coi tuoi artisti e con gli artisti del mio signore Davide, tuo padre.
\par 15 Ora dunque mandi il mio signore ai suoi servi il grano, l'orzo, l'olio ed il vino, di cui egli ha parlato;
\par 16 e noi, dal canto nostro, taglieremo del legname del Libano, quanto te ne abbisognerà; te lo spediremo per mare su zattere fino a Jafo, e tu lo farai trasportare a Gerusalemme'.
\par 17 Salomone fece fare il conto di tutti gli stranieri che si trovavano nel paese d'Israele, e dei quali già Davide suo padre avea fatto il censimento; e se ne trovò centocinquantatremilaseicento;
\par 18 e ne prese settantamila per portar pesi, ottantamila per tagliar pietre nella montagna, e tremilaseicento per sorvegliare e far lavorare il popolo.

\chapter{3}

\par 1 Salomone cominciò a costruire la casa dell'Eterno a Gerusalemme, sul monte Moriah, dove l'Eterno era apparso a Davide suo padre, nel luogo che Davide aveva preparato, nell'aia di Ornan, il Gebuseo.
\par 2 Egli cominciò la costruzione il secondo giorno del secondo mese del quarto anno del suo regno.
\par 3 Or queste son le misure dei fondamenti gettati da Salomone per la costruzione della casa di Dio. La lunghezza, in cubiti dell'antica misura, era di sessanta cubiti; la larghezza, di venti cubiti.
\par 4 Il portico, sul davanti della casa, avea venti cubiti di lunghezza, rispondenti alla larghezza della casa, e centoventi d'altezza. Salomone ricoprì d'oro finissimo l'interno della casa.
\par 5 Egli ricoprì la casa maggiore di legno di cipresso, poi la rivestì d'oro finissimo e vi fece scolpire delle palme e delle catenelle.
\par 6 Rivestì questa casa di pietre preziose, per ornamento; e l'oro era quello di Parvaim.
\par 7 Rivestì pure d'oro la casa, le travi, gli stipiti, le pareti e le porte; e sulle pareti fece dei cherubini d'intaglio.
\par 8 E costruì il luogo santissimo. Esso avea venti cubiti di lunghezza, corrispondenti alla larghezza della casa, e venti cubiti di larghezza. Lo ricoprì d'oro finissimo, del valore di seicento talenti;
\par 9 e il peso dell'oro per i chiodi ascendeva a cinquanta sicli. Rivestì anche d'oro le camere superiori.
\par 10 Nel luogo santissimo fece scolpire due statue di cherubini, che furono ricoperti d'oro.
\par 11 Le ali dei cherubini aveano venti cubiti di lunghezza. L'ala del primo, lunga cinque cubiti, toccava la parete della casa; l'altra ala, pure di cinque cubiti, toccava l'ala del secondo cherubino.
\par 12 L'ala del secondo cherubino, lunga cinque cubiti, toccava la parete della casa; l'altra ala, pure di cinque cubiti, arrivava all'ala dell'altro cherubino.
\par 13 Le ali di questi cherubini, spiegate, misuravano venti cubiti. Essi stavano ritti in piè, e aveano le facce volte verso la sala.
\par 14 E fece il velo di filo violaceo, porporino, scarlatto e di bisso, e vi fece ricamare dei cherubini.
\par 15 Fece pure davanti alla casa due colonne di trentacinque cubiti d'altezza; e il capitello in cima a ciascuna, era di cinque cubiti.
\par 16 E fece delle catenelle, come quelle che erano nel santuario, e le pose in cima alle colonne; e fece cento melagrane, che sospese alle catenelle.
\par 17 E rizzò le colonne dinanzi al tempio: una a destra e l'altra a sinistra; e chiamò quella di destra Jakin, e quella di sinistra Boaz.

\chapter{4}

\par 1 Poi fece un altare di rame lungo venti cubiti, largo venti cubiti e alto dieci cubiti.
\par 2 Fece pure il mare di getto, che avea dieci cubiti da un orlo all'altro; era di forma perfettamente rotonda, avea cinque cubiti d'altezza, e una corda di trenta cubiti ne misurava la circonferenza.
\par 3 Sotto all'orlo lo circondavano delle figure di buoi, dieci per cubito, facendo tutto il giro del mare; erano disposti in due ordini ed erano stati fusi insieme col mare.
\par 4 Questo posava su dodici buoi, dei quali tre guardavano a settentrione, tre a occidente, tre a mezzogiorno, e tre ad oriente: il mare stava su di essi, e le parti posteriori de' buoi erano vòlte verso il di dentro.
\par 5 Esso aveva lo spessore d'un palmo; il suo orlo, fatto come l'orlo d'una coppa, avea la forma d'un fior di giglio; il mare poteva contenere tremila bati.
\par 6 Fece pure dieci conche, e ne pose cinque a destra e cinque a sinistra, perché servissero alle purificazioni; vi si lavava ciò che serviva agli olocausti. Il mare era destinato alle abluzioni dei sacerdoti.
\par 7 E fece i dieci candelabri d'oro, conformemente alle norme che li concernevano, e li pose nel tempio, cinque a destra e cinque a sinistra.
\par 8 Fece anche dieci tavole, che pose nel tempio, cinque a destra e cinque a sinistra. E fece cento bacini d'oro.
\par 9 Fece pure il cortile dei sacerdoti, e il gran cortile con le sue porte, delle quali ricoprì di rame i battenti.
\par 10 E pose il mare al lato destro della casa, verso sud-est.
\par 11 Huram fece pure i vasi per le ceneri, le palette ed i bacini. Così Huram compì l'opera che avea fatta per il re Salomone nella casa di Dio:
\par 12 le due colonne, le due palle dei capitelli in cima alle colonne; i due reticolati per coprire le due palle dei capitelli in cima alle colonne,
\par 13 le quattrocento melagrane per i due reticolati, a due ordini di melagrane per ogni reticolato, da coprire le due palle dei capitelli in cima alle colonne;
\par 14 e fece le basi e le conche sulle basi,
\par 15 il mare, ch'era unico, e i dodici buoi sotto il mare,
\par 16 e i vasi per le ceneri, le palette, i forchettoni e tutti gli utensili accessori. Maestro Huram li fece per il re Salomone, per la casa dell'Eterno, di rame tirato a pulimento.
\par 17 Il re li fece fondere nella pianura del Giordano, in un suolo argilloso, fra Succoth e Tsereda.
\par 18 Salomone fece tutti questi utensili in così gran quantità, che non se ne riscontrò il peso del rame.
\par 19 Salomone fece fabbricare tutti gli arredi della casa di Dio: l'altare d'oro, le tavole sulle quali si mettevano i pani della presentazione;
\par 20 i candelabri d'oro puro, con le loro lampade, da accendere, secondo la norma stabilita, davanti al santuario;
\par 21 i fiori, le lampade, gli smoccolatoi, d'oro del più puro;
\par 22 i coltelli, i bacini, le coppe e i bracieri, d'oro fino. Quanto alla porta della casa, i battenti interiori, all'ingresso del luogo santissimo, e le porte della casa, all'ingresso del tempio, erano d'oro.

\chapter{5}

\par 1 Così fu compiuta tutta l'opera che Salomone fece eseguire per la casa dell'Eterno. E Salomone fece portare l'argento, l'oro e tutti gli utensili che Davide suo padre avea consacrati, e li mise nei tesori della casa di Dio.
\par 2 Allora Salomone radunò a Gerusalemme gli anziani d'Israele e tutti i capi delle tribù, i principi delle famiglie patriarcali dei figliuoli d'Israele, per portar su l'arca del patto dell'Eterno, dalla città di Davide, cioè da Sion.
\par 3 Tutti gli uomini d'Israele si radunarono presso il re per la festa che cadeva il settimo mese.
\par 4 Arrivati che furono tutti gli anziani d'Israele, i Leviti presero l'arca,
\par 5 e portarono su l'arca, la tenda di convegno, e tutti gli utensili sacri che erano nella tenda. I sacerdoti ed i Leviti eseguirono il trasporto.
\par 6 Il re Salomone e tutta la raunanza d'Israele convocata presso di lui, si raccolsero davanti all'arca, e immolarono pecore e buoi in tal quantità da non potersi contare né calcolare.
\par 7 I sacerdoti portarono l'arca del patto dell'Eterno al luogo destinatole, nel santuario della casa, nel luogo santissimo, sotto le ali dei cherubini;
\par 8 poiché i cherubini aveano le ali spiegate sopra il sito dell'arca, e coprivano dall'alto l'arca e le sue stanghe.
\par 9 Le stanghe aveano una tale lunghezza che le loro estremità si vedevano sporgere dall'arca, davanti al santuario, ma non si vedevano dal di fuori. Esse son rimaste quivi fino al dì d'oggi.
\par 10 Nell'arca non v'era altro se non le due tavole di pietra che Mosè vi avea deposte sullo Horeb, quando l'Eterno fece patto coi figliuoli d'Israele, dopo che questi furono usciti dal paese d'Egitto.
\par 11 Or avvenne che mentre i sacerdoti uscivano dal luogo santo - giacché tutti i sacerdoti presenti s'erano santificati senza osservare l'ordine delle classi,
\par 12 e tutti i Leviti cantori, Asaf, Heman, Jeduthun, i loro figliuoli e i loro fratelli, vestiti di bisso, con cembali, saltèri e cetre stavano in piè a oriente dell'altare, e con essi centoventi sacerdoti che sonavan la tromba
\par 13 - mentre, dico, quelli che sonavan la tromba e quelli che cantavano, come un sol uomo, fecero udire un'unica voce per celebrare e per lodare l'Eterno, e alzarono la voce al suon delle trombe, de' cembali e degli altri strumenti musicali, e celebrarono l'Eterno dicendo: 'Celebrate l'Eterno, perch'egli è buono, perché la sua benignità dura in perpetuo', avvenne che la casa, la casa dell'Eterno, fu riempita da una nuvola,
\par 14 e i sacerdoti non poterono rimanervi per farvi l'ufficio loro, a motivo della nuvola; poiché la gloria dell'Eterno riempiva la casa di Dio.

\chapter{6}

\par 1 Allora Salomone disse: 'L'Eterno ha dichiarato che abiterebbe nella oscurità!
\par 2 E io t'ho costruito una casa per tua abitazione, un luogo ove tu dimorerai in perpetuo!'
\par 3 Poi il re voltò la faccia, e benedisse tutta la raunanza d'Israele; e tutta la raunanza d'Israele stava in piedi.
\par 4 E disse: 'Benedetto sia l'Eterno, l'Iddio d'Israele, il quale di sua propria bocca parlò a Davide mio padre, e con la sua potenza ha adempito quel che avea dichiarato dicendo:
\par 5 - Dal giorno che trassi il mio popolo d'Israele dal paese d'Egitto, io non scelsi alcuna città, fra tutte le tribù d'Israele, per edificarvi una casa, ove il mio nome dimorasse; e non scelsi alcun uomo perché fosse principe del mio popolo d'Israele;
\par 6 ma ho scelto Gerusalemme perché il mio nome vi dimori, e ho scelto Davide per regnare sul mio popolo d'Israele. -
\par 7 Or Davide, mio padre, ebbe in cuore di costruire una casa al nome dell'Eterno, dell'Iddio d'Israele;
\par 8 ma l'Eterno disse a Davide mio padre: - Quanto all'aver tu avuto in cuore di costruire una casa al mio nome, hai fatto bene ad aver questo in cuore;
\par 9 però, non sarai tu che edificherai la casa; ma il tuo figliuolo che uscirà dalle tue viscere, sarà quegli che costruirà la casa al mio nome. -
\par 10 E l'Eterno ha adempita la parola che avea pronunziata; ed io son sorto in luogo di Davide mio padre, e mi sono assiso sul trono d'Israele, come l'Eterno aveva annunziato, ed ho costruita la casa al nome dell'Eterno, dell'Iddio d'Israele.
\par 11 E quivi ho posto l'arca nella quale è il patto dell'Eterno: il patto ch'egli fermò coi figliuoli d'Israele'.
\par 12 Poi Salomone si pose davanti all'altare dell'Eterno, in presenza di tutta la raunanza d'Israele, e stese le sue mani.
\par 13 Egli, infatti, avea fatto costruire una tribuna di rame, lunga cinque cubiti, larga cinque cubiti e alta tre cubiti, e l'avea posta in mezzo al cortile; egli vi salì, si mise in ginocchio in presenza di tutta la raunanza d'Israele, stese le mani verso il cielo, e disse:
\par 14 'O Eterno, Dio d'Israele! Non v'è Dio che sia simile a te, né in cielo né in terra! Tu mantieni il patto e la misericordia verso i tuoi servi che camminano in tua presenza con tutto il cuor loro.
\par 15 Tu hai mantenuta la promessa da te fatta al tuo servo Davide, mio padre; e ciò che dichiarasti con la tua propria bocca, la tua mano oggi l'adempie.
\par 16 Ora dunque, o Eterno, Dio d'Israele, mantieni al tuo servo Davide, mio padre, la promessa che gli facesti, dicendo: - Non ti mancherà mai qualcuno che segga nel mio cospetto sul trono d'Israele, purché i tuoi figliuoli veglino sulla loro condotta, e camminino secondo la mia legge, come tu hai camminato in mia presenza.
\par 17 Ora dunque, o Eterno, Dio d'Israele, s'avveri la parola che dicesti al tuo servo Davide!
\par 18 Ma è egli proprio vero che Dio abiti cogli uomini sulla terra? Ecco, i cieli e i cieli de' cieli non ti posson contenere; quanto meno questa casa che io ho costruita!
\par 19 Nondimeno, o Eterno, Dio mio, abbi riguardo alla preghiera del tuo servo e alla sua supplicazione, ascoltando il grido e la preghiera che il tuo servo ti rivolge.
\par 20 Siano gli occhi tuoi giorno e notte aperti su questa casa, sul luogo nel quale dicesti di voler mettere il tuo nome! Ascolta la preghiera che il tuo servo farà, rivòlto a questo luogo!
\par 21 Ascolta le supplicazioni del tuo servo e del tuo popolo Israele quando pregheranno, rivòlti a questo luogo; ascoltali dal luogo della tua dimora, dai cieli; ascolta e perdona!
\par 22 Quand'uno avrà peccato contro il suo prossimo e si esigerà da lui il giuramento per costringerlo a giurare, se quegli viene a giurare davanti al tuo altare in questa casa,
\par 23 tu ascoltalo dal cielo, agisci e giudica i tuoi servi; condanna il colpevole, facendo ricadere sul suo capo i suoi atti, e dichiara giusto l'innocente, trattandolo secondo la sua giustizia.
\par 24 Quando il tuo popolo Israele sarà sconfitto dal nemico per aver peccato contro di te, se torna a te, se dà gloria al tuo nome e ti rivolge preghiere e supplicazioni in questa casa, tu esaudiscilo dal cielo,
\par 25 perdona al tuo popolo d'Israele il suo peccato, e riconducilo nel paese che desti a lui ed ai suoi padri.
\par 26 Quando il cielo sarà chiuso e non vi sarà più pioggia a motivo dei loro peccati contro di te, se essi pregano rivòlti a questo luogo, se danno gloria al tuo nome e si convertono dai loro peccati perché li avrai afflitti,
\par 27 tu esaudiscili dal cielo, perdona il loro peccato ai tuoi servi ed al tuo popolo d'Israele, ai quali avrai mostrato la buona strada per cui debbon camminare; e manda la pioggia sulla terra, che hai data come eredità al tuo popolo.
\par 28 Quando il paese sarà invaso dalla carestia o dalla peste, dalla ruggine o dal carbone, dalle locuste o dai bruci, quando il nemico assedierà il tuo popolo nel suo paese, nelle sue città, quando scoppierà qualsivoglia flagello o epidemia,
\par 29 ogni preghiera, ogni supplicazione che ti sarà rivolta da un individuo o dall'intero tuo popolo d'Israele, allorché ciascuno avrà riconosciuta la sua piaga e il suo dolore e stenderà le sue mani verso questa casa,
\par 30 tu esaudiscila dal cielo, dal luogo della tua dimora, e perdona; rendi a ciascuno secondo le sue vie, tu che conosci il cuore d'ognuno; - poiché tu solo conosci il cuore dei figliuoli degli uomini; -
\par 31 affinché essi ti temano e camminino nelle tue vie tutto il tempo che vivranno nel paese che tu desti ai padri nostri!
\par 32 Anche lo straniero, che non è del tuo popolo d'Israele, quando verrà da un paese lontano a motivo del tuo gran nome, della tua mano potente e del tuo braccio disteso, quando verrà a pregarti in questa casa,
\par 33 tu esaudiscilo dal cielo, dal luogo della tua dimora, e concedi a questo straniero tutto quello che ti domanderà, affinché tutti i popoli della terra conoscano il tuo nome per temerti, come fa il tuo popolo d'Israele, e sappiano che il tuo nome è invocato su questa casa che io ho costruita!
\par 34 Quando il tuo popolo partirà per muover guerra al suo nemico seguendo la via per la quale tu l'avrai mandato, se t'innalza preghiere rivòlto alla città che tu hai scelta, e alla casa che io ho costruita al tuo nome,
\par 35 esaudisci dal cielo le sue preghiere e le sue supplicazioni, e fagli ragione.
\par 36 Quando peccheranno contro di te - poiché non v'è uomo che non pecchi - e tu ti sarai mosso a sdegno contro di loro e li avrai abbandonati in balìa del nemico che li menerà in cattività in un paese lontano o vicino,
\par 37 se, nel paese dove saranno schiavi, rientrano in se stessi, se tornano a te e ti rivolgono supplicazioni nel paese del loro servaggio, e dicono: - Abbiam peccato, abbiamo operato iniquamente, siamo stati malvagi, -
\par 38 se tornano a te con tutto il loro cuore e con tutta l'anima loro nel paese del loro servaggio dove sono stati menati schiavi, e ti pregano, rivòlti al loro paese, il paese che tu desti ai loro padri, alla città che tu hai scelta, e alla casa che io ho costruita al tuo nome,
\par 39 esaudisci dal cielo, dal luogo della tua dimora, la loro preghiera e le loro supplicazioni, e fa' loro ragione; perdona al tuo popolo che ha peccato contro di te!
\par 40 Ora, o Dio mio, siano aperti gli occhi tuoi, e siano attente le tue orecchie alla preghiera fatta in questo luogo!
\par 41 Ed ora, lèvati, o Eterno, o Dio, vieni al luogo del tuo riposo, tu e l'arca della tua forza! I tuoi sacerdoti, o Eterno, o Dio, siano rivestiti di salvezza, e giubilino nel bene i tuoi fedeli!
\par 42 O Eterno, o Dio, non respingere la faccia del tuo unto; ricordati delle grazie fatte a Davide, tuo servo!'

\chapter{7}

\par 1 Quando Salomone ebbe finito di pregare, il fuoco scese dal cielo, consumò l'olocausto e i sacrifizi, e la gloria dell'Eterno riempì la casa;
\par 2 e i sacerdoti non potevano entrare nella casa dell'Eterno a motivo della gloria dell'Eterno che riempiva la casa dell'Eterno.
\par 3 Tutti i figliuoli d'Israele videro scendere il fuoco e la gloria dell'Eterno sulla casa, e si chinarono con la faccia a terra, si prostrarono sul pavimento, e lodarono l'Eterno, dicendo: 'Celebrate l'Eterno, perch'egli è buono, perché la sua benignità dura in perpetuo'.
\par 4 Poi il re e tutto il popolo offrirono dei sacrifizi davanti all'Eterno.
\par 5 Il re Salomone offrì in sacrifizio ventiduemila buoi e centoventimila pecore. Così il re e tutto il popolo dedicarono la casa di Dio.
\par 6 I sacerdoti stavano in piè, intenti ai loro uffici; così pure i Leviti, con gli strumenti musicali consacrati all'Eterno, che il re Davide avea fatti per lodare l'Eterno, la cui 'benignità dura in perpetuo', quando anche Davide celebrava con essi l'Eterno; e i sacerdoti sonavano la tromba dirimpetto ai Leviti, e tutto Israele stava in piedi.
\par 7 Salomone consacrò la parte di mezzo del cortile, ch'è davanti alla casa dell'Eterno; poiché offrì quivi gli olocausti e i grassi dei sacrifizi di azioni di grazie, giacché l'altare di rame che Salomone avea fatto, non poteva contenere gli olocausti, le oblazioni e i grassi.
\par 8 E in quel tempo Salomone celebrò la festa per sette giorni, e tutto Israele con lui. Ci fu una grandissima raunanza di gente, venuta da tutto il paese: dai dintorni di Hamath fino al torrente d'Egitto.
\par 9 L'ottavo giorno fecero una raunanza solenne; poiché celebrarono la dedicazione dell'altare per sette giorni, e la festa per altri sette giorni.
\par 10 Il ventitreesimo giorno del settimo mese Salomone rimandò alle sue tende il popolo allegro e col cuor contento per il bene che l'Eterno avea fatto a Davide, a Salomone e ad Israele, suo popolo.
\par 11 Salomone dunque terminò la casa dell'Eterno e la casa reale, e menò a felice compimento tutto quello che aveva avuto in cuore di fare nella casa dell'Eterno e nella sua propria casa.
\par 12 E l'Eterno apparve di notte a Salomone, e gli disse: 'Io ho esaudita la tua preghiera, e mi sono scelto questo luogo come casa dei sacrifizi.
\par 13 Quand'io chiuderò il cielo in guisa che non vi sarà più pioggia, quand'ordinerò alle locuste di divorare il paese, quando manderò la peste fra il mio popolo,
\par 14 se il mio popolo, sul quale è invocato il mio nome, si umilia, prega, cerca la mia faccia e si converte dalle sue vie malvage, io lo esaudirò dal cielo, gli perdonerò i suoi peccati, e guarirò il suo paese.
\par 15 I miei occhi saranno oramai aperti e le mie orecchie attente alla preghiera fatta in questo luogo;
\par 16 poiché ora ho scelta e santificata questa casa, affinché il mio nome vi rimanga in perpetuo, e gli occhi miei ed il mio cuore saran quivi sempre.
\par 17 E quanto a te, se tu cammini dinanzi a me come camminò Davide tuo padre, facendo tutto quello che t'ho comandato, e se osservi le mie leggi e i miei precetti,
\par 18 io stabilirò il trono del tuo regno, come promisi a Davide tuo padre, dicendo: - Non ti mancherà mai qualcuno che regni sopra Israele.
\par 19 Ma se vi ritraete da me e abbandonate le mie leggi e i miei comandamenti che io vi ho posti dinanzi, e andate invece a servire altri dèi e a prostrarvi dinanzi a loro,
\par 20 io vi sradicherò dal mio paese che v'ho dato; e rigetterò dal mio cospetto la casa che ho consacrata al mio nome, e la farò diventare la favola e lo zimbello di tutti i popoli.
\par 21 Chiunque passerà vicino a questa casa, già così eccelsa, stupirà e dirà: - Perché l'Eterno ha egli trattato in tal guisa questo paese e questa casa? -
\par 22 e si risponderà: - Perché hanno abbandonato l'Eterno, l'Iddio dei loro padri che li trasse dal paese d'Egitto, si sono invaghiti di altri dèi, si son prostrati dinanzi a loro e li hanno serviti; ecco perché l'Eterno ha fatto venire tutti questi mali su loro'.

\chapter{8}

\par 1 Or avvenne che, passati i venti anni nei quali Salomone edificò la casa dell'Eterno e la sua propria casa,
\par 2 egli ricostruì le città che Huram gli avea date, e vi fece abitare i figliuoli d'Israele.
\par 3 E Salomone marciò contro Hamath-Tsoba e se ne impadronì.
\par 4 E ricostruì Tadmor nella parte deserta del paese, e tutte le città di rifornimento in Hamath.
\par 5 Ricostruì pure Beth-Horon superiore e Beth-Horon inferiore, città forti, munite di mura, di porte e di sbarre;
\par 6 riedificò Baalath e tutte le città di rifornimento che appartenevano al re, tutte le città per i suoi carri, le città per i suoi cavalieri, insomma tutto quello che gli piacque di costruire a Gerusalemme, al Libano e in tutto il paese del suo dominio.
\par 7 Di tutta la popolazione ch'era rimasta degli Hittei, degli Amorei, dei Ferezei, degli Hivvei e dei Gebusei, che non erano d'Israele,
\par 8 vale a dire dei loro discendenti ch'eran rimasti dopo di loro nel paese e che gl'Israeliti non aveano distrutti, Salomone fece tanti servi per le comandate; e tali son rimasti fino al dì d'oggi.
\par 9 Ma de' figliuoli d'Israele Salomone non impiegò alcuno come servo per i suoi lavori; essi furono la sua gente di guerra, capi de' suoi condottieri e comandanti dei suoi carri e dei suoi cavalieri.
\par 10 I capi preposti al popolo dal re Salomone e incaricati di sorvegliarlo, erano in numero di duecentocinquanta.
\par 11 Or Salomone fece salire la figliuola di Faraone dalla città di Davide alla casa ch'egli le avea fatto costruire; perché disse: 'La moglie mia non abiterà nella casa di Davide re d'Israele, perché i luoghi dov'è entrata l'arca dell'Eterno son santi'.
\par 12 Allora Salomone offrì degli olocausti all'Eterno sull'altare dell'Eterno, ch'egli avea costruito davanti al portico;
\par 13 offriva quello che bisognava offrire, secondo l'ordine di Mosè, ogni giorno, nei sabati, nei noviluni, e nelle feste solenni, tre volte all'anno: alla festa degli azzimi, alla festa delle settimane e alla festa delle capanne.
\par 14 E stabilì nelle loro funzioni, come le avea regolate Davide suo padre, le classi dei sacerdoti, i Leviti nella loro incombenza di celebrare l'Eterno e fare il servizio in presenza de' sacerdoti giorno per giorno, e i portinai, a ciascuna porta, secondo le loro classi; poiché così aveva ordinato Davide, l'uomo di Dio.
\par 15 E non si deviò in nulla dagli ordini che il re avea dato circa i sacerdoti e i Leviti, come pure relativamente ai tesori.
\par 16 Così fu condotta tutta l'opera di Salomone dal giorno in cui fu fondata la casa dell'Eterno, fino a quando fu terminata. La casa dell'Eterno ebbe il suo perfetto compimento.
\par 17 Allora Salomone partì per Etsion-Gheber e per Eloth, sulla riva del mare, nel paese di Edom.
\par 18 E Huram, per mezzo della sua gente, gli mandò delle navi e degli uomini che conoscevano il mare; i quali andaron con la gente di Salomone ad Ofir, vi presero quattrocentocinquanta talenti d'oro, e li portarono al re Salomone.

\chapter{9}

\par 1 Or la regina di Sceba, avendo udito la fama che circondava Salomone, venne a Gerusalemme per metterlo alla prova con degli enimmi. Essa giunse con un numerosissimo seguito, con cammelli carichi di aromi, d'oro in gran quantità, e di pietre preziose: e, recatasi da Salomone, gli disse tutto quello che aveva in cuore.
\par 2 Salomone rispose a tutte le questioni propostegli da lei, e non ci fu cosa che fosse oscura per il re, e ch'ei non sapesse spiegare.
\par 3 E quando la regina di Sceba ebbe veduto la sapienza di Salomone e la casa ch'egli avea costruita,
\par 4 e le vivande della sua mensa e gli alloggi de' suoi servi e l'ordine di servizio de' suoi ufficiali e le loro vesti e i suoi coppieri e le loro vesti e gli olocausti ch'egli offriva nella casa dell'Eterno, rimase fuor di sé dalla maraviglia.
\par 5 E disse al re: 'Quello che avevo sentito dire nel mio paese dei fatti tuoi e della tua sapienza era dunque vero.
\par 6 Ma io non ci ho creduto finché non son venuta io stessa, e non ho visto con gli occhi miei; ed ora, ecco, non m'era stata riferita neppure la metà della grandezza della tua sapienza! Tu sorpassi la fama che me n'era giunta!
\par 7 Beata la tua gente, beati questi tuoi servi che stanno del continuo dinanzi a te, ed ascoltano la tua sapienza!
\par 8 Sia benedetto l'Eterno, il tuo Dio, il quale t'ha gradito, mettendoti sul suo trono, onde tu regni per l'Eterno, per il tuo Dio! Iddio ti ha stabilito re per far ragione e giustizia, perch'egli ama Israele e vuol conservarlo in perpetuo'.
\par 9 Poi ella donò al re centoventi talenti d'oro, grandissima quantità di aromi e delle pietre preziose. Non vi furon più tali aromi, come quelli che la regina di Sceba diede al re Salomone.
\par 10 (I servi di Huram e i servi di Salomone che portavano oro da Ofir, portavano anche del legno di sandalo e delle pietre preziose;
\par 11 e di questo legno di sandalo il re fece delle scale per la casa dell'Eterno e per la casa reale, delle cetre e de' saltèri per i cantori. Del legno come questo non se n'era mai visto prima nel paese di Giuda).
\par 12 Il re Salomone diede alla regina di Sceba tutto quello ch'essa bramò e chiese, oltre all'equivalente di quello ch'essa avea portato al re. Poi ella si rimise in cammino, e coi suoi servi se ne tornò al suo paese.
\par 13 Or il peso dell'oro che giungeva ogni anno a Salomone, era di seicentosessantasei talenti,
\par 14 oltre quello che percepiva dai trafficanti e dai negozianti che gliene portavano, da tutti i re d'Arabia e dai governatori del paese che recavano a Salomone dell'oro e dell'argento.
\par 15 E il re Salomone fece fare duecento scudi grandi d'oro battuto, per ognuno dei quali impiegò seicento sicli d'oro battuto,
\par 16 e trecento altri scudi d'oro battuto, per ognuno dei quali impiegò trecento sicli d'oro; e il re li mise nella casa della 'Foresta del Libano'.
\par 17 Il re fece pure un gran trono d'avorio, che rivestì d'oro puro.
\par 18 Questo trono aveva sei gradini e una predella d'oro connessi col trono; v'erano dei bracci da un lato e dall'altro del seggio, due leoni stavano presso i bracci,
\par 19 e dodici leoni stavano sui sei gradini, da una parte e dall'altra. Niente di simile era ancora stato fatto in verun altro regno.
\par 20 E tutte le coppe del re Salomone erano d'oro, e tutto il vasellame della casa della 'Foresta del Libano' era d'oro puro; dell'argento non si faceva alcun conto al tempo di Salomone.
\par 21 Poiché il re aveva delle navi che andavano a Tarsis con la gente di Huram; e una volta ogni tre anni venivano le navi da Tarsis, recando oro, argento, avorio, scimmie e pavoni.
\par 22 Così il re Salomone fu il più grande di tutti i re della terra per ricchezze e per sapienza.
\par 23 E tutti i re della terra cercavan di veder Salomone per udir la sapienza che Dio gli avea messa in cuore.
\par 24 E ognun d'essi gli portava il suo dono: vasi d'argento, vasi d'oro, vesti, armi, aromi, cavalli, muli; e questo avveniva ogni anno.
\par 25 Salomone aveva delle scuderie per quattromila cavalli, dei carri, e dodicimila cavalieri, che distribuiva nelle città dove teneva i suoi carri, e in Gerusalemme presso di sé.
\par 26 Egli signoreggiava su tutti i re, dal fiume sino al paese de' Filistei e sino ai confini d'Egitto.
\par 27 E il re fece sì che l'argento era in Gerusalemme così comune come le pietre, e i cedri tanto abbondanti quanto i sicomori della pianura.
\par 28 E si menavano a Salomone de' cavalli dall'Egitto e da tutti i paesi.
\par 29 Or il rimanente delle azioni di Salomone, le prime e le ultime, sono scritte nel libro di Nathan, il profeta, nella profezia di Ahija di Scilo, e nelle visioni di Jeddo il veggente, relative a Geroboamo, figliuolo di Nebat.
\par 30 Salomone regnò a Gerusalemme, su tutto Israele, quarant'anni.
\par 31 Poi Salomone s'addormentò coi suoi padri, e fu sepolto nella città di Davide suo padre; e Roboamo suo figliuolo regnò in luogo suo.

\chapter{10}

\par 1 Roboamo andò a Sichem, perché tutto Israele era venuto a Sichem per farlo re.
\par 2 Quando Geroboamo, figliuolo di Nebat, ebbe di ciò notizia, si trovava ancora in Egitto, dov'era fuggito per scampare dal re Salomone; e tornò dall'Egitto.
\par 3 Lo mandarono a chiamare, e Geroboamo e tutto Israele vennero a parlare a Roboamo, e gli dissero:
\par 4 'Tuo padre ha reso duro il nostro giogo; ora rendi tu più lieve la dura servitù e il giogo pesante che tuo padre ci ha imposti, e noi ti serviremo'. Ed egli rispose loro:
\par 5 'Tornate da me fra tre giorni'. E il popolo se ne andò.
\par 6 Il re Roboamo si consigliò coi vecchi ch'erano stati al servizio del re Salomone suo padre mentre era vivo, e disse: 'Che mi consigliate voi di rispondere a questo popolo?'
\par 7 E quelli gli parlarono così: 'Se ti mostri benevolo verso questo popolo, e gli compiaci, e se gli parli con bontà, ti sarà servo per sempre'.
\par 8 Ma Roboamo abbandonò il consiglio datogli dai vecchi, e si consigliò coi giovani ch'eran cresciuti con lui ed erano stati al suo servizio,
\par 9 e disse loro: 'Come consigliate voi che rispondiamo a questo popolo che m'ha parlato dicendo: - Allevia il giogo che tuo padre ci ha imposto?' -
\par 10 E i giovani ch'eran cresciuti con lui gli parlarono così: 'Ecco quel che dirai a questo popolo che s'è rivolto a te dicendo: - Tuo padre ha reso pesante il nostro giogo, e tu ce lo allevia! - Gli risponderai così: - Il mio dito mignolo è più grosso del corpo di mio padre;
\par 11 ora, mio padre vi ha caricati d'un giogo pesante, ma io lo renderò più pesante ancora; mio padre vi ha castigati con la frusta, e io vi castigherò coi flagelli a punte'.
\par 12 Tre giorni dopo, Geroboamo e tutto il popolo vennero da Roboamo, come aveva ordinato il re dicendo: 'Tornate da me fra tre giorni'.
\par 13 E il re rispose loro duramente, abbandonando il consiglio che i vecchi gli aveano dato;
\par 14 e parlò loro secondo il consiglio de' giovani, dicendo: 'Mio padre ha reso pesante il vostro giogo, ma io lo renderò più pesante ancora; mio padre vi ha castigati con la frusta, e io vi castigherò coi flagelli a punte'.
\par 15 Così il re non diede ascolto al popolo; perché questa era cosa diretta da Dio, affinché si adempisse la parola che l'Eterno avea pronunziata per mezzo di Ahija di Scilo a Geroboamo, figliuolo di Nebat.
\par 16 E quando tutto Israele vide che il re non gli dava ascolto, rispose al re, dicendo: 'Che abbiam noi da fare con Davide? Noi non abbiamo nulla di comune col figliuolo d'Isai! Ognuno alle sue tende, o Israele! Provvedi ora alla tua casa, o Davide!' E tutto Israele se ne andò alle sue tende.
\par 17 Ma sui figliuoli d'Israele che abitavano nelle città di Giuda, regnò Roboamo.
\par 18 E il re Roboamo mandò loro Adoram, preposto ai tributi; ma i figliuoli d'Israele lo lapidarono ed egli morì. E il re Roboamo salì in fretta sopra un carro per fuggire a Gerusalemme.
\par 19 Così Israele si ribellò alla casa di Davide, ed è rimasto ribelle fino al dì d'oggi.

\chapter{11}

\par 1 Roboamo, giunto che fu a Gerusalemme, radunò la casa di Giuda e di Beniamino, centottantamila uomini, guerrieri scelti, per combattere contro Israele e restituire il regno a Roboamo.
\par 2 Ma la parola dell'Eterno fu così rivolta a Scemaia, uomo di Dio:
\par 3 'Parla a Roboamo, figliuolo di Salomone, re di Giuda, e a tutto Israele in Giuda e in Beniamino, e di' loro:
\par 4 - Così parla l'Eterno: Non salite a combattere contro i vostri fratelli! Ognuno se ne torni a casa sua; perché questo è avvenuto per voler mio'. Quelli ubbidirono alla parola dell'Eterno, e se ne tornaron via rinunziando a marciare contro Geroboamo.
\par 5 Roboamo abitò in Gerusalemme, e costruì delle città fortificate in Giuda.
\par 6 Costruì Bethlehem, Etam, Tekoa,
\par 7 Beth-Tsur, Soco, Adullam,
\par 8 Gath, Maresha, Zif,
\par 9 Adoraim, Lakis, Azeka,
\par 10 Tsorea, Ajalon ed Hebron, che erano in Giuda e in Beniamino, e ne fece delle città fortificate.
\par 11 Munì queste città fortificate, vi pose dei comandanti, e dei magazzini di viveri, d'olio e di vino;
\par 12 e in ognuna di queste città mise scudi e lance, e le rese straordinariamente forti. E Giuda e Beniamino furon per lui.
\par 13 I sacerdoti e i Leviti di tutto Israele vennero da tutte le loro contrade a porsi accanto a lui;
\par 14 poiché i Leviti abbandonarono i loro contadi e le loro proprietà, e vennero in Giuda e a Gerusalemme; perché Geroboamo, con i suoi figliuoli, li avea cacciati perché non esercitassero più l'ufficio di sacerdoti dell'Eterno,
\par 15 e s'era creato dei sacerdoti per gli alti luoghi, per i demoni, e per i vitelli che avea fatti.
\par 16 E quelli di tutte le tribù d'Israele che aveano in cuore di cercare l'Eterno, l'Iddio d'Israele, seguirono i Leviti a Gerusalemme per offrir sacrifizi all'Eterno, all'Iddio dei loro padri;
\par 17 e fortificarono così il regno di Giuda e resero stabile Roboamo, figliuolo di Salomone, durante tre anni; perché per tre anni seguiron la via di Davide e di Salomone.
\par 18 Roboamo prese per moglie Mahalath, figliuola di Jerimoth, figliuolo di Davide e di Abihail, figliuola di Eliab, figliuolo d'Isai.
\par 19 Essa gli partorì questi figliuoli: Jeush, Scemaria e Zaham.
\par 20 Dopo di lei, prese Maaca, figliuola d'Absalom, la quale gli partorì Abija, Attai, Ziza e Scelomith.
\par 21 E Roboamo amò Maaca, figliuola di Absalom, più di tutte le sue mogli e di tutte le sue concubine; perché ebbe diciotto mogli, e sessanta concubine, e generò ventotto figliuoli e sessanta figliuole.
\par 22 Roboamo stabilì Abija, figliuolo di Maaca, come capo della famiglia e principe de' suoi fratelli, perché aveva in mente di farlo re.
\par 23 E, con avvedutezza, sparse tutti i suoi figliuoli per tutte le contrade di Giuda e di Beniamino, in tutte le città fortificate, dette loro viveri in abbondanza, e cercò per loro molte mogli.

\chapter{12}

\par 1 Quando Roboamo fu bene stabilito e fortificato nel regno, egli, e tutto Israele con lui, abbandonò la legge dell'Eterno.
\par 2 E l'anno quinto del regno di Roboamo, Scishak re d'Egitto, salì contro Gerusalemme, perch'essi erano stati infedeli all'Eterno.
\par 3 Egli avea milleduecento carri e sessantamila cavalieri; con lui venne dall'Egitto un popolo innumerevole di Libî, di Sukkei e di Etiopi;
\par 4 s'impadronì delle città fortificate che appartenevano a Giuda, e giunse fino a Gerusalemme.
\par 5 E il profeta Scemaia si recò da Roboamo e dai capi di Giuda, che s'erano raccolti in Gerusalemme all'avvicinarsi di Scishak, e disse loro: 'Così dice l'Eterno: - Voi avete abbandonato me, quindi anch'io ho abbandonato voi nelle mani di Scishak'.
\par 6 Allora i principi d'Israele e il re si umiliarono, e dissero: 'L'Eterno è giusto'.
\par 7 E quando l'Eterno vide che s'erano umiliati, la parola dell'Eterno fu così rivolta a Scemaia: 'Essi si sono umiliati; io non li distruggerò, ma concederò loro fra poco un mezzo di scampo, e la mia ira non si rovescerà su Gerusalemme per mezzo di Scishak.
\par 8 Nondimeno gli saranno soggetti, e impareranno la differenza che v'è tra il servire a me e il servire ai regni degli altri paesi'.
\par 9 Scishak, re d'Egitto, salì dunque contro Gerusalemme e portò via i tesori della casa dell'Eterno e i tesori della casa del re; portò via ogni cosa; prese pure gli scudi d'oro che Salomone avea fatti;
\par 10 invece de' quali, il re Roboamo fece fare degli scudi di rame, e li affidò ai capitani della guardia che custodiva la porta della casa del re.
\par 11 E ogni volta che il re entrava nella casa dell'Eterno, quei della guardia venivano, e li portavano; poi li riportavano nella sala della guardia.
\par 12 Così, perch'egli s'era umiliato, l'Eterno rimosse da lui l'ira sua e non volle distruggerlo del tutto; e v'erano anche in Giuda delle cose buone.
\par 13 Il re Roboamo dunque si rese forte in Gerusalemme, e continuò a regnare. Avea quarantun anni quando cominciò a regnare, e regnò diciassette anni a Gerusalemme, la città che l'Eterno s'era scelta fra tutte le tribù d'Israele, per stabilirvi il suo nome. Sua madre si chiamava Naama, l'Ammonita.
\par 14 Ed egli fece il male, perché non applicò il cuor suo alla ricerca dell'Eterno.
\par 15 Or le azioni di Roboamo, le prime e le ultime, sono scritte nelle storie del profeta Scemaia e d'Iddo, il veggente, nei registri genealogici. E vi fu guerra continua fra Roboamo e Geroboamo.
\par 16 E Roboamo s'addormentò coi suoi padri e fu sepolto nella città di Davide. Ed Abija, suo figliuolo, regnò in luogo suo.

\chapter{13}

\par 1 Il diciottesimo anno del regno di Geroboamo, Abija cominciò a regnare sopra Giuda.
\par 2 Regnò tre anni in Gerusalemme. Sua madre si chiamava Micaia, figliuola d'Uriel, da Ghibea. E ci fu guerra tra Abija e Geroboamo.
\par 3 Abija entrò in guerra con un esercito di prodi guerrieri, quattrocentomila uomini scelti; e Geroboamo si dispose in ordine di battaglia contro di lui con ottocentomila uomini scelti, tutti forti e valorosi.
\par 4 Ed Abija si levò e disse, dall'alto del monte Tsemaraim, ch'è nella contrada montuosa d'Efraim: 'O Geroboamo, e tutto Israele, ascoltatemi!
\par 5 Non dovreste voi sapere che l'Eterno, l'Iddio d'Israele, ha dato per sempre il regno sopra Israele a Davide, a Davide ed ai suoi figliuoli, con un patto inviolabile?
\par 6 Eppure, Geroboamo, figliuolo di Nebat, servo di Salomone, figliuolo di Davide, s'è levato, e s'è ribellato contro il suo signore;
\par 7 e della gente da nulla, degli uomini perversi, si son raccolti attorno a lui, e si son fatti forti contro Roboamo, figliuolo di Salomone, allorché Roboamo era giovane, e timido di cuore, e non potea tener loro fronte.
\par 8 E ora voi credete di poter tener fronte al regno dell'Eterno, ch'è nelle mani dei figliuoli di Davide; e siete una gran moltitudine, e avete con voi i vitelli d'oro che Geroboamo vi ha fatti per vostri dèi.
\par 9 Non avete voi cacciati i sacerdoti dell'Eterno, i figliuoli d'Aaronne ed i Leviti? e non vi siete voi fatti de' sacerdoti al modo de' popoli d'altri paesi? Chiunque è venuto con un giovenco e con sette montoni per esser consacrato, è diventato sacerdote di quelli che non sono dèi.
\par 10 Quanto a noi, l'Eterno è nostro Dio, e non l'abbiamo abbandonato; i sacerdoti al servizio dell'Eterno son figliuoli d'Aaronne, e i Leviti son quelli che celebran le funzioni.
\par 11 Ogni mattina e ogni sera essi ardono in onor dell'Eterno gli olocausti e il profumo fragrante, mettono in ordine i pani della presentazione sulla tavola pura e ogni sera accendono il candelabro d'oro con le sue lampade; poiché noi osserviamo i comandamenti dell'Eterno, del nostro Dio; ma voi l'avete abbandonato.
\par 12 Ed ecco, noi abbiam con noi, alla nostra testa, Iddio e i suoi sacerdoti e le trombe squillanti, per sonar la carica contro di voi. O figliuoli d'Israele, non combattete contro l'Eterno, ch'è l'Iddio de' vostri padri, perché non vincerete!'
\par 13 Intanto Geroboamo li prese per di dietro mediante un'imboscata; in modo che le truppe di Geroboamo stavano in faccia a Giuda, che avea dietro l'imboscata.
\par 14 Que' di Giuda si volsero indietro, ed eccoli costretti a combattere davanti e di dietro. Allora gridarono all'Eterno, e i sacerdoti dettero nelle trombe.
\par 15 La gente di Giuda mandò un grido; e avvenne che, al grido della gente di Giuda, Iddio sconfisse Geroboamo e tutto Israele davanti ad Abija ed a Giuda.
\par 16 I figliuoli d'Israele fuggirono d'innanzi a Giuda, e Dio li diede nelle loro mani.
\par 17 Abija e il suo popolo ne fecero una grande strage; dalla parte d'Israele caddero morti cinquecentomila uomini scelti.
\par 18 Così i figliuoli d'Israele, in quel tempo, furono umiliati, e i figliuoli di Giuda ripresero vigore, perché s'erano appoggiati sull'Eterno, sull'Iddio dei loro padri.
\par 19 Abija inseguì Geroboamo, e gli prese delle città: Bethel e le città che ne dipendevano, Jeshana e le città che ne dipendevano, Efraim e le città che ne dipendevano.
\par 20 E Geroboamo, al tempo d'Abija, non ebbe più forza; e colpito dall'Eterno, egli morì.
\par 21 Ma Abija divenne potente, prese quattordici mogli, e generò ventidue figliuoli e sedici figliuole.
\par 22 Il resto delle azioni di Abija, la sua condotta e le sue parole, trovasi scritto nelle memorie del profeta Iddo.
\par 23 E Abija s'addormentò coi suoi padri, e fu sepolto nella città di Davide; e Asa, suo figliuolo, regnò in luogo suo; e al suo tempo il paese ebbe requie per dieci anni.

\chapter{14}

\par 1 Asa fece ciò ch'è buono e retto agli occhi dell'Eterno, del suo Dio.
\par 2 Tolse via gli altari degli dèi stranieri, e gli alti luoghi; spezzò le statue, abbatté gl'idoli d'Astarte;
\par 3 e ordinò a Giuda di cercare l'Eterno, l'Iddio de' suoi padri, e di mettere ad effetto la sua legge ed i suoi comandamenti.
\par 4 Tolse anche via da tutte le città di Giuda gli alti luoghi e le colonne solari; e, sotto di lui, il regno ebbe requie.
\par 5 Egli costruì delle città fortificate in Giuda, giacché il paese era tranquillo, e in quegli anni non v'era alcuna guerra contro di lui, perché l'Eterno gli avea data requie.
\par 6 Egli diceva a quei di Giuda: 'Costruiamo queste città, e circondiamole di mura, di torri, di porte e di sbarre; il paese è ancora a nostra disposizione, perché abbiamo cercato l'Eterno, il nostro Dio; noi l'abbiamo cercato, ed egli ci ha dato riposo d'ogni intorno'. Essi dunque si misero a costruire, e prosperarono.
\par 7 Asa aveva un esercito di trecentomila uomini di Giuda che portavano scudo e lancia, e di duecentottantamila di Beniamino che portavano scudo e tiravan d'arco, tutti uomini forti e valorosi.
\par 8 Zerah, l'Etiopo, uscì contro di loro con un esercito d'un milione d'uomini e trecento carri, e si avanzò fino a Maresha.
\par 9 Asa gli mosse contro, e si disposero in ordine di battaglia nella valle di Tsefatha presso Maresha.
\par 10 Allora Asa invocò l'Eterno, il suo Dio, e disse: 'O Eterno, per te non v'è differenza tra il dar soccorso a chi è in gran numero, e il darlo a chi è senza forza; soccorrici, o Eterno, o nostro Dio! poiché su te noi ci appoggiamo, e nel tuo nome siam venuti contro questa moltitudine. Tu sei l'Eterno, il nostro Dio; non la vinca l'uomo a petto di te!'
\par 11 E l'Eterno sconfisse gli Etiopi davanti ad Asa e davanti a Giuda, e gli Etiopi si diedero alla fuga.
\par 12 Ed Asa e la gente ch'era con lui li inseguirono fino a Gherar; e degli Etiopi ne caddero tanti, che non ne rimase più uno di vivo; poiché furono rotti davanti all'Eterno e davanti al suo esercito. E Asa ed i suoi portaron via un immenso bottino;
\par 13 e batteron tutte le città nei dintorni di Gherar, perché lo spavento dell'Eterno s'era impadronito d'esse; e quelli saccheggiarono tutte le città; perché v'era molto bottino;
\par 14 fecero pure man bassa sui chiusi delle mandre, e menaron via gran numero di pecore e di cammelli. Poi tornarono a Gerusalemme.

\chapter{15}

\par 1 Allora lo spirito di Dio s'impadronì di Azaria, figliuolo di Oded,
\par 2 il quale uscì ad incontrare Asa, e gli disse: 'Asa, e voi tutto Giuda e Beniamino, ascoltatemi! L'Eterno è con voi, quando voi siete con lui; se lo cercate, egli si farà trovare da voi; ma, se lo abbandonate, egli vi abbandonerà.
\par 3 Per lungo tempo Israele è stato senza vero Dio, senza sacerdote che lo ammaestrasse, e senza legge;
\par 4 ma nella sua distretta ei s'è convertito all'Eterno, all'Iddio d'Israele, l'ha cercato, ed egli s'è lasciato trovare da lui.
\par 5 In quel tempo, non v'era pace né per chi andava né per chi veniva; perché fra tutti gli abitanti de' vari paesi v'erano grandi agitazioni,
\par 6 ed essi erano schiacciati, nazione da nazione, e città da città; poiché Iddio li conturbava con ogni sorta di tribolazioni.
\par 7 Ma voi, siate forti, non vi lasciate illanguidire le braccia, perché l'opera vostra avrà la sua mercede'.
\par 8 Quando Asa ebbe udite queste parole, e la profezia del profeta Oded, prese animo, e fece sparire le abominazioni da tutto il paese di Giuda e di Beniamino, e dalle città che avea prese nella contrada montuosa d'Efraim; e ristabilì l'altare dell'Eterno, ch'era davanti al portico dell'Eterno.
\par 9 Poi radunò tutto Giuda e Beniamino, e quelli di Efraim, di Manasse e di Simeone, che dimoravano fra loro; giacché gran numero di quei d'Israele eran passati dalla sua parte, vedendo che l'Eterno, il suo Dio, era con lui.
\par 10 Essi dunque si radunarono a Gerusalemme il terzo mese del quindicesimo anno del regno d'Asa.
\par 11 E in quel giorno offrirono in sacrifizio all'Eterno, della preda che avean portata, settecento buoi e settemila pecore;
\par 12 e convennero nel patto di cercare l'Eterno, l'Iddio dei loro padri, con tutto il loro cuore e con tutta l'anima loro;
\par 13 e chiunque non cercasse l'Eterno, l'Iddio d'Israele, doveva esser messo a morte, grande o piccolo che fosse, uomo o donna.
\par 14 E si unirono per giuramento all'Eterno con gran voce e con acclamazioni, al suon delle trombe e dei corni.
\par 15 Tutto Giuda si rallegrò di questo giuramento; perché avean giurato di tutto cuore, avean cercato l'Eterno con grande ardore ed egli s'era lasciato trovare da loro. E l'Eterno diede loro requie d'ogn'intorno.
\par 16 Il re Asa destituì pure dalla dignità di regina sua madre Maaca, perch'essa avea rizzato un'immagine ad Astarte; e Asa abbatté l'immagine, la fece a pezzi e la bruciò presso al torrente Kidron.
\par 17 Nondimeno, gli alti luoghi non furono eliminati da Israele; quantunque il cuore d'Asa fosse integro, durante l'intera sua vita.
\par 18 Egli fece portare nella casa dell'Eterno le cose che suo padre avea consacrate, e quelle che avea consacrate egli stesso: argento, oro, vasi.
\par 19 E non ci fu più guerra alcuna fino al trentacinquesimo anno del regno di Asa.

\chapter{16}

\par 1 L'anno trentesimosesto del regno di Asa, Baasa, re d'Israele, salì contro Giuda, ed edificò Rama per impedire che alcuno andasse e venisse dalla parte di Asa, re di Giuda.
\par 2 Allora Asa trasse dell'argento e dell'oro dai tesori della casa dell'Eterno e della casa del re, e inviò dei messi a Ben-Hadad, re di Siria, che abitava a Damasco, per dirgli:
\par 3 'Siavi alleanza fra me e te, come vi fu tra il padre mio e il padre tuo. Ecco, io ti mando dell'argento e dell'oro; va', rompi la tua alleanza con Baasa, re d'Israele, ond'egli si ritiri da me'.
\par 4 Ben-Hadad diè ascolto al re Asa; mandò i capi del suo esercito contro le città d'Israele, i quali espugnarono Ijon, Dan, Abel-Maim, e tutte le città d'approvvigionamento di Neftali.
\par 5 E quando Baasa ebbe udito questo, cessò di edificare Rama, e sospese i suoi lavori.
\par 6 Allora il re Asa convocò tutti que' di Giuda, e quelli portaron via le pietre e il legname di cui Baasa s'era servito per la costruzione di Rama; e con essi Asa edificò Gheba e Mitspa.
\par 7 In quel tempo, Hanani, il veggente, si recò da Asa, re di Giuda, e gli disse: 'Poiché tu ti sei appoggiato sul re di Siria invece d'appoggiarti sull'Eterno, ch'è il tuo Dio, l'esercito del re di Siria è scampato dalle tue mani.
\par 8 Gli Etiopi ed i Libî non formavan essi un grande esercito con una moltitudine immensa di carri e di cavalieri? Eppure l'Eterno, perché tu t'eri appoggiato su lui, li diede nelle tue mani.
\par 9 Poiché l'Eterno scorre collo sguardo tutta la terra per spiegar la sua forza a pro di quelli che hanno il cuore integro verso di lui. In questo tu hai agito da insensato; poiché, da ora innanzi, avrai delle guerre'.
\par 10 Asa s'indignò contro il veggente, e lo fece mettere in prigione, tanto questa cosa lo aveva irritato contro di lui. E, al tempo stesso, Asa incrudelì anche contro alcuni del popolo.
\par 11 Or ecco, le azioni d'Asa, le prime e le ultime, si trovano scritte nel libro dei re di Giuda e d'Israele.
\par 12 Il trentanovesimo anno del suo regno, Asa ebbe una malattia ai piedi; la sua malattia fu gravissima; e, nondimeno, nella sua malattia non ricorse all'Eterno, ma ai medici.
\par 13 E Asa si addormentò coi suoi padri; morì il quarantunesimo anno del suo regno,
\par 14 e fu sepolto nel sepolcro ch'egli avea fatto scavare per sé nella città di Davide. Fu steso sopra un letto pieno di profumi e di varie sorta d'aromi composti con arte di profumiere; e ne bruciarono una grandissima quantità in onor suo.

\chapter{17}

\par 1 Giosafat, figliuolo di Asa, regnò in luogo di lui, e si fortificò contro Israele;
\par 2 collocò dei presidî in tutte le città fortificate di Giuda, e pose delle guarnigioni nel paese di Giuda e nelle città di Efraim, che Asa suo padre avea conquistate.
\par 3 E l'Eterno fu con Giosafat, perch'egli camminò nelle vie che Davide suo padre avea seguite da principio, e cercò, non i Baali,
\par 4 ma l'Iddio di suo padre; e si condusse secondo i suoi comandamenti, senza imitare quel che faceva Israele.
\par 5 Perciò l'Eterno assicurò il possesso del regno nelle mani di Giosafat; tutto Giuda gli recava dei doni, ed egli ebbe ricchezza e gloria in abbondanza.
\par 6 Il suo coraggio crebbe, seguendo le vie dell'Eterno; e fece anche sparire da Giuda gli alti luoghi e gl'idoli d'Astarte.
\par 7 Il terzo anno del suo regno mandò i suoi capi Ben-Hail, Obadia, Zaccaria, Natanaele e Micaiah, a insegnare nelle città di Giuda;
\par 8 e con essi mandò i Leviti Scemaia, Nethania, Zebadia, Asael, Scemiramoth, Gionathan, Adonia, Tobia e Tob-Adonia, e i sacerdoti Elishama e Jehoram.
\par 9 Ed essi insegnarono in Giuda, avendo seco il libro della legge dell'Eterno; percorsero tutte le città di Giuda, e istruirono il popolo.
\par 10 Il terrore dell'Eterno s'impadronì di tutti i regni dei paesi che circondavano Giuda, sì che non mossero guerra a Giosafat.
\par 11 E una parte de' Filistei recò a Giosafat dei doni, e un tributo in argento; anche gli Arabi gli menarono del bestiame: settemilasettecento montoni e settemilasettecento capri.
\par 12 Giosafat raggiunse un alto grado di grandezza, ed edificò in Giuda castelli e città d'approvvigionamento.
\par 13 Fece eseguire molti lavori nelle città di Giuda, ed ebbe a Gerusalemme de' guerrieri, uomini forti e valorosi.
\par 14 Eccone il censimento secondo le loro case patriarcali. Di Giuda: capi di migliaia: Adna, il capo, con trecentomila uomini forti e valorosi;
\par 15 dopo di lui, Johanan, il capo, con duecentottantamila uomini;
\par 16 dopo questo, Amasia, figliuolo di Zicri, il quale s'era volontariamente consacrato all'Eterno, con duecentomila uomini forti e valorosi.
\par 17 Di Beniamino: Eliada, uomo forte e valoroso, con duecentomila uomini, armati d'arco e di scudo;
\par 18 e, dopo di lui, Jozabad con centottantamila uomini pronti per la guerra.
\par 19 Tutti questi erano al servizio del re, senza contare quelli ch'egli avea collocati nelle città fortificate, in tutto il paese di Giuda.

\chapter{18}

\par 1 Giosafat ebbe ricchezze e gloria in abbondanza, e contrasse parentela con Achab.
\par 2 In capo a qualche anno, scese a Samaria da Achab; e Achab fece uccidere per lui e per la gente ch'era con lui un gran numero di pecore e di buoi, e lo indusse a salir seco contro Ramoth di Galaad.
\par 3 Achab, re d'Israele, disse a Giosafat, re di Giuda: 'Vuoi venire con me a Ramoth di Galaad?' Giosafat gli rispose: 'Fa' conto di me come di te stesso, della mia gente come della tua, e verremo con te alla guerra'.
\par 4 E Giosafat disse al re d'Israele: 'Ti prego, consulta oggi la parola dell'Eterno'.
\par 5 Allora il re d'Israele radunò i profeti, in numero di quattrocento, e disse loro: 'Dobbiam noi andare a far guerra a Ramoth di Galaad, o no?' Quelli risposero: 'Va', e Dio la darà nelle mani del re'.
\par 6 Ma Giosafat disse: 'Non v'ha egli qui alcun altro profeta dell'Eterno da poter consultare?'
\par 7 Il re d'Israele rispose a Giosafat: 'V'è ancora un uomo per mezzo del quale si potrebbe consultare l'Eterno; ma io l'odio perché non mi predice mai nulla di buono, ma sempre del male: è Micaiah, figliuolo d'Imla'. E Giosafat disse: 'Il re non dica così'.
\par 8 Allora il re d'Israele chiamò un eunuco, e gli disse: 'Fa' venir presto Micaiah, figliuolo d'Imla'.
\par 9 Or il re d'Israele e Giosafat, re di Giuda, sedevano ciascuno sul suo trono, vestiti de' loro abiti reali, nell'aia ch'è all'ingresso della porta di Samaria; e tutti i profeti profetavano dinanzi ad essi.
\par 10 Sedekia, figliuolo di Kenaana, s'era fatto delle corna di ferro, e disse: 'Così dice l'Eterno: - Con queste corna darai di cozzo nei Sirî finché tu li abbia completamente distrutti'.
\par 11 E tutti i profeti profetavano nello stesso modo, dicendo: 'Sali contro Ramoth di Galaad, e vincerai; l'Eterno la darà nelle mani del re'.
\par 12 Or il messo ch'era andato a chiamar Micaiah, gli parlò così: 'Ecco, i profeti tutti, ad una voce, predicono del bene al re; ti prego, sia il tuo parlare come quello d'ognun d'essi, e predici del bene!'
\par 13 Ma Micaiah rispose: 'Com'è vero che l'Eterno vive, io dirò quel che l'Eterno mi dirà'.
\par 14 E, come fu giunto dinanzi al re, il re gli disse: 'Micaiah, dobbiamo noi andare a far guerra a Ramoth di Galaad, o no?' Quegli rispose: 'Andate pure, e vincerete; i nemici saranno dati nelle vostre mani'.
\par 15 E il re gli disse: 'Quante volte dovrò io scongiurarti di non dirmi se non la verità nel nome dell'Eterno?'
\par 16 Micaiah rispose: 'Ho veduto tutto Israele disperso su per i monti, come pecore che non hanno pastore; e l'Eterno ha detto: - Questa gente non ha padrone; se ne torni ciascuno in pace a casa sua'. -
\par 17 E il re d'Israele disse a Giosafat: 'Non te l'ho io detto che costui non mi predirebbe nulla di buono, ma soltanto del male?'
\par 18 E Micaiah replicò: 'Perciò ascoltate la parola dell'Eterno. Io ho veduto l'Eterno che sedeva sul suo trono, e tutto l'esercito celeste che gli stava a destra e a sinistra.
\par 19 E l'Eterno disse: - Chi sedurrà Achab, re d'Israele, affinché salga a Ramoth di Galaad e vi perisca? - E uno rispose in un modo e l'altro in un altro.
\par 20 Allora si fece avanti uno spirito, il quale si presentò dinanzi all'Eterno, e disse: - Lo sedurrò io. - L'Eterno gli disse: - E come? -
\par 21 Quegli rispose: - Io uscirò, e sarò spirito di menzogna in bocca a tutti i suoi profeti. - L'Eterno gli disse: - Sì, riuscirai a sedurlo; esci, e fa' così. -
\par 22 Ed ora ecco che l'Eterno ha posto uno spirito di menzogna in bocca a questi tuoi profeti; ma l'Eterno ha pronunziato del male contro di te'.
\par 23 Allora Sedekia, figliuolo di Kenaana, si accostò, diede uno schiaffo a Micaiah, e disse: 'Per dove è passato lo spirito dell'Eterno quand'è uscito da me per parlare a te?'
\par 24 Micaiah rispose: 'Lo vedrai il giorno che andrai di camera in camera per nasconderti!'
\par 25 E il re d'Israele disse ai suoi servi: 'Prendete Micaiah, menatelo da Amon, governatore della città, e da Joas, figliuolo del re,
\par 26 e dite loro: - Così dice il re: Mettete costui in prigione, nutritelo di pan d'afflizione e d'acqua d'afflizione, finch'io ritorni sano e salvo'.
\par 27 E Micaiah disse: 'Se tu ritorni sano e salvo, non sarà l'Eterno quegli che avrà parlato per bocca mia'. E aggiunse: 'Udite questo, o voi, popoli tutti!'
\par 28 Il re d'Israele e Giosafat, re di Giuda, saliron dunque contro Ramoth di Galaad.
\par 29 E il re d'Israele disse a Giosafat: 'Io mi travestirò per andar in battaglia; ma tu mettiti i tuoi abiti reali'. Il re d'Israele si travestì, e andarono in battaglia.
\par 30 Or il re di Siria avea dato quest'ordine ai capitani dei suoi carri: 'Non combattete contro veruno, piccolo o grande, ma contro il solo re d'Israele'.
\par 31 E quando i capitani dei carri scòrsero Giosafat, dissero: 'Quello è il re d'Israele'; e lo circondarono per attaccarlo; ma Giosafat mandò un grido, e l'Eterno lo soccorse; e Dio li attirò lungi da lui.
\par 32 E allorché i capitani dei carri s'accorsero ch'egli non era il re d'Israele, cessarono d'assalirlo.
\par 33 Or qualcuno scoccò a caso la freccia del suo arco, e ferì il re d'Israele tra la corazza e le falde; onde il re disse al suo cocchiere: 'Vòlta, menami fuori del campo, perché son ferito'.
\par 34 Ma la battaglia fu così accanita quel giorno, che il re fu trattenuto sul suo carro in faccia ai Sirî fino alla sera, e sul tramontare del sole morì.

\chapter{19}

\par 1 Giosafat, re di Giuda, tornò sano e salvo a casa sua a Gerusalemme.
\par 2 E il veggente Jehu, figliuolo di Hanani, andò incontro a Giosafat, e gli disse: 'Dovevi tu dare aiuto ad un empio e amar quelli che odiano l'Eterno? Per questo fatto hai attirato su di te l'ira dell'Eterno.
\par 3 Nondimeno si son trovate in te delle buone cose, giacché hai fatti sparire dal paese gl'idoli d'Astarte, e hai applicato il cuor tuo alla ricerca di Dio'.
\par 4 Giosafat rimase a Gerusalemme; poi fece di nuovo un giro fra il popolo, da Beer-Sceba alla contrada montuosa d'Efraim, e lo ricondusse all'Eterno, all'Iddio de' suoi padri.
\par 5 E stabilì dei giudici nel paese, in tutte le città fortificate di Giuda, città per città, e disse ai giudici:
\par 6 'Badate bene a quello che fate; poiché voi amministrate la giustizia, non per servire ad un uomo ma per servire all'Eterno; il quale sarà con voi negli affari della giustizia.
\par 7 Or dunque il timor dell'Eterno sia in voi; agite con circospezione, poiché presso l'Eterno, ch'è l'Iddio nostro, non v'è né perversità, né riguardo a qualità di persone, né accettazione di doni'.
\par 8 Giosafat, tornato che fu a Gerusalemme, stabilì anche quivi dei Leviti, dei sacerdoti e dei capi delle case patriarcali d'Israele per render giustizia nel nome dell'Eterno, e per sentenziare nelle liti.
\par 9 E diede loro i suoi ordini, dicendo: 'Voi farete così, con timore dell'Eterno, con fedeltà e con cuore integro:
\par 10 In qualunque lite che vi sia portata dinanzi dai vostri fratelli dimoranti nelle loro città, sia che si tratti d'un omicidio o d'una legge o d'un comandamento o d'uno statuto o d'un precetto, illuminateli, affinché non si rendano colpevoli verso l'Eterno, e l'ira sua non piombi su voi e sui vostri fratelli. Così facendo, voi non vi renderete colpevoli.
\par 11 Ed ecco, il sommo sacerdote Amaria vi sarà preposto per tutti gli affari che concernono l'Eterno; e Zebadia, figliuolo d'Ismaele, capo della casa di Giuda, per tutti gli affari che concernono il re; e avete a vostra disposizione dei Leviti, come magistrati. Fatevi cuore, mettetevi all'opra, e l'Eterno sia con l'uomo dabbene!'

\chapter{20}

\par 1 Dopo queste cose, i figliuoli di Moab, e i figliuoli di Ammon, e con loro de' Maoniti, mossero contro Giosafat per fargli guerra.
\par 2 E vennero dei messi a informare Giosafat, dicendo: 'Una gran moltitudine s'avanza contro di te dall'altra parte del mare, dalla Siria, ed è giunta a Hatsatson-Thamar', che è En-Ghedi.
\par 3 E Giosafat ebbe paura, si dispose a cercare l'Eterno, e bandì un digiuno per tutto Giuda.
\par 4 Giuda si radunò per implorare aiuto dall'Eterno, e da tutte quante le città di Giuda venivan gli abitanti a cercare l'Eterno.
\par 5 E Giosafat, stando in piè in mezzo alla raunanza di Giuda e di Gerusalemme, nella casa dell'Eterno, davanti al cortile nuovo, disse:
\par 6 'O Eterno, Dio de' nostri padri, non sei tu l'Iddio dei cieli? e non sei tu che signoreggi su tutti i regni delle nazioni? e non hai tu nelle tue mani la forza e la potenza, in guisa che nessuno ti può resistere?
\par 7 Non sei tu quegli, o Dio nostro, che cacciasti gli abitanti di questo paese d'innanzi al tuo popolo d'Israele, e lo desti per sempre alla progenie d'Abrahamo, il quale ti amò?
\par 8 E quelli l'hanno abitato e v'hanno edificato un santuario per il tuo nome, dicendo:
\par 9 Quando c'incolga qualche calamità, spada, giudizio, peste o carestia, noi ci presenteremo dinanzi a questa casa e dinanzi a te, poiché il tuo nome è in questa casa; e a te grideremo nella nostra tribolazione, e tu ci udrai e ci salverai.
\par 10 Ed ora ecco che i figliuoli d'Ammon e di Moab e quei del monte di Seir, nelle terre dei quali non permettesti ad Israele d'entrare quando veniva dal paese d'Egitto, ed egli li lasciò da parte e non li distrusse,
\par 11 eccoli che ora ci ricompensano, venendo a cacciarci dalla eredità di cui ci hai dato il possesso.
\par 12 O Dio nostro, non farai tu giudizio di costoro? Poiché noi siamo senza forza, di fronte a questa gran moltitudine che s'avanza contro di noi; e non sappiamo che fare, ma gli occhi nostri sono su te!' -
\par 13 E tutto Giuda, perfino i bambini, le mogli, i figliuoli, stavano in piè davanti all'Eterno.
\par 14 Allora lo spirito dell'Eterno investì in mezzo alla raunanza Jahaziel, figliuolo di Zaccaria, figliuolo di Benaia, figliuolo di Jeiel, figliuolo di Mattania, il Levita, di tra i figliuoli d'Asaf.
\par 15 E Jahaziel disse: 'Porgete orecchio, voi tutti di Giuda, e voi abitanti di Gerusalemme, e tu, o re Giosafat! Così vi dice l'Eterno: - Non temete e non vi sgomentate a motivo di questa gran moltitudine; poiché questa non è battaglia vostra, ma di Dio.
\par 16 Domani, scendete contro di loro; eccoli che vengon su per la salita di Tsits, e voi li troverete all'estremità della valle, dirimpetto al deserto di Jeruel.
\par 17 Questa battaglia non l'avete a combatter voi: presentatevi, tenetevi fermi, e vedrete la liberazione che l'Eterno vi darà. O Giuda, o Gerusalemme, non temete e non vi sgomentate; domani, uscite contro di loro, e l'Eterno sarà con voi'. -
\par 18 Allora Giosafat chinò la faccia a terra, e tutto Giuda e gli abitanti di Gerusalemme si prostrarono dinanzi all'Eterno e l'adorarono.
\par 19 E i Leviti di tra i figliuoli dei Kehathiti e di tra i figliuoli dei Korahiti si levarono per lodare ad altissima voce l'Eterno, l'Iddio d'Israele.
\par 20 La mattina seguente si levarono di buon'ora, e si misero in cammino verso il deserto di Tekoa; e come si mettevano in cammino, Giosafat, stando in piedi, disse: 'Ascoltatemi, o Giuda, e voi abitanti di Gerusalemme! Credete nell'Eterno, ch'è l'Iddio vostro, e sarete al sicuro; credete ai suoi profeti, e trionferete!'
\par 21 E dopo aver tenuto consiglio col popolo, stabilì dei cantori che, vestiti in santa magnificenza, cantassero le lodi dell'Eterno, e camminando alla testa dell'esercito, dicessero: 'Celebrate l'Eterno, perché la sua benignità dura in perpetuo!'
\par 22 E com'essi cominciavano i canti di gioia e di lode, l'Eterno tese un'imboscata contro i figliuoli di Ammon e di Moab e contro quelli del monte Seir ch'eran venuti contro Giuda; e rimasero sconfitti.
\par 23 I figliuoli di Ammon e di Moab assalirono gli abitanti del monte di Seir per votarli allo sterminio e distruggerli; e quand'ebbero annientati gli abitanti di Seir, si diedero a distruggersi a vicenda.
\par 24 E quando que' di Giuda furon giunti sull'altura donde si scorge il deserto, volsero lo sguardo verso la moltitudine, ed ecco i cadaveri che giacevano a terra; nessuno era scampato.
\par 25 Allora Giosafat e la sua gente andarono a far bottino delle loro spoglie; e fra i cadaveri trovarono abbondanza di ricchezze, di vesti e di oggetti preziosi; e se ne appropriarono più che ne potessero portare; tre giorni misero a portar via il bottino, tant'era copioso.
\par 26 Il quarto giorno si radunarono nella Valle di Benedizione, dove benedissero l'Eterno; per questo, quel luogo è stato chiamato Valle di Benedizione fino al dì d'oggi.
\par 27 Tutti gli uomini di Giuda e di Gerusalemme, con a capo Giosafat, partirono con gioia per tornare a Gerusalemme, perché l'Eterno li avea ricolmi d'allegrezza, liberandoli dai loro nemici.
\par 28 Ed entrarono in Gerusalemme e nella casa dell'Eterno al suono de' saltèri, delle cetre e delle trombe.
\par 29 E il terrore di Dio s'impadronì di tutti i regni degli altri paesi, quando udirono che l'Eterno avea combattuto contro i nemici d'Israele.
\par 30 E il regno di Giosafat ebbe requie; il suo Dio gli diede pace d'ogni intorno.
\par 31 Così Giosafat regnò sopra Giuda. Avea trentacinque anni quando cominciò a regnare, e regnò venticinque anni a Gerusalemme; e il nome di sua madre era Azuba, figliuola di Scilhi.
\par 32 Egli camminò per le vie di Asa suo padre, e non se ne allontanò, facendo quel ch'è giusto agli occhi dell'Eterno.
\par 33 Nondimeno gli alti luoghi non scomparvero, perché il popolo non aveva ancora il cuore fermamente unito all'Iddio dei suoi padri.
\par 34 Or il rimanente delle azioni di Giosafat, le prime e le ultime, si trovano scritte nella Storia di Jehu, figliuolo di Hanani, inserita nel libro dei re d'Israele.
\par 35 Dopo questo, Giosafat, re di Giuda, si associò col re d'Israele Achazia, che aveva una condotta empia;
\par 36 e se lo associò, per costruire delle navi che andassero a Tarsis; e le costruirono ad Etsion-Gheber.
\par 37 Allora Eliezer, figliuolo di Dodava da Maresha, profetizzò contro Giosafat, dicendo: 'Perché ti sei associato con Achazia, l'Eterno ha disperse le opere tue'. E le navi furono infrante, e non poterono

\chapter{21}

\par 1 fare il viaggio di Tarsis. E Giosafat s'addormentò coi suoi padri, e con essi fu sepolto nella città di Davide; e Jehoram, suo figliuolo, regnò in luogo suo.
\par 2 Jehoram avea de' fratelli, figliuoli di Giosafat: Azaria, Jehiel, Zaccaria, Azariahu, Micael e Scefatia; tutti questi erano figliuoli di Giosafat, re d'Israele;
\par 3 e il padre loro avea fatto ad essi grandi doni d'argento, d'oro e di cose preziose, con delle città fortificate in Giuda, ma avea lasciato il regno a Jehoram, perch'era il primogenito.
\par 4 Or quando Jehoram ebbe preso possesso del regno di suo padre e vi si fu solidamente stabilito, fece morir di spada tutti i suoi fratelli, come pure alcuni dei capi d'Israele.
\par 5 Jehoram avea trentadue anni quando cominciò a regnare, e regnò otto anni in Gerusalemme.
\par 6 E camminò per la via dei re d'Israele come avea fatto la casa di Achab, poiché avea per moglie una figliuola di Achab; e fece ciò ch'è male agli occhi dell'Eterno.
\par 7 Nondimeno l'Eterno non volle distrugger la casa di Davide, a motivo del patto che avea fermato con Davide, e della promessa che avea fatta di lasciar sempre una lampada a lui ed ai suoi figliuoli.
\par 8 Ai tempi di lui, Edom si ribellò, sottraendosi al giogo di Giuda, e si dette un re.
\par 9 Allora Jehoram partì coi suoi capi e con tutti i suoi carri; e, levatosi di notte, sconfisse gli Edomiti che l'aveano circondato, e i capi dei carri.
\par 10 Così Edom si è ribellato sottraendosi al giogo di Giuda fino al dì d'oggi. In quel medesimo tempo, anche Libna si ribellò e si sottrasse al giogo di Giuda, perché Jehoram aveva abbandonato l'Eterno, l'Iddio de' suoi padri.
\par 11 Jehoram fece anch'egli degli alti luoghi sui monti di Giuda, spinse gli abitanti di Gerusalemme alla prostituzione, e sviò Giuda.
\par 12 E gli giunse uno scritto da parte del profeta Elia, che diceva: 'Così dice l'Eterno, l'Iddio di Davide tuo padre: - Perché tu non hai camminato per le vie di Giosafat, tuo padre, e per le vie d'Asa, re di Giuda,
\par 13 ma hai camminato per la via dei re d'Israele; perché hai spinto alla prostituzione Giuda e gli abitanti di Gerusalemme; come la casa di Achab v'ha spinto Israele, e perché hai ucciso i tuoi fratelli, membri della famiglia di tuo padre, ch'eran migliori di te,
\par 14 ecco, l'Eterno colpirà con una gran piaga il tuo popolo, i tuoi figliuoli, le tue mogli, e tutto quello che t'appartiene;
\par 15 e tu avrai una grave malattia, una malattia d'intestini, che s'inasprirà di giorno in giorno, finché gl'intestini ti vengan fuori per effetto del male'.
\par 16 E l'Eterno risvegliò contro Jehoram lo spirito de' Filistei e degli Arabi, che confinano con gli Etiopi;
\par 17 ed essi salirono contro Giuda, l'invasero, e portaron via tutte le ricchezze che si trovavano nella casa del re, e anche i suoi figliuoli e le sue mogli, in guisa che non gli rimase altro figliuolo se non Joachaz, ch'era il più piccolo.
\par 18 Dopo tutto questo l'Eterno lo colpì con una malattia incurabile d'intestini.
\par 19 E, con l'andar del tempo, verso la fine del secondo anno, gl'intestini gli venner fuori, in sèguito alla malattia; e morì, in mezzo ad atroci sofferenze; e il suo popolo non bruciò profumi in onore di lui, come avea fatto per i suoi padri.
\par 20 Aveva trentadue anni quando cominciò a regnare, e regnò otto anni in Gerusalemme. Se ne andò senza esser rimpianto, e fu sepolto nella città di Davide, ma non nei sepolcri dei re.

\chapter{22}

\par 1 Gli abitanti di Gerusalemme, in luogo di Jehoram, proclamarono re Achazia, il più giovine de' suoi figliuoli; poiché la truppa ch'era entrata con gli Arabi nel campo, aveva ucciso tutti i più grandi d'età. Così regnò Achazia, figliuolo di Jehoram, re di Giuda.
\par 2 Achazia avea quarantadue anni quando cominciò a regnare, e regnò un anno in Gerusalemme. Sua madre si chiamava Athalia, figliuola di Omri.
\par 3 Anch'egli camminò per le vie della casa di Achab, perché sua madre, ch'era sua consigliera, lo spingeva ad agire empiamente.
\par 4 Egli fece ciò ch'è male agli occhi dell'Eterno, come quei della casa di Achab, perché, dopo la morte di suo padre, questi furono suoi consiglieri, per sua rovina.
\par 5 E fu pure dietro loro consiglio ch'egli andò con Jehoram, figliuolo di Achab re d'Israele, a combattere contro Hazael, re di Siria, a Ramoth di Galaad, e i Sirî ferirono Joram;
\par 6 e questi tornò a Jzreel per farsi curare delle ferite che avea ricevute dai Sirî a Ramah, quando combatteva contro Hazael, re di Siria. Ed Achazia, figliuolo di Jehoram re di Giuda, scese ad Jzreel a vedere Jehoram, figliuolo di Achab, perché questi era ammalato.
\par 7 Or fu volontà di Dio che Achazia, per sua rovina, si recasse da Joram; perché quando fu giunto, uscì con Jehoram contro Jehu, figliuolo di Nimsci, che l'Eterno aveva unto per sterminare la casa di Achab;
\par 8 e come Jehu facea giustizia della casa di Achab, trovò i capi di Giuda e i figliuoli de' fratelli di Achazia ch'erano al servizio di Achazia, e li uccise.
\par 9 E fe' cercare Achazia, che s'era nascosto in Samaria; e Achazia fu preso, menato a Jehu, messo a morte, e poi seppellito; perché si diceva: 'È il figliuolo di Giosafat, che cercava l'Eterno con tutto il cuor suo'. E nella casa di Achazia non rimase più alcuno che fosse capace di regnare.
\par 10 Or quando Athalia, madre di Achazia, vide che il suo figliuolo era morto, si levò e distrusse tutta la stirpe reale della casa di Giuda.
\par 11 Ma Jehoshabet, figliuola del re, prese Joas, figliuolo di Achazia, lo trafugò di mezzo ai figliuoli del re ch'eran messi a morte, e lo pose con la sua balia nella camera dei letti. Così Jehoshabet, figliuola del re Jehoram, moglie del sacerdote Jehoiada (era sorella d'Achazia), lo nascose alle ricerche d'Athalia, che non lo mise a morte.
\par 12 Ed egli rimase nascosto presso di loro nella casa di Dio per sei anni; intanto, Athalia regnava sul paese.

\chapter{23}

\par 1 Il settimo anno, Jehoiada, fattosi animo, fece lega coi capi-centurie Azaria figliuolo di Jehoram, Ismaele figliuolo di Johanan, Azaria figliuolo di Obed, Maaseia figliuolo di Adaia, ed Elishafat, figliuolo di Zicri.
\par 2 Essi percorsero Giuda, radunarono i Leviti di tutte le città di Giuda e i capi delle case patriarcali d'Israele, e vennero a Gerusalemme.
\par 3 E tutta la raunanza strinse lega col re nella casa di Dio. E Jehoiada disse loro: 'Ecco, il figliuolo del re regnerà, come l'Eterno ha promesso relativamente ai figliuoli di Davide.
\par 4 Ecco quello che voi farete: un terzo di quelli tra voi che entrano in servizio il giorno del sabato, sacerdoti e Leviti, starà di guardia alle porte del tempio;
\par 5 un altro terzo starà nella casa del re, e l'altro terzo alla porta di Jesod. Tutto il popolo starà nei cortili della casa dell'Eterno.
\par 6 Ma nessuno entri nella casa dell'Eterno, tranne i sacerdoti e i Leviti di servizio; questi entreranno, perché son consacrati; ma tutto il popolo s'atterrà all'ordine dell'Eterno.
\par 7 I Leviti circonderanno il re, da ogni lato, ognuno colle armi alla mano; e chiunque cercherà di penetrare nella casa di Dio, sia messo a morte; e voi starete col re, quando entrerà e quando uscirà'.
\par 8 I Leviti e tutto Giuda eseguirono tutti gli ordini dati dal sacerdote Jehoiada; ognun d'essi prese i suoi uomini: quelli che entravano in servizio il giorno del sabato, e quelli che uscivan di servizio il giorno del sabato; poiché il sacerdote Jehoiada non avea licenziato le mute uscenti.
\par 9 Il sacerdote Jehoiada diede ai capi-centurie le lance, le targhe e gli scudi che aveano appartenuto a Davide e si trovavano nella casa di Dio.
\par 10 E dispose tutto il popolo attorno al re, ciascuno con l'arma in mano, dal lato destro al lato sinistro della casa, presso l'altare e presso la casa.
\par 11 Allora menaron fuori il figliuolo del re, gli posero in testa il diadema, gli consegnarono la legge, e lo proclamarono re; Jehoiada e i suoi figliuoli lo unsero, ed esclamarono: 'Viva il re!'
\par 12 Or quando Athalia udì il rumore del popolo che accorreva ed acclamava il re, andò verso il popolo nella casa dell'Eterno;
\par 13 guardò, ed ecco che il re stava in piedi sul suo palco, all'ingresso; i capitani e i trombettieri erano accanto al re; tutto il popolo del paese era in festa e sonava le trombe; e i cantori, coi loro strumenti musicali, dirigevano i canti di lode. Allora Athalia si stracciò le vesti, e gridò: 'Congiura! congiura!'
\par 14 Ma il sacerdote Jehoiada fece venir fuori i capi-centurie che comandavano l'esercito, e disse loro: 'Fatela uscire di tra le file; e chiunque la seguirà sia ucciso di spada!' Poiché il sacerdote avea detto: 'Non sia messa a morte nella casa dell'Eterno'.
\par 15 Così quelli le fecero largo, ed ella giunse alla casa del re per la strada della porta dei cavalli; e quivi fu uccisa.
\par 16 E Jehoiada fermò tra sé, tutto il popolo ed il re, il patto, per il quale Israele doveva essere il popolo dell'Eterno.
\par 17 E tutto il popolo entrò nel tempio di Baal, e lo demolì; fece interamente in pezzi i suoi altari e le sue immagini, e uccise dinanzi agli altari Mattan, sacerdote di Baal.
\par 18 Poi Jehoiada affidò la sorveglianza della casa dell'Eterno ai sacerdoti levitici, che Davide avea ripartiti in classi preposte alla casa dell'Eterno per offrire olocausti all'Eterno, com'è scritto nella legge di Mosè, con gioia e con canto di lodi, secondo le disposizioni di Davide.
\par 19 E collocò i portinai alle porte della casa dell'Eterno, affinché nessuno v'entrasse che fosse impuro per qualsivoglia ragione.
\par 20 E prese i capi-centurie, gli uomini ragguardevoli, quelli che avevano autorità sul popolo e tutto il popolo del paese, e fece scendere il re dalla casa dell'Eterno. Entrarono nella casa del re per la porta superiore, e fecero sedere il re sul trono reale.
\par 21 E tutto il popolo del paese fu in festa e la città rimase tranquilla, quando Athalia fu uccisa di spada.

\chapter{24}

\par 1 Joas aveva sette anni quando cominciò a regnare, e regnò quarant'anni a Gerusalemme. Sua madre si chiamava Tsibia da Beer-Sceba.
\par 2 Joas fece ciò ch'è giusto agli occhi dell'Eterno durante tutto il tempo che visse il sacerdote Jehoiada.
\par 3 E Jehoiada prese per lui due mogli, dalle quali egli ebbe de' figliuoli e delle figliuole.
\par 4 Dopo queste cose venne in cuore a Joas di restaurare la casa dell'Eterno.
\par 5 Radunò i sacerdoti e i Leviti, e disse loro: 'Andate per le città di Giuda, e raccogliete anno per anno in tutto Israele del danaro per restaurare la casa dell'Iddio vostro; e guardate di sollecitar la cosa'. Ma i Leviti non s'affrettarono.
\par 6 Allora il re chiamò Jehoiada loro capo e gli disse: 'Perché non hai tu procurato che i Leviti portassero da Giuda e da Gerusalemme la tassa che Mosè, servo dell'Eterno, e la raunanza d'Israele stabilirono per la tenda della testimonianza?'
\par 7 Poiché i figliuoli di quella scellerata donna d'Athalia aveano saccheggiato la casa di Dio e aveano perfino adoperato per i Baali tutte le cose consacrate della casa dell'Eterno.
\par 8 Il re dunque comandò che si facesse una cassa e che la si mettesse fuori, alla porta della casa dell'Eterno.
\par 9 Poi fu intimato in Giuda e in Gerusalemme che si portasse all'Eterno la tassa che Mosè, servo di Dio, aveva imposta ad Israele nel deserto.
\par 10 E tutti i capi e tutto il popolo se ne rallegrarono e portarono il danaro e lo gettarono nella cassa finché tutti ebbero pagato.
\par 11 Or quand'era il momento che i Leviti doveano portar la cassa agl'ispettori reali, perché vedevano che v'era molto danaro, il segretario del re e il commissario del sommo sacerdote venivano a vuotare la cassa; la prendevano, poi la riportavano al suo posto; facevan così ogni giorno, e raccolsero danaro in abbondanza.
\par 12 E il re e Jehoiada lo davano a quelli incaricati d'eseguire i lavori della casa dell'Eterno; e questi pagavano degli scalpellini e de' legnaiuoli per restaurare la casa dell'Eterno, e anche de' lavoratori di ferro e di rame per restaurare la casa dell'Eterno.
\par 13 Così gl'incaricati dei lavori si misero all'opera, e per le loro mani furon compiute le riparazioni; essi rimisero la casa di Dio in buono stato, e la consolidarono.
\par 14 E, quand'ebbero finito, portarono davanti al re e davanti a Jehoiada il rimanente del danaro, col quale si fecero degli utensili per la casa dell'Eterno: degli utensili per il servizio e per gli olocausti, delle coppe, e altri utensili d'oro e d'argento. E durante tutta la vita di Jehoiada, si offrirono del continuo olocausti nella casa dell'Eterno.
\par 15 Ma Jehoiada, fattosi vecchio e sazio di giorni, morì; quando morì, avea centotrent'anni;
\par 16 e fu sepolto nella città di Davide coi re, perché avea fatto del bene in Israele, per il servizio di Dio e della sua casa.
\par 17 Dopo la morte di Jehoiada, i capi di Giuda vennero al re e si prostrarono dinanzi a lui; allora il re diè loro ascolto;
\par 18 ed essi abbandonarono la casa dell'Eterno, dell'Iddio dei loro padri, servirono gl'idoli d'Astarte e gli altri idoli; e questa loro colpa trasse l'ira dell'Eterno su Giuda e su Gerusalemme.
\par 19 L'Eterno mandò loro bensì dei profeti per ricondurli a sé e questi protestarono contro la loro condotta, ma essi non vollero ascoltarli.
\par 20 Allora lo spirito di Dio investì Zaccaria, figliuolo del sacerdote Jehoiada, il quale, in piè, dominando il popolo, disse loro: 'Così dice Iddio: - Perché trasgredite voi i comandamenti dell'Eterno? Voi non prospererete; poiché avete abbandonato l'Eterno, anch'egli vi abbandonerà'.
\par 21 Ma quelli fecero una congiura contro di lui, e lo lapidarono per ordine del re, nel cortile della casa dell'Eterno.
\par 22 E il re Joas non si ricordò della benevolenza usata verso lui da Jehoiada, padre di Zaccaria, e gli uccise il figliuolo; il quale, morendo, disse: 'L'Eterno lo veda e ne ridomandi conto!'
\par 23 E avvenne che, scorso l'anno, l'esercito dei Sirî salì contro Joas, e venne in Giuda e a Gerusalemme. Essi misero a morte fra il popolo tutti i capi, e ne mandarono tutte le spoglie al re di Damasco.
\par 24 E benché l'esercito dei Sirî fosse venuto con piccolo numero d'uomini, pure l'Eterno diè loro nelle mani un esercito grandissimo, perché quelli aveano abbandonato l'Eterno, l'Iddio dei loro padri. Così i Sirî fecero giustizia di Joas.
\par 25 E quando questi si furon partiti da lui, lasciandolo in gravi sofferenze, i suoi servi ordirono contro di lui una congiura, perch'egli avea versato il sangue dei figliuoli del sacerdote Jehoiada, e lo uccisero nel suo letto. Così morì, e fu sepolto nella città di Davide, ma non nei sepolcri dei re.
\par 26 Quelli che congiurarono contro di lui furono Zabad, figliuolo di Scimeath, una Ammonita, e Jozabad, figliuolo di Scimrith, una Moabita.
\par 27 Or quanto concerne i suoi figliuoli, il gran numero di tributi impostigli e il restauro della casa di Dio, si trova scritto nelle memorie del libro dei re. E Amatsia, suo figliuolo, regnò in luogo suo.

\chapter{25}

\par 1 Amatsia avea venticinque anni quando cominciò a regnare, e regnò ventinove anni a Gerusalemme. Sua madre si chiamava Jehoaddan, da Gerusalemme.
\par 2 Egli fece ciò ch'è giusto agli occhi dell'Eterno, ma non di tutto cuore.
\par 3 Or come il regno fu bene assicurato nelle sue mani, egli fece morire quei servi suoi che aveano ucciso il re suo padre.
\par 4 Ma non fece morire i loro figliuoli, conformandosi a quello ch'è scritto nella legge, nel libro di Mosè, dove l'Eterno ha dato questo comandamento: 'I padri non saranno messi a morte a cagion dei figliuoli, né i figliuoli saranno messi a morte a cagion dei padri; ma ciascuno sarà messo a morte a cagione del proprio peccato'.
\par 5 Poi Amatsia radunò quei di Giuda, e li distribuì secondo le loro case patriarcali sotto capi di migliaia e sotto capi di centinaia, per tutto Giuda e Beniamino; ne fece il censimento dall'età di venti anni in su, e trovò trecentomila uomini scelti, atti alla guerra e capaci di maneggiare la lancia e lo scudo.
\par 6 E assoldò anche centomila uomini d'Israele, forti e valorosi, per cento talenti d'argento.
\par 7 Ma un uomo di Dio venne a lui, e gli disse: 'O re, l'esercito d'Israele non vada teco, poiché l'Eterno non è con Israele, con tutti questi figliuoli d'Efraim!
\par 8 Ma, se vuoi andare, portati pure valorosamente nella battaglia; ma Iddio ti abbatterà dinanzi al nemico; perché Dio ha il potere di soccorrere e di abbattere'.
\par 9 Amatsia disse all'uomo di Dio: 'E che fare circa que' cento talenti che ho dati all'esercito d'Israele?' L'uomo di Dio rispose: 'L'Eterno è in grado di darti molto più di questo'.
\par 10 Allora Amatsia separò l'esercito che gli era venuto da Efraim, affinché se ne tornasse al suo paese; ma questa gente fu gravemente irritata contro Giuda, e se ne tornò a casa, accesa d'ira.
\par 11 Amatsia, preso animo, si mise alla testa del suo popolo, andò nella valle del Sale, e sconfisse diecimila uomini de' figliuoli di Seir;
\par 12 e i figliuoli di Giuda ne catturarono vivi altri diecimila; li menarono in cima alla Ròcca, e li precipitarono giù dall'alto della Ròcca, sì che tutti rimasero sfracellati.
\par 13 Ma gli uomini dell'esercito che Amatsia avea licenziati perché non andassero seco alla guerra, piombarono sulle città di Giuda, da Samaria fino a Beth-Horon; ne uccisero tremila abitanti, e portaron via molta preda.
\par 14 E Amatsia, tornato che fu dalla sconfitta degl'Idumei, si fece portare gli dèi de' figliuoli di Seir, li stabilì come suoi dèi, si prostrò dinanzi ad essi, e bruciò de' profumi in loro onore.
\par 15 Per il che l'Eterno s'accese d'ira contro Amatsia, e gli mandò un profeta per dirgli: 'Perché hai tu cercato gli dèi di questo popolo, che non hanno liberato il popolo loro dalla tua mano?'
\par 16 E mentr'egli parlava al re, questi gli disse: 'T'abbiam noi forse fatto consigliere del re? Vattene! Perché vorresti essere ucciso?' Allora il profeta se ne andò, dicendo: 'Io so che Dio ha deciso di distruggerti, perché hai fatto questo, e non hai dato ascolto al mio consiglio'.
\par 17 Allora Amatsia, re di Giuda, dopo aver preso consiglio, inviò de' messi a Joas, figliuolo di Joahaz, figliuolo di Jehu, re d'Israele, per dirgli: 'Vieni, mettiamoci a faccia a faccia!'
\par 18 E Joas, re d'Israele, fece dire ad Amatsia, re di Giuda: 'Lo spino del Libano mandò a dire al cedro del Libano: - Da' la tua figliuola per moglie al mio figliuolo. - Ma le bestie selvagge del Libano passarono, e calpestarono lo spino.
\par 19 Tu hai detto: - Ecco, io ho sconfitto gl'Idumei! - e il tuo cuore, reso orgoglioso, t'ha portato a gloriarti. Stattene a casa tua. Perché impegnarti in una disgraziata impresa che menerebbe alla ruina te e Giuda con te?'
\par 20 Ma Amatsia non gli volle dar retta; perché la cosa era diretta da Dio affinché fossero dati in man del nemico, perché avean cercato gli dèi di Edom.
\par 21 Allora Joas, re d'Israele, salì, ed egli ed Amatsia, re di Giuda, si trovarono a faccia a faccia a Beth-Scemesh, che apparteneva a Giuda.
\par 22 Giuda rimase sconfitto da Israele, e que' di Giuda fuggirono, ognuno alla sua tenda.
\par 23 E Joas, re d'Israele, fece prigioniero a Beth-Scemesh Amatsia, re di Giuda, figliuolo di Joas, figliuolo di Joahaz; lo menò a Gerusalemme, e fece una breccia di quattrocento cubiti nelle mura di Gerusalemme, dalla porta di Efraim alla porta dell'angolo.
\par 24 E prese tutto l'oro e l'argento e tutti i vasi che si trovavano nella casa di Dio in custodia di Obed-Edom, e i tesori della casa del re; prese pure degli ostaggi, e se ne tornò a Samaria.
\par 25 Amatsia, figliuolo di Joas, re di Giuda, visse ancora quindici anni, dopo la morte di Joas, figliuolo di Joahaz, re d'Israele.
\par 26 Il rimanente delle azioni di Amatsia, le prime e le ultime, si trova scritto nel libro dei re di Giuda e d'Israele.
\par 27 Dopo che Amatsia ebbe abbandonato l'Eterno, fu ordita contro di lui una congiura a Gerusalemme, ed egli fuggì a Lakis; ma lo fecero inseguire fino a Lakis, e quivi fu messo a morte.
\par 28 Di là fu trasportato sopra cavalli, e quindi sepolto coi suoi padri nella città di Giuda.

\chapter{26}

\par 1 Allora tutto il popolo di Giuda prese Uzzia che aveva allora sedici anni, e lo fece re in luogo di Amatsia suo padre.
\par 2 Egli riedificò Eloth e la riconquistò a Giuda, dopo che il re si fu addormentato coi suoi padri.
\par 3 Uzzia avea sedici anni quando cominciò a regnare, e regnò cinquantadue anni a Gerusalemme. Sua madre si chiamava Jecolia, ed era di Gerusalemme.
\par 4 Egli fece ciò ch'è giusto agli occhi dell'Eterno, interamente come avea fatto Amatsia suo padre.
\par 5 Si diè con diligenza a cercare Iddio mentre visse Zaccaria, che avea l'intelligenza delle visioni di Dio; e finché cercò l'Eterno, Iddio lo fece prosperare.
\par 6 Egli uscì e mosse guerra ai Filistei, abbatté le mura di Gath, le mura di Jabne e le mura di Asdod, ed edificò delle città nel territorio di Asdod e in quello dei Filistei.
\par 7 E Dio gli diede aiuto contro i Filistei, contro gli Arabi che abitavano a Gur-Baal, e contro i Maoniti.
\par 8 E gli Ammoniti pagavano un tributo ad Uzzia; e la sua fama si sparse sino ai confini dell'Egitto, perch'era divenuto potentissimo.
\par 9 Uzzia costruì pure delle torri a Gerusalemme sulla porta dell'angolo, sulla porta della valle e sullo svolto, e le fortificò.
\par 10 Costruì delle torri nel deserto, e scavò molte cisterne perché avea gran quantità di bestiame; e ne scavò pure nella parte bassa del paese e nella pianura; ed avea de' lavoranti e de' vignaiuoli per i monti e nelle terre fruttifere, perché amava l'agricoltura.
\par 11 Uzzia aveva inoltre un esercito di combattenti che andava alla guerra per schiere, composte secondo il numero del censimento fattone dal segretario Jeiel e dal commissario Maaseia, e messe sotto il comando di Hanania, uno dei generali del re.
\par 12 Il numero totale dei capi delle case patriarcali, degli uomini forti e valorosi, era di duemilaseicento.
\par 13 Essi avevano al loro comando un esercito di trecentosettemilacinquecento combattenti, atti a entrare in guerra con gran valore, per sostenere il re contro il nemico.
\par 14 E Uzzia fornì a tutto l'esercito, scudi, lance, elmi, corazze, archi, e fionde da scagliar sassi.
\par 15 E fece fare a Gerusalemme delle macchine inventate da ingegneri per collocarle sulle torri e sugli angoli, per scagliar saette e grosse pietre. La sua fama andò lungi, perch'egli fu maravigliosamente soccorso, finché divenne potente.
\par 16 Ma quando fu divenuto potente, il suo cuore, insuperbitosi, si pervertì, ed egli commise una infedeltà contro l'Eterno, il suo Dio, entrando nel tempio dell'Eterno per bruciare dell'incenso sull'altare dei profumi.
\par 17 Ma il sacerdote Azaria entrò dopo di lui con ottanta sacerdoti dell'Eterno, uomini coraggiosi,
\par 18 i quali si opposero al re Uzzia, e gli dissero: 'Non spetta a te, o Uzzia, di offrir de' profumi all'Eterno; ma ai sacerdoti, figliuoli d'Aaronne, che son consacrati per offrire i profumi! Esci dal santuario, poiché tu hai commesso una infedeltà! E questo non ti tornerà a gloria dinanzi a Dio, all'Eterno'.
\par 19 Allora Uzzia, che teneva in mano un turibolo per offrire il profumo, si adirò; e mentre s'adirava contro i sacerdoti, la lebbra gli scoppiò sulla fronte, in presenza dei sacerdoti, nella casa dell'Eterno, presso l'altare dei profumi.
\par 20 Il sommo sacerdote Azaria e tutti gli altri sacerdoti lo guardarono, ed ecco che avea la lebbra sulla fronte; lo fecero uscire precipitosamente, ed egli stesso s'affrettò ad andarsene fuori, perché l'Eterno l'avea colpito.
\par 21 Il re Uzzia fu lebbroso fino al giorno della sua morte e stette nell'infermeria come lebbroso, perché era escluso dalla casa dell'Eterno; e Jotham, suo figliuolo, era a capo della casa reale e rendea giustizia al popolo del paese.
\par 22 Il rimanente delle azioni di Uzzia, le prime e le ultime, è stato scritto dal profeta Isaia, figliuolo di Amots.
\par 23 Uzzia s'addormentò coi suoi padri e fu sepolto coi suoi padri nel campo delle sepolture destinato ai re, perché si diceva: 'È lebbroso'. E Jotham, suo figliuolo, regnò in luogo suo.

\chapter{27}

\par 1 Jotham avea venticinque anni quando cominciò a regnare, e regnò sedici anni a Gerusalemme. Sua madre si chiamava Jerusha, figliuola di Tsadok.
\par 2 Egli fece ciò ch'è giusto agli occhi dell'Eterno, interamente come avea fatto Uzzia suo padre; soltanto non entrò nel tempio dell'Eterno, e il popolo continuava a corrompersi.
\par 3 Egli costruì la porta superiore della casa dell'Eterno, e fece molti lavori sulle mura di Ofel.
\par 4 Costruì parimente delle città nella contrada montuosa di Giuda, e dei castelli e delle torri nelle foreste.
\par 5 E mosse guerra al re dei figliuoli di Ammon, e vinse gli Ammoniti. I figliuoli di Ammon gli diedero quell'anno cento talenti d'argento, diecimila cori di grano e diecimila d'orzo; e altrettanto gli pagarono il secondo e il terzo anno.
\par 6 Così Jotham divenne potente, perché camminò con costanza nel cospetto dell'Eterno, del suo Dio.
\par 7 Il rimanente delle azioni di Jotham, tutte le sue guerre e le sue imprese si trovano scritte nel libro dei re d'Israele e di Giuda.
\par 8 Avea venticinque anni quando cominciò a regnare, e regnò sedici anni a Gerusalemme.
\par 9 Jotham s'addormentò coi suoi padri, e fu sepolto nella città di Davide. Ed Achaz, suo figliuolo, regnò in luogo suo.

\chapter{28}

\par 1 Achaz avea vent'anni quando cominciò a regnare, e regnò sedici anni a Gerusalemme. Egli non fece ciò ch'è giusto agli occhi dell'Eterno, come avea fatto Davide suo padre;
\par 2 ma seguì la via dei re d'Israele, e fece perfino delle immagini di getto per i Baali,
\par 3 bruciò dei profumi nella valle del figliuolo di Hinnom, ed arse i suoi figliuoli nel fuoco, seguendo le abominazioni delle genti che l'Eterno avea cacciate d'innanzi ai figliuoli d'Israele;
\par 4 e offriva sacrifizi e profumi sugli alti luoghi, sulle colline, e sotto ogni albero verdeggiante.
\par 5 Perciò l'Eterno, il suo Dio, lo diè nelle mani del re di Siria; e i Sirî lo sconfissero, e gli presero un gran numero di prigionieri che menarono a Damasco. E fu anche dato in mano del re d'Israele, che gl'inflisse una grande sconfitta.
\par 6 Infatti Pekah, figliuolo di Remalia, uccise in un giorno, in Giuda, centoventimila uomini, tutta gente valorosa, perché aveano abbandonato l'Eterno, l'Iddio dei loro padri.
\par 7 Zicri, un prode d'Efraim, uccise Maaseia, figliuolo del re, Azrikam, maggiordomo della casa reale, ed Elkana, che teneva il secondo posto dopo il re.
\par 8 E i figliuoli d'Israele menaron via, di tra i loro fratelli, duecentomila prigionieri, fra donne, figliuoli e figliuole; e ne trassero pure una gran preda, che portarono a Samaria.
\par 9 Or v'era quivi un profeta dell'Eterno, per nome Oded. Egli uscì incontro all'esercito che tornava a Samaria, e disse loro: 'Ecco, l'Eterno, l'Iddio de' vostri padri, nella sua ira contro Giuda, ve li ha dati nelle mani; e voi li avete uccisi con tal furore, ch'è giunto fino al cielo.
\par 10 Ed ora, pretendete di sottomettervi come schiavi e come schiave i figliuoli e le figliuole di Giuda e di Gerusalemme! Ma voi, voi stessi, non siete forse colpevoli verso l'Eterno, l'Iddio vostro?
\par 11 Ascoltatemi dunque, e rimandate i prigionieri che avete fatti tra i vostri fratelli; poiché l'ardente ira dell'Eterno vi sovrasta'.
\par 12 Allora alcuni tra i capi de' figliuoli d'Efraim, Azaria figliuolo di Johanan, Berekia figliuolo di Mescillemoth, Ezechia figliuolo di Shallum e Amasa figliuolo di Hadlai, sorsero contro quelli che tornavano dalla guerra,
\par 13 e dissero loro: 'Voi non menerete qua dentro i prigionieri; perché voi vi proponete cosa che ci renderà colpevoli dinanzi all'Eterno, accrescendo il numero dei nostri peccati e delle nostre colpe; poiché noi siamo già grandemente colpevoli, e l'ira dell'Eterno arde contro Israele'.
\par 14 Allora i soldati abbandonarono i prigionieri e la preda in presenza dei capi e di tutta la raunanza.
\par 15 E gli uomini già ricordati per nome si levarono e presero i prigionieri; del bottino si servirono per rivestire tutti quelli di loro ch'erano ignudi; li rivestirono, li calzarono, diedero loro da mangiare e da bere, li unsero, condussero sopra degli asini tutti quelli che cascavan dalla fatica, e li menarono a Gerico, la città delle palme, dai loro fratelli; poi se ne tornarono a Samaria.
\par 16 In quel tempo, il re Achaz mandò a chieder soccorso ai re d'Assiria.
\par 17 - Or gli Edomiti erano venuti di nuovo, aveano sconfitto Giuda e menati via de' prigionieri.
\par 18 I Filistei pure aveano invaso le città della pianura e del mezzogiorno di Giuda, e avean preso Beth-Scemesh, Ajalon, Ghederoth, Soco e le città che ne dipendevano, Timnah e le città che ne dipendevano, Ghimzo e le città che ne dipendevano, e vi s'erano stabiliti.
\par 19 Poiché l'Eterno aveva umiliato Giuda a motivo di Achaz, re d'Israele, perché avea rotto ogni freno in Giuda, e avea commesso ogni sorta d'infedeltà contro l'Eterno. -
\par 20 E Tilgath-Pilneser, re d'Assiria, mosse contro di lui, lo ridusse alle strette, e non lo sostenne affatto.
\par 21 Poiché Achaz avea spogliato la casa dell'Eterno, la casa del re e dei capi, e avea dato tutto al re d'Assiria; ma a nulla gli era giovato.
\par 22 E nel tempo in cui si trovava alle strette, questo medesimo re Achaz continuò più che mai a commettere delle infedeltà contro l'Eterno.
\par 23 Offrì dei sacrifizi agli dèi di Damasco, che l'avevano sconfitto, e disse: 'Giacché gli dèi dei re di Siria aiutan quelli, io offrirò loro de' sacrifizi ed aiuteranno anche me'. Ma furono invece la rovina di lui e di tutto Israele.
\par 24 Achaz radunò gli utensili della casa di Dio, fece a pezzi gli utensili della casa di Dio, chiuse le porte della casa dell'Eterno, si fece degli altari a tutte le cantonate di Gerusalemme,
\par 25 e stabilì degli alti luoghi in ognuna delle città di Giuda per offrire dei profumi ad altri dèi. Così provocò ad ira l'Eterno, l'Iddio de' suoi padri.
\par 26 Il rimanente delle sue azioni e di tutti i suoi portamenti, i primi e gli ultimi, si trova scritto nel libro dei re di Giuda e d'Israele.
\par 27 Achaz si addormentò coi suoi padri, e fu sepolto in città, a Gerusalemme, perché non lo vollero mettere nei sepolcri dei re d'Israele. Ed Ezechia, suo figliuolo, regnò in luogo suo.

\chapter{29}

\par 1 Ezechia avea venticinque anni quando cominciò a regnare, e regnò ventinove anni a Gerusalemme. Sua madre si chiamava Abija, figliuola di Zaccaria.
\par 2 Egli fece ciò ch'è giusto agli occhi dell'Eterno, interamente come avea fatto Davide suo padre.
\par 3 Nel primo anno del suo regno, nel primo mese, riaperse le porte della casa dell'Eterno, e le restaurò.
\par 4 Fece venire i sacerdoti e i Leviti, li radunò sulla piazza orientale,
\par 5 e disse loro: 'Ascoltatemi, o Leviti! Ora santificatevi, e santificate la casa dell'Eterno, dell'Iddio de' vostri padri, e portate fuori dal santuario ogni immondezza.
\par 6 Poiché i nostri padri sono stati infedeli e hanno fatto ciò ch'è male agli occhi dell'Eterno, dell'Iddio nostro, l'hanno abbandonato, han cessato di volger la faccia verso la dimora dell'Eterno, e le han voltato le spalle.
\par 7 Ed hanno chiuse le porte del portico, hanno spente le lampade, non hanno più bruciato profumi né offerto olocausti nel santuario all'Iddio d'Israele.
\par 8 Perciò l'ira dell'Eterno ha colpito Giuda e Gerusalemme; ed ei li ha abbandonati alle vessazioni, alla desolazione ed agli scherni, come vedete con gli occhi vostri.
\par 9 Ed ecco che, a causa di questo, i nostri padri son periti di spada, e i nostri figliuoli, le nostre figliuole e le nostre mogli sono in cattività.
\par 10 Or io ho in cuore di fare un patto con l'Eterno, coll'Iddio d'Israele, affinché l'ardore della sua ira si allontani da noi.
\par 11 Figliuoli miei, non siate negligenti; poiché l'Eterno vi ha scelti affinché stiate davanti a lui per servirgli, per esser suoi ministri, e per offrirgli profumi'.
\par 12 Allora i Leviti si levarono: Mahath, figliuolo d'Amasai, Joel, figliuolo di Azaria, de' figliuoli di Kehath. Dei figliuoli di Merari: Kish, figliuolo d'Abdi, e Azaria, figliuolo di Jehalleleel. Dei Ghershoniti: Joah, figliuolo di Zimma, e Edem, figliuolo di Joah.
\par 13 Dei figliuoli di Elitsafan: Scimri e Jeiel. Dei figliuoli di Asaf: Zaccaria e Mattania.
\par 14 Dei figliuoli di Heman: Jehiel e Scimei. Dei figliuoli di Jeduthun: Scemaia e Uzziel.
\par 15 Ed essi adunarono i loro fratelli e, dopo essersi santificati, vennero a purificare la casa dell'Eterno, secondo l'ordine del re, conformemente alle parole dell'Eterno.
\par 16 E i sacerdoti entrarono nell'interno della casa dell'Eterno per purificarla, e portaron fuori, nel cortile della casa dell'Eterno, tutte le immondezze che trovarono nel tempio dell'Eterno; e i Leviti le presero per portarle fuori e gettarle nel torrente Kidron.
\par 17 Cominciarono queste purificazioni il primo giorno del primo mese; e l'ottavo giorno dello stesso mese vennero al portico dell'Eterno, e misero otto giorni a purificare la casa dell'Eterno; il sedicesimo giorno del primo mese aveano finito.
\par 18 Allora vennero al re Ezechia, nel suo palazzo, e gli dissero: 'Noi abbiam purificata tutta la casa dell'Eterno, l'altare degli olocausti con tutti i suoi utensili, la tavola dei pani della presentazione con tutti i suoi utensili;
\par 19 come pure abbiamo rimesso in buono stato e purificati tutti gli utensili che il re Achaz avea profanati durante il suo regno, quando si rese infedele; ed ecco, stanno davanti all'altare dell'Eterno'.
\par 20 Allora Ezechia, levatosi di buon'ora, adunò i capi della città, e salì alla casa dell'Eterno.
\par 21 Essi menarono sette giovenchi, sette montoni e sette agnelli; e sette capri, come sacrifizio per il peccato, a pro del regno, del santuario e di Giuda. E il re ordinò ai sacerdoti, figliuoli d'Aaronne, d'offrirli sull'altare dell'Eterno.
\par 22 I sacerdoti scannarono i giovenchi, e ne raccolsero il sangue, e lo sparsero sull'altare; scannarono i montoni, e ne sparsero il sangue sull'altare; e scannarono gli agnelli, e ne sparsero il sangue sull'altare.
\par 23 Poi menarono i capri del sacrifizio per il peccato, davanti al re e alla raunanza, e questi posarono su d'essi le loro mani.
\par 24 I sacerdoti li scannarono, e ne offrirono il sangue sull'altare come sacrifizio per il peccato, per fare l'espiazione dei peccati di tutto Israele; giacché il re aveva ordinato che si offrisse l'olocausto e il sacrifizio per il peccato, a pro di tutto Israele.
\par 25 Il re stabilì i Leviti nella casa dell'Eterno, con cembali, con saltèri e con cetre, secondo l'ordine di Davide, di Gad, il veggente del re, e del profeta Nathan; poiché tale era il comandamento dato dall'Eterno per mezzo de' suoi profeti.
\par 26 E i Leviti presero il loro posto con gli strumenti di Davide; e i sacerdoti, con le trombe.
\par 27 Allora Ezechia ordinò che si offrisse l'olocausto sull'altare; e nel momento in cui si cominciò l'olocausto, cominciò pure il canto dell'Eterno e il suono delle trombe, con l'accompagnamento degli strumenti di Davide, re d'Israele.
\par 28 E tutta la raunanza si prostrò, e i cantori cominciarono a cantare e le trombe a sonare; e tutto questo continuò sino alla fine dell'olocausto.
\par 29 E quando l'offerta dell'olocausto fu finita, il re e tutti quelli ch'erano con lui s'inchinarono e si prostrarono.
\par 30 Poi il re Ezechia e i capi ordinarono ai Leviti di celebrare le lodi dell'Eterno con le parole di Davide e del veggente Asaf; e quelli le celebrarono con gioia, e s'inchinarono e si prostrarono.
\par 31 Allora Ezechia prese a dire: 'Ora che vi siete consacrati all'Eterno, avvicinatevi, e offrite vittime e sacrifizi di lode nella casa dell'Eterno'. E la raunanza menò vittime e offrì sacrifizi di azioni di grazie; e tutti quelli che aveano il cuore ben disposto, offrirono olocausti.
\par 32 Il numero degli olocausti offerti dalla raunanza fu di settanta giovenchi, cento montoni, duecento agnelli: tutto per l'olocausto all'Eterno.
\par 33 E furon pure consacrati seicento buoi e tremila pecore.
\par 34 Ma i sacerdoti erano troppo pochi, e non potevano scorticare tutti gli olocausti; perciò i loro fratelli, i Leviti, li aiutarono finché l'opera fu compiuta, e finché gli altri sacerdoti si furon santificati; perché i Leviti avean messo più rettitudine di cuore a santificarsi, dei sacerdoti.
\par 35 E v'era pure abbondanza d'olocausti, oltre ai grassi de' sacrifizi d'azioni di grazie e alle libazioni degli olocausti. Così fu ristabilito il servizio della casa dell'Eterno.
\par 36 Ed Ezechia e tutto il popolo si rallegrarono che Dio avesse ben disposto il popolo, perché la cosa s'era fatta subitamente.

\chapter{30}

\par 1 Poi Ezechia inviò de' messi a tutto Israele e a Giuda, e scrisse pure lettere ad Efraim ed a Manasse, perché venissero alla casa dell'Eterno a Gerusalemme, a celebrar la Pasqua in onore dell'Eterno, dell'Iddio d'Israele.
\par 2 Il re, i suoi capi e tutta la raunanza, in un consiglio tenuto a Gerusalemme, avevano deciso di celebrare la Pasqua il secondo mese;
\par 3 giacché non la potevan celebrare al tempo debito, perché i sacerdoti non s'erano santificati in numero sufficiente, e il popolo non s'era radunato in Gerusalemme.
\par 4 La cosa piacque al re e a tutta la raunanza;
\par 5 e stabilirono di proclamare un bando per tutto Israele, da Beer-Sceba fino a Dan, perché la gente venisse a Gerusalemme a celebrar la Pasqua in onore dell'Eterno, dell'Iddio d'Israele; poiché per l'addietro essa non era stata celebrata in modo generale, secondo ch'è prescritto.
\par 6 I corrieri dunque andarono con le lettere del re e dei suoi capi per tutto Israele e Giuda; e, conformemente all'ordine del re, dissero: 'Figliuoli d'Israele, tornate all'Eterno, all'Iddio d'Abrahamo, d'Isacco e d'Israele, ond'egli torni al residuo che di voi è scampato dalle mani dei re d'Assiria.
\par 7 E non siate come i vostri padri e come i vostri fratelli, che sono stati infedeli all'Eterno, all'Iddio dei loro padri, in guisa ch'ei li ha dati in preda alla desolazione, come voi vedete.
\par 8 Ora non indurate le vostre cervici, come i padri vostri; date la mano all'Eterno, venite al suo santuario ch'egli ha santificato in perpetuo, e servite l'Eterno, il vostro Dio, onde l'ardente ira sua si ritiri da voi.
\par 9 Poiché, se tornate all'Eterno, i vostri fratelli e i vostri figliuoli troveranno pietà in quelli che li hanno menati schiavi, e ritorneranno in questo paese; giacché l'Eterno, il vostro Dio, è clemente e misericordioso, e non volgerà la faccia lungi da voi, se a lui tornate'.
\par 10 Quei corrieri dunque passarono di città in città nel paese di Efraim e di Manasse, e fino a Zabulon; ma la gente si facea beffe di loro e li scherniva.
\par 11 Nondimeno, alcuni uomini di Ascer, di Manasse e di Zabulon si umiliarono, e vennero a Gerusalemme.
\par 12 Anche in Giuda la mano di Dio operò in guisa da dar loro un medesimo cuore per mettere ad effetto l'ordine del re e dei capi, secondo la parola dell'Eterno.
\par 13 Un gran popolo si riunì a Gerusalemme per celebrare la festa degli azzimi, il secondo mese: fu una raunanza immensa.
\par 14 Si levarono e tolsero via gli altari sui quali si offrivan sacrifizi a Gerusalemme, tolsero via tutti gli altari sui quali si offrivan profumi, e li gettarono nel torrente Kidron.
\par 15 Poi immolarono l'agnello pasquale, il quattordicesimo giorno del secondo mese. I sacerdoti e i Leviti, i quali, presi da vergogna, s'eran santificati, offrirono olocausti nella casa dell'Eterno;
\par 16 e occuparono il posto assegnato loro dalla legge di Mosè, uomo di Dio. I sacerdoti facevano l'aspersione del sangue, che ricevevano dalle mani de' Leviti.
\par 17 Siccome ve n'erano molti, nella raunanza, che non s'erano santificati, i Leviti aveano l'incarico d'immolare gli agnelli pasquali, consacrandoli all'Eterno, per tutti quelli che non eran puri.
\par 18 Poiché una gran parte del popolo, molti d'Efraim, di Manasse, d'Issacar e di Zabulon non s'erano purificati, e mangiarono la Pasqua, senza conformarsi a quello ch'è scritto. Ma Ezechia pregò per loro, dicendo:
\par 19 'L'Eterno, che è buono, perdoni a chiunque ha disposto il proprio cuore alla ricerca di Dio, dell'Eterno, ch'è l'Iddio de' suoi padri, anche senz'avere la purificazione richiesta dal santuario'.
\par 20 E l'Eterno esaudì Ezechia, e perdonò al popolo.
\par 21 Così i figliuoli d'Israele che si trovarono a Gerusalemme, celebrarono la festa degli azzimi per sette giorni con grande allegrezza; e ogni giorno i Leviti e i sacerdoti celebravano l'Eterno con gli strumenti consacrati ad accompagnar le sue lodi.
\par 22 Ezechia parlò al cuore di tutti i Leviti che mostravano grande intelligenza nel servizio dell'Eterno; e si fecero i pasti della festa durante i sette giorni, offrendo sacrifizi di azioni di grazie, e lodando l'Eterno, l'Iddio dei loro padri.
\par 23 E tutta la raunanza deliberò di celebrare la festa per altri sette giorni; e la celebrarono con allegrezza durante questi sette giorni;
\par 24 poiché Ezechia, re di Giuda, avea donato alla raunanza mille giovenchi e settemila pecore, e i capi pure avean donato alla raunanza mille tori e diecimila pecore; e i sacerdoti in gran numero, s'erano santificati.
\par 25 Tutta la raunanza di Giuda, i sacerdoti, i Leviti, tutta la raunanza di quelli venuti da Israele e gli stranieri giunti dal paese d'Israele o stabiliti in Giuda, furono in festa.
\par 26 Così vi fu gran gioia in Gerusalemme; dal tempo di Salomone, figliuolo di Davide, re d'Israele, non v'era stato nulla di simile in Gerusalemme.
\par 27 Poi i sacerdoti Leviti si levarono e benedissero il popolo, e la loro voce fu udita, e la loro preghiera giunse fino al cielo, fino alla santa dimora dell'Eterno.

\chapter{31}

\par 1 Quando tutte queste cose furon compiute, tutti gl'Israeliti che si trovavano quivi partirono per le città di Giuda, e frantumarono le statue, abbatterono gl'idoli d'Astarte, demolirono gli alti luoghi e gli altari in tutto Giuda e Beniamino, e in Efraim e in Manasse, in guisa che nulla più ne rimase. Poi tutti i figliuoli d'Israele se ne tornarono nelle loro città, ciascuno nel proprio possesso.
\par 2 Ezechia ristabilì le classi de' sacerdoti e de' Leviti nelle loro funzioni, ognuno secondo il genere del suo servizio, sacerdoti e Leviti, per gli olocausti e i sacrifizi di azioni di grazie, per il servizio, per la lode e per il canto, entro le porte del campo dell'Eterno.
\par 3 Stabilì pure la parte che il re preleverebbe dai suoi beni per gli olocausti, per gli olocausti del mattino e della sera, per gli olocausti dei sabati, dei noviluni e delle feste, come sta scritto nella legge dell'Eterno;
\par 4 e ordinò al popolo, agli abitanti di Gerusalemme, di dare ai sacerdoti e ai Leviti la loro parte, affinché potessero darsi all'adempimento della legge dell'Eterno.
\par 5 Non appena quest'ordine fu pubblicato, i figliuoli d'Israele dettero in gran quantità le primizie del grano, del vino, dell'olio, del miele, e di tutti i prodotti dei campi; e portarono la decima d'ogni cosa, in abbondanza.
\par 6 I figliuoli d'Israele e di Giuda che abitavano nelle città di Giuda menarono anch'essi la decima dei buoi e delle pecore, e la decima delle cose sante che erano consacrate all'Eterno, al loro Dio, e delle quali si fecero tanti mucchi.
\par 7 Cominciarono a fare quei mucchi il terzo mese, e finirono il settimo mese.
\par 8 Ezechia e i capi vennero a vedere que' mucchi, e benedissero l'Eterno e il suo popolo d'Israele.
\par 9 Ed Ezechia interrogò i sacerdoti e i Leviti, relativamente a que' mucchi;
\par 10 e il sommo sacerdote Azaria, della casa di Tsadok, gli rispose: 'Da che s'è cominciato a portar le offerte nella casa dell'Eterno, noi abbiam mangiato, ci siamo saziati, e v'è rimasta roba in abbondanza, perché l'Eterno ha benedetto il suo popolo; ed ecco qui la gran quantità ch'è rimasta'.
\par 11 Allora Ezechia ordinò che si preparassero delle stanze nella casa dell'Eterno; e furon preparate.
\par 12 E vi riposero fedelmente le offerte, la decima e le cose consacrate; Conania, il Levita, n'ebbe la sovrintendenza, e Scimei, suo fratello, veniva in secondo luogo.
\par 13 Jehiel, Ahazia, Nahath, Asahel, Jerimoth, Jozabad, Eliel, Ismakia, Mahath e Benaia erano impiegati sotto la direzione di Conania e del suo fratello Scimei, per ordine del re Ezechia e d'Azaria, capo della casa di Dio.
\par 14 Il Levita Kore, figliuolo di Imna, guardiano della porta orientale, era preposto ai doni volontari fatti a Dio per distribuire le offerte fatte all'Eterno e le cose santissime.
\par 15 Sotto di lui stavano Eden, Miniamin, Jeshua, Scemaia, Amaria, Scecania, nelle città dei sacerdoti, come uomini di fiducia, per fare le distribuzioni ai loro fratelli grandi e piccoli, secondo le loro classi,
\par 16 eccettuati i maschi ch'erano registrati nelle loro genealogie dall'età di tre anni in su, cioè tutti quelli che entravano giornalmente nella casa dell'Eterno per fare il loro servizio secondo le loro funzioni e secondo le loro classi.
\par 17 (La registrazione dei sacerdoti si faceva secondo le loro case patriarcali, e quella dei Leviti dall'età di vent'anni in su, secondo le loro funzioni e secondo le loro classi).
\par 18 Dovean fare le distribuzioni a quelli di tutta la raunanza ch'eran registrati con tutti i loro bambini, con le loro mogli, coi loro figliuoli e con le loro figliuole; poiché nel loro ufficio di fiducia amministravano i doni sacri.
\par 19 E per i sacerdoti, figliuoli d'Aaronne, che dimoravano in campagna, nei contadi delle loro città, v'erano in ogni città degli uomini designati per nome per distribuire le porzioni a tutti i maschi di tra i sacerdoti, e a tutti i Leviti registrati nelle genealogie.
\par 20 Ezechia fece così per tutto Giuda; fece ciò ch'è buono, retto e vero dinanzi all'Eterno, al suo Dio.
\par 21 In tutto quello che prese a fare per il servizio della casa di Dio, per la legge e per i comandamenti, cercando il suo Dio, mise tutto il cuore nell'opera sua, e prosperò.

\chapter{32}

\par 1 Dopo queste cose e questi atti di fedeltà di Ezechia, Sennacherib, re d'Assiria, venne, entrò in Giuda, e cinse d'assedio le città fortificate, con l'intenzione d'impadronirsene.
\par 2 E quando Ezechia vide che Sennacherib era giunto e si proponeva d'attaccar Gerusalemme,
\par 3 deliberò coi suoi capi e con i suoi uomini valorosi di turar le sorgenti d'acqua ch'eran fuori della città; ed essi gli prestarono aiuto.
\par 4 Si radunò dunque un gran numero di gente e turarono tutte le sorgenti e il torrente che scorreva attraverso il paese. 'E perché', dicevan essi, 'i re d'Assiria, venendo, troverebbero essi abbondanza d'acqua?'
\par 5 Ezechia prese animo, ricostruì tutte le mura dov'erano rotte, rialzò le torri, costruì l'altro muro di fuori, fortificò Millo nella città di Davide, e fece fare gran quantità d'armi e di scudi.
\par 6 Diede dei capi militari al popolo, li riunì presso di sé sulla piazza della porta della città, e parlò al loro cuore, dicendo:
\par 7 'Siate forti, e fatevi animo! Non temete e non vi sgomentate a motivo del re d'Assiria e della gran gente che l'accompagna; giacché con noi è uno più grande di ciò ch'è con lui.
\par 8 Con lui è un braccio di carne; con noi è l'Eterno, il nostro Dio, per aiutarci e combattere le nostre battaglie'. E il popolo fu rassicurato dalle parole di Ezechia, re di Giuda.
\par 9 Dopo questo, Sennacherib, re d'Assiria, mentre stava di fronte a Lakis con tutte le sue forze, mandò i suoi servi a Gerusalemme per dire a Ezechia, re di Giuda, e a tutti que' di Giuda che si trovavano a Gerusalemme:
\par 10 'Così parla Sennacherib, re degli Assiri: In chi confidate voi per rimanervene così assediati in Gerusalemme?
\par 11 Ezechia non v'inganna egli per ridurvi a morir di fame e di sete, quando dice: - L'Eterno, il nostro Dio, ci libererà dalle mani del re d'Assiria? -
\par 12 Non è egli lo stesso Ezechia che ha soppresso gli alti luoghi e gli altari dell'Eterno, e che ha detto a Giuda e a Gerusalemme: - Voi adorerete dinanzi a un unico altare e su quello offrirete profumi? -
\par 13 Non sapete voi quello che io e i miei padri abbiam fatto a tutti i popoli degli altri paesi? Gli dèi delle nazioni di que' paesi hanno essi potuto liberare i loro paesi dalla mia mano?
\par 14 Qual è fra tutti gli dèi di queste nazioni che i miei padri hanno sterminate, quello che abbia potuto liberare il suo popolo dalla mia mano? E potrebbe il vostro Dio liberar voi dalla mia mano?!
\par 15 Or dunque Ezechia non v'inganni e non vi seduca in questa maniera; non gli prestate fede! Poiché nessun dio d'alcuna nazione o d'alcun regno ha potuto liberare il suo popolo dalla mia mano o dalla mano de' miei padri; quanto meno potrà l'Iddio vostro liberar voi dalla mia mano!'
\par 16 I servi di Sennacherib parlarono ancora contro l'Eterno Iddio e contro il suo servo Ezechia.
\par 17 Sennacherib scrisse pure delle lettere, insultando l'Eterno, l'Iddio d'Israele, e parlando contro di lui, in questi termini: 'Come gli dèi delle nazioni degli altri paesi non han potuto liberare i loro popoli dalla mia mano, così neanche l'Iddio d'Ezechia potrà liberare dalla mia mano il popolo suo'.
\par 18 I servi di Sennacherib gridarono ad alta voce, in lingua giudaica, rivolgendosi al popolo di Gerusalemme che stava sulle mura, per spaventarlo e atterrirlo, e potersi così impadronire della città.
\par 19 E parlarono dell'Iddio di Gerusalemme come degli dèi dei popoli della terra, che sono opera di mano d'uomo.
\par 20 Allora il re Ezechia e il profeta Isaia, figliuolo di Amots, pregarono a questo proposito, e alzarono fino al cielo il loro grido.
\par 21 E l'Eterno mandò un angelo che sterminò nel campo del re d'Assiria tutti gli uomini forti e valorosi, i principi ed i capi. E il re se ne tornò svergognato al suo paese. E come fu entrato nella casa del suo dio, i suoi propri figliuoli lo uccisero quivi di spada.
\par 22 Così l'Eterno salvò Ezechia e gli abitanti di Gerusalemme dalla mano di Sennacherib, re d'Assiria, e dalla mano di tutti gli altri, e li protesse d'ogn'intorno.
\par 23 E molti portarono a Gerusalemme delle offerte all'Eterno, e degli oggetti preziosi a Ezechia, re di Giuda, il quale, da allora, sorse in gran considerazione agli occhi di tutte le nazioni.
\par 24 In quel tempo, Ezechia fu malato a morte; egli pregò l'Eterno, e l'Eterno gli parlò, e gli concesse un segno.
\par 25 Ma Ezechia non fu riconoscente del beneficio che avea ricevuto; giacché il suo cuore s'inorgoglì, e l'ira dell'Eterno si volse contro di lui, contro Giuda e contro Gerusalemme.
\par 26 Nondimeno Ezechia si umiliò dell'essersi inorgoglito in cuor suo: tanto egli, quanto gli abitanti di Gerusalemme; perciò l'ira dell'Eterno non venne sopra loro durante la vita d'Ezechia.
\par 27 Ezechia ebbe immense ricchezze e grandissima gloria: e si fece de' tesori per riporvi argento, oro, pietre preziose, aromi, scudi, ogni sorta d'oggetti di valore;
\par 28 de' magazzini per i prodotti di grano, vino, olio; delle stalle per ogni sorta di bestiame, e degli ovili per le pecore.
\par 29 Si edificò delle città, ed ebbe greggi e mandre in abbondanza, perché Dio gli avea dato dei beni in gran copia.
\par 30 Ezechia fu quegli che turò la sorgente superiore delle acque di Ghihon, che condusse giù direttamente, dal lato occidentale della città di Davide. Ezechia riuscì felicemente in tutte le sue imprese.
\par 31 Nondimeno, quando i capi di Babilonia gl'inviarono dei messi per informarsi del prodigio ch'era avvenuto nel paese, Iddio lo abbandonò, per metterlo alla prova, affin di conoscere tutto quello ch'egli aveva in cuore.
\par 32 Le rimanenti azioni di Ezechia e le sue opere pie trovansi scritte nella visione del profeta Isaia, figliuolo d'Amots, inserita nel libro dei re di Giuda e d'Israele.
\par 33 Ezechia s'addormentò coi suoi padri, e fu sepolto sulla salita dei sepolcri de' figliuoli di Davide; e alla sua morte, tutto Giuda e gli abitanti di Gerusalemme gli resero onore. E Manasse, suo figliuolo, regnò in luogo suo.

\chapter{33}

\par 1 Manasse avea dodici anni quando cominciò a regnare, e regnò cinquantacinque anni a Gerusalemme.
\par 2 Egli fece ciò ch'è male agli occhi dell'Eterno, seguendo le abominazioni delle nazioni che l'Eterno avea cacciate d'innanzi ai figliuoli d'Israele.
\par 3 Riedificò gli alti luoghi che Ezechia suo padre avea demoliti, eresse altari ai Baali, fece degl'idoli d'Astarte, e adorò tutto l'esercito del cielo e lo servì.
\par 4 Eresse pure degli altari ad altri dèi nella casa dell'Eterno, riguardo alla quale l'Eterno avea detto: 'In Gerusalemme sarà in perpetuo il mio nome!'
\par 5 Eresse altari a tutto l'esercito del cielo nei due cortili della casa dell'Eterno.
\par 6 Fece passare i suoi figliuoli pel fuoco nella valle del figliuolo di Hinnom; si dette alla magia, agl'incantesimi, alla stregoneria, e istituì di quelli che evocavano gli spiriti e predicevan l'avvenire; s'abbandonò interamente a fare ciò ch'è male agli occhi dell'Eterno, provocandolo ad ira.
\par 7 Mise l'immagine scolpita dell'idolo che avea fatto, nella casa di Dio, riguardo alla quale Dio avea detto a Davide e a Salomone suo figliuolo: 'In questa casa, e a Gerusalemme, che io ho scelta fra tutte le tribù d'Israele, porrò il mio nome in perpetuo;
\par 8 e farò che Israele non muova più il piede dal paese ch'io ho assegnato ai vostri padri, purché essi abbian cura di mettere in pratica tutto quello che ho loro comandato, cioè tutta la legge, i precetti e le prescrizioni, dati per mezzo di Mosè'.
\par 9 Ma Manasse indusse Giuda e gli abitanti di Gerusalemme a sviarsi, e a far peggio delle nazioni che l'Eterno avea distrutte d'innanzi ai figliuoli d'Israele.
\par 10 L'Eterno parlò a Manasse e al suo popolo, ma essi non ne fecero caso.
\par 11 Allora l'Eterno fece venire contro di loro i capi dell'esercito del re d'Assiria, che misero Manasse nei ferri; e, legatolo con catene di rame, lo menarono a Babilonia.
\par 12 E quand'ei fu in distretta, implorò l'Eterno, il suo Dio, e s'umiliò profondamente davanti all'Iddio de' suoi padri.
\par 13 A lui rivolse le sue preghiere, ed egli s'arrese ad esse, esaudì le sue supplicazioni, e lo ricondusse a Gerusalemme nel suo regno. Allora Manasse riconobbe che l'Eterno è Dio.
\par 14 Dopo questo, Manasse costruì, fuori della città di Davide, a occidente, verso Ghihon nella valle, un muro che si prolungava fino alla porta dei pesci; lo fe' girare attorno ad Ofel, e lo tirò su a grande altezza; e pose dei capi militari in tutte le città fortificate di Giuda;
\par 15 e tolse dalla casa dell'Eterno gli dèi stranieri e l'idolo, abbatté tutti gli altari che aveva costruiti sul monte della casa dell'Eterno e a Gerusalemme, e gettò tutto fuori della città.
\par 16 Poi ristabilì l'altare dell'Eterno e v'offrì sopra dei sacrifizi di azioni di grazie e di lode, e ordinò a Giuda che servisse all'Eterno, all'Iddio d'Israele.
\par 17 Nondimeno il popolo continuava a offrir sacrifizi sugli alti luoghi; però, soltanto all'Eterno, al suo Dio.
\par 18 Il rimanente delle azioni di Manasse, la preghiera che rivolse al suo Dio, e le parole che i veggenti gli rivolsero nel nome dell'Eterno, dell'Iddio d'Israele, son cose scritte nella storia dei re d'Israele.
\par 19 E la sua preghiera, e come Dio s'arrese ad essa, tutti i suoi peccati e tutte le sue infedeltà, i luoghi dove costruì degli alti luoghi e pose degl'idoli d'Astarte e delle immagini scolpite, prima che si fosse umiliato, sono cose scritte nel libro di Hozai.
\par 20 Poi Manasse s'addormentò coi suoi padri, e fu sepolto in casa sua. E Amon, suo figliuolo, regnò in luogo suo.
\par 21 Amon avea ventidue anni quando cominciò a regnare, e regnò due anni a Gerusalemme.
\par 22 Egli fece ciò ch'è male agli occhi dell'Eterno, come avea fatto Manasse suo padre; offriva sacrifizi a tutte le immagini scolpite fatte da Manasse suo padre, e le serviva.
\par 23 Egli non s'umiliò dinanzi all'Eterno, come s'era umiliato Manasse suo padre; anzi Amon si rese sempre più colpevole.
\par 24 E i suoi servi ordirono una congiura contro di lui, e lo uccisero in casa sua.
\par 25 Ma il popolo del paese mise a morte tutti quelli che avean congiurato contro il re Amon, e fece re, in sua vece, Giosia suo figliuolo.

\chapter{34}

\par 1 Giosia aveva otto anni quando cominciò a regnare, e regnò trentun anni a Gerusalemme.
\par 2 Egli fece ciò ch'è giusto agli occhi dell'Eterno, e camminò per le vie di Davide suo padre senza scostarsene né a destra né a sinistra.
\par 3 L'ottavo anno del suo regno, mentre era ancora giovinetto, cominciò a cercare l'Iddio di Davide suo padre; e il dodicesimo anno cominciò a purificare Giuda e Gerusalemme dagli alti luoghi, dagl'idoli d'Astarte, dalle immagini scolpite e dalle immagini fuse.
\par 4 E in sua presenza furon demoliti gli altari de' Baali e abbattute le colonne solari che v'eran sopra; e frantumò gl'idoli d'Astarte, le immagini scolpite e le statue; e le ridusse in polvere, che sparse sui sepolcri di quelli che aveano offerto loro de' sacrifizi;
\par 5 e bruciò le ossa dei sacerdoti sui loro altari, e così purificò Giuda e Gerusalemme.
\par 6 Lo stesso fece nelle città di Manasse, d'Efraim, di Simeone, e fino a Neftali: da per tutto, in mezzo alle loro rovine,
\par 7 demolì gli altari, frantumò e ridusse in polvere gl'idoli d'Astarte e le immagini scolpite, abbatté tutte le colonne solari in tutto il paese d'Israele, e tornò a Gerusalemme.
\par 8 L'anno diciottesimo del suo regno, dopo aver purificato il paese e la casa dell'Eterno, mandò Shafan, figliuolo di Atsalia, Maaseia, governatore della città, e Joah, figliuolo di Joachaz, l'archivista, per restaurare la casa dell'Eterno, del suo Dio.
\par 9 E quelli si recarono dal sommo sacerdote Hilkia, e fu loro consegnato il danaro ch'era stato portato nella casa di Dio, e che i Leviti custodi della soglia aveano raccolto in Manasse, in Efraim, in tutto il rimanente d'Israele, in tutto Giuda e Beniamino, e fra gli abitanti di Gerusalemme.
\par 10 Ed essi lo rimisero nelle mani dei direttori preposti ai lavori della casa dell'Eterno, e i direttori lo dettero a quelli che lavoravano nella casa dell'Eterno per ripararla e restaurarla.
\par 11 Lo dettero ai legnaiuoli ed ai costruttori, per comprar delle pietre da tagliare, e del legname per l'armatura e la travatura delle case che i re di Giuda aveano distrutte.
\par 12 E quegli uomini facevano il loro lavoro con fedeltà; e ad essi eran preposti Jahath e Obadia, Leviti di tra i figliuoli di Merari, e Zaccaria e Meshullam di tra i figliuoli di Kehath, per la direzione, e tutti quelli tra i Leviti ch'erano abili a sonare strumenti musicali.
\par 13 Questi sorvegliavan pure i portatori di pesi, e dirigevano tutti gli operai occupati nei diversi lavori; e fra i Leviti addetti a que' lavori ve n'eran di quelli ch'erano segretari, commissari, portinai.
\par 14 Or mentre si traeva fuori il danaro ch'era stato portato nella casa dell'Eterno, il sacerdote Hilkia trovò il libro della Legge dell'Eterno, data per mezzo di Mosè.
\par 15 E Hilkia parlò a Shafan, il segretario, e gli disse: 'Ho trovato nella casa dell'Eterno il libro della legge'. E Hilkia diede il libro a Shafan.
\par 16 E Shafan portò il libro al re, e gli fece al tempo stesso la sua relazione, dicendo: 'I tuoi servi hanno fatto tutto quello ch'è stato loro ordinato.
\par 17 Hanno versato il danaro che s'è trovato nella casa dell'Eterno, e l'hanno consegnato a quelli che son preposti ai lavori e agli operai'.
\par 18 E Shafan, il segretario, disse ancora al re: 'Il sacerdote Hilkia m'ha dato un libro'. E Shafan lo lesse in presenza del re.
\par 19 Quando il re ebbe udite le parole della legge, si stracciò le vesti.
\par 20 Poi il re diede quest'ordine a Hilkia, ad Ahikam, figliuolo di Shafan, ad Abdon, figliuolo di Mica, a Shafan il segretario, e ad Asaia, servo del re:
\par 21 'Andate a consultare l'Eterno per me e per ciò che rimane d'Israele e di Giuda, riguardo alle parole di questo libro che s'è trovato; giacché grande è l'ira dell'Eterno che s'è riversata su noi, perché i nostri padri non hanno osservata la parola dell'Eterno, e non hanno messo in pratica tutto quello ch'è scritto in questo libro'.
\par 22 Hilkia e quelli che il re avea designati andarono dalla profetessa Hulda, moglie di Shallum, figliuolo di Tokhath, figliuolo di Hasra, il guardaroba. Essa dimorava a Gerusalemme, nel secondo quartiere; e quelli le parlarono nel senso indicato dal re.
\par 23 Ed ella disse loro: 'Così dice l'Eterno, l'Iddio d'Israele: Dite all'uomo che vi ha mandati da me:
\par 24 - Così dice l'Eterno: Ecco, io farò venire delle sciagure su questo luogo e sopra i suoi abitanti, farò venire tutte le maledizioni che sono scritte nel libro, ch'è stato letto in presenza del re di Giuda.
\par 25 Poiché essi m'hanno abbandonato ed hanno offerto profumi ad altri dèi per provocarmi ad ira con tutte le opere delle loro mani, la mia ira s'è riversata su questo luogo e non si estinguerà. -
\par 26 Quanto al re di Giuda che v'ha mandati a consultare l'Eterno, gli direte questo: Così dice l'Eterno, l'Iddio d'Israele, riguardo alle parole che tu hai udite:
\par 27 Giacché il tuo cuore è stato toccato, giacché ti sei umiliato dinanzi a Dio, udendo le sue parole contro questo luogo e contro i suoi abitanti, giacché ti sei umiliato dinanzi a me e ti sei stracciate le vesti e hai pianto dinanzi a me, anch'io t'ho ascoltato, dice l'Eterno.
\par 28 Ecco, io ti riunirò coi tuoi padri, e sarai raccolto in pace nel tuo sepolcro; e gli occhi tuoi non vedranno tutte le sciagure ch'io farò venire su questo luogo e sopra i suoi abitanti'. E quelli riferirono al re la risposta.
\par 29 Allora il re mandò a far raunare presso di sé tutti gli anziani di Giuda e di Gerusalemme.
\par 30 E il re salì alla casa dell'Eterno con tutti gli uomini di Giuda, tutti gli abitanti di Gerusalemme, i sacerdoti e i Leviti, e tutto il popolo, grandi e piccoli, e lesse in loro presenza tutte le parole del libro del patto, ch'era stato trovato nella casa dell'Eterno.
\par 31 Il re, stando in piedi sul palco, fece un patto dinanzi all'Eterno, impegnandosi di seguire l'Eterno, d'osservare i suoi comandamenti, i suoi precetti e le sue leggi con tutto il cuore e con tutta l'anima, per mettere in pratica le parole del patto scritte in quel libro.
\par 32 E fece aderire al patto tutti quelli che si trovavano a Gerusalemme e in Beniamino; e gli abitanti di Gerusalemme si conformarono al patto di Dio, dell'Iddio de' loro padri.
\par 33 E Giosia fece sparire tutte le abominazioni da tutti i paesi che appartenevano ai figliuoli d'Israele, e impose a tutti quelli che si trovavano in Israele, di servire all'Eterno, al loro Dio. Durante tutto il tempo della vita di Giosia essi non cessarono di seguire l'Eterno, l'Iddio dei loro padri.

\chapter{35}

\par 1 Giosia celebrò la Pasqua in onore dell'Eterno a Gerusalemme; e l'agnello pasquale fu immolato il quattordicesimo giorno del mese.
\par 2 Egli stabilì i sacerdoti nei loro uffici, e li incoraggiò a compiere il servizio nella casa dell'Eterno.
\par 3 E disse ai Leviti che ammaestravano tutto Israele ed erano consacrati all'Eterno: 'Collocate pure l'arca santa nella casa che Salomone, figliuolo di Davide, re d'Israele, ha edificata; voi non dovete più portarla sulle spalle; ora servite l'Eterno, il vostro Dio, e il suo popolo d'Israele;
\par 4 e tenetevi pronti secondo le vostre case patriarcali, secondo le vostre classi, conformemente a quello che hanno disposto per iscritto Davide, re d'Israele e Salomone suo figliuolo;
\par 5 e statevene nel santuario secondo i rami delle case patriarcali dei vostri fratelli, figliuoli del popolo, e secondo la classificazione della casa paterna dei Leviti.
\par 6 Immolate la Pasqua, santificatevi, e preparatela per i vostri fratelli, conformandovi alla parola dell'Eterno trasmessa per mezzo di Mosè'.
\par 7 Giosia diede alla gente del popolo, a tutti quelli che si trovavan quivi, del bestiame minuto: agnelli e capretti, in numero di trentamila: tutti per la Pasqua; e tremila buoi; e questo proveniva dai beni particolari del re.
\par 8 E i suoi principi fecero anch'essi un dono spontaneo al popolo, ai sacerdoti ed ai Leviti. Hilkia, Zaccaria e Jehiel, conduttori della casa di Dio, dettero ai sacerdoti per i sacrifizi della Pasqua, duemilaseicento capi di minuto bestiame e trecento buoi.
\par 9 Conania, Scemaia e Nethaneel suoi fratelli, e Hashabia, Jeiel e Jozabad, capi dei Leviti, dettero ai Leviti, per i sacrifizi della Pasqua, cinquemila capi di minuto bestiame e cinquecento buoi.
\par 10 Così, il servizio essendo preparato, i sacerdoti si misero al loro posto; e così pure i Leviti, secondo le loro classi, conformemente all'ordine del re.
\par 11 Poi fu immolata la Pasqua; i sacerdoti sparsero il sangue ricevuto dalle mani dei Leviti, e questi scorticarono le vittime.
\par 12 E i Leviti misero da parte quello che doveva essere arso, per darlo ai figliuoli del popolo, secondo i rami delle case paterne, perché l'offrissero all'Eterno, secondo ch'è scritto nel libro di Mosè. E lo stesso fecero per i buoi.
\par 13 Poi arrostirono le vittime pasquali sul fuoco, secondo ch'è prescritto; ma le altre vivande consacrate le cossero in pignatte, in caldaie ed in pentole, e s'affrettarono a portarle a tutti i figliuoli del popolo.
\par 14 Poi prepararono la Pasqua per se stessi e per i sacerdoti, perché i sacerdoti, figliuoli d'Aaronne, furono occupati fino alla notte a mettere sull'altare ciò che doveva esser arso, e i grassi; perciò i Leviti fecero preparativi per se stessi e per i sacerdoti, figliuoli d'Aaronne.
\par 15 I cantori, figliuoli d'Asaf, erano al loro posto, conformemente all'ordine di Davide, d'Asaf, di Heman e di Jeduthun, il veggente del re; e i portinai stavano a ciascuna porta; essi non ebbero bisogno d'allontanarsi dal loro servizio, perché i Leviti, loro fratelli, preparavan la Pasqua per loro.
\par 16 Così, in quel giorno, tutto il servizio dell'Eterno fu preparato per far la Pasqua e per offrire olocausti sull'altare dell'Eterno, conformemente all'ordine del re Giosia.
\par 17 I figliuoli d'Israele che si trovavan quivi, celebrarono allora la Pasqua e la festa degli azzimi per sette giorni.
\par 18 Nessuna Pasqua, come quella, era stata celebrata in Israele dai giorni del profeta Samuele; né alcuno dei re d'Israele avea celebrato una Pasqua pari a quella celebrata da Giosia, dai sacerdoti e dai Leviti, da tutto Giuda e Israele che si trovavan colà, e dagli abitanti di Gerusalemme.
\par 19 Questa Pasqua fu celebrata il diciottesimo anno del regno di Giosia.
\par 20 Dopo tutto questo, quando Giosia ebbe restaurato il tempio, Neco, re d'Egitto, salì per combattere a Carkemish, sull'Eufrate; e Giosia gli mosse contro.
\par 21 Ma Neco gl'inviò dei messi per dirgli: 'Che v'è egli fra me e te, o re di Giuda? Io non salgo oggi contro di te, ma contro una casa con la quale sono in guerra; e Dio m'ha comandato di far presto; bada dunque di non opporti a Dio, il quale è meco, affinch'egli non ti distrugga'.
\par 22 Ma Giosia non volle tornare indietro; anzi, si travestì per assalirlo, e non diede ascolto alle parole di Neco, che venivano dalla bocca di Dio. E venne a dar battaglia nella valle di Meghiddo.
\par 23 E gli arcieri tirarono al re Giosia; e il re disse ai suoi servi: 'Portatemi via di qui, perché son ferito gravemente'.
\par 24 I suoi servi lo tolsero dal carro e lo misero sopra un secondo carro ch'era pur suo, e lo menarono a Gerusalemme. E morì, e fu sepolto nel sepolcreto de' suoi padri. Tutto Giuda e Gerusalemme piansero Giosia.
\par 25 Geremia compose un lamento sopra Giosia; e tutti i cantori e tutte le cantatrici hanno parlato di Giosia nei loro lamenti fino al dì d'oggi, e ne hanno stabilito un'usanza in Israele. Essi si trovano scritti tra i lamenti.
\par 26 Il rimanente delle azioni di Giosia, le sue opere pie secondo i precetti della legge dell'Eterno,
\par 27 le sue azioni prime ed ultime, sono cose scritte nel libro dei re d'Israele e di Giuda.

\chapter{36}

\par 1 Allora il popolo del paese prese Joachaz, figliuolo di Giosia, e lo fece re a Gerusalemme, in luogo di suo padre.
\par 2 Joachaz avea ventitre anni quando cominciò a regnare e regnò tre mesi a Gerusalemme.
\par 3 Il re d'Egitto lo depose a Gerusalemme, e gravò il paese di un'indennità di cento talenti d'argento e d'un talento d'oro.
\par 4 E il re d'Egitto fece re sopra Giuda e sopra Gerusalemme Eliakim, fratello di Joachaz, e gli mutò il nome in quello di Joiakim. Neco prese Joachaz, fratello di lui, e lo menò in Egitto.
\par 5 Joiakim avea venticinque anni quando cominciò a regnare; regnò undici anni a Gerusalemme, e fece ciò ch'è male agli occhi dell'Eterno, il suo Dio.
\par 6 Nebucadnetsar, re di Babilonia, salì contro di lui, e lo legò con catene di rame per menarlo a Babilonia.
\par 7 Nebucadnetsar portò pure a Babilonia parte degli utensili della casa dell'Eterno, e li mise nel suo palazzo a Babilonia.
\par 8 Il rimanente delle azioni di Joiakim, le abominazioni che commise e tutto quello di cui si rese colpevole, sono cose scritte nel libro dei re d'Israele e di Giuda. E Joiakin, suo figliuolo, regnò in luogo suo.
\par 9 Joiakin aveva otto anni quando cominciò a regnare; regnò tre mesi e dieci giorni a Gerusalemme, e fece ciò ch'è male agli occhi dell'Eterno.
\par 10 L'anno seguente il re Nebucadnetsar mandò a prenderlo, lo fece menare a Babilonia con gli utensili preziosi della casa dell'Eterno, e fece re di Giuda e di Gerusalemme Sedekia, fratello di Joiakin.
\par 11 Sedekia avea ventun anni quando cominciò a regnare, e regnò a Gerusalemme undici anni.
\par 12 Egli fece ciò ch'è male agli occhi dell'Eterno, del suo Dio, e non s'umiliò dinanzi al profeta Geremia, che gli parlava da parte dell'Eterno.
\par 13 E si ribellò pure a Nebucadnetsar, che l'avea fatto giurare nel nome di Dio; e indurò la sua cervice ed il suo cuore rifiutando di convertirsi all'Eterno, all'Iddio d'Israele.
\par 14 Tutti i capi dei sacerdoti e il popolo moltiplicarono anch'essi le loro infedeltà, seguendo tutte le abominazioni delle genti; e contaminarono la casa dell'Eterno, ch'egli avea santificata a Gerusalemme.
\par 15 E l'Eterno, l'Iddio de' loro padri, mandò loro a più riprese degli ammonimenti, per mezzo dei suoi messaggeri, poiché voleva risparmiare il suo popolo e la sua propria dimora:
\par 16 ma quelli si beffarono de' messaggeri di Dio, sprezzarono le sue parole e schernirono i suoi profeti, finché l'ira dell'Eterno contro il suo popolo arrivò al punto che non ci fu più rimedio.
\par 17 Allora egli fece salire contro ad essi il re dei Caldei, che uccise di spada i loro giovani nella casa del loro santuario, e non risparmiò né giovane, né fanciulla, né vecchiaia, né canizie. L'Eterno gli diè nelle mani ogni cosa.
\par 18 Nebucadnetsar portò a Babilonia tutti gli utensili della casa di Dio, grandi e piccoli, i tesori della casa dell'Eterno, e i tesori del re e dei suoi capi.
\par 19 I Caldei incendiarono la casa di Dio, demolirono le mura di Gerusalemme, dettero alle fiamme tutti i suoi palazzi, e ne distrussero tutti gli oggetti preziosi.
\par 20 E Nebucadnetsar menò in cattività a Babilonia quelli ch'erano scampati dalla spada; ed essi furono assoggettati a lui ed ai suoi figliuoli, fino all'avvento del regno di Persia
\par 21 (affinché s'adempisse la parola dell'Eterno pronunziata per bocca di Geremia), fino a che il paese avesse goduto de' suoi sabati; difatti esso dovette riposare per tutto il tempo della sua desolazione, finché furon compiuti i settant'anni.
\par 22 Nel primo anno di Ciro, re di Persia, affinché s'adempisse la parola dell'Eterno pronunziata per bocca di Geremia, l'Eterno destò lo spirito di Ciro, re di Persia, il quale, a voce e per iscritto, fece pubblicare per tutto il suo regno quest'editto:
\par 23 'Così dice Ciro, re di Persia: L'Eterno, l'Iddio de' cieli, m'ha dato tutti i regni della terra, ed egli m'ha comandato di edificargli una casa in Gerusalemme, ch'è in Giuda. Chiunque tra voi è del suo popolo, sia l'Eterno, il suo Dio, con lui, e parta!'


\end{document}