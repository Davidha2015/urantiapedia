\begin{document}

\title{Esdra}


\chapter{1}

\par 1 Nel primo anno di Ciro, re di Persia, affinché s'adempisse la parola dell'Eterno pronunziata per bocca di Geremia, l'Eterno destò lo spirito di Ciro, re di Persia, il quale, a voce e per iscritto, fece pubblicare per tutto il suo regno quest'editto:
\par 2 'Così dice Ciro, re di Persia: L'Eterno, l'Iddio de' cieli, m'ha dato tutti i regni della terra, ed egli m'ha comandato di edificargli una casa a Gerusalemme, ch'è in Giuda.
\par 3 Chiunque tra voi è del suo popolo, sia il suo Dio con lui, e salga a Gerusalemme, ch'è in Giuda, ed edifichi la casa dell'Eterno, dell'Iddio d'Israele, dell'Iddio ch'è a Gerusalemme.
\par 4 Tutti quelli che rimangono ancora del popolo dell'Eterno, in qualunque luogo dimorino, la gente del luogo li assista con argento, con oro, con doni in natura, bestiame, aggiungendovi offerte volontarie per la casa dell'Iddio ch'è a Gerusalemme'.
\par 5 Allora i capi famiglia di Giuda e di Beniamino, i sacerdoti e i Leviti, tutti quelli ai quali Iddio avea destato lo spirito, si levarono per andare a ricostruire la casa dell'Eterno ch'è a Gerusalemme.
\par 6 E tutti i loro vicini d'ogn'intorno li fornirono d'oggetti d'argento, d'oro, di doni in natura, di bestiame, di cose preziose, oltre a tutte le offerte volontarie.
\par 7 Il re Ciro trasse fuori gli utensili della casa dell'Eterno che Nebucadnetsar avea portati via da Gerusalemme e posti nella casa del suo dio.
\par 8 Ciro, re di Persia, li fece ritirare per mezzo di Mithredath, il tesoriere, che li consegnò a Sceshbatsar, principe di Giuda.
\par 9 Eccone il numero: trenta bacini d'oro, mille bacini d'argento, ventinove coltelli,
\par 10 trenta coppe d'oro, quattrocentodieci coppe d'argento di second'ordine, mille altri utensili.
\par 11 Tutti gli oggetti d'oro e d'argento erano in numero di cinquemilaquattrocento. Sceshbatsar li riportò tutti, quando gli esuli furon ricondotti da Babilonia a Gerusalemme.

\chapter{2}

\par 1 Questi son gli uomini della provincia che tornarono dalla cattività, quelli che Nebucadnetsar, re di Babilonia, avea menati schiavi a Babilonia, e che tornarono a Gerusalemme e in Giuda, ognuno nella sua città.
\par 2 Essi vennero con Zorobabel, Jeshua, Nehemia, Seraia, Reelaia, Mardocheo, Bishan, Mispar, Bigvai, Rehum, Baana. Numero degli uomini del popolo d'Israele.
\par 3 Figliuoli di Parosh, duemilacentosettantadue.
\par 4 Figliuoli di Scefatia, trecentosettantadue.
\par 5 Figliuoli di Arah, settecentosettantacinque.
\par 6 Figliuoli di Pahath-Moab, discendenti di Jeshua e di Joab, duemilaottocentododici.
\par 7 Figliuoli di Elam, milleduecentocinquantaquattro.
\par 8 Figliuoli di Zattu, novecentoquarantacinque.
\par 9 Figliuoli di Zaccai, settecentosessanta.
\par 10 Figliuoli di Bani, seicentoquarantadue.
\par 11 Figliuoli di Bebai, seicentoventitre.
\par 12 Figliuoli di Azgad, milleduecentoventidue.
\par 13 Figliuoli di Adonikam, seicentosessantasei.
\par 14 Figliuoli di Bigvai, duemilacinquantasei.
\par 15 Figliuoli di Adin, quattrocentocinquantaquattro.
\par 16 Figliuoli di Ater, della famiglia di Ezechia, novantotto.
\par 17 Figliuoli di Betsai, trecentoventitre.
\par 18 Figliuoli di Jorah, centododici.
\par 19 Figliuoli di Hashum, duecentoventitre.
\par 20 Figliuoli di Ghibbar, novantacinque.
\par 21 Figliuoli di Bethlehem, centoventitre.
\par 22 Gli uomini di Netofa, cinquantasei.
\par 23 Gli uomini di Anatoth, centoventotto.
\par 24 Gli uomini di Azmaveth, quarantadue.
\par 25 Gli uomini di Kiriath-Arim, di Kefira e di Beeroth, settecentoquarantatre.
\par 26 Gli uomini di Rama e di Gheba, seicentoventuno.
\par 27 Gli uomini di Micmas, centoventidue.
\par 28 Gli uomini di Bethel e d'Ai, duecentoventitre.
\par 29 I figliuoli di Nebo, cinquantadue.
\par 30 I figliuoli di Magbish, centocinquantasei.
\par 31 I figliuoli d'un altro Elam, milleduecentocinquantaquattro.
\par 32 I figliuoli di Harim, trecentoventi.
\par 33 I figliuoli di Lod, di Hadid e d'Ono, settecentoventicinque.
\par 34 I figliuoli di Gerico, trecentoquarantacinque.
\par 35 I figliuoli di Senea, tremilaseicentotrenta.
\par 36 Sacerdoti: figliuoli di Jedaia, della casa di Jeshua, novecentosettantatre.
\par 37 Figliuoli d'Immer, millecinquantadue.
\par 38 Figliuoli di Pashur, milleduecentoquarantasette.
\par 39 Figliuoli di Harim, millediciassette.
\par 40 Leviti: figliuoli di Jeshua e di Kadmiel, discendenti di Hodavia, settantaquattro.
\par 41 Cantori: figliuoli di Asaf, centoventotto.
\par 42 Figliuoli de' portinai: figliuoli di Shallum, figliuoli di Ater, figliuoli di Talmon, figliuoli di Akkub, figliuoli di Hatita, figliuoli di Shobai, in tutto, centotrentanove.
\par 43 Nethinei: i figliuoli di Tsiha, i figliuoli di Hasufa, i figliuoli di Tabbaoth,
\par 44 i figliuoli di Keros, i figliuoli di Siaha, i figliuoli di Padon,
\par 45 i figliuoli di Lebana, i figliuoli di Hagaba, i figliuoli di Akkub,
\par 46 i figliuoli di Hagab, i figliuoli di Samlai, i figliuoli di Hanan,
\par 47 i figliuoli di Ghiddel, i figliuoli di Gahar, i figliuoli di Reaia,
\par 48 i figliuoli di Retsin, i figliuoli di Nekoda, i figliuoli di Gazzam,
\par 49 i figliuoli di Uzza, i figliuoli di Paseah, i figliuoli di Besai,
\par 50 i figliuoli d'Asna, i figliuoli di Mehunim, i figliuoli di Nefusim,
\par 51 i figliuoli di Bakbuk, i figliuoli di Hakufa, i figliuoli di Harhur,
\par 52 i figliuoli di Batsluth, i figliuoli di Mehida, i figliuoli di Harsha, i figliuoli di Barkos,
\par 53 i figliuoli di Sisera, i figliuoli di Thamah,
\par 54 i figliuoli di Netsiah, i figliuoli di Hatifa.
\par 55 Figliuoli dei servi di Salomone: i figliuoli di Sotai, i figliuoli di Soferet, i figliuoli di Peruda, i figliuoli di Jaala,
\par 56 i figliuoli di Darkon, i figliuoli di Ghiddel,
\par 57 i figliuoli di Scefatia, i figliuoli di Hattil, i figliuoli di Pokereth-Hatsebaim, i figliuoli d'Ami.
\par 58 Tutti i Nethinei e i figliuoli de' servi di Salomone ammontarono a trecentonovantadue.
\par 59 Ed ecco quelli che tornarono da Tel-Melah, da Tel-Harsha, da Kerub-Addan, da Immer, e che non poterono indicare la loro casa patriarcale e la loro discendenza per provare ch'erano d'Israele:
\par 60 i figliuoli di Delaia, i figliuoli di Tobia, i figliuoli di Nekoda, in tutto, seicentocinquantadue.
\par 61 E di tra i figliuoli de' sacerdoti: i figliuoli di Habaia, i figliuoli di Hakkots, i figliuoli di Barzillai, che avea preso per moglie una delle figliuole di Barzillai, il Galaadita, e fu chiamato col nome loro.
\par 62 Questi cercarono i loro titoli genealogici, ma non li trovarono; furon quindi esclusi, come impuri, dal sacerdozio;
\par 63 e il governatore disse loro di non mangiare cose santissime finché non si presentasse un sacerdote per consultar Dio con l'Urim e il Thummim.
\par 64 La raunanza, tutt'assieme, noverava quarantaduemilatrecentosessanta persone,
\par 65 senza contare i loro servi e le loro serve, che ammontavano a settemilatrecentotrentasette. Avean pure duecento cantori e cantatrici.
\par 66 Aveano settecentotrentasei cavalli, duecentoquarantacinque muli,
\par 67 quattrocentotrentacinque cammelli e seimilasettecentoventi asini.
\par 68 Alcuni dei capi famiglia, come furon giunti alla casa dell'Eterno ch'è a Gerusalemme, offriron dei doni volontari per la casa di Dio, per rimetterla in piè sul luogo di prima.
\par 69 Dettero al tesoro dell'opera, secondo i loro mezzi, sessantunmila dariche d'oro, cinquemila mine d'argento e cento vesti sacerdotali.
\par 70 I sacerdoti, i Leviti, la gente del popolo, i cantori, i portinai, i Nethinei, si stabiliron nelle loro città; e tutti gl'Israeliti, nelle città rispettive.

\chapter{3}

\par 1 Or come fu giunto il settimo mese, e i figliuoli d'Israele si furono stabiliti nelle loro città, il popolo si adunò come un sol uomo a Gerusalemme.
\par 2 Allora Jeshua, figliuolo di Jotsadak, coi suoi fratelli sacerdoti, e Zorobabel, figliuolo di Scealtiel, coi suoi fratelli, si levarono e costruirono l'altare dell'Iddio d'Israele, per offrirvi sopra degli olocausti, com'è scritto nella legge di Mosè, uomo di Dio.
\par 3 Ristabilirono l'altare sulle sue basi, benché avessero paura a motivo dei popoli delle terre vicine, e vi offriron sopra olocausti all'Eterno: gli olocausti del mattino e della sera.
\par 4 E celebrarono la festa delle Capanne, nel modo ch'è scritto, e offersero giorno per giorno olocausti, secondo il numero prescritto per ciascun giorno;
\par 5 poi offersero l'olocausto perpetuo, gli olocausti dei noviluni e di tutte le solennità sacre all'Eterno, e quelli di chiunque faceva qualche offerta volontaria all'Eterno.
\par 6 Dal primo giorno del settimo mese cominciarono a offrire olocausti all'Eterno; ma le fondamenta del tempio dell'Eterno non erano ancora state gettate.
\par 7 E diedero del danaro agli scalpellini ed ai legnaiuoli, e de' viveri e delle bevande e dell'olio ai Sidonî e ai Tirî perché portassero per mare sino a Jafo del legname di cedro del Libano, secondo la concessione che Ciro, re di Persia, avea loro fatta.
\par 8 Il secondo anno del loro arrivo alla casa di Dio a Gerusalemme, il secondo mese, Zorobabel, figliuolo di Scealtiel, Jeshua, figliuolo di Jotsadak, con gli altri loro fratelli sacerdoti e Leviti, e tutti quelli ch'eran tornati dalla cattività a Gerusalemme, si misero all'opra; e incaricarono i Leviti dai vent'anni in su di dirigere i lavori della casa dell'Eterno.
\par 9 E Jeshua, coi suoi figliuoli, e i suoi fratelli, Kadmiel coi suoi figliuoli, figliuoli di Giuda, si presentarono come un sol uomo per dirigere quelli che lavoravano alla casa di Dio; lo stesso fecero i figliuoli di Henadad coi loro figliuoli e coi loro fratelli Leviti.
\par 10 E quando i costruttori gettaron le fondamenta del tempio dell'Eterno, vi si fecero assistere i sacerdoti vestiti de' loro paramenti, con delle trombe, e i Leviti, figliuoli d'Asaf, con de' cembali, per lodare l'Eterno, secondo le direzioni date da Davide, re d'Israele.
\par 11 Ed essi cantavano rispondendosi a vicenda, celebrando e lodando l'Eterno, "perch'egli è buono, perché la sua benignità verso Israele dura in perpetuo". E tutto il popolo mandava alti gridi di gioia, lodando l'Eterno, perché s'eran gettate le fondamenta della casa dell'Eterno.
\par 12 E molti sacerdoti, Leviti e capi famiglia anziani che avean veduta la prima casa, piangevano ad alta voce mentre si gettavano le fondamenta della nuova casa. Molti altri invece alzavan le loro voci, gridando per allegrezza;
\par 13 in guisa che non si potea discernere il rumore delle grida d'allegrezza da quello del pianto del popolo; perché il popolo mandava di gran gridi, e il rumore se n'udiva di lontano.

\chapter{4}

\par 1 Or i nemici di Giuda e di Beniamino, avendo saputo che quelli ch'erano stati in cattività edificavano un tempio all'Eterno, all'Iddio d'Israele,
\par 2 s'avvicinarono a Zorobabel ed ai capi famiglia, e dissero loro: 'Noi edificheremo con voi, giacché, come voi, noi cerchiamo il vostro Dio, e gli offriamo de' sacrifizi dal tempo di Esar-Haddon, re d'Assiria, che ci fece salir qui'.
\par 3 Ma Zorobabel, Jeshua, e gli altri capi famiglia d'Israele risposero loro: 'Non spetta a voi ed a noi insieme di edificare una casa al nostro Dio; noi soli la edificheremo all'Eterno, all'Iddio d'Israele, come Ciro, re di Persia, ce l'ha comandato'.
\par 4 Allora la gente del paese si mise a scoraggiare il popolo di Giuda, a molestarlo per impedirgli di fabbricare,
\par 5 e a comprare de' consiglieri per frustrare il suo divisamento; e questo durò per tutta la vita di Ciro, re di Persia, e fino al regno di Dario, re di Persia.
\par 6 Sotto il regno d'Assuero, al principio del suo regno, scrissero un'accusa contro gli abitanti di Giuda e di Gerusalemme.
\par 7 Poi, al tempo d'Artaserse, Bishlam, Mithredath, Tabeel e gli altri loro colleghi scrissero ad Artaserse, re di Persia. La lettera era scritta in caratteri aramaici e tradotta in aramaico.
\par 8 Rehum il governatore e Scimshai il segretario scrissero una lettera contro Gerusalemme al re Artaserse, in questi termini:
\par 9 - La data: 'Rehum il governatore, Scimshai il segretario, e gli altri loro colleghi di Din, d'Afarsathac, di Tarpel, d'Afaras, d'Erec, di Babilonia, di Shushan, di Deha, d'Elam,
\par 10 e gli altri popoli che il grande e illustre Osnapar ha trasportati e stabiliti nella città di Samaria, e gli altri che stanno di là dal fiume...' ecc.
\par 11 Ecco la copia della lettera che inviarono al re Artaserse: 'I tuoi servi, la gente d'oltre il fiume, ecc.
\par 12 Sappia il re che i Giudei che son partiti da te e giunti fra noi a Gerusalemme, riedificano la città ribelle e malvagia, ne rialzano le mura e ne restaurano le fondamenta.
\par 13 Sappia dunque il re che, se questa città si riedifica e se le sue mura si rialzano, essi non pagheranno più né tributo né imposta né pedaggio, e il tesoro dei re n'avrà a soffrire.
\par 14 Or siccome noi mangiamo il sale del palazzo e non ci sembra conveniente lo stare a vedere il danno del re, mandiamo al re questa informazione.
\par 15 Si facciano delle ricerche nel libro delle memorie de' tuoi padri; e nel libro delle memorie troverai e apprenderai che questa città è una città ribelle, perniciosa a re ed a province, e che fin da tempi antichi vi si son fatte delle sedizioni; per queste ragioni, la città è stata distrutta.
\par 16 Noi facciamo sapere al re che, se questa città si riedifica e le sue mura si rialzano, tu non avrai più possessi da questo lato del fiume'.
\par 17 Il re mandò questa risposta a Rehum il governatore, a Scimshai il segretario, e al resto dei loro colleghi che stavano a Samaria e altrove di là dal fiume: 'Salute, ecc.
\par 18 La lettera che ci avete mandata, è stata esattamente letta in mia presenza;
\par 19 ed io ho dato ordine di far delle ricerche; e s'è trovato che fin da tempi antichi cotesta città è insorta contro ai re e vi si son fatte delle sedizioni e delle rivolte.
\par 20 Vi sono stati a Gerusalemme dei re potenti, che signoreggiarono su tutto il paese ch'è di là dal fiume, e ai quali si pagavano tributi, imposte e pedaggi.
\par 21 Date dunque ordine che quella gente sospenda i lavori, e che cotesta città non si riedifichi prima che ordine ne sia dato da me.
\par 22 E badate di non esser negligenti in questo, onde il danno non venga a crescere in pregiudizio dei re'.
\par 23 Non appena la copia della lettera del re Artaserse fu letta in presenza di Rehum, di Scimshai il segretario, e dei loro colleghi, essi andarono in fretta a Gerusalemme dai Giudei, e li obbligarono, a mano armata, a sospendere i lavori.
\par 24 Allora fu sospesa l'opera della casa di Dio a Gerusalemme e rimase sospesa fino al secondo anno del regno di Dario, re di Persia.

\chapter{5}

\par 1 Or i profeti Aggeo e Zaccaria, figliuolo d'Iddo, profetarono nel nome dell'Iddio d'Israele, ai Giudei ch'erano in Giuda ed a Gerusalemme.
\par 2 Allora Zorobabel, figliuolo di Scealtiel, e Jeshua, figliuolo di Jotsadak, si levarono e ricominciarono a edificare la casa di Dio a Gerusalemme; e con essi erano i profeti di Dio, che li secondavano.
\par 3 In quel medesimo tempo giunsero da loro Tattenai, governatore d'oltre il fiume, Scethar-Boznai e i loro colleghi, e parlaron loro così: 'Chi v'ha dato ordine di edificare questa casa e di rialzare queste mura?'
\par 4 Poi aggiunsero: 'Quali sono i nomi degli uomini che costruirono quest'edifizio?'
\par 5 Ma sugli anziani dei Giudei vegliava l'occhio del loro Dio e quelli non li fecero cessare i lavori, finché la cosa non fosse stata sottoposta a Dario, e da lui fosse giunta una risposta in proposito.
\par 6 Copia della lettera mandata al re Dario da Tattenai, governatore d'oltre il fiume, da Scethar-Boznai, e dai suoi colleghi, gli Afarsakiti, ch'erano oltre il fiume.
\par 7 Gl'inviarono un rapporto così concepito: 'Al re Dario, perfetta salute!
\par 8 Sappia il re che noi siamo andati nella provincia di Giuda, alla casa del gran Dio. Essa si costruisce con blocchi di pietra, e nelle pareti s'interpongono dei legnami; l'opera vien fatta con cura e progredisce nelle loro mani.
\par 9 Noi abbiamo interrogato quegli anziani, e abbiam parlato loro così: - Chi v'ha dato ordine di edificare questa casa e di rialzare queste mura? -
\par 10 Abbiamo anche domandato loro i loro nomi per notificarteli, mettendo in iscritto i nomi degli uomini che stanno loro a capo.
\par 11 E questa è la risposta che ci hanno data: - Noi siamo i servi dell'Iddio del cielo e della terra, e riedifichiamo la casa ch'era stata edificata già molti anni fa: un gran re d'Israele l'aveva edificata e compiuta.
\par 12 Ma avendo i nostri padri provocato ad ira l'Iddio del cielo, Iddio li diede in mano di Nebucadnetsar, re di Babilonia, il Caldeo, il quale distrusse questa casa, e menò il popolo in cattività a Babilonia.
\par 13 Ma il primo anno di Ciro, re di Babilonia, il re Ciro diè ordine che questa casa di Dio fosse riedificata.
\par 14 E il re Ciro trasse pure dal tempio di Babilonia gli utensili d'oro e d'argento della casa di Dio, che Nebucadnetsar avea portati via dal tempio di Gerusalemme e trasportati nel tempio di Babilonia; li fece consegnare a uno chiamato Sceshbatsar, ch'egli aveva fatto governatore, e gli disse:
\par 15 Prendi questi utensili, va' a riporli nel tempio di Gerusalemme, e la casa di Dio sia riedificata dov'era.
\par 16 Allora lo stesso Sceshbatsar venne e gettò le fondamenta della casa di Dio a Gerusalemme; da quel tempo fino ad ora essa è in costruzione, ma non è ancora finita.
\par 17 Or dunque, se così piaccia al re, si faccian delle ricerche nella casa dei tesori del re a Babilonia, per accertare se vi sia stato un ordine dato dal re Ciro per la costruzione di questa casa a Gerusalemme; e ci trasmetta il re il suo beneplacito a questo riguardo'. -

\chapter{6}

\par 1 Allora il re Dario ordinò che si facessero delle ricerche nella casa degli archivi dov'erano riposti i tesori a Babilonia;
\par 2 e nel castello d'Ahmetha, ch'è nella provincia di Media, si trovò un rotolo, nel quale stava scritto così:
\par 3 'Memoria. - Il primo anno del re Ciro, il re Ciro ha pubblicato quest'editto, concernente la casa di Dio a Gerusalemme: La casa sia riedificata per essere un luogo dove si offrono dei sacrifizi; e le fondamenta che se ne getteranno, siano solide. Abbia sessanta cubiti d'altezza, sessanta cubiti di larghezza,
\par 4 tre ordini di blocchi di pietra e un ordine di travatura nuova; e la spesa sia pagata dalla casa reale.
\par 5 E inoltre, gli utensili d'oro e d'argento della casa di Dio, che Nebucadnetsar avea tratti dal tempio di Gerusalemme e trasportati a Babilonia, siano restituiti e riportati al tempio di Gerusalemme, nel luogo dov'erano prima, e posti nella casa di Dio'.
\par 6 'Or dunque tu, Tattenai, governatore d'oltre il fiume, tu, Scethar-Boznai, e voi, loro colleghi d'Afarsak, che state di là dal fiume, statevene lontani da quel luogo!
\par 7 Lasciate continuare i lavori di quella casa di Dio; il governatore de' Giudei e gli anziani de' Giudei riedifichino quella casa di Dio nel sito di prima.
\par 8 E questo è l'ordine ch'io do relativamente al vostro modo di procedere verso quegli anziani de' Giudei nella ricostruzione di quella casa di Dio: le spese, detratte dalle entrate del re provenienti dai tributi d'oltre il fiume, siano puntualmente pagate a quegli uomini, affinché i lavori non siano interrotti.
\par 9 E le cose necessarie per gli olocausti all'Iddio dei cieli: vitelli, montoni, agnelli; e frumento, sale, vino, olio, siano forniti ai sacerdoti di Gerusalemme a loro richiesta, giorno per giorno e senza fallo,
\par 10 affinché offrano sacrifizi di odor soave all'Iddio del cielo, e preghino per la vita del re e de' suoi figliuoli.
\par 11 E questo è pure l'ordine ch'io do: Se qualcuno contravverrà a questo decreto, si tragga dalla casa di lui una trave, la si rizzi, vi sia egli inchiodato sopra, e la sua casa, per questo motivo, diventi un letamaio.
\par 12 L'Iddio che ha fatto di quel luogo la dimora del suo nome, distrugga ogni re ed ogni popolo che stendesse la mano per trasgredire la mia parola, per distruggere la casa di Dio ch'è in Gerusalemme! Io, Dario, ho emanato questo decreto: sia eseguito con ogni prontezza'.
\par 13 Allora Tattenai, governatore d'oltre il fiume, Scethar-Boznai e i loro colleghi, poiché il re Dario avea così decretato, eseguirono puntualmente i suoi ordini.
\par 14 E gli anziani de' Giudei tirarono innanzi e fecero progredire la fabbrica, aiutati dalle parole ispirate del profeta Aggeo, e di Zaccaria figliuolo d'Iddo. E finirono i loro lavori di costruzione secondo il comandamento dell'Iddio d'Israele, e secondo gli ordini di Ciro, di Dario e d'Artaserse, re di Persia.
\par 15 E la casa fu finita il terzo giorno del mese d'Adar, il sesto anno del regno di Dario.
\par 16 I figliuoli d'Israele, i sacerdoti, i Leviti e gli altri reduci dalla cattività celebrarono con gioia la dedicazione di questa casa di Dio.
\par 17 E per la dedicazione di questa casa di Dio offrirono cento giovenchi, duecento montoni, quattrocento agnelli; e come sacrifizio per il peccato per tutto Israele, dodici capri, secondo il numero delle tribù d'Israele.
\par 18 E stabilirono i sacerdoti secondo le loro classi, e i Leviti secondo le loro divisioni, per il servizio di Dio a Gerusalemme, come sta scritto nel libro di Mosè.
\par 19 Poi, i reduci dalla cattività celebrarono la Pasqua il quattordicesimo giorno del primo mese,
\par 20 poiché i sacerdoti e i Leviti s'erano purificati come se non fossero stati che un sol uomo; tutti erano puri; e immolarono la Pasqua per tutti i reduci dalla cattività, per i sacerdoti loro fratelli, e per loro stessi.
\par 21 Così i figliuoli d'Israele ch'eran tornati dalla cattività e tutti quelli che s'eran separati dall'impurità della gente del paese e che s'unirono a loro per cercare l'Eterno, l'Iddio d'Israele, mangiarono la Pasqua.
\par 22 E celebrarono con gioia la festa degli azzimi per sette giorni, perché l'Eterno li avea rallegrati, e avea piegato in lor favore il cuore del re d'Assiria, in modo da fortificare le loro mani nell'opera della casa di Dio, dell'Iddio d'Israele.

\chapter{7}

\par 1 Or dopo queste cose, sotto il regno d'Artaserse, re di Persia, giunse Esdra, figliuolo di Seraia, figliuolo d'Azaria, figliuolo di Hilkia,
\par 2 figliuolo di Shallum, figliuolo di Tsadok, figliuolo d'Ahitub,
\par 3 figliuolo d'Amaria, figliuolo d'Azaria, figliuolo di Meraioth,
\par 4 figliuolo di Zerahia, figliuolo di Uzzi,
\par 5 figliuolo di Bukki, figliuolo di Abishua, figliuolo di Fineas, figliuolo di Eleazar, figliuolo d'Aaronne, il sommo sacerdote.
\par 6 Quest'Esdra veniva da Babilonia; era uno scriba versato nella legge di Mosè data dall'Eterno, dall'Iddio d'Israele; e siccome la mano dell'Eterno, del suo Dio, era su lui, il re gli concedette tutto quello che domandò.
\par 7 E alcuni dei figliuoli d'Israele e alcuni de' sacerdoti, de' Leviti, de' cantori, de' portinai e de' Nethinei saliron pure con lui a Gerusalemme, il settimo anno del re Artaserse.
\par 8 Esdra giunse a Gerusalemme il quinto mese, nel settimo anno del re.
\par 9 Infatti, avea fissata la partenza da Babilonia per il primo giorno del primo mese, e arrivò a Gerusalemme il primo giorno del quinto mese, assistito dalla benefica mano del suo Dio.
\par 10 Poiché Esdra aveva applicato il cuore allo studio ed alla pratica della legge dell'Eterno, e ad insegnare in Israele le leggi e le prescrizioni divine.
\par 11 Or ecco la copia della lettera data dal re Artaserse a Esdra, sacerdote e scriba, scriba versato nei comandamenti e nelle leggi dati dall'Eterno ad Israele:
\par 12 'Artaserse, re dei re, a Esdra, sacerdote, scriba versato nella legge dell'Iddio del cielo, ecc.
\par 13 Da me è decretato che nel mio regno, chiunque del popolo d'Israele, de' suoi sacerdoti e dei Leviti sarà disposto a partire con te per Gerusalemme, vada pure;
\par 14 giacché tu sei mandato da parte del re e dai suoi sette consiglieri per informarti in Giuda e in Gerusalemme come vi sia osservata la legge del tuo Dio, la quale tu hai nelle mani,
\par 15 e per portare l'argento e l'oro che il re ed i suoi consiglieri hanno volenterosamente offerto all'Iddio d'Israele, la cui dimora è a Gerusalemme,
\par 16 e tutto l'argento e l'oro che troverai in tutta la provincia di Babilonia, e i doni volontari fatti dal popolo e dai sacerdoti per la casa del loro Dio a Gerusalemme.
\par 17 Tu avrai quindi cura di comprare con questo danaro de' giovenchi, dei montoni, degli agnelli, e ciò che occorre per le relative oblazioni e libazioni, e li offrirai sull'altare della casa del vostro Dio ch'è a Gerusalemme.
\par 18 Del rimanente dell'argento e dell'oro farete, tu e i tuoi fratelli, quel che meglio vi parrà, conformandovi alla volontà del vostro Dio.
\par 19 Quanto agli utensili che ti son dati per il servizio della casa dell'Iddio tuo, rimettili davanti all'Iddio di Gerusalemme.
\par 20 E qualunque altra spesa ti occorrerà di fare per la casa del tuo Dio, ne trarrai l'ammontare dal tesoro della casa reale.
\par 21 Io, il re Artaserse, do ordine a tutti i tesorieri d'oltre il fiume di consegnare senza dilazione a Esdra, sacerdote e scriba, versato nella legge dell'Iddio del cielo, tutto quello che vi chiederà,
\par 22 fino a cento talenti d'argento, a cento cori di grano, a cento bati di vino, a cento bati d'olio, e a una quantità illimitata di sale.
\par 23 Tutto quello ch'è comandato dall'Iddio del cielo sia puntualmente fatto per la casa dell'Iddio del cielo. Perché l'ira di Dio dovrebbe ella venire sopra il regno, sopra il re e i suoi figliuoli?
\par 24 Vi facciamo inoltre sapere che non è lecito a nessuno esigere alcun tributo o imposta o pedaggio da alcuno de' sacerdoti, de' Leviti, de' cantori, de' portinai, de' Nethinei e de' servi di questa casa di Dio.
\par 25 E tu, Esdra, secondo la sapienza di cui il tuo Dio ti ha dotato, stabilisci de' magistrati e de' giudici che amministrino la giustizia a tutto il popolo d'oltre il fiume, a tutti quelli che conoscono le leggi del tuo Dio; e fatele voi conoscere a chi non le conosce.
\par 26 E di chiunque non osserverà la legge del tuo Dio e la legge del re farete pronta giustizia, punendolo con la morte o col bando o con multa pecuniaria o col carcere'.
\par 27 Benedetto sia l'Eterno, l'Iddio de' nostri padri, che ha così disposto il cuore del re ad onorare la casa dell'Eterno, a Gerusalemme,
\par 28 e che m'ha conciliato la benevolenza del re, de' suoi consiglieri e di tutti i suoi potenti capi! Ed io, fortificato dalla mano dell'Eterno, del mio Dio, ch'era su me, radunai i capi d'Israele perché partissero meco.

\chapter{8}

\par 1 Questi sono i capi delle case patriarcali e la lista genealogica di quelli che tornaron meco da Babilonia, sotto il regno di Artaserse.
\par 2 Dei figliuoli di Fineas, Ghershom; de' figliuoli d'Ithamar, Daniele; dei figliuoli di Davide, Hattush.
\par 3 Dei figliuoli di Scecania: dei figliuoli di Parosh, Zaccaria, e con lui furono registrati centocinquanta maschi.
\par 4 Dei figliuoli di Pahath-Moab, Elioenai, figliuolo di Zerahia, e con lui duecento maschi.
\par 5 Dei figliuoli di Scecania, il figliuolo di Jahaziel, e con lui trecento maschi.
\par 6 Dei figliuoli di Adin, Ebed, figliuolo di Jonathan, e con lui cinquanta maschi.
\par 7 Dei figliuoli di Elam, Isaia, figliuolo di Athalia, e con lui settanta maschi.
\par 8 Dei figliuoli di Scefatia, Zebadia, figliuolo di Micael, e con lui ottanta maschi.
\par 9 Dei figliuoli di Joab, Obadia, figliuolo di Jehiel, e con lui duecentodiciotto maschi.
\par 10 Dei figliuoli di Scelomith, il figliuolo di Josifia, e con lui centosessanta maschi.
\par 11 Dei figliuoli di Bebai, Zaccaria, figliuolo di Bebai, e con lui ventotto maschi.
\par 12 Dei figliuoli d'Azgad, Johanan, figliuolo di Hakkatan, e con lui centodieci maschi.
\par 13 Dei figliuoli d'Adonikam, gli ultimi, de' quali questi sono i nomi: Elifelet, Jehiel, Scemaia, e con loro sessanta maschi.
\par 14 E dei figliuoli di Bigvai, Uthai e Zabbud, e con lui settanta maschi.
\par 15 Io li radunai presso al fiume che scorre verso Ahava, e quivi stemmo accampati tre giorni; e, avendo fatta la rassegna del popolo e dei sacerdoti, non trovai tra loro alcun figliuolo di Levi.
\par 16 Allora feci chiamare i capi Eliezer, Ariel, Scemaia, Elnathan, Jarib, Elnathan, Nathan, Zaccaria, Meshullam, e i dottori Joiarib ed Elnathan,
\par 17 e ordinai loro d'andare dal capo Iddo, che stava a Casifia, e posi loro in bocca le parole che dovean dire a Iddo e a suo fratello, ch'eran preposti al luogo di Casifia, perché ci menassero degli uomini per fare il servizio della casa del nostro Dio.
\par 18 E siccome la benefica mano del nostro Dio era su noi, ci menarono Scerebia, uomo intelligente, dei figliuoli di Mahli, figliuolo di Levi, figliuolo d'Israele, e con lui i suoi figliuoli e i suoi fratelli, in numero di diciotto;
\par 19 Hashabia, e con lui Isaia, dei figliuoli di Merari, i suoi fratelli e i suoi figliuoli, in numero di venti;
\par 20 e dei Nethinei, che Davide e i capi aveano messo al servizio de' Leviti, duecentoventi Nethinei, tutti quanti designati per nome.
\par 21 E colà, presso il fiume Ahava, io bandii un digiuno per umiliarci nel cospetto del nostro Dio, per chiedergli un buon viaggio per noi, per i nostri bambini, e per tutto quello che ci apparteneva;
\par 22 perché, io mi vergognavo di chiedere al re una scorta armata e de' cavalieri per difenderci per istrada dal nemico, giacché avevamo detto al re: 'La mano del nostro Dio assiste tutti quelli che lo cercano; ma la sua potenza e la sua ira sono contro tutti quelli che l'abbandonano'.
\par 23 Così digiunammo e invocammo il nostro Dio a questo proposito, ed egli ci esaudì.
\par 24 Allora io separai dodici dei capi sacerdoti: Scerebia, Hashabia e dieci dei loro fratelli,
\par 25 e pesai loro l'argento, l'oro, gli utensili, ch'eran l'offerta fatta per la casa del nostro Dio dal re, dai suoi consiglieri, dai suoi capi, e da tutti quei d'Israele che si trovan colà.
\par 26 Rimisi dunque nelle loro mani seicentocinquanta talenti d'argento, degli utensili d'argento per il valore di cento talenti, cento talenti d'oro,
\par 27 venti coppe d'oro del valore di mille dariche, due vasi di rame lucente finissimo, prezioso come l'oro,
\par 28 e dissi loro: 'Voi siete consacrati all'Eterno; questi utensili sono sacri, e quest'argento e quest'oro sono un'offerta volontaria fatta all'Eterno, all'Iddio de' vostri padri;
\par 29 vigilate e custoditeli, finché li pesiate in presenza dei capi sacerdoti, dei Leviti e dei capi delle famiglie d'Israele a Gerusalemme, nelle camere della casa dell'Eterno'.
\par 30 I sacerdoti e i Leviti dunque ricevettero pesato l'argento e l'oro, e gli utensili, per portarli a Gerusalemme nella casa del nostro Dio.
\par 31 E noi ci partimmo dal fiume d'Ahava il dodicesimo giorno del primo mese per andare a Gerusalemme; e la mano di Dio fu su noi, e ci liberò dalla mano del nemico e da ogni insidia, durante il viaggio.
\par 32 Arrivammo a Gerusalemme; e dopo esserci riposati quivi tre giorni,
\par 33 il quarto giorno pesammo nella casa del nostro Dio l'argento, l'oro e gli utensili, che consegnammo al sacerdote Meremoth, figliuolo d'Uria; con lui era Eleazar, figliuolo di Fineas, e con loro erano i Leviti Jozabad, figliuolo di Jeshua, e Noadia, figliuolo di Binnu.
\par 34 Tutto fu contato e pesato, e nello stesso tempo il peso di tutto fu messo per iscritto.
\par 35 Gli esuli, tornati dalla cattività, offersero in olocausti all'Iddio d'Israele dodici giovenchi per tutto Israele, novantasei montoni, settantasette agnelli; e, come sacrifizio per il peccato, dodici capri: tutto questo, in olocausto all'Eterno.
\par 36 E presentarono i decreti del re ai satrapi del re e ai governatori d'oltre il fiume, i quali favoreggiarono il popolo e la casa di Dio.

\chapter{9}

\par 1 Or quando queste cose furon finite, i capi s'accostarono a me, dicendo: 'Il popolo d'Israele, i sacerdoti e i Leviti non si son separati dai popoli di questi paesi, ma si conformano alle abominazioni de' Cananei, degli Hittei, de' Ferezei, dei Gebusei, degli Ammoniti, dei Moabiti, degli Egiziani e degli Amorei.
\par 2 Poiché hanno preso delle loro figliuole per sé e per i propri figliuoli, e hanno mescolato la stirpe santa coi popoli di questi paesi; e i capi e i magistrati sono stati i primi a commettere questa infedeltà'.
\par 3 Quand'io ebbi udito questo, mi stracciai le vesti e il mantello, mi strappai i capelli della testa e della barba, e mi misi a sedere, costernato.
\par 4 Allora tutti quelli che tremavano alle parole dell'Iddio d'Israele si radunarono presso di me a motivo della infedeltà di quelli ch'eran tornati dalla cattività; e io rimasi così seduto e costernato, fino al tempo dell'oblazione della sera.
\par 5 E al momento dell'oblazione della sera, m'alzai dalla mia umiliazione, colle vesti e col mantello stracciati; caddi in ginocchio; stesi le mani verso l'Eterno, il mio Dio, e dissi:
\par 6 'O mio Dio, io son confuso; e mi vergogno, o mio Dio, d'alzare a te la mia faccia; poiché le nostre iniquità si son moltiplicate fino al disopra del nostro capo, e la nostra colpa è sì grande che arriva al cielo.
\par 7 Dal tempo dei nostri padri fino al dì d'oggi siamo stati grandemente colpevoli; e a motivo delle nostre iniquità, noi, i nostri re, i nostri sacerdoti, siamo stati dati in mano dei re dei paesi stranieri, in balìa della spada, dell'esilio, della rapina e dell'obbrobrio, come anch'oggi si vede.
\par 8 Ed ora, per un breve istante, l'Eterno, il nostro Dio, ci ha fatto grazia, lasciandoci alcuni superstiti, e concedendoci un asilo nel suo santo luogo, affin d'illuminare gli occhi nostri, e di darci un po' di respiro in mezzo al nostro servaggio.
\par 9 Poiché noi siamo schiavi; ma il nostro Dio non ci ha abbandonati nel nostro servaggio; ché anzi ha fatto sì che trovassimo benevolenza presso i re di Persia, i quali ci hanno dato tanto respiro da poter rimettere in piè la casa dell'Iddio nostro e restaurarne le rovine, e ci hanno concesso un ricovero in Giuda ed in Gerusalemme.
\par 10 Ed ora, o nostro Dio, che direm noi dopo questo? Poiché noi abbiamo abbandonati i tuoi comandamenti,
\par 11 quelli che ci desti per mezzo de' tuoi servi i profeti, dicendo: - Il paese nel quale entrate per prenderne possesso, è un paese reso impuro dalla impurità dei popoli di questi paesi, dalle abominazioni con le quali l'hanno riempito da un capo all'altro con le loro contaminazioni.
\par 12 Or dunque non date le vostre figliuole ai loro figliuoli, e non prendete le loro figliuole per i vostri figliuoli, e non cercate mai la loro prosperità né il loro benessere, e così diventerete forti, mangerete i migliori prodotti del paese, e lo lascerete in retaggio perpetuo ai vostri figliuoli.
\par 13 - Ora, dopo tutto quello che ci è avvenuto a motivo delle nostre azioni malvage e delle nostre grandi colpe, giacché tu, o nostro Dio, ci hai puniti meno severamente di quanto le nostre iniquità avrebbero meritato, e hai conservato di noi un residuo come questo,
\par 14 torneremmo noi di nuovo a violare i tuoi comandamenti e ad imparentarci coi popoli che commettono queste abominazioni? L'ira tua non s'infiammerebbe essa contro di noi sino a consumarci e a non lasciar più né residuo né superstite?
\par 15 O Eterno, Dio d'Israele, tu sei giusto, e perciò noi siamo oggi ridotti ad un residuo di scampati. Ed eccoci dinanzi a te a riconoscere la nostra colpa; poiché per cagion d'essa, noi non potremmo sussistere nel tuo cospetto!'

\chapter{10}

\par 1 Or mentre Esdra pregava e faceva questa confessione piangendo e prostrato davanti alla casa di Dio, si raunò intorno a lui una grandissima moltitudine di gente d'Israele, uomini, donne e fanciulli; e il popolo piangeva dirottamente.
\par 2 Allora Scecania, figliuolo di Jehiel, uno de' figliuoli di Elam, prese a dire a Esdra: 'Noi siamo stati infedeli al nostro Dio, sposando donne straniere prese dai popoli di questo paese; nondimeno, rimane ancora, a questo riguardo, una speranza a Israele.
\par 3 Facciamo un patto col nostro Dio impegnandoci a rimandare tutte queste donne e i figliuoli nati da esse, come consigliano il mio signore e quelli che tremano dinanzi ai comandamenti del nostro Dio. E facciasi quel che vuole la legge.
\par 4 Lèvati, poiché questo è affar tuo, e noi sarem teco. Fatti animo, ed agisci!'
\par 5 Allora Esdra si levò, fece giurare ai capi de' sacerdoti, de' Leviti, e di tutto Israele che farebbero com'era stato detto. E quelli giurarono.
\par 6 Poi Esdra si levò d'innanzi alla casa di Dio, e andò nella camera di Johanan, figliuolo di Eliascib; e come vi fu entrato, non mangiò pane né bevve acqua, perché facea cordoglio per la infedeltà di quelli ch'erano stati in esilio.
\par 7 E si bandì in Giuda e a Gerusalemme che tutti quelli della cattività si adunassero a Gerusalemme;
\par 8 e che chiunque non venisse entro tre giorni seguendo il consiglio dei capi e degli anziani, tutti i suoi beni gli sarebbero confiscati, ed egli stesso sarebbe escluso dalla raunanza de' reduci dalla cattività.
\par 9 Così tutti gli uomini di Giuda e di Beniamino s'adunarono a Gerusalemme entro i tre giorni. Era il ventesimo giorno del nono mese. Tutto il popolo stava sulla piazza della casa di Dio, tremante per cagion di questa cosa ed a causa della gran pioggia.
\par 10 E il sacerdote Esdra si levò e disse loro: 'Voi avete commesso una infedeltà, sposando donne straniere, e avete accresciuta la colpa d'Israele.
\par 11 Ma ora rendete omaggio all'Eterno, all'Iddio de' vostri padri, e fate quel che a lui piace! Separatevi dai popoli di questo paese e dalle donne straniere!'
\par 12 Allora tutta la raunanza rispose e disse ad alta voce: 'Sì, dobbiam fare come tu hai detto!
\par 13 Ma il popolo è in gran numero, e il tempo è molto piovoso e non possiamo stare allo scoperto; e questo non è affar d'un giorno o due, poiché siamo stati numerosi a commettere questo peccato.
\par 14 Rimangano dunque qui i capi di tutta la raunanza; e tutti quelli che nelle nostre città hanno sposato donne straniere vengano a tempi determinati, con gli anziani e i giudici d'ogni città, finché non sia rimossa da noi l'ardente ira del nostro Dio, per questa infedeltà'.
\par 15 Jonathan, figliuolo di Asael, e Jahzia, figliuolo di Tikva, appoggiati da Meshullam e dal Levita Hubbetai, furono i soli ad opporsi a questo;
\par 16 ma quei della cattività fecero a quel modo; e furono scelti il sacerdote Esdra e alcuni capi famiglia secondo le loro case patriarcali, tutti designati per nome, i quali cominciarono a tener adunanza il primo giorno del decimo mese, per esaminare i fatti.
\par 17 Il primo giorno del primo mese aveano finito quanto concerneva tutti quelli che aveano sposato donne straniere.
\par 18 Tra i figliuoli de' sacerdoti questi si trovarono, che aveano sposato donne straniere: de' figliuoli di Jeshua, figliuolo di Jotsadak, e tra i suoi fratelli: Maaseia, Eliezer, Jarib e Ghedalia,
\par 19 i quali promisero, dando la mano, di mandar via le loro mogli, e offrirono un montone come sacrifizio per la loro colpa.
\par 20 Dei figliuoli d'Immer: Hanani e Zebadia.
\par 21 De' figliuoli di Harim: Maaseia, Elia, Scemaia, Jehiel ed Uzzia.
\par 22 Dei figliuoli di Pashur: Elioenai, Maaseia, Ishmael, Nethaneel, Jozabad, Elasa.
\par 23 Dei Leviti: Jozabad, Scimei, Kelaia, detto anche Kelita, Petahia, Giuda, ed Eliezer.
\par 24 De' cantori: Eliascib. De' portinai: Shallum, Telem e Uri.
\par 25 E degl'Israeliti: de' figliuoli di Parosh: Ramia, Izzia, Malkia, Mijamin, Eleazar, Malkia e Benaia.
\par 26 De' figliuoli di Elam: Mattania, Zaccaria, Jehiel, Abdi, Jeremoth ed Elia.
\par 27 De' figliuoli di Zattu: Elioenai, Eliascib, Mattania, Jeremoth, Zabad e Aziza.
\par 28 De' figliuoli di Bebai: Johanan, Hanania, Zabbai, Athlai.
\par 29 De' figliuoli di Bani: Meshullam, Malluc, Adaia, Jashub, Sceal, e Ramoth.
\par 30 De' figliuoli di Pahath-Moab: Adna, Kelal, Benaia, Maaseia, Mattania, Betsaleel, Binnui e Manasse.
\par 31 De' figliuoli di Harim: Eliezer, Isscia, Malkia, Scemaia, Simeone,
\par 32 Beniamino, Malluc, Scemaria.
\par 33 De' figliuoli di Hashum: Mattenai, Mattatta, Zabad, Elifelet, Jeremai, Manasse, Scimei.
\par 34 De' figliuoli di Bani: Maadai, Amram, Uel,
\par 35 Benaia, Bedia, Keluhu,
\par 36 Vania, Meremoth, Eliascib,
\par 37 Mattania, Mattenai, Jaasai,
\par 38 Bani, Binnui, Scimei,
\par 39 Scelemia, Nathan, Adaia,
\par 40 Macnadbai, Shashai, Sharai,
\par 41 Azarel, Scelemia, Scemaria,
\par 42 Shallum, Amaria, Giuseppe.
\par 43 De' figliuoli di Nebo: Jeiel, Mattithia, Zabad, Zebina, Jaddai, Joel, Benaia.
\par 44 Tutti questi avevan preso delle mogli straniere; e ve n'eran di quelli che da queste mogli avevano avuto de' figliuoli.


\end{document}