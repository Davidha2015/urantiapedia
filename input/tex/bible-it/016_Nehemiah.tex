\begin{document}

\title{Nehemiah}


\chapter{1}

\par 1 Parole di Nehemia, figliuolo di Hacalia. Or avvenne che nel mese di Kisleu dell'anno ventesimo, mentr'io mi trovavo nel castello di Susan,
\par 2 Hanani, uno de' miei fratelli, e alcuni altri uomini arrivarono da Giuda. Io li interrogai riguardo ai Giudei scampati, superstiti della cattività, e riguardo a Gerusalemme.
\par 3 E quelli mi dissero: 'I superstiti della cattività son là, nella provincia, in gran miseria e nell'obbrobrio; le mura di Gerusalemme restano rotte, e le sue porte, consumate dal fuoco'.
\par 4 Com'ebbi udite queste parole, io mi posi a sedere, piansi, feci cordoglio per parecchi giorni, e digiunai e pregai dinanzi all'Iddio del cielo.
\par 5 E dissi: 'O Eterno, Dio del cielo, Dio grande e tremendo; che mantieni il patto, e la misericordia con quei che t'amano e osservano i tuoi comandamenti,
\par 6 siano le tue orecchie attente, i tuoi occhi aperti, ed ascolta la preghiera del tuo servo, la quale io fo adesso dinanzi a te, giorno e notte, per i figliuoli d'Israele, tuoi servi, confessando i peccati de' figliuoli d'Israele: peccati, che noi abbiam commessi contro di te; sì, che io e la casa di mio padre abbiamo commessi!
\par 7 Noi ci siam condotti malvagiamente contro di te, e non abbiamo osservato i comandamenti, le leggi e le prescrizioni che tu desti a Mosè, tuo servo.
\par 8 Deh, ricordati della parola che ordinasti a Mosè, tuo servo, di pronunziare: - Se sarete infedeli, io vi disperderò fra i popoli;
\par 9 ma se tornerete a me e osserverete i miei comandamenti e li metterete in pratica, quand'anche i vostri dispersi fossero agli estremi confini del mondo, io di là li raccoglierò; e li ricondurrò al luogo che ho scelto per farne la dimora del mio nome. -
\par 10 Or questi sono tuoi servi, tuo popolo; tu li hai redenti con la tua gran potenza e con la tua forte mano.
\par 11 O Signore, te ne prego, siano le tue orecchie attente alla preghiera del tuo servo e alla preghiera de' tuoi servi, che hanno a cuore di temere il tuo nome; e concedi oggi, ti prego, buon successo al tuo servo, e fa' ch'ei trovi pietà agli occhi di quest'uomo'. Allora io ero coppiere del re.

\chapter{2}

\par 1 L'anno ventesimo del re Artaserse, nel mese di Nisan, come il vino stava dinanzi al re, io presi il vino e glielo porsi. Or io non ero mai stato triste in sua presenza.
\par 2 E il re mi disse: 'Perché hai l'aspetto triste? eppure non sei malato; non può esser altro che un'afflizione del cuore'. Allora io ebbi grandissima paura,
\par 3 e dissi al re: 'Viva il re in eterno! Come potrebbe il mio aspetto non esser triste quando la città dove sono i sepolcri de' miei padri è distrutta e le sue porte son consumate dal fuoco?'
\par 4 E il re mi disse: 'Che cosa domandi?' Allora io pregai l'Iddio del cielo;
\par 5 poi risposi al re: 'Se così piace al re e il tuo servo ha incontrato favore agli occhi tuoi, mandami in Giudea, nella città dove sono i sepolcri de' miei padri, perché io la riedifichi'.
\par 6 E il re, che avea la regina seduta allato, mi disse: 'Quanto durerà il tuo viaggio? e quando ritornerai?' La cosa piacque al re, ei mi lasciò andare, e io gli fissai un termine di tempo.
\par 7 Poi dissi al re: 'Se così piace al re, mi si diano delle lettere per i governatori d'oltre il fiume affinché mi lascino passare ed entrare in Giuda,
\par 8 e una lettera per Asaf, guardiano del parco del re, affinché mi dia del legname per costruire le porte del castello annesso alla casa dell'Eterno, per le mura della città, e per la casa che abiterò io'. E il re mi diede le lettere, perché la benefica mano del mio Dio era su me.
\par 9 Io giunsi presso i governatori d'oltre il fiume, e diedi loro le lettere del re. Il re avea mandati meco dei capi dell'esercito e dei cavalieri.
\par 10 E quando Samballat, lo Horonita, e Tobia, il servo Ammonita, furono informati del mio arrivo, ebbero gran dispiacere della venuta d'un uomo che procurava il bene de' figliuoli d'Israele.
\par 11 Così giunsi a Gerusalemme; e quando v'ebbi passato tre giorni,
\par 12 mi levai di notte, presi meco pochi uomini, e non dissi nulla ad alcuno di quello che Dio m'avea messo in cuore di fare per Gerusalemme; non avevo meco altro giumento che quello ch'io cavalcavo.
\par 13 Ed uscii di notte per la porta della Valle, e mi diressi verso la sorgente del Dragone e la porta del Letame, considerando le mura di Gerusalemme, com'erano rotte e come le sue porte erano consumate dal fuoco.
\par 14 Passai presso la porta della Sorgente e il serbatoio del Re, ma non v'era posto per cui il giumento ch'io cavalcavo potesse passare.
\par 15 Allora risalii di notte la valle, sempre considerando le mura; poi, rientrato per la porta della Valle, me ne tornai a casa.
\par 16 I magistrati non sapevano né dov'io fossi andato né che cosa facessi. Fino a quel momento, io non avevo detto nulla né ai Giudei né ai sacerdoti né ai notabili né ai magistrati né ad alcuno di quelli che si occupavano di lavori.
\par 17 Allora io dissi loro: 'Voi vedete la misera condizione nella quale ci troviamo; Gerusalemme è distrutta, e le sue porte son consumate dal fuoco! Venite, riedifichiamo le mura di Gerusalemme, e non sarem più nell'obbrobrio!'
\par 18 E narrai loro come la benefica mano del mio Dio era stata su me, senza omettere le parole che il re m'avea dette. E quelli dissero: 'Leviamoci, e mettiamoci a costruire!' E si fecero animo per metter mano alla buona impresa.
\par 19 Ma quando Samballat, lo Horonita, e Tobia, il servo Ammonita, e Ghescem, l'Arabo, seppero la cosa, si fecero beffe di noi, e ci sprezzarono dicendo: 'Che cosa state facendo? Vi volete forse ribellare contro al re?'
\par 20 Allora io risposi e dissi loro: 'L'Iddio del cielo è quegli che ci darà buon successo. Noi, suoi servi, ci leveremo e costruiremo; ma voi non avete né parte né diritto né ricordanza in Gerusalemme'.

\chapter{3}

\par 1 Eliascib, sommo sacerdote, si levò co' suoi fratelli sacerdoti e costruirono la porta delle Pecore; la consacrarono e vi misero le sue imposte; continuarono a costruire fino alla torre di Mea, che consacrarono, e fino alla Torre di Hananeel.
\par 2 Allato a Eliascib lavorarono gli uomini di Gerico, e allato a loro lavorò Zaccur, figliuolo d'Imri.
\par 3 I figliuoli di Senaa costruirono la porta de' Pesci, ne fecero l'intelaiatura, e vi posero le imposte, le serrature e le sbarre.
\par 4 Allato a loro lavorò alle riparazioni Meremoth, figliuolo d'Uria, figliuolo di Hakkots; allato a loro lavorò alle riparazioni Meshullam, figliuolo di Berekia, figliuolo di Mescezabeel; allato a loro lavorò alle riparazioni Tsadok, figliuolo di Baana;
\par 5 allato a loro lavorarono alle riparazioni i Tekoiti; ma i principali fra loro non piegarono i loro colli a lavorare all'opera del loro signore.
\par 6 Joiada, figliuolo di Paseah, e Meshullam, figliuolo di Besodeia, restaurarono la porta Vecchia; ne fecero l'intelaiatura, e vi posero le imposte, le serrature e le sbarre.
\par 7 Allato a loro lavorarono alle riparazioni Melatia, il Gabaonita, Jadon, il Meronothita, e gli uomini di Gabaon e di Mitspa, che dipendevano dalla sede del governatore d'oltre il fiume;
\par 8 allato a loro lavorò alle riparazioni Uzziel, figliuolo di Harhaia, di tra gli orefici, e allato a lui lavorò Hanania, di tra i profumieri. Essi lasciarono stare Gerusalemme com'era, fino al muro largo.
\par 9 Allato a loro lavorò alle riparazioni Refaia, figliuolo di Hur, capo della metà del distretto di Gerusalemme.
\par 10 Allato a loro lavorò alle riparazioni dirimpetto alla sua casa, Jedaia, figliuolo di Harumaf, e allato a lui lavorò Hattush, figliuolo di Hashabneia.
\par 11 Malkia, figliuolo di Harim e Hasshub, figliuolo di Pahath-Moab, restaurarono un'altra parte delle mura e la torre de' Forni.
\par 12 Allato a loro lavorò alle riparazioni, con le sue figliuole, Shallum, figliuolo di Hallohesh, capo della metà del distretto di Gerusalemme.
\par 13 Hanun e gli abitanti di Zanoah restaurarono la porta della Valle; la costruirono, vi posero le imposte, le serrature e le sbarre. Fecero inoltre mille cubiti di muro fino alla porta del Letame.
\par 14 Malkia, figliuolo di Recab, capo del distretto di Beth-Hakkerem restaurò la porta del Letame; la costruì, vi pose le imposte, le serrature, le sbarre.
\par 15 Shallum, figliuolo di Col-Hozeh, capo del distretto di Mitspa, restaurò la porta della Sorgente; la costruì, la coperse, vi pose le imposte, le serrature e le sbarre. Fece inoltre il muro del serbatoio di Scelah, presso il giardino del re, fino alla scalinata per cui si scende dalla città di Davide.
\par 16 Dopo di lui Neemia, figliuolo di Azbuk, capo della metà del distretto di Beth-Zur, lavorò alle riparazioni fin dirimpetto ai sepolcri di Davide, fino al serbatoio ch'era stato costruito, e fino alla casa de' prodi.
\par 17 Dopo di lui lavorarono alle riparazioni i Leviti, sotto Rehum, figliuolo di Bani; e allato a lui lavorò per il suo distretto Hashabia, capo della metà del distretto di Keila.
\par 18 Dopo di lui lavorarono alle riparazioni i loro fratelli, sotto Bavvai, figliuolo di Henadad, capo della metà del distretto di Keila;
\par 19 e allato a lui Ezer, figliuolo di Jeshua, capo di Mitspa, restaurò un'altra parte delle mura, dirimpetto alla salita dell'arsenale, all'angolo.
\par 20 Dopo di lui Baruc, figliuolo di Zaccai, ne restaurò con ardore un'altra parte, dall'angolo fino alla porta della casa di Eliascib, il sommo sacerdote.
\par 21 Dopo di lui Meremoth, figliuolo di Uria, figliuolo di Hakkoz, ne restaurò un'altra parte, dalla porta della casa di Eliascib fino all'estremità della casa di Eliascib.
\par 22 Dopo di lui lavorarono i sacerdoti che abitavano il contado.
\par 23 Dopo di loro Beniamino e Hashub lavorarono dirimpetto alla loro casa. Dopo di loro Azaria, figliuolo di Maaseia, figliuolo di Anania, lavorò presso la sua casa.
\par 24 Dopo di lui Binnui, figliuolo di Henadad, restaurò un'altra parte delle mura, dalla casa di Azaria fino allo svolto, e fino all'angolo.
\par 25 Palal, figliuolo d'Uzai, lavorò dirimpetto allo svolto e alla torre sporgente dalla casa superiore del re, che dà sul cortile della prigione. Dopo di lui lavorò Pedaia, figliuolo di Parosh.
\par 26 - I Nethinei che abitavano sulla collina, lavorarono fino dirimpetto alla porta delle Acque, verso oriente, e dirimpetto alla torre sporgente.
\par 27 Dopo di loro i Tekoiti ne restaurarono un'altra parte, dirimpetto alla gran torre sporgente e fino al muro della collina.
\par 28 I sacerdoti lavorarono alle riparazioni al disopra della porta de' Cavalli, ciascuno dirimpetto alla propria casa.
\par 29 Dopo di loro Tsadok, figliuolo d'Immer, lavorò dirimpetto alla sua casa. Dopo di lui lavorò Scemaia, figliuolo di Scecania, guardiano della porta orientale.
\par 30 Dopo di lui Hanania, figliuolo di Scelemia, e Hanun, sesto figliuolo di Tsalaf, restaurarono un'altra parte delle mura. Dopo di loro Meshullam, figliuolo di Berekia, lavorò difaccia alla sua camera.
\par 31 Dopo di lui Malkja, uno degli orefici, lavorò fino alle case de' Nethinei e de' mercanti, dirimpetto alla porta di Hammifkad e fino alla salita dell'angolo.
\par 32 E gli orefici e i mercanti lavorarono alle riparazioni fra la salita dell'angolo e la porta delle Pecore.

\chapter{4}

\par 1 Quando Samballat udì che noi edificavamo le mura, si adirò, s'indignò fuor di modo, si fe' beffe de' Giudei,
\par 2 e disse in presenza de' suoi fratelli e de' soldati di Samaria: 'Che fanno questi spossati Giudei? Si lasceranno fare? Offriranno sacrifizi? Finiranno in un giorno? Faranno essi rivivere delle pietre sepolte sotto mucchi di polvere e consumate dal fuoco?'
\par 3 Tobia l'Ammonita, che gli stava accanto, disse: 'Edifichino pure! Se una volpe vi salta su farà crollare il loro muro di pietra!'
\par 4 Ascolta, o Dio nostro, come siamo sprezzati! Fa' ricadere sul loro capo il loro vituperio, e abbandonali al saccheggio in un paese di schiavitù!
\par 5 E non coprire la loro iniquità, e non sia cancellato dal tuo cospetto il loro peccato; poiché t'hanno provocato ad ira in presenza dei costruttori.
\par 6 Noi dunque riedificammo le mura, che furon da pertutto compiute fino alla metà della loro altezza; e il popolo avea preso a cuore il lavoro.
\par 7 Ma quando Samballat, Tobia, gli Arabi, gli Ammoniti e gli Asdodei ebbero udito che la riparazione delle mura di Gerusalemme progrediva, e che le brecce cominciavano a chiudersi, n'ebbero grandissimo sdegno,
\par 8 e tutti quanti assieme congiurarono di venire ad attaccare Gerusalemme e a crearvi del disordine.
\par 9 Allora noi pregammo l'Iddio nostro e mettemmo contro di loro delle sentinelle di giorno e di notte per difenderci dai loro attacchi.
\par 10 Que' di Giuda dicevano: 'Le forze de' portatori di pesi vengon meno, e le macerie sono molte; noi non potremo costruir le mura!'
\par 11 E i nostri avversari dicevano: 'Essi non sapranno e non vedranno nulla, finché noi giungiamo in mezzo a loro; allora li uccideremo, e farem cessare i lavori'.
\par 12 E siccome i Giudei che dimoravano vicino a loro vennero dieci volte a riferirci la cosa da tutti i luoghi di loro provenienza,
\par 13 io, nelle parti più basse del posto, dietro le mura, in luoghi aperti, disposi il popolo per famiglie, con le loro spade, le loro lance, i loro archi.
\par 14 E, dopo aver tutto ben esaminato, mi levai, e dissi ai notabili, ai magistrati e al resto del popolo: 'Non li temete! Ricordatevi del Signore, grande e tremendo; e combattete per i vostri fratelli, per i vostri figliuoli e figliuole, per le vostre mogli e per le vostre case!'
\par 15 Quando i nostri nemici udirono ch'eravamo informati della cosa, Iddio frustrò il loro disegno, e noi tutti tornammo alle mura, ognuno al suo lavoro.
\par 16 Da quel giorno, la metà de' miei servi lavorava, e l'altra metà stava armata di lance, di scudi, d'archi, di corazze; e i capi eran dietro a tutta la casa di Giuda.
\par 17 Quelli che costruivan le mura e quelli che portavano o caricavano i pesi, con una mano lavoravano, e con l'altra tenevano la loro arma;
\par 18 e tutti i costruttori, lavorando, portavan ciascuno la spada cinta ai fianchi. Il trombettiere stava accanto a me.
\par 19 E io dissi ai notabili, ai magistrati e al resto del popolo: 'L'opera è grande ed estesa, e noi siamo sparsi sulle mura, e distanti l'uno dall'altro.
\par 20 Dovunque udrete il suon della tromba, quivi raccoglietevi presso di noi; l'Iddio nostro combatterà per noi'.
\par 21 Così continuavamo i lavori, mentre la metà della mia gente teneva impugnata la lancia, dallo spuntar dell'alba all'apparir delle stelle.
\par 22 In quel medesimo tempo, io dissi al popolo: 'Ciascuno di voi resti la notte dentro Gerusalemme coi suoi servi, per far con noi la guardia durante la notte e riprendere il lavoro di giorno'.
\par 23 Io poi, i miei fratelli, i miei servi e gli uomini di guardia che mi seguivano, non ci spogliavamo; ognuno avea l'arma a portata di mano.

\chapter{5}

\par 1 Or si levò un gran lamento da parte di que' del popolo e delle loro mogli contro ai Giudei, loro fratelli.
\par 2 Ve n'eran che dicevano: 'Noi, i nostri figliuoli e le nostre figliuole siamo numerosi; ci si dia del grano perché possiam mangiare e vivere!'
\par 3 Altri dicevano: 'Impegniamo i nostri campi, le nostre vigne e le nostre case per assicurarci del grano durante la carestia!'
\par 4 Altri ancora dicevano: 'Noi abbiam preso del danaro a imprestito sui nostri campi e sulle nostre vigne per pagare il tributo del re.
\par 5 Ora la nostra carne è come la carne de' nostri fratelli, i nostri figliuoli son come i loro figliuoli; ed ecco che dobbiam sottoporre i nostri figliuoli e le nostre figliuole alla schiavitù, e alcune delle nostre figliuole son già ridotte schiave; e noi non possiamo farci nulla, giacché i nostri campi e le nostre vigne sono in mano d'altri'.
\par 6 Quand'udii i loro lamenti e queste parole, io m'indignai forte.
\par 7 E, dopo matura riflessione, ripresi aspramente i notabili e i magistrati, e dissi loro: 'Come! voi prestate su pegno ai vostri fratelli?' E convocai contro di loro una grande raunanza,
\par 8 e dissi loro: 'Noi secondo la nostra possibilità, abbiamo riscattato i nostri fratelli Giudei che s'eran venduti ai pagani; e voi stessi vendereste i vostri fratelli, ed essi si venderebbero a noi!' Allora quelli si tacquero, e non seppero che rispondere.
\par 9 Io dissi pure: 'Quello che voi fate non è ben fatto. Non dovreste voi camminare nel timore del nostro Dio per non essere oltraggiati dai pagani nostri nemici?
\par 10 Anch'io e i miei fratelli e i miei servi abbiam dato loro in prestito danaro e grano. Vi prego, condoniamo loro questo debito.
\par 11 Rendete loro oggi i loro campi, le loro vigne, i loro uliveti e le loro case, e la centesima del danaro, del grano, del vino e dell'olio, che avete esatto da loro come interesse'.
\par 12 Quelli risposero: 'Restituiremo tutto, e non domanderemo più nulla da loro; faremo come tu dici'. Allora chiamai i sacerdoti, e in loro presenza li feci giurare che avrebbero mantenuta la promessa.
\par 13 Io scossi inoltre il mio mantello, e dissi: 'Così scuota Iddio dalla sua casa e dai suoi beni chiunque non avrà mantenuto questa promessa, e così sia egli scosso e resti senza nulla!' E tutta la raunanza disse: 'Amen!' E celebrarono l'Eterno. E il popolo mantenne la promessa.
\par 14 Di più, dal giorno che il re mi stabilì loro governatore nel paese di Giuda, dal ventesimo anno fino al trentaduesimo anno del re Artaserse, durante dodici anni, io e i miei fratelli non mangiammo della provvisione assegnata al governatore.
\par 15 I governatori che mi avean preceduto aveano gravato il popolo, ricevendone pane e vino, oltre a quaranta sicli d'argento; perfino i loro servi angariavano il popolo; ma io non ho fatto così, perché ho avuto timor di Dio.
\par 16 Anzi ho messo mano ai lavori di riparazione di queste mura, e non abbiamo comprato verun campo, e tutta la mia gente s'è raccolta là a lavorare.
\par 17 E avevo alla mia mensa centocinquanta uomini, Giudei e magistrati, oltre quelli che venivano a noi dalle nazioni circonvicine.
\par 18 E quel che mi si preparava per ogni giorno era un bue, sei capri scelti di bestiame minuto, e dell'uccellame; e ogni dieci giorni si preparava ogni sorta di vini in abbondanza; e, nondimeno, io non ho mai chiesta la provvisione assegnata al governatore, perché il popolo era già gravato abbastanza a motivo dei lavori.
\par 19 O mio Dio, ricordati, per farmi del bene, di tutto quello che ho fatto per questo popolo.

\chapter{6}

\par 1 Or quando Samballat e Tobia e Ghescem, l'Arabo, e gli altri nostri nemici ebbero udito che io avevo riedificate le mura e che non v'era più rimasta alcuna breccia - quantunque allora io non avessi ancora messe le imposte alle porte -
\par 2 Samballat e Ghescem mi mandarono a dire: 'Vieni, e troviamoci assieme in uno dei villaggi della valle di Ono'. Or essi pensavano a farmi del male.
\par 3 E io inviai loro dei messi per dire: 'Io sto facendo un gran lavoro, e non posso scendere. Perché il lavoro rimarrebb'egli sospeso mentr'io lo lascerei per scendere da voi?'
\par 4 Essi mandarono quattro volte a dirmi la stessa cosa, e io risposi loro nello stesso modo.
\par 5 Allora Samballat mi mandò a dire la stessa cosa la quinta volta per mezzo del suo servo che aveva in mano una lettera aperta,
\par 6 nella quale stava scritto: 'Corre voce fra queste genti, e Gashmu l'afferma, che tu e i Giudei meditate di ribellarvi; e che perciò tu ricostruisci le mura; e, stando a quel che si dice, tu diventeresti loro re,
\par 7 e avresti perfino stabiliti de' profeti per far la tua proclamazione a Gerusalemme, dicendo: - V'è un re in Giuda! - Or questi discorsi saranno riferiti al re. Vieni dunque, e consultiamoci assieme'.
\par 8 Ma io gli feci rispondere: 'Le cose non stanno come tu dici, ma sei tu che le inventi!'
\par 9 Perché tutta quella gente ci voleva impaurire e diceva: 'Le loro mani si rilasseranno e il lavoro non si farà più'. Ma tu, o Dio, fortifica ora le mie mani!
\par 10 Ed io andai a casa di Scemaia, figliuolo di Delaia, figliuolo di Mehetabeel, che s'era quivi rinchiuso; ed egli mi disse: 'Troviamoci assieme nella casa di Dio, dentro al tempio, e chiudiamo le porte del tempio; poiché coloro verranno ad ucciderti, e verranno a ucciderti di notte'.
\par 11 Ma io risposi: 'Un uomo come me si dà egli alla fuga? E un uomo qual son io potrebb'egli entrare nel tempio e vivere? No, io non v'entrerò'.
\par 12 E io compresi ch'ei non era mandato da Dio, ma avea pronunziata quella profezia contro di me, perché Tobia e Samballat l'aveano pagato.
\par 13 E l'aveano pagato per impaurirmi e indurmi ad agire a quel modo e a peccare, affin di aver materia da farmi una cattiva riputazione e da coprirmi d'onta.
\par 14 O mio Dio, ricordati di Tobia, di Samballat e di queste loro opere! Ricordati anche della profetessa Noadia e degli altri profeti che han cercato di spaventarmi!
\par 15 Or le mura furon condotte a fine il venticinquesimo giorno di Elul, in cinquantadue giorni.
\par 16 E quando tutti i nostri nemici l'ebber saputo, tutte le nazioni circonvicine furon prese da timore, e restarono grandemente avvilite ai loro propri occhi, perché riconobbero che quest'opera s'era compiuta con l'aiuto del nostro Dio.
\par 17 In quei giorni, anche dei notabili di Giuda mandavano frequenti lettere a Tobia, e ne ricevevano da Tobia,
\par 18 giacché molti in Giuda gli eran legati per giuramento, perch'egli era genero di Scecania figliuolo di Arah, e Johanan, suo figliuolo, avea sposata la figliuola di Meshullam, figliuolo di Berekia.
\par 19 Essi dicevan del bene di lui perfino in presenza mia, e gli riferivan le mie parole. E Tobia mandava lettere per impaurirmi.

\chapter{7}

\par 1 Or quando le mura furon riedificate ed io ebbi messo a posto le porte, e i portinai, i cantori e i Leviti furono stabiliti nei loro uffici,
\par 2 io detti il comando di Gerusalemme ad Hanani, mio fratello, e ad Hanania governatore del castello, perch'era un uomo fedele e timorato di Dio più di tanti altri.
\par 3 E dissi loro: 'Le porte di Gerusalemme non s'aprano finché il sole scotti; e mentre le guardie saranno ancora al loro posto, si chiudano e si sbarrino le porte; e si stabiliscano per far la guardia, gli abitanti di Gerusalemme, ciascuno al suo turno e ciascuno davanti alla propria casa'.
\par 4 Or la città era spaziosa e grande; ma dentro v'era poca gente, e non vi s'eran fabbricate case.
\par 5 E il mio Dio mi mise in cuore di radunare i notabili, i magistrati e il popolo, per farne il censimento. E trovai il registro genealogico di quelli ch'eran tornati dall'esilio la prima volta, e vi trovai scritto quanto segue:
\par 6 Questi son quei della provincia che tornarono dalla cattività; quelli che Nebucadnetsar, re di Babilonia, avea menati in cattività, e che tornarono a Gerusalemme e in Giuda, ciascuno nella sua città.
\par 7 Essi tornarono con Zorobabele, Jeshua, Nehemia, Azaria, Raamia, Nahamani, Mardocheo, Bilshan, Mispereth, Bigvai, Nehum e Baana. Censimento degli uomini del popolo d'Israele:
\par 8 Figliuoli di Parosh, duemilacentosettantadue.
\par 9 Figliuoli di Scefatia, trecentosettantadue.
\par 10 Figliuoli di Ara, seicentocinquantadue.
\par 11 Figliuoli di Pahath-Moab, dei figliuoli di Jeshua e di Joab, duemilaottocentodiciotto.
\par 12 Figliuoli di Elam, milleduecentocinquantaquattro.
\par 13 Figliuoli di Zattu, ottocentoquarantacinque.
\par 14 Figliuoli di Zaccai, settecentosessanta.
\par 15 Figliuoli di Binnui, seicentoquarantotto.
\par 16 Figliuoli di Bebai, seicentoventotto.
\par 17 Figliuoli di Azgad, duemilatrecentoventidue.
\par 18 Figliuoli di Adonikam, seicentosessantasette.
\par 19 Figliuoli di Bigvai, duemilasessantasette.
\par 20 Figliuoli di Adin, seicentocinquantacinque.
\par 21 Figliuoli di Ater, della famiglia d'Ezechia, novantotto.
\par 22 Figliuoli di Hashum, trecentoventotto.
\par 23 Figliuoli di Bezai, trecentoventiquattro.
\par 24 Figliuoli di Harif, centododici.
\par 25 Figliuoli di Gabaon, novantacinque.
\par 26 Uomini di Bethlehem e di Netofa, centottantotto.
\par 27 Uomini di Anathoth, centoventotto.
\par 28 Uomini di Beth-Azmaveth, quarantadue.
\par 29 Uomini di Kiriath-Jearim, di Kefira e di Beeroth, settecentoquarantatre.
\par 30 Uomini di Rama e di Gheba, seicentoventuno.
\par 31 Uomini di Micmas, centoventidue.
\par 32 Uomini di Bethel e d'Ai, centoventitre.
\par 33 Uomini d'un altro Nebo, cinquantadue.
\par 34 Figliuoli d'un altro Elam, milleduecentocinquantaquattro.
\par 35 Figliuoli di Harim, trecentoventi.
\par 36 Figliuoli di Gerico, trecentoquarantacinque.
\par 37 Figliuoli di Lod, di Hadid e d'Ono, settecentoventuno.
\par 38 Figliuoli di Senaa, tremilanovecentotrenta.
\par 39 Sacerdoti: figliuoli di Jedaia, della casa di Jeshua, novecentosessantatre.
\par 40 Figliuoli di Immer, millecinquantadue.
\par 41 Figliuoli di Pashur, milleduecentoquarantasette.
\par 42 Figliuoli di Harim, millediciassette.
\par 43 Leviti: figliuoli di Jeshua e di Kadmiel, de' figliuoli di Hodeva, settantaquattro.
\par 44 Cantori: figliuoli di Asaf, centoquarantotto.
\par 45 Portinai: figliuoli di Shallum, figliuoli di Ater, figliuoli di Talmon, figliuoli di Akkub, figliuoli di Hatita, figliuoli di Shobai, centotrentotto.
\par 46 Nethinei: figliuoli di Tsiha, figliuoli di Hasufa, figliuoli di Tabbaoth,
\par 47 figliuoli di Keros, figliuoli di Sia, figliuoli di Padon,
\par 48 figliuoli di Lebana, figliuoli di Hagaba, figliuoli di Salmai,
\par 49 figliuoli di Hanan, figliuoli di Ghiddel, figliuoli di Gahar,
\par 50 figliuoli di Reaia, figliuoli di Retsin, figliuoli di Nekoda,
\par 51 figliuoli di Gazzam, figliuoli di Uzza, figliuoli di Paseah,
\par 52 figliuoli di Besai, figliuoli di Meunim, figliuoli di Nefiscesim,
\par 53 figliuoli di Bakbuk, figliuoli di Hakufa, figliuoli di Harhur,
\par 54 figliuoli di Bazlith, figliuoli di Mehida, figliuoli di Harsha,
\par 55 figliuoli di Barkos, figliuoli di Sisera, figliuoli di Temah,
\par 56 figliuoli di Netsiah, figliuoli di Hatifa.
\par 57 Figliuoli dei servi di Salomone: figliuoli di Satai, figliuoli di Sofereth, figliuoli di Perida,
\par 58 figliuoli di Jala, figliuoli di Darkon, figliuoli di Ghiddel,
\par 59 figliuoli di Scefatia, figliuoli di Hattil, figliuoli di Pokereth-Hatsebaim, figliuoli di Amon.
\par 60 Totale dei Nethinei e de' figliuoli dei servi di Salomone, trecentonovantadue.
\par 61 Ed ecco quelli che tornarono da Tel-Melah, da Tel-Harsha, da Kerub-Addon e da Immer, e che non avean potuto stabilire la loro genealogia patriarcale per dimostrare ch'erano Israeliti:
\par 62 figliuoli di Delaia, figliuoli di Tobia, figliuoli di Nekoda, seicentoquarantadue.
\par 63 Di tra i sacerdoti: figliuoli di Habaia, figliuoli di Hakkots, figliuoli di Barzillai, il quale avea sposato una delle figliuole di Barzillai, il Galaadita, e fu chiamato col nome loro.
\par 64 Questi cercarono i loro titoli genealogici, ma non li trovarono, e furon quindi esclusi, come impuri, dal sacerdozio;
\par 65 e il governatore disse loro di non mangiare cose santissime finché non si presentasse un sacerdote per consultar Dio con l'Urim e il Thummim.
\par 66 La raunanza, tutt'assieme, noverava quarantaduemilatrecentosessanta persone,
\par 67 senza contare i loro servi e le loro serve, che ammontavano a settemilatrecentotrentasette. Avevan pure duecentoquarantacinque cantori e cantatrici.
\par 68 Avevano settecentotrentasei cavalli, duecentoquarantacinque muli,
\par 69 quattrocentotrentacinque cammelli, seimilasettecentoventi asini.
\par 70 Alcuni dei capi famiglia offriron dei doni per l'opera. Il governatore diede al tesoro mille dariche d'oro, cinquanta coppe, cinquecentotrenta vesti sacerdotali.
\par 71 E tra i capi famiglia ve ne furono che dettero al tesoro dell'opera ventimila dariche d'oro e duemiladuecento mine d'argento.
\par 72 Il resto del popolo dette ventimila dariche d'oro, duemila mine d'argento e sessantasette vesti sacerdotali.
\par 73 I sacerdoti, i Leviti, i portinai, i cantori, la gente del popolo, i Nethinei e tutti gl'Israeliti si stabilirono nelle loro città. Come fu giunto il settimo mese, e i figliuoli d'Israele si furono stabiliti nelle loro città,

\chapter{8}

\par 1 tutto il popolo si radunò come un sol uomo sulla piazza ch'è davanti alla porta delle Acque, e disse a Esdra, lo scriba, che portasse il libro della legge di Mosè che l'Eterno avea data a Israele.
\par 2 E il primo giorno del settimo mese, il sacerdote Esdra portò la legge davanti alla raunanza, composta d'uomini, di donne e di tutti quelli ch'eran capaci d'intendere.
\par 3 E lesse il libro sulla piazza ch'è davanti alla porta delle Acque, dalla mattina presto fino a mezzogiorno, in presenza degli uomini, delle donne, e di quelli ch'eran capaci d'intendere; e tutto il popolo teneva tese le orecchie a sentire il libro della legge.
\par 4 Esdra, lo scriba, stava sopra una tribuna di legno, ch'era stata fatta apposta, e accanto a lui stavano, a destra, Mattithia, Scema, Anaia, Uria, Hilkia e Maaseia; a sinistra, Pedaia, Mishael, Malkia, Hashum, Hashbaddana, Zaccaria e Meshullam.
\par 5 Esdra aprì il libro in presenza di tutto il popolo, poiché stava in luogo più eminente; e, com'ebbe aperto il libro, tutto il popolo s'alzò in piedi.
\par 6 Esdra benedisse l'Eterno, l'Iddio grande, e tutto il popolo rispose: 'Amen, amen', alzando le mani; e s'inchinarono, e si prostrarono con la faccia a terra dinanzi all'Eterno.
\par 7 Jeshua, Bani, Scerebia, Jamin, Akkub, Shabbethai, Hodia, Maaseia, Kelita, Azaria, Jozabad, Hanan, Pelaia e gli altri Leviti spiegavano la legge al popolo, e il popolo stava in piedi al suo posto.
\par 8 Essi leggevano nel libro della legge di Dio distintamente; e ne davano il senso, per far capire al popolo quel che s'andava leggendo.
\par 9 Nehemia, ch'era il governatore, Esdra, sacerdote e scriba, e i Leviti che ammaestravano il popolo, dissero a tutto il popolo: 'Questo giorno è consacrato all'Eterno, al vostro Dio; non fate cordoglio e non piangete!' Poiché tutto il popolo piangeva, ascoltando le parole della legge.
\par 10 Poi Nehemia disse loro: 'Andate, mangiate vivande grasse e bevete vini dolci, e mandate delle porzioni a quelli che nulla hanno di preparato per loro; perché questo giorno è consacrato al Signor nostro; non v'attristate; perché il gaudio dell'Eterno è la vostra forza'.
\par 11 I Leviti facevano far silenzio a tutto il popolo, dicendo: 'Tacete, perché questo giorno è santo; non v'attristate!'
\par 12 E tutto il popolo se n'andò a mangiare, a bere, a mandar porzioni ai poveri, e a far gran festa, perché aveano intese le parole ch'erano state loro spiegate.
\par 13 Il secondo giorno, i capi famiglia di tutto il popolo, i sacerdoti e i Leviti si radunarono presso Esdra, lo scriba, per esaminare le parole della legge.
\par 14 E trovarono scritto nella legge che l'Eterno avea data per mezzo di Mosè, che i figliuoli d'Israele doveano dimorare in capanne durante la festa del settimo mese,
\par 15 e che in tutte le loro città e in Gerusalemme si dovea pubblicar questo bando: 'Andate al monte, e portatene rami d'ulivo, rami d'ulivastro, rami di mirto, rami di palma e rami d'alberi ombrosi, per fare delle capanne, come sta scritto'.
\par 16 Allora il popolo andò fuori, portò i rami, e si fecero ciascuno la sua capanna sul tetto della propria casa, nei loro cortili, nei cortili della casa di Dio, sulla piazza della porta delle Acque, e sulla piazza della porta d'Efraim.
\par 17 Così tutta la raunanza di quelli ch'eran tornati dalla cattività si fece delle capanne, e dimorò nelle capanne. Dal tempo di Giosuè, figliuolo di Nun, fino a quel giorno, i figliuoli d'Israele non avean più fatto nulla di simile. E vi fu grandissima allegrezza.
\par 18 Ed Esdra fece la lettura del libro della legge di Dio ogni giorno, dal primo all'ultimo; la festa si celebrò durante sette giorni, e l'ottavo vi fu solenne raunanza, com'è ordinato.

\chapter{9}

\par 1 Or il ventiquattresimo giorno dello stesso mese, i figliuoli d'Israele si radunarono, vestiti di sacco e coperti di terra, per celebrare un digiuno.
\par 2 Quelli che appartenevano alla progenie d'Israele si separarono da tutti gli stranieri, si presentarono dinanzi a Dio, e confessarono i loro peccati e le iniquità dei loro padri.
\par 3 S'alzarono in piè nel posto dove si trovavano, e fu fatta la lettura del libro della legge dell'Eterno, del loro Dio, per un quarto del giorno; e per un altro quarto essi fecero la confessione de' peccati, e si prostrarono davanti all'Eterno, al loro Dio.
\par 4 Jeshua, Bani, Kadmiel, Scebania, Bunni, Scerebia, Bani e Kenani salirono sulla tribuna dei Leviti e gridarono ad alta voce all'Eterno, al loro Dio.
\par 5 E i Leviti Jeshua, Kadmiel, Bani, Hashabneia, Scerebia, Hodia, Scebania e Pethahia dissero: 'Levatevi e benedite l'Eterno, il vostro Dio, d'eternità in eternità! Si benedica il nome tuo glorioso, ch'è esaltato al disopra d'ogni benedizione e d'ogni lode!
\par 6 Tu, tu solo sei l'Eterno! tu hai fatto i cieli, i cieli de' cieli e tutto il loro esercito, la terra e tutto ciò che sta sovr'essa, i mari e tutto ciò ch'è in essi, e tu fai vivere tutte queste cose, e l'esercito de' cieli t'adora.
\par 7 Tu sei l'Eterno, l'Iddio che scegliesti Abramo, lo traesti fuori da Ur de' Caldei, e gli desti il nome d'Abrahamo;
\par 8 tu trovasti il cuor suo fedele davanti a te, e fermasti con lui un patto, promettendogli di dare alla sua progenie il paese de' Cananei, degli Hittei, degli Amorei, de' Ferezei, de' Gebusei e de' Ghirgasei; tu hai mantenuta la tua parola, perché sei giusto.
\par 9 Tu vedesti l'afflizione de' nostri padri in Egitto e udisti il loro grido presso il mar Rosso;
\par 10 e operasti miracoli e prodigi contro Faraone, contro tutti i suoi servi, contro tutto il popolo del suo paese, perché sapevi ch'essi aveano trattato i nostri padri con prepotenza; e ti facesti un nome com'è quello che hai al dì d'oggi.
\par 11 E fendesti il mare davanti a loro, sì che passarono per mezzo al mare sull'asciutto; e quelli che l'inseguivano tu li precipitasti nell'abisso, come una pietra in fondo ad acque potenti.
\par 12 E li conducesti di giorno con una colonna di nuvola, e di notte con una colonna di fuoco per rischiarar loro la via per la quale dovean camminare.
\par 13 E scendesti sul monte Sinai e parlasti con loro dal cielo e desti loro prescrizioni giuste e leggi di verità, buoni precetti e buoni comandamenti:
\par 14 e facesti loro conoscere il tuo santo sabato, e desti loro comandamenti, precetti e una legge per mezzo di Mosè, tuo servo;
\par 15 e desti loro pane dal cielo quand'erano affamati, e facesti scaturire acqua dalla rupe quand'erano assetati e dicesti loro che andassero a prender possesso del paese che avevi giurato di dar loro.
\par 16 Ma essi, i nostri padri, si condussero con superbia, indurarono le loro cervici, e non ubbidirono ai tuoi comandamenti;
\par 17 rifiutarono d'ubbidire, e non si ricordarono delle maraviglie che tu avevi fatte a pro loro; e indurarono le loro cervici; e, nella loro ribellione, si vollero dare un capo per tornare alla loro schiavitù. Ma tu sei un Dio pronto a perdonare, misericordioso, pieno di compassione, lento all'ira e di gran benignità, e non li abbandonasti.
\par 18 Neppure quando si fecero un vitello di getto e dissero: - Ecco il tuo Dio che t'ha tratto fuori dall'Egitto! e t'oltraggiarono gravemente,
\par 19 tu nella tua immensa misericordia, non li abbandonasti nel deserto: la colonna di nuvola che stava su loro non cessò di guidarli durante il giorno per il loro cammino, e la colonna di fuoco non cessò di rischiarar loro la via per la quale doveano camminare.
\par 20 E desti loro il tuo buono spirito per istruirli, e non rifiutasti la tua manna alle loro bocche, e desti loro dell'acqua quand'erano assetati.
\par 21 Per quarant'anni li sostentasti nel deserto, e non mancò loro nulla; le loro vesti non si logorarono e i loro piedi non si gonfiarono.
\par 22 E desti loro regni e popoli, e li spartisti fra loro per contrade; ed essi possedettero il paese di Sihon, cioè il paese del re di Heshbon, e il paese di Og re di Bashan.
\par 23 E moltiplicasti i loro figliuoli come le stelle del cielo, e li introducesti nel paese in cui avevi detto ai padri loro che li faresti entrare per possederlo.
\par 24 E i loro figliuoli v'entrarono e presero possesso del paese; tu umiliasti dinanzi a loro i Cananei che abitavano il paese, e li desti nelle loro mani coi loro re e coi popoli del paese, perché li trattassero come loro piaceva.
\par 25 Ed essi s'impadronirono di città fortificate e d'una terra fertile, e possedettero case piene d'ogni bene, cisterne bell'e scavate, vigne, uliveti, alberi fruttiferi in abbondanza, e mangiarono e si saziarono e ingrassarono e vissero in delizie, per la tua gran bontà.
\par 26 Ma essi furon disubbidienti, si ribellarono contro di te, si gettaron la tua legge dietro le spalle, uccisero i tuoi profeti che li scongiuravano di tornare a te, e t'oltraggiarono gravemente.
\par 27 Perciò tu li desti nelle mani de' loro nemici, che li oppressero; ma al tempo della loro distretta essi gridarono a te, e tu li esaudisti dal cielo; e, nella tua immensa misericordia, tu desti loro de' liberatori, che li salvarono dalle mani dei loro nemici.
\par 28 Ma quando aveano riposo, ricominciavano a fare il male dinanzi a te; perciò tu li abbandonavi nelle mani dei loro nemici, i quali diventavan loro dominatori; poi, quando ricominciavano a gridare a te, tu li esaudivi dal cielo; e così, nella tua misericordia, più volte li salvasti.
\par 29 Tu li scongiuravi per farli tornare alla tua legge; ma essi s'inorgoglivano e non ubbidivano ai tuoi comandamenti, peccavano contro le tue prescrizioni che fanno vivere chi le mette in pratica; la loro spalla rifiutava il giogo, essi induravano le loro cervici e non voleano ubbidire.
\par 30 E pazientasti con essi molti anni, e li scongiurasti per mezzo del tuo spirito e per bocca dei tuoi profeti; ma essi non vollero prestare orecchio, e tu li desti nelle mani de' popoli de' paesi stranieri.
\par 31 Però, nella tua immensa compassione, tu non li sterminasti del tutto, e non li abbandonasti, perché sei un Dio clemente e misericordioso.
\par 32 Ora dunque, o Dio nostro, Dio grande, potente e tremendo, che mantieni il patto e la misericordia, non paian poca cosa agli occhi tuoi tutte queste afflizioni che son piombate addosso a noi, ai nostri re, ai nostri capi, ai nostri sacerdoti, ai nostri profeti, ai nostri padri, a tutto il tuo popolo, dal tempo dei re d'Assiria al dì d'oggi.
\par 33 Tu sei stato giusto in tutto quello che ci è avvenuto, poiché tu hai agito fedelmente, mentre noi ci siam condotti empiamente.
\par 34 I nostri re, i nostri capi, i nostri sacerdoti, i nostri padri non hanno messa in pratica la tua legge e non hanno ubbidito né ai comandamenti né agli ammonimenti coi quali tu li scongiuravi.
\par 35 Ed essi, mentre godevano del loro regno, dei grandi benefizi che tu largivi loro e del vasto e fertile paese che tu avevi messo a loro disposizione, non ti servirono e non abbandonarono le loro opere malvage.
\par 36 E oggi eccoci schiavi! eccoci schiavi nel paese che tu desti ai nostri padri, perché ne mangiassero i frutti e ne godessero i beni.
\par 37 Ed esso moltiplica i suoi prodotti per i re ai quali tu ci hai sottoposti a cagion dei nostri peccati, e che son padroni dei nostri corpi e del nostro bestiame a loro talento; e noi siamo in gran distretta'.
\par 38 A motivo di tutto questo, noi fermammo un patto stabile e lo mettemmo per iscritto; e i nostri capi, i nostri Leviti e i nostri sacerdoti vi apposero il loro sigillo.

\chapter{10}

\par 1 Quelli che v'apposero il loro sigillo furono i seguenti: Nehemia, il governatore, figliuolo di Hacalia, e Sedecia,
\par 2 Seraia, Azaria, Geremia,
\par 3 Pashur, Amaria, Malkija,
\par 4 Hattush, Scebania, Malluc,
\par 5 Harim, Meremoth, Obadia,
\par 6 Daniele, Ghinnethon, Baruc,
\par 7 Meshullam, Abija, Mijamin,
\par 8 Maazia, Bilgai, Scemaia. Questi erano sacerdoti.
\par 9 Leviti: Jeshua, figliuolo di Azania, Binnui de' figliuoli di Henadad, Kadmiel,
\par 10 e i loro fratelli Scebania, Hodia,
\par 11 Kelita, Pelaia, Hanan, Mica,
\par 12 Rehob, Hashabia, Zaccur, Scerebia,
\par 13 Scebania, Hodia, Bani, Beninu.
\par 14 Capi del popolo: Parosh, Pahath-Moab, Elam, Zattu, Bani,
\par 15 Bunni, Azgad,
\par 16 Bebai, Adonia, Bigvai, Adin,
\par 17 Ater, Ezechia, Azzur,
\par 18 Hodia, Hashum,
\par 19 Betsai, Harif, Anatoth,
\par 20 Nebai, Magpiash, Meshullam,
\par 21 Hezir, Mescezabeel, Tsadok,
\par 22 Jaddua, Pelatia, Hanan, Anaia,
\par 23 Hosea, Hanania, Hasshub,
\par 24 Hallohesh, Pilha, Shobek,
\par 25 Rehum, Hashabna, Maaseia,
\par 26 Ahiah, Hanan, Anan,
\par 27 Malluc, Harim, Baana.
\par 28 Il resto del popolo, i sacerdoti, i Leviti, i portinai, i cantori, i Nethinei e tutti quelli che s'eran separati dai popoli dei paesi stranieri per aderire alla legge di Dio, le loro mogli, i loro figliuoli e le loro figliuole, tutti quelli che aveano conoscimento e intelligenza,
\par 29 s'unirono ai loro fratelli più ragguardevoli tra loro, e s'impegnarono con esecrazione e giuramento a camminare nella legge di Dio data per mezzo di Mosè servo di Dio, ad osservare e mettere in pratica tutti i comandamenti dell'Eterno, del Signor nostro, le sue prescrizioni e le sue leggi,
\par 30 a non dare le nostre figliuole ai popoli del paese e a non prendere le figliuole loro per i nostri figliuoli,
\par 31 a non comprar nulla in giorno di sabato o in altro giorno sacro, dai popoli che portassero a vendere in giorno di sabato qualsivoglia sorta di merci o di derrate, a lasciare in riposo la terra ogni settimo anno, e a non esigere il pagamento di verun debito.
\par 32 C'imponemmo pure per legge di dare ogni anno il terzo d'un siclo per il servizio della casa del nostro Dio,
\par 33 per i pani della presentazione, per l'oblazione perpetua, per l'olocausto perpetuo dei sabati, dei noviluni, delle feste, per le cose consacrate, per i sacrifizi d'espiazione a pro d'Israele, e per tutta l'opera della casa del nostro Dio;
\par 34 e tirando a sorte, noi sacerdoti, Leviti e popolo, regolammo quel che concerne l'offerta delle legna, affin di portarle, secondo le nostre case patriarcali, alla casa del nostro Dio, a tempi fissi, anno per anno, perché bruciassero sull'altare dell'Eterno, del nostro Dio, come sta scritto nella legge;
\par 35 e c'impegnammo a portare ogni anno nella casa dell'Eterno le primizie del nostro suolo e le primizie d'ogni frutto di qualunque albero,
\par 36 come anche i primogeniti de' nostri figliuoli e del nostro bestiame conforme sta scritto nella legge, e i primogeniti delle nostre mandre e de' nostri greggi per presentarli nella casa del nostro Dio ai sacerdoti che fanno il servizio nella casa del nostro Dio.
\par 37 E c'impegnammo pure di portare ai sacerdoti nelle camere della casa del nostro Dio, le primizie della nostra pasta, le nostre offerte prelevate, le primizie de' frutti di qualunque albero, del vino e dell'olio, e di dare la decima delle rendite del nostro suolo ai Leviti, i quali debbon prendere essi stessi queste decime in tutti i luoghi da noi coltivati.
\par 38 E un sacerdote, figliuolo d'Aaronne, sarà coi Leviti quando preleveranno le decime; e i Leviti porteranno la decima della decima alla casa del nostro Dio nelle stanze che servono di magazzino,
\par 39 poiché in quelle stanze i figliuoli d'Israele e i figliuoli di Levi debbon portare l'offerta prelevata sul frumento, sul vino e sull'olio; quivi sono gli utensili del santuario, i sacerdoti che fanno il servizio, i portinai e i cantori. Noi c'impegnammo così a non abbandonare la casa del nostro Dio.

\chapter{11}

\par 1 I capi del popolo si stabilirono a Gerusalemme; il resto del popolo tirò a sorte per farne venire uno su dieci ad abitar Gerusalemme, la città santa; gli altri nove doveano rimanere nelle altre città.
\par 2 E il popolo benedisse tutti quelli che s'offrirono volenterosamente d'abitare in Gerusalemme.
\par 3 Ecco i capi della provincia che si stabilirono a Gerusalemme, mentre che, nelle città di Giuda, ognuno si stabilì nella sua proprietà, nella sua città: Israeliti, sacerdoti, Leviti, Nethinei, e figliuoli dei servi di Salomone.
\par 4 A Gerusalemme dunque si stabilirono de' figliuoli di Giuda, e de' figliuoli di Beniamino. - Dei figliuoli di Giuda: Atahia, figliuolo d'Uzzia, figliuolo di Zaccaria, figliuolo d'Amaria, figliuolo di Scefatia, figliuolo di Mahalaleel, de' figliuoli di Perets,
\par 5 e Maaseia, figliuolo di Baruc, figliuolo di Col-Hozeh, figliuolo di Hazaia, figliuolo di Adaia, figliuolo di Joiarib, figliuolo di Zaccaria, figliuolo dello Scilonita.
\par 6 Totale dei figliuoli di Perets che si stabilirono a Gerusalemme: quattrocentosessantotto uomini valorosi.
\par 7 De' figliuoli di Beniamino, questi: Sallu, figliuolo di Mashullam, figliuolo di Joed, figliuolo di Pedaia, figliuolo di Kolaia, figliuolo di Maaseia, figliuolo d'Ithiel, figliuolo d'Isaia;
\par 8 e, dopo lui, Gabbai, Sallai: in tutto, novecentoventotto.
\par 9 Gioele, figliuolo di Zicri, era loro capo, e Giuda, figliuolo di Hassenua, era il secondo capo della città.
\par 10 Dei sacerdoti: Jedaia, figliuolo di Joiarib, Jakin,
\par 11 Seraia, figliuolo di Hilkia, figliuolo di Meshullam, figliuolo di Tsadok, figliuolo di Meraioth, figliuolo di Ahitub, preposto alla casa di Dio,
\par 12 e i loro fratelli addetti all'opera della casa, in numero di ottocentoventidue; e Adaia, figliuolo di Jeroham, figliuolo di Pelalia, figliuolo di Amtsi, figliuolo di Zaccaria, figliuolo di Pashur, figliuolo di Malkija,
\par 13 e i suoi fratelli, capi delle case patriarcali, in numero di duecentoquarantadue; e Amashsai, figliuolo d'Azareel, figliuolo d'Ahzai, figliuolo di Meshillemoth, figliuolo d'Immer,
\par 14 e i loro fratelli, uomini valorosi, in numero di centoventotto. Zabdiel, figliuolo di Ghedolim, era loro capo.
\par 15 Dei Leviti: Scemaia, figliuolo di Hashub, figliuolo di Azricam, figliuolo di Hashabia, figliuolo di Bunni,
\par 16 Shabhethai e Jozabad, preposti al servizio esterno della casa di Dio, di fra i capi dei Leviti;
\par 17 e Mattania, figliuolo di Mica, figliuolo di Zabdi, figliuolo d'Asaf, il capo cantore che intonava le laudi al momento della preghiera, e Bakbukia che gli veniva secondo tra i suoi fratelli, e Abda figliuolo di Shammua, figliuolo di Galal, figliuolo di Jeduthun.
\par 18 Totale de' Leviti nella città santa: duecentottantaquattro.
\par 19 I portinai: Akkub, Talmon e i loro fratelli, custodi delle porte, centosettantadue.
\par 20 Il resto d'Israele, i sacerdoti, i Leviti, si stabilirono in tutte le città di Giuda, ciascuno nella sua proprietà.
\par 21 I Nethinei si stabilirono sulla collina, e Tsiha e Ghishpa erano a capo dei Nethinei.
\par 22 Il capo dei Leviti a Gerusalemme era Uzzi, figliuolo di Bani, figliuolo di Hashabia, figliuolo di Mattania, figliuolo di Mica, de' figliuoli d'Asaf, ch'erano i cantori addetti al servizio della casa di Dio;
\par 23 poiché v'era un ordine del re che concerneva i cantori, e v'era una provvisione assicurata loro giorno per giorno.
\par 24 E Pethahia, figliuolo di Mescezabeel, de' figliuoli di Zerach, figliuolo di Giuda, era commissario del re per tutti gli affari del popolo.
\par 25 Quanto ai villaggi con le loro campagne, alcuni de' figliuoli di Giuda si stabilirono in Kiriath-Arba e ne' luoghi che ne dipendevano, in Dibon e nei luoghi che ne dipendevano, in Jekabtseel e ne' villaggi che ne dipendevano,
\par 26 in Jeshua, in Molada, in Beth-Paleth,
\par 27 in Atsar-Shual, in Beer-Sceba e ne' luoghi che ne dipendevano,
\par 28 in Tsiklag, in Mecona e ne' luoghi che ne dipendevano,
\par 29 in En-Rimmon, in Tsora,
\par 30 in Jarmuth, in Zanoah, in Adullam e nei loro villaggi, in Lakis e nelle sue campagne, in Azeka e ne' luoghi che ne dipendevano. Si stabilirono da Beer-Sceba fino alla valle di Hinnom.
\par 31 I figliuoli di Beniamino si stabilirono da Gheba in là, a Micmas, ad Aijah, a Bethel e ne' luoghi che ne dipendevano,
\par 32 ad Anathoth, a Nob, ad Anania,
\par 33 a Atsor, a Rama, a Ghittaim,
\par 34 a Hadid, a Tseboim, a Neballath,
\par 35 a Lod ed a Ono, valle degli artigiani.
\par 36 Dei Leviti alcune classi appartenenti a Giuda furono unite a Beniamino.

\chapter{12}

\par 1 Questi sono i sacerdoti e i Leviti che tornarono con Zorobabel, figliuolo di Scealthiel, e con Jeshua: Seraia, Geremia,
\par 2 Esdra, Amaria, Malluc,
\par 3 Hattush, Scecania, Rehum,
\par 4 Meremoth, Iddo, Ghinnethoi,
\par 5 Abija, Mijamin, Maadia,
\par 6 Bilga, Scemaia, Joiarib,
\par 7 Jedaia, Sallu, Amok, Hilkia, Jedaia. Questi erano i capi de' sacerdoti e dei loro fratelli al tempo di Jeshua.
\par 8 Leviti: Jeshua, Binnui, Kadmiel, Scerebia, Giuda, Mattania, che dirigeva coi suoi fratelli il canto delle laudi.
\par 9 Bakbukia e Unni, loro fratelli, s'alternavan con loro secondo il loro turno.
\par 10 Jeshua generò Joiakim; Joiakim generò Eliascib; Eliascib generò Joiada;
\par 11 Joiada generò Jonathan; Jonathan generò Jaddua.
\par 12 Ecco quali erano, al tempo di Joiakim, i capi di famiglie sacerdotali: della famiglia di Seraia, Meraia; di quella di Geremia, Hanania;
\par 13 di quella d'Esdra, Meshullam; di quella d'Amaria, Johanan;
\par 14 di quella di Melicu, Jonathan; di quella di Scebania, Giuseppe;
\par 15 di quella di Harim, Adna; di quella di Meraioth, Helkai;
\par 16 di quella d'Iddo, Zaccaria; di quella di Ghinnethon, Meshullam;
\par 17 di quella d'Abija, Zicri; di quella di Miniamin...; di quella di Moadia, Piltai;
\par 18 di quella di Bilga, Shammua; di quella di Scemaia, Jonathan;
\par 19 di quella di Joiarib, Mattenai; di quella di Jedaia, Uzzi;
\par 20 di quella di Sallai, Kallai; di quella di Amok, Eber;
\par 21 di quella di Hilkia, Hashabia; di quella di Jedaia, Nethaneel.
\par 22 Quanto ai Leviti, i capi famiglia furono iscritti al tempo di Eliascib, di Joiada, di Johanan e di Jaddua; e i sacerdoti, sotto il regno di Dario, il Persiano.
\par 23 I capi delle famiglie levitiche furono iscritti nel libro delle Cronache fino al tempo di Johanan, figliuolo di Eliascib.
\par 24 I capi dei Leviti Hashabia, Scerebia, Jeshua, figliuolo di Kadmiel, e i loro fratelli s'alternavano con essi per lodare e celebrare l'Eterno, conforme all'ordine di Davide, uomo di Dio, per mute, secondo il loro turno.
\par 25 Mattania, Bakbukia, Obadia, Meshullam, Talmon, Akkub erano portinai, e facevan la guardia ai magazzini delle porte.
\par 26 Questi vivevano al tempo di Joiakim, figliuolo di Jeshua, figliuolo di Jotsadak e al tempo di Nehemia, il governatore, e di Esdra, sacerdote e scriba.
\par 27 Alla dedicazione delle mura di Gerusalemme si mandarono a cercare i Leviti di tutti i luoghi dov'erano, per farli venire a Gerusalemme affin di fare la dedicazione con gioia, con laudi e cantici e suon di cembali, saltèri e cetre.
\par 28 E i figliuoli dei cantori si radunarono dal distretto intorno a Gerusalemme, dai villaggi dei Netofathiti,
\par 29 da Beth-Ghilgal e dal territorio di Gheba e d'Azmaveth; poiché i cantori s'erano edificati dei villaggi ne' dintorni di Gerusalemme.
\par 30 I sacerdoti e i Leviti si purificarono e purificarono il popolo, le porte e le mura.
\par 31 Poi io feci salire sulle mura i capi di Giuda, e formai due grandi cori coi relativi cortei. Il primo s'incamminò dal lato destro, sulle mura, verso la porta del Letame;
\par 32 e dietro questo coro camminavano Hoshaia, la metà dei capi di Giuda,
\par 33 Azaria, Esdra, Meshullam, Giuda,
\par 34 Beniamino, Scemaia, Geremia,
\par 35 dei figliuoli di sacerdoti con le trombe, Zaccaria, figliuolo di Jonathan, figliuolo di Scemaia, figliuolo di Mattania, figliuolo di Micaia, figliuolo di Zaccur, figliuolo d'Asaf,
\par 36 e i suoi fratelli Scemaia, Azareel, Milalai, Ghilalai, Maai, Nethaneel, Giuda, Hanani, con gli strumenti musicali di Davide, uomo di Dio. Esdra, lo scriba, camminava alla loro testa.
\par 37 Giunti che furono alla porta della Sorgente, montarono, dirimpetto a loro, la scalinata della città di Davide, là dove le mura salgono al di sopra del livello della casa di Davide, e giunsero alla porta delle Acque, a oriente.
\par 38 Il secondo coro s'incamminò nel senso opposto; e io gli andavo dietro, con l'altra metà del popolo, sopra le mura. Passando al disopra della torre de' Forni, esso andò fino alle mura larghe;
\par 39 poi al disopra della porta d'Efraim, della porta Vecchia, della porta dei Pesci, della torre di Hananeel, della torre di Mea, fino alla porta delle Pecore; e il coro si fermò alla porta della Prigione.
\par 40 I due cori si fermarono nella casa di Dio; e così feci io, con la metà de' magistrati ch'era meco,
\par 41 e i sacerdoti Eliakim, Maaseia, Miniamin, Micaia, Elioenai, Zaccaria, Hanania con le trombe,
\par 42 e Maaseia, Scemaia, Eleazar, Uzzi, Johanan, Malkija, Elam, Ezer. E i cantori fecero risonar forte le loro voci, diretti da Izrahia.
\par 43 In quel giorno il popolo offrì numerosi sacrifizi, e si rallegrò perché Iddio gli avea concesso una gran gioia. Anche le donne e i fanciulli si rallegrarono; e la gioia di Gerusalemme si sentiva di lontano.
\par 44 In quel tempo, degli uomini furon preposti alle stanze che servivan da magazzini delle offerte, delle primizie e delle decime, onde vi raccogliessero dai contadi delle città le parti assegnate dalla legge ai sacerdoti e ai Leviti; poiché i Giudei gioivano a vedere i sacerdoti ed i Leviti ai loro posti;
\par 45 e questi osservavano ciò che si riferiva al servizio del loro Dio e alle purificazioni; come facevano, dal canto loro, i cantori e i portinai conforme all'ordine di Davide e di Salomone suo figliuolo.
\par 46 Poiché, anticamente, al tempo di Davide e di Asaf v'erano de' capi de' cantori e de' canti di laude e di azioni di grazie a Dio.
\par 47 Tutto Israele, al tempo di Zorobabele e di Nehemia, dava giorno per giorno le porzioni assegnate ai cantori ed ai portinai; dava ai Leviti le cose consacrate, e i Leviti davano ai figliuoli d'Aaronne le cose consacrate che loro spettavano.

\chapter{13}

\par 1 In quel tempo si lesse in presenza del popolo il libro di Mosè, e vi si trovò scritto che l'Ammonita e il Moabita non debbono mai in perpetuo entrare nella raunanza di Dio,
\par 2 perché non eran venuti incontro ai figliuoli d'Israele con del pane e dell'acqua, e perché aveano prezzolato a loro danno Balaam, per maledirli; ma il nostro Dio convertì la maledizione in benedizione.
\par 3 E quando il popolo ebbe udita la legge, separò da Israele ogni elemento straniero.
\par 4 Or prima di questo, il sacerdote Eliascib, ch'era preposto alle camere della casa del nostro Dio ed era parente di Tobia,
\par 5 avea messo a disposizione di quest'ultimo una camera grande là dove, prima d'allora, si riponevano le offerte, l'incenso, gli utensili, la decima del grano, del vino e dell'olio, tutto ciò che spettava per legge ai Leviti, ai cantori, ai portinai, e la parte che se ne prelevava per i sacerdoti.
\par 6 Ma quando si faceva tutto questo, io non ero a Gerusalemme; perché l'anno trentaduesimo di Artaserse, re di Babilonia, ero tornato presso il re; e in capo a qualche tempo avendo ottenuto un congedo dal re,
\par 7 tornai a Gerusalemme, e m'accorsi del male che Eliascib avea fatto per amor di Tobia, mettendo a sua disposizione una camera nei cortili della casa di Dio.
\par 8 La cosa mi dispiacque fortemente, e feci gettare fuori dalla camera tutte le masserizie appartenenti a Tobia;
\par 9 poi ordinai che si purificassero quelle camere, e vi feci ricollocare gli utensili della casa di Dio, le offerte e l'incenso.
\par 10 Seppi pure che le porzioni dovute ai Leviti non erano state date, e che i Leviti e i cantori, incaricati del servizio, se n'eran fuggiti, ciascuno alla sua terra.
\par 11 E io censurai i magistrati, e dissi loro: 'Perché la casa di Dio è ella stata abbandonata?' Poi radunai i Leviti e i cantori e li ristabilii nei loro uffici.
\par 12 Allora tutto Giuda portò nei magazzini le decime del frumento, del vino e dell'olio;
\par 13 e affidai la sorveglianza dei magazzini al sacerdote Scelemia, allo scriba Tsadok, e a Pedaia uno dei Leviti; ai quali aggiunsi Hanan, figliuolo di Zaccur, figliuolo di Mattania, perché erano reputati uomini fedeli. Il loro ufficio era di fare le ripartizioni tra i loro fratelli.
\par 14 Ricòrdati per questo di me, o Dio mio, e non cancellare le opere pie che ho fatte per la casa del mio Dio e per il suo servizio!
\par 15 In que' giorni osservai in Giuda di quelli che calcavano l'uva negli strettoi in giorno di sabato, altri che portavano, caricandolo sugli asini, del grano od anche del vino, dell'uva, dei fichi, e ogni sorta di cose, che facean venire a Gerusalemme in giorno di sabato; e io li rimproverai a motivo del giorno in cui vendevano le loro derrate.
\par 16 C'erano anche dei Sirî, stabiliti a Gerusalemme, che portavano del pesce e ogni sorta di cose, e le vendevano ai figliuoli di Giuda in giorno di sabato, e in Gerusalemme.
\par 17 Allora io censurai i notabili di Giuda, e dissi loro: 'Che vuol dire questa mala azione che fate, profanando il giorno del sabato?
\par 18 I nostri padri non fecero essi così? e l'Iddio nostro fece, per questo, cader su noi e su questa città tutti questi mali. E voi accrescete l'ira ardente contro ad Israele, profanando il sabato!'
\par 19 E non appena le porte di Gerusalemme cominciarono ad esser nell'ombra, prima del sabato, io ordinai che le porte fossero chiuse, e che non si riaprissero fino a dopo il sabato; e collocai alcuni de' miei servi alle porte, affinché nessun carico entrasse in città durante il sabato.
\par 20 Così i mercanti e i venditori d'ogni sorta di cose una o due volte passarono la notte fuori di Gerusalemme.
\par 21 Allora io li rimproverai, e dissi loro: 'Perché passate voi la notte davanti alle mura? Se lo rifate, vi farò arrestare'. Da quel momento non vennero più il sabato.
\par 22 Io ordinai anche ai Leviti che si purificassero e venissero a custodire le porte per santificare il giorno del sabato. Anche per questo ricòrdati di me, o mio Dio, e abbi pietà di me secondo la grandezza della tua misericordia!
\par 23 In quei giorni vidi pure dei Giudei che s'erano ammogliati con donne di Ashdod, di Ammon e di Moab;
\par 24 e la metà dei loro figliuoli parlava l'asdodeo, ma non sapeva parlare la lingua de' Giudei; conosceva soltanto la lingua di questo o quest'altro popolo.
\par 25 E io li censurai, li maledissi, ne picchiai alcuni, strappai loro i capelli, e li feci giurare nel nome di Dio che non darebbero le loro figliuole ai figliuoli di costoro, e non prenderebbero le figliuole di coloro per i loro figliuoli né per loro stessi.
\par 26 E dissi: 'Salomone, re d'Israele, non peccò egli forse appunto in questo? e, certo, fra le molte nazioni, non ci fu re simile a lui; era amato dal suo Dio, e Dio l'aveva fatto re di tutto Israele; nondimeno, le donne straniere fecero peccare anche lui.
\par 27 E s'avrà egli a dir di voi che commettete questo gran male, che siete infedeli al nostro Dio, prendendo mogli straniere?'
\par 28 Uno de' figliuoli di Joiada, figliuolo di Eliascib, il sommo sacerdote, era genero di Samballat, lo Horonita; e io lo cacciai lungi da me.
\par 29 Ricòrdati di loro, o mio Dio, poiché hanno contaminato il sacerdozio e il patto fermato dal sacerdozio e dai Leviti!
\par 30 Così purificai il popolo da ogni elemento straniero, e ristabilii i servizi varî de' sacerdoti e de' Leviti, assegnando a ciascuno il suo lavoro.
\par 31 Ordinai pure il da farsi circa l'offerta delle legna ai tempi stabiliti, e circa le primizie.
\par 32 Ricòrdati di me, mio Dio, per farmi del bene!


\end{document}