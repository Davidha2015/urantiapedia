\begin{document}

\title{Giobbe}


\chapter{1}

\par 1 C'era nel paese di Uz un uomo che si chiamava Giobbe. Quest'uomo era integro e retto; temeva Iddio e fuggiva il male.
\par 2 Gli erano nati sette figliuoli e tre figliuole;
\par 3 possedeva settemila pecore, tremila cammelli, cinquecento paia di bovi, cinquecento asine e una servitù molto numerosa. E quest'uomo era il più grande di tutti gli Orientali.
\par 4 I suoi figliuoli solevano andare gli uni dagli altri e darsi un convito, ciascuno nel suo giorno: e mandavano a chiamare le loro tre sorelle perché venissero a mangiare e a bere con loro.
\par 5 E quando la serie dei giorni di convito era finita, Giobbe li faceva venire per purificarli; si levava di buon mattino, e offriva un olocausto per ciascun d'essi, perché diceva: 'Può darsi che i miei figliuoli abbian peccato ed abbiano rinnegato Iddio in cuor loro'. E Giobbe faceva sempre così.
\par 6 Or accadde un giorno, che i figliuoli di Dio vennero a presentarsi davanti all'Eterno, e Satana venne anch'egli in mezzo a loro.
\par 7 E l'Eterno disse a Satana: 'Donde vieni?' E Satana rispose all'Eterno: 'Dal percorrere la terra e dal passeggiare per essa'.
\par 8 E l'Eterno disse a Satana: 'Hai tu notato il mio servo Giobbe? Non ce n'è un altro sulla terra che come lui sia integro, retto, tema Iddio e fugga il male'.
\par 9 E Satana rispose all'Eterno: 'È egli forse per nulla che Giobbe teme Iddio?
\par 10 Non l'hai tu circondato d'un riparo, lui, la sua casa, e tutto quel che possiede? Tu hai benedetto l'opera delle sue mani, e il suo bestiame ricopre tutto il paese.
\par 11 Ma stendi un po' la tua mano, tocca quanto egli possiede, e vedrai se non ti rinnega in faccia'.
\par 12 E l'Eterno disse a Satana: 'Ebbene! tutto quello che possiede è in tuo potere; soltanto, non stender la mano sulla sua persona'. - E Satana si ritirò dalla presenza dell'Eterno.
\par 13 Or accadde che un giorno, mentre i suoi figliuoli e le sue figliuole mangiavano e bevevano del vino in casa del loro fratello maggiore, giunse a Giobbe un messaggero a dirgli:
\par 14 'I buoi stavano arando e le asine pascevano lì appresso,
\par 15 quand'ecco i Sabei son piombati loro addosso e li hanno portati via; hanno passato a fil di spada i servitori, e io solo son potuto scampare per venire a dirtelo'.
\par 16 Quello parlava ancora, quando ne giunse un altro a dire: 'Il fuoco di Dio è caduto dal cielo, ha colpito le pecore e i servitori, e li ha divorati; e io solo son potuto scampare per venire a dirtelo'.
\par 17 Quello parlava ancora, quando ne giunse un altro a dire: 'I Caldei hanno formato tre bande, si son gettati sui cammelli e li han portati via; hanno passato a fil di spada i servitori, e io solo son potuto scampare per venire a dirtelo'.
\par 18 Quello parlava ancora, quando ne giunse un altro a dire: 'I tuoi figliuoli e le tue figliuole mangiavano e bevevano del vino in casa del loro fratello maggiore;
\par 19 ed ecco che un gran vento, venuto dall'altra parte del deserto, ha investito i quattro canti della casa, ch'è caduta sui giovani; ed essi sono morti; e io solo son potuto scampare per venire a dirtelo'.
\par 20 Allora Giobbe si alzò e si stracciò il mantello e si rase il capo e si prostrò a terra e adorò e disse:
\par 21 'Nudo sono uscito dal seno di mia madre, e nudo tornerò in seno della terra; l'Eterno ha dato, l'Eterno ha tolto; sia benedetto il nome dell'Eterno'.
\par 22 In tutto questo Giobbe non peccò e non attribuì a Dio nulla di mal fatto.

\chapter{2}

\par 1 Or accadde un giorno, che i figliuoli di Dio vennero a presentarsi davanti all'Eterno, e Satana venne anch'egli in mezzo a loro a presentarsi davanti all'Eterno.
\par 2 E l'Eterno disse a Satana: 'Donde vieni?' E Satana rispose all'Eterno: 'Dal percorrere la terra e dal passeggiare per essa'. E l'Eterno disse a Satana:
\par 3 'Hai tu notato il mio servo Giobbe? Non ce n'è un altro sulla terra che come lui sia integro, retto, tema Iddio e fugga il male. Egli si mantiene saldo nella sua integrità benché tu m'abbia incitato contro di lui per rovinarlo senza alcun motivo'.
\par 4 E Satana rispose all'Eterno: 'Pelle per pelle! L'uomo dà tutto quel che possiede per la sua vita;
\par 5 ma stendi un po' la tua mano, toccagli le ossa e la carne, e vedrai se non ti rinnega in faccia'.
\par 6 E l'Eterno disse a Satana: 'Ebbene esso è in tuo potere; soltanto, rispetta la sua vita'.
\par 7 E Satana si ritirò dalla presenza dell'Eterno e colpì Giobbe d'un'ulcera maligna dalla pianta de' piedi al sommo del capo; e Giobbe prese un còccio per grattarsi, e stava seduto nella cenere.
\par 8 E sua moglie gli disse: 'Ancora stai saldo nella tua integrità?
\par 9 Ma lascia stare Iddio, e muori!'
\par 10 E Giobbe a lei: 'Tu parli da donna insensata! Abbiamo accettato il bene dalla mano di Dio, e rifiuteremmo d'accettare il male?' - In tutto questo Giobbe non peccò con le sue labbra.
\par 11 Or tre amici di Giobbe, Elifaz di Teman, Bildad di Suach e Tsofar di Naama, avendo udito tutti questi mali che gli eran piombati addosso, partirono, ciascuno dal suo paese e si misero d'accordo per venire a condolersi con lui e a consolarlo.
\par 12 E, levati gli occhi da lontano, essi non lo riconobbero, e alzarono la voce e piansero; si stracciarono i mantelli e si cosparsero il capo di polvere gittandola verso il cielo.
\par 13 E rimasero seduti per terra, presso a lui, sette giorni e sette notti; e nessuno di loro gli disse verbo, perché vedevano che il suo dolore era molto grande.

\chapter{3}

\par 1 Allora Giobbe aprì la bocca e maledisse il giorno della sua nascita.
\par 2 E prese a dire così:
\par 3 "Perisca il giorno ch'io nacqui e la notte che disse: 'È concepito un maschio!'
\par 4 Quel giorno si converta in tenebre, non se ne curi Iddio dall'alto, né splenda sovr'esso raggio di luce!
\par 5 Se lo riprendano le tenebre e l'ombra di morte, resti sovr'esso una fitta nuvola, le ecclissi lo riempian di paura!
\par 6 Quella notte diventi preda d'un buio cupo, non abbia la gioia di contar tra i giorni dell'anno, non entri nel novero de' mesi!
\par 7 Quella notte sia notte sterile, e non vi s'oda grido di gioia.
\par 8 La maledicano quei che maledicono i giorni e sono esperti nell'evocare il drago.
\par 9 Si oscurino le stelle del suo crepuscolo, aspetti la luce e la luce non venga, e non miri le palpebre dell'alba,
\par 10 poiché non chiuse la porta del seno che mi portava, e non celò l'affanno agli occhi miei.
\par 11 Perché non morii nel seno di mia madre? Perché non spirai appena uscito dalle sue viscere?
\par 12 Perché trovai delle ginocchia per ricevermi e delle mammelle da poppare?
\par 13 Ora mi giacerei tranquillo, dormirei, ed avrei così riposo
\par 14 coi re e coi consiglieri della terra che si edificarono mausolei,
\par 15 coi principi che possedean dell'oro e che empiron d'argento le lor case;
\par 16 o, come l'aborto nascosto, non esisterei, sarei come i feti che non videro la luce.
\par 17 Là cessano gli empi di tormentare gli altri. Là riposano gli stanchi,
\par 18 là i prigioni han requie tutti insieme, senz'udir voce d'aguzzino.
\par 19 Piccoli e grandi sono là del pari, e lo schiavo è libero del suo padrone.
\par 20 Perché dar la luce all'infelice e la vita a chi ha l'anima nell'amarezza,
\par 21 i quali aspettano la morte che non viene, e la ricercano più che i tesori nascosti,
\par 22 e si rallegrerebbero fino a giubilarne, esulterebbero se trovassero una tomba?
\par 23 Perché dar vita a un uomo la cui via è oscura, e che Dio ha stretto in un cerchio?
\par 24 Io sospiro anche quando prendo il mio cibo, e i miei gemiti si spandono com'acqua.
\par 25 Non appena temo un male, ch'esso mi colpisce; e quel che pavento, mi piomba addosso.
\par 26 Non trovo posa, né requie, né pace, il tormento è continuo!"

\chapter{4}

\par 1 Allora Elifaz di Teman rispose e disse:
\par 2 "Se provassimo a dirti una parola ti darebbe fastidio? Ma chi potrebbe trattener le parole?
\par 3 Ecco tu n'hai ammaestrati molti, hai fortificato le mani stanche;
\par 4 le tue parole hanno rialzato chi stava cadendo, hai raffermato le ginocchia vacillanti;
\par 5 e ora che il male piomba su te, tu ti lasci abbattere; ora ch'è giunto fino a te, sei tutto smarrito.
\par 6 La tua pietà non è forse la tua fiducia, e l'integrità della tua vita la speranza tua?
\par 7 Ricorda: quale innocente perì mai? e dove furono gli uomini retti mai distrutti?
\par 8 Io per me ho visto che coloro che arano iniquità e seminano tormenti, ne mietono i frutti.
\par 9 Al soffio di Dio essi periscono, dal vento del suo corruccio son consumati.
\par 10 Spenta è la voce del ruggente, sono spezzati i denti dei leoncelli.
\par 11 Perisce per mancanza di preda il forte leone, e restan dispersi i piccini della leonessa.
\par 12 Una parola m'è furtivamente giunta, e il mio orecchio ne ha còlto il lieve sussurro.
\par 13 Fra i pensieri delle visioni notturne, quando un sonno profondo cade sui mortali,
\par 14 uno spavento mi prese, un tremore, che mi fece fremer tutte l'ossa.
\par 15 Uno spirito mi passò dinanzi, e i peli mi si rizzarono addosso.
\par 16 Si fermò, ma non riconobbi il suo sembiante; una figura mi stava davanti agli occhi e udii una voce sommessa che diceva:
\par 17 'Può il mortale esser giusto dinanzi a Dio? Può l'uomo esser puro dinanzi al suo Fattore?
\par 18 Ecco, Iddio non si fida de' suoi propri servi, e trova difetti nei suoi angeli;
\par 19 quanto più in quelli che stanno in case d'argilla, che han per fondamento la polvere e son schiacciati al par delle tignuole!
\par 20 Tra la mattina e la sera sono infranti; periscono per sempre, senza che alcuno se ne accorga.
\par 21 La corda della lor tenda, ecco, è strappata, e muoion senza posseder la sapienza'.

\chapter{5}

\par 1 Chiama pure! C'è forse chi ti risponda? E a qual dei santi vorrai tu rivolgerti?
\par 2 No, il cruccio non uccide che l'insensato e l'irritazione non fa morir che lo stolto.
\par 3 Io ho veduto l'insensato prender radice, ma ben tosto ho dovuto maledirne la dimora.
\par 4 I suoi figli van privi di soccorso, sono oppressi alla porta, e non c'è chi li difenda.
\par 5 L'affamato gli divora la raccolta, gliela rapisce perfino di tra le spine; e l'assetato gli trangugia i beni.
\par 6 Ché la sventura non spunta dalla terra né il dolore germina dal suolo;
\par 7 ma l'uomo nasce per soffrire, come la favilla per volare in alto.
\par 8 Io però vorrei cercar di Dio, e a Dio vorrei esporre la mia causa:
\par 9 a lui, che fa cose grandi, imperscrutabili, maraviglie senza numero;
\par 10 che spande la pioggia sopra la terra e manda le acque sui campi;
\par 11 che innalza quelli ch'erano abbassati e pone in salvo gli afflitti in luogo elevato;
\par 12 che sventa i disegni degli astuti sicché le loro mani non giungono ad eseguirli:
\par 13 che prende gli abili nella loro astuzia, sì che il consiglio degli scaltri va in rovina.
\par 14 Di giorno essi incorron nelle tenebre, in pien mezzodì brancolan come di notte;
\par 15 ma Iddio salva il meschino dalla spada della lor bocca, e il povero di man del potente.
\par 16 E così pel misero v'è speranza, mentre l'iniquità ha la bocca chiusa.
\par 17 Beato l'uomo che Dio castiga! E tu non isdegnar la correzione dell'Onnipotente;
\par 18 giacché egli fa la piaga, poi la fascia; egli ferisce, ma le sue mani guariscono.
\par 19 In sei distrette egli sarà il tuo liberatore e in sette il male non ti toccherà.
\par 20 In tempo di carestia ti scamperà dalla morte, in tempo di guerra dai colpi della spada.
\par 21 Sarai sottratto al flagello della lingua, non temerai quando verrà il disastro.
\par 22 In mezzo al disastro e alla fame riderai, non paventerai le belve della terra;
\par 23 perché avrai per alleate le pietre del suolo, e gli animali de' campi saran teco in pace.
\par 24 Saprai sicura la tua tenda; e, visitando i tuoi pascoli, vedrai che non ti manca nulla.
\par 25 Saprai che la tua progenie moltiplica, che i tuoi rampolli crescono come l'erba dei campi.
\par 26 Scenderai maturo nella tomba, come la bica di mannelle che si ripone a suo tempo.
\par 27 Ecco quel che abbiam trovato, riflettendo. Così è. Tu ascolta, e fanne tuo pro".

\chapter{6}

\par 1 Allora Giobbe rispose e disse:
\par 2 "Ah, se il mio travaglio si pesasse, se le mie calamità si mettessero tutte insieme sulla bilancia!
\par 3 Sarebbero trovati più pesanti che la sabbia del mare. Ecco perché le mie parole sono temerarie.
\par 4 Ché le saette dell'Onnipotente mi trafiggono, lo spirito mio ne sugge il veleno; i terrori di Dio si schierano in battaglia contro me.
\par 5 L'asino salvatico raglia forse quand'ha l'erba davanti? mugghia forse il bue davanti alla pastura?
\par 6 Si può egli mangiar ciò ch'è scipito e senza sale? c'è qualche gusto in un chiaro d'uovo?
\par 7 L'anima mia rifiuta di toccare una simil cosa, essa è per me come un cibo ripugnante.
\par 8 Oh, m'avvenisse pur quello che chiedo, e mi desse Iddio quello che spero!
\par 9 Volesse pure Iddio schiacciarmi, stender la mano e tagliare il filo de' miei giorni!
\par 10 Sarebbe questo un conforto per me, esulterei nei dolori ch'egli non mi risparmia; giacché non ho rinnegato le parole del Santo.
\par 11 Che è mai la mia forza perch'io speri ancora? Che fine m'aspetta perch'io sia paziente?
\par 12 La mia forza è essa forza di pietra? e la mia carne, carne di rame?
\par 13 Non son io ridotto senza energia, e non m'è forse tolta ogni speranza di guarire?
\par 14 Pietà deve l'amico a colui che soccombe, quand'anche abbandoni il timor dell'Onnipotente.
\par 15 Ma i fratelli miei si son mostrati infidi come un torrente, come l'acqua di torrenti che passano.
\par 16 Il ghiaccio li rende torbidi, e la neve vi si scioglie;
\par 17 ma passato il tempo delle piene, svaniscono; quando sentono il caldo, scompariscono dal loro luogo.
\par 18 Le carovane che si dirigon là mutano strada, s'inoltran nel deserto, e vi periscono.
\par 19 Le carovane di Tema li cercavan collo sguardo, i viandanti di Sceba ci contavan su,
\par 20 ma furon delusi nella loro fiducia; giunti sul luogo, rimasero confusi.
\par 21 Tali siete divenuti voi per me: vedete uno che fa orrore, e vi prende la paura.
\par 22 V'ho forse detto: 'Datemi qualcosa' o 'co' vostri beni fate un donativo a favor mio',
\par 23 o 'liberatemi dalla stretta del nemico', o 'scampatemi di man dei prepotenti'?
\par 24 Ammaestratemi, e mi starò in silenzio; fatemi capire in che cosa ho errato.
\par 25 Quanto sono efficaci le parole rette! Ma la vostra riprensione che vale?
\par 26 Volete dunque biasimar delle parole? Ma le parole d'un disperato se le porta il vento!
\par 27 Voi sareste capaci di trar la sorte sull'orfano, e di contrattare il vostro amico!
\par 28 Ma pure vi piaccia di rivolgervi a guardarmi, e vedete s'io vi menta in faccia.
\par 29 Mutate consiglio! Non vi sia in voi iniquità! Mutate consiglio, la mia giustizia sussiste.
\par 30 V'è qualche iniquità sulla mia lingua? Il mio palato non distingue più quel ch'è male?

\chapter{7}

\par 1 La vita dell'uomo sulla terra è una milizia; i giorni suoi son simili ai giorni d'un operaio.
\par 2 Come lo schiavo anela l'ombra e come l'operaio aspetta il suo salario,
\par 3 così a me toccan mesi di sciagura, e mi sono assegnate notti di dolore.
\par 4 Non appena mi corico, dico: 'Quando mi leverò?' Ma la notte si prolunga, e mi sazio d'agitazioni infino all'alba.
\par 5 La mia carne è coperta di vermi e di croste terrose, la mia pelle si richiude, poi riprende a suppurare.
\par 6 I miei giorni sen vanno più veloci della spola, si consumano senza speranza.
\par 7 Ricordati, che la mia vita è un soffio! L'occhio mio non vedrà più il bene.
\par 8 Lo sguardo di chi ora mi vede non mi potrà più scorgere; gli occhi tuoi mi cercheranno, ma io non sarò più.
\par 9 La nuvola svanisce e si dilegua; così chi scende nel soggiorno de' morti non ne risalirà;
\par 10 non tornerà più nella sua casa, e il luogo ove stava non lo riconoscerà più.
\par 11 Io, perciò, non terrò chiusa la bocca; nell'angoscia del mio spirito io parlerò, mi lamenterò nell'amarezza dell'anima mia.
\par 12 Son io forse il mare o un mostro marino che tu ponga intorno a me una guardia?
\par 13 Quando dico: 'Il mio letto mi darà sollievo, il mio giaciglio allevierà la mia pena',
\par 14 tu mi sgomenti con sogni, e mi spaventi con visioni;
\par 15 sicché l'anima mia preferisce soffocare, preferisce a queste ossa la morte.
\par 16 Io mi vo struggendo; non vivrò sempre; deh, lasciami stare; i giorni miei non son che un soffio.
\par 17 Che cosa è l'uomo che tu ne faccia tanto caso, che tu ponga mente ad esso,
\par 18 e lo visiti ogni mattina e lo metta alla prova ad ogni istante?
\par 19 Quando cesserai di tener lo sguardo fisso su me? Quando mi darai tempo d'inghiottir la mia saliva?
\par 20 Se ho peccato, che ho fatto a te, o guardiano degli uomini? Perché hai fatto di me il tuo bersaglio? A tal punto che son divenuto un peso a me stesso?
\par 21 E perché non perdoni le mie trasgressioni e non cancelli la mia iniquità? Poiché presto giacerò nella polvere; e tu mi cercherai, ma io non sarò più".

\chapter{8}

\par 1 Allora Bildad di Suach rispose e disse:
\par 2 "Fino a quando terrai tu questi discorsi e saran le parole della tua bocca come un vento impetuoso?
\par 3 Iddio perverte egli il giudizio? L'Onnipotente perverte egli la giustizia?
\par 4 Se i tuoi figliuoli han peccato contro lui, egli li ha dati in balìa del loro misfatto;
\par 5 ma tu, se ricorri a Dio e implori grazia dall'Onnipotente,
\par 6 se proprio sei puro e integro, certo egli sorgerà in tuo favore, e restaurerà la dimora della tua giustizia.
\par 7 Così sarà stato piccolo il tuo principio, ma la tua fine sarà grande oltre modo.
\par 8 Interroga le passate generazioni, rifletti sull'esperienza de' padri;
\par 9 giacché noi siam d'ieri e non sappiamo nulla; i nostri giorni sulla terra non son che un'ombra;
\par 10 ma quelli certo t'insegneranno, ti parleranno, e dal loro cuore trarranno discorsi.
\par 11 Può il papiro crescere ove non c'è limo? Il giunco viene egli su senz'acqua?
\par 12 Mentre son verdi ancora, e senza che li si tagli, prima di tutte l'erbe, seccano.
\par 13 Tale la sorte di tutti quei che dimenticano Dio, e la speranza dell'empio perirà.
\par 14 La sua baldanza è troncata, la sua fiducia, è come una tela di ragno.
\par 15 Egli s'appoggia alla sua casa, ma essa non regge; vi s'aggrappa, ma quella non sta salda.
\par 16 Egli verdeggia al sole, e i suoi rami si protendono sul suo giardino;
\par 17 le sue radici s'intrecciano sul mucchio delle macerie, penetra fra le pietre della casa.
\par 18 Ma divelto che sia dal suo luogo, questo lo rinnega e gli dice: 'Non ti ho mai veduto!'
\par 19 Ecco il gaudio che gli procura la sua condotta! E dalla polvere altri dopo lui germoglieranno.
\par 20 No, Iddio non rigetta l'uomo integro, né porge aiuto a quelli che fanno il male.
\par 21 Egli renderà ancora il sorriso alla tua bocca, e sulle tue labbra metterà canti d'esultanza.
\par 22 Quelli che t'odiano saran coperti di vergogna, e la tenda degli empi sparirà".

\chapter{9}

\par 1 Allora Giobbe rispose e disse:
\par 2 "Sì, certo, io so ch'egli è così; e come sarebbe il mortale giusto davanti a Dio?
\par 3 Se all'uomo piacesse di piatir con Dio, non potrebbe rispondergli sovra un punto fra mille.
\par 4 Dio è savio di cuore, è grande in potenza; chi gli ha tenuto fronte e se n'è trovato bene?
\par 5 Egli trasporta le montagne senza che se ne avvedano, nel suo furore le sconvolge.
\par 6 Egli scuote la terra dalle sue basi, e le sue colonne tremano.
\par 7 Comanda al sole, ed esso non si leva; mette un sigillo sulle stelle.
\par 8 Da solo spiega i cieli, e cammina sulle più alte onde del mare.
\par 9 È il creatore dell'Orsa, d'Orione, delle Pleiadi, e delle misteriose regioni del cielo australe.
\par 10 Egli fa cose grandi e imperscrutabili, maraviglie senza numero.
\par 11 Ecco, ei mi passa vicino, ed io nol veggo; mi scivola daccanto e non me n'accorgo.
\par 12 Ecco afferra la preda, e chi si opporrà? Chi oserà dirgli: 'Che fai?'
\par 13 Iddio non ritira la sua collera; sotto di lui si curvano i campioni della superbia.
\par 14 E io, come farei a rispondergli, a sceglier le mie parole per discuter con lui?
\par 15 Avessi anche ragione, non gli replicherei, ma chiederei mercé al mio giudice.
\par 16 S'io lo invocassi ed egli mi rispondesse, non però crederei che avesse dato ascolto alla mia voce;
\par 17 egli che mi piomba addosso dal seno della tempesta, che moltiplica senza motivo le mie piaghe,
\par 18 che non mi lascia riprender fiato, e mi sazia d'amarezza.
\par 19 Se si tratta di forza, ecco, egli è potente; se di diritto, ei dice: 'Chi mi fisserà un giorno per comparire'?
\par 20 Fossi pur giusto, la mia bocca stessa mi condannerebbe; fossi pure integro, essa mi farebbe dichiarar perverso.
\par 21 Integro! Sì, lo sono! di me non mi preme, io disprezzo la vita!
\par 22 Per me è tutt'uno! perciò dico: 'Egli distrugge ugualmente l'integro ed il malvagio.
\par 23 Se un flagello, a un tratto, semina la morte, egli ride dello sgomento degli innocenti.
\par 24 La terra è data in balìa dei malvagi; ei vela gli occhi ai giudici di essa; se non è lui, chi è dunque'?
\par 25 E i miei giorni se ne vanno più veloci d'un corriere; fuggono via senz'aver visto il bene;
\par 26 passan rapidi come navicelle di giunchi, come l'aquila che piomba sulla preda.
\par 27 Se dico: 'Voglio dimenticare il mio lamento, deporre quest'aria triste e rasserenarmi',
\par 28 sono spaventato di tutti i miei dolori, so che non mi terrai per innocente.
\par 29 Io sarò condannato; perché dunque affaticarmi invano?
\par 30 Quand'anche mi lavassi con la neve e mi nettassi le mani col sapone,
\par 31 tu mi tufferesti nel fango d'una fossa, le mie vesti stesse m'avrebbero in orrore.
\par 32 Dio non è un uomo come me, perch'io gli risponda e che possiam comparire in giudizio assieme.
\par 33 Non c'è fra noi un arbitro, che posi la mano su tutti e due!
\par 34 Ritiri Iddio d'addosso a me la sua verga; cessi dallo spaventarmi il suo terrore;
\par 35 allora io parlerò senza temerlo, giacché sento di non essere quel colpevole che sembro.

\chapter{10}

\par 1 L'anima mia prova disgusto della vita; vo' dar libero corso al mio lamento, vo' parlar nell'amarezza dell'anima mia!
\par 2 Io dirò a Dio: 'Non mi condannare! Fammi sapere perché contendi meco!'
\par 3 Ti par egli ben fatto d'opprimere, di sprezzare l'opera delle tue mani e di favorire i disegni de' malvagi?
\par 4 Hai tu occhi di carne? Vedi tu come vede l'uomo?
\par 5 I tuoi giorni son essi come i giorni del mortale, i tuoi anni son essi come gli anni degli umani,
\par 6 che tu investighi tanto la mia iniquità, che t'informi così del mio peccato,
\par 7 pur sapendo ch'io non son colpevole, e che non v'è chi mi liberi dalla tua mano?
\par 8 Le tue mani m'hanno formato, m'hanno fatto tutto quanto... e tu mi distruggi!
\par 9 Deh, ricordati che m'hai plasmato come argilla... e tu mi fai ritornare in polvere!
\par 10 Non m'hai tu colato come il latte e fatto rapprender come il cacio?
\par 11 Tu m'hai rivestito di pelle e di carne, e m'hai intessuto d'ossa e di nervi.
\par 12 Mi sei stato largo di vita e di grazia, la tua provvidenza ha vegliato sul mio spirito,
\par 13 ed ecco quello che nascondevi in cuore! Sì, lo so, questo meditavi:
\par 14 se avessi peccato, l'avresti ben tenuto a mente, e non m'avresti assolto dalla mia iniquità.
\par 15 Se fossi stato malvagio, guai a me! Se giusto, non avrei osato alzar la fronte, sazio d'ignominia, spettatore della mia miseria.
\par 16 Se l'avessi alzata, m'avresti dato la caccia come ad un leone e contro di me avresti rinnovato le tue maraviglie;
\par 17 m'avresti messo a fronte nuovi testimoni, e avresti raddoppiato il tuo sdegno contro di me; legioni su legioni m'avrebbero assalito.
\par 18 E allora, perché m'hai tratto dal seno di mia madre? Sarei spirato senza che occhio mi vedesse!
\par 19 Sarei stato come se non fossi mai esistito, m'avrebbero portato dal seno materno alla tomba!
\par 20 Non son forse pochi i giorni che mi restano? Cessi egli dunque, mi lasci stare, ond'io mi rassereni un poco,
\par 21 prima ch'io me ne vada, per non più tornare, nella terra delle tenebre e dell'ombra di morte:
\par 22 terra oscura come notte profonda, ove regnano l'ombra di morte ed il caos, il cui chiarore è come notte oscura".

\chapter{11}

\par 1 Allora Tsofar di Naama rispose e disse:
\par 2 "Cotesta abbondanza di parole rimarrà ella senza risposta? Basterà egli esser loquace per aver ragione?
\par 3 Varranno le tue ciance a far tacere la gente? Farai tu il beffardo, senza che alcuno ti confonda?
\par 4 Tu dici a Dio: 'Quel che sostengo è giusto, e io sono puro nel tuo cospetto'.
\par 5 Ma, oh se Iddio volesse parlare e aprir la bocca per risponderti
\par 6 e rivelarti i segreti della sua sapienza - poiché infinita è la sua intelligenza - vedresti allora come Iddio dimentichi parte della colpa tua.
\par 7 Puoi tu scandagliare le profondità di Dio? arrivare a conoscere appieno l'Onnipotente?
\par 8 Si tratta di cose più alte del cielo... e tu che faresti? di cose più profonde del soggiorno de' morti... come le conosceresti?
\par 9 La lor misura è più lunga della terra, più larga del mare.
\par 10 Se Dio passa, se incarcera, se chiama in giudizio, chi s'opporrà?
\par 11 Poich'egli conosce gli uomini perversi, scopre senza sforzo l'iniquità.
\par 12 Ma l'insensato diventerà savio, quando un puledro d'onàgro diventerà uomo.
\par 13 Tu, però, se ben disponi il cuore, e protendi verso Dio le palme,
\par 14 se allontani il male ch'è nelle tue mani, e non alberghi l'iniquità nelle tue tende,
\par 15 allora alzerai la fronte senza macchia, sarai incrollabile, e non avrai paura di nulla;
\par 16 dimenticherai i tuoi affanni; te ne ricorderai come d'acqua passata;
\par 17 la tua vita sorgerà più fulgida del meriggio, l'oscurità sarà come la luce del mattino.
\par 18 Sarai fiducioso perché avrai speranza; ti guarderai bene attorno e ti coricherai sicuro.
\par 19 Ti metterai a giacere e niuno ti spaventerà; e molti cercheranno il tuo favore.
\par 20 Ma gli occhi degli empi verranno meno; non vi sarà più rifugio per loro, e non avranno altra speranza che di esalar l'anima".

\chapter{12}

\par 1 Allora Giobbe rispose e disse:
\par 2 "Voi, certo, valete quanto un popolo, e con voi morrà la sapienza.
\par 3 Ma del senno ne ho anch'io al par di voi, non vi son punto inferiore; e cose come codeste chi non le sa?
\par 4 Io dunque dovrei essere il ludibrio degli amici! Io che invocavo Iddio, ed ei mi rispondeva; il ludibrio io, l'uomo giusto, integro!
\par 5 Lo sprezzo alla sventura è nel pensiero di chi vive contento; esso è sempre pronto per coloro a cui vacilla il piede.
\par 6 Sono invece tranquille le tende de' ladroni e chi provoca Iddio, chi si fa un dio della propria forza, se ne sta al sicuro.
\par 7 Ma interroga un po' gli animali, e te lo insegneranno; gli uccelli del cielo, e te lo mostreranno;
\par 8 o parla alla terra ed essa te lo insegnerà, e i pesci del mare te lo racconteranno.
\par 9 Chi non sa, fra tutte queste creature, che la mano dell'Eterno ha fatto ogni cosa,
\par 10 ch'egli tiene in mano l'anima di tutto quel che vive, e lo spirito di ogni essere umano?
\par 11 L'orecchio non discerne esso le parole, come il palato assaggia le vivande?
\par 12 Nei vecchi si trova la sapienza e la lunghezza di giorni dà intelligenza.
\par 13 Ma in Dio stanno la saviezza e la potenza, a lui appartengono il consiglio e l'intelligenza.
\par 14 Ecco, egli abbatte, e niuno può ricostruire; chiude un uomo in prigione, e non v'è chi gli apra.
\par 15 Ecco, egli trattiene le acque, e tutto inaridisce; le lascia andare, ed esse sconvolgono la terra.
\par 16 Egli possiede la forza e l'abilità; da lui dipendono chi erra e chi fa errare.
\par 17 Egli manda scalzi i consiglieri, colpisce di demenza i giudici.
\par 18 Scioglie i legami dell'autorità dei re e cinge i loro fianchi di catene.
\par 19 Manda scalzi i sacerdoti, e rovescia i potenti.
\par 20 Priva della parola i più eloquenti, e toglie il discernimento ai vecchi.
\par 21 Sparge lo sprezzo sui nobili, e rallenta la cintura ai forti.
\par 22 Rivela le cose recondite, facendole uscir dalle tenebre, e trae alla luce ciò ch'è avvolto in ombra di morte.
\par 23 Aggrandisce i popoli e li annienta, amplia le nazioni e le riconduce nei loro confini;
\par 24 toglie il senno ai capi della terra, e li fa errare in solitudini senza sentiero.
\par 25 Van brancolando nelle tenebre, senza alcuna luce, e li fa barcollare come ubriachi.

\chapter{13}

\par 1 Ecco, l'occhio mio tutto questo l'ha veduto; l'orecchio mio l'ha udito e l'ha inteso.
\par 2 Quel che sapete voi lo so pur io, non vi sono punto inferiore.
\par 3 Ma io vorrei parlare con l'Onnipotente, avrei caro di ragionar con Dio;
\par 4 giacché voi siete de' fabbri di menzogne, siete tutti quanti dei medici da nulla.
\par 5 Oh se serbaste il silenzio! esso vi conterebbe come sapienza.
\par 6 Ascoltate, vi prego, quel che ho da rimproverarvi; state attenti alle ragioni delle mie labbra!
\par 7 Volete dunque difendere Iddio parlando iniquamente? sostener la sua causa con parole di frode?
\par 8 Volete aver riguardo alla sua persona? e costituirvi gli avvocati di Dio?
\par 9 Sarà egli un bene per voi quando vi scruterà a fondo? credete ingannarlo come s'inganna un uomo?
\par 10 Certo egli vi riprenderà severamente se nel vostro segreto avete dei riguardi personali.
\par 11 La maestà sua non vi farà sgomenti? Il suo terrore non piomberà su di voi?
\par 12 I vostri detti memorandi son massime di cenere; i vostri baluardi son baluardi d'argilla.
\par 13 Tacete! lasciatemi stare! voglio parlare io, e m'avvenga quello che può!
\par 14 Perché prenderei la mia carne coi denti? Metterò piuttosto la mia vita nelle mie mani.
\par 15 Ecco, egli m'ucciderà; non spero più nulla; ma io difenderò in faccia a lui la mia condotta!
\par 16 Anche questo servirà alla mia salvezza; poiché un empio non ardirebbe presentarsi a lui.
\par 17 Ascoltate attentamente il mio discorso, porgete orecchio a quanto sto per dichiararvi.
\par 18 Ecco, io ho disposto ogni cosa per la causa, so che sarò riconosciuto giusto.
\par 19 V'è qualcuno che voglia farmi opposizione? Se v'è io mi taccio e vo' morire.
\par 20 Ma, o Dio, concedimi solo due cose, e non mi nasconderò dal tuo cospetto:
\par 21 ritirami d'addosso la tua mano, e fa' che i tuoi terrori non mi spaventin più.
\par 22 Poi interpellami, ed io risponderò; o parlerò io, e tu replicherai.
\par 23 Quante sono le mie iniquità, quanti i miei peccati? Fammi conoscere la mia trasgressione, il mio peccato!
\par 24 Perché nascondi il tuo volto, e mi tieni in conto di nemico?
\par 25 Vuoi tu atterrire una foglia portata via dal vento? Vuoi tu perseguitare una pagliuzza inaridita?
\par 26 tu che mi condanni a pene così amare, e mi fai espiare i falli della mia giovinezza,
\par 27 tu che metti i miei piedi nei ceppi; che spii tutti i miei movimenti, e tracci una linea intorno alla pianta de' miei piedi?
\par 28 Intanto questo mio corpo si disfa come legno tarlato, come un abito ròso dalle tignuole.

\chapter{14}

\par 1 L'uomo, nato di donna, vive pochi giorni, e sazio d'affanni.
\par 2 Spunta come un fiore, poi è reciso; fugge come un'ombra, e non dura.
\par 3 E sopra un essere così tu tieni gli occhi aperti! e mi fai comparir teco in giudizio!
\par 4 Chi può trarre una cosa pura da una impura? Nessuno.
\par 5 Giacché i suoi giorni son fissati, e il numero de' suoi mesi dipende da te, e tu gli hai posto un termine ch'egli non può varcare,
\par 6 storna da lui lo sguardo, sì ch'egli abbia un po' di requie, e possa godere come un operaio la fine della sua giornata.
\par 7 Per l'albero, almeno c'è speranza; se è tagliato, rigermoglia e continua a mettere rampolli.
\par 8 Quando la sua radice è invecchiata sotto terra, e il suo tronco muore nel suolo,
\par 9 a sentir l'acqua, rinverdisce e mette rami come una pianta nuova.
\par 10 Ma l'uomo muore e perde ogni forza; il mortale spira e... dov'è egli?
\par 11 Le acque del lago se ne vanno, il fiume vien meno e si prosciuga;
\par 12 così l'uomo giace, e non risorge più; finché non vi sian più cieli, ei non si risveglierà né sarà più destato dal suo sonno.
\par 13 Oh, volessi tu nascondermi nel soggiorno dei morti, tenermi occulto finché l'ira tua sia passata, fissarmi un termine, e poi ricordarti di me!...
\par 14 Se l'uomo, dopo morto, potesse ritornare in vita, aspetterei tutti i giorni della mia fazione, finché giungesse l'ora del mio cambio;
\par 15 tu mi chiameresti e io risponderei, tu brameresti rivedere l'opera delle tue mani.
\par 16 Ma ora tu conti i miei passi, tu osservi i miei peccati;
\par 17 le mie trasgressioni sono sigillate in un sacco, e alle mie iniquità, altre ne aggiungi.
\par 18 La montagna frana e scompare, la rupe è divelta dal suo luogo,
\par 19 le acque rodono la pietra, le loro inondazioni trascinan via la terra: così tu distruggi la speranza dell'uomo.
\par 20 Tu lo sopraffai una volta per sempre, ed egli se ne va; gli muti il sembiante, e lo mandi via.
\par 21 Se i suoi figliuoli salgono in onore, egli lo ignora; se vengono in dispregio, ei non lo vede;
\par 22 questo solo sente: che il suo corpo soffre, che l'anima sua è in lutto".

\chapter{15}

\par 1 Allora Elifaz di Teman rispose e disse:
\par 2 "Il savio risponde egli con vana scienza? si gonfia egli il petto di vento?
\par 3 Si difende egli con ciarle inutili e con parole che non giovan nulla?
\par 4 Tu, poi, distruggi il timor di Dio, menomi il rispetto religioso che gli è dovuto.
\par 5 La tua iniquità ti detta le parole, e adoperi il linguaggio degli astuti.
\par 6 Non io, la tua bocca stessa ti condanna; le tue labbra stesse depongono contro a te.
\par 7 Sei tu il primo uomo che nacque? Fosti tu formato prima de' monti?
\par 8 Hai tu sentito quel che s'è detto nel Consiglio di Dio? Hai tu fatto incetta della sapienza per te solo?
\par 9 Che sai tu che noi non sappiamo? Che conoscenza hai tu che non sia pur nostra?
\par 10 Ci son fra noi degli uomini canuti ed anche de' vecchi più attempati di tuo padre.
\par 11 Fai tu sì poco caso delle consolazioni di Dio e delle dolci parole che t'abbiam rivolte?
\par 12 Dove ti trascina il cuore, e che voglion dire codeste torve occhiate?
\par 13 Come! tu volgi la tua collera contro Dio, e ti lasci uscir di bocca tali parole?
\par 14 Che è mai l'uomo per esser puro, il nato di donna per esser giusto?
\par 15 Ecco, Iddio non si fida nemmeno de' suoi santi, i cieli non son puri agli occhi suoi;
\par 16 quanto meno quest'essere abominevole e corrotto, l'uomo, che tracanna l'iniquità come l'acqua!
\par 17 Io voglio ammaestrarti; porgimi ascolto, e ti racconterò quello che ho visto,
\par 18 quello che i Savi hanno riferito senza nulla celare di quel che sapean dai padri,
\par 19 ai quali soli è stato dato il paese; e in mezzo ai quali non è passato lo straniero.
\par 20 L'empio è tormentato tutti i suoi giorni, e pochi son gli anni riservati al prepotente.
\par 21 Sempre ha negli orecchi rumori spaventosi, e in piena pace gli piomba addosso il distruttore.
\par 22 Non ha speranza d'uscir dalle tenebre, e si sente destinato alla spada.
\par 23 Va errando in cerca di pane; dove trovarne? ei sa che a lui dappresso è pronto il giorno tenebroso.
\par 24 La distretta e l'angoscia lo riempion di paura, l'assalgono a guisa di re pronto alla pugna,
\par 25 perché ha teso la mano contro Dio, ha sfidato l'Onnipotente,
\par 26 gli s'è slanciato, audacemente contro, sotto il folto de' suoi scudi convessi.
\par 27 Avea la faccia coperta di grasso, i fianchi carichi di pinguedine;
\par 28 s'era stabilito in città distrutte, in case disabitate, destinate a diventar mucchi di sassi.
\par 29 Ei non s'arricchirà, la sua fortuna non sarà stabile; né le sue possessioni si stenderanno sulla terra.
\par 30 Non potrà liberarsi dalle tenebre, il vento infocato farà seccare i suoi rampolli, e sarà portato via dal soffio della bocca di Dio.
\par 31 Non confidi nella vanità; è un'illusione; poiché avrà la vanità per ricompensa.
\par 32 La sua fine verrà prima del tempo, e i suoi rami non rinverdiranno più.
\par 33 Sarà come vigna da cui si strappi l'uva ancor acerba, come l'ulivo da cui si scuota il fiore;
\par 34 poiché sterile è la famiglia del profano, e il fuoco divora le tende ov'entrano presenti.
\par 35 L'empio concepisce malizia, e partorisce rovina; ei si prepara in seno il disinganno".

\chapter{16}

\par 1 Allora Giobbe rispose e disse:
\par 2 "Di cose come codeste, ne ho udite tante! Siete tutti dei consolatori molesti!
\par 3 Non ci sarà egli una fine alle parole vane? Che cosa ti provoca a rispondere?
\par 4 Anch'io potrei parlare come voi, se voi foste al posto mio; potrei mettere assieme delle parole contro a voi e su di voi scrollare il capo;
\par 5 potrei farvi coraggio con la bocca; e il conforto delle mie labbra vi calmerebbe.
\par 6 Se parlo, il mio dolore non ne sarà lenito; e se cesso di parlare, che sollievo ne avrò?
\par 7 Ora, purtroppo, Dio, m'ha ridotto senza forze, ha desolato tutta la mia casa;
\par 8 m'ha coperto di grinze e questo testimonia contro a me, la mia magrezza si leva ad accusarmi in faccia.
\par 9 La sua ira mi lacera, mi perseguita, digrigna i denti contro di me. Il mio nemico aguzza gli occhi su di me.
\par 10 Apron larga contro a me la bocca, mi percuoton per obbrobrio le guance, si metton tutt'insieme a darmi addosso.
\par 11 Iddio mi dà in balìa degli empi, mi getta in mano dei malvagi.
\par 12 Vivevo in pace, ed egli m'ha scosso con violenza, m'ha preso per la nuca, m'ha frantumato, m'ha posto per suo bersaglio.
\par 13 I suoi arcieri mi circondano, egli mi trafigge i reni senza pietà, sparge a terra il mio fiele.
\par 14 Apre sopra di me breccia su breccia, mi corre addosso come un guerriero.
\par 15 Mi son cucito un cilicio sulla pelle, ho prostrato la mia fronte nella polvere.
\par 16 Il mio viso è rosso di pianto, e sulle mie palpebre si stende l'ombra di morte.
\par 17 Eppure, le mie mani non commisero mai violenza, e la mia preghiera fu sempre pura.
\par 18 O terra, non coprire il mio sangue, e non vi sia luogo ove si fermi il mio grido!
\par 19 Già fin d'ora, ecco, il mio Testimonio è in cielo, il mio Garante è nei luoghi altissimi.
\par 20 Gli amici mi deridono, ma a Dio si volgon piangenti gli occhi miei;
\par 21 sostenga egli le ragioni dell'uomo presso Dio, le ragioni del figliuol d'uomo contro i suoi compagni!
\par 22 Poiché, pochi anni ancora, e me ne andrò per una via senza ritorno.

\chapter{17}

\par 1 Il mio soffio vitale si spegne, i miei giorni si estinguono, il sepolcro m'aspetta!
\par 2 Sono attorniato di schernitori e non posso chiuder occhio per via delle lor parole amare.
\par 3 O Dio, da' un pegno, sii tu il mio mallevadore presso di te; se no, chi metterà la sua nella mia mano?
\par 4 Poiché tu hai chiuso il cuor di costoro alla ragione, e però non li farai trionfare.
\par 5 Chi denunzia un amico sì che diventi preda altrui, vedrà venir meno gli occhi de' suoi figli.
\par 6 Egli m'ha reso la favola dei popoli, e son divenuto un essere a cui si sputa in faccia.
\par 7 L'occhio mio si oscura pel dolore, tutte le mie membra non son più che un'ombra.
\par 8 Gli uomini retti ne son colpiti di stupore, e l'innocente insorge contro l'empio;
\par 9 ma il giusto si attiene saldo alla sua via, e chi ha le mani pure viepiù si fortifica.
\par 10 Quanto a voi tutti, tornate pure, fatevi avanti, ma fra voi non troverò alcun savio.
\par 11 I miei giorni passano, i miei disegni, i disegni cari al mio cuore, sono distrutti,
\par 12 e costoro pretendon che la notte sia giorno, che la luce sia vicina, quando tutto è buio!
\par 13 Se aspetto come casa mia il soggiorno de' morti, se già mi son fatto il letto nelle tenebre,
\par 14 se ormai dico al sepolcro 'tu sei mio padre' e ai vermi: 'siete mia madre e mia sorella',
\par 15 dov'è dunque la mia speranza? questa speranza mia chi la può scorgere?
\par 16 Essa scenderà alle porte del soggiorno de' morti, quando nella polvere troverem riposo assieme".

\chapter{18}

\par 1 Allora Bildad di Suach rispose e disse:
\par 2 "Quando porrete fine alle parole? Fate senno, e poi parleremo.
\par 3 Perché siamo considerati come bruti e perché siamo agli occhi vostri degli esseri impuri?
\par 4 O tu, che nel tuo cruccio laceri te stesso, dovrà la terra, per cagion tua, essere abbandonata e la roccia esser rimossa dal suo luogo?
\par 5 Sì, la luce dell'empio si spegne, e la fiamma del suo fuoco non brilla.
\par 6 La luce si oscura nella sua tenda, e la lampada che gli sta sopra si spegne.
\par 7 I passi che facea nella sua forza si raccorciano, e i suoi propri disegni lo menano a ruina.
\par 8 Poiché i suoi piedi lo traggon nel tranello, e va camminando sulle reti.
\par 9 Il laccio l'afferra pel tallone, e la trappola lo ghermisce.
\par 10 Sta nascosta in terra per lui un'insidia, e sul sentiero lo aspetta un agguato.
\par 11 Paure lo atterriscono d'ogn'intorno, lo inseguono, gli stanno alle calcagna.
\par 12 La sua forza vien meno dalla fame, la calamità gli sta pronta al fianco.
\par 13 Gli divora a pezzo a pezzo la pelle, gli divora le membra il primogenito della morte.
\par 14 Egli è strappato dalla sua tenda che credea sicura, e fatto scendere verso il re degli spaventi.
\par 15 Nella sua tenda dimora chi non è de' suoi, e la sua casa è cosparsa di zolfo.
\par 16 In basso s'inaridiscono le sue radici, in alto son tagliati i suoi rami.
\par 17 La sua memoria scompare dal paese, più non s'ode il suo nome per le campagne.
\par 18 È cacciato dalla luce nelle tenebre, ed è bandito dal mondo.
\par 19 Non lascia tra il suo popolo né figli, né nipoti, nessun superstite dov'egli soggiornava.
\par 20 Quei d'occidente son stupiti della sua sorte, e quei d'oriente ne son presi d'orrore.
\par 21 Certo son tali le dimore dei perversi e tale è il luogo di chi non conosce Iddio".

\chapter{19}

\par 1 Allora Giobbe rispose e disse:
\par 2 "Fino a quando affliggerete l'anima mia e mi tormenterete coi vostri discorsi?
\par 3 Son già dieci volte che m'insultate, e non vi vergognate di malmenarmi.
\par 4 Dato pure ch'io abbia errato, il mio errore concerne me solo.
\par 5 Ma se proprio volete insuperbire contro di me e rimproverarmi la vergogna in cui mi trovo,
\par 6 allora sappiatelo: chi m'ha fatto torto e m'ha avvolto nelle sue reti è Dio.
\par 7 Ecco, io grido: 'Violenza!' e nessuno risponde; imploro aiuto, ma non c'è giustizia!
\par 8 Dio m'ha sbarrato la via e non posso passare, ha coperto di tenebre il mio cammino.
\par 9 M'ha spogliato della mia gloria, m'ha tolto dal capo la corona.
\par 10 M'ha demolito a brano a brano, e io me ne vo! ha sradicata come un albero la mia speranza.
\par 11 Ha acceso l'ira sua contro di me, e m'ha considerato come suo nemico.
\par 12 Le sue schiere son venute tutte insieme, si sono spianata la via fino a me, han posto il campo intorno alla mia tenda.
\par 13 Egli ha allontanato da me i miei fratelli, i miei conoscenti si son del tutto alienati da me.
\par 14 M'hanno abbandonato i miei parenti, gl'intimi miei m'hanno dimenticato.
\par 15 I miei domestici e le mie serve mi trattan da straniero; agli occhi loro io sono un estraneo.
\par 16 Chiamo il mio servo, e non risponde, devo supplicarlo con la mia bocca.
\par 17 Il mio fiato ripugna alla mia moglie, faccio pietà a chi nacque dal seno di mia madre.
\par 18 Perfino i bimbi mi sprezzano; se cerco d'alzarmi, mi scherniscono.
\par 19 Tutti gli amici più stretti m'hanno in orrore, e quelli che amavo mi si son vòlti contro.
\par 20 Le mie ossa stanno attaccate alla mia pelle, alla mia carne, non m'è rimasto che la pelle de' denti.
\par 21 Pietà, pietà di me, voi, miei amici! ché la man di Dio m'ha colpito.
\par 22 Perché perseguitarmi come fa Dio? Perché non siete mai sazi della mia carne?
\par 23 Oh se le mie parole fossero scritte! se fossero consegnate in un libro!
\par 24 se con lo scarpello di ferro e col piombo fossero incise nella roccia per sempre!...
\par 25 Ma io so che il mio Vindice vive, e che alla fine si leverà sulla polvere.
\par 26 E quando, dopo la mia pelle, sarà distrutto questo corpo, senza la mia carne, vedrò Iddio.
\par 27 Io lo vedrò a me favorevole; lo contempleranno gli occhi miei, non quelli d'un altro... il cuore, dalla brama, mi si strugge in seno!
\par 28 Se voi dite: Come lo perseguiteremo, come troveremo in lui la causa prima dei suoi mali?
\par 29 Temete per voi stessi la spada, ché furiosi sono i castighi della spada, affinché sappiate che v'è una giustizia".

\chapter{20}

\par 1 Allora Tsofar di Naama rispose e disse:
\par 2 "Quel che tu dici mi spinge a risponderti e ne suscita in me il fervido impulso.
\par 3 Ho udito rimproveri che mi fanno oltraggio; ma lo spirito mio mi darà una risposta assennata.
\par 4 Non lo sai tu che in ogni tempo, da che l'uomo è stato posto sulla terra,
\par 5 il trionfo de' malvagi è breve, e la gioia degli empi non dura che un istante?
\par 6 Quando la sua altezza giungesse fino al cielo ed il suo capo toccasse le nubi,
\par 7 l'empio perirà per sempre come lo sterco suo; quelli che lo vedevano diranno: 'Dov'è?'
\par 8 Se ne volerà via come un sogno, e non si troverà più; dileguerà come una visione notturna.
\par 9 L'occhio che lo guardava, cesserà di vederlo, e la sua dimora più non lo scorgerà.
\par 10 I suoi figli si raccomanderanno ai poveri, e le sue mani restituiranno la sua ricchezza.
\par 11 Il vigor giovanile che gli riempiva l'ossa giacerà nella polvere con lui.
\par 12 Il male è dolce alla sua bocca, se lo nasconde sotto la lingua,
\par 13 lo risparmia, non lo lascia andar giù, lo trattiene sotto al suo palato:
\par 14 ma il cibo gli si trasforma nelle viscere, e gli diventa in corpo veleno d'aspide.
\par 15 Ha trangugiato ricchezze e le vomiterà; Iddio stesso gliele ricaccerà dal ventre.
\par 16 Ha succhiato veleno d'aspide, la lingua della vipera l'ucciderà.
\par 17 Non godrà più la vista d'acque perenni, né di rivi fluenti di miele e di latte.
\par 18 Renderà il frutto delle sue fatiche, senza poterlo ingoiare. Pari alla sua ricchezza sarà la restituzione che ne dovrà fare, e così non godrà dei suoi beni.
\par 19 Perché ha oppresso e abbandonato il povero, s'è impadronito di case che non avea costruite;
\par 20 perché la sua ingordigia non conobbe requie, egli non salverà nulla di ciò che ha tanto bramato.
\par 21 La sua voracità non risparmiava nulla, perciò il suo benessere non durerà.
\par 22 Nel colmo dell'abbondanza, si troverà in penuria; la mano di chiunque ebbe a soffrir tormenti si leverà contro lui.
\par 23 Quando starà per riempirsi il ventre, ecco Iddio manderà contro a lui l'ardor della sua ira; gliela farà piovere addosso per servirgli il cibo.
\par 24 Se scampa alle armi di ferro, lo trafigge l'arco di rame.
\par 25 Si strappa il dardo, esso gli esce dal corpo, la punta sfolgorante gli vien fuori dal fiele, lo assalgono i terrori della morte.
\par 26 Buio profondo è riservato a' suoi tesori; lo consumerà un fuoco non attizzato dall'uomo, che divorerà quel che resta nella sua tenda.
\par 27 Il cielo rivelerà la sua iniquità, e la terra insorgerà contro di lui.
\par 28 Le rendite della sua casa se n'andranno, portate via nel giorno dell'ira di Dio.
\par 29 Tale la parte che Dio riserba all'empio, tale il retaggio che Dio gli destina".

\chapter{21}

\par 1 Allora Giobbe rispose e disse:
\par 2 "Porgete bene ascolto alle mie parole, e sia questa la consolazione che mi date.
\par 3 Sopportatemi, lasciate ch'io parli, e quando avrò parlato tu mi potrai deridere.
\par 4 Mi lagno io forse d'un uomo? E come farei a non perder la pazienza?
\par 5 Guardatemi, stupite, e mettetevi la mano sulla bocca.
\par 6 Quando ci penso, ne sono smarrito, e la mia carne è presa da raccapriccio.
\par 7 Perché mai vivono gli empi? Perché arrivano alla vecchiaia ed anche crescon di forze?
\par 8 La loro progenie prospera, sotto ai loro sguardi, intorno ad essi, e i lor rampolli fioriscon sotto gli occhi loro.
\par 9 La loro casa è in pace, al sicuro da spaventi, e la verga di Dio non li colpisce.
\par 10 Il loro toro monta e non falla, la loro vacca figlia senz'abortire.
\par 11 Mandan fuori come un gregge i loro piccini, e i loro figliuoli saltano e ballano.
\par 12 Cantano a suon di timpano e di cetra, e si rallegrano al suon della zampogna.
\par 13 Passano felici i loro giorni, poi scendono in un attimo nel soggiorno dei morti.
\par 14 Eppure, diceano a Dio: 'Ritirati da noi! Noi non ci curiamo di conoscer le tue vie!
\par 15 Che è l'Onnipotente perché lo serviamo? che guadagneremo a pregarlo?'
\par 16 Ecco, non hanno essi in mano la loro felicità? (lungi da me il consiglio degli empi!)
\par 17 Quando avvien mai che la lucerna degli empi si spenga, che piombi loro addosso la ruina, e che Dio, nella sua ira, li retribuisca di pene?
\par 18 Quando son essi mai come paglia al vento, come pula portata via dall'uragano?
\par 19 'Iddio', mi dite, 'serba castigo pei figli dell'empio'. Ma punisca lui stesso! che lo senta lui,
\par 20 che vegga con gli occhi propri la sua ruina, e beva egli stesso l'ira dell'Onnipotente!
\par 21 E che importa all'empio della sua famiglia dopo di lui, quando il numero dei suoi mesi è ormai compiuto?
\par 22 S'insegnerà forse a Dio la scienza? a lui che giudica quelli di lassù?
\par 23 L'uno muore in mezzo al suo benessere, quand'è pienamente tranquillo e felice,
\par 24 ha i secchi pieni di latte, e fresco il midollo dell'ossa.
\par 25 L'altro muore con l'amarezza nell'anima, senz'aver mai gustato il bene.
\par 26 Ambedue giacciono ugualmente nella polvere e i vermi li ricoprono.
\par 27 Ah! li conosco i vostri pensieri, e i piani che formate per abbattermi!
\par 28 Voi dite: 'E dov'è la casa del prepotente? dov'è la tenda che albergava gli empi?'
\par 29 Non avete dunque interrogato quelli che hanno viaggiato? Voi non vorrete negare quello che attestano;
\par 30 che, cioè, il malvagio è risparmiato nel dì della ruina, che nel giorno dell'ira egli sfugge.
\par 31 Chi gli rimprovera in faccia la sua condotta? Chi gli rende quel che ha fatto?
\par 32 Egli è portato alla sepoltura con onore, e veglia egli stesso sulla sua tomba.
\par 33 Lievi sono a lui le zolle della valle; dopo, tutta la gente segue le sue orme; e, anche prima, una folla immensa fu come lui.
\par 34 Perché dunque m'offrite consolazioni vane? Delle vostre risposte altro non resta che falsità".

\chapter{22}

\par 1 Allora Elifaz di Teman rispose e disse:
\par 2 "Può l'uomo recar qualche vantaggio a Dio? No; il savio non reca vantaggio che a se stesso.
\par 3 Se sei giusto, ne vien forse qualche diletto all'Onnipotente? Se sei integro nella tua condotta, ne ritrae egli un guadagno?
\par 4 È forse per la paura che ha di te ch'egli ti castiga o vien teco in giudizio?
\par 5 La tua malvagità non è essa grande e le tue iniquità non sono esse infinite?
\par 6 Tu, per un nulla, prendevi pegno da' tuoi fratelli, spogliavi delle loro vesti i mezzo ignudi.
\par 7 Allo stanco non davi a bere dell'acqua, all'affamato rifiutavi del pane.
\par 8 La terra apparteneva al più forte, e l'uomo influente vi piantava la sua dimora.
\par 9 Rimandavi a vuoto le vedove, e le braccia degli orfani eran spezzate.
\par 10 Ecco perché sei circondato di lacci, e spaventato da sùbiti terrori.
\par 11 O non vedi le tenebre che t'avvolgono e la piena d'acque che ti sommerge?
\par 12 Iddio non è egli lassù ne' cieli? Guarda lassù le stelle eccelse, come stanno in alto!
\par 13 E tu dici: 'Iddio che sa? Può egli giudicare attraverso il buio?
\par 14 Fitte nubi lo coprono e nulla vede; egli passeggia sulla volta de' cieli'.
\par 15 Vuoi tu dunque seguir l'antica via per cui camminarono gli uomini iniqui,
\par 16 che furon portati via prima del tempo, e il cui fondamento fu come un torrente che scorre?
\par 17 Essi dicevano a Dio: 'Ritirati da noi!' e chiedevano che mai potesse far per loro l'Onnipotente.
\par 18 Eppure Iddio avea riempito le loro case di beni! Ah lungi da me il consiglio degli empi!
\par 19 I giusti, vedendo la loro ruina, ne gioiscono e l'innocente si fa beffe di loro:
\par 20 'Vedete se non son distrutti gli avversari nostri! la loro abbondanza l'ha divorata il fuoco!'
\par 21 Riconciliati dunque con Dio; avrai pace, e ti sarà resa la prosperità.
\par 22 Ricevi istruzioni dalla sua bocca, e riponi le sue parole nel tuo cuore.
\par 23 Se torni all'Onnipotente, se allontani l'iniquità dalle tue tende, sarai ristabilito.
\par 24 Getta l'oro nella polvere e l'oro d'Ophir tra i ciottoli del fiume
\par 25 e l'Onnipotente sarà il tuo oro, egli ti sarà come l'argento acquistato con fatica.
\par 26 Allora farai dell'Onnipotente la tua delizia, e alzerai la faccia verso Dio.
\par 27 Lo pregherai, egli t'esaudirà, e tu scioglierai i voti che avrai fatto.
\par 28 Quello che imprenderai, ti riuscirà; sul tuo cammino risplenderà la luce.
\par 29 Se ti abbassano, tu dirai: 'In alto!' e Dio soccorrerà chi ha gli occhi a terra;
\par 30 libererà anche chi non è innocente, ei sarà salvo per la purità delle tue mani".

\chapter{23}

\par 1 Allora Giobbe rispose e disse:
\par 2 "Anche oggi il mio lamento è una rivolta, per quanto io cerchi di comprimere il mio gemito.
\par 3 Oh sapessi dove trovarlo! potessi arrivare fino al suo trono!
\par 4 Esporrei la mia causa dinanzi a lui, riempirei d'argomenti la mia bocca.
\par 5 Saprei quel che mi risponderebbe, e capirei quello che avrebbe da dirmi.
\par 6 Contenderebbe egli meco con la sua gran potenza? No! invece, mi presterebbe attenzione.
\par 7 Là sarebbe un uomo retto a discutere con lui, e sarei dal mio giudice assolto per sempre.
\par 8 Ma, ecco, se vo ad oriente, egli non c'è; se ad occidente, non lo trovo;
\par 9 se a settentrione, quando vi opera, io non lo veggo; si nasconde egli nel mezzodì, io non lo scorgo.
\par 10 Ma la via ch'io batto ei la sa; se mi mettesse alla prova, ne uscirei come l'oro.
\par 11 Il mio piede ha seguito fedelmente le sue orme, mi son tenuto sulla sua via senza deviare;
\par 12 non mi sono scostato dai comandamenti delle sue labbra, ho riposto nel mio seno le parole della sua bocca.
\par 13 Ma la sua decisione è una; chi lo farà mutare? Quello ch'ei desidera, lo fa;
\par 14 egli seguirà quel che di me ha decretato; e di cose come queste ne ha molte in mente.
\par 15 Perciò nel suo cospetto io sono atterrito; quando ci penso, ho paura di lui.
\par 16 Iddio m'ha tolto il coraggio, l'Onnipotente mi ha spaventato.
\par 17 Questo mi annienta: non le tenebre, non la fitta oscurità che mi ricopre.

\chapter{24}

\par 1 Perché non sono dall'Onnipotente fissati dei tempi in cui renda la giustizia? Perché quelli che lo conoscono non veggono quei giorni?
\par 2 Gli empi spostano i termini, rapiscono greggi e li menano a pascere;
\par 3 portano via l'asino dell'orfano, prendono in pegno il bove della vedova;
\par 4 mandano via dalla strada i bisognosi, i poveri del paese si nascondono tutti insieme.
\par 5 Eccoli, che come onàgri del deserto escono al loro lavoro in cerca di cibo; solo il deserto dà pane a' lor figliuoli.
\par 6 Raccolgono nei campi la loro pastura, raspollano nella vigna dell'empio;
\par 7 passan la notte ignudi, senza vestito, senza una coperta che li ripari dal freddo.
\par 8 Bagnati dagli acquazzoni di montagna, per mancanza di rifugio, si stringono alle rocce.
\par 9 Ce n'è di quelli che strappano dalla mammella l'orfano, che prendono pegni dai poveri!
\par 10 E questi se ne vanno, ignudi, senza vestiti; hanno fame, e portano i covoni.
\par 11 Fanno l'olio nel recinto dell'empio; calcan l'uva nel tino e patiscon la sete.
\par 12 Sale dalle città il gemito de' morenti; l'anima de' feriti implora aiuto, e Dio non si cura di codeste infamie!
\par 13 Ve ne son di quelli che si ribellano alla luce, non ne conoscono le vie, non ne battono i sentieri.
\par 14 L'assassino si leva sul far del giorno, e ammazza il meschino e il povero; la notte fa il ladro.
\par 15 L'occhio dell'adultero spia il crepuscolo, dicendo: 'Nessuno mi vedrà!' e si copre d'un velo la faccia.
\par 16 I ladri, di notte, sfondano le case; di giorno, si tengono rinchiusi; non conoscono la luce.
\par 17 Il mattino è per essi come ombra di morte: appena lo scorgono provano i terrori del buio.
\par 18 Voi dite: 'L'empio è una festuca sulla faccia dell'acque; la sua parte sulla terra è maledetta: non prenderà più la via delle vigne.
\par 19 Come la siccità e il calore assorbon le acque della neve, così il soggiorno de' morti inghiottisce chi ha peccato.
\par 20 Il seno che lo portò, l'oblia; i vermi ne fanno il loro pasto delizioso, nessuno più lo ricorda.
\par 21 L'iniquo sarà troncato come un albero: ei che divorava la sterile, priva di figli, e non faceva del bene alla vedova!'
\par 22 Invece, Iddio con la sua forza prolunga i giorni dei prepotenti, i quali risorgono, quand'ormai disperavan della vita.
\par 23 Dà loro sicurezza, fiducia, e i suoi occhi vegliano sul loro cammino.
\par 24 Salgono in alto, poi scompaiono ad un tratto; cadono, son mietuti come gli altri mortali; son falciati come le spighe del grano maturo.
\par 25 Se così non è, chi mi smentirà, chi annienterà il mio dire?"

\chapter{25}

\par 1 Allora Bildad di Suach rispose e disse:
\par 2 "A Dio appartiene il dominio e il terrore: egli fa regnare la pace ne' suoi luoghi altissimi.
\par 3 Le sue legioni si posson forse contare? Su chi non si leva la sua luce?
\par 4 Come può dunque l'uomo esser giusto dinanzi a Dio? Come può esser puro il nato dalla donna?
\par 5 Ecco, la luna stessa manca di chiarore, e le stelle non son pure agli occhi di lui;
\par 6 quanto meno l'uomo, ch'è un verme, il figliuol d'uomo ch'è un vermicciuolo!"

\chapter{26}

\par 1 Allora Giobbe rispose e disse:
\par 2 "Come hai bene aiutato il debole! Come hai sorretto il braccio senza forza!
\par 3 Come hai ben consigliato chi è privo di sapienza! E che abbondanza di sapere tu gli hai comunicato!
\par 4 Ma a chi ti credi di aver parlato? E di chi è lo spirito che parla per mezzo tuo?
\par 5 Dinanzi a Dio tremano le ombre disotto alle acque ed ai loro abitanti.
\par 6 Dinanzi a lui il soggiorno dei morti è nudo, l'abisso è senza velo.
\par 7 Egli distende il settentrione sul vuoto, sospende la terra sul nulla.
\par 8 Rinchiude le acque nelle sue nubi, e le nubi non scoppiano per il peso.
\par 9 Nasconde l'aspetto del suo trono, vi distende sopra le sue nuvole.
\par 10 Ha tracciato un cerchio sulla faccia dell'acque, là dove la luce confina colle tenebre.
\par 11 Le colonne del cielo sono scosse, e tremano alla sua minaccia.
\par 12 Con la sua forza egli solleva il mare, con la sua intelligenza ne abbatte l'orgoglio.
\par 13 Al suo soffio il cielo torna sereno, la sua mano trafigge il drago fuggente.
\par 14 Ecco, questi non son che gli estremi lembi dell'azione sua. Non ce ne giunge all'orecchio che un breve sussurro; ma il tuono delle sue potenti opere chi lo può intendere?"

\chapter{27}

\par 1 Giobbe riprese il suo discorso e disse:
\par 2 "Come vive Iddio che mi nega giustizia, come vive l'Onnipotente che mi amareggia l'anima,
\par 3 finché avrò fiato e il soffio di Dio sarà nelle mie nari,
\par 4 le mie labbra, no, non diranno nulla d'ingiusto, e la mia lingua non proferirà falsità.
\par 5 Lungi da me l'idea di darvi ragione! Fino all'ultimo respiro non mi lascerò togliere la mia integrità.
\par 6 Ho preso a difendere la mia giustizia e non cederò; il cuore non mi rimprovera uno solo de' miei giorni.
\par 7 Sia trattato da malvagio il mio nemico e da perverso chi si leva contro di me!
\par 8 Quale speranza rimane mai all'empio quando Iddio gli toglie, gli rapisce l'anima?
\par 9 Iddio presterà egli orecchio al grido di lui, quando gli verrà sopra la distretta?
\par 10 Potrà egli prendere il suo diletto nell'Onnipotente? invocare Iddio in ogni tempo?
\par 11 Io vi mostrerò il modo d'agire di Dio, non vi nasconderò i disegni dell'Onnipotente.
\par 12 Ma queste cose voi tutti le avete osservate e perché dunque vi perdete in vani discorsi?
\par 13 Ecco la parte che Dio riserba all'empio, l'eredità che l'uomo violento riceve dall'Onnipotente.
\par 14 Se ha figli in gran numero son per la spada; la sua progenie non avrà pane da saziarsi.
\par 15 I superstiti son sepolti dalla morte, e le vedove loro non li piangono.
\par 16 Se accumula l'argento come polvere, se ammucchia vestiti come fango;
\par 17 li ammucchia, sì, ma se ne vestirà il giusto, e l'argento l'avrà come sua parte l'innocente.
\par 18 La casa ch'ei si edifica è come quella della tignuola, come il capanno che fa il guardiano della vigna.
\par 19 Va a letto ricco, ma per l'ultima volta; apre gli occhi e non è più.
\par 20 Terrori lo sorprendono come acque; nel cuor della notte lo rapisce un uragano.
\par 21 Il vento d'oriente lo porta via, ed egli se ne va; lo spazza in un turbine al luogo suo.
\par 22 Iddio gli scaglia addosso i suoi dardi, senza pietà, per quanto egli tenti di scampare a' suoi colpi.
\par 23 La gente batte le mani quando cade, e fischia dietro a lui quando lascia il luogo dove stava.

\chapter{28}

\par 1 Ha una miniera l'argento, e l'oro un luogo dove lo si affina.
\par 2 Il ferro si cava dal suolo, e la pietra fusa dà il rame.
\par 3 L'uomo ha posto fine alle tenebre, egli esplora i più profondi recessi, per trovar le pietre che son nel buio, nell'ombra di morte.
\par 4 Scava un pozzo lontan dall'abitato; il piede più non serve a quei che vi lavorano; son sospesi, oscillano lungi dai mortali.
\par 5 Dalla terra esce il pane, ma, nelle sue viscere, è sconvolta come dal fuoco.
\par 6 Le sue rocce son la dimora dello zaffiro, e vi si trova della polvere d'oro.
\par 7 L'uccello di rapina non conosce il sentiero che vi mena, né l'ha mai scorto l'occhio del falco.
\par 8 Le fiere superbe non vi hanno messo piede, e il leone non v'è passato mai.
\par 9 L'uomo stende la mano sul granito, rovescia dalle radici le montagne.
\par 10 Pratica trafori per entro le rocce, e l'occhio suo scorge quanto v'è di prezioso.
\par 11 Infrena le acque perché non gemano, e le cose nascoste trae fuori alla luce.
\par 12 Ma la Sapienza, dove trovarla? E dov'è il luogo della Intelligenza?
\par 13 L'uomo non ne sa la via, non la si trova sulla terra de' viventi.
\par 14 L'abisso dice: 'Non è in me'; il mare dice: 'Non sta da me'.
\par 15 Non la si ottiene in cambio d'oro, né la si compra a peso d'argento.
\par 16 Non la si acquista con l'oro di Ofir, con l'ònice prezioso o con lo zaffiro.
\par 17 L'oro ed il vetro non reggono al suo confronto, non la si dà in cambio di vasi d'oro fino.
\par 18 Non si parli di corallo, di cristallo; la Sapienza val più delle perle.
\par 19 Il topazio d'Etiopia non può starle a fronte, l'oro puro non ne bilancia il valore.
\par 20 Donde vien dunque la Sapienza? E dov'è il luogo della Intelligenza?
\par 21 Essa è nascosta agli occhi d'ogni vivente, è celata agli uccelli del cielo.
\par 22 L'abisso e la morte dicono: 'Ne abbiamo avuto qualche sentore'.
\par 23 Dio solo conosce la via che vi mena, egli solo sa il luogo dove dimora,
\par 24 perché il suo sguardo giunge sino alle estremità della terra, perch'egli vede tutto quel ch'è sotto i cieli.
\par 25 Quando regolò il peso del vento e fissò la misura dell'acque,
\par 26 quando dette una legge alla pioggia e tracciò la strada al lampo dei tuoni,
\par 27 allora la vide e la rivelò, la stabilì ed anche l'investigò.
\par 28 E disse all'uomo: 'Ecco: temere il Signore: questa è la Sapienza, e fuggire il male è l'Intelligenza'."

\chapter{29}

\par 1 Giobbe riprese il suo discorso e disse:
\par 2 "Oh foss'io come ne' mesi d'una volta, come ne' giorni in cui Dio mi proteggeva,
\par 3 quando la sua lampada mi risplendeva sul capo, e alla sua luce io camminavo nelle tenebre!
\par 4 Oh fossi com'ero a' giorni della mia maturità, quando Iddio vegliava amico sulla mia tenda,
\par 5 quando l'Onnipotente stava ancora meco, e avevo i miei figliuoli dintorno;
\par 6 quando mi lavavo i piedi nel latte e dalla roccia mi fluivano ruscelli d'olio!
\par 7 Allorché uscivo per andare alla porta della città e mi facevo preparare il seggio sulla piazza,
\par 8 i giovani, al vedermi, si ritiravano, i vecchi s'alzavano e rimanevano in piedi;
\par 9 i maggiorenti cessavan di parlare e si mettevan la mano sulla bocca;
\par 10 la voce dei capi diventava muta, la lingua s'attaccava al loro palato.
\par 11 L'orecchio che mi udiva, mi diceva beato; l'occhio che mi vedeva mi rendea testimonianza,
\par 12 perché salvavo il misero che gridava aiuto, e l'orfano che non aveva chi lo soccorresse.
\par 13 Scendea su me la benedizione di chi stava per perire, e facevo esultare il cuor della vedova.
\par 14 La giustizia era il mio vestimento ed io il suo; la probità era come il mio mantello e il mio turbante.
\par 15 Ero l'occhio del cieco, il piede dello zoppo;
\par 16 ero il padre de' poveri, e studiavo a fondo la causa dello sconosciuto.
\par 17 Spezzavo la ganascia all'iniquo, e gli facevo lasciar la preda che avea fra i denti.
\par 18 E dicevo: 'Morrò nel mio nido, e moltiplicherò i miei giorni come la rena;
\par 19 le mie radici si stenderanno verso l'acque, la rugiada passerà la notte sui miei rami;
\par 20 la mia gloria sempre si rinnoverà, e l'arco rinverdirà nella mia mano'.
\par 21 Gli astanti m'ascoltavano pieni d'aspettazione, si tacevan per udire il mio parere.
\par 22 Quand'avevo parlato, non replicavano; la mia parola scendeva su loro come una rugiada.
\par 23 E m'aspettavan come s'aspetta la pioggia; aprivan larga la bocca come a un acquazzone di primavera.
\par 24 Io sorridevo loro quand'erano sfiduciati; e non potevano oscurar la luce del mio volto.
\par 25 Quando andavo da loro, mi sedevo come capo, ed ero come un re fra le sue schiere, come un consolatore in mezzo agli afflitti.

\chapter{30}

\par 1 E ora servo di zimbello a dei più giovani di me, i cui padri non mi sarei degnato di mettere fra i cani del mio gregge!
\par 2 E a che m'avrebbe servito la forza delle lor mani? Gente incapace a raggiungere l'età matura,
\par 3 smunta dalla miseria e dalla fame, ridotta a brucare il deserto, la terra da tempo nuda e desolata,
\par 4 strappando erba salsa presso ai cespugli, ed avendo per pane radici di ginestra.
\par 5 Sono scacciati di mezzo agli uomini, grida lor dietro la gente come dietro al ladro,
\par 6 abitano in burroni orrendi, nelle caverne della terra e fra le rocce;
\par 7 ragliano fra i cespugli, si sdraiano alla rinfusa sotto i rovi;
\par 8 gente da nulla, razza senza nome, cacciata via dal paese a bastonate.
\par 9 E ora io sono il tema delle loro canzoni, il soggetto dei loro discorsi.
\par 10 Mi aborrono, mi fuggono, non si trattengono dallo sputarmi in faccia.
\par 11 Non han più ritegno, m'umiliano, rompono ogni freno in mia presenza.
\par 12 Questa genìa si leva alla mia destra, m'incalzano, e si appianano le vie contro di me per distruggermi.
\par 13 Hanno sovvertito il mio cammino, lavorano alla mia ruina, essi che nessuno vorrebbe soccorrere!
\par 14 S'avanzano come per un'ampia breccia, si precipitano innanzi in mezzo alle ruine.
\par 15 Terrori mi si rovesciano addosso; l'onor mio è portato via come dal vento, è passata come una nube la mia felicità.
\par 16 E ora l'anima mia si strugge in me, m'hanno colto i giorni dell'afflizione.
\par 17 La notte mi trafigge, mi stacca l'ossa, e i dolori che mi rodono non hanno posa.
\par 18 Per la gran violenza del mio male la mia veste si sforma, mi si serra addosso come la tunica.
\par 19 Iddio m'ha gettato nel fango, e rassomiglio alla polvere e alla cenere.
\par 20 Io grido a te, e tu non mi rispondi; ti sto dinanzi, e tu mi stai a considerare!
\par 21 Ti sei mutato in nemico crudele verso di me; mi perseguiti con la potenza della tua mano.
\par 22 Mi levi per aria, mi fai portar via dal vento, e mi annienti nella tempesta.
\par 23 Giacché, lo so, tu mi meni alla morte, alla casa di convegno di tutti i viventi.
\par 24 Ma chi sta per perire non protende la mano? e nell'angoscia sua non grida al soccorso?
\par 25 Non piangevo io forse per chi era nell'avversità? l'anima mia non era ella angustiata per il povero?
\par 26 Speravo il bene, ed è venuto il male; aspettavo la luce, ed è venuta l'oscurità!
\par 27 Le mie viscere bollono e non hanno requie, son venuti per me giorni d'afflizione.
\par 28 Me ne vo tutto annerito, ma non dal sole; mi levo in mezzo alla raunanza, e grido aiuto;
\par 29 son diventato fratello degli sciacalli, compagno degli struzzi.
\par 30 La mia pelle è nera, e cade a pezzi; le mie ossa son calcinate dall'arsura.
\par 31 La mia cetra non dà più che accenti di lutto, e la mia zampogna voce di pianto.

\chapter{31}

\par 1 Io avevo stretto un patto con gli occhi miei; come dunque avrei fissati gli sguardi sopra una vergine?
\par 2 Che parte mi avrebbe assegnata Iddio dall'alto e quale eredità m'avrebbe data l'Onnipotente dai luoghi eccelsi?
\par 3 La sventura non è ella per il perverso e le sciagure per quelli che fanno il male?
\par 4 Iddio non vede egli le mie vie? non conta tutti i miei passi?
\par 5 Se ho camminato insieme alla menzogna, se il piede mio s'è affrettato dietro alla frode
\par 6 (Iddio mi pesi con la bilancia giusta e riconoscerà la mia integrità)
\par 7 se i miei passi sono usciti dalla retta via, se il mio cuore è ito dietro ai miei occhi, se qualche sozzura mi s'è attaccata alle mani,
\par 8 ch'io semini e un altro mangi, e quel ch'è cresciuto nei miei campi sia sradicato!
\par 9 Se il mio cuore s'è lasciato sedurre per amor d'una donna, se ho spiato la porta del mio prossimo,
\par 10 che mia moglie giri la macina ad un altro, e che altri abusino di lei!
\par 11 Poiché quella è una scelleratezza, un misfatto punito dai giudici,
\par 12 un fuoco che consuma fino a perdizione, e che avrebbe distrutto fin dalle radici ogni mia fortuna.
\par 13 Se ho disconosciuto il diritto del mio servo e della mia serva, quand'eran meco in lite,
\par 14 che farei quando Iddio si levasse per giudicarmi, e che risponderei quando mi esaminasse?
\par 15 Chi fece me nel seno di mia madre non fece anche lui? non ci ha formati nel seno materno uno stesso Iddio?
\par 16 Se ho rifiutato ai poveri quel che desideravano, se ho fatto languire gli occhi della vedova,
\par 17 se ho mangiato da solo il mio pezzo di pane senza che l'orfano ne mangiasse la sua parte,
\par 18 io che fin da giovane l'ho allevato come un padre, io che fin dal seno di mia madre sono stato guida alla vedova,
\par 19 se ho visto uno perire per mancanza di vesti o il povero senza una coperta,
\par 20 se non m'hanno benedetto i suoi fianchi, ed egli non s'è riscaldato colla lana dei miei agnelli,
\par 21 se ho levato la mano contro l'orfano perché mi sapevo sostenuto alla porta...
\par 22 che la mia spalla si stacchi dalla sua giuntura, il mio braccio si spezzi e cada!
\par 23 E invero mi spaventava il castigo di Dio, ed ero trattenuto dalla maestà di lui.
\par 24 Se ho riposto la mia fiducia nell'oro, se all'oro fino ho detto: 'Tu sei la mia speranza',
\par 25 se mi son rallegrato che le mie ricchezze fosser grandi e la mia mano avesse molto accumulato,
\par 26 se, contemplando il sole che raggiava e la luna che procedeva lucente nel suo corso,
\par 27 il mio cuore, in segreto, s'è lasciato sedurre e la mia bocca ha posato un bacio sulla mano
\par 28 (misfatto anche questo punito dai giudici ché avrei difatti rinnegato l'Iddio ch'è di sopra),
\par 29 se mi son rallegrato della sciagura del mio nemico ed ho esultato quando gli ha incòlto sventura
\par 30 (io, che non ho permesso alle mie labbra di peccare chiedendo la sua morte con imprecazione),
\par 31 se la gente della mia tenda non ha detto: 'Chi è che non si sia saziato della carne delle sue bestie?'
\par 32 (lo straniero non passava la notte fuori; le mie porte erano aperte al viandante),
\par 33 se, come fan gli uomini, ho coperto i miei falli celando nel petto la mia iniquità,
\par 34 perché avevo paura della folla e dello sprezzo delle famiglie al punto da starmene queto e non uscir di casa...
\par 35 Oh, avessi pure chi m'ascoltasse!... ecco qua la mia firma! l'Onnipotente mi risponda! Scriva l'avversario mio la sua querela,
\par 36 ed io la porterò attaccata alla mia spalla, me la cingerò come un diadema!
\par 37 Gli renderò conto di tutt'i miei passi, a lui m'appresserò come un principe!
\par 38 Se la mia terra mi grida contro, se tutti i suoi solchi piangono,
\par 39 se ne ho mangiato il frutto senza pagarla, se ho fatto sospirare chi la coltivava,
\par 40 che invece di grano mi nascano spine, invece d'orzo mi crescano zizzanie!" Qui finiscono i discorsi di Giobbe.

\chapter{32}

\par 1 Quei tre uomini cessarono di rispondere a Giobbe perché egli si credeva giusto.
\par 2 Allora l'ira di Elihu, figliuolo di Barakeel il Buzita, della tribù di Ram, s'accese:
\par 3 s'accese contro Giobbe, perché riteneva giusto se stesso anziché Dio; s'accese anche contro i tre amici di lui perché non avean trovato che rispondere, sebbene condannassero Giobbe.
\par 4 Ora, siccome quelli erano più attempati di lui,
\par 5 Elihu aveva aspettato a parlare a Giobbe; ma quando vide che dalla bocca di quei tre uomini non usciva più risposta, s'accese d'ira.
\par 6 Ed Elihu, figliuolo di Barakeel il Buzita, rispose e disse: "Io son giovine d'età e voi siete vecchi; perciò mi son tenuto indietro e non ho ardito esporvi il mio pensiero.
\par 7 Dicevo: 'Parleranno i giorni, e il gran numero degli anni insegnerà la sapienza'.
\par 8 Ma, nell'uomo, quel che lo rende intelligente è lo spirito, è il soffio dell'Onnipotente.
\par 9 Non quelli di lunga età sono sapienti, né i vecchi son quelli che comprendono il giusto.
\par 10 Perciò dico: 'Ascoltatemi; vi esporrò anch'io il mio pensiero'.
\par 11 Ecco, ho aspettato i vostri discorsi, ho ascoltato i vostri argomenti, mentre andavate cercando altre parole.
\par 12 V'ho seguito attentamente, ed ecco, nessun di voi ha convinto Giobbe, nessuno ha risposto alle sue parole.
\par 13 Non avete dunque ragione di dire: 'Abbiam trovato la sapienza! Dio soltanto lo farà cedere; non l'uomo!'
\par 14 Egli non ha diretto i suoi discorsi contro a me, ed io non gli risponderò colle vostre parole.
\par 15 Eccoli sconcertati! non rispondon più, non trovan più parole.
\par 16 Ed ho aspettato che non parlassero più, che restassero e non rispondessero più.
\par 17 Ma ora risponderò anch'io per mio conto, esporrò anch'io il mio pensiero!
\par 18 Perché son pieno di parole, e lo spirito ch'è dentro di me mi stimola.
\par 19 Ecco, il mio seno è come vin rinchiuso, è simile ad otri pieni di vin nuovo, che stanno per scoppiare.
\par 20 Parlerò dunque e mi solleverò, aprirò le labbra e risponderò!
\par 21 E lasciate ch'io parli senza riguardi personali, senza adulare alcuno;
\par 22 poiché adulare io non so; se lo facessi, il mio Fattore tosto mi torrebbe di mezzo.

\chapter{33}

\par 1 Ma pure, ascolta, o Giobbe, il mio dire, porgi orecchio a tutte le mie parole!
\par 2 Ecco, apro la bocca, la lingua parla sotto il mio palato.
\par 3 Nelle mie parole è la rettitudine del mio cuore; e le mie labbra diran sinceramente quello che so.
\par 4 Lo spirito di Dio mi ha creato, e il soffio dell'Onnipotente mi dà la vita.
\par 5 Se puoi, rispondimi; prepara le tue ragioni, fatti avanti!
\par 6 Ecco, io sono uguale a te davanti a Dio; anch'io, fui tratto dall'argilla.
\par 7 Spavento di me non potrà quindi sgomentarti, e il peso della mia autorità non ti potrà schiacciare.
\par 8 Davanti a me tu dunque hai detto (e ho bene udito il suono delle tue parole):
\par 9 'Io sono puro, senza peccato; sono innocente, non c'è iniquità in me;
\par 10 ma Dio trova contro me degli appigli ostili, mi tiene per suo nemico;
\par 11 mi mette i piedi nei ceppi, spia tutti i miei movimenti'.
\par 12 E io ti rispondo: In questo non hai ragione; giacché Dio è più grande dell'uomo.
\par 13 Perché contendi con lui? poich'egli non rende conto d'alcuno dei suoi atti.
\par 14 Iddio parla, bensì, una volta ed anche due, ma l'uomo non ci bada;
\par 15 parla per via di sogni, di visioni notturne, quando un sonno profondo cade sui mortali, quando sui loro letti essi giacciono assopiti;
\par 16 allora egli apre i loro orecchi e dà loro in segreto degli ammonimenti,
\par 17 per distoglier l'uomo dal suo modo d'agire e tener lungi da lui la superbia;
\par 18 per salvargli l'anima dalla fossa, la vita dal dardo mortale.
\par 19 L'uomo è anche ammonito sul suo letto, dal dolore, dall'agitazione incessante delle sue ossa;
\par 20 quand'egli ha in avversione il pane, e l'anima sua schifa i cibi più squisiti;
\par 21 la carne gli si consuma, e sparisce, mentre le ossa, prima invisibili, gli escon fuori,
\par 22 l'anima sua si avvicina alla fossa, e la sua vita a quelli che danno la morte.
\par 23 Ma se, presso a lui, v'è un angelo, un interprete, uno solo fra i mille, che mostri all'uomo il suo dovere,
\par 24 Iddio ha pietà di lui e dice: 'Risparmialo, che non scenda nella fossa! Ho trovato il suo riscatto'.
\par 25 Allora la sua carne divien fresca più di quella d'un bimbo; egli torna ai giorni della sua giovinezza;
\par 26 implora Dio, e Dio gli è propizio; gli dà di contemplare il suo volto con giubilo, e lo considera di nuovo come giusto.
\par 27 Ed egli va cantando fra la gente e dice: 'Avevo peccato, pervertito la giustizia, e non sono stato punito come meritavo.
\par 28 Iddio ha riscattato l'anima mia, onde non scendesse nella fossa e la mia vita si schiude alla luce!'
\par 29 Ecco, tutto questo Iddio lo fa due, tre volte, all'uomo,
\par 30 per ritrarre l'anima di lui dalla fossa, perché su di lei splenda la luce della vita.
\par 31 Sta' attento, Giobbe, dammi ascolto; taci, ed io parlerò.
\par 32 Se hai qualcosa da dire, rispondi, parla, ché io vorrei poterti dar ragione.
\par 33 Se no, tu dammi ascolto, taci, e t'insegnerò la saviezza".

\chapter{34}

\par 1 Elihu riprese a parlare e disse:
\par 2 "O voi savi, ascoltate le mie parole! Voi che siete intelligenti, prestatemi orecchio!
\par 3 Poiché l'orecchio giudica dei discorsi, come il palato assapora le vivande.
\par 4 Scegliamo quello ch'è giusto, riconosciamo fra noi quello ch'è buono.
\par 5 Giobbe ha detto: 'Sono giusto, ma Dio mi nega giustizia;
\par 6 ho ragione, e passo da bugiardo; la mia ferita è incurabile, e sono senza peccato'.
\par 7 Dov'è l'uomo che al par di Giobbe tracanni gli empi scherni come l'acqua,
\par 8 cammini in compagnia de' malfattori, e vada assieme con gli scellerati?
\par 9 Poiché ha detto: 'Non giova nulla all'uomo l'avere il suo diletto in Dio'.
\par 10 Ascoltatemi dunque, o uomini di senno! Lungi da Dio il male, lungi dall'Onnipotente l'iniquità!
\par 11 Poich'egli rende all'uomo secondo le sue opere, e fa trovare a ognuno il salario della sua condotta.
\par 12 No, di certo Iddio non commette ingiustizie! l'Onnipotente non perverte il diritto.
\par 13 Chi gli ha dato il governo della terra? Chi ha affidato l'universo alla sua cura?
\par 14 S'ei non ponesse mente che a se stesso, se ritirasse a sé il suo spirito e il suo soffio,
\par 15 ogni carne perirebbe d'un tratto, e l'uomo ritornerebbe in polvere.
\par 16 Se tu se' intelligente, ascolta questo, porgi orecchio alla voce delle mie parole.
\par 17 Uno che odiasse la giustizia potrebbe governare? E osi tu condannare il Giusto, il Potente,
\par 18 che chiama i re 'uomini da nulla' e i principi 'scellerati',
\par 19 che non porta rispetto all'apparenza de' grandi, che non considera il ricco più del povero, perché son tutti opera delle sue mani?
\par 20 In un attimo, essi muoiono; nel cuor della notte, la gente del popolo è scossa e scompare, i potenti son portati via, senza man d'uomo.
\par 21 Perché Iddio tien gli occhi aperti sulle vie de' mortali, e vede tutti i lor passi.
\par 22 Non vi son tenebre, non v'è ombra di morte, ove possa nascondersi chi opera iniquamente.
\par 23 Dio non ha bisogno d'osservare a lungo un uomo per trarlo davanti a lui in giudizio.
\par 24 Egli fiacca i potenti, senza inchiesta; e ne stabilisce altri al loro posto;
\par 25 poich'egli conosce le loro azioni; li abbatte nella notte, e son fiaccati;
\par 26 li colpisce come dei malvagi, in presenza di tutti,
\par 27 perché si sono sviati da lui e non hanno posto mente ad alcuna delle sue vie;
\par 28 han fatto salire a lui il gemito del povero, ed egli ha dato ascolto al gemito degli infelici.
\par 29 Quando Iddio dà requie chi lo condannerà? Chi potrà contemplarlo quando nasconde il suo volto a una nazione ovvero a un individuo,
\par 30 per impedire all'empio di regnare, per allontanar dal popolo le insidie?
\par 31 Quell'empio ha egli detto a Dio: 'Io porto la mia pena, non farò più il male,
\par 32 mostrami tu quel che non so vedere; se ho agito perversamente, non lo farò più'?
\par 33 Dovrà forse Iddio render la giustizia a modo tuo, che tu lo critichi? Ti dirà forse: 'Scegli tu, non io, quello che sai, dillo'?
\par 34 La gente assennata e ogni uomo savio che m'ascolta, mi diranno:
\par 35 'Giobbe parla senza giudizio, le sue parole sono senza intendimento'.
\par 36 Ebbene, sia Giobbe provato sino alla fine! poiché le sue risposte son quelle degli iniqui,
\par 37 poiché aggiunge al peccato suo la ribellione, batte le mani in mezzo a noi, e moltiplica le sue parole contro Dio".

\chapter{35}

\par 1 Poi Elihu riprese il discorso e disse:
\par 2 "Credi tu d'aver ragione quando dici: 'Dio non si cura della mia giustizia'?
\par 3 Infatti hai detto: 'Che mi giova? che guadagno io di più a non peccare?'
\par 4 Io ti darò la risposta: a te ed agli amici tuoi.
\par 5 Considera i cieli, e vedi! guarda le nuvole, come sono più in alto di te!
\par 6 Se pecchi, che torto gli fai? Se moltiplichi i tuoi misfatti, che danno gli rechi?
\par 7 Se sei giusto, che gli dai? Che ricev'egli dalla tua mano?
\par 8 La tua malvagità non nuoce che al tuo simile, e la tua giustizia non giova che ai figli degli uomini.
\par 9 Si grida per le molte oppressioni, si levano lamenti per la violenza dei grandi;
\par 10 ma nessuno dice: 'Dov'è Dio, il mio creatore, che nella notte concede canti di gioia,
\par 11 che ci fa più intelligenti delle bestie de' campi e più savi degli uccelli del cielo?'
\par 12 Si grida, sì, ma egli non risponde, a motivo della superbia dei malvagi.
\par 13 Certo, Dio non dà ascolto a lamenti vani; l'Onnipotente non ne fa nessun caso.
\par 14 E tu, quando dici che non lo scorgi, la causa tua gli sta dinanzi; sappilo aspettare!
\par 15 Ma ora, perché la sua ira non punisce, perch'egli non prende rigorosa conoscenza delle trasgressioni,
\par 16 Giobbe apre vanamente le labbra e accumula parole senza conoscimento".

\chapter{36}

\par 1 Poi Elihu seguitando disse:
\par 2 "Aspetta un po', io t'istruirò; perché c'è da dire ancora a pro di Dio.
\par 3 Io trarrò la mia scienza da lontano e renderò giustizia a colui che m'ha fatto.
\par 4 Per certo, le mie parole non son bugiarde; ti sta dinanzi un uomo dotato di perfetta scienza.
\par 5 Ecco, Iddio è potente, ma non disdegna nessuno; è potente per la forza dell'intelletto suo.
\par 6 Ei non lascia viver l'empio, e fa ragione ai miseri.
\par 7 Non storna lo sguardo suo dai giusti, ma li pone coi re sul trono, ve li fa sedere per sempre, e così li esalta.
\par 8 Se gli uomini son talora stretti da catene, se son presi nei legami dell'afflizione,
\par 9 Dio fa lor conoscere la lor condotta, le loro trasgressioni, giacché si sono insuperbiti;
\par 10 egli apre così i loro orecchi a' suoi ammonimenti, e li esorta ad abbandonare il male.
\par 11 Se l'ascoltano, se si sottomettono, finiscono i loro giorni nel benessere, e gli anni loro nella gioia;
\par 12 ma, se non l'ascoltano, periscon trafitti da' suoi dardi, muoiono per mancanza d'intendimento.
\par 13 Gli empi di cuore s'abbandonano alla collera, non implorano Iddio quand'ei gl'incatena;
\par 14 così muoiono nel fior degli anni, e la lor vita finisce come quella dei dissoluti;
\par 15 ma Dio libera l'afflitto mediante l'afflizione, e gli apre gli orecchi mediante la sventura.
\par 16 Te pure ei vuol trarre dalle fauci della distretta, al largo, dove non è più angustia, e coprir la tua mensa tranquilla di cibi succulenti.
\par 17 Ma, se giudichi le vie di Dio come fan gli empi, il giudizio e la sentenza di lui ti piomberanno addosso.
\par 18 Bada che la collera non ti trasporti alla bestemmia, e la grandezza del riscatto non t'induca a fuorviare!
\par 19 Farebbe egli caso delle tue ricchezze? Non han valore per lui, né l'oro, né tutta la possanza dell'opulenza.
\par 20 Non anelare a quella notte che porta via i popoli dal luogo loro.
\par 21 Guàrdati bene dal volgerti all'iniquità, tu che sembri preferirla all'afflizione!
\par 22 Vedi, Iddio è eccelso nella sua potenza; chi può insegnare come lui?
\par 23 Chi gli prescrive la via da seguire? Chi osa dirgli: 'Tu hai fatto male?'
\par 24 Pensa piuttosto a magnificar le sue opere; gli uomini le celebrano nei loro canti,
\par 25 tutti le ammirano, il mortale le contempla da lungi.
\par 26 Sì, Iddio è grande e noi non lo possiam conoscere; incalcolabile è il numero degli anni suoi.
\par 27 Egli attrae a sé le gocciole dell'acqua; dai vapori ch'egli ha formato stilla la pioggia.
\par 28 Le nubi la spandono, la rovesciano sulla folla dei mortali.
\par 29 E chi può capire lo spiegamento delle nubi, i fragori che scoppiano nel suo padiglione?
\par 30 Ecco, ora egli spiega intorno a sé la sua luce, or prende per coperta le profondità del mare.
\par 31 Per tal modo punisce i popoli, e dà loro del cibo in abbondanza.
\par 32 S'empie di fulmini le mani, e li lancia contro gli avversari.
\par 33 Il rombo del tuono annunzia ch'ei viene, gli animali lo presenton vicino.

\chapter{37}

\par 1 A tale spettacolo il cuor mi trema e balza fuor del suo luogo.
\par 2 Udite, udite il fragore della sua voce, il rombo che esce dalla sua bocca!
\par 3 Egli lo lancia sotto tutti i cieli e il suo lampo guizza fino ai lembi della terra.
\par 4 Dopo il lampo, una voce rugge; egli tuona con la sua voce maestosa; e quando s'ode la voce, il fulmine non è già più nella sua mano.
\par 5 Iddio tuona con la sua voce maravigliosamente; grandi cose egli fa che noi non intendiamo.
\par 6 Dice alla neve: 'Cadi sulla terra!' lo dice al nembo della pioggia, al nembo delle piogge torrenziali.
\par 7 Rende inerte ogni mano d'uomo, onde tutti i mortali, che son opera sua, imparino a conoscerlo.
\par 8 Le bestie selvagge vanno nel covo, e stan ritirate entro le tane.
\par 9 Dai recessi del sud viene l'uragano, dagli aquiloni il freddo.
\par 10 Al soffio di Dio si forma il ghiaccio e si contrae la distesa dell'acque.
\par 11 Egli carica pure le nubi d'umidità, disperde lontano le nuvole che portano i suoi lampi
\par 12 ed esse, da lui guidate, vanno vagando nei lor giri per eseguir quanto ei loro comanda sopra la faccia di tutta la terra;
\par 13 e le manda o come flagello, o come beneficio alla sua terra, o come prova della sua bontà.
\par 14 Porgi l'orecchio a questo, o Giobbe; fermati, e considera le maraviglie di Dio!
\par 15 Sai tu come Iddio le diriga e faccia guizzare il lampo dalle sue nubi?
\par 16 Conosci tu l'equilibrio delle nuvole, le maraviglie di colui la cui scienza è perfetta?
\par 17 Sai tu come mai gli abiti tuoi sono caldi quando la terra s'assopisce sotto il soffio dello scirocco?
\par 18 Puoi tu, come lui, distendere i cieli e farli solidi come uno specchio di metallo?
\par 19 Insegnaci tu che dirgli!... Nelle tenebre nostre, noi non abbiam parole.
\par 20 Gli si annunzierà forse ch'io voglio parlare? Ma chi mai può bramare d'essere inghiottito?
\par 21 Nessuno può fissare il sole che sfolgora ne' cieli quando v'è passato il vento a renderli tersi.
\par 22 Dal settentrione viene l'oro; ma Dio è circondato da una maestà terribile;
\par 23 l'Onnipotente noi non lo possiam scoprire. Egli è grande in forza, in equità, in perfetta giustizia; egli non opprime alcuno.
\par 24 Perciò gli uomini lo temono; ei non degna d'uno sguardo chi si presume savio".

\chapter{38}

\par 1 Allora l'Eterno rispose a Giobbe dal seno della tempesta, e disse:
\par 2 "Chi è costui che oscura i miei disegni con parole prive di senno?
\par 3 Orsù, cingiti i lombi come un prode; io ti farò delle domande e tu insegnami!
\par 4 Dov'eri tu quand'io fondavo la terra? Dillo, se hai tanta intelligenza.
\par 5 Chi ne fissò le dimensioni? giacché tu il sai! O chi tirò sovr'essa la corda da misurare?
\par 6 Su che furon poggiate le sue fondamenta, o chi ne pose la pietra angolare
\par 7 quando le stelle del mattino cantavan tutte assieme e tutti i figli di Dio davan in gridi di giubilo?
\par 8 Chi chiuse con porte il mare balzante fuor dal seno materno,
\par 9 quando gli detti le nubi per vestimento e per fasce l'oscurità,
\par 10 quando gli tracciai de' confini, gli misi sbarre e porte,
\par 11 e dissi: 'Fin qui tu verrai, e non oltre; qui si fermerà l'orgoglio de' tuoi flutti?'
\par 12 Hai tu mai, in vita tua, comandato al mattino? o insegnato il suo luogo all'aurora,
\par 13 perch'ella afferri i lembi della terra, e ne scuota via i malvagi?
\par 14 La terra si trasfigura come creta sotto il sigillo, e appar come vestita d'un ricco manto;
\par 15 i malfattori sono privati della luce loro, e il braccio, alzato già, è spezzato.
\par 16 Sei tu penetrato fino alle sorgenti del mare? hai tu passeggiato in fondo all'abisso?
\par 17 Le porte della morte ti son esse state scoperte? Hai tu veduto le porte dell'ombra di morte?
\par 18 Hai tu abbracciato collo sguardo l'ampiezza della terra? Parla, se la conosci tutta!
\par 19 Dov'è la via che guida al soggiorno della luce? E la tenebra dov'è la sua dimora?
\par 20 Le puoi tu menare verso i loro domini, e sai tu bene i sentieri per ricondurle a casa?
\par 21 Lo sai di sicuro! ché tu eri, allora, già nato, e il numero de' tuoi giorni è grande!...
\par 22 Sei tu entrato ne' depositi della neve? Li hai visti i depositi della grandine
\par 23 ch'io tengo in serbo per i tempi della distretta, pel giorno della battaglia e della guerra?
\par 24 Per quali vie si diffonde la luce e si sparge il vento orientale sulla terra?
\par 25 Chi ha aperto i canali all'acquazzone e segnato la via al lampo dei tuoni,
\par 26 perché la pioggia cada sulla terra inabitata, sul deserto ove non sta alcun uomo,
\par 27 e disseti le solitudini desolate, sì che vi germogli e cresca l'erba?
\par 28 Ha forse la pioggia un padre? o chi genera le gocce della rugiada?
\par 29 Dal seno di chi esce il ghiaccio, e la brina del cielo chi la dà alla luce?
\par 30 Le acque, divenute come pietra, si nascondono, e la superficie dell'abisso si congela.
\par 31 Sei tu che stringi i legami delle Pleiadi, o potresti tu scioglier le catene d'Orione?
\par 32 Sei tu che, al suo tempo, fai apparire le costellazioni e guidi la grand'Orsa insieme a' suoi piccini?
\par 33 Conosci tu le leggi del cielo? e regoli tu il dominio di esso sulla terra?
\par 34 Puoi tu levar la voce fino alle nubi, e far che abbondanza di pioggia ti ricopra?
\par 35 I fulmini parton forse al tuo comando? Ti dicono essi: 'Eccoci qua'?
\par 36 Chi ha messo negli strati delle nubi sapienza, o chi ha dato intelletto alla meteora?
\par 37 Chi conta con sapienza le nubi? e gli otri del cielo chi li versa
\par 38 allorché la polvere stemperata diventa come una massa in fusione e le zolle de' campi si saldan fra loro?
\par 39 Sei tu che cacci la preda per la leonessa, che sazi la fame de' leoncelli
\par 40 quando si appiattano nelle tane e si mettono in agguato nella macchia?
\par 41 Chi provvede il pasto al corvo quando i suoi piccini gridano a Dio e vanno errando senza cibo?

\chapter{39}

\par 1 Sai tu quando le capre selvagge delle rocce figliano? Hai tu osservato quando le cerve partoriscono?
\par 2 Conti tu i mesi della lor pregnanza e sai tu il momento in cui debbono sgravarsi?
\par 3 S'accosciano, fanno i lor piccini, e son tosto liberate delle loro doglie;
\par 4 i lor piccini si fanno forti, crescono all'aperto, se ne vanno, e non tornan più alle madri.
\par 5 Chi manda libero l'onàgro, e chi scioglie i legami all'asino selvatico,
\par 6 al quale ho dato per dimora il deserto, e la terra salata per abitazione?
\par 7 Egli si beffa del frastuono della città, e non ode grida di padrone.
\par 8 Batte le montagne della sua pastura, e va in traccia d'ogni filo di verde.
\par 9 Il bufalo vorrà egli servirti o passar la notte presso alla tua mangiatoia?
\par 10 Legherai tu il bufalo con una corda perché faccia il solco? erpicherà egli le valli dietro a te?
\par 11 Ti fiderai di lui perché la sua forza è grande? Lascerai a lui il tuo lavoro?
\par 12 Conterai su lui perché ti porti a casa la raccolta e ti ammonti il grano sull'aia?
\par 13 Lo struzzo batte allegramente l'ali; ma le penne e le piume di lui son esse pietose?
\par 14 No, poich'egli abbandona sulla terra le proprie uova e le lascia scaldar sopra la sabbia.
\par 15 Egli dimentica che un piede le potrà schiacciare, e che le bestie dei campi le potran calpestare.
\par 16 Tratta duramente i suoi piccini, quasi non fosser suoi; la sua fatica sarà vana, ma ciò non lo turba,
\par 17 ché Iddio l'ha privato di sapienza, e non gli ha impartito intelligenza.
\par 18 Ma quando si leva e piglia lo slancio, si beffa del cavallo e di chi lo cavalca.
\par 19 Sei tu che dài al cavallo il coraggio? che gli vesti il collo d'una fremente criniera?
\par 20 Sei tu che lo fai saltar come la locusta? Il fiero suo nitrito incute spavento.
\par 21 Raspa la terra nella valle ed esulta della sua forza; si slancia incontro alle armi.
\par 22 Della paura si ride, non trema, non indietreggia davanti alla spada.
\par 23 Gli risuona addosso il turcasso, la folgorante lancia e il dardo.
\par 24 Con fremente furia divora la terra. Non sta più fermo quando suona la tromba.
\par 25 Com'ode lo squillo, dice: Aha! e fiuta da lontano la battaglia, la voce tonante dei capi, e il grido di guerra.
\par 26 È l'intelligenza tua che allo sparviere fa spiccare il volo e spiegar l'ali verso mezzogiorno?
\par 27 È forse al tuo comando che l'aquila si leva in alto e fa il suo nido nei luoghi elevati?
\par 28 Abita nelle rocce e vi pernotta; sta sulla punta delle rupi, sulle vette scoscese;
\par 29 di là spia la preda, e i suoi occhi miran lontano.
\par 29 di là spia la preda, e i suoi occhi miran lontano.

\chapter{40}

\par 1 L'Eterno continuò a rispondere a Giobbe e disse:
\par 2 "Il censore dell'Onnipotente vuole ancora contendere con lui? Colui che censura Iddio ha egli una risposta a tutto questo?"
\par 3 Allora Giobbe rispose all'Eterno e disse:
\par 4 "Ecco, io son troppo meschino; che ti risponderei? Io mi metto la mano sulla bocca.
\par 5 Ho parlato una volta, ma non riprenderò la parola, due volte... ma non lo farò più".
\par 6 L'Eterno allora rispose a Giobbe dal seno della tempesta, e disse:
\par 7 "Orsù, cingiti i lombi come un prode; ti farò delle domande e tu insegnami!
\par 8 Vuoi tu proprio annullare il mio giudizio? condannar me per giustificar te stesso?
\par 9 Hai tu un braccio pari a quello di Dio? o una voce che tuoni come la sua?
\par 10 Su via, adornati di maestà, di grandezza, rivestiti di splendore, di magnificenza!
\par 11 Da' libero corso ai furori dell'ira tua; mira tutti i superbi e abbassali!
\par 12 Mira tutti i superbi e umiliali! e schiaccia gli empi dovunque stanno!
\par 13 Seppelliscili tutti assieme nella polvere, copri di bende la lor faccia nel buio della tomba!
\par 14 Allora, anch'io ti loderò, perché la tua destra t'avrà dato la vittoria.
\par 15 Guarda l'ippopotamo che ho fatto al par di te; esso mangia l'erba come il bove.
\par 16 Ecco la sua forza è nei suoi lombi, e il vigor suo nei muscoli del ventre.
\par 17 Stende rigida come un cedro la coda; i nervi delle sue cosce sono intrecciati insieme.
\par 18 Le sue ossa sono tubi di rame; le sue membra, sbarre di ferro.
\par 19 Esso è il capolavoro di Dio; colui che lo fece l'ha fornito di falce,
\par 20 perché i monti gli producon la pastura; e là tutte le bestie de' campi gli scherzano intorno.
\par 21 Si giace sotto i loti, nel folto de' canneti, in mezzo alle paludi.
\par 22 I loti lo copron dell'ombra loro, i salci del torrente lo circondano.
\par 23 Straripi pure il fiume, ei non trema; rimane calmo, anche se avesse un Giordano alla gola.
\par 24 Potrebbe alcuno impadronirsene assalendolo di fronte? o prenderlo colle reti per forargli il naso?
\par 25 Prenderai tu il coccodrillo all'amo? Gli assicurerai la lingua colla corda?
\par 26 Gli passerai un giunco per le narici? Gli forerai le mascelle con l'uncino?
\par 27 Ti rivolgerà egli molte supplicazioni? Ti dirà egli delle parole dolci?
\par 28 Farà egli teco un patto perché tu lo prenda per sempre al tuo servizio?
\par 29 Scherzerai tu con lui come fosse un uccello? L'attaccherai a un filo per divertir le tue ragazze?
\par 30 Ne trafficheranno forse i pescatori? Lo spartiranno essi fra i negozianti?
\par 31 Gli coprirai tu la pelle di dardi e la testa di ramponi?
\par 32 Mettigli un po' le mani addosso!... Ti ricorderai del combattimento e non ci tornerai!

\chapter{41}

\par 1 Ecco, fallace è la speranza di chi l'assale; basta scorgerlo e s'è atterrati.
\par 2 Nessuno è tanto ardito da provocarlo. E chi dunque oserà starmi a fronte?
\par 3 Chi mi ha anticipato alcun che perch'io glielo debba rendere? Sotto tutti i cieli, ogni cosa è mia.
\par 4 E non vo' tacer delle sue membra, della sua gran forza, della bellezza della sua armatura.
\par 5 Chi l'ha mai spogliato della sua corazza? Chi è penetrato fra la doppia fila dei suoi denti?
\par 6 Chi gli ha aperti i due battenti della gola? Intorno alla chiostra de' suoi denti sta il terrore.
\par 7 Superbe son le file de' suoi scudi, strettamente uniti come da un sigillo.
\par 8 Uno tocca l'altro, e tra loro non passa l'aria.
\par 9 Sono saldati assieme, si tengono stretti, sono inseparabili.
\par 10 I suoi starnuti danno sprazzi di luce; i suoi occhi son come le palpebre dell'aurora.
\par 11 Dalla sua bocca partono vampe, ne scappan fuori scintille di fuoco.
\par 12 Dalle sue narici esce un fumo, come da una pignatta che bolla o da una caldaia.
\par 13 L'alito suo accende i carboni, e una fiamma gli erompe dalla gola.
\par 14 Nel suo collo risiede la forza, dinanzi a lui salta il terrore.
\par 15 Compatte sono in lui le parti flosce della carne, gli stanno salde addosso, non si muovono.
\par 16 Il suo cuore è duro come il sasso, duro come la macina di sotto.
\par 17 Quando si rizza, tremano i più forti, e dalla paura son fuori di sé.
\par 18 Invano lo si attacca con la spada; a nulla valgon lancia, giavellotto, corazza.
\par 19 Il ferro è per lui come paglia; il rame, come legno tarlato.
\par 20 La figlia dell'arco non lo mette in fuga; le pietre della fionda si mutano per lui in stoppia.
\par 21 Stoppia gli par la mazza e si ride del fremer della lancia.
\par 22 Il suo ventre è armato di punte acute, e lascia come tracce d'erpice sul fango.
\par 23 Fa bollire l'abisso come una caldaia, del mare fa come un gran vaso da profumi.
\par 24 Si lascia dietro una scia di luce; l'abisso par coperto di bianca chioma.
\par 25 Non v'è sulla terra chi lo domi; è stato fatto per non aver paura.
\par 26 Guarda in faccia tutto ciò ch'è eccelso, è re su tutte le belve più superbe".

\chapter{42}

\par 1 Allora Giobbe rispose all'Eterno e disse:
\par 2 "Io riconosco che tu puoi tutto, e che nulla può impedirti d'eseguire un tuo disegno.
\par 3 Chi è colui che senza intendimento offusca il tuo disegno?... Sì, ne ho parlato; ma non lo capivo; son cose per me troppo maravigliose ed io non le conosco.
\par 4 Deh, ascoltami, io parlerò; io ti farò delle domande e tu insegnami!
\par 5 Il mio orecchio avea sentito parlar di te ma ora l'occhio mio t'ha veduto.
\par 6 Perciò mi ritratto, mi pento sulla polvere e sulla cenere".
\par 7 Dopo che ebbe rivolto questi discorsi a Giobbe, l'Eterno disse a Elifaz di Teman: 'L'ira mia è accesa contro te e contro i tuoi due amici, perché non avete parlato di me secondo la verità, come ha fatto il mio servo Giobbe.
\par 8 Ora dunque prendetevi sette tori e sette montoni, venite a trovare il mio servo Giobbe e offriteli in olocausto per voi stessi. Il mio servo Giobbe pregherà per voi; ed io avrò riguardo a lui per non punir la vostra follia; poiché non avete parlato di me secondo la verità, come ha fatto il mio servo Giobbe'.
\par 9 Elifaz di Teman e Bildad di Suach e Tsofar di Naama se ne andarono e fecero come l'Eterno aveva loro ordinato; e l'Eterno ebbe riguardo a Giobbe.
\par 10 E quando Giobbe ebbe pregato per i suoi amici, l'Eterno lo ristabilì nella condizione di prima e gli rese il doppio di tutto quello che già gli era appartenuto.
\par 11 Tutti i suoi fratelli, tutte le sue sorelle e tutte le sue conoscenze di prima vennero a trovarlo, mangiarono con lui in casa sua, gli fecero le loro condoglianze e lo consolarono di tutti i mali che l'Eterno gli avea fatto cadere addosso; e ognuno d'essi gli dette un pezzo d'argento e un anello d'oro.
\par 12 E l'Eterno benedì gli ultimi anni di Giobbe più de' primi; ed ei s'ebbe quattordicimila pecore, seimila cammelli, mille paia di bovi e mille asine.
\par 13 E s'ebbe pure sette figliuoli e tre figliuole;
\par 14 e chiamò la prima, Colomba; la seconda, Cassia; la terza, Cornustibia.
\par 15 E in tutto il paese non c'eran donne così belle come le figliuole di Giobbe; e il padre assegnò loro una eredità tra i loro fratelli.
\par 16 Giobbe, dopo questo, visse centoquarant'anni, e vide i suoi figliuoli e i figliuoli dei suoi figliuoli, fino alla quarta generazione.
\par 17 Poi Giobbe morì vecchio e sazio di giorni.


\end{document}