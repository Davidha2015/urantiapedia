\begin{document}

\title{Proverbs}


\chapter{1}

\par 1 Proverbi di Salomone, figliuolo di Davide, re d'Israele;
\par 2 perché l'uomo conosca la sapienza e l'istruzione, e intenda i detti sensati;
\par 3 perché riceva istruzione circa l'assennatezza, la giustizia, l'equità, la dirittura;
\par 4 per dare accorgimento ai semplici, e conoscenza e riflessione al giovane.
\par 5 Il savio ascolterà, e accrescerà il suo sapere; l'uomo intelligente ne ritrarrà buone direzioni
\par 6 per capire i proverbi e le allegorie, le parole dei savi e i loro enigmi.
\par 7 Il timore dell'Eterno è il principio della scienza; gli stolti disprezzano la sapienza e l'istruzione.
\par 8 Ascolta, figliuol mio, l'istruzione di tuo padre e non ricusare l'insegnamento di tua madre;
\par 9 poiché saranno una corona di grazia sul tuo capo, e monili al tuo collo.
\par 10 Figliuol mio, se i peccatori ti voglion sedurre, non dar loro retta.
\par 11 Se dicono: - 'Vieni con noi; mettiamoci in agguato per uccidere; tendiamo insidie senza motivo all'innocente;
\par 12 inghiottiamoli vivi, come il soggiorno de' morti, e tutt'interi come quelli che scendon nella fossa;
\par 13 noi troveremo ogni sorta di beni preziosi, empiremo le nostre case di bottino;
\par 14 tu trarrai a sorte la tua parte con noi, non ci sarà fra noi tutti che una borsa sola' -
\par 15 figliuol mio, non t'incamminare con essi; trattieni il tuo piè lungi dal loro sentiero;
\par 16 poiché i loro piedi corrono al male ed essi s'affrettano a spargere il sangue.
\par 17 Si tende invano la rete dinanzi a ogni sorta d'uccelli;
\par 18 ma costoro pongono agguati al loro proprio sangue, e tendono insidie alla stessa loro vita.
\par 19 Tal è la sorte di chiunque è avido di guadagno; esso tòglie la vita a chi lo possiede.
\par 20 La sapienza grida per le vie, fa udire la sua voce per le piazze;
\par 21 nei crocicchi affollati ella chiama, all'ingresso delle porte, in città, pronunzia i suoi discorsi:
\par 22 'Fino a quando, o scempi, amerete la scempiaggine? fino a quando gli schernitori prenderanno gusto a schernire e gli stolti avranno in odio la scienza?
\par 23 Volgetevi a udire la mia riprensione; ecco, io farò sgorgare su voi lo spirito mio, vi farò conoscere le mie parole...
\par 24 Ma poiché, quand'ho chiamato avete rifiutato d'ascoltare, quand'ho steso la mano nessun vi ha badato,
\par 25 anzi avete respinto ogni mio consiglio e della mia correzione non ne avete voluto sapere,
\par 26 anch'io mi riderò delle vostre sventure, mi farò beffe quando lo spavento vi piomberà addosso;
\par 27 quando lo spavento vi piomberà addosso come una tempesta, quando la sventura v'investirà come un uragano, e vi cadranno addosso la distretta e l'angoscia.
\par 28 Allora mi chiameranno, ma io non risponderò; mi cercheranno con premura ma non mi troveranno.
\par 29 Poiché hanno odiato la scienza e non hanno scelto il timor dell'Eterno
\par 30 e non hanno voluto sapere dei miei consigli e hanno disdegnato ogni mia riprensione,
\par 31 si pasceranno del frutto della loro condotta, e saranno saziati dei loro propri consigli.
\par 32 Poiché il pervertimento degli scempi li uccide, e lo sviarsi degli stolti li fa perire;
\par 33 ma chi m'ascolta se ne starà al sicuro, sarà tranquillo, senza paura d'alcun male'.

\chapter{2}

\par 1 Figliuol mio, se ricevi le mie parole e serbi con cura i miei comandamenti,
\par 2 prestando orecchio alla sapienza e inclinando il cuore all'intelligenza;
\par 3 sì, se chiami il discernimento e rivolgi la tua voce all'intelligenza,
\par 4 se la cerchi come l'argento e ti dai a scavarla come un tesoro,
\par 5 allora intenderai il timor dell'Eterno, e troverai la conoscenza di Dio.
\par 6 Poiché l'Eterno dà la sapienza; dalla sua bocca procedono la scienza e l'intelligenza.
\par 7 Egli tiene in serbo per gli uomini retti un aiuto potente, uno scudo per quelli che camminano integramente,
\par 8 affin di proteggere i sentieri della equità e di custodire la via dei suoi fedeli.
\par 9 Allora intenderai la giustizia, l'equità, la rettitudine, tutte le vie del bene.
\par 10 Perché la sapienza t'entrerà nel cuore, e la scienza sarà gradevole all'anima tua;
\par 11 la riflessione veglierà su te, e l'intelligenza ti proteggerà;
\par 12 ti scamperà così dalla via malvagia, dalla gente che parla di cose perverse,
\par 13 da quelli che lasciano i sentieri della rettitudine per camminare nella via delle tenebre,
\par 14 che godono a fare il male e si compiacciono delle perversità del malvagio,
\par 15 che seguono sentieri storti e battono vie tortuose.
\par 16 Ti scamperà dalla donna adultera, dalla infedele che usa parole melate,
\par 17 che ha abbandonato il compagno della sua giovinezza e ha dimenticato il patto del suo Dio.
\par 18 Poiché la sua casa pende verso la morte, e i suoi sentieri menano ai defunti.
\par 19 Nessuno di quelli che vanno da lei ne ritorna, nessuno riprende i sentieri della vita.
\par 20 Così camminerai per la via dei buoni, e rimarrai nei sentieri dei giusti.
\par 21 Ché gli uomini retti abiteranno la terra, e quelli che sono integri vi rimarranno;
\par 22 ma gli empi saranno sterminati di sulla terra e gli sleali ne saranno divelti.

\chapter{3}

\par 1 Figliuol mio, non dimenticare il mio insegnamento, e il tuo cuore osservi i miei comandamenti,
\par 2 perché ti procureranno lunghi giorni, anni di vita e di prosperità.
\par 3 Bontà e verità non ti abbandonino; lègatele al collo, scrivile sulla tavola del tuo cuore;
\par 4 troverai così grazia e buon senno agli occhi di Dio e degli uomini.
\par 5 Confidati nell'Eterno con tutto il cuore, e non t'appoggiare sul tuo discernimento.
\par 6 Riconoscilo in tutte le tue vie, ed egli appianerà i tuoi sentieri.
\par 7 Non ti stimar savio da te stesso; temi l'Eterno e ritirati dal male;
\par 8 questo sarà la salute del tuo corpo, e un refrigerio alle tue ossa.
\par 9 Onora l'Eterno con i tuoi beni e con le primizie d'ogni tua rendita;
\par 10 i tuoi granai saran ripieni d'abbondanza e i tuoi tini traboccheranno di mosto.
\par 11 Figliuol mio, non disdegnare la correzione dell'Eterno, e non ti ripugni la sua riprensione;
\par 12 ché l'Eterno riprende colui ch'egli ama, come un padre il figliuolo che gradisce.
\par 13 Beato l'uomo che ha trovato la sapienza, e l'uomo che ottiene l'intelligenza!
\par 14 Poiché il guadagno ch'essa procura è preferibile a quel dell'argento, e il profitto che se ne trae val più dell'oro fino.
\par 15 Essa è più pregevole delle perle, e quanto hai di più prezioso non l'equivale.
\par 16 Lunghezza di vita è nella sua destra; ricchezza e gloria nella sua sinistra.
\par 17 Le sue vie son vie dilettevoli, e tutti i suoi sentieri sono pace.
\par 18 Essa è un albero di vita per quei che l'afferrano, e quei che la ritengon fermamente sono beati.
\par 19 Con la sapienza l'Eterno fondò la terra, e con l'intelligenza rese stabili i cieli.
\par 20 Per la sua scienza gli abissi furono aperti, e le nubi distillano la rugiada.
\par 21 Figliuol mio, queste cose non si dipartano mai dagli occhi tuoi! Ritieni la saviezza e la riflessione!
\par 22 Esse saranno la vita dell'anima tua e un ornamento al tuo collo.
\par 23 Allora camminerai sicuro per la tua via, e il tuo piede non inciamperà.
\par 24 Quando ti metterai a giacere non avrai paura; giacerai, e il sonno tuo sarà dolce.
\par 25 Non avrai da temere i sùbiti spaventi, né la ruina degli empi, quando avverrà;
\par 26 perché l'Eterno sarà la tua sicurezza, e preserverà il tuo piede da ogn'insidia.
\par 27 Non rifiutare un benefizio a chi vi ha diritto, quand'è in tuo potere di farlo.
\par 28 Non dire al tuo prossimo: 'Va' e torna' e 'te lo darò domani', quand'hai di che dare.
\par 29 Non macchinare il male contro il tuo prossimo, mentr'egli abita fiducioso con te.
\par 30 Non intentar causa ad alcuno senza motivo, allorché non t'ha fatto alcun torto.
\par 31 Non portare invidia all'uomo violento, e non scegliere alcuna delle sue vie;
\par 32 poiché l'Eterno ha in abominio l'uomo perverso, ma l'amicizia sua è per gli uomini retti.
\par 33 La maledizione dell'Eterno è nella casa dell'empio, ma egli benedice la dimora dei giusti.
\par 34 Se schernisce gli schernitori, fa grazia agli umili.
\par 35 I savi erederanno la gloria, ma l'ignominia è la parte degli stolti.

\chapter{4}

\par 1 Figliuoli, ascoltate l'istruzione di un padre, e state attenti a imparare il discernimento;
\par 2 perché io vi do una buona dottrina; non abbandonate il mio insegnamento.
\par 3 Quand'ero ancora fanciullo presso mio padre, tenero ed unico presso mia madre,
\par 4 egli mi ammaestrava e mi diceva: 'Il tuo cuore ritenga le mie parole; osserva i miei comandamenti, e vivrai.
\par 5 Acquista sapienza, acquista intelligenza; non dimenticare le parole della mia bocca, e non te ne sviare;
\par 6 non abbandonare la sapienza, ed essa ti custodirà; amala, ed essa ti proteggerà.
\par 7 Il principio della sapienza è: Acquista la sapienza. Sì, a costo di quanto possiedi, acquista l'intelligenza.
\par 8 Esaltala, ed essa t'innalzerà; essa ti coprirà di gloria, quando l'avrai abbracciata.
\par 9 Essa ti metterà sul capo una corona di grazia, ti farà dono d'un magnifico diadema'.
\par 10 Ascolta, figliuol mio, ricevi le mie parole, e anni di vita ti saranno moltiplicati.
\par 11 Io ti mostro la via della sapienza, t'avvio per i sentieri della rettitudine.
\par 12 Se cammini, i tuoi passi non saran raccorciati; e se corri, non inciamperai.
\par 13 Afferra saldamente l'istruzione, non la lasciar andare; serbala, perch'essa è la tua vita.
\par 14 Non entrare nel sentiero degli empi, e non t'inoltrare per la via de' malvagi;
\par 15 schivala, non passare per essa; allontanatene, e va' oltre.
\par 16 Poiché essi non posson dormire se non han fatto del male, e il sonno è loro tolto se non han fatto cader qualcuno.
\par 17 Essi mangiano il pane dell'empietà, e bevono il vino della violenza;
\par 18 ma il sentiero dei giusti è come la luce che spunta e va via più risplendendo, finché sia giorno perfetto.
\par 19 La via degli empi è come il buio; essi non scorgono ciò che li farà cadere.
\par 20 Figliuol mio, sta' attento alle mie parole, inclina l'orecchio ai miei detti;
\par 21 non si dipartano mai dai tuoi occhi, serbali nel fondo del cuore;
\par 22 poiché sono vita per quelli che li trovano, e salute per tutto il loro corpo.
\par 23 Custodisci il tuo cuore più d'ogni altra cosa, poiché da esso procedono le sorgenti della vita.
\par 24 Rimuovi da te la perversità della bocca, e allontana da te la falsità delle labbra.
\par 25 Gli occhi tuoi guardino bene in faccia, e le tue palpebre si dirigano dritto davanti a te.
\par 26 Appiana il sentiero dei tuoi piedi, e tutte le tue vie siano ben preparate.
\par 27 Non piegare né a destra né a sinistra; ritira il tuo piede dal male.

\chapter{5}

\par 1 Figliuol mio, sta' attento alla mia sapienza, inclina l'orecchio alla mia intelligenza,
\par 2 affinché tu conservi l'accorgimento, e le tue labbra ritengano la scienza.
\par 3 Poiché le labbra dell'adultera stillano miele, e la sua bocca è più morbida dell'olio;
\par 4 ma la fine cui mena è amara come l'assenzio, è acuta come una spada a due tagli.
\par 5 I suoi piedi scendono alla morte, i suoi passi fan capo al soggiorno dei defunti.
\par 6 Lungi dal prendere il sentiero della vita, le sue vie sono erranti, e non sa dove va.
\par 7 Or dunque, figliuoli, ascoltatemi, e non vi dipartite dalle parole della mia bocca.
\par 8 Tieni lontana da lei la tua via, e non t'accostare alla porta della sua casa,
\par 9 per non dare ad altri il fiore della tua gioventù, e i tuoi anni al tiranno crudele;
\par 10 perché degli stranieri non si sazino de' tuoi beni, e le tue fatiche non vadano in casa d'altri;
\par 11 perché tu non abbia a gemere quando verrà la tua fine, quando la tua carne e il tuo corpo saran consumati,
\par 12 e tu non dica: 'Come ho fatto a odiare la correzione, e come ha potuto il cuor mio sprezzare la riprensione?
\par 13 come ho fatto a non ascoltare la voce di chi m'ammaestrava, e a non porger l'orecchio a chi m'insegnava?
\par 14 poco mancò che non mi trovassi immerso in ogni male, in mezzo al popolo ed all'assemblea'.
\par 15 Bevi l'acqua della tua cisterna, l'acqua viva del tuo pozzo.
\par 16 Le tue fonti debbon esse spargersi al di fuori? e i tuoi rivi debbon essi scorrer per le strade?
\par 17 Siano per te solo, e non per degli stranieri con te.
\par 18 Sia benedetta la tua fonte, e vivi lieto con la sposa della tua gioventù.
\par 19 Cerva d'amore, cavriola di grazia, le sue carezze t'inebrino in ogni tempo, e sii del continuo rapito nell'affetto suo.
\par 20 E perché, figliuol mio, t'invaghiresti d'un'estranea, e abbracceresti il seno della donna altrui?
\par 21 Ché le vie dell'uomo stan davanti agli occhi dell'Eterno, il quale osserva tutti i sentieri di lui.
\par 22 L'empio sarà preso nelle proprie iniquità, e tenuto stretto dalle funi del suo peccato.
\par 23 Egli morrà per mancanza di correzione, andrà vacillando per la grandezza della sua follia.

\chapter{6}

\par 1 Figliuol mio, se ti sei reso garante per il tuo prossimo, se ti sei impegnato per un estraneo,
\par 2 sei còlto nel laccio dalle parole della tua bocca, sei preso dalle parole della tua bocca.
\par 3 Fa' questo, figliuol mio; disimpegnati, perché sei caduto in mano del tuo prossimo. Va', gettati ai suoi piedi, insisti,
\par 4 non dar sonno ai tuoi occhi né sopore alle tue palpebre;
\par 5 disimpegnati come il cavriolo di man del cacciatore, come l'uccello di mano dell'uccellatore.
\par 6 Va', pigro, alla formica; considera il suo fare, e diventa savio!
\par 7 Essa non ha né capo, né sorvegliante, né padrone;
\par 8 prepara il suo cibo nell'estate, e raduna il suo mangiare durante la raccolta.
\par 9 Fino a quando, o pigro, giacerai? quando ti desterai dal tuo sonno?
\par 10 Dormire un po', sonnecchiare un po', incrociare un po' le mani per riposare...
\par 11 e la tua povertà verrà come un ladro, e la tua indigenza, come un uomo armato.
\par 12 L'uomo da nulla, l'uomo iniquo cammina colla falsità sulle labbra;
\par 13 ammicca cogli occhi, parla coi piedi, fa segni con le dita;
\par 14 ha la perversità nel cuore, macchina del male in ogni tempo, semina discordie;
\par 15 perciò la sua ruina verrà ad un tratto, in un attimo sarà distrutto, senza rimedio.
\par 16 Sei cose odia l'Eterno, anzi sette gli sono in abominio:
\par 17 gli occhi alteri, la lingua bugiarda, le mani che spandono sangue innocente,
\par 18 il cuore che medita disegni iniqui, i piedi che corron frettolosi al male,
\par 19 il falso testimonio che proferisce menzogne, e chi semina discordie tra fratelli.
\par 20 Figliuol mio, osserva i precetti di tuo padre, e non trascurare gl'insegnamenti di tua madre;
\par 21 tienteli del continuo legati sul cuore e attaccati al collo.
\par 22 Quando camminerai, ti guideranno; quando giacerai, veglieranno su te; quando ti risveglierai, ragioneranno teco.
\par 23 Poiché il precetto è una lampada e l'insegnamento una luce, e le correzioni della disciplina son la via della vita,
\par 24 per guardarti dalla donna malvagia, dalle parole lusinghevoli della straniera.
\par 25 Non bramare in cuor tuo la sua bellezza, e non ti lasciar prendere dalle sue palpebre;
\par 26 ché per una donna corrotta uno si riduce a un pezzo di pane, e la donna adultera sta in agguato contro un'anima preziosa.
\par 27 Uno si metterà forse del fuoco in seno senza che i suoi abiti si brucino?
\par 28 camminerà forse sui carboni accesi senza scottarsi i piedi?
\par 29 Così è di chi va dalla moglie del prossimo; chi la tocca non rimarrà impunito.
\par 30 Non si disprezza il ladro che ruba per saziarsi quand'ha fame;
\par 31 se è còlto, restituirà anche il settuplo, darà tutti i beni della sua casa.
\par 32 Ma chi commette un adulterio è privo di senno; chi fa questo vuol rovinar se stesso.
\par 33 Troverà ferite ed ignominia, e l'obbrobrio suo non sarà mai cancellato;
\par 34 ché la gelosia rende furioso il marito, il quale sarà senza pietà nel dì della vendetta;
\par 35 non avrà riguardo a riscatto di sorta, e anche se tu moltiplichi i regali, non sarà soddisfatto.

\chapter{7}

\par 1 Figliuol mio, ritieni le mie parole, e fa' tesoro de' miei comandamenti.
\par 2 Osserva i miei comandamenti e vivrai; custodisci il mio insegnamento come la pupilla degli occhi.
\par 3 Lègateli alle dita, scrivili sulla tavola del tuo cuore.
\par 4 Di' alla sapienza: 'Tu sei mia sorella', e chiama l'intelligenza amica tua,
\par 5 affinché ti preservino dalla donna altrui, dall'estranea che usa parole melate.
\par 6 Ero alla finestra della mia casa, e dietro alla mia persiana stavo guardando,
\par 7 quando vidi, tra gli sciocchi, scòrsi, tra i giovani, un ragazzo privo di senno,
\par 8 che passava per la strada, presso all'angolo dov'essa abitava, e si dirigeva verso la casa di lei,
\par 9 al crepuscolo, sul declinar del giorno, allorché la notte si faceva nera, oscura.
\par 10 Ed ecco farglisi incontro una donna in abito da meretrice e astuta di cuore,
\par 11 turbolenta e proterva, che non teneva piede in casa:
\par 12 ora in istrada, ora per le piazze, e in agguato presso ogni canto.
\par 13 Essa lo prese, lo baciò, e sfacciatamente gli disse:
\par 14 'Dovevo fare un sacrifizio di azioni di grazie; oggi ho sciolto i miei voti;
\par 15 perciò ti son venuta incontro per cercarti, e t'ho trovato.
\par 16 Ho guarnito il mio letto di morbidi tappeti, di coperte ricamate con filo d'Egitto;
\par 17 l'ho profumato di mirra, d'aloè e di cinnamomo.
\par 18 Vieni inebriamoci d'amore fino al mattino, sollazziamoci in amorosi piaceri;
\par 19 giacché il mio marito non è a casa; è andato in viaggio lontano;
\par 20 ha preso seco un sacchetto di danaro, non tornerà a casa che al plenilunio'.
\par 21 Ella lo sedusse con le sue molte lusinghe, lo trascinò con la dolcezza delle sue labbra.
\par 22 Egli le andò dietro subito, come un bove va al macello, come uno stolto è menato ai ceppi che lo castigheranno,
\par 23 come un uccello s'affretta al laccio, senza sapere ch'è teso contro la sua vita, finché una freccia gli trapassi il fegato.
\par 24 Or dunque, figliuoli, ascoltatemi, e state attenti alle parole della mia bocca.
\par 25 Il tuo cuore non si lasci trascinare nelle vie d'una tal donna; non ti sviare per i suoi sentieri;
\par 26 ché molti ne ha fatti cadere feriti a morte, e grande è la moltitudine di quelli che ha uccisi.
\par 27 La sua casa è la via del soggiorno de' defunti, la strada che scende ai penetrali della morte.

\chapter{8}

\par 1 La sapienza non grida ella? e l'intelligenza non fa ella udire la sua voce?
\par 2 Ella sta in piè al sommo dei luoghi elevati, sulla strada, ai crocicchi;
\par 3 grida presso le porte, all'ingresso della città, nei viali che menano alle porte;
\par 4 'Chiamo voi, o uomini principali, e la mia voce si rivolge ai figli del popolo.
\par 5 Imparate, o semplici, l'accorgimento, e voi, stolti, diventate intelligenti di cuore!
\par 6 Ascoltate, perché dirò cose eccellenti, e le mie labbra s'apriranno a insegnar cose rette.
\par 7 Poiché la mia bocca esprime il vero, e le mie labbra abominano l'empietà.
\par 8 Tutte le parole della mia bocca son conformi a giustizia, non v'è nulla di torto o di perverso in esse.
\par 9 Son tutte piane per l'uomo intelligente, e rette per quelli che han trovato la scienza.
\par 10 Ricevete la mia istruzione anziché l'argento, e la scienza anziché l'oro scelto;
\par 11 poiché la sapienza val più delle perle, e tutti gli oggetti preziosi non la equivalgono.
\par 12 Io, la sapienza, sto con l'accorgimento, e trovo la scienza della riflessione.
\par 13 Il timore dell'Eterno è odiare il male; io odio la superbia, l'arroganza, la via del male e la bocca perversa.
\par 14 A me appartiene il consiglio e il buon successo; io sono l'intelligenza, a me appartiene la forza.
\par 15 Per mio mezzo regnano i re, e i principi decretano ciò ch'è giusto.
\par 16 Per mio mezzo governano i capi, i nobili, tutti i giudici della terra.
\par 17 Io amo quelli che m'amano, e quelli che mi cercano mi trovano.
\par 18 Con me sono ricchezze e gloria, i beni permanenti e la giustizia.
\par 19 Il mio frutto è migliore dell'oro fino, e il mio prodotto val più che argento eletto.
\par 20 Io cammino per la via della giustizia, per i sentieri dell'equità,
\par 21 per far eredi di beni reali quelli che m'amano, e per riempire i loro tesori.
\par 22 L'Eterno mi formò al principio de' suoi atti, prima di fare alcuna delle opere sue, ab antico.
\par 23 Fui stabilita ab eterno, dal principio, prima che la terra fosse.
\par 24 Fui generata quando non c'erano ancora abissi, quando ancora non c'erano sorgenti rigurgitanti d'acqua.
\par 25 Fui generata prima che i monti fossero fondati, prima ch'esistessero le colline,
\par 26 quand'egli ancora non avea fatto né la terra né i campi né le prime zolle della terra coltivabile.
\par 27 Quand'egli disponeva i cieli io ero là; quando tracciava un circolo sulla superficie dell'abisso,
\par 28 quando condensava le nuvole in alto, quando rafforzava le fonti dell'abisso,
\par 29 quando assegnava al mare il suo limite perché le acque non oltrepassassero il suo cenno, quando poneva i fondamenti della terra,
\par 30 io ero presso di lui come un artefice, ero del continuo esuberante di gioia, mi rallegravo in ogni tempo nel suo cospetto;
\par 31 mi rallegravo nella parte abitabile della sua terra, e trovavo la mia gioia tra i figliuoli degli uomini.
\par 32 Ed ora, figliuoli, ascoltatemi; beati quelli che osservano le mie vie!
\par 33 Ascoltate l'istruzione, siate savi, e non la rigettate!
\par 34 Beato l'uomo che m'ascolta, che veglia ogni giorno alle mie porte, che vigila alla soglia della mia casa!
\par 35 Poiché chi mi trova trova la vita, e ottiene favore dall'Eterno.
\par 36 Ma chi pecca contro di me, fa torto all'anima sua; tutti quelli che m'odiano, amano la morte'.

\chapter{9}

\par 1 La sapienza ha fabbricato la sua casa, ha lavorato le sue colonne, in numero di sette;
\par 2 ha ammazzato i suoi animali, ha drogato il suo vino, ed ha anche apparecchiato la sua mensa.
\par 3 Ha mandato fuori le sue ancelle; dall'alto dei luoghi elevati della città ella grida:
\par 4 'Chi è sciocco venga qua!' A quelli che son privi di senno dice:
\par 5 'Venite, mangiate del mio pane e bevete del vino che ho drogato!
\par 6 Lasciate, o sciocchi, la stoltezza e vivrete, e camminate per la via dell'intelligenza!'
\par 7 Chi corregge il beffardo s'attira vituperio, e chi riprende l'empio riceve affronto.
\par 8 Non riprendere il beffardo, per tema che t'odi; riprendi il savio, e t'amerà.
\par 9 Istruisci il savio e diventerà più savio che mai; ammaestra il giusto e accrescerà il suo sapere.
\par 10 Il principio della sapienza è il timor dell'Eterno, e conoscere il Santo è l'intelligenza.
\par 11 Poiché per mio mezzo ti saran moltiplicati i giorni, e ti saranno aumentati anni di vita.
\par 12 Se sei savio, sei savio per te stesso; se sei beffardo tu solo ne porterai la pena.
\par 13 La follia è una donna turbolenta, sciocca, che non sa nulla, nulla.
\par 14 Siede alla porta di casa, sopra una sedia, ne' luoghi elevati della città,
\par 15 per gridare a quelli che passan per la via, che van diritti per la loro strada:
\par 16 'Chi è sciocco venga qua!' E a chi è privo di senno dice:
\par 17 'Le acque rubate son dolci, e il pane mangiato di nascosto è soave'.
\par 18 Ma egli non sa che quivi sono i defunti, che i suoi convitati son nel fondo del soggiorno de' morti.

\chapter{10}

\par 1 Proverbi di Salomone. Un figliuol savio rallegra suo padre, ma un figliuolo stolto è il cordoglio di sua madre.
\par 2 I tesori d'empietà non giovano, ma la giustizia libera dalla morte.
\par 3 L'Eterno non permette che il giusto soffra la fame, ma respinge insoddisfatta l'avidità degli empi.
\par 4 Chi lavora con mano pigra impoverisce, ma la mano dei diligenti fa arricchire.
\par 5 Chi raccoglie nella estate è un figliuolo prudente, ma chi dorme durante la raccolta è un figliuolo che fa vergogna.
\par 6 Benedizioni vengono sul capo dei giusti, ma la violenza cuopre la bocca degli empi.
\par 7 La memoria del giusto è in benedizione, ma il nome degli empi marcisce.
\par 8 Il savio di cuore accetta i precetti, ma lo stolto di labbra va in precipizio.
\par 9 Chi cammina nella integrità cammina sicuro, ma chi va per vie tortuose sarà scoperto.
\par 10 Chi ammicca con l'occhio cagiona dolore, e lo stolto di labbra va in precipizio.
\par 11 La bocca del giusto è una fonte di vita, ma la bocca degli empi nasconde violenza.
\par 12 L'odio provoca liti, ma l'amore cuopre ogni fallo.
\par 13 Sulle labbra dell'uomo intelligente si trova la sapienza, ma il bastone è per il dosso di chi è privo di senno.
\par 14 I savi tengono in serbo la scienza, ma la bocca dello stolto è una rovina imminente.
\par 15 I beni del ricco sono la sua città forte; la rovina de' poveri è la loro povertà.
\par 16 Il lavoro del giusto serve alla vita, le entrate dell'empio servono al peccato.
\par 17 Chi tien conto della correzione, segue il cammino della vita; ma chi non fa caso della riprensione si smarrisce.
\par 18 Chi dissimula l'odio ha labbra bugiarde, e chi spande la calunnia è uno stolto.
\par 19 Nella moltitudine delle parole non manca la colpa, ma chi frena le sue labbra è prudente.
\par 20 La lingua del giusto è argento eletto; il cuore degli empi val poco.
\par 21 Le labbra del giusto pascono molti, ma gli stolti muoiono per mancanza di senno.
\par 22 Quel che fa ricchi è la benedizione dell'Eterno e il tormento che uno si dà non le aggiunge nulla.
\par 23 Commettere un delitto, per lo stolto, è come uno spasso; tale è la sapienza per l'uomo accorto.
\par 24 All'empio succede quello che teme, ma ai giusti è concesso quel che desiderano.
\par 25 Come procella che passa, l'empio non è più, ma il giusto ha un fondamento eterno.
\par 26 Come l'aceto ai denti e il fumo agli occhi, così è il pigro per chi lo manda.
\par 27 Il timor dell'Eterno accresce i giorni ma gli anni degli empi saranno accorciati.
\par 28 L'aspettazione dei giusti è letizia, ma la speranza degli empi perirà.
\par 29 La via dell'Eterno è una fortezza per l'uomo integro, ma una rovina per gli operatori d'iniquità.
\par 30 Il giusto non sarà mai smosso, ma gli empi non abiteranno la terra.
\par 31 La bocca del giusto sgorga sapienza, ma la lingua perversa sarà soppressa.
\par 32 Le labbra del giusto conoscono ciò che è grato, ma la bocca degli empi è piena di perversità.

\chapter{11}

\par 1 La bilancia falsa è un abominio per l'Eterno, ma il peso giusto gli è grato.
\par 2 Venuta la superbia, viene anche l'ignominia; ma la sapienza è con gli umili.
\par 3 L'integrità degli uomini retti li guida, ma la perversità dei perfidi è la loro rovina.
\par 4 Le ricchezze non servono a nulla nel giorno dell'ira, ma la giustizia salva da morte.
\par 5 La giustizia dell'uomo integro gli appiana la via, ma l'empio cade per la sua empietà.
\par 6 La giustizia degli uomini retti li libera, ma i perfidi restan presi nella loro propria malizia.
\par 7 Quando un empio muore, la sua speranza perisce, e l'aspettazione degl'iniqui è annientata.
\par 8 Il giusto è tratto fuor dalla distretta, e l'empio ne prende il posto.
\par 9 Con la sua bocca l'ipocrita rovina il suo prossimo, ma i giusti sono liberati dalla loro perspicacia.
\par 10 Quando i giusti prosperano, la città gioisce; ma quando periscono gli empi son gridi di giubilo.
\par 11 Per la benedizione degli uomini retti la città è esaltata, ma è sovvertita dalla bocca degli empi.
\par 12 Chi sprezza il prossimo è privo di senno, ma l'uomo accorto tace.
\par 13 Chi va sparlando svela i segreti, ma chi ha lo spirito leale tien celata la cosa.
\par 14 Quando manca una savia direzione il popolo cade; nel gran numero de' consiglieri sta la salvezza.
\par 15 Chi si fa mallevadore d'un altro ne soffre danno, ma chi odia la mallevadoria è sicuro.
\par 16 La donna graziosa ottiene la gloria, e gli uomini forti ottengon la ricchezza.
\par 17 L'uomo benigno fa del bene a se stesso, ma il crudele tortura la sua propria carne.
\par 18 L'empio fa un'opera fallace, ma chi semina giustizia ha una ricompensa sicura.
\par 19 Così la giustizia mena alla vita, ma chi va dietro al male s'incammina alla morte.
\par 20 I perversi di cuore sono un abominio per l'Eterno, ma gl'integri nella loro condotta gli sono graditi.
\par 21 No, certo, il malvagio non rimarrà impunito, ma la progenie dei giusti scamperà.
\par 22 Una donna bella, ma senza giudizio, è un anello d'oro nel grifo d'un porco.
\par 23 Il desiderio dei giusti è il bene soltanto, ma la prospettiva degli empi è l'ira.
\par 24 C'è chi spande liberalmente e diventa più ricco, e c'è chi risparmia più del dovere e non fa che impoverire.
\par 25 L'anima benefica sarà nell'abbondanza, e chi annaffia sarà egli pure annaffiato.
\par 26 Chi detiene il grano è maledetto dal popolo, ma la benedizione è sul capo di chi lo vende.
\par 27 Chi procaccia il bene s'attira benevolenza, ma chi cerca il male, male gl'incoglierà.
\par 28 Chi confida nelle sue ricchezze cadrà, ma i giusti rinverdiranno a guisa di fronde.
\par 29 Chi getta lo scompiglio in casa sua erediterà vento, e lo stolto sarà lo schiavo di chi ha il cuor savio.
\par 30 Il frutto del giusto è un albero di vita, e il savio fa conquista d'anime.
\par 31 Ecco, il giusto riceve la sua retribuzione sulla terra, quanto più l'empio e il peccatore!

\chapter{12}

\par 1 Chi ama la correzione ama la scienza, ma chi odia la riprensione è uno stupido.
\par 2 L'uomo buono ottiene il favore dell'Eterno, ma l'Eterno condanna l'uomo pien di malizia.
\par 3 L'uomo non diventa stabile con l'empietà, ma la radice dei giusti non sarà mai smossa.
\par 4 La donna virtuosa è la corona del marito, ma quella che fa vergogna gli è un tarlo nell'ossa.
\par 5 I pensieri dei giusti sono equità, ma i disegni degli empi son frode.
\par 6 Le parole degli empi insidiano la vita, ma la bocca degli uomini retti procura liberazione.
\par 7 Gli empi, una volta rovesciati, non sono più, ma la casa dei giusti rimane in piedi.
\par 8 L'uomo è lodato in proporzione del suo senno, ma chi ha il cuore pervertito sarà sprezzato.
\par 9 È meglio essere in umile stato ed avere un servo, che fare il borioso e mancar di pane.
\par 10 Il giusto ha cura della vita del suo bestiame, ma le viscere degli empi sono crudeli.
\par 11 Chi coltiva la sua terra avrà pane da saziarsi, ma chi va dietro ai fannulloni è privo di senno.
\par 12 L'empio agogna la preda de' malvagi, ma la radice dei giusti porta il suo frutto.
\par 13 Nel peccato delle labbra sta un'insidia funesta, ma il giusto uscirà dalla distretta.
\par 14 Per il frutto della sua bocca l'uomo è saziato di beni, e ad ognuno è reso secondo l'opera delle sue mani.
\par 15 La via dello stolto è diritta agli occhi suoi, ma chi ascolta i consigli è savio.
\par 16 Lo stolto lascia scorger subito il suo cruccio, ma chi dissimula un affronto è uomo accorto.
\par 17 Chi dice la verità proclama ciò ch'è giusto, ma il falso testimonio parla con inganno.
\par 18 C'è chi, parlando inconsultamente, trafigge come spada, ma la lingua de' savi reca guarigione.
\par 19 Il labbro veridico è stabile in perpetuo, ma la lingua bugiarda non dura che un istante.
\par 20 L'inganno è nel cuore di chi macchina il male, ma per chi nutre propositi di pace v'è gioia.
\par 21 Nessun male incoglie al giusto, ma gli empi son pieni di guai.
\par 22 Le labbra bugiarde sono un abominio per l'Eterno, ma quelli che agiscono con sincerità gli sono graditi.
\par 23 L'uomo accorto nasconde quello che sa, ma il cuor degli stolti proclama la loro follia.
\par 24 La mano dei diligenti dominerà, ma la pigra sarà tributaria.
\par 25 Il cordoglio ch'è nel cuore dell'uomo l'abbatte, ma la parola buona lo rallegra.
\par 26 Il giusto indica la strada al suo compagno, ma la via degli empi li fa smarrire.
\par 27 Il pigro non arrostisce la sua caccia, ma la solerzia è per l'uomo un tesoro prezioso.
\par 28 Nel sentiero della giustizia sta la vita, e nella via ch'essa traccia non v'è morte.

\chapter{13}

\par 1 Il figliuol savio ascolta l'istruzione di suo padre, ma il beffardo non ascolta rimproveri.
\par 2 Per il frutto delle sue labbra uno gode del bene, ma il desiderio dei perfidi è la violenza.
\par 3 Chi custodisce la sua bocca preserva la propria vita; chi apre troppo le labbra va incontro alla rovina.
\par 4 L'anima del pigro desidera, e non ha nulla, ma l'anima dei diligenti sarà soddisfatta appieno.
\par 5 Il giusto odia la menzogna, ma l'empio getta sugli altri vituperio ed onta.
\par 6 La giustizia protegge l'uomo che cammina nella integrità, ma l'empietà atterra il peccatore.
\par 7 C'è chi fa il ricco e non ha nulla; c'è chi fa il povero e ha di gran beni.
\par 8 La ricchezza d'un uomo serve come riscatto della sua vita, ma il povero non ode mai minacce.
\par 9 La luce dei giusti è gaia, ma la lampada degli empi si spegne.
\par 10 Dall'orgoglio non vien che contesa, ma la sapienza è con chi dà retta ai consigli.
\par 11 La ricchezza male acquistata va scemando, ma chi accumula a poco a poco l'aumenta.
\par 12 La speranza differita fa languire il cuore, ma il desiderio adempiuto è un albero di vita.
\par 13 Chi sprezza la parola si costituisce, di fronte ad essa, debitore, ma chi rispetta il comandamento sarà ricompensato.
\par 14 L'insegnamento del savio è una fonte di vita per schivare le insidie della morte.
\par 15 Buon senno procura favore, ma il procedere dei perfidi è duro.
\par 16 Ogni uomo accorto agisce con conoscenza, ma l'insensato fa sfoggio di follia.
\par 17 Il messo malvagio cade in sciagure, ma l'ambasciatore fedele reca guarigione.
\par 18 Miseria e vergogna a chi rigetta la correzione, ma chi dà retta alla riprensione è onorato.
\par 19 Il desiderio adempiuto è dolce all'anima, ma agl'insensati fa orrore l'evitare il male.
\par 20 Chi va coi savi diventa savio, ma il compagno degl'insensati diventa cattivo.
\par 21 Il male perseguita i peccatori ma il giusto è ricompensato col bene.
\par 22 L'uomo buono lascia una eredità ai figli de' suoi figli, ma la ricchezza del peccatore è riserbata al giusto.
\par 23 Il campo lavorato dal povero dà cibo in abbondanza, ma v'è chi perisce per mancanza di equità.
\par 24 Chi risparmia la verga odia il suo figliuolo, ma chi l'ama, lo corregge per tempo.
\par 25 Il giusto ha di che mangiare a sazietà, ma il ventre degli empi manca di cibo.

\chapter{14}

\par 1 La donna savia edifica la sua casa, ma la stolta l'abbatte con le proprie mani.
\par 2 Chi cammina nella rettitudine teme l'Eterno, ma chi è pervertito nelle sue vie lo sprezza.
\par 3 Nella bocca dello stolto germoglia la superbia, ma le labbra dei savi son la loro custodia.
\par 4 Dove mancano i buoi è vuoto il granaio, ma l'abbondanza della raccolta sta nella forza del bove.
\par 5 Il testimonio fedele non mentisce, ma il testimonio falso spaccia menzogne.
\par 6 Il beffardo cerca la sapienza e non la trova, ma per l'uomo intelligente la scienza è cosa facile.
\par 7 Vattene lungi dallo stolto; sulle sue labbra certo non hai trovato scienza.
\par 8 La sapienza dell'uomo accorto sta nel discernere la propria strada; ma la follia degli stolti non è che inganno.
\par 9 Gli insensati si burlano delle colpe commesse, ma il favore dell'Eterno sta fra gli uomini retti.
\par 10 Il cuore conosce la sua propria amarezza, e alla sua gioia non può prender parte un estraneo.
\par 11 La casa degli empi sarà distrutta, ma la tenda degli uomini retti fiorirà.
\par 12 V'è tal via che all'uomo par dritta, ma finisce col menare alla morte.
\par 13 Anche ridendo, il cuore può esser triste; e l'allegrezza può finire in dolore.
\par 14 Lo sviato di cuore avrà la ricompensa dal suo modo di vivere, e l'uomo dabbene, quella delle opere sue.
\par 15 Lo scemo crede tutto quel che si dice, ma l'uomo prudente bada ai suoi passi.
\par 16 Il savio teme, ed evita il male; ma lo stolto è arrogante e presuntuoso.
\par 17 Chi è pronto all'ira commette follie, e l'uomo pien di malizia diventa odioso.
\par 18 Gli scemi ereditano stoltezza, ma i prudenti s'incoronano di scienza.
\par 19 I malvagi si chinano dinanzi ai buoni, e gli empi alle porte de' giusti.
\par 20 Il povero è odiato anche dal suo compagno, ma gli amici del ricco son molti.
\par 21 Chi sprezza il prossimo pecca, ma beato chi ha pietà dei miseri!
\par 22 Quelli che meditano il male non son forse traviati? ma quelli che meditano il bene trovan grazia e fedeltà.
\par 23 In ogni fatica v'è profitto, ma il chiacchierare mena all'indigenza.
\par 24 La corona de' savi è la loro ricchezza, ma la follia degli stolti non è che follia.
\par 25 Il testimonio verace salva delle vite, ma chi spaccia bugie non fa che ingannare.
\par 26 V'è una gran sicurezza nel timor dell'Eterno; Egli sarà un rifugio per i figli di chi lo teme.
\par 27 Il timor dell'Eterno è fonte di vita e fa schivare le insidie della morte.
\par 28 La moltitudine del popolo è la gloria del re, ma la scarsezza de' sudditi è la rovina del principe.
\par 29 Chi è lento all'ira ha un gran buon senso, ma chi è pronto ad andare in collera mostra la sua follia.
\par 30 Un cuor calmo è la vita del corpo, ma l'invidia è la carie dell'ossa.
\par 31 Chi opprime il povero oltraggia Colui che l'ha fatto, ma chi ha pietà del bisognoso, l'onora.
\par 32 L'empio è travolto dalla sua sventura, ma il giusto spera anche nella morte.
\par 33 La sapienza riposa nel cuore dell'uomo intelligente, ma in mezzo agli stolti si fa tosto conoscere.
\par 34 La giustizia innalza una nazione, ma il peccato è la vergogna dei popoli.
\par 35 Il favore del re è per il servo prudente, ma la sua ira è per chi gli fa onta.

\chapter{15}

\par 1 La risposta dolce calma il furore, ma la parola dura eccita l'ira.
\par 2 La lingua dei savi è ricca di scienza, ma la bocca degli stolti sgorga follia.
\par 3 Gli occhi dell'Eterno sono in ogni luogo, osservando i cattivi ed i buoni.
\par 4 La lingua che calma, è un albero di vita; ma la lingua perversa strazia lo spirito.
\par 5 L'insensato disdegna l'istruzione di suo padre, ma chi tien conto della riprensione diviene accorto.
\par 6 Nella casa del giusto v'è grande abbondanza, ma nell'entrate dell'empio c'è turbolenza.
\par 7 Le labbra dei savi spargono scienza, ma non così il cuore degli stolti.
\par 8 Il sacrifizio degli empi è in abominio all'Eterno, ma la preghiera degli uomini retti gli è grata.
\par 9 La via dell'empio è in abominio all'Eterno, ma egli ama chi segue la giustizia.
\par 10 Una dura correzione aspetta chi lascia la diritta via; chi odia la riprensione morrà.
\par 11 Il soggiorno de' morti e l'abisso stanno dinanzi all'Eterno; quanto più i cuori de' figliuoli degli uomini!
\par 12 Il beffardo non ama che altri lo riprenda; egli non va dai savi.
\par 13 Il cuore allegro rende ilare il volto, ma quando il cuore è triste, lo spirito è abbattuto.
\par 14 Il cuor dell'uomo intelligente cerca la scienza, ma la bocca degli stolti si pasce di follia.
\par 15 Tutt'i giorni dell'afflitto sono cattivi, ma il cuor contento è un convito perenne.
\par 16 Meglio poco col timor dell'Eterno, che gran tesoro con turbolenza.
\par 17 Meglio un piatto d'erbe, dov'è l'amore, che un bove ingrassato, dov'è l'odio.
\par 18 L'uomo iracondo fa nascere contese, ma chi è lento all'ira acqueta le liti.
\par 19 La via del pigro è come una siepe di spine, ma il sentiero degli uomini retti è piano.
\par 20 Il figliuol savio rallegra il padre, ma l'uomo stolto disprezza sua madre.
\par 21 La follia è una gioia per chi è privo di senno, ma l'uomo prudente cammina retto per la sua via.
\par 22 I disegni falliscono, dove mancano i consigli; ma riescono, dove son molti i consiglieri.
\par 23 Uno prova allegrezza quando risponde bene; e com'è buona una parola detta a tempo!
\par 24 Per l'uomo sagace la via della vita mena in alto, e gli fa evitare il soggiorno de' morti, in basso.
\par 25 L'Eterno spianta la casa dei superbi, ma rende stabili i confini della vedova.
\par 26 I pensieri malvagi sono in abominio all'Eterno, ma le parole benevole son pure agli occhi suoi.
\par 27 Chi è avido di lucro conturba la sua casa, ma chi odia i regali vivrà.
\par 28 Il cuor del giusto medita la sua risposta, ma la bocca degli empi sgorga cose malvage.
\par 29 L'Eterno è lungi dagli empi, ma ascolta la preghiera dei giusti.
\par 30 Uno sguardo lucente rallegra il cuore; una buona notizia impingua l'ossa.
\par 31 L'orecchio attento alla riprensione che mena a vita, dimorerà fra i savi.
\par 32 Chi rigetta l'istruzione disprezza l'anima sua, ma chi dà retta alla riprensione acquista senno.
\par 33 Il timor dell'Eterno è scuola di sapienza; e l'umiltà precede la gloria.

\chapter{16}

\par 1 All'uomo, i disegni del cuore; ma la risposta della lingua vien dall'Eterno.
\par 2 Tutte le vie dell'uomo a lui sembran pure, ma l'Eterno pesa gli spiriti.
\par 3 Rimetti le cose tue nell'Eterno, e i tuoi disegni avran buona riuscita.
\par 4 L'Eterno ha fatto ogni cosa per uno scopo; anche l'empio, per il dì della sventura.
\par 5 Chi è altero d'animo è in abominio all'Eterno; certo è che non rimarrà impunito.
\par 6 Con la bontà e con la fedeltà l'iniquità si espia, e col timor dell'Eterno si evita il male.
\par 7 Quando l'Eterno gradisce le vie d'un uomo, riconcilia con lui anche i nemici.
\par 8 Meglio poco con giustizia, che grandi entrate senza equità.
\par 9 Il cuor dell'uomo medita la sua via, ma l'Eterno dirige i suoi passi.
\par 10 Sulle labbra del re sta una sentenza divina; quando pronunzia il giudizio la sua bocca non erra.
\par 11 La stadera e le bilance giuste appartengono all'Eterno, tutti i pesi del sacchetto son opera sua.
\par 12 I re hanno orrore di fare il male, perché il trono è reso stabile con la giustizia.
\par 13 Le labbra giuste sono gradite ai re; essi amano chi parla rettamente.
\par 14 Ira del re vuol dire messaggeri di morte, ma l'uomo savio la placherà.
\par 15 La serenità del volto del re dà la vita, e il suo favore è come nube di pioggia primaverile.
\par 16 L'acquisto della sapienza oh quanto è migliore di quello dell'oro, e l'acquisto dell'intelligenza preferibile a quel dell'argento!
\par 17 La strada maestra dell'uomo retto è evitare il male; chi bada alla sua via preserva l'anima sua.
\par 18 La superbia precede la rovina, e l'alterezza dello spirito precede la caduta.
\par 19 Meglio esser umile di spirito coi miseri, che spartir la preda coi superbi.
\par 20 Chi presta attenzione alla Parola se ne troverà bene, e beato colui che confida nell'Eterno!
\par 21 Il savio di cuore è chiamato intelligente, e la dolcezza delle labbra aumenta il sapere.
\par 22 Il senno, per chi lo possiede, è fonte di vita, ma la stoltezza è il castigo degli stolti.
\par 23 Il cuore del savio gli rende assennata la bocca, e aumenta il sapere sulle sue labbra.
\par 24 Le parole soavi sono un favo di miele: dolcezza all'anima, salute al corpo.
\par 25 V'è tal via che all'uomo par diritta, ma finisce col menare alla morte.
\par 26 La fame del lavoratore lavora per lui, perché la sua bocca lo stimola.
\par 27 L'uomo cattivo va scavando ad altri del male; sulle sue labbra c'è come un fuoco divorante.
\par 28 L'uomo perverso semina contese, e il maldicente disunisce gli amici migliori.
\par 29 L'uomo violento trascina il compagno, e lo mena per una via non buona.
\par 30 Chi chiude gli occhi per macchinar cose perverse, chi si morde le labbra, ha già compiuto il male.
\par 31 I capelli bianchi sono una corona d'onore; la si trova sulla via della giustizia.
\par 32 Chi è lento all'ira val più del prode guerriero; chi padroneggia se stesso val più di chi espugna città.
\par 33 Si gettan le sorti nel grembo, ma ogni decisione vien dall'Eterno.

\chapter{17}

\par 1 È meglio un tozzo di pan secco con la pace, che una casa piena di carni con la discordia.
\par 2 Il servo sagace dominerà sul figlio che fa onta, e avrà parte all'eredità insieme coi fratelli.
\par 3 La coppella è per l'argento e il fornello per l'oro, ma chi prova i cuori è l'Eterno.
\par 4 Il malvagio dà ascolto alle labbra inique, e il bugiardo dà retta alla cattiva lingua.
\par 5 Chi beffa il povero oltraggia Colui che l'ha fatto; chi si rallegra dell'altrui sventura non rimarrà impunito.
\par 6 I figliuoli de' figliuoli son la corona de' vecchi, e i padri son la gloria dei loro figliuoli.
\par 7 Un parlar solenne non s'addice all'uomo da nulla; quanto meno s'addicono ad un principe labbra bugiarde!
\par 8 Il regalo è una pietra preziosa agli occhi di chi lo possiede; dovunque si volga, egli riesce.
\par 9 Chi copre i falli si procura amore, ma chi sempre vi torna su, disunisce gli amici migliori.
\par 10 Un rimprovero fa più impressione all'uomo intelligente, che cento percosse allo stolto.
\par 11 Il malvagio non cerca che ribellione, ma un messaggero crudele gli sarà mandato contro.
\par 12 Meglio imbattersi in un'orsa derubata dei suoi piccini, che in un insensato nella sua follia.
\par 13 Il male non si dipartirà dalla casa di chi rende il male per il bene.
\par 14 Cominciare una contesa è dar la stura all'acqua; perciò ritirati prima che la lite s'inasprisca.
\par 15 Chi assolve il reo e chi condanna il giusto sono ambedue in abominio all'Eterno.
\par 16 A che serve il danaro in mano allo stolto? ad acquistar saviezza?... Ma se non ha senno.
\par 17 L'amico ama in ogni tempo; è nato per essere un fratello nella distretta.
\par 18 L'uomo privo di senno dà la mano e fa sicurtà per altri davanti al suo prossimo.
\par 19 Chi ama le liti ama il peccato; chi alza troppo la sua porta, cerca la rovina.
\par 20 Chi ha il cuor falso non trova bene, e chi ha la lingua perversa cade nella sciagura.
\par 21 Chi genera uno stolto ne avrà cordoglio, e il padre dell'uomo da nulla non avrà gioia.
\par 22 Un cuore allegro è un buon rimedio, ma uno spirito abbattuto secca l'ossa.
\par 23 L'empio accetta regali di sottomano per pervertire le vie della giustizia.
\par 24 La sapienza sta dinanzi a chi ha intelligenza, ma gli occhi dello stolto vagano agli estremi confini della terra.
\par 25 Il figliuolo stolto è il cordoglio del padre e l'amarezza di colei che l'ha partorito.
\par 26 Non è bene condannare il giusto, foss'anche ad un'ammenda, né colpire i principi per la loro probità.
\par 27 Chi modera le sue parole possiede la scienza, e chi ha lo spirito calmo è un uomo prudente.
\par 28 Anche lo stolto, quando tace, passa per savio; chi tien chiuse le labbra è uomo intelligente.

\chapter{18}

\par 1 Chi si separa dagli altri cerca la propria soddisfazione e s'arrabbia contro tutto ciò ch'è profittevole.
\par 2 Lo stolto prende piacere, non nella prudenza, ma soltanto nel manifestare ciò che ha nel cuore.
\par 3 Quando viene l'empio, viene anche lo sprezzo; e, con la vergogna, viene l'obbrobrio.
\par 4 Le parole della bocca d'un uomo sono acque profonde; la fonte di sapienza è un rivo che scorre perenne.
\par 5 Non è bene aver per l'empio dei riguardi personali, per far torto al giusto nel giudizio.
\par 6 Le labbra dello stolto menano alle liti, e la sua bocca chiama le percosse.
\par 7 La bocca dello stolto è la sua rovina, e le sue labbra, sono un laccio per l'anima sua.
\par 8 Le parole del maldicente son come ghiottonerie, e penetrano fino nell'intimo delle viscere.
\par 9 Anche colui ch'è infingardo nel suo lavoro è fratello del dissipatore.
\par 10 Il nome dell'Eterno è una forte torre; il giusto vi corre, e vi trova un alto rifugio.
\par 11 I beni del ricco son la sua città forte; son come un'alta muraglia... nella sua immaginazione.
\par 12 Prima della rovina, il cuor dell'uomo s'innalza, ma l'umiltà precede la gloria.
\par 13 Chi risponde prima d'aver ascoltato, mostra la sua follia, e rimane confuso.
\par 14 Lo spirito dell'uomo lo sostiene quand'egli è infermo; ma lo spirito abbattuto chi lo solleverà?
\par 15 Il cuore dell'uomo intelligente acquista la scienza, e l'orecchio dei savi la cerca.
\par 16 I regali che uno fa gli apron la strada e gli danno adito ai grandi.
\par 17 Il primo a perorare la propria causa par che abbia ragione; ma vien l'altra parte, e scruta quello a fondo.
\par 18 La sorte fa cessare le liti e decide fra i grandi.
\par 19 Un fratello offeso è più inespugnabile d'una città forte; e le liti tra fratelli son come le sbarre d'un castello.
\par 20 Col frutto della sua bocca l'uomo sazia il corpo; si sazia col provento delle sue labbra.
\par 21 Morte e vita sono in potere della lingua; chi l'ama ne mangerà i frutti.
\par 22 Chi ha trovato moglie ha trovato un bene e ha ottenuto un favore dall'Eterno.
\par 23 Il povero parla supplicando, e il ricco risponde con durezza.
\par 24 Chi ha molti amici li ha per sua disgrazia; ma v'è tale amico, ch'è più affezionato d'un fratello.

\chapter{19}

\par 1 Meglio un povero che cammina nella sua integrità, di colui ch'è perverso di labbra ed anche stolto.
\par 2 L'ardore stesso, senza conoscenza, non è cosa buona: e chi cammina in fretta sbaglia strada.
\par 3 La stoltezza dell'uomo ne perverte la via, ma il cuor di lui s'irrita contro l'Eterno.
\par 4 Le ricchezze procurano gran numero d'amici, ma il povero è abbandonato anche dal suo compagno.
\par 5 Il falso testimonio non rimarrà impunito, e chi spaccia menzogne non avrà scampo.
\par 6 Molti corteggiano l'uomo generoso, e tutti sono amici dell'uomo munificente.
\par 7 Tutti i fratelli del povero l'odiano; quanto più gli amici suoi s'allontaneranno da lui! Ei li sollecita con parole, ma già sono scomparsi.
\par 8 Chi acquista senno ama l'anima sua; e chi serba con cura la prudenza troverà del bene.
\par 9 Il falso testimonio non rimarrà impunito, e chi spaccia menzogne perirà.
\par 10 Vivere in delizie non s'addice allo stolto; quanto meno s'addice allo schiavo dominare sui principi!
\par 11 Il senno rende l'uomo lento all'ira, ed egli stima sua gloria il passar sopra le offese.
\par 12 L'ira del re è come il ruggito d'un leone, ma il suo favore è come rugiada sull'erba.
\par 13 Un figliuolo stolto è una grande sciagura per suo padre, e le risse d'una moglie sono il gocciolar continuo d'un tetto.
\par 14 Casa e ricchezze sono un'eredità dei padri, ma una moglie giudiziosa è un dono dell'Eterno.
\par 15 La pigrizia fa cadere nel torpore, e l'anima indolente patirà la fame.
\par 16 Chi osserva il comandamento ha cura dell'anima sua, ma chi non si dà pensiero della propria condotta morrà.
\par 17 Chi ha pietà del povero presta all'Eterno, che gli contraccambierà l'opera buona.
\par 18 Castiga il tuo figliuolo, mentre c'è ancora speranza, ma non ti lasciar andare sino a farlo morire.
\par 19 L'uomo dalla collera violenta dev'esser punito; ché, se lo scampi, dovrai tornare daccapo.
\par 20 Ascolta il consiglio e ricevi l'istruzione, affinché tu diventi savio per il resto della vita.
\par 21 Ci sono molti disegni nel cuor dell'uomo, ma il piano dell'Eterno è quello che sussiste.
\par 22 Ciò che rende caro l'uomo è la bontà, e un povero val più d'un bugiardo.
\par 23 Il timor dell'Eterno mena alla vita; chi l'ha si sazia, e passa la notte non visitato da alcun male.
\par 24 Il pigro tuffa la mano nel piatto, e non fa neppure tanto da portarla alla bocca.
\par 25 Percuoti il beffardo, e il semplice si farà accorto; riprendi l'intelligente, e imparerà la scienza.
\par 26 Il figlio che fa vergogna e disonore, rovina suo padre e scaccia sua madre.
\par 27 Cessa, figliuol mio, d'ascoltar l'istruzione, se ti vuoi allontanare dalle parole della scienza.
\par 28 Il testimonio iniquo si burla della giustizia, e la bocca degli empi trangugia l'iniquità.
\par 29 I giudicî son preparati per i beffardi e le percosse per il dosso degli stolti.

\chapter{20}

\par 1 Il vino è schernitore, la bevanda alcoolica è turbolenta, e chiunque se ne lascia sopraffare non è savio.
\par 2 Il terrore che incute il re è come il ruggito d'un leone; chi lo irrita pecca contro la propria vita.
\par 3 È una gloria per l'uomo l'astenersi dalle contese, ma chiunque è insensato mostra i denti.
\par 4 Il pigro non ara a causa del freddo; alla raccolta verrà a cercare, ma non ci sarà nulla.
\par 5 I disegni nel cuor dell'uomo sono acque profonde, ma l'uomo intelligente saprà attingervi.
\par 6 Molta gente vanta la propria bontà; ma un uomo fedele chi lo troverà?
\par 7 I figliuoli del giusto, che cammina nella sua integrità, saranno beati dopo di lui.
\par 8 Il re, assiso sul trono dove rende giustizia, dissipa col suo sguardo ogni male.
\par 9 Chi può dire: 'Ho nettato il mio cuore, sono puro dal mio peccato?'
\par 10 Doppio peso e doppia misura sono ambedue in abominio all'Eterno.
\par 11 Anche il fanciullo dà a conoscere con i suoi atti se la sua condotta sarà pura e retta.
\par 12 L'orecchio che ascolta e l'occhio che vede, li ha fatti ambedue l'Eterno.
\par 13 Non amare il sonno, che tu non abbia a impoverire; tieni aperti gli occhi, e avrai pane da saziarti.
\par 14 'Cattivo! cattivo!' dice il compratore; ma, andandosene, si vanta dell'acquisto.
\par 15 C'è dell'oro e abbondanza di perle, ma le labbra ricche di scienza son cosa più preziosa.
\par 16 Prendigli il vestito, giacché ha fatta cauzione per altri; fatti dare dei pegni, poiché s'è reso garante di stranieri.
\par 17 Il pane frodato è dolce all'uomo; ma, dopo, avrà la bocca piena di ghiaia.
\par 18 I disegni son resi stabili dal consiglio; fa' dunque la guerra con una savia direzione.
\par 19 Chi va sparlando palesa i segreti; perciò non t'immischiare con chi apre troppo le labbra.
\par 20 Chi maledice suo padre e sua madre, la sua lucerna si spegnerà nelle tenebre più fitte.
\par 21 L'eredità acquistata troppo presto da principio, alla fine non sarà benedetta.
\par 22 Non dire: 'Renderò il male'; spera nell'Eterno, ed egli ti salverà.
\par 23 Il peso doppio è in abominio all'Eterno, e la bilancia falsa non è cosa buona.
\par 24 I passi dell'uomo li dirige l'Eterno; come può quindi l'uomo capir la propria via?
\par 25 È pericoloso per l'uomo prender leggermente un impegno sacro, e non riflettere che dopo aver fatto un voto.
\par 26 Il re savio passa gli empi al vaglio, dopo aver fatto passare la ruota su loro.
\par 27 Lo spirito dell'uomo è una lucerna dell'Eterno che scruta tutti i recessi del cuore.
\par 28 La bontà e la fedeltà custodiscono il re; e con la bontà egli rende stabile il suo trono.
\par 29 La gloria dei giovani sta nella loro forza, e la bellezza dei vecchi, nella loro canizie.
\par 30 Le battiture che piagano guariscono il male; e così le percosse che vanno al fondo delle viscere.

\chapter{21}

\par 1 Il cuore del re, nella mano dell'Eterno, è come un corso d'acqua; egli lo volge dovunque gli piace.
\par 2 Tutte le vie dell'uomo gli paion diritte, ma l'Eterno pesa i cuori.
\par 3 Praticare la giustizia e l'equità è cosa che l'Eterno preferisce ai sacrifizi.
\par 4 Gli occhi alteri e il cuor gonfio, lucerna degli empi, sono peccato.
\par 5 I disegni dell'uomo diligente menano sicuramente all'abbondanza, ma chi troppo s'affretta non fa che cader nella miseria.
\par 6 I tesori acquistati con lingua bugiarda sono un soffio fugace di gente che cerca la morte.
\par 7 La violenza degli empi li porta via, perché rifiutano di praticare l'equità.
\par 8 La via del colpevole è tortuosa, ma l'innocente opera con rettitudine.
\par 9 Meglio abitare sul canto d'un tetto, che una gran casa con una moglie rissosa.
\par 10 L'anima dell'empio desidera il male; il suo amico stesso non trova pietà agli occhi di lui.
\par 11 Quando il beffardo è punito, il semplice diventa savio; e quando s'istruisce il savio, egli acquista scienza.
\par 12 Il Giusto tien d'occhio la casa dell'empio, e precipita gli empi nelle sciagure.
\par 13 Chi chiude l'orecchio al grido del povero, griderà anch'egli, e non gli sarà risposto.
\par 14 Un dono fatto in segreto placa la collera, e un regalo dato di sottomano, l'ira violenta.
\par 15 Far ciò ch'è retto è una gioia per il giusto, ma è una rovina per gli artefici d'iniquità.
\par 16 L'uomo che erra lungi dalle vie del buon senso, riposerà nell'assemblea dei trapassati.
\par 17 Chi ama godere sarà bisognoso, chi ama il vino e l'olio non arricchirà.
\par 18 L'empio serve di riscatto al giusto; e il perfido, agli uomini retti.
\par 19 Meglio abitare in un deserto, che con una donna rissosa e stizzosa.
\par 20 In casa del savio c'è dei tesori preziosi e dell'olio, ma l'uomo stolto dà fondo a tutto.
\par 21 Chi ricerca la giustizia e la bontà troverà vita, giustizia e gloria.
\par 22 Il savio dà la scalata alla città dei forti, e abbatte il baluardo in cui essa confidava.
\par 23 Chi custodisce la sua bocca e la sua lingua preserva l'anima sua dalle distrette.
\par 24 Il nome del superbo insolente è: beffardo; egli fa ogni cosa con furore di superbia.
\par 25 I desideri del pigro l'uccidono perché le sue mani rifiutano di lavorare.
\par 26 C'è chi da mane a sera brama avidamente, ma il giusto dona senza mai rifiutare.
\par 27 Il sacrifizio dell'empio è cosa abominevole; quanto più se l'offre con intento malvagio!
\par 28 Il testimonio bugiardo perirà, ma l'uomo che ascolta potrà sempre parlare.
\par 29 L'empio fa la faccia tosta, ma l'uomo retto rende ferma la sua condotta.
\par 30 Non c'è sapienza, non intelligenza, non consiglio che valga contro l'Eterno.
\par 31 Il cavallo è pronto per il dì della battaglia, ma la vittoria appartiene all'Eterno.

\chapter{22}

\par 1 La buona riputazione è da preferirsi alle molte ricchezze; e la stima, all'argento e all'oro.
\par 2 Il ricco e il povero s'incontrano; l'Eterno li ha fatti tutti e due.
\par 3 L'uomo accorto vede venire il male, e si nasconde; ma i semplici tirano innanzi, e ne portan la pena.
\par 4 Il frutto dell'umiltà e del timor dell'Eterno è ricchezza e gloria e vita.
\par 5 Spine e lacci sono sulla via del perverso; chi ha cura dell'anima sua se ne tien lontano.
\par 6 Inculca al fanciullo la condotta che deve tenere; anche quando sarà vecchio non se ne dipartirà.
\par 7 Il ricco signoreggia sui poveri, e chi prende in prestito è schiavo di chi presta.
\par 8 Chi semina iniquità miete sciagura, e la verga della sua collera è infranta.
\par 9 L'uomo dallo sguardo benevolo sarà benedetto, perché dà del suo pane al povero.
\par 10 Caccia via il beffardo, se n'andranno le contese, e cesseran le liti e gli oltraggi.
\par 11 Chi ama la purità del cuore e ha la grazia sulle labbra, ha il re per amico.
\par 12 Gli occhi dell'Eterno proteggono la scienza, ma egli rende vane le parole del perfido.
\par 13 Il pigro dice: 'Là fuori c'è un leone; sarò ucciso per la strada'.
\par 14 La bocca delle donne corrotte è una fossa profonda; colui ch'è in ira all'Eterno, vi cadrà dentro.
\par 15 La follia è legata al cuore del fanciullo, ma la verga della correzione l'allontanerà da lui.
\par 16 Chi opprime il povero, l'arricchisce; chi dona al ricco, non fa che impoverirlo.
\par 17 Porgi l'orecchio e ascolta le parole dei Savi ed applica il cuore alla mia scienza.
\par 18 Ti sarà dolce custodirle in petto, e averle tutte pronte sulle tue labbra.
\par 19 Ho voluto istruirti oggi, sì, proprio te, perché la tua fiducia sia posta nell'Eterno.
\par 20 Non ho io già da tempo scritto per te consigli e insegnamenti
\par 21 per farti conoscere cose certe, parole vere, onde tu possa risponder parole vere a chi t'interroga?
\par 22 Non derubare il povero perch'è povero, e non opprimere il misero alla porta;
\par 23 ché l'Eterno difenderà la loro causa, e spoglierà della vita chi avrà spogliato loro.
\par 24 Non fare amicizia con l'uomo iracondo e non andare con l'uomo violento,
\par 25 che tu non abbia ad imparare le sue vie e ad esporre a un'insidia l'anima tua.
\par 26 Non esser di quelli che dan la mano, che fanno sicurtà per debiti.
\par 27 Se non hai di che pagare, perché esporti a farti portar via il letto?
\par 28 Non spostare il termine antico, che fu messo dai tuoi padri.
\par 29 Hai tu veduto un uomo spedito nelle sue faccende? Egli starà al servizio dei re; non starà al servizio della gente oscura.

\chapter{23}

\par 1 Quando ti siedi a mensa con un principe, rifletti bene a chi ti sta dinanzi;
\par 2 e mettiti un coltello alla gola, se tu sei ingordo.
\par 3 Non bramare i suoi bocconi delicati; sono un cibo ingannatore.
\par 4 Non t'affannare per diventar ricco, smetti dall'applicarvi la tua intelligenza.
\par 5 Vuoi tu fissar lo sguardo su ciò che scompare? Giacché la ricchezza si fa dell'ali, come l'aquila che vola verso il cielo.
\par 6 Non mangiare il pane di chi ha l'occhio maligno, e non bramare i suoi cibi delicati;
\par 7 poiché, nell'intimo suo, egli è calcolatore: 'Mangia e bevi!' ti dirà; ma il cuor suo non è con te.
\par 8 Vomiterai il boccone che avrai mangiato, e avrai perduto le tue belle parole.
\par 9 Non rivolger la parola allo stolto, perché sprezzerà il senno de' tuoi discorsi.
\par 10 Non spostare il termine antico, e non entrare nei campi degli orfani;
\par 11 ché il Vindice loro è potente; egli difenderà la causa loro contro di te.
\par 12 Applica il tuo cuore all'istruzione, e gli orecchi alle parole della scienza.
\par 13 Non risparmiare la correzione al fanciullo; se lo batti con la verga, non ne morrà;
\par 14 lo batterai con la verga, ma libererai l'anima sua dal soggiorno de' morti.
\par 15 Figliuol mio, se il tuo cuore è savio, anche il mio cuore si rallegrerà;
\par 16 le viscere mie esulteranno quando le tue labbra diranno cose rette.
\par 17 Il tuo cuore non porti invidia ai peccatori, ma perseveri sempre nel timor dell'Eterno;
\par 18 poiché c'è un avvenire, e la tua speranza non sarà frustrata.
\par 19 Ascolta, figliuol mio, sii savio, e dirigi il cuore per la diritta via.
\par 20 Non esser di quelli che son bevitori di vino, che son ghiotti mangiatori di carne;
\par 21 ché il beone ed il ghiotto impoveriranno e i dormiglioni n'andran vestiti di cenci.
\par 22 Da' retta a tuo padre che t'ha generato, e non disprezzar tua madre quando sarà vecchia.
\par 23 Acquista verità e non la vendere, acquista sapienza, istruzione e intelligenza.
\par 24 Il padre del giusto esulta grandemente; chi ha generato un savio, ne avrà gioia.
\par 25 Possan tuo padre e tua madre rallegrarsi, e possa gioire colei che t'ha partorito!
\par 26 Figliuol mio, dammi il tuo cuore, e gli occhi tuoi prendano piacere nelle mie vie;
\par 27 perché la meretrice è una fossa profonda, e la straniera, un pozzo stretto.
\par 28 Anch'essa sta in agguato come un ladro, e accresce fra gli uomini il numero de' traditori.
\par 29 Per chi sono gli 'ahi'? per chi gli 'ahimè'? per chi le liti? per chi i lamenti? per chi le ferite senza ragione? per chi gli occhi rossi?
\par 30 Per chi s'indugia a lungo presso il vino, per quei che vanno a gustare il vin drogato.
\par 31 Non guardare il vino quando rosseggia, quando scintilla nel calice e va giù così facilmente!
\par 32 Alla fine, esso morde come un serpente e punge come un basilisco.
\par 33 I tuoi occhi vedranno cose strane, e il tuo cuore farà dei discorsi pazzi.
\par 34 Sarai come chi giace in mezzo al mare, come chi giace in cima a un albero di nave.
\par 35 Dirai: 'M'hanno picchiato... e non m'han fatto male; m'hanno percosso... e non me ne sono accorto; quando mi sveglierò?... tornerò a cercarne ancora!'

\chapter{24}

\par 1 Non portare invidia ai malvagi, e non desiderare di star con loro,
\par 2 perché il loro cuore medita rapine, e le loro labbra parlan di nuocere.
\par 3 La casa si edifica con la sapienza, e si rende stabile con la prudenza;
\par 4 mediante la scienza, se ne riempiono le stanze d'ogni specie di beni preziosi e gradevoli.
\par 5 L'uomo savio è pien di forza, e chi ha conoscimento accresce la sua potenza;
\par 6 infatti, con savie direzioni potrai condur bene la guerra, e la vittoria sta nel gran numero de' consiglieri.
\par 7 La sapienza è troppo in alto per lo stolto; egli non apre mai la bocca alla porta di città.
\par 8 Chi pensa a mal fare sarà chiamato esperto in malizia.
\par 9 I disegni dello stolto sono peccato, e il beffardo è l'abominio degli uomini.
\par 10 Se ti perdi d'animo nel giorno dell'avversità, la tua forza è poca.
\par 11 Libera quelli che son condotti a morte, e salva quei che, vacillando, vanno al supplizio.
\par 12 Se dici: 'Ma noi non ne sapevamo nulla!...' Colui che pesa i cuori, non lo vede egli? Colui che veglia sull'anima tua non lo sa forse? E non renderà egli a ciascuno secondo le opere sue?
\par 13 Figliuol mio, mangia del miele perché è buono; un favo di miele sarà dolce al tuo palato.
\par 14 Così conosci la sapienza per il bene dell'anima tua! Se la trovi, c'è un avvenire, e la speranza tua non sarà frustrata.
\par 15 O empio, non tendere insidie alla dimora del giusto! non devastare il luogo ove riposa!
\par 16 ché il giusto cade sette volte e si rialza, ma gli empi son travolti dalla sventura.
\par 17 Quando il tuo nemico cade, non ti rallegrare; quand'è rovesciato, il cuor tuo non ne gioisca,
\par 18 che l'Eterno nol vegga e gli dispiaccia e non storni l'ira sua da lui.
\par 19 Non t'irritare a motivo di chi fa il male, e non portare invidia agli empi;
\par 20 perché non c'è avvenire per il malvagio; la lucerna degli empi sarà spenta.
\par 21 Figliuol mio, temi l'Eterno e il re, e non far lega cogli amatori di novità;
\par 22 la loro calamità sopraggiungerà improvvisa, e chi sa la triste fine dei loro anni?
\par 23 Anche queste sono massime dei Savi. Non è bene, in giudizio, aver dei riguardi personali.
\par 24 Chi dice all'empio: 'Tu sei giusto', i popoli lo malediranno, lo esecreranno le nazioni.
\par 25 Ma quelli che sanno punire se ne troveranno bene, e su loro scenderanno benedizioni e prosperità.
\par 26 Dà un bacio sulle labbra chi dà una risposta giusta.
\par 27 Metti in buon ordine gli affari tuoi di fuori, metti in assetto i tuoi campi, poi ti fabbricherai la casa.
\par 28 Non testimoniare, senza motivo, contro il tuo prossimo; vorresti tu farti ingannatore con le tue parole?
\par 29 Non dire: 'Come ha fatto a me così farò a lui; renderò a costui secondo l'opera sua'.
\par 30 Passai presso il campo del pigro e presso la vigna dell'uomo privo di senno;
\par 31 ed ecco le spine vi crescean da per tutto, i rovi ne coprivano il suolo, e il muro di cinta era in rovina.
\par 32 Considerai la cosa, e mi posi a riflettere; e da quel che vidi trassi una lezione:
\par 33 Dormire un po', sonnecchiare un po', incrociare un po' le mani per riposare...
\par 34 e la tua povertà verrà come un ladro, e la tua indigenza, come un uomo armato.

\chapter{25}

\par 1 Ecco altri proverbi di Salomone, raccolti dalla gente di Ezechia, re di Giuda.
\par 2 È gloria di Dio nascondere le cose; ma la gloria dei re sta nell'investigarle.
\par 3 L'altezza del cielo, la profondità della terra e il cuore dei re non si possono investigare.
\par 4 Togli dall'argento le scorie, e ne uscirà un vaso per l'artefice,
\par 5 togli l'empio dalla presenza del re, e il suo trono sarà reso stabile dalla giustizia.
\par 6 Non fare il vanaglorioso in presenza del re, e non ti porre nel luogo dei grandi;
\par 7 poiché è meglio ti sia detto: 'Sali qui', anziché essere abbassato davanti al principe che gli occhi tuoi hanno veduto.
\par 8 Non t'affrettare a intentar processi, che alla fine tu non sappia che fare, quando il tuo prossimo t'avrà svergognato.
\par 9 Difendi la tua causa contro il tuo prossimo, ma non rivelare il segreto d'un altro,
\par 10 onde chi t'ode non t'abbia a vituperare, e la tua infamia non si cancelli più.
\par 11 Le parole dette a tempo son come pomi d'oro in vasi d'argento cesellato.
\par 12 Per un orecchio docile, chi riprende con saviezza è un anello d'oro, un ornamento d'oro fino.
\par 13 Il messaggero fedele, per quelli che lo mandano, è come il fresco della neve al tempo della mèsse; esso ristora l'anima del suo padrone.
\par 14 Nuvole e vento, ma punta pioggia; ecco l'uomo che si vanta falsamente della sua liberalità.
\par 15 Con la pazienza si piega un principe, e la lingua dolce spezza dell'ossa.
\par 16 Se trovi del miele, mangiane quanto ti basta; che, satollandotene, tu non abbia poi a vomitarlo.
\par 17 Metti di rado il piede in casa del prossimo, ond'egli, stufandosi di te, non abbia ad odiarti.
\par 18 L'uomo che attesta il falso contro il suo prossimo, è un martello, una spada, una freccia acuta.
\par 19 La fiducia in un perfido, nel dì della distretta, è un dente rotto, un piede slogato.
\par 20 Cantar delle canzoni a un cuor dolente è come togliersi l'abito in giorno di freddo, e mettere aceto sul nitro.
\par 21 Se il tuo nemico ha fame, dagli del pane da mangiare; se ha sete, dagli dell'acqua da bere;
\par 22 ché, così, raunerai dei carboni accesi sul suo capo, e l'Eterno ti ricompenserà.
\par 23 Il vento del nord porta la pioggia, e la lingua che sparla di nascosto fa oscurare il viso.
\par 24 Meglio abitare sul canto d'un tetto, che in una gran casa con una moglie rissosa.
\par 25 Una buona notizia da paese lontano è come acqua fresca a persona stanca ed assetata.
\par 26 Il giusto che vacilla davanti all'empio, è come una fontana torbida e una sorgente inquinata.
\par 27 Mangiar troppo miele non è bene, ma scrutare cose difficili è un onore.
\par 28 L'uomo che non si sa padroneggiare, è una città smantellata, priva di mura.

\chapter{26}

\par 1 Come la neve non conviene all'estate, né la pioggia al tempo della mèsse, così non conviene la gloria allo stolto.
\par 2 Come il passero vaga qua e là e la rondine vola, così la maledizione senza motivo, non raggiunge l'effetto.
\par 3 La frusta per il cavallo, la briglia per l'asino, e il bastone per il dosso degli stolti.
\par 4 Non rispondere allo stolto secondo la sua follia, che tu non gli abbia a somigliare.
\par 5 Rispondi allo stolto secondo la sua follia, perché non abbia a credersi savio.
\par 6 Chi affida messaggi a uno stolto si taglia i piedi e s'abbevera di pene.
\par 7 Come le gambe dello zoppo son senza forza, così è una massima in bocca degli stolti.
\par 8 Chi onora uno stolto fa come chi getta una gemma in un mucchio di sassi.
\par 9 Una massima in bocca agli stolti è come un ramo spinoso in mano a un ubriaco.
\par 10 Chi impiega lo stolto e il primo che capita, è come un arciere che ferisce tutti.
\par 11 Lo stolto che ricade nella sua follia, è come il cane che torna al suo vomito.
\par 12 Hai tu visto un uomo che si crede savio? C'è più da sperare da uno stolto che da lui.
\par 13 Il pigro dice: 'C'è un leone nella strada, c'è un leone per le vie!'
\par 14 Come la porta si volge sui cardini, così il pigro sul suo letto.
\par 15 Il pigro tuffa la mano nel piatto; gli par fatica riportarla alla bocca.
\par 16 Il pigro si crede più savio di sette uomini che danno risposte sensate.
\par 17 Il passante che si riscalda per una contesa che non lo concerne, è come chi afferra un cane per le orecchie.
\par 18 Come un pazzo che avventa tizzoni, frecce e morte,
\par 19 così è colui che inganna il prossimo, e dice: 'Ho fatto per ridere!'
\par 20 Quando mancan le legna, il fuoco si spegne; e quando non c'è maldicente, cessan le contese.
\par 21 Come il carbone dà la brace, e le legna danno la fiamma, così l'uomo rissoso accende le liti.
\par 22 Le parole del maldicente son come ghiottonerie, e penetrano fino nell'intimo delle viscere.
\par 23 Labbra ardenti e un cuor malvagio son come schiuma d'argento spalmata sopra un vaso di terra.
\par 24 Chi odia, parla con dissimulazione; ma, dentro, cova la frode.
\par 25 Quando parla con voce graziosa, non te ne fidare, perché ha sette abominazioni in cuore.
\par 26 L'odio suo si nasconde sotto la finzione, ma la sua malvagità si rivelerà nell'assemblea.
\par 27 Chi scava una fossa vi cadrà, e la pietra torna addosso a chi la rotola.
\par 28 La lingua bugiarda odia quelli che ha ferito, e la bocca lusinghiera produce rovina.

\chapter{27}

\par 1 Non ti vantare del domani, poiché non sai quel che un giorno possa produrre.
\par 2 Altri ti lodi, non la tua bocca; un estraneo, non le tue labbra.
\par 3 La pietra è grave e la rena pesante, ma l'irritazione dello stolto pesa più dell'uno e dell'altra.
\par 4 L'ira è crudele e la collera impetuosa; ma chi può resistere alla gelosia?
\par 5 Meglio riprensione aperta, che amore occulto.
\par 6 Fedeli son le ferite di chi ama; frequenti i baci di chi odia.
\par 7 Chi è sazio calpesta il favo di miele; ma, per chi ha fame, ogni cosa amara è dolce.
\par 8 Come l'uccello che va ramingo lungi dal nido, così è l'uomo che va ramingo lungi da casa.
\par 9 L'olio e il profumo rallegrano il cuore; così fa la dolcezza d'un amico coi suoi consigli cordiali.
\par 10 Non abbandonare il tuo amico né l'amico di tuo padre, e non andare in casa del tuo fratello nel dì della tua sventura; un vicino dappresso val meglio d'un fratello lontano.
\par 11 Figliuolo mio, sii savio e rallegrami il cuore, così potrò rispondere a chi mi vitupera.
\par 12 L'uomo accorto vede il male e si nasconde, ma gli scempi passan oltre e ne portan la pena.
\par 13 Prendigli il vestito giacché ha fatto cauzione per altri; fatti dare dei pegni, poiché s'è reso garante di stranieri.
\par 14 Chi benedice il prossimo ad alta voce, di buon mattino, sarà considerato come se lo maledicesse.
\par 15 Un gocciolar continuo in giorno di gran pioggia e una donna rissosa son cose che si somigliano.
\par 16 Chi la vuol trattenere vuol trattenere il vento, e stringer l'olio nella sua destra.
\par 17 Il ferro forbisce il ferro; così un uomo ne forbisce un altro.
\par 18 Chi ha cura del fico ne mangerà il frutto; e chi veglia sul suo padrone sarà onorato.
\par 19 Come nell'acqua il viso risponde al viso, così il cuor dell'uomo risponde al cuore dell'uomo.
\par 20 Il soggiorno dei morti e l'abisso sono insaziabili, e insaziabili son gli occhi degli uomini.
\par 21 Il crogiuolo è per l'argento, il forno fusorio per l'oro, e l'uomo è provato dalla bocca di chi lo loda.
\par 22 Anche se tu pestassi lo stolto in un mortaio in mezzo al grano col pestello, la sua follia non lo lascerebbe.
\par 23 Guarda di conoscer bene lo stato delle tue pecore, abbi gran cura delle tue mandre;
\par 24 perché le ricchezze non duran sempre, e neanche una corona dura d'età in età.
\par 25 Quando è levato il fieno, subito rispunta la fresca verdura e le erbe dei monti sono raccolte.
\par 26 Gli agnelli ti danno da vestire, i becchi di che comprarti un campo,
\par 27 e il latte delle capre basta a nutrir te, a nutrir la tua famiglia e a far vivere le tue serve.

\chapter{28}

\par 1 L'empio fugge senza che alcuno lo perseguiti, ma il giusto se ne sta sicuro come un leone.
\par 2 Per i suoi misfatti i capi d'un paese son numerosi, ma, con un uomo intelligente e pratico delle cose, l'ordine dura.
\par 3 Un povero che opprime i miseri è come una pioggia che devasta e non dà pane.
\par 4 Quelli che abbandonano la legge, lodano gli empi; ma quelli che l'osservano, fan loro la guerra.
\par 5 Gli uomini dati al male non comprendono ciò ch'è giusto, ma quelli che cercano l'Eterno comprendono ogni cosa.
\par 6 Meglio il povero che cammina nella sua integrità, del perverso che cammina nella doppiezza, ed è ricco.
\par 7 Chi osserva la legge è un figliuolo intelligente, ma il compagno dei ghiottoni fa vergogna a suo padre.
\par 8 Chi accresce i suoi beni con gl'interessi e l'usura, li aduna per colui che ha pietà dei poveri.
\par 9 Se uno volge altrove gli orecchi per non udire la legge, la sua stessa preghiera è un abominio.
\par 10 Chi induce i giusti a battere una mala via cadrà egli stesso nella fossa che ha scavata; ma gli uomini integri erediteranno il bene.
\par 11 Il ricco si reputa savio, ma il povero ch'è intelligente, lo scruta.
\par 12 Quando i giusti trionfano, la gloria è grande; ma, quando gli empi s'innalzano, la gente si nasconde.
\par 13 Chi copre le sue trasgressioni non prospererà, ma chi le confessa e le abbandona otterrà misericordia.
\par 14 Beato l'uomo ch'è sempre timoroso! ma chi indura il suo cuore cadrà nella sfortuna.
\par 15 Un empio che domina un popolo povero, è un leone ruggente, un orso affamato.
\par 16 Il principe senza prudenza fa molte estorsioni, ma chi odia il lucro disonesto prolunga i suoi giorni.
\par 17 L'uomo su cui pesa un omicidio, fuggirà fino alla fossa; nessuno lo fermi!
\par 18 Chi cammina integramente sarà salvato, ma il perverso che batte doppie vie, cadrà a un tratto.
\par 19 Chi lavora la sua terra avrà abbondanza di pane; ma chi va dietro ai fannulloni avrà abbondanza di miseria.
\par 20 L'uomo fedele sarà colmato di benedizioni, ma chi ha fretta d'arricchire non rimarrà impunito.
\par 21 Aver de' riguardi personali non è bene; per un pezzo di pane l'uomo talvolta diventa trasgressore.
\par 22 L'uomo invidioso ha fretta d'arricchire, e non sa che gli piomberà addosso la miseria.
\par 23 Chi riprende qualcuno gli sarà alla fine più accetto di chi lo lusinga con le sue parole.
\par 24 Chi ruba a suo padre e a sua madre e dice: 'Non è un delitto!', è compagno del dissipatore.
\par 25 Chi ha l'animo avido fa nascere contese, ma chi confida nell'Eterno sarà saziato.
\par 26 Chi confida nel proprio cuore è uno stolto, ma chi cammina saviamente scamperà.
\par 27 Chi dona al povero non sarà mai nel bisogno, ma colui che chiude gli occhi, sarà coperto di maledizioni.
\par 28 Quando gli empi s'innalzano, la gente si nasconde; ma quando periscono, si moltiplicano i giusti.

\chapter{29}

\par 1 L'uomo che, essendo spesso ripreso, irrigidisce il collo, sarà di subito fiaccato, senza rimedio.
\par 2 Quando i giusti son numerosi, il popolo si rallegra: ma quando domina l'empio, il popolo geme.
\par 3 L'uomo che ama la sapienza, rallegra suo padre; ma chi frequenta le meretrici dissipa i suoi beni.
\par 4 Il re, con la giustizia, rende stabile il paese; ma chi pensa solo a imporre tasse, lo rovina.
\par 5 L'uomo che lusinga il prossimo, gli tende una rete davanti ai piedi.
\par 6 Nella trasgressione del malvagio v'è un'insidia; ma il giusto canta e si rallegra.
\par 7 Il giusto prende conoscenza della causa de' miseri, ma l'empio non ha intendimento né conoscenza.
\par 8 I beffardi soffian nel fuoco delle discordie cittadine, ma i savi calmano le ire.
\par 9 Se un savio viene a contesa con uno stolto, quello va in collera e ride, e non c'è da intendersi.
\par 10 Gli uomini di sangue odiano chi è integro, ma gli uomini retti ne proteggono la vita.
\par 11 Lo stolto dà sfogo a tutta la sua ira, ma il savio rattiene la propria.
\par 12 Quando il sovrano dà retta alle parole menzognere, tutti i suoi ministri sono empi.
\par 13 Il povero e l'oppressore s'incontrano; l'Eterno illumina gli occhi d'ambedue.
\par 14 Il re che fa ragione ai miseri secondo verità, avrà il trono stabilito in perpetuo.
\par 15 La verga e la riprensione danno sapienza; ma il fanciullo lasciato a se stesso, fa vergogna a sua madre.
\par 16 Quando abbondano gli empi, abbondano le trasgressioni; ma i giusti ne vedranno la ruina.
\par 17 Correggi il tuo figliuolo; egli ti darà conforto, e procurerà delizie all'anima tua.
\par 18 Quando non c'è visioni, il popolo è senza freno; ma beato colui che osserva la legge!
\par 19 Uno schiavo non si corregge a parole; anche se comprende, non ubbidisce.
\par 20 Hai tu visto un uomo precipitoso nel suo parlare? C'è più da sperare da uno stolto che da lui.
\par 21 Se uno alleva delicatamente da fanciullo il suo servo, questo finirà per voler essere figliuolo.
\par 22 L'uomo iracondo fa nascere contese, e l'uomo collerico abbonda in trasgressioni.
\par 23 L'orgoglio abbassa l'uomo, ma chi è umile di spirito ottiene gloria.
\par 24 Chi fa società col ladro odia l'anima sua; egli ode la esecrazione e non dice nulla.
\par 25 La paura degli uomini costituisce un laccio, ma chi confida nell'Eterno è al sicuro.
\par 26 Molti cercano il favore del principe, ma l'Eterno fa giustizia ad ognuno.
\par 27 L'uomo iniquo è un abominio per i giusti, e colui che cammina rettamente è un abominio per gli empi.

\chapter{30}

\par 1 Parole di Agur, figliuolo di Jaké. Sentenze pronunziate da quest'uomo per Itiel, per Itiel ed Ucal.
\par 2 Certo, io sono più stupido d'ogni altro, e non ho l'intelligenza d'un uomo.
\par 3 Non ho imparato la sapienza, e non ho la conoscenza del Santo.
\par 4 Chi è salito in cielo e n'è disceso? Chi ha raccolto il vento nel suo pugno? Chi ha racchiuse l'acque nella sua veste? Chi ha stabilito tutti i confini della terra? Qual è il suo nome e il nome del suo figlio? Lo sai tu?
\par 5 Ogni parola di Dio è affinata col fuoco. Egli è uno scudo per chi confida in lui.
\par 6 Non aggiunger nulla alle sue parole, ch'egli non t'abbia a riprendere, e tu non sia trovato bugiardo.
\par 7 Io t'ho chiesto due cose: non me le rifiutare, prima ch'io muoia:
\par 8 allontana da me vanità e parola mendace; non mi dare né povertà né ricchezze, cibami del pane che m'è necessario,
\par 9 ond'io, essendo sazio, non giunga a rinnegarti, e a dire: 'Chi è l'Eterno?' ovvero, diventato povero, non rubi, e profani il nome del mio Dio.
\par 10 Non calunniare il servo presso al suo padrone, ch'ei non ti maledica e tu non abbia a subirne la pena.
\par 11 V'è una razza di gente che maledice suo padre e non benedice sua madre.
\par 12 V'è una razza di gente che si crede pura, e non è lavata dalla sua sozzura.
\par 13 V'è una razza di gente che ha gli occhi alteri e come! e le palpebre superbe.
\par 14 V'è una razza di gente i cui denti sono spade e i mascellari, coltelli, per divorare del tutto i miseri sulla terra, e i bisognosi fra gli uomini.
\par 15 La mignatta ha due figliuole, che dicono: 'Dammi' 'dammi!' Ci son tre cose che non si sazian mai, anzi quattro, che non dicon mai: 'Basta!'
\par 16 Il soggiorno dei morti, il seno sterile, la terra che non si sazia d'acqua, e il fuoco, che non dice mai: 'Basta!'
\par 17 L'occhio di chi si fa beffe del padre e disdegna d'ubbidire alla madre, lo caveranno i corvi del torrente, lo divoreranno gli aquilotti.
\par 18 Ci son tre cose per me troppo maravigliose; anzi quattro, ch'io non capisco:
\par 19 la traccia dell'aquila nell'aria, la traccia del serpente sulla roccia, la traccia della nave in mezzo al mare, la traccia dell'uomo nella giovane.
\par 20 Tale è la condotta della donna adultera: essa mangia, si pulisce la bocca, e dice: 'Non ho fatto nulla di male!'
\par 21 Per tre cose la terra trema, anzi per quattro, che non può sopportare:
\par 22 per un servo quando diventa re, per un uomo da nulla quando ha pane a sazietà,
\par 23 per una donna, non mai chiesta, quando giunge a maritarsi, e per una serva quando diventa erede della padrona.
\par 24 Ci son quattro animali fra i più piccoli della terra, e nondimeno pieni di saviezza:
\par 25 le formiche, popolo senza forze, che si preparano il cibo durante l'estate;
\par 26 i conigli, popolo non potente, che fissano la loro dimora nelle rocce;
\par 27 le locuste, che non hanno re, e procedon tutte, divise per schiere;
\par 28 la lucertola, che puoi prender con le mani, eppur si trova nei palazzi dei re.
\par 29 Queste tre creature hanno una bella andatura, anche queste quattro hanno un passo magnifico:
\par 30 il leone, ch'è il più forte degli animali, e non indietreggia dinanzi ad alcuno;
\par 31 il cavallo dai fianchi serrati, il capro, e il re alla testa dei suoi eserciti.
\par 32 Se hai agito follemente cercando d'innalzarti, o se hai pensato del male, mettiti la mano sulla bocca;
\par 33 perché, come chi sbatte la panna ne fa uscire il burro, chi comprime il naso ne fa uscire il sangue, così chi spreme l'ira ne fa uscire contese.

\chapter{31}

\par 1 Parole del re Lemuel. Sentenze con le quali sua madre lo ammaestrò.
\par 2 Che ti dirò, figlio mio? che ti dirò, figlio delle mie viscere? che ti dirò, o figlio dei miei voti?
\par 3 Non dare il tuo vigore alle donne, né i tuoi costumi a quelle che perdono i re.
\par 4 Non s'addice ai re, o Lemuel, non s'addice ai re bere del vino, né ai principi, bramar la cervogia:
\par 5 che a volte, avendo bevuto, non dimentichino la legge, e non disconoscano i diritti d'ogni povero afflitto.
\par 6 Date della cervogia a chi sta per perire, e del vino a chi ha l'anima amareggiata;
\par 7 affinché bevano, dimentichino la loro miseria, e non si ricordin più dei loro travagli.
\par 8 Apri la tua bocca in favore del mutolo, per sostener la causa di tutti i derelitti;
\par 9 apri la tua bocca, giudica con giustizia, fa' ragione al misero ed al bisognoso.
\par 10 Una donna forte e virtuosa chi la troverà? il suo pregio sorpassa di molto quello delle perle.
\par 11 Il cuore del suo marito confida in lei, ed egli non mancherà mai di provviste.
\par 12 Ella gli fa del bene, e non del male, tutti i giorni della sua vita.
\par 13 Ella si procura della lana e del lino, e lavora con diletto con le proprie mani.
\par 14 Ella è simile alle navi dei mercanti: fa venire il suo cibo da lontano.
\par 15 Ella si alza quando ancora è notte, distribuisce il cibo alla famiglia e il còmpito alle sue donne di servizio.
\par 16 Ella posa gli occhi sopra un campo, e l'acquista; col guadagno delle sue mani pianta una vigna.
\par 17 Ella si ricinge di forza i fianchi, e fa robuste le sue braccia.
\par 18 Ella s'accorge che il suo lavoro rende bene; la sua lucerna non si spegne la notte.
\par 19 Ella mette la mano alla ròcca, e le sue dita maneggiano il fuso.
\par 20 Ella stende le palme al misero, e porge le mani al bisognoso.
\par 21 Ella non teme la neve per la sua famiglia, perché tutta la sua famiglia è vestita di lana scarlatta.
\par 22 Ella si fa dei tappeti, ha delle vesti di lino finissimo e di porpora.
\par 23 Il suo marito è rispettato alle porte, quando si siede fra gli Anziani del paese.
\par 24 Ella fa delle tuniche e le vende, e delle cinture che dà al mercante.
\par 25 Forza e dignità sono il suo manto, ed ella si ride dell'avvenire.
\par 26 Ella apre la bocca con sapienza, ed ha sulla lingua insegnamenti di bontà.
\par 27 Ella sorveglia l'andamento della sua casa, e non mangia il pane di pigrizia.
\par 28 I suoi figliuoli sorgono e la proclaman beata, e il suo marito la loda, dicendo:
\par 29 'Molte donne si son portate valorosamente, ma tu le superi tutte!'
\par 30 La grazia è fallace e la bellezza è cosa vana; ma la donna che teme l'Eterno è quella che sarà lodata.
\par 31 Datele del frutto delle sue mani, e le opere sue la lodino alle porte!


\end{document}