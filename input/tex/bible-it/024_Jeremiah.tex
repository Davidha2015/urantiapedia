\begin{document}

\title{Jeremiah}


\chapter{1}

\par 1 Parole di Geremia, figliuolo di Hilkia, uno dei sacerdoti che stavano ad Anatoth, nel paese di Beniamino.
\par 2 La parola dell'Eterno gli fu rivolta al tempo di Giosia, figliuolo d'Amon, re di Giuda, l'anno tredicesimo del suo regno, e al tempo di Jehoiakim,
\par 3 figliuolo di Giosia, re di Giuda, sino alla fine dell'anno undecimo di Sedechia, figliuolo di Giosia, re di Giuda, fino a quando Gerusalemme fu menata in cattività, il che avvenne nel quinto mese.
\par 4 La parola dell'Eterno mi fu rivolta, dicendo:
\par 5 'Prima ch'io ti avessi formato nel seno di tua madre, io t'ho conosciuto; e prima che tu uscissi dal suo seno, io t'ho consacrato e t'ho costituito profeta delle nazioni'.
\par 6 E io risposi: 'Ahimè, Signore, Eterno, io non so parlare, poiché non sono che un fanciullo'.
\par 7 Ma l'Eterno mi disse: 'Non dire: - Sono un fanciullo, - poiché tu andrai da tutti quelli ai quali ti manderò, e dirai tutto quello che io ti comanderò.
\par 8 Non li temere, perché io son teco per liberarti, dice l'Eterno'.
\par 9 Poi l'Eterno stese la mano e mi toccò la bocca; e l'Eterno disse: 'Ecco, io ho messo le mie parole nella tua bocca.
\par 10 Vedi, io ti costituisco oggi sulle nazioni e sopra i regni, per svellere, per demolire, per abbattere, per distruggere, per edificare e per piantare'.
\par 11 Poi la parola dell'Eterno mi fu rivolta, dicendo: - 'Geremia, che vedi?' Io risposi: 'Vedo un ramo di mandorlo'. E l'Eterno mi disse:
\par 12 'Hai veduto bene, poiché io vigilo sulla mia parola per mandarla ad effetto'.
\par 13 E la parola dell'Eterno mi fu rivolta per la seconda volta, dicendo: 'Che vedi?' Io risposi: 'Vedo una caldaia che bolle ed ha la bocca vòlta dal settentrione in qua'. E l'Eterno mi disse:
\par 14 'Dal settentrione verrà fuori la calamità su tutti gli abitanti del paese.
\par 15 Poiché, ecco, io sto per chiamare tutti i popoli dei regni del settentrione, dice l'Eterno; essi verranno, e porranno ognuno il suo trono all'ingresso delle porte di Gerusalemme, contro tutte le sue mura all'intorno, e contro tutte le città di Giuda.
\par 16 E pronunzierò i miei giudizi contro di loro, a motivo di tutta la loro malvagità, perché m'hanno abbandonato e hanno offerto il loro profumo ad altri dèi e si son prostrati dinanzi all'opera delle loro mani.
\par 17 Tu dunque, cingiti i lombi, lèvati, e di' loro tutto quello che io ti comanderò. Non ti sgomentare per via di loro, ond'io non ti renda sgomento in loro presenza.
\par 18 Ecco, oggi io ti stabilisco come una città fortificata, come una colonna di ferro e come un muro di rame contro tutto il paese, contro i re di Giuda, contro i suoi principi, contro i suoi sacerdoti e contro il popolo del paese.
\par 19 Essi ti faranno la guerra, ma non ti vinceranno, perché io son teco per liberarti, dice l'Eterno'.

\chapter{2}

\par 1 La parola dell'Eterno mi fu ancora rivolta, dicendo: Va', e grida agli orecchi di Gerusalemme:
\par 2 Così dice l'Eterno: Io mi ricordo dell'affezione che avevi per me quand'eri giovane, del tuo amore quand'eri fidanzata, allorché tu mi seguivi nel deserto, in una terra non seminata.
\par 3 Israele era consacrato all'Eterno, le primizie della sua rendita; tutti quelli che lo divoravano si rendevan colpevoli, e la calamità piombava su loro, dice l'Eterno.
\par 4 Ascoltate la parola dell'Eterno, o casa di Giacobbe, e voi tutte le famiglie della casa d'Israele!
\par 5 Così parla l'Eterno: Quale iniquità hanno trovata i vostri padri in me, che si sono allontanati da me, e sono andati dietro alla vanità, e son diventati essi stessi vanità?
\par 6 Essi non hanno detto: 'Dov'è l'Eterno che ci ha tratti fuori dal paese d'Egitto, che ci ha menati per il deserto, per un paese di solitudine e di crepacci, per un paese d'aridità e d'ombra di morte, per un paese per il quale nessuno passò mai e dove non abitò mai nessuno?'
\par 7 E io v'ho condotti in un paese ch'è un frutteto, perché ne mangiaste i frutti ed i buoni prodotti; ma voi, quando vi siete entrati, avete contaminato il mio paese e avete fatto della mia eredità un'abominazione.
\par 8 I sacerdoti non hanno detto: 'Dov'è l'Eterno?' i depositari della legge non m'hanno conosciuto, i pastori mi sono stati infedeli, i profeti hanno profetato nel nome di Baal, e sono andati dietro a cose che non giovano a nulla.
\par 9 Perciò io contenderò ancora in giudizio con voi, dice l'Eterno, e contenderò coi figliuoli de' vostri figliuoli.
\par 10 Passate dunque nelle isole di Kittim, e guardate! Mandate a Kedar e osservate bene, e guardate se avvenne mai qualcosa di simile!
\par 11 V'ha egli una nazione che abbia cambiato i suoi dèi, quantunque non siano dèi? Ma il mio popolo ha cambiato la sua gloria per ciò che non giova a nulla.
\par 12 O cieli, stupite di questo; inorridite e restate attoniti, dice l'Eterno.
\par 13 Poiché il mio popolo ha commesso due mali: ha abbandonato me, la sorgente d'acqua viva, e s'è scavato delle cisterne, delle cisterne screpolate, che non tengono l'acqua.
\par 14 Israele è egli uno schiavo? è egli uno schiavo nato in casa? Perché dunque è egli diventato una preda?
\par 15 I leoncelli ruggono contro di lui, e fanno udire la loro voce, e riducono il suo paese in una desolazione; le sue città sono arse, e non vi son più abitanti.
\par 16 Perfino gli abitanti di Nof e di Tahpanes ti divorano il cranio.
\par 17 Tutto questo non ti succede egli perché hai abbandonato l'Eterno, il tuo Dio, mentr'egli ti menava per la buona via?
\par 18 E ora, che hai tu da fare sulla via che mena in Egitto per andare a bere l'acqua del Nilo? o che hai tu da fare sulla via che mena in Assiria per andare a bere l'acqua del fiume?
\par 19 La tua propria malvagità è quella che ti castiga, e le tue infedeltà sono la tua punizione. Sappi dunque e vedi che mala ed amara cosa è abbandonare l'Eterno, il tuo Dio, e il non aver di me alcun timore, dice il Signore, l'Eterno degli eserciti.
\par 20 Già da lungo tempo tu hai spezzato il tuo giogo, rotti i tuoi legami, e hai detto: 'Non voglio più servire!' Ma sopra ogni alto colle e sotto ogni albero verdeggiante ti sei buttata giù come una prostituta.
\par 21 Eppure, io t'avevo piantato come una nobile vigna tutta del miglior ceppo; come dunque mi ti sei mutato in rampolli degenerati di una vigna straniera?
\par 22 Quand'anche tu ti lavassi col nitro e usassi molto sapone, la tua iniquità lascerebbe una macchia dinanzi a me, dice il Signore, l'Eterno.
\par 23 Come puoi tu dire: 'Io non mi son contaminata, non sono andata dietro ai Baal?' Guarda i tuoi passi nella valle, riconosci quello che hai fatto, dromedaria leggera e vagabonda!
\par 24 Asina selvatica, avvezza al deserto, che aspira l'aria nell'ardore della sua passione, chi le impedirà di soddisfare la sua brama? Tutti quelli che la cercano non hanno da affaticarsi; la trovano nel suo mese.
\par 25 Guarda che il tuo piede non si scalzi e che la tua gola non s'inaridisca! Ma tu hai detto: 'Non c'è rimedio; no, io amo gli stranieri, e andrò dietro a loro!'
\par 26 Come il ladro è confuso quand'è còlto sul fatto, così son confusi quelli della casa d'Israele: essi, i loro re, i loro capi, i loro sacerdoti e i loro profeti,
\par 27 i quali dicono al legno: 'Tu sei mio padre', e alla pietra: 'Tu ci hai dato la vita!' Poich'essi m'han voltato le spalle e non la faccia; ma nel tempo della loro sventura dicono: 'Lèvati e salvaci!'
\par 28 E dove sono i tuoi dèi che ti sei fatti? Si lèvino, se ti posson salvare nel tempo della tua sventura! Perché, o Giuda, tu hai tanti dèi quante città.
\par 29 Perché contendereste meco? Voi tutti mi siete stati infedeli, dice l'Eterno.
\par 30 Invano ho colpito i vostri figliuoli; non ne hanno ricevuto correzione; la vostra spada ha divorato i vostri profeti, come un leone distruttore.
\par 31 O generazione, considera la parola dell'Eterno! Son io stato un deserto per Israele? o un paese di fitte tenebre? Perché dice il mio popolo: 'Noi siamo liberi, non vogliamo tornar più a te?'
\par 32 La fanciulla può essa dimenticare i suoi ornamenti, o la sposa la sua cintura? Eppure, il mio popolo ha dimenticato me, da giorni innumerevoli.
\par 33 Come sei brava a trovar la via per correr dietro ai tuoi amori! Perfino alle male femmine hai insegnato i tuoi modi!
\par 34 Fino nei lembi della tua veste si trova il sangue di poveri innocenti, che tu non hai còlto in flagrante delitto di scasso;
\par 35 eppure, dopo tutto questo, tu dici: 'Io sono innocente; certo, l'ira sua s'è stornata da me'. Ecco, io entrerò in giudizio con te, perché hai detto: 'Non ho peccato'.
\par 36 Perché hai tanta premura di mutare il tuo cammino? Anche dall'Egitto riceverai confusione, come già l'hai ricevuta dall'Assiria.
\par 37 Anche di là uscirai con le mani sul capo; perché l'Eterno rigetta quelli ne' quali tu confidi, e tu non riuscirai nel tuo intento per loro mezzo.

\chapter{3}

\par 1 L'Eterno dice: Se un uomo ripudia la sua moglie e questa se ne va da lui e si marita a un altro, quell'uomo torna egli forse ancora da lei? Il paese stesso non ne sarebb'egli tutto profanato? E tu, che ti sei prostituita con molti amanti, ritorneresti a me? dice l'Eterno.
\par 2 Alza gli occhi verso le alture, e guarda: Dov'è che non ti sei prostituita? Tu sedevi per le vie ad aspettare i passanti, come fa l'Arabo nel deserto, e hai contaminato il paese con le tue prostituzioni e con le tue malvagità.
\par 3 Perciò le grandi piogge sono state trattenute, e non v'è stata pioggia di primavera; ma tu hai avuto una fronte da prostituta, e non hai voluto vergognarti.
\par 4 E ora, non è egli vero? tu gridi a me: 'Padre mio, tu sei stato l'amico della mia giovinezza!
\par 5 Sarà egli adirato in perpetuo? Serberà egli la sua ira sino alla fine?' Ecco, tu parli così, ma intanto commetti a tutto potere delle male azioni!
\par 6 L'Eterno mi disse al tempo del re Giosia: 'Hai tu veduto quello che la infedele Israele ha fatto? È andata sopra ogni alto monte e sotto ogni albero verdeggiante, e quivi s'è prostituita.
\par 7 Io dicevo: Dopo che avrà fatto tutte queste cose, essa tornerà a me; ma non è ritornata; e la sua sorella, la perfida Giuda, l'ha visto.
\par 8 E benché io avessi ripudiato l'infedele Israele a cagione di tutti i suoi adulteri e le avessi dato la sua lettera di divorzio, ho visto che la sua sorella, la perfida Giuda, non ha avuto alcun timore, ed è andata a prostituirsi anch'essa.
\par 9 Col rumore delle sue prostituzioni Israele ha contaminato il paese, e ha commesso adulterio con la pietra e col legno;
\par 10 e nonostante tutto questo, la sua perfida sorella non è tornata a me con tutto il suo cuore, ma con finzione, dice l'Eterno'.
\par 11 E l'Eterno mi disse: 'La infedele Israele s'è mostrata più giusta della perfida Giuda'.
\par 12 Va', proclama queste parole verso il settentrione, e di': Torna, o infedele Israele, dice l'Eterno; io non vi mostrerò un viso accigliato, giacché io son misericordioso, dice l'Eterno, e non serbo l'ira in perpetuo.
\par 13 Soltanto riconosci la tua iniquità: tu sei stata infedele all'Eterno, al tuo Dio, hai vòlto qua e là i tuoi passi verso gli stranieri, sotto ogni albero verdeggiante, e non hai dato ascolto alla mia voce, dice l'Eterno.
\par 14 Tornate o figliuoli traviati, dice l'Eterno, poiché io sono il vostro signore, e vi prenderò, uno da una città, due da una famiglia, e vi ricondurrò a Sion;
\par 15 e vi darò dei pastori secondo il mio cuore, che vi pasceranno con conoscenza e con intelligenza.
\par 16 E quando sarete moltiplicati e avrete fruttato nel paese, allora, dice l'Eterno, non si dirà più: 'L'arca del patto dell'Eterno!' non vi si penserà più, non la si menzionerà più, non la si rimpiangerà più, non se ne farà un'altra.
\par 17 Allora Gerusalemme sarà chiamata 'il trono dell'Eterno'; tutte le nazioni si raduneranno a Gerusalemme nel nome dell'Eterno, e non cammineranno più secondo la caparbietà del loro cuore malvagio.
\par 18 In quei giorni, la casa di Giuda camminerà con la casa d'Israele, e verranno assieme dal paese del settentrione al paese ch'io detti in eredità ai vostri padri.
\par 19 Io avevo detto: 'Oh qual posto ti darò tra i miei figliuoli! Che paese delizioso ti darò! la più bella eredità delle nazioni!' Avevo detto: 'Tu mi chiamerai: - Padre mio! - e non cesserai di seguirmi'.
\par 20 Ma, proprio come una donna è infedele al suo amante, così voi mi siete stati infedeli, o casa d'Israele! dice l'Eterno.
\par 21 Una voce s'è fatta udire sulle alture; sono i pianti, le supplicazioni de' figliuoli d'Israele, perché hanno pervertito la loro via, hanno dimenticato l'Eterno, il loro Dio.
\par 22 'Tornate, o figliuoli traviati, io vi guarirò dei vostri traviamenti!' 'Eccoci, noi veniamo a te, perché tu sei l'Eterno, il nostro Dio.
\par 23 Sì, certo, vano è il soccorso che s'aspetta dalle alture, dalle feste strepitose sui monti; sì, nell'Eterno, nel nostro Dio, sta la salvezza d'Israele.
\par 24 Quella vergogna, che son gl'idoli, ha divorato il prodotto della fatica de' nostri padri fin dalla nostra giovinezza, le loro pecore e i loro buoi, i loro figliuoli e le loro figliuole.
\par 25 Giaciamoci nella nostra vergogna e ci copra la nostra ignominia! poiché abbiam peccato contro l'Eterno, il nostro Dio: noi e i nostri padri, dalla nostra fanciullezza fino a questo giorno; e non abbiam dato ascolto alla voce dell'Eterno, ch'è il nostro Dio'.

\chapter{4}

\par 1 O Israele, se tu torni, dice l'Eterno, se tu torni a me, e se togli dal mio cospetto le tue abominazioni, se non vai più vagando qua e là
\par 2 e giuri per l'Eterno che vive! con verità, con rettitudine e con giustizia, allora le nazioni saranno benedette in te, e in te si glorieranno.
\par 3 Poiché così parla l'Eterno a quei di Giuda e di Gerusalemme: Dissodatevi un campo nuovo, e non seminate fra le spine!
\par 4 Circoncidetevi per l'Eterno, circoncidete i vostri cuori, o uomini di Giuda e abitanti di Gerusalemme, affinché il mio furore non scoppi come un fuoco, e non si infiammi sì che nessuno possa spegnerlo, a motivo della malvagità delle vostre azioni!
\par 5 Annunziate in Giuda, bandite questo in Gerusalemme, e dite: 'Suonate le trombe nel paese!' gridate forte e dite: 'Adunatevi ed entriamo nelle città forti!'
\par 6 Alzate la bandiera verso Sion, cercate un rifugio, non vi fermate, perch'io faccio venire dal settentrione una calamità e una grande rovina.
\par 7 Un leone balza fuori dal folto del bosco, e un distruttore di nazioni s'è messo in via, ha lasciato il suo luogo, per ridurre il tuo paese in desolazione, sì che le tue città saranno rovinate e prive d'abitanti.
\par 8 Perciò, cingetevi di sacchi, fate cordoglio, mandate lamenti! perché l'ardente ira dell'Eterno non s'è stornata da noi.
\par 9 E in quel giorno avverrà, dice l'Eterno, che il cuore del re e il cuore de' capi verranno meno, i sacerdoti saranno attoniti, e i profeti stupefatti.
\par 10 Allora io dissi: 'Ahi! Signore, Eterno! tu hai dunque ingannato questo popolo e Gerusalemme dicendo: Voi avrete pace mentre la spada penetra fino all'anima'.
\par 11 In quel tempo si dirà a questo popolo e a Gerusalemme: Un vento ardente viene dalle alture del deserto verso la figliuola del mio popolo, non per vagliare, non per nettare il grano;
\par 12 un vento anche più impetuoso di quello verrà da parte mia; ora anch'io pronunzierò la sentenza contro di loro.
\par 13 Ecco, l'invasore sale come fan le nuvole, e i suoi carri sono come un turbine; i suoi cavalli son più rapidi delle aquile. 'Guai a noi! poiché siam devastati!'
\par 14 O Gerusalemme, netta il tuo cuore dalla malvagità, affinché tu sia salvata. Fino a quando albergheranno in te i tuoi pensieri iniqui?
\par 15 Poiché una voce che viene da Dan annunzia la calamità, e la bandisce dai colli d'Efraim.
\par 16 'Avvertitene le nazioni, fatelo sapere a Gerusalemme: degli assedianti vengono da un paese lontano, e mandan le loro grida contro le città di Giuda'.
\par 17 Si son posti contro Gerusalemme da ogni lato, a guisa di guardie d'un campo, perch'ella s'è ribellata contro di me, dice l'Eterno.
\par 18 Il tuo procedere e le tue azioni t'hanno attirato queste cose; quest'è il frutto della tua malvagità; sì, è amaro; sì, è cosa che t'arriva al cuore.
\par 19 Le mie viscere! le mie viscere! Io sento un gran dolore! Oh le pareti del mio cuore! Il mio cuore mi batte in petto! Io non posso tacermi; poiché, anima mia, tu odi il suon della tromba, il grido di guerra.
\par 20 S'annunzia rovina sopra rovina, poiché tutto il paese è devastato. Le mie tende sono distrutte ad un tratto, i miei padiglioni, in un attimo.
\par 21 Fino a quando vedrò la bandiera e udrò il suon della tromba?
\par 22 Veramente il mio popolo è stolto, non mi conosce; son de' figliuoli insensati, e non hanno intelligenza; sono sapienti per fare il male; ma il bene non lo sanno fare.
\par 23 Io guardo la terra, ed ecco è desolata e deserta; i cieli, e son senza luce.
\par 24 Guardo i monti, ed ecco tremano, e tutti i colli sono agitati.
\par 25 Guardo, ed ecco non c'è uomo, e tutti gli uccelli del cielo son volati via.
\par 26 Guardo, ed ecco il Carmelo è un deserto, e tutte le sue città sono abbattute dinanzi all'Eterno, dinanzi all'ardente sua ira.
\par 27 Poiché così parla l'Eterno: Tutto il paese sarà desolato, ma io non lo finirò del tutto.
\par 28 A motivo di questo, la terra fa cordoglio, e i cieli di sopra s'oscurano; perché io l'ho detto, l'ho stabilito, e non me ne pento, e non mi ritratterò.
\par 29 Al rumore dei cavalieri e degli arcieri tutte le città sono in fuga; tutti entrano nel folto de' boschi, montano sulle rocce; tutte le città sono abbandonate, e non v'è più alcun abitante.
\par 30 E tu che stai per esser devastata, che fai? Hai un bel vestirti di scarlatto, un bel metterti i tuoi ornamenti d'oro, un bell'ingrandirti gli occhi col belletto! Invano t'abbellisci; i tuoi amanti ti sprezzano, voglion la tua vita.
\par 31 Poiché io odo de' gridi come di donna ch'è nei dolori; un'angoscia come quella di donna nel suo primo parto; è la voce della figliuola di Sion, che sospira ansimando e stende le mani: 'Ahi me lassa! che l'anima mia vien meno dinanzi agli uccisori'.

\chapter{5}

\par 1 Andate attorno per le vie di Gerusalemme, e guardate, e informatevi, e cercate per le sue piazze se vi trovate un uomo, se ve n'è uno solo che operi giustamente, che cerchi la fedeltà; e io perdonerò Gerusalemme.
\par 2 Anche quando dicono: 'Com'è vero che l'Eterno vive', è certo che giurano il falso.
\par 3 O Eterno, gli occhi tuoi non cercano essi la fedeltà? Tu li colpisci, e quelli non sentono nulla; tu li consumi, e quelli rifiutano di ricevere la correzione; essi han reso il loro volto più duro della roccia, rifiutano di convertirsi.
\par 4 Io dicevo: 'Questi non son che i miseri; sono insensati perché non conoscono la via dell'Eterno, la legge del loro Dio;
\par 5 io andrò dai grandi e parlerò loro, perch'essi conoscono la via dell'Eterno, la legge del loro Dio'; ma anch'essi tutti quanti hanno spezzato il giogo, hanno rotto i legami.
\par 6 Perciò il leone della foresta li uccide, il lupo del deserto li distrugge, il leopardo sta in agguato presso le loro città; chiunque ne uscirà sarà sbranato, perché le loro trasgressioni son numerose, le loro infedeltà sono aumentate.
\par 7 Perché ti perdonerei io? I tuoi figliuoli m'hanno abbandonato, e giurano per degli dèi che non esistono. Io li ho satollati ed essi si danno all'adulterio, e s'affollano nelle case di prostituzione.
\par 8 Sono come tanti stalloni ben pasciuti ed ardenti; ognun d'essi nitrisce dietro la moglie del prossimo.
\par 9 Non li punirei io per queste cose? dice l'Eterno; e l'anima mia non si vendicherebbe d'una simile nazione?
\par 10 Salite sulle sue mura e distruggete, ma non la finite del tutto; portate via i suoi tralci, perché non son dell'Eterno!
\par 11 Poiché la casa d'Israele e la casa di Giuda m'hanno tradito, dice l'Eterno.
\par 12 Rinnegano l'Eterno, e dicono: 'Non esiste; nessun male ci verrà addosso, noi non vedremo né spada né fame;
\par 13 i profeti non sono che vento, e nessuno parla in essi. Quel che minacciano sia fatto a loro!'
\par 14 Perciò così parla l'Eterno, l'Iddio degli eserciti: Perché avete detto quelle parole, ecco, io farò che la parola mia sia come fuoco nella tua bocca, che questo popolo sia come legno, e che quel fuoco lo divori.
\par 15 Ecco, io faccio venire da lungi una nazione contro di voi, o casa d'Israele, dice l'Eterno; una nazione valorosa, una nazione antica, una nazione della quale tu non conosci la lingua e non intendi le parole.
\par 16 Il suo turcasso è un sepolcro aperto; tutti quanti son dei prodi.
\par 17 Essa divorerà le tue mèssi e il tuo pane, divorerà i tuoi figliuoli e le tue figliuole, divorerà le tue pecore e i tuoi buoi, divorerà le tue vigne e i tuoi fichi; abbatterà con la spada le tue città forti nelle quali confidi.
\par 18 Ma anche in quei giorni, dice l'Eterno, io non ti finirò del tutto.
\par 19 E quando direte: 'Perché l'Eterno, il nostro Dio, ci ha egli fatto tutto questo?' tu risponderai loro: 'Come voi m'avete abbandonato e avete servito degli dèi stranieri nel vostro paese, così servirete degli stranieri in un paese che non è vostro'.
\par 20 Annunziate questo alla casa di Giacobbe, banditelo in Giuda, e dite:
\par 21 Ascoltate ora questo, o popolo stolto e senza cuore, che ha occhi e non vede, che ha orecchi e non ode.
\par 22 Non mi temerete voi? dice l'Eterno; non temerete voi dinanzi a me che ho posto la rena per limite al mare, barriera eterna, ch'esso non oltrepasserà mai? I suoi flutti s'agitano, ma sono impotenti; muggono, ma non la sormontano.
\par 23 Ma questo popolo ha un cuore indocile e ribelle; si voltano indietro e se ne vanno.
\par 24 Non dicono in cuor loro: 'Temiamo l'Eterno, il nostro Dio, che dà la pioggia a suo tempo: la pioggia della prima e dell'ultima stagione, che ci mantiene le settimane fissate per la mietitura'.
\par 25 Le vostre iniquità hanno sconvolto queste cose, e i vostri peccati v'han privato del benessere.
\par 26 Poiché fra il mio popolo si trovan degli empi che spiano, come uccellatori in agguato; essi tendon tranelli, acchiappano uomini.
\par 27 Come una gabbia è piena d'uccelli, così le loro case son piene di frode; perciò diventan grandi e s'arricchiscono.
\par 28 Ingrassano, hanno il volto lucente, oltrepassano ogni limite di male. Non difendono la causa, la causa dell'orfano, eppur prosperano; e non fanno giustizia nei processi de' poveri.
\par 29 E non punirei io queste cose? dice l'Eterno; e l'anima mia non si vendicherebbe di una simile nazione?
\par 30 Cose spaventevoli e orride si fanno nel paese:
\par 31 i profeti profetano bugiardamente; i sacerdoti governano agli ordini de' profeti; e il mio popolo ha piacere che sia così. E che farete voi quando verrà la fine?

\chapter{6}

\par 1 O figliuoli di Beniamino, cercate un rifugio lungi dal mezzo di Gerusalemme, e sonate la tromba in Tekoa, e innalzate un segnale su Bethkerem! perché dal settentrione s'avanza una calamità, una grande ruina.
\par 2 La bella, la voluttuosa figliuola di Sion io la distruggo!
\par 3 Verso di lei vengono de' pastori coi loro greggi; essi piantano le loro tende intorno a lei; ognun d'essi bruca dal suo lato.
\par 4 'Preparate l'attacco contro di lei; levatevi, saliamo in pien mezzodì!' 'Guai a noi! ché il giorno declina, e le ombre della sera s'allungano!'
\par 5 'Levatevi, saliamo di notte, e distruggiamo i suoi palazzi!'
\par 6 Poiché così parla l'Eterno degli eserciti: Abbattete i suoi alberi, ed elevate un bastione contro Gerusalemme; quella è la città che dev'esser punita; dovunque, in mezzo a lei, non v'è che oppressione.
\par 7 Come un pozzo fa scaturire le sue acque, così ella fa scaturire la sua malvagità; in lei non si sente parlar che di violenza e di rovina; dinanzi a me stanno continuamente sofferenze e piaghe.
\par 8 Correggiti, o Gerusalemme, affinché l'anima mia non si alieni da te, e io non faccia di te un deserto, una terra disabitata!
\par 9 Così parla l'Eterno degli eserciti: Il resto d'Israele sarà interamente racimolato come una vigna; mettivi e rimettivi la mano, come fa il vendemmiatore sui tralci.
\par 10 A chi parlerò io, chi prenderò a testimonio perché m'ascolti? Ecco, l'orecchio loro è incirconciso, ed essi sono incapaci di prestare attenzione; ecco, la parola dell'Eterno è diventata per loro un obbrobrio, e non vi trovano più alcun piacere.
\par 11 Ma io son pieno del furore dell'Eterno; sono stanco di contenermi. Rivèrsalo ad un tempo sui bambini per la strada e sulle adunate dei giovani; poiché il marito e la moglie, il vecchio e l'uomo carico d'anni saranno tutti presi.
\par 12 Le loro case saran passate ad altri; e così pure i loro campi e le loro mogli; poiché io stenderò la mia mano sugli abitanti del paese, dice l'Eterno.
\par 13 Perché dal più piccolo al più grande, son tutti quanti avidi di guadagno; dal profeta al sacerdote, tutti praticano la menzogna.
\par 14 Essi curano alla leggera la piaga del mio popolo; dicono: 'Pace, pace', mentre pace non v'è.
\par 15 Saranno confusi perché commettono delle abominazioni; non si vergognano affatto, non sanno che cosa sia arrossire; perciò cadranno fra quelli che cadono; quand'io li visiterò saranno rovesciati, dice l'Eterno.
\par 16 Così dice l'Eterno: Fermatevi sulle vie, e guardate, e domandate quali siano i sentieri antichi, dove sia la buona strada, e incamminatevi per essa; e voi troverete riposo alle anime vostre! Ma quelli rispondono: 'Non c'incammineremo per essa!'
\par 17 Io ho posto presso a voi delle sentinelle: 'State attenti al suon della tromba!' Ma quelli rispondono: 'Non staremo attenti'.
\par 18 Perciò, ascoltate, o nazioni! Sappiate, o assemblea de' popoli, quello che avverrà loro.
\par 19 Ascolta, o terra! Ecco, io fo venire su questo popolo una calamità, frutto de' loro pensieri; perché non hanno prestato attenzione alle mie parole; e quanto alla mia legge, l'hanno rigettata.
\par 20 Che m'importa dell'incenso che viene da Seba, della canna odorosa che vien dal paese lontano? I vostri olocausti non mi sono graditi, e i vostri sacrifizi non mi piacciono.
\par 21 Perciò così parla l'Eterno: Ecco, io porrò dinanzi a questo popolo delle pietre d'intoppo, nelle quali inciamperanno assieme padri e figliuoli, vicini ed amici, e periranno.
\par 22 Così parla l'Eterno: Ecco, un popolo viene dal paese di settentrione, e una grande nazione si muove dalle estremità della terra.
\par 23 Essi impugnano l'arco ed il dardo; son crudeli, non hanno pietà; la loro voce è come il muggito del mare; montan cavalli; son pronti a combattere come un solo guerriero, contro di te, o figliuola di Sion.
\par 24 Noi ne abbiamo udito la fama, e le nostre mani si sono infiacchite; l'angoscia ci coglie, un dolore come di donna che partorisce.
\par 25 Non uscite nei campi, non camminate per le vie, perché la spada del nemico è là, e il terrore d'ogn'intorno.
\par 26 O figliuola del mio popolo, cingiti d'un sacco, avvoltolati nella cenere, prendi il lutto come per un figliuolo unico, fa' udire un amaro lamento, perché il devastatore ci piomba addosso improvviso.
\par 27 Io t'avevo messo fra il mio popolo come un saggiatore di metalli, perché tu conoscessi e saggiassi la loro via.
\par 28 Essi son tutti de' ribelli fra i ribelli, vanno attorno seminando calunnie, son rame e ferro, son tutti dei corrotti.
\par 29 Il mantice soffia con forza, il piombo è consumato dal fuoco; invano si cerca di raffinare, ché le scorie non si staccano.
\par 30 Saranno chiamati: argento di rifiuto, perché l'Eterno li ha rigettati.

\chapter{7}

\par 1 La parola che fu rivolta a Geremia da parte dell'Eterno, dicendo:
\par 2 Fermati alla porta della casa dell'Eterno, e quivi proclama questa parola: Ascoltate la parola dell'Eterno, o voi tutti uomini di Giuda ch'entrate per queste porte per prostrarvi dinanzi all'Eterno!
\par 3 Così parla l'Eterno degli eserciti, l'Iddio d'Israele: Emendate le vostre vie e le vostre opere, ed io vi farò dimorare in questo luogo.
\par 4 Non ponete la vostra fiducia in parole fallaci, dicendo: 'Questo è il tempio dell'Eterno, il tempio dell'Eterno, il tempio dell'Eterno!'
\par 5 Ma se emendate veramente le vostre vie e le vostre opere, se praticate sul serio la giustizia gli uni verso gli altri,
\par 6 se non opprimete lo straniero, l'orfano e la vedova, se non spargete sangue innocente in questo luogo e non andate per vostra sciagura dietro ad altri dèi,
\par 7 io altresì vi farò abitare in questo luogo, nel paese che ho dato ai vostri padri in sempiterno.
\par 8 Ecco, voi mettete la vostra fiducia in parole fallaci, che non giovano a nulla.
\par 9 Come! Voi rubate, uccidete, commettete adulterî, giurate il falso, offrite profumi a Baal, andate dietro ad altri dèi che prima non conoscevate,
\par 10 e poi venite a presentarvi davanti a me, in questa casa, sulla quale è invocato il mio nome, e dite 'Siamo salvi!' - e ciò per compiere tutte queste abominazioni?! È ella forse, agli occhi vostri, una spelonca di ladroni
\par 11 questa casa sulla quale è invocato il mio nome? Ecco, tutto questo io l'ho veduto, dice l'Eterno.
\par 12 Andate dunque al mio luogo ch'era a Silo, dove avevo da prima stanziato il mio nome, e guardate come l'ho trattato, a motivo della malvagità del mio popolo d'Israele.
\par 13 Ed ora, poiché avete commesso tutte queste cose, dice l'Eterno, poiché v'ho parlato, parlato fin dal mattino, e voi non avete dato ascolto, poiché v'ho chiamati e voi non avete risposto,
\par 14 io tratterò questa casa, sulla quale è invocato il mio nome e nella quale riponete la vostra fiducia, e il luogo che ho dato a voi e ai vostri padri, come ho trattato Silo;
\par 15 e vi caccerò dal mio cospetto, come ho cacciato tutti i vostri fratelli, tutta la progenie d'Efraim.
\par 16 E tu non intercedere per questo popolo, non innalzare per essi supplicazioni o preghiere, e non insistere presso di me, perché non t'esaudirò.
\par 17 Non vedi tu quello che fanno nelle città di Giuda e nelle vie di Gerusalemme?
\par 18 I figliuoli raccolgon le legna, i padri accendono il fuoco, e le donne intridon la pasta per far delle focacce alla regina del cielo e per far delle libazioni ad altri dèi, per offendermi.
\par 19 È proprio me che offendono? dice l'Eterno; non offendon essi loro stessi, a loro propria confusione?
\par 20 Perciò così parla il Signore, l'Eterno: Ecco, la mia ira, il mio furore, si riversa su questo luogo, sugli uomini e sulle bestie, sugli alberi della campagna e sui frutti della terra; essa consumerà ogni cosa e non si estinguerà.
\par 21 Così parla l'Eterno degli eserciti, l'Iddio d'Israele: Aggiungete i vostri olocausti ai vostri sacrifizi, e mangiatene la carne!
\par 22 Poiché io non parlai ai vostri padri e non diedi loro alcun comandamento, quando li trassi fuori dal paese d'Egitto, intorno ad olocausti ed a sacrifizi;
\par 23 ma questo comandai loro: 'Ascoltate la mia voce, e sarò il vostro Dio, e voi sarete il mio popolo; camminate in tutte le vie ch'io vi prescrivo affinché siate felici'.
\par 24 Ma essi non ascoltarono, non prestarono orecchio, ma camminarono seguendo i consigli e la caparbietà del loro cuore malvagio, e invece di andare avanti si sono vòlti indietro.
\par 25 Dal giorno che i vostri padri uscirono dal paese d'Egitto fino al dì d'oggi, io v'ho mandato tutti i miei servi, i profeti, e ve l'ho mandati ogni giorno, fin dal mattino;
\par 26 ma essi non m'hanno ascoltato, non hanno prestato orecchio; hanno fatto il collo duro; si son condotti peggio de' loro padri.
\par 27 Di' loro tutte queste cose, ma essi non t'ascolteranno; chiamali, ma essi non ti risponderanno.
\par 28 Perciò dirai loro: Questa è la nazione che non ascolta la voce dell'Eterno, del suo Dio, e che non vuol accettare correzione; la fedeltà è perita, è venuta meno nella loro bocca.
\par 29 Ràditi la chioma, e buttala via, e leva sulle alture un lamento, poiché l'Eterno rigetta e abbandona la generazione ch'è divenuta oggetto della sua ira.
\par 30 I figliuoli di Giuda hanno fatto ciò ch'è male agli occhi miei, dice l'Eterno; hanno collocato le loro abominazioni nella casa sulla quale è invocato il mio nome, per contaminarla.
\par 31 Hanno edificato gli alti luoghi di Tofet, nella valle del figliuolo di Hinnom, per bruciarvi nel fuoco i loro figliuoli e le loro figliuole: cosa che io non avevo comandata, e che non m'era mai venuta in mente.
\par 32 Perciò, ecco, i giorni vengono, dice l'Eterno, che non si dirà più 'Tofet' né 'la valle del figliuolo di Hinnom', ma 'la valle del massacro', e, per mancanza di spazio, si seppelliranno i morti a Tofet.
\par 33 E i cadaveri di questo popolo serviran di pasto agli uccelli del cielo e alle bestie della terra; e non vi sarà alcuno che li scacci.
\par 34 E farò cessare nelle città di Giuda e per le strade di Gerusalemme i gridi di gioia e i gridi d'esultanza, il canto dello sposo e il canto della sposa, perché il paese sarà una desolazione.

\chapter{8}

\par 1 In quel tempo, dice l'Eterno, si trarranno dai loro sepolcri le ossa dei re di Giuda, e le ossa dei suoi principi, le ossa dei sacerdoti, le ossa dei profeti, le ossa degli abitanti di Gerusalemme,
\par 2 e le si esporranno dinanzi al sole, dinanzi alla luna e dinanzi a tutto l'esercito del cielo, i quali essi hanno amato, hanno servito, hanno seguito, hanno consultato, e dinanzi ai quali si sono prostrati; non si raccoglieranno, non si seppelliranno, ma saranno come letame sulla faccia della terra.
\par 3 E la morte sarà preferibile alla vita per tutto il residuo che rimarrà di questa razza malvagia, in tutti i luoghi dove li avrò cacciati, dice l'Eterno degli eserciti.
\par 4 E tu di' loro: Così parla l'Eterno: Se uno cade non si rialza forse? Se uno si svia, non torna egli indietro?
\par 5 Perché dunque questo popolo di Gerusalemme si svia egli d'uno sviamento perpetuo? Essi persistono nella malafede, e rifiutano di convertirsi.
\par 6 Io sto attento ed ascolto: essi non parlano come dovrebbero; nessuno si pente della sua malvagità e dice: 'Che ho io fatto?' Ognuno riprende la sua corsa, come il cavallo che si slancia alla battaglia.
\par 7 Anche la cicogna conosce nel cielo le sue stagioni; la tortora, la rondine e la gru osservano il tempo quando debbon venire, ma il mio popolo non conosce quel che l'Eterno ha ordinato.
\par 8 Come potete voi dire: 'Noi siam savi e la legge dell'Eterno è con noi!' Sì certo, ma la penna bugiarda degli scribi ne ha falsato il senso.
\par 9 I savi saranno confusi, saranno costernati, saranno presi; ecco, hanno rigettato la parola dell'Eterno; che sapienza possono essi avere?
\par 10 Perciò io darò le loro mogli ad altri, e i loro campi a de' nuovi possessori; poiché dal più piccolo al più grande, son tutti avidi di guadagno; dal profeta al sacerdote, tutti praticano la menzogna.
\par 11 Essi curano alla leggera la piaga del mio popolo; dicono: 'Pace, pace', mentre pace non v'è.
\par 12 Essi saranno confusi perché commettono delle abominazioni: non si vergognano affatto, non sanno che cosa sia arrossire; perciò cadranno fra quelli che cadono; quand'io li visiterò saranno rovesciati, dice l'Eterno.
\par 13 Certo io li sterminerò, dice l'Eterno. Non v'è più uva sulla vite, non più fichi sul fico, e le foglie sono appassite! Io ho dato loro de' nemici che passeranno sui loro corpi.
\par 14 'Perché ce ne stiamo qui seduti? Adunatevi ed entriamo nelle città forti, per quivi perire! Poiché l'Eterno, il nostro Dio, ci condanna a perire, ci fa bere delle acque avvelenate, perché abbiam peccato contro l'Eterno.
\par 15 Noi aspettavamo la pace, ma nessun bene giunge; aspettavamo un tempo di guarigione, ed ecco il terrore!'
\par 16 S'ode da Dan lo sbuffare de' suoi cavalli; al rumore del nitrito de' suoi destrieri, trema tutto il paese; poiché vengono, divorano il paese e tutto ciò che contiene, la città e i suoi abitanti.
\par 17 Poiché, ecco, io mando contro di voi de' serpenti, degli aspidi, contro i quali non v'è incantagione che valga; e vi morderanno, dice l'Eterno.
\par 18 Ove trovar conforto nel mio dolore? Il cuore mi langue in seno.
\par 19 Ecco il grido d'angoscia della figliuola del mio popolo da terra lontana: 'L'Eterno non è egli più in Sion? Il suo re non è egli più in mezzo a lei?' 'Perché m'hanno provocato ad ira con le loro immagini scolpite e con vanità straniere?'
\par 20 'La mèsse è passata, l'estate è finita, e noi non siamo salvati'.
\par 21 Per la piaga della figliuola del mio popolo io son tutto affranto; sono in lutto, sono in preda alla costernazione.
\par 22 Non v'è egli balsamo in Galaad? Non v'è egli colà alcun medico? Perché dunque la piaga della figliuola del mio popolo non è stata medicata?

\chapter{9}

\par 1 Oh fosse pur la mia testa mutata in acqua, e fosser gli occhi miei una fonte di lacrime! Io piangerei giorno e notte gli uccisi della figliuola del mio popolo!
\par 2 Oh se avessi nel deserto un rifugio da viandanti! Io abbandonerei il mio popolo e me n'andrei lungi da costoro, perché son tutti adulteri, un'adunata di traditori.
\par 3 Tendono la lingua, ch'è il loro arco, per scoccar menzogne; son diventati potenti nel paese, ma non per agir con fedeltà; poiché procedono di malvagità in malvagità, e non conoscono me, dice l'Eterno.
\par 4 Si guardi ciascuno dal suo amico, e nessuno si fidi del suo fratello; poiché ogni fratello non fa che ingannare, ed ogni amico va spargendo calunnie.
\par 5 L'uno gabba l'altro, e non dice la verità, esercitano la loro lingua a mentire, s'affannano a fare il male.
\par 6 La tua dimora è la malafede; ed è per malafede che costoro rifiutano di conoscermi, dice l'Eterno.
\par 7 Perciò, così parla l'Eterno degli eserciti: Ecco, io li fonderò nel crogiuolo per saggiarli; poiché che altro farei riguardo alla figliuola del mio popolo?
\par 8 La loro lingua è un dardo micidiale; essa non dice che menzogne; con la bocca ognuno parla di pace al suo prossimo, ma nel cuore gli tende insidie.
\par 9 Non li punirei io per queste cose? dice l'Eterno; e l'anima mia non si vendicherebbe di una simile nazione?
\par 10 Io vo' dare in pianto ed in gemito, per i monti, e vo' dare in lamento per i pascoli del deserto, perché son arsi, talché niuno più vi passa, e non vi s'ode più voce di bestiame; gli uccelli del cielo e le bestie sono fuggite, sono scomparse.
\par 11 Io ridurrò Gerusalemme in un monte di ruine, in un ricetto di sciacalli; e farò delle città di Giuda una desolazione senza abitanti.
\par 12 Chi è il savio che capisca queste cose? Chi è colui al quale la bocca dell'Eterno ha parlato perché ei ne dia l'annunzio? Perché il paese è egli distrutto, desolato come un deserto talché niuno vi passa?
\par 13 L'Eterno risponde: Perché costoro hanno abbandonato la mia legge ch'io avevo loro posta dinanzi, e non hanno dato ascolto alla mia voce, e non l'hanno seguita nella lor condotta,
\par 14 ma han seguito la caparbietà del cuor loro, e sono andati dietro ai Baali, come i loro padri insegnaron loro.
\par 15 Perciò, così parla l'Eterno degli eserciti, l'Iddio di Israele: Ecco, io farò mangiar dell'assenzio a questo popolo e gli farò bere dell'acqua avvelenata.
\par 16 Io li disperderò fra le nazioni, che né loro né i loro padri han conosciuto; e manderò dietro a loro la spada, finché io li abbia consumati.
\par 17 Così parla l'Eterno degli eserciti: Pensate a chiamare delle piagnone, e ch'esse vengano! Mandate a cercare le più avvedute e ch'esse vengano
\par 18 e s'affrettino a fare un lamento su noi, sì che i nostri occhi si struggano in lacrime, e l'acqua fluisca dalle nostre palpebre.
\par 19 Poiché una voce di lamento si fa udire da Sion: 'Come siamo devastati! Siamo coperti di confusione, perché dobbiamo abbandonare il paese, ora che hanno abbattuto le nostre dimore'.
\par 20 Donne, ascoltate la parola dell'Eterno, e i vostri orecchi ricevan la parola della sua bocca! Insegnate alle vostre figliuole de' lamenti, e ognuna insegni alla sua compagna de' canti funebri!
\par 21 Poiché la morte è salita per le nostre finestre, è entrata nei nostri palazzi per far sparire i bambini dalle strade e i giovani dalle piazze.
\par 22 Di': Così parla l'Eterno: I cadaveri degli uomini giaceranno come letame sull'aperta campagna, come una mannella che il mietitore si lascia dietro e che nessuno raccoglie.
\par 23 Così parla l'Eterno: Il savio non si glorî della sua saviezza, il forte non si glorî della sua forza, il ricco non si glorî della sua ricchezza;
\par 24 ma chi si gloria si glorî di questo: che ha intelligenza e conosce me, che sono l'Eterno, che esercita la benignità, il diritto e la giustizia sulla terra; perché di queste cose mi compiaccio, dice l'Eterno.
\par 25 Ecco, i giorni vengono, dice l'Eterno, ch'io punirò tutti i circoncisi che sono incirconcisi:
\par 26 l'Egitto, Giuda, Edom, i figliuoli di Ammon, Moab, e tutti quelli che si tagliano i canti della barba, e abitano nel deserto; poiché tutte le nazioni sono incirconcise, e tutta la casa d'Israele è incirconcisa di cuore.

\chapter{10}

\par 1 Ascoltate la parola che l'Eterno vi rivolge, o casa d'Israele!
\par 2 Così parla l'Eterno: Non imparate a camminare nella via delle nazioni, e non abbiate paura de' segni del cielo, perché sono le nazioni quelle che ne hanno paura.
\par 3 Poiché i costumi dei popoli sono vanità; giacché si taglia un albero nella foresta e le mani dell'operaio lo lavorano con l'ascia;
\par 4 lo si adorna d'argento e d'oro, lo si fissa con chiodi e coi martelli perché non si muova.
\par 5 Cotesti dèi son come pali in un orto di cocomeri, e non parlano; bisogna portarli, perché non possono camminare. Non li temete! perché non possono fare alcun male, e non è in loro potere di far del bene.
\par 6 Non v'è alcuno pari a te, o Eterno; tu sei grande, e grande in potenza è il tuo nome.
\par 7 Chi non ti temerebbe, o re delle nazioni? Poiché questo t'è dovuto; giacché fra tutti i savi delle nazioni e in tutti i loro regni non v'è alcuno pari a te.
\par 8 Ma costoro tutti insieme sono stupidi e insensati; non è che una dottrina di vanità; non è altro che legno;
\par 9 argento battuto in lastre portato da Tarsis, oro venuto da Ufaz, opera di scultore e di man d'orefice; son vestiti di porpora e di scarlatto, son tutti lavoro d'abili artefici.
\par 10 Ma l'Eterno è il vero Dio, egli è l'Iddio vivente, e il re eterno; per l'ira sua trema la terra, e le nazioni non possono reggere dinanzi al suo sdegno.
\par 11 Così direte loro: 'Gli dèi che non han fatto i cieli e la terra, scompariranno di sulla terra e di sotto il cielo'.
\par 12 Egli, con la sua potenza, ha fatto la terra; con la sua sapienza ha stabilito fermamente il mondo; con la sua intelligenza ha disteso i cieli.
\par 13 Quando fa udire la sua voce c'è un rumor d'acque nel cielo; ei fa salire i vapori dalle estremità della terra, fa guizzare i lampi per la pioggia e trae il vento dai suoi serbatoi;
\par 14 ogni uomo allora diventa stupido, privo di conoscenza; ogni orafo ha vergogna delle sue immagini scolpite; perché le sue immagini fuse sono una menzogna, e non v'è soffio vitale in loro.
\par 15 Sono vanità, lavoro d'inganno; nel giorno del castigo, periranno.
\par 16 A loro non somiglia Colui ch'è la parte di Giacobbe; perché Egli è quel che ha formato tutte le cose, e Israele è la tribù della sua eredità. Il suo nome è l'Eterno degli eserciti.
\par 17 Raccogli da terra il tuo bagaglio, o tu che sei cinta d'assedio!
\par 18 Poiché così parla l'Eterno: Ecco, questa volta io lancerò lontano gli abitanti del paese, e li stringerò da presso affinché non isfuggano.
\par 19 Guai a me a motivo della mia ferita! La mia piaga è dolorosa; ma io ho detto: 'Questo è il mio male, e lo devo sopportare'.
\par 20 Le mie tende son guaste, e tutto il mio cordame è rotto; i miei figliuoli sono andati lungi da me e non sono più; non v'è più alcuno che stenda la mia tenda, che drizzi i miei padiglioni.
\par 21 Perché i pastori sono stati stupidi, e non hanno cercato l'Eterno; perciò non hanno prosperato, e tutto il loro gregge è stato disperso.
\par 22 Ecco, un rumore giunge, un gran tumulto arriva dal paese del settentrione, per ridurre le città di Giuda in desolazione, in un ricetto di sciacalli.
\par 23 O Eterno, io so che la via dell'uomo non è in suo potere, e che non è in poter dell'uomo che cammina il dirigere i suoi passi.
\par 24 O Eterno, correggimi, ma con giusta misura; non nella tua ira, che tu non abbia a ridurmi a poca cosa!
\par 25 Riversa la tua ira sulle nazioni che non ti conoscono, e sui popoli che non invocano il tuo nome; poiché hanno divorato Giacobbe; sì, lo hanno divorato, l'han consumato, han desolato la sua dimora.

\chapter{11}

\par 1 La parola che fu rivolta a Geremia da parte dell'Eterno, in questi termini:
\par 2 'Ascoltate le parole di questo patto, e parlate agli uomini di Giuda e agli abitanti di Gerusalemme!
\par 3 Di' loro: - Così parla l'Eterno, l'Iddio d'Israele: Maledetto l'uomo che non ascolta le parole di questo patto,
\par 4 che io comandai ai vostri padri il giorno che li feci uscire dal paese d'Egitto, dalla fornace di ferro, dicendo: Ascoltate la mia voce e fate tutto quello che vi comanderò, e voi sarete mio popolo e io sarò vostro Dio,
\par 5 affinché io possa mantenere il giuramento che feci ai vostri padri, di dar loro un paese dove scorre il latte e il miele, come oggi vedete ch'esso è'. Allora io risposi: 'Amen, o Eterno!'
\par 6 L'Eterno mi disse: 'Proclama tutte queste parole nelle città di Giuda e per le strade di Gerusalemme, dicendo: - Ascoltate le parole di questo patto, e mettetele ad effetto!
\par 7 Poiché io ho scongiurato i vostri padri dal giorno che li trassi fuori dal paese d'Egitto fino a questo giorno, li ho scongiurati fin dal mattino, dicendo: - Ascoltate la mia voce! -
\par 8 Ma essi non l'hanno ascoltata, non hanno prestato orecchio, e hanno camminato, seguendo ciascuno la caparbietà del loro cuore malvagio; perciò io ho fatto venir su loro tutto quello che avevo detto in quel patto che io avevo comandato loro d'osservare, e ch'essi non hanno osservato'.
\par 9 Poi l'Eterno mi disse: 'Esiste una congiura fra gli uomini di Giuda e fra gli abitanti di Gerusalemme.
\par 10 Son tornati alle iniquità dei loro padri antichi, i quali ricusarono di ascoltare le mie parole; e sono andati anch'essi dietro ad altri dèi, per servirli; la casa d'Israele e la casa di Giuda hanno rotto il patto, che io avevo fatto coi loro padri.
\par 11 Perciò, così parla l'Eterno: - Ecco, io faccio venir su loro una calamità, alla quale non potranno sfuggire. Essi grideranno a me, ma io non li ascolterò.
\par 12 Allora le città di Giuda e gli abitanti di Gerusalemme andranno a gridare agli dèi ai quali offron profumi; ma essi non li salveranno, nel tempo della calamità!
\par 13 Poiché, o Giuda, tu hai tanti dèi quante sono le tue città; e quante sono le strade di Gerusalemme, tanti altari avete eretti all'infamia, altari per offrir profumi a Baal.
\par 14 E tu non pregare per questo popolo, non ti mettere a gridare né a far supplicazioni per loro; perché io non li esaudirò quando grideranno a me a motivo della calamità che li avrà colpiti.
\par 15 Che ha da fare l'amato mio nella mia casa? Delle scelleratezze? Forse che dei voti e della carne consacrata allontaneranno da te la calamità perché tu possa rallegrarti?
\par 16 L'Eterno t'aveva chiamato 'Ulivo verdeggiante, adorno di bei frutti'. Al rumore di un gran tumulto, egli v'appicca il fuoco e i rami ne sono infranti.
\par 17 L'Eterno degli eserciti che t'avea piantato pronunzia del male contro di te, a motivo della malvagità commessa a loro danno dalla casa d'Israele e dalla casa di Giuda allorché m'hanno provocato ad ira, offrendo profumi a Baal'.
\par 18 L'Eterno me l'ha fatto sapere, ed io l'ho saputo; allora tu m'hai mostrato le loro azioni.
\par 19 Io ero come un docile agnello che si mena al macello; io non sapevo che ordissero macchinazioni contro di me, dicendo: - 'Distruggiamo l'albero col suo frutto e sterminiamolo dalla terra dei viventi; affinché il suo nome non sia più ricordato'. -
\par 20 Ma, o Eterno degli eserciti, giusto giudice, che scruti le reni ed il cuore, io vedrò la tua vendetta su di loro, poiché a te io rimetto la mia causa.
\par 21 Perciò, così parla l'Eterno riguardo a que' di Anatoth, che cercan la tua vita e dicono: 'Non profetare nel nome dell'Eterno, se non vuoi morire per le nostre mani';
\par 22 perciò, così parla l'Eterno degli eserciti: Ecco, io sto per punirli; i giovani morranno per la spada, i loro figliuoli e le loro figliuole morranno di fame;
\par 23 e non resterà di loro alcun residuo; poiché io farò venire la calamità su quei d'Anatoth, l'anno in cui li visiterò.

\chapter{12}

\par 1 Tu sei giusto, o Eterno, quand'io contendo teco; nondimeno io proporrò le mie ragioni: Perché prospera la via degli empi? Perché son tutti a loro agio quelli che procedono perfidamente?
\par 2 Tu li hai piantati, essi hanno messo radice, crescono ed anche portano frutto; tu sei vicino alla loro bocca, ma lontano dal loro interiore.
\par 3 E tu, o Eterno, tu mi conosci, tu mi vedi, tu provi qual sia il mio cuore verso di te. Trascinali al macello come pecore, e preparali per il giorno del massacro!
\par 4 Fino a quando farà cordoglio il paese, e si seccherà l'erba di tutta la campagna? Per la malvagità degli abitanti, le bestie e gli uccelli sono sterminati. Poiché quelli dicono: 'Egli non vedrà la nostra fine'.
\par 5 - Se, correndo con de' pedoni, questi ti stancano, come potrai lottare coi cavalli? E se non ti senti al sicuro che in terra di pace, come farai quando il Giordano sarà gonfio?
\par 6 Perché perfino i tuoi fratelli e la casa di tuo padre ti tradiscono; anch'essi ti gridan dietro a piena voce: non li credere quando ti diranno delle buone parole.
\par 7 Io ho lasciato la mia casa, ho abbandonato la mia eredità; ho dato quello che l'anima mia ha di più caro, nelle mani de' suoi nemici.
\par 8 La mia eredità è divenuta per me come un leone nella foresta; ha mandato contro di me il suo ruggito; perciò io l'ho odiata.
\par 9 La mia eredità è stata per me come l'uccello rapace screziato; gli uccelli rapaci si gettan contro di lei da ogni parte. Andate, radunate tutte le bestie della campagna, fatele venire a divorare!
\par 10 Molti pastori guastano la mia vigna, calpestano la porzione che m'è toccata, riducono la mia deliziosa porzione in un deserto desolato.
\par 11 La riducono in una desolazione; e, tutta desolata, fa cordoglio dinanzi a me; tutto il paese è desolato, perché nessuno lo prende a cuore.
\par 12 Su tutte le alture del deserto giungono devastatori, perché la spada dell'Eterno divora il paese da un'estremità all'altra; nessuna carne ha pace.
\par 13 Han seminato grano, e raccolgono spine; si sono affannati senz'alcun profitto. Vergognatevi di ciò che raccogliete a motivo dell'ardente ira dell'Eterno!
\par 14 Così parla l'Eterno contro tutti i miei malvagi vicini, che toccano l'eredità ch'io ho data a possedere al mio popolo d'Israele: Ecco, io li svellerò dal loro paese, svellerò la casa di Giuda di fra loro;
\par 15 ma, dopo che li avrò divelti, avrò di nuovo compassione di loro, e li ricondurrò ciascuno nella sua eredità, ciascuno nel suo paese.
\par 16 E se pure imparano le vie del mio popolo e a giurare per il mio nome dicendo: 'l'Eterno vive', come hanno insegnato al mio popolo a giurare per Baal, saranno saldamente stabiliti in mezzo al mio popolo.
\par 17 Ma, se non danno ascolto, io svellerò quella nazione; la svellerò e la distruggerò, dice l'Eterno.

\chapter{13}

\par 1 Così mi ha detto l'Eterno: 'Va', comprati una cintura di lino, mettitela sui fianchi, ma non la porre nell'acqua'.
\par 2 Così io comprai la cintura, secondo la parola dell'Eterno, e me la misi sui fianchi.
\par 3 E la parola dell'Eterno mi fu indirizzata per la seconda volta, in questi termini:
\par 4 'Prendi la cintura che hai comprata e che hai sui fianchi; va' verso l'Eufrate, e quivi nascondila nella fessura d'una roccia'.
\par 5 E io andai, e la nascosi presso l'Eufrate, come l'Eterno mi aveva comandato.
\par 6 Dopo molti giorni l'Eterno mi disse: 'Lèvati, va' verso l'Eufrate, e togli di là la cintura, che io t'avevo comandato di nascondervi'.
\par 7 E io andai verso l'Eufrate, e scavai, e tolsi la cintura dal luogo dove l'avevo nascosta; ed ecco, la cintura era guasta, e non era più buona a nulla.
\par 8 Allora la parola dell'Eterno mi fu rivolta in questi termini:
\par 9 Così parla l'Eterno: 'In questo modo io distruggerò l'orgoglio di Giuda e il grande orgoglio di Gerusalemme,
\par 10 di questo popolo malvagio che ricusa di ascoltare le mie parole, che cammina seguendo la caparbietà del suo cuore, e va dietro ad altri dèi per servirli e per prostrarsi dinanzi a loro; esso diventerà come questa cintura, che non è più buona a nulla.
\par 11 Poiché, come la cintura aderisce ai fianchi dell'uomo, così io avevo strettamente unita a me tutta la casa d'Israele e tutta la casa di Giuda, dice l'Eterno, perché fossero mio popolo, mia fama, mia lode, mia gloria; ma essi non han voluto dare ascolto.
\par 12 Tu dirai dunque loro questa parola: Così parla l'Eterno, l'Iddio d'Israele: 'Ogni vaso sarà riempito di vino'; e quando essi ti diranno: 'Non lo sappiamo noi che ogni vaso si riempie di vino?'
\par 13 Allora tu di' loro: Così parla l'Eterno: Ecco, io empirò d'ebbrezza tutti gli abitanti di questo paese, i re che seggono sul trono di Davide, i sacerdoti, i profeti, e tutti gli abitanti di Gerusalemme.
\par 14 Li sbatterò l'uno contro l'altro, padri e figli assieme, dice l'Eterno; io non risparmierò alcuno; nessuna pietà, nessuna compassione, m'impedirà di distruggerli.
\par 15 Ascoltate, porgete orecchio! non insuperbite, perché l'Eterno parla.
\par 16 Date gloria all'Eterno, al vostro Dio, prima ch'ei faccia venir le tenebre, e prima che i vostri piedi inciampino sui monti avvolti nel crepuscolo, e voi aspettiate la luce ed egli ne faccia un'ombra di morte, e la muti in oscurità profonda.
\par 17 Ma se voi non date ascolto, l'anima mia piangerà in segreto, a motivo del vostro orgoglio, gli occhi miei piangeranno dirottamente, si scioglieranno in lacrime, perché il gregge dell'Eterno sarà menato in cattività.
\par 18 Di' al re e alla regina: 'Sedetevi in terra! perché la vostra gloriosa corona vi cade di testa'.
\par 19 Le città del mezzogiorno sono chiuse, e non v'è più chi le apra; tutto Giuda è menato in cattività, è menato in esilio tutto quanto.
\par 20 Alzate gli occhi, e guardate quelli che vengono dal settentrione; dov'è il gregge, il magnifico gregge, che t'era stato dato?
\par 21 Che dirai tu quand'Egli ti punirà? Ma tu stessa hai insegnato ai tuoi amici a dominar su te. Non ti piglieranno i dolori, come piglian la donna che sta per partorire?
\par 22 E se tu dici in cuor tuo: 'Perché m'avvengon queste cose?' Per la grandezza della tua iniquità i lembi della tua veste ti son rimboccati, e i tuoi calcagni sono violentemente scoperti.
\par 23 Un moro può egli mutar la sua pelle o un leopardo le sue macchie? Allora anche voi, abituati come siete a fare il male, potrete fare il bene?
\par 24 E io li disperderò, come stoppia portata via dal vento del deserto.
\par 25 È questa la tua sorte, la parte ch'io ti misuro, dice l'Eterno, perché tu m'hai dimenticato, e hai riposto la tua fiducia nella menzogna.
\par 26 E io pure ti rovescerò i lembi della veste sul viso, sì che si vegga la tua vergogna.
\par 27 Io ho visto le tue abominazioni, i tuoi adulterî, i tuoi nitriti, l'infamia della tua prostituzione sulle colline e per i campi. Guai a te, o Gerusalemme! Quando avverrà mai che tu ti purifichi?'

\chapter{14}

\par 1 La parola dell'Eterno che fu rivolta a Geremia in occasione della siccità.
\par 2 Giuda è in lutto, e le assemblee delle sue porte languiscono, giacciono per terra in abito lugubre; il grido di Gerusalemme sale al cielo.
\par 3 I nobili fra loro mandano i piccoli a cercar dell'acqua; e questi vanno alle cisterne, non trovano acqua, e tornano coi loro vasi vuoti; sono pieni di vergogna, di confusione, e si coprono il capo.
\par 4 Il suolo è costernato perché non v'è stata pioggia nel paese; i lavoratori sono pieni di confusione e si cuoprono il capo.
\par 5 Perfino la cerva nella campagna figlia, e abbandona il suo parto perché non v'è erba;
\par 6 e gli onàgri si fermano sulle alture, aspirano l'aria come gli sciacalli; i loro occhi sono spenti, perché non c'è verdura.
\par 7 O Eterno, se le nostre iniquità testimoniano contro di noi, opera per amor del tuo nome; poiché le nostre infedeltà son molte; noi abbiam peccato contro di te.
\par 8 O speranza d'Israele, suo salvatore in tempo di distretta, perché saresti nel paese come un forestiero, come un viandante che vi si ferma per passarvi la notte?
\par 9 Perché saresti come un uomo sopraffatto, come un prode che non può salvare? Eppure, o Eterno, tu sei in mezzo a noi, e il tuo nome è invocato su noi; non ci abbandonare!
\par 10 Così parla l'Eterno a questo popolo: Essi amano andar vagando; non trattengono i loro piedi; perciò l'Eterno non li gradisce, si ricorda ora della loro iniquità, e punisce i loro peccati.
\par 11 E l'Eterno mi disse: 'Non pregare per il bene di questo popolo.
\par 12 Se digiunano, non ascolterò il loro grido; se fanno degli olocausti e delle offerte, non li gradirò; anzi io sto per consumarli con la spada, con la fame, con la peste'.
\par 13 Allora io dissi: 'Ah, Signore, Eterno! ecco, i profeti dicon loro: - Voi non vedrete la spada, né avrete mai la fame; ma io vi darò una pace sicura in questo luogo'. -
\par 14 E l'Eterno mi disse: 'Que' profeti profetizzano menzogne nel mio nome; io non li ho mandati, non ho dato loro alcun ordine, e non ho parlato loro; le profezie che vi fanno sono visioni menzognere, divinazioni, vanità, imposture del loro proprio cuore.
\par 15 Perciò così parla l'Eterno riguardo ai profeti che profetano nel mio nome benché io non li abbia mandati, e dicono: - Non vi sarà né spada né fame in questo paese; - que' profeti saranno consumati dalla spada e dalla fame;
\par 16 e quelli ai quali essi profetizzano saranno gettati per le vie di Gerusalemme morti di fame e di spada, essi, le loro mogli, e i loro figliuoli e le loro figliuole, né vi sarà chi dia loro sepoltura; e riverserò su loro la loro malvagità'.
\par 17 Di' loro dunque questa parola: Struggansi gli occhi miei in lacrime giorno e notte, senza posa; poiché la vergine figliuola del mio popolo è stata fiaccata in modo straziante, ha ricevuto un colpo tremendo.
\par 18 Se esco per i campi, ecco degli uccisi per la spada; se entro in città, ecco i languenti per fame; perfino il profeta, perfino il sacerdote vanno a mendicare in un paese che non conoscono.
\par 19 Hai tu dunque reietto Giuda? Ha l'anima tua preso in disgusto Sion? Perché ci colpisci senza che ci sia guarigione per noi? Noi aspettavamo la pace, ma nessun bene giunge; aspettavamo un tempo di guarigione, ed ecco il terrore.
\par 20 O Eterno, noi riconosciamo la nostra malvagità, l'iniquità dei nostri padri; poiché noi abbiam peccato contro di te.
\par 21 Per amor del tuo nome, non disdegnare, non disonorare il trono della tua gloria; ricordati del tuo patto con noi; non lo annullare!
\par 22 Fra gl'idoli vani delle genti, ve n'ha egli che possa far piovere? O è forse il cielo che dà gli acquazzoni? Non sei tu, o Eterno, tu, l'Iddio nostro? Perciò noi speriamo in te, poiché tu hai fatto tutte queste cose.

\chapter{15}

\par 1 Ma l'Eterno mi disse: 'Quand'anche Mosè e Samuele si presentassero davanti a me, l'anima mia non si piegherebbe verso questo popolo; caccialo via dalla mia presenza, e ch'ei se ne vada!
\par 2 E se pur ti dicono: - Dove ce ne andremo? tu risponderai loro: - Così dice l'Eterno: Alla morte, i destinati alla morte; alla spada, i destinati alla spada; alla fame, i destinati alla fame; alla cattività, i destinati alla cattività.
\par 3 Io manderò contro di loro quattro specie di flagelli, dice l'Eterno: la spada, per ucciderli; i cani, per trascinarli; gli uccelli del cielo e le bestie della terra, per divorarli e per distruggerli.
\par 4 E farò sì che saranno agitati per tutti i regni della terra, a cagione di Manasse, figliuolo di Ezechia, re di Giuda, e di tutto quello ch'egli ha fatto in Gerusalemme.
\par 5 Poiché chi avrebbe pietà di te, o Gerusalemme? Chi ti compiangerebbe? Chi s'incomoderebbe per domandarti come stai?
\par 6 Tu m'hai respinto, dice l'Eterno; ti sei tirata indietro; perciò io stendo la mano contro di te, e ti distruggo; sono stanco di pentirmi.
\par 7 Io ti ventolo col ventilabro alle porte del paese, privo di figli il mio popolo, e lo faccio perire, poiché non si converte dalle sue vie.
\par 8 Le sue vedove son più numerose della rena del mare; io faccio venire contro di loro, contro la madre de' giovani, un nemico che devasta in pien mezzodì; faccio piombar su lei, a un tratto, angoscia e terrore.
\par 9 Colei che avea partorito sette figliuoli è languente, esala lo spirito; il suo sole tramonta mentr'è giorno ancora; è coperta di vergogna, di confusione; e il rimanente di loro io lo do in balìa della spada de' loro nemici, dice l'Eterno'.
\par 10 Me infelice! o madre mia, poiché m'hai fatto nascere uomo di lite e di contesa per tutto il paese! Io non do né prendo in imprestito, e nondimeno tutti mi maledicono.
\par 11 L'Eterno dice: Per certo, io ti riserbo un avvenire felice; io farò che il nemico ti rivolga supplicazioni nel tempo dell'avversità, nel tempo dell'angoscia.
\par 12 Il ferro potrà esso spezzare il ferro del settentrione ed il rame?
\par 13 Le tue facoltà e i tuoi tesori io li darò gratuitamente come preda, a cagione di tutti i tuoi peccati, e dentro tutti i tuoi confini.
\par 14 E li farò passare coi tuoi nemici in un paese che non conosci; perché un fuoco s'è acceso nella mia ira, che arderà contro di voi.
\par 15 Tu sai tutto, o Eterno; ricordati di me, visitami, e vendicami dei miei persecutori; nella tua longanimità, non mi portar via! riconosci che per amor tuo io porto l'obbrobrio.
\par 16 Tosto che ho trovato le tue parole, io le ho divorate; e le tue parole sono state la mia gioia, l'allegrezza del mio cuore, perché il tuo nome è invocato su me, o Eterno, Dio degli eserciti.
\par 17 Io non mi son seduto nell'assemblea di quelli che ridono, e non mi son rallegrato, ma per cagion della tua mano mi son seduto solitario, perché tu mi riempivi d'indignazione.
\par 18 Perché il mio dolore è desso perpetuo, e la mia piaga, incurabile, ricusa di guarire? Vuoi tu essere per me come una sorgente fallace, come un'acqua che non dura?
\par 19 Perciò, così parla l'Eterno: Se tu torni a me, io ti ricondurrò, e tu ti terrai dinanzi a me; e se tu separi ciò ch'è prezioso da ciò ch'è vile, tu sarai come la mia bocca; ritorneranno essi a te, ma tu non tornerai a loro.
\par 20 Io ti farò essere per questo popolo un forte muro di rame; essi combatteranno contro di te, ma non potranno vincerti, perché io sarò teco per salvarti e per liberarti, dice l'Eterno.
\par 20 Io ti farò essere per questo popolo un forte muro di rame; essi combatteranno contro di te, ma non potranno vincerti, perché io sarò teco per salvarti e per liberarti, dice l'Eterno.

\chapter{16}

\par 1 La parola dell'Eterno mi fu rivolta in questi termini:
\par 2 Non ti prender moglie e non aver figliuoli né figliuole in questo luogo.
\par 3 Poiché così parla l'Eterno riguardo ai figliuoli e alle figliuole che nascono in questo paese, e alle madri che li partoriscono, e ai padri che li generano in questo paese:
\par 4 Essi morranno consunti dalle malattie, non saranno rimpianti, e non avranno sepoltura; serviranno di letame sulla faccia del suolo; saranno consumati dalla spada e dalla fame, e i loro cadaveri saran pasto agli uccelli del cielo, e alle bestie della terra.
\par 5 Poiché così parla l'Eterno: Non entrare nella casa del lutto, non andare a far cordoglio con loro né a compiangerli, perché, dice l'Eterno, io ho ritirato da questo popolo la mia pace, la mia benignità, la mia compassione.
\par 6 Grandi e piccoli morranno in questo paese; non avranno sepoltura, non si farà cordoglio per loro, nessuno si farà incisioni addosso o si raderà per loro;
\par 7 non si romperà per loro il pane del lutto per consolarli d'un morto, e non si offrirà loro a bere la coppa della consolazione per un padre o per una madre.
\par 8 Parimente non entrare in alcuna casa di convito per sederti con loro a mangiare ed a bere.
\par 9 Poiché così parla l'Eterno degli eserciti, l'Iddio d'Israele: Ecco, io farò cessare in questo luogo, davanti ai vostri occhi, ai giorni vostri, il grido di gioia, il grido d'allegrezza, la voce dello sposo e la voce della sposa.
\par 10 E avverrà che quando tu annunzierai a questo popolo tutte queste cose, essi ti diranno: - Perché l'Eterno ha egli pronunziato contro di noi tutta questa grande calamità? Qual è la nostra iniquità? Qual è il peccato che abbiam commesso contro l'Eterno, il nostro Dio? -
\par 11 Allora tu risponderai loro: 'Perché i vostri padri m'hanno abbandonato, dice l'Eterno, sono andati dietro ad altri dèi, li hanno serviti e si son prostrati dinanzi a loro, hanno abbandonato me e non hanno osservato la mia legge.
\par 12 E voi avete fatto anche peggio dei vostri padri; perché, ecco, ciascuno cammina seguendo la caparbietà del suo cuore malvagio, per non dare ascolto a me;
\par 13 perciò io vi caccerò da questo paese in un paese che né voi né i vostri padri avete conosciuto; e quivi servirete giorno e notte ad altri dèi, perché io non vi farò grazia di sorta'.
\par 14 Perciò, ecco, i giorni vengono, dice l'Eterno, che non si dirà più: 'L'Eterno è vivente, egli che trasse i figliuoli d'Israele fuori del paese d'Egitto',
\par 15 ma: 'L'Eterno è vivente, egli che ha tratto i figliuoli d'Israele fuori del paese del settentrione e di tutti gli altri paesi ne' quali egli li aveva cacciati'; e io li ricondurrò nel loro paese, che avevo dato ai loro padri.
\par 16 Ecco, io mando gran numero di pescatori a pescarli, dice l'Eterno; e poi, manderò gran numero di cacciatori a dar loro la caccia sopra ogni monte, sopra ogni collina e nelle fessure delle rocce.
\par 17 Poiché i miei occhi sono su tutte le loro vie; esse non sono nascoste d'innanzi alla mia faccia, e la loro iniquità non rimane occulta agli occhi miei.
\par 18 E prima darò loro al doppio la retribuzione della loro iniquità e del loro peccato, perché hanno profanato il mio paese, con quei cadaveri che sono i loro idoli esecrandi, ed hanno empito la mia eredità delle loro abominazioni.
\par 19 O Eterno, mia forza, mia fortezza, e mio rifugio nel giorno della distretta! A te verranno le nazioni dalle estremità della terra, e diranno: 'I nostri padri non hanno ereditato che menzogne, vanità, e cose che non giovano a nulla'.
\par 20 L'uomo si farebbe egli degli dèi? Ma già cotesti non sono dèi.
\par 21 Perciò, ecco, io farò loro conoscere, questa volta farò loro conoscere la mia mano e la mia potenza; e sapranno che il mio nome è l'Eterno.

\chapter{17}

\par 1 Il peccato di Giuda è scritto con uno stilo di ferro, con una punta di diamante; è scolpito sulla tavola del loro cuore e sui corni de' vostri altari.
\par 2 Come si ricordano dei loro figliuoli, così si ricordano dei loro altari e dei loro idoli d'Astarte presso gli alberi verdeggianti sugli alti colli.
\par 3 O mia montagna che domini la campagna, io darò i tuoi beni e tutti i tuoi tesori e i tuoi alti luoghi come preda, a cagione de' peccati che tu hai commessi entro tutti i tuoi confini!
\par 4 E tu, per tua colpa, perderai l'eredità ch'io t'avevo data, e ti farò servire ai tuoi nemici in un paese che non conosci; perché avete acceso il fuoco della mia ira, ed esso arderà in perpetuo.
\par 5 Così parla l'Eterno: Maledetto l'uomo che confida nell'uomo e fa della carne il suo braccio, e il cui cuore si ritrae dall'Eterno!
\par 6 Egli è come una tamerice nella pianura sterile; e quando giunge il bene, ei non lo vede; dimora in luoghi aridi, nel deserto, in terra salata, senza abitanti.
\par 7 Benedetto l'uomo che confida nell'Eterno, e la cui fiducia è l'Eterno!
\par 8 Egli è come un albero piantato presso all'acque, che distende le sue radici lungo il fiume; non s'accorge quando vien la caldura, e il suo fogliame riman verde; nell'anno della siccità non è in affanno, e non cessa di portar frutto.
\par 9 Il cuore è ingannevole più d'ogni altra cosa, e insanabilmente maligno; chi lo conoscerà?
\par 10 - Io, l'Eterno, che investigo il cuore, che metto alla prova le reni, per retribuire ciascuno secondo le sue vie, secondo il frutto delle sue azioni.
\par 11 Chi acquista ricchezze, ma non con giustizia, è come la pernice che cova uova che non ha fatte; nel bel mezzo de' suoi giorni egli deve lasciarle, e quando arriva la sua fine, non è che uno stolto.
\par 12 Trono di gloria, eccelso fin dal principio, è il luogo del nostro santuario.
\par 13 Speranza d'Israele, o Eterno, tutti quelli che t'abbandonano saranno confusi; quelli che s'allontanano da te saranno iscritti sulla polvere, perché hanno abbandonato l'Eterno, la sorgente delle acque vive.
\par 14 Guariscimi, o Eterno, e sarò guarito; salvami e sarò salvo; poiché tu sei la mia lode.
\par 15 Ecco, essi mi dicono: 'Dov'è la parola dell'Eterno? ch'essa si compia, dunque!'
\par 16 Quanto a me, io non mi son rifiutato d'esser loro pastore agli ordini tuoi, né ho desiderato il giorno funesto, tu lo sai; quello ch'è uscito dalle mie labbra è stato manifesto dinanzi a te.
\par 17 Non esser per me uno spavento; tu sei il mio rifugio nel giorno della calamità.
\par 18 Siano confusi i miei persecutori; non io sia confuso; siano spaventati essi; non io sia spaventato; fa' venir su loro il giorno della calamità, e colpiscili di doppia distruzione!
\par 19 Così m'ha detto l'Eterno: Va', e fermati alla porta de' figliuoli del popolo per la quale entrano ed escono i re di Giuda, e a tutte le porte di Gerusalemme e di' loro:
\par 20 Ascoltate la parola dell'Eterno, o re di Giuda, e tutto Giuda, e voi tutti gli abitanti di Gerusalemme, ch'entrate per queste porte!
\par 21 Così parla l'Eterno: Per amore delle anime vostre, guardatevi dal portare alcun carico e dal farlo passare per le porte di Gerusalemme, in giorno di sabato;
\par 22 e non traete fuori delle vostre case alcun carico e non fate lavoro alcuno in giorno di sabato; ma santificate il giorno del sabato, com'io comandai ai vostri padri.
\par 23 Essi, però, non diedero ascolto, non porsero orecchio, ma indurarono la loro cervice per non ascoltare, e per non ricevere istruzione.
\par 24 E se voi mi date attentamente ascolto, dice l'Eterno, se non fate entrare alcun carico per le porte di questa città in giorno di sabato, ma santificate il giorno del sabato e non fate in esso alcun lavoro,
\par 25 i re ed i principi che seggono sul trono di Davide entreranno per le porte di questa città montati su carri e su cavalli: v'entreranno essi, i loro principi, gli uomini di Giuda, gli abitanti di Gerusalemme; e questa città sarà abitata in perpetuo,
\par 26 e dalle città di Giuda, dai luoghi circonvicini di Gerusalemme, dal paese di Beniamino, dal piano, dal monte e dal mezzodì, si verrà a portare olocausti, vittime, oblazioni, incenso, e ad offrire sacrifizi d'azioni di grazie nella casa dell'Eterno.
\par 27 Ma, se non mi date ascolto e non santificate il giorno del sabato e non v'astenete dal portar de' carichi e dall'introdurne per le porte di Gerusalemme in giorno di sabato, io accenderò un fuoco alle porte della città, ed esso divorerà i palazzi di Gerusalemme, e non s'estinguerà.

\chapter{18}

\par 1 La parola che fu rivolta a Geremia da parte dell'Eterno, in questi termini:
\par 2 'Lèvati, scendi in casa del vasaio, e quivi ti farò udire le mie parole'.
\par 3 Allora io scesi in casa del vasaio, ed ecco egli stava lavorando alla ruota;
\par 4 e il vaso che faceva si guastò, come succede all'argilla in man del vasaio, ed egli da capo ne fece un altro vaso come a lui parve bene di farlo.
\par 5 E la parola dell'Eterno mi fu rivolta in questi termini:
\par 6 'O casa d'Israele, non posso io far di voi quello che fa questo vasaio? dice l'Eterno. Ecco, quel che l'argilla è in mano al vasaio, voi lo siete in mano mia, o casa d'Israele!
\par 7 A un dato momento io parlo riguardo a una nazione, riguardo a un regno, di svellere, d'abbattere, di distruggere;
\par 8 ma se quella nazione contro la quale ho parlato, si converte dalla sua malvagità, io mi pento del male che avevo pensato di farle.
\par 9 Ed ad un altro dato momento io parlo riguardo a una nazione, a un regno, di edificare e di piantare;
\par 10 ma, se quella nazione fa ciò ch'è male agli occhi miei senza dare ascolto alla mia voce, io mi pento del bene di cui avevo parlato di colmarla'.
\par 11 Or dunque parla agli uomini di Giuda e agli abitanti di Gerusalemme, e di': Così parla l'Eterno: Ecco, io preparo contro di voi del male, e formo contro di voi un disegno. Si converta ora ciascun di voi dalla sua via malvagia, ed emendate le vostre vie e le vostre azioni!
\par 12 Ma costoro dicono: 'È inutile; noi vogliamo camminare seguendo i nostri propri pensieri, e vogliamo agire ciascuno seguendo la caparbietà del nostro cuore malvagio'.
\par 13 Perciò, così parla l'Eterno: Chiedete dunque fra le nazioni chi ha udito cotali cose! La vergine d'Israele ha fatto una cosa orribile, enorme.
\par 14 La neve del Libano scompare essa mai dalle rocce che dominano la campagna? O le acque che vengon di lontano, fresche, correnti, s'asciugan esse mai?
\par 15 Eppure il mio popolo m'ha dimenticato, offre profumi agl'idoli vani; l'han tratto a inciampare nelle sue vie, ch'erano i sentieri antichi, per seguire sentieri laterali, una via non appianata,
\par 16 e per far così del loro paese una desolazione, un oggetto di perpetuo scherno; talché tutti quelli che vi passano rimangono stupiti e scuotono il capo.
\par 17 Io li disperderò dinanzi al nemico, come fa il vento orientale; io volterò loro le spalle e non la faccia nel giorno della loro calamità.
\par 18 Ed essi hanno detto: 'Venite, ordiamo macchinazioni contro Geremia; poiché l'insegnamento della legge non verrà meno per mancanza di sacerdoti, né il consiglio per mancanza di savi, né la parola per mancanza di profeti. Venite, colpiamolo con la lingua, e non diamo retta ad alcuna delle sue parole'.
\par 19 Tu dunque, o Eterno, volgi a me la tua attenzione, e odi la voce di quelli che contendono meco.
\par 20 Il male sarà esso reso per il bene? Poiché essi hanno scavato una fossa per l'anima mia. Ricordati com'io mi son presentato dinanzi a te per parlare in loro favore, e per stornare da loro l'ira tua.
\par 21 Perciò abbandona i loro figliuoli alla fame; dàlli essi stessi in balìa della spada; le loro mogli siano orbate di figliuoli, e rimangan vedove; i loro mariti sian feriti a morte; i loro giovani sian colpiti dalla spada in battaglia.
\par 22 Un grido s'oda uscire dalle loro case, quando tu farai piombar su loro a un tratto le bande nemiche: poiché hanno scavata una fossa per pigliarmi, e han teso de' lacci ai miei piedi.
\par 23 E tu, o Eterno, conosci tutti i loro disegni contro di me per farmi morire; non perdonare la loro iniquità, non cancellare il loro peccato d'innanzi ai tuoi occhi! Siano essi rovesciati davanti a te! Agisci contro di loro nel giorno della tua ira!

\chapter{19}

\par 1 Così ha detto l'Eterno: Va', compra una brocca di terra da un vasaio, e prendi teco alcuni degli anziani del popolo e degli anziani dei sacerdoti;
\par 2 récati nella valle del figliuolo d'Hinnom ch'è all'ingresso della porta dei Vasai, e quivi proclama le parole che io ti dirò.
\par 3 Dirai così: Ascoltate la parola dell'Eterno, o re di Giuda, e abitanti di Gerusalemme! Così parla l'Eterno degli eserciti, l'Iddio d'Israele: Ecco, io fo venire sopra questo luogo una calamità, che farà intronar gli orecchi di chi n'udrà parlare;
\par 4 poiché m'hanno abbandonato, hanno profanato questo luogo e vi hanno offerto profumi ad altri dèi, che né essi, né i loro padri, né i re di Giuda hanno conosciuti, e hanno riempito questo luogo di sangue d'innocenti;
\par 5 hanno edificato degli alti luoghi a Baal, per bruciare nel fuoco i loro figliuoli in olocausto a Baal; cosa che io non avevo comandata, della quale non avevo parlato mai, e che non m'era mai venuta in cuore.
\par 6 Perciò, ecco, i giorni vengono, dice l'Eterno, che questo luogo non sarà più chiamato 'Tofet', né 'la valle del figliuolo d'Hinnom', ma 'la valle del Massacro'.
\par 7 Ed io frustrerò i disegni di Giuda e di Gerusalemme in questo luogo, e farò sì che costoro cadano per la spada dinanzi ai loro nemici, e per man di coloro che cercano la loro vita; e darò i loro cadaveri in pasto agli uccelli del cielo e alle bestie della terra.
\par 8 E farò di questa città una desolazione, un oggetto di scherno; chiunque passerà presso di lei rimarrà stupito, e si metterà a fischiare per tutte le sue piaghe.
\par 9 E farò loro mangiare la carne de' loro figliuoli e la carne delle loro figliuole, e mangeranno la carne gli uni degli altri, durante l'assedio e la distretta in cui li stringeranno i loro nemici e quelli che cercano la loro vita.
\par 10 Poi tu spezzerai la brocca in presenza di quegli uomini che saranno andati teco, e dirai loro:
\par 11 Così parla l'Eterno degli eserciti: Così spezzerò questo popolo e questa città, come si spezza un vaso di vasaio, che non si può più accomodare; e si seppelliranno i morti a Tofet, per mancanza di luogo per seppellire.
\par 12 Così, dice l'Eterno, farò a questo luogo ed ai suoi abitanti, rendendo questa città simile a Tofet.
\par 13 E le case di Gerusalemme, e le case dei re di Giuda, saranno come il luogo di Tofet, immonde; tutte le case, cioè, sopra i cui tetti essi hanno offerto profumi a tutto l'esercito del cielo, e han fatto libazioni ad altri dèi.
\par 14 E Geremia tornò da Tofet, dove l'Eterno l'avea mandato a profetare; si fermò nel cortile della casa dell'Eterno, e disse a tutto il popolo:
\par 15 'Così parla l'Eterno degli eserciti, l'Iddio d'Israele: Ecco, io fo venire sopra questa città e sopra tutte le città che da lei dipendono tutte le calamità che ho annunziate contro di lei, perché hanno indurato la loro cervice, per non dare ascolto alle mie parole'.

\chapter{20}

\par 1 Or Pashur, figliuolo d'Immer, sacerdote e capo-soprintendente della casa dell'Eterno, udì Geremia che profetizzava queste cose.
\par 2 E Pashur percosse il profeta Geremia, e lo mise nei ceppi nella prigione ch'era nella porta superiore di Beniamino, nella casa dell'Eterno.
\par 3 E il giorno seguente, Pashur fe' uscire Geremia di carcere. E Geremia gli disse: 'L'Eterno non ti chiama più Pashur, ma Magor-Missabib.
\par 4 Poiché così parla l'Eterno: Io ti renderò un oggetto di terrore a te stesso e a tutti i tuoi amici; essi cadranno per la spada dei loro nemici, e i tuoi occhi lo vedranno; e darò tutto Giuda in mano del re di Babilonia, che li menerà in cattività in Babilonia, e li colpirà con la spada.
\par 5 E darò tutte le ricchezze di questa città e tutto il suo guadagno e tutte le sue cose preziose, darò tutti i tesori dei re di Giuda in mano dei loro nemici che ne faranno lor preda, li piglieranno, e li porteranno via a Babilonia.
\par 6 E tu, Pashur, e tutti quelli che abitano in casa tua, andrete in cattività; tu andrai a Babilonia, e quivi morrai, e quivi sarai sepolto, tu, con tutti i tuoi amici, ai quali hai profetizzato menzogne'.
\par 7 Tu m'hai persuaso, o Eterno, e io mi son lasciato persuadere, tu m'hai fatto forza, e m'hai vinto; io son diventato ogni giorno un oggetto di scherno, ognuno si fa beffe di me.
\par 8 Poiché ogni volta ch'io parlo, grido, grido: 'Violenza e saccheggio!' Sì, la parola dell'Eterno è per me un obbrobrio, uno scherno d'ogni giorno.
\par 9 E s'io dico: 'Io non lo mentoverò più, non parlerò più nel suo nome', v'è nel mio cuore come un fuoco ardente, chiuso nelle mie ossa; e mi sforzo di contenerlo, ma non posso.
\par 10 Poiché odo le diffamazioni di molti, lo spavento mi vien da ogni lato: 'Denunziatelo, e noi lo denunzieremo'. Tutti quelli coi quali vivevo in pace spiano s'io inciampo, e dicono: 'Forse si lascerà sedurre, e noi prevarremo contro di lui, e ci vendicheremo di lui'.
\par 11 Ma l'Eterno è meco, come un potente eroe; perciò i miei persecutori inciamperanno e non prevarranno; saranno coperti di confusione, perché non sono riusciti; l'onta loro sarà eterna, non sarà dimenticata.
\par 12 Ma, o Eterno degli eserciti, che provi il giusto, che vedi le reni e il cuore, io vedrò, sì, la vendetta che prenderai di loro, poiché a te io affido la mia causa!
\par 13 Cantate all'Eterno, lodate l'Eterno, poich'egli libera l'anima dell'infelice dalla mano dei malfattori!
\par 14 Maledetto sia il giorno ch'io nacqui! Il giorno che mia madre mi partorì non sia benedetto!
\par 15 Maledetto sia l'uomo che portò a mio padre la notizia: 'T'è nato un maschio', e lo colmò di gioia!
\par 16 Sia quell'uomo come le città che l'Eterno ha distrutte senza pentirsene! Oda egli delle grida il mattino, e clamori di guerra sul mezzodì;
\par 17 poich'egli non m'ha fatto morire fin dal seno materno. Così mia madre sarebbe stata la mia tomba, e la sua gravidanza, senza fine.
\par 18 Perché son io uscito dal seno materno per vedere tormento e dolore, e per finire i miei giorni nella vergogna?

\chapter{21}

\par 1 La parola che fu rivolta a Geremia da parte dell'Eterno, quando il re Sedekia gli mandò Pashur, figliuolo di Malchia, e Sefonia, figliuolo di Maaseia, il sacerdote, per dirgli:
\par 2 'Deh, consulta per noi l'Eterno; poiché Nebucadnetsar, re di Babilonia, ci fa la guerra; forse l'Eterno farà a pro nostro qualcuna delle sue maraviglie, in guisa che quegli si ritragga da noi'.
\par 3 Allora Geremia disse loro: Direte così a Sedekia:
\par 4 - Così parla l'Eterno, l'Iddio d'Israele: Ecco, io sto per far rientrare nelle città le armi di guerra che sono nelle vostre mani e con le quali voi combattete, fuori delle mura, contro il re di Babilonia, e contro i Caldei che vi assediano, e le raccoglierò in mezzo a questa città.
\par 5 E io stesso combatterò contro di voi con mano distesa e con braccio potente, con ira, con furore, con grande indignazione.
\par 6 E colpirò gli abitanti di questa città, uomini e bestie; e morranno d'un'orrenda peste.
\par 7 Poi, dice l'Eterno, io darò Sedekia, re di Giuda, e i suoi servi, e il popolo, e coloro che in questa città saranno scampati dalla peste, dalla spada e dalla fame, in mano di Nebucadnetsar re di Babilonia, in mano dei loro nemici, in mano di quelli che cercano la loro vita; e Nebucadnetsar li passerà a fil di spada; non li risparmierà, e non ne avrà né pietà né compassione.
\par 8 E a questo popolo dirai: Così parla l'Eterno: Ecco, io pongo dinanzi a voi la via della vita e la via della morte.
\par 9 Colui che rimarrà in questa città morrà per la spada, per la fame o per la peste; ma chi ne uscirà per arrendersi ai Caldei che vi assediano vivrà, e avrà la vita per suo bottino.
\par 10 Poiché io volgo la mia faccia contro questa città per farle del male e non del bene, dice l'Eterno; essa sarà data in mano del re di Babilonia, ed egli la darà alle fiamme.
\par 11 Alla casa dei re di Giuda di': Ascoltate la parola dell'Eterno:
\par 12 O casa di Davide, così dice l'Eterno: Amministrate la giustizia fin dal mattino, e liberate dalla mano dell'oppressore, colui a cui è tolto il suo, affinché l'ira mia non divampi a guisa di fuoco, e arda sì che nessuno la possa spegnere, per la malvagità delle vostre azioni.
\par 13 Eccomi contro te, o abitatrice della valle, roccia della pianura, dice l'Eterno. Voi che dite: 'Chi scenderà contro di noi? Chi entrerà nelle nostre dimore?'
\par 14 io vi punirò secondo il frutto delle vostre azioni, dice l'Eterno; e appiccherò il fuoco a questa selva di Gerusalemme, ed esso divorerà tutto quello che la circonda.

\chapter{22}

\par 1 Così parla l'Eterno: Scendi nella casa del re di Giuda, e pronunzia quivi questa parola, e di':
\par 2 Ascolta la parola dell'Eterno, o re di Giuda, che siedi sul trono di Davide: tu, i tuoi servitori e il tuo popolo, che entrate per queste porte!
\par 3 Così parla l'Eterno: Fate ragione e giustizia, liberate dalla mano dell'oppressore colui al quale è tolto il suo, non fate torto né violenza allo straniero, all'orfano e alla vedova, e non spargete sangue innocente, in questo luogo.
\par 4 Poiché, se metterete realmente ad effetto questa parola, dei re assisi sul trono di Davide entreranno per le porte di questa casa, montati su carri e su cavalli: essi, i loro servitori e il loro popolo.
\par 5 Ma, se non date ascolto a queste parole, io giuro per me stesso, dice l'Eterno, che questa casa sarà ridotta in una rovina.
\par 6 Poiché così parla l'Eterno riguardo alla casa del re di Giuda: Tu eri per me come Galaad, come la vetta del Libano. Ma, certo, io ti ridurrò simile a un deserto, a delle città disabitate.
\par 7 Preparo contro di te dei devastatori, armati ciascuno delle sue armi; essi abbatteranno i cedri tuoi più belli, e li getteranno nel fuoco.
\par 8 Molte nazioni passeranno presso questa città, e ognuno dirà all'altro: 'Perché l'Eterno ha egli fatto così a questa grande città?'
\par 9 E si risponderà: 'Perché hanno abbandonato il patto dell'Eterno, del loro Dio, perché si son prostrati davanti ad altri dèi, e li hanno serviti'.
\par 10 Non piangete per il morto, non vi affliggete per lui; ma piangete, piangete per colui che se ne va, perché non tornerà più e non vedrà più il suo paese natìo.
\par 11 Poiché così parla l'Eterno, riguardo a Shallum, figliuolo di Giosia, re di Giuda, che regnava in luogo di Giosia suo padre, e ch'è uscito da questo luogo: Egli non vi ritornerà più;
\par 12 ma morrà nel luogo dove l'hanno menato in cattività, e non vedrà più questo paese.
\par 13 Guai a colui ch'edifica la sua casa senza giustizia, e le sue camere senza equità; che fa lavorare il prossimo per nulla, e non gli paga il suo salario;
\par 14 e dice: 'Mi edificherò una casa grande e delle camere spaziose', e vi fa eseguire delle finestre, la riveste di legno di cedro e la dipinge di rosso!
\par 15 Regni tu forse perché hai la passione del cedro? Tuo padre non mangiava egli e non beveva? Ma faceva ciò ch'è retto e giusto, e tutto gli andava bene.
\par 16 Egli giudicava la causa del povero e del bisognoso, e tutto gli andava bene. Questo non è egli conoscermi? dice l'Eterno.
\par 17 Ma tu non hai occhi né cuore che per la tua cupidigia, per spargere sangue innocente, e per fare oppressione e violenza.
\par 18 Perciò, così parla l'Eterno riguardo a Joiakim, figliuolo di Giosia, re di Giuda: Non se ne farà cordoglio, dicendo: 'Ahimè, fratel mio, ahimè sorella!' Non se ne farà cordoglio, dicendo: 'Ahimè, signore, ahimè sua maestà!'
\par 19 Sarà sepolto come si seppellisce un asino, trascinato e gettato fuori delle porte di Gerusalemme.
\par 20 Sali sul Libano e grida, alza la voce in Basan, e grida dall'Abarim, perché tutti i tuoi amanti sono distrutti.
\par 21 Io t'ho parlato al tempo della tua prosperità, ma tu dicevi: 'Io non ascolterò'. Questo è stato il tuo modo di fare fin dalla tua fanciullezza; tu non hai mai dato ascolto alla mia voce.
\par 22 Tutti i tuoi pastori saranno pastura del vento, e i tuoi amanti andranno in cattività; e allora sarai svergognata, confusa, per tutta la tua malvagità.
\par 23 O tu che dimori sul Libano, che t'annidi fra i cedri, come farai pietà quando ti coglieranno i dolori, le doglie pari a quelle d'una donna di parto!
\par 24 Com'è vero ch'io vivo, dice l'Eterno, quand'anche Conia, figliuolo di Joiakim, re di Giuda, fosse un sigillo nella mia destra, io ti strapperei di lì.
\par 25 Io ti darò in mano di quelli che cercan la tua vita, in mano di quelli de' quali hai paura, in mano di Nebucadnetsar, re di Babilonia, in mano de' Caldei.
\par 26 E caccerò te e tua madre che t'ha partorito, in un paese straniero dove non siete nati, e quivi morrete.
\par 27 Ma quanto al paese al quale brameranno tornare, essi non vi torneranno.
\par 28 Questo Conia è egli dunque un vaso spezzato, infranto? È egli un oggetto che non fa più alcun piacere? Perché son dunque cacciati, egli e la sua progenie, lanciati in un paese che non conoscono?
\par 29 O paese, o paese, o paese, ascolta la parola dell'Eterno!
\par 30 Così parla l'Eterno: Inscrivete quest'uomo come privo di figliuoli, come un uomo che non prospererà durante i suoi giorni; perché nessuno della sua progenie giungerà a sedersi sul trono di Davide, ed a regnare ancora su Giuda.

\chapter{23}

\par 1 Guai ai pastori che distruggono e disperdono il gregge del mio pascolo! dice l'Eterno.
\par 2 Perciò così parla l'Eterno, l'Iddio d'Israele, riguardo ai pastori che pascono il mio popolo: Voi avete disperse le mie pecore, le avete scacciate, e non ne avete avuto cura; ecco, io vi punirò, per la malvagità delle vostre azioni, dice l'Eterno.
\par 3 E raccoglierò il rimanente delle mie pecore da tutti i paesi dove le ho cacciate, e le ricondurrò ai loro pascoli, e saranno feconde, e moltiplicheranno.
\par 4 E costituirò su loro de' pastori che le pastureranno, ed esse non avranno più paura né spavento, e non ne mancherà alcuna, dice l'Eterno.
\par 5 Ecco, i giorni vengono, dice l'Eterno, quand'io farò sorgere a Davide un germoglio giusto, il quale regnerà da re e prospererà, e farà ragione e giustizia nel paese.
\par 6 Ai giorni d'esso, Giuda sarà salvato, e Israele starà sicuro nella sua dimora: e questo sarà il nome col quale sarà chiamato: 'l'Eterno nostra giustizia'.
\par 7 Perciò, ecco, i giorni vengono, dice l'Eterno, che non si dirà più: 'L'Eterno è vivente, egli che ha tratto i figliuoli d'Israele fuori del paese d'Egitto',
\par 8 ma: 'l'Eterno è vivente, egli che ha tratto fuori e ha ricondotto la progenie della casa d'Israele dal paese del settentrione, e da tutti i paesi dove io li avevo cacciati'; ed essi dimoreranno nel loro paese.
\par 9 Contro i profeti. Il cuore mi si spezza in seno, tutte le mie ossa tremano; io sono come un ubriaco, come un uomo sopraffatto dal vino, a cagione dell'Eterno e a cagione delle sue parole sante.
\par 10 Poiché il paese è pieno di adulteri; poiché il paese fa cordoglio a motivo della maledizione che lo colpisce; i pascoli del deserto sono inariditi. La corsa di costoro è diretta al male, la loro forza non tende al bene.
\par 11 Profeti e sacerdoti sono empi, nella mia casa stessa ho trovato la loro malvagità, dice l'Eterno.
\par 12 Perciò la loro via sarà per loro come luoghi lùbrici in mezzo alle tenebre; essi vi saranno spinti, e cadranno; poiché io farò venir su loro la calamità, l'anno in cui li visiterò, dice l'Eterno.
\par 13 Avevo ben visto cose insulse tra i profeti di Samaria; profetizzavano nel nome di Baal, e traviavano il mio popolo d'Israele.
\par 14 Ma fra i profeti di Gerusalemme ho visto cose nefande: commettono adulteri, procedono con falsità, fortificano le mani dei malfattori, talché nessuno si converte dalla sua malvagità; tutti quanti sono per me come Sodoma, e gli abitanti di Gerusalemme, come quei di Gomorra.
\par 15 Perciò così parla l'Eterno degli eserciti riguardo ai profeti: Ecco, io farò loro mangiare dell'assenzio, e farò loro bere dell'acqua avvelenata; poiché dai profeti di Gerusalemme l'empietà s'è sparsa per tutto il paese.
\par 16 Così parla l'Eterno degli eserciti: Non ascoltate le parole de' profeti che vi profetizzano; essi vi pascono di cose vane; vi espongono le visioni del loro proprio cuore, e non ciò che procede dalla bocca dell'Eterno.
\par 17 Dicono del continuo a quei che mi sprezzano: 'L'Eterno ha detto: Avrete pace'; e a tutti quelli che camminano seguendo la caparbietà del proprio cuore: 'Nessun male v'incoglierà';
\par 18 poiché chi ha assistito al consiglio dell'Eterno, chi ha veduto, chi ha udito la sua parola? Chi ha prestato orecchio alla sua parola e l'ha udita?
\par 19 Ecco, la tempesta dell'Eterno, il furore scoppia, la tempesta scroscia, scroscia sul capo degli empi.
\par 20 L'ira dell'Eterno non si acqueterà, finché non abbia eseguito, compiuto i disegni del suo cuore; negli ultimi giorni, lo capirete appieno.
\par 21 Io non ho mandato quei profeti; ed essi son corsi; io non ho parlato loro, ed essi hanno profetizzato.
\par 22 Se avessero assistito al mio consiglio, avrebbero fatto udire le mie parole al mio popolo, e li avrebbero stornati dalla loro cattiva via e dalla malvagità delle loro azioni.
\par 23 Son io soltanto un Dio da vicino, dice l'Eterno, e non un Dio da lungi?
\par 24 Potrebbe uno nascondersi in luogo occulto sì ch'io non lo vegga? dice l'Eterno. Non riempio io il cielo e la terra? dice l'Eterno.
\par 25 Io ho udito quel che dicono i profeti che profetizzano menzogne nel mio nome, dicendo: 'Ho avuto un sogno! ho avuto un sogno!'
\par 26 Fino a quando durerà questo? Hanno essi in mente, questi profeti che profetizzan menzogne, questi profeti dell'inganno del cuor loro,
\par 27 pensan essi di far dimenticare il mio nome al mio popolo coi loro sogni che si raccontan l'un l'altro, come i loro padri dimenticarono il mio nome per Baal?
\par 28 Il profeta che ha avuto un sogno, racconti il sogno, e colui che ha udito la mia parola riferisca la mia parola fedelmente. Che ha da fare la paglia col frumento? dice l'Eterno.
\par 29 La mia parola non è essa come il fuoco? dice l'Eterno; e come un martello che spezza il sasso?
\par 30 Perciò, ecco, dice l'Eterno, io vengo contro i profeti che ruban gli uni agli altri le mie parole.
\par 31 Ecco, dice l'Eterno, io vengo contro i profeti che fan parlar la loro propria lingua, eppure dicono: 'Egli dice'.
\par 32 Ecco, dice l'Eterno, io vengo contro quelli che profetizzano sogni falsi, che li raccontano e traviano il mio popolo con le loro menzogne e con la loro temerità, benché io non li abbia mandati e non abbia dato loro alcun ordine, ed essi non possan recare alcun giovamento a questo popolo, dice l'Eterno.
\par 33 Se questo popolo o un profeta o un sacerdote ti domandano: 'Qual è l'oracolo dell'Eterno?' Tu risponderai loro: 'Qual oracolo? Io vi rigetterò, dice l'Eterno'.
\par 34 E quanto al profeta, al sacerdote, o al popolo che dirà: 'Oracolo dell'Eterno', io lo punirò: lui, e la sua casa.
\par 35 Direte così, ognuno al suo vicino, ognuno al suo fratello: 'Che ha risposto l'Eterno?' e: 'Che ha detto l'Eterno?'
\par 36 Ma l'oracolo dell'Eterno non lo mentoverete più; poiché la parola di ciascuno sarà per lui il suo oracolo, giacché avete tòrte le parole dell'Iddio vivente, dell'Eterno degli eserciti, dell'Iddio nostro.
\par 37 Tu dirai così al profeta: 'Che t'ha risposto l'Eterno?' e: 'Che ha detto l'Eterno?'
\par 38 E se dite ancora: 'Oracolo dell'Eterno', allora l'Eterno parla così: 'Siccome avete detto questa parola 'oracolo dell'Eterno', benché io v'avessi mandato a dire: 'Non dite più: - Oracolo dell'Eterno',
\par 39 ecco, io vi dimenticherò del tutto, e vi rigetterò lungi dalla mia faccia, voi e la città che avevo data a voi e ai vostri padri,
\par 40 e vi coprirò d'un obbrobrio eterno e d'un'eterna vergogna, che non saran mai dimenticati'.

\chapter{24}

\par 1 L'Eterno mi fece vedere due canestri di fichi, posti davanti al tempio dell'Eterno, dopo che Nebucadnetsar, re di Babilonia, ebbe menato via da Gerusalemme e trasportato in cattività a Babilonia Jeconia, figliuolo di Joiakim, re di Giuda, i capi di Giuda, i falegnami e i fabbri.
\par 2 Uno de' canestri conteneva de' fichi molto buoni, come sono i fichi primaticci; e l'altro canestro conteneva dei fichi molto cattivi, che non si potevano mangiare, tanto eran cattivi.
\par 3 E l'Eterno mi disse: 'Che vedi, Geremia?' Io risposi: 'De' fichi; quelli buoni, molto buoni, e quelli cattivi, molto cattivi, da non potersi mangiare, tanto sono cattivi'.
\par 4 E la parola dell'Eterno mi fu rivolta in questi termini:
\par 5 'Così parla l'Eterno, l'Iddio d'Israele: Quali sono questi fichi buoni, tali saranno que' di Giuda che ho mandati da questo luogo in cattività nel paese de' Caldei; io li riguarderò con favore;
\par 6 l'occhio mio si poserà con favore su loro; e li ricondurrò in questo paese; li stabilirò fermamente, e non li distruggerò più; li pianterò, e non li sradicherò più.
\par 7 E darò loro un cuore, per conoscer me che sono l'Eterno; saranno mio popolo, e io sarò loro Dio, perché si convertiranno a me con tutto il loro cuore.
\par 8 E come si trattano questi fichi cattivi che non si posson mangiare, tanto son cattivi, così, dice l'Eterno, io tratterò Sedekia, re di Giuda, e i suoi principi, e il residuo di que' di Gerusalemme, quelli che son rimasti in questo paese e quelli che abitano nel paese d'Egitto;
\par 9 e farò sì che saranno agitati e maltrattati per tutti i regni della terra; che diventeranno oggetto d'obbrobrio, di proverbio, di sarcasmo e di maledizione in tutti i luoghi dove li caccerò.
\par 10 E manderò contro di loro la spada, la fame, la peste, finché siano scomparsi dal suolo che avevo dato a loro e ai loro padri.

\chapter{25}

\par 1 La parola che fu rivolta a Geremia riguardo a tutto il popolo di Giuda, nel quarto anno di Joiakim, figliuolo di Giosia, re di Giuda (era il primo anno di Nebucadnetsar, re di Babilonia),
\par 2 e che Geremia pronunziò davanti a tutto il popolo di Giuda e a tutti gli abitanti di Gerusalemme, dicendo:
\par 3 Dal tredicesimo anno di Giosia, figliuolo di Amon, re di Giuda, fino ad oggi, son già ventitre anni che la parola dell'Eterno m'è stata rivolta, e che io v'ho parlato del continuo, fin dal mattino, ma voi non avete dato ascolto.
\par 4 L'Eterno vi ha pure mandato tutti i suoi servitori, i profeti; ve li ha mandati del continuo fin dal mattino, ma voi non avete ubbidito, né avete pòrto l'orecchio per ascoltare.
\par 5 Essi hanno detto: 'Convertasi ciascun di voi dalla sua cattiva via e dalla malvagità delle sue azioni, e voi abiterete di secolo in secolo sul suolo che l'Eterno ha dato a voi e ai vostri padri;
\par 6 e non andate dietro ad altri dèi per servirli e per prostrarvi dinanzi a loro; non mi provocate con l'opera delle vostre mani, e io non vi farò male alcuno'.
\par 7 Ma voi non mi avete dato ascolto, dice l'Eterno, per provocarmi, a vostro danno, con l'opera delle vostre mani.
\par 8 Perciò, così dice l'Eterno degli eserciti: Giacché non avete dato ascolto alle mie parole, ecco,
\par 9 io manderò a prendere tutte le nazioni del settentrione, dice l'Eterno, e manderò a chiamare Nebucadnetsar re di Babilonia, mio servitore, e le farò venire contro questo paese e contro i suoi abitanti, e contro tutte le nazioni che gli stanno d'intorno, e li voterò allo sterminio e li abbandonerò alla desolazione, alla derisione, a una solitudine perpetua.
\par 10 E farò cessare fra loro i gridi di gioia e i gridi d'esultanza, il canto dello sposo e il canto della sposa, il rumore della macina, e la luce della lampada.
\par 11 E tutto questo paese sarà ridotto in una solitudine e in una desolazione, e queste nazioni serviranno il re di Babilonia per settant'anni.
\par 12 Ma quando saran compiuti i settant'anni, io punirò il re di Babilonia e quella nazione, dice l'Eterno, a motivo della loro iniquità, e punirò il paese de' Caldei, e lo ridurrò in una desolazione perpetua.
\par 13 E farò venire su quel paese tutte le cose che ho annunziate contro di lui, tutto ciò ch'è scritto in questo libro, ciò che Geremia ha profetizzato contro tutte le nazioni.
\par 14 Infatti, nazioni numerose e re potenti ridurranno in servitù i Caldei stessi; e io li retribuirò secondo le loro azioni, secondo l'opera delle loro mani.
\par 15 Poiché così m'ha parlato l'Eterno, l'Iddio d'Israele: Prendi di mano mia questa coppa del vino della mia ira, e danne a bere a tutte le nazioni alle quali ti manderò.
\par 16 Esse berranno, barcolleranno, saran come pazze, a motivo della spada ch'io manderò fra loro.
\par 17 E io presi la coppa di mano dell'Eterno, e ne diedi a bere a tutte le nazioni alle quali l'Eterno mi mandava:
\par 18 a Gerusalemme e alle città di Giuda, ai suoi re ed ai suoi principi, per abbandonarli alla rovina, alla desolazione, alla derisione, alla maledizione, come oggi si vede;
\par 19 a Faraone, re d'Egitto, ai suoi servitori, ai suoi principi, a tutto il suo popolo;
\par 20 a tutta la mescolanza di popoli, a tutti i re del paese di Ur, a tutti i re del paese de' Filistei, ad Askalon, a Gaza, a Ekron, e al residuo d'Asdod;
\par 21 a Edom, a Moab, e ai figliuoli d'Ammon;
\par 22 a tutti i re di Tiro, a tutti i re di Sidon, e ai re delle isole d'oltremare;
\par 23 a Dedan, a Tema, a Buz, e a tutti quelli che si radono i canti della barba;
\par 24 a tutti i re d'Arabia, e a tutti i re della mescolanza di popoli che abita nel deserto;
\par 25 a tutti i re di Zimri, a tutti i re d'Elam,
\par 26 e a tutti i re di Media e a tutti i re del settentrione, vicini o lontani, agli uni e agli altri, e a tutti i regni del mondo che sono sulla faccia della terra. E il re di Sceshac ne berrà dopo di loro.
\par 27 Tu dirai loro: Così parla l'Eterno degli eserciti, l'Iddio d'Israele: Bevete, ubriacatevi, vomitate, cadete senza rialzarvi più, dinanzi alla spada ch'io mando fra voi.
\par 28 E se ricusano di prender dalla tua mano la coppa per bere, di' loro: Così dice l'Eterno degli eserciti: Voi berrete in ogni modo!
\par 29 Poiché, ecco, io comincio a punire la città sulla quale è invocato il mio nome, e voi rimarreste del tutto impuniti? Voi non rimarrete impuniti; poiché io chiamerò la spada su tutti gli abitanti della terra, dice l'Eterno degli eserciti.
\par 30 E tu, profetizza loro tutte queste cose, e di' loro: L'Eterno rugge dall'alto, e fa risonare la sua voce dalla sua santa dimora; egli rugge fieramente contro la sua residenza; manda un grido, come quelli che calcan l'uva, contro tutti gli abitanti della terra.
\par 31 Il rumore ne giunge fino all'estremità della terra; poiché l'Eterno ha una lite con le nazioni, egli entra in giudizio contro ogni carne; gli empi, li dà in balìa della spada, dice l'Eterno.
\par 32 Così parla l'Eterno degli eserciti: Ecco, una calamità passa di nazione in nazione, e un gran turbine si leva dalle estremità della terra.
\par 33 In quel giorno, gli uccisi dall'Eterno copriranno la terra dall'una all'altra estremità di essa, e non saranno rimpianti, né raccolti, né seppelliti; serviranno di letame sulla faccia del suolo.
\par 34 Urlate, o pastori, gridate, voltolatevi nella polvere, o guide del gregge! Poiché è giunto il tempo in cui dovete essere scannati; io vi frantumerò, e cadrete, come un vaso prezioso.
\par 35 Ai pastori mancherà ogni rifugio, e le guide del gregge non avranno via di scampo.
\par 36 S'ode il grido de' pastori e l'urlo delle guide del gregge; poiché l'Eterno devasta il loro pascolo;
\par 37 e i tranquilli ovili son ridotti al silenzio, a motivo dell'ardente ira dell'Eterno.
\par 38 Egli ha abbandonato il suo ricetto, come un leoncello, perché il loro paese è diventato una desolazione, a motivo del furor della spada crudele, a motivo dell'ardente ira dell'Eterno.

\chapter{26}

\par 1 Nel principio del regno di Joiakim figliuolo di Giosia, re di Giuda, fu pronunziata questa parola da parte dell'Eterno:
\par 2 Così parla l'Eterno: Preséntati nel cortile della casa dell'Eterno, e di' a tutte le città di Giuda che vengono a prostrarsi nella casa dell'Eterno, tutte le parole che io ti comando di dir loro; non ne detrarre verbo.
\par 3 Forse daranno ascolto, e si convertiranno ciascuno dalla sua via malvagia; e io mi pentirò del male che penso di far loro per la malvagità delle loro azioni.
\par 4 Tu dirai loro: Così parla l'Eterno: Se non date ascolto, se non camminate secondo la mia legge che vi ho posta dinanzi,
\par 5 se non date ascolto alle parole dei miei servitori, i profeti, i quali vi mando, che vi ho mandati fin dal mattino e non li avete ascoltati,
\par 6 io tratterò questa casa come Sciloh, e farò che questa città serva di maledizione presso tutte le nazioni della terra.
\par 7 Or i sacerdoti, i profeti e tutto il popolo udirono Geremia che pronunziava queste parole nella casa dell'Eterno.
\par 8 E avvenne che, come Geremia ebbe finito di pronunziare tutto quello che l'Eterno gli aveva comandato di dire a tutto il popolo, i sacerdoti, i profeti e tutto il popolo lo presero, dicendo: - 'Tu devi morire! -
\par 9 Perché hai profetizzato nel nome dell'Eterno dicendo: - Questa casa sarà come Sciloh e questa città sarà devastata, e priva d'abitanti?' - E tutto il popolo s'adunò contro Geremia nella casa dell'Eterno.
\par 10 Quando i capi di Giuda ebbero udite queste cose, salirono dalla casa del re alla casa dell'Eterno, e si sedettero all'ingresso della porta nuova della casa dell'Eterno.
\par 11 E i sacerdoti e i profeti parlarono ai capi e a tutto il popolo, dicendo: 'Quest'uomo merita la morte, perché ha profetizzato contro questa città, nel modo che avete udito coi vostri propri orecchi'.
\par 12 Allora Geremia parlò a tutti i capi e a tutto il popolo, dicendo: 'L'Eterno mi ha mandato a profetizzare contro questa casa e contro questa città tutte le cose che avete udite.
\par 13 Or dunque, emendate le vostre vie e le vostre azioni, date ascolto alla voce dell'Eterno, del vostro Dio, e l'Eterno si pentirà del male che ha pronunziato contro di voi.
\par 14 Quanto a me, eccomi nelle vostre mani; fate di me quello che vi parrà buono e giusto.
\par 15 Soltanto sappiate per certo che, se mi fate morire, mettete del sangue innocente addosso a voi, a questa città e ai suoi abitanti, perché l'Eterno m'ha veramente mandato a voi per farvi udire tutte queste parole'.
\par 16 Allora i capi e tutto il popolo dissero ai sacerdoti e ai profeti: 'Quest'uomo non merita la morte, perché ci ha parlato nel nome dell'Eterno, del nostro Dio'.
\par 17 E alcuni degli anziani del paese si levarono e parlaron così a tutta la raunanza del popolo:
\par 18 'Michea, il Morashtita, profetizzò ai giorni d'Ezechia, re di Giuda, e parlò a tutto il popolo di Giuda in questi termini: Così dice l'Eterno degli eserciti: Sion sarà arata come un campo, Gerusalemme diventerà un monte di ruine, e la montagna del tempio, un'altura boscosa.
\par 19 Ezechia, re di Giuda, e tutto Giuda lo misero essi a morte? Ezechia non temette egli l'Eterno, e non supplicò egli l'Eterno, sì che l'Eterno si pentì del male che aveva pronunziato contro di loro? E noi stiamo per fare un gran male a danno delle anime nostre'.
\par 20 Vi fu anche un altro uomo che profetizzò nel nome dell'Eterno: Uria, figliuolo di Scemaia di Kiriath-Jearim, il quale profetizzò contro questa città e contro questo paese, in tutto e per tutto come Geremia;
\par 21 e quando il re Joiakim, tutti i suoi uomini prodi e tutti i suoi capi ebbero udito le sue parole, il re cercò di farlo morire; ma Uria lo seppe, ebbe paura, fuggì e andò in Egitto;
\par 22 e il re Joiakim mandò degli uomini in Egitto, cioè Elnathan, figliuolo di Acbor, e altra gente con lui.
\par 23 Questi trassero Uria fuori d'Egitto, e lo menarono al re Joiakim, il quale lo colpì con la spada, e gettò il suo cadavere fra le sepolture de' figliuoli del popolo.
\par 24 Ma la mano di Ahikam, figliuolo di Shafan, fu con Geremia, e impedì che fosse dato in man del popolo per esser messo a morte.

\chapter{27}

\par 1 Nel principio del regno di Joiakim, figliuolo di Giosia, re di Giuda, questa parola fu rivolta dall'Eterno a Geremia in questi termini:
\par 2 'Così m'ha detto l'Eterno: Fatti de' legami e dei gioghi, e mettiteli sul collo;
\par 3 poi mandali al re di Edom, al re di Moab, al re de' figliuoli di Ammon, al re di Tiro e al re di Sidone, mediante gli ambasciatori che son venuti a Gerusalemme da Sedekia, re di Giuda;
\par 4 e ordina loro che dicano ai loro signori: Così parla l'Eterno degli eserciti, l'Iddio d'Israele: Direte questo ai vostri signori:
\par 5 Io ho fatto la terra, gli uomini e gli animali che sono sulla faccia della terra, con la mia gran potenza e col mio braccio steso; e do la terra a chi mi par bene.
\par 6 E ora do tutti questi paesi in mano di Nebucadnetsar, re di Babilonia, mio servitore; e gli do pure gli animali della campagna perché gli siano soggetti.
\par 7 E tutte le nazioni saranno soggette a lui, al suo figliuolo e al figliuolo del suo figliuolo, finché giunga il tempo anche pel suo paese; e allora molte nazioni e grandi re lo ridurranno in servitù.
\par 8 E avverrà che la nazione o il regno che non vorrà sottomettersi a lui, a Nebucadnetsar re di Babilonia, e non vorrà piegare il collo sotto il giogo del re di Babilonia, quella nazione io la punirò, dice l'Eterno, con la spada, con la fame, con la peste, finché io non l'abbia sterminata per mano di lui.
\par 9 Voi dunque non ascoltate i vostri profeti, né i vostri indovini, né i vostri sognatori, né i vostri pronosticatori, né i vostri maghi che vi dicono: - Non sarete asserviti al re di Babilonia! -
\par 10 Poiché essi vi profetizzano menzogna, per allontanarvi dal vostro paese, perché io vi scacci e voi periate.
\par 11 Ma la nazione che piegherà il suo collo sotto il giogo del re di Babilonia e gli sarà soggetta, io la lascerò stare nel suo paese, dice l'Eterno; ed essa lo coltiverà e vi dimorerà'.
\par 12 Io parlai dunque a Sedekia, re di Giuda, in conformità di tutte queste parole, e dissi: 'Piegate il collo sotto il giogo del re di Babilonia, sottomettetevi a lui e al suo popolo, e vivrete.
\par 13 Perché morreste, tu e il tuo popolo, per la spada, per la fame e per la peste, come l'Eterno ha detto della nazione che non si assoggetterà al re di Babilonia?
\par 14 E non date ascolto alle parole de' profeti che vi dicono: - Non sarete asserviti al re di Babilonia! - perché vi profetizzano menzogna.
\par 15 Poiché io non li ho mandati, dice l'Eterno; ma profetizzano falsamente nel mio nome, perché io vi scacci, e voi periate: voi e i profeti che vi profetizzano'.
\par 16 Parlai pure ai sacerdoti e a tutto questo popolo, e dissi: 'Così parla l'Eterno: Non date ascolto alle parole dei vostri profeti i quali vi profetizzano, dicendo: - Ecco, gli arredi della casa dell'Eterno saranno in breve riportati da Babilonia, - perché vi profetizzano menzogna.
\par 17 Non date loro ascolto; sottomettetevi al re di Babilonia, e vivrete. Perché questa città sarebb'ella ridotta una desolazione?
\par 18 Se sono profeti, e se la parola dell'Eterno è con loro, intercedano ora presso l'Eterno degli eserciti perché gli arredi che son rimasti nella casa dell'Eterno, nella casa del re di Giuda e in Gerusalemme, non vadano a Babilonia.
\par 19 Perché così parla l'Eterno degli eserciti riguardo alle colonne, al mare, alle basi e al resto degli arredi rimasti in questa città,
\par 20 e che non furon presi da Nebucadnetsar, re di Babilonia, quando menò in cattività da Gerusalemme in Babilonia, Jeconia, figliuolo di Joiakim, re di Giuda, e tutti i nobili di Giuda e di Gerusalemme;
\par 21 così, dico, parla l'Eterno degli eserciti, l'Iddio d'Israele, riguardo agli arredi che rimangono nella casa dell'Eterno, nella casa del re di Giuda e in Gerusalemme:
\par 22 saranno portati a Babilonia, e quivi resteranno, finché io li cercherò, dice l'Eterno, e li farò risalire e ritornare in questo luogo'.

\chapter{28}

\par 1 In quello stesso anno, al principio del regno di Sedekia, re di Giuda, l'anno quarto, il quinto mese, Anania, figliuolo di Azzur, profeta, ch'era di Gabaon, mi parlò nella casa dell'Eterno, in presenza dei sacerdoti e di tutto il popolo, dicendo:
\par 2 'Così parla l'Eterno degli eserciti, l'Iddio d'Israele: Io spezzo il giogo del re di Babilonia.
\par 3 Entro due anni, io farò tornare in questo luogo tutti gli arredi della casa dell'Eterno, che Nebucadnetsar, re di Babilonia, ha tolti da questo luogo e ha portati a Babilonia;
\par 4 e ricondurrò in questo luogo, dice l'Eterno, Jeconia, figliuolo di Joiakim, re di Giuda, e tutti que' di Giuda che sono stati menati in cattività in Babilonia; perché spezzerò il giogo del re di Babilonia'.
\par 5 E il profeta Geremia rispose al profeta Anania in presenza de' sacerdoti e in presenza di tutto il popolo che si trovava nella casa dell'Eterno.
\par 6 Il profeta Geremia disse: 'Amen! Così faccia l'Eterno! L'Eterno mandi ad effetto quel che tu hai profetizzato, e faccia tornare da Babilonia in questo luogo gli arredi della casa dell'Eterno e tutti quelli che sono stati menati in cattività!
\par 7 Però, ascolta ora questa parola che io pronunzio in presenza tua e in presenza di tutto il popolo.
\par 8 I profeti che apparvero prima di me e prima di te fin dai tempi antichi, profetarono contro molti paesi e contro grandi regni la guerra, la fame, la peste.
\par 9 Quanto al profeta che profetizza la pace, allorché si sarà adempiuta la sua parola, egli sarà riconosciuto come un vero mandato dall'Eterno'.
\par 10 Allora il profeta Anania prese il giogo di sul collo del profeta Geremia e lo spezzò.
\par 11 E Anania parlò in presenza di tutto il popolo, e disse: 'Così parla l'Eterno: In questo modo io spezzerò il giogo di Nebucadnetsar, re di Babilonia, di sul collo di tutte le nazioni, entro lo spazio di due anni'. E il profeta Geremia se ne andò.
\par 12 Allora la parola dell'Eterno fu rivolta a Geremia, dopo che il profeta Anania ebbe spezzato il giogo di sul collo del profeta Geremia, e disse:
\par 13 'Va', e di' ad Anania: Così parla l'Eterno: Tu hai spezzato un giogo di legno, ma hai fatto, invece di quello, un giogo di ferro.
\par 14 Poiché così parla l'Eterno degli eserciti, l'Iddio d'Israele: Io metto un giogo di ferro sul collo di tutte queste nazioni perché siano assoggettate a Nebucadnetsar, re di Babilonia; ed esse gli saranno assoggettate; e gli do pure gli animali della campagna'.
\par 15 E il profeta Geremia disse al profeta Anania: 'Ascolta, Anania! L'Eterno non t'ha mandato, e tu hai indotto questo popolo a confidar nella menzogna.
\par 16 Perciò, così parla l'Eterno: Ecco, io ti scaccio di sulla faccia della terra; quest'anno morrai, perché hai parlato di ribellione contro l'Eterno'.
\par 17 E il profeta Anania morì quello stesso anno, nel settimo mese.

\chapter{29}

\par 1 Or queste son le parole della lettera che il profeta Geremia mandò da Gerusalemme al residuo degli anziani in cattività, ai sacerdoti, ai profeti e a tutto il popolo che Nebucadnetsar avea menato in cattività da Gerusalemme in Babilonia,
\par 2 dopo che il re Jeconia, la regina, gli eunuchi, i principi di Giuda e di Gerusalemme, i falegnami e i fabbri furono usciti da Gerusalemme.
\par 3 La lettera fu portata per man di Elasa, figliuolo di Shafan, e di Ghemaria, figliuolo di Hilkia, che Sedekia, re di Giuda, mandava a Babilonia da Nebucadnetsar, re di Babilonia. Essa diceva:
\par 4 Così parla l'Eterno degli eserciti, l'Iddio d'Israele, a tutti i deportati ch'egli ha fatto menare in cattività da Gerusalemme in Babilonia:
\par 5 Fabbricate delle case e abitatele; piantate de' giardini e mangiatene il frutto;
\par 6 prendete delle mogli e generate figliuoli e figliuole; prendete delle mogli per i vostri figliuoli, date marito alle vostre figliuole perché faccian figliuoli e figliuole; e moltiplicate là dove siete, e non diminuite.
\par 7 Cercate il bene della città dove io vi ho fatti menare in cattività e pregate l'Eterno per essa; poiché dal bene d'essa dipende il vostro bene.
\par 8 Poiché così dice l'Eterno degli eserciti, l'Iddio d'Israele: I vostri profeti che sono in mezzo a voi e i vostri indovini non v'ingannino, e non date retta ai sogni che fate.
\par 9 Giacché quelli vi profetano falsamente nel mio nome; io non li ho mandati, dice l'Eterno.
\par 10 Poiché così parla l'Eterno: Quando settant'anni saranno compiuti per Babilonia, io vi visiterò e manderò ad effetto per voi la mia buona parola, facendovi tornare in questo luogo.
\par 11 Poiché io so i pensieri che medito per voi, dice l'Eterno: pensieri di pace e non di male, per darvi un avvenire e una speranza.
\par 12 Voi m'invocherete, verrete a pregarmi e io v'esaudirò.
\par 13 Voi mi cercherete e mi troverete, perché mi cercherete con tutto il vostro cuore;
\par 14 e io mi lascerò trovare da voi, dice l'Eterno, e vi farò tornare dalla vostra cattività; vi raccoglierò di fra tutte le nazioni e da tutti i luoghi dove vi ho cacciati, dice l'Eterno; e vi ricondurrò nel luogo donde vi ho fatti andare in cattività.
\par 15 Voi dite: 'L'Eterno ci ha suscitato de' profeti in Babilonia'.
\par 16 Ebbene, così parla l'Eterno riguardo al re che siede sul trono di Davide, riguardo a tutto il popolo che abita in questa città, ai vostri fratelli che non sono andati con voi in cattività;
\par 17 così parla l'Eterno degli eserciti: Ecco, io manderò contro di loro la spada, la fame, la peste, e li renderò come quegli orribili fichi che non si posson mangiare, tanto sono cattivi.
\par 18 E li inseguirò con la spada, con la fame, con la peste; farò sì che saranno agitati fra tutti i regni della terra, e li abbandonerò alla esecrazione, allo stupore, alla derisione e al vituperio fra tutte le nazioni dove li caccerò;
\par 19 perché non han dato ascolto alle mie parole, dice l'Eterno, che io ho mandate loro a dire dai miei servitori i profeti, del continuo, fin dal mattino; ma essi non han dato ascolto, dice l'Eterno.
\par 20 Ascoltate dunque la parola dell'Eterno, o voi tutti, che io ho mandati in cattività da Gerusalemme in Babilonia!
\par 21 Così parla l'Eterno degli eserciti, l'Iddio d'Israele, riguardo ad Achab, figliuolo di Kolaia, e riguardo a Sedekia, figliuolo di Maaseia, che vi profetizzano la menzogna nel mio nome: Ecco, io do costoro in mano di Nebucadnetsar, re di Babilonia, ed ei li metterà a morte davanti agli occhi vostri;
\par 22 da essi trarrà una formula di maledizione fra tutti quei di Giuda che sono in cattività in Babilonia, e si dirà: 'L'Eterno ti tratti come Sedekia e come Achab, che il re di Babilonia ha fatti arrostire al fuoco!'
\par 23 Perché costoro han fatto delle cose nefande in Israele, han commesso adulterio con le mogli del loro prossimo, e hanno pronunziato in mio nome parole di menzogna; il che io non avevo loro comandato. Io stesso lo so, e ne son testimone, dice l'Eterno.
\par 24 E quanto a Scemaia il Nehelamita, gli parlerai in questo modo:
\par 25 Così dice l'Eterno degli eserciti, l'Iddio d'Israele: Tu hai mandato in tuo nome una lettera a tutto il popolo che è in Gerusalemme, a Sofonia, figliuolo di Maaseia, il sacerdote, e a tutti i sacerdoti, per dire:
\par 26 'L'Eterno ti ha costituito sacerdote in luogo del sacerdote Jehoiada, perché vi siano nella casa dell'Eterno de' sovrintendenti per sorvegliare ogni uomo che è pazzo e che fa il profeta, e perché tu lo metta ne' ceppi e ai ferri.
\par 27 E ora perché non reprimi tu Geremia d'Anatoth che fa il profeta tra voi,
\par 28 e ci ha perfino mandato a dire a Babilonia: La cattività sarà lunga; fabbricate delle case e abitatele; piantate dei giardini e mangiatene il frutto?'
\par 29 - Or il sacerdote Sofonia lesse questa lettera in presenza del profeta Geremia. -
\par 30 E la parola dell'Eterno fu rivolta a Geremia, dicendo:
\par 31 Manda a dire a tutti quelli che sono in cattività: Così parla l'Eterno riguardo a Scemaia il Nehelamita: Poiché Scemaia vi ha profetato, benché io non l'abbia mandato, e vi ha fatto confidare nella menzogna,
\par 32 così parla l'Eterno: Ecco, io punirò Scemaia il Nehelamita, e la sua progenie; non vi sarà alcuno de' suoi discendenti che abiti in mezzo a questo popolo, ed egli non vedrà il bene che io farò al mio popolo, dice l'Eterno; poich'egli ha parlato di ribellione contro l'Eterno.

\chapter{30}

\par 1 La parola che fu rivolta a Geremia dall'Eterno, in questi termini:
\par 2 'Così parla l'Eterno, l'Iddio d'Israele: Scriviti in un libro tutte le parole che t'ho dette;
\par 3 poiché, ecco, i giorni vengono, dice l'Eterno, quando io ritrarrò dalla cattività il mio popolo d'Israele e di Giuda, dice l'Eterno, e li ricondurrò nel paese che diedi ai loro padri, ed essi lo possederanno'.
\par 4 Queste sono le parole che l'Eterno ha pronunziate riguardo ad Israele ed a Giuda.
\par 5 Così parla l'Eterno: Noi udiamo un grido di terrore, di spavento, e non di pace.
\par 6 Informatevi e guardate se un maschio partorisce! Perché dunque vedo io tutti gli uomini con le mani sui fianchi come donna partoriente? Perché tutte le facce son diventate pallide?
\par 7 Ahimè, perché quel giorno è grande; non ve ne fu mai altro di simile; è un tempo di distretta per Giacobbe; ma pure ei ne sarà salvato.
\par 8 In quel giorno, dice l'Eterno degli eserciti, io spezzerò il suo giogo di sul tuo collo, e romperò i tuoi legami; e gli stranieri non ti faran più loro schiavo;
\par 9 ma quei d'Israele serviranno l'Eterno, il loro Dio, e Davide lor re, che io susciterò loro.
\par 10 Tu dunque, o Giacobbe, mio servitore, non temere, dice l'Eterno; non ti sgomentare, o Israele; poiché, ecco, io ti salverò dal lontano paese, salverò la tua progenie dalla terra della sua cattività; Giacobbe ritornerà, sarà in riposo, sarà tranquillo, e nessuno più lo spaventerà.
\par 11 Poiché io son teco, dice l'Eterno, per salvarti; io annienterò tutte le nazioni fra le quali t'ho disperso, ma non annienterò te; però, ti castigherò con giusta misura, e non ti lascerò del tutto impunito.
\par 12 Così parla l'Eterno: La tua ferita è incurabile, la tua piaga è grave.
\par 13 Nessuno prende in mano la tua causa per fasciar la tua piaga; tu non hai medicamenti atti a guarirla.
\par 14 Tutti i tuoi amanti t'hanno dimenticata, non si curano più di te; poiché io t'ho percossa come si percuote un nemico, t'ho inflitto la correzione d'un uomo crudele, per la grandezza della tua iniquità, perché i tuoi peccati sono andati aumentando.
\par 15 Perché gridi a causa della tua ferita? Il tuo dolore è insanabile. Io ti ho fatto queste cose per la grandezza della tua iniquità, perché i tuoi peccati sono andati aumentando.
\par 16 Nondimeno, tutti quelli che ti divorano saran divorati, tutti i tuoi nemici, tutti quanti, andranno in cattività; quelli che ti spogliano saranno spogliati, quelli che ti saccheggiano li abbandonerò al saccheggio.
\par 17 Ma io medicherò le tue ferite, ti guarirò delle tue piaghe, dice l'Eterno, poiché ti chiaman 'la scacciata', 'la Sion di cui nessuno si cura'.
\par 18 Così parla l'Eterno: Ecco, io traggo dalla cattività le tende di Giacobbe, ed ho pietà delle sue dimore; le città saranno riedificate sulle loro rovine, e i palazzi saranno abitati come di consueto.
\par 19 E ne usciranno azioni di grazie, voci di gente festeggiante. Io li moltiplicherò e non saranno più ridotti a pochi; li renderò onorati e non saran più avviliti.
\par 20 I suoi figliuoli saranno come furono un tempo, la sua raunanza sarà stabilita dinanzi a me, e io punirò tutti i loro oppressori.
\par 21 Il loro principe sarà uno d'essi, e chi li signoreggerà uscirà di mezzo a loro; io lo farò avvicinare, ed egli verrà a me; poiché chi disporrebbe il suo cuore ad accostarsi a me? dice l'Eterno.
\par 22 Voi sarete mio popolo, e io sarò vostro Dio.
\par 23 Ecco la tempesta dell'Eterno; il furore scoppia; la tempesta imperversa; scroscia sul capo degli empi.
\par 24 L'ardente ira dell'Eterno non s'acqueterà, finché non abbia eseguiti, compiuti i disegni del suo cuore; negli ultimi giorni, lo capirete.

\chapter{31}

\par 1 In quel tempo, dice l'Eterno, io sarò l'Iddio di tutte le famiglie d'Israele, ed esse saranno il mio popolo.
\par 2 Così parla l'Eterno: Il popolo scampato dalla spada ha trovato grazia nel deserto; io sto per dar riposo a Israele.
\par 3 Da tempi lontani l'Eterno m'è apparso. 'Sì, io t'amo d'un amore eterno; perciò ti prolungo la mia bontà.
\par 4 Io ti riedificherò, e tu sarai riedificata, o vergine d'Israele! Tu sarai di nuovo adorna de' tuoi tamburelli, e uscirai in mezzo alle danze di quei che si rallegrano.
\par 5 Pianterai ancora delle vigne sui monti di Samaria; i piantatori pianteranno e raccoglieranno il frutto.
\par 6 Poiché il giorno verrà, quando le guardie grideranno sul monte d'Efraim: Levatevi, saliamo a Sion, all'Eterno, ch'è il nostro Dio'.
\par 7 Poiché così parla l'Eterno: Levate canti di gioia per Giacobbe, date in gridi, per il capo delle nazioni; fate udire delle laudi, e dite: 'O Eterno, salva il tuo popolo, il residuo d'Israele!'
\par 8 Ecco, io li riconduco dal paese del settentrione, e li raccolgo dalle estremità della terra; fra loro sono il cieco e lo zoppo, la donna incinta e quella in doglie di parto: una gran moltitudine, che ritorna qua.
\par 9 Vengono piangenti; li conduco supplichevoli; li meno ai torrenti d'acqua, per una via diritta dove non inciamperanno; perché son diventato un padre per Israele, ed Efraim è il mio primogenito.
\par 10 O nazioni, ascoltate la parola dell'Eterno, e proclamatela alle isole lontane, e dite: 'Colui che ha disperso Israele lo raccoglie, e lo custodisce come un pastore il suo gregge'.
\par 11 Poiché l'Eterno ha riscattato Giacobbe, l'ha redento dalla mano d'uno più forte di lui.
\par 12 E quelli verranno e canteranno di gioia sulle alture di Sion, e affluiranno verso i beni dell'Eterno: al frumento, al vino, all'olio, al frutto de' greggi e degli armenti; e l'anima loro sarà come un giardino annaffiato, e non continueranno più a languire.
\par 13 Allora la vergine si rallegrerà nella danza, i giovani gioiranno insieme ai vecchi; io muterò il loro lutto in gioia, li consolerò, li rallegrerò liberandoli del loro dolore.
\par 14 Satollerò di grasso l'anima de' sacerdoti, ed il mio popolo sarà saziato dei miei beni, dice l'Eterno.
\par 15 Così parla l'Eterno: S'è udita una voce in Rama, un lamento, un pianto amaro; Rachele piange i suoi figliuoli; ella rifiuta d'esser consolata de' suoi figliuoli, perché non sono più.
\par 16 Così parla l'Eterno: Trattieni la tua voce dal piangere, i tuoi occhi dal versar lagrime; poiché l'opera tua sarà ricompensata, dice l'Eterno: essi ritorneranno dal paese del nemico;
\par 17 e v'è speranza per il tuo avvenire, dice l'Eterno; i tuoi figliuoli ritorneranno nelle loro frontiere.
\par 18 Io odo, odo Efraim che si rammarica: 'Tu m'hai castigato, e io sono stato castigato, come un giovenco non domato; convertimi, e io mi convertirò, giacché tu sei l'Eterno, il mio Dio.
\par 19 Dopo che mi sono sviato, io mi son pentito; e dopo che ho riconosciuto il mio stato, mi son battuto l'anca; io son coperto di vergogna, confuso, perché porto l'obbrobrio della mia giovinezza'.
\par 20 - Efraim è egli dunque per me un figliuolo sì caro? un figliuolo prediletto? Dacché io parlo contro di lui, è più vivo e continuo il ricordo che ho di esso; perciò le mie viscere si commuovono per lui, ed io certo ne avrò pietà, dice l'Eterno.
\par 21 Rizza delle pietre miliari, fatti de' pali indicatori, poni ben mente alla strada, alla via che hai seguita. Ritorna, o vergine d'Israele, torna a queste città che son tue!
\par 22 Fino a quando n'andrai tu vagabonda, o figliuola infedele? Poiché l'Eterno crea una cosa nuova sulla terra: la donna che corteggia l'uomo.
\par 23 Così parla l'Eterno degli eserciti, l'Iddio d'Israele: Ancora si dirà questa parola nel paese di Giuda e nelle sue città, quando li avrò fatti tornare dalla cattività: 'L'Eterno ti benedica, o dimora di giustizia, o monte di santità!'
\par 24 Là si stabiliranno assieme Giuda e tutte le sue città: gli agricoltori e quei che menano i greggi.
\par 25 Poiché io ristorerò l'anima stanca, e sazierò ogni anima languente.
\par 26 A questo punto mi sono svegliato e ho guardato; e il mio sonno m'è stato dolce.
\par 27 Ecco, i giorni vengono, dice l'Eterno, ch'io seminerò la casa d'Israele e la casa di Giuda di semenza d'uomini e di semenza d'animali.
\par 28 E avverrà che, come ho vegliato su loro per svellere e per demolire, per rovesciare, per distruggere e per nuocere, così veglierò su loro per edificare e per piantare, dice l'Eterno.
\par 29 In quei giorni non si dirà più: 'I padri han mangiato l'agresto, e i denti de' figliuoli si sono allegati',
\par 30 ma ognuno morrà per la propria iniquità: chiunque mangerà l'agresto ne avrà i denti allegati.
\par 31 Ecco, i giorni vengono, dice l'Eterno, che io farò un nuovo patto con la casa d'Israele e con la casa di Giuda;
\par 32 non come il patto che fermai coi loro padri il giorno che li presi per mano per trarli fuori dal paese d'Egitto: patto ch'essi violarono, benché io fossi loro Signore, dice l'Eterno;
\par 33 ma questo è il patto che farò con la casa d'Israele, dopo quei giorni, dice l'Eterno: io metterò la mia legge nell'intimo loro, la scriverò sul loro cuore, e io sarò loro Dio, ed essi saranno mio popolo.
\par 34 E non insegneranno più ciascuno il suo compagno e ciascuno il suo fratello, dicendo: 'Conoscete l'Eterno!' poiché tutti mi conosceranno, dal più piccolo al più grande, dice l'Eterno. Poiché io perdonerò la loro iniquità, e non mi ricorderò più del loro peccato.
\par 35 Così parla l'Eterno, che ha dato il sole come luce del giorno, e le leggi alla luna e alle stelle perché sian luce alla notte; che solleva il mare sì che ne muggon le onde; colui che ha nome: l'Eterno degli eserciti.
\par 36 Se quelle leggi vengono a mancare dinanzi a me, dice l'Eterno, allora anche la progenie d'Israele cesserà d'essere in perpetuo una nazione nel mio cospetto.
\par 37 Così parla l'Eterno: Se i cieli di sopra possono esser misurati, e le fondamenta della terra di sotto, scandagliate, allora anch'io rigetterò tutta la progenie d'Israele per tutto quello ch'essi hanno fatto, dice l'Eterno.
\par 38 Ecco, i giorni vengono, dice l'Eterno, che questa città sarà riedificata in onore dell'Eterno, dalla torre di Hananeel alla porta dell'angolo.
\par 39 E di là la corda per misurare sarà tirata in linea retta fino al colle di Gareb, e girerà dal lato di Goah.
\par 40 E tutta la valle de' cadaveri e delle ceneri e tutti i campi fino al torrente di Kidron, fino all'angolo della porta de' cavalli verso oriente, saranno consacrati all'Eterno, e non saranno più sconvolti né distrutti in perpetuo.

\chapter{32}

\par 1 La parola che fu rivolta a Geremia dall'Eterno nel decimo anno di Sedekia, re di Giuda, che fu l'anno diciottesimo di Nebucadnetsar.
\par 2 L'esercito del re di Babilonia assediava allora Gerusalemme, e il profeta Geremia era rinchiuso nel cortile della prigione ch'era nella casa del re di Giuda.
\par 3 Ve l'aveva fatto rinchiudere Sedekia, re di Giuda, col dirgli: 'Perché vai tu profetizzando dicendo: - Così parla l'Eterno: Ecco, io do questa città in man del re di Babilonia, ed ei la prenderà;
\par 4 e Sedekia, re di Giuda, non scamperà dalle mani de' Caldei, ma sarà per certo dato in mano del re di Babilonia, e parlerà con lui bocca a bocca, e lo vedrà faccia a faccia;
\par 5 e Nebucadnetsar menerà Sedekia a Babilonia, ed egli resterà quivi finch'io lo visiti, dice l'Eterno; se combattete contro i Caldei voi non riuscirete a nulla'.
\par 6 E Geremia disse: 'La parola dell'Eterno m'è stata rivolta in questi termini:
\par 7 Ecco, Hanameel, figliuolo di Shallum, tuo zio, viene da te per dirti: Còmprati il mio campo ch'è ad Anatoth, poiché tu hai diritto di riscatto per comprarlo'.
\par 8 E Hanameel, figliuolo del mio zio, venne da me, secondo la parola dell'Eterno, nel cortile della prigione, e mi disse: 'Ti prego, compra il mio campo ch'è ad Anatoth, nel territorio di Beniamino; giacché tu hai il diritto di successione e il diritto di riscatto, còmpratelo!' Allora riconobbi che questa era parola dell'Eterno.
\par 9 E io comprai da Hanameel, figliuolo del mio zio, il campo ch'era ad Anatoth, gli pesai il danaro, diciassette sicli d'argento.
\par 10 Scrissi tutto questo in un atto, lo sigillai, chiamai i testimoni, e pesai il danaro nella bilancia.
\par 11 Poi presi l'atto di compra, quello sigillato contenente i termini e le condizioni, e quello aperto,
\par 12 e consegnai l'atto di compra a Baruc, figliuolo di Neria, figliuolo di Mahseia, in presenza di Hanameel mio cugino, in presenza dei testimoni che avevano sottoscritto l'atto di compra, e in presenza di tutti i Giudei che sedevano nel cortile della prigione.
\par 13 Poi, davanti a loro, diedi quest'ordine a Baruc:
\par 14 'Così parla l'Eterno degli eserciti, l'Iddio d'Israele: Prendi questi atti, l'atto di compra, tanto quello ch'è sigillato, quanto quello ch'è aperto, e mettili in un vaso di terra, perché si conservino lungo tempo.
\par 15 Poiché così parla l'Eterno degli eserciti, l'Iddio d'Israele: Si compreranno ancora delle case, de' campi e delle vigne, in questo paese'.
\par 16 E dopo ch'io ebbi consegnato l'atto di compra a Baruc, figliuolo di Neria, pregai l'Eterno, dicendo:
\par 17 'Ah, Signore, Eterno! Ecco, tu hai fatto il cielo e la terra con la tua gran potenza e col tuo braccio disteso: non v'è nulla di troppo difficile per te;
\par 18 tu usi benignità verso mille generazioni, e retribuisci l'iniquità dei padri in seno ai figliuoli, dopo di loro; tu sei l'Iddio grande, potente, il cui nome è l'Eterno degli eserciti;
\par 19 tu sei grande in consiglio e potente in opere; e hai gli occhi aperti su tutte le vie de' figliuoli degli uomini, per rendere a ciascuno secondo le sue opere e secondo il frutto delle sue azioni;
\par 20 tu hai fatto nel paese d'Egitto, in Israele e fra gli altri uomini, fino a questo giorno, miracoli e prodigi, e ti sei acquistato un nome qual è oggi;
\par 21 tu traesti il tuo popolo fuori dal paese d'Egitto con miracoli e prodigi, con mano potente e braccio steso, con gran terrore;
\par 22 e desti loro questo paese che avevi giurato ai loro padri di dar loro: un paese dove scorre il latte e il miele.
\par 23 Ed essi v'entrarono e ne presero possesso, ma non hanno ubbidito alla tua voce e non han camminato secondo la tua legge; tutto quello che avevi loro comandato di fare essi non l'hanno fatto; perciò tu hai fatto venir su di essi tutti questi mali.
\par 24 Ecco, le opere d'assedio giungono fino alla città per prenderla; e la città, vinta dalla spada, dalla fame e dalla peste, è data in man dei Caldei che combattono contro di lei. Quello che tu hai detto è avvenuto, ed ecco, tu lo vedi.
\par 25 Eppure, o Signore, o Eterno, tu m'hai detto: Còmprati con danaro il campo, e chiama de' testimoni... e la città è data in man de' Caldei'.
\par 26 Allora la parola dell'Eterno fu rivolta a Geremia in questi termini:
\par 27 'Ecco, io sono l'Eterno, l'Iddio d'ogni carne; v'ha egli qualcosa di troppo difficile per me?
\par 28 Perciò, così parla l'Eterno: Ecco, io do questa città in man dei Caldei, in mano di Nebucadnetsar, re di Babilonia, il quale la prenderà;
\par 29 e i Caldei che combattono contro questa città v'entreranno, v'appiccheranno il fuoco e la incendieranno, con le case sui tetti delle quali hanno offerto profumi a Baal e fatto libazioni ad altri dèi, per provocarmi ad ira.
\par 30 Poiché i figliuoli d'Israele e i figliuoli di Giuda, non hanno fatto altro, fin dalla loro fanciullezza, che quel ch'è male agli occhi miei; giacché i figliuoli d'Israele non hanno fatto che provocarmi ad ira con l'opera delle loro mani, dice l'Eterno.
\par 31 Poiché questa città, dal giorno che fu edificata fino ad oggi, è stata una continua provocazione alla mia ira e al mio furore, sicché la voglio toglier via dalla mia presenza,
\par 32 a motivo di tutto il male che i figliuoli d'Israele e i figliuoli di Giuda hanno fatto per provocarmi ad ira; essi, i loro re, i loro principi, i loro sacerdoti, i loro profeti, gli uomini di Giuda, e gli abitanti di Gerusalemme.
\par 33 E m'hanno voltato non la faccia, ma le spalle; e sebbene io li abbia ammaestrati del continuo fin dalla mattina, essi non han dato ascolto per ricevere la correzione.
\par 34 Ma hanno messo le loro abominazioni nella casa sulla quale è invocato il mio nome, per contaminarla.
\par 35 E hanno edificato gli alti luoghi di Baal che sono nella valle de' figliuoli d'Hinnom, per far passare per il fuoco i loro figliuoli e le loro figliuole offrendoli a Moloc; una cosa siffatta io non l'ho comandata loro; e non m'è venuto mai in mente che si dovesse commettere una tale abominazione, facendo peccare Giuda.
\par 36 Ma ora, in seguito a tutto questo, così parla l'Eterno, l'Iddio d'Israele, riguardo a questa città, della quale voi dite: Ella è data in mano del re di Babilonia, per la spada, per la fame e per la peste:
\par 37 Ecco, li raccoglierò da tutti i paesi dove li ho cacciati nella mia ira, nel mio furore, nella mia grande indignazione; e li farò tornare in questo luogo, e ve li farò dimorare al sicuro;
\par 38 ed essi saranno mio popolo, e io sarò loro Dio;
\par 39 e darò loro uno stesso cuore, una stessa via, perché mi temano in perpetuo per il loro bene e per quello dei loro figliuoli dopo di loro.
\par 40 E farò con loro un patto eterno, che non mi ritrarrò più da loro per cessare di far loro del bene; e metterò il mio timore nel loro cuore, perché non si dipartano da me.
\par 41 E metterò la mia gioia nel far loro del bene e li pianterò in questo paese con fedeltà, con tutto il mio cuore, con tutta l'anima mia.
\par 42 Poiché così parla l'Eterno: Come ho fatto venire su questo popolo tutto questo gran male, così farò venire su lui tutto il bene che gli prometto.
\par 43 Si compreranno de' campi in questo paese, del quale voi dite: È desolato; non v'è più né uomo né bestia; è dato in man de' Caldei.
\par 44 Si compreranno dei campi con danaro, se ne scriveranno gli atti, si sigilleranno, si chiameranno testimoni, nel paese di Beniamino e ne' luoghi intorno a Gerusalemme, nelle città di Giuda, nelle città della contrada montuosa, nelle città della pianura, nelle città del mezzogiorno; poiché io farò tornare quelli che sono in cattività, dice l'Eterno'.

\chapter{33}

\par 1 La parola dell'Eterno fu rivolta per la seconda volta a Geremia in questi termini, mentr'egli era ancora rinchiuso nel cortile della prigione:
\par 2 Così parla l'Eterno, che sta per far questo, l'Eterno che lo concepisce per mandarlo ad effetto, colui che ha nome l'Eterno:
\par 3 Invocami, e io ti risponderò, e t'annunzierò cose grandi e impenetrabili, che tu non conosci.
\par 4 Poiché così parla l'Eterno, l'Iddio d'Israele, riguardo alle case di questa città, e riguardo alle case dei re di Giuda che saran diroccate per far fronte ai terrapieni ed alla spada del nemico
\par 5 quando si verrà a combattere contro i Caldei, e a riempire quelle case di cadaveri d'uomini, che io percuoterò nella mia ira e nel mio furore, e per le cui malvagità io nasconderò la mia faccia a questa città:
\par 6 Ecco, io recherò ad essa medicazione e rimedi, e guarirò i suoi abitanti, e aprirò loro un tesoro di pace e di verità.
\par 7 E farò tornare dalla cattività Giuda e Israele, e li ristabilirò com'erano prima;
\par 8 e li purificherò di tutta l'iniquità colla quale hanno peccato contro di me; e perdonerò loro tutte le iniquità colle quali hanno peccato contro di me, e si sono ribellati a me.
\par 9 E questa città sarà per me un palese argomento di gioia, di lode e di gloria fra tutte le nazioni della terra, che udranno tutto il bene ch'io sto per far loro, e temeranno e tremeranno a motivo di tutto il bene e di tutta la pace ch'io procurerò a Gerusalemme.
\par 10 Così parla l'Eterno: In questo luogo, del quale voi dite: 'È un deserto, non v'è più uomo né bestia', nelle città di Giuda, e per le strade di Gerusalemme che son desolate e dove non è più né uomo, né abitante, né bestia,
\par 11 s'udranno ancora i gridi di gioia, i gridi d'esultanza, la voce dello sposo e la voce della sposa, la voce di quelli che dicono: 'Celebrate l'Eterno degli eserciti, poiché l'Eterno è buono, poiché la sua benignità dura in perpetuo', e che portano offerte di azioni di grazie nella casa dell'Eterno. Poiché io farò tornare i deportati del paese, e lo ristabilirò com'era prima, dice l'Eterno.
\par 12 Così parla l'Eterno degli eserciti: In questo luogo ch'è deserto, dove non v'è più né uomo né bestia, e in tutte le sue città vi saranno ancora delle dimore di pastori, che faranno riposare i loro greggi.
\par 13 Nelle città della contrada montuosa, nelle città della pianura, nelle città del mezzogiorno, nel paese di Beniamino, nei dintorni di Gerusalemme e nelle città di Giuda le pecore passeranno ancora sotto la mano di colui che le conta, dice l'Eterno.
\par 14 Ecco, i giorni vengono, dice l'Eterno, che io manderò ad effetto la buona parola che ho pronunziata riguardo alla casa d'Israele e riguardo alla casa di Giuda.
\par 15 In que' giorni e in quel tempo, io farò germogliare a Davide un germe di giustizia, ed esso farà ragione e giustizia nel paese.
\par 16 In que' giorni, Giuda sarà salvato, e Gerusalemme abiterà al sicuro, e questo è il nome onde sarà chiamato: 'l'Eterno, nostra giustizia'.
\par 17 Poiché così parla l'Eterno: Non verrà mai meno a Davide chi segga sul trono della casa d'Israele,
\par 18 e ai sacerdoti levitici non verrà mai meno nel mio cospetto chi offra olocausti, chi faccia fumare le offerte, e chi faccia tutti i giorni i sacrifizi.
\par 19 E la parola dell'Eterno fu rivolta a Geremia in questi termini:
\par 20 Così parla l'Eterno: Se voi potete annullare il mio patto col giorno e il mio patto con la notte, sì che il giorno e la notte non vengano al tempo loro,
\par 21 allora si potrà anche annullare il mio patto con Davide mio servitore, sì ch'egli non abbia più figliuolo che regni sul suo trono, e coi sacerdoti levitici miei ministri.
\par 22 Come non si può contare l'esercito del cielo né misurare la rena del mare, così io moltiplicherò la progenie di Davide, mio servitore, e i Leviti che fanno il mio servizio.
\par 23 La parola dell'Eterno fu rivolta a Geremia in questi termini:
\par 24 Non hai tu posto mente alle parole di questo popolo quando va dicendo: 'Le due famiglie che l'Eterno aveva scelte, le ha rigettate?' Così disprezzano il mio popolo, che agli occhi loro non è più una nazione.
\par 25 Così parla l'Eterno: Se io non ho stabilito il mio patto col giorno e con la notte, e se non ho fissato le leggi del cielo e della terra,
\par 26 allora rigetterò anche la progenie di Giacobbe e di Davide mio servitore, e non prenderò più dal suo legnaggio i reggitori della progenie d'Abrahamo, d'Isacco e di Giacobbe! poiché io farò tornare i loro esuli, e avrò pietà di loro.

\chapter{34}

\par 1 La parola che fu rivolta dall'Eterno in questi termini a Geremia, quando Nebucadnetsar, re di Babilonia, e tutto il suo esercito, e tutti i regni della terra sottoposti al suo dominio, e tutti i popoli combattevano contro Gerusalemme e contro tutte le sue città:
\par 2 Così parla l'Eterno, l'Iddio d'Israele: Va', parla a Sedekia, re di Giuda, e digli: Così parla l'Eterno: Ecco, io do questa città in mano del re di Babilonia, il quale la darà alle fiamme;
\par 3 e tu non scamperai dalla sua mano, ma sarai certamente preso, e sarai dato in sua mano; i tuoi occhi vedranno gli occhi del re di Babilonia; egli ti parlerà da bocca a bocca, e tu andrai a Babilonia.
\par 4 Nondimeno, o Sedekia, re di Giuda, ascolta la parola dell'Eterno: Così parla l'Eterno riguardo a te: Tu non morrai per la spada;
\par 5 tu morrai in pace; e come si arsero aromi per i tuoi padri, gli antichi re tuoi predecessori, così se ne arderanno per te; e si farà cordoglio per te, dicendo: 'Ahimè, signore!...' poiché son io quegli che pronunzia questa parola, dice l'Eterno.
\par 6 E il profeta Geremia disse tutte queste parole a Sedekia, re di Giuda, a Gerusalemme,
\par 7 mentre l'esercito del re di Babilonia combatteva contro Gerusalemme e contro tutte le città di Giuda che resistevano ancora, cioè contro Lachis e Azeka, ch'eran tutto quello che rimaneva, in fatto di città fortificate, fra le città di Giuda.
\par 8 La parola che fu rivolta dall'Eterno a Geremia, dopo che il re Sedekia ebbe fatto un patto con tutto il popolo di Gerusalemme di proclamare l'emancipazione,
\par 9 per la quale ognuno doveva rimandare in libertà il suo schiavo e la sua schiava, ebreo ed ebrea, e nessuno doveva tener più in schiavitù alcun suo fratello giudeo.
\par 10 E tutti i capi e tutto il popolo ch'erano entrati nel patto di rimandare in libertà ciascuno il proprio servo e la propria serva e di non tenerli più in ischiavitù ubbidirono, e li rimandarono;
\par 11 ma poi mutarono, e fecero ritornare gli schiavi e le schiave che avevano affrancati, e li riassoggettarono ad essere loro schiavi e schiave.
\par 12 La parola dell'Eterno fu dunque rivolta dall'Eterno a Geremia, in questi termini:
\par 13 Così parla l'Eterno, l'Iddio d'Israele: Io fermai un patto coi vostri padri il giorno che li trassi fuori dal paese d'Egitto, dalla casa di servitù, e dissi loro:
\par 14 'Al termine di sette anni, ciascuno di voi rimandi libero il suo fratello ebreo, che si sarà venduto a lui; ti serva sei anni, poi rimandalo da casa tua libero'; ma i vostri padri non ubbidirono e non prestarono orecchio.
\par 15 E voi eravate oggi tornati a fare ciò ch'è retto agli occhi miei, proclamando l'emancipazione ciascuno al suo prossimo, e avevate fermato un patto nel mio cospetto, nella casa sulla quale è invocato il mio nome;
\par 16 ma siete tornati indietro, e avete profanato il mio nome; ciascun di voi ha fatto ritornare il suo schiavo e la sua schiava che avevate rimandati in libertà a loro piacere, e li avete assoggettati ad essere vostri schiavi e schiave.
\par 17 Perciò, così parla l'Eterno: Voi non mi avete ubbidito proclamando l'emancipazione ciascuno al suo fratello e ciascuno al suo prossimo; ecco: io proclamo la vostra emancipazione, dice l'Eterno, per andare incontro alla spada, alla peste e alla fame, e farò che sarete agitati per tutti i regni della terra.
\par 18 E darò gli uomini che hanno trasgredito il mio patto e non hanno messo ad effetto le parole del patto che aveano fermato nel mio cospetto, passando in mezzo alle parti del vitello che aveano tagliato in due;
\par 19 darò, dico, i capi di Giuda e i capi di Gerusalemme, gli eunuchi, i sacerdoti e tutto il popolo del paese che passarono in mezzo alle parti del vitello,
\par 20 in mano dei loro nemici, e in mano di quelli che cercano la loro vita; e i loro cadaveri serviranno di pasto agli uccelli del cielo e alle bestie della terra.
\par 21 E darò Sedekia, re di Giuda, e i suoi capi in mano dei loro nemici, e in mano di quelli che cercano la loro vita, e in mano dell'esercito del re di Babilonia, che s'è allontanato da voi.
\par 22 Ecco, io darò l'ordine, dice l'Eterno, e li farò ritornare contro questa città; essi combatteranno contro di lei, la prenderanno, la daranno alle fiamme; e io farò delle città di Giuda una desolazione senz'abitanti.

\chapter{35}

\par 1 La parola che fu rivolta a Geremia dall'Eterno, al tempo di Joiakim, figliuolo di Giosia, re di Giuda, in questi termini:
\par 2 'Va' alla casa dei Recabiti, e parla loro: ménali nella casa dell'Eterno, in una delle camere, e offri loro del vino da bere'.
\par 3 Allora io presi Jaazania, figliuolo di Geremia, figliuolo di Habazzinia, i suoi fratelli, tutti i suoi figliuoli e tutta la casa dei Recabiti,
\par 4 e li menai nella casa dell'Eterno, nella camera dei figliuoli di Hanan figliuolo d'Igdalia, uomo di Dio, la quale era presso alla camera de' capi, sopra la camera di Maaseia, figliuolo di Shallum, guardiano della soglia;
\par 5 e misi davanti ai figliuoli della casa dei Recabiti dei vasi pieni di vino e delle coppe, e dissi loro: 'Bevete del vino'.
\par 6 Ma quelli risposero: 'Noi non beviamo vino; perché Gionadab, figliuolo di Recab, nostro padre, ce l'ha proibito, dicendo: - Non berrete mai in perpetuo vino, né voi né i vostri figliuoli;
\par 7 e non edificherete case, non seminerete alcuna semenza, non pianterete vigne, e non ne possederete alcuna, ma abiterete in tende tutti i giorni della vostra vita, affinché viviate lungamente nel paese dove state come forestieri.
\par 8 E noi abbiamo ubbidito alla voce di Gionadab, figliuolo di Recab, nostro padre, in tutto quello che ci ha comandato: non beviamo vino durante tutti i nostri giorni, tanto noi, che le nostre mogli, i nostri figliuoli e le nostre figliuole;
\par 9 non edifichiamo case per abitarvi, non abbiamo vigna, campo, né sementa;
\par 10 abitiamo in tende, e abbiamo ubbidito e fatto tutto quello che Gionadab, nostro padre, ci ha comandato.
\par 11 Ma quando Nebucadnetsar, re di Babilonia, è salito contro il paese, abbiam detto: - Venite, ritiriamoci a Gerusalemme, per paura dell'esercito dei Caldei e dell'esercito di Siria. E così ci siamo stabiliti a Gerusalemme'.
\par 12 Allora la parola dell'Eterno fu rivolta a Geremia in questi termini:
\par 13 'Così parla l'Eterno degli eserciti, l'Iddio d'Israele: Va' e di' agli uomini di Giuda e agli abitanti di Gerusalemme: - Non riceverete voi dunque la lezione, imparando ad ubbidire alle mie parole? dice l'Eterno.
\par 14 Le parole di Gionadab, figliuolo di Recab, che comandò ai suoi figliuoli di non bever vino, sono state messe ad effetto; ed essi fino al dì d'oggi non hanno bevuto vino, in ubbidienza all'ordine del padre loro; e io v'ho parlato, parlato fin dal mattino, e voi non m'avete dato ascolto;
\par 15 ho continuato a mandarvi ogni mattina tutti i miei servitori i profeti per dirvi: - Convertitevi dunque ciascuno dalla sua via malvagia, emendate le vostre azioni, non andate dietro ad altri dèi per servirli, e abiterete nel paese che ho dato a voi ed ai vostri padri; ma voi non avete prestato orecchio, e non m'avete ubbidito.
\par 16 Sì, i figliuoli di Gionadab, figliuolo di Recab, hanno messo ad effetto l'ordine dato dal padre loro, ma questo popolo non mi ha ubbidito!
\par 17 Perciò, così parla l'Eterno, l'Iddio degli eserciti, l'Iddio d'Israele: Ecco, io faccio venire su Giuda e su tutti gli abitanti di Gerusalemme tutto il male che ho pronunziato contro di loro, perché ho parlato loro, ed essi non hanno ascoltato; perché li ho chiamati, ed essi non hanno risposto'.
\par 18 E alla casa dei Recabiti Geremia disse: 'Così parla l'Eterno degli eserciti, l'Iddio d'Israele: Poiché avete ubbidito all'ordine di Gionadab, vostro padre, e avete osservato tutti i suoi precetti, e avete fatto tutto quello ch'egli vi avea prescritto,
\par 19 così parla l'Eterno degli eserciti, l'Iddio d'Israele: A Gionadab, figliuolo di Recab, non verranno mai meno in perpetuo discendenti, che stiano davanti alla mia faccia'.

\chapter{36}

\par 1 Or avvenne, l'anno quarto di Joiakim, figliuolo di Giosia, re di Giuda, che questa parola fu rivolta dall'Eterno a Geremia, in questi termini:
\par 2 'Prenditi un rotolo da scrivere e scrivici tutte le parole che t'ho dette contro Israele, contro Giuda e contro tutte le nazioni, dal giorno che cominciai a parlarti, cioè dal tempo di Giosia, fino a quest'oggi.
\par 3 Forse quei della casa di Giuda, udendo tutto il male ch'io penso di far loro, si convertiranno ciascuno dalla sua via malvagia, e io perdonerò la loro iniquità e il loro peccato'.
\par 4 Allora Geremia chiamò Baruc, figliuolo di Neria; e Baruc scrisse in un rotolo da scrivere, a dettatura di Geremia, tutte le parole che l'Eterno avea dette a Geremia.
\par 5 Poi Geremia diede quest'ordine a Baruc: 'Io sono impedito, e non posso entrare nella casa dell'Eterno;
\par 6 perciò, va' tu, e leggi dal libro che hai scritto a mia dettatura, le parole dell'Eterno, in presenza del popolo, nella casa dell'Eterno, il giorno del digiuno; e leggile anche in presenza di tutti quei di Giuda, che saran venuti dalle loro città.
\par 7 Forse, presenteranno le loro supplicazioni all'Eterno, e si convertiranno ciascuno dalla sua via malvagia; perché l'ira e il furore che l'Eterno ha espresso contro questo popolo, sono grandi'.
\par 8 E Baruc, figliuolo di Neria, fece tutto quello che gli aveva ordinato il profeta Geremia, e lesse dal libro le parole dell'Eterno.
\par 9 Or l'anno quinto di Joiakim, figliuolo di Giosia, re di Giuda, il nono mese, fu pubblicato un digiuno nel cospetto dell'Eterno, per tutto il popolo di Gerusalemme e per tutto il popolo venuto dalle città di Giuda a Gerusalemme.
\par 10 E Baruc lesse dal libro le parole di Geremia in presenza di tutto il popolo, nella casa dell'Eterno, nella camera di Ghemaria, figliuolo di Shafan, segretario, nel cortile superiore, all'ingresso della porta nuova della casa dell'Eterno.
\par 11 Or Micaia, figliuolo di Ghemaria, figliuolo di Shafan, udì tutte le parole dell'Eterno, lette dal libro;
\par 12 scese nella casa del re, nella camera del segretario, ed ecco che quivi stavan seduti tutti i capi: Elishama il segretario, Delaia figliuolo di Scemaia, Elnathan figliuolo di Acbor, Ghemaria figliuolo di Shafan, Sedekia figliuolo di Hanania, e tutti gli altri capi.
\par 13 E Micaia riferì loro tutte le parole che aveva udite mentre Baruc leggeva il libro in presenza del popolo.
\par 14 Allora tutti i capi mandarono Jehudi, figliuolo di Nethania, figliuolo di Scelemia, figliuolo di Cusci, a Baruc per dirgli: 'Prendi in mano il rotolo dal quale tu hai letto in presenza del popolo, e vieni'. E Baruc, figliuolo di Neria, prese in mano il rotolo, e venne a loro.
\par 15 Ed essi gli dissero: 'Siediti, e leggilo qui a noi'. E Baruc lo lesse in loro presenza.
\par 16 E quand'essi ebbero udito tutte quelle parole, si volsero spaventati gli uni agli altri, e dissero a Baruc: 'Non mancheremo di riferire tutte queste parole al re'.
\par 17 Poi chiesero a Baruc: 'Dicci ora come hai scritto tutte queste parole uscite dalla sua bocca'.
\par 18 E Baruc rispose loro: 'Egli m'ha dettato di bocca sua tutte queste parole, e io le ho scritte con inchiostro nel libro'.
\par 19 Allora i capi dissero a Baruc: 'Vatti a nascondere tanto tu quanto Geremia; e nessuno sappia dove siete'.
\par 20 Poi andarono dal re, nel cortile, riposero il rotolo nella camera di Elishama, segretario, e riferirono al re tutte quelle parole.
\par 21 E il re mandò Jehudi a prendere il rotolo; ed egli lo prese dalla camera di Elishama, segretario. E Jehudi lo lesse in presenza del re, e in presenza di tutti i capi che stavano in piè allato al re.
\par 22 Or il re stava seduto nel suo palazzo d'inverno - era il nono mese -, e il braciere ardeva davanti a lui.
\par 23 E quando Jehudi ebbe letto tre o quattro colonne, il re tagliò il libro col temperino e lo gettò nel fuoco del braciere, dove il rotolo fu interamente consumato dal fuoco del braciere.
\par 24 Né il re né alcuno dei suoi servitori che udirono tutte quelle parole, rimasero spaventati o si stracciarono le vesti.
\par 25 E benché Elnathan, Delaia e Ghemaria supplicassero il re perché non bruciasse il rotolo, egli non volle dar loro ascolto.
\par 26 E il re ordinò a Jerahmeel, figliuolo del re, a Sesaia figliuolo di Azriel, e a Scelemia figliuolo di Abdeel, di pigliare Baruc, segretario, e il profeta Geremia. Ma l'Eterno li nascose.
\par 27 E dopo che il re ebbe bruciato il rotolo e le parole che Baruc aveva scritte a dettatura di Geremia, la parola dell'Eterno fu rivolta a Geremia in questi termini:
\par 28 'Prenditi di nuovo un altro rotolo, e scrivici tutte le parole di prima ch'erano nel primo rotolo, che Joiakim re di Giuda ha bruciato.
\par 29 E riguardo a Joiakim, re di Giuda, tu dirai: Così parla l'Eterno: Tu hai bruciato quel rotolo, dicendo: - Perché hai scritto in esso che il re di Babilonia verrà certamente e distruggerà questo paese e farà sì che non vi sarà più né uomo né bestia? -
\par 30 Perciò così parla l'Eterno riguardo a Joiakim re di Giuda: Egli non avrà alcuno che segga sul trono di Davide, e il suo cadavere sarà gettato fuori, esposto al caldo del giorno e al gelo della notte.
\par 31 E io punirò lui, la sua progenie e i suoi servitori della loro iniquità, e farò venire su loro, sugli abitanti di Gerusalemme e sugli uomini di Giuda tutto il male che ho pronunziato contro di loro senza ch'essi abbian dato ascolto'.
\par 32 E Geremia prese un altro rotolo e lo diede a Baruc, figliuolo di Neria, segretario, il quale vi scrisse, a dettatura di Geremia, tutte le parole del libro che Joiakim, re di Giuda, avea bruciato nel fuoco; e vi furono aggiunte molte altre parole simili a quelle.

\chapter{37}

\par 1 Or il re Sedekia, figliuolo di Giosia, regnò in luogo di Conia, figliuolo di Joiakim, e fu costituito re nel paese di Giuda da Nebucadnetsar, re di Babilonia.
\par 2 Ma né egli, né i suoi servitori, né il popolo del paese dettero ascolto alle parole che l'Eterno avea pronunziate per mezzo del profeta Geremia.
\par 3 Il re Sedekia mandò Jehucal, figliuolo di Scelemia, e Sofonia, figliuolo di Maaseia, il sacerdote, dal profeta Geremia, per dirgli: 'Deh, prega per noi l'Eterno, l'Iddio nostro'.
\par 4 Or Geremia andava e veniva fra il popolo, e non era ancora stato messo in prigione.
\par 5 L'esercito di Faraone era uscito d'Egitto; e come i Caldei che assediavano Gerusalemme n'ebbero ricevuto la notizia, tolsero l'assedio a Gerusalemme.
\par 6 Allora la parola dell'Eterno fu rivolta al profeta Geremia, in questi termini:
\par 7 'Così parla l'Eterno, l'Iddio d'Israele: Dite così al re di Giuda che vi ha mandati da me per consultarmi: Ecco, l'esercito di Faraone ch'era uscito in vostro soccorso, è tornato nel suo paese, in Egitto;
\par 8 e i Caldei torneranno, e combatteranno contro questa città, la prenderanno, e la daranno alle fiamme.
\par 9 Così parla l'Eterno: Non ingannate voi stessi dicendo: - Certo, i Caldei se n'andranno da noi, - perché non se n'andranno.
\par 10 Anzi, quand'anche voi sconfiggeste tutto l'esercito de' Caldei che combatte contro di voi, e non ne rimanesse che degli uomini feriti, questi si leverebbero, ciascuno nella sua tenda, e darebbero questa città alle fiamme'.
\par 11 Or quando l'esercito de' Caldei si fu ritirato d'innanzi a Gerusalemme a motivo dell'esercito di Faraone,
\par 12 Geremia uscì da Gerusalemme per andare nel paese di Beniamino, per ricever quivi la sua porzione in mezzo al popolo.
\par 13 Ma quando fu alla porta di Beniamino, c'era quivi un capitano della guardia, per nome Ireia, figliuolo di Scelemia, figliuolo di Hanania, il quale arrestò il profeta Geremia dicendo: 'Tu vai ad arrenderti ai Caldei'.
\par 14 E Geremia rispose: 'È falso; io non vado ad arrendermi ai Caldei'; ma l'altro non gli diede ascolto; arrestò Geremia, e lo menò dai capi.
\par 15 E i capi s'adirarono contro Geremia, lo percossero, e lo misero in prigione nella casa di Gionathan, il segretario; perché di quella avean fatto un carcere.
\par 16 Quando Geremia fu entrato nella prigione sotterranea fra le segrete, e vi fu rimasto molti giorni,
\par 17 il re Sedekia lo mandò a prendere, lo interrogò in casa sua, di nascosto, e gli disse: 'C'è egli qualche parola da parte dell'Eterno?' E Geremia rispose: 'Sì, c'è'. E aggiunse: 'Tu sarai dato in mano del re di Babilonia'.
\par 18 E Geremia disse inoltre al re Sedekia: 'Che peccato ho io commesso contro di te o contro i tuoi servitori o contro questo popolo, che m'avete messo in prigione?
\par 19 E dove sono ora i vostri profeti che vi profetavano dicendo: - Il re di Babilonia non verrà contro di voi né contro questo paese? -
\par 20 Ora ascolta, ti prego, o re, mio signore; e la mia supplicazione giunga bene accolta nel tuo cospetto; non mi far tornare nella casa di Gionathan lo scriba, sì ch'io vi muoia'.
\par 21 Allora il re Sedekia ordinò che Geremia fosse custodito nel cortile della prigione, e gli fosse dato tutti i giorni un pane dalla via de' fornai, finché tutto il pane della città fosse consumato. Così Geremia rimase nel cortile della prigione.

\chapter{38}

\par 1 Scefatia figliuolo di Mattan, Ghedalia figliuolo di Pashur, Jucal figliuolo di Scelamia e Pashur figliuolo di Malkia, udirono le parole che Geremia rivolgeva a tutto il popolo, dicendo:
\par 2 'Così parla l'Eterno: Chi rimarrà in questa città morrà di spada, di fame o di peste; ma chi andrà ad arrendersi ai Caldei avrà salva la vita, la vita sarà il suo bottino, e vivrà.
\par 3 Così parla l'Eterno: Questa città sarà certamente data in mano dell'esercito del re di Babilonia, che la prenderà'.
\par 4 E i capi dissero al re: 'Deh, sia quest'uomo messo a morte! poich'egli rende fiacche le mani degli uomini di guerra che rimangono in questa città, e le mani di tutto il popolo, tenendo loro cotali discorsi; quest'uomo non cerca il bene, ma il male di questo popolo'.
\par 5 Allora il re Sedekia disse: 'Ecco, egli è in mano vostra; poiché il re non può nulla contro di voi'.
\par 6 Allora essi presero Geremia e lo gettarono nella cisterna di Malkia, figliuolo del re, ch'era nel cortile della prigione; vi calarono Geremia con delle funi. Nella cisterna non c'era acqua ma solo fango, e Geremia affondò nel fango.
\par 7 Or Ebed-melec, etiopo, eunuco che stava nella casa del re, udì che aveano messo Geremia nella cisterna. - Il re stava allora seduto alla porta di Beniamino. -
\par 8 Ebed-melec uscì dalla casa del re, e parlò al re dicendo:
\par 9 'O re, mio signore, quegli uomini hanno male agito in tutto quello che hanno fatto al profeta Geremia, che hanno gettato nella cisterna; egli morrà di fame là dov'è, giacché non v'è più pane in città'.
\par 10 E il re diede quest'ordine ad Ebed-melec, l'etiopo: 'Prendi teco di qui trenta uomini, e tira su il profeta Geremia dalla cisterna prima che muoia'.
\par 11 Ebed-melec prese seco quegli uomini, entrò nella casa del re, sotto il Tesoro; prese di lì dei pezzi di stoffa logora e de' vecchi stracci, e li calò a Geremia, nella cisterna, con delle funi.
\par 12 Ed Ebed-melec, l'etiopo, disse a Geremia: 'Mettiti ora questi pezzi di stoffa logora e questi stracci sotto le ascelle, sotto le funi'. E Geremia fece così.
\par 13 E quelli trassero su Geremia con quelle funi, e lo fecero salir fuori dalla cisterna. E Geremia rimase nel cortile della prigione.
\par 14 Allora il re Sedekia mandò a prendere il profeta Geremia, e se lo fece condurre al terzo ingresso della casa dell'Eterno; e il re disse a Geremia: 'Io ti domando una cosa; non mi celar nulla'.
\par 15 E Geremia rispose a Sedekia: 'Se te la dico, non è egli certo che mi farai morire? E se ti do qualche consiglio, non mi darai ascolto'.
\par 16 E il re Sedekia giurò in segreto a Geremia, dicendo: 'Com'è vero che l'Eterno, il quale ci ha dato questa vita, vive, io non ti farò morire, e non ti darò in mano di questi uomini che cercan la tua vita'.
\par 17 Allora Geremia disse a Sedekia: 'Così parla l'Eterno, l'Iddio degli eserciti, l'Iddio d'Israele: Se tu ti vai ad arrendere ai capi del re di Babilonia, avrai salva la vita; questa città non sarà data alle fiamme, e vivrai tu con la tua casa;
\par 18 ma se non vai ad arrenderti ai capi del re di Babilonia, questa città sarà data in mano de' Caldei che la daranno alle fiamme, e tu non scamperai dalle loro mani'.
\par 19 E il re Sedekia disse a Geremia: 'Io temo que' Giudei che si sono arresi ai Caldei, ch'io non abbia ad esser dato nelle loro mani, e ch'essi non mi scherniscano'.
\par 20 Ma Geremia rispose: 'Tu non sarai dato nelle loro mani. Deh! ascolta la voce dell'Eterno in questo che ti dico: tutto andrà bene per te, e tu vivrai.
\par 21 Ma se rifiuti d'uscire, ecco quello che l'Eterno m'ha fatto vedere:
\par 22 Tutte le donne rimaste nella casa del re di Giuda saranno menate fuori ai capi del re di Babilonia; e queste donne diranno: - 'I tuoi familiari amici t'hanno incitato, t'hanno vinto; i tuoi piedi sono affondati nel fango, e quelli si son ritirati'. -
\par 23 E tutte le tue mogli coi tuoi figliuoli saranno menate ai Caldei; e tu non scamperai dalle loro mani, ma sarai preso e dato in mano del re di Babilonia, e questa città sarà data alle fiamme'.
\par 24 E Sedekia disse a Geremia: 'Nessuno sappia nulla di queste parole, e tu non morrai.
\par 25 E se i capi odono che io ho parlato teco e vengono da te a dirti: - Dichiaraci quello che tu hai detto al re; non ce lo celare, e non ti faremo morire; e il re che t'ha detto?... -
\par 26 rispondi loro: Io ho presentato al re la mia supplicazione, ch'egli non mi facesse ritornare nella casa di Gionàthan, per morirvi'.
\par 27 E tutti i capi vennero a Geremia, e lo interrogarono; ma egli rispose loro secondo tutte le parole che il re gli aveva comandate, e quelli lo lasciarono in pace, perché la cosa non s'era divulgata.
\par 28 E Geremia rimase nel cortile della prigione fino al giorno che Gerusalemme fu presa.

\chapter{39}

\par 1 Quando Gerusalemme fu presa - il nono anno di Sedekia, re di Giuda, il decimo mese, Nebucadnetsar, re di Babilonia, venne con tutto il suo esercito contro Gerusalemme e la cinse d'assedio;
\par 2 l'undecimo anno di Sedekia, il quarto mese, il nono giorno, una breccia fu fatta nella città
\par 3 - tutti i capi del re di Babilonia entrarono, e si stabilirono alla porta di mezzo: Nergalsaretser, Samgarnebu, Sarsekim, capo degli eunuchi, Nergalsaretser, capo dei magi, e tutti gli altri capi del re di Babilonia.
\par 4 E quando Sedekia, re di Giuda, e tutta la gente di guerra li ebbero veduti, fuggirono, uscirono di notte dalla città per la via del giardino reale, per la porta fra le due mura, e presero la via della pianura.
\par 5 Ma l'esercito de' Caldei li inseguì, e raggiunse Sedekia nelle campagne di Gerico. Lo presero, lo menaron su da Nebucadnetsar, re di Babilonia, a Ribla, nel paese di Hamath, dove il re pronunziò la sua sentenza su di lui.
\par 6 E il re di Babilonia fece scannare i figliuoli di Sedekia, a Ribla, sotto gli occhi di lui; il re di Babilonia fece pure scannare tutti i notabili di Giuda;
\par 7 poi fece cavar gli occhi a Sedekia, e lo fe' legare con una doppia catena di rame per menarlo in Babilonia.
\par 8 I Caldei incendiarono la casa del re e le case del popolo, e abbatterono le mura di Gerusalemme;
\par 9 e Nebuzaradan, capo delle guardie, menò in cattività a Babilonia il residuo della gente ch'era ancora nella città, quelli ch'erano andati ad arrendersi a lui, e il resto del popolo.
\par 10 Ma Nebuzaradan, capo delle guardie, lasciò nel paese di Giuda alcuni de' più poveri fra il popolo i quali non avevano nulla, e diede loro in quel giorno vigne e campi.
\par 11 Or Nebucadnetsar, re di Babilonia, avea dato a Nebuzaradan, capo delle guardie, quest'ordine riguardo a Geremia:
\par 12 'Prendilo, veglia su lui, e non gli fare alcun male, ma comportati verso di lui com'egli ti dirà'.
\par 13 Così Nebuzaradan, capo delle guardie, Nebushazban, capo degli eunuchi, Nergal-saretser, capo de' magi, e tutti i capi del re di Babilonia
\par 14 mandarono a far trarre Geremia fuori dal cortile della prigione, e lo consegnarono a Ghedalia, figliuolo di Ahikam, figliuolo di Shafan, perché fosse menato a casa; e così egli abitò fra il popolo.
\par 15 Or la parola dell'Eterno fu rivolta a Geremia in questi termini, mentr'egli era rinchiuso nel cortile della prigione:
\par 16 'Va' e parla ad Ebed-melec l'etiopo e digli: Così parla l'Eterno degli eserciti, l'Iddio d'Israele: Ecco, io sto per adempiere su questa città, per il suo male e non per il suo bene, le parole che ho pronunziate, ed in quel giorno esse si avvereranno in tua presenza.
\par 17 Ma in quel giorno io ti libererò, dice l'Eterno; e tu non sarai dato in mano degli uomini che temi;
\par 18 poiché, certo, io ti farò scampare, e tu non cadrai per la spada; la tua vita sarà il tuo bottino, giacché hai posto la tua fiducia in me, dice l'Eterno'.

\chapter{40}

\par 1 La parola che fu rivolta dall'Eterno a Geremia, dopo che Nebuzaradan, capo delle guardie, l'ebbe rimandato da Rama. Quando questi lo fece prendere, Geremia era incatenato in mezzo a tutti quelli di Gerusalemme e di Giuda, che dovevano esser menati in cattività a Babilonia.
\par 2 Il capo delle guardie prese dunque Geremia, e gli disse: 'L'Eterno, il tuo Dio, aveva pronunziato questo male contro questo luogo;
\par 3 e l'Eterno l'ha fatto venire e ha fatto come aveva detto, perché voi avete peccato contro l'Eterno, e non avete dato ascolto alla sua voce; perciò questo v'è avvenuto.
\par 4 Ora ecco, io ti sciolgo oggi dalle catene che hai alle mani; se ti piace di venire con me a Babilonia, vieni; e io avrò cura di te; ma se non t'aggrada di venir con me a Babilonia, rimantene; ecco, tutto il paese ti sta dinanzi; va' dove ti piacerà e ti converrà d'andare'.
\par 5 E come Geremia non si decideva a tornare con lui, l'altro aggiunse: 'Torna da Ghedalia, figliuolo di Ahikam, figliuolo di Shafan, che il re di Babilonia ha stabilito sulle città di Giuda, e dimora con lui in mezzo al popolo; ovvero va' dovunque ti piacerà'. E il capo delle guardie gli diede delle provviste e un regalo, e l'accomiatò.
\par 6 E Geremia andò da Ghedalia, figliuolo di Ahikam, a Mitspa, e dimorò con lui in mezzo al popolo che era rimasto nel paese.
\par 7 Or quando tutti i capi delle forze che erano per le campagne ebbero inteso, essi e la loro gente, che il re di Babilonia aveva stabilito Ghedalia, figliuolo di Ahikam, sul paese, e che gli aveva affidato gli uomini, le donne, i bambini, e quelli tra i poveri del paese che non erano stati menati in cattività a Babilonia,
\par 8 si recarono da Ghedalia a Mitspa: erano Ismael, figliuolo di Nethania, Johanan e Gionathan, figliuoli di Kareah, Seraia, figliuolo di Tanhumeth, i figliuoli di Efai di Netofa, e Jezania, figliuolo del Maacatita: essi e i loro uomini.
\par 9 E Ghedalia, figliuolo di Ahikam, figliuolo di Shafan, giurò loro e alla lor gente, dicendo: 'Non temete di servire i Caldei; abitate nel paese, servite il re di Babilonia, e tutto andrà bene per voi.
\par 10 Quanto a me, ecco, io risiederò a Mitspa per tenermi agli ordini dei Caldei, che verranno da noi; e voi raccogliete il vino, le frutta d'estate e l'olio; metteteli nei vostri vasi, e dimorate nelle città di cui avete preso possesso'.
\par 11 Anche tutti i Giudei ch'erano in Moab, fra gli Ammoniti, nel paese d'Edom e in tutti i paesi, quand'udirono che il re di Babilonia aveva lasciato un residuo in Giuda e che avea stabilito su di loro Ghedalia, figliuolo di Ahikam, figliuolo di Shafan,
\par 12 se ne tornarono da tutti i luoghi dov'erano stati dispersi, e si recarono nel paese di Giuda da Ghedalia, a Mitspa; e raccolsero vino e frutta d'estate in grande abbondanza.
\par 13 Or Johanan, figliuolo di Kareah, e tutti i capi delle forze che erano per la campagna, vennero da Ghedalia a Mitspa, e gli dissero:
\par 14 'Sai tu che Baalis, re degli Ammoniti, ha mandato Ismael, figliuolo di Nethania, per toglierti la vita?' Ma Ghedalia, figliuolo di Ahikam, non credette loro.
\par 15 Allora Johanan, figliuolo di Kareah, disse segretamente a Ghedalia, a Mitspa: 'Lasciami andare a uccidere Ismael, figliuolo di Nethania; nessuno lo risaprà; e perché ti toglierebbe egli la vita, e tutti i Giudei che si son raccolti presso di te andrebbero essi dispersi, e il residuo di Giuda perirebb'egli?'
\par 16 Ma Ghedalia, figliuolo di Ahikam, disse a Johanan, figliuolo di Kareah: 'Non lo fare, perché quello che tu dici d'Ismael è falso'.

\chapter{41}

\par 1 E il settimo mese, Ismael, figliuolo di Nethania, figliuolo di Elishama, della stirpe reale e uno dei grandi del re, venne con dieci uomini, da Ghedalia, figliuolo di Ahikam, a Mitspa; e quivi, a Mitspa, mangiarono assieme.
\par 2 Poi Ismael, figliuolo di Nethania, si levò coi dieci uomini ch'eran con lui, e colpirono con la spada Ghedalia, figliuolo di Ahikam, figliuolo di Shafan. Così fecero morire colui che il re di Babilonia aveva stabilito sul paese.
\par 3 Ismael uccise pure tutti i Giudei ch'erano con Ghedalia a Mitspa, e i Caldei, uomini di guerra, che si trovavan quivi.
\par 4 Il giorno dopo ch'egli ebbe ucciso Ghedalia, prima che alcuno ne sapesse nulla,
\par 5 giunsero da Sichem, da Sciloh e da Samaria, ottanta uomini che avevano la barba rasa, le vesti stracciate e delle incisioni sul corpo; e avevano in mano delle offerte e dell'incenso per presentarli nella casa dell'Eterno.
\par 6 E Ismael, figliuolo di Nethania, uscì loro incontro da Mitspa; e, camminando, piangeva; e come li ebbe incontrati, disse loro: 'Venite da Ghedalia, figliuolo di Ahikam'.
\par 7 E quando furono entrati in mezzo alla città, Ismael, figliuolo di Nethania, assieme agli uomini che aveva seco, li scannò e li gettò nella cisterna.
\par 8 Or fra quelli, ci furon dieci uomini, che dissero a Ismael: 'Non ci uccidere, perché abbiamo nei campi delle provviste nascoste di grano, d'orzo, d'olio e di miele'. Allora egli si trattenne, e non li mise a morte coi loro fratelli.
\par 9 Or la cisterna nella quale Ismael gettò tutti i cadaveri degli uomini ch'egli uccise con Ghedalia, è quella che il re Asa aveva fatta fare per tema di Baasa, re d'Israele; e Ismael, figliuolo di Nethania, la riempì di uccisi.
\par 10 Poi Ismael menò via prigionieri tutto il rimanente del popolo che si trovava a Mitspa: le figliuole del re, e tutto il suo popolo ch'era rimasto a Mitspa, e sul quale Nebuzaradan, capo delle guardie, aveva stabilito Ghedalia, figliuolo di Ahikam; Ismael, figliuolo di Nethania, li menò via prigionieri, e partì per recarsi dagli Ammoniti.
\par 11 Ma quando Johanan, figliuolo di Kareah, e tutti i capi delle forze ch'eran con lui furono informati di tutto il male che Ismael, figliuolo di Nethania, aveva fatto,
\par 12 presero tutti gli uomini, e andarono a combattere contro Ismael, figliuolo di Nethania; e lo trovarono presso le grandi acque che sono a Gabaon.
\par 13 E quando tutto il popolo ch'era con Ismael vide Johanan, figliuolo di Kareah, e tutti i capi delle forze ch'erano con lui, si rallegrò;
\par 14 e tutto il popolo che Ismael aveva menato prigioniero da Mitspa fece voltafaccia, e andò a unirsi a Johanan, figliuolo di Kareah.
\par 15 Ma Ismael, figliuolo di Nethania, scampò con otto uomini d'innanzi a Johanan, e se ne andò fra gli Ammoniti.
\par 16 E Johanan, figliuolo di Kareah, e tutti i capi delle forze ch'erano con lui, presero tutto il rimanente del popolo, che Ismael, figliuolo di Nethania, aveva menati via da Mitspa, dopo ch'egli ebbe ucciso Ghedalia, figliuolo d'Ahikam: uomini, gente di guerra, donne, fanciulli, eunuchi; e li ricondussero da Gabaon;
\par 17 e partirono, e si fermarono a Geruth-Kimham presso Bethlehem, per poi continuare a recarsi in Egitto,
\par 18 a motivo de' Caldei dei quali avevano paura, perché Ismael, figliuolo di Nethania, aveva ucciso Ghedalia, figliuolo di Ahikam, che il re di Babilonia aveva stabilito sul paese.

\chapter{42}

\par 1 Tutti i capi delle forze, Johanan, figliuolo di Kareah, Jezania, figliuolo di Hosaia, e tutto il popolo, dal più piccolo al più grande, s'accostarono,
\par 2 e dissero al profeta Geremia: 'Deh, siati accetta la nostra supplicazione, e prega l'Eterno, il tuo Dio, per noi, per tutto questo residuo (poiché, di molti che eravamo, siamo rimasti pochi, come lo vedono gli occhi tuoi);
\par 3 affinché l'Eterno, il tuo Dio, ci mostri la via per la quale dobbiamo camminare, e che cosa dobbiam fare'.
\par 4 E il profeta Geremia disse loro: 'Ho inteso; ecco, io pregherò l'Eterno, il vostro Dio, come avete detto; e tutto quello che l'Eterno vi risponderà, ve lo farò conoscere; e nulla ve ne celerò'.
\par 5 E quelli dissero a Geremia: 'L'Eterno sia un testimonio verace e fedele contro di noi, se non facciamo tutto quello che l'Eterno, il tuo Dio, ti manderà a dirci.
\par 6 Sia la sua risposta gradevole o sgradevole, noi ubbidiremo alla voce dell'Eterno, del nostro Dio, al quale ti mandiamo, affinché bene ce ne venga, per aver ubbidito alla voce dell'Eterno, del nostro Dio'.
\par 7 Dopo dieci giorni, la parola dell'Eterno fu rivolta a Geremia.
\par 8 E Geremia chiamò Johanan, figliuolo di Kareah; tutti i capi delle forze ch'erano con lui, e tutto il popolo, dal più piccolo al più grande, e disse loro:
\par 9 'Così parla l'Eterno, l'Iddio d'Israele, al quale m'avete mandato perché io gli presentassi la vostra supplicazione:
\par 10 Se continuate a dimorare in questo paese, io vi ci stabilirò, e non vi distruggerò; vi pianterò, e non vi sradicherò; perché mi pento del male che v'ho fatto.
\par 11 Non temete il re di Babilonia, del quale avete paura; non lo temete, dice l'Eterno, perché io sono con voi per salvarvi e per liberarvi dalla sua mano;
\par 12 io vi farò trovar compassione dinanzi a lui; egli avrà compassione di voi, e vi farà tornare nel vostro paese.
\par 13 Ma se dite: - Noi non rimarremo in questo paese, - se non ubbidite alla voce dell'Eterno, del vostro Dio, e dite:
\par 14 - No, andremo nel paese d'Egitto, dove non vedremo la guerra, non udremo suon di tromba, e dove non avrem più fame di pane, e quivi dimoreremo, -
\par 15 ebbene, ascoltate allora la parola dell'Eterno, o superstiti di Giuda! Così parla l'Eterno degli eserciti, l'Iddio d'Israele: Se siete decisi a recarvi in Egitto, e se andate a dimorarvi,
\par 16 la spada che temete vi raggiungerà là, nel paese d'Egitto, e la fame che paventate vi starà alle calcagna là in Egitto, e quivi morrete.
\par 17 Tutti quelli che avranno deciso di andare in Egitto per dimorarvi, vi morranno di spada, di fame o di peste; nessun di loro scamperà, sfuggirà al male ch'io farò venire su loro.
\par 18 Poiché così parla l'Eterno degli eserciti, l'Iddio d'Israele: Come la mia ira e il mio furore si son riversati sugli abitanti di Gerusalemme, così il mio furore si riverserà su voi, quando sarete entrati in Egitto; e sarete abbandonati alla esecrazione, alla desolazione, alla maledizione e all'obbrobrio, e non vedrete mai più questo luogo.
\par 19 O superstiti di Giuda! l'Eterno parla a voi: Non andate in Egitto! Sappiate bene che quest'oggi io v'ho premuniti.
\par 20 Voi ingannate voi stessi, a rischio della vostra vita; poiché m'avete mandato dall'Eterno, dal vostro Dio, dicendo: - Prega l'Eterno, il nostro Dio, per noi; e tutto quello che l'Eterno, il nostro Dio, dirà, faccelo sapere esattamente, e noi lo faremo. -
\par 21 E io ve l'ho fatto sapere quest'oggi; ma voi non ubbidite alla voce dell'Eterno, del vostro Dio, né a nulla di quanto egli m'ha mandato a dirvi.
\par 22 Or dunque sappiate bene che voi morrete di spada, di fame e di peste, nel luogo dove desiderate andare per dimorarvi'.

\chapter{43}

\par 1 Or quando Geremia ebbe finito di dire al popolo tutte le parole dell'Eterno, del loro Dio, tutte le parole che l'Eterno, il loro Dio, l'aveva incaricato di dir loro,
\par 2 Azaria, figliuolo di Hosaia, e Johanan, figliuolo di Kareah, e tutti gli uomini superbi dissero a Geremia: 'Tu dici il falso; l'Eterno, il nostro Dio, non t'ha mandato a dire: - Non entrate in Egitto per dimorarvi, -
\par 3 ma Baruc, figliuolo di Neria, t'incita contro di noi per darci in man de' Caldei, per farci morire o per farci menare in cattività a Babilonia'.
\par 4 Così Johanan, figliuolo di Kareah, tutti i capi delle forze e tutto il popolo non ubbidirono alla voce dell'Eterno, che ordinava loro di dimorare nel paese di Giuda.
\par 5 E Johanan, figliuolo di Kareah, e tutti i capi delle forze presero tutti i superstiti di Giuda i quali, di fra tutte le nazioni dov'erano stati dispersi, erano ritornati per dimorare nel paese di Giuda:
\par 6 gli uomini, le donne, i fanciulli, le figliuole del re e tutte le persone che Nebuzaradan, capo delle guardie, aveva lasciate con Ghedalia, figliuolo di Ahikam, figliuolo di Shafan, come pure il profeta Geremia, e Baruc, figliuolo di Neria,
\par 7 ed entrarono nel paese d'Egitto, perché non ubbidirono alla voce dell'Eterno; e giunsero a Tahpanes.
\par 8 E la parola dell'Eterno fu rivolta a Geremia a Tahpanes in questi termini:
\par 9 'Prendi nelle tue mani delle grosse pietre, e nascondile nell'argilla della fornace da mattoni ch'è all'ingresso della casa di Faraone a Tahpanes, in presenza degli uomini di Giuda.
\par 10 E di' loro: Così parla l'Eterno degli eserciti, l'Iddio d'Israele: Ecco, io manderò a prendere Nebucadnetsar, re di Babilonia, mio servitore, e porrò il suo trono su queste pietre che io ho nascoste, ed egli stenderà su d'esse il suo padiglione reale,
\par 11 e verrà e colpirà il paese d'Egitto: chi deve andare alla morte, andrà alla morte; chi in cattività, andrà in cattività; chi deve cader di spada, cadrà per la spada.
\par 12 Ed io appiccherò il fuoco alle case degli dèi d'Egitto. Nebucadnetsar brucerà le case e menerà in cattività gl'idoli, e s'avvolgerà del paese d'Egitto come il pastore s'avvolge nella sua veste; e ne uscirà in pace.
\par 13 Frantumerà pure le statue del tempio del sole, che è nel paese d'Egitto, e darà alle fiamme le case degli dèi d'Egitto'.

\chapter{44}

\par 1 La parola che fu rivolta a Geremia in questi termini, riguardo a tutti i Giudei che dimoravano nel paese di Egitto, che dimoravano a Migdol, a Tahpanes, a Nof e nel paese di Pathros:
\par 2 Così parla l'Eterno degli eserciti, l'Iddio d'Israele: Voi avete veduto tutto il male che io ho fatto venire sopra Gerusalemme e sopra tutte le città di Giuda; ed ecco, oggi sono una desolazione e non v'è chi abiti in esse,
\par 3 a motivo della malvagità che hanno commessa per provocarmi ad ira, andando a far profumi e servire altri dèi, i quali né essi, né voi, né i vostri padri avevate mai conosciuti.
\par 4 E io vi ho mandato tutti i miei servitori, i profeti, ve li ho mandati del continuo, fin dal mattino, a dirvi: - Deh, non fate questa cosa abominevole che io odio; -
\par 5 ma essi non hanno ubbidito, non han prestato orecchio, non si sono stornati dalla loro malvagità, non han cessato di offrir profumi ad altri dèi;
\par 6 perciò il mio furore, la mia ira si son riversati, e han divampato nelle città di Giuda e nelle vie di Gerusalemme, che son ridotte deserte e desolate, come oggi si vede.
\par 7 E ora così parla l'Eterno, l'Iddio degli eserciti, l'Iddio d'Israele: Perché commettete questo gran male contro voi stessi, tanto da farvi sterminare dal mezzo di Giuda, uomini e donne, bambini e lattanti, sì che non rimanga di voi alcun residuo?
\par 8 Perché provocarmi ad ira con l'opera delle vostre mani, facendo profumi ad altri dèi nel paese d'Egitto dove siete venuti a dimorare? Così vi farete sterminare e sarete abbandonati alla maledizione e all'obbrobrio fra tutte le nazioni della terra.
\par 9 Avete voi dimenticato le malvagità dei vostri padri, le malvagità dei re di Giuda, le malvagità delle loro mogli, le malvagità vostre e le malvagità commesse dalle vostre mogli nel paese di Giuda e per le vie di Gerusalemme?
\par 10 Fino ad oggi, non v'è stata contrizione da parte loro, non hanno avuto timore, non hanno camminato secondo la mia legge e secondo i miei statuti, che io avevo messo dinanzi a voi e dinanzi ai vostri padri.
\par 11 Perciò così parla l'Eterno degli eserciti, l'Iddio d'Israele: Ecco, io volgo la mia faccia contro di voi per il vostro male, e per distruggere tutto Giuda.
\par 12 E prenderò i superstiti di Giuda che si sono ostinati a venire nel paese d'Egitto per dimorarvi, e saranno tutti consumati; cadranno nel paese d'Egitto; saranno consumati dalla spada e dalla fame, dal più piccolo al più grande; periranno per la spada e per la fame, e saranno abbandonati alla esecrazione, alla desolazione, alla maledizione e all'obbrobrio.
\par 13 E punirò quelli che dimorano nel paese d'Egitto, come ho punito Gerusalemme con la spada, con la fame e con la peste;
\par 14 e nessuno si salverà o scamperà dei superstiti di Giuda che son venuti a stare nel paese d'Egitto colla speranza di tornare poi nel paese di Giuda, ove desiderano rientrare per dimorarvi; essi, ad eccezione di alcuni fuggiaschi, non vi ritorneranno.
\par 15 Allora tutti gli uomini i quali sapevano che le loro mogli offrivan profumi ad altri dèi, tutte le donne che si trovavano quivi, riunite in gran numero, e tutto il popolo che dimorava nel paese d'Egitto a Pathros, risposero a Geremia, dicendo:
\par 16 'Quanto alla parola che ci hai detta nel nome dell'Eterno, noi non ti ubbidiremo,
\par 17 ma vogliamo mettere interamente ad effetto tutto quello che la nostra bocca ha espresso: offrir profumi alla regina del cielo, farle delle libazioni, come già abbiam fatto noi, i nostri padri, i nostri re, i nostri capi, nelle città di Giuda e per le vie di Gerusalemme; e avevamo allora abbondanza di pane, stavamo bene e non sentivamo alcun male;
\par 18 ma da che abbiam cessato d'offrir profumi alla regina del cielo e di farle delle libazioni, abbiamo avuto mancanza d'ogni cosa, e siamo stati consumati dalla spada e dalla fame.
\par 19 E quando offriamo profumi alla regina del cielo e le facciamo delle libazioni, è egli senza il consenso dei nostri mariti che le facciamo delle focacce a sua immagine e le offriamo delle libazioni?'
\par 20 E Geremia parlò a tutto il popolo, agli uomini, alle donne e a tutto il popolo che gli aveva risposto a quel modo, e disse:
\par 21 'Non sono forse i profumi che avete offerti nelle città di Giuda e per le vie di Gerusalemme, voi, i vostri padri, i vostri re, i vostri capi e il popolo del paese, quelli che l'Eterno ha ricordato e che gli son tornati in mente?
\par 22 L'Eterno non l'ha più potuto sopportare, a motivo della malvagità delle vostre azioni, e a motivo delle abominazioni che avete commesse; perciò il vostro paese è stato abbandonato alla devastazione, alla desolazione e alla maledizione, senza che vi sia più chi l'abiti, come si vede al dì d'oggi.
\par 23 Perché voi avete offerto que' profumi e avete peccato contro l'Eterno e non avete ubbidito alla voce dell'Eterno e non avete camminato secondo la sua legge, i suoi statuti e le sue testimonianze, perciò v'è avvenuto questo male che oggi si vede'.
\par 24 Poi Geremia disse a tutto il popolo e a tutte le donne: 'Ascoltate la parola dell'Eterno, o voi tutti di Giuda, che siete nel paese d'Egitto!
\par 25 Così parla l'Eterno degli eserciti, l'Iddio d'Israele: Voi e le vostre mogli lo dite con la vostra bocca e lo mettete ad effetto con le vostre mani; voi dite: - Vogliamo adempiere i voti che abbiamo fatti, offrendo profumi alla regina del cielo e facendole delle libazioni. - Sì, voi adempite i vostri voti: sì, voi mandate ad effetto i vostri voti;
\par 26 perciò ascoltate la parola dell'Eterno, o voi tutti di Giuda, che dimorate nel paese d'Egitto! Ecco, io lo giuro per il mio gran nome, dice l'Eterno; in tutto il paese d'Egitto il mio nome non sarà più invocato dalla bocca d'alcun uomo di Giuda che dica: - Il Signore, l'Eterno, vive! -
\par 27 Ecco, io vigilo su loro per il loro male, e non per il loro bene; e tutti gli uomini di Giuda che sono nel paese d'Egitto saranno consumati dalla spada e dalla fame, finché non siano interamente scomparsi.
\par 28 E quelli che saranno scampati dalla spada ritorneranno dal paese d'Egitto nel paese di Giuda in ben piccolo numero; e tutto il rimanente di Giuda, quelli che son venuti nel paese d'Egitto per dimorarvi, riconosceranno qual è la parola che sussiste, la mia o la loro.
\par 29 E questo vi sarà per segno, dice l'Eterno, che io vi punirò in questo luogo, affinché riconosciate che le mie parole contro di voi saranno del tutto messe ad effetto, per il vostro male:
\par 30 così parla l'Eterno: - Ecco, io darò Faraone Hofra, re d'Egitto, in mano de' suoi nemici, in mano di quelli che cercano la sua vita, come ho dato Sedekia, re di Giuda, in mano di Nebucadnetsar, re di Babilonia, suo nemico, che cercava la vita di lui'.

\chapter{45}

\par 1 La parola che il profeta Geremia rivolse a Baruc, figliuolo di Neria, quando questi scrisse queste parole in un libro, a dettatura di Geremia, l'anno quarto di Joiakim, figliuolo di Giosia, re di Giuda. Egli disse:
\par 2 'Così parla l'Eterno, l'Iddio d'Israele, riguardo a te, Baruc:
\par 3 Tu dici: Guai a me! poiché l'Eterno aggiunge tristezza al mio dolore; io m'affanno a gemere, e non trovo requie.
\par 4 Digli così: Così parla l'Eterno: Ecco, ciò che ho edificato, io lo distruggerò; ciò che ho piantato, io lo sradicherò; e questo farò in tutto il paese.
\par 5 E tu cercheresti grandi cose per te? Non le cercare! poiché, ecco, io farò venir del male sopra ogni carne, dice l'Eterno, ma a te darò la vita come bottino, in tutti i luoghi dove tu andrai'.

\chapter{46}

\par 1 Parola dell'Eterno che fu rivolta a Geremia riguardo alle nazioni.
\par 2 Riguardo all'Egitto. Circa l'esercito di Faraone Neco, re d'Egitto, che era presso al fiume Eufrate a Carkemish, e che Nebucadnetsar, re di Babilonia, sconfisse il quarto anno di Joiakim, figliuolo di Giosia, re di Giuda.
\par 3 Preparate lo scudo e la targa, e avvicinatevi per la battaglia.
\par 4 Attaccate i cavalli, e voi, cavalieri, montate, e presentatevi con gli elmi in capo; forbite le lance, indossate le corazze!
\par 5 Perché li veggo io sbigottiti, vòlti in rotta? I loro prodi sono sconfitti, si danno alla fuga senza volgersi indietro; d'ogn'intorno è terrore, dice l'Eterno.
\par 6 Il veloce non fugga, il prode non scampi! Al settentrione, presso il fiume Eufrate vacillano e cadono.
\par 7 Chi è colui che sale come il Nilo, e le cui acque s'agitano come quelle de' fiumi?
\par 8 È l'Egitto, che sale come il Nilo, e le cui acque s'agitano come quelle de' fiumi. Egli dice: 'Io salirò, ricoprirò la terra, distruggerò le città e i loro abitanti'.
\par 9 All'assalto! cavalli; al galoppo! carri; si facciano avanti i prodi, quei d'Etiopia e di Put che portan lo scudo e que' di Lud che maneggiano e tendono l'arco.
\par 10 Questo giorno, per il Signore, per l'Eterno degli eserciti, è giorno di vendetta in cui si vendica de' suoi nemici. La spada divorerà, si sazierà, s'inebrierà del loro sangue; poiché il Signore, l'Eterno degli eserciti, immola le vittime nel paese del settentrione, presso il fiume Eufrate.
\par 11 Sali a Galaad, prendi del balsamo, o vergine, figliuola d'Egitto! Invano moltiplichi i rimedi; non v'è medicatura che valga per te.
\par 12 Le nazioni odono la tua ignominia, e la terra è piena del tuo grido; poiché il prode vacilla appoggiandosi al prode, ambedue cadono assieme.
\par 13 Parola che l'Eterno rivolse al profeta Geremia sulla venuta di Nebucadnetsar, re di Babilonia, per colpire il paese d'Egitto.
\par 14 Annunziatelo in Egitto, banditelo a Migdol, banditelo a Nof e Tahpanes! Dite: 'Lèvati, preparati, poiché la spada divora tutto ciò che ti circonda'.
\par 15 Perché i tuoi prodi son essi atterrati? Non posson resistere perché l'Eterno li abbatte.
\par 16 Egli ne fa vacillar molti; essi cadono l'un sopra l'altro, e dicono: 'Andiamo, torniamo al nostro popolo e al nostro paese natìo, sottraendoci alla spada micidiale'.
\par 17 Là essi gridano: 'Faraone, re d'Egitto, non è che un vano rumore, ha lasciato passare il tempo fissato'.
\par 18 Com'è vero ch'io vivo, dice il Re che ha nome l'Eterno degli eserciti, il nemico verrà come un Tabor fra le montagne, come un Carmel che s'avanza sul mare.
\par 19 O figliuola che abiti l'Egitto, fa' il tuo bagaglio per la cattività! poiché Nof diventerà una desolazione; sarà devastata, nessuno v'abiterà più.
\par 20 L'Egitto è una giovenca bellissima, ma viene un tafano, viene dal settentrione.
\par 21 Anche i mercenari che sono in mezzo all'Egitto son come vitelli da ingrasso; anch'essi volgono il dorso, fuggon tutti assieme, non resistono; poiché piomba su loro il giorno della loro calamità, il tempo della loro visitazione.
\par 22 La sua voce giunge come quella d'un serpente; poiché s'avanzano con un esercito, marcian contro a lui con scuri, come tanti tagliaboschi.
\par 23 Essi abbattono la sua foresta, dice l'Eterno, benché sia impenetrabile, perché quelli son più numerosi delle locuste, non si posson contare.
\par 24 La figliuola dell'Egitto è coperta d'onta, è data in mano del popolo del settentrione.
\par 25 L'Eterno degli eserciti, l'Iddio d'Israele, dice: Ecco, io punirò Amon di No, Faraone, l'Egitto, i suoi dèi, i suoi re, Faraone e quelli che confidano in lui;
\par 26 li darò in mano di quei che cercano la loro vita, in mano di Nebucadnetsar, re di Babilonia, e in mano de' suoi servitori; ma, dopo questo, l'Egitto sarà abitato come ai giorni di prima, dice l'Eterno.
\par 27 Tu dunque non temere, o Giacobbe, mio servitore, non ti sgomentare, o Israele! poiché, ecco, io ti salverò dal lontano paese, salverò la tua progenie dalla terra della sua cattività; Giacobbe ritornerà, sarà in riposo, sarà tranquillo, e nessuno più lo spaventerà.
\par 28 Tu non temere, o Giacobbe, mio servitore, dice l'Eterno; poiché io son teco, io annienterò tutte le nazioni fra le quali t'ho disperso, ma non annienterò te; però ti castigherò con giusta misura, e non ti lascerò del tutto impunito.

\chapter{47}

\par 1 La parola dell'Eterno che fu rivolta al profeta Geremia riguardo ai Filistei, prima che Faraone colpisse Gaza.
\par 2 Così parla l'Eterno: Ecco, delle acque salgono dal settentrione; formano un torrente che straripa; esse inondano il paese e tutto ciò che contiene, le città e i loro abitanti; gli uomini mandano grida, tutti gli abitanti del paese urlano.
\par 3 Per lo strepito dell'unghie de' suoi potenti destrieri, per il rumore de' suoi carri e il fracasso delle ruote, i padri non si voltan verso i figliuoli, tanto le lor mani son divenute fiacche,
\par 4 perché giunge il giorno in cui tutti i Filistei saranno devastati, in cui saran soppressi i restanti ausiliari di Tiro e di Sidone, poiché l'Eterno devasterà i Filistei, ciò che resta dell'isola di Caftor.
\par 5 Gaza è divenuta calva, Askalon è ridotta al silenzio. Resti degli Anakim, fino a quando vi farete delle incisioni?
\par 6 O spada dell'Eterno, quando sarà che ti riposerai? Rientra nel tuo fodero, fermati e rimani tranquilla!
\par 7 Come ti potresti tu riposare? L'Eterno le dà i suoi ordini, le addita Askalon e il lido del mare.

\chapter{48}

\par 1 Riguardo a Moab. Così parla l'Eterno degli eserciti, l'Iddio d'Israele: Guai a Nebo! poiché è devastata; Kiriathaim è coperta d'onta, è presa; Misgab è coperta d'onta e sbigottita.
\par 2 Il vanto di Moab non è più; in Heshbon macchinan del male contro di lui: 'Venite, distruggiamolo, e non sia più nazione'. Tu pure, o Madmen, sarai ridotta al silenzio; la spada t'inseguirà.
\par 3 Delle grida vengon da Horonaim: Devastazione e gran rovina!
\par 4 Moab è infranto, i suoi piccini fanno udire i lor gridi.
\par 5 Poiché su per la salita di Luhith si piange, si sale piangendo perché giù per la discesa di Horonaim s'ode il grido angoscioso della rotta.
\par 6 Fuggite, salvate le vostre persone, siano esse come una tamerice nel deserto!
\par 7 Poiché, siccome ti sei confidato nelle tue opere e nei tuoi tesori, anche tu sarai preso; e Kemosh andrà in cattività, coi suoi sacerdoti e coi suoi capi.
\par 8 Il devastatore verrà contro tutte le città, e nessuna città scamperà; la valle perirà e la pianura sarà distrutta, come l'Eterno ha detto.
\par 9 Date delle ali a Moab, poiché bisogna che voli via; le sue città diventeranno una desolazione, senza che più v'abiti alcuno.
\par 10 Maledetto colui che fa l'opera dell'Eterno fiaccamente, maledetto colui che trattiene la spada dallo spargere il sangue!
\par 11 Moab era tranquillo fin dalla sua giovinezza, riposava sulle sue fecce non è stato travasato da vaso a vaso, non è andato in cattività; per questo ha conservato il suo sapore, e il suo profumo non s'è alterato.
\par 12 Perciò ecco, i giorni vengono, dice l'Eterno, ch'io gli manderò de' travasatori, che lo travaseranno; vuoteranno i suoi vasi, frantumeranno le sue ànfore.
\par 13 E Moab avrà vergogna di Kemosh, come la casa d'Israele ha avuto vergogna di Bethel, in cui avea riposto la sua fiducia.
\par 14 Come potete dire: 'Noi siamo uomini prodi, uomini valorosi per la battaglia?'
\par 15 Moab è devastato; le sue città salgono in fumo, il fiore dei suoi giovani scende al macello, dice il re, che ha nome l'Eterno degli eserciti.
\par 16 La calamità di Moab sta per giungere, la sua sciagura viene a gran passi.
\par 17 Compiangetelo voi tutti che lo circondate, e voi tutti che conoscete il suo nome, dite: 'Come s'è spezzato quel forte scettro, quel magnifico bastone?'
\par 18 O figliuola che abiti in Dibon, scendi dalla tua gloria, siedi sul suolo riarso, poiché il devastatore di Moab sale contro di te, distrugge le tue fortezze.
\par 19 O tu che abiti in Aroer, fermati per la strada, e guarda; interroga il fuggiasco e colei che scampa, e di': 'Che è successo?'
\par 20 Moab è coperto d'onta, perché è infranto; mandate urli! gridate! annunziate sull'Arnon che Moab è devastato!
\par 21 Un castigo è venuto sul paese della pianura, sopra Holon, sopra Jahats, su Mefaath,
\par 22 su Dibon, su Nebo, su Beth-Diblathaim,
\par 23 su Kiriathaim, su Beth-Gamul, su Beth-Meon,
\par 24 su Kerioth, su Botsra, su tutte le città del paese di Moab, lontane e vicine.
\par 25 Il corno di Moab è tagliato, il suo braccio è spezzato, dice l'Eterno.
\par 26 Inebriatelo, poich'egli s'è innalzato contro l'Eterno, e si rotoli Moab nel suo vomito, e diventi anch'egli un oggetto di scherno!
\par 27 Israele non è egli stato per te un oggetto di scherno? Era egli forse stato trovato fra i ladri, che ogni volta che parli di lui tu scuoti il capo?
\par 28 Abbandonate le città e andate a stare nelle rocce, o abitanti di Moab! Siate come le colombe che fanno il lor nido sull'orlo de' precipizi.
\par 29 Noi abbiamo udito l'orgoglio di Moab, l'orgogliosissimo popolo, la sua arroganza, la sua superbia, la sua fierezza, l'alterigia del suo cuore.
\par 30 Io conosco la sua tracotanza, dice l'Eterno, ch'è mal fondata; le sue vanterie non hanno approdato a nulla di stabile.
\par 31 Perciò, io alzo un lamento su Moab, io do in gridi per tutto Moab; perciò si geme per quei di Kir-Heres.
\par 32 O vigna di Sibma, io piango per te più ancora che per Jazer; i tuoi rami andavan oltre il mare, arrivavano fino al mare di Jazer; il devastatore è piombato sui tuoi frutti d'estate e sulla tua vendemmia.
\par 33 La gioia e l'allegrezza sono scomparse dalla fertile campagna e dal paese di Moab; io ho fatto venir meno il vino negli strettoi; non si pigia più l'uva con gridi di gioia; il grido che s'ode non è più il grido di gioia.
\par 34 Gli alti lamenti di Heshbon giungon fino a Elealeh; si fanno udire fin verso Jahats; da Tsoar fino a Horonaim, fino a Eglath-Sceliscia; perfino le acque di Nimrim son prosciugate.
\par 35 E io farò venir meno in Moab dice l'Eterno, chi salga sull'alto luogo, e chi offra profumi ai suoi dèi.
\par 36 Perciò il mio cuore geme per Moab come gemono i flauti, il mio cuore geme come gemono i flauti per quei di Kir-Heres, perché tutto quello che aveano ammassato è perduto.
\par 37 Poiché tutte le teste sono rasate, tutte le barbe sono tagliate, su tutte le mani ci son delle incisioni, e sui fianchi, dei sacchi.
\par 38 Su tutti i tetti di Moab e nelle sue piazze, da per tutto, è lamento; poiché io ho frantumato Moab, come un vaso di cui non si fa stima di sorta, dice l'Eterno.
\par 39 Com'è stato infranto! Urlate! Come Moab ha vòlto vergognosamente le spalle! Come Moab è diventato lo scherno e lo spavento di tutti quelli che gli stanno dintorno!
\par 40 Poiché così parla l'Eterno: Ecco, il nemico fende l'aria come l'aquila, spiega le sue ali verso Moab.
\par 41 Kerioth è presa, le fortezze sono occupate, e il cuore dei prodi di Moab, in quel giorno, è come il cuore d'una donna in doglie di parto.
\par 42 Moab sarà distrutto, non sarà più popolo, perché s'è innalzato contro l'Eterno.
\par 43 Spavento, fossa, laccio ti soprastanno, o abitante di Moab! dice l'Eterno.
\par 44 Chi fugge dinanzi allo spavento, cade nella fossa; chi risale dalla fossa, riman preso al laccio; perché io fo venire su lui, su Moab, l'anno in cui dovrà render conto, dice l'Eterno.
\par 45 All'ombra di Heshbon i fuggiaschi si fermano, spossati; ma un fuoco esce da Heshbon, una fiamma di mezzo a Sihon, che divora i fianchi di Moab, il sommo del capo dei figli del tumulto.
\par 46 Guai a te, o Moab! Il popolo di Kemosh è perduto! poiché i tuoi figliuoli son portati via in cattività, e in cattività son menate le tue figliuole.
\par 47 Ma io farò tornar Moab dalla cattività negli ultimi giorni, dice l'Eterno. Fin qui il giudizio su Moab.

\chapter{49}

\par 1 Riguardo ai figliuoli di Ammon. Così parla l'Eterno: Israele non ha egli figliuoli? Non ha egli erede? Perché dunque Malcom prend'egli possesso di Gad, e il suo popolo abita nelle città d'esso?
\par 2 Perciò, ecco, i giorni vengono, dice l'Eterno, ch'io farò udire il grido di guerra contro Rabbah de' figliuoli d'Ammon; essa diventerà un mucchio di ruine, le sue città saranno consumate dal fuoco; allora Israele spodesterà quelli che l'aveano spodestato, dice l'Eterno.
\par 3 Urla, o Heshbon, poiché Ai è devastata; gridate, o città di Rabbah, cingetevi di sacchi, date in lamenti, correte qua e là lungo le chiusure, poiché Malcom va in cattività insieme coi suoi sacerdoti e coi suoi capi.
\par 4 Perché ti glori tu delle tue valli, della tua fertile valle, o figliuola infedele, che confidavi nei tuoi tesori e dicevi: 'Chi verrà contro di me?'
\par 5 Ecco, io ti fo venire addosso da tutti i tuoi dintorni del terrore, dice il Signore, l'Eterno degli eserciti; e voi sarete scacciati, in tutte le direzioni, e non vi sarà chi raduni i fuggiaschi.
\par 6 Ma, dopo questo, io trarrò dalla cattività i figliuoli di Ammon, dice l'Eterno.
\par 7 Riguardo a Edom. Così parla l'Eterno degli eserciti: Non v'è egli più saviezza in Teman? Gl'intelligenti non sanno essi più consigliare? La loro saviezza è dessa svanita?
\par 8 Fuggite, voltate le spalle, nascondetevi profondamente, o abitanti di Dedan! Poiché io fo venire la calamità sopra Esaù, il tempo della sua punizione.
\par 9 Se de' vendemmiatori venissero a te non lascerebbero essi dei racimoli? Se de' ladri venissero a te di notte non guasterebbero più di quanto a loro bastasse.
\par 10 Ma io nuderò Esaù, scoprirò i suoi nascondigli, ed ei non si potrà nascondere; la sua prole, i suoi fratelli, i suoi vicini saran distrutti, ed ei non sarà più.
\par 11 Lascia i tuoi orfani, io li farò vivere, e le tue vedove confidino in me!
\par 12 Poiché così parla l'Eterno: Ecco, quelli che non eran destinati a bere la coppa, la dovranno bere; e tu andresti del tutto impunito? Non andrai impunito, tu la berrai certamente.
\par 13 Poiché io lo giuro per me stesso, dice l'Eterno, Botsra diverrà una desolazione, un obbrobrio, un deserto, una maledizione, e tutte le sue città saranno delle solitudini eterne.
\par 14 Io ho ricevuto un messaggio dall'Eterno, e un messaggero è stato inviato fra le nazioni: 'Adunatevi, venite contro di lei, levatevi per la battaglia!'
\par 15 Poiché, ecco, io ti rendo piccolo fra le nazioni, e sprezzato fra gli uomini.
\par 16 Lo spavento che ispiravi, l'orgoglio del tuo cuore t'han sedotto, o tu che abiti nelle fessure delle rocce, che occupi il sommo delle colline; ma quand'anche tu facessi il tuo nido tant'alto quanto quello dell'aquila, io ti farò precipitar di lassù, dice l'Eterno.
\par 17 E Edom diventerà una desolazione; chiunque passerà presso di lui rimarrà stupito, e si darà a fischiare a motivo di tutte le sue piaghe.
\par 18 Come avvenne al sovvertimento di Sodoma, di Gomorra e di tutte le città a loro vicine, dice l'Eterno, nessuno più abiterà quivi, non vi dimorerà più alcun figliuol d'uomo.
\par 19 Ecco, egli sale come un leone dalle rive lussureggianti del Giordano contro la forte dimora; io ne farò fuggire a un tratto Edom, e stabilirò su di essa colui che io ho scelto. Poiché chi è simile a me? Chi m'ordinerà di comparire in giudizio? Qual è il pastore che possa starmi a fronte?
\par 20 Perciò, ascoltate il disegno che l'Eterno ha concepito contro Edom, e i pensieri che medita contro gli abitanti di Teman! Certo, saran trascinati via come i più piccoli del gregge, certo, la loro dimora sarà devastata.
\par 21 Al rumore della loro caduta trema la terra; s'ode il loro grido fino al mar Rosso.
\par 22 Ecco, il nemico sale, fende l'aria, come l'aquila, spiega le sue ali verso Botsra; e il cuore dei prodi d'Edom, in quel giorno, è come il cuore d'una donna in doglie di parto.
\par 23 Riguardo a Damasco. Hamath e Arpad sono confuse, poiché hanno udito una cattiva notizia; vengon meno; è un'agitazione come quella del mare, che non può calmarsi.
\par 24 Damasco divien fiacca, si volta per fuggire, un tremito l'ha còlta; angoscia e dolori si sono impadroniti di lei, come di donna che partorisce.
\par 25 'Come mai non è stata risparmiata la città famosa, la città della mia gioia?'
\par 26 Così i suoi giovani cadranno nelle sue piazze, e tutti i suoi uomini di guerra periranno in quel giorno, dice l'Eterno degli eserciti.
\par 27 Ed io appiccherò il fuoco alle mura di Damasco, ed esso divorerà i palazzi di Ben-Hadad.
\par 28 Riguardo a Kedar e ai regni di Hatsor, che Nebucadnetsar, re di Babilonia, sconfisse. Così parla l'Eterno: Levatevi, salite contro Kedar, distruggete i figliuoli dell'oriente!
\par 29 Le lor tende, i loro greggi saranno presi; saranno portati via i loro padiglioni, tutti i loro bagagli, i loro cammelli; si griderà loro: 'Spavento da tutte le parti!'
\par 30 Fuggite, dileguatevi ben lungi, nascondetevi profondamente, o abitanti di Hatsor, dice l'Eterno; poiché Nebucadnetsar, re di Babilonia, ha formato un disegno contro di voi, ha concepito un piano contro di voi.
\par 31 Levatevi, salite contro una nazione che gode pace ed abita in sicurtà, dice l'Eterno; che non ha né porte né sbarre, e dimora solitaria.
\par 32 Siano i loro cammelli dati in preda, e la moltitudine del loro bestiame diventi bottino! Io disperderò a tutti i venti quelli che si tagliano i canti della barba, e farò venire la loro calamità da tutte le parti, dice l'Eterno.
\par 33 Hatsor diventerà un ricetto di sciacalli, una desolazione in perpetuo; nessuno più abiterà quivi, non vi dimorerà più alcun figliuol d'uomo.
\par 34 La parola dell'Eterno che fu rivolta in questi termini al profeta Geremia riguardo ad Elam, al principio del regno di Sedekia, re di Giuda:
\par 35 Così parla l'Eterno degli eserciti: Ecco, io spezzo l'arco di Elam, la sua principal forza.
\par 36 Io farò venire contro Elam i quattro venti dalle quattro estremità del cielo; li disperderò a tutti quei venti, e non ci sarà nazione, dove non arrivino de' fuggiaschi d'Elam.
\par 37 Renderò gli Elamiti spaventati dinanzi ai loro nemici, e dinanzi a quelli che cercan la loro vita; farò piombare su loro la calamità, la mia ira ardente, dice l'Eterno; manderò la spada ad inseguirli, finch'io non li abbia consumati.
\par 38 E metterò il mio trono in Elam, e ne farò perire i re ed i capi, dice l'Eterno.
\par 39 Ma negli ultimi giorni avverrà ch'io trarrò Elam dalla cattività, dice l'Eterno.

\chapter{50}

\par 1 Parola che l'Eterno pronunziò riguardo a Babilonia, riguardo al paese de' Caldei, per mezzo del profeta Geremia:
\par 2 Annunziatelo fra le nazioni, proclamatelo, issate una bandiera, proclamatelo, non lo celate! Dite: 'Babilonia è presa! Bel è coperto d'onta, Merodac è infranto! le sue immagini son coperte d'onta; i suoi idoli, infranti!'
\par 3 Poiché dal settentrione sale contro di lei una nazione che ne ridurrà il paese in un deserto, e non vi sarà più alcuno che abiti in lei; uomini e bestie fuggiranno, se n'andranno.
\par 4 In que' giorni, in quel tempo, dice l'Eterno, i figliuoli d'Israele e i figliuoli di Giuda torneranno assieme; cammineranno piangendo, e cercheranno l'Eterno, il loro Dio.
\par 5 Domanderanno qual è la via di Sion, volgeranno le loro facce in direzione d'essa, e diranno: 'Venite, unitevi all'Eterno con un patto eterno, che non si dimentichi più!'
\par 6 Il mio popolo era un gregge di pecore smarrite; i loro pastori le aveano sviate, sui monti dell'infedeltà; esse andavano di monte in colle, avean dimenticato il luogo del loro riposo.
\par 7 Tutti quelli che le trovavano, le divoravano; e i loro nemici, dicevano: 'Noi non siamo colpevoli, poich'essi han peccato contro l'Eterno, dimora della giustizia, contro l'Eterno, speranza de' loro padri'.
\par 8 Fuggite di mezzo a Babilonia, uscite dal paese de' Caldei, e siate come de' capri davanti al gregge!
\par 9 Poiché, ecco, io suscito e fo salire contro Babilonia un'adunata di grandi nazioni dal paese del settentrione, ed esse si schiereranno contro di lei; e da quel lato sarà presa. Le loro frecce son come quelle d'un valente arciere: nessuna d'esse ritorna a vuoto.
\par 10 E la Caldea sarà depredata; tutti quelli che la prederanno saranno saziati, dice l'Eterno.
\par 11 Sì, gioite, sì, rallegratevi, o voi che avete saccheggiato la mia eredità, sì, saltate come una giovenca che trebbia il grano, nitrite come forti destrieri!
\par 12 La madre vostra è tutta coperta d'onta, colei che v'ha partoriti, arrossisce; ecco, essa è l'ultima delle nazioni, un deserto, una terra arida, una solitudine.
\par 13 A motivo dell'ira dell'Eterno non sarà più abitata, sarà una completa solitudine; chiunque passerà presso a Babilonia rimarrà stupito, e fischierà per tutte le sue piaghe.
\par 14 Schieratevi contro Babilonia d'ogn'intorno, o voi tutti che tirate d'arco! Tirate contro di lei, non risparmiate le frecce! poich'essa ha peccato contro l'Eterno.
\par 15 Levate contro di lei il grido di guerra, d'ogn'intorno; ella si arrende; le sue colonne cadono, le sue mura crollano, perché questa è la vendetta dell'Eterno! Vendicatevi di lei! Fate a lei com'essa ha fatto!
\par 16 Sterminate da Babilonia colui che semina, e colui che maneggia la falce al tempo della mèsse. Per scampare alla spada micidiale ritorni ciascuno al suo popolo, fugga ciascuno verso il proprio paese!
\par 17 Israele è una pecora smarrita, a cui de' leoni han dato la caccia; il re d'Assiria, pel primo, l'ha divorata; e quest'ultimo, Nebucadnetsar, re di Babilonia, le ha frantumate le ossa.
\par 18 Perciò così parla l'Eterno degli eserciti, l'Iddio d'Israele: Ecco, io punirò il re di Babilonia e il suo paese, come ho punito il re d'Assiria.
\par 19 E ricondurrò Israele ai suoi pascoli; egli pasturerà al Carmel e in Basan, e l'anima sua si sazierà sui colli d'Efraim e in Galaad.
\par 20 In quei giorni, in quel tempo, dice l'Eterno, si cercherà l'iniquità d'Israele, ma essa non sarà più, e i peccati di Giuda, ma non si troveranno; poiché io perdonerò a quelli che avrò lasciati di resto.
\par 21 Sali contro il paese di Merathaim e contro gli abitanti di Pekod! Inseguili colla spada, votali allo sterminio, dice l'Eterno, e fa' esattamente come io t'ho comandato!
\par 22 S'ode nel paese un grido di guerra, e grande è il disastro.
\par 23 Come mai s'è rotto, s'è spezzato il martello di tutta la terra? Come mai Babilonia è divenuta una desolazione fra le nazioni?
\par 24 Io t'ho teso un laccio, e tu, o Babilonia, vi sei stata presa, senza che te n'accorgessi; sei stata trovata, ed arrestata, perché ti sei messa in guerra contro l'Eterno.
\par 25 L'Eterno ha aperto la sua armeria, e ha tratto fuori le armi della sua indignazione; poiché questa è un'opera che il Signore, l'Eterno degli eserciti, ha da compiere nel paese de' Caldei.
\par 26 Venite contro a lei da tutte le parti, aprite i suoi granai, ammucchiatela come tante mannelle, votatela allo sterminio, che nulla ne resti!
\par 27 Uccidete tutti i suoi tori, fateli scendere al macello! Guai a loro! poiché il loro giorno è giunto, il giorno della loro visitazione.
\par 28 S'ode la voce di quelli che fuggono, che scampano dal paese di Babilonia per annunziare in Sion la vendetta dell'Eterno, del nostro Dio, la vendetta del suo tempio.
\par 29 Convocate contro Babilonia gli arcieri, tutti quelli che tirano d'arco; accampatevi contro a lei d'ogn'intorno, nessuno ne scampi; rendetele secondo le sue opere, fate interamente a lei com'ella ha fatto; poich'ella è stata arrogante contro l'Eterno, contro il Santo d'Israele.
\par 30 Perciò i suoi giovani cadranno nelle sue piazze, e tutti i suoi uomini di guerra periranno in quel giorno, dice l'Eterno.
\par 31 Eccomi a te, o arrogante, dice il Signore, l'Eterno degli eserciti; poiché il tuo giorno è giunto, il tempo ch'io ti visiterò.
\par 32 L'arrogante vacillerà, cadrà, e non vi sarà chi la rialzi; e io appiccherò il fuoco alle sue città, ed esso divorerà tutti i suoi dintorni.
\par 33 Così parla l'Eterno degli eserciti: I figliuoli d'Israele e i figliuoli di Giuda sono oppressi insieme; tutti quelli che li han menati in cattività li tengono, e rifiutano di lasciarli andare.
\par 34 Il loro vindice è forte; ha nome l'Eterno degli eserciti; certo egli difenderà la loro causa, dando requie alla terra e gettando lo scompiglio fra gli abitanti di Babilonia.
\par 35 La spada sovrasta ai Caldei, dice l'Eterno, agli abitanti di Babilonia, ai suoi capi, ai suoi savi.
\par 36 La spada sovrasta ai millantatori, che risulteranno insensati; la spada sovrasta ai suoi prodi, che saranno atterriti;
\par 37 la spada sovrasta ai suoi cavalli, ai suoi carri, a tutta l'accozzaglia di gente ch'è in mezzo a lei, la quale diventerà come tante donne; la spada sovrasta ai suoi tesori, che saran saccheggiati.
\par 38 La siccità sovrasta alle sue acque, che saran prosciugate; poiché è un paese d'immagini scolpite, vanno in delirio per quegli spauracchi dei loro idoli.
\par 39 Perciò gli animali del deserto con gli sciacalli si stabiliranno quivi, e vi si stabiliranno gli struzzi; nessuno vi dimorerà più in perpetuo, non sarà più abitata d'età in età.
\par 40 Come avvenne quando Dio sovvertì Sodoma, Gomorra, e le città loro vicine, dice l'Eterno, nessuno più abiterà quivi, non vi dimorerà più alcun figliuol d'uomo.
\par 41 Ecco, un popolo viene dal settentrione; una grande nazione e molti re sorgono dalle estremità della terra.
\par 42 Essi impugnano l'arco ed il dardo; son crudeli, non hanno pietà; la loro voce è come il muggito del mare; montan cavalli; son pronti a combattere come un solo guerriero, contro di te, o figliuola di Babilonia!
\par 43 Il re di Babilonia n'ode la fama, e le sue mani s'illanguidiscono; l'angoscia lo coglie, un dolore come di donna che partorisce.
\par 44 Ecco, egli sale come un leone dalle rive lussureggianti del Giordano contro la forte dimora; io ne farò fuggire ad un tratto gli abitanti e stabilirò su di essa colui che io ho scelto. Poiché chi è simile a me? chi m'ordinerà di comparire in giudizio? Qual è il pastore che possa starmi a fronte?
\par 45 Perciò, ascoltate il disegno che l'Eterno ha concepito contro Babilonia, e i pensieri che medita contro il paese de' Caldei! Certo, saran trascinati via come i più piccoli del gregge, certo, la loro dimora sarà devastata.
\par 46 Al rumore della presa di Babilonia trema la terra, e se n'ode il grido fra le nazioni.

\chapter{51}

\par 1 Così parla l'Eterno: Ecco, io faccio levare contro Babilonia e contro gli abitanti di questo paese, ch'è il cuore de' miei nemici, un vento distruttore.
\par 2 E mando contro Babilonia degli stranieri che la ventoleranno, e vuoteranno il suo paese; poiché, nel giorno della calamità, piomberanno su di lei da tutte le parti.
\par 3 Tenda l'arciere il suo arco contro chi tende l'arco, e contro chi s'erge fieramente nella sua corazza! Non risparmiate i suoi giovani, votate allo sterminio tutto il suo esercito!
\par 4 Cadano uccisi nel paese de' Caldei, crivellati di ferite per le vie di Babilonia!
\par 5 Poiché Israele e Giuda non son vedovati del loro Dio, dell'Eterno degli eserciti; e il paese de' Caldei è pieno di colpe contro il Santo d'Israele.
\par 6 Fuggite di mezzo a Babilonia, e salvi ognuno la sua vita, guardate di non perire per l'iniquità di lei! Poiché questo è il tempo della vendetta dell'Eterno; egli le dà la sua retribuzione.
\par 7 Babilonia era nelle mani dell'Eterno una coppa d'oro, che inebriava tutta la terra; le nazioni han bevuto del suo vino, perciò le nazioni son divenute deliranti.
\par 8 A un tratto Babilonia è caduta, è frantumata. Mandate su di lei alti lamenti, prendete del balsamo pel suo dolore; forse guarirà!
\par 9 Noi abbiam voluto guarire Babilonia, ma essa non è guarita; abbandonatela, e andiamocene ognuno al nostro paese; poiché la sua punizione arriva sino al cielo, s'innalza fino alle nuvole.
\par 10 L'Eterno ha prodotto in luce la giustizia della nostra causa; venite, raccontiamo in Sion l'opera dell'Eterno, del nostro Dio.
\par 11 Forbite le saette, imbracciate gli scudi! L'Eterno ha eccitato lo spirito dei re dei Medi, perché il suo disegno contro Babilonia è di distruggerla; poiché questa è la vendetta dell'Eterno, la vendetta del suo tempio.
\par 12 Alzate la bandiera contro le mura di Babilonia! Rinforzate le guardie, ponete le sentinelle, preparate gli agguati! Poiché l'Eterno ha divisato e già mette ad effetto ciò che ha detto contro gli abitanti di Babilonia.
\par 13 O tu che abiti in riva alle grandi acque, tu che abbondi di tesori, la tua fine è giunta, il termine delle tue rapine!
\par 14 L'Eterno degli eserciti l'ha giurato per se stesso: Sì, certo, io t'empirò d'uomini come di locuste ed essi leveranno contro di te gridi di trionfo.
\par 15 Egli, con la sua potenza, ha fatto la terra, con la sua sapienza ha stabilito fermamente il mondo; con la sua intelligenza ha disteso i cieli.
\par 16 Quando fa udire la sua voce, v'è un rumor d'acque nel cielo, ei fa salire i vapori dalle estremità della terra, fa guizzare i lampi per la pioggia e trae il vento dai suoi serbatoi;
\par 17 ogni uomo allora diventa stupido, privo di conoscenza, ogni orafo ha vergogna delle sue immagini scolpite; perché le sue immagini fuse sono una menzogna, e non v'è soffio vitale in loro.
\par 18 Sono vanità, lavoro d'inganno; nel giorno del castigo, periranno.
\par 19 A loro non somiglia Colui ch'è la parte di Giacobbe; perché Egli è quel che ha formato tutte le cose, e Israele è la tribù della sua eredità. Il suo nome è l'Eterno degli eserciti.
\par 20 O Babilonia, tu sei stata per me un martello, uno strumento di guerra; con te ho schiacciato le nazioni, con te ho distrutto i regni;
\par 21 con te ho schiacciato cavalli e cavalieri, con te ho schiacciato i carri e chi vi stava sopra;
\par 22 con te ho schiacciato uomini e donne, con te ho schiacciato vecchi e bambini, con te ho schiacciato giovani e fanciulle;
\par 23 con te ho schiacciato i pastori e i lor greggi, con te ho schiacciato i lavoratori e i lor buoi aggiogati; con te ho schiacciato governatori e magistrati.
\par 24 Ma, sotto gli occhi vostri, io renderò a Babilonia e a tutti gli abitanti della Caldea tutto il male che han fatto a Sion, dice l'Eterno.
\par 25 Eccomi a te, o montagna di distruzione, dice l'Eterno; a te che distruggi tutta la terra! Io stenderò la mia mano su di te, ti rotolerò giù dalle rocce, e farò di te una montagna bruciata.
\par 26 E da te non si trarrà più pietra angolare, né pietre da fondamenta; ma tu sarai una desolazione perpetua, dice l'Eterno.
\par 27 Issate una bandiera sulla terra! Sonate la tromba fra le nazioni! Preparate le nazioni contro di lei, chiamate a raccolta contro di lei i regni d'Ararat, di Minni e d'Ashkenaz! Costituite contro di lei de' generali! Fate avanzare i cavalli come locuste dalle ali ritte.
\par 28 Preparate contro di lei le nazioni, i re di Media, i suoi governatori, tutti i suoi magistrati, e tutti i paesi de' suoi dominî.
\par 29 La terra trema, è in doglia perché i disegni dell'Eterno contro Babilonia s'effettuano: di ridurre il paese di Babilonia in un deserto senz'abitanti.
\par 30 I prodi di Babilonia cessan di combattere; se ne stanno nelle loro fortezze; la loro bravura è venuta meno, son come donne; le sue abitazioni sono in fiamme, le sbarre delle sue porte sono spezzate.
\par 31 Un corriere incrocia l'altro, un messaggero incrocia l'altro, per annunziare al re di Babilonia che la sua città è presa da ogni lato,
\par 32 che i guadi son occupati, che le paludi sono in preda alle fiamme, che gli uomini di guerra sono allibiti.
\par 33 Poiché così parla l'Eterno degli eserciti, l'Iddio d'Israele: La figliuola di Babilonia è come un'aia al tempo in cui la si trebbia; ancora un poco, e verrà per lei il tempo della mietitura.
\par 34 Nebucadnetsar, re di Babilonia, ci ha divorati, ci ha schiacciati, ci ha posti là come un vaso vuoto; ci ha inghiottiti come un dragone; ha empito il suo ventre con le nostre delizie, ci ha cacciati via.
\par 35 'La violenza che m'è fatta e la mia carne ricadano su Babilonia', dirà l'abitante di Sion; 'Il mio sangue ricada sugli abitanti di Caldea', dirà Gerusalemme.
\par 36 Perciò, così parla l'Eterno: Ecco, io difenderò la tua causa, e farò la tua vendetta! io prosciugherò il suo mare, disseccherò la sua sorgente,
\par 37 e Babilonia diventerà un monte di ruine, un ricetto di sciacalli, un oggetto di stupore e di scherno, un luogo senz'abitanti.
\par 38 Essi ruggiranno assieme come leoni, grideranno come piccini di leonesse.
\par 39 Quando saranno riscaldati, darò loro da bere, li inebrierò perché stiano allegri, e poi s'addormentino d'un sonno perpetuo, e non si risveglino più, dice l'Eterno.
\par 40 Io li farò scendere al macello come agnelli, come montoni, come capri.
\par 41 Come mai è stata presa Sceshac, ed è stata conquistata colei ch'era il vanto di tutta la terra? Come mai Babilonia è ella diventata una desolazione fra le nazioni?
\par 42 Il mare è salito su Babilonia; essa è stata coperta dal tumulto de' suoi flutti.
\par 43 Le sue città son diventate una desolazione, una terra arida, un deserto, un paese dove non abita alcuno, per dove non passa alcun figliuol d'uomo.
\par 44 Io punirò Bel in Babilonia, e gli trarrò di gola ciò che ha trangugiato, e le nazioni non affluiranno più a lui; perfin le mura di Babilonia son cadute.
\par 45 O popolo mio, uscite di mezzo a lei, e salvi ciascuno la sua vita d'innanzi all'ardente ira dell'Eterno!
\par 46 Il vostro cuore non s'avvilisca, e non vi spaventate delle voci che s'udranno nel paese; poiché un anno correrà una voce, e l'anno seguente correrà un'altra voce; vi sarà nel paese violenza, dominatore contro dominatore.
\par 47 Perciò, ecco, i giorni vengono ch'io farò giustizia delle immagini scolpite di Babilonia, e tutto il suo paese sarà coperto d'onta, e tutti i suoi feriti a morte cadranno in mezzo a lei.
\par 48 E i cieli, la terra, e tutto ciò ch'è in essi, giubileranno su Babilonia, perché i devastatori piomberanno su lei dal settentrione, dice l'Eterno.
\par 49 Come Babilonia ha fatto cadere i feriti a morte d'Israele, così in Babilonia cadranno i feriti a morte di tutto il paese.
\par 50 O voi che siete scampati dalla spada, partite, non vi fermate, ricordatevi da lungi dell'Eterno, e Gerusalemme vi ritorni in cuore!
\par 51 Noi eravamo coperti d'onta all'udire gli oltraggi, la vergogna ci copriva la faccia, perché gli stranieri eran venuti nel santuario della casa dell'Eterno.
\par 52 Perciò, ecco, i giorni vengono, dice l'Eterno, ch'io farò giustizia delle sue immagini scolpite, e in tutto il suo paese gemeranno i feriti a morte.
\par 53 Quand'anche Babilonia s'elevasse fino al cielo, quand'anche rendesse inaccessibili i suoi alti baluardi, le verranno da parte mia dei devastatori, dice l'Eterno.
\par 54 Giunge da Babilonia un grido, la notizia d'un gran disastro dalla terra de' Caldei.
\par 55 Poiché l'Eterno devasta Babilonia, e fa cessare il suo grande rumore; le onde dei devastatori muggono come grandi acque, se ne ode il fracasso;
\par 56 poiché il devastatore piomba su lei, su Babilonia, i suoi prodi son presi, i loro archi spezzati, giacché l'Eterno è l'Iddio delle retribuzioni, non manca di rendere ciò ch'è dovuto.
\par 57 Io inebrierò i suoi capi e i suoi savi, i suoi governatori, i suoi magistrati, i suoi prodi, ed essi s'addormenteranno d'un sonno eterno, e non si risveglieranno più, dice il Re, che ha nome l'Eterno degli eserciti.
\par 58 Così parla l'Eterno degli eserciti: Le larghe mura di Babilonia saranno spianate al suolo, le sue alte porte saranno incendiate, sicché i popoli avran lavorato per nulla, le nazioni si saranno stancate per il fuoco.
\par 59 Ordine, dato dal profeta Geremia a Seraia, figliuolo di Neria, figliuolo di Mahaseia, quando si recò a Babilonia con Sedekia, re di Giuda, il quarto anno del regno di Sedekia. Seraia era capo dei ciambellani.
\par 60 Geremia scrisse in un libro tutto il male che doveva accadere a Babilonia, cioè tutte queste parole che sono scritte riguardo a Babilonia.
\par 61 E Geremia disse a Seraia: 'Quando sarai arrivato a Babilonia, avrai cura di leggere tutte queste parole,
\par 62 e dirai: - O Eterno, tu hai detto di questo luogo che lo avresti distrutto, sì che non sarebbe più abitato né da uomo, né da bestia, e che sarebbe ridotto in una desolazione perpetua. -
\par 63 E quando avrai finito di leggere questo libro, tu vi legherai una pietra, lo getterai in mezzo all'Eufrate,
\par 64 e dirai: - Così affonderà Babilonia, e non si rialzerà più, a motivo del male ch'io faccio venire su di lei; cadrà esausta'. Fin qui, le parole di Geremia.

\chapter{52}

\par 1 Sedekia aveva ventun anni quando cominciò a regnare, e regnò a Gerusalemme undici anni. Sua madre si chiamava Hamutal, figliuola di Geremia da Libna.
\par 2 Egli fece ciò ch'è male agli occhi dell'Eterno, in tutto e per tutto come avea fatto Joiakim.
\par 3 E a causa dell'ira dell'Eterno contro Gerusalemme e Giuda, le cose arrivarono al punto che l'Eterno li cacciò dalla sua presenza. E Sedekia si ribellò al re di Babilonia.
\par 4 L'anno nono del regno di Sedekia, il decimo giorno del decimo mese, Nebucadnetsar, re di Babilonia, venne con tutto il suo esercito contro Gerusalemme; s'accampò contro di lei, e la circondò di posti fortificati.
\par 5 E la città fu assediata fino all'undecimo anno del re Sedekia.
\par 6 Il nono giorno del quarto mese, la carestia era grave nella città; e non c'era più pane per il popolo del paese.
\par 7 Allora fu fatta una breccia alla città, e tutta la gente di guerra fuggì uscendo di notte dalla città, per la via della porta fra le due mura, in prossimità del giardino del re, mentre i Caldei stringevano la città da ogni parte; e i fuggiaschi presero la via della pianura;
\par 8 ma l'esercito dei Caldei inseguì il re, raggiunse Sedekia nelle pianure di Gerico, e tutto l'esercito di lui si disperse e l'abbandonò.
\par 9 Allora i Caldei presero il re, e lo condussero al re di Babilonia a Ribla nel paese di Hamath; ed egli pronunziò la sua sentenza contro di lui.
\par 10 Il re di Babilonia fece scannare i figliuoli di Sedekia in presenza di lui, fece pure scannare tutti i capi di Giuda a Ribla.
\par 11 Poi fece cavar gli occhi a Sedekia; e il re di Babilonia lo fece incatenare con una doppia catena di rame e lo menò a Babilonia, e lo mise in prigione, dove rimase fino al giorno della sua morte.
\par 12 Or il decimo giorno del quinto mese - era il diciannovesimo anno di Nebucadnetsar, re di Babilonia - Nebuzaradan, capitano della guardia del corpo, al servizio del re di Babilonia, giunse a Gerusalemme,
\par 13 e arse la casa dell'Eterno e la casa del re, diede alle fiamme tutte le case di Gerusalemme, e arse tutte le case ragguardevoli.
\par 14 E tutto l'esercito dei Caldei ch'era col capitano della guardia atterrò da tutte le parti le mura di Gerusalemme.
\par 15 Nebuzaradan, capitano della guardia, menò in cattività una parte dei più poveri del popolo, i superstiti ch'erano rimasti nella città, i fuggiaschi che s'erano arresi al re di Babilonia, e il resto della popolazione.
\par 16 Ma Nebuzaradan, capitano della guardia, lasciò alcuni dei più poveri del paese a coltivar le vigne ed i campi.
\par 17 I Caldei spezzarono le colonne di rame ch'erano nella casa dell'Eterno, le basi, il mar di rame ch'era nella casa dell'Eterno, e ne portaron via il rame a Babilonia.
\par 18 Presero le pignatte, le palette, i coltelli, i bacini, le coppe, e tutti gli utensili di rame coi quali si faceva il servizio.
\par 19 Il capo della guardia prese pure le coppe, i bracieri, i bacini, le pignatte, i candelabri, le tazze e i calici, l'oro di ciò ch'era d'oro e l'argento di ciò ch'era d'argento.
\par 20 Quanto alle due colonne, al mare e ai dodici buoi di rame che servivano di base e che Salomone avea fatti per la casa dell'Eterno, il rame di tutti questi oggetti aveva un peso incalcolabile.
\par 21 L'altezza di una di queste colonne era di diciotto cubiti, e a misurarla in giro ci voleva un filo di dodici cubiti; aveva uno spessore di quattro dita, ed era vuota;
\par 22 e v'era su un capitello di rame; e l'altezza d'ogni capitello era di cinque cubiti; attorno al capitello v'erano un reticolato e delle melagrane, ogni cosa di rame; lo stesso era della seconda colonna, adorna pure di melagrane.
\par 23 V'erano novantasei melagrane da ogni lato, e tutte le melagrane attorno al reticolato ammontavano a cento.
\par 24 Il capitano della guardia prese Seraia, il sommo sacerdote, Sofonia, il secondo sacerdote, e i tre custodi della soglia,
\par 25 e prese nella città un eunuco che comandava la gente di guerra, sette uomini di fra i consiglieri intimi del re che furon trovati nella città, il segretario del capo dell'esercito che arruolava il popolo del paese, e sessanta privati che furono anch'essi trovati nella città.
\par 26 Nebuzaradan, capitano della guardia, li prese e li condusse al re di Babilonia a Ribla,
\par 27 e il re di Babilonia li fece colpire e mettere a morte a Ribla, nel paese di Hamath.
\par 28 Così Giuda fu menato in cattività lungi dal suo paese. Questo è il popolo che Nebucadnetsar menò in cattività: il settimo anno, tremilaventitre Giudei;
\par 29 il diciottesimo anno del suo regno, menò in cattività da Gerusalemme ottocentotrentadue persone;
\par 30 il ventitreesimo anno di Nebucadnetsar, Nebuzaradan, capitano della guardia, menò in cattività settecentoquarantacinque Giudei: in tutto, quattromilaseicento persone.
\par 31 Il trentasettesimo anno della cattività di Joiakin, re di Giuda, il venticinquesimo giorno del dodicesimo mese, Evil-Merodac, re di Babilonia, l'anno stesso che cominciò a regnare, fece grazia a Joiakin, re di Giuda, e lo trasse di prigione;
\par 32 gli parlò benignamente, e mise il trono d'esso più in alto di quello degli altri re ch'eran con lui a Babilonia.
\par 33 Gli fece mutare i suoi vestiti di prigione; e Joiakin mangiò sempre a tavola con lui per tutto il tempo ch'ei visse.
\par 34 E quanto al suo mantenimento, durante tutto il tempo che visse, esso gli fu dato del continuo da parte del re di Babilonia, giorno per giorno, fino al giorno della sua morte.


\end{document}