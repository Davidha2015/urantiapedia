\begin{document}

\title{Daniele}


\chapter{1}

\par 1 Il terzo anno del regno di Joiakim, re di Giuda, Nebucadnetsar, re di Babilonia, venne contro Gerusalemme, e l'assediò.
\par 2 Il Signore gli diede nelle mani Joiakim, re di Giuda, e una parte degli utensili della casa di Dio; e Nebucadnetsar portò gli utensili nel paese di Scinear, nella casa del suo dio, e li mise nella casa del tesoro del suo dio.
\par 3 E il re disse ad Ashpenaz, capo de' suoi eunuchi, di menargli alcuni de' figliuoli d'Israele di stirpe reale e di famiglie nobili,
\par 4 giovani senza difetti fisici, belli d'aspetto, dotati d'ogni sorta di talenti, istruiti e intelligenti, tali che avessero attitudine a stare nel palazzo del re; e d'insegnar loro la letteratura e la lingua de' Caldei.
\par 5 Il re assegnò loro una porzione giornaliera delle vivande della mensa reale, e del vino ch'egli beveva; e disse di mantenerli per tre anni, dopo i quali sarebbero passati al servizio del re.
\par 6 Or fra questi c'erano, di tra i figliuoli di Giuda, Daniele, Hanania, Mishael e Azaria;
\par 7 e il capo degli eunuchi diede loro altri nomi: a Daniele pose nome Beltsatsar; ad Hanania, Shadrac; a Mishael, Meshac, e ad Azaria, Abed-nego.
\par 8 E Daniele prese in cuor suo la risoluzione di non contaminarsi con le vivande del re e col vino che il re beveva; e chiese al capo degli eunuchi di non obbligarlo a contaminarsi;
\par 9 e Dio fece trovare a Daniele grazia e compassione presso il capo degli eunuchi.
\par 10 E il capo degli eunuchi disse a Daniele: 'Io temo il re, mio signore, il quale ha fissato il vostro cibo e le vostre bevande; e perché vedrebb'egli il vostro volto più triste di quello dei giovani della vostra medesima età? Voi mettereste in pericolo la mia testa presso il re'.
\par 11 Allora Daniele disse al maggiordomo, al quale il capo degli eunuchi aveva affidato la cura di Daniele, di Hanania, di Mishael e d'Azaria:
\par 12 'Ti prego, fa' coi tuoi servi una prova di dieci giorni, e ci siano dati de' legumi per mangiare, e dell'acqua per bere;
\par 13 poi ti si faccia vedere l'aspetto nostro e l'aspetto de' giovani che mangiano le vivande del re; e secondo quel che vedrai, ti regolerai coi tuoi servi'.
\par 14 Quegli accordò loro quanto domandavano, e li mise alla prova per dieci giorni.
\par 15 E alla fine de' dieci giorni, essi avevano miglior aspetto ed erano più grassi di tutti i giovani che aveano mangiato le vivande del re.
\par 16 Così il maggiordomo portò via il cibo e il vino ch'eran loro destinati, e dette loro de' legumi.
\par 17 E a tutti questi quattro giovani Iddio dette conoscenza e intelligenza in tutta la letteratura, e sapienza; e Daniele s'intendeva d'ogni sorta di visioni e di sogni.
\par 18 E alla fine del tempo fissato dal re perché que' giovani gli fossero menati, il capo degli eunuchi li presentò a Nebucadnetsar.
\par 19 Il re parlò con loro; e fra tutti que' giovani non se ne trovò alcuno che fosse come Daniele, Hanania, Mishael e Azaria; e questi furono ammessi al servizio del re.
\par 20 E su tutti i punti che richiedevano sapienza e intelletto, e sui quali il re li interrogasse, il re li trovava dieci volte superiori a tutti i magi ed astrologi ch'erano in tutto il suo regno.
\par 21 Così continuò Daniele fino al primo anno del re Ciro.

\chapter{2}

\par 1 Il secondo anno del regno di Nebucadnetsar, Nebucadnetsar ebbe dei sogni; il suo spirito ne fu turbato, e il suo sonno fu rotto.
\par 2 Il re fece chiamare i magi, gli astrologi, gl'incantatori e i Caldei, perché gli spiegassero i suoi sogni. Ed essi vennero e si presentarono al re.
\par 3 E il re disse loro: 'Ho fatto un sogno; e il mio spirito è turbato, perché vorrei comprendere il sogno'.
\par 4 Allora i Caldei risposero al re, in aramaico: 'O re, possa tu vivere in perpetuo! Racconta il sogno ai tuoi servi, e noi ne daremo la interpretazione'.
\par 5 Il re replicò, e disse ai Caldei: 'La mia decisione è presa: se voi non mi fate conoscere il sogno e la sua interpretazione, sarete fatti a pezzi; e le vostre case saran ridotte in tanti immondezzai;
\par 6 ma se mi dite il sogno e la sua interpretazione, riceverete da me doni, ricompense e grandi onori; ditemi dunque il sogno e la sua interpretazione'.
\par 7 Quelli risposero una seconda volta, e dissero: 'Dica il re il sogno ai suoi servi, e noi ne daremo l'interpretazione'.
\par 8 Il re replicò, e disse: 'Io m'accorgo che di certo voi volete guadagnar tempo, perché vedete che la mia decisione è presa;
\par 9 se dunque non mi fate conoscere il sogno, non c'è che un'unica sentenza per voi; e voi vi siete messi d'accordo per dire davanti a me delle parole bugiarde e perverse, aspettando che mutino i tempi. Perciò ditemi il sogno, e io saprò che siete in grado di darmene l'interpretazione'.
\par 10 I Caldei risposero in presenza del re, e dissero: 'Non c'è uomo sulla terra che possa far conoscere quello che il re domanda; così non c'è mai stato re, per grande e potente che fosse, il quale abbia domandato una cosa siffatta a un mago, a un astrologo, o a un Caldeo.
\par 11 La cosa che il re domanda è ardua; e non v'è alcuno che la possa far conoscere al re, tranne gli dèi, la cui dimora non è fra i mortali'.
\par 12 A questo, il re s'adirò, montò in furia, e ordinò che tutti i savi di Babilonia fossero fatti perire.
\par 13 E il decreto fu promulgato, e i savi dovevano essere uccisi; e si cercavano Daniele e i suoi compagni per uccidere anche loro.
\par 14 Allora Daniele si rivolse in modo prudente e sensato ad Arioc, capo delle guardie del re, il quale era uscito per uccidere i savi di Babilonia.
\par 15 Prese la parola e disse ad Arioc, ufficiale del re: 'Perché questo decreto così perentorio da parte del re?' Allora Arioc fece sapere la cosa a Daniele.
\par 16 E Daniele entrò dal re, e gli chiese di dargli tempo; che avrebbe fatto conoscere al re l'interpretazione del sogno.
\par 17 Allora Daniele andò a casa sua, e informò della cosa Hanania, Mishael e Azaria, suoi compagni,
\par 18 perché implorassero la misericordia dell'Iddio del cielo, a proposito di questo segreto, onde Daniele e i suoi compagni non fossero messi a morte col resto dei savi di Babilonia.
\par 19 Allora il segreto fu rivelato a Daniele in una visione notturna. E Daniele benedisse l'Iddio del cielo.
\par 20 Daniele prese a dire: 'Sia benedetto il nome di Dio, d'eternità in eternità! poiché a lui appartengono la sapienza e la forza.
\par 21 Egli muta i tempi e le stagioni; depone i re e li stabilisce, dà la sapienza ai savi, e la scienza a quelli che hanno intelletto.
\par 22 Egli rivela le cose profonde e occulte; conosce ciò ch'è nelle tenebre, e la luce dimora con lui.
\par 23 O Dio de' miei padri, io ti rendo gloria e lode, perché m'hai dato sapienza e forza, e m'hai fatto conoscere quello che t'abbiam domandato, rivelandoci la cosa che il re vuole'.
\par 24 Daniele entrò quindi da Arioc, a cui il re aveva dato l'incarico di far perire i savi di Babilonia; entrò, e gli disse così: 'Non far perire i savi di Babilonia! Conducimi davanti al re, e io darò al re l'interpretazione'.
\par 25 Allora Arioc menò in tutta fretta Daniele davanti al re, e gli parlò così: 'Io ho trovato, fra i Giudei che sono in cattività, un uomo che darà al re l'interpretazione'.
\par 26 Il re prese a dire a Daniele, che si chiamava Beltsatsar: 'Sei tu capace di farmi conoscere il sogno che ho fatto e la sua interpretazione?'
\par 27 Daniele rispose in presenza del re, e disse: 'Il segreto che il re domanda, né savi, né incantatori, né magi, né astrologi possono svelarlo al re;
\par 28 ma v'è nel cielo un Dio che rivela i segreti, ed egli ha fatto conoscere al re Nebucadnetsar quello che avverrà negli ultimi giorni. Ecco quali erano il tuo sogno e le visioni della tua mente quand'eri a letto.
\par 29 I tuoi pensieri, o re, quand'eri a letto, si riferivano a quello che deve avvenire da ora innanzi; e colui che rivela i segreti t'ha fatto conoscere quello che avverrà.
\par 30 E quanto a me, questo segreto m'è stato rivelato, non per una sapienza ch'io possegga superiore a quella di tutti gli altri viventi, ma perché l'interpretazione ne sia data al re, e tu possa conoscere quel che preoccupava il tuo cuore.
\par 31 Tu, o re, guardavi, ed ecco una grande statua; questa statua, ch'era immensa e d'uno splendore straordinario, si ergeva dinanzi a te, e il suo aspetto era terribile.
\par 32 La testa di questa statua era d'oro fino; il suo petto e le sue braccia eran d'argento; il suo ventre e le sue cosce, di rame;
\par 33 le sue gambe, di ferro; i suoi piedi, in parte di ferro e in parte d'argilla.
\par 34 Tu stavi guardando, quand'ecco una pietra si staccò, senz'opera di mano, e colpì i piedi di ferro e d'argilla della statua, e li frantumò.
\par 35 Allora il ferro, l'argilla, il rame, l'argento e l'oro furon frantumati insieme, e diventarono come la pula sulle aie d'estate; il vento li portò via, e non se ne trovò più traccia; ma la pietra che avea colpito la statua diventò un gran monte, che riempì tutta la terra.
\par 36 Questo è il sogno; ora ne daremo l'interpretazione davanti al re.
\par 37 Tu, o re, sei il re dei re, al quale l'Iddio del cielo ha dato l'impero, la potenza, la forza e la gloria;
\par 38 e dovunque dimorano i figliuoli degli uomini, le bestie della campagna e gli uccelli del cielo, egli te li ha dati nelle mani, e t'ha fatto dominare sopra essi tutti. La testa d'oro sei tu;
\par 39 e dopo di te sorgerà un altro regno, inferiore al tuo; poi un terzo regno, di rame, che dominerà sulla terra;
\par 40 poi vi sarà un quarto regno, forte come il ferro; poiché, come il ferro spezza ed abbatte ogni cosa, così, pari al ferro che tutto frantuma, esso spezzerà ogni cosa.
\par 41 E come hai visto i piedi e le dita, in parte d'argilla di vasaio e in parte di ferro, così quel regno sarà diviso; ma vi sarà in lui qualcosa della consistenza del ferro, giacché tu hai visto il ferro mescolato con la molle argilla.
\par 42 E come le dita de' piedi erano in parte di ferro e in parte d'argilla, così quel regno sarà in parte forte e in parte fragile.
\par 43 Tu hai visto il ferro mescolato con la molle argilla, perché quelli si mescoleranno mediante connubî umani; ma non saranno uniti l'uno all'altro, nello stesso modo che il ferro non s'amalgama con l'argilla.
\par 44 E al tempo di questi re, l'Iddio del cielo farà sorgere un regno, che non sarà mai distrutto, e che non passerà sotto la dominazione d'un altro popolo; quello spezzerà e annienterà tutti quei regni; ma esso sussisterà in perpetuo,
\par 45 nel modo che hai visto la pietra staccarsi dal monte, senz'opera di mano, e spezzare il ferro, il rame, l'argilla, l'argento e l'oro. Il grande Iddio ha fatto conoscere al re ciò che deve avvenire d'ora innanzi; il sogno è verace, e la interpretazione n'è sicura'.
\par 46 Allora il re Nebucadnetsar cadde sulla sua faccia, si prostrò davanti a Daniele, e ordinò che gli fossero presentati offerte e profumi.
\par 47 Il re parlò a Daniele, e disse: 'In verità il vostro Dio è l'Iddio degli dèi, il Signore dei re, e il rivelatore dei segreti, giacché tu hai potuto rivelare questo segreto'.
\par 48 Allora il re elevò Daniele in dignità, lo colmò di numerosi e ricchi doni, gli diede il comando di tutta la provincia di Babilonia, e lo stabilì capo supremo di tutti i savi di Babilonia.
\par 49 E Daniele ottenne dal re che Shadrac, Meshac e Abed-nego fossero preposti agli affari della provincia di Babilonia; ma Daniele stava alla corte del re.

\chapter{3}

\par 1 Il re Nebucadnetsar fece una statua d'oro, alta sessanta cubiti e larga sei cubiti, e la eresse nella pianura di Dura, nella provincia di Babilonia.
\par 2 E il re Nebucadnetsar mandò a radunare i satrapi, i prefetti, i governatori, i giudici, i tesorieri, i giureconsulti, i presidenti e tutte le autorità delle province, perché venissero alla inaugurazione della statua che il re Nebucadnetsar aveva eretta.
\par 3 Allora i satrapi, i prefetti e i governatori, i giudici, i tesorieri, i giureconsulti, i presidenti e tutte le autorità delle province s'adunarono per la inaugurazione della statua, che il re Nebucadnetsar aveva eretta; e stavano in piedi davanti alla statua che Nebucadnetsar aveva eretta.
\par 4 E l'araldo gridò forte: 'A voi, popoli, nazioni e lingue è imposto che,
\par 5 nel momento in cui udrete il suono del corno, del flauto, della cetra, della lira, del saltèro, della zampogna e d'ogni sorta di strumenti, vi prostriate per adorare la statua d'oro che il re Nebucadnetsar ha eretta;
\par 6 e chiunque non si prostrerà per adorare, sarà immantinente gettato in mezzo a una fornace di fuoco ardente'.
\par 7 Non appena quindi tutti i popoli ebbero udito il suono del corno, del flauto, della cetra, della lira, del saltèro e d'ogni sorta di strumenti, tutti i popoli, tutte le nazioni e lingue si prostrarono e adorarono la statua d'oro, che il re Nebucadnetsar aveva eretta.
\par 8 Allora, in quello stesso momento, alcuni uomini caldei si fecero avanti, e accusarono i Giudei;
\par 9 e, rivolgendosi al re Nebucadnetsar, gli dissero: 'O re, possa tu vivere in perpetuo!
\par 10 Tu, o re, hai emanato un decreto, per il quale chiunque ha udito il suono del corno, del flauto, della cetra, della lira, del saltèro, della zampogna e d'ogni sorta di strumenti deve prostrarsi per adorare la statua d'oro;
\par 11 e chiunque non si prostra e non adora, dev'esser gettato in mezzo a una fornace di fuoco ardente.
\par 12 Or vi sono degli uomini giudei, che tu hai preposti agli affari della provincia di Babilonia: Shadrac, Meshac e Abed-nego; cotesti uomini, o re, non ti tengono in alcun conto; non servono i tuoi dèi, e non adorano la statua d'oro che tu hai eretta'.
\par 13 Allora Nebucadnetsar, irritato e furioso, ordinò che gli fossero menati Shadrac, Meshac e Abed-nego; e quegli uomini furon menati in presenza del re.
\par 14 Nebucadnetsar, rivolgendosi a loro, disse: 'Shadrac, Meshac, Abed-nego, lo fate deliberatamente di non servire i miei dèi e di non adorare la statua d'oro che io ho eretto?
\par 15 Ora, se non appena udrete il suono del corno, del flauto, della cetra, della lira, del saltèro, della zampogna e d'ogni sorta di strumenti, siete pronti a prostrarvi per adorare la statua che io ho fatto, bene; ma se non l'adorate, sarete immantinente gettati in mezzo a una fornace di fuoco ardente; e qual è quel dio che vi libererà dalle mie mani?'
\par 16 Shadrac, Meshac e Abed-nego risposero al re, dicendo: 'O Nebucadnetsar, noi non abbiam bisogno di darti risposta su questo.
\par 17 Ecco, il nostro Dio che noi serviamo, è potente da liberarci, e ci libererà dalla fornace del fuoco ardente, e dalla tua mano, o re.
\par 18 Se no, sappi o re, che noi non serviremo i tuoi dèi e non adoreremo la statua d'oro che tu hai eretto'.
\par 19 Allora Nebucadnetsar fu ripieno di furore, e l'aspetto del suo viso fu mutato verso Shadrac, Meshac e Abed-nego. Egli riprese la parola, e ordinò che si accendesse la fornace sette volte più di quello che s'era pensato di fare;
\par 20 poi comandò ad alcuni uomini de' più vigorosi del suo esercito di legare Shadrac, Meshac e Abed-nego, e di gettarli nella fornace del fuoco ardente.
\par 21 Allora questi tre uomini furon legati con le loro tuniche, le loro sopravvesti, i loro mantelli e tutti i loro vestiti, e furon gettati in mezzo alla fornace del fuoco ardente.
\par 22 E siccome l'ordine del re era perentorio e la fornace era straordinariamente riscaldata, la fiamma del fuoco uccise gli uomini che vi avevano gettato dentro Shadrac, Meshac e Abed-nego.
\par 23 E quei tre uomini, Shadrac, Meshac e Abed-nego, caddero legati in mezzo alla fornace del fuoco ardente.
\par 24 Allora il re Nebucadnetsar fu spaventato, si levò in gran fretta, e prese a dire ai suoi consiglieri: 'Non abbiam noi gettato in mezzo al fuoco tre uomini legati?' Quelli risposero e dissero al re: 'Certo o re!'
\par 25 Ed egli riprese a dire: 'Ecco, io vedo quattro uomini, sciolti, che camminano in mezzo al fuoco, senz'aver sofferto danno alcuno; e l'aspetto del quarto è come quello d'un figlio degli dèi'.
\par 26 Poi Nebucadnetsar s'avvicinò alla bocca della fornace del fuoco ardente, e prese a dire: 'Shadrac, Meshac, Abed-nego, servi dell'Iddio altissimo, uscite, venite!' E Shadrac, Meshac e Abed-nego uscirono di mezzo al fuoco.
\par 27 E i satrapi, i prefetti, i governatori e i consiglieri del re, essendosi adunati, guardarono quegli uomini, e videro che il fuoco non aveva avuto alcun potere sul loro corpo, che i capelli del loro capo non erano stati arsi, che le loro tuniche non erano alterate, e ch'essi non avevano odor di fuoco.
\par 28 E Nebucadnetsar prese a dire: 'Benedetto sia l'Iddio di Shadrac, di Meshac e di Abed-nego, il quale ha mandato il suo angelo, e ha liberato i suoi servi, che hanno confidato in lui, hanno trasgredito l'ordine del re, e hanno esposto i loro corpi, per non servire e non adorare altro dio che il loro!
\par 29 Perciò, io faccio questo decreto: che chiunque, a qualsiasi popolo, nazione o lingua appartenga, dirà male dell'Iddio di Shadrac, Meshac e Abed-nego, sia fatto a pezzi, e la sua casa sia ridotta in un immondezzaio; perché non v'è alcun altro dio che possa salvare a questo modo'.
\par 30 Allora il re fece prosperare Shadrac, Meshac e Abed-nego nella provincia di Babilonia.

\chapter{4}

\par 1 'Il re Nebucadnetsar a tutti i popoli, a tutte le nazioni e lingue, che abitano su tutta la terra. La vostra pace abbondi.
\par 2 M'è parso bene di far conoscere i segni e i prodigi che l'Iddio altissimo ha fatto nella mia persona.
\par 3 Come son grandi i suoi segni! Come son potenti i suoi prodigi! Il suo regno è un regno eterno, e il suo dominio dura di generazione in generazione.
\par 4 Io, Nebucadnetsar, stavo tranquillo in casa mia, e fiorente nel mio palazzo.
\par 5 Ebbi un sogno, che mi spaventò; e i pensieri che m'assalivano sul mio letto, e le visioni del mio spirito m'empiron di terrore.
\par 6 Ordine fu dato da parte mia di condurre davanti a me tutti i savi di Babilonia, perché mi facessero conoscere l'interpretazione del sogno.
\par 7 Allora vennero i magi, gl'incantatori, i Caldei e gli astrologi; io dissi loro il sogno, ma essi non poterono farmene conoscere l'interpretazione.
\par 8 Alla fine si presentò davanti a me Daniele, che si chiama Beltsatsar, dal nome del mio dio, e nel quale è lo spirito degli dèi santi; e io gli raccontai il sogno: -
\par 9 Beltsatsar, capo de' magi, siccome io so che lo spirito degli dèi santi è in te, e che nessun segreto t'è difficile, dimmi le visioni che ho avuto nel mio sogno, e la loro interpretazione.
\par 10 Ed ecco le visioni della mia mente quand'ero sul mio letto. Io guardavo, ed ecco un albero in mezzo alla terra, la cui altezza era grande.
\par 11 L'albero era cresciuto e diventato forte, e la sua vetta giungeva al cielo, e lo si vedeva dalle estremità di tutta la terra.
\par 12 Il suo fogliame era bello, il suo frutto abbondante, c'era in lui nutrimento per tutti; le bestie de' campi si riparavano sotto la sua ombra, gli uccelli del cielo dimoravano fra i suoi rami, e ogni creatura si nutriva d'esso.
\par 13 Nelle visioni della mia mente, quand'ero sul mio letto, io guardavo, ed ecco uno dei santi Veglianti scese dal cielo,
\par 14 gridò con forza, e disse così: - Abbattete l'albero, e tagliatene i rami; scotétene il fogliame, e dispergetene il frutto; fuggano gli animali di sotto a lui, e gli uccelli di tra i suoi rami!
\par 15 Però, lasciate in terra il ceppo delle sue radici, ma in catene di ferro e di rame, fra l'erba de' campi; e sia bagnato dalla rugiada del cielo, e abbia con gli animali la sua parte d'erba della terra.
\par 16 Gli sia mutato il cuore; e invece d'un cuor d'uomo, gli sia dato un cuore di bestia; e passino su di lui sette tempi.
\par 17 La cosa è decretata dai Veglianti, e la sentenza emana dai santi, affinché i viventi conoscano che l'Altissimo domina sul regno degli uomini, ch'egli lo dà a chi vuole, e vi innalza l'infimo degli uomini.
\par 18 Questo è il sogno che io, il re Nebucadnetsar, ho fatto; e tu, Beltsatsar, danne l'interpretazione, giacché tutti i savi del mio regno non me lo possono interpretare; ma tu puoi, perché lo spirito degli dèi santi è in te'. -
\par 19 Allora Daniele, il cui nome è Beltsatsar, rimase per un momento stupefatto, e i suoi pensieri lo spaventavano. Il re prese a dire: 'Beltsatsar, il sogno e la interpretazione non ti spaventino!' Beltsatsar rispose, e disse: 'Signor mio, il sogno s'avveri per i tuoi nemici, e la sua interpretazione per i tuoi avversari!
\par 20 L'albero che il re ha visto, ch'era divenuto grande e forte, la cui vetta giungeva al cielo e che si vedeva da tutti i punti della terra,
\par 21 l'albero dal fogliame bello, dal frutto abbondante e in cui era nutrimento per tutti, sotto il quale si riparavano le bestie dei campi e fra i cui rami dimoravano gli uccelli del cielo,
\par 22 sei tu, o re; tu, che sei divenuto grande e forte, la cui grandezza s'è accresciuta e giunge fino al cielo, e il cui dominio s'estende fino alle estremità della terra.
\par 23 E quanto al santo Vegliante che hai visto scendere dal cielo e che ha detto: - Abbattete l'albero e distruggetelo, ma lasciatene in terra il ceppo delle radici, in catene di ferro e di rame, fra l'erba de' campi, e sia bagnato dalla rugiada del cielo, e abbia la sua parte con gli animali della campagna finché sian passati sopra di lui sette tempi -
\par 24 eccone l'interpretazione, o re; è un decreto dell'Altissimo, che sarà eseguito sul re mio signore:
\par 25 tu sarai cacciato di fra gli uomini e la tua dimora sarà con le bestie dei campi; ti sarà data a mangiare dell'erba come ai buoi; sarai bagnato dalla rugiada del cielo, e passeranno su di te sette tempi, finché tu non riconosca che l'Altissimo domina sul regno degli uomini, e lo dà a chi vuole.
\par 26 E quanto all'ordine di lasciare il ceppo delle radici dell'albero, ciò significa che il tuo regno ti sarà ristabilito, dopo che avrai riconosciuto che il cielo domina.
\par 27 Perciò, o re, ti sia gradito il mio consiglio! Poni fine ai tuoi peccati con la giustizia, e alle tue iniquità con la compassione verso gli afflitti; e, forse, la tua prosperità potrà esser prolungata'.
\par 28 Tutto questo avvenne al re Nebucadnetsar.
\par 29 In capo a dodici mesi egli passeggiava sul palazzo reale di Babilonia.
\par 30 Il re prese a dire: 'Non è questa la gran Babilonia che io ho edificata come residenza reale con la forza della mia potenza e per la gloria della mia maestà?'
\par 31 Il re aveva ancora la parola in bocca, quando una voce discese dal cielo: 'Sappi, o re Nebucadnetsar, che il tuo regno t'è tolto;
\par 32 e tu sarai cacciato di fra gli uomini, la tua dimora sarà con le bestie dei campi; ti sarà data a mangiare dell'erba come ai buoi, e passeranno su di te sette tempi, finché tu non riconosca che l'Altissimo domina sul regno degli uomini e lo dà a chi vuole'.
\par 33 In quel medesimo istante quella parola si adempì su Nebucadnetsar. Egli fu cacciato di fra gli uomini, mangiò l'erba come i buoi, e il suo corpo fu bagnato dalla rugiada del cielo, finché il pelo gli crebbe come le penne alle aquile, e le unghie come agli uccelli.
\par 34 'Alla fine di que' giorni, io, Nebucadnetsar, alzai gli occhi al cielo, la ragione mi tornò, e benedissi l'Altissimo, e lodai e glorificai colui che vive in eterno, il cui dominio è un dominio perpetuo, e il cui regno dura di generazione in generazione.
\par 35 Tutti gli abitanti della terra son da lui reputati un nulla; egli agisce come vuole con l'esercito del cielo e con gli abitanti della terra; e non v'è alcuno che possa fermare la sua mano o dirgli: - Che fai? -
\par 36 In quel tempo la ragione mi tornò; la gloria del mio regno, la mia maestà, il mio splendore mi furono restituiti; i miei consiglieri e i miei grandi mi cercarono, e io fui ristabilito nel mio regno, e la mia grandezza fu accresciuta più che mai.
\par 37 Ora, io, Nebucadnetsar, lodo, esalto e glorifico il Re del cielo, perché tutte le sue opere sono verità, e le sue vie, giustizia, ed egli ha il potere di umiliare quelli che camminano superbamente'.

\chapter{5}

\par 1 Il re Belsatsar fece un gran convito a mille de' suoi grandi; e bevve del vino in presenza dei mille.
\par 2 Belsatsar, mentre stava assaporando il vino, ordinò che si recassero i vasi d'oro e d'argento che Nebucadnetsar suo padre aveva portati via dal tempio di Gerusalemme, perché il re, i suoi grandi, le sue mogli e le sue concubine se ne servissero per bere.
\par 3 Allora furon recati i vasi d'oro ch'erano stati portati via dal tempio, dalla casa di Dio, ch'era in Gerusalemme; e il re, i suoi grandi, le sue mogli e le sue concubine se ne servirono per bere.
\par 4 Bevvero del vino, e lodarono gli dèi d'oro, d'argento, di rame, di ferro, di legno e di pietra.
\par 5 In quel momento, apparvero delle dita d'una mano d'uomo, che si misero a scrivere, difaccia al candelabro, sull'intonaco della parete del palazzo reale. E il re vide quel mozzicone di mano che scriveva.
\par 6 Allora il re mutò di colore, e i suoi pensieri lo spaventarono; le giunture de' suoi fianchi si rilassarono, e i suoi ginocchi cominciarono a urtarsi l'uno contro l'altro.
\par 7 Il re gridò forte che si facessero entrare gl'incantatori, i Caldei e gli astrologi; e il re prese a dire ai savi di Babilonia: 'Chiunque leggerà questo scritto e me ne darà l'interpretazione sarà rivestito di porpora, avrà al collo una collana d'oro, e sarà terzo nel governo del regno'.
\par 8 Allora entrarono tutti i savi del re; ma non poteron leggere lo scritto, né darne al re l'interpretazione.
\par 9 Allora il re Belsatsar fu preso da grande spavento, mutò di colore, e i suoi grandi furono costernati.
\par 10 La regina, com'ebbe udite le parole del re e dei suoi grandi, entrò nella sala del convito. La regina prese a dire: 'O re, possa tu vivere in perpetuo! I tuoi pensieri non ti spaventino, e non mutar di colore!
\par 11 C'è un uomo nel tuo regno, in cui è lo spirito degli dèi santi; e al tempo di tuo padre si trovò in lui una luce, un intelletto e una sapienza, pari alla sapienza degli dèi; e il re Nebucadnetsar tuo padre, il padre tuo, o re, lo stabilì capo dei magi, degli incantatori, de' Caldei e degli astrologi,
\par 12 perché in lui, in questo Daniele, a cui il re avea posto nome Beltsatsar, fu trovato uno spirito straordinario, conoscenza, intelletto, facoltà di interpretare i sogni, di spiegare enigmi, e di risolvere questioni difficili. Si chiami dunque Daniele ed egli darà l'interpretazione'.
\par 13 Allora Daniele fu introdotto alla presenza del re; e il re parlò a Daniele, e gli disse: 'Sei tu Daniele, uno de' Giudei che il re mio padre menò in cattività da Giuda?
\par 14 Io ho sentito dire di te che lo spirito degli dèi è in te, e che in te si trova luce, intelletto, e una sapienza straordinaria.
\par 15 Ora, i savi e gl'incantatori sono stati introdotti alla mia presenza, per leggere questo scritto e per farmene conoscere l'interpretazione; ma non han potuto darmi l'interpretazione della cosa.
\par 16 Però, ho sentito dire di te che tu puoi dare interpretazioni e risolvere questioni difficili; ora, se puoi leggere questo scritto e farmene conoscere l'interpretazione, tu sarai rivestito di porpora, avrai al collo una collana d'oro, e sarai terzo nel governo del regno'.
\par 17 Allora Daniele prese a dire in presenza del re: 'Tienti i tuoi doni, e da' a un altro le tue ricompense; nondimeno io leggerò lo scritto al re e gliene farò conoscere l'interpretazione.
\par 18 O re, l'Iddio altissimo avea dato a Nebucadnetsar tuo padre, regno, grandezza, gloria e maestà;
\par 19 e a motivo della grandezza ch'Egli gli aveva dato, tutti i popoli, tutte le nazioni e lingue temevano e tremavano alla sua presenza; egli faceva morire chi voleva, lasciava in vita chi voleva; innalzava chi voleva, abbassava chi voleva.
\par 20 Ma quando il suo cuore divenne altero e il suo spirito s'indurò fino a diventare arrogante, fu deposto dal suo trono reale, e gli fu tolta la sua gloria;
\par 21 fu cacciato di tra i figliuoli degli uomini, il suo cuore fu reso simile a quello delle bestie, e la sua dimora fu con gli asini selvatici; gli fu data a mangiare dell'erba come ai buoi, e il suo corpo fu bagnato dalla rugiada del cielo, finché non riconobbe che l'Iddio altissimo domina sul regno degli uomini, e ch'egli vi stabilisce sopra chi vuole.
\par 22 E tu, o Belsatsar, suo figliuolo, non hai umiliato il tuo cuore, quantunque tu sapessi tutto questo;
\par 23 ma ti sei innalzato contro il Signore del cielo; ti sono stati portati davanti i vasi della sua casa, e tu, i tuoi grandi, le tue mogli e le tue concubine ve ne siete serviti per bere; e tu hai lodato gli dèi d'argento, d'oro, di rame, di ferro, di legno e di pietra, i quali non vedono, non odono, non hanno conoscenza di sorta, e non hai glorificato l'Iddio che ha nella sua mano il tuo soffio vitale, e da cui dipendono tutte le tue vie.
\par 24 Perciò è stato mandato, da parte sua, quel mozzicone di mano, che ha tracciato quello scritto.
\par 25 Questo è lo scritto ch'è stato tracciato: MENE, MENE, TEKEL, UFARSIN.
\par 26 E questa è l'interpretazione delle parole: MENE: Dio ha fatto il conto del tuo regno, e vi ha posto fine.
\par 27 TEKEL: tu sei stato pesato con la bilancia e sei stato trovato mancante.
\par 28 PERES: il tuo regno è diviso, e dato ai Medi e ai Persiani'.
\par 29 Allora, per ordine di Belsatsar, Daniele fu rivestito di porpora, gli fu messa al collo una collana d'oro, e fu proclamato che egli sarebbe terzo nel governo del regno.
\par 30 In quella stessa notte, Belsatsar, re de' Caldei, fu ucciso;
\par 31 e Dario, il Medo, ricevette il regno, all'età di sessantadue anni.

\chapter{6}

\par 1 Parve bene a Dario di stabilire sul regno centoventi satrapi, i quali fossero per tutto il regno;
\par 2 e sopra questi, tre capi, uno de' quali era Daniele, perché questi satrapi rendessero loro conto, e il re non avesse a soffrire alcun danno.
\par 3 Or questo Daniele si distingueva più dei capi e dei satrapi, perché c'era in lui uno spirito straordinario; e il re pensava di stabilirlo sopra tutto il regno.
\par 4 Allora i capi e i satrapi cercarono di trovare un'occasione d'accusar Daniele circa l'amministrazione del regno; ma non potevano trovare alcuna occasione, né alcun motivo di riprensione, perch'egli era fedele, e non c'era da trovare in lui alcunché di male o da riprendere.
\par 5 Quegli uomini dissero dunque: 'Noi non troveremo occasione alcuna d'accusar questo Daniele, se non la troviamo in quel che concerne la legge del suo Dio'.
\par 6 Allora quei capi e quei satrapi vennero tumultuosamente presso al re, e gli dissero: 'O re Dario, possa tu vivere in perpetuo!
\par 7 Tutti i capi del regno, i prefetti e i satrapi, i consiglieri e i governatori si sono concertati perché il re promulghi un decreto e pubblichi un severo divieto, per i quali, chiunque, entro lo spazio di trenta giorni, rivolgerà qualche richiesta a qualsivoglia dio o uomo tranne che a te, o re, sia gettato nella fossa de' leoni.
\par 8 Ora, o re, promulga il divieto e firmane l'atto perché sia immutabile, conformemente alla legge dei Medi e de' Persiani, che è irrevocabile'.
\par 9 Il re Dario quindi firmò il decreto e il divieto.
\par 10 E quando Daniele seppe che il decreto era firmato, entrò in casa sua; e, tenendo le finestre della sua camera superiore aperte verso Gerusalemme, tre volte al giorno si metteva in ginocchi, pregava e rendeva grazie al suo Dio, come soleva fare per l'addietro.
\par 11 Allora quegli uomini accorsero tumultuosamente, e trovaron Daniele che faceva richieste e supplicazioni al suo Dio.
\par 12 Poi s'accostarono al re, e gli parlarono del divieto reale: 'Non hai tu firmato un divieto, per il quale chiunque entro lo spazio di trenta giorni farà qualche richiesta a qualsivoglia dio o uomo tranne che a te, o re, dev'esser gettato nella fossa de' leoni?' Il re rispose e disse: 'La cosa è stabilita, conformemente alla legge dei Medi e dei Persiani, che è irrevocabile'.
\par 13 Allora quelli ripresero a dire in presenza del re: 'Daniele, che è fra quelli che sono stati menati in cattività da Giuda, non tiene in alcun conto né te, o re, né il divieto che tu hai firmato, ma prega il suo Dio tre volte al giorno'.
\par 14 Quand'ebbe udito questo, il re ne fu dolentissimo, e si mise in cuore di liberar Daniele; e fino al tramonto del sole fece di tutto per salvarlo.
\par 15 Ma quegli uomini vennero tumultuosamente al re, e gli dissero: 'Sappi, o re, che è legge dei Medi e de' Persiani che nessun divieto o decreto promulgato dal re possa essere mutato'.
\par 16 Allora il re diede l'ordine, e Daniele fu menato e gettato nella fossa de' leoni. E il re parlò a Daniele, e gli disse: 'L'Iddio tuo, che tu servi del continuo, sarà quegli che ti libererà'.
\par 17 E fu portata una pietra, che fu messa sulla bocca della fossa; e il re la sigillò col suo anello e con l'anello de' suoi grandi, perché nulla fosse mutato riguardo a Daniele.
\par 18 Allora il re se ne andò al suo palazzo, e passò la notte in digiuno; non si fece venir alcuna concubina e il sonno fuggì da lui.
\par 19 Poi il re si levò la mattina di buon'ora, appena fu giorno, e si recò in fretta alla fossa de' leoni.
\par 20 E come fu vicino alla fossa, chiamò Daniele con voce dolorosa, e il re prese a dire a Daniele: 'Daniele, servo dell'Iddio vivente! Il tuo Dio, che tu servi del continuo, t'ha egli potuto liberare dai leoni?'
\par 21 Allora Daniele disse al re: 'O re, possa tu vivere in perpetuo!
\par 22 Il mio Dio ha mandato il suo angelo, e ha chiuso la bocca de' leoni che non m'hanno fatto alcun male, perché io sono stato trovato innocente nel suo cospetto; e anche davanti a te, o re, non ho fatto alcun male'.
\par 23 Allora il re fu ricolmo di gioia, e ordinò che Daniele fosse tratto fuori dalla fossa; e Daniele fu tratto fuori dalla fossa, e non si trovò su di lui lesione di sorta, perché s'era confidato nel suo Dio.
\par 24 E per ordine del re furon menati quegli uomini che avevano accusato Daniele, e furon gettati nella fossa de' leoni, essi, i loro figliuoli e le loro mogli; e non erano ancora giunti in fondo alla fossa, che i leoni furono loro addosso, e fiaccaron loro tutte le ossa.
\par 25 Allora il re Dario scrisse a tutti i popoli, a tutte le nazioni e lingue che abitavano su tutta la terra: 'La vostra pace abbondi!
\par 26 Io decreto che in tutto il dominio del mio regno si tema e si tremi nel cospetto dell'Iddio di Daniele; poich'Egli è l'Iddio vivente, che sussiste in eterno; il suo regno non sarà mai distrutto, e il suo dominio durerà sino alla fine.
\par 27 Egli libera e salva, e opera segni e prodigi in cielo e in terra; Egli è quei che ha liberato Daniele dalle branche dei leoni'.
\par 28 E questo Daniele prosperò sotto il regno di Dario, e sotto il regno di Ciro, il Persiano.

\chapter{7}

\par 1 Il primo anno di Belsatsar, re di Babilonia, Daniele, mentr'era a letto, fece un sogno, ed ebbe delle visioni nella sua mente. Poi scrisse il sogno, e narrò la sostanza delle cose.
\par 2 Daniele dunque prese a dire: Io guardavo, nella mia visione notturna, ed ecco scatenarsi sul mar grande i quattro venti del cielo.
\par 3 E quattro grandi bestie salirono dal mare, una diversa dall'altra.
\par 4 La prima era come un leone, ed avea delle ali d'aquila. Io guardai, finché non le furono strappate le ali; e fu sollevata da terra, fu fatta stare in piedi come un uomo, e le fu dato un cuor d'uomo.
\par 5 Ed ecco una seconda bestia, simile ad un orso; essa rizzavasi sopra un lato, avea tre costole in bocca fra i denti; e le fu detto: 'Lèvati, mangia molta carne!'
\par 6 Dopo questo, io guardavo, ed eccone un'altra simile ad un leopardo, che aveva addosso quattro ali d'uccello; questa bestia avea quattro teste, e le fu dato il dominio.
\par 7 Dopo questo, io guardavo, nelle visioni notturne, ed ecco una quarta bestia spaventevole, terribile e straordinariamente forte; aveva dei denti grandi, di ferro; divorava e sbranava, e calpestava il resto coi piedi; era diversa da tutte le bestie che l'avevano preceduta, e aveva dieci corna.
\par 8 Io esaminavo quelle corna, ed ecco che un altro piccolo corno spuntò tra quelle, e tre delle prime corna furono divelte dinanzi ad esso; ed ecco che quel corno avea degli occhi simili a occhi d'uomo, e una bocca che proferiva grandi cose.
\par 9 Io continuai a guardare fino al momento in cui furono collocati de' troni, e un vegliardo s'assise. La sua veste era bianca come la neve, e i capelli del suo capo eran come lana pura; fiamme di fuoco erano il suo trono e le ruote d'esso erano fuoco ardente.
\par 10 Un fiume di fuoco sgorgava e scendeva dalla sua presenza; mille migliaia lo servivano, e diecimila miriadi gli stavan davanti. Il giudizio si tenne, e i libri furono aperti.
\par 11 Allora io guardai a motivo delle parole orgogliose che il corno proferiva; guardai, finché la bestia non fu uccisa, e il suo corpo distrutto, gettato nel fuoco per esser arso.
\par 12 Quanto alle altre bestie, il dominio fu loro tolto; ma fu loro concesso un prolungamento di vita per un tempo determinato.
\par 13 Io guardavo, nelle visioni notturne, ed ecco venire sulle nuvole del cielo uno simile a un figliuol d'uomo; egli giunse fino al vegliardo, e fu fatto accostare a lui.
\par 14 E gli furon dati dominio, gloria e regno, perché tutti i popoli, tutte le nazioni e lingue lo servissero; il suo dominio è un dominio eterno che non passerà, e il suo regno, un regno che non sarà distrutto.
\par 15 Quanto a me, Daniele, il mio spirito fu turbato dentro di me, e le visioni della mia mente mi spaventarono.
\par 16 M'accostai a uno degli astanti, e gli domandai la verità intorno a tutto questo; ed egli mi parlò, e mi dette l'interpretazione di quelle cose:
\par 17 'Queste quattro grandi bestie, sono quattro re che sorgeranno dalla terra;
\par 18 poi i santi dell'Altissimo riceveranno il regno e lo possederanno per sempre, d'eternità in eternità'.
\par 19 Allora desiderai saper la verità intorno alla quarta bestia, ch'era diversa da tutte le altre, straordinariamente terribile, che aveva i denti di ferro e le unghie di rame, che divorava, sbranava, e calpestava il resto coi piedi,
\par 20 e intorno alle dieci corna che aveva in capo, e intorno all'altro corno che spuntava, e davanti al quale tre erano cadute: a quel corno che avea degli occhi, e una bocca proferente cose grandi, e che appariva maggiore delle altre corna.
\par 21 Io guardai, e quello stesso corno faceva guerra ai santi e aveva il sopravvento,
\par 22 finché non giunse il vegliardo e il giudicio fu dato ai santi dell'Altissimo, e venne il tempo che i santi possederono il regno.
\par 23 Ed egli mi parlò così: 'La quarta bestia è un quarto regno sulla terra, che differirà da tutti i regni, divorerà tutta la terra, la calpesterà e la frantumerà.
\par 24 Le dieci corna sono dieci re che sorgeranno da questo regno; e, dopo quelli, ne sorgerà un altro, che sarà diverso dai precedenti, e abbatterà tre re.
\par 25 Egli proferirà parole contro l'Altissimo, ridurrà allo stremo i santi dell'Altissimo, e penserà di mutare i tempi e la legge; i santi saran dati nelle sue mani per un tempo, dei tempi, e la metà d'un tempo.
\par 26 Poi si terrà il giudizio e gli sarà tolto il dominio, che verrà distrutto ed annientato per sempre.
\par 27 E il regno e il dominio e la grandezza dei regni che sono sotto tutti i cieli saranno dati al popolo dei santi dell'Altissimo; il suo regno è un regno eterno, e tutti i domini lo serviranno e gli ubbidiranno'.
\par 28 Qui finirono le parole rivoltemi. Quanto a me, Daniele, i miei pensieri mi spaventarono molto, e mutai di colore; ma serbai la cosa nel cuore.

\chapter{8}

\par 1 Il terzo anno del regno del re Belsatsar, io, Daniele, ebbi una visione, dopo quella che avevo avuta al principio del regno.
\par 2 Ero in visione; e, mentre guardavo, ero a Susan, la residenza reale, che è nella provincia di Elam; e, nella visione, mi trovavo presso il fiume Ulai.
\par 3 Alzai gli occhi, guardai, ed ecco, ritto davanti al fiume, un montone che aveva due corna; e le due corna erano alte, ma una era più alta dell'altra, e la più alta veniva sull'ultima.
\par 4 Vidi il montone che cozzava a occidente, a settentrione e a mezzogiorno; nessuna bestia gli poteva tener fronte, e non c'era nessuno che la potesse liberare dalla sua potenza; esso faceva quel che voleva, e diventò grande.
\par 5 E com'io stavo considerando questo, ecco venire dall'occidente un capro, che percorreva tutta la superficie della terra senza toccare il suolo; e questo capro aveva un corno cospicuo fra i suoi occhi.
\par 6 Esso venne fino al montone dalle due corna che avevo visto ritto davanti al fiume, e gli s'avventò contro, nel furore della sua forza.
\par 7 E lo vidi giungere vicino al montone, pieno di rabbia contro di lui, investirlo, e spezzargli le due corna; il montone non ebbe la forza di tenergli fronte, e il capro lo atterrò e lo calpestò; e non ci fu nessuno che potesse liberare il montone dalla potenza d'esso.
\par 8 Il capro diventò sommamente grande; ma, quando fu potente, il suo gran corno si spezzò; e, in luogo di quello, sorsero quattro corna cospicue, verso i quattro venti del cielo.
\par 9 E dall'una d'esse uscì un piccolo corno, che diventò molto grande verso mezzogiorno, verso levante, e verso il paese splendido.
\par 10 S'ingrandì, fino a giungere all'esercito del cielo; fece cadere in terra parte di quell'esercito e delle stelle, e le calpestò.
\par 11 S'elevò anzi fino al capo di quell'esercito, gli tolse il sacrifizio perpetuo, e il luogo del suo santuario fu abbattuto.
\par 12 L'esercito gli fu dato in mano col sacrifizio perpetuo a motivo della ribellione; e il corno gettò a terra la verità, e prosperò nelle sue imprese.
\par 13 Poi udii un santo che parlava; e un altro santo disse a quello che parlava: 'Fino a quando durerà la visione del sacrifizio continuo e la ribellione che produce la desolazione, abbandonando il luogo santo e l'esercito ad esser calpestati?'
\par 14 Egli mi disse: 'Fino a duemilatrecento sere e mattine; poi il santuario sarà purificato'.
\par 15 E avvenne che, mentre io, Daniele, avevo questa visione e cercavo d'intenderla, ecco starmi ritta davanti come una figura d'uomo.
\par 16 E udii la voce d'un uomo in mezzo all'Ulai, che gridò, e disse: 'Gabriele, spiega a colui la visione'.
\par 17 Ed esso venne presso al luogo dove io stavo; alla sua venuta io fui spaventato, e caddi sulla mia faccia; ma egli mi disse: 'Intendi bene, o figliuol d'uomo! perché questa visione concerne il tempo della fine'.
\par 18 E com'egli mi parlava, io mi lasciai andare con la faccia a terra, profondamente assopito; ma egli mi toccò, e mi fece stare in piedi.
\par 19 E disse: 'Ecco, io ti farò conoscere quello che avverrà nell'ultimo tempo dell'indignazione; poiché si tratta del tempo fissato per la fine.
\par 20 Il montone con due corna che hai veduto, rappresenta i re di Media e di Persia.
\par 21 Il becco peloso è il re di Grecia; e il gran corno fra i suoi due occhi è il primo re.
\par 22 Quanto al corno spezzato, al cui posto ne son sorti quattro, questi sono quattro regni che sorgeranno da questa nazione, ma non con la stessa sua potenza.
\par 23 E alla fine del loro regno, quando i ribelli avranno colmato la misura delle loro ribellioni, sorgerà un re dall'aspetto feroce, ed esperto in stratagemmi.
\par 24 La sua potenza sarà grande, ma non sarà potenza sua; egli farà prodigiose ruine, prospererà nelle sue imprese, e distruggerà i potenti e il popolo dei santi.
\par 25 A motivo della sua astuzia farà prosperare la frode nelle sue mani; s'inorgoglirà in cuor suo, e in piena pace distruggerà molta gente; insorgerà contro il principe de' principi, ma sarà infranto, senz'opera di mano.
\par 26 E la visione delle sere e delle mattine, di cui è stato parlato, è vera. Tu tieni segreta la visione, perché si riferisce a un tempo lontano'.
\par 27 E io, Daniele, svenni, e fui malato vari giorni; poi m'alzai, e feci gli affari del re. Io ero stupito della visione, ma nessuno se ne avvide.

\chapter{9}

\par 1 Nell'anno primo di Dario, figliuolo d'Assuero, della stirpe dei Medi, che fu fatto re del regno dei Caldei,
\par 2 il primo anno del suo regno, io, Daniele, meditando sui libri, vidi che il numero degli anni di cui l'Eterno avea parlato al profeta Geremia, e durante i quali Gerusalemme dovea essere in ruine, era di settant'anni.
\par 3 E volsi la mia faccia verso il Signore Iddio, per dispormi alla preghiera e alle supplicazioni, col digiuno, col sacco e con la cenere.
\par 4 E feci la mia preghiera e la mia confessione all'Eterno, al mio Dio, dicendo: 'O Signore, Dio grande e tremendo, che mantieni il patto e continui la benignità a quelli che t'amano e osservano i tuoi comandamenti!
\par 5 Noi abbiamo peccato, ci siam condotti iniquamente, abbiamo operato malvagiamente, ci siamo ribellati, e ci siamo allontanati dai tuoi comandamenti e dalle tue prescrizioni,
\par 6 non abbiam dato ascolto ai profeti, tuoi servi, che hanno parlato in tuo nome ai nostri re, ai nostri capi, ai nostri padri, e a tutto il popolo del paese.
\par 7 A te, o Signore, la giustizia; a noi, la confusione della faccia, come avviene al dì d'oggi: agli uomini di Giuda, agli abitanti di Gerusalemme e a tutto Israele, vicini e lontani, in tutti i paesi dove li hai cacciati, a motivo delle infedeltà che hanno commesse contro di te.
\par 8 O Signore, a noi la confusione della faccia, ai nostri re, ai nostri capi e ai nostri padri, perché abbiam peccato contro di te.
\par 9 Al Signore, ch'è il nostro Dio, appartengono la misericordia e il perdono; poiché noi ci siamo ribellati a lui,
\par 10 e non abbiam dato ascolto alla voce dell'Eterno, dell'Iddio nostro, per camminare secondo le sue leggi, ch'egli ci avea poste dinanzi mediante i profeti suoi servi.
\par 11 Sì, tutto Israele ha trasgredito la tua legge, s'è sviato per non ubbidire alla tua voce; e così su noi si son riversate le maledizioni e imprecazioni che sono scritte nella legge di Mosè, servo di Dio, perché noi abbiam peccato contro di lui.
\par 12 Ed egli ha mandato ad effetto le parole che aveva pronunziate contro di noi e contro i nostri giudici che ci governano, facendo venir su noi una calamità così grande, che sotto tutto il cielo nulla mai è stato fatto di simile a quello ch'è stato fatto a Gerusalemme.
\par 13 Com'è scritto nella legge di Mosè, tutta questa calamità ci è venuta addosso; e, nondimeno, non abbiamo implorato il favore dell'Eterno, del nostro Dio, ritraendoci dalle nostre iniquità e rendendoci attenti alla sua verità.
\par 14 E l'Eterno ha vegliato su questa calamità, e ce l'ha fatta venire addosso; perché l'Eterno, il nostro Dio, è giusto in tutto quello che ha fatto, ma noi non abbiamo ubbidito alla sua voce.
\par 15 Ed ora, o Signore, Iddio nostro, che traesti il tuo popolo fuori del paese d'Egitto con mano potente, e ti facesti il nome che hai oggi, noi abbiamo peccato, abbiamo operato malvagiamente.
\par 16 O Signore, secondo tutte le tue opere di giustizia, fa', ti prego, che la tua ira e il tuo furore si ritraggano dalla tua città di Gerusalemme, il tuo monte santo; poiché per i nostri peccati e per le iniquità de' nostri padri, Gerusalemme e il tuo popolo sono esposti al vituperio di tutti quelli che ci circondano.
\par 17 Ora dunque, o Dio nostro, ascolta la preghiera del tuo servo e le sue supplicazioni, e fa' risplendere il tuo volto sul tuo desolato santuario, per amor del Signore!
\par 18 O mio Dio, inclina il tuo orecchio ed ascolta; apri gli occhi e guarda le nostre desolazioni, e la città sulla quale è invocato il tuo nome; poiché noi umilmente presentiamo le nostre supplicazioni nel tuo cospetto, fondati non sulle nostre opere giuste, ma sulle tue grandi compassioni.
\par 19 O Signore, ascolta! Signore, perdona! Signore, sii attento ed agisci; non indugiare, per amor di te stesso, o mio Dio, perché il tuo nome è invocato sulla tua città e sul tuo popolo!'
\par 20 Mentre io parlavo ancora, pregando e confessando il mio peccato e il peccato del mio popolo d'Israele, e presentavo la mia supplicazione all'Eterno, al mio Dio, per il monte santo del mio Dio,
\par 21 mentre stavo ancora parlando in preghiera, quell'uomo, Gabriele, che avevo visto nella visione da principio, mandato con rapido volo, s'avvicinò a me, verso l'ora dell'oblazione della sera.
\par 22 E mi ammaestrò, mi parlò, e disse: 'Daniele, io son venuto ora per darti intendimento.
\par 23 Al principio delle tue supplicazioni, una parola è uscita; e io son venuto a comunicartela, poiché tu sei grandemente amato. Fa' dunque attenzione alla parola, e intendi la visione!
\par 24 Settanta settimane son fissate riguardo al tuo popolo e alla tua santa città, per far cessare la trasgressione, per metter fine al peccato, per espiare l'iniquità, e addurre una giustizia eterna, per suggellare visione e profezia, e per ungere un luogo santissimo.
\par 25 Sappilo dunque, e intendi! Dal momento in cui è uscito l'ordine di restaurare e riedificare Gerusalemme fino all'apparire di un unto, di un capo, vi sono sette settimane; e in sessantadue settimane essa sarà restaurata e ricostruita, piazze e mura, ma in tempi angosciosi.
\par 26 Dopo le sessantadue settimane, un unto sarà soppresso, nessuno sarà per lui. E il popolo d'un capo che verrà, distruggerà la città e il santuario; la sua fine verrà come un'inondazione; ed è decretato che vi saranno delle devastazioni sino alla fine della guerra.
\par 27 Egli stabilirà un saldo patto con molti, durante una settimana; e in mezzo alla settimana farà cessare sacrifizio e oblazione; e sulle ali delle abominazioni verrà un devastatore; e questo, finché la completa distruzione, che è decretata, non piombi sul devastatore'.

\chapter{10}

\par 1 Il terzo anno di Ciro, re di Persia, una parola fu rivelata a Daniele, che si chiamava Beltsatsar; e la parola è verace, e predice una gran lotta. Egli capì la parola, ed ebbe l'intelligenza della visione.
\par 2 In quel tempo, io, Daniele, feci cordoglio per tre settimane intere.
\par 3 Non mangiai alcun cibo prelibato, né carne né vino entrarono nella mia bocca, e non mi unsi affatto, sino alla fine delle tre settimane.
\par 4 E il ventiquattresimo giorno del primo mese, come io mi trovavo in riva al gran fiume, che è lo Hiddekel,
\par 5 alzai gli occhi, guardai, ed ecco un uomo, vestito di lino, con attorno ai fianchi una cintura d'oro d'Ufaz.
\par 6 Il suo corpo era come un crisolito, la sua faccia avea l'aspetto della folgore, i suoi occhi eran come fiamme di fuoco, le sue braccia e i suoi piedi parevano terso rame, e il suono della sua voce era come il rumore d'una moltitudine.
\par 7 Io solo, Daniele, vidi la visione; gli uomini ch'erano meco non la videro, ma un gran terrore piombò su loro, e fuggirono a nascondersi.
\par 8 E io rimasi solo, ed ebbi questa grande visione. In me non rimase più forza; il mio viso mutò colore fino a rimanere sfigurato, e non mi restò alcun vigore.
\par 9 Udii il suono delle sue parole; e, all'udire il suono delle sue parole caddi profondamente assopito, con la faccia a terra.
\par 10 Ed ecco, una mano mi toccò, e mi fece stare sulle ginocchia e sulle palme delle mani.
\par 11 E mi disse: 'Daniele, uomo grandemente amato, cerca d'intendere le parole che ti dirò, e rizzati in piedi nel luogo dove sei; perché ora io sono mandato da te'. E quand'egli m'ebbe detta questa parola, io mi rizzai in piedi, tutto tremante.
\par 12 Ed egli mi disse: 'Non temere, Daniele; poiché dal primo giorno che ti mettesti in cuore d'intendere e d'umiliarti nel cospetto del tuo Dio, le tue parole furono udite, e io son venuto a motivo delle tue parole.
\par 13 Ma il capo del regno di Persia m'ha resistito ventun giorni; però ecco, Micael, uno dei primi capi, è venuto in mio soccorso, e io son rimasto là presso i re di Persia.
\par 14 E ora son venuto a farti comprendere ciò che avverrà al tuo popolo negli ultimi giorni; perché è ancora una visione che concerne l'avvenire'.
\par 15 E mentr'egli mi rivolgeva queste parole, io abbassai gli occhi al suolo, e rimasi muto.
\par 16 Ed ecco uno che avea sembianza d'un figliuol d'uomo, mi toccò le labbra. Allora io aprii la bocca, parlai, e dissi a colui che mi stava davanti: 'Signor mio, a motivo di questa visione m'ha colto lo spasimo, e non m'è più rimasto alcun vigore.
\par 17 E come potrebbe questo servo del mio signore parlare a cotesto signor mio? Poiché oramai nessun vigore mi resta, e mi manca fino il respiro'.
\par 18 Allora colui che avea la sembianza d'uomo mi toccò di nuovo, e mi fortificò.
\par 19 E disse: 'O uomo grandemente amato, non temere! La pace sia teco! Sii forte, sii forte'. E quand'egli ebbe parlato meco, io ripresi forza, e dissi: 'Il mio signore parli pure poiché tu m'hai fortificato'.
\par 20 Ed egli disse: 'Sai tu perché io son venuto da te? Ora me ne torno a combattere col capo della Persia; e quand'io uscirò a combattere ecco che verrà il capo di Javan.
\par 21 Ma io ti voglio far conoscere ciò che è scritto nel libro della verità; e non v'è nessuno che mi sostenga contro quelli là

\chapter{11}

\par 1 tranne Micael vostro capo; e io, il primo anno di Dario, il Medo, mi tenni presso di lui per sostenerlo e difenderlo.
\par 2 E ora ti farò conoscere la verità. Ecco, sorgeranno ancora in Persia tre re; poi il quarto diventerà molto più ricco di tutti gli altri; e quando sarà diventato forte per le sue ricchezze, solleverà tutti contro il regno di Javan.
\par 3 Allora sorgerà un re potente, che eserciterà un gran dominio e farà quel che vorrà.
\par 4 Ma quando sarà sorto, il suo regno sarà infranto, e sarà diviso verso i quattro venti del cielo; esso non apparterrà alla progenie di lui, né avrà una potenza pari a quella che aveva lui; giacché il suo regno sarà sradicato e passerà ad altri; non ai suoi eredi.
\par 5 E il re del mezzogiorno diventerà forte; ma uno dei suoi capi diventerà più forte di lui, e dominerà; e il suo dominio sarà potente.
\par 6 E alla fine di vari anni, essi faran lega assieme; e la figliuola del re del mezzogiorno verrà al re del settentrione per fare un accordo; ma essa non potrà conservare la forza del proprio braccio, né quegli e il suo braccio potranno resistere; e lei e quelli che l'hanno condotta, e colui che l'ha generata, e colui che l'ha sostenuta per un tempo, saran dati alla morte.
\par 7 E uno de' rampolli delle sue radici sorgerà a prendere il posto di quello; esso verrà all'esercito, entrerà nelle fortezze del re di settentrione, verrà alle prese con quelli, e rimarrà vittorioso;
\par 8 e menerà anche in cattività in Egitto i loro dèi, con le loro immagini fuse e coi loro preziosi arredi d'argento e d'oro; e per vari anni si terrà lungi dal re del settentrione.
\par 9 E questi marcerà contro il re del mezzogiorno, ma tornerà nel proprio paese.
\par 10 E i suoi figliuoli entreranno in guerra, e raduneranno una moltitudine di grandi forze; l'un d'essi si farà avanti, si spanderà come un torrente, e passerà oltre; poi tornerà e spingerà le ostilità sino alla fortezza del re del mezzogiorno.
\par 11 Il re del mezzogiorno s'inasprirà, si farà innanzi e moverà guerra a lui, al re del settentrione, il quale arrolerà una gran moltitudine; ma quella moltitudine sarà data in mano del re del mezzogiorno.
\par 12 La moltitudine sarà portata via, e il cuore di lui s'inorgoglirà; ma, per quanto ne abbia abbattuto delle diecine di migliaia, non sarà per questo più forte.
\par 13 E il re del settentrione arrolerà di nuovo una moltitudine più numerosa della prima; e in capo a un certo numero d'anni egli si farà avanti con un grosso esercito e con molto materiale.
\par 14 E in quel tempo molti insorgeranno contro il re del mezzogiorno; e degli uomini violenti di fra il tuo popolo insorgeranno per dar compimento alla visione, ma cadranno.
\par 15 E il re del settentrione verrà; innalzerà de' bastioni, e s'impadronirà di una città fortificata; e né le forze del mezzogiorno, né le truppe scelte avran la forza di resistere.
\par 16 E quegli che sarà venuto contro di lui farà ciò che gli piacerà, non essendovi chi possa stargli a fronte; e si fermerà nel paese splendido, il quale sarà interamente in suo potere.
\par 17 Egli si proporrà di venire con le forze di tutto il suo regno, ma farà un accomodamento col re del mezzogiorno; e gli darà la figliuola per distruggergli il regno; ma il piano non riuscirà, e il paese non gli apparterrà.
\par 18 Poi si dirigerà verso le isole, e ne prenderà molte; ma un generale farà cessare l'obbrobrio ch'ei voleva infliggergli, e lo farà ricadere addosso a lui.
\par 19 Poi il re si dirigerà verso le fortezze del proprio paese; ma inciamperà, cadrà, e non lo si troverà più.
\par 20 Poi, in luogo di lui, sorgerà uno che farà passare un esattore di tributi attraverso il paese che è la gloria del regno; ma in pochi giorni sarà distrutto, non nell'ira, né in battaglia.
\par 21 Poi, in luogo suo, sorgerà un uomo spregevole, a cui non sarà stata conferita la maestà reale; ma verrà senza rumore, e s'impadronirà del regno a forza di lusinghe.
\par 22 E le forze che inonderanno il paese saranno sommerse davanti a lui, saranno infrante, come pure un capo dell'alleanza.
\par 23 E, nonostante la lega fatta con quest'ultimo, agirà con frode, salirà, e diverrà vittorioso con poca gente.
\par 24 E, senza rumore, invaderà le parti più grasse della provincia, e farà quello che non fecero mai né i suoi padri, né i padri dei suoi padri: distribuirà bottino, spoglie e beni e mediterà progetti contro le fortezze; questo, per un certo tempo.
\par 25 Poi raccoglierà le sue forze e il suo coraggio contro il re del mezzogiorno, mediante un grande esercito. E il re del mezzogiorno s'impegnerà in guerra con un grande e potentissimo esercito; ma non potrà tener fronte, perché si faranno delle macchinazioni contro di lui.
\par 26 Quelli che mangeranno alla sua mensa saranno la sua rovina, il suo esercito si dileguerà come un torrente, e molti cadranno uccisi.
\par 27 E quei due re cercheranno in cuor loro di farsi del male; e, alla stessa mensa, si diranno delle menzogne; ma ciò non riuscirà, perché la fine non verrà che al tempo fissato.
\par 28 E quegli tornerà al suo paese con grandi ricchezze; il suo cuore formerà dei disegni contro al patto santo, ed egli li eseguirà, poi tornerà al suo paese.
\par 29 Al tempo stabilito, egli marcerà di nuovo contro il mezzogiorno; ma quest'ultima volta la cosa non riuscirà come la prima;
\par 30 poiché delle navi di Kittim moveranno contro di lui; ed egli si perderà d'animo; poi di nuovo s'indignerà contro il patto santo, ed eseguirà i suoi disegni, e tornerà ad intendersi con quelli che avranno abbandonato il patto santo.
\par 31 Delle forze mandate da lui si presenteranno e profaneranno il santuario, la fortezza, sopprimeranno il sacrifizio continuo, e vi collocheranno l'abominazione, che cagiona la desolazione.
\par 32 E per via di lusinghe corromperà quelli che agiscono empiamente contro il patto; ma il popolo di quelli che conoscono il loro Dio mostrerà fermezza, e agirà.
\par 33 E i savi fra il popolo ne istruiranno molti; ma saranno abbattuti dalla spada e dal fuoco, dalla cattività e dal saccheggio, per un certo tempo.
\par 34 E quando saranno così abbattuti, saran soccorsi con qualche piccolo aiuto; ma molti s'uniranno a loro con finti sembianti.
\par 35 E di que' savi ne saranno abbattuti alcuni, per affinarli, per purificarli e per imbiancarli sino al tempo della fine, perché questa non avverrà che al tempo stabilito.
\par 36 E il re agirà a suo talento, si estollerà, si magnificherà al disopra d'ogni dio, e proferirà cose inaudite contro l'Iddio degli dèi; prospererà finché l'indignazione sia esaurita; poiché quello ch'è decretato si compirà.
\par 37 Egli non avrà riguardo agli dèi de' suoi padri; non avrà riguardo né alla divinità favorita delle donne, né ad alcun dio, perché si magnificherà al disopra di tutti.
\par 38 Ma onorerà l'iddio delle fortezze nel suo luogo di culto; onorerà con oro, con argento, con pietre preziose e con oggetti di valore un dio che i suoi padri non conobbero.
\par 39 E agirà contro le fortezze ben munite, aiutato da un dio straniero; quelli che lo riconosceranno egli ricolmerà di gloria, li farà dominare su molti, e spartirà fra loro delle terre come ricompense.
\par 40 E al tempo della fine, il re del mezzogiorno verrà a cozzo con lui; e il re del settentrione gli piomberà addosso come la tempesta, con carri e cavalieri, e con molte navi; penetrerà ne' paesi e, tutto inondando, passerà oltre.
\par 41 Entrerà pure nel paese splendido, e molte popolazioni saranno abbattute; ma queste scamperanno dalle sue mani: Edom, Moab e la parte principale de' figliuoli di Ammon.
\par 42 Egli stenderà la mano anche su diversi paesi, e il paese d'Egitto non scamperà.
\par 43 E s'impadronirà de' tesori d'oro e d'argento, e di tutte le cose preziose dell'Egitto; e i Libi e gli Etiopi saranno al suo séguito.
\par 44 Ma notizie dall'oriente e dal settentrione lo spaventeranno; ed egli partirà con gran furore, per distruggere e votare allo sterminio molti.
\par 45 E pianterà le tende del suo palazzo fra i mari e il bel monte santo; poi giungerà alla sua fine, e nessuno gli darà aiuto.

\chapter{12}

\par 1 E in quel tempo sorgerà Micael, il gran capo, il difensore de' figliuoli del tuo popolo; e sarà un tempo d'angoscia, quale non se n'ebbe mai da quando esiston nazioni fino a quell'epoca; e in quel tempo, il tuo popolo sarà salvato; tutti quelli, cioè, che saran trovati iscritti nel libro.
\par 2 E molti di coloro che dormono nella polvere della terra si risveglieranno: gli uni per la vita eterna, gli altri per l'obbrobrio, per una eterna infamia.
\par 3 E i savi risplenderanno come lo splendore della distesa, e quelli che ne avranno condotti molti alla giustizia, risplenderanno come le stelle, in sempiterno.
\par 4 E tu, Daniele, tieni nascoste queste parole, e sigilla il libro sino al tempo della fine; molti lo studieranno con cura, e la conoscenza aumenterà'.
\par 5 Poi, io, Daniele, guardai, ed ecco due altri uomini in piedi: l'uno di qua sulla sponda del fiume,
\par 6 e l'altro di là, sull'altra sponda del fiume. E l'un d'essi disse all'uomo vestito di lino, che stava sopra le acque del fiume: 'Quando sarà la fine di queste maraviglie?'
\par 7 E io udii l'uomo vestito di lino, che stava sopra le acque del fiume, il quale, alzata la man destra e la man sinistra al cielo, giurò per colui che vive in eterno, che ciò sarà per un tempo, per dei tempi e per la metà d'un tempo; e quando la forza del popolo santo sarà interamente infranta, allora tutte queste cose si compiranno.
\par 8 E io udii, ma non compresi; e dissi: 'Signor mio, qual sarà la fine di queste cose?'
\par 9 Ed egli rispose: 'Va', Daniele; poiché queste parole son nascoste e sigillate sino al tempo della fine.
\par 10 Molti saranno purificati, imbiancati, affinati; ma gli empi agiranno empiamente, e nessuno degli empi capirà, ma capiranno i savi.
\par 11 E dal tempo che sarà soppresso il sacrifizio continuo e sarà rizzata l'abominazione che cagiona la desolazione, vi saranno milleduecentonovanta giorni.
\par 12 Beato chi aspetta e giunge a milletrecentotrentacinque giorni!
\par 13 Ma tu avviati verso la fine; tu ti riposerai, e poi sorgerai per ricevere la tua parte d'eredità alla fine de' giorni'.


\end{document}