\begin{document}

\title{Jonah}


\chapter{1}

\par 1 La parola dell'Eterno fu rivolta a Giona, figliuolo di Amittai, in questi termini:
\par 2 'Lèvati, va' a Ninive, la gran città, e predica contro di lei; perché la loro malvagità è salita nel mio cospetto'.
\par 3 Ma Giona si levò per fuggirsene a Tarsis, lungi dal cospetto dell'Eterno; e scese a Giaffa, dove trovò una nave che andava a Tarsis; e, pagato il prezzo del suo passaggio, s'imbarcò per andare con quei della nave a Tarsis, lungi dal cospetto dell'Eterno.
\par 4 Ma l'Eterno scatenò un gran vento sul mare, e vi fu sul mare una forte tempesta, sì che la nave minacciava di sfasciarsi.
\par 5 I marinai ebbero paura, e ognuno gridò al suo dio e gettarono a mare le mercanzie ch'erano a bordo, per alleggerire la nave; ma Giona era sceso nel fondo della nave, s'era coricato, e dormiva profondamente.
\par 6 Il capitano gli si avvicinò, e gli disse: 'Che fai tu qui a dormire? Lèvati, invoca il tuo dio! Forse Dio si darà pensiero di noi, e non periremo'.
\par 7 Poi dissero l'uno all'altro: 'Venite, tiriamo a sorte, per sapere a cagione di chi ci capita questa disgrazia'. Tirarono a sorte, e la sorte cadde su Giona.
\par 8 Allora essi gli dissero: 'Dicci dunque a cagione di chi ci capita questa disgrazia! Qual è la tua occupazione? donde vieni? qual è il tuo paese? e a che popolo appartieni?'
\par 9 Egli rispose loro: 'Sono Ebreo, e temo l'Eterno, l'Iddio del cielo, che ha fatto il mare e la terra ferma'.
\par 10 Allora quegli uomini furon presi da grande spavento, e gli dissero: 'Perché hai fatto questo?' Poiché quegli uomini sapevano ch'egli fuggiva lungi dal cospetto dell'Eterno, giacché egli avea dichiarato loro la cosa.
\par 11 E quelli gli dissero: 'Che ti dobbiam fare perché il mare si calmi per noi?' Poiché il mare si faceva sempre più tempestoso.
\par 12 Egli rispose loro: 'Pigliatemi e gettatemi in mare, e il mare si calmerà per voi; perché io so che questa forte tempesta vi piomba addosso per cagion mia'.
\par 13 Nondimeno quegli uomini davan forte nei remi per ripigliar terra; ma non potevano, perché il mare si faceva sempre più tempestoso e minaccioso.
\par 14 Allora gridarono all'Eterno, e dissero: 'Deh, o Eterno, non lasciar che periamo per risparmiar la vita di quest'uomo, e non ci mettere addosso del sangue innocente; perché tu, o Eterno, hai fatto quel che ti è piaciuto'.
\par 15 Poi presero Giona e lo gettarono in mare; e la furia del mare si calmò.
\par 16 E quegli uomini furon presi da un gran timore dell'Eterno; offrirono un sacrifizio all'Eterno, e fecero dei voti.

\chapter{2}

\par 1 E l'Eterno fece venire un gran pesce per inghiottir Giona; e Giona fu nel ventre del pesce tre giorni e tre notti.
\par 2 E Giona pregò l'Eterno, il suo Dio, dal ventre del pesce, e disse:
\par 3 Io ho gridato all'Eterno dal fondo della mia distretta, ed egli m'ha risposto; dalle viscere del soggiorno dei morti ho gridato, e tu hai udito la mia voce.
\par 4 Tu m'hai gettato nell'abisso, nel cuore del mare; la corrente mi ha circondato e tutte le tue onde e tutti i tuoi flutti mi son passati sopra.
\par 5 E io dicevo: Io son cacciato via lungi dal tuo sguardo! Come vedrei io ancora il tuo tempio santo?
\par 6 Le acque m'hanno attorniato fino all'anima; l'abisso m'ha avvolto; le alghe mi si sono attorcigliate al capo.
\par 7 Io son disceso fino alle radici dei monti; la terra con le sue sbarre mi ha rinchiuso per sempre; ma tu hai fatto risalir la mia vita dalla fossa, o Eterno, Dio mio!
\par 8 Quando l'anima mia veniva meno in me, io mi son ricordato dell'Eterno, e la mia preghiera è giunta fino a te, nel tuo tempio santo.
\par 9 Quelli che onorano le vanità bugiarde abbandonano la fonte della loro grazia;
\par 10 ma io t'offrirò sacrifizi, con canti di lode; adempirò i voti che ho fatto. La salvezza appartiene all'Eterno.
\par 11 E l'Eterno diè l'ordine al pesce, e il pesce vomitò Giona sull'asciutto.

\chapter{3}

\par 1 E la parola dell'Eterno fu rivolta a Giona per la seconda volta, in questi termini:
\par 2 'Lèvati, va' a Ninive, la gran città, e proclamale quello che io ti comando'.
\par 3 E Giona si levò, e andò a Ninive, secondo la parola dell'Eterno. Or Ninive era una grande città dinanzi a Dio, di tre giornate di cammino.
\par 4 E Giona cominciò a inoltrarsi nella città per il cammino d'una giornata, e predicava e diceva: 'Ancora quaranta giorni, e Ninive sarà distrutta!'
\par 5 E i Niniviti credettero a Dio, bandirono un digiuno, e si vestirono di sacchi, dai più grandi ai più piccoli.
\par 6 Ed essendo la notizia giunta al re di Ninive, questi s'alzò dal trono, si tolse di dosso il manto, si coprì d'un sacco, e si mise a sedere sulla cenere.
\par 7 E per decreto del re e dei suoi grandi, fu pubblicato in Ninive un bando di questo tenore: 'Uomini e bestie, armenti e greggi, non assaggino nulla; non si pascano e non bevano acqua;
\par 8 uomini e bestie si coprano di sacchi e gridino con forza a Dio; e ognuno si converta dalla sua via malvagia, e dalla violenza perpetrata dalle sue mani.
\par 9 Chi sa che Dio non si volga, non si penta, e non acqueti l'ardente sua ira, sì che noi non periamo'.
\par 10 E Dio vide quel che facevano, vide che si convertivano dalla loro via malvagia, e si pentì del male che avea parlato di far loro; e non lo fece.

\chapter{4}

\par 1 Ma Giona ne provò un gran dispiacere, e ne fu irritato; e pregò l'Eterno, dicendo:
\par 2 'O Eterno, non è egli questo ch'io dicevo, mentr'ero ancora nel mio paese? Perciò m'affrettai a fuggirmene a Tarsis; perché sapevo che sei un Dio misericordioso, pietoso, lento all'ira, di gran benignità, e che ti penti del male minacciato.
\par 3 Or dunque, o Eterno, ti prego, riprenditi la mia vita; poiché per me val meglio morire che vivere'.
\par 4 E l'Eterno gli disse: 'Fai tu bene a irritarti così?'
\par 5 Poi Giona uscì dalla città, e si mise a sedere a oriente della città; si fece quivi una capanna, e vi sedette sotto, all'ombra, stando a vedere quello che succederebbe alla città.
\par 6 E Dio, l'Eterno, per guarirlo della sua irritazione, fece crescere un ricino, che montò su di sopra a Giona per fargli ombra al capo; e Giona provò una grandissima gioia a motivo di quel ricino.
\par 7 Ma l'indomani, allo spuntar dell'alba, Iddio fece venire un verme, il quale attaccò il ricino, ed esso si seccò.
\par 8 E come il sole fu levato, Iddio fece soffiare un vento soffocante d'oriente, e il sole picchiò sul capo di Giona, sì ch'egli venne meno, e chiese di morire, dicendo: 'Meglio è per me morire che vivere'.
\par 9 E Dio disse a Giona: 'Fai tu bene a irritarti così a motivo del ricino?' Egli rispose: 'Sì, faccio bene a irritarmi fino alla morte'.
\par 10 E l'Eterno disse: 'Tu hai pietà del ricino per il quale non hai faticato, e che non hai fatto crescere, che è nato in una notte e in una notte è perito:
\par 11 e io non avrei pietà di Ninive, la gran città, nella quale si trovano più di centoventimila persone che non sanno distinguere la loro destra dalla loro sinistra, e tanta quantità di bestiame?'


\end{document}