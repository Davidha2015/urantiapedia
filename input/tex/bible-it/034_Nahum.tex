\begin{document}

\title{Nahum}


\chapter{1}

\par 1 Oracolo relativo a Ninive; libro della visione di Nahum d'Elkosh.
\par 2 L'Eterno è un Dio geloso e vendicatore; l'Eterno è vendicatore e pieno di furore; l'Eterno si vendica dei suoi avversari, e serba il cruccio per i suoi nemici.
\par 3 L'Eterno è lento all'ira, è grande in forza, ma non tiene il colpevole per innocente. L'Eterno cammina nel turbine e nella tempesta, e le nuvole son la polvere de' suoi piedi.
\par 4 Egli sgrida il mare e lo prosciuga, dissecca tutti i fiumi. Basan langue, langue il Carmelo, e langue il fiore del Libano.
\par 5 I monti tremano davanti a lui, si struggono i colli; la terra si solleva alla sua presenza, e il mondo con tutti i suoi abitanti.
\par 6 Chi può reggere davanti alla sua indignazione? chi può sussistere sotto l'ardore della sua ira? Il suo furore si spande come fuoco, e le roccie scoscendono davanti a lui.
\par 7 L'Eterno è buono; è una fortezza nel giorno della distretta, ed egli conosce quelli che si rifugiano in lui.
\par 8 Ma con una irrompente inondazione egli farà una totale distruzione del luogo ov'è Ninive, e inseguirà i propri nemici fin nelle tenebre.
\par 9 Che meditate voi contro l'Eterno? Egli farà una distruzione totale; la distretta non sorgerà due volte.
\par 10 Poiché fossero pur intrecciati come spine e fradici pel vino ingollato, saran divorati del tutto, come stoppia secca.
\par 11 Da te è uscito colui che ha meditato del male contro l'Eterno, che ha macchinato scelleratezze.
\par 12 Così parla l'Eterno: Anche se in piena forza e numerosi, saranno falciati e scompariranno; e s'io t'ho afflitta non t'affliggerò più.
\par 13 Ora spezzerò il suo giogo d'addosso a te, e infrangerò i tuoi legami.
\par 14 E quanto a te, popolo di Ninive, l'Eterno ha dato quest'ordine: che non vi sia più posterità del tuo nome; io sterminerò dalla casa delle tue divinità le immagini scolpite e le immagini fuse; io ti preparerò la tomba perché sei divenuto spregevole.
\par 15 Ecco, sui monti, i piedi di colui che reca buone novelle, che annunzia la pace! Celebra le tue feste, o Giuda, sciogli i tuoi voti; poiché lo scellerato non passerà più in mezzo a te; egli è sterminato interamente.

\chapter{2}

\par 1 Un distruttore sale contro di te o Ninive; custodisci bene la fortezza, sorveglia le strade, fortificati i fianchi, raccogli tutte quante le tue forze!
\par 2 Poiché l'Eterno ristabilisce la gloria di Giacobbe, e la gloria d'Israele; perché i saccheggiatori li han saccheggiati, e han distrutto i loro tralci.
\par 3 Lo scudo de' suoi prodi è tinto in rosso, i suoi guerrieri veston di scarlatto; il giorno in cui ei si prepara, l'acciaio de' carri scintilla, e si brandiscon le lance di cipresso.
\par 4 I carri si slancian furiosamente per le strade, si precipitano per le piazze; il loro aspetto è come di fiaccole, guizzan come folgori.
\par 5 Il re si ricorda de' suoi prodi ufficiali; essi inciampano nella loro marcia, si precipitano verso le mura, e la difesa è preparata.
\par 6 Le porte de' fiumi s'aprono, e il palazzo crolla.
\par 7 È fatto! Ninive è spogliata nuda e portata via; le sue serve gemono con voce di colombe, e si picchiano il petto.
\par 8 Essa è stata come un serbatoio pieno d'acqua, dal giorno che esiste; e ora fuggono! 'Fermate! fermate!' ma nessuno si volta.
\par 9 Predate l'argento, predate l'oro! Vi son de' tesori senza fine, dei monti d'oggetti preziosi d'ogni sorta.
\par 10 Essa è vuotata, spogliata, devastata; i cuori si struggono, le ginocchia tremano, tutti i fianchi si contorcono, tutti i volti impallidiscono.
\par 11 Dov'è questo ricetto di leoni, questo luogo dove facevano il pasto i leoncelli, dove passeggiavano il leone, la leonessa e i leoncini, senza che alcuno li spaventasse?
\par 12 Quivi il leone sbranava per i suoi piccini, strangolava per le sue leonesse, ed empiva le sue grotte di preda, e le sue tane di rapina.
\par 13 Eccomi a te, dice l'Eterno degli eserciti; io arderò i tuoi carri che andranno in fumo, e la spada divorerà i tuoi leoncelli; io sterminerò dalla terra la tua preda, e più non s'udrà la voce de' tuoi messaggeri.

\chapter{3}

\par 1 Guai alla città di sangue, che è tutta piena di menzogna e di violenza e che non cessa di far preda!
\par 2 S'ode rumor di sferza, strepito di ruote, galoppo di cavalli, balzar di carri.
\par 3 I cavalieri danno la carica, fiammeggiano le spade, sfolgoran le lance, i feriti abbondano, s'ammontano i cadaveri, sono infiniti i morti, s'inciampa nei cadaveri.
\par 4 E questo a cagione delle tante fornicazioni dell'avvenente prostituta, dell'abile incantatrice, che vendeva le nazioni con le sue fornicazioni, e i popoli coi suoi incantesimi.
\par 5 Eccomi a te, dice l'Eterno degli eserciti; io t'alzerò i lembi della veste fin sulla faccia e mostrerò alle nazioni la tua nudità, e ai regni la tua vergogna;
\par 6 e ti getterò addosso delle immondizie, t'avvilirò, e ti esporrò in spettacolo.
\par 7 Tutti quelli che ti vedranno fuggiranno lungi da te, e diranno: 'Ninive è devastata! Chi la compiangerà? Dove ti cercherei dei consolatori?'
\par 8 Vali tu meglio di No-Amon, ch'era assisa tra i fiumi, circondata dalle acque, che aveva il mare per baluardo, il mare per mura?
\par 9 L'Etiopia e l'Egitto eran la sua forza, e non avea limiti; Put ed i Libî erano i suoi ausiliari.
\par 10 Eppure, anch'essa è stata deportata, è andata in cattività; anche i bambini suoi sono stati sfracellati a ogni canto di strada; s'è tirata la sorte sopra i suoi uomini onorati, e tutti i suoi grandi sono stati messi in catene.
\par 11 Tu pure sarai ubriacata, t'andrai a nascondere; tu pure cercherai un rifugio davanti al nemico.
\par 12 Tutte le tue fortezze saranno come fichi dai frutti primaticci, che, quando li si scuote, cadono in bocca di chi li vuol mangiare.
\par 13 Ecco, il tuo popolo, in mezzo a te, son tante donne; le porte del tuo paese sono spalancate davanti ai tuoi nemici, il fuoco ha divorato le tue sbarre.
\par 14 Attingiti pure acqua per l'assedio! Rinforza le tue fortificazioni! Entra nella malta, pesta l'argilla! Restaura la fornace da mattoni!
\par 15 Là il fuoco ti divorerà, la spada ti distruggerà; ti divorerà come la cavalletta, fossi tu pur numerosa come le cavallette, fossi tu pur numerosa come le locuste.
\par 16 Tu hai moltiplicato i tuoi mercanti, più delle stelle del cielo; le cavallette spogliano ogni cosa e volano via.
\par 17 I tuoi principi son come le locuste, i tuoi ufficiali come sciami di giovani locuste che s'accampano lungo le siepi in giorno di freddo, e quando il sole si leva volano via, e non si conosce più il posto dov'erano.
\par 18 O re d'Assiria, i tuoi pastori si sono addormentati; i tuoi valorosi ufficiali riposano; il tuo popolo è disperso su per i monti, e non v'è chi li raduni.
\par 19 Non v'è rimedio per la tua ferita; la tua piaga è grave; tutti quelli che udranno parlare di te batteranno le mani alla tua sorte; poiché su chi non è passata del continuo la tua malvagità?


\end{document}