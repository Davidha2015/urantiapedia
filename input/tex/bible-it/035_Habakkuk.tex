\begin{document}

\title{Habakkuk}


\chapter{1}

\par 1 Oracolo che il profeta Habacuc ebbe per visione.
\par 2 Fino a quando, o Eterno, griderò, senza che tu mi dia ascolto? Io grido a te: 'Violenza!' e tu non salvi.
\par 3 Perché mi fai veder l'iniquità, e tolleri lo spettacolo della perversità? e perché mi stanno dinanzi la rapina e la violenza? Vi son liti, e sorge la discordia.
\par 4 Perciò la legge è senza forza e il diritto non fa strada, perché l'empio aggira il giusto, e il diritto n'esce pervertito.
\par 5 Vedete fra le nazioni, guardate, maravigliatevi e siate stupefatti! Poiché io sto per fare ai vostri giorni un'opera, che voi non credereste, se ve la raccontassero.
\par 6 Perché, ecco, io sto per suscitare i Caldei, questa nazione aspra e impetuosa, che percorre la terra quant'è larga, per impadronirsi di dimore che non son sue.
\par 7 È terribile, formidabile; il suo diritto e la sua grandezza emanano da lui stesso.
\par 8 I suoi cavalli son più veloci de' leopardi, più agili de' lupi sulla sera; i suoi cavalieri procedon con fierezza; i suoi cavalieri vengon di lontano, volan come l'aquila che piomba sulla preda.
\par 9 Tutta quella gente viene per darsi alla violenza, le lor facce bramose son tese in avanti, e ammassan prigionieri senza numero come la rena.
\par 10 Si fan beffe dei re, e i principi son per essi oggetto di scherno; si ridono di tutte le fortezze; ammontano un po' di terra, e le prendono.
\par 11 Poi passan come il vento; passan oltre e si rendon colpevoli, questa lor forza è il loro dio.
\par 12 Non sei tu ab antico, o Eterno, il mio Dio, il mio Santo? Noi non morremo! O Eterno, tu l'hai posto, questo popolo, per esercitare i tuoi giudizi, tu, o Ròcca, l'hai stabilito per infliggere i tuoi castighi.
\par 13 Tu, che hai gli occhi troppo puri per sopportar la vista del male, e che non puoi tollerar lo spettacolo dell'iniquità, perché guardi i perfidi, e taci quando il malvagio divora l'uomo ch'è più giusto di lui?
\par 14 E perché rendi gli uomini come i pesci del mare e come i rettili, che non hanno signore?
\par 15 Il Caldeo li trae tutti su con l'amo, li piglia nella sua rete, li raccoglie nel suo giacchio; perciò si rallegra ed esulta.
\par 16 Per questo fa sacrifizi alla sua rete, e offre profumi al suo giacchio; perché per essi la sua parte è grassa, e il suo cibo è succulento.
\par 17 Dev'egli per questo seguitare a vuotar la sua rete, e massacrar del continuo le nazioni senza pietà?

\chapter{2}

\par 1 Io starò alla mia vedetta, mi porrò sopra una torre, e starò attento a quello che l'Eterno mi dirà, e a quello che dovrò rispondere circa la rimostranza che ho fatto.
\par 2 E l'Eterno mi rispose e disse: 'Scrivi la visione, incidila su delle tavole, perché si possa leggere speditamente;
\par 3 poiché è una visione per un tempo già fissato; ella s'affretta verso la fine, e non mentirà; se tarda, aspettala; poiché per certo verrà; non tarderà'.
\par 4 Ecco, l'anima sua è gonfia, non è retta in lui; ma il giusto vivrà per la sua fede.
\par 5 E poi, il vino è perfido; l'uomo arrogante non può starsene tranquillo; egli allarga le sue brame come il soggiorno de' morti; è come la morte e non si può saziare, ma raduna presso di sé tutte le nazioni, raccoglie intorno a sé tutti i popoli.
\par 6 Tutti questi non faranno contro di lui proverbi, sarcasmi, enigmi? Si dirà: 'Guai a colui che accumula ciò che non è suo! Fino a quando? Guai a colui che si carica di pegni!'
\par 7 I tuoi creditori non si leveranno essi ad un tratto? I tuoi tormentatori non si desteranno essi? E tu diventerai loro preda.
\par 8 Poiché tu hai saccheggiato molte nazioni, tutto il resto dei popoli ti saccheggerà, a motivo del sangue umano sparso, della violenza fatta ai paesi, alle città e a tutti i loro abitanti.
\par 9 Guai a colui ch'è avido d'illecito guadagno per la sua casa, per porre il suo nido in alto e mettersi al sicuro dalla mano della sventura!
\par 10 Tu hai divisato l'onta della tua casa, sterminando molti popoli; e hai peccato contro te stesso.
\par 11 Poiché la pietra grida dalla parete, e la trave le risponde dall'armatura di legname.
\par 12 Guai a colui che edifica la città col sangue, e fonda una città sull'iniquità!
\par 13 Ecco, questo non procede egli dall'Eterno che i popoli s'affatichino per il fuoco, e le nazioni si stanchino per nulla?
\par 14 Poiché la terra sarà ripiena della conoscenza della gloria dell'Eterno, come le acque coprono il fondo del mare.
\par 15 Guai a colui che dà da bere al prossimo, a te che gli versi il tuo veleno e l'ubriachi, per guardare la sua nudità!
\par 16 Tu sarai saziato d'onta anziché di gloria; bevi anche tu, e scopri la tua incirconcisione! La coppa della destra dell'Eterno farà il giro fino a te, e l'ignominia coprirà la tua gloria.
\par 17 Poiché la violenza fatta al Libano e la devastazione che spaventava le bestie, ricadranno su te, a motivo del sangue umano sparso, della violenza fatta ai paesi, alle città e a tutti i loro abitanti.
\par 18 A che giova l'immagine scolpita perché l'artefice la scolpisca? A che giova l'immagine fusa che insegna la menzogna, perché l'artefice si confidi nel suo lavoro, fabbricando idoli muti?
\par 19 Guai a chi dice al legno: 'Svègliati!' e alla pietra muta: 'Lèvati!' Può essa ammaestrare? Ecco, è ricoperta d'oro e d'argento, ma non v'è in lei spirito alcuno.
\par 20 Ma l'Eterno è nel suo tempio santo; tutta la terra faccia silenzio in presenza sua!

\chapter{3}

\par 1 Preghiera del profeta Habacuc. Sopra Scighionoth.
\par 2 O Eterno, io ho udito il tuo messaggio, e son preso da timore; o Eterno, da' vita all'opera tua nel corso degli anni! Nel corso degli anni falla conoscere! Nell'ira, ricordati d'aver pietà!
\par 3 Iddio viene da Teman, e il santo viene dal monte di Paran. Sela. La sua gloria copre i cieli, e la terra è piena della sua lode.
\par 4 Il suo splendore è pari alla luce; dei raggi partono dalla sua mano; ivi si nasconde la sua potenza.
\par 5 Davanti a lui cammina la peste, la febbre ardente segue i suoi passi.
\par 6 Egli si ferma, e scuote la terra; guarda, e fa tremar le nazioni; i monti eterni si frantumano, i colli antichi s'abbassano; le sue vie son quelle d'un tempo.
\par 7 Io vedo nell'afflizione le tende d'Etiopia; i padiglioni del paese di Madian tremano.
\par 8 O Eterno, t'adiri tu contro i fiumi? È egli contro i fiumi che s'accende l'ira tua, o contro il mare che va il tuo sdegno, che tu avanzi sui tuoi cavalli, sui tuoi carri di vittoria?
\par 9 Il tuo arco è messo a nudo; i dardi lanciati dalla tua parola sono esecrazioni. Sela. Tu fendi la terra in tanti letti di fiumi.
\par 10 I monti ti vedono e tremano; passa una piena d'acque: l'abisso fa udir la sua voce, e leva in alto le mani.
\par 11 Il sole e la luna si fermano nella loro dimora; si cammina alla luce delle tue saette, al lampeggiare della tua lancia sfolgorante.
\par 12 Tu percorri la terra nella tua indignazione, tu schiacci le nazioni nella tua ira.
\par 13 Tu esci per salvare il tuo popolo, per liberare il tuo unto; tu abbatti la sommità della casa dell'empio, e la demolisci da capo a fondo. Sela.
\par 14 Tu trafiggi coi lor propri dardi la testa de' suoi capi, che vengon come un uragano per disperdermi, mandando gridi di gioia, come se già divorassero il misero nei loro nascondigli.
\par 15 Coi tuoi cavalli tu calpesti il mare, le grandi acque spumeggianti.
\par 16 Ho udito, e le mie viscere fremono, le mie labbra tremano a quella voce; un tarlo m'entra nelle ossa, e io tremo qui dove sto, a dover aspettare in silenzio il dì della distretta, quando il nemico salirà contro il popolo per assalirlo.
\par 17 Poiché il fico non fiorirà, non ci sarà più frutto nelle vigne; il prodotto dell'ulivo fallirà, i campi non daran più cibo, i greggi verranno a mancare negli ovili, e non ci saran più buoi nelle stalle;
\par 18 ma io mi rallegrerò nell'Eterno, esulterò nell'Iddio della mia salvezza.
\par 19 L'Eterno, il Signore, è la mia forza; egli renderà i miei piedi come quelli delle cerve, e mi farà camminare sui miei alti luoghi. Al capo de' musici. Per strumenti a corda.


\end{document}