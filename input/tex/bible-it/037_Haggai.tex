\begin{document}

\title{Haggai}


\chapter{1}

\par 1 Il secondo anno del re Dario, il sesto mese, il primo giorno del mese, la parola dell'Eterno fu rivolta, per mezzo del profeta Aggeo, a Zorobabele, figliuolo di Scealtiel, governatore di Giuda, e a Giosuè, figliuolo di Jehotsadak, sommo sacerdote, in questi termini:
\par 2 'Così parla l'Eterno degli eserciti: Questo popolo dice: Il tempo non è giunto, il tempo in cui la casa dell'Eterno dev'essere riedificata'.
\par 3 Perciò la parola dell'Eterno fu rivolta loro per mezzo del profeta Aggeo, in questi termini:
\par 4 'È egli il tempo per voi stessi d'abitare le vostre case ben rivestite di legno, mentre questa casa giace in rovina?
\par 5 Or dunque così parla l'Eterno degli eserciti: Ponete ben mente alle vostre vie!
\par 6 Voi avete seminato molto, e avete raccolto poco; voi mangiate, ma non fino ad esser sazi; bevete, ma non fino a soddisfare la sete; vi vestite, ma non v'è chi si riscaldi; chi guadagna un salario mette il suo salario in una borsa forata.
\par 7 Così parla l'Eterno degli eserciti: Ponete ben mente alle vostre vie!
\par 8 Salite nella contrada montuosa, recate del legname, e costruite la casa; e io mi compiacerò d'essa, e sarò glorificato, dice l'Eterno.
\par 9 Voi v'aspettavate molto, ed ecco v'è poco; e quando l'avete portato in casa, io ci ho soffiato sopra. Perché? dice l'Eterno degli eserciti. A motivo della mia casa che giace in rovina, mentre ognun di voi si dà premura per la propria casa.
\par 10 Perciò il cielo, sopra di voi, è rimasto chiuso, sì che non c'è stata rugiada, e la terra ha ritenuto il suo prodotto.
\par 11 Ed io ho chiamato la siccità sul paese, sui monti, sul grano, sul vino, sull'olio, su tutto ciò che il suolo produce, sugli uomini, sul bestiame, e su tutto il lavoro delle mani'.
\par 12 E Zorobabele, figliuolo di Scealtiel, e Giosuè, figliuolo di Jehotsadak, il sommo sacerdote, e tutto il rimanente del popolo, diedero ascolto alla voce dell'Eterno, del loro Dio, e alle parole del profeta Aggeo, secondo il messaggio che l'Eterno, il loro Dio, gli aveva affidato; e il popolo temette l'Eterno.
\par 13 E Aggeo, messaggero dell'Eterno, disse al popolo, in virtù della missione avuta dall'Eterno: 'Io son con voi, dice l'Eterno'.
\par 14 E l'Eterno destò lo spirito di Zorobabele, figliuolo di Scealtiel, governatore di Giuda, e lo spirito di Giosuè, figliuolo di Jehotsadak, sommo sacerdote, e lo spirito di tutto il resto del popolo; ed essi vennero e misero mano all'opera nella casa dell'Eterno degli eserciti, il loro Dio,
\par 15 il ventiquattresimo giorno del mese, il sesto mese, il secondo anno del re Dario.

\chapter{2}

\par 1 Il settimo mese, il ventunesimo giorno del mese, la parola dell'Eterno fu rivelata per mezzo del profeta Aggeo, in questi termini:
\par 2 'Parla ora a Zorobabele, figliuolo di Scealtiel, governatore di Giuda, e a Giosuè, figliuolo di Jehotsadak, sommo sacerdote, e al resto del popolo, e di' loro:
\par 3 Chi è rimasto fra voi che abbia veduto questa casa nella sua prima gloria? E come la vedete adesso? Così com'è, non è essa come nulla agli occhi vostri?
\par 4 E ora, fortìficati, Zorobabele! dice l'Eterno, fortìficati, Giosuè, figliuolo di Jehotsadak, sommo sacerdote! fortìficati, o popolo tutto del paese! dice l'Eterno; e mettetevi all'opra! poiché io sono con voi, dice l'Eterno degli eserciti,
\par 5 secondo il patto che feci con voi quando usciste dall'Egitto, e il mio spirito dimora tra voi, non temete!
\par 6 Poiché così parla l'Eterno degli eserciti: Ancora una volta, fra poco, io farò tremare i cieli, la terra, il mare, e l'asciutto;
\par 7 farò tremare tutte le nazioni, le cose più preziose di tutte le nazioni affluiranno, ed io empirò di gloria questa casa, dice l'Eterno degli eserciti.
\par 8 Mio è l'argento e mio è l'oro, dice l'Eterno degli eserciti.
\par 9 La gloria di quest'ultima casa sarà più grande di quella della prima, dice l'Eterno degli eserciti; e in questo luogo io darò la pace, dice l'Eterno degli eserciti'.
\par 10 Il ventiquattresimo giorno del nono mese, il secondo anno di Dario, la parola dell'Eterno fu rivelata per mezzo del profeta Aggeo, in questi termini:
\par 11 'Così parla l'Eterno degli eserciti: Interroga i sacerdoti sulla legge intorno a questo punto:
\par 12 - Se uno porta nel lembo della sua veste della carne consacrata, e con quel suo lembo tocca del pane, o una vivanda cotta, o del vino, o dell'olio, o qualsivoglia altro cibo, quelle cose diventeranno esse consacrate? - I sacerdoti risposero e dissero: - No. -
\par 13 E Aggeo disse: - Se uno, essendo impuro a motivo d'un morto, tocca qualcuna di quelle cose, diventerà essa impura? - I sacerdoti risposero e dissero: - Sì, diventerà impura. -
\par 14 Allora Aggeo replicò e disse: - Così è questo popolo, così è questa nazione nel mio cospetto, dice l'Eterno; e così è tutta l'opera delle loro mani; e tutto quello che m'offrono là è impuro.
\par 15 Ed ora, ponete ben mente a ciò ch'è avvenuto fino a questo giorno, prima che fosse messa pietra su pietra nel tempio dell'Eterno!
\par 16 Durante tutto quel tempo, quand'uno veniva a un mucchio di venti misure, non ve n'erano che dieci; quand'uno veniva al tino per cavarne cinquanta misure, non ve n'eran che venti.
\par 17 Io vi colpii col carbonchio, colla ruggine, con la grandine, in tutta l'opera delle vostre mani; ma voi non tornaste a me, dice l'Eterno.
\par 18 Ponete ben mente a ciò ch'è avvenuto fino a questo giorno, fino al ventiquattresimo giorno del nono mese, dal giorno che il tempio dell'Eterno fu fondato; ponetevi ben mente!
\par 19 V'è egli ancora del grano nel granaio? La stessa vigna, il fico, il melagrano, l'ulivo, nulla producono! Da questo giorno, io vi benedirò'.
\par 20 E la parola dell'Eterno fu indirizzata per la seconda volta ad Aggeo, il ventiquattresimo giorno del mese, in questi termini:
\par 21 'Parla a Zorobabele, governatore di Giuda, e digli: Io farò tremare i cieli e la terra,
\par 22 rovescerò il trono dei regni e distruggerò la forza dei regni delle nazioni; rovescerò i carri e quelli che vi montano; i cavalli e i loro cavalieri cadranno, l'uno per la spada dell'altro.
\par 23 In quel giorno, dice l'Eterno degli eserciti, io ti prenderò, o Zorobabele, figliuolo di Scealtiel, mio servo, dice l'Eterno, e ti terrò come un sigillo, perché io t'ho scelto, dice l'Eterno degli eserciti'.


\end{document}