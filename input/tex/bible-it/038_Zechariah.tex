\begin{document}

\title{Zechariah}


\chapter{1}

\par 1 L'ottavo mese, il secondo anno di Dario, la parola dell'Eterno fu rivolta al profeta Zaccaria, figliuolo di Berekia, figliuolo d'Iddo, il profeta, in questi termini:
\par 2 'L'Eterno è stato gravemente adirato contro i vostri padri.
\par 3 Tu, dunque, di' loro: Così parla l'Eterno degli eserciti: Tornate a me, dice l'Eterno degli eserciti, e io tornerò a voi; dice l'Eterno degli eserciti.
\par 4 Non siate come i vostri padri, ai quali i profeti precedenti si rivolgevano, dicendo: Così parla l'Eterno degli eserciti: - Ritraetevi dalle vostre vie malvage, dalle vostre malvage azioni! Ma essi non dettero ascolto, e non prestarono attenzione a me, dice l'Eterno.
\par 5 I vostri padri dove son essi? E i profeti potevan essi vivere in perpetuo?
\par 6 Ma le mie parole e i miei decreti, dei quali avevo dato incarico ai miei servi i profeti, non arrivarono essi a colpire i padri vostri? Allora essi si convertirono, e dissero: L'Eterno degli eserciti ci ha trattati secondo le nostre vie e secondo le nostre azioni, come avea risoluto di fare'.
\par 7 Il ventiquattresimo giorno dell'undicesimo mese, che è il mese di Scebat, nel secondo anno di Dario, la parola dell'Eterno fu rivolta a Zaccaria, figliuolo di Berekia, figliuolo d'Iddo, il profeta, in questi termini:
\par 8 Io ebbi, di notte, una visione; ed ecco un uomo montato sopra un cavallo rosso; egli stava fra le piante di mortella in un luogo profondo; e dietro a lui c'erano de' cavalli rossi, sauri e bianchi.
\par 9 E io dissi: 'Che son questi, signor mio?' E l'angelo che parlava meco mi disse: 'Io ti farò vedere che cosa son questi'.
\par 10 E l'uomo che stava fra le piante di mortella prese a dire: 'Questi son quelli che l'Eterno ha mandati a percorrere la terra'.
\par 11 E quelli si rivolsero all'angelo dell'Eterno che stava fra le piante di mortella, e dissero: 'Noi abbiamo percorso la terra, ed ecco tutta la terra è in riposo e tranquilla'.
\par 12 Allora l'angelo dell'Eterno prese a dire: 'O Eterno degli eserciti, fino a quando non avrai tu pietà di Gerusalemme e delle città di Giuda, contro le quali sei stato indignato durante questi settant'anni?'
\par 13 E l'Eterno rivolse all'angelo che parlava meco, delle buone parole, delle parole di conforto.
\par 14 E l'angelo che parlava meco mi disse: 'Grida e di': Così parla l'Eterno degli eserciti: Io provo una gran gelosia per Gerusalemme e per Sion;
\par 15 e provo un grande sdegno contro le nazioni che se ne stanno ora tranquille, e che, quand'io m'indignai un poco contro di essa, contribuirono ad accrescer la sua disgrazia.
\par 16 Perciò così parla l'Eterno: Io mi volgo di nuovo a Gerusalemme con compassione; la mia casa vi sarà ricostruita, dice l'Eterno degli eserciti, e la corda sarà di nuovo tirata su Gerusalemme.
\par 17 Grida ancora, e di': Così parla l'Eterno degli eserciti: Le mie città rigurgiteranno ancora di beni, e l'Eterno consolerà ancora Sion, e sceglierà ancora Gerusalemme'.
\par 18 Poi alzai gli occhi, guardai, ed ecco quattro corna.
\par 19 E io dissi all'angelo che parlava meco: 'Che son queste?' Egli mi rispose: 'Queste son le corna che hanno disperso Giuda, Israele e Gerusalemme'.
\par 20 E l'Eterno mi fece vedere quattro fabbri.
\par 21 E io dissi: 'Questi, che vengono a fare?' Egli rispose e mi disse: 'Quelle là son le corna che hanno disperso Giuda, sì che nessuno alzava più il capo; ma questi qui vengono per spaventarle, per abbattere le corna delle nazioni, che hanno alzato il loro corno contro il paese di Giuda per disperderne gli abitanti'.

\chapter{2}

\par 1 E alzai gli occhi, guardai, ed ecco un uomo che aveva in mano una corda da misurare.
\par 2 E io dissi: 'Dove vai?' Egli mi rispose: 'Vado a misurare Gerusalemme, per vedere qual ne sia la larghezza, e quale la lunghezza'.
\par 3 Ed ecco, l'angelo che parlava meco si fece avanti, e un altro gli uscì incontro,
\par 4 e gli disse: 'Corri, parla a quel giovane, e digli: Gerusalemme sarà abitata come una città senza mura, tanta sarà la quantità di gente e di bestiame che si troverà in mezzo ad essa;
\par 5 e io, dice l'Eterno, sarò per lei un muro di fuoco tutt'attorno, e sarò la sua gloria in mezzo a lei.
\par 6 Olà, fuggite dal paese del settentrione, dice l'Eterno: perché io vi ho sparsi ai quattro venti dei cieli, dice l'Eterno.
\par 7 Olà, Sion, mettiti in salvo, tu che abiti con la figliuola di Babilonia!
\par 8 Poiché così parla l'Eterno degli eserciti: È per rivendicare la sua gloria, ch'egli mi ha mandato verso le nazioni che han fatto di voi la loro preda; perché chi tocca voi tocca la pupilla dell'occhio suo;
\par 9 infatti, ecco, io sto per agitare la mia mano contro di loro, ed esse diventeranno preda di quelli ch'eran loro asserviti, e voi conoscerete che l'Eterno degli eserciti m'ha mandato.
\par 10 Manda gridi di gioia, rallegrati, o figliuola di Sion! poiché ecco, io sto per venire, e abiterò in mezzo a te, dice l'Eterno.
\par 11 E in quel giorno molte nazioni s'uniranno all'Eterno, e diventeranno mio popolo; e io abiterò in mezzo a te, e tu conoscerai che l'Eterno degli eserciti m'ha mandato a te.
\par 12 E l'Eterno possederà Giuda come sua parte nella terra santa, e sceglierà ancora Gerusalemme.
\par 13 Ogni carne faccia silenzio in presenza dell'Eterno! poich'egli s'è destato dalla sua santa dimora'.

\chapter{3}

\par 1 E mi fece vedere il sommo sacerdote Giosuè, che stava in piè davanti all'angelo dell'Eterno, e Satana che gli stava alla destra per accusarlo.
\par 2 E l'Eterno disse a Satana: 'Ti sgridi l'Eterno, o Satana! ti sgridi l'Eterno che ha scelto Gerusalemme! Non è questi un tizzone strappato dal fuoco?'
\par 3 Or Giosuè era vestito di vestiti sudici, e stava in piè davanti all'angelo.
\par 4 E l'angelo prese a dire a quelli che gli stavano davanti: 'Levategli di dosso i vestiti sudici!' Poi disse a Giosuè: 'Guarda, io ti ho tolto di dosso la tua iniquità, e t'ho vestito di abiti magnifici'.
\par 5 E io dissi: 'Gli sia messa in capo una tiara pura!' E quelli gli posero in capo una tiara pura, e gli misero delle vesti; e l'angelo dell'Eterno era quivi presente.
\par 6 E l'angelo dell'Eterno fece a Giosuè questo solenne ammonimento:
\par 7 'Così parla l'Eterno degli eserciti: Se tu cammini nelle mie vie, e osservi quello che t'ho comandato, anche tu governerai la mia casa e custodirai i miei cortili, e io ti darò libero accesso fra quelli che stanno qui davanti a me.
\par 8 Ascolta dunque, o Giosuè, sommo sacerdote, tu e i tuoi compagni che stan seduti davanti a te! Poiché questi uomini servon di segni. Ecco, io faccio venire il mio servo, il Germoglio.
\par 9 Poiché, guardate la pietra che io ho posta davanti a Giosuè; sopra un'unica pietra stanno sette occhi; ecco, io v'inciderò quello che vi deve essere inciso, dice l'Eterno degli eserciti; e torrò via l'iniquità di questo paese in un sol giorno.
\par 10 In quel giorno, dice l'Eterno degli eserciti, voi vi inviterete gli uni gli altri sotto la vigna e sotto il fico'.

\chapter{4}

\par 1 E l'angelo che parlava meco tornò, e mi svegliò come si sveglia un uomo dal sonno.
\par 2 E mi disse: 'Che vedi?' Io risposi: 'Ecco, vedo un candelabro tutto d'oro, che ha in cima un vaso, ed è munito delle sue sette lampade, e di sette tubi per le lampade che stanno in cima;
\par 3 e vicino al candelabro stanno due ulivi; l'uno a destra del vaso, e l'altro alla sua sinistra'.
\par 4 E io presi a dire all'angelo che parlava meco: 'Che significan queste cose, signor mio?'
\par 5 L'angelo che parlava meco rispose e mi disse: 'Non sai quel che significhino queste cose?' E io dissi: 'No, mio signore'.
\par 6 Allora egli rispondendo, mi disse: 'È questa la parola che l'Eterno rivolge a Zorobabele: Non per potenza né per forza, ma per lo spirito mio, dice l'Eterno degli eserciti.
\par 7 Chi sei tu, o gran monte, davanti a Zorobabele? Tu diventerai pianura; ed egli porterà innanzi la pietra della vetta, in mezzo alle grida di: Grazia, grazia su di lei!'
\par 8 E la parola dell'Eterno mi fu rivolta in questi termini:
\par 9 'Le mani di Zorobabele hanno gettato le fondamenta di questa casa, e le sue mani la finiranno; e tu saprai che l'Eterno degli eserciti mi ha mandato a voi.
\par 10 Poiché chi potrebbe sprezzare il giorno delle piccole cose, quando quei sette là, gli occhi dell'Eterno che percorrono tutta la terra, vedono con gioia il piombino in mano a Zorobabele?'
\par 11 E io risposi e gli dissi: 'Che significano questi due ulivi a destra e a sinistra del candelabro?'
\par 12 E per la seconda volta io presi a dire: 'Che significano questi due ramoscelli d'ulivo che stanno allato ai due condotti d'oro per cui scorre l'olio dorato?'
\par 13 Ed egli rispose e mi disse: 'Non sai che significhino queste cose?' Io risposi: 'No, signor mio'.
\par 14 Allora egli disse: 'Questi sono i due unti che stanno presso il Signore di tutta la terra'.

\chapter{5}

\par 1 E io alzai di nuovo gli occhi, guardai, ed ecco un rotolo che volava.
\par 2 E l'angelo mi disse: 'Che vedi?' Io risposi: 'Vedo un rotolo che vola, la cui lunghezza è di venti cubiti, e la larghezza di dieci cubiti'.
\par 3 Ed egli mi disse: 'Questa è la maledizione che si spande sopra tutto il paese; poiché ogni ladro, a tenor di essa, sarà estirpato da questo luogo, e ogni spergiuro, a tenor di essa, sarà estirpato da questo luogo.
\par 4 Io la faccio uscire, dice l'Eterno degli eserciti, ed essa entrerà nella casa del ladro, e nella casa di colui che giura il falso nel mio nome; si stabilirà in mezzo a quella casa, e la consumerà col legname e le pietre che contiene'.
\par 5 E l'angelo che parlava meco uscì, e mi disse: 'Alza gli occhi, e guarda che cosa esce là'.
\par 6 Io risposi: 'Che cos'è?' Egli disse: 'È l'efa che esce'. Poi aggiunse: 'In tutto il paese non hanno occhio che per quello'.
\par 7 Ed ecco, fu alzata una piastra di piombo, e in mezzo all'efa stava seduta una donna.
\par 8 Ed egli disse: 'Questa è la malvagità'; e la gettò in mezzo all'efa, poi gettò la piastra di piombo sulla bocca dell'efa.
\par 9 Poi alzai gli occhi, guardai, ed ecco due donne che s'avanzavano; il vento soffiava nelle loro ali, e le ali che avevano eran come ali di cicogna; ed esse sollevarono l'efa fra terra e cielo.
\par 10 E io dissi all'angelo che parlava meco: 'Dove portano esse l'efa?'
\par 11 Egli mi rispose: 'Nel paese di Scinear, per costruirgli quivi una casa; e quando sarà preparata, esso sarà posto quivi al suo luogo'.

\chapter{6}

\par 1 E alzai di nuovo gli occhi, guardai, ed ecco quattro carri che uscivano di fra i due monti; e i monti eran monti di rame.
\par 2 Al primo carro c'erano dei cavalli rossi; al secondo carro, dei cavalli neri;
\par 3 al terzo carro dei cavalli bianchi, e al quarto carro dei cavalli chiazzati di rosso.
\par 4 E io presi a dire all'angelo che parlava meco: 'Che son questi, mio signore?'
\par 5 L'angelo rispose e mi disse: 'Questi sono i quattro venti del cielo, che escono, dopo essersi presentati al Signore di tutta la terra.
\par 6 Il carro dei cavalli neri va verso il paese del settentrione; i cavalli bianchi lo seguono; i chiazzati vanno verso il paese del mezzogiorno,
\par 7 e i rossi escono e chiedono d'andare a percorrere la terra'. E l'angelo disse loro: 'Andate, percorrete la terra!' Ed essi percorsero la terra.
\par 8 Poi egli mi chiamò, e mi parlò così: 'Ecco, quelli che escono verso il paese del settentrione acquetano la mia ira sul paese del settentrione'.
\par 9 E la parola dell'Eterno mi fu rivolta in questi termini:
\par 10 Prendi da quelli della cattività, cioè da Heldai, da Tobia e da Jedaia - e rècati oggi stesso in casa di Giosia, figliuolo di Sofonia, dov'essi sono giunti da Babilonia, -
\par 11 prendi dell'argento e dell'oro, fanne delle corone, e mettile sul capo di Giosuè, figliuolo di Jehotsadak, sommo sacerdote,
\par 12 e parlagli e digli: 'Così parla l'Eterno degli eserciti: Ecco un uomo, che ha nome il Germoglio; egli germoglierà nel suo luogo, ed edificherà il tempio dell'Eterno;
\par 13 egli edificherà il tempio dell'Eterno, e porterà le insegne della gloria, e si assiderà e dominerà sul suo trono, sarà sacerdote sul suo trono, e vi sarà fra i due un consiglio di pace.
\par 14 E le corone saranno per Helem, per Tobia, per Jedaia e per Hen, figliuolo di Sofonia, un ricordo nel tempio dell'Eterno.
\par 15 E quelli che son lontani verranno e lavoreranno alla costruzione del tempio dell'Eterno; e voi conoscerete che l'Eterno degli eserciti m'ha mandato a voi. Questo avverrà se date veramente ascolto alla voce dell'Eterno, del vostro Dio'.

\chapter{7}

\par 1 E avvenne che il quarto anno del re Dario la parola dell'Eterno fu rivolta a Zaccaria, il quarto giorno del nono mese, cioè di Chisleu.
\par 2 Quelli di Bethel avean mandato Saretser e Reghem-melec con la loro gente per implorare il favore dell'Eterno,
\par 3 e per parlare ai sacerdoti della casa dell'Eterno degli eserciti e ai profeti, in questo modo: 'Dobbiam noi continuare a piangere il quinto mese e a fare astinenza come abbiam fatto per tanti anni?'
\par 4 E la parola dell'Eterno mi fu rivolta in questi termini:
\par 5 'Parla a tutto il popolo del paese e ai sacerdoti, e di': Quando avete digiunato e fatto cordoglio il quinto e il settimo mese durante questi settant'anni, avete voi digiunato per me, proprio per me?
\par 6 E quando mangiate e quando bevete, non siete voi che mangiate, voi che bevete?
\par 7 Non dovreste voi dare ascolto alle parole che l'Eterno degli eserciti ha proclamate per mezzo dei profeti di prima, quando Gerusalemme era abitata e tranquilla, con le sue città all'intorno, ed eran pure abitati il mezzogiorno e la pianura?'
\par 8 E la parola dell'Eterno fu rivolta a Zaccaria, in questi termini:
\par 9 'Così parlava l'Eterno degli eserciti: Fate giustizia fedelmente, e mostrate l'uno per l'altro bontà e compassione;
\par 10 non opprimete la vedova né l'orfano, lo straniero né il povero; e nessuno di voi macchini del male contro il fratello nel suo cuore.
\par 11 Ma essi rifiutarono di fare attenzione, opposero una spalla ribelle, e si tapparono gli orecchi per non udire.
\par 12 Resero il loro cuore duro come il diamante, per non ascoltare la legge e le parole che l'Eterno degli eserciti mandava loro per mezzo del suo spirito, per mezzo dei profeti di prima; perciò ci fu grande indignazione da parte dell'Eterno degli eserciti.
\par 13 E avvenne che siccome egli chiamava, e quelli non davano ascolto, così quelli chiameranno, e io non darò ascolto, dice l'Eterno degli eserciti;
\par 14 e li disperderò fra tutte le nazioni ch'essi non hanno mai conosciute, e il paese rimarrà desolato dietro a loro, senza più nessuno che vi passi o vi ritorni. D'un paese delizioso essi han fatto una desolazione'.

\chapter{8}

\par 1 E la parola dell'Eterno degli eserciti mi fu rivolta in questi termini:
\par 2 'Così parla l'Eterno degli eserciti: Io provo per Sion una grande gelosia, e sono furiosamente geloso di lei.
\par 3 Così parla l'Eterno: Io torno a Sion, e dimorerò in mezzo a Gerusalemme; Gerusalemme si chiamerà la Città della fedeltà, e il monte dell'Eterno degli eserciti, Monte della santità.
\par 4 Così parla l'Eterno degli eserciti: Ci saranno ancora dei vecchi e delle vecchie che si sederanno nelle piazze di Gerusalemme, e ognuno avrà il bastone in mano a motivo della grave età.
\par 5 E le piazze della città saranno piene di ragazzi e di ragazze che si divertiranno nelle piazze.
\par 6 Così parla l'Eterno degli eserciti: Se questo parrà maraviglioso agli occhi del resto di questo popolo in quei giorni, sarà esso maraviglioso anche agli occhi miei? dice l'Eterno degli eserciti.
\par 7 Così parla l'Eterno degli eserciti: Ecco, io salvo il mio popolo dal paese del levante e dal paese del ponente;
\par 8 e li ricondurrò, ed essi abiteranno in mezzo a Gerusalemme; ed essi saranno mio popolo, e io sarò loro Dio con fedeltà e con giustizia.
\par 9 Così parla l'Eterno degli eserciti: Le vostre mani sian forti, o voi che udite in questi giorni queste parole dalla bocca dei profeti che parlarono al tempo che la casa dell'Eterno, il tempio, fu fondata, per essere ricostruita.
\par 10 Prima di quel tempo non v'era salario per il lavoro dell'uomo, né salario per il lavoro delle bestie; non v'era alcuna sicurezza per quelli che andavano e venivano, a motivo del nemico; e io mettevo gli uni alle prese con gli altri.
\par 11 Ma ora io non son più per il rimanente di questo popolo com'ero nei tempi addietro, dice l'Eterno degli eserciti.
\par 12 Poiché vi sarà sementa di pace; la vigna darà il suo frutto, il suolo i suoi prodotti, e i cieli daranno la loro rugiada; e darò al rimanente di questo popolo il possesso di tutte queste cose.
\par 13 E avverrà che, come siete stati una maledizione fra le nazioni, così, o casa di Giuda e casa d'Israele, io vi salverò e sarete una benedizione. Non temete! Le vostre mani siano forti!
\par 14 Poiché così parla l'Eterno degli eserciti: Come io pensai di farvi del male quando i vostri padri mi provocarono ad ira, dice l'Eterno degli eserciti, e non mi pentii,
\par 15 così di nuovo ho pensato in questi giorni di far del bene a Gerusalemme e alla casa di Giuda; non temete!
\par 16 Queste son le cose che dovete fare: dite la verità ciascuno al suo prossimo; fate giustizia, alle vostre porte, secondo verità e per la pace;
\par 17 nessuno macchini in cuor suo alcun male contro il suo prossimo, e non amate il falso giuramento; perché tutte queste cose io le odio, dice l'Eterno'.
\par 18 E la parola dell'Eterno degli eserciti mi fu rivolta in questi termini:
\par 19 'Così parla l'Eterno degli eserciti: Il digiuno del quarto, il digiuno del quinto, il digiuno del settimo e il digiuno del decimo mese diventeranno per la casa di Giuda una gioia, un gaudio, delle feste d'esultanza; amate dunque la verità e la pace.
\par 20 Così parla l'Eterno degli eserciti: Verranno ancora dei popoli e gli abitanti di molte città;
\par 21 e gli abitanti dell'una andranno all'altra e diranno: Andiamo, andiamo a implorare il favore dell'Eterno, e a cercare l'Eterno degli eserciti! Anch'io voglio andare!
\par 22 E molti popoli e delle nazioni potenti verranno a cercare l'Eterno degli eserciti a Gerusalemme, e a implorare il favore dell'Eterno.
\par 23 Così parla l'Eterno degli eserciti: In quei giorni avverrà che dieci uomini di tutte le lingue delle nazioni piglieranno un Giudeo per il lembo della veste, e diranno: Noi andremo con voi, perché abbiamo udito che Dio è con voi'.

\chapter{9}

\par 1 Oracolo, parola dell'Eterno, contro il paese di Hadrac, e che si ferma sopra Damasco; poiché l'Eterno ha l'occhio su tutti gli uomini e su tutte le tribù d'Israele.
\par 2 Essa si ferma pure sopra Hamath, ai confini di Damasco, su Tiro e Sidone perché son così savie!
\par 3 Tiro s'è costruita una fortezza, ed ha ammassato argento come polvere, e oro come fango di strada.
\par 4 Ecco, l'Eterno s'impadronirà di essa, getterà la sua potenza nel mare, ed essa sarà consumata dal fuoco.
\par 5 Askalon lo vedrà e avrà paura; anche Gaza, e si torcerà dal gran dolore; e così Ekron, perché la sua speranza sarà confusa; e Gaza non avrà più re, e Askalon non sarà più abitata.
\par 6 Dei bastardi abiteranno in Asdod, ed io annienterò l'orgoglio dei Filistei.
\par 7 Ma io toglierò il sangue dalla bocca del Filisteo e le abominazioni di fra i suoi denti, e anch'egli sarà un residuo per il nostro Dio; sarà come un capo in Giuda, ed Ekron, come il Gebuseo.
\par 8 Ed io m'accamperò attorno alla mia casa per difenderla da ogni esercito, da chi va e viene; e nessun esattor di tributi passerà più da loro, perché ora ho visto con gli occhi miei.
\par 9 Esulta grandemente, o figliuola di Sion, manda gridi d'allegrezza, o figliuola di Gerusalemme; ecco, il tuo re viene a te; egli è giusto e vittorioso, umile e montato sopra un asino, sopra un puledro d'asina.
\par 10 Io farò sparire i carri da Efraim, i cavalli da Gerusalemme, e gli archi di guerra saranno annientati. Egli parlerà di pace alle nazioni, il suo dominio si estenderà da un mare all'altro, e dal fiume sino alle estremità della terra.
\par 11 E te pure, Israele, a motivo del sangue del tuo patto, io trarrò i tuoi prigionieri dalla fossa senz'acqua.
\par 12 Tornate alla fortezza, o voi prigionieri della speranza! Anch'oggi io ti dichiaro che ti renderò il doppio.
\par 13 Poiché io piego Giuda come un arco, armo l'arco con Efraim, e solleverò i tuoi figliuoli, o Sion, contro i tuoi figliuoli, o Javan, e ti renderò simile alla spada di un prode.
\par 14 L'Eterno apparirà sopra di loro, e la sua freccia partirà come il lampo. Il Signore, l'Eterno, sonerà la tromba, e avanzerà coi turbini del mezzogiorno.
\par 15 L'Eterno degli eserciti li proteggerà; ed essi divoreranno, calpesteranno le pietre di fionda: berranno, schiamazzeranno come eccitati dal vino, e saran pieni come coppe da sacrifizi, come i canti dell'altare.
\par 16 E l'Eterno, il loro Dio, li salverà, in quel giorno, come il gregge del suo popolo; poiché saranno come pietre d'un diadema, che rifulgeranno sulla sua terra.
\par 17 Poiché qual prosperità sarà la loro! e quanta sarà la loro bellezza! Il grano farà crescere i giovani, e il mosto le fanciulle.

\chapter{10}

\par 1 Chiedete all'Eterno la pioggia nella stagione di primavera! L'Eterno che produce i lampi, darà loro abbondanza di pioggia, ad ognuno erba nel proprio campo.
\par 2 Poiché gl'idoli domestici dicono cose vane, gl'indovini vedono menzogne, i sogni mentiscono e danno un vano conforto; perciò costoro vanno errando come pecore, sono afflitti, perché non v'è pastore.
\par 3 La mia ira s'è accesa contro i pastori, e io punirò i capri; poiché l'Eterno degli eserciti visita il suo gregge, la casa di Giuda, e ne fa come il suo cavallo d'onore nella battaglia.
\par 4 Da lui uscirà la pietra angolare, da lui il piuolo, da lui l'arco di battaglia, da lui usciranno tutti i capi assieme.
\par 5 E saranno come prodi che calpesteranno il nemico in battaglia, nel fango delle strade; e combatteranno perché l'Eterno è con loro; ma quelli che son montati sui cavalli saran confusi.
\par 6 E io fortificherò la casa di Giuda, e salverò la casa di Giuseppe, e li ricondurrò perché ho pietà di loro; e saranno come se non li avessi mai scacciati, perché io sono l'Eterno, il loro Dio, e li esaudirò.
\par 7 E quelli d'Efraim saranno come un prode, e il loro cuore si rallegrerà come per effetto del vino; i loro figliuoli lo vedranno e si rallegreranno, il loro cuore esulterà nell'Eterno.
\par 8 Io fischierò loro e li raccoglierò, perché io li voglio riscattare; ed essi moltiplicheranno come già moltiplicarono.
\par 9 Io li disseminerò fra i popoli, ed essi si ricorderanno di me nei paesi lontani; e vivranno coi loro figliuoli, e torneranno.
\par 10 Io li farò tornare dal paese d'Egitto, e li raccoglierò dall'Assiria; li farò venire nel paese di Galaad e al Libano, e non vi si troverà posto sufficiente per loro.
\par 11 Egli passerà per il mare della distretta; ma nel mare egli colpirà i flutti, e tutte le profondità del fiume saranno prosciugate; l'orgoglio dell'Assiria sarà abbattuto, e lo scettro d'Egitto sarà tolto via.
\par 12 Io li renderò forti nell'Eterno, ed essi cammineranno nel suo nome, dice l'Eterno.

\chapter{11}

\par 1 Libano, apri le tue porte, e il fuoco divori i tuoi cedri!
\par 2 Urla, cipresso, perché il cedro è caduto, e gli alberi magnifici son devastati! Urlate, querce di Basan, perché la foresta impenetrabile è abbattuta!
\par 3 S'odono i lamenti de' pastori perché la loro magnificenza è devastata; s'ode il ruggito dei leoncelli perché le rive lussureggianti del Giordano son devastate.
\par 4 Così parla l'Eterno, il mio Dio: 'Pasci le mie pecore destinate al macello,
\par 5 che i compratori uccidono senza rendersi colpevoli, e delle quali i venditori dicono: - Sia benedetto l'Eterno! io m'arricchisco, - e che i loro pastori non risparmiano affatto.
\par 6 Poiché io non risparmierò più gli abitanti del paese, dice l'Eterno, anzi, ecco, io abbandonerò gli uomini, ognuno in balìa del suo prossimo e in balìa del suo re; essi schiacceranno il paese, e io non libererò alcuno dalle lor mani'.
\par 7 Allora io mi misi a pascere le pecore destinate al macello, e perciò le più misere del gregge; e mi presi due verghe; chiamai l'una Favore e l'altra Vincoli, e mi misi a pascere il gregge.
\par 8 E sterminai i tre pastori in un mese; l'anima mia perdette la pazienza con loro, e anche l'anima loro m'avea preso a sdegno.
\par 9 E io dissi: 'Non vi pascerò più; la moribonda muoia, quella che sta per perire perisca, e quelle che restano, divorino l'una la carne dell'altra'.
\par 10 E presi la mia verga Favore e la spezzai, per annullare il patto che avevo stretto con tutti i popoli.
\par 11 E quello fu annullato in quel giorno; e le pecore più misere del gregge che m'osservavano, conobbero che quella era la parola dell'Eterno.
\par 12 E io dissi loro: 'Se vi par bene, datemi il mio salario; se no, lasciate stare'. Ed essi mi pesarono il mio salario; trenta sicli d'argento.
\par 13 E l'Eterno mi disse: 'Gettalo per il vasaio, questo magnifico prezzo al quale m'hanno stimato!' E io presi i trenta sicli d'argento, e li gettai nella casa dell'Eterno per il vasaio.
\par 14 Poi spezzai l'altra verga Vincoli, per rompere la fratellanza fra Giuda e Israele.
\par 15 E l'Eterno mi disse: 'Prenditi anche gli arnesi d'un pastore insensato.
\par 16 Perché, ecco, io susciterò nel paese un pastore che non si curerà delle pecore che periscono, non cercherà le disperse, non guarirà le ferite, non nutrirà quelle che stanno in piè, ma mangerà la carne delle grasse, e strapperà loro fino le unghie'.
\par 17 Guai al pastore da nulla, che abbandona il gregge! La spada gli colpirà il braccio e l'occhio destro. Il braccio gli seccherà del tutto, e l'occhio destro gli si spegnerà interamente.

\chapter{12}

\par 1 Oracolo, parola dell'Eterno, riguardo a Israele. Parola dell'Eterno che ha disteso i cieli e fondata la terra, e che ha formato lo spirito dell'uomo dentro di lui:
\par 2 Ecco, io farò di Gerusalemme una coppa di stordimento per tutti i popoli all'intorno; e questo concernerà anche Giuda, quando si cingerà d'assedio Gerusalemme.
\par 3 E in quel giorno avverrà che io farò di Gerusalemme una pietra pesante per tutti i popoli; tutti quelli che se la caricheranno addosso ne saranno malamente feriti, e tutte le nazioni della terra s'aduneranno contro di lei.
\par 4 In quel giorno, dice l'Eterno, io colpirò di smarrimento tutti i cavalli, e di delirio quelli che li montano; io aprirò i miei occhi sulla casa di Giuda, ma colpirò di cecità tutti i cavalli dei popoli.
\par 5 E i capi di Giuda diranno in cuor loro: 'Gli abitanti di Gerusalemme son la mia forza nell'Eterno degli eserciti, loro Dio'.
\par 6 In quel giorno, io renderò i capi di Giuda come un braciere ardente in mezzo a delle legna, come una torcia accesa in mezzo a dei covoni; essi divoreranno a destra e a sinistra tutti i popoli d'ogn'intorno; e Gerusalemme sarà ancora abitata nel suo proprio luogo, a Gerusalemme.
\par 7 L'Eterno salverà prima le tende di Giuda, perché la gloria della casa di Davide e la gloria degli abitanti di Gerusalemme non s'innalzi al disopra di Giuda.
\par 8 In quel giorno l'Eterno proteggerà gli abitanti di Gerusalemme; e colui che fra loro vacilla sarà in quel giorno come Davide, e la casa di Davide sarà come Dio, come l'angelo dell'Eterno davanti a loro.
\par 9 E in quel giorno avverrà che io avrò cura di distruggere tutte le nazioni che verranno contro Gerusalemme.
\par 10 E spanderò sulla casa di Davide e sugli abitanti di Gerusalemme lo spirito di grazia e di supplicazione; ed essi riguarderanno a me, a colui ch'essi hanno trafitto, e ne faran cordoglio come si fa cordoglio per un figliuolo unico, e lo piangeranno amaramente come si piange amaramente un primogenito.
\par 11 In quel giorno vi sarà un gran lutto in Gerusalemme, pari al lutto di Hadadrimmon nella valle di Meghiddon.
\par 12 E il paese farà cordoglio, ogni famiglia da sé; la famiglia della casa di Davide da sé, e le loro mogli da sé; la famiglia della casa di Nathan da sé, e le loro mogli da sé;
\par 13 la famiglia della casa di Levi da sé, e le loro mogli da sé; la famiglia degli Scimeiti da sé, e le loro mogli da sé;
\par 14 tutte le famiglie rimaste ognuna da sé, e le mogli da sé.

\chapter{13}

\par 1 In quel giorno vi sarà una fonte aperta per la casa di Davide e per gli abitanti di Gerusalemme, per il peccato e per l'impurità.
\par 2 E in quel giorno avverrà, dice l'Eterno degli eserciti, che io sterminerò dal paese i nomi degl'idoli, e non se ne farà più menzione; e i profeti pure, e gli spiriti immondi farò sparire dal paese.
\par 3 E avverrà che, se qualcuno farà ancora il profeta, suo padre e sua madre che l'hanno generato gli diranno: 'Tu non vivrai, perché dici delle menzogne nel nome dell'Eterno'; e suo padre e sua madre che l'hanno generato lo trafiggeranno perché fa il profeta.
\par 4 E in quel giorno avverrà che i profeti avranno vergogna, ognuno della visione che proferiva quando profetava; e non si metteranno più il mantello di pelo per mentire.
\par 5 E ognuno d'essi dirà: 'Io non son profeta; sono un coltivatore del suolo; qualcuno mi comprò fin dalla mia giovinezza'.
\par 6 E gli si dirà: 'Che son quelle ferite che hai nelle mani?' Ed egli risponderà: 'Son le ferite che ho ricevuto nella casa dei miei amici'.
\par 7 Dèstati, o spada, contro il mio pastore, e contro l'uomo che mi è compagno! dice l'Eterno degli eserciti. Colpisci il pastore, e sian disperse le pecore! ma io volgerò la mia mano sui piccoli.
\par 8 E in tutto il paese avverrà, dice l'Eterno, che i due terzi vi saranno sterminati, periranno, ma l'altro terzo vi sarà lasciato.
\par 9 E metterò quel terzo nel fuoco, e lo affinerò come si affina l'argento, lo proverò come si prova l'oro; essi invocheranno il mio nome e io li esaudirò; io dirò: 'È il mio popolo!' ed esso dirà: 'L'Eterno è il mio Dio!'

\chapter{14}

\par 1 Ecco, viene un giorno dell'Eterno, in cui le tue spoglie saranno spartite in mezzo a te.
\par 2 Io adunerò tutte le nazioni per far guerra a Gerusalemme, e la città sarà presa, le case saranno saccheggiate, e le donne violate; la metà della città andrà in cattività, ma il resto del popolo non sarà sterminato dalla città.
\par 3 Poi l'Eterno si farà innanzi e combatterà contro quelle nazioni, com'egli combatté, le tante volte, il dì della battaglia.
\par 4 I suoi piedi si poseranno in quel giorno sul monte degli Ulivi ch'è dirimpetto a Gerusalemme a levante, e il monte degli Ulivi si spaccherà per il mezzo, da levante a ponente, sì da formare una gran valle; e metà del monte si ritirerà verso settentrione, e l'altra metà verso mezzogiorno.
\par 5 E voi fuggirete per la valle de' miei monti, poiché la valle de' monti s'estenderà fino ad Atsal; fuggirete, come fuggiste davanti al terremoto ai giorni di Uzzia, re di Giuda; e l'Eterno, il mio Dio, verrà, e tutti i suoi santi con lui.
\par 6 E in quel giorno avverrà che non vi sarà più luce; gli astri brillanti ritireranno il loro splendore.
\par 7 Sarà un giorno unico, conosciuto dall'Eterno; non sarà né giorno né notte, ma in sulla sera vi sarà luce.
\par 8 E in quel giorno avverrà che delle acque vive usciranno da Gerusalemme; metà delle quali volgerà verso il mare orientale, e metà verso il mare occidentale, tanto d'estate quanto d'inverno.
\par 9 E l'Eterno sarà re di tutta la terra; in quel giorno l'Eterno sarà l'unico, e unico sarà il suo nome.
\par 10 Tutto il paese sarà mutato in pianura, da Gheba a Rimmon a mezzogiorno di Gerusalemme; e Gerusalemme sarà innalzata e abitata nel suo luogo, dalla porta di Beniamino al luogo della prima porta, la porta degli angoli, e dalla torre di Hananeel agli strettoi del re.
\par 11 E la gente abiterà in essa, e non ci sarà più nulla di votato allo sterminio, e Gerusalemme se ne starà al sicuro.
\par 12 E questa sarà la piaga con la quale l'Eterno colpirà tutti i popoli che avran mosso guerra a Gerusalemme: la loro carne si consumerà mentre stanno in piedi, gli occhi si struggeranno loro nelle orbite, la lor lingua si consumerà nella lor bocca.
\par 13 E avverrà in quel giorno che vi sarà tra loro un gran tumulto prodotto dall'Eterno; ognun d'essi afferrerà la mano dell'altro, e la mano dell'uno si leverà contro la mano dell'altro.
\par 14 E Giuda stesso combatterà contro Gerusalemme; e le ricchezze di tutte le nazioni all'intorno saranno ammassate: oro, argento, vesti in grande abbondanza.
\par 15 E la piaga che colpirà i cavalli, i muli, i cammelli, gli asini e tutte le bestie che saranno in quegli accampamenti, sarà simile a quell'altra piaga.
\par 16 E avverrà che tutti quelli che saran rimasti di tutte le nazioni venute contro Gerusalemme, saliranno d'anno in anno a prostrarsi davanti al Re, all'Eterno degli eserciti, e a celebrare la festa delle Capanne.
\par 17 E quanto a quelli delle famiglie della terra che non saliranno a Gerusalemme per prostrarsi davanti al Re, all'Eterno degli eserciti, non cadrà pioggia su loro.
\par 18 E se la famiglia d'Egitto non sale e non viene, neppur su lei ne cadrà; sarà colpita dalla piaga con cui l'Eterno colpirà le nazioni che non saliranno a celebrare la festa delle Capanne.
\par 19 Tale sarà la punizione dell'Egitto, e la punizione di tutte le nazioni che non saliranno a celebrare la festa delle Capanne.
\par 20 In quel giorno si leggerà sui sonagli dei cavalli: SANTITÀ ALL'ETERNO! E le caldaie nella casa dell'Eterno saranno come i bacini davanti all'altare.
\par 21 Ogni caldaia in Gerusalemme ed in Giuda sarà consacrata all'Eterno degli eserciti; tutti quelli che offriranno sacrifizi ne verranno a prendere per cuocervi le carni; e in quel giorno non vi saran più Cananei nella casa dell'Eterno degli eserciti.


\end{document}