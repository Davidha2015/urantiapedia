\begin{document}

\title{Matthew}


\chapter{1}

\par 1 Genealogia di Gesù Cristo figliuolo di Davide, figliuolo d'Abramo.
\par 2 Abramo generò Isacco; Isacco generò Giacobbe; Giacobbe generò Giuda e i suoi fratelli;
\par 3 Giuda generò Fares e Zara da Tamar; Fares generò Esrom; Esrom generò Aram;
\par 4 Aram generò Aminadab; Aminadab generò Naasson; Naasson generò Salmon;
\par 5 Salmon generò Booz da Rahab; Booz generò Obed da Ruth; Obed generò Iesse,
\par 6 e Iesse generò Davide, il re. E Davide generò Salomone da quella ch'era stata moglie d'Uria;
\par 7 Salomone generò Roboamo; Roboamo generò Abia; Abia generò Asa;
\par 8 Asa generò Giosafat; Giosafat generò Ioram; Ioram generò Uzzia;
\par 9 Uzzia generò Ioatam; Ioatam generò Achaz; Achaz generò Ezechia;
\par 10 Ezechia generò Manasse; Manasse generò Amon; Amon generò Giosia;
\par 11 Giosia generò Ieconia e i suoi fratelli al tempo della deportazione in Babilonia.
\par 12 E dopo la deportazione in Babilonia, Ieconia generò Salatiel; Salatiel generò Zorobabel;
\par 13 Zorobabel generò Abiud; Abiud generò Eliachim; Eliachim generò Azor;
\par 14 Azor generò Sadoc; Sadoc generò Achim; Achim generò Eliud;
\par 15 Eliud generò Eleazaro; Eleazaro generò Mattan; Mattan generò Giacobbe;
\par 16 Giacobbe generò Giuseppe, il marito di Maria, dalla quale nacque Gesù, che è chiamato Cristo.
\par 17 Così da Abramo fino a Davide sono in tutto quattordici generazioni; e da Davide fino alla deportazione in Babilonia, quattordici generazioni; e dalla deportazione in Babilonia fino a Cristo, quattordici generazioni.
\par 18 Or la nascita di Gesù Cristo avvenne in questo modo. Maria, sua madre, era stata promessa sposa a Giuseppe; e prima che fossero venuti a stare insieme, si trovò incinta per virtù dello Spirito Santo.
\par 19 E Giuseppe, suo marito, essendo uomo giusto e non volendo esporla ad infamia, si propose di lasciarla occultamente.
\par 20 Ma mentre avea queste cose nell'animo, ecco che un angelo del Signore gli apparve in sogno, dicendo: Giuseppe, figliuol di Davide, non temere di prender teco Maria tua moglie; perché ciò che in lei è generato, è dallo Spirito Santo.
\par 21 Ed ella partorirà un figliuolo, e tu gli porrai nome Gesù, perché è lui che salverà il suo popolo dai loro peccati.
\par 22 Or tutto ciò avvenne, affinché si adempiesse quello che era stato detto dal Signore per mezzo del profeta:
\par 23 Ecco, la vergine sarà incinta e partorirà un figliuolo, al quale sarà posto nome Emmanuele, che, interpretato, vuol dire: "Iddio con noi".
\par 24 E Giuseppe, destatosi dal sonno, fece come l'angelo del Signore gli avea comandato, e prese con sé sua moglie;
\par 25 e non la conobbe finch'ella non ebbe partorito un figlio; e gli pose nome Gesù.

\chapter{2}

\par 1 Or essendo Gesù nato in Betleem di Giudea, ai dì del re Erode, ecco dei magi d'Oriente arrivarono in Gerusalemme, dicendo:
\par 2 Dov'è il re de' Giudei che è nato? Poiché noi abbiam veduto la sua stella in Oriente e siam venuti per adorarlo.
\par 3 Udito questo, il re Erode fu turbato, e tutta Gerusalemme con lui.
\par 4 E radunati tutti i capi sacerdoti e gli scribi del popolo, s'informò da loro dove il Cristo dovea nascere.
\par 5 Ed essi gli dissero: In Betleem di Giudea; poiché così è scritto per mezzo del profeta:
\par 6 E tu, Betleem, terra di Giuda, non sei punto la minima fra le città principali di Giuda; perché da te uscirà un Principe, che pascerà il mio popolo Israele.
\par 7 Allora Erode, chiamati di nascosto i magi, s'informò esattamente da loro del tempo in cui la stella era apparita;
\par 8 e mandandoli a Betleem, disse loro: Andate e domandate diligentemente del fanciullino; e quando lo avrete trovato, fatemelo sapere, affinché io pure venga ad adorarlo.
\par 9 Essi dunque, udito il re, partirono; ed ecco la stella che aveano veduta in Oriente, andava dinanzi a loro, finché, giunta al luogo dov'era il fanciullino, vi si fermò sopra.
\par 10 Ed essi, veduta la stella, si rallegrarono di grandissima allegrezza.
\par 11 Ed entrati nella casa, videro il fanciullino con Maria sua madre; e prostratisi, lo adorarono; ed aperti i loro tesori, gli offrirono dei doni: oro, incenso e mirra.
\par 12 Poi, essendo stati divinamente avvertiti in sogno di non ripassare da Erode, per altra via tornarono al loro paese.
\par 13 Partiti che furono, ecco un angelo del Signore apparve in sogno a Giuseppe e gli disse: Lèvati, prendi il fanciullino e sua madre, e fuggi in Egitto, e sta' quivi finch'io non tel dica; perché Erode cercherà il fanciullino per farlo morire.
\par 14 Egli dunque, levatosi, prese di notte il fanciullino e sua madre, e si ritirò in Egitto;
\par 15 ed ivi stette fino alla morte di Erode, affinché si adempiesse quello che fu detto dal Signore per mezzo del profeta: Fuor d'Egitto chiamai il mio figliuolo.
\par 16 Allora Erode, vedutosi beffato dai magi, si adirò gravemente, e mandò ad uccidere tutti i maschi ch'erano in Betleem e in tutto il suo territorio dall'età di due anni in giù, secondo il tempo del quale s'era esattamente informato dai magi.
\par 17 Allora si adempié quello che fu detto per bocca del profeta Geremia:
\par 18 Un grido è stato udito in Rama; un pianto ed un lamento grande: Rachele piange i suoi figliuoli e ricusa d'esser consolata, perché non sono più.
\par 19 Ma dopo che Erode fu morto, ecco un angelo del Signore apparve in sogno a Giuseppe in Egitto, e gli disse:
\par 20 Lèvati, prendi il fanciullino e sua madre, e vattene nel paese d'Israele; perché son morti coloro che cercavano la vita del fanciullino.
\par 21 Ed egli, levatosi, prese il fanciullino e sua madre ed entrò nel paese d'Israele.
\par 22 Ma udito che in Giudea regnava Archelao invece d'Erode, suo padre, temette d'andar colà; ed essendo stato divinamente avvertito in sogno, si ritirò nelle parti della Galilea;
\par 23 e venne ad abitare in una città detta Nazaret, affinché si adempiesse quello ch'era stato detto dai profeti, ch'egli sarebbe chiamato Nazareno.

\chapter{3}

\par 1 Or in que' giorni comparve Giovanni il Battista, predicando nel deserto della Giudea e dicendo:
\par 2 Ravvedetevi, poiché il regno de' cieli è vicino.
\par 3 Di lui parlò infatti il profeta Isaia quando disse: V'è una voce d'uno che grida nel deserto: Preparate la via del Signore, addirizzate i suoi sentieri.
\par 4 Or esso Giovanni aveva il vestimento di pelo di cammello ed una cintura di cuoio intorno a' fianchi; ed il suo cibo erano locuste e miele selvatico.
\par 5 Allora Gerusalemme e tutta la Giudea e tutto il paese d'intorno al Giordano presero ad accorrere a lui;
\par 6 ed erano battezzati da lui nel fiume Giordano, confessando i loro peccati.
\par 7 Ma vedendo egli molti dei Farisei e dei Sadducei venire al suo battesimo, disse loro: Razza di vipere, chi v'ha insegnato a fuggir dall'ira a venire?
\par 8 Fate dunque de' frutti degni del ravvedimento.
\par 9 E non pensate di dir dentro di voi: Abbiamo per padre Abramo; perché io vi dico che Iddio può da queste pietre far sorgere de' figliuoli ad Abramo.
\par 10 E già la scure è posta alla radice degli alberi; ogni albero dunque che non fa buon frutto, sta per esser tagliato e gittato nel fuoco.
\par 11 Ben vi battezzo io con acqua, in vista del ravvedimento; ma colui che viene dietro a me è più forte di me, ed io non son degno di portargli i calzari; egli vi battezzerà con lo Spirito Santo e con fuoco.
\par 12 Egli ha il suo ventilabro in mano, e netterà interamente l'aia sua, e raccoglierà il suo grano nel granaio, ma arderà la pula con fuoco inestinguibile.
\par 13 Allora Gesù dalla Galilea si recò al Giordano da Giovanni per esser da lui battezzato.
\par 14 Ma questi vi si opponeva dicendo: Son io che ho bisogno d'esser battezzato da te, e tu vieni a me?
\par 15 Ma Gesù gli rispose: Lascia fare per ora; poiché conviene che noi adempiamo così ogni giustizia. Allora Giovanni lo lasciò fare.
\par 16 E Gesù, tosto che fu battezzato, salì fuor dell'acqua; ed ecco i cieli s'apersero, ed egli vide lo Spirito di Dio scendere come una colomba e venir sopra lui.
\par 17 Ed ecco una voce dai cieli che disse: Questo è il mio diletto Figliuolo, nel quale mi sono compiaciuto.

\chapter{4}

\par 1 Allora Gesù fu condotto dallo Spirito su nel deserto, per esser tentato dal diavolo.
\par 2 E dopo che ebbe digiunato quaranta giorni e quaranta notti, alla fine ebbe fame.
\par 3 E il tentatore, accostatosi, gli disse: Se tu sei Figliuol di Dio, di' che queste pietre divengan pani.
\par 4 Ma egli rispondendo disse: Sta scritto: Non di pane soltanto vivrà l'uomo, ma d'ogni parola che procede dalla bocca di Dio.
\par 5 Allora il diavolo lo menò seco nella santa città e lo pose sul pinnacolo del tempio,
\par 6 e gli disse: Se tu sei Figliuol di Dio, gettati giù; poiché sta scritto: Egli darà ordine ai suoi angeli intorno a te, ed essi ti porteranno sulle loro mani, che talora tu non urti col piede contro una pietra.
\par 7 Gesù gli disse: Egli è altresì scritto: Non tentare il Signore Iddio tuo.
\par 8 Di nuovo il diavolo lo menò seco sopra un monte altissimo, e gli mostrò tutti i regni del mondo e la lor gloria, e gli disse:
\par 9 Tutte queste cose io te le darò, se, prostrandoti, tu mi adori.
\par 10 Allora Gesù gli disse: Va', Satana, poiché sta scritto: Adora il Signore Iddio tuo, ed a lui solo rendi il culto.
\par 11 Allora il diavolo lo lasciò; ed ecco degli angeli vennero a lui e lo servivano.
\par 12 Or Gesù, avendo udito che Giovanni era stato messo in prigione, si ritirò in Galilea.
\par 13 E, lasciata Nazaret, venne ad abitare in Capernaum, città sul mare, ai confini di Zabulon e di Neftali,
\par 14 affinché si adempiesse quello ch'era stato detto dal profeta Isaia:
\par 15 Il paese di Zabulon e il paese di Neftali, sulla via del mare, al di là del Giordano, la Galilea dei Gentili,
\par 16 il popolo che giaceva nelle tenebre, ha veduto una gran luce; su quelli che giacevano nella contrada e nell'ombra della morte, una luce s'è levata.
\par 17 Da quel tempo Gesù cominciò a predicare e a dire: Ravvedetevi, perché il regno de' cieli è vicino.
\par 18 Or passeggiando lungo il mare della Galilea, egli vide due fratelli, Simone detto Pietro, e Andrea suo fratello, i quali gettavano la rete in mare; poiché erano pescatori.
\par 19 E disse loro: Venite dietro a me, e vi farò pescatori d'uomini.
\par 20 Ed essi, lasciate prontamente le reti, lo seguirono.
\par 21 E passato più oltre, vide due altri fratelli, Giacomo di Zebedeo e Giovanni, suo fratello, i quali nella barca, con Zebedeo loro padre, rassettavano le reti: e li chiamò.
\par 22 Ed essi, lasciata subito la barca e il padre loro, lo seguirono.
\par 23 E Gesù andava attorno per tutta la Galilea, insegnando nelle loro sinagoghe e predicando l'evangelo del Regno, sanando ogni malattia ed ogni infermità fra il popolo.
\par 24 E la sua fama si sparse per tutta la Siria; e gli recarono tutti i malati colpiti da varie infermità e da varî dolori, indemoniati, lunatici, paralitici; ed ei li guarì.
\par 25 E grandi folle lo seguirono dalla Galilea e dalla Decapoli e da Gerusalemme e dalla Giudea e d'oltre il Giordano.

\chapter{5}

\par 1 E Gesù, vedendo le folle, salì sul monte; e postosi a sedere, i suoi discepoli si accostarono a lui.
\par 2 Ed egli, aperta la bocca, li ammaestrava dicendo:
\par 3 Beati i poveri in ispirito, perché di loro è il regno de' cieli.
\par 4 Beati quelli che fanno cordoglio, perché essi saranno consolati.
\par 5 Beati i mansueti, perché essi erederanno la terra.
\par 6 Beati quelli che sono affamati ed assetati della giustizia, perché essi saranno saziati.
\par 7 Beati i misericordiosi, perché a loro misericordia sarà fatta.
\par 8 Beati i puri di cuore, perché essi vedranno Iddio.
\par 9 Beati quelli che s'adoperano alla pace, perché essi saran chiamati figliuoli di Dio.
\par 10 Beati i perseguitati per cagion di giustizia, perché di loro è il regno dei cieli.
\par 11 Beati voi, quando v'oltraggeranno e vi perseguiteranno e, mentendo, diranno contro di voi ogni sorta di male per cagion mia.
\par 12 Rallegratevi e giubilate, perché il vostro premio è grande ne' cieli; poiché così hanno perseguitato i profeti che sono stati prima di voi.
\par 13 Voi siete il sale della terra; ora, se il sale diviene insipido, con che lo si salerà? Non è più buono a nulla se non ad esser gettato via e calpestato dagli uomini.
\par 14 Voi siete la luce del mondo; una città posta sopra un monte non può rimaner nascosta;
\par 15 e non si accende una lampada per metterla sotto il moggio; anzi la si mette sul candeliere ed ella fa lume a tutti quelli che sono in casa.
\par 16 Così risplenda la vostra luce nel cospetto degli uomini, affinché veggano le vostre buone opere e glorifichino il Padre vostro che è ne' cieli.
\par 17 Non pensate ch'io sia venuto per abolire la legge od i profeti; io son venuto non per abolire ma per compire:
\par 18 poiché io vi dico in verità che finché non siano passati il cielo e la terra, neppure un iota o un apice della legge passerà, che tutto non sia adempiuto.
\par 19 Chi dunque avrà violato uno di questi minimi comandamenti ed avrà così insegnato agli uomini, sarà chiamato minimo nel regno de' cieli; ma chi li avrà messi in pratica ed insegnati, esso sarà chiamato grande nel regno dei cieli.
\par 20 Poiché io vi dico che se la vostra giustizia non supera quella degli scribi e de' Farisei, voi non entrerete punto nel regno dei cieli.
\par 21 Voi avete udito che fu detto agli antichi: Non uccidere, e Chiunque avrà ucciso sarà sottoposto al tribunale;
\par 22 ma io vi dico: Chiunque s'adira contro al suo fratello, sarà sottoposto al tribunale; e chi avrà detto al suo fratello 'raca', sarà sottoposto al Sinedrio; e chi gli avrà detto 'pazzo', sarà condannato alla geenna del fuoco.
\par 23 Se dunque tu stai per offrire la tua offerta sull'altare, e quivi ti ricordi che il tuo fratello ha qualcosa contro di te,
\par 24 lascia quivi la tua offerta dinanzi all'altare, e va' prima a riconciliarti col tuo fratello; e poi vieni ad offrir la tua offerta.
\par 25 Fa' presto amichevole accordo col tuo avversario mentre sei ancora per via con lui; che talora il tuo avversario non ti dia in man del giudice, e il giudice in man delle guardie, e tu sii cacciato in prigione.
\par 26 Io ti dico in verità che di là non uscirai, finché tu non abbia pagato l'ultimo quattrino.
\par 27 Voi avete udito che fu detto: Non commettere adulterio.
\par 28 Ma io vi dico che chiunque guarda una donna per appetirla, ha già commesso adulterio con lei nel suo cuore.
\par 29 Ora, se l'occhio tuo destro ti fa cadere in peccato, cavalo e gettalo via da te; poiché val meglio per te che uno dei tuoi membri perisca, e non sia gettato l'intero tuo corpo nella geenna.
\par 30 E se la tua man destra ti fa cadere in peccato, mozzala e gettala via da te; poiché val meglio per te che uno dei tuoi membri perisca, e non vada l'intero tuo corpo nella geenna.
\par 31 Fu detto: Chiunque ripudia sua moglie, le dia l'atto del divorzio.
\par 32 Ma io vi dico: Chiunque manda via la moglie, salvo che per cagion di fornicazione, la fa essere adultera; e chiunque sposa colei ch'è mandata via, commette adulterio.
\par 33 Avete udito pure che fu detto agli antichi: Non ispergiurare, ma attieni al Signore i tuoi giuramenti.
\par 34 Ma io vi dico: Del tutto non giurate, né per il cielo, perché è il trono di Dio;
\par 35 né per la terra, perché è lo sgabello dei suoi piedi; né per Gerusalemme, perché è la città del gran Re.
\par 36 Non giurar neppure per il tuo capo, poiché tu non puoi fare un solo capello bianco o nero.
\par 37 Ma sia il vostro parlare: Sì, sì; no, no; poiché il di più vien dal maligno.
\par 38 Voi avete udito che fu detto: Occhio per occhio e dente per dente.
\par 39 Ma io vi dico: Non contrastate al malvagio; anzi, se uno ti percuote sulla guancia destra, porgigli anche l'altra;
\par 40 ed a chi vuol litigar teco e toglierti la tunica, lasciagli anche il mantello.
\par 41 E se uno ti vuol costringere a far seco un miglio, fanne con lui due.
\par 42 Da' a chi ti chiede, e a chi desidera da te un imprestito, non voltar le spalle.
\par 43 Voi avete udito che fu detto: Ama il tuo prossimo e odia il tuo nemico.
\par 44 Ma io vi dico: Amate i vostri nemici e pregate per quelli che vi perseguitano,
\par 45 affinché siate figliuoli del Padre vostro che è nei cieli; poiché Egli fa levare il suo sole sopra i malvagi e sopra i buoni, e fa piovere sui giusti e sugli ingiusti.
\par 46 Se infatti amate quelli che vi amano, che premio ne avete? Non fanno anche i pubblicani lo stesso?
\par 47 E se fate accoglienze soltanto ai vostri fratelli, che fate di singolare? Non fanno anche i pagani altrettanto?
\par 48 Voi dunque siate perfetti, com'è perfetto il Padre vostro celeste.

\chapter{6}

\par 1 Guardatevi dal praticare la vostra giustizia nel cospetto degli uomini per esser osservati da loro; altrimenti non ne avrete premio presso il Padre vostro che è nei cieli.
\par 2 Quando dunque fai limosina, non far sonar la tromba dinanzi a te, come fanno gl'ipocriti nelle sinagoghe e nelle strade, per essere onorati dagli uomini. Io vi dico in verità che cotesto è il premio che ne hanno.
\par 3 Ma quando tu fai limosina, non sappia la tua sinistra quel che fa la destra,
\par 4 affinché la tua limosina si faccia in segreto; e il Padre tuo che vede nel segreto, te ne darà la ricompensa.
\par 5 E quando pregate, non siate come gl'ipocriti; poiché essi amano di fare orazione stando in piè nelle sinagoghe e ai canti delle piazze per esser veduti dagli uomini. Io vi dico in verità che cotesto è il premio che ne hanno.
\par 6 Ma tu, quando preghi, entra nella tua cameretta, e serratone l'uscio fa' orazione al Padre tuo che è nel segreto; e il Padre tuo che vede nel segreto, te ne darà la ricompensa.
\par 7 E nel pregare non usate soverchie dicerie come fanno i pagani, i quali pensano d'essere esauditi per la moltitudine delle loro parole.
\par 8 Non li rassomigliate dunque, poiché il Padre vostro sa le cose di cui avete bisogno, prima che gliele chiediate.
\par 9 Voi dunque pregate così: Padre nostro che sei nei cieli, sia santificato il tuo nome;
\par 10 venga il tuo regno; sia fatta la tua volontà anche in terra com'è fatta nel cielo.
\par 11 Dacci oggi il nostro pane cotidiano;
\par 12 e rimettici i nostri debiti come anche noi li abbiamo rimessi ai nostri debitori;
\par 13 e non ci esporre alla tentazione, ma liberaci dal maligno.
\par 14 Poiché se voi perdonate agli uomini i loro falli, il Padre vostro celeste perdonerà anche a voi;
\par 15 ma se voi non perdonate agli uomini, neppure il Padre vostro perdonerà i vostri falli.
\par 16 E quando digiunate, non siate mesti d'aspetto come gl'ipocriti; poiché essi si sfigurano la faccia per far vedere agli uomini che digiunano. Io vi dico in verità che cotesto è il premio che ne hanno.
\par 17 Ma tu, quando digiuni, ungiti il capo e lavati la faccia,
\par 18 affinché non apparisca agli uomini che tu digiuni, ma al Padre tuo che è nel segreto; e il Padre tuo, che vede nel segreto, te ne darà la ricompensa.
\par 19 Non vi fate tesori sulla terra, ove la tignola e la ruggine consumano, e dove i ladri sconficcano e rubano;
\par 20 ma fatevi tesori in cielo, ove né tignola né ruggine consumano, e dove i ladri non sconficcano né rubano.
\par 21 Perché dov'è il tuo tesoro, quivi sarà anche il tuo cuore.
\par 22 La lampada del corpo è l'occhio. Se dunque l'occhio tuo è sano, tutto il tuo corpo sarà illuminato;
\par 23 ma se l'occhio tuo è viziato, tutto il tuo corpo sarà nelle tenebre. Se dunque la luce che è in te è tenebre, esse tenebre quanto grandi saranno!
\par 24 Niuno può servire a due padroni; perché o odierà l'uno ed amerà l'altro, o si atterrà all'uno e sprezzerà l'altro. Voi non potete servire a Dio ed a Mammona.
\par 25 Perciò vi dico: Non siate con ansietà solleciti per la vita vostra di quel che mangerete o di quel che berrete; né per il vostro corpo, di che vi vestirete. Non è la vita più del nutrimento, e il corpo più del vestito?
\par 26 Guardate gli uccelli del cielo: non seminano, non mietono, non raccolgono in granai, e il Padre vostro celeste li nutrisce. Non siete voi assai più di loro?
\par 27 E chi di voi può con la sua sollecitudine aggiungere alla sua statura pure un cubito?
\par 28 E intorno al vestire, perché siete con ansietà solleciti? Considerate come crescono i gigli della campagna; essi non faticano e non filano;
\par 29 eppure io vi dico che nemmeno Salomone, con tutta la sua gloria, fu vestito come uno di loro.
\par 30 Or se Iddio riveste in questa maniera l'erba de' campi che oggi è e domani è gettata nel forno, non vestirà Egli molto più voi, o gente di poca fede?
\par 31 Non siate dunque con ansietà solleciti, dicendo: Che mangeremo? che berremo? o di che ci vestiremo?
\par 32 Poiché sono i pagani che ricercano tutte queste cose; e il Padre vostro celeste sa che avete bisogno di tutte queste cose.
\par 33 Ma cercate prima il regno e la giustizia di Dio, e tutte queste cose vi saranno sopraggiunte.
\par 34 Non siate dunque con ansietà solleciti del domani; perché il domani sarà sollecito di se stesso. Basta a ciascun giorno il suo affanno.

\chapter{7}

\par 1 Non giudicate acciocché non siate giudicati;
\par 2 perché col giudicio col quale giudicate, sarete giudicati; e con la misura onde misurate, sarà misurato a voi.
\par 3 E perché guardi tu il bruscolo che è nell'occhio del tuo fratello, mentre non iscorgi la trave che è nell'occhio tuo?
\par 4 Ovvero, come potrai tu dire al tuo fratello: Lascia ch'io ti tragga dall'occhio il bruscolo, mentre ecco la trave è nell'occhio tuo?
\par 5 Ipocrita, trai prima dall'occhio tuo la trave, e allora ci vedrai bene per trarre il bruscolo dall'occhio del tuo fratello.
\par 6 Non date ciò ch'è santo ai cani e non gettate le vostre perle dinanzi ai porci, che talora non le pestino co' piedi e rivolti contro a voi non vi sbranino.
\par 7 Chiedete e vi sarà dato; cercate e troverete; picchiate e vi sarà aperto;
\par 8 perché chiunque chiede riceve; chi cerca trova, e sarà aperto a chi picchia.
\par 9 E qual è l'uomo fra voi, il quale, se il figliuolo gli chiede un pane gli dia una pietra?
\par 10 Oppure se gli chiede un pesce gli dia un serpente?
\par 11 Se dunque voi che siete malvagi, sapete dar buoni doni ai vostri figliuoli, quanto più il Padre vostro che è ne' cieli darà Egli cose buone a coloro che gliele domandano!
\par 12 Tutte le cose dunque che voi volete che gli uomini vi facciano, fatele anche voi a loro; perché questa è la legge ed i profeti.
\par 13 Entrate per la porta stretta, poiché larga è la porta e spaziosa la via che mena alla perdizione, e molti son quelli che entran per essa.
\par 14 Stretta invece è la porta ed angusta la via che mena alla vita, e pochi son quelli che la trovano.
\par 15 Guardatevi dai falsi profeti i quali vengono a voi in vesti da pecore, ma dentro son lupi rapaci.
\par 16 Voi li riconoscerete dai loro frutti. Si colgon forse delle uve dalle spine, o dei fichi dai triboli?
\par 17 Così, ogni albero buono fa frutti buoni; ma l'albero cattivo fa frutti cattivi.
\par 18 Un albero buono non può far frutti cattivi, né un albero cattivo far frutti buoni.
\par 19 Ogni albero che non fa buon frutto, è tagliato e gettato nel fuoco.
\par 20 Voi li riconoscerete dunque dai loro frutti.
\par 21 Non chiunque mi dice: Signore, Signore, entrerà nel regno dei cieli, ma chi fa la volontà del Padre mio che è ne' cieli.
\par 22 Molti mi diranno in quel giorno: Signore, Signore, non abbiam noi profetizzato in nome tuo, e in nome tuo cacciato demonî, e fatte in nome tuo molte opere potenti?
\par 23 E allora dichiarerò loro: Io non vi conobbi mai; dipartitevi da me, voi tutti operatori d'iniquità.
\par 24 Perciò chiunque ode queste mie parole e le mette in pratica sarà paragonato ad un uomo avveduto che ha edificata la sua casa sopra la roccia.
\par 25 E la pioggia è caduta, e son venuti i torrenti, e i venti hanno soffiato e hanno investito quella casa; ma ella non è caduta, perché era fondata sulla roccia.
\par 26 E chiunque ode queste mie parole e non le mette in pratica sarà paragonato ad un uomo stolto che ha edificata la sua casa sulla rena.
\par 27 E la pioggia è caduta, e son venuti i torrenti, e i venti hanno soffiato ed hanno fatto impeto contro quella casa; ed ella è caduta, e la sua ruina è stata grande.
\par 28 Ed avvenne che quando Gesù ebbe finiti questi discorsi, le turbe stupivano del suo insegnamento,
\par 28 Ed avvenne che quando Gesù ebbe finiti questi discorsi, le turbe stupivano del suo insegnamento,

\chapter{8}

\par 1 Or quando egli fu sceso dal monte, molte turbe lo seguirono.
\par 2 Ed ecco un lebbroso, accostatosi, gli si prostrò dinanzi dicendo: Signore, se vuoi, tu puoi mondarmi.
\par 3 E Gesù, stesa la mano, lo toccò dicendo: Lo voglio, sii mondato. E in quell'istante egli fu mondato dalla sua lebbra.
\par 4 E Gesù gli disse: Guarda di non dirlo a nessuno: ma va', mostrati al sacerdote e fa' l'offerta che Mosè ha prescritto; e ciò serva loro di testimonianza.
\par 5 Or quand'egli fu entrato in Capernaum, un centurione venne a lui pregandolo e dicendo:
\par 6 Signore, il mio servitore giace in casa paralitico, gravemente tormentato.
\par 7 Gesù gli disse: Io verrò e lo guarirò. Ma il centurione, rispondendo disse:
\par 8 Signore, io non son degno che tu entri sotto al mio tetto; ma di' soltanto una parola e il mio servitore sarà guarito.
\par 9 Poiché anch'io son uomo sottoposto ad altri ed ho sotto di me dei soldati; e dico a uno: Va', ed egli va; e ad un altro: Vieni, ed egli viene; e al mio servo: Fa' questo, ed egli lo fa.
\par 10 E Gesù, udito questo, ne restò maravigliato, e disse a quelli che lo seguivano: Io vi dico in verità che in nessuno, in Israele, ho trovato cotanta fede.
\par 11 Or io vi dico che molti verranno di Levante e di Ponente e sederanno a tavola con Abramo e Isacco e Giacobbe, nel regno dei cieli;
\par 12 ma i figliuoli del regno saranno gettati nelle tenebre di fuori. Quivi sarà il pianto e lo stridor dei denti.
\par 13 E Gesù disse al centurione: Va'; e come hai creduto, siati fatto. E il servitore fu guarito in quell'ora stessa.
\par 14 Poi Gesù, entrato nella casa di Pietro, vide la suocera di lui che giaceva in letto con la febbre; ed egli le toccò la mano e la febbre la lasciò.
\par 15 Ella si alzò e si mise a servirlo.
\par 16 Poi, venuta la sera, gli presentarono molti indemoniati; ed egli, con la parola, scacciò gli spiriti e guarì tutti i malati,
\par 17 affinché si adempisse quel che fu detto per bocca del profeta Isaia: Egli stesso ha preso le nostre infermità, ed ha portato le nostre malattie.
\par 18 Or Gesù, vedendo una gran folla intorno a sé, comandò che si passasse all'altra riva.
\par 19 Allora uno scriba, accostatosi, gli disse: Maestro, io ti seguirò dovunque tu vada.
\par 20 E Gesù gli disse: Le volpi hanno delle tane e gli uccelli del cielo dei nidi, ma il Figliuol dell'uomo non ha dove posare il capo.
\par 21 E un altro dei discepoli gli disse: Signore, permettimi d'andare prima a seppellir mio padre.
\par 22 Ma Gesù gli disse: Séguitami, e lascia i morti seppellire i loro morti.
\par 23 Ed essendo egli entrato nella barca, i suoi discepoli lo seguirono.
\par 24 Ed ecco farsi in mare una così gran burrasca, che la barca era coperta dalle onde; ma Gesù dormiva.
\par 25 E i suoi discepoli, accostatisi, lo svegliarono dicendo: Signore, salvaci, siam perduti.
\par 26 Ed egli disse loro: Perché avete paura, o gente di poca fede? Allora, levatosi, sgridò i venti ed il mare, e si fece gran bonaccia.
\par 27 E quegli uomini ne restaron maravigliati e dicevano: Che uomo è mai questo che anche i venti e il mare gli obbediscono?
\par 28 E quando fu giunto all'altra riva, nel paese de' Gadareni, gli si fecero incontro due indemoniati, usciti dai sepolcri, così furiosi, che niuno potea passar per quella via.
\par 29 Ed ecco si misero a gridare: Che v'è fra noi e te, Figliuol di Dio? Sei tu venuto qua prima del tempo per tormentarci?
\par 30 Or lungi da loro v'era un gran branco di porci che pasceva.
\par 31 E i demonî lo pregavano dicendo: Se tu ci scacci, mandaci in quel branco di porci.
\par 32 Ed egli disse loro: Andate. Ed essi, usciti, se ne andarono nei porci; ed ecco tutto il branco si gettò a precipizio giù nel mare, e perirono nelle acque.
\par 33 E quelli che li pasturavano fuggirono; e andati nella città raccontarono ogni cosa e il fatto degl'indemoniati.
\par 34 Ed ecco tutta la città uscì incontro a Gesù; e, come lo videro, lo pregarono che si partisse dai loro confini.

\chapter{9}

\par 1 E Gesù, entrato in una barca, passò all'altra riva e venne nella sua città.
\par 2 Ed ecco gli portarono un paralitico steso sopra un letto. E Gesù, veduta la fede loro, disse al paralitico: Figliuolo, sta' di buon animo, i tuoi peccati ti sono rimessi.
\par 3 Ed ecco alcuni degli scribi dissero dentro di sé: Costui bestemmia.
\par 4 E Gesù, conosciuti i loro pensieri, disse: Perché pensate voi cose malvage ne' vostri cuori?
\par 5 Poiché, che cos'è più facile, dire: I tuoi peccati ti sono rimessi o dire: Lèvati e cammina?
\par 6 Or affinché sappiate che il Figliuol dell'uomo ha sulla terra autorità di rimettere i peccati: Lèvati (disse al paralitico), prendi il tuo letto e vattene a casa.
\par 7 Ed egli, levatosi, se ne andò a casa sua.
\par 8 E le turbe, veduto ciò, furon prese da timore, e glorificarono Iddio che avea data cotale autorità agli uomini.
\par 9 Poi Gesù, partitosi di là, passando, vide un uomo, chiamato Matteo, che sedeva al banco della gabella; e gli disse: Seguimi. Ed egli, levatosi, lo seguì.
\par 10 Ed avvenne che, essendo Gesù a tavola in casa di Matteo, ecco, molti pubblicani e peccatori vennero e si misero a tavola con Gesù e co' suoi discepoli.
\par 11 E i Farisei, veduto ciò, dicevano ai suoi discepoli: Perché il vostro maestro mangia coi pubblicani e coi peccatori?
\par 12 Ma Gesù, avendoli uditi, disse: Non sono i sani che hanno bisogno del medico, ma i malati.
\par 13 Or andate e imparate che cosa significhi: Voglio misericordia, e non sacrifizio; poiché io non son venuto a chiamar de' giusti, ma dei peccatori.
\par 14 Allora gli s'accostarono i discepoli di Giovanni e gli dissero: Perché noi ed i Farisei digiuniamo, e i tuoi discepoli non digiunano?
\par 15 E Gesù disse loro: Gli amici dello sposo possono essi far cordoglio, finché lo sposo è con loro? Ma verranno i giorni che lo sposo sarà loro tolto, ed allora digiuneranno.
\par 16 Or niuno mette un pezzo di stoffa nuova sopra un vestito vecchio; perché quella toppa porta via qualcosa dal vestito, e lo strappo si fa peggiore.
\par 17 Neppur si mette del vin nuovo in otri vecchi; altrimenti gli otri si rompono, il vino si spande e gli otri si perdono; ma si mette il vin nuovo in otri nuovi, e l'uno e gli altri si conservano.
\par 18 Mentr'egli diceva loro queste cose, ecco uno dei capi della sinagoga, accostatosi, s'inchinò dinanzi a lui e gli disse: La mia figliuola è pur ora trapassata; ma vieni, metti la mano su lei ed ella vivrà.
\par 19 E Gesù, alzatosi, lo seguiva co' suoi discepoli.
\par 20 Ed ecco una donna, malata d'un flusso di sangue da dodici anni, accostatasi per di dietro, gli toccò il lembo della veste.
\par 21 Perché diceva fra sé: Sol ch'io tocchi la sua veste, sarò guarita.
\par 22 E Gesù, voltatosi e vedutala, disse: Sta' di buon animo, figliuola; la tua fede t'ha guarita. E da quell'ora la donna fu guarita.
\par 23 E quando Gesù fu giunto alla casa del capo della sinagoga, ed ebbe veduto i sonatori di flauto e la moltitudine che facea grande strepito, disse loro: Ritiratevi;
\par 24 perché la fanciulla non è morta, ma dorme. E si ridevano di lui.
\par 25 Ma quando la moltitudine fu messa fuori, egli entrò, e prese la fanciulla per la mano, ed ella si alzò.
\par 26 E se ne divulgò la fama per tutto quel paese.
\par 27 Come Gesù partiva di là, due ciechi lo seguirono, gridando e dicendo: Abbi pietà di noi, o Figliuol di Davide!
\par 28 E quand'egli fu entrato nella casa, que' ciechi si accostarono a lui. E Gesù disse loro: Credete voi ch'io possa far questo? Essi gli risposero: Sì, o Signore.
\par 29 Allora toccò loro gli occhi dicendo: Siavi fatto secondo la vostra fede.
\par 30 E gli occhi loro furono aperti. E Gesù fece loro un severo divieto, dicendo: Guardate che niuno lo sappia.
\par 31 Ma quelli, usciti fuori, sparsero la fama di lui per tutto quel paese.
\par 32 Or come quei ciechi uscivano, ecco che gli fu presentato un uomo muto indemoniato.
\par 33 E cacciato che fu il demonio, il muto parlò. E le turbe si maravigliarono dicendo: Mai non s'è vista cosa tale in Israele.
\par 34 Ma i Farisei dicevano: Egli caccia i demonî per l'aiuto del principe dei demonî.
\par 35 E Gesù andava attorno per tutte le città e per i villaggi, insegnando nelle loro sinagoghe e predicando l'evangelo del Regno, e sanando ogni malattia ed ogni infermità.
\par 36 E vedendo le turbe, n'ebbe compassione, perch'erano stanche e sfinite, come pecore che non hanno pastore.
\par 37 Allora egli disse ai suoi discepoli: Ben è la mèsse grande, ma pochi son gli operai.
\par 38 Pregate dunque il Signor della mèsse che spinga degli operai nella sua mèsse.

\chapter{10}

\par 1 Poi, chiamati a sé i suoi dodici discepoli, diede loro potestà di cacciare gli spiriti immondi, e di sanare qualunque malattia e qualunque infermità.
\par 2 Or i nomi de' dodici apostoli son questi: Il primo Simone detto Pietro, e Andrea suo fratello; Giacomo di Zebedeo e Giovanni suo fratello;
\par 3 Filippo e Bartolommeo; Toma e Matteo il pubblicano; Giacomo d'Alfeo e Taddeo;
\par 4 Simone il Cananeo e Giuda l'Iscariota, quello stesso che poi lo tradì.
\par 5 Questi dodici mandò Gesù, dando loro queste istruzioni: Non andate fra i Gentili, e non entrate in alcuna città de' Samaritani,
\par 6 ma andate piuttosto alle pecore perdute della casa d'Israele.
\par 7 E andando, predicate e dite: Il regno de' cieli è vicino.
\par 8 Sanate gl'infermi, risuscitate i morti, mondate i lebbrosi, cacciate i demonî; gratuitamente avete ricevuto, gratuitamente date.
\par 9 Non fate provvisione né d'oro, né d'argento, né di rame nelle vostre cinture,
\par 10 né di sacca da viaggio, né di due tuniche, né di calzari, né di bastone, perché l'operaio è degno del suo nutrimento.
\par 11 Or in qualunque città o villaggio sarete entrati, informatevi chi sia ivi degno; e dimorate da lui finché partiate.
\par 12 E quando entrerete nella casa, salutatela.
\par 13 E se quella casa n'è degna, venga la pace vostra su lei: se poi non ne è degna la vostra pace torni a voi.
\par 14 E se alcuno non vi riceve né ascolta le vostre parole, uscendo da quella casa o da quella città, scotete la polvere da' vostri piedi.
\par 15 In verità io vi dico che il paese di Sodoma e di Gomorra, nel giorno del giudizio, sarà trattato con meno rigore di quella città.
\par 16 Ecco, io vi mando come pecore in mezzo ai lupi; siate dunque prudenti come i serpenti e semplici come le colombe.
\par 17 E guardatevi dagli uomini; perché vi metteranno in man de' tribunali e vi flagelleranno nelle loro sinagoghe;
\par 18 e sarete menati davanti a governatori e re per cagion mia, per servir di testimonianza dinanzi a loro ed ai Gentili.
\par 19 Ma quando vi metteranno nelle loro mani, non siate in ansietà del come parlerete o di quel che avrete a dire; perché in quell'ora stessa vi sarà dato ciò che avrete a dire.
\par 20 Poiché non siete voi che parlate, ma è lo Spirito del Padre vostro che parla in voi.
\par 21 Or il fratello darà il fratello a morte, e il padre il figliuolo; e i figliuoli si leveranno contro i genitori e li faranno morire.
\par 22 E sarete odiati da tutti a cagion del mio nome; ma chi avrà perseverato sino alla fine sarà salvato.
\par 23 E quando vi perseguiteranno in una città, fuggite in un'altra; perché io vi dico in verità che non avrete finito di percorrere le città d'Israele, prima che il Figliuol dell'uomo sia venuto.
\par 24 Un discepolo non è da più del maestro, né un servo da più del suo signore.
\par 25 Basti al discepolo di essere come il suo maestro, e al servo d'essere come il suo signore. Se hanno chiamato Beelzebub il padrone, quanto più chiameranno così quei di casa sua!
\par 26 Non li temete dunque; poiché non v'è niente di nascosto che non abbia ad essere scoperto, né di occulto che non abbia a venire a notizia.
\par 27 Quello ch'io vi dico nelle tenebre, ditelo voi nella luce; e quel che udite dettovi all'orecchio, predicatelo sui tetti.
\par 28 E non temete coloro che uccidono il corpo, ma non possono uccider l'anima; temete piuttosto colui che può far perire e l'anima e il corpo nella geenna.
\par 29 Due passeri non si vendon essi per un soldo? Eppure non ne cade uno solo in terra senza del Padre vostro.
\par 30 Ma quant'è a voi, perfino i capelli del vostro capo son tutti contati.
\par 31 Non temete dunque; voi siete da più di molti passeri.
\par 32 Chiunque adunque mi riconoscerà davanti agli uomini, anch'io riconoscerò lui davanti al Padre mio che è ne' cieli.
\par 33 Ma chiunque mi rinnegherà davanti agli uomini, anch'io rinnegherò lui davanti al Padre mio che è nei cieli.
\par 34 Non pensate ch'io sia venuto a metter pace sulla terra; non son venuto a metter pace, ma spada.
\par 35 Perché son venuto a dividere il figlio da suo padre, e la figlia da sua madre, e la nuora dalla suocera;
\par 36 e i nemici dell'uomo saranno quelli stessi di casa sua.
\par 37 Chi ama padre o madre più di me, non è degno di me; e chi ama figliuolo o figliuola più di me, non è degno di me;
\par 38 e chi non prende la sua croce e non vien dietro a me, non è degno di me.
\par 39 Chi avrà trovato la vita sua la perderà; e chi avrà perduto la sua vita per cagion mia, la troverà.
\par 40 Chi riceve voi riceve me; e chi riceve me, riceve colui che mi ha mandato.
\par 41 Chi riceve un profeta come profeta, riceverà premio di profeta; e chi riceve un giusto come giusto, riceverà premio di giusto.
\par 42 E chi avrà dato da bere soltanto un bicchier d'acqua fresca ad uno di questi piccoli, perché è un mio discepolo, io vi dico in verità che non perderà punto il suo premio.

\chapter{11}

\par 1 Ed avvenne che quando ebbe finito di dar le sue istruzioni ai suoi dodici discepoli, Gesù si partì di là per insegnare e predicare nelle loro città.
\par 2 Or Giovanni, avendo nella prigione udito parlare delle opere del Cristo, mandò a dirgli per mezzo dei suoi discepoli:
\par 3 Sei tu colui che ha da venire, o ne aspetteremo noi un altro?
\par 4 E Gesù rispondendo disse loro: Andate a riferire a Giovanni quello che udite e vedete:
\par 5 i ciechi ricuperano la vista e gli zoppi camminano; i lebbrosi sono mondati e i sordi odono; i morti risuscitano, e l'Evangelo è annunziato ai poveri.
\par 6 E beato colui che non si sarà scandalizzato di me!
\par 7 Or com'essi se ne andavano, Gesù prese a dire alle turbe intorno a Giovanni: Che andaste a vedere nel deserto? Una canna dimenata dal vento? Ma che andaste a vedere?
\par 8 Un uomo avvolto in morbide vesti? Ecco, quelli che portano delle vesti morbide stanno nelle dimore dei re.
\par 9 Ma perché andaste? Per vedere un profeta? Sì, vi dico e uno più che profeta.
\par 10 Egli è colui del quale è scritto: Ecco, io mando il mio messaggero davanti al tuo cospetto, che preparerà la via dinanzi a te.
\par 11 In verità io vi dico, che fra i nati di donna non è sorto alcuno maggiore di Giovanni Battista; però, il minimo nel regno dei cieli è maggiore di lui.
\par 12 Or dai giorni di Giovanni Battista fino ad ora, il regno de' cieli è preso a forza ed i violenti se ne impadroniscono.
\par 13 Poiché tutti i profeti e la legge hanno profetato fino a Giovanni.
\par 14 E se lo volete accettare, egli è l'Elia che dovea venire. Chi ha orecchi oda.
\par 15 Ma a chi assomiglierò io questa generazione?
\par 16 Ella è simile ai fanciulli seduti nelle piazze che gridano ai loro compagni e dicono:
\par 17 Vi abbiam sonato il flauto, e voi non avete ballato; abbiam cantato de' lamenti, e voi non avete fatto cordoglio.
\par 18 Difatti è venuto Giovanni non mangiando né bevendo, e dicono: Ha un demonio!
\par 19 È venuto il Figliuol dell'uomo mangiando e bevendo, e dicono: Ecco un mangiatore ed un beone, un amico dei pubblicani e de' peccatori! Ma la sapienza è stata giustificata dalle opere sue.
\par 20 Allora egli prese a rimproverare le città nelle quali era stata fatta la maggior parte delle sue opere potenti, perché non si erano ravvedute.
\par 21 Guai a te, Corazin! Guai a te, Betsaida! Perché se in Tiro e Sidone fossero state fatte le opere potenti compiute fra voi, già da gran tempo si sarebbero pentite, con cilicio e cenere.
\par 22 E però vi dichiaro che nel giorno del giudizio la sorte di Tiro e di Sidone sarà più tollerabile della vostra.
\par 23 E tu, o Capernaum, sarai tu forse innalzata fino al cielo? No, tu scenderai fino nell'Ades. Perché se in Sodoma fossero state fatte le opere potenti compiute in te, ella sarebbe durata fino ad oggi.
\par 24 E però, io lo dichiaro, nel giorno del giudizio la sorte del paese di Sodoma sarà più tollerabile della tua.
\par 25 In quel tempo Gesù prese a dire: Io ti rendo lode, o Padre, Signor del cielo e della terra, perché hai nascoste queste cose ai savî e agli intelligenti, e le hai rivelate ai piccoli fanciulli.
\par 26 Sì, Padre, perché così t'è piaciuto.
\par 27 Ogni cosa m'è stata data in mano dal Padre mio; e niuno conosce appieno il Figliuolo, se non il Padre, e niuno conosce appieno il Padre, se non il Figliuolo e colui al quale il Figliuolo avrà voluto rivelarlo.
\par 28 Venite a me, voi tutti che siete travagliati ed aggravati, e io vi darò riposo.
\par 29 Prendete su voi il mio giogo ed imparate da me, perch'io son mansueto ed umile di cuore; e voi troverete riposo alle anime vostre;
\par 30 poiché il mio giogo è dolce e il mio carico è leggero.

\chapter{12}

\par 1 In quel tempo Gesù passò in giorno di sabato per i seminati; e i suoi discepoli ebbero fame e presero a svellere delle spighe ed a mangiare.
\par 2 E i Farisei, veduto ciò, gli dissero: Ecco, i tuoi discepoli fanno quel che non è lecito di fare in giorno di sabato.
\par 3 Ma egli disse loro: Non avete voi letto quel che fece Davide, quando ebbe fame, egli e coloro ch'eran con lui?
\par 4 Come egli entrò nella casa di Dio, e come mangiarono i pani di presentazione i quali non era lecito di mangiare né a lui, né a quelli ch'eran con lui, ma ai soli sacerdoti?
\par 5 Ovvero, non avete voi letto nella legge che nei giorni di sabato, i sacerdoti nel tempio violano il sabato e non ne son colpevoli?
\par 6 Or io vi dico che v'è qui qualcosa di più grande del tempio.
\par 7 E se sapeste che cosa significhi: Voglio misericordia e non sacrifizio, voi non avreste condannato gl'innocenti;
\par 8 perché il Figliuol dell'uomo è signore del sabato.
\par 9 E, partitosi di là, venne nella loro sinagoga.
\par 10 Ed ecco un uomo che avea una mano secca. Ed essi, affin di poterlo accusare, fecero a Gesù questa domanda: È egli lecito far delle guarigioni in giorno di sabato?
\par 11 Ed egli disse loro: Chi è colui fra voi che, avendo una pecora, s'ella cade in giorno di sabato in una fossa non la prenda e la tragga fuori?
\par 12 Or quant'è un uomo da più d'una pecora! È dunque lecito di far del bene in giorno di sabato.
\par 13 Allora disse a quell'uomo: Stendi la tua mano. E colui la stese, ed ella tornò sana come l'altra.
\par 14 Ma i Farisei, usciti, tennero consiglio contro di lui, col fine di farlo morire.
\par 15 Ma Gesù, saputolo, si partì di là; e molti lo seguirono, ed egli li guarì tutti;
\par 16 e ordinò loro severamente di non farlo conoscere,
\par 17 affinché si adempisse quanto era stato detto per bocca del profeta Isaia:
\par 18 Ecco il mio Servitore che ho scelto; il mio diletto, in cui l'anima mia si è compiaciuta. Io metterò lo Spirito mio sopra lui, ed egli annunzierà giudicio alle genti.
\par 19 Non contenderà, né griderà, né alcuno udrà la sua voce nelle piazze.
\par 20 Ei non triterà la canna rotta e non spegnerà il lucignolo fumante, finché non abbia fatto trionfar la giustizia.
\par 21 E nel nome di lui le genti spereranno.
\par 22 Allora gli fu presentato un indemoniato, cieco e muto; ed egli lo sanò, talché il mutolo parlava e vedeva.
\par 23 E tutte le turbe stupivano e dicevano: Non è costui il figliuol di Davide?
\par 24 Ma i Farisei, udendo ciò, dissero: Costui non caccia i demonî se non per l'aiuto di Beelzebub, principe dei demonî.
\par 25 E Gesù, conosciuti i loro pensieri, disse loro: Ogni regno diviso in parti contrarie sarà ridotto in deserto; ed ogni città o casa divisa in parti contrarie non potrà reggere.
\par 26 E se Satana caccia Satana, egli è diviso contro se stesso; come dunque potrà sussistere il suo regno?
\par 27 E se io caccio i demonî per l'aiuto di Beelzebub, per l'aiuto di chi li cacciano i vostri figliuoli? Per questo, essi stessi saranno i vostri giudici.
\par 28 Ma se è per l'aiuto dello Spirito di Dio che io caccio i demonî, è dunque pervenuto fino a voi il regno di Dio.
\par 29 Ovvero, come può uno entrar nella casa dell'uomo forte e rapirgli le sue masserizie, se prima non abbia legato l'uomo forte? Allora soltanto gli prederà la casa.
\par 30 Chi non è con me, è contro di me; e chi non raccoglie con me, disperde.
\par 31 Perciò io vi dico: Ogni peccato e bestemmia sarà perdonata agli uomini; ma la bestemmia contro lo Spirito non sarà perdonata.
\par 32 Ed a chiunque parli contro il Figliuol dell'uomo, sarà perdonato; ma a chiunque parli contro lo Spirito Santo, non sarà perdonato né in questo mondo né in quello avvenire.
\par 33 O voi fate l'albero buono e buono pure il suo frutto, o fate l'albero cattivo e cattivo pure il suo frutto; perché dal frutto si conosce l'albero.
\par 34 Razza di vipere, come potete dir cose buone, essendo malvagi? Poiché dall'abbondanza del cuore la bocca parla.
\par 35 L'uomo dabbene dal suo buon tesoro trae cose buone; e l'uomo malvagio dal suo malvagio tesoro trae cose malvage.
\par 36 Or io vi dico che d'ogni parola oziosa che avranno detta, gli uomini renderan conto nel giorno del giudizio;
\par 37 poiché dalle tue parole sarai giustificato, e dalle tue parole sarai condannato.
\par 38 Allora alcuni degli scribi e dei Farisei presero a dirgli: Maestro, noi vorremmo vederti operare un segno.
\par 39 Ma egli rispose loro: Questa generazione malvagia e adultera chiede un segno; e segno non le sarà dato, tranne il segno del profeta Giona.
\par 40 Poiché, come Giona stette nel ventre del pesce tre giorni e tre notti, così starà il Figliuol dell'uomo nel cuor della terra tre giorni e tre notti.
\par 41 I Niniviti risorgeranno nel giudizio con questa generazione e la condanneranno, perché essi si ravvidero alla predicazione di Giona; ed ecco qui vi è più che Giona!
\par 42 La regina del Mezzodì risusciterà nel giudizio con questa generazione e la condannerà; perché ella venne dalle estremità della terra per udir la sapienza di Salomone; ed ecco qui v'è più che Salomone!
\par 43 Or quando lo spirito immondo è uscito da un uomo, va attorno per luoghi aridi, cercando riposo e non lo trova.
\par 44 Allora dice: Ritornerò nella mia casa donde sono uscito; e giuntovi, la trova vuota, spazzata e adorna.
\par 45 Allora va e prende seco altri sette spiriti peggiori di lui, i quali, entrati, prendon quivi dimora; e l'ultima condizione di cotest'uomo divien peggiore della prima. Così avverrà anche a questa malvagia generazione.
\par 46 Mentre Gesù parlava ancora alle turbe, ecco sua madre e i suoi fratelli che, fermatisi di fuori, cercavano di parlargli.
\par 47 E uno gli disse: Ecco, tua madre e i tuoi fratelli son là fuori che cercano di parlarti.
\par 48 Ma egli, rispondendo, disse a colui che gli parlava: Chi è mia madre, e chi sono i miei fratelli?
\par 49 E, stendendo la mano sui suoi discepoli, disse: Ecco mia madre e i miei fratelli!
\par 50 Poiché chiunque avrà fatta la volontà del Padre mio che è ne' cieli, esso mi è fratello e sorella e madre.

\chapter{13}

\par 1 In quel giorno Gesù, uscito di casa, si pose a sedere presso al mare;
\par 2 e molte turbe si raunarono attorno a lui; talché egli, montato in una barca, vi sedette; e tutta la moltitudine stava sulla riva.
\par 3 Ed egli insegnò loro molte cose in parabole, dicendo:
\par 4 Ecco, il seminatore uscì a seminare. E mentre seminava, una parte del seme cadde lungo la strada; gli uccelli vennero e la mangiarono.
\par 5 E un'altra cadde ne' luoghi rocciosi ove non avea molta terra; e subito spuntò, perché non avea terreno profondo;
\par 6 ma, levatosi il sole, fu riarsa; e perché non avea radice, si seccò.
\par 7 E un'altra cadde sulle spine; e le spine crebbero e l'affogarono.
\par 8 E un'altra cadde nella buona terra e portò frutto, dando qual cento, qual sessanta, qual trenta per uno.
\par 9 Chi ha orecchi da udire oda.
\par 10 Allora i discepoli, accostatisi, gli dissero: Perché parli loro in parabole?
\par 11 Ed egli rispose loro: Perché a voi è dato di conoscere i misteri del regno dei cieli; ma a loro non è dato.
\par 12 Perché a chiunque ha, sarà dato, e sarà nell'abbondanza; ma a chiunque non ha, sarà tolto anche quello che ha.
\par 13 Perciò parlo loro in parabole, perché, vedendo, non vedono; e udendo, non odono e non intendono.
\par 14 E s'adempie in loro la profezia d'Isaia che dice: Udrete co' vostri orecchi e non intenderete; guarderete co' vostri occhi e non vedrete:
\par 15 perché il cuore di questo popolo s'è fatto insensibile, son divenuti duri d'orecchi ed hanno chiuso gli occhi, che talora non veggano con gli occhi e non odano con gli orecchi e non intendano col cuore e non si convertano, ed io non li guarisca.
\par 16 Ma beati gli occhi vostri, perché veggono; ed i vostri orecchi, perché odono!
\par 17 Poiché in verità io vi dico che molti profeti e giusti desiderarono di vedere le cose che voi vedete, e non le videro; e di udire le cose che voi udite, e non le udirono.
\par 18 Voi dunque ascoltate che cosa significhi la parabola del seminatore:
\par 19 Tutte le volte che uno ode la parola del Regno e non la intende, viene il maligno e porta via quel ch'è stato seminato nel cuore di lui: questi è colui che ha ricevuto la semenza lungo la strada.
\par 20 E quegli che ha ricevuto la semenza in luoghi rocciosi, è colui che ode la Parola e subito la riceve con allegrezza;
\par 21 però non ha radice in sé, ma è di corta durata; e quando venga tribolazione o persecuzione a cagion della parola, è subito scandalizzato.
\par 22 E quegli che ha ricevuto la semenza fra le spine, è colui che ode la Parola; poi le cure mondane e l'inganno delle ricchezze affogano la Parola, e così riesce infruttuosa.
\par 23 Ma quei che ha ricevuto la semenza in buona terra, è colui che ode la Parola e l'intende; che porta del frutto e rende l'uno il cento, l'altro il sessanta e l'altro il trenta.
\par 24 Egli propose loro un'altra parabola, dicendo: il regno de' cieli è simile ad un uomo che ha seminato buona semenza nel suo campo.
\par 25 Ma mentre gli uomini dormivano, venne il suo nemico e seminò delle zizzanie in mezzo al grano e se ne andò.
\par 26 E quando l'erba fu nata ed ebbe fatto frutto, allora apparvero anche le zizzanie.
\par 27 E i servitori del padron di casa vennero a dirgli: Signore, non hai tu seminato buona semenza nel tuo campo? Come mai, dunque, c'è della zizzania?
\par 28 Ed egli disse loro: Un nemico ha fatto questo. E i servitori gli dissero: Vuoi tu che l'andiamo a cogliere?
\par 29 Ma egli rispose: No, che talora, cogliendo le zizzanie, non sradichiate insiem con esse il grano.
\par 30 Lasciate che ambedue crescano assieme fino alla mietitura; e al tempo della mietitura, io dirò ai mietitori: Cogliete prima le zizzanie, e legatele in fasci per bruciarle; ma il grano, raccoglietelo nel mio granaio.
\par 31 Egli propose loro un'altra parabola, dicendo: Il regno de' cieli è simile ad un granel di senapa che un uomo prende e semina nel suo campo.
\par 32 Esso è bene il più piccolo di tutti i semi; ma quando è cresciuto, è maggiore de' legumi e diviene albero; tanto che gli uccelli del cielo vengono a ripararsi tra i suoi rami.
\par 33 Disse loro un'altra parabola: Il regno de' cieli è simile al lievito che una donna prende e nasconde in tre staia di farina, finché la pasta sia tutta lievitata.
\par 34 Tutte queste cose disse Gesù in parabole alle turbe e senza parabola non diceva loro nulla,
\par 35 affinché si adempisse quel ch'era stato detto per mezzo del profeta: Aprirò in parabole la mia bocca; esporrò cose occulte fin dalla fondazione del mondo.
\par 36 Allora Gesù, lasciate le turbe, tornò a casa; e i suoi discepoli gli s'accostarono, dicendo: Spiegaci la parabola delle zizzanie del campo.
\par 37 Ed egli, rispondendo, disse loro: Colui che semina la buona semenza, è il Figliuol dell'uomo;
\par 38 il campo è il mondo, la buona semenza sono i figliuoli del Regno; le zizzanie sono i figliuoli del maligno;
\par 39 il nemico che le ha seminate, è il diavolo; la mietitura è la fine dell'età presente; i mietitori sono angeli.
\par 40 Come dunque si raccolgono le zizzanie e si bruciano col fuoco, così avverrà alla fine dell'età presente.
\par 41 Il Figliuol dell'uomo manderà i suoi angeli che raccoglieranno dal suo regno tutti gli scandali e tutti gli operatori d'iniquità,
\par 42 e li getteranno nella fornace del fuoco. Quivi sarà il pianto e lo stridor dei denti.
\par 43 Allora i giusti risplenderanno come il sole nel regno del Padre loro. Chi ha orecchi, oda.
\par 44 Il regno de' cieli è simile ad un tesoro nascosto nel campo, che un uomo, dopo averlo trovato, nasconde; e per l'allegrezza che ne ha, va e vende tutto quello che ha, e compra quel campo.
\par 45 Il regno de' cieli è anche simile ad un mercante che va in cerca di belle perle,
\par 46 e trovata una perla di gran prezzo, se n'è andato, ha venduto tutto quel che aveva, e l'ha comperata.
\par 47 Il regno de' cieli è anche simile ad una rete che, gettata in mare, ha raccolto ogni sorta di pesci;
\par 48 quando è piena, i pescatori la traggono a riva; e, postisi a sedere, raccolgono il buono in vasi, e buttano via quel che non val nulla.
\par 49 Così avverrà alla fine dell'età presente. Verranno gli angeli, toglieranno i malvagi di mezzo ai giusti,
\par 50 e li getteranno nella fornace del fuoco. Ivi sarà il pianto e lo stridor de' denti.
\par 51 Avete intese tutte queste cose? Essi gli risposero: Sì.
\par 52 Allora disse loro: Per questo, ogni scriba ammaestrato pel regno de' cieli è simile ad un padron di casa il quale trae fuori dal suo tesoro cose nuove e cose vecchie.
\par 53 Or quando Gesù ebbe finite queste parabole, partì di là.
\par 54 E recatosi nella sua patria, li ammaestrava nella lor sinagoga, talché stupivano e dicevano: Onde ha costui questa sapienza e queste opere potenti?
\par 55 Non è questi il figliuol del falegname? Sua madre non si chiama ella Maria, e i suoi fratelli, Giacomo, Giuseppe, Simone e Giuda?
\par 56 E le sue sorelle non sono tutte fra noi? Donde dunque vengono a lui tutte queste cose?
\par 57 E si scandalizzavano di lui. Ma Gesù disse loro: Un profeta non è sprezzato che nella sua patria e in casa sua.
\par 58 E non fece quivi molte opere potenti a cagione della loro incredulità.

\chapter{14}

\par 1 In quel tempo Erode, il tetrarca, udì la fama di Gesù,
\par 2 e disse ai suoi servitori: Costui è Giovanni Battista; egli è risuscitato dai morti, e però agiscono in lui le potenze miracolose.
\par 3 Perché Erode, fatto arrestare Giovanni, lo aveva incatenato e messo in prigione a motivo di Erodiada, moglie di Filippo suo fratello; perché Giovanni gli diceva:
\par 4 È non t'è lecito d'averla.
\par 5 E benché desiderasse farlo morire, temette il popolo che lo teneva per profeta.
\par 6 Ora, come si celebrava il giorno natalizio di Erode, la figliuola di Erodiada ballò nel convito e piacque ad Erode;
\par 7 ond'egli promise con giuramento di darle tutto quello che domanderebbe.
\par 8 Ed ella, spintavi da sua madre, disse: Dammi qui in un piatto la testa di Giovanni Battista.
\par 9 E il re ne fu contristato; ma, a motivo de' giuramenti e de' commensali, comandò che le fosse data,
\par 10 e mandò a far decapitare Giovanni nella prigione.
\par 11 E la testa di lui fu portata in un piatto e data alla fanciulla, che la portò a sua madre.
\par 12 E i discepoli di Giovanni andarono a prenderne il corpo e lo seppellirono; poi vennero a darne la nuova a Gesù.
\par 13 Udito ciò, Gesù si ritirò di là in barca verso un luogo deserto, in disparte; e le turbe, saputolo, lo seguitarono a piedi dalle città.
\par 14 E Gesù, smontato dalla barca, vide una gran moltitudine; n'ebbe compassione, e ne guarì gl'infermi.
\par 15 Or, facendosi sera, i suoi discepoli gli si accostarono e gli dissero: Il luogo è deserto e l'ora è già passata; licenzia dunque le folle, affinché vadano pei villaggi a comprarsi da mangiare.
\par 16 Ma Gesù disse loro: Non hanno bisogno d'andarsene; date lor voi da mangiare!
\par 17 Ed essi gli risposero: Non abbiam qui altro che cinque pani e due pesci.
\par 18 Ed egli disse: Portatemeli qua.
\par 19 Ed avendo ordinato alle turbe di accomodarsi sull'erba, prese i cinque pani e i due pesci e, levati gli occhi al cielo, rese grazie; poi, spezzati i pani, li diede ai discepoli e i discepoli alle turbe.
\par 20 E tutti mangiarono e furon sazî; e si portaron via, dei pezzi avanzati, dodici ceste piene.
\par 21 E quelli che avevano mangiato eran circa cinquemila uomini, oltre le donne e i fanciulli.
\par 22 Subito dopo, Gesù obbligò i suoi discepoli a montar nella barca ed a precederlo sull'altra riva, mentr'egli licenzierebbe le turbe.
\par 23 E licenziatele, si ritirò in disparte sul monte per pregare. E fattosi sera, era quivi tutto solo.
\par 24 Frattanto la barca, già di molti stadi lontana da terra, era sbattuta dalle onde perché il vento era contrario.
\par 25 Ma alla quarta vigilia della notte Gesù andò verso loro, camminando sul mare.
\par 26 E i discepoli, vedendolo camminar sul mare, si turbarono e dissero: È un fantasma! E dalla paura gridarono.
\par 27 Ma subito Gesù parlò loro e disse: State di buon animo, son io; non temete!
\par 28 E Pietro gli rispose: Signore, se sei tu, comandami di venir a te sulle acque.
\par 29 Ed egli disse: Vieni! E Pietro, smontato dalla barca, camminò sulle acque e andò verso Gesù.
\par 30 Ma vedendo il vento, ebbe paura; e cominciando a sommergersi, gridò: Signore, salvami!
\par 31 E Gesù, stesa subito la mano, lo afferrò e gli disse: O uomo di poca fede, perché hai dubitato?
\par 32 E quando furono montati nella barca, il vento s'acquetò.
\par 33 Allora quelli che erano nella barca si prostrarono dinanzi a lui, dicendo: veramente tu sei Figliuol di Dio!
\par 34 E, passati all'altra riva, vennero nel paese di Gennezaret.
\par 35 E la gente di quel luogo, avendolo riconosciuto, mandò per tutto il paese all'intorno, e gli presentaron tutti i malati,
\par 36 e lo pregavano che lasciasse loro toccare non foss'altro che il lembo del suo vestito; e tutti quelli che lo toccarono, furon completamente guariti.

\chapter{15}

\par 1 Allora s'accostarono a Gesù dei Farisei e degli scribi venuti da Gerusalemme, e gli dissero: Perché i tuoi discepoli trasgrediscono la tradizione degli antichi?
\par 2 poiché non si lavano le mani quando prendono cibo.
\par 3 Ma egli rispose loro: E voi, perché trasgredite il comandamento di Dio a motivo della vostra tradizione?
\par 4 Dio, infatti, ha detto: Onora tuo padre e tua madre; e: Chi maledice padre o madre sia punito di morte; voi, invece, dite:
\par 5 Se uno dice a suo padre o a sua madre: 'Quello con cui potrei assisterti è offerta a Dio',
\par 6 egli non è più obbligato ad onorar suo padre o sua madre. E avete annullata la parola di Dio a cagion della vostra tradizione.
\par 7 Ipocriti, ben profetò Isaia di voi quando disse:
\par 8 Questo popolo mi onora con le labbra, ma il cuor loro è lontano da me.
\par 9 Ma invano mi rendono il loro culto, insegnando dottrine che son precetti d'uomini.
\par 10 E chiamata a sé la moltitudine, disse loro: Ascoltate e intendete.
\par 11 Non è quel che entra nella bocca che contamina l'uomo; ma quel che esce dalla bocca, ecco quel che contamina l'uomo.
\par 12 Allora i suoi discepoli, accostatisi, gli dissero: Sai tu che i Farisei, quand'hanno udito questo discorso, ne son rimasti scandalizzati?
\par 13 Ed egli rispose loro: Ogni pianta che il Padre mio celeste non ha piantata, sarà sradicata.
\par 14 Lasciateli; sono ciechi, guide di ciechi; or se un cieco guida un altro cieco, ambedue cadranno nella fossa.
\par 15 Pietro allora prese a dirgli: Spiegaci la parabola.
\par 16 E Gesù disse: Siete anche voi tuttora privi d'intendimento?
\par 17 Non capite voi che tutto quello che entra nella bocca va nel ventre ed è gittato fuori nella latrina?
\par 18 Ma quel che esce dalla bocca viene dal cuore, ed è quello che contamina l'uomo.
\par 19 Poiché dal cuore vengono pensieri malvagi, omicidî, adulterî, fornicazioni, furti, false testimonianze, diffamazioni.
\par 20 Queste son le cose che contaminano l'uomo; ma il mangiare con le mani non lavate non contamina l'uomo.
\par 21 E partitosi di là, Gesù si ritirò nelle parti di Tiro e di Sidone.
\par 22 Quand'ecco una donna cananea di que' luoghi venne fuori e si mise a gridare: Abbi pietà di me, Signore, figliuol di Davide; la mia figliuola è gravemente tormentata da un demonio.
\par 23 Ma egli non le rispose parola. E i suoi discepoli, accostatisi, lo pregavano dicendo: Licenziala, perché ci grida dietro.
\par 24 Ma egli rispose: Io non sono stato mandato che alle pecore perdute della casa d'Israele.
\par 25 Ella però venne e gli si prostrò dinanzi, dicendo: Signore, aiutami!
\par 26 Ma egli rispose: Non è bene prendere il pan dei figliuoli per buttarlo ai cagnolini.
\par 27 Ma ella disse: Dici bene, Signore; eppure anche i cagnolini mangiano dei minuzzoli che cadono dalla tavola dei lor padroni.
\par 28 Allora Gesù le disse: O donna, grande è la tua fede; ti sia fatto come vuoi. E da quell'ora la sua figliuola fu guarita.
\par 29 Partitosi di là, Gesù venne presso al mar di Galilea; e, salito sul monte, si pose quivi a sedere.
\par 30 E gli si accostarono molte turbe che avean seco degli zoppi, dei ciechi, de' muti, degli storpi e molti altri malati; li deposero a' suoi piedi, e Gesù li guarì;
\par 31 talché la folla restò ammirata a veder che i muti parlavano, che gli storpi eran guariti, che gli zoppi camminavano, che i ciechi vedevano, e ne dette gloria all'Iddio d'Israele.
\par 32 E Gesù, chiamati a sé i suoi discepoli, disse: Io ho pietà di questa moltitudine; poiché già da tre giorni sta con me e non ha da mangiare; e non voglio rimandarli digiuni, che talora non vengano meno per via.
\par 33 E i discepoli gli dissero: Donde potremmo avere, in un luogo deserto, tanti pani da saziare così gran folla?
\par 34 E Gesù chiese loro: Quanti pani avete? Ed essi risposero: Sette e pochi pescetti.
\par 35 Allora egli ordinò alla folla di accomodarsi per terra.
\par 36 Poi prese i sette pani ed i pesci; e dopo aver rese grazie, li spezzò e diede ai discepoli, e i discepoli alle folle.
\par 37 E tutti mangiarono e furon saziati; e de' pezzi avanzati si levaron sette panieri pieni.
\par 38 Or quelli che aveano mangiato eran quattromila persone, senza contare le donne e i fanciulli.
\par 39 E, licenziate le turbe, Gesù entrò nella barca e venne al paese di Magadan.

\chapter{16}

\par 1 Ed accostatisi a lui i Farisei e i Sadducei, per metterlo alla prova, gli chiesero di mostrar loro un segno dal cielo.
\par 2 Ma egli, rispondendo, disse loro: Quando si fa sera, voi dite: Bel tempo, perché il cielo rosseggia!
\par 3 e la mattina dite: Oggi tempesta, perché il cielo rosseggia cupo! L'aspetto del cielo lo sapete dunque discernere, e i segni de' tempi non arrivate a discernerli?
\par 4 Questa generazione malvagia e adultera chiede un segno, e segno non le sarà dato se non quello di Giona. E, lasciatili, se ne andò.
\par 5 Or i discepoli, passati all'altra riva, s'erano dimenticati di prender de' pani.
\par 6 E Gesù disse loro: Vedete di guardarvi dal lievito dei Farisei e de' Sadducei.
\par 7 Ed essi ragionavan fra loro e dicevano: Egli è perché non abbiam preso de' pani.
\par 8 Ma Gesù, accortosene, disse: O gente di poca fede, perché ragionate fra voi del non aver de' pani?
\par 9 Non capite ancora e non vi ricordate de' cinque pani dei cinquemila uomini e quante ceste ne levaste?
\par 10 né dei sette pani de' quattromila uomini e quanti panieri ne levaste?
\par 11 Come mai non capite che non è di pani ch'io vi parlavo? Ma guardatevi dal lievito de' Farisei e de' Sadducei.
\par 12 Allora intesero che non avea lor detto di guardarsi dal lievito del pane, ma dalla dottrina dei Farisei e de' Sadducei.
\par 13 Poi Gesù, venuto nelle parti di Cesarea di Filippo, domandò ai suoi discepoli: Chi dice la gente che sia il Figliuol dell'uomo?
\par 14 Ed essi risposero: Gli uni dicono Giovanni Battista; altri, Elia; altri, Geremia o uno de' profeti. Ed egli disse loro: E voi, chi dite ch'io sia?
\par 15 Simon Pietro, rispondendo, disse:
\par 16 Tu sei il Cristo, il Figliuol dell'Iddio vivente.
\par 17 E Gesù, replicando, gli disse: Tu sei beato, o Simone, figliuol di Giona, perché non la carne e il sangue t'hanno rivelato questo, ma il Padre mio che è ne' cieli.
\par 18 E io altresì ti dico: Tu sei Pietro, e su questa pietra edificherò la mia Chiesa, e le porte dell'Ades non la potranno vincere.
\par 19 Io ti darò le chiavi del regno dei cieli; e tutto ciò che avrai legato sulla terra sarà legato ne' cieli, e tutto ciò che avrai sciolto in terra sarà sciolto ne' cieli.
\par 20 Allora vietò ai suoi discepoli di dire ad alcuno ch'egli era il Cristo.
\par 21 Da quell'ora Gesù cominciò a dichiarare ai suoi discepoli che doveva andare a Gerusalemme e soffrir molte cose dagli anziani, dai capi sacerdoti e dagli scribi, ed esser ucciso, e risuscitare il terzo giorno.
\par 22 E Pietro, trattolo da parte, cominciò a rimproverarlo, dicendo: Tolga ciò Iddio, Signore; questo non ti avverrà mai.
\par 23 Ma Gesù, rivoltosi, disse a Pietro: Vattene via da me, Satana; tu mi sei di scandalo. Tu non hai il senso delle cose di Dio, ma delle cose degli uomini.
\par 24 Allora Gesù disse ai suoi discepoli: Se uno vuol venire dietro a me, rinunzi a se stesso e prenda la sua croce e mi segua.
\par 25 Perché chi vorrà salvare la sua vita, la perderà; ma chi avrà perduto la sua vita per amor mio, la troverà.
\par 26 E che gioverà egli a un uomo se, dopo aver guadagnato tutto il mondo, perde poi l'anima sua? O che darà l'uomo in cambio dell'anima sua?
\par 27 Perché il Figliuol dell'uomo verrà nella gloria del Padre suo, con i suoi angeli, ed allora renderà a ciascuno secondo l'opera sua.
\par 28 In verità io vi dico che alcuni di coloro che son qui presenti non gusteranno la morte, finché non abbian visto il Figliuol dell'uomo venire nel suo regno.

\chapter{17}

\par 1 Sei giorni dopo, Gesù prese seco Pietro, Giacomo e Giovanni suo fratello, e li condusse sopra un alto monte, in disparte.
\par 2 E fu trasfigurato dinanzi a loro; la sua faccia risplendé come il sole, e i suoi vestiti divennero candidi come la luce.
\par 3 Ed ecco apparvero loro Mosè ed Elia, che stavan conversando con lui.
\par 4 E Pietro prese a dire a Gesù: Signore, egli è bene che stiamo qui; se vuoi, farò qui tre tende: una per te, una per Mosè ed una per Elia.
\par 5 Mentr'egli parlava ancora, ecco una nuvola luminosa li coperse della sua ombra, ed ecco una voce dalla nuvola che diceva: Questo è il mio diletto Figliuolo, nel quale mi sono compiaciuto; ascoltatelo.
\par 6 E i discepoli, udito ciò, caddero con la faccia a terra, e furon presi da gran timore.
\par 7 Ma Gesù, accostatosi, li toccò e disse: Levatevi, e non temete.
\par 8 Ed essi, alzati gli occhi, non videro alcuno, se non Gesù tutto solo.
\par 9 Poi, mentre scendevano dal monte, Gesù diede loro quest'ordine: Non parlate di questa visione ad alcuno, finché il Figliuol dell'uomo sia risuscitato dai morti.
\par 10 E i discepoli gli domandarono: Perché dunque dicono gli scribi che prima deve venir Elia?
\par 11 Ed egli, rispondendo, disse loro: Certo, Elia deve venire e ristabilire ogni cosa.
\par 12 Ma io vi dico: Elia è già venuto, e non l'hanno riconosciuto; anzi, gli hanno fatto tutto quello che hanno voluto; così anche il Figliuol dell'uomo ha da patire da loro.
\par 13 Allora i discepoli intesero ch'era di Giovanni Battista ch'egli aveva loro parlato.
\par 14 E quando furon venuti alla moltitudine, un uomo gli s'accostò, gettandosi in ginocchio davanti a lui,
\par 15 e dicendo: Signore, abbi pietà del mio figliuolo, perché è lunatico e soffre molto; spesso, infatti, cade nel fuoco e spesso nell'acqua.
\par 16 L'ho menato ai tuoi discepoli, e non l'hanno potuto guarire.
\par 17 E Gesù, rispondendo, disse: O generazione incredula e perversa! Fino a quando sarò con voi? Fino a quando vi sopporterò? Menatemelo qua.
\par 18 E Gesù sgridò l'indemoniato, e il demonio uscì da lui; e da quell'ora il fanciullo fu guarito.
\par 19 Allora i discepoli, accostatisi a Gesù in disparte, gli chiesero: Perché non l'abbiam potuto cacciar noi?
\par 20 E Gesù rispose loro: A cagion della vostra poca fede; perché in verità io vi dico: Se avete fede quanto un granel di senapa, potrete dire a questo monte: Passa di qua là, e passerà; e niente vi sarà impossibile.
\par 21 Cha
\par 22 Or com'essi percorrevano insieme la Galilea, Gesù disse loro: Il Figliuol dell'uomo sta per esser dato nelle mani degli uomini;
\par 23 e l'uccideranno, e al terzo giorno risusciterà. Ed essi ne furono grandemente contristati.
\par 24 E quando furon venuti a Capernaum, quelli che riscotevano le didramme si accostarono a Pietro e dissero: Il vostro maestro non paga egli le didramme?
\par 25 Egli rispose: Sì. E quando fu entrato in casa, Gesù lo prevenne e gli disse: Che te ne pare, Simone? i re della terra da chi prendono i tributi o il censo? dai loro figliuoli o dagli stranieri?
\par 26 Dagli stranieri, rispose Pietro. Gesù gli disse: I figliuoli, dunque, ne sono esenti.
\par 27 Ma, per non scandalizzarli, vattene al mare, getta l'amo e prendi il primo pesce che verrà su; e, apertagli la bocca, troverai uno statère. Prendilo, e dàllo loro per me e per te.

\chapter{18}

\par 1 In quel mentre i discepoli s'accostarono a Gesù, dicendo: Chi è dunque il maggiore nel regno dei cieli?
\par 2 Ed egli, chiamato a sé un piccolo fanciullo, lo pose in mezzo a loro e disse:
\par 3 In verità io vi dico: Se non mutate e non diventate come i piccoli fanciulli, non entrerete punto nel regno dei cieli.
\par 4 Chi pertanto si abbasserà come questo piccolo fanciullo, è lui il maggiore nel regno de' cieli.
\par 5 E chiunque riceve un cotal piccolo fanciullo nel nome mio, riceve me.
\par 6 Ma chi avrà scandalizzato uno di questi piccoli che credono in me, meglio per lui sarebbe che gli fosse appesa al collo una macina da mulino e fosse sommerso nel fondo del mare.
\par 7 Guai al mondo per gli scandali! Poiché, ben è necessario che avvengan degli scandali; ma guai all'uomo per cui lo scandalo avviene!
\par 8 Ora, se la tua mano od il tuo piede t'è occasion di peccato, mozzali e gettali via da te; meglio è per te l'entrar nella vita monco o zoppo, che l'aver due mani o due piedi ed esser gettato nel fuoco eterno.
\par 9 E se l'occhio tuo t'è occasion di peccato, cavalo e gettalo via da te; meglio è per te l'entrar nella vita con un occhio solo, che l'aver due occhi ed esser gettato nella geenna del fuoco.
\par 10 Guardatevi dal disprezzare alcuno di questi piccoli; perché io vi dico che gli angeli loro, ne' cieli, vedono del continuo la faccia del Padre mio che è ne' cieli.
\par 11 nam
\par 12 Che vi par egli? Se un uomo ha cento pecore e una di queste si smarrisce, non lascerà egli le novantanove sui monti per andare in cerca della smarrita?
\par 13 E se gli riesce di ritrovarla, in verità vi dico ch'ei si rallegra più di questa che delle novantanove che non si erano smarrite.
\par 14 Così è voler del Padre vostro che è nei cieli, che neppure un solo di questi piccoli perisca.
\par 15 Se poi il tuo fratello ha peccato contro di te, va' e riprendilo fra te e lui solo. Se t'ascolta, avrai guadagnato il tuo fratello;
\par 16 ma, se non t'ascolta, prendi teco ancora una o due persone, affinché ogni parola sia confermata per bocca di due o tre testimoni.
\par 17 E se rifiuta d'ascoltarli, dillo alla chiesa; e se rifiuta di ascoltare anche la chiesa, siati come il pagano e il pubblicano.
\par 18 Io vi dico in verità che tutte le cose che avrete legate sulla terra, saranno legate nel cielo; e tutte le cose che avrete sciolte sulla terra, saranno sciolte nel cielo.
\par 19 Ed anche in verità vi dico: Se due di voi sulla terra s'accordano a domandare una cosa qualsiasi, quella sarà loro concessa dal Padre mio che è nei cieli.
\par 20 Poiché dovunque due o tre son raunati nel nome mio, quivi son io in mezzo a loro.
\par 21 Allora Pietro, accostatosi, gli disse: Signore, quante volte, peccando il mio fratello contro di me, gli perdonerò io? fino a sette volte?
\par 22 E Gesù a lui: Io non ti dico fino a sette volte, ma fino a settanta volte sette.
\par 23 Perciò il regno de' cieli è simile ad un re che volle fare i conti coi suoi servitori.
\par 24 E avendo cominciato a fare i conti, gli fu presentato uno, ch'era debitore di diecimila talenti.
\par 25 E non avendo egli di che pagare, il suo signore comandò che fosse venduto lui con la moglie e i figliuoli e tutto quant'avea, e che il debito fosse pagato.
\par 26 Onde il servitore, gettatosi a terra, gli si prostrò dinanzi, dicendo: Abbi pazienza con me, e ti pagherò tutto.
\par 27 E il signore di quel servitore, mosso a compassione, lo lasciò andare, e gli rimise il debito.
\par 28 Ma quel servitore, uscito, trovò uno de' suoi conservi che gli dovea cento denari; e afferratolo, lo strangolava, dicendo: Paga quel che devi!
\par 29 Onde il conservo, gettatosi a terra, lo pregava dicendo: Abbi pazienza con me, e ti pagherò.
\par 30 Ma colui non volle; anzi andò e lo cacciò in prigione, finché avesse pagato il debito.
\par 31 Or i suoi conservi, veduto il fatto, ne furono grandemente contristati, e andarono a riferire al loro signore tutto l'accaduto.
\par 32 Allora il suo signore lo chiamò a sé e gli disse: Malvagio servitore, io t'ho rimesso tutto quel debito, perché tu me ne supplicasti;
\par 33 non dovevi anche tu aver pietà del tuo conservo, com'ebbi anch'io pietà di te?
\par 34 E il suo signore, adirato, lo diede in man degli aguzzini fino a tanto che avesse pagato tutto quel che gli doveva.
\par 35 Così vi farà anche il Padre mio celeste, se ognun di voi non perdona di cuore al proprio fratello.

\chapter{19}

\par 1 Or avvenne che quando Gesù ebbe finiti questi ragionamenti, si partì dalla Galilea e se ne andò sui confini della Giudea oltre il Giordano.
\par 2 E molte turbe lo seguirono, e quivi guarì i loro malati.
\par 3 E de' Farisei s'accostarono a lui tentandolo, e dicendo: È egli lecito di mandar via, per qualunque ragione, la propria moglie?
\par 4 Ed egli, rispondendo, disse loro: Non avete voi letto che il Creatore da principio li creò maschio e femmina, e disse:
\par 5 Perciò l'uomo lascerà il padre e la madre e s'unirà con la sua moglie e i due saranno una sola carne?
\par 6 Talché non son più due, ma una sola carne; quello dunque che Iddio ha congiunto, l'uomo nol separi.
\par 7 Essi gli dissero: Perché dunque comandò Mosè di darle un atto di divorzio e mandarla via?
\par 8 Gesù disse loro: Fu per la durezza dei vostri cuori che Mosè vi permise di mandar via le vostre mogli; ma da principio non era così.
\par 9 Ed io vi dico che chiunque manda via sua moglie, quando non sia per cagion di fornicazione, e ne sposa un'altra, commette adulterio.
\par 10 I discepoli gli dissero: Se tale è il caso dell'uomo rispetto alla donna, non conviene di prender moglie.
\par 11 Ma egli rispose loro: Non tutti son capaci di praticare questa parola, ma quelli soltanto ai quali è dato.
\par 12 Poiché vi son degli eunuchi, i quali son nati così dal seno della madre; vi son degli eunuchi, i quali sono stati fatti tali dagli uomini, e vi son degli eunuchi, i quali si son fatti eunuchi da sé a cagion del regno de' cieli. Chi è in grado di farlo lo faccia.
\par 13 Allora gli furono presentati dei bambini perché imponesse loro le mani e pregasse; ma i discepoli sgridarono coloro che glieli presentavano.
\par 14 Gesù però disse: Lasciate i piccoli fanciulli e non vietate loro di venire a me, perché di tali è il regno de' cieli.
\par 15 E imposte loro le mani, si partì di là.
\par 16 Ed ecco un tale, che gli s'accostò e gli disse: Maestro, che farò io di buono per aver la vita eterna?
\par 17 E Gesù gli rispose: Perché m'interroghi tu intorno a ciò ch'è buono? Uno solo è il buono. Ma se vuoi entrar nella vita osserva i comandamenti.
\par 18 Quali? gli chiese colui. E Gesù rispose: Questi: Non uccidere; non commettere adulterio; non rubare; non dir falsa testimonianza;
\par 19 onora tuo padre e tua madre, e ama il tuo prossimo come te stesso.
\par 20 E il giovane a lui: Tutte queste cose le ho osservate; che mi manca ancora?
\par 21 Gesù gli disse: Se vuoi esser perfetto, va', vendi ciò che hai e dàllo ai poveri, ed avrai un tesoro nei cieli: poi, vieni e seguitami.
\par 22 Ma il giovane, udita questa parola, se ne andò contristato, perché avea di gran beni.
\par 23 E Gesù disse ai suoi discepoli: Io vi dico in verità che un ricco malagevolmente entrerà nel regno dei cieli.
\par 24 E da capo vi dico: È più facile a un cammello passare per la cruna d'un ago, che ad un ricco entrare nel regno di Dio.
\par 25 I suoi discepoli, udito questo, sbigottirono forte e dicevano: Chi dunque può esser salvato?
\par 26 E Gesù, riguardatili fisso, disse loro: Agli uomini questo è impossibile; ma a Dio ogni cosa è possibile.
\par 27 Allora Pietro, replicando, gli disse: Ecco, noi abbiamo lasciato ogni cosa e t'abbiam seguitato; che ne avremo dunque?
\par 28 E Gesù disse loro: Io vi dico in verità che nella nuova creazione, quando il Figliuol dell'uomo sederà sul trono della sua gloria, anche voi che m'avete seguitato, sederete su dodici troni a giudicare le dodici tribù d'Israele.
\par 29 E chiunque avrà lasciato case, o fratelli, o sorelle, o padre, o madre, o figliuoli, o campi per amor del mio nome, ne riceverà cento volte tanti, ed erederà la vita eterna.
\par 30 Ma molti primi saranno ultimi; e molti ultimi, primi.

\chapter{20}

\par 1 Poiché il regno de' cieli è simile a un padron di casa, il quale, in sul far del giorno, uscì a prender ad opra de' lavoratori per la sua vigna.
\par 2 E avendo convenuto coi lavoratori per un denaro al giorno, li mandò nella sua vigna.
\par 3 Ed uscito verso l'ora terza, ne vide degli altri che se ne stavano sulla piazza disoccupati,
\par 4 e disse loro: Andate anche voi nella vigna, e vi darò quel che sarà giusto. Ed essi andarono.
\par 5 Poi, uscito ancora verso la sesta e la nona ora, fece lo stesso.
\par 6 Ed uscito verso l'undicesima, ne trovò degli altri in piazza e disse loro: Perché ve ne state qui tutto il giorno inoperosi?
\par 7 Essi gli dissero: Perché nessuno ci ha presi a giornata. Egli disse loro: Andate anche voi nella vigna.
\par 8 Poi, fattosi sera, il padron della vigna disse al suo fattore: Chiama i lavoratori e paga loro la mercede, cominciando dagli ultimi fino ai primi.
\par 9 Allora, venuti quei dell'undicesima ora, ricevettero un denaro per uno.
\par 10 E venuti i primi, pensavano di ricever di più; ma ricevettero anch'essi un denaro per uno.
\par 11 E ricevutolo, mormoravano contro al padron di casa, dicendo:
\par 12 Questi ultimi non han fatto che un'ora e tu li hai fatti pari a noi che abbiamo portato il peso della giornata e il caldo.
\par 13 Ma egli, rispondendo a un di loro, disse: Amico, io non ti fo alcun torto; non convenisti meco per un denaro?
\par 14 Prendi il tuo, e vattene; ma io voglio dare a quest'ultimo quanto a te.
\par 15 Non m'è lecito far del mio ciò che voglio? o vedi tu di mal occhio ch'io sia buono?
\par 16 Così gli ultimi saranno primi, e i primi ultimi.
\par 17 Poi Gesù, stando per salire a Gerusalemme, trasse da parte i suoi dodici discepoli; e, cammin facendo, disse loro:
\par 18 Ecco, noi saliamo a Gerusalemme, e il Figliuol dell'uomo sarà dato nelle mani de' capi sacerdoti e degli scribi;
\par 19 ed essi lo condanneranno a morte, e lo metteranno nelle mani dei Gentili per essere schernito e flagellato e crocifisso; ma il terzo giorno risusciterà.
\par 20 Allora la madre de' figliuoli di Zebedeo s'accostò a lui co' suoi figliuoli, prostrandosi e chiedendogli qualche cosa.
\par 21 Ed egli le domandò: Che vuoi? Ella gli disse: Ordina che questi miei due figliuoli seggano l'uno alla tua destra e l'altro alla tua sinistra, nel tuo regno.
\par 22 E Gesù, rispondendo, disse: Voi non sapete quel che chiedete. Potete voi bere il calice che io sto per bere? Essi gli dissero: Sì, lo possiamo.
\par 23 Egli disse loro: Voi certo berrete il mio calice; ma quant'è al sedermi a destra o a sinistra non sta a me il darlo, ma è per quelli a cui è stato preparato dal Padre mio.
\par 24 E i dieci, udito ciò, furono indignati contro i due fratelli.
\par 25 Ma Gesù, chiamatili a sé, disse: Voi sapete che i principi delle nazioni le signoreggiano, e che i grandi usano potestà sopra di esse.
\par 26 Ma non è così tra voi; anzi, chiunque vorrà esser grande fra voi, sarà vostro servitore;
\par 27 chiunque fra voi vorrà esser primo, sarà vostro servitore;
\par 28 appunto come il Figliuol dell'uomo non è venuto per esser servito ma per servire, e per dar la vita sua come prezzo di riscatto per molti.
\par 29 E come uscivano da Gerico, una gran moltitudine lo seguì.
\par 30 Ed ecco che due ciechi, seduti presso la strada, avendo udito che Gesù passava, si misero a gridare: Abbi pietà di noi, Signore, figliuol di Davide!
\par 31 Ma la moltitudine li sgridava, perché tacessero; essi però gridavan più forte: Abbi pietà di noi, Signore, figliuol di Davide!
\par 32 E Gesù, fermatosi, li chiamò e disse: Che volete ch'io vi faccia?
\par 33 Ed essi: Signore, che s'aprano gli occhi nostri.
\par 34 Allora Gesù, mosso a pietà, toccò gli occhi loro, e in quell'istante ricuperarono la vista e lo seguirono.

\chapter{21}

\par 1 E quando furon vicini a Gerusalemme e furon giunti a Betfage, presso al monte degli Ulivi, Gesù mandò due discepoli,
\par 2 dicendo loro: Andate nella borgata che è dirimpetto a voi; e subito troverete un'asina legata, e un puledro con essa; scioglieteli e menatemeli.
\par 3 E se alcuno vi dice qualcosa, direte che il Signore ne ha bisogno, e subito li manderà.
\par 4 Or questo avvenne affinché si adempisse la parola del profeta:
\par 5 Dite alla figliuola di Sion: Ecco il tuo re viene a te, mansueto, e montato sopra un'asina, e un asinello, puledro d'asina.
\par 6 E i discepoli andarono e fecero come Gesù avea loro ordinato:
\par 7 menarono l'asina e il puledro, vi misero sopra i loro mantelli, e Gesù vi si pose a sedere.
\par 8 E la maggior parte della folla stese i mantelli sulla via; e altri tagliavano de' rami dagli alberi e li stendeano sulla via.
\par 9 E le turbe che precedevano e quelle che seguivano, gridavano: Osanna al Figliuolo di Davide! Benedetto colui che viene nel nome del Signore! Osanna nei luoghi altissimi!
\par 10 Ed essendo egli entrato in Gerusalemme, tutta la città fu commossa e si diceva:
\par 11 Chi è costui? E le turbe dicevano: Questi è Gesù, il profeta che è da Nazaret di Galilea.
\par 12 E Gesù entrò nel tempio e cacciò fuori tutti quelli che quivi vendevano e compravano; e rovesciò le tavole dei cambiamonete e le sedie de' venditori di colombi.
\par 13 E disse loro: Egli è scritto: La mia casa sarà chiamata casa d'orazione; ma voi ne fate una spelonca di ladroni.
\par 14 Allora vennero a lui, nel tempio, de' ciechi e degli zoppi, ed egli li sanò.
\par 15 Ma i capi sacerdoti e gli scribi, vedute le maraviglie che avea fatte, e i fanciulli che gridavano nel tempio: Osanna al figliuol di Davide, ne furono indignati, e gli dissero: Odi tu quel che dicono costoro?
\par 16 E Gesù disse loro: Sì. Non avete mai letto: Dalla bocca dei fanciulli e de' lattanti hai tratto lode?
\par 17 E, lasciatili, se ne andò fuor della città a Betania, dove albergò.
\par 18 E la mattina, tornando in città, ebbe fame.
\par 19 E vedendo un fico sulla strada, gli si accostò, ma non vi trovò altro che delle foglie; e gli disse: Mai più in eterno non nasca frutto da te. E subito il fico si seccò.
\par 20 E i discepoli, veduto ciò, si maravigliarono, dicendo: Come s'è in un attimo seccato il fico?
\par 21 E Gesù, rispondendo, disse loro: Io vi dico in verità: Se aveste fede e non dubitaste, non soltanto fareste quel ch'è stato fatto al fico; ma se anche diceste a questo monte: Togliti di là e gettati nel mare, sarebbe fatto.
\par 22 E tutte le cose che domanderete nella preghiera, se avete fede, le otterrete.
\par 23 E quando fu venuto nel tempio, i capi sacerdoti e gli anziani del popolo si accostarono a lui, mentr'egli insegnava, e gli dissero: Con quale autorità fai tu queste cose? E chi t'ha data codesta autorità?
\par 24 E Gesù, rispondendo, disse loro: Anch'io vi domanderò una cosa: e se voi mi rispondete, anch'io vi dirò con quale autorità faccio queste cose.
\par 25 Il battesimo di Giovanni, d'onde veniva? dal cielo o dagli uomini? Ed essi ragionavan fra loro, dicendo: Se diciamo: Dal cielo, egli ci dirà: Perché dunque non gli credeste?
\par 26 E se diciamo: Dagli uomini, temiamo la moltitudine, perché tutti tengono Giovanni per profeta.
\par 27 Risposero dunque a Gesù, dicendo: Non lo sappiamo. E anch'egli disse loro: E neppur io vi dirò con quale autorità io fo queste cose.
\par 28 Or che vi par egli? Un uomo avea due figliuoli. Accostatosi al primo disse: Figliuolo, va' oggi a lavorare nella vigna.
\par 29 Ed egli, rispondendo, disse: Vado, signore; ma non vi andò.
\par 30 E accostatosi al secondo, gli disse lo stesso. Ma egli, rispondendo, disse: Non voglio; ma poi, pentitosi, v'andò.
\par 31 Qual de' due fece la volontà del padre? Essi gli dissero: L'ultimo. E Gesù a loro: Io vi dico in verità: I pubblicani e le meretrici vanno innanzi a voi nel regno di Dio.
\par 32 Poiché Giovanni è venuto a voi per la via della giustizia, e voi non gli avete creduto; ma i pubblicani e le meretrici gli hanno creduto; e voi, che avete veduto questo, neppur poi vi siete pentiti per credere a lui.
\par 33 Udite un'altra parabola: Vi era un padron di casa, il quale piantò una vigna e le fece attorno una siepe, e vi scavò un luogo da spremer l'uva, e vi edificò una torre; poi l'allogò a de' lavoratori, e se n'andò in viaggio.
\par 34 Or quando fu vicina la stagione de' frutti, mandò i suoi servitori dai lavoratori per ricevere i frutti della vigna.
\par 35 Ma i lavoratori, presi i servitori, uno ne batterono, uno ne uccisero, e un altro ne lapidarono.
\par 36 Da capo mandò degli altri servitori, in maggior numero de' primi; e coloro li trattarono nello stesso modo.
\par 37 Finalmente, mandò loro il suo figliuolo, dicendo: Avranno rispetto al mio figliuolo.
\par 38 Ma i lavoratori, veduto il figliuolo, dissero tra di loro: Costui è l'erede; venite, uccidiamolo, e facciam nostra la sua eredità.
\par 39 E presolo, lo cacciaron fuori della vigna, e l'uccisero.
\par 40 Quando dunque sarà venuto il padron della vigna, che farà egli a quei lavoratori?
\par 41 Essi gli risposero: Li farà perir malamente, cotesti scellerati, e allogherà la vigna ad altri lavoratori, i quali gliene renderanno il frutto a suo tempo.
\par 42 Gesù disse loro: Non avete mai letto nelle Scritture: La pietra che gli edificatori hanno riprovata è quella ch'è divenuta pietra angolare; ciò è stato fatto dal Signore, ed è cosa maravigliosa agli occhi nostri?
\par 43 Perciò io vi dico che il Regno di Dio vi sarà tolto, e sarà dato ad una gente che ne faccia i frutti.
\par 44 E chi cadrà su questa pietra sarà sfracellato; ed ella stritolerà colui sul quale cadrà.
\par 45 E i capi sacerdoti e i Farisei, udite le sue parabole, si avvidero che parlava di loro;
\par 46 e cercavano di pigliarlo, ma temettero le turbe che lo teneano per profeta.

\chapter{22}

\par 1 E Gesù prese di nuovo a parlar loro in parabole dicendo:
\par 2 Il regno de' cieli è simile ad un re, il quale fece le nozze del suo figliuolo.
\par 3 E mandò i suoi servitori a chiamare gl'invitati alle nozze; ma questi non vollero venire.
\par 4 Di nuovo mandò degli altri servitori, dicendo: Dite agli invitati: Ecco, io ho preparato il mio pranzo; i miei buoi ed i miei animali ingrassati sono ammazzati, e tutto è pronto; venite alle nozze.
\par 5 Ma quelli, non curandosene, se n'andarono, chi al suo campo, chi al suo traffico;
\par 6 gli altri poi, presi i suoi servitori, li oltraggiarono e li uccisero.
\par 7 Allora il re s'adirò, e mandò le sue truppe a sterminare quegli omicidi e ad ardere la loro città.
\par 8 Quindi disse ai suoi servitori: Le nozze, sì, sono pronte; ma gl'invitati non ne erano degni.
\par 9 Andate dunque sui crocicchi delle strade e chiamate alle nozze quanti troverete.
\par 10 E quei servitori, usciti per le strade, raunarono tutti quelli che trovarono, cattivi e buoni; e la sala delle nozze fu ripiena di commensali.
\par 11 Or il re, entrato per vedere quelli che erano a tavola, notò quivi un uomo che non vestiva l'abito di nozze.
\par 12 E gli disse: Amico, come sei entrato qua senza aver un abito da nozze? E colui ebbe la bocca chiusa.
\par 13 Allora il re disse ai servitori: Legatelo mani e piedi e gettatelo nelle tenebre di fuori. Ivi sarà il pianto e lo stridor de' denti.
\par 14 Poiché molti son chiamati, ma pochi eletti.
\par 15 Allora i Farisei, ritiratisi, tennero consiglio per veder di coglierlo in fallo nelle sue parole.
\par 16 E gli mandarono i loro discepoli con gli Erodiani a dirgli: Maestro, noi sappiamo che sei verace e insegni la via di Dio secondo verità, e non ti curi d'alcuno, perché non guardi all'apparenza delle persone.
\par 17 Dicci dunque: Che te ne pare? È egli lecito pagare il tributo a Cesare, o no?
\par 18 Ma Gesù, conosciuta la loro malizia, disse: Perché mi tentate, ipocriti?
\par 19 Mostratemi la moneta del tributo. Ed essi gli porsero un denaro. Ed egli domandò loro:
\par 20 Di chi è questa effigie e questa iscrizione?
\par 21 Gli risposero: Di Cesare. Allora egli disse loro: Rendete dunque a Cesare quel ch'è di Cesare, e a Dio quel ch'è di Dio.
\par 22 Ed essi, udito ciò, si maravigliarono; e, lasciatolo, se ne andarono.
\par 23 In quell'istesso giorno vennero a lui de' Sadducei, i quali dicono che non v'è risurrezione, e gli domandarono:
\par 24 Maestro, Mosè ha detto: Se uno muore senza figliuoli, il fratel suo sposi la moglie di lui e susciti progenie al suo fratello.
\par 25 Or v'erano fra di noi sette fratelli; e il primo, ammogliatosi, morì; e, non avendo prole, lasciò sua moglie al suo fratello.
\par 26 Lo stesso fece pure il secondo, poi il terzo, fino al settimo.
\par 27 Infine, dopo tutti, morì anche la donna.
\par 28 Alla risurrezione, dunque, di quale dei sette sarà ella moglie? Poiché tutti l'hanno avuta.
\par 29 Ma Gesù, rispondendo, disse loro: Voi errate, perché non conoscete le Scritture, né la potenza di Dio.
\par 30 Perché alla risurrezione né si prende né si dà moglie; ma i risorti son come angeli ne' cieli.
\par 31 Quanto poi alla risurrezione dei morti, non avete voi letto quel che vi fu insegnato da Dio,
\par 32 quando disse: Io sono l'Iddio di Abramo e l'Iddio d'Isacco e l'Iddio di Giacobbe? Egli non è l'Iddio dei morti, ma de' viventi.
\par 33 E le turbe, udite queste cose, stupivano della sua dottrina.
\par 34 Or i Farisei, udito ch'egli avea chiusa la bocca a' Sadducei, si raunarono insieme;
\par 35 e uno di loro, dottor della legge, gli domandò, per metterlo alla prova:
\par 36 Maestro, qual è, nella legge, il gran comandamento?
\par 37 E Gesù gli disse: Ama il Signore Iddio tuo con tutto il tuo cuore e con tutta l'anima tua e con tutta la mente tua.
\par 38 Questo è il grande e il primo comandamento.
\par 39 Il secondo, simile ad esso, è: Ama il tuo prossimo come te stesso.
\par 40 Da questi due comandamenti dipendono tutta la legge ed i profeti.
\par 41 Or essendo i Farisei raunati, Gesù li interrogò, dicendo:
\par 42 Che vi par egli del Cristo? di chi è egli figliuolo? Essi gli risposero: Di Davide.
\par 43 Ed egli a loro: Come dunque Davide, parlando per lo Spirito, lo chiama Signore, dicendo:
\par 44 Il Signore ha detto al mio Signore: Siedi alla mia destra finché io abbia posto i tuoi nemici sotto i tuoi piedi?
\par 45 Se dunque Davide lo chiama Signore, com'è egli suo figliuolo?
\par 46 E nessuno potea replicargli parola; e da quel giorno nessuno ardì più interrogarlo.

\chapter{23}

\par 1 Allora Gesù parlò alle turbe e ai suoi discepoli,
\par 2 dicendo: Gli scribi e i Farisei seggono sulla cattedra di Mosè.
\par 3 Fate dunque ed osservate tutte le cose che vi diranno, ma non fate secondo le opere loro; perché dicono e non fanno.
\par 4 Difatti, legano de' pesi gravi e li mettono sulle spalle della gente; ma loro non li voglion muovere neppur col dito.
\par 5 Tutte le loro opere le fanno per essere osservati dagli uomini; difatti allargano le lor filatterie ed allungano le frange de' mantelli;
\par 6 ed amano i primi posti ne' conviti e i primi seggi nelle sinagoghe
\par 7 e i saluti nelle piazze e d'esser chiamati dalla gente: "Maestro!"
\par 8 Ma voi non vi fate chiamar "Maestro"; perché uno solo è il vostro maestro, e voi siete tutti fratelli.
\par 9 E non chiamate alcuno sulla terra vostro padre, perché uno solo è il Padre vostro, quello che è ne' cieli.
\par 10 E non vi fate chiamare guide, perché una sola è la vostra guida, il Cristo:
\par 11 ma il maggiore fra voi sia vostro servitore.
\par 12 Chiunque s'innalzerà sarà abbassato, e chiunque si abbasserà sarà innalzato.
\par 13 Ma guai a voi, scribi e Farisei ipocriti, perché serrate il regno de' cieli dinanzi alla gente, poiché né vi entrate voi, né lasciate entrare quelli che cercano di entrare.
\par 14 nam
\par 15 Guai a voi, scribi e Farisei ipocriti, perché scorrete mare e terra per fare un proselito; e fatto che sia, lo rendete figliuol della geenna il doppio di voi.
\par 16 Guai a voi, guide cieche, che dite: Se uno giura per il tempio, non è nulla; ma se giura per l'oro del tempio, resta obbligato.
\par 17 Stolti e ciechi, poiché qual è maggiore: l'oro, o il tempio che santifica l'oro?
\par 18 E se uno, voi dite, giura per l'altare, non è nulla; ma se giura per l'offerta che c'è sopra, resta obbligato.
\par 19 Ciechi, poiché qual è maggiore: l'offerta, o l'altare che santifica l'offerta?
\par 20 Chi dunque giura per l'altare, giura per esso e per tutto quel che c'è sopra;
\par 21 e chi giura per il tempio, giura per esso e per Colui che l'abita;
\par 22 e chi giura per il cielo, giura per il trono di Dio e per Colui che vi siede sopra.
\par 23 Guai a voi, scribi e Farisei ipocriti, perché pagate la decima della menta e dell'aneto e del comino, e trascurate le cose più gravi della legge: il giudicio, e la misericordia, e la fede. Queste son le cose che bisognava fare, senza tralasciar le altre.
\par 24 Guide cieche, che colate il moscerino e inghiottite il cammello.
\par 25 Guai a voi, scribi e Farisei ipocriti, perché nettate il di fuori del calice e del piatto, mentre dentro son pieni di rapina e d'intemperanza.
\par 26 Fariseo cieco, netta prima il di dentro del calice e del piatto, affinché anche il di fuori diventi netto.
\par 27 Guai a voi, scribi e Farisei ipocriti, perché siete simili a sepolcri imbiancati, che appaion belli di fuori, ma dentro son pieni d'ossa di morti e d'ogni immondizia.
\par 28 Così anche voi, di fuori apparite giusti alla gente; ma dentro siete pieni d'ipocrisia e d'iniquità.
\par 29 Guai a voi, scribi e Farisei ipocriti, perché edificate i sepolcri ai profeti, e adornate le tombe de' giusti e dite:
\par 30 Se fossimo stati ai dì de' nostri padri, non saremmo stati loro complici nello spargere il sangue dei profeti!
\par 31 Talché voi testimoniate contro voi stessi, che siete figliuoli di coloro che uccisero i profeti.
\par 32 E voi, colmate pure la misura dei vostri padri!
\par 33 Serpenti, razza di vipere, come scamperete al giudizio della geenna?
\par 34 Perciò, ecco, io vi mando de' profeti e de' savi e degli scribi; di questi, alcuni ne ucciderete e metterete in croce; altri ne flagellerete nelle vostre sinagoghe e li perseguiterete di città in città,
\par 35 affinché venga su voi tutto il sangue giusto sparso sulla terra, dal sangue del giusto Abele, fino al sangue di Zaccaria, figliuol di Barachia, che voi uccideste fra il tempio e l'altare.
\par 36 Io vi dico in verità che tutte queste cose verranno su questa generazione.
\par 37 Gerusalemme, Gerusalemme, che uccidi i profeti e lapidi quelli che ti sono mandati, quante volte ho voluto raccogliere i tuoi figliuoli, come la gallina raccoglie i suoi pulcini sotto le ali; e voi non avete voluto!
\par 38 Ecco, la vostra casa sta per esservi lasciata deserta.
\par 39 Poiché vi dico che d'ora innanzi non mi vedrete più, finché diciate: Benedetto colui che viene nel nome del Signore!

\chapter{24}

\par 1 E come Gesù usciva dal tempio e se n'andava, i suoi discepoli gli s'accostarono per fargli osservare gli edifizî del tempio.
\par 2 Ma egli rispose loro: Le vedete tutte queste cose? Io vi dico in verità: Non sarà lasciata qui pietra sopra pietra che non sia diroccata.
\par 3 E stando egli seduto sul monte degli Ulivi, i discepoli gli s'accostarono in disparte, dicendo: Dicci: Quando avverranno queste cose, e quale sarà il segno della tua venuta e della fine dell'età presente?
\par 4 E Gesù, rispondendo, disse loro: Guardate che nessuno vi seduca.
\par 5 Poiché molti verranno sotto il mio nome, dicendo: Io sono il Cristo, e ne sedurranno molti.
\par 6 Or voi udirete parlar di guerre e di rumori di guerre; guardate di non turbarvi, perché bisogna che questo avvenga, ma non sarà ancora la fine.
\par 7 Poiché si leverà nazione contro nazione e regno contro regno; ci saranno carestie e terremoti in varî luoghi;
\par 8 ma tutto questo non sarà che principio di dolori.
\par 9 Allora vi getteranno in tribolazione e v'uccideranno, e sarete odiati da tutte le genti a cagion del mio nome.
\par 10 E allora molti si scandalizzeranno, e si tradiranno e si odieranno a vicenda.
\par 11 E molti falsi profeti sorgeranno e sedurranno molti.
\par 12 E perché l'iniquità sarà moltiplicata, la carità dei più si raffredderà.
\par 13 Ma chi avrà perseverato sino alla fine sarà salvato.
\par 14 E questo evangelo del Regno sarà predicato per tutto il mondo, onde ne sia resa testimonianza a tutte le genti; e allora verrà la fine.
\par 15 Quando dunque avrete veduta l'abominazione della desolazione, della quale ha parlato il profeta Daniele, posta in luogo santo (chi legge pongavi mente),
\par 16 allora quelli che saranno nella Giudea, fuggano ai monti;
\par 17 chi sarà sulla terrazza non scenda per toglier quello che è in casa sua;
\par 18 e chi sarà nel campo non torni indietro a prender la sua veste.
\par 19 Or guai alle donne che saranno incinte, ed a quelle che allatteranno in que' giorni!
\par 20 E pregate che la vostra fuga non avvenga d'inverno né di sabato;
\par 21 perché allora vi sarà una grande afflizione; tale, che non v'è stata l'uguale dal principio del mondo fino ad ora, né mai più vi sarà.
\par 22 E se quei giorni non fossero stati abbreviati, nessuno scamperebbe; ma, a cagion degli eletti, que' giorni saranno abbreviati.
\par 23 Allora, se alcuno vi dice: 'Il Cristo eccolo qui, eccolo là', non lo credete;
\par 24 perché sorgeranno falsi cristi e falsi profeti, e faranno gran segni e prodigî da sedurre, se fosse possibile, anche gli eletti.
\par 25 Ecco, ve l'ho predetto. Se dunque vi dicono: Eccolo, è nel deserto, non v'andate;
\par 26 eccolo è nelle stanze interne, non lo credete;
\par 27 perché, come il lampo esce da levante e si vede fino a ponente, così sarà la venuta del Figliuol dell'uomo.
\par 28 Dovunque sarà il carname, quivi si raduneranno le aquile.
\par 29 Or subito dopo l'afflizione di que' giorni, il sole si oscurerà, e la luna non darà il suo splendore, e le stelle cadranno dal cielo, e le potenze de' cieli saranno scrollate.
\par 30 E allora apparirà nel cielo il segno del Figliuol dell'uomo; ed allora tutte le tribù della terra faranno cordoglio, e vedranno il Figliuol dell'uomo venir sulle nuvole del cielo con gran potenza e gloria.
\par 31 E manderà i suoi angeli con gran suono di tromba a radunare i suoi eletti dai quattro venti, dall'un capo all'altro de' cieli.
\par 32 Or imparate dal fico questa similitudine: Quando già i suoi rami si fanno teneri e metton le foglie, voi sapete che l'estate è vicina.
\par 33 Così anche voi, quando vedrete tutte queste cose, sappiate che egli è vicino, proprio alle porte.
\par 34 Io vi dico in verità che questa generazione non passerà prima che tutte queste cose siano avvenute.
\par 35 Il cielo e la terra passeranno, ma le mie parole non passeranno.
\par 36 Ma quant'è a quel giorno ed a quell'ora nessuno li sa, neppure gli angeli dei cieli, neppure il Figliuolo, ma il Padre solo.
\par 37 E come fu ai giorni di Noè, così sarà alla venuta del Figliuol dell'uomo.
\par 38 Infatti, come ne' giorni innanzi al diluvio si mangiava e si beveva, si prendea moglie e s'andava a marito, sino al giorno che Noè entrò nell'arca,
\par 39 e di nulla si avvide la gente, finché venne il diluvio che portò via tutti quanti, così avverrà alla venuta del Figliuol dell'uomo.
\par 40 Allora due saranno nel campo; l'uno sarà preso e l'altro lasciato;
\par 41 due donne macineranno al mulino: l'una sarà presa e l'altra lasciata.
\par 42 Vegliate, dunque, perché non sapete in qual giorno il vostro Signore sia per venire.
\par 43 Ma sappiate questo, che se il padron di casa sapesse a qual vigilia il ladro deve venire, veglierebbe e non lascerebbe forzar la sua casa.
\par 44 Perciò, anche voi siate pronti; perché, nell'ora che non pensate, il Figliuol dell'uomo verrà.
\par 45 Qual è mai il servitore fedele e prudente che il padrone abbia costituito sui domestici per dar loro il vitto a suo tempo?
\par 46 Beato quel servitore che il padrone, arrivando, troverà così occupato!
\par 47 Io vi dico in verità che lo costituirà su tutti i suoi beni.
\par 48 Ma, s'egli è un malvagio servitore che dica in cuor suo: Il mio padrone tarda a venire;
\par 49 e comincia a battere i suoi conservi, e a mangiare e bere con gli ubriaconi,
\par 50 il padrone di quel servitore verrà nel giorno che non se l'aspetta, e nell'ora che non sa;
\par 51 e lo farà lacerare a colpi di flagello, e gli assegnerà la sorte degl'ipocriti. Ivi sarà il pianto e lo stridor de' denti.

\chapter{25}

\par 1 Allora il regno de' cieli sarà simile a dieci vergini le quali, prese le loro lampade, uscirono a incontrar lo sposo.
\par 2 Or cinque d'esse erano stolte e cinque avvedute;
\par 3 le stolte, nel prendere le loro lampade, non avean preso seco dell'olio;
\par 4 mentre le avvedute, insieme con le loro lampade, avean preso dell'olio ne' vasi.
\par 5 Or tardando lo sposo, tutte divennero sonnacchiose e si addormentarono.
\par 6 E sulla mezzanotte si levò un grido: Ecco lo sposo, uscitegli incontro!
\par 7 Allora tutte quelle vergini si destarono e acconciaron le loro lampade.
\par 8 E le stolte dissero alle avvedute: Dateci del vostro olio, perché le nostre lampade si spengono.
\par 9 Ma le avvedute risposero: No, che talora non basti per noi e per voi; andate piuttosto da' venditori e compratevene!
\par 10 Ma, mentre quelle andavano a comprarne, arrivò lo sposo; e quelle che eran pronte, entraron con lui nella sala delle nozze, e l'uscio fu chiuso.
\par 11 All'ultimo vennero anche le altre vergini, dicendo: Signore, Signore, aprici!
\par 12 Ma egli, rispondendo, disse: Io vi dico in verità: Non vi conosco.
\par 13 Vegliate dunque, perché non sapete né il giorno né l'ora.
\par 14 Poiché avverrà come di un uomo il quale, partendo per un viaggio, chiamò i suoi servitori e affidò loro i suoi beni;
\par 15 e all'uno diede cinque talenti, a un altro due, e a un altro uno; a ciascuno secondo la sua capacità; e partì.
\par 16 Subito, colui che avea ricevuto i cinque talenti andò a farli fruttare, e ne guadagnò altri cinque.
\par 17 Parimente, quello de' due ne guadagnò altri due.
\par 18 Ma colui che ne avea ricevuto uno, andò e, fatta una buca in terra, vi nascose il danaro del suo padrone.
\par 19 Or dopo molto tempo, ecco il padrone di que' servitori a fare i conti con loro.
\par 20 E colui che avea ricevuto i cinque talenti, venne e presentò altri cinque talenti, dicendo: Signore, tu m'affidasti cinque talenti; ecco, ne ho guadagnati altri cinque.
\par 21 E il suo padrone gli disse: Va bene, buono e fedel servitore; sei stato fedele in poca cosa, ti costituirò sopra molte cose; entra nella gioia del tuo Signore.
\par 22 Poi, presentatosi anche quello de' due talenti, disse: Signore, tu m'affidasti due talenti; ecco, ne ho guadagnati altri due.
\par 23 Il suo padrone gli disse: Va bene, buono e fedel servitore; sei stato fedele in poca cosa, ti costituirò sopra molte cose; entra nella gioia del tuo Signore.
\par 24 Poi, accostatosi anche quello che avea ricevuto un talento solo, disse: Signore, io sapevo che tu sei uomo duro, che mieti dove non hai seminato, e raccogli dove non hai sparso;
\par 25 ebbi paura, e andai a nascondere il tuo talento sotterra; eccoti il tuo.
\par 26 E il suo padrone, rispondendo, gli disse: Servo malvagio ed infingardo, tu sapevi ch'io mieto dove non ho seminato e raccolgo dove non ho sparso;
\par 27 dovevi dunque portare il mio danaro dai banchieri; e al mio ritorno, avrei ritirato il mio con interesse.
\par 28 Toglietegli dunque il talento, e datelo a colui che ha i dieci talenti.
\par 29 Poiché a chiunque ha sarà dato, ed egli sovrabbonderà; ma a chi non ha sarà tolto anche quello che ha.
\par 30 E quel servitore disutile, gettatelo nelle tenebre di fuori. Ivi sarà il pianto e lo stridor dei denti.
\par 31 Or quando il Figliuol dell'uomo sarà venuto nella sua gloria, avendo seco tutti gli angeli, allora sederà sul trono della sua gloria.
\par 32 E tutte le genti saranno radunate dinanzi a lui ed egli separerà gli uni dagli altri, come il pastore separa le pecore dai capri;
\par 33 e metterà le pecore alla sua destra e i capri alla sinistra.
\par 34 Allora il Re dirà a quelli della sua destra: Venite, voi, i benedetti del Padre mio; eredate il regno che v'è stato preparato sin dalla fondazione del mondo.
\par 35 Perché ebbi fame, e mi deste da mangiare; ebbi sete, e mi deste da bere; fui forestiere e m'accoglieste;
\par 36 fui ignudo, e mi rivestiste; fui infermo, e mi visitaste; fui in prigione, e veniste a trovarmi.
\par 37 Allora i giusti gli risponderanno: Signore, quando mai t'abbiam veduto aver fame e t'abbiam dato da mangiare? o aver sete e t'abbiam dato da bere?
\par 38 Quando mai t'abbiam veduto forestiere e t'abbiamo accolto? o ignudo e t'abbiam rivestito?
\par 39 Quando mai t'abbiam veduto infermo o in prigione e siam venuti a trovarti?
\par 40 E il Re, rispondendo, dirà loro: In verità vi dico che in quanto l'avete fatto ad uno di questi miei minimi fratelli, l'avete fatto a me.
\par 41 Allora dirà anche a coloro dalla sinistra: Andate via da me, maledetti, nel fuoco eterno, preparato pel diavolo e per i suoi angeli!
\par 42 Perché ebbi fame e non mi deste da mangiare; ebbi sete e non mi deste da bere;
\par 43 fui forestiere e non m'accoglieste; ignudo, e non mi rivestiste; infermo ed in prigione, e non mi visitaste.
\par 44 Allora anche questi gli risponderanno, dicendo: Signore, quando t'abbiam veduto aver fame, o sete, o esser forestiere, o ignudo, o infermo, o in prigione, e non t'abbiamo assistito?
\par 45 Allora risponderà loro, dicendo: In verità vi dico che in quanto non l'avete fatto ad uno di questi minimi, non l'avete fatto neppure a me.
\par 46 E questi se ne andranno a punizione eterna; ma i giusti a vita eterna.

\chapter{26}

\par 1 Ed avvenne che quando Gesù ebbe finiti tutti questi ragionamenti, disse ai suoi discepoli:
\par 2 Voi sapete che fra due giorni è la Pasqua, e il Figliuol dell'uomo sarà consegnato per esser crocifisso.
\par 3 Allora i capi sacerdoti e gli anziani del popolo si raunarono nella corte del sommo sacerdote detto Caiàfa,
\par 4 e deliberarono nel loro consiglio di pigliar Gesù con inganno e di farlo morire.
\par 5 Ma dicevano: Non durante la festa, perché non accada tumulto nel popolo.
\par 6 Or essendo Gesù in Betania, in casa di Simone il lebbroso,
\par 7 venne a lui una donna che aveva un alabastro d'olio odorifero di gran prezzo, e lo versò sul capo di lui che stava a tavola.
\par 8 Veduto ciò, i discepoli furono indignati e dissero: A che questa perdita?
\par 9 Poiché quest'olio si sarebbe potuto vender caro, e il denaro darlo ai poveri.
\par 10 Ma Gesù, accortosene, disse loro: Perché date noia a questa donna? Ella ha fatto un'azione buona verso di me.
\par 11 Perché i poveri li avete sempre con voi; ma me non mi avete sempre.
\par 12 Poiché costei, versando quest'olio sul mio corpo, l'ha fatto in vista della mia sepoltura.
\par 13 In verità vi dico che per tutto il mondo, dovunque sarà predicato questo evangelo, anche quello che costei ha fatto, sarà raccontato in memoria di lei.
\par 14 Allora uno dei dodici, detto Giuda Iscariot, andò dai capi sacerdoti e disse loro:
\par 15 Che mi volete dare, e io ve lo consegnerò? Ed essi gli contarono trenta sicli d'argento.
\par 16 E da quell'ora cercava il momento opportuno di tradirlo.
\par 17 Or il primo giorno degli azzimi, i discepoli s'accostarono a Gesù e gli dissero: Dove vuoi che ti prepariamo da mangiar la pasqua?
\par 18 Ed egli disse: Andate in città dal tale, e ditegli: Il Maestro dice: Il mio tempo è vicino; farò la pasqua da te, co' miei discepoli.
\par 19 E i discepoli fecero come Gesù avea loro ordinato, e prepararono la pasqua.
\par 20 E quando fu sera, si mise a tavola co' dodici discepoli.
\par 21 E mentre mangiavano, disse: In verità io vi dico: Uno di voi mi tradirà.
\par 22 Ed essi, grandemente attristati, cominciarono a dirgli ad uno ad uno: Sono io quello, Signore?
\par 23 Ma egli, rispondendo, disse: Colui che ha messo con me la mano nel piatto, quello mi tradirà.
\par 24 Certo, il Figliuol dell'uomo se ne va, come è scritto di lui; ma guai a quell'uomo per cui il Figliuol dell'uomo è tradito! Meglio sarebbe per cotest'uomo, se non fosse mai nato.
\par 25 E Giuda, che lo tradiva, prese a dire: Sono io quello, Maestro? E Gesù a lui: L'hai detto.
\par 26 Or mentre mangiavano, Gesù prese del pane; e fatta la benedizione, lo ruppe, e dandolo a' suoi discepoli, disse: Prendete, mangiate, questo è il mio corpo.
\par 27 Poi, preso un calice e rese grazie, lo diede loro, dicendo:
\par 28 Bevetene tutti, perché questo è il mio sangue, il sangue del patto, il quale è sparso per molti per la remissione dei peccati.
\par 29 Io vi dico che d'ora in poi non berrò più di questo frutto della vigna, fino al giorno che lo berrò nuovo con voi nel regno del Padre mio.
\par 30 E dopo ch'ebbero cantato l'inno, uscirono per andare al monte degli Ulivi.
\par 31 Allora Gesù disse loro: Questa notte voi tutti avrete in me un'occasion di caduta; perché è scritto: Io percoterò il pastore, e le pecore della greggia saranno disperse.
\par 32 Ma dopo che sarò risuscitato, vi precederò in Galilea.
\par 33 Ma Pietro, rispondendo, gli disse: Quand'anche tu fossi per tutti un'occasion di caduta, non lo sarai mai per me.
\par 34 Gesù gli disse: In verità ti dico che questa stessa notte, prima che il gallo canti, tu mi rinnegherai tre volte.
\par 35 E Pietro a lui: Quand'anche mi convenisse morir teco, non però ti rinnegherò. E lo stesso dissero pure tutti i discepoli.
\par 36 Allora Gesù venne con loro in un podere detto Getsemani, e disse ai discepoli: Sedete qui finché io sia andato là ed abbia orato.
\par 37 E presi seco Pietro e i due figliuoli di Zebedeo, cominciò ad esser contristato ed angosciato.
\par 38 Allora disse loro: L'anima mia è oppressa da tristezza mortale; rimanete qui e vegliate meco.
\par 39 E andato un poco innanzi, si gettò con la faccia a terra, pregando, e dicendo: Padre mio, se è possibile, passi oltre da me questo calice! Ma pure, non come voglio io, ma come tu vuoi.
\par 40 Poi venne a' discepoli, e li trovò che dormivano, e disse a Pietro: Così, non siete stati capaci di vegliar meco un'ora sola?
\par 41 Vegliate ed orate, affinché non cadiate in tentazione; ben è lo spirito pronto, ma la carne è debole.
\par 42 Di nuovo, per la seconda volta, andò e pregò, dicendo: Padre mio, se non è possibile che questo calice passi oltre da me, senza ch'io lo beva, sia fatta la tua volontà.
\par 43 E tornato, li trovò che dormivano perché gli occhi loro erano aggravati.
\par 44 E lasciatili, andò di nuovo e pregò per la terza volta, ripetendo le medesime parole.
\par 45 Poi venne ai discepoli e disse loro: Dormite pure oramai, e riposatevi! Ecco, l'ora è giunta, e il Figliuol dell'uomo è dato nelle mani dei peccatori.
\par 46 Levatevi, andiamo; ecco, colui che mi tradisce è vicino.
\par 47 E mentre parlava ancora, ecco arrivar Giuda, uno dei dodici, e con lui una gran turba con spade e bastoni, da parte de' capi sacerdoti e degli anziani del popolo.
\par 48 Or colui che lo tradiva, avea dato loro un segnale, dicendo: Quello che bacerò, è lui; pigliatelo.
\par 49 E in quell'istante, accostatosi a Gesù, gli disse: Ti saluto, Maestro! e gli dette un lungo bacio.
\par 50 Ma Gesù gli disse: Amico, a far che sei tu qui? Allora, accostatisi, gli misero le mani addosso, e lo presero.
\par 51 Ed ecco, uno di coloro ch'eran con lui, stesa la mano alla spada, la sfoderò; e percosso il servitore del sommo sacerdote, gli spiccò l'orecchio.
\par 52 Allora Gesù gli disse: Riponi la tua spada al suo posto, perché tutti quelli che prendon la spada, periscon per la spada.
\par 53 Credi tu forse ch'io non potrei pregare il Padre mio che mi manderebbe in quest'istante più di dodici legioni d'angeli?
\par 54 Come dunque si adempirebbero le Scritture, secondo le quali bisogna che così avvenga?
\par 55 In quel punto Gesù disse alle turbe: Voi siete usciti con spade e bastoni come contro ad un ladrone, per pigliarmi. Ogni giorno sedevo nel tempio ad insegnare, e voi non m'avete preso;
\par 56 ma tutto questo è avvenuto affinché si adempissero le scritture dei profeti. Allora tutti i discepoli, lasciatolo, se ne fuggirono.
\par 57 Or quelli che aveano preso Gesù, lo menarono a Caiàfa, sommo sacerdote, presso il quale erano raunati gli scribi e gli anziani.
\par 58 E Pietro lo seguiva da lontano, finché giunsero alla corte del sommo sacerdote; ed entrato dentro, si pose a sedere con le guardie, per veder la fine.
\par 59 Or i capi sacerdoti e tutto il Sinedrio cercavano qualche falsa testimonianza contro a Gesù per farlo morire;
\par 60 e non ne trovavano alcuna, benché si fossero fatti avanti molti falsi testimoni.
\par 61 Finalmente, se ne fecero avanti due che dissero: Costui ha detto: Io posso disfare il tempio di Dio e riedificarlo in tre giorni.
\par 62 E il sommo sacerdote, levatosi in piedi, gli disse: Non rispondi tu nulla? Che testimoniano costoro contro a te? Ma Gesù taceva.
\par 63 E il sommo sacerdote gli disse: Ti scongiuro per l'Iddio vivente a dirci se tu se' il Cristo, il Figliuol di Dio.
\par 64 Gesù gli rispose: Tu l'hai detto; anzi vi dico che da ora innanzi vedrete il Figliuol dell'uomo sedere alla destra della Potenza, e venire su le nuvole del cielo.
\par 65 Allora il sommo sacerdote si stracciò le vesti, dicendo: Egli ha bestemmiato; che bisogno abbiamo più di testimoni? Ecco, ora avete udito la sua bestemmia.
\par 66 Che ve ne pare? Ed essi, rispondendo, dissero: È reo di morte.
\par 67 Allora gli sputarono in viso e gli diedero de' pugni; e altri lo schiaffeggiarono,
\par 68 dicendo: O Cristo profeta, indovinaci: Chi t'ha percosso?
\par 69 Pietro, intanto, stava seduto fuori nella corte; e una serva gli si accostò, dicendo: Anche tu eri con Gesù il Galileo.
\par 70 Ma egli lo negò davanti a tutti, dicendo: Non so quel che tu dica.
\par 71 E come fu uscito fuori nell'antiporto, un'altra lo vide e disse a coloro ch'eran quivi: Anche costui era con Gesù Nazareno.
\par 72 Ed egli daccapo lo negò giurando: Non conosco quell'uomo.
\par 73 Di lì a poco, gli astanti, accostatisi, dissero a Pietro: Per certo tu pure sei di quelli, perché anche la tua parlata ti dà a conoscere.
\par 74 Allora egli cominciò ad imprecare ed a giurare: Non conosco quell'uomo! E in quell'istante il gallo cantò.
\par 75 E Pietro si ricordò della parola di Gesù che gli aveva detto: Prima che il gallo canti, tu mi rinnegherai tre volte. E uscito fuori, pianse amaramente.

\chapter{27}

\par 1 Poi, venuta la mattina, tutti i capi sacerdoti e gli anziani del popolo tennero consiglio contro a Gesù per farlo morire.
\par 2 E legatolo, lo menarono via e lo consegnarono a Pilato, il governatore.
\par 3 Allora Giuda, che l'avea tradito, vedendo che Gesù era stato condannato, si pentì, e riportò i trenta sicli d'argento ai capi sacerdoti ed agli anziani,
\par 4 dicendo: Ho peccato, tradendo il sangue innocente. Ma essi dissero: Che c'importa?
\par 5 Pensaci tu. Ed egli, lanciati i sicli nel tempio, s'allontanò e andò ad impiccarsi.
\par 6 Ma i capi sacerdoti, presi quei sicli, dissero: Non è lecito metterli nel tesoro delle offerte, perché son prezzo di sangue.
\par 7 E tenuto consiglio, comprarono con quel danaro il campo del vasaio da servir di sepoltura ai forestieri.
\par 8 Perciò quel campo, fino al dì d'oggi, è stato chiamato: Campo di sangue.
\par 9 Allora s'adempì quel che fu detto dal profeta Geremia: E presero i trenta sicli d'argento, prezzo di colui ch'era stato messo a prezzo, messo a prezzo dai figliuoli d'Israele;
\par 10 e li dettero per il campo del vasaio, come me l'avea ordinato il Signore.
\par 11 Or Gesù comparve davanti al governatore; e il governatore lo interrogò, dicendo: Sei tu il re de' Giudei? E Gesù gli disse: Sì, lo sono.
\par 12 E accusato da' capi sacerdoti e dagli anziani, non rispose nulla.
\par 13 Allora Pilato gli disse: Non odi tu quante cose testimoniano contro di te?
\par 14 Ma egli non gli rispose neppure una parola: talché il governatore se ne maravigliava grandemente.
\par 15 Or ogni festa di Pasqua il governatore soleva liberare alla folla un carcerato, qualunque ella volesse.
\par 16 Avevano allora un carcerato famigerato, di nome Barabba.
\par 17 Essendo dunque radunati, Pilato domandò loro: Chi volete che vi liberi, Barabba, o Gesù detto Cristo?
\par 18 Poiché egli sapeva che glielo avevano consegnato per invidia.
\par 19 Or mentre egli sedeva in tribunale, la moglie gli mandò a dire: Non aver nulla a che fare con quel giusto, perché oggi ho sofferto molto in sogno a cagion di lui.
\par 20 Ma i capi sacerdoti e gli anziani persuasero le turbe a chieder Barabba e far perire Gesù.
\par 21 E il governatore prese a dir loro: Qual de' due volete che vi liberi? E quelli dissero: Barabba.
\par 22 E Pilato a loro: Che farò dunque di Gesù detto Cristo? Tutti risposero: Sia crocifisso.
\par 23 Ma pure, riprese egli, che male ha fatto? Ma quelli vie più gridavano: Sia crocifisso!
\par 24 E Pilato, vedendo che non riusciva a nulla, ma che si sollevava un tumulto, prese dell'acqua e si lavò le mani in presenza della moltitudine, dicendo: Io sono innocente del sangue di questo giusto; pensateci voi.
\par 25 E tutto il popolo, rispondendo, disse: Il suo sangue sia sopra noi e sopra i nostri figliuoli.
\par 26 Allora egli liberò loro Barabba; e dopo aver fatto flagellare Gesù, lo consegnò perché fosse crocifisso.
\par 27 Allora i soldati del governatore, tratto Gesù nel pretorio, radunarono attorno a lui tutta la coorte.
\par 28 E spogliatolo, gli misero addosso un manto scarlatto;
\par 29 e intrecciata una corona di spine, gliela misero sul capo, e una canna nella man destra; e inginocchiatisi dinanzi a lui, lo beffavano, dicendo: Salve, re dei Giudei!
\par 30 E sputatogli addosso, presero la canna, e gli percotevano il capo.
\par 31 E dopo averlo schernito, lo spogliarono del manto, e lo rivestirono delle sue vesti; poi lo menaron via per crocifiggerlo.
\par 32 Or nell'uscire trovarono un Cireneo chiamato Simone, e lo costrinsero a portar la croce di Gesù.
\par 33 E venuti ad un luogo detto Golgota, che vuol dire: Luogo del teschio, gli dettero a bere del vino mescolato con fiele;
\par 34 ma Gesù, assaggiatolo, non volle berne.
\par 35 Poi, dopo averlo crocifisso, spartirono i suoi vestimenti, tirando a sorte;
\par 36 e postisi a sedere, gli facevan quivi la guardia.
\par 37 E al disopra del capo gli posero scritto il motivo della condanna: QUESTO È GESÙ, IL RE DE' GIUDEI.
\par 38 Allora furon con lui crocifissi due ladroni, uno a destra e l'altro a sinistra.
\par 39 E coloro che passavano di lì, lo ingiuriavano, scotendo il capo e dicendo:
\par 40 Tu che disfai il tempio e in tre giorni lo riedifichi, salva te stesso, se tu sei Figliuol di Dio, e scendi giù di croce!
\par 41 Similmente, i capi sacerdoti con gli scribi e gli anziani, beffandosi, dicevano:
\par 42 Ha salvato altri e non può salvar se stesso! Da che è il re d'Israele, scenda ora giù di croce, e noi crederemo in lui.
\par 43 S'è confidato in Dio; lo liberi ora, s'Ei lo gradisce, poiché ha detto: Son Figliuol di Dio.
\par 44 E nello stesso modo lo vituperavano anche i ladroni crocifissi con lui.
\par 45 Or dall'ora sesta si fecero tenebre per tutto il paese, fino all'ora nona.
\par 46 E verso l'ora nona, Gesù gridò con gran voce: Elì, Elì, lamà sabactanì? cioè: Dio mio, Dio mio, perché mi hai abbandonato?
\par 47 Ma alcuni degli astanti, udito ciò, dicevano: Costui chiama Elia.
\par 48 E subito uno di loro corse a prendere una spugna; e inzuppatala d'aceto e postala in cima ad una canna, gli diè da bere.
\par 49 Ma gli altri dicevano: Lascia, vediamo se Elia viene a salvarlo.
\par 50 E Gesù, avendo di nuovo gridato con gran voce, rendé lo spirito.
\par 51 Ed ecco, la cortina del tempio si squarciò in due, da cima a fondo, e la terra tremò, e le rocce si schiantarono,
\par 52 e le tombe s'aprirono, e molti corpi de' santi che dormivano, risuscitarono;
\par 53 ed usciti dai sepolcri dopo la risurrezione di lui, entrarono nella santa città, ed apparvero a molti.
\par 54 E il centurione e quelli che con lui facean la guardia a Gesù, visto il terremoto e le cose avvenute, temettero grandemente, dicendo: Veramente, costui era Figliuol di Dio.
\par 55 Ora quivi erano molte donne che guardavano da lontano, le quali avean seguitato Gesù dalla Galilea per assisterlo;
\par 56 tra le quali erano Maria Maddalena, e Maria madre di Giacomo e di Jose, e la madre de' figliuoli di Zebedeo.
\par 57 Poi, fattosi sera, venne un uomo ricco di Arimatea, chiamato Giuseppe, il quale era divenuto anch'egli discepolo di Gesù.
\par 58 Questi, presentatosi a Pilato, chiese il corpo di Gesù. Allora Pilato comandò che il corpo gli fosse rilasciato.
\par 59 E Giuseppe, preso il corpo, lo involse in un panno lino netto,
\par 60 e lo pose nella propria tomba nuova, che aveva fatta scavar nella roccia, e dopo aver rotolata una gran pietra contro l'apertura del sepolcro, se ne andò.
\par 61 Or Maria Maddalena e l'altra Maria eran quivi, sedute dirimpetto al sepolcro.
\par 62 E l'indomani, che era il giorno successivo alla Preparazione, i capi sacerdoti ed i Farisei si radunarono presso Pilato, dicendo:
\par 63 Signore, ci siamo ricordati che quel seduttore, mentre viveva ancora, disse: Dopo tre giorni, risusciterò.
\par 64 Ordina dunque che il sepolcro sia sicuramente custodito fino al terzo giorno; che talora i suoi discepoli non vengano a rubarlo e dicano al popolo: È risuscitato dai morti; così l'ultimo inganno sarebbe peggiore del primo.
\par 65 Pilato disse loro: Avete una guardia: andate, assicuratevi come credete.
\par 66 Ed essi andarono ad assicurare il sepolcro, sigillando la pietra, e mettendovi la guardia.

\chapter{28}

\par 1 Or nella notte del sabato, quando già albeggiava, il primo giorno della settimana, Maria Maddalena e l'altra Maria vennero a visitare il sepolcro.
\par 2 Ed ecco si fece un gran terremoto; perché un angelo del Signore, sceso dal cielo, si accostò, rotolò la pietra, e vi sedette sopra.
\par 3 Il suo aspetto era come di folgore; e la sua veste, bianca come neve.
\par 4 E per lo spavento che n'ebbero, le guardie tremarono e rimasero come morte.
\par 5 Ma l'angelo prese a dire alle donne: Voi, non temete; perché io so che cercate Gesù, che è stato crocifisso.
\par 6 Egli non è qui, poiché è risuscitato come avea detto; venite a vedere il luogo dove giaceva.
\par 7 E andate presto a dire a' suoi discepoli: Egli è risuscitato da' morti, ed ecco, vi precede in Galilea; quivi lo vedrete. Ecco, ve l'ho detto.
\par 8 E quelle, andatesene prestamente dal sepolcro con spavento ed allegrezza grande, corsero ad annunziar la cosa a' suoi discepoli.
\par 9 Quand'ecco Gesù si fece loro incontro, dicendo: Vi saluto! Ed esse, accostatesi, gli strinsero i piedi e l'adorarono.
\par 10 Allora Gesù disse loro: Non temete; andate ad annunziare a' miei fratelli che vadano in Galilea; là mi vedranno.
\par 11 Or mentre quelle andavano, ecco alcuni della guardia vennero in città, e riferirono ai capi sacerdoti tutte le cose ch'erano avvenute.
\par 12 Ed essi, radunatisi con gli anziani, e tenuto consiglio, dettero una forte somma di danaro a' soldati, dicendo:
\par 13 Dite così: I suoi discepoli vennero di notte e lo rubarono mentre dormivamo.
\par 14 E se mai questo viene alle orecchie del governatore, noi lo persuaderemo e vi metteremo fuor di pena.
\par 15 Ed essi, preso il danaro, fecero secondo le istruzioni ricevute; e quel dire è stato divulgato fra i Giudei, fino al dì d'oggi.
\par 16 Quanto agli undici discepoli, essi andarono in Galilea sul monte che Gesù avea loro designato.
\par 17 E vedutolo, l'adorarono; alcuni però dubitarono.
\par 18 E Gesù, accostatosi, parlò loro, dicendo: Ogni potestà m'è stata data in cielo e sulla terra.
\par 19 Andate dunque, ammaestrate tutti i popoli, battezzandoli nel nome del Padre e del Figliuolo e dello Spirito Santo,
\par 20 insegnando loro d'osservar tutte quante le cose che v'ho comandate. Ed ecco, io sono con voi tutti i giorni, sino alla fine dell'età presente.


\end{document}