\begin{document}

\title{Luke}


\chapter{1}

\par 1 Poiché molti hanno intrapreso ad ordinare una narrazione de' fatti che si son compiuti tra noi,
\par 2 secondo che ce li hanno tramandati quelli che da principio ne furono testimoni oculari e che divennero ministri della Parola,
\par 3 è parso bene anche a me, dopo essermi accuratamente informato d'ogni cosa dall'origine, di scrivertene per ordine, o eccellentissimo Teofilo,
\par 4 affinché tu riconosca la certezza delle cose che ti sono state insegnate.
\par 5 Ai dì d'Erode, re della Giudea, v'era un certo sacerdote di nome Zaccaria, della muta di Abia; e sua moglie era delle figliuole d'Aaronne e si chiamava Elisabetta.
\par 6 Or erano ambedue giusti nel cospetto di Dio, camminando irreprensibili in tutti i comandamenti e precetti del Signore.
\par 7 E non aveano figliuoli, perché Elisabetta era sterile, ed erano ambedue avanzati in età.
\par 8 Or avvenne che esercitando Zaccaria il sacerdozio dinanzi a Dio nell'ordine della sua muta,
\par 9 secondo l'usanza del sacerdozio, gli toccò a sorte d'entrar nel tempio del Signore per offrirvi il profumo;
\par 10 e tutta la moltitudine del popolo stava di fuori in preghiera nell'ora del profumo.
\par 11 E gli apparve un angelo del Signore, ritto alla destra dell'altare de' profumi.
\par 12 E Zaccaria, vedutolo, fu turbato e preso da spavento.
\par 13 Ma l'angelo gli disse: Non temere, Zaccaria, perché la tua preghiera è stata esaudita; e tua moglie Elisabetta ti partorirà un figliuolo al quale porrai nome Giovanni.
\par 14 E tu ne avrai gioia ed allegrezza, e molti si rallegreranno per la sua nascita.
\par 15 Poiché sarà grande nel cospetto del Signore; non berrà né vino né cervogia, e sarà ripieno dello Spirito Santo fin dal seno di sua madre,
\par 16 e convertirà molti de' figliuoli d'Israele al Signore Iddio loro;
\par 17 ed egli andrà innanzi a lui con lo spirito e con la potenza d'Elia, per volgere i cuori de' padri ai figliuoli e i ribelli alla saviezza de' giusti, affin di preparare al Signore un popolo ben disposto.
\par 18 E Zaccaria disse all'angelo: A che conoscerò io questo? Perch'io son vecchio e mia moglie è avanti nell'età.
\par 19 E l'angelo, rispondendo, gli disse: Io son Gabriele, che sto davanti a Dio; e sono stato mandato a parlarti e recarti questa buona notizia.
\par 20 Ed ecco, tu sarai muto, e non potrai parlare fino al giorno che queste cose avverranno, perché non hai creduto alle mie parole che si adempiranno a suo tempo.
\par 21 Il popolo intanto stava aspettando Zaccaria, e si maravigliava che s'indugiasse tanto nel tempio.
\par 22 Ma quando fu uscito, non potea parlar loro; e capirono che avea avuto una visione nel tempio; ed egli faceva loro dei segni e rimase muto.
\par 23 E quando furon compiuti i giorni del suo ministerio, egli se ne andò a casa sua.
\par 24 Or dopo que' giorni, Elisabetta sua moglie rimase incinta; e si tenne nascosta per cinque mesi, dicendo:
\par 25 Ecco quel che il Signore ha fatto per me ne' giorni nei quali ha rivolto a me lo sguardo per togliere il mio vituperio fra gli uomini.
\par 26 Al sesto mese l'angelo Gabriele fu mandato da Dio in una città di Galilea detta Nazaret
\par 27 ad una vergine fidanzata ad un uomo chiamato Giuseppe, della casa di Davide; e il nome della vergine era Maria.
\par 28 E l'angelo, entrato da lei, disse: Ti saluto, o favorita dalla grazia; il Signore è teco.
\par 29 Ed ella fu turbata a questa parola, e si domandava che cosa volesse dire un tal saluto.
\par 30 E l'angelo le disse: Non temere, Maria, perché hai trovato grazia presso Dio.
\par 31 Ed ecco tu concepirai nel seno e partorirai un figliuolo e gli porrai nome Gesù.
\par 32 Questi sarà grande, e sarà chiamato Figliuol dell'Altissimo, e il Signore Iddio gli darà il trono di Davide suo padre,
\par 33 ed egli regnerà sulla casa di Giacobbe in eterno, e il suo regno non avrà mai fine.
\par 34 E Maria disse all'angelo: Come avverrà questo, poiché non conosco uomo?
\par 35 E l'angelo rispondendo, le disse: Lo Spirito Santo verrà su di te e la potenza dell'Altissimo ti coprirà dell'ombra sua; perciò ancora il santo che nascerà, sarà chiamato Figliuolo di Dio.
\par 36 Ed ecco, Elisabetta, tua parente, ha concepito anche lei un figliuolo nella sua vecchiaia; e questo è il sesto mese per lei, ch'era chiamata sterile;
\par 37 poiché nessuna parola di Dio rimarrà inefficace.
\par 38 E Maria disse: Ecco, io son l'ancella del Signore; siami fatto secondo la tua parola. E l'angelo si partì da lei.
\par 39 In que' giorni Maria si levò e se ne andò in fretta nella regione montuosa, in una città di Giuda,
\par 40 ed entrò in casa di Zaccaria e salutò Elisabetta.
\par 41 E avvenne che come Elisabetta ebbe udito il saluto di Maria, il bambino le balzò nel seno; ed Elisabetta fu ripiena di Spirito Santo,
\par 42 e a gran voce esclamò: Benedetta sei tu fra le donne, e benedetto è il frutto del tuo seno!
\par 43 E come mai m'è dato che la madre del mio Signore venga da me?
\par 44 Poiché ecco, non appena la voce del tuo saluto m'è giunta agli orecchi, il bambino m'è per giubilo balzato nel seno.
\par 45 E beata è colei che ha creduto, perché le cose dettele da parte del Signore avranno compimento.
\par 46 E Maria disse: "L'anima mia magnifica il Signore,
\par 47 e lo spirito mio esulta in Dio mio Salvatore,
\par 48 poich'egli ha riguardato alla bassezza della sua ancella. Perché ecco, d'ora innanzi tutte le età mi chiameranno beata,
\par 49 poiché il Potente mi ha fatto grandi cose. Santo è il suo nome;
\par 50 e la sua misericordia è d'età in età per quelli che lo temono.
\par 51 Egli ha operato potentemente col suo braccio; ha disperso quelli ch'eran superbi ne' pensieri del cuor loro;
\par 52 ha tratto giù dai troni i potenti, ed ha innalzato gli umili;
\par 53 ha ricolmato di beni i famelici, e ha rimandati a vuoto i ricchi.
\par 54 Ha soccorso Israele, suo servitore, ricordandosi della misericordia
\par 55 di cui avea parlato ai nostri padri, verso Abramo e verso la sua progenie in perpetuo".
\par 56 E Maria rimase con Elisabetta circa tre mesi; poi se ne tornò a casa sua.
\par 57 Or compiutosi per Elisabetta il tempo di partorire, diè alla luce un figliuolo.
\par 58 E i suoi vicini e i parenti udirono che il Signore avea magnificata la sua misericordia verso di lei, e se ne rallegravano con essa.
\par 59 Ed ecco che nell'ottavo giorno vennero a circoncidere il bambino, e lo chiamavano Zaccaria dal nome di suo padre.
\par 60 Allora sua madre prese a parlare e disse: No, sarà invece chiamato Giovanni.
\par 61 Ed essi le dissero: Non v'è alcuno nel tuo parentado che porti questo nome.
\par 62 E per cenni domandavano al padre come voleva che fosse chiamato.
\par 63 Ed egli, chiesta una tavoletta, scrisse così: Il suo nome è Giovanni. E tutti si maravigliarono.
\par 64 In quell'istante la sua bocca fu aperta e la sua lingua sciolta, ed egli parlava benedicendo Iddio.
\par 65 E tutti i lor vicini furon presi da timore; e tutte queste cose si divulgavano per tutta la regione montuosa della Giudea.
\par 66 E tutti quelli che le udirono, le serbarono in cuor loro e diceano: Che sarà mai questo bambino? Perché la mano del Signore era con lui.
\par 67 E Zaccaria, suo padre, fu ripieno dello Spirito Santo, e profetò, dicendo:
\par 68 "Benedetto sia il Signore, l'Iddio d'Israele, perché ha visitato e riscattato il suo popolo,
\par 69 e ci ha suscitato un potente salvatore nella casa di Davide suo servitore
\par 70 (come avea promesso ab antico per bocca de' suoi profeti);
\par 71 uno che ci salverà da' nostri nemici e dalle mani di tutti quelli che ci odiano.
\par 72 Egli usa così misericordia verso i nostri padri e si ricorda del suo santo patto,
\par 73 del giuramento che fece ad Abramo nostro padre,
\par 74 affine di concederci che, liberati dalla mano dei nostri nemici, gli servissimo senza paura,
\par 75 in santità e giustizia, nel suo cospetto, tutti i giorni della nostra vita.
\par 76 E tu, piccol fanciullo, sarai chiamato profeta dell'Altissimo, perché andrai davanti alla faccia del Signore per preparar le sue vie,
\par 77 per dare al suo popolo conoscenza della salvezza mediante la remissione de' loro peccati,
\par 78 dovuta alle viscere di misericordia del nostro Dio, per le quali l'Aurora dall'alto ci visiterà
\par 79 per risplendere su quelli che giacciono in tenebre ed ombra di morte, per guidare i nostri passi verso la via della pace".
\par 80 Or il bambino cresceva e si fortificava in ispirito; e stette ne' deserti fino al giorno in cui dovea manifestarsi ad Israele.

\chapter{2}

\par 1 Or in que' dì avvenne che un decreto uscì da parte di Cesare Augusto, che si facesse un censimento di tutto l'impero.
\par 2 Questo censimento fu il primo fatto mentre Quirino governava la Siria.
\par 3 E tutti andavano a farsi registrare, ciascuno alla sua città.
\par 4 Or anche Giuseppe salì di Galilea, dalla città di Nazaret, in Giudea, alla città di Davide, chiamata Betleem, perché era della casa e famiglia di Davide,
\par 5 a farsi registrare con Maria sua sposa, che era incinta.
\par 6 E avvenne che, mentre eran quivi, si compié per lei il tempo del parto;
\par 7 ed ella diè alla luce il suo figliuolo primogenito, e lo fasciò, e lo pose a giacere in una mangiatoia, perché non v'era posto per loro nell'albergo.
\par 8 Or in quella medesima contrada v'eran de' pastori che stavano ne' campi e facean di notte la guardia al loro gregge.
\par 9 E un angelo del Signore si presentò ad essi e la gloria del Signore risplendé intorno a loro, e temettero di gran timore.
\par 10 E l'angelo disse loro: Non temete, perché ecco, vi reco il buon annunzio di una grande allegrezza che tutto il popolo avrà:
\par 11 Oggi, nella città di Davide, v'è nato un salvatore, che è Cristo, il Signore.
\par 12 E questo vi servirà di segno: troverete un bambino fasciato e coricato in una mangiatoia.
\par 13 E ad un tratto vi fu con l'angelo una moltitudine dell'esercito celeste, che lodava Iddio e diceva:
\par 14 Gloria a Dio ne' luoghi altissimi, pace in terra fra gli uomini ch'Egli gradisce!
\par 15 E avvenne che quando gli angeli se ne furono andati da loro verso il cielo, i pastori presero a dire tra loro: Passiamo fino a Betleem e vediamo questo che è avvenuto, e che il Signore ci ha fatto sapere.
\par 16 E andarono in fretta, e trovarono Maria e Giuseppe ed il bambino giacente nella mangiatoia;
\par 17 e vedutolo, divulgarono ciò ch'era loro stato detto di quel bambino.
\par 18 E tutti quelli che li udirono, si maravigliarono delle cose dette loro dai pastori.
\par 19 Or Maria serbava in sé tutte quelle cose, collegandole insieme in cuor suo.
\par 20 E i pastori se ne tornarono, glorificando e lodando Iddio per tutto quello che aveano udito e visto, com'era loro stato annunziato.
\par 21 E quando furono compiuti gli otto giorni in capo ai quali e' doveva esser circonciso, gli fu posto il nome di Gesù, che gli era stato dato dall'angelo prima ch'ei fosse concepito nel seno.
\par 22 E quando furon compiuti i giorni della loro purificazione secondo la legge di Mosè, portarono il bambino in Gerusalemme per presentarlo al Signore,
\par 23 com'è scritto nella legge del Signore: Ogni maschio primogenito sarà chiamato santo al Signore,
\par 24 e per offrire il sacrificio di cui parla la legge del Signore, di un paio di tortore o di due giovani piccioni.
\par 25 Ed ecco, v'era in Gerusalemme un uomo di nome Simeone; e quest'uomo era giusto e timorato di Dio, e aspettava la consolazione d'Israele; e lo Spirito Santo era sopra lui;
\par 26 e gli era stato rivelato dallo Spirito Santo che non vedrebbe la morte prima d'aver veduto il Cristo del Signore.
\par 27 Ed egli, mosso dallo Spirito, venne nel tempio; e come i genitori vi portavano il bambino Gesù per adempiere a suo riguardo le prescrizioni della legge,
\par 28 se lo prese anch'egli nelle braccia, e benedisse Iddio e disse:
\par 29 "Ora, o mio Signore, tu lasci andare in pace il tuo servo, secondo la tua parola;
\par 30 poiché gli occhi miei han veduto la tua salvezza,
\par 31 che hai preparata dinanzi a tutti i popoli
\par 32 per esser luce da illuminar le genti, e gloria del tuo popolo Israele".
\par 33 E il padre e la madre di Gesù restavano maravigliati delle cose che dicevan di lui.
\par 34 E Simeone li benedisse, e disse a Maria, madre di lui: Ecco, questi è posto a caduta ed a rialzamento di molti in Israele, e per segno a cui si contradirà
\par 35 (e a te stessa una spada trapasserà l'anima), affinché i pensieri di molti cuori siano rivelati.
\par 36 V'era anche Anna, profetessa, figliuola di Fanuel, della tribù di Aser, la quale era molto attempata. Dopo esser vissuta col marito sette anni dalla sua verginità,
\par 37 era rimasta vedova ed avea raggiunto gli ottantaquattro anni. Ella non si partiva mai dal tempio, servendo a Dio notte e giorno con digiuni ed orazioni.
\par 38 Sopraggiunta in quell'istessa ora, lodava anch'ella Iddio e parlava del bambino a tutti quelli che aspettavano la redenzione di Gerusalemme.
\par 39 E come ebbero adempiuto tutte le prescrizioni della legge del Signore, tornarono in Galilea, a Nazaret, loro città.
\par 40 E il bambino cresceva e si fortificava, essendo ripieno di sapienza; e la grazia di Dio era sopra lui.
\par 41 Or i suoi genitori andavano ogni anno a Gerusalemme per la festa di Pasqua.
\par 42 E quando egli fu giunto ai dodici anni, salirono a Gerusalemme, secondo l'usanza della festa;
\par 43 e passati i giorni della festa, come se ne tornavano, il fanciullo Gesù rimase in Gerusalemme all'insaputa dei genitori;
\par 44 i quali, stimando ch'egli fosse nella comitiva, camminarono una giornata, e si misero a cercarlo fra i parenti e i conoscenti;
\par 45 e, non avendolo trovato, tornarono a Gerusalemme facendone ricerca.
\par 46 Ed avvenne che tre giorni dopo lo trovarono nel tempio, seduto in mezzo a' dottori, che li ascoltava e faceva loro delle domande;
\par 47 e tutti quelli che l'udivano, stupivano del suo senno e delle sue risposte.
\par 48 E, vedutolo, sbigottirono; e sua madre gli disse: Figliuolo, perché ci hai fatto così? Ecco, tuo padre ed io ti cercavamo, stando in gran pena.
\par 49 Ed egli disse loro: Perché mi cercavate? Non sapevate ch'io dovea trovarmi nella casa del Padre mio?
\par 50 Ed essi non intesero la parola ch'egli avea lor detta.
\par 51 E discese con loro, e venne a Nazaret, e stava loro sottomesso. E sua madre serbava tutte queste cose in cuor suo.
\par 52 E Gesù cresceva in sapienza e in statura, e in grazia dinanzi a Dio e agli uomini.

\chapter{3}

\par 1 Or nell'anno decimoquinto dell'impero di Tiberio Cesare, essendo Ponzio Pilato governatore della Giudea, ed Erode tetrarca della Galilea, e Filippo, suo fratello, tetrarca dell'Iturea e della Traconitide, e Lisania tetrarca dell'Abilene,
\par 2 sotto i sommi sacerdoti Anna e Caiàfa, la parola di Dio fu diretta a Giovanni, figliuol di Zaccaria nel deserto.
\par 3 Ed egli andò per tutta la contrada d'intorno al Giordano, predicando un battesimo di ravvedimento per la remissione de' peccati,
\par 4 secondo che è scritto nel libro delle parole del profeta Isaia: V'è una voce d'uno che grida nel deserto: Preparate la via del Signore, addirizzate i suoi sentieri.
\par 5 Ogni valle sarà colmata ed ogni monte ed ogni colle sarà abbassato; le vie tortuose saran fatte diritte e le scabre saranno appianate;
\par 6 ed ogni carne vedrà la salvezza di Dio.
\par 7 Giovanni dunque diceva alle turbe che uscivano per esser battezzate da lui: Razza di vipere, chi v'ha mostrato a fuggir dall'ira a venire?
\par 8 Fate dunque dei frutti degni del ravvedimento, e non vi mettete a dire in voi stessi: Noi abbiamo Abramo per padre! Perché vi dico che Iddio può da queste pietre far sorgere dei figliuoli ad Abramo.
\par 9 E ormai è anche posta la scure alla radice degli alberi; ogni albero dunque che non fa buon frutto, vien tagliato e gittato nel fuoco.
\par 10 E le turbe lo interrogavano, dicendo: E allora, che dobbiam fare?
\par 11 ed egli rispondeva loro: Chi ha due tuniche, ne faccia parte a chi non ne ha; e chi ha da mangiare, faccia altrettanto.
\par 12 Or vennero anche dei pubblicani per esser battezzati, e gli dissero: Maestro, che dobbiam fare?
\par 13 Ed egli rispose loro: Non riscotete nulla di più di quello che v'è ordinato.
\par 14 Lo interrogaron pure de' soldati, dicendo: E noi, che dobbiamo fare? Ed egli a loro: Non fate estorsioni, né opprimete alcuno con false denunzie e contentatevi della vostra paga.
\par 15 Or stando il popolo in aspettazione e domandandosi tutti in cuor loro riguardo a Giovanni se talora non fosse lui il Cristo,
\par 16 Giovanni rispose, dicendo a tutti: Ben vi battezzo io con acqua; ma vien colui che è più forte di me, al quale io non son degno di sciogliere il legaccio dei calzari. Egli vi battezzerà con lo Spirito Santo e col fuoco.
\par 17 Egli ha in mano il suo ventilabro per nettare interamente l'aia sua, e raccogliere il grano nel suo granaio; ma quant'è alla pula la brucerà con fuoco inestinguibile.
\par 18 Così, con molte e varie esortazioni, evangelizzava il popolo;
\par 19 ma Erode, il tetrarca, essendo da lui ripreso riguardo ad Erodiada, moglie di suo fratello, e per tutte le malvagità ch'esso Erode avea commesse,
\par 20 aggiunse a tutte le altre anche questa, di rinchiudere Giovanni in prigione.
\par 21 Or avvenne che come tutto il popolo si faceva battezzare, essendo anche Gesù stato battezzato, mentre stava pregando, s'aprì il cielo,
\par 22 e lo Spirito Santo scese su lui in forma corporea a guisa di colomba; e venne una voce dal cielo: Tu sei il mio diletto Figliuolo; in te mi sono compiaciuto.
\par 23 E Gesù, quando cominciò anch'egli ad insegnare, avea circa trent'anni ed era figliuolo, come credevasi, di Giuseppe,
\par 24 di Heli, di Matthat, di Levi, di Melchi, di Jannai, di Giuseppe,
\par 25 di Mattatia, di Amos, di Naum, di Esli, di Naggai,
\par 26 di Maath, di Mattatia, di Semein, di Josech, di Joda,
\par 27 di Joanan, di Rhesa, di Zorobabele, di Salatiel, di Neri,
\par 28 di Melchi, di Addi, di Cosam, di Elmadam, di Er,
\par 29 di Gesù, di Eliezer, di Jorim, di Matthat,
\par 30 di Levi, di Simeone, di Giuda, di Giuseppe, di Jonam, di Eliakim,
\par 31 di Melea, di Menna, di Mattatha, di Nathan, di Davide,
\par 32 di Jesse, di Jobed, di Boos, di Sala, di Naasson,
\par 33 di Aminadab, di Admin, di Arni, di Esrom, di Fares, di Giuda,
\par 34 di Giacobbe, d'Isacco, d'Abramo, di Tara, di Nachor,
\par 35 di Seruch, di Ragau, di Falek, di Eber, di Sala,
\par 36 di Cainam, di Arfacsad, di Sem, di Noè,
\par 37 di Lamech, di Mathusala, di Enoch, di Jaret, di Maleleel,
\par 38 di Cainam, di Enos, di Seth, di Adamo, di Dio.

\chapter{4}

\par 1 Or Gesù, ripieno dello Spirito Santo, se ne ritornò dal Giordano, e fu condotto dallo Spirito nel deserto per quaranta giorni, ed era tentato dal diavolo.
\par 2 E durante quei giorni non mangiò nulla; e dopo che quelli furon trascorsi, ebbe fame.
\par 3 E il diavolo gli disse: Se tu sei Figliuol di Dio, di' a questa pietra che diventi pane.
\par 4 E Gesù gli rispose: Sta scritto: Non di pane soltanto vivrà l'uomo.
\par 5 E il diavolo, menatolo in alto, gli mostrò in un attimo tutti i regni del mondo e gli disse:
\par 6 Ti darò tutta quanta questa potenza e la gloria di questi regni: perch'essa mi è stata data, e la do a chi voglio.
\par 7 Se dunque tu ti prostri ad adorarmi, sarà tutta tua.
\par 8 E Gesù, rispondendo, gli disse: Sta scritto: Adora il Signore Iddio tuo, e a lui solo rendi il tuo culto.
\par 9 Poi lo menò a Gerusalemme e lo pose sul pinnacolo del tempio e gli disse: Se tu sei Figliuolo di Dio, gettati giù di qui;
\par 10 perché sta scritto: Egli ordinerà ai suoi angeli intorno a te, che ti proteggano;
\par 11 ed essi ti porteranno sulle mani, che talora tu non urti col piede contro una pietra.
\par 12 E Gesù, rispondendo, gli disse: È stato detto: Non tentare il Signore Iddio tuo.
\par 13 Allora il diavolo, finita che ebbe ogni sorta di tentazione, si partì da lui fino ad altra occasione.
\par 14 E Gesù, nella potenza dello spirito, se ne tornò in Galilea; e la sua fama si sparse per tutta la contrada circonvicina.
\par 15 E insegnava nelle loro sinagoghe, glorificato da tutti.
\par 16 E venne a Nazaret, dov'era stato allevato; e com'era solito, entrò in giorno di sabato nella sinagoga, e alzatosi per leggere,
\par 17 gli fu dato il libro del profeta Isaia; e aperto il libro trovò quel passo dov'era scritto:
\par 18 Lo Spirito del Signore è sopra di me; per questo egli mi ha unto per evangelizzare i poveri; mi ha mandato a bandir liberazione a' prigionieri, ed ai ciechi ricupero della vista; a rimettere in libertà gli oppressi,
\par 19 e a predicare l'anno accettevole del Signore.
\par 20 Poi, chiuso il libro e resolo all'inserviente, si pose a sedere; e gli occhi di tutti nella sinagoga erano fissi in lui.
\par 21 Ed egli prese a dir loro: Oggi, s'è adempiuta questa scrittura, e voi l'udite.
\par 22 E tutti gli rendeano testimonianza, e si maravigliavano delle parole di grazia che uscivano dalla sua bocca, e dicevano: Non è costui il figliuol di Giuseppe?
\par 23 Ed egli disse loro: Certo, voi mi citerete questo proverbio: medico, cura te stesso; fa' anche qui nella tua patria tutto quello che abbiamo udito essere avvenuto in Capernaum!
\par 24 Ma egli disse: In verità vi dico che nessun profeta è ben accetto nella sua patria.
\par 25 Anzi, vi dico in verità che ai dì d'Elia, quando il cielo fu serrato per tre anni e sei mesi e vi fu gran carestia in tutto il paese, c'eran molte vedove in Israele;
\par 26 eppure a nessuna di esse fu mandato Elia, ma fu mandato a una vedova in Sarepta di Sidon.
\par 27 E al tempo del profeta Eliseo, c'eran molti lebbrosi in Israele; eppure nessuno di loro fu mondato, ma lo fu Naaman il Siro.
\par 28 E tutti, nella sinagoga, furon ripieni d'ira all'udir queste cose.
\par 29 E levatisi, lo cacciaron fuori della città, e lo menarono fin sul ciglio del monte sul quale era fabbricata la loro città, per precipitarlo giù.
\par 30 Ma egli, passando in mezzo a loro, se ne andò.
\par 31 E scese a Capernaum, città di Galilea; e vi stava ammaestrando la gente nei giorni di sabato.
\par 32 Ed essi stupivano della sua dottrina perché parlava con autorità.
\par 33 Or nella sinagoga si trovava un uomo posseduto da uno spirito d'immondo demonio, il quale gridò con gran voce: Ahi!
\par 34 Che v'è fra noi e te, o Gesù Nazareno? Se' tu venuto per perderci? Io so chi tu sei: il Santo di Dio!
\par 35 E Gesù lo sgridò, dicendo: Ammutolisci, ed esci da quest'uomo! E il demonio, gettatolo a terra in mezzo alla gente, uscì da lui senza fargli alcun male.
\par 36 E tutti furon presi da sbigottimento e ragionavan fra loro, dicendo: Qual parola è questa? Egli comanda con autorità e potenza agli spiriti immondi, ed essi escono.
\par 37 E la sua fama si spargeva in ogni parte della circostante contrada.
\par 38 Poi, levatosi ed uscito dalla sinagoga, entrò in casa di Simone. Or la suocera di Simone era travagliata da una gran febbre; e lo pregarono per lei.
\par 39 Ed egli, chinatosi verso di lei, sgridò la febbre, e la febbre la lasciò; ed ella, alzatasi prontamente, si mise a servirli.
\par 40 E sul tramontar del sole, tutti quelli che aveano degli infermi di varie malattie, li menavano a lui; ed egli li guariva, imponendo le mani a ciascuno.
\par 41 Anche i demonî uscivano da molti, gridando e dicendo: Tu sei il Figliuol di Dio! Ed egli li sgridava e non permetteva loro di parlare, perché sapevano ch'egli era il Cristo.
\par 42 Poi, fattosi giorno, uscì e andò in un luogo deserto; e le turbe lo cercavano e giunsero fino a lui; e lo trattenevano perché non si partisse da loro.
\par 43 Ma egli disse loro: Anche alle altre città bisogna ch'io evangelizzi il regno di Dio; poiché per questo sono stato mandato.
\par 44 E andava predicando per le sinagoghe della Galilea.

\chapter{5}

\par 1 Or avvenne che essendogli la moltitudine addosso per udir la parola di Dio, e stando egli in piè sulla riva del lago di Gennesaret,
\par 2 vide due barche ferme a riva, dalle quali erano smontati i pescatori e lavavano le reti.
\par 3 E montato in una di quelle barche che era di Simone, lo pregò di scostarsi un po' da terra; poi, sedutosi, d'in sulla barca ammaestrava le turbe.
\par 4 E com'ebbe cessato di parlare, disse a Simone: Prendi il largo, e calate le reti per pescare.
\par 5 E Simone, rispondendo, disse: Maestro, tutta la notte ci siamo affaticati, e non abbiam preso nulla; però, alla tua parola, calerò le reti.
\par 6 E fatto così, presero una tal quantità di pesci, che le reti si rompevano.
\par 7 E fecero segno a' loro compagni dell'altra barca, di venire ad aiutarli. E quelli vennero, e riempirono ambedue le barche, talché affondavano.
\par 8 Simon Pietro, veduto ciò, si gettò a' ginocchi di Gesù, dicendo: Signore, dipàrtiti da me, perché son uomo peccatore.
\par 9 Poiché spavento avea preso lui e tutti quelli ch'eran con lui, per la presa di pesci che avean fatta;
\par 10 e così pure Giacomo e Giovanni, figliuoli di Zebedeo, ch'eran soci di Simone. E Gesù disse a Simone: Non temere: da ora innanzi sarai pescator d'uomini.
\par 11 Ed essi, tratte le barche a terra, lasciarono ogni cosa e lo seguirono.
\par 12 Ed avvenne che, trovandosi egli in una di quelle città, ecco un uomo pien di lebbra, il quale, veduto Gesù e gettatosi con la faccia a terra, lo pregò dicendo: Signore, se tu vuoi, tu puoi mondarmi.
\par 13 Ed egli, stesa la mano, lo toccò dicendo: Lo voglio, sii mondato. E in quell'istante la lebbra sparì da lui.
\par 14 E Gesù gli comandò di non dirlo a nessuno: Ma va', gli disse, mostrati al sacerdote ed offri per la tua purificazione quel che ha prescritto Mosè; e ciò serva loro di testimonianza.
\par 15 Però la fama di lui si spandeva sempre più; e molte turbe si adunavano per udirlo ed esser guarite delle loro infermità.
\par 16 Ma egli si ritirava ne' luoghi deserti e pregava.
\par 17 Ed avvenne, in uno di que' giorni, ch'egli stava insegnando; ed eran quivi seduti de' Farisei e de' dottori della legge, venuti da tutte le borgate della Galilea, della Giudea e da Gerusalemme; e la potenza del Signore era con lui per compier delle guarigioni.
\par 18 Ed ecco degli uomini che portavano sopra un letto un paralitico, e cercavano di portarlo dentro e di metterlo davanti a lui.
\par 19 E non trovando modo d'introdurlo a motivo della calca, salirono sul tetto, e fatta un'apertura fra i tegoli, lo calaron giù col suo lettuccio, in mezzo alla gente, davanti a Gesù.
\par 20 Ed egli, veduta la loro fede, disse: O uomo, i tuoi peccati ti sono rimessi.
\par 21 Allora gli scribi e i Farisei cominciarono a ragionare, dicendo: Chi è costui che pronunzia bestemmie? Chi può rimettere i peccati se non Dio solo?
\par 22 Ma Gesù, conosciuti i loro ragionamenti, prese a dir loro: Che ragionate nei vostri cuori?
\par 23 Che cosa è più agevole dire: I tuoi peccati ti son rimessi, oppur dire: Lèvati e cammina?
\par 24 Ora, affinché sappiate che il Figliuol dell'uomo ha sulla terra autorità di rimettere i peccati: Io tel dico (disse al paralitico), lèvati, togli il tuo lettuccio, e vattene a casa tua.
\par 25 E in quell'istante, alzatosi in presenza loro e preso il suo giaciglio, se ne andò a casa sua, glorificando Iddio.
\par 26 E tutti furon presi da stupore e glorificavano Iddio; e pieni di spavento, dicevano: Oggi abbiamo visto cose strane.
\par 27 E dopo queste cose, egli uscì e notò un pubblicano, di nome Levi, che sedeva al banco della gabella, e gli disse: Seguimi.
\par 28 Ed egli, lasciata ogni cosa, si levò e si mise a seguirlo.
\par 29 E Levi gli fece un gran convito in casa sua; e c'era gran folla di pubblicani e d'altri che erano a tavola con loro.
\par 30 E i Farisei ed i loro scribi mormoravano contro i discepoli di Gesù, dicendo: Perché mangiate e bevete coi pubblicani e coi peccatori?
\par 31 E Gesù, rispondendo disse loro: I sani non hanno bisogno del medico, bensì i malati.
\par 32 Io non son venuto a chiamar de' giusti, ma de' peccatori a ravvedimento.
\par 33 Ed essi gli dissero: I discepoli di Giovanni digiunano spesso e fanno orazioni; così pure i discepoli de' Farisei; mentre i tuoi mangiano e bevono.
\par 34 E Gesù disse loro: Potete voi far digiunare gli amici dello sposo, mentre lo sposo è con loro?
\par 35 Ma verranno i giorni per questo; e quando lo sposo sarà loro tolto, allora, in que' giorni, digiuneranno.
\par 36 Disse loro anche una parabola: Nessuno strappa un pezzo da un vestito nuovo per metterlo ad un vestito vecchio; altrimenti strappa il nuovo, e il pezzo tolto dal nuovo non si adatta al vecchio.
\par 37 E nessuno mette vin nuovo in otri vecchi; altrimenti il vin nuovo rompe gli otri, il vino si spande, e gli otri vanno perduti.
\par 38 Ma il vin nuovo va messo in otri nuovi.
\par 38 Ma il vin nuovo va messo in otri nuovi.

\chapter{6}

\par 1 Or avvenne che in un giorno di sabato egli passava per i seminati; e i suoi discepoli svellevano delle spighe, e sfregandole con le mani, mangiavano.
\par 2 Ed alcuni de' Farisei dissero: Perché fate quel che non è lecito nel giorno del sabato?
\par 3 E Gesù, rispondendo, disse loro: Non avete letto neppure quel che fece Davide, quand'ebbe fame, egli e coloro ch'eran con lui?
\par 4 Com'entrò nella casa di Dio, e prese i pani di presentazione, e ne mangiò e ne diede anche a coloro che eran con lui, quantunque non sia lecito mangiarne se non ai soli sacerdoti?
\par 5 E diceva loro: Il Figliuol dell'uomo è Signore del sabato.
\par 6 Or avvenne in un altro sabato ch'egli entrò nella sinagoga, e si mise ad insegnare. E quivi era un uomo che avea la mano destra secca.
\par 7 Or gli scribi e i Farisei l'osservavano per vedere se farebbe una guarigione in giorno di sabato, per trovar di che accusarlo.
\par 8 Ma egli conosceva i loro pensieri, e disse all'uomo che avea la mano secca: Lèvati, e sta' su nel mezzo! Ed egli, alzatosi, stette su.
\par 9 Poi Gesù disse loro: Io domando a voi: È lecito, in giorno di sabato, di far del bene o di far del male? di salvare una persona o di ucciderla?
\par 10 E girato lo sguardo intorno su tutti loro, disse a quell'uomo: Stendi la mano! Egli fece così, e la sua mano tornò sana.
\par 11 Ed essi furon ripieni di furore e discorreano fra loro di quel che potrebbero fare a Gesù.
\par 12 Or avvenne in que' giorni ch'egli se ne andò sul monte a pregare, e passò la notte in orazione a Dio.
\par 13 E quando fu giorno, chiamò a sé i suoi discepoli, e ne elesse dodici, ai quali dette anche il nome di apostoli:
\par 14 Simone, che nominò anche Pietro, e Andrea, fratello di lui, e Giacomo e Giovanni, e Filippo e Bartolommeo,
\par 15 e Matteo e Toma, e Giacomo d'Alfeo e Simone chiamato Zelota,
\par 16 e Giuda di Giacomo, e Giuda Iscariot che divenne poi traditore.
\par 17 E sceso con loro, si fermò sopra un ripiano, insieme con gran folla dei suoi discepoli e gran quantità di popolo da tutta la Giudea e da Gerusalemme e dalla marina di Tiro e di Sidone,
\par 18 i quali eran venuti per udirlo e per esser guariti delle loro infermità.
\par 19 E quelli che eran tormentati da spiriti immondi, erano guariti; e tutta la moltitudine cercava di toccarlo, perché usciva da lui una virtù che sanava tutti.
\par 20 Ed egli, alzati gli occhi verso i suoi discepoli, diceva: Beati voi che siete poveri, perché il Regno di Dio è vostro.
\par 21 Beati voi che ora avete fame, perché sarete saziati. Beati voi che ora piangete, perché riderete.
\par 22 Beati voi, quando gli uomini v'avranno odiati, e quando v'avranno sbanditi d'infra loro, e v'avranno vituperati ed avranno ripudiato il vostro nome come malvagio, per cagione del Figliuol dell'uomo.
\par 23 Rallegratevi in quel giorno e saltate di letizia, perché, ecco, il vostro premio è grande ne' cieli; poiché i padri loro facean lo stesso a' profeti.
\par 24 Ma guai a voi, ricchi, perché avete già la vostra consolazione.
\par 25 Guai a voi che siete ora satolli, perché avrete fame. Guai a voi che ora ridete, perché farete cordoglio e piangerete.
\par 26 Guai a voi quando tutti gli uomini diran bene di voi, perché i padri loro facean lo stesso coi falsi profeti.
\par 27 Ma a voi che ascoltate, io dico: Amate i vostri nemici; fate del bene a quelli che v'odiano;
\par 28 benedite quelli che vi maledicono, pregate per quelli che v'oltraggiano.
\par 29 A chi ti percuote su una guancia, porgigli anche l'altra; e a chi ti toglie il mantello non impedire di prenderti anche la tunica.
\par 30 Da' a chiunque ti chiede; e a chi ti toglie il tuo, non glielo ridomandare.
\par 31 E come volete che gli uomini facciano a voi, fate voi pure a loro.
\par 32 E se amate quelli che vi amano, qual grazia ve ne viene? Poiché anche i peccatori amano quelli che li amano.
\par 33 E se fate del bene a quelli che vi fanno del bene, qual grazia ve ne viene? Anche i peccatori fanno lo stesso.
\par 34 E se prestate a quelli dai quali sperate ricevere, qual grazia ne avete? Anche i peccatori prestano ai peccatori per riceverne altrettanto.
\par 35 Ma amate i vostri nemici, e fate del bene e prestate senza sperarne alcun che, e il vostro premio sarà grande e sarete figliuoli dell'Altissimo; poich'Egli è benigno verso gl'ingrati e malvagi.
\par 36 Siate misericordiosi com'è misericordioso il Padre vostro.
\par 37 Non giudicate, e non sarete giudicati; non condannate, e non sarete condannati; perdonate, e vi sarà perdonato.
\par 38 Date, e vi sarà dato: vi sarà versata in seno buona misura, pigiata, scossa, traboccante; perché con la misura onde misurate, sarà rimisurato a voi.
\par 39 Poi disse loro anche una parabola: Un cieco può egli guidare un cieco? Non cadranno tutti e due nella fossa?
\par 40 Un discepolo non è da più del maestro; ma ogni discepolo perfetto sarà come il suo maestro.
\par 41 Or perché guardi tu il bruscolo che è nell'occhio del tuo fratello, mentre non iscorgi la trave che è nell'occhio tuo proprio?
\par 42 Come puoi dire al tuo fratello: Fratello, lascia ch'io ti tragga il bruscolo che hai nell'occhio, mentre tu stesso non vedi la trave ch'è nell'occhio tuo? Ipocrita, trai prima dall'occhio tuo la trave, e allora ci vedrai bene per trarre il bruscolo che è nell'occhio del tuo fratello.
\par 43 Non v'è infatti albero buono che faccia frutto cattivo, né v'è albero cattivo che faccia frutto buono;
\par 44 poiché ogni albero si riconosce dal suo proprio frutto; perché non si colgon fichi dalle spine, né si vendemmia uva dal pruno.
\par 45 L'uomo buono dal buon tesoro del suo cuore reca fuori il bene; e l'uomo malvagio, dal malvagio tesoro reca fuori il male; poiché dall'abbondanza del cuore parla la sua bocca.
\par 46 Perché mi chiamate Signore, Signore, e non fate quel che dico?
\par 47 Chiunque viene a me ed ascolta le mie parole e le mette in pratica, io vi mostrerò a chi somiglia.
\par 48 Somiglia ad un uomo il quale, edificando una casa, ha scavato e scavato profondo, ed ha posto il fondamento sulla roccia; e venuta una piena, la fiumana ha investito quella casa e non ha potuto scrollarla perché era stata edificata bene.
\par 49 Ma chi ha udito e non ha messo in pratica, somiglia ad un uomo che ha edificato una casa sulla terra, senza fondamento; la fiumana l'ha investita, e subito è crollata; e la ruina di quella casa è stata grande.

\chapter{7}

\par 1 Dopo ch'egli ebbe finiti tutti i suoi ragionamenti al popolo che l'ascoltava, entrò in Capernaum.
\par 2 Or il servitore d'un certo centurione, che l'avea molto caro, era malato e stava per morire;
\par 3 e il centurione, avendo udito parlar di Gesù, gli mandò degli anziani de' Giudei per pregarlo che venisse a salvare il suo servitore.
\par 4 Ed essi, presentatisi a Gesù, lo pregavano istantemente, dicendo: Egli è degno che tu gli conceda questo;
\par 5 perché ama la nostra nazione, ed è lui che ci ha edificata la sinagoga.
\par 6 E Gesù s'incamminò con loro; e ormai non si trovava più molto lontano dalla casa, quando il centurione mandò degli amici a dirgli: Signore, non ti dare questo incomodo, perch'io non son degno che tu entri sotto il mio tetto;
\par 7 e perciò non mi son neppure reputato degno di venire da te; ma dillo con una parola, e sia guarito il mio servitore.
\par 8 Poiché anch'io son uomo sottoposto alla potestà altrui, ed ho sotto di me de' soldati; e dico ad uno: Va', ed egli va; e ad un altro: Vieni, ed egli viene; e al mio servitore: Fa' questo, ed egli lo fa.
\par 9 Udito questo, Gesù restò maravigliato di lui; e rivoltosi alla moltitudine che lo seguiva, disse: Io vi dico che neppure in Israele ho trovato una cotanta fede!
\par 10 E quando gl'inviati furon tornati a casa, trovarono il servitore guarito.
\par 11 E avvenne in seguito, ch'egli s'avviò ad una città chiamata Nain, e i suoi discepoli e una gran moltitudine andavano con lui.
\par 12 E come fu presso alla porta della città, ecco che si portava a seppellire un morto, figliuolo unico di sua madre; e questa era vedova; e una gran moltitudine della città era con lei.
\par 13 E il Signore, vedutala, ebbe pietà di lei e le disse: Non piangere!
\par 14 E accostatosi, toccò la bara; i portatori si fermarono, ed egli disse: Giovinetto, io tel dico, lèvati!
\par 15 E il morto si levò a sedere e cominciò a parlare. E Gesù lo diede a sua madre.
\par 16 Tutti furon presi da timore, e glorificavano Iddio dicendo: Un gran profeta è sorto fra noi; e: Dio ha visitato il suo popolo.
\par 17 E questo dire intorno a Gesù si sparse per tutta la Giudea e per tutto il paese circonvicino.
\par 18 E i discepoli di Giovanni gli riferirono tutte queste cose.
\par 19 Ed egli, chiamati a sé due dei suoi discepoli, li mandò al Signore a dirgli: Sei tu colui che ha da venire o ne aspetteremo noi un altro?
\par 20 E quelli, presentatisi a Gesù, gli dissero: Giovanni Battista ci ha mandati da te a dirti: Sei tu colui che ha da venire, o ne aspetteremo noi un altro?
\par 21 In quella stessa ora, Gesù guarì molti di malattie, di flagelli e di spiriti maligni, e a molti ciechi donò la vista.
\par 22 E, rispondendo, disse loro: Andate a riferire a Giovanni quel che avete veduto e udito: i ciechi ricuperano la vista, gli zoppi camminano, i lebbrosi sono mondati, i sordi odono, i morti risuscitano, l'Evangelo è annunziato ai poveri.
\par 23 E beato colui che non si sarà scandalizzato di me!
\par 24 Quando i messi di Giovanni se ne furono andati, Gesù prese a dire alle turbe intorno a Giovanni: che andaste a vedere nel deserto? Una canna dimenata dal vento?
\par 25 Ma che andaste a vedere? Un uomo avvolto in morbide vesti? Ecco, quelli che portano de' vestimenti magnifici e vivono in delizie, stanno nei palazzi dei re.
\par 26 Ma che andaste a vedere? Un profeta? Sì, vi dico, e uno più che profeta.
\par 27 Egli è colui del quale è scritto: Ecco, io mando il mio messaggero davanti al tuo cospetto che preparerà la tua via dinanzi a te.
\par 28 Io ve lo dico: Fra i nati di donna non ve n'è alcuno maggiore di Giovanni; però, il minimo nel regno di Dio è maggiore di lui.
\par 29 E tutto il popolo che l'ha udito, ed anche i pubblicani, hanno reso giustizia a Dio, facendosi battezzare del battesimo di Giovanni;
\par 30 ma i Farisei e i dottori della legge hanno reso vano per loro stessi il consiglio di Dio, non facendosi battezzare da lui.
\par 31 A chi dunque assomiglierò gli uomini di questa generazione? E a chi sono simili?
\par 32 Sono simili ai fanciulli che stanno a sedere in piazza, e gridano gli uni agli altri: Vi abbiam sonato il flauto e non avete ballato; abbiam cantato dei lamenti e non avete pianto.
\par 33 Difatti è venuto Giovanni Battista non mangiando pane né bevendo vino, e voi dite: Ha un demonio.
\par 34 È venuto il Figliuol dell'uomo mangiando e bevendo, e voi dite: Ecco un mangiatore ed un beone, un amico dei pubblicani e de' peccatori!
\par 35 Ma alla sapienza è stata resa giustizia da tutti i suoi figliuoli.
\par 36 Or uno de' Farisei lo pregò di mangiare da lui; ed egli, entrato in casa del Fariseo, si mise a tavola.
\par 37 Ed ecco, una donna che era in quella città, una peccatrice, saputo ch'egli era a tavola in casa del Fariseo, portò un alabastro d'olio odorifero;
\par 38 e stando a' piedi di lui, di dietro, piangendo cominciò a rigargli di lagrime i piedi, e li asciugava coi capelli del suo capo; e gli baciava e ribaciava i piedi e li ungeva con l'olio.
\par 39 Il Fariseo che l'avea invitato, veduto ciò, disse fra sé: Costui, se fosse profeta, saprebbe chi e quale sia la donna che lo tocca; perché è una peccatrice.
\par 40 E Gesù, rispondendo, gli disse: Simone, ho qualcosa da dirti. Ed egli:
\par 41 Maestro, di' pure. - Un creditore avea due debitori; l'uno gli dovea cinquecento denari e l'altro cinquanta.
\par 42 E non avendo essi di che pagare, condonò il debito ad ambedue. Chi di loro dunque l'amerà di più?
\par 43 Simone, rispondendo, disse: Stimo sia colui al quale ha condonato di più. E Gesù gli disse: Hai giudicato rettamente.
\par 44 E voltosi alla donna, disse a Simone: Vedi questa donna? Io sono entrato in casa tua, e tu non m'hai dato dell'acqua ai piedi; ma ella mi ha rigato i piedi di lagrime e li ha asciugati co' suoi capelli.
\par 45 Tu non m'hai dato alcun bacio; ma ella, da che sono entrato, non ha smesso di baciarmi i piedi.
\par 46 Tu non m'hai unto il capo d'olio; ma ella m'ha unto i piedi di profumo.
\par 47 Per la qual cosa, io ti dico: Le sono rimessi i suoi molti peccati, perché ha molto amato; ma colui a cui poco è rimesso, poco ama.
\par 48 Poi disse alla donna: I tuoi peccati ti sono rimessi.
\par 49 E quelli che erano a tavola con lui, cominciarono a dire dentro di sé: Chi è costui che rimette anche i peccati?
\par 50 Ma egli disse alla donna: La tua fede t'ha salvata; vattene in pace.

\chapter{8}

\par 1 Ed avvenne in appresso che egli andava attorno di città in città e di villaggio in villaggio, predicando ed annunziando la buona novella del regno di Dio;
\par 2 e con lui erano i dodici e certe donne che erano state guarite da spiriti maligni e da infermità: Maria, detta Maddalena, dalla quale erano usciti sette demonî,
\par 3 e Giovanna, moglie di Cuza, amministratore di Erode, e Susanna ed altre molte che assistevano Gesù ed i suoi coi loro beni.
\par 4 Or come si raunava gran folla e la gente d'ogni città accorreva a lui, egli disse in parabola:
\par 5 Il seminatore uscì a seminar la sua semenza; e mentre seminava, una parte del seme cadde lungo la strada, e fu calpestato e gli uccelli del cielo lo mangiarono.
\par 6 Ed un'altra cadde sulla roccia; e come fu nato seccò perché non avea umore.
\par 7 Ed un'altra cadde in mezzo alle spine; e le spine, nate insieme col seme, lo soffocarono.
\par 8 Ed un'altra parte cadde nella buona terra; e nato che fu, fruttò il cento per uno. Dicendo queste cose, esclamava: Chi ha orecchi da udire, oda.
\par 9 E i suoi discepoli gli domandarono che volesse dir questa parabola.
\par 10 Ed egli disse: a voi è dato di conoscere i misteri del regno di Dio; ma agli altri se ne parla in parabole, affinché vedendo non veggano, e udendo non intendano.
\par 11 Or questo è il senso della parabola: Il seme è la parola di Dio.
\par 12 Quelli lungo la strada son coloro che hanno udito; ma poi viene il diavolo e porta via la parola dal cuor loro, affinché non credano e non siano salvati.
\par 13 E quelli sulla roccia son coloro i quali, quando hanno udito la Parola, la ricevono con allegrezza; ma costoro non hanno radice, credono per un tempo, e quando viene la prova, si traggono indietro.
\par 14 E quel ch'è caduto fra le spine, son coloro che hanno udito, ma se ne vanno e restan soffocati dalle cure e dalle ricchezze e dai piaceri della vita, e non arrivano a maturità.
\par 15 E quel ch'è in buona terra, son coloro i quali, dopo aver udita la Parola, la ritengono in cuore onesto e buono, e portan frutto con perseveranza.
\par 16 Or niuno, accesa una lampada, la copre con un vaso, o la mette sotto il letto; anzi la mette sul candeliere, acciocché chi entra vegga la luce.
\par 17 Poiché non v'è nulla di nascosto che non abbia a diventar manifesto, né di segreto che non abbia a sapersi ed a farsi palese.
\par 18 Badate dunque come ascoltate: perché a chi ha sarà dato; ma a chi non ha, anche quel che pensa d'avere gli sarà tolto.
\par 19 Or sua madre e i suoi fratelli vennero a lui; e non poteano avvicinarglisi a motivo della folla.
\par 20 E gli fu riferito: Tua madre e i tuoi fratelli son là fuori, che ti voglion vedere.
\par 21 Ma egli, rispondendo, disse loro: Mia madre e miei fratelli son quelli che ascoltano la parola di Dio e la mettono in pratica.
\par 22 Or avvenne, in un di quei giorni, ch'egli entrò in una barca co' suoi discepoli, e disse loro: Passiamo all'altra riva del lago. E presero il largo.
\par 23 E mentre navigavano, egli si addormentò; e calò sul lago un turbine di vento, talché la barca s'empiva d'acqua, ed essi pericolavano.
\par 24 E accostatisi, lo svegliarono, dicendo: Maestro, Maestro, noi periamo! Ma egli, destatosi, sgridò il vento e i flutti che s'acquetarono, e si fe' bonaccia.
\par 25 Poi disse loro: Dov'è la fede vostra? Ma essi, impauriti e maravigliati, diceano l'uno all'altro: Chi è mai costui che comanda anche ai venti ed all'acqua e gli ubbidiscono?
\par 26 E navigarono verso il paese dei Geraseni che è dirimpetto alla Galilea.
\par 27 E quando egli fu smontato a terra, gli si fece incontro un uomo della città, il quale era posseduto da demonî, e da lungo tempo non indossava vestito, e non abitava casa ma stava ne' sepolcri.
\par 28 Or quando ebbe veduto Gesù, dato un gran grido, gli si prostrò dinanzi, e disse con gran voce: Che v'è fra me e te, o Gesù, Figliuolo dell'Iddio altissimo? Ti prego, non mi tormentare.
\par 29 Poiché Gesù comandava allo spirito immondo d'uscir da quell'uomo; molte volte infatti esso se n'era impadronito; e benché lo si fosse legato con catene e custodito in ceppi, avea spezzato i legami, ed era portato via dal demonio ne' deserti.
\par 30 E Gesù gli domandò: Qual è il tuo nome? Ed egli rispose: Legione; perché molti demonî erano entrati in lui.
\par 31 Ed essi lo pregavano che non comandasse loro d'andar nell'abisso.
\par 32 Or c'era quivi un branco numeroso di porci che pascolava pel monte; e que' demonî lo pregarono di permetter loro d'entrare in quelli. Ed egli lo permise loro.
\par 33 E i demonî, usciti da quell'uomo, entrarono ne' porci; e quel branco si avventò a precipizio giù nel lago ed affogò.
\par 34 E quando quelli che li pasturavano videro ciò ch'era avvenuto, se ne fuggirono e portaron la notizia in città e per la campagna.
\par 35 E la gente uscì fuori a veder l'accaduto; e venuta a Gesù, trovò l'uomo, dal quale erano usciti i demonî, che sedeva a' piedi di Gesù, vestito ed in buon senno; e s'impaurirono.
\par 36 E quelli che aveano veduto, raccontarono loro come l'indemoniato era stato liberato.
\par 37 E l'intera popolazione della circostante regione de' Geraseni pregò Gesù che se n'andasse da loro; perch'eran presi da grande spavento. Ed egli, montato nella barca, se ne tornò indietro.
\par 38 E l'uomo dal quale erano usciti i demonî, lo pregava di poter stare con lui, ma Gesù lo licenziò, dicendo:
\par 39 Torna a casa tua, e racconta le grandi cose che Iddio ha fatte per te. Ed egli se ne andò per tutta la città, proclamando quanto grandi cose Gesù avea fatte per lui.
\par 40 Al suo ritorno, Gesù fu accolto dalla folla, perché tutti lo stavano aspettando.
\par 41 Ed ecco venire un uomo, chiamato Iairo, che era capo della sinagoga; e gittatosi ai piedi di Gesù, lo pregava d'entrare in casa sua,
\par 42 perché avea una figlia unica di circa dodici anni, e quella stava per morire. Or mentre Gesù v'andava, la moltitudine l'affollava.
\par 43 E una donna che avea un flusso di sangue da dodici anni ed avea spesa ne' medici tutta la sua sostanza senza poter esser guarita da alcuno,
\par 44 accostatasi per di dietro, gli toccò il lembo della veste; e in quell'istante il suo flusso ristagnò.
\par 45 E Gesù domandò: Chi m'ha toccato? E siccome tutti negavano, Pietro e quelli ch'eran con lui, risposero: Maestro, le turbe ti stringono e t'affollano.
\par 46 Ma Gesù replicò: Qualcuno m'ha toccato, perché ho sentito che una virtù è uscita da me.
\par 47 E la donna, vedendo che non era rimasta inosservata, venne tutta tremante, e gittatasi a' suoi piedi, dichiarò, in presenza di tutto il popolo, per qual motivo l'avea toccato e com'era stata guarita in un istante.
\par 48 Ma egli le disse: Figliuola, la tua fede t'ha salvata; vattene in pace.
\par 49 Mentr'egli parlava ancora, venne uno da casa del capo della sinagoga, a dirgli: La tua figliuola è morta; non incomodar più oltre il Maestro.
\par 50 Ma Gesù, udito ciò, rispose a Iairo: Non temere; solo abbi fede, ed ella sarà salva.
\par 51 Ed arrivato alla casa, non permise ad alcuno d'entrarvi con lui, salvo che a Pietro, a Giovanni, a Giacomo e al padre e alla madre della fanciulla.
\par 52 Or tutti piangevano e facean cordoglio per lei. Ma egli disse: Non piangete; ella non è morta, ma dorme.
\par 53 E si ridevano di lui, sapendo ch'era morta.
\par 54 Ma egli, presala per la mano, disse ad alta voce: Fanciulla, lèvati!
\par 55 E lo spirito di lei tornò; ella s'alzò subito, ed egli comandò che le si desse da mangiare.
\par 56 E i genitori di lei sbigottirono; ma egli ordinò loro di non dire ad alcuno quel che era avvenuto.

\chapter{9}

\par 1 Ora Gesù, chiamati assieme i dodici, diede loro potestà ed autorità su tutti i demonî e di guarir le malattie.
\par 2 E li mandò a predicare il regno di Dio e a guarire gl'infermi.
\par 3 E disse loro: Non prendete nulla per viaggio: né bastone, né sacca, né pane, né danaro, e non abbiate tunica di ricambio.
\par 4 E in qualunque casa sarete entrati, in quella dimorate e da quella ripartite.
\par 5 E quant'è a quelli che non vi riceveranno, uscendo dalla loro città, scotete la polvere dai vostri piedi, in testimonianza contro a loro.
\par 6 Ed essi, partitisi, andavano attorno di villaggio in villaggio, evangelizzando e facendo guarigioni per ogni dove.
\par 7 Ora, Erode il tetrarca udì parlare di tutti que' fatti; e n'era perplesso, perché taluni dicevano: Giovanni è risuscitato dai morti;
\par 8 altri dicevano: È apparso Elia; ed altri: È risuscitato uno degli antichi profeti.
\par 9 Ma Erode disse: Giovanni l'ho fatto decapitare; chi è dunque costui del quale sento dir tali cose? E cercava di vederlo.
\par 10 E gli apostoli, essendo ritornati, raccontarono a Gesù tutte le cose che aveano fatte; ed egli, presili seco, si ritirò in disparte verso una città chiamata Betsaida.
\par 11 Ma le turbe, avendolo saputo, lo seguirono; ed egli, accoltele, parlava loro del regno di Dio, e guariva quelli che avean bisogno di guarigione.
\par 12 Or il giorno cominciava a declinare; e i dodici, accostatisi, gli dissero: Licenzia la moltitudine, affinché se ne vada per i villaggi e per le campagne d'intorno per albergarvi e per trovarvi da mangiare, perché qui siamo in un luogo deserto.
\par 13 Ma egli disse loro: Date lor voi da mangiare. Ed essi risposero: Noi non abbiamo altro che cinque pani e due pesci; se pur non andiamo noi a comprar dei viveri per tutto questo popolo.
\par 14 Poiché v'eran cinquemila uomini. Ed egli disse ai suoi discepoli: Fateli accomodare a cerchi d'una cinquantina.
\par 15 E così li fecero accomodar tutti.
\par 16 Poi Gesù prese i cinque pani e i due pesci; e levati gli occhi al cielo, li benedisse, li spezzò e li dava ai suoi discepoli per metterli dinanzi alla gente.
\par 17 E tutti mangiarono e furon sazî; e de' pezzi loro avanzati si portaron via dodici ceste.
\par 18 Or avvenne che mentr'egli stava pregando in disparte, i discepoli erano con lui; ed egli domandò loro: Chi dicono le turbe ch'io sia?
\par 19 E quelli risposero: Gli uni dicono Giovanni Battista; altri Elia; ed altri, uno dei profeti antichi risuscitato.
\par 20 Ed egli disse loro: E voi, chi dite ch'io sia? E Pietro, rispondendo, disse: Il Cristo di Dio.
\par 21 Ed egli vietò loro severamente di dirlo ad alcuno, e aggiunse:
\par 22 Bisogna che il Figliuol dell'uomo soffra molte cose, e sia reietto dagli anziani e dai capi sacerdoti e dagli scribi, e sia ucciso, e risusciti il terzo giorno.
\par 23 Diceva poi a tutti: Se uno vuol venire dietro a me, rinunzi a se stesso, prenda ogni giorno la sua croce e mi seguiti.
\par 24 Perché chi vorrà salvare la sua vita, la perderà; ma chi avrà perduto la propria vita per me, esso la salverà.
\par 25 Infatti, che giova egli all'uomo l'aver guadagnato tutto il mondo, se poi ha perduto o rovinato se stesso?
\par 26 Perché se uno ha vergogna di me e delle mie parole, il Figliuol dell'uomo avrà vergogna di lui, quando verrà nella gloria sua e del Padre e de' santi angeli.
\par 27 Or io vi dico in verità che alcuni di coloro che son qui presenti non gusteranno la morte, finché non abbian veduto il regno di Dio.
\par 28 Or avvenne che circa otto giorni dopo questi ragionamenti, Gesù prese seco Pietro, Giovanni e Giacomo, e salì sul monte per pregare.
\par 29 E mentre pregava, l'aspetto del suo volto fu mutato, e la sua veste divenne candida sfolgorante.
\par 30 Ed ecco, due uomini conversavano con lui; ed erano Mosè ed Elia,
\par 31 i quali, appariti in gloria, parlavano della dipartenza ch'egli stava per compiere in Gerusalemme.
\par 32 Or Pietro e quelli ch'eran con lui, erano aggravati dal sonno; e quando si furono svegliati, videro la sua gloria e i due uomini che stavan con lui.
\par 33 E come questi si partivano da lui, Pietro disse a Gesù: Maestro, egli è bene che stiamo qui; facciamo tre tende: una per te, una per Mosè, ed una per Elia; non sapendo quel che si dicesse.
\par 34 E mentre diceva così, venne una nuvola che li coperse della sua ombra; e i discepoli temettero quando quelli entrarono nella nuvola.
\par 35 Ed una voce venne dalla nuvola, dicendo: Questo è il mio figliuolo, l'eletto mio; ascoltatelo.
\par 36 E mentre si faceva quella voce, Gesù si trovò solo. Ed essi tacquero, e non riferirono in quei giorni ad alcuno nulla di quel che aveano veduto.
\par 37 Or avvenne il giorno seguente che essendo essi scesi dal monte, una gran moltitudine venne incontro a Gesù.
\par 38 Ed ecco, un uomo dalla folla esclamò: Maestro, te ne prego, volgi lo sguardo al mio figliuolo; è l'unico ch'io abbia;
\par 39 ed ecco uno spirito lo prende, e subito egli grida, e lo spirito lo getta in convulsione facendolo schiumare, e a fatica si diparte da lui, fiaccandolo tutto.
\par 40 Ed ho pregato i tuoi discepoli di cacciarlo, ma non hanno potuto.
\par 41 E Gesù, rispondendo, disse: O generazione incredula e perversa, fino a quando sarò io con voi e vi sopporterò?
\par 42 Mena qua il tuo figliuolo. E mentre il fanciullo si avvicinava, il demonio lo gettò per terra e lo torse in convulsione; ma Gesù sgridò lo spirito immondo, guarì il fanciullo, e lo rese a suo padre.
\par 43 E tutti sbigottivano della grandezza di Dio.
\par 44 Ora, mentre tutti si maravigliavano di tutte le cose che Gesù faceva, egli disse ai suoi discepoli: Voi, tenete bene a mente queste parole: Il Figliuol dell'uomo sta per esser dato nelle mani degli uomini.
\par 45 Ma essi non capivano quel detto ch'era per loro coperto d'un velo, per modo che non lo intendevano, e temevano d'interrogarlo circa quel detto.
\par 46 Poi sorse fra loro una disputa sul chi di loro fosse il maggiore.
\par 47 Ma Gesù, conosciuto il pensiero del loro cuore, prese un piccolo fanciullo, se lo pose accanto, e disse loro:
\par 48 Chi riceve questo piccolo fanciullo nel nome mio, riceve me; e chi riceve me, riceve Colui che m'ha mandato. Poiché chi è il minimo fra tutti voi, quello è grande.
\par 49 Or Giovanni prese a dirgli: Maestro, noi abbiam veduto un tale che cacciava i demonî nel tuo nome, e glielo abbiamo vietato perché non ti segue con noi.
\par 50 Ma Gesù gli disse: Non glielo vietate, perché chi non è contro voi è per voi.
\par 51 Poi, come s'avvicinava il tempo della sua assunzione, Gesù si mise risolutamente in via per andare a Gerusalemme.
\par 52 E mandò davanti a sé de' messi, i quali, partitisi, entrarono in un villaggio de' Samaritani per preparargli alloggio.
\par 53 Ma quelli non lo ricevettero perché era diretto verso Gerusalemme.
\par 54 Veduto ciò, i suoi discepoli Giacomo e Giovanni dissero: Signore, vuoi tu che diciamo che scenda fuoco dal cielo e li consumi?
\par 55 Ma egli, rivoltosi, li sgridò.
\par 56 E se ne andarono in un altro villaggio.
\par 57 Or avvenne che mentre camminavano per la via, qualcuno gli disse: Io ti seguiterò dovunque tu andrai.
\par 58 E Gesù gli rispose: Le volpi hanno delle tane e gli uccelli del cielo dei nidi, ma il Figliuol dell'uomo non ha dove posare il capo.
\par 59 E ad un altro disse: Seguitami. Ed egli rispose: Permettimi prima d'andare a seppellir mio padre.
\par 60 Ma Gesù gli disse: Lascia i morti seppellire i loro morti; ma tu va' ad annunziare il regno di Dio.
\par 61 E un altro ancora gli disse: Ti seguiterò, Signore, ma permettimi prima d'accomiatarmi da que' di casa mia.
\par 62 Ma Gesù gli disse: Nessuno che abbia messo la mano all'aratro e poi riguardi indietro, è adatto al regno di Dio.

\chapter{10}

\par 1 Or dopo queste cose, il Signore designò altri settanta discepoli, e li mandò a due a due dinanzi a sé, in ogni città e luogo dove egli stesso era per andare.
\par 2 E diceva loro: Ben è la mèsse grande, ma gli operai son pochi; pregate dunque il Signor della mèsse che spinga degli operai nella sua mèsse.
\par 3 Andate; ecco, io vi mando come agnelli in mezzo ai lupi.
\par 4 Non portate né borsa, né sacca, né calzari, e non salutate alcuno per via.
\par 5 In qualunque casa sarete entrati, dite prima: Pace a questa casa!
\par 6 E se v'è quivi alcun figliuolo di pace, la vostra pace riposerà su lui; se no, ella tornerà a voi.
\par 7 Or dimorate in quella stessa casa, mangiando e bevendo di quello che hanno, perché l'operaio è degno della sua mercede. Non passate di casa in casa.
\par 8 E in qualunque città sarete entrati, se vi ricevono, mangiate di ciò che vi sarà messo dinanzi,
\par 9 guarite gl'infermi che saranno in essa, e dite loro: Il regno di Dio s'è avvicinato a voi.
\par 10 Ma in qualunque città sarete entrati, se non vi ricevono, uscite sulle piazze e dite:
\par 11 Perfino la polvere che dalla vostra città s'è attaccata a' nostri piedi, noi la scotiamo contro a voi; sappiate tuttavia questo, che il regno di Dio s'è avvicinato a voi.
\par 12 Io vi dico che in quel giorno la sorte di Sodoma sarà più tollerabile della sorte di quella città.
\par 13 Guai a te, Corazin! Guai a te, Betsaida; perché se in Tiro e in Sidone fossero state fatte le opere potenti compiute fra voi, già anticamente si sarebbero ravvedute, prendendo il cilicio, e sedendo nella cenere.
\par 14 E però, nel giorno del giudicio, la sorte di Tiro e di Sidone sarà più tollerabile della vostra.
\par 15 E tu, o Capernaum, sarai tu forse innalzata fino al cielo? No, tu sarai abbassata fino nell'Ades!
\par 16 Chi ascolta voi ascolta me; chi sprezza voi sprezza me, e chi sprezza me sprezza Colui che mi ha mandato.
\par 17 Or i settanta tornarono con allegrezza, dicendo: Signore, anche i demonî ci sono sottoposti nel tuo nome.
\par 18 Ed egli disse loro: Io mirava Satana cader dal cielo a guisa di folgore.
\par 19 Ecco, io v'ho dato la potestà di calcar serpenti e scorpioni, e tutta la potenza del nemico; e nulla potrà farvi del male.
\par 20 Pure, non vi rallegrate perché gli spiriti vi son sottoposti, ma rallegratevi perché i vostri nomi sono scritti ne' cieli.
\par 21 In quella stessa ora, Gesù giubilò per lo Spirito Santo, e disse: Io ti rendo lode, o Padre, Signore del cielo e della terra, perché hai nascoste queste cose ai savi e agl'intelligenti, e le hai rivelate ai piccoli fanciulli! Sì, o Padre, perché così ti è piaciuto.
\par 22 Ogni cosa m'è stata data in mano dal Padre mio; e nessuno conosce chi è il Figliuolo, se non il Padre; né chi è il Padre, se non il Figliuolo e colui al quale il Figliuolo voglia rivelarlo.
\par 23 E rivoltosi a' suoi discepoli, disse loro in disparte: Beati gli occhi che veggono le cose che voi vedete!
\par 24 Poiché vi dico che molti profeti e re han bramato di veder le cose che voi vedete, e non le hanno vedute; e di udir le cose che voi udite, e non le hanno udite.
\par 25 Ed ecco, un certo dottor della legge si levò per metterlo alla prova, e gli disse: Maestro, che dovrò fare per eredar la vita eterna?
\par 26 Ed egli gli disse: Nella legge che sta scritto? Come leggi?
\par 27 E colui, rispondendo, disse: Ama il Signore Iddio tuo con tutto il tuo cuore, e con tutta l'anima tua, e con tutta la forza tua, e con tutta la mente tua, e il tuo prossimo come te stesso.
\par 28 E Gesù gli disse: Tu hai risposto rettamente; fa' questo, e vivrai.
\par 29 Ma colui, volendo giustificarsi, disse a Gesù: E chi è il mio prossimo?
\par 30 Gesù, replicando, disse: Un uomo scendeva da Gerusalemme a Gerico, e s'imbatté in ladroni i quali, spogliatolo e feritolo, se ne andarono, lasciandolo mezzo morto.
\par 31 Or, per caso, un sacerdote scendeva per quella stessa via; e veduto colui, passò oltre dal lato opposto.
\par 32 Così pure un levita, giunto a quel luogo e vedutolo, passò oltre dal lato opposto.
\par 33 Ma un Samaritano che era in viaggio giunse presso a lui; e vedutolo, n'ebbe pietà;
\par 34 e accostatosi, fasciò le sue piaghe, versandovi sopra dell'olio e del vino; poi lo mise sulla propria cavalcatura, lo menò ad un albergo e si prese cura di lui.
\par 35 E il giorno dopo, tratti fuori due denari, li diede all'oste e gli disse: prenditi cura di lui; e tutto ciò che spenderai di più, quando tornerò in su, te lo renderò.
\par 36 Quale di questi tre ti pare essere stato il prossimo di colui che s'imbatté ne' ladroni?
\par 37 E quello rispose: Colui che gli usò misericordia. E Gesù gli disse: Va', e fa' tu il simigliante.
\par 38 Or mentre essi erano in cammino, egli entrò in un villaggio; e una certa donna, per nome Marta, lo ricevette in casa sua.
\par 39 Ell'avea una sorella chiamata Maria la quale, postasi a sedere a' piedi di Gesù, ascoltava la sua parola.
\par 40 Ma Marta era affaccendata intorno a molti servigi; e venne e disse: Signore, non t'importa che mia sorella m'abbia lasciata sola a servire? Dille dunque che m'aiuti.
\par 41 Ma il Signore, rispondendo, le disse: Marta, Marta, tu ti affanni e t'inquieti di molte cose, ma di una cosa sola fa bisogno.
\par 42 E Maria ha scelto la buona parte che non le sarà tolta.

\chapter{11}

\par 1 Ed avvenne che essendo egli in orazione in un certo luogo, com'ebbe finito, uno de' suoi discepoli gli disse: Signore, insegnaci a pregare come anche Giovanni ha insegnato ai suoi discepoli.
\par 2 Ed egli disse loro: Quando pregate, dite: Padre, sia santificato il tuo nome; venga il tuo regno;
\par 3 dacci di giorno in giorno il nostro pane cotidiano;
\par 4 e perdonaci i nostri peccati, poiché anche noi perdoniamo ad ogni nostro debitore; e non ci esporre alla tentazione.
\par 5 Poi disse loro: Se uno d'infra voi ha un amico e va da lui a mezzanotte e gli dice: Amico, prestami tre pani,
\par 6 perché m'è giunto di viaggio in casa un amico, e non ho nulla da mettergli dinanzi;
\par 7 e se colui dal di dentro gli risponde: Non mi dar molestia; già è serrata la porta, e i miei fanciulli son meco a letto, io non posso alzarmi per darteli,
\par 8 - io vi dico che quand'anche non s'alzasse a darglieli perché gli è amico, pure, per la importunità sua, si leverà e gliene darà quanti ne ha di bisogno.
\par 9 Io altresì vi dico: Chiedete, e vi sarà dato; cercate e troverete; picchiate, e vi sarà aperto.
\par 10 Poiché chiunque chiede riceve, chi cerca trova, e sarà aperto a chi picchia.
\par 11 E chi è quel padre tra voi che, se il figliuolo gli chiede un pane, gli dia una pietra? O se gli chiede un pesce, gli dia invece una serpe?
\par 12 Oppure anche se gli chiede un uovo, gli dia uno scorpione?
\par 13 Se voi dunque, che siete malvagi, sapete dare buoni doni ai vostri figliuoli, quanto più il vostro Padre celeste donerà lo Spirito Santo a coloro che glielo domandano!
\par 14 Or egli stava cacciando un demonio che era muto; ed avvenne che quando il demonio fu uscito, il muto parlò; e le turbe si maravigliarono.
\par 15 Ma alcuni di loro dissero: È per l'aiuto di Beelzebub, principe dei demonî, ch'egli caccia i demonî.
\par 16 Ed altri, per metterlo alla prova, chiedevano da lui un segno dal cielo.
\par 17 Ma egli, conoscendo i loro pensieri, disse loro: Ogni regno diviso in parti contrarie è ridotto in deserto, e una casa divisa contro se stessa, rovina.
\par 18 Se dunque anche Satana è diviso contro se stesso, come potrà reggere il suo regno? Poiché voi dite che è per l'aiuto di Beelzebub che io caccio i demonî.
\par 19 E se io caccio i demonî per l'aiuto di Beelzebub, i vostri figliuoli per l'aiuto di chi li caccian essi? Perciò, essi stessi saranno i vostri giudici.
\par 20 Ma se è per il dito di Dio che io caccio i demonî, è dunque pervenuto fino a voi il regno di Dio.
\par 21 Quando l'uomo forte, ben armato, guarda l'ingresso della sua dimora, quel ch'e' possiede è al sicuro;
\par 22 ma quando uno più forte di lui sopraggiunge e lo vince, gli toglie tutta l'armatura nella quale si confidava, e ne spartisce le spoglie.
\par 23 Chi non è con me, è contro di me; e chi non raccoglie con me, disperde.
\par 24 Quando lo spirito immondo è uscito da un uomo, va attorno per luoghi aridi, cercando riposo; e non trovandone, dice: Ritornerò nella mia casa donde sono uscito;
\par 25 e giuntovi, la trova spazzata e adorna.
\par 26 Allora va e prende seco altri sette spiriti peggiori di lui, ed entrano ad abitarla; e l'ultima condizione di quell'uomo divien peggiore della prima.
\par 27 Or avvenne che, mentre egli diceva queste cose, una donna di fra la moltitudine alzò la voce e gli disse: Beato il seno che ti portò e le mammelle che tu poppasti! Ma egli disse:
\par 28 Beati piuttosto quelli che odono la parola di Dio e l'osservano!
\par 29 E affollandosi intorno a lui le turbe, egli prese a dire: Questa generazione è una generazione malvagia; ella chiede un segno; e segno alcuno non le sarà dato, salvo il segno di Giona.
\par 30 Poiché come Giona fu un segno per i Niniviti, così anche il Figliuol dell'uomo sarà per questa generazione.
\par 31 La regina del Mezzodì risusciterà nel giudizio con gli uomini di questa generazione e li condannerà; perché ella venne dalle estremità della terra per udir la sapienza di Salomone; ed ecco qui v'è più che Salomone.
\par 32 I Niniviti risusciteranno nel giudizio con questa generazione e la condanneranno; perché essi si ravvidero alla predicazione di Giona; ed ecco qui v'è più che Giona.
\par 33 Nessuno, quand'ha acceso una lampada, la mette in un luogo nascosto o sotto il moggio; anzi la mette sul candeliere, affinché coloro che entrano veggano la luce.
\par 34 La lampada del tuo corpo è l'occhio; se l'occhio tuo è sano, anche tutto il tuo corpo è illuminato; ma se è viziato, anche il tuo corpo è nelle tenebre.
\par 35 Guarda dunque che la luce che è in te non sia tenebre.
\par 36 Se dunque tutto il tuo corpo è illuminato, senz'aver parte alcuna tenebrosa, sarà tutto illuminato come quando la lampada t'illumina col suo splendore.
\par 37 Or mentr'egli parlava, un Fariseo lo invitò a desinare da lui. Ed egli, entrato, si mise a tavola.
\par 38 E il Fariseo, veduto questo, si maravigliò che non si fosse prima lavato, avanti il desinare.
\par 39 E il Signore gli disse: Voi altri Farisei nettate il di fuori della coppa e del piatto, ma l'interno vostro è pieno di rapina e di malvagità.
\par 40 Stolti, Colui che ha fatto il di fuori, non ha anche fatto il di dentro?
\par 41 Date piuttosto in elemosina quel ch'è dentro al piatto; ed ecco, ogni cosa sarà netta per voi.
\par 42 Ma guai a voi, Farisei, poiché pagate la decima della menta, della ruta e d'ogni erba, e trascurate la giustizia e l'amor di Dio! Queste son le cose che bisognava fare, senza tralasciar le altre.
\par 43 Guai a voi, Farisei, perché amate i primi seggi nelle sinagoghe, e i saluti nelle piazze.
\par 44 Guai a voi, perché siete come quei sepolcri che non si vedono, e chi vi cammina sopra non ne sa niente.
\par 45 Allora uno dei dottori della legge, rispondendo, gli disse: Maestro, parlando così, fai ingiuria anche a noi.
\par 46 Ed egli disse: Guai anche a voi, dottori della legge, perché caricate le genti di pesi difficili a portare e voi non toccate quei pesi neppur con un dito!
\par 47 Guai a voi, perché edificate i sepolcri de' profeti, e i vostri padri li uccisero.
\par 48 Voi dunque testimoniate delle opere de' vostri padri e le approvate: perché essi li uccisero, e voi edificate loro de' sepolcri.
\par 49 E per questo la sapienza di Dio ha detto: Io manderò loro dei profeti e degli apostoli; e ne uccideranno alcuni e ne perseguiteranno altri,
\par 50 affinché il sangue di tutti i profeti sparso dalla fondazione del mondo sia ridomandato a questa generazione;
\par 51 dal sangue di Abele fino al sangue di Zaccaria che fu ucciso fra l'altare ed il tempio; sì, vi dico, sarà ridomandato a questa generazione.
\par 52 Guai a voi, dottori della legge, poiché avete tolta la chiave della scienza! Voi stessi non siete entrati, ed avete impedito quelli che entravano.
\par 53 E quando fu uscito di là, gli scribi e i Farisei cominciarono a incalzarlo fieramente ed a trargli di bocca risposte a molte cose; tendendogli de' lacci,
\par 54 per coglier qualche parola che gli uscisse di bocca.

\chapter{12}

\par 1 Intanto, essendosi la moltitudine radunata a migliaia, così da calpestarsi gli uni gli altri, Gesù cominciò prima di tutto a dire ai suoi discepoli: Guardatevi dal lievito de' Farisei, che è ipocrisia.
\par 2 Ma non v'è niente di coperto che non abbia ad essere scoperto, né di occulto che non abbia ad esser conosciuto.
\par 3 Perciò tutto quel che avete detto nelle tenebre, sarà udito nella luce; e quel che avete detto all'orecchio nelle stanze interne, sarà proclamato sui tetti.
\par 4 Ma a voi che siete miei amici, io dico: Non temete coloro che uccidono il corpo, e che dopo ciò, non possono far nulla di più;
\par 5 ma io vi mostrerò chi dovete temere: Temete colui che, dopo aver ucciso, ha potestà di gettar nella geenna. Sì, vi dico, temete Lui.
\par 6 Cinque passeri non si vendon per due soldi? Eppure non uno d'essi è dimenticato dinanzi a Dio;
\par 7 anzi, perfino i capelli del vostro capo son tutti contati. Non temete dunque; voi siete da più di molti passeri.
\par 8 Or io vi dico: Chiunque mi avrà riconosciuto davanti agli uomini, anche il Figliuol dell'uomo riconoscerà lui davanti agli angeli di Dio;
\par 9 ma chi mi avrà rinnegato davanti agli uomini, sarà rinnegato davanti agli angeli di Dio.
\par 10 E a chiunque avrà parlato contro il Figliuol dell'uomo, sarà perdonato; ma a chi avrà bestemmiato contro lo Spirito Santo, non sarà perdonato.
\par 11 Quando poi vi condurranno davanti alle sinagoghe e ai magistrati e alle autorità, non state in ansietà del come o del che avrete a rispondere a vostra difesa, o di quel che avrete a dire;
\par 12 perché lo Spirito Santo v'insegnerà in quell'ora stessa quel che dovrete dire.
\par 13 Or uno della folla gli disse: Maestro, di' a mio fratello che divida con me l'eredità.
\par 14 Ma Gesù gli rispose: O uomo, chi mi ha costituito su voi giudice o spartitore? Poi disse loro:
\par 15 Badate e guardatevi da ogni avarizia; perché non è dall'abbondanza de' beni che uno possiede, ch'egli ha la sua vita.
\par 16 E disse loro questa parabola: La campagna d'un certo uomo ricco fruttò copiosamente;
\par 17 ed egli ragionava così fra se medesimo: Che farò, poiché non ho dove riporre i miei raccolti? E disse:
\par 18 Questo farò: demolirò i miei granai e ne fabbricherò dei più vasti, e vi raccoglierò tutto il mio grano e i miei beni,
\par 19 e dirò all'anima mia: Anima, tu hai molti beni riposti per molti anni; riposati, mangia, bevi, godi.
\par 20 Ma Dio gli disse: Stolto, questa notte stessa l'anima tua ti sarà ridomandata; e quel che hai preparato, di chi sarà?
\par 21 Così è di chi tesoreggia per sé, e non è ricco in vista di Dio.
\par 22 Poi disse ai suoi discepoli: Perciò vi dico: Non siate con ansietà solleciti per la vita vostra di quel che mangerete; né per il corpo di che vi vestirete;
\par 23 poiché la vita è più del nutrimento, e il corpo è più del vestito.
\par 24 Considerate i corvi: non seminano, non mietono; non hanno dispensa né granaio, eppure Dio li nutrisce. Di quanto non siete voi da più degli uccelli?
\par 25 E chi di voi può con la sua sollecitudine aggiungere alla sua statura pure un cubito?
\par 26 Se dunque non potete far nemmeno ciò ch'è minimo, perché siete in ansiosa sollecitudine del rimanente?
\par 27 Considerate i gigli, come crescono; non faticano e non filano; eppure io vi dico che Salomone stesso, con tutta la sua gloria, non fu vestito come uno di loro.
\par 28 Or se Dio riveste così l'erba che oggi è nel campo e domani è gettata nel forno, quanto più vestirà voi, o gente di poca fede?
\par 29 Anche voi non cercate che mangerete e che berrete, e non ne state in sospeso;
\par 30 poiché tutte queste cose son le genti del mondo che le ricercano; ma il Padre vostro sa che ne avete bisogno.
\par 31 Cercate piuttosto il suo regno, e queste cose vi saranno sopraggiunte.
\par 32 Non temere, o piccol gregge; poiché al Padre vostro è piaciuto di darvi il regno.
\par 33 Vendete i vostri beni, e fatene elemosina; fatevi delle borse che non invecchiano, un tesoro che non venga meno ne' cieli, ove ladro non s'accosta e tignuola non guasta.
\par 34 Perché dov'è il vostro tesoro, quivi sarà anche il vostro cuore.
\par 35 I vostri fianchi siano cinti, e le vostre lampade accese;
\par 36 e voi siate simili a quelli che aspettano il loro padrone quando tornerà dalle nozze, per aprirgli appena giungerà e picchierà.
\par 37 Beati que' servitori che il padrone, arrivando, troverà vigilanti! In verità io vi dico che egli si cingerà, li farà mettere a tavola e passerà a servirli.
\par 38 E se giungerà alla seconda o alla terza vigilia e li troverà così, beati loro!
\par 39 Or sappiate questo, che se il padron di casa sapesse a che ora verrà il ladro, veglierebbe e non si lascerebbe sconficcar la casa.
\par 40 Anche voi siate pronti, perché nell'ora che non pensate, il Figliuol dell'uomo verrà.
\par 41 E Pietro disse: Signore, questa parabola la dici tu per noi, o anche per tutti?
\par 42 E il Signore rispose: E qual è mai l'economo fedele e avveduto che il padrone costituirà sui suoi domestici per dar loro a suo tempo la loro misura di viveri?
\par 43 Beato quel servitore che il padrone, al suo arrivo, troverà facendo così.
\par 44 In verità io vi dico che lo costituirà su tutti i suoi beni.
\par 45 Ma se quel servitore dice in cuor suo: Il mio padrone mette indugio a venire; e comincia a battere i servi e le serve, e a mangiare e bere ed ubriacarsi,
\par 46 il padrone di quel servitore verrà nel giorno che non se l'aspetta e nell'ora che non sa; e lo farà lacerare a colpi di flagello, e gli assegnerà la sorte degl'infedeli.
\par 47 Or quel servitore che ha conosciuto la volontà del suo padrone e non ha preparato né fatto nulla per compiere la volontà di lui, sarà battuto di molti colpi;
\par 48 ma colui che non l'ha conosciuta e ha fatto cose degne di castigo, sarà battuto di pochi colpi. E a chi molto è stato dato, molto sarà ridomandato; e a chi molto è stato affidato, tanto più si richiederà.
\par 49 Io son venuto a gettare un fuoco sulla terra; e che mi resta a desiderare, se già è acceso?
\par 50 Ma v'è un battesimo del quale ho da esser battezzato; e come sono angustiato finché non sia compiuto!
\par 51 Pensate voi ch'io sia venuto a metter pace in terra? No, vi dico; ma piuttosto divisione;
\par 52 perché, da ora innanzi, se vi sono cinque persone in una casa, saranno divise tre contro due, e due contro tre;
\par 53 saranno divisi il padre contro il figliuolo, e il figliuolo contro il padre; la madre contro la figliuola, e la figliuola contro la madre; la suocera contro la nuora, e la nuora contro la suocera.
\par 54 Diceva poi ancora alle turbe: Quando vedete una nuvola venir su da ponente, voi dite subito: Viene la pioggia; e così succede.
\par 55 E quando sentite soffiar lo scirocco, dite: Farà caldo, e avviene così.
\par 56 Ipocriti, ben sapete discernere l'aspetto della terra e del cielo; e come mai non sapete discernere questo tempo?
\par 57 E perché non giudicate da voi stessi ciò che è giusto?
\par 58 Quando vai col tuo avversario davanti al magistrato, fa' di tutto, mentre sei per via, per liberarti da lui; che talora e' non ti tragga dinanzi al giudice, e il giudice ti dia in man dell'esecutore giudiziario, e l'esecutore ti cacci in prigione.
\par 59 Io ti dico che non uscirai di là, finché tu non abbia pagato fino all'ultimo spicciolo.

\chapter{13}

\par 1 In quello stesso tempo vennero alcuni a riferirgli il fatto dei Galilei il cui sangue Pilato aveva mescolato coi loro sacrifici.
\par 2 E Gesù, rispondendo, disse loro: Pensate voi che quei Galilei fossero più peccatori di tutti i Galilei perché hanno sofferto tali cose?
\par 3 No, vi dico; ma se non vi ravvedete, tutti similmente perirete.
\par 4 O quei diciotto sui quali cadde la torre in Siloe e li uccise, pensate voi che fossero più colpevoli di tutti gli abitanti di Gerusalemme?
\par 5 No, vi dico; ma se non vi ravvedete, tutti al par di loro perirete.
\par 6 Disse pure questa parabola: Un tale aveva un fico piantato nella sua vigna; e andò a cercarvi del frutto, e non ne trovò.
\par 7 Disse dunque al vignaiuolo: Ecco, sono ormai tre anni che vengo a cercar frutto da questo fico, e non ne trovo; taglialo; perché sta lì a rendere improduttivo anche il terreno?
\par 8 Ma l'altro, rispondendo, gli disse: Signore, lascialo ancora quest'anno, finch'io l'abbia scalzato e concimato;
\par 9 e forse darà frutto in avvenire; se no, lo taglierai.
\par 10 Or egli stava insegnando in una delle sinagoghe in giorno di sabato.
\par 11 Ed ecco una donna, che da diciotto anni aveva uno spirito d'infermità, ed era tutta curvata e incapace di raddrizzarsi in alcun modo.
\par 12 E Gesù, vedutala, la chiamò a sé e le disse: Donna, tu sei liberata dalla tua infermità.
\par 13 E pose le mani su lei, ed ella in quell'istante fu raddrizzata e glorificava Iddio.
\par 14 Or il capo della sinagoga, sdegnato che Gesù avesse fatta una guarigione in giorno di sabato, prese a dire alla moltitudine: Ci son sei giorni ne' quali s'ha da lavorare; venite dunque in quelli a farvi guarire, e non in giorno di sabato.
\par 15 Ma il Signore gli rispose e disse: Ipocriti, non scioglie ciascun di voi, di sabato, il suo bue o il suo asino dalla mangiatoia per menarlo a bere?
\par 16 E costei, ch'è figliuola d'Abramo, e che Satana avea tenuta legata per ben diciott'anni, non doveva esser sciolta da questo legame in giorno di sabato?
\par 17 E mentre diceva queste cose, tutti i suoi avversari erano confusi, e tutta la moltitudine si rallegrava di tutte le opere gloriose da lui compiute.
\par 18 Diceva dunque: A che è simile il regno di Dio, e a che l'assomiglierò io?
\par 19 Esso è simile ad un granel di senapa che un uomo ha preso e gettato nel suo orto; ed è cresciuto ed è divenuto albero; e gli uccelli del cielo si son riparati sui suoi rami.
\par 20 E di nuovo disse: A che assomiglierò il regno di Dio?
\par 21 Esso è simile al lievito che una donna ha preso e nascosto in tre staia di farina, finché tutta sia lievitata.
\par 22 Ed egli attraversava man mano le città ed i villaggi, insegnando, e facendo cammino verso Gerusalemme.
\par 23 E un tale gli disse: Signore, son pochi i salvati?
\par 24 Ed egli disse loro: Sforzatevi d'entrare per la porta stretta, perché io vi dico che molti cercheranno d'entrare e non potranno.
\par 25 Da che il padron di casa si sarà alzato ed avrà serrata la porta, e voi, stando di fuori, comincerete a picchiare alla porta, dicendo: Signore, aprici, egli, rispondendo, vi dirà: Io non so d'onde voi siate.
\par 26 Allora comincerete a dire: Noi abbiam mangiato e bevuto in tua presenza, e tu hai insegnato nelle nostre piazze!
\par 27 Ed egli dirà: Io vi dico che non so d'onde voi siate; dipartitevi da me voi tutti operatori d'iniquità.
\par 28 Quivi sarà il pianto e lo stridor de' denti, quando vedrete Abramo e Isacco e Giacobbe e tutti i profeti nel regno di Dio, e che voi ne sarete cacciati fuori.
\par 29 E ne verranno d'oriente e d'occidente, e da settentrione e da mezzogiorno, che si porranno a mensa nel regno di Dio.
\par 30 Ed ecco, ve ne son degli ultimi che saranno primi, e de' primi che saranno ultimi.
\par 31 In quello stesso momento vennero alcuni Farisei a dirgli: Parti, e vattene di qui, perché Erode ti vuol far morire.
\par 32 Ed egli disse loro: Andate a dire a quella volpe: Ecco, io caccio i demonî e compio guarigioni oggi e domani, e il terzo giorno giungo al mio termine.
\par 33 D'altronde, bisogna ch'io cammini oggi e domani e posdomani, perché non può essere che un profeta muoia fuori di Gerusalemme.
\par 34 Gerusalemme, Gerusalemme, che uccidi i profeti e lapidi quelli che ti son mandati, quante volte ho voluto raccogliere i tuoi figliuoli, come la gallina raccoglie i suoi pulcini sotto le ali; e voi non avete voluto!
\par 35 Ecco, la vostra casa sta per esservi lasciata deserta. E io vi dico che non mi vedrete più, finché venga il giorno che diciate: Benedetto colui che viene nel nome del Signore!

\chapter{14}

\par 1 E avvenne che, essendo egli entrato in casa di uno de' principali Farisei in giorno di sabato per prender cibo, essi lo stavano osservando.
\par 2 Ed ecco, gli stava dinanzi un uomo idropico.
\par 3 E Gesù prese a dire ai dottori della legge ed ai Farisei: È egli lecito o no far guarigioni in giorno di sabato? Ma essi tacquero.
\par 4 Allora egli, presolo, lo guarì e lo licenziò.
\par 5 Poi disse loro: Chi di voi, se un figliuolo od un bue cade in un pozzo, non lo trae subito fuori in giorno di sabato?
\par 6 Ed essi non potevano risponder nulla in contrario.
\par 7 Notando poi come gl'invitati sceglievano i primi posti, disse loro questa parabola:
\par 8 Quando sarai invitato a nozze da qualcuno, non ti mettere a tavola al primo posto, che talora non sia stato invitato da lui qualcuno più ragguardevole di te,
\par 9 e chi ha invitato te e lui non venga a dirti: Cedi il posto a questo! e tu debba con tua vergogna cominciare allora ad occupare l'ultimo posto.
\par 10 Ma quando sarai invitato, va a metterti all'ultimo posto, affinché quando colui che t'ha invitato verrà, ti dica: Amico, sali più in su. Allora ne avrai onore dinanzi a tutti quelli che saran teco a tavola.
\par 11 Poiché chiunque s'innalza sarà abbassato e chi si abbassa sarà innalzato.
\par 12 E diceva pure a colui che lo aveva invitato: Quando fai un desinare o una cena, non chiamare i tuoi amici, né i tuoi fratelli, né i tuoi parenti, né i vicini ricchi; che talora anch'essi non t'invitino, e ti sia reso il contraccambio;
\par 13 ma quando fai un convito, chiama i poveri, gli storpi, gli zoppi, i ciechi;
\par 14 e sarai beato, perché non hanno modo di rendertene il contraccambio; ma il contraccambio ti sarà reso alla risurrezione de' giusti.
\par 15 Or uno de' commensali, udite queste cose, gli disse: Beato chi mangerà del pane nel regno di Dio!
\par 16 Ma Gesù gli disse: Un uomo fece una gran cena e invitò molti;
\par 17 e all'ora della cena, mandò il suo servitore a dire agl'invitati: Venite, perché tutto è già pronto.
\par 18 E tutti, ad una voce, cominciarono a scusarsi. Il primo gli disse: Ho comprato un campo e ho necessità d'andarlo a vedere; ti prego, abbimi per iscusato.
\par 19 E un altro disse: Ho comprato cinque paia di buoi, e vado a provarli; ti prego, abbimi per iscusato.
\par 20 E un altro disse: Ho preso moglie, e perciò non posso venire.
\par 21 E il servitore, tornato, riferì queste cose al suo signore. Allora il padron di casa, adiratosi, disse al suo servitore: Va' presto per le piazze e per le vie della città, e mena qua i poveri, gli storpi, i ciechi e gli zoppi.
\par 22 Poi il servitore disse: Signore, s'è fatto come hai comandato, e ancora c'è posto.
\par 23 E il signore disse al servitore: Va' fuori per le strade e lungo le siepi, e costringili ad entrare, affinché la mia casa sia piena.
\par 24 Perché io vi dico che nessuno di quegli uomini ch'erano stati invitati assaggerà la mia cena.
\par 25 Or molte turbe andavano con lui; ed egli, rivoltosi, disse loro:
\par 26 Se uno viene a me e non odia suo padre, e sua madre, e la moglie, e i fratelli, e le sorelle, e finanche la sua propria vita, non può esser mio discepolo.
\par 27 E chi non porta la sua croce e non vien dietro a me, non può esser mio discepolo.
\par 28 Infatti chi è fra voi colui che, volendo edificare una torre, non si metta prima a sedere e calcoli la spesa per vedere se ha da poterla finire?
\par 29 Che talora, quando ne abbia posto il fondamento e non la possa finire, tutti quelli che la vedranno prendano a beffarsi di lui, dicendo:
\par 30 Quest'uomo ha cominciato a edificare e non ha potuto finire.
\par 31 Ovvero, qual è il re che, partendo per muover guerra ad un altro re, non si metta prima a sedere ed esamini se possa con diecimila uomini affrontare colui che gli vien contro con ventimila?
\par 32 Se no, mentre quello è ancora lontano, gli manda un'ambasciata e chiede di trattar la pace.
\par 33 Così dunque ognun di voi che non rinunzi a tutto quello che ha, non può esser mio discepolo.
\par 34 Il sale, certo, è buono; ma se anche il sale diventa insipido, con che gli si darà sapore?
\par 35 Non serve né per terra, né per concime; lo si butta via. Chi ha orecchi da udire, oda.

\chapter{15}

\par 1 Or tutti i pubblicani e i peccatori s'accostavano a lui per udirlo.
\par 2 E così i Farisei come gli scribi mormoravano, dicendo: Costui accoglie i peccatori e mangia con loro.
\par 3 Ed egli disse loro questa parabola:
\par 4 Chi è l'uomo fra voi, che, avendo cento pecore, se ne perde una, non lasci le novantanove nel deserto e non vada dietro alla perduta finché non l'abbia ritrovata?
\par 5 E trovatala, tutto allegro se la mette sulle spalle;
\par 6 e giunto a casa, chiama assieme gli amici e i vicini, e dice loro: Rallegratevi meco, perché ho ritrovato la mia pecora ch'era perduta.
\par 7 Io vi dico che così vi sarà in cielo più allegrezza per un solo peccatore che si ravvede, che per novantanove giusti i quali non han bisogno di ravvedimento.
\par 8 Ovvero, qual è la donna che avendo dieci dramme, se ne perde una, non accenda un lume e non spazzi la casa e non cerchi con cura finché non l'abbia ritrovata?
\par 9 E quando l'ha trovata, chiama assieme le amiche e le vicine, dicendo: Rallegratevi meco, perché ho ritrovato la dramma che avevo perduta.
\par 10 Così, vi dico, v'è allegrezza dinanzi agli angeli di Dio per un solo peccatore che si ravvede.
\par 11 Disse ancora: Un uomo avea due figliuoli;
\par 12 e il più giovane di loro disse al padre: Padre, dammi la parte de' beni che mi tocca. Ed egli spartì fra loro i beni.
\par 13 E di lì a poco, il figliuolo più giovane, messa insieme ogni cosa, se ne partì per un paese lontano, e quivi dissipò la sostanza, vivendo dissolutamente.
\par 14 E quand'ebbe speso ogni cosa, una gran carestia sopravvenne in quel paese, sicché egli cominciò ad esser nel bisogno.
\par 15 E andò, e si mise con uno degli abitanti di quel paese, il quale lo mandò nei suoi campi, a pasturare i porci.
\par 16 Ed egli avrebbe bramato empirsi il corpo de' baccelli che i porci mangiavano, ma nessuno gliene dava.
\par 17 Ma rientrato in sé, disse: Quanti servi di mio padre hanno pane in abbondanza, ed io qui mi muoio di fame!
\par 18 Io mi leverò e me n'andrò a mio padre, e gli dirò: Padre, ho peccato contro il cielo e contro te:
\par 19 non son più degno d'esser chiamato tuo figliuolo; trattami come uno de' tuoi servi.
\par 20 Egli dunque si levò e venne a suo padre; ma mentr'egli era ancora lontano, suo padre lo vide e fu mosso a compassione, e corse, e gli si gettò al collo, e lo baciò e ribaciò.
\par 21 E il figliuolo gli disse: Padre, ho peccato contro il cielo e contro te; non son più degno d'esser chiamato tuo figliuolo.
\par 22 Ma il padre disse ai suoi servitori: Presto, portate qua la veste più bella e rivestitelo, e mettetegli un anello al dito e de' calzari a' piedi;
\par 23 e menate fuori il vitello ingrassato, ammazzatelo, e mangiamo e rallegriamoci,
\par 24 perché questo mio figliuolo era morto, ed è tornato a vita; era perduto, ed è stato ritrovato. E si misero a far gran festa.
\par 25 Or il figliuolo maggiore era a' campi; e come tornando fu vicino alla casa, udì la musica e le danze.
\par 26 E chiamato a sé uno de' servitori, gli domandò che cosa ciò volesse dire.
\par 27 Quello gli disse: È giunto tuo fratello, e tuo padre ha ammazzato il vitello ingrassato, perché l'ha riavuto sano e salvo.
\par 28 Ma egli si adirò e non volle entrare; onde suo padre uscì fuori e lo pregava d'entrare.
\par 29 Ma egli, rispondendo, disse al padre: Ecco, da tanti anni ti servo, e non ho mai trasgredito un tuo comando; a me però non hai mai dato neppure un capretto da far festa con i miei amici;
\par 30 ma quando è venuto questo tuo figliuolo che ha divorato i tuoi beni con le meretrici, tu hai ammazzato per lui il vitello ingrassato.
\par 31 E il padre gli disse: Figliuolo, tu sei sempre meco, ed ogni cosa mia è tua;
\par 32 ma bisognava far festa e rallegrarsi, perché questo tuo fratello era morto, ed è tornato a vita; era perduto, ed è stato ritrovato.

\chapter{16}

\par 1 Gesù diceva ancora ai suoi discepoli: V'era un uomo ricco che avea un fattore, il quale fu accusato dinanzi a lui di dissipare i suoi beni.
\par 2 Ed egli lo chiamò e gli disse: Che cos'è questo che odo di te? Rendi conto della tua amministrazione, perché tu non puoi più esser mio fattore.
\par 3 E il fattore disse fra sé: Che farò io, dacché il padrone mi toglie l'amministrazione? A zappare non son buono: a mendicare mi vergogno.
\par 4 So bene quel che farò, affinché, quando dovrò lasciare l'amministrazione, ci sia chi mi riceva in casa sua.
\par 5 Chiamati quindi a sé ad uno ad uno i debitori del suo padrone, disse al primo:
\par 6 Quanto devi al mio padrone? Quello rispose: Cento bati d'olio. Egli disse: Prendi la tua scritta, siedi, e scrivi presto: Cinquanta.
\par 7 Poi disse ad un altro: E tu, quanto devi? Quello rispose: Cento cori di grano. Egli disse: Prendi la tua scritta, e scrivi: Ottanta.
\par 8 E il padrone lodò il fattore infedele perché aveva operato con avvedutezza; poiché i figliuoli di questo secolo, nelle relazioni con que' della loro generazione, sono più accorti dei figliuoli della luce.
\par 9 Ed io vi dico: Fatevi degli amici con le ricchezze ingiuste; affinché, quand'esse verranno meno, quelli vi ricevano ne' tabernacoli eterni.
\par 10 Chi è fedele nelle cose minime, è pur fedele nelle grandi; e chi è ingiusto nelle cose minime, è pure ingiusto nelle grandi.
\par 11 Se dunque non siete stati fedeli nelle ricchezze ingiuste, chi vi affiderà le vere?
\par 12 E se non siete stati fedeli nell'altrui, chi vi darà il vostro?
\par 13 Nessun domestico può servire a due padroni: perché o odierà l'uno e amerà l'altro, o si atterrà all'uno e sprezzerà l'altro. Voi non potete servire a Dio ed a Mammona.
\par 14 Or i Farisei, che amavano il danaro, udivano tutte queste cose e si facean beffe di lui.
\par 15 Ed egli disse loro: Voi siete quelli che vi proclamate giusti dinanzi agli uomini; ma Dio conosce i vostri cuori; poiché quel che è eccelso fra gli uomini, è abominazione dinanzi a Dio.
\par 16 La legge ed i profeti hanno durato fino a Giovanni; da quel tempo è annunziata la buona novella del regno di Dio, ed ognuno v'entra a forza.
\par 17 Più facile è che passino cielo e terra, che un apice solo della legge cada.
\par 18 Chiunque manda via la moglie e ne sposa un'altra, commette adulterio; e chiunque sposa una donna mandata via dal marito, commette adulterio.
\par 19 Or v'era un uomo ricco, il quale vestiva porpora e bisso, ed ogni giorno godeva splendidamente;
\par 20 e v'era un pover'uomo chiamato Lazzaro, che giaceva alla porta di lui, pieno d'ulceri,
\par 21 e bramoso di sfamarsi con le briciole che cadevano dalla tavola del ricco; anzi perfino venivano i cani a leccargli le ulceri.
\par 22 Or avvenne che il povero morì e fu portato dagli angeli nel seno d'Abramo; morì anche il ricco, e fu seppellito.
\par 23 E nell'Ades, essendo ne' tormenti, alzò gli occhi e vide da lontano Abramo, e Lazzaro nel suo seno;
\par 24 ed esclamò: Padre Abramo, abbi pietà di me, e manda Lazzaro a intingere la punta del dito nell'acqua per rinfrescarmi la lingua, perché son tormentato in questa fiamma.
\par 25 Ma Abramo disse: Figliuolo, ricordati che tu ricevesti i tuoi beni in vita tua, e che Lazzaro similmente ricevette i mali; ma ora qui egli è consolato, e tu sei tormentato.
\par 26 E oltre a tutto questo, fra noi e voi è posta una gran voragine, perché quelli che vorrebbero passar di qui a voi non possano, né di là si passi da noi.
\par 27 Ed egli disse: Ti prego, dunque, o padre, che tu lo mandi a casa di mio padre,
\par 28 perché ho cinque fratelli, affinché attesti loro queste cose, onde non abbiano anch'essi a venire in questo luogo di tormento.
\par 29 Abramo disse: Hanno Mosè e i profeti; ascoltin quelli.
\par 30 Ed egli: No, padre Abramo; ma se uno va a loro dai morti, si ravvedranno.
\par 31 Ma Abramo rispose: Se non ascoltano Mosè e i profeti, non si lasceranno persuadere neppure se uno dei morti risuscitasse.

\chapter{17}

\par 1 Disse poi ai suoi discepoli: È impossibile che non avvengano scandali: ma guai a colui per cui avvengono!
\par 2 Meglio per lui sarebbe che una macina da mulino gli fosse messa al collo e fosse gettato nel mare, piuttosto che scandalizzare un solo di questi piccoli.
\par 3 Badate a voi stessi! Se il tuo fratello pecca, riprendilo; e se si pente, perdonagli.
\par 4 E se ha peccato contro te sette volte al giorno, e sette volte torna a te e ti dice: Mi pento, perdonagli.
\par 5 Allora gli apostoli dissero al Signore: Aumentaci la fede.
\par 6 E il Signore disse: Se aveste fede quant'è un granel di senapa, potreste dire a questo moro: Sràdicati e trapiantati nel mare, e vi ubbidirebbe.
\par 7 Or chi di voi, avendo un servo ad arare o pascere, quand'ei torna a casa dai campi, gli dirà: Vieni presto a metterti a tavola?
\par 8 Non gli dirà invece: Preparami la cena, e cingiti a servirmi finch'io abbia mangiato e bevuto, e poi mangerai e berrai tu?
\par 9 Si ritiene egli forse obbligato al suo servo perché ha fatto le cose comandategli?
\par 10 Così anche voi, quand'avrete fatto tutto ciò che v'è comandato, dite: Noi siamo servi inutili; abbiam fatto quel ch'eravamo in obbligo di fare.
\par 11 Ed avvenne che, nel recarsi a Gerusalemme, egli passava sui confini della Samaria e della Galilea.
\par 12 E come entrava in un certo villaggio, gli si fecero incontro dieci uomini lebbrosi, i quali, fermatisi da lontano,
\par 13 alzaron la voce dicendo: Gesù, Maestro, abbi pietà di noi!
\par 14 E, vedutili, egli disse loro: Andate a mostrarvi a' sacerdoti. E avvenne che, mentre andavano, furon mondati.
\par 15 E uno di loro, vedendo che era guarito, tornò indietro, glorificando Iddio ad alta voce;
\par 16 e si gettò ai suoi piedi con la faccia a terra, ringraziandolo; e questo era un Samaritano.
\par 17 Gesù, rispondendo, disse: I dieci non sono stati tutti mondati? E i nove altri dove sono?
\par 18 Non si è trovato alcuno che sia tornato per dar gloria a Dio fuor che questo straniero?
\par 19 E gli disse: Lèvati e vattene: la tua fede t'ha salvato.
\par 20 Interrogato poi dai Farisei sul quando verrebbe il regno di Dio, rispose loro dicendo: Il regno di Dio non viene in maniera da attirar gli sguardi; né si dirà:
\par 21 Eccolo qui, o eccolo là; perché ecco, il regno di Dio è dentro di voi.
\par 22 Disse pure ai suoi discepoli: Verranno giorni che desidererete vedere uno de' giorni del Figliuol dell'uomo, e non lo vedrete.
\par 23 E vi si dirà: Eccolo là, eccolo qui; non andate, e non li seguite;
\par 24 perché com'è il lampo che balenando risplende da un'estremità all'altra del cielo, così sarà il Figliuol dell'uomo nel suo giorno.
\par 25 Ma prima bisogna ch'e' soffra molte cose, e sia reietto da questa generazione.
\par 26 E come avvenne a' giorni di Noè, così pure avverrà a' giorni del Figliuol dell'uomo.
\par 27 Si mangiava, si beveva, si prendea moglie, s'andava a marito, fino al giorno che Noè entrò nell'arca, e venne il diluvio che li fece tutti perire.
\par 28 Nello stesso modo che avvenne anche ai giorni di Lot; si mangiava, si beveva, si comprava, si vendeva, si piantava, si edificava;
\par 29 ma nel giorno che Lot uscì di Sodoma piovve dal cielo fuoco e zolfo, che li fece tutti perire.
\par 30 Lo stesso avverrà nel giorno che il Figliuol dell'uomo sarà manifestato.
\par 31 In quel giorno, chi sarà sulla terrazza ed avrà la sua roba in casa, non scenda a prenderla; e parimente, chi sarà nei campi non torni indietro.
\par 32 Ricordatevi della moglie di Lot.
\par 33 Chi cercherà di salvare la sua vita, la perderà; ma chi la perderà, la preserverà.
\par 34 Io ve lo dico: In quella notte, due saranno in un letto; l'uno sarà preso, e l'altro lasciato.
\par 35 Due donne macineranno assieme; l'una sarà presa, e l'altra lasciata.
\par 36 tex
\par 37 I discepoli risposero: Dove sarà, Signore? Ed egli disse loro: Dove sarà il corpo, ivi anche le aquile si raduneranno.

\chapter{18}

\par 1 Propose loro ancora questa parabola per mostrare che doveano del continuo pregare e non stancarsi.
\par 2 In una certa città v'era un giudice, che non temeva Iddio né avea rispetto per alcun uomo;
\par 3 e in quella città vi era una vedova, la quale andava da lui dicendo: Fammi giustizia del mio avversario.
\par 4 Ed egli per un tempo non volle farlo; ma poi disse fra sé: benché io non tema Iddio e non abbia rispetto per alcun uomo,
\par 5 pure, poiché questa vedova mi dà molestia, le farò giustizia, che talora, a forza di venire, non finisca col rompermi la testa.
\par 6 E il Signore disse: Ascoltate quel che dice il giudice iniquo.
\par 7 E Dio non farà egli giustizia ai suoi eletti che giorno e notte gridano a lui, e sarà egli tardo per loro?
\par 8 Io vi dico che farà loro prontamente giustizia. Ma quando il Figliuol dell'uomo verrà, troverà egli la fede sulla terra?
\par 9 E disse ancora questa parabola per certuni che confidavano in se stessi di esser giusti e disprezzavano gli altri:
\par 10 Due uomini salirono al tempio per pregare; l'uno Fariseo, e l'altro pubblicano.
\par 11 Il Fariseo, stando in piè, pregava così dentro di sé: O Dio, ti ringrazio ch'io non sono come gli altri uomini, rapaci, ingiusti, adulteri; né pure come quel pubblicano.
\par 12 Io digiuno due volte la settimana; pago la decima su tutto quel che posseggo.
\par 13 Ma il pubblicano, stando da lungi, non ardiva neppure alzar gli occhi al cielo; ma si batteva il petto, dicendo: O Dio, sii placato verso me peccatore!
\par 14 Io vi dico che questi scese a casa sua giustificato, piuttosto che quell'altro; perché chiunque s'innalza sarà abbassato; ma chi si abbassa sarà innalzato.
\par 15 Or gli recavano anche i bambini, perché li toccasse; ma i discepoli, veduto questo, sgridavano quelli che glieli recavano.
\par 16 Ma Gesù chiamò a sé i bambini, e disse: Lasciate i piccoli fanciulli venire a me, e non glielo vietate, perché di tali è il regno di Dio.
\par 17 In verità io vi dico che chiunque non avrà ricevuto il regno di Dio come un piccolo fanciullo, non entrerà punto in esso.
\par 18 E uno dei principali lo interrogò, dicendo: Maestro buono, che farò io per ereditare la vita eterna?
\par 19 E Gesù gli disse: Perché mi chiami buono? Nessuno è buono, salvo uno solo, cioè Iddio.
\par 20 Tu sai i comandamenti: Non commettere adulterio; non uccidere; non rubare; non dir falsa testimonianza; onora tuo padre e tua madre.
\par 21 Ed egli rispose: Tutte queste cose io le ho osservate fin dalla mia giovinezza.
\par 22 E Gesù, udito questo, gli disse: Una cosa ti manca ancora; vendi tutto ciò che hai, e distribuiscilo ai poveri, e tu avrai un tesoro nel cielo; poi vieni e seguitami.
\par 23 Ma egli, udite queste cose, ne fu grandemente attristato, perch'era molto ricco.
\par 24 E Gesù, vedendolo a quel modo, disse: Quanto malagevolmente coloro che hanno delle ricchezze entreranno nel regno di Dio!
\par 25 Poiché è più facile a un cammello passare per la cruna d'un ago, che ad un ricco entrare nel regno di Dio.
\par 26 E quelli che udirono questo dissero: Chi dunque può essere salvato?
\par 27 Ma egli rispose: Le cose impossibili agli uomini sono possibili a Dio.
\par 28 E Pietro disse: Ecco, noi abbiam lasciato le nostre case, e t'abbiam seguitato.
\par 29 Ed egli disse loro: Io vi dico in verità che non v'è alcuno che abbia lasciato casa, o moglie, o fratelli, o genitori, o figliuoli per amor del regno di Dio,
\par 30 il quale non ne riceva molte volte tanto in questo tempo, e nel secolo avvenire la vita eterna.
\par 31 Poi, presi seco i dodici, disse loro: Ecco, noi saliamo a Gerusalemme, e saranno adempiute rispetto al Figliuol dell'uomo tutte le cose scritte dai profeti;
\par 32 poiché egli sarà dato in man de' Gentili, e sarà schernito ed oltraggiato e gli sputeranno addosso;
\par 33 e dopo averlo flagellato, l'uccideranno; ma il terzo giorno risusciterà.
\par 34 Ed essi non capirono nulla di queste cose; quel parlare era per loro oscuro, e non intendevano le cose dette loro.
\par 35 Or avvenne che com'egli si avvicinava a Gerico, un certo cieco sedeva presso la strada, mendicando;
\par 36 e, udendo la folla che passava, domandò che cosa fosse.
\par 37 E gli fecero sapere che passava Gesù il Nazareno.
\par 38 Allora egli gridò: Gesù figliuol di Davide, abbi pietà di me!
\par 39 E quelli che precedevano lo sgridavano perché tacesse; ma lui gridava più forte: Figliuol di Davide, abbi pietà di me!
\par 40 E Gesù, fermatosi, comandò che gli fosse menato; e quando gli fu vicino, gli domandò:
\par 41 Che vuoi tu ch'io ti faccia? Ed egli disse: Signore, ch'io ricuperi la vista.
\par 42 E Gesù gli disse: Ricupera la vista; la tua fede t'ha salvato.
\par 43 E in quell'istante ricuperò la vista, e lo seguiva glorificando Iddio; e tutto il popolo, veduto ciò, diede lode a Dio.

\chapter{19}

\par 1 E Gesù, essendo entrato in Gerico, attraversava la città.
\par 2 Ed ecco, un uomo, chiamato per nome Zaccheo, il quale era capo dei pubblicani ed era ricco,
\par 3 cercava di veder chi era Gesù, ma non poteva a motivo della folla, perché era piccolo di statura.
\par 4 Allora corse innanzi, e montò sopra un sicomoro, per vederlo, perch'egli avea da passar per quella via.
\par 5 E come Gesù fu giunto in quel luogo, alzati gli occhi, gli disse: Zaccheo, scendi presto, perché oggi debbo albergare in casa tua.
\par 6 Ed egli s'affrettò a scendere e l'accolse con allegrezza.
\par 7 E veduto ciò, tutti mormoravano, dicendo: È andato ad albergare da un peccatore!
\par 8 Ma Zaccheo, presentatosi al Signore, gli disse: Ecco, Signore, la metà de' miei beni la do ai poveri; e se ho frodato qualcuno di qualcosa gli rendo il quadruplo.
\par 9 E Gesù gli disse: Oggi la salvezza è entrata in questa casa, poiché anche questo è figliuolo d'Abramo:
\par 10 poiché il Figliuol dell'uomo è venuto per cercare e salvare ciò che era perito.
\par 11 Or com'essi ascoltavano queste cose, Gesù aggiunse una parabola, perché era vicino a Gerusalemme ed essi pensavano che il regno di Dio stesse per esser manifestato immediatamente.
\par 12 Disse dunque: Un uomo nobile se n'andò in un paese lontano per ricevere l'investitura d'un regno e poi tornare.
\par 13 E chiamati a sé dieci suoi servitori, diede loro dieci mine, e disse loro: Trafficate finch'io venga.
\par 14 Ma i suoi concittadini l'odiavano, e gli mandaron dietro un'ambasciata per dire: Non vogliamo che costui regni su noi.
\par 15 Ed avvenne, quand'e' fu tornato, dopo aver ricevuto l'investitura del regno, ch'egli fece venire quei servitori ai quali avea dato il danaro, per sapere quanto ognuno avesse guadagnato, trafficando.
\par 16 Si presentò il primo e disse: Signore, la tua mina ne ha fruttate altre dieci.
\par 17 Ed egli gli disse: Va bene, buon servitore; poiché sei stato fedele in cosa minima, abbi potestà su dieci città.
\par 18 Poi venne il secondo, dicendo: La tua mina, signore, ha fruttato cinque mine.
\par 19 Ed egli disse anche a questo: E tu sii sopra cinque città.
\par 20 Poi ne venne un altro che disse: Signore, ecco la tua mina che ho tenuta riposta in un fazzoletto,
\par 21 perché ho avuto paura di te che sei uomo duro; tu prendi quel che non hai messo, e mieti quel che non hai seminato.
\par 22 E il padrone a lui: Dalle tue parole ti giudicherò, servo malvagio! Tu sapevi ch'io sono un uomo duro, che prendo quel che non ho messo e mieto quel che non ho seminato;
\par 23 e perché non hai messo il mio danaro alla banca, ed io, al mio ritorno, l'avrei riscosso con l'interesse?
\par 24 Poi disse a coloro ch'eran presenti: Toglietegli la mina, e datela a colui che ha le dieci mine:
\par 25 - Essi gli dissero: Signore, egli ha dieci mine. -
\par 26 Io vi dico che a chiunque ha sarà dato; ma a chi non ha sarà tolto anche quello che ha.
\par 27 Quanto poi a quei miei nemici che non volevano che io regnassi su loro, menateli qua e scannateli in mia presenza.
\par 28 E dette queste cose, Gesù andava innanzi, salendo a Gerusalemme.
\par 29 E avvenne che come fu vicino a Betfage e a Betania presso al monte detto degli Ulivi, mandò due de' discepoli, dicendo:
\par 30 Andate nella borgata dirimpetto, nella quale entrando, troverete legato un puledro d'asino, sopra il quale non è mai montato alcuno; scioglietelo e menatemelo.
\par 31 E se qualcuno vi domanda perché lo sciogliete, direte così: Il Signore ne ha bisogno.
\par 32 E quelli ch'erano mandati, partirono e trovarono le cose com'egli avea lor detto.
\par 33 E com'essi scioglievano il puledro, i suoi padroni dissero loro: Perché sciogliete il puledro?
\par 34 Essi risposero: Il Signore ne ha bisogno.
\par 35 E lo menarono a Gesù; e gettati i loro mantelli sul puledro, vi fecero montar Gesù.
\par 36 E mentre egli andava innanzi, stendevano i loro mantelli sulla via.
\par 37 E com'era già presso la città, alla scesa del monte degli Ulivi, tutta la moltitudine dei discepoli cominciò con allegrezza a lodare Iddio a gran voce per tutte le opere potenti che aveano vedute,
\par 38 dicendo: Benedetto il Re che viene nel nome del Signore; pace in cielo e gloria ne' luoghi altissimi!
\par 39 e alcuni de' Farisei di tra la folla gli dissero: Maestro, sgrida i tuoi discepoli!
\par 40 Ed egli, rispondendo, disse: Io vi dico che se costoro si tacciono, le pietre grideranno.
\par 41 E come si fu avvicinato, vedendo la città, pianse su lei, dicendo:
\par 42 Oh se tu pure avessi conosciuto in questo giorno quel ch'è per la tua pace! Ma ora è nascosto agli occhi tuoi.
\par 43 Poiché verranno su te de' giorni nei quali i tuoi nemici ti faranno attorno delle trincee, e ti circonderanno e ti stringeranno da ogni parte;
\par 44 e atterreranno te e i tuoi figliuoli dentro di te, e non lasceranno in te pietra sopra pietra, perché tu non hai conosciuto il tempo nel quale sei stata visitata.
\par 45 Poi, entrato nel tempio, cominciò a cacciar quelli che in esso vendevano,
\par 46 dicendo loro: Egli è scritto: La mia casa sarà una casa d'orazione, ma voi ne avete fatto una spelonca di ladroni.
\par 47 Ed ogni giorno insegnava nel tempio. Ma i capi sacerdoti e gli scribi e i primi fra il popolo cercavano di farlo morire;
\par 48 ma non sapevano come fare, perché tutto il popolo, ascoltandolo, pendeva dalle sue labbra.

\chapter{20}

\par 1 E avvenne un di quei giorni, che mentre insegnava al popolo nel tempio ed evangelizzava, sopraggiunsero i capi sacerdoti e gli scribi con gli anziani, e gli parlaron così:
\par 2 Dicci con quale autorità tu fai queste cose, e chi t'ha data codesta autorità.
\par 3 Ed egli, rispondendo, disse loro: Anch'io vi domanderò una cosa:
\par 4 Il battesimo di Giovanni era dal cielo o dagli uomini?
\par 5 Ed essi ragionavan fra loro, dicendo: Se diciamo: Dal cielo, egli ci dirà: Perché non gli credeste?
\par 6 Ma se diciamo: Dagli uomini, tutto il popolo ci lapiderà, perché è persuaso che Giovanni era un profeta.
\par 7 E risposero che non sapevano d'onde fosse.
\par 8 E Gesù disse loro: Neppur io vi dico con quale autorità fo queste cose.
\par 9 Poi prese a dire al popolo questa parabola: Un uomo piantò una vigna, l'allogò a dei lavoratori, e se n'andò in viaggio per lungo tempo.
\par 10 E nella stagione mandò a que' lavoratori un servitore perché gli dessero del frutto della vigna; ma i lavoratori, battutolo, lo rimandarono a mani vuote.
\par 11 Ed egli di nuovo mandò un altro servitore; ma essi, dopo aver battuto e vituperato anche questo, lo rimandarono a mani vuote.
\par 12 Ed egli ne mandò ancora un terzo; ed essi, dopo aver ferito anche questo, lo scacciarono.
\par 13 Allora il padron della vigna disse: Che farò? Manderò il mio diletto figliuolo; forse a lui porteranno rispetto.
\par 14 Ma quando i lavoratori lo videro, fecero tra loro questo ragionamento: Costui è l'erede; uccidiamolo, affinché l'eredità diventi nostra.
\par 15 E cacciatolo fuor dalla vigna, lo uccisero. Che farà loro dunque il padron della vigna?
\par 16 Verrà e distruggerà que' lavoratori, e darà la vigna ad altri. Ed essi, udito ciò, dissero: Così non sia!
\par 17 Ma egli, guardatili in faccia, disse: Che vuol dir dunque questo che è scritto: La pietra che gli edificatori hanno riprovata è quella che è diventata pietra angolare?
\par 18 Chiunque cadrà su quella pietra sarà sfracellato; ed ella stritolerà colui sul quale cadrà.
\par 19 E gli scribi e i capi sacerdoti cercarono di mettergli le mani addosso in quella stessa ora, ma temettero il popolo; poiché si avvidero bene ch'egli avea detto quella parabola per loro.
\par 20 Ed essendosi messi ad osservarlo, gli mandarono delle spie che simulassero d'esser giusti per coglierlo in parole, affin di darlo in man dell'autorità e del potere del governatore.
\par 21 E quelli gli fecero una domanda, dicendo: Maestro, noi sappiamo che tu parli e insegni dirittamente, e non hai riguardi personali, ma insegni la via di Dio secondo verità:
\par 22 È egli lecito a noi pagare il tributo a Cesare o no?
\par 23 Ma egli avvedutosi della loro astuzia, disse loro:
\par 24 Mostratemi un denaro; di chi porta l'effigie e l'iscrizione? Ed essi dissero: Di Cesare.
\par 25 Ed egli a loro: Rendete dunque a Cesare quel ch'è di Cesare, e a Dio quel ch'è di Dio.
\par 26 Ed essi non poteron coglierlo in parole dinanzi al popolo; e maravigliati della sua risposta, si tacquero.
\par 27 Poi, accostatisi alcuni dei Sadducei, i quali negano che ci sia risurrezione, lo interrogarono, dicendo:
\par 28 Maestro, Mosè ci ha scritto che se il fratello di uno muore avendo moglie ma senza figliuoli, il fratello ne prenda la moglie e susciti progenie a suo fratello.
\par 29 Or v'erano sette fratelli. Il primo prese moglie, e morì senza figliuoli.
\par 30 Il secondo pure la sposò;
\par 31 poi il terzo; e così fu dei sette; non lasciaron figliuoli, e morirono.
\par 32 In ultimo, anche la donna morì.
\par 33 Nella risurrezione dunque, la donna, di chi di loro sarà moglie? Perché i sette l'hanno avuta per moglie.
\par 34 E Gesù disse loro: I figliuoli di questo secolo sposano e sono sposati;
\par 35 ma quelli che saranno reputati degni d'aver parte al secolo avvenire e alla risurrezione dai morti, non sposano e non sono sposati,
\par 36 perché neanche possono più morire, giacché son simili agli angeli e son figliuoli di Dio, essendo figliuoli della risurrezione.
\par 37 Che poi i morti risuscitino anche Mosè lo dichiarò nel passo del "pruno", quando chiama il Signore l'Iddio d'Abramo, l'Iddio d'Isacco e l'Iddio di Giacobbe.
\par 38 Or Egli non è un Dio di morti, ma di viventi; poiché per lui vivono tutti.
\par 39 E alcuni degli scribi, rispondendo, dissero: Maestro, hai detto bene.
\par 40 E non ardivano più fargli alcuna domanda.
\par 41 Ed egli disse loro: Come dicono che il Cristo è figliuolo di Davide?
\par 42 Poiché Davide stesso, nel libro dei Salmi, dice: Il Signore ha detto al mio Signore: Siedi alla mia destra,
\par 43 finché io abbia posto i tuoi nemici per sgabello de' tuoi piedi.
\par 44 Davide dunque lo chiama Signore; e com'è egli suo figliuolo?
\par 45 E udendolo tutto il popolo, egli disse a' suoi discepoli:
\par 46 Guardatevi dagli scribi, i quali passeggian volentieri in lunghe vesti ed amano le salutazioni nelle piazze e i primi seggi nelle sinagoghe e i primi posti nei conviti;
\par 47 essi che divorano le case delle vedove e fanno per apparenza lunghe orazioni. Costoro riceveranno maggior condanna.

\chapter{21}

\par 1 Poi, alzati gli occhi, Gesù vide dei ricchi che gettavano i loro doni nella cassa delle offerte.
\par 2 Vide pure una vedova poveretta che vi gettava due spiccioli;
\par 3 e disse: In verità vi dico che questa povera vedova ha gettato più di tutti;
\par 4 poiché tutti costoro hanno gettato nelle offerte, del loro superfluo; ma costei, del suo necessario, v'ha gettato tutto quanto avea per vivere.
\par 5 E facendo alcuni notare come il tempio fosse adorno di belle pietre e di doni consacrati, egli disse:
\par 6 Quant'è a queste cose che voi contemplate, verranno i giorni che non sarà lasciata pietra sopra pietra che non sia diroccata.
\par 7 Ed essi gli domandarono: Maestro, quando avverranno dunque queste cose? E quale sarà il segno del tempo in cui queste cose staranno per succedere?
\par 8 Ed egli disse: Guardate di non esser sedotti; perché molti verranno sotto il mio nome, dicendo: Son io; e: Il tempo è vicino; non andate dietro a loro.
\par 9 E quando udrete parlar di guerre e di sommosse, non siate spaventati; perché bisogna che queste cose avvengano prima; ma la fine non verrà subito dopo.
\par 10 Allora disse loro: Si leverà nazione contro nazione e regno contro regno;
\par 11 vi saranno gran terremoti, e in diversi luoghi pestilenze e carestie; vi saranno fenomeni spaventevoli e gran segni dal cielo.
\par 12 Ma prima di tutte queste cose, vi metteranno le mani addosso e vi perseguiteranno, dandovi in man delle sinagoghe e mettendovi in prigione, traendovi dinanzi a re e governatori, a cagion del mio nome.
\par 13 Ma ciò vi darà occasione di render testimonianza.
\par 14 Mettetevi dunque in cuore di non premeditar come rispondere a vostra difesa,
\par 15 perché io vi darò una parola e una sapienza alle quali tutti i vostri avversari non potranno contrastare né contraddire.
\par 16 Or voi sarete traditi perfino da genitori, da fratelli, da parenti e da amici; faranno morire parecchi di voi;
\par 17 e sarete odiati da tutti a cagion del mio nome;
\par 18 ma neppure un capello del vostro capo perirà.
\par 19 Con la vostra perseveranza guadagnerete le anime vostre.
\par 20 Quando vedrete Gerusalemme circondata d'eserciti, sappiate allora che la sua desolazione è vicina.
\par 21 Allora quelli che sono in Giudea, fuggano ai monti; e quelli che sono nella città, se ne partano; e quelli che sono per la campagna, non entrino in lei.
\par 22 Perché quelli son giorni di vendetta, affinché tutte le cose che sono scritte, siano adempite.
\par 23 Guai alle donne che saranno incinte, e a quelle che allatteranno in que' giorni! Perché vi sarà gran distretta nel paese ed ira su questo popolo.
\par 24 E cadranno sotto il taglio della spada, e saran menati in cattività fra tutte le genti; e Gerusalemme sarà calpestata dai Gentili, finché i tempi de' Gentili siano compiuti.
\par 25 E vi saranno de' segni nel sole, nella luna e nelle stelle; e sulla terra, angoscia delle nazioni, sbigottite dal rimbombo del mare e delle onde;
\par 26 gli uomini venendo meno per la paurosa aspettazione di quel che sarà per accadere al mondo; poiché le potenze de' cieli saranno scrollate.
\par 27 E allora vedranno il Figliuol dell'uomo venir sopra le nuvole con potenza e gran gloria.
\par 28 Ma quando queste cose cominceranno ad avvenire, rialzatevi, levate il capo, perché la vostra redenzione è vicina.
\par 29 E disse loro una parabola: Guardate il fico e tutti gli alberi;
\par 30 quando cominciano a germogliare, voi, guardando, riconoscete da voi stessi che l'estate è ormai vicina.
\par 31 Così anche voi quando vedrete avvenir queste cose, sappiate che il regno di Dio è vicino.
\par 32 In verità io vi dico che questa generazione non passerà prima che tutte queste cose siano avvenute.
\par 33 Il cielo e la terra passeranno, ma le mie parole non passeranno.
\par 34 Badate a voi stessi, che talora i vostri cuori non siano aggravati da crapula, da ubriachezza e dalle ansiose sollecitudini di questa vita, e che quel giorno non vi venga addosso all'improvviso come un laccio;
\par 35 perché verrà sopra tutti quelli che abitano sulla faccia di tutta la terra.
\par 36 Vegliate dunque, pregando in ogni tempo, affinché siate in grado di scampare a tutte queste cose che stanno per accadere, e di comparire dinanzi al Figliuol dell'uomo.
\par 37 Or di giorno egli insegnava nel tempio; e la notte usciva e la passava sul monte detto degli Ulivi.
\par 38 E tutto il popolo, la mattina di buon'ora, veniva a lui nel tempio per udirlo.

\chapter{22}

\par 1 Or la festa degli azzimi, detta la Pasqua, s'avvicinava;
\par 2 e i capi sacerdoti e gli scribi cercavano il modo di farlo morire, perché temevano il popolo.
\par 3 E Satana entrò in Giuda, chiamato Iscariota, che era del numero de' dodici.
\par 4 Ed egli andò a conferire coi capi sacerdoti e i capitani sul come lo darebbe loro nelle mani.
\par 5 Ed essi se ne rallegrarono e pattuirono di dargli del denaro.
\par 6 Ed egli prese l'impegno, e cercava l'opportunità di farlo di nascosto alla folla.
\par 7 Or venne il giorno degli azzimi, nel quale si dovea sacrificar la pasqua.
\par 8 E Gesù mandò Pietro e Giovanni, dicendo: Andate a prepararci la pasqua, affinché la mangiamo.
\par 9 Ed essi gli dissero: Dove vuoi che la prepariamo?
\par 10 Ed egli disse loro: Ecco, quando sarete entrati nella città, vi verrà incontro un uomo che porterà una brocca d'acqua; seguitelo nella casa dov'egli entrerà.
\par 11 E dite al padron di casa: Il Maestro ti manda a dire: Dov'è la stanza nella quale mangerò la pasqua co' miei discepoli?
\par 12 Ed egli vi mostrerà di sopra una gran sala ammobiliata; quivi apparecchiate.
\par 13 Ed essi andarono e trovaron com'egli avea lor detto, e prepararon la pasqua.
\par 14 E quando l'ora fu venuta, egli si mise a tavola, e gli apostoli con lui.
\par 15 Ed egli disse loro: Ho grandemente desiderato di mangiar questa pasqua con voi, prima ch'io soffra;
\par 16 poiché io vi dico che non la mangerò più finché sia compiuta nel regno di Dio.
\par 17 E avendo preso un calice, rese grazie e disse: Prendete questo e distribuitelo fra voi;
\par 18 perché io vi dico che oramai non berrò più del frutto della vigna, finché sia venuto il regno di Dio.
\par 19 Poi, avendo preso del pane, rese grazie e lo ruppe e lo diede loro, dicendo: Questo è il mio corpo il quale è dato per voi: fate questo in memoria di me.
\par 20 Parimente ancora, dopo aver cenato, dette loro il calice dicendo: Questo calice è il nuovo patto nel mio sangue, il quale è sparso per voi.
\par 21 Del resto, ecco, la mano di colui che mi tradisce è meco a tavola.
\par 22 Poiché il Figliuol dell'uomo, certo, se ne va, secondo che è determinato; ma guai a quell'uomo dal quale è tradito!
\par 23 Ed essi cominciarono a domandarsi gli uni gli altri chi sarebbe mai quel di loro che farebbe questo.
\par 24 Nacque poi anche una contesa fra loro per sapere chi di loro fosse reputato il maggiore.
\par 25 Ma egli disse loro: I re delle nazioni le signoreggiano, e quelli che hanno autorità su di esse son chiamati benefattori.
\par 26 Ma tra voi non ha da esser così; anzi, il maggiore fra voi sia come il minore, e chi governa come colui che serve.
\par 27 Poiché, chi è maggiore, colui che è a tavola oppur colui che serve? Non è forse colui che è a tavola? Ma io sono in mezzo a voi come colui che serve.
\par 28 Or voi siete quelli che avete perseverato meco nelle mie prove;
\par 29 e io dispongo che vi sia dato un regno, come il Padre mio ha disposto che fosse dato a me,
\par 30 affinché mangiate e beviate alla mia tavola nel mio regno, e sediate sui troni, giudicando le dodici tribù d'Israele.
\par 31 Simone, Simone, ecco, Satana ha chiesto di vagliarvi come si vaglia il grano;
\par 32 ma io ho pregato per te affinché la tua fede non venga meno; e tu, quando sarai convertito, conferma i tuoi fratelli.
\par 33 Ma egli gli disse: Signore, con te son pronto ad andare e in prigione e alla morte.
\par 34 E Gesù: Pietro, io ti dico che oggi il gallo non canterà, prima che tu abbia negato tre volte di conoscermi.
\par 35 Poi disse loro: Quando vi mandai senza borsa, senza sacca da viaggio e senza calzari, vi mancò mai niente? Ed essi risposero: Niente. Ed egli disse loro:
\par 36 Ma ora, chi ha una borsa la prenda; e parimente una sacca; e chi non ha spada, venda il mantello e ne compri una.
\par 37 Poiché io vi dico che questo che è scritto deve esser adempito in me: Ed egli è stato annoverato tra i malfattori. Infatti, le cose che si riferiscono a me stanno per compiersi.
\par 38 Ed essi dissero: Signore, ecco qui due spade! Ma egli disse loro: Basta!
\par 39 Poi, essendo uscito, andò, secondo il suo solito, al monte degli Ulivi; e anche i discepoli lo seguirono.
\par 40 E giunto che fu sul luogo, disse loro: Pregate, chiedendo di non entrare in tentazione.
\par 41 Ed egli si staccò da loro circa un tiro di sasso; e postosi in ginocchio pregava, dicendo:
\par 42 Padre, se tu vuoi, allontana da me questo calice! Però, non la mia volontà, ma la tua sia fatta.
\par 43 E un angelo gli apparve dal cielo a confortarlo.
\par 44 Ed essendo in agonia, egli pregava vie più intensamente; e il suo sudore divenne come grosse gocce di sangue che cadeano in terra.
\par 45 E alzatosi dall'orazione, venne ai discepoli e li trovò che dormivano di tristezza,
\par 46 e disse loro: Perché dormite? Alzatevi e pregate, affinché non entriate in tentazione.
\par 47 Mentre parlava ancora, ecco una turba; e colui che si chiamava Giuda, uno dei dodici, la precedeva, e si accostò a Gesù per baciarlo.
\par 48 Ma Gesù gli disse: Giuda, tradisci tu il Figliuol dell'uomo con un bacio?
\par 49 E quelli ch'eran con lui, vedendo quel che stava per succedere, dissero: Signore, percoterem noi con la spada?
\par 50 E uno di loro percosse il servitore del sommo sacerdote, e gli spiccò l'orecchio destro.
\par 51 Ma Gesù rivolse loro la parola e disse: Lasciate, basta! E toccato l'orecchio di colui, lo guarì.
\par 52 E Gesù disse ai capi sacerdoti e ai capitani del tempio e agli anziani che eran venuti contro a lui: Voi siete usciti con spade e bastoni, come contro a un ladrone;
\par 53 mentre ero ogni giorno con voi nel tempio, non mi avete mai messe le mani addosso; ma questa è l'ora vostra e la potestà delle tenebre.
\par 54 E presolo, lo menaron via e lo condussero dentro la casa del sommo sacerdote; e Pietro seguiva da lontano.
\par 55 E avendo essi acceso un fuoco in mezzo alla corte ed essendosi posti a sedere insieme, Pietro si sedette in mezzo a loro.
\par 56 E una certa serva, vedutolo sedere presso il fuoco, e avendolo guardato fisso, disse: Anche costui era con lui.
\par 57 Ma egli negò, dicendo: Donna, io non lo conosco.
\par 58 E poco dopo, un altro, vedutolo, disse: Anche tu sei di quelli. Ma Pietro rispose: O uomo, non lo sono.
\par 59 E trascorsa circa un'ora, un altro affermava lo stesso, dicendo: Certo, anche costui era con lui, poich'egli è Galileo.
\par 60 Ma Pietro disse: O uomo, io non so quel che tu ti dica. E subito, mentr'egli parlava ancora, il gallo cantò.
\par 61 E il Signore, voltatosi, riguardò Pietro; e Pietro si ricordò della parola del Signore com'ei gli avea detto: Prima che il gallo canti oggi, tu mi rinnegherai tre volte.
\par 62 E uscito fuori pianse amaramente.
\par 63 E gli uomini che tenevano Gesù, lo schernivano percuotendolo;
\par 64 e avendolo bendato gli domandavano: Indovina, profeta, chi t'ha percosso?
\par 65 E molte altre cose dicevano contro a lui, bestemmiando.
\par 66 E come fu giorno, gli anziani del popolo, i capi sacerdoti e gli scribi si radunarono, e lo menarono nel loro Sinedrio, dicendo:
\par 67 Se tu sei il Cristo, diccelo. Ma egli disse loro: Se ve lo dicessi, non credereste;
\par 68 e se io vi facessi delle domande, non rispondereste.
\par 69 Ma da ora innanzi il Figliuol dell'uomo sarà seduto alla destra della potenza di Dio.
\par 70 E tutti dissero: Sei tu dunque il Figliuol di Dio? Ed egli rispose loro: Voi lo dite, poiché io lo sono.
\par 71 E quelli dissero: Che bisogno abbiamo ancora di testimonianza? Noi stessi l'abbiamo udito dalla sua propria bocca.

\chapter{23}

\par 1 Poi, levatasi tutta l'assemblea, lo menarono a Pilato.
\par 2 E cominciarono ad accusarlo, dicendo: Abbiam trovato costui che sovvertiva la nostra nazione e che vietava di pagare i tributi a Cesare, e diceva d'esser lui il Cristo re.
\par 3 E Pilato lo interrogò, dicendo: Sei tu il re dei Giudei? Ed egli, rispondendo, gli disse: Sì, lo sono.
\par 4 E Pilato disse ai capi sacerdoti e alle turbe: Io non trovo colpa alcuna in quest'uomo.
\par 5 Ma essi insistevano, dicendo: Egli solleva il popolo insegnando per tutta la Giudea; ha cominciato dalla Galilea ed è giunto fin qui.
\par 6 Quando Pilato udì questo, domandò se quell'uomo fosse Galileo.
\par 7 E saputo ch'egli era della giurisdizione d'Erode, lo rimandò a Erode ch'era anch'egli a Gerusalemme in que' giorni.
\par 8 Erode, come vide Gesù, se ne rallegrò grandemente, perché da lungo tempo desiderava vederlo, avendo sentito parlar di lui; e sperava di vedergli fare qualche miracolo.
\par 9 E gli rivolse molte domande, ma Gesù non gli rispose nulla.
\par 10 Or i capi sacerdoti e gli scribi stavan là, accusandolo con veemenza.
\par 11 Ed Erode co' suoi soldati, dopo averlo vilipeso e schernito, lo vestì di un manto splendido, e lo rimandò a Pilato.
\par 12 E in quel giorno, Erode e Pilato divennero amici, perché per l'addietro erano stati in inimicizia fra loro.
\par 13 E Pilato, chiamati assieme i capi sacerdoti e i magistrati e il popolo, disse loro:
\par 14 Voi mi avete fatto comparir dinanzi quest'uomo come sovvertitore del popolo; ed ecco, dopo averlo in presenza vostra esaminato, non ho trovato in lui alcuna delle colpe di cui l'accusate;
\par 15 e neppure Erode, poiché egli l'ha rimandato a noi; ed ecco, egli non ha fatto nulla che sia degno di morte.
\par 16 Io dunque, dopo averlo castigato, lo libererò.
\par 17 tex
\par 18 Ma essi gridarono tutti insieme: Fa' morir costui, e liberaci Barabba!
\par 19 (Barabba era stato messo in prigione a motivo di una sedizione avvenuta in città e di un omicidio).
\par 20 E Pilato da capo parlò loro, desiderando liberar Gesù;
\par 21 ma essi gridavano: Crocifiggilo, crocifiggilo!
\par 22 E per la terza volta egli disse loro: Ma che male ha egli fatto? Io non ho trovato nulla in lui, che meriti la morte. Io dunque, dopo averlo castigato, lo libererò.
\par 23 Ma essi insistevano con gran grida, chiedendo che fosse crocifisso; e le loro grida finirono con avere il sopravvento.
\par 24 E Pilato sentenziò che fosse fatto quello che domandavano.
\par 25 E liberò colui che era stato messo in prigione per sedizione ed omicidio, e che essi aveano richiesto; ma abbandonò Gesù alla loro volontà.
\par 26 E mentre lo menavan via, presero un certo Simon, cireneo, che veniva dalla campagna, e gli misero addosso la croce, perché la portasse dietro a Gesù.
\par 27 Or lo seguiva una gran moltitudine di popolo e di donne che facean cordoglio e lamento per lui.
\par 28 Ma Gesù, voltatosi verso di loro, disse: Figliuole di Gerusalemme, non piangete per me, ma piangete per voi stesse e per i vostri figliuoli.
\par 29 Perché ecco, vengono i giorni nei quali si dirà: Beate le sterili, e i seni che non han partorito, e le mammelle che non hanno allattato.
\par 30 Allora prenderanno a dire ai monti: Cadeteci addosso; ed ai colli: Copriteci.
\par 31 Poiché se fan queste cose al legno verde, che sarà egli fatto al secco?
\par 32 Or due altri, due malfattori, eran menati con lui per esser fatti morire.
\par 33 E quando furon giunti al luogo detto "il Teschio", crocifissero quivi lui e i malfattori, l'uno a destra e l'altro a sinistra.
\par 34 E Gesù diceva: Padre, perdona loro, perché non sanno quello che fanno. Poi, fatte delle parti delle sue vesti, trassero a sorte.
\par 35 E il popolo stava a guardare. E anche i magistrati si facean beffe di lui, dicendo: Ha salvato altri, salvi se stesso, se è il Cristo, l'Eletto di Dio!
\par 36 E i soldati pure lo schernivano, accostandosi, presentandogli dell'aceto e dicendo:
\par 37 Se tu sei il re de' Giudei, salva te stesso!
\par 38 E v'era anche questa iscrizione sopra il suo capo: QUESTO È IL RE DEI GIUDEI.
\par 39 E uno de' malfattori appesi lo ingiuriava, dicendo: Non se' tu il Cristo? Salva te stesso e noi!
\par 40 Ma l'altro, rispondendo, lo sgridava e diceva: Non hai tu nemmeno timor di Dio, tu che ti trovi nel medesimo supplizio?
\par 41 E per noi è cosa giusta, perché riceviamo la condegna pena de' nostri fatti, ma questi non ha fatto nulla di male.
\par 42 E diceva: Gesù, ricordati di me quando sarai venuto nel tuo regno!
\par 43 E Gesù gli disse: Io ti dico in verità che oggi tu sarai meco in paradiso.
\par 44 Ora era circa l'ora sesta, e si fecero tenebre per tutto il paese, fino all'ora nona, essendosi oscurato il sole.
\par 45 La cortina del tempio si squarciò pel mezzo.
\par 46 E Gesù, gridando con gran voce, disse: Padre, nelle tue mani rimetto lo spirito mio. E detto questo spirò.
\par 47 E il centurione, veduto ciò che era accaduto, glorificava Iddio dicendo: Veramente, quest'uomo era giusto.
\par 48 E tutte le turbe che si erano raunate a questo spettacolo, vedute le cose che erano successe, se ne tornavano battendosi il petto.
\par 49 Ma tutti i suoi conoscenti e le donne che lo avevano accompagnato dalla Galilea, stavano a guardare queste cose da lontano.
\par 50 Ed ecco un uomo per nome Giuseppe, che era consigliere, uomo dabbene e giusto,
\par 51 il quale non avea consentito alla deliberazione e all'operato degli altri, ed era da Arimatea, città de' Giudei, e aspettava il regno di Dio,
\par 52 venne a Pilato e chiese il corpo di Gesù.
\par 53 E trattolo giù di croce, lo involse in un panno lino e lo pose in una tomba scavata nella roccia, dove niuno era ancora stato posto.
\par 54 Era il giorno della Preparazione, e stava per cominciare il sabato.
\par 55 E le donne che eran venute con Gesù dalla Galilea, avendo seguito Giuseppe, guardarono la tomba, e come v'era stato posto il corpo di Gesù.
\par 56 Poi, essendosene tornate, prepararono aromi ed oli odoriferi.

\chapter{24}

\par 1 Durante il sabato si riposarono, secondo il comandamento; ma il primo giorno della settimana, la mattina molto per tempo, esse si recarono al sepolcro, portando gli aromi che aveano preparato.
\par 2 E trovarono la pietra rotolata dal sepolcro.
\par 3 Ma essendo entrate, non trovarono il corpo del Signor Gesù.
\par 4 Ed avvenne che mentre se ne stavano perplesse di ciò, ecco che apparvero dinanzi a loro due uomini in vesti sfolgoranti;
\par 5 ed essendo esse impaurite, e chinando il viso a terra, essi dissero loro: Perché cercate il vivente fra i morti?
\par 6 Egli non è qui, ma è risuscitato; ricordatevi com'egli vi parlò quand'era ancora in Galilea,
\par 7 dicendo che il Figliuol dell'uomo doveva esser dato nelle mani d'uomini peccatori ed esser crocifisso, e il terzo giorno risuscitare.
\par 8 Ed esse si ricordarono delle sue parole;
\par 9 e tornate dal sepolcro, annunziarono tutte queste cose agli undici e a tutti gli altri.
\par 10 Or quelle che dissero queste cose agli apostoli erano: Maria Maddalena, Giovanna, Maria madre di Giacomo, e le altre donne che eran con loro.
\par 11 E quelle parole parvero loro un vaneggiare, e non prestaron fede alle donne.
\par 12 Ma Pietro, levatosi, corse al sepolcro; ed essendosi chinato a guardare, vide le sole lenzuola; e se ne andò maravigliandosi fra se stesso di quel che era avvenuto.
\par 13 Ed ecco, due di loro se ne andavano in quello stesso giorno a un villaggio nominato Emmaus, distante da Gerusalemme sessanta stadi;
\par 14 e discorrevano tra loro di tutte le cose che erano accadute.
\par 15 Ed avvenne che mentre discorrevano e discutevano insieme, Gesù stesso si accostò e cominciò a camminare con loro.
\par 16 Ma gli occhi loro erano impediti così da non riconoscerlo.
\par 17 Ed egli domandò loro: Che discorsi son questi che tenete fra voi cammin facendo? Ed essi si fermarono tutti mesti.
\par 18 E l'un de' due, per nome Cleopa, rispondendo, gli disse: Tu solo, tra i forestieri, stando in Gerusalemme, non hai saputo le cose che sono in essa avvenute in questi giorni?
\par 19 Ed egli disse loro: Quali? Ed essi gli risposero: Il fatto di Gesù Nazareno, che era un profeta potente in opere e in parole dinanzi a Dio e a tutto il popolo;
\par 20 e come i capi sacerdoti e i nostri magistrati l'hanno fatto condannare a morte, e l'hanno crocifisso.
\par 21 Or noi speravamo che fosse lui che avrebbe riscattato Israele; invece, con tutto ciò, ecco il terzo giorno da che queste cose sono avvenute.
\par 22 Vero è che certe donne d'infra noi ci hanno fatto stupire; essendo andate la mattina di buon'ora al sepolcro,
\par 23 e non avendo trovato il corpo di lui, son venute dicendo d'aver avuto anche una visione d'angeli, i quali dicono ch'egli vive.
\par 24 E alcuni de' nostri sono andati al sepolcro, e hanno trovato la cosa così come aveano detto le donne; ma lui non l'hanno veduto.
\par 25 Allora Gesù disse loro: O insensati e tardi di cuore a credere a tutte le cose che i profeti hanno dette!
\par 26 Non bisognava egli che il Cristo soffrisse queste cose ed entrasse quindi nella sua gloria?
\par 27 E cominciando da Mosè e da tutti i profeti, spiegò loro in tutte le Scritture le cose che lo concernevano.
\par 28 E quando si furono avvicinati al villaggio dove andavano, egli fece come se volesse andar più oltre.
\par 29 Ed essi gli fecero forza, dicendo: Rimani con noi, perché si fa sera e il giorno è già declinato. Ed egli entrò per rimaner con loro.
\par 30 E quando si fu messo a tavola con loro, prese il pane, lo benedisse, e spezzatolo lo dette loro.
\par 31 E gli occhi loro furono aperti, e lo riconobbero; ma egli sparì d'innanzi a loro.
\par 32 Ed essi dissero l'uno all'altro: Non ardeva il cuor nostro in noi mentr'egli ci parlava per la via, mentre ci spiegava le Scritture?
\par 33 E levatisi in quella stessa ora, tornarono a Gerusalemme e trovarono adunati gli undici e quelli ch'eran con loro,
\par 34 i quali dicevano: Il Signore è veramente risuscitato ed è apparso a Simone.
\par 35 Ed essi pure raccontarono le cose avvenute loro per la via, e come era stato da loro riconosciuto nello spezzare il pane.
\par 36 Or mentr'essi parlavano di queste cose, Gesù stesso comparve in mezzo a loro, e disse: Pace a voi!
\par 37 Ma essi, smarriti e impauriti, pensavano di vedere uno spirito.
\par 38 Ed egli disse loro: Perché siete turbati? E perché vi sorgono in cuore tali pensieri?
\par 39 Guardate le mie mani ed i miei piedi, perché son ben io; palpatemi e guardate; perché uno spirito non ha carne e ossa come vedete che ho io.
\par 40 E detto questo, mostrò loro le mani e i piedi.
\par 41 Ma siccome per l'allegrezza non credevano ancora, e si stupivano, disse loro: Avete qui nulla da mangiare?
\par 42 Essi gli porsero un pezzo di pesce arrostito;
\par 43 ed egli lo prese, e mangiò in loro presenza.
\par 44 Poi disse loro: Queste son le cose che io vi dicevo quand'ero ancora con voi: che bisognava che tutte le cose scritte di me nella legge di Mosè, ne' profeti e nei Salmi, fossero adempiute.
\par 45 Allora aprì loro la mente per intendere le Scritture, e disse loro:
\par 46 Così è scritto, che il Cristo soffrirebbe, e risusciterebbe dai morti il terzo giorno,
\par 47 e che nel suo nome si predicherebbe ravvedimento e remission dei peccati a tutte le genti, cominciando da Gerusalemme.
\par 48 Or voi siete testimoni di queste cose.
\par 49 Ed ecco, io mando su voi quello che il Padre mio ha promesso; quant'è a voi, rimanete in questa città, finché dall'alto siate rivestiti di potenza.
\par 50 Poi li condusse fuori fino presso Betania; e levate in alto le mani, li benedisse.
\par 51 E avvenne che mentre li benediceva, si dipartì da loro e fu portato su nel cielo.
\par 52 Ed essi, adoratolo, tornarono a Gerusalemme con grande allegrezza;
\par 53 ed erano del continuo nel tempio, benedicendo Iddio.


\end{document}