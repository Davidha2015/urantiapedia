\begin{document}

\title{Atti degli Apostoli}


\chapter{1}

\par 1 Nel mio primo libro, o Teofilo, parlai di tutto quel che Gesù prese e a fare e ad insegnare,
\par 2 fino al giorno che fu assunto in cielo, dopo aver dato per lo Spirito Santo dei comandamenti agli apostoli che avea scelto.
\par 3 Ai quali anche, dopo ch'ebbe sofferto, si presentò vivente con molte prove, facendosi veder da loro per quaranta giorni, e ragionando delle cose relative al regno di Dio.
\par 4 E trovandosi con essi, ordinò loro di non dipartirsi da Gerusalemme, ma di aspettarvi il compimento della promessa del Padre, la quale, egli disse, avete udita da me.
\par 5 Poiché Giovanni battezzò sì con acqua, ma voi sarete battezzati con lo Spirito Santo tra non molti giorni.
\par 6 Quelli dunque che erano raunati, gli domandarono: Signore, è egli in questo tempo che ristabilirai il regno ad Israele?
\par 7 Egli rispose loro: Non sta a voi di sapere i tempi o i momenti che il Padre ha riserbato alla sua propria autorità.
\par 8 Ma voi riceverete potenza quando lo Spirito Santo verrà su voi, e mi sarete testimoni e in Gerusalemme, e in tutta la Giudea e Samaria, e fino all'estremità della terra.
\par 9 E dette queste cose, mentr'essi guardavano, fu elevato; e una nuvola, accogliendolo, lo tolse d'innanzi agli occhi loro.
\par 10 E come essi aveano gli occhi fissi in cielo, mentr'egli se ne andava, ecco che due uomini in vesti bianche si presentarono loro e dissero:
\par 11 Uomini Galilei, perché state a guardare verso il cielo? Questo Gesù che è stato tolto da voi ed assunto dal cielo, verrà nella medesima maniera che l'avete veduto andare in cielo.
\par 12 Allora essi tornarono a Gerusalemme dal monte chiamato dell'Uliveto, il quale è vicino a Gerusalemme, non distandone che un cammin di sabato.
\par 13 E come furono entrati, salirono nella sala di sopra ove solevano trattenersi Pietro e Giovanni e Giacomo e Andrea, Filippo e Toma, Bartolomeo e Matteo, Giacomo d'Alfeo, e Simone lo Zelota, e Giuda di Giacomo.
\par 14 Tutti costoro perseveravano di pari consentimento nella preghiera, con le donne, e con Maria, madre di Gesù, e coi fratelli di lui.
\par 15 E in que' giorni, Pietro, levatosi in mezzo ai fratelli (il numero delle persone adunate saliva a circa centoventi), disse:
\par 16 Fratelli, bisognava che si adempisse la profezia della Scrittura pronunziata dallo Spirito Santo per bocca di Davide intorno a Giuda, che fu la guida di quelli che arrestarono Gesù.
\par 17 Poiché egli era annoverato fra noi, e avea ricevuto la sua parte di questo ministerio.
\par 18 Costui dunque acquistò un campo col prezzo della sua iniquità; ed essendosi precipitato, gli si squarciò il ventre, e tutte le sue interiora si sparsero.
\par 19 E ciò è divenuto così noto a tutti gli abitanti di Gerusalemme, che quel campo è stato chiamato nel loro proprio linguaggio Acheldama, cioè, Campo di sangue.
\par 20 Poiché è scritto nel libro dei Salmi: Divenga la sua dimora deserta, e non vi sia chi abiti in essa: e: L'ufficio suo lo prenda un altro.
\par 21 Bisogna dunque che fra gli uomini che sono stati in nostra compagnia tutto il tempo che il Signor Gesù è andato e venuto fra noi,
\par 22 a cominciare dal battesimo di Giovanni fino al giorno ch'egli, tolto da noi, è stato assunto in cielo, uno sia fatto testimone con noi della risurrezione di lui.
\par 23 E ne presentarono due: Giuseppe, detto Barsabba, il quale era soprannominato Giusto, e Mattia.
\par 24 E, pregando, dissero: Tu, Signore, che conosci i cuori di tutti, mostra quale di questi due hai scelto
\par 25 per prendere in questo ministerio ed apostolato il posto che Giuda ha abbandonato per andarsene al suo luogo.
\par 26 E li trassero a sorte, e la sorte cadde su Mattia, che fu associato agli undici apostoli.

\chapter{2}

\par 1 E come il giorno della Pentecoste fu giunto, tutti erano insieme nel medesimo luogo.
\par 2 E di subito si fece dal cielo un suono come di vento impetuoso che soffia, ed esso riempì tutta la casa dov'essi sedevano.
\par 3 E apparvero loro delle lingue come di fuoco che si dividevano, e se ne posò una su ciascuno di loro.
\par 4 E tutti furon ripieni dello Spirito Santo, e cominciarono a parlare in altre lingue, secondo che lo Spirito dava loro d'esprimersi.
\par 5 Or in Gerusalemme si trovavan di soggiorno dei Giudei, uomini religiosi d'ogni nazione di sotto il cielo.
\par 6 Ed essendosi fatto quel suono, la moltitudine si radunò e fu confusa, perché ciascuno li udiva parlare nel suo proprio linguaggio.
\par 7 E tutti stupivano e si maravigliavano, dicendo: Ecco, tutti costoro che parlano non son eglino Galilei?
\par 8 E com'è che li udiamo parlare ciascuno nel nostro proprio natìo linguaggio?
\par 9 Noi Parti, Medi, Elamiti, abitanti della Mesopotamia, della Giudea e della Cappadocia, del Ponto e dell'Asia,
\par 10 della Frigia e della Panfilia, dell'Egitto e delle parti della Libia Cirenaica, e avventizî Romani,
\par 11 tanto Giudei che proseliti, Cretesi ed Arabi, li udiamo parlar delle cose grandi di Dio nelle nostre lingue.
\par 12 E tutti stupivano ed eran perplessi dicendosi l'uno all'altro: Che vuol esser questo?
\par 13 Ma altri, beffandosi, dicevano: Son pieni di vin dolce.
\par 14 Ma Pietro, levatosi in piè con gli undici, alzò la voce e parlò loro in questa maniera: Uomini giudei, e voi tutti che abitate in Gerusalemme, siavi noto questo, e prestate orecchio alle mie parole.
\par 15 Perché costoro non sono ebbri, come voi supponete, poiché non è che la terza ora del giorno:
\par 16 ma questo è quel che fu detto per mezzo del profeta Gioele:
\par 17 E avverrà negli ultimi giorni, dice Iddio, che io spanderò del mio Spirito sopra ogni carne; e i vostri figliuoli e le vostre figliuole profeteranno, e i vostri giovani vedranno delle visioni, e i vostri vecchi sogneranno dei sogni.
\par 18 E anche sui miei servi e sulle mie serventi, in quei giorni, spanderò del mio Spirito, e profeteranno.
\par 19 E farò prodigi su nel cielo, e segni giù sulla terra; sangue, e fuoco, e vapor di fumo.
\par 20 Il sole sarà mutato in tenebre, e la luna in sangue, prima che venga il grande e glorioso giorno, che è il giorno del Signore.
\par 21 Ed avverrà che chiunque avrà invocato il nome del Signore sarà salvato.
\par 22 Uomini israeliti, udite queste parole: Gesù il Nazareno, uomo che Dio ha accreditato fra voi mediante opere potenti e prodigî e segni che Dio fece per mezzo di lui fra voi, come voi stessi ben sapete,
\par 23 quest'uomo, allorché vi fu dato nelle mani per il determinato consiglio e per la prescienza di Dio, voi, per man d'iniqui, inchiodandolo sulla croce, lo uccideste;
\par 24 ma Dio lo risuscitò, avendo sciolto gli angosciosi legami della morte, perché non era possibile ch'egli fosse da essa ritenuto.
\par 25 Poiché Davide dice di lui: Io ho avuto del continuo il Signore davanti agli occhi, perché egli è alla mia destra, affinché io non sia smosso.
\par 26 Perciò s'è rallegrato il cuor mio, e ha giubilato la mia lingua, e anche la mia carne riposerà in isperanza;
\par 27 poiché tu non lascerai l'anima mia nell'Ades, e non permetterai che il tuo Santo vegga la corruzione.
\par 28 Tu m'hai fatto conoscere le vie della vita; tu mi riempirai di letizia con la tua presenza.
\par 29 Uomini fratelli, ben può liberamente dirvisi intorno al patriarca Davide, ch'egli morì e fu sepolto; e la sua tomba è ancora al dì d'oggi fra noi.
\par 30 Egli dunque, essendo profeta e sapendo che Dio gli avea con giuramento promesso che sul suo trono avrebbe fatto sedere uno dei suoi discendenti,
\par 31 antivedendola, parlò della risurrezione di Cristo, dicendo che non sarebbe stato lasciato nell'Ades, e che la sua carne non avrebbe veduto la corruzione.
\par 32 Questo Gesù, Iddio l'ha risuscitato; del che noi tutti siamo testimoni.
\par 33 Egli dunque, essendo stato esaltato dalla destra di Dio, e avendo ricevuto dal Padre lo Spirito Santo promesso, ha sparso quello che ora vedete e udite.
\par 34 Poiché Davide non è salito in cielo; anzi egli stesso dice: Il Signore ha detto al mio Signore: Siedi alla mia destra,
\par 35 finché io abbia posto i tuoi nemici per sgabello de' tuoi piedi.
\par 36 Sappia dunque sicuramente tutta la casa d'Israele che Iddio ha fatto e Signore e Cristo quel Gesù che voi avete crocifisso.
\par 37 Or essi, udite queste cose, furon compunti nel cuore, e dissero a Pietro e agli altri apostoli: Fratelli, che dobbiam fare?
\par 38 E Pietro a loro: Ravvedetevi, e ciascun di voi sia battezzato nel nome di Gesù Cristo, per la remission de' vostri peccati, e voi riceverete il dono dello Spirito Santo.
\par 39 Poiché per voi è la promessa, e per i vostri figliuoli, e per tutti quelli che son lontani, per quanti il Signore Iddio nostro ne chiamerà.
\par 40 E con molte altre parole li scongiurava e li esortava dicendo: Salvatevi da questa perversa generazione.
\par 41 Quelli dunque i quali accettarono la sua parola furon battezzati; e in quel giorno furono aggiunte a loro circa tremila persone.
\par 42 Ed erano perseveranti nell'attendere all'insegnamento degli apostoli, nella comunione fraterna, nel rompere il pane e nelle preghiere.
\par 43 E ogni anima era presa da timore; e molti prodigî e segni eran fatti dagli apostoli.
\par 44 E tutti quelli che credevano erano insieme, ed aveano ogni cosa in comune;
\par 45 e vendevano le possessioni ed i beni, e li distribuivano a tutti, secondo il bisogno di ciascuno.
\par 46 E tutti i giorni, essendo di pari consentimento assidui al tempio, e rompendo il pane nelle case, prendevano il loro cibo assieme con letizia e semplicità di cuore,
\par 47 lodando Iddio, e avendo il favore di tutto il popolo. E il Signore aggiungeva ogni giorno alla loro comunità quelli che erano sulla via della salvazione.

\chapter{3}

\par 1 Or Pietro e Giovanni salivano al tempio per la preghiera dell'ora nona.
\par 2 E si portava un certo uomo, zoppo fin dalla nascita, che ogni giorno deponevano alla porta del tempio detta 'Bella', per chieder l'elemosina a coloro che entravano nel tempio.
\par 3 Costui, veduto Pietro e Giovanni che stavan per entrare nel tempio, domandò loro l'elemosina.
\par 4 E Pietro, con Giovanni, fissando gli occhi su lui, disse: Guarda noi!
\par 5 Ed egli li guardava intentamente, aspettando di ricever qualcosa da loro.
\par 6 Ma Pietro disse: Dell'argento e dell'oro io non ne ho; ma quello che ho te lo do: Nel nome di Gesù Cristo il Nazareno; cammina!
\par 7 E presolo per la man destra, lo sollevò; e in quell'istante le piante e le caviglie de' piedi gli si raffermarono.
\par 8 E d'un salto si rizzò in piè e cominciò a camminare; ed entrò con loro nel tempio, camminando, e saltando, e lodando Iddio.
\par 9 E tutto il popolo lo vide che camminava e lodava Iddio;
\par 10 e lo riconoscevano per quello che sedeva a chieder l'elemosina alla porta 'Bella' del tempio; e furono ripieni di sbigottimento e di stupore per quel che gli era avvenuto.
\par 11 E mentre colui teneva stretti a sé Pietro e Giovanni, tutto il popolo, attonito, accorse a loro al portico detto di Salomone.
\par 12 E Pietro, veduto ciò, parlò al popolo, dicendo: Uomini israeliti, perché vi maravigliate di questo? O perché fissate gli occhi su noi, come se per la nostra propria potenza o pietà avessimo fatto camminar quest'uomo?
\par 13 L'Iddio d'Abramo, d'Isacco e di Giacobbe, l'Iddio de' nostri padri ha glorificato il suo Servitore Gesù, che voi metteste in man di Pilato e rinnegaste dinanzi a lui, mentre egli avea giudicato di doverlo liberare.
\par 14 Ma voi rinnegaste il Santo ed il Giusto, e chiedeste che vi fosse concesso un omicida;
\par 15 e uccideste il Principe della vita, che Dio ha risuscitato dai morti; del che noi siamo testimoni.
\par 16 E per la fede nel suo nome, il suo nome ha raffermato quest'uomo che vedete e conoscete; ed è la fede che si ha per mezzo di lui, che gli ha dato questa perfetta guarigione in presenza di voi tutti.
\par 17 Ed ora, fratelli, io so che lo faceste per ignoranza, al pari dei vostri rettori.
\par 18 Ma quello che Dio aveva preannunziato per bocca di tutti i profeti, cioè, che il suo Cristo soffrirebbe, Egli l'ha adempiuto in questa maniera.
\par 19 Ravvedetevi dunque e convertitevi, onde i vostri peccati siano cancellati,
\par 20 affinché vengano dalla presenza del Signore dei tempi di rifrigerio e ch'Egli vi mandi il Cristo che v'è stato destinato,
\par 21 cioè Gesù, che il cielo deve tenere accolto fino ai tempi della restaurazione di tutte le cose; tempi dei quali Iddio parlò per bocca dei suoi santi profeti, che sono stati fin dal principio.
\par 22 Mosè, infatti, disse: Il Signore Iddio vi susciterà di fra i vostri fratelli un profeta come me; ascoltatelo in tutte le cose che vi dirà.
\par 23 E avverrà che ogni anima la quale non avrà ascoltato codesto profeta, sarà del tutto distrutta di fra il popolo.
\par 24 E tutti i profeti, da Samuele in poi, quanti hanno parlato, hanno anch'essi annunziato questi giorni.
\par 25 Voi siete i figliuoli de' profeti, e del patto che Dio fece coi vostri padri, dicendo ad Abramo: E nella tua progenie tutte le nazioni della terra saranno benedette.
\par 26 A voi per i primi Iddio, dopo aver suscitato il suo Servitore, l'ha mandato per benedirvi, convertendo ciascun di voi dalle sue malvagità.

\chapter{4}

\par 1 Or mentr'essi parlavano al popolo, i sacerdoti e il capitano del tempio e i Sadducei sopraggiunsero,
\par 2 essendo molto crucciati perché ammaestravano il popolo e annunziavano in Gesù la risurrezione dei morti.
\par 3 E misero loro le mani addosso, e li posero in prigione fino al giorno seguente, perché già era sera.
\par 4 Ma molti di coloro che aveano udito la Parola credettero; e il numero degli uomini salì a circa cinquemila.
\par 5 E il dì seguente, i loro capi, con gli anziani e gli scribi, si radunarono in Gerusalemme,
\par 6 con Anna, il sommo sacerdote, e Caiàfa, e Giovanni, e Alessandro e tutti quelli che erano della famiglia dei sommi sacerdoti.
\par 7 E fatti comparir quivi in mezzo Pietro e Giovanni, domandarono: Con qual potestà, o in nome di chi avete voi fatto questo?
\par 8 Allora Pietro, ripieno dello Spirito Santo, disse loro: Rettori del popolo ed anziani,
\par 9 se siamo oggi esaminati circa un beneficio fatto a un uomo infermo, per sapere com'è che quest'uomo è stato guarito,
\par 10 sia noto a tutti voi e a tutto il popolo d'Israele che ciò è stato fatto nel nome di Gesù Cristo il Nazareno, che voi avete crocifisso, e che Dio ha risuscitato dai morti; in virtù d'esso quest'uomo comparisce guarito, in presenza vostra.
\par 11 Egli è la pietra che è stata da voi edificatori sprezzata, ed è divenuta la pietra angolare.
\par 12 E in nessun altro è la salvezza; poiché non v'è sotto il cielo alcun altro nome che sia stato dato agli uomini, per il quale noi abbiamo ad esser salvati.
\par 13 Or essi, veduta la franchezza di Pietro e di Giovanni, e avendo capito che erano popolani senza istruzione, si maravigliarono e riconoscevano che erano stati con Gesù.
\par 14 E vedendo l'uomo, ch'era stato guarito, quivi presente con loro, non potevano dir nulla contro.
\par 15 Ma quand'ebbero comandato loro di uscire dal concistoro, conferiron fra loro dicendo:
\par 16 Che faremo a questi uomini? Che un evidente miracolo sia stato fatto per loro mezzo, è noto a tutti gli abitanti di Gerusalemme, e noi non lo possiamo negare.
\par 17 Ma affinché ciò non si sparga maggiormente fra il popolo, divietiam loro con minacce che non parlino più ad alcuno in questo nome.
\par 18 E avendoli chiamati, ingiunsero loro di non parlare né insegnare affatto nel nome di Gesù.
\par 19 Ma Pietro e Giovanni, rispondendo, dissero loro: Giudicate voi se è giusto nel cospetto di Dio, di ubbidire a voi anzi che a Dio.
\par 20 Poiché, quanto a noi, non possiamo non parlare delle cose che abbiam vedute e udite.
\par 21 Ed essi, minacciatili di nuovo, li lasciarono andare, non trovando nulla da poterli castigare, per cagion del popolo; perché tutti glorificavano Iddio per quel ch'era stato fatto.
\par 22 Poiché l'uomo, in cui questo miracolo della guarigione era stato compiuto, avea più di quarant'anni.
\par 23 Or essi, essendo stati rimandati, vennero ai loro, e riferirono tutte le cose che i capi sacerdoti e gli anziani aveano loro dette.
\par 24 Ed essi, uditele, alzaron di pari consentimento la voce a Dio, e dissero: Signore, tu sei Colui che ha fatto il cielo, la terra, il mare e tutte le cose che sono in essi;
\par 25 Colui che mediante lo Spirito Santo, per bocca del padre nostro e tuo servitore Davide, ha detto: Perché hanno fremuto le genti, e hanno i popoli divisate cose vane?
\par 26 I re della terra si son fatti avanti, e i principi si son raunati assieme contro al Signore, e contro al suo Unto.
\par 27 E invero in questa città, contro al tuo santo Servitore Gesù che tu hai unto, si sono raunati Erode e Ponzio Pilato, insiem coi Gentili e con tutto il popolo d'Israele,
\par 28 per far tutte le cose che la tua mano e il tuo consiglio aveano innanzi determinato che avvenissero.
\par 29 E adesso, Signore, considera le loro minacce, e concedi ai tuoi servitori di annunziar la tua parola con ogni franchezza,
\par 30 stendendo la tua mano per guarire, e perché si faccian segni e prodigî mediante il nome del tuo santo Servitore Gesù.
\par 31 E dopo ch'ebbero pregato, il luogo dov'erano raunati tremò; e furon tutti ripieni dello Spirito Santo, e annunziavano la parola di Dio con franchezza.
\par 32 E la moltitudine di coloro che aveano creduto, era d'un sol cuore e d'un'anima sola; né v'era chi dicesse sua alcuna delle cose che possedeva, ma tutto era comune tra loro.
\par 33 E gli apostoli con gran potenza rendevan testimonianza della risurrezione del Signor Gesù; e gran grazia era sopra tutti loro.
\par 34 Poiché non v'era alcun bisognoso fra loro; perché tutti coloro che possedevan poderi o case li vendevano, portavano il prezzo delle cose vendute,
\par 35 e lo mettevano ai piedi degli apostoli; poi, era distribuito a ciascuno, secondo il bisogno.
\par 36 Or Giuseppe, soprannominato dagli apostoli Barnaba (il che, interpretato, vuol dire: Figliuol di consolazione), levita, cipriota di nascita,
\par 37 avendo un campo, lo vendé e portò i danari e li mise ai piedi degli apostoli.

\chapter{5}

\par 1 Ma un certo uomo, chiamato Anania, con Saffira sua moglie, vendé un possesso,
\par 2 e tenne per sé parte del prezzo, essendone consapevole anche la moglie; e portatane una parte, la pose ai piedi degli apostoli.
\par 3 Ma Pietro disse: Anania, perché ha Satana così riempito il cuor tuo da farti mentire allo Spirito Santo e ritener parte del prezzo del podere?
\par 4 Se questo restava invenduto, non restava tuo? E una volta vendutolo, non ne era il prezzo in tuo potere? Perché ti sei messa in cuore questa cosa? Tu non hai mentito agli uomini ma a Dio.
\par 5 E Anania, udendo queste parole, cadde e spirò. E gran paura prese tutti coloro che udiron queste cose.
\par 6 E i giovani, levatisi, avvolsero il corpo, e portatolo fuori, lo seppellirono.
\par 7 Or avvenne, circa tre ore dopo, che la moglie di lui, non sapendo ciò che era avvenuto, entrò.
\par 8 E Pietro, rivolgendosi a lei: Dimmi, le disse, avete voi venduto il podere per tanto? Ed ella rispose: Sì, per tanto.
\par 9 Ma Pietro a lei: Perché vi siete accordati a tentare lo Spirito del Signore? Ecco, i piedi di quelli che hanno seppellito il tuo marito sono all'uscio e ti porteranno via.
\par 10 Ed ella in quell'istante cadde ai suoi piedi, e spirò. E i giovani, entrati, la trovarono morta; e portatala via, la seppellirono presso al suo marito.
\par 11 E gran paura ne venne alla chiesa intera e a tutti coloro che udivano queste cose.
\par 12 E molti segni e prodigî eran fatti fra il popolo per le mani degli apostoli; e tutti di pari consentimento si ritrovavano sotto il portico di Salomone.
\par 13 Ma, degli altri, nessuno ardiva unirsi a loro; il popolo però li magnificava.
\par 14 E di più in più si aggiungevano al Signore dei credenti, uomini e donne, in gran numero;
\par 15 tanto che portavano perfino gli infermi per le piazze, e li mettevano su lettucci e giacigli, affinché, quando Pietro passava, l'ombra sua almeno ne adombrasse qualcuno.
\par 16 E anche la moltitudine accorreva dalle città vicine a Gerusalemme, portando dei malati e dei tormentati da spiriti immondi; e tutti quanti eran sanati.
\par 17 Or il sommo sacerdote e tutti quelli che eran con lui, cioè la setta de' Sadducei, si levarono, pieni di invidia,
\par 18 e misero le mani sopra agli apostoli, e li gettarono nella prigione pubblica.
\par 19 Ma un angelo del Signore, nella notte, aprì le porte della prigione; e condottili fuori, disse:
\par 20 Andate, presentatevi nel tempio e quivi annunziate al popolo tutte le parole di questa Vita.
\par 21 Ed essi, avendo ciò udito, entrarono sullo schiarir del giorno nel tempio, e insegnavano. Or il sommo sacerdote e coloro che eran con lui vennero, e convocarono il Sinedrio e tutti gli anziani dei figliuoli d'Israele, e mandarono alla prigione per far menare dinanzi a loro gli apostoli.
\par 22 Ma le guardie che vi andarono non li trovarono nella prigione; e tornate, fecero il loro rapporto dicendo:
\par 23 La prigione l'abbiam trovata serrata con ogni diligenza, e le guardie in piè davanti alle porte; ma, avendo aperto, non abbiam trovato alcuno dentro.
\par 24 Quando il capitano del tempio e i capi sacerdoti udiron queste cose, erano perplessi sul conto loro, non sapendo che cosa ciò potesse essere.
\par 25 Ma sopraggiunse uno che disse loro: Ecco, gli uomini che voi metteste in prigione sono nel tempio, e stanno quivi ammaestrando il popolo.
\par 26 Allora il capitano del tempio, con le guardie, andò e li menò via, non però con violenza, perché temevano d'esser lapidati dal popolo.
\par 27 E avendoli menati, li presentarono al Sinedrio; e il sommo sacerdote li interrogò,
\par 28 dicendo: Noi vi abbiamo del tutto vietato di insegnare in cotesto nome; ed ecco, avete riempita Gerusalemme della vostra dottrina, e volete trarci addosso il sangue di cotesto uomo.
\par 29 Ma Pietro e gli altri apostoli, rispondendo, dissero: Bisogna ubbidire a Dio anziché agli uomini.
\par 30 L'Iddio de' nostri padri ha risuscitato Gesù, che voi uccideste appendendolo al legno.
\par 31 Esso ha Iddio esaltato con la sua destra, costituendolo Principe e Salvatore, per dare ravvedimento a Israele, e remission dei peccati.
\par 32 E noi siam testimoni di queste cose; e anche lo Spirito Santo, che Dio ha dato a coloro che gli ubbidiscono.
\par 33 Ma essi, udendo queste cose, fremevano d'ira, e facevan proposito d'ucciderli.
\par 34 Ma un certo Fariseo, chiamato per nome Gamaliele, dottor della legge, onorato da tutto il popolo, levatosi in piè nel Sinedrio, comandò che gli apostoli fossero per un po' messi fuori.
\par 35 Poi disse loro: Uomini Israeliti, badate bene, circa questi uomini, a quel che state per fare.
\par 36 Poiché, prima d'ora, sorse Teuda, dicendosi esser qualche gran cosa; e presso a lui si raccolsero intorno a quattrocento uomini; ed egli fu ucciso; e tutti quelli che gli aveano prestata fede, furono sbandati e ridotti a nulla.
\par 37 Dopo costui, sorse Giuda il Galileo, a' dì del censimento, e si trascinò dietro della gente; anch'egli perì, e tutti coloro che gli aveano prestata fede furon dispersi.
\par 38 E adesso io vi dico: Non vi occupate di questi uomini, e lasciateli stare; perché, se questo disegno o quest'opera è dagli uomini, sarà distrutta;
\par 39 ma se è da Dio, voi non li potrete distruggere, se non volete trovarvi a combattere anche contro Dio.
\par 40 Ed essi furon del suo parere; e chiamati agli apostoli, li batterono, e ordinarono loro di non parlare nel nome di Gesù, e li lasciarono andare.
\par 41 Ed essi se ne andarono dalla presenza del Sinedrio, rallegrandosi d'essere stati reputati degni di esser vituperati per il nome di Gesù.
\par 42 E ogni giorno, nel tempio e per le case, non ristavano d'insegnare e di annunziare la buona novella che Gesù è il Cristo.

\chapter{6}

\par 1 Or in que' giorni, moltiplicandosi il numero dei discepoli, sorse un mormorio degli Ellenisti contro gli Ebrei, perché le loro vedove erano trascurate nell'assistenza quotidiana.
\par 2 E i dodici, raunata la moltitudine dei discepoli, dissero: Non è convenevole che noi lasciamo la parola di Dio per servire alle mense.
\par 3 Perciò, fratelli, cercate di trovar fra voi sette uomini, de' quali si abbia buona testimonianza, pieni di Spirito e di sapienza, e che noi incaricheremo di quest'opera.
\par 4 Ma quant'è a noi, continueremo a dedicarci alla preghiera e al ministerio della Parola.
\par 5 E questo ragionamento piacque a tutta la moltitudine; ed elessero Stefano, uomo pieno di fede e di Spirito Santo, Filippo, Procoro, Nicanore, Timone, Parmena e Nicola, proselito di Antiochia;
\par 6 e li presentarono agli apostoli, i quali, dopo aver pregato, imposero loro le mani.
\par 7 E la parola di Dio si diffondeva, e il numero dei discepoli si moltiplicava grandemente in Gerusalemme; e anche una gran quantità di sacerdoti ubbidiva alla fede.
\par 8 Or Stefano, pieno di grazia e di potenza, faceva gran prodigî e segni fra il popolo.
\par 9 Ma alcuni della sinagoga detta dei Liberti, e de' Cirenei, e degli Alessandrini, e di quei di Cilicia e d'Asia, si levarono a disputare con Stefano;
\par 10 e non potevano resistere alla sapienza e allo Spirito con cui egli parlava.
\par 11 Allora subornarono degli uomini, che dissero: Noi l'abbiamo udito dir parole di bestemmia contro Mosè e contro Dio.
\par 12 E commossero il popolo e gli anziani e gli scribi; e venutigli addosso, lo afferrarono e lo menarono al Sinedrio;
\par 13 e presentarono dei falsi testimoni, che dicevano: Quest'uomo non cessa di proferir parole contro il luogo santo e contro la legge.
\par 14 Infatti gli abbiamo udito dire che quel Nazareno, Gesù, distruggerà questo luogo e muterà gli usi che Mosè ci ha tramandati.
\par 15 E tutti coloro che sedevano nel Sinedrio, avendo fissati in lui gli occhi, videro la sua faccia simile alla faccia d'un angelo.

\chapter{7}

\par 1 E il sommo sacerdote disse: Stanno queste cose proprio così?
\par 2 Ed egli disse: Fratelli e padri, ascoltate. L'Iddio della gloria apparve ad Abramo, nostro padre, mentr'egli era in Mesopotamia, prima che abitasse in Carran,
\par 3 e gli disse: Esci dal tuo paese e dal tuo parentado, e vieni nel paese che io ti mostrerò.
\par 4 Allora egli uscì dal paese de' Caldei, e abitò in Carran; e di là, dopo che suo padre fu morto, Iddio lo fece venire in questo paese, che ora voi abitate.
\par 5 E non gli diede alcuna eredità in esso, neppure un palmo di terra, ma gli promise di darne la possessione a lui e alla sua progenie dopo di lui, quand'egli non aveva ancora alcun figliuolo.
\par 6 E Dio parlò così: La sua progenie soggiornerà in terra straniera, e sarà ridotta in servitù e maltrattata per quattrocent'anni.
\par 7 Ma io giudicherò la nazione alla quale avranno servito, disse Iddio; e dopo questo essi partiranno e mi renderanno il loro culto in questo luogo.
\par 8 E gli dette il patto della circoncisione; e così Abramo generò Isacco, e lo circoncise l'ottavo giorno; e Isacco generò Giacobbe, e Giacobbe i dodici patriarchi.
\par 9 E i patriarchi, portando invidia a Giuseppe, lo venderono perché fosse menato in Egitto; ma Dio era con lui,
\par 10 e lo liberò da tutte le sue distrette, e gli diede grazia e sapienza davanti a Faraone, re d'Egitto, che lo costituì governatore dell'Egitto e di tutta la sua casa.
\par 11 Or sopravvenne una carestia e una gran distretta in tutto l'Egitto e in Canaan; e i nostri padri non trovavano viveri.
\par 12 Ma avendo Giacobbe udito che in Egitto v'era del grano, vi mandò una prima volta i nostri padri.
\par 13 E la seconda volta, Giuseppe fu riconosciuto dai suoi fratelli, e Faraone conobbe di che stirpe fosse Giuseppe.
\par 14 E Giuseppe mandò a chiamare Giacobbe suo padre, e tutto il suo parentado, che era di settantacinque anime.
\par 15 E Giacobbe scese in Egitto, e morirono egli e i padri nostri,
\par 16 i quali furon trasportati a Sichem, e posti nel sepolcro che Abramo avea comprato a prezzo di danaro dai figliuoli di Emmor in Sichem.
\par 17 Ma come si avvicinava il tempo della promessa che Dio aveva fatta ad Abramo, il popolo crebbe e moltiplicò in Egitto,
\par 18 finché sorse sull'Egitto un altro re, che non sapeva nulla di Giuseppe.
\par 19 Costui, procedendo con astuzia contro la nostra stirpe, trattò male i nostri padri, li costrinse ad esporre i loro piccoli fanciulli perché non vivessero.
\par 20 In quel tempo nacque Mosè, ed era divinamente bello; e fu nutrito per tre mesi in casa di suo padre;
\par 21 e quando fu esposto, la figliuola di Faraone lo raccolse e se lo allevò come figliuolo.
\par 22 E Mosè fu educato in tutta la sapienza degli Egizi ed era potente nelle sue parole ed opere.
\par 23 Ma quando fu pervenuto all'età di quarant'anni, gli venne in animo d'andare a visitare i suoi fratelli, i figliuoli d'Israele.
\par 24 E vedutone uno a cui era fatto torto, lo difese e vendicò l'oppresso, uccidendo l'Egizio.
\par 25 Or egli pensava che i suoi fratelli intenderebbero che Dio li voleva salvare per mano di lui; ma essi non l'intesero.
\par 26 E il giorno seguente egli comparve fra loro, mentre contendevano, e cercava di riconciliarli, dicendo: O uomini, voi siete fratelli, perché fate torto gli uni agli altri?
\par 27 Ma colui che facea torto al suo prossimo lo respinse, dicendo: Chi ti ha costituito rettore e giudice su noi?
\par 28 Vuoi tu uccider me come ieri uccidesti l'Egizio?
\par 29 A questa parola Mosè fuggì, e dimorò come forestiero nel paese di Madian, dove ebbe due figliuoli.
\par 30 E in capo a quarant'anni, un angelo gli apparve nel deserto del monte Sinai, nella fiamma d'un pruno ardente.
\par 31 E Mosè, veduto ciò, si maravigliò della visione; e come si accostava per osservare, si fece udire questa voce del Signore:
\par 32 Io son l'Iddio de' tuoi padri, l'Iddio d'Abramo, d'Isacco e di Giacobbe. E Mosè, tutto tremante, non ardiva osservare.
\par 33 E il Signore gli disse: Sciogliti i calzari dai piedi; perché il luogo dove stai è terra santa.
\par 34 Certo, io ho veduto l'afflizione del mio popolo che è in Egitto, e ho udito i loro sospiri, e son disceso per liberarli; or dunque vieni; io ti manderò in Egitto.
\par 35 Quel Mosè che aveano rinnegato dicendo: Chi ti ha costituito rettore e giudice? Iddio lo mandò loro come capo e come liberatore con l'aiuto dell'angelo che gli era apparito nel pruno.
\par 36 Egli li condusse fuori, avendo fatto prodigi e segni nel paese di Egitto, nel mar Rosso e nel deserto per quaranta anni.
\par 37 Questi è il Mosè che disse ai figliuoli d'Israele: Il Signore Iddio vostro vi susciterà un Profeta d'infra i vostri fratelli, come me.
\par 38 Questi è colui che nell'assemblea del deserto fu con l'angelo che gli parlava sul monte Sinai, e co' padri nostri, e che ricevette rivelazioni viventi per darcele.
\par 39 A lui i nostri padri non vollero essere ubbidienti, ma lo ripudiarono, e rivolsero i loro cuori all'Egitto,
\par 40 dicendo ad Aaronne: Facci degl'iddii che vadano davanti a noi; perché quant'è a questo Mosè che ci ha condotti fuori del paese d'Egitto, noi non sappiamo quel che ne sia avvenuto.
\par 41 E in quei giorni fecero un vitello, e offersero un sacrificio all'idolo, e si rallegrarono delle opere delle loro mani.
\par 42 Ma Dio si rivolse da loro e li abbandonò al culto dell'esercito del cielo, com'è scritto nel libro dei profeti: Casa d'Israele, mi offriste voi vittime e sacrificî durante quarant'anni nel deserto?
\par 43 Anzi, voi portaste la tenda di Moloc e la stella del dio Romfàn, immagini che voi faceste per adorarle. Perciò io vi trasporterò al di là di Babilonia.
\par 44 Il tabernacolo della testimonianza fu coi nostri padri nel deserto, come avea comandato Colui che avea detto a Mosè che lo facesse secondo il modello che avea veduto.
\par 45 E i nostri padri, guidati da Giosuè, ricevutolo, lo introdussero nel paese posseduto dalle genti che Dio scacciò d'innanzi ai nostri padri. Quivi rimase fino ai giorni di Davide,
\par 46 il quale trovò grazia nel cospetto di Dio, e chiese di preparare una dimora all'Iddio di Giacobbe.
\par 47 Ma Salomone fu quello che gli edificò una casa.
\par 48 L'Altissimo però non abita in templi fatti da man d'uomo, come dice il profeta:
\par 49 Il cielo è il mio trono, e la terra lo sgabello de' miei piedi. Qual casa mi edificherete voi? dice il Signore; o qual sarà il luogo del mio riposo?
\par 50 Non ha la mia mano fatte tutte queste cose?
\par 51 Gente di collo duro e incirconcisa di cuore e d'orecchi, voi contrastate sempre allo Spirito Santo; come fecero i padri vostri; così fate anche voi.
\par 52 Qual dei profeti non perseguitarono i padri vostri? E uccisero quelli che preannunziavano la venuta del Giusto, del quale voi ora siete stati i traditori e gli uccisori;
\par 53 voi, che avete ricevuto la legge promulgata dagli angeli, e non l'avete osservata.
\par 54 Essi, udendo queste cose, fremevan di rabbia ne' loro cuori e digrignavano i denti contro di lui.
\par 55 Ma egli, essendo pieno dello Spirito Santo, fissati gli occhi al cielo, vide la gloria di Dio e Gesù che stava alla destra di Dio,
\par 56 e disse: Ecco, io vedo i cieli aperti, e il Figliuol dell'uomo in piè alla destra di Dio.
\par 57 Ma essi, gettando di gran gridi, si turarono gli orecchi, e tutti insieme si avventarono sopra lui;
\par 58 e cacciatolo fuor della città, si diedero a lapidarlo; e i testimoni deposero le loro vesti ai piedi di un giovane, chiamato Saulo.
\par 59 E lapidavano Stefano che invocava Gesù e diceva: Signor Gesù, ricevi il mio spirito.
\par 60 Poi, postosi in ginocchio, gridò ad alta voce: Signore, non imputar loro questo peccato. E detto questo si addormentò.

\chapter{8}

\par 1 E Saulo era consenziente all'uccisione di lui. E vi fu in quel tempo una gran persecuzione contro la chiesa che era in Gerusalemme. Tutti furon dispersi per le contrade della Giudea e della Samaria, salvo gli apostoli.
\par 2 E degli uomini timorati seppellirono Stefano e fecero gran cordoglio di lui.
\par 3 Ma Saulo devastava la chiesa, entrando di casa in casa; e trattine uomini e donne, li metteva in prigione.
\par 4 Coloro dunque che erano stati dispersi se ne andarono di luogo in luogo, annunziando la Parola.
\par 5 E Filippo, disceso nella città di Samaria, vi predicò il Cristo.
\par 6 E le folle di pari consentimento prestavano attenzione alle cose dette da Filippo, udendo e vedendo i miracoli ch'egli faceva.
\par 7 Poiché gli spiriti immondi uscivano da molti che li avevano, gridando con gran voce; e molti paralitici e molti zoppi erano guariti.
\par 8 E vi fu grande allegrezza in quella città.
\par 9 Or v'era un certo uomo, chiamato Simone, che già da tempo esercitava nella città le arti magiche, e facea stupire la gente di Samaria, dandosi per un qualcosa di grande.
\par 10 Tutti, dal più piccolo al più grande, gli davano ascolto, dicendo: Costui è 'la potenza di Dio', che si chiama 'la Grande'.
\par 11 E gli davano ascolto, perché già da lungo tempo li avea fatti stupire con le sue arti magiche.
\par 12 Ma quand'ebbero creduto a Filippo che annunziava loro la buona novella relativa al regno di Dio e al nome di Gesù Cristo, furon battezzati, uomini e donne.
\par 13 E Simone credette anch'egli; ed essendo stato battezzato, stava sempre con Filippo; e vedendo i miracoli e le gran potenti opere ch'eran fatti, stupiva.
\par 14 Or gli apostoli ch'erano a Gerusalemme, avendo inteso che la Samaria avea ricevuto la parola di Dio, vi mandarono Pietro e Giovanni.
\par 15 I quali, essendo discesi là, pregarono per loro affinché ricevessero lo Spirito Santo;
\par 16 poiché non era ancora disceso sopra alcuno di loro, ma erano stati soltanto battezzati nel nome del Signor Gesù.
\par 17 Allora imposero loro le mani, ed essi ricevettero lo Spirito Santo.
\par 18 Or Simone, vedendo che per l'imposizione delle mani degli apostoli era dato lo Spirito Santo, offerse loro del danaro, dicendo:
\par 19 Date anche a me questa potestà, che colui al quale io imponga le mani riceva lo Spirito Santo.
\par 20 Ma Pietro gli disse: Vada il tuo danaro teco in perdizione, poiché hai stimato che il dono di Dio si acquisti con danaro.
\par 21 Tu, in questo, non hai parte né sorte alcuna; perché il tuo cuore non è retto dinanzi a Dio.
\par 22 Ravvediti dunque di questa tua malvagità; e prega il Signore affinché, se è possibile, ti sia perdonato il pensiero del tuo cuore.
\par 23 Poiché io ti veggo in fiele amaro e in legami di iniquità.
\par 24 E Simone, rispondendo, disse: Pregate voi il Signore per me affinché nulla di ciò che avete detto mi venga addosso.
\par 25 Essi dunque, dopo aver reso testimonianza alla parola del Signore, ed averla annunziata, se ne tornarono a Gerusalemme, evangelizzando molti villaggi dei Samaritani.
\par 26 Or un angelo del Signore parlò a Filippo, dicendo: Lèvati, e vattene dalla parte di mezzodì, sulla via che scende da Gerusalemme a Gaza. Ella è una via deserta.
\par 27 Ed egli, levatosi, andò. Ed ecco un Etiopo, un eunuco, ministro di Candace, regina degli Etiopi, il quale era sovrintendente di tutti i tesori di lei, era venuto a Gerusalemme per adorare
\par 28 e stava tornandosene, seduto sul suo carro, e leggeva il profeta Isaia.
\par 29 E lo Spirito disse a Filippo: Accostati, e raggiungi codesto carro.
\par 30 Filippo accorse, l'udì che leggeva il profeta Isaia e disse: Intendi tu le cose che leggi?
\par 31 Ed egli rispose: E come potrei intenderle, se alcuno non mi guida? E pregò Filippo che montasse e sedesse con lui.
\par 32 Or il passo della Scrittura ch'egli leggeva era questo: Egli è stato menato all'uccisione come una pecora; e come un agnello che è muto dinanzi a colui che lo tosa, così egli non ha aperto la bocca.
\par 33 Nel suo abbassamento fu tolta via la sua condanna; chi descriverà la sua generazione? Poiché la sua vita è stata tolta dalla terra.
\par 34 E l'eunuco, rivolto a Filippo, gli disse: Di chi, ti prego, dice questo il profeta? Di se stesso, oppure d'un altro?
\par 35 E Filippo prese a parlare, e cominciando da questo passo della Scrittura gli annunziò Gesù.
\par 36 E cammin facendo, giunsero a una cert'acqua. E l'eunuco disse: Ecco dell'acqua; che impedisce che io sia battezzato?
\par 37 Nam
\par 38 E comandò che il carro si fermasse; e discesero ambedue nell'acqua, Filippo e l'eunuco; e Filippo lo battezzò.
\par 39 E quando furon saliti fuori dall'acqua, lo Spirito del Signore rapì Filippo; e l'eunuco, continuando il suo cammino tutto allegro, non lo vide più.
\par 40 Poi Filippo si ritrovò in Azot; e, passando, evangelizzò tutte le città, finché venne a Cesarea.

\chapter{9}

\par 1 Or Saulo, tuttora spirante minaccia e strage contro i discepoli del Signore, venne al sommo sacerdote,
\par 2 e gli chiese delle lettere per le sinagoghe di Damasco, affinché, se ne trovasse di quelli che seguivano la nuova via, uomini e donne, li potesse menar legati a Gerusalemme.
\par 3 E mentre era in cammino, avvenne che, avvicinandosi a Damasco, di subito una luce dal cielo gli sfolgorò d'intorno.
\par 4 Ed essendo caduto in terra, udì una voce che gli diceva: Saulo, Saulo, perché mi perseguiti?
\par 5 Ed egli disse: Chi sei, Signore? E il Signore: Io son Gesù che tu perseguiti;
\par 6 ma lèvati, entra nella città e ti sarà detto ciò che devi fare.
\par 7 Or gli uomini che faceano il viaggio con lui ristettero attoniti, udendo ben la voce, ma non vedendo alcuno.
\par 8 E Saulo si levò da terra; ma quando aprì gli occhi, non vedeva nulla; e quelli, menandolo per la mano, lo condussero a Damasco.
\par 9 E rimase tre giorni senza vedere, e non mangiò né bevve.
\par 10 Or in Damasco v'era un certo discepolo, chiamato Anania; e il Signore gli disse in visione: Anania! Ed egli rispose: Eccomi, Signore.
\par 11 E il Signore a lui: Lèvati, vattene nella strada detta Diritta e cerca, in casa di Giuda, un uomo chiamato Saulo, da Tarso; poiché ecco, egli è in preghiera,
\par 12 e ha veduto un uomo, chiamato Anania, entrare e imporgli le mani perché ricuperi la vista.
\par 13 Ma Anania rispose: Signore, io ho udito dir da molti di quest'uomo, quanti mali abbia fatto ai tuoi santi in Gerusalemme.
\par 14 E qui ha potestà dai capi sacerdoti d'incatenare tutti coloro che invocano il tuo nome.
\par 15 Ma il Signore gli disse: Va', perché egli è uno strumento che ho eletto per portare il mio nome davanti ai Gentili, ed ai re, ed ai figliuoli d'Israele;
\par 16 poiché io gli mostrerò quante cose debba patire per il mio nome.
\par 17 E Anania se ne andò, ed entrò in quella casa; e avendogli imposte le mani, disse: Fratello Saulo, il Signore, cioè Gesù, che ti è apparso sulla via per la quale tu venivi, mi ha mandato perché tu ricuperi la vista e sii ripieno dello Spirito Santo.
\par 18 E in quell'istante gli caddero dagli occhi come delle scaglie, e ricuperò la vista; poi, levatosi, fu battezzato.
\par 19 E avendo preso cibo, riacquistò le forze. E Saulo rimase alcuni giorni coi discepoli che erano a Damasco.
\par 20 E subito si mise a predicar nelle sinagoghe che Gesù è il Figliuol di Dio.
\par 21 E tutti coloro che l'udivano, stupivano e dicevano: Non è costui quel che in Gerusalemme infieriva contro quelli che invocano questo nome ed è venuto qui allo scopo di menarli incatenati ai capi sacerdoti?
\par 22 Ma Saulo vie più si fortificava e confondeva i Giudei che abitavano a Damasco, dimostrando che Gesù è il Cristo.
\par 23 E passati molti giorni, i Giudei si misero d'accordo per ucciderlo;
\par 24 ma il loro complotto venne a notizia di Saulo. Essi facevan perfino la guardia alle porte, giorno e notte, per ucciderlo;
\par 25 ma i discepoli, presolo di notte, lo calarono a basso giù dal muro in una cesta.
\par 26 E quando fu giunto a Gerusalemme, tentava d'unirsi ai discepoli; ma tutti lo temevano, non credendo ch'egli fosse un discepolo.
\par 27 Ma Barnaba, presolo con sé, lo menò agli apostoli, e raccontò loro come per cammino avea veduto il Signore e il Signore gli avea parlato, e come in Damasco avea predicato con franchezza nel nome di Gesù.
\par 28 Da allora, Saulo andava e veniva con loro in Gerusalemme, e predicava con franchezza nel nome del Signore;
\par 29 discorreva pure e discuteva con gli Ellenisti; ma questi cercavano d'ucciderlo.
\par 30 E i fratelli, avendolo saputo, lo condussero a Cesarea, e di là lo mandarono a Tarso.
\par 31 Così la Chiesa, per tutta la Giudea, la Galilea e la Samaria avea pace, essendo edificata; e camminando nel timor del Signore e nella consolazione dello Spirito Santo, moltiplicava.
\par 32 Or avvenne che Pietro, andando qua e là da tutti, venne anche ai santi che abitavano in Lidda.
\par 33 E quivi trovò un uomo, chiamato Enea, che già da otto anni giaceva in un lettuccio, essendo paralitico.
\par 34 E Pietro gli disse: Enea, Gesù Cristo ti sana; lèvati e rifatti il letto. Ed egli subito si levò.
\par 35 E tutti gli abitanti di Lidda e del pian di Saron lo videro e si convertirono al Signore.
\par 36 Or in Ioppe v'era una certa discepola, chiamata Tabita, il che, interpretato, vuol dire Gazzella. Costei abbondava in buone opere e faceva molte elemosine.
\par 37 E avvenne in que' giorni ch'ella infermò e morì. E dopo averla lavata, la posero in una sala di sopra.
\par 38 E perché Lidda era vicina a Ioppe, i discepoli, udito che Pietro era là, gli mandarono due uomini per pregarlo che senza indugio venisse fino a loro.
\par 39 Pietro allora, levatosi, se ne venne con loro. E come fu giunto, lo menarono nella sala di sopra; e tutte le vedove si presentarono a lui piangendo, e mostrandogli tutte le tuniche e i vestiti che Gazzella faceva, mentr'era con loro.
\par 40 Ma Pietro, messi tutti fuori, si pose in ginocchio, e pregò; e voltatosi verso il corpo, disse: Tabita, lèvati. Ed ella aprì gli occhi; e veduto Pietro, si mise a sedere.
\par 41 Ed egli le diè la mano, e la sollevò; e chiamati i santi e le vedove, la presentò loro in vita.
\par 42 E ciò fu saputo per tutta Ioppe, e molti credettero nel Signore.
\par 43 E Pietro dimorò molti giorni in Ioppe, da un certo Simone coiaio.

\chapter{10}

\par 1 Or v'era in Cesarea un uomo, chiamato Cornelio, centurione della coorte detta l''Italica',
\par 2 il quale era pio e temente Iddio con tutta la sua casa, e faceva molte elemosine al popolo e pregava Dio del continuo.
\par 3 Egli vide chiaramente in visione, verso l'ora nona del giorno, un angelo di Dio che entrò da lui e gli disse: Cornelio!
\par 4 Ed egli, guardandolo fisso, e preso da spavento, rispose: Che v'è, Signore? E l'angelo gli disse: Le tue preghiere e le tue elemosine son salite come una ricordanza davanti a Dio.
\par 5 Ed ora, manda degli uomini a Ioppe, e fa' chiamare un certo Simone, che è soprannominato Pietro.
\par 6 Egli alberga da un certo Simone coiaio, che ha la casa presso al mare.
\par 7 E come l'angelo che gli parlava se ne fu partito, Cornelio chiamò due dei suoi domestici, e un soldato pio di quelli che si tenean del continuo presso di lui;
\par 8 e raccontata loro ogni cosa, li mandò a Ioppe.
\par 9 Or il giorno seguente, mentre quelli erano in viaggio e si avvicinavano alla città, Pietro salì sul terrazzo della casa, verso l'ora sesta, per pregare.
\par 10 E avvenne ch'ebbe fame e desiderava prender cibo; e come gliene preparavano, fu rapito in estasi;
\par 11 e vide il cielo aperto, e scenderne una certa cosa, simile a un gran lenzuolo che, tenuto per i quattro capi, veniva calato in terra.
\par 12 In esso erano dei quadrupedi, dei rettili della terra e degli uccelli del cielo, di ogni specie.
\par 13 E una voce gli disse: Lèvati, Pietro; ammazza e mangia.
\par 14 Ma Pietro rispose: In niun modo, Signore, poiché io non ho mai mangiato nulla d'immondo né di contaminato.
\par 15 E una voce gli disse di nuovo la seconda volta: Le cose che Dio ha purificate, non le far tu immonde.
\par 16 E questo avvenne per tre volte; e subito il lenzuolo fu ritirato in cielo.
\par 17 E come Pietro stava perplesso in se stesso sul significato della visione avuta, ecco gli uomini mandati da Cornelio, i quali, avendo domandato della casa di Simone, si fermarono alla porta.
\par 18 E avendo chiamato, domandarono se Simone, soprannominato Pietro, albergasse lì.
\par 19 E come Pietro stava pensando alla visione, lo Spirito gli disse: Ecco tre uomini che ti cercano.
\par 20 Lèvati dunque, scendi, e va' con loro, senza fartene scrupolo, perché sono io che li ho mandati.
\par 21 E Pietro, sceso verso quegli uomini, disse loro: Ecco, io son quello che cercate: qual è la cagione per la quale siete qui?
\par 22 Ed essi risposero: Cornelio centurione, uomo giusto e temente Iddio, e del quale rende buona testimonianza tutta la nazion de' Giudei, è stato divinamente avvertito da un santo angelo, di farti chiamare in casa sua e d'ascoltar quel che avrai da dirgli.
\par 23 Allora, fattili entrare, li albergò. E il giorno seguente andò con loro; e alcuni dei fratelli di Ioppe l'accompagnarono.
\par 24 E il giorno di poi entrarono in Cesarea. Or Cornelio li stava aspettando e avea chiamato i suoi parenti e i suoi intimi amici.
\par 25 E come Pietro entrava, Cornelio, fattoglisi incontro, gli si gittò ai piedi, e l'adorò.
\par 26 Ma Pietro lo rialzò, dicendo: Lèvati, anch'io sono uomo!
\par 27 E discorrendo con lui, entrò e trovò molti radunati quivi.
\par 28 E disse loro: Voi sapete come non sia lecito ad un Giudeo di aver relazioni con uno straniero o d'entrare da lui; ma Dio mi ha mostrato che non debbo chiamare alcun uomo immondo o contaminato.
\par 29 È per questo che, essendo stato chiamato, venni senza far obiezioni. Io vi domando dunque: Per qual cagione m'avete mandato a chiamare?
\par 30 E Cornelio disse: Sono appunto adesso quattro giorni che io stavo pregando, all'ora nona, in casa mia, quand'ecco un uomo mi si presentò davanti, in veste risplendente,
\par 31 e disse: Cornelio, la tua preghiera è stata esaudita, e le tue elemosine sono state ricordate nel cospetto di Dio.
\par 32 Manda dunque a Ioppe a far chiamare Simone, soprannominato Pietro; egli alberga in casa di Simone coiaio, presso al mare.
\par 33 Perciò, in quell'istante io mandai da te, e tu hai fatto bene a venire; ora dunque siamo tutti qui presenti davanti a Dio, per udir tutte le cose che ti sono state comandate dal Signore.
\par 34 Allora Pietro, prendendo a parlare, disse: In verità io comprendo che Dio non ha riguardo alla qualità delle persone;
\par 35 ma che in qualunque nazione, chi lo teme ed opera giustamente gli è accettevole.
\par 36 E questa è la parola ch'Egli ha diretta ai figliuoli d'Israele, annunziando pace per mezzo di Gesù Cristo. Esso è il Signore di tutti.
\par 37 Voi sapete quello che è avvenuto per tutta la Giudea cominciando dalla Galilea, dopo il battesimo predicato da Giovanni:
\par 38 vale a dire, la storia di Gesù di Nazaret; come Iddio l'ha unto di Spirito Santo e di potenza; e come egli è andato attorno facendo del bene, e guarendo tutti coloro che erano sotto il dominio del diavolo, perché Iddio era con lui.
\par 39 E noi siam testimoni di tutte le cose ch'egli ha fatte nel paese de' Giudei e in Gerusalemme; ed essi l'hanno ucciso, appendendolo ad un legno.
\par 40 Esso ha Iddio risuscitato il terzo giorno, e ha fatto sì ch'egli si manifestasse
\par 41 non a tutto il popolo, ma ai testimoni ch'erano prima stati scelti da Dio; cioè a noi, che abbiamo mangiato e bevuto con lui dopo la sua risurrezione dai morti.
\par 42 Ed egli ci ha comandato di predicare al popolo e di testimoniare ch'egli è quello che da Dio è stato costituito Giudice dei vivi e dei morti.
\par 43 Di lui attestano tutti i profeti che chiunque crede in lui riceve la remission de' peccati mediante il suo nome.
\par 44 Mentre Pietro parlava così, lo Spirito Santo cadde su tutti coloro che udivano la Parola.
\par 45 E tutti i credenti circoncisi che erano venuti con Pietro, rimasero stupiti che il dono dello Spirito Santo fosse sparso anche sui Gentili;
\par 46 poiché li udivano parlare in altre lingue, e magnificare Iddio.
\par 47 Allora Pietro prese a dire: Può alcuno vietar l'acqua perché non siano battezzati questi che hanno ricevuto lo Spirito Santo come noi stessi?
\par 48 E comandò che fossero battezzati nel nome di Gesù Cristo. Allora essi lo pregarono di rimanere alcuni giorni con loro.

\chapter{11}

\par 1 Or gli apostoli e i fratelli che erano per la Giudea, intesero che i Gentili aveano anch'essi ricevuto la parola di Dio.
\par 2 E quando Pietro fu salito a Gerusalemme, quelli della circoncisione questionavano con lui, dicendo:
\par 3 Tu sei entrato da uomini incirconcisi, e hai mangiato con loro.
\par 4 Ma Pietro prese a raccontar loro le cose per ordine fin dal principio, dicendo:
\par 5 Io ero nella città di Ioppe in preghiera, ed in un'estasi, ebbi una visione; una certa cosa simile a un gran lenzuolo tenuto per i quattro capi, scendeva giù dal cielo, e veniva fino a me;
\par 6 ed io fissatolo, lo considerai bene, e vidi i quadrupedi della terra, le fiere, i rettili, e gli uccelli del cielo.
\par 7 E udii anche una voce che mi diceva: Pietro, lèvati, ammazza e mangia.
\par 8 Ma io dissi: In niun modo, Signore; poiché nulla d'immondo o di contaminato mi è mai entrato in bocca.
\par 9 Ma una voce mi rispose per la seconda volta dal cielo: Le cose che Dio ha purificate, non le far tu immonde.
\par 10 E ciò avvenne per tre volte; poi ogni cosa fu ritirata in cielo.
\par 11 Ed ecco che in quell'istante tre uomini, mandatimi da Cesarea, si presentarono alla casa dov'eravamo.
\par 12 E lo Spirito mi disse che andassi con loro, senza farmene scrupolo. Or anche questi sei fratelli vennero meco, ed entrammo in casa di quell'uomo.
\par 13 Ed egli ci raccontò come avea veduto l'angelo che si era presentato in casa sua e gli avea detto: Manda a Ioppe, e fa' chiamare Simone, soprannominato Pietro;
\par 14 il quale ti parlerà di cose, per le quali sarai salvato tu e tutta la casa tua.
\par 15 E come avevo cominciato a parlare, lo Spirito Santo scese su loro, com'era sceso su noi da principio.
\par 16 Mi ricordai allora della parola del Signore, che diceva: 'Giovanni ha battezzato con acqua, ma voi sarete battezzati con lo Spirito Santo'.
\par 17 Se dunque Iddio ha dato a loro lo stesso dono che ha dato anche a noi che abbiam creduto nel Signor Gesù Cristo, chi ero io da potermi opporre a Dio?
\par 18 Essi allora, udite queste cose, si acquetarono e glorificarono Iddio, dicendo: Iddio dunque ha dato il ravvedimento anche ai Gentili affinché abbiano vita.
\par 19 Quelli dunque ch'erano stati dispersi dalla persecuzione avvenuta a motivo di Stefano, passarono fino in Fenicia, in Cipro e in Antiochia, non annunziando la Parola ad alcuno, se non ai Giudei soltanto.
\par 20 Ma alcuni di loro, che erano Ciprioti e Cirenei, venuti in Antiochia, si misero a parlare anche ai Greci, annunziando il Signor Gesù.
\par 21 E la mano del Signore era con loro; e gran numero di gente, avendo creduto, si convertì al Signore.
\par 22 E la notizia del fatto venne agli orecchi della chiesa ch'era in Gerusalemme; onde mandarono Barnaba fino ad Antiochia.
\par 23 Ed esso, giunto là e veduta la grazia di Dio, si rallegrò, e li esortò tutti ad attenersi al Signore con fermo proponimento di cuore,
\par 24 poiché egli era un uomo dabbene, e pieno di Spirito Santo e di fede. E gran moltitudine fu aggiunta al Signore.
\par 25 Poi Barnaba se ne andò a Tarso, a cercar Saulo; e avendolo trovato, lo menò ad Antiochia.
\par 26 E avvenne che per lo spazio d'un anno intero parteciparono alle raunanze della chiesa, ed ammaestrarono un gran popolo; e fu in Antiochia che per la prima volta i discepoli furon chiamati Cristiani.
\par 27 Or in que' giorni, scesero de' profeti da Gerusalemme ad Antiochia.
\par 28 E un di loro, chiamato per nome Agabo, levatosi, predisse per lo Spirito che ci sarebbe stata una gran carestia per tutta la terra; ed essa ci fu sotto Claudio.
\par 29 E i discepoli determinarono di mandare, ciascuno secondo le sue facoltà, una sovvenzione ai fratelli che abitavano in Giudea,
\par 30 il che difatti fecero, mandandola agli anziani, per mano di Barnaba e di Saulo.

\chapter{12}

\par 1 Or intorno a quel tempo, il re Erode mise mano a maltrattare alcuni della chiesa;
\par 2 e fece morir per la spada Giacomo, fratello di Giovanni.
\par 3 E vedendo che ciò era grato ai Giudei, continuò e fece arrestare anche Pietro. Or erano i giorni degli azzimi.
\par 4 E presolo, lo mise in prigione, dandolo in guardia a quattro mute di soldati di quattro l'una; perché, dopo la Pasqua, voleva farlo comparire dinanzi al popolo.
\par 5 Pietro dunque era custodito nella prigione; ma fervide preghiere eran fatte dalla chiesa a Dio per lui.
\par 6 Or quando Erode stava per farlo comparire, la notte prima, Pietro stava dormendo in mezzo a due soldati, legato con due catene; e le guardie davanti alla porta custodivano la prigione.
\par 7 Ed ecco, un angelo del Signore sopraggiunse, e una luce risplendé nella cella; e l'angelo, percosso il fianco a Pietro, lo svegliò dicendo: Lèvati prestamente. E le catene gli caddero dalle mani.
\par 8 E l'angelo disse: Cingiti, e lègati i sandali. E Pietro fece così. Poi gli disse: Mettiti il mantello, e seguimi.
\par 9 Ed egli, uscito, lo seguiva, non sapendo che fosse vero quel che avveniva per mezzo dell'angelo, ma pensando di avere una visione.
\par 10 Or com'ebbero passata la prima e la seconda guardia, vennero alla porta di ferro che mette in città, la quale si aperse loro da sé; ed essendo usciti, s'inoltrarono per una strada: e in quell'istante l'angelo si partì da lui.
\par 11 E Pietro, rientrato in sé, disse: Ora conosco per certo che il Signore ha mandato il suo angelo e mi ha liberato dalla mano di Erode e da tutta l'aspettazione del popolo dei Giudei.
\par 12 E considerando la cosa, venne alla casa di Maria, madre di Giovanni soprannominato Marco, dove molti fratelli stavano raunati e pregavano.
\par 13 E avendo Pietro picchiato all'uscio del vestibolo, una serva, chiamata Rode, venne ad ascoltare;
\par 14 e riconosciuta la voce di Pietro, per l'allegrezza non aprì l'uscio, ma corse dentro ad annunziare che Pietro stava davanti alla porta.
\par 15 E quelli le dissero: Tu sei pazza! Ma ella asseverava che era così. Ed essi dicevano: È il suo angelo.
\par 16 Ma Pietro continuava a picchiare, e quand'ebbero aperto, lo videro e stupirono.
\par 17 Ma egli, fatto lor cenno con la mano che tacessero, raccontò loro in qual modo il Signore l'avea tratto fuor della prigione. Poi disse: Fate sapere queste cose a Giacomo ed ai fratelli. Ed essendo uscito, se ne andò in un altro luogo.
\par 18 Or, fattosi giorno, vi fu non piccol turbamento fra i soldati, perché non sapevano che cosa fosse avvenuto di Pietro.
\par 19 Ed Erode, cercatolo, e non avendolo trovato, esaminate le guardie, comandò che fosser menate al supplizio. Poi, sceso di Giudea a Cesarea, vi si trattenne.
\par 20 Or Erode era fortemente adirato contro i Tirî e i Sidonî; ma essi di pari consentimento si presentarono a lui; e guadagnato il favore di Blasto, ciambellano del re, chiesero pace, perché il loro paese traeva i viveri dal paese del re.
\par 21 Nel giorno fissato, Erode, indossato l'abito reale, e postosi a sedere sul trono, li arringava pubblicamente.
\par 22 E il popolo si mise a gridare: Voce d'un dio, e non d'un uomo!
\par 23 In quell'istante, un angelo del Signore lo percosse, perché non avea dato a Dio la gloria; e morì, roso dai vermi.
\par 24 Ma la parola di Dio progrediva e si spandeva di più in più.
\par 25 E Barnaba e Saulo, compiuta la loro missione, tornarono da Gerusalemme, prendendo seco Giovanni soprannominato Marco.

\chapter{13}

\par 1 Or nella chiesa d'Antiochia v'eran dei profeti e dei dottori: Barnaba, Simeone chiamato Niger, Lucio di Cirene, Manaen, fratello di latte di Erode il tetrarca, e Saulo.
\par 2 E mentre celebravano il culto del Signore e digiunavano, lo Spirito Santo disse: Mettetemi a parte Barnaba e Saulo per l'opera alla quale li ho chiamati.
\par 3 Allora, dopo aver digiunato e pregato, imposero loro le mani, e li accomiatarono.
\par 4 Essi dunque, mandati dallo Spirito Santo, scesero a Seleucia, e di là navigarono verso Cipro.
\par 5 E giunti a Salamina, annunziarono la parola di Dio nelle sinagoghe de' Giudei: e aveano seco Giovanni come aiuto.
\par 6 Poi, traversata tutta l'isola fino a Pafo, trovarono un certo mago, un falso profeta giudeo, che avea nome Bar-Gesù,
\par 7 il quale era col proconsole Sergio Paolo, uomo intelligente. Questi, chiamati a sé Barnaba e Saulo, chiese d'udir la parola di Dio.
\par 8 Ma Elima, il mago (perché così s'interpreta questo suo nome), resisteva loro, cercando di stornare il proconsole dalla fede.
\par 9 Ma Saulo, chiamato anche Paolo, pieno dello Spirito Santo, guardandolo fisso, gli disse:
\par 10 O pieno d'ogni frode e d'ogni furberia, figliuol del diavolo, nemico d'ogni giustizia, non cesserai tu di pervertir le diritte vie del Signore?
\par 11 Ed ora, ecco, la mano del Signore è sopra te, e sarai cieco, senza vedere il sole, per un certo tempo. E in quell'istante, caligine e tenebre caddero su lui; e andando qua e là cercava chi lo menasse per la mano.
\par 12 Allora il proconsole, visto quel che era accaduto, credette, essendo stupito della dottrina del Signore.
\par 13 Or Paolo e i suoi compagni, imbarcatisi a Pafo, arrivarono a Perga di Panfilia; ma Giovanni, separatosi da loro, ritornò a Gerusalemme.
\par 14 Ed essi, passando oltre Perga, giunsero ad Antiochia di Pisidia; e recatisi il sabato nella sinagoga, si posero a sedere.
\par 15 E dopo la lettura della legge e dei profeti, i capi della sinagoga mandarono a dir loro: Fratelli, se avete qualche parola d'esortazione da rivolgere al popolo, ditela.
\par 16 Allora Paolo, alzatosi, e fatto cenno con la mano, disse: Uomini israeliti, e voi che temete Iddio, udite.
\par 17 L'Iddio di questo popolo d'Israele elesse i nostri padri, e fece grande il popolo durante la sua dimora nel paese di Egitto, e con braccio levato, ne lo trasse fuori.
\par 18 E per lo spazio di circa quarant'anni, sopportò i loro modi nel deserto.
\par 19 Poi, dopo aver distrutte sette nazioni nel paese di Canaan, distribuì loro come eredità il paese di quelle.
\par 20 E dopo queste cose, per circa quattrocentocinquant'anni, diede loro de' giudici fino al profeta Samuele.
\par 21 Dopo chiesero un re; e Dio diede loro Saul, figliuolo di Chis, della tribù di Beniamino, per lo spazio di quarant'anni.
\par 22 Poi, rimossolo, suscitò loro Davide per re, al quale rese anche questa testimonianza: Io ho trovato Davide, figliuolo di Iesse, un uomo secondo il mio cuore, che eseguirà ogni mio volere.
\par 23 Dalla progenie di lui Iddio, secondo la sua promessa, ha suscitato a Israele un Salvatore nella persona di Gesù,
\par 24 avendo Giovanni, prima della venuta di lui, predicato il battesimo del ravvedimento a tutto il popolo d'Israele.
\par 25 E come Giovanni terminava la sua carriera diceva: Che credete voi che io sia? Io non sono il Messia; ma ecco, dietro a me viene uno, del quale io non son degno di sciogliere i calzari.
\par 26 Fratelli miei, figliuoli della progenie d'Abramo, e voi tutti che temete Iddio, a noi è stata mandata la parola di questa salvezza.
\par 27 Poiché gli abitanti di Gerusalemme e i loro capi, avendo disconosciuto questo Gesù e le dichiarazioni de' profeti che si leggono ogni sabato, le adempirono, condannandolo.
\par 28 E benché non trovassero in lui nulla che fosse degno di morte, chiesero a Pilato che fosse fatto morire.
\par 29 E dopo ch'ebber compiute tutte le cose che erano scritte di lui, lo trassero giù dal legno, e lo posero in un sepolcro.
\par 30 Ma Iddio lo risuscitò dai morti;
\par 31 e per molti giorni egli si fece vedere da coloro ch'eran con lui saliti dalla Galilea a Gerusalemme, i quali sono ora suoi testimoni presso il popolo.
\par 32 E noi vi rechiamo la buona novella che la promessa fatta ai padri,
\par 33 Iddio l'ha adempiuta per noi, loro figliuoli, risuscitando Gesù, siccome anche è scritto nel salmo secondo: Tu sei il mio Figliuolo, oggi Io ti ho generato.
\par 34 E siccome lo ha risuscitato dai morti per non tornar più nella corruzione, Egli ha detto così: Io vi manterrò le sacre e fedeli promesse fatte a Davide.
\par 35 Difatti egli dice anche in un altro luogo: Tu non permetterai che il tuo Santo vegga la corruzione.
\par 36 Poiché Davide, dopo aver servito al consiglio di Dio nella sua generazione, si è addormentato, ed è stato riunito coi suoi padri, e ha veduto la corruzione:
\par 37 ma colui che Dio ha risuscitato, non ha veduto la corruzione.
\par 38 Siavi dunque noto, fratelli, che per mezzo di lui v'è annunziata la remissione dei peccati;
\par 39 e per mezzo di lui, chiunque crede è giustificato di tutte le cose delle quali voi non avete potuto esser giustificati per la legge di Mosè.
\par 40 Guardate dunque che non venga su voi quello che è detto nei profeti:
\par 41 Vedete, o sprezzatori, e maravigliatevi, e dileguatevi, perché io fo un'opera ai dì vostri, un'opera che voi non credereste, se qualcuno ve la narrasse.
\par 42 Or, mentre uscivano, furon pregati di parlar di quelle medesime cose al popolo il sabato seguente.
\par 43 E dopo che la raunanza si fu sciolta, molti de' Giudei e de' proseliti pii seguiron Paolo e Barnaba; i quali, parlando loro, li persuasero a perseverare nella grazia di Dio.
\par 44 E il sabato seguente, quasi tutta la città si radunò per udir la parola di Dio.
\par 45 Ma i Giudei, vedendo le moltitudini, furon ripieni d'invidia, e bestemmiando contradicevano alle cose dette da Paolo.
\par 46 Ma Paolo e Barnaba dissero loro francamente: Era necessario che a voi per i primi si annunziasse la parola di Dio; ma poiché la respingete e non vi giudicate degni della vita eterna, ecco, noi ci volgiamo ai Gentili.
\par 47 Perché così ci ha ordinato il Signore, dicendo: Io ti ho posto per esser luce de' Gentili, affinché tu sia strumento di salvezza fino alle estremità della terra.
\par 48 E i Gentili, udendo queste cose, si rallegravano e glorificavano la parola di Dio; e tutti quelli che erano ordinati a vita eterna, credettero.
\par 49 E la parola del Signore si spandeva per tutto il paese.
\par 50 Ma i Giudei istigarono le donne pie e ragguardevoli e i principali uomini della città, e suscitarono una persecuzione contro Paolo e Barnaba, e li scacciarono dai loro confini.
\par 51 Ma essi, scossa la polvere de' loro piedi contro loro, se ne vennero ad Iconio.
\par 52 E i discepoli eran pieni d'allegrezza e di Spirito Santo.

\chapter{14}

\par 1 Or avvenne che in Iconio pure Paolo e Barnaba entrarono nella sinagoga dei Giudei e parlarono in maniera che una gran moltitudine di Giudei e dei Greci credette.
\par 2 Ma i Giudei, rimasti disubbidienti, misero su e inasprirono gli animi dei Gentili contro i fratelli.
\par 3 Essi dunque dimoraron quivi molto tempo, predicando con franchezza, fidenti nel Signore, il quale rendeva testimonianza alla parola della sua grazia, concedendo che per le lor mani si facessero segni e prodigî.
\par 4 Ma la popolazione della città era divisa; gli uni tenevano per i Giudei, e gli altri per gli apostoli.
\par 5 Ma essendo scoppiato un moto dei Gentili e dei Giudei coi loro capi, per recare ingiuria agli apostoli e lapidarli,
\par 6 questi, conosciuta la cosa, se ne fuggirono nelle città di Licaonia, Listra e Derba e nel paese d'intorno;
\par 7 e quivi si misero ad evangelizzare.
\par 8 Or in Listra c'era un certo uomo, impotente nei piedi, che stava sempre a sedere, essendo zoppo dalla nascita, e non aveva mai camminato.
\par 9 Egli udì parlare Paolo, il quale, fissati in lui gli occhi, e vedendo che avea fede da esser sanato,
\par 10 disse ad alta voce: Lèvati ritto in piè. Ed egli saltò su, e si mise a camminare.
\par 11 E le turbe, avendo veduto ciò che Paolo avea fatto, alzarono la voce, dicendo in lingua licaonica: Gli dèi hanno preso forma umana, e sono discesi fino a noi.
\par 12 E chiamavano Barnaba, Giove e Paolo, Mercurio, perché era il primo a parlare.
\par 13 E il sacerdote di Giove, il cui tempio era all'entrata della città, mandò dinanzi alle porte tori e ghirlande, e volea sacrificare con le turbe.
\par 14 Ma gli apostoli Barnaba e Paolo, udito ciò, si stracciarono i vestimenti, e saltarono in mezzo alla moltitudine, esclamando:
\par 15 Uomini, perché fate queste cose? Anche noi siamo uomini della stessa natura che voi; e vi predichiamo che da queste cose vane vi convertiate all'Iddio vivente, che ha fatto il cielo, la terra, il mare e tutte le cose che sono in essi;
\par 16 che nelle età passate ha lasciato camminare nelle loro vie tutte le nazioni,
\par 17 benché non si sia lasciato senza testimonianza, facendo del bene, mandandovi dal cielo piogge e stagioni fruttifere, dandovi cibo in abbondanza, e letizia nei vostri cuori.
\par 18 E dicendo queste cose, a mala pena trattennero le turbe dal sacrificar loro.
\par 19 Or sopraggiunsero quivi de' Giudei da Antiochia e da Iconio; i quali, avendo persuaso le turbe, lapidarono Paolo e lo trascinarono fuori dalla città, credendolo morto.
\par 20 Ma essendosi i discepoli raunati intorno a lui, egli si rialzò, ed entrò nella città; e il giorno seguente partì con Barnaba per Derba.
\par 21 E avendo evangelizzata quella città e fatti molti discepoli, se ne tornarono a Listra, a Iconio ed Antiochia,
\par 22 confermando gli animi dei discepoli, esortandoli a perseverare nella fede, e dicendo loro che dobbiamo entrare nel regno di Dio attraverso molte tribolazioni.
\par 23 E fatti eleggere per ciascuna chiesa degli anziani, dopo aver pregato e digiunato, raccomandarono i fratelli al Signore, nel quale aveano creduto.
\par 24 E traversata la Pisidia, vennero in Panfilia.
\par 25 E dopo aver annunziata la Parola in Perga, discesero ad Attalia;
\par 26 e di là navigarono verso Antiochia, di dove erano stati raccomandati alla grazia di Dio per l'opera che aveano compiuta.
\par 27 Giunti colà e raunata la chiesa, riferirono tutte le cose che Dio avea fatte per mezzo di loro, e come avea aperta la porta della fede ai Gentili.
\par 28 E stettero non poco tempo coi discepoli.

\chapter{15}

\par 1 Or alcuni, discesi dalla Giudea, insegnavano ai fratelli: Se voi non siete circoncisi secondo il rito di Mosè, non potete esser salvati.
\par 2 Ed essendo nata una non piccola dissensione e controversia fra Paolo e Barnaba, e costoro, fu deciso che Paolo, Barnaba e alcuni altri dei fratelli salissero a Gerusalemme agli apostoli ed anziani per trattar questa questione.
\par 3 Essi dunque, accompagnati per un tratto dalla chiesa, traversarono la Fenicia e la Samaria, raccontando la conversione dei Gentili; e cagionavano grande allegrezza a tutti i fratelli.
\par 4 Poi, giunti a Gerusalemme, furono accolti dalla chiesa, dagli apostoli e dagli anziani, e riferirono quanto grandi cose Dio avea fatte con loro.
\par 5 Ma alcuni della setta de' Farisei che aveano creduto, si levarono dicendo: Bisogna circoncidere i Gentili, e comandar loro d'osservare la legge di Mosè.
\par 6 Allora gli apostoli e gli anziani si raunarono per esaminar la questione.
\par 7 Ed essendone nata una gran discussione, Pietro si levò in piè, e disse loro: Fratelli, voi sapete che fin dai primi giorni Iddio scelse fra voi me, affinché dalla bocca mia i Gentili udissero la parola del Vangelo e credessero.
\par 8 E Dio, conoscitore dei cuori, rese loro testimonianza, dando lo Spirito Santo a loro, come a noi;
\par 9 e non fece alcuna differenza fra noi e loro, purificando i cuori loro mediante la fede.
\par 10 Perché dunque tentate adesso Iddio mettendo sul collo de' discepoli un giogo che né i padri nostri né noi abbiam potuto portare?
\par 11 Anzi, noi crediamo d'esser salvati per la grazia del Signore Gesù, nello stesso modo che loro.
\par 12 E tutta la moltitudine si tacque; e stavano ad ascoltar Barnaba e Paolo, che narravano quali segni e prodigî Iddio aveva fatto per mezzo di loro fra i Gentili.
\par 13 E quando si furon taciuti, Giacomo prese a dire:
\par 14 Fratelli, ascoltatemi. Simone ha narrato come Dio ha primieramente visitato i Gentili, per trarre da questi un popolo per il suo nome.
\par 15 E con ciò s'accordano le parole de' profeti, siccome è scritto:
\par 16 Dopo queste cose io tornerò e edificherò di nuovo la tenda di Davide, che è caduta; e restaurerò le sue ruine, e la rimetterò in piè,
\par 17 affinché il rimanente degli uomini e tutti i Gentili sui quali è invocato il mio nome,
\par 18 cerchino il Signore, dice il Signore che fa queste cose, le quali a lui son note ab eterno.
\par 19 Per la qual cosa io giudico che non si dia molestia a quelli dei Gentili che si convertono a Dio;
\par 20 ma che si scriva loro di astenersi dalle cose contaminate nei sacrifizî agl'idoli, dalla fornicazione, dalle cose soffocate, e dal sangue.
\par 21 Poiché Mosè fin dalle antiche generazioni ha chi lo predica in ogni città, essendo letto nelle sinagoghe ogni sabato.
\par 22 Allora parve bene agli apostoli e agli anziani con tutta la chiesa, di mandare ad Antiochia con Paolo e Barnaba, certi uomini scelti fra loro, cioè: Giuda, soprannominato Barsabba, e Sila, uomini autorevoli tra i fratelli;
\par 23 e scrissero così per loro mezzo: Gli apostoli e i fratelli anziani, ai fratelli di fra i Gentili che sono in Antiochia, in Siria ed in Cilicia, salute.
\par 24 Poiché abbiamo inteso che alcuni, partiti di fra noi, vi hanno turbato coi loro discorsi, sconvolgendo le anime vostre, benché non avessimo dato loro mandato di sorta,
\par 25 è parso bene a noi, riuniti di comune accordo, di scegliere degli uomini e di mandarveli assieme ai nostri cari Barnaba e Paolo,
\par 26 i quali hanno esposto la propria vita per il nome del Signor nostro Gesù Cristo.
\par 27 Vi abbiam dunque mandato Giuda e Sila; anch'essi vi diranno a voce le medesime cose.
\par 28 Poiché è parso bene allo Spirito Santo ed a noi di non imporvi altro peso all'infuori di queste cose, che sono necessarie;
\par 29 cioè: che v'asteniate dalle cose sacrificate agl'idoli, dal sangue, dalle cose soffocate, e dalla fornicazione; dalle quali cose ben farete a guardarvi. State sani.
\par 30 Essi dunque, dopo essere stati accomiatati, scesero ad Antiochia; e radunata la moltitudine, consegnarono la lettera.
\par 31 E quando i fratelli l'ebbero letta, si rallegrarono della consolazione che recava.
\par 32 E Giuda e Sila, anch'essi, essendo profeti, con molte parole li esortarono e li confermarono.
\par 33 E dopo che furon dimorati quivi alquanto tempo, furon dai fratelli congedati in pace perché se ne tornassero a quelli che li aveano inviati.
\par 34 ddd
\par 35 Ma Paolo e Barnaba rimasero ad Antiochia insegnando ed evangelizzando, con molti altri ancora, la parola del Signore.
\par 36 E dopo varî giorni, Paolo disse a Barnaba: Torniamo ora a visitare i fratelli in ogni città dove abbiamo annunziato la parola del Signore, per vedere come stanno.
\par 37 Barnaba voleva prender con loro anche Giovanni, detto Marco.
\par 38 Ma Paolo giudicava che non dovessero prendere a compagno colui che si era separato da loro fin dalla Panfilia, e che non era andato con loro all'opera.
\par 39 E ne nacque un'aspra contesa, tanto che si separarono; e Barnaba, preso seco Marco, navigò verso Cipro;
\par 40 ma Paolo, sceltosi Sila, partì, raccomandato dai fratelli alla grazia del Signore.
\par 41 E percorse la Siria e la Cilicia, confermando le chiese.

\chapter{16}

\par 1 E venne anche a Derba e a Listra; ed ecco, quivi era un certo discepolo, di nome Timoteo, figliuolo di una donna giudea credente, ma di padre greco.
\par 2 Di lui rendevano buona testimonianza i fratelli che erano in Listra ed in Iconio.
\par 3 Paolo volle ch'egli partisse con lui; e presolo, lo circoncise a cagion de' Giudei che erano in quei luoghi; perché tutti sapevano che il padre di lui era greco.
\par 4 E passando essi per le città, trasmisero loro, perché le osservassero, le decisioni prese dagli apostoli e dagli anziani che erano a Gerusalemme.
\par 5 Le chiese dunque erano confermate nella fede, e crescevano in numero di giorno in giorno.
\par 6 Poi traversarono la Frigia e il paese della Galazia, avendo lo Spirito Santo vietato loro d'annunziar la Parola in Asia;
\par 7 e giunti sui confini della Misia, tentarono d'andare in Bitinia; ma lo Spirito di Gesù non lo permise loro;
\par 8 e passata la Misia, discesero in Troas.
\par 9 E Paolo ebbe di notte una visione: Un uomo macedone gli stava dinanzi, e lo pregava dicendo: Passa in Macedonia e soccorrici.
\par 10 E com'egli ebbe avuta quella visione, cercammo subito di partire per la Macedonia, tenendo per certo che Dio ci avea chiamati là, ad annunziar loro l'Evangelo.
\par 11 Perciò, salpando da Troas, tirammo diritto, verso Samotracia, e il giorno seguente verso Neapoli;
\par 12 e di là ci recammo a Filippi, che è città primaria di quella parte della Macedonia, ed è colonia romana; e dimorammo in quella città alcuni giorni.
\par 13 E nel giorno di sabato andammo fuori della porta, presso al fiume, dove supponevamo fosse un luogo d'orazione; e postici a sedere, parlavamo alle donne ch'eran quivi radunate.
\par 14 E una certa donna, di nome Lidia, negoziante di porpora, della città di Tiatiri, che temeva Dio, ci stava ad ascoltare; e il Signore le aprì il cuore, per renderla attenta alle cose dette da Paolo.
\par 15 E dopo che fu battezzata con quei di casa, ci pregò dicendo: Se mi avete giudicata fedele al Signore, entrate in casa mia, e dimoratevi. E ci fece forza.
\par 16 E avvenne, come andavamo al luogo d'orazione, che incontrammo una certa serva, che avea uno spirito indovino e con l'indovinare procacciava molto guadagno ai suoi padroni.
\par 17 Costei, messasi a seguir Paolo e noi, gridava: Questi uomini son servitori dell'Iddio altissimo, e vi annunziano la via della salvezza.
\par 18 Così fece per molti giorni; ma essendone Paolo annoiato, si voltò e disse allo spirito: Io ti comando, nel nome di Gesù Cristo, che tu esca da costei. Ed esso uscì in quell'istante.
\par 19 Ma i padroni di lei, vedendo che la speranza del loro guadagno era svanita, presero Paolo e Sila, e li trassero sulla pubblica piazza davanti ai magistrati,
\par 20 e presentatili ai pretori, dissero: Questi uomini, che son Giudei, perturbano la nostra città,
\par 21 e predicano dei riti che non è lecito a noi che siam Romani né di ricevere, né di osservare.
\par 22 E la folla si levò tutta insieme contro a loro; e i pretori, strappate loro di dosso le vesti, comandarono che fossero battuti con le verghe.
\par 23 E dopo aver loro date molte battiture, li cacciarono in prigione, comandando al carceriere di custodirli sicuramente.
\par 24 Il quale, ricevuto un tal ordine, li cacciò nella prigione più interna, e serrò loro i piedi nei ceppi.
\par 25 Or sulla mezzanotte Paolo e Sila, pregando, cantavano inni a Dio; e i carcerati li ascoltavano.
\par 26 Ad un tratto si fece un gran terremoto, talché la prigione fu scossa dalle fondamenta; e in quell'istante tutte le porte si apersero, e i legami di tutti si sciolsero.
\par 27 Il carceriere, destatosi, e vedute le porte della prigione aperte, tratta la spada, stava per uccidersi, pensando che i carcerati fossero fuggiti.
\par 28 Ma Paolo gridò ad alta voce: Non ti far male alcuno, perché siam tutti qui.
\par 29 E quegli, chiesto un lume, saltò dentro, e tutto tremante si gettò ai piedi di Paolo e di Sila;
\par 30 e menatili fuori, disse: Signori, che debbo io fare per esser salvato?
\par 31 Ed essi risposero: Credi nel Signor Gesù, e sarai salvato tu e la casa tua.
\par 32 Poi annunziarono la parola del Signore a lui e a tutti coloro che erano in casa sua.
\par 33 Ed egli, presili in quell'istessa ora della notte, lavò loro le piaghe; e subito fu battezzato lui con tutti i suoi.
\par 34 E menatili su in casa sua, apparecchiò loro la tavola, e giubilava con tutta la sua casa, perché avea creduto in Dio.
\par 35 Or come fu giorno, i pretori mandarono i littori a dire: Lascia andar quegli uomini.
\par 36 E il carceriere riferì a Paolo queste parole, dicendo: I pretori hanno mandato a mettervi in libertà; or dunque uscite, e andatevene in pace.
\par 37 Ma Paolo disse loro: Dopo averci pubblicamente battuti senza essere stati condannati, noi che siam cittadini romani, ci hanno cacciato in prigione; e ora ci mandan via celatamente? No davvero! Anzi, vengano essi stessi a menarci fuori.
\par 38 E i littori riferirono queste parole ai pretori; e questi ebbero paura quando intesero che eran Romani;
\par 39 e vennero, e li pregarono di scusarli; e menatili fuori, chiesero loro d'andarsene dalla città.
\par 40 Allora essi, usciti di prigione, entrarono in casa di Lidia; e veduti i fratelli, li confortarono, e si partirono.

\chapter{17}

\par 1 Ed essendo passati per Amfipoli e per Apollonia, vennero a Tessalonica, dov'era una sinagoga dei Giudei;
\par 2 e Paolo, secondo la sua usanza, entrò da loro, e per tre sabati tenne loro ragionamenti tratti dalle Scritture,
\par 3 spiegando e dimostrando ch'era stato necessario che il Cristo soffrisse e risuscitasse dai morti; e il Cristo, egli diceva, è quel Gesù che io v'annunzio.
\par 4 E alcuni di loro furon persuasi, e si unirono a Paolo e Sila; e così fecero una gran moltitudine di Greci pii, e non poche delle donne principali.
\par 5 Ma i Giudei, mossi da invidia, presero con loro certi uomini malvagi fra la gente di piazza; e raccolta una turba, misero in tumulto la città; e, assalita la casa di Giasone, cercavano di trar Paolo e Sila fuori al popolo.
\par 6 Ma non avendoli trovati, trascinarono Giasone e alcuni de' fratelli dinanzi ai magistrati della città, gridando: Costoro che hanno messo sossopra il mondo, son venuti anche qua,
\par 7 e Giasone li ha accolti; ed essi tutti vanno contro agli statuti di Cesare, dicendo che c'è un altro re, Gesù.
\par 8 E misero sossopra la moltitudine e i magistrati della città, che udivano queste cose.
\par 9 E questi, dopo che ebbero ricevuta una cauzione da Giasone e dagli altri, li lasciarono andare.
\par 10 E i fratelli, subito, di notte, fecero partire Paolo e Sila per Berea; ed essi, giuntivi, si recarono nella sinagoga de' Giudei.
\par 11 Or questi furono più generosi di quelli di Tessalonica, in quanto che ricevettero la Parola con ogni premura, esaminando tutti i giorni le Scritture per vedere se le cose stavan così.
\par 12 Molti di loro, dunque, credettero, e non piccol numero di nobildonne greche e d'uomini.
\par 13 Ma quando i Giudei di Tessalonica ebbero inteso che la parola di Dio era stata annunziata da Paolo anche in Berea, vennero anche là, agitando e mettendo sossopra le turbe.
\par 14 E i fratelli, allora, fecero partire immediatamente Paolo, conducendolo fino al mare; e Sila e Timoteo rimasero ancora quivi.
\par 15 Ma coloro che accompagnavano Paolo, lo condussero fino ad Atene; e ricevuto l'ordine di dire a Sila e a Timoteo che quanto prima venissero a lui, si partirono.
\par 16 Or mentre Paolo li aspettava in Atene, lo spirito gli s'inacerbiva dentro a veder la città piena d'idoli.
\par 17 Egli dunque ragionava nella sinagoga coi Giudei e con le persone pie; e sulla piazza, ogni giorno, con quelli che vi si trovavano.
\par 18 E anche certi filosofi epicurei e stoici conferivan con lui. E alcuni dicevano: Che vuol dire questo cianciatore? E altri: Egli pare essere un predicatore di divinità straniere; perché annunziava Gesù e la risurrezione.
\par 19 E presolo con sé, lo condussero su nell'Areopàgo, dicendo: Potremmo noi sapere qual sia questa nuova dottrina che tu proponi?
\par 20 Poiché tu ci rechi agli orecchi delle cose strane. Noi vorremmo dunque sapere che cosa voglian dire queste cose.
\par 21 Or tutti gli Ateniesi e i forestieri che dimoravan quivi, non passavano il tempo in altro modo, che a dire o ad ascoltare quel che c'era di più nuovo.
\par 22 E Paolo, stando in piè in mezzo all'Areopàgo, disse: Ateniesi, io veggo che siete in ogni cosa quasi troppo religiosi.
\par 23 Poiché, passando, e considerando gli oggetti del vostro culto, ho trovato anche un altare sul quale era scritto: AL DIO SCONOSCIUTO. Ciò dunque che voi adorate senza conoscerlo, io ve l'annunzio.
\par 24 L'Iddio che ha fatto il mondo e tutte le cose che sono in esso, essendo Signore del cielo e della terra, non abita in templi fatti d'opera di mano;
\par 25 e non è servito da mani d'uomini; come se avesse bisogno di alcuna cosa; Egli, che dà a tutti la vita, il fiato ed ogni cosa.
\par 26 Egli ha tratto da un solo tutte le nazioni degli uomini perché abitino su tutta la faccia della terra, avendo determinato le epoche loro assegnate, e i confini della loro abitazione,
\par 27 affinché cerchino Dio, se mai giungano a trovarlo, come a tastoni, benché Egli non sia lungi da ciascun di noi.
\par 28 Difatti, in lui viviamo, ci moviamo, e siamo, come anche alcuni de' vostri poeti han detto: 'Poiché siamo anche sua progenie'.
\par 29 Essendo dunque progenie di Dio, non dobbiam credere che la Divinità sia simile ad oro, ad argento, o a pietra scolpiti dall'arte e dall'immaginazione umana.
\par 30 Iddio dunque, passando sopra ai tempi dell'ignoranza, fa ora annunziare agli uomini che tutti, per ogni dove, abbiano a ravvedersi,
\par 31 perché ha fissato un giorno, nel quale giudicherà il mondo con giustizia, per mezzo dell'uomo ch'Egli ha stabilito; del che ha fatto fede a tutti, avendolo risuscitato dai morti.
\par 32 Quando udiron mentovar la risurrezione de' morti, alcuni se ne facevano beffe; ed altri dicevano: Su questo noi ti sentiremo un'altra volta.
\par 33 Così Paolo uscì dal mezzo di loro.
\par 34 Ma alcuni si unirono a lui e credettero, fra i quali anche Dionisio l'Areopagita, una donna chiamata Damaris, e altri con loro.

\chapter{18}

\par 1 Dopo queste cose egli, partitosi da Atene, venne a Corinto.
\par 2 E trovato un certo Giudeo, per nome Aquila, oriundo del Ponto, venuto di recente dall'Italia insieme con Priscilla sua moglie, perché Claudio avea comandato che tutti i Giudei se ne andassero da Roma, s'unì a loro.
\par 3 E siccome era del medesimo mestiere, dimorava con loro, e lavoravano; poiché di mestiere, eran fabbricanti di tende.
\par 4 E ogni sabato discorreva nella sinagoga, e persuadeva Giudei e Greci.
\par 5 Ma quando Sila e Timoteo furon venuti dalla Macedonia, Paolo si diè tutto quanto alla predicazione, testimoniando ai Giudei che Gesù era il Cristo.
\par 6 Però, contrastando essi e bestemmiando, egli scosse le sue vesti e disse loro: Il vostro sangue ricada sul vostro capo; io ne son netto; da ora innanzi andrò ai Gentili.
\par 7 E partitosi di là, entrò in casa d'un tale, chiamato Tizio Giusto, il quale temeva Iddio, ed avea la casa contigua alla sinagoga.
\par 8 E Crispo, il capo della sinagoga, credette nel Signore con tutta la sua casa; e molti dei Corinzî, udendo Paolo, credevano, ed eran battezzati.
\par 9 E il Signore disse di notte in visione a Paolo: Non temere, ma parla e non tacere;
\par 10 perché io son teco, e nessuno metterà le mani su te per farti del male; poiché io ho un gran popolo in questa città.
\par 11 Ed egli dimorò quivi un anno e sei mesi, insegnando fra loro la parola di Dio.
\par 12 Poi, quando Gallione fu proconsole d'Acaia, i Giudei, tutti d'accordo, si levaron contro Paolo, e lo menarono dinanzi al tribunale, dicendo:
\par 13 Costui va persuadendo gli uomini ad adorare Iddio in modo contrario alla legge.
\par 14 E come Paolo stava per aprir bocca, Gallione disse ai Giudei: Se si trattasse di qualche ingiustizia o di qualche mala azione, o Giudei, io vi ascolterei pazientemente, come ragion vuole.
\par 15 Ma se si tratta di questioni intorno a parole, a nomi, e alla vostra legge, provvedeteci voi; io non voglio esser giudice di codeste cose.
\par 16 E li mandò via dal tribunale.
\par 17 Allora tutti, afferrato Sostène, il capo della sinagoga, lo battevano davanti al tribunale. E Gallione non si curava affatto di queste cose.
\par 18 Quanto a Paolo, ei rimase ancora molti giorni a Corinto; poi, preso commiato dai fratelli, navigò verso la Siria, con Priscilla ed Aquila, dopo essersi fatto tosare il capo a Cencrea, perché avea fatto un voto.
\par 19 Come furon giunti ad Efeso, Paolo li lasciò quivi; egli, intanto, entrato nella sinagoga, si pose a discorrere coi Giudei.
\par 20 E pregandolo essi di dimorare da loro più a lungo, non acconsentì;
\par 21 ma dopo aver preso commiato e aver detto che, Dio volendo, sarebbe tornato da loro un'altra volta, salpò da Efeso.
\par 22 E sbarcato a Cesarea, salì a Gerusalemme, e salutata la chiesa, scese ad Antiochia.
\par 23 Ed essendosi fermato quivi alquanto tempo, si partì, percorrendo di luogo in luogo il paese della Galazia e la Frigia, confermando tutti i discepoli.
\par 24 Or un certo Giudeo, per nome Apollo, oriundo d'Alessandria, uomo eloquente e potente nelle Scritture, arrivò ad Efeso.
\par 25 Egli era stato ammaestrato nella via del Signore; ed essendo fervente di spirito, parlava e insegnava accuratamente le cose relative a Gesù, benché avesse conoscenza soltanto del battesimo di Giovanni.
\par 26 Egli cominciò pure a parlar francamente nella sinagoga. Ma Priscilla ed Aquila, uditolo, lo presero seco e gli esposero più appieno la via di Dio.
\par 27 Poi, volendo egli passare in Acaia, i fratelli ve lo confortarono, e scrissero ai discepoli che l'accogliessero. Giunto là, egli fu di grande aiuto a quelli che avevan creduto mediante la grazia;
\par 28 perché con gran vigore confutava pubblicamente i Giudei, dimostrando per le Scritture che Gesù è il Cristo.

\chapter{19}

\par 1 Or avvenne, mentre Apollo era a Corinto, che Paolo, avendo traversato la parte alta del paese, venne ad Efeso; e vi trovò alcuni discepoli, ai quali disse:
\par 2 Riceveste voi lo Spirito Santo quando credeste? Ed essi a lui: Non abbiamo neppur sentito dire che ci sia lo Spirito Santo.
\par 3 Ed egli disse loro: Di che battesimo siete dunque stati battezzati? Ed essi risposero: Del battesimo di Giovanni.
\par 4 E Paolo disse: Giovanni battezzò col battesimo di ravvedimento, dicendo al popolo che credesse in colui che veniva dopo di lui, cioè, in Gesù.
\par 5 Udito questo, furon battezzati nel nome del Signor Gesù;
\par 6 e dopo che Paolo ebbe loro imposto le mani, lo Spirito Santo scese su loro, e parlavano in altre lingue, e profetizzavano.
\par 7 Erano, in tutto, circa dodici uomini.
\par 8 Poi entrò nella sinagoga, e quivi seguitò a parlare francamente per lo spazio di tre mesi, discorrendo con parole persuasive delle cose relative al regno di Dio.
\par 9 Ma siccome alcuni s'indurivano e rifiutavano di credere, dicendo male della nuova Via dinanzi alla moltitudine, egli, ritiratosi da loro, separò i discepoli, discorrendo ogni giorno nella scuola di Tiranno.
\par 10 E questo continuò due anni; talché tutti coloro che abitavano nell'Asia, Giudei e Greci, udirono la parola del Signore.
\par 11 E Iddio faceva de' miracoli straordinari per le mani di Paolo;
\par 12 al punto che si portavano sui malati degli asciugatoi e de' grembiuli che erano stati sul suo corpo, e le malattie si partivano da loro, e gli spiriti maligni se ne uscivano.
\par 13 Or alcuni degli esorcisti giudei che andavano attorno, tentarono anch'essi d'invocare il nome del Signor Gesù su quelli che aveano degli spiriti maligni, dicendo: Io vi scongiuro, per quel Gesù che Paolo predica.
\par 14 E quelli che facevan questo, eran sette figliuoli di un certo Sceva, Giudeo, capo sacerdote.
\par 15 E lo spirito maligno, rispondendo, disse loro: Gesù, lo conosco, e Paolo so chi è; ma voi chi siete?
\par 16 E l'uomo che avea lo spirito maligno si avventò su due di loro; li sopraffece, e fe' loro tal violenza, che se ne fuggirono da quella casa, nudi e feriti.
\par 17 E questo venne a notizia di tutti, Giudei e Greci, che abitavano in Efeso; e tutti furon presi da spavento, e il nome del Signor Gesù era magnificato.
\par 18 E molti di coloro che aveano creduto, venivano a confessare e a dichiarare le cose che aveano fatte.
\par 19 E buon numero di quelli che aveano esercitato le arti magiche, portarono i loro libri assieme, e li arsero in presenza di tutti; e calcolatone il prezzo, trovarono che ascendeva a cinquantamila dramme d'argento.
\par 20 Così la parola di Dio cresceva potentemente e si rafforzava.
\par 21 Compiute che furon queste cose, Paolo si mise in animo d'andare a Gerusalemme, passando per la Macedonia e per l'Acaia. Dopo che sarò stato là, diceva, bisogna ch'io veda anche Roma.
\par 22 E mandati in Macedonia due di quelli che lo aiutavano, Timoteo ed Erasto, egli si trattenne ancora in Asia per qualche tempo.
\par 23 Or in quel tempo nacque non piccol tumulto a proposito della nuova Via.
\par 24 Poiché un tale, chiamato Demetrio, orefice, che faceva de' tempietti di Diana in argento, procurava non poco guadagno agli artigiani.
\par 25 Raunati questi e gli altri che lavoravan di cotali cose, disse: Uomini, voi sapete che dall'esercizio di quest'arte viene la nostra prosperità.
\par 26 E voi vedete e udite che questo Paolo ha persuaso e sviato gran moltitudine non solo in Efeso, ma quasi in tutta l'Asia, dicendo che quelli fatti con le mani non sono dèi.
\par 27 E non solo v'è pericolo che questo ramo della nostra arte cada in discredito, ma che anche il tempio della gran dea Diana sia reputato per nulla, e che sia perfino spogliata della sua maestà colei, che tutta l'Asia e il mondo adorano.
\par 28 Ed essi, udite queste cose, accesi di sdegno, si misero a gridare: Grande è la Diana degli Efesini!
\par 29 E tutta la città fu ripiena di confusione; e traendo seco a forza Gaio e Aristarco, Macedoni, compagni di viaggio di Paolo, si precipitaron tutti d'accordo verso il teatro.
\par 30 Paolo voleva presentarsi al popolo, ma i discepoli non glielo permisero.
\par 31 E anche alcuni de' magistrati dell'Asia che gli erano amici, mandarono a pregarlo che non s'arrischiasse a venire nel teatro.
\par 32 Gli uni dunque gridavano una cosa, e gli altri un'altra, perché l'assemblea era una confusione; e i più non sapevano per qual cagione si fossero raunati.
\par 33 E di fra la moltitudine trassero Alessandro, che i Giudei spingevano innanzi. E Alessandro, fatto cenno con la mano, voleva arringare il popolo a loro difesa.
\par 34 Ma quando ebbero riconosciuto che era Giudeo, tutti, ad una voce, per circa due ore, si posero a gridare: Grande è la Diana degli Efesini!
\par 35 Ma il segretario, avendo acquetata la turba, disse: Uomini di Efeso, chi è che non sappia che la città degli Efesini è la guardiana del tempio della gran Diana e dell'immagine caduta da Giove?
\par 36 Essendo dunque queste cose fuor di contestazione, voi dovete acquetarvi e non far nulla di precipitato;
\par 37 poiché avete menato qua questi uomini, i quali non sono né sacrileghi, né bestemmiatori della nostra dea.
\par 38 Se dunque Demetrio e gli artigiani che son con lui hanno qualcosa contro qualcuno, ci sono i tribunali, e ci sono i proconsoli; si facciano citare gli uni e gli altri.
\par 39 Se poi volete ottenere qualcosa intorno ad altri affari, la questione si risolverà in un'assemblea legale.
\par 40 Perché noi siamo in pericolo d'essere accusati di sedizione per la raunata d'oggi, non essendovi ragione alcuna con la quale noi possiamo giustificare questo assembramento.
\par 41 E dette queste cose, sciolse l'adunanza.

\chapter{20}

\par 1 Or dopo che fu cessato il tumulto, Paolo, fatti chiamare i discepoli ed esortatili, li abbracciò e si partì per andare in Macedonia.
\par 2 E dopo aver traversato quelle parti, e averli con molte parole esortati, venne in Grecia.
\par 3 Quivi si fermò tre mesi; poi avendogli i Giudei teso delle insidie mentre stava per imbarcarsi per la Siria, decise di tornare per la Macedonia.
\par 4 E lo accompagnarono Sòpatro di Berea, figlio di Pirro, e i Tessalonicesi Aristarco e Secondo, e Gaio di Derba e Timoteo, e della provincia d'Asia Tichico e Trofimo.
\par 5 Costoro, andati innanzi, ci aspettarono a Troas.
\par 6 E noi, dopo i giorni degli azzimi, partimmo da Filippi, e in capo a cinque giorni li raggiungemmo a Troas, dove dimorammo sette giorni.
\par 7 E nel primo giorno della settimana, mentre eravamo radunati per rompere il pane, Paolo, dovendo partire il giorno seguente, si mise a ragionar con loro, e prolungò il suo discorso fino a mezzanotte.
\par 8 Or nella sala di sopra, dove eravamo radunati, c'erano molte lampade;
\par 9 e un certo giovinetto, chiamato Eutico, che stava seduto sul davanzale della finestra, fu preso da profondo sonno; e come Paolo tirava in lungo il suo dire, sopraffatto dal sonno, cadde giù dal terzo piano, e fu levato morto.
\par 10 Ma Paolo, sceso a basso, si buttò su di lui, e abbracciatolo, disse: Non fate tanto strepito, perché l'anima sua è in lui.
\par 11 Ed essendo risalito, ruppe il pane e prese cibo; e dopo aver ragionato lungamente sino all'alba, senz'altro si partì.
\par 12 Il ragazzo poi fu ricondotto vivo, ed essi ne furono oltre modo consolati.
\par 13 Quanto a noi, andati innanzi a bordo, navigammo verso Asso, con intenzione di prender quivi Paolo con noi; poiché egli avea fissato così, volendo fare quel tragitto per terra.
\par 14 E avendoci incontrati ad Asso, lo prendemmo con noi, e venimmo a Mitilene.
\par 15 E di là, navigando, arrivammo il giorno dopo dirimpetto a Chio; e il giorno seguente approdammo a Samo, e il giorno di poi giungemmo a Mileto.
\par 16 Poiché Paolo avea deliberato di navigare oltre Efeso, per non aver a consumar tempo in Asia; giacché si affrettava per trovarsi, se gli fosse possibile, a Gerusalemme il giorno della Pentecoste.
\par 17 E da Mileto mandò ad Efeso a far chiamare gli anziani della chiesa.
\par 18 E quando furon venuti a lui, egli disse loro: Voi sapete in qual maniera, dal primo giorno che entrai nell'Asia, io mi son sempre comportato con voi,
\par 19 servendo al Signore con ogni umiltà, e con lacrime, fra le prove venutemi dalle insidie dei Giudei;
\par 20 come io non mi son tratto indietro dall'annunziarvi e dall'insegnarvi in pubblico e per le case, cosa alcuna di quelle che vi fossero utili,
\par 21 scongiurando Giudei e Greci a ravvedersi dinanzi a Dio e a credere nel Signor nostro Gesù Cristo.
\par 22 Ed ora, ecco, vincolato nel mio spirito, io vo a Gerusalemme, non sapendo le cose che quivi mi avverranno;
\par 23 salvo che lo Spirito Santo mi attesta in ogni città che legami ed afflizioni m'aspettano.
\par 24 Ma io non fo alcun conto della vita, quasi mi fosse cara, pur di compiere il mio corso e il ministerio che ho ricevuto dal Signor Gesù, che è di testimoniare dell'Evangelo della grazia di Dio.
\par 25 Ed ora, ecco, io so che voi tutti fra i quali sono passato predicando il Regno, non vedrete più la mia faccia.
\par 26 Perciò io vi protesto quest'oggi che son netto del sangue di tutti;
\par 27 perché io non mi son tratto indietro dall'annunziarvi tutto il consiglio di Dio.
\par 28 Badate a voi stessi e a tutto il gregge, in mezzo al quale lo Spirito Santo vi ha costituiti vescovi, per pascere la chiesa di Dio, la quale egli ha acquistata col proprio sangue.
\par 29 Io so che dopo la mia partenza entreranno fra voi de' lupi rapaci, i quali non risparmieranno il gregge;
\par 30 e di fra voi stessi sorgeranno uomini che insegneranno cose perverse per trarre i discepoli dietro a sé.
\par 31 Perciò vegliate, ricordandovi che per lo spazio di tre anni, notte e giorno, non ho cessato d'ammonire ciascuno con lacrime.
\par 32 E ora, io vi raccomando a Dio e alla parola della sua grazia; a lui che può edificarvi e darvi l'eredità con tutti i santificati.
\par 33 Io non ho bramato né l'argento, né l'oro, né il vestito d'alcuno.
\par 34 Voi stessi sapete che queste mani hanno provveduto ai bisogni miei e di coloro che eran meco.
\par 35 In ogni cosa vi ho mostrato ch'egli è con l'affaticarsi così, che bisogna venire in aiuto ai deboli, e ricordarsi delle parole del Signor Gesù, il quale disse egli stesso: Più felice cosa è il dare che il ricevere.
\par 36 Quando ebbe dette queste cose, si pose in ginocchio e pregò con tutti loro.
\par 37 E si fece da tutti un gran piangere; e gettatisi al collo di Paolo, lo baciavano,
\par 38 dolenti sopra tutto per la parola che avea detta, che non vedrebbero più la sua faccia. E l'accompagnarono alla nave.

\chapter{21}

\par 1 Or dopo che ci fummo staccati da loro, salpammo, e per diritto corso giungemmo a Cos, e il giorno seguente a Rodi, e di là a Patara;
\par 2 e trovata una nave che passava in Fenicia, vi montammo su, e facemmo vela.
\par 3 Giunti in vista di Cipro, e lasciatala a sinistra, navigammo verso la Siria, e approdammo a Tiro, perché quivi si dovea scaricar la nave.
\par 4 E trovati i discepoli, dimorammo quivi sette giorni. Essi, mossi dallo Spirito, dicevano a Paolo di non metter piede in Gerusalemme;
\par 5 quando però fummo al termine di quei giorni, partimmo per continuare il viaggio, accompagnati da tutti loro, con le mogli e i figliuoli, fin fuori della città; e postici in ginocchio sul lido, facemmo orazione e ci dicemmo addio;
\par 6 poi montammo sulla nave, e quelli se ne tornarono alle case loro.
\par 7 E noi, terminando la navigazione, da Tiro arrivammo a Tolemaide; e salutati i fratelli, dimorammo un giorno con loro.
\par 8 E partiti l'indomani, giungemmo a Cesarea; ed entrati in casa di Filippo l'evangelista, ch'era uno dei sette, dimorammo con lui.
\par 9 Or egli avea quattro figliuole non maritate, le quali profetizzavano.
\par 10 Eravamo quivi da molti giorni, quando scese dalla Giudea un certo profeta, di nome Agabo,
\par 11 il quale, venuto da noi, prese la cintura di Paolo, se ne legò i piedi e le mani, e disse: Questo dice lo Spirito Santo: Così legheranno i Giudei a Gerusalemme l'uomo di cui è questa cintura, e lo metteranno nelle mani dei Gentili.
\par 12 Quando udimmo queste cose, tanto noi che quei del luogo lo pregavamo di non salire a Gerusalemme.
\par 13 Paolo allora rispose: Che fate voi, piangendo e spezzandomi il cuore? Poiché io son pronto non solo ad esser legato, ma anche a morire a Gerusalemme per il nome del Signor Gesù.
\par 14 E non lasciandosi egli persuadere, ci acquetammo, dicendo: Sia fatta la volontà del Signore.
\par 15 Dopo que' giorni, fatti i nostri preparativi, salimmo a Gerusalemme.
\par 16 E vennero con noi anche alcuni de' discepoli di Cesarea, menando seco un certo Mnasone di Cipro, antico discepolo, presso il quale dovevamo albergare.
\par 17 Quando fummo giunti a Gerusalemme, i fratelli ci accolsero lietamente.
\par 18 E il giorno seguente, Paolo si recò con noi da Giacomo; e vi si trovarono tutti gli anziani.
\par 19 Dopo averli salutati, Paolo si mise a raccontare ad una ad una le cose che Dio avea fatte fra i Gentili, per mezzo del suo ministerio.
\par 20 Ed essi, uditele, glorificavano Iddio. Poi, dissero a Paolo: Fratello, tu vedi quante migliaia di Giudei ci sono che hanno creduto; e tutti sono zelanti per la legge.
\par 21 Or sono stati informati di te, che tu insegni a tutti i Giudei che sono fra i Gentili, ad abbandonare Mosè, dicendo loro di non circoncidere i figliuoli, e di non conformarsi ai riti.
\par 22 Che devesi dunque fare? È inevitabile che una moltitudine di loro si raduni, perché udranno che tu se' venuto.
\par 23 Fa' dunque questo che ti diciamo: Noi abbiamo quattro uomini che hanno fatto un voto;
\par 24 prendili teco, e purificati con loro, e paga le spese per loro, onde possano radersi il capo; così tutti conosceranno che non c'è nulla di vero nelle informazioni che hanno ricevute di te; ma che tu pure ti comporti da osservatore della legge.
\par 25 Quanto ai Gentili che hanno creduto, noi abbiamo loro scritto, avendo deciso che debbano astenersi dalle cose sacrificate agl'idoli, dal sangue, dalle cose soffocate, e dalla fornicazione.
\par 26 Allora Paolo, il giorno seguente, prese seco quegli uomini, e dopo essersi con loro purificato, entrò nel tempio, annunziando di voler compiere i giorni della purificazione, fino alla presentazione dell'offerta per ciascun di loro.
\par 27 Or come i sette giorni eran presso che compiuti, i Giudei dell'Asia, vedutolo nel tempio, sollevarono tutta la moltitudine, e gli misero le mani addosso, gridando:
\par 28 Uomini Israeliti, venite al soccorso; questo è l'uomo che va predicando a tutti e da per tutto contro il popolo, contro la legge, e contro questo luogo; e oltre a ciò, ha menato anche de' Greci nel tempio, e ha profanato questo santo luogo.
\par 29 Infatti, aveano veduto prima Trofimo d'Efeso in città con Paolo, e pensavano ch'egli l'avesse menato nel tempio.
\par 30 Tutta la città fu commossa, e si fece un concorso di popolo; e preso Paolo, lo trassero fuori del tempio; e subito le porte furon serrate.
\par 31 Or com'essi cercavan d'ucciderlo, arrivò su al tribuno della coorte la voce che tutta Gerusalemme era sossopra.
\par 32 Ed egli immediatamente prese con sé de' soldati e de' centurioni, e corse giù ai Giudei, i quali, veduto il tribuno e i soldati, cessarono di batter Paolo.
\par 33 Allora il tribuno, accostatosi, lo prese, e comandò che fosse legato con due catene; poi domandò chi egli fosse, e che cosa avesse fatto.
\par 34 E nella folla gli uni gridavano una cosa, e gli altri un'altra; onde, non potendo saper nulla di certo a cagion del tumulto, comandò ch'egli fosse menato nella fortezza.
\par 35 Quando Paolo arrivò alla gradinata dovette, per la violenza della folla, esser portato dai soldati,
\par 36 perché il popolo in gran folla lo seguiva, gridando: Toglilo di mezzo!
\par 37 Or come Paolo stava per esser introdotto nella fortezza, disse al tribuno: Mi è egli lecito dirti qualcosa? Quegli rispose: Sai tu il greco?
\par 38 Non sei tu dunque quell'Egiziano che tempo fa sollevò e menò nel deserto que' quattromila briganti?
\par 39 Ma Paolo disse: Io sono un Giudeo, di Tarso, cittadino di quella non oscura città di Cilicia; e ti prego che tu mi permetta di parlare al popolo.
\par 40 E avendolo egli permesso, Paolo, stando in piè sulla gradinata, fece cenno con la mano al popolo. E fattosi gran silenzio, parlò in lingua ebraica, dicendo:

\chapter{22}

\par 1 Fratelli e padri, ascoltate ciò che ora vi dico a mia difesa.
\par 2 E quand'ebbero udito ch'egli parlava loro in lingua ebraica, tanto più fecero silenzio. Poi disse:
\par 3 Io sono un Giudeo, nato a Tarso di Cilicia, ma allevato in questa città, ai piedi di Gamaliele, educato nella rigida osservanza della legge dei padri, e fui zelante per la causa di Dio, come voi tutti siete oggi;
\par 4 e perseguitai a morte questa Via, legando e mettendo in prigione uomini e donne,
\par 5 come me ne son testimoni il sommo sacerdote e tutto il concistoro degli anziani, dai quali avendo pure ricevuto lettere per i fratelli, mi recavo a Damasco per menare legati a Gerusalemme anche quelli ch'eran quivi, perché fossero puniti.
\par 6 Or avvenne che mentre ero in cammino e mi avvicinavo a Damasco, sul mezzogiorno, di subito dal cielo mi folgoreggiò d'intorno una gran luce.
\par 7 Caddi in terra, e udii una voce che mi disse: Saulo, Saulo, perché mi perseguiti?
\par 8 E io risposi: Chi sei, Signore? Ed egli mi disse: Io sono Gesù il Nazareno, che tu perseguiti.
\par 9 Or coloro ch'erano meco, videro ben la luce, ma non udirono la voce di colui che mi parlava.
\par 10 E io dissi: Signore, che debbo fare? E il Signore mi disse: Lèvati, va' a Damasco, e quivi ti saranno dette tutte le cose che t'è ordinato di fare.
\par 11 E siccome io non ci vedevo più per il fulgore di quella luce, fui menato per mano da coloro che eran meco, e così venni a Damasco.
\par 12 Or un certo Anania, uomo pio secondo la legge, al quale tutti i Giudei che abitavan quivi rendean buona testimonianza, venne a me;
\par 13 e standomi vicino, mi disse: Fratello Saulo, ricupera la vista. Ed io in quell'istante ricuperai la vista, e lo guardai.
\par 14 Ed egli disse: L'Iddio de' nostri padri ti ha destinato a conoscer la sua volontà, e a vedere il Giusto, e a udire una voce dalla sua bocca.
\par 15 Poiché tu gli sarai presso tutti gli uomini un testimone delle cose che hai vedute e udite.
\par 16 Ed ora, che indugi? Lèvati, e sii battezzato, e lavato dei tuoi peccati, invocando il suo nome.
\par 17 Or avvenne, dopo ch'io fui tornato a Gerusalemme, che mentre pregavo nel tempio fui rapito in estasi,
\par 18 e vidi Gesù che mi diceva: Affrettati, ed esci prestamente da Gerusalemme, perché essi non riceveranno la tua testimonianza intorno a me.
\par 19 E io dissi: Signore, eglino stessi sanno che io incarceravo e battevo nelle sinagoghe quelli che credevano in te;
\par 20 e quando si spandeva il sangue di Stefano tuo testimone, anch'io ero presente e approvavo, e custodivo le vesti di coloro che l'uccidevano.
\par 21 Ed egli mi disse: Va', perché io ti manderò lontano, ai Gentili.
\par 22 L'ascoltarono fino a questa parola; e poi alzarono la voce, dicendo: Togli via un tal uomo dal mondo; perché non è degno di vivere.
\par 23 Com'essi gridavano e gettavan via le loro vesti e lanciavano la polvere in aria,
\par 24 il tribuno comandò ch'egli fosse menato dentro la fortezza e inquisito mediante i flagelli, affin di sapere per qual cagione gridassero così contro a lui.
\par 25 E come l'ebbero disteso e legato con le cinghie, Paolo disse al centurione che era presente: V'è egli lecito flagellare un uomo che è cittadino romano, e non è stato condannato?
\par 26 E il centurione, udito questo, venne a riferirlo al tribuno, dicendo: Che stai per fare? perché quest'uomo è Romano.
\par 27 Il tribuno venne a Paolo, e gli chiese: Dimmi, sei tu Romano?
\par 28 Ed egli rispose: Sì. E il tribuno replicò: Io ho acquistato questa cittadinanza per gran somma di denaro. E Paolo disse: Io, invece, l'ho di nascita.
\par 29 Allora quelli che stavan per inquisirlo, si ritrassero subito da lui; e anche il tribuno ebbe paura, quand'ebbe saputo che egli era Romano; perché l'avea fatto legare.
\par 30 E il giorno seguente, volendo saper con certezza di che cosa egli fosse accusato dai Giudei, lo sciolse, e comandò ai capi sacerdoti e a tutto il Sinedrio di radunarsi; e menato giù Paolo, lo fe' comparire dinanzi a loro.

\chapter{23}

\par 1 E Paolo, fissati gli occhi nel Sinedrio, disse: Fratelli, fino a questo giorno, mi son condotto dinanzi a Dio in tutta buona coscienza.
\par 2 E il sommo sacerdote Anania comandò a coloro ch'erano presso a lui di percuoterlo sulla bocca.
\par 3 Allora Paolo gli disse: Iddio percoterà te, parete scialbata; tu siedi per giudicarmi secondo la legge, e violando la legge comandi che io sia percosso?
\par 4 E coloro ch'eran quivi presenti, dissero: Ingiurii tu il sommo sacerdote di Dio?
\par 5 E Paolo disse: Fratelli, io non sapevo che fosse sommo sacerdote; perché sta scritto: 'Non dirai male del principe del tuo popolo'.
\par 6 Or Paolo, sapendo che una parte eran Sadducei e l'altra Farisei, esclamò nel Sinedrio: Fratelli, io son Fariseo, figliuol di Farisei; ed è a motivo della speranza e della risurrezione dei morti, che son chiamato in giudizio.
\par 7 E com'ebbe detto questo, nacque contesa tra i Farisei e i Sadducei, e l'assemblea fu divisa.
\par 8 Poiché i Sadducei dicono che non v'è risurrezione, né angelo, né spirito; mentre i Farisei affermano l'una e l'altra cosa.
\par 9 E si fece un gridar grande; e alcuni degli scribi del partito de' Farisei, levatisi, cominciarono a disputare, dicendo: Noi non troviamo male alcuno in quest'uomo; e se gli avesse parlato uno spirito o un angelo?
\par 10 E facendosi forte la contesa, il tribuno, temendo che Paolo non fosse da loro fatto a pezzi, comandò ai soldati di scendere giù, e di portarlo via dal mezzo di loro, e di menarlo nella fortezza.
\par 11 E la notte seguente il Signore si presentò a Paolo, e gli disse: Sta' di buon cuore; perché come hai reso testimonianza di me a Gerusalemme, così bisogna che tu la renda anche a Roma.
\par 12 E quando fu giorno, i Giudei s'adunarono, e con imprecazioni contro se stessi fecer voto di non mangiare né bere finché non avessero ucciso Paolo.
\par 13 Or coloro che avean fatta questa congiura eran più di quaranta.
\par 14 E vennero ai capi sacerdoti e agli anziani, e dissero: Noi abbiam fatto voto con imprecazione contro noi stessi, di non mangiare cosa alcuna, finché non abbiam ucciso Paolo.
\par 15 Or dunque voi col Sinedrio presentatevi al tribuno per chiedergli di menarlo giù da voi, come se voleste conoscer più esattamente il fatto suo; e noi, innanzi ch'ei giunga, siam pronti ad ucciderlo.
\par 16 Ma il figliuolo della sorella di Paolo, udite queste insidie, venne; ed entrato nella fortezza, riferì la cosa a Paolo.
\par 17 E Paolo, chiamato a sé uno dei centurioni, disse: Mena questo giovane al tribuno, perché ha qualcosa da riferirgli.
\par 18 Egli dunque, presolo, lo menò al tribuno, e disse: Paolo, il prigione, mi ha chiamato e m'ha pregato che ti meni questo giovane, il quale ha qualcosa da dirti.
\par 19 E il tribuno, presolo per la mano e ritiratosi in disparte, gli domandò: Che cos'hai da riferirmi?
\par 20 Ed egli rispose: I Giudei si son messi d'accordo per pregarti che domani tu meni giù Paolo nel Sinedrio, come se volessero informarsi più appieno del fatto suo;
\par 21 ma tu non dar loro retta, perché più di quaranta uomini di loro gli tendono insidie e con imprecazioni contro se stessi han fatto voto di non mangiare né bere, finché non l'abbiano ucciso; ed ora son pronti, aspettando la tua promessa.
\par 22 Il tribuno dunque licenziò il giovane, ordinandogli di non palesare ad alcuno che gli avesse fatto saper queste cose.
\par 23 E chiamati due de' centurioni, disse loro: Tenete pronti fino dalla terza ora della notte duecento soldati, settanta cavalieri e duecento lancieri, per andar fino a Cesarea;
\par 24 e abbiate pronte delle cavalcature per farvi montar su Paolo e condurlo sano e salvo al governatore Felice.
\par 25 E scrisse una lettera del seguente tenore:
\par 26 Claudio Lisia, all'eccellentissimo governatore Felice, salute.
\par 27 Quest'uomo era stato preso dai Giudei, ed era sul punto d'esser da loro ucciso, quand'io son sopraggiunto coi soldati e l'ho sottratto dalle loro mani, avendo inteso che era Romano.
\par 28 E volendo sapere di che l'accusavano, l'ho menato nel loro Sinedrio.
\par 29 E ho trovato che era accusato intorno a questioni della loro legge, ma che non era incolpato di nulla che fosse degno di morte o di prigione.
\par 30 Essendomi però stato riferito che si tenderebbe un agguato contro quest'uomo, l'ho subito mandato a te, ordinando anche ai suoi accusatori di dir davanti a te quello che hanno contro di lui.
\par 31 I soldati dunque, secondo ch'era loro stato ordinato, presero Paolo e lo condussero di notte ad Antipatrìda.
\par 32 E il giorno seguente, lasciati partire i cavalieri con lui, tornarono alla fortezza.
\par 33 E quelli, giunti a Cesarea e consegnata la lettera al governatore, gli presentarono anche Paolo.
\par 34 Ed egli avendo letta la lettera e domandato a Paolo di qual provincia fosse, e inteso che era di Cilicia, gli disse:
\par 35 Io ti udirò meglio quando saranno arrivati anche i tuoi accusatori. E comandò che fosse custodito nel palazzo d'Erode.

\chapter{24}

\par 1 Cinque giorni dopo, il sommo sacerdote Anania discese con alcuni anziani e con un certo Tertullo, oratore; e si presentarono al governatore per accusar Paolo.
\par 2 Questi essendo stato chiamato, Tertullo cominciò ad accusarlo, dicendo:
\par 3 Siccome in grazia tua godiamo molta pace, e per la tua previdenza sono state fatte delle riforme a pro di questa nazione, noi in tutto e per tutto lo riconosciamo, o eccellentissimo Felice, con ogni gratitudine.
\par 4 Ora, per non trattenerti troppo a lungo, ti prego che, secondo la tua condiscendenza, tu ascolti quel che abbiamo a dirti in breve.
\par 5 Abbiam dunque trovato che quest'uomo è una peste, che eccita sedizioni fra tutti i Giudei del mondo, ed è capo della setta de' Nazarei.
\par 6 Egli ha perfino tentato di profanare il tempio; onde noi l'abbiamo preso;
\par 7 tex
\par 8 e da lui, esaminandolo, potrai tu stesso aver piena conoscenza di tutte le cose, delle quali noi l'accusiamo.
\par 9 I Giudei si unirono anch'essi nelle accuse, affermando che le cose stavan così.
\par 10 E Paolo, dopo che il governatore gli ebbe fatto cenno che parlasse, rispose: Sapendo che già da molti anni tu sei giudice di questa nazione, parlo con più coraggio a mia difesa.
\par 11 Poiché tu puoi accertarti che non son più di dodici giorni ch'io salii a Gerusalemme per adorare;
\par 12 ed essi non mi hanno trovato nel tempio, né nelle sinagoghe, né in città a discutere con alcuno, né a far adunata di popolo;
\par 13 e non posson provarti le cose delle quali ora m'accusano.
\par 14 Ma questo ti confesso, che secondo la Via ch'essi chiamano setta, io adoro l'Iddio dei padri, credendo tutte le cose che sono scritte nella legge e nei profeti;
\par 15 avendo in Dio la speranza che nutrono anche costoro che ci sarà una risurrezione de' giusti e degli ingiusti.
\par 16 Per questo anch'io m'esercito ad aver del continuo una coscienza pura dinanzi a Dio e dinanzi agli uomini.
\par 17 Or dopo molti anni, io son venuto a portar elemosine alla mia nazione e a presentar offerte.
\par 18 Mentre io stavo facendo questo, mi hanno trovato purificato nel tempio, senza assembramento e senza tumulto;
\par 19 ed erano alcuni Giudei dell'Asia; questi avrebbero dovuto comparire dinanzi a te ed accusarmi, se avevano cosa alcuna contro a me.
\par 20 D'altronde dicano costoro qual misfatto hanno trovato in me, quando mi presentai dinanzi al Sinedrio;
\par 21 se pur non si tratti di quest'unica parola che gridai, quando comparvi dinanzi a loro: È a motivo della risurrezione de' morti, che io son oggi giudicato da voi.
\par 22 Or Felice, che ben conosceva quel che concerneva questa Via, li rimandò a un'altra volta, dicendo: Quando sarà sceso il tribuno Lisia, esaminerò il fatto vostro.
\par 23 E ordinò al centurione che Paolo fosse custodito, ma lasciandogli una qualche libertà, e non vietando ad alcuno de' suoi di rendergli de' servigi.
\par 24 Or alcuni giorni dopo, Felice, venuto con Drusilla sua moglie, che era giudea, mandò a chiamar Paolo, e l'ascoltò circa la fede in Cristo Gesù.
\par 25 Ma ragionando Paolo di giustizia, di temperanza e del giudizio a venire, Felice, tutto spaventato, replicò: Per ora, vattene; e quando ne troverò l'opportunità, ti manderò a chiamare.
\par 26 Egli sperava, in pari tempo, che da Paolo gli sarebbe dato del denaro; per questo lo mandava spesso a chiamare e discorreva con lui.
\par 27 Or in capo a due anni, Felice ebbe per successore Porcio Festo; e Felice, volendo far cosa grata ai Giudei, lasciò Paolo in prigione.

\chapter{25}

\par 1 Festo dunque, essendo giunto nella sua provincia, tre giorni dopo salì da Cesarea a Gerusalemme.
\par 2 E i capi sacerdoti e i principali dei Giudei gli presentarono le loro accuse contro a Paolo;
\par 3 e lo pregavano, chiedendo per favore contro a lui, che lo facesse venire a Gerusalemme. Essi intanto avrebbero posto insidie per ucciderlo per via.
\par 4 Festo allora rispose che Paolo era custodito a Cesarea, e che egli stesso dovea partir presto.
\par 5 Quelli dunque di voi, diss'egli, che possono, scendano meco; e se v'è in quest'uomo qualche colpa, lo accusino.
\par 6 Rimasto presso di loro non più di otto o dieci giorni, discese in Cesarea; e il giorno seguente, postosi a sedere in tribunale, comandò che Paolo gli fosse menato dinanzi.
\par 7 E com'egli fu giunto, i Giudei che eran discesi da Gerusalemme gli furono attorno, portando contro lui molte e gravi accuse, che non potevano provare; mentre Paolo diceva a sua difesa:
\par 8 Io non ho peccato né contro la legge de' Giudei, né contro il tempio, né contro Cesare.
\par 9 Ma Festo, volendo far cosa grata ai Giudei, disse a Paolo: Vuoi tu salire a Gerusalemme ed esser quivi giudicato davanti a me intorno a queste cose?
\par 10 Ma Paolo rispose: Io sto qui dinanzi al tribunale di Cesare, ove debbo esser giudicato; io non ho fatto torto alcuno ai Giudei, come anche tu sai molto bene.
\par 11 Se dunque sono colpevole e ho commesso cosa degna di morte, non ricuso di morire; ma se nelle cose delle quali costoro mi accusano non c'è nulla di vero, nessuno mi può consegnare per favore nelle loro mani. Io mi appello a Cesare.
\par 12 Allora Festo, dopo aver conferito col consiglio, rispose: Tu ti sei appellato a Cesare; a Cesare andrai.
\par 13 E dopo alquanti giorni il re Agrippa e Berenice arrivarono a Cesarea, per salutar Festo.
\par 14 E trattenendosi essi quivi per molti giorni, Festo raccontò al re il caso di Paolo, dicendo: V'è qui un uomo che è stato lasciato prigione da Felice, contro il quale,
\par 15 quando fui a Gerusalemme, i capi sacerdoti e gli anziani de' Giudei mi sporsero querela, chiedendomi di condannarlo.
\par 16 Risposi loro che non è usanza de' Romani di consegnare alcuno, prima che l'accusato abbia avuto gli accusatori a faccia, e gli sia stato dato modo di difendersi dall'accusa.
\par 17 Essendo eglino dunque venuti qua, io senza indugio, il giorno seguente, sedetti in tribunale, e comandai che quell'uomo mi fosse menato dinanzi.
\par 18 I suoi accusatori però, presentatisi, non gli imputavano alcuna delle male azioni che io supponevo;
\par 19 ma aveano contro lui certe questioni intorno alla propria religione e intorno a un certo Gesù morto, che Paolo affermava esser vivente.
\par 20 Ed io, stando in dubbio sul come procedere in queste cose, gli dissi se voleva andare a Gerusalemme, e quivi esser giudicato intorno a queste cose.
\par 21 Ma avendo Paolo interposto appello per esser riserbato al giudizio dell'imperatore, io comandai che fosse custodito, finché lo mandassi a Cesare.
\par 22 E Agrippa disse a Festo: Anch'io vorrei udir cotesto uomo. Ed egli rispose: Domani l'udrai.
\par 23 Il giorno seguente dunque, essendo venuti Agrippa e Berenice con molta pompa, ed entrati nella sala d'udienza coi tribuni e coi principali della città, Paolo, per ordine di Festo, fu menato quivi.
\par 24 E Festo disse: Re Agrippa, e voi tutti che siete qui presenti con noi, voi vedete quest'uomo, a proposito del quale tutta la moltitudine de' Giudei s'è rivolta a me, in Gerusalemme e qui, gridando che non deve viver più oltre.
\par 25 Io però non ho trovato che avesse fatto cosa alcuna degna di morte, ed essendosi egli stesso appellato all'imperatore, ho deliberato di mandarglielo.
\par 26 E siccome non ho nulla di certo da scriverne al mio signore, l'ho menato qui davanti a voi, e principalmente davanti a te, o re Agrippa, affinché, dopo esame, io abbia qualcosa da scrivere.
\par 27 Perché non mi par cosa ragionevole mandare un prigioniero, senza notificar le accuse che gli son mosse contro.

\chapter{26}

\par 1 E Agrippa disse a Paolo: T'è permesso parlare a tua difesa. Allora Paolo, distesa la mano, disse a sua difesa:
\par 2 Re Agrippa, io mi reputo felice di dovermi oggi scolpare dinanzi a te di tutte le cose delle quali sono accusato dai Giudei,
\par 3 principalmente perché tu hai conoscenza di tutti i riti e di tutte le questioni che son fra i Giudei, perciò ti prego di ascoltarmi pazientemente.
\par 4 Quale sia stato il mio modo di vivere dalla mia giovinezza, fin dal principio trascorsa in mezzo alla mia nazione e in Gerusalemme, tutti i Giudei lo sanno,
\par 5 poiché mi hanno conosciuto fin d'allora e sanno, se pur vogliono renderne testimonianza, che, secondo la più rigida setta della nostra religione, son vissuto Fariseo.
\par 6 E ora son chiamato in giudizio per la speranza della promessa fatta da Dio ai nostri padri;
\par 7 della qual promessa le nostre dodici tribù, che servono con fervore a Dio notte e giorno, sperano di vedere il compimento. E per questa speranza, o re, io sono accusato dai Giudei!
\par 8 Perché mai si giudica da voi cosa incredibile che Dio risusciti i morti?
\par 9 Quant'è a me, avevo sì pensato anch'io di dover fare molte cose contro il nome di Gesù il Nazareno.
\par 10 E questo difatti feci a Gerusalemme; e avutane facoltà dai capi sacerdoti serrai nelle prigioni molti de' santi; e quando erano messi a morte, io detti il mio voto.
\par 11 E spesse volte, per tutte le sinagoghe li costrinsi con pene a bestemmiare; e infuriato oltremodo contro di loro, li perseguitai fino nelle città straniere.
\par 12 Il che facendo, come andavo a Damasco con potere e commissione de' capi sacerdoti,
\par 13 io vidi, o re, per cammino a mezzo giorno, una luce dal cielo, più risplendente del sole, la quale lampeggiò intorno a me ed a coloro che viaggiavan meco.
\par 14 Ed essendo noi tutti caduti in terra, udii una voce che mi disse in lingua ebraica: Saulo, Saulo, perché mi perseguiti? Ei t'è duro di ricalcitrar contro gli stimoli.
\par 15 E io dissi: Chi sei tu, Signore? E il Signore rispose: Io son Gesù, che tu perseguiti.
\par 16 Ma lèvati, e sta' in piè; perché per questo ti sono apparito: per stabilirti ministro e testimone delle cose che tu hai vedute, e di quelle per le quali ti apparirò ancora,
\par 17 liberandoti da questo popolo e dai Gentili, ai quali io ti mando
\par 18 per aprir loro gli occhi, onde si convertano dalle tenebre alla luce e dalla potestà di Satana a Dio, e ricevano, per la fede in me, la remissione dei peccati e la loro parte d'eredità fra i santificati.
\par 19 Perciò, o re Agrippa, io non sono stato disubbidiente alla celeste visione;
\par 20 ma, prima a que' di Damasco, poi a Gerusalemme e per tutto il paese della Giudea e ai Gentili, ho annunziato che si ravveggano e si convertano a Dio, facendo opere degne del ravvedimento.
\par 21 Per questo i Giudei, avendomi preso nel tempio, tentavano d'uccidermi.
\par 22 Ma per l'aiuto che vien da Dio, son durato fino a questo giorno, rendendo testimonianza a piccoli e a grandi, non dicendo nulla all'infuori di quello che i profeti e Mosè hanno detto dover avvenire, cioè:
\par 23 che il Cristo soffrirebbe, e che egli, il primo a risuscitar dai morti, annunzierebbe la luce al popolo ed ai Gentili.
\par 24 Or mentre ei diceva queste cose a sua difesa, Festo disse ad alta voce: Paolo, tu vaneggi; la molta dottrina ti mette fuor di senno.
\par 25 Ma Paolo disse: Io non vaneggio, eccellentissimo Festo; ma pronunzio parole di verità, e di buon senno.
\par 26 Poiché il re, al quale io parlo con franchezza, conosce queste cose; perché son persuaso che nessuna di esse gli è occulta; poiché questo non è stato fatto in un cantuccio.
\par 27 O re Agrippa, credi tu ai profeti? Io so che tu ci credi.
\par 28 E Agrippa disse a Paolo: Per poco non mi persuadi a diventar cristiano.
\par 29 E Paolo: Piacesse a Dio che per poco o per molto, non solamente tu, ma anche tutti quelli che oggi m'ascoltano, diventaste tali, quale sono io, all'infuori di questi legami.
\par 30 Allora il re si alzò, e con lui il governatore, Berenice, e quanti sedevano con loro;
\par 31 e ritiratisi in disparte, parlavano gli uni agli altri, dicendo: Quest'uomo non fa nulla che meriti morte o prigione.
\par 32 E Agrippa disse a Festo: Quest'uomo poteva esser liberato, se non si fosse appellato a Cesare.

\chapter{27}

\par 1 Or quando fu determinato che faremmo vela per l'Italia, Paolo e certi altri prigionieri furon consegnati a un centurione, per nome Giulio, della coorte Augusta.
\par 2 E montati sopra una nave adramittina, che doveva toccare i porti della costa d'Asia, salpammo, avendo con noi Aristarco, Macedone di Tessalonica.
\par 3 Il giorno seguente arrivammo a Sidone; e Giulio, usando umanità verso Paolo, gli permise d'andare dai suoi amici per ricevere le loro cure.
\par 4 Poi, essendo partiti di là, navigammo sotto Cipro, perché i venti erano contrari.
\par 5 E passato il mar di Cilicia e di Panfilia, arrivammo a Mira di Licia.
\par 6 E il centurione, trovata quivi una nave alessandrina che facea vela per l'Italia, ci fe' montare su quella.
\par 7 E navigando per molti giorni lentamente, e pervenuti a fatica, per l'impedimento del vento, di faccia a Gnido, veleggiammo sotto Creta, di rincontro a Salmone;
\par 8 e costeggiandola con difficoltà, venimmo a un certo luogo, detto Beiporti, vicino al quale era la città di Lasea.
\par 9 Or essendo trascorso molto tempo, ed essendo la navigazione ormai pericolosa, poiché anche il Digiuno era già passato, Paolo li ammonì dicendo loro:
\par 10 Uomini, io veggo che la navigazione si farà con pericolo e grave danno, non solo del carico e della nave, ma anche delle nostre persone.
\par 11 Ma il centurione prestava più fede al pilota e al padron della nave che alle cose dette da Paolo.
\par 12 E siccome quel porto non era adatto a svernare, i più furono di parere di partir di là per cercare d'arrivare a Fenice, porto di Creta che guarda a Libeccio e a Maestro, e di passarvi l'inverno.
\par 13 Essendosi intanto levato un leggero scirocco, e credendo essi d'esser venuti a capo del loro proposito, levate le àncore, si misero a costeggiare l'isola di Creta più da presso.
\par 14 Ma poco dopo si scatenò giù dall'isola un vento turbinoso, che si chiama Euraquilone;
\par 15 ed essendo la nave portata via e non potendo reggere al vento, la lasciammo andare, ed eravamo portati alla deriva.
\par 16 E passati rapidamente sotto un'isoletta chiamata Clauda, a stento potemmo avere in nostro potere la scialuppa.
\par 17 E quando l'ebbero tirata su, ricorsero a ripari, cingendo la nave di sotto; e temendo di esser gettati sulla Sirti, calarono le vele, ed eran così portati via.
\par 18 E siccome eravamo fieramente sbattuti dalla tempesta, il giorno dopo cominciarono a far getto del carico.
\par 19 E il terzo giorno, con le loro proprie mani, buttarono in mare gli arredi della nave.
\par 20 E non apparendo né sole né stelle già da molti giorni, ed essendoci sopra non piccola tempesta, era ormai tolta ogni speranza di scampare.
\par 21 Or dopo che furono stati lungamente senza prender cibo, Paolo si levò in mezzo a loro, e disse: Uomini, bisognava darmi ascolto, non partire da Creta, e risparmiar così questo pericolo e questa perdita.
\par 22 Ora però vi esorto a star di buon cuore, perché non vi sarà perdita della vita d'alcun di voi ma solo della nave.
\par 23 Poiché un angelo dell'Iddio, al quale appartengo e ch'io servo, m'è apparso questa notte,
\par 24 dicendo: Paolo, non temere; bisogna che tu comparisca dinanzi a Cesare ed ecco, Iddio ti ha donato tutti coloro che navigano teco.
\par 25 Perciò, o uomini, state di buon cuore, perché ho fede in Dio che avverrà come mi è stato detto.
\par 26 Ma dobbiamo esser gettati sopra un'isola.
\par 27 E la quattordicesima notte da che eravamo portati qua e là per l'Adriatico, verso la mezzanotte i marinai sospettavano d'esser vicini a terra;
\par 28 e calato lo scandaglio, trovarono venti braccia; poi, passati un po' più oltre e scandagliato di nuovo, trovarono quindici braccia.
\par 29 Temendo allora di percuotere in luoghi scogliosi, gettarono da poppa quattro àncore, aspettando ansiosamente che facesse giorno.
\par 30 Or cercando i marinai di fuggir dalla nave, e avendo calato la scialuppa in mare col pretesto di voler calare le àncore dalla prua,
\par 31 Paolo disse al centurione ed ai soldati: Se costoro non restano nella nave, voi non potete scampare.
\par 32 Allora i soldati tagliaron le funi della scialuppa, e la lasciaron cadere.
\par 33 E mentre si aspettava che facesse giorno, Paolo esortava tutti a prender cibo, dicendo: Oggi son quattordici giorni che state aspettando, sempre digiuni, senza prender nulla.
\par 34 Perciò, io v'esorto a prender cibo, perché questo contribuirà alla vostra salvezza; poiché non perirà neppure un capello del capo d'alcun di voi.
\par 35 Detto questo, preso del pane, rese grazie a Dio, in presenza di tutti; poi, rottolo, cominciò a mangiare.
\par 36 E tutti, fatto animo, presero anch'essi del cibo.
\par 37 Or eravamo sulla nave, fra tutti, dugentosettantasei persone.
\par 38 E saziati che furono, alleggerirono la nave, gettando il frumento in mare.
\par 39 Quando fu giorno, non riconoscevano il paese; ma scòrsero una certa baia che aveva una spiaggia, e deliberarono, se fosse loro possibile, di spingervi la nave.
\par 40 E staccate le àncore, le lasciarono andare in mare; sciolsero al tempo stesso i legami dei timoni, e alzato l'artimone al vento, traevano al lido.
\par 41 Ma essendo incorsi in un luogo che avea il mare d'ambo i lati, vi fecero arrenar la nave; e mentre la prua, incagliata, rimaneva immobile, la poppa si sfasciava per la violenza delle onde.
\par 42 Or il parere de' soldati era d'uccidere i prigionieri, perché nessuno fuggisse a nuoto.
\par 43 Ma il centurione, volendo salvar Paolo, li distolse da quel proposito, e comandò che quelli che sapevan nuotare si gettassero in mare per andarsene i primi a terra,
\par 44 e gli altri vi arrivassero, chi sopra tavole, e chi sopra altri pezzi della nave. E così avvenne che tutti giunsero salvi a terra.

\chapter{28}

\par 1 E dopo che fummo scampati, riconoscemmo che l'isola si chiamava Malta.
\par 2 E i barbari usarono verso noi umanità non comune; poiché, acceso un gran fuoco, ci accolsero tutti, a motivo della pioggia che cadeva, e del freddo.
\par 3 Or Paolo, avendo raccolto una quantità di legna secche e avendole poste sul fuoco, una vipera, sentito il caldo, uscì fuori, e gli si attaccò alla mano.
\par 4 E quando i barbari videro la bestia che gli pendeva dalla mano, dissero fra loro: Certo, quest'uomo è un omicida, perché essendo scampato dal mare, pur la Giustizia divina non lo lascia vivere.
\par 5 Ma Paolo, scossa la bestia nel fuoco, non ne risentì male alcuno.
\par 6 Or essi si aspettavano ch'egli enfierebbe o cadrebbe di subito morto; ma dopo aver lungamente aspettato, veduto che non gliene avveniva alcun male, mutarono parere, e cominciarono a dire ch'egli era un dio.
\par 7 Or ne' dintorni di quel luogo v'erano dei poderi dell'uomo principale dell'isola, chiamato Publio, il quale ci accolse, e ci albergò tre giorni amichevolmente.
\par 8 E accadde che il padre di Publio giacea malato di febbre e di dissenteria. Paolo andò a trovarlo; e dopo aver pregato, gl'impose le mani e lo guarì.
\par 9 Avvenuto questo, anche gli altri che aveano delle infermità nell'isola, vennero, e furono guariti;
\par 10 ed essi ci fecero grandi onori; e quando salpammo, ci portarono a bordo le cose necessarie.
\par 11 Tre mesi dopo, partimmo sopra una nave alessandrina che avea per insegna Castore e Polluce, e che avea svernato nell'isola.
\par 12 E arrivati a Siracusa, vi restammo tre giorni.
\par 13 E di là, costeggiando, arrivammo a Reggio. E dopo un giorno, levatosi un vento di scirocco, in due giorni arrivammo a Pozzuoli.
\par 14 E avendo quivi trovato de' fratelli, fummo pregati di rimanere presso di loro sette giorni. E così venimmo a Roma.
\par 15 Or i fratelli, avute nostre notizie, di là ci vennero incontro sino al Foro Appio e alle Tre Taverne; e Paolo quando li ebbe veduti, rese grazie a Dio e prese animo.
\par 16 E giunti che fummo a Roma, a Paolo fu concesso d'abitar da sé col soldato che lo custodiva.
\par 17 E tre giorni dopo, Paolo convocò i principali fra i Giudei; e quando furon raunati, disse loro: Fratelli, senza aver fatto nulla contro il popolo né contro i riti dei padri, io fui arrestato in Gerusalemme, e di là dato in man de' Romani.
\par 18 I quali, avendomi esaminato, volevano rilasciarmi perché non era in me colpa degna di morte.
\par 19 Ma opponendovisi i Giudei, fui costretto ad appellarmi a Cesare, senza però aver in animo di portare alcuna accusa contro la mia nazione.
\par 20 Per questa ragione dunque vi ho chiamati per vedervi e per parlarvi; perché egli è a causa della speranza d'Israele ch'io sono stretto da questa catena.
\par 21 Ma essi gli dissero: Noi non abbiamo ricevuto lettere dalla Giudea intorno a te, né è venuto qui alcuno de' fratelli a riferire o a dir male di te.
\par 22 Ben vorremmo però sentir da te quel che tu pensi; perché, quant'è a cotesta setta, ci è noto che da per tutto essa incontra opposizione.
\par 23 E avendogli fissato un giorno, vennero a lui nel suo alloggio in gran numero; ed egli da mane a sera esponeva loro le cose, testimoniando del regno di Dio e persuadendoli di quel che concerne Gesù, con la legge di Mosè e coi profeti.
\par 24 E alcuni restaron persuasi delle cose dette; altri invece non credettero.
\par 25 E non essendo d'accordo fra loro, si ritirarono, dopo che Paolo ebbe detta quest'unica parola: Ben parlò lo Spirito Santo ai vostri padri per mezzo del profeta Isaia dicendo:
\par 26 Va' a questo popolo e di': Voi udrete coi vostri orecchi e non intenderete; guarderete coi vostri occhi, e non vedrete;
\par 27 perché il cuore di questo popolo s'è fatto insensibile, son divenuti duri di orecchi, e hanno chiuso gli occhi, che talora non veggano con gli occhi, e non odano con gli orecchi, e non intendano col cuore, e non si convertano, ed io non li guarisca.
\par 28 Sappiate dunque che questa salvazione di Dio è mandata ai Gentili; ed essi presteranno ascolto.
\par 29 int
\par 30 E Paolo dimorò due anni interi in una casa da lui presa a fitto, e riceveva tutti coloro che venivano a trovarlo,
\par 31 predicando il regno di Dio, e insegnando le cose relative al Signor Gesù Cristo con tutta franchezza e senza che alcuno glielo impedisse.


\end{document}