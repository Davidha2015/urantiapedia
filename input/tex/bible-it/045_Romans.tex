\begin{document}

\title{Romani}


\chapter{1}

\par 1 Paolo, servo di Cristo Gesù, chiamato ad essere apostolo, appartato per l'Evangelo di Dio,
\par 2 ch'Egli avea già promesso per mezzo de' suoi profeti nelle sante Scritture
\par 3 e che concerne il suo Figliuolo,
\par 4 nato dal seme di Davide secondo la carne, dichiarato Figliuol di Dio con potenza secondo lo spirito di santità mediante la sua risurrezione dai morti; cioè Gesù Cristo nostro Signore,
\par 5 per mezzo del quale noi abbiam ricevuto grazia e apostolato per trarre all'ubbidienza della fede tutti i Gentili, per amor del suo nome -
\par 6 fra i quali Gentili siete voi pure, chiamati da Gesù Cristo -
\par 7 a quanti sono in Roma, amati da Dio, chiamati ad esser santi, grazia a voi e pace da Dio nostro Padre e dal Signor Gesù Cristo.
\par 8 Prima di tutto io rendo grazie all'Iddio mio per mezzo di Gesù Cristo per tutti voi perché la vostra fede è pubblicata per tutto il mondo.
\par 9 Poiché Iddio, al quale servo nello spirito mio annunziando l'Evangelo del suo Figliuolo, mi è testimone ch'io non resto dal far menzione di voi in tutte le mie preghiere,
\par 10 chiedendo che in qualche modo mi sia porta finalmente, per la volontà di Dio, l'occasione propizia di venire a voi.
\par 11 Poiché desidero vivamente di vedervi per comunicarvi qualche dono spirituale affinché siate fortificati;
\par 12 o meglio, perché quando sarò tra voi ci confortiamo a vicenda mediante la fede che abbiamo in comune, voi ed io.
\par 13 Or, fratelli, non voglio che ignoriate che molte volte mi son proposto di recarmi da voi (ma finora ne sono stato impedito) per avere qualche frutto anche fra voi come fra il resto dei Gentili.
\par 14 Io son debitore tanto ai Greci quanto ai Barbari, tanto ai savî quanto agli ignoranti;
\par 15 ond'è che, per quanto sta in me, io son pronto ad annunziar l'Evangelo anche a voi che siete in Roma.
\par 16 Poiché io non mi vergogno dell'Evangelo; perché esso è potenza di Dio per la salvezza d'ogni credente; del Giudeo prima e poi del Greco;
\par 17 poiché in esso la giustizia di Dio è rivelata da fede a fede, secondo che è scritto: Ma il giusto vivrà per fede.
\par 18 Poiché l'ira di Dio si rivela dal cielo contro ogni empietà ed ingiustizia degli uomini che soffocano la verità con l'ingiustizia;
\par 19 infatti quel che si può conoscer di Dio è manifesto in loro, avendolo Iddio loro manifestato;
\par 20 poiché le perfezioni invisibili di lui, la sua eterna potenza e divinità, si vedon chiaramente sin dalla creazione del mondo, essendo intese per mezzo delle opere sue;
\par 21 ond'è che essi sono inescusabili, perché, pur avendo conosciuto Iddio, non l'hanno glorificato come Dio, né l'hanno ringraziato; ma si son dati a vani ragionamenti, e l'insensato loro cuore s'è ottenebrato.
\par 22 Dicendosi savî, son divenuti stolti,
\par 23 e hanno mutato la gloria dell'incorruttibile Iddio in immagini simili a quelle dell'uomo corruttibile, e d'uccelli e di quadrupedi e di rettili.
\par 24 Per questo, Iddio li ha abbandonati, nelle concupiscenze de' loro cuori, alla impurità, perché vituperassero fra loro i loro corpi;
\par 25 essi, che hanno mutato la verità di Dio in menzogna, e hanno adorato e servito la creatura invece del Creatore, che è benedetto in eterno. Amen.
\par 26 Perciò Iddio li ha abbandonati a passioni infami: poiché le loro femmine hanno mutato l'uso naturale in quello che è contro natura,
\par 27 e similmente anche i maschi, lasciando l'uso naturale della donna, si sono infiammati nella loro libidine gli uni per gli altri, commettendo uomini con uomini cose turpi, e ricevendo in loro stessi la condegna mercede del proprio traviamento.
\par 28 E siccome non si son curati di ritenere la conoscenza di Dio, Iddio li ha abbandonati ad una mente reproba, perché facessero le cose che sono sconvenienti,
\par 29 essendo essi ricolmi d'ogni ingiustizia, malvagità, cupidigia, malizia; pieni d'invidia, d'omicidio, di contesa, di frode, di malignità;
\par 30 delatori, maldicenti, abominevoli a Dio, insolenti, superbi, vanagloriosi, inventori di mali, disubbidienti ai genitori,
\par 31 insensati, senza fede nei patti, senza affezione naturale, spietati;
\par 32 i quali, pur conoscendo che secondo il giudizio di Dio quelli che fanno codeste cose son degni di morte, non soltanto le fanno, ma anche approvano chi le commette.

\chapter{2}

\par 1 Perciò, o uomo, chiunque tu sii che giudichi, sei inescusabile; poiché nel giudicare gli altri, tu condanni te stesso; perché tu che giudichi, fai le medesime cose.
\par 2 Or noi sappiamo che il giudizio di Dio su quelli che fanno tali cose è conforme a verità.
\par 3 E pensi tu, o uomo che giudichi quelli che fanno tali cose e le fai tu stesso, di scampare al giudizio di Dio?
\par 4 Ovvero sprezzi tu le ricchezze della sua benignità, della sua pazienza e della sua longanimità, non riconoscendo che la benignità di Dio ti trae a ravvedimento?
\par 5 Tu invece, seguendo la tua durezza e il tuo cuore impenitente, t'accumuli un tesoro d'ira, per il giorno dell'ira e della rivelazione del giusto giudizio di Dio,
\par 6 il quale renderà a ciascuno secondo le sue opere:
\par 7 vita eterna a quelli che con la perseveranza nel bene oprare cercano gloria e onore e immortalità;
\par 8 ma a quelli che son contenziosi e non ubbidiscono alla verità ma ubbidiscono alla ingiustizia, ira e indignazione.
\par 9 Tribolazione e angoscia sopra ogni anima d'uomo che fa il male; del Giudeo prima, e poi del Greco;
\par 10 ma gloria e onore e pace a chiunque opera bene; al Giudeo prima e poi al Greco;
\par 11 poiché dinanzi a Dio non c'è riguardo a persone.
\par 12 Infatti, tutti coloro che hanno peccato senza legge, periranno pure senza legge; e tutti coloro che hanno peccato avendo legge, saranno giudicati con quella legge;
\par 13 poiché non quelli che ascoltano la legge son giusti dinanzi a Dio, ma quelli che l'osservano saranno giustificati.
\par 14 Infatti, quando i Gentili che non hanno legge, adempiono per natura le cose della legge, essi, che non hanno legge, son legge a se stessi;
\par 15 essi mostrano che quel che la legge comanda è scritto nei loro cuori per la testimonianza che rende loro la coscienza, e perché i loro pensieri si accusano od anche si scusano a vicenda.
\par 16 Tutto ciò si vedrà nel giorno in cui Dio giudicherà i segreti degli uomini per mezzo di Gesù Cristo, secondo il mio Evangelo.
\par 17 Or se tu ti chiami Giudeo, e ti riposi sulla legge, e ti glorii in Dio,
\par 18 e conosci la sua volontà, e discerni la differenza delle cose essendo ammaestrato dalla legge,
\par 19 e ti persuadi d'esser guida de' ciechi, luce di quelli che sono nelle tenebre,
\par 20 educatore degli scempî, maestro dei fanciulli, perché hai nella legge la formula della conoscenza e della verità,
\par 21 come mai, dunque, tu che insegni agli altri non insegni a te stesso? Tu che predichi che non si deve rubare, rubi?
\par 22 Tu che dici che non si deve commettere adulterio, commetti adulterio? Tu che hai in abominio gl'idoli, saccheggi i templi?
\par 23 Tu che meni vanto della legge, disonori Dio trasgredendo la legge?
\par 24 Poiché, siccome è scritto, il nome di Dio, per cagion vostra, è bestemmiato fra i Gentili.
\par 25 Infatti ben giova la circoncisione se tu osservi la legge; ma se tu sei trasgressore della legge, la tua circoncisione diventa incirconcisione.
\par 26 E se l'incirconciso osserva i precetti della legge, la sua incirconcisione non sarà essa reputata circoncisione?
\par 27 E così colui che è per natura incirconciso, se adempie la legge, giudicherà te, che con la lettera e la circoncisione sei un trasgressore della legge.
\par 28 Poiché Giudeo non è colui che è tale all'esterno; né è circoncisione quella che è esterna, nella carne;
\par 29 ma Giudeo è colui che lo è interiormente; e la circoncisione è quella del cuore, in ispirito, non in lettera; d'un tal Giudeo la lode procede non dagli uomini, ma da Dio.

\chapter{3}

\par 1 Qual è dunque il vantaggio del Giudeo? O qual è la utilità della circoncisione?
\par 2 Grande per ogni maniera; prima di tutto, perché a loro furono affidati gli oracoli di Dio.
\par 3 Poiché che vuol dire se alcuni sono stati increduli? Annullerà la loro incredulità la fedeltà di Dio?
\par 4 Così non sia; anzi, sia Dio riconosciuto verace, ma ogni uomo bugiardo, siccome è scritto: Affinché tu sia riconosciuto giusto nelle tue parole, e resti vincitore quando sei giudicato.
\par 5 Ma se la nostra ingiustizia fa risaltare la giustizia di Dio, che diremo noi? Iddio è egli ingiusto quando dà corso alla sua ira? (Io parlo umanamente).
\par 6 Così non sia; perché, altrimenti, come giudicherà egli il mondo?
\par 7 Ma se per la mia menzogna la verità di Dio è abbondata a sua gloria, perché son io ancora giudicato come peccatore?
\par 8 E perché (secondo la calunnia che ci è lanciata e la massima che taluni ci attribuiscono), perché non "facciamo il male affinché ne venga il bene?" La condanna di quei tali è giusta.
\par 9 Che dunque? Abbiam noi qualche superiorità? Affatto; perché abbiamo dianzi provato che tutti, Giudei e Greci, sono sotto il peccato,
\par 10 siccome è scritto: Non v'è alcun giusto, neppure uno.
\par 11 Non v'è alcuno che abbia intendimento, non v'è alcuno che ricerchi Dio.
\par 12 Tutti si sono sviati, tutti quanti son divenuti inutili. Non v'è alcuno che pratichi la bontà, no, neppur uno.
\par 13 La loro gola è un sepolcro aperto; con le loro lingue hanno usato frode; v'è un veleno di aspidi sotto le loro labbra.
\par 14 La loro bocca è piena di maledizione e d'amarezza.
\par 15 I loro piedi son veloci a spargere il sangue.
\par 16 Sulle lor vie è rovina e calamità,
\par 17 e non hanno conosciuto la via della pace.
\par 18 Non c'è timor di Dio dinanzi agli occhi loro.
\par 19 Or noi sappiamo che tutto quel che la legge dice, lo dice a quelli che son sotto la legge, affinché ogni bocca sia turata, e tutto il mondo sia sottoposto al giudizio di Dio;
\par 20 poiché per le opere della legge nessuno sarà giustificato al suo cospetto; giacché mediante la legge è data la conoscenza del peccato.
\par 21 Ora, però, indipendentemente dalla legge, è stata manifestata una giustizia di Dio, attestata dalla legge e dai profeti:
\par 22 vale a dire la giustizia di Dio mediante la fede in Gesù Cristo, per tutti i credenti; poiché non v'è distinzione,
\par 23 difatti, tutti hanno peccato e son privi della gloria di Dio,
\par 24 e son giustificati gratuitamente per la sua grazia, mediante la redenzione che è in Cristo Gesù,
\par 25 il quale Iddio ha prestabilito come propiziazione mediante la fede nel sangue d'esso, per dimostrare la sua giustizia, avendo Egli usato tolleranza verso i peccati commessi in passato, al tempo della sua divina pazienza;
\par 26 per dimostrare, dico, la sua giustizia nel tempo presente; ond'Egli sia giusto e giustificante colui che ha fede in Gesù.
\par 27 Dov'è dunque il vanto? Esso è escluso. Per qual legge? Delle opere? No, ma per la legge della fede;
\par 28 poiché noi riteniamo che l'uomo è giustificato mediante la fede, senza le opere della legge.
\par 29 Iddio è Egli forse soltanto l'Iddio de' Giudei? Non è Egli anche l'Iddio de' Gentili? Certo lo è anche de' Gentili,
\par 30 poiché v'è un Dio solo, il quale giustificherà il circonciso per fede, e l'incirconciso parimente mediante la fede.
\par 31 Annulliamo noi dunque la legge mediante la fede? Così non sia; anzi, stabiliamo la legge.

\chapter{4}

\par 1 Che diremo dunque che l'antenato nostro Abramo abbia ottenuto secondo la carne?
\par 2 Poiché se Abramo è stato giustificato per le opere, egli avrebbe di che gloriarsi; ma dinanzi a Dio egli non ha di che gloriarsi; infatti, che dice la Scrittura?
\par 3 Or Abramo credette a Dio, e ciò gli fu messo in conto di giustizia.
\par 4 Or a chi opera, la mercede non è messa in conto di grazia, ma di debito;
\par 5 mentre a chi non opera ma crede in colui che giustifica l'empio, la sua fede gli è messa in conto di giustizia.
\par 6 Così pure Davide proclama la beatitudine dell'uomo al quale Iddio imputa la giustizia senz'opere, dicendo:
\par 7 Beati quelli le cui iniquità son perdonate, e i cui peccati sono coperti.
\par 8 Beato l'uomo al quale il Signore non imputa il peccato.
\par 9 Questa beatitudine è ella soltanto per i circoncisi o anche per gli incirconcisi? Poiché noi diciamo che la fede fu ad Abramo messa in conto di giustizia.
\par 10 In che modo dunque gli fu messa in conto? Quand'era circonciso, o quand'era incirconciso? Non quand'era circonciso, ma quand'era incirconciso;
\par 11 poi ricevette il segno della circoncisione, qual suggello della giustizia ottenuta per la fede che avea quand'era incirconciso, affinché fosse il padre di tutti quelli che credono essendo incirconcisi, onde anche a loro sia messa in conto la giustizia;
\par 12 e il padre dei circoncisi, di quelli, cioè, che non solo sono circoncisi, ma seguono anche le orme della fede del nostro padre Abramo quand'era ancora incirconciso.
\par 13 Poiché la promessa d'esser erede del mondo non fu fatta ad Abramo o alla sua progenie in base alla legge, ma in base alla giustizia che vien dalla fede.
\par 14 Perché, se quelli che son della legge sono eredi, la fede è resa vana, e la promessa è annullata;
\par 15 poiché la legge genera ira; ma dove non c'è legge, non c'è neppur trasgressione.
\par 16 Perciò l'eredità è per fede, affinché sia per grazia; onde la promessa sia sicura per tutta la progenie; non soltanto per quella che è sotto la legge, ma anche per quella che ha la fede d'Abramo, il quale è padre di noi tutti
\par 17 (secondo che è scritto: Io ti ho costituito padre di molte nazioni) dinanzi al Dio a cui egli credette, il quale fa rivivere i morti, e chiama le cose che non sono, come se fossero.
\par 18 Egli, sperando contro speranza, credette, per diventar padre di molte nazioni, secondo quel che gli era stato detto: Così sarà la tua progenie.
\par 19 E senza venir meno nella fede, egli vide bensì che il suo corpo era svigorito (avea quasi cent'anni), e che Sara non era più in grado d'esser madre;
\par 20 ma, dinanzi alla promessa di Dio, non vacillò per incredulità, ma fu fortificato per la sua fede dando gloria a Dio
\par 21 ed essendo pienamente convinto che ciò che avea promesso, Egli era anche potente da effettuarlo.
\par 22 Ond'è che ciò gli fu messo in conto di giustizia.
\par 23 Or non per lui soltanto sta scritto che questo gli fu messo in conto di giustizia,
\par 24 ma anche per noi ai quali sarà così messo in conto; per noi che crediamo in Colui che ha risuscitato dai morti Gesù, nostro Signore,
\par 25 il quale è stato dato a cagione delle nostre offese, ed è risuscitato a cagione della nostra giustificazione.

\chapter{5}

\par 1 Giustificati dunque per fede, abbiam pace con Dio per mezzo di Gesù Cristo, nostro Signore,
\par 2 mediante il quale abbiamo anche avuto, per la fede, l'accesso a questa grazia nella quale stiamo saldi; e ci gloriamo nella speranza della gloria di Dio;
\par 3 e non soltanto questo, ma ci gloriamo anche nelle afflizioni, sapendo che l'afflizione produce pazienza, la pazienza esperienza,
\par 4 e la esperienza speranza.
\par 5 Or la speranza non rende confusi, perché l'amor di Dio è stato sparso nei nostri cuori per lo Spirito Santo che ci è stato dato.
\par 6 Perché, mentre eravamo ancora senza forza, Cristo, a suo tempo, è morto per gli empî.
\par 7 Poiché a mala pena uno muore per un giusto; ma forse per un uomo dabbene qualcuno ardirebbe morire;
\par 8 ma Iddio mostra la grandezza del proprio amore per noi, in quanto che, mentre eravamo ancora peccatori, Cristo è morto per noi.
\par 9 Tanto più dunque, essendo ora giustificati per il suo sangue, sarem per mezzo di lui salvati dall'ira.
\par 10 Perché, se mentre eravamo nemici siamo stati riconciliati con Dio mediante la morte del suo Figliuolo, tanto più ora, essendo riconciliati, saremo salvati mediante la sua vita.
\par 11 E non soltanto questo, ma anche ci gloriamo in Dio per mezzo del nostro Signor Gesù Cristo, per il quale abbiamo ora ottenuto la riconciliazione.
\par 12 Perciò, siccome per mezzo d'un sol uomo il peccato è entrato nel mondo, e per mezzo del peccato v'è entrata la morte, e in questo modo la morte è passata su tutti gli uomini, perché tutti hanno peccato...
\par 13 Poiché, fino alla legge, il peccato era nel mondo; ma il peccato non è imputato quando non v'è legge.
\par 14 Eppure, la morte regnò, da Adamo fino a Mosè, anche su quelli che non avean peccato con una trasgressione simile a quella d'Adamo, il quale è il tipo di colui che dovea venire.
\par 15 Però, la grazia non è come il fallo. Perché se per il fallo di quell'uno i molti sono morti, molto più la grazia di Dio e il dono fattoci dalla grazia dell'unico uomo Gesù Cristo, hanno abbondato verso i molti.
\par 16 E riguardo al dono non avviene quel che è avvenuto nel caso dell'uno che ha peccato; poiché il giudizio da un unico fallo ha fatto capo alla condanna; mentre la grazia, da molti falli, ha fatto capo alla giustificazione.
\par 17 Perché, se per il fallo di quell'uno la morte ha regnato mediante quell'uno, tanto più quelli che ricevono l'abbondanza della grazia e del dono della giustizia, regneranno nella vita per mezzo di quell'uno che è Gesù Cristo.
\par 18 - Come dunque con un sol fallo la condanna si è estesa a tutti gli uomini, così, con un solo atto di giustizia la giustificazione che dà vita s'è estesa a tutti gli uomini.
\par 19 Poiché, siccome per la disubbidienza di un solo uomo i molti sono stati costituiti peccatori, così anche per l'ubbidienza d'un solo, i molti saran costituiti giusti.
\par 20 Or la legge è intervenuta affinché il fallo abbondasse; ma dove il peccato è abbondato, la grazia è sovrabbondata,
\par 21 affinché, come il peccato regnò nella morte, così anche la grazia regni, mediante la giustizia, a vita eterna, per mezzo di Gesù Cristo, nostro Signore.

\chapter{6}

\par 1 Che direm dunque? Rimarremo noi nel peccato onde la grazia abbondi?
\par 2 Così non sia. Noi che siam morti al peccato, come vivremmo ancora in esso?
\par 3 O ignorate voi che quanti siamo stati battezzati in Cristo Gesù, siamo stati battezzati nella sua morte?
\par 4 Noi siam dunque stati con lui seppelliti mediante il battesimo nella sua morte, affinché, come Cristo è risuscitato dai morti mediante la gloria del Padre, così anche noi camminassimo in novità di vita.
\par 5 Perché, se siamo divenuti una stessa cosa con lui per una morte somigliante alla sua, lo saremo anche per una risurrezione simile alla sua, sapendo questo:
\par 6 che il nostro vecchio uomo è stato crocifisso con lui, affinché il corpo del peccato fosse annullato, onde noi non serviamo più al peccato;
\par 7 poiché colui che è morto, è affrancato dal peccato.
\par 8 Ora, se siamo morti con Cristo, noi crediamo che altresì vivremo con lui,
\par 9 sapendo che Cristo, essendo risuscitato dai morti, non muore più; la morte non lo signoreggia più.
\par 10 Poiché il suo morire fu un morire al peccato, una volta per sempre; ma il suo vivere è un vivere a Dio.
\par 11 Così anche voi fate conto d'esser morti al peccato, ma viventi a Dio, in Cristo Gesù.
\par 12 Non regni dunque il peccato nel vostro corpo mortale per ubbidirgli nelle sue concupiscenze;
\par 13 e non prestate le vostre membra come strumenti d'iniquità al peccato; ma presentate voi stessi a Dio come di morti fatti viventi, e le vostre membra come strumenti di giustizia a Dio;
\par 14 perché il peccato non vi signoreggerà, poiché non siete sotto la legge, ma sotto la grazia.
\par 15 Che dunque? Peccheremo noi perché non siamo sotto la legge ma sotto la grazia? Così non sia.
\par 16 Non sapete voi che se vi date a uno come servi per ubbidirgli, siete servi di colui a cui ubbidite: o del peccato che mena alla morte o dell'ubbidienza che mena alla giustizia?
\par 17 Ma sia ringraziato Iddio che eravate bensì servi del peccato, ma avete di cuore ubbidito a quel tenore d'insegnamento che v'è stato trasmesso;
\par 18 ed essendo stati affrancati dal peccato, siete divenuti servi della giustizia.
\par 19 Io parlo alla maniera degli uomini, per la debolezza della vostra carne; poiché, come già prestaste le vostre membra a servizio della impurità e della iniquità per commettere l'iniquità, così prestate ora le vostre membra a servizio della giustizia per la vostra santificazione.
\par 20 Poiché, quando eravate servi del peccato, eravate liberi riguardo alla giustizia.
\par 21 Qual frutto dunque avevate allora delle cose delle quali oggi vi vergognate? poiché la fine loro è la morte.
\par 22 Ma ora, essendo stati affrancati dal peccato e fatti servi a Dio, voi avete per frutto la vostra santificazione, e per fine la vita eterna:
\par 23 poiché il salario del peccato è la morte; ma il dono di Dio è la vita eterna in Cristo Gesù, nostro Signore.

\chapter{7}

\par 1 O ignorate voi, fratelli (poiché io parlo a persone che hanno conoscenza della legge), che la legge signoreggia l'uomo per tutto il tempo ch'egli vive?
\par 2 Infatti la donna maritata è per la legge legata al marito mentre egli vive; ma se il marito muore, ella è sciolta dalla legge che la lega al marito.
\par 3 Ond'è che se mentre vive il marito ella passa ad un altro uomo, sarà chiamata adultera; ma se il marito muore, ella è libera di fronte a quella legge; in guisa che non è adultera se divien moglie d'un altro uomo.
\par 4 Così, fratelli miei, anche voi siete divenuti morti alla legge mediante il corpo di Cristo, per appartenere ad un altro, cioè a colui che è risuscitato dai morti, e questo affinché portiamo del frutto a Dio.
\par 5 Poiché, mentre eravamo nella carne, le passioni peccaminose, destate dalla legge, agivano nelle nostre membra per portar del frutto per la morte;
\par 6 ma ora siamo stati sciolti dai legami della legge, essendo morti a quella che ci teneva soggetti, talché serviamo in novità di spirito, e non in vecchiezza di lettera.
\par 7 Che diremo dunque? La legge è essa peccato? Così non sia; anzi io non avrei conosciuto il peccato, se non per mezzo della legge; poiché io non avrei conosciuto la concupiscenza, se la legge non avesse detto: Non concupire.
\par 8 Ma il peccato, còlta l'occasione, per mezzo del comandamento, produsse in me ogni concupiscenza; perché senza la legge il peccato è morto.
\par 9 E ci fu un tempo, nel quale, senza legge, vivevo; ma, venuto il comandamento, il peccato prese vita, ed io morii;
\par 10 e il comandamento ch'era inteso a darmi vita, risultò che mi dava morte.
\par 11 Perché il peccato, còlta l'occasione, per mezzo del comandamento, mi trasse in inganno; e, per mezzo d'esso, m'uccise.
\par 12 Talché la legge è santa, e il comandamento è santo e giusto e buono.
\par 13 Ciò che è buono diventò dunque morte per me? Così non sia; ma è il peccato che m'è divenuto morte, onde si palesasse come peccato, cagionandomi la morte mediante ciò che è buono; affinché, per mezzo del comandamento, il peccato diventasse estremamente peccante.
\par 14 Noi sappiamo infatti che la legge è spirituale; ma io son carnale, venduto schiavo al peccato.
\par 15 Perché io non approvo quello che faccio; poiché non faccio quel che voglio, ma faccio quello che odio.
\par 16 Ora, se faccio quello che non voglio, io ammetto che la legge è buona;
\par 17 e allora non son più io che lo faccio; ma è il peccato che abita in me.
\par 18 Difatti, io so che in me, vale a dire nella mia carne, non abita alcun bene; poiché ben trovasi in me il volere, ma il modo di compiere il bene, no.
\par 19 Perché il bene che voglio, non lo fo; ma il male che non voglio, quello fo.
\par 20 Ora, se ciò che non voglio è quello che fo, non son più io che lo compio, ma è il peccato che abita in me.
\par 21 Io mi trovo dunque sotto questa legge: che volendo io fare il bene, il male si trova in me.
\par 22 Poiché io mi diletto nella legge di Dio, secondo l'uomo interno;
\par 23 ma veggo un'altra legge nelle mie membra, che combatte contro la legge della mia mente, e mi rende prigione della legge del peccato che è nelle mie membra.
\par 24 Misero me uomo! chi mi trarrà da questo corpo di morte?
\par 25 Grazie siano rese a Dio per mezzo di Gesù Cristo, nostro Signore. Così dunque, io stesso con la mente servo alla legge di Dio, ma con la carne alla legge del peccato.

\chapter{8}

\par 1 Non v'è dunque ora alcuna condanna per quelli che sono in Cristo Gesù;
\par 2 perché la legge dello Spirito della vita in Cristo Gesù mi ha affrancato dalla legge del peccato e della morte.
\par 3 Poiché quel che era impossibile alla legge, perché la carne la rendeva debole, Iddio l'ha fatto; mandando il suo proprio Figliuolo in carne simile a carne di peccato e a motivo del peccato, ha condannato il peccato nella carne,
\par 4 affinché il comandamento della legge fosse adempiuto in noi, che camminiamo non secondo la carne, ma secondo lo spirito.
\par 5 Poiché quelli che son secondo la carne, hanno l'animo alle cose della carne; ma quelli che son secondo lo spirito, hanno l'animo alle cose dello spirito.
\par 6 Perché ciò a cui la carne ha l'animo è morte, ma ciò a cui lo spirito ha l'animo, è vita e pace;
\par 7 poiché ciò a cui la carne ha l'animo è inimicizia contro Dio, perché non è sottomesso alla legge di Dio, e neppure può esserlo;
\par 8 e quelli che sono nella carne, non possono piacere a Dio.
\par 9 Or voi non siete nella carne ma nello spirito, se pur lo Spirito di Dio abita in voi; ma se uno non ha lo Spirito di Cristo, egli non è di lui.
\par 10 E se Cristo è in voi, ben è il corpo morto a cagion del peccato; ma lo spirito è la vita a cagion della giustizia.
\par 11 E se lo Spirito di colui che ha risuscitato Gesù dai morti abita in voi, Colui che ha risuscitato Cristo Gesù dai morti vivificherà anche i vostri corpi mortali per mezzo del suo Spirito che abita in voi.
\par 12 Così dunque, fratelli, noi siam debitori non alla carne per viver secondo la carne;
\par 13 perché se vivete secondo la carne, voi morrete; ma se mediante lo Spirito mortificate gli atti del corpo, voi vivrete;
\par 14 poiché tutti quelli che son condotti dallo Spirito di Dio, son figliuoli di Dio.
\par 15 Poiché voi non avete ricevuto lo spirito di servitù per ricader nella paura; ma avete ricevuto lo spirito d'adozione, per il quale gridiamo: Abba! Padre!
\par 16 Lo Spirito stesso attesta insieme col nostro spirito, che siamo figliuoli di Dio;
\par 17 e se siamo figliuoli, siamo anche eredi; eredi di Dio e coeredi di Cristo, se pur soffriamo con lui, affinché siamo anche glorificati con lui.
\par 18 Perché io stimo che le sofferenze del tempo presente non siano punto da paragonare con la gloria che ha da essere manifestata a nostro riguardo.
\par 19 Poiché la creazione con brama intensa aspetta la manifestazione dei figliuoli di Dio;
\par 20 perché la creazione è stata sottoposta alla vanità, non di sua propria volontà, ma a cagion di colui che ve l'ha sottoposta,
\par 21 non senza speranza però che la creazione stessa sarà anch'ella liberata dalla servitù della corruzione, per entrare nella libertà della gloria dei figliuoli di Dio.
\par 22 Poiché sappiamo che fino ad ora tutta la creazione geme insieme ed è in travaglio;
\par 23 non solo essa, ma anche noi, che abbiamo le primizie dello Spirito, anche noi stessi gemiamo in noi medesimi, aspettando l'adozione, la redenzione del nostro corpo.
\par 24 Poiché noi siamo stati salvati in isperanza. Or la speranza di quel che si vede, non è speranza; difatti, quello che uno vede, perché lo spererebbe egli ancora?
\par 25 Ma se speriamo quel che non vediamo, noi l'aspettiamo con pazienza.
\par 26 Parimente ancora, lo Spirito sovviene alla nostra debolezza; perché noi non sappiamo pregare come si conviene; ma lo Spirito intercede egli stesso per noi con sospiri ineffabili;
\par 27 e Colui che investiga i cuori conosce qual sia il sentimento dello Spirito, perché esso intercede per i santi secondo Iddio.
\par 28 Or noi sappiamo che tutte le cose cooperano al bene di quelli che amano Dio, i quali son chiamati secondo il suo proponimento.
\par 29 Perché quelli che Egli ha preconosciuti, li ha pure predestinati ad esser conformi all'immagine del suo Figliuolo, ond'egli sia il primogenito fra molti fratelli;
\par 30 e quelli che ha predestinati, li ha pure chiamati; e quelli che ha chiamati, li ha pure giustificati; e quelli che ha giustificati, li ha pure glorificati.
\par 31 Che diremo dunque a queste cose? Se Dio è per noi, chi sarà contro di noi?
\par 32 Colui che non ha risparmiato il suo proprio Figliuolo, ma l'ha dato per tutti noi, come non ci donerà egli anche tutte le cose con lui?
\par 33 Chi accuserà gli eletti di Dio? Iddio è quel che li giustifica.
\par 34 Chi sarà quel che li condanni? Cristo Gesù è quel che è morto; e, più che questo, è risuscitato; ed è alla destra di Dio; ed anche intercede per noi.
\par 35 Chi ci separerà dall'amore di Cristo? Sarà forse la tribolazione, o la distretta, o la persecuzione, o la fame, o la nudità, o il pericolo, o la spada?
\par 36 Come è scritto: Per amor di te noi siamo tutto il giorno messi a morte; siamo stati considerati come pecore da macello.
\par 37 Anzi, in tutte queste cose, noi siam più che vincitori, in virtù di colui che ci ha amati.
\par 38 Poiché io son persuaso che né morte, né vita, né angeli, né principati, né cose presenti, né cose future,
\par 39 né potestà, né altezza, né profondità, né alcun'altra creatura potranno separarci dall'amore di Dio, che è in Cristo Gesù, nostro Signore.

\chapter{9}

\par 1 Io dico la verità in Cristo, non mento, la mia coscienza me lo attesta per lo Spirito Santo:
\par 2 io ho una grande tristezza e un continuo dolore nel cuor mio;
\par 3 perché vorrei essere io stesso anatema, separato da Cristo, per amor dei miei fratelli, miei parenti secondo la carne,
\par 4 che sono Israeliti, ai quali appartengono l'adozione e la gloria e i patti e la legislazione e il culto e le promesse;
\par 5 dei quali sono i padri, e dai quali è venuto, secondo la carne, il Cristo, che è sopra tutte le cose Dio benedetto in eterno. Amen.
\par 6 Però non è che la parola di Dio sia caduta a terra; perché non tutti i discendenti da Israele sono Israele;
\par 7 né per il fatto che son progenie d'Abramo, son tutti figliuoli d'Abramo; anzi: In Isacco ti sarà nominata una progenie.
\par 8 Cioè, non i figliuoli della carne sono figliuoli di Dio: ma i figliuoli della promessa son considerati come progenie.
\par 9 Poiché questa è una parola di promessa: In questa stagione io verrò, e Sara avrà un figliuolo.
\par 10 Non solo; ma anche a Rebecca avvenne la medesima cosa quand'ebbe concepito da uno stesso uomo, vale a dire Isacco nostro padre, due gemelli;
\par 11 poiché, prima che fossero nati e che avessero fatto alcun che di bene o di male, affinché rimanesse fermo il proponimento dell'elezione di Dio, che dipende non dalle opere ma dalla volontà di colui che chiama,
\par 12 le fu detto: Il maggiore servirà al minore;
\par 13 secondo che è scritto: Ho amato Giacobbe, ma ho odiato Esaù.
\par 14 Che diremo dunque? V'è forse ingiustizia in Dio? Così non sia.
\par 15 Poiché Egli dice a Mosè: Io avrò mercé di chi avrò mercé, e avrò compassione di chi avrò compassione.
\par 16 Non dipende dunque né da chi vuole né da chi corre, ma da Dio che fa misericordia.
\par 17 Poiché la Scrittura dice a Faraone: Appunto per questo io t'ho suscitato: per mostrare in te la mia potenza, e perché il mio nome sia pubblicato per tutta la terra.
\par 18 Così dunque Egli fa misericordia a chi vuole, e indura chi vuole.
\par 19 Tu allora mi dirai: Perché si lagna Egli ancora? Poiché chi può resistere alla sua volontà?
\par 20 Piuttosto, o uomo, chi sei tu che replichi a Dio? La cosa formata dirà essa a colui che la formò: Perché mi facesti così?
\par 21 Il vasaio non ha egli potestà sull'argilla, da trarre dalla stessa massa un vaso per uso nobile, e un altro per uso ignobile?
\par 22 E che v'è mai da replicare se Dio, volendo mostrare la sua ira e far conoscere la sua potenza, ha sopportato con molta longanimità de' vasi d'ira preparati per la perdizione,
\par 23 e se, per far conoscere le ricchezze della sua gloria verso de' vasi di misericordia che avea già innanzi preparati per la gloria,
\par 24 li ha anche chiamati (parlo di noi) non soltanto fra i Giudei ma anche di fra i Gentili?
\par 25 Così Egli dice anche in Osea: Io chiamerò mio popolo quello che non era mio popolo, e 'amata' quella che non era amata;
\par 26 e avverrà che nel luogo ov'era loro stato detto: 'Voi non siete mio popolo', quivi saran chiamati figliuoli dell'Iddio vivente.
\par 27 E Isaia esclama riguardo a Israele: Quand'anche il numero dei figliuoli d'Israele fosse come la rena del mare, il rimanente solo sarà salvato;
\par 28 perché il Signore eseguirà la sua parola sulla terra, in modo definitivo e reciso.
\par 29 E come Isaia avea già detto prima: Se il Signor degli eserciti non ci avesse lasciato un seme, saremmo divenuti come Sodoma e saremmo stati simili a Gomorra.
\par 30 Che diremo dunque? Diremo che i Gentili, i quali non cercavano la giustizia, hanno conseguito la giustizia, ma la giustizia che vien dalla fede;
\par 31 mentre Israele, che cercava la legge della giustizia, non ha conseguito la legge della giustizia.
\par 32 Perché? Perché l'ha cercata non per fede, ma per opere. Essi hanno urtato nella pietra d'intoppo,
\par 33 siccome è scritto: Ecco, io pongo in Sion una pietra d'intoppo e una roccia d'inciampo; ma chi crede in lui non sarà svergognato.

\chapter{10}

\par 1 Fratelli, il desiderio del mio cuore e la mia preghiera a Dio per loro è che siano salvati.
\par 2 Poiché io rendo loro testimonianza che hanno zelo per le cose di Dio, ma zelo senza conoscenza.
\par 3 Perché, ignorando la giustizia di Dio, e cercando di stabilir la loro propria, non si son sottoposti alla giustizia di Dio;
\par 4 poiché il termine della legge è Cristo, per esser giustizia ad ognuno che crede.
\par 5 Infatti Mosè descrive così la giustizia che vien dalla legge: L'uomo che farà quelle cose, vivrà per esse.
\par 6 Ma la giustizia che vien dalla fede dice così: Non dire in cuor tuo: Chi salirà in cielo? (questo è un farne scendere Cristo) né:
\par 7 Chi scenderà nell'abisso? (questo è un far risalire Cristo d'infra i morti).
\par 8 Ma che dice ella? La parola è presso di te, nella tua bocca e nel tuo cuore; questa è la parola della fede che noi predichiamo;
\par 9 perché, se con la bocca avrai confessato Gesù come Signore, e avrai creduto col cuore che Dio l'ha risuscitato dai morti, sarai salvato;
\par 10 infatti col cuore si crede per ottener la giustizia e con la bocca si fa confessione per esser salvati.
\par 11 Difatti la Scrittura dice: Chiunque crede in lui, non sarà svergognato.
\par 12 Poiché non v'è distinzione fra Giudeo e Greco; perché lo stesso Signore è Signore di tutti, ricco verso tutti quelli che lo invocano;
\par 13 poiché chiunque avrà invocato il nome del Signore, sarà salvato.
\par 14 Come dunque invocheranno colui nel quale non hanno creduto? E come crederanno in colui del quale non hanno udito parlare? E come udiranno, se non v'è chi predichi?
\par 15 E come predicheranno se non son mandati? Siccome è scritto: Quanto son belli i piedi di quelli che annunziano buone novelle!
\par 16 Ma tutti non hanno ubbidito alla Buona Novella; perché Isaia dice: Signore, chi ha creduto alla nostra predicazione?
\par 17 Così la fede vien dall'udire e l'udire si ha per mezzo della parola di Cristo.
\par 18 Ma io dico: Non hanno essi udito? Anzi, la loro voce è andata per tutta la terra, e le loro parole fino agli estremi confini del mondo.
\par 19 Ma io dico: Israele non ha egli compreso? Mosè pel primo dice: Io vi moverò a gelosia di una nazione che non è nazione; contro una nazione senza intelletto provocherò il vostro sdegno.
\par 20 E Isaia si fa ardito e dice: Sono stato trovato da quelli che non mi cercavano; sono stato chiaramente conosciuto da quelli che non chiedevan di me.
\par 21 Ma riguardo a Israele dice: Tutto il giorno ho teso le mani verso un popolo disubbidiente e contradicente.

\chapter{11}

\par 1 Io dico dunque: Iddio ha egli reietto il suo popolo? Così non sia; perché anch'io sono Israelita, della progenie d'Abramo, della tribù di Beniamino.
\par 2 Iddio non ha reietto il suo popolo, che ha preconosciuto. Non sapete voi quel che la Scrittura dice, nella storia d'Elia? Com'egli ricorre a Dio contro Israele, dicendo:
\par 3 Signore, hanno ucciso i tuoi profeti, hanno demoliti i tuoi altari, e io son rimasto solo, e cercano la mia vita?
\par 4 Ma che gli rispose la voce divina? Mi son riserbato settemila uomini, che non han piegato il ginocchio davanti a Baal.
\par 5 E così anche nel tempo presente, v'è un residuo secondo l'elezione della grazia.
\par 6 Ma se è per grazia, non è più per opere; altrimenti, grazia non è più grazia.
\par 7 Che dunque? Quel che Israele cerca, non l'ha ottenuto; mentre il residuo eletto l'ha ottenuto;
\par 8 e gli altri sono stati indurati, secondo che è scritto: Iddio ha dato loro uno spirito di stordimento, degli occhi per non vedere e degli orecchi per non udire, fino a questo giorno.
\par 9 E Davide dice: La loro mensa sia per loro un laccio, una rete, un inciampo, e una retribuzione.
\par 10 Siano gli occhi loro oscurati in guisa che non veggano, e piega loro del continuo la schiena.
\par 11 Io dico dunque: Hanno essi così inciampato da cadere? Così non sia; ma per la loro caduta la salvezza è giunta ai Gentili per provocar loro a gelosia.
\par 12 Ora se la loro caduta è la ricchezza del mondo e la loro diminuzione la ricchezza de' Gentili, quanto più lo sarà la loro pienezza!
\par 13 Ma io parlo a voi, o Gentili: In quanto io sono apostolo dei Gentili, glorifico il mio ministerio,
\par 14 per veder di provocare a gelosia quelli del mio sangue e di salvarne alcuni.
\par 15 Poiché, se la loro reiezione è la riconciliazione del mondo, che sarà la loro riammissione, se non una vita d'infra i morti?
\par 16 E se la primizia è santa, anche la massa è santa; e se la radice è santa, anche i rami son santi.
\par 17 E se pure alcuni de' rami sono stati troncati, e tu, che sei olivastro, sei stato innestato in luogo loro e sei divenuto partecipe della radice e della grassezza dell'ulivo,
\par 18 non t'insuperbire contro ai rami; ma, se t'insuperbisci, sappi che non sei tu che porti la radice, ma la radice che porta te.
\par 19 Allora tu dirai: Sono stati troncati dei rami perché io fossi innestato.
\par 20 Bene: sono stati troncati per la loro incredulità, e tu sussisti per la fede; non t'insuperbire, ma temi.
\par 21 Perché se Dio non ha risparmiato i rami naturali, non risparmierà neppur te.
\par 22 Vedi dunque la benignità e la severità di Dio; la severità verso quelli che son caduti; ma verso te la benignità di Dio, se pur tu perseveri nella sua benignità; altrimenti, anche tu sarai reciso.
\par 23 Ed anche quelli, se non perseverano nella loro incredulità, saranno innestati; perché Dio è potente da innestarli di nuovo.
\par 24 Poiché se tu sei stato tagliato dall'ulivo per sua natura selvatico, e sei stato contro natura innestato nell'ulivo domestico, quanto più essi, che son dei rami naturali, saranno innestati nel loro proprio ulivo?
\par 25 Perché, fratelli, non voglio che ignorate questo mistero, affinché non siate presuntuosi; che cioè, un induramento parziale s'è prodotto in Israele, finché sia entrata la pienezza dei Gentili;
\par 26 e così tutto Israele sarà salvato, secondo che è scritto: Il liberatore verrà da Sion;
\par 27 Egli allontanerà da Giacobbe l'empietà; e questo sarà il mio patto con loro, quand'io torrò via i loro peccati.
\par 28 Per quanto concerne l'Evangelo, essi sono nemici per via di voi; ma per quanto concerne l'elezione, sono amati per via dei loro padri;
\par 29 perché i doni e la vocazione di Dio sono senza pentimento.
\par 30 Poiché, siccome voi siete stati in passato disubbidienti a Dio ma ora avete ottenuto misericordia per la loro disubbidienza,
\par 31 così anch'essi sono stati ora disubbidienti, onde, per la misericordia a voi usata, ottengano essi pure misericordia.
\par 32 Poiché Dio ha rinchiuso tutti nella disubbidienza per far misericordia a tutti.
\par 33 O profondità della ricchezza e della sapienza e della conoscenza di Dio! Quanto inscrutabili sono i suoi giudizî, e incomprensibili le sue vie!
\par 34 Poiché: Chi ha conosciuto il pensiero del Signore? O chi è stato il suo consigliere?
\par 35 O chi gli ha dato per il primo, e gli sarà contraccambiato?
\par 36 Poiché da lui, per mezzo di lui e per lui son tutte le cose. A lui sia la gloria in eterno. Amen.

\chapter{12}

\par 1 Io vi esorto dunque, fratelli, per le compassioni di Dio, a presentare i vostri corpi in sacrificio vivente, santo, accettevole a Dio; il che è il vostro culto spirituale.
\par 2 E non vi conformate a questo secolo, ma siate trasformati mediante il rinnovamento della vostra mente, affinché conosciate per esperienza qual sia la volontà di Dio, la buona, accettevole e perfetta volontà.
\par 3 Per la grazia che m'è stata data, io dico quindi a ciascuno fra voi che non abbia di sé un concetto più alto di quel che deve avere, ma abbia di sé un concetto sobrio, secondo la misura della fede che Dio ha assegnata a ciascuno.
\par 4 Poiché, siccome in un solo corpo abbiamo molte membra e tutte le membra non hanno un medesimo ufficio,
\par 5 così noi, che siamo molti, siamo un solo corpo in Cristo, e, individualmente, siamo membra l'uno dell'altro.
\par 6 E siccome abbiamo dei doni differenti secondo la grazia che ci è stata data, se abbiamo dono di profezia, profetizziamo secondo la proporzione della nostra fede;
\par 7 se di ministerio, attendiamo al ministerio; se d'insegnamento, all'insegnare;
\par 8 se di esortazione, all'esortare; chi dà, dia con semplicità; chi presiede, lo faccia con diligenza; chi fa opere pietose, le faccia con allegrezza.
\par 9 L'amore sia senza ipocrisia. Aborrite il male, e attenetevi fermamente al bene.
\par 10 Quanto all'amor fraterno, siate pieni d'affezione gli uni per gli altri; quanto all'onore, prevenitevi gli uni gli altri;
\par 11 quanto allo zelo, non siate pigri; siate ferventi nello spirito, servite il Signore;
\par 12 siate allegri nella speranza, pazienti nell'afflizione, perseveranti nella preghiera;
\par 13 provvedete alle necessità dei santi, esercitate con premura l'ospitalità.
\par 14 Benedite quelli che vi perseguitano; benedite e non maledite.
\par 15 Rallegratevi con quelli che sono allegri; piangete con quelli che piangono.
\par 16 Abbiate fra voi un medesimo sentimento; non abbiate l'animo alle cose alte, ma lasciatevi attirare dalle umili. Non vi stimate savî da voi stessi.
\par 17 Non rendete ad alcuno male per male. Applicatevi alle cose che sono oneste, nel cospetto di tutti gli uomini.
\par 18 Se è possibile, per quanto dipende da voi, vivete in pace con tutti gli uomini.
\par 19 Non fate le vostre vendette, cari miei, ma cedete il posto all'ira di Dio; poiché sta scritto: A me la vendetta; io darò la retribuzione, dice il Signore.
\par 20 Anzi, se il tuo nemico ha fame, dagli da mangiare; se ha sete, dagli da bere; poiché, facendo così, tu raunerai dei carboni accesi sul suo capo.
\par 21 Non esser vinto dal male, ma vinci il male col bene.

\chapter{13}

\par 1 Ogni persona sia sottoposta alle autorità superiori; perché non v'è autorità se non da Dio; e le autorità che esistono, sono ordinate da Dio;
\par 2 talché chi resiste all'autorità, si oppone all'ordine di Dio; e quelli che vi si oppongono, si attireranno addosso una pena;
\par 3 poiché i magistrati non son di spavento alle opere buone, ma alle cattive. Vuoi tu non aver paura dell'autorità? Fa' quel ch'è bene, e avrai lode da essa;
\par 4 perché il magistrato è un ministro di Dio per il tuo bene; ma se fai quel ch'è male, temi, perché egli non porta la spada invano; poich'egli è un ministro di Dio, per infliggere una giusta punizione contro colui che fa il male.
\par 5 Perciò è necessario star soggetti non soltanto a motivo della punizione, ma anche per motivo di coscienza.
\par 6 Poiché è anche per questa ragione che voi pagate i tributi; perché si tratta di ministri di Dio, i quali attendono del continuo a quest'ufficio.
\par 7 Rendete a tutti quel che dovete loro: il tributo a chi dovete il tributo; la gabella a chi la gabella; il timore a chi il timore; l'onore a chi l'onore.
\par 8 Non abbiate altro debito con alcuno se non d'amarvi gli uni gli altri; perché chi ama il prossimo ha adempiuto la legge.
\par 9 Infatti il non commettere adulterio, non uccidere, non rubare, non concupire e qualsiasi altro comandamento si riassumono in questa parola: Ama il tuo prossimo come te stesso.
\par 10 L'amore non fa male alcuno al prossimo; l'amore, quindi, è l'adempimento della legge.
\par 11 E questo tanto più dovete fare, conoscendo il tempo nel quale siamo; poiché è ora ormai che vi svegliate dal sonno; perché la salvezza ci è adesso più vicina di quando credemmo.
\par 12 La notte è avanzata, il giorno è vicino; gettiam dunque via le opere delle tenebre, e indossiamo le armi della luce.
\par 13 Camminiamo onestamente, come di giorno; non in gozzoviglie ed ebbrezze; non in lussuria e lascivie; non in contese ed invidie;
\par 14 ma rivestitevi del Signor Gesù Cristo, e non abbiate cura della carne per soddisfarne le concupiscenze.

\chapter{14}

\par 1 Quanto a colui che è debole nella fede, accoglietelo, ma non per discutere opinioni.
\par 2 L'uno crede di poter mangiare di tutto, mentre l'altro, che è debole, mangia legumi.
\par 3 Colui che mangia di tutto, non sprezzi colui che non mangia di tutto; e colui che non mangia di tutto, non giudichi colui che mangia di tutto; perché Dio l'ha accolto.
\par 4 Chi sei tu che giudichi il domestico altrui? Se sta in piedi o se cade è cosa che riguarda il suo padrone; ma egli sarà tenuto in piè, perché il Signore è potente da farlo stare in piè.
\par 5 L'uno stima un giorno più d'un altro; l'altro stima tutti i giorni uguali; sia ciascuno pienamente convinto nella propria mente.
\par 6 Chi ha riguardo al giorno, lo fa per il Signore; e chi mangia di tutto, lo fa per il Signore, poiché rende grazie a Dio; e chi non mangia di tutto fa così per il Signore, e rende grazie a Dio.
\par 7 Poiché nessuno di noi vive per se stesso, e nessuno muore per se stesso;
\par 8 perché, se viviamo, viviamo per il Signore; e se moriamo, moriamo per il Signore; sia dunque che viviamo o che moriamo, noi siamo del Signore.
\par 9 Poiché a questo fine Cristo è morto ed è tornato in vita: per essere il Signore e de' morti e de' viventi.
\par 10 Ma tu, perché giudichi il tuo fratello? E anche tu perché disprezzi il tuo fratello? Poiché tutti compariremo davanti al tribunale di Dio;
\par 11 infatti sta scritto: Com'io vivo, dice il Signore, ogni ginocchio si piegherà davanti a me, ed ogni lingua darà gloria a Dio.
\par 12 Così dunque ciascun di noi renderà conto di se stesso a Dio.
\par 13 Non ci giudichiamo dunque più gli uni gli altri, ma giudicate piuttosto che non dovete porre pietra d'inciampo sulla via del fratello, né essergli occasion di caduta.
\par 14 Io so e son persuaso nel Signor Gesù che nessuna cosa è impura in se stessa; però se uno stima che una cosa è impura, per lui è impura.
\par 15 Ora, se a motivo di un cibo il tuo fratello è contristato, tu non procedi più secondo carità. Non perdere, col tuo cibo, colui per il quale Cristo è morto!
\par 16 Il privilegio che avete, non sia dunque oggetto di biasimo;
\par 17 perché il regno di Dio non consiste in vivanda né in bevanda, ma è giustizia, pace ed allegrezza nello Spirito Santo.
\par 18 Poiché chi serve in questo a Cristo, è gradito a Dio e approvato dagli uomini.
\par 19 Cerchiamo dunque le cose che contribuiscono alla pace e alla mutua edificazione.
\par 20 Non disfare, per un cibo, l'opera di Dio. Certo, tutte le cose son pure ma è male quand'uno mangia dando intoppo.
\par 21 È bene non mangiar carne, né bever vino, né far cosa alcuna che possa esser d'intoppo al fratello.
\par 22 Tu, la convinzione che hai, serbala per te stesso dinanzi a Dio. Beato colui che non condanna se stesso in quello che approva.
\par 23 Ma colui che sta in dubbio, se mangia è condannato, perché non mangia con convinzione; e tutto quello che non vien da convinzione è peccato.

\chapter{15}

\par 1 Or noi che siam forti, dobbiam sopportare le debolezze de' deboli e non compiacere a noi stessi.
\par 2 Ciascun di noi compiaccia al prossimo nel bene, a scopo di edificazione.
\par 3 Poiché anche Cristo non compiacque a se stesso; ma, com'è scritto: Gli oltraggi di quelli che ti oltraggiano son caduti sopra di me.
\par 4 Perché tutto quello che fu scritto per l'addietro, fu scritto per nostro ammaestramento, affinché mediante la pazienza e mediante la consolazione delle Scritture, noi riteniamo la speranza.
\par 5 Or l'Iddio della pazienza e della consolazione vi dia d'aver fra voi un medesimo sentimento secondo Cristo Gesù,
\par 6 affinché d'un solo animo e d'una stessa bocca glorifichiate Iddio, il Padre del nostro Signor Gesù Cristo.
\par 7 Perciò accoglietevi gli uni gli altri, siccome anche Cristo ha accolto noi per la gloria di Dio;
\par 8 poiché io dico che Cristo è stato fatto ministro de' circoncisi, a dimostrazione della veracità di Dio, per confermare le promesse fatte ai padri;
\par 9 mentre i Gentili hanno da glorificare Iddio per la sua misericordia, secondo che è scritto: Per questo ti celebrerò fra i Gentili e salmeggerò al tuo nome.
\par 10 Ed è detto ancora: Rallegratevi, o Gentili, col suo popolo.
\par 11 E altrove: Gentili, lodate tutti il Signore, e tutti i popoli lo celebrino.
\par 12 E di nuovo Isaia dice: Vi sarà la radice di Iesse, e Colui che sorgerà a governare i Gentili; in lui spereranno i Gentili.
\par 13 Or l'Iddio della speranza vi riempia d'ogni allegrezza e d'ogni pace nel vostro credere, onde abbondiate nella speranza, mediante la potenza dello Spirito Santo.
\par 14 Ora, fratelli miei, sono io pure persuaso, a riguardo vostro, che anche voi siete pieni di bontà, ricolmi d'ogni conoscenza, capaci anche d'ammonirvi a vicenda.
\par 15 Ma vi ho scritto alquanto arditamente, come per ricordarvi quel che già sapete, a motivo della grazia che mi è stata fatta da Dio,
\par 16 d'esser ministro di Cristo Gesù per i Gentili, esercitando il sacro servigio del Vangelo di Dio, affinché l'offerta de' Gentili sia accettevole, essendo santificata dallo Spirito Santo.
\par 17 Io ho dunque di che gloriarmi in Cristo Gesù, per quel che concerne le cose di Dio;
\par 18 perché io non ardirei dir cosa che Cristo non abbia operata per mio mezzo, in vista dell'ubbidienza de' Gentili, in parola e in opera,
\par 19 con potenza di segni e di miracoli, con potenza dello Spirito Santo. Così da Gerusalemme e dai luoghi intorno fino all'Illiria, ho predicato dovunque l'Evangelo di Cristo,
\par 20 avendo l'ambizione di predicare l'Evangelo là dove Cristo non fosse già stato nominato, per non edificare sul fondamento altrui;
\par 21 come è scritto: Coloro ai quali nulla era stato annunziato di lui, lo vedranno; e coloro che non ne avevano udito parlare, intenderanno.
\par 22 Per questa ragione appunto sono stato le tante volte impedito di venire a voi;
\par 23 ma ora, non avendo più campo da lavorare in queste contrade, e avendo già da molti anni gran desiderio di recarmi da voi,
\par 24 quando andrò in Ispagna, spero, passando, di vedervi e d'esser da voi aiutato nel mio viaggio a quella volta, dopo che mi sarò in parte saziato di voi.
\par 25 Ma per ora vado a Gerusalemme a portarvi una sovvenzione per i santi;
\par 26 perché la Macedonia e l'Acaia si son compiaciute di raccogliere una contribuzione a pro dei poveri fra i santi che sono in Gerusalemme.
\par 27 Si sono compiaciute, dico; ed è anche un debito ch'esse hanno verso di loro; perché se i Gentili sono stati fatti partecipi dei loro beni spirituali, sono anche in obbligo di sovvenir loro con i beni materiali.
\par 28 Quando dunque avrò compiuto questo servizio e consegnato questo frutto, andrò in Ispagna passando da voi;
\par 29 e so che, recandomi da voi, verrò con la pienezza delle benedizioni di Cristo.
\par 30 Ora, fratelli, io v'esorto per il Signor nostro Gesù Cristo e per la carità dello Spirito, a combatter meco nelle vostre preghiere a Dio per me,
\par 31 affinché io sia liberato dai disubbidienti di Giudea, e la sovvenzione che porto a Gerusalemme sia accettevole ai santi,
\par 32 in modo che, se piace a Dio, io possa recarmi da voi con allegrezza e possa con voi ricrearmi.
\par 33 Or l'Iddio della pace sia con tutti voi. Amen.

\chapter{16}

\par 1 Vi raccomando Febe, nostra sorella, che è diaconessa della chiesa di Cencrea,
\par 2 perché la riceviate nel Signore, in modo degno dei santi, e le prestiate assistenza, in qualunque cosa ella possa aver bisogno di voi; poiché ella pure ha prestato assistenza a molti e anche a me stesso.
\par 3 Salutate Prisca ed Aquila, miei compagni d'opera in Cristo Gesù,
\par 4 i quali per la vita mia hanno esposto il loro proprio collo; ai quali non io solo ma anche tutte le chiese dei Gentili rendono grazie.
\par 5 Salutate anche la chiesa che è in casa loro. Salutate il mio caro Epeneto, che è la primizia dell'Asia per Cristo.
\par 6 Salutate Maria, che si è molto affaticata per voi.
\par 7 Salutate Andronico e Giunio, miei parenti e compagni di prigione, i quali sono segnalati fra gli apostoli, e anche sono stati in Cristo prima di me.
\par 8 Salutate Ampliato, il mio diletto nel Signore.
\par 9 Salutate Urbano, nostro compagno d'opera in Cristo, e il mio caro Stachi.
\par 10 Salutate Apelle, che ha fatto le sue prove in Cristo. Salutate que' di casa di Aristobulo.
\par 11 Salutate Erodione, mio parente. Salutate que' di casa di Narcisso che sono nel Signore.
\par 12 Salutate Trifena e Trifosa, che si affaticano nel Signore. Salutate la cara Perside che si è molto affaticata nel Signore.
\par 13 Salutate Rufo, l'eletto nel Signore, e sua madre che è pur mia.
\par 14 Salutate Asincrito, Flegonte, Erme, Patroba, Erma, e i fratelli che son con loro.
\par 15 Salutate Filologo e Giulia, Nereo e sua sorella, e Olimpia, e tutti i santi che son con loro.
\par 16 Salutatevi gli uni gli altri con un santo bacio. Tutte le chiese di Cristo vi salutano.
\par 17 Or io v'esorto, fratelli, tenete d'occhio quelli che fomentano le dissensioni e gli scandali contro l'insegnamento che avete ricevuto, e ritiratevi da loro.
\par 18 Poiché quei tali non servono al nostro Signor Gesù Cristo, ma al proprio ventre; e con dolce e lusinghiero parlare seducono il cuore de' semplici.
\par 19 Quanto a voi, la vostra ubbidienza è giunta a conoscenza di tutti. Io dunque mi rallegro per voi, ma desidero che siate savî nel bene e semplici per quel che concerne il male.
\par 20 E l'Iddio della pace triterà tosto Satana sotto ai vostri piedi. La grazia del Signor nostro Gesù Cristo sia con voi.
\par 21 Timoteo, mio compagno d'opera, vi saluta, e vi salutano pure Lucio, Giasone e Sosipatro, miei parenti.
\par 22 Io, Terzio, che ho scritto l'epistola, vi saluto nel Signore.
\par 23 Gaio, che ospita me e tutta la chiesa, vi saluta. Erasto, il tesoriere della città, e il fratello Quarto vi salutano.
\par 24 tex
\par 25 Or a Colui che vi può fortificare secondo il mio Evangelo e la predicazione di Gesù Cristo, conformemente alla rivelazione del mistero che fu tenuto occulto fin dai tempi più remoti
\par 26 ma è ora manifestato, e, mediante le Scritture profetiche, secondo l'ordine dell'eterno Iddio, è fatto conoscere a tutte le nazioni per addurle all'ubbidienza della fede,
\par 27 a Dio solo savio, per mezzo di Gesù Cristo, sia la gloria nei secoli dei secoli. Amen.


\end{document}