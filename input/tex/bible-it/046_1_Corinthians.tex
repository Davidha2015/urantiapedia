\begin{document}

\title{1 Corinthians}


\chapter{1}

\par 1 Paolo, chiamato ad essere apostolo di Cristo Gesù per la volontà di Dio, e il fratello Sostene,
\par 2 alla chiesa di Dio che è in Corinto, ai santificati in Cristo Gesù, chiamati ad esser santi, con tutti quelli che in ogni luogo invocano il nome del Signor nostro Gesù Cristo, Signor loro e nostro,
\par 3 grazia a voi e pace da Dio nostro Padre e dal Signor Cristo.
\par 4 Io rendo del continuo grazie all'Iddio mio per voi della grazia di Dio che vi è stata data in Cristo Gesù;
\par 5 perché in lui siete stati arricchiti in ogni cosa, in ogni dono di parola e in ogni conoscenza,
\par 6 essendo stata la testimonianza di Cristo confermata tra voi;
\par 7 in guisa che non difettate d'alcun dono, mentre aspettate la manifestazione del Signor nostro Gesù Cristo,
\par 8 il quale anche vi confermerà sino alla fine, onde siate irreprensibili nel giorno del nostro Signore Gesù Cristo.
\par 9 Fedele è l'Iddio dal quale siete stati chiamati alla comunione del suo Figliuolo Gesù Cristo, nostro Signore.
\par 10 Ora, fratelli, io v'esorto, per il nome del nostro Signor Gesù Cristo, ad aver tutti un medesimo parlare, e a non aver divisioni fra voi, ma a stare perfettamente uniti in una medesima mente e in un medesimo sentire.
\par 11 Perché, fratelli miei, m'è stato riferito intorno a voi da quei di casa Cloe, che vi son fra voi delle contese.
\par 12 Voglio dire che ciascun di voi dice: Io son di Paolo; e io d'Apollo; e io di Cefa; e io di Cristo.
\par 13 Cristo è egli diviso? Paolo è egli stato crocifisso per voi? O siete voi stati battezzati nel nome di Paolo?
\par 14 Io ringrazio Dio che non ho battezzato alcun di voi, salvo Crispo e Gaio;
\par 15 cosicché nessuno può dire che foste battezzati nel mio nome.
\par 16 Ho battezzato anche la famiglia di Stefana; del resto non so se ho battezzato alcun altro.
\par 17 Perché Cristo non mi ha mandato a battezzare ma ad evangelizzare; non con sapienza di parola, affinché la croce di Cristo non sia resa vana.
\par 18 Poiché la parola della croce è pazzia per quelli che periscono; ma per noi che siam sulla via della salvazione, è la potenza di Dio; poich'egli è scritto:
\par 19 Io farò perire la sapienza dei savî, e annienterò l'intelligenza degli intelligenti.
\par 20 Dov'è il savio? Dov'è lo scriba? Dov'è il disputatore di questo secolo? Iddio non ha egli resa pazza la sapienza di questo mondo?
\par 21 Poiché, visto che nella sapienza di Dio il mondo non ha conosciuto Dio con la propria sapienza, è piaciuto a Dio di salvare i credenti mediante la pazzia della predicazione.
\par 22 Poiché i Giudei chiedon de' miracoli, e i Greci cercan sapienza;
\par 23 ma noi predichiamo Cristo crocifisso, che per i Giudei è scandalo, e per i Gentili, pazzia;
\par 24 ma per quelli i quali son chiamati, tanto Giudei quanto Greci, predichiamo Cristo, potenza di Dio e sapienza di Dio;
\par 25 poiché la pazzia di Dio è più savia degli uomini, e la debolezza di Dio è più forte degli uomini.
\par 26 Infatti, fratelli, guardate la vostra vocazione: non ci son tra voi molti savî secondo la carne, non molti potenti, non molti nobili;
\par 27 ma Dio ha scelto le cose pazze del mondo per svergognare i savî; e Dio ha scelto le cose deboli del mondo per svergognare le forti;
\par 28 e Dio ha scelto le cose ignobili del mondo, e le cose sprezzate, anzi le cose che non sono, per ridurre al niente le cose che sono,
\par 29 affinché nessuna carne si glorî nel cospetto di Dio.
\par 30 E a lui voi dovete d'essere in Cristo Gesù, il quale ci è stato fatto da Dio sapienza, e giustizia, e santificazione, e redenzione,
\par 31 affinché, com'è scritto: Chi si gloria, si glorî nel Signore.

\chapter{2}

\par 1 Quant'è a me, fratelli, quando venni a voi, non venni ad annunziarvi la testimonianza di Dio con eccellenza di parola o di sapienza;
\par 2 poiché mi proposi di non saper altro fra voi, fuorché Gesù Cristo e lui crocifisso.
\par 3 Ed io sono stato presso di voi con debolezza, e con timore, e con gran tremore;
\par 4 e la mia parola e la mia predicazione non hanno consistito in discorsi persuasivi di sapienza umana, ma in dimostrazione di Spirito e di potenza,
\par 5 affinché la vostra fede fosse fondata non sulla sapienza degli uomini, ma sulla potenza di Dio.
\par 6 Nondimeno fra quelli che son maturi noi esponiamo una sapienza, una sapienza però non di questo secolo né de' principi di questo secolo che stan per essere annientati,
\par 7 ma esponiamo la sapienza di Dio misteriosa ed occulta che Dio avea innanzi i secoli predestinata a nostra gloria,
\par 8 e che nessuno de' principi di questo mondo ha conosciuta; perché, se l'avessero conosciuta, non avrebbero crocifisso il Signor della gloria.
\par 9 Ma, com'è scritto: Le cose che occhio non ha vedute, e che orecchio non ha udite e che non son salite in cuor d'uomo, son quelle che Dio ha preparate per coloro che l'amano.
\par 10 Ma a noi Dio le ha rivelate per mezzo dello Spirito; perché lo spirito investiga ogni cosa, anche le cose profonde di Dio.
\par 11 Infatti, chi, fra gli uomini, conosce le cose dell'uomo se non lo spirito dell'uomo che è in lui? E così nessuno conosce le cose di Dio, se non lo Spirito di Dio.
\par 12 Or noi abbiam ricevuto non lo spirito del mondo, ma lo Spirito che vien da Dio, affinché conosciamo le cose che ci sono state donate da Dio;
\par 13 e noi ne parliamo non con parole insegnate dalla sapienza umana, ma insegnate dallo Spirito, adattando parole spirituali a cose spirituali.
\par 14 Or l'uomo naturale non riceve le cose dello Spirito di Dio, perché gli sono pazzia; e non le può conoscere, perché le si giudicano spiritualmente.
\par 15 Ma l'uomo spirituale giudica d'ogni cosa, ed egli stesso non è giudicato da alcuno.
\par 16 Poiché chi ha conosciuto la mente del Signore da poterlo ammaestrare? Ma noi abbiamo la mente di Cristo.

\chapter{3}

\par 1 Ed io, fratelli, non ho potuto parlarvi come a spirituali, ma ho dovuto parlarvi come a carnali, come a bambini in Cristo.
\par 2 V'ho nutriti di latte, non di cibo solido, perché non eravate ancora da tanto; anzi, non lo siete neppure adesso, perché siete ancora carnali.
\par 3 Infatti, poiché v'è tra voi gelosia e contesa, non siete voi carnali, e non camminate voi secondo l'uomo?
\par 4 Quando uno dice: Io son di Paolo; e un altro: Io son d'Apollo; non siete voi uomini carnali?
\par 5 Che cos'è dunque Apollo? E che cos'è Paolo? Son dei ministri, per mezzo dei quali voi avete creduto; e lo sono secondo che il Signore ha dato a ciascun di loro.
\par 6 Io ho piantato, Apollo ha annaffiato, ma è Dio che ha fatto crescere;
\par 7 talché né colui che pianta né colui che annaffia sono alcun che, ma Iddio che fa crescere, è tutto.
\par 8 Ora, colui che pianta e colui che annaffia sono una medesima cosa, ma ciascuno riceverà il proprio premio secondo la propria fatica.
\par 9 Poiché noi siamo collaboratori di Dio, voi siete il campo di Dio, l'edificio di Dio.
\par 10 Io, secondo la grazia di Dio che m'è stata data, come savio architetto, ho posto il fondamento; altri vi edifica sopra. Ma badi ciascuno com'egli vi edifica sopra;
\par 11 poiché nessuno può porre altro fondamento che quello già posto, cioè Cristo Gesù.
\par 12 Ora, se uno edifica su questo fondamento oro, argento, pietre di valore, legno, fieno, paglia,
\par 13 l'opera d'ognuno sarà manifestata, perché il giorno di Cristo la paleserà; poiché quel giorno ha da apparire qual fuoco; e il fuoco farà la prova di quel che sia l'opera di ciascuno.
\par 14 Se l'opera che uno ha edificata sul fondamento sussiste, ei ne riceverà ricompensa;
\par 15 se l'opera sua sarà arsa, ei ne avrà il danno; ma egli stesso sarà salvo, però come attraverso il fuoco.
\par 16 Non sapete voi che siete il tempio di Dio, e che lo Spirito di Dio abita in voi?
\par 17 Se uno guasta il tempio di Dio, Iddio guasterà lui; poiché il tempio di Dio è santo; e questo tempio siete voi.
\par 18 Nessuno s'inganni. Se qualcuno fra voi s'immagina d'esser savio in questo secolo, diventi pazzo affinché diventi savio;
\par 19 perché la sapienza di questo mondo è pazzia presso Dio. Infatti è scritto: Egli prende i savî nella loro astuzia;
\par 20 e altrove: Il Signore conosce i pensieri dei savî, e sa che sono vani.
\par 21 Nessuno dunque si glorî degli uomini, perché ogni cosa è vostra:
\par 22 e Paolo, e Apollo, e Cefa, e il mondo, e la vita, e la morte, e le cose presenti, e le cose future, tutto è vostro;
\par 23 e voi siete di Cristo, e Cristo è di Dio.

\chapter{4}

\par 1 Così ci stimi ognuno come dei ministri di Cristo e degli amministratori de' misteri di Dio.
\par 2 Del resto quel che si richiede dagli amministratori, è che ciascuno sia trovato fedele.
\par 3 A me poi pochissimo importa d'esser giudicato da voi o da un tribunale umano; anzi, non mi giudico neppur da me stesso.
\par 4 Poiché non ho coscienza di colpa alcuna; non per questo però sono giustificato; ma colui che mi giudica, è il Signore.
\par 5 Cosicché non giudicate di nulla prima del tempo, finché sia venuto il Signore, il quale metterà in luce le cose occulte delle tenebre, e manifesterà i consigli de' cuori; e allora ciascuno avrà la sua lode da Dio.
\par 6 Or, fratelli, queste cose le ho per amor vostro applicate a me stesso e ad Apollo, onde per nostro mezzo impariate a praticare il 'non oltre quel che è scritto'; affinché non vi gonfiate d'orgoglio esaltando l'uno a danno dell'altro.
\par 7 Infatti chi ti distingue dagli altri? E che hai tu che non l'abbia ricevuto? E se pur l'hai ricevuto, perché ti glorî come se tu non l'avessi ricevuto?
\par 8 Già siete saziati, già siete arricchiti, senza di noi siete giunti a regnare! E fosse pure che voi foste giunti a regnare, affinché anche noi potessimo regnare con voi!
\par 9 Poiché io stimo che Dio abbia messi in mostra noi, gli apostoli, ultimi fra tutti, come uomini condannati a morte; poiché siamo divenuti uno spettacolo al mondo, e agli angeli, e agli uomini.
\par 10 Noi siamo pazzi a cagion di Cristo, ma voi siete savî in Cristo; noi siamo deboli, ma voi siete forti; voi siete gloriosi, ma noi siamo sprezzati.
\par 11 Fino a questa stessa ora, noi abbiamo e fame e sete; noi siamo ignudi, e siamo schiaffeggiati, e non abbiamo stanza ferma,
\par 12 e ci affatichiamo lavorando con le nostre proprie mani; ingiuriati, benediciamo; perseguitati, sopportiamo; diffamati, esortiamo;
\par 13 siamo diventati e siam tuttora come la spazzatura del mondo, come il rifiuto di tutti.
\par 14 Io vi scrivo queste cose non per farvi vergogna, ma per ammonirvi come miei cari figliuoli.
\par 15 Poiché quand'anche aveste diecimila pedagoghi in Cristo, non avete però molti padri; perché son io che vi ho generati in Cristo Gesù, mediante l'Evangelo.
\par 16 Io vi esorto dunque: Siate miei imitatori.
\par 17 Appunto per questo vi ho mandato Timoteo, che è mio figliuolo diletto e fedele nel Signore; egli vi ricorderà quali siano le mie vie in Cristo Gesù, com'io insegni da per tutto, in ogni chiesa.
\par 18 Or alcuni si sono gonfiati come se io non dovessi recarmi da voi;
\par 19 ma, se il Signore vorrà, mi recherò presto da voi, e conoscerò non il parlare ma la potenza di coloro che si son gonfiati;
\par 20 perché il regno di Dio non consiste in parlare, ma in potenza.
\par 21 Che volete? Che venga da voi con la verga, o con amore e con spirito di mansuetudine?

\chapter{5}

\par 1 Si ode addirittura affermare che v'è tra voi fornicazione; e tale fornicazione, che non si trova neppure fra i Gentili; al punto che uno di voi si tiene la moglie di suo padre.
\par 2 E siete gonfi, e non avete invece fatto cordoglio perché colui che ha commesso quell'azione fosse tolto di mezzo a voi!
\par 3 Quanto a me, assente di persona ma presente in ispirito, ho già giudicato, come se fossi presente, colui che ha perpetrato un tale atto.
\par 4 Nel nome del Signore Gesù, essendo insieme adunati voi e lo spirito mio, con la potestà del Signor nostro Gesù,
\par 5 ho deciso che quel tale sia dato in man di Satana, a perdizione della carne, onde lo spirito sia salvo nel giorno del Signor Gesù.
\par 6 Il vostro vantarvi non è buono. Non sapete voi che un po' di lievito fa lievitare tutta la pasta?
\par 7 Purificatevi dal vecchio lievito, affinché siate una nuova pasta, come già siete senza lievito. Poiché anche la nostra pasqua, cioè Cristo, è stata immolata.
\par 8 Celebriamo dunque la festa, non con vecchio lievito, né con lievito di malizia e di malvagità, ma con gli azzimi della sincerità e della verità.
\par 9 V'ho scritto nella mia epistola di non mischiarvi coi fornicatori;
\par 10 non del tutto però coi fornicatori di questo mondo, o con gli avari e i rapaci, e con gl'idolatri; perché altrimenti dovreste uscire dal mondo;
\par 11 ma quel che v'ho scritto è di non mischiarvi con alcuno che, chiamandosi fratello, sia un fornicatore, o un avaro, o un idolatra, o un oltraggiatore, o un ubriacone, o un rapace; con un tale non dovete neppur mangiare.
\par 12 Poiché, ho io forse da giudicar que' di fuori? Non giudicate voi quelli di dentro?
\par 13 Que' di fuori li giudica Iddio. Togliete il malvagio di mezzo a voi stessi.

\chapter{6}

\par 1 Ardisce alcun di voi, quando ha una lite con un altro, chiamarlo in giudizio dinanzi agli ingiusti, anziché dinanzi ai santi?
\par 2 Non sapete voi che i santi giudicheranno il mondo? E se il mondo è giudicato da voi, siete voi indegni di giudicar delle cose minime?
\par 3 Non sapete voi che giudicheremo gli angeli? Quanto più possiamo giudicare delle cose di questa vita!
\par 4 Quando dunque avete da giudicar di cose di questa vita, costituitene giudici quelli che sono i meno stimati nella chiesa.
\par 5 Io dico questo per farvi vergogna. Così non v'è egli tra voi neppure un savio che sia capace di pronunziare un giudizio fra un fratello e l'altro?
\par 6 Ma il fratello processa il fratello, e lo fa dinanzi agl'infedeli.
\par 7 Certo è già in ogni modo un vostro difetto l'aver fra voi dei processi. Perché non patite piuttosto qualche torto? Perché non patite piuttosto qualche danno?
\par 8 Invece, siete voi che fate torto e danno; e ciò a dei fratelli.
\par 9 Non sapete voi che gli ingiusti non erederanno il regno di Dio? Non v'illudete; né i fornicatori, né gl'idolatri, né gli adulteri, né gli effeminati, né i sodomiti,
\par 10 né i ladri, né gli avari, né gli ubriachi, né gli oltraggiatori, né i rapaci erederanno il regno di Dio.
\par 11 E tali eravate alcuni; ma siete stati lavati, ma siete stati santificati, ma siete stati giustificati nel nome del Signor Gesù Cristo, e mediante lo Spirito dell'Iddio nostro.
\par 12 Ogni cosa m'è lecita, ma non ogni cosa è utile. Ogni cosa m'è lecita, ma io non mi lascerò dominare da cosa alcuna.
\par 13 Le vivande son per il ventre, e il ventre è per le vivande; ma Iddio distruggerà e queste e quello. Il corpo però non è per la fornicazione, ma è per il Signore, e il Signore è per il corpo;
\par 14 e Dio, come ha risuscitato il Signore, così risusciterà anche noi mediante la sua potenza.
\par 15 Non sapete voi che i vostri corpi sono membra di Cristo? Torrò io dunque le membra di Cristo per farne membra d'una meretrice? Così non sia.
\par 16 Non sapete voi che chi si unisce a una meretrice è un corpo solo con lei? Poiché, dice Iddio, i due diventeranno una sola carne.
\par 17 Ma chi si unisce al Signore è uno spirito solo con lui.
\par 18 Fuggite la fornicazione. Ogni altro peccato che l'uomo commetta è fuori del corpo; ma il fornicatore pecca contro il proprio corpo.
\par 19 E non sapete voi che il vostro corpo è il tempio dello Spirito Santo che è in voi, il quale avete da Dio, e che non appartenete a voi stessi?
\par 20 Poiché foste comprati a prezzo; glorificate dunque Dio nel vostro corpo.

\chapter{7}

\par 1 Or quant'è alle cose delle quali m'avete scritto, è bene per l'uomo di non toccar donna;
\par 2 ma, per evitar le fornicazioni, ogni uomo abbia la propria moglie, e ogni donna il proprio marito.
\par 3 Il marito renda alla moglie quel che le è dovuto; e lo stesso faccia la moglie verso il marito.
\par 4 La moglie non ha potestà sul proprio corpo, ma il marito; e nello stesso modo il marito non ha potestà sul proprio corpo, ma la moglie.
\par 5 Non vi private l'un dell'altro, se non di comun consenso, per un tempo, affin di darvi alla preghiera; e poi ritornate assieme, onde Satana non vi tenti a motivo della vostra incontinenza.
\par 6 Ma questo dico per concessione, non per comando;
\par 7 perché io vorrei che tutti gli uomini fossero come son io; ma ciascuno ha il suo proprio dono da Dio; l'uno in un modo, l'altro in un altro.
\par 8 Ai celibi e alle vedove, però, dico che è bene per loro che se ne stiano come sto anch'io.
\par 9 Ma se non si contengono, sposino; perché è meglio sposarsi che ardere.
\par 10 Ma ai coniugi ordino non io ma il Signore, che la moglie non si separi dal marito,
\par 11 (e se mai si separa, rimanga senza maritarsi o si riconcilî col marito); e che il marito non lasci la moglie.
\par 12 Ma agli altri dico io, non il Signore: Se un fratello ha una moglie non credente ed ella è contenta di abitar con lui, non la lasci;
\par 13 e la donna che ha un marito non credente, s'egli consente ad abitar con lei, non lasci il marito;
\par 14 perché il marito non credente è santificato nella moglie, e la moglie non credente è santificata nel marito credente; altrimenti i vostri figliuoli sarebbero impuri, mentre ora sono santi.
\par 15 Però, se il non credente si separa, si separi pure; in tali casi, il fratello o la sorella non sono vincolati; ma Dio ci ha chiamati a vivere in pace;
\par 16 perché, o moglie, che sai tu se salverai il marito? Ovvero tu, marito, che sai tu se salverai la moglie?
\par 17 Del resto, ciascuno seguiti a vivere nella condizione assegnatagli dal Signore, e nella quale si trovava quando Iddio lo chiamò. E così ordino in tutte le chiese.
\par 18 È stato alcuno chiamato essendo circonciso? Non faccia sparir la sua circoncisione. È stato alcuno chiamato essendo incirconciso? Non si faccia circoncidere.
\par 19 La circoncisione è nulla e la incirconcisione è nulla ma l'osservanza de' comandamenti di Dio è tutto.
\par 20 Ognuno rimanga nella condizione in cui era quando fu chiamato.
\par 21 Sei tu stato chiamato essendo schiavo? Non curartene, ma se puoi divenir libero è meglio valerti dell'opportunità.
\par 22 Poiché colui che è stato chiamato nel Signore, essendo schiavo, è un affrancato del Signore; parimente colui che è stato chiamato essendo libero, è schiavo di Cristo.
\par 23 Voi siete stati riscattati a prezzo; non diventate schiavi degli uomini.
\par 24 Fratelli, ognuno rimanga dinanzi a Dio nella condizione nella quale si trovava quando fu chiamato.
\par 25 Or quanto alle vergini, io non ho comandamento dal Signore; ma do il mio parere, come avendo ricevuto dal Signore la grazia d'esser fedele.
\par 26 Io stimo dunque che a motivo della imminente distretta sia bene per loro di restar come sono; poiché per l'uomo in genere è bene di starsene così.
\par 27 Sei tu legato a una moglie? Non cercare d'esserne sciolto. Sei tu sciolto da moglie? Non cercar moglie.
\par 28 Se però prendi moglie non pecchi; e se una vergine si marita, non pecca; ma tali persone avranno tribolazione nella carne, e io vorrei risparmiarvela.
\par 29 Ma questo io dichiaro, fratelli, che il tempo è ormai abbreviato; talché d'ora innanzi, anche quelli che hanno moglie, siano come se non l'avessero;
\par 30 e quelli che piangono, come se non piangessero; e quelli che si rallegrano, come se non si rallegrassero; e quelli che comprano, come se non possedessero;
\par 31 e quelli che usano di questo mondo, come se non ne usassero, perché la figura di questo mondo passa.
\par 32 Or io vorrei che foste senza sollecitudine. Chi non è ammogliato ha cura delle cose del Signore, del come potrebbe piacere al Signore;
\par 33 ma colui che è ammogliato, ha cura delle cose del mondo, del come potrebbe piacere alla moglie.
\par 34 E v'è anche una differenza tra la donna maritata e la vergine: la non maritata ha cura delle cose del Signore, affin d'esser santa di corpo e di Spirito; ma la maritata ha cura delle cose del mondo, del come potrebbe piacere al marito.
\par 35 Or questo dico per l'utile vostro proprio; non per tendervi un laccio, ma in vista di ciò che è decoroso e affinché possiate consacrarvi al Signore senza distrazione.
\par 36 Ma se alcuno crede far cosa indecorosa verso la propria figliuola nubile s'ella passi il fior dell'età, e se così bisogna fare, faccia quello che vuole; egli non pecca; la dia a marito.
\par 37 Ma chi sta fermo in cuor suo, e non è stretto da necessità ma è padrone della sua volontà, e ha determinato in cuor suo di serbar vergine la sua figliuola, fa bene.
\par 38 Perciò, chi dà la sua figliuola a marito fa bene, e chi non la dà a marito fa meglio.
\par 39 La moglie è vincolata per tutto il tempo che vive suo marito; ma, se il marito muore, ella è libera di maritarsi a chi vuole, purché sia nel Signore.
\par 40 Nondimeno ella è più felice, a parer mio, se rimane com'è; e credo d'aver anch'io lo Spirito di Dio.

\chapter{8}

\par 1 Quanto alle carni sacrificate agl'idoli, noi sappiamo che tutti abbiamo conoscenza. La conoscenza gonfia, ma la carità edifica.
\par 2 Se alcuno si pensa di conoscer qualcosa, egli non conosce ancora come si deve conoscere;
\par 3 ma se alcuno ama Dio, esso è conosciuto da lui.
\par 4 Quanto dunque, al mangiar delle carni sacrificate agl'idoli, noi sappiamo che l'idolo non è nulla nel mondo, e che non c'è alcun Dio fuori d'un solo.
\par 5 Poiché, sebbene vi siano de' cosiddetti dèi tanto in cielo che in terra, come infatti ci sono molti dèi e molti signori,
\par 6 nondimeno, per noi c'è un Dio solo, il Padre, dal quale sono tutte le cose, e noi per la gloria sua, e un solo Signore, Gesù Cristo, mediante il quale sono tutte le cose, e mediante il quale siam noi.
\par 7 Ma non in tutti è la conoscenza; anzi, alcuni, abituati finora all'idolo, mangiano di quelle carni, com'essendo cosa sacrificata a un idolo; e la loro coscienza, essendo debole, ne è contaminata.
\par 8 Ora non è un cibo che ci farà graditi a Dio; se non mangiamo, non abbiamo nulla di meno; e se mangiamo, non abbiamo nulla di più.
\par 9 Ma badate che questo vostro diritto non diventi un intoppo per i deboli.
\par 10 Perché se alcuno vede te, che hai conoscenza, seduto a tavola in un tempio d'idoli, la sua coscienza, s'egli è debole, non sarà ella incoraggiata a mangiar delle carni sacrificate agl'idoli?
\par 11 E così, per la tua conoscenza, perisce il debole, il fratello per il quale Cristo è morto.
\par 12 Ora, peccando in tal modo contro i fratelli, e ferendo la loro coscienza che è debole, voi peccate contro Cristo.
\par 13 Perciò, se un cibo scandalizza il mio fratello, io non mangerò mai più carne, per non scandalizzare il mio fratello.

\chapter{9}

\par 1 Non sono io libero? Non sono io apostolo? Non ho io veduto Gesù, il Signor nostro? Non siete voi l'opera mia nel Signore?
\par 2 Se per altri non sono apostolo lo sono almeno per voi; perché il suggello del mio apostolato siete voi, nel Signore.
\par 3 Questa è la mia difesa di fronte a quelli che mi sottopongono ad inchiesta.
\par 4 Non abbiam noi il diritto di mangiare e di bere?
\par 5 Non abbiamo noi il diritto di condurre attorno con noi una moglie, sorella in fede, siccome fanno anche gli altri apostoli e i fratelli del Signore e Cefa?
\par 6 O siamo soltanto io e Barnaba a non avere il diritto di non lavorare?
\par 7 Chi è mai che fa il soldato a sue proprie spese? Chi è che pianta una vigna e non ne mangia del frutto? O chi è che pasce un gregge e non si ciba del latte del gregge?
\par 8 Dico io queste cose secondo l'uomo? Non le dice anche la legge?
\par 9 Difatti, nella legge di Mosè è scritto: Non metter la musoliera al bue che trebbia il grano. Forse che Dio si dà pensiero dei buoi?
\par 10 O non dice Egli così proprio per noi? Certo, per noi fu scritto così; perché chi ara deve arare con speranza; e chi trebbia il grano deve trebbiarlo colla speranza d'averne la sua parte.
\par 11 Se abbiam seminato per voi i beni spirituali, è egli gran che se mietiam i vostri beni materiali?
\par 12 Se altri hanno questo diritto su voi, non l'abbiamo noi molto più? Ma noi non abbiamo fatto uso di questo diritto; anzi sopportiamo ogni cosa, per non creare alcun ostacolo all'Evangelo di Cristo.
\par 13 Non sapete voi che quelli i quali fanno il servigio sacro mangiano di quel che è offerto nel tempio? e che coloro i quali attendono all'altare, hanno parte all'altare?
\par 14 Così ancora, il Signore ha ordinato che coloro i quali annunziano l'Evangelo vivano dell'Evangelo.
\par 15 Io però non ho fatto uso d'alcuno di questi diritti, e non ho scritto questo perché si faccia così a mio riguardo; poiché preferirei morire, anziché veder qualcuno render vano il mio vanto.
\par 16 Perché se io evangelizzo, non ho da trarne vanto, poiché necessità me n'è imposta; e guai a me, se non evangelizzo!
\par 17 Se lo faccio volenterosamente, ne ho ricompensa; ma se non lo faccio volenterosamente è pur sempre un'amministrazione che m'è affidata.
\par 18 Qual è dunque la mia ricompensa? Questa: che annunziando l'Evangelo, io offra l'Evangelo gratuitamente, senza valermi del mio diritto nell'Evangelo.
\par 19 Poiché, pur essendo libero da tutti, mi son fatto servo a tutti, per guadagnarne il maggior numero;
\par 20 e coi Giudei, mi son fatto Giudeo, per guadagnare i Giudei; con quelli che son sotto la legge, mi son fatto come uno sotto la legge (benché io stesso non sia sottoposto alla legge), per guadagnare quelli che son sotto la legge;
\par 21 con quelli che son senza legge, mi son fatto come se fossi senza legge (benché io non sia senza legge riguardo a Dio, ma sotto la legge di Cristo), per guadagnare quelli che son senza legge.
\par 22 Coi deboli mi son fatto debole, per guadagnare i deboli; mi faccio ogni cosa a tutti, per salvarne ad ogni modo alcuni.
\par 23 E tutto fo a motivo dell'Evangelo, affin d'esserne partecipe anch'io.
\par 24 Non sapete voi che coloro i quali corrono nello stadio, corrono ben tutti, ma uno solo ottiene il premio? Correte in modo da riportarlo.
\par 25 Chiunque fa l'atleta è temperato in ogni cosa; e quelli lo fanno per ricevere una corona corruttibile; ma noi, una incorruttibile.
\par 26 Io quindi corro ma non in modo incerto, lotto al pugilato, ma non come chi batte l'aria;
\par 27 anzi, tratto duramente il mio corpo e lo riduco in schiavitù, che talora, dopo aver predicato agli altri, io stesso non sia riprovato.

\chapter{10}

\par 1 Perché, fratelli, non voglio che ignoriate che i nostri padri furon tutti sotto la nuvola, e tutti passarono attraverso il mare,
\par 2 e tutti furon battezzati, nella nuvola e nel mare, per esser di Mosè,
\par 3 e tutti mangiarono lo stesso cibo spirituale,
\par 4 e tutti bevvero la stessa bevanda spirituale, perché beveano alla roccia spirituale che li seguiva; e la roccia era Cristo.
\par 5 Ma della maggior parte di loro Iddio non si compiacque, poiché furono atterrati nel deserto.
\par 6 Or queste cose avvennero per servir d'esempio a noi, onde non siam bramosi di cose malvage, come coloro ne furon bramosi;
\par 7 onde non diventiate idolatri come alcuni di loro, secondo che è scritto: Il popolo si sedette per mangiare e per bere, poi s'alzò per divertirsi;
\par 8 onde non fornichiamo come taluni di loro fornicarono, e ne caddero, in un giorno solo, ventitremila;
\par 9 onde non tentiamo il Signore, come alcuni di loro lo tentarono, e perirono morsi dai serpenti.
\par 10 E non mormorate come alcuni di loro mormorarono, e perirono colpiti dal distruttore.
\par 11 Or queste cose avvennero loro per servire d'esempio, e sono state scritte per ammonizione di noi, che ci troviamo agli ultimi termini dei tempi.
\par 12 Perciò, chi si pensa di stare ritto, guardi di non cadere.
\par 13 Niuna tentazione vi ha còlti, che non sia stata umana; or Iddio è fedele e non permetterà che siate tentati al di là delle vostre forze; ma con la tentazione vi darà anche la via d'uscirne, onde la possiate sopportare.
\par 14 Perciò, cari miei, fuggite l'idolatria.
\par 15 Io parlo come a persone intelligenti; giudicate voi di quello che dico.
\par 16 Il calice della benedizione che noi benediciamo, non è egli la comunione col sangue di Cristo? Il pane, che noi rompiamo, non è egli la comunione col corpo di Cristo?
\par 17 Siccome v'è un unico pane, noi, che siam molti, siamo un corpo unico, perché partecipiamo tutti a quell'unico pane.
\par 18 Guardate l'Israele secondo la carne; quelli che mangiano i sacrificî non hanno essi comunione con l'altare?
\par 19 Che dico io dunque? Che la carne sacrificata agl'idoli sia qualcosa? Che un idolo sia qualcosa?
\par 20 Tutt'altro; io dico che le carni che i Gentili sacrificano, le sacrificano ai demonî e non a Dio; or io non voglio che abbiate comunione coi demonî.
\par 21 Voi non potete bere il calice del Signore e il calice dei demonî; voi non potete partecipare alla mensa del Signore e alla mensa dei demonî.
\par 22 O vogliam noi provocare il Signore a gelosia? Siamo noi più forti di lui?
\par 23 Ogni cosa è lecita ma non ogni cosa è utile; ogni cosa è lecita ma non ogni cosa edifica.
\par 24 Nessuno cerchi il proprio vantaggio, ma ciascuno cerchi l'altrui.
\par 25 Mangiate di tutto quello che si vende al macello senza fare inchieste per motivo di coscienza;
\par 26 perché al Signore appartiene la terra e tutto quello ch'essa contiene.
\par 27 Se qualcuno de' non credenti v'invita, e voi volete andarci, mangiate di tutto quello che vi è posto davanti, senza fare inchieste per motivo di coscienza.
\par 28 Ma se qualcuno vi dice: Questa è cosa di sacrificî, non ne mangiate per riguardo a colui che v'ha avvertito, e per riguardo alla coscienza;
\par 29 alla coscienza, dico, non tua, ma di quell'altro; infatti, perché la mia libertà sarebb'ella giudicata dalla coscienza altrui?
\par 30 E se io mangio di una cosa con rendimento di grazie, perché sarei biasimato per quello di cui io rendo grazie?
\par 31 Sia dunque che mangiate, sia che beviate, sia che facciate alcun'altra cosa, fate tutto alla gloria di Dio.
\par 32 Non siate d'intoppo né ai Giudei, né ai Greci, né alla Chiesa di Dio:
\par 33 sì come anch'io compiaccio a tutti in ogni cosa, non cercando l'utile mio proprio, ma quello de' molti, affinché siano salvati.

\chapter{11}

\par 1 Siate miei imitatori, come anch'io lo sono di Cristo.
\par 2 Or io vi lodo perché vi ricordate di me in ogni cosa, e ritenete i miei insegnamenti quali ve li ho trasmessi.
\par 3 Ma io voglio che sappiate che il capo d'ogni uomo è Cristo, che il capo della donna è l'uomo, e che il capo di Cristo è Dio.
\par 4 Ogni uomo che prega o profetizza a capo coperto, fa disonore al suo capo;
\par 5 ma ogni donna che prega o profetizza senz'avere il capo coperto da un velo, fa disonore al suo capo, perché è lo stesso che se fosse rasa.
\par 6 Perché se la donna non si mette il velo, si faccia anche tagliare i capelli! Ma se è cosa vergognosa per una donna il farsi tagliare i capelli o radere il capo, si metta un velo.
\par 7 Poiché, quanto all'uomo, egli non deve velarsi il capo, essendo immagine e gloria di Dio; ma la donna è la gloria dell'uomo;
\par 8 perché l'uomo non viene dalla donna, ma la donna dall'uomo;
\par 9 e l'uomo non fu creato a motivo della donna, ma la donna a motivo dell'uomo.
\par 10 Perciò la donna deve, a motivo degli angeli, aver sul capo un segno dell'autorità da cui dipende.
\par 11 D'altronde, nel Signore, né la donna è senza l'uomo, né l'uomo senza la donna.
\par 12 Poiché, siccome la donna viene dall'uomo, così anche l'uomo esiste per mezzo della donna, e ogni cosa è da Dio.
\par 13 Giudicatene voi stessi: È egli conveniente che una donna preghi Iddio senz'esser velata?
\par 14 La natura stessa non v'insegna ella che se l'uomo porta la chioma, ciò è per lui un disonore?
\par 15 Mentre se una donna porta la chioma, ciò è per lei un onore; perché la chioma le è data a guisa di velo.
\par 16 Se poi ad alcuno piace d'esser contenzioso, noi non abbiamo tale usanza; e neppur le chiese di Dio.
\par 17 Mentre io vi do queste istruzioni, io non vi lodo del fatto che vi radunate non per il meglio ma per il peggio.
\par 18 Poiché, prima di tutto, sento che quando v'adunate in assemblea, ci son fra voi delle divisioni; e in parte lo credo;
\par 19 perché bisogna che ci sian fra voi anche delle sètte, affinché quelli che sono approvati, siano manifesti fra voi.
\par 20 Quando poi vi radunate assieme, quel che fate, non è mangiar la Cena del Signore;
\par 21 poiché, al pasto comune, ciascuno prende prima la propria cena; e mentre l'uno ha fame, l'altro è ubriaco.
\par 22 Non avete voi delle case per mangiare e bere? O disprezzate voi la chiesa di Dio e fate vergogna a quelli che non hanno nulla? Che vi dirò? Vi loderò io? In questo io non vi lodo.
\par 23 Poiché ho ricevuto dal Signore quello che anche v'ho trasmesso; cioè, che il Signor Gesù, nella notte che fu tradito, prese del pane;
\par 24 e dopo aver rese grazie, lo ruppe e disse: Questo è il mio corpo che è dato per voi; fate questo in memoria di me.
\par 25 Parimente, dopo aver cenato, prese anche il calice, dicendo: Questo calice è il nuovo patto nel mio sangue; fate questo, ogni volta che ne berrete, in memoria di me.
\par 26 Poiché ogni volta che voi mangiate questo pane e bevete di questo calice, voi annunziate la morte del Signore, finch'egli venga.
\par 27 Perciò, chiunque mangerà il pane o berrà del calice del Signore indegnamente, sarà colpevole verso il corpo ed il sangue del Signore.
\par 28 Or provi l'uomo se stesso, e così mangi del pane e beva del calice;
\par 29 poiché chi mangia e beve, mangia e beve un giudicio su se stesso, se non discerne il corpo del Signore.
\par 30 Per questa cagione molti fra voi sono infermi e malati, e parecchi muoiono.
\par 31 Ora, se esaminassimo noi stessi, non saremmo giudicati;
\par 32 ma quando siamo giudicati, siam corretti dal Signore, affinché non siam condannati col mondo.
\par 33 Quando dunque, fratelli miei, v'adunate per mangiare, aspettatevi gli uni gli altri.
\par 34 Se qualcuno ha fame, mangi a casa, onde non vi aduniate per attirar su voi un giudicio. Le altre cose regolerò quando verrò.

\chapter{12}

\par 1 Circa i doni spirituali, fratelli, non voglio che siate nell'ignoranza.
\par 2 Voi sapete che quando eravate Gentili eravate trascinati dietro agl'idoli muti, secondo che vi si menava.
\par 3 Perciò vi fo sapere che nessuno, parlando per lo Spirito di Dio, dice: Gesù è anatema! e nessuno può dire: Gesù è il Signore! se non per lo Spirito Santo.
\par 4 Or vi è diversità di doni, ma v'è un medesimo Spirito.
\par 5 E vi è diversità di ministerî, ma non v'è che un medesimo Signore.
\par 6 E vi è varietà di operazioni, ma non v'è che un medesimo Iddio, il quale opera tutte le cose in tutti.
\par 7 Or a ciascuno è data la manifestazione dello Spirito per l'utile comune.
\par 8 Infatti, a uno è data mediante lo Spirito parola di sapienza; a un altro, parola di conoscenza, secondo il medesimo Spirito;
\par 9 a un altro, fede, mediante il medesimo Spirito; a un altro, doni di guarigioni, per mezzo del medesimo Spirito; a un altro, potenza d'operar miracoli;
\par 10 a un altro, profezia; a un altro, il discernimento degli spiriti; a un altro, diversità di lingue, e ad un altro, la interpretazione delle lingue;
\par 11 ma tutte queste cose le opera quell'uno e medesimo Spirito, distribuendo i suoi doni a ciascuno in particolare come Egli vuole.
\par 12 Poiché, siccome il corpo è uno ed ha molte membra, e tutte le membra del corpo, benché siano molte, formano un unico corpo, così ancora è di Cristo.
\par 13 Infatti noi tutti abbiam ricevuto il battesimo di un unico Spirito per formare un unico corpo, e Giudei e Greci, e schiavi e liberi; e tutti siamo stati abbeverati di un unico Spirito.
\par 14 E infatti il corpo non si compone di un membro solo, ma di molte membra.
\par 15 Se il piè dicesse: Siccome io non sono mano, non son del corpo, non per questo non sarebbe del corpo.
\par 16 E se l'orecchio dicesse: Siccome io non son occhio, non son del corpo, non per questo non sarebbe del corpo.
\par 17 Se tutto il corpo fosse occhio, dove sarebbe l'udito? Se tutto fosse udito, dove sarebbe l'odorato?
\par 18 Ma ora Iddio ha collocato ciascun membro nel corpo, come ha voluto.
\par 19 E se tutte le membra fossero un unico membro, dove sarebbe il corpo?
\par 20 Ma ora ci son molte membra, ma c'è un unico corpo;
\par 21 e l'occhio non può dire alla mano: Io non ho bisogno di te; né il capo può dire ai piedi: Non ho bisogno di voi.
\par 22 Al contrario, le membra del corpo che paiono essere più deboli, sono invece necessarie;
\par 23 e quelle parti del corpo che noi stimiamo esser le meno onorevoli, noi le circondiamo di maggior onore; e le parti nostre meno decorose son fatte segno di maggior decoro,
\par 24 mentre le parti nostre decorose non ne hanno bisogno; ma Dio ha costrutto il corpo in modo da dare maggior onore alla parte che ne mancava,
\par 25 affinché non ci fosse divisione nel corpo, ma le membra avessero la medesima cura le une per le altre.
\par 26 E se un membro soffre, tutte le membra soffrono con lui; e se un membro è onorato, tutte le membra ne gioiscono con lui.
\par 27 Or voi siete il corpo di Cristo, e membra d'esso, ciascuno per parte sua.
\par 28 E Dio ha costituito nella Chiesa primieramente degli apostoli; in secondo luogo dei profeti; in terzo luogo de' dottori; poi, i miracoli; poi i doni di guarigione, le assistenze, i doni di governo, la diversità delle lingue.
\par 29 Tutti sono eglino apostoli? Son forse tutti profeti? Son forse tutti dottori? Fan tutti de' miracoli?
\par 30 Tutti hanno eglino i doni delle guarigioni? Parlan tutti in altre lingue? Interpretano tutti?
\par 31 Ma desiderate ardentemente i doni maggiori. E ora vi mostrerò una via, che è la via per eccellenza.

\chapter{13}

\par 1 Quand'io parlassi le lingue degli uomini e degli angeli, se non ho carità, divento un rame risonante o uno squillante cembalo.
\par 2 E quando avessi il dono di profezia e conoscessi tutti i misteri e tutta la scienza, e avessi tutta la fede in modo da trasportare i monti, se non ho carità, non son nulla.
\par 3 E quando distribuissi tutte le mie facoltà per nutrire i poveri, e quando dessi il mio corpo ad essere arso, se non ho carità, ciò niente mi giova.
\par 4 La carità è paziente, è benigna; la carità non invidia; la carità non si vanta, non si gonfia,
\par 5 non si comporta in modo sconveniente, non cerca il proprio interesse, non s'inasprisce, non sospetta il male,
\par 6 non gode dell'ingiustizia, ma gioisce con la verità;
\par 7 soffre ogni cosa, crede ogni cosa, spera ogni cosa, sopporta ogni cosa.
\par 8 La carità non verrà mai meno. Quanto alle profezie, esse verranno abolite; quanto alle lingue, esse cesseranno; quanto alla conoscenza, essa verrà abolita;
\par 9 poiché noi conosciamo in parte, e in parte profetizziamo;
\par 10 ma quando la perfezione sarà venuta, quello che è solo in parte, sarà abolito.
\par 11 Quand'ero fanciullo, parlavo da fanciullo, pensavo da fanciullo, ragionavo da fanciullo; ma quando son diventato uomo, ho smesso le cose da fanciullo.
\par 12 Poiché ora vediamo come in uno specchio, in modo oscuro; ma allora vedremo faccia a faccia: ora conosco in parte; ma allora conoscerò appieno, come anche sono stato appieno conosciuto.
\par 13 Or dunque queste tre cose durano: fede, speranza, carità; ma la più grande di esse è la carità.

\chapter{14}

\par 1 Procacciate la carità, non lasciando però di ricercare i doni spirituali, e principalmente il dono di profezia.
\par 2 Perché chi parla in altra lingua non parla agli uomini, ma a Dio; poiché nessuno l'intende, ma in ispirito proferisce misteri.
\par 3 Chi profetizza, invece, parla agli uomini un linguaggio di edificazione, di esortazione e di consolazione.
\par 4 Chi parla in altra lingua edifica se stesso; ma chi profetizza edifica la chiesa.
\par 5 Or io ben vorrei che tutti parlaste in altre lingue; ma molto più che profetaste; chi profetizza è superiore a chi parla in altre lingue, a meno ch'egli interpreti, affinché la chiesa ne riceva edificazione.
\par 6 Infatti, fratelli, s'io venissi a voi parlando in altre lingue, che vi gioverei se la mia parola non vi recasse qualche rivelazione, o qualche conoscenza, o qualche profezia, o qualche insegnamento?
\par 7 Perfino le cose inanimate che dànno suono, quali il flauto o la cetra, se non dànno distinzione di suoni, come si conoscerà quel ch'è suonato col flauto o con la cetra?
\par 8 E se la tromba dà un suono sconosciuto, chi si preparerà alla battaglia?
\par 9 Così anche voi, se per il vostro dono di lingue non proferite un parlare intelligibile, come si capirà quel che dite? Parlerete in aria.
\par 10 Ci sono nel mondo tante e tante specie di parlari, e niun parlare è senza significato.
\par 11 Se quindi io non intendo il significato del parlare, sarò un barbaro per chi parla, e chi parla sarà un barbaro per me.
\par 12 Così anche voi, poiché siete bramosi dei doni spirituali, cercate di abbondarne per l'edificazione della chiesa.
\par 13 Perciò, chi parla in altra lingua preghi di poter interpretare;
\par 14 poiché, se prego in altra lingua, ben prega lo spirito mio, ma la mia intelligenza rimane infruttuosa.
\par 15 Che dunque? Io pregherò con lo spirito, ma pregherò anche con l'intelligenza; salmeggerò con lo spirito, ma salmeggerò anche con l'intelligenza.
\par 16 Altrimenti, se tu benedici Iddio soltanto con lo spirito, come potrà colui che occupa il posto del semplice uditore dire 'Amen' al tuo rendimento di grazie, poiché non sa quel che tu dici?
\par 17 Quanto a te, certo, tu fai un bel ringraziamento; ma l'altro non è edificato.
\par 18 Io ringrazio Dio che parlo in altre lingue più di tutti voi;
\par 19 ma nella chiesa preferisco dir cinque parole intelligibili per istruire anche gli altri, che dirne diecimila in altra lingua.
\par 20 Fratelli, non siate fanciulli per senno; siate pur bambini quanto a malizia, ma quanto a senno, siate uomini fatti.
\par 21 Egli è scritto nella legge: Io parlerò a questo popolo per mezzo di gente d'altra lingua, e per mezzo di labbra straniere; e neppur così mi ascolteranno, dice il Signore.
\par 22 Pertanto le lingue servono di segno, non per i credenti, ma per i non credenti: la profezia, invece, serve di segno non per i non credenti, ma per i credenti.
\par 23 Quando dunque tutta la chiesa si raduna assieme, se tutti parlano in altre lingue, ed entrano degli estranei o dei non credenti, non diranno essi che siete pazzi?
\par 24 Ma se tutti profetizzano, ed entra qualche non credente o qualche estraneo, egli è convinto da tutti,
\par 25 è scrutato da tutti, i segreti del suo cuore son palesati; e così, gettandosi giù con la faccia a terra, adorerà Dio, proclamando che Dio è veramente fra voi.
\par 26 Che dunque, fratelli? Quando vi radunate, avendo ciascun di voi un salmo, o un insegnamento, o una rivelazione, o un parlare in altra lingua, o una interpretazione, facciasi ogni cosa per l'edificazione.
\par 27 Se c'è chi parla in altra lingua, siano due o tre al più, a farlo; e l'un dopo l'altro; e uno interpreti;
\par 28 e se non v'è chi interpreti, si tacciano nella chiesa e parlino a se stessi e a Dio.
\par 29 Parlino due o tre profeti, e gli altri giudichino;
\par 30 e se una rivelazione è data a uno di quelli che stanno seduti, il precedente si taccia.
\par 31 Poiché tutti, uno ad uno, potete profetare; affinché tutti imparino e tutti sian consolati;
\par 32 e gli spiriti de' profeti son sottoposti a' profeti,
\par 33 perché Dio non è un Dio di confusione, ma di pace.
\par 34 Come si fa in tutte le chiese de' santi, tacciansi le donne nelle assemblee, perché non è loro permesso di parlare, ma debbono star soggette, come dice anche la legge.
\par 35 E se vogliono imparar qualcosa, interroghino i loro mariti a casa; perché è cosa indecorosa per una donna parlare in assemblea.
\par 36 La parola di Dio è forse proceduta da voi? O è dessa forse pervenuta a voi soli?
\par 37 Se qualcuno si stima esser profeta o spirituale, riconosca che le cose che io vi scrivo sono comandamenti del Signore.
\par 38 E se qualcuno lo vuole ignorare, lo ignori.
\par 39 Pertanto, fratelli, bramate il profetare, e non impedite il parlare in altre lingue;
\par 40 ma ogni cosa sia fatta con decoro e con ordine.

\chapter{15}

\par 1 Fratelli, io vi rammento l'Evangelo che v'ho annunziato, che voi ancora avete ricevuto, nel quale ancora state saldi, e mediante il quale siete salvati,
\par 2 se pur lo ritenete quale ve l'ho annunziato; a meno che non abbiate creduto invano.
\par 3 Poiché io v'ho prima di tutto trasmesso, come l'ho ricevuto anch'io, che Cristo è morto per i nostri peccati, secondo le Scritture;
\par 4 che fu seppellito; che risuscitò il terzo giorno, secondo le Scritture;
\par 5 che apparve a Cefa, poi ai Dodici.
\par 6 Poi apparve a più di cinquecento fratelli in una volta, dei quali la maggior parte rimane ancora in vita e alcuni sono morti.
\par 7 Poi apparve a Giacomo; poi a tutti gli Apostoli;
\par 8 e, ultimo di tutti, apparve anche a me, come all'aborto;
\par 9 perché io sono il minimo degli apostoli; e non son degno d'esser chiamato apostolo, perché ho perseguitato la Chiesa di Dio.
\par 10 Ma per la grazia di Dio io sono quello che sono; e la grazia sua verso di me non è stata vana; anzi, ho faticato più di loro tutti; non già io, però, ma la grazia di Dio che è con me.
\par 11 Sia dunque io o siano loro, così noi predichiamo, e così voi avete creduto.
\par 12 Or se si predica che Cristo è risuscitato dai morti, come mai alcuni fra voi dicono che non v'è risurrezione de' morti?
\par 13 Ma se non v'è risurrezione dei morti, neppur Cristo è risuscitato;
\par 14 e se Cristo non è risuscitato, vana dunque è la nostra predicazione, e vana pure è la vostra fede.
\par 15 E noi siamo anche trovati falsi testimoni di Dio, poiché abbiam testimoniato di Dio, ch'Egli ha risuscitato il Cristo; il quale Egli non ha risuscitato, se è vero che i morti non risuscitano.
\par 16 Difatti, se i morti non risuscitano, neppur Cristo è risuscitato;
\par 17 e se Cristo non è risuscitato, vana è la vostra fede; voi siete ancora nei vostri peccati.
\par 18 Anche quelli che dormono in Cristo, son dunque periti.
\par 19 Se abbiamo sperato in Cristo per questa vita soltanto, noi siamo i più miserabili di tutti gli uomini.
\par 20 Ma ora Cristo è risuscitato dai morti, primizia di quelli che dormono.
\par 21 Infatti, poiché per mezzo d'un uomo è venuta la morte, così anche per mezzo d'un uomo è venuta la risurrezione dei morti.
\par 22 Poiché, come tutti muoiono in Adamo, così anche in Cristo saran tutti vivificati;
\par 23 ma ciascuno nel suo proprio ordine: Cristo, la primizia; poi quelli che son di Cristo, alla sua venuta;
\par 24 poi verrà la fine, quand'egli avrà rimesso il regno nelle mani di Dio Padre, dopo che avrà ridotto al nulla ogni principato, ogni potestà ed ogni potenza.
\par 25 Poiché bisogna ch'egli regni finché abbia messo tutti i suoi nemici sotto i suoi piedi.
\par 26 L'ultimo nemico che sarà distrutto, sarà la morte.
\par 27 Difatti, Iddio ha posto ogni cosa sotto i piedi di esso; ma quando dice che ogni cosa gli è sottoposta, è chiaro che Colui che gli ha sottoposto ogni cosa, ne è eccettuato.
\par 28 E quando ogni cosa gli sarà sottoposta, allora anche il Figlio stesso sarà sottoposto a Colui che gli ha sottoposto ogni cosa, affinché Dio sia tutto in tutti.
\par 29 Altrimenti, che faranno quelli che son battezzati per i morti? Se i morti non risuscitano affatto, perché dunque son essi battezzati per loro?
\par 30 E perché anche noi siamo ogni momento in pericolo?
\par 31 Ogni giorno sono esposto alla morte; sì, fratelli, com'è vero ch'io mi glorio di voi, in Cristo Gesù, nostro Signore.
\par 32 Se soltanto per fini umani ho lottato con le fiere ad Efeso, che utile ne ho io? Se i morti non risuscitano, mangiamo e beviamo, perché domani morremo.
\par 33 Non v'ingannate: Le cattive compagnie corrompono i buoni costumi.
\par 34 Svegliatevi a vita di giustizia, e non peccate; perché alcuni non hanno conoscenza di Dio; lo dico a vostra vergogna.
\par 35 Ma qualcuno dirà: Come risuscitano i morti? E con qual corpo tornano essi?
\par 36 Insensato, quel che tu semini non è vivificato, se prima non muore;
\par 37 e quanto a quel che tu semini, non semini il corpo che ha da nascere, ma un granello ignudo, come capita, di frumento, o di qualche altro seme;
\par 38 e Dio gli dà un corpo secondo che l'ha stabilito; e ad ogni seme, il proprio corpo.
\par 39 Non ogni carne è la stessa carne; ma altra è la carne degli uomini, altra la carne delle bestie, altra quella degli uccelli, altra quella de' pesci.
\par 40 Ci sono anche de' corpi celesti, e de' corpi terrestri; ma altra è la gloria de' celesti, e altra quella de' terrestri.
\par 41 Altra è la gloria del sole, altra la gloria della luna, e altra la gloria delle stelle; perché un astro è differente dall'altro in gloria.
\par 42 Così pure della risurrezione de' morti. Il corpo è seminato corruttibile, e risuscita incorruttibile;
\par 43 è seminato ignobile, e risuscita glorioso; è seminato debole, e risuscita potente;
\par 44 è seminato corpo naturale, e risuscita corpo spirituale. Se c'è un corpo naturale, c'è anche un corpo spirituale.
\par 45 Così anche sta scritto: Il primo uomo, Adamo, fu fatto anima vivente; l'ultimo Adamo è spirito vivificante.
\par 46 Però, ciò che è spirituale non vien prima; ma prima, ciò che è naturale; poi vien ciò che è spirituale.
\par 47 Il primo uomo, tratto dalla terra, è terreno; il secondo uomo è dal cielo.
\par 48 Quale è il terreno, tali sono anche i terreni; e quale è il celeste, tali saranno anche i celesti.
\par 49 E come abbiam portato l'immagine del terreno, così porteremo anche l'immagine del celeste.
\par 50 Or questo dico, fratelli, che carne e sangue non possono eredare il regno di Dio; né la corruzione può eredare la incorruttibilità.
\par 51 Ecco, io vi dico un mistero: Non tutti morremo, ma tutti saremo mutati,
\par 52 in un momento, in un batter d'occhio, al suon dell'ultima tromba. Perché la tromba sonerà, e i morti risusciteranno incorruttibili, e noi saremo mutati.
\par 53 Poiché bisogna che questo corruttibile rivesta incorruttibilità, e che questo mortale rivesta immortalità.
\par 54 E quando questo corruttibile avrà rivestito incorruttibilità, e questo mortale avrà rivestito immortalità, allora sarà adempiuta la parola che è scritta: La morte è stata sommersa nella vittoria.
\par 55 O morte, dov'è la tua vittoria? O morte, dov'è il tuo dardo?
\par 56 Or il dardo della morte è il peccato, e la forza del peccato è la legge;
\par 57 ma ringraziato sia Dio, che ci dà la vittoria per mezzo del Signor nostro Gesù Cristo.
\par 58 Perciò fratelli miei diletti, state saldi, incrollabili, abbondanti sempre nell'opera del Signore, sapendo che la vostra fatica non è vana nel Signore.

\chapter{16}

\par 1 Or quanto alla colletta per i santi, come ho ordinato alle chiese di Galazia, così fate anche voi.
\par 2 Ogni primo giorno della settimana ciascun di voi metta da parte a casa quel che potrà secondo la prosperità concessagli, affinché, quando verrò, non ci sian più collette da fare.
\par 3 E quando sarò giunto, quelli che avrete approvati, io li manderò con lettere a portare la vostra liberalità a Gerusalemme;
\par 4 e se converrà che ci vada anch'io, essi verranno meco.
\par 5 Io poi mi recherò da voi, quando sarò passato per la Macedonia;
\par 6 perché passerò per la Macedonia; ma da voi forse mi fermerò alquanto, ovvero anche passerò l'inverno, affinché voi mi facciate proseguire per dove mi recherò.
\par 7 Perché, questa volta, io non voglio vedervi di passaggio; poiché spero di fermarmi qualche tempo da voi, se il Signore lo permette.
\par 8 Ma mi fermerò in Efeso fino alla Pentecoste,
\par 9 perché una larga porta mi è qui aperta ad un lavoro efficace, e vi son molti avversarî.
\par 10 Or se viene Timoteo, guardate che stia fra voi senza timore; perch'egli lavora nell'opera del Signore, come faccio anch'io.
\par 11 Nessuno dunque lo sprezzi; ma fatelo proseguire in pace, affinché venga da me; poiché io l'aspetto coi fratelli.
\par 12 Quanto al fratello Apollo, io l'ho molto esortato a recarsi da voi coi fratelli; ma egli assolutamente non ha avuto volontà di farlo adesso; andrà però quando ne avrà l'opportunità.
\par 13 Vegliate, state fermi nella fede, portatevi virilmente, fortificatevi.
\par 14 Tutte le cose vostre sian fatte con carità.
\par 15 Or, fratelli, voi conoscete la famiglia di Stefana; sapete che è la primizia dell'Acaia, e che si è dedicata al servizio dei santi;
\par 16 io v'esorto a sottomettervi anche voi a cotali persone, e a chiunque lavora e fatica nell'opera comune.
\par 17 E io mi rallegro della venuta di Stefana, di Fortunato e d'Acaico, perché essi hanno riempito il vuoto prodotto dalla vostra assenza;
\par 18 poiché hanno ricreato lo spirito mio ed il vostro; sappiate apprezzare cotali persone.
\par 19 Le chiese dell'Asia vi salutano. Aquila e Priscilla, con la chiesa che è in casa loro, vi salutano molto nel Signore.
\par 20 Tutti i fratelli vi salutano. Salutatevi gli uni gli altri con un santo bacio.
\par 21 Il saluto, di mia propria mano: di me, Paolo.
\par 22 Se qualcuno non ama il Signore, sia anatema. Maràn-atà.
\par 23 La grazia del Signor Gesù sia con voi.
\par 24 L'amor mio è con tutti voi in Cristo Gesù.


\end{document}