\begin{document}

\title{Galati}


\chapter{1}

\par 1 Paolo, apostolo (non dagli uomini né per mezzo d'alcun uomo, ma per mezzo di Gesù Cristo e di Dio Padre che l'ha risuscitato dai morti),
\par 2 e tutti i fratelli che sono meco, alle chiese della Galazia;
\par 3 grazia a voi e pace da Dio Padre e dal Signor nostro Gesù Cristo,
\par 4 che ha dato se stesso per i nostri peccati affin di strapparci al presente secolo malvagio, secondo la volontà del nostro Dio e Padre,
\par 5 al quale sia la gloria ne' secoli dei secoli. Amen.
\par 6 Io mi maraviglio che così presto voi passiate da Colui che vi ha chiamati mediante la grazia di Cristo, a un altro vangelo.
\par 7 Il quale poi non è un altro vangelo; ma ci sono alcuni che vi turbano e vogliono sovvertire l'Evangelo di Cristo.
\par 8 Ma quand'anche noi, quand'anche un angelo del cielo vi annunziasse un vangelo diverso da quello che v'abbiamo annunziato, sia egli anatema.
\par 9 Come l'abbiamo detto prima d'ora, torno a ripeterlo anche adesso: Se alcuno vi annunzia un vangelo diverso da quello che avete ricevuto, sia anatema.
\par 10 Vado io forse cercando di conciliarmi il favore degli uomini, ovvero quello di Dio? O cerco io di piacere agli uomini? Se cercassi ancora di piacere agli uomini, non sarei servitore di Cristo.
\par 11 E invero, fratelli, io vi dichiaro che l'Evangelo da me annunziato non è secondo l'uomo;
\par 12 poiché io stesso non l'ho ricevuto né l'ho imparato da alcun uomo, ma l'ho ricevuto per rivelazione di Gesù Cristo.
\par 13 Difatti voi avete udito quale sia stata la mia condotta nel passato, quando ero nel giudaismo; come perseguitavo a tutto potere la Chiesa di Dio e la devastavo,
\par 14 e mi segnalavo nel giudaismo più di molti della mia età fra i miei connazionali, essendo estremamente zelante delle tradizioni dei miei padri.
\par 15 Ma quando Iddio, che m'aveva appartato fin dal seno di mia madre e m'ha chiamato mediante la sua grazia, si compiacque
\par 16 di rivelare in me il suo Figliuolo perch'io lo annunziassi fra i Gentili, io non mi consigliai con carne e sangue,
\par 17 e non salii a Gerusalemme da quelli che erano stati apostoli prima di me, ma subito me ne andai in Arabia; quindi tornai di nuovo a Damasco.
\par 18 Di poi, in capo a tre anni, salii a Gerusalemme per visitar Cefa, e stetti da lui quindici giorni;
\par 19 e non vidi alcun altro degli apostoli; ma solo Giacomo, il fratello del Signore.
\par 20 Ora, circa le cose che vi scrivo, ecco, nel cospetto di Dio vi dichiaro che non mentisco.
\par 21 Poi venni nelle contrade della Siria e della Cilicia;
\par 22 ma ero sconosciuto, di persona, alle chiese della Giudea, che sono in Cristo;
\par 23 esse sentivan soltanto dire: Colui che già ci perseguitava, ora predica la fede, che altra volta cercava di distruggere.
\par 24 E per causa mia glorificavano Iddio.

\chapter{2}

\par 1 Poi, passati quattordici anni, salii di nuovo a Gerusalemme con Barnaba, prendendo anche Tito con me.
\par 2 E vi salii in seguito ad una rivelazione, ed esposi loro l'Evangelo che io predico fra i Gentili, ma lo esposi privatamente ai più ragguardevoli, onde io non corressi o non avessi corso in vano.
\par 3 Ma neppur Tito, che era con me, ed era greco, fu costretto a farsi circoncidere;
\par 4 e questo a cagione dei falsi fratelli, introdottisi di soppiatto, i quali s'erano insinuati fra noi per spiare la libertà che abbiamo in Cristo Gesù, col fine di ridurci in servitù.
\par 5 Alle imposizioni di costoro noi non cedemmo neppur per un momento, affinché la verità del Vangelo rimanesse ferma tra voi.
\par 6 Ma quelli che godono di particolare considerazione (quali già siano stati a me non importa; Iddio non ha riguardi personali), quelli dico, che godono maggior considerazione non m'imposero nulla di più;
\par 7 anzi, quando videro che a me era stata affidata la evangelizzazione degli incirconcisi, come a Pietro quella de' circoncisi
\par 8 (poiché Colui che avea operato in Pietro per farlo apostolo della circoncisione aveva anche operato in me per farmi apostolo de' Gentili),
\par 9 e quando conobbero la grazia che m'era stata accordata, Giacomo e Cefa e Giovanni, che son reputati colonne, dettero a me ed a Barnaba la mano d'associazione perché noi andassimo ai Gentili, ed essi ai circoncisi;
\par 10 soltanto ci raccomandarono di ricordarci dei poveri; e questo mi sono studiato di farlo.
\par 11 Ma quando Cefa fu venuto ad Antiochia, io gli resistei in faccia perch'egli era da condannare.
\par 12 Difatti, prima che fossero venuti certuni provenienti da Giacomo, egli mangiava coi Gentili; ma quando costoro furono arrivati, egli prese a ritrarsi e a separarsi per timor di quelli della circoncisione.
\par 13 E gli altri Giudei si misero a simulare anch'essi con lui; talché perfino Barnaba fu trascinato dalla loro simulazione.
\par 14 Ma quando vidi che non procedevano con dirittura rispetto alla verità del Vangelo, io dissi a Cefa in presenza di tutti: Se tu, che sei Giudeo, vivi alla Gentile e non alla giudaica, come mai costringi i Gentili a giudaizzare?
\par 15 Noi che siam Giudei di nascita e non peccatori di fra i Gentili,
\par 16 avendo pur nondimeno riconosciuto che l'uomo non è giustificato per le opere della legge ma lo è soltanto per mezzo della fede in Cristo Gesù, abbiamo anche noi creduto in Cristo Gesù affin d'esser giustificati per la fede in Cristo e non per le opere della legge, poiché per le opere della legge nessuna carne sarà giustificata.
\par 17 Ma se nel cercare d'esser giustificati in Cristo, siamo anche noi trovati peccatori, Cristo è egli un ministro di peccato? Così non sia.
\par 18 Perché se io riedifico le cose che ho distrutte, mi dimostro trasgressore.
\par 19 Poiché per mezzo della legge io son morto alla legge per vivere a Dio.
\par 20 Sono stato crocifisso con Cristo, e non son più io che vivo, ma è Cristo che vive in me; e la vita che vivo ora nella carne, la vivo nella fede nel Figliuol di Dio il quale m'ha amato, e ha dato se stesso per me.
\par 21 Io non annullo la grazia di Dio; perché se la giustizia si ottiene per mezzo della legge, Cristo è dunque morto inutilmente.

\chapter{3}

\par 1 O Galati insensati, chi v'ha ammaliati, voi, dinanzi agli occhi dei quali Gesù Cristo crocifisso è stato ritratto al vivo?
\par 2 Questo soltanto desidero saper da voi: Avete voi ricevuto lo Spirito per la via delle opere della legge o per la predicazione della fede?
\par 3 Siete voi così insensati? Dopo aver cominciato con lo Spirito, volete ora raggiungere la perfezione con la carne?
\par 4 Avete voi sofferto tante cose invano? se pure è proprio invano.
\par 5 Colui dunque che vi somministra lo Spirito ed opera fra voi de' miracoli, lo fa Egli per la via delle opere della legge o per la predicazione della fede?
\par 6 Siccome Abramo credette a Dio e ciò gli fu messo in conto di giustizia,
\par 7 riconoscete anche voi che coloro i quali hanno la fede, son figliuoli d'Abramo.
\par 8 E la Scrittura, prevedendo che Dio giustificherebbe i Gentili per la fede, preannunziò ad Abramo questa buona novella: In te saranno benedette tutte le genti.
\par 9 Talché coloro che hanno la fede, sono benedetti col credente Abramo.
\par 10 Poiché tutti coloro che si basano sulle opere della legge sono sotto maledizione; perché è scritto: Maledetto chiunque non persevera in tutte le cose scritte nel libro della legge per metterle in pratica!
\par 11 Or che nessuno sia giustificato per la legge dinanzi a Dio, è manifesto perché il giusto vivrà per fede.
\par 12 Ma la legge non si basa sulla fede; anzi essa dice: Chi avrà messe in pratica queste cose, vivrà per via di esse.
\par 13 Cristo ci ha riscattati dalla maledizione della legge, essendo divenuto maledizione per noi (poiché sta scritto: Maledetto chiunque è appeso al legno),
\par 14 affinché la benedizione d'Abramo venisse sui Gentili in Cristo Gesù, affinché ricevessimo, per mezzo della fede, lo Spirito promesso.
\par 15 Fratelli, io parlo secondo le usanze degli uomini: Un patto che sia stato validamente concluso, sia pur soltanto un patto d'uomo, nessuno l'annulla o vi aggiunge alcun che.
\par 16 Or le promesse furon fatte ad Abramo e alla sua progenie. Non dice: "E alle progenie", come se si trattasse di molte; ma come parlando di una sola, dice: "E alla tua progenie", ch'è Cristo.
\par 17 Or io dico: Un patto già prima debitamente stabilito da Dio, la legge, che venne quattrocentotrent'anni dopo, non lo invalida in guisa da annullare la promessa.
\par 18 Perché, se l'eredità viene dalla legge, essa non viene più dalla promessa; ora ad Abramo Dio l'ha donata per via di promessa.
\par 19 Che cos'è dunque la legge? Essa fu aggiunta a motivo delle trasgressioni, finché venisse la progenie alla quale era stata fatta la promessa; e fu promulgata per mezzo d'angeli, per mano d'un mediatore.
\par 20 Ora, un mediatore non è mediatore d'uno solo; Dio, invece, è uno solo.
\par 21 La legge è essa dunque contraria alle promesse di Dio? Così non sia; perché se fosse stata data una legge capace di produrre la vita, allora sì, la giustizia sarebbe venuta dalla legge;
\par 22 ma la Scrittura ha rinchiuso ogni cosa sotto peccato, affinché i beni promessi alla fede in Gesù Cristo fossero dati ai credenti.
\par 23 Ma prima che venisse la fede eravamo tenuti rinchiusi in custodia sotto la legge, in attesa della fede che doveva esser rivelata.
\par 24 Talché la legge è stata il nostro pedagogo per condurci a Cristo, affinché fossimo giustificati per fede.
\par 25 Ma ora che la fede è venuta, noi non siamo più sotto pedagogo;
\par 26 perché siete tutti figliuoli di Dio, per la fede in Cristo Gesù.
\par 27 Poiché voi tutti che siete stati battezzati in Cristo vi siete rivestiti di Cristo.
\par 28 Non c'è qui né Giudeo né Greco; non c'è né schiavo né libero; non c'è né maschio né femmina; poiché voi tutti siete uno in Cristo Gesù.
\par 29 E se siete di Cristo, siete dunque progenie d'Abramo; eredi, secondo la promessa.

\chapter{4}

\par 1 Or io dico: Fin tanto che l'erede è fanciullo, non differisce in nulla dal servo, benché sia padrone di tutto;
\par 2 ma è sotto tutori e curatori fino al tempo prestabilito dal padre.
\par 3 Così anche noi, quando eravamo fanciulli, eravamo tenuti in servitù sotto gli elementi del mondo;
\par 4 ma quando giunse la pienezza de' tempi, Iddio mandò il suo Figliuolo, nato di donna, nato sotto la legge,
\par 5 per riscattare quelli che erano sotto la legge, affinché noi ricevessimo l'adozione di figliuoli.
\par 6 E perché siete figliuoli, Dio ha mandato lo Spirito del suo Figliuolo nei nostri cuori, che grida: Abba, Padre.
\par 7 Talché tu non sei più servo, ma figliuolo; e se sei figliuolo, sei anche erede per grazia di Dio.
\par 8 In quel tempo, è vero, non avendo conoscenza di Dio, voi avete servito a quelli che per natura non sono dèi;
\par 9 ma ora che avete conosciuto Dio, o piuttosto che siete stati conosciuti da Dio, come mai vi rivolgete di nuovo ai deboli e poveri elementi, ai quali volete di bel nuovo ricominciare a servire?
\par 10 Voi osservate giorni e mesi e stagioni ed anni.
\par 11 Io temo, quanto a voi, d'essermi invano affaticato per voi.
\par 12 Siate come son io, fratelli, ve ne prego, perché anch'io sono come voi.
\par 13 Voi non mi faceste alcun torto; anzi sapete bene che fu a motivo di una infermità della carne che vi evangelizzai la prima volta;
\par 14 e quella mia infermità corporale che era per voi una prova, voi non la sprezzaste né l'aveste a schifo; al contrario, mi accoglieste come un angelo di Dio. Come Cristo Gesù stesso.
\par 15 Dove son dunque le vostre proteste di gioia? Poiché io vi rendo questa testimonianza: che, se fosse stato possibile, vi sareste cavati gli occhi e me li avreste dati.
\par 16 Son io dunque divenuto vostro nemico dicendovi la verità?
\par 17 Costoro son zelanti di voi, ma non per fini onesti; anzi vi vogliono staccare da noi perché il vostro zelo si volga a loro.
\par 18 Or è una bella cosa essere oggetto dello zelo altrui nel bene, in ogni tempo, e non solo quando son presente fra voi.
\par 19 Figliuoletti miei, per i quali io son di nuovo in doglie finché Cristo sia formato in voi,
\par 20 oh come vorrei essere ora presente fra voi e cambiar tono perché son perplesso riguardo a voi!
\par 21 Ditemi: Voi che volete esser sotto la legge, non ascoltate voi la legge?
\par 22 Poiché sta scritto che Abramo ebbe due figliuoli: uno dalla schiava, e uno dalla donna libera;
\par 23 ma quello dalla schiava nacque secondo la carne; mentre quello dalla libera nacque in virtù della promessa.
\par 24 Le quali cose hanno un senso allegorico; poiché queste donne sono due patti, l'uno, del monte Sinai, genera per la schiavitù, ed è Agar.
\par 25 Infatti Agar è il monte Sinai in Arabia, e corrisponde alla Gerusalemme del tempo presente, la quale è schiava coi suoi figliuoli.
\par 26 Ma la Gerusalemme di sopra è libera, ed essa è nostra madre.
\par 27 Poich'egli è scritto: Rallegrati, o sterile che non partorivi! Prorompi in grida, tu che non avevi sentito doglie di parto! Poiché i figliuoli dell'abbandonata saranno più numerosi di quelli di colei che aveva il marito.
\par 28 Ora voi, fratelli, siete figliuoli della promessa alla maniera d'Isacco.
\par 29 Ma come allora colui ch'era nato secondo la carne perseguitava il nato secondo lo Spirito, così succede anche ora.
\par 30 Ma che dice la Scrittura? Caccia via la schiava e il suo figliuolo; perché il figliuolo della schiava non sarà erede col figliuolo della libera.
\par 31 Perciò, fratelli, noi non siam figliuoli della schiava, ma della libera.

\chapter{5}

\par 1 Cristo ci ha affrancati perché fossimo liberi; state dunque saldi, e non vi lasciate di nuovo porre sotto il giogo della schiavitù!
\par 2 Ecco, io, Paolo, vi dichiaro che, se vi fate circoncidere, Cristo non vi gioverà nulla.
\par 3 E da capo protesto ad ogni uomo che si fa circoncidere, ch'egli è obbligato ad osservare tutta quanta la legge.
\par 4 Voi che volete esser giustificati per la legge, avete rinunziato a Cristo; siete scaduti dalla grazia.
\par 5 Poiché, quanto a noi, è in ispirito, per fede, che aspettiamo la speranza della giustizia.
\par 6 Infatti, in Cristo Gesù, né la circoncisione né l'incirconcisione hanno valore alcuno; quel che vale è la fede operante per mezzo dell'amore.
\par 7 Voi correvate bene; chi vi ha fermati perché non ubbidiate alla verità?
\par 8 Una tal persuasione non viene da Colui che vi chiama.
\par 9 Un po' di lievito fa lievitare tutta la pasta.
\par 10 Riguardo a voi, io ho questa fiducia nel Signore, che non la penserete diversamente; ma colui che vi conturba ne porterà la pena, chiunque egli sia.
\par 11 Quanto a me, fratelli, s'io predico ancora la circoncisione, perché sono ancora perseguitato? Lo scandalo della croce sarebbe allora tolto via.
\par 12 Si facessero pur anche evirare quelli che vi mettono sottosopra!
\par 13 Perché, fratelli, voi siete stati chiamati a libertà; soltanto non fate della libertà un'occasione alla carne, ma per mezzo dell'amore servite gli uni gli altri;
\par 14 poiché tutta la legge è adempiuta in quest'unica parola: Ama il tuo prossimo come te stesso.
\par 15 Ma se vi mordete e divorate gli uni gli altri, guardate di non esser consumati gli uni dagli altri.
\par 16 Or io dico: Camminate per lo Spirito e non adempirete i desiderî della carne.
\par 17 Perché la carne ha desiderî contrarî allo Spirito, e lo Spirito ha desiderî contrarî alla carne; sono cose opposte fra loro; in guisa che non potete fare quel che vorreste.
\par 18 Ma se siete condotti dallo Spirito, voi non siete sotto la legge.
\par 19 Or le opere della carne sono manifeste, e sono: fornicazione, impurità, dissolutezza,
\par 20 idolatria, stregoneria, inimicizie, discordia, gelosia, ire, contese, divisioni,
\par 21 sètte, invidie, ubriachezze, gozzoviglie, e altre simili cose; circa le quali io vi prevengo, come anche v'ho già prevenuti, che quelli che fanno tali cose non erederanno il regno di Dio.
\par 22 Il frutto dello Spirito, invece, è amore, allegrezza, pace, longanimità, benignità, bontà, fedeltà, dolcezza, temperanza;
\par 23 contro tali cose non c'è legge.
\par 24 E quelli che son di Cristo hanno crocifisso la carne con le sue passioni e le sue concupiscenze.
\par 25 Se viviamo per lo Spirito, camminiamo altresì per lo Spirito.
\par 26 Non siamo vanagloriosi, provocandoci e invidiandoci gli uni gli altri.

\chapter{6}

\par 1 Fratelli, quand'anche uno sia stato còlto in qualche fallo, voi, che siete spirituali, rialzatelo con spirito di mansuetudine. E bada bene a te stesso, che talora anche tu non sii tentato.
\par 2 Portate i pesi gli uni degli altri, e così adempirete la legge di Cristo.
\par 3 Poiché se alcuno si stima esser qualcosa pur non essendo nulla, egli inganna se stesso.
\par 4 Ciascuno esamini invece l'opera propria; e allora avrà motivo di gloriarsi rispetto a se stesso soltanto, e non rispetto ad altri.
\par 5 Poiché ciascuno porterà il suo proprio carico.
\par 6 Colui che viene ammaestrato nella Parola faccia parte di tutti i suoi beni a chi l'ammaestra.
\par 7 Non v'ingannate; non si può beffarsi di Dio; poiché quello che l'uomo avrà seminato, quello pure mieterà.
\par 8 Perché chi semina per la propria carne, mieterà dalla carne corruzione; ma chi semina per lo Spirito, mieterà dallo Spirito vita eterna.
\par 9 E non ci scoraggiamo nel far il bene; perché, se non ci stanchiamo, mieteremo a suo tempo.
\par 10 Così dunque, secondo che ne abbiamo l'opportunità, facciam del bene a tutti; ma specialmente a quei della famiglia dei credenti.
\par 11 Guardate con che grosso carattere v'ho scritto, di mia propria mano.
\par 12 Tutti coloro che vogliono far bella figura nella carne, vi costringono a farvi circoncidere, e ciò al solo fine di non esser perseguitati per la croce di Cristo.
\par 13 Poiché neppur quelli stessi che son circoncisi, osservano la legge; ma vogliono che siate circoncisi per potersi gloriare della vostra carne.
\par 14 Ma quanto a me, non sia mai ch'io mi glorî d'altro che della croce del Signor nostro Gesù Cristo, mediante la quale il mondo, per me, è stato crocifisso, e io sono stato crocifisso per il mondo.
\par 15 Poiché tanto la circoncisione che l'incirconcisione non son nulla; quel che importa è l'essere una nuova creatura.
\par 16 E su quanti cammineranno secondo questa regola siano pace e misericordia, e così siano sull'Israele di Dio.
\par 17 Da ora in poi nessuno mi dia molestia, perché io porto nel mio corpo le stimmate di Gesù.
\par 18 La grazia del Signore nostro Gesù Cristo sia col vostro spirito, fratelli. Amen.


\end{document}