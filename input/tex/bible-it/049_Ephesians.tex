\begin{document}

\title{Ephesians}


\chapter{1}

\par 1 Paolo, apostolo di Cristo Gesù per volontà di Dio, ai santi che sono in Efeso ed ai fedeli in Cristo Gesù.
\par 2 Grazia a voi e pace da Dio, Padre nostro, e dal Signor Gesù Cristo.
\par 3 Benedetto sia l'Iddio e Padre del nostro Signor Gesù Cristo, il quale ci ha benedetti d'ogni benedizione spirituale ne' luoghi celesti in Cristo,
\par 4 siccome in lui ci ha eletti, prima della fondazione del mondo, affinché fossimo santi ed irreprensibili dinanzi a lui nell'amore,
\par 5 avendoci predestinati ad essere adottati, per mezzo di Gesù Cristo, come suoi figliuoli, secondo il beneplacito della sua volontà:
\par 6 a lode della gloria della sua grazia, la quale Egli ci ha largita nell'amato suo.
\par 7 Poiché in lui noi abbiamo la redenzione mediante il suo sangue, la remissione de' peccati, secondo le ricchezze della sua grazia;
\par 8 della quale Egli è stato abbondante in verso noi, dandoci ogni sorta di sapienza e di intelligenza,
\par 9 col farci conoscere il mistero della sua volontà, giusta il disegno benevolo ch'Egli avea già prima in se stesso formato,
\par 10 per tradurlo in atto nella pienezza dei tempi, e che consiste nel raccogliere sotto un sol capo, in Cristo, tutte le cose: tanto quelle che son nei cieli, quanto quelle che son sopra la terra.
\par 11 In lui, dico, nel quale siamo pur stati fatti eredi, a ciò predestinati conforme al proposito di Colui che opera tutte le cose secondo il consiglio della propria volontà,
\par 12 affinché fossimo a lode della sua gloria, noi, che per i primi abbiamo sperato in Cristo.
\par 13 In lui voi pure, dopo aver udito la parola della verità, l'evangelo della vostra salvazione, in lui avendo creduto, avete ricevuto il suggello dello Spirito Santo che era stato promesso,
\par 14 il quale è pegno della nostra eredità fino alla piena redenzione di quelli che Dio s'è acquistati, a lode della sua gloria.
\par 15 Perciò anch'io, avendo udito parlare della fede vostra nel Signor Gesù e del vostro amore per tutti i santi,
\par 16 non resto mai dal render grazie per voi, facendo di voi menzione nelle mie orazioni,
\par 17 affinché l'Iddio del Signor nostro Gesù Cristo, il Padre della gloria, vi dia uno spirito di sapienza e di rivelazione per la piena conoscenza di lui,
\par 18 ed illumini gli occhi del vostro cuore, affinché sappiate a quale speranza Egli v'abbia chiamati, qual sia la ricchezza della gloria della sua eredità nei santi,
\par 19 e qual sia verso noi che crediamo, l'immensità della sua potenza.
\par 20 La qual potente efficacia della sua forza Egli ha spiegata in Cristo, quando lo risuscitò dai morti, e lo fece sedere alla propria destra ne' luoghi celesti,
\par 21 al di sopra di ogni principato e autorità e potestà e signoria, e d'ogni altro nome che si nomina non solo in questo mondo, ma anche in quello a venire.
\par 22 Ogni cosa Ei gli ha posta sotto ai piedi, e l'ha dato per capo supremo alla Chiesa,
\par 23 che è il corpo di lui, il compimento di colui che porta a compimento ogni cosa in tutti.

\chapter{2}

\par 1 E voi pure ha vivificati, voi ch'eravate morti ne' vostri falli e ne' vostri peccati,
\par 2 ai quali un tempo vi abbandonaste seguendo l'andazzo di questo mondo, seguendo il principe della potestà dell'aria, di quello spirito che opera al presente negli uomini ribelli;
\par 3 nel numero dei quali noi tutti pure, immersi nelle nostre concupiscenze carnali, siamo vissuti altra volta ubbidendo alle voglie della carne e dei pensieri, ed eravamo per natura figliuoli d'ira, come gli altri.
\par 4 Ma Dio, che è ricco in misericordia, per il grande amore del quale ci ha amati,
\par 5 anche quand'eravamo morti nei falli, ci ha vivificati con Cristo (egli è per grazia che siete stati salvati),
\par 6 e ci ha risuscitati con lui e con lui ci ha fatti sedere ne' luoghi celesti in Cristo Gesù,
\par 7 per mostrare nelle età a venire l'immensa ricchezza della sua grazia, nella benignità ch'Egli ha avuta per noi in Cristo Gesù.
\par 8 Poiché gli è per grazia che voi siete stati salvati, mediante la fede; e ciò non vien da voi; è il dono di Dio.
\par 9 Non è in virtù d'opere, affinché niuno si glorî;
\par 10 perché noi siamo fattura di lui, essendo stati creati in Cristo Gesù per le buone opere, le quali Iddio ha innanzi preparate affinché le pratichiamo.
\par 11 Perciò, ricordatevi che un tempo voi, Gentili di nascita, chiamati i non circoncisi da quelli che si dicono i circoncisi, perché tali sono nella carne per mano d'uomo, voi, dico, ricordatevi che
\par 12 in quel tempo eravate senza Cristo, esclusi dalla cittadinanza d'Israele ed estranei ai patti della promessa, non avendo speranza, ed essendo senza Dio nel mondo.
\par 13 Ma ora, in Cristo Gesù, voi che già eravate lontani, siete stati avvicinati mediante il sangue di Cristo.
\par 14 Poiché è lui ch'è la nostra pace; lui che dei due popoli ne ha fatto un solo ed ha abbattuto il muro di separazione
\par 15 con l'abolire nella sua carne la causa dell'inimicizia, la legge fatta di comandamenti in forma di precetti, affin di creare in se stesso dei due un solo uomo nuovo, facendo la pace;
\par 16 ed affin di riconciliarli ambedue in un corpo unico con Dio, mediante la sua croce, sulla quale fece morire l'inimicizia loro.
\par 17 E con la sua venuta ha annunziato la buona novella della pace a voi che eravate lontani, e della pace a quelli che eran vicini.
\par 18 Poiché per mezzo di lui e gli uni e gli altri abbiamo accesso al Padre in un medesimo Spirito.
\par 19 Voi dunque non siete più né forestieri né avventizî; ma siete concittadini dei santi e membri della famiglia di Dio,
\par 20 essendo stati edificati sul fondamento degli apostoli e de' profeti, essendo Cristo Gesù stesso la pietra angolare,
\par 21 sulla quale l'edificio intero, ben collegato insieme, si va innalzando per essere un tempio santo nel Signore.
\par 22 Ed in lui voi pure entrate a far parte dell'edificio, che ha da servire di dimora a Dio per lo Spirito.

\chapter{3}

\par 1 Per questa cagione io, Paolo, il carcerato di Cristo Gesù per voi, o Gentili...
\par 2 (Poiché senza dubbio avete udito di quale grazia Iddio m'abbia fatto dispensatore per voi;
\par 3 come per rivelazione mi sia stato fatto conoscere il mistero, di cui più sopra vi ho scritto in poche parole;
\par 4 le quali leggendo, potete capire la intelligenza che io ho del mistero di Cristo.
\par 5 Il qual mistero, nelle altre età, non fu dato a conoscere ai figliuoli degli uomini nel modo che ora, per mezzo dello Spirito, è stato rivelato ai santi apostoli e profeti di Lui;
\par 6 vale a dire, che i Gentili sono eredi con noi, membra con noi d'un medesimo corpo e con noi partecipi della promessa fatta in Cristo Gesù mediante l'Evangelo,
\par 7 del quale io sono stato fatto ministro, in virtù del dono della grazia di Dio largitami secondo la virtù della sua potenza.
\par 8 A me, dico, che son da meno del minimo di tutti i santi, è stata data questa grazia di recare ai Gentili il buon annunzio delle non investigabili ricchezze di Cristo,
\par 9 e di manifestare a tutti quale sia il piano seguito da Dio riguardo al mistero che è stato fin dalle più remote età nascosto in Dio, il Creatore di tutte le cose,
\par 10 affinché nel tempo presente, ai principati ed alle potestà, ne' luoghi celesti, sia data a conoscere, per mezzo della Chiesa, la infinitamente varia sapienza di Dio,
\par 11 conforme al proponimento eterno ch'Egli ha mandato ad effetto nel nostro Signore, Cristo Gesù;
\par 12 nel quale abbiamo la libertà d'accostarci a Dio, con piena fiducia, mediante la fede in lui.
\par 13 Perciò io vi chieggo che non veniate meno nell'animo a motivo delle tribolazioni ch'io patisco per voi, poiché esse sono la vostra gloria).
\par 14 ...Per questa cagione, dico, io piego le ginocchia dinanzi al Padre,
\par 15 dal quale ogni famiglia ne' cieli e sulla terra prende nome,
\par 16 perch'Egli vi dia, secondo le ricchezze della sua gloria, d'esser potentemente fortificati mediante lo Spirito suo, nell'uomo interiore,
\par 17 e faccia sì che Cristo abiti per mezzo della fede nei vostri cuori,
\par 18 affinché, essendo radicati e fondati nell'amore, siate resi capaci di abbracciare con tutti i santi qual sia la larghezza, la lunghezza, l'altezza e la profondità dell'amore di Cristo,
\par 19 e di conoscere questo amore che sorpassa ogni conoscenza, affinché giungiate ad esser ripieni di tutta la pienezza di Dio.
\par 20 Or a Colui che può, mediante la potenza che opera in noi, fare infinitamente al di là di quel che domandiamo o pensiamo,
\par 21 a Lui sia la gloria nella Chiesa e in Cristo Gesù, per tutte le età, ne' secoli de' secoli. Amen.

\chapter{4}

\par 1 Io dunque, il carcerato nel Signore, vi esorto a condurvi in modo degno della vocazione che vi è stata rivolta,
\par 2 con ogni umiltà e mansuetudine, con longanimità, sopportandovi gli uni gli altri con amore,
\par 3 studiandovi di conservare l'unità dello Spirito col vincolo della pace.
\par 4 V'è un corpo unico ed un unico Spirito, come pure siete stati chiamati ad un'unica speranza, quella della vostra vocazione.
\par 5 V'è un solo Signore, una sola fede, un solo battesimo,
\par 6 un Dio unico e Padre di tutti, che è sopra tutti, fra tutti ed in tutti.
\par 7 Ma a ciascun di noi la grazia è stata data secondo la misura del dono largito da Cristo.
\par 8 Egli è per questo che è detto: Salito in alto, egli ha menato in cattività un gran numero di prigioni ed ha fatto dei doni agli uomini.
\par 9 Or questo è salito che cosa vuol dire se non che egli era anche disceso nelle parti più basse della terra?
\par 10 Colui che è disceso, è lo stesso che è salito al disopra di tutti i cieli, affinché riempisse ogni cosa.
\par 11 Ed è lui che ha dato gli uni, come apostoli; gli altri, come profeti; gli altri, come evangelisti; gli altri, come pastori e dottori,
\par 12 per il perfezionamento dei santi, per l'opera del ministerio, per la edificazione del corpo di Cristo,
\par 13 finché tutti siamo arrivati all'unità della fede e della piena conoscenza del Figliuol di Dio, allo stato d'uomini fatti, all'altezza della statura perfetta di Cristo;
\par 14 affinché non siamo più de' bambini, sballottati e portati qua e là da ogni vento di dottrina, per la frode degli uomini, per l'astuzia loro nelle arti seduttrici dell'errore,
\par 15 ma che, seguitando verità in carità, noi cresciamo in ogni cosa verso colui che è il capo, cioè Cristo.
\par 16 Da lui tutto il corpo ben collegato e ben connesso mediante l'aiuto fornito da tutte le giunture, trae il proprio sviluppo nella misura del vigore d'ogni singola parte, per edificar se stesso nell'amore.
\par 17 Questo dunque io dico ed attesto nel Signore, che non vi conduciate più come si conducono i pagani nella vanità de' loro pensieri,
\par 18 con l'intelligenza ottenebrata, estranei alla vita di Dio, a motivo della ignoranza che è in loro, a motivo dell'induramento del cuor loro.
\par 19 Essi, avendo perduto ogni sentimento, si sono abbandonati alla dissolutezza fino a commettere ogni sorta di impurità con insaziabile avidità.
\par 20 Ma quant'è a voi, non è così che avete imparato a conoscer Cristo.
\par 21 Se pur l'avete udito ed in lui siete stati ammaestrati secondo la verità che è in Gesù,
\par 22 avete imparato, per quanto concerne la vostra condotta di prima, a spogliarvi del vecchio uomo che si corrompe seguendo le passioni ingannatrici;
\par 23 ad essere invece rinnovati nello spirito della vostra mente,
\par 24 e a rivestire l'uomo nuovo che è creato all'immagine di Dio nella giustizia e nella santità che procedono dalla verità.
\par 25 Perciò, bandita la menzogna, ognuno dica la verità al suo prossimo perché siamo membra gli uni degli altri.
\par 26 Adiratevi e non peccate; il sole non tramonti sopra il vostro cruccio
\par 27 e non fate posto al diavolo.
\par 28 Chi rubava non rubi più, ma s'affatichi piuttosto a lavorare onestamente con le proprie mani, onde abbia di che far parte a colui che ha bisogno.
\par 29 Niuna mala parola esca dalla vostra bocca; ma se ne avete alcuna buona che edifichi, secondo il bisogno, ditela, affinché conferisca grazia a chi l'ascolta.
\par 30 E non contristate lo Spirito Santo di Dio col quale siete stati suggellati per il giorno della redenzione.
\par 31 Sia tolta via da voi ogni amarezza, ogni cruccio ed ira e clamore e parola offensiva con ogni sorta di malignità.
\par 32 Siate invece gli uni verso gli altri benigni, misericordiosi, perdonandovi a vicenda, come anche Dio vi ha perdonati in Cristo.

\chapter{5}

\par 1 Siate dunque imitatori di Dio, come figliuoli suoi diletti;
\par 2 camminate nell'amore come anche Cristo vi ha amati e ha dato se stesso per noi in offerta e sacrificio a Dio, qual profumo d'odor soave.
\par 3 Ma come si conviene a dei santi, né fornicazione, né alcuna impurità, né avarizia, sia neppur nominata fra voi;
\par 4 né disonestà, né buffonerie, né facezie scurrili, che son cose sconvenienti; ma piuttosto, rendimento di grazie.
\par 5 Poiché voi sapete molto bene che niun fornicatore o impuro, o avaro (che è un idolatra), ha eredità nel regno di Cristo e di Dio.
\par 6 Niuno vi seduca con vani ragionamenti; poiché è per queste cose che l'ira di Dio viene sugli uomini ribelli.
\par 7 Non siate dunque loro compagni;
\par 8 perché già eravate tenebre, ma ora siete luce nel Signore. Conducetevi come figliuoli di luce
\par 9 (poiché il frutto della luce consiste in tutto ciò che è bontà e giustizia e verità),
\par 10 esaminando che cosa sia accetto al Signore.
\par 11 E non partecipate alle opere infruttuose delle tenebre; anzi, piuttosto riprendetele;
\par 12 poiché egli è disonesto pur di dire le cose che si fanno da costoro in occulto.
\par 13 Ma tutte le cose, quando sono riprese dalla luce, diventano manifeste; poiché tutto ciò che è manifesto, è luce.
\par 14 Perciò dice: Risvegliati, o tu che dormi, e risorgi da' morti, e Cristo t'inonderà di luce.
\par 15 Guardate dunque con diligenza come vi conducete; non da stolti, ma da savî;
\par 16 approfittando delle occasioni, perché i giorni sono malvagi.
\par 17 Perciò non siate disavveduti, ma intendete bene quale sia la volontà del Signore.
\par 18 E non v'inebriate di vino; esso porta alla dissolutezza; ma siate ripieni dello Spirito,
\par 19 parlandovi con salmi ed inni e canzoni spirituali, cantando e salmeggiando col cuor vostro al Signore;
\par 20 rendendo del continuo grazie d'ogni cosa a Dio e Padre, nel nome del Signor nostro Gesù Cristo;
\par 21 sottoponendovi gli uni agli altri nel timore di Cristo.
\par 22 Mogli, siate soggette ai vostri mariti, come al Signore;
\par 23 poiché il marito è capo della moglie, come anche Cristo è capo della Chiesa, egli, che è il Salvatore del corpo.
\par 24 Ma come la Chiesa è soggetta a Cristo, così debbono anche le mogli esser soggette a' loro mariti in ogni cosa.
\par 25 Mariti, amate le vostre mogli, come anche Cristo ha amato la Chiesa e ha dato se stesso per lei,
\par 26 affin di santificarla, dopo averla purificata col lavacro dell'acqua mediante la Parola,
\par 27 affin di far egli stesso comparire dinanzi a sé questa Chiesa, gloriosa, senza macchia, senza ruga o cosa alcun simile, ma santa ed irreprensibile.
\par 28 Allo stesso modo anche i mariti debbono amare le loro mogli, come i loro proprî corpi. Chi ama sua moglie ama se stesso.
\par 29 Poiché niuno ebbe mai in odio la sua carne; anzi la nutre e la cura teneramente, come anche Cristo fa per la Chiesa,
\par 30 poiché noi siamo membra del suo corpo.
\par 31 Perciò l'uomo lascerà suo padre e sua madre e s'unirà a sua moglie, e i due diverranno una stessa carne.
\par 32 Questo mistero è grande; dico questo, riguardo a Cristo ed alla Chiesa.
\par 33 Ma d'altronde, anche fra voi, ciascuno individualmente così ami sua moglie, come ama se stesso; e altresì la moglie rispetti il marito.

\chapter{6}

\par 1 Figliuoli, ubbidite nel Signore ai vostri genitori, poiché ciò è giusto.
\par 2 Onora tuo padre e tua madre (è questo il primo comandamento con promessa)
\par 3 affinché ti sia bene e tu abbia lunga vita sulla terra.
\par 4 E voi, padri, non provocate ad ira i vostri figliuoli, ma allevateli in disciplina e in ammonizione del Signore.
\par 5 Servi, ubbidite ai vostri signori secondo la carne, con timore e tremore, nella semplicità del cuor vostro, come a Cristo,
\par 6 non servendo all'occhio come per piacere agli uomini, ma, come servi di Cristo, facendo il voler di Dio d'animo;
\par 7 servendo con benevolenza, come se serviste il Signore e non gli uomini;
\par 8 sapendo che ognuno, quand'abbia fatto qualche bene, ne riceverà la retribuzione dal Signore, servo o libero che sia.
\par 9 E voi, signori, fate altrettanto rispetto a loro; astenendovi dalle minacce, sapendo che il Signor vostro e loro è nel cielo, e che dinanzi a lui non v'è riguardo a qualità di persone.
\par 10 Del rimanente, fortificatevi nel Signore e nella forza della sua possanza.
\par 11 Rivestitevi della completa armatura di Dio, onde possiate star saldi contro le insidie del diavolo;
\par 12 poiché il combattimento nostro non è contro sangue e carne, ma contro i principati, contro le potestà, contro i dominatori di questo mondo di tenebre, contro le forze spirituali della malvagità, che sono ne' luoghi celesti.
\par 13 Perciò, prendete la completa armatura di Dio, affinché possiate resistere nel giorno malvagio, e dopo aver compiuto tutto il dover vostro, restare in piè.
\par 14 State dunque saldi, avendo presa la verità a cintura dei fianchi, essendovi rivestiti della corazza della giustizia
\par 15 e calzati i piedi della prontezza che dà l'Evangelo della pace;
\par 16 prendendo oltre a tutto ciò lo scudo della fede, col quale potrete spegnere tutti i dardi infocati del maligno.
\par 17 Prendete anche l'elmo della salvezza e la spada dello Spirito, che è la Parola di Dio;
\par 18 orando in ogni tempo, per lo Spirito, con ogni sorta di preghiere e di supplicazioni; ed a questo vegliando con ogni perseveranza e supplicazione per tutti i santi,
\par 19 ed anche per me, acciocché mi sia dato di parlare apertamente per far conoscere con franchezza il mistero dell'Evangelo,
\par 20 per il quale io sono ambasciatore in catena; affinché io l'annunzî francamente, come convien ch'io ne parli.
\par 21 Or acciocché anche voi sappiate lo stato mio e quello ch'io fo, Tichico, il caro fratello e fedel ministro del Signore, vi farà saper tutto.
\par 22 Ve l'ho mandato apposta affinché abbiate conoscenza dello stato nostro ed ei consoli i vostri cuori.
\par 23 Pace a' fratelli e amore con fede, da Dio Padre e dal Signor Gesù Cristo.
\par 24 La grazia sia con tutti quelli che amano il Signor nostro Gesù Cristo con purità incorrotta.


\end{document}