\begin{document}

\title{Colossesi}


\chapter{1}

\par 1 Paolo, apostolo di Cristo Gesù per volontà di Dio, e il fratello Timoteo,
\par 2 ai santi e fedeli fratelli in Cristo che sono in Colosse, grazia a voi e pace da Dio nostro Padre.
\par 3 Noi rendiamo grazie a Dio, Padre del Signor nostro Gesù Cristo, nelle continue preghiere che facciamo per voi,
\par 4 avendo udito parlare della vostra fede in Cristo Gesù e dell'amore che avete per tutti i santi,
\par 5 a motivo della speranza che vi è riposta nei cieli; speranza che avete da tempo conosciuta mediante la predicazione della verità del Vangelo
\par 6 che è pervenuto sino a voi, come sta portando frutto e crescendo in tutto il mondo nel modo che fa pure tra voi dal giorno che udiste e conosceste la grazia di Dio in verità,
\par 7 secondo quel che avete imparato da Epafra, il nostro caro compagno di servizio, che è fedel ministro di Cristo per voi,
\par 8 e che ci ha anche fatto conoscere il vostro amore nello Spirito.
\par 9 Perciò anche noi, dal giorno che abbiamo ciò udito, non cessiamo di pregare per voi, e di domandare che siate ripieni della profonda conoscenza della volontà di Dio in ogni sapienza e intelligenza spirituale,
\par 10 affinché camminiate in modo degno del Signore per piacergli in ogni cosa, portando frutto in ogni opera buona e crescendo nella conoscenza di Dio;
\par 11 essendo fortificati in ogni forza secondo la potenza della sua gloria, onde possiate essere in tutto pazienti e longanimi;
\par 12 e rendendo grazie con allegrezza al Padre che vi ha messi in grado di partecipare alla sorte dei santi nella luce.
\par 13 Egli ci ha riscossi dalla potestà delle tenebre e ci ha trasportati nel regno del suo amato Figliuolo,
\par 14 nel quale abbiamo la redenzione, la remissione dei peccati;
\par 15 il quale è l'immagine dell'invisibile Iddio, il primogenito d'ogni creatura;
\par 16 poiché in lui sono state create tutte le cose, che sono nei cieli e sulla terra; le visibili e le invisibili; siano troni, siano signorie, siano principati, siano potestà; tutte le cose sono state create per mezzo di lui e in vista di lui;
\par 17 ed egli è avanti ogni cosa, e tutte le cose sussistono in lui.
\par 18 Ed egli è il capo del corpo, cioè della Chiesa; egli che è il principio, il primogenito dai morti, onde in ogni cosa abbia il primato.
\par 19 Poiché in lui si compiacque il Padre di far abitare tutta la pienezza
\par 20 e di riconciliare con sé tutte le cose per mezzo di lui, avendo fatto la pace mediante il sangue della croce d'esso; per mezzo di lui, dico, tanto le cose che sono sulla terra, quanto quelle che sono nei cieli.
\par 21 E voi, che già eravate estranei e nemici nella vostra mente e nelle vostre opere malvage,
\par 22 ora Iddio vi ha riconciliati nel corpo della carne di lui, per mezzo della morte d'esso, per farvi comparire davanti a sé santi e immacolati e irreprensibili,
\par 23 se pur perseverate nella fede, fondati e saldi, e non essendo smossi dalla speranza dell'Evangelo che avete udito, che fu predicato in tutta la creazione sotto il cielo, e del quale io, Paolo, sono stato fatto ministro.
\par 24 Ora mi rallegro nelle mie sofferenze per voi; e quel che manca alle afflizioni di Cristo lo compio nella mia carne a pro del corpo di lui che è la Chiesa;
\par 25 della quale io sono stato fatto ministro, secondo l'ufficio datomi da Dio per voi di annunziare nella sua pienezza la parola di Dio,
\par 26 cioè, il mistero, che è stato occulto da tutti i secoli e da tutte le generazioni, ma che ora è stato manifestato ai santi di lui;
\par 27 ai quali Iddio ha voluto far conoscere qual sia la ricchezza della gloria di questo mistero fra i Gentili, che è Cristo in voi, speranza della gloria;
\par 28 il quale noi proclamiamo, ammonendo ciascun uomo e ciascun uomo ammaestrando in ogni sapienza, affinché presentiamo ogni uomo, perfetto in Cristo.
\par 29 A questo fine io m'affatico, combattendo secondo l'energia sua, che opera in me con potenza.

\chapter{2}

\par 1 Poiché desidero che sappiate qual arduo combattimento io sostengo per voi e per quelli di Laodicea e per tutti quelli che non hanno veduto la mia faccia;
\par 2 affinché siano confortati nei loro cuori essendo stretti insieme dall'amore, mirando a tutte le ricchezze della piena certezza dell'intelligenza, per giungere alla completa conoscenza del mistero di Dio:
\par 3 cioè di Cristo, nel quale tutti i tesori della sapienza e della conoscenza sono nascosti.
\par 4 Questo io dico affinché nessuno v'inganni con parole seducenti,
\par 5 perché, sebbene sia assente di persona, pure son con voi in ispirito, rallegrandomi e mirando il vostro ordine e la fermezza della vostra fede in Cristo.
\par 6 Come dunque avete ricevuto Cristo Gesù il Signore, così camminate uniti a lui,
\par 7 essendo radicati ed edificati in lui e confermati nella fede, come v'è stato insegnato, e abbondando in azioni di grazie.
\par 8 Guardate che non vi sia alcuno che faccia di voi sua preda con la filosofia e con vanità ingannatrice secondo la tradizione degli uomini, gli elementi del mondo, e non secondo Cristo;
\par 9 poiché in lui abita corporalmente tutta la pienezza della Deità,
\par 10 e in lui voi avete tutto pienamente. Egli è il capo d'ogni principato e d'ogni potestà;
\par 11 in lui voi siete anche stati circoncisi d'una circoncisione non fatta da mano d'uomo, ma della circoncisione di Cristo, che consiste nello spogliamento del corpo della carne:
\par 12 essendo stati con lui sepolti nel battesimo, nel quale siete anche stati risuscitati con lui mediante la fede nella potenza di Dio che ha risuscitato lui dai morti.
\par 13 E voi, che eravate morti ne' falli e nella incirconcisione della vostra carne, voi, dico, Egli ha vivificati con lui, avendoci perdonato tutti i falli,
\par 14 avendo cancellato l'atto accusatore scritto in precetti, il quale ci era contrario; e quell'atto ha tolto di mezzo, inchiodandolo sulla croce;
\par 15 e avendo spogliato i principati e le potestà ne ha fatto un pubblico spettacolo, trionfando su di loro per mezzo della croce.
\par 16 Nessuno dunque vi giudichi quanto al mangiare o al bere, o rispetto a feste, o a novilunî o a sabati,
\par 17 che sono l'ombra di cose che doveano avvenire; ma il corpo è di Cristo.
\par 18 Nessuno a suo talento vi defraudi del vostro premio per via d'umiltà e di culto degli angeli affidandosi alle proprie visioni, gonfiato di vanità dalla sua mente carnale,
\par 19 e non attenendosi al Capo, dal quale tutto il corpo, ben fornito e congiunto insieme per via delle giunture e articolazioni, prende l'accrescimento che viene da Dio.
\par 20 Se siete morti con Cristo agli elementi del mondo, perché, come se viveste nel mondo, vi lasciate imporre de' precetti, quali:
\par 21 Non toccare, non assaggiare, non maneggiare
\par 22 (cose tutte destinate a perire con l'uso), secondo i comandamenti e le dottrine degli uomini?
\par 23 Quelle cose hanno, è vero, riputazione di sapienza per quel tanto che è in esse di culto volontario, di umiltà, e di austerità nel trattare il corpo; ma non hanno alcun valore e servon solo a soddisfare la carne.

\chapter{3}

\par 1 Se dunque voi siete stati risuscitati con Cristo, cercate le cose di sopra dove Cristo è seduto alla destra di Dio.
\par 2 Abbiate l'animo alle cose di sopra, non a quelle che son sulla terra;
\par 3 poiché voi moriste, e la vita vostra è nascosta con Cristo in Dio.
\par 4 Quando Cristo, la vita nostra, sarà manifestato, allora anche voi sarete con lui manifestati in gloria.
\par 5 Fate dunque morire le vostre membra che son sulla terra: fornicazione, impurità, lussuria, mala concupiscenza e cupidigia, la quale è idolatria.
\par 6 Per queste cose viene l'ira di Dio sui figliuoli della disubbidienza;
\par 7 e in quelle camminaste un tempo anche voi, quando vivevate in esse.
\par 8 Ma ora deponete anche voi tutte queste cose: ira, collera, malignità, maldicenza, e non vi escano di bocca parole disoneste.
\par 9 Non mentite gli uni agli altri,
\par 10 giacché avete svestito l'uomo vecchio coi suoi atti e rivestito il nuovo, che si va rinnovando in conoscenza ad immagine di Colui che l'ha creato.
\par 11 Qui non c'è Greco e Giudeo, circoncisione e incirconcisione, barbaro, Scita, schiavo, libero, ma Cristo è ogni cosa e in tutti.
\par 12 Vestitevi dunque, come eletti di Dio, santi ed amati, di tenera compassione, di benignità, di umiltà, di dolcezza, di longanimità;
\par 13 sopportandovi gli uni gli altri e perdonandovi a vicenda, se uno ha di che dolersi d'un altro. Come il Signore vi ha perdonati, così fate anche voi.
\par 14 E sopra tutte queste cose vestitevi della carità che è il vincolo della perfezione.
\par 15 E la pace di Cristo, alla quale siete stati chiamati per essere un sol corpo, regni nei vostri cuori; e siate riconoscenti.
\par 16 La parola di Cristo abiti in voi doviziosamente; ammaestrandovi ed ammonendovi gli uni gli altri con ogni sapienza, cantando di cuore a Dio, sotto l'impulso della grazia, salmi, inni, e cantici spirituali.
\par 17 E qualunque cosa facciate, in parola o in opera, fate ogni cosa nel nome del Signor Gesù, rendendo grazie a Dio Padre per mezzo di lui.
\par 18 Mogli, siate soggette ai vostri mariti, come si conviene nel Signore.
\par 19 Mariti, amate le vostre mogli, e non v'inasprite contro a loro.
\par 20 Figliuoli, ubbidite ai vostri genitori in ogni cosa, poiché questo è accettevole al Signore.
\par 21 Padri, non irritate i vostri figliuoli, affinché non si scoraggino.
\par 22 Servi, ubbidite in ogni cosa ai vostri padroni secondo la carne; non servendoli soltanto quando vi vedono come per piacere agli uomini, ma con semplicità di cuore, temendo il Signore.
\par 23 Qualunque cosa facciate, operate di buon animo, come per il Signore e non per gli uomini;
\par 24 sapendo che dal Signore riceverete per ricompensa l'eredità.
\par 25 Servite a Cristo il Signore! Poiché chi fa torto riceverà la retribuzione del torto che avrà fatto; e non ci son riguardi personali.

\chapter{4}

\par 1 Padroni, date ai vostri servi ciò che è giusto ed equo, sapendo che anche voi avete un Padrone nel cielo.
\par 2 Perseverate nella preghiera, vegliando in essa con rendimento di grazie;
\par 3 pregando in pari tempo anche per noi, affinché Iddio ci apra una porta per la Parola onde possiamo annunziare il mistero di Cristo, a cagion del quale io mi trovo anche prigione;
\par 4 e che io lo faccia conoscere, parlandone come debbo.
\par 5 Conducetevi con saviezza verso quelli di fuori, approfittando delle opportunità.
\par 6 Il vostro parlare sia sempre con grazia, condito con sale, per sapere come dovete rispondere a ciascuno.
\par 7 Tutte le cose mie ve le farà sapere Tichico, il caro fratello e fedel ministro e mio compagno di servizio nel Signore.
\par 8 Ve l'ho mandato appunto per questo: affinché sappiate lo stato nostro ed egli consoli i vostri cuori;
\par 9 e con lui ho mandato il fedele e caro fratello Onesimo, che è dei vostri. Essi vi faranno sapere tutte le cose di qua.
\par 10 Vi salutano Aristarco, il mio compagno di prigione, e Marco, il cugino di Barnaba (intorno al quale avete ricevuto degli ordini; se viene da voi, accoglietelo), e Gesù, detto Giusto, i quali sono della circoncisione;
\par 11 e fra questi sono i soli miei collaboratori per il regno di Dio, che mi siano stati di conforto.
\par 12 Epafra, che è dei vostri e servo di Cristo Gesù, vi saluta. Egli lotta sempre per voi nelle sue preghiere affinché perfetti e pienamente accertati stiate fermi in tutta la volontà di Dio.
\par 13 Poiché io gli rendo questa testimonianza ch'egli si dà molta pena per voi e per quelli di Laodicea e per quelli di Jerapoli.
\par 14 Luca, il medico diletto, e Dema vi salutano.
\par 15 Salutate i fratelli che sono in Laodicea, e Ninfa e la chiesa che è in casa sua.
\par 16 E quando questa epistola sarà stata letta fra voi, fate che sia letta anche nella chiesa dei Laodicesi, e che anche voi leggiate quella che vi sarà mandata da Laodicea.
\par 17 E dite ad Archippo: Bada al ministerio che hai ricevuto nel Signore, per adempierlo.
\par 18 Il saluto è di mia propria mano, di me, Paolo. Ricordatevi delle mie catene. La grazia sia con voi.


\end{document}