\begin{document}

\title{1 Thessalonians}


\chapter{1}

\par 1 Paolo, Silvano e Timoteo alla chiesa dei Tessalonicesi che è in Dio Padre e nel Signor Gesù Cristo, grazia a voi e pace.
\par 2 Noi rendiamo del continuo grazie a Dio per voi tutti, facendo di voi menzione nelle nostre preghiere,
\par 3 ricordandoci del continuo nel cospetto del nostro Dio e Padre, dell'opera della vostra fede, delle fatiche del vostro amore e della costanza della vostra speranza nel nostro Signor Gesù Cristo;
\par 4 conoscendo, fratelli amati da Dio, la vostra elezione.
\par 5 Poiché il nostro Evangelo non vi è stato annunziato soltanto con parole, ma anche con potenza, con lo Spirito Santo e con gran pienezza di convinzione; e infatti voi sapete quel che siamo stati fra voi per amor vostro.
\par 6 E voi siete divenuti imitatori nostri e del Signore, avendo ricevuto la Parola in mezzo a molte afflizioni, con allegrezza dello Spirito Santo;
\par 7 talché siete diventati un esempio a tutti i credenti della Macedonia e dell'Acaia.
\par 8 Poiché da voi la parola del Signore ha echeggiato non soltanto nella Macedonia e nell'Acaia, ma la fama della fede che avete in Dio si è sparsa in ogni luogo; talché non abbiam bisogno di parlarne;
\par 9 perché eglino stessi raccontano di noi quale sia stata la nostra venuta tra voi, e come vi siete convertiti dagl'idoli a Dio per servire all'Iddio vivente e vero, e per aspettare dai cieli il suo Figliuolo,
\par 10 il quale Egli ha risuscitato dai morti: cioè, Gesù che ci libera dall'ira a venire.

\chapter{2}

\par 1 Voi stessi, fratelli, sapete che la nostra venuta tra voi non è stata invano;
\par 2 anzi, sebbene avessimo prima patito e fossimo stati oltraggiati, come sapete, a Filippi, pur ci siamo rinfrancati nell'Iddio nostro, per annunziarvi l'Evangelo di Dio in mezzo a molte lotte.
\par 3 Poiché la nostra esortazione non procede da impostura, né da motivi impuri, né è fatta con frode;
\par 4 ma siccome siamo stati approvati da Dio che ci ha stimati tali da poterci affidare l'Evangelo, parliamo in modo da piacere non agli uomini, ma a Dio che prova i nostri cuori.
\par 5 Difatti, non abbiamo mai usato un parlar lusinghevole, come ben sapete, né pretesti ispirati da cupidigia; Iddio ne è testimone.
\par 6 E non abbiam cercato gloria dagli uomini, né da voi, né da altri, quantunque, come apostoli di Cristo, avessimo potuto far valere la nostra autorità;
\par 7 invece, siamo stati mansueti in mezzo a voi, come una nutrice che cura teneramente i proprî figliuoli.
\par 8 Così, nel nostro grande affetto per voi, eravamo disposti a darvi non soltanto l'Evangelo di Dio, ma anche le nostre proprie vite, tanto ci eravate divenuti cari.
\par 9 Perché, fratelli, voi la ricordate la nostra fatica e la nostra pena; egli è lavorando notte e giorno per non essere d'aggravio ad alcuno di voi, che v'abbiam predicato l'Evangelo di Dio.
\par 10 Voi siete testimoni, e Dio lo è pure, del modo santo, giusto e irreprensibile con cui ci siamo comportati verso voi che credete;
\par 11 e sapete pure che, come fa un padre coi suoi figliuoli, noi abbiamo esortato,
\par 12 confortato e scongiurato ciascun di voi a condursi in modo degno di Dio, che vi chiama al suo regno e alla sua gloria.
\par 13 E per questa ragione anche noi rendiamo del continuo grazie a Dio: perché quando riceveste da noi la parola della predicazione, cioè la parola di Dio, voi l'accettaste non come parola d'uomini, ma, quale essa è veramente, come parola di Dio, la quale opera efficacemente in voi che credete.
\par 14 Poiché, fratelli, voi siete divenuti imitatori delle chiese di Dio che sono in Cristo Gesù nella Giudea; in quanto che anche voi avete sofferto dai vostri connazionali le stesse cose che quelle chiese hanno sofferto dai Giudei,
\par 15 i quali hanno ucciso e il Signor Gesù e i profeti, hanno cacciato noi, e non piacciono a Dio, e sono avversi a tutti gli uomini,
\par 16 divietandoci di parlare ai Gentili perché sieno salvati. Essi vengono così colmando senza posa la misura dei loro peccati; ma ormai li ha raggiunti l'ira finale.
\par 17 Quant'è a noi, fratelli, orbati di voi per breve tempo, di persona, non di cuore, abbiamo tanto maggiormente cercato, con gran desiderio, di veder la vostra faccia.
\par 18 Perché abbiam voluto, io Paolo almeno, non una ma due volte, venir a voi; ma Satana ce lo ha impedito.
\par 19 Qual è infatti la nostra speranza, o la nostra allegrezza, o la corona di cui ci gloriamo? Non siete forse voi, nel cospetto del nostro Signor Gesù quand'egli verrà?
\par 20 Sì, certo, la nostra gloria e la nostra allegrezza siete voi.

\chapter{3}

\par 1 Perciò, non potendo più reggere, stimammo bene di esser lasciati soli ad Atene;
\par 2 e mandammo Timoteo, nostro fratello e ministro di Dio nella propagazione del Vangelo di Cristo, per confermarvi e confortarvi nella vostra fede,
\par 3 affinché nessuno fosse scosso in mezzo a queste afflizioni; poiché voi stessi sapete che a questo siamo destinati.
\par 4 Perché anche quando eravamo fra voi, vi predicevamo che saremmo afflitti; come anche è avvenuto, e voi lo sapete.
\par 5 Perciò anch'io, non potendo più resistere, mandai ad informarmi della vostra fede, per tema che il tentatore vi avesse tentati, e la nostra fatica fosse riuscita vana.
\par 6 Ma ora che Timoteo è giunto qui da presso a voi e ci ha recato liete notizie della vostra fede e del vostro amore, e ci ha detto che serbate del continuo buona ricordanza di noi bramando di vederci, come anche noi bramiamo vedervi,
\par 7 per questa ragione, fratelli, siamo stati consolati a vostro riguardo, in mezzo a tutte le nostre distrette e afflizioni, mediante la vostra fede;
\par 8 perché ora viviamo, se voi state saldi nel Signore.
\par 9 Poiché quali grazie possiam noi rendere a Dio, a vostro riguardo, per tutta l'allegrezza della quale ci rallegriamo a cagion di voi nel cospetto dell'Iddio nostro,
\par 10 mentre notte e giorno preghiamo intensamente di poter vedere la vostra faccia e supplire alle lacune della vostra fede?
\par 11 Ora Iddio stesso, nostro Padre, e il Signor nostro Gesù ci appianino la via per venir da voi;
\par 12 e quant'è a voi, il Signore vi accresca e vi faccia abbondare in amore gli uni verso gli altri e verso tutti, come anche noi abbondiamo verso voi,
\par 13 per confermare i vostri cuori, onde siano irreprensibili in santità nel cospetto di Dio nostro Padre, quando il Signor nostro Gesù verrà con tutti i suoi santi.

\chapter{4}

\par 1 Del rimanente, fratelli, come avete imparato da noi il modo in cui vi dovete condurre e piacere a Dio (ed è così che già vi conducete), vi preghiamo e vi esortiamo nel Signor Gesù a vie più progredire.
\par 2 Poiché sapete quali comandamenti vi abbiamo dati per la grazia del Signor Gesù.
\par 3 Perché questa è la volontà di Dio: che vi santifichiate, che v'asteniate dalla fornicazione,
\par 4 che ciascun di voi sappia possedere il proprio corpo in santità ed onore,
\par 5 non dandosi a passioni di concupiscenza come fanno i pagani i quali non conoscono Iddio;
\par 6 e che nessuno soverchi il fratello né lo sfrutti negli affari; perché il Signore è un vendicatore in tutte queste cose, siccome anche v'abbiamo innanzi detto e protestato.
\par 7 Poiché Iddio ci ha chiamati non a impurità, ma a santificazione.
\par 8 Chi dunque sprezza questi precetti, non sprezza un uomo, ma quell'Iddio, il quale anche vi comunica il dono del suo Santo Spirito.
\par 9 Or quanto all'amor fraterno non avete bisogno che io ve ne scriva, giacché voi stessi siete stati ammaestrati da Dio ad amarvi gli uni gli altri;
\par 10 e invero voi lo fate verso tutti i fratelli che sono nell'intera Macedonia. Ma v'esortiamo, fratelli, che vie più abbondiate in questo, e vi studiate di vivere in quiete,
\par 11 di fare i fatti vostri e di lavorare con le vostre mani, come v'abbiamo ordinato di fare,
\par 12 onde camminiate onestamente verso quelli di fuori, e non abbiate bisogno di nessuno.
\par 13 Or, fratelli, non vogliamo che siate in ignoranza circa quelli che dormono, affinché non siate contristati come gli altri che non hanno speranza.
\par 14 Poiché, se crediamo che Gesù morì e risuscitò, così pure, quelli che si sono addormentati, Iddio, per mezzo di Gesù, li ricondurrà con esso lui.
\par 15 Poiché questo vi diciamo per parola del Signore: che noi viventi, i quali saremo rimasti fino alla venuta del Signore, non precederemo quelli che si sono addormentati;
\par 16 perché il Signore stesso, con potente grido, con voce d'arcangelo e con la tromba di Dio, scenderà dal cielo, e i morti in Cristo risusciteranno i primi;
\par 17 poi noi viventi, che saremo rimasti, verremo insieme con loro rapiti sulle nuvole, a incontrare il Signore nell'aria; e così saremo sempre col Signore.
\par 18 Consolatevi dunque gli uni gli altri con queste parole.

\chapter{5}

\par 1 Or quanto ai tempi ed ai momenti, fratelli, non avete bisogno che vi se ne scriva;
\par 2 perché voi stessi sapete molto bene che il giorno del Signore verrà come viene un ladro nella notte.
\par 3 Quando diranno: Pace e sicurezza, allora di subito una improvvisa ruina verrà loro addosso, come le doglie alla donna incinta; e non scamperanno affatto.
\par 4 Ma voi, fratelli, non siete nelle tenebre, sì che quel giorno abbia a cogliervi a guisa di ladro;
\par 5 poiché voi tutti siete figliuoli di luce e figliuoli del giorno; noi non siamo della notte né delle tenebre;
\par 6 non dormiamo dunque come gli altri; ma vegliamo e siamo sobrî.
\par 7 Poiché quelli che dormono, dormono di notte; e quelli che s'inebriano, s'inebriano di notte;
\par 8 ma noi, che siamo del giorno, siamo sobrî, avendo rivestito la corazza della fede e dell'amore, e preso per elmo la speranza della salvezza.
\par 9 Poiché Iddio non ci ha destinati ad ira, ma ad ottener salvezza per mezzo del Signor nostro Gesù Cristo,
\par 10 il quale è morto per noi affinché, sia che vegliamo sia che dormiamo, viviamo insieme con lui.
\par 11 Perciò, consolatevi gli uni gli altri, ed edificatevi l'un l'altro, come d'altronde già fate.
\par 12 Or, fratelli, vi preghiamo di avere in considerazione coloro che faticano fra voi, che vi son preposti nel Signore e vi ammoniscono,
\par 13 e di tenerli in grande stima ed amarli a motivo dell'opera loro. Vivete in pace fra voi.
\par 14 V'esortiamo, fratelli, ad ammonire i disordinati, a confortare gli scoraggiati, a sostenere i deboli, ad esser longanimi verso tutti.
\par 15 Guardate che nessuno renda ad alcuno male per male; anzi procacciate sempre il bene gli uni degli altri, e quello di tutti.
\par 16 Siate sempre allegri;
\par 17 non cessate mai di pregare;
\par 18 in ogni cosa rendete grazie, poiché tale è la volontà di Dio in Cristo Gesù verso di voi.
\par 19 Non spegnete lo Spirito;
\par 20 non disprezzate le profezie;
\par 21 ma esaminate ogni cosa e ritenete il bene;
\par 22 astenetevi da ogni specie di male.
\par 23 Or l'Iddio della pace vi santifichi Egli stesso completamente; e l'intero essere vostro, lo spirito, l'anima ed il corpo, sia conservato irreprensibile, per la venuta del Signor nostro Gesù Cristo.
\par 24 Fedele è Colui che vi chiama, ed Egli farà anche questo.
\par 25 Fratelli, pregate per noi.
\par 26 Salutate tutti i fratelli con un santo bacio.
\par 27 Io vi scongiuro per il Signore a far sì che questa epistola sia letta a tutti i fratelli.
\par 28 La grazia del Signor nostro Gesù Cristo sia con voi.


\end{document}