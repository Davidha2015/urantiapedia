\begin{document}

\title{II Timoteo}


\chapter{1}

\par 1 Paolo, apostolo di Cristo Gesù per volontà di Dio, secondo la promessa della vita che è in Cristo Gesù,
\par 2 a Timoteo, mio diletto figliuolo, grazia, misericordia, pace da Dio Padre e da Cristo Gesù nostro Signore.
\par 3 Io rendo grazie a Dio, il quale servo con pura coscienza, come l'han servito i miei antenati, ricordandomi sempre di te nelle mie preghiere giorno e notte,
\par 4 bramando, memore come sono delle tue lacrime, di vederti per esser ricolmo d'allegrezza.
\par 5 Io ricordo infatti la fede non finta che è in te, la quale abitò prima nella tua nonna Loide e nella tua madre Eunice, e, son persuaso, abita in te pure.
\par 6 Per questa ragione ti ricordo di ravvivare il dono di Dio che è in te per la imposizione delle mie mani.
\par 7 Poiché Iddio ci ha dato uno spirito non di timidità, ma di forza e d'amore e di correzione.
\par 8 Non aver dunque vergogna della testimonianza del Signor nostro, né di me che sono in catene per lui; ma soffri anche tu per l'Evangelo, sorretto dalla potenza di Dio;
\par 9 il quale ci ha salvati e ci ha rivolto una santa chiamata, non secondo le nostre opere, ma secondo il proprio proponimento e la grazia che ci è stata fatta in Cristo Gesù avanti i secoli,
\par 10 ma che è stata ora manifestata coll'apparizione del Salvator nostro Cristo Gesù, il quale ha distrutto la morte e ha prodotto in luce la vita e l'immortalità mediante l'Evangelo,
\par 11 in vista del quale io sono stato costituito banditore e apostolo e dottore.
\par 12 Ed è pure per questa cagione che soffro queste cose; ma non me ne vergogno, perché so in chi ho creduto, e son persuaso ch'egli è potente da custodire il mio deposito fino a quel giorno.
\par 13 Attienti con fede e con l'amore che è in Cristo Gesù al modello delle sane parole che udisti da me.
\par 14 Custodisci il buon deposito per mezzo dello Spirito Santo che abita in noi.
\par 15 Tu sai questo: che tutti quelli che sono in Asia mi hanno abbandonato; fra i quali, Figello ed Ermogene.
\par 16 Conceda il Signore misericordia alla famiglia d'Onesiforo, poiché egli m'ha spesse volte confortato e non si è vergognato della mia catena;
\par 17 anzi, quando è venuto a Roma, mi ha cercato premurosamente e m'ha trovato.
\par 18 Gli conceda il Signore di trovar misericordia presso il Signore in quel giorno; e quanti servigî egli abbia reso in Efeso tu sai molto bene.

\chapter{2}

\par 1 Tu dunque, figliuol mio, fortificati nella grazia che è in Cristo Gesù,
\par 2 e le cose che hai udite da me in presenza di molti testimoni, affidale ad uomini fedeli, i quali siano capaci d'insegnarle anche ad altri.
\par 3 Sopporta anche tu le sofferenze, come un buon soldato di Cristo Gesù.
\par 4 Uno che va alla guerra non s'impaccia delle faccende della vita; e ciò, affin di piacere a colui che l'ha arruolato.
\par 5 Parimente se uno lotta come atleta non è coronato, se non ha lottato secondo le leggi.
\par 6 Il lavoratore che fatica dev'essere il primo ad aver la sua parte de' frutti.
\par 7 Considera quello che dico, poiché il Signore ti darà intelligenza in ogni cosa.
\par 8 Ricordati di Gesù Cristo, risorto d'infra i morti, progenie di Davide, secondo il mio Vangelo;
\par 9 per il quale io soffro afflizione fino ad essere incatenato come un malfattore, ma la parola di Dio non è incatenata.
\par 10 Perciò io sopporto ogni cosa per amor degli eletti, affinché anch'essi conseguano la salvezza che è in Cristo Gesù con gloria eterna.
\par 11 Certa è questa parola: che se muoiamo con lui, con lui anche vivremo;
\par 12 se abbiam costanza nella prova, con lui altresì regneremo;
\par 13 se lo rinnegheremo, anch'egli ci rinnegherà; se siamo infedeli, egli rimane fedele, perché non può rinnegare se stesso.
\par 14 Ricorda loro queste cose, scongiurandoli nel cospetto di Dio che non faccian dispute di parole, che a nulla giovano e sovvertono chi le ascolta.
\par 15 Studiati di presentar te stesso approvato dinanzi a Dio: operaio che non abbia ad esser confuso, che tagli rettamente la parola della verità.
\par 16 Ma schiva le profane ciance, perché quelli che vi si danno progrediranno nella empietà
\par 17 e la loro parola andrà rodendo come fa la cancrena; fra i quali sono Imeneo e Fileto;
\par 18 uomini che si sono sviati dalla verità, dicendo che la risurrezione è già avvenuta, e sovvertono la fede di alcuni.
\par 19 Ma pure il solido fondamento di Dio rimane fermo, portando questo sigillo: 'Il Signore conosce quelli che son suoi', e: 'Ritraggasi dall'iniquità chiunque nomina il nome del Signore'.
\par 20 Or in una gran casa non ci son soltanto dei vasi d'oro e d'argento, ma anche dei vasi di legno e di terra; e gli uni son destinati a un uso nobile e gli altri ad un uso ignobile.
\par 21 Se dunque uno si serba puro da quelle cose, sarà un vaso nobile, santificato, atto al servigio del padrone, preparato per ogni opera buona.
\par 22 Ma fuggi gli appetiti giovanili e procaccia giustizia, fede, amore, pace con quelli che di cuor puro invocano il Signore.
\par 23 Ma schiva le quistioni stolte e scempie, sapendo che generano contese.
\par 24 Or il servitore del Signore non deve contendere, ma dev'essere mite inverso tutti, atto ad insegnare, paziente,
\par 25 correggendo con dolcezza quelli che contradicono, se mai avvenga che Dio conceda loro di ravvedersi per riconoscere la verità;
\par 26 in guisa che, tornati in sé, escano dal laccio del diavolo, che li avea presi prigionieri perché facessero la sua volontà.

\chapter{3}

\par 1 Or sappi questo, che negli ultimi giorni verranno dei tempi difficili;
\par 2 perché gli uomini saranno egoisti, amanti del danaro, vanagloriosi, superbi, bestemmiatori, disubbidienti ai genitori, ingrati, irreligiosi,
\par 3 senz'affezione naturale, mancatori di fede, calunniatori, intemperanti, spietati, senza amore per il bene,
\par 4 traditori, temerarî, gonfi, amanti del piacere anziché di Dio,
\par 5 aventi le forme della pietà, ma avendone rinnegata la potenza.
\par 6 Anche costoro schiva! Poiché del numero di costoro son quelli che s'insinuano nelle case e cattivano donnicciuole cariche di peccati, agitate da varie cupidigie,
\par 7 che imparan sempre e non possono mai pervenire alla conoscenza della verità.
\par 8 E come Jannè e Iambrè contrastarono a Mosè, così anche costoro contrastano alla verità: uomini corrotti di mente, riprovati quanto alla fede.
\par 9 Ma non andranno più oltre, perché la loro stoltezza sarà manifesta a tutti, come fu quella di quegli uomini.
\par 10 Quanto a te, tu hai tenuto dietro al mio insegnamento, alla mia condotta, a' miei propositi, alla mia fede, alla mia pazienza, al mio amore, alla mia costanza,
\par 11 alle mie persecuzioni, alle mie sofferenze, a quel che mi avvenne ad Antiochia, ad Iconio ed a Listra. Sai quali persecuzioni ho sopportato; e il Signore mi ha liberato da tutte.
\par 12 E d'altronde tutti quelli che voglion vivere piamente in Cristo Gesù saranno perseguitati;
\par 13 mentre i malvagi e gli impostori andranno di male in peggio, seducendo ed essendo sedotti.
\par 14 Ma tu persevera nelle cose che hai imparate e delle quali sei stato accertato, sapendo da chi le hai imparate,
\par 15 e che fin da fanciullo hai avuto conoscenza degli Scritti sacri, i quali possono renderti savio a salute mediante la fede che è in Cristo Gesù.
\par 16 Ogni Scrittura è ispirata da Dio e utile ad insegnare, a riprendere, a correggere, a educare alla giustizia,
\par 17 affinché l'uomo di Dio sia compiuto, appieno fornito per ogni opera buona.

\chapter{4}

\par 1 Io te ne scongiuro nel cospetto di Dio e di Cristo Gesù che ha da giudicare i vivi e i morti, e per la sua apparizione e per il suo regno:
\par 2 Predica la Parola, insisti a tempo e fuor di tempo, riprendi, sgrida, esorta con grande pazienza e sempre istruendo.
\par 3 Perché verrà il tempo che non sopporteranno la sana dottrina; ma per prurito d'udire si accumuleranno dottori secondo le loro proprie voglie
\par 4 e distoglieranno le orecchie dalla verità e si volgeranno alle favole.
\par 5 Ma tu sii vigilante in ogni cosa, soffri afflizioni, fa' l'opera d'evangelista, compi tutti i doveri del tuo ministerio.
\par 6 Quanto a me io sto per esser offerto a mo' di libazione, e il tempo della mia dipartenza è giunto.
\par 7 Io ho combattuto il buon combattimento, ho finito la corsa, ho serbata la fede;
\par 8 del rimanente mi è riservata la corona di giustizia che il Signore, il giusto giudice, mi assegnerà in quel giorno; e non solo a me, ma anche a tutti quelli che avranno amato la sua apparizione.
\par 9 Studiati di venir tosto da me;
\par 10 poiché Dema, avendo amato il presente secolo, mi ha lasciato e se n'è andato a Tessalonica. Crescente è andato in Galazia, Tito in Dalmazia. Luca solo è meco.
\par 11 Prendi Marco e menalo teco; poich'egli mi è molto utile per il ministerio.
\par 12 Quanto a Tichico l'ho mandato ad Efeso.
\par 13 Quando verrai porta il mantello che ho lasciato a Troas da Carpo, e i libri, specialmente le pergamene.
\par 14 Alessandro, il ramaio, mi ha fatto del male assai. Il Signore gli renderà secondo le sue opere.
\par 15 Da lui guardati anche tu, poiché egli ha fortemente contrastato alle nostre parole.
\par 16 Nella mia prima difesa nessuno s'è trovato al mio fianco, ma tutti mi hanno abbandonato; non sia loro imputato!
\par 17 Ma il Signore è stato meco e m'ha fortificato, affinché il Vangelo fosse per mezzo mio pienamente proclamato e tutti i Gentili l'udissero; e sono stato liberato dalla gola del leone.
\par 18 Il Signore mi libererà da ogni mala azione e mi salverà nel suo regno celeste. A lui sia la gloria ne' secoli dei secoli. Amen.
\par 19 Saluta Prisca ed Aquila e la famiglia d'Onesiforo.
\par 20 Erasto è rimasto a Corinto; e Trofimo l'ho lasciato infermo a Mileto.
\par 21 Studiati di venire prima dell'inverno. Ti salutano Eubulo e Pudente e Lino e Claudia ed i fratelli tutti.
\par 22 Il Signore sia col tuo spirito. La grazia sia con voi.


\end{document}