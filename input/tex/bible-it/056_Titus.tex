\begin{document}

\title{Titus}


\chapter{1}

\par 1 Paolo, servitor di Dio e apostolo di Gesù Cristo per la fede degli eletti di Dio e la conoscenza della verità che è secondo pietà,
\par 2 nella speranza della vita eterna la quale Iddio, che non può mentire, promise avanti i secoli,
\par 3 manifestando poi nei suoi proprî tempi la sua parola mediante la predicazione che è stata a me affidata per mandato di Dio, nostro Salvatore,
\par 4 a Tito, mio vero figliuolo secondo la fede che ci è comune, grazia e pace da Dio Padre e da Cristo Gesù, nostro Salvatore.
\par 5 Per questa ragione t'ho lasciato in Creta: perché tu dia ordine alle cose che rimangono a fare, e costituisca degli anziani per ogni città, come t'ho ordinato;
\par 6 quando si trovi chi sia irreprensibile, marito d'una sola moglie, avente figliuoli fedeli, che non siano accusati di dissolutezza né insubordinati.
\par 7 Poiché il vescovo bisogna che sia irreprensibile, come economo di Dio; non arrogante, non iracondo, non dedito al vino, non manesco, non cupido di disonesto guadagno,
\par 8 ma ospitale, amante del bene, assennato, giusto, santo, temperante,
\par 9 attaccato alla fedel Parola quale gli è stata insegnata, onde sia capace d'esortare nella sana dottrina e di convincere i contradittori.
\par 10 Poiché vi son molti ribelli, cianciatori e seduttori di menti, specialmente fra quelli della circoncisione, ai quali bisogna turar la bocca;
\par 11 uomini che sovvertono le case intere, insegnando cose che non dovrebbero, per amor di disonesto guadagno.
\par 12 Uno dei loro, un loro proprio profeta, disse: 'I Cretesi son sempre bugiardi, male bestie, ventri pigri'.
\par 13 Questa testimonianza è verace. Riprendili perciò severamente, affinché siano sani nella fede,
\par 14 non dando retta a favole giudaiche né a comandamenti d'uomini che voltan le spalle alla verità.
\par 15 Tutto è puro per quelli che son puri; ma per i contaminati ed increduli niente è puro; anzi, tanto la mente che la coscienza loro son contaminate.
\par 16 Fanno professione di conoscere Iddio; ma lo rinnegano con le loro opere, essendo abominevoli, e ribelli, e incapaci di qualsiasi opera buona.

\chapter{2}

\par 1 Ma tu esponi le cose che si convengono alla sana dottrina:
\par 2 Che i vecchi siano sobrî, gravi, assennati, sani nella fede, nell'amore, nella pazienza:
\par 3 che le donne attempate abbiano parimente un portamento convenevole a santità, non siano maldicenti né dedite a molto vino, siano maestre di ciò che è buono;
\par 4 onde insegnino alle giovani ad amare i mariti, ad amare i figliuoli,
\par 5 ad esser assennate, caste, date ai lavori domestici, buone, soggette ai loro mariti, affinché la Parola di Dio non sia bestemmiata.
\par 6 Esorta parimente i giovani ad essere assennati,
\par 7 dando te stesso in ogni cosa come esempio di opere buone; mostrando nell'insegnamento purità incorrotta, gravità,
\par 8 parlar sano, irreprensibile, onde l'avversario resti confuso, non avendo nulla di male da dire di noi.
\par 9 Esorta i servi ad esser sottomessi ai loro padroni, a compiacerli in ogni cosa, a non contradirli,
\par 10 a non frodarli, ma a mostrar sempre lealtà perfetta, onde onorino la dottrina di Dio, nostro Salvatore, in ogni cosa.
\par 11 Poiché la grazia di Dio, salutare per tutti gli uomini, è apparsa
\par 12 e ci ammaestra a rinunziare all'empietà e alle mondane concupiscenze, per vivere in questo mondo temperatamente, giustamente e piamente,
\par 13 aspettando la beata speranza e l'apparizione della gloria del nostro grande Iddio e Salvatore, Cristo Gesù;
\par 14 il quale ha dato se stesso per noi affin di riscattarci da ogni iniquità e di purificarsi un popolo suo proprio, zelante nelle opere buone.
\par 15 Insegna queste cose, ed esorta e riprendi con ogni autorità. Niuno ti sprezzi.

\chapter{3}

\par 1 Ricorda loro che stiano soggetti ai magistrati e alle autorità, che siano ubbidienti, pronti a fare ogni opera buona,
\par 2 che non dicano male d'alcuno, che non siano contenziosi, che siano benigni, mostrando ogni mansuetudine verso tutti gli uomini.
\par 3 Perché anche noi eravamo una volta insensati, ribelli, traviati, servi di varie concupiscenze e voluttà, menanti la vita in malizia ed invidia, odiosi e odiantici gli uni gli altri.
\par 4 Ma quando la benignità di Dio, nostro Salvatore, e il suo amore verso gli uomini sono stati manifestati,
\par 5 Egli ci ha salvati non per opere giuste che noi avessimo fatte, ma secondo la sua misericordia, mediante il lavacro della rigenerazione e il rinnovamento dello Spirito Santo,
\par 6 ch'Egli ha copiosamente sparso su noi per mezzo di Gesù Cristo, nostro Salvatore,
\par 7 affinché, giustificati per la sua grazia, noi fossimo fatti eredi secondo la speranza della vita eterna.
\par 8 Certa è questa parola, e queste cose voglio che tu affermi con forza, affinché quelli che han creduto a Dio abbiano cura d'attendere a buone opere. Queste cose sono buone ed utili agli uomini.
\par 9 Ma quanto alle quistioni stolte, alle genealogie, alle contese, e alle dispute intorno alla legge, stattene lontano, perché sono inutili e vane.
\par 10 L'uomo settario, dopo una prima e una seconda ammonizione, schivalo,
\par 11 sapendo che un tal uomo è pervertito e pecca, condannandosi da sé.
\par 12 Quando t'avrò mandato Artemas o Tichico, studiati di venir da me a Nicopoli, perché ho deciso di passar quivi l'inverno.
\par 13 Provvedi con cura al viaggio di Zena, il legista, e d'Apollo, affinché nulla manchi loro.
\par 14 Ed imparino anche i nostri ad attendere a buone opere per provvedere alle necessità, onde non stiano senza portar frutto.
\par 15 Tutti quelli che son meco ti salutano. Saluta quelli che ci amano in fede. La grazia sia con tutti voi!


\end{document}