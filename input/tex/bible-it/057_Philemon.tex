\begin{document}

\title{Filemone}


\chapter{1}

\par 1 Paolo, prigione di Cristo Gesù, e il fratello Timoteo, a Filemone, nostro diletto e compagno d'opera,
\par 2 e alla sorella Apfia, e ad Archippo, nostro compagno d'armi, e alla chiesa che è in casa tua,
\par 3 grazia a voi e pace da Dio nostro Padre e dal Signor Gesù Cristo.
\par 4 Io rendo sempre grazie all'Iddio mio, facendo menzione di te nelle mie preghiere,
\par 5 giacché odo parlare dell'amore e della fede che hai nel Signor Gesù e verso tutti i santi,
\par 6 e domando che la nostra comunione di fede sia efficace nel farti riconoscere ogni bene che si compia in noi, alla gloria di Cristo.
\par 7 Poiché ho provato una grande allegrezza e consolazione pel tuo amore, perché il cuore dei santi è stato ricreato per mezzo tuo, o fratello.
\par 8 Perciò, benché io abbia molta libertà in Cristo di comandarti quel che convien fare,
\par 9 preferisco fare appello alla tua carità, semplicemente come Paolo, vecchio, e adesso anche prigione di Cristo Gesù;
\par 10 ti prego per il mio figliuolo che ho generato nelle mie catene,
\par 11 per Onesimo che altra volta ti fu disutile, ma che ora è utile a te ed a me.
\par 12 Io te l'ho rimandato, lui, ch'è quanto dire, le viscere mie.
\par 13 Avrei voluto tenerlo presso di me, affinché in vece tua mi servisse nelle catene che porto a motivo del Vangelo;
\par 14 ma, senza il tuo parere, non ho voluto far nulla, affinché il tuo beneficio non fosse come forzato, ma volontario.
\par 15 Infatti, per questo, forse, egli è stato per breve tempo separato da te, perché tu lo ricuperassi per sempre;
\par 16 non più come uno schiavo, ma come da più di uno schiavo, come un fratello caro specialmente a me, ma ora quanto più a te, e nella carne e nel Signore!
\par 17 Se dunque tu mi tieni per un consocio, ricevilo come faresti di me.
\par 18 Che se t'ha fatto alcun torto o ti deve qualcosa, addebitalo a me.
\par 19 Io, Paolo, lo scrivo di mio proprio pugno: Io lo pagherò; per non dirti che tu mi sei debitore perfino di te stesso.
\par 20 Sì, fratello, io vorrei da te un qualche utile nel Signore; deh, ricrea il mio cuore in Cristo.
\par 21 Ti scrivo confidando nella tua ubbidienza, sapendo che tu farai anche al di là di quel che dico.
\par 22 Preparami al tempo stesso un alloggio, perché spero che, per le vostre preghiere, io vi sarò donato.
\par 23 Epafra, mio compagno di prigione in Cristo Gesù, ti saluta.
\par 24 Così fanno Marco, Aristarco, Dema, Luca, miei compagni d'opera.
\par 25 La grazia del Signor Gesù Cristo sia con lo spirito vostro.


\end{document}