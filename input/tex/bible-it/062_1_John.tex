\begin{document}

\title{1 John}


\chapter{1}

\par 1 Quel che era dal principio, quel che abbiamo udito, quel che abbiam veduto con gli occhi nostri, quel che abbiamo contemplato e che le nostre mani hanno toccato della Parola della vita
\par 2 (e la vita è stata manifestata e noi l'abbiam veduta e ne rendiamo testimonianza, e vi annunziamo la vita eterna che era presso il Padre e che ci fu manifestata),
\par 3 quello, dico, che abbiam veduto e udito, noi l'annunziamo anche a voi, affinché voi pure abbiate comunione con noi, e la nostra comunione è col Padre e col suo Figliuolo, Gesù Cristo.
\par 4 E noi vi scriviamo queste cose affinché la nostra allegrezza sia compiuta.
\par 5 Or questo è il messaggio che abbiamo udito da lui e che vi annunziamo: che Dio è luce; e che in Lui non vi son tenebre alcune.
\par 6 Se diciamo che abbiam comunione con lui e camminiamo nelle tenebre, noi mentiamo e non mettiamo in pratica la verità;
\par 7 ma se camminiamo nella luce, com'Egli è nella luce, abbiam comunione l'uno con l'altro, e il sangue di Gesù, suo Figliuolo, ci purifica da ogni peccato.
\par 8 Se diciamo d'esser senza peccato, inganniamo noi stessi, e la verità non è in noi.
\par 9 Se confessiamo i nostri peccati, Egli è fedele e giusto da rimetterci i peccati e purificarci da ogni iniquità.
\par 10 Se diciamo di non aver peccato, lo facciamo bugiardo, e la sua parola non è in noi.

\chapter{2}

\par 1 Figliuoletti miei, io vi scrivo queste cose affinché non pecchiate; e se alcuno ha peccato, noi abbiamo un avvocato presso il Padre, cioè Gesù Cristo, il giusto;
\par 2 ed egli è la propiziazione per i nostri peccati; e non soltanto per i nostri, ma anche per quelli di tutto il mondo.
\par 3 E da questo sappiamo che l'abbiam conosciuto: se osserviamo i suoi comandamenti.
\par 4 Chi dice: Io l'ho conosciuto e non osserva i suoi comandamenti, è bugiardo, e la verità non è in lui;
\par 5 ma chi osserva la sua parola, l'amor di Dio è in lui veramente compiuto.
\par 6 Da questo conosciamo che siamo in lui: chi dice di dimorare in lui, deve, nel modo ch'egli camminò, camminare anch'esso.
\par 7 Diletti, non è un nuovo comandamento ch'io vi scrivo, ma un comandamento vecchio, che aveste dal principio: il comandamento vecchio è la Parola che avete udita.
\par 8 E però è un comandamento nuovo ch'io vi scrivo; il che è vero in lui ed in voi; perché le tenebre stanno passando, e la vera luce già risplende.
\par 9 Chi dice d'esser nella luce e odia il suo fratello, è tuttora nelle tenebre.
\par 10 Chi ama il suo fratello dimora nella luce e non v'è in lui nulla che lo faccia inciampare.
\par 11 Ma chi odia il suo fratello è nelle tenebre e cammina nelle tenebre e non sa ov'egli vada, perché le tenebre gli hanno accecato gli occhi.
\par 12 Figliuoletti, io vi scrivo perché i vostri peccati vi sono rimessi per il suo nome.
\par 13 Padri, vi scrivo perché avete conosciuto Colui che è dal principio. Giovani, vi scrivo perché avete vinto il maligno.
\par 14 Figliuoletti, v'ho scritto perché avete conosciuto il Padre. Padri, v'ho scritto perché avete conosciuto Colui che è dal principio. Giovani, v'ho scritto perché siete forti, e la parola di Dio dimora in voi, e avete vinto il maligno.
\par 15 Non amate il mondo né le cose che sono nel mondo. Se uno ama il mondo, l'amor del Padre non è in lui.
\par 16 Poiché tutto quello che è nel mondo: la concupiscenza della carne, la concupiscenza degli occhi e la superbia della vita non è dal Padre, ma è dal mondo.
\par 17 E il mondo passa via con la sua concupiscenza; ma chi fa la volontà di Dio dimora in eterno.
\par 18 Figliuoletti, è l'ultima ora; e come avete udito che l'anticristo deve venire, fin da ora sono sorti molti anticristi; onde conosciamo che è l'ultima ora.
\par 19 Sono usciti di fra noi, ma non eran de' nostri; perché, se fossero stati de' nostri, sarebbero rimasti con noi; ma sono usciti affinché fossero manifestati e si vedesse che non tutti sono dei nostri.
\par 20 Quanto a voi, avete l'unzione dal Santo, e conoscete ogni cosa.
\par 21 Io vi ho scritto non perché non conoscete la verità, ma perché la conoscete, e perché tutto quel ch'è menzogna non ha che fare colla verità.
\par 22 Chi è il mendace se non colui che nega che Gesù è il Cristo? Esso è l'anticristo, che nega il Padre e il Figliuolo.
\par 23 Chiunque nega il Figliuolo, non ha neppure il Padre; chi confessa il Figliuolo ha anche il Padre.
\par 24 Quant'è a voi, dimori in voi quel che avete udito dal principio. Se quel che avete udito dal principio dimora in voi, anche voi dimorerete nel Figliuolo e nel Padre.
\par 25 E questa è la promessa ch'egli ci ha fatta: cioè la vita eterna.
\par 26 Vi ho scritto queste cose intorno a quelli che cercano di sedurvi.
\par 27 Ma quant'è a voi, l'unzione che avete ricevuta da lui dimora in voi, e non avete bisogno che alcuno v'insegni; ma siccome l'unzione sua v'insegna ogni cosa, ed è verace, e non è menzogna, dimorate in lui come essa vi ha insegnato.
\par 28 Ed ora, figliuoletti, dimorate in lui, affinché, quando egli apparirà, abbiam confidanza e alla sua venuta non abbiam da ritrarci da lui, coperti di vergogna.
\par 29 Se sapete che egli è giusto, sappiate che anche tutti quelli che praticano la giustizia son nati da lui.

\chapter{3}

\par 1 Vedete di quale amore ci è stato largo il Padre, dandoci d'esser chiamati figliuoli di Dio! E tali siamo. Per questo non ci conosce il mondo: perché non ha conosciuto lui.
\par 2 Diletti, ora siam figliuoli di Dio, e non è ancora reso manifesto quel che saremo. Sappiamo che quand'egli sarà manifestato saremo simili a lui, perché lo vedremo com'egli è.
\par 3 E chiunque ha questa speranza in lui, si purifica com'esso è puro.
\par 4 Chi fa il peccato commette una violazione della legge; e il peccato è la violazione della legge.
\par 5 E voi sapete ch'egli è stato manifestato per togliere i peccati; e in lui non c'è peccato.
\par 6 Chiunque dimora in lui non pecca; chiunque pecca non l'ha veduto, né l'ha conosciuto.
\par 7 Figliuoletti, nessuno vi seduca. Chi opera la giustizia è giusto, come egli è giusto.
\par 8 Chi commette il peccato è dal diavolo, perché il diavolo pecca dal principio. Per questo il Figliuol di Dio è stato manifestato: per distruggere le opere del diavolo.
\par 9 Chiunque è nato da Dio non commette peccato, perché il seme d'Esso dimora in lui; e non può peccare perché è nato da Dio.
\par 10 Da questo sono manifesti i figliuoli di Dio e i figliuoli del diavolo: chiunque non opera la giustizia non è da Dio; e così pure chi non ama il suo fratello.
\par 11 Poiché questo è il messaggio che avete udito dal principio:
\par 12 che ci amiamo gli uni gli altri, e non facciamo come Caino, che era dal maligno, e uccise il suo fratello. E perché l'uccise? Perché le sue opere erano malvage, e quelle del suo fratello erano giuste.
\par 13 Non vi maravigliate, fratelli, se il mondo vi odia.
\par 14 Noi sappiamo che siam passati dalla morte alla vita, perché amiamo i fratelli. Chi non ama rimane nella morte.
\par 15 Chiunque odia il suo fratello è omicida; e voi sapete che nessun omicida ha la vita eterna dimorante in se stesso.
\par 16 Noi abbiamo conosciuto l'amore da questo: che Egli ha data la sua vita per noi; noi pure dobbiam dare la nostra vita per i fratelli.
\par 17 Ma se uno ha dei beni di questo mondo, e vede il suo fratello nel bisogno, e gli chiude le proprie viscere, come dimora l'amor di Dio in lui?
\par 18 Figliuoletti, non amiamo a parole e con la lingua, ma a fatti e in verità.
\par 19 Da questo conosceremo che siam della verità e renderem sicuri i nostri cuori dinanzi a Lui.
\par 20 Poiché se il cuor nostro ci condanna, Dio è più grande del cuor nostro, e conosce ogni cosa.
\par 21 Diletti, se il cuor nostro non ci condanna, noi abbiam confidanza dinanzi a Dio;
\par 22 e qualunque cosa chiediamo la riceviamo da Lui, perché osserviamo i suoi comandamenti e facciam le cose che gli son grate.
\par 23 E questo è il suo comandamento: che crediamo nel nome del suo Figliuolo Gesù Cristo, e ci amiamo gli uni gli altri, com'Egli ce ne ha dato il comandamento.
\par 24 E chi osserva i suoi comandamenti dimora in Lui, ed Egli in esso. E da questo conosciamo ch'Egli dimora in noi: dallo Spirito ch'Egli ci ha dato.

\chapter{4}

\par 1 Diletti, non crediate ad ogni spirito, ma provate gli spiriti per sapere se son da Dio; perché molti falsi profeti sono usciti fuori nel mondo.
\par 2 Da questo conoscete lo Spirito di Dio: ogni spirito che confessa Gesù Cristo venuto in carne, è da Dio;
\par 3 e ogni spirito che non confessa Gesù, non è da Dio, e quello è lo spirito dell'anticristo, del quale avete udito che deve venire; ed ora è già nel mondo.
\par 4 Voi siete da Dio, figliuoletti, e li avete vinti; perché Colui che è in voi è più grande di colui che è nel mondo.
\par 5 Costoro sono del mondo; perciò parlano come chi è del mondo, e il mondo li ascolta.
\par 6 Noi siamo da Dio; chi conosce Iddio ci ascolta; chi non è da Dio non ci ascolta. Da questo conosciamo lo spirito della verità e lo spirito dell'errore.
\par 7 Diletti, amiamoci gli uni gli altri; perché l'amore è da Dio, e chiunque ama è nato da Dio e conosce Iddio.
\par 8 Chi non ama non ha conosciuto Iddio; perché Dio è amore.
\par 9 In questo s'è manifestato per noi l'amor di Dio: che Dio ha mandato il suo unigenito Figliuolo nel mondo, affinché, per mezzo di lui, vivessimo.
\par 10 In questo è l'amore: non che noi abbiamo amato Iddio, ma che Egli ha amato noi, e ha mandato il suo Figliuolo per essere la propiziazione per i nostri peccati.
\par 11 Diletti, se Dio ci ha così amati, anche noi dobbiamo amarci gli uni gli altri.
\par 12 Nessuno vide giammai Iddio; se ci amiamo gli uni gli altri, Iddio dimora in noi, e l'amor di Lui diventa perfetto in noi.
\par 13 Da questo conosciamo che dimoriamo in lui ed Egli in noi: ch'Egli ci ha dato del suo Spirito.
\par 14 E noi abbiamo veduto e testimoniamo che il Padre ha mandato il Figliuolo per essere il Salvatore del mondo.
\par 15 Chi confessa che Gesù è il Figliuol di Dio, Iddio dimora in lui, ed egli in Dio.
\par 16 E noi abbiam conosciuto l'amore che Dio ha per noi, e vi abbiam creduto. Dio è amore; e chi dimora nell'amore dimora in Dio, e Dio dimora in lui.
\par 17 In questo l'amore è reso perfetto in noi, affinché abbiamo confidanza nel giorno del giudizio: che quale egli è, tali siamo anche noi in questo mondo.
\par 18 Nell'amore non c'è paura; anzi, l'amor perfetto caccia via la paura; perché la paura implica apprensione di castigo; e chi ha paura non è perfetto nell'amore.
\par 19 Noi amiamo perché Egli ci ha amati il primo.
\par 20 Se uno dice: Io amo Dio, e odia il suo fratello, è bugiardo; perché chi non ama il suo fratello che ha veduto, non può amar Dio che non ha veduto.
\par 21 E questo è il comandamento che abbiam da lui: che chi ama Dio ami anche il suo fratello.

\chapter{5}

\par 1 Chiunque crede che Gesù è il Cristo, è nato da Dio; e chiunque ama Colui che ha generato, ama anche chi è stato da lui generato.
\par 2 Da questo conosciamo che amiamo i figliuoli di Dio: quando amiamo Dio e osserviamo i suoi comandamenti.
\par 3 Perché questo è l'amor di Dio: che osserviamo i suoi comandamenti; e i suoi comandamenti non sono gravosi.
\par 4 Poiché tutto quello che è nato da Dio vince il mondo; e questa è la vittoria che ha vinto il mondo: la nostra fede.
\par 5 Chi è colui che vince il mondo, se non colui che crede che Gesù è il Figliuol di Dio?
\par 6 Questi è colui che è venuto con acqua e con sangue, cioè, Gesù Cristo; non con l'acqua soltanto, ma con l'acqua e col sangue. Ed è lo Spirito che ne rende testimonianza, perché lo Spirito è la verità.
\par 7 Poiché tre son quelli che rendon testimonianza:
\par 8 lo Spirito, l'acqua ed il sangue, e i tre sono concordi.
\par 9 Se accettiamo la testimonianza degli uomini, maggiore è la testimonianza di Dio; e la testimonianza di Dio è quella ch'Egli ha resa circa il suo Figliuolo.
\par 10 Chi crede nel Figliuol di Dio ha quella testimonianza in sé; chi non crede a Dio l'ha fatto bugiardo, perché non ha creduto alla testimonianza che Dio ha resa circa il proprio Figliuolo.
\par 11 E la testimonianza è questa: Iddio ci ha data la vita eterna, e questa vita è nel suo Figliuolo.
\par 12 Chi ha il Figliuolo ha la vita; chi non ha il Figliuolo di Dio, non ha la vita.
\par 13 Io v'ho scritto queste cose affinché sappiate che avete la vita eterna, voi che credete nel nome del Figliuol di Dio.
\par 14 E questa è la confidanza che abbiamo in lui: che se domandiamo qualcosa secondo la sua volontà, Egli ci esaudisce;
\par 15 e se sappiamo ch'Egli ci esaudisce in quel che gli chiediamo, noi sappiamo di aver le cose che gli abbiamo domandate.
\par 16 Se uno vede il suo fratello commettere un peccato che non mena a morte, pregherà, e Dio gli darà la vita: a quelli, cioè, che commettono peccato che non meni a morte. V'è un peccato che mena a morte; non è per quello che dico di pregare.
\par 17 Ogni iniquità è peccato; e v'è un peccato che non mena a morte.
\par 18 Noi sappiamo che chiunque è nato da Dio non pecca; ma colui che nacque da Dio lo preserva, e il maligno non lo tocca.
\par 19 Noi sappiamo che siam da Dio, e che tutto il mondo giace nel maligno;
\par 20 ma sappiamo che il Figliuol di Dio è venuto e ci ha dato intendimento per conoscere Colui che è il vero; e noi siamo in Colui che è il vero Dio, nel suo Figliuolo Gesù Cristo. Quello è il vero Dio e la vita eterna.
\par 21 Figliuoletti, guardatevi dagl'idoli.


\end{document}