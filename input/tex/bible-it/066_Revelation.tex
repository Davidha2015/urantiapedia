\begin{document}

\title{Apocalisse}


\chapter{1}

\par 1 La rivelazione di Gesù Cristo, che Dio gli ha data per mostrare ai suoi servitori le cose che debbono avvenire in breve; ed egli l'ha fatta conoscere mandandola per mezzo del suo angelo al suo servitore Giovanni,
\par 2 il quale ha attestato la parola di Dio e la testimonianza di Gesù Cristo, tutto ciò ch'egli ha veduto.
\par 3 Beato chi legge e beati coloro che ascoltano le parole di questa profezia e serbano le cose che sono scritte in essa, poiché il tempo è vicino!
\par 4 Giovanni alle sette chiese che sono nell'Asia: Grazia a voi e pace da Colui che è, che era e che viene, e dai sette Spiriti che son davanti al suo trono,
\par 5 e da Gesù Cristo, il fedel testimone, il primogenito dei morti e il principe dei re della terra. A lui che ci ama, e ci ha liberati dai nostri peccati col suo sangue,
\par 6 e ci ha fatti essere un regno e sacerdoti all'Iddio e Padre suo, a lui siano la gloria e l'imperio nei secoli dei secoli. Amen.
\par 7 Ecco, egli viene colle nuvole; ed ogni occhio lo vedrà; lo vedranno anche quelli che lo trafissero, e tutte le tribù della terra faranno cordoglio per lui. Sì, Amen.
\par 8 Io son l'Alfa e l'Omega, dice il Signore Iddio che è, che era e che viene, l'Onnipotente.
\par 9 Io, Giovanni, vostro fratello e partecipe con voi della tribolazione, del regno e della costanza in Gesù, ero nell'isola chiamata Patmo a motivo della parola di Dio e della testimonianza di Gesù.
\par 10 Fui rapito in Ispirito nel giorno di Domenica, e udii dietro a me una gran voce, come d'una tromba, che diceva:
\par 11 Quel che tu vedi, scrivilo in un libro e mandalo alle sette chiese: a Efeso, a Smirne, a Pergamo, a Tiatiri, a Sardi, a Filadelfia e a Laodicea.
\par 12 E io mi voltai per veder la voce che mi parlava; e come mi fui voltato, vidi sette candelabri d'oro;
\par 13 e in mezzo ai candelabri Uno somigliante a un figliuol d'uomo, vestito d'una veste lunga fino ai piedi, e cinto d'una cintura d'oro all'altezza del petto.
\par 14 E il suo capo e i suoi capelli erano bianchi come candida lana, come neve; e i suoi occhi erano come una fiamma di fuoco;
\par 15 e i suoi piedi eran simili a terso rame, arroventato in una fornace; e la sua voce era come la voce di molte acque.
\par 16 Ed egli teneva nella sua man destra sette stelle; e dalla sua bocca usciva una spada a due tagli, acuta, e il suo volto era come il sole quando splende nella sua forza.
\par 17 E quando l'ebbi veduto, caddi ai suoi piedi come morto; ed egli mise la sua man destra su di me, dicendo: Non temere;
\par 18 io sono il primo e l'ultimo, e il Vivente; e fui morto, ma ecco son vivente per i secoli dei secoli, e tengo le chiavi della morte e dell'Ades.
\par 19 Scrivi dunque le cose che hai vedute, quelle che sono e quelle che devono avvenire in appresso,
\par 20 il mistero delle sette stelle che hai vedute nella mia destra, e dei sette candelabri d'oro. Le sette stelle sono gli angeli delle sette chiese, e i sette candelabri sono le sette chiese.

\chapter{2}

\par 1 All'angelo della chiesa d'Efeso scrivi: Queste cose dice colui che tiene le sette stelle nella sua destra, e che cammina in mezzo ai sette candelabri d'oro:
\par 2 Io conosco le tue opere e la tua fatica e la tua costanza e che non puoi sopportare i malvagi e hai messo alla prova quelli che si chiamano apostoli e non lo sono, e li hai trovati mendaci;
\par 3 e hai costanza e hai sopportato molte cose per amor del mio nome, e non ti sei stancato.
\par 4 Ma ho questo contro di te: che hai lasciato il tuo primo amore.
\par 5 Ricordati dunque donde sei caduto, e ravvediti, e fa' le opere di prima; se no, verrò a te, e rimoverò il tuo candelabro dal suo posto, se tu non ti ravvedi.
\par 6 Ma tu hai questo: che odii le opere dei Nicolaiti, le quali odio anch'io.
\par 7 Chi ha orecchio ascolti ciò che lo Spirito dice alle chiese. A chi vince io darò a mangiare dell'albero della vita, che sta nel paradiso di Dio.
\par 8 E all'angelo della chiesa di Smirne scrivi: Queste cose dice il primo e l'ultimo, che fu morto e tornò in vita:
\par 9 Io conosco la tua tribolazione e la tua povertà (ma pur sei ricco) e le calunnie lanciate da quelli che dicono d'esser Giudei e non lo sono, ma sono una sinagoga di Satana.
\par 10 Non temere quel che avrai da soffrire; ecco, il diavolo sta per cacciare alcuni di voi in prigione, perché siate provati: e avrete una tribolazione di dieci giorni. Sii fedele fino alla morte, e io ti darò la corona della vita.
\par 11 Chi ha orecchio ascolti ciò che lo Spirito dice alle chiese. Chi vince non sarà punto offeso dalla morte seconda.
\par 12 E all'angelo della chiesa di Pergamo scrivi: Queste cose dice colui che ha la spada acuta a due tagli:
\par 13 Io conosco dove tu abiti, cioè là dov'è il trono di Satana; eppur tu ritieni fermamente il mio nome, e non rinnegasti la mia fede, neppur nei giorni in cui Antipa, il mio fedel testimone, fu ucciso fra voi, dove abita Satana.
\par 14 Ma ho alcune poche cose contro di te: cioè, che tu hai quivi di quelli che professano la dottrina di Balaam, il quale insegnava a Balac a porre un intoppo davanti ai figliuoli d'Israele, inducendoli a mangiare delle cose sacrificate agli idoli e a fornicare.
\par 15 Così hai anche tu di quelli che in simil guisa professano la dottrina dei Nicolaiti.
\par 16 Ravvediti dunque; se no, verrò tosto a te, e combatterò contro a loro con la spada della mia bocca.
\par 17 Chi ha orecchio ascolti ciò che lo Spirito dice alle chiese. A chi vince io darò della manna nascosta, e gli darò una pietruzza bianca, e sulla pietruzza scritto un nome nuovo che nessuno conosce, se non colui che lo riceve.
\par 18 E all'angelo della chiesa di Tiatiri scrivi: Queste cose dice il Figliuol di Dio, che ha gli occhi come fiamma di fuoco, e i cui piedi son come terso rame:
\par 19 Io conosco le tue opere e il tuo amore e la tua fede e il tuo ministerio e la tua costanza, e che le tue opere ultime sono più abbondanti delle prime.
\par 20 Ma ho questo contro a te: che tu tolleri quella donna Jezabel, che si dice profetessa e insegna e seduce i miei servitori perché commettano fornicazione e mangino cose sacrificate agl'idoli.
\par 21 E io le ho dato tempo per ravvedersi, ed ella non vuol ravvedersi della sua fornicazione.
\par 22 Ecco, io getto lei sopra un letto di dolore, e quelli che commettono adulterio con lei in una gran tribolazione, se non si ravvedono delle opere d'essa.
\par 23 E metterò a morte i suoi figliuoli; e tutte le chiese conosceranno che io son colui che investigo le reni ed i cuori; e darò a ciascun di voi secondo le opere vostre.
\par 24 Ma agli altri di voi in Tiatiri che non professate questa dottrina e non avete conosciuto le profondità di Satana (come le chiaman loro), io dico: Io non v'impongo altro peso.
\par 25 Soltanto, quel che avete tenetelo fermamente finché io venga.
\par 26 E a chi vince e persevera nelle mie opere sino alla fine io darò potestà sulle nazioni,
\par 27 ed egli le reggerà con una verga di ferro frantumandole a mo' di vasi d'argilla; come anch'io ho ricevuto potestà dal Padre mio.
\par 28 E gli darò la stella mattutina.
\par 29 Chi ha orecchio ascolti ciò che lo Spirito dice alle chiese.

\chapter{3}

\par 1 E all'angelo della chiesa di Sardi scrivi: Queste cose dice colui che ha i sette Spiriti di Dio e le sette stelle: Io conosco le tue opere: tu hai nome di vivere e sei morto.
\par 2 Sii vigilante e rafferma il resto che sta per morire; poiché non ho trovato le opere tue compiute nel cospetto del mio Dio.
\par 3 Ricordati dunque di quanto hai ricevuto e udito; e serbalo, e ravvediti. Che se tu non vegli, io verrò come un ladro, e tu non saprai a quale ora verrò su di te.
\par 4 Ma tu hai alcuni pochi in Sardi che non hanno contaminato le loro vesti; essi cammineranno meco in vesti bianche, perché ne son degni.
\par 5 Chi vince sarà così vestito di vesti bianche, ed io non cancellerò il suo nome dal libro della vita, e confesserò il suo nome nel cospetto del Padre mio e nel cospetto dei suoi angeli.
\par 6 Chi ha orecchio ascolti ciò che lo Spirito dice alle chiese.
\par 7 E all'angelo della chiesa di Filadelfia scrivi: Queste cose dice il santo, il verace, colui che ha la chiave di Davide, colui che apre e nessuno chiude, colui che chiude e nessuno apre:
\par 8 Io conosco le tue opere. Ecco, io ti ho posta dinanzi una porta aperta, che nessuno può chiudere, perché, pur avendo poca forza, hai serbata la mia parola, e non hai rinnegato il mio nome.
\par 9 Ecco, io ti do di quelli della sinagoga di Satana, i quali dicono d'esser Giudei e non lo sono, ma mentiscono; ecco, io li farò venire a prostrarsi dinanzi ai tuoi piedi, e conosceranno ch'io t'ho amato.
\par 10 Perché tu hai serbata la parola della mia costanza, anch'io ti guarderò dall'ora del cimento che ha da venire su tutto il mondo, per mettere alla prova quelli che abitano sulla terra.
\par 11 Io vengo tosto; tieni fermamente quello che hai, affinché nessuno ti tolga la tua corona.
\par 12 Chi vince io lo farò una colonna nel tempio del mio Dio, ed egli non ne uscirà mai più; e scriverò su lui il nome del mio Dio e il nome della città del mio Dio, della nuova Gerusalemme che scende dal cielo d'appresso all'Iddio mio, ed il mio nuovo nome.
\par 13 Chi ha orecchio ascolti ciò che lo Spirito dice alle chiese.
\par 14 E all'angelo della chiesa di Laodicea scrivi: Queste cose dice l'Amen, il testimone fedele e verace, il principio della creazione di Dio:
\par 15 Io conosco le tue opere: tu non sei né freddo né fervente. Oh fossi tu pur freddo o fervente!
\par 16 Così, perché sei tiepido, e non sei né freddo, né fervente, io ti vomiterò dalla mia bocca.
\par 17 Poiché tu dici: Io son ricco, e mi sono arricchito, e non ho bisogno di nulla, e non sai che tu sei infelice fra tutti, e miserabile e povero e cieco e nudo,
\par 18 io ti consiglio di comprare da me dell'oro affinato col fuoco, affinché tu arricchisca; e delle vesti bianche, affinché tu ti vesta e non apparisca la vergogna della tua nudità; e del collirio per ungertene gli occhi, affinché tu vegga.
\par 19 Tutti quelli che amo, io li riprendo e li castigo; abbi dunque zelo e ravvediti.
\par 20 Ecco, io sto alla porta e picchio: se uno ode la mia voce ed apre la porta, io entrerò da lui e cenerò con lui ed egli meco.
\par 21 A chi vince io darò di seder meco sul mio trono, come anch'io ho vinto e mi son posto a sedere col Padre mio sul suo trono.
\par 22 Chi ha orecchio ascolti ciò che lo Spirito dice alle chiese.

\chapter{4}

\par 1 Dopo queste cose io vidi, ed ecco una porta aperta nel cielo, e la prima voce che avevo udita parlante meco a guisa di tromba, mi disse: Sali qua, e io ti mostrerò le cose che debbono avvenire da ora innanzi.
\par 2 E subito fui rapito in ispirito; ed ecco un trono era posto nel cielo, e sul trono v'era uno a sedere.
\par 3 E Colui che sedeva era nell'aspetto simile a una pietra di diaspro e di sardònico; e attorno al trono c'era un arcobaleno che, a vederlo, somigliava a uno smeraldo.
\par 4 E attorno al trono c'erano ventiquattro troni; e sui troni sedevano ventiquattro anziani, vestiti di bianche vesti, e aveano sui loro capi delle corone d'oro.
\par 5 E dal trono procedevano lampi e voci e tuoni; e davanti al trono c'erano sette lampade ardenti, che sono i sette Spiriti di Dio;
\par 6 e davanti al trono c'era come un mare di vetro, simile al cristallo; e in mezzo al trono e attorno al trono, quattro creature viventi, piene d'occhi davanti e di dietro.
\par 7 E la prima creatura vivente era simile a un leone, e la seconda simile a un vitello, e la terza avea la faccia come d'un uomo, e la quarta era simile a un'aquila volante.
\par 8 E le quattro creature viventi avevano ognuna sei ali, ed eran piene d'occhi all'intorno e di dentro, e non restavan mai, giorno e notte, di dire: Santo, santo, santo è il Signore Iddio, l'Onnipotente, che era, che è, e che viene.
\par 9 E ogni volta che le creature viventi rendon gloria e onore e grazie a Colui che siede sul trono, a Colui che vive nei secoli dei secoli,
\par 10 i ventiquattro anziani si prostrano davanti a Colui che siede sul trono e adorano Colui che vive ne' secoli dei secoli e gettano le loro corone davanti al trono, dicendo:
\par 11 Degno sei, o Signore, e Iddio nostro, di ricever la gloria e l'onore e la potenza: poiché tu creasti tutte le cose, e per la tua volontà esistettero e furon create.

\chapter{5}

\par 1 E vidi nella destra di Colui che sedeva sul trono, un libro scritto di dentro e di fuori, sigillato con sette suggelli.
\par 2 E vidi un angelo potente che bandiva con gran voce: Chi è degno d'aprire il libro e di romperne i suggelli?
\par 3 E nessuno, né in cielo, né sulla terra, né sotto la terra, poteva aprire il libro, o guardarlo.
\par 4 E io piangevo forte perché non s'era trovato nessuno che fosse degno d'aprire il libro, o di guardarlo.
\par 5 E uno degli anziani mi disse: Non piangere; ecco, il Leone che è della tribù di Giuda, il Rampollo di Davide, ha vinto per aprire il libro e i suoi sette suggelli.
\par 6 Poi vidi, in mezzo al trono e alle quattro creature viventi e in mezzo agli anziani, un Agnello in piedi, che pareva essere stato immolato, ed avea sette corna e sette occhi che sono i sette Spiriti di Dio, mandati per tutta la terra.
\par 7 Ed esso venne e prese il libro dalla destra di Colui che sedeva sul trono.
\par 8 E quando ebbe preso il libro, le quattro creature viventi e i ventiquattro anziani si prostrarono davanti all'Agnello, avendo ciascuno una cetra e delle coppe d'oro piene di profumi, che sono le preghiere dei santi.
\par 9 E cantavano un nuovo cantico, dicendo: Tu sei degno di prendere il libro e d'aprirne i suggelli, perché sei stato immolato e hai comprato a Dio, col tuo sangue, gente d'ogni tribù e lingua e popolo e nazione,
\par 10 e ne hai fatto per il nostro Dio un regno e de' sacerdoti; e regneranno sulla terra.
\par 11 E vidi, e udii una voce di molti angeli attorno al trono e alle creature viventi e agli anziani; e il numero loro era di miriadi di miriadi, e di migliaia di migliaia,
\par 12 che dicevano con gran voce: Degno è l'Agnello che è stato immolato di ricever la potenza e le ricchezze e la sapienza e la forza e l'onore e la gloria e la benedizione.
\par 13 E tutte le creature che sono nel cielo e sulla terra e sotto la terra e sul mare e tutte le cose che sono in essi, le udii che dicevano: A Colui che siede sul trono e all'Agnello siano la benedizione e l'onore e la gloria e l'imperio, nei secoli dei secoli.
\par 14 E le quattro creature viventi dicevano: Amen! E gli anziani si prostrarono e adorarono.

\chapter{6}

\par 1 Poi vidi quando l'Agnello ebbe aperto uno dei sette suggelli; e udii una delle quattro creature viventi, che diceva con voce come di tuono: Vieni.
\par 2 E vidi, ed ecco un cavallo bianco; e colui che lo cavalcava aveva un arco; e gli fu data una corona, ed egli uscì fuori da vincitore, e per vincere.
\par 3 E quando ebbe aperto il secondo suggello, io udii la seconda creatura vivente che diceva: Vieni.
\par 4 E uscì fuori un altro cavallo, rosso; e a colui che lo cavalcava fu dato di toglier la pace dalla terra affinché gli uomini si uccidessero gli uni gli altri, e gli fu data una grande spada.
\par 5 E quando ebbe aperto il terzo suggello, io udii la terza creatura vivente che diceva: Vieni. Ed io vidi, ed ecco un cavallo nero; e colui che lo cavalcava aveva una bilancia in mano.
\par 6 E udii come una voce in mezzo alle quattro creature viventi che diceva: Una chènice di frumento per un denaro e tre chènici d'orzo per un denaro; e non danneggiare né l'olio né il vino.
\par 7 E quando ebbe aperto il quarto suggello, io udii la voce della quarta creatura vivente che diceva: Vieni.
\par 8 E io vidi, ed ecco un cavallo giallastro; e colui che lo cavalcava avea nome la Morte; e gli teneva dietro l'Ades. E fu loro data potestà sopra la quarta parte della terra di uccidere con la spada, con la fame, con la mortalità e con le fiere della terra.
\par 9 E quando ebbe aperto il quinto suggello, io vidi sotto l'altare le anime di quelli ch'erano stati uccisi per la parola di Dio e per la testimonianza che aveano resa;
\par 10 e gridarono con gran voce, dicendo: Fino a quando, o nostro Signore che sei santo e verace, non fai tu giudicio e non vendichi il nostro sangue su quelli che abitano sopra la terra?
\par 11 E a ciascun d'essi fu data una veste bianca e fu loro detto che si riposassero ancora un po' di tempo, finché fosse completo il numero dei loro conservi e dei loro fratelli, che hanno ad essere uccisi come loro.
\par 12 Poi vidi quand'ebbe aperto il sesto suggello: e si fece un gran terremoto; e il sole divenne nero come un cilicio di crine, e tutta la luna diventò come sangue;
\par 13 e le stelle del cielo caddero sulla terra come quando un fico scosso da un gran vento lascia cadere i suoi fichi immaturi.
\par 14 E il cielo si ritrasse come una pergamena che si arrotola; e ogni montagna e ogni isola fu rimossa dal suo luogo.
\par 15 E i re della terra e i grandi e i capitani e i ricchi e i potenti e ogni servo e ogni libero si nascosero nelle spelonche e nelle rocce dei monti;
\par 16 e dicevano ai monti e alle rocce: Cadeteci addosso e nascondeteci dal cospetto di Colui che siede sul trono e dall'ira dell'Agnello;
\par 17 perché è venuto il gran giorno della sua ira, e chi può reggere in piè?

\chapter{7}

\par 1 Dopo questo, io vidi quattro angeli che stavano in piè ai quattro canti della terra, ritenendo i quattro venti della terra affinché non soffiasse vento alcuno sulla terra, né sopra il mare, né sopra alcun albero.
\par 2 E vidi un altro angelo che saliva dal sol levante, il quale aveva il suggello dell'Iddio vivente; ed egli gridò con gran voce ai quattro angeli ai quali era dato di danneggiare la terra e il mare, dicendo:
\par 3 Non danneggiate la terra, né il mare, né gli alberi, finché abbiam segnato in fronte col suggello i servitori dell'Iddio nostro.
\par 4 E udii il numero dei segnati: centoquarantaquattromila segnati di tutte le tribù dei figliuoli d'Israele:
\par 5 Della tribù di Giuda dodicimila segnati, della tribù di Ruben dodicimila, della tribù di Gad dodicimila,
\par 6 della tribù di Aser dodicimila, della tribù di Neftali dodicimila, della tribù di Manasse dodicimila,
\par 7 della tribù di Simeone dodicimila, della tribù di Levi dodicimila, della tribù di Issacar dodicimila,
\par 8 della tribù di Zabulon dodicimila, della tribù di Giuseppe dodicimila, della tribù di Beniamino dodicimila segnati.
\par 9 Dopo queste cose vidi, ed ecco una gran folla che nessun uomo poteva noverare, di tutte le nazioni e tribù e popoli e lingue, che stava in piè davanti al trono e davanti all'Agnello, vestiti di vesti bianche e con delle palme in mano.
\par 10 E gridavano con gran voce dicendo: La salvezza appartiene all'Iddio nostro il quale siede sul trono, ed all'Agnello.
\par 11 E tutti gli angeli stavano in piè attorno al trono e agli anziani e alle quattro creature viventi; e si prostrarono sulle loro facce davanti al trono, e adorarono Iddio dicendo:
\par 12 Amen! All'Iddio nostro la benedizione e la gloria e la sapienza e le azioni di grazie e l'onore e la potenza e la forza, nei secoli dei secoli! Amen.
\par 13 E uno degli anziani mi rivolse la parola dicendomi: Questi che son vestiti di vesti bianche chi son dessi, e donde son venuti?
\par 14 Io gli risposi: Signor mio, tu lo sai. Ed egli mi disse: Essi son quelli che vengono dalla gran tribolazione, e hanno lavato le loro vesti, e le hanno imbiancate nel sangue dell'Agnello.
\par 15 Perciò son davanti al trono di Dio, e gli servono giorno e notte nel suo tempio; e Colui che siede sul trono spiegherà su loro la sua tenda.
\par 16 Non avranno più fame e non avranno più sete, non li colpirà più il sole né alcuna arsura;
\par 17 perché l'Agnello che è in mezzo al trono li pasturerà e li guiderà alle sorgenti delle acque della vita; e Iddio asciugherà ogni lagrima dagli occhi loro.

\chapter{8}

\par 1 E quando l'Agnello ebbe aperto il settimo suggello, si fece silenzio nel cielo per circa lo spazio di mezz'ora.
\par 2 E io vidi i sette angeli che stanno in piè davanti a Dio, e furon date loro sette trombe.
\par 3 E un altro angelo venne e si fermò presso l'altare, avendo un turibolo d'oro; e gli furon dati molti profumi affinché li unisse alle preghiere di tutti i santi sull'altare d'oro che era davanti al trono.
\par 4 E il fumo dei profumi, unendosi alle preghiere dei santi, salì dalla mano dell'angelo al cospetto di Dio.
\par 5 Poi l'angelo prese il turibolo e l'empì del fuoco dell'altare e lo gettò sulla terra; e ne seguirono tuoni e voci e lampi e un terremoto.
\par 6 E i sette angeli che avean le sette trombe si prepararono a sonare.
\par 7 E il primo sonò, e vi fu grandine e fuoco, mescolati con sangue, che furon gettati sulla terra; e la terza parte della terra fu arsa, e la terza parte degli alberi fu arsa, ed ogni erba verde fu arsa.
\par 8 Poi sonò il secondo angelo, e una massa simile ad una gran montagna ardente fu gettata nel mare; e la terza parte del mare divenne sangue,
\par 9 e la terza parte delle creature viventi che erano nel mare morì, e la terza parte delle navi perì.
\par 10 Poi sonò il terzo angelo, e cadde dal cielo una grande stella, ardente come una torcia; e cadde sulla terza parte dei fiumi e sulle fonti delle acque.
\par 11 Il nome della stella è Assenzio; e la terza parte delle acque divenne assenzio; e molti uomini morirono a cagione di quelle acque, perché eran divenute amare.
\par 12 Poi sonò il quarto angelo, e la terza parte del sole fu colpita e la terza parte della luna e la terza parte delle stelle affinché la loro terza parte si oscurasse e il giorno non risplendesse per la sua terza parte e lo stesso avvenisse della notte.
\par 13 E guardai e udii un'aquila che volava in mezzo al cielo e diceva con gran voce: Guai, guai, guai a quelli che abitano sulla terra, a cagione degli altri suoni di tromba dei tre angeli che debbono ancora sonare!

\chapter{9}

\par 1 Poi sonò il quinto angelo, e io vidi una stella caduta dal cielo sulla terra; e ad esso fu data la chiave del pozzo dell'abisso.
\par 2 Ed egli aprì il pozzo dell'abisso; e dal pozzo salì un fumo simile al fumo di una gran fornace; e il sole e l'aria furono oscurati dal fumo del pozzo.
\par 3 E dal fumo uscirono sulla terra delle locuste; e fu dato loro un potere pari al potere che hanno gli scorpioni della terra.
\par 4 E fu loro detto di non danneggiare l'erba della terra, né alcuna verdura, né albero alcuno, ma soltanto gli uomini che non aveano il suggello di Dio in fronte.
\par 5 E fu loro dato, non di ucciderli, ma di tormentarli per cinque mesi; e il tormento che cagionavano era come quello prodotto da uno scorpione quando ferisce un uomo.
\par 6 E in quei giorni gli uomini cercheranno la morte e non la troveranno, e desidereranno di morire, e la morte fuggirà da loro.
\par 7 E nella forma le locuste eran simili a cavalli pronti alla guerra; e sulle teste aveano come delle corone simili ad oro e le loro facce eran come facce d'uomini.
\par 8 E aveano dei capelli come capelli di donne, e i denti eran come denti di leoni.
\par 9 E aveano degli usberghi come usberghi di ferro; e il rumore delle loro ali era come il rumore di carri, tirati da molti cavalli correnti alla battaglia.
\par 10 E aveano delle code come quelle degli scorpioni, e degli aculei; e nelle code stava il loro potere di danneggiare gli uomini per cinque mesi.
\par 11 E aveano come re sopra di loro l'angelo dell'abisso, il cui nome in ebraico è Abaddon, e in greco Apollion.
\par 12 Il primo guaio è passato: ecco, vengono ancora due guai dopo queste cose.
\par 13 Poi il sesto angelo sonò, e io udii una voce dalle quattro corna dell'altare d'oro che era davanti a Dio,
\par 14 la quale diceva al sesto angelo che avea la tromba: Sciogli i quattro angeli che son legati sul gran fiume Eufrate.
\par 15 E furono sciolti i quattro angeli che erano stati preparati per quell'ora, per quel giorno e mese e anno, per uccidere la terza parte degli uomini.
\par 16 E il numero degli eserciti della cavalleria era di venti migliaia di decine di migliaia; io udii il loro numero.
\par 17 Ed ecco come mi apparvero nella visione i cavalli e quelli che li cavalcavano: aveano degli usberghi di fuoco, di giacinto e di zolfo; e le teste dei cavalli erano come teste di leoni; e dalle loro bocche usciva fuoco e fumo e zolfo.
\par 18 Da queste tre piaghe: dal fuoco, dal fumo e dallo zolfo che usciva dalle loro bocche fu uccisa la terza parte degli uomini.
\par 19 Perché il potere dei cavalli era nella loro bocca e nelle loro code; poiché le loro code eran simili a serpenti e aveano delle teste, e con esse danneggiavano.
\par 20 E il resto degli uomini che non furono uccisi da queste piaghe, non si ravvidero delle opere delle loro mani sì da non adorar più i demonî e gl'idoli d'oro e d'argento e di rame e di pietra e di legno, i quali non possono né vedere, né udire, né camminare;
\par 21 e non si ravvidero dei loro omicidî, né delle loro malìe, né della loro fornicazione, né dei loro furti.

\chapter{10}

\par 1 Poi vidi un altro angelo potente che scendeva dal cielo, avvolto in una nuvola; sopra il suo capo era l'arcobaleno; la sua faccia era come il sole, e i suoi piedi come colonne di fuoco;
\par 2 e aveva in mano un libretto aperto; ed egli posò il suo piè destro sul mare e il sinistro sulla terra;
\par 3 e gridò con gran voce, nel modo che rugge il leone; e quando ebbe gridato, i sette tuoni fecero udire le loro voci.
\par 4 E quando i sette tuoni ebbero fatto udire le loro voci, io stavo per scrivere; ma udii una voce dal cielo che mi disse: Suggella le cose che i sette tuoni hanno proferite, e non le scrivere.
\par 5 E l'angelo che io avea veduto stare in piè sul mare e sulla terra,
\par 6 levò la man destra al cielo e giurò per Colui che vive nei secoli dei secoli, il quale ha creato il cielo e le cose che sono in esso e la terra e le cose che sono in essa e il mare e le cose che sono in esso, che non ci sarebbe più indugio;
\par 7 ma che nei giorni della voce del settimo angelo, quand'egli sonerebbe, si compirebbe il mistero di Dio, secondo ch'Egli ha annunziato ai suoi servitori, i profeti.
\par 8 E la voce che io avevo udita dal cielo mi parlò di nuovo e disse: Va', prendi il libro che è aperto in mano all'angelo che sta in piè sul mare e sulla terra.
\par 9 E io andai dall'angelo, dicendogli di darmi il libretto. Ed egli mi disse: Prendilo, e divoralo: esso sarà amaro alle tue viscere, ma in bocca ti sarà dolce come miele.
\par 10 Presi il libretto di mano all'angelo, e lo divorai; e mi fu dolce in bocca, come miele; ma quando l'ebbi divorato, le mie viscere sentirono amarezza.
\par 11 E mi fu detto: Bisogna che tu profetizzi di nuovo sopra molti popoli e nazioni e lingue e re.

\chapter{11}

\par 1 Poi mi fu data una canna simile a una verga; e mi fu detto: Lèvati e misura il tempio di Dio e l'altare e novera quelli che vi adorano;
\par 2 ma tralascia il cortile che è fuori del tempio, e non lo misurare, perché esso è stato dato ai Gentili, e questi calpesteranno la santa città per quarantadue mesi.
\par 3 E io darò ai miei due testimoni di profetare, ed essi profeteranno per milleduecentosessanta giorni, vestiti di cilicio.
\par 4 Questi sono i due olivi e i due candelabri che stanno nel cospetto del Signor della terra.
\par 5 E se alcuno li vuole offendere, esce dalla lor bocca un fuoco che divora i loro nemici; e se alcuno li vuole offendere bisogna ch'ei sia ucciso in questa maniera.
\par 6 Essi hanno il potere di chiudere il cielo onde non cada pioggia durante i giorni della loro profezia; e hanno potestà sulle acque di convertirle in sangue, potestà di percuotere la terra di qualunque piaga, quante volte vorranno.
\par 7 E quando avranno compiuta la loro testimonianza, la bestia che sale dall'abisso moverà loro guerra e li vincerà e li ucciderà.
\par 8 E i loro corpi morti giaceranno sulla piazza della gran città, che spiritualmente si chiama Sodoma ed Egitto, dove anche il Signor loro è stato crocifisso.
\par 9 E gli uomini dei varî popoli e tribù e lingue e nazioni vedranno i loro corpi morti per tre giorni e mezzo, e non lasceranno che i loro corpi morti siano posti in un sepolcro.
\par 10 E gli abitanti della terra si rallegreranno di loro e faranno festa e si manderanno regali gli uni agli altri, perché questi due profeti avranno tormentati gli abitanti della terra.
\par 11 E in capo ai tre giorni e mezzo uno spirito di vita procedente da Dio entrò in loro, ed essi si drizzarono in piè e grande spavento cadde su quelli che li videro.
\par 12 Ed essi udirono una gran voce dal cielo che diceva loro: Salite qua. Ed essi salirono al cielo nella nuvola, e i loro nemici li videro.
\par 13 E in quell'ora si fece un gran terremoto, e la decima parte della città cadde, e settemila persone furono uccise nel terremoto; e il rimanente fu spaventato e dette gloria all'Iddio del cielo.
\par 14 Il secondo guaio è passato; ed ecco, il terzo guaio verrà tosto.
\par 15 Ed il settimo angelo sonò, e si fecero gran voci nel cielo, che dicevano: Il regno del mondo è venuto ad essere del Signor nostro e del suo Cristo; ed egli regnerà ne' secoli dei secoli.
\par 16 E i ventiquattro anziani seduti nel cospetto di Dio sui loro troni si gettaron giù sulle loro facce e adorarono Iddio, dicendo:
\par 17 Noi ti ringraziamo, o Signore Iddio onnipotente che sei e che eri, perché hai preso in mano il tuo gran potere, ed hai assunto il regno.
\par 18 Le nazioni s'erano adirate, ma l'ira tua è giunta, ed è giunto il tempo di giudicare i morti, di dare il loro premio ai tuoi servitori, i profeti, ed ai santi e a quelli che temono il tuo nome, e piccoli e grandi, e di distruggere quelli che distruggon la terra.
\par 19 E il tempio di Dio che è nel cielo fu aperto, e si vide nel suo tempio l'arca del suo patto, e vi furono lampi e voci e tuoni e un terremoto ed una forte gragnuola.

\chapter{12}

\par 1 Poi apparve un gran segno nel cielo: una donna rivestita del sole con la luna sotto i piedi, e sul capo una corona di dodici stelle.
\par 2 Ella era incinta, e gridava nelle doglie tormentose del parto.
\par 3 E apparve un altro segno nel cielo; ed ecco un gran dragone rosso che aveva sette teste e dieci corna e sulle teste sette diademi.
\par 4 E la sua coda trascinava la terza parte delle stelle del cielo e le gettò sulla terra. E il dragone si fermò davanti alla donna che stava per partorire, affin di divorarne il figliuolo, quando l'avrebbe partorito.
\par 5 Ed ella partorì un figliuolo maschio, che ha da reggere tutte le nazioni con verga di ferro; e il figliuolo di lei fu rapito presso a Dio ed al suo trono.
\par 6 E la donna fuggì nel deserto, dove ha un luogo preparato da Dio, affinché vi sia nutrita per milleduecentosessanta giorni.
\par 7 E vi fu battaglia in cielo: Michele e i suoi angeli combatterono col dragone, e il dragone e i suoi angeli combatterono,
\par 8 ma non vinsero, e il luogo loro non fu più trovato nel cielo.
\par 9 E il gran dragone, il serpente antico, che è chiamato Diavolo e Satana, il seduttore di tutto il mondo, fu gettato giù; fu gettato sulla terra, e con lui furon gettati gli angeli suoi.
\par 10 Ed io udii una gran voce nel cielo che diceva: Ora è venuta la salvezza e la potenza ed il regno dell'Iddio nostro, e la potestà del suo Cristo, perché è stato gettato giù l'accusatore dei nostri fratelli, che li accusava dinanzi all'Iddio nostro, giorno e notte.
\par 11 Ma essi l'hanno vinto a cagion del sangue dell'Agnello e a cagion della parola della loro testimonianza; e non hanno amata la loro vita, anzi l'hanno esposta alla morte.
\par 12 Perciò rallegratevi, o cieli, e voi che abitate in essi. Guai a voi, o terra, o mare! Perché il diavolo è disceso a voi con gran furore, sapendo di non aver che breve tempo.
\par 13 E quando il dragone si vide gettato sulla terra, perseguitò la donna che avea partorito il figliuolo maschio.
\par 14 Ma alla donna furon date le due ali della grande aquila affinché se ne volasse nel deserto, nel suo luogo, dove è nutrita un tempo, dei tempi e la metà d'un tempo, lungi dalla presenza del serpente.
\par 15 E il serpente gettò dalla sua bocca, dietro alla donna, dell'acqua a guisa di fiume, per farla portar via dalla fiumana.
\par 16 Ma la terra soccorse la donna; e la terra aprì la sua bocca e inghiottì il fiume che il dragone avea gettato fuori dalla propria bocca.
\par 17 E il dragone si adirò contro la donna e andò a far guerra col rimanente della progenie d'essa, che serba i comandamenti di Dio e ritiene la testimonianza di Gesù.
\par 18 E si fermò sulla riva del mare.

\chapter{13}

\par 1 E vidi salir dal mare una bestia che aveva dieci corna e sette teste, e sulle corna dieci diademi, e sulle teste nomi di bestemmia.
\par 2 E la bestia ch'io vidi era simile a un leopardo, e i suoi piedi erano come di orso, e la sua bocca come bocca di leone; e il dragone le diede la propria potenza e il proprio trono e grande potestà.
\par 3 E io vidi una delle sue teste come ferita a morte; e la sua piaga mortale fu sanata; e tutta la terra maravigliata andò dietro alla bestia;
\par 4 e adorarono il dragone perché avea dato il potere alla bestia; e adorarono la bestia dicendo: Chi è simile alla bestia? e chi può guerreggiare con lei?
\par 5 E le fu data una bocca che proferiva parole arroganti e bestemmie e le fu data potestà di agire per quarantadue mesi.
\par 6 Ed essa aprì la bocca per bestemmiare contro Dio, per bestemmiare il suo nome e il suo tabernacolo e quelli che abitano nel cielo.
\par 7 E le fu dato di far guerra ai santi e di vincerli; e le fu data potestà sopra ogni tribù e popolo e lingua e nazione.
\par 8 E tutti gli abitanti della terra i cui nomi non sono scritti fin dalla fondazione del mondo nel libro della vita dell'Agnello che è stato immolato, l'adoreranno.
\par 9 Se uno ha orecchio, ascolti. Se uno mena in cattività andrà in cattività,
\par 10 se uno uccide con la spada, bisogna che sia ucciso con la spada. Qui sta la costanza e la fede dei santi.
\par 11 Poi vidi un'altra bestia, che saliva dalla terra, ed avea due corna come quelle d'un agnello, ma parlava come un dragone.
\par 12 Ed esercitava tutta la potestà della prima bestia, alla sua presenza; e facea sì che la terra e quelli che abitano in essa adorassero la prima bestia la cui piaga mortale era stata sanata.
\par 13 E operava grandi segni, fino a far scendere del fuoco dal cielo sulla terra in presenza degli uomini.
\par 14 E seduceva quelli che abitavano sulla terra coi segni che le era dato di fare in presenza della bestia, dicendo agli abitanti della terra di fare un'immagine della bestia che avea ricevuta la ferita della spada ed era tornata in vita.
\par 15 E le fu concesso di dare uno spirito all'immagine della bestia, onde l'immagine della bestia parlasse e facesse sì che tutti quelli che non adorassero l'immagine della bestia fossero uccisi.
\par 16 E faceva sì che a tutti, piccoli e grandi, ricchi e poveri, liberi e servi, fosse posto un marchio sulla mano destra o sulla fronte;
\par 17 e che nessuno potesse comprare o vendere se non chi avesse il marchio, cioè il nome della bestia o il numero del suo nome.
\par 18 Qui sta la sapienza. Chi ha intendimento conti il numero della bestia, poiché è numero d'uomo; e il suo numero è 666.

\chapter{14}

\par 1 Poi vidi, ed ecco l'Agnello che stava in piè sul monte Sion, e con lui erano centoquarantaquattromila persone che aveano il suo nome e il nome di suo Padre scritto sulle loro fronti.
\par 2 E udii una voce dal cielo come rumore di molte acque e come rumore di gran tuono; e la voce che udii era come il suono prodotto da arpisti che suonano le loro arpe.
\par 3 E cantavano un cantico nuovo davanti al trono e davanti alle quattro creature viventi ed agli anziani; e nessuno poteva imparare il cantico se non quei centoquarantaquattromila, i quali sono stati riscattati dalla terra.
\par 4 Essi son quelli che non si sono contaminati con donne, poiché son vergini. Essi son quelli che seguono l'Agnello dovunque vada. Essi sono stati riscattati di fra gli uomini per esser primizie a Dio ed all'Agnello.
\par 5 E nella bocca loro non è stata trovata menzogna: sono irreprensibili.
\par 6 Poi vidi un altro angelo che volava in mezzo al cielo, recante l'evangelo eterno per annunziarlo a quelli che abitano sulla terra, e ad ogni nazione e tribù e lingua e popolo;
\par 7 e diceva con gran voce: Temete Iddio e dategli gloria poiché l'ora del suo giudizio è venuta; e adorate Colui che ha fatto il cielo e la terra e il mare e le fonti delle acque.
\par 8 Poi un altro, un secondo angelo, seguì dicendo: Caduta, caduta è Babilonia la grande, che ha fatto bere a tutte le nazioni del vino dell'ira della sua fornicazione.
\par 9 E un altro, un terzo angelo, tenne dietro a quelli, dicendo con gran voce: Se qualcuno adora la bestia e la sua immagine e ne prende il marchio sulla fronte o sulla mano,
\par 10 beverà anch'egli del vino dell'ira di Dio mesciuto puro nel calice della sua ira: e sarà tormentato con fuoco e zolfo nel cospetto dei santi angeli e nel cospetto dell'Agnello.
\par 11 E il fumo del loro tormento sale ne' secoli dei secoli; e non hanno requie né giorno né notte quelli che adorano la bestia e la sua immagine e chiunque prende il marchio del suo nome.
\par 12 Qui è la costanza dei santi che osservano i comandamenti di Dio e la fede in Gesù.
\par 13 E udii una voce dal cielo che diceva: Scrivi: Beati i morti che da ora innanzi muoiono nel Signore. Sì, dice lo Spirito, essendo che si riposano dalle loro fatiche, poiché le loro opere li seguono.
\par 14 E vidi ed ecco una nuvola bianca; e sulla nuvola assiso uno simile a un figliuol d'uomo, che avea sul capo una corona d'oro, e in mano una falce tagliente.
\par 15 E un altro angelo uscì dal tempio, gridando con gran voce a colui che sedeva sulla nuvola: Metti mano alla tua falce e mieti; poiché l'ora di mietere è giunta, perché la mèsse della terra è ben matura.
\par 16 E colui che sedeva sulla nuvola lanciò la sua falce sulla terra e la terra fu mietuta.
\par 17 E un altro angelo uscì dal tempio che è nel cielo, avendo anch'egli una falce tagliente.
\par 18 E un altro angelo, che avea potestà sul fuoco, uscì dall'altare, e gridò con gran voce a quello che avea la falce tagliente, dicendo: Metti mano alla tua falce tagliente, e vendemmia i grappoli della vigna della terra, perché le sue uve sono mature.
\par 19 E l'angelo lanciò la sua falce sulla terra e gettò le uve nel gran tino dell'ira di Dio.
\par 20 E il tino fu calcato fuori della città, e dal tino uscì del sangue che giungeva sino ai freni dei cavalli per una distesa di milleseicento stadî.

\chapter{15}

\par 1 Poi vidi nel cielo un altro segno grande e maraviglioso: sette angeli che aveano sette piaghe, le ultime; poiché con esse si compie l'ira di Dio.
\par 2 E vidi come un mare di vetro e di fuoco e quelli che aveano ottenuta vittoria sulla bestia e sulla sua immagine e sul numero del suo nome, i quali stavano in piè sul mare di vetro avendo delle arpe di Dio.
\par 3 E cantavano il cantico di Mosè, servitore di Dio, e il cantico dell'Agnello, dicendo: Grandi e maravigliose sono le tue opere, o Signore Iddio onnipotente; giuste e veraci sono le tue vie, o Re delle nazioni.
\par 4 Chi non temerà, o Signore, e chi non glorificherà il tuo nome? Poiché tu solo sei santo; e tutte le nazioni verranno e adoreranno nel tuo cospetto, poiché i tuoi giudicî sono stati manifestati.
\par 5 E dopo queste cose vidi, e il tempio del tabernacolo della testimonianza fu aperto nel cielo;
\par 6 e i sette angeli che recavano le sette piaghe usciron dal tempio, vestiti di lino puro e risplendente, e col petto cinto di cinture d'oro.
\par 7 E una delle quattro creature viventi diede ai sette angeli sette coppe d'oro piene dell'ira di Dio, il quale vive nei secoli dei secoli.
\par 8 E il tempio fu ripieno di fumo a cagione della gloria di Dio e della sua potenza; e nessuno poteva entrare nel tempio finché fosser compiute le sette piaghe dei sette angeli.

\chapter{16}

\par 1 E udii una gran voce dal tempio che diceva ai sette angeli: Andate e versate sulla terra le sette coppe dell'ira di Dio.
\par 2 E il primo andò e versò la sua coppa sulla terra; e un'ulcera maligna e dolorosa colpì gli uomini che aveano il marchio della bestia e che adoravano la sua immagine.
\par 3 Poi il secondo angelo versò la sua coppa nel mare; ed esso divenne sangue come di morto; ed ogni essere vivente che si trovava nel mare morì.
\par 4 Poi il terzo angelo versò la sua coppa nei fiumi e nelle fonti delle acque; e le acque diventarono sangue.
\par 5 E udii l'angelo delle acque che diceva: Sei giusto, tu che sei e che eri, tu, il Santo, per aver così giudicato.
\par 6 Hanno sparso il sangue dei santi e dei profeti, e tu hai dato loro a bere del sangue; essi ne son degni!
\par 7 E udii l'altare che diceva: Sì, o Signore Iddio onnipotente, i tuoi giudicî sono veraci e giusti.
\par 8 Poi il quarto angelo versò la sua coppa sul sole; e al sole fu dato di bruciare gli uomini col fuoco.
\par 9 E gli uomini furon arsi dal gran calore; e bestemmiarono il nome di Dio che ha la potestà su queste piaghe, e non si ravvidero per dargli gloria.
\par 10 Poi il quinto angelo versò la sua coppa sul trono della bestia; e il regno d'essa divenne tenebroso, e gli uomini si mordevano la lingua per il dolore,
\par 11 e bestemmiarono l'Iddio del cielo a motivo de' loro dolori e delle loro ulceri; e non si ravvidero delle loro opere.
\par 12 Poi il sesto angelo versò la sua coppa sul gran fiume Eufrate, e l'acqua ne fu asciugata affinché fosse preparata la via ai re che vengono dal levante.
\par 13 E vidi uscir dalla bocca del dragone e dalla bocca della bestia e dalla bocca del falso profeta tre spiriti immondi simili a rane;
\par 14 perché sono spiriti di demonî che fan de' segni e si recano dai re di tutto il mondo per radunarli per la battaglia del gran giorno dell'Iddio Onnipotente.
\par 15 (Ecco, io vengo come un ladro; beato colui che veglia e serba le sue vesti onde non cammini ignudo e non si veggano le sue vergogne).
\par 16 Ed essi li radunarono nel luogo che si chiama in ebraico Harmaghedon.
\par 17 Poi il settimo angelo versò la sua coppa nell'aria; e una gran voce uscì dal tempio, dal trono, dicendo: È fatto.
\par 18 E si fecero lampi e voci e tuoni; e ci fu un gran terremoto, tale, che da quando gli uomini sono stati sulla terra, non si ebbe mai terremoto così grande e così forte.
\par 19 E la gran città fu divisa in tre parti, e le città delle nazioni caddero; e Dio si ricordò di Babilonia la grande per darle il calice del vino del furor dell'ira sua.
\par 20 Ed ogni isola fuggì e i monti non furon più trovati.
\par 21 E cadde dal cielo sugli uomini una gragnuola grossa del peso di circa un talento; e gli uomini bestemmiarono Iddio a motivo della piaga della gragnuola; perché la piaga d'essa era grandissima.

\chapter{17}

\par 1 E uno dei sette angeli che aveano le sette coppe venne, e mi parlò dicendo: Vieni; io ti mostrerò il giudicio della gran meretrice, che siede su molte acque
\par 2 e con la quale hanno fornicato i re della terra; e gli abitanti della terra sono stati inebriati del vino della sua fornicazione.
\par 3 Ed egli, nello Spirito, mi trasportò in un deserto; e io vidi una donna che sedeva sopra una bestia di colore scarlatto, piena di nomi di bestemmia e avente sette teste e dieci corna.
\par 4 E la donna era vestita di porpora e di scarlatto, adorna d'oro, di pietre preziose e di perle; aveva in mano un calice d'oro pieno di abominazioni e delle immondizie della sua fornicazione,
\par 5 e sulla fronte avea scritto un nome: Mistero, Babilonia la grande, la madre delle meretrici e delle abominazioni della terra.
\par 6 E vidi la donna ebbra del sangue dei santi e del sangue dei martiri di Gesù. E quando l'ebbi veduta, mi maravigliai di gran maraviglia.
\par 7 E l'angelo mi disse: Perché ti maravigli? Io ti dirò il mistero della donna e della bestia che la porta, la quale ha le sette teste e le dieci corna.
\par 8 La bestia che hai veduta era, e non è, e deve salire dall'abisso e andare in perdizione. E quelli che abitano sulla terra i cui nomi non sono stati scritti nel libro della vita fin dalla fondazione del mondo, si maraviglieranno vedendo che la bestia era, e non è, e verrà di nuovo.
\par 9 Qui sta la mente che ha sapienza. Le sette teste sono sette monti sui quali la donna siede;
\par 10 e sono anche sette re: cinque son caduti, uno è, e l'altro non è ancora venuto; e quando sarà venuto, ha da durar poco.
\par 11 E la bestia che era, e non è, è anch'essa un ottavo re, e viene dai sette, e se ne va in perdizione.
\par 12 E le dieci corna che hai vedute sono dieci re, che non hanno ancora ricevuto regno; ma riceveranno potestà, come re, assieme alla bestia, per un'ora.
\par 13 Costoro hanno uno stesso pensiero e daranno la loro potenza e la loro autorità alla bestia.
\par 14 Costoro guerreggeranno contro l'Agnello, e l'Agnello li vincerà, perché egli è il Signor dei signori e il Re dei re; e vinceranno anche quelli che sono con lui, i chiamati, gli eletti e fedeli.
\par 15 Poi mi disse: Le acque che hai vedute e sulle quali siede la meretrice, son popoli e moltitudini e nazioni e lingue.
\par 16 E le dieci corna che hai vedute e la bestia odieranno la meretrice e la renderanno desolata e nuda, e mangeranno le sue carni e la consumeranno col fuoco.
\par 17 Poiché Iddio ha messo in cuor loro di eseguire il suo disegno e di avere un medesimo pensiero e di dare il loro regno alla bestia finché le parole di Dio siano adempite.
\par 18 E la donna che hai veduta è la gran città che impera sui re della terra.

\chapter{18}

\par 1 E dopo queste cose vidi un altro angelo che scendeva dal cielo, il quale aveva gran potestà; e la terra fu illuminata dalla sua gloria.
\par 2 Ed egli gridò con voce potente, dicendo: Caduta, caduta è Babilonia la grande, ed è divenuta albergo di demonî e ricetto d'ogni spirito immondo e ricetto d'ogni uccello immondo e abominevole.
\par 3 Poiché tutte le nazioni han bevuto del vino dell'ira della sua fornicazione, e i re della terra han fornicato con lei, e i mercanti della terra si sono arricchiti con la sua sfrenata lussuria.
\par 4 Poi udii un'altra voce dal cielo che diceva: Uscite da essa, o popolo mio, affinché non siate partecipi de' suoi peccati e non abbiate parte alle sue piaghe;
\par 5 poiché i suoi peccati si sono accumulati fino al cielo e Dio si è ricordato delle iniquità di lei.
\par 6 Rendetele il contraccambio di quello ch'ella vi ha fatto, e rendetele al doppio la retribuzione delle sue opere; nel calice in cui ha mesciuto ad altri, mescetele il doppio.
\par 7 Quanto ella ha glorificato se stessa ed ha lussureggiato, tanto datele di tormento e di cordoglio. Poiché ella dice in cuor suo: Io seggo regina e non son vedova e non vedrò mai cordoglio;
\par 8 perciò in uno stesso giorno verranno le sue piaghe, mortalità e cordoglio e fame, e sarà consumata dal fuoco; poiché potente è il Signore Iddio che l'ha giudicata.
\par 9 E i re della terra che fornicavano e lussureggiavan con lei la piangeranno e faran cordoglio per lei quando vedranno il fumo del suo incendio;
\par 10 e standosene da lungi per tema del suo tormento diranno: Ahi! ahi! Babilonia, la gran città, la potente città! Il tuo giudicio è venuto in un momento!
\par 11 I mercanti della terra piangeranno e faranno cordoglio per lei, perché nessuno compera più le loro mercanzie:
\par 12 mercanzie d'oro, d'argento, di pietre preziose, di perle, di lino fino, di porpora, di seta, di scarlatto; e ogni sorta di legno odoroso, e ogni sorta d'oggetti d'avorio e ogni sorta d'oggetti di legno preziosissimo e di rame, di ferro e di marmo,
\par 13 e la cannella e le essenze, e i profumi, e gli unguenti, e l'incenso, e il vino, e l'olio, e il fior di farina, e il grano, e i buoi, e le pecore, e i cavalli, e i carri, e i corpi e le anime d'uomini.
\par 14 E i frutti che l'anima tua appetiva se ne sono andati lungi da te; e tutte le cose delicate e sontuose son perdute per te e non si troveranno mai più.
\par 15 I mercanti di queste cose che sono stati arricchiti da lei se ne staranno da lungi per tema del suo tormento, piangendo e facendo cordoglio, e dicendo:
\par 16 Ahi! ahi! La gran città ch'era vestita di lino fino e di porpora e di scarlatto, e adorna d'oro e di pietre preziose e di perle! Una cotanta ricchezza è stata devastata in un momento.
\par 17 E tutti i piloti e tutti i naviganti e i marinari e quanti trafficano sul mare se ne staranno da lungi;
\par 18 e vedendo il fumo dell'incendio d'essa esclameranno dicendo: Qual città era simile a questa gran città?
\par 19 E si getteranno della polvere sul capo e grideranno, piangendo e facendo cordoglio e dicendo: Ahi! ahi! La gran città nella quale tutti coloro che aveano navi in mare si erano arricchiti con la sua magnificenza! In un momento ella è stata ridotta in un deserto.
\par 20 Rallègrati d'essa, o cielo, e voi santi, ed apostoli e profeti, rallegratevi poiché Dio, giudicandola, vi ha reso giustizia.
\par 21 Poi un potente angelo sollevò una pietra grossa come una gran macina, e la gettò nel mare dicendo: Così sarà con impeto precipitata Babilonia, la gran città, e non sarà più ritrovata.
\par 22 E in te non sarà più udito suono di arpisti né di musici né di flautisti né di sonatori di tromba; né sarà più trovato in te artefice alcuno d'arte qualsiasi, né s'udrà più in te rumor di macina.
\par 23 E non rilucerà più in te lume di lampada e non s'udrà più in te voce di sposo e di sposa; perché i tuoi mercanti erano i principi della terra, perché tutte le nazioni sono state sedotte dalle tue malìe,
\par 24 e in lei è stato trovato il sangue dei profeti e dei santi e di tutti quelli che sono stati uccisi sopra la terra.

\chapter{19}

\par 1 Dopo queste cose udii come una gran voce d'una immensa moltitudine nel cielo, che diceva: Alleluia! La salvazione e la gloria e la potenza appartengono al nostro Dio;
\par 2 perché veraci e giusti sono i suoi giudicî; poiché Egli ha giudicata la gran meretrice che corrompeva la terra con la sua fornicazione e ha vendicato il sangue de' suoi servitori, ridomandandolo dalla mano di lei.
\par 3 E dissero una seconda volta: Alleluia! Il suo fumo sale per i secoli dei secoli.
\par 4 E i ventiquattro anziani e le quattro creature viventi si gettarono giù e adorarono Iddio che siede sul trono, dicendo: Amen! Alleluia!
\par 5 E una voce partì dal trono dicendo: Lodate il nostro Dio, voi tutti suoi servitori, voi che lo temete, piccoli e grandi.
\par 6 Poi udii come la voce di una gran moltitudine e come il suono di molte acque e come il rumore di forti tuoni, che diceva: Alleluia! poiché il Signore Iddio nostro, l'Onnipotente, ha preso a regnare.
\par 7 Rallegriamoci e giubiliamo e diamo a lui la gloria, poiché son giunte le nozze dell'Agnello, e la sua sposa s'è preparata;
\par 8 e le è stato dato di vestirsi di lino fino, risplendente e puro; poiché il lino fino son le opere giuste dei santi.
\par 9 E l'angelo mi disse: Scrivi: Beati quelli che sono invitati alla cena delle nozze dell'Agnello. E mi disse: Queste sono le veraci parole di Dio.
\par 10 E io mi prostrai ai suoi piedi per adorarlo. Ed egli mi disse: Guàrdati dal farlo; io sono tuo conservo e de' tuoi fratelli che serbano la testimonianza di Gesù; adora Iddio! Perché la testimonianza di Gesù è lo spirito della profezia.
\par 11 Poi vidi il cielo aperto, ed ecco un cavallo bianco; e colui che lo cavalcava si chiama il Fedele e il Verace; ed egli giudica e guerreggia con giustizia.
\par 12 E i suoi occhi erano una fiamma di fuoco, e sul suo capo v'eran molti diademi; e portava scritto un nome che nessuno conosce fuorché lui.
\par 13 Era vestito d'una veste tinta di sangue, e il suo nome è: la Parola di Dio.
\par 14 Gli eserciti che sono nel cielo lo seguivano sopra cavalli bianchi, ed eran vestiti di lino fino bianco e puro.
\par 15 E dalla bocca gli usciva una spada affilata per percuoter con essa le nazioni; ed egli le reggerà con una verga di ferro, e calcherà il tino del vino dell'ardente ira dell'Onnipotente Iddio.
\par 16 E sulla veste e sulla coscia porta scritto questo nome: RE DEI RE, SIGNOR DEI SIGNORI.
\par 17 Poi vidi un angelo che stava in piè nel sole, ed egli gridò con gran voce, dicendo a tutti gli uccelli che volano in mezzo al cielo:
\par 18 Venite, adunatevi per il gran convito di Dio, per mangiar carni di re e carni di capitani e carni di prodi e carni di cavalli e di cavalieri, e carni d'ogni sorta d'uomini liberi e schiavi, piccoli e grandi.
\par 19 E vidi la bestia e i re della terra e i loro eserciti radunati per muover guerra a colui che cavalcava il cavallo e all'esercito suo.
\par 20 E la bestia fu presa, e con lei fu preso il falso profeta che avea fatto i miracoli davanti a lei, coi quali aveva sedotto quelli che aveano preso il marchio della bestia e quelli che adoravano la sua immagine. Ambedue furon gettati vivi nello stagno ardente di fuoco e di zolfo.
\par 21 E il rimanente fu ucciso con la spada che usciva dalla bocca di colui che cavalcava il cavallo; e tutti gli uccelli si satollarono delle loro carni.

\chapter{20}

\par 1 Poi vidi un angelo che scendeva dal cielo e avea la chiave dell'abisso e una gran catena in mano.
\par 2 Ed egli afferrò il dragone, il serpente antico, che è il Diavolo e Satana, e lo legò per mille anni,
\par 3 lo gettò nell'abisso che chiuse e suggellò sopra di lui onde non seducesse più le nazioni finché fossero compiti i mille anni; dopo di che egli ha da essere sciolto per un po' di tempo.
\par 4 Poi vidi dei troni; e a coloro che vi si sedettero fu dato il potere di giudicare. E vidi le anime di quelli che erano stati decollati per la testimonianza di Gesù e per la parola di Dio, e di quelli che non aveano adorata la bestia né la sua immagine, e non aveano preso il marchio sulla loro fronte e sulla loro mano; ed essi tornarono in vita, e regnarono con Cristo mille anni.
\par 5 Il rimanente dei morti non tornò in vita prima che fosser compiti i mille anni. Questa è la prima risurrezione.
\par 6 Beato e santo è colui che partecipa alla prima risurrezione. Su loro non ha potestà la morte seconda, ma saranno sacerdoti di Dio e di Cristo e regneranno con lui quei mille anni.
\par 7 E quando i mille anni saranno compiti, Satana sarà sciolto dalla sua prigione
\par 8 e uscirà per sedurre le nazioni che sono ai quattro canti della terra, Gog e Magog, per adunarle per la battaglia; il loro numero è come la rena del mare.
\par 9 E salirono sulla distesa della terra e attorniarono il campo dei santi e la città diletta; ma dal cielo discese del fuoco e le divorò.
\par 10 E il diavolo che le avea sedotte fu gettato nello stagno di fuoco e di zolfo, dove sono anche la bestia e il falso profeta; e saran tormentati giorno e notte, nei secoli dei secoli.
\par 11 Poi vidi un gran trono bianco e Colui che vi sedeva sopra, dalla cui presenza fuggiron terra e cielo; e non fu più trovato posto per loro.
\par 12 E vidi i morti, grandi e piccoli, che stavan ritti davanti al trono; ed i libri furono aperti; e un altro libro fu aperto, che è il libro della vita; e i morti furon giudicati dalle cose scritte nei libri, secondo le opere loro.
\par 13 E il mare rese i morti ch'erano in esso; e la morte e l'Ades resero i loro morti, ed essi furon giudicati, ciascuno secondo le sue opere.
\par 14 E la morte e l'Ades furon gettati nello stagno di fuoco. Questa è la morte seconda, cioè, lo stagno di fuoco.
\par 15 E se qualcuno non fu trovato scritto nel libro della vita, fu gettato nello stagno di fuoco.

\chapter{21}

\par 1 Poi vidi un nuovo cielo e una nuova terra, perché il primo cielo e la prima terra erano passati, e il mare non era più.
\par 2 E vidi la santa città, la nuova Gerusalemme, scender giù dal cielo d'appresso a Dio, pronta come una sposa adorna per il suo sposo.
\par 3 E udii una gran voce dal trono, che diceva: Ecco il tabernacolo di Dio con gli uomini; ed Egli abiterà con loro, ed essi saranno suoi popoli, e Dio stesso sarà con loro e sarà loro Dio;
\par 4 e asciugherà ogni lagrima dagli occhi loro e la morte non sarà più; né ci saran più cordoglio, né grido, né dolore, poiché le cose di prima sono passate.
\par 5 E Colui che siede sul trono disse: Ecco, io fo ogni cosa nuova, ed aggiunse: Scrivi, perché queste parole sono fedeli e veraci.
\par 6 Poi mi disse: È compiuto. Io son l'Alfa e l'Omega, il principio e la fine. A chi ha sete io darò gratuitamente della fonte dell'acqua della vita.
\par 7 Chi vince erediterà queste cose; e io gli sarò Dio, ed egli mi sarà figliuolo;
\par 8 ma quanto ai codardi, agl'increduli, agli abominevoli, agli omicidi, ai fornicatori, agli stregoni, agli idolatri e a tutti i bugiardi, la loro parte sarà nello stagno ardente di fuoco e di zolfo, che è la morte seconda.
\par 9 E venne uno dei sette angeli che aveano le sette coppe piene delle sette ultime piaghe; e parlò meco, dicendo: Vieni e ti mostrerò la sposa, la moglie dell'Agnello.
\par 10 E mi trasportò in ispirito su di una grande ed alta montagna, e mi mostrò la santa città, Gerusalemme, che scendeva dal cielo d'appresso a Dio, avendo la gloria di Dio.
\par 11 Il suo luminare era simile a una pietra preziosissima, a guisa d'una pietra di diaspro cristallino.
\par 12 Avea un muro grande ed alto; avea dodici porte, e alle porte dodici angeli, e sulle porte erano scritti dei nomi, che sono quelli delle dodici tribù dei figliuoli d'Israele.
\par 13 A oriente c'eran tre porte; a settentrione tre porte; a mezzogiorno tre porte, e ad occidente tre porte.
\par 14 E il muro della città avea dodici fondamenti, e su quelli stavano i dodici nomi dei dodici apostoli dell'Agnello.
\par 15 E colui che parlava meco aveva una misura, una canna d'oro, per misurare la città, le sue porte e il suo muro.
\par 16 E la città era quadrangolare, e la sua lunghezza era uguale alla larghezza; egli misurò la città con la canna, ed era dodicimila stadi; la sua lunghezza, la sua larghezza e la sua altezza erano uguali.
\par 17 Ne misurò anche il muro, ed era di centoquarantaquattro cubiti, a misura d'uomo, cioè d'angelo.
\par 18 Il muro era costruito di diaspro e la città era d'oro puro, simile a vetro puro.
\par 19 I fondamenti del muro della città erano adorni d'ogni maniera di pietre preziose. Il primo fondamento era di diaspro; il secondo di zaffiro; il terzo di calcedonio; il quarto di smeraldo;
\par 20 il quinto di sardonico; il sesto di sardio; il settimo di crisolito; l'ottavo di berillo; il nono di topazio; il decimo di crisopazio; l'undecimo di giacinto; il dodicesimo di ametista.
\par 21 E le dodici porte eran dodici perle, e ognuna delle porte era fatta d'una perla; e la piazza della città era d'oro puro simile a vetro trasparente.
\par 22 E non vidi in essa alcun tempio, perché il Signore Iddio, l'Onnipotente, e l'Agnello sono il suo tempio.
\par 23 E la città non ha bisogno di sole, né di luna che risplendano in lei, perché la illumina la gloria di Dio, e l'Agnello è il suo luminare.
\par 24 E le nazioni cammineranno alla sua luce; e i re della terra vi porteranno la loro gloria.
\par 25 E le sue porte non saranno mai chiuse di giorno (la notte quivi non sarà più);
\par 26 e in lei si porterà la gloria e l'onore delle nazioni.
\par 27 E niente d'immondo e nessuno che commetta abominazione o falsità, v'entreranno; ma quelli soltanto che sono scritti nel libro della vita dell'Agnello.

\chapter{22}

\par 1 Poi mi mostrò il fiume dell'acqua della vita, limpido come cristallo, che procedeva dal trono di Dio e dell'Agnello.
\par 2 In mezzo alla piazza della città e d'ambo i lati del fiume stava l'albero della vita che dà dodici raccolti, e porta il suo frutto ogni mese; e le foglie dell'albero sono per la guarigione delle nazioni.
\par 3 E non ci sarà più alcuna cosa maledetta; e in essa sarà il trono di Dio e dell'Agnello;
\par 4 i suoi servitori gli serviranno ed essi vedranno la sua faccia e avranno in fronte il suo nome.
\par 5 E non ci sarà più notte; ed essi non avranno bisogno di luce di lampada, né di luce di sole, perché li illuminerà il Signore Iddio, ed essi regneranno nei secoli dei secoli.
\par 6 Poi mi disse: Queste parole sono fedeli e veraci; e il Signore, l'Iddio degli spiriti dei profeti, ha mandato il suo angelo per mostrare ai suoi servitori le cose che debbono avvenire in breve.
\par 7 Ecco, io vengo tosto. Beato chi serba le parole della profezia di questo libro.
\par 8 E io, Giovanni, son quello che udii e vidi queste cose. E quando le ebbi udite e vedute, mi prostrai per adorare ai piedi dell'angelo che mi avea mostrate queste cose.
\par 9 Ma egli mi disse: Guàrdati dal farlo; io sono tuo conservo e de' tuoi fratelli, i profeti, e di quelli che serbano le parole di questo libro. Adora Iddio.
\par 10 Poi mi disse: Non suggellare le parole della profezia di questo libro, perché il tempo è vicino.
\par 11 Chi è ingiusto sia ingiusto ancora; chi è contaminato si contamini ancora; e chi è giusto pratichi ancora la giustizia e chi è santo si santifichi ancora.
\par 12 Ecco, io vengo tosto, e il mio premio è meco per rendere a ciascuno secondo che sarà l'opera sua.
\par 13 Io son l'Alfa e l'Omega, il primo e l'ultimo, il principio e la fine.
\par 14 Beati coloro che lavano le loro vesti per aver diritto all'albero della vita e per entrare per le porte nella città!
\par 15 Fuori i cani, gli stregoni, i fornicatori, gli omicidi, gli idolatri e chiunque ama e pratica la menzogna.
\par 16 Io Gesù ho mandato il mio angelo per attestarvi queste cose in seno alle chiese. Io son la radice e la progenie di Davide, la lucente stella mattutina.
\par 17 E lo Spirito e la sposa dicono: Vieni. E chi ode dica: Vieni. E chi ha sete venga: chi vuole, prenda in dono dell'acqua della vita.
\par 18 Io lo dichiaro a ognuno che ode le parole della profezia di questo libro: Se alcuno vi aggiunge qualcosa, Dio aggiungerà ai suoi mali le piaghe descritte in questo libro;
\par 19 e se alcuno toglie qualcosa dalle parole del libro di questa profezia, Iddio gli torrà la sua parte dell'albero della vita e della città santa, delle cose scritte in questo libro.
\par 20 Colui che attesta queste cose, dice: Sì; vengo tosto! Amen! Vieni, Signor Gesù!
\par 21 La grazia del Signor Gesù sia con tutti.


\end{document}