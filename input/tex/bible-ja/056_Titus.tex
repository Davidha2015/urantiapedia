\begin{document}

\title{Titus}

Tit 1:1  神の僕、イエス・キリストの使徒パウロから――わたしが使徒とされたのは、神に選ばれた者たちの信仰を強め、また、信心にかなう真理の知識を彼らに得させるためであり、
Tit 1:2  偽りのない神が永遠の昔に約束された永遠のいのちの望みに基くのである。
Tit 1:3  神は、定められた時に及んで、御言を宣教によって明らかにされたが、わたしは、わたしたちの救主なる神の任命によって、この宣教をゆだねられたのである――
Tit 1:4  信仰を同じうするわたしの真実の子テトスへ。父なる神とわたしたちの救主キリスト・イエスから、恵みと平安とが、あなたにあるように。
Tit 1:5  あなたをクレテにおいてきたのは、わたしがあなたに命じておいたように、そこにし残してあることを整理してもらい、また、町々に長老を立ててもらうためにほかならない。
Tit 1:6  長老は、責められる点がなく、ひとりの妻の夫であって、その子たちも不品行のうわさをたてられず、親不孝をしない信者でなくてはならない。
Tit 1:7  監督たる者は、神に仕える者として、責められる点がなく、わがままでなく、軽々しく怒らず、酒を好まず、乱暴でなく、利をむさぼらず、
Tit 1:8  かえって、旅人をもてなし、善を愛し、慎み深く、正しく、信仰深く、自制する者であり、
Tit 1:9  教にかなった信頼すべき言葉を守る人でなければならない。それは、彼が健全な教によって人をさとし、また、反対者の誤りを指摘することができるためである。
Tit 1:10  実は、法に服さない者、空論に走る者、人の心を惑わす者が多くおり、とくに、割礼のある者の中に多い。
Tit 1:11  彼らの口を封ずべきである。彼らは恥ずべき利のために、教えてはならないことを教えて、数々の家庭を破壊してしまっている。
Tit 1:12  クレテ人のうちのある預言者が「クレテ人は、いつもうそつき、たちの悪いけもの、なまけ者の食いしんぼう」と言っているが、
Tit 1:13  この非難はあたっている。だから、彼らをきびしく責めて、その信仰を健全なものにし、
Tit 1:14  ユダヤ人の作り話や、真理からそれていった人々の定めなどに、気をとられることがないようにさせなさい。
Tit 1:15  きよい人には、すべてのものがきよい。しかし、汚れている不信仰な人には、きよいものは一つもなく、その知性も良心も汚れてしまっている。
Tit 1:16  彼らは神を知っていると、口では言うが、行いではそれを否定している。彼らは忌まわしい者、また不従順な者であって、いっさいの良いわざに関しては、失格者である。
Tit 2:1  しかし、あなたは、健全な教にかなうことを語りなさい。
Tit 2:2  老人たちには自らを制し、謹厳で、慎み深くし、また、信仰と愛と忍耐とにおいて健全であるように勧め、
Tit 2:3  年老いた女たちにも、同じように、たち居ふるまいをうやうやしくし、人をそしったり大酒の奴隷になったりせず、良いことを教える者となるように、勧めなさい。
Tit 2:4  そうすれば、彼女たちは、若い女たちに、夫を愛し、子供を愛し、
Tit 2:5  慎み深く、純潔で、家事に努め、善良で、自分の夫に従順であるように教えることになり、したがって、神の言がそしりを受けないようになるであろう。
Tit 2:6  若い男にも、同じく、万事につけ慎み深くあるように、勧めなさい。
Tit 2:7  あなた自身を良いわざの模範として示し、人を教える場合には、清廉と謹厳とをもってし、
Tit 2:8  非難のない健全な言葉を用いなさい。そうすれば、反対者も、わたしたちについてなんの悪口も言えなくなり、自ら恥じいるであろう。
Tit 2:9  奴隷には、万事につけその主人に服従して、喜ばれるようになり、反抗をせず、
Tit 2:10  盗みをせず、どこまでも心をこめた真実を示すようにと、勧めなさい。そうすれば、彼らは万事につけ、わたしたちの救主なる神の教を飾ることになろう。
Tit 2:11  すべての人を救う神の恵みが現れた。
Tit 2:12  そして、わたしたちを導き、不信心とこの世の情欲とを捨てて、慎み深く、正しく、信心深くこの世で生活し、
Tit 2:13  祝福に満ちた望み、すなわち、大いなる神、わたしたちの救主キリスト・イエスの栄光の出現を待ち望むようにと、教えている。
Tit 2:14  このキリストが、わたしたちのためにご自身をささげられたのは、わたしたちをすべての不法からあがない出して、良いわざに熱心な選びの民を、ご自身のものとして聖別するためにほかならない。
Tit 2:15  あなたは、権威をもってこれらのことを語り、勧め、また責めなさい。だれにも軽んじられてはならない。
Tit 3:1  あなたは彼らに勧めて、支配者、権威ある者に服し、これに従い、いつでも良いわざをする用意があり、
Tit 3:2  だれをもそしらず、争わず、寛容であって、すべての人に対してどこまでも柔和な態度を示すべきことを、思い出させなさい。
Tit 3:3  わたしたちも以前には、無分別で、不従順な、迷っていた者であって、さまざまの情欲と快楽との奴隷になり、悪意とねたみとで日を過ごし、人に憎まれ、互に憎み合っていた。
Tit 3:4  ところが、わたしたちの救主なる神の慈悲と博愛とが現れたとき、
Tit 3:5  わたしたちの行った義のわざによってではなく、ただ神のあわれみによって、再生の洗いを受け、聖霊により新たにされて、わたしたちは救われたのである。
Tit 3:6  この聖霊は、わたしたちの救主イエス・キリストをとおして、わたしたちの上に豊かに注がれた。
Tit 3:7  これは、わたしたちが、キリストの恵みによって義とされ、永遠のいのちを望むことによって、御国をつぐ者となるためである。
Tit 3:8  この言葉は確実である。わたしは、あなたがそれらのことを主張するのを願っている。それは、神を信じている者たちが、努めて良いわざを励むことを心がけるようになるためである。これは良いことであって、人々の益となる。
Tit 3:9  しかし、愚かな議論と、系図と、争いと、律法についての論争とを、避けなさい。それらは無益かつ空虚なことである。
Tit 3:10  異端者は、一、二度、訓戒を加えた上で退けなさい。
Tit 3:11  たしかに、こういう人たちは、邪道に陥り、自ら悪と知りつつも、罪を犯しているからである。
Tit 3:12  わたしがアルテマスかテキコかをあなたのところに送ったなら、急いでニコポリにいるわたしの所にきなさい。わたしは、そこで冬を過ごすことにした。
Tit 3:13  法学者ゼナスと、アポロとを、急いで旅につかせ、不自由のないようにしてあげなさい。
Tit 3:14  わたしたちの仲間も、さし迫った必要に備えて、努めて良いわざを励み、実を結ばぬ者とならないように、心がけるべきである。
Tit 3:15  わたしと共にいる一同の者から、あなたによろしく。わたしたちを愛している信徒たちに、よろしく。恵みが、あなたがた一同と共にあるように。


\end{document}