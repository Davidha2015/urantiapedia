\begin{document}

\title{느헤미야}


\chapter{1}

\par 1 하가랴의 아들 느헤미야의 말이라 아닥사스다왕 제 이십년 기슬르월에 내가 수산궁에 있더니
\par 2 나의 한 형제 중 하나니가 두어 사람과 함께 유다에서 이르렀기로 내가 그 사로잡힘을 면하고 남아 있는 유다 사람과 예루살렘 형편을 물은즉
\par 3 저희가 내게 이르되 사로잡힘을 면하고 남은 자가 그 도에서 큰 환난을 만나고 능욕을 받으며 예루살렘성은 훼파되고 성문들은 소화되었다 하는지라
\par 4 내가 이 말을 듣고 앉아서 울고 수일 동안 슬퍼하며 하늘의 하나님 앞에 금식하며 기도하여
\par 5 가로되 하늘의 하나님 여호와 크고 두려우신 하나님이여 주를 사랑하고 주의 계명을 지키는 자에게 언약을 지키시며 긍휼을 베푸시는 주여 간구하나이다
\par 6 이제 종이 주의 종 이스라엘 자손을 위하여 주야로 기도하오며 이스라엘 자손의 주 앞에 범죄함을 자복하오니 주는 귀를 기울이시며 눈을 여시사 종의 기도를 들으시옵소서 나와 나의 아비 집이 범죄하여
\par 7 주를 향하여 심히 악을 행하여 주의 종 모세에게 주께서 명하신 계명과 율례와 규례를 지키지 아니하였나이다
\par 8 옛적에 주께서 주의 종 모세에게 명하여 가라사대 만일 너희가 범죄하면 내가 너희를 열국 중에 흩을 것이요
\par 9 만일 내게로 돌아와서 내 계명을 지켜 행하면 너희 쫓긴 자가 하늘 끝에 있을지라도 내가 거기서부터 모아 내 이름을 두려고 택한 곳에 돌아오게 하리라 하신 말씀을 이제 청컨대 기억하옵소서
\par 10 이들은 주께서 일찍 큰 권능과 강한 손으로 구속하신 주의 종이요 주의 백성이니이다
\par 11 주여 구하오니 귀를 기울이사 종의 기도와 주의 이름을 경외하기를 기뻐하는 종들의 기도를 들으시고 오늘날 종으로 형통하여 이 사람 앞에서 은혜를 입게 하옵소서 하였나니 그 때에 내가 왕의 술 관원이 되었었느니라

\chapter{2}

\par 1 아닥사스다왕 이십년 니산월에 왕의 앞에 술이 있기로 내가 들어 왕에게 드렸는데 이전에는 내가 왕의 앞에서 수색이 없었더니
\par 2 왕이 내게 이르시되 네가 병이 없거늘 어찌하여 얼굴에 수색이 있느냐 이는 필연 네 마음에 근심이 있음이로다 그 때에 내가 크게 두려워하여
\par 3 왕께 대답하되 왕은 만세수를 하옵소서 나의 열조의 묘실 있는 성읍이 이제까지 황무하고 성문이 소화되었사오니 내가 어찌 얼굴에 수색이 없사오리이까
\par 4 왕이 내게 이르시되 그러면 네가 무엇을 원하느냐 하시기로 내가 곧 하늘의 하나님께 묵도하고
\par 5 왕에게 고하되 왕이 만일 즐겨하시고 종이 왕의 목전에서 은혜를 얻었사오면 나를 유다 땅 나의 열조의 묘실 있는 성읍에 보내어 그 성을 중건하게 하옵소서 하였는데
\par 6 그 때에 왕후도 왕의 곁에 앉았더라 왕이 내게 이르시되 네가 몇날에 행할 길이며 어느 때에 돌아오겠느냐 하고 왕이 나를 보내기를 즐겨하시기로 내가 기한을 정하고
\par 7 내가 또 왕에게 아뢰되 왕이 만일 즐겨하시거든 강 서편 총독들에게 내리시는 조서를 내게 주사 저희로 나를 용납하여 유다까지 통과하게 하시고
\par 8 또 왕의 삼림 감독 아삽에게 조서를 내리사 저로 전에 속한 영문의 문과 성곽과 나의 거할 집을 위하여 들보 재목을 주게 하옵소서 하매 내 하나님의 선한 손이 나를 도우심으로 왕이 허락하고
\par 9 군대 장관과 마병을 보내어 나와 함께하게 하시기로 내가 강 서편에 있는 총독들에게 이르러 왕의 조서를 전하였더니
\par 10 호론 사람 산발랏과 종 되었던 암몬 사람 도비야가 이스라엘 자손을 흥왕케 하려는 사람이 왔다 함을 듣고 심히 근심하더라
\par 11 내가 예루살렘에 이르러 거한지 삼일에
\par 12 내 하나님이 내 마음을 감화하사 예루살렘을 위하여 행하게 하신 일을 내가 아무 사람에게도 말하지 아니하고 밤에 일어나 두어 사람과 함께 나갈새 내가 탄 짐승 외에는 다른 짐승이 없더라
\par 13 그 밤에 골짜기 문으로 나가서 용정으로 분문에 이르는 동안에 보니 예루살렘 성벽이 다 무너졌고 성문은 소화되었더라
\par 14 앞으로 행하여 샘문과 왕의 못에 이르러는 탄 짐승이 지나갈 곳이 없는지라
\par 15 그 밤에 시내를 좇아 올라가서 성벽을 살펴 본 후에 돌이켜 골짜기 문으로 들어와서 돌아 왔으나
\par 16 방백들은 내가 어디 갔었으며 무엇을 하였는지 알지 못하였고 나도 그 일을 유다 사람들에게나 제사장들에게나 귀인들에게나 방백들에게나 그 외에 일하는 자들에게 고하지 아니하다가
\par 17 후에 저희에게 이르기를 우리의 당한 곤경은 너희도 목도하는바라 예루살렘이 황무하고 성문이 소화되었으니 자 예루살렘 성을 중건하여 다시 수치를 받지 말자 하고
\par 18 또 저희에게 하나님의 선한 손이 나를 도우신 일과 왕이 내게 이른 말씀을 고하였더니 저희의 말이 일어나 건축하자 하고 모두 힘을 내어 이 선한 일을 하려 하매
\par 19 호론 사람 산발랏과 종이 되었던 암몬 사람 도비야와 아라비아 사람 게셈이 이 말을 듣고 우리를 업신여기고 비웃어 가로되 너희의 하는 일이 무엇이냐 왕을 배반코자 하느냐 하기로
\par 20 내가 대답하여 가로되 하늘의 하나님이 우리로 형통케 하시리니 그의 종 우리가 일어나 건축하려니와 오직 너희는 예루살렘에서 아무 기업도 없고 권리도 없고 명록도 없다 하였느니라

\chapter{3}

\par 1 때에 대제사장 엘리아십이 그 형제 제사장들과 함께 일어나 양문을 건축하여 성별하고 문짝을 달고 또 성벽을 건축하여 함메아 망대에서부터 하나넬 망대까지 성별하였고
\par 2 그 다음은 여리고 사람들이 건축하였고 또 그 다음은 이므리의 아들 삭굴이 건축하였으며
\par 3 어문은 하스나아의 자손들이 건축하여 그 들보를 얹고 문짝을 달고 자물쇠와 빗장을 갖추었고
\par 4 그 다음은 학고스의 손자 우리아의 아들 므레못이 중수하였고 그 다음은 므세사벨의 손자 베레갸의 아들 므술람이 중수하였고 그 다음은 바아나의 아들 사독이 중수하였고
\par 5 그 다음은 드고아 사람들이 중수하였으나 그 귀족들은 그 주의 역사에 담부치 아니하였으며
\par 6 옛 문은 바세아의 아들 요야다와 브소드야의 아들 므술람이 중수하여 그 들보를 얹고 문짝을 달고 자물쇠와 빗장을 갖추었고
\par 7 그 다음은 기브온 사람 믈라댜와 메로놋 사람 야돈이 강 서편 총독의 관할에 속한 기브온 사람들과 미스바 사람들로 더불어 중수하였고
\par 8 그 다음은 금장색 할해야의 아들 웃시엘등이 중수하였고 그 다음은 향품 장사 하나냐등이 중수하되 저희가 예루살렘 넓은 성벽까지 하였고
\par 9 그 다음은 예루살렘 지방 절반을 다스리는 자 후르의 아들 르바야가 중수하였고
\par 10 하루맙의 아들 여다야는 자기 집과 마주 대한 곳을 중수하였고 그 다음은 하삽느야의 아들 핫두스가 중수하였고
\par 11 하림의 아들 말기야와 바핫모압의 아들 핫숩이 한 부분과 풀무 망대를 중수하였고
\par 12 그 다음은 예루살렘 지방 절반을 다스리는 자 할로헤스의 아들 살룸과 그 딸들이 중수하였고
\par 13 골짜기 문은 하눈과 사노아 거민이 중수하여 문을 세우며 문짝을 달고 자물쇠와 빗장을 갖추고 또 분문까지 성벽 일천 규빗을 중수하였고
\par 14 분문은 벧학게렘 지방을 다스리는 레갑의 아들 말기야가 중수하여 문을 세우며 문짝을 달고 자물쇠와 빗장을 갖추었고
\par 15 샘문은 미스바 지방을 다스리는 골호세의 아들 살룬이 중수하여 문을 세우고 덮으며 문짝을 달며 자물쇠와 빗장을 갖추고 또 왕의 동산 근처 셀라 못가의 성벽을 중수하여 다윗성에서 내려오는 층계까지 이르렀고
\par 16 그 다음은 벧술 지방 절반을 다스리는 자 아스북의 아들 느헤미야가 중수하여 다윗의 묘실과 마주 대한 곳에 이르고 또 파서 만든 못을 지나 용사의 집까지 이르렀고
\par 17 그 다음은 레위 사람 바니의 아들 르훔이 중수하였고 그 다음은 그일라 지방 절반을 다스리는 자 하사뱌가 그 지방을 대표하여 중수하였고
\par 18 그 다음은 그 형제 그일라 지방 절반을 다스리는 자 헤나닷의 아들 바왜가 중수하였고
\par 19 그 다음은 미스바를 다스리는 자 예수아의 아들 에셀이 한 부분을 중수하여 성 굽이에 있는 군기고 맞은편까지 이르렀고
\par 20 그 다음은 삽배의 아들 바룩이 한 부분을 힘써 중수하여 성 굽이에서부터 대제사장 엘리아십의 집 문에 이르렀고
\par 21 그 다음은 학고스의 손자 우리야의 아들 므레못이 한 부분을 중수하여 엘리아십의 집 문에서부터 엘리아십의 집 모퉁이에 이르렀고
\par 22 그 다음은 평지에 사는 제사장들이 중수하였고
\par 23 그 다음은 베냐민과 핫숩이 자기 집 맞은편 부분을 중수하였고 그 다음은 아나냐의 손자 마아세야의 아들 아사랴가 자기 집에서 가까운 부분을 중수하였고
\par 24 그 다음은 헤나닷의 아들 빈누이가 한 부분을 중수하되 아사랴의 집에서부터 성 굽이를 지나 성 모퉁이에 이르렀고
\par 25 우새의 아들 발랄은 성 굽이 맞은편과 왕의 윗 궁에서 내어민 망대 맞은편 곧 시위청에서 가까운 부분을 중수하였고 그 다음은 바로스의 아들 브다야가 중수하였고
\par 26 (때에 느디님 사람은 오벨에 거하여 동편 수문과 마주 대한 곳에서부터 내어민 망대까지 미쳤느니라)
\par 27 그 다음은 드고아 사람들이 한 부분을 중수하여 내어민 큰 망대 와 마주 대한 곳에서부터 오벨 성벽까지 이르렀느니라
\par 28 마문 위로부터는 제사장들이 각각 자기 집과 마주 대한 부분을 중수하였고
\par 29 그 다음은 임멜의 아들 사독이 자기 집과 마주 대한 부분을 중수하였고 그 다음은 동문지기 스가냐의 아들 스마야가 중수하였고
\par 30 그 다음은 셀레먀의 아들 하나냐와 살랍의 여섯째 아들 하눈이 한 부분을 중수하였고 그 다음은 베레갸의 아들 므술람이 자기 침방과 마주 대한 부분을 중수하였고
\par 31 그 다음은 금장색 말기야가 함밉갓 문과 마주 대한 부분을 중수하여 느디님 사람과 상고들의 집에서부터 성 모퉁이 누에 이르렀고
\par 32 성 모퉁이 누에서 양문까지는 금장색과 상고들이 중수하였느니라

\chapter{4}

\par 1 산발랏이 우리가 성을 건축한다 함을 듣고 크게 분노하여 유다 사람을 비웃으며
\par 2 자기 형제들과 사마리아 군대 앞에서 말하여 가로되 이 미약한 유다 사람들의 하는 일이 무엇인가 스스로 견고케 하려는가 제사를 드리려는가 하루에 필역하려는가 소화된 돌을 흙무더기에서 다시 일으키려는가 하고
\par 3 암몬 사람 도비야는 곁에 섰다가 가로되 저들의 건축하는 성벽은 여우가 올라가도 곧 무너지리라 하더라
\par 4 우리 하나님이여 들으시옵소서 우리가 업신여김을 당하나이다 원컨대 저희의 욕하는 것으로 자기의 머리에 돌리사 노략거리가 되어 이방에 사로잡히게 하시고
\par 5 주의 앞에서 그 악을 덮어 두지 마옵시며 그 죄를 도말하지 마옵소서 저희가 건축하는 자 앞에서 주의 노를 격동하였음이니이다 하고
\par 6 이에 우리가 성을 건축하여 전부가 연락되고 고가 절반에 미쳤으니 이는 백성이 마음들여 역사하였음이니라
\par 7 산발랏과 도비야와 아라비아 사람들과 암몬 사람들과 아스돗 사람들이 예루살렘 성이 중수되어 그 퇴락한 곳이 수보되어 간다 함을 듣고 심히 분하여
\par 8 다 함께 꾀하기를 예루살렘으로 가서 쳐서 요란하게 하자 하기로
\par 9 우리가 우리 하나님께 기도하며 저희를 인하여 파숫군을 두어 주야로 방비하는데
\par 10 유다 사람들은 이르기를 흙 무더기가 아직도 많거늘 담부하는 자의 힘이 쇠하였으니 우리가 성을 건축하지 못하리라 하고
\par 11 우리의 대적은 이르기를 저희가 알지 못하고 보지 못하는 사이에 우리가 저희 중에 달려 들어가서 살륙하여 역사를 그치게 하리라 하고
\par 12 그 대적의 근처에 거하는 유다 사람들도 그 각처에서 와서 열 번이나 우리에게 고하기를 너희가 우리에게로 와야 하리라 하기로
\par 13 내가 성 뒤 낮고 넓은 곳에 백성으로 그 종족을 따라 칼과 창과 활을 가지고 서게 하고
\par 14 내가 돌아 본 후에 일어나서 귀인들과 민장과 남은 백성에게 고하기를 너희는 저희를 두려워 말고 지극히 크시고 두려우신 주를 기억하고 너희 형제와 자녀와 아내와 집을 위하여 싸우라 하였었느니라
\par 15 우리의 대적이 자기의 뜻을 우리가 알았다 함을 들으니라 하나님이 저희의 꾀를 폐하셨으므로 우리가 다 성에 돌아와서 각각 역사하였는데
\par 16 그 때로부터 내 종자의 절반은 역사하고 절반은 갑옷을 입고 창과 방패와 활을 가졌고 민장은 유다 온 족속의 뒤에 있었으며
\par 17 성을 건축하는 자와 담부하는 자는 다 각각 한 손으로 일을 하며 한 손에는 병기를 잡았는데
\par 18 건축하는 자는 각각 칼을 차고 건축하며 나팔 부는 자는 내 곁에 섰었느니라
\par 19 내가 귀인들과 민장들과 남은 백성에게 이르기를 이 역사는 크고 넓으므로 우리가 성에서 나뉘어 상거가 먼즉
\par 20 너희가 무론 어디서든지 나팔 소리를 듣거든 그리로 모여서 우리에게로 나아오라 우리 하나님이 우리를 위하여 싸우시리라 하였느니라
\par 21 우리가 이같이 역사하는데 무리의 절반은 동틀 때부터 별이 나기까지 창을 잡았었으며
\par 22 그 때에 내가 또 백성에게 고하기를 사람마다 그 종자와 함께 예루살렘 안에서 잘지니 밤에는 우리를 위하여 파수하겠고 낮에는 역사하리라 하고
\par 23 내나 내 형제들이나 종자들이나 나를 좇아 파수하는 사람들이나 다 그 옷을 벗지 아니하였으며 물을 길으러 갈 때에도 기계를 잡았었느니라

\chapter{5}

\par 1 때에 백성이 그 아내와 함께 크게 부르짖어 그 형제 유다 사람을 원망하는데
\par 2 혹은 말하기를 우리와 우리 자녀가 많으니 곡식을 얻어 먹고 살아야 하겠다 하고
\par 3 혹은 말하기를 우리의 밭과 포도원과 집이라도 전당 잡히고 이 흉년을 위하여 곡식을 얻자 하고
\par 4 혹은 말하기를 우리는 밭과 포도원으로 돈을 빚내어 세금을 바쳤도다
\par 5 우리 육체도 우리 형제의 육체와 같고 우리 자녀도 저희 자녀 같거늘 이제 우리 자녀를 종으로 파는도다 우리 딸 중에 벌써 종된 자가 있으나 우리의 밭과 포도원이 이미 남의 것이 되었으니 속량할 힘이 없도다
\par 6 내가 백성의 부르짖음과 이런 말을 듣고 크게 노하여
\par 7 중심에 계획하고 귀인과 민장을 꾸짖어 이르기를 너희가 각기 형제에게 취리를 하는도다 하고 대회를 열고 저희를 쳐서
\par 8 이르기를 우리는 이방인의 손에 팔린 우리 형제 유다 사람들을 우리의 힘을 다하여 속량하였거늘 너희는 너희 형제를 팔고자 하느냐 더구나 우리의 손에 팔리게 하겠느냐 하매 저희가 잠잠하여 말이 없기로
\par 9 내가 또 이르기를 너희의 소위가 좋지 못하도다 우리 대적 이방 사람의 비방을 생각하고 우리 하나님을 경외함에 행할 것이 아니냐
\par 10 나와 내 형제와 종자들도 역시 돈과 곡식을 백성에게 취하여 주나니 우리가 그 이식 받기를 그치자
\par 11 그런즉 너희는 오늘이라도 그 밭과 포도원과 감람원과 집이며 취한바 돈이나 곡식이나 새 포도주나 기름의 백분지 일을 돌려 보내라 하였더니
\par 12 저희가 말하기를 우리가 당신의 말씀대로 행하여 돌려 보내고 아무 것도 요구하지 아니하리이다 하기로 내가 제사장들을 불러 저희에게 그 말대로 행하리라는 맹세를 시키게 하고
\par 13 내가 옷자락을 떨치며 이르기를 이 말대로 행치 아니하는 자는 하나님이 또한 이와 같이 그 집과 산업에서 떨치실지니 저는 곧 이렇게 떨쳐져 빌지로다 하매 회중이 다 아멘 하고 여호와를 찬송하고 백성들이 그 말한대로 행하였느니라
\par 14 내가 유다 땅 총독으로 세움을 받은 때 곧 아닥사스다왕 이십년 부터 삼십 이년까지 십 이년 동안은 나와 내 형제가 총독의 녹을 먹지 아니하였느니라
\par 15 이전 총독들은 백성에게 토색하여 양식과 포도주와 또 은 사십 세겔을 취하였고 그 종자들도 백성을 압제하였으나 나는 하나님을 경외하므로 이같이 행치 아니하고
\par 16 도리어 이 성 역사에 힘을 다하며 땅을 사지 아니하였고 나의 모든 종자도 모여서 역사를 하였으며
\par 17 또 내 상에는 유다 사람들과 민장들 일백 오십인이 있고 그 외에도 우리 사면 이방인 중에서 우리에게 나아온 자들이 있었는데
\par 18 매일 나를 위하여 소 하나와 살진 양 여섯을 준비하며 닭도 많이 준비하고 열흘에 한번씩은 각종 포도주를 갖추었나니 비록 이같이 하였을지라도 내가 총독의 녹을 요구하지 아니하였음은 백성의 부역이 중함이니라
\par 19 내 하나님이여 내가 이 백성을 위하여 행한 모든 일을 생각하시고 내게 은혜를 베푸시옵소서

\chapter{6}

\par 1 산발랏과 도비야와 아라비아 사람 게셈과 그 나머지 우리의 대적이 내가 성을 건축하여 그 퇴락한 곳을 남기지 아니하였다 함을 들었는데 내가 아직 성문에 문짝을 달지 못한 때라
\par 2 산발랏과 게셈이 내게 보내어 이르기를 오라 우리가 오노 평지 한 촌에서 서로 만나자 하니 실상은 나를 해코자 함이라
\par 3 내가 곧 저희에게 사자들을 보내어 이르기를 내가 이제 큰 역사를 하니 내려가지 못하겠노라 어찌하여 역사를 떠나 정지하게 하고 너희에게로 내려 가겠느냐 하매
\par 4 저희가 네번이나 이같이 내게 보내되 나는 여전히 대답하였더니
\par 5 산발랏이 다섯번째는 그 종자의 손에 봉하지 않은 편지를 들려 내게 보내었는데
\par 6 그 글에 이르기를 이방 중에도 소문이 있고 가스무도 말하기를 네가 유다 사람들로 더불어 모반하려 하여 성을 건축한다 하나니 네가 그 말과 같이 왕이 되려 하는도다
\par 7 또 네가 선지자를 세워 예루살렘에서 너를 들어 선전하기를 유다에 왕이 있다 하게 하였으니 이 말이 왕에게 들릴지라 그런즉 너는 이제 오라 함께 의논하자 하였기로
\par 8 내가 보내어 저에게 이르기를 너의 말한바 이런 일은 없는 일이요 네 마음에서 지어낸 것이라 하였나니
\par 9 이는 저희가 다 우리를 두렵게 하고자 하여 말하기를 저희 손이 피곤하여 역사를 정지하고 이루지 못하리라 함이라 이제 내 손을 힘있게 하옵소서 하였노라
\par 10 이 후에 므헤다벨의 손자 들라야의 아들 스마야가 두문불출하기로 내가 그 집에 가니 저가 이르기를 저희가 너를 죽이러 올터이니 우리가 하나님의 전으로 가서 외소 안에 있고 그 문을 닫자 저희가 필연 밤에 와서 너를 죽이리라 하기로
\par 11 내가 이르기를 나 같은 자가 어찌 도망하며 나 같은 몸이면 누가 외소에 들어가서 생명을 보존하겠느냐 나는 들어가지 않겠노라 하고
\par 12 깨달은즉 저는 하나님의 보내신바가 아니라 도비야와 산발랏에게 뇌물을 받고 내게 이런 에언을 함이라
\par 13 저희가 뇌물을 준 까닭은 나를 두렵게 하고 이렇게 함으로 범죄하게 하고 악한 말을 지어 나를 비방하려 함이었느니라
\par 14 내 하나님이여 도비야와 산발랏과 여선지 노아댜와 그 남은 선지자들 무릇 나를 두렵게 하고자 한 자의 소위를 기억하옵소서 하였노라
\par 15 성 역사가 오십 이일만에 엘룰월 이십 오일에 끝나매
\par 16 우리 모든 대적과 사면 이방 사람들이 이를 듣고 다 두려워하여 스스로 낙담하였으니 이는 이 역사를 우리 하나님이 이루신 것을 앎이니라
\par 17 그 때에 유다의 귀인들이 여러번 도비야에게 편지하였고 도비야의 편지도 저희에게 이르렀으니
\par 18 도비야는 아라의 아들 스가냐의 사위가 되었고 도비야의 아들 여호하난도 베레갸의 아들 므술람의 딸을 취하였으므로 유다에서 저와 동맹한 자가 많음이라
\par 19 저희들이 도비야의 선행을 내 앞에 말하고 또 나의 말도 저에게 전하매 도비야가 항상 내게 편지하여 나를 두렵게 하고자 하였느니라

\chapter{7}

\par 1 성이 건축되매 문짝을 달고 문지기와 노래하는 자들과 레위 사람들을 세운 후에
\par 2 내 아우 하나니와 영문의 관원 하나냐로 함께 예루살렘을 다스리게 하였는데 하나냐는 위인이 충성되어 하나님을 경외함이 무리에서 뛰어난 자라
\par 3 내가 저희에게 이르기를 해가 높이 뜨기 전에는 예루살렘 성문을 열지 말고 아직 파수할 때에 곧 문을 닫고 빗장을 지르며 또 예루살렘 거민으로 각각 반차를 따라 파수하되 자기 집 맞은편을 지키게 하라 하였노니
\par 4 그 성은 광대하고 거민은 희소하여 가옥을 오히려 건축하지 못하였음이니라
\par 5 내 하나님이 내 마음을 감동하사 귀인들과 민장과 백성을 모아 그 보계대로 계수하게 하신고로 내가 처음으로 돌아온 자의 보계를 얻었는데 거기 기록한 것을 보면
\par 6 옛적에 바벨론왕 느부갓네살에게 사로잡혀 갔던 자 중에서 놓임을 받고 예루살렘과 유다로 돌아와 각기 본성에 이른 자 곧
\par 7 스룹바벨과 예수아와 느헤미야와 아사랴와 라아먀와 나하마니와 모르드개와 빌산과 미스베렛과 비그왜와 느훔과 바아나등과 함께 나온 이스라엘 백성의 명수가 이러하니라
\par 8 바로스 자손이 이천 일백 칠십 이명이요
\par 9 스바댜 자손이 삼백 칠십 이명이요
\par 10 아라 자손이 육백 오십 이명이요
\par 11 바핫모압 자손 곧 예수아와 요압 자손이 이천 팔백 십 팔명이요
\par 12 엘람 자손이 일천 이백 오십 사명이요
\par 13 삿두 자손이 팔백 사십 오명이요
\par 14 삭개 자손이 칠백 육십명이요
\par 15 빈누이 자손이 육백 사십 팔명이요
\par 16 브배 자손이 육백 이십 팔명이요
\par 17 아스갓 자손이 이천 삼백 이십 이명이요
\par 18 아도니감 자손이 육백 륙십 칠명이요
\par 19 비그왜 자손이 이천 육십 칠명이요
\par 20 아딘 자손이 육백 오십 오명이요
\par 21 아델 자손 곧 히스기야 자손이 구십 팔명이요
\par 22 하숨 자손이 삼백 이십 팔명이요
\par 23 베새 자손이 삼백 이십 사명이요
\par 24 하립 자손이 일백 십 이명이요
\par 25 기브온 사람이 구십 오명이요
\par 26 베들레헴과 느도바 사람이 일백 팔십 팔명이요
\par 27 아나돗 사람이 일백 이십 팔명이요
\par 28 벧아스마웹 사람이 사십 이명이요
\par 29 기럇여아림과 그비라와 브에롯 사람이 칠백 사십 삼명이요
\par 30 라마와 게바 사람이 육백 이십 일명이요
\par 31 믹마스 사람이 일백 이십 이명이요
\par 32 벧엘과 아이 사람이 일백 이십 삼명이요
\par 33 기타 느보 사람이 오십 이명이요
\par 34 기타 엘람 자손이 일천 이백 오십 사명이요
\par 35 하림 자손이 삼백 이십명이요
\par 36 여리고 자손이 삼백 사십 오명이요
\par 37 로드와 하딧과 오노 자손이 칠백 이십 일명이요
\par 38 스나아 자손이 삼천 구백 삼십명이었느니라
\par 39 제사장들은 예수아의 집 여다야 자손이 구백 칠십 삼명이요
\par 40 임멜 자손이 일천 오십 이명이요
\par 41 바수훌 자손이 일천 이백 사십 칠명이요
\par 42 하림 자손이 일천 십 칠명이였느니라
\par 43 레위 사람들은 호드야 자손 곧 예수아와 갓미엘 자손이 칠십 사명이요
\par 44 노래하는 자들은 아삽 자손이 일백 사십 팔명이요
\par 45 문지기들은 살룸 자손과 아델 자손과 달문 자손과 악굽 자손과 하디다 자손과 소배 자손이 모두 일백 삼십 팔명이었느니라
\par 46 느디님 사람들은 시하 자손과 하수바 자손과 답바옷 자손과
\par 47 게로스 자손과 시아 자손과 바돈 자손과 르바나 자손과
\par 48 하가바 자손과 살매 자손과
\par 49 하난 자손과 깃델 자손과 가할 자손과
\par 50 르아야 자손과 르신 자손과 느고다 자손과
\par 51 갓삼 자손과 웃사 자손과 바세아 자손과
\par 52 베새 자손과 므우님 자손과 느비스심 자손과
\par 53 박북 자손과 하그바 자손과 할훌 자손과
\par 54 바슬릿 자손과 므히다 자손과 하르사 자손과
\par 55 바르고스 자손과 시스라 자손과 데마 자손과
\par 56 느시야 자손과 하디바 자손이었느니라
\par 57 솔로몬의 신복의 자손은 소대 자손과 소베렛 자손과 브리다 자손과
\par 58 야알라 자손과 다르곤 자손과 깃델 자손과
\par 59 스바댜 자손과 핫딜 자손과 보게렛하스바임 자손과 아몬 자손이니
\par 60 모든 느디님 사람과 솔로몬의 신복의 자손이 삼백 구십 이명이었느니라
\par 61 델멜라와 델하르사와 그룹과 앗돈과 임멜로부터 올라온 자가 있으나 그 종족과 보계가 이스라엘에 속하였는지는 증거할 수 없으니
\par 62 저희는 들라야 자손과 도비야 자손과 느고다 자손이라 도합이 육백 사십 이명이요
\par 63 제사장 중에는 호바야 자손과 학고스 자손과 바르실래 자손이니 바르실래는 길르앗 사람 바르실래의 딸 중에 하나로 아내를 삼고 바르실래의 이름으로 이름한 자라
\par 64 이 사람들이 보계 중에서 자기 이름을 찾아도 얻지 못한고로 저희를 부정하게 여겨 제사장의 직분을 행치 못하게 하고
\par 65 방백이 저희에게 명하여 우림과 둠밈을 가진 제사장이 일어나기 전에는 지성물을 먹지 말라 하였느니라
\par 66 온 회중의 합계가 사만 이천 삼백 육십명이요
\par 67 그 외에 노비가 칠천 삼백 삼십 칠명이요 노래하는 남녀가 이백 사십 오명이요
\par 68 말이 칠백 삼십 륙이요 노새가 이백 사십 오요
\par 69 약대가 사백 삼십 오요 나귀가 육천 칠백 이십이었느니라
\par 70 어떤 족장들은 역사를 위하여 보조하였고 방백은 금 일천 다릭과 대접 오십과 제사장의 의복 오백 삼십 벌을 보물 곳간에 드렸고
\par 71 또 어떤 족장들은 금 이만 다릭과 은 이천 이백 마네를 역사 곳간에 드렸고
\par 72 그 나머지 백성은 금 이만 다릭과 은 이천 마네와 제사장의 의복 육십 칠벌을 드렸느니라
\par 73 이와 같이 제사장들과 레위 사람들과 문지기들과 노래하는 자들과 백성 몇명과 느디님 사람들과 온 이스라엘이 다 그 본성에 거하였느니라

\chapter{8}

\par 1 이스라엘 자손이 그 본성에 거하였더니 칠월에 이르러는 모든 백성이 일제히 수문 앞 광장에 모여 학사 에스라에게 여호와께서 이스라엘에게 명하신 모세의 율법책을 가지고 오기를 청하매
\par 2 칠월 일일에 제사장 에스라가 율법책을 가지고 남자 여자 무릇 알아 들을만한 회중 앞에 이르러
\par 3 "수문 앞 광장에서 새벽부터 오정까지 남자, 여자 무릇 알아 들을만한 자의 앞에서 읽으매 뭇백성이 그 율법책에 귀를 기울였는데"
\par 4 때에 학사 에스라가 특별히 지은 나무 강단에 서매 그 우편에 선 자는 맛디댜와 스마와 아나야와 우리야와 힐기야와 마아세야요 그 좌편에 선 자는 브다야와 미사엘과 말기야와 하숨과 하스밧다나와 스가랴와 므술람이라
\par 5 학사 에스라가 모든 백성 위에 서서 저희 목전에 책을 펴니 책을 펼 때에 모든 백성이 일어서니라
\par 6 에스라가 광대하신 하나님 여호와를 송축하매 모든 백성이 손을 들고 아멘 아멘 응답하고 몸을 굽혀 얼굴을 땅에 대고 여호와께 경배하였느니라
\par 7 예수아와 바니와 세레뱌와 야민과 악굽과 사브대와 호디야와 마아세야와 그리다와 아사랴와 요사밧과 하난과 블라야와 레위 사람들이 다 그 처소에 섰는 백성에게 율법을 깨닫게 하는데
\par 8 하나님의 율법책을 낭독하고 그 뜻을 해석하여 백성으로 그 낭독하는 것을 다 깨닫게 하매
\par 9 백성이 율법의 말씀을 듣고 다 우는지라 총독 느헤미야와 제사장겸 학사 에스라와 백성을 가르치는 레위 사람들이 모든 백성에게 이르기를 오늘은 너희 하나님 여호와의 성일이니 슬퍼하지 말며 울지말라 하고
\par 10 느헤미야가 또 이르기를 너희는 가서 살진 것을 먹고 단 것을 마시되 예비치 못한 자에게는 너희가 나누어 주라 이 날은 우리 주의 성일이니 근심하지 말라 여호와를 기뻐하는 것이 너희의 힘이니라 하고
\par 11 레위 사람들도 모든 백성을 정숙케 하여 이르기를 오늘은 성일이니 마땅히 종용하고 근심하지 말라 하매
\par 12 모든 백성이 곧 가서 먹고 마시며 나누어 주고 크게 즐거워하였으니 이는 그 읽어 들린 말을 밝히 앎이니라
\par 13 그 이튿날 뭇백성의 족장들과 제사장들과 레위 사람들이 율법의 말씀을 밝히 알고자 하여 학사 에스라의 곳에 모여서
\par 14 율법책을 본즉 여호와께서 모세로 명하시기를 이스라엘 자손은 칠월 절기에 초막에 거할지니라 하였고
\par 15 또 일렀으되 모든 성읍과 예루살렘에 공포하여 이르기를 너희는 산에 가서 감람나무 가지와 들 감람나무 가지와 화석류나무 가지와 종려나무 가지와 기타 무성한 나무 가지를 취하여 기록한바를 따라 초막을 지으라 하라 하였는지라
\par 16 백성이 이에 나가서 나무 가지를 취하여 혹은 지붕 위에 혹은 뜰 안에 혹은 하나님의 전 뜰에 혹은 수문 광장에 혹은 에브라임 문광장에 초막을 짓되
\par 17 사로잡혔다가 돌아온 회 무리가 다 초막을 짓고 그 안에 거하니 눈의 아들 여호수아 때로부터 그 날까지 이스라엘 자손이 이같이 행함이 없었으므로 이에 크게 즐거워하며
\par 18 에스라는 첫날부터 끝날까지 날마다 하나님의 율법책을 낭독하고 무리가 칠일 동안 절기를 지키고 제 팔일에 규례를 따라 성회를 열었느니라

\chapter{9}

\par 1 그 달 이십 사일에 이스라엘 자손이 다 모여 금식하며 굵은 베를 입고 티끌을 무릅쓰며
\par 2 모든 이방 사람과 절교하고 서서 자기의 죄와 열조의 허물을 자복하고
\par 3 이 날에 낮 사분지 일은 그 처소에 서서 그 하나님 여호와의 율법책을 낭독하고 낮 사분지 일은 죄를 자복하며 그 하나님 여호와께 경배하는데
\par 4 레위 사람 예수아와 바니와 갓미엘과 스바냐와 분니와 세레뱌와 바니와 그나니는 대에 올라서서 큰 소리로 그 하나님 여호와께 부르짖고
\par 5 또 레위 사람 예수아와 갓미엘과 바니와 하삽느야와 세레뱌와 호다야와 스바냐와 브다히야는 이르기를 너희 무리는 마땅히 일어나 영원부터 영원까지 계신 너희 하나님 여호와를 송축할지어다 주여 주의 영화로운 이름을 송축하올 것은 주의 이름이 존귀하여 모든 송축이나 찬양에서 뛰어남이니이다
\par 6 오직 주는 여호와시라 하늘과 하늘들의 하늘과 일월 성신과 땅과 땅 위의 만물과 바다와 그 가운데 모든 것을 지으시고 다 보존하시오니 모든 천군이 주께 경배하나이다
\par 7 주는 하나님 여호와시라 옛적에 아브람을 택하시고 갈대아 우르에서 인도하여 내시고 아브라함이라는 이름을 주시고
\par 8 그 마음이 주 앞에서 충성됨을 보시고 더불어 언약을 세우사 가나안 족속과 헷 족속과 아모리 족속과 브리스 족속과 여부스 족속과 기르가스 족속의 땅을 그 씨에게 주리라 하시더니 그 말씀대로 이루셨사오니 주는 의로우심이로소이다
\par 9 주께서 우리 열조가 애굽에서 고난 받는 것을 감찰하시며 홍해에서 부르짖음을 들으시고
\par 10 이적과 기사를 베푸사 바로와 그 모든 신하와 그 나라 온 백성을 치셨사오니 이는 저희가 우리의 열조에게 교만히 행함을 아셨음이라 오늘날과 같이 명예를 얻으셨나이다
\par 11 주께서 또 우리 열조 앞에서 바다를 갈라지게 하시사 저희로 바다 가운데를 육지 같이 통과하게 하시고 쫓아 오는 자를 돌을 큰 물에 던짐 같이 깊은 물에 던지시고
\par 12 낮에는 구름 기둥으로 인도하시고 밤에는 불 기둥으로 그 행할 길을 비취셨사오며
\par 13 또 시내 산에 강림하시고 하늘에서부터 저희와 말씀하사 정직한 규례와 진정한 율법과 선한 율례와 계명을 저희에게 주시고
\par 14 거룩한 안식일을 저희에게 알리시며 주의 종 모세로 계명과 율례와 율법을 저희에게 명하시고
\par 15 저희의 주림을 인하여 하늘에서 양식을 주시며 저희의 목마름을 인하여 반석에서 물을 내시고 또 주께서 옛적에 손을 들어 맹세하시고 주마 하신 땅을 들어가서 차지하라 명하셨사오나
\par 16 저희와 우리 열조가 교만히 하고 목을 굳게 하여 주의 명령을 듣지 아니하고
\par 17 거역하며 주께서 저희 가운데 행하신 기사를 생각지 아니하고 목을 굳게하며 패역하여 스스로 한 두목을 세우고 종 되었던 땅으로 돌아가고자 하였사오나 오직 주는 사유하시는 하나님이시라 은혜로우시며 긍휼히 여기시며 더디 노하시며 인자가 풍부하시므로 저희를 버리지 아니하셨나이다
\par 18 또 저희가 송아지를 부어 만들고 이르기를 이는 곧 너희를 인도하여 애굽에서 나오게 하신 하나님이라 하여 크게 설만하게 하였사오나
\par 19 주께서는 연하여 긍휼을 베푸사 저희를 광야에 버리지 아니하시고 낮에는 구름 기둥으로 길을 인도하시며 밤에는 불 기둥으로 그 행할 길을 비취사 떠나게 아니하셨사오며
\par 20 또 주의 선한 신을 주사 저희를 가르치시며 주의 만나로 저희 입에 끊어지지 않게 하시고 저희의 목마름을 인하여 물을 주시사
\par 21 사십년 동안을 들에서 기르시되 결핍함이 없게 하시므로 그 옷이 해어지지 아니하였고 발이 부릍지 아니하였사오며
\par 22 또 나라들과 족속들을 저희에게 각각 나누어 주시매 저희가 시혼의 땅 곧 헤스본 왕의 땅과 바산 왕 옥의 땅을 차지하였나이다
\par 23 주께서 그 자손을 하늘의 별같이 많게 하시고 전에 그 열조에게 명하사 들어가서 차지하라고 하신 땅으로 인도하여 이르게 하셨으므로
\par 24 그 자손이 들어가서 땅을 차지하되 주께서 그 땅 가나안 거민으로 저희 앞에 복종케 하실 때에 가나안 사람과 그 왕들과 본토 여러 족속을 저희 손에 붙여 임의로 행하게 하시매
\par 25 저희가 견고한 성들과 기름진 땅을 취하고 모든 아름다운 물건을 채운 집과 파서 만든 우물과 포도원과 감람원과 허다한 과목을 차지하여 배불리 먹어 살지고 주의 큰 복을 즐겼사오나
\par 26 저희가 오히려 순종치 아니하고 주를 거역하며 주의 율법을 등뒤에 두고 주께로 돌아오기를 권면하는 선지자들을 죽여 크게 설만하게 행하였나이다
\par 27 그러므로 주께서 그 대적의 손에 붙이사 곤고를 당하게 하시매 저희가 환난을 당하여 주께 부르짖을 때에 주께서 하늘에서 들으시고 크게 긍휼을 발하사 구원자들을 주어 대적의 손에서 구원하셨거늘
\par 28 저희가 평강을 얻은 후에 다시 주 앞에서 악을 행하므로 주께서 그 대적의 손에 버려 두사 대적에게 제어를 받게 하시다가 저희가 돌이켜서 주께 부르짖으매 주께서 하늘에서 들으시고 여러번 긍휼을 발하사 건져내시고
\par 29 다시 주의 율법을 복종하게 하시려고 경계하셨으나 저희가 교만히 행하여 사람이 준행하면 그 가운데서 삶을 얻는 주의 계명을 듣지 아니하며 주의 규례를 범하여 고집하는 어깨를 내어밀며 목을 굳게 하여 듣지 아니하였나이다
\par 30 그러나 주께서 여러 해 동안 용서하시고 또 선지자로 말미암아 주의 신으로 저희를 경계하시되 저희가 듣지 아니하므로 열방 사람의 손에 붙이시고도
\par 31 주의 긍휼이 크시므로 저희를 아주 멸하지 아니하시며 버리지도 아니하셨사오니 주는 은혜로우시고 긍휼히 여기시는 하나님이심이니이다
\par 32 우리 하나님이여 광대하시고 능하시고 두려우시며 언약과 인자를 지키시는 하나님이여 우리와 우리 열왕과 방백들과 제사장들과 선지자들과 열조와 주의 모든 백성이 앗수르 열왕의 때로부터 오늘날까지 당한바 환난을 이제 작게 여기시지 마옵소서
\par 33 그러나 우리의 당한 모든 일에 주는 공의로우시니 우리는 악을 행하였사오나 주는 진실히 행하셨음이니이다
\par 34 우리 열왕과 방백들과 제사장들과 열조가 주의 율법을 지키지 아니하며 주의 명령과 주의 경계하신 말씀을 순종치 아니하고
\par 35 저희가 그 나라와 주의 베푸신 큰 복과 자기 앞에 주신 넓고 기름진 땅을 누리면서도 주를 섬기지 아니하며 악행을 그치지 아니한고로
\par 36 우리가 오늘날 종이 되었삽는데 곧 주께서 우리 열조에게 주사 그 실과를 먹고 그 아름다운 소산을 누리게 하신 땅에서 종이 되었나이다
\par 37 우리의 죄로 인하여 주께서 우리 위에 세우신 이방 열왕이 이 땅의 많은 소산을 얻고 저희가 우리의 몸과 육축을 임의로 관할하오니 우리의 곤난이 심하오며
\par 38 우리가 이 모든 일을 인하여 이제 견고한 언약을 세워 기록하고 우리의 방백들과 레위 사람들과 제사장들이 다 인을 치나이다 하였느니라

\chapter{10}

\par 1 그 인친 자는 하가랴의 아들 방백 느헤미야와 시드기야
\par 2 "스라야, 아사랴, 예레미야,"
\par 3 "바스훌, 아마랴, 말기야,"
\par 4 "핫두스, 스바냐, 말룩,"
\par 5 "하림, 므레못, 오바댜,"
\par 6 "다니엘, 긴느돈, 바룩,"
\par 7 "므술람, 아비야, 미야민,"
\par 8 "마아시야, 빌개, 스마야니 이는 다 제사장이요"
\par 9 "또 레위 사람 곧 아사냐의 아들 예수아, 헤나닷의 자손 중 빈누이, 갓미엘과"
\par 10 "그 형제 스바냐, 호디야, 그리다, 블라야, 하난"
\par 11 "미가, 르홉, 하사뱌, "
\par 12 "삭굴, 세레뱌, 스바냐, "
\par 13 "호디야, 바니, 브니누요"
\par 14 "또 백성의 두목들 곧 바로스, 바핫모압, 엘람, 삿두, 바니, "
\par 15 "분니, 아스갓, 베배, "
\par 16 "아도니야, 비그왜, 아딘, "
\par 17 "아델, 히스기야, 앗술, "
\par 18 "호디야, 하숨, 베새, "
\par 19 "하립, 아나돗, 노배, "
\par 20 "막비아스, 므술람, 헤실, "
\par 21 "므세사벨, 사독, 얏두아, "
\par 22 "블라댜, 하난, 아나야, "
\par 23 "호세아, 하나냐, 핫숩, "
\par 24 "할르헤스, 빌하, 소벡, "
\par 25 "르훔, 하삽나, 마아세야, "
\par 26 "아히야, 하난, 아난, "
\par 27 "말룩, 하림, 바아나이었느니라"
\par 28 그 남은 백성과 제사장들과 레위 사람들과 문지기들과 노래하는 자들과 느디님 사람들과 및 이방 사람과 절교하고 하나님의 율법을 준행하는 모든 자와 그 아내와 그 자녀들 무릇 지식과 총명이 있는 자가
\par 29 다 그 형제 귀인들을 좇아 저주로 맹세하기를 우리가 하나님의 종 모세로 주신 하나님의 율법을 좇아 우리 주 여호와의 모든 계명과 규례와 율례를 지켜
\par 30 우리 딸은 이 땅 백성에게 주지 아니하고 우리 아들을 위하여 저희 딸을 데려오지 아니하며
\par 31 혹시 이 땅 백성이 안식일에 물화나 식물을 가져다가 팔려 할지라도 우리가 안식일이나 성일에는 사지 않겠고 제 칠년마다 땅을 쉬게 하고 모든 빚을 탕감하리라 하였고
\par 32 우리가 또 스스로 규례를 정하기를 해마다 각기 세겔의 삼분 일을 수납하여 하나님의 전을 위하여 쓰게 하되
\par 33 곧 진설병과 항상 드리는 소제와 항상 드리는 번제와 안식일과 초하루와 정한 절기에 쓸 것과 성물과 이스라엘을 위하는 속죄제와 우리 하나님의 전의 모든 일을 위하여 쓰게 하였고
\par 34 또 우리 제사장들과 레위 사람들과 백성들이 제비 뽑아 각기 종족대로 해마다 정한 기한에 나무를 우리 하나님의 전에 드려서 율법에 기록한대로 우리 하나님 여호와의 단에 사르게 하였고
\par 35 해마다 우리 토지 소산의 맏물과 각종 과목의 첫 열매를 여호와의 전에 드리기로 하였고
\par 36 또 우리의 맏아들들과 생축의 처음 난 것과 우양의 처음 난 것을 율법에 기록된대로 우리 하나님의 전으로 가져다가 우리 하나님의 전에서 섬기는 제사장들에게 주고
\par 37 또 처음 익은 밀의 가루와 거제물과 각종 과목의 열매와 새 포도주와 기름을 제사장들에게로 가져다가 우리 하나님의 전 골방에 두고 또 우리 물산의 십일조를 레위 사람들에게 주리라 하였나니 이 레위 사람들은 우리의 모든 성읍에서 물산의 십일조를 받는 자임이며
\par 38 레위 사람들이 십일조를 받을 때에는 아론의 자손 제사장 하나가 함께 있을 것이요 레위 사람들은 그 십일조의 십분 일을 가져다가 우리 하나님의 전 골방 곧 곳간에 두되
\par 39 곧 이스라엘 자손과 레위 자손이 거제로 드린바 곡식과 새 포도주와 기름을 가져다가 성소의 기명을 두는 골방 곧 섬기는 제사장들과 및 문지기들과 노래하는 자들이 있는 골방에 둘 것이라 그리하여 우리가 우리 하나님의 전을 버리지 아니하리라

\chapter{11}

\par 1 백성의 두목들은 예루살렘에 머물렀고 그 남은 백성은 제비 뽑아 십분의 일은 거룩한 성 예루살렘에 와서 거하게 하고 그 구분은 다른 성읍에 거하게 하였으며
\par 2 무릇 예루살렘에 거하기를 자원하는 자는 백성들이 위하여 복을 빌었느니라
\par 3 이스라엘과 제사장들과 레위 사람들과 느디님 사람들과 솔로몬의 신복의 자손은 유다 여러 성읍에서 각각 그 본성 본 기업에 거하였고 예루살렘에 거한 그 도의 두목들은 이러하니
\par 4 예루살렘에 거한 자는 유다 자손과 베냐민 자손 몇명이라 유다 자손 중에는 베레스 자손 아다야니 저는 웃시야의 아들이요 스가랴의 손자요 아마랴의 증손이요 스바댜의 현손이요 마할랄렐의 오대손이며
\par 5 또 마아세야니 저는 바룩의 아들이요 골호세의 손자요 하사야의 증손이요 아다야의 현손이요 요야립의 오대손이요 스가랴의 육대손이요 실로 사람의 칠대손이라
\par 6 예루살렘에 거한 베레스 자손의 도합이 사백 육십 팔명이니 다 용사이었느니라
\par 7 베냐민 자손은 살루니 저는 므술람의 아들이요 요엣의 손자요 브다야의 증손이요 골라야의 현손이요 마아세야의 오대손이요 이디엘의 육대손이요 여사야의 칠대손이며
\par 8 그 다음은 갑배와 살래 등이니 도합이 구백 이십 팔명이라
\par 9 시그리의 아들 요엘이 그 감독이 되었고 핫스누아의 아들 유다는 버금이 되어 성읍을 다스렸느니라
\par 10 제사장 중에는 요야립의 아들 여다야와 야긴이며
\par 11 또 하나님의 전을 맡은 자 스라야니 저는 힐기야의 아들이요 므술람의 손자요 사독의 증손이요 므라욧의 현손이요 아히둡의 오대손이며
\par 12 또 전에서 일하는 그 형제니 도합이 팔백 이십 이명이요 또 아다야니 저는 여로함의 아들이요 블라야의 손자요 암시의 증손이요 스가랴의 현손이요 바스훌의 오대손이요 말기야의 육대손이며
\par 13 또 그 형제의 족장된 자니 도합이 이백 사십 이명이요 또 아맛새니 저는 아사렐의 아들이요 아흐새의 손자요 므실레못의 증손이요 임멜의 현손이며
\par 14 또 그 형제의 큰 용사니 도합이 일백 이십 팔명이라 하그돌림의 아들 삽디엘이 그 감독이 되었느니라
\par 15 레위 사람 중에는 스마야니 저는 핫숩의 아들이요 아스리감의 손자요 하사뱌의 증손이요 분니의 현손이며
\par 16 또 레위 사람의 족장 삽브대와 요사밧이니 저희는 하나님의 전 바깥 일을 맡았고
\par 17 또 아삽의 증손 삽디의 손자 미가의 아들 맛다냐니 저는 기도할 때에 감사하는 말씀을 인도하는 어른이 되었고 형제 중에 박부갸가 버금이 되었으며 또 여두둔의 증손 갈랄의 손자 삼무아의 아들 압다니
\par 18 거룩한 성에 레위 사람의 도합이 이백 팔십 사명이었느니라
\par 19 성 문지기는 악굽과 달몬과 그 형제니 도합이 일백 칠십 이명이며
\par 20 그 나머지 이스라엘 백성과 제사장과 레위 사람은 유다 모든 성읍에 흩어져 각각 자기 기업에 거하였고
\par 21 느디님 사람은 오벨에 거하니 시하와 기스바가 그 두목이 되었느니라
\par 22 노래하는 자 아삽 자손 곧 미가의 현손 맛다냐의 증손 하사뱌의 손자 바니의 아들 웃시는 예루살렘에 거하는 레위 사람의 감독이 되어 하나님의 전 일을 맡아 다스렸으니
\par 23 이는 왕의 명대로 노래하는 자에게 날마다 양식을 정하여 주는 것이 있음이며
\par 24 유다의 아들 세라의 자손 곧 므세사벨의 아들 브다히야는 왕의 수하에서 백성의 일을 다스렸느니라
\par 25 향리와 들로 말하면 유다 자손의 더러는 기럇 아바와 그 촌과 디본과 그 촌과 여갑스엘과 그 동네에 거하며
\par 26 또 예수아와 몰라다와 벧벨렛과
\par 27 하살수알과 브엘세바와 그 촌에 거하며
\par 28 또 시글락과 므고나와 그 촌에 거하며
\par 29 또 에느림몬과 소라와 야르뭇에 거하며
\par 30 또 사노아와 아둘람과 그 동네와 라기스와 그 들과 아세가와 그 촌에 거하였으니 저희는 브엘세바에서부터 힌놈의 골짜기까지 장막을 쳤으며
\par 31 또 베냐민 자손은 게바에서부터 믹마스와 아야와 벧엘과 그 촌에 거하며
\par 32 아나돗과 놉과 아나냐와
\par 33 하솔과 라마와 깃다임과
\par 34 하딧과 스보임과 느발랏과
\par 35 로드와 오노와 공장 골짜기에 거하였으며
\par 36 유다에 있던 레위 사람의 어떤 반열은 베냐민과 합하였느니라

\chapter{12}

\par 1 스알디엘의 아들 스룹바벨과 및 예수아를 좇아 돌아온 제사장과 레위 사람은 이러하니라 제사장은 스라야와 예레미야와 에스라와
\par 2 아마랴와 말룩과 핫두스와
\par 3 스가냐와 르훔과 므레못과
\par 4 잇도와 긴느도이와 아비야와
\par 5 미야민과 마아댜와 빌가와
\par 6 스마야와 요야립과 여다야와
\par 7 살루와 아목과 힐기야와 여다야니 이상은 예수아 때에 제사장과 그 형제의 어른이었느니라
\par 8 레위 사람은 예수아와 빈누이와 갓미엘과 세레뱌와 유다와 맛다냐니 이 맛다냐는 그 형제와 함께 찬송하는 일을 맡았고
\par 9 또 그 형제 박부갸와 운노는 직무를 따라 저의 맞은편에 있으며
\par 10 예수아는 요야김을 낳았고 요야김은 엘리아십을 낳았고 엘리아십은 요야다를 낳았고
\par 11 요야다는 요나단을 낳았고 요나단은 얏두아를 낳았느니라
\par 12 요야김 때에 제사장의 족장 된 자는 스라야 족속에는 므라야요 예레미야 족속에는 하나냐요
\par 13 에스라 족속에는 므술람이요 아마랴 족속에는 여호하난이요
\par 14 말루기 족속에는 요나단이요 스바냐 족속에는 요셉이요
\par 15 하림 족속에는 아드나요 므라욧 족속에는 헬개요
\par 16 잇도 족속에는 스가랴요 긴느돈 족속에는 므술람이요
\par 17 아비야 족속에는 시그리요 미냐민 곧 모아댜 족속에는 빌대요
\par 18 빌가 족속에는 삼무아요 스마야 족속에는 여호나단이요
\par 19 요야립 족속에는 맛드내요 여다야 족속에는 웃시요
\par 20 살래 족속에는 갈래요 아목 족속에는 에벨이요
\par 21 힐기야 족속에는 하사뱌요 여다야 족속에는 느다넬이었느니라
\par 22 엘리아십과 요야다와 요하난과 얏두아 때에 레위 사람의 족장이 모두 책에 기록되었고 바사 왕 다리오때에 제사장도 책에 기록되었고
\par 23 레위 자손의 족장들은 엘리아십의 아들 요하난 때까지 역대지략에 기록되었으며
\par 24 레위 사람의 어른은 하사뱌와 세레뱌와 갓미엘의 아들 예수아라 저희가 그 형제의 맞은편에 있어 하나님의 사람 다윗의 명한대로 반차를 따라 주를 찬양하며 감사하고
\par 25 맛다냐와 박부갸와 오바댜와 므술람과 달몬과 악굽은 다 문지기로서 반차대로 문 안의 곳간을 파수하였나니
\par 26 이상 모든 사람은 요사닥의 손자 예수아의 아들 요야김과 방백 느헤미야와 제사장 겸 서기관 에스라 때에 있었느니라
\par 27 예루살렘 성곽이 낙성되니 각처에서 레위 사람들을 찾아 예루살렘으로 데려다가 감사하며 노래하며 제금 치며 비파와 수금을 타며 즐거이 봉헌식을 행하려 하매
\par 28 이에 노래하는 자들이 예루살렘 사방 들과 느도바 사람의 동네에서 모여 오고
\par 29 또 벧길갈과 게바와 아스마웹 들에서 모여 왔으니 이 노래하는 자들은 자기를 위하여 예루살렘 사방에 동네를 세웠음이라
\par 30 제사장들과 레위 사람들이 몸을 정결케 하고 또 백성과 성문과 성을 정결케 하니라
\par 31 이에 내가 유다의 방백들로 성 위에 오르게 하고 또 감사 찬송하는 자의 큰 무리를 두 떼로 나누어 성 위로 항렬을 지어 가게 하는데 한 떼는 우편으로 분문을 향하여 가게하니
\par 32 따르는 자는 호세야와 유다 방백의 절반이요
\par 33 또 아사랴와 에스라와 므술람과
\par 34 유다와 베냐민과 스마야와 예레미야며
\par 35 또 제사장의 자손 몇이 나팔을 잡았으니 요나단의 아들 스마야의 손자 맛다냐의 증손 미가야의 현손 삭굴의 오대손 아삽의 육대손 스가랴와
\par 36 그 형제 스마야와 아사렐과 밀랄래와 길랄래와 마애와 느다넬과 유다와 하나니라 다 하나님의 사람 다윗의 악기를 잡았고 학사 에스라가 앞서서
\par 37 샘문으로 말미암아 전진하여 성으로 올라가는 곳에 이르러 다윗성의 층계로 올라가서 다윗의 궁 윗 길에서 동향하여 수문에 이르렀고
\par 38 감사 찬송하는 다른 떼는 저희를 마주 진행하는데 내가 백성의 절반으로 더불어 그 뒤를 따라 성 위로 행하여 풀무 망대 윗 길로 성 넓은 곳에 이르고
\par 39 에브라임 문 위로 말미암아 옛문과 어문과 하나넬 망대와 함메아 망대를 지나 양문에 이르러 감옥 문에 그치매
\par 40 이에 감사 찬송하는 두 떼와 나와 민장의 절반은 하나님의 전에 섰고
\par 41 제사장 엘리아김과 마아세야와 미냐민과 미가야와 엘료에내와 스가랴와 하나냐는 다 나팔을 잡았고
\par 42 또 마아세야와 스마야와 엘르아살과 웃시와 여호하난과 말기야와 엘람과 에셀이 함께 있으며 노래하는 자는 크게 찬송하였는데 그 감독은 예스라히야라
\par 43 이 날에 무리가 크게 제사를 드리고 심히 즐거워하였으니 이는 하나님이 크게 즐거워하게 하셨음이라 부녀와 어린 아이도 즐거워 하였으므로 예루살렘의 즐거워하는 소리가 멀리 들렸느니라
\par 44 그 날에 사람을 세워 곳간을 맡기고 제사장들과 레위 사람들에게 돌릴 것 곧 율볍에 정한대로 거제물과 처음 익은 것과 십일조를 모든 성읍 밭에서 거두어 이 곳간에 쌓게 하였노니 이는 유다 사람이 섬기는 제사장들과 레위 사람들을 인하여 즐거워함을 인함이라
\par 45 저희는 하나님을 섬기는 일과 결례의 일을 힘썼으며 노래하는 자들과 문지기들도 그러하여 모두 다윗과 그 아들 솔로몬의 명을 좇아 행하였으니
\par 46 옛적 다윗과 아삽의 때에는 노래하는 자의 두목이 있어서 하나님께 찬송하는 노래와 감사하는 노래를 하였음이며
\par 47 스룹바벨과 느헤미야 때에는 온 이스라엘이 노래하는 자들과 문지기들에게 날마다 쓸 것을 주되 그 구별한 것을 레위 사람들에게 주고 레위 사람들은 그것을 또 구별하여 아론 자손에게 주었느니라

\chapter{13}

\par 1 그 날에 모세의 책을 낭독하여 백성에게 들렸는데 그 책에 기록하기를 암몬 사람과 모압 사람은 영영히 하나님의 회에 들어오지 못하리니
\par 2 이는 저희가 양식과 물로 이스라엘 자손을 영접지 아니하고 도리어 발람에게 뇌물을 주어 저주하게 하였음이라 그러나 우리 하나님이 그 저주를 돌이켜 복이 되게 하셨다 하였는지라
\par 3 백성이 이 율법을 듣고 곧 섞인 무리를 이스라엘 가운데서 몰수히 분리케 하였느니라
\par 4 이전에 우리 하나님의 전 골방을 맡은 제사장 엘리아십이 도비야와 연락이 있었으므로
\par 5 도비야를 위하여 한 큰 방을 갖추었으니 그 방은 원래 소제물과 유향과 기명과 또 레위 사람들과 노래하는 자들과 문지기들에게 십일조로 주는 곡물과 새 포도주와 기름과 또 제사장들에게 주는 거제물을 두는 곳이라
\par 6 그 때에는 내가 예루살렘에 있지 아니하였었느니라 바벨론 왕 아닥사스다 삼십 이년에 내가 왕에게 나아갔다가 며칠 후에 왕에게 말미를 청하고
\par 7 예루살렘에 이르러서야 엘리아십이 도비야를 위하여 하나님의 전 뜰에 방을 갖춘 악한 일을 안지라
\par 8 내가 심히 근심하여 도비야의 세간을 그 방 밖으로 다 내어 던지고
\par 9 명하여 그 방을 정결케 하고 하나님의 전의 기명과 소제물과 유향을 다시 그리로 들여 놓았느니라
\par 10 내가 또 알아 본즉 레위 사람들의 받을 것을 주지 아니 하였으므로 그 직무를 행하는 레위 사람들과 노래하는 자들이 각각 그 전리로 도망하였기로
\par 11 내가 모든 민장을 꾸짖어 이르기를 하나님의 전이 어찌하여 버린바 되었느냐 하고 곧 레위 사람을 불러 모아 다시 그 처소에 세웠더니
\par 12 이에 온 유다가 곡식과 새 포도주와 기름의 십일조를 가져다가 곳간에 들이므로
\par 13 내가 제사장 셀레먀와 서기관 사독과 레위 사람 브다야로 고지기를 삼고 맛다냐의 손자 삭굴의 아들 하난으로 버금을 삼았나니 이는 저희가 충직한 자로 인정됨이라 그 직분은 형제들에게 분배하는 일이었느니라
\par 14 내 하나님이여 이 일을 인하여 나를 기억하옵소서 내 하나님의 전과 그 모든 직무를 위하여 나의 행한 선한 일을 도말하지 마옵소서
\par 15 그 때에 내가 본즉 유다에게 어떤 사람이 안식일에 술틀을 밟고 곡식단을 나귀에 실어 운반하며 포도주와 포도와 무화과와 여러가지 짐을 지고 안식일에 예루살렘에 들어와서 식물을 팔기로 그 날에 내가 경계하였고
\par 16 또 두로 사람이 예루살렘에 거하며 물고기와 각양 물건을 가져다가 안식일에 유다 자손에게 예루살렘에서도 팔기로
\par 17 내가 유다 모든 귀인을 꾸짖어 이르기를 너희가 어찌 이 악을 행하여 안식일을 범하느냐
\par 18 너희 열조가 이같이 행하지 아니 하였느냐 그러므로 우리 하나님이 이 모든 재앙으로 우리와 이 성읍에 내리신 것이 아니냐 이제 너희가 오히려 안식일을 범하여 진노가 이스라엘에게 임함이 더욱 심하게 하는도다 하고
\par 19 안식일 전 예루살렘 성문이 어두워 갈 때에 내가 명하여 성문을 닫고 안식일이 지나기 전에는 열지 말라 하고 내 종자 두어 사람을 성문마다 세워서 안식일에 아무 짐도 들어 오지 못하게 하매
\par 20 장사들과 각양 물건 파는 자들이 한두번 예루살렘성 밖에서 자므로
\par 21 내가 경계하여 이르기를 너희가 어찌하여 성 밑에서 자느냐 다시 이같이 하면 내가 잡으리라 하였더니 그 후부터는 안식일에 저희가 다시 오지 아니하였느니라
\par 22 내가 또 레위 사람들을 명하여 몸을 정결케 하고 와서 성문을 지켜서 안식일로 거룩하게 하라 하였느니라 나의 하나님이여 나를 위하여 이 일도 기억하옵시고 주의 큰 은혜대로 나를 아끼시옵소서
\par 23 그 때에 내가 또 본즉 유다 사람이 아스돗과 암몬과 모압 여인을 취하여 아내를 삼았는데
\par 24 그 자녀가 아스돗 방언을 절반쯤은 하여도 유다 방언은 못하니 그 하는 말이 각 족속의 방언이므로
\par 25 내가 책망하고 저주하며 두어 사람을 때리고 그 머리털을 뽑고 이르되 너희는 너희 딸들로 저희 아들들에게 주지 말고 너희 아들들이나 너희를 위하여 저희 딸을 데려오지 않겠다고 하나님을 가리켜 맹세하라 하고
\par 26 또 이르기를 옛적에 이스라엘 왕 솔로몬이 이 일로 범죄하지 아니하였느냐 저는 열국 중에 비길 왕이 없이 하나님의 사랑을 입은 자라 하나님이 저로 왕을 삼아 온 이스라엘을 다스리게 하셨으나 이방 여인이 저로 범죄케 하였나니
\par 27 너희가 이방 여인을 취하여 크게 악을 행하여 우리 하나님께 범죄하는 것을 우리가 어찌 용납하겠느냐
\par 28 대제사장 엘리아십의 손자 요야다의 아들 하나가 호론 사람 산발랏의 사위가 되었으므로 내가 쫓아내어 나를 떠나게 하였느니라
\par 29 내 하나님이여 저희가 제사장의 직분을 더럽히고 제사장의 직분과 레위 사람에 대한 언약을 어기었사오니 저희를 기억하옵소서
\par 30 내가 이와 같이 저희로 이방 사람을 떠나게 하여 깨끗하게 하고 또 제사장과 레위 사람의 반열을 세워 각각 그 일을 맡게 하고
\par 31 또 정한 기한에 나무와 처음 익은 것을 드리게 하였사오니 내 하나님이여 나를 기억하사 복을 주옵소서


\end{document}