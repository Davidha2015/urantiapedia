\begin{document}

\title{Esther}

Est 1:1  이 일은 아하수에로왕 때에 된 것이니 아하수에로는 인도로 구스까지 일백 이십 칠도를 치리하는 왕이라
Est 1:2  당시에 아하수에로왕이 수산 궁에서 즉위하고
Est 1:3  위에 있은지 삼년에 그 모든 방백과 신복을 위하여 잔치를 베푸니 바사와 메대의 장수와 각 도의 귀족과 방백들이 다 왕 앞에 있는지라
Est 1:4  왕이 여러 날 곧 일백 팔십일 동안에 그 영화로운 나라의 부함과 위엄의 혁혁함을 나타내니라
Est 1:5  이 날이 다하매 왕이 또 도성 수산 대소 인민을 위하여 왕궁 후원 뜰에서 칠일 동안 잔치를 베풀새
Est 1:6  "백색, 녹색, 청색 휘장을 자색 가는 베줄로 대리석 기둥 은고리에 매고 금과 은으로 만든 걸상을 화반석, 백석, 운모석, 흑석을 깐 땅에 진설하고"
Est 1:7  금잔으로 마시게 하니 잔의 식양이 각기 다르고 왕의 풍부한대로 어주가 한이 없으며
Est 1:8  마시는 것도 규모가 있어 사람으로 억지로 하지 않게 하니 이는 왕이 모든 궁내 관리에게 명하여 각 사람으로 마음대로 하게 함이더라
Est 1:9  왕후 와스디도 아하수에로 왕궁에서 부녀들을 위하여 잔치를 베푸니라
Est 1:10  제 칠일에 왕이 주흥이 일어나서 어전 내시 므후만과 비스다와 하르보나와 빅다와 아박다와 세달과 가르가스 일곱 사람을 명하여
Est 1:11  왕후 와스디를 청하여 왕후의 면류관을 정제하고 왕의 앞으로 나아오게 하여 그 아리따움을 뭇 백성과 방백들에게 보이게 하라 하니 이는 왕후의 용모가 보기에 좋음이라
Est 1:12  그러나 왕후 와스디가 내시의 전하는 왕명을 좇아 오기를 싫어하니 왕이 진노하여 중심이 불 붙는듯하더라
Est 1:13  왕이 사례를 아는 박사들에게 묻되 (왕이 규례와 법률을 아는 자에게 묻는 전례가 있는데
Est 1:14  때에 왕에게 가까이 하여 왕의 기색을 살피며 나라 첫 자리에 앉은 자는 바사와 메대의 일곱 방백 곧 가르스나와 세달과 아드마다와 다시스와 메레스와 마르스나와 므무간이라)
Est 1:15  왕후 와스디가 내시의 전하는 아하수에로 왕명을 좇지 아니하니 규례대로 하면 어떻게 처치할꼬
Est 1:16  므무간이 왕과 방백 앞에서 대답하여 가로되 왕후 와스디가 왕에게만 잘못할뿐 아니라 아하수에로왕의 각 도 방백과 뭇 백성에게도 잘못하였나이다
Est 1:17  아하수에로왕이 명하여 왕후 와스디를 청하여도 오지 아니하였다 하는 왕후의 행위의 소문이 모든 부녀에게 전파되면 저희도 그 남편을 멸시할 것인즉
Est 1:18  오늘이라도 바사와 메대의 귀부인들이 왕후의 행위를 듣고 왕의 모든 방백에게 그렇게 말하리니 멸시와 분노가 많이 일어나리이다
Est 1:19  왕이 만일 선히 여기실진대 와스디로 다시는 왕 앞에 오지 못하게 하는 조서를 내리되 바사와 메대의 법률 중에 기록하여 변역함이 없게 하고 그 왕후의 위를 저보다 나은 사람에게 주소서
Est 1:20  왕의 조서가 이 광대한 전국에 반포되면 귀천을 무론하고 모든 부녀가 그 남편을 존경하리이다
Est 1:21  왕과 방백들이 그 말을 선히 여긴지라 왕이 므무간의 말대로 행하여
Est 1:22  각 도 각 백성의 문자와 방언대로 모든 도에 조서를 내려 이르기를 남편으로 그 집을 주관하게 하고 자기 민족의 방언대로 말하게 하라 하였더라
Est 2:1  그 후에 아하수에로왕의 노가 그치매 와스디와 그의 행한 일과 그에 대하여 내린 조서를 생각하거늘
Est 2:2  왕의 시신이 아뢰되 왕은 왕을 위하어 아리따운 처녀들을 구하게 하시되
Est 2:3  전국 각 도에 관리를 명령하여 아리따운 처녀를 다 도성 수산으로 모아 후궁으로 들여 궁녀를 주관하는 내시 헤개의 손에 붙여 그 몸을 정결케 하는 물품을 주게 하시고
Est 2:4  왕의 눈에 아름다운 처녀로 와스디를 대신하여 왕후를 삼으소서 왕이 그 말을 선히 여겨 그대로 행하니라
Est 2:5  도성 수산에 한 유다인이 있으니 이름은 모르드개라 저는 베냐민 자손이니 기스의 증손이요 시므이의 손자요 야일의 아들이라
Est 2:6  전에 바벨론 왕 느부갓네살이 예루살렘에서 유다 왕 여고냐와 백성을 사로잡아 갈 때에 모르드개도 함께 사로잡혔더라
Est 2:7  저의 삼촌의 딸 하닷사 곧 에스더는 부모가 없고 용모가 곱고 아리따운 처녀라 그 부모가 죽은 후에 모르드개가 자기 딸 같이 양육하더라
Est 2:8  왕의 조명이 반포되매 처녀들이 도성 수산에 많이 모여 헤개의 수하에 나아갈 때에 에스더도 왕궁으로 이끌려 가서 궁녀를 주관하는 헤개의 수하에 속하니
Est 2:9  헤개가 이 처녀를 기뻐하여 은혜를 베풀어 몸을 정결케 할 물품과 일용품을 곧 주며 또 왕궁에서 의례히 주는 일곱 궁녀를 주고 에스더와 그 궁녀들을 후궁 아름다운 처소로 옮기더라
Est 2:10  에스더가 자기의 민족과 종족을 고하지 아니하니 이는 모르드개가 명하여 고하지 말라 하였음이라
Est 2:11  모르드개가 날마다 후궁 뜰 앞으로 왕래하며 에스더의 안부와 어떻게 될 것을 알고자 하더라
Est 2:12  처녀마다 차례대로 아하수에로왕에게 나아가기 전에 여자에 대하여 정한 규례대로 열 두달 동안을 행하되 여섯달은 향품과 여자에게 쓰는 다른 물품을 써서 몸을 정결케 하는 기한을 마치며
Est 2:13  처녀가 왕에게 나아갈 때에는 그 구하는 것을 다 주어 후궁에서 왕궁으로 가지고 가게 하고
Est 2:14  저녁이면 갔다가 아침에는 둘째 후궁으로 돌아와서 비빈을 주관하는 내시 사아스가스의 수하에 속하고 왕이 저를 기뻐하여 그 이름을 부르지 아니하면 다시 왕에게 나아가지 못하더라
Est 2:15  모르드개의 삼촌 아비하일의 딸 곧 모르드개가 자기의 딸 같이 양육하는 에스더가 차례대로 왕에게 나아갈 때에 궁녀를 주관하는 내시 헤개의 정한 것 외에는 다른것을 구하지 아니하였으나 모든 보는 자에게 굄을 얻더라
Est 2:16  아하수에로왕의 칠년 시월 곧 데벳월에 에스더가 이끌려 왕궁에 들어가서 왕의 앞에 나아가니
Est 2:17  왕이 모든 여자보다 에스더를 더욱 사랑하므로 저가 모든 처녀보다 왕의 앞에 더욱 은총을 얻은지라 왕이 그 머리에 면류관을 씌우고 와스디를 대신하여 왕후를 삼은 후에
Est 2:18  왕이 크게 잔치를 베푸니 이는 에스더를 위한 잔치라 모든 방백과 신복을 향응하고 또 각 도의 세금을 면제하고 왕의 풍부함을 따라 크게 상주니라
Est 2:19  처녀들을 다시 모을 때에는 모르드개가 대궐 문에 앉았더라
Est 2:20  에스더가 모르드개의 명한대로 그 종족과 민족을 고하지 아니 하니 저가 모르드개의 명을 양육 받을 때와 같이 좇음이더라
Est 2:21  모르드개가 대궐 문에 앉았을 때에 문 지킨 왕의 내시 빅단과 데레스 두 사람이 아하수에로왕을 원한하여 모살하려 하거늘
Est 2:22  모르드개가 알고 왕후 에스더에게 고하니 에스더가 모르드개의 이름으로 왕에게 고한지라
Est 2:23  사실하여 실정을 얻었으므로 두 사람을 나무에 달고 그 일을 왕의 앞에서 궁중 일기에 기록하니라
Est 3:1  그 후에 아하수에로왕이 아각 사람 함므다다의 아들 하만의 지위를 높이 올려 모든 함께 있는 대신 위에 두니
Est 3:2  대궐 문에 있는 왕의 모든 신복이 다 왕의 명대로 하만에게 꿇어 절하되 모르드개는 꿇지도 아니하고 절하지도 아니하니
Est 3:3  대궐 문에 있는 왕의 신복이 모르드개에게 이르되 너는 어찌하여 왕의 명령을 거역하느냐 하고
Est 3:4  날마다 권하되 모르드개가 듣지 아니하고 자기는 유다인임을 고하였더니 저희가 모르드개의 일이 어찌 되나 보고자 하여 하만에게 고하였더라
Est 3:5  하만이 모르드개가 꿇지도 아니하고 절하지도 아니함을 보고 심히 노하더니
Est 3:6  저희가 모르드개의 민족을 하만에게 고한고로 하만이 모르드개만 죽이는 것이 경하다 하고 아하수에로의 온 나라에 있는 유다인 곧 모르드개의 민족을 다 멸하고자 하더라
Est 3:7  아하수에로왕 십 이년 정월 곧 니산월에 무리가 하만 앞에서 날과 달에 대하여 부르 곧 제비를 뽑아 십이월 곧 아달월을 얻은지라
Est 3:8  하만이 아하수에로왕에게 아뢰되 한 민족이 왕의 나라 각 도 백성 중에 흩어져 거하는데 그 법률이 만민보다 달라서 왕의 법률을 지키지 아니하오니 용납하는 것이 왕에게 무익하니이다
Est 3:9  왕이 옳게 여기시거든 조서를 내려 저희를 진멸하소서 내가 은 일만 달란트를 왕의 일을 맡은 자의 손에 부쳐 왕의 부고에 드리리이다
Est 3:10  왕이 반지를 손에서 빼어 유다인의 대적 곧 아각 사람 함므다다의 아들 하만에게 주며
Est 3:11  이르되 그 은을 네게 주고 그 백성도 그리하노니 너는 소견에 좋을대로 행하라 하더라
Est 3:12  정월 십 삼일에 왕의 서기관이 소집되어 하만의 명을 따라 왕의 대신과 각 도 방백과 각 민족의 관원에게 아하수에로왕의 이름으로 조서를 쓰되 곧 각 도의 문자와 각 민족의 방언대로 쓰고 왕의 반지로 인치니라
Est 3:13  이에 그 조서를 역졸에게 부쳐 왕의 각 도에 보내니 십이월 곧 아달월 십 삼일 하루 동안에 모든 유다인을 노소나 어린 아이나 부녀를 무론하고 죽이고 도륙하고 진멸하고 또 그 재산을 탈취하라 하였고
Est 3:14  이 명령을 각 도에 전하기 위하여 조서의 초본을 모든 민족에게 선포하여 그 날을 위하여 준비하게 하라 하였더라
Est 3:15  역졸이 왕의 명을 받들어 급히 나가매 그 조서가 도성 수산에도 반포되니 왕은 하만과 함께 앉아 마시되 수산성은 어지럽더라
Est 4:1  모르드개가 이 모든 일을 알고 그 옷을 찢고 굵은 베를 입으며 재를 무릅쓰고 성중에 나가서 대성 통곡하며
Est 4:2  대궐 문 앞까지 이르렀으니 굵은 베를 입은 자는 대궐 문에 들어가지 못함이라
Est 4:3  왕의 조명이 각 도에 이르매 유다인이 크게 애통하여 금식하며 곡읍하며 부르짖고 굵은 베를 입고 재에 누운 자가 무수하더라
Est 4:4  에스더의 시녀와 내시가 나아와 고하니 왕후가 심히 근심하여 입을 의복을 모르드개에게 보내어 그 굵은 베를 벗기고자 하나 모르드개가 받지 아니하는지라
Est 4:5  에스더가 왕의 명으로 자기에게 근시하는 내시 하닥을 불러 명하여 모르드개에게 가서 이것이 무슨 일이며 무슨 연고인가 알아 보라 하매
Est 4:6  하닥이 대궐 문앞 성중 광장에 있는 모르드개에게 이르니
Est 4:7  모르드개가 자기의 당한 모든 일과 하만이 유다인을 멸하려고 왕의 부고에 바치기로 한 은의 정확한 수효를 하닥에게 말하고
Est 4:8  또 유다인을 진멸하라고 수산궁에서 내린 조서 초본을 하닥에게 주어 에스더에게 뵈어 알게 하고 또 저에게 부탁하여 왕에게 나아가서 그 앞에서 자기의 민족을 위하여 간절히 구하라 하니
Est 4:9  하닥이 돌아와 모드드개의 말을 에스더에게 고하매
Est 4:10  에스더가 하닥에게 이르되 너는 모르드개에게 고하기를
Est 4:11  왕의 신복과 왕의 각 도 백성이 다 알거니와 무론 남녀하고 부름을 받지 아니하고 안뜰에 들어가서 왕에게 나아가면 오직 죽이는 법이요 왕이 그 자에게 금홀을 내어 밀어야 살것이라 이제 내가 부름을 입어 왕에게 나아가지 못한지가 이미 삼십일이라 하라
Est 4:12  그가 에스더의 말로 모르드개에게 고하매
Est 4:13  모르드개가 그를 시켜 에스더에게 회답하되 너는 왕궁에 있으니 모든 유다인 중에 홀로 면하리라 생각지 말라
Est 4:14  이 때에 네가 만일 잠잠하여 말이 없으면 유다인은 다른데로 말미암아 놓임과 구원을 얻으려니와 너와 네 아비 집은 멸망하리라 네가 왕후의 위를 얻은 것이 이 때를 위함이 아닌지 누가 아느냐
Est 4:15  에스더가 명하여 모르드개에게 회답하되
Est 4:16  당신은 가서 수산에 있는 유다인을 다 모으고 나를 위하여 금식하되 밤낮 삼일을 먹지도 말고 마시지도 마소서 나도 나의 시녀로 더불어 이렇게 금식한 후에 규례를 어기고 왕에게 나아가리니 죽으면 죽으리이다
Est 4:17  모르드개가 가서 에스더의 명한대로 다 행하니라
Est 5:1  제 삼일에 에스더가 왕후의 예복을 입고 왕궁 안뜰 곧 어전 맞은편에 서니 왕이 어전에서 전 문을 대하여 보좌에 앉았다가
Est 5:2  왕후 에스더가 뜰에 선 것을 본즉 심히 사랑스러우므로 손에 잡았던 금홀을 그에게 내어미니 에스더가 가까이 가서 금홀 끝을 만진지라
Est 5:3  왕이 이르되 왕후 에스더여 그대의 소원이 무엇이며 요구가 무엇이뇨 나라의 절반이라도 그대에게 주겠노라
Est 5:4  에스더가 가로되 오늘 내가 왕을 위하여 잔치를 베풀었사오니 왕이 선히 여기시거든 하만과 함께 임하소서
Est 5:5  왕이 가로되 에스더의 말한대로 하도록 하만을 급히 부르라 하고 이에 왕이 하만과 함께 에스더의 베푼 잔치에 나아가니라
Est 5:6  잔치의 술을 마실 때에 왕이 에스더에게 이르되 그대의 소청이 무엇이뇨 곧 허락하겠노라 그대의 요구가 무엇이뇨 나라의 절반이라 할지라도 시행하겠노라
Est 5:7  "에스더가 대답하여 가로되 나의 소청, 나의 요구가 이러하니이다"
Est 5:8  내가 만일 왕의 목전에서 은혜를 입었고 왕이 내 소청을 허락하시며 내 요구를 시행하시기를 선히 여기시거든 내가 왕과 하만을 위하여 베푸는 잔치에 또 나아오소서 내일은 왕의 말씀대로 하리이다
Est 5:9  이 날에 하만이 마음이 기뻐 즐거이 나오더니 모르드개가 대궐 문에 있어 일어나지도 아니하고 몸을 움직이지도 아니하는 것을 보고 심히 노하나
Est 5:10  참고 집에 돌아와서 사람을 보내어 그 친구들과 그 아내 세레스를 청하여
Est 5:11  자기의 부성한 영광과 자녀가 많은 것과 왕이 자기를 들어 왕의 모든 방백이나 신복들보다 높인 것을 다 말하고
Est 5:12  또 가로되 왕후 에스더가 그 베푼 잔치에 왕과 함께 오기를 허락 받은 자는 나밖에 없었고 내일도 왕과 함께 청함을 받았느니라
Est 5:13  그러나 유다 사람 모르드개가 대궐 문에 앉은 것을 보는 동안에는 이 모든 일이 만족하지 아니하도다
Est 5:14  그 아내 세레스와 모든 친구가 이르되 오십 규빗이나 높은 나무를 세우고 내일 왕에게 모르드개를 그 나무에 달기를 구하고 왕과 함께 즐거이 잔치에 나아가소서 하만이 그 말을 선히 여기고 명하여 나무를 세우니라
Est 6:1  이 밤에 왕이 잠이 오지 아니하므로 명하여 역대 일기를 가져다가 자기 앞에서 읽히더니
Est 6:2  그 속에 기록하기를 문 지킨 왕의 두 내시 빅다나와 데레스가 아하수에로왕을 모살하려 하는 것을 모르드개가 고발하였다 하였는지라
Est 6:3  왕이 가로되 이 일을 인하여 무슨 존귀와 관작을 모르드개에게 베풀었느냐 시신이 대답하되 아무 것도 베풀지 아니하였나이다
Est 6:4  왕이 가로되 누가 뜰에 있느냐 마침 하만이 자기가 세운 나무에 모르드개 달기를 왕께 구하고자 하여 왕궁 바깥 뜰에 이른지라
Est 6:5  시신이 고하되 하만이 뜰에 섰나이다 왕이 가로되 들어 오게 하라 하니
Est 6:6  하만이 들어오거늘 왕이 묻되 왕이 존귀케 하기를 기뻐하는 사람에게 어떻게 하여야 하겠느뇨 하만이 심중에 이르되 왕이 존귀케 하기를 기뻐하시는 자는 나 외에 누구리요 하고
Est 6:7  왕께 아뢰되 왕께서 사람을 존귀케 하시려면
Est 6:8  왕의 입으시는 왕복과 왕의 타시는 말과 머리에 쓰시는 왕관을 취하고
Est 6:9  그 왕복과 말을 왕의 방백 중 가장 존귀한 자의 손에 붙여서 왕이 존귀케 하시기를 기뻐하시는 사람에게 옷을 입히고 말을 태워서 성중 거리로 다니며 그 앞에서 반포하여 이르기를 왕이 존귀케 하기를 기뻐하시는 사람에게는 이같이 할것이라 하게 하소서
Est 6:10  이에 왕이 하만에게 이르되 너는 네 말대로 속히 왕복과 말을 취하여 대궐 문에 앉은 유다 사람 모르드개에게 행하되 무릇 네가 말한 것에서 조금도 빠짐이 없이 하라
Est 6:11  하만이 왕복과 말을 취하여 모르드개에게 옷을 입히고 말을 태워 성중 거리로 다니며 그 앞에서 반포하되 왕이 존귀케 하시기를 기뻐하시는 사람에게는 이같이 할 것이라 하니라
Est 6:12  모르드개는 다시 대궐 문으로 돌아오고 하만은 번뇌하여 머리를 싸고 급히 집으로 돌아와서
Est 6:13  자기의 당한 모든 일을 그 아내 세레스와 모든 친구에게 고하매 그 중 지혜로운 자와 그 아내 세레스가 가로되 모르드개가 과연 유다 족속이면 당신이 그 앞에서 굴욕을 당하기 시작하였으니 능히 저를 이기지 못하고 분명히 그 앞에 엎드러지리이다
Est 6:14  아직 말이 그치지 아니하여서 왕의 내시들이 이르러 하만을 데리고 에스더의 베푼 잔치에 빨리 나아가니라
Est 7:1  왕이 하만과 함께 또 왕후 에스더의 잔치에 나아가니라
Est 7:2  왕이 이 둘째날 잔치에 술을 마실 때에 다시 에스더에게 물어 가로되 왕후 에스더여 그대의 소청이 무엇이뇨 곧 허락하겠노라 그대의 요구가 무엇이뇨 곧 나라의 절반이라 할지라도 시행하겠노라
Est 7:3  왕후 에스더가 대답하여 가로되 왕이여 내가 만일 왕의 목전에서 은혜를 입었으며 왕이 선히 여기시거든 내 소청대로 내 생명을 내게 주시고 내 요구대로 내 민족을 내게 주소서
Est 7:4  나와 내 민족이 팔려서 죽임과 도륙함과 진멸함을 당하게 되었나이다 만일 우리가 노비로 팔렸더면 내가 잠잠하였으리이다 그래도 대적이 왕의 손해를 보충하지 못하였으리이다
Est 7:5  아하수에로왕이 왕후 에스더에게 일러 가로되 감히 이런 일을 심중에 품은 자가 누구며 그가 어디 있느뇨
Est 7:6  에스더가 가로되 대적과 원수는 이 악한 하만이니이다 하니 하만이 왕과 왕후 앞에서 두려워하거늘
Est 7:7  왕이 노하여 일어나서 잔치 자리를 떠나 왕궁 후원으로 들어가니라 하만이 일어서서 왕후 에스더에게 생명을 구하니 이는 왕이 자기에게 화를 내리기로 결심한줄 앎이더라
Est 7:8  왕이 후원으로부터 잔치 자리에 돌아오니 하만이 에스더의 앉은 걸상 위에 엎드렸거늘 왕이 가로되 저가 궁중 내 앞에서 왕후를 강간까지 하고자 하는가 이 말이 왕의 입에서 나오매 무리가 하만의 얼굴을 싸더라
Est 7:9  왕을 모신 내시 중에 하르보나가 왕에게 아뢰되 왕을 위하여 충성된 말로 고발한 모르드개를 달고자 하여 하만이 고가 오십 규빗 되는 나무를 준비하였는데 이제 그 나무가 하만의 집에 섰나이다 왕이 가로되 하만을 그 나무에 달라 하매
Est 7:10  모르드개를 달고자 한 나무에 하만을 다니 왕의 노가 그치니라
Est 8:1  당일에 아하수에로왕이 유다인의 대적 하만의 집을 왕후 에스더에게 주니라 에스더가 모르드개는 자기에게 어떻게 관계됨을 왕께 고한고로 모르드개가 왕의 앞에 나아오니
Est 8:2  왕이 하만에게 거둔 반지를 빼어 모르드개에게 준지라 에스더가 모르드개로 하만의 집을 주관하게 하니라
Est 8:3  에스더가 다시 왕의 앞에서 말씀하며 왕의 발 아래 엎드려 아각 사람 하만이 유다인을 해하려 한 악한 꾀를 제하기를 울며 구하니
Est 8:4  왕이 에스더를 향하여 금홀을 내어미는지라 에스더가 일어나 왕의 앞에 서서
Est 8:5  가로되 왕이 만일 즐겨하시며 내가 왕의 목전에 은혜를 입었고 또 왕이 이 일을 선히 여기시며 나를 기쁘게 보실진대 조서를 내리사 아각 사람 함므다다의 아들 하만이 왕의 각 도에 있는 유다인을 멸하려고 꾀하고 쓴 조서를 취소하소서
Est 8:6  내가 어찌 내 민족의 화 당함을 참아 보며 내 친척의 멸망함을 참아 보리이까
Est 8:7  아하수에로왕이 왕후 에스더와 유다인 모르드개에게 이르되 하만이 유다인을 살해하려 하므로 나무에 달렸고 내가 그 집으로 에스더에게 주었으니
Est 8:8  너희는 왕의 명의로 유다인에게 조서를 뜻대로 쓰고 왕의 반지로 인을 칠지어다 왕의 이름을 쓰고 왕의 반지로 인친 조서는 누구든지 취소할 수 없음이니라
Est 8:9  그때 시완월 곧 삼월 이십 삼일에 왕의 서기관이 소집되고 무릇 모르드개의 시키는대로 조서를 써서 인도로부터 구스까지의 일백 이십 칠도 유다인과 대신과 방백과 관원에게 전할새 각 도의 문자와 각 민족의 방언과 유다인의 문자와 방언대로 쓰되
Est 8:10  아하수에로왕의 명의로 쓰고 왕의 반지로 인을 치고 그 조서를 역졸들에게 부쳐 전하게 하니 저희는 왕궁에서 길러서 왕의 일에 쓰는 준마를 타는 자들이라
Est 8:11  조서에는 왕이 여러 고을에 있는 유다인에게 허락하여 저희로 함께 모여 스스로 생명을 보호하여 각 도의 백성 중 세력을 가지고 저희를 치려하는 자와 그 처자를 죽이고 도륙하고 진멸하고 그 재산을 탈취하게 하되
Est 8:12  아하수에로왕의 각 도에서 아달월 곧 십이월 십 삼일 하루 동안에 하게 하였고
Est 8:13  이 조서 초본을 각 도에 전하고 각 민족에게 반포하고 유다인으로 예비하였다가 그 날에 대적에게 원수를 갚게 한지라
Est 8:14  왕의 명이 심히 급하매 역졸이 왕의 일에 쓰는 준마를 타고 빨리 나가고 그 조서가 도성 수산에도 반포되니라
Est 8:15  모르드개가 푸르고 흰 조복을 입고 큰 금면류관을 쓰고 자색 가는 베 겉옷을 입고 왕의 앞에서 나오니 수산성이 즐거이 부르며 기뻐하고
Est 8:16  유다인에게는 영광과 즐거움과 기쁨과 존귀함이 있는지라
Est 8:17  "왕의 조명이 이르는 각 도, 각 읍에서 유다인이 즐기고 기뻐하여 잔치를 베풀고 그 날로 경절을 삼으니 본토 백성이 유다인을 두려워하여 유다인 되는 자가 많더라"
Est 9:1  아달월 곧 십이월 십 삼일은 왕의 조명을 행하게 된 날이라 유다인의 대적이 저희를 제어하기를 바랐더니 유다인이 도리어 자기를 미워하는 자를 제어하게 된 그 날에
Est 9:2  "유다인들이 아하수에로왕의 각 도, 각 읍에 모여 자기를 해하고자 하는 자를 죽이려 하니 모든 민족이 저희를 두려워하여 능히 막을 자가 없고"
Est 9:3  각 도 모든 관원과 대신과 방백과 왕의 사무를 보는 자들이 모르드개를 두려워하므로 다 유다인을 도우니
Est 9:4  모르드개가 왕궁에서 존귀하여 점점 창대하매 이 사람 모르드개의 명성이 각 도에 퍼지더라
Est 9:5  유다인이 칼로 그 모든 대적을 쳐서 도륙하고 진멸하고 자기를 미워하는 자에게 마음대로 행하고
Est 9:6  유다인이 또 도성 수산에서 오백인을 죽이고 멸하고
Est 9:7  또 바산다다와 달본과 아스바다와
Est 9:8  보라다와 아달리야와 아리다다와
Est 9:9  바마스다와 아리새와 아리대와 왜사다
Est 9:10  곧 함므다다의 손자요 유다인의 대적 하만의 열 아들을 죽였으나 그 재산에는 손을 대지 아니하였더라
Est 9:11  그 날에 도성 수산에서 도륙한 자의 수효를 왕께 고하니
Est 9:12  왕이 왕후 에스더에게 이르되 유다인이 도성 수산에서 이미 오백인을 죽이고 멸하고 또 하만의 열 아들을 죽였으니 왕의 다른 도에서는 어떠하였겠느뇨 이제 그대의 소청이 무엇이뇨 곧 허락하겠노라 그대의 요구가 무엇이뇨 또한 시행하겠노라
Est 9:13  에스더가 가로되 왕이 만일 선히 여기시거든 수산에 거하는 유다인으로 내일도 오늘날 조서대로 행하게 하시고 하만의 열 아들의 시체를 나무에 달게하소서
Est 9:14  왕이 그대로 행하기를 허락하고 조서를 수산에 내리니 하만의 열 아들의 시체가 달리니라
Est 9:15  아달월 십 사일에도 수산에 있는 유다인이 모여 또 삼백인을 수산에서 도륙하되 그 재산에는 손을 대지 아니하였고
Est 9:16  왕의 각 도에 있는 다른 유다인들이 모여 스스로 생명을 보호하여 대적들에게서 벗어나며 자기를 미워하는 자 칠만 오천인을 도륙하되 그 재산에는 손을 대지 아니하였더라
Est 9:17  아달월 십 삼일에 그 일을 행하였고 십 사일에 쉬며 그 날에 잔치를 베풀어 즐겼고
Est 9:18  수산에 거한 유다인은 십 삼일과 십 사일에 모였고 십 오일에 쉬며 이 날에 잔치를 베풀어 즐긴지라
Est 9:19  그러므로 촌촌의 유다인 곧 성이 없는 고을 고을에 거하는 자들이 아달월 십 사일로 경절을 삼아 잔치를 베풀고 즐기며 서로 예물을 주더라
Est 9:20  모르드개가 이 일을 기록하고 아하수에로왕의 각 도에 있는 모든 유다인에게 무론 원근하고 글을 보내어 이르기를
Est 9:21  한 규례를 세워 해마다 아달월 십 사일과 십 오일을 지키라
Est 9:22  이 달 이 날에 유다인이 대적에게서 벗어나서 평안함을 얻어 슬픔이 변하여 기쁨이 되고 애통이 변하여 길한 날이 되었으니 이 두 날을 지켜 잔치를 베풀고 즐기며 서로 예물을 주며 가난한 자를 구제하라 하매
Est 9:23  유다인이 자기들의 이미 시작한대로 또는 모르드개의 보낸 글대로 계속하여 행하였으니
Est 9:24  곧 아각 사람 함므다다의 아들 모든 유다인의 대적 하만이 유다인을 진멸하기를 꾀하고 부르 곧 제비를 뽑아 저희를 죽이고 멸하려 하였으나
Est 9:25  에스더가 왕의 앞에 나아감을 인하여 왕이 조서를 내려 하만이 유다인을 해하려던 악한 꾀를 그 머리에 돌려보내어 하만과 그 여러 아들을 나무에 달게 하였으므로
Est 9:26  무리가 부르의 이름을 좇아 이 두 날을 부림이라 하고 유다인이 이 글의 모든 말과 이 일에 보고 당한 것을 인하여
Est 9:27  뜻을 정하고 자기와 자손과 자기와 화합한 자들이 해마다 그 기록한 정기에 이 두 날을 연하여 지켜 폐하지 아니하기로 작정하고
Est 9:28  "각 도, 각 읍, 각 집에서 대대로 이 두 날을 기념하여 지키되 이 부림일을 유다인 중에서 폐하지 않게 하고 그 자손 중에서도 기념함이 폐하지 않게 하였더라"
Est 9:29  아비하일의 딸 왕후 에스더와 유다인 모르드개가 전권으로 글을 쓰고 부림에 대한 이 둘째 편지를 굳이 지키게 하되
Est 9:30  화평하고 진실한 말로 편지를 써서 아하수에로의 나라 일백 이십 칠도에 있는 유다 모든 사람에게 보내어
Est 9:31  정한 기한에 이 부림일을 지키게 하였으니 이는 유다인 모르드개와 왕후 에스더의 명한 바와 유다인이 금식하며 부르짖은 것을 인하여 자기와 자기 자손을 위하여 정한 바가 있음이더라
Est 9:32  에스더의 명령이 이 부림에 대한 일을 견고히 하였고 그 일이 책에 기록되었더라
Est 10:1  아하수에로왕이 그 본토와 바다 섬들로 공을 바치게 하였더라
Est 10:2  왕의 능력의 모든 행적과 모르드개를 높여 존귀케 한 사적이 메대와 바사 열왕의 일기에 기록되지 아니하였느냐
Est 10:3  유다인 모르드개가 아하수에로 왕의 다음이 되고 유다인 중에 존대하여 그 허다한 형제에게 굄을 받고 그 백성의 이익을 도모하며 그 모든 종족을 안위하였더라


\end{document}