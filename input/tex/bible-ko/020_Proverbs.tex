\begin{document}

\title{잠언}


\chapter{1}

\par 1 다윗의 아들 이스라엘 왕 솔로몬의 잠언이라
\par 2 이는 지혜와 훈계를 알게 하며 명철의 말씀을 깨닫게 하며
\par 3 "지혜롭게, 의롭게, 공평하게, 정직하게, 행할일에 대하여 훈계를 받게 하며"
\par 4 어리석은 자로 슬기롭게 하며 젊은 자에게 지식과 근신함을 주기위한 것이니
\par 5 지혜있는 자는 듣고 학식이 더할 것이요 명철한 자는 모략을 얻을 것이라
\par 6 잠언과 비유와 지혜있는 자의 말과 그 오묘한 말을 깨달으리라
\par 7 여호와를 경외하는 것이 지식의 근본이어늘 미련한 자는 지혜와 훈계를 멸시하느니라
\par 8 내 아들아 네 아비의 훈계를 들으며 네 어미의 법을 떠나지 말라
\par 9 이는 네 머리의 아름다운 관이요 네 목의 금사슬이니라
\par 10 내 아들아 악한 자가 너를 꾈지라도 좇지 말라
\par 11 그들이 네게 말하기를 우리와 함께 가자 우리가 가만히 엎드렸다가 사람의 피를 흘리자 죄없는 자를 까닭없이 숨어 기다리다가
\par 12 음부 같이 그들을 산 채로 삼키며 무덤에 내려가는 자 같게 통으로 삼키자
\par 13 우리가 온갖 보화를 얻으며 빼앗은 것으로 우리 집에 채우리니
\par 14 너는 우리와 함께 제비를 뽑고 우리가 함께 전대 하나만 두자 할지라도
\par 15 내 아들아 그들과 함께 길에 다니지 말라 네 발을 금하여 그 길을 밟지 말라
\par 16 대저 그 발은 악으로 달려가며 피를 흘리는데 빠름이니라
\par 17 무릇 새가 그물 치는 것을 보면 헛 일이겠거늘
\par 18 그들의 가만히 엎드림은 자기의 피를 흘릴 뿐이요 숨어 기다림은 자기의 생명을 해할 뿐이니
\par 19 무릇 이를 탐하는 자의 길은 다 이러하여 자기의 생명을 잃게 하느니라
\par 20 지혜가 길거리에서 부르며 광장에서 소리를 높이며
\par 21 훤화하는 길 머리에서 소리를 지르며 성문 어귀와 성중에서 그 소리를 발하여 가로되
\par 22 너희 어리석은 자들은 어리석음을 좋아하며 거만한 자들은 거만을 기뻐하며 미련한 자들은 지식을 미워하니 어느 때까지 하겠느냐
\par 23 나의 책망을 듣고 돌이키라 보라 내가 나의 신을 너희에게 부어주며 나의 말을 너희에게 보이리라
\par 24 내가 부를지라도 너희가 듣기 싫어 하였고 내가 손을 펼지라도 돌아보는 자가 없었고
\par 25 도리어 나의 모든 교훈을 멸시하며 나의 책망을 받지 아니하였은즉
\par 26 너희가 재앙을 만날 때에 내가 웃을 것이며 너희에게 두려움이 임할 때에 내가 비웃으리라
\par 27 너희의 두려움이 광풍같이 임하겠고 너희의 재앙이 폭풍같이 임하리니
\par 28 그 때에 너희가 나를 부르리라 그래도 내가 대답지 아니하겠고 부지런히 나를 찾으리라 그래도 나를 만나지 못하리니
\par 29 대저 너희가 지식을 미워하며 여호와 경외하기를 즐거워하지 아니하며
\par 30 나의 교훈을 받지 아니하고 나의 모든 책망을 업신여겼음이라
\par 31 그러므로 자기 행위의 열매를 먹으며 자기 꾀에 배부르리라
\par 32 어리석은 자의 퇴보는 자기를 죽이며 미련한 자의 안일은 자기를 멸망시키려니와
\par 33 오직 나를 듣는 자는 안연히 살며 재앙의 두려움이 없이 평안하리라

\chapter{2}

\par 1 내 아들아 네가 만일 나의 말을 받으며 나의 계명을 네게 간직하며
\par 2 네 귀를 지혜에 기울이며 네 마음을 명철에 두며
\par 3 지식을 불러 구하며 명철을 얻으려고 소리를 높이며
\par 4 은을 구하는 것 같이 그것을 구하며 감추인 보배를 찾는 것 같이 그것을 찾으면
\par 5 여호와 경외하기를 깨달으며 하나님을 알게 되리니
\par 6 대저 여호와는 지혜를 주시며 지식과 명철을 그 입에서 내심이며
\par 7 그는 정직한 자를 위하여 완전한 지혜를 예비하시며 행실이 온전한 자에게 방패가 되시나니
\par 8 대저 그는 공평의 길을 보호하시며 그 성도들의 길을 보전하려 하심이니라
\par 9 그런즉 네가 공의와 공평과 정직 곧 모든 선한 길을 깨달을 것이라
\par 10 곧 지혜가 네 마음에 들어가며 지식이 네 영혼에 즐겁게 될 것이요
\par 11 근신이 너를 지키며 명철이 너를 보호하여
\par 12 악한 자의 길과 패역을 말하는 자에게서 건져내리라
\par 13 이 무리는 정직한 길을 떠나 어두운 길로 행하며
\par 14 행악하기를 기뻐하며 악인의 패역을 즐거워하나니
\par 15 그 길은 구부러지고 그 행위는 패역하리라
\par 16 "지혜가 또 너를 음녀에게서, 말로 호리는 이방 계집에게서 구원 하리니"
\par 17 그는 소시의 짝을 버리며 그 하나님의 언약을 잊어버린자라
\par 18 "그 집은 사망으로, 그 길은 음부로 기울어졌나니"
\par 19 누구든지 그에게로 가는 자는 돌아오지 못하며 또 생명길을 얻지 못하느니라
\par 20 지혜가 너로 선한 자의 길로 행하게 하며 또 의인의 길을 지키게하리니
\par 21 대저 정직한 자는 땅에 거하며 완전한 자는 땅에 남아 있으리라
\par 22 그러나 악인은 땅에서 끊어지겠고 궤휼한 자는 땅에서 뽑히리라

\chapter{3}

\par 1 내 아들아 나의 법을 잊어버리지 말고 네 마음으로 나의 명령을 지키라
\par 2 그리하면 그것이 너로 장수하여 많은 해를 누리게 하며 평강을 더하게 하리라
\par 3 인자와 진리로 네게서 떠나지 않게 하고 그것을 네 목에 매며 네 마음판에 새기라
\par 4 그리하면 네가 하나님과 사람 앞에서 은총과 귀중히 여김을 받으리라
\par 5 너는 마음을 다하여 여호와를 의뢰하고 네 명철을 의지하지 말라
\par 6 너는 범사에 그를 인정하라 그리하면 네 길을 지도하시리라
\par 7 스스로 지혜롭게 여기지 말지어다 여호와를 경외하며 악을 떠날지어다
\par 8 이것이 네 몸에 양약이 되어 네 골수로 윤택하게 하리라
\par 9 네 재물과 네 소산물의 처음 익은 열매로 여호와를 공경하라
\par 10 그리하면 네 창고가 가득히 차고 네 즙틀에 새 포도즙이 넘치리라
\par 11 내 아들아 여호와의 징계를 경히 여기지 말라 그 꾸지람을 싫어 하지 말라
\par 12 대저 여호와께서 그 사랑하시는 자를 징계하시기를 마치 아비가 그 기뻐하는 아들을 징계함 같이 하시느니라
\par 13 지혜를 얻은 자와 명철을 얻은 자는 복이 있나니
\par 14 이는 지혜를 얻는 것이 은을 얻는 것보다 낫고 그 이익이 정금보다 나음이니라
\par 15 지혜는 진주보다 귀하니 너의 사모하는 모든 것으로 이에 비교할수 없도다
\par 16 그 우편 손에는 장수가 있고 그 좌편 손에는 부귀가 있나니
\par 17 그 길은 즐거운 길이요 그 첩경은 다 평강이니라
\par 18 지혜는 그 얻은 자에게 생명 나무라 지혜를 가진 자는 복되도다
\par 19 여호와께서는 지혜로 땅을 세우셨으며 명철로 하늘을 굳게 펴셨고
\par 20 그 지식으로 해양이 갈라지게 하셨으며 공중에서 이슬이 내리게 하셨느니라
\par 21 내 아들아 완전한 지혜와 근신을 지키고 이것들로 네 눈 앞에서 떠나지 않게 하라
\par 22 그리하면 그것이 네 영혼의 생명이 되며 네 목에 장식이 되리니
\par 23 네가 네 길을 안연히 행하겠고 네 발이 거치지 아니하겠으며
\par 24 네가 누울 때에 두려워하지 아니하겠고 네가 누운즉 네 잠이 달리로다
\par 25 너는 창졸간의 두려움이나 악인의 멸망이 임할 때나 두려워하지 말라
\par 26 대저 여호와는 너의 의지할 자이시라 네 발을 지켜 걸리지 않게 하시리라
\par 27 네 손이 선을 베풀 힘이 있거든 마땅히 받을 자에게 베풀기를 아끼지 말며
\par 28 네게 있거든 이웃에게 이르기를 갔다가 다시 오라 내일 주겠노라 하지말며
\par 29 네 이웃이 네 곁에서 안연히 살거든 그를 모해하지 말며
\par 30 사람이 네게 악을 행하지 아니하였거든 까닭없이 더불어 다투지 말며
\par 31 포학한 자를 부러워하지 말며 그 아무 행위든지 좇지 말라
\par 32 대저 패역한 자는 여호와의 미워하심을 입거니와 정직한 자에게는 그의 교통하심이 있으며
\par 33 악인의 집에는 여호와의 저주가 있거니와 의인의 집에는 복이 있느니라
\par 34 진실로 그는 거만한 자를 비웃으시며 겸손한 자에게 은혜를 베푸시나니
\par 35 지혜로운 자는 영광을 기업으로 받거니와 미련한 자의 현달함은 욕이 되느니라

\chapter{4}

\par 1 아들들아 아비의 훈계를 들으며 명철을 얻기에 주의하라
\par 2 내가 선한 도리를 너희에게 전하노니 내 법을 떠나지 말라
\par 3 나도 내 아버지에게 아들이었었으며 내 어머니 보기에 유약한 외아들이었었노라
\par 4 아버지가 내게 가르쳐 이르기를 내 말을 네 마음에 두라 내 명령을 지키라 그리하면 살리라
\par 5 지혜를 얻으며 명철을 얻으라 내 입의 말을 잊지 말며 어기지 말라
\par 6 지혜를 버리지 말라 그가 너를 보호하리라 그를 사랑하라 그가 너를 지키리라
\par 7 지혜가 제일이니 지혜를 얻으라 무릇 너의 얻은 것을 가져 명철을 얻을지니라
\par 8 그를 높이라 그리하면 그가 너를 높이 들리라 만일 그를 품으면 그가 너를 영화롭게 하리라
\par 9 그가 아름다운 관을 네 머리에 두겠고 영화로운 면류관을 네게 주리라 하였느니라
\par 10 내 아들아 들으라 내 말을 받으라 그리하면 네 생명의 해가 길리라
\par 11 내가 지혜로운 길로 네게 가르쳤으며 정직한 첩경으로 너를 인도하였은즉
\par 12 다닐 때에 네 걸음이 곤란하지 아니하겠고 달려갈 때에 실족하지 아니하리라
\par 13 훈계를 굳게 잡아 놓치지 말고 지키라 이것이 네 생명이니라
\par 14 사특한 자의 첩경에 들어가지 말며 악인의 길로 다니지 말지어다
\par 15 그 길을 피하고 지나가지 말며 돌이켜 떠나갈지어다
\par 16 그들은 악을 행하지 못하면 자지 못하며 사람을 넘어뜨리지 못하면 잠이 오지 아니하며
\par 17 불의의 떡을 먹으며 강포의 술을 마심이니라
\par 18 의인의 길은 돋는 햇볕 같아서 점점 빛나서 원만한 광명에 이르거니와
\par 19 악인의 길은 어둠 같아서 그가 거쳐 넘어져도 그것이 무엇인지 깨닫지 못하느니라
\par 20 내 아들아 내 말에 주의하며 나의 이르는 것에 네 귀를 기울이라
\par 21 그것을 네 눈에서 떠나게 말며 네 마음 속에 지키라
\par 22 그것은 얻는 자에게 생명이 되며 그 온 육체의 건강이 됨이니라
\par 23 무릇 지킬만한 것보다 더욱 네 마음을 지키라 생명의 근원이 이에서 남이니라
\par 24 궤휼을 네 입에서 버리며 사곡을 네 입술에서 멀리하라
\par 25 네 눈은 바로 보며 네 눈꺼풀은 네 앞을 곧게 살펴
\par 26 네 발의 행할 첩경을 평탄케 하며 네 모든 길을 든든히하라
\par 27 우편으로나 좌편으로나 치우치지 말고 네 발을 악에서 떠나게 하라

\chapter{5}

\par 1 내 아들아 내 지혜에 주의하며 내 명철에 네 귀를 기울여서
\par 2 근신을 지키며 네 입술로 지식을 지키도록 하라
\par 3 대저 음녀의 입술은 꿀을 떨어뜨리며 그 입은 기름보다 미끄러우나
\par 4 나중은 쑥 같이 쓰고 두 날 가진 칼같이 날카로우며
\par 5 그 발은 사지로 내려가며 그 걸음은 음부로 나아가나니
\par 6 그는 생명의 평탄한 길을 찾지 못하며 자기 길이 든든치 못하여 그것을 깨닫지 못하느니라
\par 7 그런즉 아들들아 나를 들으며 내 입의 말을 버리지 말고
\par 8 네 길을 그에게서 멀리하라 그 집 문에도 가까이 가지 말라
\par 9 두렵건대 네 존영이 남에게 잃어버리게 되며 네 수한이 잔포자에게 빼앗기게 될까 하노라
\par 10 두렵건대 타인이 네 재물로 충족하게 되며 네 수고한 것이 외인의 집에 있게될까 하노라
\par 11 두렵건대 마지막에 이르러 네 몸 네 육체가 쇠패할 때에 네가 한탄하여
\par 12 말하기를 내가 어찌하여 훈계를 싫어하며 내 마음이 꾸지람을 가벼이 여기고
\par 13 내 선생의 목소리를 청종치 아니하며 나를 가르치는 이에게 귀를 기울이지 아니하였던고
\par 14 많은 무리들이 모인 중에서 모든 악에 거의 빠지게 되었었노라 하게 될까 하노라
\par 15 너는 네 우물에서 물을 마시며 네 샘에서 흐르는 물을 마시라
\par 16 어찌하여 네 샘물을 집 밖으로 넘치게 하겠으며 네 도랑물을 거리로 흘러가게 하겠느냐
\par 17 그 물로 네게만 있게 하고 타인으로 더불어 그것을 나누지 말라
\par 18 네 샘으로 복되게 하라 네가 젊어서 취한 아내를 즐거워하라
\par 19 그는 사랑스러운 암사슴 같고 아름다운 암노루 같으니 너는 그 품을 항상 족하게 여기며 그 사랑을 항상 연모하라
\par 20 내 아들아 어찌하여 음녀를 연모하겠으며 어찌하여 이방 계집의 가슴을 안겠느냐
\par 21 대저 사람의 길은 여호와의 눈 앞에 있나니 그가 그 모든 길을 평탄케 하시느니라
\par 22 악인은 자기의 악에 걸리며 그 죄의 줄에 매이나니
\par 23 그는 훈계를 받지 아니함을 인하여 죽겠고 미련함이 많음을 인하여 혼미하게 되느니라

\chapter{6}

\par 1 내 아들아 네가 만일 이웃을 위하여 담보하며 타인을 위하여 보증하였으면
\par 2 네 입의 말로 네가 얽혔으며 네 입의 말로 인하여 잡히게 되었느니라
\par 3 내 아들아 네가 네 이웃의 손에 빠졌은즉 이같이 하라 너는 곧 가서 겸손히 네 이웃에게 간구하여 스스로 구원하되
\par 4 네 눈으로 잠들게 하지 말며 눈꺼풀로 감기게 하지 말고
\par 5 노루가 사냥군의 손에서 벗어나는 것 같이 새가 그물 치는 자의 손에서 벗어나는 것 같이 스스로 구원하라
\par 6 게으른 자여 개미에게로 가서 그 하는 것을 보고 지혜를 얻으라
\par 7 개미는 두령도 없고 간역자도 없고 주권자도 없으되
\par 8 먹을 것을 여름 동안에 예비하며 추수 때에 양식을 모으느니라
\par 9 게으른 자여 네가 어느 때까지 눕겠느냐 네가 어느 때에 잠이 깨어 일어나겠느나
\par 10 "좀더 자자, 좀더 졸자, 손을 모으고 좀더 눕자 하면"
\par 11 네 빈궁이 강도 같이 오며 네 곤핍이 군사 같이 이르리라
\par 12 불량하고 악한 자는 그 행동에 궤휼한 입을 벌리며
\par 13 눈짓을 하며 발로 뜻을 보이며 손가락질로 알게 하며
\par 14 그 마음에 패역을 품으며 항상 악을 꾀하여 다툼을 일으키는자라
\par 15 그러므로 그 재앙이 갑자기 임한즉 도움을 얻지 못하고 당장에 패망하리라
\par 16 여호와의 미워하시는 것 곧 그 마음에 싫어하시는 것이 육 칠 가지니
\par 17 곧 교만한 눈과 거짓된 혀와 무죄한 자의 피를 흘리는 손과
\par 18 악한 계교를 꾀하는 마음과 빨리 악으로 달려가는 발과
\par 19 거짓을 말하는 망령된 증인과 및 형제 사이를 이간하는 자니라
\par 20 내 아들아 네 아비의 명령을 지키며 네 어미의 법을 떠나지 말고
\par 21 그것을 항상 네 마음에 새기며 네 목에 매라
\par 22 그것이 너의 다닐 때에 너를 인도하며 너의 잘 때에 너를 보호하며 너의 깰 때에 너로 더불어 말하리니
\par 23 대저 명령은 등불이요 법은 빛이요 훈계의 책망은 곧 생명의 길이라
\par 24 "이것이 너를 지켜서 악한 계집에게, 이방 계집의 혀로 호리는 말에 빠지지 않게 하리라"
\par 25 네 마음에 그 아름다운 색을 탐하지 말며 그 눈꺼풀에 홀리지 말라
\par 26 음녀로 인하여 사람이 한조각 떡만 남게 됨이며 음란한 계집은 귀한 생명을 사냥함이니라
\par 27 사람이 불을 품에 품고야 어찌 그 옷이 타지 아니하겠으며
\par 28 사람이 숯불을 밟고야 어찌 그 발이 데지 아니하겠느나
\par 29 남의 아내와 통간하는 자도 이와 같을 것이라 무릇 그를 만지기만 하는 자도 죄 없게 되지 아니하리라
\par 30 도적이 만일 주릴 때에 배를 채우려고 도적질하면 사람이 그를 멸시치는 아니하려니와
\par 31 들키면 칠배를 갚아야 하리니 심지어 자기 집에 있는 것을 다 내어주게 되리라
\par 32 부녀와 간음하는 자는 무지한 자라 이것을 행하는 자는 자기의 영혼을 망하게 하며
\par 33 상함과 능욕을 받고 부끄러움을 씻을 수 없게 되나니
\par 34 그 남편이 투기함으로 분노하여 원수를 갚는 날에 용서하지 아니하고
\par 35 아무 벌금도 돌아 보지 아니하며 많은 선물을 줄지라도 듣지 아니하리라

\chapter{7}

\par 1 내 아들아 내 말을 지키며 내 명령을 네게 간직하라
\par 2 내 명령을 지켜서 살며 내 법을 네 눈동자처럼 지키라
\par 3 이것을 네 손가락에 매며 이것을 네 마음판에 새기라
\par 4 지혜에게 너는 내 누이라 하며 명철에게 너는 내 친족이라 하라
\par 5 "그리하면 이것이 너를 지켜서 음녀에게, 말로 호리는 이방 계집에게 빠지지 않게 하리라"
\par 6 "내가 내 집 들창으로, 살창으로 내어다보다가"
\par 7 "어리석은자 중에, 소년 중에 한 지혜 없는 자를 보았노라"
\par 8 그가 거리를 지나 음녀의 골목 모퉁이로 가까이 하여 그 집으로 들어가는데
\par 9 "저물 때, 황혼 때, 깊은밤 흑암 중에라"
\par 10 그 때에 기생의 옷을 입은 간교한 계집이 그를 맞으니
\par 11 이 계집은 떠들며 완패하며 그 발이 집에 머물지 아니하여
\par 12 "어떤 때에는 거리, 어떤 때에는 광장 모퉁이, 모퉁이에 서서 사람을 기다리는 자라"
\par 13 그 계집이 그를 붙잡고 입을 맞추며 부끄러움을 모르는 얼굴로 말하되
\par 14 내가 화목제를 드려서 서원한 것을 오늘날 갚았노라
\par 15 이러므로 내가 너를 맞으려고 나와서 네 얼굴을 찾다가 너를 만났도다
\par 16 내 침상에는 화문 요와 애굽의 문채 있는 이불을 폈고
\par 17 몰약과 침향과 계피를 뿌렸노라
\par 18 오라 우리가 아침까지 흡족하게 서로 사랑하며 사랑함으로 희락 하자
\par 19 남편은 집을 떠나 먼 길을 갔는데
\par 20 은 주머니를 가졌은즉 보름에나 집에 돌아오리라 하여
\par 21 여러가지 고운 말로 혹하게 하며 입술의 호리는 말로 꾀므로
\par 22 소년이 곧 그를 따랐으니 소가 푸주로 가는 것 같고 미련한자가 벌을 받으려고 쇠사슬에 매이러 가는 것과 일반이라
\par 23 필경은 살이 그 간을 뚫기까지에 이를 것이라 새가 빨리 그물로 들어가되 그 생명을 잃어버릴 줄을 알지 못함과 일반이니라
\par 24 아들들아 나를 듣고 내 입의 말에 주의하라
\par 25 네 마음이 음녀의 길로 치우치지 말며 그 길에 미혹지 말지어다
\par 26 대저 그가 많은 사람을 상하여 엎드러지게 하였나니 그에게 죽은자가 허다하니라
\par 27 그 집은 음부의 길이라 사망의 방으로 내려가느니라

\chapter{8}

\par 1 지혜가 부르지 아니하느냐 명철이 소리를 높이지 아니하느냐
\par 2 그가 길가의 높은 곳과 사거리에 서며
\par 3 성문 곁과 문 어귀와 여러 출입하는 문에서 불러 가로되
\par 4 사람들아 내가 너희를 부르며 내가 인자들에게 소리를 높이노라
\par 5 어리석은 자들아 너희는 명철할지니라 미련한 자들아 너희는 마음이 밝을지니라 너희는 들을지어다
\par 6 내가 가장 선한 것을 말하리라 내 입술을 열어 정직을 내리라
\par 7 내 입은 진리를 말하며 내 입술은 악을 미워하느니라
\par 8 내 입의 말은 다 의로운즉 그 가운데 굽은 것과 패역한 것이 없나니
\par 9 이는 다 총명 있는 자의 밝히 아는 바요 지식 얻은 자의 정직히 여기는 바니라
\par 10 너희가 은을 받지 말고 나의 훈계를 받으며 정금보다 지식을 얻으라
\par 11 대저 지혜는 진주보다 나으므로 무릇 원하는 것을 이에 비교할 수 없음이니라
\par 12 나 지혜는 명철로 주소를 삼으며 지식과 근신을 찾아 얻나니
\par 13 여호와를 경외하는 것은 악을 미워하는 것이라 나는 교만과 거만과 악한 행실과 패역한 입을 미워하느니라
\par 14 내게는 도략과 참 지식이 있으며 나는 명철이라 내게 능력이 있으므로
\par 15 나로 말미암아 왕들이 치리하며 방백들이 공의를 세우며
\par 16 나로 말미암아 재상과 존귀한자 곧 세상의 모든 재판관들이 다스리느니라
\par 17 나를 사랑하는 자들이 나의 사랑을 입으며 나를 간절히 찾는 자가 나를 만날 것이니라
\par 18 부귀가 내게 있고 장구한 재물과 의도 그러하니라
\par 19 내 열매는 금이나 정금보다 나으며 내 소득은 천은보다 나으니라
\par 20 나는 의로운 길로 행하며 공평한 길 가운데로 다니나니
\par 21 이는 나를 사랑하는 자로 재물을 얻어서 그 곳간에 채우게 하려 함이니라
\par 22 여호와께서 그 조화의 시작 곧 태초에 일하시기 전에 나를 가지셨으며
\par 23 "만세 전부터 상고부터, 땅이 생기기 전부터, 내가 세움을 입었나니"
\par 24 아직 바다가 생기지 아니하였고 큰 샘들이 있기 전에 내가 이미 났으며
\par 25 산이 세우심을 입기 전에 언덕이 생기기 전에 내가 이미 났으니
\par 26 하나님이 아직 땅도 들도 세상 진토의 근원도 짓지 아니하셨을 때에라
\par 27 그가 하늘을 지으시며 궁창으로 해면에 두르실 때에 내가 거기 있었고
\par 28 그가 위로 구름 하늘을 견고하게 하시며 바다의 샘들을 힘있게 하시며
\par 29 바다의 한계를 정하여 물로 명령을 거스리지 못하게 하시며 또 땅의 기초를 정하실 때에
\par 30 내가 그 곁에 있어서 창조자가 되어 날마다 그 기뻐하신 바가 되었으며 항상 그 앞에서 즐거워하였으며
\par 31 사람이 거처할 땅에서 즐거워하며 인자들을 기뻐하였었느니라
\par 32 아들들아 이제 내게 들으라 내 도를 지키는 자가 복이 있느니라
\par 33 훈계를 들어서 지혜를 얻으라 그것을 버리지 말라
\par 34 누구든지 내게 들으며 날마다 내 문 곁에서 기다리며 문설주 옆에서 기다리는 자는 복이 있나니
\par 35 대저 나를 얻는 자는 생명을 얻고 여호와께 은총을 얻을 것임이니라
\par 36 그러나 나를 잃는 자는 자기의 영혼을 해하는 자라 무릇 나를 미워하는 자는 사망을 사랑하느니라

\chapter{9}

\par 1 지혜가 그 집을 짓고 일곱 기둥을 다듬고
\par 2 짐승을 잡으며 포도주를 혼합하여 상을 갖추고
\par 3 그 여종을 보내어 성중 높은 곳에서 불러 이르기를
\par 4 무릇 어리석은 자는 이리로 돌이키라 또 지혜 없는 자에게 이르기를
\par 5 너는 와서 내 식물을 먹으며 내 혼합한 포도주를 마시고
\par 6 어리석음을 버리고 생명을 얻으라 명철의 길을 행하라 하느니라
\par 7 거만한 자를 징계하는 자는 도리어 능욕을 받고 악인을 책망하는자는 도리어 흠을 잡히느니라
\par 8 거만한 자를 책망하지 말라 그가 너를 미워할까 두려우니라 지혜있는 자를 책망하라 그가 너를 사랑하리라
\par 9 지혜 있는 자에게 교훈을 더하라 그가 더욱 지혜로와질 것이요 의로운 사람을 가르치라 그의 학식이 더하리라
\par 10 여호와를 경외하는 것이 지혜의 근본이요 거룩하신 자를 아는 것이 명철이니라
\par 11 나 지혜로 말미암아 네 날이 많아질 것이요 네 생명의 해가 더하리라
\par 12 네가 만일 지혜로우면 그 지혜가 네게 유익할 것이나 네가 만일 거만하면 너 홀로 해를 당하리라
\par 13 미련한 계집이 떠들며 어리석어서 아무 것도 알지 못하고
\par 14 자기 집 문에 앉으며 성읍 높은 곳에 있는 자리에 앉아서
\par 15 자기 길을 바로 가는 행객을 불러 이르되
\par 16 무릇 어리석은 자는 이리로 돌이키라 또 지혜없는 자에게 이르기를
\par 17 도적질한 물이 달고 몰래 먹는 떡이 맛이 있다 하는도다
\par 18 오직 그 어리석은 자는 죽은 자가 그의 곳에 있는 것과 그의 객들이 음부 깊은 곳에 있는 것을 알지 못하느니라

\chapter{10}

\par 1 솔로몬의 잠언이라 지혜로운 아들은 아비로 기쁘게 하거니와 미련한 아들은 어미의 근심이니라
\par 2 불의의 재물은 무익하여도 의리는 죽음에서 건지느니라
\par 3 여호와께서 의인의 영혼은 주리지 않게 하시나 악인의 소욕은 물리치시느니라
\par 4 손을 게으르게 놀리는 자는 가난하게 되고 손이 부지런한 자는 부하게 되느니라
\par 5 여름에 거두는 자는 지혜로운 아들이나 추수 때에 자는 자는 부끄러움을 끼치는 아들이니라
\par 6 의인의 머리에는 복이 임하거늘 악인의 입은 독을 머금었느니라
\par 7 의인을 기념할 때에는 칭찬하거니와 악인의 이름은 썩으리라
\par 8 마음이 지혜로운 자는 명령을 받거니와 입이 미련한 자는 패망하리라
\par 9 바른 길로 행하는 자는 걸음이 평안하려니와 굽은 길로 행하는 자는 드러나리라
\par 10 눈짓하는 자는 근심을 끼치고 입이 미련한 자는 패망하느니라
\par 11 의인의 입은 생명의 샘이라도 악인의 입은 독을 머금었느니라
\par 12 미움은 다툼을 일으켜도 사랑은 모든 허물을 가리우느니라
\par 13 명철한 자의 입술에는 지혜가 있어도 지혜 없는 자의 등을 위하여는 채찍이 있느니라
\par 14 지혜로운 자는 지식을 간직하거니와 미련한 자의 입은 멸망에 가까우니라
\par 15 부자의 재물은 그의 견고한 성이요 가난한 자의 궁핍은 그의 패망이니라
\par 16 의인의 수고는 생명에 이르고 악인의 소득은 죄에 이르느니라
\par 17 훈계를 지키는 자는 생명길로 행하여도 징계를 버리는 자는 그릇가느니라
\par 18 미워함을 감추는 자는 거짓의 입술을 가진 자요 참소하는 자는 미련한 자니라
\par 19 말이 많으면 허물을 면키 어려우나 그 입술을 제어하는 자는 지혜가 있느니라
\par 20 의인의 혀는 천은과 같거니와 악인의 마음은 가치가 적으니라
\par 21 의인의 입술은 여러 사람을 교육하나 미련한 자는 지식이 없으므로 죽느니라
\par 22 여호와께서 복을 주시므로 사람으로 부하게 하시고 근심을 겸하여 주지 아니하시느니라
\par 23 미련한 자는 행악으로 낙을 삼는 것 같이 명철한 자는 지혜로 낙을 삼느니라
\par 24 악인에게는 그의 두려워하는 것이 임하거니와 의인은 그 원하는 것이 이루어지느니라
\par 25 회리바람이 지나가면 악인은 없어져도 의인은 영원한 기초 같으니라
\par 26 게으른 자는 그 부리는 사람에게 마치 이에 초 같고 눈에 연기 같으니라
\par 27 여호와를 경외하면 장수하느니라 그러나 악인의 연세는 짧아지느니라
\par 28 의인의 소망은 즐거움을 이루어도 악인의 소망은 끊어지느니라
\par 29 여호와의 도가 정직한 자에게는 산성이요 행악하는 자에게는 멸망이니라
\par 30 의인은 영영히 이동되지 아니하여도 악인은 땅에 거하지 못하게 되느니라
\par 31 의인의 입은 지혜를 내어도 패역한 혀는 베임을 당할 것이니라
\par 32 의인의 입술은 기쁘게 할 것을 알거늘 악인의 입은 패역을 말하느니라

\chapter{11}

\par 1 속이는 저울은 여호와께서 미워하셔도 공평한 추는 그가 기뻐하시느니라
\par 2 교만이 오면 욕도 오거니와 겸손한 자에게는 지혜가 있느니라
\par 3 정직한 자의 성실은 자기를 인도하거니와 사특한 자의 패역은 자기를 망케하느니라
\par 4 재물은 진노하시는 날에 무익하나 의리는 죽음을 면케 하느니라
\par 5 완전한 자는 그 의로 인하여 그 길이 곧게 되려니와 악한 자는 그 악을 인하여 넘어지리라
\par 6 정직한 자는 그 의로 인하여 구원을 얻으려니와 사특한 자는 자기의 악에 잡히리라
\par 7 악인은 죽을 때에 그 소망이 끊어지나니 불의의 소망이 없어지느니라
\par 8 의인은 환난에서 구원을 얻고 악인은 와서 그를 대신하느니라
\par 9 사특한 자는 입으로 그 이웃을 망하게 하여도 의인은 그 지식으로 말미암아 구원을 얻느니라
\par 10 의인이 형통하면 성읍이 즐거워하고 악인이 패망하면 기뻐 외치느니라
\par 11 성읍은 정직한 자의 축원을 인하여 진흥하고 악한 자의 입을 인하여 무너지느니라
\par 12 지혜 없는 자는 그 이웃을 멸시하나 명철한 자는 잠잠하느니라
\par 13 두루 다니며 한담하는 자는 남의 비밀을 누설하나 마음이 신실한자는 그런 것을 숨기느니라
\par 14 도략이 없으면 백성이 망하여도 모사가 많으면 평안을 누리느니라
\par 15 타인을 위하여 보증이 되는 자는 손해를 당하여도 보증이 되기를 싫어하는 자는 평안하니라
\par 16 유덕한 여자는 존영을 얻고 근면한 남자는 재물을 얻느니라
\par 17 인자한 자는 자기의 영혼을 이롭게 하고 잔인한 자는 자기의 몸을 해롭게 하느니라
\par 18 악인의 삯은 허무하되 의를 뿌린 자의 상은 확실하니라
\par 19 의를 굳게 지키는 자는 생명에 이르고 악을 따르는 자는 사망에 이르느니라
\par 20 마음이 패려한 자는 여호와의 미움을 받아도 행위가 온전한 자는 그의 기뻐하심을 받느니라
\par 21 악인은 피차 손을 잡을지라도 벌을 면치 못할 것이나 의인의 자손은 구원을 얻으리라
\par 22 아름다운 여인이 삼가지 아니하는 것은 마치 돼지 코에 금고리 같으니라
\par 23 의인의 소원은 오직 선하나 악인의 소망은 진노를 이루느니라
\par 24 흩어 구제하여도 더욱 부하게 되는 일이 있나니 과도히 아껴도 가난하게 될 뿐이니라
\par 25 구제를 좋아하는 자는 풍족하여질 것이요 남을 윤택하게 하는 자는 윤택하여지리라
\par 26 곡식을 내지 아니하는 자는 백성에게 저주를 받을 것이나 파는 자는 그 머리에 복이 임하리라
\par 27 선을 간절히 구하는 자는 은총을 얻으려니와 악을 더듬어 찾는 자에게는 악이 임하리라
\par 28 자기의 재물을 의지하는 자는 패망하려니와 의인은 푸른 잎사귀 같아서 번성하리라
\par 29 자기 집을 해롭게 하는 자의 소득은 바람이라 미련한 자는 마음이 지혜로운 자의 종이 되리라
\par 30 의인의 열매는 생명나무라 지혜로운 자는 사람을 얻느니라
\par 31 보라 의인이라도 이 세상에서 보응을 받겠거든 하물며 악인과 죄인이리요

\chapter{12}

\par 1 훈계를 좋아하는 자는 지식을 좋아하나니 징계를 싫어하는 자는 짐승과 같으니라
\par 2 선인은 여호와께 은총을 받으려니와 악을 꾀하는 자는 정죄하심을 받으리라
\par 3 사람이 악으로 굳게 서지 못하나니 의인의 뿌리는 움직이지 아니하느니라
\par 4 어진 여인은 그 지아비의 면류관이나 욕을 끼치는 여인은 그 지아비로 뼈가 썩음 같게 하느니라
\par 5 의인의 생각은 공직하여도 악인의 도모는 궤휼이니라
\par 6 악인의 말은 사람을 엿보아 피를 흘리자 하는 것이어니와 정직한 자의 입은 사람을 구원하느니라
\par 7 악인은 엎드러져서 소멸되려니와 의인의 집은 서있으리라
\par 8 사람은 그 지혜대로 칭찬을 받으려니와 마음이 패려한 자는 멸시를 받으리라
\par 9 비천히 여김을 받을지라도 종을 부리는 자는 스스로 높은체 하고도 음식이 핍절한 자보다 나으니라
\par 10 의인은 그 육축의 생명을 돌아보나 악인의 긍휼은 잔인이니라
\par 11 자기의 토지를 경작하는 자는 먹을 것이 많거니와 방탕한 것을 따르는 자는 지혜가 없느니라
\par 12 악인은 불의의 이를 탐하나 의인은 그 뿌리로 말미암아 결실하느니라
\par 13 악인은 입술의 허물로 인하여 그물에 걸려도 의인은 환난에서 벗어나느니라
\par 14 사람은 입의 열매로 인하여 복록에 족하며 그 손의 행하는 대로 자기가 받느니라
\par 15 미련한 자는 자기 행위를 바른 줄로 여기나 지혜로운 자는 권고를 듣느니라
\par 16 미련한 자는 분노를 당장에 나타내거니와 슬기로운 자는 수욕을 참느니라
\par 17 진리를 말하는 자는 의를 나타내어도 거짓 증인은 궤휼을 말하느니라
\par 18 혹은 칼로 찌름같이 함부로 말하거니와 지혜로운 자의 혀는 양약 같으니라
\par 19 진실한 입술은 영원히 보존되거니와 거짓 혀는 눈 깜짝일 동안만 있을뿐이니라
\par 20 악을 꾀하는 자의 마음에는 궤휼이 있고 화평을 논하는 자에게는 희락이 있느니라
\par 21 의인에게는 아무 재앙도 임하지 아니하려니와 악인에게는 앙화가 가득하리라
\par 22 거짓 입술은 여호와께 미움을 받아도 진실히 행하는 자는 그의 기뻐하심을 받느니라
\par 23 슬기로운 자는 지식을 감추어 두어도 미련한 자의 마음은 미련한것을 전파하느니라
\par 24 부지런한 자의 손은 사람을 다스리게 되어도 게으른 자는 부림을 받느니라
\par 25 근심이 사람의 마음에 있으면 그것으로 번뇌케 하나 선한 말은 그것을 즐겁게 하느니라
\par 26 의인은 그 이웃의 인도자가 되나 악인의 소행은 자기를 미혹하게하느니라
\par 27 게으른 자는 그 잡을 것도 사냥하지 아니하나니 사람의 부귀는 부지런한 것이니라
\par 28 의로운 길에 생명이 있나니 그 길에는 사망이 없느니라

\chapter{13}

\par 1 지혜로운 아들은 아비의 훈계를 들으나 거만한 자는 꾸지람을 즐겨 듣지 아니하느니라
\par 2 사람은 입의 열매로 인하여 복록을 누리거니와 마음이 궤사한 자는 강포를 당하느니라
\par 3 입을 지키는 자는 그 생명을 보전하나 입술을 크게 벌리는 자에게는 멸망이 오느니라
\par 4 게으른 자는 마음으로 원하여도 얻지 못하나 부지런한 자의 마음은 풍족함을 얻느니라
\par 5 의인은 거짓말을 미워하나 악인은 행위가 흉악하여 부끄러운데 이르느니라
\par 6 의는 행실이 정직한 자를 보호하고 악은 죄인을 패망케 하느니라
\par 7 스스로 부한 체 하여도 아무 것도 없는 자가 있고 스스로 가난한체 하여도 재물이 많은 자가 있느니라
\par 8 사람의 재물이 그 생명을 속할 수는 있으나 가난한 자는 협박을 받을 일이 없느니라
\par 9 의인의 빛은 환하게 빛나고 악인의 등불은 꺼지느니라
\par 10 교만에서는 다툼만 일어날 뿐이라 권면을 듣는 자는 지혜가 있느니라
\par 11 망령되이 얻은 재물은 줄어가고 손으로 모은 것은 늘어가느니라
\par 12 소망이 더디 이루게 되면 그것이 마음을 상하게 하나니 소원이 이루는 것은 곧 생명나무니라
\par 13 말씀을 멸시하는 자는 패망을 이루고 계명을 두려워하는 자는 상을 얻느니라
\par 14 지혜 있는 자의 교훈은 생명의 샘이라 사람으로 사망의 그물을 벗어나게 하느니라
\par 15 선한 지혜는 은혜를 베푸나 궤사한 자의 길은 험하니라
\par 16 무릇 슬기로운 자는 지식으로 행하여도 미련한 자는 자기의 미련한 것을 나타내느니라
\par 17 악한 사자는 재앙에 빠져도 충성된 사신은 양약이 되느니라
\par 18 훈계를 저버리는 자에게는 궁핍과 수욕이 이르거니와 경계를 지키는 자는 존영을 얻느니라
\par 19 소원을 성취하면 마음에 달아도 미련한 자는 악에서 떠나기를 싫어하느니라
\par 20 지혜로운 자와 동행하면 지혜를 얻고 미련한 자와 사귀면 해를 받느니라
\par 21 재앙은 죄인을 따르고 선한 보응은 의인에게 이르느니라
\par 22 선인은 그 산업을 자자 손손에게 끼쳐도 죄인의 재물은 의인을 위하여 쌓이느니라
\par 23 가난한 자는 밭을 경작하므로 양식이 많아지거늘 혹 불의로 인하여 가산을 탕패하는 자가 있느니라
\par 24 초달을 차마 못하는 자는 그 자식을 미워함이라 자식을 사랑하는자는 근실히 징계하느니라
\par 25 의인은 포식하여도 악인의 배는 주리느니라

\chapter{14}

\par 1 무릇 지혜로운 여인은 그 집을 세우되 미련한 여인은 자기 손으 로 그것을 허느니라
\par 2 정직하게 행하는 자는 여호와를 경외하여도 패역하게 행하는 자는 여호와를 경멸히 여기느니라
\par 3 미련한 자는 교만하여 입으로 매를 자청하고 지혜로운 자는 입술로 스스로 보전하느니라
\par 4 소가 없으면 구유는 깨끗하려니와 소의 힘으로 얻는 것이 많으니라
\par 5 신실한 증인은 거짓말을 아니하여도 거짓 증인은 거짓말을 뱉느니라
\par 6 거만한 자는 지혜를 구하여도 얻지 못하거니와 명철한 자는 지식얻기가 쉬우니라
\par 7 너는 미련한 자의 앞을 떠나라 그 입술에 지식 있음을 보지 못함이니라
\par 8 슬기로운 자의 지혜는 자기의 길을 아는 것이라도 미련한 자의 어리석음은 속이는 것이니라
\par 9 미련한 자는 죄를 심상히 여겨도 정직한 자 중에는 은혜가 있느니라
\par 10 마음의 고통은 자기가 알고 마음의 즐거움도 타인이 참예하지 못하느니라
\par 11 악한 자의 집은 망하겠고 정직한 자의 장막은 흥하리라
\par 12 어떤 길은 사람의 보기에 바르나 필경은 사망의 길이니라
\par 13 웃을 때에도 마음에 슬픔이 있고 즐거움의 끝에도 근심이 있느니라
\par 14 마음이 패려한 자는 자기 행위로 보응이 만족하겠고 선한 사람도 자기의 행위로 그러하리라
\par 15 어리석은 자는 온갖 말을 믿으나 슬기로운 자는 그 행동을 삼가느니라
\par 16 지혜로운 자는 두려워하여 악을 떠나나 어리석은 자는 방자하여 스스로 믿느니라
\par 17 노하기를 속히 하는 자는 어리석은 일을 행하고 악한 계교를 꾀하는 자는 미움을 받느니라
\par 18 어리석은 자는 어리석음으로 기업을 삼아도 슬기로운 자는 지식으로 면류관을 삼느니라
\par 19 악인은 선인 앞에 엎드리고 불의자는 의인의 문에 엎드리느니라
\par 20 가난한 자는 그 이웃에게도 미움을 받게 되나 부요한 자는 친구가 많으니라
\par 21 그 이웃을 업신여기는 자는 죄를 범하는 자요 빈곤한 자를 불쌍 히 여기는 자는 복이 있는 자니라
\par 22 악을 도모하는 자는 그릇 가는 것이 아니냐 선을 도모하는 자에 게는 인자와 진리가 있으리라
\par 23 모든 수고에는 이익이 있어도 입술의 말은 궁핍을 이룰 뿐이니라
\par 24 지혜로운 자의 재물은 그의 면류관이요 미련한 자의 소유는 다만 그 미련한 것이니라
\par 25 진실한 증인은 사람의 생명을 구원하여도 거짓말을 뱉는 사람은 속이느니라
\par 26 여호와를 경외하는 자에게는 견고한 의뢰가 있나니 그 자녀들에게 피난처가 있으리라
\par 27 여호와를 경외하는 것은 생명의 샘이라 사망의 그물에서 벗어나게 하느니라
\par 28 백성이 많은 것은 왕의 영광이요 백성이 적은 것은 주권자의 패망이니라
\par 29 노하기를 더디 하는 자는 크게 명철하여도 마음이 조급한 자는 어리석음을 나타내느니라
\par 30 마음의 화평은 육신의 생명이나 시기는 뼈의 썩음이니라
\par 31 가난한 사람을 학대하는 자는 그를 지으신 이를 멸시하는 자요 궁핍한 사람을 불쌍히 여기는 자는 주를 존경하는 자니라
\par 32 악인은 그 환난에 엎드러져도 의인은 그 죽음에도 소망이 있느니라
\par 33 지혜는 명철한 자의 마음에 머물거니와 미련한 자의 속에 있는 것은 나타나느니라
\par 34 의는 나라로 영화롭게 하고 죄는 백성을 욕되게 하느니라
\par 35 슬기롭게 행하는 신하는 왕의 은총을 입고 욕을 끼치는 신하는 그의 진노를 당하느니라

\chapter{15}

\par 1 유순한 대답은 분노를 쉬게 하여도 과격한 말은 노를 격동하느니라
\par 2 지혜 있는 자의 혀는 지식을 선히 베풀고 미련한 자의 입은 미련한 것을 쏟느니라
\par 3 여호와의 눈은 어디서든지 악인과 선인을 감찰하시느니라
\par 4 온량한 혀는 곧 생명 나무라도 패려한 혀는 마음을 상하게 하느니라
\par 5 아비의 훈계를 업신여기는 자는 미련한 자요 경계를 받는 자는 슬기를 얻을 자니라
\par 6 의인의 집에는 많은 보물이 있어도 악인의 소득은 고통이 되느니라
\par 7 지혜로운 자의 입술은 지식을 전파하여도 미련한 자의 마음은 정함이 없느니라
\par 8 악인의 제사는 여호와께서 미워하셔도 정직한 자의 기도는 그가 기뻐하시느니라
\par 9 악인의 길은 여호와께서 미워하셔도 의를 따라가는 자는 그가 사랑하시느니라
\par 10 도를 배반하는 자는 엄한 징계를 받을 것이요 견책을 싫어하는 자는 죽을 것이니라
\par 11 음부와 유명도 여호와의 앞에 드러나거든 하물며 인생의 마음이리요
\par 12 거만한 자는 견책 받기를 좋아하지 아니하며 지혜 있는 자에게로 가지도 아니하느니라
\par 13 마음의 즐거움은 얼굴을 빛나게 하여도 마음의 근심은 심령을 상하게 하느니라
\par 14 명철한 자의 마음은 지식을 요구하고 미련한 자의 입은 미련한 것을 즐기느니라
\par 15 고난 받는 자는 그 날이 다 험악하나 마음이 즐거운자는 항상 잔치하느니라
\par 16 가산이 적어도 여호와를 경외하는 것이 크게 부하고 번뇌하는 것보다 나으니라
\par 17 여간 채소를 먹으며 서로 사랑하는 것이 살진 소를 먹으며 서로 미워하는 것보다 나으니라
\par 18 분을 쉽게 내는 자는 다툼을 일으켜도 노하기를 더디하는 자는 시비를 그치게 하느니라
\par 19 게으른 자의 길은 가시울타리 같으나 정직한 자의 길은 대로니라
\par 20 지혜로운 아들은 아비를 즐겁게 하여도 미련한 자는 어미를 업신여기느니라
\par 21 무지한 자는 미련한 것을 즐겨하여도 명철한 자는 그 길을 바르게 하느니라
\par 22 의논이 없으면 경영이 파하고 모사가 많으면 경영이 성립하느니라
\par 23 사람은 그 입의 대답으로 말미암아 기쁨을 얻나니 때에 맞은 말이 얼마나 아름다운고
\par 24 지혜로운 자는 위로 향한 생명길로 말미암음으로 그 아래 있는 음부를 떠나게 되느니라
\par 25 여호와는 교만한 자의 집을 허시며 과부의 지계를 정하시느니라
\par 26 악한 꾀는 여호와의 미워하시는 것이라도 선한 말은 정결하니라
\par 27 이를 탐하는 자는 자기 집을 해롭게 하나 뇌물을 싫어하는 자는 사느니라
\par 28 의인의 마음은 대답할 말을 깊이 생각하여도 악인의 입은 악을 쏟느니라
\par 29 여호와는 악인을 멀리 하시고 의인의 기도를 들으시느니라
\par 30 눈의 밝은 것은 마음을 기쁘게 하고 좋은 기별은 뼈를 윤택하게 하느니라
\par 31 생명의 경계를 듣는 귀는 지혜로운 자 가운데 있느니라
\par 32 훈계 받기를 싫어하는 자는 자기의 영혼을 경히 여김이라 견책을 달게 받는 자는 지식을 얻느니라
\par 33 여호와를 경외하는 것은 지혜의 훈계라 겸손은 존귀의 앞잡이니 라

\chapter{16}

\par 1 마음의 경영은 사람에게 있어도 말의 응답은 여호와께로서 나느니라
\par 2 사람의 행위가 자기 보기에는 모두 깨끗하여도 여호와는 심령을 감찰하시느니라
\par 3 너의 행사를 여호와께 맡기라 그리하면 너의 경영하는 것이 이루리라
\par 4 여호와께서 온갖 것을 그 씌움에 적당하게 지으셨나니 악인도 악한 날에 적당하게 하셨느니라
\par 5 무릇 마음이 교만한 자를 여호와께서 미워하시나니 피차 손을 잡을지라도 벌을 면치 못하리라
\par 6 인자와 진리로 인하여 죄악이 속하게 되고 여호와를 경외함으로 인하여 악에서 떠나게 되느니라
\par 7 사람의 행위가 여호와를 기쁘시게 하면 그 사람의 원수라도 그로 더불어 화목하게 하시느니라
\par 8 적은 소득이 의를 겸하면 많은 소득이 불의를 겸한 것보다 나으니라
\par 9 사람이 마음으로 자기의 길을 계획할지라도 그 걸음을 인도하는 자는 여호와시니라
\par 10 하나님의 말씀이 왕의 입술에 있은즉 재판할 때에 그 입이 그릇하지 아니하리라
\par 11 공평한 간칭과 명칭은 여호와의 것이요 주머니 속의 추돌들도 다 그의 지으신 것이니라
\par 12 악을 행하는 것은 왕의 미워할 바니 이는 그 보좌가 공의로 말미암아 굳게 섬이니라
\par 13 의로운 입술은 왕들의 기뻐하는 것이요 정직히 말하는 자는 그들의 사랑을 입느니라
\par 14 왕의 진노는 살륙의 사자와 같아도 지혜로운 사람은 그것을 쉬게하리라
\par 15 왕의 희색에 생명이 있나니 그 은택이 늦은 비를 내리는 구름과 같으니라
\par 16 지혜를 얻는 것이 금을 얻는 것보다 얼마나 나은고 명철을 얻는 것이 은을 얻는 것보다 더욱 나으니라
\par 17 악을 떠나는 것은 정직한 사람의 대로니 그 길을 지키는 자는 자기의 영혼을 보전하느니라
\par 18 교만은 패망의 선봉이요 거만한 마음은 넘어짐의 앞잡이니라
\par 19 겸손한 자와 함께하여 마음을 낮추는 것이 교만한 자와 함께 하여 탈취물을 나누는 것보다 나으니라
\par 20 삼가 말씀에 주의하는 자는 좋은 것을 얻나니 여호와를 의지하는자가 복이 있느니라
\par 21 마음이 지혜로운 자가 명철하다 일컬음을 받고 입이 선한 자가 남의 학식을 더하게 하느니라
\par 22 명철한 자에게는 그 명철이 생명의 샘이 되거니와 미련한 자에게는 그 미련한 것이 징계가 되느니라
\par 23 지혜로운 자의 마음은 그 입을 슬기롭게 하고 또 그 입술에 지식을 더하느니라
\par 24 선한 말은 꿀송이 같아서 마음에 달고 뼈에 양약이 되느니라
\par 25 어떤 길은 사람의 보기에 바르나 필경은 사망의 길이니라
\par 26 노력하는 자는 식욕을 인하여 애쓰나니 이는 그 입이 자기를 독촉함이니라
\par 27 불량한 자는 악을 꾀하나니 그 입술에는 맹렬한 불 같은 것이 있느니라
\par 28 패려한 자는 다툼을 일으키고 말장이는 친한 벗을 이간하느니라
\par 29 강포한 사람은 그 이웃을 꾀어 불선한 길로 인도하느니라
\par 30 눈을 감는 자는 패역한 일을 도모하며 입술을 닫는 자는 악한 일을 이루느니라
\par 31 백발은 영화의 면류관이라 의로운 길에서 얻으리라
\par 32 노하기를 더디하는 자는 용사보다 낫고 자기의 마음을 다스리는 자는 성을 빼앗는 자보다 나으니라
\par 33 사람이 제비는 뽑으나 일을 작정하기는 여호와께 있느니라

\chapter{17}

\par 1 마른 떡 한 조각만 있고도 화목하는 것이 육선이 집에 가득하고 다투는 것보다 나으니라
\par 2 슬기로운 종은 주인의 부끄러움을 끼치는 아들을 다스리겠고 또 그 아들들 중에서 유업을 나눠 얻으리라
\par 3 "도가니는 은을, 풀무는 금을 연단하거니와 여호와는 마음을 연단 하시느니라"
\par 4 악을 행하는 자는 궤사한 입술을 잘 듣고 거짓말을 하는 자는 악한 혀에 귀를 기울이느니라
\par 5 가난한 자를 조롱하는 자는 이를 지으신 주를 멸시하는 자요 사람의 재앙을 기뻐하는 자는 형벌을 면치 못할 자니라
\par 6 손자는 노인의 면류관이요 아비는 자식의 영화니라
\par 7 분외의 말을 하는 것도 미련한 자에게 합당치 아니하거든 하물며 거짓말을 하는 것이 존귀한 자에게 합당하겠느냐
\par 8 뇌물은 임자의 보기에 보석 같은즉 어디로 향하든지 형통케 하느니라
\par 9 허물을 덮어 주는 자는 사랑을 구하는 자요 그것을 거듭 말하는 자는 친한 벗을 이간하는 자니라
\par 10 한 마디로 총명한 자를 경계하는 것이 매 백개로 미련한 자를 때리는 것보다 더욱 깊이 박이느니라
\par 11 악한 자는 반역만 힘쓰나니 그러므로 그에게 잔인한 사자가 보냄을 입으리라
\par 12 차라리 새끼 빼앗긴 암콤을 만날지언정 미련한 일을 행하는 미련한 자를 만나지 말 것이니라
\par 13 누구든지 악으로 선을 갚으면 악이 그 집을 떠나지 아니하리라
\par 14 다투는 시작은 방축에서 물이 새는 것 같은즉 싸움이 일어나기 전에 시비를 그칠 것이니라
\par 15 악인을 의롭다 하며 의인을 악하다 하는 이 두 자는 다 여호와의 미워하심을 입느니라
\par 16 미련한 자는 무지하거늘 손에 값을 가지고 지혜를 사려 함은 어찜인고
\par 17 친구는 사랑이 끊이지 아니하고 형제는 위급한 때까지 위하여 났느니라
\par 18 지혜없는 자는 남의 손을 잡고 그 이웃 앞에서 보증이 되느니라
\par 19 다툼을 좋아하는 자는 죄과를 좋아하는 자요 자기 문을 높이는 자는 파괴를 구하는 자니라
\par 20 마음이 사특한 자는 복을 얻지 못하고 혀가 패역한 자는 재앙에 빠지느니라
\par 21 미련한 자를 낳는 자는 근심을 당하나니 미련한 자의 아비는 낙이 없느니라
\par 22 마음의 즐거움은 양약이라도 심령의 근심은 뼈로 마르게 하느니라
\par 23 악인은 사람의 품에서 뇌물을 받고 재판을 굽게 하느니라
\par 24 지혜는 명철한 자의 앞에 있거늘 미련한 자는 눈을 땅 끝에 두느니라
\par 25 미련한 아들은 그 아비의 근심이 되고 그 어미의 고통이 되느니 라
\par 26 의인을 벌하는 것과 귀인을 정직하다고 때리는 것이 선치 못하니라
\par 27 말을 아끼는 자는 지식이 있고 성품이 안존한 자는 명철하니라
\par 28 미련한 자라도 잠잠하면 지혜로운 자로 여기우고 그 입술을 닫히면 슬기로운 자로 여기우느니라

\chapter{18}

\par 1 무리에게서 스스로 나뉘는 자는 자기 소욕을 따르는 자라 온갖 참 지혜를 배척하느니라
\par 2 미련한 자는 명철을 기뻐하지 아니하고 자기의 의사를 드러내기만 기뻐하느니라
\par 3 악한 자가 이를 때에는 멸시도 따라오고 부끄러운 것이 이를 때에는 능욕도 함께 오느니라
\par 4 명철한 사람의 입의 말은 깊은 물과 같고 지혜의 샘은 솟쳐 흐르는 내와 같으니라
\par 5 악인을 두호하는 것과 재판할 때에 의인을 억울하게 하는 것이 선하지 아니하니라
\par 6 미련한 자의 입술은 다툼을 일으키고 그 입은 매를 자청하느니라
\par 7 미련한 자의 입은 그의 멸망이 되고 그 입술은 그의 영혼의 그물이 되느니라
\par 8 남의 말하기를 좋아하는 자의 말은 별식과 같아서 뱃 속 깊은 데로 내려가느니라
\par 9 자기의 일을 게을리 하는 자는 패가 하는 자의 형제니라
\par 10 여호와의 이름은 견고한 망대라 의인은 그리로 달려가서 안전함을 얻느니라
\par 11 부자의 재물은 그의 견고한 성이라 그가 높은 성벽 같이 여기느니라
\par 12 사람의 마음의 교만은 멸망의 선봉이요 겸손은 존귀의 앞잡이니라
\par 13 사연을 듣기 전에 대답하는 자는 미련하여 욕을 당하느니라
\par 14 사람의 심령은 그 병을 능히 이기려니와 심령이 상하면 그것을 누가 일으키겠느냐
\par 15 명철한 자의 마음은 지식을 얻고 지혜로운 자의 귀는 지식을 구하느니라
\par 16 선물은 그 사람의 길을 너그럽게 하며 또 존귀한 자의 앞으로 그를 인도하느니라
\par 17 송사에 원고의 말이 바른 것 같으나 그 피고가 와서 밝히느니라
\par 18 제비 뽑는 것은 다툼을 그치게 하여 강한 자 사이에 해결케 하느니라
\par 19 노엽게 한 형제와 화목하기가 견고한 성을 취하기 보다 어려운즉 이러한 다툼은 산성 문빗장 같으니라
\par 20 사람은 입에서 나오는 열매로 하여 배가 부르게 되나니 곧 그 입술에서 나는 것으로하여 만족케 되느니라
\par 21 죽고 사는 것이 혀의 권세에 달렸나니 혀를 쓰기 좋아하는 자는 그 열매를 먹으리라
\par 22 아내를 얻는 자는 복을 얻고 여호와께 은총을 받는 자니라
\par 23 가난한 자는 간절한 말로 구하여도 부자는 엄한 말로 대답하느니라
\par 24 많은 친구를 얻는 자는 해를 당하게 되거니와 어떤 친구는 형제보다 친밀하니라

\chapter{19}

\par 1 성실히 행하는 가난한 자는 입술이 패려하고 미련한 자 보다 나으니라
\par 2 지식 없는 소원은 선치 못하고 발이 급한 사람은 그릇하느니라
\par 3 사람이 미련하므로 자기 길을 굽게 하고 마음으로 여호와를 원망하느니라
\par 4 재물은 많은 친구를 더하게 하나 가난한즉 친구가 끊어지느니라
\par 5 거짓 증인은 벌을 면치 못할 것이요 거짓말을 내는 자도 피치 못하리라
\par 6 너그러운 사람에게는 은혜를 구하는 자가 많고 선물을 주기를 좋아하는 자에게는 사람마다 친구가 되느니라
\par 7 가난한 자는 그 형제들에게도 미움을 받거든 하물며 친구야 그를 멀리 아니하겠느냐 따라가며 말하려 할지라도 그들이 없어졌으리라
\par 8 지혜를 얻는 자는 자기 영혼을 사랑하고 명철을 지키는 자는 복을 얻느니라
\par 9 거짓 증인은 벌을 면치 못할 것이요 거짓말을 내는 자는 망할 것이니라
\par 10 미련한 자가 사치하는 것이 적당치 못하거든 하물며 종이 방백을 다스림이랴
\par 11 노하기를 더디하는 것이 사람의 슬기요 허물을 용서하는 것이 자기의 영광이니라
\par 12 왕의 노함은 사자의 부르짖음 같고 그의 은택은 풀 위에 이슬 같으니라
\par 13 미련한 아들은 그 아비의 재앙이요 다투는 아내는 이어 떨어지는 물방울이니라
\par 14 집과 재물은 조상에게서 상속하거니와 슬기로운 아내는 여호와께로서 말미암느니라
\par 15 게으름이 사람으로 깊이 잠들게 하나니 해태한 사람은 주릴 것이니라
\par 16 계명을 지키는 자는 자기의 영혼을 지키거니와 그 행실을 삼가지 아니하는 자는 죽으리라
\par 17 가난한 자를 불쌍히 여기는 것은 여호와께 꾸이는 것이니 그 선행을 갚아 주시리라
\par 18 네가 네 아들에게 소망이 있은즉 그를 징계하고 죽일 마음은 두지 말지니라
\par 19 노하기를 맹렬히 하는 자는 벌을 받을 것이라 네가 그를 건져 주면 다시 건져 주게 되리라
\par 20 너는 권고를 들으며 훈계를 받으라 그리하면 네가 필경은 지혜롭게 되리라
\par 21 사람의 마음에는 많은 계획이 있어도 오직 여호와의 뜻이 완전히 서리라
\par 22 사람은 그 인자함으로 남에게 사모함을 받느니라 가난한 자는 거짓말하는 자보다 나으니라
\par 23 여호와를 경외하는 것은 사람으로 생명에 이르게 하는 것이라 경외하는 자는 족하게 지내고 재앙을 만나지 아니하느니라
\par 24 게으른 자는 그 손을 그릇에 넣고도 입으로 올리기를 괴로와하느니라
\par 25 거만한 자를 때리라 그리하면 어리석은 자도 경성하리라 명철한 자를 견책하라 그리하면 그가 지식을 얻으리라
\par 26 아비를 구박하고 어미를 쫓아 내는 자는 부끄러움을 끼치며 능욕을 부르는 자식이니라
\par 27 내 아들아 지식의 말씀에서 떠나게 하는 교훈을 듣지 말지니라
\par 28 망령된 증인은 공의를 업신여기고 악인의 입은 죄악을 삼키느니라
\par 29 심판은 거만한 자를 위하여 예비된 것이요 채찍은 어리석은 자의 등을 위하여 예비된 것이니라

\chapter{20}

\par 1 포도주는 거만케 하는 것이요 독주는 떠들게 하는 것이라 무릇 이에 미혹되는 자에게는 지혜가 없느니라
\par 2 왕의 진노는 사자의 부르짖음 같으니 그를 노하게 하는 것은 자기의 생명을 해하는 것이니라
\par 3 다툼을 멀리 하는 것이 사람에게 영광이어늘 미련한 자마다 다툼을 일으키느니라
\par 4 게으른 자는 가을에 밭 같지 아니하나니 그러므로 거둘 때에는 구걸할지라도 얻지 못하리라
\par 5 사람의 마음에 있는 모략은 깊은 물 같으니라 그럴찌라도 명철한 사람은 그것을 길어 내느니라
\par 6 많은 사람은 각기 자기의 인자함을 자랑하나니 충성된 자를 누가 만날 수 있으랴
\par 7 완전히 행하는 자가 의인이라 그 후손에게 복이 있느니라
\par 8 심판 자리에 앉은 왕은 그 눈으로 모든 악을 흩어지게 하느니라
\par 9 "내가 내 마음을 정하게 하였다, 내 죄를 깨끗하게 하였다 할 자가 누구뇨"
\par 10 한결 같지 않은 저울 추와 말은 다 여호와께서 미워하시느니라
\par 11 비록 아이라도 그 동작으로 자기의 품행의 청결하며 정직한 여부를 나타내느니라
\par 12 듣는 귀와 보는 눈은 다 여호와의 지으신 것이니라
\par 13 너는 잠자기를 좋아하지 말라 네가 빈궁하게 될까 두려우니라 네눈을 뜨라 그리하면 양식에 족하리라
\par 14 사는 자가 물건이 좋지 못하다 좋지 못하다 하다가 돌아간 후에는 자랑하느니라
\par 15 세상에 금도 있고 진주도 많거니와 지혜로운 입술이 더욱 귀한 보배니라
\par 16 타인을 위하여 보증이 된 자의 옷을 취하라 외인들의 보증이 된 자는 그 몸을 볼모 잡힐지니라
\par 17 속이고 취한 식물은 맛이 좋은듯하나 후에는 그 입에 모래가 가득하게 되리라
\par 18 무릇 경영은 의논함으로 성취하나니 모략을 베풀고 전쟁할지니라
\par 19 두루 다니며 한담하는 자는 남의 비밀를 누설하나니 입술을 벌린자를 사귀지 말지니라
\par 20 자기의 아비나 어미를 저주하는 자는 그 등불이 유암중에 꺼짐을 당하리라
\par 21 처음에 속히 잡은 산업은 마침내 복이 되지 아니하느니라
\par 22 너는 악을 갚겠다 말하지 말고 여호와를 기다리라 그가 너를 구원하시리라
\par 23 한결 같지 않은 저울 추는 여호와의 미워하시는 것이요 속이는 저울은 좋지 못한 것이니라
\par 24 사람의 걸음은 여호와께로서 말미암나니 사람이 어찌 자기의 길을 알 수 있으랴
\par 25 함부로 이 물건을 거룩하다하여 서원하고 그 후에 살피면 그것이 그물이 되느니라
\par 26 지혜로운 왕은 악인을 키질하며 타작하는 바퀴로 그 위에 굴리느니라
\par 27 사람의 영혼은 여호와의 등불이라 사람의 깊은 속을 살피느니라
\par 28 왕은 인자와 진리로 스스로 보호하고 그 위도 인자함으로 말미암아 견고하니라
\par 29 젊은 자의 영화는 그 힘이요 늙은 자의 아름다운 것은 백발이니라
\par 30 상하게 때리는 것이 악을 없이 하나니 매는 사람의 속에 깊이 들어가느니라

\chapter{21}

\par 1 왕의 마음이 여호와의 손에 있음이 마치 보의 물과 같아서 그가 임의로 인도하시느니라
\par 2 사람의 행위가 자기 보기에는 모두 정직하여도 여호와는 심령을 감찰하시느니라
\par 3 의와 공평을 행하는 것은 제사 드리는 것보다 여호와께서 기쁘게 여기시느니라
\par 4 눈이 높은 것과 마음이 교만한 것과 악인의 형통한 것은 다 죄니라
\par 5 부지런한 자의 경영은 풍부함에 이를 것이나 조급한 자는 궁핍함에 이를 따름이니라
\par 6 속이는 말로 재물을 모으는 것은 죽음을 구하는 것이라 곧 불려 다니는 안개니라
\par 7 악인의 강포는 자기를 소멸하나니 이는 공의 행하기를 싫어함이니라
\par 8 죄를 크게 범한 자의 길은 심히 구부러지고 깨끗한 자의 길은 곧으니라
\par 9 다투는 여인과 함께 큰 집에서 나는 것보다 움막에서 혼자 사는 것이 나으니라
\par 10 악인의 마음은 남의 재앙을 원하나니 그 이웃도 그 앞에서 은혜를 입지 못하느니라
\par 11 거만한 자가 벌을 받으면 어리석은 자는 경성하겠고 지혜로운 자가 교훈을 받으면 지식이 더 하리라
\par 12 의로우신 자는 악인의 집을 감찰하시고 악인을 환난에 던지시느니라
\par 13 귀를 막아 가난한 자의 부르짖는 소리를 듣지 아니하면 자기의 부르짖을 때에도 들을 자가 없으리라
\par 14 은밀한 선물은 노를 쉬게 하고 품의 뇌물은 맹렬한 분을 그치게 하느니라
\par 15 공의를 행하는 것이 의인에게는 즐거움이요 죄인에게는 패망이니라
\par 16 명철의 길을 떠난 사람은 사망의 회중에 거하리라
\par 17 연락을 좋아하는 자는 가난하게 되고 술과 기름을 좋아하는 자는 부하게 되지 못하느니라
\par 18 악인은 의인의 대속이 되고 궤사한 자는 정직한 자의 대신이 되느니라
\par 19 다투며 성내는 여인과 함께 사는 것보다 광야에서 혼자 사는 것이 나으니라
\par 20 지혜있는 자의 집에는 귀한 보배와 기름이 있으나 미련한 자는 이것을 다 삼켜버리느니라
\par 21 의와 인자를 따라 구하는 자는 생명과 의와 영광을 얻느니라
\par 22 지혜로운 자는 용사의 성에 올라가서 그 성의 견고히 의뢰하는 것을 파하느니라
\par 23 입과 혀를 지키는 자는 그 영혼을 환난에서 보전하느니라
\par 24 무례하고 교만한 자를 이름하여 망령된 자라 하나니 이는 넘치는 교만으로 행함이니라
\par 25 게으른 자의 정욕이 그를 죽이나니 이는 그 손으로 일하기를 싫어 함이니라
\par 26 어떤 자는 종일토록 탐하기만 하나 의인은 아끼지 아니하고 시제하느니라
\par 27 악인의 제물은 본래 가증하거든 하물며 악한 뜻으로 드리는 것이랴
\par 28 거짓 증인은 패망하려니와 확실한 증인의 말은 힘이 있느니라
\par 29 악인은 그 얼굴을 굳게 하나 정직한 자는 그 행위를 삼가느니라
\par 30 지혜로도 명철로도 모략으로도 여호와를 당치 못하느니라
\par 31 싸울 날을 위하여 마병을 예비하거니와 이김은 여호와께 있느니라

\chapter{22}

\par 1 많은 재물보다 명예를 택할 것이요 은이나 금보다 은총을 더욱 택할 것이니라
\par 2 빈부가 섞여 살거니와 무릇 그들을 지으신 이는 여호와시니라
\par 3 슬기로운 자는 재앙을 보면 숨어 피하여도 어리석은 자들은 나아가다가 해를 받느니라
\par 4 겸손과 여호와를 경외함의 보응은 재물과 영광과 생명이니라
\par 5 패역한 자의 길에는 가시와 올무가 있거니와 영혼을 지키는 자는 이를 멀리 하느니라
\par 6 마땅히 행할 길을 아이에게 가르치라 그리하면 늙어도 그것을 떠나지 아니하리라
\par 7 부자는 가난한 자를 주관하고 빚진 자는 채주의 종이 되느니라
\par 8 악을 뿌리는 자는 재앙을 거두리니 그 분노의 기세가 쇠하리라
\par 9 선한 눈을 가진 자는 복을 받으리니 이는 양식을 가난한 자에게 줌이니라
\par 10 거만한 자를 쫓아내면 다툼이 쉬고 싸움과 수욕이 그치느니라
\par 11 마음의 정결을 사모하는 자의 입술에는 덕이 있으므로 임금이 그의 친구가 되느니라
\par 12 여호와께서는 지식있는 자를 그 눈으로 지키시나 궤사한 자의 말은 패하게 하시느니라
\par 13 게으른 자는 말하기를 사자가 밖에 있은즉 내가 나가면 거리에서 찢기겠다 하느니라
\par 14 음녀의 입은 깊은 함정이라 여호와의 노를 당한 자는 거기 빠지리라
\par 15 아이의 마음에는 미련한 것이 얽혔으나 징계하는 채찍이 이를 멀리 쫓아내리라
\par 16 이를 얻으려고 가난한 자를 학대하는 자와 부자에게 주는 자는 가난하여질 뿐이니라
\par 17 너는 귀를 기울여 지혜있는 자의 말씀을 들으며 내 지식에 마음을 둘지어다
\par 18 이것을 네 속에 보존하며 네 입술에 있게 함이 아름다우니라
\par 19 내가 너로 여호와를 의뢰하게 하려 하여 이것을 오늘 특별히 네게 알게 하였노니
\par 20 내가 모략과 지식의 아름다운 것을 기록하여
\par 21 너로 진리의 확실한 말씀을 깨닫게 하며 또 너를 보내는 자에게 진리의 말씀으로 회답하게 하려 함이 아니냐
\par 22 약한 자를 약하다고 탈취하지 말며 곤고한 자를 성문에서 압제하지 말라
\par 23 대저 여호와께서 신원하여 주시고 또 그를 노략하는 자의 생명을 빼앗으시리라
\par 24 노를 품는 자와 사귀지 말며 울분한 자와 동행하지 말지니
\par 25 그 행위를 본받아서 네 영혼을 올무에 빠칠까 두려움이니라
\par 26 너는 사람으로 더불어 손을 잡지 말며 남의 빚에 보증이 되지 말라
\par 27 만일 갚을 것이 없으면 네 누운 침상도 빼앗길 것이라 네가 어찌그리하겠느냐
\par 28 네 선조의 세운 옛 지계석을 옮기지 말지니라
\par 29 네가 자기 사업에 근실한 사람을 보았느냐 이러한 사람은 왕 앞에 설 것이요 천한 자 앞에 서지 아니하리라

\chapter{23}

\par 1 네가 관원과 함께 앉아 음식을 먹게 되거든 삼가 네 앞에 있는 자가 누구인지 생각하며
\par 2 네가 만일 탐식자여든 네 목에 칼을 둘 것이니라
\par 3 그 진찬을 탐하지 말라 그것은 간사하게 베푼 식물이니라
\par 4 부자 되기에 애쓰지 말고 네 사사로운 지혜를 버릴지어다
\par 5 네가 어찌 허무한 것에 주목하겠느냐 정녕히 재물은 날개를 내어 하늘에 나는 독수리처럼 날아가리라
\par 6 악한 눈이 있는 자의 음식을 먹지 말며 그 진찬을 탐하지 말지어다
\par 7 대저 그 마음의 생각이 어떠하면 그 위인도 그러한즉 그가 너더러 먹고 마시라 할지라도 그 마음은 너와 함께하지 아니함이라
\par 8 네가 조금 먹은 것도 토하겠고 네 아름다운 말도 헛된 데로 돌아가리라
\par 9 미련한 자의 귀에 말하지 말지니 이는 그가 네 지혜로운 말을 업신여길 것임이니라
\par 10 옛 지계석을 옮기지 말며 외로운 자식의 밭을 침범하지 말찌어다
\par 11 대저 그들의 구속자는 강하시니 너를 대적하사 그 원을 펴시리라
\par 12 훈계에 착심하며 지식의 말씀에 귀를 기울이라
\par 13 아이를 훈계하지 아니치 말라 채찍으로 그를 때릴지라도 죽지 아니하리라
\par 14 그를 채찍으로 때리면 그 영혼을 음부에서 구원하리라
\par 15 내 아들아 만일 네 마음이 지혜로우면 나 곧 내 마음이 즐겁겠고
\par 16 만일 네 입술이 정직을 말하면 내 속이 유쾌하리라
\par 17 네 마음으로 죄인의 형통을 부러워하지 말고 항상 여호와를 경외하라
\par 18 정녕히 네 장래가 있겠고 네 소망이 끊어지지 아니하리라
\par 19 내 아들아 너는 듣고 지혜를 얻어 네 마음을 정로로 인도할지니라
\par 20 술을 즐겨하는 자와 고기를 탐하는 자로 더불어 사귀지 말라
\par 21 술 취하고 탐식하는 자는 가난하여질 것이요 잠자기를 즐겨하는자는 해어진 옷을 입을 것임이니라
\par 22 너 낳은 아비에게 청종하고 네 늙은 어미를 경히 여기지 말지니라
\par 23 진리를 사고서 팔지 말며 지혜와 훈계와 명철도 그리할지니라
\par 24 의인의 아비는 크게 즐거울 것이요 지혜로운 자식을 낳은 자는 그를 인하여 즐거울 것이니라
\par 25 네 부모를 즐겁게 하며 너 낳은 어미를 기쁘게 하라
\par 26 내 아들아 네 마음을 내게 주며 네 눈으로 내 길을 즐거워할지어다
\par 27 대저 음녀는 깊은 구렁이요 이방 여인은 좁은 함정이라
\par 28 그는 강도 같이 매복하며 인간에 궤사한 자가 많아지게 하느니라
\par 29 재앙이 뉘게 있느뇨 근심이 뉘게 있느뇨 분쟁이 뉘게 있느뇨 원망이 뉘게 있느뇨 까닭없는 창상이 뉘게 있느뇨 붉은 눈이 뉘게 있느뇨
\par 30 술에 잠긴 자에게 있고 혼합한 술을 구하러 다니는 자에게 있느니라
\par 31 포도주는 붉고 잔에서 번쩍이며 순하게 내려가나니 너는 그것을 보지도 말지어다
\par 32 이것이 마침내 뱀 같이 물 것이요 독사 같이 쏠 것이며
\par 33 또 네 눈에는 괴이한 것이 보일 것이요 네 마음은 망령된 것을 발할 것이며
\par 34 너는 바다 가운데 누운 자 같을 것이요 돛대 위에 누운 자 같을 것이며
\par 35 네가 스스로 말하기를 사람이 나를 때려도 나는 아프지 아니하고 나를 상하게 하여도 내게 감각이 없도다 내가 언제나 깰까 다시 술을 찾겠다 하리라

\chapter{24}

\par 1 너는 악인의 형통을 부러워하지 말며 그와 함께 있기도 원하지 말지어다
\par 2 그들의 마음은 강포를 품고 그 입술은 잔해를 말함이니라
\par 3 집은 지혜로 말미암아 건축되고 명철로 말미암아 견고히 되며
\par 4 또 방들은 지식으로 말미암아 각종 귀하고 아름다운 보배로 채우게 되느니라
\par 5 지혜 있는 자는 강하고 지식 있는 자는 힘을 더하나니
\par 6 너는 모략으로 싸우라 승리는 모사가 많음에 있느니라
\par 7 지혜는 너무 높아서 미련한 자의 미치지 못할 것이므로 그는 성 문에서 입을 열지 못하느니라
\par 8 악을 행하기를 꾀하는 자를 일컬어 사특한 자라 하느니라
\par 9 미련한 자의 생각은 죄요 거만한 자는 사람의 미움을 받느니라
\par 10 네가 만일 환난날에 낙담하면 네 힘의 미약함을 보임이니라
\par 11 너는 사망으로 끌려가는 자를 건져주며 살륙을 당하게 된 자를 구원하지 아니치 말라
\par 12 네가 말하기를 나는 그것을 알지 못하였노라 할지라도 마음을 저울질 하시는 이가 어찌 통찰하지 못하시겠으며 네 영혼을 지키시는 이가 어찌 알지 못하시겠느냐 그가 각 사람의 행위대로 보증 하시리라
\par 13 내 아들아 꿀을 먹으라 이것이 좋으니라 송이꿀을 먹으라 이것이 네 입에 다니라
\par 14 지혜가 네 영혼에게 이와 같은 줄을 알라 이것을 얻으면 정녕히 네 장래가 있겠고 네 소망이 끊어지지 아니하리라
\par 15 악한 자여 의인의 집을 엿보지 말며 그 쉬는 처소를 헐지 말지니라
\par 16 대저 의인은 일곱번 넘어질지라도 다시 일어나려니와 악인은 재앙으로 인하여 엎드러지느니라
\par 17 네 원수가 넘어질 때에 즐거워하지 말며 그가 엎드러질 때에 마음에 기뻐하지 말라
\par 18 여호와께서 이것을 보시고 기뻐 아니하사 그 진노를 그에게서 옮기실까 두려우니라
\par 19 너는 행악자의 득의함을 인하여 분을 품지 말며 악인의 형통을 부러워하지 말라
\par 20 대저 행악자는 장래가 없겠고 악인의 등불은 꺼지리라
\par 21 내 아들아 여호와와 왕을 경외하고 반역자로 더불어 사귀지 말라
\par 22 대저 그들의 재앙은 속히 임하리니 이 두 자의 멸망을 누가 알랴
\par 23 이것도 지혜로운 자의 말씀이라 재판할 때에 낯을 보아주는 것이 옳지 못하니라
\par 24 무릇 악인더러 옳다 하는 자는 백성에게 저주를 받을 것이요 국민에게 미움을 받으려니와
\par 25 오직 그를 견책하는 자는 기쁨을 얻을 것이요 또 좋은 복을 받으리라
\par 26 적당한 말로 대답함은 입맞춤과 같으니라
\par 27 네 일을 밖에서 다스리며 밭에서 예비하고 그 후에 네 집을 세울지니라
\par 28 너는 까닭없이 네 이웃을 쳐서 증인이 되지 말며 네 입술로 속이지 말지니라
\par 29 너는 그가 내게 행함 같이 나도 그에게 행하여 그 행한대로 갚겠다 말하지 말지니라
\par 30 내가 증왕에 게으른 자의 밭과 지혜 없는 자의 포도원을 지나며 본즉
\par 31 가시덤불이 퍼졌으며 거친 풀이 지면에 덮였고 돌담이 무너졌기 로
\par 32 내가 보고 생각이 깊었고 내가 보고 훈계를 받았었노라
\par 33 "네가 좀더 자자, 좀더 졸자, 손을 모으고 좀더 눕자 하니 네 빈궁이 강도 같이 오며"
\par 34 네 곤핍이 군사 같이 이르리라

\chapter{25}

\par 1 이것도 솔로몬의 잠언이요 유다 왕 히스기야의 신하들의 편집한 것이니라
\par 2 일을 숨기는 것은 하나님의 영화요 일을 살피는 것은 왕의 영화니라
\par 3 하늘의 높음과 땅의 깊음 같이 왕의 마음은 헤아릴 수 없느니라
\par 4 은에서 찌끼를 제하라 그리하면 장색의 쓸만한 그릇이 나올 것이요
\par 5 왕 앞에서 악한 자를 제하라 그리하면 그 위가 의로 말미암아 견고히 서리라
\par 6 왕 앞에서 스스로 높은 체 하지말며 대인의 자리에 서지 말라
\par 7 이는 사람이 너더러 이리로 올라오라 하는 것이 네 눈에 보이는 귀인 앞에서 저리로 내려가라 하는 것보다 나음이니라
\par 8 너는 급거히 나가서 다투지 말라 마침내 네가 이웃에게 욕을 보게 될 때에 네가 어찌 할 줄을 알지 못할까 두려우니라
\par 9 너는 이웃과 다투거든 변론만 하고 남의 은밀한 일을 누설하지 말라
\par 10 듣는 자가 너를 꾸짖을 터이요 또 수욕이 네게서 떠나지 아니할까 두려우니라
\par 11 경우에 합당한 말은 아로새긴 은쟁반에 금사과니라
\par 12 슬기로운 자의 책망은 청종하는 귀에 금고리와 정금 장식이니라
\par 13 충성된 사자는 그를 보낸 이에게 마치 추수하는 날에 얼음 냉수 같아서 능히 그 주인의 마음을 시원케 하느니라
\par 14 선물한다고 거짓 자랑하는 자는 비 없는 구름과 바람 같으니라
\par 15 오래 참으면 관원이 그 말을 용납하나니 부드러운 혀는 뼈를 꺾느니라
\par 16 너는 꿀을 만나거든 족하리만큼 먹으라 과식하므로 토할까 두려우니라
\par 17 너는 이웃집에 자주 다니지 말라 그가 너를 싫어하며 미워할까 두려우니라
\par 18 그 이웃을 쳐서 거짓 증거하는 사람은 방망이요 칼이요 뾰족한 살이니라
\par 19 환난날에 진실치 못한 자를 의뢰하는 의뢰는 부러진 이와 위골된 발 같으니라
\par 20 마음이 상한 자에게 노래하는 것은 추운 날에 옷을 벗음 같고 쏘다 위에 초를 부음 같으니라
\par 21 네 원수가 배고파하거든 식물을 먹이고 목말라하거든 물을 마시우라
\par 22 그리하는 것은 핀 숯으로 그의 머리에 놓는 것과 일반이요 여호와께서는 네게 상을 주시리라
\par 23 북풍이 비를 일으킴 같이 참소하는 혀는 사람의 얼굴에 분을 일으키느니라
\par 24 다투는 여인과 함께 큰 집에서 사는 것보다 움막에서 혼자 사는 것이 나으니라
\par 25 먼 땅에서 오는 좋은 기별은 목마른 사람에게 냉수 같으니라
\par 26 의인이 악인 앞에 굴복하는 것은 우물의 흐리어짐과 샘의 더러워짐 같으니라
\par 27 꿀을 많이 먹는 것이 좋지 못하고 자기의 영예를 구하는 것이 헛되니라
\par 28 자기의 마음을 제어하지 아니하는 자는 성읍이 무너지고 성벽이 없는 것 같으니라

\chapter{26}

\par 1 미련한 자에게는 영예가 적당하지 아니하니 마치 여름에 눈오는 것과 추수 때에 비오는 것 같으니라
\par 2 까닭 없는 저주는 참새의 떠도는 것과 제비의 날아가는 것 같이 이르지 아니하느니라
\par 3 말에게는 채찍이요 나귀에게는 자갈이요 미련한 자의 등에는 막대기니라
\par 4 미련한 자의 어리석은 것을 따라 대답하지 말라 두렵건대 네가 그와 같을까 하노라
\par 5 미련한 자의 어리석은 것을 따라 그에게 대답하라 두렵건대 그가 스스로 지혜롭게 여길까 하노라
\par 6 미련한자 편에 기별하는 것은 자기의 발을 베어 버림이라 해를 받느니라
\par 7 저는 자의 다리는 힘 없이 달렸나니 미련한 자의 입의 잠언도 그러하니라
\par 8 미련한 자에게 영예를 주는 것은 돌을 물매에 매는 것과 같으니라
\par 9 미련한 자의 입의 잠언은 술 취한 자의 손에 든 가시나무 같으니라
\par 10 장인이 온갖 것을 만들지라도 미련한 자를 고용하는 것은 지나가는 자를 고용함과 같으니라
\par 11 개가 그 토한 것을 도로 먹는 것 같이 미련한 자는 그 미련한 것을 거듭 행하느니라
\par 12 네가 스스로 지혜롭게 여기는 자를 보느냐 그보다 미련한 자에게 오히려 바랄 것이 있느니라
\par 13 게으른 자는 길에 사자가 있다 거리에 사자가 있다 하느니라
\par 14 문짝이 돌쩌귀를 따라서 도는 것 같이 게으른 자는 침상에서 구으느니라
\par 15 게으른 자는 그 손을 그릇에 넣고도 입으로 올리기를 괴로와 하느니라
\par 16 게으른 자는 선히 대답하는 사람 일곱보다 자기를 지혜롭게 여기느니라
\par 17 길로 지나다가 자기에게 상관없는 다툼을 간섭하는 자는 개 귀를 잡는 자와 같으니라
\par 18 횃불을 던지며 살을 쏘아서 사람을 죽이는 미친 사람이 있나니
\par 19 자기 이웃을 속이고 말하기를 내가 희롱하였노라 하는 자도 그러하니라
\par 20 나무가 다하면 불이 꺼지고 말장이가 없어지면 다툼이 쉬느니라
\par 21 숯불 위에 숯을 더하는 것과 타는 불에 나무를 더하는 것 같이 다툼을 좋아하는 자는 시비를 일으키느니라
\par 22 남의 말하기를 좋아하는 자의 말은 별식과 같아서 뱃 속 깊은 데로 내려가느니라
\par 23 온유한 입술에 악한 마음은 낮은 은을 입힌 토기니라
\par 24 감정있는 자는 입술로는 꾸미고 속에는 궤휼을 품나니
\par 25 그 말이 좋을지라도 믿지 말 것은 그 마음에 일곱 가지 가증한 것이 있음이라
\par 26 궤휼로 그 감정을 감출지라도 그 악이 회중 앞에 드러나리라
\par 27 함정을 파는 자는 그것에 빠질 것이요 돌을 굴리는 자는 도리어 그것에 치이리라
\par 28 거짓말하는 자는 자기의 해한 자를 미워하고 아첨하는 입은 패망을 일으키느니라

\chapter{27}

\par 1 너는 내일 일을 자랑하지 말라 하루 동안에 무슨 일이 날는지 네가 알 수 없음이니라
\par 2 타인으로 너를 칭찬하게 하고 네 입으로는 말며 외인으로 너를 칭찬하게 하고 네 입술로는 말지니라
\par 3 돌은 무겁고 모래도 가볍지 아니하거니와 미련한 자의 분노는 이 둘보다 무거우니라
\par 4 분은 잔인하고 노는 창수 같거니와 투기 앞에야 누가 서리요
\par 5 면책은 숨은 사랑보다 나으니라
\par 6 친구의 통책은 충성에서 말미암은 것이나 원수의 자주 입맞춤은 거짓에서 난 것이니라
\par 7 배부른 자는 꿀이라도 싫어하고 주린 자에게는 쓴 것이라도 다니라
\par 8 본향을 떠나 유리하는 사람은 보금자리를 떠나 떠도는 새와 같으니라
\par 9 기름과 향이 사람의 마음을 즐겁게 하나니 친구의 충성된 권고가 이와 같이 아름다우니라
\par 10 네 친구와 네 아비의 친구를 버리지 말며 네 환난날에 형제의 집에 들어가지 말지어다 가까운 이웃이 먼 형제보다 나으니라
\par 11 내 아들아 지혜를 얻고 내 마음을 기쁘게 하라 그리하면 나를 비방하는 자에게 내가 대답할 수 있겠노라
\par 12 슬기로운 자는 재앙을 보면 숨어 피하여도 어리석은 자들은 나아가다가 해를 받느니라
\par 13 타인을 위하여 보증이 된 자의 옷을 취하라 외인들의 보증이 된 자는 그 몸을 볼모로 잡힐지니라
\par 14 이른 아침에 큰 소리로 그 이웃을 축복하면 도리어 저주 같이 여기게 되리라
\par 15 다투는 부녀는 비오는 날에 이어 떨어지는 물방울이라
\par 16 그를 제어하기가 바람을 제어하는 것 같고 오른손으로 기름을 움키는 것 같으니라
\par 17 철이 철을 날카롭게 하는 것 같이 사람이 그 친구의 얼굴을 빛나게 하느니라
\par 18 무화과나무를 지키는 자는 그 과실을 먹고 자기 주인을 시종하는 자는 영화를 얻느니라
\par 19 물에 비취이면 얼굴이 서로 같은 것 같이 사람의 마음도 서로 비취느니라
\par 20 음부와 유명은 만족함이 없고 사람의 눈도 만족함이 없느니라
\par 21 "도가니로 은을, 풀무로 금을, 칭찬으로 사람을 시련하느니라"
\par 22 미련한 자를 곡물과 함께 절구에 넣고 공이로 찧을지라도 그의 미련은 벗어지지 아니하느니라
\par 23 네 양떼의 형편을 부지런히 살피며 네 소떼에 마음을 두라
\par 24 대저 재물은 영영히 있지 못하나니 면류관이 어찌 대대에 있으랴
\par 25 풀을 벤 후에는 새로 움이 돋나니 산에서 꼴을 거둘 것이니라
\par 26 어린 양의 털은 네 옷이 되며 염소는 밭을 사는 값이 되며
\par 27 염소의 젖은 넉넉하여 너와 네 집 사람의 식물이 되며 네 여종의 먹을 것이 되느니라

\chapter{28}

\par 1 악인은 쫓아 오는 자가 없어도 도망하나 의인은 사자 같이 담대하니라
\par 2 나라는 죄가 있으면 주관자가 많아져도 명철과 지식 있는 사람으로 말미암아 장구하게 되느니라
\par 3 가난한 자를 학대하는 가난한 자는 곡식을 남기지 아니하는 폭우같으니라
\par 4 율법을 버린 자는 악인을 칭찬하나 율법을 지키는 자는 악인을 대적하느니라
\par 5 악인은 공의를 깨닫지 못하나 여호와를 찾는 자는 모든 것을 깨닫느니라
\par 6 성실히 행하는 가난한 자는 사곡히 행하는 부자보다 나으니라
\par 7 율법을 지키는 자는 지혜로운 아들이요 탐식자를 사귀는 자는 아비를 욕되게 하는 자니라
\par 8 중한 변리로 자기 재산을 많아지게 하는 것은 가난한 사람 불쌍히 여기는 자를 위하여 그 재산을 저축하는 것이니라
\par 9 사람이 귀를 돌이키고 율법을 듣지 아니하면 그의 기도도 가증하니라
\par 10 정직한 자를 악한 길로 유인하는 자는 스스로 자기 함정에 빠져도 성실한 자는 복을 얻느니라
\par 11 부자는 자기를 지혜롭게 여겨도 명철한 가난한 자는 그를 살펴 아느니라
\par 12 의인이 득의하면 큰 영화가 있고 악인이 일어나면 사람이 숨느니라
\par 13 자기의 죄를 숨기는 자는 형통치 못하나 죄를 자복하고 버리는 자는 불쌍히 여김을 받으리라
\par 14 항상 경외하는 자는 복되거니와 마음을 강퍅하게 하는 자는 재앙에 빠지리라
\par 15 가난한 백성을 압제하는 악한 관원은 부르짖는 사자와 주린 곰 같으니라
\par 16 무지한 치리자는 포학을 크게 행하거니와 탐욕을 미워하는 자는 장수하리라
\par 17 사람의 피를 흘린 자는 함정으로 달려갈 것이니 그를 막지 말지니라
\par 18 성실히 행하는 자는 구원을 얻을 것이나 사곡히 행하는 자는 곧 넘어지리라
\par 19 자기의 토지를 경작하는 자는 먹을 것이 많으려니와 방탕을 좇는자는 궁핍함이 많으리라
\par 20 충성된 자는 복이 많아도 속히 부하고자 하는 자는 형벌을 면치 못하리라
\par 21 사람의 낯을 보아주는 것이 좋지 못하고 한 조각 떡을 인하여 범법하는 것도 그러하니라
\par 22 악한 눈이 있는 자는 재물을 얻기에만 급하고 빈궁이 자기에게로 임할 줄은 알지 못하느니라
\par 23 사람을 경책하는 자는 혀로 아첨하는 자보다 나중에 더욱 사랑을 받느니라
\par 24 부모의 물건을 도적질하고 죄가 아니라 하는 자는 멸망케 하는 자의 동류니라
\par 25 마음이 탐하는 자는 다툼을 일으키나 여호와를 의지하는 자는 풍족하게 되느니라
\par 26 자기의 마음을 믿는 자는 미련한 자요 지혜롭게 행하는 자는 구원을 얻을 자니라
\par 27 가난한 자를 구제하는 자는 궁핍하지 아니 하려니와 못본체 하는자에게는 저주가 많으리라
\par 28 악인이 일어나면 사람이 숨고 그가 멸망하면 의인이 많아지느니라

\chapter{29}

\par 1 자주 책망을 받으면서도 목이 곧은 사람은 갑자기 패망을 당하고 피하지 못하리라
\par 2 의인이 많아지면 백성이 즐거워하고 악인이 권세를 잡으면 백성이 탄식하느니라
\par 3 지혜를 사모하는 자는 아비를 즐겁게 하여도 창기를 사귀는 자는 재물을 없이하느니라
\par 4 왕은 공의로 나라를 견고케 하나 뇌물을 억지로 내게 하는 자는 나라를 멸망시키느니라
\par 5 이웃에게 아첨하는 것은 그의 발 앞에 그물을 치는 것이니라
\par 6 악인의 범죄하는 것은 스스로 올무가 되게 하는 것이나 의인은 노래하고 기뻐하느니라
\par 7 의인은 가난한 자의 사정을 알아 주나 악인은 알아 줄 지식이 없느니라
\par 8 모만한 자는 성읍을 요란케 하여도 슬기로운 자는 노를 그치게 하느니라
\par 9 지혜로운 자와 미련한 자가 다투면 지혜로운 자가 노하든지 웃든지 그 다툼이 그침이 없느니라
\par 10 피 흘리기를 좋아하는 자는 온전한 자를 미워하고 정직한 자의 생명을 찾느니라
\par 11 어리석은 자는 그 노를 다 드러내어도 지혜로운 자는 그 노를 억제하느니라
\par 12 관원이 거짓말을 신청하면 그 하인은 다 악하니라
\par 13 가난한 자와 포학한 자가 섞여 살거니와 여호와께서는 그들의 눈에 빛을 주시느니라
\par 14 왕이 가난한 자를 성실히 신원하면 그 위가 영원히 견고하리라
\par 15 채찍과 꾸지람이 지혜를 주거늘 임의로 하게 버려두면 그 자식은 어미를 욕되게 하느니라
\par 16 악인이 많아지면 죄도 많아지나니 의인은 그들의 망함을 보리라
\par 17 네 자식을 징계하라 그리하면 그가 너를 평안하게 하겠고 또 네 마음에 기쁨을 주리라
\par 18 묵시가 없으면 백성이 방자히 행하거니와 율법을 지키는 자는 복이 있느니라
\par 19 종은 말로만 하면 고치지 아니하나니 이는 그가 알고도 청종치 아니함이니라
\par 20 네가 언어에 조급한 사람을 보느냐 그보다 미련한 자에게 오히려 바랄 것이 있느니라
\par 21 종을 어렸을 때부터 곱게 양육하면 그가 나중에는 자식인체하리라
\par 22 노하는 자는 다툼을 일으키고 분하여 하는 자는 범죄함이 많으니라
\par 23 사람이 교만하면 낮아지게 되겠고 마음이 겸손하면 영예를 얻으리라
\par 24 도적과 짝하는 자는 자기의 영혼을 미워하는 자라 그는 맹세함을 들어도 직고하지 아니하느니라
\par 25 사람을 두려워하면 올무에 걸리게 되거니와 여호와를 의지하는 자는 안전하리라
\par 26 주권자에게 은혜를 구하는 자가 많으나 사람의 일의 작정은 여호와께로 말미암느니라
\par 27 불의한 자는 의인에게 미움을 받고 정직한 자는 악인에게 미움을받느니라

\chapter{30}

\par 1 이 말씀은 야게의 아들 아굴의 잠언이니 그가 이디엘과 우갈에게 이른 것이니라
\par 2 나는 다른 사람에게 비하면 짐승이라 내게는 사람의 총명이 있지아니하니라
\par 3 나는 지혜를 배우지 못하였고 또 거룩하신 자를 아는 지식이 없거니와
\par 4 "하늘에 올라갔다가 내려온 자가 누구인지, 바람을 그 장중에 모은자가 누구인지, 물을 옷에 싼자가 누구인지, 땅의 모든 끝을 정한 자가 누구인지, 그이름이 무엇인지, 그 아들의 이름이 무엇인지 너는 아느냐"
\par 5 하나님의 말씀은 다 순전하며 하나님은 그를 의지하는 자의 방패시니라
\par 6 너는 그 말씀에 더하지 말라 그가 너를 책망하시겠고 너는 거짓말 하는 자가 될까 두려우니라
\par 7 내가 두 가지 일을 주께 구하였사오니 나의 죽기 전에 주시옵소서
\par 8 곧 허탄과 거짓말을 내게서 멀리 하옵시며 나로 가난하게도 마옵시고 부하게도 마옵시고 오직 필요한 양식으로 내게 먹이시옵소서
\par 9 혹 내가 배불러서 하나님을 모른다 여호와가 누구냐 할까 하오며 혹 내가 가난하여 도적질하고 내 하나님의 이름을 욕되게 할까 두려워함이니이다
\par 10 너는 종을 그 상전에게 훼방하지 말라 그가 너를 저주하겠고 너는 죄책을 당할까 두려우니라
\par 11 아비를 저주하며 어미를 축복하지 아니하는 무리가 있느니라
\par 12 스스로 깨끗한 자로 여기면서 오히려 그 더러운 것을 씻지 아니하는 무리가 있느니라
\par 13 눈이 심히 높으며 그 눈꺼풀이 높이 들린 무리가 있느니라
\par 14 앞니는 장검 같고 어금니는 군도 같아서 가난한 자를 땅에서 삼키며 궁핍한 자를 사람 중에서 삼키는 무리가 있느니라
\par 15 거머리에게는 두 딸이 있어 다고 다고 하느니라 족한 줄을 알지 못하여 족하다 하지 아니하는 것 서넛이 있나니
\par 16 곧 음부와 아이 배지 못하는 태와 물로 채울 수 없는 땅과 족하다 하지 아니하는 불이니라
\par 17 아비를 조롱하며 어미 순종하기를 싫어하는 자의 눈은 골짜기의 까마귀에게 쪼이고 독수리 새끼에게 먹히리라
\par 18 내가 심히 기이히 여기고도 깨닫지 못하는 것 서넛이 있나니
\par 19 곧 공중에 날아 다니는 독수리의 자취와 바다로 지나다니는 배의 자취와 남자가 여자와 함께 한 자취며
\par 20 음녀의 자취도 그러하니라 그가 먹고 그 입을 씻음 같이 말하기를 내가 악을 행치 아니하였다 하느니라
\par 21 세상을 진동시키며 세상으로 견딜 수 없게 하는 것 서넛이 있나니
\par 22 곧 종이 임금된 것과 미련한 자가 배부른 것과
\par 23 꺼림을 받는 계집이 시집간 것과 계집 종이 주모를 이은 것이니라
\par 24 땅에 작고도 가장 지혜로운 것 넷이 있나니
\par 25 곧 힘이 없는 종류로되 먹을 것을 여름에 예비하는 개미와
\par 26 약한 종류로되 집을 바위 사이에 짓는 사반과
\par 27 임군이 없으되 다 떼를 지어 나아가는 메뚜기와
\par 28 손에 잡힐만하여도 왕궁에 있는 도마뱀이니라
\par 29 잘 걸으며 위풍 있게 다니는 것 서넛이 있나니
\par 30 곧 짐승 중에 가장 강하여 아무 짐승 앞에서도 물러가지 아니하는 사자와
\par 31 사냥개와 수염소와 및 당할 수 없는 왕이니라
\par 32 만일 네가 미련하여 스스로 높은체 하였거나 혹 악한 일을 도모하였거든 네 손으로 입을 막으라
\par 33 대저 젖을 저으면 뻐터가 되고 코를 비틀면 피가 나는 것 같이 노를 격동하면 다툼이 남이니라

\chapter{31}

\par 1 르무엘왕의 말씀한바 곧 그 어머니가 그를 훈계한 잠언이라
\par 2 내 아들아 내가 무엇을 말할꼬 내 태에서 난 아들아 내가 무엇을 말할꼬 서원대로 얻은 아들아 내가 무엇을 말할꼬
\par 3 네 힘을 여자들에게 쓰지 말며 왕들을 멸망시키는 일을 행치 말지어다
\par 4 르무엘아 포도주를 마시는 것이 왕에게 마땅치 아니하고 왕에게 마땅치 아니하며 독주를 찾는 것이 주권자에게 마땅치 않도다
\par 5 술을 마시다가 법을 잊어버리고 모든 간곤한 백성에게 공의를 굽게 할까 두려우니라
\par 6 "독주는 죽게된 자에게, 포도주는 마음에 근심하는 자에게 줄지어다"
\par 7 그는 마시고 빈궁한 것을 잊어버리겠고 다시 그 고통을 기억지 아니하리라
\par 8 너는 벙어리와 고독한 자의 송사를 위하여 입을 열지니라
\par 9 너는 입을 열어 공의로 재판하여 간곤한 자와 궁핍한 자를 신원 할지니라
\par 10 누가 현숙한 여인을 찾아 얻겠느냐 그 값은 진주보다 더하니라
\par 11 그런 자의 남편의 마음은 그를 믿나니 산업이 핍절치 아니하겠으며
\par 12 그런 자는 살아 있는 동안에 그 남편에게 선을 행하고 악을 행치아니하느니라
\par 13 그는 양털과 삼을 구하여 부지런히 손으로 일하며
\par 14 상고의 배와 같아서 먼 데서 양식을 가져오며
\par 15 밤이 새기 전에 일어나서 그 집 사람에게 식물을 나눠주며 여종 에게 일을 정하여 맡기며
\par 16 밭을 간품하여 사며 그 손으로 번 것을 가지고 포도원을 심으며
\par 17 힘으로 허리를 묶으며 그 팔을 강하게 하며
\par 18 자기의 무역하는 것이 이로운 줄을 깨닫고 밤에 등불을 끄지 아니하고
\par 19 손으로 솜뭉치를 들고 손가락으로 가락을 잡으며
\par 20 그는 간곤한 자에게 손을 펴며 궁핍한 자를 위하여 손을 내밀며
\par 21 그 집 사람들은 다 홍색 옷을 입었으므로 눈이 와도 그는 집 사람을 위하여 두려워하지 아니하며
\par 22 그는 자기를 위하여 아름다운 방석을 지으며 세마포와 자색 옷을 입으며
\par 23 그 남편은 그 땅의 장로로 더불어 성문에 앉으며 사람의 아는 바가 되며
\par 24 그는 베로 옷을 지어 팔며 띠를 만들어 상고에게 맡기며
\par 25 능력과 존귀로 옷을 삼고 후일을 웃으며
\par 26 입을 열어 지혜를 베풀며 그 혀로 인애의 법을 말하며
\par 27 그 집안 일을 보살피고 게을리 얻은 양식을 먹지 아니하나니
\par 28 그 자식들은 일어나 사례하며 그 남편은 칭찬하기를
\par 29 덕행 있는 여자가 많으나 그대는 여러 여자보다 뛰어난다 하느니라
\par 30 고운 것도 거짓되고 아름다운 것도 헛되나 오직 여호와를 경외하는 여자는 칭찬을 받을 것이라
\par 31 그 손의 열매가 그에게로 돌아갈 것이요 그 행한 일을 인하여 성문에서 칭찬을 받으리라


\end{document}