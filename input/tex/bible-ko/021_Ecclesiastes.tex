\begin{document}

\title{전도서}


\chapter{1}

\par 1 다윗의 아들 예루살렘 왕 전도자의 말씀이라
\par 2 전도자가 가로되 헛되고 헛되며 헛되고 헛되니 모든 것이 헛되도다
\par 3 사람이 해 아래서 수고하는 모든 수고가 자기에게 무엇이 유익한고
\par 4 한 세대는 가고 한 세대는 오되 땅은 영원히 있도다
\par 5 해는 떴다가 지며 그 떴던 곳으로 빨리 돌아가고
\par 6 바람은 남으로 불다가 북으로 돌이키며 이리 돌며 저리 돌아 불 던 곳으로 돌아가고
\par 7 모든 강물은 다 바다로 흐르되 바다를 채우지 못하며 어느 곳으로 흐르든지 그리로 연하여 흐르느니라
\par 8 만물의 피곤함을 사람이 말로 다 할 수 없나니 눈은 보아도 족함이 없고 귀는 들어도 차지 아니하는도다
\par 9 이미 있던 것이 후에 다시 있겠고 이미 한 일을 후에 다시 할지라 해 아래는 새 것이 없나니
\par 10 무엇을 가리켜 이르기를 보라 이것이 새 것이라 할 것이 있으랴 오래 전 세대에도 이미 있었느니라
\par 11 이전 세대를 기억함이 없으니 장래 세대도 그 후 세대가 기억함이 없으리라
\par 12 나 전도자는 예루살렘에서 이스라엘 왕이 되어
\par 13 마음을 다하며 지혜를 써서 하늘 아래서 행하는 모든 일을 궁구 하며 살핀즉 이는 괴로운 것이니 하나님이 인생들에게 주사 수고하게 하신 것이라
\par 14 내가 해 아래서 행하는 모든 일을 본즉 다 헛되어 바람을 잡으려는 것이로다
\par 15 구부러진 것을 곧게 할 수 없고 이지러진 것을 셀 수 없도다
\par 16 내가 마음 가운데 말하여 이르기를 내가 큰 지혜를 많이 얻었으므로 나보다 먼저 예루살렘에 있던 자보다 낫다 하였나니 곧 내 마음이 지혜와 지식을 많이 만나 보았음이로다
\par 17 내가 다시 지혜를 알고자 하며 미친 것과 미련한 것을 알고자 하여 마음을 썼으나 이것도 바람을 잡으려는 것인줄을 깨달았도다
\par 18 지혜가 많으면 번뇌도 많으니 지식을 더하는 자는 근심을 더하느니라

\chapter{2}

\par 1 나는 내 마음에 이르기를 자 내가 시험적으로 너를 즐겁게 하리니 너는 낙을 누리라 하였으나 본즉 이것도 헛되도다
\par 2 내가 웃음을 논하여 이르기를 미친 것이라 하였고 희락을 논하여 이르기를 저가 무엇을 하는가 하였노라
\par 3 내 마음이 궁구하기를 내가 어떻게 하여야 내 마음에 지혜로 다스림을 받으면서 술로 내 육신을 즐겁게 할까 또 어떻게 하여야 어리석음을 취하여서 천하 인생의 종신토록 생활함에 어떤것이 쾌락인지 알까 하여
\par 4 나의 사업을 크게 하였노라 내가 나를 위하여 집들을 지으며 포도원을 심으며
\par 5 여러 동산과 과원을 만들고 그 가운데 각종 과목을 심었으며
\par 6 수목을 기르는 삼림에 물주기 위하여 못을 팠으며
\par 7 노비는 사기도 하였고 집에서 나게도 하였으며 나보다 먼저 예루살렘에 있던 모든 자보다도 소와 양떼의 소유를 많게 하였으며
\par 8 은금과 왕들의 보배와 여러 도의 보배를 쌓고 또 노래하는 남녀와 인생들의 기뻐하는 처와 첩들을 많이 두었노라
\par 9 내가 이같이 창성하여 나보다 먼저 예루살렘에 있던 모든 자보다 지나고 내 지혜도 내게 여전하여
\par 10 무엇이든지 내 마음이 즐거워하는 것을 내가 막지 아니하였으니 이는 나의 모든 수고를 내 마음이 기뻐하였음이라 이것이 나의 모든 수고로 말미암아 얻은 분복이로다
\par 11 그 후에 본즉 내 손으로 한 모든 일과 수고한 모든 수고가 다 헛되어 바람을 잡으려는 것이며 해 아래서 무익한 것이로다
\par 12 내가 돌이켜 지혜와 망령됨과 어리석음을 보았나니 왕의 뒤에 오는 자는 무슨 일을 행할꼬 행한지 오랜 일일 뿐이리라
\par 13 내가 보건대 지혜가 우매보다 뛰어남이 빛이 어두움보다 뛰어남 같도다
\par 14 지혜자는 눈이 밝고 우매자는 어두움에 다니거니와 이들의 당하는 일이 일반인 줄을 내가 깨닫고
\par 15 심중에 이르기를 우매자의 당한 것을 나도 당하리니 내가 심중에 이르기를 이것도 헛되도다
\par 16 지혜자나 우매자나 영원토록 기억함을 얻지 못하나니 후일에는 다 잊어버린지 오랠 것임이라 오호라 지혜자의 죽음이 우매자의 죽음과 일반이로다
\par 17 이러므로 내가 사는 것을 한하였노니 이는 해 아래서 하는 일이 내게 괴로움이요 다 헛되어 바람을 잡으려는 것임이로다
\par 18 내가 해 아래서 나의 수고한 모든 수고를 한하였노니 이는 내 뒤를 이을 자에게 끼치게 됨이라
\par 19 "그 사람이 지혜자일지, 우매자일지야 누가 알랴마는 내가 해 아래서 내 지혜를 나타내어 수고한 모든 결과를 저가 다 관리하리니 이것도 헛되도다"
\par 20 이러므로 내가 해 아래서 수고한 모든 수고에 대하여 도리어 마음으로 실망케 하였도다
\par 21 어떤 사람은 그 지혜와 지식과 재주를 써서 수고하였어도 그 얻은 것을 수고하지 아니한 자에게 업으로 끼치리니 이것도 헛된 것이라 큰 해로다
\par 22 사람이 해 아래서 수고하는 모든 수고와 마음에 애쓰는 것으로 소득이 무엇이랴
\par 23 일평생에 근심하며 수고하는 것이 슬픔뿐이라 그 마음이 밤에도 쉬지 못하나니 이것도 헛되도다
\par 24 사람이 먹고 마시며 수고하는 가운데서 심령으로 낙을 누리게 하는 것보다 나은 것이 없나니 내가 이것도 본즉 하나님의 손에서 나는 것이로다
\par 25 먹고 즐거워하는 일에 누가 나보다 승하랴
\par 26 하나님이 그 기뻐하시는 자에게는 지혜와 지식과 희락을 주시나 죄인에게는 노고를 주시고 저로 모아 쌓게 하사 하나님을 기뻐하는 자에게 주게 하시나니 이것도 헛되어 바람을 잡으려는 것이로다

\chapter{3}

\par 1 천하에 범사가 기한이 있고 모든 목적이 이룰 때가 있나니
\par 2 날 때가 있고 죽을 때가 있으며 심을 때가 있고 심은 것을 뽑을 때가 있으며
\par 3 죽일 때가 있고 치료시킬 때가 있으며 헐 때가 있고 세울 때가 있으며
\par 4 울 때가 있고 웃을 때가 있으며 슬퍼할 때가 있고 춤출 때가 있으며
\par 5 돌을 던져 버릴 때가 있고 돌을 거둘 때가 있으며 안을 때가 있고 안는 일을 멀리 할 때가 있으며
\par 6 찾을 때가 있고 잃을 때가 있으며 지킬 때가 있고 버릴 때가 있으며
\par 7 찢을 때가 있고 꿰멜 때가 있으며 잠잠할 때가 있고 말할 때가 있으며
\par 8 사랑할 때가 있고 미워할 때가 있으며 전쟁할 때가 있고 평화할 때가 있느니라
\par 9 일하는 자가 그 수고로 말미암아 무슨 이익이 있으랴
\par 10 하나님이 인생들에게 노고를 주사 애쓰게 하신 것을 내가 보았노라
\par 11 하나님이 모든 것을 지으시되 때를 따라 아름답게 하셨고 또 사람에게 영원을 사모하는 마음을 주셨느니라 그러나 하나님의 하시는 일의 시종을 사람으로 측량할 수 없게 하셨도다
\par 12 사람이 사는 동안에 기뻐하며 선을 행하는 것보다 나은 것이 없는 줄을 내가 알았고
\par 13 사람마다 먹고 마시는 것과 수고함으로 낙을 누리는 것이 하나님의 선물인 줄을 또한 알았도다
\par 14 무릇 하나님의 행하시는 것은 영원히 있을 것이라 더 할 수도 없고 덜 할 수도 없나니 하나님이 이같이 행하심은 사람으로 그 앞에서 경외하게 하려 하심인 줄을 내가 알았도다
\par 15 이제 있는 것이 옛적에 있었고 장래에 있을 것도 옛적에 있었나니 하나님은 이미 지난 것을 다시 찾으시느니라
\par 16 내가 해 아래서 또 보건대 재판하는 곳에 악이 있고 공의를 행하는 곳에도 악이 있도다
\par 17 내가 심중에 이르기를 의인과 악인을 하나님이 심판하시리니 이는 모든 목적과 모든 일이 이룰 때가 있음이라 하였으며
\par 18 내가 심중에 이르기를 인생의 일에 대하여 하나님이 저희를 시험하시리니 저희로 자기가 짐승보다 다름이 없는 줄을 깨닫게 하려하심이라 하였노라
\par 19 인생에게 임하는 일이 일반이라 다 동일한 호흡이 있어서 이의 죽음같이 저도 죽으니 사람이 짐승보다 뛰어남이 없음은 모든 것이 헛됨이로다
\par 20 다 흙으로 말미암았으므로 다 흙으로 돌아가나니 다 한 곳으로 가거니와
\par 21 인생의 혼은 위로 올라가고 짐승의 혼은 아래 곧 땅으로 내려가는 줄을 누가 알랴
\par 22 그러므로 내 소견에는 사람이 자기 일에 즐거워하는 것보다 나은것이 없나니 이는 그의 분복이라 그 신후사를 보게 하려고 저를 도로 데리고 올 자가 누구이랴

\chapter{4}

\par 1 내가 돌이켜 해 아래서 행하는 모든 학대를 보았도다 오호라 학 대받는 자가 눈물을 흘리되 저희에게 위로자가 없도다 저희를 학대하는 자의 손에는 권세가 있으나 저희에게는 위로자가 없도다
\par 2 그러므로 나는 살아 있는 산 자보다 죽은 지 오랜 죽은 자를 복되다 하였으며
\par 3 이 둘보다도 출생하지 아니하여 해 아래서 행하는 악을 보지 못한 자가 더욱 낫다 하였노라
\par 4 내가 또 본즉 사람이 모든 수고와 여러 가지 교묘한 일로 인하여 이웃에게 시기를 받으니 이것도 헛되어 바람을 잡으려는 것이로다
\par 5 우매자는 손을 거두고 자기 살을 먹느니라
\par 6 한 손에만 가득하고 평온함이 두 손에 가득하고 수고하며 바람을 잡으려는 것보다 나으니라
\par 7 내가 또 돌이켜 해 아래서 헛된 것을 보았도다
\par 8 어떤 사람은 아들도 없고 형제도 없으니 아무도 없이 홀로 있으나 수고하기를 마지 아니하며 부를 눈에 족하게 여기지 아니하면서도 이르기를 내가 누구를 위하여 수고하고 내 심령으로 낙을 누리지 못하게 하는고 하나니 이것도 헛되어 무익한 노고로다
\par 9 두 사람이 한 사람보다 나음은 저희가 수고함으로 좋은 상을 얻을 것임이라
\par 10 혹시 저희가 넘어지면 하나가 그 동무를 붙들어 일으키려니와 홀로 있어 넘어지고 붙들어 일으킬 자가 없는 자에게는 화가 있으리라
\par 11 두 사람이 함께 누우면 따뜻하거니와 한 사람이면 어찌 따뜻하랴
\par 12 한 사람이면 패하겠거니와 두 사람이면 능히 당하나니 삼겹 줄은 쉽게 끊어지지 아니하느니라
\par 13 가난하여도 지혜로운 소년은 늙고 둔하여 간함을 받을줄 모르는 왕보다 나으니
\par 14 저는 그 나라에서 나면서 가난한 자로서 옥에서 나와서 왕이 되었음이니라
\par 15 내가 본즉 해 아래서 다니는 인생들이 왕의 버금으로 대신하여 일어난 소년과 함께 있으매
\par 16 저희 치리를 받는 백성들이 무수하였을지라도 후에 오는 자들은 저를 기뻐하지 아니하리니 이것도 헛되어 바람을 잡으려는 것이로다

\chapter{5}

\par 1 너는 하나님의 전에 들어갈 때에 네 발을 삼갈지어다 가까이 하여 말씀을 듣는 것이 우매자의 제사 드리는 것보다 나으니 저희는 악을 행하면서도 깨닫지 못함이니라
\par 2 너는 하나님 앞에서 함부로 입을 열지 말며 급한 마음으로 말을 내지 말라 하나님은 하늘에 계시고 너는 땅에 있음이니라 그런즉 마땅히 말을 적게 할 것이라
\par 3 일이 많으면 꿈이 생기고 말이 많으면 우매자의 소리가 나타나느니라
\par 4 네가 하나님께 서원하였거든 갚기를 더디게 말라 하나님은 우매 자를 기뻐하지 아니하시나니 서원한 것을 갚으라
\par 5 서원하고 갚지 아니하는 것보다 서원하지 아니하는 것이 나으니
\par 6 네 입으로 네 육체를 범죄케 말라 사자 앞에서 내가 서원한 것이 실수라고 말하지 말라 어찌 하나님으로 네 말소리를 진노하사 네손으로 한 것을 멸하시게 하랴
\par 7 꿈이 많으면 헛된 것이 많고 말이 많아도 그러하니 오직 너는 하나님을 경외할지니라
\par 8 너는 어느 도에서든지 빈민을 학대하는 것과 공의를 박멸하는 것을 볼지라도 그것을 이상히 여기지 말라 높은 자보다 더 높은 자가 감찰하고 그들보다 더 높은 자들이 있음이니라
\par 9 땅의 이익은 뭇 사람을 위하여 있나니 왕도 밭의 소산을 받느니라
\par 10 은을 사랑하는 자는 은으로 만족함이 없고 풍부를 사랑하는 자는 소득으로 만족함이 없나니 이것도 헛되도다
\par 11 재산이 더하면 먹는 자도 더하나니 그 소유주가 눈으로 보는 외에 무엇이 유익하랴
\par 12 노동자는 먹는 것이 많든지 적든지 잠을 달게 자거니와 부자는 배부름으로 자지 못하느니라
\par 13 내가 해 아래서 큰 폐단되는 것을 보았나니 곧 소유주가 재물을 자기에게 해 되도록 지키는 것이라
\par 14 그 재물이 재난을 인하여 패하나니 비록 아들은 낳았으나 그 손에 아무 것도 없느니라
\par 15 저가 모태에서 벌거벗고 나왔은즉 그 나온 대로 돌아가고 수고하여 얻은 것을 아무 것도 손에 가지고 가지 못하리니
\par 16 이것도 폐단이라 어떻게 왔든지 그대로 가리니 바람을 잡으려는 수고가 저에게 무엇이 유익하랴
\par 17 일평생을 어두운 데서 먹으며 번뇌와 병과 분노가 저에게 있느니라
\par 18 사람이 하나님의 주신바 그 일평생에 먹고 마시며 해 아래서 수고하는 모든 수고 중에서 낙을 누리는 것이 선하고 아름다움을 내가 보았나니 이것이 그의 분복이로다
\par 19 어떤 사람에게든지 하나님이 재물과 부요를 주사 능히 누리게 하시며 분복을 받아 수고함으로 즐거워하게 하신 것은 하나님의 선물이라
\par 20 저는 그 생명의 날을 깊이 관념치 아니하리니 이는 하나님이 저의 마음의 기뻐하는 것으로 응하심이라

\chapter{6}

\par 1 내가 해 아래서 한가지 폐단 있는 것을 보았나니 이는 사람에게 중한 것이라
\par 2 어떤 사람은 그 심령의 모든 소원에 부족함이 없어 재물과 부요와 존귀를 하나님께 받았으나 능히 누리게 하심을 얻지 못하였으므로 다른 사람이 누리나니 이것도 헛되어 악한 병이로다
\par 3 사람이 비록 일백 자녀를 낳고 또 장수하여 사는 날이 많을지라도 그 심령에 낙이 족하지 못하고 또 그 몸이 매장되지 못하면 나는 이르기를 낙태된 자가 저보다 낫다 하노니
\par 4 낙태된 자는 헛되이 왔다가 어두운 중에 가매 그 이름이 어두움에 덮이니
\par 5 햇빛을 보지 못하고 알지 못하나 이가 저보다 평안함이라
\par 6 저가 비록 천 년의 갑절을 산다 할지라도 낙을 누리지 못하면 마침내 다 한 곳으로 돌아가는 것뿐이 아니냐
\par 7 사람의 수고는 다 그 입을 위함이나 그 식욕은 차지 아니하느니라
\par 8 지혜자가 우매자보다 나은 것이 무엇이뇨 인생 앞에서 행할 줄 아는 가난한 자는 무엇이 유익한고
\par 9 눈으로 보는 것이 심령의 공상보다 나으나 이것도 헛되어 바람을 잡으려는 것이로다
\par 10 이미 있는 무엇이든지 오래 전부터 그 이름이 칭한 바 되었으며 사람이 무엇인지도 이미 안 바 되었나니 자기보다 강한 자와 능히 다툴 수 없느니라
\par 11 헛된 것을 더하게 하는 많은 일이 있나니 사람에게 무엇이 유익하랴
\par 12 헛된 생명의 모든 날을 그림자같이 보내는 일평생에 사람에게 무엇이 낙인지 누가 알며 그 신후에 해 아래서 무슨 일이 있을 것을 누가 능히 그에게 고하리요

\chapter{7}

\par 1 아름다운 이름이 보배로운 기름보다 낫고 죽는 날이 출생하는 날보다 나으며
\par 2 초상집에 가는 것이 잔치집에 가는 것보다 나으니 모든 사람의 결국이 이와 같이 됨이라 산 자가 이것에 유심하리로다
\par 3 슬픔이 웃음보다 나음은 얼굴에 근심함으로 마음이 좋게 됨이니라
\par 4 지혜자의 마음은 초상집에 있으되 우매자의 마음은 연락하는 집에 있느니라
\par 5 사람이 지혜자의 책망을 듣는 것이 우매자의 노래를 듣는 것보다 나으니라
\par 6 우매자의 웃음 소리는 솥 밑에서 가시나무의 타는 소리 같으니 이것도 헛되니라
\par 7 탐학이 지혜자를 우매하게 하고 뇌물이 사람의 명철을 망케 하느니라
\par 8 일의 끝이 시작보다 낫고 참는 마음이 교만한 마음보다 나으니
\par 9 급한 마음으로 노를 발하지 말라 노는 우매자의 품에 머무름이니라
\par 10 옛날이 오늘보다 나은 것이 어찜이냐 하지 말라 이렇게 묻는 것이 지혜가 아니니라
\par 11 지혜는 유업같이 아름답고 햇빛을 보는 자에게 유익하도다
\par 12 지혜도 보호하는 것이 되고 돈도 보호하는 것이 되나 지식이 더욱 아름다움은 지혜는 지혜 얻은 자의 생명을 보존함이니라
\par 13 하나님의 행하시는 일을 보라 하나님이 굽게 하신 것을 누가 능히 곧게 하겠느냐
\par 14 형통한 날에는 기뻐하고 곤고한 날에는 생각하라 하나님이 이 두가지를 병행하게 하사 사람으로 그 장래 일을 능히 헤아려 알지 못하게 하셨느니라
\par 15 내가 내 헛된 날에 이 모든 일을 본즉 자기의 의로운 중에서 멸망하는 의인이 있고 자기의 악행 중에서 장수하는 악인이 있으니
\par 16 지나치게 의인이 되지 말며 지나치게 지혜자도 되지 말라 어찌하여 스스로 패망케 하겠느냐
\par 17 지나치게 악인이 되지 말며 우매자도 되지 말라 어찌하여 기한 전에 죽으려느냐
\par 18 너는 이것을 잡으며 저것을 놓지 마는 것이 좋으니 하나님을 경외하는 자는 이 모든 일에서 벗어날 것임이니라
\par 19 지혜가 지혜자로 성읍 가운데 열 유사보다 능력이 있게 하느니라
\par 20 선을 행하고 죄를 범치 아니하는 의인은 세상에 아주 없느니라
\par 21 무릇 사람의 말을 들으려고 마음을 두지 말라 염려컨대 네 종이 너를 저주하는 것을 들으리라
\par 22 너도 가끔 사람을 저주한 것을 네 마음이 아느니라
\par 23 내가 이 모든 것을 지혜로 시험하며 스스로 이르기를 내가 지혜자가 되리라 하였으나 지혜가 나를 멀리하였도다
\par 24 무릇 된 것이 멀고 깊고 깊도다 누가 능히 통달하랴
\par 25 내가 돌이켜 전심으로 지혜와 명철을 살피고 궁구하여 악한 것이 어리석은 것이요 어리석은 것이 미친 것인 줄을 알고자 하였더니
\par 26 내가 깨달은즉 마음이 올무와 그물같고 손이 포승같은 여인은 사망보다 독한 자라 하나님을 기뻐하는 자는 저를 피하려니와 죄인은 저에게 잡히리로다
\par 27 전도자가 가로되 내가 낱낱이 살펴 그 이치를 궁구하여 이것을 깨달았노라
\par 28 내 마음에 찾아도 아직 얻지 못한 것이 이것이라 일천 남자 중에서 하나를 얻었거니와 일천 여인 중에서는 하나도 얻지 못하였느니라
\par 29 나의 깨달은 것이 이것이라 곧 하나님이 사람을 정직하게 지으셨으나 사람은 많은 꾀를 낸 것이니라

\chapter{8}

\par 1 지혜자와 같은 자 누구며 사리의 해석을 아는 자 누구냐 사람의 지혜는 그 사람의 얼굴에 광채가 나게 하나니 그 얼굴의 사나운 것이 변하느니라
\par 2 내가 권하노니 왕의 명령을 지키라 이미 하나님을 가리켜 맹세하였음이니라
\par 3 왕 앞에서 물러가기를 급거히 말며 악한 것을 일삼지 말라 왕은 그 하고자 하는 것을 다 행함이니라
\par 4 왕의 말은 권능이 있나니 누가 이르기를 왕께서 무엇을 하시나이까 할 수 있으랴
\par 5 무릇 명령을 지키는 자는 화를 모르리라 지혜자의 마음은 시기와 판단을 분변하나니
\par 6 무론 무슨 일에든지 시기와 판단이 있으므로 사람에게 임하는 화가 심함이니라
\par 7 사람이 장래 일을 알지 못하나니 장래 일을 가르칠 자가 누구이랴
\par 8 생기를 주장하여 생기로 머무르게할 사람도 없고 죽는 날을 주장할 자도 없고 전쟁할 때에 모면할 자도 없으며 악이 행악자를 건져낼 수도 없느니라
\par 9 내가 이런 것들을 다 보고 마음을 다하여 해 아래서 행하는 모든일을 살핀즉 사람이 사람을 주장하여 해롭게 하는 때가 있으며
\par 10 내가 본즉 악인은 장사 지낸 바 되어 무덤에 들어 갔고 선을 행한 자는 거룩한 곳에서 떠나 성읍 사람의 잊어버린 바 되었으니 이것도 헛되도다
\par 11 악한 일에 징벌이 속히 실행되지 않으므로 인생들이 악을 행하기에 마음이 담대하도다
\par 12 죄인이 백번 악을 행하고도 장수하거니와 내가 정녕히 아노니 하나님을 경외하여 그 앞에서 경외하는 자가 잘 될 것이요
\par 13 악인은 잘 되지 못하며 장수하지 못하고 그 날이 그림자와 같으리니 이는 하나님 앞에 경외하지 아니함이니라
\par 14 세상에 행하는 헛된 일이 있나니 곧 악인의 행위대로 받는 의인도 있고 의인의 행위대로 받는 악인도 있는 것이라 내가 이르노니 이것도 헛되도다
\par 15 이에 내가 희락을 칭찬하노니 이는 사람이 먹고 마시고 즐거워하는 것보다 해 아래서 나은 것이 없음이라 하나님이 사람으로 해 아래서 살게 하신 날 동안 수고하는 중에 이것이 항상 함께 있을 것이니라
\par 16 내가 마음을 다하여 지혜를 알고자 하며 세상에서 하는 노고를 보고자 하는 동시에(밤낮으로 자지 못하는 자도 있도다)
\par 17 하나님의 모든 행사를 살펴보니 해 아래서 하시는 일을 사람이 능히 깨달을 수 없도다 사람이 아무리 애써 궁구할지라도 능히 깨닫지 못하나니 비록 지혜자가 아노라 할지라도 능히 깨닫지 못하리로다

\chapter{9}

\par 1 내가 마음을 다하여 이 모든 일을 궁구하며 살펴본즉 의인과 지혜자나 그들의 행하는 일이나 다 하나님의 손에 있으니 사랑을 받을는지 미움을 받을는지 사람이 알지 못하는 것은 모두 그 미래임이니라
\par 2 모든 사람에게 임하는 모든 것이 일반이라 의인과 악인이며 선하고 깨끗한 자와 깨끗지 않은 자며 제사를 드리는 자와 제사를 드리지 아니하는 자의 결국이 일반이니 선인과 죄인이며 맹세하는 자와 맹세하기를 무서워하는 자가 일반이로다
\par 3 모든 사람의 결국이 일반인 그것은 해 아래서 모든 일 중에 악한것이니 곧 인생의 마음에 악이 가득하여 평생에 미친 마음을 품다가 후에는 죽은 자에게로 돌아가는 것이라
\par 4 모든 산 자 중에 참예한 자가 소망이 있음은 산 개가 죽은 사자 보다 나음이니라
\par 5 무릇 산 자는 죽을 줄을 알되 죽은 자는 아무 것도 모르며 다시는 상도 받지 못하는 것은 그 이름이 잊어버린 바 됨이라
\par 6 그 사랑함과 미워함과 시기함이 없어진지 오래니 해 아래서 행하는 모든 일에 저희가 다시는 영영히 분복이 없느니라
\par 7 너는 가서 기쁨으로 네 식물을 먹고 즐거운 마음으로 네 포도주를 마실지어다 이는 하나님이 너의 하는 일을 벌써 기쁘게 받으셨음이니라
\par 8 네 의복을 항상 희게하며 네 머리에 향 기름을 그치지 않게 할지니라
\par 9 네 헛된 평생의 모든 날 곧 하나님이 해 아래서 네게 주신 모든 헛된 날에 사랑하는 아내와 함께 즐겁게 살지어다 이는 네가 일 평생에 해 아래서 수고하고 얻은 분복이니라
\par 10 무릇 네 손이 일을 당하는대로 힘을 다하여 할찌어다 네가 장차 들어갈 음부에는 일도 없고 계획도 없고 지식도 없고 지혜도 없음이니라
\par 11 내가 돌이켜 해 아래서 보니 빠른 경주자라고 선착하는 것이 아니며 유력자라고 전쟁에 승리하는 것이 아니며 지혜자라고 식물을 얻는 것이 아니며 명철자라고 재물을 얻는 것이 아니며 기능자라고 은총을 입는 것이 아니니 이는 시기와 우연이 이 모든 자에게 임함이라
\par 12 대저 사람은 자기의 시기를 알지 못하나니 물고기가 재앙의 그물에 걸리고 새가 올무에 걸림같이 인생도 재앙의 날이 홀연히 임하면 거기 걸리느니라
\par 13 내가 또 해 아래서 지혜를 보고 크게 여긴 것이 이러하니
\par 14 곧 어떤 작고 인구가 많지 않은 성읍에 큰 임금이 와서 에워싸고 큰 흉벽을 쌓고 치고자 할 때에
\par 15 그 성읍 가운데 가난한 지혜자가 있어서 그 지혜로 그 성읍을 건진 것이라 그러나 이 가난한 자를 기억하는 사람이 없었도다
\par 16 그러므로 내가 이르기를 지혜가 힘보다 낫다마는 가난한 자의 지혜가 멸시를 받고 그 말이 신청되지 아니한다 하였노라
\par 17 종용히 들리는 지혜자의 말이 우매자의 어른의 호령보다 나으니라
\par 18 지혜가 병기보다 나으니라 그러나 한 죄인이 많은 선을 패궤케 하느니라

\chapter{10}

\par 1 죽은 파리가 향기름으로 악취가 나게 하는 것 같이 적은 우매가 지혜와 존귀로 패하게 하느니라
\par 2 지혜자의 마음은 오른편에 있고 우매자의 마음은 왼편에 있느니라
\par 3 우매자는 길에 행할 때에도 지혜가 결핍하여 각 사람에게 자기의 우매한 것을 말하느니라
\par 4 주권자가 네게 분을 일으키거든 너는 네 자리를 떠나지 말라 공순이 큰 허물을 경하게 하느니라
\par 5 내가 해 아래서 한 가지 폐단 곧 주권자에게서 나는 허물인듯한 것을 보았노니
\par 6 우매자가 크게 높은 지위를 얻고 부자가 낮은 지위에 앉는도다
\par 7 또 보았노니 종들은 말을 타고 방백들은 종처럼 땅에 걸어 다니는도다
\par 8 함정을 파는 자는 거기 빠질 것이요 담을 허는 자는 뱀에게 물리리라
\par 9 돌을 떠내는 자는 그로 인하여 상할 것이요 나무를 쪼개는 자는 그로 인하여 위험을 당하리라
\par 10 무딘 철 연장 날을 갈지 아니하면 힘이 더 드느니라 오직 지혜는 성공하기에 유익하니라
\par 11 방술을 베풀기 전에 뱀에게 물렸으면 술객은 무용하니라
\par 12 지혜자의 입의 말은 은혜로우나 우매자의 입술은 자기를 삼키나니
\par 13 그 입의 말의 시작은 우매요 끝은 광패니라
\par 14 우매자는 말을 많이 하거니와 사람이 장래 일을 알지 못하나니 신후사를 알게 할 자가 누구이냐
\par 15 우매자들의 수고는 제각기 곤하게 할 뿐이라 저희는 성읍에 들어갈 줄도 알지 못함이니라
\par 16 왕은 어리고 대신들은 아침에 연락하는 이 나라여 화가 있도다
\par 17 왕은 귀족의 아들이요 대신들은 취하려 함이 아니라 기력을 보하려고 마땅한 때에 먹는 이 나라여 복이 있도다
\par 18 게으른즉 석가래가 퇴락하고 손이 풀어진즉 집이 새느니라
\par 19 잔치는 희락을 위하여 베푸는 것이요 포도주는 생명을 기쁘게 하는 것이나 돈은 범사에 응용되느니라
\par 20 심중에라도 왕을 저주하지 말며 침방에서라도 부자를 저주하지 말라 공중의 새가 그 소리를 전하고 날짐승이 그 일을 전파할 것임이니라

\chapter{11}

\par 1 너는 네 식물을 물 위에 던지라 여러날 후에 도로 찾으리라
\par 2 일곱에게나 여덟에게 나눠줄지어다 무슨 재앙이 땅에 임할는지 네가 알지 못함이니라
\par 3 구름에 비가 가득하면 땅에 쏟아지며 나무가 남으로나 북으로나 쓰러지면 그 쓰러진 곳에 그냥 있으리라
\par 4 풍세를 살펴보는 자는 파종하지 아니할 것이요 구름을 바라보는 자는 거두지 아니하리라
\par 5 바람의 길이 어떠함과 아이 밴 자의 태에서 뼈가 어떻게 자라는 것을 네가 알지 못함같이 만사를 성취하시는 하나님의 일을 네가알지 못하느니라
\par 6 "너는 아침에 씨를 뿌리고 저녁에도 손을 거두지 말라 이것이 잘 될는지, 저것이 잘 될는지, 혹 둘이 다 잘 될는지 알지 못함이니라"
\par 7 빛은 실로 아름다운 것이라 눈으로 해를 보는 것이 즐거운 일이로다
\par 8 사람이 여러 해를 살면 항상 즐거워할지로다 그러나 캄캄한 날이 많으리니 그날을 생각할지로다 장래 일은 다 헛되도다
\par 9 청년이여 네 어린 때를 즐거워 하며 네 청년의 날을 마음에 기뻐하여 마음에 원하는 길과 네 눈이 보는대로 좇아 행하라 그러나 하나님이 이 모든 일로 인하여 너를 심판하실 줄 알라
\par 10 그런즉 근심으로 네 마음에서 떠나게 하며 악으로 네 몸에서 물러가게 하라 어릴 때와 청년의 때가 다 헛되니라

\chapter{12}

\par 1 "너는 청년의 때 곧 곤고한 날이 이르기 전, 나는 아무 낙이 없다고 할 해가 가깝기 전에 너의 창조자를 기억하라"
\par 2 해와 빛과 달과 별들이 어둡기 전에 비 뒤에 구름이 다시 일어나기 전에 그리하라
\par 3 그런 날에는 집을 지키는 자들이 떨 것이며 힘있는 자들이 구부러질 것이며 맷돌질 하는 자들이 적으므로 그칠 것이며 창들로 내어다 보는 자가 어두워질 것이며
\par 4 길거리 문들이 닫혀질 것이며 맷돌 소리가 적어질 것이며 새의 소리를 인하여 일어날 것이며 음악하는 여자들은 다 쇠하여질 것이며
\par 5 그런 자들은 높은 곳을 두려워할 것이며 길에서는 놀랄 것이며 살구나무가 꽃이 필 것이며 메뚜기도 짐이 될 것이며 원욕이 그치리니 이는 사람이 자기 영원한 집으로 돌아가고 조문자들이 거리로 왕래하게 됨이라
\par 6 은줄이 풀리고 금 그릇이 깨어지고 항아리가 샘 곁에서 깨어지고 바퀴가 우물 위에서 깨어지고
\par 7 흙은 여전히 땅으로 돌아가고 신은 그 주신 하나님께로 돌아가기전에 기억하라
\par 8 전도자가 가로되 헛되고 헛되도다 모든 것이 헛되도다
\par 9 전도자가 지혜로움으로 여전히 백성에게 지식을 가르쳤고 또 묵상하고 궁구하여 잠언을 많이 지었으며
\par 10 전도자가 힘써 아름다운 말을 구하였나니 기록한 것은 정직하여 진리의 말씀이니라
\par 11 지혜자의 말씀은 찌르는 채찍같고 회중의 스승의 말씀은 잘 박힌 못 같으니 다 한 목자의 주신 바니라
\par 12 내 아들아 또 경계를 받으라 여러 책을 짓는 것은 끝이 없고 많이 공부하는 것은 몸을 피곤케 하느니라
\par 13 일의 결국을 다 들었으니 하나님을 경외하고 그 명령을 지킬지어다 이것이 사람의 본분이니라
\par 14 하나님은 모든 행위와 모든 은밀한 일을 선악간에 심판하시리라


\end{document}