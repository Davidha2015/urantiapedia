\begin{document}

\title{다니엘}


\chapter{1}

\par 1 유다 왕 여호야김이 위에 있은지 삼년에 바벨론 왕 느부갓네살이 예루살렘에 이르러 그것을 에워쌌더니
\par 2 주께서 유다 왕 여호야김과 하나님의 전 기구 얼마를 그의 손에 붙이시매 그가 그것을 가지고 시날 땅 자기 신의 묘에 이르러 그 신의 보고에 두었더라
\par 3 왕이 환관장 아스부나스에게 명하여 이스라엘 자손 중에서 왕족과 귀족의 몇 사람
\par 4 곧 흠이 없고 아름다우며 모든 재주를 통달하며 지식이 구비하며 학문에 익숙하여 왕궁에 모실 만한 소년을 데려오게 하였고 그들에게 갈대아 사람의 학문과 방언을 가르치게 하였고
\par 5 또 왕이 지정하여 자기의 진미와 자기의 마시는 포도주에서 그들의 날마다 쓸 것을 주어 삼년을 기르게 하였으니 이는 그 후에 그들로 왕의 앞에 모셔 서게 하려 함이었더라
\par 6 그들 중에 유다 자손 곧 다니엘과 하나냐와 미사엘과 아사랴가 있었더니
\par 7 환관장이 그들의 이름을 고쳐 다니엘은 벨드사살이라 하고 하나냐는 사드락이라 하고 미사엘은 메삭이라 하고 아사랴는 아벳느고라 하였더라
\par 8 다니엘은 뜻을 정하여 왕의 진미와 그의 마시는 포도주로 자기를 더럽히지 아니하리라 하고 자기를 더럽히지 않게 하기를 환관장에게 구하니
\par 9 하나님이 다니엘로 환관장에게 은혜와 긍휼을 얻게 하신지라
\par 10 환관장이 다니엘에게 이르되 내가 내 주 왕을 두려워하노라 그가 너희 먹을 것과 너희 마실 것을 지정하셨거늘 너희의 얼굴이 초췌하여 동무 소년들만 못한 것을 그로 보시게 할것이 무엇이냐 그렇게 되면 너희 까닭에 내 머리가 왕 앞에서 위태하게 되리라 하니라
\par 11 환관장이 세워 다니엘과 하나냐와 미사엘과 아사랴를 감독하게 한 자에게 다니엘이 말하되
\par 12 청하오니 당신의 종들을 열흘 동안 시험하여 채식을 주어 먹게 하고 물을 주어 마시게 한 후에
\par 13 당신 앞에서 우리의 얼굴과 왕의 진미를 먹는 소년들의 얼굴을 비교하여 보아서 보이는 대로 종들에게 처분하소서 하매
\par 14 그가 그들의 말을 좇아 열흘을 시험하더니
\par 15 열흘 후에 그들의 얼굴이 더욱 아름답고 살이 더욱 윤택하여 왕의 진미를 먹는 모든 소년보다 나아 보인지라
\par 16 이러므로 감독하는 자가 그들에게 분정된 진미와 마실 포도주를 제하고 채식을 주니라
\par 17 하나님이 이 네 소년에게 지식을 얻게 하시며 모든 학문과 재주에 명철하게 하신 외에 다니엘은 또 모든 이상과 몽조를 깨달아 알더라
\par 18 왕의 명한바 그들을 불러 들일 기한이 찼으므로 환관장이 그들을 데리고 느부갓네살 앞으로 들어갔더니
\par 19 왕이 그들과 말하여 보매 무리 중에 다니엘과 하나냐와 미사엘과 아사랴와 같은 자 없으므로 그들로 왕 앞에 모시게 하고
\par 20 왕이 그들에게 모든 일을 묻는 중에 그 지혜와 총명이 온 나라 박수와 술객보다 십배나 나은 줄을 아니라
\par 21 다니엘은 고레스 왕 원년까지 있으니라

\chapter{2}

\par 1 느부갓네살이 위에 있은지 이년에 꿈을 꾸고 그로 인하여 마음이 번민하여 잠을 이루지 못한지라
\par 2 왕이 그 꿈을 자기에게 고하게 하려고 명하여 박수와 술객과 점장이와 갈대아 술사를 부르매 그들이 들어와서 왕의 앞에 선지라
\par 3 왕이 그들에게 이르되 내가 꿈을 꾸고 그 꿈을 알고자 하여 마음이 번민하도다
\par 4 갈대아 술사들이 아람 방언으로 왕에게 말하되 왕이여 만세수를 하옵소서 왕은 그 꿈을 종들에게 이르시면 우리가 해석하여 드리겠나이다
\par 5 왕이 갈대아 술사에게 대답하여 가로되 내가 명령을 내렸나니 너희가 만일 꿈과 그 해석을 나로 알게 하지 아니하면 너희 몸을 쪼갤 것이며 너희 집으로 거름터를 삼을 것이요
\par 6 너희가 만일 꿈과 그 해석을 보이면 너희가 선물과 상과 큰 영광을 내게서 얻으리라 그런즉 꿈과 그 해석을 내게 보이라
\par 7 그들이 다시 대답하여 가로되 청컨대 왕은 꿈을 종들에게 이르소서 그리하시면 우리가 해석하여 드리겠나이다
\par 8 왕이 대답하여 가로되 내가 분명히 아노라 너희가 나의 명령이 내렸음을 보았으므로 시간을 천연하려 함이로다
\par 9 너희가 만일 이 꿈을 나로 알게 하지 아니하면 너희를 처치할 법이 오직 하나이니 이는 너희가 거짓말과 망령된 말을 내 앞에서 꾸며 말하여 때가 변하기를 기다리려 함이니라 이제 그 꿈을 내게 알게 하라 그리하면 너희가 그 해석도 보일 줄을 내가 알리라
\par 10 갈대아 술사들이 왕 앞에 대답하여 가로되 세상에는 왕의 그 일을 보일 자가 하나도 없으므로 크고 권력 있는 왕이 이런 것으로 박수에게나 술객에게나 갈대아 술사에게 물은 자가 절대로 있지 아니하였나이다
\par 11 왕의 물으신 것은 희한한 일이라 육체와 함께 거하지 아니하는 신들 외에는 왕 앞에 그것을 보일 자가 없나이다 한지라
\par 12 왕이 이로 인하여 진노하고 통분하여 바벨론 모든 박사를 다 멸하라 명하니라
\par 13 왕의 명령이 내리매 박사들은 죽게 되었고 다니엘과 그 동무도 죽이려고 찾았더라
\par 14 왕의 시위대 장관 아리옥이 바벨론 박사들을 죽이러 나가매 다니엘이 명철하고 슬기로운 말로
\par 15 왕의 장관 아리옥에게 물어 가로되 왕의 명령이 어찌 그리 급하뇨 아리옥이 그 일을 다니엘에게 고하매
\par 16 다니엘이 들어가서 왕께 구하기를 기한하여 주시면 왕에게 그 해석을 보여 드리겠다 하니라
\par 17 이에 다니엘이 자기 집으로 돌아가서 그 동무 하나냐와 미사엘과 아사랴에게 그 일을 고하고
\par 18 하늘에 계신 하나님이 이 은밀한 일에 대하여 긍휼히 여기사 자기 다니엘과 동무들이 바벨론의 다른 박사와 함께 죽임을 당치 않게 하시기를 그들로 구하게 하니라
\par 19 이에 이 은밀한 것이 밤에 이상으로 다니엘에게 나타나 보이매 다니엘이 하늘에 계신 하나님을 찬송하니라
\par 20 다니엘이 말하여 가로되 영원 무궁히 하나님의 이름을 찬송할 것은 지혜와 권능이 그에게 있음이로다
\par 21 그는 때와 기한을 변하시며 왕들을 폐하시고 왕들을 세우시며 지혜자에게 지혜를 주시고 지식자에게 총명을 주시는도다
\par 22 그는 깊고 은밀한 일을 나타내시고 어두운데 있는 것을 아시며 또 빛이 그와 함께 있도다
\par 23 나의 열조의 하나님이여 주께서 이제 내게 지혜와 능력을 주시고 우리가 주께 구한바 일을 내게 알게 하셨사오니 내가 주께 감사하고 주를 찬양하나이다 곧 주께서 왕의 그 일을 내게 보이셨나이다 하니라
\par 24 이에 다니엘이 왕이 바벨론 박사들을 멸하라 명한 아리옥에게로 가서 이르매 그에게 이같이 이르되 바벨론 박사들을 멸하지 말고 나를 왕의 앞으로 인도하라 그리하면 내가 그 해석을 왕께 보여 드리리라
\par 25 이에 아리옥이 다니엘을 데리고 급히 왕의 앞에 들어가서 고하되 내가 사로잡혀온 유다 자손 중에서 한 사람을 얻었나이다 그가 그 해석을 왕께 아시게 하리이다
\par 26 왕이 대답하여 벨드사살이라 이름한 다니엘에게 이르되 내가 얻은 꿈과 그 해석을 네가 능히 내게 알게 하겠느냐
\par 27 다니엘이 왕 앞에 대답하여 가로되 왕의 물으신바 은밀한 것은 박사나 술객이나 박수나 점장이가 능히 왕께 보일 수 없으되
\par 28 오직 은밀한 것을 나타내실 자는 하늘에 계신 하나님이시라 그가 느부갓네살왕에게 후일에 될 일을 알게 하셨나이다 왕의 꿈 곧 왕이 침상에서 뇌 속으로 받은 이상은 이러하니이다
\par 29 왕이여 왕이 침상에 나아가서 장래 일을 생각하실 때에 은밀한 것을 나타내시는 이가 장래 일을 왕에게 알게 하셨사오며
\par 30 내게 이 은밀한 것을 나타내심은 내 지혜가 다른 인생보다 나은 것이 아니라 오직 그 해석을 왕에게 알려서 왕의 마음으로 생각 하던 것을 왕으로 알게하려 하심이니이다
\par 31 왕이여 왕이 한 큰 신상을 보셨나이다 그 신상이 왕의 앞에 섰는데 크고 광채가 특심하며 그 모양이 심히 두려우니
\par 32 그 우상의 머리는 정금이요 가슴과 팔들은 은이요 배와 넓적다리는 놋이요
\par 33 그 종아리는 철이요 그 발은 얼마는 철이요 얼마는 진흙이었나이다
\par 34 또 왕이 보신즉 사람의 손으로 하지 아니하고 뜨인 돌이 신상의 철과 진흙의 발을 쳐서 부숴뜨리매
\par 35 때에 철과 진흙과 놋과 은과 금이 다 부숴져 여름 타작 마당의 겨같이 되어 바람에 불려 간 곳이 없었고 우상을 친돌은 태산을 이루어 온 세계에 가득하였었나이다
\par 36 그 꿈이 이러한즉 내가 이제 그 해석을 왕 앞에 진술하리이다
\par 37 왕이여 왕은 열왕의 왕이시라 하늘의 하나님이 나라와 권세와 능력과 영광을 왕에게 주셨고
\par 38 인생들과 들짐승과 공중의 새들 어느곳에 있는 것을 무론하고 그것들을 왕의 손에 붙이사 다 다스리게 하셨으니 왕은 곧 그 금머리니이다
\par 39 왕의 후에 왕만 못한 다른 나라가 일어날 것이요 세째로 또 놋 같은 나라가 일어나서 온 세계를 다스릴 것이며
\par 40 네째 나라는 강하기가 철 같으리니 철은 모든 물건을 부숴뜨리고 이기는 것이라 철이 모든 것을 부수는 것 같이 그 나라가 뭇 나라를 부숴뜨리고 빻을 것이며
\par 41 왕께서 그 발과 발가락이 얼마는 토기장이의 진흙이요 얼마는 철인 것을 보셨은즉 그 나라가 나누일 것이며 왕께서 철과 진흙이 섞인 것을 보셨은즉 그 나라가 철의 든든함이 있을 것이나
\par 42 그 발가락이 얼마는 철이요 얼마는 진흙인즉 그 나라가 얼마는 든든하고 얼마는 부숴질 만할 것이며
\par 43 왕께서 철과 진흙이 섞인 것을 보셨은즉 그들이 다른 인종과 서로 섞일 것이나 피차에 합하지 아니함이 철과 진흙이 합하지 않음과 같으리이다
\par 44 이 열왕의 때에 하늘의 하나님이 한 나라를 세우시리니 이것은 영원히 망하지도 아니할 것이요 그 국권이 다른 백성에게로 돌아가지도 아니할 것이요 도리어 이 모든 나라를 쳐서 멸하고 영원히 설 것이라
\par 45 왕이 사람의 손으로 아니하고 산에서 뜨인 돌이 철과 놋과 진흙과 은과 금을 부숴뜨린 것을 보신 것은 크신 하나님이 장래 일을 왕께 알게 하신 것이라 이 꿈이 참되고 이 해석이 확실하니이다
\par 46 이에 느부갓네살 왕이 엎드려 다니엘에게 절하고 명하여 예물과 향품을 그에게 드리게 하니라
\par 47 왕이 대답하여 다니엘에게 이르되 너희 하나님은 참으로 모든 신의 신이시요 모든 왕의 주재시로다 네가 능히 이 은밀한 것을 나타내었으니 네 하나님은 또 은밀한 것을 나타내시는 자시로다
\par 48 왕이 이에 다니엘을 높여 귀한 선물을 많이 주며 세워 바벨론 모든 박사의 어른을 삼았으며
\par 49 왕이 또 다니엘의 청구대로 사드락과 메삭과 아벳느고를 세워 바벨론 도의 일을 다스리게 하였고 다니엘은 왕궁에 있었더라

\chapter{3}

\par 1 느부갓네살 왕이 금으로 신상을 만들었으니 고는 육십 규빗이요 광은 여섯 규빗이라 그것을 바벨론 도의 두라 평지에 세웠더라
\par 2 느부갓네살 왕이 보내어 방백과 수령과 도백과 재판관과 재무관과 모사와 법률사와 각 도 모든 관원을 자기 느부갓네살 왕의 세운 신상의 낙성 예식에 참집하게 하매
\par 3 이에 방백과 수령과 도백과 재판관과 재무관과 모사와 법률사와 각 도 모든 관원이 느부갓네살 왕의 세운 신상의 낙성 예식에 참집하여 느부갓네살의 세운 신상 앞에 서니라
\par 4 반포하는 자가 크게 외쳐 가로되 백성들과 나라들과 각 방언하는 자들아 왕이 너희 무리에게 명하시나니
\par 5 너희는 나팔과 피리와 수금과 삼현금과 양금과 생황과 및 모든 악기 소리를 들을 때에 엎드리어 느부갓네살 왕의 세운 금신상에게 절하라
\par 6 누구든지 엎드리어 절하지 아니하는 자는 즉시 극렬히 타는 풀무에 던져 넣으리라 하매
\par 7 모든 백성과 나라들과 각 방언하는 자들이 나팔과 피리와 수금과 삼현금과 양금과 및 모든 악기 소리를 듣자 곧 느부갓네살 왕의 세운 금신상에게 엎드리어 절하니라
\par 8 그 때에 어떤 갈대아 사람들이 나아와 유다 사람들을 참소하니라
\par 9 그들이 느부갓네살왕에게 고하여 가로되 왕이여 만세수를 하옵소서
\par 10 왕이여 왕이 명령을 내리사 무릇 사람마다 나팔과 피리와 수금과 삼현금과 양금과 생황과 및 모든 악기 소리를 듣거든 엎드리어 금 신상에게 절할 것이라
\par 11 누구든지 엎드리어 절하지 아니하는 자는 극렬히 타는 풀무 가운데 던져 넣음을 당하리라 하지 아니하셨나이까
\par 12 이제 몇 유다 사람 사드락과 메삭과 아벳느고는 왕이 세워 바벨론 도를 다스리게 하신 자이어늘 왕이여 이 사람들이 왕을 높이지 아니하며 왕의 신들을 섬기지 아니하며 왕이 세우신 금 신상에게 절하지 아니하나이다
\par 13 느부갓네살 왕이 노하고 분하여 사드락과 메삭과 아벳느고를 끌어 오라 명하매 드디어 그 사람들을 왕의 앞으로 끌어온지라
\par 14 "느부갓네살이 그들에게 물어 가로되 사드락, 메삭, 아벳느고야 너희가 내 신을 섬기지 아니하며 내가 세운 그 신상에게 절하지 아니하니 짐짓 그리하였느냐"
\par 15 이제라도 너희가 예비하였다가 언제든지 나팔과 피리와 수금과 삼현금과 양금과 생황과 및 모든 악기 소리를 듣거든 내가 만든 신상 앞에 엎드리어 절하면 좋거니와 너희가 만일 절하지 아니하면 즉시 너희를 극렬히 타는 풀무 가운데 던져 넣을 것이니 능히 너희를 내 손에서 건져 낼 신이 어떤 신이겠느냐
\par 16 사드락과 메삭과 아벳느고가 왕에게 대답하여 가로되 느부갓네살이여 우리가 이 일에 대하여 왕에게 대답할 필요가 없나이다
\par 17 만일 그럴 것이면 왕이여 우리가 섬기는 우리 하나님이 우리를 극렬히 타는 풀무 가운데서 능히 건져 내시겠고 왕의 손에서도 건져내시리이다
\par 18 그리 아니하실지라도 왕이여 우리가 왕의 신들을 섬기지도 아니하고 왕의 세우신 금 신상에게 절하지도 아니할 줄을 아옵소서
\par 19 느부갓네살이 분이 가득하여 사드락과 메삭과 아벳느고를 향하여 낯빛을 변하고 명하여 이르되 그 풀무를 뜨겁게 하기를 평일보다 칠배나 뜨겁게 하라 하고
\par 20 군대 중 용사 몇 사람을 명하여 사드락과 메삭과 아벳느고를 결박하여 극렬히 타는 풀무 가운데 던지라 하니
\par 21 이 사람들을 고의와 속옷과 겉옷과 별다른 옷을 입은채 결박하여 극렬히 타는 풀무 가운데 던질 때에
\par 22 왕의 명령이 엄하고 풀무가 심히 뜨거우므로 불꽃이 사드락과 메삭과 아벳느고를 붙든 사람을 태워 죽였고
\par 23 이 세 사람 사드락과 메삭과 아벳느고는 결박된채 극렬히 타는 풀무 가운데 떨어졌더라
\par 24 때에 느부갓네살 왕이 놀라 급히 일어나서 모사들에게 물어 가로되 우리가 결박하여 불가운데 던진 자는 세 사람이 아니었느냐 그들이 왕에게 대답하여 가로되 왕이여 옳소이다
\par 25 왕이 또 말하여 가로되 내가 보니 결박되지 아니한 네 사람이 불 가운데로 다니는데 상하지도 아니하였고 그 네째의 모양은 신들의 아들과 같도다 하고
\par 26 "느부갓네살이 극렬히 타는 풀무 아구 가까이 가서 불러 가로되 지극히 높으신 하나님의 종 사드락, 메삭, 아벳느고야 나와서 이리로 오라 하매 사드락과 메삭과 아벳느고가 불 가운데서 나온지라"
\par 27 방백과 수령과 도백과 왕의 모사들이 모여 이 사람들을 본즉 불이 능히 그 몸을 해하지 못하였고 머리털도 그슬리지 아니하였고 고의 빛도 변하지 아니하였고 불 탄 냄새도 없었더라
\par 28 느부갓네살이 말하여 가로되 사드락과 메삭과 아벳느고의 하나님을 찬송할지로다 그가 그 사자를 보내사 자기를 의뢰하고 그 몸을 버려서 왕의 명을 거역하고 그 하나님 밖에는 다른 신을 섬기지 아니하며 그에게 절하지 아니한 종들을 구원하셨도다
\par 29 그러므로 내가 이제 조서를 내리노니 각 백성과 각 나라와 각 방언하는 자가 무릇 사드락과 메삭과 아벳느고의 하나님께 설만히 말하거든 그 몸을 쪼개고 그 집으로 거름터를 삼을지니 이는 이 같이 사람을 구원할 다른 신이 없음이니라 하고
\par 30 왕이 드디어 사드락과 메삭과 아벳느고를 바벨론 도에서 더욱 높이니라

\chapter{4}

\par 1 느부갓네살 왕은 천하에 거하는 백성들과 나라들과 각 방언하는 자에게 조서하노라 원하노니 너희에게 많은 평강이 있을지어다
\par 2 지극히 높으신 하나님이 내게 행하신 이적과 기사를 내가 알게 하기를 즐겨하노라
\par 3 크도다 그 이적이여 능하도다 그 기사여 그 나라는 영원한 나라요 그 권병은 대대에 이르리로다
\par 4 나 느부갓네살이 내 집에 편히 있으며 내 궁에서 평강할 때에
\par 5 한 꿈을 꾸고 그로 인하여 두려워하였으되 곧 내 침상에서 생각 하는 것과 뇌 속으로 받은 이상을 인하여 번민하였었노라
\par 6 이러므로 내가 명을 내려 바벨론 모든 박사를 내 앞으로 불러다가 그 꿈의 해석을 내게 알게 하라 하매
\par 7 박수와 술객과 갈대아 술사와 점장이가 들어왔기로 내가 그 꿈을 그들에게 고하였으나 그들이 그 해석을 내게 알게 하지 못하였느니라
\par 8 그 후에 다니엘이 내 앞에 들어왔으니 그는 내 신의 이름을 좇아 벨드사살이라 이름한 자요 그의 안에는 거룩한 신들의 영이 있는자라 내가 그에게 꿈을 고하여 가로되
\par 9 박수장 벨드사살아 네 안에는 거룩한 신들의 영이 있은즉 아무 은밀한 것이라도 네게는 어려울 것이 없는 줄을 내가 아노니 내 꿈에 본 이상의 해석을 내게 고하라
\par 10 내가 침상에서 나의 뇌 속으로 받은 이상이 이러하니라 내가 본즉 땅의 중앙에 한 나무가 있는데 고가 높더니
\par 11 그 나무가 자라서 견고하여지고 그 고는 하늘에 닿았으니 땅 끝에서도 보이겠고
\par 12 그 잎사귀는 아름답고 그 열매는 많아서 만민의 식물이 될 만하고 들짐승이 그 그늘에 있으며 공중에 나는 새는 그 가지에 깃들이고 무릇 혈기 있는 자가 거기서 식물을 얻더라
\par 13 내가 침상에서 뇌 속으로 받은 이상 가운데 또 본즉 한 순찰자 한 거룩한 자가 하늘에서 내려왔는데
\par 14 그가 소리 질러 외쳐서 이처럼 이르기를 그 나무를 베고 그 가지를 찍고 그 잎사귀를 떨고 그 열매를 헤치고 짐승들로 그 아래서 떠나게 하고 새들을 그 가지에서 쫓아내라
\par 15 그러나 그 뿌리의 그루터기를 땅에 남겨두고 철과 놋줄로 동이고 그것으로 들 청초 가운데 있게 하라 그것이 하늘 이슬에 젖고 땅의 풀 가운데서 짐승으로 더불어 그 분량을 같이 하리라
\par 16 또 그 마음은 변하여 인생의 마음 같지 아니하고 짐승의 마음을 받아 일곱 때를 지나리라
\par 17 이는 순찰자들의 명령대로요 거룩한 자들의 말대로니 곧 인생으로 지극히 높으신 자가 인간 나라를 다스리시며 자기의 뜻대로 그것을 누구에게든지 주시며 또 지극히 천한 자로 그 위에 세우시는 줄을 알게 하려 함이니라 하였느니라
\par 18 나 느부갓네살 왕이 이 꿈을 꾸었나니 너 벨드사살아 그 해석을 밝히 말하라 내 나라 모든 박사가 능히 그 해석을 내게 알게 하지 못하였으나 오직 너는 능히 하리니 이는 거룩한 신들의 영이 네 안에 있음이니라
\par 19 벨드사살이라 이름한 다니엘이 얼마 동안 놀라 벙벙하며 마음이 번민하여 하는지라 왕이 그에게 말하여 이르기를 벨드사살아 너는 이 꿈과 그 해석을 인하여 번민할 것이 아니니라 벨드사살이 대답하여 가로되 내 주여 그 꿈은 왕을 미워하는 자에게 응하기를 원하며 그 해석은 왕의 대적에게 응하기를 원하나이다
\par 20 왕의 보신 그 나무가 자라서 견고하여지고 그 고는 하늘에 닿았으니 땅 끝에서도 보이겠고
\par 21 그 잎사귀는 아름답고 그 열매는 많아서 만민의 식물이 될만하고 들짐승은 그 아래 거하며 공중에 나는 새는 그 가지에 깃들이더라 하시오니
\par 22 왕이여 이 나무는 곧 왕이시라 이는 왕이 자라서 견고하여지고 창대하사 하늘에 닿으시며 권세는 땅 끝까지 미치심이니이다
\par 23 "왕이 보신즉 한 순찰자, 한 거룩한 자가 하늘에서 내려와서 이르기를 그 나무를 베고 멸하라 그러나 그 뿌리의 그루터기는 땅에 남겨두고 철과 놋줄로 동이고 그것을 들 청초 가운데 있게 하라 그것이 하늘 이슬에 젖고 또 들짐승으로 더불어 그 분량을 같이 하며 일곱 때를 지내리라 하더라 하시오니"
\par 24 왕이여 그 해석은 이러하니이다 곧 지극히 높으신 자의 명정하신것이 내 주 왕에게 미칠 것이라
\par 25 왕이 사람에게서 쫓겨나서 들짐승과 함께 거하며 소처럼 풀을 먹으며 하늘 이슬에 젖을 것이요 이와 같이 일곱 때를 지낼 것이라 그때에 지극히 높으신 자가 인간 나라를 다스리시며 자기의 뜻대로 그것을 누구에게든지 주시는 줄을 아시리이다
\par 26 또 그들이 그 나무 뿌리의 그루터기를 남겨 두라 하였은즉 하나님이 다스리시는 줄을 왕이 깨달은 후에야 왕의 나라가 견고하리이다
\par 27 그런즉 왕이여 나의 간하는 것을 받으시고 공의를 행함으로 죄를 속하고 가난한 자를 긍휼히 여김으로 죄악을 속하소서 그리하시면 왕의 평안함이 혹시 장구하리이다 하였느니라
\par 28 이 모든 일이 다 나 느부갓네살 왕에게 임하였느니라
\par 29 열 두달이 지난 후에 내가 바벨론 궁 지붕에서 거닐새
\par 30 나 왕이 말하여 가로되 이 큰 바벨론은 내가 능력과 권세로 건설하여 나의 도성을 삼고 이것으로 내 위엄의 영광을 나타낸 것이 아니냐 하였더니
\par 31 이 말이 오히려 나 왕의 입에 있을 때에 하늘에서 소리가 내려 가로되 느부갓네살 왕아 네게 말하노니 나라의 위가 네게서 떠났느니라
\par 32 네가 사람에게서 쫓겨나서 들짐승과 함께 거하며 소처럼 풀을 먹을 것이요 이와 같이 일곱 때를 지내서 지극히 높으신 자가 인간나라를 다스리시며 자기의 뜻대로 그것을 누구에게든지 주시는 줄을 알기까지 이르리라 하더니
\par 33 그 동시에 이 일이 나 느부갓네살에게 응하므로 내가 사람에게 쫓겨나서 소처럼 풀을 먹으며 몸이 하늘 이슬에 젖고 머리털이 독수리 털과 같았고 손톱은 새 발톱과 같았었느니라
\par 34 그 기한이 차매 나 느부갓네살이 하늘을 우러러 보았더니 내 총명이 다시 내게로 돌아온지라 이에 내가 지극히 높으신 자에게 감사하며 영생하시는 자를 찬양하고 존경하였노니 그 권세는 영원한 권세요 그 나라는 대대로 이르리로다
\par 35 "땅의 모든 거민을 없는 것 같이 여기시며 하늘의 군사에게든지, 땅의 거민에게든지 그는 자기 뜻대로 행하시나니 누가 그의 손을 금하든지 혹시 이르기를 네가 무엇을 하느냐 할 자가 없도다"
\par 36 그 동시에 내 총명이 내게로 돌아왔고 또 나라 영광에 대하여도 내 위엄과 광명이 내게로 돌아왔고 또 나의 모사들과 관원들이 내게 조회하니 내가 내 나라에서 다시 세움을 입고 또 지극한 위세가 내게 더하였느니라
\par 37 그러므로 지금 나 느부갓네살이 하늘의 왕을 찬양하며 칭송하며 존경하노니 그의 일이 다 진실하고 그의 행하심이 의로우시므로 무릇 교만하게 행하는 자를 그가 능히 낮추심이니라

\chapter{5}

\par 1 벨사살왕이 그 귀인 일천명을 위하여 큰 잔치를 배설하고 그 일천명 앞에서 술을 마시니라
\par 2 "벨사살이 술을 마실 때에 명하여 그 부친 느부갓네살이 예루살렘 전에서 취하여 온 금, 은 기명을 가져오게 하였으니 이는 왕과 귀인들과 왕후들과 빈궁들이 다 그것으로 마시려 함이었더라"
\par 3 이에 예루살렘 하나님의 전 성소 중에서 취하여 온 금 기명을 가져오매 왕이 그 귀인들과 왕후들과 빈궁들로 더불어 그것으로 마시고
\par 4 "무리가 술을 마시고는 그 금, 은, 동, 철, 목, 석으로 만든 신들을 찬양하니라"
\par 5 그 때에 사람의 손가락이 나타나서 왕궁 촛대 맞은편 분벽에 글자를 쓰는데 왕이 그 글자 쓰는 손가락을 본지라
\par 6 이에 왕의 즐기던 빛이 변하고 그 생각이 번민하여 넓적다리 마디가 녹는 듯하고 그 무릎이 서로 부딪힌지라
\par 7 왕이 크게 소리하여 술객과 갈대아 술사와 점장이를 불러오게 하고 바벨론 박사들에게 일러 가로되 무론 누구든지 이 글자를 읽고 그 해석을 내게 보이면 자주옷을 입히고 금사슬로 그 목에 드리우고 그로 나라의 세째 치리자를 삼으리라 하니라
\par 8 때에 왕의 박사가 다 들어왔으나 능히 그 글자를 읽지 못하여 그 해석을 왕께 알게 하지 못하는지라
\par 9 그러므로 벨사살 왕이 크게 번민하여 그 낯빛이 변하였고 귀인들도 다 놀라니라
\par 10 태후가 왕과 그 귀인들의 말로 인하여 잔치하는 궁에 들어 왔더니 이에 말하여 가로되 왕이여 만세수를 하옵소서 왕의 생각을 번민케 말며 낯빛을 변할 것이 아니니이다
\par 11 왕의 나라에 거룩한 신들의 영이 있는 사람이 있으니 곧 왕의 부친 때에 있던 자로서 명철과 총명과 지혜가 있어 신들의 지혜와 같은 자라 왕의 부친 느부갓네살 왕이 그를 세워 박수와 술객과 갈대아 술사와 점장이의 어른을 삼으셨으니
\par 12 왕이 벨드사살이라 이름한 이 다니엘의 마음이 민첩하고 지식과 총명이 있어 능히 꿈을 해석하며 은밀한 말을 밝히며 의문을 파할 수 있었음이라 이제 다니엘을 부르소서 그리하시면 그가 그 해석을 알려드리리이다
\par 13 이에 다니엘이 부름을 입어 왕의 앞에 나오매 왕이 다니엘에게 말하여 가로되 네가 우리 부왕이 유다에서 사로잡아 온 유다 자손 중의 그 다니엘이냐
\par 14 내가 네게 대하여 들은즉 네 안에는 신들의 영이 있으므로 네가 명철과 총명과 비상한 지혜가 있다 하도다
\par 15 지금 여러 박사와 술객을 내 앞에 불러다가 그들로 이 글을 읽고 그 해석을 내게 알게 하라 하였으나 그들이 다 능히 그 해석을 내게 보이지 못하였느니라
\par 16 내가 네게 대하여 들은즉 너는 해석을 잘하고 의문을 파한다 하도다 그런즉 이제 네가 이 글을 읽고 그 해석을 내게 알게 하면 네게 자주옷을 입히고 금사슬을 네 목에 드리우고 너로 나라의 세째 치리자를 삼으리라
\par 17 다니엘이 왕에게 대답하여 가로되 왕의 예물은 왕이 스스로 취하시며 왕의 상급은 다른 사람에게 주옵소서 그럴지라도 내가 왕을 위하여 이 글을 읽으며 그 해석을 아시게 하리이다
\par 18 왕이여 지극히 높으신 하나님이 왕의 부친 느부갓네살에게 나라와 큰 권세와 영광과 위엄을 주셨고
\par 19 그에게 큰 권세를 주셨으므로 백성들과 나라들과 각 방언하는 자들이 그의 앞에서 떨며 두려워하였으며 그는 임의로 죽이며 임의로 살리며 임의로 높이며 임의로 낮추었더니
\par 20 그가 마음이 높아지며 뜻이 강퍅하여 교만을 행하므로 그 왕위가 폐한 바 되며 그 영광을 빼앗기고
\par 21 인생 중에서 쫓겨나서 그 마음이 들짐승의 마음과 같았고 또 들 나귀와 함께 거하며 또 소처럼 풀을 먹으며 그 몸이 하늘 이슬에 젖었으며 지극히 높으신 하나님이 인간 나라를 다스리시며 자기의 뜻대로 누구든지 그 위에 세우시는 줄을 알기까지 이르게 되었었나이다
\par 22 벨사살이여 왕은 그의 아들이 되어서 이것을 다 알고도 오히려 마음을 낮추지 아니하고
\par 23 "도리어 스스로 높여서 하늘의 주재를 거역하고 그 전 기명을 왕의 앞으로 가져다가 왕과 귀인들과 왕후들과 빈궁들이 다 그것으로 술을마시고 왕이 또 보지도 듣지도 알지도 못하는 금, 은, 동, 철과 목, 석으로 만든 신상들을 찬양하고 도리어 왕의 호흡을 주장하시고 왕의 모든길을 작정하시는 하나님께는 영광을 돌리지 아니한지라"
\par 24 이러므로 그의 앞에서 이 손가락이 나와서 이 글을 기록하였나이다 (23절에서 기록을 다 못함)
\par 25 기록한 글자는 이것이니 곧 메네 메네 데겔 우바르신이라
\par 26 그 뜻을 해석하건대 메네는 하나님이 이미 왕의 나라의 시대를 세어서 그것을 끝나게 하셨다 함이요
\par 27 데겔은 왕이 저울에 달려서 부족함이 뵈었다 함이요
\par 28 베레스는 왕의 나라가 나뉘어서 메대와 바사 사람에게 준바 되었다 함이니이다
\par 29 이에 벨사살이 명하여 무리로 다니엘에게 자주옷을 입히게 하며 금 사슬로 그의 목에 드리우게 하고 그를 위하여 조서를 내려 나라의 세째 치리자를 삼으니라
\par 30 그날 밤에 갈대아 왕 벨사살이 죽임을 당하였고
\par 31 메대 사람 다리오가 나라를 얻었는데 때에 다리오는 육십 이세였더라

\chapter{6}

\par 1 다리오가 자기의 심원대로 방백 일백 이십 명을 세워 전국을 통치하게 하고
\par 2 또 그들 위에 총리 셋을 두었으니 다니엘이 그 중에 하나이라 이는 방백들로 총리에게 자기의 직무를 보고하게 하여 왕에게 손해가 없게 하려함이었더라
\par 3 다니엘은 마음이 민첩하여 총리들과 방백들 위에 뛰어나므로 왕이 그를 세워 전국을 다스리게 하고자 한지라
\par 4 "이에 총리들과 방백들이 국사에 대하여 다니엘을 고소할 틈을 얻고자 하였으나 능히 아무 틈, 아무 허물을 얻지 못하였으니 이는 그가 충성되어 아무 그릇함도 없고 아무 허물도 없음이었더라"
\par 5 그 사람들이 가로되 이 다니엘은 그 하나님의 율법에 대하여 그 틈을 얻지 못하면 그를 고소할 수 없으리라 하고
\par 6 이에 총리들과 방백들이 모여 왕에게 나아가서 그에게 말하되 다리오 왕이여 만세수를 하옵소서
\par 7 나라의 모든 총리와 수령과 방백과 모사와 관원이 의논하고 왕에게 한 율법을 세우며 한 금령을 정하실 것을 구하려 하였는데 왕이여 그것은 곧 이제부터 삼십 일 동안에 누구든지 왕 외에 어느신에게나 사람에게 무엇을 구하면 사자굴에 던져 넣기로 한 것이니이다
\par 8 그런즉 원컨대 금령을 세우시고 그 조서에 어인을 찍어서 메대와 바사의 변개치 아니하는 규례를 따라 그것을 다시 고치지 못하게 하옵소서 하매
\par 9 이에 다리오 왕이 조서에 어인을 찍어 금령을 내니라
\par 10 다니엘이 이 조서에 어인이 찍힌 것을 알고도 자기 집에 돌아가서는 그 방의 예루살렘으로 향하여 열린 창에서 전에 행하던대로 하루 세번씩 무릎을 꿇고 기도하며 그 하나님께 감사하였더라
\par 11 그 무리들이 모여서 다니엘이 자기 하나님 앞에 기도하며 간구하는 것을 발견하고
\par 12 이에 그들이 나아가서 왕의 금령에 대하여 왕께 아뢰되 왕이여 왕이 이미 금령에 어인을 찍어서 이제부터 삼십 일 동안에 누구든지 왕 외에 어느 신에게나 사람에게 구하면 사자굴에 던져 넣기로 하지 아니하였나이까 왕이 대답하여 가로되 이 일이 적실하니 메대와 바사의 변개치 아니하는 규례대로 된 것이니라
\par 13 그들이 왕 앞에서 대답하여 가로되 왕이여 사로잡혀 온 유다 자손 중에 그 다니엘이 왕과 왕의 어인이 찍힌 금령을 돌아보지 아니하고 하루 세번씩 기도하나이다
\par 14 왕이 이 말을 듣고 그로 인하여 심히 근심하여 다니엘을 구원하려고 마음을 쓰며 그를 건져 내려고 힘을 다하여 해가 질 때까지 이르매
\par 15 그 무리들이 또 모여 왕에게로 나아와서 왕께 말씀하되 왕이여 메대와 바사의 규례를 아시거니와 왕의 세우신 금령과 법도는 변개하지 못할 것이니이다
\par 16 이에 왕이 명하매 다니엘을 끌어다가 사자굴에 던져 넣는지라 왕이 다니엘에게 일러 가로되 너의 항상 섬기는 네 하나님이 너를 구원하시리라 하니라
\par 17 이에 돌을 굴려다가 굴 아구를 막으매 왕이 어인과 귀인들의 인을 쳐서 봉하였으니 이는 다니엘 처치한 것을 변개함이 없게 하려 함이었더라
\par 18 왕이 궁에 돌아가서는 밤이 맞도록 금식하고 그 앞에 기악을 그치고 침수를 폐하니라
\par 19 이튿날에 왕이 새벽에 일어나 급히 사자굴로 가서
\par 20 다니엘의 든 굴에 가까이 이르러는 슬피 소리질러 다니엘에게 물어 가로되 사시는 하나님의 종 다니엘아 너의 항상 섬기는 네 하나님이 사자에게서 너를 구원하시기에 능하셨느냐
\par 21 다니엘이 왕에게 고하되 왕이여 원컨대 왕은 만세수를 하옵소서
\par 22 나의 하나님이 이미 그 천사를 보내어 사자들의 입을 봉하셨으므로 사자들이 나를 상해치 아니하였사오니 이는 나의 무죄함이 그앞에 명백함이오며 또 왕이여 나는 왕의 앞에도 해를 끼치지 아니하였나이다
\par 23 왕이 심히 기뻐서 명하여 다니엘을 굴에서 올리라 하매 그들이 다니엘을 굴에서 올린즉 그 몸이 조금도 상하지 아니하였으니 이는 그가 자기 하나님을 의뢰함이었더라
\par 24 왕이 명을 내려 다니엘을 참소한 사람들을 끌어오게 하고 그들을 그 처자들과 함께 사자굴에 던져 넣게 하였더니 그들이 굴 밑에 닿기 전에 사자가 곧 그들을 움켜서 그 뼈까지도 부숴뜨렸더라
\par 25 이에 다리오 왕이 온 땅에 있는 모든 백성과 나라들과 각 방언하는 자들에게 조서를 내려 가로되 원컨대 많은 평강이 너희에게 있을지어다
\par 26 내가 이제 조서를 내리노라 내 나라 관할 아래 있는 사람들은 다 다니엘의 하나님 앞에서 떨며 두려워할지니 그는 사시는 하나님 이시요 영원히 변치 않으실 자시며 그 나라는 망하지 아니할 것이요 그 권세는 무궁할 것이며
\par 27 그는 구원도 하시며 건져내기도 하시며 하늘에서든지 땅에서든지 이적과 기사를 행하시는 자로서 다니엘을 구원하여 사자의 입에서 벗어나게 하셨음이니라 하였더라
\par 28 이 다니엘이 다리오 왕의 시대와 바사 사람 고레스 왕의 시대에 형통하였더라

\chapter{7}

\par 1 바벨론 왕 벨사살 원년에 다니엘이 그 침상에서 꿈을 꾸며 뇌 속으로 이상을 받고 그 꿈을 기록하며 그 일의 대략을 진술하니라
\par 2 다니엘이 진술하여 가로되 내가 밤에 이상을 보았는데 하늘의 네바람이 큰 바다로 몰려 불더니
\par 3 큰 짐승 넷이 바다에서 나왔는데 그 모양이 각각 다르니
\par 4 첫째는 사자와 같은데 독수리의 날개가 있더니 내가 볼 사이에 그 날개가 뽑혔고 또 땅에서 들려서 사람처럼 두 발로 서게 함을 입었으며 또 사람의 마음을 받았으며
\par 5 다른 짐승 곧 둘째는 곰과 같은데 그것이 몸 한편을 들었고 그 입의 잇사이에는 세 갈빗대가 물렸는데 그에게 말하는 자가 있어 이르기를 일어나서 많은 고기를 먹으라 하였으며
\par 6 그 후에 내가 또 본즉 다른 짐승 곧 표범과 같은 것이 있는데 그 등에는 새의 날개 넷이 있고 그 짐승에게 또 머리 넷이 있으며 또 권세를 받았으며
\par 7 내가 밤 이상 가운데 그 다음에 본 네째 짐승은 무섭고 놀라우며 또 극히 강하며 또 큰 철 이가 있어서 먹고 부숴뜨리고 그 나머지를 발로 밟았으며 이 짐승은 전의 모든 짐승과 다르고 또 열 뿔이 있으므로
\par 8 내가 그 뿔을 유심히 보는 중 다른 작은 뿔이 그 사이에서 나더니 먼저 뿔 중에 셋이 그 앞에 뿌리까지 뽑혔으며 이 작은 뿔에는 사람의 눈 같은 눈이 있고 또 입이 있어 큰 말을 하였느니라
\par 9 내가 보았는데 왕좌가 놓이고 옛적부터 항상 계신 이가 좌정하셨는데 그 옷은 희기가 눈 같고 그 머리털은 깨끗한 양의 털같고 그 보좌는 불꽃이요 그 바퀴는 붙는 불이며
\par 10 불이 강처럼 흘러 그 앞에서 나오며 그에게 수종하는 자는 천천이요 그 앞에 시위한 자는 만만이며 심판을 베푸는데 책들이 펴 놓였더라
\par 11 그때에 내가 그 큰 말하는 작은 뿔의 목소리로 인하여 주목하여 보는 사이에 짐승이 죽임을 당하고 그 시체가 상한 바 되어 붙는 불에 던진 바 되었으며
\par 12 그 남은 모든 짐승은 그 권세를 빼았겼으나 그 생명은 보존되어 정한 시기가 이르기를 기다리게 되었더라
\par 13 내가 또 밤 이상 중에 보았는데 인자 같은 이가 하늘 구름을 타고 와서 옛적부터 항상 계신 자에게 나아와 그 앞에 인도되매
\par 14 그에게 권세와 영광과 나라를 주고 모든 백성과 나라들과 각 방언하는 자로 그를 섬기게 하였으니 그 권세는 영원한 권세라 옮기지 아니할 것이요 그 나라는 폐하지 아니할 것이니라
\par 15 나 다니엘이 중심에 근심하며 내 뇌 속에 이상이 나로 번민케 한지라
\par 16 내가 그 곁에 모신 자 중 하나에게 나아가서 이 모든 일의 진상을 물으매 그가 내게 고하여 그 일의 해석을 알게 하여 가로되
\par 17 그 네 큰 짐승은 네 왕이라 세상에 일어날 것이로되
\par 18 지극히 높으신 자의 성도들이 나라를 얻으리니 그 누림이 영원하고 영원하고 영원하리라
\par 19 이에 내가 네째 짐승의 진상을 알고자 하였으니 곧 그것은 모든 짐승과 달라서 심히 무섭고 그 이는 철이요 그 발톱은 놋이며 먹고 부숴뜨리고 나머지는 발로 밟았으며
\par 20 또 그것의 머리에는 열 뿔이 있고 그 외에 또 다른 뿔이 나오매 세 뿔이 그 앞에 빠졌으며 그 뿔에는 눈도 있고 큰 말하는 입도 있고 그 모양이 동류보다 강하여 보인 것이라
\par 21 내가 본즉 이 뿔이 성도들로 더불어 싸워 이기었더니
\par 22 옛적부터 항상 계신 자가 와서 지극히 높으신 자의 성도를 위하여 신원하셨고 때가 이르매 성도가 나라를 얻었더라
\par 23 모신 자가 이처럼 이르되 네째 짐승은 곧 땅의 네째 나라인데 이는 모든 나라보다 달라서 천하를 삼키고 밞아 부숴뜨릴 것이며
\par 24 그 열 뿔은 이 나라에서 일어날 열 왕이요 그 후에 또 하나가 일어나리니 그는 먼저 있던 자들과 다르고 또 세 왕을 복종시킬 것이며
\par 25 그가 장차 말로 지극히 높으신 자를 대적하며 또 지극히 높으신 자의 성도를 괴롭게 할 것이며 그가 또 때와 법을 변개코자 할 것이며 성도는 그의 손에 붙인 바 되어 한 때와 두 때와 반 때를 지내리라
\par 26 그러나 심판이 시작된즉 그는 권세를 빼앗기고 끝까지 멸망할 것이요
\par 27 나라와 권세와 온 천하 열국의 위세가 지극히 높으신 자의 성민에게 붙인 바 되리니 그의 나라는 영원한 나라이라 모든 권세 있는 자가 다 그를 섬겨 복종하리라 하여
\par 28 그 말이 이에 그친지라 나 다니엘은 중심이 번민하였으며 내 낯빛이 변하였으나 내가 이 일을 마음에 감추었느니라

\chapter{8}

\par 1 나 다니엘에게 처음에 나타난 이상 후 벨사살 왕 삼년에 다시 이상이 나타나니라
\par 2 내가 이상을 보았는데 내가 그것을 볼 때에 내 몸은 엘람도 수산성에 있었고 내가 이상을 보기는 을래 강변에서니라
\par 3 내가 눈을 들어본즉 강 가에 두 뿔 가진 수양이 섰는데 그 두 뿔이 다 길어도 한 뿔은 다른 뿔보다도 길었고 그 긴 것은 나중에 난 것이더라
\par 4 내가 본즉 그 수양이 서와 북과 남을 향하여 받으나 그것을 당할 짐승이 하나도 없고 그 손에서 능히 구할이가 절대로 없으므로 그것이 임의로 행하고 스스로 강대하더라
\par 5 내가 생각할 때에 한 수염소가 서편에서부터 와서 온 지면에 두루 다니되 땅에 닿지 아니하며 그 염소 두 눈 사이에는 현저한 뿔이 있더라
\par 6 그것이 두 뿔 가진 수양 곧 내가 본바 강가에 섰던 양에게로 나아가되 분노한 힘으로 그것에게로 달려가더니
\par 7 내가 본즉 그것이 수양에게로 가까이 나아가서는 더욱 성내어 그 수양을 땅에 엎드러뜨리고 짓밟았으나 능히 수양을 그 손에서 벗어나게 할 이가 없었더라
\par 8 수염소가 스스로 심히 강대하여 가더니 강성할 때에 그 큰 뿔이 꺾이고 그 대신에 현저한 뿔 넷이 하늘 사방을 향하여 났더라
\par 9 그 중 한 뿔에서 또 작은 뿔 하나가 나서 남편과 동편과 또 영화로운 땅을 향하여 심히 커지더니
\par 10 그것이 하늘 군대에 미칠만큼 커져서 그 군대와 별 중에 몇을 땅에 떨어뜨리고 그것을 짓밞고
\par 11 또 스스로 높아져서 군대의 주재를 대적하며 그에게 매일 드리는 제사를 제하여 버렸고 그의 성소를 헐었으며
\par 12 범죄함을 인하여 백성과 매일 드리는 제사가 그것에게 붙인바 되었고 그것이 또 진리를 땅에 던지며 자의로 행하여 형통하였더라
\par 13 내가 들은즉 거룩한 자가 말하더니 다른 거룩한 자가 그 말하는 자에게 묻되 이상에 나타난바 매일 드리는 제사와 망하게 하는 죄악에 대한 일과 성소와 백성이 내어준바 되며 짓밟힐 일이 어느때까지 이를꼬 하매
\par 14 그가 내게 이르되 이천 삼백 주야까지니 그 때에 성소가 정결하게 함을 입으리라 하였느니라
\par 15 나 다니엘이 이 이상을 보고 그 뜻을 알고자 할 때에 사람 모양 같은 것이 내 앞에 섰고
\par 16 내가 들은즉 을래 강 두 언덕 사이에서 사람의 목소리가 있어 외쳐 이르되 가브리엘아 이 이상을 이 사람에게 깨닫게 하라 하더니
\par 17 그가 나의 선 곳으로 나아왔는데 그 나아올 때에 내가 두려워서 얼굴을 땅에 대고 엎드리매 그가 내게 이르되 인자야 깨달아 알라 이 이상은 정한 때 끝에 관한 것이니라
\par 18 그가 내게 말할 때에 내가 얼굴을 땅에 대고 엎드리어 깊이 잠들매 그가 나를 어루만져서 일으켜 세우며
\par 19 가로되 진노하시는 때가 마친 후에 될 일을 내가 네게 알게 하리니 이 이상은 정한 때 끝에 관한 일임이니라
\par 20 네가 본바 두 뿔 가진 수양은 곧 메대와 바사 왕들이요
\par 21 털이 많은 수염소는 곧 헬라 왕이요 두 눈 사이에 있는 큰 뿔은 곧 그 첫째 왕이요
\par 22 이 뿔이 꺾이고 그 대신에 네 뿔이 났은즉 그 나라 가운데서 네 나라가 일어나되 그 권세만 못하리라
\par 23 이 네 나라 마지막 때에 패역자들이 가득할 즈음에 한 왕이 일어나리니 그 얼굴은 엄장하며 궤휼에 능하며
\par 24 그 권세가 강할 것이나 자기의 힘으로 말미암은 것이 아니며 그가 장차 비상하게 피괴를 행하고 자의로 행하여 형통하며 강한 자들과 거룩한 백성을 멸하리라
\par 25 그가 꾀를 베풀어 제 손으로 궤휼을 이루고 마음에 스스로 큰 체하며 또 평화한 때에 많은 무리를 멸하며 또 스스로 서서 만왕의 왕을 대적할 것이나 그가 사람의 손을 말미암지 않고 깨어지리라
\par 26 이미 말한바 주야에 대한 이상이 확실하니 너는 그 이상을 간수 하라 이는 여러 날 후의 일임이니라
\par 27 이에 나 다니엘이 혼절하여 수일을 앓다가 일어나서 왕의 일을 보았느니라 내가 그 이상을 인하여 놀랐고 그 뜻을 깨닫는 사람도 없었느니라

\chapter{9}

\par 1 메대 족속 아하수에로의 아들 다리오가 갈대아 나라 왕으로 세움을 입던 원년
\par 2 곧 그 통치 원년에 나 다니엘이 서책으로 말미암아 여호와의 말씀이 선지자 예레미야에게 임하여 고하신 그 년수를 깨달았나니 곧 예루살렘의 황무함이 칠십 년만에 마치리라 하신 것이니라
\par 3 내가 금식하며 베옷을 입고 재를 무릅쓰고 주 하나님께 기도하며 간구하기를 결심하고
\par 4 "내 하나님 여호와께 기도하며 자복하여 이르기를 크시고 두려워 할 주 하나님, 주를 사랑하고 주의 계명을 지키는 자를 위하여 언약을 지키시고 그에게 인자를 베푸시는 자시여"
\par 5 우리는 이미 범죄하여 패역하며 행악하며 반역하여 주의 법도와 규례를 떠났사오며
\par 6 우리가 또 주의 종 선지자들이 주의 이름으로 우리의 열왕과 우리의 방백과 열조와 온 국민에게 말씀한 것을 듣지 아니하였나이다
\par 7 주여 공의는 주께로 돌아가고 수욕은 우리 얼굴로 돌아옴이 오늘날과 같아서 유다 사람들과 예루살렘 거민들과 이스라엘이 가까운데 있는자나 먼데 있는 자가 다 주께서 쫓아 보내신 각국에서 수욕을 입었사오니 이는 그들이 주께 죄를 범하였음이니이다
\par 8 주여 수욕이 우리에게 돌아오고 우리의 열왕과 우리의 방백과 열조에게 돌아온 것은 우리가 주께 범죄하였음이니이다 마는
\par 9 주 우리 하나님께는 긍휼과 사유하심이 있사오니 이는 우리가 주께 패역하였음이오며
\par 10 우리 하나님 여호와의 목소리를 청종치 아니하며 여호와께서 그 종 선지자들에게 부탁하여 우리 앞에 세우신 율법을 행치 아니하였음이니이다
\par 11 온 이스라엘이 주의 율법을 범하고 치우쳐 가서 주의 목소리를 청종치 아니하였으므로 이 저주가 우리에게 내렸으되 곧 하나님의 종 모세의 율법 가운데 기록된 맹세대로 되었사오니 이는 우리가 주께 범죄하였음이니이다
\par 12 주께서 큰 재앙을 우리에게 내리사 우리와 및 우리를 재판하던 재판관을 쳐서 하신 말씀을 이루셨사오니 온 천하에 예루살렘에 임한 일 같은 것이 없나이다
\par 13 모세의 율법에 기록된 대로 이 모든 재앙이 이미 우리에게 임하였사오나 우리는 우리의 죄악을 떠나고 주의 진리를 깨닫도록 우리 하나님 여호와의 은총을 간구치 아니하였나이다
\par 14 이러므로 여호와께서 이 재앙을 간직하여 두셨다가 우리에게 임하게 하셨사오니 우리의 하나님 여호와는 행하시는 모든 일이 공의로우시나 우리가 그 목소리를 청종치 아니하였음이니이다
\par 15 강한 손으로 주의 백성을 애굽 땅에서 인도하여 내시고 오늘과 같이 명성을 얻으신 우리 주 하나님이여 우리가 범죄하였고 악을 행하였나이다
\par 16 "주여 내가 구하옵나니 주는 주의 공의를 좇으사 주의 분노를 주 의 성 예루살렘, 주의 거룩한 산에서 떠나게 하옵소서 이는 우리의 죄와 우리의 열조의 죄악을 인하여 예루살렘과 주의 백성이 사면에 있는 자에게 수욕을 받음이니이다"
\par 17 그러하온즉 우리 하나님이여 지금 주의 얼굴 종의 기도와 간구를 들으시고 주를 위하여 주의 얼굴 빛을 주의 황폐한 성소에 비취시옵소서
\par 18 나의 하나님이여 귀를 기울여 들으시며 눈을 떠서 우리의 황폐된 상황과 주의 이름으로 일컫는 성을 보옵소서 우리가 주의 앞에 간구하옵는 것은 우리의 의를 의지하여 하는 것이 아니요 주의 큰 긍휼을 의지하여 함이오니
\par 19 주여 들으소서 주여 용서하소서 주여 들으시고 행하소서 지체치 마옵소서 나의 하나님이여 주 자신을 위하여 하시옵소서 이는 주의 성과 주의 백성이 주의 이름으로 일컫는 바 됨이니이다
\par 20 내가 이같이 말하여 기도하며 내 죄와 및 내 백성 이스라엘의 죄를 자복하고 내 하나님의 거룩한 산을 위하여 내 하나님 여호와 앞에 간구할 때
\par 21 곧 내가 말하여 기도할 때에 이전 이상 중에 본 그 사람 가브리엘이 빨리 날아서 저녁 제사를 드릴 때 즈음에 내게 이르더니
\par 22 내게 가르치며 내게 말하여 가로되 다니엘아 내가 이제 네게 지혜와 총명을 주려고 나왔나니
\par 23 곧 네가 기도를 시작할 즈음에 명령이 내렸으므로 이제 네게 고하러 왔느니라 너는 크게 은총을 입은 자라 그런즉 너는 이 일을 생각하고 그 이상을 깨달을지니라
\par 24 네 백성과 네 거룩한 성을 위하여 칠십 이레로 기한을 정하였나니 허물이 마치며 죄가 끝나며 죄악이 영속되며 영원한 의가 드러나며 이상과 예언이 응하며 또 지극히 거룩한 자가 기름부음을 받으리라
\par 25 그러므로 너는 깨달아 알지니라 예루살렘을 중건하라는 영이 날 때부터 기름부음을 받은 자 곧 왕이 일어나기까지 일곱 이레와 육십 이 이레가 지날 것이요 그 때 곤란한 동안에 성이 중건되어 거리와 해자가 이룰 것이며
\par 26 육십 이 이레 후에 기름부음을 받은 자가 끊어져 없어질 것이며 장차 한 왕의 백성이 와서 그 성읍과 성소를 훼파하려니와 그의 종말은 홍수에 엄몰됨 같을 것이며 또 끝까지 전쟁이 있으리니 황폐할 것이 작정되었느니라
\par 27 그가 장차 많은 사람으로 더불어 한 이레 동안의 언약을 굳게 정하겠고 그가 그 이레의 절반에 제사와 예물을 금지할 것이며 또 잔포하여 미운 물건이 날개를 의지하여 설 것이며 또 이미 정한 종말까지 진노가 황페케 하는 자에게 쏟아지리라 하였느니라

\chapter{10}

\par 1 바사 왕 고레스 삼년에 한 일이 벨드사살이라 이름한 다니엘에게 나타났는데 그 일이 참되니 곧 큰 전쟁에 관한 것이라 다니엘이 그 일을 분명히 알았고 그 이상을 깨달으니라
\par 2 그때에 나 다니엘이 세 이레 동안을 슬퍼하며
\par 3 세 이레가 차기까지 좋은 떡을 먹지 아니하며 고기와 포도주를 입에 넣지 아니하며 또 기름을 바르지 아니하니라
\par 4 정월 이십 사일에 내가 힛데겔이라 하는 큰 강가에 있었는데
\par 5 그때에 내가 눈을 들어 바라본즉 한 사람이 세마포 옷을 입었고 허리에는 우바스 정금 띠를 띠었고
\par 6 그 몸은 황옥 같고 그 얼굴은 번갯빛 같고 그 눈은 횃불 같고 그 팔과 발은 빛난 놋과 같고 그 말소리는 무리의 소리와 같더라
\par 7 이 이상은 나 다니엘이 홀로 보았고 나와 함께한 사람들은 이 이상은 보지 못하였어도 그들이 크게 떨며 도망하여 숨었었느니라
\par 8 그러므로 나만 홀로 있어서 이 큰 이상을 볼 때에 내 몸에 힘이 빠졌고 나의 아름다운 빛이 변하여 썩은 듯하였고 나의 힘이 다 없어졌으나
\par 9 내가 그 말소리를 들었는데 그 말소리를 들을 때에 내가 얼굴을 땅에 대고 깊이 잠들었었느니라
\par 10 한 손이 있어 나를 어루만지기로 내가 떨더니 그가 내 무릎과 손바닥이 땅에 닿게 일으키고
\par 11 내게 이르되 은총을 크게 받은 사람 다니엘아 내가 네게 이르는 말을 깨닫고 일어서라 내가 네게 보내심을 받았느니라 그가 내게 이 말을 한 후에 내가 떨며 일어서매
\par 12 그가 이르되 다니엘아 두려워하지 말라 네가 깨달으려 하여 네 하나님 앞에 스스로 겸비케 하기로 결심하던 첫 날부터 네 말이 들으신 바 되었으므로 내가 네 말로 인하여 왔느니라
\par 13 그런데 바사국 군이 이십 일일 동안 나를 막았으므로 내가 거기 바사국 왕들과 함께 머물러 있더니 군장 중 하나 미가엘이 와서 나를 도와주므로
\par 14 이제 내가 말일에 네 백성의 당할 일을 네게 깨닫게 하러 왔노라 대저 이 이상은 오래 후의 일이니라
\par 15 그가 이런 말로 내게 이를 때에 내가 곧 얼굴을 땅에 향하고 벙벙하였더니
\par 16 인자와 같은 이가 있어 내 입술을 만진지라 내가 곧 입을 열어 내 앞에 섰는 자에게 말하여 가로되 내 주여 이 이상을 인하여 근심이 내게 더하므로 내가 힘이 없어졌나이다
\par 17 내 몸에 힘이 없어졌고 호흡이 남지 아니하였사오니 내 주의 이 종이 어찌 능히 내 주로 더불어 말씀할 수 있으리이까
\par 18 또 사람의 모양 같은 것 하나가 나를 만지며 나로 강건케 하여
\par 19 가로되 은총을 크게 받은 사람이여 두려워하지 말라 평안하라 강건하라 강건하라 그가 이같이 내게 말하매 내가 곧 힘이 나서 가로되 내 주께서 나로 힘이 나게 하셨사오니 말씀하옵소서
\par 20 그가 이르되 내가 어찌하여 네게 나아온 것을 네가 아느냐 이제 내가 돌아가서 바사 군과 싸우려니와 내가 나간 후에는 헬라군이 이를 것이라
\par 21 오직 내가 먼저 진리의 글에 기록된 것으로 네게 보이리라 나를 도와서 그들을 대적하는 자는 너희 군 미가엘 뿐이니라

\chapter{11}

\par 1 내가 또 메대 사람 다리오 원년에 일어나 그를 돕고 강하게 한 일이 있었느니라
\par 2 이제 내가 참된 것을 네게 보이리라 보라 바사에서 또 세 왕이 일어날 것이요 그 후의 네째는 그들보다 심히 부요할 것이며 그가 그 부요함으로 강하여진 후에는 모든 사람을 격동시켜 헬라국을 칠 것이며
\par 3 장차 한 능력 있는 왕이 일어나서 큰 권세로 다스리며 임의로 행하리라
\par 4 그러나 그가 강성할 때에 그 나라가 갈라져 천하 사방에 나누일 것이나 그 자손에게로 돌아가지도 아니할 것이요 또 자기가 주장하던 권세대로도 되지 아니하리니 이는 그 나라가 뽑혀서 이 외의 사람들에게로 돌아갈 것임이니라
\par 5 남방의 왕은 강할 것이나 그 군들 중에 하나는 그보다 강하여 권세를 떨치리니 그 권세가 심히 클 것이요
\par 6 몇 해 후에 그들이 서로 맹약하리니 곧 남방 왕의 딸이 북방 왕에게 나아가서 화친하리라 그러나 이 공주의 힘이 쇠하고 그 왕은 서지도 못하며 권세가 없어질 뿐 아니라 이 공주와 그를 데리고 온 자와 그를 낳은 자와 그 때에 도와주던 자가 다 버림을 당하리라
\par 7 그러나 이 공주의 본족에서 난 자 중에 하나가 그의 위를 이어 북방 왕의 군대를 치러 와서 그의 성에 들어가서 그들을 이기고
\par 8 그 신들과 부어만든 우상들과 그 은과 금의 아름다운 기구를 다 노략하여 애굽으로 가져갈 것이요 몇 해 동안은 그가 북방 왕을 치지 아니하리라
\par 9 북방 왕이 남방 왕의 나라로 쳐 들어갈 것이나 자기 본국으로 물러 가리라
\par 10 그 아들들이 전쟁을 준비하고 심히 많은 군대를 모아서 물의 넘침 같이 나아올 것이며 그가 또 와서 남방 왕의 견고한 성까지 칠 것이요
\par 11 남방 왕은 크게 노하여 나와서 북방 왕과 싸울 것이라 북방 왕이 큰 무리를 일으킬 것이나 그 무리가 그의 손에 붙인바 되리라
\par 12 그가 큰 무리를 사로잡은 후에 그 마음이 스스로 높아져서 수만 명을 엎드러뜨릴 것이나 그 세력은 더하지 못할 것이요
\par 13 북방 왕은 돌아가서 다시 대군을 전보다 더 많이 준비하였다가 몇 때 곧 몇 해 후에 대군과 많은 물건을 거느리고 오리라
\par 14 그 때에 여러 사람이 일어나서 남방 왕을 칠 것이요 네 백성 중에서도 강포한 자가 스스로 높아져서 이상을 이루려 할 것이나 그들이 도리어 넘어지리라
\par 15 이에 북방 왕은 와서 토성을 쌓고 견고한 성읍을 취할 것이요 남방 군대는 그를 당할 힘이 없을 것이므로
\par 16 오직 와서 치는 자가 임의로 행하리니 능히 그 앞에 설 사람이 없겠고 그가 영화로운 땅에 설 것이요 그 손에 멸망이 있으리라
\par 17 그가 결심하고 전국의 힘을 다하여 이르렀다가 그와 화친할 것이요 또 여자의 딸을 그에게 주어 그 나라를 패망케 하려 할 것이나 이루지 못하리니 그에게 무익하리라
\par 18 그 후에 그가 얼굴을 섬들로 돌이켜 많이 취할 것이나 한 대장이 있어서 그의 보이는 수욕을 씻고 그 수욕을 그에게로 돌릴 것이므로
\par 19 그가 드디어 그 얼굴을 돌이켜 자기 땅 산성들로 향할 것이나 거쳐 넘어지고 다시는 보이지 아니하리라
\par 20 그 위를 이을 자가 토색하는 자로 그 나라의 아름다운 곳으로 두루 다니게 할 것이나 그는 분노함이나 싸움이 없이 몇 날이 못되어 망할 것이요
\par 21 또 그 위를 이을 자는 한 비천한 사람이라 나라 영광을 그에게 주지 아니할 것이나 그가 평안한 때를 타서 궤휼로 그 나라를 얻을 것이며
\par 22 넘치는 물 같은 군대가 그에게 넘침을 입어 패할 것이요 동맹한 왕도 그렇게 될 것이며
\par 23 그와 약조한 후에 그는 거짓을 행하여 올라올 것이요 적은 백성을 거느리고 강하게 될 것이며
\par 24 그가 평안한 때에 그 도의 가장 기름진 곳에 들어와서 그 열조와 열조의 조상이 행하지 못하던 것을 행할 것이요 그는 노략하며 탈취한 재물을 무리에게 흩어주며 모략을 베풀어 얼마 동안 산성들을 칠 것인데 때가 이르기까지 그리하리라
\par 25 그가 그 힘을 떨치며 용맹을 발하여 큰 군대를 거느리고 남방 왕도 심히 크고 강한 군대를 거느리고 맞아 싸울 것이나 능히 당하지 못하리니 이는 그들이 모략을 베풀어 그를 침이니라
\par 26 자기의 진미를 먹는 자가 그를 멸하리니 그 군대가 흩어질 것이요 많은 자가 엎드러져 죽으리라
\par 27 이 두 왕이 마음에 서로 해코자 하여 한 밥상에 앉았을 때에 거짓말을 할 것이라 일이 형통하지 못하리니 이는 작정된 기한에 미쳐서 그 일이 끝날 것임이니라
\par 28 북방 왕은 많은 재물을 가지고 본국으로 돌아가리니 그는 마음으로 거룩한 언약을 거스리며 임의로 행하고 본토로 돌아갈 것이며
\par 29 작정된 기한에 그가 다시 나와서 남방에 이를 것이나 이번이 그 전번만 못하리니
\par 30 이는 깃딤의 배들이 이르러 그를 칠 것임이라 그가 낙심하고 돌아가며 거룩한 언약을 한하고 임의로 행하며 돌아가서는 거룩한 언약을 배반하는 자를 중히 여길 것이며
\par 31 군대는 그의 편에 서서 성소 곧 견고한 곳을 더럽히며 매일 드리는 제사를 폐하며 멸망케 하는 미운 물건을 세울 것이며
\par 32 그가 또 언약을 배반하고 악행하는 자를 궤휼로 타락시킬 것이나 오직 자기의 하나님을 아는 백성은 강하여 용맹을 발하리라
\par 33 백성 중에 지혜로운 자가 많은 사람을 가르칠 것이나 그들이 칼 날과 불꽃과 사로잡힘과 약탈을 당하여 여러 날 동안 쇠패하리라
\par 34 그들이 쇠패할 때에 도움을 조금 얻을 것이나 많은 사람은 궤휼로 그들과 친합할 것이며
\par 35 또 그들 중 지혜로운 자 몇 사람이 쇠패하여 무리로 연단되며 정결케 되며 희게 되어 마지막 때까지 이르게 하리니 이는 작정된 기한이 있음이니라
\par 36 이 왕이 자기 뜻대로 행하며 스스로 높여 모든 신보다 크다 하며 비상한 말로 신들의 신을 대적하며 형통하기를 분노하심이 쉴 때까지 하리니 이는 그 작정된 일이 반드시 이룰 것임이니라
\par 37 그가 모든 것보다 스스로 크다 하고 그 열조의 신들과 여자의 사모하는 것을 돌아보지 아니하며 아무 신이든지 돌아보지 아니할 것이나
\par 38 "그 대신에 세력의 신을 공경할 것이요 또 그 열조가 알지 못하던 신에게 금, 은 보석과 보물을 드려 공경할 것이며"
\par 39 그는 이방 신을 힘입어 크게 견고한 산성들을 취할 것이요 무릇 그를 안다 하는 자에게는 영광을 더하여 여러 백성을 다스리게도하며 그에게서 뇌물을 받고 땅을 나눠주기도 하리라
\par 40 마지막 때에 남방 왕이 그를 찌르리니 북방 왕이 병거와 마병과 많은 배로 회리바람처럼 그에게로 마주 와서 그 여러 나라에 들어가며 물이 넘침 같이 지나갈 것이요
\par 41 그가 또 영화로운 땅에 들어갈 것이요 많은 나라를 패망케 할 것이나 오직 에돔과 모압과 암몬 자손의 존귀한 자들은 그 손에서 벗어나리라
\par 42 그가 열국에 그 손을 펴리니 애굽 땅도 면치 못할 것이므로
\par 43 그가 권세로 애굽의 금 은과 모든 보물을 잡을 것이요 리비아 사람과 구스 사람이 그의 시종이 되리라
\par 44 그러나 동북에서부터 소문이 이르러 그로 번민케 하므로 그가 분노하여 나가서 많은 무리를 다 도륙하며 진멸코자 할 것이요
\par 45 그가 장막 궁전을 바다와 영화롭고 거룩한 산 사이에 베풀 것이나 그의 끝이 이르리니 도와줄 자가 없으리라

\chapter{12}

\par 1 그 때에 네 민족을 호위하는 대군 미가엘이 일어날 것이요 또 환난이 있으리니 이는 개국 이래로 그때까지 없던 환난일 것이며 그 때에 네 백성 중 무릇 책에 기록된 모든 자가 구원을 얻을 것이라
\par 2 땅의 티끌 가운데서 자는 자 중에 많이 깨어 영생을 얻는 자도 있겠고 수욕을 받아서 무궁히 부끄러움을 입을 자도 있을 것이며
\par 3 지혜 있는 자는 궁창의 빛과 같이 빛날 것이요 많은 사람을 옳은데로 돌아오게 한 자는 별과 같이 영원토록 비취리라
\par 4 다니엘아 마지막 때까지 이 말을 간수하고 이 글을 봉함하라 많은 사람이 빨리 왕래하며 지식이 더하리라
\par 5 나 다니엘이 본즉 다른 두 사람이 있어 하나는 강 이편 언덕에 섰고 하나는 강 저편 언덕에 섰더니
\par 6 그 중에 하나가 세마포 옷을 입은 자 곧 강물 위에 있는 자에게 이르되 이 기사의 끝이 어느 때까지냐 하기로
\par 7 내가 들은즉 그 세마포 옷을 입고 강물 위에 있는 자가 그 좌우 손을 들어 하늘을 향하여 영생하시는 자를 가리켜 맹세하여 가로되 반드시 한때 두때 반때를 지나서 성도의 권세가 다 깨어지기까지니 그렇게 되면 이 모든 일이 다 끝나리라 하더라
\par 8 내가 듣고도 깨닫지 못한지라 내가 가로되 내 주여 이 모든 일의 결국이 어떠하겠삽나이까
\par 9 그가 가로되 다니엘아 갈지어다 대저 이 말은 마지막 때까지 간수하고 봉함할 것임이니라
\par 10 많은 사람이 연단을 받아 스스로 정결케 하며 희게 할 것이나 악한 사람은 악을 행하리니 악한 자는 아무도 깨닫지 못하되 오직 지혜있는 자는 깨달으리라
\par 11 매일 드리는 제사를 폐하며 멸망케 할 미운 물건을 세울 때부터 일천 이백구십 일을 지낼 것이요
\par 12 기다려서 일천 삼백 삼십 오일까지 이르는 그 사람은 복이 있으리라
\par 13 너는 가서 마지막을 기다리라 이는 네가 평안히 쉬다가 끝날에는 네 업을 누릴 것임이니라


\end{document}