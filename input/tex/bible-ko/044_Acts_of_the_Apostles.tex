\begin{document}

\title{사도행전}


\chapter{1}

\par 1 데오빌로여 내가 먼저 쓴 글에는 무릇 예수의 행하시며 가르치시기를 시작하심부터
\par 2 그의 택하신 사도들에게 성령으로 명하시고 승천하신 날까지의 일을 기록하였노라
\par 3 해 받으신 후에 또한 저희에게 확실한 많은 증거로 친히 사심을 나타내사 사십 일 동안 저희에게 보이시며 하나님 나라의 일을 말씀하시니라
\par 4 사도와 같이 모이사 저희에게 분부하여 가라사대 예루살렘을 떠나지 말고 내게 들은바 아버지의 약속하신 것을 기다리라
\par 5 요한은 물로 세례를 베풀었으나 너희는 몇 날이 못되어 성령으로 세례를 받으리라 하셨느니라
\par 6 저희가 모였을 때에 예수께 묻자와 가로되 주께서 이스라엘 나라를 회복하심이 이 때니이까 하니
\par 7 가라사대 때와 기한은 아버지께서 자기의 권한에 두셨으니 너희의 알바 아니요
\par 8 오직 성령이 너희에게 임하시면 너희가 권능을 받고 예루살렘과 온 유대와 사마리아와 땅끝까지 이르러 내 증인이 되리라 하시니라
\par 9 이 말씀을 마치시고 저희 보는데서 올리워 가시니 구름이 저를 가리워 보이지 않게 하더라
\par 10 올라가실 때에 제자들이 자세히 하늘을 쳐다 보고 있는 데 흰옷 입은 두 사람이 저희 곁에 서서
\par 11 가로되 갈릴리 사람들아 어찌하여 서서 하늘을 쳐다 보느냐 너희 가운데서 하늘로 올리우신 이 예수는 하늘로 가심을 본 그대로 오시리라 하였느니라
\par 12 제자들이 감람원이라 하는 산으로부터 예루살렘에 돌아오니 이 산은 예루살렘에서 가까와 안식일에 가기 알맞은 길이라
\par 13 "들어가 저희 유하는 다락에 올라가니 베드로, 요한, 야고보, 안드레와 빌립, 도마와 바돌로매, 마태와 및 알패오의 아들 야고보, 셀롯인 시몬, 야고보의 아들 유다가 다 거기 있어"
\par 14 여자들과 예수의 모친 마리아와 예수의 아우들로 더불어 마음을 같이하여 전혀 기도에 힘쓰니라
\par 15 모인 무리의 수가 한 일백 이십 명이나 되더라 그 때에 베드로가 그 형제 가운데 일어서서 가로되
\par 16 형제들아 성령이 다윗의 입을 의탁하사 예수 잡는 자들을 지로한 유다를 가리켜 미리 말씀하신 성경이 응하였으니 마땅하도다
\par 17 이 사람이 본래 우리 수 가운데 참예하여 이 직무의 한 부분을 맡았던 자라
\par 18 (이 사람이 불의의 삯으로 밭을 사고 후에 몸이 곤두박질하여 배가 터져 창자가 다 흘러 나온지라
\par 19 이 일이 예루살렘에 사는 모든 사람에게 알게 되어 본방언에 그 밭을 이르되 아겔다마라 하니 이는 피밭이라는 뜻이라)
\par 20 시편에 기록하였으되 그의 거처로 황폐하게 하시며 거기 거하는 자가 없게 하소서 하였고 또 일렀으되 그 직분을 타인이 취하게 하소서 하였도다
\par 21 이러하므로 요한의 세례로부터 우리 가운데서 올리워 가신 날까지 주 예수께서 우리 가운데 출입하실 때에
\par 22 항상 우리와 함께 다니던 사람 중에 하나를 세워 우리로 더불어 예수의 부활하심을 증거할 사람이 되게 하여야 하리라 하거늘
\par 23 저희가 두 사람을 천하니 하나는 바사바라고도 하고 별명은 유스도라고 하는 요셉이요 하나는 맛디아라
\par 24 저희가 기도하여 가로되 뭇사람의 마음을 아시는 주여 이 두 사람 중에 누가 주의 택하신 바 되어
\par 25 봉사와 및 사도의 직무를 대신 할 자를 보이시옵소서 유다는 이를 버리옵고 제 곳으로 갔나이다 하고
\par 26 제비 뽑아 맛디아를 얻으니 저가 열 한 사도의 수에 가입하니라

\chapter{2}

\par 1 오순절날이 이미 이르매 저희가 다 같이 한 곳에 모였더니
\par 2 홀연히 하늘로부터 급하고 강한 바람 같은 소리가 있어 저희 앉은 온 집에 가득하며
\par 3 불의 혀같이 갈라지는 것이 저희에게 보여 각 사람 위에 임하여 있더니
\par 4 저희가 다 성령의 충만함을 받고 성령이 말하게 하심을 따라 다른 방언으로 말하기를 시작하니라
\par 5 그 때에 경건한 유대인이 천하 각국으로부터 와서 예루살렘에 우거하더니
\par 6 이 소리가 나매 큰 무리가 모여 각각 자기의 방언으로 제자들의 말하는 것을 듣고 소동하여
\par 7 다 놀라 기이히 여겨 이르되 보라 이 말하는 사람이 다 갈릴리 사람이 아니냐
\par 8 우리가 우리 각 사람의 난 곳 방언으로 듣게 되는 것이 어찜이뇨
\par 9 "우리는 바대인과 메대인과 엘람인과 또 메소보다미아 유대와 가바도기아, 본도와 아시아,"
\par 10 "브루기아와 밤빌리아, 애굽과 및 구레네에 가까운 리비야 여러 지방에 사는 사람들과 로마로부터 온 나그네 곧 유대인과 유대교에 들어 온 사람들과"
\par 11 그레데인과 아라비아인들이라 우리가 다 우리의 각 방언으로 하나님의 큰 일을 말함을 듣는도다 하고
\par 12 다 놀라며 의혹하여 서로 가로되 이 어찐 일이냐 하며
\par 13 또 어떤이들은 조롱하여 가로되 저희가 새 술이 취하였다 하더라
\par 14 베드로가 열 한 사도와 같이 서서 소리를 높여 가로되 유대인들과 예루살렘에 사는 모든 사람들아 이 일을 너희로 알게 할 것이니 내 말에 귀를 기울이라
\par 15 때가 제 삼시니 너희 생각과 같이 이 사람들이 취한 것이 아니라
\par 16 이는 곧 선지자 요엘로 말씀하신 것이니 일렀으되
\par 17 하나님이 가라사대 말세에 내가 내 영으로 모든 육체에게 부어 주리니 너희의 자녀들은 예언할 것이요 너희의 젊은이들은 환상을 보고 너희의 늙은이들은 꿈을 꾸리라
\par 18 그 때에 내가 내 영으로 내 남종과 여종들에게 부어 주리니 저희가 예언할 것이요
\par 19 또 내가 위로 하늘에서는 기사와 아래로 땅에서는 징조를 베풀리니 곧 피와 불과 연기로다
\par 20 주의 크고 영화로운 날이 이르기 전에 해가 변하여 어두워지고 달이 변하여 피가 되리라
\par 21 누구든지 주의 이름을 부르는 자는 구원을 얻으리라 하였느니라
\par 22 이스라엘 사람들아 이 말을 들으라 너희도 아는바에 하나님께서 나사렛 예수로 큰 권능과 기사와 표적을 너희 가운데서 베푸사 너희 앞에서 그를 증거하셨느니라
\par 23 그가 하나님의 정하신 뜻과 미리 아신대로 내어준바 되었거늘 너희가 법 없는 자들의 손을 빌어 못 박아 죽였으나
\par 24 하나님께서 사망의 고통을 풀어 살리셨으니 이는 그가 사망에게 매여 있을 수 없었음이라
\par 25 다윗이 저를 가리켜 가로되 내가 항상 내 앞에 계신 주를 뵈웠음이여 나로 요동치 않게 하기 위하여 그가 내 우편에 계시도다
\par 26 이러므로 내 마음이 기뻐하였고 내 입술도 즐거워하였으며 육체는 희망에 거하리니
\par 27 이는 내 영혼을 음부에 버리지 아니하시며 주의 거룩한 자로 썩음을 당치 않게 하실 것임이로다
\par 28 주께서 생명의 길로 내게 보이셨으니 주의 앞에서 나로 기쁨이 충만하게 하시리로다 하였으니
\par 29 형제들아 내가 조상 다윗에 대하여 담대히 말할 수 있노니 다윗이 죽어 장사되어 그 묘가 오늘까지 우리 중에 있도다
\par 30 그는 선지자라 하나님이 이미 맹세하사 그 자손 중에서 한 사람을 그 위에 앉게 하리라 하심을 알고
\par 31 미리 보는고로 그리스도의 부활하심을 말하되 저가 음부에 버림이 되지 않고 육신이 썩음을 당하지 아니하시리라 하더니
\par 32 이 예수를 하나님이 살리신지라 우리가 다 이 일에 증인이로다
\par 33 하나님이 오른손으로 예수를 높이시매 그가 약속하신 성령을 아버지께 받아서 너희 보고 듣는 이것을 부어 주셨느니라
\par 34 다윗은 하늘에 올라가지 못하였으나 친히 말하여 가로되 주께서 내 주에게 말씀하시기를
\par 35 내가 네 원수로 네 발등상 되게 하기까지는 너는 내 우편에 앉았으라 하셨도다 하였으니
\par 36 그런즉 이스라엘 온 집이 정녕 알지니 너희가 십자가에 못 박은 이 예수를 하나님이 주와 그리스도가 되게 하셨느니라 하니라
\par 37 저희가 이 말을 듣고 마음에 찔려 베드로와 다른 사도들에게 물어 가로되 형제들아 우리가 어찌할꼬 하거늘
\par 38 베드로가 가로되 너희가 회개하여 각각 예수 그리스도의 이름으로 세례를 받고 죄 사함을 얻으라 그리하면 성령을 선물로 받으리니
\par 39 이 약속은 너희와 너희 자녀와 모든 먼데 사람 곧 주 우리 하나님이 얼마든지 부르시는 자들에게 하신 것이라 하고
\par 40 또 여러 말로 확증하며 권하여 가로되 너희가 이 패역한 세대에서 구원을 받으라 하니
\par 41 그 말을 받는 사람들은 세례를 받으매 이 날에 제자의 수가 삼천이나 더하더라
\par 42 저희가 사도의 가르침을 받아 서로 교제하며 떡을 떼며 기도하기를 전혀 힘쓰니라
\par 43 사람마다 두려워하는데 사도들로 인하여 기사와 표적이 많이 나타나니
\par 44 믿는 사람이 다 함께 있어 모든 물건을 서로 통용하고
\par 45 또 재산과 소유를 팔아 각 사람의 필요를 따라 나눠 주고
\par 46 날마다 마음을 같이 하여 성전에 모이기를 힘쓰고 집에서 떡을 떼며 기쁨과 순전한 마음으로 음식을 먹고
\par 47 하나님을 찬미하며 또 온 백성에게 칭송을 받으니 주께서 구원 받는 사람을 날마다 더하게 하시니라

\chapter{3}

\par 1 제 구시 기도 시간에 베드로와 요한이 성전에 올라갈새
\par 2 나면서 앉은뱅이 된 자를 사람들이 메고 오니 이는 성전에 들어가는 사람들에게 구걸하기 위하여 날마다 미문이라는 성전 문에 두는 자라
\par 3 그가 베드로와 요한이 성전에 들어 가려함을 보고 구걸하거늘
\par 4 베드로가 요한으로 더불어 주목하여 가로되 우리를 보라 하니
\par 5 그가 저희에게 무엇을 얻을까 하여 바라보거늘
\par 6 베드로가 가로되 은과 금은 내게 없거니와 내게 있는 것으로 네게 주노니 곧 나사렛 예수 그리스도의 이름으로 걸으라 하고
\par 7 오른손을 잡아 일으키니 발과 발목이 곧 힘을 얻고
\par 8 뛰어 서서 걸으며 그들과 함께 성전으로 들어 가면서 걷기도 하고 뛰기도 하며 하나님을 찬미하니
\par 9 모든 백성이 그 걷는 것과 및 하나님을 찬미함을 보고
\par 10 그 본래 성전 미문에 앉아 구걸하던 사람인줄 알고 그의 당한 일을 인하여 심히 기이히 여기며 놀라니라
\par 11 나은 사람이 베드로와 요한을 붙잡으니 모든 백성이 크게 놀라며 달려 나아가 솔로몬의 행각이라 칭하는 행각에 모이거늘
\par 12 베드로가 이것을 보고 백성에게 말하되 이스라엘 사람들아 이 일을 왜 기이히 여기느냐 우리 개인의 권능과 경건으로 이 사람을 걷게 한 것처럼 왜 우리를 주목하느냐
\par 13 아브라함과 이삭과 야곱의 하나님 곧 우리 조상의 하나님이 그 종 예수를 영화롭게 하셨느니라 너희가 저를 넘겨주고 빌라도가 놓아 주기로 결안한 것을 너희가 그 앞에서 부인하였으니
\par 14 너희가 거룩하고 의로운 자를 부인하고 도리어 살인한 사람을 놓아 주기를 구하여
\par 15 생명의 주를 죽였도다 그러나 하나님이 죽은자 가운데서 살리셨으니 우리가 이 일에 증인이로라
\par 16 그 이름을 믿으므로 그 이름이 너희 보고 아는 이 사람을 성하게 하였나니 예수로 말미암아 난 믿음이 너희 모든 사람 앞에서 이같이 완전히 낫게 하였느니라
\par 17 형제들아 너희가 알지 못하여서 그리 하였으며 너희 관원들도 그리 한줄 아노라
\par 18 그러나 하나님이 모든 선지자의 입을 의탁하사 자기의 그리스도의 해 받으실 일을 미리 알게 하신 것을 이와 같이 이루셨느니라
\par 19 그러므로 너희가 회개하고 돌이켜 너희 죄 없이 함을 받으라 이같이 하면 유쾌하게 되는 날이 주 앞으로부터 이를 것이요
\par 20 또 주께서 너희를 위하여 예정하신 그리스도 곧 예수를 보내시리니
\par 21 하나님이 영원 전부터 거룩한 선지자의 입을 의탁하여 말씀하신바 만유를 회복하실 때까지는 하늘이 마땅히 그를 받아두리라
\par 22 모세가 말하되 주 하나님이 너희를 위하여 너희 형제 가운데서 나 같은 선지자 하나를 세울 것이니 너희가 무엇이든지 그 모든 말씀을 들을 것이라
\par 23 누구든지 그 선지자의 말을 듣지 아니하는 자는 백성 중에서 멸망 받으리라 하였고
\par 24 또한 사무엘 때부터 옴으로 말한 모든 선지자도 이 때를 가리켜 말하였느니라
\par 25 너희는 선지자들의 자손이요 또 하나님이 너희 조상으로 더불어 세우신 언약의 자손이라 아브라함에게 이르시기를 땅 위의 모든 족속이 너의 씨를 인하여 복을 받으리라 하셨으니
\par 26 하나님이 그 종을 세워 복 주시려고 너희에게 먼저 보내사 너희로 하여금 돌이켜 각각 그 악함을 버리게 하셨느니라

\chapter{4}

\par 1 사도들이 백성에게 말할 때에 제사장들과 성전 맡은 자와 사두개인들이 이르러
\par 2 백성을 가르침과 예수를 들어 죽은자 가운데서 부활하는 도 전함을 싫어하여
\par 3 저희를 잡으매 날이 이미 저문고로 이튿날까지 가두었으나
\par 4 말씀을 들은 사람 중에 믿는 자가 많으니 남자의 수가 약 오천이나 되었더라
\par 5 이튿날에 관원과 장로와 서기관들이 예루살렘에 모였는데
\par 6 대제사장 안나스와 가야바와 요한과 알렉산더와 및 대제사장의 문중이 다 참예하여
\par 7 사도들을 가운데 세우고 묻되 너희가 무슨 권세와 뉘 이름으로 이 일을 행하였느냐
\par 8 이에 베드로가 성령이 충만하여 가로되 백성의 관원과 장로들아
\par 9 만일 병인에게 행한 착한 일에 대하여 이 사람이 어떻게 구원을 얻었느냐고 오늘 우리에게 질문하면
\par 10 너희와 모든 이스라엘 백성들은 알라 너희가 십자가에 못 박고 하나님이 죽은자 가운데서 살리신 나사렛예수 그리스도의 이름으로 이 사람이 건강하게 되어 너희 앞에 섰느니라
\par 11 이 예수는 너희 건축자들의 버린 돌로서 집 모퉁이의 머릿돌이 되었느니라
\par 12 다른이로서는 구원을 얻을 수 없나니 천하 인간에 구원을 얻을만한 다른 이름을 우리에게 주신 일이 없음이니라 하였더라
\par 13 저희가 베드로와 요한이 기탄없이 말함을 보고 그 본래 학문 없는 범인으로 알았다가 이상히 여기며 또 그 전에 예수와 함께 있던 줄도 알고
\par 14 또 병 나은 사람이 그들과 함께 섰는 것을 보고 힐난할 말이 없는지라
\par 15 명하여 공회에서 나가라 하고 서로 의논하여 가로되
\par 16 이 사람들을 어떻게 할꼬 저희로 인하여 유명한 표적 나타난 것이 예루살렘에 사는 모든 사람에게 알려졌으니 우리도 부인할 수 없는지라
\par 17 이것이 민간에 더 퍼지지 못하게 저희를 위협하여 이 후에는 이 이름으로 아무 사람에게도 말하지 말게 하자 하고
\par 18 그들을 불러 경계하여 도무지 예수의 이름으로 말하지도 말고 가르치지도 말라 하니
\par 19 베드로와 요한이 대답하여 가로되 하나님 앞에서 너희 말 듣는 것이 하나님 말씀 듣는 것보다 옳은가 판단하라
\par 20 우리는 보고 들은 것을 말하지 아니할 수 없다 하니
\par 21 관원들이 백성을 인하여 저희를 어떻게 벌할 도리를 찾지 못하고 다시 위협하여 놓아 주었으니 이는 모든 사람이 그 된 일을 보고 하나님께 영광을 돌림이러라
\par 22 이 표적으로 병 나은 사람은 사십 여 세나 되었더라
\par 23 사도들이 놓이매 그 동류에게 가서 제사장들과 장로들의 말을 다 고하니
\par 24 저희가 듣고 일심으로 하나님께 소리를 높여 가로되 대주재여 천지와 바다와 그 가운데 만유를 지은 이시요
\par 25 또 주의 종 우리 조상 다윗의 입을 의탁하사 성령으로 말씀하시기를 어찌하여 열방이 분노하며 족속들이 허사를 경영하였는고
\par 26 세상의 군왕들이 나서며 관원들이 함께 모여 주와 그 그리스도를 대적하도다 하신 이로소이다
\par 27 과연 헤롯과 본디오 빌라도는 이방인과 이스라엘 백성과 합동하여 하나님의 기름부으신 거룩한 종 예수를 거스려
\par 28 하나님의 권능과 뜻대로 이루려고 예정하신 그것을 행하려고 이 성에 모였나이다
\par 29 주여 이제도 저희의 위협함을 하감하옵시고 또 종들로 하여금 담대히 하나님의 말씀을 전하게 하여 주옵시며
\par 30 손을 내밀어 병을 낫게 하옵시고 표적과 기사가 거룩한 종 예수의 이름으로 이루어지게 하옵소서 하더라
\par 31 빌기를 다하매 모인 곳이 진동하더니 무리가 다 성령이 충만하여 담대히 하나님의 말씀을 전하니라
\par 32 믿는 무리가 한 마음과 한 뜻이 되어 모든 물건을 서로 통용하고 제 재물을 조금이라도 제 것이라 하는 이가 하나도 없더라
\par 33 사도들이 큰 권능으로 주 예수의 부활을 증거하니 무리가 큰 은혜를 얻어
\par 34 그 중에 핍절한 사람이 없으니 이는 밭과 집 있는 자는 팔아 그 판 것의 값을 가져다가
\par 35 사도들의 발 앞에 두매 저희가 각 사람의 필요를 따라 나눠줌이러라
\par 36 구브로에서 난 레위족인이 있으니 이름은 요셉이라 사도들이 일컬어 바나바 (번역하면 권위자)라 하니
\par 37 그가 밭이 있으매 팔아 값을 가지고 사도들의 발 앞에 두니라

\chapter{5}

\par 1 아나니아라 하는 사람이 그 아내 삽비라로 더불어 소유를 팔아
\par 2 그 값에서 얼마를 감추매 그 아내도 알더라 얼마를 가져다가 사도들의 발 앞에 두니
\par 3 베드로가 가로되 아나니아야 어찌하여 사단이 네 마음에 가득하여 네가 성령을 속이고 땅 값 얼마를 감추었느냐
\par 4 땅이 그대로 있을 때에는 네 땅이 아니며 판 후에도 네 임의로 할 수가 없더냐 어찌하여 이 일을 네 마음에 두었느냐 사람에게 거짓말 한것이 아니요 하나님께로다
\par 5 아나니아가 이 말을 듣고 엎드러져 혼이 떠나니 이 일을 듣는 사람이 다 크게 두려워하더라
\par 6 젊은 사람들이 일어나 시신을 싸서 메고 나가 장사하니라
\par 7 세 시간쯤 지나 그 아내가 그 생긴 일을 알지 못하고 들어 오니
\par 8 베드로가 가로되 그 땅 판 값이 이것 뿐이냐 내게 말하라 하니 가로되 예 이뿐이로라
\par 9 베드로가 가로되 너희가 어찌 함께 꾀하여 주의 영을 시험하려 하느냐 보라 네 남편을 장사하고 오는 사람들의 발이 문앞에 이르렀으니 또 너를 메어 내가리라 한대
\par 10 곧 베드로의 발 앞에 엎드러져 혼이 떠나는지라 젊은 사람들이 들어와 죽은 것을 보고 메어다가 그 남편곁에 장사하니
\par 11 온 교회와 이 일을 듣는 사람들이 다 크게 두려워하니라
\par 12 사도들의 손으로 민간에 표적과 기사가 많이 되매 믿는 사람이 다 마음을 같이하여 솔로몬 행각에 모이고
\par 13 그 나머지는 감히 그들과 상종하는 사람이 없으나 백성이 칭송하더라
\par 14 믿고 주께로 나오는 자가 더 많으니 남녀의 큰 무리더라
\par 15 심지어 병든 사람을 메고 거리에 나가 침대와 요 위에 뉘우고 베드로가 지날 때에 혹 그 그림자라도 뉘게 덮일까 바라고
\par 16 예루살렘 근읍 허다한 사람들도 모여 병든 사람과 더러운 귀신에게 괴로움 받는 사람을 데리고 와서 다 나음을 얻으니라
\par 17 대제사장과 그와 함께 있는 사람 즉 사두개인의 당파가 다 마음에 시기가 가득하여 일어나서
\par 18 사도들을 잡아다가 옥에 가두었더니
\par 19 주의 사자가 밤에 옥문을 열고 끌어내어 가로되
\par 20 가서 성전에 서서 이 생명의 말씀을 다 백성에게 말하라 하매
\par 21 저희가 듣고 새벽에 성전에 들어가서 가르치더니 대제사장과 그와 함께 있는 사람들이 와서 공회와 이스라엘 족속의 원로들을 다 모으고 사람을 옥에 보내어 사도들을 잡아오라 하니
\par 22 관속들이 가서 옥에서 사도들을 보지 못하고 돌아와 말하여
\par 23 가로되 우리가 보니 옥은 든든하게 잠기고 지킨 사람들이 문에 섰으되 문을 열고 본즉 그 안에는 한 사람도 없더이다 하니
\par 24 성전 맡은 자와 제사장들이 이 말을 듣고 의혹하여 이 일이 어찌 될까 하더니
\par 25 사람이 와서 고하되 보소서 옥에 가두었던 사람들이 성전에 서서 백성을 가르치더이다 하니
\par 26 성전 맡은 자가 관속들과 같이 가서 저희를 잡아 왔으나 강제로 못함은 백성들이 돌로 칠까 두려워함이러라
\par 27 저희를 끌어다가 공회 앞에 세우니 대제사장이 물어
\par 28 가로되 우리가 이 이름으로 사람을 가르치지 말라고 엄금하였으되 너희가 너희 교를 예루살렘에 가득하게 하니 이 사람의 피를 우리에게로 돌리고자 함이로다
\par 29 베드로와 사도들이 대답하여 가로되 사람보다 하나님을 순종하는것이 마땅하니라
\par 30 너희가 나무에 달아 죽인 예수를 우리 조상의 하나님이 살리시고
\par 31 이스라엘로 회개케 하사 죄 사함을 얻게 하시려고 그를 오른손으로 높이사 임금과 구주를 삼으셨느니라
\par 32 우리는 이 일에 증인이요 하나님이 자기를 순종하는 사람들에게 주신 성령도 그러하니라 하더라
\par 33 저희가 듣고 크게 노하여 사도들을 없이하고자 할새
\par 34 바리새인 가말리엘은 교법사로 모든 백성에게 존경을 받는 자라 공회 중에 일어나 명하사 사도들을 잠간 밖에 나가게 하고
\par 35 말하되 이스라엘 사람들아 너희가 이 사람들에게 대하여 어떻게 하려는 것을 조심하라
\par 36 이전에 드다가 일어나 스스로 자랑하매 사람이 약 사백이나 따르더니 그가 죽임을 당하매 좇던 사람이 다 흩어져 없어졌고
\par 37 그 후 호적할 때에 갈릴리 유다가 일어나 백성을 꾀어 좇게 하다가 그도 망한즉 좇던 사람이 다 흩어졌느니라
\par 38 이제 내가 너희에게 말하노니 이 사람들을 상관 말고 버려두라 이 사상과 소행이 사람에게로서 났으면 무너질 것이요
\par 39 만일 하나님께로서 났으면 너희가 저희를 무너뜨릴 수 없겠고 도리어 하나님을 대적하는 자가 될까 하노라 하니
\par 40 저희가 옳게 여겨 사도들을 불러들여 채찍질하며 예수의 이름으로 말하는 것을 금하고 놓으니
\par 41 사도들은 그 이름을 위하여 능욕 받는 일에 합당한 자로 여기심을 기뻐하면서 공회 앞을 떠나니라
\par 42 저희가 날마다 성전에 있든지 집에 있든지 예수는 그리스도라 가르치기와 전도하기를 쉬지 아니하니라

\chapter{6}

\par 1 그 때에 제자가 더 많아졌는데 헬라파 유대인들이 자기의 과부들이 그 매일 구제에 빠지므로 히브리파 사람을 원망한대
\par 2 열 두 사도가 모든 제자를 불러 이르되 우리가 하나님의 말씀을 제쳐 놓고 공궤를 일삼는 것이 마땅치 아니하니
\par 3 형제들아 너희 가운데서 성령과 지혜가 충만하여 칭찬 듣는 사람 일곱을 택하라 우리가 이 일을 저희에게 맡기고
\par 4 우리는 기도하는 것과 말씀 전하는 것을 전무하리라 하니
\par 5 온 무리가 이 말을 기뻐하여 믿음과 성령이 충만한 사람 스데반과 또 빌립과 브로고로와 니가노르와 디몬과 바메나와 유대교에 입교한 안디옥 사람 니골라를 택하여
\par 6 사도들 앞에 세우니 사도들이 기도하고 그들에게 안수하니라
\par 7 하나님의 말씀이 점점 왕성하여 예루살렘에 있는 제자의 수가 더 심히 많아지고 허다한 제사장의 무리도 이 도에 복종하니라
\par 8 스데반이 은혜와 권능이 충만하여 큰 기사와 표적을 민간에 행하니
\par 9 리버디노 구레네인 알렉산드리아인 길리기아와 아시아에서 온 사람들의 회당이라는 각 회당에서 어떤 자들이 일어나 스데반으로 더불어 변론할새
\par 10 스데반이 지혜와 성령으로 말함을 저희가 능히 당치 못하여
\par 11 사람들을 가르쳐 말시키되 이 사람이 모세와 및 하나님을 모독하는 말 하는 것을 우리가 들었노라 하게 하고
\par 12 백성과 장로와 서기관들을 충동시켜 와서 잡아 가지고 공회에 이르러
\par 13 거짓 증인들을 세우니 가로되 이 사람이 이 거룩한 곳과 율법을 거스려 말하기를 마지 아니하는도다
\par 14 그의 말에 이 나사렛 예수가 이곳을 헐고 또 모세가 우리에게 전하여 준 규례를 고치겠다 함을 우리가 들었노라 하거늘
\par 15 공회 중에 앉은 사람들이 다 스데반을 주목하여 보니 그 얼굴이 천사의 얼굴과 같더라

\chapter{7}

\par 1 대제사장이 가로되 이것이 사실이냐
\par 2 스데반이 가로되 여러분 부형들이여 들으소서 우리 조상 아브라함이 하란에 있기 전 메소보다미아에 있을 때에 영광의 하나님이 그에게 보여
\par 3 가라사대 네 고향과 친척을 떠나 내가 네게 보일 땅으로 가라 하시니
\par 4 아브라함이 갈대아 사람의 땅을 떠나 하란에 거하다가 그 아비가 죽으매 하나님이 그를 거기서 너희 시방 거하는 이 땅으로 옮기셨느니라
\par 5 그러나 여기서 발 붙일만큼도 유업을 주지 아니하시고 다만 이 땅을 아직 자식도 없는 저와 저의 씨에게 소유로 주신다고 약속하셨으며
\par 6 하나님이 또 이같이 말씀하시되 그 씨가 다른 땅에 나그네 되리니 그 땅 사람이 종을 삼아 사백 년 동안을 괴롭게 하리라 하시고
\par 7 또 가라사대 종 삼는 나라를 내가 심판하리니 그 후에 저희가 나와서 이곳에서 나를 섬기리라 하시고
\par 8 "할례의 언약을 아브라함에게 주셨더니 그가 이삭을 낳아 여드레만에 할례를 행하고 이삭이 야곱을, 야곱이 우리 열 두 조상을 낳으니"
\par 9 여러 조상이 요셉을 시기하여 애굽에 팔았더니 하나님이 저와 함께 계셔
\par 10 그 모든 환난에서 건져내사 애굽 왕 바로 앞에서 은총과 지혜를 주시매 바로가 저를 애굽과 자기 온 집의 치리자로 세웠느니라
\par 11 그 때에 애굽과 가나안 온 땅에 흉년들어 큰 환난이 있을새 우리 조상들이 양식이 없는지라
\par 12 야곱이 애굽에 곡식 있다는 말을 듣고 먼저 우리 조상들을 보내고
\par 13 또 재차 보내매 요셉이 자기 형제들에게 알게 되고 또 요셉의 친족이 바로에게 드러나게 되니라
\par 14 요셉이 보내어 그 부친 야곱과 온 친족 일흔 다섯 사람을 청하였더니
\par 15 야곱이 애굽으로 내려가 자기와 우리 조상들이 거기서 죽고
\par 16 세겜으로 옮기워 아브라함이 세겜 하몰의 자손에게서 은으로 값주고 산 무덤에 장사되니라
\par 17 하나님이 아브라함에게 약속하신 때가 가까우매 이스라엘 백성이 애굽에서 번성하여 많아졌더니
\par 18 요셉을 알지 못하는 새 임금이 애굽 왕위에 오르매
\par 19 그가 우리 족속에게 궤계를 써서 조상들을 괴롭게 하여 그 어린 아이들을 내어버려 살지 못하게 하려 할새
\par 20 그 때에 모세가 났는데 하나님 보시기에 아름다운지라 그 부친의 집에서 석 달을 길리우더니
\par 21 버리운 후에 바로의 딸이 가져다가 자기 아들로 기르매
\par 22 모세가 애굽 사람의 학술을 다 배워 그 말과 행사가 능하더라
\par 23 나이 사십이 되매 그 형제 이스라엘 자손을 돌아볼 생각이 나더니
\par 24 한 사람의 원통한 일 당함을 보고 보호하여 압제 받는 자를 위하여 원수를 갚아 애굽 사람을 쳐 죽이니라
\par 25 저는 그 형제들이 하나님께서 자기의 손을 빌어 구원하여 주시는 것을 깨달으리라고 생각하였으나 저희가 깨닫지 못하였더라
\par 26 이튿날 이스라엘 사람이 싸울 때에 모세가 와서 화목시키려 하여 가로되 너희는 형제라 어찌 서로 해하느냐 하니
\par 27 그 동무를 해하는 사람이 모세를 밀뜨려 가로되 누가 너를 관원과 재판장으로 우리 위에 세웠느냐
\par 28 네가 어제 애굽 사람을 죽임과 같이 또 나를 죽이려느냐 하니
\par 29 모세가 이 말을 인하여 도주하여 미디안 땅에서 나그네 되어 거기서 아들 둘을 낳으니라
\par 30 사십 년이 차매 천사가 시내산 광야 가시나무떨기 불꽃 가운데서 그에게 보이거늘
\par 31 모세가 이 광경을 보고 기이히 여겨 알아보려고 가까이 가니 주의 소리 있어
\par 32 나는 네 조상의 하나님 즉 아브라함과 이삭과 야곱의 하나님이로라 하신대 모세가 무서워 감히 알아보지 못하더라
\par 33 주께서 가라사대 네 발의 신을 벗으라 너 섰는 곳은 거룩한 땅이니라
\par 34 내 백성이 애굽에서 괴로움 받음을 내가 정녕히 보고 그 탄식하는 소리를 듣고 저희를 구원하려고 내려왔노니 시방 내가 너를 애굽으로 보내리라 하시니라
\par 35 저희 말이 누가 너를 관원과 재판장으로 세웠느냐 하며 거절하던 그 모세를 하나님은 가시나무떨기 가운데서 보이던 천사의 손을 의탁하여 관원과 속량하는 자로 보내셨으니
\par 36 이 사람이 백성을 인도하여 나오게 하고 애굽과 홍해와 광야에서 사십 년간 기사와 표적을 행하였느니라
\par 37 이스라엘 자손을 대하여 하나님이 너희 형제 가운데서 나와 같은 선지자를 세우리라 하던 자가 곧 이 모세라
\par 38 시내산에서 말하던 그 천사와 및 우리 조상들과 함께 광야 교회에 있었고 또 생명의 도를 받아 우리에게 주던 자가 이 사람이라
\par 39 우리 조상들이 모세에게 복종치 아니하고자하여 거절하며 그 마음이 도리어 애굽으로 행하여
\par 40 아론더러 이르되 우리를 인도할 신들을 우리를 위하여 만들라 애굽 땅에서 우리를 인도하던 이 모세는 어떻게 되었는지 알지 못하노라 하고
\par 41 그 때에 저희가 송아지를 만들어 그 우상 앞에 제사하며 자기 손으로 만든 것을 기뻐하더니
\par 42 하나님이 돌이키사 저희를 그 하늘의 군대 섬기는 일에 버려두셨으니 이는 선지자의 책에 기록된바 이스라엘의 집이여 사십년을 광야에서 너희가 희생과 제물을 내게 드린 일이 있었느냐
\par 43 몰록의 장막과 신 레판의 별을 받들었음이여 이것은 너희가 절하고자 하여 만든 형상이로다 내가 너희를 바벨론 밖에 옮기리라 함과 같으니라
\par 44 광야에서 우리 조상들에게 증거의 장막이 있었으니 이것은 모세에게 말씀하신 이가 명하사 저가 본 그 식대로 만들게 하신 것이라
\par 45 우리 조상들이 그것을 받아 하나님이 저희 앞에서 쫓아내신 이방인의 땅을 점령할 때에 여호수아와 함께 가지고 들어가서 다윗 때까지 이르니라
\par 46 다윗이 하나님 앞에서 은혜를 받아 야곱의 집을 위하여 하나님의 처소를 준비케 하여 달라 하더니
\par 47 솔로몬이 그를 위하여 집을 지었느니라
\par 48 그러나 지극히 높으신 이는 손으로 지은 곳에 계시지 아니하시나니 선지자의 말한바
\par 49 주께서 가라사대 하늘은 나의 보좌요 땅은 나의 발등상이니 너희가 나를 위하여 무슨 집을 짓겠으며 나의 안식할 처소가 어디뇨
\par 50 이 모든 것이 다 내 손으로 지은 것이 아니냐 함과 같으니라
\par 51 목이 곧고 마음과 귀에 할례를 받지 못한 사람들아 너희가 항상 성령을 거스려 너희 조상과 같이 너희도 하는도다
\par 52 너희 조상들은 선지자 중에 누구를 핍박지 아니하였느냐 의인이 오시리라 예고한 자들을 저희가 죽였고 이제 너희는 그 의인을 잡아준 자요 살인한 자가 되나니
\par 53 너희가 천사의 전한 율법을 받고도 지키지 아니하였도다 하니라
\par 54 저희가 이말을 듣고 마음에 찔려 저를 향하여 이를 갈거늘
\par 55 스데반이 성령이 충만하여 하늘을 우러러 주목하여 하나님의 영광과 및 예수께서 하나님 우편에 서신 것을 보고
\par 56 말하되 보라 하늘이 열리고 인자가 하나님 우편에 서신 것을 보노라 한대
\par 57 저희가 큰 소리를 지르며 귀를 막고 일심으로 그에게 달려들어
\par 58 성 밖에 내치고 돌로 칠새 증인들이 옷을 벗어 사울이라 하는 청년의 발앞에 두니라
\par 59 저희가 돌로 스데반을 치니 스데반이 부르짖어 가로되 주 예수여 내 영혼을 받으시옵소서 하고
\par 60 무릎을 꿇고 크게 불러 가로되 주여 이 죄를 저들에게 돌리지 마옵소서 이 말을 하고 자니라

\chapter{8}

\par 1 사울이 그의 죽임 당함을 마땅히 여기더라 그 날에 예루살렘에 있는 교회에 큰 핍박이 나서 사도 외에는 다 유대와 사마리아 모든 땅으로 흩어지니라
\par 2 경건한 사람들이 스데반을 장사하고 위하여 크게 울더라
\par 3 사울이 교회를 잔멸할새 각집에 들어가 남녀를 끌어다가 옥에 넘기니라
\par 4 그 흩어진 사람들이 두루 다니며 복음의 말씀을 전할새
\par 5 빌립이 사마리아 성에 내려가 그리스도를 백성에게 전파하니
\par 6 무리가 빌립의 말도 듣고 행하는 표적도 보고 일심으로 그의 말하는 것을 좇더라
\par 7 많은 사람에게 붙었던 더러운 귀신들이 크게 소리를 지르며 나가고 또 많은 중풍병자와 앉은뱅이가 나으니
\par 8 그 성에 큰 기쁨이 있더라
\par 9 그 성에 시몬이라 하는 사람이 전부터 있어 마술을 행하여 사마리아 백성을 놀라게 하며 자칭 큰 자라 하니
\par 10 낮은 사람부터 높은 사람까지 다 청종하여 가로되 이 사람은 크다 일컫는 하나님의 능력이라 하더라
\par 11 오래 동안 그 마술에 놀랐으므로 저희가 청종하더니
\par 12 빌립이 하나님 나라와 및 예수 그리스도의 이름에 관하여 전도함을 저희가 믿고 남녀가 다 세례를 받으니
\par 13 시몬도 믿고 세례를 받은 후에 전심으로 빌립을 따라 다니며 그 나타나는 표적과 큰 능력을 보고 놀라니라
\par 14 예루살렘에 있는 사도들이 사마리아도 하나님의 말씀을 받았다 함을 듣고 베드로와 요한을 보내매
\par 15 그들이 내려가서 저희를 위하여 성령 받기를 기도하니
\par 16 이는 아직 한 사람에게도 성령 내리신 일이 없고 오직 주 예수의 이름으로 세례만 받을 뿐이러라
\par 17 이에 두 사도가 저희에게 안수하매 성령을 받는지라
\par 18 시몬이 사도들의 안수함으로 성령 받는 것을 보고 돈을 드려
\par 19 가로되 이 권능을 내게도 주어 누구든지 내가 안수하는 사람은 성령을 받게 하여 주소서 하니
\par 20 베드로가 가로되 네가 하나님의 선물을 돈 주고 살 줄로 생각하였으니 네 은과 네가 함께 망할지어다
\par 21 하나님 앞에서 네 마음이 바르지 못하니 이 도에는 네가 관계도 없고 분깃 될것도 없느니라
\par 22 그러므로 너의 이 악함을 회개하고 주께 기도하라 혹 마음에 품은 것을 사하여 주시리라
\par 23 내가 보니 너는 악독이 가득하며 불의에 매인바 되었도다
\par 24 시몬이 대답하여 가로되 나를 위하여 주께 기도하여 말한 것이 하나도 내게 임하지 말게 하소서 하니라
\par 25 두 사도가 주의 말씀을 증거하여 말한 후 예루살렘으로 돌아갈새 사마리아인의 여러 촌에서 복음을 전하니라
\par 26 주의 사자가 빌립더러 일러 가로되 일어나서 남으로 향하여 예루살렘에서 가사로 내려가는 길까지 가라 하니 그 길은 광야라
\par 27 일어나 가서 보니 에디오피아 사람 곧 에디오피아 여왕 간다게의 모든 국고를 맡은 큰 권세가 있는 내시가 예배하러 예루살렘에 왔다가
\par 28 돌아 가는데 병거를 타고 선지자 아사야의 글을 읽더라
\par 29 성령이 빌립더러 이르시되 이 병거로 가까이 나아가라 하시거늘
\par 30 빌립이 달려가서 선지자 이사야의 글 읽는 것을 듣고 말하되 읽는 것을 깨닫느뇨
\par 31 대답하되 지도하는 사람이 없으니 어찌 깨달을 수 있느뇨 하고 빌립을 청하여 병거에 올라 같이 앉으라 하니라
\par 32 읽는 성경 귀절은 이것이니 일렀으되 저가 사지로 가는 양과 같이 끌리었고 털 깎는 자 앞에 있는 어린 양의 잠잠함과 같이 그 입을 열지 아니하였도다
\par 33 낮을 때에 공변된 판단을 받지 못하였으니 누가 가히 그 세대를 말하리요 그 생명이 땅에서 빼앗김이로다 하였거늘
\par 34 내시가 빌립더러 말하되 청컨대 묻노니 선지자가 이말 한 것이 누구를 가리킴이뇨 자기를 가리킴이뇨 타인을 가리킴이뇨
\par 35 빌립이 입을 열어 이 글에서 시작하여 예수를 가르쳐 복음을 전하니
\par 36 길 가다가 물 있는 곳에 이르러 내시가 말하되 보라 물이 있으니 내가 세례를 받음에 무슨 거리낌이 있느뇨
\par 37 (없 음�
\par 38 이에 명하여 병거를 머물고 빌립과 내시가 둘 다 물에 내려가 빌립이 세례를 주고
\par 39 둘이 물에서 올라갈새 주의 영이 빌립을 이끌어 간지라 내시는 혼연히 길을 가므로 그를 다시 보지 못하니라
\par 40 빌립은 아소도에 나타나 여러 성을 지나 다니며 복음을 전하고 가이사랴에 이르니라

\chapter{9}

\par 1 사울이 주의 제자들을 대하여 여전히 위협과 살기가 등등하여 대제사장에게 가서
\par 2 다메섹 여러 회당에 갈 공문을 청하니 이는 만일 그 도를 좇는 사람을 만나면 무론남녀하고 결박하여 예루살렘으로 잡아 오려 함이라
\par 3 사울이 행하여 다메섹에 가까이 가더니 홀연히 하늘로서 빛이 저를 둘러 비추는지라
\par 4 땅에 엎드러져 들으매 소리 있어 가라사대 사울아 사울아 네가 어찌하여 나를 핍박하느냐 하시거늘
\par 5 대답하되 주여 뉘시오니이까 가라사대 나는 네가 핍박하는 예수라
\par 6 네가 일어나 성으로 들어가라 행할 것을 네게 이를 자가 있느니라 하시니
\par 7 같이 가던 사람들은 소리만 듣고 아무도 보지 못하여 말을 못하고 섰더라
\par 8 사울이 땅에서 일어나 눈은 떴으나 아무 것도 보지 못하고 사람의 손에 끌려 다메섹으로 들어가서
\par 9 사흘 동안을 보지 못하고 식음을 전폐하니라
\par 10 그 때에 다메섹에 아나니아라 하는 제자가 있더니 주께서 환상 중에 불러 가라사대 아나니아야 하시거늘 대답하되 주여 내가 여기 있나이다 하니
\par 11 주께서 가라사대 일어나 직가라 하는 거리로 가서 유다 집에서 다소 사람 사울이라 하는 자를 찾으라 저가 기도하는 중이다
\par 12 저가 아나니아라 하는 사람이 들어와서 자기에게 안수하여 다시 보게 하는 것을 보았느니라 하시거늘
\par 13 아나니아가 대답하되 주여 이 사람에 대하여 내가 여러 사람에게 듣사온즉 그가 예루살렘에서 주의 성도에게 적지 않은 해를 끼쳤다 하더니
\par 14 여기서도 주의 이름을 부르는 모든 자를 결박할 권세를 대제사장들에게 받았나이다 하거늘
\par 15 주께서 가라사대 가라 이 사람은 내 이름을 이방인과 임금들과 이스라엘 자손들 앞에 전하기 위하여 택한 나의 그릇이라
\par 16 그가 내 이름을 위하여 해를 얼마나 받아야 할 것을 내가 그에게 보이리라 하시니
\par 17 아나니아가 떠나 그 집에 들어가서 그에게 안수하여 가로되 형제 사울아 주 곧 네가 오는 길에서 나타나시던 예수께서 나를 보내어 너로 다시 보게 하시고 성령으로 충만하게 하신다 하니
\par 18 즉시 사울의 눈에서 비늘 같은 것이 벗어져 다시 보게 된지라 일어나 세례를 받고
\par 19 음식을 먹으매 강건하여지니라 사울이 다메섹에 있는 제자들과 함께 며칠 있을새
\par 20 즉시로 각 회당에서 예수의 하나님의 아들이심을 전파하니
\par 21 듣는 사람이 다 놀라 말하되 이사람이 예루살렘에서 이 이름 부르는 사람을 잔해 하던 자가 아니냐 여기 온 것도 저희를 결박하여 대제사장들에게 끌어 가고자 함이 아니냐 하더라
\par 22 사울은 힘을 더 얻어 예수를 그리스도라 증명하여 다메섹에 사는 유대인들을 굴복시키니라
\par 23 여러 날이 지나매 유대인들이 사울 죽이기를 공모하더니
\par 24 그 계교가 사울에게 알려지니라 저희가 그를 죽이려고 밤낮으로 성문까지 지키거늘
\par 25 그의 제자들이 밤에 광주리에 사울을 담아 성에서 달아 내리니라
\par 26 사울이 예루살렘에 가서 제자들을 사귀고자 하나 다 두려워하여 그의 제자 됨을 믿지 아니하니
\par 27 바나바가 데리고 사도들에게 가서 그가 길에서 어떻게 주를 본 것과 주께서 그에게 말씀하신 일과 다메섹에서 그가 어떻게 예수의 이름으로 담대히 말하던 것을 말하니라
\par 28 사울이 제자들과 함께 있어 예루살렘에 출입하며
\par 29 또 주 예수의 이름으로 담대히 말하고 헬라파 유대인들과 함께 말하며 변론하니 그 사람들이 죽이려고 힘쓰거늘
\par 30 형제들이 알고 가이사랴로 데리고 내려가서 다소로 보내니라
\par 31 그리하여 온 유대와 갈릴리와 사마리아 교회가 평안하여 든든히 서 가고 주를 경외함과 성령의 위로로 진행하여 수가 더 많아지니라
\par 32 때에 베드로가 사방으로 두루 행하다가 룻다에 사는 성도들에게도 내려갔더니
\par 33 거기서 애니아라 하는 사람을 만나매 그가 중풍병으로 상 위에 누운지 팔년이라
\par 34 베드로가 가로되 애니아야 예수 그리스도께서 너를 낫게 하시니 일어나 네 자리를 정돈하라 한대 곧 일어나니
\par 35 룻다와 사론에 사는 사람들이 다 그를 보고 주께로 돌아가니라
\par 36 욥바에 다비다라 하는 여제자가 있으니 그 이름을 번역하면 도르가라 선행과 구제하는 일이 심히 많더니
\par 37 그 때에 병들어 죽으매 시체를 씻어 다락에 뉘우니라
\par 38 룻다가 욥바에 가까운지라 제자들이 베드로가 거기 있음을 듣고 두 사람을 보내어 지체 말고 오라고 간청하니
\par 39 베드로가 일어나 저희와 함께 가서 이르매 저희가 데리고 다락에 올라가니 모든 과부가 베드로의 곁에 서서 울며 도르가가 저희와 함께 있을 때에 지은 속옷과 겉옷을 다 내어 보이거늘
\par 40 베드로가 사람을 다 내어보내고 무릎을 꿇고 기도하고 돌이켜 시체를 향하여 가로되 다비다야 일어나라 하니 그가 눈을 떠 베드로를 보고 일어나 앉는지라
\par 41 베드로가 손을 내밀어 일으키고 성도들과 과부들을 불러들여 그의 산 것을 보이니
\par 42 온 욥바 사람이 알고 많이 주를 믿더라
\par 43 베드로가 욥바에 여러 날 있어 시몬이라 하는 피장의 집에서 유하니라

\chapter{10}

\par 1 가이사랴에 고넬료라 하는 사람이 있으니 이달리야대라 하는 군대의 백부장이라
\par 2 그가 경건하여 온 집으로 더불어 하나님을 경외하며 백성을 많이 구제하고 하나님께 항상 기도하더니
\par 3 하루는 제 구시쯤 되어 환상 중에 밝히 보매 하나님의 사자가 들어와 가로되 고넬료야 하니
\par 4 고넬료가 주목하여 보고 두려워 가로되 주여 무슨 일이니이까 천사가 가로되 네 기도와 구제가 하나님 앞에 상달하여 기억하신 바가 되었으니
\par 5 네가 지금 사람들을 욥바에 보내어 베드로라 하는 시몬을 청하라
\par 6 저는 피장 시몬의 집에 우거하니 그 집은 해변에 있느니라 하더라
\par 7 마침 말하던 천사가 떠나매 고넬료가 집안 하인 둘과 종졸 가운데 경건한 사람 하나를 불러
\par 8 이 일을 다 고하고 욥바로 보내니라
\par 9 이튿날 저희가 행하여 성에 가까이 갔을 그 때에 베드로가 기도하려고 지붕에 올라가니 시간은 제 육시더라
\par 10 시장하여 먹고자 하매 사람이 준비할 때에 비몽사몽간에
\par 11 하늘이 열리며 한 그릇이 내려오는 것을 보니 큰 보자기 같고 네 귀를 매어 땅에 드리웠더라
\par 12 그 안에는 땅에 있는 각색 네 발 가진 짐승과 기는 것과 공중에 나는 것들이 있는데
\par 13 또 소리가 있으되 베드로야 일어나 잡아 먹으라 하거늘
\par 14 베드로가 가로되 주여 그럴 수 없나이다 속되고 깨끗지 아니한 물건을 내가 언제든지 먹지 아니하였삽나이다 한대
\par 15 또 두번째 소리 있으되 하나님께서 깨끗케 하신 것을 네가 속되다 하지 말라 하더라
\par 16 이런 일이 세 번 있은 후 그 그릇이 곧 하늘로 올리워 가니라
\par 17 베드로가 본바 환상이 무슨 뜻인지 속으로 의심하더니 마침 고넬료의 보낸 사람들이 시몬의 집을 찾아 문 밖에 서서
\par 18 불러 묻되 베드로라 하는 시몬이 여기 우거하느냐 하거늘
\par 19 베드로가 그 환상에 대하여 생각할 때에 성령께서 저더러 말씀하시되 두 사람이 너를 찾으니
\par 20 일어나 내려가 의심치 말고 함께 가라 내가 저희를 보내었느니라 하시니
\par 21 베드로가 내려가 그 사람들을 보고 가로되 내가 곧 너희의 찾는 사람이니 너희가 무슨 일로 왔느냐
\par 22 저희가 대답하되 백부장 고넬료는 의인이요 하나님을 경외하는 자라 유대 온 족속이 칭찬하더니 저가 거룩한 천사의 지시를 받아 너를 그 집으로 청하여 말을 들으려 하느니라 한대
\par 23 베드로가 불러 들여 유숙하게 하니라 이튿날 일어나 저희와 함께 갈새 욥바 두어 형제도 함께 가니라
\par 24 이튿날 가이사랴에 들어가니 고넬료가 일가와 가까운 친구들을 모아 기다리더니
\par 25 마침 베드로가 들어올 때에 고넬료가 맞아 발앞에 엎드리어 절하니
\par 26 베드로가 일으켜 가로되 일어서라 나도 사람이라 하고
\par 27 더불어 말하며 들어가 여러 사람의 모인것을 보고
\par 28 이르되 유대인으로서 이방인을 교제하는 것과 가까이 하는 것이 위법인 줄은 너희도 알거니와 하나님께서 내게 지시하사 아무도 속되다 하거나 깨끗지 않다 하지 말라 하시기로
\par 29 부름을 사양치 아니하고 왔노라 묻노니 무슨 일로 나를 불렀느뇨
\par 30 고넬료가 가로되 나흘 전 이맘때까지 내 집에서 제 구 시 기도를 하는데 홀연히 한 사람이 빛난 옷을 입고 내 앞에 서서
\par 31 말하되 고넬료야 하나님이 네 기도를 들으시고 네 구제를 기억하셨으니
\par 32 사람을 욥바에 보내어 베드로라 하는 시몬을 청하라 저가 바닷가 피장 시몬의 집에 우거하느니라 하시기로
\par 33 내가 곧 당신에게 사람을 보내었더니 오셨으니 잘하였나이다 이제 우리는 주께서 당신에게 명하신 모든 것을 듣고자 하여 다 하나님 앞에 있나이다
\par 34 베드로가 입을 열어 가로되 내가 참으로 하나님은 사람의 외모를 취하지 아니하시고
\par 35 각 나라중 하나님을 경외하며 의를 행하는 사람은 하나님이 받으시는줄 깨달았도다
\par 36 만유의 주 되신 예수 그리스도로 말미암아 화평의 복음을 전하사 이스라엘 자손들에게 보내신 말씀
\par 37 곧 요한이 그 세례를 반포한 후에 갈릴리에서 시작되어 온 유대에 두루 전파된 그것을 너희도 알거니와
\par 38 하나님이 나사렛 예수에게 성령과 능력을 기름붓듯 하셨으매 저가 두루 다니시며 착한 일을 행하시고 마귀에게 눌린 모든 자를 고치셨으니 이는 하나님이 함께 하셨음이라
\par 39 우리는 유대인의 땅과 예루살렘에서 그의 행하신 모든 일에 증인이라 그를 저희가 나무에 달아 죽였으나
\par 40 하나님이 사흘만에 다시 살리사 나타내시되
\par 41 모든 백성에게 하신 것이 아니요 오직 미리 택하신 증인 곧 죽은자 가운데서 일어나신 후 모시고 음식을 먹은 우리에게 하신 것이라
\par 42 우리를 명하사 백성에게 전도하되 하나님이 산 자와 죽은 자의 재판장으로 정하신 자가 곧 이 사람인 것을 증거하게 하셨고
\par 43 저에 대하여 모든 선지자도 증거하되 저를 믿는 사람들이 다 그 이름을 힘입어 죄 사함을 받는다 하였느니라
\par 44 베드로가 이 말 할 때에 성령이 말씀 듣는 모든 사람에게 내려오시니
\par 45 베드로와 함께 온 할례 받은 신자들이 이방인들에게도 성령 부어 주심을 인하여 놀라니
\par 46 이는 방언을 말하며 하나님 높임을 들음이러라
\par 47 이에 베드로가 가로되 이 사람들이 우리와 같이 성령을 받았으니 누가 능히 물로 세례 줌을 금하리요 하고
\par 48 명하여 예수 그리스도의 이름으로 세례를 주라 하니라 저희가 베드로에게 수일 더 유하기를 청하니라

\chapter{11}

\par 1 유대에 있는 사도들과 형제들이 이방인들도 하나님 말씀을 받았다 함을 들었더니
\par 2 베드로가 예루살렘에 올라갔을 때에 할례자들이 힐난하여
\par 3 가로되 네가 무할례자의 집에 들어가 함께 먹었다 하니
\par 4 베드로가 저희에게 이 일을 차례로 설명하여
\par 5 가로되 내가 욥바 성에서 기도할 때에 비몽사몽간에 환상을 보니 큰 보자기 같은 그릇을 네 귀를 매어 하늘로부터 내리워 내 앞에까지 드리우거늘
\par 6 이것을 주목하여 보니 땅에 네 발 가진 것과 들짐승과 기는 것과 공중에 나는 것들이 보이더라
\par 7 또 들으니 소리 있어 내게 이르되 베드로야 일어나 잡아 먹으라 하거늘
\par 8 내가 가로되 주여 그럴 수 없나이다 속되거나 깨끗지 아니한 물건은 언제든지 내 입에 들어간 일이 없나이다 하니
\par 9 또 하늘로부터 두 번째 소리 있어 내게 대답하되 하나님이 깨끗하게 하신 것을 네가 속되다 말라 하더라
\par 10 이런 일이 세번 있은 후에 모든 것이 다시 하늘로 끌려 올라가더라
\par 11 마침 세 사람이 내 우거한 집 앞에 섰으니 가이사랴에서 내게로 보낸 사람이라
\par 12 성령이 내게 명하사 아무 의심 말고 함께 가라 하시매 이 여섯 형제도 나와 함께 가서 그 사람의 집에 들어가니
\par 13 그가 우리에게 말하기를 천사가 내 집에 서서 말하되 네가 사람을 욥바에 보내어 베드로라 하는 시몬을 청하라
\par 14 그가 너와 네 온 집의 구원 얻을 말씀을 네게 이르리라 함을 보았다 하거늘
\par 15 내가 말을 시작할 때에 성령이 저희에게 임하시기를 처음 우리에게 하신 것과 같이 하는지라
\par 16 내가 주의 말씀에 요한은 물로 세례를 주었으나 너희는 성령으로 세례 받으리라 하신 것이 생각났노라
\par 17 그런즉 하나님이 우리가 주 예수 그리스도를 믿을 때에 주신 것과 같은 선물을 저희에게도 주셨으니 내가 누구관대 하나님을 능히 막겠느냐 하더라
\par 18 저희가 이 말을 듣고 잠잠하여 하나님께 영광을 돌려 가로되 그러면 하나님께서 이방인에게도 생명 얻는 회개를 주셨도다 하니라
\par 19 때에 스데반의 일로 일어난 환난을 인하여 흩어진 자들이 베니게와 구브로와 안디옥까지 이르러 도를 유대인에게만 전하는데
\par 20 그 중에 구브로와 구레네 몇 사람이 안디옥에 이르러 헬라인에게도 말하여 주 예수를 전파하니
\par 21 주의 손이 그들과 함께 하시매 수다한 사람이 믿고 주께 돌아오더라
\par 22 예루살렘 교회가 이 사람들의 소문을 듣고 바나바를 안디옥까지 보내니
\par 23 저가 이르러 하나님의 은혜를 보고 기뻐하여 모든 사람에게 굳은 마음으로 주께 붙어 있으라 권하니
\par 24 바나바는 착한 사람이요 성령과 믿음이 충만한 자라 이에 큰 무리가 주께 더하더라
\par 25 바나바가 사울을 찾으러 다소에 가서
\par 26 만나매 안디옥에 데리고 와서 둘이 교회에 일 년간 모여 있어 큰 무리를 가르쳤고 제자들이 안디옥에서 비로소 그리스도인이라 일컫음을 받게 되었더라
\par 27 그 때에 선지자들이 예루살렘에서 안디옥에 이르니
\par 28 그 중에 아가보라 하는 한 사람이 일어나 성령으로 말하되 천하가 크게 흉년 들리라 하더니 글라우디오 때에 그렇게 되니라
\par 29 제자들이 각각 그 힘대로 유대에 사는 형제들에게 부조를 보내기로 작정하고
\par 30 이를 실행하여 바나바와 사울의 손으로 장로들에게 보내니라

\chapter{12}

\par 1 그 때에 헤롯왕이 손을 들어 교회 중 몇 사람을 해하려하여
\par 2 요한의 형제 야고보를 칼로 죽이니
\par 3 유대인들이 이 일을 기뻐하는 것을 보고 베드로도 잡으려 할새 때는 무교절일이라
\par 4 잡으매 옥에 가두어 군사 넷씩인 네 패에게 맡겨 지키고 유월절 후에 백성 앞에 끌어 내고자 하더라
\par 5 이에 베드로는 옥에 갇혔고 교회는 그를 위하여 간절히 하나님께 빌더라
\par 6 헤롯이 잡아 내려고 하는 그 전날 밤에 베드로가 두 군사 틈에서 두 쇠사슬애 매여 누워 자는데 파숫군들이 문 밖에서 옥을 지키더니
\par 7 홀연히 주의 사자가 곁에 서매 옥중에 광채가 조요하며 또 베드로의 옆구리를 쳐 깨워 가로되 급히 일어나라 하니 쇠사슬이 그 손에서 벗어지더라
\par 8 천사가 가로되 띠를 띠고 신을 들메라 하거늘 베드로가 그대로 하니 천사가 또 가로되 겉옷을 입고 따라 오라 한대
\par 9 베드로가 나와서 따라갈새 천사의 하는 것이 참인줄 알지 못하고 환상을 보는가 하니라
\par 10 이에 첫째와 둘째 파수를 지나 성으로 통한 쇠문에 이르니 문이 절로 열리는지라 나와 한 거리를 지나매 천사가 곧 떠나더라
\par 11 이에 베드로가 정신이 나서 가로되 내가 이제야 참으로 주께서 그의 천사를 보내어 나를 헤롯의 손과 유대 백성의 모든 기대에서 벗어나게 하신줄 알겠노라 하여
\par 12 깨닫고 마가라 하는 요한의 어머니 마리아의 집에 가니 여러 사람이 모여 기도하더라
\par 13 베드로가 대문을 두드린대 로데라 하는 계집아이가 영접하러 나왔다가
\par 14 베드로의 음성인줄 알고 기뻐하여 문을 미처 열지 못하고 달려 들어가 말하되 베드로가 대문 밖에 섰더라 하니
\par 15 저희가 말하되 네가 미쳤다 하나 계집아이는 힘써 말하되 참말이라 하니 저희가 말하되 그러면 그의 천사라 하더라
\par 16 베드로가 문 두드리기를 그치지 아니하니 저희가 문을 열어 베드로를 보고 놀라는지라
\par 17 베드로가 저희에게 손짓하여 종용하게 하고 주께서 자기를 이끌어 옥에서 나오게 하던 일을 말하고 또 야고보와 형제들에게 이 말을 전하라 하고 떠나 다른 곳으로 가니라
\par 18 날이 새매 군사들은 베드로가 어떻게 되었는지 알지 못하여 적지 않게 소동하니
\par 19 헤롯이 그를 찾아도 보지 못하매 파숫군들을 심문하고 죽이라 명하니라 헤롯이 유대를 떠나 가이사랴로 내려가서 거하니라
\par 20 헤롯이 두로와 시돈 사람들을 대단히 노여워하나 저희 지방이 왕국에서 나는 양식을 쓰는고로 일심으로 그에게 나아와 왕의 침소 맡은 신하 블라스도를 친하여 화목하기를 청한지라
\par 21 헤롯이 날을 택하여 왕복을 입고 위에 앉아 백성을 효유한대
\par 22 백성들이 크게 부르되 이것은 신의 소리요 사람의 소리는 아니라 하거늘
\par 23 헤롯이 영광을 하나님께로 돌리지 아니하는 고로 주의 사자가 곧 치니 충이 먹어 죽으니라
\par 24 하나님의 말씀은 흥왕하여 더하더라
\par 25 바나바와 사울이 부조의 일을 마치고 마가라 하는 요한을 데리고 예루살렘에서 돌아오니라

\chapter{13}

\par 1 안디옥 교회에 선지자들과 교사들이 있으니 곧 바나바와 니게르라 하는 시므온과 구레네 사람 루기오와 분봉왕 헤롯의 젖동생 마나엔과 및 사울이라
\par 2 주를 섬겨 금식할 때에 성령이 가라사대 내가 불러 시키는 일을 위하여 바나바와 사울을 따로 세우라 하시니
\par 3 이에 금식하며 기도하고 두 사람에게 안수하여 보내니라
\par 4 두 사람이 성령의 보내심을 받아 실루기아에 내려가 거기서 배 타고 구브로에 가서
\par 5 살라미에 이르러 하나님의 말씀을 유대인의 여러 회당에서 전할새 요한을 수종자로 두었더라
\par 6 온 섬 가운데로 지나서 바보에 이르러 바예수라 하는 유대인 거짓 선지자 박수를 만나니
\par 7 그가 총독 서기오 바울과 함께 있으니 서기오 바울은 지혜 있는 사람이라 바나바와 사울을 불러 하나님 말씀을 듣고자 하더라
\par 8 이 박수 엘루마는 (이 이름을 번역하면 박수라) 저희를 대적하여 총독으로 믿지 못하게 힘쓰니
\par 9 바울이라고 하는 사울이 성령이 충만하여 그를 주목하고
\par 10 가로되 모든 궤계와 악행이 가득한 자요 마귀의 자식이요 모든 의의 원수여 주의 바른 길을 굽게 하기를 그치지 아니하겠느냐
\par 11 보라 이제 주의 손이 네 위에 있으니 네가 소경이 되어 얼마 동안 해를 보지 못하리라 하니 즉시 안개와 어두움이 그를 덮어 인도할 사람을 두루 구하는지라
\par 12 이에 총독이 그렇게 된 것을 보고 믿으며 주의 가르치심을 기이히 여기니라
\par 13 바울과 및 동행하는 사람들이 바보에서 배 타고 밤빌리아에 있는 버가에 이르니 요한은 저희에게서 떠나 예루살렘으로 돌아가고
\par 14 저희는 버가로부터 지나 비시디아 안디옥에 이르러 안식일에 회당에 들어가 앉으니라
\par 15 율법과 선지자의 글을 읽은 후에 회당장들이 사람을 보내어 물어 가로되 형제들아 만일 백성을 권할 말이 있거든 말하라 하니
\par 16 바울이 일어나 손짓하며 말하되 이스라엘 사람들과 및 하나님을 경외하는 사람들아 들으라
\par 17 이 이스라엘 백성의 하나님이 우리 조상들을 택하시고 애굽 땅에서 나그네 된 그 백성을 높여 큰 권능으로 인도하여 내사
\par 18 광야에서 약 사십년간 저희 소행을 참으시고
\par 19 가나안 땅 일곱 족속을 멸하사 그 땅을 기업으로 주시고 (약 사백 오십 년간)
\par 20 그 후에 선지자 사무엘 때까지 사사를 주셨더니
\par 21 그 후에 저희가 왕을 구하거늘 하나님이 베냐민 지파 사람 기스의 아들 사울을 사십 년간 주셨다가
\par 22 폐하시고 다윗을 왕으로 세우시고 증거하여 가라사대 내가 이새의 아들 다윗을 만나니 내 마음에 합한 사람이라 내 뜻을 다 이루게 하리라 하시더니
\par 23 하나님이 약속하신 대로 이 사람의 씨에서 이스라엘을 위하여 구주를 세우셨으니 곧 예수라
\par 24 그 오시는 앞에 요한이 먼저 회개의 세례를 이스라엘 모든 백성에게 전파하니라
\par 25 요한이 그 달려 갈 길을 마칠 때에 말하되 너희가 나를 누구로 생각하느냐 나는 그리스도가 아니라 내 뒤에 오시는 이가 있으니 나는 그 발의 신 풀기도 감당치 못하리라 하였으니
\par 26 형제들 아브라함의 후예와 너희 중 하나님을 경외하는 사람들아 이 구원의 말씀을 우리에게 보내셨거늘
\par 27 예루살렘에 사는 자들과 저희 관원들이 예수와 및 안식일마다 외우는바 선지자들의 말을 알지 못하므로 예수를 정죄하여 선지자들의 말을 응하게 하였도다
\par 28 죽일 죄를 하나도 찾지 못하였으나 빌라도에게 죽여 달라 하였으니
\par 29 성경에 저를 가리켜 기록한 말씀을 다 응하게 한 것이라 후에 나무에서 내려다가 무덤에 두었으나
\par 30 하나님이 죽은자 가운데서 저를 살리신지라
\par 31 갈릴리로부터 예루살렘에 함께 올라간 사람들에게 여러날 보이셨으니 저희가 이제 백성 앞에 그의 증인이라
\par 32 우리도 조상들에게 주신 약속을 너희에게 전파하노니
\par 33 곧 하나님이 예수를 일으키사 우리 자녀들에게 이 약속을 이루게 하셨다 함이라 시편 둘째 편에 기록한 바와 같이 너는 내 아들이라 오늘 너를 낳았다 하셨고
\par 34 또 하나님께서 죽은 자 가운데서 저를 일으키사 다시 썩음을 당하지 않게 하실 것을 가르쳐 가라사대 내가 다윗의 거룩하고 미쁜 은사를 너희에게 주리라 하셨으니
\par 35 그러므로 또 다른 편에 일렀으되 주의 거룩한 자로 썩음을 당하지 않게 하시리라 하셨느니라
\par 36 다윗은 당시에 하나님의 뜻을 좇아 섬기다가 잠들어 그 조상들과 함께 묻혀 썩음을 당하였으되
\par 37 하나님의 살리신 이는 썩음을 당하지 아니하였나니
\par 38 그러므로 형제들아 너희가 알 것은 이 사람을 힘입어 죄 사함을 너희에게 전하는 이것이며
\par 39 또 모세의 율법으로 너희가 의롭다 하심을 얻지 못하던 모든 일에도 이 사람을 힘입어 믿는 자마다 의롭다 하심을 얻는 이것이라
\par 40 그런즉 너희는 선지자들로 말씀하신 것이 너희에게 미칠까 삼가라
\par 41 일렀으되 보라 멸시하는 사람들아 너희는 놀라고 망하라 내가 너희 때를 당하여 한 일을 행할 것이니 사람이 너희에게 이를지라도 도무지 믿지 못할 일이라 하였느니라 하니라
\par 42 저희가 나갈새 사람들이 청하되 다음 안식일에도 이 말씀을 하라하더라
\par 43 폐회한 후에 유대인과 유대교에 입교한 경건한 사람들이 많이 바울과 바나바를 좇으니 두 사도가 더불어 말하고 항상 하나님의 은혜 가운데 있으라 권하니라
\par 44 그 다음 안식일에는 온 성이 거의 다 하나님 말씀을 듣고자 하여 모이니
\par 45 유대인들이 그 무리를 보고 시기가 가득하여 바울의 말한 것을 변박하고 비방하거늘
\par 46 바울과 바나바가 담대히 말하여 가로되 하나님의 말씀을 마땅히 먼저 너희에게 전할 것이로되 너희가 버리고 영생 얻음에 합당치 않은 자로 자처하기로 우리가 이방인에게로 향하노라
\par 47 주께서 이같이 우리를 명하시되 내가 너를 이방의 빛을 삼아 너로 땅 끝까지 구원하게 하리라 하셨느니라 하니
\par 48 이방인들이 듣고 기뻐하여 하나님의 말씀을 찬송하며 영생을 주시기로 작정된 자는 다 믿더라
\par 49 주의 말씀이 그 지방에 두루 퍼지니라
\par 50 이에 유대인들이 경건한 귀부인들과 그 성내 유력자들을 선동하여 바울과 바나바를 핍박케 하여 그 지경에서 쫓아내니
\par 51 두 사람이 저희를 향하여 발에 티끌을 떨어 버리고 이고니온으로 가거늘
\par 52 제자들은 기쁨과 성령이 충만하니라

\chapter{14}

\par 1 이에 이고니온에서 두 사도가 함께 유대인의 회당에 들어가 말하니 유대와 헬라의 허다한 무리가 믿더라
\par 2 그러나 순종치 아니하는 유대인들이 이방인들의 마음을 선동하여 형제들에게 악감을 품게 하거늘
\par 3 두 사도가 오래 있어 주를 힘입어 담대히 말하니 주께서 저희 손으로 표적과 기사를 행하게 하여 주사 자기 은혜의 말씀을 증거하시니
\par 4 그 성내 무리가 나뉘어 유대인을 좇는 자도 있고 두 사도를 좇는 자도 있는지라
\par 5 이방인과 유대인과 그 관원들이 두 사도를 능욕하며 돌로 치려고 달려드니
\par 6 저희가 알고 도망하여 루가오니아의 두 성 루스드라와 더베와 및 그 근방으로 가서
\par 7 거기서 복음을 전하니라
\par 8 루스드라에 발을 쓰지 못하는 한 사람이 있어 앉았는데 나면서 앉은뱅이 되어 걸어 본 적이 없는 자라
\par 9 바울의 말하는 것을 듣거늘 바울이 주목하여 구원받을만한 믿음이 그에게 있는것을 보고
\par 10 큰 소리로 가로되 네 발로 바로 일어서라 하니 그 사람이 뛰어 걷는지라
\par 11 무리가 바울의 행한 일을 보고 루가오니아 방언으로 소리질러 가로되 신들이 사람의 형상으로 우리 가운데 내려 오셨다 하여
\par 12 바나바는 쓰스라 하고 바울은 그 중에 말하는 자이므로 허메라 하더라
\par 13 성밖 쓰스신당의 제사장이 소와 화관들을 가지고 대문 앞에 와서 무리와 함께 제사하고자 하니
\par 14 두 사도 바나바와 바울이 듣고 옷을 찢고 무리 가운데 뛰어 들어가서 소리질러
\par 15 가로되 여러분이여 어찌하여 이러한 일을 하느냐 우리도 너희와 같은 성정을 가진 사람이라 너희에게 복음을 전하는 것은 이 헛된 일을 버리고 천지와 바다와 그 가운데 만유를 지으시고 살아 계신 하나님께로 돌아 오라 함이라
\par 16 하나님이 지나간 세대에는 모든 족속으로 자기의 길들을 다니게 묵인하셨으나
\par 17 그러나 자기를 증거하지 아니하신 것이 아니니 곧 너희에게 하늘로서 비를 내리시며 결실기를 주시는 선한 일을 하사 음식과 기쁨으로 너희 마음에 만족케 하셨느니라 하고
\par 18 이렇게 말하여 겨우 무리를 말려 자기들에게 제사를 못하게 하니라
\par 19 유대인들이 안디옥과 이고니온에서 와서 무리를 초인하여 돌로 바울을 쳐서 죽은 줄로 알고 성밖에 끌어 내치니라
\par 20 제자들이 둘러 섰을 때에 바울이 일어나 성에 들어갔다가 이튿날 바나바와 함께 더베로 가서
\par 21 복음을 그 성에서 전하여 많은 사람을 제자로 삼고 루스드라와 이고니온과 안디옥으로 돌아가서
\par 22 제자들의 마음을 굳게 하여 이 믿음에 거하라 권하고 또 우리가 하나님 나라에 들어가려면 많은 환난을 겪어야 할 것이라 하고
\par 23 각 교회에서 장로들을 택하여 금식 기도하며 저희를 그 믿은바 주께 부탁하고
\par 24 비시디아 가운데로 지나가서 밤빌리아에 이르러
\par 25 도를 버가에서 전하고 앗달리아로 내려가서
\par 26 거기서 배 타고 안디옥에 이르니 이곳은 두 사도의 이룬 그 일을 위하여 전에 하나님의 은혜에 부탁하던 곳이라
\par 27 이르러 교회를 모아 하나님이 함께 행하신 모든 일과 이방인들에게 믿음의 문을 여신 것을 고하고
\par 28 제자들과 함께 오래 있으니라

\chapter{15}

\par 1 어떤 사람들이 유대로부터 내려와서 형제들을 가르치되 너희가 모세의 법대로 할례를 받지 아니하면 능히 구원을 얻지 못하리라 하니라
\par 2 바울과 바나바와 저희 사이에 적지 아니한 다툼과 변론이 일어난지라 형제들이 이 문제에 대하여 바울과 바나바와 및 그 중에 몇 사람을 예루살렘에 있는 사도와 장로들에게 보내기로 작정하니
\par 3 저희가 교회의 전송을 받고 베니게와 사마리아로 다녀가며 이방인들의 주께 돌아온 일을 말하여 형제들을 다 크게 기쁘게 하더라
\par 4 예루살렘에 이르러 교회와 사도와 장로들에게 영접을 받고 하나님이 자기들과 함께 계셔 행하신 모든 일을 말하매
\par 5 바리새파 중에 믿는 어떤 사람들이 일어나 말하되 이방인에게 할례 주고 모세의 율법을 지키라 명하는 것이 마땅하다 하니라
\par 6 사도와 장로들이 이 일을 의논하러 모여
\par 7 많은 변론이 있은 후에 베드로가 일어나 말하되 형제들아 너희도 알거니와 하나님이 이방인들로 내 입에서 복음의 말씀을 들어 믿게 하시려고 오래 전부터 너희 가운데서 나를 택하시고
\par 8 또 마음을 아시는 하나님이 우리에게와 같이 저희에게도 성령을 주어 증거하시고
\par 9 믿음으로 저희 마음을 깨끗이 하사 저희나 우리나 분간치 아니하셨느니라
\par 10 그런데 지금 너희가 어찌하여 하나님을 시험하여 우리 조상과 우리도 능히 메지 못하던 멍에를 제자들의 목에 두려느냐
\par 11 우리가 저희와 동일하게 주 예수의 은혜로 구원 받는 줄을 믿노라 하니라
\par 12 온 무리가 가만히 있어 바나바와 바울이 하나님이 자기들로 말미암아 이방인 중에서 행하신 표적과 기사 고하는 것을 듣더니
\par 13 말을 마치매 야고보가 대답하여 가로되 형제들아 내 말을 들으라
\par 14 하나님이 처음으로 이방인 중에서 자기 이름을 위할 백성을 취하시려고 저희를 권고하신 것을 시므온이 고하였으니
\par 15 선지자들의 말씀이 이와 합하도다 기록된바
\par 16 이 후에 내가 돌아와서 다윗의 무너진 장막을 다시 지으며 또 그 퇴락한 것을 다시 지어 일으키리니
\par 17 이는 그 남은 사람들과 내 이름으로 일컬음을 받는 모든 이방인들로 주를 찾게 하려 함이라 하셨으니
\par 18 즉 예로부터 이것을 알게 하시는 주의 말씀이라 함과 같으니라
\par 19 그러므로 내 의견에는 이방인중에서 하나님께로 돌아 오는 자들을 괴롭게 말고
\par 20 다만 우상의 더러운 것과 음행과 목매어 죽인 것과 피를 멀리 하라고 편지하는 것이 가하니
\par 21 이는 예로부터 각 성에서 모세를 전하는 자가 있어 안식일마다 회당에서 그 글을 읽음이니라 하더라
\par 22 이에 사도와 장로와 온 교회가 그 중에서 사람을 택하여 바울과 바나바와 함께 안디옥으로 보내기를 가결하니 곧 형제 중에 인도자인 바사바라 하는 유다와 실라더라
\par 23 그 편에 편지를 부쳐 이르되 사도와 장로된 형제들은 안디옥과 수리아와 길리기아에 있는 이방인 형제들에게 문안하노라
\par 24 들은즉 우리 가운데서 어떤 사람들이 우리의 시킨 것도 없이 나가서 말로 너희를 괴롭게 하고 마음을 혹하게 한다 하기로
\par 25 사람을 택하여 우리 주 예수 그리스도의 이름을 위하여 생명을 아끼지 아니하는 자인 우리의 사랑하는 바나바와 바울과 함께 너희에게 보내기를 일치 가결하였노라
\par 26 tex
\par 27 그리하여 유다와 실라를 보내니 저희도 이 일을 말로 전하리라
\par 28 성령과 우리는 이 요긴한 것들 외에 아무 짐도 너희에게 지우지 아니하는 것이 가한줄 알았노니
\par 29 우상의 제물과 피와 목매어 죽인 것과 음행을 멀리 할지니라 이에 스스로 삼가면 잘되리라 평안함을 원하노라 하였더라
\par 30 저희가 작별하고 안디옥에 내려가 무리를 모은 후에 편지를 전하니
\par 31 읽고 그 위로한 말을 기뻐하더라
\par 32 유다와 실라도 선지자라 여러 말로 형제를 권면하여 굳게 하고
\par 33 얼마 있다가 평안히 가라는 전송을 형제들에게 받고 자기를 보내던 사람들에게로 돌아가되
\par 34 (없 음+
\par 35 바울과 바나바는 안디옥에서 유하여 다수한 다른 사람들과 함께 주의 말씀을 가르치며 전파하니라
\par 36 수일 후에 바울이 바나바더러 말하되 우리가 주의 말씀을 전한 각 성으로 다시 가서 형제들이 어떠한가 방문하자 하니
\par 37 바나바는 마가라 하는 요한도 데리고 가고자 하나
\par 38 바울은 밤빌리아에서 자기들을 떠나 한가지로 일하러 가지 아니한 자를 데리고 가는 것이 옳지 않다 하여
\par 39 서로 심히 다투어 피차 갈라 서니 바나바는 마가를 데리고 배 타고 구브로로 가고
\par 40 바울은 실라를 택한 후에 형제들에게 주의 은혜에 부탁함을 받고 떠나
\par 41 수리아와 길리기아로 다녀가며 교회들을 굳게 하니라

\chapter{16}

\par 1 바울이 더베와 루스드라에도 이르매 거기 디모데라 하는 제자가 있으니 그 모친은 믿는 유대 여자요 부친은 헬라인이라
\par 2 디모데는 루스드라와 이고니온에 있는 형제들에게 칭찬 받는 자니
\par 3 바울이 그를 데리고 떠나고자 할새 그 지경에 있는 유대인을 인하여 그를 데려다가 할례를 행하니 이는 그 사람들이 그의 부친은 헬라인인줄 다 앎이러라
\par 4 여러 성으로 다녀 갈 때에 예루살렘에 있는 사도와 장로들의 작정한 규례를 저희에게 주어 지키게 하니
\par 5 이에 여러 교회가 믿음이 더 굳어지고 수가 날마다 더하니라
\par 6 성령이 아시아에서 말씀을 전하지 못하게 하시거늘 브루기아와 갈라디아 땅으로 다녀가
\par 7 무시아 앞에 이르러 비두니아로 가고자 애쓰되 예수의 영이 허락지 아니하시는지라
\par 8 무시아를 지나 드로아로 내려갔는데
\par 9 밤에 환상이 바울에게 보이니 마게도냐 사람 하나가 서서 그에게 청하여 가로되 마게도냐로 건너와서 우리를 도우라 하거늘
\par 10 바울이 이 환상을 본 후에 우리가 곧 마게도냐로 떠나기를 힘쓰니 이는 하나님이 저 사람들에게 복음을 전하라고 우리를 부르신 줄로 인정함이러라
\par 11 드로아에서 배로 떠나 사모드라게로 직행하여 이튿날 네압볼리로 가고
\par 12 거기서 빌립보에 이르니 이는 마게도냐 지경 첫성이요 또 로마의 식민지라 이 성에서 수일을 유하다가
\par 13 안식일에 우리가 기도처가 있는가 하여 문밖 강 가에 나가 거기 앉아서 모인 여자들에게 말하더니
\par 14 두아디라성의 자주 장사로서 하나님을 공경하는 루디아라 하는 한 여자가 들었는데 주께서 그 마음을 열어 바울의 말을 청종하게 하신지라
\par 15 저와 그 집이 다 세례를 받고 우리에게 청하여 가로되 만일 나를 주 믿는 자로 알거든 내 집에 들어와 유하라 하고 강권하여 있게 하니라
\par 16 우리가 기도하는 곳에 가다가 점하는 귀신 들린 여종 하나를 만나니 점으로 그 주인들을 크게 이하게 하는 자라
\par 17 바울과 우리를 좇아와서 소리질러 가로되 이 사람들은 지극히 높은 하나님의 종으로 구원의 길을 너희에게 전하는 자라 하며
\par 18 이같이 여러 날을 하는지라 바울이 심히 괴로와하여 돌이켜 그 귀신에게 이르되 예수 그리스도의 이름으로 내가 네게 명하노니 그에게서 나오라 하니 귀신이 즉시 나오니라
\par 19 종의 주인들은 자기 이익의 소망이 끊어진 것을 보고 바울과 실라를 잡아 가지고 저자로 관원들에게 끌어 갔다가
\par 20 상관들 앞에 데리고 가서 말하되 이 사람들이 유대인인데 우리 성을 심히 요란케 하여
\par 21 로마 사람인 우리가 받지도 못하고 행치도 못할 풍속을 전한다 하거늘
\par 22 무리가 일제히 일어나 송사하니 상관들이 옷을 찢어 벗기고 매로 치라 하여
\par 23 많이 친 후에 옥에 가두고 간수에게 분부하여 든든히 지키라 하니
\par 24 그가 이러한 영을 받아 저희를 깊은 옥에 가두고 그 발을 착고에 든든히 채웠더니
\par 25 밤중쯤 되어 바울과 실라가 기도하고 하나님을 찬미하매 죄수들이 듣더라
\par 26 이에 홀연히 큰 지진이 나서 옥터가 움직이고 문이 곧 다 열리며 모든 사람의 매인 것이 다 벗어진지라
\par 27 간수가 자다가 깨어 옥문들이 열린 것을 보고 죄수들이 도망한줄 생각하고 검을 빼어 자결하려 하거늘
\par 28 바울이 크게 소리질러 가로되 네 몸을 상하지 말라 우리가 다 여기 있노라 하니
\par 29 간수가 등불을 달라고 하며 뛰어 들어가 무서워 떨며 바울과 실라 앞에 부복하고
\par 30 저희를 데리고 나가 가로되 선생들아 내가 어떻게 하여야 구원을 얻으리이까 하거늘
\par 31 가로되 주 예수를 믿으라 그리하면 너와 네 집이 구원을 얻으리라 하고
\par 32 주의 말씀을 그 사람과 그 집에 있는 모든 사람에게 전하더라
\par 33 밤 그 시에 간수가 저희를 데려다가 그 맞은 자리를 씻기고 자기와 그 권속이 다 세례를 받은 후
\par 34 저희를 데리고 자기 집에 올라가서 음식을 차려주고 저와 온 집이 하나님을 믿었으므로 크게 기뻐하니라
\par 35 날이 새매 상관들이 아전을 보내어 이 사람들을 놓으라 하니
\par 36 간수가 이 말대로 바울에게 고하되 상관들이 사람을 보내어 너희를 놓으라 하였으니 이제는 나가서 평안히 가라 하거늘
\par 37 바울이 이르되 로마 사람인 우리를 죄도 정치 아니하고 공중 앞에서 때리고 옥에 가두었다가 이제는 가만히 우리를 내어 보내고자 하느냐 아니라 저희가 친히 와서 우리를 데리고 나가야하리라 한대
\par 38 아전들이 이 말로 상관들에게 고하니 저희가 로마 사람이라 하는 말을 듣고 두려워하여
\par 39 와서 권하여 데리고 나가 성에서 떠나기를 청하니
\par 40 두 사람이 옥에서 나가 루디아의 집에 들어가서 형제들을 만나보고 위로하고 가니라

\chapter{17}

\par 1 저희가 암비볼리와 아볼로니아로 다녀가 데살로니가에 이르니 거기 유대인의 회당이 있는지라
\par 2 바울이 자기의 규례대로 저희에게로 들어가서 세 안식일에 성경을 가지고 강론하며
\par 3 뜻을 풀어 그리스도가 해를 받고 죽은 자 가운데서 다시 살아야 할것을 증명하고 이르되 내가 너희에게 전하는 이 예수가 곧 그리스도라 하니
\par 4 그 중에 어떤 사람 곧 경건한 헬라인의 큰 무리와 적지 않은 귀부인도 권함을 받고 바울과 실라를 좇으나
\par 5 그러나 유대인들은 시기하여 저자의 어떤 괴악한 사람들을 데리고 떼를 지어 성을 소동케 하여 야손의 집에 달려들어 저희를 백성에게 끌어 내려고 찾았으나
\par 6 발견치 못하매 야손과 및 형제를 끌고 읍장들 앞에 가서 소리질러 가로되 천하를 어지럽게 하던 이 사람들이 여기도 이르매
\par 7 야손이 들였도다 이 사람들이 다 가이사의 명을 거역하여 말하되 다른 임금 곧 예수라 하는 이가 있다 하더이다 하니
\par 8 무리와 읍장들이 이 말을 듣고 소동하여
\par 9 야손과 그 나머지 사람들에게 보를 받고 놓으니라
\par 10 밤에 형제들이 곧 바울과 실라를 베뢰아로 보내니 저희가 이르러 유대인의 회당에 들어가니라
\par 11 베뢰아 사람은 데살로니가에 있는 사람보다 더 신사적이어서 간절한 마음으로 말씀을 받고 이것이 그러한가 하여 날마다 성경을 상고하므로
\par 12 그중에 믿는 사람이 많고 또 헬라의 귀부인과 남자가 적지 아니하나
\par 13 데살로니가에 있는 유대인들이 바울이 하나님 말씀을 베뢰아에서도 전하는 줄을 알고 거기도 가서 무리를 움직여 소동케 하거늘
\par 14 형제들이 곧 바울을 내어 보내어 바다까지 가게 하되 실라와 디모데는 아직 거기 유하더라
\par 15 바울을 인도하는 사람들이 데리고 아덴까지 이르러 바울에게서 실라와 디모데를 자기에게로 속히 오게 하라는 명을 받고 떠나니라
\par 16 바울이 아덴에서 저희를 기다리다가 온 성에 우상이 가득한 것을 보고 마음에 분하여
\par 17 회당에서는 유대인과 경건한 사람들과 또 저자에서는 날마다 만나는 사람들과 변론하니
\par 18 어떤 에비구레오와 스도이고 철학자들도 바울과 쟁론할새 혹은 이르되 이 말장이가 무슨 말을 하고자 하느뇨 하고 혹은 이르되 이방 신들을 전하는 사람인가보다 하니 이는 바울이 예수와 또 몸의 부활 전함을 인함이러라
\par 19 붙들어 가지고 아레오바고로 가며 말하기를 우리가 너의 말하는 이 새 교가 무엇인지 알수 있겠느냐
\par 20 네가 무슨 이상한 것을 우리 귀에 들려 주니 그 무슨 뜻인지 알고자 하노라 하니
\par 21 모든 아덴 사람과 거기서 나그네 된 외국인들이 가장 새로 되는 것을 말하고 듣는 이외에 달리는 시간을 쓰지 않음이더라
\par 22 바울이 아레오바고 가운데 서서 말하되 아덴 사람들아 너희를 보니 범사에 종교성이 많도다
\par 23 내가 두루 다니며 너희의 위하는 것들을 보다가 알지 못하는 신에게라고 새긴 단도 보았으니 그런즉 너희가 알지 못하고 위하는 그것을 내가 너희에게 알게 하리라
\par 24 우주와 그 가운데 있는 만유를 지으신 신께서는 천지의 주재시니 손으로 지은 전에 계시지 아니하시고
\par 25 또 무엇이 부족한 것처럼 사람의 손으로 섬김을 받으시는 것이 아니니 이는 만민에게 생명과 호흡과 만물을 친히 주시는 자이심이라
\par 26 인류의 모든 족속을 한 혈통으로 만드사 온 땅에 거하게 하시고 저희의 년대를 정하시며 거주의 경계를 한하셨으니
\par 27 이는 사람으로 하나님을 혹 더듬어 찾아 발견케 하려 하심이로되 그는 우리 각 사람에게서 멀리 떠나 계시지 아니하도다
\par 28 우리가 그를 힘입어 살며 기동하며 있느니라 너희 시인 중에도 어떤 사람들의 말과 같이 우리가 그의 소생이라 하니
\par 29 이와 같이 신의 소생이 되었은즉 신을 금이나 은이나 돌에다 사람의 기술과 고안으로 새긴 것들과 같이 여길 것이 아니니라
\par 30 알지 못하던 시대에는 하나님이 허물치 아니하셨거니와 이제는 어디든지 사람을 다 명하사 회개하라 하셨으니
\par 31 이는 정하신 사람으로 하여금 천하를 공의로 심판할 날을 작정하시고 이에 저를 죽은 자 가운데서 다시 살리신 것으로 모든 사람에게 믿을만한 증거를 주셨음이니라 하니라
\par 32 저희가 죽은 자의 부활을 듣고 혹은 기롱도 하고 혹은 이 일에 대하여 네 말을 다시 듣겠다 하니
\par 33 이에 바울이 저희 가운데서 떠나매
\par 34 몇 사람이 그를 친하여 믿으니 그 중 아레오바고 관원 디오누시오와 다마리라 하는 여자와 또 다른 사람들도 있었더라

\chapter{18}

\par 1 이 후에 바울이 아덴을 떠나 고린도에 이르러
\par 2 아굴라라 하는 본도에서 난 유대인 하나를 만나니 글라우디오가 모든 유대인을 명하여 로마에서 떠나라 한고로 그가 그 아내 브리스길라와 함께 이달리야로부터 새로 온지라 바울이 그들에게 가매
\par 3 업이 같으므로 함께 거하여 일을 하니 그 업은 장막을 만드는 것이더라
\par 4 안식일마다 바울이 회당에서 강론하고 유대인과 헬라인을 권면하니라
\par 5 실라와 디모데가 마게도냐로서 내려오매 바울이 하나님의 말씀에 붙잡혀 유대인들에게 예수는 그리스도라 밝히 증거하니
\par 6 저희가 대적하여 훼방하거늘 바울이 옷을 떨어 가로되 너희 피가 너희 머리로 돌아갈 것이요 나는 깨끗하니라 이 후에는 이방인에게로 가리라 하고
\par 7 거기서 옮겨 하나님을 공경하는 디도 유스도라 하는 사람의 집에 들어가니 그 집이 회당 옆이라
\par 8 또 회당장 그리스보가 온 집으로 더불어 주를 믿으며 수다한 고린도 사람도 듣고 믿어 세례를 받더라
\par 9 밤에 주께서 환상 가운데 바울에게 말씀하시되 두려워하지 말며 잠잠하지 말고 말하라
\par 10 내가 너와 함께 있으매 아무 사람도 너를 대적하여 해롭게 할 자가 없을 것이니 이는 이 성중에 내 백성이 많음이라 하시더라
\par 11 일년 육개월을 유하며 그들 가운데서 하나님의 말씀을 가르치니라
\par 12 갈리오가 아가야 총독 되었을 때에 유대인이 일제히 일어나 바울을 대적하여 재판 자리로 데리고 와서
\par 13 말하되 이 사람이 율법을 어기어 하나님을 공경하라고 사람들을 권한다 하거늘
\par 14 바울이 입을 열고자 할 때에 갈리오가 유대인들에게 이르되 너희 유대인들아 만일 무슨 부정한 일이나 괴악한 행동이었으면 내가 너희 말을 들어주는 것이 가하거니와
\par 15 만일 문제가 언어와 명칭과 너희 법에 관한 것이면 너희가 스스로 처리하라 나는 이러한 일에 재판장 되기를 원치 아니하노라 하고
\par 16 저희를 재판 자리에서 쫓아내니
\par 17 모든 사람이 회당장 소스데네를 잡아 재판 자리 앞에서 때리되 갈리오가 이 일을 상관치 아니하니라
\par 18 바울은 더 여러 날 유하다가 형제들을 작별하고 배 타고 수리아로 떠나갈 새 브리스길라와 아굴라도 함께 하더라 바울이 일찍 서원이 있으므로 겐그레아에서 머리를 깎았더라
\par 19 에베소에 와서 저희를 거기 머물러 두고 자기는 회당에 들어가서 유대인들과 변론하니
\par 20 여러 사람이 더 오래 있기를 청하되 허락지 아니하고
\par 21 작별하여 가로되 만일 하나님의 뜻이면 너희에게 돌아오리라 하고 배를 타고 에베소를 떠나
\par 22 가이사랴에서 상륙하여 올라가 교회의 안부를 물은 후에 안디옥으로 내려가서
\par 23 얼마 있다가 떠나 갈라디아와 브루기아 땅을 차례로 다니며 모든 제자를 굳게 하니라
\par 24 알렉산드리아에서 난 아볼로라 하는 유대인이 에베소에 이르니 이 사람은 학문이 많고 성경에 능한 자라
\par 25 그가 일찍 주의 도를 배워 열심으로 예수에 관한 것을 자세히 말하며 가르치나 요한의 세례만 알 따름이라
\par 26 그가 회당에서 담대히 말하기를 시작하거늘 브리스길라와 아굴라가 듣고 데려다가 하나님의 도를 더 자세히 풀어 이르더라
\par 27 아볼로가 아가야로 건너가고자 하니 형제들이 저를 장려하며 제자들에게 편지하여 영접하라 하였더니 저가 가매 은혜로 말미암아 믿은 자들에게 많은 유익을 주니
\par 28 이는 성경으로써 예수는 그리스도라고 증거하여 공중 앞에서 유력하게 유대인의 말을 이김일러라

\chapter{19}

\par 1 아볼로가 고린도에 있을 때에 바울이 윗지방으로 다녀 에베소에 와서 어떤 제자들을 만나
\par 2 가로되 너희가 믿을 때에 성령을 받았느냐 가로되 아니라 우리는 성령이 있음도 듣지 못하였노라
\par 3 바울이 가로되 그러면 너희가 무슨 세례를 받았느냐 대답하되 요한의 세례로라
\par 4 바울이 가로되 요한이 회개의 세례를 베풀며 백성에게 말하되 내뒤에 오시는 이를 믿으라 하였으니 이는 곧 예수라 하거늘
\par 5 저희가 듣고 주 예수의 이름으로 세례를 받으니
\par 6 바울이 그들에게 안수하매 성령이 그들에게 임하시므로 방언도 하고 예언도 하니
\par 7 모두 열 두 사람쯤 되니라
\par 8 바울이 회당에 들어가 석 달 동안을 담대히 하나님 나라에 대하여 강론하며 권면하되
\par 9 어떤 사람들은 마음이 굳어 순종치 않고 무리 앞에서 이 도를 비방하거늘 바울이 그들을 떠나 제자들을 따로 세우고 두란노 서원에서 날마다 강론하여
\par 10 이같이 두 해 동안을 하매 아시아에 사는 자는 유대인이나 헬라인이나 다 주의 말씀을 듣더라
\par 11 하나님이 바울의 손으로 희한한 능을 행하게 하시니
\par 12 심지어 사람들이 바울의 몸에서 손수건이나 앞치마를 가져다가 병든 사람에게 얹으면 그 병이 떠나고 악귀도 나가더라
\par 13 이에 돌아다니며 마술하는 어떤 유대인들이 시험적으로 악귀 들린 자들에게 대하여 주 예수의 이름을 불러 말하되 내가 바울의 전파하는 예수를 빙자하여 너희를 명하노라 하더라
\par 14 유대의 한 제사장 스게와의 일곱 아들도 이 일을 행하더니
\par 15 악귀가 대답하여 가로되 예수도 내가 알고 바울도 내가 알거니와 너희는 누구냐 하며
\par 16 악귀 들린 사람이 그 두사람에게 뛰어올라 억제하여 이기니 저희가 상하여 벗은 몸으로 그 집에서 도망하는지라
\par 17 에베소에 거하는 유대인과 헬라인들이 다 이 일을 알고 두려워하며 주 예수의 이름을 높이고
\par 18 믿은 사람들이 많이 와서 자복하여 행한 일을 고하며
\par 19 또 마술을 행하던 많은 사람이 그 책을 모아 가지고 와서 모든 사람 앞에서 불사르니 그 책 값을 계산한즉 은 오만이나 되더라
\par 20 이와 같이 주의 말씀이 힘이 있어 흥왕하여 세력을 얻으니라
\par 21 이 일이 다 된 후 바울이 마게도냐와 아가야로 다녀서 예루살렘에 가기를 경영하여 가로되 내가 거기 갔다가 후에 로마도 보아야 하리라 하고
\par 22 자기를 돕는 사람 중에서 디모데와 에라스도 두 사람을 마게도냐로 보내고 자기는 아시아에 얼마간 더 있으니라
\par 23 그 때쯤 되어 이 도로 인하여 적지 않은 소동이 있었으니
\par 24 즉 데메드리오라 하는 어떤 은장색이 아데미의 은감실을 만들어 직공들로 적지 않은 벌이를 하게 하더니
\par 25 그가 그 직공들과 이러한 영업하는 자들을 모아 이르되 여러분도 알거니와 우리의 유족한 생활이 이 업에 있는데
\par 26 이 바울이 에베소 뿐아니라 거의 아시아 전부를 통하여 허다한 사람을 권유하여 말하되 사람의 손으로 만든 것들은 신이 아니라하니 이는 그대들도 보고 들은 것이라
\par 27 우리의 이 영업만 천하여질 위험이 있을 뿐아니라 큰 여신 아데미의 전각도 경홀히 여김이 되고 온 아시아와 천하가 위하는 그의 위엄도 떨어질까 하노라 하더라
\par 28 저희가 이 말을 듣고 분이 가득하여 외쳐 가로되 크다 에베소 사람의 아데미여 하니
\par 29 온 성이 요란하여 바울과 같이 다니는 마게도냐 사람 가이오와 아리스다고를 잡아가지고 일제히 연극장으로 달려들어 가는지라
\par 30 바울이 백성 가운데로 들어가고자 하나 제자들이 말리고
\par 31 또 아시아 관원 중에 바울의 친구된 어떤이들이 그에게 통지하여 연극장에 들어가지 말라 권하더라
\par 32 사람들이 외쳐 혹은 이 말을 혹은 저 말을 하니 모인 무리가 분란하여 태반이나 어찌하여 모였는지 알지 못하더라
\par 33 유대인들이 무리 가운데서 알렉산더를 권하여 앞으로 밀어내니 알렉산더가 손짓하며 백성에게 발명하려 하나
\par 34 저희는 그가 유대인인줄 알고 다 한 소리로 외쳐 가로되 크다 에베소 사람의 아데미여 하기를 두 시 동안이나 하더니
\par 35 서기장이 무리를 안돈시키고 이르되 에베소 사람들아 에베소 성이 큰 아데미와 및 쓰스에게서 내려온 우상의 전각지기가 된줄을 누가 알지 못하겠느냐
\par 36 이 일이 그렇지 않다 할 수 없으니 너희가 가만히 있어서 무엇이든지 경솔히 아니하여야 하리라
\par 37 전각의 물건을 도적질하지도 아니하였고 우리 여신을 훼방하지도 아니한 이 사람들을 너희가 잡아 왔으니
\par 38 만일 데메드리오와 및 그와 함께 있는 직공들이 누구에게 송사할 것이 있거든 재판 날도 있고 총독들도 있으니 피차 고소할 것이요
\par 39 만일 그 외에 무엇을 원하거든 정식으로 민회에서 결단할지라
\par 40 오늘 아무 까닭도 없는 이 일에 우리가 소요의 사건으로 책망 받을 위험이 있고 우리가 이 불법 집회에 관하여 보고할 재료가 없다 하고
\par 41 이에 그 모임을 흩어지게 하니라

\chapter{20}

\par 1 소요가 그치매 바울이 제자들을 불러 권한 후에 작별하고 떠나 마게도냐로 가니라
\par 2 그 지경으로 다녀가며 여러 말로 제자들에게 권하고 헬라에 이르러
\par 3 거기 석 달을 있다가 배 타고 수리아로 가고자 할 그 때에 유대인들이 자기를 해하려고 공모하므로 마게도냐로 다녀 돌아가기를 작정하니
\par 4 아시아까지 함께 가는 자는 베뢰아 사람 부로의 아들 소바더와 데살로니가 사람 아리스다고와 세군도와 더베 사람 가이오와 및 디모데와 아시아 사람 두기고와 드로비모라
\par 5 그들은 먼저 가서 드로아에서 우리를 기다리더라
\par 6 우리는 무교절 후에 빌립보에서 배로 떠나 닷새만에 드로아에 있는 그들에게 가서 이레를 머무니라
\par 7 안식 후 첫날에 우리가 떡을 떼려 하여 모였더니 바울이 이튿날 떠나고자 하여 저희에게 강론할새 말을 밤중까지 계속하매
\par 8 우리의 모인 윗 다락에 등불을 많이 켰는데
\par 9 유두고라 하는 청년이 창에 걸터 앉았다가 깊이 졸더니 바울이 강론하기를 더 오래 하매 졸음을 이기지 못하여 삼 층 누에서 떨어지거늘 일으켜 보니 죽었는지라
\par 10 바울이 내려가서 그 위에 엎드려 그 몸을 안고 말하되 떠들지 말라 생명이 저에게 있다 하고
\par 11 올라가 떡을 떼어 먹고 오래 동안 곧 날이 새기까지 이야기하고 떠나니라
\par 12 사람들이 살아난 아이를 데리고 와서 위로를 적지 않게 받았더라
\par 13 우리는 앞서 배를 타고 앗소에서 바울을 태우려고 그리로 행선하니 이는 자기가 도보로 가고자 하여 이렇게 정하여 준 것이라
\par 14 바울이 앗소에서 우리를 만나니 우리가 배에 올리고 미둘레네에 가서
\par 15 거기서 떠나 이튿날 기오 앞에 오고 그 이튿날 사모에 들리고 또 그 다음날 밀레도에 이르니라
\par 16 바울이 아시아에서 지체치 않기 위하여 에베소를 지나 행선하기로 작정하였으니 이는 될 수 있는대로 오순절 안에 예루살렘에 이르려고 급히 감이러라
\par 17 바울이 밀레도에서 사람을 에베소로 보내어 교회 장로들을 청하니
\par 18 오매 저희에게 말하되 아시아에 들어온 첫날부터 지금까지 내가 항상 너희 가운데서 어떻게 행한 것을 너희도 아는바니
\par 19 곧 모든 겸손과 눈물이며 유대인의 간계를 인하여 당한 시험을 참고 주를 섬긴 것과
\par 20 유익한 것은 무엇이든지 공중 앞에서나 각 집에서나 꺼림이 없이 너희에게 전하여 가르치고
\par 21 유대인과 헬라인들에게 하나님께 대한 회개와 우리 주 예수 그리스도께 대한 믿음을 증거한 것이라
\par 22 보라 이제 나는 심령에 매임을 받아 예루살렘으로 가는데 저기서 무슨 일을 만날는지 알지 못하노라
\par 23 오직 성령이 각 성에서 내게 증거하여 결박과 환난이 나를 기다린다 하시나
\par 24 나의 달려갈 길과 주 예수께 받은 사명 곧 하나님의 은혜의 복음 증거하는 일을 마치려 함에는 나의 생명을 조금도 귀한 것으로 여기지 아니하노라
\par 25 보라 내가 너희 중에 왕래하며 하나님 나라를 전파하였으나 지금은 너희가 다 내 얼굴을 다시 보지 못할줄 아노라
\par 26 그러므로 오늘 너희에게 증거하노니 모든 사람의 피에 대하여 내가 깨끗하니
\par 27 이는 내가 꺼리지 않고 하나님의 뜻을 다 너희에게 전하였음이라
\par 28 너희는 자기를 위하여 또는 온 양떼를 위하여 삼가라 성령이 저들 가운데 너희로 감독자를 삼고 하나님이 자기 피로 사신 교회를 치게 하셨느니라
\par 29 내가 떠난 후에 흉악한 이리가 너희에게 들어와서 그 양떼를 아끼지 아니하며
\par 30 또한 너희 중에서도 제자들을 끌어 자기를 좇게 하려고 어그러진 말을 하는 사람들이 일어날 줄을 내가 아노니
\par 31 그러므로 너희가 일깨어 내가 삼년이나 밤낮 쉬지 않고 눈물로 각 사람을 훈계하던 것을 기억하라
\par 32 지금 내가 너희를 주와 및 그 은혜의 말씀께 부탁하노니 그 말씀이 너희를 능히 든든히 세우사 거룩케 하심을 입은 모든 자 가운데 기업이 있게 하시리라
\par 33 내가 아무의 은이나 금이나 의복을 탐하지 아니하였고
\par 34 너희 아는 바에 이 손으로 나와 내 동행들의 쓰는 것을 당하여
\par 35 범사에 너희에게 모본을 보였노니 곧 이같이 수고하여 약한 사람들을 돕고 또 주 예수의 친히 말씀하신바 주는 것이 받는 것보다 복이 있다 하심을 기억하여야 할지니라
\par 36 이 말을 한 후 무릎을 꿇고 저희 모든 사람과 함께 기도하니
\par 37 다 크게 울며 바울의 목을 안고 입을 맞추고
\par 38 다시 그 얼굴을 보지 못하리라 한 말을 인하여 더욱 근심하고 배에까지 그를 전송하니라

\chapter{21}

\par 1 우리가 저희를 작별하고 행선하여 바로 고스로 가서 이튿날 로도에 이르러 거기서부터 바다라로 가서
\par 2 베니게로 건너가는 배를 만나서 타고 가다가
\par 3 구브로를 바라보고 이를 왼편에 두고 수리아로 행선하여 두로에서 상륙하니 거기서 배가 짐을 풀려 함이러라
\par 4 제자들을 찾아 거기서 이레를 머물더니 그 제자들이 성령의 감동으로 바울더러 예루살렘에 들어가지 말라 하더라
\par 5 이 여러 날을 지난 후 우리가 떠나갈새 저희가 다 그 처자와 함께 성문 밖까지 전송하거늘 우리가 바닷가에서 무릎을 꿇어 기도하고
\par 6 서로 작별한 후 우리는 배에 오르고 저희는 집으로 돌아가니라
\par 7 두로로부터 수로를 다 행하여 돌레마이에 이르러 형제들에게 안부를 묻고 그들과 함께 하루를 있다가
\par 8 이튿날 떠나 가이사랴에 이르러 일곱 집사 중 하나인 전도자 빌립의 집에 들어가서 유하니라
\par 9 그에게 딸 넷이 있으니 처녀로 예언하는 자라
\par 10 여러 날 있더니 한 선지자 아가보라 하는 이가 유대로부터 내려와
\par 11 우리에게 와서 바울의 띠를 가져다가 자기 수족을 잡아매고 말하기를 성령이 말씀하시되 예루살렘에서 유대인들이 이같이 이 띠 임자를 결박하여 이방인의 손에 넘겨주리라 하거늘
\par 12 우리가 그 말을 듣고 그곳 사람들로 더불어 바울에게 예루살렘으로 올라가지 말라 권하니
\par 13 바울이 대답하되 너희가 어찌하여 울어 내 마음을 상하게 하느냐 나는 주 예수의 이름을 위하여 결박 받을 뿐아니라 예루살렘에서 죽을 것도 각오하였노라 하니
\par 14 저가 권함을 받지 아니하므로 우리가 주의 뜻대로 이루어지이다 하고 그쳤노라
\par 15 이 여러 날 후에 행장을 준비하여 예루살렘으로 올라갈새
\par 16 가이사랴의 몇 제자가 함께 가며 한 오랜 제자 구브로 사람 나손을 데리고 가니 이는 우리가 그의 집에 유하려 함이라
\par 17 예루살렘에 이르니 형제들이 우리를 기꺼이 영접하거늘
\par 18 그 이튿날 바울이 우리와 함께 야고보에게로 들어가니 장로들도 다 있더라
\par 19 바울이 문안하고 하나님이 자기의 봉사로 말미암아 이방 가운데서 하신 일을 낱낱이 고하니
\par 20 저희가 듣고 하나님께 영광을 돌리고 바울더러 이르되 형제여 그대도 보는 바에 유대인 중에 믿는 자 수만명이 있으니 다 율법에 열심 있는 자라
\par 21 네가 이방에 있는 모든 유대인을 가르치되 모세를 배반하고 아들들에게 할례를 하지 말고 또 규모를 지키지 말라 한다 함을 저희가 들었도다
\par 22 그러면 어찌할꼬 저희가 필연 그대의 온 것을 들으리니
\par 23 우리의 말하는 이대로 하라 서원한 네 사람이 우리에게 있으니
\par 24 저희를 데리고 함께 결례를 행하고 저희를 위하여 비용을 내어 머리를 깎게 하라 그러면 모든 사람이 그대에게 대하여 들은 것이 헛된 것이고 그대로 율법을 지켜 행하는 줄로 알 것이라
\par 25 주를 믿는 이방인에게는 우리가 우상의 제물과 피와 목매어 죽인 것과 음행을 피할 것을 결의하고 편지하였느니라 하니
\par 26 바울이 이 사람들을 데리고 이튿날 저희와 함께 결례를 행하고 성전에 들어가서 각 사람을 위하여 제사 드릴 때까지의 결례의 만기 된것을 고하니라
\par 27 그 이레가 거의 차매 아시아로부터 온 유대인들이 성전에서 바울을 보고 모든 무리를 충동하여 그를 붙들고
\par 28 외치되 이스라엘 사람들아 도우라 이 사람은 각처에서 우리 백성과 율법과 이곳을 훼방하여 모든 사람을 가르치는 그 자인데 또 헬라인을 데리고 성전에 들어가서 이 거룩한 곳을 더럽게 하였다 하니
\par 29 이는 저희가 전에 에베소 사람 드로비모가 바울과 함께 성내에 있음을 보고 바울이 저를 성전에 데리고 들어간 줄로 생각함일러라
\par 30 온 성이 소동하여 백성이 달려와 모여 바울을 잡아 성전 밖으로 끌고 나가니 문들이 곧 닫히더라
\par 31 저희가 그를 죽이려 할 때에 온 예루살렘의 요란하다는 소문이 군대의 천부장에게 들리매
\par 32 저가 급히 군사들과 백부장들을 거느리고 달려 내려가니 저희가 천부장과 군사들을 보고 바울 치기를 그치는지라
\par 33 이에 천부장이 가까이 가서 바울을 잡아 두 쇠사슬로 결박하라 명하고 누구며 무슨 일을 하였느냐 물으니
\par 34 무리 가운데서 어떤이는 이 말로 어떤이는 저 말로 부르짖거늘 천부장이 소동을 인하여 그 실상을 알 수 없어 그를 영문 안으로 데려가라 명하니라
\par 35 바울이 층대에 이를 때에 무리의 포행을 인하여 군사들에게 들려가니
\par 36 이는 백성의 무리가 그를 없이 하자고 외치며 따라 감이러라
\par 37 바울을 데리고 영문으로 들어가려 할 그 때에 바울이 천부장더러 이르되 내가 당신에게 말할 수 있느뇨 가로되 네가 헬라 말을 아느냐
\par 38 그러면 네가 이전에 난을 일으켜 사천의 자객을 거느리고 광야로 가던 애굽인이 아니냐
\par 39 바울이 가로되 나는 유대인이라 소읍이 아닌 길리기아 다소성의 시민이니 청컨대 백성에게 말하기를 허락하라 하니
\par 40 천부장이 허락하거늘 바울이 층대 위에 서서 백성에게 손짓하여 크게 종용히 한 후에 히브리 방언으로 말하여 가로되

\chapter{22}

\par 1 부형들아 내가 지금 너희 앞에서 변명하는 말을 들으라 하더라
\par 2 저희가 그 히브리 방언으로 말함을 듣고 더욱 종용한지라 이어 가로되
\par 3 나는 유대인으로 길리기아 다소에서 났고 이 성에서 자라 가말리엘의 문하에서 우리 조상들의 율법의 엄한 교훈을 받았고 오늘 너희 모든 사람처럼 하나님께 대하여 열심하는 자라
\par 4 내가 이 도를 핍박하여 사람을 죽이기까지 하고 남녀를 결박하여 옥에 넘겼노니
\par 5 이에 대제사장과 모든 장로들이 내 증인이라 또 내가 저희에게서 다메섹 형제들에게 가는 공문을 받아 가지고 거기 있는 자들도 결박하여 예루살렘으로 끌어다가 형벌 받게 하려고 가더니
\par 6 가는데 다메섹에 가까왔을 때에 오정쯤 되어 홀연히 하늘로서 큰 빛이 나를 둘러 비취매
\par 7 내가 땅에 엎드러져 들으니 소리 있어 가로되 사울아 사울아 네가 왜 나를 핍박하느냐 하시거늘
\par 8 내가 대답하되 주여 뉘시니이까 하니 가라사대 나는 네가 핍박하는 나사렛 예수라 하시더라
\par 9 나와 함께 있는 사람들이 빛은 보면서도 나더러 말하시는 이의 소리는 듣지 못하더라
\par 10 내가 가로되 주여 무엇을 하리이까 주께서 가라사대 일어나 다멕섹으로 들어가라 정한바 너희 모든 행할 것을 거기서 누가 이르리라 하시거늘
\par 11 나는 그 빛의 광채를 인하여 볼 수 없게 되었으므로 나와 함께 있는 사람들의 손에 끌려 다메섹에 들어갔노라
\par 12 율법에 의하면 경건한 사람으로 거기 사는 모든 유대인들에게 칭찬을 듣는 아나니아라 하는 이가
\par 13 내게 와 곁에 서서 말하되 형제 사울아 다시 보라 하거늘 즉시 그를 쳐다보았노라
\par 14 그가 또 가로되 우리 조상들의 하나님이 너를 택하여 너로 하여금 자기 뜻을 알게 하시며 저 의인을 보게 하시고 그 입에서 나오는 음성을 듣게 하셨으니
\par 15 네가 그를 위하여 모든 사람 앞에서 너의 보고 들은 것에 증인이 되리라
\par 16 이제는 왜 주저하느뇨 일어나 주의 이름을 불러 세례를 받고 너의 죄를 씻으라 하더라
\par 17 후에 내가 예루살렘으로 돌아와서 성전에서 기도할 때에 비몽사몽간에
\par 18 보매 주께서 내게 말씀하시되 속히 예루살렘에서 나가라 저희는 네가 내게 대하여 증거하는 말을 듣지 아니하리라 하시거늘
\par 19 내가 말하기를 주여 내가 주 믿는 사람들을 가두고 또 각 회당에서 때리고
\par 20 또 주의 증인 스데반의 피를 흘릴 적에 내가 곁에 서서 찬성하고 그 죽이는 사람들의 옷을 지킨줄 저희도 아나이다
\par 21 나더러 또 이르시되 떠나가라 내가 너를 멀리 이방인에게로 보내리라 하셨느니라
\par 22 이 말 하는 것까지 저희가 듣다가 소리질러 가로되 이러한 놈은 세상에서 없이 하자 살려 둘 자가 아니라 하여
\par 23 떠들며 옷을 벗어 던지고 티끌을 공중에 날리니
\par 24 천부장이 바울을 영문 안으로 데려가라 명하고 저희가 무슨 일로 그를 대하여 떠드나 알고자 하여 채찍질하며 신문하라 한대
\par 25 가죽줄로 바울을 매니 바울이 곁에 섰는 백부장더러 이르되 너희가 로마 사람 된 자를 죄도 정치 아니하고 채찍질할 수 있느냐 하니
\par 26 백부장이 듣고 가서 천부장에게 전하여 가로되 어찌하려 하느뇨 이는 로마사람이라 하니
\par 27 천부장이 와서 바울에게 말하되 네가 로마 사람이냐 내게 말하라 가로되 그러하다
\par 28 천부장이 대답하되 나는 돈을 많이 들여 이 시민권을 얻었노라 바울이 가로되 나는 나면서부터로라 하니
\par 29 신문하려던 사람들이 곧 그에게서 물러가고 천부장도 그가 로마 사람인줄 알고 또는 그 결박한 것을 인하여 두려워하니라
\par 30 이튿날 천부장이 무슨 일로 유대인들이 그를 송사하는지 실상을 알고자 하여 그 결박을 풀고 명하여 제사장들과 온 공회를 모으고 바울을 데리고 내려가서 저희 앞에 세우니라

\chapter{23}

\par 1 바울이 공회를 주목하여 가로되 여러분 형제들아 오늘날까지 내가 범사에 양심을 따라 하나님을 섬겼노라 하거늘
\par 2 대제사장 아나니아가 바울 곁에 섰는 사람들에게 그 입을 치라 명하니
\par 3 바울이 가로되 회칠한 담이여 하나님이 너를 치시리로다 네가 나를 율법대로 판단한다고 앉아서 율법을 어기고 나를 치라 하느냐 하니
\par 4 곁에 선 사람들이 말하되 하나님의 대제사장을 네가 욕하느냐
\par 5 바울이 가로되 형제들아 나는 그가 대제사장인줄 알지 못하였노라 기록하였으되 너희 백성의 관원을 비방치 말라 하였느니라 하더라
\par 6 바울이 그 한 부분은 사두개인이요 한 부분은 바리새인인줄 알고 공회에서 외쳐 가로되 여러분 형제들아 나는 바리새인이요 또 바리새인의 아들이라 죽은 자의 소망 곧 부활을 인하여 내가 심문을 받노라
\par 7 그 말을 한즉 바리새인과 사두개인 사이에 다툼이 생겨 무리가 나누이니
\par 8 이는 사두개인은 부활도 없고 천사도 없고 영도 없다 하고 바리새인은 다 있다 함이라
\par 9 크게 훤화가 일어날새 바리새인 편에서 몇 서기관이 일어나 다투어 가로되 우리가 이 사람을 보매 악한 것이 없도다 혹 영이나 혹 천사가 저더러 말하였으면 어찌 하겠느뇨 하여
\par 10 큰 분쟁이 생기니 천부장이 바울이 저희에게 찢겨질까 하여 군사를 명하여 내려가 무리 가운데서 빼앗아 가지고 영문으로 들어가라 하니라
\par 11 그날 밤에 주께서 바울 곁에 서서 이르시되 담대하라 네가 예루살렘에서 나의 일을 증거한 것 같이 로마에서도 증거하여야 하리라 하시니라
\par 12 날이 새매 유대인들이 당을 지어 맹세하되 바울을 죽이기 전에는 먹지도 아니하고 마시지도 아니하겠다 하고
\par 13 이같이 동맹한 자가 사십여명이더라
\par 14 대제사장들과 장로들에게 가서 말하되 우리가 바울을 죽이기 전에는 아무 것도 먹지 않기로 굳게 맹세하였으니
\par 15 이제 너희는 그의 사실을 더 자세히 알아볼 양으로 공회와 함께 천부장에게 청하여 바울을 너희에게로 데리고 내려오게 하라 우리는 그가 가까이 오기 전에 죽이기로 준비하였노라 하더니
\par 16 바울의 생질이 그들이 매복하여 있다 함을 듣고 와서 영문에 들어가 바울에게 고한지라
\par 17 바울이 한 백부장을 청하여 가로되 이 청년을 천부장에게로 인도하라 그에게 무슨 할 말이 있다 하니
\par 18 천부장에게로 데리고 가서 가로되 죄수 바울이 나를 불러 이 청년이 당신께 할 말이 있다 하여 데리고 가기를 청하더이다 하매
\par 19 천부장이 그 손을 잡고 물러가서 종용히 묻되 내게 할 말이 무엇이냐
\par 20 대답하되 유대인들이 공모하기를 저희들이 바울에 대하여 더 자세한 것을 묻기 위함이라 하고 내일 그를 데리고 공회로 내려오기를 당신께 청하자 하였으니
\par 21 당신은 저희 청함을 좇지 마옵소서 저희 중에서 바울을 죽이기 전에는 먹지도 않고 마시지도 않기로 맹세한 자 사십 여명이 그를 죽이려고 숨어서 지금 다 준비하고 당신의 허락만 기다리나이다 하매
\par 22 이에 천부장이 청년을 보내며 경계하되 이 일을 내게 고하였다고 아무에게도 이르지 말라 하고
\par 23 백부장 둘을 불러 이르되 밤 제삼시에 가이사랴까지 갈 보병 이백명과 마병 칠십명과 창군 이백명을 준비하라 하고
\par 24 또 바울을 태워 총독 벨릭스에게로 무사히 보내기 위하여 짐승을 준비하라 명하며
\par 25 또 이 아래와 같이 편지하니 일렀으되
\par 26 글라우디오 루시아는 총독 벨릭스 각하에게 문안하노이다
\par 27 이 사람이 유대인들에게 잡혀 죽게 된 것을 내가 로마 사람인줄 들어 알고 군사를 거느리고 가서 구원하여다가
\par 28 유대인들이 무슨 일로 그를 송사하는지 알고자 하여 저희 공회로 데리고 내려갔더니
\par 29 송사하는 것이 저희 율법 문제에 관한 것뿐이요 한 가지도 죽이거나 결박할 사건이 없음을 발견하였나이다
\par 30 그러나 이 사람을 해하려는 간계가 있다고 누가 내게 알게 하기로 곧 당신께로 보내며 또 송사하는 사람들도 당신 앞에서 그를 대하여 말하라 하였나이다 하였더라
\par 31 보병이 명을 받은 대로 밤에 바울을 데리고 안디바드리에 이르러
\par 32 이튿날 마병으로 바울을 호송하게 하고 영문으로 돌아 가니라
\par 33 저희가 가이사랴에 들어가서 편지를 총독에게 드리고 바울을 그 앞에 세우니
\par 34 총독이 읽고 바울더러 어느 영지 사람이냐 물어 길리기아 사람인줄 알고
\par 35 가로되 너를 송사하는 사람들이 오거든 네 말을 들으리라 하고 헤롯궁에 그를 지키라 명하니라

\chapter{24}

\par 1 닷새 후에 대제사장 아나니아가 어떤 장로들과 한 변사 더둘로와 함께 내려와서 총독 앞에서 바울을 고소하니라
\par 2 바울을 부르매 더둘로가 송사하여 가로되
\par 3 벨릭스 각하여 우리가 당신을 힘입어 태평을 누리고 또 이 민족이 당신의 선견을 인하여 여러 가지로 개량된 것을 우리가 어느 모양으로나 어느 곳에서나 감사 무지하옵나이다
\par 4 당신을 더 괴롭게 아니하려 하여 우리가 대강 여짜옵나니 관용하여 들으시기를 원하나이다
\par 5 우리가 보니 이 사람은 염병이라 천하에 퍼진 유대인을 다 소요케 하는 자요 나사렛 이단의 괴수라
\par 6 저가 또 성전을 더럽게 하려 하므로 우리가 잡았사오니
\par 7 당신이 친히 그를 심문하시면
\par 8 우리의 송사하는 이 모든 일을 아실 수 있나이다 하니
\par 9 유대인들도 이에 참가하여 이 말이 옳다 주장하니라
\par 10 총독이 바울에게 머리로 표시하여 말하라 하니 그가 대답하되 당신이 여러 해 전부터 이 민족의 재판장 된 것을 내가 알고 내 사건에 대하여 기쁘게 변명하나이다
\par 11 당신이 아실 수 있는 바와 같이 내가 예루살렘에 예배하러 올라간지 열 이틀 밖에 못되었고
\par 12 저희는 내가 성전에서 아무와 변론하는 것이나 회당과 또는 성중에서 무리를 소동케 하는 것을 보지 못하였으니
\par 13 이제 나를 송사하는 모든 일에 대하여 저희가 능히 당신 앞에 내세울 것이 없나이다
\par 14 그러나 이것을 당신께 고백하리이다 나는 저희가 이단이라 하는 도를 좇아 조상의 하나님을 섬기고 율법과 및 선지자들의 글에 기록된 것을 다 믿으며
\par 15 저희의 기다리는바 하나님께 향한 소망을 나도 가졌으니 곧 의인과 악인의 부활이 있으리라 함이라
\par 16 이것을 인하여 나도 하나님과 사람을 대하여 항상 양심에 거리낌이 없기를 힘쓰노라
\par 17 여러 해 만에 내가 내 민족을 구제할 것과 제물을 가지고 와서
\par 18 드리는 중에 내가 결례를 행하였고 모임도 없고 소동도 없이 성전에 있는 것을 저희가 보았나이다 그러나 아시아로부터 온 어떤 유대인들이 있었으니
\par 19 저희가 만일 나를 반대할 사건이 있으면 마땅히 당신 앞에 와서 송사하였을 것이요
\par 20 그렇지 않으면 이 사람들이 내가 공회 앞에 섰을 때에 무슨 옳지 않은 것을 보았는가 말하라 하소서
\par 21 오직 내가 저희 가운데 서서 외치기를 내가 죽은 자의 부활에 대하여 오늘 너희 앞에 심문을 받는다고 한 이 한 소리가 있을 따름이니이다 하니
\par 22 벨릭스가 이 도에 관한 것을 더 자세히 아는고로 연기하여 가로되 천부장 루시아가 내려 오거든 너희 일을 처결하리라 하고
\par 23 백부장을 명하여 바울을 지키되 자유를 주며 친구 중 아무나 수종하는 것을 금치 말라 하니라
\par 24 수일 후에 벨릭스가 그 아내 유대 여자 드루실라와 함께 와서 바울을 불러 그리스도 예수 믿는 도를 듣거늘
\par 25 바울이 의와 절제와 장차 오는 심판을 강론하니 벨릭스가 두려워하여 대답하되 시방은 가라 내가 틈이 있으면 너를 부르리라 하고
\par 26 동시에 또 바울에게서 돈을 받을까 바라는고로 더 자주 불러 같이 이야기하더라
\par 27 이태를 지내서 보르기오 베스도가 벨릭스의 소임을 대신하니 벨릭스가 유대인의 마음을 얻고자 하여 바울을 구류하여 두니라

\chapter{25}

\par 1 베스도가 도임한지 삼일 후에 가이사랴에서 예루살렘으로 올라가니
\par 2 대제사장들과 유대인 중 높은 사람들이 바울을 고소할새
\par 3 베스도의 호의로 바울을 예루살렘으로 옮겨 보내기를 청하니 이는 길에 매복하였다가 그를 죽이고자 함이러라
\par 4 베스도가 대답하여 바울이 가이사랴에 구류된 것과 자기도 미구에 떠나갈 것을 말하고
\par 5 또 가로되 너희 중 유력한 자들은 나아 함께 내려가서 그 사람에게 만일 옳지 아니한 일이 있거든 송사하라 하니라
\par 6 베스도가 그들 가운데서 팔일 혹 십일을 지낸 후 가이사랴로 내려가서 이튿날 재판 자리에 앉고 바울을 데려오라 명하니
\par 7 그가 나오매 예루살렘에서 내려온 유대인들이 둘러 서서 여러가지 중대한 사건으로 송사하되 능히 증명하지 못한지라
\par 8 바울이 변명하여 가로되 유대인의 율법이나 성전이나 가이사에게나 내가 도무지 죄를 범하지 아니하였노라 하니
\par 9 베스도가 유대인의 마음을 얻고자하여 바울더러 묻되 네가 예루살렘에 올라가서 이 사건에 대하여 내 앞에서 심문을 받으려느냐
\par 10 바울이 가로되 내가 가이사의 재판 자리 앞에 섰으니 마땅히 거기서 심문을 받을 것이라 당신도 잘 아시는 바에 내가 유대인들에게 불의를 행한 일이 없나이다
\par 11 만일 내가 불의를 행하여 무슨 사죄를 범하였으면 죽기를 사양치아니할 것이나 만일 이 사람들의 나를 송사하는 것이 다 사실이 아니면 누구든지 나를 그들에게 내어 줄수 없삽나이다 내가 가이사께 호소하노라 한대
\par 12 베스도가 배석자들과 상의하고 가로되 네가 가이사에게 호소하였으니 가이사에게 갈 것이라 하니라
\par 13 수일 후에 아그립바왕과 버니게가 베스도에게 문안하러 가이샤랴에 와서
\par 14 여러 날을 있더니 베스도가 바울의 일로 왕에게 고하여 가로되 벨릭스가 한 사람을 구류하여 두었는데
\par 15 내가 예루살렘에 있을 때에 유대인의 대제사장들과 장로들이 그를 고소하여 정죄하기를 청하기에
\par 16 내가 대답하되 무릇 피고가 원고들 앞에서 고소 사건에 대하여 변명할 기회가 있기 전에 내어주는 것이 로마 사람의 법이 아니라 하였노라
\par 17 그러므로 저희가 나와 함께 여기 오매 내가 지체하지 아니하고 이튿날 재판 자리에 앉아 명하여 그 사람을 데려 왔으나
\par 18 원고들이 서서 나의 짐작하던 것 같은 악행의 사건은 하나도 제출치 아니하고
\par 19 오직 자기들의 종교와 또는 예수라 하는 이의 죽은 것을 살았다고 바울이 주장하는 그 일에 관한 문제로 송사하는 것뿐이라
\par 20 내가 이 일을 어떻게 사실할는지 의심이 있어서 바울에게 묻되 예루살렘에 올라가서 이 일에 심문을 받으려느냐 한즉
\par 21 바울은 황제의 판결을 받도록 자기를 지켜 주기를 호소하므로 내가 그를 가이사에게 보내기까지 지켜두라 명하였노라 하니
\par 22 아그립바가 베스도더러 이르되 나도 이 사람의 말을 듣고자 하노라 베스도가 가로되 내일 들으시리이다 하더라
\par 23 이튿날 아그립바와 버니게가 크게 위의를 베풀고 와서 천부장들과 성중의 높은 사람들과 함께 신문소에 들어오고 베스도의 명으로 바울을 데려오니
\par 24 베스도가 말하되 아그립바왕과 여기 같이 있는 여러분이여 당신들이 보는 이 사람은 유대의 모든 무리가 크게 외치되 살려 두지 못할 사람이라고 하여 예루살렘에서와 여기서도 내게 청원하였으나
\par 25 나는 살피건대 죽일 죄를 범한 일이 없더이다 그러나 저가 황제에게 호소한고로 보내기를 작정하였나이다
\par 26 그에게 대하여 황제께 확실한 사실을 아뢸 것이 없으므로 심문한 후 상소할 재료가 있을까 하여 당신들 앞 특히 아그립바왕 당신앞에 그를 내어 세웠나이다
\par 27 그 죄목을 베풀지 아니하고 죄수를 보내는 것이 무리한 일인줄 아나이다 하였더라

\chapter{26}

\par 1 아그립바가 바울더러 이르되 너를 위하여 말하기를 네게 허락하노라 하니 이에 바울이 손을 들어 변명하되
\par 2 아그립바왕이여 유대인이 모든 송사하는 일을 오늘 당신 앞에서 변명하게 된 것을 다행히 여기옵나이다
\par 3 특히 당신이 유대인의 모든 풍속과 및 문제를 아심이니이다 그러므로 내 말을 너그러이 들으시기를 바라옵나이다
\par 4 내가 처음부터 내 민족 중에와 예루살렘에서 젊었을 때 생활한 상태를 유대인이 다 아는바라
\par 5 일찍부터 나를 알았으니 저희가 증거하려 하면 내가 우리 종교의 가장 엄한 파를 좇아 바리새인의 생활을 하였다고 할 것이라
\par 6 이제도 여기 서서 심문 받는 것은 하나님이 우리 조상에게 약속하신 것을 바라는 까닭이니
\par 7 이 약속은 우리 열 두 지파가 밤낮으로 간절히 하나님을 받들어 섬김으로 얻기를 바라는 바인데 아그립바왕이여 이 소망을 인하여 내가 유대인들에게 송사를 받는 것이니이다
\par 8 당신들은 하나님이 죽은 사람 다시 살리심을 어찌하여 못 믿을 것으로 여기나이까
\par 9 나도 나사렛 예수의 이름을 대적하여 범사를 행하여야 될줄 스스로 생각하고
\par 10 예루살렘에서 이런 일을 행하여 대제사장들에게서 권세를 얻어 가지고 많은 성도를 옥에 가두며 또 죽일 때에 내가 가편 투표를 하였고
\par 11 또 모든 회당에서 여러번 형벌하여 강제로 모독하는 말을 하게하고 저희를 대하여 심히 격분하여 외국 성까지도 가서 핍박하였고
\par 12 그 일로 대제사장들의 권세와 위임을 받고 다메섹으로 갔나이다
\par 13 왕이여 때가 정오나 되어 길에서 보니 하늘로서 해보다 더 밝은 빛이 나와 내 동행들을 둘러 비추는지라
\par 14 우리가 다 땅에 엎드러지매 내가 소리를 들으니 히브리 방언으로 이르되 사울아 사울아 네가 어찌하여 나를 핍박하느냐 가시채를 뒷발질하기가 네게 고생이니라
\par 15 내가 대답하되 주여 뉘시니이까 주께서 가라사대 나는 네가 핍박하는 예수라
\par 16 일어나 네 발로 서라 내가 네게 나타난 것은 곧 네가 나를 본 일과 장차 내가 네게 나타날 일에 너로 사환과 증인을 삼으려 함이니
\par 17 이스라엘과 이방인들에게서 내가 너를 구원하여 저희에게 보내어
\par 18 그 눈을 뜨게 하여 어두움에서 빛으로 사단의 권세에서 하나님께로 돌아가게 하고 죄 사함과 나를 믿어 거룩케 된 무리 가운데서 기업을 얻게 하리라 하더이다
\par 19 아그립바 왕이여 그러므로 하늘에서 보이신 것을 내가 거스리지 아니하고
\par 20 먼저 다메섹에와 또 예루살렘에 있는 사람과 유대 온 땅과 이방인에게까지 회개하고 하나님께로 돌아가서 회개에 합당한 일을 행하라 선전하므로
\par 21 유대인들이 성전에서 나를 잡아 죽이고자 하였으나
\par 22 하나님의 도우심을 받아 내가 오늘까지 서서 높고 낮은 사람 앞에서 증거하는 것은 선지자들과 모세가 반드시 되리라고 말한 것 밖에 없으니
\par 23 곧 그리스도가 고난을 받으실 것과 죽은 자 가운데서 먼저 다시 살아나사 이스라엘과 이방인들에게 빛을 선전하시리라 함이니이다 하니라
\par 24 바울이 이같이 변명하매 베스도가 크게 소리하여 가로되 바울아 네가 미쳤도다 네 많은 학문이 너를 미치게 한다 하니
\par 25 바울이 가로되 베스도 각하여 내가 미친 것이 아니요 참되고 정신차린 말을 하나이다
\par 26 왕께서는 이 일을 아시기로 내가 왕께 담대히 말하노니 이 일에 하나라도 아시지 못함이 없는줄 믿나이다 이 일은 한편 구석에서 행한 것이 아니로소이다
\par 27 아그립바왕이여 선지자를 믿으시나이까 믿으시는 줄 아나이다
\par 28 아그립바가 바울더러 이르되 네가 적은 말로 나를 권하여 그리스도인이 되게 하려 하는도다
\par 29 바울이 가로되 말이 적으나 많으나 당신 뿐아니라 오늘 네 말을 듣는 모든 사람도 다 이렇게 결박한 것 외에는 나와 같이 되기를 하나님께 원하노이다 하니라
\par 30 왕과 총독과 버니게와 그 함께 앉은 사람들이 다 일어나서
\par 31 물러가 서로 말하되 이 사람은 사형이나 결박을 당할만한 행사가 없다 하더라
\par 32 이에 아그립바가 베스도더러 일러 가로되 이 사람이 만일 가이사에게 호소하지 아니하였더면 놓을 수 있을뻔하였다 하니라

\chapter{27}

\par 1 우리의 배 타고 이달리야로 갈 일이 작정되매 바울과 다른 죄수 몇 사람을 아구사도대의 백부장 율리오란 사람에게 맡기니
\par 2 아시아 해변 각처로 가려 하는 아드라뭇데노 배에 우리가 올라 행선할새 마게도냐의 데살로니가 사람 아리스다고도 함께 하니라
\par 3 이튿날 시돈에 대니 율리오가 바울을 친절히 하여 친구들에게 가서 대접 받음을 허락하더니
\par 4 또 거기서 우리가 떠나가다가 바람의 거스림을 피하여 구브로 해안을 의지하고 행선하여
\par 5 길리기아와 밤빌리아 바다를 건너 루기아의 무라성에 이르러
\par 6 거기서 백부장이 이달리야로 가려하는 알렉산드리아 배를 만나 우리를 오르게 하니
\par 7 배가 더디 가 여러 날만에 간신히 니도 맞은편에 이르러 풍세가 더 허락지 아니하므로 살모네 앞을 지나 그레데 해안을 의지하고 행선하여
\par 8 간신히 그 연안을 지나 미항이라는 곳에 이르니 라새아성에서 가깝더라
\par 9 여러 날이 걸려 금식하는 절기가 이미 지났으므로 행선하기가 위태한지라 바울이 저희를 권하여
\par 10 말하되 여러분이여 내가 보니 이번 행선이 하물과 배만 아니라 우리 생명에도 타격과 많은 손해가 있으리라 하되
\par 11 백부장이 선장과 선주의 말을 바울의 말보다 더 믿더라
\par 12 "그 항구가 과동하기에 불편하므로 거기서 떠나 아무쪼록 뵈닉스에 가서 과동하자 하는 자가 더 많으니 뵈닉스는 그레데 항구라 한편은 동북을, 한편은 동남을 향하였더라"
\par 13 남풍이 순하게 불매 저희가 득의한줄 알고 닻을 감아 그레데 해변을 가까이 하고 행선하더니
\par 14 얼마 못되어 섬 가운데로서 유라굴로라는 광풍이 대작하니
\par 15 배가 밀려 바람을 맞추어 갈 수 없어 가는 대로 두고 쫓겨 가다가
\par 16 가우다라는 작은 섬 아래로 지나 간신히 거루를 잡아
\par 17 끌어 올리고 줄을 가지고 선체를 둘러 감고 스르디스에 걸릴까 두려워 연장을 내리고 그냥 쫓겨가더니
\par 18 우리가 풍랑으로 심히 애쓰다가 이튿날 사공들이 짐을 바다에 풀어 버리고
\par 19 사흘째 되는 날에 배의 기구를 저희 손으로 내어 버리니라
\par 20 여러 날 동안 해와 별이 보이지 아니하고 큰 풍랑이 그대로 있으매 구원의 여망이 다 없어졌더라
\par 21 여러 사람이 오래 먹지 못하였으매 바울이 가운데 서서 말하되 여러분이여 내 말을 듣고 그레데에서 떠나지 아니하여 이 타격과 손상을 면하였더면 좋을뻔 하였느니라
\par 22 내가 너희를 권하노니 이제는 안심하라 너희 중 생명에는 아무 손상이 없겠고 오직 배 뿐이리라
\par 23 나의 속한바 곧 나의 섬기는 하나님의 사자가 어제 밤에 내 곁에 서서 말하되
\par 24 바울아 두려워 말라 네가 가이사 앞에 서야 하겠고 또 하나님께서 너와 함께 행선하는 자를 다 네게 주셨다 하였으니
\par 25 그러므로 여러분이여 안심하라 나는 내게 말씀하신 그대로 되리라고 하나님을 믿노라
\par 26 그러나 우리가 한 섬에 걸리리라 하더라
\par 27 열 나흘째 되는 날 밤에 우리가 아드리아 바다에 이리 저리 쫓겨 가더니 밤중쯤 되어 사공들이 어느 육지에 가까와지는 줄을 짐작하고
\par 28 물을 재어보니 이십 길이 되고 조금 가다가 다시 재니 열다섯 길이라
\par 29 암초에 걸릴까 하여 고물로 닻 넷을 주고 날이 새기를 고대하더니
\par 30 사공들이 도망하고자 하여 이물에서 닻을 주려는체하고 거루를 바다에 내려 놓거늘
\par 31 바울이 백부장과 군사들에게 이르되 이 사람들이 배에 있지 아니하면 너희가 구원을 얻지 못하리라 하니
\par 32 이에 군사들이 거룻줄을 끊어 떼어 버리니라
\par 33 날이 새어가매 바울이 여러 사람을 음식 먹으라 권하여 가로되 너희가 기다리고 기다리며 먹지 못하고 주린 지가 오늘까지 열 나흘인즉
\par 34 음식 먹으라 권하노니 이것이 너희 구원을 위하는 것이요 너희중 머리터럭 하나라도 잃을 자가 없느니라 하고
\par 35 떡을 가져다가 모든 사람 앞에서 하나님께 축사하고 떼어 먹기를 시작하매
\par 36 저희도 다 안심하고 받아 먹으니
\par 37 배에 있는 우리의 수는 전부 이백 칠십 륙인이러라
\par 38 배부르게 먹고 밀을 바다에 버려 배를 가볍게 하였더니
\par 39 날이 새매 어느 땅인지 알지 못하나 경사진 해안으로 된 항만이 눈에 띄거늘 배를 거기에 들여다 댈 수 있는가 의논한 후
\par 40 닻을 끊어 바다에 버리는 동시에 킷줄을 늦추고 돛을 달고 바람을 맞추어 해안을 향하여 들어가다가
\par 41 두 물이 합하여 흐르는 곳을 당하여 배를 걸매 이물은 부딪혀 움직일 수 없이 붙고 고물은 큰 물결에 깨어져가니
\par 42 군사들은 죄수가 헤엄쳐서 도망할까 하여 저희를 죽이는 것이 좋다 하였으나
\par 43 백부장이 바울을 구원하려 하여 저희의 뜻을 막고 헤엄칠줄 아는 사람들을 명하여 물에 뛰어 내려 먼저 육지에 나가게 하고
\par 44 그 남은 사람들은 널조각 혹은 배 물건에 의지하여 나가게 하니 마침내 사람들이 다 상륙하여 구원을 얻으니라

\chapter{28}

\par 1 우리가 구원을 얻은 후에 안즉 그 섬은 멜리데라 하더라
\par 2 토인들이 우리에게 특별한 동정을 하여 비가 오고 날이 차매 불을 피워 우리를 다 영접하더라
\par 3 바울이 한뭇 나무를 거두어 불에 넣으니 뜨거움을 인하여 독사가 나와 그 손을 물고 있는지라
\par 4 토인들이 이 짐승이 그 손에 달림을 보고 서로 말하되 진실로 이사람은 살인한 자로다 바다에서는 구원을 얻었으나 공의가 살지 못하게 하심이로다 하더니
\par 5 바울이 그 짐승을 불에 떨어버리매 조금도 상함이 없더라
\par 6 그가 붓든지 혹 갑자기 엎드러져 죽을 줄로 저희가 기다렸더니 오래 기다려도 그에게 아무 이상이 없음을 보고 돌려 생각하여 말하되 신이라 하더라
\par 7 이 섬에 제일 높은 사람 보블리오라 하는 이가 그 근처에 토지가 있는지라 그가 우리를 영접하여 사흘이나 친절히 유숙하게 하더니
\par 8 보블리오의 부친이 열병과 이질에 걸려 누웠거늘 바울이 들어가서 기도하고 그에게 안수하여 낫게 하매
\par 9 이러므로 섬 가운데 다른 병든 사람들이 와서 고침을 받고
\par 10 후한 예로 우리를 대접하고 떠날 때에 우리 쓸 것을 배에 올리더라
\par 11 석 달 후에 그 섬에서 과동한 알렉산드리아 배를 우리가 타고 떠나니 그 배 기호는 디오스구로라
\par 12 수라구사에 대고 사흘을 있다가
\par 13 거기서 둘러가서 레기온에 이르러 하루를 지난 후 남풍이 일어나므로 이튿날 보디올에 이르러
\par 14 거기서 형제를 만나 저희의 청함을 받아 이레를 함께 유하다가 로마로 가니라
\par 15 거기 형제들이 우리 소식을 듣고 압비오 저자와 삼관까지 맞으러 오니 바울이 저희를 보고 하나님께 사례하고 담대한 마음을 얻으니라
\par 16 우리가 로마에 들어가니 바울은 자기를 지키는 한 군사와 함께 따로 있게 허락하더라
\par 17 사흘 후에 바울이 유대인 중 높은 사람들을 청하여 모인 후에 이르되 여러분 형제들아 내가 이스라엘 백성이나 우리 조상의 규모를 배척한 일이 없는데 예루살렘에서 로마인의 손에 죄수로 내어준 바 되었으니
\par 18 로마인은 나를 심문하여 죽일 죄목이 없으므로 놓으려 하였으나
\par 19 유대인들이 반대하기로 내가 마지못하여 가이사에게 호소함이요 내 민족을 송사하려는 것이 아니로라
\par 20 이러하므로 너희를 보고 함께 이야기하려고 청하였노니 이스라엘의 소망을 인하여 내가 이 쇠사슬에 매인바 되었노라
\par 21 저희가 가로되 우리가 유대에서 네게 대한 편지도 받은 일이 없고 또 형제 중 누가 와서 네게 대하여 좋지 못한 것을 고하든지 이야기한 일도 없느니라
\par 22 이에 우리가 너의 사상이 어떠한가 듣고자 하노니 이 파에 대하여는 어디서든지 반대를 받는줄 우리가 앎이라 하더라
\par 23 저희가 일자를 정하고 그의 우거하는 집에 많이 오니 바울이 아침부터 저녁까지 강론하여 하나님 나라를 증거하고 모세의 율법과 선지자의 말을 가지고 예수의 일로 권하더라
\par 24 그 말을 믿는 사람도 있고 믿지 아니하는 사람도 있어
\par 25 서로 맞지 아니하여 흩어질 때에 바울이 한 말로 일러 가로되 성령이 선지자 이사야로 너희 조상들에게 말씀하신 것이 옳도다
\par 26 일렀으되 이 백성에게 가서 말하기를 너희가 듣기는 들어도 도무지 깨닫지 못하며 보기는 보아도 도무지 알지 못하는도다
\par 27 이 백성들의 마음이 완악하여져서 그 귀로는 둔하게 듣고 그 눈을 감았으니 이는 눈으로 보고 귀로 듣고 마음으로 깨달아 돌아와 나의 고침을 받을까 함이라 하였으니
\par 28 그런즉 하나님의 이 구원을 이방인에게로 보내신줄 알라 저희는 또한 들으리라 하더라
\par 29 (없 음1
\par 30 바울이 온 이태를 자기 셋집에 유하며 자기에게 오는 사람을 다 영접하고
\par 31 담대히 하나님 나라를 전파하며 주 예수 그리스도께 관한 것을 가르치되 금하는 사람이 없었더라


\end{document}