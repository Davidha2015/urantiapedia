\begin{document}

\title{갈라디아서}


\chapter{1}

\par 1 사람들에게서 난 것도 아니요 사람으로 말미암은 것도 아니요 오직 예수 그리스도와 및 죽은 자 가운데서 그리스도를 살리신 하나님 아버지로 말미암아 사도된 바울은
\par 2 함께 있는 모든 형제로 더불어 갈라디아 여러 교회들에게
\par 3 우리 하나님 아버지와 주 예수 그리스도로 좇아 은혜와 평강이 있기를 원하노라
\par 4 그리스도께서 하나님 곧 우리 아버지의 뜻을 따라 이 악한 세대에서 우리를 건지시려고 우리 죄를 위하여 자기 몸을 드리셨으니
\par 5 영광이 저에게 세세토록 있을지어다 아멘
\par 6 그리스도의 은혜로 너희를 부르신 이를 이같이 속히 떠나 다른 복음 좇는 것을 내가 이상히 여기노라
\par 7 다른 복음은 없나니 다만 어떤 사람들이 너희를 요란케 하여 그리스도의 복음을 변하려 함이라
\par 8 그러나 우리나 혹 하늘로부터 온 천사라도 우리가 너희에게 전한 복음 외에 다른 복음을 전하면 저주를 받을지어다
\par 9 우리가 전에 말하였거니와 내가 지금 다시 말하노니 만일 누구든지 너희의 받은 것 외에 다른 복음을 전하면 저주를 받을지어다
\par 10 이제 내가 사람들에게 좋게 하랴 하나님께 좋게 하랴 사람들에게 기쁨을 구하랴 내가 지금까지 사람의 기쁨을 구하는 것이었다면 그리스도의 종이 아니니라
\par 11 형제들아 내가 너희에게 알게 하노니 내가 전한 복음이 사람의 뜻을 따라 된 것이 아니라
\par 12 이는 내가 사람에게서 받은 것도 아니요 배운 것도 아니요 오직 예수 그리스도의 계시로 말미암은 것이라
\par 13 내가 이전에 유대교에 있을 때에 행한 일을 너희가 들었거니와 하나님의 교회를 심히 핍박하여 잔해하고
\par 14 내가 내 동족 중 여러 연갑자보다 유대교를 지나치게 믿어 내 조상의 유전에 대하여 더욱 열심이 있었으나
\par 15 그러나 내 어머니의 태로부터 나를 택정하시고 은혜로 나를 부르신 이가
\par 16 그 아들을 이방에 전하기 위하여 그를 내 속에 나타내시기를 기뻐하실 때에 내가 곧 혈육과 의논하지 아니하고
\par 17 또 나보다 먼저 사도 된 자들을 만나려고 예루살렘으로 가지 아니하고 오직 아라비아로 갔다가 다시 다메섹으로 돌아갔노라
\par 18 그 후 삼년만에 내가 게바를 심방하려고 예루살렘에 올라가서 저와 함께 십 오일을 유할새
\par 19 주의 형제 야고보 외에 다른 사도들을 보지 못하였노라
\par 20 보라 내가 너희에게 쓰는 것은 하나님 앞에서 거짓말이 아니로라
\par 21 그 후에 내가 수리아와 길리기아 지방에 이르렀으나
\par 22 유대에 그리스도 안에 있는 교회들이 나를 얼굴로 알지 못하고
\par 23 다만 우리를 핍박하던 자가 전에 잔해하던 그 믿음을 지금 전한다 함을 듣고
\par 24 나로 말미암아 영광을 하나님께 돌리니라

\chapter{2}

\par 1 십 사년 후에 내가 바나바와 함께 디도를 데리고 다시 예루살렘에 올라갔노니
\par 2 계시를 인하여 올라가 내가 이방 가운데서 전파하는 복음을 저희에게 제출하되 유명한 자들에게 사사로이 한 것은 내가 달음질하는 것이나 달음질 한 것이 헛되지 않게 하려 함이라
\par 3 그러나 나와 함께 있는 헬라인 디도라도 억지로 할례를 받게 아니하였으니
\par 4 이는 가만히 들어온 거짓 형제 까닭이라 저희가 가만히 들어온것은 그리스도 예수 안에서 우리의 가진 자유를 엿보고 우리를 종으로 삼고자 함이로되
\par 5 우리가 일시라도 복종치 아니하였으니 이는 복음의 진리로 너희 가운데 항상 있게 하려 함이라
\par 6 유명하다는 이들 중에 (본래 어떤이들이든지 내게 상관이 없으며 하나님은 사람의 외모를 취하지 아니하시나니) 저 유명한 이들은 내게 더하여 준 것이 없고
\par 7 도리어 내가 무할례자에게 복음 전함을 맡기를 베드로가 할례자에게 맡음과 같이 한 것을 보고
\par 8 베드로에게 역사하사 그를 할례자의 사도로 삼으신 이가 또한 내게 역사하사 나를 이방인에게 사도로 삼으셨느니라
\par 9 "또 내게 주신 은혜를 알므로 기둥같이 여기는 야고보와 게바와 요한도 나와 바나바에게 교제의 악수를 하였으니 이는 우리는 이방인에게로, 저희는 할례자에게로 가게 하려 함이라"
\par 10 다만 우리에게 가난한 자들 생각하는 것을 부탁하였으니 이것을 나도 본래 힘써 행하노라
\par 11 게바가 안디옥에 이르렀을 때에 책망할 일이 있기로 내가 저를 면책하였노라
\par 12 야고보에게서 온 어떤 이들이 이르기 전에 게바가 이방인과 함께 먹다가 저희가 오매 그가 할례자들을 두려워하여 떠나 물러가매
\par 13 남은 유대인들도 저와 같이 외식하므로 바나바도 저희의 외식에 유혹되었느니라
\par 14 그러므로 나는 저희가 복음의 진리를 따라 바로 행하지 아니함을보고 모든 자 앞에서 게바에게 이르되 네가 유대인으로서 이방을 좇고 유대인답게 살지 아니하면서 어찌하여 억지로 이방인을 유대인답게 살게 하려느냐 하였노라
\par 15 우리는 본래 유대인이요 이방 죄인이 아니로되
\par 16 사람이 의롭게 되는 것은 율법의 행위에서 난 것이 아니요 오직 예수 그리스도를 믿음으로 말미암는줄 아는고로 우리도 그리스도 예수를 믿나니 이는 우리가 율법의 행위에서 아니고 그리스도를 믿음으로서 의롭다 함을 얻으려 함이라 율법의 행위로서는 의롭다 함을 얻을 육체가 없느니라
\par 17 만일 우리가 그리스도 안에서 의롭게 되려 하다가 죄인으로 나타나면 그리스도께서 죄를 짓게 하는 자냐 결코 그럴 수 없느니라
\par 18 만일 내가 헐었던 것을 다시 세우면 내가 나를 범법한 자로 만드는 것이라
\par 19 내가 율법으로 말미암아 율법을 향하여 죽었나니 이는 하나님을 향하여 살려 함이니라
\par 20 내가 그리스도와 함께 십자가에 못 박혔나니 그런즉 이제는 내가 산 것이 아니요 오직 내 안에 그리스도께서 사신 것이라 이제 내가 육체 가운데 사는 것은 나를 사랑하사 나를 위하여 자기 몸을 버리신 하나님의 아들을 믿는 믿음 안에서 사는 것이라
\par 21 내가 하나님의 은혜를 폐하지 아니하노니 만일 의롭게 되는 것이 율법으로 말미암으면 그리스도께서 헛되이 죽으셨느니라

\chapter{3}

\par 1 어리석도다 갈라디아 사람들아 예수 그리스도께서 십자가에 못 박히신 것이 너희 눈앞에 밝히 보이거늘 누가 너희를 꾀더냐
\par 2 내가 너희에게 다만 이것을 알려 하노니 너희가 성령을 받은 것은 율법의 행위로냐 듣고 믿음으로냐
\par 3 너희가 이같이 어리석으냐 성령으로 시작하였다가 이제는 육체로 마치겠느냐
\par 4 너희가 이같이 많은 괴로움을 헛되이 받았느냐 과연 헛되냐
\par 5 너희에게 성령을 주시고 너희 가운데서 능력을 행하시는 이의 일이 율법의 행위에서냐 듣고 믿음에서냐
\par 6 아브라함이 하나님을 믿으매 이것을 그에게 의로 정하셨다 함과 같으니라
\par 7 그런즉 믿음으로 말미암은 자들은 아브라함의 아들인줄 알지어다
\par 8 또 하나님이 이방을 믿음으로 말미암아 의로 정하실 것을 성경이 미리 알고 먼저 아브라함에게 복음을 전하되 모든 이방이 너를 인하여 복을 받으리라 하였으니
\par 9 그러므로 믿음으로 말미암은 자는 믿음이 있는 아브라함과 함께 복을 받느니라
\par 10 무릇 율법 행위에 속한 자들은 저주 아래 있나니 기록된바 누구든지 율법 책에 기록된대로 온갖 일을 항상 행하지 아니하는자는 저주 아래 있는 자라 하였음이라
\par 11 또 하나님 앞에서 아무나 율법으로 말미암아 의롭게 되지 못할 것이 분명하니 이는 의인이 믿음으로 살리라 하였음이니라
\par 12 율법은 믿음에서 난 것이 아니라 이를 행하는 자는 그 가운데서 살리라 하였느니라
\par 13 그리스도께서 우리를 위하여 저주를 받은바 되사 율법의 저주에서 우리를 속량하셨으니 기록된바 나무에 달린 자마다 저주 아래 있는 자라 하였음이라
\par 14 이는 그리스도 예수 안에서 아브라함의 복이 이방인에게 미치게 하고 또 우리로 하여금 믿음으로 말미암아 성령의 약속을 받게 하려 함이니라
\par 15 형제들아 사람의 예대로 말하노니 사람의 언약이라도 정한 후에는 아무나 폐하거나 더하거나 하지 못하느니라
\par 16 이 약속들은 아브라함과 그 자손에게 말씀하신 것인데 여럿을 가리켜 그 자손들이라 하지 아니하시고 오직 하나를 가리켜 네 자손이라 하셨으니 곧 그리스도라
\par 17 내가 이것을 말하노니 하나님의 미리 정하신 언약을 사백 삼십년 후에 생긴 율법이 없이 하지 못하여 그 약속을 헛되게 하지 못하리라
\par 18 만일 그 유업이 율법에서 난 것이면 약속에서 난 것이 아니리라 그러나 하나님이 약속으로 말미암아 아브라함에게 은혜로 주신 것이라
\par 19 그런즉 율법은 무엇이냐 범법함을 인하여 더한 것이라 천사들로 말미암아 중보의 손을 빌어 베푸신 것인데 약속하신 자손이 오시기까지 있을 것이라
\par 20 중보는 한편만 위한 자가 아니니 오직 하나님은 하나이시니라
\par 21 그러면 율법이 하나님의 약속들을 거스리느냐 결코 그럴 수 없느니라 만일 능히 살게 하는 율법을 주셨더면 의가 반드시 율법으로 말미암았으리라
\par 22 그러나 성경이 모든 것을 죄 아래 가두었으니 이는 예수 그리스도를 믿음으로 말미암은 약속을 믿는 자들에게 주려 함이니라
\par 23 믿음이 오기 전에 우리가 율법 아래 매인바 되고 계시될 믿음의 때까지 갇혔느니라
\par 24 이같이 율법이 우리를 그리스도에게로 인도하는 몽학선생이 되어 우리로 하여금 믿음으로 말미암아 의롭다 함을 얻게 하려 함이니라
\par 25 믿음이 온 후로는 우리가 몽학선생 아래 있지 아니하도다
\par 26 너희가 다 믿음으로 말미암아 그리스도 예수 안에서 하나님의 아들이 되었으니
\par 27 누구든지 그리스도와 합하여 세례를 받은 자는 그리스도로 옷입었느니라
\par 28 너희는 유대인이나 헬라인이나 종이나 자주자나 남자나 여자 없이 다 그리스도 예수 안에서 하나이니라
\par 29 너희가 그리스도께 속한 자면 곧 아브라함의 자손이요 약속대로 유업을 이을 자니라

\chapter{4}

\par 1 내가 또 말하노니 유업을 이을 자가 모든 것의 주인이나 어렸을 동안에는 종과 다름이 없어서
\par 2 그 아버지의 정한 때까지 후견인과 청지기 아래 있나니
\par 3 이와 같이 우리도 어렸을 때에 이 세상 초등 학문 아래 있어서 종노릇 하였더니
\par 4 때가 차매 하나님이 그 아들을 보내사 여자에게서 나게 하시고 율법 아래 나게 하신 것은
\par 5 율법 아래 있는 자들을 속량하시고 우리로 아들의 명분을 얻게 하려 하심이라
\par 6 너희가 아들인고로 하나님이 그 아들의 영을 우리 마음 가운데 보내사 아바 아버지라 부르게 하셨느니라
\par 7 그러므로 네가 이 후로는 종이 아니요 아들이니 아들이면 하나님으로 말미암아 유업을 이을 자니라
\par 8 그러나 너희가 그 때에는 하나님을 알지 못하여 본질상 하나님이 아닌 자들에게 종노릇하였더니
\par 9 이제는 너희가 하나님을 알뿐더러 하나님의 아신바 되었거늘 어찌하여 다시 약하고 천한 초등 학문으로 돌아가서 다시 저희에게 종노릇하려 하느냐
\par 10 너희가 날과 달과 절기와 해를 삼가 지키니
\par 11 내가 너희를 위하여 수고한 것이 헛될까 두려워하노라
\par 12 형제들아 내가 너희와 같이 되었은즉 너희도 나와 같이 되기를 구하노라 너희가 내게 해롭게 하지 아니하였느니라
\par 13 내가 처음에 육체의 약함을 인하여 너희에게 복음을 전한 것을 너희가 아는바라
\par 14 너희를 시험하는 것이 내 육체에 있으되 이것을 너희가 업신여기지도 아니하며 버리지도 아니하고 오직 나를 하나님의 천사와 같이 또는 그리스도 예수와 같이 영접하였도다
\par 15 너희의 복이 지금 어디 있느냐 내가 너희에게 증거하노니 너희가 할 수만 있었더면 너희의 눈이라도 빼어 나를 주었으리라
\par 16 그런즉 내가 너희에게 참된 말을 하므로 원수가 되었느냐
\par 17 저희가 너희를 대하여 열심내는 것이 좋은 뜻이 아니요 오직 너희를 이간 붙여 너희로 저희를 대하여 열심 내게 하려 함이라
\par 18 좋은 일에 대하여 열심으로 사모함을 받음은 내가 너희를 대하였을 때뿐 아니라 언제든지 좋으니라
\par 19 나의 자녀들아 너희 속에 그리스도의 형상이 이루기까지 다시 너를 위하여 해산하는 수고를 하노니
\par 20 내가 이제라도 너희와 함께 있어 내 음성을 변하려 함은 너희를 대하여 의심이 있음이라
\par 21 내게 말하라 율법 아래 있고자 하는 자들아 율법을 듣지 못하였느냐
\par 22 기록된바 아브라함이 두 아들이 있으니 하나는 계집종에게서 하나는 자유하는 여자에게서 났다 하였으나
\par 23 계집 종에게서는 육체를 따라 났고 자유하는 여자에게서는 약속으로 말미암았느니라
\par 24 이것은 비유니 이 여자들은 두 언약이라 하나는 시내산으로부터 종을 낳은 자니 곧 하가라
\par 25 이 하가는 아라비아에 있는 시내산으로 지금 있는 예루살렘과 같은 데니 저가 그 자녀들로 더불어 종노릇하고
\par 26 오직 위에 있는 예루살렘은 자유자니 곧 우리 어머니라
\par 27 기록된바 잉태치 못한 자여 즐거워하라 구로치 못한 자여 소리질러 외치라 이는 홀로 사는 자의 자녀가 남편 있는 자의 자녀보다 많음이라 하였으니
\par 28 형제들아 너희는 이삭과 같이 약속의 자녀라
\par 29 그러나 그 때에 육체를 따라 난 자가 성령을 따라 난 자를 핍박한 것 같이 이제도 그러하도다
\par 30 그러나 성경이 무엇을 말하느뇨 계집 종과 그 아들을 내어 쫓으라 계집 종의 아들이 자유하는 여자의 아들로 더불어 유업을 얻지 못하리라 하였느니라
\par 31 그런즉 형제들아 우리는 계집 종의 자녀가 아니요 자유하는 여자의 자녀니라

\chapter{5}

\par 1 그리스도께서 우리로 자유케 하려고 자유를 주셨으니 그러므로 굳세게 서서 다시는 종의 멍에를 메지 말라
\par 2 보라 나 바울은 너희에게 말하노니 너희가 만일 할례를 받으면 그리스도께서 너희에게 아무 유익이 없으리라
\par 3 내가 할례를 받는 각 사람에게 다시 증거하노니 그는 율법 전체를 행할 의무를 가진 자라
\par 4 율법 안에서 의롭다 함을 얻으려 하는 너희는 그리스도에게서 끊어지고 은혜에서 떨어진 자로다
\par 5 우리가 성령으로 믿음을 좇아 의의 소망을 기다리노니
\par 6 그리스도 예수 안에서는 할례나 무할례가 효력이 없되 사랑으로써 역사하는 믿음 뿐이니라
\par 7 너희가 달음질을 잘 하더니 누가 너희를 막아 진리를 순종치 않게 하더냐
\par 8 그 권면이 너희를 부르신 이에게서 난 것이 아니라
\par 9 적은 누룩이 온 덩이에 퍼지느니라
\par 10 나는 너희가 아무 다른 마음도 품지 아니할 줄을 주 안에서 확신하노라 그러나 너희를 요동케 하는 자는 누구든지 심판을 받으리라
\par 11 형제들아 내가 지금까지 할례를 전하면 어찌하여 지금까지 핍박을 받으리요 그리하였으면 십자가의 거치는 것이 그쳤으리니
\par 12 너희를 어지럽게 하는 자들이 스스로 베어 버리기를 원하노라
\par 13 형제들아 너희가 자유를 위하여 부르심을 입었으나 그러나 그 자유로 육체의 기회를 삼지 말고 오직 사랑으로 서로 종노릇하라
\par 14 온 율법은 네 이웃 사랑하기를 네 몸같이 하라 하신 한 말씀에 이루었나니
\par 15 만일 서로 물고 먹으면 피차 멸망할까 조심하라
\par 16 내가 이르노니 너희는 성령을 좇아 행하라 그리하면 육체의 욕심을 이루지 아니하리라
\par 17 육체의 소욕은 성령을 거스리고 성령의 소욕은 육체를 거스리나니 이 둘이 서로 대적함으로 너희의 원하는 것을 하지 못하게 하려 함이니라
\par 18 너희가 만일 성령의 인도하시는 바가 되면 율법 아래 있지 아니하리라
\par 19 육체의 일은 현저하니 곧 음행과 더러운 것과 호색과
\par 20 우상 숭배와 술수와 원수를 맺는 것과 분쟁과 시기와 분냄과 당 짓는 것과 분리함과 이단과
\par 21 투기와 술 취함과 방탕함과 또 그와 같은 것들이라 전에 너희에게 경계한것 같이 경계하노니 이런 일을 하는 자들은 하나님의 나라를 유업으로 받지 못할 것이요
\par 22 오직 성령의 열매는 사랑과 희락과 화평과 오래 참음과 자비와 양선과 충성과
\par 23 온유와 절제니 이같은 것을 금지할 법이 없느니라
\par 24 그리스도 예수의 사람들은 육체와 함께 그 정과 욕심을 십자가에 못 박았느니라
\par 25 만일 우리가 성령으로 살면 또한 성령으로 행할지니
\par 26 헛된 영광을 구하여 서로 격동하고 서로 투기하지 말지니라

\chapter{6}

\par 1 형제들아 사람이 만일 무슨 범죄한 일이 드러나거든 신령한 너희는 온유한 심령으로 그러한 자를 바로잡고 네 자신을 돌아보아 너도 시험을 받을까 두려워하라
\par 2 너희가 짐을 서로 지라 그리하여 그리스도의 법을 성취하라
\par 3 만일 누가 아무 것도 되지 못하고 된줄로 생각하면 스스로 속임이니라
\par 4 각각 자기의 일을 살피라 그리하면 자랑할 것이 자기에게만 있고 남에게는 있지 아니하리니
\par 5 각각 자기의 짐을 질 것임이니라
\par 6 가르침을 받는 자는 말씀을 가르치는 자와 모든 좋은 것을 함께 하라
\par 7 스스로 속이지 말라 하나님은 만홀히 여김을 받지 아니하시나니 사람이 무엇으로 심든지 그대로 거두리라
\par 8 자기의 육체를 위하여 심는 자는 육체로부터 썩어진 것을 거두고 성령을 위하여 심는 자는 성령으로부터 영생을 거두리라
\par 9 우리가 선을 행하되 낙심하지 말지니 피곤하지 아니하면 때가 이르매 거두리라
\par 10 그러므로 우리는 기회 있는대로 모든 이에게 착한 일을 하되 더욱 믿음의 가정들에게 할 지니라
\par 11 내 손으로 너희에게 이렇게 큰 글자로 쓴 것을 보라
\par 12 무릇 육체의 모양을 내려 하는 자들이 억지로 너희로 할례 받게 함은 저희가 그리스도의 십자가를 인하여 핍박을 면하려 함뿐이라
\par 13 할례 받은 저희라도 스스로 율법은 지키지 아니하고 너희로 할례 받게 하려 하는 것은 너희의 육체로 자랑하려 함이니라
\par 14 그러나 내게는 우리 주 예수 그리스도의 십자가 외에 결코 자랑할 것이 없으니 그리스도로 말미암아 세상이 나를 대하여 십자가에 못 박히고 내가 또한 세상을 대하여 그러하니라
\par 15 할례나 무할례가 아무 것도 아니로되 오직 새로 지으심을 받은 자 뿐이니라
\par 16 무릇 이 규례를 행하는 자에게와 하나님의 이스라엘에게 평강과 긍휼이 있을지어다
\par 17 이 후로는 누구든지 나를 괴롭게 말라 내가 내 몸에 예수의 흔적을 가졌노라
\par 18 형제들아 우리 주 예수 그리스도의 은혜가 너희 심령에 있을지어다 아멘


\end{document}