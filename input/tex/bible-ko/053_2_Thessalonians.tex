\begin{document}

\title{데살로니가후서}


\chapter{1}

\par 1 바울과 실루아노와 디모데는 하나님 우리 아버지와 주 예수 그리스도 안에 있는 데살로니가인의 교회에 편지하노니
\par 2 하나님 아버지와 주 예수 그리스도로부터 은혜와 평강이 너희에게 있을지어다
\par 3 형제들아 우리가 너희를 위하여 항상 하나님께 감사할지니 이것이 당연함은 너희 믿음이 더욱 자라고 너희가 다 각기 서로 사랑함이 풍성함이며
\par 4 그리고 너희의 참는 모든 핍박과 환난 중에서 너희 인내와 믿음을 인하여 하나님의 여러 교회에서 우리가 친히 자랑함이라
\par 5 이는 하나님의 공의로운 심판의 표요 너희로 하여금 하나님의 나라에 합당한 자로 여기심을 얻게 하려 함이니 그 나라를 위하여 너희가 또한 고난을 받으리니
\par 6 너희로 환난 받게 하는 자들에게는 환난으로 갚으시고
\par 7 환난 받는 너희에게는 우리와 함께 안식으로 갚으시는 것이 하나님의 공의시니 주 예수께서 저의 능력의 천사들과 함께 하늘로부터 불꽃 중에 나타나실 때에
\par 8 하나님을 모르는 자들과 우리 주 예수의 복음을 복종치 않는 자들에게 형벌을 주시리니
\par 9 이런 자들이 주의 얼굴과 그의 힘의 영광을 떠나 영원한 멸망의 형벌을 받으리로다
\par 10 그날에 강림하사 그의 성도들에게서 영광을 얻으시고 모든 믿는 자에게서 기이히 여김을 얻으시리라(우리의 증거가 너희에게 믿어졌음이라)
\par 11 이러므로 우리도 항상 너희를 위하여 기도함은 우리 하나님이 너희를 그 부르심에 합당한 자로 여기시고 모든 선을 기뻐함과 믿음의 역사를 능력으로 이루게 하시고
\par 12 우리 하나님과 주 예수 그리스도의 은혜대로 우리 주 예수의 이름이 너희 가운데서 영광을 얻으시고 너희도 그 안에서 영광을 얻게 하려 함이니라

\chapter{2}

\par 1 형제들아 우리가 너희에게 구하는 것은 우리 주 예수 그리스도의 강림하심과 우리가 그 앞에 모임에 관하여
\par 2 혹 영으로나 혹 말로나 혹 우리에게서 받았다 하는 편지로나 주의 날이 이르렀다고 쉬 동심하거나 두려워하거나 하지 아니할 그것이라
\par 3 누가 아무렇게 하여도 너희가 미혹하지 말라 먼저 배도하는 일이있고 저 불법의 사람 곧 멸망의 아들이 나타나기 전에는 이르지 아니하리니
\par 4 저는 대적하는 자라 범사에 일컫는 하나님이나 숭배함을 받는 자위에 뛰어나 자존하여 하나님 성전에 앉아 자기를 보여 하나님이라 하느니라
\par 5 내가 너희와 함께 있을 때에 이 일을 너희에게 말한 것을 기억하지 못하느냐
\par 6 저로 하여금 저의 때에 나타나게 하려 하여 막는 것을 지금도 너희가 아나니
\par 7 불법의 비밀이 이미 활동하였으나 지금 막는 자가 있어 그 중에서 옮길 때까지 하리라
\par 8 그 때에 불법한 자가 나타나리니 주 예수께서 그 입의 기운으로 저를 죽이시고 강림하여 나타나심으로 폐하시리라
\par 9 악한 자의 임함은 사단의 역사를 따라 모든 능력과 표적과 거짓 기적과
\par 10 불의의 모든 속임으로 멸망하는 자들에게 임하리니 이는 저희가 진리의 사랑을 받지 아니하여 구원함을 얻지 못함이니라
\par 11 이러므로 하나님이 유혹을 저의 가운데 역사하게 하사 거짓 것을 믿게 하심은
\par 12 진리를 믿지 않고 불의를 좋아하는 모든 자로 심판을 받게 하려 하심이니라
\par 13 주의 사랑하시는 형제들아 우리가 항상 너희를 위하여 마땅히 하나님께 감사할 것은 하나님이 처음부터 너희를 택하사 성령의 거룩하게 하심과 진리를 믿음으로 구원을 얻게 하심이니
\par 14 이를 위하여 우리 복음으로 너희를 부르사 우리 주 예수 그리스도의 영광을 얻게 하려 하심이니라
\par 15 이러므로 형제들아 굳게 서서 말로나 우리 편지로 가르침을 받은 유전을 지키라
\par 16 우리 주 예수 그리스도와 우리를 사랑하시고 영원한 위로와 좋은 소망을 은혜로 주신 하나님 우리 아버지께
\par 17 너희 마음을 위로하시고 모든 선한 일과 말에 굳게 하시기를 원하노라

\chapter{3}

\par 1 종말로 형제들아 너희는 우리를 위하여 기도하기를 주의 말씀이 너희 가운데서와 같이 달음질하여 영광스럽게 되고
\par 2 또한 우리를 무리하고 악한 사람들에게서 건지옵소서 하라 믿음은 모든 사람의 것이 아님이라
\par 3 주는 미쁘사 너희를 굳게 하시고 악한 자에게서 지키시리라
\par 4 너희에게 대하여는 우리의 명한 것을 너희가 행하고 또 행할 줄을 우리가 주 안에서 확신하노니
\par 5 주께서 너희 마음을 인도하여 하나님의 사랑과 그리스도의 인내에 들어가게 하시기를 원하노라
\par 6 형제들아 우리 주 예수 그리스도의 이름으로 너희를 명하노니 규모 없이 행하고 우리에게 받은 유전대로 행하지 아니하는 모든 형제에게서 떠나라
\par 7 어떻게 우리를 본받아야 할 것을 너희가 스스로 아나니 우리가 너희 가운데서 규모 없이 행하지 아니하며
\par 8 누구에게서든지 양식을 값없이 먹지 않고 오직 수고하고 애써 주야로 일함은 너희 아무에게도 누를 끼치지 아니하려 함이니
\par 9 우리에게 권리가 없는 것이 아니요 오직 스스로 너희에게 본을 주어 우리를 본받게 하려 함이니라
\par 10 우리가 너희와 함께 있을 때에도 너희에게 명하기를 누구든지 일하기 싫어하거든 먹지도 말게 하라 하였더니
\par 11 우리가 들은즉 너희 가운데 규모 없이 행하여 도무지 일하지 아니하고 일만 만드는 자들이 있다 하니
\par 12 이런 자들에게 우리가 명하고 주 예수 그리스도 안에서 권하기를 종용히 일하여 자기 양식을 먹으라 하노라
\par 13 형제들아 너희는 선을 행하다가 낙심치 말라
\par 14 누가 이 편지에 한 우리 말을 순종치 아니하거든 그 사람을 지목하여 사귀지 말고 저로 하여금 부끄럽게 하라
\par 15 그러나 원수와 같이 생각지 말고 형제같이 권하라
\par 16 평강의 주께서 친히 때마다 일마다 너희에게 평강을 주시기를 원하노라 주는 너희 모든 사람과 함께 하실지어다
\par 17 나 바울은 친필로 문안하노니 이는 편지마다 표적이기로 이렇게 쓰노라
\par 18 우리 주 예수 그리스도의 은혜가 너희 무리에게 있을지어다


\end{document}