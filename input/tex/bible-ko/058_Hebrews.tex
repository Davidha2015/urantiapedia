\begin{document}

\title{히브리서}


\chapter{1}

\par 1 옛적에 선지자들로 여러 부분과 여러 모양으로 우리 조상들에게 말씀하신 하나님이
\par 2 이 모든 날 마지막에 아들로 우리에게 말씀하셨으니 이 아들을 만유의 후사로 세우시고 또 저로 말미암아 모든 세계를 지으셨느니라
\par 3 이는 하나님의 영광의 광채시요 그 본체의 형상이시라 그의 능력의 말씀으로 만물을 붙드시며 죄를 정결케 하는 일을 하시고 높은 곳에 계신 위엄의 우편에 앉으셨느니라
\par 4 저가 천사보다 얼마큼 뛰어남은 저희보다 더욱 아름다운 이름을 기업으로 얻으심이니
\par 5 하나님께서 어느 때에 천사 중 누구에게 네가 내 아들이라 오늘날 내가 너를 낳았다 하셨으며 또 다시 나는 그에게 아버지가 되고 그는 내게 아들이 되리라 하셨느뇨
\par 6 또 맏아들을 이끌어 세상에 다시 들어오게 하실 때에 하나님의 모든 천사가 저에게 경배할지어다 말씀하시며
\par 7 또 천사들에 관하여는 그는 그의 천사들을 바람으로 그의 사역자들을 불꽃으로 삼으시느니라 하셨으되
\par 8 아들에 관하여는 하나님이여 주의 보좌가 영영하며 주의 나라의 홀은 공평한 홀이니이다
\par 9 네가 의를 사랑하고 불법을 미워하였으니 그러므로 하나님 곧 너의 하나님이 즐거움의 기름을 네게 주어 네 동류들보다 승하게 하셨도다 하였고
\par 10 또 주여 태초에 주께서 땅의 기초를 두셨으며 하늘도 주의 손으로 지으신 바라
\par 11 그것들은 멸망할 것이나 오직 주는 영존할 것이요 그것들은 다 옷과 같이 낡아지리니
\par 12 의복처럼 갈아 입을 것이요 그것들이 옷과 같이 변할 것이나 주는 여전하여 연대가 다함이 없으리라 하였으나
\par 13 어느 때에 천사 중 누구에게 내가 네 원수로 네 발등상 되게 하기까지 너는 내 우편에 앉았으라 하셨느뇨
\par 14 모든 천사들은 부리는 영으로서 구원 얻을 후사들을 위하여 섬기라고 보내심이 아니뇨

\chapter{2}

\par 1 그러므로 모든 들은 것을 우리가 더욱 간절히 삼갈지니 혹 흘러 떠내려 갈까 염려하노라
\par 2 천사들로 하신 말씀이 견고하게 되어 모든 범죄들과 순종치 아니함이 공변된 보응을 받았거든
\par 3 우리가 이같이 큰 구원을 등한히 여기면 어찌 피하리요 이 구원은 처음에 주로 말씀하신 바요 들은 자들이 우리에게 확증한 바니
\par 4 하나님도 표적들과 기사들과 여러가지 능력과 및 자기 뜻을 따라 성령의 나눠 주신 것으로써 저희와 함께 증거하셨느니라
\par 5 하나님이 우리의 말한 바 장차 오는 세상을 천사들에게는 복종케하심이 아니라
\par 6 오직 누가 어디 증거하여 가로되 사람이 무엇이관대 주께서 저를 생각하시며 인자가 무엇이관대 주께서 저를 권고하시나이까
\par 7 저를 잠간 동안 천사보다 못하게 하시며 영광과 존귀로 관 씌우시며
\par 8 만물을 그 발 아래 복종케 하셨느니라 하였으니 만물로 저에게 복종케 하셨은즉 복종치 않은 것이 하나도 없으나 지금 우리가 만물이 아직 저에게 복종한 것을 보지 못하고
\par 9 오직 우리가 천사들보다 잠간 동안 못하게 하심을 입은 자 곧 죽음의 고난 받으심을 인하여 영광과 존귀로 관 쓰신 예수를 보니 이를 행하심은 하나님의 은혜로 말미암아 모든 사람을 위하여 죽음을 맛보려 하심이라
\par 10 만물이 인하고 만물이 말미암은 자에게는 많은 아들을 이끌어 영광에 들어가게 하시는 일에 저희 구원의 주를 고난으로 말미암아 온전케 하심이 합당하도다
\par 11 거룩하게 하시는 자와 거룩하게 함을 입은 자들이 다 하나에서 난지라 그러므로 형제라 부르시기를 부끄러워 아니하시고
\par 12 이르시되 내가 주의 이름을 내 형제들에게 선포하고 내가 주를 교회 중에서 찬송하리라 하셨으며
\par 13 또 다시 내가 그를 의지하리라 하시고 또 다시 볼지어다 나와 및 하나님께서 내게 주신 자녀라 하셨으니
\par 14 자녀들은 혈육에 함께 속하였으매 그도 또한 한 모양으로 혈육에 함께 속하심은 사망으로 말미암아 사망의 세력을 잡은 자 곧 마귀를 없이 하시며
\par 15 또 죽기를 무서워하므로 일생에 매여 종노릇하는 모든 자들을 놓아주려 하심이니
\par 16 이는 실로 천사들을 붙들어 주려 하심이 아니요 오직 아브라함의 자손을 붙들어 주려 하심이라
\par 17 그러므로 저가 범사에 형제들과 같이 되심이 마땅하도다 이는 하나님의 일에 자비하고 충성된 대제사장이 되어 백성의 죄를 구속하려 하심이라
\par 18 자기가 시험을 받아 고난을 당하셨은즉 시험 받는 자들을 능히 도우시느니라

\chapter{3}

\par 1 그러므로 함께 하늘의 부르심을 입은 거룩한 형제들아 우리의 믿는 도리의 사도시며 대제사장이신 예수를 깊이 생각하라
\par 2 저가 자기를 세우신 이에게 충성하시기를 모세가 하나님의 온 집에서 한 것과 같으니
\par 3 저는 모세보다 더욱 영광을 받을 만한 것이 마치 집 지은 자가 그 집보다 더욱 존귀함 같으니라
\par 4 집마다 지은 이가 있으니 만물을 지으신 이는 하나님이시라
\par 5 또한 모세는 장래의 말할 것을 증거하기 위하여 하나님의 온 집에서 사환으로 충성하였고
\par 6 그리스도는 그의 집 맡은 아들로 충성하였으니 우리가 소망의 담대함과 자랑을 끝까지 견고히 잡으면 그의 집이라
\par 7 그러므로 성령이 이르신바와 같이 오늘날 너희가 그의 음성을 듣거든
\par 8 노하심을 격동하여 광야에서 시험하던 때와 같이 너희 마음을 강퍅케 하지 말라
\par 9 거기서 너희 열조가 나를 시험하여 증험하고 사십 년 동안에 나의 행사를 보았느니라
\par 10 그러므로 내가 이 세대를 노하여 가로되 저희가 항상 마음이 미혹되어 내 길을 알지 못하는도다 하였고
\par 11 내가 노하여 맹세한 바와 같이 저희는 내 안식에 들어오지 못하리라 하셨다 하였으니
\par 12 형제들아 너희가 삼가 혹 너희중에 누가 믿지 아니하는 악심을 품고 살아 계신 하나님에게서 떨어질까 염려할 것이요
\par 13 오직 오늘이라 일컫는 동안에 매일 피차 권면하여 너희 중에 누구든지 죄의 유혹으로 강퍅케 됨을 면하라
\par 14 우리가 시작할 때에 확실한 것을 끝까지 견고히 잡으면 그리스도와 함께 참예한 자가 되리라
\par 15 성경에 일렀으되 오늘날 너희가 그의 음성을 듣거든 노하심을 격동할 때와 같이 너희 마음을 강퍅케 하지 말라 하였으니
\par 16 듣고 격노케 하던 자가 누구뇨 모세를 좇아 애굽에서 나온 모든 이가 아니냐
\par 17 또 하나님이 사십 년 동안에 누구에게 노하셨느뇨 범죄하여 그 시체가 광야에 엎드러진 자에게가 아니냐
\par 18 또 하나님이 누구에게 맹세하사 그의 안식에 들어오지 못하리라 하셨느뇨 곧 순종치 아니하던 자에게가 아니냐
\par 19 이로 보건대 저희가 믿지 아니하므로 능히 들어가지 못한 것이라

\chapter{4}

\par 1 그러므로 우리는 두려워할지니 그의 안식에 들어갈 약속이 남아 있을지라도 너희 중에 혹 미치지 못할 자가 있을까 함이라
\par 2 저희와 같이 우리도 복음 전함을 받은자이나 그러나 그 들은 바 말씀이 저희에게 유익되지 못한 것은 듣는 자가 믿음을 화합지 아니함이라
\par 3 이미 믿는 우리들은 저 안식에 들어가는도다 그 말씀하신 바와 같으니 내가 노하여 맹세한 바와 같이 저희가 내 안식에 들어오지 못하리라 하셨다 하였으나 세상을 창조할 때부터 그일이 이루었느니라
\par 4 제 칠일에 관하여는 어디 이렇게 일렀으되 하나님은 제 칠일에 그의 모든 일을 쉬셨다 하였으며
\par 5 또 다시 거기 저희가 내 안식에 들어오지 못하리라 하였으니
\par 6 그러면 거기 들어갈 자들이 남아 있거니와 복음 전함을 먼저 받은자들은 순종치 아니함을 인하여 들어가지 못하였으므로
\par 7 오랜 후에 다윗의 글에 다시 어느 날을 정하여 오늘날이라고 미리 이같이 일렀으되 오늘날 너희가 그의 음성을 듣거든 너희 마음을 강퍅케 말라 하였나니
\par 8 만일 여호수아가 저희에게 안식을 주었더면 그 후에 다른 날을 말씀하지 아니하셨으리라
\par 9 그런즉 안식할 때가 하나님의 백성에게 남아 있도다
\par 10 이미 그의 안식에 들어간 자는 하나님이 자기 일을 쉬심과 같이 자기 일을 쉬느니라
\par 11 그러므로 우리가 저 안식에 들어가기를 힘쓸지니 이는 누구든지 저 순종치 아니하는 본에 빠지지 않게 하려 함이라
\par 12 하나님의 말씀은 살았고 운동력이 있어 죄우에 날선 어떤 검보다도 예리하여 혼과 영과 및 관절과 골수를 찔러 쪼개기까지하며 또 마음의 생각과 뜻을 감찰하나니
\par 13 지으신 것이 하나라도 그 앞에 나타나지 않음이 없고 오직 만물이 우리를 상관하시는 자의 눈앞에 벌거벗은 것같이 드러나느니라
\par 14 그러므로 우리에게 큰 대제사장이 있으니 승천하신 자 곧 하나님 아들 예수시라 우리가 믿는 도리를 굳게 잡을지어다
\par 15 우리에게 있는 대제사장은 우리 연약함을 체휼하지 아니하는 자가 아니요 모든 일에 우리와 한결같이 시험을 받은 자로되 죄는 없으시니라
\par 16 그러므로 우리가 긍휼하심을 받고 때를 따라 돕는 은혜를 얻기 위하여 은혜의 보좌 앞에 담대히 나아갈 것이니라

\chapter{5}

\par 1 대제사장마다 사람 가운데서 취한자이므로 하나님께 속한 일에 사람을 위하여 예물과 속죄하는 제사를 드리게 하나니
\par 2 저가 무식하고 미혹한 자를 능히 용납할 수 있는 것은 자기도 연약에 싸여 있음이니라
\par 3 이러므로 백성을 위하여 속죄제를 드림과 같이 또한 자기를 위하여 드리는 것이 마땅하니라
\par 4 이 존귀는 아무나 스스로 취하지 못하고 오직 아론과 같이 하나님의 부르심을 입은 자라야 할 것이니라
\par 5 또한 이와 같이 그리스도께서 대제사장 되심도 스스로 영광을 취하심이 아니요 오직 말씀하신 이가 저더러 이르시되 너는 내 아들이니 오늘날 내가 너를 낳았다 하셨고
\par 6 또한 이와 같이 다른 데 말씀하시되 네가 영원히 멜기세덱의 반차를 좇는 제사장이라 하셨으니
\par 7 그는 육체에 계실 때에 자기를 죽음에서 능히 구원하실 이에게 심한 통곡과 눈물로 간구와 소원을 올렸고 그의 경외하심을 인하여 들으심을 얻었느니라
\par 8 그가 아들이시라도 받으신 고난으로 순종함을 배워서
\par 9 온전하게 되었은즉 자기를 순종하는 모든 자에게 영원한 구원의 근원이 되시고
\par 10 하나님께 멜기세덱의 반차를 좇은 대제사장이라 칭하심을 받았느니라
\par 11 멜기세덱에 관하여는 우리가 할 말이 많으나 너희의 듣는 것이 둔하므로 해석하기 어려우니라
\par 12 때가 오래므로 너희가 마땅히 선생이 될 터인데 너희가 다시 하나님의 말씀의 초보가 무엇인지 누구에게 가르침을 받아야 할 것이니 젖이나 먹고 단단한 식물을 못 먹을 자가 되었도다
\par 13 대저 젖을 먹는 자마다 어린 아이니 저희는 말씀을 경험하지 못한 자요
\par 14 단단한 식물은 장성한 자의 것이니 저희는 지각을 사용하므로 연단을 받아 선악을 분변하는 자들이니라

\chapter{6}

\par 1 그러므로 우리가 그리스도 도의 초보를 버리고 죽은 행실을 회개함과 하나님께 대한 신앙과
\par 2 세례들과 안수와 죽은 자의 부활과 영원한 심판에 관한 교훈의 터를 다시 닦지 말고 완전한 데 나아갈지니라
\par 3 하나님께서 허락하시면 우리가 이것을 하리라
\par 4 한번 비췸을 얻고 하늘의 은사를 맛보고 성령에 참예한바 되고
\par 5 하나님의 선한 말씀과 내세의 능력을 맛보고
\par 6 타락한 자들은 다시 새롭게 하여 회개케 할 수 없나니 이는 자기가 하나님의 아들을 다시 십자가에 못 박아 현저히 욕을 보임이라
\par 7 땅이 그 위에 자주 내리는 비를 흡수하여 밭 가는 자들의 쓰기에 합당한 채소를 내면 하나님께 복을 받고
\par 8 만일 가시와 엉겅퀴를 내면 버림을 당하고 저주함에 가까와 그 마지막은 불사름이 되리라
\par 9 사랑하는 자들아 우리가 이같이 말하나 너희에게는 이보다 나은 것과 구원에 가까운 것을 확신하노라
\par 10 하나님이 불의치 아니하사 너희 행위와 그의 이름을 위하여 나타낸 사랑으로 이미 성도를 넘긴 것과 이제도 섬기는 것을 잊어버리지 아니 하시느니라
\par 11 우리가 간절히 원하는 것은 너희 각 사람이 동일한 부지런을 나타내어 끝까지 소망의 풍성함에 이르러
\par 12 게으르지 아니하고 믿음과 오래 참음으로 말미암아 약속들을 기업으로 받는 자들을 본받는 자 되게 하려는 것이니라
\par 13 하나님이 아브라함에게 약속하실 때에 가리켜 맹세할 자가 자기보다 더 큰 이가 없으므로 자기를 가리켜 맹세하여
\par 14 가라사대 내가 반드시 너를 복주고 복 주며 너를 번성케 하고 번성케 하리라 하셨더니
\par 15 저가 이같이 오래 참아 약속을 받았느니라
\par 16 사람들은 자기보다 더 큰 자를 가리켜 맹세하나니 맹세는 저희 모든 다투는 일에 최후 확정이니라
\par 17 하나님은 약속을 기업으로 받는 자들에게 그 뜻이 변치 아니함을 충분히 나타내시려고 그 일에 맹세로 보증하셨나니
\par 18 이는 하나님이 거짓말을 하실수 없는 이 두 가지 변치 못할 사실을 인하여 앞에 있는 소망을 얻으려고 피하여 가는 우리로 큰 안위를 받게 하려 하심이라
\par 19 우리가 이 소망이 있는 것은 영혼의 닻 같아서 튼튼하고 견고하여 휘장 안에 들어 가나니
\par 20 그리로 앞서 가신 예수께서 멜기세덱의 반차를 좇아 영원히 대제사장이 되어 우리를 위하여 들어 가셨느니라

\chapter{7}

\par 1 이 멜기세덱은 살렘 왕이요 지극히 높으신 하나님의 제사장이라 여러 임금을 쳐서 죽이고 돌아오는 아브라함을 만나 복을 빈 자라
\par 2 아브라함이 일체 십분의 일을 그에게 나눠 주니라 그 이름을 번역한 즉 첫째 의의 왕이요 또 살렘 왕이니 곧 평강의 왕이요
\par 3 아비도 없고 어미도 없고 족보도 없고 시작한 날도 없고 생명의 끝도 없어 하나님 아들과 방불하여 항상 제사장으로 있느니라
\par 4 이 사람의 어떻게 높은 것을 생각하라 조상 아브라함이 노략물 중 좋은 것으로 십분의 일을 저에게 주었느니라
\par 5 레위의 아들들 가운데 제사장의 직분을 받는 자들이 율법을 좇아 아브라함의 허리에서 난 자라도 자기 형제인 백성에게서 십분의 일을 취하라는 명령을 가졌으나
\par 6 레위 족보에 들지 아니한 멜기세덱은 아브라함에게서 십분의 일을 취하고 그 얻은 자를 위하여 복을 빌었나니
\par 7 폐일언하고 낮은 자가 높은 자에게 복 빎을 받느니라
\par 8 또 여기는 죽을 자들이 십분의 일을 받으나 저기는 산다고 증거를 얻은 자가 받았느니라
\par 9 또한 십분의 일을 받는 레위도 아브라함으로 말미암아 십분의 일을 바쳤다 할 수 있나니
\par 10 이는 멜기세덱이 아브라함을 만날 때에 레위는 아직 자기 조상의 허리에 있었음이니라
\par 11 레위 계통의 제사 직분으로 말미암아 온전함을 얻을 수 있었으면(백성이 그 아래서 율법을 받았으니) 어찌하여 아론의 반차를 좇지않고 멜기세덱의 반차를 좇는 별다른 한 제사장을 세울 필요가 있느뇨
\par 12 제사 직분이 변역한즉 율법도 반드시 변역하리니
\par 13 이것은 한 사람도 제단 일을 받들지 않는 다른 지파에 속한 자를 가리켜 말한 것이라
\par 14 우리 주께서 유다로 좇아 나신것이 분명하도다 이 지파에는 모세가 제사장들에 관하여 말한 것이 하나도 없고
\par 15 멜기세덱과 같은 별다른 한 제사장이 일어난 것을 보니 더욱 분명하도다
\par 16 그는 육체에 상관된 계명의 법을 좇지 아니하고 오직 무궁한 생명의 능력을 좇아 된 것이니
\par 17 증거하기를 네가 영원히 멜기세덱의 반차를 좇는 제사장이라 하였도다
\par 18 전엣 계명이 연약하며 무익하므로 폐하고
\par 19 (율법은 아무 것도 온전케 못할지라)이에 더 좋은 소망이 생기니 이것으로 우리가 하나님께 가까이 가느니라
\par 20 또 예수께서 제사장 된 것은 맹세 없이 된 것이 아니니
\par 21 (저희는 맹세 없이 제사장이 되었으되 오직 예수는 자기에게 말씀하신 자로 말미암아 맹세로 되신 것이라 주께서 맹세하시고 뉘우치지 아니하시리니 네가 영원히 제사장이라 하셨도다)
\par 22 이와 같이 예수는 더 좋은 언약의 보증이 되셨느니라
\par 23 저희 제사장 된 자의 수효가 많은 것은 죽음을 인하여 항상 있지 못함이로되
\par 24 예수는 영원히 계시므로 그 제사 직분도 갈리지 아니하나니
\par 25 그러므로 자기를 힘입어 하나님께 나아가는 자들을 온전히 구원하실 수 있으니 이는 그가 항상 살아서 저희를 위하여 간구하심이니라
\par 26 이러한 대제사장은 우리에게 합당하니 거룩하고 악이 없고 더러움이 없고 죄인에게서 떠나 계시고 하늘보다 높이 되신 자라
\par 27 저가 저 대제사장들이 먼저 자기 죄를 위하고 다음에 백성의 죄를 위하여 날마다 제사 드리는 것과 같이 할 필요가 없으니 이는 저가 단번에 자기를 드려 이루셨음이니라
\par 28 율법은 약점을 가진 사람들을 제사장으로 세웠거니와 율법 후에 하신 맹세의 말씀은 영원히 온전케 되신 아들을 세우셨느니라

\chapter{8}

\par 1 이제 하는 말의 중요한 것은 이러한 대제사장이 우리에게 있는 것이라 그가 하늘에서 위엄의 보좌 우편에 앉으셨으니
\par 2 성소와 참 장막에 부리는 자라 이 장막은 주께서 베푸신 것이요 사람이 한 것이 아니니라
\par 3 대제사장마다 예물과 제사 드림을 위하여 세운 자니 이러므로 저도 무슨 드릴 것이 있어야 할지니라
\par 4 예수께서 만일 땅에 계셨더면 제사장이 되지 아니하셨을 것이니 이는 율법을 좇아 예물을 드리는 제사장이 있음이라
\par 5 저희가 섬기는 것은 하늘에 있는 것의 모형과 그림자라 모세가 장막을 지으려 할 때에 지시하심을 얻음과 같으니 가라사대 삼가 모든 것을 산에서 네게 보이던 본을 좇아 지으라 하셨느니라
\par 6 그러나 이제 그가 더 아름다운 직분을 얻으셨으니 이는 더 좋은 약속으로 세우신 더 좋은 언약의 중보시라
\par 7 저 첫 언약이 무흠하였더면 둘째 것을 요구할 일이 없었으려니와
\par 8 저희를 허물하여 일렀으되 주께서 가라사대 볼지어다 날이 이르리니 내가 이스라엘 집과 유다 집으로 새 언약을 세우리라
\par 9 또 주께서 가라사대 내가 저희 열조들의 손을 잡고 애굽 땅에서 인도하여 내던 날에 저희와 세운 언약과 같지 아니하도다 저희는 내 언약 안에 머물러 있지 아니하므로 내가 저희를 돌아보지 아니하였노라
\par 10 또 주께서 가라사대 그 날 후에 내가 이스라엘 집으로 세울 언약이 이것이니 내 법을 저희 생각에 두고 저희 마음에 이것을 기록 하리라 나는 저희에게 하나님이 되고 저희는 내게 백성이 되리라
\par 11 또 각각 자기 나라 사람과 각각 자기 형제를 가르쳐 이르기를 주를 알라 하지 아니할 것은 저희가 작은 자로부터 큰 자까지 다 나를 앎이니라
\par 12 내가 저희 불의를 긍휼히 여기고 저희 죄를 다시 기억하지 아니 하리라 하셨느니라
\par 13 새 언약이라 말씀하셨으매 첫것은 낡아지게 하신 것이니 낡아지고 쇠하는 것은 없어져가는 것이니라

\chapter{9}

\par 1 첫 언약에도 섬기는 예법과 세상에 속한 성소가 있더라
\par 2 예비한 첫 장막이 있고 그 안에 등대와 상과 진설병이 있으니 이는 성소라 일컫고
\par 3 또 둘째 휘장 뒤에 있는 장막을 지성소라 일컫나니
\par 4 금향로와 사면을 금으로 싼 언약궤가 있고 그 안에 만나를 담은 금항아리와 아론의 싹 난 지팡이와 언약의 비석들이 있고
\par 5 그 위에 속죄소를 덮는 영광의 그룹들이 있으니 이것들에 관하여는 이제 낱낱이 말할 수 없노라
\par 6 이 모든 것을 이같이 예비하였으니 제사장들이 항상 첫 장막에 들어가 섬기는 예를 행하고
\par 7 오직 둘째 장막은 대제사장이 홀로 일년 일차씩 들어가되 피 없이는 아니하나니 이 피는 자기와 백성의 허물을 위하여 드리는 것이라
\par 8 성령이 이로써 보이신 것은 첫장막이 서 있을 동안에 성소에 들어가는 길이 아직 나타나지 아니한 것이라
\par 9 이 장막은 현재까지의 비유니 이에 의지하여 드리는 예물과 제사가 섬기는 자로 그 양심상으로 온전케 할 수 없나니
\par 10 이런 것은 먹고 마시는 것과 여러 가지 씻는 것과 함께 육체의 예법만 되어 개혁할 때까지 맡겨 둔 것이니라
\par 11 그리스도께서 장래 좋은 일의 대제사장으로 오사 손으로 짓지 아니한 곧 이 창조에 속하지 아니한 더 크고 온전한 장막으로 말미암아
\par 12 염소와 송아지의 피로 아니하고 오직 자기 피로 영원한 속죄를 이루사 단번에 성소에 들어가셨느니라
\par 13 염소와 황소의 피와 및 암송아지의 재로 부정한 자에게 뿌려 그 육체를 정결케 하여 거룩케 하거든
\par 14 하물며 영원하신 성령으로 말미암아 흠 없는 자기를 하나님께 드린 그리스도의 피가 어찌 너희 양심으로 죽은 행실에서 깨끗하게 하고 살아계신 하나님을 섬기게 못하겠느뇨
\par 15 이를 인하여 그는 새 언약의 중보니 이는 첫 언약 때에 범한 죄를 속하려고 죽으사 부르심을 입은 자로 하여금 영원한 기업의 약속을 얻게 하려 하심이니라
\par 16 유언은 유언한 자가 죽어야 되나니
\par 17 유언은 그 사람이 죽은 후에야 견고한즉 유언한 자가 살았을 때에는 언제든지 효력이 없느니라
\par 18 이러므로 첫 언약도 피 없이 세운 것이 아니니
\par 19 모세가 율법대로 모든 계명을 온 백성에게 말한 후에 송아지와 염소의 피와 및 물과 붉은 양털과 우슬초를 취하여 그 책과 온 백성에게 뿌려
\par 20 이르되 이는 하나님이 너희에게 명하신 언약의 피라 하고
\par 21 또한 이와 같이 피로써 장막과 섬기는 일에 쓰는 모든 그릇에 뿌렸느니라
\par 22 율법을 좇아 거의 모든 물건이 피로써 정결케 되나니 피흘림이 없은즉 사함이 없느니라
\par 23 그러므로 하늘에 있는 것들의 모형은 이런 것들로써 정결케 할 필요가 있었으나 하늘에 있는 그것들은 이런 것들보다 더 좋은 제물로 할지니라
\par 24 그리스도께서는 참 것의 그림자인 손으로 만든 성소에 들어가지 아니하시고 오직 참 하늘에 들어가사 이제 우리를 위하여 하나님 앞에 나타나시고
\par 25 대제사장이 해마다 다른 것의 피로써 성소에 들어가는 것같이 자주 자기를 드리려고 아니하실지니
\par 26 그리하면 그가 세상을 창조할 때부터 자주 고난을 받았어야 할 것이로되 이제 자기를 단번에 제사로 드려 죄를 없게 하시려고 세상 끝에 나타나셨느니라
\par 27 한번 죽는 것은 사람에게 정하신 것이요 그 후에는 심판이 있으리니
\par 28 이와 같이 그리스도도 많은 사람의 죄를 담당하시려고 단번에 드리신 바 되셨고 구원에 이르게 하기 위하여 죄와 상관 없이 자기를 바라는 자들에게 두 번째 나타나시리라

\chapter{10}

\par 1 율법은 장차 오는 좋은 일의 그림자요 참 형상이 아니므로 해마다 늘 드리는 바 같은 제사로는 나아오는 자들을 언제든지 온전케 할 수 없느니라
\par 2 그렇지 아니하면 섬기는 자들이 단번에 정결케 되어 다시 죄를 깨닫는 일이 없으리니 어찌 드리는 일을 그치지 아니하였으리요
\par 3 그러나 이 제사들은 해마다 죄를 생각하게 하는 것이 있나니
\par 4 이는 황소와 염소의 피가 능히 죄를 없이 하지 못함이라
\par 5 그러므로 세상에 임하실 때에 가라사대 하나님이 제사와 예물을 원치아니하시고 오직 나를 위하여 한 몸을 예비하셨도다
\par 6 전체로 번제함과 속죄제는 기뻐하지 아니하시나니
\par 7 이에 내가 말하기를 하나님이여 보시옵소서 두루마리 책에 나를 가리켜 기록한 것과 같이 하나님의 뜻을 행하러 왔나이다 하시니라
\par 8 위에 말씀하시기를 제사와 예물과 전체로 번제함과 속죄제는 원치도 아니하고 기뻐하지도 아니하신다 하셨고(이는 다 율법을 따라 드리는 것이라)
\par 9 그 후에 말씀하시기를 보시옵소서 내가 하나님의 뜻을 행하러 왔나이다 하셨으니 그 첫 것을 폐하심은 둘째 것을 세우려 하심이니라
\par 10 이 뜻을 좇아 예수 그리스도의 몸을 단번에 드리심으로 말미암아 우리가 거룩함을 얻었노라
\par 11 제사장마다 매일 서서 섬기며 자주 같은 제사를 드리되 이 제사는 언제든지 죄를 없게 하지 못하거니와
\par 12 오직 그리스도는 죄를 위하여 한 영원한 제사를 드리시고 하나님 우편에 앉으사
\par 13 그 후에 자기 원수들로 자기 발등상이 되게 하실 때까지 기다리시나니
\par 14 저가 한 제물로 거룩하게 된 자들을 영원히 온전케 하셨느니라
\par 15 또한 성령이 우리에게 증거하시되
\par 16 주께서 가라사대 그 날 후로는 저희와 세울 언약이 이것이라 하시고 내 법을 저희 마음에 두고 저희 생각에 기록하리라 하신후에
\par 17 또 저희 죄와 저희 불법을 내가 다시 기억지 아니하리라 하셨으니
\par 18 이것을 사하셨은즉 다시 죄를 위하여 제사 드릴 것이 없느니라
\par 19 그러므로 형제들아 우리가 예수의 피를 힘입어 성소에 들어갈 담력을 얻었나니
\par 20 그 길은 우리를 위하여 휘장 가운데로 열어 놓으신 새롭고 산 길이요 휘장은 곧 저의 육체니라
\par 21 또 하나님의 집 다스리는 큰 제사장이 계시매
\par 22 우리가 마음에 뿌림을 받아 양심의 악을 깨닫고 몸을 맑은 물로 씻었으나 참 마음과 온전한 믿음으로 하나님께 나아가자
\par 23 또 약속하신 이는 미쁘시니 우리가 믿는 도리의 소망을 움직이지 말고 굳게 잡아
\par 24 서로 돌아보아 사랑과 선행을 격려하며
\par 25 모이기를 폐하는 어떤 사람들의 습관과 같이 하지 말고 오직 권하여 그 날이 가까움을 볼수록 더욱 그리하자
\par 26 우리가 진리를 아는 지식을 받은 후 짐짓 죄를 범한 즉 다시 속죄하는 제사가 없고
\par 27 오직 무서운 마음으로 심판을 기다리는 것과 대적하는 자를 소멸할 맹렬한 불만 있으리라
\par 28 모세의 법을 폐한 자도 두 세 증인을 인하여 불쌍히 여김을 받지 못하고 죽었거든
\par 29 하물며 하나님 아들을 밟고 자기를 거룩하게 한 언약의 피를 부정한 것으로 여기고 은혜의 성령을 욕되게 하는 자의 당연히 받을 형벌이 얼마나 더 중하겠느냐 너희는 생각하라
\par 30 원수 갚는 것이 내게 있으니 내가 갚으리라 하시고 또 다시 주께서 그의 백성을 심판하리라 말씀하신 것을 우리가 아노니
\par 31 살아계신 하나님의 손에 빠져 들어가는 것이 무서울진저
\par 32 전날에 너희가 빛을 받은 후에 고난의 큰 싸움에 참은 것을 생각하라
\par 33 혹 비방과 환난으로써 사람에게 구경거리가 되고 혹 이런 형편에 있는 자들로 사귀는 자 되었으니
\par 34 너희가 갇힌 자를 동정하고 너희 산업을 빼앗기는 것도 기쁘게 당한 것은 더 낫고 영구한 산업이 있는 줄 앎이라
\par 35 그러므로 너희 담대함을 버리지 말라 이것이 큰 상을 얻느니라
\par 36 너희에게 인내가 필요함은 너희가 하나님의 뜻을 행한 후에 약속을 받기 위함이라
\par 37 잠시 잠간 후면 오실 이가 오시리니 지체하지 아니하시리라
\par 38 오직 나의 의인은 믿음으로 말미암아 살리라 또한 뒤로 물러가면 내 마음이 저를 기뻐하지 아니하리라 하셨느니라
\par 39 우리는 뒤로 물러가 침륜에 빠질 자가 아니요 오직 영혼을 구원함에 이르는 믿음을 가진 자니라

\chapter{11}

\par 1 믿음은 바라는 것들의 실상이요 보지 못하는 것들의 증거니
\par 2 선진들이 이로써 증거를 얻었으니라
\par 3 믿음으로 모든 세계가 하나님의 말씀으로 지어진 줄을 우리가 아나니 보이는 것은 나타난 것으로 말미암아 된 것이 아니니라
\par 4 믿음으로 아벨은 가인보다 더 나은 제사를 하나님께 드림으로 의로운 자라 하시는 증거를 얻었으니 하나님이 그 예물에 대하여 증거하심이라 저가 죽었으나 그 믿음으로써 오히려 말하느니라
\par 5 믿음으로 에녹은 죽음을 보지 않고 옮기웠으니 하나님이 저를 옮기심으로 다시 보이지 아니하니라 저는 옮기우기 전에 하나님을 기쁘시게 하는 자라 하는 증거를 받았느니라
\par 6 믿음이 없이는 기쁘시게 못하나니 하나님께 나아가는 자는 반드시 그가 계신 것과 또한 그가 자기를 찾는 자들에게 상 주시는 이심을 믿어야 할지니라
\par 7 믿음으로 노아는 아직 보지 못하는 일에 경고하심을 받아 경외함으로 방주를 예비하여 그 집을 구원하였으니 이로 말미암아 세상을 정죄하고 믿음을 좇는 의의 후사가 되었느니라
\par 8 믿음으로 아브라함은 부르심을 받았을 때에 순종하여 장래 기업으로 받을 땅에 나갈새 갈 바를 알지 못하고 나갔으며
\par 9 믿음으로 저가 외방에 있는 것같이 약속하신 땅에 우거하여 동일한 약속을 유업으로 함께 받은 이삭과 야곱으로 더불어 장막에 거하였으니
\par 10 이는 하나님의 경영하시고 지으실 터가 있는 성을 바랐음이니라
\par 11 믿음으로 사라 자신도 나이 늙어 단산하였으나 잉태하는 힘을 얻었으니 이는 약속하신 이를 미쁘신 줄 앎이라
\par 12 이러므로 죽은 자와 방불한 한 사람으로 말미암아 하늘에 허다한 별과 또 해변의 무수한 모래와 같이 많이 생육하였느니라
\par 13 이 사람들은 다 믿음을 따라 죽었으며 약속을 받지 못하였으되 그것들을 멀리서 보고 환영하며 또 땅에서는 외국인과 나그네로라 증거하였으니
\par 14 이같이 말하는 자들은 본향 찾는 것을 나타냄이라
\par 15 저희가 나온바 본향을 생각하였더면 돌아갈 기회가 있었으려니와
\par 16 저희가 이제는 더 나은 본향을 사모하니 곧 하늘에 있는 것이라 그러므로 하나님이 저희 하나님이라 일컬음 받으심을 부끄러워 아니하시고 저희를 위하여 한 성을 예비하셨느니라
\par 17 아브라함은 시험을 받을 때에 믿음으로 이삭을 드렸으니 저는 약속을 받은 자로되 그 독생자를 드렸느니라
\par 18 저에게 이미 말씀하시기를 네 자손이라 칭할 자는 이삭으로 말미암으리라 하셨으니
\par 19 저가 하나님이 능히 죽은 자 가운데서 다시 살리실 줄로 생각한지라 비유컨대 죽은 자 가운데서 도로 받은 것이니라
\par 20 믿음으로 이삭은 장차 오는 일에 대하여 야곱과 에서에게 축복하였으며
\par 21 믿음으로 야곱은 죽을 때에 요셉의 각 아들에게 축복하고 그 지팡이 머리에 의지하여 경배하였으며
\par 22 믿음으로 요셉은 임종시에 이스라엘 자손들의 떠날 것을 말하고 또 자기 해골을 위하여 명하였으며
\par 23 믿음으로 모세가 났을 때에 그 부모가 아름다운 아이임을 보고 석달 동안 숨겨 임금의 명령을 무서워 아니하였으며
\par 24 믿음으로 모세는 장성하여 바로의 공주의 아들이라 칭함을 거절하고
\par 25 도리어 하나님의 백성과 함께 고난 받기를 잠시 죄악의 낙을 누리는 것보다 더 좋아하고
\par 26 그리스도를 위하여 받는 능욕을 애굽의 모든 보화보다 더 큰 재물로 여겼으니 이는 상 주심을 바라봄이라
\par 27 믿음으로 애굽을 떠나 임금의 노함을 무서워 아니하고 곧 보이지 아니하는 자를 보는 것같이 하여 참았으며
\par 28 믿음으로 유월절과 피 뿌리는 예를 정하였으니 이는 장자를 멸하는 자로 저희를 건드리지 않게 하려 한 것이며
\par 29 믿음으로 저희가 홍해를 육지같이 건넜으나 애굽 사람들은 이것을 시험하다가 빠져 죽었으며
\par 30 믿음으로 칠일 동안 여리고를 두루 다니매 성이 무너졌으며
\par 31 믿음으로 기생 라합은 정탐군을 평안히 영접하였으므로 순종치 아니한 자와 함께 멸망치 아니하였도다
\par 32 내가 무슨 말을 더 하리요 기드온 바락 삼손 입다와 다윗과 사무엘과 및 선지자들의 일을 말하려면 내게 시간이 부족하리로다
\par 33 저희가 믿음으로 나라들을 이기기도 하며 의를 행하기도 하며 약속을 받기도 하며 사자들의 입을 막기도 하며
\par 34 불의 세력을 멸하기도 하며 칼날을 피하기도 하며 연약한 가운데서 강하게 되기도 하며 전쟁에 용맹되어 이방 사람들의 진을 물리치기도 하며
\par 35 여자들은 자기의 죽은 자를 부활로 받기도 하며 또 어떤 이들은 더 좋은 부활을 얻고자 하여 악형을 받되 구차히 면하지 아니하였으며
\par 36 또 어떤 이들은 희롱과 채찍질뿐 아니라 결박과 옥에 갇히는 시험도 받았으며
\par 37 돌로 치는 것과 톱으로 켜는 것과 시험과 칼에 죽는 것을 당하고 양과 염소의 가죽을 입고 유리하여 궁핍과 환난과 학대를 받았으니
\par 38 (이런 사람은 세상이 감당치 못하도다)저희가 광야와 산중과 암혈과 토굴에 유리하였느니라
\par 39 이 사람들이 다 믿음으로 말미암아 증거를 받았으나 약속을 받지 못하였으니
\par 40 이는 하나님이 우리를 위하여 더 좋은 것을 예비하셨은즉 우리가 아니면 저희로 온전함을 이루지 못하게 하려 하심이니라

\chapter{12}

\par 1 이러므로 우리에게 구름같이 둘러싼 허다한 증인들이 있으니 모든 무거운 것과 얽매이기 쉬운 죄를 벗어 버리고 인내로써 우리 앞에 당한 경주를 경주하며
\par 2 믿음의 주요 또 온전케 하시는 이인 예수를 바라보자 저는 그 앞에 있는 즐거움을 위하여 십자가를 참으사 부끄러움을 개의치 아니하시더니 하나님 보좌 우편에 앉으셨느니라
\par 3 너희가 피곤하여 낙심치 않기 위하여 죄인들의 이같이 자기에게 거역한 일을 참으신 자를 생각하라
\par 4 너희가 죄와 싸우되 아직 피 흘리기 까지는 대항치 아니하고
\par 5 또 아들들에게 권하는 것같이 너희에게 권면하신 말씀을 잊었도다 일렀으되 내 아들아 주의 징계 하심을 경히 여기지 말며 그에게 꾸지람을 받을 때에 낙심하지 말라
\par 6 주께서 그 사랑하시는 자를 징계 하시고 그의 받으시는 아들마다 채찍질 하심이니라 하였으니
\par 7 너희가 참음은 징계를 받기 위함이라 하나님이 아들과 같이 너희를 대우 하시나니 어찌 아비가 징계하지 않는 아들이 있으리요
\par 8 징계는 다 받는 것이거늘 너희에게 없으면 사생자요 참 아들이 아니니라
\par 9 또 우리 육체의 아버지가 우리를 징계하여도 공경하였거늘 하물며 모든 영의 아버지께 더욱 복종하여 살려 하지 않겠느냐
\par 10 저희는 잠시 자기의 뜻대로 우리를 징계하였거니와 오직 하나님은 우리의 유익을 위하여 그의 거룩하심에 참예케 하시느니라
\par 11 무릇 징계가 당시에는 즐거워 보이지 않고 슬퍼 보이나 후에 그로 말미암아 연달한 자에게는 의의 평강한 열매를 맺나니
\par 12 그러므로 피곤한 손과 연약한 무릎을 일으켜 세우고
\par 13 너희 발을 위하여 곧은 길을 만들어 저는 다리로 하여금 어그러지지 않고 고침을 받게하라
\par 14 모든 사람으로 더불어 화평함과 거룩함을 좆으라 이것이 없이는 아무도 주를 보지 못하리라
\par 15 너희는 돌아보아 하나님 은혜에 이르지 못하는 자가 있는가 두려워하고 또 쓴 뿌리가 나서 괴롭게 하고 많은 사람이 이로 말미암아 더러움을 입을까 두려워하고
\par 16 음행하는 자와 혹 한 그릇 식물을 위하여 장자의 명분을 판 에서와 같이 망령된자가 있을까 두려워 하라
\par 17 너희의 아는 바와 같이 저가 그후에 축복을 기업으로 받으려고 눈물을 흘리며 구하되 버린 바가 되어 회개할 기회를 얻지 못하였느니라
\par 18 너희의 이른 곳은 만질만한 불 붙는 산과 흑운과 흑암과 폭풍과
\par 19 나팔소리와 말하는 소리가 아니라 그 소리를 듣는 자들은 더 말씀 하지 아니하시기를 구하였으니
\par 20 이는 짐승이라도 산에 이르거든 돌로 침을 당하리라 하신 명을 저희가 견디지 못함이라
\par 21 그 보이는 바가 이렇듯이 무섭기로 모세도 이르되 내가 심히 두렵고 떨린다 하였으나
\par 22 그러나 너희가 이른 곳은 시온산과 살아계신 하나님의 도성인 하늘의 예루살렘과 천만 천사와
\par 23 하늘에 기록한 장자들의 총회와 교회와 만민의 심판자이신 하나님과 및 온전케 된 의인의 영들과
\par 24 새 언약의 중보이신 예수와 및 아벨의 피보다 더 낫게 말하는 뿌린 피니라
\par 25 너희는 삼가 말하신 자를 거역 하지 말라 땅에서 경고하신 자를 거역한 저희가 피하지 못하였거든 하물며 하늘로 좇아 경고하신 자를 배반하는 우리일까보냐
\par 26 그 때에는 그 소리가 땅을 진동하였거니와 이제는 약속하여 가라사대 내가 또 한번 땅만 아니라 하늘도 진동하리라 하셨느니라
\par 27 이 또 한번이라 하심은 진동치 아니하는 것을 영존케 하기 위하여 진동할 것들 곧 만든 것들의 변동될 것을 나타내심이니라
\par 28 그러므로 우리가 진동치 못할 나라를 받았은즉 은혜를 받자 이로 말미암아 경건함과 두려움으로 하나님을 기쁘시게 섬길지니
\par 29 우리 하나님은 소멸하는 불이심이니라

\chapter{13}

\par 1 형제 사랑하기를 계속하고
\par 2 손님 대접하기를 잊지 말라 이로써 부지중에 천사들을 대접한 이들이 있었느니라
\par 3 자기도 함께 갇힌 것같이 갇힌자를 생각하고 자기도 몸을 가졌은즉 학대 받는 자를 생각하라
\par 4 모든 사람은 혼인을 귀히 여기고 침소를 더럽히지 않게 하라 음행하는 자들과 간음하는 자들을 하나님이 심판하시리라
\par 5 돈을 사랑치 말고 있는 바를 족한 줄로 알라 그가 친히 말씀하시기를 내가 과연 너희를 버리지 아니하고 과연 너희를 떠나지 아니하리라 하셨느니라
\par 6 그러므로 우리가 담대히 가로되 주는 나를 돕는 자시니 내가 무서워 아니하겠노라 사람이 내게 어찌하리요 하노라
\par 7 하나님의 말씀을 너희에게 이르고 너희를 인도하던 자들을 생각하며 저희 행실의 종말을 주의하여 보고 저희 믿음을 본받으라
\par 8 예수 그리스도는 어제나 오늘이나 영원토록 동일하시니라
\par 9 여러가지 다른 교훈에 끌리지 말라 마음은 은혜로써 굳게 함이 아름답고 식물로써 할것이 아니니 식물로 말미암아 행한자는 유익을 얻지 못하였느니라
\par 10 우리에게 제단이 있는데 그 위에 있는 제물을 장막에서 섬기는 자들이 이 제단에게 먹을 권이 없나니
\par 11 이는 죄를 위한 짐승의 피는 대제사장이 가지고 성소에 들어가고 그 육체는 영문밖에서 불사름이니라
\par 12 그러므로 예수도 자기 피로써 백성을 거룩케 하려고 성문밖에서 고난을 받으셨느니라
\par 13 그런즉 우리는 그 능욕을 지고 영문밖으로 그에게 나아가서
\par 14 우리가 여기는 영구한 도성이 없고 오직 장차 올 것을 찾나니
\par 15 이러므로 우리가 예수로 말미암아 항상 찬미의 제사를 하나님께 드리자 이는 그 이름을 증거하는 입술의 열매니라
\par 16 오직 선을 행함과 서로 나눠 주기를 잊지말라 이같은 제사는 하나님이 기뻐 하시느니라
\par 17 너희를 인도하는 자들에게 순종하고 복종하라 저희는 너희 영혼을 위하여 경성하기를 자기가 회계할 자인 것 같이 하느니라 저희로 하여금 즐거움으로 이것을 하게하고 근심으로 하게말라 그렇지 않으면 너희에게 유익이 없느니라
\par 18 우리를 위하여 기도하라 우리가 모든일에 선하게 행하려 하므로 우리에게 선한 양심이 있는 줄을 확신하노라
\par 19 내가 더 속히 너희에게 돌아가기를 위하여 너희 기도함을 더욱 원하노라
\par 20 양의 큰 목자이신 우리 주 예수를 영원한 언약의 피로 죽은 자 가운데서 이끌어 내신 평강의 하나님이
\par 21 모든 선한 일에 너희를 온전케 하사 자기뜻을 행하게 하시고 그앞에 즐거운 것을 예수 그리스도로 말미암아 우리속에 이루시기를 원하노라 영광이 그에게 세세 무궁토록 있을지어다 아멘
\par 22 형제들아 내가 너희를 권하노니 권면의 말을 용납하라 내가 간단히 너희에게 썼느니라
\par 23 우리형제 디모데가 놓인 것을 너희가 알라 그가 속히 오면 내가 저와 함께가서 너희를 보리라
\par 24 너희를 인도하는 자와 및 모든 성도에게 문안하라 이달리랴에서 온 자들도 너희에게 문안 하느니라
\par 25 은혜가 너희 모든 사람에게 있을지어다


\end{document}