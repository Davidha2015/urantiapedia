\begin{document}

\title{Pradžios knyga}


\chapter{1}


\par 1 Pradžioje Dievas sutvėrė dangų ir žemę. 
\par 2 Žemė buvo be pavidalo ir tuščia, tamsa gaubė gelmes, ir Dievo Dvasia sklandė virš vandenų. 
\par 3 Dievas tarė: “Teatsiranda šviesa!” Ir atsirado šviesa. 
\par 4 Dievas matė šviesą ir, kad tai buvo gerai, ir Dievas atskyrė šviesą nuo tamsos. 
\par 5 Dievas pavadino šviesą diena, o tamsą naktimi. Tai buvo vakaras ir rytas­ pirmoji diena. 
\par 6 Dievas tarė: “Teatsiranda tvirtuma tarp vandenų, ir ji teatskiria vandenis nuo vandenų!” 
\par 7 Dievas padarė tvirtumą ir atskyrė vandenis, kurie buvo po tvirtuma, nuo vandenų, kurie buvo virš tvirtumos. Ir taip įvyko. 
\par 8 Dievas pavadino tvirtumą dangumi. Tai buvo vakaras ir rytas­antroji diena. 
\par 9 Dievas tarė: “Tesusirenka vandenys, kurie yra po dangumi, į vieną vietą ir tepasirodo sausuma!” Ir taip įvyko. 
\par 10 Dievas pavadino sausumą žeme, o vandenų samplūdį­jūromis. Ir Dievas matė, kad tai buvo gerai. 
\par 11 Dievas tarė: “Tegul žemė išaugina žolę, augalus, duodančius sėklą, ir vaismedžius, nešančius vaisių pagal jų rūšį, kuriuose yra jų sėkla!” Ir taip įvyko. 
\par 12 Žemė išaugino žolę, augalus, duodančius sėklą pagal jų rūšį, ir medžius, nešančius vaisius pagal jų rūšį, kuriuose yra jų sėkla. Ir Dievas matė, kad tai buvo gerai. 
\par 13 Tai buvo vakaras ir rytas­ trečioji diena. 
\par 14 Dievas tarė: “Teatsiranda šviesos dangaus tvirtumoje dienai nuo nakties atskirti ir tebūna jos ženklai pažymėti laikus, dienas ir metus. 
\par 15 Jos težiba dangaus tvirtumoje ir apšviečia žemę!” Ir taip įvyko. 
\par 16 Dievas padarė dvi dideles šviesas: didesniąją­ dienai ir mažesniąją nakčiai valdyti, ir taip pat žvaigždes. 
\par 17 Dievas išdėstė jas dangaus tvirtumoje, kad šviestų žemei, 
\par 18 valdytų dieną bei naktį ir atskirtų šviesą nuo tamsos. Ir Dievas matė, kad tai buvo gerai. 
\par 19 Tai buvo vakaras ir rytas­ ketvirtoji diena. 
\par 20 Dievas tarė: “Tegul vandenys knibždėte knibžda gyvūnais ir paukščiai teskraido virš žemės, padangėse!” 
\par 21 Taip Dievas sutvėrė didelius jūros gyvūnus ir visus kitus gyvius, kurie atsirado iš vandens, ir visus paukščius pagal jų rūšį. Ir Dievas matė, kad tai buvo gerai. 
\par 22 Dievas juos palaimino, tardamas: “Būkite vaisingi, dauginkitės ir pripildykite vandenis jūrose, o paukščiai tepripildo žemę!” 
\par 23 Tai buvo vakaras ir rytas­ penktoji diena. 
\par 24 Dievas tarė: “Tegul žemė išaugina gyvūnus pagal jų rūšį: gyvulius, roplius ir laukinius žvėris, kiekvieną pagal savo rūšį!” Ir taip įvyko. 
\par 25 Dievas padarė laukinius žvėris, gyvulius ir visokius roplius, kiekvieną pagal jų rūšį. Ir Dievas matė, kad tai buvo gerai. 
\par 26 Dievas tarė: “Padarykime žmogų pagal mūsų atvaizdą ir panašumą. Jie tevaldo jūros žuvis, padangių paukščius, gyvulius ir visą žemę bei visus roplius, kurie gyvena ant žemės!” 
\par 27 Ir Dievas sutvėrė žmogų pagal savo atvaizdą; pagal Dievo atvaizdą sutvėrė Jis jį; vyrą ir moterį sutvėrė Jis. 
\par 28 Dievas juos palaimino ir tarė: “Būkite vaisingi ir dauginkitės, pripildykite žemę ir užvaldykite ją, viešpataukite jūros žuvims, padangių paukščiams ir kiekvienam gyvam padarui, kuris kruta ant žemės!” 
\par 29 Dievas tarė: “Aš jums daviau įvairias žoles, turinčias sėklą, kurios auga žemės paviršiuje, ir visus medžius, kurių vaisius turi sėklą; jums tebūna tai maistas. 
\par 30 Ir visiems žemės gyvūnams, visiems padangių paukščiams ir visiems, kas kruta ant žemės, kas turi gyvybę, daviau visus žaliuojančius augalus maistui”. Ir taip įvyko. 
\par 31 Dievas matė visa, ką buvo padaręs, ir tai buvo labai gerai. Buvo vakaras ir rytas­ šeštoji diena.





\chapter{2}


\par 1 Taip buvo sutvertas dangus, žemė ir visi jų pulkai. 
\par 2 Dievas septintą dieną užbaigė savo darbus ir ilsėjosi septintą dieną po visų savo darbų, kuriuos atliko. 
\par 3 Dievas palaimino septintą dieną ir ją pašventino, nes joje ilsėjosi po visų savo darbų, kuriuos Dievas sukūrė ir padarė. 
\par 4 Tokia yra dangaus ir žemės kilmė, kai jie buvo sukurti tą dieną, kurią Viešpats Dievas sutvėrė žemę ir dangų, 
\par 5 ir visus lauko augalus, kurių dar nebuvo žemėje, ir visas lauko žoles, kurios dar nežėlė; nes Viešpats Dievas nesiuntė į žemę lietaus ir nebuvo žmogaus žemei įdirbti. 
\par 6 Migla kilo nuo žemės ir drėkino jos paviršių. 
\par 7 Ir Viešpats Dievas padarė žmogų iš žemės dulkių ir įkvėpė į jo šnerves gyvybės kvapą. Taip žmogus tapo gyva siela. 
\par 8 Viešpats Dievas sukūrė sodą Edene rytuose ir ten apgyvendino žmogų, kurį buvo sutvėręs. 
\par 9 Viešpats Dievas išaugino iš žemės visokių medžių, gražių pasižiūrėti ir nešančių gerus vaisius maistui; taip pat gyvybės medį sodo viduryje ir medį pažinimo gero ir blogo. 
\par 10 Upė tekėjo iš Edeno sodui drėkinti; nuo ten ji šakojosi į keturias upes. 
\par 11 Pirmosios vardas Pišonas. Ji teka aplink visą Havilos šalį, kur randamas auksas. 
\par 12 Tos šalies auksas yra geras. Ten randa bdeliją ir onikso akmenį. 
\par 13 Antrosios upės vardas Gihonas. Ji teka aplink visą Kušo šalį. 
\par 14 Trečiosios upės vardas Tigras. Ji teka į rytus nuo Asūro. O ketvirtoji upė yra Eufratas. 
\par 15 Ir paėmė Viešpats Dievas žmogų ir apgyvendino jį Edeno sode, kad žmogus jį įdirbtų ir prižiūrėtų. 
\par 16 Viešpats Dievas įsakė žmogui: “Nuo kiekvieno sodo medžio tau leista valgyti, 
\par 17 bet nuo medžio pažinimo gero ir blogo nevalgyk, nes tą dieną, kurią valgysi jo vaisių, tikrai mirsi”. 
\par 18 Viešpats Dievas tarė: “Negerai žmogui būti vienam. Aš padarysiu jam tinkamą padėjėją”. 
\par 19 Viešpats Dievas, padaręs iš žemės visus žvėris bei padangių paukščius, juos atvedė prie Adomo, kad matytų, kaip jis juos pavadins; kaip Adomas pavadino kiekvieną gyvą padarą, toks ir yra jo vardas. 
\par 20 Adomas davė vardus visiems gyvuliams, padangių paukščiams ir visiems lauko žvėrims, tačiau tarp jų neatsirado padėjėjo, tinkamo žmogui. 
\par 21 Tada Viešpats Dievas giliai užmigdė Adomą, išėmė vieną jo šonkaulių ir tą vietą užpildė kūnu. 
\par 22 Po to Viešpats Dievas iš šonkaulio, kurį išėmė iš žmogaus, padarė moterį ir ją atvedė pas žmogų. 
\par 23 Tada Adomas tarė: “Štai kaulas iš mano kaulų ir kūnas iš mano kūno! Šita bus vadinama moterimi, nes iš vyro ji paimta”. 
\par 24 Todėl vyras paliks savo tėvą bei motiną ir susijungs su savo žmona; ir juodu taps vienu kūnu. 
\par 25 Jie abu­žmogus ir jo žmona­buvo nuogi, tačiau nesigėdijo.



\chapter{3}

\par 1 Gyvatė buvo gudresnė už visus žemės gyvūnus, kuriuos Viešpats Dievas sutvėrė. Ji tarė moteriai: “Ar tikrai Dievas pasakė: ‘Nevalgykite nuo visų sodo medžių’?” 
\par 2 Moteris atsakė gyvatei: “Mums leista valgyti sodo medžių vaisius, 
\par 3 išskyrus vaisius medžio, kuris yra sodo viduryje. Dievas įsakė: ‘Nevalgykite nuo jo ir nelieskite jo, kad nemirtumėte’ ”. 
\par 4 Gyvatė atsakė: “Nemirsite! 
\par 5 Dievas žino, kad tą dieną, kurią valgysite nuo jo, atsivers jūsų akys ir jūs tapsite kaip dievai, pažindami gera ir bloga”. 
\par 6 Kai moteris pamatė, kad medžio vaisiai yra tinkami maistui, patrauklūs akims ir, vieną suvalgius, galima įsigyti išminties, ji paėmė jo vaisių, pati valgė ir davė savo vyrui, ir jis valgė. 
\par 7 Tada atsivėrė abiejų akys ir jie suprato esą nuogi; juodu supynė figmedžio lapus ir pasidarė prijuostes. 
\par 8 Dienai atvėsus, išgirdę Viešpaties Dievo, vaikščiojančio sode, balsą, Adomas ir jo žmona pasislėpė nuo Viešpaties Dievo veido tarp sodo medžių. 
\par 9 Viešpats Dievas pašaukė Adomą: “Kur tu esi?” 
\par 10 O tas atsiliepė: “Išgirdau Tavo balsą ir, išsigandęs, kad esu nuogas, pasislėpiau”. 
\par 11 Dievas tarė: “Kas tau pasakė, kad tu nuogas? Gal valgei nuo medžio, nuo kurio tau įsakiau nevalgyti?” 
\par 12 Žmogus atsakė: “Moteris, kurią Tu man davei, davė man nuo to medžio, ir aš valgiau”. 
\par 13 Tada Viešpats Dievas tarė moteriai: “Kodėl tu taip padarei?” Moteris atsakė: “Gyvatė mane apgavo, ir aš valgiau”. 
\par 14 Tada Viešpats Dievas tarė gyvatei: “Kadangi taip padarei, esi prakeikta tarp visų gyvulių ir laukinių žvėrių. Tu slinksi pilvu ir dulkes ėsi per visą savo gyvenimą! 
\par 15 Aš sukelsiu priešiškumą tarp tavęs ir moters, tarp tavo sėklos ir moters sėklos. Ji sutrins tau galvą, o tu gelsi jai į kulnį”. 
\par 16 Moteriai Jis tarė: “Aš padauginsiu tavo nėštumo vargus ir su skausmu tu gimdysi vaikus; tave trauks prie tavo vyro, o jis tau viešpataus”. 
\par 17 O Adomui Jis tarė: “Kadangi tu paklausei savo žmonos ir valgei nuo medžio, apie kurį tau buvau įsakęs: ‘Nevalgyk nuo jo’,­prakeikta bus žemė dėl tavęs! Vargdamas turėsi maitintis iš jos visą savo gyvenimą. 
\par 18 Erškėčius ir usnis ji augins tau, ir tu valgysi lauko augalus. 
\par 19 Valgysi prakaitu uždirbtą duoną, kol sugrįši į žemę, iš kurios esi paimtas. Esi dulkė ir dulke vėl pavirsi”. 
\par 20 Adomas pavadino savo žmoną Ieva, nes ji tapo visų gyvųjų motina. 
\par 21 Viešpats Dievas padarė Adomui ir jo žmonai kailinius rūbus ir jais apvilko juos. 
\par 22 Tada Viešpats Dievas tarė: “Štai žmogus tapo kaip vienas iš mūsų, pažindamas gera ir bloga; ir dabar, kad jis, ištiesęs savo ranką, neskintų nuo gyvybės medžio ir nevalgytų, ir negyventų per amžius”. 
\par 23 Todėl Viešpats Dievas išvarė jį iš Edeno sodo dirbti žemę, iš kurios jis buvo paimtas. 
\par 24 Išvaręs žmogų, į rytus nuo Edeno sodo Viešpats pastatė cherubus su švytruojančiu ugniniu kardu saugoti kelią prie gyvybės medžio.



\chapter{4}

\par 1 Ir Adomas pažino savo žmoną Ievą, ir ji tapo nėščia. Ji pagimdė Kainą ir tarė: “Įsigijau sūnų Viešpaties pagalba”. 
\par 2 Ji dar pagimdė jo brolį Abelį. Abelis buvo avių piemuo, o Kainas­žemdirbys. 
\par 3 Kuriam laikui praėjus, Kainas aukojo Viešpačiui iš žemės vaisių. 
\par 4 Taip pat ir Abelis aukojo iš savo bandos riebiausių pirmagimių. Viešpats pažvelgė į Abelį ir jo auką, 
\par 5 tačiau į Kainą ir jo auką Jis nepažvelgė. Todėl Kainas labai supyko, ir jo veidas paniuro. 
\par 6 Viešpats tarė Kainui: “Kodėl tu supykai ir tavo veidas paniuro? 
\par 7 Darydamas gera, argi nebūsi priimtas? O jei gera nedarai, nuodėmė tyko prie durų. Ji traukia tave, tačiau tu turi viešpatauti jai”. 
\par 8 Kainas kalbėjo savo broliui Abeliui. Jiems esant laukuose, Kainas užpuolė savo brolį Abelį ir jį užmušė. 
\par 9 Tada Viešpats paklausė Kaino: “Kur yra tavo brolis Abelis?” O jis atsakė: “Nežinau. Argi aš esu savo brolio sargas?” 
\par 10 Tada Viešpats tarė: “Ką padarei? Tavo brolio kraujas šaukiasi manęs nuo žemės. 
\par 11 Taigi dabar esi prakeiktas ant žemės, kuri atsivėrė ir priėmė iš tavo rankos tavo brolio kraują. 
\par 12 Kai tu ją dirbsi, ji nebeduos tau derliaus. Tu būsi klajūnas ir benamis žemėje”. 
\par 13 Tada Kainas tarė Viešpačiui: “Mano bausmė yra per didelė, kad galėčiau ją pakelti. 
\par 14 Tu šiandien mane išvarai iš žemės. Aš turėsiu slėptis nuo Tavęs ir būsiu klajūnas ir benamis žemėje. Kas mane sutiks, užmuš”. 
\par 15 Viešpats jam atsakė: “Kas užmuš Kainą, tam septyneriopai bus atkeršyta!” Viešpats paženklino Kainą žyme, kad nė vienas, sutikęs jį, jo nenužudytų. 
\par 16 Kainas pasitraukė iš Viešpaties akivaizdos ir apsigyveno Nodo šalyje, į rytus nuo Edeno. 
\par 17 Kainas pažino savo žmoną, ji pastojo ir pagimdė Henochą. Kainas pastatė miestą ir tą miestą pavadino savo sūnaus vardu­Henochas. 
\par 18 Henocho sūnus buvo Iradas, Irado sūnus­Mehujaelis, Mehujaelio­Metušaelis, Metušaelio­ Lamechas. 
\par 19 Lamechas vedė dvi žmonas. Pirmosios vardas buvo Ada, antrosios­Cila. 
\par 20 Ada pagimdė Jabalą; jis buvo tėvas tų, kurie gyvena palapinėse ir laiko gyvulius. 
\par 21 Jo brolis, vardu Jubalas, buvo arfininkų ir vamzdininkų tėvas. 
\par 22 Cila pagimdė Tubal Kainą, kuris gamino visokius įrankius iš vario ir geležies. Tubal Kaino sesuo buvo Naama. 
\par 23 Lamechas tarė savo žmonoms: “Ada ir Cila, klausykite! Jūs, Lamecho žmonos, įsidėmėkite, ką sakau: aš užmušiau vyrą už man padarytą žaizdą, jaunuolį už randą! 
\par 24 Jei už Kainą bus atkeršyta septyneriopai, tai už Lamechą­septyniasdešimt septynis kartus!” 
\par 25 Adomas vėl pažino savo žmoną, ir ji pagimdė sūnų, vardu Setas, sakydama: “Dievas man davė kitą sūnų vietoje Abelio, kurį Kainas užmušė”. 
\par 26 Taip pat ir Setas turėjo sūnų, vardu Enas. Tuomet žmonės pradėjo šauktis Viešpaties vardo.



\chapter{5}

\par 1 Šita yra Adomo palikuonių knyga. Kai Dievas sutvėrė žmogų, Jis padarė jį panašų į Dievą. 
\par 2 Jis sutvėrė vyrą ir moterį, palaimino juos ir pavadino juos Adomu tą dieną, kai jie buvo sutverti. 
\par 3 Kai Adomas buvo šimto trisdešimties metų, jam gimė sūnus pagal jo panašumą ir atvaizdą, kurį pavadino Setu. 
\par 4 Po Seto gimimo Adomas dar gyveno aštuonis šimtus metų ir susilaukė sūnų bei dukterų. 
\par 5 Taigi Adomo amžius buvo devyni šimtai trisdešimt metų, ir jis mirė. 
\par 6 Setas, būdamas šimto penkerių metų, susilaukė Eno. 
\par 7 Setas, gimus Enui, dar gyveno aštuonis šimtus septynerius metus ir susilaukė sūnų bei dukterų. 
\par 8 Taigi Seto amžius buvo devyni šimtai dvylika metų, ir jis mirė. 
\par 9 Enas, būdamas devyniasdešimties metų, susilaukė Kainamo. 
\par 10 Enas, gimus Kainamui, dar gyveno aštuonis šimtus penkiolika metų ir susilaukė sūnų bei dukterų. 
\par 11 Taigi Eno amžius buvo devyni šimtai penkeri metai, ir jis mirė. 
\par 12 Kainamas, būdamas septyniasdešimties metų, susilaukė Malaleelio. 
\par 13 Kainamas, gimus Malaleeliui, dar gyveno aštuonis šimtus keturiasdešimt metų ir susilaukė sūnų bei dukterų. 
\par 14 Taigi Kainamo amžius buvo devyni šimtai dešimt metų, ir jis mirė. 
\par 15 Malaleelis, būdamas šešiasdešimt penkerių metų, susilaukė Jareto. 
\par 16 Malaleelis, gimus Jaretui, dar gyveno aštuonis šimtus trisdešimt metų ir susilaukė sūnų bei dukterų. 
\par 17 Taigi Malaleelio amžius buvo aštuoni šimtai devyniasdešimt penkeri metai, ir jis mirė. 
\par 18 Jaretas, būdamas šimto šešiasdešimt dvejų metų, susilaukė Henocho. 
\par 19 Jaretas, gimus Henochui, dar gyveno aštuonis šimtus metų ir susilaukė sūnų bei dukterų. 
\par 20 Taigi Jareto amžius buvo devyni šimtai šešiasdešimt dveji metai, ir jis mirė. 
\par 21 Henochas, būdamas šešiasdešimt penkerių metų, susilaukė Matūzalio. 
\par 22 Henochas, gimus Matūzaliui, vaikščiojo su Dievu tris šimtus metų ir susilaukė sūnų bei dukterų. 
\par 23 Taigi Henocho amžius buvo trys šimtai šešiasdešimt penkeri metai. 
\par 24 Henochas vaikščiojo su Dievu, ir jo nebeliko, nes Dievas jį pasiėmė. 
\par 25 Matūzalis, būdamas šimto aštuoniasdešimt septynerių metų, susilaukė Lamecho. 
\par 26 Matūzalis, gimus Lamechui, dar gyveno septynis šimtus aštuoniasdešimt dvejus metus ir susilaukė sūnų bei dukterų. 
\par 27 Taigi Matūzalio amžius buvo devyni šimtai šešiasdešimt devyneri metai, ir jis mirė. 
\par 28 Lamechas, būdamas šimto aštuoniasdešimt dvejų metų, susilaukė sūnaus, 
\par 29 kurį jis pavadino Nojumi, sakydamas: “Šitas mus paguos mūsų darbuose ir mūsų rankų triūse žemėje, kurią Viešpats prakeikė”. 
\par 30 Lamechas, gimus Nojui, dar gyveno penkis šimtus devyniasdešimt penkerius metus ir susilaukė sūnų bei dukterų. 
\par 31 Taigi Lamecho amžius buvo septyni šimtai septyniasdešimt septyneri metai, ir jis mirė. 
\par 32 Nojus, būdamas penkių šimtų metų, susilaukė Semo, Chamo ir Jafeto.



\chapter{6}

\par 1 Kai žmonių padaugėjo žemėje ir jiems gimė dukterų, 
\par 2 Dievo sūnūs matydami, kad žmonių dukterys gražios, ėmė jas sau į žmonas. 
\par 3 Tada Viešpats tarė: “Mano dvasia nekovos su žmonėmis amžinai, nes jie tėra kūnas; jų dienos bus šimtas dvidešimt metų!” 
\par 4 Anomis dienomis žemėje buvo milžinų. Kai Dievo sūnūs vesdavo žmonių dukteris ir jos pagimdydavo jiems vaikų, jie būdavo galiūnais, senovėje garsiais vyrais. 
\par 5 Viešpats, matydamas, kad žmonių nedorybės žemėje buvo didelės ir jų širdies siekiai buvo vien tik pikti, 
\par 6 gailėjosi, kad Jis žemėje sutvėrė žmogų, ir sielojosi savo širdyje. 
\par 7 Ir Dievas tarė: “Aš išnaikinsiu žmones, kuriuos sutvėriau, nuo žemės paviršiaus; tiek žmones, tiek gyvulius, roplius ir padangių paukščius, nes Aš gailiuosi, kad juos padariau”. 
\par 8 Tačiau Nojus rado malonę Viešpaties akyse. 
\par 9 Tokia yra Nojaus giminės istorija. Nojus buvo teisus ir tobulas vyras savo kartoje; jis vaikščiojo su Dievu. 
\par 10 Nojus turėjo tris sūnus: Semą, Chamą ir Jafetą. 
\par 11 Dievo akivaizdoje žemė buvo sugedusi ir pilna nusikaltimų. 
\par 12 Dievas pažiūrėjo į žemę ir matė, kad ji sugedusi, nes kiekvienas kūnas žemėje iškreipė savo kelius. 
\par 13 Ir Dievas tarė Nojui: “Aš nusprendžiau padaryti galą kiekvienam kūnui, nes per juos žemė pasidarė pilna nusikaltimų. Aš išnaikinsiu juos nuo žemės paviršiaus. 
\par 14 Pasidaryk arką iš sakuoto medžio, arkoje padaryk pertvaras ir ištepk ją derva iš vidaus ir iš lauko. 
\par 15 Arka turi būti trijų šimtų uolekčių ilgio, penkiasdešimties uolekčių pločio ir trisdešimties uolekčių aukščio. 
\par 16 Padaryk arkai langą uolektis nuo viršaus; arkos duris padaryk jos šone; įrenk joje apatinį, vidurinį ir viršutinį aukštus. 
\par 17 Aš užtvindysiu žemę vandenimis, kad išnaikinčiau kiekvieną kūną, kuriame yra gyvybė. Visa, kas yra žemėje, pražus. 
\par 18 Bet Aš padarysiu sandorą su tavimi. Į arką įeisite tu, tavo sūnūs, tavo žmona ir tavo sūnų žmonos su tavimi. 
\par 19 Į arką įsivesk po du kiekvienos rūšies gyvius­patiną ir patelę, kad išliktų gyvi su tavimi. 
\par 20 Iš paukščių, iš gyvulių ir iš visų žemės roplių pagal jų rūšį teįeina pas tave po du, kad išliktų gyvi. 
\par 21 Pasiimk visokio maisto ir susikrauk į arką, ir tai tebūna maistas tau ir jiems!” 
\par 22 Nojus padarė viską, ką Dievas jam įsakė.



\chapter{7}


\par 1 Viešpats tarė Nojui: “Eik į arką tu ir visi tavo artimieji, nes tave radau teisų šioje kartoje. 
\par 2 Pasiimk iš visų švarių gyvulių po septynetą patinų ir patelių, o iš visų nešvarių gyvulių­po patiną ir patelę, 
\par 3 taip pat ir iš padangių paukščių po septynetą patinų ir patelių, kad išlaikytum jų rūšį visos žemės paviršiuje. 
\par 4 Po septynių dienų Aš užleisiu lietų ant žemės keturiasdešimčiai dienų ir keturiasdešimčiai naktų ir išnaikinsiu žemės paviršiuje visas gyvas būtybes, kurias esu padaręs”. 
\par 5 Ir Nojus padarė viską, ką Viešpats jam įsakė. 
\par 6 Nojus buvo šešių šimtų metų, kai tvanas prasidėjo. 
\par 7 Tuomet Nojus, jo sūnūs, jo žmona ir sūnų žmonos suėjo su juo į arką, gelbėdamiesi nuo tvano. 
\par 8 Švarių ir nešvarių gyvulių, paukščių ir roplių 
\par 9 poros suėjo į Nojaus arką, kaip Dievas buvo įsakęs Nojui. 
\par 10 Septynioms dienoms praėjus, tvano vandenys užplūdo žemę. 
\par 11 Nojui sulaukus šešių šimtų metų, antro mėnesio septynioliktą dieną pratrūko visi didžiosios gelmės šaltiniai ir dangaus langai atsidarė. 
\par 12 Lijo keturiasdešimt dienų ir keturiasdešimt naktų. 
\par 13 Tą pačią dieną į arką įėjo Nojus ir jo sūnūs: Semas, Chamas ir Jafetas, Nojaus žmona ir trys jo sūnų žmonos. 
\par 14 Jie ir visi žvėrys, galvijai, ropliai ir visi paukščiai pagal savo rūšis 
\par 15 suėjo į Nojaus arką, po du visų kūnų, turinčių gyvybės kvapą. 
\par 16 Įėjusieji buvo patinas ir patelė kiekvieno kūno, kaip Dievas jam buvo įsakęs. Tuomet Viešpats uždarė arką iš lauko pusės. 
\par 17 Lietus tęsėsi keturiasdešimt dienų, vandens daugėjo ir jis pakėlė arką aukštyn nuo žemės. 
\par 18 Vanduo kilo ir užplūdo žemę, o arka plūduriavo vandens paviršiuje. 
\par 19 Vanduo pakilo taip aukštai, kad apsėmė visus aukštuosius kalnus, kurie stūksojo po dangumi. 
\par 20 Vanduo pakilo penkiolika uolekčių virš kalnų ir juos apdengė. 
\par 21 Ir žuvo kiekvienas kūnas, kuris judėjo ant žemės: paukščiai, galvijai, žvėrys, ropliai ir visi žmonės. 
\par 22 Visa, kas buvo gyva ir gyveno sausumoje, išmirė. 
\par 23 Taip Dievas išnaikino visas gyvas būtybes, kurios buvo žemės paviršiuje, tiek žmones, tiek gyvulius, roplius ir padangių paukščius. Išliko tik Nojus ir tie, kurie buvo su juo arkoje. 
\par 24 Vanduo laikėsi žemėje šimtą penkiasdešimt dienų.



\chapter{8}

\par 1 Dievas atsiminė Nojų, visus žvėris bei visus gyvulius, kurie buvo su juo arkoje. Jis leido vėjui pūsti, ir vanduo pradėjo slūgti. 
\par 2 Užsidarė gelmės šaltiniai bei dangaus langai, ir lietus liovėsi. 
\par 3 Tada vandenys pamažu seko žemėje. Šimtui penkiasdešimčiai dienų praėjus, vandens ėmė mažėti. 
\par 4 Septinto mėnesio septynioliktą dieną arka sustojo Ararato kalnuose. 
\par 5 O vandenys nuolat seko iki dešimtojo mėnesio. Dešimtojo mėnesio pirmąją dieną pasirodė kalnų viršūnės. 
\par 6 Praėjus keturiasdešimčiai dienų, Nojus atidarė arkos langą, kurį buvo įstatęs, 
\par 7 ir išleido varną. Tas skraidė šen ir ten, kol vandenys nuseko ant žemės. 
\par 8 Po to jis išleido balandį, norėdamas sužinoti, ar vandenys jau nusekę žemės paviršiuje. 
\par 9 Balandis nerado vietos, kur nutūpęs galėtų pailsėti, ir sugrįžo, nes vanduo tebebuvo apsėmęs visą žemės paviršių. Nojus ištiesė ranką ir paėmė jį į arką. 
\par 10 Palaukęs dar septynias dienas, jis vėl išleido iš arkos balandį. 
\par 11 Vakare balandis sugrįžo, laikydamas snape šviežiai nuskintą alyvmedžio lapą. Taip Nojus sužinojo, kad vandenys nusekę ant žemės. 
\par 12 Palaukęs dar kitas septynias dienas, jis vėl išleido balandį, kuris daugiau nebesugrįžo. 
\par 13 Šeši šimtai pirmaisiais metais, pirmojo mėnesio pirmąją dieną vanduo visai nuseko ant žemės. Tada Nojus nuėmė arkos dangtį ir pasižiūrėjo; ir štai žemės paviršius buvo nudžiūvęs. 
\par 14 Antrojo mėnesio dvidešimt septintąją dieną žemė buvo sausa. 
\par 15 Tada Dievas tarė Nojui: 
\par 16 “Išeik iš arkos tu, tavo žmona, sūnūs ir sūnų žmonos. 
\par 17 Visus gyvūnus, kurie su tavimi: paukščius, gyvulius ir roplius, išsivesk, kad jie paplistų žemėje, veistųsi ir daugėtų”. 
\par 18 Ir išėjo Nojus, su juo jo sūnūs, žmona ir sūnų žmonos. 
\par 19 Visi žvėrys, ropliai ir paukščiai, visa, kas kruta žemėje, kiekvienas pagal savo rūšį, išėjo iš arkos. 
\par 20 Nojus pastatė Viešpačiui aukurą ir, paėmęs iš visų švarių gyvulių ir paukščių, aukojo deginamąsias aukas. 
\par 21 Viešpats, užuodęs malonų kvapą, tarė savo širdyje: “Aš daugiau nebeprakeiksiu žemės dėl žmogaus, nes žmogaus širdis palinkusi į pikta nuo pat jo jaunystės, ir daugiau nebeišnaikinsiu viso to, kas gyva, kaip esu padaręs. 
\par 22 Kol žemė bus, nenustos sėja ir pjūtis, šaltis ir šiluma, vasara ir žiema, diena ir naktis!”



\chapter{9}


\par 1 Dievas laimino Nojų bei jo sūnus ir tarė: “Būkite vaisingi, dauginkitės ir pripildykite žemę. 
\par 2 Tesibijo jūsų ir tedreba prieš jus visi žemės žvėrys, visi padangių sparnuočiai, visa, kas gyva žemėje, ir visos jūros žuvys. Visa tai atiduota į jūsų rankas. 
\par 3 Visa, kas juda ir gyva, bus jums maistui; visa jums duodu, kaip daviau žaliuojančius augalus. 
\par 4 Tik mėsos su gyvybe, kuri yra kraujyje, nevalgykite. 
\par 5 Iš tiesų už jūsų gyvybės kraują Aš pareikalausiu iš kiekvieno žvėries ir žmogaus, kuris pralietų savo brolio kraują. 
\par 6 Kas pralieja žmogaus kraują, jo kraujas taip pat bus pralietas, nes žmogus sutvertas pagal Dievo atvaizdą. 
\par 7 Jūs būkite vaisingi ir dauginkitės, pliskite po žemę ir pripildykite ją!” 
\par 8 Dievas tarė Nojui ir jo sūnums kartu su juo: 
\par 9 “Aš darau sandorą su jumis ir jūsų palikuonimis, kurie gyvens po jūsų, 
\par 10 ir su visais gyvais padarais, visais paukščiais, galvijais ir žvėrimis, kurie išėjo su jumis iš arkos. 
\par 11 Aš darau savo sandorą su jumis: jokio kūno nebepražudysiu tvano vandenimis, ir tvanas nebesunaikins žemės”. 
\par 12 Ir Dievas tarė: “Ženklas sandoros, kurią darau tarp savęs ir jūsų bei visų gyvų padarų, kurie yra su jumis, per visas būsimas kartas 
\par 13 bus lankas debesyse. Jis tebūna sandoros ženklu tarp manęs ir žemės. 
\par 14 Sutelkęs debesis viršum žemės ir lankui pasirodžius debesyse, 
\par 15 atsiminsiu savo sandorą, kuri yra tarp manęs ir jūsų bei visų gyvų padarų, kad daugiau nebūtų tvano, sunaikinančio kiekvieną kūną žemėje. 
\par 16 Kai lankas bus debesyse, Aš jį pamatysiu ir atsiminsiu amžinąją sandorą tarp Dievo ir visų gyvų padarų, turinčių kūną, kurie yra ant žemės”. 
\par 17 Dievas tarė Nojui: “Tai yra sandoros ženklas, kurią Aš darau tarp savęs ir kiekvieno kūno, gyvenančio žemėje”. 
\par 18 Nojaus sūnūs, kurie išėjo iš arkos, buvo Semas, Chamas ir Jafetas; Chamas buvo Kanaano tėvas. 
\par 19 Šie trys yra Nojaus sūnūs ir iš jų atsirado visi žemės gyventojai. 
\par 20 Nojus pradėjo dirbti žemę ir įsiveisė vynuogyną. 
\par 21 Išgėręs vyno, pasigėrė ir gulėjo apsinuoginęs savo palapinėje. 
\par 22 Chamas, Kanaano tėvas, pamatęs savo tėvo nuogumą, pasakė broliams, kurie buvo lauke. 
\par 23 Semas ir Jafetas paėmė apsiaustą ir abu, užsimetę ant pečių, priėjo atbuli, ir apdengė savo tėvo nuogumą; jų veidai buvo nukreipti į priešingą pusę ir jie nematė savo tėvo nuogumo. 
\par 24 Nojus, išsiblaivęs nuo vyno ir sužinojęs, ką jam padarė jaunesnysis sūnus, 
\par 25 tarė: “Tebūna prakeiktas Kanaanas! Vergų vergas tebūna jis savo broliams! 
\par 26 Palaimintas Viešpats, Semo Dievas, o Kanaanas bus jo vergas! 
\par 27 Teišplečia Dievas Jafetą, ir tegu jis gyvena Semo palapinėse, o Kanaanas bus jo vergas!” 
\par 28 Tvanui praėjus, Nojus dar gyveno tris šimtus penkiasdešimt metų. 
\par 29 Taigi Nojaus amžius buvo devyni šimtai penkiasdešimt metų, ir jis mirė.



\chapter{10}


\par 1 Tai yra Nojaus sūnų Semo, Chamo ir Jafeto palikuonys. Tvanui praėjus, jie susilaukė sūnų. 
\par 2 Jafeto sūnūs: Gomeras, Magogas, Madajas, Javanas, Tubalas, Mešechas ir Tyras. 
\par 3 Gomero sūnūs: Aškenazas, Rifatas ir Togarma. 
\par 4 Javano sūnūs: Eliša, Taršišas, Kitimas ir Dodanimas. 
\par 5 Iš šitų atsirado tautų grupės, gyvenančios savo žemėse, kiekviena su savo kalba ir pagal savo giminę savo tautose. 
\par 6 Chamo sūnūs: Kušas, Mizraimas, Putas ir Kanaanas. 
\par 7 Kušo sūnūs: Seba, Havila, Sabta, Rama ir Sabtecha. Ramos sūnūs: Šeba ir Dedanas. 
\par 8 Kušui gimė ir Nimrodas, kuris tapo galingas žemėje. 
\par 9 Jis buvo smarkus medžiotojas Viešpaties akyse. Todėl sakoma: “Smarkus medžiotojas kaip Nimrodas Viešpaties akyse”. 
\par 10 Jo karalystės pradžia buvo Babelė, Erechas, Akadas ir Kalnė Šinaro šalyje. 
\par 11 Iš čia jis išvyko į Asūrą ir pasistatė Ninevę, Rehobot Irą, Kelachą 
\par 12 ir Reseną tarp Ninevės ir Kelacho; tai yra didysis miestas. 
\par 13 Mizraimo sūnūs: Ludas, Anamimas, Lehabas, Naftuchimas, 
\par 14 Patrusimas, Kasluhas, iš kurio kilo filistinai, ir Kaftoras. 
\par 15 Kanaano palikuonys: pirmagimis Sidonas, Hetas, 
\par 16 jebusiečiai, amoritai, girgašai, 
\par 17 hivai, arkai, sinai, 
\par 18 arvadiečiai, cemarai ir hematiečiai. Taip kanaaniečių giminės išsiplėtė. 
\par 19 Kanaaniečių ribos tęsėsi nuo Sidono link Geraro iki Gazos, link Sodomos, Gomoros, Admos bei Ceboimų iki Lešos. 
\par 20 Tai Chamo palikuonys pagal jų gentis, kalbas, šalis bei tautas. 
\par 21 Semas, visų Ebero sūnų tėvas, vyresnysis Jafeto brolis, taip pat turėjo sūnų. 
\par 22 Semo sūnūs: Elamas, Asūras, Arfaksadas, Ludas ir Aramas. 
\par 23 Aramo sūnūs: Ucas, Hulas, Geteras ir Mašas. 
\par 24 Arfaksado sūnus­Sala, o Salos­Eberas. 
\par 25 Eberas turėjo du sūnus: vienas vardu Falekas, nes jo dienomis buvo padalinta žemė, antrasis­Joktanas. 
\par 26 Joktano sūnūs: Almodadas, Šelefas, Hacarmavetas, Jerachas, 
\par 27 Hadoramas, Uzalas, Dikla, 
\par 28 Obalas, Abimaelis, Šeba, 
\par 29 Ofyras, Havila ir Jobabas; visi jie yra Joktano sūnūs. 
\par 30 Jie gyveno nuo Mešos iki Sefaro kalno rytuose. 
\par 31 Šitie yra Semo palikuonys pagal jų gimines, kalbas, šalis bei tautas. 
\par 32 Šitos yra Nojaus sūnų giminės pagal jų palikuonis ir tautas; iš jų atsirado tautos žemėje po tvano.



\chapter{11}


\par 1 Visi žemės gyventojai kal bėjo viena kalba. 
\par 2 Besikeldami toliau į rytus, jie rado lygumą Šinaro krašte ir ten apsigyveno. 
\par 3 Jie kalbėjosi: “Pasidirbinkime plytų ir gerai jas išdekime”. Plytas jie naudojo vietoje akmenų, o dervą­vietoje kalkių. 
\par 4 Jie tarėsi: “Pasistatykime miestą ir bokštą, kurio viršūnė siektų dangų. Išgarsinkime savo vardą prieš išsiskirstydami į visus kraštus!” 
\par 5 Viešpats nužengė pasižiūrėti miesto ir bokšto, kurį žmonės statė, 
\par 6 ir tarė: “Jie yra viena tauta ir visi kalba viena kalba. Jie pradėjo tai daryti ir nėra nieko, ko jie negalėtų pasiekti, jeigu nusprendžia. 
\par 7 Nusileiskime ir sumaišykime jų kalbą, kad jie nebesuprastų vienas kito!” 
\par 8 Viešpats juos išsklaidė po visą žemės paviršių, ir jie nustojo statę miestą. 
\par 9 Todėl tą miestą praminė Babele, nes ten Viešpats sumaišė jų kalbą ir iš ten Viešpats išsklaidė juos į visus žemės kraštus. 
\par 10 Šitie yra Semo palikuonys: Semas, būdamas šimto metų, dvejiems metams praėjus po tvano, susilaukė sūnaus Arfaksado. 
\par 11 Semas, gimus Arfaksadui, dar gyveno penkis šimtus metų ir susilaukė sūnų bei dukterų. 
\par 12 Kai Arfaksadui buvo trisdešimt penkeri metai, gimė sūnus Sala. 
\par 13 Po to Arfaksadas dar gyveno keturis šimtus trejus metus ir susilaukė sūnų bei dukterų. 
\par 14 Kai Salai buvo trisdešimt metų, gimė Eberas. 
\par 15 Po to Sala dar gyveno keturis šimtus trejus metus ir susilaukė sūnų bei dukterų. 
\par 16 Kai Eberui buvo trisdešimt ketveri metai, gimė Falekas. 
\par 17 Po to Eberas dar gyveno keturis šimtus trisdešimt metų ir susilaukė sūnų bei dukterų. 
\par 18 Kai Falekui buvo trisdešimt metų, gimė Ragaujas. 
\par 19 Po to Falekas dar gyveno du šimtus devynerius metus ir susilaukė sūnų bei dukterų. 
\par 20 Kai Ragaujui buvo trisdešimt dveji metai, gimė Seruchas. 
\par 21 Po to Ragaujas dar gyveno du šimtus septynerius metus ir susilaukė sūnų bei dukterų. 
\par 22 Kai Seruchui buvo trisdešimt metų, gimė Nachoras. 
\par 23 Po to Seruchas dar gyveno du šimtus metų ir susilaukė sūnų bei dukterų. 
\par 24 Kai Nachorui buvo dvidešimt devyneri metai, gimė Tara. 
\par 25 Po to Nachoras dar gyveno šimtą devyniolika metų ir susilaukė sūnų bei dukterų. 
\par 26 Kai Tarai buvo septyniasdešimt metų, gimė Abromas, Nahoras ir Haranas. 
\par 27 Šitie yra Taros palikuonys: Abromas, Nahoras ir Haranas, o Harano sūnus­Lotas. 
\par 28 Haranas mirė, jo tėvui Tarai dar gyvam tebeesant, savo gimtoje šalyje, Chaldėjos Ūre. 
\par 29 Abromas ir Nahoras vedė. Abromo žmonos vardas buvo Saraja, o Nahoro­Milka; ji buvo duktė Harano, Milkos ir Iskos tėvo. 
\par 30 Saraja buvo nevaisinga: ji neturėjo vaikų. 
\par 31 Tara ėmė savo sūnų Abromą ir savo sūnaus Harano sūnų Lotą, marčią Sarają, sūnaus Abromo žmoną, ir jie iškeliavo iš Chaldėjos Ūro į Kanaano šalį. Atėję ligi Charano, jie ten įsikūrė. 
\par 32 Taros amžius buvo du šimtai penkeri metai, ir jis mirė Charane.



\chapter{12}

\par 1 Viešpats tarė Abromui: “Palik savo šalį, gimines, tėvų namus ir eik į kraštą, kurį tau parodysiu. 
\par 2 Aš padarysiu tave didele tauta, tave laiminsiu ir padarysiu tavo vardą garsų; ir tu būsi palaiminimu. 
\par 3 Aš laiminsiu tuos, kurie tave laimina, ir prakeiksiu tuos, kurie tave keikia, ir tavyje bus palaimintos visos žemės giminės”. 
\par 4 Abromas iškeliavo, kaip Viešpats jam pasakė. Su juo drauge išėjo Lotas. Abromas buvo septyniasdešimt penkerių metų, kai jis išėjo iš Charano. 
\par 5 Abromas paėmė savo žmoną Sarają, brolio sūnų Lotą, visą turtą, kurį jie turėjo, žmones, kuriuos buvo įsigijęs Charane, ir išėjo į Kanaano šalį. Atėjęs į tą kraštą, 
\par 6 Abromas pasiekė Sichemo vietovę, Morės slėnį. Tuo laiku toje šalyje gyveno kanaaniečiai. 
\par 7 Ten Viešpats pasirodė Abromui ir tarė: “Tavo palikuonims duosiu šitą šalį”. Jis ten pastatė aukurą Viešpačiui, kuris jam pasirodė. 
\par 8 Iš ten jis keliavo į rytus nuo Betelio ir pasistatė palapinę. Betelis buvo vakaruose, Ajas­rytuose. Ten jis pastatė aukurą Viešpačiui ir šaukėsi Viešpaties vardo. 
\par 9 Abromas keliavo vis toliau į pietus. 
\par 10 Kilus badui toje šalyje, Abromas nuvyko pagyventi į Egiptą, nes šalyje buvo didelis badas. 
\par 11 Jiems priartėjus prie Egipto, jis tarė savo žmonai Sarajai: “Aš žinau, kad tu esi graži moteris 
\par 12 ir tave pamatę egiptiečiai sakys: ‘Ši yra jo žmona’. Tada jie užmuš mane, o tave pasilaikys. 
\par 13 Sakyk tad, jog esi mano sesuo, kad man gerai būtų dėl tavęs ir išlikčiau gyvas tavo dėka”. 
\par 14 Atsitiko, kad, kai tik Abromas įėjo į Egiptą, egiptiečiai pastebėjo, jog moteris labai graži. 
\par 15 Ją pamatę, faraono kunigaikščiai kalbėjo apie jos grožį faraonui. Moteris tuojau buvo paimta į faraono namus, 
\par 16 o su Abromu jis gerai elgėsi dėl jos: jis turėjo avių, galvijų, asilų, tarnų, tarnaičių, asilių ir kupranugarių. 
\par 17 Viešpats baudė faraoną ir jo namus dėl Sarajos, Abromo žmonos. 
\par 18 Faraonas tada pasišaukė Abromą ir tarė: “Ką tu man padarei? Kodėl man nesakei, kad ji tavo žmona? 
\par 19 Kodėl sakei, kad ji tavo sesuo? Aš norėjau paimti ją sau už žmoną. Taigi štai tavo žmona, imk ją ir keliauk savo keliais”. 
\par 20 Faraonas įsakė savo žmonėms palydėti jį, jo žmoną ir visa, ką jis turėjo.



\chapter{13}


\par 1 Abromas su žmona ir viskuo, ką jis turėjo, išėjo iš Egipto į pietų šalį. Su jais iškeliavo ir Lotas. 
\par 2 Abromas buvo labai turtingas: turėjo daug gyvulių, sidabro ir aukso. 
\par 3 Jis keliavo atgal tuo pačiu keliu, kuriuo buvo atėjęs, Betelio link iki tos vietos, kur pradžioje tarp Betelio ir Ajo buvo pasistatęs palapinę, 
\par 4 iki tos vietos, kur buvo anksčiau pasidaręs aukurą. Čia Abromas šaukėsi Viešpaties vardo. 
\par 5 Lotas, kuris keliavo su Abromu, taip pat turėjo avių, galvijų ir palapinių. 
\par 6 Kraštas negalėjo sutalpinti jų ir išmaitinti jų gyvulių, nes jų bandos buvo per didelės. 
\par 7 Kilo vaidas tarp Abromo ir Loto piemenų. Tada tebegyveno tame krašte kanaaniečiai ir perizai. 
\par 8 Abromas tarė Lotui: “Tenebūna vaidų tarp manęs ir tavęs, tarp mano piemenų ir tavo piemenų, nes esame broliai! 
\par 9 Ar ne visa šalis tau atvira? Skirkis nuo manęs! Jei eisi į kairę, aš eisiu į dešinę, o jei tu eisi į dešinę, tai aš eisiu į kairę”. 
\par 10 Lotas pakėlė akis ir pamatė, kad visa Jordano apylinkė­prieš Viešpačiui sunaikinant Sodomą ir Gomorą­Coaro link buvo gerai drėkinama kaip Viešpaties sodas, kaip Egipto šalis. 
\par 11 Lotas pasirinko Jordano lygumą ir patraukė į rytus. Taip juodu išsiskyrė. 
\par 12 Abromas gyveno Kanaano šalyje, o Lotas­lygumos miestuose: Sodomos link statėsi palapines. 
\par 13 Sodomos žmonės buvo nedori ir labai nusidėję Viešpačiui. 
\par 14 Viešpats tarė Abromui, kai Lotas nuo jo atsiskyrė: “Pakelk akis ir pažvelk iš vietos, kurioje esi, į šiaurę ir į pietus, į rytus ir į vakarus! 
\par 15 Visą žemę, kurią matai, duosiu tau ir tavo palikuonims visam laikui. 
\par 16 Padarysiu tavo palikuonių tiek daug, kiek dulkių ant žemės. Jei kas galėtų suskaičiuoti žemės dulkes, tai ir tavo palikuonis galėtų suskaičiuoti. 
\par 17 Kelkis, pereik žemę išilgai ir skersai, nes Aš tau ją duosiu!” 
\par 18 Abromas susivyniojo palapinę ir nuėjo iki Mamrės ąžuolų Hebrone. Ten jis pastatė Viešpačiui aukurą.



\chapter{14}


\par 1 Anomis dienomis karalius Amrafelis iš Šinaro, karalius Arjochas iš Elasaro, Elamo karalius Kedorlaomeras ir Goimo karalius Tidalas 
\par 2 pradėjo karą su Sodomos karaliumi Bera, Gomoros karaliumi Birša, Admos karaliumi Šinabu, Ceboimo karaliumi Šemeberu ir karaliumi Belos, dar vadinamo Coaru. 
\par 3 Visi šitie, sudarę sąjungą, ėjo į Sidimo slėnį, kur dabar yra Druskos jūra. 
\par 4 Dvylika metų jie tarnavo Kedorlaomeriui, o tryliktaisiais metais sukilo. 
\par 5 Keturioliktaisiais metais atėjo Kedorlaomeris ir jo sąjungininkai. Jie sumušė refajus Aštarot Karnaime, zūzus Hame, emus Šave Kirjataimo lygumoje 
\par 6 ir horus Seyro kalne iki El Parano, kuris siekia dykumą. 
\par 7 Paskui jie atėjo į En Mišpatą, tai yra Kadešą, ir nusiaubė visą amalekiečių kraštą ir amoritus, kurie gyveno Hacecon Tamaroje. 
\par 8 Tada išėjo Sodomos, Gomoros, Admos, Ceboimo ir Belos karaliai ir išsirikiavo kovai Sidimo slėnyje: 
\par 9 prieš Elamo karalių Kedorlaomerį, Goimo karalių Tidalą, Šinaro karalių Amrafelį ir Elasaro karalių Arjochą­keturi karaliai prieš penkis. 
\par 10 Sidimo slėnis buvo pilnas dervos duobių. Bėgdami Sodomos ir Gomoros karaliai įkrito į jas, o likusieji pabėgo į kalnus. 
\par 11 Laimėtojai, pasigrobę visą Sodomos bei Gomoros turtą ir visas atsargas, nuėjo. 
\par 12 Jie taip pat paėmė Abromo brolio sūnų Lotą, kuris gyveno Sodomoje, ir jo turtą. 
\par 13 Vienas iš pabėgusiųjų atėjęs pranešė tai Abromui, hebrajui, gyvenančiam amorito Mamrės ąžuolyne; Mamrė ir jo broliai Eškolas ir Aneras buvo Abromo sąjungininkai. 
\par 14 Abromas išgirdęs, kad jo brolio sūnus buvo išvestas į nelaisvę, apginklavo tris šimtus aštuoniolika savo išlavintų tarnų, gimusių jo namuose, ir vijosi priešą iki Dano. 
\par 15 Jis, padalinęs savo būrį, užpuolė juos nakčia, sumušė ir vijosi juos ligi Hobos, kuri yra į kairę nuo Damasko, 
\par 16 ir atsiėmė jų pagrobtą turtą, Lotą ir jo turtą, moteris ir žmones. 
\par 17 Kai Abromas sugrįžo, nugalėjęs Kedorlaomerį ir karalius, kurie buvo su juo, Sodomos karalius išėjo jo pasitikti į Šavės slėnį­Karaliaus slėnį. 
\par 18 Melchizedekas, Salemo karalius, atnešė duonos ir vyno. Jis buvo aukščiausiojo Dievo kunigas. 
\par 19 Jis laimino jį, sakydamas: “Tebūna Abromas palaimintas aukščiausiojo Dievo, dangaus ir žemės valdovo! 
\par 20 Tebūna šlovinamas aukščiausiasis Dievas, kuris atidavė tavo priešus į tavo rankas!” Ir Abromas davė jam dešimtinę nuo visko. 
\par 21 Sodomos karalius tarė Abromui: “Atiduok man žmones, o turtą pasilaikyk!” 
\par 22 Bet Abromas tarė Sodomos karaliui: “Aš, pakėlęs ranką, prisiekiu aukščiausiuoju Dievu, dangaus ir žemės valdovu: 
\par 23 nei siūlo galo, nei kurpių dirželio, nei ko nors kito, kas tau priklauso, aš neimsiu, kad nesakytum: ‘Aš padariau Abromą turtingą!’ 
\par 24 Aš nieko neimsiu, išskyrus tai, ką suvalgė jaunuoliai ir kas priklauso vyrams: Anerui, Eškolui ir Mamrei, kurie ėjo su manimi. Jie tepasiima savo dalį”.



\chapter{15}


\par 1 Po šių įvykių Viešpats prabilo į Abromą regėjime: “Nebijok, Abromai! Aš esu tavo skydas ir labai didelis atlygis!” 
\par 2 Abromas tarė: “Viešpatie Dieve! Ką Tu man duosi? Aš neturiu vaikų, mano namų paveldėtojas bus damaskietis Eliezeras. 
\par 3 Man nedavei vaikų, ir štai tarnas, gimęs mano namuose, yra mano įpėdinis”. 
\par 4 Viešpats atsakė jam: “Ne šitas bus tavo paveldėtojas, bet tas, kuris gims iš tavęs”. 
\par 5 Ir, išvedęs jį laukan, tarė: “Pažvelk į dangų ir, jei gali, suskaičiuok žvaigždes! Tiek bus tavo palikuonių!” 
\par 6 Abromas patikėjo Dievu, ir tai jam buvo įskaityta teisumu. 
\par 7 Dievas kalbėjo: “Aš esu Viešpats, kuris tave išvedžiau iš Chaldėjos miesto Ūro, kad tu paveldėtum šitą šalį”. 
\par 8 Abromas tarė: “Viešpatie Dieve! Iš kur galiu žinoti, kad ją paveldėsiu?” 
\par 9 Tada Jis tarė jam: “Atvesk man trejų metų karvę, trejų metų ožką, trejų metų aviną, balandį ir jauną karvelį”. 
\par 10 Abromas, atvedęs visus gyvulius, padalino pusiau ir padėjo kiekvieną gabalą vienas prieš kitą, tačiau paukščių nedalino. 
\par 11 Plėšrieji paukščiai atskrido prie mėsos, bet Abromas juos nubaidė. 
\par 12 Saulei leidžiantis, gilus miegas apėmė Abromą, siaubas ir didelė tamsa jį apėmė. 
\par 13 Tada Viešpats tarė Abromui: “Žinok, kad tavo palikuonys bus ateiviai svetimame krašte, jie vergaus ir bus spaudžiami keturis šimtus metų. 
\par 14 Tautą, kuriai jie vergaus, Aš teisiu, ir tada jie iš ten išeis su dideliu turtu. 
\par 15 O tu ramybėje nueisi pas savo tėvus ir būsi palaidotas, sulaukęs žilos senatvės. 
\par 16 Ketvirtoji karta sugrįš čia, nes amoritų nusikaltimų saikas dar nėra pilnas”. 
\par 17 Nusileidus saulei ir sutemus, štai pasirodė rūkstanti krosnis ir liepsnojantis deglas ir praėjo tarp tų mėsos gabalų. 
\par 18 Tą dieną Viešpats padarė su Abromu sandorą, sakydamas: “Tavo palikuonims atidaviau visą žemę nuo Egipto upės iki didžiosios Eufrato upės: 
\par 19 kenitus, kenazus, kadmonitus, 
\par 20 hetitus, perizus, refajus, 
\par 21 amoritus, kanaaniečius, girgašus ir jebusiečius”.



\chapter{16}


\par 1 Abromo žmona Saraja ne turėjo vaikų. Ji turėjo tarnaitę egiptietę, vardu Hagara. 
\par 2 Saraja tarė Abromui: “Viešpats nedavė man vaikų gimdyti. Prašau, įeik pas mano tarnaitę, galbūt per ją aš turėsiu vaikų”. Abromas paklausė Sarajos. 
\par 3 Abromui išgyvenus dešimt metų Kanaano šalyje, Saraja, Abromo žmona, savo tarnaitę egiptietę Hagarą davė savo vyrui Abromui už žmoną. 
\par 4 Jis įėjo pas Hagarą, ir ji pastojo. Hagara pastojusi su panieka ėmė žiūrėti į savo valdovę. 
\par 5 Tada Saraja tarė Abromui: “Širdgėla, kurią turiu, tegula ant tavęs! Aš pati daviau savo tarnaitę tau į glėbį, tačiau ji, pasijutusi nėščia, su panieka ėmė žiūrėti į mane. Viešpats tebūna teisėjas tarp manęs ir tavęs”. 
\par 6 Abromas tarė Sarajai: “Tavo tarnaitė yra tavo rankose. Elkis su ja, kaip tau patinka”. Kai Saraja ėmė ją spausti, ta pabėgo. 
\par 7 Viešpaties angelas, radęs ją prie vandens šaltinio dykumoje, prie kelio į Šūrą, 
\par 8 tarė: “Hagara, Sarajos tarnaite, iš kur atėjai ir kur eini?” Ji atsakė: “Bėgu nuo savo valdovės Sarajos”. 
\par 9 Viešpaties angelas jai tarė: “Sugrįžk pas savo valdovę ir nusižemink prieš ją. 
\par 10 Aš taip padauginsiu tavo palikuonis, kad jų net suskaičiuoti nebus galima. 
\par 11 Štai tu esi nėščia ir pagimdysi sūnų. Tu jį pavadinsi Izmaeliu, nes Viešpats išgirdo apie tavo priespaudą. 
\par 12 Tavo sūnus gyvens kaip laukinis asilas: jis bus prieš visus ir visi prieš jį, jis gyvens šalia savo brolių”. 
\par 13 Ir Viešpatį, kuris su ja kalbėjo, Hagara pavadino: “Tu esi Dievas, kuris mane matai”. Nes ji sakė: “Aš tikrai mačiau Dievą, kuris mato mane”. 
\par 14 Todėl tą šulinį pavadino Lahai Roiju. Jis yra tarp Kadešo ir Beredo. 
\par 15 Hagara pagimdė Abromui sūnų. Abromas pavadino jį vardu Izmaelis. 
\par 16 Abromui buvo aštuoniasdešimt šešeri metai, kai Hagara pagimdė jam sūnų.



\chapter{17}

\par 1 Kai Abromas buvo devyniasdešimt devynerių metų amžiaus, Viešpats pasirodė Abromui ir tarė: “Aš esu Dievas Visagalis, vaikščiok mano akivaizdoje ir būk tobulas. 
\par 2 Aš sudarysiu sandorą su tavimi ir labai padauginsiu tavo palikuonis”. 
\par 3 Abromas puolė kniūbsčias, o Dievas kalbėjo: 
\par 4 “Štai manoji sandora su tavimi: tu tapsi daugelio tautų tėvu, 
\par 5 todėl nebesivadinsi Abromu, bet tavo vardas bus Abraomas, nes Aš tave padariau daugelio tautų tėvu. 
\par 6 Ir padarysiu tave nepaprastai vaisingą. Iš tavęs kils tautos ir karaliai. 
\par 7 Aš patvirtinsiu sandorą tarp savęs ir tavęs bei tavo palikuonių kaip amžiną sandorą, kad būčiau Dievas tau ir tavo palikuonims po tavęs. 
\par 8 Aš duosiu tau ir tavo palikuonims žemę, kurioje dabar esi ateivis, visą Kanaano žemę amžinai nuosavybei ir būsiu jiems Dievas”. 
\par 9 Toliau Dievas sakė Abraomui: “Laikyk mano sandorą tu ir tavo palikuonys po tavęs per kartų kartas. 
\par 10 Šita yra mano sandora, kurios jūs privalote laikytis. Kiekvienas vyras tarp jūsų bus apipjaustytas! 
\par 11 Jūs apipjaustysite savo kūną, ir tai bus tarp manęs ir jūsų esančios sandoros ženklas. 
\par 12 Kiekvienas berniukas aštuonių dienų tarp jūsų bus apipjaustytas: gimęs namuose ar iš svetimšalio nupirktas, kuris nėra iš tavo palikuonių. 
\par 13 Kas gimsta tavo namuose ir ką nusiperki už savo pinigus, tas turi būti apipjaustytas. Ir mano sandora jūsų kūne bus amžina sandora. 
\par 14 O neapipjaustytas vyras bus išnaikintas iš savo tautos, nes jis sulaužė mano sandorą”. 
\par 15 Dievas toliau sakė Abraomui: “Savo žmonos Sarajos nebevadink Saraja, bet Sara bus jos vardas. 
\par 16 Aš ją palaiminsiu ir tau duosiu iš jos sūnų. Aš ją taip laiminsiu, kad iš jos kils tautos ir tautų karaliai”. 
\par 17 Tada Abraomas puolė veidu į žemę, juokėsi ir sakė savo širdyje: “Ar šimtamečiui begali kas gimti ir ar Sara, sulaukusi devyniasdešimties metų amžiaus, begali gimdyti?” 
\par 18 Abraomas tarė Dievui: “Kad nors Izmaelis gyventų Tavo akivaizdoje!” 
\par 19 Bet Dievas atsakė: “Tikrai tavo žmona Sara pagimdys tau sūnų ir tu jį pavadinsi Izaoku! Aš sudarysiu su juo amžiną sandorą ir su jo palikuonimis. 
\par 20 Ir dėl Izmaelio Aš išklausiau tave! Aš jį palaiminau ir padarysiu jį vaisingą, ir nepaprastai padauginsiu: jis bus dvylikos kunigaikščių tėvas, ir padarysiu iš jo didelę tautą. 
\par 21 Tačiau savo sandorą sudarysiu su Izaoku, kurį Sara pagimdys tau šiuo laiku kitais metais”. 
\par 22 Baigęs kalbėti su juo, Dievas pasišalino. 
\par 23 Tada Abraomas ėmė savo sūnų Izmaelį ir visus vyrus, jo namuose gimusius ir už pinigus įsigytus, ir apipjaustė jų kūnus tą pačią dieną, kaip Dievas jam buvo sakęs. 
\par 24 Abraomas buvo apipjaustytas devyniasdešimt devynerių metų, 
\par 25 o jo sūnus Izmaelis­trylikos metų. 
\par 26 Tą pačią dieną buvo apipjaustytas Abraomas ir jo sūnus Izmaelis. 
\par 27 Ir visi jo namų vyrai, gimusieji jo namuose ir už pinigus įsigytieji, buvo apipjaustyti kartu su juo.



\chapter{18}


\par 1 Viešpats pasirodė Abraomui prie Mamrės ąžuolų, kai jis sėdėjo palapinės prieangyje pačioje dienos kaitroje. 
\par 2 Jis, pakėlęs akis, pamatė tris vyrus, stovinčius prieš jį. Jis išbėgo iš palapinės ir, nusilenkęs iki žemės, 
\par 3 tarė: “Mano Viešpatie, jei radau malonę Tavo akyse, prašau, neaplenk savo tarno! 
\par 4 Leiskite atnešti kiek vandens nusiplauti kojoms, pailsėkite po medžiu, 
\par 5 kol atnešiu duonos kąsnį jums pasistiprinti. Po to galėsite toliau keliauti, nes tam juk atėjote pas savo tarną”. Jie tarė: “Daryk taip, kaip sakei!” 
\par 6 Abraomas nuskubėjo į palapinę pas Sarą ir tarė: “Skubiai įmaišyk tris saikus geriausių miltų ir iškepk papločių”. 
\par 7 Abraomas nubėgo pas gyvulius ir, paėmęs rinktinį veršiuką, padavė tarnui, o tas skubėjo jį paruošti. 
\par 8 Jis ėmė sviesto, pieno ir veršiuką, kurį buvo paruošęs, ir patiekė jiems. O jis pats, jiems valgant, stovėjo prie jų po medžiu. 
\par 9 Jie paklausė jį: “Kur yra tavo žmona Sara?” Jis atsakė: “Palapinėje”. 
\par 10 Vienas iš jų tarė: “Aš tikrai sugrįšiu pas tave kitais metais šiuo laiku, ir tavo žmona Sara turės sūnų!” Tuo metu Sara klausėsi palapinės prieangyje, kuris buvo už jų. 
\par 11 Abraomas ir Sara buvo seni, sulaukę žilos senatvės. Sarai nebebūdavo to, kas būna moterims. 
\par 12 Todėl Sara savyje juokėsi: “Būdama pasenusi ir mano viešpačiui esant senam, argi dar turėsiu malonumą?” 
\par 13 Viešpats tarė Abraomui: “Kodėl Sara juokėsi, sakydama: ‘Ar aš iš tikrųjų gimdysiu, būdama pasenusi?’ 
\par 14 Ar yra kas nors Viešpačiui neįmanoma? Kitais metais, numatytu laiku, Aš sugrįšiu pas tave, ir Sara turės sūnų!” 
\par 15 Sara gynėsi, sakydama: “Aš nesijuokiau”, nes išsigando. O Jis tarė: “Ne! Tu juokeisi!” 
\par 16 Po to tie vyrai pakilo ir ėjo Sodomos link, o Abraomas ėjo su jais, norėdamas juos palydėti. 
\par 17 Viešpats tarė: “Ar Aš slėpsiu nuo Abraomo, ką ketinu daryti? 
\par 18 Juk Abraomas tikrai taps didele ir galinga tauta, jame bus palaimintos visos žemės tautos. 
\par 19 Nes Aš žinau, kad jis įsakys savo vaikams ir savo namams po savęs laikytis Viešpaties kelio ir daryti, kas yra teisinga ir teisu, kad Viešpats galėtų ištesėti Abraomui, ką Jis kalbėjo apie jį”. 
\par 20 Viešpats tarė: “Sodomos ir Gomoros šauksmas yra garsus, o jų nuodėmė­labai sunki. 
\par 21 Aš nusileisiu ir pažiūrėsiu, ar jų nusikaltimai atitinka šauksmą, pasiekusį mane. Jeigu ne, Aš sužinosiu”. 
\par 22 Tie vyrai ėjo toliau, o Abraomas pasiliko bestovįs Viešpaties akivaizdoje. 
\par 23 Abraomas priartėjęs tarė: “Ar sunaikinsi teisųjį kartu su nusikaltėliu? 
\par 24 Galbūt penkiasdešimt teisiųjų yra mieste. Ar tikrai sunaikinsi ir neatleisi tai vietovei dėl penkių dešimčių teisiųjų? 
\par 25 Tai nėra Tavo būdas nužudyti teisųjį su nusikaltėliu, kad teisusis gautų tą patį kaip piktadarys! Ar visos žemės Teisėjas pasielgs neteisingai?” 
\par 26 Viešpats tarė: “Jei Sodomos mieste rasiu penkiasdešimt teisiųjų, tai pasigailėsiu visos vietovės”. 
\par 27 Abraomas atsakė: “Štai, išdrįsau kalbėti Viešpačiui, nors esu dulkė ir pelenai. 
\par 28 Galbūt iki penkiasdešimt teisiųjų trūks penkių. Ar dėl keturiasdešimt penkių sunaikinsi visą miestą?” Jis tarė: “Nesunaikinsiu, jei ten rasiu keturiasdešimt penkis”. 
\par 29 Abraomas toliau kalbėjo: “Galbūt ten atsiras tik keturiasdešimt?” Jis atsakė: “Dėl keturiasdešimties nesunaikinsiu”. 
\par 30 Tada jis tarė: “Nesirūstink, Viešpatie, kad drįstu kalbėti. Galbūt ten atsiras tik trisdešimt”. O Jis atsakė: “Nieko nedarysiu, jei ten rasiu trisdešimt”. 
\par 31 Tada jis tarė: “Štai išdrįsau kalbėti Viešpačiui. Galbūt ten atsiras dvidešimt!” O Jis tarė: “Nesunaikinsiu ir dėl dvidešimties”. 
\par 32 Tada jis tarė: “Nesirūstink, Viešpatie, jei išdrįsiu dar kartą kalbėti. Galbūt ten atsiras dešimt?” O Jis atsakė: “Nesunaikinsiu ir dėl dešimties”. 
\par 33 Viešpats, baigęs kalbėti su Abraomu, nuėjo, o Abraomas sugrįžo į savo vietą.



\chapter{19}

\par 1 Kai du angelai vakare atė jo į Sodomą, Lotas sėdėjo Sodomos vartuose. Lotas, pamatęs juos, atsikėlė jų pasitikti ir nusilenkė iki žemės. 
\par 2 Jis tarė: “Mano viešpačiai, prašau, užsukite į savo tarno namus, pernakvokite ir nusiplaukite kojas. Anksti atsikėlę, galėsite eiti savo keliu”. Bet jie atsakė: “Ne, mes nakvosime gatvėje”. 
\par 3 Jis taip maldavo juos, kad jie užsuko pas jį ir įėjo į jo namus. Jis paruošė jiems vaišes, iškepė neraugintos duonos, ir jie valgė. 
\par 4 Jiems dar neatsigulus, visi Sodomos miesto vyrai, jauni ir seni, iš visų miesto dalių apsupo namus. 
\par 5 Jie pašaukė Lotą ir tarė: “Kur yra tie vyrai, kurie šįvakar atėjo pas tave? Išvesk juos laukan, kad mes juos pažintume”. 
\par 6 Lotas išėjo pas juos į prieangį ir, užrakinęs duris, 
\par 7 tarė: “Mano broliai, prašau, nesielkite taip piktai! 
\par 8 Aš turiu dvi dukteris, kurios dar nepažino vyro. Leiskite man jas išvesti pas jus ir darykite su jomis, kaip jums patinka. Tik tiems vyrams nieko nedarykite, nes jie atėjo po mano stogu”. 
\par 9 Bet jie tarė: “Šalin! Jis čia atvyko, kad gyventų kaip ateivis, o nori teisėju būti! Dabar mes pasielgsime su tavimi pikčiau negu su jais”. Jie smarkiai veržėsi prie Loto, norėdami išlaužti duris. 
\par 10 Tačiau vyrai savo rankomis įtempė Lotą į namą ir užrakino duris. 
\par 11 O tuos, kurie buvo prie namo durų, jie apakino, mažus ir didelius, kad jie nebesurastų durų. 
\par 12 Lotui juodu tarė: “Ar turi čia ką nors iš savųjų: žentus, sūnus, dukteris? Išvesk juos iš šios vietos! 
\par 13 Mes sunaikinsime šitą vietą, kadangi jų garsus šauksmas pasiekė Viešpatį ir Jis mus siuntė ją sunaikinti”. 
\par 14 Lotas išėjęs kalbėjo žentams, kurie buvo vedę jo dukteris: “Išeikite iš šios vietos, nes Viešpats sunaikins miestą”. Bet žentams atrodė, kad jis juokauja. 
\par 15 Išaušus angelai ragino Lotą, sakydami: “Imk žmoną ir abi dukteris, kurios čia yra, kad nebūtumėte sunaikinti dėl miesto kaltės”. 
\par 16 Kadangi jis delsė, tai tie vyrai nutvėrė jį už rankos, jo žmoną ir abi dukteris, nes Viešpats jų pasigailėjo, ir išvedę paleido už miesto ribų. 
\par 17 Išvedę juos, tarė: “Gelbėk savo gyvybę! Nežiūrėk atgal ir nesustok kur nors apylinkėje! Bėk į kalną, kad nežūtum!” 
\par 18 Lotas jiems atsakė: “O ne, mano Viešpatie! 
\par 19 Aš, Tavo tarnas, radau malonę Tavo akyse, ir man parodei didelį gailestingumą, išgelbėdamas mano gyvybę. Negaliu bėgti į kalną, kad kas bloga nenutiktų ir nenumirčiau. 
\par 20 Štai arti yra miestas. Leisk man į jį bėgti­jis yra mažas, ir aš jame išsigelbėsiu”. 
\par 21 Jis tarė jam: “Štai išklausiau tave ir dėl šito. Aš nesunaikinsiu miesto, apie kurį kalbėjai. 
\par 22 Skubėk, gelbėkis tenai. Nes Aš nieko negaliu daryti, kol nuvyksi ten”. Todėl tą miestą pavadino Coaru. 
\par 23 Saulei tekant, Lotas įėjo į Coarą. 
\par 24 Tuomet Viešpats siuntė ant Sodomos ir Gomoros sieros ir ugnies lietų. 
\par 25 Jis sunaikino tuos miestus, visą apylinkę, visus miesto gyventojus ir augalus. 
\par 26 Bet Loto žmona pažvelgė atgal ir pavirto druskos stulpu. 
\par 27 Anksti rytą Abraomas atėjo į tą vietą, kur jis stovėjo Viešpaties akivaizdoje, 
\par 28 ir pažvelgė Sodomos ir Gomoros link ir į visą jų apylinkę; jis matė kylančius nuo žemės dūmus kaip iš krosnies. 
\par 29 Dievas, sunaikindamas tos apylinkės miestus, atsiminė Abraomą ir išvedė Lotą iš pražūties, kai sugriovė miestus, kuriuose Lotas gyveno. 
\par 30 Lotas ir jo abi dukterys išėjo iš Coaro ir apsigyveno kalne, nes jis bijojo gyventi Coare. Jie apsigyveno oloje, jis ir abi jo dukterys. 
\par 31 Tada vyresnioji tarė jaunesniajai: “Mūsų tėvas senas, ir žemėje nebeliko vyro, kuris galėtų įeiti pas mus, kaip priimta visoje žemėje. 
\par 32 Eime, nugirdysime vynu savo tėvą ir atsigulsime prie jo, kad iš tėvo susilauktume palikuonių!” 
\par 33 Jos tą naktį nugirdė vynu savo tėvą. Po to vyresnioji įėjo ir gulėjo su savo tėvu, o tas nepajuto, kada ji atsigulė nė kada atsikėlė. 
\par 34 Kitą dieną vyresnioji tarė jaunesniajai: “Aš praėjusią naktį gulėjau su savo tėvu. Nugirdykime jį vynu ir šiąnakt. Po to eik, atsigulk prie jo, kad iš savo tėvo susilauktum palikuonio!” 
\par 35 Taigi jos ir kitą naktį nugirdė vynu tėvą. Paskui jaunesnioji įėjo ir gulėjo su juo, o jis nepajuto, kada ji atsigulė nė kada atsikėlė. 
\par 36 Taip abi Loto dukterys pastojo nuo savo tėvo. 
\par 37 Vyresnioji pagimdė sūnų ir jį pavadino Moabu. Jis yra ligi šiol tebegyvenančių moabitų tėvas. 
\par 38 Jaunesnioji pagimdė sūnų ir jį pavadino Amonu. Jis yra ligi šiol tebegyvenančių amonitų tėvas.



\chapter{20}

\par 1 Iš ten Abraomas traukė toliau į pietų šalį ir apsigyveno Gerare tarp Kadešo ir Šūro. 
\par 2 Abraomas sakė apie savo žmoną Sarą: “Ji yra mano sesuo”. Tada Abimelechas, Geraro karalius, paėmė Sarą. 
\par 3 Bet Dievas sapne pasirodė Abimelechui ir tarė: “Tu mirsi dėl moters, kurią paėmei, nes ji turi vyrą”. 
\par 4 Abimelechas dar nebuvo jos palietęs. Jis tarė: “Viešpatie, ar žudysi niekuo nekaltą tautą? 
\par 5 Argi Abraomas man nesakė: ‘Ji yra mano sesuo’? Ir ar ji pati nesakė: ‘Jis yra mano brolis’? Nekalta širdimi ir tyromis rankomis tai padariau”. 
\par 6 Dievas tarė jam sapne: “Aš žinau, kad tai darei nekalta širdimi, todėl tave sulaikiau, kad man nenusidėtum ir jos nepaliestum. 
\par 7 Taigi dabar sugrąžink vyrui žmoną, nes jis yra pranašas. Jis melsis už tave, ir tu išliksi gyvas. O jei nesugrąžinsi, tai žinok, kad tikrai mirsi su visais savaisiais!” 
\par 8 Abimelechas, atsikėlęs anksti rytą, sušaukė visus savo tarnus ir papasakojo jiems sapną. Žmonės labai nusigando. 
\par 9 Abimelechas pasikvietė Abraomą ir jam tarė: “Ką mums padarei? Ir kuo aš tau nusidėjau, kad užtraukei man ir mano karalystei tokią didelę nuodėmę? Tu neturėjai pasielgti su manimi taip, kaip pasielgei”. 
\par 10 Abimelechas tęsė toliau: “Ką galvojai taip darydamas?” 
\par 11 Abraomas atsakė: “Aš galvojau, kad šioje šalyje nėra Dievo baimės ir dėl mano žmonos jie užmuš mane. 
\par 12 Bet ji iš tikrųjų yra mano sesuo: mano tėvo duktė, tik ne mano motinos, ir ji tapo mano žmona. 
\par 13 Kai Dievas mane išvedė iš mano tėvo namų, aš jai tariau: “Padaryk man tą malonę­visur, kur nueisime, sakyk apie mane: ‘Jis yra mano brolis’ ”. 
\par 14 Abimelechas davė Abraomui avių, galvijų, tarnų bei tarnaičių ir sugrąžino jam jo žmoną Sarą. 
\par 15 Abimelechas tarė: “Štai mano kraštas tau atviras, įsikurk, kur tau patinka!” 
\par 16 O Sarai jis pasakė: “Aš daviau tavo broliui tūkstantį sidabrinių. Tebūna tai įrodymas visiems, kurie yra su tavimi, ir kitiems, kad esi nekalta”. 
\par 17 Abraomas meldė Dievą, ir Jis pagydė Abimelechą, jo žmoną ir tarnaites, kad jos gimdytų. 
\par 18 Nes Viešpats buvo padaręs nevaisingomis visas moteris Abimelecho namuose dėl Abraomo žmonos Saros.



\chapter{21}

\par 1 Viešpats aplankė Sarą ir įvykdė, ką jai buvo pažadėjęs. 
\par 2 Sara pastojo ir sulaukusiam senatvės Abraomui pagimdė sūnų tuo metu, kurį Dievas buvo jam nurodęs. 
\par 3 Abraomas pavadino iš Saros gimusį sūnų Izaoku. 
\par 4 Izaokas pagal Dievo įsakymą aštuntą dieną buvo apipjaustytas. 
\par 5 Abraomas buvo šimto metų, kai jam gimė sūnus Izaokas. 
\par 6 Sara tarė: “Dievas man suteikė juoko, ir visi kiti, kas išgirs, juoksis su manimi. 
\par 7 Kas būtų pasakęs Abraomui, kad Sara maitins kūdikį? Aš pagimdžiau jam sūnų jo senatvėje”. 
\par 8 Kai vaikas paaugo ir buvo nujunkytas, Abraomas tą dieną iškėlė didelį pokylį. 
\par 9 Sara pamatė egiptietės Hagaros sūnų, kurį ta pagimdė Abraomui, besišaipantį iš Izaoko, 
\par 10 ir tarė Abraomui: “Išvaryk šitą vergę ir jos sūnų! Jis nebus paveldėtojas drauge su mano sūnumi Izaoku”. 
\par 11 Tai labai nepatiko Abraomui dėl jo sūnaus. 
\par 12 Tačiau Dievas tarė Abraomui: “Nesisielok dėl berniuko ir dėl vergės! Visa, ką Sara tau sako, klausyk jos! Nes iš Izaoko tau bus pašaukti palikuonys. 
\par 13 Bet ir vergės sūnų padarysiu didele tauta, nes jis yra tavo palikuonis”. 
\par 14 Abraomas, atsikėlęs anksti rytą, ėmė duonos bei odinę vandens ir davė Hagarai, uždėdamas jai ant pečių, atidavė vaiką ir išleido. Ji išėjusi klaidžiojo Beer Šebos dykumoje. 
\par 15 Išsibaigus vandeniui odinėje, ji paliko vaiką po vienu krūmokšniu. 
\par 16 Paėjusi atsisėdo priešais jį lanko šūvio atstumu. Ji sakė: “Negaliu matyti mirštančio vaiko”. Ir graudžiai verkė. 
\par 17 Dievas išgirdo berniuko balsą, ir Dievo angelas iš dangaus tarė Hagarai: “Kas tau, Hagara? Nebijok! Dievas išgirdo berniuko balsą. 
\par 18 Kelkis, imk berniuką ir laikyk jį tvirtai savo rankose, nes Aš jį padarysiu didele tauta!” 
\par 19 Dievas atvėrė jai akis, ir ji pamatė šulinį. Nuėjusi pripildė odinę vandens ir pagirdė berniuką. 
\par 20 Dievas buvo su juo. Jis užaugo, gyveno dykumoje ir tapo šauliu. 
\par 21 Jis gyveno Parano dykumoje, ir jo motina parinko jam žmoną iš Egipto šalies. 
\par 22 Anuo metu Abimelechas ir jo kariuomenės vadas Picholas kalbėjo Abraomui: “Dievas yra su tavimi visame, ką tu darai. 
\par 23 Dabar tad prisiek Dievu, kad nekenksi nei man, nei mano vaikams, nei jų palikuonims, bet kaip aš maloningai su tavimi elgiausi, taip ir tu elgsiesi su manimi ir mano kraštu, kuriame gyveni kaip ateivis!” 
\par 24 Abraomas atsakė: “Prisiekiu”. 
\par 25 Tačiau Abraomas priekaištavo Abimelechui dėl šulinio, kurį Abimelecho tarnai buvo pasigrobę. 
\par 26 Abimelechas atsakė: “Aš nežinau, kas tai padarė. Tu man nieko nesakei, aš nieko apie tai negirdėjau iki šios dienos”. 
\par 27 Abraomas davė Abimelechui avių ir jaučių, ir juodu sudarė sandorą. 
\par 28 Abraomas atskyrė septynis ėriukus. 
\par 29 Abimelechas klausė Abraomo: “Ką gi reiškia šie septyni ėriukai, kuriuos tu atskyrei?” 
\par 30 Jis atsakė: “Tuos septynis ėriukus turi priimti iš manęs kaip įrodymą, kad aš iškasiau šitą šulinį”. 
\par 31 Ta vieta buvo pavadinta Beer Šeba, nes ten jie abu prisiekė. 
\par 32 Taip juodu padarė sutartį Beer Šeboje. Abimelechas ir jo kariuomenės vadas Picholas sugrįžo į filistinų kraštą. 
\par 33 Abraomas pasodino giraitę Beer Šeboje ir ten šaukėsi Viešpaties, amžinojo Dievo, vardo. 
\par 34 Abraomas gyveno ilgą laiką kaip ateivis filistinų krašte.



\chapter{22}

\par 1 Po šių įvykių Dievas mėgi no Abraomą. Jis tarė jam: “Abraomai!” Tas atsiliepė: “Aš čia!” 
\par 2 Tada Jis tarė: “Imk Izaoką, savo vienintelį sūnų, kurį myli, ir eik į Morijos šalį, ten aukok jį kaip deginamąją auką ant kalno, kurį tau parodysiu!” 
\par 3 Abraomas atsikėlė anksti rytą, pasibalnojo asilą, pasiėmė jaunuolius ir Izaoką, savo sūnų, prisiskaldė malkų deginamajai aukai ir išėjo į vietą, kurią jam Dievas buvo nurodęs. 
\par 4 Trečią dieną Abraomas iš tolo pamatė tą vietą. 
\par 5 Abraomas tarė savo jaunuoliams: “Pasilikite čia su asilu, o mes su sūnumi nueisime ten ir pagarbinę sugrįšime pas jus”. 
\par 6 Abraomas, paėmęs malkas deginamajai aukai, uždėjo ant savo sūnaus Izaoko pečių, o pats pasiėmė ugnies ir peilį. Jiems beeinant, 
\par 7 Izaokas tarė savo tėvui: “Mano tėve!” O tas atsiliepė: “Aš čia, sūnau!” Jis klausė: “Štai ugnis ir malkos! Bet kur yra ėriukas deginamajai aukai?” 
\par 8 Abraomas atsakė: “Dievas parūpins sau ėriuką deginamajai aukai, mano sūnau!” Taip juodu ėjo toliau. 
\par 9 Jiems atėjus į vietą, kurią Dievas buvo nurodęs, Abraomas pastatė aukurą, uždėjo ant jo malkas, surišo savo sūnų Izaoką ir jį uždėjo ant aukuro. 
\par 10 Abraomas ištiesė savo ranką ir paėmė peilį, kad nužudytų sūnų. 
\par 11 Viešpaties angelas pašaukė jį iš dangaus: “Abraomai! Abraomai!” Tas atsakė: “Aš čia!” 
\par 12 “Nekelk savo rankos prieš vaiką ir nieko jam nedaryk! Dabar žinau, kad bijai Dievo ir nepagailėjai man savo vienintelio sūnaus”. 
\par 13 Abraomas, pakėlęs akis, pamatė netoliese aviną, įstrigusį ragais į tankų krūmokšnį. Jis paėmė jį ir aukojo deginamąją auką savo sūnaus vietoje. 
\par 14 Abraomas pavadino tą vietą “Viešpats mato”. Ir šiandien dar sakoma: “Ant kalno, kur Viešpats mato”. 
\par 15 Viešpaties angelas antrą kartą pašaukė Abraomą iš dangaus 
\par 16 ir tarė: “Savimi prisiekiu,­sako Viešpats,­kadangi tu tai padarei ir nepagailėjai savo vienintelio sūnaus, 
\par 17 Aš laiminte tave palaiminsiu ir dauginte padauginsiu tavo palikuonis, kad jų bus kaip žvaigždžių danguje ir kaip smilčių jūros pakrantėje. Tavo palikuonys užims savo priešų vartus, 
\par 18 ir tavo palikuonyse bus palaimintos visos žemės tautos dėl to, kad paklausei mano balso”. 
\par 19 Abraomas sugrįžo pas jaunuolius, ir jie nuėjo į Beer Šebą; ir Abraomas gyveno Beer Šeboje. 
\par 20 Po šių įvykių Abraomui buvo pranešta: “Milka pagimdė sūnų tavo broliui Nahorui: 
\par 21 pirmagimį Ucą ir Būzą, jo brolį, ir Kemuelį, Aramo tėvą, 
\par 22 Kesedą, Hazoją, Pildašą, Idlafą ir Betuelį”. 
\par 23 Betuelio duktė buvo Rebeka. Šituos aštuonis Milka pagimdė Abraomo broliui Nahorui. 
\par 24 Be to, jo sugulovė Reuma pagimdė Tebachą, Gahamą, Tahašą ir Maaką.



\chapter{23}


\par 1 Sara gyveno šimtą dvidešimt septynerius metus. 
\par 2 Ji mirė Kirjat Arboje, tai yra Hebrone, Kanaano krašte. Ir Abraomas atėjo Saros apraudoti ir apverkti. 
\par 3 Abraomas paliko mirusiąją ir kalbėjo Heto vaikams: 
\par 4 “Aš esu ateivis ir svečias tarp jūsų. Duokite man nuosavybėn žemės kapui palaidoti mirusiąją”. 
\par 5 Heto vaikai atsakė Abraomui: 
\par 6 “Paklausyk mūsų, viešpatie! Tu esi Dievo kunigaikštis tarp mūsų. Laidok savo mirusiąją geriausiame mūsų kape! Nė vienas iš mūsų tau neatsakys kapo, kad galėtum palaidoti mirusiąją”. 
\par 7 Abraomas atsistojęs nusilenkė hetitams, to krašto žmonėms, 
\par 8 ir toliau kalbėjo: “Jei sutinkate, kad palaidočiau savo mirusiąją, tai paklausykite manęs ir prašykite už mane Efroną, Coharo sūnų, 
\par 9 kad jis man parduotų Machpelos olą, kuri jam priklauso ir yra jo lauko gale! Už tiek, kiek ji verta, jis man ją teparduoda nuosavybėn kapinėms”. 
\par 10 Tuo metu Efronas sėdėjo tarp Heto vaikų. Hetitas Efronas, girdint hetitams, atėjusiems prie miesto vartų, kalbėjo: 
\par 11 “Ne, mano viešpatie, paklausyk manęs! Žemę aš tau dovanoju ir olą, kuri yra joje. Savo tautiečių akyse aš tau ją dovanoju. Laidok savo mirusiąją”. 
\par 12 Abraomas nusilenkė to krašto žmonėms 
\par 13 ir, jiems girdint, kalbėjo Efronui: “Malonėk paklausyti manęs! Aš duosiu tau pinigus už lauką. Paimk juos iš manęs, kad galėčiau ten palaidoti savo mirusiąją”. 
\par 14 Efronas atsakė Abraomui: 
\par 15 “Paklausyk manęs, viešpatie! Tas žemės sklypas vertas keturių šimtų šekelių sidabro. Ką tai reiškia man ar tau? Palaidok savo mirusiąją”. 
\par 16 Abraomas sutiko su Efronu. Pirklių naudojamais pinigais jis atsvėrė Efronui keturis šimtus šekelių sidabro, kurį šis, hetitams girdint, minėjo. 
\par 17 Taip Efrono sklypas su ola, kuris buvo Machpeloje, ties Mamre, visi medžiai lauke, kurie augo aplinkui, tapo 
\par 18 Abraomo nuosavybe, matant hetitams, kurie buvo ten. 
\par 19 Abraomas palaidojo savo žmoną Sarą Machpelos lauko oloje, esančioje ties Mamre, Hebrone, Kanaano šalyje. 
\par 20 Taip sklypas ir ola, kuri buvo jame, iš hetitų perėjo Abraomo nuosavybėn kapinėms.



\chapter{24}


\par 1 Abraomas paseno ir sulaukė žilos senatvės. Viešpats viskuo jį laimino. 
\par 2 Abraomas tarė savo namų vyriausiajam tarnui, kuris prižiūrėjo visa, kas jam priklausė: “Dėk savo ranką po mano šlaunimi, 
\par 3 kad prisaikdinčiau tave Viešpaties, žemės ir dangaus Dievo vardu, kad neimsi žmonos mano sūnui iš kanaaniečių dukterų, tarp kurių gyvenu, 
\par 4 bet vyksi į mano kraštą, pas mano gimines, ir ten paimsi žmoną mano sūnui Izaokui”. 
\par 5 Tarnas jam atsakė: “O gal ta moteris nenorės eiti su manimi į šitą kraštą? Ar tuomet turėsiu sugrąžinti tavo sūnų į tą kraštą, iš kurio išvykai?” 
\par 6 Abraomas jam tarė: “Saugokis, kad negrąžintum mano sūnaus tenai! 
\par 7 Viešpats, dangaus Dievas, kuris mane išvedė iš mano tėvo namų, iš mano gimtojo krašto, man kalbėjo ir prisiekė: ‘Tavo palikuonims duosiu šį kraštą’. Jis siųs savo angelą pirma tavęs, ir tu iš ten paimsi žmoną mano sūnui. 
\par 8 O jei ta moteris nenorės eiti su tavimi, būsi laisvas nuo šito įpareigojimo. Tik mano sūnaus nesugrąžink tenai!” 
\par 9 Tada tarnas padėjo savo ranką po savo valdovo Abraomo šlaunimi ir jam prisiekė. 
\par 10 Tarnas ėmė dešimt savo valdovo kupranugarių, geriausių bei brangiausių dovanų ir išvyko į Mesopotamiją, į Nahoro miestą. 
\par 11 Vakare leido kupranugariams pailsėti už miesto, prie vandens šulinio, tuo metu, kai moterys eina semti vandens. 
\par 12 Jis meldėsi: “Viešpatie, mano valdovo Abraomo Dieve, duok man sėkmę šiandien ir tuo parodyk savo malonę mano valdovui Abraomui! 
\par 13 Štai stoviu prie šulinio, o miesto gyventojų dukterys ateis semti vandens. 
\par 14 Jei mergaitė, kuriai sakysiu: ‘Prašau, palenk savo ąsotį ir leisk man atsigerti’, atsakys: ‘Gerk! Aš ir tavo kupranugarius pagirdysiu’, ji bus ta, kurią paskyrei savo tarnui Izaokui. Iš to suprasiu, kad parodei malonę mano valdovui”. 
\par 15 Jam dar nebaigus kalbėti, atėjo Rebeka, Abraomo brolio Nahoro žmonos Milkos sūnaus Betuelio duktė, nešina ąsočiu ant peties. 
\par 16 Mergina buvo labai graži, mergaitė, kurios joks vyras nebuvo pažinęs. Ji, nusileidusi prie šulinio ir pasisėmusi vandens, lipo aukštyn. 
\par 17 Tarnas nuskubėjo jos pasitikti ir tarė: “Duok man truputį vandens atsigerti iš savo ąsočio!” 
\par 18 Ji atsakė: “Gerk, mano viešpatie!” Skubiai nuleidusi ąsotį sau ant rankos, davė jam gerti. 
\par 19 Jam atsigėrus, ji tarė: “Ir tavo kupranugarius pagirdysiu”. 
\par 20 Skubiai išpylusi savo ąsotį į lovį, vėl nubėgo prie šulinio. Taip ji pagirdė visus jo kupranugarius. 
\par 21 Tuo metu jis tylomis ją stebėjo, norėdamas patirti, ar Viešpats padarė jo kelionę sėkmingą, ar ne. 
\par 22 Kupranugariams atsigėrus, jis išėmė auksinę sagtį kaktai, sveriančią pusę šekelio, ir dvi apyrankes, sveriančias dešimt šekelių aukso, 
\par 23 ir klausė: “Kieno duktė esi? Ar yra tavo tėvo namuose mums vietos pernakvoti?” 
\par 24 Ji atsakė: “Aš esu duktė Betuelio, Milkos sūnaus, kurį ji pagimdė Nahorui”. 
\par 25 Ji tęsė: “Šiaudų ir pašaro pas mus daug, taip pat ir vietos nakvynei”. 
\par 26 Vyras nusilenkė ir pagarbino Viešpatį: 
\par 27 “Tebūna palaimintas Viešpats, mano valdovo Abraomo Dievas, kuris buvo malonus ir teisingas mano valdovui ir atvedė mane teisingu keliu į mano valdovo brolio namus!” 
\par 28 Mergaitė, nubėgusi namo, pranešė visiems, kas atsitiko. 
\par 29 Rebeka turėjo brolį, vardu Labaną. Ir Labanas išbėgo pas vyrą prie šulinio. 
\par 30 Pamatęs sagtį ir apyrankes ant sesers rankų ir išgirdęs sesers Rebekos žodžius: “Taip kalbėjo tas vyras”, atėjo jis pas tą vyrą, kuris stovėjo šalia kupranugarių prie šulinio, 
\par 31 ir jam tarė: “Ateik pas mus, Viešpaties palaimintasis! Ko stovi lauke? Aš paruošiau namą ir vietą kupranugariams”. 
\par 32 Taip jis tą vyrą parsivedė į savo namus, nubalnojo kupranugarius, padavė šiaudų bei pašaro jiems ir vandens jam ir su juo buvusiems vyrams kojoms nuplauti. 
\par 33 Ir jam buvo paduota maisto, bet jis tarė: “Aš nevalgysiu, kol nepasakysiu, dėl ko esu siųstas”. Labanas tarė: “Kalbėk!” 
\par 34 Jis tarė: “Aš esu Abraomo tarnas. 
\par 35 Viešpats labai palaimino mano valdovą: jis tapo didžiu ir Jis jam davė avių ir galvijų, sidabro ir aukso, tarnų ir tarnaičių, kupranugarių ir asilų. 
\par 36 Mano valdovo žmona Sara senatvėje pagimdė sūnų mano valdovui, kuriam jis atidavė viską, ką turėjo. 
\par 37 Mano valdovas mane prisaikdino: ‘Neimk mano sūnui žmonos iš kanaaniečių dukterų, kurių šalyje gyvenu, 
\par 38 bet keliauk į mano tėvo namus, pas mano gimines, ir ten surask mano sūnui žmoną’. 
\par 39 Tada atsakiau savo valdovui: ‘O gal ta moteris nesutiks keliauti su manimi?’ 
\par 40 Bet jis man tarė: ‘Viešpats, kurio akivaizdoje vaikščioju, siųs su tavimi angelą ir padarys tavo kelionę sėkmingą, ir tu paimsi mano sūnui žmoną iš mano giminės ir iš mano tėvo namų. 
\par 41 Jeigu, tau atvykus pas mano gimines, jie tau jos neduos, būsi laisvas nuo priesaikos’. 
\par 42 Šiandien, atėjęs prie šulinio, tariau: ‘Viešpatie, mano valdovo Abraomo Dieve, jei darai mano kelionę sėkmingą, 
\par 43 tai aš dabar atsistosiu prie šito vandens šulinio. Ta mergaitė, kuriai atėjus semti vandens tarsiu: ‘Duok man truputį vandens atsigerti iš savo ąsočio’, 
\par 44 o ji atsakys: ‘Gerk, ir tavo kupranugarius pagirdysiu’, bus moteris, kurią Viešpats paskyrė mano valdovo sūnui’. 
\par 45 Man dar nebaigus kalbėti, atėjo Rebeka su ąsočiu ant peties ir, nusileidusi prie šulinio, sėmė. Tada jai tariau: ‘Duok man gerti!’ 
\par 46 Ji, skubiai nuleidusi ąsotį nuo peties, tarė: ‘Gerk! Aš ir tavo kupranugarius pagirdysiu!’ Aš gėriau, o ji pagirdė ir kupranugarius. 
\par 47 Po to paklausiau: ‘Kieno tu duktė?’ Ji atsakė: ‘Esu duktė Betuelio, Nahoro sūnaus, kurį Milka jam pagimdė’. Tada užkabinau sagtį ant jos kaktos ir uždėjau apyrankes jai ant rankų. 
\par 48 Nusilenkęs pagarbinau Viešpatį ir palaiminau mano valdovo Abraomo Viešpatį Dievą, kuris mane atvedė teisingu keliu, kad imčiau mano valdovo brolio dukterį jo sūnui. 
\par 49 Taigi dabar, jei norite parodyti mano valdovui malonę ir ištikimybę, sakykite, o jei ne, tai pasakykite man, kad galėčiau pasukti į dešinę ar į kairę”. 
\par 50 Tada Labanas ir Betuelis atsakė: “Tai Viešpaties padaryta. Negalime nei prieštarauti, nei pritarti. 
\par 51 Štai Rebeka yra tavo akivaizdoje. Imk ją ir eik, tebūna ji tavo valdovo sūnaus žmona, kaip Viešpats kalbėjo”. 
\par 52 Išgirdęs jų žodžius, Abraomo tarnas pagarbino Viešpatį, nusilenkdamas iki žemės. 
\par 53 Tarnas, išėmęs sidabrinių ir auksinių indų bei drabužių, juos dovanojo Rebekai; be to, jis dovanojo vertingų daiktų jos broliui ir motinai. 
\par 54 Jie valgė, gėrė ir pasiliko tenai per naktį. Rytą, jiems atsikėlus, jis tarė: “Leiskite man keliauti pas mano valdovą”. 
\par 55 Jos brolis ir motina prašė: “Tepasilieka mergina pas mus kurį laiką, nors dešimt dienų, po to išleisime”. 
\par 56 Jis atsakė: “Netrukdykite manęs, nes Viešpats padarė mano kelionę sėkmingą. Leiskite man grįžti pas savo valdovą”. 
\par 57 Jie tarė: “Pašaukime mergaitę ir jos paklauskime”. 
\par 58 Pašaukę Rebeką, klausė: “Ar keliausi su šiuo vyru?” Ji atsakė: “Taip, keliausiu”. 
\par 59 Tada jie išlydėjo Rebeką, jos auklę, Abraomo tarną ir jo vyrus. 
\par 60 Atsisveikindami jie laimino ją: “Mūsų sesuo, tapk nesuskaitomų tūkstančių motina, tavo palikuonys tevaldo savo priešų miestų vartus!” 
\par 61 Tada Rebeka ir jos tarnaitės išjojo ant kupranugarių, sekdamos tą vyrą. 
\par 62 Tuo metu Izaokas ėjo keliu nuo Lahai Roijo šulinio, nes jis gyveno pietų krašte. 
\par 63 Pavakary Izaokas buvo išėjęs į lauką pamąstyti. Pakėlęs akis, jis pamatė ateinančius kupranugarius. 
\par 64 Rebeka, pamačiusi Izaoką, nulipo nuo kupranugario 
\par 65 ir klausė tarną: “Kas tas vyras, kuris eina mums priešais per lauką?” Tarnas atsakė: “Jis yra mano valdovas!” Tada ji apsigaubė šydu. 
\par 66 Tarnas papasakojo Izaokui viską, ką buvo padaręs. 
\par 67 Izaokas įsivedė mergaitę į savo motinos palapinę. Jis paėmė Rebeką, ir ji tapo jo žmona, ir jis pamilo ją. Izaokas buvo paguostas po savo motinos mirties.



\chapter{25}

\par 1 Abraomas dar vedė kitą žmoną, vardu Ketūra. 
\par 2 Ji pagimdė Zimraną, Jokšaną, Medaną, Midjaną, Išbaką ir Šuachą. 
\par 3 Jokšanas turėjo du sūnus: Šebą ir Dedaną. Dedano sūnūs buvo: Ašūras, Letušas ir Leumas. 
\par 4 Midjano sūnūs buvo: Efa, Eferas, Henochas, Abida ir Eldava. Visi šitie yra Ketūros vaikai. 
\par 5 Abraomas atidavė Izaokui visa, ką turėjo. 
\par 6 O sugulovių sūnums Abraomas davė dovanų ir, dar gyvas būdamas, juos išsiuntė į rytų šalį, toliau nuo Izaoko. 
\par 7 Abraomas išgyveno šimtą septyniasdešimt penkerius metus. 
\par 8 Abraomas mirė sulaukęs žilos senatvės ir pasisotinęs gyvenimu. Jis susijungė su savo tauta. 
\par 9 Jį palaidojo jo sūnūs Izaokas ir Izmaelis Machpelos oloje, Coharo sūnaus hetito Efrono lauke, kuris buvo ties Mamre. 
\par 10 Tą lauką Abraomas buvo pirkęs iš hetitų. Ten yra palaidoti Abraomas ir jo žmona Sara. 
\par 11 Abraomui mirus, Dievas laimino jo sūnų Izaoką, kuris gyveno prie Lahai Roijo šulinio. 
\par 12 Šitie yra palikuonys Izmaelio, Abraomo sūnaus, kurį egiptietė Hagara, Saros tarnaitė, pagimdė Abraomui. 
\par 13 Šitie yra jų vardai, kaip jie buvo vadinami savo giminėse: Izmaelio pirmagimis Nebajotas ir Kedaras, Adbeelis, Mibsamas, 
\par 14 Mišma, Dūma, Masa, 
\par 15 Hadaras, Tema, Jetūras, Nafišas ir Kedma. 
\par 16 Šitie yra Izmaelio sūnūs ir jų vardai pagal jų miestus ir gyvenvietes. Dvylika kunigaikščių savo giminėse. 
\par 17 Izmaelis gyveno šimtą trisdešimt septynerius metus. Jis mirė ir susijungė su savo tauta. 
\par 18 Izmaelio palikuonys gyveno nuo Havilos iki Šūro, priešais Egiptą, Ašūro link. Jis mirė visų savo brolių akivaizdoje. 
\par 19 Šita yra Abraomo sūnaus Izaoko giminė. Abraomas turėjo sūnų Izaoką. 
\par 20 Izaokas, turėdamas keturiasdešimt metų, vedė Rebeką, Betuelio iš Mesopotamijos dukterį, Labano seserį. 
\par 21 Izaokas meldėsi už savo žmoną, nes ji buvo nevaisinga. Viešpats išklausė jo maldą, ir jo žmona Rebeka pastojo. 
\par 22 Kūdikiai kovojo tarpusavyje jos įsčiose, ir ji tarė: “Jei taip yra, tai kodėl man taip?” Ji nuėjo pasiklausti Viešpaties. 
\par 23 Viešpats jai tarė: “Dvi tautos yra tavo įsčiose, dvi giminės gims iš tavęs ir persiskirs. Viena giminė bus galingesnė už kitą, vyresnysis tarnaus jaunesniajam”. 
\par 24 Atėjus metui gimdyti, gimė dvynukai. 
\par 25 Pirmasis buvo visas plaukuotas; jie pavadino jį Ezavu. 
\par 26 Jo brolis gimdamas laikėsi Ezavo kulnies; jį pavadino Jokūbu. Tuo laiku Izaokui buvo šešiasdešimt metų. 
\par 27 Berniukams užaugus, Ezavas tapo geru medžiotoju, laukų žmogumi, o Jokūbas buvo ramus ir mėgo gyventi palapinėse. 
\par 28 Izaokas mylėjo Ezavą, nes mėgo jo sumedžiotą žvėrieną, bet Rebeka labiau mylėjo Jokūbą. 
\par 29 Kartą Jokūbas išsivirė viralą, o Ezavas parėjo iš lauko nuvargęs. 
\par 30 Ezavas tarė Jokūbui: “Duok man savo raudonojo viralo, nes aš esu nuvargęs!” Todėl jį praminė Edomu. 
\par 31 O Jokūbas pasakė: “Parduok man savo pirmagimio teisę”. 
\par 32 Ezavas tarė: “Aš mirštu, ką gi man padės pirmagimystė?” 
\par 33 Jokūbas pasakė: “Prisiek man!” Taip jis prisiekė ir pardavė Jokūbui savo pirmagimio teisę. 
\par 34 Tada Jokūbas davė Ezavui duonos ir lęšių viralo. Jis valgė, gėrė ir pavalgęs išėjo. Taip Ezavas paniekino savo pirmagimystę.



\chapter{26}

\par 1 Šalyje vėl kilo badas kaip anksčiau Abraomo laikais. Izaokas nuėjo pas filistinų karalių Abimelechą į Gerarą. 
\par 2 Jam pasirodė Viešpats ir tarė: “Neik į Egiptą. Gyvenk žemėje, kurią tau nurodysiu. 
\par 3 Būk kaip ateivis šitoje šalyje. Aš būsiu su tavimi ir laiminsiu tave, nes tau ir tavo palikuonims duosiu visas šias žemes ir ištesėsiu priesaiką, kurią daviau tavo tėvui Abraomui. 
\par 4 Aš padauginsiu tavo palikuonis, kad jų bus tiek, kiek danguje žvaigždžių, ir duosiu jiems visas šias žemes. Tavo palikuonyse bus palaimintos visos žemės giminės, 
\par 5 nes Abraomas paklausė mano balso ir laikėsi mano įstatymų, įsakymų, nuostatų ir nurodymų”. 
\par 6 Izaokas pasiliko Gerare. 
\par 7 Tos vietos vyrams, teiraujantis apie jo žmoną, jis sakė: “Ji mano sesuo”, nes jis bijojo sakyti: “Ji mano žmona”, kad tos vietos vyrai neužmuštų jo dėl Rebekos, nes ji buvo graži. 
\par 8 Pagyvenus ten ilgesnį laiką, pasitaikė, kad filistinų karalius Abimelechas, žiūrėdamas pro langą, pamatė Izaoką, glamonėjantį savo žmoną Rebeką. 
\par 9 Abimelechas pasišaukė Izaoką ir tarė: “Man aišku, kad ji tavo žmona! Kodėl sakei: ‘Ji mano sesuo’?” Izaokas jam atsakė: “Bijojau, kad nereikėtų dėl jos mirti”. 
\par 10 Tada Abimelechas tarė: “Kodėl mums taip padarei? Juk kas nors galėjo sugulti su tavo žmona, ir tu būtum apkaltinęs mus!” 
\par 11 Abimelechas įspėjo visą tautą: “Kas palies šitą vyrą ar jo žmoną, bus baudžiamas mirtimi”. 
\par 12 Izaokas įdirbo žemę ir gavo tais metais šimteriopą derlių, nes Viešpats jį laimino. 
\par 13 Ir jis tapo didis žmogus, ir toliau augo, ir plėtėsi, kol tapo labai didis. 
\par 14 Jis turėjo daug gyvulių ir avių, ir didelį skaičių tarnų, todėl jam pavydėjo filistinai. 
\par 15 Visus šulinius, kuriuos jo tėvo tarnai buvo iškasę Abraomo dienomis, filistinai užvertė žemėmis. 
\par 16 Abimelechas tarė Izaokui: “Pasitrauk nuo mūsų, nes tu pasidarei daug galingesnis už mus!” 
\par 17 Izaokas pasitraukė iš ten, pasistatė palapinę Geraros slėnyje ir ten gyveno. 
\par 18 Tada Izaokas vėl atkasė šulinius, kuriuos Abraomas, jo tėvas, buvo iškasęs ir filistinai, Abraomui mirus, buvo užvertę. Jis juos pavadino tais pačiais vardais, kuriais jo tėvas juos buvo pavadinęs. 
\par 19 Izaoko tarnai kasė šulinį slėnyje ir rado vandens versmę. 
\par 20 Geraro piemenys ginčijosi su Izaoko piemenimis: “Mums priklauso vanduo!” Jis pavadino tą šulinį Eseku, nes jie ginčijosi su juo. 
\par 21 Po to jis iškasė kitą šulinį, bet jie ir dėl to susiginčijo. Jis jį pavadino Sitna. 
\par 22 Iš ten jis kėlėsi toliau ir vėl iškasė šulinį. Dėl šito jie nebesiginčijo. Jis jį pavadino Rehobotu ir tarė: “Dabar mums Viešpats suteikė daug vietos, galėsime plėstis šalyje”. 
\par 23 Iš ten jis persikėlė į Beer Šebą. 
\par 24 Tą naktį pasirodė jam Viešpats ir tarė: “Aš esu tavo tėvo Abraomo Dievas. Nebijok! Aš esu su tavimi ir dėl mano tarno Abraomo laiminsiu tave bei padauginsiu tavo palikuonis”. 
\par 25 Izaokas čia pastatė aukurą ir šaukėsi Viešpaties vardo, ir čia jis ištiesė savo palapinę. Jo tarnai toje vietoje iškasė šulinį. 
\par 26 Abimelechas iš Geraro atvyko pas jį su savo draugu Achuzatu ir kariuomenės vadu Picholu. 
\par 27 Izaokas paklausė: “Ko atėjote? Juk jūs nekenčiate manęs ir mane išvarėte!” 
\par 28 Jie atsakė: “Mes aiškiai matome, kad Viešpats yra su tavimi, todėl sakome: ‘Tarkimės, sudarykime sandorą!’ 
\par 29 Nesielk su mumis piktai, juk ir mes tavęs neskriaudėme, gerai elgėmės su tavimi ir išleidome ramybėje. Tu esi Viešpaties palaimintasis!” 
\par 30 Izaokas iškėlė jiems pokylį, jie valgė ir gėrė. 
\par 31 Atsikėlę anksti rytą, jie sudarė sandorą. Po to Izaokas juos išleido, ir jie iškeliavo ramybėje. 
\par 32 Tą pačią dieną Izaoko tarnai atėję pranešė jam apie naujai iškastą šulinį, sakydami: “Mes radome vandens”. 
\par 33 Ir jis pavadino jį Šiba, todėl tas miestas ligi šios dienos tebevadinamas Beer Šeba. 
\par 34 Ezavas, turėdamas keturiasdešimt metų, vedė hetito Beerio dukterį Juditą ir hetito Elono dukterį Basmatą. 
\par 35 Jos apkartino Izaoko ir Rebekos gyvenimą.



\chapter{27}


\par 1 Izaokas paseno, ir jo akys taip aptemo, kad jis nebegalėjo matyti. Jis pasišaukė savo vyresnįjį sūnų Ezavą ir tarė jam: “Mano sūnau”. Tas atsiliepė: “Aš čia”. 
\par 2 Jis tarė: “Aš jau pasenau, nežinau savo mirties dienos. 
\par 3 Imk savo medžioklės įrankius, strėlinę ir lanką, ir, išėjęs į lauką, sumedžiok ką nors. 
\par 4 Paruošk man valgį, kokį mėgstu, atnešk jį man, kad valgyčiau ir mano siela tave palaimintų, prieš man numirštant”. 
\par 5 Rebeka girdėjo Izaoką kalbant savo sūnui Ezavui. Kai Ezavas išėjo į lauką medžioti, 
\par 6 Rebeka tarė savo sūnui Jokūbui: “Aš girdėjau tėvą kalbant tavo broliui Ezavui: 
\par 7 ‘Sumedžiojęs ką, paruošk man skanų valgį, kad pavalgęs galėčiau tave palaiminti Viešpaties akivaizdoje prieš savo mirtį’. 
\par 8 Taigi dabar, sūnau, klausyk mano patarimo, ką tau sakysiu. 
\par 9 Eik ir išrink iš kaimenės du geriausius ožiukus ir atnešk, kad paruoščiau iš jų tėvo mėgstamą valgį. 
\par 10 Tu jį įneši tėvui, kad jis valgytų ir tave palaimintų prieš savo mirtį”. 
\par 11 Bet Jokūbas atsakė savo motinai Rebekai: “Mano brolio Ezavo kūnas apaugęs plaukais, o aš žmogus neplaukuotas. 
\par 12 Jei mano tėvas mane palytės, tada pasirodysiu kaip apgavikas. Taip užsitrauksiu prakeikimą­ne palaiminimą”. 
\par 13 Tačiau motina jam atsakė: “Sūnau, tas prakeikimas tekrinta ant manęs! Tik klausyk manęs ir nuėjęs atnešk, ką sakiau!” 
\par 14 Taigi jis nuėjęs atnešė motinai ožiukus, o ji pagamino skanų valgį, kurį mėgo tėvas. 
\par 15 Tada Rebeka, paėmusi savo vyriausiojo sūnaus Ezavo geriausius drabužius, kurie buvo namie, apvilko jais jaunesnįjį sūnų Jokūbą, 
\par 16 o ožiukų kailiais apvyniojo jo neplaukuotas rankas ir kaklą. 
\par 17 Tada ji padavė paruoštą valgį ir duonos savo sūnui Jokūbui. 
\par 18 Jokūbas, įėjęs pas savo tėvą, tarė: “Mano tėve!” O tas atsiliepė: “Aš čia. Kas tu esi, mano sūnau?” 
\par 19 Jokūbas atsakė: “Aš esu Ezavas, tavo pirmagimis. Padariau, kaip man liepei. Kelkis, sėsk ir valgyk, ką sumedžiojau, kad tavo siela palaimintų mane”. 
\par 20 Izaokas paklausė: “Kaipgi, mano sūnau, taip greitai suradai?” Tas atsakė: “Viešpats, tavo Dievas, suteikė man laimės”. 
\par 21 Izaokas tarė Jokūbui: “Prieik, kad galėčiau paliesti tave, mano sūnau, ir įsitikinčiau, ar tu tikrai esi mano sūnus Ezavas”. 
\par 22 Jokūbas priėjo prie savo tėvo. Tas, jį palietęs, tarė: “Balsas Jokūbo, bet rankos Ezavo”. 
\par 23 Jis neatpažino jo, nes rankos buvo plaukuotos kaip jo brolio Ezavo; taip Izaokas palaimino Jokūbą. 
\par 24 Tėvas paklausė: “Ar tu tikrai esi mano sūnus Ezavas?” Tas atsiliepė: “Taip, esu”. 
\par 25 Izaokas tarė: “Atnešk, ką sumedžiojai, kad mano siela galėtų tave palaiminti”. Jokūbas atnešė jam, ir šis valgė, ir jis atnešė jam vyno, ir šis gėrė. 
\par 26 Tada jo tėvas Izaokas jam tarė: “Prieik ir pabučiuok mane, sūnau!” 
\par 27 Šis priėjęs pabučiavo jį, o tėvas, užuodęs Ezavo drabužių kvapą, laimindamas jį tarė: “Mano sūnaus kvapas, kaip kvapas laukų, kuriuos palaimino Viešpats. 
\par 28 Tau Dievas teduoda dangaus rasos, derlingos žemės ir apsčiai javų bei vyno! 
\par 29 Tetarnauja tau tautos ir tenusilenkia prieš tave giminės! Viešpatauk savo broliams, ir tesilenkia prieš tave tavo motinos sūnūs! Kas tave keiktų, tebūna prakeiktas, o kas tave laimintų, tebūna palaimintas!” 
\par 30 Izaokui baigus laiminti Jokūbą ir jam tik išėjus iš savo tėvo Izaoko, jo brolis Ezavas grįžo iš medžioklės. 
\par 31 Jis irgi paruošė skanų valgį ir, atnešęs tėvui, tarė: “Kelkis, tėve, ir valgyk savo sūnaus medžioklės laimikio, kad tavo siela mane palaimintų!” 
\par 32 Bet Izaokas klausė: “Kas tu esi?” Šis atsakė: “Aš esu tavo sūnus, tavo pirmagimis Ezavas”. 
\par 33 Tada Izaokas išsigando ir drebėdamas tarė: “Kas gi buvo tas, kuris anksčiau sumedžiojo ir man atnešė valgį? Aš, prieš tau pareinant, valgiau ir jį palaiminau. Jis ir bus palaimintas!” 
\par 34 Ezavas, išgirdęs savo tėvo žodžius, pradėjo labai garsiai ir graudžiai verkti, sakydamas tėvui: “Mano tėve, palaimink ir mane!” 
\par 35 Bet tėvas atsakė: “Tavo brolis klasta gavo tavo palaiminimą”. 
\par 36 Ezavas tarė: “Teisingai jį pavadino Jokūbu. Juk jis jau du kartus apgavo mane: paėmė mano pirmagimio teisę ir štai dabar­tavo palaiminimą. Nejaugi tu man nepalikai palaiminimo?” 
\par 37 Izaokas atsakė Ezavui: “Aš jį padariau tavo valdovu ir visus jo brolius atidaviau jam tarnais, javais ir vynu jį aprūpinau. Ką gi galiu padaryti dėl tavęs, mano sūnau?” 
\par 38 Ezavas tarė tėvui: “Tėve, ar tik vieną turi palaiminimą? Palaimink ir mane!” Ir Ezavas balsu verkė. 
\par 39 Jo tėvas Izaokas atsakė: “Tu neturėsi derlingos žemės savo gyvenvietėje ir dangaus rasos. 
\par 40 Savo kardu tu maitinsies ir savo broliui tarnausi. Bet ateis laikas, kada pasipriešinsi ir nusimesi jo jungą”. 
\par 41 Ezavas nekentė Jokūbo dėl tėvo palaiminimo. Ir Ezavas sakė savo širdyje: “Artėja gedulo dienos dėl tėvo, tada užmušiu savo brolį Jokūbą!” 
\par 42 Rebekai buvo perduoti jos vyresniojo sūnaus žodžiai. Ji tada pasišaukė savo jaunesnįjį sūnų Jokūbą ir tarė: “Tavo brolis Ezavas rengiasi atkeršyti tau ir nori užmušti tave. 
\par 43 Taigi dabar, mano sūnau, klausyk manęs! Bėk pas mano brolį Labaną į Charaną 
\par 44 ir gyvenk pas jį, kol paliaus tavo brolio rūstybė, 
\par 45 kol tavo brolio pyktis atsileis ir jis pamirš, ką jam padarei! Po to aš nusiųsiu ką nors, kad tave pargabentų. Kodėl turėčiau jūsų abiejų netekti vieną dieną?” 
\par 46 Rebeka tarė Izaokui: “Man įgriso mano gyvenimas dėl hetitų dukterų. Jei dar ir Jokūbas ves hetitę, tai kam man begyventi?”



\chapter{28}

\par 1 Izaokas pasišaukė Jokūbą, palaimino jį ir jam įsakė: “Neimk žmonos iš kanaaniečių giminės. 
\par 2 Keliauk į Mesopotamiją, į tavo motinos tėvo Betuelio namus, ir iš tavo motinos brolio Labano dukterų pasirink žmoną, 
\par 3 o visagalis Dievas telaimina tave ir tepadaro tave vaisingą, ir tepadaugina tave, kad iš tavęs kiltų daugybė tautų! 
\par 4 Jis tesuteikia tau ir tavo palikuonims Abraomo palaiminimą, kad paveldėtum žemę, kurioje esi svetimšalis, kurią Dievas atidavė Abraomui”. 
\par 5 Izaokas išleido Jokūbą. Tas išėjo į Mesopotamiją pas Labaną, siro Betuelio sūnų, Jokūbo ir Ezavo motinos Rebekos brolį. 
\par 6 Ezavas pamatė, kad Izaokas palaimino Jokūbą ir jį išsiuntė į Mesopotamiją žmonos pasirinkti ir, laimindamas jį, įsakė: “Neimk žmonos iš kanaaniečių dukterų”. 
\par 7 Jokūbas paklausė savo tėvo ir iškeliavo į Mesopotamiją. 
\par 8 Ezavas įsitikino, kad kanaanietės nepatinka jo tėvui Izaokui. 
\par 9 Tada Ezavas, nuėjęs pas Izmaelį, be savo turimųjų žmonų dar vedė Mahalatą, Abraomo sūnaus Izmaelio dukterį, Nebajoto seserį. 
\par 10 Jokūbas, išvykęs iš Beer Šebos, keliavo į Charaną. 
\par 11 Jis, pasiekęs vieną vietovę, ten pasiliko nakvoti, nes saulė jau buvo nusileidusi. Paėmęs vieną iš ten gulinčių akmenų, pasidėjo priegalviu ir atsigulė. 
\par 12 Jis sapnavo kopėčias, pastatytas ant žemės, kurių viršus siekė dangų, o Dievo angelai jomis laipiojo aukštyn ir žemyn. 
\par 13 Kopėčių viršuje stovėjo Viešpats ir tarė: “Aš esu Viešpats, tavo tėvo Abraomo ir Izaoko Dievas. Tą žemę, ant kurios guli, atiduosiu tau ir tavo palikuonims. 
\par 14 O tavo palikuonių bus kaip žemės dulkių; tu išsiplėsi į vakarus ir į rytus, į šiaurę ir į pietus; tavyje ir tavo palikuonyse bus palaimintos visos žemės giminės! 
\par 15 Aš būsiu su tavimi ir tave saugosiu, ir lydėsiu visur, ir vėl tave parvesiu į šitą žemę; nepaliksiu tavęs, kol įvykdysiu tai, ką esu pažadėjęs”. 
\par 16 Jokūbas, pabudęs iš miego, tarė: “Tikrai Viešpats yra šitoje vietoje, o aš to nežinojau!” 
\par 17 Jis nusigandęs tarė: “Kokia baisi šita vieta! Čia ne kas kita, kaip Dievo namai, dangaus vartai!” 
\par 18 Jokūbas, atsikėlęs anksti rytą, paėmė akmenį, kurį buvo pasidėjęs priegalviu, pastatė jį paminklu ir užpylė aliejaus ant jo. 
\par 19 Jis pavadino tą vietą Beteliu; anksčiau tas miestas vadinosi Lūzas. 
\par 20 Jokūbas padarė įžadą: “Jei Viešpats Dievas bus su manimi, mane saugos šitame kely ir duos man duonos valgyti ir drabužių apsivilkti, 
\par 21 jei ramybėje sugrįšiu į savo tėvo namus, tada Viešpats bus mano Dievas. 
\par 22 Ir šitas akmuo, kurį pastačiau paminklu, bus Dievo namai. Ir iš visko, ką man suteiksi, atiduosiu Tau dešimtąją dalį”.



\chapter{29}


\par 1 Jokūbas keliaudamas atėjo į rytų šalį. 
\par 2 Jis pamatė šulinį ir prie jo sugulusias tris avių bandas; iš to šulinio girdydavo bandas. Ant šulinio angos buvo užristas didelis akmuo. 
\par 3 Suvarius visas bandas, atrisdavo tą akmenį nuo šulinio angos ir pagirdydavo avis, po to vėl užrisdavo tą akmenį. 
\par 4 Jokūbas klausė: “Broliai, iš kur jūs esate?” Tie atsakė: “Iš Charano”. 
\par 5 Jis vėl klausė: “Ar pažįstate Labaną, Nahoro sūnų?” Jie atsakė: “Pažįstame”. 
\par 6 Jis klausė: “Kaip jam sekasi?” Tie atsakė: “Gerai. Štai jo duktė Rachelė ateina su avimis!” 
\par 7 Jokūbas tarė: “Dar anksti, ne laikas suvaryti gyvulius. Pagirdykite avis ir ganykite!” 
\par 8 Bet jie atsakė: “Negalime, kol suvarys visas bandas ir nuris tą akmenį nuo šulinio angos. Tik tada pagirdysime avis”. 
\par 9 Jam bekalbant su jais, Rachelė atėjo su savo tėvo avimis, kurias ji ganė. 
\par 10 Jokūbas, pamatęs Rachelę, savo dėdės Labano dukterį, ir savo dėdės Labano avis, priėjęs nurito akmenį nuo šulinio angos ir pagirdė savo motinos brolio Labano avis. 
\par 11 Po to Jokūbas pabučiavo Rachelę ir balsu pravirko. 
\par 12 Jokūbas pasisakė jai esąs jos tėvo brolis, Rebekos sūnus. Ta nubėgusi pranešė tėvui. 
\par 13 Labanas, išgirdęs žinią apie savo sesers sūnų Jokūbą, atbėgo jo pasitikti; jį apkabinęs ir pabučiavęs, nusivedė į savo namus, o tas viską išsipasakojo Labanui. 
\par 14 Labanas jam tarė: “Tikrai tu esi mano kūnas ir mano kaulas!” Jis gyveno pas jį visą mėnesį. 
\par 15 Ir Labanas tarė Jokūbui: “Ar dėl to, kad esi mano brolis, turėtum man veltui tarnauti? Pasakyk, kuo tau atlyginti?” 
\par 16 Labanas turėjo dvi dukteris: vyresnioji vardu Lėja, o jaunesnioji­Rachelė. 
\par 17 Lėjos akys buvo silpnos, o Rachelė buvo gražaus veido ir dailios išvaizdos. 
\par 18 Jokūbas pamilo Rachelę ir pasakė: “Aš tau tarnausiu septynerius metus už jaunesniąją dukterį Rachelę”. 
\par 19 Labanas atsakė: “Mieliau aš ją duosiu tau negu kitam. Lik pas mane!” 
\par 20 Jokūbas tarnavo už Rachelę septynerius metus. Kadangi jis mylėjo ją, jam tas laikas atrodė kaip kelios dienos. 
\par 21 Po to Jokūbas tarė Labanui: “Duok man mano žmoną, nes atėjo laikas, kad pas ją įeičiau!” 
\par 22 Labanas sukvietė visus tos vietovės žmones ir iškėlė puotą. 
\par 23 Vakare jis įvedė savo dukterį Lėją pas jį, ir jis įėjo pas ją. 
\par 24 Labanas davė Lėjai savo tarnaitę Zilpą. 
\par 25 Rytui išaušus, pasirodė, kad tai buvo Lėja. Tada jis tarė Labanui: “Ką man padarei! Ar ne už Rachelę tarnavau? Kam tad mane apgavai?” 
\par 26 Labanas atsakė: “Mūsų krašte taip nedaro, kad išleistų jaunesniąją anksčiau už vyresniąją. 
\par 27 Pabaik šią savaitę, po to duosiu tau Rachelę, už kurią tarnausi dar kitus septynerius metus!” 
\par 28 Jokūbas sutiko ir pabaigė tą savaitę. Tada Labanas davė jam savo dukterį Rachelę už žmoną. 
\par 29 Labanas davė savo dukteriai Rachelei tarnaitę Bilhą. 
\par 30 Jis įėjo ir pas Rachelę ir mylėjo ją labiau už Lėją; ir tarnavo dar kitus septynerius metus. 
\par 31 Viešpats matydamas, kad Jokūbas Lėjos nemylėjo, padarė ją vaisingą, o Rachelę­nevaisingą. 
\par 32 Lėja pastojo ir pagimdė sūnų, kurį pavadino Rubenu, nes ji sakė: “Viešpats atsižvelgė į mano sielvartą; dabar mane mylės mano vyras”. 
\par 33 Ji pagimdė kitą sūnų ir tarė: “Kadangi Viešpats išgirdo, kad manęs nemyli, Jis davė man dar ir šitą”. Ji pavadino jį Simeonu. 
\par 34 Po to ji vėl pagimdė sūnų ir tarė: “Dabar mano vyras prisiriš prie manęs, nes aš jam pagimdžiau tris sūnus”. Todėl ji pavadino jį Leviu. 
\par 35 Ji vėl pastojo ir pagimdė sūnų, ir tarė: “Dabar šlovinsiu Viešpatį”. Todėl ji pavadino jį Judu. Po to ji liovėsi gimdžiusi.



\chapter{30}

\par 1 Rachelė matydama, kad ji nevaisinga, pavydėjo savo seseriai Lėjai ir tarė Jokūbui: “Duok man vaikų, kitaip aš mirsiu!” 
\par 2 Jokūbas, supykęs ant Rachelės, tarė: “Ar aš Dievas, kuris tau vaikų neduoda?” 
\par 3 Tada ji tarė: “Štai mano tarnaitė Bilha. Įeik pas ją, kad ji pagimdytų ant mano kelių ir aš galėčiau turėti vaikų iš jos”. 
\par 4 Ji davė jam už žmoną savo tarnaitę Bilhą, ir Jokūbas įėjo pas ją. 
\par 5 Bilha pastojo ir pagimdė Jokūbui sūnų. 
\par 6 Tada Rachelė tarė: “Dievas teisingai nusprendė dėl manęs, išklausydamas mano balsą ir davė man sūnų”. Todėl ji pavadino jį Danu. 
\par 7 Rachelės tarnaitė Bilha pastojo ir pagimdė Jokūbui antrą sūnų. 
\par 8 Tada Rachelė tarė: “Didžiose grumtynėse grūmiausi su savo seserimi ir nugalėjau”. Ir ji pavadino jį Neftaliu. 
\par 9 Lėja matydama, kad nebegali daugiau gimdyti, davė Jokūbui savo tarnaitę Zilpą už žmoną. 
\par 10 Zilpa, Lijos tarnaitė, pagimdė Jokūbui sūnų. 
\par 11 Tada Lėja tarė: “Laimingai!” Ir ji pavadino jį Gadu. 
\par 12 Vėliau Zilpa pagimdė Jokūbui antrą sūnų. 
\par 13 Tada Lėja tarė: “Aš laimingoji! Nes moterys vadins mane palaiminta”. Ir ji pavadino jį Ašeru. 
\par 14 Kviečių pjūties metu Rubenas išėjęs rado mandragorų ir juos parnešė savo motinai Lėjai. Tada Rachelė tarė Lėjai: “Duok man savo sūnaus mandragorų”. 
\par 15 Bet ji atsakė: “Ar negana tau, kad turi mano vyrą, ar nori atimti ir mano sūnaus mandragorus?” Rachelė tarė: “Tegul jis šią naktį praleidžia su tavimi už tavo sūnaus mandragorus!” 
\par 16 Jokūbui pareinant vakare iš lauko, Lėja išėjo jo pasitikti ir tarė: “Tu eisi pas mane, nes aš tave pasamdžiau už mano sūnaus mandragorus”. Taip jis praleido su ja tą naktį. 
\par 17 Dievas išklausė Lėją; ji pagimdė Jokūbui penktąjį sūnų. 
\par 18 Tada Lėja tarė: “Dievas man atlygino, nes aš daviau savo tarnaitę savo vyrui”. Ji pavadino jį Isacharu. 
\par 19 Lėja vėl pastojo ir pagimdė Jokūbui šeštąjį sūnų. 
\par 20 Tada Lėja tarė: “Dievas apdovanojo mane gera dovana; dabar mano vyras gyvens su manimi, nes aš jam pagimdžiau šešis sūnus”. Ir ji praminė jį Zabulonu. 
\par 21 Po to ji pagimdė dukterį ir ją pavadino Dina. 
\par 22 Dievas atsiminė Rachelę, išklausė ją ir padarė vaisingą. 
\par 23 Ji pagimdė sūnų ir tarė: “Dievas pašalino mano gėdą”. 
\par 24 Ji pavadino jį Juozapu, sakydama: “Viešpats duos man dar kitą sūnų!” 
\par 25 Rachelei pagimdžius Juozapą, Jokūbas tarė Labanui: “Paleisk mane, grįšiu į tėvynę, į savo šalį! 
\par 26 Duok man mano žmonas ir mano vaikus, už kuriuos tau tarnavau, ir leisk man eiti. Tu juk žinai, kaip aš tau tarnavau!” 
\par 27 Labanas jam tarė: “O kad aš rasčiau malonę tavo akyse! Aš patyriau, kad Viešpats laimino mane dėl tavęs. 
\par 28 Nustatyk tu pats sau užmokestį, ir aš tau jį duosiu!” 
\par 29 Jokūbas atsakė: “Tu pats žinai, kaip tau tarnavau ir kokia tapo tavo banda mano priežiūroje. 
\par 30 Tu mažai turėjai prieš man atvykstant, bet dabar tai smarkiai padaugėjo, nes Viešpats tave laimino, kai aš atėjau. O dabar ar ne laikas man pasirūpinti savo namais?” 
\par 31 Labanas tarė: “Ką turiu tau duoti?” Jokūbas atsakė: “Nieko man neduok! Jei sutiksi su mano reikalavimu, aš vėl ganysiu ir saugosiu tavo kaimenę: 
\par 32 šiandien pereisiu visas avių bandas, išskirdamas iš jų kiekvieną dėmėtą bei lopiniuotą avį ir kiekvieną juodą avį, ir visas lopiniuotas bei dėmėtas ožkas. Tai bus mano atlyginimas. 
\par 33 Mano sąžiningumas kalbės už mane, kai ateis laikas man atsiimti užmokestį tavo akivaizdoje. Visa, kas nebus dėmėta bei lopiniuota tarp ožkų ir avių, tebūna kaip mano pavogta!” 
\par 34 Labanas atsakė: “Sutinku. Tebūna kaip sakai!” 
\par 35 Ir jis atskyrė tą dieną ožkas ir avis, ožius ir avinus­lopiniuotus ir dėmėtus; visus vienos spalvos gyvulius atidavė savo sūnums. 
\par 36 Labanas nustatė, kad tarp jo ir Jokūbo būtų trijų dienų atstumas. Jokūbas ganė likusią Labano kaimenę. 
\par 37 Jokūbas, paėmęs žalias drebulių, migdolų ir liepų lazdeles, išlupinėjo jose dryžius, kad tose vietose, kur buvo žievė, būtų balta. 
\par 38 Jis tas išpjaustytas lazdeles sudėjo į lovius, į kuriuos pilamas vanduo, prie kurių bandos ateidavo gerti, ir atėję gerti imdavo poruotis. 
\par 39 Ir avys poruodavosi, žiūrėdamos į lazdeles; ir jos vesdavo dryžuotus, dėmėtus ir lopiniuotus ėriukus. 
\par 40 Jokūbas perskyrė savo bandą. Jis sudėjo lazdeles taip, kad jo avys ir Labano avys matytų jas. Jis laikė savo bandas atskirai ir nesuleisdavo jų su Labano bandomis. 
\par 41 Kai poravosi stipresnieji gyvuliai, Jokūbas įdėdavo lazdeles į lovius taip, kad gyvuliai matytų jas ir poruotųsi. 
\par 42 Silpnesniems gyvuliams poruojantis, jis neįdėdavo tų lazdelių. Taigi silpnesnieji teko Labanui, o stipresnieji Jokūbui. 
\par 43 Taip šis žmogus nepaprastai pralobo. Jis turėjo daug tarnų ir tarnaičių, galvijų, avių, kupranugarių ir asilų.



\chapter{31}

\par 1 Jokūbas girdėjo Labano sūnus kalbant: “Jokūbas pasiglemžė visa, kas priklausė mūsų tėvui. Iš mūsų tėvo jis įsigijo visą šitą turtą”. 
\par 2 Be to, Jokūbas pastebėjo, kad Labanas jo atžvilgiu nebuvo toks pat kaip anksčiau. 
\par 3 Viešpats tarė Jokūbui: “Grįžk į tėvų šalį pas savo gimines. Aš būsiu su tavimi!” 
\par 4 Jokūbas pasišaukė Rachelę ir Lėją į lauką prie savo bandos 
\par 5 ir joms tarė: “Aš matau, kad jūsų tėvas mano atžvilgiu nebėra toks kaip anksčiau. Bet mano tėvo Dievas buvo su manimi. 
\par 6 Jūs pačios žinote, kaip visomis jėgomis tarnavau jūsų tėvui. 
\par 7 Jūsų tėvas apgaudinėjo mane ir dešimt kartų keitė mano atlyginimą. Tačiau Dievas neleido jam manęs skriausti. 
\par 8 Jei jis sakė: ‘Dėmėtieji tebūna tavo atlyginimas’, visos ožkos ir avys vedė dėmėtus. O jei jis sakė: ‘Dryžuotieji tebūna tavo atlyginimas’, visos avys ir ožkos vedė dryžuotus. 
\par 9 Taip Dievas atėmė jūsų tėvo gyvulius ir man atidavė. 
\par 10 Gyvulių poravimosi metu sapne mačiau, kad dryžuotieji, dėmėti ir kerši patinai eina prie patelių. 
\par 11 Tada Dievo angelas sapne man tarė: ‘Jokūbai!’ Aš atsiliepiau: ‘Aš čia!’ 
\par 12 Jis tarė: ‘Žiūrėk, visi dryžuoti, dėmėti ir kerši patinai eina prie patelių! Aš mačiau visa, ką Labanas tau darė. 
\par 13 Aš esu Betelio Dievas, kur tu patepei akmens paminklą ir davei įžadą. Išeik iš šitos šalies ir sugrįžk į savo gimtinę!’ ” 
\par 14 Tada Rachelė ir Lėja kalbėjo: “Ar mums dar yra dalis tėvo namuose? 
\par 15 Argi mes nelaikomos svetimomis? Juk jis pardavė mus ir gautus pinigus už mus naudojo sau. 
\par 16 Iš tikrųjų visi turtai, kuriuos Dievas atėmė iš mūsų tėvo, priklauso mums ir mūsų vaikams. Taigi dabar daryk visa, ką Dievas tau įsakė”. 
\par 17 Jokūbas užsodino ant kupranugarių savo vaikus ir žmonas, 
\par 18 išsivarė visus savo gyvulius ir pasiėmė visą savo mantą, kurią jis buvo įsigijęs Mesopotamijoje, kad eitų pas savo tėvą Izaoką į Kanaano šalį. 
\par 19 Labanas tuo metu kirpo avis. Tada Rachelė pavogė dievukus, kurie priklausė jos tėvui. 
\par 20 Jokūbas apgavo sirą Labaną, nes pabėgo, nieko nesakęs. 
\par 21 Jis pasiėmė viską, kas jam priklausė; persikėlęs per upę, pasuko į Gileado kalnyną. 
\par 22 Trečią dieną Labanui pranešė, kad Jokūbas pabėgo. 
\par 23 Tada Labanas, pasiėmęs savo brolius, vijosi jį septynias dienas ir pasivijo Gileado kalnyne. 
\par 24 Dievas sapne atėjo pas sirą Labaną ir tarė: “Saugokis, nekalbėk su Jokūbu šiurkščiai!” 
\par 25 Jokūbas jau buvo pasistatęs palapinę kalnyne, kai Labanas jį pasivijo. Labanas su savo broliais taip pat pasistatė palapinę Gileado kalnyne 
\par 26 ir tarė Jokūbui: “Kodėl taip pasielgei ir iškeliavai nieko man nesakęs, slaptai išsivarydamas mano dukteris kaip karo belaisves? 
\par 27 Kodėl slapčia pabėgai ir pasislėpei nuo manęs? Jei būtum man pasisakęs, būčiau išlydėjęs tave iškilmingai, su dainomis, būgnais ir arfomis. 
\par 28 Tu neleidai man pabučiuoti vaikaičių ir dukterų. Tu pasielgei neprotingai. 
\par 29 Aš galėčiau tau pakenkti, bet tavo tėvo Dievas sapne pasakė man: ‘Saugokis, nekalbėk su Jokūbu šiurkščiai’. 
\par 30 Žinau, tu išsiilgai savo tėvo namų ir todėl iškeliavai, bet kodėl pavogei mano dievukus?” 
\par 31 Jokūbas atsakė Labanui: “Pabėgau bijodamas, kad prievarta neatimtum iš manęs savo dukterų. 
\par 32 O dėl vagystės, tai tas, pas kurį rasi savo dievukus, temiršta! Mūsų akivaizdoje ieškok ir pasiimk, kas tavo”. Jokūbas nežinojo, kad Rachelė buvo pavogusi dievukus. 
\par 33 Labanas patikrino Jokūbo, Lėjos ir abiejų tarnaičių palapines, bet nieko nerado. Tada jis, išėjęs iš Lėjos palapinės, įėjo į Rachelės palapinę. 
\par 34 Bet Rachelė dievukus buvo paslėpusi kupranugario balne ir atsisėdusi ant jų. Labanas iškrėtė visą palapinę, bet nerado. 
\par 35 Ji tarė savo tėvui: “Nepyk, mano viešpatie, kad negaliu atsikelti, nes su manimi vyksta tai, kas darosi moteriškėms”. Jis ieškojo, bet dievukų nerado. 
\par 36 Jokūbas supyko ir barė Labaną: “Kuo nusikaltau, kuo nusidėjau, kad su tokiu užsidegimu mane vijaisi 
\par 37 ir iškrėtei visus mano daiktus? Ką radai iš savo turtų? Pavesk tą reikalą mano ir savo giminaičiams, tegul jie išsprendžia mudviejų bylą! 
\par 38 Dvidešimt metų aš pas tave tarnavau. Tavo avys ir ožkos nebuvo bergždžios, ir tavo bandos avinų aš nevalgiau. 
\par 39 Kas žvėrių sudraskyta, nenešiau tau. Aš pats turėjau atlyginti nuostolį. Iš manęs reikalavai atlyginti, kas pavogta dieną ar naktį. 
\par 40 Aš dieną kenčiau kaitrą, naktį­šaltį, ir miegas bėgo nuo mano akių. 
\par 41 Taip dvidešimt metų tarnavau tavo namuose: keturiolika metų už dukteris ir šešerius metus už bandą. Tu dešimt kartų keitei mano atlyginimą! 
\par 42 Jei mano tėvo Dievas, Abraomo Dievas, kurio bijojosi Izaokas, nebūtų buvęs su manimi, tikrai dabar būtum mane išleidęs tuščiomis rankomis. Mano priespaudą ir vargą matė Dievas ir praėjusią naktį sudraudė tave”. 
\par 43 Labanas atsakė Jokūbui: “Dukterys yra mano dukterys, vaikaičiai­mano vaikaičiai, banda­ mano banda, ir visa, ką matai, man priklauso. O ką galiu šiandien daryti savo dukterims ir jų vaikams? 
\par 44 Todėl ateik ir padarykime sandorą­aš ir tu. Ir tai tebūna liudijimas tarp manęs ir tavęs!” 
\par 45 Tada Jokūbas, suradęs akmenį, pastatė paminklą. 
\par 46 Po to Jokūbas liepė savo giminaičiams: “Pririnkite akmenų!” Tie pririnkę sukrovė juos, ir ant tos krūvos jie valgė. 
\par 47 Labanas tuos akmenis pavadino Jegar Sahaduta, o Jokūbas­ Galedu. 
\par 48 Labanas tarė: “Šita akmenų krūva yra liudytoja tarp manęs ir tavęs”. Todėl ji vadinama Galedu 
\par 49 ir Micpa, nes jis sakė: “Viešpats tegu stebi mane ir tave, kai būsime vienas nuo kito atsiskyrę! 
\par 50 Jeigu tu skriausi mano dukteris ar vesi dar daugiau žmonų, nors nėra žmonių tarp mūsų, bet Dievas yra mūsų liudytojas”. 
\par 51 Labanas toliau kalbėjo Jokūbui: “Štai akmenų krūva ir paminklas, kurį pastačiau tarp mūsų. 
\par 52 Ši krūva bus liudytojas ir paminklas bus liudytojas, kad aš neisiu pas tave pro šitą akmenų krūvą, nė tu eisi pas mane pro šitą akmenų krūvą ir šitą paminklą su piktu kėslu! 
\par 53 Abraomo ir Nahoro Dievas, jų tėvų Dievas, tebūna teisėjas tarp mudviejų!” Jokūbas tada prisiekė Tuo, kurio bijojo jo tėvas Izaokas. 
\par 54 Tada Jokūbas aukojo kalne ir pasikvietė savo giminaičius valgyti. Jie valgė ir pasiliko ant kalno visą naktį. 
\par 55 Labanas, atsikėlęs anksti rytą, pabučiavo savo vaikaičius bei dukteris ir juos palaimino. Jis atsiskyrė nuo jų ir sugrįžo į savo vietovę.



\chapter{32}


\par 1 Jokūbui keliaujant toliau, jį pasitiko Dievo angelai. 
\par 2 Jis, išvydęs juos, tarė: “Tai Dievo stovykla!” Ir pavadino tą vietą Mahanaimu. 
\par 3 Jokūbas siuntė pirma savęs pasiuntinius pas savo brolį Ezavą į Seyro žemę, Edomo kraštan. 
\par 4 Jis įsakė jiems: “Taip kalbėsite mano valdovui Ezavui: ‘Taip sako tavo tarnas Jokūbas: ‘Viešėjau pas Labaną ir ten užtrukau iki šios dienos. 
\par 5 Turiu jaučių, asilų, avių, tarnų bei tarnaičių ir siunčiu pranešti savo valdovui, kad surasčiau malonę jo akyse’ ”. 
\par 6 Pasiuntiniai, sugrįžę pas Jokūbą, pranešė: “Buvome nuėję pas tavo brolį Ezavą, jis ateina tavęs pasitikti su keturiais šimtais vyrų!” 
\par 7 Jokūbas labai išsigando ir susirūpino. Jis padalino žmones, avis, galvijus bei kupranugarius į du būrius 
\par 8 ir tarė: “Jei Ezavas užpuls vieną būrį ir jį sumuš, tai bent likęs išsigelbės”. 
\par 9 Jokūbas meldėsi: “Mano tėvo Abraomo ir mano tėvo Izaoko Dieve, Viešpatie, kuris man sakei: ‘Grįžk į savo šalį pas savo gimines, ir Aš tau gera darysiu’. 
\par 10 Aš nevertas net mažiausios Tavo malonės ir ištikimybės, kurią parodei savo tarnui. Aš tik su lazda perėjau Jordaną, o dabar turiu du būrius. 
\par 11 Išgelbėk mane iš mano brolio Ezavo rankos, nes aš jo bijau, kad atėjęs nenužudytų manęs ir motinų su vaikais. 
\par 12 Tu juk sakei: ‘Aš tikrai darysiu tau gera ir padauginsiu tavo palikuonis, kad jie bus kaip jūros smiltys ir jų neįmanoma bus suskaičiuoti dėl gausybės’ ”. 
\par 13 Jokūbas tą naktį nakvojo toje vietoje. Rytą jis parinko dovanų savo broliui Ezavui iš to, ką turėjo: 
\par 14 du šimtus ožkų ir dvidešimt ožių, du šimtus avių ir dvidešimt avinų, 
\par 15 trisdešimt kupranugarių su kumeliukais, keturiasdešimt karvių ir dešimt jaučių, dvidešimt asilių ir dešimt asilų. 
\par 16 Jis juos atidavė tarnams ir išsiuntė po būrį atskirai, sakydamas: “Eikite pirma manęs, palikdami tarpus tarp bandų!” 
\par 17 Jis įsakė pirmajam: “Kai tave sutiks mano brolis Ezavas, klausdamas: ‘Kam tu priklausai? Kur eini? Kam priklauso šita banda?’, 
\par 18 tai atsakyk: ‘Tavo tarnui Jokūbui. Tai dovana, siunčiama mano valdovui Ezavui; štai ir jis pats ateina paskui mus’ ”. 
\par 19 Jis taip įsakė antrajam, trečiajam ir visiems, kurie ginė bandas: “Kalbėkite šitais žodžiais Ezavui, kai jį sutiksite, 
\par 20 ir pridurkite: ‘Tavo tarnas Jokūbas taip pat ateina paskui mus’ ”. Jokūbas galvojo: “Aš jį permaldausiu dovanomis, kurias siunčiu pirma savęs, paskui sutiksiu jį patį. Gal jis mane draugiškai sutiks?” 
\par 21 Pasiųstos dovanos išėjo pirma jo, o jis pats tą naktį nakvojo stovykloje. 
\par 22 Tą pačią naktį atsikėlęs jis paėmė abi žmonas, abi tarnaites ir vienuolika sūnų ir perbrido Jaboko brastą. 
\par 23 Jis perkėlė per upelį juos ir visa, ką turėjo. 
\par 24 Jokūbas pasiliko vienas. Ten jis grūmėsi su vienu vyru iki aušros. 
\par 25 Matydamas, kad neįstengia jo įveikti, tas vyras smogė Jokūbui į šlaunį ir išnarino Jokūbo šlaunies sąnarį. 
\par 26 Tada tas vyras tarė: “Paleisk mane, nes jau aušta!” Jokūbas atsakė: “Nepaleisiu tavęs, jei manęs nepalaiminsi!” 
\par 27 Tas klausė: “Kuo tu vardu?” Jis atsakė: “Jokūbas”. 
\par 28 Tada jis tarė: “Tu nebebūsi vadinamas Jokūbu, bet Izraeliu, nes tu kovojai su Dievu ir su žmonėmis ir nugalėjai”. 
\par 29 Jokūbas klausė: “Pasakyk man savo vardą”. Bet tas atsakė: “Kam gi klausi mano vardo?” Ir jis ten jį palaimino. 
\par 30 Jokūbas pavadino tą vietą Penieliu: “Aš regėjau Dievą veidas į veidą ir išlikau gyvas”. 
\par 31 Kai jis perėjo Penielį, patekėjo saulė, ir jis šlubavo viena koja. 
\par 32 Todėl iki šios dienos Izraelio vaikai nevalgo šlaunies raumenų, nes Jokūbo šlaunis buvo sužeista.



\chapter{33}

\par 1 Jokūbas, pakėlęs akis, pamatė ateinantį Ezavą su keturiais šimtais vyrų. Jis paskirstė vaikus tarp Lėjos, Rachelės ir abiejų tarnaičių: 
\par 2 sustatė tarnaites ir jų vaikus pirmoje eilėje, Lėją ir jos vaikus už jų, o Rachelę ir Juozapą paskutinėje eilėje. 
\par 3 Jis pats nuėjo pirma jų ir septynis kartus nusilenkė iki žemės, kol susitiko su broliu. 
\par 4 Ezavas bėgo prie jo, apkabino jį ir bučiavo, puolęs jam ant kaklo, ir jie abu verkė. 
\par 5 Pamatęs žmonas ir vaikus, jis klausė: “Kas šitie?” Tas atsakė: “Vaikai, kuriais Dievas apdovanojo tavo tarną”. 
\par 6 Tada priartėjusios tarnaitės su savo vaikais nusilenkė. 
\par 7 Paskui priartėjo Lėja ir jos vaikai ir nusilenkė. Pagaliau priartėjo Juozapas ir Rachelė ir taip pat nusilenkė. 
\par 8 Ezavas toliau klausė: “Kam tie būriai, kuriuos sutikau?” Jis atsakė: “Kad rasčiau malonę savo valdovo akyse!” 
\par 9 Ezavas atsakė: “Mano broli, aš turiu užtektinai, pasilaikyk, ką turi!” 
\par 10 Jokūbas tarė: “O ne! Jei radau malonę tavo akyse, tai priimk iš manęs šią dovaną. Aš matau tavo veidą, tarsi matyčiau Dievo veidą, ir tu esi man malonus. 
\par 11 Prašau, priimk palaiminimą, kurį tau atnešiau. Nes Dievas buvo man maloningas, ir aš visko turiu”. Jokūbui prašant, brolis priėmė dovaną. 
\par 12 Tada Ezavas tarė: “Dabar keliaukime­aš eisiu tavo priešakyje!” 
\par 13 Bet Jokūbas atsakė: “Mano valdovas žino, kad vaikai gležni ir kad dalis avių bei galvijų yra jaunikliai; jei juos per greitai varysime nors vieną dieną, jie išgaiš. 
\par 14 Mano valdove, eik pirma savo tarno, o aš pamažu toliau judėsiu, kaip įstengia eiti gyvuliai ir vaikai, kol nueisiu pas savo valdovą į Seyrą”. 
\par 15 Tada Ezavas sakė: “Leisk man palikti su tavimi dalį savo žmonių!” Bet tas atsakė: “Kam to reikia? Kad tik surasčiau malonę savo valdovo akyse!” 
\par 16 Ezavas tą dieną sugrįžo į Seyrą, 
\par 17 o Jokūbas judėjo toliau į Sukotą. Ten jis pasistatė namus, o gyvuliams­ pastoges. Todėl pavadino tą vietą Sukotu. 
\par 18 Jokūbas, keliaudamas iš Mesopotamijos, laimingai atvyko į Sichemą, kuris yra Kanaano krašte, ir apsistojo šalia miesto. 
\par 19 Sklypą, kuriame pasistatė palapinę, jis nusipirko iš Sichemo tėvo Hamoro sūnų už šimtą monetų. 
\par 20 Jis ten pastatė aukurą ir jį pavadino: “Izraelio Dievo galybė”.



\chapter{34}


\par 1 Kartą Lėjos duktė Dina išėjo pasižiūrėti tos šalies dukterų. 
\par 2 Šalies kunigaikščio hivo Hamoro sūnus Sichemas, ją pamatęs, nutvėrė ir išprievartavo. 
\par 3 Jo siela prisirišo prie Jokūbo dukters Dinos. Jis pamilo mergaitę ir meiliai kalbėjo su ja. 
\par 4 Sichemas prašė savo tėv o Hamoro: “Leisk man vesti šią mergaitę!” 
\par 5 Jokūbas sužinojo, kad Sichemas išprievartavo jo dukterį Diną; kadangi jo sūnūs buvo prie gyvulių laukuose, Jokūbas tylėjo, kol jie pareis. 
\par 6 Sichemo tėvas Hamoras atėjo pas Jokūbą, norėdamas su juo pasikalbėti. 
\par 7 Jokūbo sūnūs, tai išgirdę, parėjo iš laukų. Jie įsižeidė ir labai supyko, nes Sichemas padarė gėdą Izraeliui, gulėdamas su Jokūbo dukterimi, nors nederėjo taip daryti. 
\par 8 Hamoras kalbėjosi su jais: “Mano sūnaus Sichemo siela ilgisi jūsų dukters. Prašau, leiskite mano sūnui ją vesti. 
\par 9 Susigiminiuokime: duokite mums savo dukteris, o mūsų dukteris veskite! 
\par 10 Gyvenkite pas mus. Kraštas jums yra atviras. Pasilikite ir laisvai jame gyvenkite ir įsigykite čia nuosavybę”. 
\par 11 Ir Sichemas kalbėjo Dinos tėvui ir broliams: “O kad rasčiau malonę jūsų akyse! Ko tik iš manęs paprašysite, duosiu. 
\par 12 Prašykite pačio didžiausio kraičio ir dovanos; aš viską duosiu, ko paprašysite, tik leiskite man vesti mergaitę!” 
\par 13 Jokūbo sūnūs klastingai kalbėjo su Sichemu ir jo tėvu Hamoru, nes Sichemas buvo išprievartavęs jų seserį Diną. 
\par 14 Jie sakė jiems: “Mes negalime to padaryti­išleisti savo seserį už vyro, kuris yra neapipjaustytas, nes tai būtų mums negarbė ir gėda. 
\par 15 Sutiksime su jumis tik su sąlyga, jei jūs tapsite kaip mes ir kiekvienas vyras tarp jūsų bus apipjaustytas. 
\par 16 Tada mes leisime jums vesti savo dukteris ir vesime jūsų; liksime pas jus gyventi ir tapsime viena tauta. 
\par 17 Bet jei mūsų nepaklausysite ir neapsipjaustysite, pasiimsime savo dukterį ir išeisime”. 
\par 18 Jų žodžiai patiko Hamorui ir jo sūnui Sichemui. 
\par 19 Jaunuolis nedelsė įvykdyti pasiūlymo, nes jis buvo įsimylėjęs Jokūbo dukterį. O jis buvo žymiausias savo tėvo namuose. 
\par 20 Hamoras ir jo sūnus Sichemas atėjo prie miesto vartų ir kalbėjo savo miesto vyrams: 
\par 21 “Šitie žmonės yra taikingi mūsų atžvilgiu. Jie telieka gyventi ir laisvai verstis mūsų krašte. Kraštas juk platus! Jų dukteris vesime, o savo dukteris leisime tekėti už jų. 
\par 22 Tie žmonės sutinka gyventi pas mus ir tapti viena tauta tik su šita sąlyga, jei kiekvienas mūsų vyras apsipjaustys, kaip jie yra apipjaustyti. 
\par 23 Jų galvijai, jų manta ir visi gyvuliai priklausys mums. Sutikime su jais, ir jie liks pas mus gyventi!” 
\par 24 Visi miesto vyrai paklausė Hamoro ir jo sūnaus Sichemo ir buvo apipjaustyti. 
\par 25 Trečią dieną, kai jiems labai skaudėjo, du Jokūbo sūnūs, Simeonas ir Levis, Dinos broliai, pasiėmė savo kardus ir, drąsiai atėję į miestą, išžudė visus vyrus. 
\par 26 Jie taip pat nužudė Hamorą ir jo sūnų Sichemą, paėmė Diną iš Sichemo namų ir išėjo. 
\par 27 Jokūbo sūnūs atėjo prie nužudytųjų ir apiplėšė miestą, keršydami už sesers išniekinimą. 
\par 28 Pasiėmė jų avis, galvijus, asilus ir visa, kas buvo mieste ir laukuose. 
\par 29 Pagrobė visą jų turtą, vaikus ir žmonas išsivedė į nelaisvę ir išplėšė viską, kas buvo namuose. 
\par 30 Jokūbas tarė Simeonui ir Leviui: “Jūs pridarėte man bėdos, padarydami mane nekenčiamą tarp šios šalies gyventojų, tarp kanaaniečių ir perizų. Mūsų labai mažai; jie susirinks prieš mane ir nužudys mane. Taip aš ir mano namai bus sunaikinti”. 
\par 31 Sūnūs atsakė: “Argi jam buvo leista pasielgti su mūsų seserimi kaip su paleistuve?”



\chapter{35}

\par 1 Dievas tarė Jokūbui: “Kelkis ir eik į Betelį, ir apsistok ten; pastatyk aukurą Dievui, kuris tau pasirodė, kai bėgai nuo savo brolio Ezavo”. 
\par 2 Jokūbas įsakė saviesiems ir visiems, buvusiems su juo: “Pašalinkite svetimus dievus, kurie yra tarp jūsų, apsivalykite ir pakeiskite drabužius! 
\par 3 Eikime į Betelį, ten pastatysiu aukurą Dievui, kuris mane išklausė pavojuje ir buvo su manimi kelyje, kuriuo aš ėjau!” 
\par 4 Jie atidavė Jokūbui visus svetimus dievus, kuriuos jie turėjo, ir auskarus, o Jokūbas juos užkasė po ąžuolu prie Sichemo. 
\par 5 Jie iškeliavo, ir siaubas nuo Dievo apėmė aplinkinius miestus, kad niekas nedrįso vytis Jokūbo sūnų. 
\par 6 Taip Jokūbas ir visi su juo esantys žmonės atėjo į Lūzą, dar vadinamą Beteliu, kuri yra Kanaano šalyje. 
\par 7 Ten jis pastatė aukurą ir tą vietą pavadino El Betelis, nes ten jam pasirodė Dievas, kai jis bėgo nuo savo brolio veido. 
\par 8 Ten mirė Debora, Rebekos auklė, ir buvo palaidota prie Betelio po ąžuolu, kurį pavadino Raudos ąžuolu. 
\par 9 Dievas vėl pasirodė Jokūbui, kai jis atvyko iš Mesopotamijos, ir jį palaimino. 
\par 10 Ir Dievas tarė jam: “Tavo vardas Jokūbas, bet tu nebesivadinsi Jokūbu. Tavo vardas bus Izraelis! 
\par 11 Aš esu Dievas Visagalis. Būk vaisingas ir dauginkis! Tauta ir daugelis tautų atsiras iš tavęs, ir karaliai išeis iš tavo strėnų! 
\par 12 Tą žemę, kurią daviau Abraomui ir Izaokui, atiduosiu tau ir po tavęs duosiu tavo palikuonims”. 
\par 13 Tada Dievas pasitraukė nuo jo iš tos vietos, kur su juo kalbėjo. 
\par 14 Jokūbas pastatė akmeninį paminklą toje vietoje, kur Dievas kalbėjo su juo, išliejo ant jo geriamąją auką ir aliejaus. 
\par 15 Jokūbas pavadino tą vietą, kur Dievas su juo kalbėjo, Beteliu. 
\par 16 Iš Betelio jie keliavo toliau. Nepasiekus Efratos, Rachelė gimdė, ir jos gimdymas buvo sunkus. 
\par 17 Jai esant gimdymo kančiose, pribuvėja jai tarė: “Nebijok! Ir šį kartą turėsi sūnų”. 
\par 18 Kai jos siela buvo beatsiskirianti, nes ji buvo prie mirties, ji pavadino jį Ben Oniu, bet tėvas jį pavadino Benjaminu. 
\par 19 Ir Rachelė mirė ir buvo palaidota prie kelio, einančio į Efratą, tai yra Betliejų. 
\par 20 Jokūbas pastatė ant jos kapo paminklą; tas Rachelės kapo paminklas tebestovi iki šios dienos. 
\par 21 Izraelis keliavo toliau ir apsistojęs pasistatė palapines anapus Edero bokšto. 
\par 22 Izraeliui gyvenant anoje šalyje, Rubenas miegojo su savo tėvo sugulove Bilha. Izraelis tai sužinojo. Jokūbo sūnų buvo dvylika. 
\par 23 Lėjos sūnūs: Jokūbo pirmagimis Rubenas, Simeonas, Levis, Judas, Isacharas ir Zabulonas. 
\par 24 Rachelės sūnūs: Juozapas ir Benjaminas. 
\par 25 Rachelės tarnaitės Bilhos sūnūs: Danas ir Neftalis. 
\par 26 Lėjos tarnaitės Zilpos sūnūs: Gadas ir Ašeras. Šitie yra Jokūbo sūnūs, gimę jam Mesopotamijoje. 
\par 27 Jokūbas atėjo pas savo tėvą Izaoką į Mamrę, į Kirjat Arbos miestą, tai yra Hebroną, kur Abraomas ir Izaokas buvo ateiviai. 
\par 28 Izaokas sulaukė šimto aštuoniasdešimties metų. 
\par 29 Ir Izaokas atidavė savo dvasią, ir mirė, ir susijungė su savo tauta, būdamas senas ir pasisotinęs gyvenimu. Jį palaidojo jo sūnūs Ezavas ir Jokūbas.



\chapter{36}

\par 1 Šitie yra Ezavo, kuris yra Edomas, palikuonys. 
\par 2 Ezavas vedė žmonas kanaanietes: hetito Elono dukterį Adą, hivo Cibeono sūnaus Anos dukterį Oholibamą 
\par 3 ir Basmatą, Izmaelio dukterį, Nebajoto seserį. 
\par 4 Ados sūnus­Elifazas. Basmatos sūnus­Reuelis. 
\par 5 Oholibamos sūnūs: Jeušas, Jalamas ir Korachas. Šitie yra Ezavo sūnūs, gimę jam Kanaano žemėje. 
\par 6 Ezavas pasiėmė savo žmonas, sūnus, dukteris, visus savo žmones, gyvulių bandas ir visą nuosavybę, kurią buvo įsigijęs Kanaano krašte, ir išvyko į kitą šalį, pasitraukdamas nuo savo brolio Jokūbo. 
\par 7 Jų turtai buvo per dideli, kad jie galėtų gyventi kartu, o žemė, kurioje jie buvo ateiviai, nebegalėjo išmaitinti jų gyvulių. 
\par 8 Ezavas apsigyveno Seyro kalnyne. Ezavas yra Edomas. 
\par 9 Šitie yra Ezavo, edomitų tėvo palikuonys Seyro kalnyne. 
\par 10 Ezavo sūnų vardai: Ezavo žmonos Ados sūnus­Elifazas, Ezavo žmonos Basmatos sūnus­Reuelis. 
\par 11 Elifazo sūnūs: Temanas, Omaras, Cefojas, Gatamas ir Kenazas. 
\par 12 Timna buvo Ezavo sūnaus Elifazo sugulovė ir pagimdė Elifazui Amaleką. Šitie yra Ezavo žmonos Ados sūnūs. 
\par 13 Reuelio sūnūs: Nahatas ir Zerachas, Šama ir Miza. Jie yra Ezavo žmonos Basmatos sūnūs. 
\par 14 Ezavo žmonos Oholibamos, Cibeono sūnaus Anos dukters, sūnūs: Jeušas, Jalamas ir Korachas. 
\par 15 Šie yra Ezavo sūnų kunigaikščiai—Ezavo pirmagimio Elifazo sūnūs: Temanas, Omaras, Cefojas, Kenazas, 
\par 16 Korachas, Gatamas, Amalekas­iš Elifazo kilę kunigaikščiai Edomo krašte. Jie yra Ados sūnūs. 
\par 17 Ezavo sūnaus Reuelio sūnūs: Nahatas, Zerachas, Šama, Miza­iš Reuelio kilę kunigaikščiai Edomo krašte. Jie Ezavo žmonos Basmatos sūnūs. 
\par 18 Ezavo žmonos Oholibamos sūnūs: Jeušas, Jalamas ir Korachas. Tai iš Anos dukters Oholibamos, Ezavo žmonos, kilę kunigaikščiai. 
\par 19 Jie yra Ezavo, tai yra Edomo sūnūs­tos šalies kunigaikščiai. 
\par 20 Horo Seyro sūnūs, gyvenę krašte: Lotanas, Šobalas, Cibeonas, Ana, 
\par 21 Dišonas, Eceras, Dišanas. Jie horų, Seyro sūnų, kunigaikščiai Edomo krašte. 
\par 22 Lotano vaikai: Horis, Hemamas ir Lotano sesuo Timna. 
\par 23 Sobalio sūnūs: Alvanas, Manahatas, Ebalas, Šefojas ir Onamas. 
\par 24 Cibeono sūnūs: Aja ir Ana. Ana, beganydamas savo tėvo Cibeono asilus, dykumoje rado šiltąsias versmes. 
\par 25 Anos vaikai: Dišonas ir duktė Oholibama. 
\par 26 Dišono sūnūs: Hemdanas, Ešbanas, Itranas ir Keranas. 
\par 27 Ecero sūnūs: Bilhanas, Zaavanas ir Akanas. 
\par 28 Dišano sūnūs: Ucas ir Aranas. 
\par 29 Horai: Lotanas, Šobalas, Zibeonas, Anas, 
\par 30 Dišonas, Eceras ir Dišanas­horų kunigaikščiai Seyro krašte. 
\par 31 Šitie karaliai valdė Edomo kraštą, kai izraelitai dar neturėjo savo kilmės karaliaus. 
\par 32 Edome karaliumi buvo Beoro sūnus Bela: jo miestas vadinosi Dinhaba. 
\par 33 Belai mirus, jo vietą užėmė Jobabas, Zeracho iš Bocros sūnus. 
\par 34 Jobabui mirus, jo vietoje karaliavo Hušamas iš Temano šalies. 
\par 35 Hušamui mirus, sostą paėmė Bedado sūnus Hadadas, kuris sumušė Midjaną Moabo laukuose ir kurio miestas vadinosi Avitas. 
\par 36 Hadadui mirus, jo vietą užėmė Samla iš Masrekos. 
\par 37 Samlai mirus, karaliavo Saulius iš Rehoboto. 
\par 38 Sauliui mirus, sostas atiteko Achboro sūnui Baal Hananui. 
\par 39 Achboro sūnui Baal Hananui mirus, jo vietą užėmė Hadaras. Jo miestas vadinosi Pavas. Jo žmona buvo vardu Mehetabelė; ji buvo Me Zahabo dukters Matredos duktė. 
\par 40 Ezavo kunigaikščių vardai pagal jų kilmę ir vietos pavadinimą: Timna, Alva, Jetetas, 
\par 41 Oholibama, Ela, Pinonas, 
\par 42 Kenazas, Temanas, Mibcaras, 
\par 43 Magdielis, Iramas. Šitie Edomo kunigaikščiai gyveno jų nuosavame krašte. Ezavas­edomitų protėvis.



\chapter{37}

\par 1 Jokūbas apsigyveno Kanaano šalyje, kur jo tėvas buvo ateivis. 
\par 2 Tokia yra Jokūbo istorija. Juozapas, būdamas septyniolikos metų, ganė su savo broliais avis; vaikinas gyveno su savo tėvo žmonų Bilhos ir Zilpos sūnumis. Juozapas pranešdavo tėvui, kai jie ką pikta kalbėdavo. 
\par 3 Izraelis mylėjo Juozapą labiau už kitus savo sūnus, nes jis gimė jam sulaukus žilos senatvės. Tėvas jam padarė margą apdarą. 
\par 4 Jo broliai, pastebėję, kad jų tėvas Juozapą myli labiau už visus brolius, neapkentė jo ir nesugyveno su juo. 
\par 5 Kartą Juozapas sapnavo sapną ir jį papasakojo savo broliams. Tada jie ėmė dar labiau jo nekęsti. 
\par 6 Jis jiems tarė: “Pasiklausykite mano sapno: 
\par 7 štai mes rišome pėdus laukuose; mano pėdas atsistojo ir stovėjo tiesus, o jūsų pėdai sustojo aplinkui ir nusilenkė prieš mano pėdą”. 
\par 8 Broliai jam atsakė: “Bene būsi mūsų karalius? O gal mus valdysi?” Ir jie dar labiau jo neapkentė dėl jo sapnų ir jo kalbų. 
\par 9 Jis sapnavo dar kitą sapną ir papasakojo savo broliams: “Sapnavau dar vieną sapną, kad saulė, mėnulis ir vienuolika žvaigždžių lenkėsi prieš mane”. 
\par 10 Kai jis papasakojo tą sapną savo tėvui ir savo broliams, tėvas jį subarė: “Koks čia sapnas! Nejaugi aš, tavo motina ir broliai ateisime ir nusilenksime iki žemės prieš tave?” 
\par 11 Jo broliai pavydėjo jam, bet tėvas įsidėmėjo tuos žodžius. 
\par 12 Kartą jo broliai ganė tėvo avis prie Sichemo. 
\par 13 Tėvas tarė Juozapui: “Ar tavo broliai negano prie Sichemo? Eikš, aš tave pasiųsiu pas juos!” Jis atsiliepė: “Aš čia!” 
\par 14 “Eik, pažiūrėk, kaip tavo broliams ten sekasi, ir parėjęs pranešk man”. Taigi tėvas išsiuntė Juozapą iš Hebrono slėnio į Sichemą. 
\par 15 Vienas vyras sutiko Juozapą beklaidžiojantį lauke ir paklausė: “Ko ieškai?” 
\par 16 Jis atsakė: “Ieškau savo brolių. Pasakyk man, kur jie gano?” 
\par 17 Tas vyras atsakė: “Jie išėjo iš čia. Nugirdau juos kalbant: ‘Eikime į Dotaną’ ”. Juozapas ėjo paskui savo brolius ir rado juos Dotane. 
\par 18 Jie, iš tolo pamatę jį ateinant, slapta susimokė jį nužudyti. 
\par 19 Jie sakė vienas kitam: “Štai ateina sapnuotojas! 
\par 20 Dabar užmuškime jį, įmeskime į duobę ir sakykime: ‘Plėšrus žvėris jį suėdė’; tada pamatysime, kas bus iš jo sapnų!” 
\par 21 Bet Rubenas, tai išgirdęs, išgelbėjo jį iš jų rankų, sakydamas: “Nežudykime jo. 
\par 22 Nepraliekite kraujo! Įmeskite jį į šitą dykumoje esančią duobę, bet nesutepkite savo rankų!” Jis taip kalbėjo, norėdamas jį išgelbėti iš jų rankų ir sugrąžinti tėvui. 
\par 23 Juozapui atėjus pas brolius, jie nutraukė nuo jo margąjį apdarą, kuriuo jis vilkėjo, 
\par 24 ir, sugriebę jį, įmetė į duobę, kurioje nebuvo vandens. 
\par 25 Tada jie susėdo valgyti. Staiga jie pamatė izmaelitų karavaną, ateinantį iš Gileado, ir jų kupranugarius, nešančius kvepalų, balzamo ir miros. Jie traukė į Egiptą. 
\par 26 Judas tarė savo broliams: “Ką laimėsime, užmušę savo brolį ir nuslėpę jo kraują? 
\par 27 Parduokime jį izmaelitams ir nesutepkime savo rankų. Juk jis mūsų brolis, mūsų kūnas!” Broliai paklausė jo. 
\par 28 Einant pro šalį Midjano pirkliams, jie, ištraukę Juozapą iš duobės, pardavė izmaelitams už dvidešimt sidabrinių; tie nusivedė Juozapą į Egiptą. 
\par 29 Rubenas, sugrįžęs prie duobės ir pamatęs, kad Juozapo nebėra, perplėšė savo drabužius. 
\par 30 Sugrįžęs pas brolius, tarė: “Vaiko nebėra! Kur aš eisiu!” 
\par 31 Jie paėmė Juozapo apdarą ir, papjovę ožį, tą apdarą pamirkė jo kraujyje 
\par 32 ir pasiuntė jį tėvui, sakydami: “Štai ką radome. Pažiūrėk, ar tai ne tavo sūnaus apdaras?” 
\par 33 Pažinęs jį, tėvas tarė: “Tai mano sūnaus apdaras! Plėšrus žvėris jį suėdė! Juozapas tikrai sudraskytas!” 
\par 34 Jokūbas persiplėšė drabužius, užsivilko ašutinę ir daugelį dienų gedėjo savo sūnaus. 
\par 35 Visi jo sūnūs ir dukterys guodė jį, tačiau jis nesidavė guodžiamas ir tarė: “Aš gedėdamas nueisiu į mirusiųjų buveinę pas savo sūnų”. Taip jį apraudojo jo tėvas. 
\par 36 Tuo metu midjaniečiai Egipte pardavė jį Potifarui, faraono rūmų valdininkui, sargybos viršininkui.



\chapter{38}

\par 1 Judas pasitraukė nuo savo brolių ir apsigyveno pas vieną adulamietį, vardu Hyras. 
\par 2 Ten Judas pamatė kanaaniečio Šūvos dukterį ir paėmė ją, ir įėjo pas ją. 
\par 3 Ji pastojo ir pagimdė sūnų, kurį pavadino Eru. 
\par 4 Ji pastojo dar kartą ir pagimdė sūnų, kurį pavadino Onanu. 
\par 5 Ir dar kartą ji pastojo, ir pagimdė sūnų, kurį pavadino Šela. Judas tada gyveno Kezibe. 
\par 6 Judas paėmė savo pirmagimiui Erui žmoną, vardu Tamara. 
\par 7 Eras, Judo pirmagimis, buvo nedoras Viešpaties akyse, ir Viešpats jį numarino. 
\par 8 Tada Judas tarė sūnui Onanui: “Įeik pas savo brolio žmoną ir vesk ją, ir atgaivink savo brolio palikuonis”. 
\par 9 Onanas žinojo, kad palikuonys nepriklausys jam. Todėl įeidamas pas savo brolio žmoną išliejo sėklą žemėn, kad neduotų broliui palikuonių. 
\par 10 Viešpačiui nepatiko jo elgesys, todėl ir jį numarino. 
\par 11 Judas tarė savo marčiai Tamarai: “Gyvenk kaip našlė savo tėvo namuose, kol užaugs mano sūnus Šela”. Nes jis bijojo, kad ir tas nemirtų kaip jo broliai. Tamara nuėjo gyventi į savo tėvo namus. 
\par 12 Po kurio laiko mirė Judo žmona, Šūvos duktė. Gedulo laikui praėjus, Judas su savo draugu adulamiečiu Hyru ėjo į Timną pas savo avių kirpėjus. 
\par 13 Tamarai buvo pranešta: “Štai tavo uošvis eina į Timną avių kirpti”. 
\par 14 Ji nusivilko savo našlės drabužius, apsidengė šydu ir atsisėdo prie Enaimo vartų, kurie yra pakelėje į Timną. Ji žinojo, kad Šela buvo užaugęs, tačiau ji nebuvo jam duota į žmonas. 
\par 15 Judas ją palaikė paleistuve, nes ji buvo uždengusi savo veidą. 
\par 16 Jis pasuko prie jos ir tarė: “Leisk man įeiti pas tave”. Jis nežinojo, kad tai buvo jo marti, o ji klausė: “Ką man duosi?” 
\par 17 Jis atsakė: “Aš tau atsiųsiu ožiuką iš savo bandos”. Ji paklausė: “Ar duosi man užstatą, kol jį atsiųsi?” 
\par 18 Jis paklausė: “Ko nori užstatu?” Ji atsakė: “Tavo antspaudo, virvės ir lazdos, kuri yra tavo rankoje”. Jis davė jai tai, įėjo pas ją, ir ji pastojo. 
\par 19 Jam išėjus, ji nusiėmė šydą, persirengė našlės drabužiais ir grįžo namo. 
\par 20 Kai Judo draugas adulamietis nuvedė ožiuką, kad atsiimtų užstatą iš moteriškės, jos nerado. 
\par 21 Jis klausinėjo vietos žmonių: “Kur yra paleistuvė, kuri buvo Enaime prie kelio?” Jie atsakė: “Čia nebuvo jokios paleistuvės”. 
\par 22 Jis tada sugrįžo pas Judą ir pasakė: “Aš jos neradau, ir tos vietos žmonės sakėsi nieko nežiną apie jokią paleistuvę”. 
\par 23 Judas tarė: “Tegul pasilaiko mano užstatą, kad neapsijuoktume. Aš jai siunčiau ožiuką, bet tu negalėjai jos surasti”. 
\par 24 Po trijų mėnesių Judui buvo pranešta: “Tavo marti Tamara paleistuvavo ir pastojo”. Judas liepė: “Atveskite ją, kad ji būtų sudeginta”. 
\par 25 Kai ją išvedė, ji pasiuntė žinią savo uošviui: “Aš pastojau nuo vyro, kuriam priklauso šitie daiktai. Ištirk, kam priklauso šitas antspaudas, virvė ir lazda”. 
\par 26 Judas atpažinęs tarė: “Ji yra teisesnė už mane, nes aš jos nedaviau savo sūnui Šelai”. Ir daugiau jis jos nepažino. 
\par 27 Gimdymo metu paaiškėjo, kad dvyniai yra jos įsčiose. 
\par 28 Jai begimdant, vienas iškišo ranką. Pribuvėja paėmė ir užrišo ant jo rankos raudoną siūlą, sakydama: “Šitas pirmas išėjo”. 
\par 29 Bet jis įtraukė ranką atgal, ir štai išėjo jo brolis. Ji tarė: “Štai kaip tu prasiveržei”. Jį tad pavadino Perecu. 
\par 30 Paskui gimė jo brolis, ant kurio rankos buvo raudonas siūlas; jį pavadino Zerachu.



\chapter{39}

\par 1 Juozapą nuvedė į Egiptą, ir egiptietis Potifaras, faraono rūmų valdininkas, sargybos viršininkas, jį nupirko iš izmaelitų. 
\par 2 Ir Viešpats buvo su Juozapu, ir jam viskas sekėsi. Jis gyveno savo valdovo, egiptiečio, namuose. 
\par 3 Jo valdovas pastebėjo, kad Viešpats buvo su juo ir kad visa, ką jis darė, Viešpats laimino. 
\par 4 Juozapas rado Potifaro akyse malonę; jis tarnavo jam, ir tas paskyrė jį savo namų prievaizdu, ir visa pavedė jam tvarkyti. 
\par 5 Nuo to laiko, kai jis paskyrė Juozapą prievaizdu savo namuose, Viešpats laimino egiptiečio namus dėl Juozapo; Viešpaties palaima buvo ant visko, ką jis turėjo namuose ir laukuose. 
\par 6 Jis visa, ką turėjo, pavedė Juozapui; pats niekuo nesirūpino, tik maistu, kurį valgė. Juozapas buvo dailus ir gražaus veido. 
\par 7 Po kurio laiko jo valdovo žmona atkreipė dėmesį į Juozapą ir tarė: “Sugulk su manimi”. 
\par 8 Bet jis jai atsakė: “Mano valdovas niekuo nesirūpina ir visa, ką jis turi, atidavė į mano rankas. 
\par 9 Šiuose namuose nėra didesnio už mane, ir jis nieko man nedraudžia išskyrus tave, nes tu esi jo žmona. Kaip tad galėčiau padaryti tokią piktadarystę ir nusidėti prieš Dievą?” 
\par 10 Ji kiekvieną dieną kalbino Juozapą, tačiau jis nesutiko sugulti ir būti su ja. 
\par 11 Vieną dieną Juozapas atėjo į namus savo reikalais ir nieko daugiau tuo metu namuose nebuvo. 
\par 12 Ji nutvėrė jį už jo drabužio ir sakė: “Sugulk su manimi”. Bet jis, išsinėręs iš drabužio, ištrūko ir išbėgo laukan. 
\par 13 Pamačiusi, kad jis paliko savo drabužį jos rankoje ir išbėgo laukan, 
\par 14 ji pasišaukė namiškius ir jiems tarė: “Žiūrėkite! Jis atvedė mums vyrą, hebrają, kad tas tyčiotųsi iš mūsų. Jis atėjo pas mane, norėdamas sugulti su manimi, bet aš ėmiau garsiai šaukti. 
\par 15 Kai jis išgirdo mane šaukiant, paliko drabužį pas mane ir išbėgo”. 
\par 16 Ji pasilaikė jo drabužį, kol grįžo valdovas. 
\par 17 Tada ji tais pačiais žodžiais kalbėjo jam: “Pas mane atėjo tas vergas, hebrajas, kurį mums atvedei, kad pasityčiotų iš manęs. 
\par 18 Bet kai aš pradėjau garsiai šaukti, jis paliko savo drabužį pas mane ir išbėgo”. 
\par 19 Valdovas, išgirdęs žmonos žodžius, kuriais ji kalbėjo: “Taip tavo vergas pasielgė su manimi”, užsidegė pykčiu. 
\par 20 Ir Juozapo valdovas paėmė jį, ir atidavė į kalėjimą, kur kalėjo karaliaus kaliniai. 
\par 21 Bet Viešpats buvo su Juozapu ir parodė jam savo gailestingumą, ir davė jam rasti malonę kalėjimo viršininko akyse. 
\par 22 Šis pavedė Juozapui rūpintis visais kaliniais ir visi darbai buvo jo priežiūroje. 
\par 23 Kalėjimo viršininkas niekuo nesidomėjo, kas buvo Juozapui pavesta, nes Viešpats buvo su juo ir visuose darbuose jam duodavo sėkmę.



\chapter{40}

\par 1 Po kurio laiko nusikalto Egipto karaliui jo vyno pilstytojas ir duonkepys. 
\par 2 Faraonas supyko ant abiejų savo valdininkų: ant vyno pilstytojų viršininko ir duonkepių viršininko. 
\par 3 Jis įsakė juos uždaryti sargybos viršininko kalėjime, kur kalėjo Juozapas. 
\par 4 Sargybos viršininkas pavedė Juozapui juos prižiūrėti ir jiems patarnauti. Jie ten sėdėjo ilgesnį laiką. 
\par 5 Egipto karaliaus vyno pilstytojas ir duonkepys tą pačią naktį sapnavo sapną, ir kiekvieno sapnas turėjo savo reikšmę. 
\par 6 Juozapas, įėjęs pas juos rytą, pastebėjo juos esant prislėgtus. 
\par 7 Jis paklausė jų: “Kodėl šiandien jūsų veidai tokie paniurę?” 
\par 8 Jie atsakė: “Sapnavome sapną, bet nėra, kas jį išaiškintų”. Juozapas jiems tarė: “Argi ne iš Dievo ateina išaiškinimas? Papasakokite juos man”. 
\par 9 Tuomet vyno pilstytojų viršininkas papasakojo savo sapną Juozapui: “Aš sapnavau, kad pasirodė vynmedis prieš mane. 
\par 10 Jis turėjo tris šakeles, išleido pumpurus, išskleidė žiedus ir subrandino vynuoges ant kekių. 
\par 11 Faraono taurę laikiau savo rankoje. Paėmiau vynuogių, išspaudžiau jas į faraono taurę ir padaviau taurę faraonui”. 
\par 12 Juozapas jam atsakė: “Štai sapno išaiškinimas: trys šakelės yra trys dienos. 
\par 13 Po trijų dienų faraonas sugrąžins tave tarnybon, ir tu padavinėsi faraonui taurę į jo ranką, kaip pirma darydavai, kai buvai jo vyno pilstytojas. 
\par 14 Atsimink mane, kai tau bus gerai, ir pasigailėk manęs, paminėk mane faraonui ir padėk man išeiti iš šitų namų. 
\par 15 Aš esu pavogtas iš hebrajų krašto ir čia nesu nusikaltimo padaręs, už kurį mane laikytų kalėjime”. 
\par 16 Kepėjų viršininkas girdėdamas, kad jis gerai išaiškino, tarė Juozapui: “O aš sapnuodamas mačiau tris pintines ant savo galvos. 
\par 17 Viršutinėje pintinėje buvo įvairių keptų valgių faraonui, ir paukščiai lesė iš tos pintinės”. 
\par 18 Juozapas atsakė: “Štai sapno išaiškinimas: trys pintinės yra trys dienos. 
\par 19 Po trijų dienų faraonas nukirs tavo galvą, tave pakabins ant medžio, ir paukščiai les tavo kūną”. 
\par 20 Trečiąją dieną buvo faraono gimtadienis ir jis iškėlė puotą visiems savo tarnams. Jis atsiminė savo vyriausiąjį vyno pilstytoją ir vyriausiąjį duonkepį. 
\par 21 Faraonas sugrąžino vyriausiąjį vyno pilstytoją į jo tarnybą, ir jis vėl padavinėjo taurę faraonui. 
\par 22 O kepėjų viršininką jis įsakė pakarti, kaip Juozapas buvo jiems išaiškinęs. 
\par 23 Tačiau vyno pilstytojų viršininkas neatsiminė Juozapo ir pamiršo jį.



\chapter{41}


\par 1 Dvejiems metams praėjus, faraonas sapnavo: jis stovėjo prie upės, 
\par 2 ir iš jos išlipo septynios karvės, gražios ir riebios, ir jos ganėsi lankoje. 
\par 3 O po jų išlipo iš upės kitos septynios karvės, bjaurios ir liesos, ir atėjo prie tų karvių upės pakrantėje. 
\par 4 Liesosios karvės surijo anas septynias gražiąsias ir riebiąsias karves. Ir faraonas pabudo. 
\par 5 Vėl užmigęs sapnavo antrą kartą: septynios varpos išaugo iš vieno stiebo, pilnos ir gražios. 
\par 6 O po jų išdygo septynios tuščios ir rytų vėjo išdžiovintos varpos. 
\par 7 Tuščiosios varpos prarijo septynias pilnąsias ir gražiąsias varpas. Faraonas pabudo ir suprato, kad tai sapnas. 
\par 8 Tą rytą faraonas buvo neramus. Jis pasikvietė visus Egipto žynius ir išminčius ir papasakojo jiems savo sapnus, bet nebuvo nė vieno, kuris galėtų juos išaiškinti. 
\par 9 Tada vyno pilstytojų viršininkas kalbėjo faraonui: “Aš šiandien prisimenu savo nusikaltimus. 
\par 10 Faraonas buvo užsirūstinęs ant savo tarnų ir atidavė mane ir vyriausiąjį duonkepį uždaryti į sargybos viršininko kalėjimą. 
\par 11 Ir mudu sapnavome sapną tą pačią naktį, ir kiekvieno sapnas turėjo savo reikšmę. 
\par 12 Su mumis buvo jaunuolis hebrajas, sargybos viršininko vergas. Mes jam papasakojome savo sapnus, ir jis mums išaiškino mūsų sapnų reikšmę. 
\par 13 Kaip jis išaiškino, taip ir įvyko: mane sugrąžino į mano tarnybą, o aną pakorė”. 
\par 14 Tada faraonas pasiuntė pakviesti Juozapą, ir jie skubiai jį išleido iš kalėjimo. Jis, nusiskutęs ir pakeitęs drabužius, atėjo pas faraoną. 
\par 15 Faraonas tarė Juozapui: “Sapnavau sapną, ir nėra nė vieno, kuris galėtų jį išaiškinti. Aš girdėjau, kad tu gerai aiškini sapnus”. 
\par 16 Juozapas atsakė faraonui: “Ne aš, o Dievas duos faraonui palankų aiškinimą”. 
\par 17 Faraonas pasakojo Juozapui: “Sapnavau stovįs ant upės kranto. 
\par 18 Iš upės išlipo septynios karvės, riebios ir gražios, ir jos ganėsi lankoje. 
\par 19 Po jų išlipo kitos septynios karvės, menkos, labai bjaurios ir liesos. Aš nesu matęs tokių bjaurių karvių visoje Egipto šalyje. 
\par 20 Liesosios ir bjauriosios karvės surijo anas septynias riebiąsias karves. 
\par 21 Tačiau nebuvo žymu, kad jos būtų ką prarijusios; jos tebebuvo liesos kaip pradžioje. Po to aš pabudau. 
\par 22 Sapne aš dar regėjau: septynios varpos išaugo iš vieno stiebo, pilnos ir gražios. 
\par 23 Po jų išdygo septynios tuščios, plonos ir rytų vėjo išdžiovintos varpos. 
\par 24 Plonosios varpos prarijo septynias gražiąsias varpas. Aš tai papasakojau žyniams, bet nė vienas negalėjo išaiškinti”. 
\par 25 Juozapas atsakė faraonui: “Faraono sapnai reiškia vieną ir tą patį. Dievas parodė faraonui, ką Jis ketina daryti. 
\par 26 Septynios gražiosios karvės yra septyneri metai ir septynios gražiosios varpos yra septyneri metai. Sapnas reiškia vieną ir tą patį. 
\par 27 O septynios plonosios ir bjauriosios karvės ir septynios tuščiosios, rytų vėjo išdžiovintos varpos yra septyneri ateinančio bado metai. 
\par 28 Todėl aš sakiau faraonui, kad Dievas parodė jam, ką Jis ketina daryti. 
\par 29 Ateina septyneri didelio pertekliaus metai visoje Egipto šalyje. 
\par 30 Bet po jų seks septyneri bado metai, per kuriuos pasimirš buvusi gausa; badas sunaikins šalį. 
\par 31 Buvęs perteklius bus užmirštas šalyje dėl bado, nes jis bus labai baisus. 
\par 32 Du kartus pasikartojęs faraono sapnas reiškia, jog tai yra tikrai Dievo nustatyta ir greitai įvyks. 
\par 33 Dabar faraonas tegul parenka protingą ir sumanų vyrą ir paskiria jį Egipto šalies valdytoju. 
\par 34 Tegul įsako faraonas paskirti prievaizdus visoje šalyje, kurie surinks penktąją Egipto šalies derliaus dalį per septynerius derlingus metus. 
\par 35 Visą šitą būsimųjų gerų metų derlių tegul supila į aruodus, esančius faraono valdomuose miestuose, ir tegul saugoja jį maistui. 
\par 36 Tas maistas bus atsarga septyneriems bado metams, kurie vargins Egipto šalį, kad kraštas nepražūtų bado metu”. 
\par 37 Tas patarimas patiko faraonui ir visiems jo tarnams, 
\par 38 ir jis tarė: “Ar rasime tokį vyrą kaip šis, kuriame būtų Dievo Dvasia?” 
\par 39 Faraonas kalbėjo Juozapui: “Kadangi Dievas tau visa tai apreiškė, tai nėra nė vieno, kuris būtų toks protingas ir sumanus kaip tu. 
\par 40 Tu būsi mano namų valdytoju, ir tavo žodžio klausys visi žmonės. Tik sostu aš būsiu aukščiau tavęs”. 
\par 41 Toliau faraonas tarė Juozapui: “Aš tave skiriu visos Egipto šalies valdytoju”. 
\par 42 Faraonas numovė nuo savo piršto žiedą ir jį užmovė Juozapui; ir aprengė jį ploniausios drobės drabužiais, ir užkabino jam ant kaklo auksinę grandinę. 
\par 43 Ir liepė jį vežti savo antruoju vežimu, ir priešakyje jo šaukti: “Klaupkitės!” Tuo būdu jis tapo visos Egipto šalies valdytoju. 
\par 44 Be to, faraonas tarė Juozapui: “Aš faraonas, ir be tavo žinios niekas nepakels nei rankos, nei kojos visoje Egipto šalyje!” 
\par 45 Faraonas pavadino Juozapą Cafnat Paneachu ir jam davė žmoną Asenatą, Ono kunigo Potiferos dukterį. Ir Juozapas keliavo po visą Egipto šalį. 
\par 46 Juozapas buvo trisdešimties metų amžiaus, kai jis stovėjo faraono, Egipto karaliaus, akivaizdoje ir išėjęs iš faraono apkeliavo visą Egipto žemę. 
\par 47 Per septynerius pertekliaus metus šalyje viskas gausiai užderėjo. 
\par 48 Jis surinko visą septynerių metų maistą ir sukrovė jį miestų sandėliuose. Aplink kiekvieną miestą esančių laukų derlių jis sukrovė tame mieste. 
\par 49 Juozapas pripildė aruodus javų kaip jūros smilčių, tiek daug, kad jų nebuvo įmanoma suskaičiuoti. 
\par 50 Prieš užeinant bado metams, Juozapui gimė du sūnūs iš Asenatos, Ono kunigo Potiferos dukters. 
\par 51 Juozapas pirmąjį pavadino Manasu: “Dievas leido man pamiršti visą mano vargą ir mano tėvo namus”. 
\par 52 Antrąjį jis pavadino Efraimu: “Dievas padarė mane vaisingą mano vargo šalyje”. 
\par 53 Pasibaigė septyneri pertekliaus metai Egipto šalyje. 
\par 54 Prasidėjo septyneri bado metai, kaip Juozapas buvo sakęs. Badas siautė visose šalyse, tačiau Egipte buvo duonos. 
\par 55 Badui prasidėjus visoje Egipto šalyje, žmonės kreipėsi į faraoną, prašydami duonos. Faraonas sakė: “Eikite pas Juozapą! Ką jis jums sakys, darykite”. 
\par 56 Badas išsiplėtė visoje žemėje. Juozapas atidarė javų sandėlius ir pardavinėjo javus egiptiečiams, nes kilo baisus badas Egipto šalyje. 
\par 57 Iš įvairių kraštų žmonės ėjo į Egiptą pas Juozapą pirkti javų, nes buvo baisus badas visose šalyse.



\chapter{42}


\par 1 Jokūbas, išgirdęs, kad javai parduodami Egipte, tarė savo sūnums: “Ko žiūrite vienas į kitą? 
\par 2 Girdėjau, kad javai parduodami Egipte. Vykite tenai ir nupirkite javų, kad gyventume ir nemirtume”. 
\par 3 Dešimt Juozapo brolių iškeliavo pirkti javų į Egiptą. 
\par 4 Tačiau Jokūbas neleido Juozapo brolio Benjamino eiti su broliais, nes bijojo, kad jam neatsitiktų nelaimė. 
\par 5 Izraelio sūnūs kartu su kitais atėjo pirkti javų, nes badas siautė Kanaano šalyje. 
\par 6 Juozapas buvo Egipto šalies valdytojas. Jis pardavinėjo javus visoms žemės tautoms. Atėję Juozapo broliai nusilenkė prieš jį iki žemės. 
\par 7 Juozapas, pamatęs savo brolius, atpažino juos, bet jis elgėsi su jais tarsi su svetimais. Šiurkščiai su jais kalbėdamas, klausė: “Iš kur atvykote?” Jie atsakė: “Iš Kanaano šalies maisto pirkti”. 
\par 8 Juozapas atpažino savo brolius, tačiau jie neatpažino jo. 
\par 9 Juozapas atsiminė sapnus, kuriuos jis sapnavo apie juos, ir jiems tarė: “Jūs esate žvalgai! Atvykote išžvalgyti silpnesniųjų šalies vietų”. 
\par 10 Jie atsakė jam: “Ne, mūsų viešpatie! Tavo tarnai atėjo nusipirkti maisto. 
\par 11 Mes visi esame vieno vyro sūnūs, dori žmonės. Tavo tarnai nėra žvalgai”. 
\par 12 Tačiau jis jiems tarė: “Ne! Jūs atėjote išžvalgyti silpnesniųjų šalies vietų!” 
\par 13 Jie atsakė: “Tavo tarnų yra dvylika brolių, vieno tėvo sūnų, Kanaano šalyje. Jauniausias liko pas tėvą namuose, o vieno jau nebėra”. 
\par 14 Bet Juozapas jiems atsakė: “Yra taip, kaip jums sakiau. Jūs esate žvalgai! 
\par 15 Taip jūs būsite ištirti. Prisiekiu, kaip gyvas faraonas, jūs neišeisite iš čia, kol atvyks jūsų jaunesnysis brolis! 
\par 16 Pasiųskite vieną iš jūsų atvesti jūsų brolį! Jūs būsite suimti, kol bus ištirti jūsų žodžiai, ar tiesą sakote, ar meluojate. Jei ne, kaip gyvas faraonas, jūs esate žvalgai!” 
\par 17 Tris dienas jis išlaikė juos suimtus. 
\par 18 Trečią dieną Juozapas jiems tarė: “Išliksite gyvi su viena sąlyga, nes aš bijau Dievo. 
\par 19 Jei jūs esate dori, vienas iš jūsų telieka suimtas, o kiti keliaukite, pargabenkite javų savo šeimoms nuo bado apsiginti. 
\par 20 Bet atveskite pas mane savo jauniausiąjį brolį, kad jūsų žodžiai pasirodytų tikri ir nemirtumėte!” 
\par 21 Jie kalbėjosi: “Iš tikrųjų esame nusikaltę savo broliui: mes matėme jo sielvartą, kai jis mus maldavo, bet neklausėme. Todėl šita bėda užklupo mus”. 
\par 22 Rubenas sakė jiems: “Ar aš jums nesakiau: ‘Nenusikalskite prieš vaiką!’ Bet jūs neklausėte. Todėl štai išieškomas jo kraujas”. 
\par 23 Jie nežinojo, kad Juozapas suprato jų kalbą, nes jis su jais kalbėjo per vertėją. 
\par 24 Pasitraukęs nuo jų, jis verkė. Tada sugrįžęs pas juos, kalbėjo toliau. Jis paėmė iš jų Simeoną ir jį surišo jų akyse. 
\par 25 Juozapas įsakė pripildyti jų maišus javais, kiekvieno pinigus įdėti atgal į maišą ir jiems duoti davinį kelionei. Tarnai taip ir padarė. 
\par 26 Susikrovę savo javų maišus ant asilų, jie iškeliavo. 
\par 27 Vienas atrišo savo maišą užeigoje, norėdamas pašerti asilą, ir pamatė pinigus. Jie buvo maišo viršuje. 
\par 28 Jis sušuko broliams: “Mano pinigai grąžinti man. Štai jie maiše!” Jų širdys nusiminė, ir jie drebėdami žiūrėjo vienas į kitą ir kalbėjo: “Ką Dievas mums padarė?” 
\par 29 Parėję pas savo tėvą Jokūbą į Kanaano šalį, jie papasakojo jam visa, kas jiems nutiko: 
\par 30 “Vyras, tos šalies valdovas, šiurkščiai kalbėjo su mumis ir mus palaikė žvalgais. 
\par 31 Mes jam sakėme: ‘Esame dori žmonės, o ne žvalgai. 
\par 32 Mes esame dvylika brolių, vieno tėvo sūnūs. Vieno nebėra, o jauniausias yra pas mūsų tėvą Kanaano šalyje’. 
\par 33 Tas vyras, šalies valdovas, mums atsakė: ‘Patikrinsiu, ar esate dori. Vieną jūsų brolį palikite pas mane, paimkite, ko reikia jūsų šeimoms nuo bado apsiginti, keliaukite 
\par 34 ir atsiveskite savo jauniausiąjį brolį, kad įsitikinčiau, jog nesate žvalgai. Tada aš jums atiduosiu jūsų brolį ir galėsite laisvai keliauti!’ ” 
\par 35 Išpylę maišus, kiekvienas rado įrištus savo pinigus. Jie ir jų tėvas, pamatę pinigus, nusigando. 
\par 36 Jų tėvas Jokūbas kalbėjo: “Jūs padarėte, kad netekau vaikų! Juozapo nebėra, Simeono nebėra, ir Benjaminą norite paimti. Viskas atsisuko prieš mane”. 
\par 37 Rubenas atsakė savo tėvui: “Nužudyk mano du sūnus, jei aš jo neparvesiu! Patikėk jį man ir aš tau jį sugrąžinsiu”. 
\par 38 Tėvas tarė: “Mano sūnus nekeliaus su jumis, nes jo brolis yra miręs ir jis likęs man vienas. Jei kelyje atsitiks jam nelaimė, tai jūs nuvarysite mano žilus plaukus su sielvartu į kapus”.



\chapter{43}


\par 1 Badas sunkiai slėgė šalį. 
\par 2 Kai jie sunaudojo javus, atsigabentus iš Egipto, jų tėvas sakė jiems: “Vėl keliaukite ir nupirkite mums maisto”. 
\par 3 Judas jam atsakė: “Tas vyras mus griežtai įspėjo nepasirodyti jam be jauniausiojo brolio. 
\par 4 Jei leisi su mumis mūsų brolį, keliausime ir nupirksime maisto. 
\par 5 Bet jeigu neleisi, neisime, nes tas vyras mums pasakė: ‘Nematysite mano veido, jei jūsų brolio nebus su jumis!’ ” 
\par 6 Izraelis atsakė: “Kodėl jūs man padarėte tokį skausmą, sakydami tam vyrui turį dar vieną brolį?” 
\par 7 Jie atsakė: “Tas vyras nuodugniai klausinėjo apie mus ir mūsų giminę: ‘Ar jūsų tėvas gyvas? Ar dar turite kokį brolį?’ Mes jam atsakėme į šituos klausimus. Argi galėjome žinoti, kad jis lieps atvesti mūsų brolį?” 
\par 8 Judas kalbėjo savo tėvui Izraeliui: “Leisk berniuką su manimi, kad galėtume vykti ir nemirtume badu mes, tu ir mūsų vaikai. 
\par 9 Aš laiduoju už jį! Iš mano rankos jo pareikalausi. Jei aš jo neparvesiu, būsiu tau nusikaltęs visą amžių. 
\par 10 Jei nebūtume delsę, būtume jau antrą kartą sugrįžę”. 
\par 11 Tada jų tėvas Izraelis tarė: “Jei taip, darykite! Pasiimkite į savo maišus geriausių šio krašto vaisių ir nugabenkite tam vyrui dovanų: truputį balzamo, medaus, kvepiančių žolių, miros, riešutų ir migdolų. 
\par 12 Dvigubai tiek pinigų pasiimkite su savimi ir pinigus, kurie buvo jūsų maišuose, grąžinkite iš savo rankų. Gal buvo kokia klaida? 
\par 13 Imkite taip pat savo brolį ir keliaukite pas tą vyrą. 
\par 14 O visagalis Dievas tesuteikia jums malonę to vyro akivaizdoje, kad jis paleistų jums jūsų brolį ir Benjaminą! O jei aš tapsiu bevaikis, tai ir būsiu bevaikis”. 
\par 15 Vyrai pasiėmė dovanų, dvigubai tiek pinigų ir Benjaminą ir, nukeliavę į Egiptą, prisistatė Juozapui. 
\par 16 Juozapas, pamatęs Benjaminą su jais, tarė savo namų prievaizdui: “Įvesk tuos vyrus į namus, papjauk gyvulį ir paruošk maisto, nes jie pietaus su manimi!” 
\par 17 Jis padarė, kaip Juozapas buvo įsakęs, ir įvedė juos į jo namus. 
\par 18 Jie nusigando, kai juos įvedė į Juozapo namus, ir vienas kitam kalbėjo: “Mus veda dėl pinigų, kuriuos radome savo maišuose, kad apkaltintų, suimtų, padarytų mus vergais ir paimtų mūsų asilus”. 
\par 19 Todėl jie kreipėsi į Juozapo namų prievaizdą, užkalbindami jį prieangyje: 
\par 20 “Valdove, paklausyk mūsų! Kai atvykome pirmą kartą pirkti maisto, 
\par 21 tai grįždami užeigoje atrišome maišus ir kiekvieno mūsų visi pinigai buvo maišuose; mes atnešėme juos atgal. 
\par 22 Be to, atsinešėme su savimi dar kitų pinigų maistui pirkti. Mes nežinome, kas įdėjo mums pinigus į maišus”. 
\par 23 Jis atsakė: “Būkite ramūs! Nebijokite! Jūsų ir jūsų tėvo Dievas įdėjo jums lobį į maišus. Aš gavau jūsų pinigus”. Ir jis atvedė Simeoną pas juos. 
\par 24 Po to jis įvedė juos į Juozapo namus ir padavė vandens nusiplauti kojoms; jis pašėrė ir jų asilus. 
\par 25 Ir jie paruošė dovaną, laukdami Juozapo ateinant vidudienį, nes jie girdėjo, kad ten valgys pietus. 
\par 26 Juozapui parėjus namo, jie atnešė jam dovaną, kurią turėjo su savimi, ir nusilenkė jam iki žemės. 
\par 27 Jis klausinėjo, kaip jiems sekasi: “Ar sveikas jūsų senasis tėvas, apie kurį pasakojote? Ar jis dar gyvas?” 
\par 28 Jie atsakė: “Tavo tarnas, mūsų tėvas, yra sveikas ir gyvas”. Jie vėl žemai nusilenkė. 
\par 29 Pamatęs savo brolį Benjaminą, savo motinos sūnų, klausė: “Ar šitas yra jūsų jauniausiasis brolis, apie kurį pasakojote?” Ir tarė jam: “Dievas tebūna tau malonus, mano sūnau!” 
\par 30 Juozapas išskubėjo, nes jis susijaudino dėl brolio ir ieškojo vietos išsiverkti. Ir įėjo į savo kambarį, ir ten verkė. 
\par 31 Jis nusiprausė veidą, išėjo ir susitvardęs tarė: “Paduokite valgį!” 
\par 32 Jie padėjo Juozapui atskirai, broliams atskirai, su juo valgiusiems egiptiečiams taip pat atskirai. Mat egiptiečiai negali valgyti drauge su hebrajais, nes tai jiems yra pasibjaurėjimas. 
\par 33 Jie susėdo prieš jį, pirmagimis pagal savo pirmagimystę, o jauniausias pagal savo jaunumą. Ir jie žvilgčiojo vienas į kitą nustebę. 
\par 34 Valgiai jiems buvo nešami iš Juozapui skirtų patiekalų, bet Benjaminas gavo penkis kartus daugiau negu kiti. Jie gėrė vyną ir linksminosi su juo.



\chapter{44}

\par 1 Juozapas įsakė savo namų prievaizdui: “Pripilk tų vyrų maišus, kiek tik jie pajėgia panešti, ir įdėk kiekvieno pinigus į jo maišą. 
\par 2 O mano sidabrinę taurę įdėk į jauniausiojo maišą kartu su pinigais už javus”. Jis padarė, kaip Juozapas buvo įsakęs. 
\par 3 Rytui išaušus, išleido juos su jų asilais. 
\par 4 Vos tik jiems išėjus iš miesto, Juozapas tarė namų prievaizdui: “Vykis tuos vyrus ir, juos pasivijęs, sakyk: ‘Kodėl jūs atsilyginote piktu už gera? 
\par 5 Kodėl pavogėte mano valdovo sidabrinę taurę? Ar ne iš jos geria mano valdovas ir ar ne su ja jis buria? Jūs blogai pasielgėte’ ”. 
\par 6 Prievaizdas, juos pasivijęs, kalbėjo tuos žodžius. 
\par 7 Jie atsakė jam: “Kodėl mūsų viešpats taip kalba? Gink Dieve, kad tavo tarnai taip pasielgtų! 
\par 8 Pinigus, kuriuos radome maišuose, atnešėme atgal iš Kanaano šalies. Kaip tad mes vogsime iš tavo valdovo namų sidabrą ar auksą? 
\par 9 Pas kurį iš mūsų ją rasi, tas temiršta, o mes visi tapsime tavo valdovo vergais”. 
\par 10 Jis atsakė: “Tebūna, kaip sakote! Pas kurį rasiu taurę, taps vergu, o visi kiti būsite nekalti”. 
\par 11 Tada kiekvienas jų skubiai pastatė savo maišą ant žemės ir atrišo jį. 
\par 12 Jis ėmė ieškoti, pradėjo nuo vyriausiojo ir baigė jauniausiuoju. Ir rado taurę Benjamino maiše. 
\par 13 Jie persiplėšė drabužius, uždėjo maišus ant savo asilų ir sugrįžo į miestą. 
\par 14 Judas ir jo broliai atėjo į Juozapo namus, nes jis dar buvo ten. Jie puolė prieš jį ant žemės. 
\par 15 Juozapas tarė jiems: “Ką padarėte! Argi nežinojote, kad toks vyras kaip aš tikrai moku burti?” 
\par 16 Judas atsakė: “Ką besakysime mūsų valdovui? Kaip bekalbėsime? Ir kuo pasiteisinsime? Dievas rado tavo tarnų kaltę. Mes esame mūsų valdovo vergai, taip pat ir tas, pas kurį rasta taurė”. 
\par 17 Juozapas atsakė: “Gink Dieve, kad taip padaryčiau! Tik tas vyras, pas kurį rasta taurė, bus mano vergas, o jūs laisvi keliaukite pas savo tėvą!” 
\par 18 Tada priartėjo prie jo Judas ir tarė: “Mano valdove, prašau, leisk savo tarnui pasiaiškinti ir nesupyk ant savo tarno, nes esi kaip faraonas. 
\par 19 Mano valdove, tu klausei savo tarnų: ‘Ar jūs turite tėvą ar brolį?’ 
\par 20 Mes atsakėme: ‘Turime seną tėvą ir jauną brolį, gimusį jam senatvėje. Jo brolis yra miręs, ir jis yra vienintelis likęs iš savo motinos; tad tėvas jį myli’. 
\par 21 Tada liepei: ‘Atveskite, kad jį pamatyčiau savo akimis!’ 
\par 22 Mes atsakėme: ‘Berniukas negali palikti savo tėvo, nes tėvas mirs, jei paliks!’ 
\par 23 Tada, valdove, pasakei savo tarnams: ‘Jei jūsų jauniausias brolis neateis su jumis, nepasirodykite mano akyse!’ 
\par 24 Kai parėjome pas tavo tarną, mūsų tėvą, jam pranešėme tavo, valdove, žodžius. 
\par 25 Ir mūsų tėvas tarė: ‘Vėl nukeliavę nupirkite kiek maisto’. 
\par 26 Mes atsakėme: ‘Negalime keliauti! Tik jei leisi jauniausią brolį su mumis, keliausime. Mes negalime pasirodyti tam žmogui be mūsų jauniausiojo brolio’. 
\par 27 Tavo tarnas, mūsų tėvas, atsakė: ‘Jūs patys žinote, kad mano žmona man pagimdė du sūnus. 
\par 28 Vienas išėjo, ir aš pasakiau: ‘Jį sudraskė žvėris’. Ir aš jo daugiau nebemačiau. 
\par 29 Jei ir šitą paimsite iš manęs ir jam atsitiks nelaimė, nuvarysite mano žilus plaukus su sielvartu į kapus’. 
\par 30 Taigi, jei dabar pareisiu pas tavo tarną, mano tėvą, ir su mumis nebus berniuko, prie kurio tėvas labai prisirišęs, 
\par 31 jis numirs, matydamas, kad jo nėra su mumis. Ir tavo tarnai nuvarys tavo tarno, mūsų tėvo, žilus plaukus su sielvartu į kapus. 
\par 32 Tavo tarnas laidavo už berniuką savo tėvui, sakydamas: ‘Jei aš jo neparvesiu, būsiu nusikaltęs tau visą amžių’, 
\par 33 todėl prašau, valdove, palik mane berniuko vietoje tau vergauti, o berniukas tegrįžta su broliais! 
\par 34 Kaip galėčiau grįžti pas savo tėvą be jauniausiojo brolio ir matyti nelaimę, kuri ištiks mano tėvą”.



\chapter{45}

\par 1 Juozapas nebegalėjo susivaldyti ir sušuko esantiems su juo: “Išeikite iš čia!” Nieko nebuvo prie jo, kai Juozapas prisipažino savo broliams. 
\par 2 Jis taip garsiai verkė, kad išgirdo egiptiečiai ir faraono namai. 
\par 3 Juozapas tarė savo broliams: “Aš esu Juozapas. Ar mano tėvas dar gyvas?” Jo broliai išsigandę negalėjo nė žodžio ištarti. 
\par 4 Juozapas tarė savo broliams: “Prieikite prie manęs!” Jiems priėjus, jis kalbėjo: “Aš esu Juozapas, jūsų brolis, kurį pardavėte į Egiptą. 
\par 5 Nesisielokite ir nebijokite, kad mane pardavėte. Jūsų gyvybei išlaikyti Dievas mane siuntė pirma jūsų! 
\par 6 Tik dveji metai, kai badas žemėje; jis dar tęsis penkerius metus, kuriais nebus nei ariama, nei pjaunama. 
\par 7 Dievas atsiuntė mane pirma jūsų, kad išsaugotų jūsų palikuonis ir išgelbėtų jūsų gyvybes dideliu išgelbėjimu. 
\par 8 Taigi ne jūs mane čia atsiuntėte, bet Dievas. Jis mane padarė tėvu faraonui, visų jo namų tvarkytoju ir Egipto šalies valdytoju. 
\par 9 Skubiai keliaukite pas mano tėvą ir jam sakykite: ‘Taip sako Juozapas: ‘Dievas mane padarė viso Egipto viešpačiu. Atvyk pas mane, negaišk! 
\par 10 Gyvensi Gošeno krašte ir būsi arti manęs tu, tavo sūnūs ir vaikaičiai, tavo avys, galvijai ir visa, kas tau priklauso. 
\par 11 Aš tave viskuo aprūpinsiu dar penkerius bado metus, kad tu nenuskurstum su savo šeima’. 
\par 12 Jūs ir mano brolis Benjaminas mato, kad aš jums kalbu. 
\par 13 Praneškite mano tėvui apie visą mano garbę Egipte ir apie visa, ką matote, ir skubiai atgabenkite čia mano tėvą!” 
\par 14 Tada jis puolė savo broliui Benjaminui ant kaklo ir abu verkė. 
\par 15 Jis bučiavo visus savo brolius verkdamas. Po to jo broliai kalbėjosi su juo. 
\par 16 Faraono namuose pasklido žinia: “Atvyko Juozapo broliai”. Tai patiko faraonui ir jo tarnams. 
\par 17 Faraonas tarė Juozapui: “Sakyk savo broliams: ‘Apkraukite savo gyvulius ir, sugrįžę į Kanaano šalį, 
\par 18 pasiimkite savo tėvą bei šeimas, ir ateikite pas mane. Aš jums duosiu Egipto geriausią dalį, kad maitintumėtės žemės gėrybėmis’. 
\par 19 Be to, jiems sakyk: ‘Pasiimkite iš Egipto šalies vežimų savo vaikams bei žmonoms ir, paėmę savo tėvą, ateikite. 
\par 20 Negailėkite savo daiktų palikti, nes geriausia, ką Egipto šalis turi, priklausys jums’ ”. 
\par 21 Izraelio sūnūs taip ir padarė. Juozapas jiems davė vežimus, kaip faraonas įsakė, ir maisto kelionei. 
\par 22 Jis davė kiekvienam jų po vieną naują drabužį, o Benjaminui­tris šimtus sidabrinių ir penkis naujus drabužius. 
\par 23 Savo tėvui jis pasiuntė dešimt asilų, nešančių Egipto gėrybes, ir dešimt asilių, apkrautų javais bei maistu kelionei. 
\par 24 Jis išleido savo brolius ir, jiems išvykstant, tarė: “Nesibarkite kelionėje!” 
\par 25 Ir jie išėjo iš Egipto, ir atėjo į Kanaano žemę pas savo tėvą Jokūbą. 
\par 26 Ir pranešė jam: “Juozapas gyvas ir yra visos Egipto šalies valdytojas”. Jokūbo širdis apmirė, nes jis netikėjo jais. 
\par 27 Ir jie persakė jam visus Juozapo žodžius, kuriuos jis jiems kalbėjo. Kai pamatė vežimus, kuriuos atsiuntė Juozapas, kad jį parvežtų, jų tėvo Jokūbo dvasia atgijo. 
\par 28 Ir Izraelis pasakė: “Užtenka! Mano sūnus Juozapas dar gyvas. Eisiu ir pamatysiu jį prieš numirdamas”.



\chapter{46}

\par 1 Izraelis išsirengė su viskuo, ką turėjo, ir, atėjęs į Beer Šebą, aukojo aukas savo tėvo Izaoko Dievui. 
\par 2 Dievas kalbėjo Izraeliui nakties regėjime: “Jokūbai, Jokūbai!” Tas atsiliepė: “Aš čia”. 
\par 3 Jis kalbėjo: “Aš esu tavo tėvo Dievas, nebijok eiti į Egiptą; ten padarysiu iš tavęs didelę tautą! 
\par 4 Aš lydėsiu tave į Egiptą ir parvesiu atgal, o Juozapas savo ranka užspaus tavo akis”. 
\par 5 Tada Jokūbas išėjo iš Beer Šebos. Izraelio sūnūs vežė tėvą Jokūbą, savo vaikus ir žmonas vežimuose, kuriuos faraonas atsiuntė jiems. 
\par 6 Jie pasiėmė gyvulius ir visą savo turtą, kurį buvo įsigiję Kanaano šalyje, ir atėjo į Egiptą; Jokūbas ir su juo visi jo palikuonys: 
\par 7 sūnūs, vaikaičiai, dukterys bei jo sūnų dukterys. Visus savo palikuonis jis atsivedė į Egiptą. 
\par 8 Šitie yra vardai Izraelio vaikų, atėjusių į Egiptą. Jokūbas, jo pirmagimis­Rubenas. 
\par 9 Rubeno sūnūs: Henochas, Paluvas, Hecronas ir Karmis. 
\par 10 Simeono sūnūs: Jemuelis, Jaminas, Ohadas, Jachinas, Coharas ir kanaanietės sūnus Saulius. 
\par 11 Levio sūnūs: Geršonas, Kehatas ir Meraris. 
\par 12 Judo sūnūs: Eras, Onanas, Šela, Perecas ir Zerachas. Tačiau Eras ir Onanas mirė Kanaano šalyje. Faro sūnūs: Esromas ir Hamulas. 
\par 13 Isacharo sūnūs: Tola, Pūva, Jobas ir Šimronas. 
\par 14 Zabulono sūnūs: Seredas, Elonas ir Jachleelis. 
\par 15 Šitie yra Lėjos sūnūs, kuriuos ji pagimdė Jokūbui Mesopotamijoje, ir duktė Dina; iš viso sūnų bei dukterų buvo trisdešimt trys asmenys. 
\par 16 Gado sūnūs: Cifjonas, Hagis, Šūnis, Ecbonas, Eris, Arodis ir Arelis. 
\par 17 Ašero vaikai: Imna, Išva, Išvis, Berija ir jų sesuo Seracha. Berijos sūnūs: Heberas ir Malkielis. 
\par 18 Šitie yra vaikai Zilpos, kurią Labanas davė savo dukteriai Lėjai; jų buvo šešiolika asmenų. 
\par 19 Jokūbo žmonos Rachelės sūnūs: Juozapas ir Benjaminas. 
\par 20 Juozapas susilaukė Egipto šalyje dviejų sūnų­Manaso ir Efraimo, kuriuos jam pagimdė Ono kunigo Potiferos duktė Asenata. 
\par 21 Benjamino sūnūs: Bela, Becheras, Ašbelis, Gera, Naamanas, Ehis, Rošas, Mupimas, Hupimas ir Ardas. 
\par 22 Šitie yra Rachelės ir Jokūbo vaikai­keturiolika asmenų. 
\par 23 Dano sūnus­Hušimas. 
\par 24 Neftalio sūnūs: Jachceelis, Gūnis, Jeceras ir Šilemas. 
\par 25 Šitie yra sūnūs Bilhos, kurią Labanas davė savo dukteriai Rachelei, iš viso septyni asmenys. 
\par 26 Asmenų, kurie atvyko su Jokūbu į Egiptą, kilusių iš jo, neskaičiuojant Jokūbo sūnų žmonų, iš viso buvo šešiasdešimt šeši asmenys. 
\par 27 Juozapo sūnūs, gimę Egipte, buvo du. Iš viso Jokūbo namams priklausančių asmenų, atvykusių į Egiptą, buvo septyniasdešimt. 
\par 28 Jokūbas pasiuntė pas Juozapą pirma savęs Judą, kad nuvestų jį į Gošeną. Jie atėjo į Gošeno kraštą. 
\par 29 Juozapas savo vežimu važiavo į Gošeną pasitikti savo tėvo Izraelio. Jį pamatęs, puolė jam ant kaklo ir apsikabinęs ilgai verkė. 
\par 30 Izraelis tarė Juozapui: “Dabar galiu mirti, nes pamačiau tavo veidą, kad tu esi gyvas”. 
\par 31 Juozapas sakė savo broliams ir visiems atvykusiems: “Eisiu pas faraoną ir pranešiu jam: ‘Mano broliai ir visi mano tėvo namiškiai, kurie gyveno Kanaano šalyje, atėjo pas mane. 
\par 32 Jie augina gyvulius ir atsigabeno avis, galvijus ir visa, ką turėjo’. 
\par 33 Kai faraonas jus pasišauks ir klaus: ‘Koks jūsų užsiėmimas?’, 
\par 34 atsakykite: ‘Tavo tarnai yra gyvulių augintojai nuo pat jaunystės iki dabar; tokie buvo ir mūsų tėvai’, kad galėtumėte gyventi Gošeno krašte, nes egiptiečiai bjaurisi gyvulių augintojais”.



\chapter{47}

\par 1 Juozapas atėjo ir pranešė faraonui: “Mano tėvas ir broliai su avimis, galvijais ir visu, ką jie turėjo, atėjo iš Kanaano krašto, ir dabar jie yra Gošeno krašte”. 
\par 2 Jis pasiėmė su savimi penkis savo brolius ir pristatė faraonui. 
\par 3 Faraonas paklausė jų: “Koks jūsų užsiėmimas?” Jie atsakė: “Tavo tarnai yra gyvulių augintojai, tokie buvo ir mūsų tėvai. 
\par 4 Mes atvykome kurį laiką pagyventi šioje šalyje, nes tavo tarnai neturi ganyklos savo avims­badas siaučia Kanaano krašte. Prašome, leisk savo tarnams apsigyventi Gošeno krašte”. 
\par 5 Faraonas sakė Juozapui: “Tavo tėvas ir broliai atėjo pas tave. 
\par 6 Egipto šalis yra tavo žinioje. Geriausioje šalies dalyje apgyvendink juos! Jie tegyvena Gošeno krašte! Ir jei žinai iš jų sumanių vyrų, paskirk juos mano bandų prižiūrėtojais”. 
\par 7 Po to Juozapas nuvedė savo tėvą Jokūbą pas faraoną. Ir Jokūbas palaimino faraoną. 
\par 8 Faraonas paklausė Jokūbą: “Kiek tau metų?” 
\par 9 Jokūbas atsakė faraonui: “Šios žemės kelionėje esu šimtą trisdešimt metų. Negausios ir sunkios buvo mano gyvenimo dienos, nepasiekiau savo tėvų amžiaus”. 
\par 10 Jokūbas palaimino faraoną ir išėjo. 
\par 11 Juozapas apgyvendino savo tėvą bei brolius ir davė jiems geriausios žemės Egipte, Ramzio krašte, kaip faraonas įsakė. 
\par 12 Juozapas aprūpino maistu savo tėvą, brolius ir visus namiškius, kiek kuriai šeimai reikėjo. 
\par 13 Visoje šalyje trūko maisto, nes badas buvo baisus; Egipto ir Kanaano šalys buvo išsekintos bado. 
\par 14 Juozapas surinko iš Egipto šalies ir Kanaano šalies visus pinigus, parduodamas javus, ir juos atgabeno į faraono iždą. 
\par 15 Kai Egipto žemėje ir Kanaano žemėje pasibaigė pinigai, egiptiečiai atėjo pas Juozapą ir tarė: “Duok mums duonos! Kodėl mes turime mirti tavo akivaizdoje dėl pinigų stokos?” 
\par 16 Juozapas atsakė: “Jei nebeturite pinigų, atveskite gyvulius, ir aš jums duosiu duonos už juos”. 
\par 17 Jie atvedė savo gyvulius, o Juozapas davė jiems duonos už arklius, avis, galvijus bei asilus. Taip jis metus maitino juos už jų gyvulius. 
\par 18 Metams praėjus, jie vėl atėjo pas jį ir sakė: “Neslėpsime nuo mūsų valdovo, kad nebeturime nei pinigų, nei gyvulių. Mums liko tik mūsų kūnai ir žemė. 
\par 19 Kodėl turime mirti tavo akivaizdoje? Pirk mus ir mūsų žemę už duoną, kad su savo žeme priklausytume faraonui. Duok mums sėklos, kad išliktume gyvi, o žemė nepavirstų dykyne”. 
\par 20 Taip Juozapas nupirko faraonui visą Egipto žemę, nes visi egiptiečiai pardavė savo laukus, kadangi juos labai kankino badas. Tuo būdu kraštas atiteko faraonui. 
\par 21 Ir jis perkėlė žmones į miestus nuo vieno Egipto pakraščio iki kito. 
\par 22 Tik kunigų žemės jis nenupirko, nes kunigai gaudavo išlaikymą iš faraono, todėl jie ir nepardavė savo žemės. 
\par 23 Juozapas sakė žmonėms: “Šiandien aš nupirkau jus ir jūsų žemę faraonui. Štai jums sėkla, apsėkite laukus. 
\par 24 Penktąją derliaus dalį atiduokite faraonui, o keturias dalis pasilaikykite laukui apsėti ir maistui jūsų šeimoms ir vaikams”. 
\par 25 Jie atsakė: “Tu išgelbėjai mums gyvybes. Teatrasime mes malonę savo valdovo akyse ir mes būsime faraono tarnais”. 
\par 26 Ir Juozapas įvedė įstatymą visame Egipto krašte, kuris galioja iki šios dienos, kad faraonui atiduodama viso derliaus penktoji dalis. Išimtį sudaro tik kunigų žemė, kuri nepriklauso faraonui. 
\par 27 Izraelis liko gyventi Egipto šalyje, Gošeno krašte; jie turėjo ten nuosavybę, išsiplėtė ir labai padaugėjo. 
\par 28 Jokūbas gyveno Egipto šalyje septyniolika metų. Jo amžius buvo šimtas keturiasdešimt septyneri metai. 
\par 29 Izraelis, nujausdamas artėjančią mirtį, pasišaukė savo sūnų Juozapą ir jam tarė: “Jei radau malonę tavo akyse, padėk savo ranką po mano šlaunimi ir elkis su manimi maloningai ir teisingai; pažadėk nepalaidoti manęs Egipte. 
\par 30 Aš norėčiau atsigulti šalia savo tėvų. Tad išgabenk mane iš Egipto ir palaidok jų kapinėse”. Juozapas atsakė: “Padarysiu, kaip prašai”. 
\par 31 Tėvas tarė: “Prisiek man!” Ir jis prisiekė. Tada Izraelis nusilenkė ant lovos galo.



\chapter{48}

\par 1 Po tų įvykių Juozapas sužinojo, kad jo tėvas serga. Jis pasiėmė su savimi abu sūnus­Manasą ir Efraimą. 
\par 2 Jokūbui buvo pranešta, kad ateina Juozapas, ir Izraelis sukaupė savo jėgas, ir atsisėdo lovoje. 
\par 3 Jokūbas tarė Juozapui: “Visagalis Dievas man pasirodė Lūzoje, Kanaano šalyje, ir palaimino mane, 
\par 4 ir pasakė: ‘Aš padarysiu tave vaisingą ir padauginsiu, padarysiu iš tavęs daugybę tautų ir šitą kraštą duosiu tavo palikuonims amžinon nuosavybėn’. 
\par 5 Abu tavo sūnūs, gimę Egipte pirmiau, negu aš atvykau pas tave, man priklauso. Efraimas ir Manasas priklausys man taip, kaip ir Rubenas su Simeonu. 
\par 6 O tavo vaikai, kurie gims po jų, tau priklausys. Jie bus vadinami pagal brolius savo paveldėtose dalyse. 
\par 7 Man keliaujant iš Mesopotamijos, kelyje mirė Rachelė; ją palaidojau pakeliui į Efratą, kuri yra Betliejus”. 
\par 8 Izraelis, pamatęs Juozapo sūnus, klausė: “Kas šitie?” 
\par 9 Juozapas atsakė tėvui: “Jie yra mano sūnūs, kuriuos Dievas man čia davė”. Jis sakė: “Atvesk juos prie manęs, kad juos palaiminčiau!” 
\par 10 Izraelio akys buvo aptemusios dėl senatvės, jis vos matė. Privedus juos prie jo, jis apkabinęs juos pabučiavo 
\par 11 ir tarė Juozapui: “Aš nesitikėjau pamatyti tave, o dabar štai Dievas man parodė net ir tavo vaikus!” 
\par 12 Tada Juozapas atitraukė sūnus nuo jo kelių ir nusilenkė iki žemės. 
\par 13 Po to Juozapas paėmė savo dešine Efraimą ties Izraelio kaire ir savo kairiąja Manasą ties Izraelio dešine ir juos privedė prie jo. 
\par 14 Izraelis, ištiesęs savo dešinę, uždėjo ant Efraimo galvos, nors jis buvo jaunesnysis, ir savo kairę ant pirmagimio Manaso galvos, taip padarydamas sąmoningai. 
\par 15 Ir jis laimino Juozapą: “Dievas, kurio akivaizdoje vaikščiojo mano tėvai Abraomas ir Izaokas, Dievas, kuris vedė mane per visą mano gyvenimą iki pat šios dienos, 
\par 16 angelas, kuris mane išgelbėjo iš viso pikto, tepalaimina šiuos vaikus! Tebūna jie vadinami mano vardu ir mano tėvų Abraomo ir Izaoko vardu ir jų palikuonys tedaugėja žemėje!” 
\par 17 Juozapas, pamatęs tėvą laikant dešinę ranką ant Efraimo galvos, buvo nepatenkintas. Jis paėmė savo tėvo ranką, norėdamas ją perkelti nuo Efraimo galvos ant Manaso galvos, 
\par 18 ir tarė: “Ne taip, tėve! Šitas pirmagimis! Padėk savo dešinę ant jo galvos”. 
\par 19 Bet jo tėvas nesutiko ir sakė: “Aš žinau, mano sūnau, aš žinau. Jis irgi taps tauta ir bus didis. Tačiau jo jaunesnysis brolis bus didesnis už jį ir jo palikuonys taps daugybe tautų”. 
\par 20 Jis juos laimino: “Tavo vardu bus laiminama Izraelyje, kur sakys: ‘Dievas tepadaro tave kaip Efraimą ir Manasą’ ”. Taip jis pastatė Efraimą pirma Manaso. 
\par 21 Izraelis tarė Juozapui: “Aš mirsiu, bet Dievas bus su jumis ir jus parves atgal į jūsų tėvų šalį. 
\par 22 Be to, aš daviau tau viena dalimi daugiau negu tavo broliams, kurią atėmiau iš amoritų savo kardu ir savo lanku”.



\chapter{49}

\par 1 Jokūbas, pasišaukęs savo sūnus, kalbėjo: “Susirinkite! Paskelbsiu jums, kas įvyks su jumis ateityje. 
\par 2 Susirinkite ir klausykite, Jokūbo sūnūs! Pasiklausykite Izraelio, savo tėvo! 
\par 3 Rubenai, tu esi mano pirmagimis, mano tvirtybė, mano pajėgumo pradžia, pirmas orumu ir galybe. 
\par 4 Neramus kaip vanduo! Tu neįsigalėsi, nes įlipai į savo tėvo lovą ir atsiguldamas sutepei mano patalą. 
\par 5 Simeonas ir Levis­broliai; smurto įrankiai jų namuose. 
\par 6 Mano siela, neik į jų pasitarimus, nesijunk į jų būrį, mano garbe. Užsirūstinę jie nužudė žmogų ir savivaliaudami sužalojo jaučius. 
\par 7 Prakeiktas tebūna jų nuožmus įtūžimas ir žiaurus pyktis! Aš juos padalinsiu Jokūbe ir išsklaidysiu Izraelyje. 
\par 8 Judai, tu susilauksi savo brolių pagarbos, tavo ranka bus ant tavo priešų sprando; tavo tėvo vaikai nusilenks prieš tave. 
\par 9 Judas­jaunas liūtas. Mano sūnus, kyląs nuo grobio. Jis sustojo, atsigulė kaip liūtas ar kaip liūtė. Kas jį prikels! 
\par 10 Iš Judo nebus atimtas skeptras nė valdžia iš jo palikuonių, kol ateis siųstasis, kuriam paklus tautos. 
\par 11 Jis riša prie vynmedžio savo asilaitį ir prie geriausio vynmedžio savo asilės jauniklį; jis plauna vyne savo drabužį ir vynuogių sultyse­apsiaustą. 
\par 12 Jo akys spindės nuo vyno ir dantys bus balti nuo pieno. 
\par 13 Zabulonas gyvens prie jūros kranto, kur priplaukia laivai; jo žemių ribos sieks Sidoną. 
\par 14 Isacharas yra stiprus asilas, gulįs tarp dviejų nešulių. 
\par 15 Matydamas, kad poilsis geras ir šalis tokia miela, jis palenkė savo petį, kad neštų, ir tapo samdomu bernu. 
\par 16 Danas teis savo tautą, kaip viena iš Izraelio giminių. 
\par 17 Danas bus gyvatė šalia kelio, angis ant tako, gelianti žirgui į kulnis taip, kad jo raitelis nuvirstų atbulas. 
\par 18 Viešpatie, aš laukiu Tavo išgelbėjimo! 
\par 19 Gadas bus užpultas priešų, bet jis vysis juos įkandin, lips jiems ant kulnų. 
\par 20 Ašero duona bus soti; jis tieks maistą net karaliams. 
\par 21 Neftalis­laisvas briedis, jis gražbylys. 
\par 22 Juozapas­jaunas vaismedis prie versmės, jo šakos nusvirusios per mūrą. 
\par 23 Šauliai erzino jį, šaudė ir nekentė jo. 
\par 24 Jo lankas pasiliko stiprus ir jo rankas sustiprino Jokūbo galingojo Dievo rankos. Iš ten ganytojas ir Izraelio uola. 
\par 25 Tavo tėvo Dievas padės tau, Visagalis laimins tave dangaus palaiminimais iš aukštybių, gelmių palaiminimais, esančiais žemai, krūtų ir įsčių palaiminimais. 
\par 26 Tavo tėvo palaiminimai pranoko mano protėvių palaiminimus iki amžinųjų kalvų tolimiausių ribų; jie bus ant Juozapo galvos ir ant galvos vainiko to, kuris buvo atskirtas nuo savo brolių. 
\par 27 Benjaminas­plėšrus vilkas; rytą jis draskys grobį, o vakare padalins jį”. 
\par 28 Tai yra dvylika Izraelio giminių, ir tai jiems kalbėjo tėvas, ir palaimino juos. Kiekvieną palaimino atskiru palaiminimu. 
\par 29 Po to jis jiems tarė: “Aš susijungsiu su savo tauta. Palaidokite mane prie mano tėvų oloje, kuri yra hetito Efrono lauke, 
\par 30 Machpelos lauko oloje, ties Mamre, Kanaano šalyje. Tą lauką Abraomas nupirko iš hetito Efrono nuosavoms kapinėms. 
\par 31 Ten palaidotas Abraomas ir jo žmona Sara, Izaokas ir jo žmona Rebeka, ten aš palaidojau ir Lėją. 
\par 32 Tai laukas ir ola, kurie buvo nupirkti iš Heto vaikų”. 
\par 33 Jokūbas, baigęs duoti nurodymus savo sūnums, įkėlė kojas į lovą, atidavė dvasią ir susijungė su savo tauta.



\chapter{50}

\par 1 Juozapas puolė prie savo mirusio tėvo, verkė ir jį bučiavo. 
\par 2 Jis paliepė savo tarnams gydytojams išbalzamuoti tėvą. Ir gydytojai išbalzamavo Izraelį. 
\par 3 Praėjo keturiasdešimt dienų, nes tiek laiko užtrunka balzamavimas. Egiptiečiai jį apraudojo septyniasdešimt dienų. 
\par 4 Pasibaigus apraudojimo laikui, Juozapas kalbėjo faraono namiškiams: “Jeigu radau jūsų akyse malonę, pasakykite faraonui, 
\par 5 kad mano tėvas mane prisaikdino palaidoti jį jo kape, kurį jis išsikasė Kanaano žemėje. Teišleidžia mane eiti ir palaidoti savo tėvą, o po to sugrįšiu”. 
\par 6 Faraonas atsakė: “Eik ir palaidok savo tėvą, kaip esi prisaikdintas”. 
\par 7 Juozapas išėjo laidoti savo tėvo; su juo keliavo visi faraono tarnai, jo namų prievaizdai, visi Egipto šalies vyresnieji, 
\par 8 visi Juozapo namiškiai, jo broliai ir visi jo tėvo namiškiai. Tik jų vaikai, avys ir galvijai liko Gošeno krašte. 
\par 9 Be to, su juo traukė vežimai ir raiteliai; taip susidarė labai didelis būrys. 
\par 10 Atėję prie Goren Haatado klojimo anapus Jordano, jie garsiai ir graudžiai raudojo; jie raudojo dėl tėvo septynias dienas. 
\par 11 Tos šalies gyventojai kanaaniečiai, matydami tokias raudas prie Goren Haatado klojimo, sakė: “Tai yra didelis gedulas egiptiečiams”. Todėl tą vietą jie praminė Abelmizraimu, kuris yra anapus Jordano. 
\par 12 Jo sūnūs padarė taip, kaip tėvas jiems buvo įsakęs. 
\par 13 Jie jį nugabeno į Kanaano šalį ir palaidojo Machpelos lauko oloje.Tą lauką, esantį ties Mamre, Abraomas nupirko iš hetito Efrono kapinėms. 
\par 14 Palaidojęs tėvą, Juozapas sugrįžo į Egiptą su broliais ir visais, kurie buvo nukeliavę su juo tėvo laidoti. 
\par 15 Jokūbui mirus, Juozapo broliai bijojo ir kalbėjosi: “Galbūt Juozapas pradės neapkęsti mūsų ir atkeršys mums už visas piktadarystes, kurias jam padarėme”. 
\par 16 Jie nusiuntė jam tokią žinią: “Tavo tėvas prieš mirdamas liepė mums, 
\par 17 kad sakytume tau: ‘Atleisk savo broliams jų nusikaltimą ir nuodėmę, nes jie piktai su tavimi pasielgė!’ Taigi dabar prašome: atleisk tavo tėvo Dievo tarnų nusikaltimą”. Juozapas verkė girdėdamas šiuos žodžius. 
\par 18 Po to jie nuėjo pas jį ir, parpuolę prieš jį, sakė: “Mes esame tavo vergai!” 
\par 19 Juozapas jiems atsakė: “Nebijokite! Argi aš užimu Dievo vietą? 
\par 20 Nors jūs man norėjote blogo, Dievas tai pavertė į gera, norėdamas įvykdyti, ką šiandien matome­išgelbėti daugybę žmonių. 
\par 21 Todėl dabar nebijokite! Aš maitinsiu jus ir jūsų vaikus”. Taip jis guodė ir ramino juos. 
\par 22 Juozapas ir jo tėvo namiškiai liko gyventi Egipte. Juozapas gyveno šimtą dešimt metų. 
\par 23 Jis matė Efraimo vaikus iki trečios kartos. Taip pat ir Manaso sūnaus Machyro sūnūs buvo padėti Juozapui ant kelių. 
\par 24 Juozapas sakė savo broliams: “Aš mirštu. Dievas tikrai aplankys jus ir išves iš šitos žemės į kraštą, kurį Jis prisiekdamas pažadėjo Abraomui, Izaokui ir Jokūbui”. 
\par 25 Po to Juozapas prisaikdino Izraelio vaikus: “Tikrai Dievas aplankys jus ir jūs išnešite iš čia mano kaulus”. 
\par 26 Juozapas mirė, sulaukęs šimto dešimties metų. Jie išbalzamavo jį ir paguldė į karstą Egipte.


\end{document}