\begin{document}

\title{Numbers}

\chapter{1}


\par 1 Antraisiais metais, izraelitams išėjus iš Egipto, antro mėnesio pirmą dieną Viešpats kalbėjo Mozei Sinajaus dykumoje Susitikimo palapinėje: 
\par 2 “Suskaičiuokite visus izraelitų vyrus pagal jų kilmę ir šeimas, 
\par 3 visus dvidešimties metų ir vyresnius vyrus, tinkamus karui. Tu ir Aaronas suskaičiuokite juos pagal jų būrius. 
\par 4 Jums padės kiekvienos giminės vyresnieji. 
\par 5 Jų vardai: iš Rubeno­Šedeūro sūnus Elicūras, 
\par 6 iš Simeono­Cūrišadajo sūnus Šelumielis, 
\par 7 iš Judo­Aminadabo sūnus Naasonas, 
\par 8 iš Isacharo­Cuaro sūnus Netanelis, 
\par 9 iš Zabulono­Helono sūnus Eliabas, 
\par 10 iš Juozapo sūnų: iš Efraimo­Amihudo sūnus Elišama ir iš Manaso­ Pedacūro sūnus Gamelielis, 
\par 11 iš Benjamino­Gideonio sūnus Abidanas, 
\par 12 iš Dano­Amišadajo sūnus Ahiezeras, 
\par 13 iš Ašero­Ochrano sūnus Pagielis, 
\par 14 iš Gado­Deuelio sūnus Eljasafas, 
\par 15 iš Neftalio­Enano sūnus Ahyra”. 
\par 16 Šitie yra tautos išrinktieji, atskirų giminių kunigaikščiai ir Izraelio tūkstančių vadai. 
\par 17 Mozė ir Aaronas su šiais išvardintais vyrais 
\par 18 antrojo mėnesio pirmą dieną surinko vyrus pagal jų gimines ir šeimas, visus turėjusius dvidešimt metų ir vyresnius, 
\par 19 kaip Viešpats buvo įsakęs Mozei, ir suskaičiavo juos Sinajaus dykumoje. 
\par 20 Izraelio pirmagimio Rubeno giminės vyrų, dvidešimties metų ir vyresnių, tinkančių eiti į karą, pagal jų kilmę ir šeimas 
\par 21 buvo suskaičiuota keturiasdešimt šeši tūkstančiai penki šimtai. 
\par 22 Iš Simeono sūnų vyrų, dvidešimties metų ir vyresnių, tinkančių eiti į karą, pagal jų kilmę ir šeimas 
\par 23 buvo suskaičiuota penkiasdešimt devyni tūkstančiai trys šimtai. 
\par 24 Iš Gado sūnų vyrų, dvidešimties metų ir vyresnių, tinkančių eiti į karą, pagal jų kilmę ir šeimas 
\par 25 buvo suskaičiuota keturiasdešimt penki tūkstančiai šeši šimtai penkiasdešimt. 
\par 26 Iš Judo sūnų vyrų, dvidešimties metų ir vyresnių, tinkančių eiti į karą, pagal jų kilmę ir šeimas 
\par 27 buvo suskaičiuota septyniasdešimt keturi tūkstančiai šeši šimtai. 
\par 28 Iš Isacharo sūnų vyrų, dvidešimties metų ir vyresnių, tinkančių eiti į karą, pagal jų kilmę ir šeimas 
\par 29 buvo suskaičiuota penkiasdešimt keturi tūkstančiai keturi šimtai. 
\par 30 Iš Zabulono sūnų vyrų, dvidešimties metų ir vyresnių, tinkančių eiti į karą, pagal jų kilmę ir šeimas 
\par 31 buvo suskaičiuota penkiasdešimt septyni tūkstančiai keturi šimtai. 
\par 32 Iš Juozapo giminės, Efraimo sūnų vyrų, dvidešimties metų ir vyresnių, tinkančių eiti į karą, pagal jų kilmę ir šeimas 
\par 33 buvo suskaičiuota keturiasdešimt tūkstančių penki šimtai. 
\par 34 Iš Manaso sūnų vyrų, dvidešimties metų ir vyresnių, tinkančių eiti į karą, pagal jų kilmę ir šeimas 
\par 35 buvo suskaičiuota trisdešimt du tūkstančiai du šimtai. 
\par 36 Iš Benjamino sūnų vyrų, dvidešimties metų ir vyresnių, tinkančių eiti į karą, pagal jų kilmę ir šeimas 
\par 37 buvo suskaičiuota trisdešimt penki tūkstančiai keturi šimtai. 
\par 38 Iš Dano sūnų vyrų, dvidešimties metų ir vyresnių, tinkančių eiti į karą, pagal jų kilmę ir šeimas 
\par 39 buvo suskaičiuota šešiasdešimt du tūkstančiai septyni šimtai. 
\par 40 Iš Ašero sūnų vyrų, dvidešimties metų ir vyresnių, tinkančių eiti į karą, pagal jų kilmę ir šeimas 
\par 41 buvo suskaičiuota keturiasdešimt vienas tūkstantis penki šimtai. 
\par 42 Iš Neftalio sūnų vyrų, dvidešimties metų ir vyresnių, tinkančių eiti į karą, pagal jų kilmę ir šeimas 
\par 43 buvo suskaičiuota penkiasdešimt trys tūkstančiai keturi šimtai. 
\par 44 Tai vyrai, kuriuos suskaičiavo Mozė, Aaronas ir dvylika Izraelio vyresniųjų, kiekvieną pagal jų kilmę, giminę ir šeimas, 
\par 45 dvidešimties metų ir vyresni, tinkantys eiti į karą. 
\par 46 Iš viso jų buvo suskaičiuota šeši šimtai trys tūkstančiai penki šimtai penkiasdešimt. 
\par 47 Nebuvo priskaičiuoti tik levitai, 
\par 48 nes Viešpats kalbėjo Mozei: 
\par 49 “Levio giminės vyrų neskaičiuok kartu su Izraelio sūnumis, 
\par 50 bet paskirk juos šventai tarnystei prie Susitikimo palapinės. Jie nešios palapinę ir visus jos daiktus ir tarnaus prie jos bei gyvens aplink palapinę. 
\par 51 Keliantis į kitą vietą, levitai išardys palapinę, o sustojus­vėl išties ją. Jei kas pašalinis prisiartintų prie jos, bus baudžiamas mirtimi. 
\par 52 Izraelitai statys savo palapines kiekvienas savoje stovykloje bei jiems paskirtoje vietoje, 
\par 53 o levitai statys savo palapines aplink Susitikimo palapinę, kad Izraelio vaikai neiššauktų mano rūstybės. Levitai eis sargybą prie Susitikimo palapinės”. 
\par 54 Izraelitai padarė visa, ką Viešpats buvo įsakęs Mozei.


\chapter{2}


\par 1 Viešpats kalbėjo Mozei ir Aaronui: 
\par 2 “Izraelitai privalo ištiesti savo palapines aplink Sandoros palapinę pagal giminės eilę, kiekviena giminė su savo vėliava. 
\par 3 Rytų pusėje savo būrių palapines išties Judas, jo vadas yra Aminadabo sūnus Naasonas. 
\par 4 Jo giminės vyrų, tinkančių eiti į karą, yra septyniasdešimt keturi tūkstančiai šeši šimtai. 
\par 5 Šalia jo įrengs savo stovyklą Isacharo giminė, jos vadas yra Cuaro sūnus Netanelis. 
\par 6 Jo karių yra penkiasdešimt keturi tūkstančiai keturi šimtai. 
\par 7 Zabulono giminės vadas yra Helono sūnus Eliabas. 
\par 8 Jo giminės vyrų, tinkančių karui, yra penkiasdešimt septyni tūkstančiai keturi šimtai. 
\par 9 Viso Judo stovykloje yra šimtas aštuoniasdešimt šeši tūkstančiai keturi šimtai vyrų. Jie eis pirmieji. 
\par 10 Pietinėje pusėje Rubeno stovyklai vadovaus Šedeūro sūnus Elicūras. 
\par 11 Jo vyrų, tinkančių karui, yra keturiasdešimt šeši tūkstančiai penki šimtai. 
\par 12 Šalia jo įrengs stovyklą Simeono giminė, kurios vadas yra Cūrišadajo sūnus Šelumielis. 
\par 13 Jo tinkančių karui vyrų skaičius yra penkiasdešimt devyni tūkstančiai trys šimtai. 
\par 14 Gado giminės vadas yra Deuelio sūnus Eljasafas. 
\par 15 Jo karių skaičius yra keturiasdešimt penki tūkstančiai šeši šimtai. 
\par 16 Viso Rubeno stovykloje yra šimtas penkiasdešimt vienas tūkstantis keturi šimtai penkiasdešimt vyrų. Rubeno būriai eis antri. 
\par 17 Levitai žygiuos viduryje ir neš Susitikimo palapinę. Jie eis ta tvarka, kaip apsistoja, kiekvienas prie savo vėliavos. 
\par 18 Vakaruose bus Efraimo sūnų stovykla. Jų vadas yra Amihudo sūnus Elišama. 
\par 19 Jo vyrų, tinkančių eiti į karą, yra keturiasdešimt tūkstančių penki šimtai. 
\par 20 Šalia jų bus Manaso giminė. Jos vadas yra Pedacūro sūnus Gamelielis. 
\par 21 Jo vyrų, tinkančių karui, yra trisdešimt du tūkstančiai du šimtai. 
\par 22 Benjamino giminės vadas yra Gideonio sūnus Abidanas. 
\par 23 Jo vyrų, tinkančių karui, yra trisdešimt penki tūkstančiai keturi šimtai. 
\par 24 Viso Efraimo stovykloje yra šimtas aštuoni tūkstančiai šimtas karių. Efraimo būriai eis treti. 
\par 25 Šiaurėje bus Dano stovykla. Jos vadas yra Amišadajo sūnus Ahiezeras. 
\par 26 Jo vyrų, tinkančių karui, yra šešiasdešimt du tūkstančiai septyni šimtai. 
\par 27 Šalia jo išties savo palapines Ašero giminė, kurios vadas yra Ochrano sūnus Pagielis. 
\par 28 Jo vyrų, tinkančių karui, yra keturiasdešimt vienas tūkstantis penki šimtai. 
\par 29 Šalia jų stovyklaus Neftalio giminė. Jos vadas yra Enano sūnus Ahyra. 
\par 30 Jo vyrų, tinkančių karui, yra penkiasdešimt trys tūkstančiai keturi šimtai. 
\par 31 Viso Dano stovykloje yra šimtas penkiasdešimt septyni tūkstančiai šeši šimtai karių. Jie eis paskutiniai”. 
\par 32 Visų izraelitų vyrų pagal jų gimines, šeimas bei kariuomenės būrius buvo šeši šimtai trys tūkstančiai penki šimtai penkiasdešimt. 
\par 33 Levitai nebuvo suskaičiuoti kartu su izraelitais, kaip Viešpats įsakė Mozei. 
\par 34 Izraelitai viską padarė taip, kaip Viešpats įsakė. Jie sustodavo ir žygiuodavo būriais, giminėmis ir šeimomis.



\chapter{3}

\par 1 Tuo metu, kai Viešpats kalbėjo Sinajaus kalne, Aarono ir Mozės palikuonys buvo: 
\par 2 Aarono pirmagimis sūnus Nadabas ir Abihuvas, Eleazaras bei Itamaras. 
\par 3 Jie buvo patepti ir pašvęsti kunigų tarnystei. 
\par 4 Bet Nadabas ir Abihuvas, aukoję svetimą ugnį Viešpačiui Sinajaus dykumose, mirė, nepalikdami vaikų. Aaronui gyvam esant, kunigų tarnystę ėjo Eleazaras ir Itamaras. 
\par 5 Viešpats kalbėjo Mozei: 
\par 6 “Pakviesk Levio giminę pas kunigą Aaroną, kad jam tarnautų. 
\par 7 Jie atliks visus darbus, susijusius su Susitikimo palapine, 
\par 8 prižiūrės jos daiktus ir tarnaus prie palapinės tarp Izraelio vaikų. 
\par 9 Atiduok levitus Aaronui ir jo sūnums; jie visi yra atiduoti jam iš Izraelio vaikų. 
\par 10 Aaroną ir jo sūnus paskirk kunigais. Jei kas iš pašalinių artinsis, bus baudžiamas mirtimi”. 
\par 11 Viešpats kalbėjo Mozei: 
\par 12 “Aš paėmiau levitus izraelitų pirmagimių vietoje. 
\par 13 Kiekvienas pirmagimis yra mano nuo to laiko, kai išžudžiau pirmagimius Egipte; tuomet pašvenčiau sau izraelitų ir jų gyvulių pirmagimius. Jie yra mano. Aš esu Viešpats”. 
\par 14 Izraelitams esant Sinajaus dykumoje, Viešpats kalbėjo Mozei: 
\par 15 “Suskaičiuok levitus pagal jų šeimas, visus vyriškos giminės asmenis, nuo vieno mėnesio amžiaus”. 
\par 16 Mozė suskaitė juos, kaip Viešpats buvo įsakęs. 
\par 17 Štai Levio sūnų vardai: Geršonas, Kehatas ir Meraris. 
\par 18 Geršono sūnūs: Libnis ir Šimis. 
\par 19 Kehato sūnūs: Amramas ir Iccharas, Hebronas ir Uzielis. 
\par 20 Merario sūnūs: Machlis ir Mušis. Tai yra Levio giminė pagal savo šeimas. 
\par 21 Iš Geršono kilo dvi šeimos: libnių ir šimių. 
\par 22 Jų, vyriškos lyties nuo vieno mėnesio amžiaus, suskaityta septyni tūkstančiai penki šimtai asmenų. 
\par 23 Geršono šeimos turėjo statyti savo palapines už Susitikimo palapinės vakarų pusėje, 
\par 24 vadovaujami Laelio sūnaus Eljasafo. 
\par 25 Jie buvo atsakingi už Susitikimo palapinę, jos uždangalus, palapinės įėjimo užuolaidą, 
\par 26 kiemo užkabas, įėjimo į kiemą užkabą, palapinės virves bei visus jos reikmenis. 
\par 27 Kehato giminei priklausė amramų, iccharų, hebronų ir uzielitų šeimos. Tai yra kehatų šeimos. 
\par 28 Jų, vyriškos lyties nuo vieno mėnesio amžiaus, suskaityta aštuoni tūkstančiai šeši šimtai asmenų. Jie prižiūrėjo šventyklą. 
\par 29 Kehato sūnų šeimos statydavo palapines pietinėje Susitikimo palapinės pusėje. 
\par 30 Jiems vadovavo Uzielio sūnus Elicafanas. 
\par 31 Jie turėjo saugoti skrynią, stalą, žvakidę, aukurus, šventyklos indus, naudojamus tarnavimo metu, uždangą ir visus reikmenis. 
\par 32 Levitų vadų vyriausias buvo Eleazaras, kunigo Aarono sūnus. Jis prižiūrės tuos, kurie rūpinasi šventykla. 
\par 33 Iš Merario yra kilusios machlių ir mušių šeimos. 
\par 34 Jų, vyriškos lyties nuo vieno mėnesio amžiaus, buvo šeši tūkstančiai du šimtai asmenų. 
\par 35 Jų vadas buvo Abihailo sūnus Cūrielis, jų stovykla buvo šiaurinėje Susitikimo palapinės pusėje. 
\par 36 Jie saugojo palapinės lentas, kartis, stulpus, jų pakojus, 
\par 37 kiemo stulpus, statomus aplinkui, su jų pakojais, kuoleliais ir virvėmis. 
\par 38 Priešais palapinę, rytinėje pusėje, pasistatydavo palapinę Mozė ir Aaronas su savo sūnumis. Jie saugojo šventyklą tarp Izraelio vaikų, nes jei prisiartintų kas iš pašalinių, turėjo būti baudžiamas mirtimi. 
\par 39 Viešpačiui įsakius, Mozė ir Aaronas suskaitė visus levitus vyriškos lyties nuo vieno mėnesio amžiaus. Jų buvo dvidešimt du tūkstančiai asmenų. 
\par 40 Viešpats tarė Mozei: “Dabar suskaičiuok visus izraelitų pirmagimius vyriškos lyties nuo vieno mėnesio amžiaus. 
\par 41 Atskirk levitus man­Aš esu Viešpats­vietoj visų Izraelio vaikų pirmagimių ir levitų gyvulius vietoj izraelitų galvijų pirmagimių”. 
\par 42 Mozė suskaitė izraelitų pirmagimius, kaip Viešpats buvo įsakęs. 
\par 43 Visų vyriškos lyties pirmagimių nuo vieno mėnesio amžiaus buvo dvidešimt du tūkstančiai du šimtai septyniasdešimt trys. 
\par 44 Viešpats kalbėjo Mozei: 
\par 45 “Imk levitus vietoje izraelitų pirmagimių ir levitų gyvulius vietoje izraelitų gyvulių. Levitai bus mano. Aš esu Viešpats. 
\par 46 Išpirk du šimtus septyniasdešimt tris izraelitų pirmagimius, nes jų yra daugiau negu levitų; 
\par 47 imk nuo kiekvieno penkis šekelius sidabro pagal šventyklos šekelį. Šekelis susideda iš dvidešimt gerų. 
\par 48 Atiduok pinigus Aaronui ir jo sūnums už tuos, kurie viršija levitų skaičių”. 
\par 49 Mozė paėmė išpirkimo pinigus už tuos, kurie viršijo levitų skaičių, 
\par 50 iš izraelitų pirmagimių paėmė pinigus: tūkstantį tris šimtus šešiasdešimt penkis šekelius sidabro pagal šventyklos šekelį. 
\par 51 Juos atidavė Aaronui ir jo sūnums, vykdydamas Viešpaties įsakymą.



\chapter{4}

\par 1 Viešpats kalbėjo Mozei ir Aaronui: 
\par 2 “Suskaičiuok Kehato sūnus levitus pagal jų kilmę ir šeimas, 
\par 3 visus nuo trisdešimties iki penkiasdešimties metų amžiaus; jie prižiūrės Susitikimo palapinę. 
\par 4 Tokios bus kehatų pareigos Susitikimo palapinėje ir Švenčiausiojoje. 
\par 5 Aaronas ir jo sūnūs, prieš iškeliaujant, nuims uždangą, kabančią prieš Švenčiausiąją, ir ja apdengs Liudijimo skrynią, 
\par 6 ant viršaus užties opšrų kailiais ir mėlynu audiniu, ir įvers kartis. 
\par 7 Taip pat padėtinės duonos stalą apdengs mėlyna drobe, sudės smilkytuvus ir indus, taures ir puodelius geriamosioms aukoms pilti, o padėtinę duoną uždės viršuje, 
\par 8 užties raudona drobe, kurią apdengs opšrų kailių uždangalu, ir įvers kartis. 
\par 9 Ims mėlyną audinį, juo apdengs žvakidę, lempas, gnybtuvus, indus nuognaiboms ir visus aliejaus indus; 
\par 10 tai įvynios į opšrų kailių uždangalą ir įvers kartis. 
\par 11 Auksinį aukurą apdengs mėlyna drobe, užties opšrų kailių uždangalu ir įvers kartis. 
\par 12 Visus apeigoms naudojamus daiktus įvynios į mėlyną audinį, ant viršaus užties opšrų kailių uždangalu ir uždės ant neštuvų. 
\par 13 Iš aukuro išims pelenus ir jį apdengs violetine drobe. 
\par 14 Sudėję visus reikmenis, kurie vartojami prie aukuro: indus anglims, šakutes, semtuvėlius ir dubenis, užties opšrų kailių uždangalu ir įvers kartis. 
\par 15 Aaronui ir jo sūnums apdengus šventyklą ir visus jos daiktus, keliantis stovyklai, kehatai atėję juos paims. Jiems negalima prisiliesti prie šventų daiktų, kad nemirtų. Tai yra kehatų pareiga Susitikimo palapinėje. 
\par 16 Kunigo Aarono sūnus Eleazaras rūpinsis aliejumi lempoms, kvapniais smilkalais, padėtine duona, patepimo aliejumi ir viskuo, ko reikia šventyklos apeigoms, taip pat visais šventykloje esančiais daiktais”. 
\par 17 Viešpats kalbėjo Mozei ir Aaronui: 
\par 18 “Neleiskite kehatų giminei žūti. 
\par 19 Tai darykite, kad jie liktų gyvi ir nemirtų, kai ateina prie šventų daiktų. Aaronas ir jo sūnūs paskirs kiekvienam darbą, kas ką turi nešti. 
\par 20 Jiems negalima žiūrėti, kas yra Švenčiausiojoje, kol ji neapdengta, kad nenumirtų”. 
\par 21 Viešpats kalbėjo Mozei: 
\par 22 “Suskaičiuok taip pat Geršono sūnus pagal jų kilmę ir šeimas 
\par 23 nuo trisdešimties iki penkiasdešimties metų amžiaus, visus, kurie tarnauja Sandoros palapinėje. 
\par 24 Geršonų pareigos yra: 
\par 25 nešti palapinės uždangalus, užuolaidą, pakabintą Susitikimo palapinės įėjime, 
\par 26 kiemo, kuris yra aplinkui aukurą ir palapinę, užkabas, virves ir prie jų naudojamus įrankius; tai yra jų tarnavimas. 
\par 27 Geršonai neš, ką lieps Aaronas ir jo sūnūs. Darbai jiems bus nurodyti ir paskirstyti. 
\par 28 Tokia bus Geršono sūnų tarnystė Susitikimo palapinėje; jie bus kunigo Aarono sūnaus Itamaro žinioje. 
\par 29 Suskaičiuok taip pat Merario sūnus pagal jų kilmę ir šeimas 
\par 30 nuo trisdešimties ligi penkiasdešimties metų amžiaus, visus, kurie tarnauja Susitikimo palapinėje. 
\par 31 Jų darbas bus nešti palapinės lentas, jos užkaiščius, stulpus ir pakojus; 
\par 32 taip pat kiemo stulpus, statomus aplinkui, jų pakojus, kuolelius, virves ir visus jų reikmenis. Kiekvienam paskirsite, ką jis turi nešti. 
\par 33 Tai yra Merario sūnų giminės pareiga ir tarnystė Susitikimo palapinėje; jie bus kunigo Aarono sūnaus Itamaro žinioje”. 
\par 34 Mozė, Aaronas ir vyresnieji suskaitė kehatus pagal jų kilmę 
\par 35 nuo trisdešimties ligi penkiasdešimties metų amžiaus, kurie gali tarnauti Susitikimo palapinėje. 
\par 36 Jų buvo du tūkstančiai septyni šimtai penkiasdešimt. 
\par 37 Juos suskaitė Mozė ir Aaronas, kaip Viešpats buvo įsakęs Mozei. 
\par 38 Buvo suskaityta taip pat ir Geršono sūnūs pagal jų kilmę ir šeimas 
\par 39 nuo trisdešimties ligi penkiasdešimties metų amžiaus, kurie gali tarnauti Susitikimo palapinėje. 
\par 40 Jų buvo du tūkstančiai šeši šimtai trisdešimt. 
\par 41 Geršono sūnus suskaitė Mozė ir Aaronas, kaip Viešpats buvo įsakęs. 
\par 42 Buvo suskaityta taip pat ir Merario sūnūs pagal jų kilmę ir šeimas 
\par 43 nuo trisdešimties ligi penkiasdešimties metų amžiaus, kurie gali tarnauti Susitikimo palapinėje. 
\par 44 Jų buvo trys tūkstančiai du šimtai. 
\par 45 Merario sūnus suskaitė Mozė ir Aaronas, kaip Viešpats buvo įsakęs Mozei. 
\par 46 Visų levitų, kuriuos Mozė, Aaronas ir Izraelio vyresnieji suskaitė pagal jų kilmę ir šeimas 
\par 47 nuo trisdešimties ligi penkiasdešimties metų amžiaus, galinčių tarnauti ir nešti Susitikimo palapinę, 
\par 48 buvo aštuoni tūkstančiai penki šimtai aštuoniasdešimt asmenų. 
\par 49 Juos suskaitė Mozė, kaip Viešpats buvo įsakęs, kiekvieną pagal jo tarnystę ir naštą.



\chapter{5}

\par 1 Viešpats kalbėjo Mozei: 
\par 2 “Įsakyk izraelitams pašalinti iš stovyklos visus raupsuotus, visus turinčius plūdimą ir susitepusius mirusiu. 
\par 3 Pašalinkite tokius vyrus ir tokias moteris, kad jie nesuteptų stovyklos, kurios viduryje Aš gyvenu”. 
\par 4 Izraelitai taip ir padarė, ir juos apgyvendino už stovyklos ribų, kaip Viešpats buvo įsakęs Mozei. 
\par 5 Viešpats kalbėjo Mozei: 
\par 6 “Sakyk izraelitams: ‘Jei vyras ar moteris padarytų kurią nors nuodėmę, peržengtų Viešpaties įsakymą ir taip nusikalstų, 
\par 7 tai jis prisipažins kaltas ir atlygins tam, kuriam nusikalto, atiduos tą patį daiktą ir, be to, pridės penktą dalį jo vertės. 
\par 8 O jei nebūtų to, kuriam nusikalto, atiduos viską Viešpačiui, ir tai priklausys kunigui, neskaitant avino, kuris turi būti aukojamas, kad jis būtų sutaikintas. 
\par 9 Visos aukos, kurias aukoja izraelitai, priklauso kunigui. 
\par 10 Ką pavieniai asmenys aukoja šventykloje ir atiduoda kunigui, yra jo nuosavybė’ ”. 
\par 11 Viešpats kalbėjo Mozei: 
\par 12 “Sakyk izraelitams: ‘Jei kieno žmona nusidėtų, būdama neištikima savo vyrui, 
\par 13 ir kitas vyras gulėtų su ja ir ją suterštų, tačiau tai įvyktų slaptai ir būtų paslėpta nuo jos vyro akių, ir nebūtų liudytojų, nes ji nebūtų sugauta, 
\par 14 ir jei pavydo dvasia apimtų vyrą, nepaisant ar žmona iš tiesų susitepusi, ar tai tik neteisingas įtarimas, 
\par 15 jis ją turi atvesti pas kunigą ir aukoti už ją dešimtą efos dalį miežinių miltų; nepils ant jų aliejaus ir nedės smilkalų, nes tai pavydo auka, primenanti nedorybę. 
\par 16 Kunigas ją atves ir pastatys Viešpaties akivaizdoje. 
\par 17 Jis, paėmęs švento vandens moliniame inde, įmes į jį truputį dulkių nuo palapinės aslos. 
\par 18 Kai moteris atsistos Viešpaties akivaizdoje, jis atidengs jos galvą ir paduos jai į ranką pavydo auką; pats gi laikys kartaus vandens, kuris neša prakeikimą, 
\par 19 ir prisaikdins ją, sakydamas: ‘Jei neturėjai santykių su svetimu vyru, tau nekenks šitas kartus vanduo. 
\par 20 Bet jei buvai neištikima savo vyrui ir susitepei gulėdama su kitu, 
\par 21 Viešpats padarys tave prakeikimu tavo tautoje, ir tavo šlaunys ims pūti, o tavo pilvas išsipūs. 
\par 22 Šitas vanduo, nešantis prakeikimą, teįeina į tavo vidų, kad tavo pilvas išsipūstų ir šlaunys imtų pūti’. Moteris atsakys: ‘Amen, amen’. 
\par 23 Kunigas surašys tuos prakeikimus ir juos nuplaus karčiuoju vandeniu. 
\par 24 Jis duos moteriai gerti kartaus vandens, nešančio prakeikimą, kad tas vanduo patektų į moters vidų. 
\par 25 Tada kunigas paims iš jos rankų pavydo auką, ją siūbuos Viešpaties akivaizdoje ir aukos ant aukuro. 
\par 26 Jis ims saują aukojamųjų miltų ir sudegins ant aukuro. Po to vėl duos moteriai gerti kartaus vandens. 
\par 27 Kai ji išgers, jei ji yra nusikaltusi svetimavimu, prakeikimo vanduo, įėjęs į ją, išpūs jos pilvą, ir jos šlaunys ims pūti, ir moteris bus prakeikimas savo tautoje. 
\par 28 O jei ji bus nekalta, jai nepakenks, ir ji galės pastoti. 
\par 29 Toks yra pavydo įstatymas. 
\par 30 Jei vyras apimtas pavydo dvasios atvestų žmoną Viešpaties akivaizdon ir kunigas padarytų visa, kas čia parašyta, 
\par 31 vyras bus nekaltas, o moteris susilauks bausmės, jeigu bus nusikaltusi’ ”.



\chapter{6}

\par 1 Viešpats kalbėjo Mozei: 
\par 2 “Sakyk izraelitams: ‘Jei vyras ar moteris padarys nazarėno įžadą, kad pasišvęstų Viešpačiui, 
\par 3 jis susilaikys nuo vyno ir stipraus gėrimo. Negers vynuogių sulčių nei jokio kito gėrimo, kas iš vynuogių išspaudžiama, nevalgys nei šviežių, nei džiovintų vynuogių. 
\par 4 Visą laiką, kol yra įžadu pasišventęs Viešpačiui, nevalgys nieko, kas iš vynuogių, net sėklų ar odelės. 
\par 5 Padaręs įžadą neskus galvos, kol pasibaigs įžado laikas. Visą savo pasišventimo laiką augins galvos plaukus. 
\par 6 Per visą savo įžado laiką nepalies mirusio, 
\par 7 net savo tėvu, motina, broliu ar seserimi nesusiteps, jei jie numirtų, nes įžado metu bus pasižadėjęs Dievui. 
\par 8 Visą savo įžado laiką jis yra šventas Viešpačiui. 
\par 9 Jei kas staiga mirtų šalia jo ir jis susiteptų, tada jis turės nusiskusti galvą septintą, apsivalymo, dieną. 
\par 10 Aštuntą dieną atneš du balandžius ar du jaunus karvelius ir prie Susitikimo palapinės paduos juos kunigui. 
\par 11 Kunigas vieną jų aukos už nuodėmę, o antrą­deginamąja auka, kad sutaikintų jį, nes jis susitepė mirusiu. Tą dieną jis pakartos įžadą 
\par 12 ir atves metinį avinėlį aukai už kaltę. Ankstesnės dienos nebus įskaitytos, nes jo pasišventimas buvo suteptas. 
\par 13 Tas yra nazarėno įstatymas. Pasibaigus įžado laikui, jį atves prie Susitikimo palapinės durų, 
\par 14 kur jis aukos auką Viešpačiui: metinį sveiką avinėlį deginamajai aukai, metinę sveiką avelę aukai už nuodėmę ir sveiką aviną padėkos aukai, 
\par 15 taip pat apšlakstytą aliejumi neraugintos duonos pintinę, neraugintų bandelių, apteptų aliejumi, ir geriamąją auką. 
\par 16 Kunigas aukos Viešpačiui auką už nuodėmę ir deginamąją auką. 
\par 17 Aviną aukos kaip padėkos auką Viešpačiui, kartu su neraugintos duonos pintine ir geriamąja auka. 
\par 18 Tada prie Susitikimo palapinės durų nuskus nazarėnui plaukus; ir kunigas jo plaukus sudegins su padėkos auka. 
\par 19 Kunigas, paėmęs išvirtą avino petį, vieną neraugintą bandelę iš pintinės bei neraugintą paplotį, įdės į nazarėno rankas po to, kai jo galva bus nuskusta. 
\par 20 Paskui, paėmęs iš nazarėno rankų, jis siūbuos viską Viešpaties akivaizdoje. Visi pašvęstieji daiktai priklausys kunigui. Po to nazarėnas galės gerti vyną. 
\par 21 Toks yra nazarėno, kuris davė įžadą, įstatymas ir auka Viešpačiui, neskaičiuojant to, ką jis aukos pagal savo įžadą, kad išpildytų pasišventimo įstatymą’ ”. 
\par 22 Viešpats kalbėjo Mozei: 
\par 23 “Pasakyk Aaronui ir jo sūnums laiminti izraelitus tokiais žodžiais: 
\par 24 ‘Viešpats telaimina ir tesaugoja tave. 
\par 25 Viešpats teparodo tau savo veidą ir tebūna tau maloningas. 
\par 26 Viešpats teatgręžia savo veidą į tave ir tesuteikia tau ramybę’. 
\par 27 Jie šauksis mano vardo izraelitams, ir Aš juos laiminsiu”.



\chapter{7}


\par 1 Tą dieną, kai pabaigė statyti palapinę ir Mozė patepė ir pašventino ją ir visus jos reikmenis, taip pat aukurą ir visus jo daiktus, 
\par 2 Izraelio giminių kunigaikščiai aukojo 
\par 3 Viešpačiui šešis dengtus vežimus ir dvylika jaučių. (Po vežimą nuo dviejų kunigaikščių ir nuo kiekvieno po jautį.) Visa tai jie atgabeno prie palapinės. 
\par 4 Viešpats sakė Mozei: 
\par 5 “Imk tas aukas palapinės reikalams ir skirstyk levitams pagal jų pareigas”. 
\par 6 Mozė paskirstė vežimus ir jaučius levitams. 
\par 7 Du vežimus ir keturis jaučius jis davė Geršono sūnums. 
\par 8 Keturis vežimus ir aštuonis jaučius davė Merario sūnums. Jų vyresnysis buvo kunigo Aarono sūnus Itamaras. 
\par 9 Kehato sūnums nedavė vežimų nei jaučių, nes jiems patikėta Švenčiausioji turi būti nešama ant pečių. 
\par 10 Aukuro patepimo dieną giminių kunigaikščiai aukojo aukuro reikalams. 
\par 11 Viešpats tarė Mozei: “Kunigaikščiai kiekvienas savo dieną teaukoja aukuro pašventimui skirtas aukas”. 
\par 12 Pirmą dieną aukojo Aminadabo sūnus Naasonas iš Judo giminės. 
\par 13 Tai buvo sidabrinis dubuo, sveriąs šimtą trisdešimt šekelių, ir sidabrinė taurė, sverianti septyniasdešimt šekelių pagal šventyklos šekelį,­abu indai buvo pilni smulkių, aliejumi apšlakstytų miltų duonos aukai; 
\par 14 be to, auksinis indelis, sveriąs dešimt šekelių, pilnas smilkalų; 
\par 15 jautis, avinas ir metinis avinėlis deginamajai aukai; 
\par 16 ožys aukai už nuodėmę; 
\par 17 du jaučiai, penki avinai, penki ožiai ir penki metiniai avinėliai padėkos aukai. Tai buvo Aminadabo sūnaus Naasono auka. 
\par 18 Antrą dieną aukojo Cuaro sūnus Netanelis, Isacharo giminės kunigaikštis: 
\par 19 sidabrinį dubenį, sveriantį šimtą trisdešimt šekelių, sidabrinę taurę, sveriančią septyniasdešimt šekelių,­abu pilnus smulkių, aliejumi apšlakstytų miltų duonos aukai; 
\par 20 auksinį indelį, sveriantį dešimt šekelių, pilną smilkalų; 
\par 21 jautį, aviną ir metinį avinėlį deginamajai aukai; 
\par 22 ožį aukai už nuodėmę; 
\par 23 padėkos aukai du jaučius, penkis avinus, penkis ožius ir penkis metinius avinėlius. Tai buvo Cuaro sūnaus Netanelio auka. 
\par 24 Trečią dieną aukojo Zabulono sūnų kunigaikštis, Helono sūnus Eliabas: 
\par 25 sidabrinį dubenį, sveriantį šimtą trisdešimt šekelių, sidabrinę taurę, sveriančią septyniasdešimt šekelių,­abu pilnus smulkių, aliejumi apšlakstytų miltų duonos aukai; 
\par 26 auksinį indelį, sveriantį dešimt šekelių, pilną smilkalų; 
\par 27 jautį, aviną ir metinį avinėlį deginamajai aukai; 
\par 28 ožį aukai už nuodėmę; 
\par 29 padėkos aukai du jaučius, penkis avinus, penkis ožius, penkis metinius avinėlius. Tai buvo Helono sūnaus Eliabo auka. 
\par 30 Ketvirtą dieną aukojo Rubeno sūnų kunigaikštis, Šedeūro sūnus Elicūras: 
\par 31 sidabrinį dubenį, sveriantį šimtą trisdešimt šekelių, sidabrinę taurę, sveriančią septyniasdešimt šekelių,­abu pilnus smulkių, aliejumi apšlakstytų miltų duonos aukai; 
\par 32 auksinį indelį, sveriantį dešimt šekelių, pilną smilkalų; 
\par 33 jautį, aviną ir metinį avinėlį deginamajai aukai; 
\par 34 ožį aukai už nuodėmę; 
\par 35 padėkos aukai du jaučius, penkis avinus, penkis ožius ir penkis metinius avinėlius. Tai buvo Šedeūro sūnaus Elicūro auka. 
\par 36 Penktą dieną aukojo Simeono sūnų kunigaikštis, Cūrišadajo sūnus Šelumielis: 
\par 37 sidabrinį dubenį, sveriantį šimtą trisdešimt šekelių, sidabrinę taurę, sveriančią septyniasdešimt šekelių,­abu pilnus smulkių, aliejumi apšlakstytų miltų duonos aukai; 
\par 38 auksinį indelį, sveriantį dešimt šekelių, pilną smilkalų; 
\par 39 jautį, aviną ir metinį avinėlį deginamajai aukai; 
\par 40 ožį aukai už nuodėmę; 
\par 41 padėkos aukai du jaučius, penkis avinus, penkis ožius ir penkis metinius avinėlius. Tai buvo Cūrišadajo sūnaus Šelumielio auka. 
\par 42 Šeštą dieną aukojo Gado sūnų kunigaikštis, Deuelio sūnus Elja-safas: 
\par 43 sidabrinį dubenį, sveriantį šimtą trisdešimt šekelių, sidabrinę taurę, sveriančią septyniasdešimt šekelių,­abu pilnus smulkių, aliejumi apšlakstytų miltų duonos aukai; 
\par 44 auksinį indelį, sveriantį dešimt šekelių, pilną smilkalų; 
\par 45 jautį, aviną ir metinį avinėlį deginamajai aukai; 
\par 46 ožį aukai už nuodėmę; 
\par 47 padėkos aukai du jaučius, penkis avinus, penkis ožius ir penkis metinius avinėlius. Tai buvo Deuelio sūnaus Eljasafo auka. 
\par 48 Septintą dieną aukojo Efraimo sūnų kunigaikštis, Amihudo sūnus Elišama: 
\par 49 sidabrinį dubenį, sveriantį šimtą trisdešimt šekelių, sidabrinę taurę, sveriančią septyniasdešimt šekelių,­abu pilnus smulkių, aliejumi apšlakstytų miltų duonos aukai; 
\par 50 auksinį indelį, sveriantį dešimt šekelių, pilną smilkalų; 
\par 51 jautį, aviną ir metinį avinėlį deginamajai aukai; 
\par 52 ožį aukai už nuodėmę; 
\par 53 padėkos aukai du jaučius, penkis avinus, penkis ožius ir penkis metinius avinėlius. Tai buvo Amihudo sūnaus Elišamos auka. 
\par 54 Aštuntą dieną aukojo Manaso sūnų kunigaikštis, Pedacūro sūnus Gamelielis: 
\par 55 sidabrinį dubenį, sveriantį šimtą trisdešimt šekelių, sidabrinę taurę, sveriančią septyniasdešimt šekelių,­abu pilnus smulkių, aliejumi apšlakstytų miltų duonos aukai; 
\par 56 auksinį indelį, sveriantį dešimt šekelių, pilną smilkalų; 
\par 57 jautį, aviną ir metinį avinėlį deginamajai aukai; 
\par 58 ožį aukai už nuodėmę; 
\par 59 padėkos aukai du jaučius, penkis avinus, penkis ožius ir penkis metinius avinėlius. Tai buvo Pedacūro sūnaus Gamelielio auka. 
\par 60 Devintą dieną aukojo Benjamino sūnų kunigaikštis, Gideonio sūnus Abidanas: 
\par 61 sidabrinį dubenį, sveriantį šimtą trisdešimt šekelių, sidabrinę taurę, sveriančią septyniasdešimt šekelių,­abu pilnus smulkių, aliejumi apšlakstytų miltų duonos aukai; 
\par 62 auksinį indelį, sveriantį dešimt šekelių, pilną smilkalų; 
\par 63 jautį, aviną ir metinį avinėlį deginamajai aukai; 
\par 64 ožį aukai už nuodėmę; 
\par 65 padėkos aukai du jaučius, penkis avinus, penkis ožius ir penkis metinius avinėlius. Tai buvo Gideonio sūnaus Abidano auka. 
\par 66 Dešimtą dieną aukojo Dano sūnų kunigaikštis, Amišadajo sūnus Ahiezeras: 
\par 67 sidabrinį dubenį, sveriantį šimtą trisdešimt šekelių, sidabrinę taurę, sveriančią septyniasdešimt šekelių,­abu pilnus smulkių, aliejumi apšlakstytų miltų duonos aukai; 
\par 68 auksinį indelį, sveriantį dešimt šekelių, pilną smilkalų; 
\par 69 jautį, aviną ir metinį avinėlį deginamajai aukai; 
\par 70 ožį aukai už nuodėmę; 
\par 71 padėkos aukai du jaučius, penkis avinus, penkis ožius ir penkis metinius avinėlius. Tai buvo Amišadajo sūnaus Ahiezero auka. 
\par 72 Vienuoliktą dieną aukojo Ašero sūnų kunigaikštis, Ochrano sūnus Pagielis: 
\par 73 sidabrinį dubenį, sveriantį šimtą trisdešimt šekelių, sidabrinę taurę, sveriančią septyniasdešimt šekelių,­abu pilnus smulkių, aliejumi apšlakstytų miltų duonos aukai; 
\par 74 auksinį indelį, sveriantį dešimt šekelių, pilną smilkalų; 
\par 75 jautį, aviną ir metinį avinėlį deginamajai aukai; 
\par 76 ožį aukai už nuodėmę; 
\par 77 padėkos aukai du jaučius, penkis avinus, penkis ožius ir penkis metinius avinėlius. Tai buvo Ochrano sūnaus Pagielio auka. 
\par 78 Dvyliktą dieną aukojo Neftalio sūnų kunigaikštis, Enano sūnus Ahyra: 
\par 79 sidabrinį dubenį, sveriantį šimtą trisdešimt šekelių, sidabrinę taurę, sveriančią septyniasdešimt šekelių,­abu pilnus smulkių, aliejumi apšlakstytų miltų duonos aukai; 
\par 80 auksinį indelį, sveriantį dešimt šekelių, pilną smilkalų; 
\par 81 jautį, aviną ir metinį avinėlį deginamajai aukai; 
\par 82 ožį aukai už nuodėmę; 
\par 83 padėkos aukai du jaučius, penkis avinus, penkis ožius ir penkis metinius avinėlius. Tai buvo Enano sūnaus Ahyros auka. 
\par 84 Tos buvo Izraelio kunigaikščių aukos aukuro pašventinimo proga: dvylika sidabrinių dubenių, dvylika sidabrinių taurių, dvylika auksinių indelių. 
\par 85 Kadangi dubuo svėrė šimtą trisdešimt šekelių sidabro ir kiekviena taurė septyniasdešimt šekelių, tai visi sidabriniai indai svėrė du tūkstančius keturis šimtus šekelių pagal šventyklos šekelį. 
\par 86 Dvylika auksinių indelių, kurie buvo pilni smilkalų, kiekvienas svėrė po dešimt šekelių; viso aukso buvo šimtas dvidešimt šekelių. 
\par 87 Jaučių deginamajai aukai buvo dvylika, avinų dvylika, metinių avinėlių dvylika kartu su duonos auka; aukai už nuodėmę buvo dvylika ožių. 
\par 88 Padėkos aukai buvo dvidešimt keturi jaučiai, šešiasdešimt avinų, šešiasdešimt ožių ir šešiasdešimt metinių avinėlių. Tos aukos buvo duotos aukuro pašventinimui po to, kai jis buvo pateptas. 
\par 89 Kai Mozė įėjo į Susitikimo palapinę kalbėtis su Dievu, jis girdėjo balsą nuo dangčio, kuris buvo ant Sandoros skrynios tarp dviejų cherubų. Iš ten Dievas kalbėjo su juo.



\chapter{8}


\par 1 Viešpats kalbėjo Mozei: 
\par 2 “Sakyk Aaronui, kad jis sustatytų lempas taip, kad visos degtų žvakidės priekyje”. 
\par 3 Aaronas sustatė lempas, kad jos degtų žvakidės priekyje, kaip Viešpats įsakė Mozei. 
\par 4 Visa žvakidė buvo nukalta iš aukso, pagal Viešpaties Mozei parodytą pavyzdį. 
\par 5 Viešpats kalbėjo Mozei, sakydamas: 
\par 6 “Imk levitus iš Izraelio vaikų ir apvalyk juos, 
\par 7 apšlakstydamas apvalomuoju vandeniu, liepk nusiskusti plaukus nuo viso kūno ir išsiplauti drabužius, tai jie bus apvalyti. 
\par 8 Jie ims veršį ir duonos aukai­ aliejumi apšlakstytų smulkių miltų, ir kitą veršį­aukai už nuodėmę. 
\par 9 Sušauk izraelitus ir pastatyk levitus prie Susitikimo palapinės. 
\par 10 Tu pastatysi levitus prieš Viešpatį, ir izraelitai uždės ant jų rankas. 
\par 11 Aaronas paaukos levitus Viešpačiui, kaip Izraelio vaikų auką, kad jie tarnautų Jam. 
\par 12 Levitai uždės rankas ant veršių galvų, kurių vieną paaukosi kaip auką už nuodėmę, o kitą­kaip deginamąją auką; taip levitai bus sutaikinti. 
\par 13 Pastatyk levitus prieš Aaroną ir jo sūnus ir paaukok juos Viešpačiui; 
\par 14 išskirk levitus iš izraelitų, kad jie būtų mano. 
\par 15 Po to jie tarnaus man Susitikimo palapinėje. Taip jie bus apvalyti ir pašvęsti man, 
\par 16 nes jie visi yra man atiduoti vietoje izraelitų pirmagimių, 
\par 17 kadangi visi izraelitų pirmagimiai yra mano­žmonės ir gyvuliai. Aš juos pašvenčiau sau nuo tos dienos, kai išžudžiau egiptiečių pirmagimius. 
\par 18 Aš paėmiau levitus vietoje izraelitų pirmagimių 
\par 19 ir pavedžiau juos Aarono ir jo sūnų priežiūrai, kad tarnautų man Susitikimo palapinėje ir sutaikintų izraelitus. Taip izraelitai bus apsaugoti nuo sunaikinimo, kuris ištiktų juos, jei kas iš jų išdrįstų artintis prie šventyklos”. 
\par 20 Mozė, Aaronas ir izraelitai pasielgė su levitais taip, kaip Viešpats įsakė Mozei: 
\par 21 juos apvalė, jie išsiplovė savo drabužius, o Aaronas paaukojo juos kaip auką Viešpačiui ir sutaikino. 
\par 22 Po to levitai ėjo į Susitikimo palapinę ir atliko jiems skirtą tarnystę Aarono ir jo sūnų priežiūroje. 
\par 23 Viešpats kalbėjo Mozei: 
\par 24 “Toks yra įstatymas levitams: dvidešimt penkerių metų ir vyresni tarnaus Susitikimo palapinėje, 
\par 25 o, sulaukę penkiasdešimties metų amžiaus, jie nebetarnaus. 
\par 26 Jie padės savo broliams saugoti Susitikimo palapinę, bet nedirbs jokio darbo. Tai įsakymas levitams, liečiąs jų tarnystę”.



\chapter{9}

\par 1 Izraelitams išėjus iš Egipto, antraisiais metais, pirmą mėnesį Viešpats kalbėjo Mozei Sinajaus dykumoje: 
\par 2 “Tegul izraelitai švenčia Paschą jai skirtu metu, 
\par 3 šio mėnesio keturioliktos dienos vakare, laikydamiesi visų apeigų ir nurodymų”. 
\par 4 Mozė pasakė izraelitams, kad jie turi švęsti Paschą. 
\par 5 Jie ją šventė pirmo mėnesio keturioliktos dienos vakare prie Sinajaus kalno. Izraelitai darė visa, kaip Viešpats buvo įsakęs Mozei. 
\par 6 Kai kurie buvo susitepę žmogaus lavonu ir tą dieną negalėjo švęsti Paschos. Jie atėjo pas Mozę bei Aaroną ir jiems tarė: 
\par 7 “Mes susitepę mirusiu. Ar dėl to mes neturime teisės aukoti Viešpačiui kartu su visais izraelitais?” 
\par 8 Mozė jiems atsakė: “Palaukite, aš paklausiu, ką Viešpats pasakys šiuo reikalu”. 
\par 9 Ir Viešpats kalbėjo Mozei: 
\par 10 “Jei kas iš izraelitų susiteptų mirusiu ar būtų kelionėje, tešvenčia Paschą 
\par 11 antro mėnesio keturioliktos dienos vakare, valgydamas neraugintą duoną su karčiomis žolėmis; 
\par 12 nieko nepaliks ligi ryto ir nė vieno avinėlio kaulo nesulaužys; laikysis visų Paschos nuostatų. 
\par 13 O jei kas nesuteptas ir nekeliaudamas nešvęstų Paschos, tas bus išnaikintas iš savo tautos, nes neaukojo Viešpačiui skirtu laiku; jis atsakys už savo nuodėmę. 
\par 14 Jei tarp jūsų būtų svetimšalis ar ateivis, kuris norėtų švęsti Paschą, tešvenčia ją Viešpačiui, laikydamasis tų pačių apeigų ir nurodymų. Jums ir ateiviui yra tas pats įstatymas”. 
\par 15 Tą dieną, kai buvo pastatyta Liudijimo palapinė, ją apdengė debesis, o naktį virš palapinės buvo tarsi ugnies liepsna. 
\par 16 Taip buvo visą laiką: dieną ją dengė debesis, o naktį­tarsi liepsna. 
\par 17 Kai palapinę dengiąs debesis pakildavo, izraelitai iškeliaudavo; toje vietoje, kur debesis nusileisdavo, izraelitai statydavo stovyklą. 
\par 18 Pagal Viešpaties įsakymą jie keliaudavo ir Jam įsakius ištiesdavo palapines. Kol debesis dengė palapinę, jie pasilikdavo toje pačioje vietoje. 
\par 19 Jei debesis virš palapinės pasilikdavo ilgą laiką, izraelitai nekeliaudavo. 
\par 20 Kartais debesis virš palapinės likdavo tik kelias dienas. Pagal Viešpaties įsakymą jie ištiesdavo palapines ir pagal Jo įsakymą iškeliaudavo. 
\par 21 Jei debesis stovėdavo virš palapinės tik vieną naktį ir, dienai brėkštant, pakildavo, jie keliaudavo toliau; jei debesis pakildavo, jie iškeliaudavo ar dieną, ar naktį. 
\par 22 O jei dvi dienas ar visą mėnesį, ar metus debesis pasilikdavo ant palapinės, izraelitai pasilikdavo toje pačioje vietoje; kai tik debesis pakildavo, jie keldavosi iš stovyklos. 
\par 23 Pagal Viešpaties įsakymą jie ištiesdavo palapines ir pagal Jo įsakymą iškeliaudavo. Jie laikėsi Viešpaties per Mozę duoto įsakymo.



\chapter{10}


\par 1 Viešpats kalbėjo Mozei: 
\par 2 “Padaryk du sidabrinius trimitus, kad galėtum sušaukti žmones ir jiems pranešti, kai reikės keltis iš stovyklos. 
\par 3 Sutrimitavus abiem trimitais, visa tauta susirinks pas tave prie Susitikimo palapinės. 
\par 4 Jei sutrimituos vienu trimitu, pas tave susirinks izraelitų kunigaikščiai. 
\par 5 Kai trimituos pavojų, pirmieji pakils tie, kurie stovyklauja rytų pusėje. 
\par 6 Trimituojant pavojų antrą sykį, pajudės pietuose esančios stovyklos. Pavojaus ženklą trimituosite skelbdami kelionės pradžią. 
\par 7 Sušaukdami tautos susirinkimą, trimituosite, bet ne pavojaus signalą. 
\par 8 Trimituos Aarono sūnūs, kunigai; tai bus amžinas įstatymas jūsų kartoms. 
\par 9 Jei eisite į karą savo žemėje prieš tuos, kurie jus spaudžia, trimituokite pavojų. Viešpats, jūsų Dievas, prisimins jus ir išgelbės iš jūsų priešų. 
\par 10 Jūsų džiaugsmo dieną, šventės dieną ir pirmą mėnesio dieną, trimituokite prie deginamųjų ir padėkos aukų kaip atminimą Dievui. Aš, Viešpats, jūsų Dievas”. 
\par 11 Antraisiais metais, antro mėnesio dvidešimtą dieną debesis pakilo nuo Liudijimo palapinės, 
\par 12 ir izraelitai iškeliavo būriais iš Sinajaus dykumos. Debesis nusileido Parano dykumoje. 
\par 13 Tai buvo pirmoji jų kelionė, kaip Viešpats buvo įsakęs Mozei. 
\par 14 Pirmieji ėjo Judo giminės žmonės tvarkingais būriais. Jų vadas buvo Aminadabo sūnus Naasonas. 
\par 15 Isacharo giminės pulkų vadas buvo Cuaro sūnus Netanelis. 
\par 16 Zabulono giminės pulkų vadas buvo Helono sūnus Eliabas. 
\par 17 Po to buvo išardyta palapinė, kuria nešini ėjo Geršono ir Merario sūnūs. 
\par 18 Paskui juos keliavo Rubeno pulkai. Jų vadas buvo Šedeūro sūnus Elicūras. 
\par 19 Simeono giminės pulkų vadas buvo Cūrišadajo sūnus Šelumielis. 
\par 20 Gado giminės pulkų vadas buvo Deuelio sūnus Eljasafas. 
\par 21 Po to ėjo Kehato sūnūs, kurie nešė Švenčiausiąją. Kai jie nuėjo, palapinė buvo jau pastatyta. 
\par 22 Toliau ėjo Efraimo giminės pulkai, kuriems vadovavo Amihudo sūnus Elišama. 
\par 23 Manaso giminės pulkų vadas buvo Pedacūro sūnus Gamelielis. 
\par 24 Benjamino giminės pulkų vadas buvo Gideonio sūnus Abidanas. 
\par 25 Paskutiniai iš stovyklos iškeliavo Dano pulkai. Jiems vadovavo Amišadajo sūnus Ahiezeras. 
\par 26 Ašero giminės pulkų vadas buvo Ochrano sūnus Pagielis. 
\par 27 Neftalio giminės pulkų vadas buvo Enano sūnus Ahyra. 
\par 28 Tokia tvarka ėjo Izraelio vaikų pulkai, kai keldavosi iš vienos vietos į kitą. 
\par 29 Mozė sakė savo žmonos broliui Hobabui, midjaniečio Reuelio sūnui: “Mes eisime į vietą, kurią Viešpats mums pažadėjo. Keliauk su mumis, mes tau gera darysime, nes Viešpats pažadėjo Izraeliui gerovę”. 
\par 30 Tas jam atsakė: “Aš nekeliausiu su jumis, bet grįšiu į savo šalį, kurioje gimiau”. 
\par 31 Mozė atsakė: “Nepalik mūsų, nes tu geriausiai žinai, kaip mes turime įsikurti dykumoje; tu būsi mums vietoje akių. 
\par 32 Jei pasiliksi su mumis, duosime tau, ką mums Viešpats duos”. 
\par 33 Jie keliavo nuo Sinajaus kalno tris dienas, ir Viešpaties Sandoros skrynia buvo nešama pirma jų į stovyklai paskirtą vietą. 
\par 34 Viešpaties debesis rodė jiems kelią, keliaujant dienos metu. 
\par 35 Sandoros skrynią pakeliant kelionei, Mozė tardavo: “Viešpatie! Teišsisklaido Tavo priešai ir bėga nuo Tavo veido tie, kurie Tavęs nekenčia”. 
\par 36 Kai sustodavo, jis tardavo: “Sugrįžk, Viešpatie, pas Izraelio tūkstančių tūkstančius!”



\chapter{11}


\par 1 Izraelitai murmėjo prieš Viešpatį, ir Viešpačiui tai nepatiko. Tai išgirdęs, Viešpats užsirūstino, Jo ugnis užsidegė tarp jų ir ėmė naikinti stovyklos pakraštį. 
\par 2 Tauta šaukėsi Mozės ir, kai Mozė pasimeldė Viešpačiui, ugnis užgeso. 
\par 3 Jie praminė tą vietą Tabera, nes Viešpaties ugnis degė tarp jų. 
\par 4 Kartu keliavo daug svetimšalių, kurie pasidavė geiduliams; su jais drauge verkė ir izraelitai, sakydami: “Kas duos mums mėsos? 
\par 5 Atsimename žuvis, kurias valgėme Egipte ir jos mums nieko nekainavo; taip pat prisimename agurkus, melionus, porus, svogūnus ir česnakus. 
\par 6 Dabar mūsų sielos išseko, nes mūsų akys nieko kito nemato­tiktai maną”. 
\par 7 Mana buvo geltona ir panaši į kalendros sėklas. 
\par 8 Žmonės vaikščiojo aplinkui ir ją rinkosi, malė girnomis ar susitrindavo grūstuvuose, virė puoduose ir kepė iš jos papločius; jos skonis buvo panašus į ragaišio su aliejumi skonį. 
\par 9 Nusileidžiant naktį ant stovyklos rasai, krisdavo ir mana. 
\par 10 Mozė išgirdo, kaip žmonės verkia savo šeimose prie palapinių angų. Viešpaties rūstybė smarkiai užsidegė, Mozė taip pat buvo nepatenkintas. 
\par 11 Jis skundėsi Viešpačiui: “Kam vargini savo tarną? Kodėl nerandu malonės Tavo akyse? Ir kodėl uždėjai man visos šitos tautos naštą? 
\par 12 Argi šita tauta yra mano vaikai, ar aš ją pagimdžiau, kad man sakai: ‘Nunešk juos, paėmęs į glėbį, kaip auklė nešioja kūdikį, į žemę, kurią Tu pažadėjai jų tėvams?’ 
\par 13 Iš kur aš imsiu mėsos, kad pasotinčiau tokią daugybę? Jie rauda ir šaukia: ‘Duok mums mėsos!’ 
\par 14 Aš negaliu vienas nešti visos šitos tautos, nes man tai per sunku. 
\par 15 Jeigu taip su manimi elgiesi, geriau nužudyk mane, jei atradau malonę Tavo akyse, kad nebūčiau varginamas tokių didelių bėdų”. 
\par 16 Viešpats atsakė Mozei: “Surink septyniasdešimt Izraelio vyresniųjų, apie kuriuos žinai, kad jie yra tautos vadovai, nuvesk juos prie Susitikimo palapinės ir atsistok ten kartu su jais. 
\par 17 Aš nužengsiu ir kalbėsiu su tavimi, paimsiu dvasios, kuri yra ant tavęs, ir suteiksiu jiems, kad jie kartu su tavimi neštų tautos naštą ir tau nereikėtų nešti vienam. 
\par 18 O tautai sakyk: ‘Pasišventinkite ir rytoj valgysite mėsos, nes Viešpats išgirdo jūsų verksmą, kai sakėte: ‘Kas mums duos mėsos? Gera mums buvo Egipte’. Viešpats jums duos mėsos, ir jūs valgysite. 
\par 19 Ne vieną dieną, ne dvi dienas, ne penkias dienas, ne dešimt dienų ir ne dvidešimt dienų, 
\par 20 bet visą mėnesį, kol nebegalėsite į ją net pažiūrėti; jūs paniekinote Viešpatį, kuris yra tarp jūsų, ir raudojote Jo akivaizdoje, sakydami: ‘Kodėl mes išėjome iš Egipto?’ ” 
\par 21 Mozė tarė: “Šitos tautos yra šeši šimtai tūkstančių pajėgių vyrų, ir Tu sakai, kad duosi jiems visą mėnesį valgyti mėsos? 
\par 22 Argi mes turėsime papjauti avis ir jaučius, kad užtektų visiems? Ar susirinks visos jūros žuvys, kad juos pasotintų?” 
\par 23 Viešpats atsakė jam: “Argi Viešpaties ranka sutrumpėjo? Dabar matysi, ar mano žodžiai išsipildys, ar ne”. 
\par 24 Mozė nuėjo ir pasakė tautai Viešpaties žodžius. Jis surinko septyniasdešimt izraelitų vyresniųjų ir juos sustatė prie palapinės. 
\par 25 Viešpats nužengė debesyje ir jam kalbėjo. Ir Jis paėmė dvasios, kuri buvo ant Mozės, ir davė septyniasdešimčiai vyresniųjų. Kai ant jų nusileido dvasia, jie pradėjo pranašauti ir nepaliovė. 
\par 26 Stovykloje buvo pasilikę du vyrai, vienas vardu Eldadas, o kitas­Medadas. Ir jie gavo tos dvasios, nes ir juodu buvo įtraukti į sąrašą, bet nebuvo atėję prie palapinės. Jie pradėjo pranašauti stovykloje. 
\par 27 Vienas jaunuolis atbėgo ir pranešė Mozei: “Eldadas ir Medadas pranašauja stovykloje!” 
\par 28 Nūno sūnus Jozuė, Mozės tarnas, tarė: “Mano valdove Moze, uždrausk jiems!” 
\par 29 Bet jis atsakė: “Ar tu pavydi dėl manęs? O, kad visa Viešpaties tauta pranašautų ir kad Viešpats duotų kiekvienam savo dvasios!” 
\par 30 Mozė ir Izraelio vyresnieji sugrįžo į stovyklą. 
\par 31 Pakilo Viešpaties siųstas vėjas ir atnešė putpelių. Jų tiek prikrito aplinkui stovyklą, kiek galima apeiti per dvi dienas ir dvi uolektys nuo žemės. 
\par 32 Žmonės rinko putpeles visą tą dieną ir naktį ir dar kitą dieną, ir kiekvienas prisirinko nemažiau kaip dešimt homerų; jas džiovino aplink stovyklą. 
\par 33 Mėsa tebebuvo jiems tarp dantų, ir jie dar nebuvo jos suvalgę, kai Viešpaties rūstybė užsidegė prieš tautą, ir Jis ištiko juos dideliu maru. 
\par 34 Ta vieta buvo pavadinta Kibrot Taavos kapinėmis, nes ten palaidojo žmones, kurie buvo pasidavę geiduliams. 
\par 35 Išėję iš Kibrot Taavos kapinių, jie nukeliavo į Hacerotą ir ten sustojo.



\chapter{12}


\par 1 Mirjama ir Aaronas priekaištavo Mozei dėl jo vedybų, nes jis buvo vedęs etiopę. 
\par 2 Jie sakė: “Argi Viešpats kalbėjo tik per vieną Mozę? Argi Jis nekalbėjo taip pat ir per mus?” Viešpats tai išgirdo. 
\par 3 Mozė gi buvo labai romus, romiausias iš visų žmonių žemėje. 
\par 4 Viešpats staiga prabilo į Mozę, Aaroną bei Mirjamą: “Jūs trys išeikite prie Susitikimo palapinės”. Jiems išėjus, 
\par 5 Viešpats nužengė debesies stulpe ir, stovėdamas palapinės įėjime, pašaukė Aaroną ir Mirjamą. Kai juodu priėjo, 
\par 6 Jis tarė jiems: “Klausykite! Jei kas tarp jūsų yra Viešpaties pranašas, Aš jam apsireiškiu regėjime arba kalbu sapne. 
\par 7 Ne taip yra su mano tarnu Moze, kuris yra ištikimas visuose mano namuose. 
\par 8 Su juo Aš kalbuosi veidas į veidą, atvirai, o ne neaiškiais žodžiais, ir jis mato mano pavidalą. Kaip judu nebijote kalbėti prieš mano tarną Mozę?” 
\par 9 Užsirūstinęs Viešpats pasitraukė. 
\par 10 Pasitraukė ir debesis nuo palapinės. Staiga Mirjamą išbėrė raupsai, ji pabalo kaip sniegas. Aaronas, pažvelgęs į ją ir pamatęs ją apdengtą raupsais, 
\par 11 tarė Mozei: “Maldauju, mano valdove, tenepasilieka ant mudviejų ši nuodėmė, kurią padarėme per savo kvailumą. 
\par 12 Tenebūna ji kaip kūdikis, negyvas gimęs, kurio kūno dalis jau sugedus”. 
\par 13 Mozė šaukėsi Viešpaties: “Maldauju, o Dieve, išgydyk ją”. 
\par 14 Viešpats jam atsakė: “Jei jos tėvas būtų spjovęs jai į veidą, argi ji nebūtų turėjusi rausti bent septynias dienas? Tebūna atskirta septynias dienas nuo stovyklos ir paskui bus pašaukta atgal”. 
\par 15 Mirjama buvo septynias dienas atskirta nuo stovyklos, ir tauta nepajudėjo iš vietos, kol Mirjama nebuvo pašaukta atgal. 
\par 16 Po to izraelitai iškeliavo iš Haceroto ir apsistojo Parano dykumoje.



\chapter{13}


\par 1 Ir Viešpats kalbėjo Mozei: 
\par 2 “Siųsk vyrus, po vieną vyresnįjį iš kiekvienos giminės, išžvalgyti Kanaano žemės, kurią duosiu izraelitams”. 
\par 3 Mozė padarė, ką Viešpats įsakė. Jis išsiuntė iš Parano dykumos kiekvienos giminės vyresnįjį: 
\par 4 Rubeno giminės­Zakūro sūnų Šamūvą, 
\par 5 Simeono giminės­Horio sūnų Šafatą, 
\par 6 Judo giminės­Jefunės sūnų Kalebą, 
\par 7 Isacharo giminės­Juozapo sūnų Igalą, 
\par 8 Efraimo giminės­Nūno sūnų Ozėją, 
\par 9 Benjamino giminės­Rafuvo sūnų Paltį, 
\par 10 Zabulono giminės­Sodžio sūnų Gadielį, 
\par 11 iš Juozapo sūnų, Manaso giminės­ Susio sūnų Gadį, 
\par 12 Dano giminės­Gemalio sūnų Amielį, 
\par 13 Ašero giminės­Mykolo sūnų Setūrą, 
\par 14 Neftalio giminės­Vofsio sūnų Nachbį, 
\par 15 Gado giminės­Machio sūnų Geuelį. 
\par 16 Tai vardai vyrų, kuriuos Mozė pasiuntė išžvalgyti kraštą. Nūno sūnų Ozėją jis pavadino Jozue. 
\par 17 Mozė, siųsdamas juos į Kanaano šalį, tarė jiems: “Pradėkite šalies pietuose ir, nuėję į kalnus, 
\par 18 apžiūrėkite žemę ir žmones, kurie ten gyvena: ar jie galingi, ar silpni, ar jų mažai, ar daug; 
\par 19 ar pati žemė gera, ar bloga; kokie miestai, ar turi mūro sienas, ar ne; 
\par 20 ar žemė derlinga, ar nualinta, ar miškinga, ar ne. Būkite drąsūs ir mums atneškite tos žemės vaisių”. Buvo metas, kada skinamos pirmosios vynuogės. 
\par 21 Jie nuėję išžvalgė žemę nuo Cino dykumos iki Rehobo, kur einama į Lebo Hamatą. 
\par 22 Jie patraukė į pietus ir atėjo į Hebroną, kur gyveno Anako sūnūs Ahimanas, Šešajas ir Talmajas. Hebronas įkurtas septyneriais metais anksčiau už Egipto miestą Coaną. 
\par 23 Nuėję prie Eškolo upelio, nupjovė šaką su viena keke vynuogių, ir du vyrai ją nešė ant karties. Taip pat paėmė granato vaisių ir figų iš tos vietos, 
\par 24 kuri buvo pavadinta Eškolu, nes ten Izraelio vaikai nupjovė vynuogių kekę. 
\par 25 Žvalgai, apėję visą šalį, sugrįžo po keturiasdešimties dienų 
\par 26 pas Mozę, Aaroną ir izraelitus į Kadešą, į Parano dykumą. Jie papasakojo visiems, ką sužinojo, ir parodė to krašto vaisius. 
\par 27 Jie sakė: “Buvome nuėję į žemę, į kurią mus siuntei. Ji tikrai plūsta pienu ir medumi, ir štai jos vaisiai; 
\par 28 bet jos gyventojai yra stiprūs, miestai dideli, apjuosti mūro sienomis. Ten matėme ir Anako palikuonių. 
\par 29 Amalekiečiai gyvena pietuose, hetitai, jebusiečiai ir amoritai kalnuose, kanaaniečiai prie jūros ir Jordano slėnyje”. 
\par 30 Bet Kalebas nuramino žmones prieš Mozę ir sakė: “Eikime ir užimkime tą žemę, nes mes esame pajėgūs juos nugalėti”. 
\par 31 Bet vyrai, buvę su juo, sakė: “Mes negalime eiti prieš tas tautas, nes jos už mus stipresnės”. 
\par 32 Ir jie skleidė tarp Izraelio vaikų blogus atsiliepimus apie tą žemę, kurią buvo išžvalgę, kalbėdami: “Žemė, kurią išžvalgėme, ryja savo gyventojus, o žmonės, kuriuos matėme, yra labai aukšto ūgio. 
\par 33 Ten matėme milžinus iš Anako giminės, ir mes buvome prieš juos kaip žiogai, ir tokie mes buvome jų akyse”.



\chapter{14}

\par 1 Visa tauta pakėlė balsus ir raudojo tą naktį. 
\par 2 Jie murmėjo prieš Mozę ir Aaroną: “Verčiau būtume mirę Egipte arba žuvę šioje dykumoje. 
\par 3 Kodėl Viešpats atvedė mus į šią žemę? Ar tam, kad žūtume nuo kardo, o mūsų žmonos ir vaikai taptų priešo grobiu? Ar ne geriau būtų grįžti į Egiptą?” 
\par 4 Jie tarėsi: “Išsirinkime vadą ir grįžkime į Egiptą”. 
\par 5 Mozė ir Aaronas puolė ant žemės prieš visą Izraelio vaikų susirinkimą. 
\par 6 Nūno sūnus Jozuė ir Jefunės sūnus Kalebas iš tų, kurie žvalgė kraštą, perplėšė savo rūbus 
\par 7 ir kalbėjo izraelitams: “Žemė, kurią apėjome ir išžvalgėme, labai gera. 
\par 8 Jei Viešpats bus maloningas, Jis mus įves į tą žemę, plūstančią pienu ir medumi. 
\par 9 Tik nesukilkite prieš Viešpatį ir nebijokite to krašto žmonių. Mes juos valgysime kaip duoną, jie neturi apsaugos, o Viešpats yra su mumis, nebijokime!” 
\par 10 Bet žmonės ketino juodu užmušti akmenimis. Ir Viešpaties šlovė pasirodė visiems izraelitams virš Susitikimo palapinės. 
\par 11 Viešpats tarė Mozei: “Ar dar ilgai šita tauta niekins mane? Kodėl jie netiki manimi, matydami visus stebuklus, kuriuos padariau jų akivaizdoje? 
\par 12 Aš juos bausiu maru ir sunaikinsiu, o iš tavęs padarysiu didesnę ir galingesnę tautą už šitą”. 
\par 13 Mozė kalbėjo Viešpačiui: “Tada egiptiečiai, iš kurių Tu išvedei šią tautą, išgirs tai, ką Tu padarei tautai, 
\par 14 ir pasakys šio krašto gyventojams, kurie girdėjo, kad Tu, Viešpatie, esi su mumis, kad Tu esi regimas veidu į veidą, kad Tavo debesis yra virš šios tautos ir kad Tu eini pirma mūsų dienos metu debesies stulpe ir naktį­ ugnies stulpe; 
\par 15 todėl jei Tu išžudysi savo žmones, tautos, kurios girdėjo apie Tavo šlovę, sakys: 
\par 16 ‘Kadangi Viešpats nepajėgė įvesti šitos tautos į žemę, kurią jiems prisiekdamas pažadėjo, tai išžudė juos dykumoje’. 
\par 17 Viešpatie, meldžiu Tave, parodyk savo galią, kaip esi pasakęs: 
\par 18 ‘Viešpats yra kantrus ir kupinas gailestingumo, atleidžiantis neteisybes ir nusikaltimus, tačiau nepaliekantis kalto nenubausto, bet baudžiantis už tėvų nusikaltimus vaikus iki trečios ir ketvirtos kartos’. 
\par 19 Maldauju, atleisk šitos tautos nusikaltimus dėl Tavo didelio gailestingumo, kaip atleisdavai jai nuo išėjimo iš Egipto iki šiol”. 
\par 20 Viešpats atsakė: “Atleidžiu, kaip prašei. 
\par 21 Kaip Aš gyvas, visa žemė bus pilna Viešpaties šlovės. 
\par 22 Kadangi visi žmonės, kurie matė mano šlovę ir stebuklus, kuriuos dariau Egipte ir dykumoje, mane gundė jau dešimt kartų ir neklausė mano balso, 
\par 23 jie neišvys žemės, kurią pažadėjau jų tėvams. Niekas iš tų, kurie mane pykdė, nematys jos. 
\par 24 Savo tarną Kalebą, kuris kupinas kitokios dvasios ir iki galo sekė manimi, įvesiu į tą žemę, į kurią jis buvo nuėjęs, ir jo palikuonys ją paveldės. 
\par 25 Kadangi amalekiečiai ir kanaaniečiai gyvena slėniuose, rytoj iš stovyklos visi keliaukite į dykumą Raudonosios jūros link”. 
\par 26 Viešpats kalbėjo Mozei ir Aaronui: 
\par 27 “Kiek dar šita pikta tauta murmės prieš mane? Aš girdžiu Izraelio vaikų murmėjimą, kai jie murma prieš mane. 
\par 28 Sakyk jiems: ‘Kaip Aš gyvas,­ sako Viešpats,­padarysiu jums taip, kaip jūs kalbėjote. 
\par 29 Šioje dykumoje liks jūsų lavonai. Visi, kurie buvote suskaičiuoti, dvidešimties metų ir vyresni, ir murmėjote prieš mane, 
\par 30 neįeisite į žemę, kurią jums pažadėjau, išskyrus Jefunės sūnų Kalebą ir Nūno sūnų Jozuę. 
\par 31 Jūsų vaikus, apie kuriuos sakėte, kad jie bus priešų grobis, įvesiu į tą žemę, kurią jūs paniekinote. 
\par 32 Jūsų lavonai kris šioje dykumoje. 
\par 33 Jūsų vaikai klajos dykumoje keturiasdešimt metų dėl jūsų paleistuvystės, kol jūsų lavonai pasiliks dykumoje. 
\par 34 Pagal skaičių dienų, kai jūs žvalgėte žemę, už keturiasdešimt dienų jūs nešiosite savo kaltes keturiasdešimt metų, už kiekvieną dieną­metai, ir jūs pažinsite, ką reiškia būti mano atmestiems. 
\par 35 Kaip kalbėjau, taip ir padarysiu visai šiai piktai tautai, kuri sukilo prieš mane­visi mirs šioje dykumoje’ ”. 
\par 36 Vyrai, kuriuos Mozė buvo išsiuntęs išžvalgyti žemę, ir kurie grįžę sukurstė tautą murmėti prieš Viešpatį, blogai kalbėdami apie kraštą 
\par 37 ir skleisdami blogus atsiliepimus apie tą žemę, buvo ištikti Viešpaties akivaizdoje ir mirė. 
\par 38 Iš visų, kurie buvo išėję žemę išžvalgyti, gyvi liko tik Nūno sūnus Jozuė ir Jefunės sūnus Kalebas. 
\par 39 Kai Mozė pasakė visus šiuos žodžius izraelitams, jie labai nuliūdo. 
\par 40 Atsikėlę anksti rytą, jie užlipo ant kalno, sakydami: “Eisime į žemę, kurią Viešpats pažadėjo, nes mes nusikaltome”. 
\par 41 Mozė jiems atsakė: “Kodėl neklausote Viešpaties įsakymo? Jūs nieko gero nelaimėsite. 
\par 42 Neikite, nes Viešpats neis su jumis, jūs žūsite nuo priešų. 
\par 43 Amalekiečiai ir kaananiečiai yra prieš jus. Jūs žūsite nuo kardo, nes neklausėte Viešpaties, todėl Viešpats nebus su jumis”. 
\par 44 Bet jie nusprendė eiti į kalnus. Tačiau nei Viešpaties Sandoros skrynia, nei Mozė nepajudėjo iš stovyklos. 
\par 45 Atėję amalekiečiai ir kaananiečiai, gyvenantys kalnuose, juos sumušė ir vijosi iki Hormos.



\chapter{15}

\par 1 Viešpats kalbėjo Mozei: 
\par 2 “Sakyk izraelitams: ‘Kai būsite įėję ir apsigyvenę pažadėtoje žemėje, kurią jums duosiu, 
\par 3 ir aukosite Viešpačiui auką iš bandos ar kaimenės­deginamąją, vykdydami įžadus, laisva valia aukodami ar savo iškilmėse, kad būtų malonus kvapas Viešpačiui, 
\par 4 kiekvienas, kuris aukos, atsineš duonos aukai dešimtą efos dalį smulkių miltų, sumaišytų su vienu ketvirtadaliu hino aliejaus, 
\par 5 ir prie deginamosios ar kitos gyvulinės aukos pridės ketvirtą hino dalį vyno prie kiekvieno avinėlio geriamajai aukai. 
\par 6 Prie kiekvieno avino duonos aukai pridėkite dvi dešimtąsias efos smulkių miltų, sumaišytų su trečdaliu hino aliejaus, 
\par 7 ir geriamajai aukai trečią dalį hino vyno, kad būtų malonus kvapas Viešpačiui. 
\par 8 Kai aukosite jautį deginamajai aukai ar įžadui įvykdyti, ar padėkos aukoms, 
\par 9 duokite prie kiekvieno jaučio tris dešimtąsias efos smulkių miltų, sumaišytų su puse hino aliejaus, 
\par 10 ir pusę hino vyno geriamajai aukai, kad būtų malonus kvapas Viešpačiui. 
\par 11 Taip darykite prie kiekvieno jaučio, avino, avinėlio ar ožio 
\par 12 pagal aukų skaičių, kurias aukosite. 
\par 13 Vietiniai gyventojai taip darys aukodami auką, kad būtų malonus kvapas Viešpačiui; 
\par 14 ir jeigu tarp jūsų esantis ateivis norės aukoti auką, jis darys taip pat, kaip ir jūs. 
\par 15 Tas pats nuostatas galioja ir jums, ir ateiviams, gyvenantiems jūsų žemėje, nes kaip jūs, taip ir ateiviai bus prieš Viešpatį. 
\par 16 Toks pat įstatymas bus jums ir ateiviams, gyvenantiems tarp jūsų’ ”. 
\par 17 Viešpats kalbėjo Mozei: 
\par 18 “Sakyk izraelitams: ‘Kai būsite pažadėtoje žemėje, kurią jums duosiu, 
\par 19 ir valgysite to krašto duonos, aukokite Viešpačiui auką. 
\par 20 Aukokite paplotį iš pirmos tešlos, kaip aukojate derliaus pirmienas. 
\par 21 Iš pirmos tešlos aukokite Viešpačiui per visas jūsų kartas. 
\par 22 Jei netyčia ko nors nevykdytumėte iš to, ką Aš įsakiau Mozei 
\par 23 ir per jį paskelbiau, nuo tos dienos, kai daviau savo įsakymus, per visas jūsų kartas, 
\par 24 jei tai padarysite dėl neapsižiūrėjimo, aukokite man veršį deginamajai aukai, kad būtų malonus kvapas, kartu su duonos ir geriamąja auka, pagal visus nuostatus, ir ožį aukai už nuodėmę. 
\par 25 Kunigas sutaikins izraelitus, ir jiems bus atleista, nes jie nusikalto netyčia, tačiau jie aukos deginamąją auką Viešpačiui ir auką už savo nuodėmę. 
\par 26 Bus atleista Izraelio tautai ir tarp jų gyvenantiems ateiviams, nes nusikaltimas įvyko dėl nežinojimo. 
\par 27 Jei vienas žmogus nusikalstų nežinodamas, teaukoja metinę ožką už savo nusikaltimą. 
\par 28 Kunigas sutaikins jį, nes nusikalto Viešpačiui nežinodamas; jis bus sutaikintas, ir kaltė jam bus atleista. 
\par 29 Vietiniams gyventojams ir ateiviams, netyčia nusikaltus, bus taikomas tas pats įstatymas. 
\par 30 Bet jei žmogus nusikalstų sąmoningai ir taip paniekintų Viešpatį, nežiūrint ar jis bus vietinis, ar ateivis, jis bus išnaikintas iš savo tautos, 
\par 31 nes paniekino Viešpaties žodį ir Jo įsakymą. Todėl ta siela bus sunaikinta. Jis atsakys už savo nusikaltimą’ ”. 
\par 32 Izraelitams būnant dykumoje, jie rado žmogų, renkantį malkas sabato dieną. 
\par 33 Jie atvedė jį pas Mozę ir Aaroną, 
\par 34 ir uždarė jį, nes nežinojo, ką su juo daryti. 
\par 35 Viešpats tarė Mozei: “Tas žmogus turi mirti, visi žmonės tegul užmuša jį akmenimis už stovyklos ribų”. 
\par 36 Išvedę už stovyklos, užmušė jį akmenimis, kaip Viešpats buvo įsakęs. 
\par 37 Viešpats kalbėjo Mozei: 
\par 38 “Įsakyk izraelitams pasidaryti savo apdarų kraštuose kutus su mėlynomis juostelėmis. 
\par 39 Pažvelgę į juosteles, atsiminkite Viešpaties įsakymus ir nesekite savo širdimis ir akimis, kurios veda į paleistuvystę. 
\par 40 Atsiminkite mano įsakymus, juos vykdykite ir būkite šventi. 
\par 41 Aš esu Viešpats, jūsų Dievas, kuris jus išvedžiau iš Egipto žemės”.



\chapter{16}


\par 1 Korachas, Levio sūnaus Kehato sūnaus Iccharo sūnus, ir iš Rubeno sūnų Eliabo sūnūs Datanas ir Abiramas bei Peleto sūnus Onas 
\par 2 sukilo prieš Mozę. Prie jų prisidėjo du šimtai penkiasdešimt izraelitų, tautos kunigaikščių, žinomų bendruomenėje ir gerbiamų vyrų. 
\par 3 Jie susirinkę kalbėjo prieš Mozę ir Aaroną: “Gana judviem; visi izraelitai yra šventi ir Viešpats yra tarp jų. Kodėl judu keliatės aukščiau Viešpaties tautos?” 
\par 4 Tai išgirdęs, Mozė puolė ant žemės 
\par 5 ir kalbėjo Korachui ir su juo esantiems: “Rytoj Viešpats parodys, kas yra šventas ir kas yra Jo. Tiems, kuriuos Jis išsirinko, Jis leis prisiartinti prie Jo. 
\par 6 Kiekvienas imkite savo smilkytuvą: tu, Korachai, ir visas tavo būrys, 
\par 7 pasiėmę ugnies, užberkite ant jos smilkalų Viešpaties akivaizdoje. Kurį Jis išsirinks, tas bus šventas. Jūs, Levio sūnūs, keliatės per aukštai!” 
\par 8 Mozė sakė Korachui: “Klausykite, Levio sūnūs! 
\par 9 Argi jums dar maža, kad Izraelio Dievas jus išsirinko iš visos tautos ir pašaukė eiti tarnystę Viešpaties palapinėje ir tarnauti tautai? 
\par 10 Jis pašaukė savo tarnystei tave ir tavo brolius levitus, tai kodėl dar reikalaujate kunigystės? 
\par 11 Kodėl sukilote tu ir tavo pasekėjai prieš Viešpatį? Kas gi yra Aaronas, kad prieš jį murmate?” 
\par 12 Mozė pasiuntė pašaukti abu Eliabo sūnus Dataną ir Abiramą. Juodu atsakė: “Mudu neisime. 
\par 13 Ar tau dar maža, kad mus išvedei iš žemės, plūstančios pienu ir medum, kad nužudytum dykumoje? Ar dar nori viešpatauti mums? 
\par 14 Ar nuvedei mus į žemę, kurioje teka pienas ir medus, ir ar davei mums paveldėti laukus ir vynuogynus? Ar ir akis nori šiems žmonėms išlupti? Mudu neateisime!” 
\par 15 Mozė, labai įpykęs, meldė Viešpatį: “Nežiūrėk į jų aukas; aš neėmiau iš jų nė asilaičio ir nė vieno nenuskriaudžiau”. 
\par 16 Mozė sakė Korachui: “Rytoj tu ir visas tavo būrys stokite Viešpaties akivaizdoje: tu, jie ir Aaronas. 
\par 17 Imkite kiekvienas savo smilkytuvą, įdėkite į juos smilkalų ir atneškite smilkytuvus Viešpaties akivaizdon­du šimtus penkiasdešimt smilkytuvų; taip pat tu ir Aaronas atneškite savo smilkytuvus”. 
\par 18 Visi įsidėjo ugnies į smilkytuvus ir užbėrę ant jos smilkalų atsistojo prie Susitikimo palapinės įėjimo, kartu su Moze su Aaronu. 
\par 19 Korachas sušaukė visus izraelitus prie Susitikimo palapinės, ir visiems matant pasirodė Viešpaties šlovė. 
\par 20 Viešpats tarė Mozei ir Aaronui: 
\par 21 “Pasitraukite iš šio susirinkimo, Aš juos bematant sunaikinsiu”. 
\par 22 Juodu puolė ant žemės, prašydami: “Dieve, Tu kiekvieno kūno dvasios Dievas. Argi vienam nusidėjus Tavo rūstybė sunaikins visus?” 
\par 23 Viešpats atsakė Mozei: 
\par 24 “Įsakyk visiems pasitraukti nuo Koracho, Datano ir Abiramo palapinių”. 
\par 25 Mozė ėjo prie Datano ir Abiramo, jį sekė Izraelio vyresnieji. 
\par 26 Jis tarė tautai: “Pasitraukite nuo šių piktadarių palapinių ir nelieskite nieko, kas jiems priklauso, kad ir jūs nežūtumėte dėl jų nuodėmių”. 
\par 27 Visiems pasitraukus nuo Koracho, Datano ir Abiramo palapinių, Datanas ir Abiramas išėję stovėjo savo palapinių angose kartu su žmonomis ir vaikais. 
\par 28 Mozė tarė: “Dabar matysite, kad Viešpats siuntė mane visa tai daryti ir kad aš nieko nedariau savo valia. 
\par 29 Jei jie mirs paprasta mirtimi, kaip miršta visi žmonės, tai Viešpats manęs nesiuntė; 
\par 30 bet jei Viešpats padarys naują dalyką, kad žemė atsivers ir prarys juos ir visa, kas jiems priklauso, ir jie gyvi pateks į mirusiųjų buveinę, žinokite, kad jie piktžodžiavo Viešpačiui”. 
\par 31 Kai tik jis baigė kalbėti, žemė prasiskyrė po jų kojomis 
\par 32 ir atsivėrusi prarijo juos su jų palapinėmis, žmonėmis ir visu lobiu. 
\par 33 Jie gyvi nugrimzdo į mirusiųjų buveinę, ir žemė apdengė juos, ir jie buvo išnaikinti iš susirinkusiųjų. 
\par 34 O visi izraelitai, kurie stovėjo aplinkui, žūstantiems šaukiant, pradėjo bėgti, nes jie sakė: “Kad tik ir mūsų žemė neprarytų”. 
\par 35 Viešpaties siųsta ugnis sunaikino tuos du šimtus penkiasdešimt vyrų, kurie aukojo smilkalus. 
\par 36 Po to Viešpats kalbėjo Mozei: 
\par 37 “Įsakyk Aarono sūnui kunigui Eleazarui surinkti smilkytuvus iš degėsių, ir išsklaidyti ugnį, nes jie pašventinti. 
\par 38 Smilkytuvus tų, kurie nusidėjo prieš savo sielas, tegul perkala į skardas ir jomis apdengia aukurą, nes jie buvo aukojami prieš Viešpatį ir tapo pašventinti. Tai bus izraelitams atsiminimo ženklas”. 
\par 39 Kunigas Eleazaras surinko varinius smilkytuvus, kuriuose aukojo gaisre žuvę, perkalė į skardas ir apdengė jomis aukurą, 
\par 40 kaip atminimą Izraelio vaikams, kaip Viešpats jam sakė per Mozę, kad svetimasis, kuris nėra Aarono palikuonis, nesiartintų aukoti smilkalų Viešpačiui ir nežūtų kaip Korachas ir visas jo būrys. 
\par 41 Kitą dieną visi izraelitai murmėjo prieš Mozę ir Aaroną: “Judu nužudėte Viešpaties žmones”. 
\par 42 Žmonėms susirinkus prieš Mozę ir Aaroną, jie atsigręžė į Susitikimo palapinę, ir štai ją apdengė debesis ir pasirodė Viešpaties šlovė. 
\par 43 Mozė ir Aaronas nuėjo prie Susitikimo palapinės. 
\par 44 Viešpats tarė Mozei: 
\par 45 “Pasitraukite nuo šių žmonių. Aš juos tuojau sunaikinsiu”. Juodu puolė ant žemės. 
\par 46 Mozė sakė Aaronui: “Imk smilkytuvą, pasisemk žarijų nuo aukuro, užberk ant jų smilkalų ir skubiai eik prie žmonių, kad juos sutaikintum, nes jau pasireiškė Viešpaties rūstybė ir prasidėjo maras”. 
\par 47 Ir Aaronas padarė, kaip Mozė įsakė, ir nubėgo į žmonių vidurį, tarp kurių jau buvo prasidėjęs maras. Jis užbėrė smilkalų ir sutaikino tautą. 
\par 48 Jis stovėjo tarp mirusiųjų ir gyvųjų. Maras liovėsi. 
\par 49 Mirė keturiolika tūkstančių septyni šimtai, neskaitant žuvusiųjų Koracho maište. 
\par 50 Kai maras liovėsi, Aaronas sugrįžo pas Mozę prie Susitikimo palapinės.



\chapter{17}


\par 1 Viešpats kalbėjo Mozei: 
\par 2 “Sakyk izraelitams, kad kiekviena giminė duotų lazdą; iš visų jų kunigaikščių pagal jų giminę dvylika lazdų. Užrašyk kiekvieno vardą ant jo lazdos. 
\par 3 Ant Levio lazdos užrašyk Aarono vardą. Nuo giminės kunigaikščio bus po vieną lazdą. 
\par 4 Jas sudėk Susitikimo palapinėje ties liudijimu, kur jums apsireiškiu. 
\par 5 Kurį iš jų išsirinksiu, to lazda pražys, taip padarysiu galą izraelitų murmėjimui, kuriuo jie prieš judu murma”. 
\par 6 Mozė pranešė tai izraelitams. Kiekvienos giminės kunigaikštis davė lazdą. Buvo dvylika lazdų, tarp jų ir Aarono lazda. 
\par 7 Mozė jas padėjo Viešpaties akivaizdoje Liudijimo palapinėje. 
\par 8 Mozė, įėjęs kitą dieną, rado žaliuojančią Aarono, Levio giminės, lazdą. Pumpurai išsiskleidė žiedais, sužaliavo lapeliais ir subrandino migdolus. 
\par 9 Mozė išnešė visas lazdas iš šventyklos prie izraelitų. Jie apžiūrėjo jas ir kiekvienas atsiėmė savo lazdą. 
\par 10 Tuomet Viešpats tarė Mozei: “Įnešk atgal Aarono lazdą prie liudijimo, kad ji būtų ženklas maištaujantiems, kad pasibaigtų jų murmėjimas prieš mane, ir jie nemirtų”. 
\par 11 Mozė padarė, kaip Viešpats įsakė. 
\par 12 Izraelitai sakė Mozei: “Štai mes mirštame, mes visi žūvame. 
\par 13 Kas tik artinasi prie Viešpaties palapinės, tas miršta. Argi visi būsime sunaikinti?”



\chapter{18}


\par 1 Viešpats kalbėjo Aaronui: “Tu ir tavo sūnūs būsite atsakingi už šventyklą ir kunigų tarnystę. 
\par 2 Tavo broliai iš Levio giminės, tavo tėvo palikuonys, prisidės prie tavęs ir tarnaus tau, bet tu ir tavo sūnūs atliksite kunigų tarnystę Liudijimo palapinėje. 
\par 3 Levitai tarnaus tau ir atliks visus palapinės darbus, tačiau jie nelies šventyklos indų ir aukuro, kad jie ir jūs nežūtumėte. 
\par 4 Tebūna jie su tavimi ir tetarnauja Susitikimo palapinėje, atlikdami visus jos darbus. Svetimasis neturi būti tarp jūsų. 
\par 5 Prižiūrėkite šventyklą ir aukurą, kad nekiltų mano rūstybė prieš izraelitus. 
\par 6 Aš jums daviau jūsų brolius levitus iš izraelitų kaip dovaną, kad jie tarnautų Susitikimo palapinėje. 
\par 7 Tu ir tavo sūnūs atlikite kunigų tarnystę. Visa, kas priklauso aukurui ir kas yra už uždangos, bus kunigų aptarnaujama; jei kas svetimas artinsis, bus baudžiamas mirtimi”. 
\par 8 Viešpats kalbėjo Aaronui: “Aš tau duodu savo aukų dalį iš visko, ką izraelitai pašvenčia. Duodu tai tau ir tavo sūnums, nes esate patepti. Tas įstatymas bus amžinas. 
\par 9 Šitos aukos, aukojamos Viešpačiui, priklausys tau: duonos auka, auka už nuodėmę ir auka už kaltę. Jos yra šventos ir teks tau ir tavo sūnums. 
\par 10 Jas valgysite šventoje vietoje, tik vyrai valgys tas aukas, nes jos šventos. 
\par 11 Visas izraelitų siūbuojamąsias aukas duodu tau, tavo sūnums ir dukterims amžina teise: kas tavo namuose nesuteptas, valgys jas. 
\par 12 Geriausio aliejaus, vynuogių ir javų pirmojo derliaus, aukojamų Viešpačiui, duodu tau. 
\par 13 Visi pirmieji jūsų krašto vaisiai, atnešami Viešpačiui, bus tavo ir, kas tavo namuose nesuteptas, juos valgys. 
\par 14 Visa, kas Izraelyje pašvęsta, bus tavo. 
\par 15 Visi pirmagimiai, kurie aukojami Viešpačiui, ar tai būtų žmonės, ar gyvuliai, priklausys tau. Žmogaus pirmagimis turi būti išpirktas ir kiekvieno nešvaraus gyvulio pirmagimį aukotojas privalės išpirkti. 
\par 16 Kūdikis vieno mėnesio amžiaus bus išperkamas už penkis šekelius sidabro pagal šventyklos šekelį, kurį sudaro dvidešimt gerų. 
\par 17 Galvijų, avių ir ožkų pirmagimių negalima išpirkti, nes jie paskirti Viešpačiui: jų kraują išliesi ant aukuro ir taukus sudeginsi, kad būtų malonus kvapas Viešpačiui. 
\par 18 Jų mėsa bus tavo kaip ir siūbuojamosios aukos krūtinė ir dešinysis petys. 
\par 19 Visas šventas aukas, kurias izraelitai aukoja Viešpačiui, duodu tau ir tavo sūnums bei dukterims amžina teise. Tai amžina druskos sandora prieš Viešpatį tau ir tavo vaikams”. 
\par 20 Viešpats kalbėjo Aaronui: “Izraelitų žemėje nieko nepaveldėsi ir neturėsi tarp jų dalies: Aš­tavo dalis ir paveldėjimas. 
\par 21 Levitams atiduodu visas Izraelio dešimtines už tarnystę, kurią jie man atlieka Susitikimo palapinėje, 
\par 22 kad izraelitai nesiartintų prie palapinės, nenusidėtų ir nemirtų. 
\par 23 Tik levitai tarnaus Susitikimo palapinėje ir bus už ją atsakingi; tai bus amžinas įstatymas visoms jūsų kartoms. Jie nieko nepaveldės tarp izraelitų. 
\par 24 Izraelio vaikų dešimtines, atnešamas Viešpačiui, atidaviau levitams. Todėl jiems pasakiau: ‘Jūs nieko nepaveldėsite tarp izraelitų’ ”. 
\par 25 Po to Viešpats kalbėjo Mozei: 
\par 26 “Taip kalbėk levitams: ‘Gavę iš izraelitų dešimtines, tų dešimtinių dešimtą dalį aukokite Viešpačiui. 
\par 27 Tai bus jums įskaityta kaip grūdai iš klojimo ar dalis nuo vynuogių spaustuvo. 
\par 28 Taip ir jūs aukosite auką Viešpačiui iš surinktų dešimtinių, atiduodami ją kunigui Aaronui. 
\par 29 Iš visko, kas duodama jums, aukokite aukas Viešpačiui, iš visko, kas geriausia, pašvęstąją dalį’. 
\par 30 Todėl sakyk jiems: ‘Kai atnešite iš visko tai, kas geriausia, bus tai įskaityta levitams kaip gauta iš klojimo ir vynuogių spaustuvo. 
\par 31 Jūs ir jūsų šeimos valgys tai bet kur, nes tai užmokestis už tarnystę, kurią atliekate Susitikimo palapinėje. 
\par 32 Nenusikalskite, pasilaikydami sau geriausius dalykus, ir nesutepkite izraelitų aukų, kad nemirtumėte’ ”.



\chapter{19}

\par 1 Viešpats kalbėjo Mozei ir Aaronui: 
\par 2 “Įsakyk izraelitams atvesti sveiką žalą karvę, dar nekinkytą į jungą, 
\par 3 ir duokite ją kunigui Eleazarui. Ji bus išvesta iš stovyklos ir papjauta jo akivaizdoje. 
\par 4 Tada Eleazaras, padažęs jos kraujyje pirštą, pašlakstys septynis kartus Susitikimo palapinės įėjimo link. 
\par 5 Paskui ji bus sudeginta jo akivaizdoje: oda, mėsa, kraujas ir mėšlas. 
\par 6 Kunigas įmes į ugnį, kurioje dega karvė, kedro medžio, yzopo ir raudonų siūlų. 
\par 7 Po to kunigas išplaus savo drabužius, apsiplaus vandeniu ir grįš į stovyklą, ir jis bus nešvarus iki vakaro. 
\par 8 Tas, kuris karvę sudegins, išplaus savo drabužius, išsimaudys ir bus nešvarus iki vakaro. 
\par 9 Kas nors nesusitepęs susems karvės pelenus ir juos išpils už stovyklos švarioje vietoje, kad jie būtų izraelitų laikomi apvalymo vandeniui padaryti, nes karvė sudeginta kaip auka už nuodėmę. 
\par 10 Kuris susems karvės pelenus, plaus savo drabužius ir bus nešvarus iki vakaro. Tai bus izraelitams ir tarp jų gyvenantiems ateiviams amžinas įstatymas. 
\par 11 Kas paliečia žmogaus lavoną, bus nešvarus septynias dienas. 
\par 12 Jis apsivalys šiuo vandeniu trečią ir septintą dieną ir bus švarus. Jei trečią dieną neapsivalys, tai septintą dieną nebus švarus. 
\par 13 Kuris palies mirusio žmogaus lavoną ir nebus apšlakstytas minėtu vandeniu, suteps Viešpaties palapinę ir bus išnaikintas iš izraelitų; kadangi jis neapšlakstytas apvalomuoju vandeniu, jis liks nešvarus. 
\par 14 Šitas yra įstatymas apie žmogų, mirusį palapinėje. Visi, kurie įeina į jo palapinę, ir visi ten esantieji bus nešvarūs septynias dienas. 
\par 15 Indas, neturintis dangčio, bus suteptas. 
\par 16 Jei kas lauke paliestų užmušto kardu ar savaime mirusio žmogaus lavoną, kaulą ar karstą, bus nešvarus septynias dienas. 
\par 17 Dėl nešvaraus žmogaus tegul ima karvės, sudegintos apvalyti nuo nuodėmės, pelenų, įberia į indą ir užpila tekančio vandens ant jų. 
\par 18 Nesusitepęs žmogus, padažęs jame yzopą, teapšlaksto visą palapinę ir visus daiktus bei susitepusius prisilietimu žmones. 
\par 19 Nesusitepęs apšlakstys nešvarųjį trečią ir septintą dieną, ir jis apsivalys septintą dieną, išplaus savo drabužius, pats išsimaudys ir bus nešvarus iki vakaro. 
\par 20 Jei kas nebus tuo būdu apvalytas, jis bus išnaikintas iš izraelitų, nes sutepė Viešpaties šventyklą; jis nebuvo apšlakstytas apvalomuoju vandeniu, todėl yra nešvarus. 
\par 21 Tai amžinas nuostatas jiems, kad tas, kuris šlakstė vandeniu, plautų savo drabužius, ir kiekvienas, palietęs apvalomąjį vandenį, bus nešvarus iki vakaro. 
\par 22 Ką nešvarus žmogus palies­suteps, kas prisiliestų suteptų daiktų­bus nešvarus iki vakaro”.



\chapter{20}

\par 1 Izraelitai atvyko į Cino dykumą pirmą mėnesį ir sustojo Kadeše. Ten mirė Mirjama ir buvo palaidota. 
\par 2 Toje vietoje nebuvo vandens. Izraelitai, susirinkę pas Mozę ir Aaroną, 
\par 3 priekaištaudami kalbėjo: “Geriau būtume žuvę su savo broliais Viešpaties akivaizdoje. 
\par 4 Kodėl jūs atvedėte Viešpaties susirinkimą į šitą dykumą, kad mes ir mūsų gyvuliai čia numirtume? 
\par 5 Kam mus išvedėte iš Egipto ir atvedėte į šitą netikusią vietą, kur negalime sėti ir neauga nei figos, nei vynuogės, nei granatai ir nėra net vandens?” 
\par 6 Mozė ir Aaronas, palikę minią, įėjo į Susitikimo palapinę ir puolė ant žemės. Jiems pasirodė Viešpaties šlovė. 
\par 7 Viešpats tarė Mozei: 
\par 8 “Imk lazdą, abu su Aaronu surinkite žmones ir jų akyse kalbėkite uolai, ir ji duos vandens. Tu išgausi jiems vandens iš uolos, ir atsigers žmonės bei jų galvijai”. 
\par 9 Mozė ėmė Viešpaties akivaizdoje buvusią lazdą, kaip Jis jam įsakė. 
\par 10 Mozė ir Aaronas surinko žmones prie uolos ir jiems kalbėjo: “Klausykite, maištininkai! Ar mudu turime iš šitos uolos išgauti jums vandens?” 
\par 11 Mozė pakėlė ranką, smogė du kartus lazda į uolą, ir pasipylė apsčiai vandens; atsigėrė žmonės ir jų gyvuliai. 
\par 12 Bet Viešpats tarė Mozei ir Aaronui: “Kadangi manimi netikėjote ir neparodėte mano šventumo izraelitų akivaizdoje, judu neįvesite šitos tautos į žemę, kurią jiems duosiu”. 
\par 13 Tai Meriba, kur izraelitai murmėjo prieš Viešpatį ir Jis parodė jiems savo šventumą. 
\par 14 Mozė išsiuntė pasiuntinius iš Kadešo pas Edomo karalių, kad pasakytų: “Taip sako tavo brolis Izraelis: ‘Tu žinai visus mūsų vargus, 
\par 15 žinai, kaip mūsų tėvai nuėjo į Egiptą ir mes ilgai ten gyvenome, kaip egiptiečiai spaudė mus ir mūsų tėvus. 
\par 16 Kai šaukėmės Viešpaties, Jis mus išklausė ir siuntė angelą, kuris mus išvedė iš Egipto. Esame sustoję Kadešo mieste, kuris yra prie pat tavo krašto sienos. 
\par 17 Maldaujame, leisk mums pereiti per tavo žemę. Mes neisime dirvomis ar vynuogynais, negersime vandens iš tavo šulinių, bet trauksime vieškeliu, nenukrypdami nei į dešinę, nei į kairę, kol pereisime tavo šalį’ ”. 
\par 18 Karalius jiems atsakė: “Neisite per mano žemę, o jei eisite, aš jus pasitiksiu su ginklu”. 
\par 19 Izraelitai sakė: “Mes eisime vieškeliu, o jei mes ir mūsų gyvuliai gersime tavo vandens, atsilyginsime, tiktai leisk pereiti per tavo žemę”. 
\par 20 Bet jis atsakė: “Neleidžiu!” Ir edomitų gausi kariuomenė išėjo prieš izraelitus. 
\par 21 Kadangi edomitai atsisakė praleisti izraelitus, tai jie pasuko kitais keliais. 
\par 22 Pakilę iš Kadešo, izraelitai atėjo prie Horo kalno. 
\par 23 Ten, prie Edomo žemės sienos, Viešpats kalbėjo Mozei ir Aaronui: 
\par 24 “Aaronas susijungs su savo tauta, nes jis neįeis į žemę, kurią pažadėjau izraelitams, dėl to, kad netikėjote mano žodžiais prie Meribos. 
\par 25 Imk Aaroną ir jo sūnų Eleazarą ir nuvesk juodu ant Horo kalno. 
\par 26 Aaronas tegul nusivelka savo drabužius ir jais apvilk jo sūnų Eleazarą. Aaronas mirs tenai”. 
\par 27 Mozė padarė, kaip Viešpats įsakė. Jie užlipo į Horo kalną, visiems susirinkusiems matant. 
\par 28 Mozė nurengė nuo Aarono drabužius ir jais apvilko jo sūnų Eleazarą. Aaronas mirė kalno viršūnėje. Mozė su Eleazaru grįžo nuo kalno. 
\par 29 Izraelitai, pamatę, kad Aaronas mirė, apraudojo jį trisdešimt dienų.



\chapter{21}

\par 1 Kanaaniečių karalius Aradas, kuris gyveno krašto pietuose, išgirdęs, kad izraelitai ateina iš Atarimo, išėjo į kovą prieš juos ir kelis paėmė į nelaisvę. 
\par 2 Izraelitai padarė įžadą Viešpačiui: “Jei atiduosi šitą tautą į mūsų rankas, sunaikinsime jos miestus”. 
\par 3 Viešpats išklausė izraelitų prašymo ir atidavė jiems kanaaniečius, kuriuos jie nugalėjo, sunaikino juos ir jų miestus. Jie praminė aną vietą Horma. 
\par 4 Nuo Horo kalno jie ėjo keliu, vedančiu Raudonosios jūros link, kad aplenktų Edomo žemę. Kelionėje tauta nerimavo, 
\par 5 ji kalbėjo prieš Dievą ir Mozę: “Kodėl mus išvedėte iš Egipto numirti dykumoje? Nėra duonos, nėra vandens, mums įgriso tas menkavertis maistas”. 
\par 6 Tada Viešpats siuntė nuodingas gyvates. Jos gėlė žmones, ir daugelis nuo jų mirė. 
\par 7 Jie kreipėsi į Mozę, sakydami: “Nusidėjome, kalbėdami prieš Viešpatį ir tave, melsk, kad pašalintų nuo mūsų gyvates”. Mozė meldėsi už tautą. 
\par 8 Viešpats sakė Mozei: “Padaryk varinę gyvatę ir iškelk ją ant stulpo; kas įgeltas į ją pažvelgs, liks gyvas”. 
\par 9 Mozė padirbdino varinę gyvatę ir iškėlė ją ant stulpo. Į ją pažvelgę, įgeltieji likdavo gyvi. 
\par 10 Izraelitai keliavo toliau ir sustojo Obote. 
\par 11 Išėję iš Oboto, ištiesė palapines Ije Abarimo dykumoje, kuri yra Moabo rytų pusėje. 
\par 12 Iš ten pasitraukę, atėjo į Zeredo slėnį. 
\par 13 Jį palikę, sustojo prie Arnono upės, tekančios dykumoje šiapus amoritų sienos. Arnono upė yra Moabo siena ir skiria moabitus nuo amoritų. 
\par 14 Todėl Viešpaties kovų knygoje pasakyta: “Kaip padarė Raudonojoje jūroje, taip padarys ir prie Arnono upės, 
\par 15 kuri pasisukus pasiekia Aro miestą ir priartėja prie moabitų sienos”. 
\par 16 Iš čia izraelitai priartėjo prie šulinio, apie kurį Viešpats kalbėjo Mozei: “Surink tautą, ir Aš duosiu jai vandens”. 
\par 17 Tada izraelitai giedojo: “Šuliny, duok vandens! Giedosime tau, 
\par 18 šuliny, kurį kunigaikščiai ir kilnieji iškasė skeptru ir lazdomis”. Iš tos dykumos jie atėjo į Mataną, 
\par 19 iš Matanos į Nahalielį, iš Nahalielio į Bamotą 
\par 20 ir iš Bamoto­į slėnį Moabo šalyje arti Pisgos kalno, kuris yra ties dykuma. 
\par 21 Izraelis nusiuntė pas amoritų karalių Sihoną pasiuntinius su prašymu: 
\par 22 “Leisk mums pereiti per tavo žemę, mes nenukrypsime į dirvas ir vynuogynus, negersime vandens iš tavo šulinių, eisime vieškeliu, kol pereisime per tavo kraštą”. 
\par 23 Bet Sihonas neleido izraelitams eiti per savo kraštą. Surinkęs savo kariuomenę, išėjo prieš juos į dykumą ir prie Jahaco kovojo su izraelitais. 
\par 24 Izraelitai sumušė jį ir užėmė Sihono kraštą nuo Arnono iki Jaboko upių ir ligi Amono krašto, kurio sienos buvo ginamos stiprios sargybos. 
\par 25 Izraelitai užėmė visus amoritų miestus ir apsigyveno juose, Hešbone ir jam priklausančiuose miesteliuose. 
\par 26 Hešbonas buvo sostinė amoritų karaliaus Sihono, kuris buvo kovojęs su buvusiu Moabo karaliumi ir užėmęs visą jam priklausančią šalį iki Arnono upės. 
\par 27 Todėl dainoje sakoma: “Ateikite į Hešboną, karaliaus Sihono miestą, kuris bus atstatytas ir sustiprintas! 
\par 28 Ugnis iš Hešbono sunaikino moabitų miestą Arą ir Arnono aukštumų viešpačius. 
\par 29 Vargas tau, Moabai! Žuvai, Kemošo tauta! Jis atidavė savo sūnus ir dukteris į nelaisvę amoritų karaliui Sihonui. 
\par 30 Jie sunaikinti nuo Hešbono iki Dibono ir iki Nofacho prie Medebos”. 
\par 31 Taip izraelitai apsigyveno amoritų žemėje. 
\par 32 Mozė išsiuntė išžvalgyti Jazerą, ir izraelitai užėmė jam priklausančius miestelius ir išvijo amoritus. 
\par 33 Jie traukė Bašano keliu; ten juos pasitiko Bašano karalius Ogas su visa savo kariuomene prie Edrėjo. 
\par 34 Viešpats kalbėjo Mozei: “Nebijok jo, nes Aš atidaviau tau jį, visą jo tautą ir žemę; padaryk su juo, kaip padarei su amoritų karaliumi Sihonu Hešbone”. 
\par 35 Izraelitai sunaikino jį, jo sūnus ir visą jo tautą taip, kad nė vieno neliko, ir užėmė šalį.



\chapter{22}

\par 1 Izraelitai keliavo ir sustojo Moabo lygumoje, anoje pusėje Jordano, ties Jerichu. 
\par 2 Ciporo sūnus Balakas matė visa, ką Izraelis padarė amoritams. 
\par 3 Moabitai labai bijojo izraelitų, nes jų buvo labai daug. 
\par 4 Moabas sakė midjaniečių vyresniesiems: “Šita tauta sunaikins mus visus, čia gyvenančius, kaip jautis sunaikina žolę iki šaknų”. O Ciporo sūnus Balakas tuo metu buvo moabitų karalius. 
\par 5 Jis siuntė pasiuntinius pas Beoro sūnų Balaamą į Petoro miestą prie upės, kad jį pakviestų, sakydami: “Iš Egipto išėjo tauta, kuri apdengė žemės paviršių ir dabar sustojo prie mano krašto sienų. 
\par 6 Ateik ir prakeik tą tautą, nes ji galingesnė už mane. Gal tada galėsiu kaip nors ją sumušti ir išvyti iš savo žemės, nes žinau, kad kurį tu laimini, tas yra palaimintas ir, kurį prakeiki, yra prakeiktas”. 
\par 7 Moabo ir Midjano vyresnieji išėjo laikydami rankose užmokestį už žyniavimą. Kai jie atėjo pas Balaamą ir jam perdavė Balako žodžius, 
\par 8 jis atsakė: “Apsinakvokite pas mane, ir aš atsakysiu, ką man pasakys Viešpats”. Ir Moabo vyresnieji pasiliko pas Balaamą. 
\par 9 Atėjo Dievas pas Balaamą ir klausė: “Kas yra šitie žmonės?” 
\par 10 Balaamas atsakė Dievui: “Moabitų karalius, Ciporo sūnus Balakas, atsiuntė juos pas mane 
\par 11 ir pranešė, kad išėjusi iš Egipto tauta apdengė žemės paviršių. Jis prašė: ‘Ateik ir prakeik ją, gal tada aš įstengsiu nugalėti juos ir išvaryti iš savo krašto’ ”. 
\par 12 Dievas atsakė Balaamui: “Neik su jais ir neprakeik tautos, nes ji yra palaiminta”. 
\par 13 Balaamas, atsikėlęs rytą, tarė Balako kunigaikščiams: “Grįžkite į savo žemę, nes Viešpats uždraudė man eiti su jumis”. 
\par 14 Ir Moabo kunigaikščiai pakilo ir, sugrįžę pas Balaką, pranešė: “Balaamas atsisakė eiti su mumis”. 
\par 15 Balakas vėl siuntė daugiau ir aukštesnės kilmės pasiuntinių. 
\par 16 Jie, atėję pas Balaamą, tarė: “Ciporo sūnus Balakas taip sako: ‘Skubiai ateik pas mane. 
\par 17 Pagerbsiu tave ir darysiu, ką įsakysi, tik ateik ir prakeik šitą tautą’ ”. 
\par 18 Balaamas atsakė Balako pasiuntiniams: “Jei Balakas duotų pilnus savo namus sidabro ir aukso, aš negalėčiau peržengti Viešpaties, mano Dievo, žodžio, ir padaryti daugiau ar mažiau. 
\par 19 Prašau, pasilikite šią naktį pas mane, kad sužinočiau, ką dar Viešpats man sakys”. 
\par 20 Naktį atėjo Dievas pas Balaamą ir jam tarė: “Jei tie žmonės tave kviečia, eik su jais, bet daryk taip, kaip tau įsakysiu”. 
\par 21 Balaamas, atsikėlęs rytą, pabalnojo savo asilę ir iškeliavo su moabitų kunigaikščiais. 
\par 22 Viešpats užsirūstino, kad Balaamas išėjo. Jiems keliaujant, Viešpaties angelas užstojo kelią Balaamui, kuris jojo ant asilės. Su juo buvo du tarnai. 
\par 23 Asilė, pamačiusi angelą stovintį su nuogu kardu, pasuko iš kelio ir ėjo lauku. Balaamas mušė ją, norėdamas grąžinti į kelią. 
\par 24 Angelas atsistojo siauroje vietoje tarp dviejų sienų, kuriomis buvo aptverti vynuogynai. 
\par 25 Pamačiusi jį, asilė šliejosi prie sienos ir prispaudė Balaamo koją. Jis vėl ją mušė. 
\par 26 Viešpaties angelas nuėjo į dar siauresnę vietą, kur nebuvo galima pasukti nei dešinėn, nei kairėn, ir atsistojo. 
\par 27 Asilė, pamačiusi stovintį angelą, sukniubo po Balaamu. Balaamas įpykęs mušė ją lazda. 
\par 28 Viešpats atvėrė asilės nasrus, ir ji kalbėjo Balaamui: “Ką aš tau padariau? Kodėl mane muši jau trečią kartą?” 
\par 29 Balaamas atsakė asilei: “Kad tyčiojiesi iš manęs. Jei turėčiau kardą, aš tave užmuščiau”. 
\par 30 Asilė atsakė Balaamui: “Argi aš ne tavo asilė, kuria visada jodinėdavai? Pasakyk, ar aš tau kada nors taip dariau?” Jis atsakė: “Niekados!” 
\par 31 Tada Viešpats atvėrė Balaamui akis. Ir jis, pamatęs Viešpaties angelą, stovintį kelyje su nuogu kardu, krito veidu į žemę. 
\par 32 Viešpaties angelas jam tarė: “Kodėl muši jau trečią kartą savo asilę? Aš atėjau tau sukliudyti, nes nepritariu tavo kelionei. 
\par 33 Jei asilė nebūtų pasukusi iš kelio tris kartus ir nebūtų pasitraukusi, būčiau tave užmušęs, o ji būtų išlikusi gyva”. 
\par 34 Balaamas tarė Viešpaties angelui: “Nusidėjau, nežinodamas, kad tu stovi prieš mane, o dabar, jei tau nepatinka, aš sugrįšiu”. 
\par 35 Viešpaties angelas tarė: “Eik su jais ir kalbėk tiktai tą, ką tau įsakysiu”. Balaamas nuėjo su Balako kunigaikščiais. 
\par 36 Tai išgirdęs, Balakas išėjo pasitikti Balaamo ligi miesto Aro, kuris yra Moabo pasienyje prie Arnono upės, 
\par 37 ir tarė Balaamui: “Aš siunčiau pasiuntinius tave pakviesti, kodėl neatėjai? Ar manai, kad aš negaliu tavęs tinkamai pagerbti?” 
\par 38 Balaamas atsakė Balakui: “Štai aš atėjau pas tave. Bet ar galiu aš ką nors pasakyti? Žodį, kurį Dievas įdės į mano lūpas, tą kalbėsiu”. 
\par 39 Balaamas su Balaku nuėjo į Kirjat Hucotą. 
\par 40 Balakas aukojo jaučius bei avis ir nusiuntė dalį aukos Balaamui ir atvykusiems su juo kunigaikščiams. 
\par 41 Rytui išaušus, Balakas nuvedė Balaamą į Baalo aukštumas, iš kur jis matė dalį izraelitų tautos.



\chapter{23}

\par 1 Tada Balaamas kalbėjo Balakui: “Pastatydink man čia septynis aukurus ir paruošk man septynis veršius ir septynis avinus”. 
\par 2 Balakas padarė, kaip Balaamas sakė. Tada juodu aukojo po veršį ir aviną ant kiekvieno aukuro. 
\par 3 Balaamas tarė Balakui: “Pastovėk prie savo deginamosios aukos, o aš pasitrauksiu ir lauksiu Viešpaties žodžio: aš tau pasakysiu, ką Jis lieps”. Jis nuėjo į nuošalią aukštumą. 
\par 4 Ten jį sutiko Dievas. Balaamas Jam kalbėjo: “Pastatydinau septynis aukurus ir aukojau ant jų po veršį ir aviną”. 
\par 5 Viešpats įdėjo žodį į Balaamo lūpas ir sakė: “Grįžk pas Balaką ir taip kalbėk”. 
\par 6 Sugrįžęs rado stovintį Balaką prie savo deginamosios aukos su visais Moabo kunigaikščiais. 
\par 7 Jis kalbėjo: “Iš Aramo mane atvedė Balakas, moabitų karalius, iš rytų šalies kalnų pašaukė mane: ‘Ateik ir prakeik Jokūbą, linkėk pikta Izraeliui’. 
\par 8 Kaipgi aš keikčiau, ko Dievas nekeikia? Kaipgi aš pasmerkčiau, ko Viešpats nesmerkia? 
\par 9 Aš matau juos nuo uolų viršūnių ir į juos žiūriu nuo kalnų. Ta tauta gyvena atskirai ir neprisideda prie kitų tautų. 
\par 10 Kas galėtų suskaityti Jokūbo dulkes ir suskaičiuoti ketvirtadalį Izraelio? O, kad galėčiau mirti teisiųjų mirtimi! O, kad mirčiau ramybėje kaip jie!” 
\par 11 Balakas sakė Balaamui: “Ką tu man padarei? Aš tave pasišaukiau, kad prakeiktum mano priešus, o tu juos palaiminai”. 
\par 12 Tas jam atsakė: “Argi aš neturiu kalbėti to, ką Viešpats įdeda į mano lūpas?” 
\par 13 Balakas sakė jam: “Eikš su manimi į kitą vietą, kur matysi izraelitų dalį, bet visų negalėsi matyti; iš ten juos prakeiksi”. 
\par 14 Kai jį nuvedė ant Pisgos kalno viršūnės, jis ten pastatydino septynis aukurus ir ant kiekvieno aukojo po veršį ir aviną. 
\par 15 Tada Balaamas tarė Balakui: “Stovėk čia prie deginamosios aukos, o aš eisiu pasitikti Viešpaties”. 
\par 16 Viešpats sutiko Balaamą ir įdėjo žodį į jo lūpas, sakydamas: “Eik vėl pas Balaką ir pasakyk tai”. 
\par 17 Sugrįžęs jis rado Balaką stovintį prie savo deginamosios aukos kartu su Moabo kunigaikščiais. Balakas klausė jo: “Ką tau kalbėjo Viešpats?” 
\par 18 Balaamas atsakė: “Balakai, klausykis ir išgirsk, Ciporo sūnau! 
\par 19 Dievas ne žmogus, kad meluotų, ir ne žmogaus sūnus, kad pakeistų savo nuomonę. Ar Jis pasakė ir nepadarys? Ar Jis kalbėjo ir neįvykdys? 
\par 20 Man įsakyta laiminti; Jis palaimino, ir aš negaliu to pakeisti. 
\par 21 Jis nerado nedorybės Jokūbe ir neįžiūrėjo neteisybės Izraelyje. Viešpats, jų Dievas, yra su jais, ir karaliaus šauksmas girdimas tarp jų. 
\par 22 Dievas juos išvedė iš Egipto, jų galybė kaip stumbro. 
\par 23 Nėra užkeikimo prieš Jokūbą nei ištarmės prieš Izraelį; ateis laikas, kai apie Izraelį sakys: ‘Štai ką padarė Dievas’. 
\par 24 Tai tauta, kuri pakils kaip liūtė ir atsistos kaip liūtas, neatsiguls, kol nesurys grobio ir neišgers užmuštųjų kraujo”. 
\par 25 Balakas tarė Balaamui: “Tu jų neprakeik, bet ir nelaimink”. 
\par 26 Balaamas atsakė Balakui: “Argi aš tau nesakiau, kad ką man Viešpats lieps, tą turėsiu daryti!” 
\par 27 Ir Balakas tarė Balaamui: “Eime, nuvesiu tave į kitą vietą, gal patiks Dievui, kad iš ten juos prakeiktum”. 
\par 28 Balakas užvedė Balaamą į Peoro kalno viršūnę, kuris yra prie dykumos. 
\par 29 Čia Balaamas sakė Balakui: “Pastatydink man čia septynis aukurus ir paruošk tiek pat veršių ir avinų aukai”. 
\par 30 Balakas padarė, kaip Balaamas įsakė. Jie aukojo po veršį ir po aviną ant kiekvieno aukuro.



\chapter{24}


\par 1 Balaamas, matydamas, kad Viešpačiui patinka laiminti Izraelį, nebėjo, kaip pirma eidavo ieškoti žyniavimo, bet atgręžė veidą į dykumą 
\par 2 ir, pakėlęs akis, pamatė izraelitus, stovyklaujančius giminėmis savo palapinėse. Dievo dvasia nužengė ant jo, 
\par 3 ir jis kalbėjo: “Kalba žmogus, Beoro sūnus Balaamas, kurio atvertos akys, 
\par 4 kuris girdi Dievo žodžius, mato Visagalio regėjimus ir krinta atvertomis akimis. 
\par 5 Kokios gražios tavo palapinės, Jokūbai, ir tavo buveinės, Izraeli! 
\par 6 Jos yra kaip besitęsią slėniai, kaip sodai paupiuose, kaip Viešpaties sodinti alavijų medžiai, lyg kedrai prie vandens! 
\par 7 Vanduo tekės upėmis, ir jų slėniuose viskas augs. Izraelitų karalius bus žymesnis už Agagą, jo karalystė bus išaukštinta. 
\par 8 Dievas išvedė jį iš Egipto, jo galybė kaip stumbro; jis suvalgys priešų tautas, sulaužys jų kaulus, pervers juos savo strėlėmis. 
\par 9 Jis atsiguls kaip liūtas, kurio niekas nedrįs pažadinti. Kas tave laimina, bus palaimintas; kas keikia, bus prakeiktas”. 
\par 10 Balakas, supykęs ant Balaamo, suplojo rankomis ir tarė: “Aš tave pasišaukiau prakeikti mano priešus, o tu juos tris kartus palaiminai. 
\par 11 Grįžk skubiai į savo kraštą! Maniau tave didžiai pagerbti, bet Viešpats atėmė iš tavęs tau skirtą pagarbą”. 
\par 12 Balaamas atsakė Balakui: “Argi aš nesakiau tavo pasiuntiniams, kuriuos atsiuntei pas mane: 
\par 13 ‘Jei Balakas man duotų pilnus savo namus sidabro ir aukso, aš negaliu peržengti Viešpaties įsakymo ir daryti gera ar bloga savo noru. Kalbėsiu tai, ką Viešpats man įsakys’. 
\par 14 Prieš grįždamas pas savo tautą, paskelbsiu tau, ką ši tauta padarys tavo tautai ateityje”. 
\par 15 Balaamas toliau kalbėjo: “Kalba žmogus, Beoro sūnus Balaamas, kurio akys atvertos, 
\par 16 kuris girdi Dievo žodžius, pažįsta Aukščiausiojo mokslą, mato Visagalio regėjimus ir krinta atvertomis akimis. 
\par 17 Aš jį matysiu, bet ne dabar, į jį žiūrėsiu, bet ne iš arti. Žvaigždė užtekės iš Jokūbo giminės, skeptras pakils Izraelyje, jis užims Moabą ir sunaikins Seto giminę. 
\par 18 Edomas ir Seyras taps priešų nuosavybe, Izraelis parodys savo jėgą. 
\par 19 Jokūbo ainiai viešpataus ir sunaikins priešų likučius”. 
\par 20 Balaamas, pažiūrėjęs į amalekiečius, kalbėjo: “Amalekas yra pirmas tarp tautų, bet jis bus sunaikintas amžiams”. 
\par 21 Pamatęs kainitus, tarė: “Stipri, saugi tavo buveinė, kaip ant uolos sukrautas lizdas. 
\par 22 Bet ir jūs, kainitai, būsite išsklaidyti, jūsų palikuonis Ašūras ištrems. 
\par 23 Ir kas išliks gyvas, kai Dievas visa tai darys? 
\par 24 Laivai atvyks iš Kitimo, pavergs ašūriečius ir sunaikins Ebero kraštą, pagaliau patys žus”. 
\par 25 Po to Balaamas sugrįžo į savo tėviškę; taip pat ir Balakas grįžo tuo keliu, kuriuo buvo atėjęs.



\chapter{25}


\par 1 Izraelitams gyvenant Šitime, tauta pradėjo paleistuvauti su Moabo dukterimis. 
\par 2 Jos kvietė izraelitus į aukojimo šventes. Ten jie valgė ir lenkėsi prieš jų dievus. 
\par 3 Izraelitai garbino Baal Peorą. Užsirūstinęs Viešpats 
\par 4 tarė Mozei: “Surink visus tautos vadus ir juos pakark saulės kaitroje, kad mano rūstybė nepaliestų Izraelio tautos”. 
\par 5 Mozė įsakė Izraelio teisėjams užmušti visus Baal Peoro garbintojus. 
\par 6 Vienas izraelitas atsivedė midjanietę į savo palapinę Mozei ir visiems izraelitams matant, tuo metu, kai jie raudojo prie Susitikimo palapinės įėjimo. 
\par 7 Tai išvydęs, kunigo Aarono sūnaus Eleazaro sūnus Finehasas pakilo iš susirinkusiųjų ir, pagriebęs ietį, 
\par 8 įėjo į palapinę paskui izraelitą, ir perdūrė juos abu kiaurai­izraelitą ir moterį per jos pilvą. Tada liovėsi maras tarp Izraelio sūnų. 
\par 9 Iš viso nuo maro mirė dvidešimt keturi tūkstančiai žmonių. 
\par 10 Viešpats tarė Mozei: 
\par 11 “Kunigo Aarono sūnaus Eleazaro sūnus Finehasas išgelbėjo izraelitus nuo mano rūstybės; jis buvo uolus dėl manęs, kad Aš nesunaikinčiau izraelitų, apimtas pavydo. 
\par 12 Todėl sakyk jam, kad Aš darau su juo taikos sandorą: 
\par 13 jam ir jo palikuonims priklausys kunigystė kaip amžina sandora, nes jis buvo uolus dėl Dievo ir sutaikino izraelitus”. 
\par 14 Izraelitas, kurį nužudė kartu su midjaniete, buvo Saluvo sūnus Zimris, Simeono giminės kunigaikštis, 
\par 15 o nužudytoji midjanietė, vardu Kozbė, buvo Midjano giminės kunigaikščio Cūro duktė. 
\par 16 Viešpats kalbėjo Mozei: 
\par 17 “Išžudykite midjaniečius, 
\par 18 nes jie pasielgė su jumis klastingai, suvedžiodami jus Baalo garbinimu ir midjaniečių kunigaikščio dukterimi Kozbe, savo seserimi, kuri buvo nužudyta dėl Peoro maro dieną”.



\chapter{26}

\par 1 Viešpats tarė Mozei ir Aarono sūnui kunigui Eleazarui: 
\par 2 “Suskaičiuokite visus izraelitus vyrus, dvidešimties metų ir vyresnius pagal jų gimines, tinkamus kariuomenei”. 
\par 3 Mozė ir kunigas Eleazaras Moabo lygumoje prie Jordano, ties Jerichu, kalbėjo tiems, kurie buvo 
\par 4 dvidešimties metų ir vyresni, kaip Viešpats buvo liepęs. Iš Egipto išėjusieji buvo: 
\par 5 Rubenas, Izraelio pirmagimis. Jo sūnūs buvo Henochas, iš kurio kilo henochai, Paluvas, iš kurio­paluvai, 
\par 6 Hecronas, iš jo­hecronai ir Karmis, iš jo­karmiai. 
\par 7 Rubeno giminės buvo keturiasdešimt trys tūkstančiai septyni šimtai trisdešimt. 
\par 8 Paluvo sūnus buvo Eliabas; 
\par 9 jo sūnūs Nemuelis, Datanas ir Abiramas; Datanas ir Abiramas prisidėjo prie Koracho maišto, kuris buvo sukeltas prieš Mozę ir Aaroną; tai buvo maištas prieš Viešpatį. 
\par 10 Atvėrusi žemė prarijo juos kartu su Korachu. Du šimtai penkiasdešimt žmonių buvo sunaikinti ugnimi. Tai buvo ženklas izraelitams. 
\par 11 Tačiau Koracho sūnūs nežuvo. 
\par 12 Simeono sūnų šeimos: Nemuelio­nemuelitai, Jamino­jaminai, Jachino­jachinai, 
\par 13 Zeracho­zerachai, Sauliaus­ sauliai. 
\par 14 Simeono giminės buvo dvidešimt du tūkstančiai du šimtai. 
\par 15 Gado sūnų šeimos: Cefono, iš jo­cefonai, Hagio­hagiai, Šūnio­šūniai, 
\par 16 Oznio­ozniai, Erio­eriai, 
\par 17 Arodo­arodai, Arelio­arelitai. 
\par 18 Gado giminės buvo keturiasdešimt tūkstančių penki šimtai. 
\par 19 Judo sūnūs buvo Eras ir Onanas. Jie mirė Kanaano žemėje. 
\par 20 Kitų Judo sūnų šeimos: Šelos­šelai, Pereco­perecai, Zeracho­zerachai. 
\par 21 Pereco giminės sūnų: Esromo­ esromitai ir Hamulo­hamulai. 
\par 22 Judo giminės buvo septyniasdešimt šeši tūkstančiai penki šimtai. 
\par 23 Isacharo sūnų: Tolos­tolai, Pūvos­pūvai, 
\par 24 Jašubo­jašubai, Šimrono­šimronai. 
\par 25 Isacharo giminės buvo šešiasdešimt keturi tūkstančiai trys šimtai. 
\par 26 Zabulono sūnų: Seredo­seredai, Elono­elonai, Jachleelio­ jachlelitai. 
\par 27 Zabulono giminės buvo šešiasdešimt tūkstančių penki šimtai. 
\par 28 Juozapo sūnūs Manasas ir Efraimas pagal savo šeimas: 
\par 29 Manasui gimė Machyras, iš kurio kilo machyrai, iš Machyro sūnaus Gileado­gileadiečiai. 
\par 30 Šitie yra Gileado sūnūs, iš kurių kilo: iš Jezero­jezerai, iš Heleko­helekai, 
\par 31 iš Asrielio­asrielitai, iš Šechemo­šechemai, 
\par 32 iš Šemidos­šemidai ir iš Hefero­heferai. 
\par 33 O Hefero sūnus Celofhadas neturėjo sūnų, tik dukteris, kurių vardai yra: Machla, Noja, Hogla, Milka ir Tirca. 
\par 34 Manaso giminės buvo penkiasdešimt du tūkstančiai septyni šimtai. 
\par 35 Efraimo sūnų šeimos: Šutelacho­šutelachai, Bechero­becherai, Tahano­tahanai. 
\par 36 Šutelacho sūnaus Erano­eranai. 
\par 37 Efraimo giminės buvo trisdešimt du tūkstančiai penki šimtai. Šitos yra Juozapo giminės šeimos. 
\par 38 Benjamino sūnų šeimos: Belos­belai, Ašbelio­ašbelitai, Ahiramo­ahiramai, 
\par 39 Šefufamo­šefufamai, Hufamo­hufamai. 
\par 40 Belos sūnūs buvo Ardas ir Naamanas. Iš Ardo­ardai, iš Naamano­naamanai. 
\par 41 Benjamino giminės buvo keturiasdešimt penki tūkstančiai šeši šimtai. 
\par 42 Dano sūnų šeimos: Šuhamo­ šuhamai. Tai Dano giminės šeimos. 
\par 43 Šuhamų buvo šešiasdešimt keturi tūkstančiai keturi šimtai. 
\par 44 Ašero sūnų šeimos: Imnos­imnai, Išvio­išviai, Berijos­berijai. 
\par 45 Iš Berijos sūnų: Hebero­heberai ir Malkielio­malkielitai. 
\par 46 Ašero dukters vardas buvo Seracha. 
\par 47 Ašero giminės buvo penkiasdešimt trys tūkstančiai keturi šimtai. 
\par 48 Neftalio sūnų šeimos: Jachceelio­jachceelitai, Gūnio­gūniai, 
\par 49 Jecero­jecerai, Šilemo­šilemai. 
\par 50 Neftalio giminės buvo keturiasdešimt penki tūkstančiai keturi šimtai. 
\par 51 Izraelitų, kurie buvo surašyti, buvo šeši šimtai vienas tūkstantis septyni šimtai trisdešimt. 
\par 52 Viešpats kalbėjo Mozei: 
\par 53 “Jiems bus išdalinta žemė paveldėjimui pagal vardų skaičių. 
\par 54 Didesnėms šeimoms duosi didesnę dalį, o mažesnėms­mažesnę, kiekvienam bus duotas paveldėjimas pagal tuos, kurie buvo suskaičiuoti. 
\par 55 Žemė bus padalyta tarp giminių burtų keliu. 
\par 56 Burtų keliu bus padalinta nuosavybė tiems, kurių yra daug, ir tiems, kurių yra mažai. 
\par 57 Levio giminės sūnų šeimos: Geršono­geršonų šeima, Kehato­kehatų šeima, Merario­merarių šeima. 
\par 58 Šitos yra Levio šeimos: libniai, hebronai, machliai, mušiai ir koriai. Kehato sūnaus Amramo 
\par 59 žmona, Levio duktė Jochebeda, gimusi Egipte, pagimdė Aaroną ir Mozę bei jų seserį Mirjamą. 
\par 60 Aarono sūnūs buvo: Nadabas, Abihuvas, Eleazaras ir Itamaras. 
\par 61 Nadabas ir Abihuvas mirė, aukodami svetimą ugnį Viešpačiui. 
\par 62 Visų Levio giminės suskaitytų vyrų, vieno mėnesio amžiaus ir vyresnių, buvo dvidešimt trys tūkstančiai. Jie nebuvo suskaityti kartu su izraelitais, nes jiems nebuvo duota dalis tarp Izraelio vaikų. 
\par 63 Tai izraelitai, kuriuos suskaičiavo Mozė ir kunigas Eleazaras Moabo lygumoje prie Jordano, ties Jerichu. 
\par 64 Tarp jų nebuvo nė vieno iš tų, kurie buvo anksčiau Mozės ir Aarono suskaičiuoti Sinajaus dykumoje, 
\par 65 nes Viešpats buvo pasakęs, kad jie visi mirs dykumoje. Nė vienas neišliko, išskyrus Jefunės sūnų Kalebą ir Nūno sūnų Jozuę.



\chapter{27}

\par 1 Machla, Noja, Hogla, Milka ir Tirca, dukterys Celofhado, sūnaus Hefero, sūnaus Gileado, sūnaus Machyro, sūnaus Manaso, sūnaus Juozapo, atėjo 
\par 2 prie Susitikimo palapinės įėjimo, kur stovėjo Mozė, kunigas Eleazaras ir visi tautos kunigaikščiai, ir tarė: 
\par 3 “Mūsų tėvas mirė dykumoje. Jis nedalyvavo maište, kuris buvo Koracho sukeltas prieš Viešpatį, bet mirė už savo nusikaltimus. Jis neturėjo sūnų. 
\par 4 Kodėl jo vardas turėtų išnykti iš jo šeimos? Ar dėl to, kad jis neturėjo sūnaus? Duokite mums paveldėjimą tarp mūsų tėvo brolių”. 
\par 5 Mozė kreipėsi tuo reikalu į Viešpatį. 
\par 6 Viešpats atsakė Mozei: 
\par 7 “Celofhado dukterys kalba teisingai. Duok joms paveldėjimą tarp jų tėvo brolių, tepaveldi jos jo nuosavybę. 
\par 8 Izraelio sūnums įsakyk: ‘Jei kas miršta nepalikdamas sūnaus, jo nuosavybę tepaveldi jo duktė. 
\par 9 Jei neturės dukters, įpėdiniais bus mirusiojo broliai. 
\par 10 O jei nebus nė brolių, jo nuosavybę paveldės jo tėvo broliai. 
\par 11 Jeigu neturės nė tėvo brolių, paveldas atiteks tiems, kurie yra jo artimiausi giminės’ ”. Tai amžinas įstatymas izraelitams, Viešpaties duotas Mozei. 
\par 12 Viešpats kalbėjo Mozei: “Užlipk į Abarimo kalną ir iš ten pažiūrėk į kraštą, kurį duosiu izraelitams. 
\par 13 Jį pamatęs, tu susijungsi su savo tauta, kaip susijungė tavo brolis Aaronas, 
\par 14 nes judu nepaklausėte manęs Cino dykumoje tautos prieštaravimo metu, kad būtų parodytas mano šventumas jų akivaizdoje prie Meribos vandenų šaltinio, Cino dykumoje Kadeše”. 
\par 15 Mozė atsakė Viešpačiui: 
\par 16 “Viešpatie, visų gyvųjų Dieve, paskirk žmogų šitai tautai vadovauti, 
\par 17 kuris eitų su jais, juos išvestų ir įvestų, kad Viešpaties susirinkimas nebūtų kaip avys be piemens”. 
\par 18 Viešpats atsakė Mozei: “Imk Nūno sūnų Jozuę, kuriame yra mano dvasia, ir ant jo uždėk savo ranką. 
\par 19 Jis teatsistoja kunigo Eleazaro ir visos tautos akivaizdoje, o tu duok jam paliepimus 
\par 20 ir suteik jam savo garbės, kad jo klausytų visi izraelitai. 
\par 21 Tuo atveju, kai reikės pagalbos, tegul kunigas Eleazaras kreipiasi į Viešpatį patarimo. Tuo būdu Jozuė galės vadovauti izraelitams”. 
\par 22 Mozė padarė, kaip Viešpats buvo įsakęs: pastatė Jozuę kunigo Eleazaro ir visos tautos akivaizdoje, 
\par 23 uždėjo jam ant galvos rankas ir suteikė jam valdžią, kaip Viešpats liepė.



\chapter{28}

\par 1 Viešpats kalbėjo Mozei: 
\par 2 “Įsakyk izraelitams skirtu laiku aukoti duonos ir deginamąsias aukas kaip malonų kvapą. 
\par 3 Šitos aukos bus: kasdien du sveiki metiniai avinėliai nuolatinei deginamajai aukai; 
\par 4 vieną aukosite rytą, o antrą­vakare; 
\par 5 ir duonos aukai dešimtą efos dalį smulkių miltų, sumaišytų su ketvirtadaliu hino tyriausio aliejaus. 
\par 6 Tai bus nuolatinė deginamoji auka, kaip buvo įsakyta Sinajaus kalne, malonus kvapas Viešpačiui. 
\par 7 Ir geriamoji auka bus ketvirtadalis hino vyno; šventykloje išliesite Viešpačiui vyną kaip geriamąją auką. 
\par 8 Antrąjį avinėlį aukosite vakare, kaip ir rytmetinę auką, su duonos ir geriamąja auka, kad būtų malonus kvapas. 
\par 9 Sabato dieną aukosite du sveikus metinius avinėlius, o duonos aukai­ dvi dešimtąsias efos smulkių miltų, sumaišytų su aliejumi, taip pat ir geriamąją auką. 
\par 10 Tai bus nuolatinė sabato deginamoji auka, priedas prie kasdieninių deginamųjų ir geriamųjų aukų. 
\par 11 Kiekvieno mėnesio pradžioje privalote aukoti deginamąją auką Viešpačiui: du sveikus veršius, vieną aviną ir septynis metinius avinėlius. 
\par 12 Duonos aukai: prie kiekvieno veršio tris dešimtąsias efos smulkių miltų, sumaišytų su aliejumi; prie kiekvieno avino dvi dešimtąsias efos smulkių miltų, sumaišytų su aliejumi, 
\par 13 ir dešimtą dalį smulkių miltų, sumaišytų su aliejumi, duonos aukai prie kiekvieno avinėlio. Tai bus malonaus kvapo deginamoji auka Viešpačiui. 
\par 14 Geriamosios vyno aukos prie kiekvienos deginamosios aukos bus: pusė hino prie kiekvieno veršio, trečdalis­ prie avino, ketvirtadalis­prie kiekvieno avinėlio. Tai kiekvieno mėnesio geriamoji auka. 
\par 15 Taip pat aukosite Viešpačiui ožį aukai už nuodėmę, neskaičiuojant nuolatinių deginamųjų ir geriamųjų aukų. 
\par 16 Pirmo mėnesio keturioliktą dieną yra Viešpaties Pascha, 
\par 17 o penkioliktą dieną­iškilmės. Neraugintą duoną valgysite septynias dienas. 
\par 18 Pirmoji diena yra izraelitams šventa diena­tą dieną nedirbsite jokio darbo, 
\par 19 aukosite deginamajai aukai du sveikus veršius, vieną aviną, septynis metinius avinėlius 
\par 20 ir duonos aukai smulkių miltų, sumaišytų su aliejumi, tris dešimtąsias efos prie veršio, dvi dešimtąsias efos prie avino 
\par 21 ir vieną dešimtąją efos prie kiekvieno avinėlio, 
\par 22 taip pat vieną ožį aukai už nuodėmę, kad būtumėte sutaikinti, 
\par 23 neskaičiuojant rytmetinės deginamosios aukos, kuri nuolat aukojama. 
\par 24 Taip darykite septynias dienas, kad būtų malonus kvapas Viešpačiui. 
\par 25 Septintoji diena bus jums šventa: tą dieną nedirbkite jokio darbo. 
\par 26 Pirmųjų vaisių šventės dieną aukosite Viešpačiui naujo derliaus aukas. Ta diena bus šventa ir iškilminga, tada nedirbsite jokio darbo. 
\par 27 Deginamajai aukai aukosite du veršius, vieną aviną bei septynis metinius avinėlius; 
\par 28 duonos aukai­su aliejumi sumaišytų smulkių miltų tris dešimtąsias efos prie kiekvieno veršio, dvi­prie avino 
\par 29 ir dešimtą dalį­prie kiekvieno avinėlio; 
\par 30 taip pat ožį, kuris aukojamas sutaikinimui. 
\par 31 Šias aukas aukosite neskaitant nuolatinių deginamųjų aukų ir kartu su jomis aukojamų duonos ir geriamųjų aukų”.



\chapter{29}


\par 1 “Pirmoji septinto mėnesio diena bus jums iškilminga ir šventa. Tą dieną nedirbsite jokio darbo­tai bus trimitavimo diena. 
\par 2 Aukosite deginamajai aukai, kaip malonų kvapą Viešpačiui, sveikus gyvulius: vieną veršį, vieną aviną ir septynis metinius avinėlius; 
\par 3 duonos aukai­tris dešimtąsias efos smulkių miltų, sumaišytų su aliejumi, prie kiekvieno veršio, dvi dešimtąsias efos prie avino, 
\par 4 vieną dešimtą dalį efos prie kiekvieno avinėlio 
\par 5 ir ožį aukai už nuodėmę, kad būtumėte sutaikinti, 
\par 6 neskaičiuojant mėnesinės deginamosios bei duonos aukos ir kasdieninės deginamosios, duonos ir geriamosios aukos, kurias aukosite pagal jų nuostatus, kad būtų malonus kvapas Viešpačiui. 
\par 7 Taip pat septinto mėnesio dešimtoji diena bus jums šventa ir iškilminga. Tą dieną varginsite savo sielas ir nedirbsite jokio darbo. 
\par 8 Deginamajai Viešpaties aukai kaip malonų kvapą aukosite sveikus gyvulius: jauną veršį, aviną ir septynis metinius avinėlius; 
\par 9 jų duonos aukai tris dešimtąsias efos smulkių miltų, sumaišytų su aliejumi, prie kiekvieno veršio, dvi dešimtąsias efos prie kiekvieno avino, 
\par 10 dešimtą dalį efos prie kiekvieno avinėlio 
\par 11 ir ožį aukai už nuodėmę, neskaitant aukos už nuodėmę sutaikinimui ir nuolatinės deginamosios aukos su duonos ir geriamosiomis aukomis. 
\par 12 Septinto mėnesio penkioliktoji diena bus jums šventa ir iškilminga, nedirbsite tą dieną jokio darbo. Septynias dienas švęsite Viešpaties garbei. 
\par 13 Aukosite deginamąją auką kaip malonų kvapą Viešpačiui: trylika sveikų jaunų veršių, du avinus, keturiolika metinių avinėlių; 
\par 14 taip pat duonos auką: smulkių, su aliejumi sumaišytų miltų po tris dešimtąsias efos prie kiekvieno veršio, kurių bus trylika, po dvi dešimtąsias efos prie kiekvieno avino, jų bus du, 
\par 15 dešimtą dalį efos prie kiekvieno avinėlio, kurių bus keturiolika, 
\par 16 ir ožį aukai už nuodėmę, be to, nuolatines deginamąsias, duonos ir geriamąsias aukas. 
\par 17 Antrą dieną aukosite: dvylika jaunų veršių, du avinus, keturiolika metinių avinėlių. 
\par 18 Prie kiekvieno veršio, avino ir avinėlio aukosite jų duonos ir geriamąją auką, kaip įsakyta, 
\par 19 ir ožį aukai už nuodėmę, neskaičiuojant nuolatinės deginamosios, duonos ir geriamosios aukos. 
\par 20 Trečią dieną aukosite: vienuolika veršių, du avinus, keturiolika metinių avinėlių 
\par 21 ir, kaip įsakyta, prie kiekvieno veršio, avino ir avinėlio duonos ir geriamąsias aukas. 
\par 22 ir ožį aukai už nuodėmę, neskaičiuojant nuolatinės deginamosios, duonos ir geriamosios aukos. 
\par 23 Ketvirtą dieną aukosite: dešimt veršių, du avinus, keturiolika metinių avinėlių 
\par 24 ir prie kiekvieno jų duonos bei geriamąsias aukas, kaip įsakyta, 
\par 25 ir ožį aukai už nuodėmę, neskaičiuojant nuolatinės deginamosios, duonos bei geriamosios aukų. 
\par 26 Penktą dieną aukosite: devynis veršius, du avinus, keturiolika metinių avinėlių, 
\par 27 prie kiekvieno jų duonos bei geriamąsias aukas, kaip įsakyta, 
\par 28 ir ožį aukai už nuodėmę, neskaičiuojant nuolatinės deginamosios, duonos bei geriamosios aukų. 
\par 29 Šeštą dieną aukosite: aštuonis veršius, du avinus ir keturiolika metinių avinėlių, 
\par 30 prie kiekvieno jų duonos bei geriamąsias aukas, kaip įsakyta, 
\par 31 ir ožį aukai už nuodėmę, neskaičiuojant nuolatinės deginamosios, duonos bei geriamosios aukų. 
\par 32 Septintą dieną aukosite: septynis veršius, du avinus ir keturiolika metinių avinėlių, 
\par 33 prie kiekvieno jų duonos ir geriamąsias aukas, kaip įsakyta, 
\par 34 bei ožį aukai už nuodėmę, neskaičiuojant nuolatinės deginamosios, duonos ir geriamosios aukų. 
\par 35 Aštunta diena yra iškilminga­nedirbsite tą dieną jokio darbo 
\par 36 ir aukosite deginamąją auką kaip malonų kvapą Viešpačiui: veršį, aviną, septynis metinius avinėlius, 
\par 37 prie kiekvieno jų duonos bei geriamąsias aukas, kaip įsakyta, 
\par 38 ir ožį aukai už nuodėmę, neskaičiuojant nuolatinės deginamosios, duonos ir geriamosios aukų. Visi aukojamieji gyvuliai turi būti sveiki. 
\par 39 Tai aukosite švenčių metu kaip priedą prie jūsų įžadų ir laisvos valios aukų, o taip pat deginamųjų, duonos, geriamųjų bei padėkos aukų”. 
\par 40 Mozė perdavė izraelitams visa, ką jam Viešpats kalbėjo.



\chapter{30}


\par 1 Ir Mozė kalbėjo izraelitų giminių vadams: “Štai ką Viešpats įsakė: 
\par 2 ‘Jei kuris vyras duoda Viešpačiui įžadą ar pasižada su priesaika, negalės savo žodžio laužyti, turės įvykdyti, ką pažadėjo. 
\par 3 Jei įžadą Viešpačiui duotų ir save priesaika suvaržytų jauna moteris, kuri gyvena savo tėvo namuose, 
\par 4 ir jei tėvas, sužinojęs apie įžadą ir priesaiką, tylėtų, ji turi laikytis įžado ir privalo įvykdyti, ką pažadėjo ir prisiekė. 
\par 5 O jei tėvas išgirdęs uždraustų tą pačią dieną, jos įžadas ir priesaika bus panaikinta, nes tėvas uždraudė, ir Viešpats jai atleis. 
\par 6 Jei ištekėtų, davusi įžadą ar priesaiką, 
\par 7 ir jos vyras sužinojęs neuždraustų jai to daryti tą pačią dieną, ji laikysis įžado ir jį įvykdys. 
\par 8 O jei jis išgirdęs tuojau uždraustų ir tuo būdu panaikintų jos įžadus, Viešpats jai atleis. 
\par 9 Našlė arba išsiskyrusi privalo įvykdyti, ką pažadėjo. 
\par 10 Jei žmona, gyvendama su vyru, padarytų įžadą arba prisiektų 
\par 11 ir jei jos vyras išgirdęs tylėtų ir neprieštarautų, ji vykdys savo įžadą. 
\par 12 O jei išgirdęs tuojau uždraustų, nereikės vykdyti pažado, nes vyras uždraudė, ir Viešpats jai atleis. 
\par 13 Kiekvieną jos įžadą ir kiekvieną priesaiką varginti savo sielą vyras gali patvirtinti arba panaikinti. 
\par 14 O jei vyras išgirdęs tylėtų ir atidėtų sprendimą kitai dienai, ji vykdys, ką pažadėjo, nes jis tylėdamas patvirtino jos įžadus. 
\par 15 Jei žinodamas vėliau prieštarautų, jis bus atsakingas už jos kaltę’ ”. 
\par 16 Tai yra Viešpaties duoti Mozei įsakymai, kurių turi laikytis vyras ir žmona, tėvas ir duktė, kuri yra netekėjusi ir gyvena tėvo namuose.



\chapter{31}

\par 1 Viešpats tarė Mozei: 
\par 2 “Atkeršyk midjaniečiams už izraelitams padarytas skriaudas ir tada susijungsi su savo tauta”. 
\par 3 Mozė įsakė apginkluoti vyrus, kurie vykdys Viešpaties kerštą midjaniečiams, 
\par 4 ir pasiųsti po tūkstantį vyrų iš kiekvienos giminės. 
\par 5 Buvo atskirta po tūkstantį vyrų iš kiekvienos giminės, iš viso dvylika tūkstančių, pasirengusių kovai. 
\par 6 Mozė juos išsiuntė su kunigo Eleazaro sūnumi Finehasu, kuriam pavedė šventus daiktus ir trimitus. 
\par 7 Jie nugalėjo midjaniečius ir išžudė visus jų vyrus, kaip Viešpats buvo įsakęs Mozei, 
\par 8 taip pat ir jų karalius: Evį, Rekemą, Cūrą, Hūrą ir Rebą. Užmušė kardu ir Beoro sūnų Balaamą. 
\par 9 Jie paėmė į nelaisvę jų moteris, vaikus, visus galvijus ir visą midjaniečių lobį, 
\par 10 sudegino jų miestus ir stovyklas. 
\par 11 Belaisvius, gyvulius ir visą karo grobį 
\par 12 izraelitai atgabeno pas Mozę ir kunigą Eleazarą į izraelitų stovyklą Moabo lygumose prie Jordano, ties Jerichu. 
\par 13 Mozė, kunigas Eleazaras ir visi izraelitų kunigaikščiai išėjo jų pasitikti už stovyklos. 
\par 14 Užsirūstinęs ant kariuomenės vadų, tūkstantininkų ir šimtininkų, kurie grįžo iš kovos lauko, 
\par 15 Mozė tarė: “Kodėl palikote gyvas moteris? 
\par 16 Argi ne jos suvedžiojo izraelitus, patariant Balaamui, nusikalsti Viešpačiui Baal Peore, dėl ko tauta buvo nubausta maru. 
\par 17 Nužudykite visus berniukus bei moteris, kurios gyveno su vyrais! 
\par 18 Mergaites, kurios nėra pažinusios vyro, pasilaikykite sau. 
\par 19 Pasilikite už stovyklos septynias dienas. Kas užmušė žmogų ar užmuštąjį palietė, apsivalykite patys ir jūsų belaisviai trečią ir septintą dieną. 
\par 20 Reikia apvalyti ir visą grobį: ar tai būtų drabužiai, ar indai, ar bet koks daiktas, padarytas iš odos, ožkų vilnų ar iš medžio”. 
\par 21 Kunigas Eleazaras kalbėjo kovoje dalyvavusiems kariams: “Štai Viešpaties nurodymas: 
\par 22 auksą, sidabrą, varį, geležį, šviną, ciną 
\par 23 ir visa, kas nedega liepsnoje, apvalykite ugnimi ir apvalomuoju vandeniu, kas dega ugnyje, apvalykite vandeniu. 
\par 24 Septintą dieną išplaukite savo drabužius ir, taip apsivalę, įeikite į stovyklą”. 
\par 25 Viešpats kalbėjo Mozei: 
\par 26 “Tu, kunigas Eleazaras ir tautos vyresnieji suskaičiuokite visą grobį­ žmones ir galvijus. 
\par 27 Padalykite grobį į dvi lygias dalis kovojusiems ir kitiems izraelitams. 
\par 28 Iš tų, kurie kovojo kare ir gavo karo grobio, atskirk Viešpaties daliai po vieną iš penkių šimtų žmonių, taip pat iš galvijų, asilų ir avių; 
\par 29 viską atiduok kunigui Eleazarui, nes tai yra auka Viešpačiui. 
\par 30 O iš nekariavusių izraelitų grobio imk kiekvieną penkiasdešimtą žmogų, taip pat galviją, asilą, avį ir kitų gyvulių ir atiduok levitams, kurie tarnauja Viešpačiui”. 
\par 31 Mozė ir Eleazaras padarė, kaip Viešpats įsakė. 
\par 32 Karo grobis buvo: šeši šimtai septyniasdešimt penki tūkstančiai avių, 
\par 33 septyniasdešimt du tūkstančiai galvijų, 
\par 34 šešiasdešimt vienas tūkstantis asilų, 
\par 35 trisdešimt du tūkstančiai nekaltų mergaičių. 
\par 36 Pusė buvo atiduota kare dalyvavusiems: trys šimtai trisdešimt septyni tūkstančiai penki šimtai avių; 
\par 37 iš jų Viešpačiui atskirta šeši šimtai septyniasdešimt penkios avys. 
\par 38 Trisdešimt šeši tūkstančiai galvijų; iš jų Viešpačiui atskirta septyniasdešimt du; 
\par 39 trisdešimt tūkstančių penki šimtai asilų, iš kurių atskirta Viešpačiui šešiasdešimt vienas asilas. 
\par 40 Šešiolika tūkstančių žmonių, iš kurių Viešpačiui teko trisdešimt du. 
\par 41 Mozė atidavė, kaip buvo liepta, Viešpaties aukos dalį Eleazarui. 
\par 42 Izraelitams, nedalyvavusiems kare, buvo duota: 
\par 43 trys šimtai trisdešimt septyni tūkstančiai penki šimtai avių, 
\par 44 trisdešimt šeši tūkstančiai galvijų, 
\par 45 trisdešimt tūkstančių penki šimtai asilų, 
\par 46 šešiolika tūkstančių belaisvių. 
\par 47 Mozė ėmė kas penkiasdešimtą gyvulį ir belaisvį ir atidavė levitams, tarnaujantiems prie Viešpaties palapinės, kaip Viešpats buvo įsakęs. 
\par 48 Kariuomenės vadai, tūkstantininkai ir šimtininkai, priėję prie Mozės, kalbėjo: 
\par 49 “Mes, tavo tarnai, suskaičiavome kareivius, kuriuos turime savo žinioje, ir nė vieno nepasigedome. 
\par 50 Todėl kiekvienas atnešame kaip dovaną Viešpačiui visus auksinius papuošalus, kuriuos paėmėme karo grobiu: grandinėles, apyrankes, žiedus, auskarus ir kaklo papuošalus, kad sutaikintum mus su Viešpačiu”. 
\par 51 Mozė ir kunigas Eleazaras priėmė auką Viešpačiui iš tūkstantininkų ir šimtininkų, visus atneštus auksinius papuošalus, 
\par 52 kurie svėrė šešiolika tūkstančių septynis šimtus penkiasdešimt šekelių. 
\par 53 Eiliniai kareiviai grobį pasilaikė sau. 
\par 54 Paimtą auksą iš tūkstantininkų ir šimtininkų Mozė ir kunigas Eleazaras įnešė į Susitikimo palapinę, kad Viešpats atmintų izraelitus.



\chapter{32}


\par 1 Rubeno ir Gado giminės turėjo daug gyvulių. Jie matė Jazero ir Gileado žemes, tinkamas gyvuliams auginti. 
\par 2 Atėję pas Mozę, kunigą Eleazarą ir izraelitų kunigaikščius, jie tarė: 
\par 3 “Ataroto, Dibono, Jazero, Nimros, Hešbono, Elealės, Sebamo, Nebojo ir Beono žemės, 
\par 4 kurios, Viešpačiui padedant, buvo užimtos izraelitų, labai tinka gyvuliams auginti, o mes, tavo tarnai, turime gyvulių. 
\par 5 Taigi jei radome malonę tavo akyse, prašome, kad mums, tavo tarnams, atiduotum ją nuosavybėn ir nevestum mūsų per Jordaną”. 
\par 6 Mozė jiems atsakė: “Argi, kai jūsų broliai kariaus, jūs čia sėdėsite? 
\par 7 Kodėl atkalbinėjate izraelitus, kad jie neitų į žemę, kurią Viešpats jiems atidavė? 
\par 8 Taip elgėsi jūsų tėvai, kai siunčiau iš Kadeš Barnėjos apžiūrėti kraštą. 
\par 9 Jie, nuėję iki Eškolo slėnio ir apžiūrėję visą kraštą, įbaugino izraelitus, kad neitų į šalį, kurią Viešpats jiems pažadėjo. 
\par 10 Tada Viešpats užsirūstinęs prisiekė: 
\par 11 ‘Šitie žmonės, kurie išėjo iš Egipto, dvidešimties metų ir vyresni, neišvys žemės, kurią pažadėjau Abraomui, Izaokui ir Jokūbui, nes jie nesekė manimi iki galo, 
\par 12 išskyrus kenazą Jefunės sūnų Kalebą ir Nūno sūnų Jozuę, kurie iki galo sekė Viešpačiu’. 
\par 13 Viešpats, užsirūstinęs ant izraelitų, leido jiems klaidžioti dykumoje, kol išmirė visa karta, kuri buvo nusikaltusi Viešpačiui. 
\par 14 Dabar jūs stojate savo tėvų vieton, kad dar labiau padidintumėte Viešpaties įtūžimą prieš Izraelį. 
\par 15 Jei nenorite Jo klausyti, Jis vėl paliks tautą dykumoje, ir jūs pražudysite visą šią tautą”. 
\par 16 Jie, prisiartinę prie Mozės, tarė: “Pastatysime tvartus galvijams, taip pat mūsų vaikams miestus; 
\par 17 mes gi patys apsiginklavę eisime į kovą izraelitų priekyje, kol įvesime juos į jų žemes. Tuo tarpu mūsų vaikai gyvens apmūrytuose miestuose dėl šio krašto gyventojų. 
\par 18 Negrįšime į savo namus, kol visi izraelitai gaus savo dalį, 
\par 19 nieko nereikalausime anoje pusėje Jordano, nes mūsų dalis yra šiapus Jordano”. 
\par 20 Mozė jiems atsakė: “Jei taip darysite ir stosite į kovą Viešpaties akivaizdoje, 
\par 21 ir visi karui tinkami vyrai apsiginklavę pereis Jordaną, iki priešas bus nugalėtas 
\par 22 ir visa žemė bus paimta, tuomet nenusikalsite nei Viešpačiui, nei Izraeliui ir ši žemė bus jūsų Viešpaties akivaizdoje. 
\par 23 Jei nedarysite, ką sakote, tai nusidėsite Viešpačiui ir žinokite, kad būsite nubausti. 
\par 24 Taigi statykite miestus savo vaikams, tvartus avims bei galvijams ir įvykdykite, ką pažadėjote”. 
\par 25 Gaditai ir rubenai atsakė Mozei: “Mes, tavo tarnai, darysime, ką mūsų valdovas liepia. 
\par 26 Savo vaikus ir moteris, avis ir galvijus paliksime Gileado miestuose, 
\par 27 mes gi, tavo tarnai, visi apsiginklavę trauksime į karą, kaip tu, valdove, sakai”. 
\par 28 Mozė įsakė kunigui Eleazarui, Nūno sūnui Jozuei ir izraelitų giminių vadams: 
\par 29 “Jei gaditai ir rubenai kartu su jumis pereis Jordaną su ginklais ir žemė bus paimta, duokite jiems paveldėti Gileadą. 
\par 30 Jei nenorės eiti kartu su jumis apsiginklavę į Kanaano žemę, tegul pasilieka tarp jūsų”. 
\par 31 Gaditai ir rubenai atsakė: “Kaip Viešpats įsakė savo tarnams, taip mes darysime. 
\par 32 Mes eisime apsiginklavę Viešpaties akivaizdoje įs Kanaano žemę, kad gautume dalį šiapus Jordano”. 
\par 33 Mozė davė gaditams ir rubenams bei pusei Juozapo sūnaus Manaso giminės amoritų karaliaus Sihono ir Bašano karaliaus Ogo karalysčių žemes su jų miestais. 
\par 34 Gaditai atstatė sutvirtintus Dibono, Ataroto, Aroero, 
\par 35 Atroto, Šofano, Jazero, Jogbohos, 
\par 36 Bet Nimros, Bet Harano miestus ir pastatė savo galvijams tvartus. 
\par 37 Rubenai atstatė Hešboną, Elealę, Kirjataimą, 
\par 38 Neboją, Baal Meoną ir Sibmą, pakeisdami jų vardus. 
\par 39 Manaso sūnaus Machyro sūnūs patraukė į Gileadą ir jį užėmė, išvydami jo gyventojus amoritus. 
\par 40 Mozė davė Gileado žemes Manaso sūnui Machyrui, kuris ten apsigyveno. 
\par 41 O Manaso sūnus Jayras nuėjęs užėmė krašto miestelius ir juos pavadino Havot Jayru. 
\par 42 Taip pat Nobachas nuėjęs užėmė Kenatą su jo kaimais ir jį pavadino savo vardu­Nobachas.



\chapter{33}


\par 1 Izraelitų sustojimo vietos, išėjus jiems iš Egipto, vadovaujant Mozei ir Aaronui. 
\par 2 Mozė jas surašė pagal stovyklų vietas, kurias jie, Viešpačiui įsakant, keisdavo. 
\par 3 Jie išėjo iš Ramzio pirmo mėnesio penkioliktą dieną, kitą dieną po Paschos, galingos rankos vedami visiems egiptiečiams matant, 
\par 4 laidojant jiems Viešpaties išžudytus pirmagimius. Ir jų dievams Viešpats įvykdė teismą. 
\par 5 Izraelitai, iškeliavę iš Ramzio, pasistatė stovyklas Sukote. 
\par 6 Iš Sukoto atvyko į Etamą, esantį dykumos pakraštyje. 
\par 7 Iš ten atvyko prie Pi Hahiroto, kuris yra priešais Baal Cefoną, ir sustojo prie Migdolo. 
\par 8 Iškeliavę iš Pi Hahiroto, perėjo jūros dugnu į dykumą ir, keliavę tris dienas per Etamo dykumą, pasistatė stovyklą Maroje. 
\par 9 Išėję iš Maros, pasiekė Elimą, kur buvo dvylika vandens šaltinių bei septyniasdešimt palmių, ir sustojo. 
\par 10 Išėję iš Elimo, ištiesė palapines prie Raudonosios jūros. 
\par 11 Iš čia išėję, sustojo Sino dykumoje. 
\par 12 Iš Sino dykumos atkeliavo į Dofką, 
\par 13 išėję iš Dofkos sustojo Aluše. 
\par 14 Iškeliavę iš Alušo, ištiesė palapines Refidime, kur stigo vandens. 
\par 15 Palikę Refidimą, sustojo Sinajaus dykumoje. 
\par 16 Išėję iš Sinajaus dykumos, atėjo į Kibrot Taavos kapines. 
\par 17 Iškeliavę iš čia, pasistatė stovyklą Hacerote. 
\par 18 Iš Haceroto atėjo į Ritmą, 
\par 19 o iš Ritmos­į Rimon Perecą. 
\par 20 Išėję iš Rimon Pereco, atvyko į Libną, 
\par 21 o išėję iš Libnos, sustojo Risoje. 
\par 22 Išėję ir Risos, atvyko į Kehelatą, 
\par 23 o išėję iš Kehelato, sustojo prie Sefero kalno. 
\par 24 Pasitraukę nuo Šefero kalno, atėjo į Haradą. 
\par 25 Išėję iš Harados, pasistatė stovyklą Makhelote. 
\par 26 Iškeliavę iš Makheloto, atėjo į Tahatą, 
\par 27 o išėję iš Tahato, sustojo Terache. 
\par 28 Išėję ir Teracho, ištiesė palapines Mitkoje, 
\par 29 o išėję iš Mitkos, pasistatė stovyklą Hašmonoje. 
\par 30 Išėję iš Hašmonos, atvyko į Moserotą, 
\par 31 o išėję iš Moseroto, pasistatė stovyklą Bene Jaakane. 
\par 32 Išėję iš Bene Jaakano, atkeliavo į Hor Gidgadą, 
\par 33 o išėję iš čia, pasistatė stovyklą Jotbatoje. 
\par 34 Iš Jotbatos atėjo į Abroną, 
\par 35 išėję iš Abronos, ištiesė palapines Ecjon Gebere, 
\par 36 o iš Ecjon Gebero atvyko į Cino dykumą, prie Kadešo. 
\par 37 Išėję iš Kadešo, jie pasistatė stovyklas prie Horo kalno Edomo krašto pasienyje. 
\par 38 Kunigas Aaronas, Viešpačiui liepiant užlipo ant Horo kalno ir mirė penkto mėnesio pirmą dieną, praėjus keturiasdešimčiai metų nuo izraelitų išėjimo iš Egipto, 
\par 39 būdamas šimto dvidešimt trejų metų amžiaus. 
\par 40 Kanaaniečių karalius Aradas, kuris gyveno Kanaano pietuose, sužinojo, kad ateina izraelitai. 
\par 41 Pasitraukę nuo Horo kalno, jie sustojo Calmone. 
\par 42 Išėję iš Calmono, jie atėjo į Punoną, 
\par 43 o išėję iš Punono, sustojo Obote. 
\par 44 Iš Oboto atėjo į Ije Abarimą, prie moabitų sienos, 
\par 45 o išėję iš Ije Abarimo, ištiesė palapines Dibon Gade. 
\par 46 Išėję iš Dibon Gado, sustojo Almon Diblataimoje, 
\par 47 o iš čia jie atėjo iki Abarimo kalnų ties Neboju. 
\par 48 Keliaudami nuo Abarimo kalnų, jie pasiekė Moabo lygumas prie Jordano ties Jerichu 
\par 49 ir pasistatė stovyklą prie Jordano nuo Bet Ješimoto ligi Abel Šitimo, moabitų lygumose. 
\par 50 Moabitų lygumose Viešpats kalbėjo Mozei: 
\par 51 “Kai pereisite Jordaną ir įeisite į Kanaano žemę, 
\par 52 išvykite visus to krašto gyventojus, sunaikinkite jų atvaizdus, sulaužykite stabus, išgriaukite visas aukštąsias stabų garbinimo vietas 
\par 53 ir apsigyvenkite šalyje, kurią Aš jums daviau paveldėti. 
\par 54 Pasidalykite ją burtų keliu: tiems, kurių bus daugiau, duokite didesnį žemės plotą, o kurių mažiau­mažesnį. Žemę išdalinkite burtų keliu giminėms ir šeimoms. 
\par 55 Jei neišvarysite žemės gyventojų, jie bus jums krislai akyse ir dygliai šonuose ir vargins jus krašte, kuriame jūs gyvensite. 
\par 56 Ir tada, ką buvau sumanęs padaryti jiems, jums padarysiu”.



\chapter{34}


\par 1 Viešpats kalbėjo Mozei: 
\par 2 “Kai įeisite į Kanaano šalį ir ją užimsite, jos sienos bus: 
\par 3 pietinė dalis prasidės nuo Cino dykumos, esančios šalia Edomo, jos rytų siena bus Druskos jūra, pradedant pietine dalimi. 
\par 4 Pietinė siena eis išilgai Akrabimo aukštumos ir tęsis iki Sinajaus, pietuose pasieks Kadeš Barnėją. Iš čia siena eis ligi Hacar Adaro ir toliau iki Acmono. 
\par 5 Nuo Acmono siena pasisuks, sieks Egipto upę ir baigsis jūros krantu. 
\par 6 Vakarų siena bus jums Didžioji jūra. 
\par 7 Šiaurėje siena prasidės nuo Didžiosios jūros ir tęsis iki Horo kalno, 
\par 8 iš čia į Lebo Hamatą iki Cedado apylinkės, 
\par 9 toliau siena tęsis iki Zifrono ir Hacar Enano. 
\par 10 Rytų siena eis nuo Hacar Enano ligi Šefamo, 
\par 11 nuo Šefamo tęsis iki Riblos į rytus nuo Aino, iš čia eis Jordano rytų puse iki Kinereto ežero 
\par 12 ir toliau Jordanu iki Sūriosios jūros. Tai bus jūsų krašto sienos”. 
\par 13 Mozė kalbėjo izraelitams: “Tai žemė, kurią paveldėsite burtų keliu. Ją Viešpats liepė išdalinti devynioms ir pusei giminės. 
\par 14 Rubeno, Gado ir pusė Manaso giminės gavo savo dalį. 
\par 15 Dvi giminės ir pusė gavo savo dalį šioje Jordano pusėje ties Jerichu”. 
\par 16 Viešpats tarė Mozei: 
\par 17 “Tai yra vardai vyrų, kurie jums padalins žemę: kunigas Eleazaras, Nūno sūnus Jozuė 
\par 18 ir iš kiekvienos giminės po vieną kunigaikštį. 
\par 19 Jų vardai: iš Judo giminės­Jefunės sūnus Kalebas, 
\par 20 iš Simeono­Amihudo sūnus Samuelis, 
\par 21 iš Benjamino­Kislono sūnus Elidadas, 
\par 22 iš Dano­Joglio sūnus Bukis, 
\par 23 iš Juozapo palikuonių, Manaso giminės­Efodo sūnus Hanielis 
\par 24 ir iš Efraimo­Šiftano sūnus Kemuelis, 
\par 25 iš Zabulono­Parnacho sūnus Elicafanas, 
\par 26 iš Isacharo­Azano sūnus Paltielis, 
\par 27 iš Ašero­Šelomio sūnus Ahihudas, 
\par 28 iš Neftalio­Amihudo sūnus Pedahelis”. 
\par 29 Šitiems vyrams Viešpats įsakė padalyti Kanaano žemę izraelitams.



\chapter{35}


\par 1 Moabo lygumose prie Jordano, ties Jerichu, Viešpats kalbėjo Mozei: 
\par 2 “Pasakyk izraelitams duoti levitams iš savo paveldėjimo miestus gyventi ir jų apylinkes ganykloms; 
\par 3 jie gyvens miestuose, jų bandos ganysis miestų apylinkėse, 
\par 4 kurios tęsis tūkstantį uolekčių aplink miestų mūrus. 
\par 5 Ganyklos sieks po du tūkstančius uolekčių visomis kryptimis: į rytus, į pietus, į vakarus ir į šiaurę, o miestai bus viduryje. 
\par 6 Iš tų miestų, kuriuos duosite levitams, atskirkite šešis prieglaudai, kad juose rastų apsaugą nusikaltę žmogžudyste. Be šitų duokite levitams dar keturiasdešimt du miestus. 
\par 7 Iš viso duokite levitams keturiasdešimt aštuonis miestus su jų ganyklomis. 
\par 8 Miestų, kuriuos duosite iš izraelitų nuosavybės levitams, daugiau paimsite iš tų giminių, kurios daugiau turi, ir mažiau iš tų, kurios mažiau turi; kiekvienas duos levitams miestų pagal savo paveldo dydį”. 
\par 9 Viešpats tarė Mozei: 
\par 10 “Pasakyk izraelitams, kad, įėję į Kanaano žemę, 
\par 11 paskirtų prieglaudos miestus netyčia nužudžiusiems žmogų. 
\par 12 Tokie miestai bus jums prieglauda nuo keršytojo, kad tas, kuris užmušė žmogų, nemirtų, kol nebus stojęs prieš bendruomenės teismą. 
\par 13 Paskirkite šešis miestus prieglaudai, 
\par 14 iš kurių trys bus Jordano rytų pusėje ir trys bus Kanaano žemėje. 
\par 15 Tie šeši miestai bus prieglauda izraelitams, taip pat ateiviams ir svetimšaliams, kad kiekvienas, netyčia nužudęs žmogų, galėtų į juos atbėgti. 
\par 16 Jei kas geležimi užmuštų žmogų, bus kaltinamas žmogžudyste ir baudžiamas mirtimi. 
\par 17 Jei kas akmeniu užmuštų žmogų, jis bus nubaustas mirtimi. 
\par 18 Jei kas bus nužudytas mediniu įrankiu, užmušėjas bus baudžiamas mirtimi. 
\par 19 Kraujo keršytojas pats užmuš žudiką, kai tik jį sutiks. 
\par 20 Jei kas iš neapykantos pastumtų žmogų ar, mesdamas kuo, jį užmuštų, 
\par 21 arba ranka suduotų taip, kad jis numirtų, nusikaltėlis bus laikomas žmogžudžiu ir baudžiamas mirtimi. Kraujo keršytojas užmuš žudiką, kai tik sutiks. 
\par 22 Jei kas neturėdamas priešiškumo kitą pastumtų, ar mestų į jį ką nors, prieš tai netykojęs, 
\par 23 ar užgautų akmeniu, kokiu galima užmušti, jo nepastebėjęs, taip, kad šis numirtų, tačiau prieš tai nebuvo jo priešas ir netykojo jam pakenkti, 
\par 24 tai bendruomenė darys teismą tarp žudiko ir kraujo keršytojo pagal šiuos nurodymus. 
\par 25 Teismas išvaduos žudiką iš kraujo keršytojo rankų ir grąžins į prieglaudos miestą, kuriame jis pasiliks iki vyriausiojo kunigo, patepto šventu aliejumi, mirties. 
\par 26 Jei žudikas išeis už prieglaudos miesto, į kurį pabėgo, sienų 
\par 27 ir sutikęs kraujo keršytojas jį užmuš, jis bus nekaltas, 
\par 28 nes anas privalėjo likti mieste ligi vyriausiojo kunigo mirties; po kunigo mirties žmogžudys sugrįš prie savo nuosavybės. 
\par 29 Šį nuostatą privalo vykdyti visos jūsų kartos, kur jūs begyventumėte. 
\par 30 Žmogžudys bus nubaustas, apklausus liudytojus; tačiau vieno liudytojo neužtenka, kad pasmerktų žmogų mirčiai. 
\par 31 Žmogžudys negali būti išpirktas, jis turi mirti. 
\par 32 Pabėgėliai negalės išsipirkti ir grįžti prie savo nuosavybės prieš vyriausiojo kunigo mirtį. 
\par 33 Nesutepkite savo žemės, nes ji sutepama krauju ir negali būti apvalyta kitaip, kaip tik krauju to, kuris praliejo kraują. 
\par 34 Nesutepkite žemės, kurią paveldėsite ir kurioje Aš gyvenu, nes Aš, Viešpats, gyvenu tarp Izraelio vaikų”.



\chapter{36}


\par 1 Manaso sūnaus Machyro sūnaus Gileado šeimų vadai iš Juozapo giminės atėjo ir kalbėjo Mozei, Izraelio vadams girdint: 
\par 2 “Tau, mūsų valdove, Viešpats įsakė, kad burtų keliu padalytum žemę tarp izraelitų ir kad mūsų brolio Celofhado dalį duotum jo dukterims. 
\par 3 Jei jas ves kitos giminės vyrai, jų dalis pereis kitai giminei ir bus atimta iš mums kritusios paveldėjimo dalies. 
\par 4 Atėjus jubiliejaus metams, jų dalis prisidės prie tos giminės žemių, į kurią jos išėjo; taip jų dalis bus atplėšta nuo mūsų giminei tekusios dalies”. 
\par 5 Mozė atsakė izraelitams pagal Viešpaties žodį: “Juozapo sūnų giminė kalba teisingai. 
\par 6 Štai ką sako Viešpats apie Celofhado dukteris: ‘Tegul jos teka už ko nori, bet tik už savo giminės vyrų, 
\par 7 kad izraelitų nuosavybė nepereitų iš vienos giminės kitai giminei. 
\par 8 Visos mergaitės, turinčios paveldėjimo teisę, privalo tekėti už tos pačios giminės vyrų, kad Izraelio vaikai paveldėtų kiekvienas savo tėvų dalį. 
\par 9 Nuosavybė iš vienos giminės nepereis kitai; visos izraelitų giminės išlaikys savo paveldo dalį’ ”. 
\par 10 Kaip Viešpats įsakė Mozei, taip padarė Celofhado dukterys: 
\par 11 Machla, Tirca, Hogla, Milka ir Noja ištekėjo už savo pusbrolių 
\par 12 iš giminės Manaso, kuris buvo Juozapo sūnus. Joms paskirtoji dalis pasiliko jų giminėje ir šeimoje. 
\par 13 Šituos nuostatus ir įsakymus Viešpats davė izraelitams per Mozę Moabo lygumose prie Jordano ties Jerichu.


\end{document}