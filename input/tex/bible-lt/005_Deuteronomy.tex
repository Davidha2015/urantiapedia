\begin{document}

\title{Deuteronomy}

\chapter{1}


\par 1 Mozė kalbėjo izraelitams šioje Jordano pusėje, dykumoje, tarp Parano, Tofelio, Labano, Haceroto ir Di Zahabo miestų. 
\par 2 Nuo Horebo per Seyro kalnus iki Kadeš Barnėjos yra vienuolikos dienų kelias. 
\par 3 Keturiasdešimtaisiais metais, vienuolikto mėnesio pirmą dieną Mozė kalbėjo izraelitams viską, ką jam Viešpats dėl jų buvo įsakęs. 
\par 4 Nugalėjęs amoritų karalių Sihoną, gyvenusį Hešbone, ir Bašano karalių Ogą, gyvenusį Aštarote, Edrėjo mieste, 
\par 5 šioje Jordano pusėje, Moabo žemėje, Mozė pradėjo aiškinti įstatymą: 
\par 6 “Viešpats, mūsų Dievas, mums kalbėjo prie Horebo: ‘Jau užtenka jums gyventi prie šio kalno. 
\par 7 Eikite pas amoritus į visas jų apylinkes: į lygumas, kalnuotas vietas, slėnius, pietų link ir išilgai jūros kranto, į Kanaano ir Libano žemes iki didžiosios Eufrato upės’. 
\par 8 Aš atvedžiau jus į žemę, kurią Viešpats prisiekė duoti jūsų tėvams Abraomui, Izaokui ir Jokūbui ir jų palikuonims. Eikite ir užimkite ją. 
\par 9 Aš jums anuo metu sakiau: 
\par 10 ‘Aš vienas neįstengiu jumis rūpintis, nes Viešpats, jūsų Dievas, jus padaugino, ir šiandien jūsų yra tiek, kiek dangaus žvaigždžių. 
\par 11 Viešpats, jūsų tėvų Dievas, dar tūkstanteriopai tepadaugina ir tepalaimina jus, kaip Jis pažadėjo. 
\par 12 Kaip aš vienas galiu nešti jūsų naštas, vargus ir vaidus? 
\par 13 Išsirinkite iš savo giminių išmintingų, sumanių ir žinomų vyrų, kad juos paskirčiau jums vadais’. 
\par 14 Tada jūs sutikote, kad taip padaryti yra gerai. 
\par 15 Aš iš jūsų giminių išmintingus bei sumanius vyrus paskyriau jums vadovauti: tūkstantininkais, šimtininkais, penkiasdešimtininkais ir dešimtininkais. 
\par 16 Įsakiau jūsų teisėjams: ‘Išklausykite bylas ir teisingai teiskite juos ir ateivius, 
\par 17 neatsižvelkite teisme į asmenis; išklausykite mažą ir didelį, nebijokite jokio žmogaus, nes teismas yra Dievo. Jeigu jums kas būtų per sunku, praneškite man, ir aš išklausysiu’. 
\par 18 Tuo metu aš jums įsakiau viską, ką privalote daryti. 
\par 19 Palikę Horebą, ėjome per didžiąją, baisiąją dykumą, kurią matėte, keliu per amoritų kalnus, kaip Viešpats, mūsų Dievas, buvo įsakęs. Kai atvykome į Kadeš Barnėją, 
\par 20 jums sakiau: ‘Atėjote į amoritų kalnyną, kurį Viešpats, mūsų Dievas, mums duoda. 
\par 21 Štai Viešpaties jums pažadėtoji žemė. Eikite ir užimkite ją, nes Viešpats, jūsų tėvų Dievas, jums ją pažadėjo. Nieko nebijokite ir nenusigąskite’. 
\par 22 Tada jūs, priėję prie manęs, tarėte: ‘Siųskime vyrus apžiūrėti žemę ir pranešti, kuriuo keliu turime keliauti ir į kuriuos miestus eiti’. 
\par 23 Man patiko ta šneka, todėl siunčiau iš jūsų dvylika vyrų, po vieną iš kiekvienos giminės. 
\par 24 Jie nuėjo į kalnuotas vietas ligi Eškolo slėnio, apžiūrėjo žemę, 
\par 25 paėmė jos vaisių ir, atnešę pas mus, tarė: ‘Gera žemė, kurią Viešpats, mūsų Dievas, mums duoda’. 
\par 26 Tačiau jūs nenorėjote eiti ir sukilote prieš Viešpaties įsakymą, 
\par 27 murmėdami savo palapinėse: ‘Viešpats mūsų nekenčia, todėl mus išvedė iš Egipto, kad amoritai mus sunaikintų. 
\par 28 Kaipgi mes eisime? Pasiuntiniai išgąsdino mus, pranešdami, kad ta tauta gausesnė ir aukštesnė už mus, kad jų miestai dideli ir sustiprinti ligi dangaus ir kad ten jie matė ir Anako palikuonių’. 
\par 29 Aš jums sakiau: ‘Nenusigąskite jų ir nebijokite, 
\par 30 nes Viešpats, jūsų Dievas, eina priekyje jūsų ir pats už jus kariaus, kaip tai darė Egipte jūsų akivaizdoje. 
\par 31 Dykumoje patys matėte, kaip Viešpats, jūsų Dievas, nešė jus, kaip žmogus neša savo kūdikį, visą kelią, kuriuo ėjote, kol atėjote į šitą vietą’. 
\par 32 Bet jūs nepatikėjote Viešpačiu, jūsų Dievu, 
\par 33 kuris ėjo pirma jūsų keliu ir nurodydavo vietą, kur ištiesti palapines; naktį rodydavo kelią ugnimi, o dieną­debesies stulpu. 
\par 34 Viešpats girdėjo jūsų kalbas ir užsirūstinęs prisiekė: 
\par 35 ‘Nė vienas iš šitos piktos kartos neišvys gerosios žemės, kurią su priesaika pažadėjau jūsų tėvams, 
\par 36 išskyrus Jefunės sūnų Kalebą; jis ją matys; Aš duosiu jam ir jo vaikams žemę, kurioje jis vaikščiojo, nes jis sekė mane iki galo’. 
\par 37 Viešpats užsirūstino ir ant manęs dėl jūsų ir tarė: ‘Ir tu neįeisi į ją, 
\par 38 bet tavo tarnas Jozuė įeis; jį padrąsink, nes jis padalys žemę izraelitams. 
\par 39 Jūsų kūdikiai, apie kuriuos sakėte, kad jie taps grobiu, ir jūsų sūnūs, kurie šiandien dar nežino skirtumo tarp gero ir pikto, įeis į ją; jiems Aš duosiu žemę, ir jie ją užims. 
\par 40 Jūs gi grįžkite į dykumą, Raudonosios jūros link’. 
\par 41 Tada man atsakėte: ‘Nusidėjome Viešpačiui, eisime ir kovosime, kaip Viešpats, mūsų Dievas, įsakė’. Jūs apsiginklavote ir buvote pasiruošę traukti į kalnus. 
\par 42 Viešpats man tarė: ‘Sakyk jiems neiti ir nekovoti, kad nežūtų nuo savo priešų, nes Aš nebūsiu su jais’. 
\par 43 Kalbėjau jums, bet jūs neklausėte, priešinotės Viešpaties įsakymui ir atkakliai žygiavote į kalnus. 
\par 44 Amoritai, gyvenantys kalnuose, vijosi kaip bitės jus nuo Seyro iki Hormos. 
\par 45 Sugrįžę raudojote Viešpaties akivaizdoje, bet Jis jūsų neklausė ir nekreipė į jus dėmesio. 
\par 46 Todėl Kadeše pasilikote ilgą laiką”.


\chapter{2}


\par 1 “Pasitraukę iš ten, atėjome į dykumą prie Raudonosios jūros, kaip man Viešpats buvo įsakęs, ir ilgai gyvenome Seyro kalnyne. 
\par 2 Viešpats kalbėjo man: 
\par 3 ‘Užtenka gyventi šiame kalnyne, traukite į šiaurę, 
\par 4 pereikite per jūsų brolių, Ezavo vaikų apgyvendintą Seyro kraštą; jie jūsų bijos. 
\par 5 Saugokitės, nekariaukite su jais, nes jų žemės neduosiu jums nė pėdos. Seyro kalnus daviau paveldėti Ezavui. 
\par 6 Pirkite iš jų maistą už pinigus, taip pat vandenį’. 
\par 7 Viešpats, jūsų Dievas, laimino visus jūsų darbus. Jis žino jūsų kelionę, kai ėjote per didžiąją dykumą; Viešpats, jūsų Dievas, buvo su jumis keturiasdešimt metų ir nieko jums nestigo. 
\par 8 Perėję per mūsų brolių, Ezavo vaikų kraštą, kurie gyveno Seyre, lygumos keliu nuo Elato ir Ecjon Gebero traukėme į Moabo dykumą. 
\par 9 Tada man Viešpats sakė: ‘Nekovok su moabitais ir nepradėk su jais karo. Jų žemės tau neduosiu, nes Arą atidaviau Loto palikuonims’. 
\par 10 Anksčiau jo gyventojai buvo emai, didelė ir galinga tauta, aukšta kaip Anako palikuonys. 
\par 11 Jie buvo laikomi milžinais, kaip ir anakiečiai. Moabitai juos vadina emais. 
\par 12 Seyre anksčiau gyveno horai, bet Ezavo palikuonys juos išvarė ir sunaikino, ir apsigyveno jų vietoje. Jie padarė kaip izraelitai savo paveldėtoje žemėje, kurią jiems davė Viešpats. 
\par 13 Viešpačiui įsakius, perėjome per Zeredo upelį. 
\par 14 Kelionės laikas nuo Kadeš Barnėjos ligi Zeredo upelio buvo trisdešimt aštuoneri metai, kol išmirė visa karta vyrų, tinkamų karui, kaip Viešpats buvo prisiekęs. 
\par 15 Iš tiesų Viešpaties ranka buvo prieš juos, kad išnaikintų juos iš tautos, kol jie visi buvo pražudyti. 
\par 16 Išmirus visiems karui tinkamiems vyrams, 
\par 17 Viešpats tarė man: 
\par 18 ‘Šiandien pereisi moabitų žemę pro miestą Arą. 
\par 19 Priartėjęs prie amonitų, saugokis nekovoti prieš juos ir nepradėk karo, nes Aš tau neduosiu amonitų žemės; Aš ją atidaviau Loto palikuonims’. 
\par 20 Kraštas buvo laikomas milžinų žeme, nes praeityje joje gyveno milžinai, kuriuos amonitai vadina zamzumais. 
\par 21 Tauta buvo didelė, gausi ir labai augalota kaip anakiečiai. Viešpats juos išnaikino amonitų akivaizdoje ir amonitus apgyvendino jų vietoje. 
\par 22 Taip pat padarė ezavitams, gyvenantiems Seyre; išnaikino horus ir jų žemę atidavė ezavitams, kurioje jie gyvena iki šios dienos. 
\par 23 Avus, gyvenusius prie Gazos, išvarė kaftoriečiai, kilę iš Kaftoro, juos išnaikino ir apsigyveno jų vietoje. 
\par 24 Viešpats tarė: ‘Pereikite Arnono upelį; štai Aš atidaviau į tavo rankas Sihoną, amoritų karalių iš Hešbono, ir jo žemę. Pradėk ją užimti ir kariauk su juo. 
\par 25 Šiandien Aš pradėsiu daryti taip, kad visos tautos bijotų tavęs ir būtų apimtos siaubo prieš tave; kurios išgirs apie tave, išsigąs ir drebės dėl tavęs’. 
\par 26 Aš siunčiau iš Kedemoto dykumos pas Hešbono karalių Sihoną pasiuntinius su taikiais žodžiais, prašydamas: 
\par 27 ‘Leisk mums pereiti per tavo kraštą; eisime vieškeliu, nenukrypsime nei į dešinę, nei į kairę. 
\par 28 Maistą valgiui pirksime už pinigus ir vandenį gėrimui už pinigus, tik leisk mums pereiti. 
\par 29 Ezavitai, gyvenantys Seyre, ir moabitai iš Aro tokiu būdu mus praleido per savo kraštą, kad pasiekę Jordaną, eitume į mums pažadėtą žemę’. 
\par 30 Bet Hešbono karalius Sihonas nenorėjo mums leisti eiti per jo kraštą, nes Viešpats, tavo Dievas, užkietino jo dvasią ir padarė jo širdį užsispyrusią, kad galėtų jį atiduoti į tavo rankas. 
\par 31 Tada Viešpats man tarė: ‘Štai Aš pradėjau atiduoti Sihoną ir jo žemę tau, pradėk užimti, kad galėtum paveldėti jo žemę’. 
\par 32 Sihonas su visais kariais išėjo prieš mus Jahace. 
\par 33 Viešpats, mūsų Dievas, atidavė jį, ir mes nugalėjome ir užėmėme kraštą. 
\par 34 Paėmėme visus miestus, išnaikinome jų vyrus, moteris ir kūdikius­nieko nepalikome, 
\par 35 tik visus galvijus ir užimtųjų miestų grobį pasiėmėme. 
\par 36 Nuo Aroero miesto, esančio Arnono upelio slėnyje, iki Gileado nebuvo miesto, kurio nebūtumėme užėmę. Viešpats, mūsų Dievas, atidavė juos visus mums. 
\par 37 Amonitų žemės, kuri yra prie Jaboko upelio, nelietėme, taip pat nelietėme kalnų miestų ir visų vietų, į kurias Viešpats, mūsų Dievas, mums uždraudė eiti”.



\chapter{3}


\par 1 “Pasukome Bašano link. Mus pasitiko Bašano karalius Ogas su savo kariuomene Edrėjyje. 
\par 2 Tada Viešpats man tarė: ‘Nebijok jo, nes atiduosiu jį, visus jo žmones ir jo žemę į tavo rankas; padaryk jam taip, kaip padarei amoritų karaliui Sihonui Hešbone’. 
\par 3 Viešpats, mūsų Dievas, atidavė į mūsų rankas Bašano karalių Ogą ir visus jo žmones. Mes juos naikinome, kol nepalikome nė vieno, 
\par 4 ir užėmėme visus jo miestus. Mums teko šešiasdešimt miestų Argobo krašte, kurį valdė Bašano karalius Ogas. 
\par 5 Visi šitie miestai buvo sustiprinti aukštomis mūro sienomis, vartais ir užkaiščiais, neskaičiuojant daugybės miestų, neturėjusių apsaugos sienų. 
\par 6 Mes juos sunaikinome, pasielgdami kaip ir su Hešbono karaliumi Sihonu,­visiškai sunaikinome vyrus, moteris ir kūdikius kiekviename mieste. 
\par 7 Galvijus ir turtą pasilaikėme kaip grobį. 
\par 8 Tuo metu užėmėme dvi amoritų karalystes, buvusias šiapus Jordano, nuo Arnono upės ligi Hermono kalno, 
\par 9 kurį sidoniečiai vadina Sirjonu, o amoritai­Senyru. 
\par 10 Užėmėme visus Ogo karalystės miestus lygumoje, visą Gileado ir Bašano žemę su miestais Salcha ir Edrėju. 
\par 11 Bašano karalius Ogas buvo paskutinis iš milžinų giminės. Jo geležinė lova yra amonitų mieste Raboje; ji devynių uolekčių ilgio ir keturių pločio. 
\par 12 Tuomet užėmėme kraštą nuo miesto Aroero, kuris yra Arnono upelio slėnyje, ligi Gileado kalno. Jo miestus daviau Rubeno ir Gado giminėms. 
\par 13 Likusią Gileado krašto dalį, visą Bašaną, priklausantį Ogo karalystei, ir Argobo kraštą atidaviau pusei Manaso giminės. Bašanas yra vadinamas milžinų kraštu. 
\par 14 Manaso sūnus Jayras užėmė Argobo kraštą iki gešūriečių ir maakų sienos ir pavadino jį savo vardu­Havot Jayru; taip jis vadinamas iki šios dienos. 
\par 15 Machyrui daviau Gileadą. 
\par 16 Rubeno ir Gado giminėms daviau Gileado kraštą ligi Arnono upelio, kuris buvo riba iki Jaboko upės, amonitų sienos; 
\par 17 toliau siena ėjo per dykumą ligi Jordano ir Kinereto apylinkių iki Druskos jūros ir Pisgos kalno šlaitų rytuose. 
\par 18 Tada jums pasakiau: ‘Viešpats, jūsų Dievas, jums duoda šitą kraštą paveldėti. Visi kovai tinką vyrai, eikite ginkluoti priekyje savo brolių izraelitų. 
\par 19 Žinau, kad turite daug gyvulių, tad žmonas, vaikus ir bandas palikite miestuose, kuriuos jums daviau. 
\par 20 Padėkite savo broliams užimti žemę už Jordano, kurią Viešpats jiems davė, tada grįžkite į savo žemę, kurią aš jums daviau’. 
\par 21 Tuo metu Jozuei kalbėjau: ‘Tavo akys matė, ką Viešpats, jūsų Dievas, padarė šitiems dviems karaliams. Jis taip pat padarys ir visoms karalystėms, į kurias tu eisi. 
\par 22 Nebijok jų, nes Viešpats, jūsų Dievas, kovos už jus’. 
\par 23 Aš tada maldavau Viešpatį: 
\par 24 ‘Viešpatie Dieve! Tu pradėjai rodyti Izraelio tautai savo didybę ir galią. Nėra kito dievo nei danguje, nei žemėje, kuris galėtų daryti tokius galingus darbus! 
\par 25 Leisk man įeiti ir pamatyti gerąją žemę anapus Jordano, tą puikųjį kalnuotą kraštą ir Libaną’. 
\par 26 Tačiau Viešpats buvo supykęs ant manęs dėl jūsų ir manęs neišklausė, ir tarė: ‘Daugiau man apie tai nekalbėk. 
\par 27 Užlipk į Pisgos viršūnę ir apžvelk savo akimis vakarus, šiaurę, pietus ir rytus; tu nepereisi Jordano. 
\par 28 Kalbėk Jozuei, jį padrąsink ir sustiprink, nes jis eis šitos tautos priekyje ir jiems išdalins žemę, kurią tu matysi’. 
\par 29 Mes pasilikome slėnyje ties Bet Peoru”.



\chapter{4}

\par 1 “Klausyk, Izraeli, įstatymų ir paliepimų, kurių jus mokau, kad juos vykdydami gyventumėte ir užėmę paveldėtumėte žemę, kurią Viešpats, jūsų Dievas, jums duoda. 
\par 2 Nieko nepridėkite prie mano įstatymų ir nieko iš jų neatimkite. Laikykitės Viešpaties, jūsų Dievo, įsakymų, kuriuos jums skelbiu. 
\par 3 Jūsų akys matė, ką Viešpats darė dėl Baal Peoro, kaip išnaikino visus jo garbintojus, buvusius tarp jūsų. 
\par 4 Jūs gi, kurie laikėtės Viešpaties, jūsų Dievo, išlikote gyvi iki šios dienos. 
\par 5 Štai aš mokiau jus nuostatų ir įstatymų, kaip Viešpats, mano Dievas, man įsakė, kad jūs vykdytumėte juos žemėje, kurią paveldėsite. 
\par 6 Laikykitės jų ir vykdykite juos, nes tai jūsų išmintis ir protas tautų akyse. Jos išgirs jūsų nuostatus ir sakys: ‘Iš tiesų ši didelė tauta yra protinga ir išmintinga’. 
\par 7 Iš tikrųjų nėra kitos tokios didingos tautos, kuriai Dievas būtų taip arti, kaip mūsų Dievas, kai Jo šaukiamės. 
\par 8 Kuri kita didelė tauta turi nuostatus ir įsakymus tokius teisingus, kaip šis įstatymas, kurį šiandien skelbiu jums? 
\par 9 Saugokis ir rūpestingai saugok savo sielą, kad neužmirštum to, ką matei savo akimis, ir tepasilieka tai tavo širdyje per visas tavo dienas. Jų mokykite savo vaikus ir vaikaičius. 
\par 10 Tą dieną, kai stovėjai Viešpaties, savo Dievo, akivaizdoje prie Horebo, Jis man kalbėjo: ‘Surink prie manęs tautą, kad išgirstų mano žodžius ir bijotų manęs, kol gyvens žemėje, ir mokytų to paties savo vaikus’. 
\par 11 Jūs priartėjote prie kalno pašlaitės, iš kur liepsnos kilo į dangų. Jis buvo apsuptas tamsa ir debesimis. 
\par 12 Viešpats kalbėjo jums iš ugnies. Jūs girdėjote Jo žodžius, bet nematėte Jo pavidalo; girdėjote tik balsą. 
\par 13 Jis jums paskelbė savo sandorą ir įsakė ją vykdyti. Jis užrašė dešimt įsakymų dviejose akmeninėse plokštėse 
\par 14 ir įsakė man mokyti jus vykdyti tuos įsakymus ir paliepimus žemėje, kurią paveldėsite. 
\par 15 Gerai įsidėmėkite, kad tą dieną, kai Viešpats jums kalbėjo Horebe iš ugnies, jūs nematėte jokio pavidalo, 
\par 16 kad kartais nedirbtumėte sau drožinių, panašių į vyrą ar moterį, 
\par 17 panašių į kurį nors gyvulį, esantį žemėje, ar skraidantį padangėje paukštį, 
\par 18 ar šliaužiantį gyvį, ar žuvį, plaukiojančią vandenyse. 
\par 19 Kad kartais, pakėlęs akis į dangų ir pamatęs saulę, mėnulį, žvaigždes, nepaklystum, jų negarbintum ir nesilenktum tiems, kuriuos Viešpats, jūsų Dievas, sutvėrė tarnauti visoms tautoms. 
\par 20 Jus gi Viešpats išvedė iš geležinės krosnies, Egipto vergijos, kad būtumėte Jo tauta. 
\par 21 Dėl jūsų Viešpats užsirūstino ant manęs ir prisiekė, kad nepereisiu Jordano ir neįeisiu į tą gerą žemę, kurią Viešpats, tavo Dievas, duoda tau paveldėti. 
\par 22 Aš turėsiu mirti šioje žemėje ir nepereisiu Jordano; tačiau jūs pereisite ir užimsite tą gerą žemę. 
\par 23 Neužmirškite Viešpaties, savo Dievo, sandoros, kurią Jis su jumis padarė ir nedarykite sau jokio drožinio ar atvaizdo, nes Viešpats tai uždraudė. 
\par 24 Viešpats, tavo Dievas, yra naikinanti ugnis, pavydus Dievas. 
\par 25 Jei gyvendami žemėje, kai susilauksite vaikų ir vaikaičių, suteršite save ir pasidarysite drožinių ar atvaizdų, ir darysite pikta Viešpaties, savo Dievo, akyse, Jį supykdydami, 
\par 26 aš šaukiu šiandien liudytoju dangų ir žemę, kad, taip elgdamiesi, jūs žūsite krašte, kurį, perėję per Jordaną, paveldėsite. Negyvensite jame ilgai, bet būsite visai sunaikinti. 
\par 27 Viešpats jus išsklaidys tarp tautų, ir jūsų liks labai mažai tarp pagonių, pas kuriuos būsite išvesti. 
\par 28 Tenai tarnausite dievams, kurie žmonių rankomis padaryti: medžiui ir akmeniui, kurie nemato ir negirdi, nevalgo ir nieko neužuodžia. 
\par 29 Bet jei ten ieškosi Viešpaties, savo Dievo, visa širdimi ir siela, tu atrasi Jį. 
\par 30 Patyręs vargą, paskutinėmis dienomis sugrįši prie Viešpaties, savo Dievo, ir klausysi Jo balso. 
\par 31 Nes Viešpats, tavo Dievas, yra gailestingas Dievas; Jis nepaliks tavęs ir nesunaikins, neužmirš sandoros, padarytos su tavo tėvais. 
\par 32 Paklausk praeities dienų, pradedant nuo tos dienos, kai Dievas sutvėrė žmogų žemėje, ir dangus nuo vieno krašto iki kito, ar įvyko kada nors toks didingas dalykas, ar girdėta tai? 
\par 33 Ar kuri nors tauta girdėjo Dievo balsą, kalbantį iš ugnies, kaip tu girdėjai, ir išliko gyva? 
\par 34 Ar kada nors Dievas yra atėjęs pasiimti tautos, esančios vergijoje, bandymais, ženklais, stebuklais, kova, galinga ranka ir baisiais siaubais, kaip Viešpats, jūsų Dievas, padarė dėl jūsų Egipte, jums visa tai matant savo akimis? 
\par 35 Jis tai parodė tau, kad žinotum, jog Viešpats yra Dievas ir kito nėra. 
\par 36 Jis tau leido išgirsti balsą iš dangaus, kad pamokytų tave, žemėje parodė tau didelę ugnį, ir tu girdėjai Jo žodžius iš ugnies. 
\par 37 Jis mylėjo tavo tėvus, todėl išsirinko jų palikuonis ir išvedė iš Egipto savo didžia galia, 
\par 38 kad išvytų už tave didesnes ir galingesnes tautas, ir tave įvestų paveldėti jų žemę. 
\par 39 Suprask tad šiandien ir palaikyk visa tai savo širdyje, kad Viešpats yra dangaus ir žemės Dievas ir jokio kito nėra. 
\par 40 Laikykis Jo įsakymų ir paliepimų, kuriuos tau skelbiu šiandien, kad tau ir tavo vaikams gerai sektųsi ir ilgai gyventum žemėje, kurią Viešpats, tavo Dievas, tau duoda visiems amžiams”. 
\par 41 Mozė paskyrė tris miestus šioje Jordano pusėje, 
\par 42 kad juose rastų prieglaudą ir išliktų gyvas žmogžudys, netyčia užmušęs savo artimą, kuris nebuvo jo priešas. 
\par 43 Iš Rubeno giminės paskyrė Becerą dykumos lygumoje, iš Gado giminės­Ramot Gileadą ir iš Manaso giminės­Golaną Bašane. 
\par 44 Tai įstatymas, Mozės duotas izraelitams. 
\par 45 Šitie įsakymai, nuostatai ir paliepimai buvo paskelbti izraelitams, jiems išėjus iš Egipto, 
\par 46 šioje Jordano pusėje, slėnyje, prieš Bet Peoro miestą, krašte amoritų karaliaus Sihono, kuris gyveno Hešbone ir buvo Mozės ir izraelitų nugalėtas, kai jie išėjo iš Egipto. 
\par 47 Jie apsigyveno Sihono ir Bašano karaliaus Ogo žemėse, dviejų amoritų karalių, buvusių į rytus nuo Jordano. 
\par 48 Šis kraštas tęsėsi nuo Aroero miesto, Arnono upelio slėnyje, ligi Siono kalno, kuris kitaip vadinamas Hermonu, 
\par 49 per visą lygumą nuo Jordano į rytus iki lygumos jūros ir Pisgos kalno šlaitų.



\chapter{5}


\par 1 Mozė sušaukė visus izraelitus ir jiems tarė: “Klausyk, Izraeli, šiandien skelbiu jums įstatymus ir paliepimus, mokykitės ir vykdykite juos. 
\par 2 Viešpats, mūsų Dievas, padarė su mumis sandorą Horebe. 
\par 3 Ne su mūsų tėvais padarė Jis sandorą, bet su mumis, kurie šiandien esame gyvi. 
\par 4 Viešpats kalbėjo su jumis veidas į veidą ant kalno iš ugnies. 
\par 5 Tada aš buvau tarpininkas tarp Dievo ir jūsų, paskelbiau jums Jo žodžius, nes jūs bijojote ugnies ir nėjote ant kalno. 
\par 6 Jis tarė: ‘Aš esu Viešpats, tavo Dievas, kuris tave išvedžiau iš Egipto žemės, iš vergijos namų. 
\par 7 Neturėk kitų dievų šalia manęs. 
\par 8 Nedaryk sau jokio drožinio nė jokio atvaizdo to, kas yra aukštai danguje, žemai žemėje ar po žeme vandenyje. 
\par 9 Nesilenk prieš juos ir netarnauk jiems. Nes Aš, Viešpats, tavo Dievas, esu pavydus Dievas, baudžiąs vaikus už tėvų kaltes iki trečios ir ketvirtos kartos tų, kurie manęs nekenčia, 
\par 10 bet rodąs gailestingumą iki tūkstantosios kartos tiems, kurie mane myli ir laikosi mano įsakymų. 
\par 11 Netark Viešpaties, savo Dievo, vardo be reikalo, nes Viešpats nepaliks be kaltės to, kuris be reikalo Jo vardą mini. 
\par 12 Sabato dieną švęsk, kaip Viešpats, tavo Dievas, įsakė. 
\par 13 Šešias dienas dirbk visus savo darbus, 
\par 14 o septintoji diena yra sabatas Viešpačiui, tavo Dievui. Tą dieną nedirbk jokio darbo nei tu, nei tavo sūnus, nei duktė, nei tarnas, nei tarnaitė, nei tavo jautis, nei asilas, nei joks tavo gyvulys, nei ateivis, kuris yra tavo namuose, kad tavo tarnas ir tarnaitė pailsėtų taip pat, kaip ir tu. 
\par 15 Atsimink, kad ir pats buvai vergas Egipte ir iš ten tave išvedė Viešpats, tavo Dievas, galinga ranka. Todėl Jis tau įsakė švęsti sabato dieną. 
\par 16 Gerbk savo tėvą ir motiną, kaip Viešpats, tavo Dievas, įsakė, kad ilgai gyventum ir tau gerai sektųsi žemėje, kurią Viešpats, tavo Dievas, tau duoda. 
\par 17 Nežudyk. 
\par 18 Nesvetimauk. 
\par 19 Nevok. 
\par 20 Neliudyk neteisingai prieš savo artimą. 
\par 21 Negeisk savo artimo žmonos nei namų, nei lauko, nei tarno, nei tarnaitės, nei jaučio, nei asilo: nieko, kas yra tavo artimo’. 
\par 22 Šituos žodžius Viešpats kalbėjo jums visiems iš ugnies, debesies ir tamsybės garsiai ir, užrašęs tai dviejose akmeninėse plokštėse, jas padavė man. 
\par 23 Išgirdę balsą iš tamsybės ir pamatę kalną degant, jūs ir visi jūsų giminių vadai bei vyresnieji priartėjote prie manęs ir tarėte: 
\par 24 ‘Štai Viešpats, mūsų Dievas, mums parodė savo šlovę ir didybę; girdėjome Jo balsą iš ugnies ir šiandien patyrėme, kad Dievui kalbant su žmogumi, galima išlikti gyviems. 
\par 25 Kodėl turėtume mirti nuo šitos didelės ugnies? Jei dar kartą išgirsime Viešpaties, mūsų Dievo, balsą, mes visi mirsime. 
\par 26 Ar yra koks kūnas, girdėjęs balsą gyvojo Dievo, kalbančio iš ugnies, kaip mes girdėjome, ir išlikęs gyvas? 
\par 27 Geriau tu prisiartink ir klausyk, ką Viešpats, mūsų Dievas, sakys, o po to pranešk mums, ir mes klausysime ir tai vykdysime’. 
\par 28 Viešpats, tai išgirdęs, man tarė: ‘Aš girdėjau tautos žodžius, jie viską gerai kalbėjo. 
\par 29 O, kad jie visuomet turėtų tokią širdį ir manęs bijotų bei laikytųsi mano įsakymų, tai jiems ir jų vaikams per amžius gerai sektųsi! 
\par 30 Eik ir sakyk jiems, kad grįžtų į savo palapines. 
\par 31 Tu gi stovėk su manimi čia: Aš tau paskelbsiu visus savo įsakymus, įstatymus ir paliepimus, kurių mokysi juos, kad vykdytų žemėje, kurią jiems duosiu paveldėti’. 
\par 32 Klausykite ir vykdykite, ką Viešpats Dievas jums įsakė; nenukrypkite nei į dešinę, nei į kairę. 
\par 33 Vaikščiokite keliais, kuriuos Viešpats, jūsų Dievas, jums nurodė, kad būtumėte gyvi, kad jums gerai sektųsi ir ilgai gyventumėte žemėje, kurią paveldėsite”.



\chapter{6}


\par 1 “Šitie yra įsakymai, nuostatai ir paliepimai, kurių Viešpats, jūsų Dievas, įsakė jus mokyti, kad juos vykdytumėte žemėje, kurią užimsite; 
\par 2 kad jūs, jūsų vaikai ir vaikaičiai per visas savo dienas bijotumėte Viešpaties, savo Dievo, ir laikytumėtės visų Jo paliepimų bei įsakymų, kuriuos jums skelbiu, ir ilgai gyventumėte. 
\par 3 Klausyk, Izraeli, ir rūpestingai vykdyk, ką Viešpats įsako; tuomet tau gerai seksis ir tu labai išsiplėsi, kaip Viešpats, tavo tėvų Dievas, pažadėjo tau, pienu ir medumi plūstančioje šalyje. 
\par 4 Klausyk, Izraeli! Viešpats, mūsų Dievas, yra vienintelis Dievas! 
\par 5 Mylėk Viešpatį, savo Dievą, visa širdimi, visa siela ir visomis jėgomis. 
\par 6 Šitie žodžiai, kuriuos tau šiandien skelbiu, tepasilieka tavo širdyje; 
\par 7 mokyk jų savo vaikus ir apie juos kalbėk, sėdėdamas savo namuose, būdamas kelionėje, guldamas ir atsikeldamas. 
\par 8 Prisitvirtink juos kaip ženklą prie savo rankos ir prie kaktos; 
\par 9 užrašyk juos ant durų staktų ir savo kiemo vartų. 
\par 10 Kai Viešpats, tavo Dievas, įves tave į žemę, kurią pažadėjo tavo tėvams, Abraomui, Izaokui ir Jokūbui, tu gausi didelius ir gerus miestus, kurių nestatei, 
\par 11 namus pilnus visokių gėrybių, kurių nekaupei, šulinius, kurių nekasei, vynuogynus ir alyvmedžių sodus, kurių nesodinai. Kai tu valgysi ir pasisotinsi, 
\par 12 saugokis, kad nepamirštum Viešpaties, kuris tave išvedė iš Egipto žemės, iš vergijos namų. 
\par 13 Bijok Viešpaties, savo Dievo, Jam vienam tarnauk ir Jo vardu prisiek. 
\par 14 Negarbink dievų svetimų tautų, kurios gyvena aplink jus, 
\par 15 nes Viešpats, tavo Dievas, kuris yra tarp jūsų, yra pavydus Dievas; kad Viešpaties, tavo Dievo, pyktis neužsidegtų prieš tave ir Jis neišnaikintų tavęs nuo žemės paviršiaus. 
\par 16 Negundykite Viešpaties, savo Dievo, kaip gundėte Jį Masoje. 
\par 17 Uoliai vykdykite Viešpaties, savo Dievo, įsakymus, paliepimus ir įstatymus, kuriuos Jis tau davė. 
\par 18 Daryk tai, kas gera ir teisinga Viešpaties akivaizdoje, kad tau gerai sektųsi ir užėmęs paveldėtum gerąją žemę, apie kurią Viešpats prisiekė tavo tėvams, 
\par 19 kad išvarys visus tavo priešus iš tavo akivaizdos, kaip Viešpats kalbėjo. 
\par 20 Kai tavo sūnus ateityje klaus: ‘Ką reiškia Viešpaties, mūsų Dievo, duoti paliepimai, įstatymai ir įsakymai?’, 
\par 21 jam atsakyk: ‘Buvome Egipte faraono vergai; Viešpats mus išvedė iš Egipto galinga ranka. 
\par 22 Jis darė Egipte mūsų akyse didelius ir baisius ženklus bei stebuklus prieš faraoną ir visus jo namus. 
\par 23 Jis mus iš ten išvedė, kad nuvestų į kraštą, kurį pažadėjo mūsų tėvams. 
\par 24 Viešpats mums įsakė vykdyti šituos įstatymus ir bijoti Viešpaties, mūsų Dievo, kad mums visuomet gerai sektųsi ir kad Jis saugotų mūsų gyvybę, kaip tai yra šiandien. 
\par 25 Tai bus mūsų teisumas, jei laikysime ir vykdysime visus šiuos įsakymus Viešpaties, savo Dievo, akivaizdoje, kaip Jis mums įsakė’ ”.



\chapter{7}


\par 1 “Kai Viešpats, tavo Dievas, nuves tave į žemę, kurią paveldėsi, ir išvys septynias tautas: hetitų, girgašų, amoritų, kanaaniečių, perizų, hivų ir jebusiečių, gausingesnes ir stipresnes už tave, 
\par 2 kai Viešpats, tavo Dievas, atiduos jas tau, išnaikink jas. Nedaryk su jomis sutarčių ir jų nesigailėk. 
\par 3 Nesigiminiuok su jomis vedybomis. Savo dukters neduok jų sūnui nei jų dukters neimk savo sūnui, 
\par 4 nes jos nukreips tavo sūnų nuo manęs tarnauti svetimiems dievams. Viešpaties rūstybė tada užsidegs prieš jus, ir Jis sunaikins tave. 
\par 5 Taigi štai ką jiems darykite: jų aukurus sugriaukite, sudaužykite stabus, iškirskite giraites ir sudeginkite drožinius. 
\par 6 Tu esi Viešpačiui, tavo Dievui, pašvęsta tauta. Viešpats, tavo Dievas, tave išsirinko, kad būtum Jam ypatinga tauta virš visų tautų, esančių žemėje. 
\par 7 Viešpats jus pamilo ir pasirinko ne dėl to, kad buvote gausesnė tauta už kitas tautas, nes jūsų tauta mažesnė už kitas tautas, 
\par 8 bet dėl to, kad Viešpats jus pamilo ir kad ištesėtų jūsų tėvams duotą priesaiką, Jis išvedė jus galinga ranka iš faraono, Egipto karaliaus, vergijos. 
\par 9 Žinok, kad Viešpats, tavo Dievas, yra ištikimas Dievas, kuris laikosi sandoros ir yra gailestingas Jį mylintiems bei Jo įsakymus vykdantiems per tūkstančius kartų. 
\par 10 Ir Jis atlygina tiems, kurie Jo nekenčia, juos sunaikindamas. Jis nedels, kad atlygintų tam, kuris Jo nekenčia. 
\par 11 Laikykis Jo įsakymų, įstatymų ir paliepimų, kuriuos tau šiandien įsakau vykdyti, 
\par 12 tuomet ir Viešpats, tavo Dievas, laikysis sandoros, kurią padarė su tavo tėvais, ir bus tau gailestingas. 
\par 13 Jis mylės tave, laimins tave ir padaugins tave; Jis palaimins tavo įsčios vaisių, ir tavo žemės vaisius: javus, vynuoges, aliejų, galvijus ir avis toje žemėje, kurią pažadėjo tavo tėvams. 
\par 14 Jūs būsite labiausiai palaiminti iš visų tautų, ir nebus nevaisingų tarp jūsų ir tarp jūsų galvijų. 
\par 15 Viešpats pašalins visas tavo negalias ir neužleis baisių Egipto ligų, kurias matei; jas siųs tiems, kurie tavęs nekenčia. 
\par 16 Išnaikink visas tautas, kurias Viešpats, tavo Dievas, tau atiduos. Nesigailėk jų, netarnauk jų dievams, kad jie netaptų tau spąstais. 
\par 17 Nesakyk savo širdyje: ‘Šitos tautos gausesnės už mus, kaip mes jas nugalėsime?’ 
\par 18 Nebijok jų, bet atsimink, ką Viešpats, tavo Dievas, padarė faraonui ir visiems egiptiečiams; 
\par 19 tuos didžius išbandymus, kuriuos tavo akys matė, ženklus, stebuklus ir ištiestą galingą ranką, kuria Viešpats, tavo Dievas, tave išvedė. Taip Viešpats, tavo Dievas, padarys visiems, kurių tu bijai. 
\par 20 Jis siųs jiems širšių, kol išnaikins visus tuos, kurie nuo tavęs pabėgs ir pasislėps. 
\par 21 Nebijokite jų, nes Viešpats, jūsų Dievas, yra tarp jūsų­Dievas, didis ir baisus. 
\par 22 Jis išvys šitas tautas iš tavo akivaizdos vieną po kitos. Tu negalėsi jų sunaikinti iš karto, kad neatsirastų daug laukinių žvėrių prieš tave. 
\par 23 Viešpats, tavo Dievas, atiduos jas tau, naikindamas jas, kol jų nebeliks. 
\par 24 Į tavo rankas Jis atiduos jų karalius, ir tu išnaikinsi jų vardus. Nė vienas tau negalės pasipriešinti, kol juos sunaikinsi. 
\par 25 Jų dievų drožinius sudegink, neimk iš jų sidabro ir aukso, kad nepatektum į spąstus. Tai yra pasibjaurėjimas Viešpačiui, tavo Dievui. 
\par 26 Neįsinešk į savo namus jokio pasibjaurėjimo, kad nebūtum prakeiktas, kaip jis. Bjaurėkis juo ir šalinkis nuo jo, nes tai prakeikta”.



\chapter{8}

\par 1 “Laikykitės įsakymų, kuriuos šiandien jums skelbiu, kad gyventumėte, daugėtumėte ir paveldėtumėte žemę, kurią Viešpats prisiekė duoti jūsų tėvams. 
\par 2 Atsimink kelią, kuriuo Viešpats, tavo Dievas, vedė tave keturiasdešimt metų per dykumą, kad palenktų ir išmėgintų tave, ir sužinotų, kas yra tavo širdyje, ar tu vykdysi Jo paliepimus. 
\par 3 Jis pažemino tave alkiu ir davė valgyti maną, kurios nepažinai nei tu, nei tavo tėvai, kad suprastum, jog žmogus gyvas ne vien duona, bet kiekvienu Dievo žodžiu. 
\par 4 Savo drabužių nesudėvėjai ir tavo kojos nesutino per keturiasdešimt metų. 
\par 5 Suprask, kad kaip žmogus auklėja sūnų, taip Viešpats, tavo Dievas, auklėja tave. 
\par 6 Vykdyk Viešpaties, savo Dievo, paliepimus, vaikščiok Jo keliais ir bijokis Jo. 
\par 7 Viešpats, tavo Dievas, įves tave į gerą, upelių ir vandens šaltinių, kurie trykšta lygumose ir kalnuose, žemę; 
\par 8 į kviečių, miežių ir vynuogynų žemę, kurioje auga figos ir granatai, į žemę alyvmedžių ir medaus, 
\par 9 kurioje nestokosi duonos ir džiaugsiesi derliaus gausumu; į žemę, kurioje akmenys yra geležis, o kalnuose galėsi kasti varį. 
\par 10 Pavalgęs ir būdamas sotus, garbink Viešpatį, savo Dievą, už gerą žemę, kurią Jis tau davė. 
\par 11 Saugokis, kad neužmirštum Viešpaties, savo Dievo, nesilaikydamas Jo paliepimų, įsakymų ir įstatymų, kuriuos tau šiandien skelbiu. 
\par 12 Kai sotus būdamas pasistatysi gerus namus, juose gyvensi 
\par 13 ir turėsi galvijų bandas, avių būrius, aukso ir daugybę visokių dalykų, saugokis, 
\par 14 kad nepasididžiuotum ir neužmirštum Viešpaties, savo Dievo, kuris tave išvedė iš Egipto, iš vergijos namų, 
\par 15 kuris vedė per didelę ir baisią dykumą, kur buvo nuodingų gyvačių ir skorpionų, kur buvo sausra ir nebuvo vandens. Jis leido iš kiečiausios uolos vandens srovei tekėti, 
\par 16 maitino tave dykumoje mana, kurios nepažino tavo tėvai, kad pažemintų tave ir išmėgintų tave ir kad galiausiai darytų tau gera, 
\par 17 kad nesakytum: ‘Mano jėga ir mano rankų stiprybė sukrovė man šiuos turtus’. 
\par 18 Atsimink Viešpatį, savo Dievą, nes Jis tau suteikia jėgų įgyti turtus, kad įtvirtintų sandorą, padarytą su tavo tėvais, kaip yra šiandien. 
\par 19 Bet jei užmirši Viešpatį, savo Dievą, ir seksi svetimus dievus, juos garbinsi ir jiems tarnausi, skelbiu jums iš anksto, kad žūsite. 
\par 20 Kaip žuvo tautos, kurias Viešpats išnaikino jums įeinant, taip ir jūs žūsite, jei neklausysite Viešpaties, jūsų Dievo, balso”.



\chapter{9}


\par 1 “Klausyk, Izraeli! Tu pereisi šiandien per Jordaną, kad nugalėtum didesnes ir stipresnes tautas už save, didelius, iki dangaus sustiprintus miestus, 
\par 2 didelius ir augalotus žmones, Anako sūnus, kuriuos pats matei ir apie kuriuos girdėjai sakant: ‘Niekas jiems negali pasipriešinti’. 
\par 3 Žinok, kad Viešpats, tavo Dievas, eis pirma tavęs kaip ryjanti ugnis ir sunaikins juos, ir parblokš juos prieš tave; taip tu juos išvysi ir sunaikinsi greitai, kaip tau Viešpats pažadėjo. 
\par 4 Kai Viešpats, tavo Dievas, juos išvarys nuo tavęs, nesakyk savo širdyje: ‘Viešpats mane įvedė dėl mano teisumo ir leido paveldėti šitą žemę’. Dėl šitų tautų piktadarysčių Viešpats išvaro jas nuo tavęs. 
\par 5 Ne dėl tavo teisumo ir širdies dorumo įeisi jų žemę paveldėti; jie yra išvaromi dėl jų pikto elgesio, kad Viešpats įvykdytų, ką su priesaika pažadėjo tavo tėvams: Abraomui, Izaokui ir Jokūbui. 
\par 6 Suprask, kad ne dėl tavo teisumo Viešpats, tavo Dievas, duos tau paveldėti šitą žemę, nes tu esi kietasprandė tauta. 
\par 7 Atsimink ir neužmiršk, kaip sukėlei Viešpaties, savo Dievo, pyktį dykumoje. Nuo tos dienos, kai išėjai iš Egipto, iki atėjai į šią vietą, tu maištavai prieš Viešpatį. 
\par 8 Ir prie Horebo Jį įpykinai taip, kad Jis užsirūstinęs norėjo tave sunaikinti. 
\par 9 Aš užlipau ant kalno paimti akmeninių plokščių, plokščių sandoros, kurią Viešpats su jumis padarė, ir pasilikau ten keturiasdešimt parų, nevalgiau duonos ir negėriau vandens. 
\par 10 Viešpats įteikė man dvi akmenines plokštes, ant kurių buvo Dievo pirštu įrašyti žodžiai, kuriuos Jis kalbėjo iš ugnies tautos susirinkimui. 
\par 11 Praėjus keturiasdešimčiai parų, Viešpats davė man dvi akmenines sandoros plokštes 
\par 12 ir tarė: ‘Eik skubiai iš čia, nes tauta, kurią išvedei iš Egipto, pasileido; jie greitai nukrypo nuo kelio, kurį jiems nurodžiau ir pasidirbdino stabą. 
\par 13 Matau, kad šita tauta kietasprandė; 
\par 14 leisk man ją sunaikinti ir išdildyti jos vardą iš po dangaus; iš tavęs padarysiu galingesnę ir didesnę tautą kaip šita’. 
\par 15 Aš leidausi žemyn nuo kalno, kuris degė ugnimi, laikydamas abiejose rankose dvi sandoros plokštes, 
\par 16 ir pamačiau, kad nusidėjote Viešpačiui, savo Dievui, nusiliedinote veršį ir nuklydote nuo kelio, kurį Jis jums nurodė. 
\par 17 Aš trenkiau abi plokštes į žemę ir jas sudaužiau jums matant, 
\par 18 puoliau ant žemės prieš Viešpatį ir, kaip pirma, keturiasdešimt parų nevalgiau duonos ir negėriau vandens dėl jūsų nuodėmių, kuriomis nusidėjote, piktai pasielgdami Viešpaties akivaizdoje ir sukeldami Jo pyktį, 
\par 19 nes bijojau Jo rūstybės, kuria užsidegęs norėjo jus sunaikinti. Viešpats išklausė mane dar ir tą kartą. 
\par 20 Jis buvo labai užsirūstinęs ant Aarono ir norėjo jį nužudyti. Aš meldžiausi ir už Aaroną. 
\par 21 Jūsų nuodėmę, veršį, kurį jūs buvote pasidarę, nutvėręs sudeginau ir, sutrupinęs į gabalėlius, visai sutrynęs į dulkes, įmečiau į upelį, tekantį nuo kalno. 
\par 22 Jūs užrūstinote Viešpatį taip pat Taberoje, Masoje ir Kibrot Taavoje. 
\par 23 Kai Jis iš Kadeš Barnėjos jus pasiuntė, sakydamas: ‘Eikite, užimkite ir paveldėkite žemę, kurią jums daviau’, jūs paniekinote Viešpaties, savo Dievo, įsakymą, netikėjote ir nenorėjote Jo klausyti. 
\par 24 Kiek jus pažįstu, visada maištavote prieš Viešpatį. 
\par 25 Aš kniūbsčias meldžiau Viešpatį keturiasdešimt parų, maldavau ir prašiau nesunaikinti jūsų, kaip buvo grasinęs. 
\par 26 Aš meldžiausi sakydamas: ‘Viešpatie Dieve! Nesunaikink paveldėjimo ir savo tautos, kurią atpirkai savo didybe ir išvedei iš Egipto galinga ranka. 
\par 27 Atsimink savo tarnus Abraomą, Izaoką ir Jokūbą, nežiūrėk šitos tautos užsispyrimo, piktadarysčių ir nuodėmių, 
\par 28 kad krašto, iš kurio mus išvedei, gyventojai nesakytų: ‘Kadangi Viešpats negalėjo jų įvesti į pažadėtąją žemę ir kadangi Jis jų nekenčia, Jis išvedė juos pražudyti dykumoje’. 
\par 29 Tačiau jie yra Tavo paveldėjimas ir Tavo tauta, kurią išvedei iš Egipto savo galia ir savo ištiesta ranka’ ”.



\chapter{10}

\par 1 “Viešpats man įsakė: ‘Išsikirsdink dvi akmenines plokštes, kaip pirmosios buvo, ir ateik pas mane ant kalno. Padirbdink medinę skrynią. 
\par 2 Aš ant plokščių užrašysiu žodžius, buvusius sudaužytose plokštėse. Plokštes įdėk į skrynią’. 
\par 3 Iš akacijos medžio padirbdinau skrynią, iškirsdinau dvi akmenines plokštes, kaip pirmosios buvo, ir užkopiau į kalną, laikydamas plokštes rankose. 
\par 4 Viešpats tose plokštėse įrašė, kaip ir pirma, dešimt įsakymų, kuriuos Viešpats jums kalbėjo iš ugnies, kai tauta buvo susirinkusi, ir padavė jas man. 
\par 5 Sugrįžęs nuo kalno, įdėjau tas dvi plokštes į skrynią, kurią buvau padirbdinęs, kaip man Viešpats buvo įsakęs, ir jos ten yra. 
\par 6 Izraelitai keliavo iš Beroto Bene Jaakano į Moserą. Aaronas čia mirė ir buvo palaidotas. Jo vietoje kunigo tarnystę pradėjo eiti jo sūnus Eleazaras. 
\par 7 Iš ten jie atvyko į Gudgodą, o iš čia­ į Jotbatą, upelių žemę. 
\par 8 Viešpats paskyrė Levio giminę Sandoros skryniai nešti, būti Jo akivaizdoje, tarnauti Jam ir laiminti Jo vardu; taip daroma iki šios dienos. 
\par 9 Levitai negavo paveldėjimo dalies, nes Viešpats yra jų dalis­ taip Viešpats, tavo Dievas, pasakė. 
\par 10 Aš pasilikau ant kalno, kaip ir pirmą kartą, keturiasdešimt parų. Viešpats išklausė mane ir nesunaikino tavęs. 
\par 11 Jis man įsakė eiti tautos priekyje ir įvesti juos į kraštą, kurį tėvams su priesaika pažadėjo. 
\par 12 O dabar, Izraeli, ko Viešpats, tavo Dievas, iš tavęs reikalauja? Tik kad bijotum Viešpaties, savo Dievo, vaikščiotum Jo keliais, Jį mylėtum ir Jam tarnautum visa savo širdimi ir visa savo siela; 
\par 13 kad laikytumeisi Viešpaties įsakymų ir įstatymų, kuriuos tau šiandien skelbiu tavo labui. 
\par 14 Viešpačiui, tavo Dievui, priklauso dangūs ir žemė bei visa, kas joje yra. 
\par 15 Viešpats pamėgo tavo tėvus, juos pamilo ir pasirinko iš visų tautų jų palikuonis, tai yra jus, kaip tai yra šiandien. 
\par 16 Apipjaustykite todėl savo širdis ir nebūkite kietasprandžiai. 
\par 17 Viešpats, jūsų Dievas, yra dievų Dievas ir viešpačių Viešpats, didis, galingas ir baisus Dievas; Jis neatsižvelgia į asmenis ir nepaperkamas kyšiais. 
\par 18 Jis teisingai elgiasi su našlaičiais ir našlėmis, myli ateivį, duodamas jam maisto ir drabužių. 
\par 19 Jūs irgi mylėkite ateivius, nes patys buvote ateiviai Egipto žemėje. 
\par 20 Bijok Viešpaties, savo Dievo, Jam vienam tarnauk, glauskis prie Jo ir Jo vardu prisiek. 
\par 21 Jis yra tavo garbė ir tavo Dievas, kuris padarė visus šiuos didelius ir baisingus dalykus, kuriuos tu matei savo akimis. 
\par 22 Tavo tėvų septyniasdešimt asmenų nukeliavo į Egiptą; ir štai dabar Viešpats, tavo Dievas, padaugino tave kaip dangaus žvaigždes”.



\chapter{11}


\par 1 “Mylėk Viešpatį, savo Dievą, visada vykdyk Jo įsakymus, įstatymus, nuostatus bei paliepimus. 
\par 2 Kalbu ne jūsų vaikams, kurie nematė ir nepažino Viešpaties, jūsų Dievo, drausmės ir Jo didybės, ir Jo ištiestos galingos rankos, 
\par 3 Jo ženklų ir darbų, padarytų faraonui, Egipto karaliui, ir visam jo kraštui, 
\par 4 visai egiptiečių kariuomenei, žirgams ir kovos vežimams, kai juos užliejo Raudonosios jūros vanduo, besivejant jus, ir Viešpats juos sunaikino, 
\par 5 ir kaip Jis jums padėjo dykumoje, iki atėjote į šitą vietą, 
\par 6 ir ką Jis padarė Rubeno sūnaus Eliabo dviem sūnums Datanui ir Abiramui, kuriuos atsivėrusi žemė prarijo su šeimomis, palapinėmis ir visu jų lobiu. 
\par 7 Jūsų akys matė visus didingus Viešpaties darbus, kuriuos Jis padarė. 
\par 8 Vykdykite visus Jo paliepimus, kuriuos jums šiandien skelbiu, kad būtumėte tvirti užimti ir paveldėti žemę, į kurią einate, 
\par 9 ir kad ilgai gyventumėte žemėje, plūstančioje pienu ir medumi, kurią Viešpats su priesaika pažadėjo jūsų tėvams ir jų palikuonims. 
\par 10 Žemė, kurios paveldėti einate, yra ne tokia kaip Egipto žemė, iš kurios išėjote, kurioje pasėtą sėklą laistėte savo rankomis atneštu vandeniu kaip daržą. 
\par 11 Žemė, į kurią einate, yra kalnų ir slėnių šalis, kurią laisto dangaus lietus, 
\par 12 Viešpats, jūsų Dievas, visada aprūpina ją, Jo akys stebi ją ištisus metus. 
\par 13 Jei klausysite mano įsakymų, kuriuos šiandien jums duodu, mylėsite Viešpatį, savo Dievą, ir Jam tarnausite visa širdimi ir visa siela, 
\par 14 Jis duos žemei lietaus reikiamu metu: ankstyvąjį lietų ir vėlyvąjį lietų; suvalysi javus, vynuoges ir aliejų. 
\par 15 Jis duos tavo gyvuliams žolės laukuose; ir tu valgysi, ir pasisotinsi. 
\par 16 Saugokitės, kad jūsų širdys nebūtų apgautos, nenuklyskite, netarnaukite svetimiems dievams ir jų negarbinkite. 
\par 17 Nes tada užsirūstinęs Viešpats uždarys dangų, kad nebebūtų lietaus, ir žemė neduos derliaus, ir jūs greitai pražūsite toje geroje žemėje, kurią Viešpats jums duoda. 
\par 18 Įsidėkite šiuos žodžius į savo širdį ir sielą, nešiokite juos kaip ženklą ant rankų ir ant kaktos. 
\par 19 Mokykite jų savo vaikus, kalbėkite apie juos, sėdėdami namuose, eidami keliu, guldami ir keldamiesi. 
\par 20 Užrašykite juos ant savo namų durų staktų ir ant vartų, 
\par 21 kad jūs ir jūsų vaikai gyventų krašte, kurį Viešpats jūsų tėvams prisiekė duoti, kol dangus bus virš žemės. 
\par 22 Jei uoliai laikysitės šių paliepimų, juos vykdysite, mylėsite Viešpatį, savo Dievą, vaikščiosite Jo keliais ir glausitės prie Jo, 
\par 23 Viešpats išvys visas šitas tautas pirma jūsų. Jūs nugalėsite jas, nors jos už jus didesnės ir galingesnės. 
\par 24 Kiekviena vieta, kur jūsų koja atsistos, bus jūsų. Nuo dykumų ir Libano, nuo didžiosios Eufrato upės iki Vakarų jūros tęsis jūsų sienos. 
\par 25 Nė vienas neatsilaikys prieš jus; Viešpats, jūsų Dievas, sukels baimę ir išgąstį visuose kraštuose, į kuriuos tik įžengsite, kaip jums pažadėjo. 
\par 26 Štai aš padedu prieš jus šiandien palaiminimą ir prakeikimą. 
\par 27 Palaiminimą­jei klausysite Viešpaties, savo Dievo, įsakymų, kuriuos šiandien jums skelbiu; 
\par 28 prakeikimą­jei neklausysite Viešpaties, savo Dievo įsakymų, nukrypsite nuo kelio, kurį šiandien jums rodau, ir seksite svetimus dievus, kurių nepažįstate. 
\par 29 Kai Viešpats, jūsų Dievas, įves jus į žemę, kurią dabar einate užimti, paskelbkite palaiminimą nuo Garizimo, o prakeikimą­ nuo Ebalo kalno, 
\par 30 kurie yra vakarinėje Jordano pusėje, kanaaniečių žemėje, ties Gilgalu, slėnyje prie Morės ąžuolo. 
\par 31 Jūs pereisite per Jordaną užimti žemes, kurias Viešpats, jūsų Dievas, jums duos, ir jūs užimsite jas ir apsigyvensite jose. 
\par 32 Vykdykite įsakymus ir paliepimus, kuriuos šiandien skelbiu jums”.



\chapter{12}


\par 1 “Šitie yra įsakymai ir paliepimai, kuriuos privalote vykdyti žemėje, kurią Viešpats, jūsų tėvų Dievas, jums duoda; vykdykite juos, kol gyvensite žemėje. 
\par 2 Sugriaukite nugalėtų tautų visas dievų garbinimo vietas: kalnuose, kalvose ir po žaliuojančiais medžiais. 
\par 3 Išardykite jų aukurus, sutrupinkite stabus, sudeginkite šventąsias giraites, sudaužykite jų drožtus dievų atvaizdus ir išnaikinkite jų vardus iš tų vietų. 
\par 4 Viešpačiui, savo Dievui, to nedarysite! 
\par 5 Eikite į vietą, kurią Viešpats, jūsų Dievas, išsirinks tarp jūsų giminių, kad ten būtų garbinamas Jo vardas. 
\par 6 Toje vietoje aukokite deginamąsias ir kitas aukas, dešimtines ir pirmųjų vaisių aukas, įžadų ir laisvos valios aukas, jaučių ir avių pirmagimius. 
\par 7 Ten valgykite Viešpaties, jūsų Dievo, akivaizdoje, džiaukitės viskuo, ką įsigysite jūs ir jūsų šeimos, kuo Viešpats, jūsų Dievas, jus palaimins. 
\par 8 Tada nedarysite to, ką mes čia šiandien darome, kiekvienas kas jam atrodo teisinga; 
\par 9 nes dar neįėjote į poilsį ir paveldą, kurį Viešpats, jūsų Dievas, jums duoda. 
\par 10 Bet kai pereisite per Jordaną ir apsigyvensite žemėje, kurią Viešpats, jūsų Dievas, jums duoda, ir kai turėsite taiką ir gyvensite ramybėje, 
\par 11 tada į tą vietą, kurią Viešpats, jūsų Dievas, išsirinks savo vardui, atnešite visa, ką įsakiau: deginamąsias aukas, dešimtines, jūsų pirmųjų vaisių aukas ir visa, ką būsite pažadėję Viešpačiui. 
\par 12 Ten Viešpaties, jūsų Dievo, akivaizdoje, džiaugsitės jūs, jūsų sūnūs ir dukterys, tarnai ir tarnaitės, taip pat ir levitai, gyvenantys jūsų miestuose, nes jie neturi jokios dalies nei paveldėjimo tarp jūsų. 
\par 13 Neaukok deginamųjų aukų kiekvienoje vietoje, kurią pamatai, 
\par 14 bet tik Viešpaties pasirinktoje vietoje. Ten aukok aukas ir daryk viską, ką tau įsakau. 
\par 15 Jei norėsi valgyti, pasipjauk gyvulį ir valgyk Viešpačiui, tavo Dievui, laiminant ten, kur tu gyveni. Ten galės valgyti švarus ir nešvarus taip, kaip leista valgyti stirną ir briedį. 
\par 16 Tačiau kraujo nevalgykite, jį išliekite žemėn kaip vandenį. 
\par 17 Savo miestuose negali valgyti javų, vyno ir aliejaus dešimtinių, galvijų ir avių pirmagimių, viso, ką pažadėsi aukoti ar norėsi aukoti laisva valia, taip pat pirmųjų vaisių. 
\par 18 Tai turi valgyti Viešpaties, savo Dievo, akivaizdoje toje vietoje, kurią Viešpats, tavo Dievas, išsirinks: tu, tavo sūnus ir duktė, tarnas ir tarnaitė, taip pat ir levitas, gyvenantis su tavimi. Džiaukis Viešpaties, savo Dievo, akivaizdoje viskuo, ką padarei. 
\par 19 Niekada neužmiršk levito savo žemėje. 
\par 20 Kai Viešpats, tavo Dievas, išplės sienas, kaip pažadėjo, ir tu norėsi valgyt mėsos, gali valgyti mėsos, kokios tik tavo siela geidžia. 
\par 21 Jei vieta, kurią Viešpats, tavo Dievas, išsirinks savo vardui, bus toli, pasipjauk galvijų arba avių, kuriuos Viešpats tau davė, kaip įsakiau, ir valgyk savo vietoje, ko tavo siela geidžia. 
\par 22 Kaip valgoma stirna ir briedis, taip valgykite visi, švarūs ir nešvarūs. 
\par 23 Atidžiai žiūrėk, kad nevalgytum kraujo, nes kraujas yra gyvybė; neleistina valgyti gyvybės drauge su mėsa, 
\par 24 išpilk jį žemėn kaip vandenį. 
\par 25 Nevalgyk jo, kad gerai sektųsi tau ir tavo vaikams, kai darysi, kas patinka Viešpačiui. 
\par 26 Ką pašvęsi ir pažadėsi Viešpačiui, atnešk į vietą, kurią Viešpats išsirinks. 
\par 27 Ten aukok deginamąsias aukas, mėsą ir kraują ant Viešpaties, tavo Dievo, aukuro: aukų kraujas bus išlietas ant aukuro, o mėsą valgyk. 
\par 28 Laikykis visko, ką įsakiau, kad gerai sektųsi tau ir tavo vaikams per amžius, kai darysi, kas gera ir kas patinka Viešpačiui, tavo Dievui. 
\par 29 Kai Viešpats, tavo Dievas, išnaikins tautas, kurių žemės eini užimti, ir kai jas nugalėjęs, gyvensi jų žemėje, 
\par 30 žiūrėk, kad nepatektum į spąstus sekdamas jomis, kai jos bus pirma tavęs išnaikintos, kad neieškotum tų tautų dievų, sakydamas: ‘Kaip šios tautos tarnavo savo dievams, taip ir aš tarnausiu’. 
\par 31 Nedaryk taip Viešpačiui, savo Dievui. Nes jos darė savo dievams tai, kas bjauru Viešpaties akyse ir tai, ko Jis nekenčia; net savo sūnus ir dukteris jos degindavo savo dievams. 
\par 32 Ką įsakau, daryk: nieko nepridėk ir neatimk”.



\chapter{13}


\par 1 “Jei tarp jūsų iškiltų pranašas ar atsirastų sapnuotojas ir paskelbtų kokį ženklą ar stebuklą, 
\par 2 kuris įvyktų, ir po to tau sakytų: ‘Sekime dievus, kurių tu nepažįsti, ir jiems tarnaukime’, 
\par 3 tokio pranašo ar sapnuotojo neklausyk, nes Viešpats, jūsų Dievas, jus bando, kad paaiškėtų, ar Jį mylite visa širdimi ir visa siela. 
\par 4 Sekite Viešpatį, jūsų Dievą, Jo bijokite, vykdykite Jo paliepimus ir klausykite Jo balso; Jam tarnaukite ir prie Jo glauskitės. 
\par 5 Tokį pranašą ar sapnuotoją užmuškite; jis kvietė jus atsitraukti nuo Viešpaties, jūsų Dievo, kuris jus išvedė iš Egipto žemės ir išpirko iš vergijos; jis bandė nukreipti jus nuo kelio, kurį Viešpats, jūsų Dievas, nurodė. Pašalinkite pikta iš savųjų tarpo. 
\par 6 Jei tavo brolis, sūnus, duktė, mylima žmona ar draugas, kurį myli kaip savo sielą, norėtų suvedžioti tave, sakydamas: ‘Eikime ir tarnaukime svetimiems dievams, kurių nepažinai nei tu, nei tavo tėvai’, 
\par 7 dievams tautų, kurios gyvena šalia tavęs ar toli nuo tavęs, nuo vieno žemės krašto iki kito, 
\par 8 nesutik su juo ir jo neklausyk. Nesigailėk ir neslėpk jo, 
\par 9 bet jį užmušk. Pirmas pakelk prieš jį ranką, kad nubaustum jį mirtimi, o paskui visa tauta prisidės. 
\par 10 Užmuškite jį akmenimis, nes jis norėjo jus atitraukti nuo Viešpaties, jūsų Dievo, kuris išvedė jus iš Egipto vergijos. 
\par 11 Visas Izraelis išgirs tai ir bijos, ir nebesielgs taip piktai tarp jūsų. 
\par 12 Jei išgirstum apie vieną iš savo miestų, kuriuos Viešpats, tavo Dievas, tau duos, kalbant: 
\par 13 ‘Tarp jūsų atsirado suvedžiotojai, kurie suklaidino miesto gyventojus, sakydami: ‘Eikime ir tarnaukime svetimiems dievams, kurių nepažįstate’, 
\par 14 rūpestingai ištirk bei išsiklausinėk, ir jei pasitvirtintų, kad tarp jūsų padaryta tokia bjaurystė, 
\par 15 išžudyk to miesto gyventojus kardu ir sunaikink jį ir visa, kas jame yra, net gyvulius. 
\par 16 Visus daiktus sukrauk aikštėje ir sudegink kartu su miestu Viešpačiui, savo Dievui. Jis telieka griuvėsiais per amžius ir niekados tenebūna atstatytas. 
\par 17 Iš tų prakeiktų daiktų nieko nepasisavink, kad Viešpats atsisakytų savo rūstybės, pasigailėtų tavęs ir padaugintų tave, kaip prisiekė tavo tėvams. 
\par 18 Klausyk Viešpaties, savo Dievo, ir laikykis visų Jo įsakymų, kuriuos šiandien tau skelbiu. Daryk, kas patinka Viešpačiui, tavo Dievui”.



\chapter{14}

\par 1 “Jūs esate Viešpaties, jūsų Dievo, vaikai: nedarykite įsipjovimų ir nesiskuskite plikai viršugalvio dėl mirusio, 
\par 2 nes esate šventi žmonės Viešpačiui, jūsų Dievui. Jis jus išsirinko iš visų žemėje esančių tautų būti Jo tauta. 
\par 3 Nevalgyk nieko pasibjaurėtino. 
\par 4 Štai gyvuliai, kuriuos leidžiama valgyti: jautis, avis, ožka, 
\par 5 briedis, stirna, stumbras, laukinė ožka, antilopė, gazelė, kalnų ožka. 
\par 6 Valgykite kiekvieną gyvulį, kuris turi skeltą nagą ir gromuliuoja. 
\par 7 Tų, kurie gromuliuoja, bet turi neskeltą nagą, nevalgykite: kupranugario, kiškio, barsuko­jie nešvarūs. 
\par 8 Kiaulė jums yra nešvari, nors ji turi skeltą nagą, bet negromuliuoja. Jos mėsos nevalgykite ir jos maitos nepalieskite. 
\par 9 Iš vandens gyvių valgykite tuos, kurie turi pelekus ir žvynus; 
\par 10 o tų, kurie be pelekų ir žvynų, nevalgykite­jie nešvarūs. 
\par 11 Valgykite visus švarius paukščius, 
\par 12 bet šių nevalgykite: erelio, grifo, jūros erelio, 
\par 13 lingės, vanagėlio, peslio su visa jo gimine, 
\par 14 varnų giminės, 
\par 15 stručio, pelėdos, žuvėdros, vanago su visa jo gimine, 
\par 16 gervės, gulbės, ibio, 
\par 17 naro, kormorano, apuoko, 
\par 18 pelikano, kėkšto ir visos jo giminės, taip pat tutlio ir šikšnosparnio. 
\par 19 Visi sparnuoti vabzdžiai nešvarūs ir nevalgomi. 
\par 20 Visus švarius sparnuočius valgykite. 
\par 21 Nieko pastipusio nevalgykite. Ateivis, gyvenantis pas jus, gali juos valgyti; svetimšaliui gali tokius gyvulius parduoti, bet tu esi Viešpačiui, savo Dievui, pašvęsta tauta. Nevirk ožiuko jo motinos piene. 
\par 22 Duok dešimtinę savo derliaus kiekvienais metais. 
\par 23 Dešimtinę javų, vyno bei aliejaus ir galvijų bei avių pirmagimius valgyk Viešpaties, savo Dievo, akivaizdoje toje vietoje, kurią Jis išsirinks savo vardui, kad išmoktum bijotis Viešpaties, savo Dievo. 
\par 24 Jei vieta, kurią Viešpats, tavo Dievas, išsirinks savo vardui, bus labai toli ir negalėsi ten nugabenti dešimtinių iš to, kuo Viešpats tave palaimino, 
\par 25 tai viską parduok ir pinigus nusinešk į vietą, kurią Viešpats, tavo Dievas, išsirinks. 
\par 26 Už tuos pinigus nusipirk, kas tau patiks: galvijų, avių, vyno bei stiprių gėrimų ir visa, ko tik norėsi; valgyk tai su visa savo šeima Viešpaties, savo Dievo, akivaizdoje ir džiaukis. 
\par 27 Neužmiršk pas tave gyvenančio levito, nes jis neturi dalies paveldėtoje žemėje. 
\par 28 Kas treji metai atnešk dešimtinę tų metų derliaus į miesto sandėlius. 
\par 29 Tai bus maistas levitui, neturinčiam dalies paveldėjimui, taip pat ateiviui, našlaičiui ir našlei, kurie gyvena tavo apylinkėje, kad Viešpats, tavo Dievas, laimintų visus tavo darbus, kuriuos darysi”.



\chapter{15}

\par 1 “Kas septintieji metai­atleidimo metai. 
\par 2 Tais metais kiekvienas skolintojas tedovanoja skolininkui skolas, ar jis būtų artimas, ar brolis, nes tai Viešpaties atleidimo metai. 
\par 3 Iš svetimšalio gali reikalauti, bet savo broliui turi dovanoti. 
\par 4 Nebent kai nebus tarp jūsų suvargusio žmogaus, nes Viešpats Dievas gausiai palaimins tave žemėje, kurią duoda tau paveldėti. 
\par 5 Jei atidžiai klausysi Viešpaties, savo Dievo, balso, ir vykdysi įsakymus, kuriuos paskelbiau šiandien, 
\par 6 Jis tave laimins, kaip pažadėjo. Tu skolinsi daugeliui tautų, bet nieko neimsi skolon, viešpatausi daugeliui tautų, o tau niekas neviešpataus. 
\par 7 Jei kuris tavo brolių, gyvenančių žemėje, kurią Viešpats Dievas tau duos, suvargtų, neužkietink širdies ir neužgniaužk prieš jį savo rankos, 
\par 8 bet atverk jam plačiai savo ranką ir skolink jam, kiek reikia patenkinti jo poreikiui. 
\par 9 Žiūrėk, kad pikta mintis neįeitų į tavo širdį: ‘Nebetoli septintieji, atleidimo metai’, ir neatstumtum beturčio brolio, nenorėdamas jam skolinti, kad jis nesišauktų Viešpaties prieš tave ir tau nebūtų nuodėmės. 
\par 10 Tu duosi jam ir neliūdėsi duodamas, nes už tai Viešpats, tavo Dievas, palaimins tave visuose tavo darbuose ir kur tu bepridėtum savo ranką. 
\par 11 Beturčių visuomet bus krašte, todėl įsakau ištiesti ranką broliui ir beturčiui, kuris gyvena tavo šalyje. 
\par 12 Jei tau parsiduos tavo brolis hebrajas, vyras ar moteris, ir ištarnaus tau šešerius metus, septintais metais paleisk jį. 
\par 13 Suteikęs jam laisvę, neišleisk jo tuščiomis rankomis. 
\par 14 Aprūpink jį iš bandos, aruodo ir vyno spaustuvo; iš to, kuo Viešpats Dievas tave palaimino, duok jam. 
\par 15 Atsimink, kad vergavai Egipto žemėje, o Viešpats Dievas išlaisvino tave. Todėl šiandien tau tai įsakau. 
\par 16 Bet jei vergas sakytų: ‘Nenoriu išeiti, nes myliu tave bei tavo namus’, nes jam yra gerai pas tave, 
\par 17 tada imk ylą ir perverk jo ausį prie namo durų; jis liks tavo vergu visam laikui. Su tarnaite pasielk taip pat. 
\par 18 Tenebūna tau sunku paleisti jį į laisvę, nes jis buvo vertas dviejų samdinių, tarnaudamas tau šešerius metus; ir Viešpats Dievas laimins visus tavo darbus. 
\par 19 Kiekvieną galvijų ir avių pirmagimį patinėlį pašvęsk Viešpačiui, savo Dievui. Nenaudok galvijų pirmagimių darbui ir nekirpk avių pirmagimių. 
\par 20 Tu ir tavo šeima valgykite juos Viešpaties, savo Dievo, akivaizdoje kiekvienais metais toje vietoje, kurią Viešpats išsirinks. 
\par 21 Jei jis turėtų kliaudą, būtų raišas ar aklas, ar turėtų kitokį trūkumą, tokio neaukok Viešpačiui, savo Dievui. 
\par 22 Tokius valgyk namuose. Juos gali valgyti švarus ir nešvarus žmogus, kaip valgo stirną ar briedį. 
\par 23 Tik nevalgyk jų kraujo, bet jį išliek žemėn kaip vandenį”.



\chapter{16}

\par 1 “Švęsk Paschą Viešpačiui Abibo mėnesį, nes tą mėnesį Viešpats Dievas išvedė tave iš Egipto. 
\par 2 Aukok Viešpačiui Dievui Paschos auką iš avių ir jaučių toje vietoje, kurią Viešpats, tavo Dievas, išsirinks savo vardui. 
\par 3 Nevalgyk jos su rauginta duona; septynias dienas valgyk sielvarto duonos be raugo, nes skubiai turėjai išeiti iš Egipto, kad per visą savo gyvenimą atsimintum dieną, kurią išėjai iš Egipto. 
\par 4 Septynias dienas neturi būti raugo visame tavo krašte; nepalik rytojui mėsos, aukotos pirmos dienos vakare. 
\par 5 Paschos aukos neaukok savo gyvenamoje vietoje, kurią Viešpats, tavo Dievas, tau duos. 
\par 6 Aukok ją toje vietoje, kurią Viešpats, tavo Dievas, išsirinks savo vardui. Paschą aukok vakare, saulei leidžiantis, tuo metu, kai išėjai iš Egipto. 
\par 7 Kepk ir valgyk toje vietoje, kurią Viešpats, tavo Dievas, išsirinks, o rytą atsikėlęs grįžk į savo palapinę. 
\par 8 Šešias dienas valgyk neraugintą duoną, o septinta diena yra iškilmingas susirinkimas Viešpačiui, tavo Dievui; tą dieną nedirbk jokio darbo. 
\par 9 Atskaityk septynias savaites nuo tos dienos, kai pradėsi pjauti javus, 
\par 10 ir švęsk Viešpačiui, savo Dievui Savaičių šventę, laisva valia atnešdamas aukų iš to, kuo Viešpats, tavo Dievas, laimino tave. 
\par 11 Viešpaties, tavo Dievo, akivaizdoje džiaukis tu, tavo sūnus, duktė, tarnas, tarnaitė, levitas, gyvenantis tavo apylinkėje, ateivis, našlaitis ir našlė, kurie su tavimi gyvena, toje vietoje, kurią Viešpats, tavo Dievas, išsirinko savo vardui. 
\par 12 Atsimink, kad buvai vergas Egipte, todėl laikykis šių nuostatų. 
\par 13 Palapinių šventę švęsk septynias dienas, kai būsi suvalęs javų ir vynuogių derlių. 
\par 14 Džiaukis šventėje tu, tavo sūnus, duktė, tarnas, tarnaitė, ateivis, našlaitis bei našlė, kurie gyvena tavo apylinkėje. 
\par 15 Septynias dienas švęsk iškilmingą šventę Viešpačiui, savo Dievui, toje vietoje, kurią Viešpats išsirinks. Tu iš tiesų džiaugsiesi, nes Viešpats, tavo Dievas, laimins visą tavo derlių ir visus darbus. 
\par 16 Kiekvienas vyras privalo pasirodyti tris kartus per metus: Neraugintos duonos, Savaičių ir Palapinių šventėse, toje vietoje, kurią Viešpats išsirinks, ir jie tenepasirodo Viešpaties akivaizdoje tuščiomis rankomis. 
\par 17 Kiekvienas teaukoja tą, ką gali, pagal tai, kaip Viešpats, tavo Dievas, tave palaimino. 
\par 18 Visose savo gyvenvietėse, kurias Viešpats, tavo Dievas, tau duos, kiekvienoje giminėje paskirsi teisėjus ir valdininkus, kad jie teisingai teistų žmones. 
\par 19 Neiškraipyk teisingumo. Nepataikauk ir neimk kyšių, nes kyšiai apakina išmintinguosius ir iškraipo teisiųjų žodžius. 
\par 20 Sek tuo, kas tikrai teisinga, kad gyventum ir paveldėtum žemę, kurią Viešpats, tavo Dievas, tau duoda. 
\par 21 Nesodink alko iš jokių medžių šalia Viešpaties, tavo Dievo, aukuro, kurį pastatysi. 
\par 22 Nesistatyk sau jokio atvaizdo, nes Viešpats, tavo Dievas, to nekenčia”.



\chapter{17}

\par 1 “Neaukok Viešpačiui, savo Dievui, avies nei jaučio su trūkumu ar kokia nors kliauda, nes tai yra pasibjaurėjimas Viešpačiui, tavo Dievui. 
\par 2 Jei tavo krašte, kurį Viešpats, tavo Dievas, tau duos, būtų vyras ar moteris, kuris nusikalto Viešpačiui, tavo Dievui, sulaužydamas sandorą 
\par 3 ir tarnaudamas svetimiems dievams bei garbindamas juos arba saulę, mėnulį ar kitą dangaus kūną, ko aš jums neįsakiau daryti, 
\par 4 ir apie tai tau bus pranešta, tai rūpestingai ištirk, ar taip iš tikrųjų yra. Jei padarytas toks nusikaltimas Izraelyje, 
\par 5 išvesk vyrą ar moterį, padariusį tą bjaurų nusikaltimą, prie miesto vartų ir užmušk akmenimis. 
\par 6 Bausi mirtimi tą, prieš kurį liudys du ar trys liudytojai. Nė vienas nebus baudžiamas mirtimi, liudijant prieš jį tik vienam. 
\par 7 Liudytojai pirmieji mes akmenį, o tada prisidės ir likusieji. Taip pikta bus pašalinta iš jūsų tarpo. 
\par 8 Jei kuri byla bus tau per sunki­dėl kraujo, išteisinimo ar sumušimo,­ir sukels ginčus tavo vartuose, atsikelk ir eik į tą vietą, kurią Viešpats, tavo Dievas išsirinks. 
\par 9 Kreipkis į kunigus iš Levio giminės ir į teisėją, kuris bus tuo metu, ir jie nurodys tau teisingą sprendimą. 
\par 10 Daryk, ką jie įsakys ir kaip pamokys. 
\par 11 Elkis pagal jų įsakymus ir nurodymus, nenukrypk nuo jų. 
\par 12 Kas įžūliai nepaklustų kunigui, kuris tarnauja Viešpačiui, tavo Dievui, ar teisėjui, toks asmuo temiršta. Taip pašalinsi pikta iš Izraelio. 
\par 13 Visa tauta išgirs, bijos ir daugiau įžūliai nesielgs. 
\par 14 Kai būsi žemėje, kurią Viešpats, tavo Dievas, tau duos, ją paveldėsi, gyvensi joje ir norėsi išsirinkti sau karalių, kaip visos tautos aplinkui, 
\par 15 tada privalai įstatyti karaliumi tą, kurį Viešpats, tavo Dievas, pasirinks iš tavo brolių. Karaliumi negali būti svetimšalis, kuris nėra tavo brolis. 
\par 16 Jis neturės daug žirgų ir neves tautos atgal į Egiptą daugiau žirgų įsigyti; nes Viešpats įsakė niekuomet tuo pačiu keliu nebegrįžti. 
\par 17 Jis neturės daug žmonų, kad jo širdis nenukryptų, ir nekraus sau sidabro ir aukso. 
\par 18 Atsisėdęs į karalystės sostą, jis pasidarys šito įstatymo nuorašą iš Levio giminės kunigų, 
\par 19 laikys jį pas save ir skaitys visas savo gyvenimo dienas, kad išmoktų bijoti Viešpaties, savo Dievo, ir laikytis žodžių bei paliepimų, surašytų įstatyme, 
\par 20 kad jo širdis nepasikeltų į puikybę prieš brolius ir jis nenukryptų nuo įstatymo nei į kairę, nei į dešinę, kad ilgai karaliautų jis ir jo vaikai Izraelyje”.



\chapter{18}


\par 1 “Kunigai ir visa Levio giminė neturės dalies ir paveldėjimo kaip visi izraelitai. Aukos Viešpačiui bus jų dalis. 
\par 2 Jie nepaveldės dalies tarp savo brolių. Viešpats bus jų dalis, kaip Jis jiems pasakė. 
\par 3 Kunigams priklauso iš tautos tokia aukos dalis: aukojant jautį ar aviną, kunigui tenka petys, abu žandikauliai ir skrandis, 
\par 4 taip pat pirmienos javų, vyno, aliejaus ir avių vilnų. 
\par 5 Iš visų tavo giminių Viešpats, tavo Dievas, išsirinko Levio giminę tarnauti Jam per amžius. 
\par 6 Jei levitas, gyvenęs tavo apylinkėje, ateitų savo noru į Viešpaties pasirinktą vietą, 
\par 7 jis tarnaus Viešpačiui, savo Dievui, kaip visi jo broliai levitai, kurie ten būna Viešpaties akivaizdoje. 
\par 8 Jis gaus tokią pat dalį kaip ir kiti, neskaitant to, ką gavo pardavęs tėviškę. 
\par 9 Kai įeisi į žemę, kurią Viešpats, tavo Dievas, tau duos, neišmok daryti tų tautų bjaurysčių. 
\par 10 Nebus tarp jūsų tokių, kurie leistų savo sūnų ar dukterį per ugnį, nei ateities spėjėjų, nei ženklų aiškintojų, nei kerėtojų, nei burtininkų, 
\par 11 nei žavėtojų, nei mirusiųjų dvasių iššaukėjų, nei žynių, nei raganių. 
\par 12 Visi, kurie taip daro, yra pasibjaurėjimas Viešpačiui, ir už tokias bjaurystes Jis išnaikins tas tautas, prieš tau užimant kraštą. 
\par 13 Būk tobulas prieš Viešpatį, savo Dievą. 
\par 14 Tautos, kurių žemę paveldėsi, klauso ženklų aiškintojų ir žynių patarimų, o tau Viešpats, tavo Dievas, ne taip skyrė. 
\par 15 Viešpats, tavo Dievas, pakels pranašą iš tavo brolių kaip mane­jo klausykite! 
\par 16 Kai prašei Viešpaties, savo Dievo, prie Horebo, sakydamas: ‘Nebenoriu daugiau girdėti Viešpaties, savo Dievo, balso ir matyti šios baisios ugnies, kad nemirčiau’, 
\par 17 Viešpats man tarė: ‘Jie teisingai kalbėjo. 
\par 18 Aš pakelsiu iš jų brolių pranašą, panašų į tave, ir įdėsiu savo žodžius į jo lūpas. Jis kalbės jiems, ką jam įsakysiu. 
\par 19 Kas nenorės paklusti mano žodžiams, kuriuos jis kalbės mano vardu, iš to išieškosiu. 
\par 20 Pranašas, kuris drįstų kalbėti mano vardu, ko Aš jam neliepiau, ar svetimų dievų vardu, bus baudžiamas mirtimi’. 
\par 21 Jei sakytum: ‘Kaip mums pažinti žodį, kurį Viešpats kalbėjo?’ 
\par 22 Ženklas bus toks: jei pranašas paskelbtų ką nors iš anksto Viešpaties vardu ir tai neįvyktų, tam Viešpats nekalbėjo, bet pranašas kalbėjo iš savo pasipūtimo, ir todėl jo nebijok”.



\chapter{19}


\par 1 “Kai Viešpats, tavo Dievas, išnaikins tautas, ir jų žemę duos tau paveldėti ir gyventi jų miestuose bei namuose, 
\par 2 išskirk tame krašte, kurį Viešpats, tavo Dievas, tau duoda, tris miestus, 
\par 3 nutiesk į juos kelią ir padalyk į tris dalis visą kraštą, kad žmogžudys galėtų ten rasti prieglaudą. 
\par 4 Toks įstatymas yra apsaugoti žmogžudį, jei jis netyčia užmuštų savo artimą, kuris anksčiau nebuvo jo priešas. 
\par 5 Jei kas nueitų su artimu į mišką malkų ir bekertant nusmukęs kirvis mirtinai sužeistų artimą, jis tebėga į vieną iš tų miestų, ir jis išliks gyvas. 
\par 6 Kad kraujo keršytojas įsikarščiavęs nesivytų žudiko ir sugavęs neužmuštų; jis nenusipelnė mirties, nes neturėjo užmuštajam jokios neapykantos. 
\par 7 Todėl įsakau paskirti tris miestus. 
\par 8 O kai Viešpats, tavo Dievas, išplės tavo sienas, kaip prisiekė tavo tėvams, ir duos visą žemę, kurią jiems pažadėjo, 
\par 9 jei vykdysi įsakymus, kuriuos tau šiandien įsakau, mylėsi Viešpatį, savo Dievą, ir vaikščiosi Jo keliais, tai pridėsi dar tris kitus miestus, 
\par 10 kad nekaltas kraujas nebūtų išlietas žemėje, kurią Viešpats, tavo Dievas, tau duos paveldėti, ir tas kraujas nebūtų ant tavęs. 
\par 11 Bet jei kas, neapkęsdamas savo artimo, jo tykotų ir, užpuolęs jį, nužudytų, jei jis pasislėptų viename iš minėtų miestų, 
\par 12 tai jo miesto vyresnieji pasiųs ir jį iš ten grąžins, ir atiduos kraujo keršytojui, kad jį nužudytų. 
\par 13 Nesigailėdamas nužudyk jį, kad nekaltas kraujas nekristų ant Izraelio ir kad tau gerai sektųsi. 
\par 14 Neperkelk savo artimo žemės ribų, kurios yra nuo seno pažymėtos krašte, kurį Viešpats, tavo Dievas, tau duos paveldėti. 
\par 15 Vieno liudytojo neužtenka žmogaus kaltei įrodyti, kad ir koks būtų nusikaltimas; dviejų ar trijų liudytojų parodymais remsis byla. 
\par 16 Jei liudytojas melagingai kaltintų žmogų įstatymo laužymu, 
\par 17 abu stos Viešpaties, kunigų ir teisėjų akivaizdon. 
\par 18 Teisėjai viską atidžiai ištirs ir, jeigu liudytojas pasirodys esąs melagiu, liudijusiu netiesą prieš savo brolį, 
\par 19 padaryk jam taip, kaip jis siekė padaryti savo broliui. Tokiu būdu pašalink pikta, 
\par 20 kad kiti išgirdę bijotų ir niekad nedrįstų taip piktai elgtis tarp jūsų. 
\par 21 Nesigailėk: gyvybė už gyvybę, akis už akį, dantis už dantį, ranka už ranką, koja už koją”.



\chapter{20}

\par 1 “Jei eisi į karą su savo priešais ir pamatysi raitelius, kovos vežimus, gausesnę kariuomenę negu tavoji, nebijok jų, nes Viešpats, tavo Dievas, kuris išvedė tave iš Egipto žemės, yra su tavimi. 
\par 2 Prieš einant į mūšį, kunigas turi ateiti ir kalbėti žmonėms: 
\par 3 ‘Klausyk, Izraeli! Šiandien jūs einate kariauti su savo priešais. Tenepasilpsta jūsų širdys, nenusigąskite, nebijokite ir nepabūkite jų, 
\par 4 nes Viešpats, jūsų Dievas, eina su jumis kovoti už jus su jūsų priešais ir išgelbėti jus’. 
\par 5 Vyresnieji kalbės žmonėms: ‘Kas pasistatė naujus namus ir jų dar nepašventė, tegrįžta į savo namus, kad nežūtų mūšyje ir kad kitas jų nepašvęstų. 
\par 6 Kas pasodino vynuogyną ir jo vaisių dar nevalgė, tegrįžta į savo namus, kad nežūtų mūšyje ir kitas jų nevalgytų. 
\par 7 Kas susižadėjo ir dar nevedęs paliko sužadėtinę, tegrįžta į savo namus, kad nežūtų mūšyje ir kitas jos nevestų’. 
\par 8 Ir dar vyresnieji sakys: ‘Ar yra bailių ir nedrąsių? Grįžkite į savo namus, kad jūsų baimė nepersiduotų jūsų broliams’. 
\par 9 Kai vyresnieji baigs kalbėti, paskirs kariuomenei vadus, kurie ves žmones. 
\par 10 Priartėjęs prie miesto, pirma pasiūlyk jam taiką. 
\par 11 Jei ją priims ir atidarys vartus, visi miesto žmonės tau tarnaus ir mokės duoklę. 
\par 12 Jei jie nenorės taikos ir pradės kovą, apsupk jį. 
\par 13 Kai Viešpats, tavo Dievas, atiduos jį į tavo rankas, išžudyk kardu visus jame esančius vyrus. 
\par 14 Moteris, vaikus, galvijus ir visa, kas yra mieste, pasilaikyk. Naudokis grobiu, kurį Viešpats, tavo Dievas, tau davė. 
\par 15 Taip daryk su visais miestais, kurie toli nuo tavęs ir nėra iš šių tautų miestų. 
\par 16 Tuose miestuose, kurie tau duoti paveldėti, nepalik nieko gyvo, kas kvėpuoja. 
\par 17 Visiškai išnaikink hetitus, amoritus, kanaaniečius, perizus, hivus ir jebusiečius, kaip Viešpats Dievas tau įsakė, 
\par 18 kad jie neišmokytų jūsų daryti tų bjaurysčių, kurias jie patys darė savo dievams, ir jūs nenusidėtumėte Viešpačiui, savo Dievui. 
\par 19 Ilgą laiką laikydamas apsuptą miestą ir prieš jį kariaudamas, neišnaikink medžių, bet valgyk jų vaisius; juk medis­ne žmogus, kad prieš jį kariautum. 
\par 20 Tik medžius, kurie neneša vaisiaus, tinkančio valgymui, gali kirsti ir pasidaryti iš jų sustiprinimus apgulčiai, kol miestas bus paimtas”.



\chapter{21}

\par 1 “Jei žemėje, kurią Viešpats, tavo Dievas, tau duoda, būtų rastas užmušto žmogaus lavonas, bet nesurastas kaltininkas, 
\par 2 vyresnieji ir teisėjai išmatuos atstumus nuo užmuštojo ligi aplinkinių miestų. 
\par 3 Artimiausio miesto vyresnieji pasiims iš kaimenės telyčią, nekinkytą į jungą, 
\par 4 nuves ją į apleistą slėnį, kuris nebuvo suartas nė apsėtas, ir čia perpjaus jai gerklę. 
\par 5 Po to ateis Levio sūnūs, kunigai, kuriuos Viešpats, tavo Dievas, išsirinko Jam tarnauti ir Jo vardu laiminti, kurių žodžiu yra išsprendžiamas kiekvienas ginčas, 
\par 6 ir miesto, nuo kurio arčiausia iki nužudytojo, vyresnieji plaus savo rankas virš telyčios, papjautos slėnyje, 
\par 7 ir sakys: ‘Mūsų rankos nepraliejo šito kraujo ir mūsų akys nematė. 
\par 8 Viešpatie, pasigailėk savo tautos, Izraelio, kurią atpirkai, ir neįskaityk jai nekalto kraujo’. Ir jie bus apvalyti nuo kraujo. 
\par 9 Taip pašalinsi nekaltai pralietą kraują tarp savųjų darydamas, kas teisu Viešpaties akyse. 
\par 10 Kai kovosi su priešu, ir Viešpats, tavo Dievas, atiduos juos į tavo rankas, ir tu paimsi juos į nelaisvę, 
\par 11 jei tarp belaisvių pamatysi gražią moterį, kurią panorėsi vesti, 
\par 12 parsivesk ją į savo namus. Ji nusiskus plaukus, nusipjaustys nagus, 
\par 13 nusivilks belaisvės drabužius ir gyvens tavo namuose, apraudodama savo tėvą ir motiną visą mėnesį. Po to gali įeiti pas ją ir būti jos vyras, o ji bus tavo žmona. 
\par 14 Jei ji tau nebepatiks, atleisi ją, bet negalėsi jos parduoti arba paversti verge, nes ją pažeminai. 
\par 15 Jei vyras turėtų dvi žmonas, vieną mylimą, o antrą­nemylimą, ir, joms pagimdžius vaikų, nemylimosios sūnus būtų pirmagimis, 
\par 16 norėdamas padalinti savo sūnums palikimą, jis negalės mylimosios sūnaus padaryti pirmagimio ir duoti jam pirmenybę nemylimosios sūnaus vietoje. 
\par 17 Jis pripažins nemylimosios sūnų pirmagimiu ir jam duos dvigubą dalį, nes tas yra jo pajėgumo pradžia ir jam priklauso pirmagimio teisė. 
\par 18 Jei sūnus būtų tėvams nepaklusnus, priešgina ir užsispyręs, jei jis nekreiptų dėmesio į pabaudimą, 
\par 19 tėvai nuves jį pas miesto vyresniuosius prie miesto vartų 
\par 20 ir sakys miesto vyresniesiems: ‘Šitas mūsų sūnus yra neklusnus, priešgina ir užsispyręs, lėbauja ir girtauja’. 
\par 21 Miesto gyventojai užmuš jį akmenimis. Taip bus pašalinta pikta iš jūsų, kad visas Izraelis, tai girdėdamas, bijotų. 
\par 22 Jei žmogus padarys nuodėmę, vertą mirties, ir tu nužudysi jį pakardamas, 
\par 23 jo kūno nepalik ant medžio per naktį, palaidok jį tą pačią dieną. Dievo prakeiktas tas, kuris kabo ant medžio. Nesutepk savo žemės, kurią Viešpats, tavo Dievas, tau duoda paveldėti”.



\chapter{22}


\par 1 “Jei matytum paklydusį savo brolio jautį ar avį, nenusigręžk, bet nuvesk jį atgal pas savo brolį. 
\par 2 Jei gyvulio savininkas gyventų toli ar būtų nepažįstamas, parsivesk jį pas save ir laikyk, kol tavo brolis ieškos jo, ir tu jam sugrąžinsi. 
\par 3 Taip pat daryk su asilu, drabužiu ir kiekvienu savo brolio daiktu, kurį radai pamestą: ką rasi, neslėpk, bet grąžink. 
\par 4 Jei matysi savo brolio asilą ar jautį, parkritusį ant kelio, nenusigręžk, bet padėk jį pakelti. 
\par 5 Moteris nesivilks vyro rūbais, ir vyras­moters; Viešpats bjaurisi tokiais, kurie taip daro. 
\par 6 Jei eidamas rastum medyje ar žemėje paukščio lizdą ir tupinčią patelę ant paukščiukų ar kiaušinių, nepasiimk jos kartu su jaunikliais. 
\par 7 Leisk nuskristi motinai, pasiimk tik paukščiukus, kad tau gerai sektųsi ir ilgai gyventum. 
\par 8 Pasistatęs naują namą, padaryk užtvarą aplink stogą, kad neužtrauktum kraujo ant savo namo, kam nors nukritus nuo jo. 
\par 9 Nesėk vynuogyne jokių kitų sėklų, kad jo vaisiai nebūtų suteršti. 
\par 10 Neark drauge jaučiu ir asilu. 
\par 11 Nedėvėk drabužio, austo iš vilnų ir linų. 
\par 12 Pasidaryk kutus prie keturių kampų savo apsiausto, kuriuo apsisiauti. 
\par 13 Jei kas vestų žmoną, o vėliau pradėtų jos nekęsti 
\par 14 ir ieškotų progos ją atleisti, apšmeiždamas: ‘Aš vedžiau šitą žmoną ir, įėjęs pas ją, radau, kad ji nėra mergaitė’, 
\par 15 tada jos tėvas ir motina atneš jos mergystės ženklus pas miesto vyresniuosius prie vartų, 
\par 16 ir tėvas sakys: ‘Aš daviau savo dukterį šitam vyrui už žmoną. Jis pradėjo jos neapkęsti 
\par 17 ir apšmeižė ją, sakydamas, kad rado, jog mano duktė nebuvo mergaitė. Štai mano dukters nekaltybės įrodymas’. Jie parodys vyresniesiems drabužį. 
\par 18 Tada miesto vyresnieji paims vyrą ir nubaus jį. 
\par 19 Be to, jis turės sumokėti šimtą šekelių sidabro baudos, kurią atiduos moteriškės tėvui, nes jis apšmeižė Izraelio mergaitę. Ji bus jo žmona, ir jis negalės jos atleisti per visas savo dienas. 
\par 20 Bet jei tai tiesa, kad ji nebuvo mergaitė, 
\par 21 jie ją atves prie tėvo namų durų, ir miesto vyrai užmuš ją akmenimis, nes ji padarė bjaurų nusikaltimą Izraelyje, paleistuvaudama tėvo namuose. Taip pašalinsi pikta iš Izraelio. 
\par 22 Jei vyras būtų sugautas gulint su kito žmona, abu: tiek vyras, tiek moteris, turės mirti. Taip pašalinsi pikta iš Izraelio. 
\par 23 Jei mergaitė būtų pažadėta vyrui ir kas nors, sutikęs ją mieste, gulėtų su ja, 
\par 24 abu bus užmušti akmenimis už miesto vartų: mergaitė už tai, kad nešaukė, būdama mieste, o vyras, kad pažemino savo artimo žmoną. Taip pašalinsi pikta iš savo žmonių. 
\par 25 Jei vyras, sutikęs susižadėjusią mergaitę laukuose, ją išprievartautų, jis vienas bus nubaustas mirtimi. 
\par 26 Mergaitei nieko nedarysi, nes ji nepadarė nuodėmės, vertos mirties. Toks nusikaltimas prilygsta žmogžudžio nusikaltimui. 
\par 27 Ji buvo laukuose, šaukė, bet nebuvo, kas ją išgelbėtų. 
\par 28 Jei vyras, sutikęs nesusižadėjusią mergaitę, pagriebtų ją ir gulėtų su ja, ir jie būtų sugauti, 
\par 29 tas vyras sumokės mergaitės tėvui penkiasdešimt šekelių sidabro ir turės ją vesti, nes ją pažemino; jis negalės jos atleisti per visas savo dienas. 
\par 30 Nė vienas neves tėvo žmonos ir neatidengs tėvo nuogumo”.



\chapter{23}


\par 1 “Eunuchas neįeis į Viešpaties tautą. 
\par 2 Pavainikis neįeis į Viešpaties tautą, nė jo palikuonys neįeis į Viešpaties tautą net iki dešimtos kartos. 
\par 3 Amonitas ir moabitas neįeis į Viešpaties tautą, nė jų palikuonys neįeis į Viešpaties tautą net iki dešimtos kartos per amžius, 
\par 4 nes jie nepasitiko jūsų kelyje su duona ir vandeniu, kai išėjote iš Egipto, bet pasisamdė prieš jus iš Mesopotamijos Beoro sūnų Balaamą, kad jus prakeiktų. 
\par 5 Bet Viešpats, tavo Dievas, neklausė Balaamo ir pavertė prakeikimą palaiminimu, nes Jis mylėjo tave. 
\par 6 Nelinkėk jiems taikos ir gerovės per visas savo dienas ir per amžius. 
\par 7 Neniekink edomitų, nes jie tavo broliai, nė egiptiečių, nes buvote ateiviai jų žemėje. 
\par 8 Jų trečios kartos vaikai įeis į Viešpaties tautą. 
\par 9 Traukdamas į karą su priešu, saugokis nuo visų blogybių. 
\par 10 Jei vyras susiteptų nuo to, kas atsitiko jam naktį, jis privalo išeiti iš stovyklos ir neįeiti į stovyklą. 
\par 11 Vakare, saulei nusileidus, apsiplovęs vėl gali grįžti į stovyklą. 
\par 12 Turėk vietą už stovyklos, kur galėtum išeiti. 
\par 13 Neškis kastuvėlį, išsikask duobelę pritūpdamas ir užpilk žemėmis išmatas. 
\par 14 Viešpats, tavo Dievas, yra stovykloje, kad tave globotų ir padėtų kovoje su priešu. Tavo stovykla tebūna šventa, kad Dievas nematytų nešvaros tarp jūsų ir nepasišalintų. 
\par 15 Vergo, kuris pas tave atbėgs, negrąžink jo šeimininkui. 
\par 16 Jis tegyvena pas tave toje vietoje, kuri jam patiks, ir kurią jis pasirinks; tu jo neišnaudok. 
\par 17 Nebus paleistuvės iš Izraelio dukterų, nė iškrypėlių iš Izraelio sūnų. 
\par 18 Neatnešk į Viešpaties, tavo Dievo, namus nei paleistuvės užmokesčio, nei šuns kainos pagal jokį įžadą; Viešpaties, tavo Dievo, akyse jie yra pasibjaurėjimas. 
\par 19 Neimk iš savo brolio palūkanų nei už paskolintus pinigus, nei už javus, nei už kurį nors kitą daiktą. 
\par 20 Iš svetimo gali imti. Broliui gi skolink be palūkanų, kad Viešpats, tavo Dievas, tave laimintų visuose tavo darbuose žemėje, kurios eini paveldėti. 
\par 21 Padaręs Viešpačiui, savo Dievui, įžadą, nedelsk jo įvykdyti, nes Viešpats, tavo Dievas, pareikalaus iš tavęs, ir tai bus tau nuodėmė. 
\par 22 Jei nepadarysi įžado, tau nebus nuodėmės. 
\par 23 Pažadėjęs laikykis žodžio, nes tu laisva valia davei įžadą Viešpačiui, savo Dievui, kai savo burna pažadėjai. 
\par 24 Savo artimo vynuogyne valgyk vynuogių kiek nori, bet su savimi nieko neišsinešk. 
\par 25 Savo artimo javų lauke gali skinti varpas savo ranka, bet pjautuvu nepjauk”.



\chapter{24}

\par 1 “Jei vyras vestų žmoną, su ja gyventų ir būtų dėl ko nors nepatenkintas ja, jis parašys skyrybų raštą, paduos jai ir atleis ją. 
\par 2 Išėjusi iš jo namų, ji galės ištekėti už kito vyro. 
\par 3 Jei ir tas pradėtų jos nemėgti ir duotų skyrybų raštą arba mirtų, 
\par 4 pirmasis vyras negalės jos vėl vesti po to, kai ji yra suteršta. Tai bjauru Viešpaties akyse. Venk tokios nuodėmės žemėje, kurią Viešpats, tavo Dievas, tau duoda. 
\par 5 Ką tik vedęs vyras bus atleistas nuo karo tarnybos ir nebus jam pavestos kitos pareigos; jis liks namuose vienerius metus pralinksminti savo žmoną. 
\par 6 Užstatui neimk nei apatinio, nei viršutinio girnų akmens, nes taip paimtum užstatu gyvybę. 
\par 7 Jei žmogus būtų pagautas vagiant savo brolį izraelitą, kad jį pavergtų arba parduotų, toks bus baudžiamas mirtimi. Pašalink pikta iš savųjų. 
\par 8 Žiūrėk, kad raupsų metu rūpestingai vykdytum Levio giminės kunigų nurodymus, kuriuos jiems daviau. 
\par 9 Atsimink, ką Viešpats, tavo Dievas, padarė Mirjamai kelyje, kai ėjote iš Egipto. 
\par 10 Ką nors skolindamas savo broliui, neik į jo namus užstato pasiimti, 
\par 11 bet palauk lauke, kol jis tau išneš. 
\par 12 Jei jis neturtingas, nelaikyk pas save per naktį to užstato. 
\par 13 Dar saulei nenusileidus, atiduok jam užstatą, kad jis miegotų savo patale ir laimintų tave. Taip tu būsi teisus Viešpaties, tavo Dievo, akyse. 
\par 14 Neišnaudok samdinio, kuris yra vargdienis ir beturtis, iš tavo brolių ar ateivių, kuris gyvena tavo apylinkėje. 
\par 15 Užmokėk jam atlyginimą tą pačią dieną, dar saulei nenusileidus, nes jis neturtingas ir iš to gyvena, kad jis nesišauktų Viešpaties dėl tavęs ir tau nebūtų nuodėmės. 
\par 16 Tėvai nebus žudomi už vaikus nė vaikai­už tėvus; kiekvienas mirs už savo nuodėmes. 
\par 17 Neiškraipyk teisme ateivio ar našlaičio bylos ir neimk iš našlės drabužio kaip užstato. 
\par 18 Atsimink, kad buvai vergas Egipte ir kad Viešpats, tavo Dievas, iš ten tave išpirko. Todėl įsakau tau taip elgtis. 
\par 19 Jei, veždamas javus, užmirši lauke pėdą, negrįžk jo pasiimti. Palik jį ateiviui, našlaičiui ir našlei, kad Viešpats, tavo Dievas, tave laimintų visuose tavo darbuose. 
\par 20 Kai skinsi alyvų vaisius, nežiūrėk antrą kartą; kas liko medžiuose, palik ateiviui, našlaičiui ir našlei. 
\par 21 Kai skinsi vynuoges, nerink pasilikusių. Palik jas ateiviui, našlaičiui ir našlei. 
\par 22 Atsimink, kad buvai vergas Egipte. Todėl įsakau tau taip daryti”.



\chapter{25}


\par 1 “Jei du kreiptųsi į teismą, teisėjai išteisins teisųjį ir pasmerks nusikaltusį. 
\par 2 Jei nusikaltėlis nubaustas plakimu, teisėjas įsakys gulti, ir jis bus nuplaktas jo akivaizdoje. Kirčių skaičius bus pagal nusikaltimą, 
\par 3 bet neviršys keturiasdešimties kirčių, kad tavo brolis nebūtų žiauriai sužalotas tavo akyse. 
\par 4 Neužrišk nasrų kuliančiam jaučiui. 
\par 5 Jei broliai gyventų kartu ir vienas jų mirtų bevaikis, mirusio žmona neištekės už svetimo, bet jos vyro brolis įeis pas ją ir ves ją, kad atliktų vyro brolio pareigą. 
\par 6 Pirmagimį jos sūnų pavadins mirusiojo brolio vardu, kad jo vardas neišnyktų iš Izraelio. 
\par 7 Jeigu jis nenorės vesti savo brolio žmonos, moteris eis prie miesto vartų ir kreipsis į vyresniuosius: ‘Mano vyro brolis nenori atstatyti savo brolio vardo Izraelyje nė manęs vesti’. 
\par 8 Tada miesto vyresnieji pasikvies jį ir kalbės su juo. Jei jis atsakys: ‘Nenoriu jos vesti’, 
\par 9 moteris vyresniųjų akyse nuaus nuo jo kojos apavą ir spjaus jam į veidą, tardama: ‘Taip atsitinka vyrui, kuris atsisako pratęsti savo brolio giminę’. 
\par 10 Jis bus vadinamas Izraelyje: ‘Nuautojo namai’. 
\par 11 Jei du vyrai susivaidiję pradėtų grumtis, ir vieno žmona, norėdama padėti savo vyrui, nutvertų kitą už lyties, 
\par 12 nukirsk jai ranką, nepasigailėk jos. 
\par 13 Nenaudok didesnių ir mažesnių svarsčių. 
\par 14 Nelaikyk savo namuose didesnio ir mažesnio saiko. 
\par 15 Naudok teisingus ir tikrus saikus bei matus, kad ilgai gyventum žemėje, kurią Viešpats, tavo Dievas, tau duoda. 
\par 16 Visi, kurie taip daro ir neteisiai elgiasi, yra pasibjaurėjimas Viešpačiui, tavo Dievui. 
\par 17 Atsimink, ką tau padarė amalekiečiai, kai išėjai iš Egipto. 
\par 18 Jie išėjo prieš tave ir tavo nuvargusius ir atsilikusius žmones išžudė, kai tu buvai pailsęs ir nuvargęs; jie nebijojo Dievo. 
\par 19 Kai Viešpats, tavo Dievas, suteiks tau poilsį nuo visų priešų žemėje, kurią tau duoda paveldėti, išnaikink Amaleko vardą nuo žemės paviršiaus; žiūrėk, neužmiršk”.



\chapter{26}

\par 1 “Kai būsi žemėje, kurią Viešpats, tavo Dievas, tau duoda paveldėti, ir ten apsigyvensi, 
\par 2 pridėk pintinę visų savo pirmojo derliaus vaisių ir nunešk ją į vietą, kurią Viešpats, tavo Dievas, pasirinks. 
\par 3 Nuėjęs pas kunigą, kuris tuo metu eis tarnystę, jam sakyk: ‘Išpažįstu šiandien Viešpaties, savo Dievo, akivaizdoje, kad esu žemėje, kurią Jis prisiekė mūsų tėvams mums duoti’. 
\par 4 Kunigas, paėmęs pintinę, padės ją prie Viešpaties, tavo Dievo, aukuro. 
\par 5 Tu sakysi Viešpaties, savo Dievo, akivaizdoje: ‘Mano tėvas buvo klajoklis aramėjas, jis nuėjo į Egiptą su labai mažu žmonių skaičiumi ir buvo ten ateivis. Jis tapo didele, stipria ir gausia tauta. 
\par 6 Egiptiečiai slėgė ir spaudė mus, užkraudami mums sunkių darbų naštą. 
\par 7 Mes šaukėmės Viešpaties, mūsų tėvų Dievo. Jis mus išklausė, pažvelgė į mūsų pažeminimą, vargą ir priespaudą; 
\par 8 išvedė mus iš Egipto galinga ranka, siaubais, ženklais ir stebuklais. 
\par 9 Jis atvedė mus į šitą kraštą ir davė žemę, plūstančią pienu ir medumi. 
\par 10 Štai dabar aukoju pirmuosius vaisius tos žemės, kurią Tu, Viešpatie, man davei’. Palik auką Viešpaties, savo Dievo, akivaizdoje ir pagarbink Jį. 
\par 11 Džiaukis tu, levitas ir ateivis, kuris gyvena tavo apylinkėje, gėrybėmis, kurias Viešpats, tavo Dievas, duoda tau ir tavo namams. 
\par 12 Kai atiduosi viso savo derliaus dešimtinę trečiais metais levitui, ateiviui, našlaičiui ir našlei, gyvenantiems su tavimi, kad jie valgytų ir pasisotintų, 
\par 13 kalbėk Viešpaties akivaizdoje: ‘Aš ėmiau iš savo namų ir daviau levitui, ateiviui, našlaičiui ir našlei, kaip man įsakei; neperžengiau Tavo įsakymų ir neužmiršau paliepimų. 
\par 14 Nevalgiau nieko pažadėto gedėdamas, nepaliečiau susitepęs ir nieko neatidaviau už mirusius. Klausiau, Viešpatie, Tavęs ir visa dariau, kaip man įsakei. 
\par 15 Pažvelk iš dangaus, savo šventos buveinės, ir laimink savo tautą Izraelį ir tą žemę, kurią mums davei, kaip prisiekei mūsų tėvams,­žemę, plūstančią pienu ir medumi’. 
\par 16 Šiandien Viešpats, tavo Dievas, tau įsakė vykdyti šiuos įsakymus ir paliepimus, todėl laikykis jų ir vykdyk juos visa savo širdimi ir visa savo siela. 
\par 17 Šiandien tu pareiškei, kad pasirinkai Viešpatį savo Dievu ir kad vaikščiosi Jo keliais, laikysiesi Jo įsakymų, paliepimų ir įstatymų ir būsi paklusnus Jo balsui. 
\par 18 Šiandien Viešpats pareiškė, kad pasirinko tave būti Jo ypatinga tauta, kaip Jis pažadėjo, kad tu įvykdytum visus Jo įsakymus, 
\par 19 kad Jis išaukštintų tavo vardą virš visų tautų, kurias sutvėrė, ir suteiktų tau garbę ir šlovę; kad tu galėtum būti šventa tauta Viešpačiui, savo Dievui, kaip Jis pasakė”.



\chapter{27}


\par 1 Mozė ir Izraelio vyresnieji įsakė tautai: “Izraeli, vykdyk visus įsakymus, kuriuos šiandien tau skelbiu. 
\par 2 Kai pereisi per Jordaną į žemę, kurią Viešpats, tavo Dievas, tau duoda, sukrauk didelius akmenis, juos aptepk kalkėmis 
\par 3 ir užrašyk ant jų visus šito įstatymo žodžius, kad įeitum į žemę, kurią Viešpats, tavo Dievas, tau duoda, žemę, plūstančią pienu ir medumi, kaip Jis pažadėjo tavo tėvams. 
\par 4 Taigi perėję per Jordaną, sukraukite tuos akmenis ant Ebalo kalno, kaip jums šiandien įsakau, ir juos aptepkite kalkėmis. 
\par 5 Pastatyk ten Viešpačiui, savo Dievui, aukurą iš netašytų akmenų. 
\par 6 Ant to nedailintų akmenų aukuro aukok deginamąsias aukas Viešpačiui, savo Dievui. 
\par 7 Padėkos aukas ten aukok, valgyk ir džiaukis Viešpaties, savo Dievo, akivaizdoje. 
\par 8 Užrašyk ant akmenų visus šito įstatymo žodžius labai aiškiai”. 
\par 9 Mozė ir Levio giminės kunigai kalbėjo visam Izraeliui: “Būk atidus ir klausyk, Izraeli. Šiandien tu tapai Viešpaties, tavo Dievo, tauta. 
\par 10 Klausyk Jo balso ir vykdyk įsakymus ir įstatymus, kuriuos tau šiandien skelbiu”. 
\par 11 Tą pačią dieną Mozė kalbėjo Izraelio tautai: 
\par 12 “Perėjus per Jordaną, ant Garizimo kalno stovės ir laimins tautą Simeono, Levio, Judo, Isacharo, Juozapo ir Benjamino giminės; 
\par 13 o ant Ebalo kalno stovės Rubeno, Gado, Ašero, Zabulono, Dano ir Neftalio giminės, kurie prakeiks. 
\par 14 Levitai garsiu balsu sakys visiems izraelitams: 
\par 15 ‘Prakeiktas žmogus, kuris amatininkų rankomis pasidaro drožtą ar lietą atvaizdą, pasibjaurėjimą Viešpačiui, ir slepia jį’. Visa tauta atsakys: ‘Amen’. 
\par 16 ‘Prakeiktas, kuris keikia savo tėvą ir motiną’. Visa tauta atsakys: ‘Amen’. 
\par 17 ‘Prakeiktas, kuris perkelia savo artimo žemės ribų ženklą’. Visa tauta atsakys: ‘Amen’. 
\par 18 ‘Prakeiktas, kuris suklaidina aklą kelyje’. Visa tauta atsakys: ‘Amen’. 
\par 19 ‘Prakeiktas, kuris iškreipia teisingumą ateivio, našlaičio ir našlės byloje’. Visa tauta atsakys: ‘Amen’. 
\par 20 ‘Prakeiktas, kuris sugula su savo tėvo žmona, nes atidengia savo tėvo nuogumą’. Visa tauta atsakys: ‘Amen’. 
\par 21 ‘Prakeiktas, kuris paleistuvauja su kuriuo nors gyvuliu’. Visa tauta atsakys: ‘Amen’. 
\par 22 ‘Prakeiktas, kuris sugula su savo seserimi, savo tėvo ar motinos dukterimi’. Visa tauta atsakys: ‘Amen’. 
\par 23 ‘Prakeiktas, kuris sugula su savo uošve’. Visa tauta atsakys: ‘Amen’. 
\par 24 ‘Prakeiktas, kuris užmuša savo artimą’. Visa tauta atsakys: ‘Amen’. 
\par 25 ‘Prakeiktas, kuris paima užmokestį, kad pralietų nekaltą kraują’. Visa tauta atsakys: ‘Amen’. 
\par 26 ‘Prakeiktas, kuris nesilaiko šito įstatymo žodžių ir jų nevykdo’. Visa tauta atsakys: ‘Amen’ ”.



\chapter{28}


\par 1 “Jei atidžiai klausysi Viešpaties, savo Dievo, balso ir vykdysi bei laikysi visus Jo įsakymus, kuriuos aš tau šiandien skelbiu, Viešpats, tavo Dievas, išaukštins tave virš visų žemės tautų. 
\par 2 Visi šie palaiminimai ateis ir pasivys tave, jei klausysi Viešpaties, savo Dievo, balso. 
\par 3 Palaimintas tu būsi mieste ir palaimintas tu būsi lauke. 
\par 4 Palaimintas bus tavo kūno vaisius, tavo žemės derlius, tavo bandų, galvijų ir avių prieauglis. 
\par 5 Palaimintos bus tavo klėtys ir atsargos. 
\par 6 Palaimintas būsi įeidamas ir palaimintas būsi išeidamas. 
\par 7 Viešpats sunaikins tavo priešus, kurie pakils prieš tave, tavo akivaizdoje; jie puls tave vienu keliu, o bėgs nuo tavęs septyniais keliais. 
\par 8 Viešpats pasiųs savo palaiminimą į tavo sandėlius ir į visus tavo rankų darbus; Jis palaimins tave žemėje, kurią Viešpats, tavo Dievas tau duoda. 
\par 9 Jei vykdysi Viešpaties, savo Dievo, įsakymus ir vaikščiosi Jo keliais, Jis patvirtins tave savo šventa tauta, kaip tau pažadėjo. 
\par 10 Visos žemės tautos, matydamos, kad tu vadinamas Viešpaties vardu, bijos tavęs. 
\par 11 Viešpats suteiks tau apsčiai gėrybių­tavo kūno vaisiaus, galvijų prieauglio ir žemės derliaus­ krašte, kurį Viešpats prisiekė tavo tėvams atiduoti. 
\par 12 Viešpats atvers tau savo gerąjį lobyną­dangų, kad duotų lietaus tavo žemei reikiamu metu ir laimintų tavo rankų darbą; tu skolinsi daugeliui tautų, bet pats iš nieko nesiskolinsi. 
\par 13 Viešpats padarys tave galva, o ne uodega; visuomet būsi viršuje, o ne apačioje, jei tik klausysi Viešpaties, savo Dievo, įsakymų, kuriuos aš tau šiandien skelbiu, ir juos vykdysi. 
\par 14 Nenukrypk nė nuo vieno žodžio, kuriuos šiandien tau skelbiu, nei į kairę, nei į dešinę sekti kitų dievų ir jiems tarnauti. 
\par 15 Bet jei neklausysi Viešpaties, savo Dievo, balso ir nesilaikysi bei nevykdysi visų Jo įsakymų ir įstatymų, kuriuos aš tau skelbiu, visi šitie prakeikimai ateis ir pasivys tave. 
\par 16 Prakeiktas tu būsi mieste ir prakeiktas tu būsi lauke. 
\par 17 Prakeiktos bus tavo klėtys ir atsargos. 
\par 18 Prakeiktas bus tavo kūno vaisius, tavo žemės derlius, tavo galvijų ir avių prieauglis. 
\par 19 Prakeiktas būsi įeidamas ir prakeiktas būsi išeidamas. 
\par 20 Viešpats siųs tau prakeikimą, sumišimą ir nepasisekimą visame, ką darysi, iki tave visai sunaikins dėl tavo piktų darbų, nes apleidai mane. 
\par 21 Viešpats siųs tau marą, kol tave nušluos nuo tos žemės, kurios paveldėti eini. 
\par 22 Viešpats baus tave drugiu, šalčiu, sausra ir kardu, kurie persekios tave iki pražūsi. 
\par 23 Dangus virš tavęs bus varinis ir žemė po tavimi­geležinė. 
\par 24 Viešpats vietoje lietaus tavo žemei siųs smilčių audras, kol tave sunaikins. 
\par 25 Viešpats leis priešams tave pavergti; vienu keliu išeisi prieš juos, o jiems puolant bėgsi septyniais; tu būsi išblaškytas po visas žemės karalystes. 
\par 26 Tavo žmonių lavonai bus maistu padangių paukščiams ir žemės žvėrims, niekas jų nenubaidys. 
\par 27 Viešpats baus tave Egipto votimis, šašais ir niežais, kurių negalėsi išgydyti. 
\par 28 Viešpats ištiks tave pamišimu, aklumu ir širdies sustingimu. 
\par 29 Vidudienį grabaliosi, kaip aklas grabalioja patamsyje, ir tau nesiseks tavo keliuose; tu būsi spaudžiamas ir išnaudojamas visą laiką, ir niekas tavęs neišgelbės. 
\par 30 Vesi žmoną, o kitas gulės su ja. Pasistatysi namą, bet jame negyvensi. Užveisi vynuogyną, bet jo vaisių nerinksi. 
\par 31 Tavo jautį papjaus tavo akyse, bet tu jo neragausi. Tavo asilą pasigaus tau matant ir tau jo negrąžins. Tavo avis paims priešas ir niekas tau nepadės. 
\par 32 Tavo sūnūs ir dukterys bus išvesti į kitas tautas; belaukdamas jų, pražiūrėsi akis ir būsi bejėgis. 
\par 33 Tavo žemės derlių ir visus darbo vaisius paims svetimieji, tu pats būsi mušamas ir žiauriai baudžiamas. 
\par 34 Tu išprotėsi nuo to, ką matys tavo akys. 
\par 35 Be to vargins tave piktos votys ir nepagydomos žaizdos, negalėsi išsigydyti nuo galvos iki kojų. 
\par 36 Tave ir tavo karalių Viešpats atiduos tautai, kurios nepažinai nei tu, nei tavo tėvai; ten tarnausi mediniams ir akmeniniams dievams. 
\par 37 Tu tapsi pasibaisėjimu ir priežodžiu tose tautose, į kurias Viešpats tave nuves. 
\par 38 Daug sėsi, bet mažai surinksi, nes skėriai viską suės. 
\par 39 Užsiveisi vynuogyną ir jį prižiūrėsi, bet vyno negersi ir vynuogių nevalgysi, nes viską kirminai sunaikins. 
\par 40 Turėsi alyvmedžių visame krašte, o aliejumi nesitepsi, nes jų vaisiai neprinokę nubyrės. 
\par 41 Tau gims sūnų ir dukterų, bet jais nesidžiaugsi, nes jie bus išvesti nelaisvėn. 
\par 42 Visus tavo medžius ir žemės vaisius suės skėriai. 
\par 43 Tavo žemėje gyvenąs ateivis labai iškils virš tavęs, tu gi labai nusmuksi. 
\par 44 Jis tau skolins, bet tu jam neturėsi ko skolinti. Jis bus galva, o tu uodega. 
\par 45 Tau ateis šitie prakeikimai, persekios ir pasivys tave, iki būsi sunaikintas, nes neklausei Viešpaties, savo Dievo, ir nevykdei Jo įstatymų bei įsakymų, kuriuos Jis tau davė. 
\par 46 Jie bus tau ir tavo palikuonims ženklu ir stebalu per amžius. 
\par 47 Netarnavai Viešpačiui, savo Dievui, su džiaugsmu ir linksma širdimi, turėdamas visko, 
\par 48 tai tarnausi priešui, kurį Viešpats tau siųs; kęsi badą, troškulį, nepriteklių ir vargą. Jis uždės tau geležinį jungą, kol tave sunaikins. 
\par 49 Jis atves tautą iš pačių žemės pakraščių, kuri puls tave kaip erelis; jos kalbos tu nesuprasi. 
\par 50 Žiaurią tautą, kuri neaplenks seno nė pagailės jauno. 
\par 51 Ji valgys tavo galvijus ir žemės derlių. Ji tau nepaliks kviečių, vyno, aliejaus, jaučių ir avių ir tu būsi sunaikintas. 
\par 52 Jie puls tavo miestus ir sugriaus stiprius ir aukštus mūrus, kuriais pasitikėjai. Apguls miestus visame tavo krašte, kurį Viešpats, tavo Dievas, tau davė. 
\par 53 Tu valgysi savo kūno vaisių, savo sūnų ir dukterų, kuriuos Dievas tau duos, kūnus, apsuptyje ir suspaudime, kuriais vargins tave tavo priešas. 
\par 54 Net aukštos kilmės ir išlepintas vyras bus negailestingas savo broliui, mylimai žmonai ir savo likusiems vaikams. 
\par 55 Jis jiems neduos savo vaikų mėsos, kurią pats valgys, nes nieko kito nebeturės priešo apgultame mieste. 
\par 56 Jautri ir išlepinta moteris, nepratusi vaikščioti basa, bus negailestinga savo mylimam vyrui, sūnui ir dukteriai 
\par 57 ir neduos jiems to, kas išeina iš jos kojų tarpo gimdant ir ką tik gimusio kūdikio, nes ji pati tai valgys slaptai priešo apgultame mieste. 
\par 58 Jei nesilaikysi ir nevykdysi visų šito įstatymo žodžių, kurie surašyti šioje knygoje, ir nebijosi šlovingo ir baisaus vardo­Viešpaties, tavo Dievo, 
\par 59 Jis siųs tau ir tavo palikuonims dideles, piktas ir ilgas negalias, 
\par 60 be to, Jis užleis ant tavęs visas Egipto ligas, kurių tu bijai, ir jos prikibs prie tavęs. 
\par 61 Taip pat ligas ir negalias, kurios neįrašytos šitoje įstatymo knygoje, Viešpats užleis ant tavęs, iki sunaikins tave. 
\par 62 Jūsų buvo tiek, kiek dangaus žvaigždžių, bet po to paliks tik mažas skaičius, nes neklausėte Viešpaties, savo Dievo. 
\par 63 Kaip Viešpats džiaugėsi darydamas jums gera ir padaugindamas jus, taip džiaugsis jus išblaškydamas ir išvydamas iš žemės, kurios paveldėti einate. 
\par 64 Viešpats išsklaidys tave tarp visų tautų nuo vieno žemės krašto iki kito. Ten tarnausite kitiems dievams: medžiui ir akmeniui, kurių nežinojai nei tu, nei tavo tėvai. 
\par 65 Tarp svetimų tautų neturėsi poilsio nė vietos kojai ramiai pastatyti, nes Viešpats duos tau baukščią širdį, nusilpusias akis ir kankinančias mintis. 
\par 66 Tavo gyvybė bus pavojuje dieną ir naktį, tu nebūsi tikras dėl jos. 
\par 67 Rytą lauksi vakaro, o vakare­ryto. Tavo širdis bus įbauginta pergyvenimų, kuriuos patyrei. 
\par 68 Viešpats sugrąžins tave laivais į Egiptą, apie kurį sakiau, kad jo niekuomet nebematysi. Ten siūlysitės savo priešams vergais ir vergėmis, bet niekas jūsų nepirks”.



\chapter{29}

\par 1 Tai yra žodžiai sandoros, kurią Viešpats įsakė Mozei padaryti su izraelitais Moabo žemėje, neskaitant sandoros, padarytos Horebe. 
\par 2 Mozė sušaukė izraelitus ir jiems tarė: “Jūs matėte, ką Viešpats padarė faraonui, visiems jo tarnams ir Egipto žemei jūsų akivaizdoje; 
\par 3 didelius išbandymus, kuriuos matė jūsų akys, nepaprastus ženklus ir stebuklus, 
\par 4 tačiau Viešpats jums nedavė iki šios dienos išminties, neatvėrė nei akių, nei ausų. 
\par 5 Aš vedžiojau jus keturiasdešimt metų dykumoje: nenusidėvėjo jūsų drabužiai ir kojų apavas nesudilo. 
\par 6 Jūs nevalgėte duonos, negėrėte vyno nė kito stipraus gėrimo, kad žinotumėte, jog Aš esu Viešpats, jūsų Dievas. 
\par 7 Kai atėjote į šitą vietą, prieš jus išėjo Hešbono karalius Sihonas ir Bašano karalius Ogas. Mes juos nugalėjome, 
\par 8 užėmėme jų žemę ir davėme ją paveldėti Rubenui, Gadui ir pusei Manaso giminės. 
\par 9 Laikykitės tad šitos sandoros žodžių ir juos vykdykite, kad jums visuomet gerai sektųsi. 
\par 10 Šiandien jūs visi stovite Viešpaties, jūsų Dievo, akivaizdoje: jūsų giminių vadai ir vyresnieji, visi Izraelio vyrai, 
\par 11 vaikai, žmonos ir stovyklos ateiviai, kurie kerta malkas ir atneša vandens, 
\par 12 kad įeitumėte į sandorą su Viešpačiu, savo Dievu, ir į sutartį, kurią šiandien Viešpats, tavo Dievas, su tavimi daro, 
\par 13 kad Jis galėtų įtvirtinti tave savo tauta ir būtų tavo Dievas, kaip kalbėjo tau ir prisiekė tavo tėvams: Abraomui, Izaokui ir Jokūbui. 
\par 14 Ne tik su jumis darau šitą sandorą ir sutartį, 
\par 15 bet su tais, kurie yra čia, Viešpaties, savo Dievo, akivaizdoje, ir su tais, kurių čia šiandien nėra. 
\par 16 Jūs žinote, kaip gyvenome Egipto žemėje, kaip keliavome per tautas. 
\par 17 Jūs matėte jų bjaurystes ir stabus­medinius, akmeninius, sidabrinius ir auksinius. 
\par 18 Tegul tarp jūsų neatsiranda vyro nei moters, nei šeimos, nei giminės, kurie nusisuktų nuo Viešpaties, savo Dievo, ir tarnautų svetimiems dievams; tegul nebūna tarp jūsų šaknies, auginančios tulžį ir metėlę; 
\par 19 kad nė vienas, išgirdęs šio prakeikimo žodžius, nelaimintų savęs savo širdyje, sakydamas: ‘Aš turėsiu ramybę, nors vaikštau pagal savo širdies įsivaizdavimus’, tarsi būtų galima lyginti girtą su ištroškusiu. 
\par 20 Viešpats nesigailės tokio, bet Jo pyktis ir pavydas užsidegs prieš tą žmogų, ir visi prakeikimai, surašyti šioje knygoje, kris ant jo, ir Viešpats išnaikins jo vardą iš po dangaus. 
\par 21 Viešpats atskirs jį pražūčiai iš visų Izraelio giminių, pagal visus prakeikimus, surašytus šitoje įstatymo knygoje. 
\par 22 Stebėsis būsimos kartos ir ateiviai, matydami šitos žemės kančias ir vargus, siųstus jiems Viešpaties. 
\par 23 Visa žemė bus siera, druska ir ugniavietė; nieko nebus joje sėjama, net žolės ten nebeaugs; bus sunaikinta lyg Sodoma ir Gomora, Adma ir Ceboimas, kuriuos Viešpats sunaikino savo rūstybėje. 
\par 24 Visos tautos stebėsis: ‘Kodėl Viešpats taip padarė šitam kraštui? Ką reiškia ta Jo rūstybė?’ 
\par 25 Žmonės atsakys: ‘Kadangi jie apleido Viešpaties, savo tėvų Dievo, sandorą, kurią Jis padarė su jais, kai išvedė juos iš Egipto žemės, 
\par 26 ir tarnavo kitiems dievams, garbindami juos, nors tie nebuvo jiems skirti, 
\par 27 Viešpaties rūstybė užsidegė prieš šitą kraštą ir Jis užleido ant jo visus prakeikimus, surašytus šitoje knygoje; 
\par 28 išrovė juos iš jų žemės savo rūstybėje ir ištrėmė į svetimą kraštą, kaip yra šiandien’. 
\par 29 Kas paslėpta, priklauso Viešpačiui, mūsų Dievui, o kas apreikšta­mums ir mūsų vaikams, kad per amžius vykdytume visus šito įstatymo žodžius”.



\chapter{30}


\par 1 “Kai visi šie dalykai ateis tau­palaiminimas ir prakeikimas, kuriuos tau paskelbiau,­tu prisiminsi juos gyvendamas tarp tautų, tarp kurių Viešpats, tavo Dievas, tave išsklaidys. 
\par 2 Tada tu su savo vaikais sugrįši pas Viešpatį, visa širdimi ir visa siela paklusi įsakymams, kuriuos tau šiandien skelbiu. 
\par 3 Tada Viešpats, tavo Dievas, pasigailės tavęs ir surinkęs iš visų tautų, tarp kurių buvai išblaškytas, sugrąžins tave iš nelaisvės. 
\par 4 Nors būtum išsklaidytas ligi dangaus pakraščių, Viešpats, tavo Dievas, ir iš ten parves tave. 
\par 5 Jis grąžins tave į žemę, kurią paveldėjo tavo tėvai, darys tau gera ir padaugins labiau už tavo tėvus. 
\par 6 Viešpats, tavo Dievas, apipjaustys tavo širdį ir tavo palikuonių širdis, kad mylėtum Viešpatį, savo Dievą, visa širdimi ir visa siela ir gyventum. 
\par 7 Visus šituos prakeikimus nukreips tavo priešams ir tiems, kurie tavęs neapkenčia ir persekioja. 
\par 8 Tu sugrįši, klausysi Viešpaties, savo Dievo, balso ir vykdysi visus Jo įsakymus, kuriuos tau šiandien skelbiu. 
\par 9 Tuomet Viešpats, tavo Dievas, laimins tavo darbus, vaikus, galvijus ir žemės derlių. Viešpats vėl džiaugsis, matydamas tavo gerovę, kaip džiaugėsi tavo tėvais, 
\par 10 jei klausysi Viešpaties, savo Dievo, balso, vykdysi Jo įsakymus bei paliepimus, kurie surašyti įstatymo knygoje, ir atsigręši į Viešpatį, savo Dievą, visa širdimi ir visa siela. 
\par 11 Šitas įsakymas, kurį šiandien skelbiu, nėra tau paslėptas ir nepasiekiamas. 
\par 12 Jis ne danguje, kad sakytum: ‘Kas už mus pakils iki dangaus ir mums jį atneš, kad klausytume ir vykdytume?’ 
\par 13 Ir ne už jūrų, kad sakytum: ‘Kas už mus perplauks jūras ir jį atneš, kad klausytume ir vykdytume?’ 
\par 14 Žodis yra labai arti tavęs­tavo burnoje ir tavo širdyje, kad jį vykdytum! 
\par 15 Šiandien leidžiu tau pasirinkti gyvenimą ir gėrį ar blogį ir mirtį. 
\par 16 Jei mylėsi Viešpatį, savo Dievą, vaikščiosi Jo keliais ir laikysies Jo paliepimų bei įsakymų, būsi gyvas; Jis padaugins ir palaimins tavo palikuonis žemėje, kurios paveldėti eini. 
\par 17 Bet jei tu priešinsies ir nenorėsi klausyti, nuklydęs garbinsi kitus dievus ir jiems tarnausi, 
\par 18 skelbiu šiandien, kad žūsi žemėje, kurios, perėjęs per Jordaną, paveldėti eini. 
\par 19 Šaukiu šiandien liudytojais dangų ir žemę, kad leidau tau pasirinkti gyvenimą ar mirtį, palaiminimą ar prakeikimą. Tad pasirink gyvenimą, kad būtum gyvas tu ir tavo palikuonys, 
\par 20 mylėtum Viešpatį, savo Dievą, klausytum Jo balso ir glaustumeis prie Jo, nes Jis yra tavo gyvenimas ir tavo dienų ilgumas; kad gyventum žemėje, kurią duoti Viešpats prisiekė tavo tėvams: Abraomui, Izaokui ir Jokūbui”.



\chapter{31}


\par 1 Mozė nuėjo ir kalbėjo šituos žodžius visam Izraeliui: 
\par 2 “Šiandien man jau šimtas dvidešimt metų, aš nebegaliu daugiau išeiti ir įeiti. Viešpats yra man pasakęs: ‘Nepereisi per Jordaną’. 
\par 3 Viešpats, tavo Dievas, eis pirma tavęs. Jis išnaikins visas tas tautas tau matant ir tu jas nugalėsi. Jozuė eis pirma tavęs, kaip Viešpats kalbėjo. 
\par 4 Viešpats padarys, kaip padarė amoritų karaliams Sihonui ir Ogui ir žemei tų, kuriuos Jis sunaikino. 
\par 5 Viešpats atiduos juos tau, kad pasielgtum su jais, kaip įsakiau. 
\par 6 Būk drąsus ir stiprus; nebijok ir neišsigąsk jų, nes Viešpats, tavo Dievas, yra su tavimi; Jis nepasitrauks ir nepaliks tavęs”. 
\par 7 Po to Mozė pasišaukė Jozuę ir jam tarė, izraelitams girdint: “Būk drąsus ir stiprus, tu įvesi šitą tautą į žemę, kurią Viešpats prisiekė duosiąs jų tėvams, ir tu ją jiems padalinsi. 
\par 8 Viešpats bus su tavimi; Jis nepasitrauks ir nepaliks tavęs; nebijok ir nenusigąsk!” 
\par 9 Mozė surašė šitą įstatymą ir padavė jį kunigams, Levio sūnums, kurie nešė Viešpaties Sandoros skrynią, ir visiems Izraelio vyresniesiems. 
\par 10 Jis jiems įsakė: “Kas septyneri metai, atleidimo metais, per Palapinių šventę, 
\par 11 visiems izraelitams susirinkus Viešpaties, tavo Dievo, pasirinktoje vietoje, skaitykite šito įstatymo žodžius visam Izraeliui girdint; 
\par 12 sušaukite žmones: vyrus, moteris, vaikus ir ateivius, gyvenančius su jais, kad klausydami mokytųsi ir bijotų Viešpaties, savo Dievo, ir vykdytų visus šio įstatymo žodžius; 
\par 13 kad jų vaikai, kurie dar jo nežino, girdėtų ir mokytųsi bijoti Viešpaties, savo Dievo, visą laiką, kol gyvensite žemėje, kurios einate paveldėti, persikėlę per Jordaną”. 
\par 14 Viešpats tarė Mozei: “Tavo mirties diena jau arti, pasišauk Jozuę ir atsistokite Susitikimo palapinėje, ten Aš jam duosiu nurodymus”. Mozė ir Jozuė nuėjo į Susitikimo palapinę, 
\par 15 o Viešpats pasirodė debesies stulpe. 
\par 16 Viešpats tarė Mozei: “Tu užmigsi su savo tėvais. Šita tauta vėl klaidžios, sekdama kitų tautų dievus žemėje, į kurią eina gyventi. Ten ji paliks mane ir sulaužys sandorą, kurią su ja padariau. 
\par 17 Tuomet mano rūstybė užsidegs prieš ją. Aš pasitrauksiu nuo jos, ir ji bus sunaikinta. Nelaimių ir vargų spaudžiama, ji supras, kad Aš ją palikau. 
\par 18 Aš negelbėsiu jos dėl visų piktybių, kurias ji darė, sekdama kitus dievus. 
\par 19 Taigi dabar užrašyk šitą giesmę ir liepk izraelitams išmokti ją giedoti, kad man šita giesmė būtų liudijimas prieš izraelitus. 
\par 20 Aš juos įvesiu į žemę, plūstančią pienu ir medumi, kaip prisiekiau jų tėvams. Kai jie pasisotins ir nutuks, jie garbins kitus dievus ir jiems tarnaus, o mane paniekins ir sulaužys mano sandorą. 
\par 21 Kai juos prislėgs nelaimės ir vargai, šita giesmė liudys prieš juos, nes ji bus jų palikuonių lūpose. Jau šiandien žinau jų mintis, ką jie darys, dar neįvedęs jų į žemę, kurią jiems pažadėjau”. 
\par 22 Mozė užrašė šią giesmę ir jos išmokė izraelitus. 
\par 23 Viešpats davė įsakymą Nūno sūnui Jozuei: “Būk drąsus ir stiprus, nes tu įvesi izraelitus į žemę, kurią pažadėjau, ir Aš būsiu su tavimi”. 
\par 24 Kai Mozė pabaigė užrašyti šito įstatymo žodžius knygoje, 
\par 25 įsakė levitams, kurie nešė Viešpaties Sandoros skrynią: 
\par 26 “Imkite šitą įstatymo knygą kaip liudijimą prieš jus ir ją padėkite prie Viešpaties, jūsų Dievo, Sandoros skrynios. 
\par 27 Aš žinau tavo maištingumą ir kietasprandiškumą. Dar man gyvam esant ir gyvenant su jumis, jūs maištavote prieš Viešpatį; juo labiau priešinsitės man mirus. 
\par 28 Sukvieskite pas mane visus giminių vadus ir vyresniuosius. Jiems girdint, paskelbsiu šituos žodžius ir liudininkais prieš jus šauksiu dangų ir žemę. 
\par 29 Aš žinau, kad man mirus jūs elgsitės piktai ir greitai nukrypsite nuo kelio, kurį jums nurodžiau. Vėliau jus užgrius nelaimės, kai, piktai elgdamiesi, sukelsite Viešpaties pyktį savo rankų darbais”. 
\par 30 Tada Mozė, visam Izraeliui girdint, kalbėjo šios giesmės žodžius.



\chapter{32}

\par 1 “Klausykite, dangūs! Aš kalbėsiu, žemė teišgirsta mano žodžius. 
\par 2 Mano pamokymai kris kaip lietus, mano žodžiai kaip rasa, kaip lietaus srovės ant žolės ir kaip lašai ant želmenų. 
\par 3 Aš skelbsiu Viešpaties vardą; atiduokite mūsų Dievui garbę! 
\par 4 Jis yra Uola; tobuli Jo darbai, visi Jo keliai pilni teisybės. Dievas ištikimas, be jokios neteisybės, Jis teisus ir teisingas. 
\par 5 Jie sugedo ir nėra Jo vaikai dėl savo išsigimimo, nedora ir iškrypusi karta. 
\par 6 Argi taip atsilygini Viešpačiui, kvaila ir neišmintinga tauta? Argi ne Jis tavo tėvas, kuris tave įsūnijo? Argi ne Jis padarė ir įtvirtino tave? 
\par 7 Atsimink senąsias dienas, apsvarstyk praeitų kartų laikus; klausk savo tėvo, jis tau pasakys, savo senelių­jie tau papasakos. 
\par 8 Aukščiausiasis skyrė tautoms kraštus, dalino žmonių vaikams žemes ir nustatė tautoms sienas pagal Izraelio vaikų skaičių. 
\par 9 Viešpaties nuosavybė yra Jo tauta, Jokūbas­Jo paveldėjimo dalis. 
\par 10 Jis ją rado negyvenamų dykumų platybėje; globojo ir rūpinosi ja, saugojo kaip savo akį. 
\par 11 Kaip erelis moko skristi savo jauniklius, pats sklando virš lizdo ištiesęs savo sparnus ir neša juos ant savo sparnų, 
\par 12 taip Viešpats vienas ją vedė; nebuvo su Juo jokio kito dievo. 
\par 13 Jis vedė ją žemės aukštumomis, maitino laukų vaisiais, davė medaus iš akmens ir aliejaus iš kietos uolos, 
\par 14 sviesto iš karvių, pieno iš avių, taukų iš ėriukų, Bašano avinų ir ožkų; gerų kviečių ir vyno iš vynuogių kraujo. 
\par 15 Nutukęs, suriebėjęs, sustorėjęs Ješurūnas paliko Dievą, savo Kūrėją, ir paniekino išgelbėjimo Uolą. 
\par 16 Svetimais dievais ir bjaurystėmis jie sukėlė Viešpaties rūstybę. 
\par 17 Jie aukojo demonams, ne Dievui, naujiems dievams, kurių jie nepažino, kurie ką tik pasirodė, kurių nebijojo jų tėvai. 
\par 18 Uolą, kuri tave pagimdė, tu paniekinai ir užmiršai Dievą, savo Kūrėją. 
\par 19 Viešpats tai matė ir bjaurėjosi jais, nes Jį supykdė Jo sūnūs ir dukterys. 
\par 20 Jis tarė: ‘Paslėpsiu nuo jų savo veidą ir žiūrėsiu, koks bus jų galas. Tai yra sugedusi karta, neištikimi vaikai. 
\par 21 Jie sukėlė mano pavydą tuo, kas nėra dievai, supykdė mane savo tuštybėmis; Aš sukelsiu jų pavydą tuo, kas nėra tauta, supykdysiu neišmanančia tauta. 
\par 22 Mano rūstybės ugnis užsidegė. Ji degins iki pragaro gelmių; ris žemę, jos vaisius ir kalnų pamatus. 
\par 23 Aš krausiu ant jų nelaimes ir šaudysiu į juos savo strėlėmis. 
\par 24 Jie bus varginami bado ir naikinami drugio bei baisiausio maro. Siųsiu jiems plėšrius žvėris ir nuodingas gyvates. 
\par 25 Lauke juos naikins kardas, o viduje­siaubas, negailėdamas jaunų, senų nė kūdikių. 
\par 26 Aš būčiau sunaikinęs juos visiškai, net prisiminimą apie juos išdildęs iš žmonių atminties, 
\par 27 bet nedariau to, kad kartais jų priešai nesugalvotų sakyti, jog jie sunaikino juos, o ne Viešpats. 
\par 28 Tai neišmintingi žmonės, neturintys supratimo. 
\par 29 O, kad jie būtų išmintingi ir suprastų tai, ir numatytų, koks bus jų galas. 
\par 30 Kaip vienas galėtų vyti tūkstantį ir du persekioti dešimt tūkstančių, jeigu jų Uola nebūtų atsisakiusi jiems padėti ir Viešpats nebūtų nuo jų pasitraukęs? 
\par 31 Mūsų Uola nėra tokia, kaip jų uola, patys mūsų priešai tai liudija. 
\par 32 Tikrai jų vynmedis yra iš Sodomos ir Gomoros laukų. Jų vynuogės yra nuodingos ir vynuogių kekės karčios. 
\par 33 Jų vynas yra slibinų ir angių nuodai. 
\par 34 Visa tai laikoma mano užantspauduotame sandėlyje. 
\par 35 Mano atlyginimas ir kerštas, kai jų kojos paslys. Jų pražūties diena arti, greitai juos ištiks tai, kas jiems skirta’. 
\par 36 Viešpats teis savo tautą ir pasigailės savo tarnų, kai jų jėgos bus išsekusios. 
\par 37 Jis sakys: ‘Kur jų dievai, uola, kuria jie pasitikėjo? 
\par 38 Kur tie, kurie valgė jų aukų taukus ir gėrė jų aukų vyną? Tegul jie pakyla ir padeda jums. 
\par 39 Supraskite, kad Aš esu vienas ir šalia manęs nėra kito dievo. Aš užmušu ir atgaivinu, sužeidžiu ir gydau; ir niekas neišgelbės iš mano rankos. 
\par 40 Aš, pakėlęs ranką į dangų, sakau­ Aš esu gyvas per amžius. 
\par 41 Aš išgaląsiu savo žibantį kardą ir teisiu. Atkeršysiu savo priešams ir atlyginsiu tiems, kurie manęs nekenčia. 
\par 42 Mano strėlės pasigers nuo kraujo, o mano kardas ris mėsą, kraują užmuštųjų ir belaisvių, priešo vadų galvas’. 
\par 43 Džiaukitės, tautos, kartu su Jo tauta, nes Jis atkeršys už savo tarnų kraują, kerštu atlygins priešams ir pasigailės savo žemės ir savo žmonių”. 
\par 44 Mozė paskelbė šią giesmę tautai, jis ir Nūno sūnus Jozuė. 
\par 45 Mozė, pabaigęs kalbėti Izraeliui, 
\par 46 tarė: “Įsidėkite į širdis visus šituos žodžius, kuriuos jums šiandien paskelbiau, perduokite juos savo vaikams ir įsakykite vykdyti viską, kas parašyta šitame įstatyme. 
\par 47 Tai ne tuščias dalykas, nes tai yra jūsų gyvenimas. Jų dėka ilgai gyvensite žemėje, kurios einate paveldėti anapus Jordano”. 
\par 48 Viešpats tą pačią dieną kalbėjo Mozei: 
\par 49 “Eik į Abarimo kalnyną ir užlipk ant Nebojo kalno, kuris yra Moabo žemėje ties Jerichu; apžvelk Kanaano žemę, kurią duosiu paveldėti izraelitams. 
\par 50 Mirsi ant to kalno ir susijungsi su savo tauta kaip tavo brolis Aaronas, kuris mirė Horo kalne ir susijungė su savo tauta. 
\par 51 Kadangi judu nusikaltote man prie Meribos vandenų Cino dykumoje, Kadeše, ir neparodėte mano šventumo tarp izraelitų, 
\par 52 tu matysi žemę, kurią duodu izraelitams, bet neįeisi į ją”.
Online Parallel Study Bible



\chapter{33}

\par 1 Palaiminimo žodžiai, kuriais Mozė, Dievo vyras, laimino izraelitus prieš mirdamas: 
\par 2 “Viešpats atėjo nuo Sinajaus ir Seyro; Jis suspindėjo nuo Parano kalno; su Juo buvo tūkstančiai šventųjų; Jo dešinėje­įstatymo liepsna. 
\par 3 Jis myli savo tautą, visi šventieji priklauso Jam, jie atsisėdo prie Jo kojų, kad išgirstų Jo žodžius. 
\par 4 Mozė paskelbė įstatymą Jokūbo palikuonims. 
\par 5 Jis buvo Ješurūno karaliumi, kai susirinko tautos vadai ir Izraelio giminės. 
\par 6 Tegyvena Rubenas, tedaugėja jo palikuonių”. 
\par 7 Palaiminimas Judui: “Viešpatie, išgirsk Judo balsą, atvesk jį pas savo tautą ir padėk jam kovoje su priešais”. 
\par 8 Apie Levį jis tarė: “Tavo Tumimas ir Urimas tebūna su šventuoju, kurį išbandei Masoje ir su kuriuo kovojai prie Meribos vandenų. 
\par 9 Jis sakė savo tėvui ir motinai: ‘Nepažįstu jūsų’, o savo broliams: ‘Nežinau jūsų’, ir atstūmė savo vaikus. Jis laikėsi Viešpaties žodžio ir sandoros. 
\par 10 Jis mokys Tavo įsakymų Jokūbą ir įstatymų Izraelį, aukos smilkalus Tavo garbei ir deginamąją auką ant Tavo aukuro. 
\par 11 Viešpatie, palaimink jo turtą ir priimk jo darbą. Gink jį nuo priešų, kurie jo nekenčia”. 
\par 12 Benjaminui jis tarė: “Tu, Viešpaties mylimasis, gyvensi Jo globoje. Jis apsaugos tave visuomet, ir tu ilsėsies Jo glėbyje”. 
\par 13 Apie Juozapą jis tarė: “Jo žemę Viešpats laimins dovanomis iš dangaus, rasa, trykštančiais šaltiniais, 
\par 14 saulėje nunokusiais geriausiais vaisiais, 
\par 15 vaisiais nuo senųjų kalnų viršūnių ir amžinųjų kalvų 
\par 16 ir žemės geriausiu derliumi. Jis bus palaimintas Apsireiškusiojo krūme. Palaiminimai teužgriūna ant Juozapo galvos, ant galvos to, kuris buvo atskirtas nuo savo brolių. 
\par 17 Jis yra stiprus kaip jautis, jo ragai lyg stumbro, kuriais jis pasieks tautas iki žemės pakraščių. Manaso tūkstančiai ir Efraimo tūkstančių tūkstančiai”. 
\par 18 Zabulonui jis tarė: “Džiaukis, Zabulonai, prekyba, o tu, Isacharai, turtais savame krašte. 
\par 19 Jie kvies tautas į kalną ir ten aukos teisingumo aukas. Jie praturtės iš jūros ir jos krantų”. 
\par 20 Apie Gadą jis pasakė: “Palaimintas tas, kuris padėjo Gadui įsigyti žemės plotus; jis kaip liūtas ilsisi, sutraiškydamas ir ranką, ir galvą. 
\par 21 Jis, pasinaudojęs pirmenybe, pasiėmė geriausią žemę kaip tos giminės vadas. Su tautos vadais jis įvykdė Viešpaties įsakymus ir įstatymus, duotus Izraeliui”. 
\par 22 Danui jis tarė: “Danas kaip jaunas liūtas iššoka iš Bašano”. 
\par 23 Neftaliui jis sakė: “Neftalis džiaugsis gerove ir Viešpaties palaiminimais; jis paveldės vakarus ir pietus”. 
\par 24 Ašerui jis tarė: “Palaimintas Ašeras sūnumis. Jis bus brolių mylimas ir jo žemėse bus daug alyvmedžių. 
\par 25 Geležies ir vario užkaiščiai saugos jo miestus. Jo gyvenimas bus saugus”. 
\par 26 “Nėra lygių Ješurūno Dievui. Iš dangaus Jis teikia tau pagalbą. Jo didybė pasireiškia aukštybėse. 
\par 27 Amžinasis Dievas yra tavo apsauga, Jo rankos­tavo prieglauda. Jis ištiesia jas į priešą ir tu sunaikini jį. 
\par 28 Izraelis gyvens saugiai, Jokūbo šaltinis bus geroje, kviečių ir vyno žemėje; Jo dangūs siųs rasą. 
\par 29 Laimingas tu, Izraeli! Kas prilygs tau? Tauta, išgelbėta Viešpaties. Jis tavo apsaugos skydas ir didybės kardas. Tavo priešai pasiduos tau, o tu mindžiosi jų sprandus”.



\chapter{34}


\par 1 Mozė iš Moabo lygumų nuėjo iki Nebojo kalno ir įlipo į Pisgos viršūnę ties Jerichu; ten Viešpats jam parodė visą Gileadą iki Dano žemių 
\par 2 ir visas Neftalio, Efraimo, Manaso ir Judo žemes iki Vakarų jūros, 
\par 3 krašto pietinę dalį, palmių miesto Jericho slėnį iki Coaro. 
\par 4 Viešpats jam tarė: “Tai žemė, kurią pažadėjau Abraomui, Izaokui ir Jokūbui duoti jų palikuonims. Tu ją matai savo akimis, bet į ją neįeisi”. 
\par 5 Viešpaties tarnas Mozė mirė Moabo žemėje, kaip Viešpats buvo sakęs. 
\par 6 Jis jį palaidojo Moabo žemės slėnyje ties Bet Peoru; bet nė vienas žmogus nežino jo kapo iki šios dienos. 
\par 7 Mozė mirė šimto dvidešimties metų; nenusilpo jo akys ir jo jėgos neišseko. 
\par 8 Izraelitai jį apraudojo trisdešimt dienų Moabo lygumose. Po to gedulas pasibaigė. 
\par 9 Nūno sūnus Jozuė buvo pilnas išminties dvasios, nes Mozė buvo uždėjęs ant jo rankas. Izraelitai jo klausė ir darė, kaip Viešpats buvo įsakęs Mozei. 
\par 10 Niekad nebebuvo Izraelyje tokio pranašo, kaip Mozė, su kuriuo Viešpats būtų kalbėjęs veidas į veidą, 
\par 11 kuris darytų tokius ženklus ir stebuklus, kuriuos daryti Viešpats pasiuntė Mozę į Egipto žemę faraonui, jo tarnams, ir visam jo kraštui; 
\par 12 kuris parodytų galingą ranką ir baisius dalykus, kuriuos Mozė darė izraelitų akivaizdoje.



\end{document}