\begin{document}

\title{Teisėjų knyga}

\chapter{1}

\par 1 Jozuei mirus, izraelitai klausė Viešpaties: “Kas iš mūsų eis pirmas kare su kanaaniečiais?” 
\par 2 Viešpats atsakė: “Judas eis pirmas; Aš atidaviau žemę į jo rankas”. 
\par 3 Tada Judas tarė savo broliui Simeonui: “Eime su manimi į mano kraštą ir kariaukime prieš kanaaniečius! Po to aš trauksiu su tavimi į tavo kraštą”. Ir Simeonas nuėjo su juo. 
\par 4 Juodu išžygiavo, ir Viešpats atidavė į jų rankas kanaaniečius ir perizus; jie išžudė prie Bezeko dešimt tūkstančių vyrų. 
\par 5 Ten jie sutiko Adoni Bezeką, kariavo prieš jį ir sumušė kanaaniečius ir perizus. 
\par 6 Adoni Bezekas pabėgo, bet jie vijosi jį ir sugavę nukirto jam rankų ir kojų nykščius. 
\par 7 Adoni Bezekas tarė: “Septyniasdešimt karalių su nukirstais rankų bei kojų nykščiais rinkdavo mano pastalėje trupinius. Kaip aš dariau, taip ir Dievas man atlygino”. Jie nuvedė jį į Jeruzalę, kur jis mirė. 
\par 8 Judas kariavo prieš Jeruzalę, užėmęs ją, gyventojus išžudė kardu ir miestą padegė. 
\par 9 Paskui jis kariavo prieš kanaaniečius, gyvenusius kalnyne, pietuose, žemumoje 
\par 10 ir Hebrone; sumušė Šešają, Ahimaną ir Talmają. Hebronas anksčiau vadinosi Kirjat Arba. 
\par 11 Iš ten jie žygiavo prieš Debyrą, kuris anksčiau vadinosi Kirjat Seferas. 
\par 12 Tada Kalebas tarė: “Kas nugalės Kirjat Seferą ir jį užims, tam duosiu savo dukterį Achsą į žmonas”. 
\par 13 Jį užėmė jaunesniojo Kalebo brolio Kenazo sūnus Otnielis. Kalebas atidavė jam savo dukterį Achsą. 
\par 14 Kai ji ištekėjo, Otnielis prikalbėjo ją prašyti iš savo tėvo dirbamos žemės. Jai nulipus nuo asilo, Kalebas klausė: “Ko nori?” 
\par 15 Ji tarė jam: “Tėve, palaimink mane! Tu davei man sausos žemės, duok man ir vandens versmių”. Tada Kalebas davė jai aukštutines ir žemutines versmes. 
\par 16 Mozės uošvio kenito palikuonys traukė iš Palmių miesto su Judu į pietus Arado link. Ten nuėję, jie apsigyveno. 
\par 17 Po to Judas su savo broliu Simeonu sumušė kanaaniečius, gyvenusius Cefate, ir sunaikino jį. Todėl tą miestą pavadino Horma. 
\par 18 Judas užėmė Gazą, Aškeloną ir Ekroną su jų apylinkėmis. 
\par 19 Viešpats buvo su Judu, ir jis užėmė kalnyną. Bet jis nepajėgė išstumti lygumos gyventojų, nes jie turėjo geležinių kovos vežimų. 
\par 20 Jie atidavė Kalebui Hebroną, kaip Mozė buvo įsakęs, iš kurio jis išvarė tris Anako sūnus. 
\par 21 Benjamino vaikai neišstūmė jebusiečių, gyvenusių Jeruzalėje; jie pasiliko gyventi su benjaminitais iki šios dienos. 
\par 22 Juozapo palikuonys žygiavo prieš Betelį, ir Viešpats buvo su jais. 
\par 23 Juozapo palikuonys siuntė žvalgus į Betelį, kuris anksčiau vadinosi Lūzas. 
\par 24 Žvalgai, sutikę žmogų, išeinantį iš miesto, jam tarė: “Parodyk mums įėjimą į miestą, mes tavęs pasigailėsime”. 
\par 25 Jis parodė jiems įėjimą į miestą. Jie išžudė miesto gyventojus kardu, bet tą vyrą ir jo šeimą paleido. 
\par 26 Tas vyras, nukeliavęs į hetitų šalį, įkūrė miestą ir jį pavadino Lūzu. Jis taip vadinasi iki šios dienos. 
\par 27 Manasas neišvarė Bet Šeano, Taanacho, Doro, Ibleamo ir Megido gyventojų iš jų miestų ir kaimų; kanaaniečiai ir toliau gyveno šalyje. 
\par 28 Izraelis sustiprėjęs privertė kanaaniečius mokėti duoklę, bet jų neišvarė. 
\par 29 Ir Efraimas neišvarė kanaaniečių, gyvenančių Gezeryje; kanaaniečiai liko gyventi Gezeryje tarp jų. 
\par 30 Ir Zabulonas neišvarė Kitrono bei Nahalolo gyventojų; ir kanaaniečiai gyveno tarp jų, mokėdami duoklę. 
\par 31 Ašeras neišvarė gyventojų iš Sidono, Achlabo, Achzibo, Helbos, Afeko ir Rehobo. 
\par 32 Ašerai gyveno tarp kanaaniečių, to krašto gyventojų, nes jie nebuvo išvaryti. 
\par 33 Neftalis irgi neišvarė Bet Šemešo nei Bet Anato gyventojų; jie liko gyventi Kanaano šalyje, bet mokėjo Neftaliui duoklę. 
\par 34 Amoritai spaudė Dano giminę kalnuose ir neleido jiems nusileisti į lygumą. 
\par 35 Amoritai toliau gyveno Har Herese, Ajalone ir Šaalbime. Tačiau Juozapo giminė nugalėjo juos ir privertė mokėti duoklę. 
\par 36 Amoritų žemės tęsėsi nuo Akrabimo aukštumos ir Selos į kalnus.


\chapter{2}

\par 1 Viešpaties angelas, atėjęs iš Gilgalos į Bochimą, tarė: “Aš išvedžiau jus iš Egipto ir atvedžiau į žemę, kurią prisiekiau duoti jūsų tėvams, ir pasakiau: ‘Aš niekada nelaužysiu savo sandoros su jumis, 
\par 2 o jūs nedarykite sutarčių su šios žemės gyventojais, sugriaukite jų aukurus’. Bet jūs nepaklusote mano balsui. Kodėl jūs taip padarėte? 
\par 3 Todėl Aš sakau, kad neišvarysiu jų iš jūsų krašto, jie bus dygliai jūsų šone, o jų dievai bus jums spąstai”. 
\par 4 Viešpaties angelui kalbant šiuos žodžius, izraelitai pakėlė savo balsus ir verkė. 
\par 5 Todėl šią vietą pavadino Bochimu. Jie čia aukojo Viešpačiui. 
\par 6 Kai Jozuė paleido tautą, kiekvienas nuėjo į savo nuosavybę ir ten apsigyveno. 
\par 7 Tauta tarnavo Viešpačiui per visas Jozuės dienas ir kol buvo gyvi vyresnieji, kurie matė visus Viešpaties darbus, padarytus Izraelio tautai. 
\par 8 Nūno sūnus Jozuė, Viešpaties tarnas, mirė sulaukęs šimto dešimties metų amžiaus. 
\par 9 Jie palaidojo jį jo nuosavybėje, Timnat Herese, Efraimo kalnyne, į šiaurę nuo Gaašo kalno. 
\par 10 Visa ta karta išmirė, užaugo kita karta, kuri nepažino Viešpaties nė matė Jo darbų Izraeliui. 
\par 11 Izraelitai nusikalto Viešpačiui, tarnaudami Baaliui. 
\par 12 Jie apleido Viešpatį, savo tėvų Dievą, kuris juos išvedė iš Egipto šalies, ir sekė svetimus dievus tautų, gyvenančių aplink juos, ir jiems lenkėsi, sukeldami Viešpaties pyktį. 
\par 13 Jie apleido Viešpatį ir tarnavo Baaliui ir Astartei. 
\par 14 Viešpaties rūstybė užsidegė prieš Izraelį, ir Jis atidavė juos į plėšikų rankas, kurie juos plėšdavo. Viešpats atidavė izraelitus jų priešams, prieš kuriuos jie nepajėgė atsilaikyti. 
\par 15 Kur jie beeidavo, Viešpaties ranka buvo prieš juos, darydama jiems pikta, kaip Viešpats jiems buvo prisiekęs. Jie buvo sunkiai varginami. 
\par 16 Tačiau Viešpats siuntė izraelitams teisėjų, kurie juos išgelbėdavo iš rankos tų, kurie juos plėšė. 
\par 17 Bet jie neklausė teisėjų ir nuėjo paleistuvauti su svetimais dievais, ir lenkėsi jiems. Jie greitai nuklydo nuo kelio, kuriuo ėjo jų tėvai, kai klausė Viešpaties įsakymų. 
\par 18 Kai Viešpats jiems duodavo teisėją, Jis būdavo su juo ir išgelbėdavo izraelitus iš jų priešų per visas to teisėjo dienas. Viešpats gailėdavosi jų, kai jie dejuodami skųsdavosi prispaudėjais. 
\par 19 Bet, teisėjui mirus, jie sugrįždavo ir susitepdavo dar labiau, negu jų tėvai, sekdami svetimus dievus, jiems tarnaudami ir juos garbindami. Jie neatsisakė savo darbų ir užsispyrimo. 
\par 20 Viešpaties rūstybė užsidegė prieš Izraelį, ir Jis tarė: “Kadangi šita tauta sulaužė mano sandorą, kurią padariau su jų tėvais, ir neklausė manęs, 
\par 21 tai ir Aš neišvarysiu iš jų krašto nė vienos tų tautų, kurias Jozuė mirdamas paliko; 
\par 22 jomis išmėginsiu Izraelį, ar jis laikysis Viešpaties kelio ir ar norės juo vaikščioti, kaip darė jo tėvai”. 
\par 23 Todėl Viešpats paliko tas tautas krašte, neišnaikino jų tuojau ir neatidavė jų į Jozuės rankas.



\chapter{3}

\par 1 Šitas tautas Viešpats paliko krašte, norėdamas išmėginti izraelitus, kurie nebuvo pergyvenę Kanaano karų, 
\par 2 kad izraelitų kartos žinotų, kas yra karas, ir išmoktų kariauti: 
\par 3 penkis filistinų kunigaikščius, visus kanaaniečius, sidoniečius ir hivus, kurie gyveno Libano kalnyne nuo Baal Hermono kalno iki Lebo Hamato slėnio. 
\par 4 Jie buvo palikti išmėginti izraelitus, ar jie klausys Viešpaties įsakymų, kuriuos Jis davė jų tėvams per Mozę. 
\par 5 Izraelitai gyveno tarp kanaaniečių, hetitų, amoritų, perizų, hivų ir jebusiečių. 
\par 6 Jie vedė jų dukteris, savo dukteris davė jų sūnums ir tarnavo jų dievams. 
\par 7 Izraelitai darė pikta Viešpaties akivaizdoje, pamiršo Viešpatį, savo Dievą, ir tarnavo Baaliams ir alkams. 
\par 8 Viešpaties rūstybė užsidegė prieš Izraelį, ir Jis atidavė jį į Mesopotamijos karaliaus Kušan Rišataimo rankas. Izraelitai tarnavo Kušan Rišataimui aštuonerius metus. 
\par 9 Kai izraelitai šaukėsi Viešpaties, Viešpats pakėlė jiems išlaisvintoją, jaunesniojo Kalebo brolio Kenazo sūnų Otnielį, kuris juos išgelbėjo. 
\par 10 Viešpaties Dvasia nužengė ant jo, ir jis tapo Izraelio teisėju. Jis išėjo į karą prieš Mesopotamijos karalių Kusan Rasataimą, ir Viešpats atidavė Kusan Rasataimą į jo rankas. 
\par 11 Kraštas ilsėjosi keturiasdešimt metų. Ir Kenazo sūnus Otnielis mirė. 
\par 12 Tuomet izraelitai vėl darė pikta Viešpaties akivaizdoje. Viešpats sustiprino Moabo karalių Egloną prieš Izraelį, kadangi jie piktai elgėsi Viešpaties akivaizdoje. 
\par 13 Jis su amonitais ir amalekiečiais pakilo prieš Izraelį, jį sumušė ir užėmė Palmių miestą. 
\par 14 Izraelitai tarnavo Moabo karaliui Eglonui aštuoniolika metų. 
\par 15 Kai izraelitai šaukėsi Viešpaties, Jis siuntė jiems gelbėtoją, Gero sūnų Ehudą, kairiarankį, iš Benjamino giminės. Izraelitai per jį siuntė dovaną Moabo karaliui Eglonui. 
\par 16 Ehudas pasidarė dviašmenį durklą vienos uolekties ilgio ir jį diržu prisijuosė po savo drabužiais dešinėje pusėje. 
\par 17 Jis pristatė dovaną Moabo karaliui Eglonui, kuris buvo labai storas vyras. 
\par 18 Įteikęs dovaną, Ehudas pasiuntė namo vyrus, nešusius dovaną, 
\par 19 o pats nuo Gilgalos akmeninių stabų sugrįžo pas Egloną ir tarė: “Aš turiu slaptą žinią tau, karaliau”. Karalius pasakė: “Tylos!” Ir visi, stovėjusieji prie jo, išėjo. 
\par 20 Ehudas nuėjo pas jį. Karalius sėdėjo vėsiame, antrame aukšte jam įrengtame kambaryje. Ehudas jam tarė: “Turiu tau žinią nuo Dievo”. Karalius atsistojo. 
\par 21 Ehudas, kairiąja ranka paėmęs durklą nuo savo dešiniojo šono, įsmeigė jį į karaliaus pilvą 
\par 22 taip, kad rankena sulindo paskui ašmenis ir taukai apdengė durklą, ir jis negalėjo jo ištraukti; ir nešvarumai išėjo lauk. 
\par 23 Ehudas išėjo į prieškambarį, uždarė ir užrakino to kambario duris. 
\par 24 Jam išėjus, atėję karaliaus tarnai pamatė, kad antro aukšto kambario durys užrakintos. Jie pagalvojo, kad jis atlieka savo reikalą vėsiame kambaryje. 
\par 25 Jie laukė, nes gėdijosi įeiti, tačiau jis neatidarė kambario durų. Pasiėmę raktą, jie atrakino ir pamatė, kad jų valdovas guli ant žemės negyvas. 
\par 26 Ehudas, kol jie delsė, pabėgo ir pro akmeninius stabus pasiekė Seyrą. 
\par 27 Atvykęs jis trimitavo Efraimo kalnuose. Išgirdę trimitą, izraelitai nuo kalnų rinkosi prie jo, ir jis jiems vadovavo. 
\par 28 Jis įsakė: “Sekite mane, nes Viešpats atidavė į jūsų rankas jūsų priešus moabitus!” Jie ėjo paskui jį ir, užėmę Jordano brastas, kuriomis pereinama į Moabą, niekam neleido pereiti. 
\par 29 Tuo metu jie nužudė apie dešimt tūkstančių moabitų, tvirtų ir narsių vyrų, ir nė vienas neištrūko. 
\par 30 Taip tuomet Moabas buvo Izraelio pavergtas. Kraštas ilsėjosi aštuoniasdešimt metų. 
\par 31 Po jo valdė Anato sūnus Šamgaras, kuris užmušė šešis šimtus filistinų lazda jaučiams varyti ir išlaisvino Izraelį.



\chapter{4}

\par 1 Ehudui mirus, izraelitai ir toliau darė pikta Viešpaties akivaizdoje. 
\par 2 Viešpats atidavė juos į Kanaano karaliaus Jabino rankas, kuris karaliavo Hacore. Jo kariuomenės vadas Sisera gyveno Harošet ha Goime. 
\par 3 Izraelitai šaukėsi Viešpaties, nes Jabinas turėjo devynis šimtus geležinių kovos vežimų ir smarkiai spaudė izraelitus dvidešimt metų. 
\par 4 Pranašė Debora, Lapidoto žmona, tuo metu buvo teisėja Izraelyje. 
\par 5 Ji gyveno po Deboros palme, tarp Ramos ir Betelio, Efraimo kalnuose, ir izraelitai ateidavo pas ją bylinėtis. 
\par 6 Ji pasišaukė Abinoamo sūnų Baraką iš Neftalio Kedešo ir tarė jam: “Viešpats, Izraelio Dievas, tau įsako eiti į Taboro kalną su dešimt tūkstančių vyrų iš Neftalio ir Zabulono giminių, 
\par 7 o Jis atves pas tave prie Kišono upelio Jabino kariuomenės vadą Siserą su jo vežimais bei visa kariuomene ir jį atiduos į tavo rankas”. 
\par 8 Barakas jai atsakė: “Jei tu eisi su manimi, aš eisiu, o jei neisi su manimi­neisiu”. 
\par 9 Ji tarė: “Aš eisiu su tavimi, tačiau tu nepasižymėsi žygyje, nes Viešpats atiduos Siserą į moters rankas”. Ir Debora su Baraku ėjo į Kedešą. 
\par 10 Barakas sušaukė Zabulono ir Neftalio vyrus į Kedešą. Jį sekė dešimt tūkstančių vyrų, ir Debora ėjo su juo. 
\par 11 Heberas, kainitas, atsiskyrė nuo kainitų, Mozės giminaičio Hobabo palikuonių. Jis pasistatė savo palapinę po ąžuolu arti Caanaimo Kedeše. 
\par 12 Kai Siserai pranešė, kad Abinoamo sūnus Barakas nužygiavo į Taboro kalną, 
\par 13 Sisera sušaukė visus savo kovos vežimus­devynis šimtus geležinių vežimų ir visus karius Harošet ha Goime prie Kišono upelio. 
\par 14 Tada Debora tarė Barakui: “Pakilk, šiandien Viešpats atidavė Siserą į tavo rankas. Tikrai Viešpats yra su tavimi!” Barakas nusileido nuo Taboro kalno su dešimčia tūkstančių vyrų. 
\par 15 Viešpats taip išgąsdino Siserą, kad jo kovos vežimai ir visa kariuomenė, pamatę Baraką, pakriko, pats Sisera, nušokęs nuo vežimo, pabėgo pėsčias. 
\par 16 O Barakas vijosi kovos vežimus ir kariuomenę iki Harošet ha Goimo. Visa Siseros kariuomenė buvo išžudyta kardu, nė vienas neišliko gyvas. 
\par 17 Sisera pėsčias nubėgo į kainito Hebero žmonos Jaelės palapinę, nes tarp Hacoro karaliaus Jabino ir kainito Hebero buvo taika. 
\par 18 Jaelė išėjo Siseros sutikti ir jį pakvietė: “Užsuk, viešpatie, užsuk pas mane, nebijok!” Jis užėjo pas ją į palapinę, o ji apklojo jį antklode. 
\par 19 Jis prašė vandens atsigerti, nes buvo labai ištroškęs. Ji, atrišusi pieno odinę, davė jam atsigerti ir vėl jį užklojo. 
\par 20 Sisera prašė jos atsistoti palapinės angoje ir, jei kas atėjęs klaustų, ar yra pas ją kas nors, atsakyti, kad nėra. 
\par 21 Hebero žmona Jaelė paėmė palapinės kuolelį, kūjį ir tylomis įėjusi kuolelį įkalė jam į smilkinį taip, kad jį prismeigė prie žemės, kai Sisera buvo labai nuvargęs ir giliai miegojo. Taip jis mirė. 
\par 22 Kai pasirodė Barakas, besivydamas Siserą, Jaelė, išėjusi jo pasitikti, tarė: “Eikš, aš tau parodysiu vyrą, kurio ieškai”. Jis įėjo į jos palapinę ir pamatė Siserą, gulintį negyvą su įsmeigtu kuoleliu smilkinyje. 
\par 23 Taip Dievas tuomet pažemino Kanaano karalių Jabiną izraelitų akyse. 
\par 24 Izraelitai vis labiau spaudė Kanaano karalių Jabiną, kol jį visai sunaikino.



\chapter{5}

\par 1 Debora ir Abinoamo sūnus Barakas tą dieną giedojo: 
\par 2 “Šlovinkite Viešpatį, nes Izraelis atkeršijo, kai žmonės noriai aukojosi. 
\par 3 Išgirskite, karaliai, klausykitės, kunigaikščiai! Aš giedosiu Viešpačiui ir girsiu Viešpatį, Izraelio Dievą! 
\par 4 Viešpatie, Tau išeinant iš Seyro ir žygiuojant iš Edomo laukų, žemė drebėjo, dangūs verkė ir iš debesų lašėjo vanduo. 
\par 5 Kalnai tirpo prieš Viešpatį, net Sinajus Viešpaties, Izraelio Dievo, akivaizdoje. 
\par 6 Anato sūnaus Šamgaro ir Jaelės dienomis keliai ištuštėjo ir keliautojai pasuko aplinkiniais keliais. 
\par 7 Gyventojai kaimuose nyko, kol aš, Debora, iškilau kaip motina Izraelyje. 
\par 8 Jie pasirinko naujus dievus, ir kilo karas prie jų vartų. Ar buvo skydas arba ietis tarp keturiasdešimties tūkstančių izraelitų? 
\par 9 Mano širdis palinkusi prie Izraelio vadovų, kurie noriai aukojasi už tautą! Girkite Viešpatį, 
\par 10 kurie jojate ant baltų asilų, sėdite ant brangių kilimų ir einate keliu. 
\par 11 Toli nuo šaulių balsų, prie vandens šulinių skelbkite Viešpaties teisius darbus, kuriuos Jis padarė Izraelio kaimams. Tada Viešpaties tauta nusileido prie vartų. 
\par 12 Pabusk, pabusk, Debora! Pabusk, pabusk ir giedok giesmę! Pakilk, Barakai, Abinoamo sūnau, vesk savo belaisvius! 
\par 13 Tam, kuris pasiliko, Jis atidavė valdyti kilniuosius tautoje. Viešpats pavedė man valdyti galinguosius. 
\par 14 Būriai iš Efraimo, gyveną Amaleko krašte, atėjo paskui Benjamino būrius, taip pat Machyro valdytojai ir Zabulono raštininkai. 
\par 15 Isacharo kunigaikščiai su Debora ir Baraku skubėjo į slėnį. Rubenas buvo pasidalinęs. 
\par 16 Ko pasilikai tarp avių gardų ir klausai kaimenės bliovimo? Rubeno pulkuose kilo nesutarimai. 
\par 17 Gileadas pasiliko anapus Jordano, Danas­prie laivų, o Ašeras liko ant jūros kranto. 
\par 18 Bet Zabulono ir Neftalio vyrai statė mirties pavojun savo gyvybes aukštumose. 
\par 19 Kanaano karaliai kariavo Taanache, prie Megido vandenų; sidabro grobio jie negavo. 
\par 20 Ir dangaus žvaigždės kovojo su Sisera. 
\par 21 Kišono upė nunešė juos. Mano siela, tu pamynei jėgą. 
\par 22 Žirgams kanopos sutrupėjo nuo bėgimo, kai bėgo jų galingieji. 
\par 23 Viešpaties angelas tarė: ‘Prakeikite Merozą, prakeikite jo gyventojus, nes jie neatėjo Viešpačiui į pagalbą kovoje su galiūnais’. 
\par 24 Kainito Hebero žmona Jaelė palaiminta labiau už visas moteris, labiau už visas moteris savo palapinėje. 
\par 25 Vandens jis prašė, bet ji davė jam pieno, brangiame dubenyje ji atnešė sviesto. 
\par 26 Į savo ranką ji paėmė kuolelį, o į savo dešinę­darbininkų kūjį. Ji užmušė Siserą, sutriuškino jo galvą ir pervėrė jam smilkinius. 
\par 27 Prie jos kojų jis susmuko, parkrito ir gulėjo, prie jos kojų susmuko, parkrito. Kur jis susmuko, ten parkrito negyvas. 
\par 28 Pro lango groteles žiūrėjo Siseros motina, laukdama jo grįžtant: ‘Kodėl taip ilgai neparvažiuoja jo kovos vežimas? Kodėl užtrunka jo kovos vežimo ratai?’ 
\par 29 Jos išmintingosios moterys jai sakė, o motina pakartojo jų žodžius: 
\par 30 ‘Tikrai jie dalinasi grobį: po vieną, o gal po dvi mergaites kiekvienam vyrui. Margi grobio audiniai Siserai. Iš abiejų pusių išsiuvinėti audiniai jo kaklui’. 
\par 31 Sunaikink visus savo priešus, Viešpatie! Tave mylintieji tebūna kaip saulė, tekanti savo jėgoje”. Kraštas ilsėjosi keturiasdešimt metų.



\chapter{6}


\par 1 Izraelitai darė pikta Viešpaties akivaizdoje, ir Viešpats juos atidavė į midjaniečių rankas septyneriems metams. 
\par 2 Midjaniečiai spaudė izraelitus, todėl jie pasidarė kalnuose lindynių, olų ir tvirtovių. 
\par 3 Izraelitams apsėjus dirvas, ateidavo midjaniečiai, amalekiečiai ir rytų gyventojai. 
\par 4 Pasistatę stovyklas, jie sunaikindavo krašto derlių iki Gazos, nepalikdami Izraeliui jokio maisto: nei avių, nei jaučių, nei asilų. 
\par 5 Jie ateidavo su savo gyvuliais ir palapinėmis, ir jų buvo daug kaip skėrių­neįmanoma suskaičiuoti nei jų, nei jų kupranugarių. Jie įsiverždavo į šalį, kad ją sunaikintų. 
\par 6 Midjaniečiai labai nuskurdino Izraelį, ir izraelitai šaukėsi Viešpaties. 
\par 7 Kai izraelitai šaukėsi Viešpaties dėl midjaniečių, 
\par 8 Jis atsiuntė izraelitams pranašą, kuris jiems paskelbė Viešpaties, Izraelio Dievo, žodžius: “Aš jus išvedžiau iš Egipto vergijos, 
\par 9 Aš išgelbėjau jus iš egiptiečių ir iš visų jūsų priešų, Aš išvariau juos nuo jūsų ir jums atidaviau jų žemę. 
\par 10 O jums pasakiau: ‘Aš esu Viešpats, jūsų Dievas; negarbinkite amoritų, kurių krašte gyvenate, dievų. Bet jūs neklausėte manęs’ ”. 
\par 11 Atėjęs Viešpaties angelas atsisėdo po ąžuolu Ofroje, kuris priklausė abiezeriui Jehoašui. Jo sūnus Gedeonas kūlė kviečius prie vyno spaustuvo, pasislėpęs nuo midjaniečių. 
\par 12 Jam pasirodė Viešpaties angelas ir tarė: “Viešpats su tavimi, galingas karžygy!” 
\par 13 Gedeonas jam atsakė: “Viešpatie, jei Viešpats yra su mumis, kodėl mums taip atsitiko? Kur yra visi Jo stebuklai, apie kuriuos pasakojo mūsų tėvai, sakydami: ‘Tikrai, Viešpats išvedė mus iš Egipto’? Dabar Viešpats mus atstūmė ir atidavė į midjaniečių rankas”. 
\par 14 Viešpats, pažiūrėjęs į Gedeoną, tarė: “Eik šioje savo jėgoje ir išgelbėk Izraelį iš midjaniečių! Aš tave siunčiu”. 
\par 15 Gedeonas klausė: “Viešpatie, kaip aš išgelbėsiu Izraelį? Mano šeima yra skurdžiausia Manaso giminėje, o aš pats mažiausias savo tėvo namuose”. 
\par 16 Viešpats jam atsakė: “Aš būsiu su tavimi, ir tu nugalėsi midjaniečius kaip vieną vyrą”. 
\par 17 Gedeonas atsakė: “Jei radau malonę Tavo akyse, tai parodyk ženklą, kad Tu kalbi su manimi. 
\par 18 Prašau, nepasitrauk iš čia, kol sugrįšiu ir atnešiu Tau dovaną”. Jis atsakė: “Aš palauksiu, iki sugrįši”. 
\par 19 Gedeonas nuskubėjo į namus. Paruošęs ožiuką ir neraugintos duonos iš vienos efos miltų, mėsą sudėjo į pintinę, sriubą supylė į puodą ir viską atnešė po ąžuolu, ir davė Jam. 
\par 20 Dievo angelas jam tarė: “Paimk mėsą bei neraugintą duoną ir padėk čia ant uolos, o sriubą išpilk!” Jis taip ir padarė. 
\par 21 Viešpaties angelas ištiesė lazdą, kurią laikė rankoje, ir jos galu palietė mėsą. Iš uolos pakilo ugnis ir sudegino mėsą bei duoną. Tada Viešpaties angelas pradingo jam iš akių. 
\par 22 Kada Gedeonas suprato, kad čia buvo Viešpaties angelas, jis tarė: “Viešpatie Dieve! Vargas man! Aš mačiau Viešpaties angelą veidas į veidą”. 
\par 23 Viešpats jam atsakė: “Ramybė tau! Nebijok, nemirsi!” 
\par 24 Gedeonas ten pastatė aukurą Viešpačiui ir pavadino jį “Viešpats yra ramybė”. Jis tebestovi iki šios dienos abiezerių Ofroje. 
\par 25 Tą pačią naktį Viešpats jam liepė: “Imk jauną jautį, priklausantį tavo tėvui, ir antrą septynmetį jautį, nugriauk Baalio aukurą, kurį pastatė tavo tėvas, ir iškirsk prie jo esančią giraitę. 
\par 26 Pastatyk aukurą Viešpačiui, savo Dievui, ant šitos uolos nurodytoje vietoje. Tuomet aukok antrą jautį kaip deginamąją auką ant giraitės, kurią iškirsi, malkų”. 
\par 27 Gedeonas su dešimčia savo tarnų padarė, kaip Viešpats jam įsakė. Tačiau, bijodamas savo tėvo namiškių ir miesto žmonių, jis padarė tai ne dienos metu, bet naktį. 
\par 28 Kai miesto žmonės, atsikėlę anksti rytą, pamatė nugriautą Baalio aukurą ir prie jo buvusią giraitę iškirstą ir ant pastatyto aukuro paaukotą antrą jautį, 
\par 29 jie kalbėjo vienas kitam: “Kas tai padarė?” Ieškodami ir klausinėdami jie sužinojo, kad tai padarė Jehoašo sūnus Gedeonas. 
\par 30 Miesto vyrai sakė Jehoašui: “Išvesk savo sūnų­jis turi mirti, nes sugriovė Baalio aukurą ir iškirto prie jo buvusią giraitę”. 
\par 31 Jehoašas tarė visiems prie jo stovintiems: “Ar jūs norite ginti Baalį ir manote jį išgelbėti? Kas gins jį, tas temiršta, nesulaukęs ryto. Jei jis yra dievas, tegul pats apgina save, nes sugriautas jo aukuras”. 
\par 32 Tą dieną Gedeonas buvo pramintas Jerubaalu, nes jo tėvas sakė: “Tegul Baalis apgina save, nes jis sugriovė jo aukurą”. 
\par 33 Midjaniečiai, amalekiečiai ir rytų gyventojai susirinkę perėjo per Jordaną ir pasistatė stovyklas Jezreelio slėnyje. 
\par 34 Viešpaties Dvasia nužengė ant Gedeono. Jis trimitu sušaukė Abiezerio palikuonis, kurie susirinko pas jį. 
\par 35 Jis išsiuntė pasiuntinius į visą Manasą, ir jie taip pat susirinko pas jį. Jis išsiuntė pasiuntinius ir pas Ašerą, Zabuloną bei Neftalį. Jie atėjo jo pasitikti. 
\par 36 Gedeonas tarė Dievui: “Jei išgelbėsi Izraelį mano ranka, kaip sakei, 
\par 37 štai aš patiesiu vilną klojime; jei rasa bus tik ant vilnos, o aplink žemė bus sausa, tai žinosiu, kad išgelbėsi Izraelį mano ranka, kaip kalbėjai”. 
\par 38 Taip ir įvyko. Rytojaus dieną atsikėlęs anksti rytą jis išgręžė iš vilnos pilną dubenį vandens. 
\par 39 Gedeonas tarė Dievui: “Teneužsidega Tavo rūstybė prieš mane, jei aš dar kartą prabilsiu. Aš prašau dar kartą leisti man pabandyti su vilna. Šį kartą tegul lieka sausa tik vilna, o ant žemės tebūna rasa”. 
\par 40 Dievas taip padarė tą naktį: sausa buvo tik vilna, o aplinkui ant žemės buvo rasa.



\chapter{7}


\par 1 Jerubaalas, kitaip Gedeonas, anksti atsikėlęs, su savo žmonėmis pasistatė stovyklą prie En Harodo versmės, o Midjano stovykla buvo į šiaurę nuo jo, prie Morės kalvos, slėnyje. 
\par 2 Viešpats tarė Gedeonui: “Turi per daug vyrų, kad atiduočiau Midjaną į tavo rankas, kad Izraelis negalėtų didžiuotis prieš mane, sakydamas: ‘Aš išsilaisvinau savo jėgomis’. 
\par 3 Paskelbk žmonėms: ‘Kas bijo ir nedrąsus, tegrįžta namo!’ ” Pasitraukė dvidešimt du tūkstančiai, o dešimt tūkstančių vyrų pasiliko. 
\par 4 Viešpats tarė Gedeonui: “Dar per daug žmonių. Nuvesk juos prie vandens. Ten Aš juos tau atrinksiu. Apie kuriuos sakysiu: ‘Tas eis su tavimi’, tuos pasiimk, o apie kuriuos sakysiu: ‘Tas neis su tavimi’, tie tegrįžta!” 
\par 5 Gedeonas nuvedė žmones prie vandens. Viešpats tarė Gedeonui: “Atskirk kiekvieną, kuris laks vandenį kaip šuo ir kuris klaupsis gerti ant kelių”. 
\par 6 Skaičius lakusių iš rankos buvo trys šimtai vyrų, o likusieji gėrė atsiklaupę ant kelių. 
\par 7 Tada Viešpats tarė Gedeonui: “Per tuos tris šimtus vyrų, kurie gėrė iš rankos, Aš jus išgelbėsiu ir atiduosiu midjaniečius į tavo rankas; visi kiti žmonės tegrįžta į savo namus”. 
\par 8 Paėmęs maisto atsargas ir trimitus iš žmonių, pasiuntė juos į savo palapines, o tuos tris šimtus vyrų paliko stovykloje. Midjaniečių stovykla buvo žemiau, slėnyje. 
\par 9 Tą pačią naktį Viešpats tarė Gedeonui: “Pulk midjaniečių stovyklą, Aš juos atidaviau į tavo rankas. 
\par 10 O jei bijai eiti, tai su savo tarnu Pura nusileisk į stovyklą 
\par 11 ir pasiklausyk, ką jie kalba. Tuomet sutvirtės tavo rankos kovoti prieš juos”. Jis nusileido su savo tarnu Pura prie stovyklos sargybų. 
\par 12 Midjaniečių, amalekiečių ir rytų gyventojų slėnyje buvo tiek daug kaip skėrių, o jų kupranugarių buvo be skaičiaus, kaip smilčių ant jūros kranto. 
\par 13 Gedeonas nuėjęs išgirdo vieną pasakojant sapną savo draugui: “Sapnavau miežinės duonos kepalą, riedantį į midjaniečių stovyklą. Jis pasiekė palapinę ir taip į ją atsitrenkė, kad ji apvirto ir subyrėjo”. 
\par 14 Jo draugas atsakė: “Tai ne kas kita kaip izraelito Gedeono, Jehoašo sūnaus, kardas. Dievas atidavė į jo rankas midjaniečius ir visą stovyklą”. 
\par 15 Gedeonas, išgirdęs sapną ir jo aiškinimą, pagarbino ir, sugrįžęs į Izraelio stovyklą, tarė: “Kilkite, Viešpats atidavė midjaniečių stovyklą į jūsų rankas”. 
\par 16 Gedeonas padalino tris šimtus vyrų į tris būrius, padavė jiems į rankas trimitus ir ąsočius su deglais viduje 
\par 17 ir įsakė: “Žiūrėkite į mane ir darykite, ką aš darysiu. 
\par 18 Kai mes visi nueisime prie midjaniečių stovyklos, aš trimituosiu ir jūs trimituokite iš visų pusių aplink stovyklą, šaukdami: ‘Viešpaties ir Gedeono kardas!’ ” 
\par 19 Gedeonas ir šimtas su juo buvusiųjų vyrų nuėjo prie stovyklos. Tik pasikeitus vidurnakčio sargybai, jie visi trimitavo ir sudaužė ąsočius, kuriuos laikė rankose. 
\par 20 Visi trys būriai trimitavo, sudaužė ąsočius ir, laikydami trimitus dešinėje rankoje bei deglus kairėje, šaukė: “Viešpaties ir Gedeono kardas!” 
\par 21 Kiekvienas iš jų stovėjo savo vietoje aplinkui stovyklą. Stovykloje kilo panika, visa stovykla lakstė, šaukė ir ėmė bėgti. 
\par 22 Trys šimtai trimitavo, ir Viešpats atgręžė midjaniečių kardus vienas prieš kitą visoje stovykloje. Jie bėgo ligi Bet Šitos, Cereros link, ligi Abel Meholos sienos, prie Tabato. 
\par 23 Susirinko izraelitai iš Neftalio, Ašero ir Manaso ir vijosi midjaniečius. 
\par 24 Gedeonas siuntė pasiuntinius į Efraimo kalnyną, sakydamas: “Nusileiskite prieš midjaniečius ir atkirskite jiems kelią į Bet Barą ir prie Jordano!” Susirinko visi efraimai ir atkirto praėjimą į Bet Barą ir prie Jordano. 
\par 25 Efraimai pagavo du midjaniečių kunigaikščius­Orebą ir Zeebą. Orebą jie nužudė ant Orebo uolos, o Zeebą­prie Zeebo vyno spaustuvo ir persekiojo midjaniečius. Orebo ir Zeebo galvas nunešė Gedeonui į kitą Jordano pusę.



\chapter{8}


\par 1 Efraimai sakė Gedeonui: “Kodėl mums taip padarei? Kodėl, nepasikvietęs mūsų, išėjai prieš midjaniečius?” Ir jie smarkiai ginčijosi su juo. 
\par 2 Jis atsakė jiems: “Ar aš ką nuveikiau palyginti su jumis? Argi efraimitų vynuogių likutis nėra didesnis negu Abiezerio visas derlius? 
\par 3 Į jūsų rankas Dievas atidavė Midjano kunigaikščius Orebą ir Zeebą. Kuo aš galiu lygintis su jumis?” Jam taip kalbant, atlyžo jų pyktis. 
\par 4 Gedeonas, atėjęs prie Jordano su trimis šimtais vyrų, persikėlė per jį. Jie pavargo, besivydami priešą. 
\par 5 Jis prašė Sukoto žmonių duonos savo kariams, kurie buvo išvargę, vydami Zebachą ir Calmuną, Midjano karalius. 
\par 6 Sukoto vyresnieji paklausė: “Ar Zebachas ir Calmuna jau yra tavo rankose, kad mes duotume duonos tavo kariuomenei?” 
\par 7 Gedeonas atsakė: “Kai Viešpats atiduos Zebachą ir Calmuną į mano rankas, aš plaksiu jus dykumų erškėčių dygliais”. 
\par 8 Jis traukė iš ten į Penuelį ir ten kalbėjo tą patį. Penuelio žmonių atsakymas buvo toks pat kaip Sukoto. 
\par 9 Tada Gedeonas jiems tarė: “Po pergalės grįždamas aš nugriausiu šitą bokštą”. 
\par 10 Zebachas ir Calmuna buvo Karkore su savo kariuomene, apie penkiolika tūkstančių vyrų. Tiek buvo likę iš visos rytų kariuomenės. Kritusiųjų buvo apie šimtas dvidešimt tūkstančių ginkluotų karių. 
\par 11 Gedeonas žygiavo keliu į rytus nuo Nobacho ir Jogbohos, kur žmonės gyveno palapinėse, ir nelauktai užpuolė midjaniečių stovyklą. 
\par 12 Kai Zebachas ir Calmuna pabėgo, Gedeonas vijosi juos ir pagavo abu Midjano karalius, o jų kariuomenėje sukėlė paniką. 
\par 13 Jehoašo sūnus Gedeonas grįžo iš mūšio dar saulei nepatekėjus. 
\par 14 Jis sugavo jaunuolį iš Sukoto ir jį išklausinėjo. Tas jam surašė Sukoto miesto kunigaikščius ir vyresniuosius, septyniasdešimt septynis vyrus. 
\par 15 Atėjęs į Sokotą, Gedeonas tarė: “Štai Zebahas ir Zalmunas, dėl kurių jūs išjuokėte mane, sakydami: ‘Ar Zebahas ir Zalmunas jau tavo rankose, kad prašai duonos savo pavargusiems žmonėms?’ ” 
\par 16 Jis Sukoto miesto vyresniuosius nuplakė dykumos erškėčiais ir taip pamokė juos. 
\par 17 Penuelio bokštą jis nugriovė ir miesto vyrus išžudė. 
\par 18 Po to jis klausė Zebachą ir Calmuną: “Kaip atrodė tie vyrai, kuriuos užmušėte Tabore?” Jie atsakė: “Taip, kaip tu. Jie atrodė kaip karaliaus sūnūs”. 
\par 19 Jis tarė: “Tai buvo mano broliai, mano motinos sūnūs. Kaip gyvas Viešpats, jei būtumėte palikę juos gyvus, nenužudyčiau jūsų”. 
\par 20 Tuomet jis sakė savo pirmagimiui Jeteriui: “Nužudyk juos!” Bet berniukas neištraukė kardo, nes bijojo, kadangi buvo dar vaikas. 
\par 21 Zebachas ir Calmuna tarė: “Tu pats nužudyk mus! Nes koks vyras, tokia ir jo jėga”. Gedeonas nužudė Zebachą ir Calmuną ir pasiėmė jų kupranugarių kaklų papuošalus. 
\par 22 Izraelitai prašė Gedeono: “Valdyk mus tu, tavo sūnus ir sūnaus sūnus, nes tu mus išgelbėjai iš Midjano rankų”. 
\par 23 Gedeonas atsakė jiems: “Nei aš, nei mano sūnus nevaldys jūsų. Viešpats bus jūsų valdovas. 
\par 24 Aš tik noriu paprašyti jūsų atiduoti man iš savo grobio auskarus”. Mat izmaelitai nešiodavo auksinius auskarus. 
\par 25 Jie atsakė: “Mielai atiduosime”. Patiesę apsiaustą, jie sumetė ant jo auskarus iš savo grobio. 
\par 26 Auksiniai auskarai svėrė tūkstantį septynis šimtus šekelių, be papuošalų, grandinėlių ir purpurinių drabužių, kuriais vilkėjo Midjano karaliai, bei grandinių, kurios buvo ant jų kupranugarių kaklų. 
\par 27 Iš to aukso Gedeonas padarė efodą ir jį pastatė savo mieste Ofroje. Visi izraelitai eidavo ten ir garbino jį, todėl jis tapo spąstais Gedeonui ir jo namams. 
\par 28 Midjaniečiai buvo nugalėti izraelitų ir daugiau nebekėlė savo galvų. Gedeono dienomis krašte buvo ramu keturiasdešimt metų. 
\par 29 Jehoašo sūnus Jerubaalas, sugrįžęs gyveno savo namuose. 
\par 30 Gedeonas turėjo septyniasdešimt sūnų, kurie gimė iš jo, nes turėjo daug žmonų. 
\par 31 Jo sugulovė, gyvenusi Sicheme, taip pat pagimdė jam sūnų, kurį jis pavadino Abimelechu. 
\par 32 Jehoašo sūnus Gedeonas mirė sulaukęs senatvės ir buvo palaidotas savo tėvo Jehoašo kape, abiezerių Ofroje. 
\par 33 Gedeonui mirus, izraelitai vėl nuėjo paleistuvauti paskui Baalį ir savo dievu padarė Baal Beritą. 
\par 34 Izraelitai neprisiminė Viešpaties, savo Dievo, kuris juos išgelbėjo iš visų aplinkui juos esančių priešų. 
\par 35 Jie neparodė malonės Jerubaalio, kitaip Gedeono, giminei, nepaisydami to, kad jis Izraeliui padarė daug gero.



\chapter{9}

\par 1 Jerubaalio sūnus Abimelechas nuėjo į Sichemą pas savo motinos brolius ir kalbėjo visai motinos giminei: 
\par 2 “Paklauskite Sichemo vyrų, ar jie norėtų, kad jiems karaliautų septyniasdešimt vyrų, visi Jerubaalio sūnūs, ar vienas? Atsiminkite, jog aš esu jūsų kūnas ir kaulas”. 
\par 3 Jo motinos broliai viską papasakojo Sichemo vyrams. Jie pritarė Abimelechui, galvodami: “Jis yra mūsų brolis!” 
\par 4 Jie davė jam iš Baal Berito šventyklos septyniasdešimt sidabrinių. Už juos Abimelechas pasisamdė niekšų ir valkatų, kurie sekė jį. 
\par 5 Atėjęs į savo tėvo namus Ofroje, nužudė savo brolius, Jerubaalio sūnus, septyniasdešimt vyrų, ant vieno akmens. Gyvas liko tik Jerubaalio jauniausiasis sūnus Jotamas, nes jis buvo pasislėpęs. 
\par 6 Po to Sichemo vyrai bei Bet Milojo gyventojai, susirinkę po ąžuolu, prie akmens stulpo, stovinčio Sicheme, paskelbė Abimelechą karaliumi. 
\par 7 Tai sužinojęs, Jotamas užlipo ant Garizimo kalno ir garsiai šaukdamas tarė: “Sichemo vyrai, klausykite manęs, kad ir Dievas jūsų klausytų. 
\par 8 Kartą medžiai susirinko patepti karalių. Jie tarė alyvmedžiui: ‘Karaliauk mums’. 
\par 9 Alyvmedis jiems atsakė: ‘Argi galiu atsisakyti savo alyvos, kuria patepami dievai ir žmonės, ir nuėjęs valdyti medžius?’ 
\par 10 Po to medžiai kreipėsi į figmedį: ‘Ateik ir karaliauk mums’. 
\par 11 Figmedis jiems atsakė: ‘Argi galiu atsisakyti savo saldumo bei puikių vaisių ir nuėjęs valdyti medžius?’ 
\par 12 Po to medžiai kreipėsi į vynmedį: ‘Ateik ir karaliauk mums’. 
\par 13 Vynmedis jiems atsakė: ‘Argi galiu atsisakyti vyno, kuris linksmina dievus ir žmones, ir nuėjęs valdyti medžius?’ 
\par 14 Pagaliau medžiai kreipėsi į erškėtį: ‘Ateik ir karaliauk mums’. 
\par 15 Erškėtis atsakė medžiams: ‘Jei tikrai jūs norite mane patepti karaliumi, ateikite ir ilsėkitės mano pavėsyje, o jei ne, tegul išeina ugnis iš erškėčio ir sudegina Libano kedrus!’ 
\par 16 Jei pasielgėte teisingai ir gerai, patepdami Abimelechą karaliumi, ir Jerubaaiui bei jo giminei pagal nuopelnus atlyginote 
\par 17 (nes mano tėvas kovojo už jus, statydamas pavojun savo gyvybę ir gelbėdamas jus iš midjaniečių rankos, 
\par 18 o jūs šiandien sukilote prieš mano tėvo giminę, nužudėte visus septyniasdešimt jo sūnų ant vieno akmens ir patepėte Sichemo karaliumi jo tarnaitės sūnų Abimelechą todėl, kad jis jūsų brolis), 
\par 19 jei šiandien teisingai ir gerai pasielgėte su Jerubaaliu ir jo gimine, tai džiaukitės Abimelechu, o jis tegul džiaugiasi jumis. 
\par 20 Bet jei ne, tegul išeina ugnis iš Abimelecho ir sudegina Sichemo miesto ir Milo gyventojus. Tegul išeina ugnis iš Sichemo ir Bet Milojo gyventojų ir sudegina Abimelechą!” 
\par 21 Jotamas pabėgo į Beerą ir ten apsigyveno, nes bijojo savo brolio Abimelecho. 
\par 22 Abimelechas valdė Izraelį trejus metus. 
\par 23 Po to Dievas sukėlė nesantaiką tarp Abimelecho ir Sichemo gyventojų (ir Sichemo gyventojai pradėjo klastingai elgtis su Abimelechu), 
\par 24 kad atkeršytų Abimelechui už septyniasdešimt Jerubaalio sūnų ir jų kraujas kristų ant to, kuris juos nužudė, ir ant Sichemo žmonių, kurie jam padėjo nužudyti brolius. 
\par 25 Sichemo gyventojai pastatė tykoti jo kalnų viršūnėse vyrus, kurie apiplėšdavo visus, kas eidavo tuo keliu. Apie tai buvo pranešta Abimelechui. 
\par 26 Į Sichemą atvyko Ebedo sūnus Gaalas su savo broliais ir apsigyveno. Sichemo gyventojai pasitikėjo juo. 
\par 27 Jie, prisiskynę vynuogių savo vynuogynuose, jas išsispaudė ir suruošė puotą. Savo dievo namuose jie valgė, gėrė ir keikė Abimelechą. 
\par 28 Ebedo sūnus Gaalas sakė: “Kas yra Abimelechas ir kas yra Sichemas, kad jam tarnautume? Argi jis ne Jerubaalio sūnus ir argi jo prievaizdas ne Zebulas? Tarnaukite Sichemo tėvo Hamoro vyrams. Kodėl mes turime tarnauti jam? 
\par 29 Jei aš būčiau tų žmonių valdovas, pašalinčiau Abimelechą, sakydamas: ‘Surink savo kariuomenę ir išeik!’ ” 
\par 30 Miesto viršininkas Zebulas girdėjo Ebedo sūnaus Gaalo žodžius ir labai supyko. 
\par 31 Jis slaptai siuntė pas Abimelechą pasiuntinius, pranešdamas: “Ebedo sūnus Galaas su savo broliais atvyko į Sichemą ir kursto miestą prieš tave. 
\par 32 Tu ir tavo vyrai pasislėpkite laukuose. 
\par 33 Rytą, saulei tekant, pulkite miestą! Gaalas su savo šalininkais išeis prieš tave. Pasielk su jais, kaip galėsi”. 
\par 34 Abimelechas ir visi jo vyrai naktį pasislėpė Sichemo laukuose, pasidalinę į keturias grupes. 
\par 35 Ebedo sūnus Gaalas išėjęs atsistojo miesto vartuose. Tada Abimelechas ir jo vyrai pasiruošė puolimui. 
\par 36 Gaalas, pamatęs žmones, tarė Zebului: “Žmonės leidžiasi nuo kalnų viršūnių”. Bet Zebulas jam atsakė: “Kalnų šešėlius tu laikai žmonėmis”. 
\par 37 Tačiau Gaalas vėl pakartojo: “Žmonės leidžiasi nuo aukštumos, o kitas būrys ateina nuo Menoimo pusės”. 
\par 38 Tada Zebulas jam tarė: “Kur tavo lūpos, kurios sakė: ‘Kas mums Abimelechas, kad jam tarnautume’. Tai vyrai, kuriuos tu niekini. Eik dabar ir kariauk su jais”. 
\par 39 Gaalas ėjo Sichemo vyrų priekyje ir kovojo su Abimelechu. 
\par 40 Abimelechas vijosi jį, ir tas bėgo nuo jo. Daug vyrų krito iki miesto vartų. 
\par 41 Abimelechas sustojo Arumoje, o Zebulas išvarė Gaalą ir jo brolius iš Sichemo. 
\par 42 Kitą rytą žmonės išėjo iš Sichemo į laukus. Apie tai pranešė Abimelechui. 
\par 43 Jis savo vyrus paskirstė į tris būrius, kurie, pasislėpę laukuose, laukė. Žmonėms išėjus iš miesto, jie puolė ir nugalėjo. 
\par 44 Abimelechas su savo būriu atskubėjo ir atsistojo miesto vartuose, kiti du būriai puolė esančius laukuose ir juos išžudė. 
\par 45 Abimelechas, kovojęs visą dieną, užėmė miestą, jame buvusius žmones išžudė, miestą sugriovė ir apibarstė druska. 
\par 46 Migdal Sichemo pilies gyventojai, tai išgirdę, suėjo į El Berito šventyklos tvirtovę. 
\par 47 Abimelechui buvo pranešta, kad visi Migdal Sichemo pilies gyventojai susirinko į vieną vietą. 
\par 48 Ir Abimelechas su savo vyrais užkopė ant Calmono kalno. Jis, paėmęs kirvį, nusikirto medžio šaką, užsidėjo ją ant peties ir įsakė žmonėms: “Ką aš darau, darykite ir jūs”. 
\par 49 Visi vyrai nusikirto po šaką ir ėjo paskui Abimelechą. Atėję sukrovė šakas aplink tvirtovę ir padegė ją. Taip mirė visi Migdal Sichemo pilies žmonės, apie tūkstantį vyrų ir moterų. 
\par 50 Po to Abimelechas nužygiavo į Tebecą, apgulė ir paėmė jį. 
\par 51 Miesto viduryje buvo stiprus bokštas, ir į jį subėgo visi miesto gyventojai. Jie, užrakinę bokšto įėjimą, užlipo ant jo stogo. 
\par 52 Abimelechas priėjo prie bokšto, norėdamas jį padegti. 
\par 53 Viena moteris numetė ant Abimelecho galvos girnų akmens gabalą ir sulaužė jo kaukolę. 
\par 54 Jis tuojau pasišaukė jaunuolį, savo ginklanešį, ir jam tarė: “Išsitrauk kardą ir nužudyk mane, kad apie mane nesakytų: ‘Moteris jį užmušė!’ ” Jo ginklanešys jį perdūrė, ir jis mirė. 
\par 55 Izraelitai pamatę, kad Abimelechas miręs, kiekvienas sugrįžo į savo namus. 
\par 56 Taip Dievas atlygino Abimelechui už jo nusikaltimą tėvui, kai jis nužudė septyniasdešimt savo brolių. 
\par 57 Taip pat ir Sichemo vyrams Dievas atlygino, ir taip išsipildė Jerubaalio sūnaus Jotamo prakeikimas.



\chapter{10}


\par 1 Abimelechui mirus, iškilo Dodojo sūnaus Pūvos sūnus Tola iš Isacharo giminės, kad išgelbėtų Izraelį. Jis gyveno Šamyre, Efraimo kalnuose, 
\par 2 ir teisė Izraelį dvidešimt trejus metus. Po to jis mirė ir buvo palaidotas Šamyre. 
\par 3 Po to iškilo gileadietis Jayras, kuris teisė Izraelį dvidešimt dvejus metus. 
\par 4 Jis turėjo trisdešimt sūnų, kurie jodinėjo ant trisdešimties asilų ir valdė trisdešimt miestų. Tuos miestus iki šios dienos Gileado krašte tebevadina Havot Jayru. 
\par 5 Jayras mirė ir buvo palaidotas Kamone. 
\par 6 Izraelitai darė pikta Viešpaties akivaizdoje, garbindami Baalį, Astartę ir Sirijos, Sidono, Moabo, amonitų ir filistinų dievus. Jie paliko Viešpatį ir netarnavo Jam. 
\par 7 Viešpaties rūstybė užsidegė prieš Izraelį. Jis atidavė juos į filistinų ir amonitų rankas. 
\par 8 Jie aštuoniolika metų vargino ir spaudė visus izraelitus, kurie gyveno Jordano rytuose, Gileade, amoritų šalyje. 
\par 9 Amonitai persikėlė per Jordaną į vakarus ir kariavo su Judo, Benjamino ir Efraimo giminėmis. Izraelis, patekęs į didelį vargą, 
\par 10 šaukėsi Viešpaties: “Mes Tau nusidėjome, palikdami savo Dievą ir tarnaudami Baaliui”. 
\par 11 Viešpats tarė: “Argi nespaudė jūsų egiptiečiai, amoritai, amonitai bei filistinai, 
\par 12 taip pat sidoniečiai, amalekiečiai ir midjaniečiai? Kai jūs šaukėtės manęs, Aš jus išgelbėjau iš jų rankų. 
\par 13 Tačiau jūs palikote mane ir tarnavote svetimiems dievams. Todėl daugiau nebegelbėsiu jūsų. 
\par 14 Eikite ir šaukitės tų dievų, kuriuos pasirinkote. Tegul jie išlaisvina jus iš vargų”. 
\par 15 Izraelitai atsakė Viešpačiui: “Mes nusidėjome. Daryk su mumis, ką nori, tik išgelbėk mus šiandien”. 
\par 16 Jie pašalino svetimus dievus iš savo tarpo ir tarnavo Viešpačiui. Tada Dievas pasigailėjo jų. 
\par 17 Amonitai susirinko ir pasistatė stovyklas Gileade, o izraelitai­ Micpoje. 
\par 18 Žmonės ir Gileado kunigaikščiai kalbėjosi: “Kas pradės kariauti su amonitais, tas taps visų Gileado gyventojų valdovu”.



\chapter{11}

\par 1 Gileadietis Jeftė buvo galingas karžygys. Tačiau jis buvo paleistuvės sūnus. Gileadas buvo Jeftės tėvas. 
\par 2 Jis turėjo ir daugiau sūnų nuo savo žmonos, kurie paaugę išvarė Jeftę, sakydami: “Tu neturi dalies mūsų tėvo namuose, nes esi kitos moters sūnus”. 
\par 3 Jeftė, pabėgęs nuo savo brolių, apsigyveno Tobo krašte. Pas jį rinkdavosi valkatos ir sekė paskui jį. 
\par 4 Kuriam laikui praėjus, amonitai vėl kariavo su Izraeliu. 
\par 5 Gileado vyresnieji pasiuntė pas Jeftę į Tobo kraštą, 
\par 6 sakydami: “Grįžk ir vadovauk mums, kad galėtume kariauti su amonitais”. 
\par 7 Jeftė atsakė Gileado vyresniesiems: “Jūs manęs nekentėte ir išvarėte iš mano tėvo namų. Dabar, kai esate spaudžiami, atėjote pas mane”. 
\par 8 Gileado vyresnieji atsakė Jeftei: “Todėl ir kreipiamės į tave, kad grįžtum ir kariautum su amonitais, ir vadovautum visiems Gileado gyventojams”. 
\par 9 Jeftė paklausė Gileado vyresniuosius: “Jei sugrįžęs kariausiu su amonitais ir Viešpats atiduos juos į mano rankas, ar aš tapsiu jūsų valdovu?” 
\par 10 Gileado vyresnieji atsakė jam: “Viešpats tebūna liudytoju, jei nepadarysime pagal tavo žodžius”. 
\par 11 Jeftė nuėjo su Gileado vyresniaisiais, ir žmonės paskelbė jį savo vadu. Jeftė kalbėjo visa tai Viešpaties akivaizdoje Micpoje. 
\par 12 Jeftė siuntė pas amonitų karalių pasiuntinius, klausdamas: “Ko nori iš manęs? Kodėl atėjai kariauti prieš mano kraštą?” 
\par 13 Amonitų karalius atsakė Jeftės pasiuntiniams: “Dėl to, kad Izraelis, atėjęs iš Egipto, užėmė mano kraštą nuo Arnono iki Jaboko ir Jordano upių. Grąžink man tai geruoju”. 
\par 14 Jeftė vėl siuntė pasiuntinius pas Amono karalių, 
\par 15 sakydamas: “Izraelitai neužėmė nei Moabo, nei amonitų šalies. 
\par 16 Jie, išėję iš Egipto, ėjo per dykumą ligi Raudonosios jūros ir atvyko į Kadešą. 
\par 17 Iš čia Izraelis siuntė pasiuntinius pas Edomo karalių, prašydamas leisti jiems pereiti per jo žemę. Bet Edomo karalius nesutiko. Jie kreipėsi taip pat ir į Moabo karalių, bet ir tas nesutiko jų praleisti. Taip Izraelis pasiliko Kadeše. 
\par 18 Po to jie dykuma apėjo Edomo bei Moabo žemes ir, atėję į rytus nuo Moabo, prie Arnono upės, pasistatė stovyklas. Jie nėjo į Moabo žemę, nes Arnonas yra Moabo krašto siena. 
\par 19 Izraelis siuntė pasiuntinius pas amoritų karalių Sihoną į Hešboną, prašydamas leisti jiems pereiti per jo kraštą. 
\par 20 Sihonas nepasitikėjo Izraeliu ir nepraleido jo. Jis surinko visą savo kariuomenę, pasistatė stovyklą Jahace ir pradėjo kovą su Izraeliu. 
\par 21 Viešpats, Izraelio Dievas, atidavė Sihoną ir visus jo žmones į Izraelio rankas, ir šie juos sumušė. Taip Izraelis užėmė visą amoritų žemę 
\par 22 nuo Arnono iki Jaboko ir nuo dykumos iki Jordano. 
\par 23 Viešpats, Izraelio Dievas, išvarė amoritus, kad tą kraštą atiduotų Izraeliui, o tu nori jame apsigyventi. 
\par 24 Argi tu negyveni ten, kur tavo dievas Kemošas tau duoda? Mes gyvename ten, kur Viešpats, mūsų Dievas, mums duoda. 
\par 25 Ar tu geresnis už Ciporo sūnų Balaką, Moabo karalių? Ar jis kada nors ginčijosi ar kovojo su Izraeliu? 
\par 26 Izraelis gyvena Hešbone ir jo apylinkėse, Aroeryje ir jo apylinkėse bei miestuose palei Arnoną jau tris šimtus metų. Kodėl per tą laiką jų neišlaisvinote? 
\par 27 Aš nekaltas prieš tave, bet tu, pradėdamas karą, piktai elgiesi su manimi. Viešpats Teisėjas tegul šiandien daro teismą tarp izraelitų ir amonitų”. 
\par 28 Amonitų karalius nekreipė dėmesio į Jeftės žodžius, kuriuos jis jam kalbėjo per pasiuntinius. 
\par 29 Tada Viešpaties Dvasia nužengė ant Jeftės ir jis perėjo per Gileadą, Manasą, toliau pro Micpą Gileade ir iš Mispos Gileado traukė prieš amonitus. 
\par 30 Jeftė padarė Viešpačiui įžadą: “Jei atiduosi amonitus į mano rankas, 
\par 31 kai aš ramybėje grįšiu nuo amonitų, pirmą, išėjusį iš mano namų manęs pasitikti, paaukosiu Viešpačiui kaip deginamąją auką”. 
\par 32 Jeftė traukė prieš amonitus, ir Viešpats atidavė juos į jo rankas. 
\par 33 Nuo Aroerio iki Minito užėmė dvidešimt miestų ir pasiekė vynuogynų slėnį be gailesčio juos žudydamas. Taip amonitai buvo pažeminti prieš izraelitus. 
\par 34 Jeftė grįžo į Micpą, į savo namus, ir štai jį pasitiko jo duktė, su tamburinu šokdama. Ji buvo jo vienintelis vaikas; jis neturėjo daugiau sūnų ar dukterų. 
\par 35 Pamatęs ją, jis perplėšė savo rūbus ir tarė: “Ak, mano dukra! Tu man suteikei daug skausmo ir esi tarp tų, kurie mane vargina. Aš daviau įžadą Viešpačiui ir nebegaliu jo atšaukti”. 
\par 36 Ji atsakė: “Mano tėve, ką pažadėjai Viešpačiui, tą daryk su manimi. Išpildyk savo pažadą. Juk Viešpats padėjo atkeršyti tavo priešams amonitams”. 
\par 37 Ji prašė tėvo duoti jai du mėnesius laiko nueiti į kalnus ir apraudoti savo mergystę kartu su draugėmis. 
\par 38 Jis sutiko. Ji su savo draugėmis nuėjo į kalnus ir apraudojo savo mergystę. 
\par 39 Po dviejų mėnesių ji sugrįžo pas savo tėvą, kuris įvykdė savo pažadą Viešpačiui. Ji nepažino vyro. Taip atsirado paprotys Izraelyje, 
\par 40 kad kas metai Izraelio dukterys išeina keturias dienas apraudoti gileadiečio Jeftės dukters.



\chapter{12}


\par 1 Efraimai susirinko, atėjo į Cafoną ir tarė Jeftei: “Kodėl, eidamas į karą prieš amonitus, nepakvietei mūsų? Mes sudeginsime tavo namus ir tave”. 
\par 2 Jeftė jiems atsakė: “Aš ir mano žmonės smarkiai susikivirčijome su amonitais. Aš jus šaukiau, bet jūs nepadėjote. 
\par 3 Matydamas, kad jūs nepadedate, aš stačiau savo gyvybę į pavojų ir ėjau prieš amonitus. Viešpats atidavė juos į mano rankas. Kodėl šiandien atėjote kariauti prieš mane?” 
\par 4 Jeftė, surinkęs visus Gileado vyrus, kariavo su Efraimu ir jį nugalėjo. Mat efraimai sakė gileadiečiams: “Jūs esate pabėgėliai iš Efraimo ir gyvenate tarp Efraimo ir Manaso”. 
\par 5 Gileadiečiai užėmė Jordano brastas, vedančias į Efraimą. Kai efraimas bėglys prašydavo praleisti, gileadiečiai jį klausdavo: “Ar tu esi efraimas?” Jam atsakius: “Ne!”, 
\par 6 jie liepdavo ištarti “šibolet!” Jis atsakydavo “sibolet”, nes negalėdavo teisingai ištarti. Tada jie nužudydavo jį prie Jordano brastų. Taip tuo metu žuvo keturiasdešimt du tūkstančiai efraimų. 
\par 7 Jeftė teisė Izraelį šešerius metus. Po to gileadietis Jeftė mirė ir buvo palaidotas viename Gileado mieste. 
\par 8 Po jo Izraelį teisė Ibcanas iš Betliejaus. 
\par 9 Jis turėjo trisdešimt sūnų ir trisdešimt dukterų. Savo dukteris jis išleido svetur, o jo sūnūs parvedė trisdešimt mergaičių iš kitų kraštų. Jis teisė Izraelį septynerius metus. 
\par 10 Po to Ibcanas mirė ir buvo palaidotas Betliejuje. 
\par 11 Po jo dešimt metų Izraelį teisė Elonas iš Zabulono giminės. 
\par 12 Elonui mirus, jis buvo palaidotas Ajalone, Zabulono krašte. 
\par 13 Po jo Izraelį teisė Hilelio sūnus Abdonas iš Piratono. 
\par 14 Jis turėjo keturiasdešimt sūnų ir trisdešimt sūnėnų, jodinėjusių ant septyniasdešimties asilų. Jis teisė Izraelį aštuonerius metus. 
\par 15 Hilelio sūnui Abdonui mirus, jis buvo palaidotas Piratone, Efraimo krašte, amalekiečių kalnyne.



\chapter{13}

\par 1 Izraelitai darė pikta Viešpaties akivaizdoje, ir Viešpats atidavė juos į filistinų rankas keturiasdešimčiai metų. 
\par 2 Coroje gyveno vienas vyras iš Dano giminės, vardu Manoachas. Jo žmona buvo nevaisinga ir negimdė. 
\par 3 Viešpaties angelas pasirodė jai ir tarė: “Tu esi nevaisinga, tačiau pastosi ir pagimdysi sūnų. 
\par 4 Negerk vyno nė stipraus gėrimo ir nevalgyk nieko nešvaraus. 
\par 5 Tu pastosi ir pagimdysi sūnų. Skustuvas tegul nepaliečia jo galvos, nes berniukas bus pašvęstas Dievui nuo pat gimimo, jis pradės išlaisvinti Izraelį iš filistinų”. 
\par 6 Moteris atėjusi tarė savo vyrui: “Dievo vyras atėjo pas mane, jo išvaizda buvo kaip angelo, ir aš labai išsigandau. Aš jo nepaklausiau, iš kur jis, o jis savo vardo man nepasakė. 
\par 7 Jis tik pasakė: ‘Tu pastosi ir pagimdysi sūnų. Žiūrėk, negerk nei vyno, nei stipraus gėrimo ir nevalgyk nieko nešvaraus, nes berniukas bus Dievui pašvęstas nuo pat gimimo iki savo mirties’ ”. 
\par 8 Manoachas meldė Viešpatį: “Viešpatie! Leisk, meldžiu, angelui, kurį buvai siuntęs, vėl ateiti pas mus ir mus pamokyti, kaip auklėti berniuką, kuris užgims”. 
\par 9 Dievas išklausė Manoachą. Dievo angelas vėl atėjo pas moterį, jai sėdint lauke. Jos vyro nebuvo šalia jos. 
\par 10 Moteris, skubiai nubėgusi, pranešė savo vyrui: “Man pasirodė tas vyras, kuris aną dieną buvo atėjęs pas mane!” 
\par 11 Pakilęs Manoachas sekė žmoną ir, atėjęs pas tą vyrą, jam tarė: “Ar tu esi vyras, kuris kalbėjo su mano žmona?” Jis atsakė: “Taip”. 
\par 12 Manoachas klausė: “Kai tavo žodžiai išsipildys, ką mes turime daryti su vaiku ir kaip elgtis su juo?” 
\par 13 Viešpaties angelas atsakė Manoachui: “Moteris turi saugotis to, ką sakiau. 
\par 14 Jai nevalia valgyti vynmedžio vaisių, gerti vyno ar stipraus gėrimo, taip pat valgyti, kas nešvaru; ji turi vykdyti, ką įsakiau”. 
\par 15 Manoachas tarė Viešpaties angelui: “Prašau, pasilik pas mus, iki prirengsiu tau ožiuką”. 
\par 16 Viešpaties angelas atsakė Manoachui: “Jei mane ir sulaikysi, aš vis tiek nevalgysiu tavo valgio; o jei prirengsi deginamąją auką, tai aukok ją Viešpačiui”. Manoachas nežinojo, kad jis buvo Viešpaties angelas. 
\par 17 Jis klausė Viešpaties angelo: “Kuo tu vardu? Mes tave pagerbsime, kai tavo žodis išsipildys”. 
\par 18 Viešpaties angelas jam atsakė: “Kodėl klausi mano vardo? Tai yra paslaptis”. 
\par 19 Manoachas aukojo Viešpačiui ožiuką ir duonos auką ant uolos. Angelas padarė stebuklą jo ir jo žmonos akivaizdoje: 
\par 20 kai liepsna kilo nuo aukuro aukštyn į dangų, Viešpaties angelas pakilo aukštyn aukuro liepsnoje. Tai matydami, jie puolė veidais į žemę. 
\par 21 Viešpaties angelas daugiau jiems nebepasirodė. Manoachas suprato, kad tai buvo Viešpaties angelas, 
\par 22 ir tarė žmonai: “Mes mirsime, nes matėme Dievą”. 
\par 23 Bet jo žmona atsakė jam: “Jei Viešpats būtų norėjęs mus nužudyti, jis nebūtų priėmęs mūsų deginamosios ir duonos aukos ir nebūtų mums parodęs to, ką matėme, ir pasakojęs to, ką girdėjome”. 
\par 24 Moteris pagimdė sūnų ir jį pavadino Samsonu. Berniukas augo, ir Viešpats jį laimino. 
\par 25 Viešpaties Dvasia pradėjo veikti jame Mahane Dano stovykloje, kuri buvo tarp Coros ir Eštaolio.



\chapter{14}

\par 1 Samsonas, nuėjęs į Timną, pamatė vieną filistinų mergaitę. 
\par 2 Sugrįžęs jis sakė tėvams: “Mačiau Timnoje filistinų mergaitę. Leiskite man ją vesti”. 
\par 3 Jo tėvai atsakė: “Nejaugi tarp tavo giminių ir visoje tautoje nėra mergaitės, kad nori vesti iš neapipjaustytų filistinų?” Bet Samsonas atsakė: “Leiskite man ją vesti, nes ji man labai patinka”. 
\par 4 Tėvai nežinojo, kad Viešpats taip padarė, ieškodamas progos prieš filistinus. Tuo metu filistinai valdė Izraelį. 
\par 5 Samsonas su tėvais ėjo į Timną. Prie Timnos vynuogynų jis sutiko jauną liūtą. 
\par 6 Viešpaties Dvasia galingai nužengė ant jo, ir jis sudraskė jį kaip ožiuką plikomis rankomis, bet viso to nepapasakojo tėvams. 
\par 7 Nuėjęs pas mergaitę, kalbėjosi, ir ji patiko Samsonui. 
\par 8 Po kurio laiko jis grįžo jos paimti. Eidamas jis pasuko iš kelio, norėdamas pamatyti liūto dvėselieną. Ir štai liūto dvėselienoje bičių spiečius ir medus. 
\par 9 Jis paėmė medaus ir ėjo toliau valgydamas. Parėjęs pas savo tėvus, jis davė ir jiems to medaus, ir jie valgė. Bet jis nepasakė jiems, kad tą medų jis buvo ėmęs iš liūto dvėselienos. 
\par 10 Jo tėvas nuėjo pas tą mergaitę, ir Samsonas suruošė ten puotą; taip darydavo jaunikiai. 
\par 11 Filistinai, jį pamatę, atvedė trisdešimt pabrolių, kad jie būtų su juo. 
\par 12 Samsonas tarė: “Aš užminsiu jums mįslę. Jei ją įminsite per septynias puotos dienas, aš jums duosiu trisdešimt apatinių ir trisdešimt išeiginių drabužių, 
\par 13 o jei neįminsite, tai jūs man duosite trisdešimt apatinių ir trisdešimt išeiginių drabužių”. Jie atsakė jam: “Užmink mums tą mįslę, norime ją išgirsti”. 
\par 14 Jis tarė jiems: “Iš ėdiko išėjo maistas, iš stipruolio­saldumas”. Jie negalėjo įminti mįslės per tris dienas. 
\par 15 Ketvirtą dieną jie tarė Samsono žmonai: “Išgauk iš savo vyro mįslės įminimą. Jei neišgausi, sudeginsime tave ir tavo tėvo namus. Argi mus čia pasikvietėte apiplėšti?” 
\par 16 Samsono žmona verkdama kalbėjo: “Tu nemyli manęs. Tu užminei mįslę mano tautos sūnums, o man jos nesakai”. Jis jai tarė: “Aš jos nepasakiau nei savo tėvui, nei motinai, kodėl turėčiau tau ją pasakyti?” 
\par 17 Ji verkė septynias dienas, kol vyko puota. Septintą dieną, netekęs kantrybės, jis pasakė mįslės įminimą. Ji viską pasakė savo tautos sūnums. 
\par 18 Septintą dieną, prieš saulės nusileidimą, miesto vyrai Samsonui tarė: “Kas saldesnis už medų? Ir kas stipresnis už liūtą?” Jis jiems atsakė: “Jei nebūtumėte arę su mano telyčia, nebūtumėt įminę mįslės”. 
\par 19 Tada Viešpaties Dvasia nužengė ant jo ir jis, nuėjęs į Aškeloną, užmušė trisdešimt vyrų, paėmė jų drabužius ir atidavė juos tiems, kurie įminė mįslę. Po to, labai supykęs, jis sugrįžo į savo tėvo namus. 
\par 20 Samsono žmona buvo atiduota vienam iš pabrolių.



\chapter{15}


\par 1 Po kurio laiko, kviečių pjūties metu, Samsonas atėjo aplankyti savo žmonos ir atnešė jai ožiuką. Samsonas sakė: “Aš noriu įeiti į kambarį pas savo žmoną”. Bet jos tėvas jo neleido, 
\par 2 tardamas: “Aš tikrai maniau, kad tu jos nekenti, todėl ją atidaviau tavo pabroliui. Argi jos jaunesnioji sesuo nėra gražesnė už ją? Imk ją vietoj anos”. 
\par 3 Tada Samsonas atsakė: “Šį kartą nenusikalsiu, atkeršydamas filistinams”. 
\par 4 Samsonas, sugavęs tris šimtus lapių, surišo jas po porą uodegomis ir tarp uodegų įrišo po deglą. 
\par 5 Uždegęs deglus, jis paleido lapes į filistinų javus. Taip jis padegė javų pėdus, nepjautus javus, vynuogynus ir alyvmedžių sodus. 
\par 6 Filistinai klausė: “Kas tai padarė?” Jiems atsakė: “Samsonas, timniečio žentas, keršydamas už žmonos atidavimą pabroliui”. Atėję filistinai sudegino ją ir jos tėvą. 
\par 7 Samsonas jiems tarė: “Nors jūs tai padarėte, nurimsiu tik tada, kai jums atkeršysiu”. 
\par 8 Jis smarkiai puolė juos ir, sulaužęs jų blauzdų ir šlaunų kaulus, nuėjo į Etamą ir apsigyveno ant uolos. 
\par 9 Tada filistinai atėję pasistatė stovyklas Judo žemėje iki Lehio. 
\par 10 Judas klausė: “Kodėl išėjote prieš mus?” Tie atsakė: “Samsono surišti ir padaryti jam taip, kaip jis mums padarė”. 
\par 11 Tada trys tūkstančiai Judo vyrų atėjo prie Etamo uolos ir klausė Samsoną: “Argi nežinai, kad mus valdo filistinai? Kodėl taip padarei?” Jis atsakė: “Kaip jie man padarė, taip aš jiems padariau”. 
\par 12 Judo vyrai tarė jam: “Mes atėjome tavęs surišti ir atiduoti filistinams”. Samsonas atsakė: “Prisiekite man, kad jūs patys manęs nenužudysite”. 
\par 13 Jie jam atsakė: “Mes tave tvirtai surišime ir atiduosime filistinams, bet patys tavęs nenužudysime”. Jie surišo jį dviem naujomis virvėmis ir nusivedė. 
\par 14 Prie Lehio filistinai pasitiko jį šūkaudami iš džiaugsmo. Viešpaties Dvasia galingai nužengė ant jo, ir virvės ant jo rankų sutrūko kaip sudegę linai ir nukrito nuo jo. 
\par 15 Jis susirado šviežią asilo žandikaulį ir, paėmęs jį, užmušė juo tūkstantį filistinų. 
\par 16 Tuomet Samsonas tarė: “Asilo žandikauliu nužudžiau tūkstantį vyrų ir suverčiau į krūvas”. 
\par 17 Taip pasakęs, jis nusviedė žandikaulį. Tą vietą pavadino Ramat Lehiu. 
\par 18 Labai ištroškęs, jis šaukėsi Viešpaties: “Tu davei man šitokią pergalę, o dabar mirštu iš troškulio ir pateksiu į neapipjaustytųjų rankas!” 
\par 19 Dievas atvėrė daubą Lehyje, ir iš jos tekėjo vanduo. Atsigėręs jis atsigaivino ir atgavo jėgas. Tą vietą pavadino En Korės versme; ji tebėra ten iki šios dienos. 
\par 20 Filistinų laikais Samsonas teisė Izraelį dvidešimt metų.



\chapter{16}


\par 1 Kartą Samsonas, nuvykęs į Gazą, pamatė paleistuvę ir užėjo pas ją. 
\par 2 Gazos gyventojai, sužinoję, kad Samsonas atėjo į miestą, visą naktį laukė jo prie miesto vartų, sakydami: “Kai prašvis rytas, nužudysime jį”. 
\par 3 Samsonas buvo pas paleistuvę iki vidurnakčio. Atsikėlęs vidurnaktį, jis nutvėrė miesto vartus su staktomis, iškėlė juos kartu su sklende, užsidėjo ant pečių ir nunešė ant kalno, esančio prie Hebrono. 
\par 4 Vėliau jis pamilo Soreko slėnyje moterį, vardu Delila. 
\par 5 Atėję filistinų kunigaikščiai tarė jai: “Sužinok iš Samsono, kur glūdi didelės jo jėgos paslaptis ir kokiu būdu jį galėtume nugalėti ir surišti. Mes tau duosime kiekvienas po tūkstantį šimtą sidabrinių”. 
\par 6 Delila klausė Samsoną: “Pasakyk man, kur glūdi tavo didelės jėgos paslaptis ir kuo galima tave surišti, kad neištrūktum?” 
\par 7 Samsonas jai atsakė: “Jei mane surištų septyniomis virvėmis iš gyslų, kurios dar nėra išdžiūvusios, tai aš būčiau toks, kaip bet kuris kitas žmogus”. 
\par 8 Filistinų kunigaikščiai atnešė jai septynias virves iš gyslų, kurios dar nebuvo išdžiūvusios, ir ji jomis surišo Samsoną. 
\par 9 Jos kambaryje laukė pasislėpę vyrai. Delila sušuko: “Samsonai, filistinai puola!” Jis sutraukė virves, kaip sutrūksta pakuliniai siūlai nuo ugnies. Iš kur jo jėga­nepaaiškėjo. 
\par 10 Delila tarė Samsonui: “Tu pasityčiojai iš manęs, meluodamas man. Dabar, prašau tavęs, pasakyk man, kuo galima tave surišti?” 
\par 11 Jis atsakė: “Jei mane surištų visiškai naujomis virvėmis, kurios dar nebuvo naudotos, tai aš turėčiau tokią jėgą, kaip bet kuris kitas žmogus”. 
\par 12 Delila, surišusi naujomis virvėmis Samsoną, tarė: “Samsonai, filistinai puola!” Tuo metu pasislėpę vyrai sėdėjo kambaryje. Jis sutraukė virves kaip siūlus. 
\par 13 Delila tarė Samsonui: “Lig šiol tu tyčiojaisi iš manęs ir man melavai. Pasakyk, kuo tave būtų galima surišti?” Jis atsakė jai: “Jei supinsi septynias mano galvos garbanas į kasas ir prikalsi vinimi, aš tapsiu toks silpnas, kaip bet kuris kitas žmogus”. 
\par 14 Jam užmigus, ji supynė septynias jo galvos garbanas į kasas, prikalė jas vinimi ir tarė: “Samsonai, filistinai puola!” Pabudęs iš miego, jis ištraukė vinį su plaukais. 
\par 15 Delila supykus tarė: “Kaip tu gali sakyti: ‘Aš myliu tave’, kai tavo širdis nėra su manimi? Jau tris kartus pasityčiojai iš manęs ir nepasakei, kur glūdi tavo didelės jėgos paslaptis”. 
\par 16 Ji nedavė jam ramybės savo kalbomis ir, primygtinai klausinėdama, taip jam įkyrėjo, 
\par 17 kad jis pasakė jai visą tiesą: “Mano galva niekados nebuvo skusta, nes aš esu pašvęstas Dievui nuo pat gimimo. Jei man nuskustų galvą, tai netekčiau jėgos ir tapčiau silpnas, kaip bet kuris kitas žmogus”. 
\par 18 Delila pamatė, kad jis atvėrė jai širdį, ir pasišaukė filistinų kunigaikščius, sakydama: “Dar kartą ateikite, nes jis man atvėrė savo širdį”. Filistinų kunigaikščiai atėjo pas ją ir atnešė pinigus. 
\par 19 Delila užmigdė Samsoną ant savo kelių ir, pasišaukusi vyrą, liepė nukirpti Samsonui septynias galvos garbanas. Jo jėga dingo. 
\par 20 Ji tarė: “Samsonai, filistinai puola!” Pabudęs iš miego, jis galvojo, kad bus taip, kaip anksčiau. Bet jis nežinojo, kad Viešpats pasitraukė nuo jo. 
\par 21 Filistinai, nutvėrę jį, išdūrė jam akis, nusivedė į Gazą ir, sukaustę varinėmis grandinėmis, pristatė jį sukti girnas kalėjime. 
\par 22 Tuo laiku jo galvos plaukai pradėjo ataugti. 
\par 23 Filistinų kunigaikščiai susirinko aukoti gausių aukų savo dievui Dagonui ir pasidžiaugti laimėjimu. Jie sakė: “Mūsų dievas atidavė į mūsų rankas Samsoną, mūsų priešą”. 
\par 24 Žmonės, jį matydami, garbino savo dievą: “Mūsų dievas atidavė į mūsų rankas mūsų priešą, kuris nusiaubė mūsų šalį ir daug mūsiškių išžudė”. 
\par 25 Įsilinksminę jie tarė: “Pašaukite Samsoną, kad jis mus palinksmintų”. Jie atvedė Samsoną iš kalėjimo ir jį pastatė tarp kolonų, kad juos linksmintų. 
\par 26 Samsonas sakė jaunuoliui, kuris vedė jį už rankos: “Privesk mane prie kolonų, ant kurių laikosi pastatas, kad galėčiau atsiremti į jas”. 
\par 27 Pastatas buvo pilnas vyrų ir moterų, ten buvo ir visi filistinų kunigaikščiai. Ant stogo buvo apie tris tūkstančius vyrų bei moterų, kurie žiūrėjo, kaip Samsonas juos linksmino. 
\par 28 Samsonas šaukėsi Viešpaties: “Viešpatie Dieve, prašau, atsimink mane! Sustiprink mane, Viešpatie, dar kartą, kad atkeršyčiau filistinams vienu smūgiu už savo akis!” 
\par 29 Samsonas įsirėmė į abi vidurines kolonas, kurios laikė pastatą, į vieną­dešine, į kitą­kaire ranka 
\par 30 ir tarė: “Mirštu drauge su filistinais!” Ir iš visų jėgų pastūmė. Pastatas griuvo ant kunigaikščių ir ant visų jame buvusiųjų. Žuvusiųjų, kuriuos jis užmušė mirdamas, buvo daugiau negu tų, kuriuos užmušė gyvendamas. 
\par 31 Po to atėjo jo broliai ir visi giminės. Jie, pasiėmę jį, nunešė ir palaidojo jo tėvo Manoacho kape tarp Coros ir Eštaolio. Samsonas teisė Izraelį dvidešimt metų.



\chapter{17}


\par 1 Efraimo aukštumose gyveno vyras, vardu Mikajas. 
\par 2 Jis tarė savo motinai: “Tu prakeikei tą, kuris paėmė iš tavęs tūkstantį šimtą sidabrinių. Tie pinigai yra pas mane. Aš juos paėmiau”. Jo motina tarė: “Viešpats telaimina tave, mano sūnau”. 
\par 3 Kai jis sugrąžino motinai tūkstantį šimtą sidabrinių, motina tarė: “Aš visus tuos pinigus pašvenčiau Viešpačiui, kad mano sūnus, ėmęs juos iš mano rankos, padarytų drožtą ir lietą atvaizdą. Todėl dabar aš atiduosiu juos tau”. 
\par 4 Bet jis sugrąžino pinigus motinai. Ji paėmė du šimtus sidabrinių ir atidavė auksakaliui, kad iš jų padarytų drožtą ir lietą atvaizdą. Juos pastatė Mikajo namuose. 
\par 5 Mikajas įrengė savo namuose dievų šventyklą, padarė efodą ir terafimą ir paskyrė vieną iš savo sūnų kunigu. 
\par 6 Tuo metu Izraelyje nebuvo karaliaus. Kiekvienas darė tai, kas jam atrodė teisinga. 
\par 7 Judo Betliejuje buvo jaunas levitas, kuris ten gyveno. 
\par 8 Palikęs Judo Betliejų, jis ieškojo kitos vietos, tinkamos apsigyventi. Keliaudamas jis atėjo pas Mikają, kuris gyveno Efraimo aukštumose. 
\par 9 Mikajas jo klausė: “Iš kur tu?” Tas atsakė: “Aš esu levitas iš Judo Betliejaus ir ieškau vietos apsigyventi”. 
\par 10 Mikajas jam atsakė: “Pasilik pas mane ir būk man tėvu ir kunigu. Aš tau duosiu metams dešimt sidabrinių, drabužius ir visą išlaikymą”. 
\par 11 Levitas sutiko apsigyventi pas jį. Jaunuolis buvo jam kaip sūnus. 
\par 12 Mikajas paskyrė levitą kunigu, ir jaunuolis gyveno jo namuose. 
\par 13 Mikajas sakė: “Dabar žinau, kad Viešpats darys man gera, nes turiu levitą kunigu”.



\chapter{18}

\par 1 Tuo metu Izraelis neturėjo karaliaus. Dano giminė ieškojo krašto, kuriame galėtų apsigyventi, nes tuo laiku jiems dar nebuvo duota paveldėjimo tarp Izraelio giminių. 
\par 2 Jie pasiuntė savo giminės penkis vyrus iš Coros ir Eštaolio išžvalgyti kraštą. Atėję į Efraimo aukštumas, jie apsinakvojo Mikajo namuose. 
\par 3 Būdami pas Mikają, jie atpažino jaunuolio levito balsą ir, užėję pas jį, paklausė: “Kas tave čia atvedė? Ką tu čia veiki? Ką tu čia turi?” 
\par 4 Jis jiems papasakojo, kad Mikajas pasamdė jį būti jo kunigu. 
\par 5 Jie prašė: “Paklausk Dievo, ar mūsų kelionė bus sėkminga?” 
\par 6 Kunigas jiems atsakė: “Eikite ramybėje. Viešpats mato kelią, kuriuo jūs einate”. 
\par 7 Žvalgai išėjo ir nuvyko į Laišą. Ten jie matė, kad žmonės gyvena nerūpestingai, pagal sidoniečių papročius, ramiai ir saugiai. Tarp jų nebuvo valdininkų, kurie jiems vadovautų. Jie gyveno toli nuo sidoniečių ir nepalaikė jokių ryšių su kitais. 
\par 8 Jiems sugrįžus pas savo brolius į Corą ir Eštaolį, tie klausė juos: “Ką matėte?” 
\par 9 Jie atsakė: “Kilkite ir eikime! Mes matėme žemę, kuri yra labai gera. Nedelskite eiti ir užimti ją. 
\par 10 Nuėję užklupsite žmones, kurie jaučiasi saugūs. Jų kraštas platus, ir Dievas jį atidavė į jūsų rankas; tai kraštas, kuriame nieko netrūksta, kas yra žemėje”. 
\par 11 Iš Coros ir Eštaolio pakilo šeši šimtai Dano giminės ginkluotų vyrų. 
\par 12 Jie atžygiavę pasistatė stovyklą Judo Kirjat Jearime. Ta vieta tebevadinama Dano stovykla iki šios dienos ir yra už Kirjat Jearimo. 
\par 13 Iš ten jie žygiavo į Efraimo aukštumas ir pasiekė Mikajo namus. 
\par 14 Tie penki vyrai, kurie išžvalgė Laišo šalį, tarė savo broliams: “Ar žinote, kad šiuose namuose yra efodas, terafimas ir drožtas bei lietas atvaizdai? Pagalvokite, ką reikėtų daryti”. 
\par 15 Jie užsuko į jaunuolio levito namus, Mikajo namus, ir pasveikino jį. 
\par 16 Ir šeši šimtai Dano giminės apsiginklavusių karių stovėjo tarpuvartėje. 
\par 17 Tuo metu vyrai, kurie buvo išžvalgyti krašto, įėjo į Mikajo namus ir paėmė drožtą atvaizdą, efodą, terafimą ir nulietą atvaizdą. Tuo laiku kunigas stovėjo tarpuvartėje su šešiais šimtais apsiginklavusių vyrų. 
\par 18 Jiems įsibrovus į Mikajo namus ir paėmus visus minėtus daiktus, kunigas klausė: “Ką darote?” 
\par 19 Jie atsakė jam: “Tylėk! Užsidenk ranka burną ir eik su mumis. Būk mums tėvu ir kunigu. Ar tau geriau būti kunigu vieno vyro namams, ar visos giminės Izraelyje?” 
\par 20 Kunigas nudžiugo. Jis paėmė efodą, terafimą bei drožtą atvaizdą ir įsimaišė tarp žmonių. 
\par 21 Po to jie, sustatę priekyje vaikus, galvijus ir vežimus, pasisuko ir keliavo toliau. 
\par 22 Jiems nutolus nuo Mikajo namų, Mikajo kaimynai susibūrę pasivijo danius. 
\par 23 Jie šaukė Dano vaikams, ir tie atsigręžę klausė Mikają: “Kas atsitiko, kad atėjai su tokiu būriu?” 
\par 24 Jis atsakė: “Jūs paėmėte mano dievus, kuriuos pasidariau, kunigą ir nuėjote. Kas gi man beliko? Ir dar klausiate, kas atsitiko?” 
\par 25 Danai atsakė: “Nutilk! Neerzink mūsų, kad įpykę vyrai neužpultų ir nenužudytų tavęs ir tavo giminės”. 
\par 26 Ir danai nuėjo savo keliu. Mikajas, matydamas, kad jie buvo stipresni už jį, sugrįžo į savo namus. 
\par 27 Danai paėmė tai, ką Mikajas buvo pasidaręs, ir kunigą. Nuėję į Laišą, užpuolė ramius ir saugiai besijaučiančius gyventojus, juos išžudė kardu, o miestą sudegino. 
\par 28 Niekas jiems nepadėjo, nes jie gyveno toli nuo Sidono ir nepalaikė jokių ryšių su kitais. Tas miestas buvo Bet Rehobo slėnyje. Danai miestą atstatė ir apsigyveno jame. 
\par 29 Tą miestą jie pavadino Izraelio sūnaus Dano, savo tėvo, vardu. Anksčiau tas miestas vadinosi Laišas. 
\par 30 Danai pasistatė drožtą atvaizdą, o Manaso sūnaus Geršomo sūnus Jehonatanas ir jo sūnūs buvo kunigais Dano giminėje iki ištrėmimo dienos. 
\par 31 Jie laikė pas save Mikajo padarytą drožtą atvaizdą visą laiką, kol Dievo šventykla buvo Šilojuje.
Online Parallel Study Bible



\chapter{19}

\par 1 Tuo metu, kai Izraelyje nebuvo karaliaus, vienas levitas gyveno kaip ateivis Efraimo kalnyno pakraštyje. Jis turėjo sugulovę iš Judo Betliejaus. 
\par 2 Sugulovė buvo jam neištikima. Ji pabėgo nuo jo į savo tėvo namus, į Judo Betliejų, ir buvo ten keturis mėnesius. 
\par 3 Jos vyras, nuėjęs į jos tėvo namus, maloniai kalbėjo su ja ir norėjo parsivesti ją atgal. Jis buvo pasiėmęs savo tarną ir porą asilų. Merginos tėvas džiaugėsi jį sutikdamas. 
\par 4 Uošvis užlaikė jį, ir jis pasiliko ten tris dienas. Jie valgė, gėrė ir nakvojo. 
\par 5 Ketvirtą dieną, atsikėlę anksti rytą, jie ruošėsi keliauti. Merginos tėvas sakė savo žentui: “Pavalgyk, o paskui galėsite keliauti”. 
\par 6 Jie abu valgė ir gėrė. Po to merginos tėvas tarė: “Pasilik nakčiai! Tegul pasidžiaugia tavo širdis”. 
\par 7 Jis norėjo keliauti, bet uošvis jį perkalbėjo, kad jis pasiliktų nakvoti. 
\par 8 Penktąją dieną atsikėlęs anksti norėjo keliauti. Merginos tėvas tarė: “Pasistiprink ir pasilik iki popietės”. Juodu pavalgė. 
\par 9 Kai levitas, jo sugulovė ir tarnas pasiruošė keliauti, uošvis vėl kalbėjo: “Žiūrėk, diena jau eina vakarop. Pasilikite nakčiai. Tegul pasidžiaugia tavo širdis, o rytoj, anksti atsikėlę, galėsite keliauti į namus”. 
\par 10 Tačiau jis nebenorėjo nakvoti ir iškeliavo. Jis atvyko iki Jebuso (dabartinė Jeruzalė). Jis turėjo su savimi porą pabalnotų asilų ir sugulovę. 
\par 11 Saulei leidžiantis, jie buvo prie Jebuso. Tarnas sakė savo šeimininkui: “Pasukime į šitą jebusiečių miestą ir nakvokime ten”. 
\par 12 O šeimininkas atsakė: “Ne, mes nesuksime į svetimtaučių miestą. Jie nėra Izraelio vaikai. Keliausime toliau iki Gibėjos miesto”. 
\par 13 Ir jis sakė savo tarnui: “Eime nakvoti į Gibėją arba į Ramą”. 
\par 14 Jie praėjo Jebusą ir keliavo toliau. Kai jie buvo prie Gibėjos miesto, priklausančio Benjaminui, nusileido saulė. 
\par 15 Ir jie pasuko į Gibėją, kad apsistotų nakčiai. Atėję jie pasiliko miesto gatvėje, nes neatsirado nė vieno, kuris būtų juos priėmęs į savo namus nakvynei. 
\par 16 Tuo metu senas vyras grįžo iš lauko darbų. Jis buvo nuo Efraimo aukštumų ir gyveno kaip ateivis Gibėjoje. Tos vietos gyventojai buvo benjaminai. 
\par 17 Jis pamatė pakeleivį miesto gatvėje. Senas žmogus paklausė: “Iš kur atvykai ir kur eini?” 
\par 18 Tas jam atsakė: “Mes einame iš Judo Betliejaus į Efraimo kalnyno pakraštį, nes ten gyvenu. Buvau nuvykęs į Judo Betliejų, o dabar einu į Viešpaties namus. Neatsirado nė vieno, kuris priimtų mane nakvoti. 
\par 19 Turime šiaudų ir pašaro asilams, taip pat duonos ir vyno man, tavo tarnaitei ir jaunuoliui, kuris yra su tavo tarnais. Mums nieko netrūksta”. 
\par 20 Senas vyras atsakė: “Ramybė tau. Visa, ko reikia, parūpinsiu, tik nenakvok gatvėje”. 
\par 21 Jis įvedė juos į savo namus ir pašėrė asilus. Jie nusiplovė kojas, valgė ir gėrė. 
\par 22 Kai jie linksmino savo širdis, miesto vyrai, Belialo sūnūs, apsupo namą ir daužė duris, šaukdami: “Išvesk tą vyrą, kuris atvyko į tavo namus, kad jį pažintume!” 
\par 23 Namų šeimininkas išėjęs tarė: “Ne, broliai. Meldžiu, nesielkite taip piktai. Šitas vyras yra svečias mano namuose, nedarykite tokios kvailystės. 
\par 24 Aš turiu dukterį, nekaltą mergaitę, ir tas vyras turi sugulovę. Aš jas išvesiu jums. Jūs galite žeminti jas ir daryti su jomis, kas jums atrodo tinkama. Tačiau su tuo vyru nesielkite taip bjauriai”. 
\par 25 Bet vyrai nenorėjo jo klausyti. Tada vyras paėmė savo sugulovę ir išvedė jiems. Jie išniekino ją ir vargino ją visą naktį. Dienai brėkštant, jie ją paleido. 
\par 26 Ta moteris atėjo auštant ir parkrito prie to vyro namo durų, kur buvo jos šeimininkas, ir gulėjo, iki prašvito. 
\par 27 Atsikėlęs rytą, jos šeimininkas atidarė duris, norėdamas keliauti. Moteris, jo sugulovė, gulėjo parkritusi prie namo durų, ištiesusi rankas ant slenksčio. 
\par 28 Jis tarė jai: “Kelkis, keliaukime”. Bet ji neatsakė. Jis ją užkėlė ant asilo ir parkeliavo į savo namus. 
\par 29 Namuose paėmė peilį ir supjaustė savo sugulovę į dvyliką gabalų, ir išsiuntė visoms Izraelio giminėms. 
\par 30 Tai matydami, visi kalbėjo: “Tokių įvykių nėra buvę nuo izraelitų išvykimo iš Egipto iki šios dienos. Apsvarstykime, pasitarkime ir nuspręskime, ką daryti”.



\chapter{20}

\par 1 Izraelitai nuo Dano iki Beer Šebos, taip pat ir Gileado krašto vyrai susirinko visi kaip vienas Micpoje. 
\par 2 Visos Izraelio tautos giminių vadai ir keturi šimtai tūkstančių ginkluotų vyrų dalyvavo susirinkime. 
\par 3 Benjaminai išgirdo, kad izraelitai susirinko Micpoje. Izraelitai klausė: “Sakykite, kaip atsitiko tokia piktadarystė?” 
\par 4 Levitas, nužudytosios moters vyras, atsakė: “Atvykau su savo sugulove į Gibėją, priklausančią Benjaminui, ir apsinakvojau. 
\par 5 Gibėjos vyrai naktį apsupo namus. Jie norėjo mane nužudyti; mano sugulovę taip nukankino, kad ji mirė. 
\par 6 Tuomet savo sugulovę supjausčiau ir jos dalis išsiunčiau į visus Izraelio paveldėtus kraštus, nes jie padarė Izraelyje bjaurų nusikaltimą. 
\par 7 Izraelitai, dabar apsvarstykite ir nutarkite, ką daryti”. 
\par 8 Visa tauta vienu balsu pasisakė: “Nė vienas iš mūsų neisime į savo palapinę ir negrįšime į namus. 
\par 9 Štai ką mes padarysime Gibėjai: mesime burtą, eidami prieš juos. 
\par 10 Išskirsime iš visų Izraelio giminių po dešimt vyrų iš šimto, po šimtą iš tūkstančio ir po tūkstantį iš dešimt tūkstančių, kad atgabentų maisto kariams, kurie eis prieš Benjamino Gibėją nubausti už padarytą Izraelyje bjaurų nusikaltimą”. 
\par 11 Visi Izraelio vyrai susirinko kariauti prieš tą miestą ir buvo kaip vienas. 
\par 12 Izraelio giminių vadai išsiuntė vyrus į visą Benjamino kraštą, klausdami: “Kodėl tarp jūsų vyksta tokios piktadarystės? 
\par 13 Išduokite tuos vyrus, Belialo vaikus, gyvenančius Gibėjoje, kad juos nubaustume mirtimi ir pašalintume pikta iš Izraelio”. Benjaminai nenorėjo klausyti savo brolių izraelitų. 
\par 14 Jie susirinko iš visų miestų į Gibėją kariauti su izraelitais. 
\par 15 Jų buvo dvidešimt šeši tūkstančiai kardais ginkluotų vyrų, neskaičiuojant Gibėjos septynių šimtų rinktinių karių. 
\par 16 Iš visų žmonių buvo septyni šimtai rinktinių vyrų, kurie buvo kairiarankiai ir galėjo pataikyti iš mėtyklės akmeniu į plauką. 
\par 17 Izraelitų, išskyrus benjaminus, buvo keturi šimtai tūkstančių ginkluotų vyrų. 
\par 18 Jie atvyko į Betelį Dievo pasiklausti, kas iš jų pirmas turi pradėti kovą su benjaminais. Viešpats atsakė: “Judas eis pirmas”. 
\par 19 Izraelitai anksti rytą pasistatė stovyklą prie Gibėjos 
\par 20 ir išsirikiavo kautynėms prieš Benjaminą ir Gibėją. 
\par 21 Benjaminai, išėję iš Gibėjos, tą dieną sunaikino dvidešimt du tūkstančius izraelitų. 
\par 22 Izraelio kariai, atgavę drąsą, vėl išsirikiavo kautynėms toje pačioje vietoje. 
\par 23 Prieš kautynes izraelitai nuvyko į Betelį ir verkė Viešpaties akivaizdoje iki vakaro, ir klausė: “Ar mums dar kartą eiti į mūšį su mūsų broliais benjaminais?” Viešpats atsakė: “Eikite prieš juos”. 
\par 24 Izraelitai antrą kartą išėjo prieš benjaminus, 
\par 25 o benjaminai išėjo jiems priešais iš Gibėjos ir išžudė aštuoniolika tūkstančių Izraelio ginkluotų karių. 
\par 26 Tada visi izraelitai ir visa tauta atėjo į Betelį. Ten jie sėdėjo verkdami Viešpaties akivaizdoje ir pasninkavo iki vakaro. Jie aukojo deginamąsias bei padėkos aukas Viešpačiui. 
\par 27 Izraelitai klausė Viešpaties (tada ten buvo Dievo Sandoros skrynia 
\par 28 ir Aarono sūnaus Eleazaro sūnus Finehasas tuo metu buvo prie jos), sakydami: “Ar mums dar kartą eiti į kovą prieš mūsų brolius benjaminus?” Viešpats atsakė: “Eikite, nes rytoj Aš juos atiduosiu į jūsų rankas”. 
\par 29 Izraelis pastatė pasalas aplink Gibėją. 
\par 30 Trečią dieną Izraelis išėjo prieš benjaminus ir išsirikiavo prieš Gibėją kaip anksčiau. 
\par 31 Benjaminai išėjo prieš tautą ir, atsitraukę nuo miesto kaip anksčiau, pradėjo žudyti žmones ant vieškelių, kurių vienas veda į Betelį, o kitas­į Gibėją. Krito apie trisdešimt Izraelio vyrų. 
\par 32 Tada benjaminai manė, kad izraelitai traukiasi nuo jų kaip anksčiau, o izraelitai bėgo, norėdami juos nuvilioti nuo miesto į vieškelius. 
\par 33 Izraelitai išsirikiavo kovai prie Baal Tamaros; tuomet Izraelio pasalos pulkai pakilo iš savo vietų už Gibėjos miesto. 
\par 34 Atėjo prieš Gibėją dešimt tūkstančių rinktinių vyrų iš viso Izraelio, ir užvirė smarki kova. Benjaminai nežinojo, kad jų laukia nelaimė. 
\par 35 Viešpats suteikė pergalę Izraeliui, ir jie sunaikino tą dieną dvidešimt penkis tūkstančius šimtą benjaminų, ginkluotų kardais. 
\par 36 Benjaminai pamatė, kad jie sumušti. Izraelitai davė benjaminams vietos, nes pasitikėjo pasala, kuri buvo pastatyta prie Gibėjos. 
\par 37 Pasalos pulkai netikėtai puolė Gibėją, užėmė miestą ir išžudė kardais gyventojus. 
\par 38 Izraelitų ir pasalos sutartas ženklas buvo padegti miestą, kad iš jo kiltų dūmai. 
\par 39 Izraelitai traukėsi iš mūšio, ir benjaminai nužudė apie trisdešimt izraelitų, manydami, kad jie bėga sumušti kaip pirmajame mūšyje. 
\par 40 Bet kai iš miesto pradėjo kilti liepsna ir dūmai, benjaminai atsigręžę pamatė, kad miestas dega. 
\par 41 Kai Izraelio kariai atsigręžė, benjaminai išsigando, nes pamatė, kad juos užklupo nelaimė. 
\par 42 Jie pasuko bėgti dykumos link. Izraelitai pasivijo juos, ir išėjusieji iš miesto taip pat naikino juos. 
\par 43 Jie apsupo benjaminus ir persekiojo, naikindami iki Gibėjos. 
\par 44 Benjaminų čia krito aštuoniolika tūkstančių vyrų, narsių karių. 
\par 45 Jiems bėgant į dykumą prie Rimono uolos, jų krito dar penki tūkstančiai vyrų, likusius izraelitai persekiojo ligi Gidomo ir nužudė dar du tūkstančius vyrų. 
\par 46 Tą dieną benjaminų žuvo dvidešimt penki tūkstančiai ginkluotų vyrų, narsių karių. 
\par 47 Šeši šimtai benjaminų nubėgo į dykumą prie Rimono uolos ir pasiliko ten keturis mėnesius. 
\par 48 Izraelitai sugrįžę išžudė visus benjaminus ir gyvulius, o miestus sudegino.



\chapter{21}

\par 1 Izraelitai prisiekė Micpoje, kad nė vienas jų neišleis savo dukters už benjamino. 
\par 2 Susirinkę Betelyje, jie sėdėjo iki vakaro Dievo akivaizdoje graudžiai verkdami 
\par 3 ir klausė: “Viešpatie, Izraelio Dieve, kodėl taip atsitiko Izraelyje? Kodėl šiandien pasigendame vienos Izraelio giminės?” 
\par 4 Kitą dieną žmonės, anksti atsikėlę, pastatė aukurą ir aukojo deginamąsias ir padėkos aukas. 
\par 5 Izraelitai klausė: “Kas iš Izraelio giminių neatvyko į susirinkimą Betelyje?” Nes jie buvo prisiekę, kad tie, kurie neateis į susirinkimą Micpoje, bus baudžiami mirtimi. 
\par 6 Izraelitai gailėjosi brolio Benjamino ir kalbėjo: “Šiandien viena Izraelio giminė sunaikinta. 
\par 7 Ką darysime, kad likusieji benjaminai gautų žmonų? Juk mes prisiekėme Viešpačiu, kad jiems neduosime savo dukterų”. 
\par 8 Jie klausė: “Kas iš Izraelio giminių neatvyko Viešpaties akivaizdon į Micpą?” Paaiškėjo, kad iš Jabeš Gileado niekas nebuvo atvykęs į susirinkimą. 
\par 9 Patikrinę pamatė, kad nebuvo nė vieno žmogaus iš Jabeš Gileado. 
\par 10 Susirinkimas pasiuntė dvylika tūkstančių rinktinių karių ir jiems įsakė kardu išžudyti Jabeš Gileado vyrus, moteris ir vaikus. 
\par 11 Jie sakė: “Nužudykite kiekvieną vyrą ir kiekvieną moterį, gulėjusią su vyru”. 
\par 12 Jabeš Gileade jie rado keturis šimtus nekaltų mergaičių, kurias atvedė stovyklon į Šilojų, Kanaano krašte. 
\par 13 Po to visas susirinkimas pasiūlė taiką benjaminams, apsistojusiems prie Rimono uolos. 
\par 14 Tuomet benjaminai sugrįžo, o izraelitai atidavė jiems mergaites iš Jabeš Gileado. Bet jiems jų neužteko. 
\par 15 Tauta gailėjosi Benjamino, nes Viešpats padarė spragą tarp Izraelio giminių. 
\par 16 Izraelio vyresnieji klausė: “Ką darysime? Kaip surasime likusiems vyrams žmonų, nes Benjamino moterys išžudytos?” 
\par 17 Jie tarė: “Benjamino paveldėjimas teks išlikusiems, kad nebūtų sunaikinta giminė Izraelyje. 
\par 18 Tačiau mes negalime duoti jiems žmonų iš savo dukterų”. Izraelitai buvo prisiekę: “Prakeiktas, kuris duotų dukterį benjaminui”. 
\par 19 Jie kalbėjo benjaminams: “Kasmet vyksta Viešpaties šventė Šilojuje, kuris yra į šiaurę nuo Betelio, į rytus nuo vieškelio, vedančio iš Betelio į Sichemą, ir į pietus nuo Lebonos. 
\par 20 Nuėję į Šilojų, pasislėpkite vynuogynuose 
\par 21 ir stebėkite. Kai išeis Šilojo dukterys žaisti ratelį, išeikite iš vynuogynų ir, pasigrobę kiekvienas sau žmoną iš Šilojo dukterų, grįžkite į Benjamino kraštą. 
\par 22 O kai ateis jų tėvai arba broliai su skundu pas mus, mes jiems sakysime: ‘Būkite jiems malonūs, nes mes nekariavome, kad gautume jiems žmonų, ir jūs patys jiems nedavėte, tai ir nenusikaltote’ ”. 
\par 23 Benjaminai taip ir padarė: jie pasiėmė tiek mergaičių, kiek jiems reikėjo žmonų. Po to kiekvienas sugrįžo į savo žemę, atstatė miestus ir juose gyveno. 
\par 24 Izraelitai taip pat sugrįžo kiekvienas į savo giminę, į savo šeimą savo žemėje. 
\par 25 Tuo metu Izraelyje nebuvo karaliaus. Kiekvienas darė tai, kas jam atrodė teisinga.



\end{document}