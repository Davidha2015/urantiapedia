\begin{document}

\title{Rutos knyga}

\chapter{1}


\par 1 Teisėjų valdymo laikais šalyje kilo badas. Vienas vyras iš Judo Betliejaus su žmona ir dviem sūnumis iškeliavo gyventi į Moabo kraštą. 
\par 2 To vyro vardas buvo Elimelechas, jo žmonos vardas Noomė, o sūnų­Machlonas ir Kiljonas, efratai iš Judo Betliejaus. Jie atvyko į Moabo kraštą ir ten apsigyveno. 
\par 3 Noomės vyras Elimelechas mirė ir ji liko su dviem sūnumis. 
\par 4 Jie vedė žmonas moabites. Viena vardu Orpa, o kita­Rūta. Jie ten išgyveno apie dešimt metų. 
\par 5 Po to abu sūnūs, Machlonas ir Kiljonas, mirė. Moteris neteko abiejų sūnų ir vyro. 
\par 6 Ji su savo marčiomis pakilo grįžti į savo kraštą, nes buvo girdėjusi būdama Moabo krašte, kad Viešpats aplankė savo tautą, duodamas jiems duonos. 
\par 7 Ji išėjo iš ten, kur gyveno kartu su savo marčiomis, ir ėjo į Judą. 
\par 8 Noomė tarė savo marčioms: “Eikite namo pas savo motinas. Viešpats tebūna malonus jums, kaip jūs buvote mirusiems ir man. 
\par 9 Viešpats jums tesuteikia poilsį vyro namuose”. Noomė pabučiavo savo marčias, kurios pakėlė balsus ir verkė. 
\par 10 Ir jos sakė: “Mes norime eiti su tavimi pas tavo žmones”. 
\par 11 Noomė atsakė: “Grįžkite, mano dukros. Kodėl jūs norite eiti su manimi? Ar manote, kad aš galiu užauginti sūnus, kurie taptų jūsų vyrais? 
\par 12 Grįžkite, mano dukros. Aš juk per sena vedyboms. Jei turėčiau viltį dar šiąnakt ištekėti ir pagimdyti sūnų, 
\par 13 argi jūs lauktumėte, kol jis užaugs? Argi dėl to susilaikytumėte nuo santuokos? Ne, mano dukros. Man labai skaudu dėl jūsų, bet Viešpaties ranka yra prieš mane”. 
\par 14 Jos pakėlė balsus ir vėl verkė. Orpa atsisveikino su savo anyta, o Rūta pasiliko su ja. 
\par 15 Noomė tarė: “Tavo brolienė sugrįžo pas savo tautą ir savo dievus. Grįžk ir tu”. 
\par 16 Tačiau Rūta atsakė: “Neversk manęs tave palikti ir sugrįžti. Kur tu eisi, ir aš eisiu; kur tu gyvensi, ir aš gyvensiu. Tavo tauta yra mano tauta ir tavo Dievas­mano Dievas. 
\par 17 Kur tu mirsi, ir aš ten mirsiu ir būsiu palaidota. Tegul Viešpats padaro man tai ir dar daugiau, bet tik mirtis atskirs mane nuo tavęs”. 
\par 18 Noomė matė, kad Rūta tvirtai pasiryžusi eiti su ja, ir nustojo atkalbinėti. 
\par 19 Taip jos keliavo, kol atėjo į Betliejų. Joms atėjus į Betliejų, visas miestas sujudo ir klausinėjo: “Ar tai Noomė?” 
\par 20 Ji atsakė: “Nevadinkite manęs Noome, vadinkite mane Mara, nes Visagalis labai apkartino mano gyvenimą. 
\par 21 Išėjau turtinga, o Viešpats parvedė tuščiomis rankomis. Kodėl vadinate mane Noome? Juk Viešpats paliudijo prieš mane, Visagalis mane nubaudė”. 
\par 22 Taip grįžo Noomė ir jos marti moabitė Rūta iš Moabo krašto. Jos atvyko į Betliejų miežių pjūties pradžioje.


\chapter{2}


\par 1 Noomės giminaitis iš vyro pusės buvo Boozas, pasiturintis vyras iš Elimelecho giminės. 
\par 2 Moabitė Rūta tarė Noomei: “Leisk man eiti į lauką varpų rinkti, kur man leis”. Noomė sakė: “Eik, mano dukra”. 
\par 3 Nuėjusi ji rinko varpas paskui pjovėjus. Jai pasitaikė rinkti lauke, kuris priklausė Boozui iš Elimelecho giminės. 
\par 4 Kaip tik tuo metu Boozas atėjo iš Betliejaus ir pasveikino pjovėjus: “Viešpats su jumis”. Tie jam atsakė: “Telaimina tave Viešpats”. 
\par 5 Tada Boozas paklausė pjovėjų prižiūrėtoją: “Kas šita moteris?” 
\par 6 Prižiūrėtojas atsakė: “Ji yra moabitė, atvykusi su Noome iš Moabo krašto. 
\par 7 Ji prašė: ‘Leisk man rinkti paskui pjovėjus nukritusias varpas’. Ji atėjo ir buvo čia nuo ryto iki dabar, tik trumpam buvo parėjusi į namus”. 
\par 8 Boozas tarė Rūtai: “Klausyk, mano dukra! Neik rinkti į kitą lauką ir nesitrauk iš čia, bet pasilik prie mano moterų. 
\par 9 Sek paskui pjovėjus. Aš įsakiau jaunuoliams tavęs neliesti. Ištroškusi eik prie ąsočių ir gerk tai, ką pjovėjai geria”. 
\par 10 Ji puolė veidu į žemę ir, nusilenkusi iki žemės, klausė: “Kodėl tu man toks malonus ir atkreipei dėmesį į mane, nors esu svetimšalė?” 
\par 11 Boozas atsakė jai: “Man viską papasakojo, kaip tu, tavo vyrui mirus, elgeisi su anyta ir kaip palikusi tėvą, motiną ir gimtąją šalį, atvykai į tautą, kurios anksčiau nepažinai. 
\par 12 Teatlygina tau Viešpats už tai, ką padarei, ir teduoda tau pilną užmokestį Viešpats, Izraelio Dievas, po kurio sparnais atėjai prisiglausti”. 
\par 13 Ji atsakė: “Mano viešpatie, kad tik rasčiau malonę tavo akyse. Tu paguodei mane, draugiškai kreipdamasis į savo tarnaitę, nors neprilygstu tavo tarnaitėms”. 
\par 14 Valgant Boozas jai tarė: “Ateik čia, valgyk duonos ir padažyk savo kąsnį vyne”. Ji atsisėdo šalia pjovėjų, o jis padavė jai paskrudintų grūdų. Ji pavalgė, o kas liko, pasilaikė. 
\par 15 Kai ji kėlėsi rinkti, Boozas įsakė savo tarnams: “Leiskite jai rinkti tarp pėdų ir nepriekaištaukite. 
\par 16 Iš pėdų tyčiomis ištraukite ir palikite jai, kad galėtų rinkti; nedrauskite jai”. 
\par 17 Ji rinko iki vakaro. Surinktas varpas iškūlė ir buvo arti efos miežių. 
\par 18 Parėjusi į miestą, ji parodė savo anytai, kiek pririnko. Be to, padavė jai likusį maistą, kurį buvo pasilaikiusi. 
\par 19 Anyta jos paklausė: “Kur tu šiandien rinkai? Tebūna palaimintas tas, kuris atkreipė dėmesį į tave!” Ji papasakojo savo anytai, pas ką dirbo, ir pasakė, kad to vyro vardas Boozas. 
\par 20 Noomė tarė savo marčiai: “Viešpats telaimina jį, nes jis išliko malonus gyviems ir mirusiems. Be to, tas vyras yra iš mūsų giminės, vienas iš artimiausių mūsų giminaičių”. 
\par 21 Rūta sakė: “Jis man patarė laikytis prie jo pjovėjų iki pjūties galo”. 
\par 22 Ir Noomė tarė savo marčiai Rūtai: “Gerai, mano dukra, kad eini su jo tarnaitėmis. Kitame lauke tave galėtų įžeisti”. 
\par 23 Ji laikėsi prie Boozo tarnaičių rinkdama varpas, kol pasibaigė miežių ir kviečių pjūtis, ir gyveno su savo anyta.



\chapter{3}

\par 1 Noomė tarė Rūtai: “Mano dukra, ar ne laikas man pasirūpinti vieta, kur tu galėtum ramiai gyventi? 
\par 2 Boozas, su kurio tarnaitėmis tu dirbai, yra mūsų giminaitis. Jis šį vakarą vėtys miežius klojime. 
\par 3 Nusiprausk, pasitepk, apsivilk geriausiu rūbu ir nueik į klojimą. Nepasirodyk jam, kol jis pavalgys ir atsiguls. 
\par 4 Kai jis atsiguls, įsidėmėk vietą, kur jis guli; priėjusi atidenk jo kojas ir atsigulk. Jis tau pasakys, ką daryti”. 
\par 5 Ji atsakė: “Visa, ką man sakai, padarysiu”. 
\par 6 Nuėjusi į klojimą, ji pasielgė taip, kaip jai anyta patarė. 
\par 7 Boozas, pavalgęs ir atsigėręs, buvo patenkintas ir atsigulė javų krūvos gale. Rūta tyliai priėjo, atidengė jo kojas ir atsigulė. 
\par 8 Vidurnaktį pabudęs žmogus nusigando, pamatęs moterį, gulinčią prie jo kojų. 
\par 9 Jis paklausė: “Kas tu esi?” Ji atsakė: “Aš esu tavo tarnaitė Rūta. Ištiesk savo apsiaustą ant savo tarnaitės, nes tu esi artimas giminaitis”. 
\par 10 Jis sakė: “Viešpats telaimina tave, mano dukra. Tavo paskutinis poelgis yra geresnis už pirmutinį, nes tu neieškojai jaunuolio, turtingo ar beturčio. 
\par 11 Nebijok, mano dukra. Visa, ko prašai, aš padarysiu. Visi šio miesto gyventojai žino, kad tu esi dora moteris. 
\par 12 Tikrai aš esu artimas tavo giminaitis, tačiau yra kitas, dar artimesnis už mane. 
\par 13 Pasilik šią naktį čia. O rytoj, jei jis tave paims,­gerai, tegul paima. O jei jis nenorės tavęs paimti, kaip Viešpats gyvas, aš tave paimsiu! Gulėk iki ryto”. 
\par 14 Ji gulėjo prie jo kojų iki ryto ir atsikėlė, kai žmogus žmogaus dar negalėjo atpažinti. Jis sakė jai: “Žiūrėk, kad niekas nesužinotų, jog moteris buvo klojime”. 
\par 15 Jis liepė jai ištiesti savo apsiaustą, kuriuo ji buvo apsisiautusi. Atseikėjęs šešis saikus miežių, jis supylė į apsiaustą ir užkėlė jai ant pečių. Taip ji grįžo į miestą. 
\par 16 Namuose anyta klausė: “Kaip tau sekėsi, mano dukra?” Ji papasakojo jai visa, kas įvyko, 
\par 17 ir sakė: “Šituos šešis saikus miežių jis man davė ir pasakė: ‘Tu neturi grįžti tuščiomis pas savo anytą’ ”. 
\par 18 Noomė tarė: “Dabar palauk, mano dukra, kol sužinosi, kaip viskas baigsis. Jis nenurims, kol visko nesutvarkys dar šiandien”.



\chapter{4}

\par 1 Boozas, nuėjęs prie vartų, atsisėdo. Pro šalį ėjo giminaitis, apie kurį Boozas buvo kalbėjęs. Jis tarė jam: “Bičiuli, sėskis čia!” Tas atėjęs atsisėdo. 
\par 2 Boozas pasišaukė dar dešimt vyrų, miesto vyresniųjų, ir tarė jiems: “Sėskitės!” Jie atsisėdo. 
\par 3 Ir jis kalbėjo giminaičiui: “Noomė, kuri grįžo iš Moabo krašto, parduoda lauką, priklausiusį mūsų broliui Elimelechui. 
\par 4 Aš norėjau pranešti tai tau. Pirk tą lauką čia sėdinčiųjų ir vyresniųjų akivaizdoje. Jei nori išpirkti, pirk, o jei nenori, pasakyk man, kad žinočiau. Nėra kito, kuris turi teisę tai išpirkti, o po tavęs­mano eilė”. Tas atsakė: “Aš pirksiu”. 
\par 5 Boozas tarė: “Tą dieną, kai pirksi lauką iš Noomės, turėsi nupirkti ir moabitę Rūtą, mirusiojo žmoną, kad išlaikytum mirusiojo vardą jo nuosavybei”. 
\par 6 Giminaitis atsakė: “Aš negaliu išpirkti jos, nes tada nukentėtų mano paties paveldėjimas. Tu pasinaudok mano teise, nes aš negaliu to padaryti”. 
\par 7 Izraelyje buvo paprotys: kai kas nors perleisdavo savo teisę į paveldėjimą kitam, jis nusiaudavo savo sandalą ir paduodavo savo artimui. Tai būdavo liudijimas Izraelyje. 
\par 8 Kai giminaitis pasakė Boozui: “Pirk tai sau”, nusiavė sandalą ir jam padavė. 
\par 9 Tada Boozas tarė vyresniesiems ir visiems žmonėms: “Jūs esate šiandien liudytojai, kad aš nupirkau iš Noomės visa, kas priklausė Elimelechui, Kiljonui ir Machlonui. 
\par 10 Taip pat ir moabitę Rūtą, Machlono žmoną, vesiu, kad išlaikyčiau mirusiojo vardą, jo nuosavybei ir kad jo vardas neišnyktų tarp jo brolių ir iš jo tėviškės. Jūs esate šiandien to įvykio liudytojai”. 
\par 11 Visi ten esantys žmonės ir vyresnieji atsakė: “Mes esame liudytojai. Viešpats telaimina moterį, ateinančią į tavo namus, kaip Rachelę ir Lėją, kurios sukūrė Izraelio namus. Tegul tau sekasi Efratoje ir tebūna garsus tavo vardas Betliejuje. 
\par 12 Tebūna tavo namai kaip namai Pereco, kurį Tamara pagimdė Judui, per palikuonis, kuriuos Viešpats tau duos iš šios jaunos moters”. 
\par 13 Boozas vedė Rūtą. Ir kai jis įėjo pas ją, Viešpats davė jai pastoti ir ji pagimdė sūnų. 
\par 14 Tuomet moterys sakė Noomei: “Palaimintas Viešpats, kuris nepaliko tavęs be įpėdinio, kad jo vardas būtų žymus Izraelyje. 
\par 15 Jis bus tau atgaiva ir pasirūpins tavimi senatvėje. Juk jį pagimdė tavo marti, kuri myli tave, ir yra tau daugiau negu septyni sūnūs”. 
\par 16 Noomė paėmė vaiką į savo prieglobstį ir buvo jo aukle. 
\par 17 Kaimynės jį pavadino Jobedu, sakydamos: “Sūnus gimė Noomei”. Jobedas buvo Jesės tėvas, karaliaus Dovydo senelis. 
\par 18 Šitie yra Pereco palikuonys: Perecs buvo Esromo tėvas, 
\par 19 Esromas­Aramo, Aramas­ Aminadabo, 
\par 20 Aminadabas­Naasono, Naasonas­Salmono, 
\par 21 Salmonas­Boozo, Boozas­Jobedo, 
\par 22 Jobedas­Jesės, o Jesė­Dovydo.



\end{document}