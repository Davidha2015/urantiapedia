\begin{document}

\title{Pirmoji Samuelio knyga}

\chapter{1}


\par 1 Efraimo aukštumoje, Ramoje gyveno vyras, vardu Elkana, efraimas. Jis buvo Cūfo sūnaus Tohuvo sūnaus Elihuvo sūnaus Jerohamo sūnus. 
\par 2 Jis turėjo dvi žmonas: Oną ir Peniną. Penina turėjo vaikų, o Ona neturėjo. 
\par 3 Elkana kasmet eidavo iš savo miesto į Šilojų melstis ir aukoti kareivijų Viešpačiui. Du Elio sūnūs, Hofnis ir Finehasas, ten buvo Viešpaties kunigais. 
\par 4 Kai Elkana aukodavo, jis duodavo aukos dalį savo žmonai Peninai ir visiems jos sūnums bei dukterims. 
\par 5 Bet Onai jis duodavo geriausią dalį, nes ją, nors Viešpats buvo uždaręs jos įsčias, mylėjo. 
\par 6 Jos priešininkė ją užgauliodavo ir erzindavo, nes Viešpats buvo uždaręs jos įsčias. 
\par 7 Taip būdavo kiekvienais metais, kai ji eidavo į Viešpaties namus. Toji taip užgauliodavo ją, kad ji verkdavo ir nevalgydavo. 
\par 8 Jos vyras Elkana klausė: “Ona, ko verki? Kodėl nevalgai? Ko liūdi? Argi aš tau nesu vertesnis už dešimtį sūnų?” 
\par 9 Ona atsistojo, kai jie pavalgė ir atsigėrė Šilojuje. Tuo metu kunigas Elis sėdėjo prie Viešpaties šventyklos durų. 
\par 10 Ona labai nuliūdusi meldėsi ir graudžiai verkė. 
\par 11 Ji davė įžadą: “Kareivijų Viešpatie, jei Tu pažvelgsi į savo tarnaitės sielvartą ir mane atsiminsi, ir nepamirši manęs, bet duosi man sūnų, tai aš atiduosiu jį Viešpačiui per visas jo gyvenimo dienas, ir skustuvas nepalies jo galvos”. 
\par 12 Ji ilgai meldėsi Viešpaties akivaizdoje, o Elis stebėjo jos lūpas. 
\par 13 Ona kalbėjo savo širdyje, jos lūpos judėjo, bet balso nesigirdėjo. Todėl Elis palaikė ją girta 
\par 14 ir tarė: “Ar ilgai būsi girta? Išsipagiriok”. 
\par 15 Ona atsakė: “Ne, viešpatie, aš esu nelaiminga moteris; aš negėriau nei vyno, nei stipraus gėrimo, tik išliejau savo širdį Viešpačiui. 
\par 16 Nelaikyk savo tarnaitės Belialo dukra, nes iš didelio sielvarto ir skausmo aš kalbėjau”. 
\par 17 Elis tarė: “Eik ramybėje, o Izraelio Dievas teįvykdo tavo prašymą”. 
\par 18 Ji atsakė: “Duok savo tarnaitei atrasti malonę tavo akyse”. Po to ji nuėjo, valgė ir nebeliūdėjo. 
\par 19 Anksti rytą atsikėlę ir pagarbinę Viešpatį, jie grįžo į savo namus, į Ramą. Elkana pažino savo žmoną Oną, ir Viešpats atsiminė ją. 
\par 20 Po kurio laiko Ona pagimdė sūnų ir jį pavadino Samueliu, sakydama: “Iš Viešpaties jį išmeldžiau”. 
\par 21 Kai Elkana ėjo su visais savo namiškiais aukoti Viešpačiui kasmetinę auką ir atlikti įžadą, 
\par 22 Ona nėjo. Ji sakė savo vyrui: “Aš neisiu, iki berniuką nujunkysiu. Po to jį nuvesiu Viešpaties akivaizdon ir ten paliksiu visam laikui”. 
\par 23 Jos vyras Elkana jai atsakė: “Daryk, kaip tau atrodo teisinga, pasilik namuose, iki jį nujunkysi. Viešpats tepatvirtina savo žodį”. Ji pasiliko ir žindė sūnų, iki jį nujunkė. 
\par 24 Nujunkiusi jį, ji paėmė berniuką, tris jaučius, efą miltų bei odinę vyno ir nuvedė jį į Šilojų Viešpaties šventyklon. Berniukas buvo dar mažas. 
\par 25 Papjovė jautį ir nuvedė berniuką pas Elį. 
\par 26 Ona sakė: “Mano viešpatie, kaip tu gyvas, aš esu ta pati moteris, kuri čia stovėjo tavo akivaizdoje ir meldėsi Viešpačiui. 
\par 27 Aš prašiau šito sūnaus. Viešpats išklausė mano prašymą ir suteikė, ko prašiau. 
\par 28 Aš jį atvedu Viešpačiui, kad visą savo gyvenimą jis Jam priklausytų”. Ir jis pagarbino ten Viešpatį.


\chapter{2}


\par 1 Ona meldėsi: “Mano širdis džiaugiasi Viešpatyje ir mano ragas išaukštintas Jame. Aš plačiai atveriu savo burną prieš savo priešus, nes džiaugiuosi Tavo išgelbėjimu. 
\par 2 Niekas nėra toks šventas, kaip Viešpats! Šalia Tavęs nėra kito ir nėra kitos uolos, kaip mūsų Dievas. 
\par 3 Nekalbėkite taip išdidžiai ir įžūliai. Viešpats yra Dievas, kuris žino viską, ir Jis pasveria visus darbus. 
\par 4 Galingųjų lankas sulaužomas, o tie, kurie klumpa, apjuosiami jėga. 
\par 5 Sotieji parsisamdė už maistą, o alkanieji nurimo. Nevaisingoji pagimdė septynetą, o turinčioji daug vaikų nusilpo. 
\par 6 Viešpats numarina ir atgaivina, nuveda į mirusiųjų buveinę ir vėl išlaisvina. 
\par 7 Viešpats padaro beturtį ir turtingą. Jis pažemina ir išaukština. 
\par 8 Jis pakelia iš dulkių vargdienį, iš sąšlavų duobės iškelia vargšą; juos pasodina greta kunigaikščių ir leidžia paveldėti jiems šlovės sostą. Viešpačiui priklauso žemės pamatai, ant jų Jis pastatė pasaulį. 
\par 9 Savo šventųjų kojas Jis palaikys, o nedorėliai nutils tamsoje. Nė vienas nelaimės savo jėga. 
\par 10 Viešpats sudaužys į gabalus savo priešus. Jis sugriaudės prieš juos iš dangaus. Viešpats teis žemės kraštus. Jis duos jėgų savo karaliui ir išaukštins pateptojo ragą”. 
\par 11 Elkana grįžo į Ramą, į savo namus, o berniukas Samuelis tarnavo Viešpačiui, prižiūrimas kunigo Elio. 
\par 12 Elio sūnūs buvo Belialo vaikai ir nepažino Viešpaties. 
\par 13 Kam nors aukojant auką, kai mėsa tebevirdavo, ateidavo kunigo tarnas, laikydamas rankoje trišakę, 
\par 14 ir smeigdavo ją į katilą, puodą ar keptuvę. Visa, ką ištraukdavo šake, pasiimdavo Elio sūnūs. Taip jie darė visiems izraelitams, aukojantiems Šilojuje. 
\par 15 Prieš sudeginant taukus, atėjęs kunigo tarnas sakydavo aukotojui: “Duok mėsos kunigui iškepti; jis nenori virtos, bet žalios”. 
\par 16 Jei aukotojas sakydavo: “Pirmiausia tegul sudegina taukus, paskui tegul ima, kiek nori”. Tarnas atsakydavo: “Duok dabar. Jei neduosi, atimsiu”. 
\par 17 Elio sūnų nuodėmė buvo labai didelė Viešpaties akyse, nes žmonės ėmė bjaurėtis aukomis Viešpačiui. 
\par 18 Samuelis tarnavo Viešpačiui, apsirengęs lininį efodą. 
\par 19 Jo motina jam pasiūdavo drabužį ir kasmet atnešdavo ateidama su savo vyru aukoti kasmetinės aukos. 
\par 20 Elis laimindavo Elkaną ir jo žmoną: “Viešpats teduoda tau dar vaikų su šita moterimi už tą, kurį ji paaukojo Viešpačiui”. Po to jie grįždavo į namus. 
\par 21 Viešpats aplankė Oną, ir ji pagimdė dar tris sūnus ir dvi dukteris. Berniukas Samuelis augo Viešpaties akivaizdoje. 
\par 22 Elis labai paseno. Jis girdėjo, ką jo sūnūs darė Izraeliui, kaip jie suguldavo su moterimis, kurios susirinkdavo prie Susitikimo palapinės. 
\par 23 Tėvas klausė: “Kodėl taip darote? Aš girdžiu apie jūsų piktus darbus iš žmonių. 
\par 24 Negerus dalykus aš girdžiu apie jus, mano sūnūs. Jūs vedate Viešpaties tautą į nusikaltimą. 
\par 25 Jei žmogus nusikalsta žmogui, jį teis teisėjas, bet jei žmogus nusideda Viešpačiui, kas jį užtars?” Jie neklausė savo tėvo, todėl Viešpats nusprendė juos nužudyti. 
\par 26 Berniukas Samuelis augo, ir jam palankiai buvo nusiteikę ir Viešpats, ir žmonės. 
\par 27 Dievo žmogus atėjo pas Elį ir jam tarė: “Taip sako Viešpats: ‘Aš apsireiškiau tavo tėvams, kai jie buvo Egipte ir tarnavo faraonui. 
\par 28 Aš tavo tėvą išsirinkau kunigu iš visų Izraelio giminių, kad jis aukotų ant mano aukuro, smilkytų smilkalus ir nešiotų efodą mano akivaizdoje. Aš paskyriau kunigams izraelitų deginamųjų aukų dalį. 
\par 29 Kodėl jūs mindžiojate kojomis aukas ir dovanas, kurias įsakiau man aukoti? Kodėl labiau už mane gerbi savo sūnus, kurie tunka mano tautos Izraelio geriausiomis aukų dalimis?’ 
\par 30 Todėl Viešpats, Izraelio Dievas, sako: ‘Tikrai buvau pažadėjęs, kad tavo tėvo ir tavo namai per amžius bus mano akivaizdoje, bet dabar sakau, kad taip nebus. Aš pagerbsiu tuos, kurie mane gerbia, o kurie mane niekina, bus paniekinti. 
\par 31 Ateina diena, kada tavo ir tavo namų ramstį sunaikinsiu; nė vienas nesulauks senatvės. 
\par 32 Tu matysi priešą mano namuose, nepaisant to, kad Dievas daro gera Izraeliui; tavo namuose niekas nesulauks senatvės per amžius. 
\par 33 Aš nesunaikinsiu ir nepašalinsiu nuo savo aukuro visų tavųjų, bet tavo akys aptems ir siela krimsis; ir visi tavo palikuonys mirs savo gražiausiame amžiuje. 
\par 34 Tai bus ženklas tau, kai abu tavo sūnūs Hofnis ir Finehasas mirs tą pačią dieną. 
\par 35 Aš pakelsiu sau ištikimą kunigą, kuris darys tai, kas yra mano širdyje ir mano mintyse. Aš jam pastatysiu tvirtus namus, ir jis vaikščios prieš mano pateptąjį per amžius. 
\par 36 Tada išlikę tavo šeimos nariai atėję parpuls prieš jį už sidabrinį pinigėlį ar duonos kąsnį ir prašys, kad paskirtų juos kunigais ir jie galėtų užsidirbti pavalgyti’ ”.



\chapter{3}


\par 1 Berniukas Samuelis tarnavo Viešpačiui Elio priežiūroje. Viešpaties žodis buvo brangus tuo laiku ir regėjimai buvo reti. 
\par 2 Kartą Elis gulėjo. Jo akys buvo nusilpę ir jis nebegalėjo gerai matyti. 
\par 3 Šventyklos lempa dar nebuvo užgesusi. Samuelis miegojo Viešpaties šventykloje, kur buvo Dievo skrynia. 
\par 4 Viešpats pašaukė Samuelį. Tas atsiliepė: “Aš čia”. 
\par 5 Nubėgęs pas Elį, Samuelis tarė: “Aš čia. Tu mane šaukei”. Tas atsakė: “Aš tavęs nešaukiau, eik ir miegok”. Jis grįžo ir atsigulė. 
\par 6 Viešpats antrą kartą pašaukė Samuelį. Atsikėlęs Samuelis vėl nuėjo pas Elį ir tarė: “Aš čia. Tu mane šaukei”. Tas atsakė: “Aš nešaukiau tavęs, mano sūnau, eik ir miegok”. 
\par 7 Samuelis dar nepažino Viešpaties ir jam dar nebuvo apreikštas Viešpaties žodis. 
\par 8 Viešpats pašaukė Samuelį trečią kartą. Atsikėlęs jis vėl nuėjo pas Elį ir sakė: “Aš čia. Tu mane šaukei”. Elis suprato, kad Viešpats šaukia berniuką, 
\par 9 ir tarė Samueliui: “Eik ir atsigulk. Jei tave vėl šauks, sakyk: ‘Kalbėk, Viešpatie, Tavo tarnas klauso’ ”. Nuėjęs Samuelis atsigulė. 
\par 10 Viešpats atėjo, atsistojo ir pašaukė, kaip pirma: “Samueli, Samueli!” Samuelis atsiliepė: “Kalbėk, Tavo tarnas klauso”. 
\par 11 Viešpats tarė Samueliui: “Štai Aš darau darbą Izraelyje. Kas apie jį išgirs, tam suspengs abiejose ausyse. 
\par 12 Tą dieną įvykdysiu visa, ką kalbėjau prieš Elį ir jo giminę. Kai Aš pradedu, Aš ir pabaigsiu. 
\par 13 Aš paskelbiau jam, kad nubausiu jo namus už nusikaltimą, apie kurį jis žino, nes jo sūnūs bjauriai elgėsi, o jis jų nesudraudė. 
\par 14 Aš prisiekiau, kad Elio šeimos kaltė nebus nuplauta nei aukomis, nei dovanomis per amžius”. 
\par 15 Samuelis gulėjo iki ryto. Atsikėlęs atidarė šventyklos duris. Jis bijojo papasakoti Eliui tą regėjimą. 
\par 16 Elis pasišaukė Samuelį: “Samueli, mano sūnau!” Jis atsakė: “Aš čia”. 
\par 17 Jis paklausė: “Ką tau sakė? Neslėpk nuo manęs. Dievas tepadaro tau tai ir dar daugiau, jei paslėpsi nuo manęs nors vieną žodį iš to, ką Jis tau sakė”. 
\par 18 Ir Samuelis jam papasakojo viską, nieko neslėpdamas. Tada Elis tarė: “Jis­Viešpats. Tedaro, kaip Jam atrodo geriausia”. 
\par 19 Samuelis augo, ir Viešpats buvo su juo. Ir nė vienas jo žodis nelikdavo neišsipildęs. 
\par 20 Visas Izraelis nuo Dano ir Beer Šebos sužinojo, kad Samuelis yra Viešpaties paskirtas pranašas. 
\par 21 Viešpats ir toliau pasirodydavo Šilojuje, kur apsireikšdavo Samueliui savo žodžiu.



\chapter{4}

\par 1 Samuelio žodis pasiekė visą Izraelį. Izraelis pasistatė stovyklą prie Eben Ezero ir ruošėsi kariauti su filistinais, o filistinai pasistatė stovyklą prie Afeko. 
\par 2 Jie išsirikiavo prieš Izraelį, puolė ir sumušė jį, nužudydami mūšyje apie keturis tūkstančius vyrų. 
\par 3 Kai žmonės sugrįžo į stovyklą, Izraelio vyresnieji tarė: “Kodėl Viešpats leido šiandien filistinams nugalėti? Atsigabenkime iš Šilojo Viešpaties Sandoros skrynią, kad ji būtų su mumis ir mus išgelbėtų nuo priešų”. 
\par 4 Buvo pasiųsti į Šiloją pasiuntiniai, kurie atgabeno kareivijų Viešpaties, gyvenančio tarp cherubų, Sandoros skrynią. Abu Elio sūnūs Hofnis ir Finehasas buvo prie Dievo Sandoros skrynios. 
\par 5 Kai Viešpaties Sandoros skrynia pasiekė stovyklą, visi izraelitai šaukė taip garsiai, kad net žemė drebėjo. 
\par 6 Filistinai, išgirdę šauksmą, klausė: “Ką reiškia šitas riksmas hebrajų stovykloje?” Ir jie sužinojo, kad buvo atgabenta Viešpaties skrynia. 
\par 7 Filistinai išsigando, nes sakė: “Dievas atėjo pas juos į stovyklą. Vargas mums, nes anksčiau taip nebuvo! 
\par 8 Vargas mums! Kas mus išgelbės iš šitų galingų dievų? Tai dievai, kurie baudė Egiptą įvairiomis bausmėmis dykumoje. 
\par 9 Filistinai, būkite drąsūs, kad nereikėtų tarnauti hebrajams, kaip jie mums tarnavo. Nusiraminkite ir kariaukite”. 
\par 10 Filistinai kovojo ir vėl nugalėjo izraelitus, kurie visi išbėgiojo į savo palapines. Įvyko didelės žudynės, ir krito trisdešimt tūkstančių pėstininkų iš Izraelio. 
\par 11 Dievo skrynia buvo priešų paimta, o Elio sūnūs Hofnis ir Finehasas žuvo. 
\par 12 Vienas benjaminas, pabėgęs iš mūšio lauko, tą pačią dieną atvyko į Šiloją su perplėštais drabužiais ir žemėm apibarstyta galva. 
\par 13 Jam atvykus į miestą, Elis sėdėjo krasėje šalia kelio ir jo širdis drebėjo dėl Dievo skrynios. Kai vyras atbėgo į miestą ir viską pranešė, visas miestas ėmė raudoti. 
\par 14 Elis, išgirdęs verksmą, paklausė: “Ką reiškia šitas triukšmas?” Tas vyras atskubėjo prie Elio ir jam pranešė. 
\par 15 Elis buvo devyniasdešimt aštuonerių metų, jo akys buvo nusilpusios ir jis nebematė. 
\par 16 Tas vyras kalbėjo Eliui: “Aš šiandien pabėgau iš mūšio lauko ir atvykau čia”. Elis paklausė: “Kas ten atsitiko, mano sūnau?” 
\par 17 Pasiuntinys atsakė: “Izraelis bėgo nuo filistinų, įvyko didelės žudynės. Tavo abu sūnūs Hofnis ir Finehasas žuvo, ir Dievo skrynia priešų rankose”. 
\par 18 Jam paminėjus Dievo skrynią, Elis atbulas krito nuo krasės šalia vartų. Jo sprandas lūžo ir jis mirė, nes buvo senas ir sunkus. Jis teisė Izraelį keturiasdešimt metų. 
\par 19 Jo marti, Finehaso žmona, buvo nėščia ir greitai turėjo gimdyti. Išgirdus, kad Dievo skrynia paimta ir kad jos uošvis ir vyras mirę, ji susilenkė ir pagimdė, nes ją suėmė gimdymo skausmai. 
\par 20 Jai mirštant, prie jos stovėjusios moterys sakė: “Nebijok, tu pagimdei sūnų”. Tačiau ji nieko neatsakė ir nebekreipė dėmesio. 
\par 21 Ji pavadino vaiką Ikabodu, sakydama: “Šlovė paliko Izraelį”, nes Dievo skrynia buvo paimta ir jos uošvis bei vyras mirę. 
\par 22 Ji pasakė: “Šlovė paliko Izraelį, nes Dievo skrynia paimta”.



\chapter{5}


\par 1 Filistinai paėmė Dievo skrynią ir nugabeno ją iš Eben Ezero į Ašdodą. 
\par 2 Filistinai paėmė Dievo skrynią, įnešė ją į Dagono šventyklą ir pastatė šalia Dagono. 
\par 3 Ašdodo gyventojai kitą rytą, atsikėlę anksti, įėjo į Dagono šventyklą ir pamatė Dagoną, kniūbsčią gulintį ant žemės prieš Viešpaties skrynią. Pakėlę Dagoną, jie pastatė jį atgal į jo vietą. 
\par 4 Kitą dieną anksti atsikėlę, jie vėl rado Dagoną gulintį ant žemės prieš Viešpaties skrynią; jo galva ir abi rankos gulėjo nukirstos ant slenksčio. Tik Dagono liemuo buvo likęs. 
\par 5 Todėl Dagono kunigai ir visi, įeinantieji į Dagono šventyklą, peržengia Dagono slenkstį Ašdode iki šios dienos. 
\par 6 Viešpaties ranka sunkiai slėgė Ašdodo gyventojus. Jis naikino juos ir baudė skaudžiais augliais Ašdode ir jo apylinkėse. 
\par 7 Ašdodo vyrai, matydami, kas darosi, tarė: “Izraelio Dievo skrynia negali pasilikti pas mus, nes Jo ranka labai spaudžia mus ir mūsų dievą Dagoną”. 
\par 8 Jie sukvietė visus filistinų kunigaikščius ir klausė: “Ką mums daryti su Izraelio Dievo skrynia?” Tie atsakė: “Izraelio Dievo skrynią reikia nugabenti į Gatą”. Jie taip ir padarė. 
\par 9 Kai ją atgabeno, Viešpaties ranka ištiko miestą labai dideliu naikinimu. Jis baudė visus to miesto žmones, mažus ir didelius, skaudžiais augliais. 
\par 10 Tuomet jie išsiuntė Dievo skrynią į Ekroną. Kai Dievo skrynia buvo atgabenta į Ekroną, ekroniečiai ėmė šaukti: “Jie atgabeno pas mus Izraelio Dievo skrynią, norėdami nužudyti mus”. 
\par 11 Taigi jie, sušaukę visus filistinų kunigaikščius, sakė: “Išsiųskime Izraelio Dievo skrynią, kad ji būtų sugrąžinta į savo vietą ir kad neišžudytų mūsų ir mūsų žmonių”. Nes pražūtingas naikinimas apėmė visą miestą ir Dievo ranka buvo labai sunki. 
\par 12 Išlikę gyvi buvo ištikti auglių, ir miesto šauksmas kilo į dangų.



\chapter{6}

\par 1 Viešpaties skrynia filistinų krašte išbuvo septynis mėnesius. 
\par 2 Filistinai pasikvietė kunigus bei žynius ir klausė: “Ką mums daryti su Viešpaties skrynia? Kaip ją pasiųsti atgal į jos vietą?” 
\par 3 Tie atsakė: “Jei siųsite atgal Izraelio Dievo skrynią, nesiųskite tuščios, bet būtinai grąžinkite su auka už kaltes. Tada pagysite ir jums paaiškės, kodėl Jo ranka neatsitraukia nuo jūsų”. 
\par 4 Jie klausė: “Kokią auką turime duoti už savo nusikaltimą?” Tie atsakė: “Penkis auksinius auglius ir penkias auksines peles pagal filistinų kunigaikščių skaičių, nes ta pati liga vargino jus ir jūsų kunigaikščius. 
\par 5 Padarykite penkis auglių atvaizdus ir pelių, kurios naikina kraštą, atvaizdus ir atiduokite Izraelio Dievui. Gal tada Jis atims savo ranką nuo jūsų, jūsų dievų ir šalies. 
\par 6 Neužkietinkite savo širdžių, kaip užkietino egiptiečiai ir faraonas. Kai Viešpats darė stebuklus tarp jų, jie buvo priversti išleisti izraelitus. 
\par 7 Imkite dvi žindančias karves, ant kurių dar nebuvo uždėtas jungas, ir įkinkykite jas į naują vežimą, o jų veršiukus parsiveskite namo. 
\par 8 Viešpaties skrynią įkelkite į vežimą, o auksinius dirbinius, kuriuos duodate kaip auką, įdėkite į atskirą dėžę ir paleiskite karves. 
\par 9 Ir žiūrėkite: jei jos eis keliu, vedančiu Bet Šemešo link, tai Viešpats yra mums siuntęs šitą didelę nelaimę; o jei ne, tada žinosime, kad ne Jo ranka ištiko mus, o taip įvyko atsitiktinai”. 
\par 10 Filistinai taip ir padarė. Dvi žindančias karves įkinkė į vežimą, o jų veršiukus uždarė namuose. 
\par 11 Jie įkėlė Viešpaties skrynią ir dėžę su auksinėmis pelėmis ir auglių atvaizdais į vežimą. 
\par 12 Karvės ėjo tiesiai į Bet Šemešą. Eidamos vieškeliu jos baubė, bet nepasuko nei dešinėn, nei kairėn. Filistinų kunigaikščiai ėjo paskui jas iki Bet Šemešo sienos. 
\par 13 Bet šemešiečiai pjovė kviečius slėnyje. Pakėlę akis, jie pamatė skrynią ir džiaugėsi ja. 
\par 14 Vežimas atvažiavo į Jozuės, bet šemešiečio, lauką ir sustojo. Ten buvo didelis akmuo. Jie suskaldė vežimą, o karves aukojo kaip deginamąją auką Viešpačiui. 
\par 15 Levitai nukėlė Viešpaties skrynią ir prie jos buvusią dėžę, kurioje buvo auksiniai dirbiniai, ir padėjo ant to didelio akmens. Bet Šemešo žmonės tą dieną aukojo deginamąsias ir kitas aukas Viešpačiui. 
\par 16 Penki filistinų kunigaikščiai, tai matydami, sugrįžo tą pačią dieną į Ekroną. 
\par 17 Filistinai aukojo už savo nusikaltimą auksinius auglius: vieną už Ašdodą, vieną už Gazą, vieną už Aškeloną, vieną už Gatą ir vieną už Ekroną. 
\par 18 Auksinės pelės buvo nuo penkių filistinų miestų, priklausančių penkiems kunigaikščiams, pradedant sutvirtintais miestais ir baigiant atvirais kaimais iki didelio akmens, ant kurio jie padėjo Viešpaties skrynią ir kuris iki šios dienos tebeguli bet šemešiečio Jozuės lauke. 
\par 19 Jis baudė Bet Šemešo gyventojus, nes jie pažiūrėjo į Viešpaties skrynios vidų, ir nužudė penkiasdešimt tūkstančių septyniasdešimt žmonių. Žmonės verkė dėl tokių didelių žudynių. 
\par 20 Bet Šemešo žmonės kalbėjo: “Kas gali išstovėti Viešpaties, švento Dievo, akivaizdoje? Kur Jis eis iš mūsų?” 
\par 21 Jie pasiuntė pasiuntinius į Kirjat Jearimą, sakydami: “Filistinai sugrąžino Viešpaties skrynią. Ateikite ir parsigabenkite ją”.



\chapter{7}

\par 1 Kirjat Jearimo vyrai parsigabeno Viešpaties skrynią ir ją įnešė į Abinadabo namus, esančius ant kalvos. Jo sūnų Eleazarą jie paskyrė saugoti Viešpaties skrynią. 
\par 2 Praėjo dvidešimt metų nuo tos dienos, kai Viešpaties skrynia buvo atgabenta į Kirjat Jearimą. Visas Izraelis atsigręžė į Viešpatį. 
\par 3 Ir Samuelis tarė visiems izraelitams: “Jei jūs visa širdimi norite grįžti prie Viešpaties, pašalinkite iš savo tarpo svetimus dievus bei Astartę, paruoškite savo širdis Viešpačiui ir tarnaukite Jam vienam; tada Jis jus išgelbės iš filistinų”. 
\par 4 Izraelitai pašalino Baalą bei Astartę ir tarnavo vienam Viešpačiui. 
\par 5 Po to Samuelis sakė: “Sušaukite visą Izraelį į Micpą; aš melsiuosi už jus Viešpačiui”. 
\par 6 Jie, susirinkę Micpoje, sėmė vandenį ir jį išliejo Viešpaties akivaizdoje, pasninkavo visą dieną ir sakė: “Mes nusidėjome Viešpačiui”. Samuelis teisė izraelitus Micpoje. 
\par 7 Filistinai išgirdo, kad izraelitai susirinko į Micpą, ir jų kunigaikščiai pakilo prieš Izraelį. Izraelitai, tai išgirdę, nusigando 
\par 8 ir tarė Samueliui: “Nepaliauk melsti Viešpatį, mūsų Dievą, kad Jis mus išgelbėtų iš filistinų”. 
\par 9 Samuelis paėmė žindomą ėriuką ir paaukojo visą Viešpačiui kaip deginamąją auką. Samuelis šaukėsi Viešpaties dėl Izraelio, ir Viešpats jį išklausė. 
\par 10 Samueliui aukojant deginamąją auką, filistinai pradėjo kovą su Izraeliu. Bet tą dieną Viešpats išgąsdino filistinus didele perkūnija taip, kad jie pakriko, ir izraelitai nugalėjo juos. 
\par 11 Išėję iš Micpos, izraelitai persekiojo filistinus ir juos žudė iki Bet Karo. 
\par 12 Tarp Micpos ir Seno Samuelis pastatė akmenį ir jį pavadino Eben Ezeriu, sakydamas: “Viešpats mums padėjo iki šios vietos”. 
\par 13 Filistinai buvo nugalėti ir daugiau nepuolė Izraelio; ir Viešpaties ranka buvo prieš filistinus per visas Samuelio dienas. 
\par 14 Miestai nuo Ekrono iki Gato, kuriuos filistinai buvo užėmę, grįžo Izraeliui. Taip pat ir jų apylinkes Izraelis atsiėmė iš filistinų. Buvo taika tarp Izraelio ir amoritų. 
\par 15 Samuelis teisė Izraelį per visas savo dienas. 
\par 16 Jis eidavo kas metai į Betelį, Gilgalą bei Micpą ir teisdavo Izraelį tose vietose. 
\par 17 Po to jis grįždavo į Ramą, nes ten buvo jo namai. Čia jis teisė Izraelį ir čia jis pastatė aukurą Viešpačiui.



\chapter{8}

\par 1 Kai Samuelis paseno, paskyrė teisėjais Izraelyje savo sūnus. 
\par 2 Jo pirmagimis sūnus buvo Joelis, o antrasis­Abija. Jie buvo teisėjai Beer Šeboje. 
\par 3 Tačiau jo sūnūs nevaikščiojo jo keliais, bet pasidavė godumui, imdavo kyšius ir iškreipdavo teisingumą. 
\par 4 Visi Izraelio vyresnieji susirinkę atėjo pas Samuelį į Ramą 
\par 5 ir sakė jam: “Tu pasenai, o tavo sūnūs nevaikšto tavo keliais. Paskirk mums karalių, kuris mus teistų, kaip yra visose tautose”. 
\par 6 Samueliui tokia kalba nepatiko, nes jie sakė: “Duok mums karalių, kuris mus teistų”. Ir Samuelis meldėsi Viešpačiui. 
\par 7 O Viešpats jam atsakė: “Klausyk tautos balso visame, ką jie tau sako. Juk ne tave jie atmetė, bet mane, kad jiems nekaraliaučiau. 
\par 8 Taip jie elgėsi nuo tos dienos, kai juos išvedžiau iš Egipto, iki šios dienos. Jie palikdavo mane ir tarnaudavo kitiems dievams. Dabar jie ir tau daro tą patį. 
\par 9 Klausyk jų balso, tačiau iš anksto rimtai juos įspėk ir jiems paskelbk teises karaliaus, kuris jiems karaliaus”. 
\par 10 Žmonėms, kurie prašė karaliaus, Samuelis pasakė visa, ką Viešpats kalbėjo. 
\par 11 Ir jis sakė: “Karalius, kuris jums karaliaus, paims jūsų sūnus ir paskirs prie savo kovos vežimų ir žirgų, ir jie turės bėgti jo vežimų priekyje. 
\par 12 Jis paskirs juos tūkstantininkais ir penkiasdešimtininkais; jie ars jo laukus, pjaus javus ir gamins ginklus bei reikmenis karo vežimams. 
\par 13 Jūsų dukterys gamins jam kvepalus ir bus virėjos bei kepėjos. 
\par 14 Geriausius jūsų laukus, vynuogynus ir alyvmedžių sodus jis atiduos savo tarnams. 
\par 15 Be to, jis ims jūsų pasėlių ir vynuogynų vaisių dešimtąją dalį ir atiduos savo valdininkams bei tarnams. 
\par 16 Jūsų tarnus ir tarnaites, geriausius jaunuolius ir jūsų asilus panaudos savo darbams. 
\par 17 Jis paims jūsų avių dešimtąją dalį, ir jūs būsite jo tarnai. 
\par 18 Tada jūs šauksite dėl karaliaus, kurį išsirinkote, bet Viešpats jūsų neišklausys”. 
\par 19 Tačiau tauta atsisakė paklusti Samuelio balsui ir sakė: “Mes norime turėti karalių, 
\par 20 kad būtume kaip visos kitos tautos, kad mūsų karalius mus teistų ir mums vadovautų kovose”. 
\par 21 Samuelis išklausė tautos žodžius ir perdavė juos Viešpačiui. 
\par 22 Ir Viešpats tarė Samueliui: “Klausyk jų balso ir paskirk jiems karalių”. Tada Samuelis tarė izraelitams: “Grįžkite kiekvienas į savo miestą”.



\chapter{9}

\par 1 Benjamine gyveno turtingas vyras, vardu Kišas, sūnus Abielio, sūnaus Ceroro, sūnaus Bekorato, sūnaus Afiacho. 
\par 2 Kišas turėjo sūnų Saulių, jauną ir gražų. Nebuvo nė vieno izraelito, gražesnio už jį; visa galva jis buvo aukštesnis už kitus savo tautos žmones. 
\par 3 Sauliaus tėvui Kišui dingo asilės. Ir Kišas sakė savo sūnui Sauliui: “Imk tarną ir eik ieškoti asilių”. 
\par 4 Jie išvaikščiojo Efraimo aukštumas ir Šališos kraštą, bet jų nerado. Paskui jie apėjo Šaalimo kraštą, bet ir ten jų nebuvo. Jie išvaikščiojo ir Benjamino žemes, bet jų nerado. 
\par 5 Kai atėjo į Cūfo šalį, Saulius tarė savo tarnui: “Grįžkime, kad mano tėvas, užuot rūpinęsis asilėmis, nepradėtų rūpintis dėl mūsų”. 
\par 6 Tarnas jam atsakė: “Šitame mieste gyvena Dievo vyras, kurį visi gerbia; visa, ką jis pasako, įvyksta. Užeikime pas jį. Gal jis pasakys, kuriuo keliu mums eiti”. 
\par 7 Saulius atsakė: “Ką mes nunešime tam vyrui eidami? Duona pasibaigė mūsų maišuose, ir mes neturime jokios dovanos, kurią galėtume nunešti Dievo vyrui”. 
\par 8 Tarnas atsakė Sauliui: “Aš turiu ketvirtį šekelio sidabro ir jį atiduosiu Dievo vyrui, kad jis parodytų mums kelią”. 
\par 9 Anksčiau, kai Izraelyje kas eidavo pasiklausti Dievo, sakydavo: “Einu pas regėtoją”. Nes tą, kurį šiandien vadina pranašu, anksčiau vadino regėtoju. 
\par 10 Saulius atsakė tarnui: “Gerai sakai, eikime”. Jie nuėjo į miestą, kur gyveno Dievo vyras. 
\par 11 Beeidami šlaitu miesto link, jie susitiko mergaičių, einančių pasisemti vandens, ir paklausė: “Ar yra čia regėtojas?” 
\par 12 Jos atsakė: “Yra. Jis kaip tik priešais jus. Skubėkite, nes jis šiandien atėjo į miestą, kadangi žmonės šiandien aukoja aukštumoje. 
\par 13 Įėję į miestą, jūs tuojau jį surasite, nes jis eis į aukštumą valgyti. Žmonės nepradės valgyti, kol jis ateis, nes jis palaimina auką; po to valgo pakviestieji. Eikite, nes dabar jį dar rasite”. 
\par 14 Įeidami į miestą, jie sutiko Samuelį, ateinantį priešais, kuris ėjo į aukštumą. 
\par 15 Viešpats buvo pasakęs Samueliui prieš dieną iki Sauliaus atėjimo: 
\par 16 “Rytoj apie tą laiką atsiųsiu pas tave vyrą iš Benjamino žemės, kad jį pateptum mano tautos Izraelio valdovu. Jis išgelbės mano tautą iš filistinų. Aš pažvelgiau į savo tautą, nes jų šauksmas pasiekė mane”. 
\par 17 Kai Samuelis pamatė Saulių, Viešpats jam tarė: “Štai vyras, apie kurį tau kalbėjau! Šitas karaliaus mano tautai”. 
\par 18 Saulius, sutikęs Samuelį tarpuvartėje, klausė: “Pasakyk man, kur gyvena regėtojas”. 
\par 19 Samuelis atsakė Sauliui: “Aš esu regėtojas. Eik pirma manęs į aukštumą, ten šiandien valgysite su manimi; rytoj tave išleisiu ir pasakysiu, kas yra tavo širdyje. 
\par 20 Dėl asilių, kurios dingo prieš tris dienas, nesirūpink­jos jau atsirado. O kam priklausys visa, kas geriausia Izraelyje? Argi ne tau ir tavo tėvo namams?” 
\par 21 Saulius atsakė: “Juk aš esu benjaminas, iš mažiausios Izraelio giminės, ir mano šeima yra menkiausia iš visų Benjamino šeimų. Kodėl man taip sakai?” 
\par 22 Samuelis Saulių ir jo tarną įsivedė į kambarį ir pasodino juos garbingiausioje vietoje tarp pakviestųjų, kurių buvo apie trisdešimt vyrų. 
\par 23 Samuelis tarė virėjui: “Duok tą dalį, kurią tau liepiau atidėti”. 
\par 24 Virėjas atnešė gyvulio petį ir padėjo prieš Saulių. Samuelis tarė: “Štai ką palikau tau, valgyk su pakviestaisiais”. Taip Saulius tą dieną valgė su Samueliu. 
\par 25 Jiems sugrįžus nuo aukštumos į miestą, Samuelis kalbėjosi su Sauliumi ant stogo. 
\par 26 Aušrai brėkštant, Samuelis pašaukė ant stogo Saulių ir tarė: “Kelkis, kad galėčiau tave išleisti”. Saulius atsikėlė ir juodu išėjo į gatvę. 
\par 27 Kai jie ėjo į miesto pakraštį, Samuelis tarė Sauliui: “Liepk tarnui eiti pirma mūsų, o tu sustok. Paskelbsiu tau Viešpaties žodį”.



\chapter{10}

\par 1 Samuelis išliejo indą aliejaus Sauliui ant galvos, pabučiavo jį ir tarė: “Viešpats patepė tave savo paveldėjimo vadu. 
\par 2 Šiandien, išsiskyręs su manimi, sutiksi du vyrus prie Rachelės kapo, Benjamino pasienyje, Celcache; jie tau sakys: ‘Asilės, kurių išėjai ieškoti, jau atsirado. Tavo tėvas, pamiršęs asiles, rūpinasi, kas su jumis atsitiko’. 
\par 3 Eidamas toliau, prieisi Taboro ąžuolą. Ten tave sutiks trys vyrai, einantys pas Dievą į Betelį. Vienas neš tris ožiukus, antras­tris duonos kepalus, o trečias­vyno odinę. 
\par 4 Jie tave pasveikins ir duos tau du duonos kepalus, kuriuos paimsi iš jų rankos. 
\par 5 Po to prieisi Dievo kalvą, kur yra filistinų įgulos stovykla. Įeidamas į miestą, sutiksi būrį pranašų, ateinančių nuo aukštumos su arfomis, būgnais, vamzdeliais ir psalteriais; ir jie pranašaus. 
\par 6 Ant tavęs nužengs Viešpaties Dvasia, ir tu pranašausi kartu su jais ir tapsi kitu žmogumi. 
\par 7 Kai visi tie ženklai įvyks su tavimi, elkis pagal aplinkybes, nes Dievas su tavimi. 
\par 8 Nueik pirma manęs į Gilgalą. Aš atėjęs aukosiu deginamąsias ir padėkos aukas. Lauk manęs septynias dienas, iki aš ateisiu ir nurodysiu, ką turi daryti”. 
\par 9 Sauliui pasisukus eiti nuo Samuelio, Dievas davė jam kitą širdį, ir visi tie ženklai įvyko tą pačią dieną. 
\par 10 Jam atėjus prie kalvos, jį pasitiko pranašų būrys. Ir Dievo Dvasia nužengė ant jo, ir jis pranašavo kartu su jais. 
\par 11 Žmonės, kurie Saulių pažino anksčiau, matydami jį su pranašais pranašaujant, kalbėjosi: “Kas atsitiko Kišo sūnui? Ar ir Saulius tarp pranašų?” 
\par 12 Vienas iš ten buvusiųjų atsakė: “O kas kitų tėvas?” Todėl tai tapo priežodžiu: “Ar ir Saulius tarp pranašų?” 
\par 13 Baigęs pranašauti, jis atėjo į aukštumą. 
\par 14 Jo dėdė klausė: “Kur buvote?” Jis atsakė: “Asilių ieškojome. Kai jų niekur neradome, nuėjome pas Samuelį”. 
\par 15 Sauliaus dėdė vėl klausė: “Prašau, pasakyk, ką jums Samuelis sakė?” 
\par 16 Saulius atsakė: “Jis mums pasakė, kad asilės atsirado”. Bet ką Samuelis jam sakė apie karalystę, Saulius jam nepasakė. 
\par 17 Samuelis sušaukė tautą į Micpą pas Viešpatį. 
\par 18 Ir jis kalbėjo izraelitams: “Taip sako Viešpats, Izraelio Dievas: ‘Aš išvedžiau Izraelį iš Egipto ir išgelbėjau iš egiptiečių rankos ir iš visų karalysčių, kurios jus spaudė, rankos’. 
\par 19 Bet jūs šiandien atmetėte Dievą, kuris jus išgelbėjo iš visų jūsų nelaimių, sakydami: ‘Paskirk mums karalių’. Dabar ateikite į Viešpaties akivaizdą giminėmis ir šeimomis”. 
\par 20 Kai Samuelis pašaukė prieiti kiekvieną Izraelio giminę, buvo išrinkta Benjamino giminė. 
\par 21 Kai Samuelis pašaukė Benjamino giminę šeimomis, buvo išrinkta Matrio šeima ir iš jos buvo išrinktas Kišo sūnus Saulius. Jie ieškojo jo, bet negalėjo surasti. 
\par 22 Todėl jie klausė Viešpaties: “Ar jis čia dar ateis?” Viešpats atsakė: “Jis pasislėpęs tarp mantos”. 
\par 23 Nubėgę vyrai atvedė jį. Kai jis atsistojo tarp žmonių, buvo visa galva aukštesnis už kitus. 
\par 24 Tada Samuelis tarė susirinkusiems: “Ar matote tą, kurį Viešpats išsirinko? Nėra jam lygaus visoje tautoje”. Tada žmonės pradėjo šaukti: “Tegyvuoja karalius!” 
\par 25 Tada Samuelis išdėstė tautai karalystės nuostatus, juos surašė į knygą ir padėjo priešais Viešpatį. Po to Samuelis leido tautai grįžti į savo namus. 
\par 26 Taip pat ir Saulius grįžo namo į Gibėją. Su juo nuėjo ir vyrai, kurių širdis palietė Dievas. 
\par 27 Bet Belialo vaikai sakė: “Kaip šitas žmogus mus išgelbės?” Jie paniekino jį ir neatnešė jam dovanų. Bet Saulius išlaikė ramybę.



\chapter{11}


\par 1 Amonitas Nahašas atžygiavo ir apsupo Jabeš Gileadą. Tada Jabešo gyventojai sakė Nahašui: “Sudarykime sandorą, ir mes tau tarnausime”. 
\par 2 Amonitas Nahašas jiems atsakė: “Aš sudarysiu su jumis sandorą su tokia sąlyga: kiekvienam iš jūsų išdursiu dešinę akį ir taip pažeminsiu visą Izraelį”. 
\par 3 Jabešo vyresnieji jam tarė: “Duok mums septynias dienas, kad galėtume pasiųsti pasiuntinius į visą Izraelio kraštą. Jei nerasime, kas mus išgelbėtų, mes tau pasiduosime”. 
\par 4 Kai pasiuntiniai atvyko į Gibėją, kur gyveno Saulius, ir pranešė šiuos žodžius žmonėms, visi žmonės pakėlė balsus ir verkė. 
\par 5 Tuo metu Saulius grįžo iš lauko su savo jaučiais. Jis paklausė: “Kas atsitiko, kad žmonės verkia?” Ir jie papasakojo jam žinias iš Jabešo. 
\par 6 Kai jis tai išgirdo, Dievo Dvasia nužengė ant Sauliaus ir jis užsidegė dideliu pykčiu. 
\par 7 Jis iškinkė jungą jaučių ir, juos sukapojęs, išsiuntinėjo gabalus po visą Izraelio kraštą, sakydamas: “Kas neis su Sauliumi ir Samueliu, taip bus padaryta jo jaučiams”. Viešpaties baimė apėmė tautą, ir jie atėjo visi kaip vienas. 
\par 8 Kai jis suskaičiavo juos Bezeke, izraelitų buvo trys šimtai tūkstančių, o Judo vyrų­trisdešimt tūkstančių. 
\par 9 Atvykusiems pasiuntiniams jie tarė: “Rytoj, kai saulė pradės kaitinti, jūs sulauksite pagalbos”. Pasiuntiniai sugrįžę pranešė apie tai Jabešo gyventojams, ir jie visi pradžiugo. 
\par 10 Jabešo vyrai sakė: “Rytoj pasiduosime jums, ir jūs galėsite daryti su mumis, ką norėsite”. 
\par 11 Kitą dieną Saulius suskirstė vyrus į tris būrius ir rytinės sargybos metu jie įsiveržė į stovyklą ir žudė amonitus, iki pradėjo kaitinti saulė. O likusieji buvo taip išblaškyti, kad dviejų neliko kartu. 
\par 12 Izraelitai klausė Samuelio: “Kas yra tie, kurie sakė: ‘Ar Saulius mums karalius?’ Atveskite juos mums, kad mes juos nužudytume”. 
\par 13 Saulius tarė: “Nė vienas nebus nužudytas šiandien, nes šiandien Viešpats išgelbėjo Izraelį”. 
\par 14 Po to Samuelis tarė tautai: “Eikime į Gilgalą ir ten atnaujinkime karalystę”. 
\par 15 Visa tauta nuėjo į Gilgalą. Jie ten paskelbė Saulių karaliumi Viešpaties akivaizdoje ir aukojo padėkos aukas Viešpačiui. Ir labai džiaugėsi ten Saulius ir visi izraelitai.



\chapter{12}


\par 1 Samuelis kalbėjo visiems izraelitams: “Aš paklausiau jūsų balso visame, ką man sakėte, ir paskyriau jums karalių. 
\par 2 Ir dabar karalius eina priekyje jūsų. Aš pasenau ir pražilau, ir mano sūnūs yra tarp jūsų. Aš gyvenau tarp jūsų nuo savo vaikystės iki šios dienos. 
\par 3 Štai aš čia. Paliudykite prieš mane Viešpaties ir Jo pateptojo akivaizdoje. Ar aš paėmiau kieno nors jautį ar asilą? Ar ką nuskriaudžiau? Ar ką išnaudojau? Iš ko paėmiau kyšį, kad užmerkčiau savo akis? Jei taip padariau, atsilyginsiu”. 
\par 4 Jie atsakė: “Tu mūsų neskriaudei, mūsų neišnaudojai ir nieko iš mūsų neėmei”. 
\par 5 Samuelis jiems tarė: “Viešpats ir Jo pateptasis yra šiandien liudytojai, kad jūs neradote nieko mano rankoje”. Jie atsakė: “Liudytojai”. 
\par 6 Samuelis toliau kalbėjo tautai: “Liudytojas yra Viešpats, kuris paskyrė Mozę bei Aaroną ir išvedė jūsų tėvus iš Egipto šalies. 
\par 7 Dabar sustokite ir klausykite, o aš bylinėsiuosi su jumis dėl visų Viešpaties teisių darbų, kuriuos Jis padarė jums ir jūsų tėvams. 
\par 8 Kai Jokūbas nuvyko į Egiptą ir jūsų tėvai šaukėsi Viešpaties, Viešpats siuntė Mozę ir Aaroną, kurie išvedė jūsų tėvus iš Egipto ir apgyvendino šitoje šalyje. 
\par 9 Kai jie užmiršo Viešpatį, savo Dievą, Jis atidavė juos į Siseros, Hacoro kariuomenės vado, rankas, į rankas filistinų ir Moabo karaliaus ir jie kariavo prieš jus. 
\par 10 Jie šaukėsi Viešpaties, sakydami: ‘Mes nusidėjome, nes apleidome Viešpatį ir tarnavome Baalui ir Astartei. Išgelbėk mus iš mūsų priešų rankos, ir mes Tau tarnausime’. 
\par 11 Viešpats siuntė Jerubaalą, Baraką, Jeftę ir Samuelį ir išgelbėjo jus iš jūsų priešų rankos, kurie supo jus, ir jūs gyvenote saugiai. 
\par 12 Pamatę amonitų karalių Nahašą atžygiuojantį prieš jus, jūs sakėte: ‘Karalius mums tekaraliauja’, nors Viešpats, jūsų Dievas, yra jūsų karalius. 
\par 13 Todėl dabar štai jūsų karalius, kurio reikalavote ir išsirinkote; Viešpats davė jums karalių. 
\par 14 Jei bijosite Viešpaties, Jam tarnausite, klausysite Jo balso ir nesipriešinsite Jo įsakymams, tai jūs ir jūsų karalius, kuris jus valdo, seksite Viešpatį, savo Dievą. 
\par 15 Bet jei neklausysite Viešpaties balso ir priešinsitės Jo įsakymams, tai Viešpaties ranka bus prieš jus, kaip buvo prieš jūsų tėvus. 
\par 16 Stovėkite ir žiūrėkite į tą didelį dalyką, kurį Viešpats darys jums matant. 
\par 17 Argi šiandien ne kviečių pjūtis? Aš šauksiuosi Viešpaties, ir Jis siųs perkūniją ir lietų, kad jūs suprastumėte ir pasimokytumėte, kokį didelį nusikaltimą padarėte Viešpaties akyse, prašydami sau karaliaus”. 
\par 18 Samuelis šaukėsi Viešpaties, ir Viešpats tą dieną siuntė perkūniją ir lietų. Visi izraelitai labai išsigando Viešpaties ir Samuelio. 
\par 19 Ir jie sakė Samueliui: “Melsk už savo tarnus Viešpatį, savo Dievą, kad nemirtume, nes prie visų savo nusikaltimų pridėjome dar vieną, prašydami sau karaliaus!” 
\par 20 Samuelis atsakė tautai: “Nebijokite. Tiesa, jūs nusikaltote, tačiau nepaliaukite sekę Viešpatį ir tarnaukite Jam visa širdimi. 
\par 21 Nenukrypkite prie tuščių dalykų, kurie negali duoti naudos nė išgelbėti, nes yra tušti. 
\par 22 Viešpats neatstums jūsų dėl savo didžio vardo, nes Viešpačiui patiko išsirinkti jus savo tauta. 
\par 23 O dėl manęs, tai taip nebus, kad aš nusidėčiau prieš Viešpatį, paliaudamas melstis už jus. Aš jus mokysiu eiti teisingu ir geru keliu. 
\par 24 Tik bijokite Viešpaties ir Jam ištikimai tarnaukite visa savo širdimi. Jūs matėte, kokių didelių dalykų Jis padarė dėl jūsų. 
\par 25 Bet jei elgsitės nedorai, pražūsite jūs ir jūsų karalius”.



\chapter{13}

\par 1 Saulius karaliavo Izraelyje metus ir antraisiais karaliavimo metais 
\par 2 Saulius išsirinko tris tūkstančius vyrų iš Izraelio; du tūkstančiai buvo su juo Michmašo ir Betelio kalnuose, o tūkstantis buvo su Jehonatanu Benjamino Gibėjoje. Likusius vyrus jis pasiuntė namo. 
\par 3 Jehonatanas sumušė filistinų įgulą Geboje, ir apie tai išgirdo visi filistinai. Saulius trimitavo visoje šalyje, sakydamas: “Teišgirsta hebrajai”. 
\par 4 Visas Izraelis išgirdo, kad Saulius sumušė filistinų įgulą ir kad Izraelis tapo pasibjaurėjimu filistinams. Visa tauta buvo pašaukta pas Saulių į Gilgalą. 
\par 5 Filistinai susirinko kovoti prieš Izraelį. Jų buvo trisdešimt tūkstančių kovos vežimų, šeši tūkstančiai raitelių ir pėstininkų kaip smilčių ant jūros kranto. Jie atžygiavo ir pasistatė stovyklą Michmaše, į rytus nuo Bet Aveno. 
\par 6 Izraelitai pamatė, kad jiems gresia pavojus; priešo spaudžiami, žmonės slapstėsi olose, urvuose, tarp uolų, aukštumose ir daubose. 
\par 7 Kai kurie perėjo Jordaną ir pabėgo į Gado ir Gileado šalį. Tačiau Saulius pasiliko Gilgale, ir visi žmonės drebėdami sekė jį. 
\par 8 Jis laukė septynias dienas, kaip Samuelis buvo paskyręs. Samueliui nepasirodžius Gilgale, žmonės pradėjo skirstytis. 
\par 9 Saulius įsakė atnešti deginamąją ir padėkos aukas, ir jis aukojo deginamąją auką. 
\par 10 Jam baigus aukoti deginamąją auką, pasirodė Samuelis. Saulius išėjo jį sutikti ir pasveikinti. 
\par 11 Samuelis klausė: “Ką tu padarei?” Saulius atsakė: “Pamačiau, kad mano žmonės pradėjo skirstytis, o tu neatvykai skirtu laiku; ir filistinai susirinko Michmaše. 
\par 12 Aš galvojau: ‘Filistinai ateis į Gilgalą, o aš dar nebūsiu maldavęs Viešpaties’. Aš įsidrąsinau ir aukojau deginamąją auką”. 
\par 13 Samuelis tarė Sauliui: “Tu neprotingai pasielgei, nesilaikydamas Viešpaties, savo Dievo, įsakymo. Viešpats būtų patvirtinęs tavo karaliavimą Izraelyje amžiams. 
\par 14 Dabar tavo karalystė nebus ilga. Viešpats surado vyrą pagal savo širdį ir paskyrė jį vadu savo tautai, nes tu nesilaikei Viešpaties įsakymo”. 
\par 15 Po to Samuelis grįžo iš Gilgalo į Benjamino Gibėją. Saulius suskaičiavo žmones, esančius su juo, kurių buvo apie šešis šimtus vyrų. 
\par 16 Saulius, jo sūnus Jehonatanas ir su jais buvusieji vyrai apsistojo Benjamino Gibėjoje, o filistinai pasistatė stovyklą Michmaše. 
\par 17 Iš filistinų stovyklos išėjo trys būriai naikinti krašto. Vienas būrys pasuko Ofros kryptimi, į Šualo šalį, 
\par 18 antras patraukė Bet Horono link, o trečias­į aukštumas, esančias prie Ceboimo slėnio, šalia dykumos. 
\par 19 Nė vieno kalvio nebuvo visoje Izraelio šalyje, nes filistinai sakė: “Kad hebrajai nepasidarytų kardų ir iečių”. 
\par 20 Kiekvienas izraelitas, norėdamas galąsti žagrę, kaplį, kirvį ar pjautuvą, turėjo eiti pas filistinus. 
\par 21 Todėl buvo atbukę jų žagrės, kapliai, šakės bei kirviai ir nesmailinti akstinai. 
\par 22 Mūšio dienai atėjus, nė vienas Sauliaus karys neturėjo nei kardo, nei ieties, išskyrus Saulių ir jo sūnų Jehonataną. 
\par 23 Ir filistinų būrys išėjo į Michmašo tarpeklį.



\chapter{14}

\par 1 Kartą Sauliaus sūnus Jehonatanas kalbėjo jaunuoliui, savo ginklanešiui: “Eikime prie filistinų įgulos, kuri stovi anoje pusėje”. Savo tėvui jis nieko nesakė. 
\par 2 Saulius buvo apsistojęs prie Gibėjos, po granatmedžiu Migrone, ir su juo buvo apie šešis šimtus vyrų. 
\par 3 Achija, sūnus Ikabodo brolio Ahitubo, sūnaus Finehaso, sūnaus Šilojo kunigo Elio, nešiojo efodą. Žmonės nežinojo, kad Jehonatanas išėjo. 
\par 4 Prie tarpeklio, per kurį Jehonatanas turėjo pereiti, kad pasiektų filistinų įgulą, buvo smailios uolos vienoje ir kitoje pusėje. Viena vadinosi Bocecas, o antroji­Senė. 
\par 5 Viena kyšojo šiaurėje, prieš Michmašą, o antroji­pietuose, prieš Gibėją. 
\par 6 Jehonatanas tarė jaunuoliui, kuris nešiojo jo ginklus: “Nueikime prie šitų neapipjaustytųjų įgulos. Gal Viešpats mums padės; juk Viešpačiui nesunku išgelbėti per kelis žmones, kaip ir per daugelį”. 
\par 7 Jo ginklanešys jam atsakė: “Daryk visa, kas yra tavo širdyje. Eik ten, o aš būsiu su tavimi, kur tu panorėsi”. 
\par 8 Jehonatanas tarė: “Mes, nuėję pas tuos vyrus, jiems pasirodysime. 
\par 9 Jei jie mums sakys: ‘Palaukite, kol mes pas jus ateisime’, tai mes stovėsime savo vietoje ir neisime prie jų. 
\par 10 O jei jie sakys: ‘Užlipkite pas mus’, tai mes užlipsime, nes Viešpats bus juos atidavęs į mūsų rankas, ir tai bus mums ženklas”. 
\par 11 Ir jie pasirodė filistinų įgulai. Filistinai sakė: “Žiūrėkite, hebrajai išlenda iš plyšių, kuriuose jie buvo pasislėpę”. 
\par 12 Filistinų įgulos vyrai šaukė Jehonatanui ir jo ginklanešiui, sakydami: “Užlipkite pas mus, mes jums kai ką parodysime”. Tuomet Jehonatanas tarė savo ginklanešiui: “Lipk paskui mane, nes Viešpats juos atidavė į Izraelio rankas”. 
\par 13 Jehonatanas lipo, kabindamasis rankomis ir kojomis, o jo ginklanešys sekė paskui jį. Filistinai krito prieš Jehonataną, o jo ginklanešys juos pribaigdavo. 
\par 14 Pirmojo puolimo metu, kurį įvykdė Jehonatanas ir jo ginklanešys, žuvo apie dvidešimt vyrų plote, kurį per pusę dienos galima suarti su pora jaučių. 
\par 15 Kilo panika stovykloje, laukuose ir tarp visų žmonių. Būriai, naikinę kraštą, taip tirtėjo, kad net žemė ėmė drebėti. Visus apėmė didelė baimė. 
\par 16 Benjamino Gibėjos stovykloje Sauliaus sargybiniai pamatė, kad filistinai pakriko ir lakstė į visas puses. 
\par 17 Saulius įsakė vyrams: “Suskaičiuokite ir pažiūrėkite, kas išėjęs iš mūsų”. Suskaičiavus pasirodė, kad nėra Jehonatano ir jo ginklanešio. 
\par 18 Tada Saulius įsakė Achijai: “Atgabenk Dievo skrynią”. Tuo metu Dievo skrynia buvo pas izraelitus. 
\par 19 Kol Saulius kalbėjo su kunigu, triukšmas filistinų stovykloje augo. Tada Saulius tarė kunigui: “Atitrauk savo ranką”. 
\par 20 Saulius ir visi jo žmonės susirinko ir išėjo į mūšį. Ten buvo labai didelė sumaištis ir filistinai žudė vienas kitą. 
\par 21 Hebrajai, kurie anksčiau buvo pas filistinus ir kartu su jais atėjo į stovyklą, prisidėjo prie izraelitų, kurie buvo su Sauliumi ir Jehonatanu. 
\par 22 Taip pat ir tie izraelitai, kurie buvo pasislėpę Efraimo kalnuose, kai išgirdo, kad filistinai bėga, persekiojo juos kovodami. 
\par 23 Taip Viešpats tą dieną išgelbėjo Izraelį. Kova nusitęsė net iki Bet Aveno. 
\par 24 Izraelio vyrai tą dieną buvo pavargę, nes Saulius prisaikdino žmones, sakydamas: “Prakeiktas tebūna tas vyras, kuris valgytų iki vakaro, kol aš atkeršysiu savo priešams”. Todėl niekas nelietė maisto. 
\par 25 Žmonės nuėjo į mišką ir rado medaus ant žemės. 
\par 26 Žmonės, įėję į mišką, matė varvantį medų, tačiau nė vienas neragavo jo, nes bijojo prakeikimo. 
\par 27 Jehonatanas negirdėjo, kaip jo tėvas prisaikdino tautą. Jis lazdos galu pakabino medaus ir pakėlė ranką prie savo burnos. Ir jo akys nušvito. 
\par 28 Vienas iš žmonių tarė: “Tavo tėvas prisaikdino tautą, sakydamas: ‘Prakeiktas tebūna tas vyras, kuris šiandien valgytų’ ”. Ir žmonės buvo išsekę. 
\par 29 Jehonatanas atsakė: “Mano tėvas pridarė bėdos žemėje. Žiūrėkite, kaip nušvito mano akys, kai paragavau truputį medaus. 
\par 30 Jei šiandien žmonės būtų sočiai pavalgę iš priešo grobio, kurį rado, jie būtų išžudę daug daugiau filistinų”. 
\par 31 Tą dieną jie naikino filistinus nuo Michmašo iki Ajalono, ir žmonės buvo labai išsekę. 
\par 32 Jie griebė avis, jaučius ir karves, pjovė ant žemės ir valgė su krauju. 
\par 33 Sauliui buvo pranešta, kad žmonės nusideda Viešpačiui, valgydami kraują. Jis tarė: “Jūs nusikaltote. Atriskite man didelį akmenį”. 
\par 34 Ir Saulius įsakė: “Atveskite kiekvieną jautį ir avį čia, papjaukite ir valgykite, kad nenusidėtumėte Viešpačiui, valgydami kraują”. Visi žmonės atvesdavo savo jaučius tą vakarą ir ten papjaudavo. 
\par 35 Saulius pastatė Viešpačiui aukurą; tai buvo pirmas aukuras, kurį jis pastatė Viešpačiui. 
\par 36 Saulius sakė: “Pulkime filistinus naktį, plėškime iki ryto ir nepalikime jų nė vieno gyvo”. Jie atsakė: “Daryk, kaip tau atrodo tinkama”. Tada kunigas tarė: “Artinkimės čia prie Dievo”. 
\par 37 Tuomet Saulius klausė Viešpatį: “Ar man pulti filistinus? Ar atiduosi juos į Izraelio rankas?” Bet Viešpats jam neatsakė tą dieną. 
\par 38 Saulius įsakė: “Susirinkite čia visi vyresnieji ir išaiškinkime, kas šiandien nusikalto. 
\par 39 Prisiekiu, kaip gyvas Viešpats, Izraelio gelbėtojas, nors tai būtų ir mano sūnus Jehonatanas, turės mirti”. Tačiau visi žmonės nieko neatsakė. 
\par 40 Tada jis tarė visiems izraelitams: “Jūs būkite vienoje pusėje, o aš ir mano sūnus Jehonatanas būsime kitoje pusėje”. Žmonės atsakė Sauliui: “Daryk, kaip tau atrodo tinkama”. 
\par 41 Tuomet Saulius kreipėsi į Viešpatį: “Izraelio Dieve, duok ženklą”. Jehonatanas ir Saulius buvo apkaltinti, o tauta pripažinta nekalta. 
\par 42 Saulius tarė: “Meskite burtą tarp manęs ir mano sūnaus Jehonatano”. Kaltė krito Jehonatanui. 
\par 43 Saulius klausė Jehonataną: “Pasakyk, ką padarei”. Jehonatanas jam atsakė: “Aš paragavau medaus, pasikabinęs lazdos galu. Ir štai aš turiu mirti”. 
\par 44 Saulius tarė: “Tegul Dievas padaro man tai ir dar daugiau, nes tu, Jehonatanai, tikrai mirsi”. 
\par 45 Ir tauta sakė Sauliui: “Argi Jehonatanas, laimėjęs Izraeliui išgelbėjimą, turėtų mirti? Jokiu būdu! Kaip Viešpats gyvas, nė vienas plaukas nenukris nuo jo galvos. Juk šiandien jis veikė kartu su Dievu”. Taip žmonės išgelbėjo Jehonataną iš mirties. 
\par 46 Saulius nebepersekiojo filistinų ir filistinai sugrįžo į savo vietas. 
\par 47 Saulius, įsitvirtinęs karaliumi Izraelyje, kariavo su visais savo priešais: su Moabu, su amonitais, su Edomu, su Cobos karaliais ir su filistinais. Prieš ką jis pasukdavo, ten laimėdavo. 
\par 48 Jis surinko kariuomenę ir nugalėjo Amaleką, ir išgelbėjo Izraelį iš rankos tų, kurie jį plėšė. 
\par 49 Sauliaus sūnūs buvo Jehonatanas, Išvis ir Malkišūva, o jo dviejų dukterų vardai: vyresniosios­Meraba ir jaunesniosios­Mikalė. 
\par 50 Sauliaus žmona buvo Ahinoama, Achimaaco duktė. Jo kariuomenės vadas buvo Abneras, Sauliaus dėdės Nero sūnus. 
\par 51 Sauliaus tėvas buvo Kišas, o Abnero­Neras, Abielio sūnus. 
\par 52 Per visas Sauliaus dienas vyko smarkus karas su filistinais. Pamatęs tvirtą ir narsų vyrą, Saulius paimdavo jį pas save.



\chapter{15}


\par 1 Samuelis tarė Sauliui: “Mane Viešpats siuntė tave patepti karaliumi Jo tautai Izraeliui. Taigi dabar klausyk Viešpaties žodžių. 
\par 2 Taip sako kareivijų Viešpats: ‘Aš prisimenu, ką Amalekas padarė Izraeliui: kaip jis tykojo kelyje, kai tas žygiavo iš Egipto. 
\par 3 Dabar eik ir užpulk Amaleką, ir visiškai sunaikink viską, kas jam priklauso, nieko nesigailėdamas. Išžudyk vyrus, moteris, vaikus ir kūdikius, jaučius, avis, kupranugarius ir asilus’ ”. 
\par 4 Saulius surinko žmones Telaime ir juos suskaičiavo; buvo du šimtai tūkstančių pėstininkų ir dešimt tūkstančių vyrų iš Judo giminės. 
\par 5 Tuomet jis atėjo prie Amaleko miesto ir slėnyje paliko pasalą. 
\par 6 Kenitams jis tarė: “Pasitraukite nuo amalekiečių, kad jūsų nesunaikinčiau kartu su jais. Jūs buvote draugiški izraelitams, jiems einant iš Egipto”. Kenitai pasitraukė nuo amalekiečių. 
\par 7 Saulius sumušė Amaleką nuo Havilos iki Šūro apylinkių, į rytus nuo Egipto. 
\par 8 Amaleko karalių Agagą jis paėmė gyvą, o visus žmones sunaikino kardu. 
\par 9 Bet Saulius ir žmonės pagailėjo Agago, geriausių avių, galvijų, penimų avinų ir apskritai viso, kas buvo gera, jie nenorėjo sunaikinti. Kas buvo nedidelės vertės, tą jie visiškai sunaikino. 
\par 10 Tada Viešpaties žodis atėjo Samueliui: 
\par 11 “Gailiuosi Saulių padaręs karaliumi, nes jis nusigręžė nuo manęs ir neįvykdė mano įsakymų”. Tai nuliūdino Samuelį, ir jis šaukėsi Viešpaties visą naktį. 
\par 12 Atsikėlęs anksti rytą, jis nuėjo pasitikti Sauliaus. Samueliui buvo pasakyta: “Saulius nuėjo į Karmelį, ten pasistatė paminklą ir iš ten jis nuvyko į Gilgalą”. 
\par 13 Samueliui atėjus, Saulius jam tarė: “Būk palaimintas Viešpaties. Aš įvykdžiau Viešpaties įsakymą”. 
\par 14 Samuelis klausė: “Ką reiškia tas avių bliovimas ir galvijų baubimas, kurį girdžiu?” 
\par 15 Saulius atsakė: “Iš Amaleko jie atsivarė juos, nes žmonės išsaugojo geriausias avis ir galvijus, norėdami paaukoti juos Viešpačiui, tavo Dievui; visa kita mes visiškai sunaikinome”. 
\par 16 Samuelis tarė Sauliui: “Palauk, ir aš pasakysiu tau, ką Viešpats man šiąnakt kalbėjo”. Ir jis atsakė: “Kalbėk”. 
\par 17 Samuelis tarė: “Kai tu buvai mažas savo akyse, tapai Izraelio giminių galva ir Viešpats tave patepė Izraelio karaliumi. 
\par 18 Ir Jis siuntė tave į kelią, sakydamas: ‘Eik ir visiškai sunaikink Amaleko nusidėjėlius. Kariauk su jais, iki visai juos išnaikinsi’. 
\par 19 Kodėl nepaklusai Viešpaties balsui ir puolei prie grobio, piktai pasielgdamas Viešpaties akyse?” 
\par 20 Saulius atsakė Samueliui: “Aš juk paklusau Viešpaties balsui ir ėjau keliu, kuriuo Viešpats mane siuntė; aš parsivedžiau Amaleko karalių Agagą, o amalekiečius visiškai sunaikinau. 
\par 21 Bet žmonės ėmė iš grobio geriausias avis ir galvijus, kurie turėjo būti sunaikinti, norėdami aukoti Viešpačiui, tavo Dievui, Gilgale”. 
\par 22 Samuelis atsakė: “Argi Viešpats labiau vertina deginamąsias aukas ir atnašas, negu paklusnumą Viešpaties balsui? Paklusti yra geriau negu aukoti ir klausyti yra geriau už avinų taukus. 
\par 23 Nepaklusnumas yra kaip žyniavimo nuodėmė ir užsispyrimas yra kaip stabmeldystė. Kadangi tu atmetei Viešpaties žodį, Jis atmetė tave, kad nebebūtum karaliumi”. 
\par 24 Saulius atsakė Samueliui: “Aš nusidėjau, nes nepaklausiau Viešpaties įsakymo ir tavo žodžių, bet, bijodamas žmonių, paklusau jų balsui. 
\par 25 Prašau, atleisk mano nuodėmę ir grįžk su manimi, kad galėčiau pagarbinti Viešpatį”. 
\par 26 Bet Samuelis atsakė Sauliui: “Aš neisiu su tavimi. Kadangi tu atmetei Viešpaties žodį, Viešpats atmetė tave, kad nebūtum Izraelio karaliumi”. 
\par 27 Samueliui apsisukus eiti, Saulius nutvėrė už jo apsiausto skverno ir tas suplyšo. 
\par 28 Ir Samuelis jam pasakė: “Viešpats šiandien atplėšė nuo tavęs Izraelio karalystę ir ją atidavė tavo artimui, geresniam už tave. 
\par 29 Izraelio Galybė nemeluoja ir nepersigalvoja, nes Jis ne žmogus, kad persigalvotų”. 
\par 30 Saulius tarė: “Aš nusidėjau. Tačiau dabar, prašau, pagerbk mane tautos vyresniųjų bei Izraelio akivaizdoje ir grįžk su manimi, kad galėčiau pagarbinti Viešpatį, tavo Dievą”. 
\par 31 Samuelis sugrįžo su Sauliumi, ir Saulius pagarbino Viešpatį. 
\par 32 Samuelis įsakė: “Atveskite pas mane amalekiečių karalių Agagą”. Agagas atėjo drebėdamas pas jį ir tarė: “Tikrai mirties kartumas jau praėjo”. 
\par 33 Samuelis tarė: “Kaip tavo kardas atimdavo moterims vaikus, taip tavo motina tepasilieka bevaikė”. Ir Samuelis sukapojo Agagą į gabalus Viešpaties akivaizdoje Gilgale. 
\par 34 Samuelis nuėjo į Ramą, o Saulius­ į savo namus Gibėjoje. 
\par 35 Samuelis iki mirties nebematė Sauliaus. Tačiau Samuelis liūdėjo dėl Sauliaus, ir Viešpats gailėjosi padaręs Saulių Izraelio karaliumi.



\chapter{16}


\par 1 Viešpats tarė: “Ar ilgai liūdėsi dėl Sauliaus, kurį Aš atmečiau, kad nebūtų Izraelio karaliumi? Prisipildyk savo ragą aliejaus ir eik; Aš pasiųsiu tave į Betliejų pas Jesę, nes vieną jo sūnų pasirinkau karaliumi”. 
\par 2 Samuelis atsakė: “Kaip aš galiu eiti? Jei išgirs Saulius, jis mane nužudys”. Viešpats tarė: “Pasiimk veršį ir nuėjęs sakyk: ‘Atėjau Viešpačiui aukoti’. 
\par 3 Pasikviesk Jesę prie aukos, o Aš tau parodysiu, ką reikia daryti. Tu patepsi tą, kurį tau nurodysiu”. 
\par 4 Samuelis padarė, kaip Viešpats jam sakė. Jam atėjus į Betliejų, miesto vyresnieji nusigando ir, jį pasitikę, klausė: “Ar su taika ateini?” 
\par 5 Jis atsakė: “Su taika! Atėjau Viešpačiui aukoti. Pasišventinkite ir eikite su manimi prie aukos”. Jis pašventino Jesę bei jo sūnus ir juos pakvietė prie aukos. 
\par 6 Kai jie atėjo, Samuelis pažiūrėjo į Eliabą ir sakė: “Tikrai šis yra Viešpaties pateptasis”. 
\par 7 Viešpats tarė Samueliui: “Nežiūrėk į jo išvaizdą nė į jo ūgį, nes Aš jį atmečiau. Viešpats mato ne taip, kaip žmogus. Žmogus žiūri į išorę, bet Viešpats žiūri į širdį”. 
\par 8 Tada Jesė pašaukė Abinadabą ir jam leido praeiti pro Samuelį. Ir jis sakė: “Ir šito Viešpats neišsirinko”. 
\par 9 Jesė leido praeiti Šamui. Ir jis sakė: “Šito Viešpats irgi neišsirinko”. 
\par 10 Taip Jesė leido visiems septyniems sūnums praeiti pro Samuelį. Samuelis tarė Jesei: “Viešpats šitų neišsirinko”. 
\par 11 Ir Samuelis sakė Jesei: “Ar čia jau visi tavo sūnūs?” Jesė atsakė: “Liko dar jauniausias, jis gano avis”. Samuelis tarė: “Pasiųsk ir atvesk jį, nes mes nesėsime, kol jis neateis”. 
\par 12 Jis pasiuntė ir atvedė jį. Jis buvo raudonskruostis, gražaus veido ir dailios išvaizdos jaunuolis. Viešpats tarė: “Patepk jį, nes tai jis”. 
\par 13 Samuelis paėmė ragą su aliejumi ir jį patepė tarp jo brolių. Tą dieną Viešpaties Dvasia nužengė ant Dovydo ir pasiliko su juo. Samuelis sugrįžo į Ramą. 
\par 14 Bet Viešpaties Dvasia pasitraukė nuo Sauliaus, ir piktoji dvasia nuo Dievo kankino jį. 
\par 15 Sauliaus tarnai kalbėjo jam: “Štai piktoji dvasia nuo Dievo kankina tave. 
\par 16 Tegul mūsų valdovas įsako savo tarnams suieškoti vyrą, kuris gražiai skambina arfa. Kai tave apims piktoji dvasia nuo Dievo, jis skambins ir nuramins tave”. 
\par 17 Saulius atsakė savo tarnams: “Suraskite vyrą, mokantį gerai skambinti, ir atveskite jį pas mane”. 
\par 18 Vienas tarnas atsiliepė: “Aš mačiau Jesės sūnų iš Betliejaus, kuris moka gražiai skambinti. Jis vyras drąsus, karingas, išmintingas kalboje, gražus ir Viešpats yra su juo”. 
\par 19 Todėl Saulius pasiuntė pasiuntinius pas Jesę, sakydamas: “Atsiųsk pas mane savo sūnų Dovydą, kuris gano avis”. 
\par 20 Jesė, užkrovęs ant asilo duonos, odinę vyno ir ožiuką, pasiuntė savo sūnų Dovydą Sauliui. 
\par 21 Dovydas atvyko pas Saulių ir pradėjo jam tarnauti. Saulius jį labai pamilo ir paskyrė savo ginklanešiu. 
\par 22 Saulius siuntė pas Jesę, sakydamas: “Telieka Dovydas tarnauti pas mane, nes jis rado malonę mano akyse”. 
\par 23 Kai piktoji dvasia nuo Dievo apimdavo Saulių, Dovydas skambindavo arfa. Sauliui palengvėdavo, ir piktoji dvasia nuo jo pasitraukdavo.



\chapter{17}

\par 1 Filistinai surinko savo kariuomenę karui. Jie susirinko Sochojo mieste, kuris priklauso Judui, ir pastatė stovyklą tarp Sochojo ir Azekos, Efesdomime. 
\par 2 Saulius ir Izraelio vyrai susirinko ir pasistatė stovyklą Elos slėnyje, ir pasiruošė kautynėms su filistinais. 
\par 3 Filistinai stovėjo ant kalno vienoje kelio pusėje, o izraelitai­ant kalno kitoje; tarp jų buvo slėnis. 
\par 4 Iš filistinų stovyklos išėjo galiūnas, vardu Galijotas iš Gato, šešių uolekčių ir vieno sprindžio ūgio. 
\par 5 Ant galvos jis turėjo varinį šalmą ir buvo apsivilkęs šarvų marškiniais, kurie svėrė penkis tūkstančius šekelių vario. 
\par 6 Variniai antblauzdžiai dengė jo blauzdas ir varinis skydas pečius. 
\par 7 Jo ieties kotas buvo kaip audėjo staklių riestuvas, o jo ieties smaigalys svėrė šešis šimtus šekelių geležies; prieš jį ėjo ginklanešys. 
\par 8 Jis sustojo ir šaukė Izraelio kariuomenei, sakydamas: “Kodėl išėjote kariauti? Argi aš ne filistinas, o jūs ne Sauliaus tarnai? Išrinkite vyrą, ir tegul jis ateina pas mane. 
\par 9 Jei jis sugebės nugalėti ir užmušti mane, tai mes jums tarnausime, o jei aš jį nugalėsiu ir užmušiu, tai jūs tapsite mūsų tarnais”. 
\par 10 Filistinas tarė: “Aš šiandien tyčiojuos iš Izraelio kariuomenės; duokite vyrą, kad su manim kautųsi”. 
\par 11 Saulius ir visas Izraelis, išgirdę šituos filistino žodžius, labai nusigando. 
\par 12 Dovydas buvo efratiečio Jesės iš Judo Betliejaus sūnus. Jesė turėjo aštuonis sūnus, jis pats Sauliaus dienomis buvo pasenęs ir vienas iš seniausių vyrų. 
\par 13 Trys vyresnieji Jesės sūnūs išėjo su Sauliumi į karą: pirmagimis Eliabas, antrasis Abinadabas ir Šama. 
\par 14 Dovydas buvo jauniausias. Trys vyresnieji išėjo su Sauliumi, 
\par 15 o Dovydas sugrįžo iš Sauliaus pas tėvą į Betliejų avių ganyti. 
\par 16 Filistinas keturiasdešimt dienų kiekvieną rytą ir vakarą išeidavo ir rodydavo save. 
\par 17 Jesė sakė savo sūnui Dovydui: “Imk efą paskrudintų grūdų bei dešimt duonos kepalų ir skubiai nunešk į stovyklą savo broliams. 
\par 18 Dešimt šitų sūrių nunešk savo brolių tūkstantininkui. Pasiteirauk, kaip sekasi tavo broliams, ir sugrįžęs pranešk man”. 
\par 19 Saulius, Dovydo broliai ir visi Izraelio vyrai buvo Elos slėnyje ir kariavo su filistinais. 
\par 20 Dovydas, atsikėlęs anksti rytą ir palikęs avis sargui, paėmė maistą ir išėjo, kaip tėvas buvo įsakęs. Jam atėjus į stovyklą, kariuomenė buvo išsirikiavusi kautynėms ir šaukė prieš mūšį. 
\par 21 Izraelitai ir filistinai stovėjo išsirikiavę kautynėms vieni prieš kitus. 
\par 22 Dovydas, palikęs daiktus pas kariuomenės mantos sargą, nubėgo į kautynių lauką ir pasveikino savo brolius. 
\par 23 Jam su jais besikalbant, pasirodė galiūnas, filistinas Galijotas iš Gato. Jis išėjo iš filistinų eilių į priekį ir kalbėjo tuos pačius žodžius. Dovydas tai girdėjo. 
\par 24 Izraelio vyrai, pamatę tą vyrą, bėgo nuo jo ir labai bijojo. 
\par 25 Jie kalbėjosi: “Ar matote šitą vyrą? Jis ateina tyčiotis iš Izraelio. Kas jį užmuš, tą karalius apdovanos dideliais turtais, duos jam savo dukterį ir jo tėvo namus atleis nuo mokesčių Izraelyje”. 
\par 26 Dovydas klausė šalia jo stovėjusių vyrų: “Ką gaus tas vyras, kuris nukaus šitą filistiną ir pašalins Izraelio gėdą? Kas yra šitas neapipjaustytas filistinas, kad tyčiotųsi iš gyvojo Dievo kariuomenės?” 
\par 27 Vyrai jam atsakė tais žodžiais, sakydami: “Tai bus vyrui, kuris jį nužudys”. 
\par 28 Jo vyriausias brolis Eliabas, išgirdęs Dovydą kalbant su vyrais, labai supyko ir tarė: “Ko čia atėjai, palikęs savo kelias avis dykumoje? Aš žinau tavo išdidumą ir tavo širdies sugedimą. Tu atėjai norėdamas pamatyti mūšį”. 
\par 29 Dovydas atsakė: “Ką aš padariau? Ar tai nėra tik žodžiai?” 
\par 30 Nusisukęs nuo jo, Dovydas atsisuko į kitą ir kalbėjo tą patį. Žmonės jam atsakydavo kaip pirma. 
\par 31 Dovydo žodžiai buvo perduoti Sauliui. Jis įsakė atvesti Dovydą. 
\par 32 Dovydas tarė Sauliui: “Te nė vieno žmogaus širdis nenusigąsta jo. Tavo tarnas eis ir kausis su šituo filistinu”. 
\par 33 Saulius atsakė Dovydui: “Tu negali kautis su šituo filistinu, nes esi jaunas, o jis yra karys nuo pat jaunystės”. 
\par 34 Dovydas atsakė Sauliui: “Tavo tarnas ganė savo tėvo avis. Jei ateidavo liūtas ar lokys ir pagriebdavo ėriuką iš bandos, 
\par 35 aš pasileisdavau jam iš paskos, mušdavau jį ir išplėšdavau grobį iš nasrų. Jei jis puldavo mane, nutverdavau jį už barzdos ir užmušdavau. 
\par 36 Tavo tarnas yra užmušęs liūtą ir lokį, ir šitam neapipjaustytam filistinui atsitiks taip, kaip jiems, nes jis tyčiojasi iš gyvojo Dievo kariuomenės. 
\par 37 Viešpats, kuris išgelbėjo mane iš liūto ir lokio nagų, išgelbės ir iš šito filistino rankų”. Saulius tarė Dovydui: “Eik, ir Viešpats tebūna su tavimi”. 
\par 38 Saulius apginklavo Dovydą savo ginklais, uždėjo varinį šalmą jam ant galvos ir apvilko šarvų marškiniais. 
\par 39 Dovydas prisijuosė ir jo kardą prie savo aprangos ir bandė eiti, nes nebuvo įpratęs. Dovydas tarė Sauliui: “Aš negaliu paeiti, nes esu neįpratęs”. Ir Dovydas nusirengė visa tai. 
\par 40 Jis pasiėmė lazdą, pasirinko iš upelio penkis glotnius akmenis, juos įsidėjo į piemens maišelį, kurį turėjo su savimi, ir laikydamas mėtyklę rankoje artėjo prie filistino. 
\par 41 Ir filistinas išėjo, ir artinosi prie Dovydo, o priešais jį ėjo ginklanešys su skydu. 
\par 42 Kai filistinas apsidairė ir pamatė Dovydą, paniekino jį, nes šis buvo raudonskruostis gražaus veido jaunuolis. 
\par 43 Filistinas sakė Dovydui: “Ar aš šuo, kad tu eini prieš mane su lazda?” Ir filistinas keikė Dovydą savo dievais. 
\par 44 Filistinas sakė Dovydui: “Ateik, aš atiduosiu tavo kūną padangių paukščiams ir lauko žvėrims”. 
\par 45 Dovydas atsakė filistinui: “Tu eini prieš mane su kardu, ietimi ir skydu, o aš einu kareivijų Viešpaties, Izraelio kariuomenės, iš kurios tyčiojiesi, Dievo vardu. 
\par 46 Šiandien Viešpats atiduos tave į mano rankas. Aš nugalėsiu tave, nukirsiu tau galvą ir atiduosiu visų filistinų karių lavonus padangių paukščiams ir lauko žvėrims, kad visa žemė žinotų, jog yra Dievas Izraelyje. 
\par 47 Ir kad visi čia susirinkę žinotų, jog ne kardu ir ietimi Viešpats gelbsti. Kova yra Viešpaties, ir Jis atiduos jus į mūsų rankas”. 
\par 48 Filistinui artėjant prie Dovydo, šis skubiai bėgo jam priešais. 
\par 49 Įkišęs ranką į maišelį, jis išsiėmė akmenį ir metė iš mėtyklės, ir pataikė filistinui į kaktą taip, kad akmuo įsmigo jam į kaktą, ir jis griuvo kniūbsčias ant žemės. 
\par 50 Taip Dovydas nugalėjo filistiną mėtykle ir akmeniu, partrenkė ir nužudė jį. Dovydas neturėjo kardo, 
\par 51 todėl pribėgo prie filistino, ištraukė iš makšties jo kardą ir nukirto jam galvą. Filistinai, pamatę, kad jų galiūnas negyvas, pasileido bėgti. 
\par 52 Izraelio ir Judo vyrai šaukdami vijo filistinus iki Gato ir Ekrono vartų. Filistinų lavonai gulėjo nuo Šaaraimo iki Gato ir Ekrono. 
\par 53 Izraelitai, baigę persekioti filistinus, sugrįžo ir išplėšė jų stovyklą. 
\par 54 Paėmęs filistino galvą, Dovydas ją nunešė į Jeruzalę, o ginklus padėjo savo palapinėje. 
\par 55 Kai Saulius matė Dovydą, išeinantį prieš filistiną, klausė kariuomenės vado Abnero: “Abnerai, kieno sūnus yra tas jaunuolis?” Abneras atsakė: “Kaip tu gyvas, karaliau, aš nežinau”. 
\par 56 Karalius įsakė: “Sužinok, kieno sūnus tas jaunuolis”. 
\par 57 Kai Dovydas, nukovęs filistiną, sugrįžo, Abneras atvedė jį pas Saulių; jis tebelaikė filistino galvą rankose. 
\par 58 Saulius jo klausė: “Jaunuoli, kieno tu sūnus?” Dovydas atsakė: “Aš esu tavo tarno Jesės iš Betliejaus sūnus”.



\chapter{18}

\par 1 Dovydui kalbant su Sauliumi, Jehonatano siela prisirišo prie Dovydo sielos, ir Jehonatanas pamilo jį kaip savo sielą. Ir Dovydas jį pamilo visa širdimi. 
\par 2 Nuo to laiko Saulius priėmė jį ir nebeleido grįžti į tėvo namus. 
\par 3 Jehonatanas padarė sandorą su Dovydu, nes mylėjo jį kaip savo sielą. 
\par 4 Jehonatanas atidavė Dovydui savo apsiaustą, drabužius, net ir kardą, lanką bei diržą. 
\par 5 Dovydas išmintingai elgėsi visur, kur tik Saulius jį pasiųsdavo; todėl Saulius jį paskyrė karių viršininku, ir tai patiko visai tautai ir Sauliaus tarnams. 
\par 6 Dovydui nugalėjus filistiną ir visiems grįžtant į namus, moterys iš visų miestų išeidavo sutikti karalių Saulių dainuodamos, šokdamos ir grodamos būgneliais bei cimbolais. 
\par 7 Moterys dainuodamos kartojo: “Saulius nukovė tūkstančius, o Dovydas­dešimtis tūkstančių”. 
\par 8 Saulius labai supyko, jam nepatiko tokios kalbos. Jis tarė: “Dovydui jos priskyrė dešimtis tūkstančių, o man tik tūkstančius; jam betrūksta tik karalystės”. 
\par 9 Nuo tos dienos Saulius ėmė stebėti Dovydą. 
\par 10 Kitą dieną piktoji dvasia nuo Dievo taip apėmė Saulių, kad jis siautė savo namuose. Tuo metu Dovydas skambino arfa kaip kasdien. Saulius laikė rankoje ietį. 
\par 11 Jis sviedė ją į Dovydą, galvodamas: “Prismeigsiu jį prie sienos”. Bet Dovydas išsisuko du kartus. 
\par 12 Saulius bijojo Dovydo, nes Viešpats buvo su juo, o nuo Sauliaus Jis buvo pasitraukęs. 
\par 13 Saulius pašalino Dovydą nuo savęs ir paskyrė tūkstantininku. Dovydas įeidavo ir išeidavo priešais tautą. 
\par 14 Dovydas išmintingai elgėsi visuose savo keliuose, ir Viešpats buvo su juo. 
\par 15 Saulius, matydamas, kad jis elgiasi labai išmintingai, bijojo jo. 
\par 16 Visas Izraelis ir Judas mylėjo Dovydą, nes jis įeidavo ir išeidavo priešais juos. 
\par 17 Kartą Saulius tarė: “Štai mano vyresnioji duktė Meraba! Ją duosiu tau į žmonas. Tik būk narsus ir kovok Viešpaties kovas”. Saulius galvojo: “Tegul ne mano ranka būna prieš jį, bet filistinų ranka”. 
\par 18 Dovydas atsakė Sauliui: “Kas aš, kas yra mano gyvenimas ir kas mano tėvo giminė Izraelyje, kad būčiau karaliaus žentu?” 
\par 19 Bet tuo metu, kai Meraba, Sauliaus duktė, turėjo būti atiduota Dovydui, ją vedė Adrielis iš Meholos. 
\par 20 Sauliaus duktė Mikalė pamilo Dovydą. Kai Saulius sužinojo, jam tai patiko. 
\par 21 Ir Saulius sakė: “Aš jam duosiu ją, kad ji būtų jam spąstai ir kad filistinų ranka būtų prieš jį”. Saulius sakė Dovydui: “Šiandien tu tapsi mano žentu su antrąja”. 
\par 22 Jis įsakė savo tarnams: “Kalbėkite Dovydui slaptai: ‘Karalius mėgsta tave ir visiems jo tarnams tu patinki. Todėl būk karaliaus žentu’ ”. 
\par 23 Sauliaus tarnai kalbėjo Dovydui šiuos žodžius. Dovydas atsakė: “Ar jūs manote, kad lengva būti karaliaus žentu? Aš juk esu neturtingas ir menkas žmogus”. 
\par 24 Tarnai pranešė Sauliui, ką girdėjo iš Dovydo. 
\par 25 Tada Saulius sakė: “Taip sakykite Dovydui: ‘Karalius nenori jokio kraičio, tik šimto filistinų odelių nuo apipjaustymo, kad būtų atkeršyta karaliaus priešams’ ”. Saulius galvojo, kad Dovydas žus nuo filistinų rankos. 
\par 26 Kai jo tarnai perdavė Dovydui tuos žodžius, Dovydui patiko tapti karaliaus žentu. Dar nebuvo praėjęs paskirtas laikas, 
\par 27 kai Dovydas su savo vyrais nuėjęs nužudė du šimtus filistinų. Dovydas atnešė jų odeles ir atidavė visas karaliui, kad galėtų tapti jo žentu. Ir Saulius atidavė jam savo dukterį Mikalę į žmonas. 
\par 28 Saulius matė ir suprato, kad Viešpats buvo su Dovydu ir kad jo duktė Mikalė myli jį. 
\par 29 Ir Saulius dar labiau ėmė bijoti Dovydo; ir Saulius tapo Dovydo priešu visam gyvenimui. 
\par 30 Filistinų kunigaikščiai kariavo su izraelitais, ir nuo karo pradžios Dovydas elgėsi išmintingiau už visus Sauliaus tarnus, ir jo vardas išgarsėjo.



\chapter{19}

\par 1 Saulius kalbėjo savo sūnui Jehonatanui ir visiems tarnams, kad jie nužudytų Dovydą. Bet Sauliaus sūnus Jehonatanas labai mėgo Dovydą. 
\par 2 Jehonatanas perspėjo Dovydą: “Mano tėvas Saulius ketina tave nužudyti. Todėl saugokis ir būk pasislėpęs iki ryto. 
\par 3 Aš išėjęs stovėsiu šalia savo tėvo lauke, kalbėsiu su tėvu apie tave. Ką sužinosiu, tau pranešiu”. 
\par 4 Jehonatanas kalbėjo gera apie Dovydą savo tėvui Sauliui: “Karaliau, nenusikalsk prieš savo tarną Dovydą, nes jis nenusikalto tau ir jo darbai tau buvo labai naudingi. 
\par 5 Jis statė į pavojų savo gyvybę, kovodamas su filistinu, ir jo dėka Viešpats suteikė didelį išgelbėjimą Izraeliui. Tu matei tai ir džiaugeisi. Kodėl dabar nori nusidėti prieš nekaltą kraują ir be priežasties nužudyti Dovydą?” 
\par 6 Saulius paklausė Jehonatano ir prisiekė: “Kaip Viešpats gyvas, jis nebus nužudytas”. 
\par 7 Tada Jehonatanas, pasišaukęs Dovydą ir jam viską papasakojęs, atvedė jį pas Saulių; Dovydas buvo Sauliaus akivaizdoje kaip anksčiau. 
\par 8 Vėl kilo karas; Dovydas išėjęs kariavo su filistinais ir daug jų nužudė, ir jie bėgo nuo jo. 
\par 9 Pikta dvasia nuo Viešpaties apėmė Saulių, ir jis sėdėjo namuose, laikydamas ietį rankoje, o Dovydas skambino arfa. 
\par 10 Saulius norėjo prismeigti ietimi Dovydą prie sienos, bet jis išsisuko ir ietis įsmigo į sieną. Dovydas išsigelbėjęs pabėgo tą pačią naktį. 
\par 11 Saulius siuntė vyrus budėti prie Dovydo namų ir rytą jį nužudyti. Bet jo žmona Mikalė pranešė Dovydui: “Jei šiąnakt neišgelbėsi savo gyvybės, rytoj būsi nužudytas”. 
\par 12 Mikalė nuleido Dovydą pro langą. Taip jis pabėgo ir išsigelbėjo. 
\par 13 Po to Mikalė paėmė statulą, paguldė į lovą, jos galvą apvyniojo ožkos kailiu ir viską apklojo apsiaustu. 
\par 14 Kai Saulius atsiuntė vyrus Dovydą suimti, Mikalė tarė: “Jis serga”. 
\par 15 Saulius vėl siuntė vyrus pas Dovydą ir įsakė: “Atgabenkite jį pas mane su lova, kad galėčiau jį nužudyti”. 
\par 16 Pasiuntiniai nuėję rado lovoje statulą, apvyniotą ožkos kailiu. 
\par 17 Tada Saulius tarė Mikalei: “Kodėl mane apgavai ir leidai mano priešui pabėgti?” Mikalė atsakė Sauliui: “Jis sakė man: ‘Išleisk mane, kad nereikėtų tavęs nužudyti’ ”. 
\par 18 Dovydas pabėgo ir išsigelbėjo. Jis nuvyko pas Samuelį į Ramą ir papasakojo viską, ką Saulius jam padarė. Po to jis ir Samuelis išėjo ir apsigyveno Najote. 
\par 19 Sauliui buvo pranešta, kad Dovydas yra Najote, Ramoje. 
\par 20 Jis siuntė vyrus suimti Dovydą. Kai jie pamatė pranašaujančių pranašų būrį ir Samuelį, stovintį jų priekyje, Dievo Dvasia nužengė ant pasiuntinių, ir jie taip pat pranašavo. 
\par 21 Tai sužinojęs, Saulius pasiuntė kitus pasiuntinius, bet ir tie pradėjo pranašauti. Ir Saulius pasiuntė vyrus trečią kartą, ir jie taip pat pranašavo. 
\par 22 Tada Saulius pats ėjo į Ramą ir, atėjęs prie didžiojo šulinio Sechuve, paklausė: “Kur yra Samuelis ir Dovydas?” Jam atsakė: “Jie yra Najote, Ramoje”. 
\par 23 Jis ėjo iš ten į Ramos Najotą. Dievo Dvasia nužengė ant jo, ir jis pranašavo visą kelią iki Ramos Najoto. 
\par 24 Nusivilkęs drabužius, jis pranašavo Samuelio akivaizdoje ir gulėjo neapsirengęs visą tą dieną ir naktį. Todėl yra sakoma: “Ar ir Saulius tarp pranašų?”



\chapter{20}

\par 1 Dovydas, pabėgęs iš Ramos Najoto, atėjo pas Jehonataną ir klausė: “Ką padariau, kuo nusikaltau, kuo nusidėjau tavo tėvui, kad jis ieško mano gyvybės?” 
\par 2 Jis jam atsakė: “Dieve gink! Tu nemirsi; štai mano tėvas visais reikalais tariasi su manimi. Kodėl jis šitą reikalą slėptų nuo manęs? Taip nebus”. 
\par 3 Dovydas atsakė: “Tavo tėvas gerai žino, kad aš radau malonę tavo akyse, ir galvoja: ‘Jehonatanas to neturi žinoti, kad nesisielotų’. Kaip Viešpats gyvas ir gyva tavo siela­tarp manęs ir mirties tik žingsnis”. 
\par 4 Jehonatanas atsakė Dovydui: “Padarysiu viską, ko trokšta tavo siela”. 
\par 5 Dovydas sakė Jehonatanui: “Štai rytoj yra jaunas mėnulis. Aš turėčiau su karaliumi sėdėti prie stalo. Leisk man eiti ir pasislėpti laukuose iki trečios dienos vakaro. 
\par 6 Jei tavo tėvas pasiges manęs, sakyk: ‘Dovydas labai prašė manęs leisti jam eiti į savo miestą Betliejų, kur jo visa šeima aukos kasmetinę auką’. 
\par 7 Jei jis sakys: ‘Gerai’, tai tavo tarnas bus saugus, o jei jis labai supyks, tai žinok, kad jis yra numatęs pikta. 
\par 8 Suteik malonę savo tarnui, nes tu padarei sandorą su manimi prieš Viešpatį. Bet jei aš nusikaltau, tu pats nužudyk mane. Kam tau mane vesti pas savo tėvą?” 
\par 9 Jehonatanas atsakė: “Tebūna tai toli nuo tavęs. Jei sužinosiu, kad mano tėvas yra numatęs tau pikta daryti, kodėl tau nepraneščiau?” 
\par 10 Dovydas klausė Jehonatano: “Kas man praneš, jei tavo tėvas rūsčiai tau atsakys?” 
\par 11 Jehonatanas atsakė Dovydui: “Eikime į lauką”. Juodu išėjo į lauką. 
\par 12 Jehonatanas tarė Dovydui: “Viešpats, Izraelio Dievas, tebūna liudytojas. Jei aš rytoj ar trečią dieną sužinosiu iš savo tėvo, kad jis tau palankus, ir nepranešiu tau, 
\par 13 tegul Viešpats padaro man tai ir dar daugiau. Jei mano tėvas sumanęs pikta prieš tave, aš tau pranešiu ir tave išsiųsiu, kad galėtum eiti ramybėje, o Viešpats bus su tavimi, kaip Jis buvo su mano tėvu. 
\par 14 Ir tu, jei aš būsiu gyvas, parodyk man Viešpaties gerumą, kad nemirčiau. 
\par 15 Ir neatimk savo maloningumo nuo mano namų per amžius, kai Viešpats išnaikins visus Dovydo priešus žemėje”. 
\par 16 Taip Jehonatanas padarė sandorą su Dovydo namais, sakydamas: “Viešpats teatkeršija visiems Dovydo priešams”. 
\par 17 Jehonatanas dar kartą prisiekė Dovydui, nes mylėjo jį kaip savo sielą. 
\par 18 Tada Jehonatanas tarė: “Rytoj jaunas mėnulis, tavęs pasiges, nes tavo vieta prie stalo bus tuščia. 
\par 19 Po trijų dienų sugrįžk ir eik į tą vietą, kurioje buvai anksčiau pasislėpęs, ir atsisėsk šalia Ezelio akmens. 
\par 20 Aš paleisiu tris strėles į tą pusę, lyg šaučiau į taikinį. 
\par 21 Tada aš pasiųsiu berniuką atnešti strėles. Jei sakysiu berniukui: ‘Strėlės guli šiapus tavęs, surink jas!’, tuomet tau negresia pavojus, kaip Viešpats gyvas. 
\par 22 O jei tarsiu jam: ‘Strėlės guli už tavęs!’ tada eik, nes tave siunčia Viešpats. 
\par 23 O dėl mudviejų pasižadėjimo, tai Viešpats bus tarp tavęs ir manęs amžinai”. 
\par 24 Dovydas pasislėpė laukuose. Jauno mėnulio dieną karalius atsisėdo valgyti. 
\par 25 Karalius kaip paprastai atsisėdo į savo vietą prie sienos, Jehonatanas atsistojo, ir Abneras atsisėdo šalia karaliaus, o Dovydo vieta liko tuščia. 
\par 26 Saulius tą dieną nieko nesakė, manydamas, kad Dovydui kas nors atsitiko ir jis yra susitepęs. 
\par 27 Bet kai Dovydo vieta buvo tuščia kitą dieną, Saulius tarė savo sūnui Jehonatanui: “Kodėl Jesės sūnus nei vakar, nei šiandien neatėjo valgyti?” 
\par 28 Jehonatanas atsakė Sauliui: “Dovydas labai prašė manęs išleisti jį į Betliejų, 
\par 29 sakydamas: ‘Leisk man eiti, nes mūsų šeima aukos mieste auką ir mano brolis liepė man atvykti. Jei radau malonę tavo akyse, išleisk mane aplankyti savo brolių’. Todėl jis neatėjo prie karaliaus stalo”. 
\par 30 Tada Saulius labai užsirūstino ir tarė Jehonatanui: “Tu iškrypėlės ir maištininkės sūnau, argi aš nežinau, kad tu išsirinkai Jesės sūnų savo ir savo motinos gėdai? 
\par 31 Kol Jesės sūnus bus gyvas, tu nebūsi karaliumi ir neturėsi karalystės. Pasiųsk, kad jis būtų atvestas pas mane, nes jis turi mirti”. 
\par 32 Jehonatanas atsakė savo tėvui Sauliui: “Kodėl jis turi būti nužudytas? Ką jis padarė?” 
\par 33 Tada Saulius sviedė ietį, norėdamas jį perdurti. Taip Jehonatanas suprato, kad jo tėvas buvo apsisprendęs Dovydą nužudyti. 
\par 34 Jehonatanas labai supykęs atsikėlė nuo stalo ir nieko nevalgė antrą mėnesio dieną; jis buvo labai susirūpinęs dėl Dovydo, kurį jo tėvas paniekino. 
\par 35 Rytą Jehonatanas išėjo į lauką tuo laiku, kurį buvo sutaręs su Dovydu, ir berniukas buvo su juo. 
\par 36 Jehonatanas sakė berniukui: “Bėk, surink strėles, kurias paleisiu”. Berniukui bėgant, jis paleido strėlę. 
\par 37 Jam nubėgus iki vietos, kur nukrito strėlė, Jehonatanas šaukė: “Ar strėlė nėra už tavęs?” 
\par 38 Ir Jehonatanas šaukė berniukui: “Greičiau! Skubėk! Nesustok!” Jehonatano berniukas surinko strėles ir atnešė savo valdovui. 
\par 39 Berniukas nieko nežinojo. Tik Jehonatanas ir Dovydas žinojo, ką tai reiškia. 
\par 40 Tuomet Jehonatanas padavė savo ginklus berniukui ir liepė juos nunešti į miestą. 
\par 41 Berniukui nuėjus, Dovydas pakilo iš už akmenų, parpuolė veidu į žemę ir tris kartus nusilenkė. Juodu pasibučiavo ir ilgai verkė, bet labiausiai Dovydas. 
\par 42 Pagaliau Jehonatanas tarė Dovydui: “Eik ramybėje, nes mudu prisiekėme Viešpaties vardu, sakydami: ‘Viešpats bus tarp tavęs ir manęs, tarp mano palikuonių ir tavo palikuonių per amžius’ ”.



\chapter{21}


\par 1 Dovydas nuėjo, o Jehonatanas sugrįžo į miestą. 
\par 2 Dovydas atėjo į Nobą pas kunigą Ahimelechą. Ahimelechas, sutikęs Dovydą, nusigando ir klausė: “Kodėl tu atėjai vienas, be palydovų?” 
\par 3 Dovydas atsakė kunigui Ahimelechui: “Karalius man įsakė sutvarkyti vieną reikalą ir sakė man: ‘Niekas neturi žinoti, kokiu reikalu tave siunčiu ir ką tau įsakau’. Aš su savo tarnais susitiksiu sutartoje vietoje. 
\par 4 Dabar, jei turi ką po ranka, duok man­penkis duonos kepalus ar ką surasi”. 
\par 5 Kunigas atsakė Dovydui: “Neturiu paprastos duonos, tik šventos. Kad jaunuoliai būtų susilaikę bent nuo moterų”. 
\par 6 Dovydas atsakė kunigui: “Moterų nelietėme jau trys dienos, nuo to laiko, kai išėjome. Jaunuolių indai šventi, o jei kelias suteptas, tai duona bus šventuose induose”. 
\par 7 Tada kunigas davė jam šventos duonos, nes neturėjo kitos, tik padėtinę, kuri buvo paimta iš Viešpaties akivaizdos pakeičiant ją šviežia duona. 
\par 8 Tą dieną ten buvo edomitas Doegas, vienas Sauliaus tarnų. Jis buvo Sauliaus piemenų vyresnysis. 
\par 9 Dovydas paklausė Ahimelecho: “Ar neturi čia ieties ar kardo? Kadangi karaliaus reikalas buvo labai skubus, nepasiėmiau nei kardo, nei kito ginklo”. 
\par 10 Kunigas tarė: “Kardas filistino Galijoto, kurį nukovei Elos slėnyje, yra čia, įvyniotas į audinį už efodo. Jei nori, pasiimk jį, nes kito čia nėra”. Dovydas tarė: “Nėra jam lygaus, duok jį man”. 
\par 11 Dovydas, bėgdamas nuo Sauliaus, atvyko pas Gato karalių Achišą. 
\par 12 Achišo tarnai klausė: “Argi tai ne Dovydas, krašto karalius? Argi ne apie jį moterys šokdamos dainavo: ‘Saulius nukovė tūkstančius, o Dovydas­dešimtis tūkstančių’ ”. 
\par 13 Dovydas įsidėjo šituos žodžius į širdį ir labai išsigando Gato karaliaus Achišo. 
\par 14 Jis pakeitė savo elgesį jų akyse ir apsimetė pamišėliu. Jis draskė vartus ir varvino seiles ant savo barzdos. 
\par 15 Achišas kalbėjo savo tarnams: “Jūs matote, kad jis beprotis. Kodėl jį atvedėte pas mane? 
\par 16 Ar man neužtenka pamišėlių, kad dar šitą man atvedėte, jog šėltų mano akyse? Nejaugi jį vesite ir į mano namus?”



\chapter{22}


\par 1 Dovydas, išėjęs iš ten, pabėgo į Adulamo olą. Tai išgirdę, jo broliai ir visi tėvo namai atėjo pas jį. 
\par 2 Žmonės, kurie buvo prislėgti, prasiskolinę ir nepatenkinti, rinkosi pas Dovydą; jis tapo jų viršininku. Su juo buvo apie keturis šimtus vyrų. 
\par 3 Iš ten Dovydas nuėjo į Moabo Micpą. Ir jis sakė Moabo karaliui: “Prašau, leisk mano tėvui ir motinai apsistoti pas jus, kol sužinosiu, ką Dievas darys su manimi”. 
\par 4 Jis paliko tėvus pas Moabo karalių; jie gyveno pas jį, kol Dovydas buvo tvirtovėje. 
\par 5 Pranašas Gadas tarė Dovydui: “Nepasilik kalnų tvirtovėje. Keliauk į Judo žemę”. Dovydas išėjo ir atėjo į Hereto mišką. 
\par 6 Saulius sužinojo, kad pasirodė Dovydas ir jo būrys. Tuo metu Saulius sėdėjo Gibėjos aukštumoje po medžiu, laikydamas ietį rankoje; visi jo tarnai stovėjo aplink jį. 
\par 7 Jis tarė savo tarnams: “Benjaminai, klausykite. Ar Jesės sūnus duos jums visiems laukų bei vynuogynų ir jus padarys tūkstantininkais ir šimtininkais, 
\par 8 kad jūs visi susimokėte prieš mane ir nebuvo nė vieno, kuris praneštų man, kad mano sūnus yra susitaręs su Jesės sūnumi? Tarp jūsų nėra nė vieno, kuris būtų manęs gailėjęsis ir man pranešęs, kad mano sūnus sukurstė mano tarną tykoti manęs, kaip šiandien matome”. 
\par 9 Tada edomitas Doegas, stovėjęs su Sauliaus tarnais, tarė: “Aš mačiau Jesės sūnų Nobe pas Ahitubo sūnų Ahimelechą, 
\par 10 kuris atsiklausė Viešpaties dėl jo, davė jam maisto ir filistino Galijoto kardą”. 
\par 11 Saulius pakvietė Ahitubo sūnų kunigą Ahimelechą ir visus jo tėvo namus, kunigus, buvusius Nobe. Jie visi atėjo pas Saulių. 
\par 12 Saulius tarė: “Paklausyk, Ahitubo sūnau”. Tas atsiliepė: “Aš čia, mano valdove”. 
\par 13 “Kodėl tu ir Jesės sūnus susitarėte prieš mane? Tu davei jam duonos, kardą ir klausei Dievą dėl jo, kad jis tykotų manęs, kaip šiandien matome”. 
\par 14 Ahimelechas tarė karaliui: “Kuris iš tavo tarnų yra toks ištikimas, kaip Dovydas, karaliaus žentas, einąs ten, kur tu liepi, ir gerbiamas tavo namuose? 
\par 15 Argi aš tik tada pradėjau klausti Dievą dėl jo? Tebūna tai toli nuo manęs. Nekaltink, karaliau, savo tarno ir mano tėvo namų, nes tavo tarnas nežinojo apie tai nei daug, nei mažai”. 
\par 16 Karalius tarė: “Tu, Ahimelechai, ir visi tavo tėvo namai neišvengsite mirties”. 
\par 17 Karalius įsakė savo sargybai: “Nužudykite Viešpaties kunigus, nes jie yra su Dovydu; jie žinojo, kad jis pabėgo, bet man nepranešė”. Bet karaliaus tarnai nenorėjo žudyti Viešpaties kunigų. 
\par 18 Tada karalius tarė Doegui: “Eik tu ir nužudyk kunigus”. Tą dieną edomitas Doegas puolė ir nužudė aštuoniasdešimt penkis vyrus, nešiojusius lininį efodą. 
\par 19 Nobe jis kardu išžudė vyrus, moteris, vaikus, kūdikius, taip pat jaučius, asilus ir avis. 
\par 20 Ištrūko tik Abjataras, Ahitubo sūnaus Ahimelecho sūnus, ir pabėgo pas Dovydą. 
\par 21 Abjataras pranešė Dovydui, kad Saulius išžudė Viešpaties kunigus. 
\par 22 Dovydas tarė Abjatarui: “Pamatęs ten stovintį edomitą Doegą, aš žinojau, kad jis tikrai praneš Sauliui. Aš kaltas, kad žuvo tavo tėvo giminės. 
\par 23 Pasilik pas mane, nebijok. Kas ieško mano gyvybės, ieško ir tavo. Bet su manimi tu esi saugus”.



\chapter{23}

\par 1 Dovydui pranešė, kad filistinai kariauja prieš Keilą ir plėšia klojimus. 
\par 2 Dovydas klausė Viešpatį: “Ar man eiti ir mušti filistinus?” Viešpats atsakė Dovydui: “Eik, sumušk filistinus ir išgelbėk Keilą”. 
\par 3 Dovydo vyrai jam tarė: “Štai mes bijome jau čia, Judo žemėje. Kaip mes eisime į Keilą prieš filistinų kariuomenę?” 
\par 4 Dovydas dar kartą klausė Viešpatį. Viešpats jam atsakė: “Kelkis ir eik į Keilą; Aš atiduosiu filistinus į tavo rankas”. 
\par 5 Dovydas ir jo vyrai, nuėję į Keilą, kariavo su filistinais, nuvarė jų galvijus ir smarkiai juos sumušė. Taip Dovydas išgelbėjo Keilos gyventojus. 
\par 6 Ahimelecho sūnus Abjataras, bėgdamas pas Dovydą į Keilą, atsinešė ir efodą. 
\par 7 Sauliui buvo pranešta, kad Dovydas atėjo į Keilą. Saulius tarė: “Dievas jį atidavė į mano rankas; jis pats užsidarė įeidamas į miestą su vartais ir užkaiščiais”. 
\par 8 Saulius sušaukė visus vyrus žygiui į Keilą prieš Dovydą ir jo vyrus. 
\par 9 Dovydas sužinojo, kad Saulius slapčia rengia jam pikta, ir tarė kunigui Abjatarui: “Atnešk efodą”. 
\par 10 Dovydas meldėsi: “Viešpatie, Izraelio Dieve, tavo tarnas išgirdo, kad Saulius rengiasi ateiti į Keilą ir sunaikinti miestą dėl manęs. 
\par 11 Ar Keilos gyventojai išduos mane į jo rankas? Ar Saulius ateis, kaip tavo tarnas girdėjo? Viešpatie, Izraelio Dieve, meldžiu, pasakyk savo tarnui”. Viešpats atsakė: “Jis ateis”. 
\par 12 Dovydas klausė: “Ar Keilos vyrai išduos mane ir mano vyrus Sauliui?” Viešpats atsakė: “Išduos”. 
\par 13 Tada Dovydas ir jo vyrai, kurių buvo apie šešis šimtus, išėję iš Keilos, ėjo, kur galėjo eiti. Kai Sauliui buvo pranešta, kad Dovydas pabėgo iš Keilos, jis atšaukė savo žygį. 
\par 14 Dovydas apsistojo dykumos tvirtovėse ir pasiliko kalne Zifo dykumoje. Saulius visą laiką jo ieškojo, tačiau Dievas neatidavė jo į Sauliaus rankas. 
\par 15 Dovydas matė, kad Saulius išėjo ieškoti jo gyvybės, ir Dovydas buvo Zifo tyruose, miške. 
\par 16 Sauliaus sūnus Jehonatanas atėjo pas Dovydą į mišką ir sustiprino jo ranką Dieve. 
\par 17 Ir jis tarė jam: “Nebijok, mano tėvas Saulius tavęs neras. Tu tapsi Izraelio karaliumi, o aš būsiu šalia tavęs. Mano tėvas Saulius tai žino”. 
\par 18 Juodu Viešpaties akivaizdoje padarė sandorą. Dovydas pasiliko miške, o Jehonatanas sugrįžo į savo namus. 
\par 19 Zifiečiai nuėjo pas Saulių į Gibėją ir pranešė: “Dovydas slapstosi pas mus kalnų tvirtovėse, miškuose ir Hachilos kalvose, į pietus nuo Jesimono. 
\par 20 Taigi dabar, karaliau, ateik, kaip tavo siela to trokšta, o mūsų darbas bus atiduoti jį į karaliaus rankas”. 
\par 21 Saulius tarė: “Viešpats telaimina jus, kad manęs pasigailėjote. 
\par 22 Eikite, viską paruoškite ir sužinokite vietą, kur jis yra ir kas jį ten matė; man pranešta, kad jis esąs labai gudrus. 
\par 23 Išžvalgykite ir sužinokite visas slėptuves, kuriose jis slapstosi; tada sugrįžkite pas mane su tikromis žiniomis ir aš eisiu su jumis. Jei jis tebėra krašte, aš surasiu jį visuose Judo tūkstančiuose”. 
\par 24 Jie nuėjo į Zifą pirma Sauliaus. Dovydas ir jo vyrai buvo Maono dykumoje, lygumoje į pietus nuo Jesimono. 
\par 25 Saulius ir jo vyrai išėjo jo ieškoti. Dovydui apie tai buvo pranešta, todėl jis pasitraukė į Maono dykumą prie uolos. Saulius, tai išgirdęs, vijosi Dovydą į Maono dykumą. 
\par 26 Saulius ėjo viena kalno puse, o Dovydas su savo vyrais­kita. Dovydas skubiai traukėsi nuo Sauliaus, nes Saulius su savo vyrais supo Dovydą ir jo vyrus, norėdami juos sugauti. 
\par 27 Tačiau pas Saulių atėjo pasiuntinys su pranešimu: “Skubiai sugrįžk, nes filistinai užpuolė kraštą”. 
\par 28 Saulius liovėsi vijęs Dovydą, sugrįžo ir kariavo su filistinais. Todėl ta vieta pavadinta Perskyrimo uola.



\chapter{24}


\par 1 Dovydas pasitraukė į En Gedžio tvirtoves. 
\par 2 Kai Saulius, nugalėjęs filistinus, sugrįžo, jam buvo pranešta, kad Dovydas yra En Gedžio dykumoje. 
\par 3 Saulius, paėmęs tris tūkstančius rinktinių vyrų iš viso Izraelio, išėjo ieškoti Dovydo ir jo vyrų laukinių ožkų gyvenamose uolose. 
\par 4 Atėjęs prie šalikelės avidžių, kur buvo ola, Saulius į ją įėjo atlikti reikalo. Dovydas ir jo vyrai sėdėjo olos gilumoje. 
\par 5 Dovydo vyrai sakė jam: “Štai diena, apie kurią Viešpats tau kalbėjo: ‘Aš atiduosiu tavo priešą į tavo rankas, kad pasielgtum su juo, kaip tau patinka’ ”. Prislinkęs Dovydas nupjovė Sauliaus apsiausto skverną. 
\par 6 Dovydo širdis smarkiai plakė, nes jis nupjovė Sauliaus skverną. 
\par 7 Jis tarė savo vyrams: “Apsaugok, Viešpatie, kad taip padaryčiau savo valdovui, Viešpaties pateptajam, pakeldamas prieš jį savo ranką, nes jis yra Viešpaties pateptasis”. 
\par 8 Dovydas sulaikė savo vyrus šiais žodžiais ir neleido jiems pakilti prieš Saulių. O Saulius, išėjęs iš olos, ėjo savo keliu. 
\par 9 Dovydas taip pat pakilo iš paskos ir, išlindęs iš olos, šaukė Sauliui: “Karaliau, mano valdove!” Sauliui atsigręžus, Dovydas nusilenkė veidu iki žemės. 
\par 10 Dovydas tarė Sauliui: “Kodėl klausai žmonių kalbų, kad Dovydas nori tau pikta? 
\par 11 Tu pats šiandien matei, kad Viešpats buvo atidavęs tave į mano rankas oloje; mane ragino tave nužudyti, bet aš pasigailėjau tavęs, sakydamas: ‘Nepakelsiu rankos prieš savo valdovą, nes jis yra Viešpaties pateptasis’. 
\par 12 Pažvelk, mano tėve, į savo apsiausto skverną mano rankoje. Iš to, kad aš, atpjaudamas tavo apsiausto skverną, tavęs nenužudžiau, suprask ir žinok, kad manyje nėra nieko pikto ir aš nesu tau nusidėjęs. Bet tu ieškai mano gyvybės, kad ją atimtum. 
\par 13 Viešpats tebūna teisėjas tarp mudviejų ir Viešpats tegul atkeršija tau už mane, bet mano ranka nepakils prieš tave. 
\par 14 Kaip sena patarlė sako: ‘Iš nedorėlių ateina nedorybės’, bet mano ranka nepakils prieš tave. 
\par 15 Prieš ką išėjo Izraelio karalius? Ką tu persekioji? Pastipusį šunį. Blusą. 
\par 16 Viešpats tebūna teisėjas ir tedaro sprendimą tarp manęs ir tavęs. Tegul mato ir gina mano bylą, ir išgelbsti mane nuo tavo rankos”. 
\par 17 Kai Dovydas baigė kalbėti, Saulius tarė: “Ar tai tavo balsas, mano sūnau Dovydai?” Ir Saulius pakėlė savo balsą ir verkė. 
\par 18 Ir jis sakė Dovydui: “Tu esi teisesnis už mane, tu man atlyginai geru, o aš tau atsilyginau piktu. 
\par 19 Šiandien tu parodei, kaip elgiesi su manimi, nes Viešpats buvo atidavęs mane į tavo rankas, tačiau tu manęs nenužudei. 
\par 20 Kas, suradęs savo priešą, paleidžia jį sveiką? Viešpats teatlygina tau geru už tai, ką tu šiandien man padarei. 
\par 21 Dabar tikrai žinau, kad tu tapsi karaliumi ir kad Izraelio karalystė bus įtvirtinta tavo rankose. 
\par 22 Dabar prisiek man Viešpačiu, kad neišnaikinsi mano palikuonių ir neišnaikinsi mano vardo iš mano tėvo namų”. 
\par 23 Dovydas prisiekė Sauliui. Po to Saulius sugrįžo į savo namus, o Dovydas ir jo vyrai pasitraukė į tvirtovę.



\chapter{25}


\par 1 Samuelis mirė, visi izraeli tai susirinko, apraudojo jį ir palaidojo jo namuose Ramoje. Dovydas pakilo ir nuėjo į Parano dykumą. 
\par 2 Maone gyveno vyras, kuris turėjo nuosavybę Karmelyje. Jis buvo labai turtingas: turėjo tris tūkstančius avių ir tūkstantį ožkų. Kartą jis kirpo avis Karmelyje. 
\par 3 Jo vardas buvo Nabalas, o jo žmonos­Abigailė. Moteris buvo išmintinga ir graži, bet jos vyras šiurkštus ir blogo elgesio; jis buvo iš Kalebo namų. 
\par 4 Dovydas išgirdo dykumoje, kad Nabalas kerpa avis. 
\par 5 Jis pasiuntė dešimt jaunuolių ir sakė jiems: “Eikite į Karmelį pas Nabalą ir pasveikinkite jį mano vardu. 
\par 6 Sakykite jam: ‘Ramybė tebūna tau, ramybė tavo namams ir ramybė viskam, ką turi. 
\par 7 Sužinojau, kad kerpamos tavo avys. Tavo piemenys buvo su mumis. Mes jiems nieko blogo nepadarėme ir jiems nieko netrūko, kol jie buvo Karmelyje. 
\par 8 Paklausk savo jaunuolių, ir jie tau patvirtins. Teatranda šitie jaunuoliai malonę tavo akyse, nes atėjome gerą dieną. Todėl prašau, duok, ką ras tavo ranka, savo tarnams ir savo sūnui Dovydui’ ”. 
\par 9 Dovydo jaunuoliai atėję kalbėjo Nabalui visus tuos žodžius Dovydo vardu ir laukė atsakymo. 
\par 10 Nabalas atsakė Dovydo tarnams: “Kas yra Dovydas? Kas yra Jesės sūnus? Šiandien yra daug pabėgusių nuo savo valdovų tarnų. 
\par 11 Argi aš atiduosiu duoną, vandenį ir mėsą, kurią prirengiau kirpėjams, žmonėms, apie kuriuos nežinau, iš kur jie?” 
\par 12 Dovydo jaunuoliai grįžo atgal ir papasakojo jam viską, ką buvo girdėję. 
\par 13 Dovydas tarė savo vyrams: “Kiekvienas prisijuoskite kardą”. Visi apsiginklavo ir maždaug keturi šimtai vyrų sekė Dovydą, o du šimtai pasiliko prie daiktų. 
\par 14 Tuo metu vienas Nabalo tarnų pranešė Nabalo žmonai Abigailei: “Dovydas atsiuntė iš dykumos pasiuntinius mūsų šeimininko pasveikinti, o tas juos išplūdo. 
\par 15 Tie vyrai buvo mums labai geri. Jie mums nepadarė jokios skriaudos, mes nieko nepasigedome per visą laiką, kurį praleidome drauge su jais. 
\par 16 Jie buvo mums siena naktį ir dieną, kai bandą ganėme netoli jų. 
\par 17 Taigi dabar pagalvok ir nuspręsk, ką darysi. Nelaimė tikrai gresia mūsų šeimininkui ir visai jo šeimynai. Jis yra Belialo vaikas, su kuriuo neįmanoma kalbėti”. 
\par 18 Abigailė skubiai paėmė du šimtus duonos kepalų, dvi odines vyno, penkias paruoštas avis, penkis saikus skrudintų grūdų, šimtą džiovintų vynuogių kekių, du šimtus figų pyragaičių ir sukrovė ant asilų. 
\par 19 Ji įsakė savo tarnams: “Eikite pirma manęs, o aš eisiu paskui jus”. Savo vyrui Nabalui ji nieko nesakė. 
\par 20 Jodama ant asilo kalno pašlaite, ji sutiko Dovydą ir jo vyrus, ateinančius priešais. 
\par 21 Dovydas buvo pasakęs: “Veltui saugojau dykumoje visa, kas Nabalui priklauso taip, kad jis nieko nepasigedo. Jis man atsilygino piktu už gera. 
\par 22 Tegul Dievas tai ir dar daugiau padaro su Dovydo priešais, jei iki ryto palikčiau gyvą bent vieną jo vyrų”. 
\par 23 Abigailė, pamačiusi Dovydą, skubiai nulipo nuo asilo, parpuolė prieš Dovydą ir nusilenkė iki žemės. 
\par 24 Parpuolusi prie jo kojų, kalbėjo: “Valdove, ant manęs tebūna kaltė. Meldžiu, leisk savo tarnaitei kalbėti ir išklausyk ją. 
\par 25 Mano valdove, prašau nekreipti dėmesio į tą Belialo žmogų Nabalą, nes koks jo vardas, toks ir jis pats. Nabalas jo vardas ir jis pilnas kvailumo. Aš, tavo tarnaitė, nemačiau savo valdovo jaunuolių, kuriuos buvai atsiuntęs. 
\par 26 Dabar, mano valdove, kaip Viešpats gyvas ir gyva tavo siela, Viešpats sulaikė tave, kad nenusikalstum, praliedamas kraują ir keršydamas savo ranka. Tavo priešai, kurie siekia pikto mano valdovui, tegul tampa kaip Nabalas. 
\par 27 Štai dovanos, kurias tavo tarnaitė atgabeno savo valdovui, atiduok jas jaunuoliams, kurie seka mano valdovą. 
\par 28 Atleisk savo tarnaitei nusikaltimą. Viešpats tikrai pastatys mano valdovui tvirtus namus, nes tu kovoji Viešpaties kovas ir tavyje nerasta blogio per visas tavo dienas. 
\par 29 O jei kas tave persekiotų ir tavo gyvybės ieškotų, tai Viešpats, tavo Dievas, saugos tave, o tavo priešų sielas Jis nusvies kaip mėtykle. 
\par 30 Kai Viešpats įvykdys visa, ką Jis tau pažadėjo, ir tave padarys Izraelio valdovu, 
\par 31 tai nereikės mano valdovui liūdėti ir jo širdis nesigrauš, kad be reikalo praliejai kraują, norėdamas pats atkeršyti. Kai Viešpats padarys gera mano valdovui, atsimink savo tarnaitę”. 
\par 32 Dovydas atsakė Abigailei: “Palaimintas Viešpats, Izraelio Dievas, kuris šiandien atsiuntė tave manęs pasitikti. 
\par 33 Palaimintas tavo patarimas ir palaiminta tu, nes šiandien mane sulaikei nuo kraujo praliejimo ir sukliudei man pačiam už save atkeršyti. 
\par 34 Tikrai, kaip gyvas Viešpats, Izraelio Dievas, kuris man neleido tavęs nuskriausti, jei nebūtum paskubėjusi manęs pasitikti, iki ryto aušros nebūtų likę Nabalui nė vieno vyro”. 
\par 35 Dovydas priėmė iš jos viską, ką ji jam atgabeno, ir tarė: “Eik rami į savo namus. Aš paklausiau tavo balso ir tavęs nepaniekinau”. 
\par 36 Abigailė sugrįžo pas Nabalą. Jis tuo metu buvo iškėlęs namuose karališką puotą. Nabalas buvo linksmas ir labai girtas, todėl ji jam nieko nesakė iki ryto. 
\par 37 Rytą, Nabalui išsiblaivius, jo žmona jam viską papasakojo. Tuomet jo širdis apmirė jame ir jis sustingo kaip akmuo. 
\par 38 Praėjus maždaug dešimčiai dienų, Viešpats ištiko Nabalą, ir jis mirė. 
\par 39 Dovydas, išgirdęs, kad Nabalas mirė, tarė: “Palaimintas Viešpats, kuris įvykdė teismą Nabalui už man padarytą įžeidimą ir apsaugojo mane nuo pikto. Viešpats sugrąžino Nabalo nusikaltimą ant jo galvos”. Ir Dovydas pasiuntė pasiuntinius ir kalbėjo Abigailei, kad nori ją vesti. 
\par 40 Dovydo tarnai, nuėję pas Abigailę į Karmelį, jai kalbėjo: “Dovydas mus atsiuntė pas tave pasakyti, kad jis nori tave vesti”. 
\par 41 Ji nusilenkė iki žemės ir tarė: “Štai tavo tarnaitė, kad tarnautų ir plautų savo valdovo tarnų kojas”. 
\par 42 Abigailė skubiai pasiruošė, užsėdo ant asilo ir su penkiomis tarnaitėmis sekė paskui Dovydo pasiuntinius ir tapo jo žmona. 
\par 43 Dovydas dar vedė Ahinoamą iš Jezreelio, ir jos abi buvo jo žmonos. 
\par 44 Saulius atidavė savo dukterį Mikalę, Dovydo žmoną, Laišo sūnui Palčiui iš Galimo.



\chapter{26}

\par 1 Zifiečiai, atėję į Gibėją, pranešė Sauliui, kad Dovydas slapstosi Hachilos kalvose, prie Jesimono. 
\par 2 Saulius su trimis tūkstančiais Izraelio rinktinių vyrų ėjo į Zifo dykumą ieškoti Dovydo. 
\par 3 Jis pasistatė stovyklą Hachilos kalvoje, prie Jesimono. Dovydas buvo dykumoje ir matė, kad Saulius ėjo į dykumą jo ieškoti. 
\par 4 Tada Dovydas išsiuntė žvalgus ir sužinojo, kad Saulius tikrai atėjo. 
\par 5 Dovydas atėjo į tą vietą, kurioje Saulius buvo pasistatęs stovyklą. Dovydas matė vietą, kur atsigulė Saulius ir Nero sūnus Abneras, jo kariuomenės vadas. Sauliaus palapinė buvo stovyklos viduryje, o kariai miegojo aplinkui. 
\par 6 Dovydas tarė hetitui Ahimelechui ir Cerujos sūnui Abišajui, Joabo broliui: “Kas eis su manimi į Sauliaus stovyklą?” Abišajis atsakė: “Aš eisiu”. 
\par 7 Dovydas ir Abišajis, naktį nuėję tarp žmonių, rado Saulių miegantį palapinėje, o jo ietis buvo įsmeigta į žemę galvūgalyje. Abneras ir kariai gulėjo aplink jį. 
\par 8 Abišajis tarė Dovydui: “Dievas šiandien atidavė tavo priešą į tavo rankas. Leisk man jį vienu smūgiu prismeigti ietimi prie žemės”. 
\par 9 Dovydas atsakė Abišajui: “Nežudyk jo! Kas pakeltų ranką prieš Viešpaties pateptąjį ir liktų nekaltas? 
\par 10 Kaip Viešpats gyvas, Viešpats jį ištiks, ir ateis jo diena mirti, arba jis žus kare! 
\par 11 Apsaugok, Viešpatie, kad pakelčiau savo ranką prieš Viešpaties pateptąjį! Dabar imk jo galvūgalyje įsmeigtą ietį bei vandens ąsotį ir eikime”. 
\par 12 Dovydas paėmė ietį ir vandens ąsotį nuo Sauliaus galvūgalio, ir jie išėjo. Niekas nematė, nežinojo ir nepabudo; jie visi miegojo, nes Viešpats siuntė jiems gilų miegą. 
\par 13 Dovydas perėjo į kitą pusę ir atsistojo kalno viršūnėje iš tolo, kad tarp jų būtų didelis atstumas. 
\par 14 Jis šaukė žmonėms bei Nero sūnui Abnerui: “Argi neatsiliepsi, Abnerai?” Abneras atsiliepė: “Kas tu esi, kad šauki karaliui?” 
\par 15 Dovydas atsakė Abnerui: “Tu esi galingas vyras. Kas tau lygus Izraelyje? Kodėl nesaugojai karaliaus, savo valdovo? Vienas žmogus buvo atėjęs karaliaus, tavo valdovo, nužudyti. 
\par 16 Negerai padarėte. Kaip Viešpats gyvas, jūs verti mirties, nes jūs nesaugojote savo valdovo, Viešpaties pateptojo. Kur dingo karaliaus ietis ir vandens ąsotis, kurie buvo jo galvūgalyje?” 
\par 17 Saulius, atpažinęs Dovydo balsą, tarė: “Ar tai tavo balsas, Dovydai, mano sūnau?” Dovydas atsakė: “Taip, mano valdove karaliau. 
\par 18 Kodėl, mano valdove, persekioji savo tarną? Ką aš padariau? Kuo nusikaltau? 
\par 19 Mano valdove karaliau, prašau, išklausyk savo tarno. Jei Viešpats tave sukurstė prieš mane, tepriima Jis auką, o jei žmonės­tegul jie būna prakeikti Viešpaties akivaizdoje, nes jie išvarė mane, kad neturėčiau dalies Viešpatyje, sakydami: ‘Eik ir tarnauk svetimiems dievams’. 
\par 20 Tenebūna pralietas mano kraujas Viešpaties akivaizdoje, nes Izraelio karalius išėjo ieškoti blusos kaip žmogus, kuris medžioja kurapką kalnuose”. 
\par 21 Saulius atsakė: “Aš nusidėjau. Sugrįžk, mano sūnau Dovydai, aš tau nieko blogo nedarysiu, nes mano siela buvo brangi tavo akyse šiandien. Aš kvailai elgiausi ir labai klydau”. 
\par 22 Dovydas atsakė: “Štai tavo ietis, karaliau! Tegul vienas iš tavo jaunuolių ateina ir ją paima. 
\par 23 Viešpats atlygins kiekvienam už jo teisumą ir ištikimybę. Viešpats buvo atidavęs tave šiandien į mano rankas, bet aš nepakėliau rankos prieš Viešpaties pateptąjį. 
\par 24 Ir štai, kaip šiandien buvo vertinga tavo gyvybė mano akyse, taip vertinga tegul būna ir mano gyvybė Viešpaties akyse; tegul Jis išlaisvina mane iš visų bėdų”. 
\par 25 Saulius atsakė Dovydui: “Būk palaimintas, mano sūnau Dovydai! Tu daug padarysi ir pasieksi”. Po to Dovydas nuėjo savo keliu, o Saulius sugrįžo į savo vietą.



\chapter{27}


\par 1 Dovydas tarė savo širdyje: “Vieną dieną aš žūsiu nuo Sauliaus rankos. Geriausia man būtų pasitraukti į filistinų šalį. Tada Saulius nebeieškos manęs Izraelyje, ir aš išsigelbėsiu nuo jo rankos”. 
\par 2 Dovydas pakilo ir su šešiais šimtais vyrų, buvusių su juo, perėjo pas Gato karalių Achišą, Maocho sūnų. 
\par 3 Ir Dovydas, abi jo žmonos, jezreelietė Ahinoama ir karmelietė Abigailė, Nabalo našlė, ir jo vyrai su savo šeimomis gyveno pas Achišą Gate. 
\par 4 Saulius sužinojo, kad Dovydas pabėgo į Gatą, ir daugiau jo nebeieškojo. 
\par 5 Dovydas tarė Achišui: “Jei radau malonę tavo akyse, prašau, leisk man gyventi kuriame nors mažame mieste. Kodėl tavo tarnas turėtų gyventi su tavimi karaliaus mieste?” 
\par 6 Achišas jam leido gyventi Ciklage. Nuo to laiko Ciklagas priklauso Judo karaliams iki šios dienos. 
\par 7 Dovydas gyveno filistinų šalyje vienerius metus ir keturis mėnesius. 
\par 8 Dovydas su savo vyrais išeidavo ir užpuldavo gešuriečius, girzus ir amalekiečius, kurie nuo senų laikų gyveno toje šalyje nuo Šūro iki Egipto. 
\par 9 Kai Dovydas užpuldavo kraštą, jis nepalikdavo gyvo nei vyro, nei moters, pasiimdavo avis, galvijus, asilus, kupranugarius, drabužius ir grįždavo, ir nueidavo pas Achišą. 
\par 10 Achišas paklausdavo: “Ką šįkart buvote užpuolę?” Dovydas atsakydavo: “Judo pietinę dalį, jerachmeelitų pietų kraštą ir kenitų pietinę dalį”. 
\par 11 Nei vyrų, nei moterų Dovydas nepalikdavo gyvų ir neatsivesdavo jų į Gatą, manydamas: “Kad jie nepraneštų apie mus, sakydami: ‘Taip padarė Dovydas ir taip jis elgiasi visą laiką, gyvendamas filistinų krašte’ ”. 
\par 12 Achišas tikėjo Dovydu, sakydamas: “Jis tapo visiškai nekenčiamas Izraelyje, todėl bus mano tarnas per amžius”.



\chapter{28}

\par 1 Tomis dienomis filistinai surinko savo kariuomenę karui prieš Izraelį. Achišas tarė Dovydui: “Tu su savo vyrais turėsi žygiuoti kartu su manimi į karą”. 
\par 2 Dovydas atsakė Achišui: “Tu dabar sužinosi, ką gali padaryti tavo tarnas”. Achišas tarė Dovydui: “Todėl aš padarysiu tave mano galvos saugotoju visam laikui”. 
\par 3 Samuelis buvo miręs, izraelitai buvo apraudoję ir palaidoję jį Ramoje, jo mieste. Saulius buvo išvaręs iš šalies mirusiųjų dvasių iššaukėjus ir burtininkus. 
\par 4 Filistinai susirinko, atėjo ir pastatė stovyklą Šuneme. Saulius sušaukė visą Izraelį, jie pasistatė stovyklą Gilbojoje. 
\par 5 Kai Saulius pamatė filistinų stovyklą, nusigando ir jo širdis ėmė labai drebėti. 
\par 6 Saulius klausė Viešpaties, bet Viešpats jam neatsakė nei per sapnus, nei per Urimą, nei per pranašus. 
\par 7 Tada Saulius tarė savo tarnams: “Suraskite man moterį, kuri iššaukia mirusiųjų dvasias, kad galėčiau nueiti pas ją patarimo”. Jo tarnai jam atsakė: “En Dore gyvena moteris, iššaukianti mirusiųjų dvasias”. 
\par 8 Saulius, pakeitęs drabužius, su dviem vyrais naktį nuėjo pas moterį. Jis tarė jai: “Paburk man per mirusiojo dvasią; iššauk man tą, kurį aš sakysiu”. 
\par 9 Moteris jam atsakė: “Tu žinai, kad Saulius išnaikino krašte mirusiųjų dvasių iššaukėjus ir burtininkus. Kodėl statai spąstus mano gyvybei, kad mane pražudytum?” 
\par 10 Saulius jai prisiekė Viešpačiu: “Kaip Viešpats gyvas, tu nebūsi nubausta dėl šito”. 
\par 11 Moteris klausė: “Ką turiu tau iššaukti?” Jis atsakė: “Iššauk man Samuelį”. 
\par 12 Moteris, pamačiusi Samuelį, garsiai sušuko ir tarė Sauliui: “Kodėl mane apgavai? Tu esi Saulius!” 
\par 13 Karalius jai tarė: “Nebijok! Ką matei?” Moteris tarė Sauliui: “Aš mačiau dvasią, kylančią iš žemės”. 
\par 14 Karalius vėl klausė: “Kaip ji atrodo?” Ji atsakė: “Kyla senas vyras su apsiaustu”. Saulius suprato, kad tai Samuelis, ir nusilenkė veidu iki žemės. 
\par 15 Samuelis tarė Sauliui: “Kodėl drumsti man ramybę, iššaukdamas mane?” Saulius atsakė: “Esu labai prislėgtas, nes filistinai kariauja prieš mane. Dievas atsitraukė nuo manęs ir man nebeatsako nei per sapnus, nei per pranašus, todėl pasišaukiau tave, kad pasakytum, ką man daryti”. 
\par 16 Samuelis atsakė: “Kodėl mane klausi, jei Viešpats atsitraukė nuo tavęs ir tapo tavo priešu? 
\par 17 Viešpats padarė, kaip Jis buvo per mane kalbėjęs. Jis atėmė iš tavęs karalystę ir ją atidavė tavo artimui Dovydui. 
\par 18 Kadangi tu nepaklusai Viešpaties balsui ir neįvykdei Viešpaties rūstybės amalekiečiams, Viešpats šiandien tau taip padarė. 
\par 19 Be to, Viešpats atiduos su tavimi ir Izraelį į filistinų rankas. Rytoj tu ir tavo sūnūs būsite pas mane. Taip pat ir Izraelio pulkus Viešpats atiduos į filistinų rankas”. 
\par 20 Saulius, labai nusigandęs dėl Samuelio žodžių, staiga visu savo ūgiu krito ant žemės. Jis neteko jėgų, nes visą dieną ir naktį buvo nevalgęs. 
\par 21 Moteris, priėjusi prie Sauliaus ir pamačiusi, kad jis labai išsigandęs, tarė: “Tavo tarnaitė paklausė tavęs, statydama į pavojų savo gyvybę, ir įvykdė, ko tu prašei. 
\par 22 Taigi dabar tu paklausyk savo tarnaitės: aš atnešiu tau kąsnelį duonos, kad turėtum jėgų ir galėtum eiti savo keliu”. 
\par 23 Bet jis atsisakė, sakydamas: “Aš nevalgysiu”. Tačiau tarnai kartu su moterimi įkalbėjo jį, ir jis paklausė jų. Jis atsikėlė nuo žemės ir atsisėdo ant lovos. 
\par 24 Moteris turėjo riebų veršį savo namuose. Ji skubiai papjovė jį, paėmusi miltų suminkė ir iškepė neraugintos duonos, 
\par 25 ir atnešė tai Sauliui bei jo tarnams. Pavalgę jie atsikėlė ir tą pačią naktį išėjo.
Online Parallel Study Bible



\chapter{29}


\par 1 Visi filistinų būriai susirinko Afeke, o izraelitai pasistatė stovyklą Jezreelyje prie versmės. 
\par 2 Filistinų kunigaikščiai ėjo su šimtais ir tūkstančiais, o Dovydas ir jo vyrai ėjo paskutinėse eilėse su Achišu. 
\par 3 Tada filistinų kunigaikščiai kalbėjo: “Ką šitie hebrajai čia daro?” Achišas tarė filistinų kunigaikščiams: “Tai Dovydas, Izraelio karaliaus Sauliaus tarnas, kuris jau seniai su manimi ir aš nieko blogo jame nepastebėjau nuo jo atėjimo pas mane”. 
\par 4 Filistinų kunigaikščiai supyko ant Achišo ir tarė: “Siųsk tą vyrą atgal! Tegul grįžta į vietą, kurią tu jam paskyrei, ir neina su mumis į mūšį, kad jo metu netaptų mums priešu. Nes kaip jis galėtų įsiteikti savo valdovui, jei ne šitų vyrų galvomis. 
\par 5 Argi ne šitas Dovydas, apie kurį dainuodavo šokdami: ‘Saulius nukovė tūkstančius, o Dovydas­dešimtis tūkstančių?’ ” 
\par 6 Tada Achišas, pasišaukęs Dovydą, jam tarė: “Kaip Viešpats gyvas, tu esi sąžiningas ir tavo įėjimas ir išėjimas su mano pulkais priimtinas man, nes aš nieko blogo neradau tavyje nuo to laiko, kai atėjai pas mane, iki šios dienos. Bet kunigaikščiams tu nepatinki. 
\par 7 Taigi grįžk ramybėje, kad nepiktintum filistinų kunigaikščių”. 
\par 8 Dovydas atsakė Achišui: “Ką aš padariau? Ką atradai savo tarne per tą laiką, kai esu su tavimi, kad negaliu eiti į karą prieš mano valdovo karaliaus priešus?” 
\par 9 Achišas atsakė Dovydui: “Tikrai mano akyse tu esi geras kaip Dievo angelas, bet filistinų kunigaikščiai pasakė: ‘Jis neis su mumis į mūšį’. 
\par 10 Anksti rytą atsikelk su savo valdovo tarnais, kurie yra atėję su tavimi, ir iškeliaukite, kai tik prašvis”. 
\par 11 Dovydas ir jo vyrai atsikėlė rytą ir pasiruošė keliauti atgal į filistinų šalį, o filistinai išėjo į Jezreelį.



\chapter{30}

\par 1 Kai Dovydas ir jo vyrai trečią dieną sugrįžo į Ciklagą, amalekiečiai buvo įsiveržę iš pietų į Ciklagą, užėmę jį ir sudeginę. 
\par 2 Jie išvedė į nelaisvę moteris; nė vienos nenužudė nuo vyriausios iki jauniausios, bet išsivarė su savimi ir nuėjo savo keliu. 
\par 3 Dovydas ir jo vyrai, grįžę į miestą, rado jį sudegintą, o jų žmonas, sūnus ir dukteris išvestus nelaisvėn. 
\par 4 Tada Dovydas ir su juo buvusieji žmonės pakėlė balsus ir verkė, kol nebeliko jėgų verkti. 
\par 5 Abi Dovydo žmonos irgi buvo išvestos nelaisvėn: jezreelietė Ahinoama ir Abigailė, karmeliečio Nabalo našlė. 
\par 6 Dovydas buvo labai nuliūdęs, nes žmonės tarėsi jį užmušti akmenimis; visi žmonės sielvartavo dėl savo sūnų ir dukterų. Bet Dovydas sustiprino save Viešpatyje, savo Dieve. 
\par 7 Dovydas paprašė kunigą Abjatarą, Ahimelecho sūnų, atnešti efodą. Abjataras atnešė efodą Dovydui. 
\par 8 Dovydas klausė Viešpaties, sakydamas: “Ar man vytis tą gaują? Ar aš juos pavysiu?” Viešpats atsakė: “Vykis! Tikrai pavysi ir išvaduosi belaisvius”. 
\par 9 Taip Dovydas ir šeši šimtai jo vyrų išėjo. Pasiekę Besoro upelį, pavargę kariai pasiliko prie jo. 
\par 10 Dovydas su keturiais šimtais vyrų toliau vijosi, o du šimtai pasiliko, kadangi buvo taip pavargę, jog neįstengė perbristi upelio. 
\par 11 Laukuose jie rado egiptietį, atvedė jį pas Dovydą ir davė jam duonos valgyti ir vandens atsigerti. 
\par 12 Ir jie davė jam džiovintų figų ir dvi kekes džiovintų vynuogių. Pavalgęs jis atsigavo, nes tris paras buvo nevalgęs ir negėręs. 
\par 13 Tuomet Dovydas klausė jį: “Kam tu priklausai ir iš kur esi?” Jis atsakė: “Aš esu egiptietis, vieno amalekiečio vergas. Mano šeimininkas paliko mane, nes prieš tris dienas susirgau. 
\par 14 Mes buvome įsiveržę į pietinį keretų kraštą ir į Judą, į pietines Kalebo žemes, ir mes sudeginome Ciklagą”. 
\par 15 Dovydas jam tarė: “Ar tu gali mus nuvesti prie to būrio?” Tas atsakė: “Prisiek man Dievu, kad manęs nenužudysi ir negrąžinsi mano šeimininkui, tada nuvesiu tave pas juos”. 
\par 16 Kai jis nuvedė juos, amalekiečiai buvo pasklidę po visą kraštą, valgė ir gėrė švęsdami, nes buvo paėmę didelį grobį, plėšdami filistinų ir Judo žemes. 
\par 17 Dovydas mušė juos nuo sutemų iki kitos dienos vakaro; nė vienas iš jų neištrūko, išskyrus keturis šimtus jaunuolių, kurie pabėgo, užsėdę ant kupranugarių. 
\par 18 Dovydas atsiėmė visa, ką amalekiečiai buvo pagrobę, ir išvadavo abi savo žmonas. 
\par 19 Nieko jie nepasigedo, mažo ar didelio: nei sūnų, nei dukterų, nei jokių daiktų, kuriuos amalekiečiai buvo paėmę; Dovydas viską atsiėmė. 
\par 20 Dovydas paėmė visas jų avis ir galvijus ir varė juos pirma savo galvijų, sakydamas: “Tai Dovydo grobis”. 
\par 21 Dovydas atėjo iki tų dviejų šimtų vyrų, kurie buvo išvargę ir nebegalėjo sekti paskui Dovydą ir buvo palikti prie Besoro upelio. Jie išėjo pasitikti Dovydo ir jo vyrų, ir Dovydas priartėjęs juos pasveikino. 
\par 22 Tada kai kurie pikti vyrai, Belialo žmonės, iš tų, kurie ėjo su Dovydu, kalbėjo: “Jie nėjo su mumis, todėl mes jiems nieko neduosime iš atgauto grobio, tik kiekvienam žmoną ir vaikus, kad, juos pasiėmę, keliautų sau”. 
\par 23 Dovydas tarė: “Mano broliai, negalima taip elgtis su tuo, ką Viešpats mums davė. Jis mus saugojo ir atidavė tą būrį, kuris mus užpuolė, į mūsų rankas. 
\par 24 Kas pritars jums šiuo klausimu? Ką gauna ėjęs į mūšį, tą gaus ir tas, kuris saugojo mantą; jie turi pasidalinti po lygiai”. 
\par 25 Taip tas nuostatas buvo įvestas Izraelyje ir galioja iki šios dienos. 
\par 26 Dovydas, sugrįžęs į Ciklagą, pasiuntė dalį grobio Judo vyresniesiems, savo draugams, sakydamas: “Štai jums dovana iš Viešpaties priešų grobio”. 
\par 27 Jis siuntė į Betelį, į pietinį Ramotą, į Jatyrą, 
\par 28 į Aroerą, į Sifmotą, į Eštemoją, 
\par 29 į Rachalą, į jerachmeelitų ir kenitų miestus, 
\par 30 į Hormą, į Bor Ašaną, į Atachą, 
\par 31 į Hebroną ir visas vietas, kur Dovydas buvo buvęs su savo vyrais.



\chapter{31}


\par 1 Filistinai kariavo su Izraeliu. Izraelio vyrai bėgo nuo filistinų ir krito nužudyti ant Gilbojos kalno. 
\par 2 Filistinai pavijo Saulių ir jo sūnus ir nužudė Jehonataną, Abinadabą ir Malkišūvą. 
\par 3 Vyko smarki kova prieš Saulių, ir šauliai pataikė į Saulių ir jį sunkiai sužeidė. 
\par 4 Tada Saulius tarė savo ginklanešiui: “Išsitrauk kardą ir juo mane perverk, kad šitie neapipjaustytieji atėję nepervertų ir neišniekintų manęs”. Bet jo ginklanešys nesutiko, nes jis labai bijojo. Tada Saulius, paėmęs savo kardą, krito ant jo. 
\par 5 Jo ginklanešys pamatęs, kad Saulius miręs, irgi puolė ant savo kardo ir mirė kartu. 
\par 6 Taip mirė Saulius, jo trys sūnūs, ginklanešys ir visi jo vyrai tą pačią dieną. 
\par 7 Izraelitai, kurie gyveno anapus slėnio ir kitoje pusėje Jordano, pamatę, kad izraelitai pabėgo ir Saulius bei jo sūnūs mirę, paliko miestus ir bėgo. Atėję filistinai apsigyveno juose. 
\par 8 Kitą dieną filistinai, atėję apiplėšti užmuštųjų, rado Saulių ir jo tris sūnus žuvusius ant Gilbojos kalno. 
\par 9 Jie nukirto jo galvą, nuvilko šarvus ir nešiojo po visą filistinų kraštą, kad praneštų apie pergalę savo stabų šventyklose ir tarp žmonių. 
\par 10 Ir jie padėjo jo šarvus Astartės šventykloje, o jo lavoną pakabino prie Bet Šeano miesto sienos. 
\par 11 Jabeš Gileado gyventojai išgirdo, ką filistinai padarė Sauliui. 
\par 12 Pakilo jų drąsiausieji vyrai ir ėjo visą naktį; nuėję nuėmė Sauliaus ir jo sūnų lavonus nuo Bet Šeano sienos, parnešė į Jabešą ir juos ten sudegino. 
\par 13 Jų kaulus paėmė ir palaidojo po medžiu Jabeše, ir jie pasninkavo septynias dienas.



\end{document}