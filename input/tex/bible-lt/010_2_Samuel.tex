\begin{document}

\title{
\par 2 Samuel}

\chapter{1}


\par 1 Atsitiko, kad Sauliui mirus, Dovydas, nugalėjęs amalekiečius, sugrįžo ir praleido dvi dienas Ciklage. 
\par 2 Trečią dieną iš Sauliaus stovyklos atėjo vyras su perplėštais drabužiais ir žemėmis apibarstyta galva. Atėjęs pas Dovydą, jis puolė ant žemės ir išreiškė jam pagarbą. 
\par 3 Dovydas jį paklausė: “Iš kur ateini?” Tas jam atsakė: “Pabėgau iš Izraelio stovyklos”. 
\par 4 Dovydas tarė: “Papasakok man, kas ten atsitiko”. Tas atsakė: “Žmonės pabėgo iš mūšio, daugelis žuvo, taip pat ir Saulius su sūnumi Jehonatanu mirė”. 
\par 5 Tuomet Dovydas klausė jaunuolį, kuris jam pranešė: “Iš kur žinai, kad Saulius ir jo sūnus Jehonatanas mirė?” 
\par 6 Jaunuolis, kuris jam tai pranešė, atsakė: “Visai atsitiktinai užėjau ant Gilbojos kalno, žiūriu, Saulius pasirėmęs ant ieties, o karo vežimai ir raiteliai skuba prie jo. 
\par 7 Atsigręžęs ir mane pamatęs, jis pašaukė mane. Aš atsiliepiau: ‘Aš čia!’ 
\par 8 Jis klausė: ‘Kas tu esi?’ Aš atsakiau jam: ‘Aš esu amalekietis’. 
\par 9 Tada jis man tarė: ‘Ateik ir nužudyk mane; aš kankinuosi, bet gyvybė dar yra manyje’. 
\par 10 Priėjau prie jo ir nužudžiau, nes buvau tikras, kad jis neišgyvens po savo kritimo. Nuėmiau karūną nuo jo galvos ir apyrankę nuo jo rankos ir atnešiau juos čia savo valdovui”. 
\par 11 Tuomet Dovydas, nutvėręs savo drabužius, perplėšė juos, taip padarė ir visi vyrai, buvę su juo. 
\par 12 Jie iki vakaro verkė, pasninkavo ir liūdėjo Sauliaus, jo sūnaus Jehonatano ir Viešpaties tautos, Izraelio namų, nes jie žuvo nuo kardo. 
\par 13 Dovydas tarė jaunuoliui, kuris jam tai pranešė: “Iš kur esi?” Tas atsakė: “Aš esu ateivio amalekiečio sūnus”. 
\par 14 Dovydas jam tarė: “Kaip išdrįsai pakelti savo ranką, kad nužudytum Viešpaties pateptąjį?” 
\par 15 Tada jis, pasišaukęs vieną iš savo jaunuolių, tarė: “Ateik ir užmušk jį”. Tas smogė jam, ir jis mirė. 
\par 16 Dovydas pasakė jam: “Tavo kraujas kris ant tavo galvos, nes tavo paties žodžiai paliudijo prieš tave, kai pasakei: ‘Aš nužudžiau Viešpaties pateptąjį’ ”. 
\par 17 Dovydas giedojo šitą raudą apie Saulių ir jo sūnų Jehonataną, 
\par 18 norėdamas, kad Judo vaikai mokytųsi Lanko giesmės, kaip parašyta Josaro knygoje. 
\par 19 “Tavo šlovė, o Izraeli, žuvo tavo aukštumose, krito galingieji! 
\par 20 Nepasakokite Gate, neskelbkite Aškelono gatvėse, kad nesilinksmintų filistinų dukterys, nedžiūgautų neapipjaustytųjų dukterys. 
\par 21 Jūs, Gilbojos kalnai, tenekrinta ant jūsų nei rasa, nei lietus, kad nebūtų ant jūsų derlingų laukų.Ten buvo pamestas galingųjų skydas, skydas Sauliaus, lyg jis nebūtų buvęs pateptas. 
\par 22 Be nukautųjų kraujo, be galingųjų taukų, Jehonatano lankas nesugrįždavo, ir Sauliaus kardas negrįždavo tuščias. 
\par 23 Saulius ir Jehonatanas, mieli ir brangūs, kartu gyveno ir mirtyje jie nebuvo perskirti. Jie buvo greitesni negu ereliai, stipresni negu liūtai. 
\par 24 Izraelio dukterys, raudokite Sauliaus, kuris jus aprengdavo purpuru su papuošalais, apsagstydavo auksiniais pagražinimais jūsų drabužius. 
\par 25 Krito galingieji kovos įkarštyje. Jehonatanas nukautas ant kalno. 
\par 26 Aš gailiuosi tavęs, mano broli Jehonatanai! Labai brangus tu man buvai. Tavo meilė man buvo nuostabesnė už moterų meilę. 
\par 27 Krito galingieji, sulaužyti jų ginklai”.


\chapter{2}


\par 1 Po to Dovydas klausė Viešpatį, sakydamas: “Ar man eiti į kurį nors Judo miestą?” Viešpats atsakė: “Eik”. Dovydas klausė: “Kur man eiti?” Jis atsakė: “Į Hebroną”. 
\par 2 Dovydas išėjo ten su savo abiem žmonom: jezreeliete Ahinoama ir karmeliete Nabalo našle Abigaile. 
\par 3 Dovydas atsivedė ir savo vyrus su jų šeimomis, ir jie apsigyveno Hebrono miestuose. 
\par 4 Judo vyrai atėję patepė Dovydą Judo karaliumi. Dovydas, sužinojęs, kad Jabeš Gileado vyrai palaidojo Saulių, 
\par 5 siuntė pasiuntinius į Jabeš Gileadą ir sakė jiems: “Viešpats telaimina jus, kad parodėte pagarbą savo valdovui Sauliui ir jį palaidojote. 
\par 6 Viešpats tebūna jums malonus ir ištikimas. Aš irgi jums darysiu gera už tai, kad jūs taip padarėte. 
\par 7 Todėl dabar tegul jūsų rankos būna sustiprintos ir būkite drąsūs. Jūsų valdovas Saulius mirė, o Judo giminė patepė mane savo karaliumi”. 
\par 8 Bet Sauliaus kariuomenės vadas Nero sūnus Abneras paėmė Sauliaus sūnų Išbošetą ir atvedė į Machanaimą. 
\par 9 Isbosetas tapo karaliumi Gileado, ašuriečių, Jezreelio, Efraimo, Benjamino ir viso Izraelio. 
\par 10 Jis buvo keturiasdešimties metų amžiaus, pradėdamas karaliauti Izraelyje, ir karaliavo dvejus metus. Bet Judo namai sekė Dovydą. 
\par 11 Dovydas buvo Hebrone Judo namų karaliumi septynerius metus ir šešis mėnesius. 
\par 12 Nero sūnus Abneras ir Sauliaus sūnus Išbošetas išėjo iš Machanaimo į Gibeoną. 
\par 13 Cerujos sūnus Joabas ir Dovydo vyrai išėjo iš Hebrono ir susitiko su jais prie Gibeono tvenkinio. Vieni sustojo vienoje tvenkinio pusėje, o kiti­kitoje. 
\par 14 Abneras tarė Joabui: “Tegul jaunuoliai išeina ir kovoja mūsų akivaizdoje”. Joabas atsakė: “Tegul išeina”. 
\par 15 Dvylika benjaminų, Sauliaus sūnaus Išbošeto tarnų, ir dvylika Dovydo tarnų išėjo vieni prieš kitus. 
\par 16 Kiekvienas, nutvėręs savo priešą už galvos, įsmeigė kardą jam į šoną. Taip jie žuvo visi kartu. Todėl ta vieta Gibeone vadinama Aštrių kardų lauku. 
\par 17 Tą dieną buvo labai smarki kova. Dovydo vyrai nugalėjo Abnerą ir Izraelio vyrus. 
\par 18 Ten buvo trys Cerujos sūnūs: Joabas, Abišajas ir Asaelis. Asaelis greitai bėgdavo lyg laukinė stirna. 
\par 19 Jis pradėjo vytis Abnerą ir nepasuko nei į dešinę, nei į kairę nuo jo. 
\par 20 Abneras atsigręžė ir paklausė: “Ar tu esi Asaelis?” Jis atsakė: “Aš”. 
\par 21 Abneras jam tarė: “Vykis kitą ir pasigavęs pasiimk jo ginklus”. Tačiau Asaelis nenorėjo pasitraukti nuo jo. 
\par 22 Abneras vėl tarė Asaeliui: “Suk į šalį nuo manęs, kad nebūčiau priverstas tave nukauti! Kaip tada galėčiau pažiūrėti tavo broliui Joabui į akis?” 
\par 23 Bet Asaelis nenorėjo pasukti. Tada Abneras taip jam smogė ieties bukuoju galu į pilvą, kad ietis išlindo per nugarą. Jis krito ir mirė vietoje. Visi, atėję į tą vietą, kur žuvo Asaelis, sustodavo. 
\par 24 Joabas ir Abišajas vijosi Abnerą. Saulei nusileidus, jie pasiekė Amos kalvą, prieš Giachą, prie kelio į Gibeono dykumą. 
\par 25 Benjaminai, Abnero kariai, susirinko ir sustojo ant kalvos viršūnės. 
\par 26 Tada Abneras pašaukė Joabą ir tarė: “Ar amžinai naikins kardas? Argi nežinai, kad tai liūdnai baigsis? Kada įsakysi nustoti žmonėms persekioti savo brolius?” 
\par 27 Joabas atsakė: “Kaip Dievas gyvas, jei nebūtum prakalbėjęs, tai persekiojimas būtų pasibaigęs tik rytą”. 
\par 28 Joabas sutrimitavo, ir visi žmonės sustojo, nebesivijo Izraelio ir nebekovojo. 
\par 29 Abneras ir jo vyrai visą naktį ėjo lyguma, persikėlė per Jordaną, perėjo per visą Bitroną ir atėjo į Machanaimą. 
\par 30 Joabas grįžo iš Abnero persekiojimo. Patikrinęs visus karius, iš Dovydo vyrų pasigedo devyniolikos vyrų ir Asaelio. 
\par 31 Bet Dovydo žmonės nukovė tris šimtus šešiasdešimt Benjamino ir Abnero vyrų. 
\par 32 Jie palaidojo Asaelį jo tėvo kape, Betliejuje. Joabas ir jo vyrai ėjo visą naktį ir, brėkštant dienai, pasiekė Hebroną.



\chapter{3}

\par 1 Tarp Sauliaus ir Dovydo namų karas buvo ilgas; Dovydas vis stiprėjo, o Sauliaus namai silpnėjo. 
\par 2 Hebrone Dovydas susilaukė sūnų: pirmagimis Amnonas iš jezreelietės Ahinoamos; 
\par 3 antrasis­Kileabas iš karmeliečio Nabalo našlės Abigailės; trečiasis­Abšalomas, Gešūro karaliaus Talmajo dukters Maakos sūnus; 
\par 4 ketvirtasis­Adonijas, Hagitos sūnus; penktasis­Šefatija, Abitalės sūnus, 
\par 5 ir šeštasis­Itramas iš Dovydo žmonos Eglos. Šitie gimė Dovydui Hebrone. 
\par 6 Vykstant karui tarp Sauliaus namų ir Dovydo namų, Abneras buvo žymiausias tarp Sauliaus šalininkų. 
\par 7 Saulius turėjo sugulovę Ricpą, Ajo dukterį. Kartą Sauliaus sūnus Išbošetas tarė Abnerui: “Kodėl tu įėjai pas mano tėvo sugulovę?” 
\par 8 Abneras labai supyko dėl tų Išbošeto žodžių ir tarė: “Ar aš esu šuns galva? Aš buvau prieš Judą, kad iki šios dienos daryčiau gera tavo tėvo Sauliaus namams, jo broliams bei draugams, ir neatidaviau tavęs į Dovydo rankas, o tu dabar mane apkaltinai dėl tos moters. 
\par 9 Tegul Dievas padaro Abnerui tai ir dar daugiau, jei aš nepadarysiu, ką Viešpats prisiekė Dovydui: 
\par 10 atimti karalystę iš Sauliaus namų ir įtvirtinti Dovydą karaliumi Izraeliui ir Judui nuo Dano iki Beer Šebos”. 
\par 11 Išbošetas negalėjo nė žodžio atsakyti Abnerui, nes jo bijojo. 
\par 12 Abneras per pasiuntinius sakė Dovydui: “Kam priklauso šita žemė? Sudaryk sąjungą su manimi, ir mano ranka bus su tavimi ir atves pas tave visą Izraelį”. 
\par 13 Dovydas atsakė: “Gerai, aš sudarysiu sąjungą su tavimi, bet su viena sąlyga: tu neišvysi mano veido, jei ateidamas neatvesi Sauliaus dukters Mikalės”. 
\par 14 Dovydas siuntė pasiuntinius pas Sauliaus sūnų Išbošetą ir sakė: “Atiduok mano žmoną Mikalę, kurią gavau už šimtą filistinų odelių”. 
\par 15 Išbošetas pasiuntė ir paėmė ją iš jos vyro, Laišo sūnaus Paltielio. 
\par 16 Jos vyras ją lydėjo verkdamas iki Bahurimo. Abneras jam pasakė: “Grįžk!”, ir jis grįžo. 
\par 17 Abneras kalbėjo Izraelio vyresniesiems: “Jau seniai jūs norėjote turėti Dovydą karaliumi. 
\par 18 Dabar padarykite tai, nes Viešpats kalbėjo Dovydui: ‘Savo tarno Dovydo ranka išgelbėsiu savo tautą Izraelį iš filistinų ir iš visų jo priešų’ ”. 
\par 19 Abneras taip pat kalbėjo ir Benjaminui. Po to Abneras vyko pas Dovydą į Hebroną pranešti jam, kas atrodė priimtina Izraeliui ir Benjamino namams. 
\par 20 Abneras atėjo pas Dovydą į Hebroną su dvidešimt vyrų. Dovydas suruošė Abnerui ir jo palydovams puotą. 
\par 21 Abneras tarė Dovydui: “Aš surinksiu pas tave, mano valdove, visą Izraelį, kad jie sudarytų su tavimi sutartį ir tu jiems karaliautum, kaip geidžia tavo širdis”. Dovydas išleido Abnerą. 
\par 22 Tuo metu Dovydo tarnai ir Joabas grįžo iš karo žygio ir parsigabeno didelį grobį, bet Abnero nebebuvo pas Dovydą Hebrone; jis buvo išvykęs ramybėje. 
\par 23 Joabui, sugrįžusiam su kariuomene, buvo pranešta: “Buvo atvykęs Nero sūnus Abneras pas karalių, ir šis išleido jį eiti ramybėje”. 
\par 24 Joabas, atėjęs pas karalių, klausė: “Ką padarei? Abneras buvo atėjęs pas tave. Kodėl jį išleidai? 
\par 25 Argi tu nepažįsti Nero sūnaus Abnero, kad jis atėjo tavęs apgauti ir sužinoti tavo įėjimą, tavo išėjimą ir visa, ką tu darai?” 
\par 26 Joabas, išėjęs iš Dovydo, siuntė vyrus paskui Abnerą; tie sugrąžino jį nuo Siros šulinio. Dovydas apie tai nieko nežinojo. 
\par 27 Abnerui sugrįžus į Hebroną, Joabas pasivedė jį nuošaliai į tarpuvartę, tarsi norėdamas slaptai pasikalbėti, ir keršydamas už savo brolio Asaelio kraują dūrė jam į pilvą taip, kad jis mirė, 
\par 28 Dovydas, tai išgirdęs, tarė: “Aš ir mano karalystė esame nekalti prieš Viešpatį dėl Nero sūnaus Abnero kraujo. 
\par 29 Tas kraujas tekrinta ant Joabo ir jo tėvo namų; Joabo giminėje tegul netrūksta paliegėlių, raupsuotų, luošų, karuose žūstančių ir badaujančių”. 
\par 30 Joabas ir jo brolis Abišajas nužudė Abnerą dėl to, kad Abneras buvo nužudęs jų brolį Asaelį mūšyje Gibeone. 
\par 31 Dovydas įsakė Joabui ir visiems žmonėms: “Perplėškite savo drabužius, užsidėkite ašutines ir raudokite Abnero”. Karalius Dovydas ėjo paskui karstą. 
\par 32 Jie palaidojo Abnerą Hebrone, ir karalius balsiai verkė prie jo kapo, verkė ir visi žmonės. 
\par 33 Karalius raudojo dėl Abnero, sakydamas: “Ar Abneras turėjo mirti tokia mirtimi? 
\par 34 Tavo rankos nebuvo surištos ir kojos nebuvo sukaustytos grandinėmis. Tu kritai, kaip žmogus krinta nuo piktadario”. Ir visi žmonės vėl verkė jo. 
\par 35 Žmonės bandė prikalbėti Dovydą valgyti tą dieną, bet Dovydas prisiekė: “Tegul Dievas padaro man tai ir dar daugiau, jei aš, prieš saulei nusileidžiant, paragaučiau duonos ar ko nors kito”. 
\par 36 Tauta sužinojo, ir tai jiems patiko, kaip patikdavo viskas, ką darė karalius. 
\par 37 Tą dieną visa Izraelio tauta suprato, kad tai buvo ne karaliaus sumanymas nužudyti Nero sūnų Abnerą. 
\par 38 Karalius tarė savo tarnams: “Argi nežinote, kad šiandien Izraelyje žuvo kunigaikštis ir garbingas žmogus? 
\par 39 Aš šiandien dar esu silpnas, nors pateptas karaliumi; šitie vyrai, Cerujos sūnūs, yra per stiprūs man. Viešpats tegul atlygina piktadariui už jo piktadarystę!”



\chapter{4}


\par 1 Kai Sauliaus sūnus išgirdo, kad Abneras žuvo Hebrone, nusviro jo rankos, ir visas Izraelis susijaudino. 
\par 2 Sauliaus sūnus turėjo du vyrus, pulkų vadus Baaną ir Rechabą, beerotiečio Rimono sūnus iš Benjamino giminės, nes Beerotas priklausė Benjaminui. 
\par 3 Beerotiečiai atbėgo į Gitaimus ir ten liko ateiviais iki šios dienos. 
\par 4 Sauliaus sūnus Jehonatanas turėjo luošą sūnų. Jis buvo penkerių metų, kai iš Jezreelio atėjo žinia apie Saulių ir Jehonataną. Jo auklė, paėmusi jį, bėgo. Jai beskubant, jis krito ir tapo luošas. Jo vardas buvo Mefi Bošetas. 
\par 5 Beerotiečio Rimono sūnūs Rechabas ir Baana atėjo pačioje dienos kaitroje į Išbošeto namus, kai tas miegojo ant lovos vidudienį. 
\par 6 Jie įėjo į jo namus tarsi kviečių pasiimti, nudūrė jį į pilvą ir pabėgo. 
\par 7 Kai jie įėjo į namus, Išbošetas miegojo ant lovos savo miegamajame. Jie užmušė jį, nukirto galvą ir, ja nešini, ėjo visą naktį per dykumą. 
\par 8 Atnešę galvą į Hebroną pas Dovydą, tarė jam: “Štai galva Sauliaus sūnaus, tavo priešo Išbošeto, kuris ieškojo tavo gyvybės. Šiandien Viešpats atkeršijo Sauliui ir jo palikuonims už mūsų karalių”. 
\par 9 Dovydas tarė Rechabui ir jo broliui Baanai, beerotiečio Rimono sūnums: “Kaip gyvas Viešpats, kuris išgelbėjo mano sielą iš visokių nelaimių, 
\par 10 tą, kuris man pranešė apie Sauliaus mirtį, galvodamas, kad atnešė man gerą žinią, aš nutvėriau ir nužudžiau Ciklage, užuot jį apdovanojęs už tokią žinią. 
\par 11 Juo labiau, kai piktadariai nužudė teisų vyrą jo namuose, gulintį lovoje. Ar aš neturiu pareikalauti jo kraujo iš jūsų rankų, pašalindamas jus nuo žemės paviršiaus?” 
\par 12 Dovydas įsakė jaunuoliams, ir tie nužudė juos, nukirto jiems rankas ir kojas ir pakorė Hebrone prie tvenkinio. O Išbošeto galvą jie palaidojo Hebrone, Abnero kape.



\chapter{5}

\par 1 Visos Izraelio giminės atėjo pas Dovydą į Hebroną ir tarė: “Mes esame tavo kūnas ir kaulai. 
\par 2 Jau anksčiau, kai Saulius buvo mūsų karalius, tu išvesdavai ir įvesdavai Izraelį, ir Viešpats tau pažadėjo: ‘Tu ganysi mano tautą Izraelį ir būsi Izraelio vadas’ ”. 
\par 3 Visi Izraelio vyresnieji atėjo pas karalių į Hebroną. Karalius Dovydas sudarė su jais Hebrone sąjungą Viešpaties akivaizdoje. Jie patepė Dovydą Izraelio karaliumi. 
\par 4 Dovydui buvo trisdešimt metų, kai jis pradėjo karaliauti, ir jis karaliavo keturiasdešimt metų. 
\par 5 Gyvendamas Hebrone, jis valdė Judą septynerius metus ir šešis mėnesius ir Jeruzalėje trisdešimt trejus metus karaliavo visam Izraeliui ir Judui. 
\par 6 Dovydas su savo žmonėmis puolė Jeruzalę. Tenykščiai gyventojai jebusiečiai sakė Dovydui: “Tu neįeisi. Tave nuvys mūsų aklieji ir raišieji”. Nes jie galvojo: “Dovydas negalės įeiti”. 
\par 7 Tačiau Dovydas paėmė Siono tvirtovę (ji yra Dovydo miestas). 
\par 8 Tuomet Dovydas tarė: “Kas nugalės jebusiečius ir pasieks vandens kanalą, tegul žudo raišuosius ir akluosius, kurių nekenčia Dovydas”. Todėl yra sakoma: “Aklas ir raišas neįeis į namus”. 
\par 9 Dovydas apsigyveno tvirtovėje ir pavadino ją Dovydo miestu. Dovydas statė aplinkui, pradėdamas nuo Milojo, ir viduje. 
\par 10 Dovydas vis daugiau įsigalėjo, nes Viešpats, kareivijų Dievas, buvo su juo. 
\par 11 Tyro karalius Hiramas siuntė pas Dovydą pasiuntinių su kedro medžiais, dailidžių bei mūrininkų, kurie pastatė Dovydui namus. 
\par 12 Dovydas suprato, kad Viešpats jį įtvirtino Izraelio karaliumi ir išaukštino jo karalystę dėl savo tautos Izraelio. 
\par 13 Dovydas, persikėlęs iš Hebrono į Jeruzalę, paėmė iš Jeruzalės daugiau sugulovių ir žmonų. Čia Dovydui gimė sūnų ir dukterų. 
\par 14 Šitie vardai tų, kurie gimė Dovydui Jeruzalėje: Šamuva, Šobabas, Natanas, Saliamonas, 
\par 15 Ibharas, Elišūva, Nefegas, Jafija, 
\par 16 Elišama, Eljada ir Elifeletas. 
\par 17 Filistinai, išgirdę, kad Dovydas pateptas Izraelio karaliumi, pradėjo ieškoti Dovydo. Dovydas, tai sužinojęs, pasitraukė į tvirtovę, 
\par 18 o filistinai atėję sustojo Rafaimų slėnyje. 
\par 19 Dovydas klausė Viešpatį, sakydamas: “Ar man eiti prieš filistinus? Ar atiduosi juos į mano rankas?” Viešpats atsakė: “Eik, nes Aš tikrai atiduosiu filistinus į tavo rankas”. 
\par 20 Dovydas atėjo į Baal Peracimus ir ten juos sumušė. Tada jis tarė: “Viešpats nušlavė mano priešus prieš mane kaip užplūdęs vanduo”. Todėl tą vietą pavadino Baal Peracimu. 
\par 21 Filistinai ten paliko savo dievų atvaizdus, kuriuos Dovydas ir jo kariai sudegino. 
\par 22 Filistinai dar kartą atėjo ir sustojo Rafaimų slėnyje. 
\par 23 Dovydas vėl klausė Viešpaties. Jis atsakė: “Nesiartink jiems iš priekio. Eik aplinkui ir pulk iš šilkmedžių pusės. 
\par 24 Išgirdęs šlamesį šilkmedžių viršūnėse, užpulk juos, nes tada Viešpats išeis pirma tavęs ir naikins filistinų kariuomenę”. 
\par 25 Dovydas taip padarė, kaip Viešpats jam įsakė. Jis mušė filistinus nuo Gebos iki Gazero.



\chapter{6}


\par 1 Dovydas surinko visus Izraelio rinktinius vyrus, iš viso trisdešimt tūkstančių. 
\par 2 Dovydas pakilo ir su visais tais žmonėmis, kurie buvo pas jį iš Baale Jehudo, nuėjo pargabenti iš ten Dievo skrynios, vadinamos kareivijų Viešpaties, gyvenančio tarp cherubų, vardu. 
\par 3 Jie įkėlė Dievo skrynią į naują vežimą ir vežė iš Abinadabo namų, esančių Gibėjoje; Uza ir Achjojas, Abinadabo sūnūs, varė naują vežimą. 
\par 4 Jie vežė ją iš Abinadabo namų Gibėjoje, lydėdami Dievo skrynią; Achjojas ėjo skrynios priekyje. 
\par 5 Dovydas ir visi izraelitai grojo Viešpaties akivaizdoje įvairiais eglės medžio instrumentais: arfomis, psalteriais, barškalais, cimbolais ir dūdelėmis. 
\par 6 Kai jie pasiekė Nachono klojimą, Uza, ištiesęs ranką, prilaikė Dievo skrynią, nes jaučiai suklupo. 
\par 7 Viešpaties rūstybė užsidegė prieš Uzą, ir Dievas ištiko jį už jo klaidą, ir Uza mirė prie Dievo skrynios. 
\par 8 Dovydas susisielojo dėl to, kad Viešpats ištiko Uzą ir pavadino tą vietą Perec Uza. 
\par 9 Dovydas išsigando tą dieną Viešpaties ir sakė: “Kaip aš galiu Viešpaties skrynią pargabenti pas save?” 
\par 10 Todėl Dovydas nenorėjo pargabenti Viešpaties skrynios pas save į Dovydo miestą ir pasiuntė ją į gatiečio Obed Edomo namus. 
\par 11 Viešpaties skrynia pasiliko gatiečio Obed Edomo namuose tris mėnesius. Viešpats laimino Obed Edomą ir visus jo namiškius. 
\par 12 Karaliui Dovydui pranešė: “Viešpats palaimino Obed Edomo namus ir visa, kas jam priklauso, dėl Dievo skrynios”. Tada Dovydas nuėjo ir su džiaugsmu parsigabeno Dievo skrynią iš Obed Edomo namų į Dovydo miestą. 
\par 13 Kai tie, kurie nešė Viešpaties skrynią, paeidavo šešis žingsnius, Dovydas aukodavo jautį ir riebų aviną. 
\par 14 Dovydas šoko prieš Viešpatį iš visų jėgų; jis buvo apsirengęs lininį efodą. 
\par 15 Taip Dovydas ir visi Izraelio namai nešė Viešpaties skrynią su šauksmais ir trimito garsais. 
\par 16 Viešpaties skrynią atgabenus į Dovydo miestą, Sauliaus duktė Mikalė žiūrėjo pro langą. Pamačiusi karalių Dovydą šokinėjantį ir šokantį prieš Viešpatį, ji paniekino jį savo širdyje. 
\par 17 Atnešę Viešpaties skrynią, jie padėjo ją palapinėje, kurią Dovydas jai buvo paruošęs. Ir Dovydas aukojo deginamąsias ir padėkos aukas Viešpačiui. 
\par 18 Kai Dovydas baigė aukoti deginamąsias ir padėkos aukas, palaimino tautą kareivijų Viešpaties vardu 
\par 19 ir padalino visiems izraelitams, vyrams ir moterims, po duonos paplotį, gabalą mėsos ir džiovintų vynuogių pyragaitį. Po to žmonės ėjo kiekvienas į savo namus. 
\par 20 Tada Dovydas sugrįžo palaiminti savo namų. Sauliaus duktė Mikalė, išėjusi Dovydo pasitikti, tarė: “Koks garbingas šiandien buvo Izraelio karalius, apsinuoginęs šiandien prieš savo tarnų tarnaites, kaip begėdiškai apsinuogina niekam tikęs žmogus”. 
\par 21 Dovydas atsakė Mikalei: “Tai buvo prieš Viešpatį, kuris pasirinko mane vietoje tavo tėvo ir vietoje jo namų, kad paskirtų mane valdovu Viešpaties tautai, Izraeliui. Todėl aš grosiu Viešpačiui 
\par 22 ir dar labiau nusižeminsiu ir tapsiu menkas savo akyse, bet tarnaičių, apie kurias kalbėjai, būsiu gerbiamas”. 
\par 23 Todėl Sauliaus duktė Mikalė nesusilaukė vaikų iki savo mirties.



\chapter{7}

\par 1 Kai karalius gyveno savo namuose ir Viešpats buvo suteikęs jam ramybę nuo visų aplinkinių priešų, 
\par 2 jis tarė pranašui Natanui: “Aš gyvenu kedro namuose, o Dievo skrynia­ palapinėje”. 
\par 3 Natanas atsakė karaliui: “Daryk visa, kas yra tavo širdyje, nes Viešpats su tavimi”. 
\par 4 Tą pačią naktį Viešpaties žodis atėjo Natanui: 
\par 5 “Eik ir kalbėk mano tarnui Dovydui: ‘Taip sako Viešpats: ‘Ar tu pastatysi man namus, kuriuose gyvenčiau? 
\par 6 Aš negyvenau namuose nuo tos dienos, kai išvedžiau izraelitus iš Egipto, iki šios dienos, bet keliavau palapinėje. 
\par 7 Visur, kur Aš keliaudavau tarp izraelitų, ar Aš kuriai nors iš Izraelio giminių, kuriai pavesdavau ganyti mano tautą Izraelį, sakiau: ‘Kodėl man nepastatote kedro namų?’ 
\par 8 Todėl sakyk mano tarnui Dovydui: ‘Aš tave paėmiau iš ganyklos, nuo avių, kad būtum vadas mano tautai Izraeliui. 
\par 9 Aš buvau su tavimi visur, kur tu ėjai, išnaikinau visus tavo priešus tavo akyse ir padariau tavo vardą garsų kaip žemės didžiūnų vardą. 
\par 10 Aš paskirsiu vietą savo tautai Izraeliui ir jį įsodinsiu, kad jis gyventų savo vietoje ir nebeklajotų ir nedorybės vaikai nevargintų jo kaip iki šiol, 
\par 11 nuo to laiko, kai įsakiau teisėjams valdyti mano tautą Izraelį; Aš suteiksiu tau ramybę nuo visų tavo priešų. Be to, Viešpats sako tau, kad Jis įkurs tau namus. 
\par 12 Kai pasibaigs tavo dienos ir tu užmigsi prie savo tėvų, Aš pakelsiu tavo palikuonį po tavęs, išėjusį iš tavo strėnų, ir įtvirtinsiu jo karalystę. 
\par 13 Jis pastatys namus mano vardui, o Aš įtvirtinsiu jo karalystės sostą amžiams. 
\par 14 Aš būsiu jam tėvas, o jis bus man sūnus. Jei jis nusikals, bausiu jį žmonių rykštėmis ir žmonių vaikų smūgiais. 
\par 15 Bet savo gailestingumo Aš neatimsiu nuo jo, kaip atėmiau nuo Sauliaus, kurį pašalinau prieš tave. 
\par 16 Tavo namai ir tavo karalystė bus įtvirtinti tau amžiams; tavo sostas bus amžinas’ ”. 
\par 17 Visus šiuos žodžius ir regėjimą Natanas persakė Dovydui. 
\par 18 Tada karalius Dovydas įėjo ir atsisėdo Viešpaties akivaizdoje, ir tarė: “Kas aš, Viešpatie Dieve, ir kas mano namai, kad mane iki čia atvedei? 
\par 19 Ir tai pasirodė dar per maža Tavo akyse, Viešpatie Dieve. Tu dar kalbėjai apie savo tarno namus tolimoje ateityje. Ar taip būna pas žmones, Viešpatie Dieve? 
\par 20 Ką gi daugiau Dovydas begali Tau sakyti? Tu pažįsti savo tarną, Viešpatie Dieve. 
\par 21 Dėl savo žodžių ir pagal savo širdį Tu padarei šiuos didelius dalykus, pranešdamas tai savo tarnui. 
\par 22 Tu esi didis, Viešpatie Dieve! Nes nėra nė vieno Tau lygaus ir nėra kito Dievo šalia Tavęs, kaip mes girdėjome savo ausimis. 
\par 23 Kokia kita tauta žemėje prilygsta Tavo tautai Izraeliui, pas kurią Dievas atėjo išpirkti jos sau ir išgarsinti savo vardą? Jis padarė didelių ir baisių dalykų, matant savo tautai, kurią Tu išpirkai sau iš Egipto, iš svetimų tautų ir jų dievų. 
\par 24 Tu išsirinkai Izraelį, kad jis būtų Tavo tauta per amžius, o Tu, Viešpatie, tapai jiems Dievu. 
\par 25 Dabar, Viešpatie Dieve, įtvirtink amžiams savo žodį, kurį kalbėjai apie savo tarną bei jo namus, ir padaryk, kaip pasakei. 
\par 26 Tegul Tavo vardas būna aukštinamas per amžius, sakant: ‘Kareivijų Viešpats yra Izraelio Dievas’, ir tegul Tavo tarno Dovydo namai būna įtvirtinti Tavo akivaizdoje. 
\par 27 Tu, kareivijų Viešpatie, Izraelio Dieve, apreiškei savo tarnui, sakydamas: ‘Aš tau pastatysiu namus’, todėl Tavo tarnas išdrįso savo širdyje kreiptis į Tave šia malda. 
\par 28 Viešpatie Dieve, Tu esi Dievas, ir Tavo žodžiai yra tiesa; ir Tu pažadėjai šitą gerovę savo tarnui. 
\par 29 Dabar teikis laiminti savo tarno namus, kad jie išliktų per amžius Tavo akivaizdoje, nes Tu, Viešpatie Dieve, tai pasakei. Tegul būna palaiminti Tavo tarno namai Tavo palaiminimu per amžius”.



\chapter{8}

\par 1 Po to Dovydas, nugalėjęs filistinus, juos pavergė ir atėmė iš jų Mefogamą. 
\par 2 Ir jis nugalėjo Moabą, ir matavo juos, suguldęs ant žemės. Jis atmatavo dvi virves nužudyti, o vieną virvę palikti gyvus. Taip moabitai tapo Dovydo tarnais ir mokėjo jam duoklę. 
\par 3 Taip pat Dovydas nugalėjo Rehobo sūnų Hadadezerą, Cobos karalių, kai tas siekė atgauti savo valdžią prie Eufrato. 
\par 4 Dovydas atėmė iš jo tūkstantį kovos vežimų, paėmė nelaisvėn septynis šimtus raitelių ir dvidešimt tūkstančių pėstininkų. Jis pakirto visiems kovos vežimų arkliams kojų gyslas ir sau pasiliko arklių tik dėl šimto kovos vežimų. 
\par 5 Kai sirai iš Damasko atėjo į pagalbą Cobos karaliui Hadadezerui, Dovydas nukovė sirų dvidešimt du tūkstančius vyrų. 
\par 6 Dovydas paskyrė įgulas Damasko Sirijoje. Taip sirai tapo Dovydo tarnais ir mokėjo duoklę. Viešpats saugojo Dovydą visur, kur jis ėjo. 
\par 7 Dovydas paėmė Hadadezero tarnų auksinius skydus ir parsigabeno į Jeruzalę, 
\par 8 o iš Hadadezero miestų Betacho ir Berotajo karalius Dovydas parsigabeno labai daug vario. 
\par 9 Hamato karalius Tojas, išgirdęs, kad Dovydas sumušė visą Hadadezero kariuomenę, 
\par 10 atsiuntė savo sūnų Joramą pas karalių Dovydą pasveikinti jį ir palaiminti, nes jis kariavo su Hadadezeru ir jį sumušė, o Hadadezeras buvo Tojaus priešas. Joramas atnešė Dovydui dovanų sidabrinių, auksinių ir varinių indų. 
\par 11 Karalius Dovydas paskyrė tuos daiktus Viešpačiui kartu su sidabru ir auksu, kurį jis paskyrė iš visų užimtų tautų: 
\par 12 Sirijos, Moabo, amonitų, filistinų, amalekiečių ir Rehobo sūnaus Hadadezero, Cobos karaliaus. 
\par 13 Dovydas įsigijo vardą, kai grįžo sumušęs Druskos slėnyje aštuoniolika tūkstančių sirų. 
\par 14 Ir jis paskyrė įgulas Edome, visame Edomo krašte, ir edomitai tapo Dovydo tarnais. Viešpats saugojo Dovydą visur, kur jis ėjo. 
\par 15 Dovydas karaliavo visame Izraelyje ir vykdė teisingumą ir teismą visai tautai. 
\par 16 Cerujos sūnus Joabas buvo kariuomenės vadas, Ahiludo sūnus Juozapatas­metraštininkas, 
\par 17 Ahitubo sūnus Cadokas ir Abjataro sūnus Ahimelechas buvo kunigai, Seraja­raštininkas, 
\par 18 Jehojados sūnus Benajas buvo keretų ir peletų viršininkas, o Dovydo sūnūs­aukšti pareigūnai.



\chapter{9}


\par 1 Ir Dovydas sakė: “Ar yra išlikęs kas nors iš Sauliaus namų, kuriam aš galėčiau parodyti gerumą dėl Jehonatano?” 
\par 2 Iš Sauliaus namų buvo tarnas, vardu Ciba. Jis buvo pašauktas pas Dovydą, ir karalius klausė: “Ar tu esi Ciba?” Tas atsakė: “Taip, tavo tarnas”. 
\par 3 Karalius klausė: “Ar išliko kas nors iš Sauliaus namų, kuriam galėčiau parodyti Dievo gerumą?” Ciba atsakė: “Yra raišas Jehonatano sūnus”. 
\par 4 Karalius vėl klausė jo: “Kur jis yra?” Ciba atsakė: “Jis gyvena Amielio sūnaus Machyro namuose, Lo Debare”. 
\par 5 Karalius Dovydas pasiuntė ir parsigabeno jį iš Lo Debaro, iš Amielio sūnaus Machyro namų. 
\par 6 Sauliaus sūnaus Jehonatano sūnus Mefi Bošetas, įėjęs pas Dovydą, puolė veidu į žemę ir nusilenkė. Dovydas tarė: “Mefi Bošetai!” Tas atsakė: “Štai tavo tarnas”. 
\par 7 Ir Dovydas kalbėjo jam: “Nebijok, aš tau būsiu geras dėl tavo tėvo Jehonatano ir sugrąžinsiu tau visas tavo tėvo Sauliaus žemes. Tu visada valgysi prie mano stalo”. 
\par 8 Mefi Bošetas nusilenkė ir tarė: “Kas yra tavo tarnas, kad tu atkreipei savo dėmesį į tokį pastipusį šunį kaip aš?” 
\par 9 Karalius pasišaukė Sauliaus tarną Cibą ir jam tarė: “Visa, kas priklausė Sauliui ir jo namams, atidaviau tavo valdovo sūnui. 
\par 10 Dirbk jo žemę su savo sūnumis ir tarnais bei nuimk derlių, kad tavo valdovo sūnus turėtų maisto, o tavo valdovo sūnus Mefi Bošetas visada valgys prie mano stalo”. Ciba turėjo penkiolika sūnų ir dvidešimt tarnų. 
\par 11 Jis atsakė karaliui: “Kaip tu, karaliau, įsakei savo tarnui, taip tavo tarnas padarys”. Mefi Bošetas valgė prie Dovydo stalo kaip vienas iš karaliaus sūnų. 
\par 12 Mefi Bošetas turėjo mažą sūnų, vardu Michėjas. Visa Cibos šeimyna buvo Mefi Bošeto tarnai. 
\par 13 Mefi Bošetas gyveno Jeruzalėje ir visada valgė prie karaliaus stalo. Jis buvo luošas abiem kojom.



\chapter{10}


\par 1 Po kurio laiko mirė amonitų karalius ir jo vietą užėmė jo sūnus Hanūnas. 
\par 2 Dovydas sakė: “Aš būsiu geras Nahašo sūnui Hanūnui, kaip jo tėvas buvo man”. Ir Dovydas siuntė savo tarnus paguosti Hanūno dėl jo tėvo mirties. Dovydo tarnams atėjus į amonitų šalį, 
\par 3 Amono kunigaikščiai tarė savo valdovui Hanūnui: “Ar manai, kad Dovydas siuntė guodėjus pas tave, norėdamas pagerbti tavo tėvą? Ar ne tam, kad miestą apžiūrėtų, viską išžvalgytų ir paskui jį sunaikintų?” 
\par 4 Hanūnas paėmė Dovydo tarnus, nuskuto jiems po pusę barzdos, nukirpo drabužius iki pusės, iki pat juostos, ir išsiuntė. 
\par 5 Kai apie tai buvo pranešta Dovydui, jis pasiuntė jų pasitikti, nes jie buvo labai sugėdinti, ir sakė jiems: “Pasilikite Jeriche, kol ataugs jūsų barzdos, o tada grįžkite”. 
\par 6 Kai amonitai suprato, kad tapo Dovydo nekenčiami, pasisamdė iš Bet Rehobo bei Cobos dvidešimt tūkstančių sirų pėstininkų, iš Maako karaliaus­tūkstantį vyrų ir iš Tobo­dvylika tūkstančių vyrų. 
\par 7 Dovydas, tai išgirdęs, pasiuntė Joabą su visa stiprių vyrų kariuomene. 
\par 8 Amonitai išsirikiavo kautynėms prie vartų, o sirai iš Cobos ir Rehobo bei Tobo ir Maako vyrai sustojo atskirai atvirame lauke. 
\par 9 Joabas, pamatęs, kad priešai išsirikiavę prieš jį iš priekio ir iš užpakalio, išrinko geriausius karius iš viso Izraelio ir išrikiavo juos prieš sirus, 
\par 10 likusius žmones jis pavedė savo broliui Abišajui, kuris išrikiavo juos prieš amonitus. 
\par 11 Tada Joabas tarė broliui: “Jei sirai bus man per stiprūs, tu ateisi man į pagalbą, o jei amonitai bus per stiprūs tau, tai aš tau padėsiu. 
\par 12 Būk drąsus ir kovokime už savo tautą ir Dievo miestus! Viešpats tedaro, kaip Jam atrodo tinkama”. 
\par 13 Joabas ir su juo buvusieji žmonės pradėjo kovą prieš sirus, ir tie pabėgo nuo jo. 
\par 14 Amonitai, pamatę, kad sirai pabėgo, bėgo nuo Abišajo ir užsidarė mieste. Joabas pasitraukė nuo amonitų ir sugrįžo į Jeruzalę. 
\par 15 Sirai, pamatę, kad pralaimėjo Izraeliui, susirinko į vieną vietą. 
\par 16 Hadadezeras pasiuntė ir pasikvietė sirus, kurie gyveno anapus upės. Jie ėjo į Helamą, vedami Hadadezero kariuomenės vado Šobacho. 
\par 17 Kai tai buvo pranešta Dovydui, jis, surinkęs visą Izraelį, persikėlė per Jordaną ir ėjo į Helamą. Ir sirai išsirikiavo prieš Dovydą, kad susikautų su juo. 
\par 18 Ir sirai bėgo nuo Izraelio. Dovydas sunaikino septynis šimtus sirų kovos vežimų bei keturiasdešimt tūkstančių raitelių ir ištiko kariuomenės vadą Šobachą, kuris ten mirė. 
\par 19 Visi karaliai, Hadadezero pavaldiniai, matydami, kad Izraelis juos nugalėjo, sudarė taiką su juo ir tarnavo jam. Nuo to laiko sirai bijodavo padėti amonitams.



\chapter{11}

\par 1 Praėjus metams, tuo laiku, kai karaliai eina į karą, Dovydas pasiuntė Joabą su savo tarnais ir visą Izraelio kariuomenę, kurie nugalėjo amonitus ir apgulė Rabą. Bet Dovydas pasiliko Jeruzalėje. 
\par 2 Kartą vakare Dovydas atsikėlė nuo savo lovos ir vaikščiojo ant karaliaus namų stogo. Nuo stogo jis pamatė besimaudančią moterį; moteris buvo labai graži. 
\par 3 Dovydas pasiuntė sužinoti, kas ji. Jam buvo pranešta, kad tai Eliamo duktė Batšeba, hetito Ūrijos žmona. 
\par 4 Dovydas pasiuntė savo tarnus ją atvesti. Ji atėjo pas jį, ir jis sugulė su ja, nes ji buvo apsivaliusi nuo savo nešvarumo. Ir ji sugrįžo į savo namus. 
\par 5 Moteris pastojo ir pasiuntė pas Dovydą pranešti: “Aš esu nėščia”. 
\par 6 Dovydas pasiuntė pasakyti Joabui, kad atsiųstų jam hetitą Ūriją. Joabas pasiuntė Ūriją pas Dovydą. 
\par 7 Kai Ūrija atėjo pas Dovydą, jis teiravosi, kaip sekasi Joabui, kariams ir kaip vyksta karas. 
\par 8 Po to Dovydas tarė Ūrijai: “Eik į savo namus ir nusiplauk kojas”. Ūrijai išėjus iš karaliaus namų, jam iš paskos nunešė karališkų valgių. 
\par 9 Bet Ūrija nėjo namo ir atsigulė karaliaus namų prieangyje su visais savo valdovo tarnais. 
\par 10 Kai Dovydui pranešė, kad Ūrija nėjo namo, Dovydas paklausė Ūrijos: “Tu atėjai iš kelionės. Kodėl neini į savo namus?” 
\par 11 Ūrija atsakė Dovydui: “Skrynia, Izraelis ir Judas gyvena palapinėse, o mano valdovas Joabas ir mano valdovo tarnai apsistoję atvirame lauke. Kaip aš galiu eiti į savo namus valgyti, gerti ir miegoti su savo žmona? Kaip tu gyvas ir gyva tavo siela, aš to nedarysiu!” 
\par 12 Dovydas tarė: “Pasilik dar šiandien čia, o rytoj aš tave išleisiu”. Ūrija pasiliko Jeruzalėje dar vieną dieną iki rytojaus. 
\par 13 Dovydas pasikvietė jį, jis valgė ir gėrė su Dovydu, ir Dovydas jį nugirdė. Vakare Ūrija išėjo miegoti kartu su savo valdovo tarnais, tačiau į savo namus nėjo. 
\par 14 Rytą Dovydas parašė Joabui laišką ir jį pasiuntė per Ūriją. 
\par 15 Laiške jis rašė: “Pastatyk Ūriją į smarkiausios kovos priekį ir atsitraukite, kad jis žūtų”. 
\par 16 Joabas, apgulęs miestą, pastatė Ūriją į tokią vietą, apie kurią žinojo, kad ten stovi drąsūs žmonės. 
\par 17 Miesto žmonės išėjo ir kovėsi su Joabu. Ir krito keletas iš tautos, iš Dovydo tarnų, ir hetitas Ūrija žuvo taip pat. 
\par 18 Joabas pasiuntė Dovydui pranešimą apie mūšio eigą. 
\par 19 Pasiuntiniui įsakė: “Kai pabaigsi pasakoti karaliui apie mūšį, 
\par 20 jei karalius supyks ir sakys tau: ‘Kodėl taip priartėjote prie miesto kovodami? Argi nežinojote, kad jie šaudys nuo sienų? 
\par 21 Kas užmušė Jerubešeto sūnų Abimelechą? Argi ne moteris, numetusi ant jo girnų akmenį nuo sienos taip, kad jis mirė Tebece? Kodėl priartėjote prie sienų?’ Atsakyk jam: ‘Tavo tarnas hetitas Ūrija taip pat miręs’ ”. 
\par 22 Pasiuntinys nuėjo ir pranešė Dovydui visa, ką Joabas buvo jam įsakęs. 
\par 23 Ir pasiuntinys sakė Dovydui: “Vyrai įveikė mus ir išėjo prieš mus į atvirą lauką, bet mes juos nustūmėme ligi miesto vartų. 
\par 24 Šauliai šaudė į tavo tarnus nuo sienų, ir keletas tavo tarnų žuvo. Ir tavo tarnas hetitas Ūrija taip pat miręs”. 
\par 25 Tuomet Dovydas tarė pasiuntiniui: “Pasakyk Joabui dėl to nenusiminti, nes kardas ryja tai vieną, tai kitą. Tegul sustiprina miesto puolimą ir jį sugriauna. Taip jį padrąsink”. 
\par 26 Ūrijos žmona, išgirdusi, kad jos vyras žuvo, gedėjo dėl savo vyro. 
\par 27 Gedului pasibaigus, Dovydas parsivedė ją į savo namus. Ji tapo jo žmona ir pagimdė jam sūnų. Bet šis dalykas, kurį padarė Dovydas, nepatiko Viešpačiui.



\chapter{12}

\par 1 Viešpats siuntė pranašą Nataną pas Dovydą. Natanas, atėjęs pas jį, tarė: “Du vyrai gyveno viename mieste. Vienas buvo turtingas, o antras­beturtis. 
\par 2 Turtingasis turėjo labai daug avių ir galvijų, 
\par 3 o beturtis nieko neturėjo, tik vieną avytę, kurią nusipirko ir prižiūrėjo. Ji augo kartu su jo vaikais, maitinosi jo valgiu, gėrė iš jo taurės ir gulėjo prie jo šono; ji jam buvo kaip duktė. 
\par 4 Kartą užėjo keleivis pas turtingąjį vyrą. Jis pagailėjo savo avių ir galvijų, kad paruoštų keleiviui maisto. Paėmęs beturčio mylimą avytę, paruošė iš jos keleiviui vaišes”. 
\par 5 Dovydas, labai supykęs ant to žmogaus, tarė Natanui: “Kaip Viešpats gyvas, tas vyras turi mirti! 
\par 6 O už avytę jis privalo atlyginti keturgubai, nes jis taip pasielgė ir neparodė gailesčio”. 
\par 7 Natanas tarė Dovydui: “Tu esi tas žmogus! Taip sako Viešpats, Izraelio Dievas: ‘Aš tave patepiau Izraelio karaliumi ir išgelbėjau iš Sauliaus rankų. 
\par 8 Aš tau atidaviau tavo valdovo namus bei jo žmonas ir tau daviau Izraelio bei Judo namus; jei to buvo maža, dar daugiau būčiau pridėjęs. 
\par 9 Kodėl paniekinai Viešpaties įsakymą, piktai elgdamasis Jo akivaizdoje? Tu nužudei hetitą Ūriją amonitų kardu ir pasiėmei jo žmoną. 
\par 10 Dabar kardas visada lydės tavo namus, nes tu paniekinai mane ir pasiėmei hetito Ūrijos žmoną. 
\par 11 Aš pakelsiu prieš tave pikta iš tavo paties namų; tau matant, tavo žmonas atiduosiu tavo artimui, ir jis suguls su tavo žmonomis prieš saulę. 
\par 12 Tu tai darei slaptai, bet Aš darysiu visam Izraeliui matant, saulės šviesoje’ ”. 
\par 13 Dovydas tarė Natanui: “Aš nusidėjau Viešpačiui”. Natanas atsakė Dovydui: “Viešpats pašalino tavo nuodėmę, tu nemirsi. 
\par 14 Bet kadangi davei progos Viešpaties priešams Dievą niekinti, sūnus, kuris tau gimė, mirs”. 
\par 15 Natanas nuėjo į savo namus, o Viešpats ištiko kūdikį, kurį Ūrijos žmona pagimdė Dovydui, ir tas sunkiai susirgo. 
\par 16 Dovydas maldavo Dievą dėl vaiko, jis pasninkavo ir pasišalinęs gulėjo ant žemės visą naktį. 
\par 17 Jo namų vyresnieji atėję norėjo jį pakelti nuo žemės, bet jis nesikėlė ir nevalgė su jais. 
\par 18 Septintą dieną kūdikis mirė. Dovydo tarnai bijojo jam pranešti apie kūdikio mirtį, galvodami: “Dar kūdikiui tebesant gyvam, mes jam kalbėjome, bet jis buvo neperkalbamas. Kai pranešime jam, kad kūdikis mirė, jis gali padaryti ką nors negero”. 
\par 19 Dovydas pastebėjo, kad jo tarnai šnibždasi, ir suprato, kad kūdikis miręs. Jis paklausė savo tarnų: “Ar kūdikis mirė?” Tie atsakė: “Mirė”. 
\par 20 Tuomet Dovydas atsikėlė nuo žemės, nusiprausė, pasitepė, pakeitė drabužius ir, nuėjęs į Viešpaties namus, pagarbino. Sugrįžęs paprašė maisto ir valgė. 
\par 21 Jo tarnai klausė: “Ką reiškia toks tavo elgesys? Tu pasninkavai ir verkei kūdikiui esant gyvam, o kai jis mirė, atsikėlei ir valgai?” 
\par 22 Jis atsakė: “Kūdikiui tebesant gyvam, pasninkavau ir verkiau, manydamas: ‘Kas žino, gal Viešpats pasigailės manęs ir kūdikis nemirs’. 
\par 23 Dabar jis mirė, tai kam gi man bepasninkauti? Ar aš galiu jį sugrąžinti? Aš nueisiu pas jį, bet jis nesugrįš pas mane”. 
\par 24 Dovydas paguodė savo žmoną Batšebą ir, įėjęs pas ją, gulėjo su ja. Ji pagimdė sūnų, kurį Dovydas pavadino Saliamonu. Viešpats pamilo vaikutį 
\par 25 ir siuntė pranašą Nataną, kad jį pavadintų Jedidiju pagal Viešpaties žodį. 
\par 26 Joabas kariavo prieš amonitus ir apgulė jų karališkąjį miestą Rabą. 
\par 27 Joabo pasiuntiniai pranešė Dovydui: “Aš kariavau prieš Rabą ir paėmiau miesto vandens atsargas. 
\par 28 Todėl dabar surink likusius karius, apgulk miestą ir jį paimk, kad aš jo nepaimčiau ir jis nebūtų pavadintas mano vardu”. 
\par 29 Dovydas, surinkęs visus žmones, nuėjo prie Rabos, kariavo prieš miestą ir jį paėmė. 
\par 30 Dovydas nuėmė amonitų karaliui nuo galvos karūną su brangiais akmenimis, sveriančią talentą aukso, ir užsidėjo ją ant galvos. Be to, jis išsigabeno iš miesto labai daug grobio. 
\par 31 Žmones jis išsivedė ir pristatė juos prie pjūklų, geležinių akėčių, kirvių ir prie krosnių plytoms degti. Taip jis padarė su visais amonitų miestais. Tada Dovydas su visais žmonėmis sugrįžo į Jeruzalę.
Online Parallel Study Bible



\chapter{13}

\par 1 Dovydo sūnus Abšalomas turėjo gražią seserį, vardu Tamara. Ją pamilo Dovydo sūnus Amnonas. 
\par 2 Amnonas taip liūdėjo, kad susirgo dėl savo sesers Tamaros. Ji buvo mergaitė, ir Amnonui atrodė sunku ką nors jai padaryti. 
\par 3 Amnonas turėjo draugą Jonadabą, Dovydo brolio Šimos sūnų. Jonadabas buvo labai gudrus vyras. 
\par 4 Jis klausė Amnoną: “Kodėl tu, karaliaus sūnau, eini liesyn diena po dienos? Gal man pasakysi?” Amnonas jam atsakė: “Aš myliu Tamarą, savo brolio Abšalomo seserį”. 
\par 5 Jonadabas jam tarė: “Apsimesk sergančiu ir nesikelk iš lovos. Kai tavo tėvas ateis aplankyti tavęs, sakyk: ‘Tegul ateina mano sesuo Tamara ir paruošia man matant valgio, kad valgyčiau iš jos rankų’ ”. 
\par 6 Amnonas atsigulė ir apsimetė sergąs. Karaliui atėjus jo aplankyti, Amnonas prašė karaliaus: “Prašau, tegul ateina mano sesuo Tamara ir paruošia man matant pora pyragaičių, kad valgyčiau iš jos rankų”. 
\par 7 Dovydas pasiuntė pas Tamarą į namus, sakydamas: “Nueik į brolio Amnono namus ir paruošk jam valgį”. 
\par 8 Ji, nuėjusi į savo brolio Amnono namus, kur jis gulėjo, suminkė jam matant ir iškepė pyragaičių. 
\par 9 Ji išėmė juos iš keptuvės ir padėjo priešais jį, bet jis atsisakė valgyti. Amnonas sakė: “Tegul visi išeina”. Ir visi žmonės išėjo. 
\par 10 Tada Amnonas tarė Tamarai: “Atnešk valgį į kambarį, kad valgyčiau iš tavo rankų”. Tamara ėmė pyragaičius, kuriuos buvo iškepusi, ir atnešė savo broliui Amnonui į kambarį. 
\par 11 Jai įnešus valgį, Amnonas, nutvėręs ją, tarė: “Gulk su manimi, mano seserie”. 
\par 12 Ji jam atsakė: “Ne, broli, neprievartauk manęs, nes taip neturi būti daroma Izraelyje. Nedaryk šitos kvailystės. 
\par 13 Kur aš dingsiu iš gėdos? Ir tu būsi kaip kvailys Izraelyje. Geriau pasikalbėk su karaliumi. Jis neatsisakys manęs tau duoti”. 
\par 14 Tačiau Amnonas nenorėjo klausyti jos ir, būdamas už ją stipresnis, išprievartavo ją. 
\par 15 Po to Amnonas pradėjo jos labai nekęsti. Ta neapykanta buvo didesnė už jo meilę. Amnonas tarė jai: “Kelkis ir eik sau!” 
\par 16 O ji jam atsakė: “Išvarydamas mane, tu pasielgi dar pikčiau, negu prieš tai”. Tačiau jis nenorėjo jos klausyti. 
\par 17 Pasišaukęs savo tarną, įsakė jam išvaryti Tamarą ir uždaryti duris paskui ją. 
\par 18 Ji vilkėjo įvairių spalvų drabužiu, nes tokius rūbus dėvėdavo karaliaus dukterys mergaitės. Jo tarnas, išvedęs ją laukan, uždarė duris paskui ją. 
\par 19 Tamara apsibarstė galvą pelenais, persiplėšė įvairiaspalvį drabužį, kurį vilkėjo, ir ėjo verkdama, uždėjusi ranką sau ant galvos. 
\par 20 Jos brolis Abšalomas klausė: “Ar tavo brolis Amnonas buvo su tavimi? Tačiau dabar tylėk, mano seserie. Jis yra tavo brolis, nesisielok dėl to”. Tamara pasiliko netekėjusi savo brolio Abšalomo namuose. 
\par 21 Karalius Dovydas, išgirdęs visa tai, labai supyko. 
\par 22 Abšalomas nekalbėjo su Amnonu nei geruoju, nei piktuoju, nes jis nekentė Amnono už tai, kad jis išprievartavo jo seserį Tamarą. 
\par 23 Praėjus dvejiems metams, Abšalomas kirpo avis Baal Hacore prie Efraimo ir kvietė visus karaliaus sūnus. 
\par 24 Abšalomas, atėjęs pas karalių, tarė: “Kerpamos mano avys. Kviečiu karalių ir jo tarnus pas save”. 
\par 25 Karalius atsakė: “Ne, sūnau. Mes visi neisime, kad tavęs neapsunkintume”. Nors Abšalomas įkalbinėjo, tačiau karalius nenorėjo eiti, bet palaimino jį. 
\par 26 Tada Abšalomas tarė: “Jei ne, prašau leisk nors mano brolį Amnoną eiti su mumis”. Karalius atsakė: “Kodėl jis turėtų eiti su tavimi?” 
\par 27 Abšalomui labai prašant, karalius leido Amnonui ir visiems savo sūnums eiti su juo. 
\par 28 Abšalomas įsakė savo tarnams: “Kai Amnonas bus įsilinksminęs nuo vyno ir aš jums pasakysiu: ‘Muškite Amnoną’, nužudykite jį; nebijokite, nes aš įsakau jums. Būkite drąsūs ir narsūs”. 
\par 29 Tarnai padarė Amnonui, kaip Abšalomas įsakė. Tada visi karaliaus sūnūs atsikėlę užsėdo kiekvienas ant savo mulo ir pabėgo. 
\par 30 Jiems dar negrįžus, Dovydą pasiekė gandas: “Abšalomas užmušė visus karaliaus sūnus ir nė vieno nepaliko”. 
\par 31 Tuomet karalius persiplėšė savo drabužius ir atsigulė ant žemės, ir visi tarnai stovėjo šalia su perplėštais drabužiais. 
\par 32 Dovydo brolio Šimos sūnus Jonadabas tarė: “Tegul mano valdovas negalvoja, kad jie nužudė visus karaliaus sūnus. Tik Amnonas vienas yra miręs Abšalomo įsakymu, nes Abšalomas buvo nusprendęs taip padaryti nuo tos dienos, kai Amnonas išprievartavo jo seserį Tamarą. 
\par 33 Mano valdove karaliau, nesisielok ir negalvok, kad visi tavo sūnūs žuvę, nes tik Amnonas vienas yra žuvęs”. 
\par 34 Abšalomas pabėgo. Jaunuolis, einąs sargybą, pakėlė akis ir pamatė daug žmonių, ateinančių pakalnės keliu. 
\par 35 Jonadabas tarė karaliui: “Štai, tavo sūnūs ateina, kaip tavo tarnas ir sakė”. 
\par 36 Jam pabaigus kalbėti, atėjo karaliaus sūnūs. Jie pakėlė balsus ir verkė, karalius ir visi jo tarnai taip pat labai verkė. 
\par 37 Abšalomas pabėgęs nuvyko pas Gešūro karalių, Amihudo sūnų Talmają. Dovydas ilgai liūdėjo savo sūnaus. 
\par 38 Abšalomas, pabėgęs į Gešūrą, ten išbuvo trejus metus. 
\par 39 Karalius pradėjo ilgėtis Abšalomo, nes jau buvo nusiraminęs dėl Amnono mirties.



\chapter{14}


\par 1 Cerujos sūnus Joabas pastebėjo, kad karaliaus širdis palinko prie Abšalomo. 
\par 2 Joabas pakvietė iš Tekojos išmintingą moterį ir tarė jai: “Apsivilk gedulo drabužiais, nesitepk aliejumi ir apsimesk gedinti ilgą laiką. 
\par 3 Nueik pas karalių ir taip jam kalbėk”. Ir Joabas įdėjo žodžius į jos lūpas. 
\par 4 Tekojietė moteris, atėjusi pas karalių, puolė veidu į žemę, išreikšdama jam pagarbą, ir tarė: “Karaliau, padėk”. 
\par 5 Karalius jos klausė: “Kas nutiko?” Ji atsakė: “Aš esu našlė, mano vyras miręs. 
\par 6 Tavo tarnaitė turėjo du sūnus. Juodu susiginčijo laukuose ir, nesant kas juos perskiria, vienas užmušė kitą. 
\par 7 Dabar visi giminės, sukilę prieš mane, sako: ‘Išduok tą, kuris užmušė savo brolį, kad jį nužudytume už jo brolį ir sunaikintume paveldėtoją’. Tuo būdu jie siekia užgesinti man likusią kibirkštėlę, kad nepaliktų ant žemės mano vyrui nei vardo, nei palikuonių”. 
\par 8 Karalius atsakė moteriai: “Eik į savo namus, aš duosiu nurodymą dėl tavęs”. 
\par 9 Tekojietė moteris tarė karaliui: “Mano valdove karaliau, kaltė tebūna ant manęs ir mano tėvo namų, o karalius ir jo sostas bus nekaltas”. 
\par 10 Karalius atsakė: “Kas tau grasins, tą atvesk pas mane, ir jis daugiau tavęs nebelies”. 
\par 11 Ji sakė: “Karaliau, atsimink Viešpatį, savo Dievą, ir nebeleisk kraujo keršytojams žudyti, kad jie nesunaikintų mano sūnaus”. Jis atsakė: “Kaip Viešpats gyvas, nė vienas tavo sūnaus plaukas nenukris žemėn”. 
\par 12 Moteris tarė: “Leisk savo tarnaitei pasakyti tau, karaliau, dar vieną dalyką”. Jis tarė: “Kalbėk”. 
\par 13 Moteris sakė: “Kodėl tu esi nusistatęs prieš Dievo tautą? Juk taip kalbėdamas, karaliau, tu apkaltini pats save, neleisdamas grįžti savo ištremtajam. 
\par 14 Mes visi mirštame ir esame kaip ant žemės išlietas vanduo, kurio nebegalima surinkti; tačiau Dievas neatima gyvybės, bet atranda būdą, kad ištremtieji nebūtų atstumti nuo Jo. 
\par 15 Aš atėjau kalbėti karaliui, savo valdovui, dėl to, kad žmonės mane įbaugino. Tavo tarnaitė galvojo: ‘Kalbėsiu karaliui, gal karalius įvykdys savo tarnaitės prašymą. 
\par 16 Karalius juk išklausys ir išgelbės savo tarnaitę iš rankos, siekiančios išnaikinti mane ir mano sūnų iš Dievo mums skirto paveldėjimo’. 
\par 17 Mano valdovo karaliaus žodis nuramino mane. Nes karalius yra kaip Dievo angelas ir skiria gera nuo pikta. Viešpats, tavo Dievas, tebūna su tavimi”. 
\par 18 Karalius tarė jai: “Neslėpk nuo manęs nieko, ko klausiu”. Moteris atsakė: “Tegul kalba karalius, mano valdovas”. 
\par 19 Karalius paklausė: “Ar ne Joabo ranka yra su tavimi šitame reikale?” Moteris sakė: “Kaip gyva tavo siela, mano valdove karaliau, niekas negali nukrypti nei į kairę, nei į dešinę nuo to, ką pasakė karalius. Taip, tavo tarnas Joabas mane pamokė ir įdėjo tuos žodžius į mano lūpas. 
\par 20 Norėdamas taip išdėstyti visą reikalą, tavo tarnas Joabas tą padarė. Bet mano valdovas yra išmintingas kaip Dievo angelas ir žino visa, kas vyksta žemėje”. 
\par 21 Karalius tarė Joabui: “Aš patenkinu tavo prašymą, eik ir parvesk jaunuolį Abšalomą”. 
\par 22 Joabas, parpuolęs veidu į žemę, nusilenkė ir dėkojo karaliui. Po to jis tarė: “Karaliau, mano valdove, šiandien žinau, kad radau malonę tavo akyse, nes patenkinai savo tarno prašymą”. 
\par 23 Joabas, nuėjęs į Gešūrą, parvedė Abšalomą į Jeruzalę. 
\par 24 Karalius įsakė: “Tegul jis gyvena savo namuose, bet mano veido nematys”. Taip Abšalomas sugrįžo į savo namus, tačiau karaliaus nematė. 
\par 25 Visame Izraelyje nebuvo gražesnio vyro už Abšalomą. Nuo galvos iki kojų padų jis neturėjo jokio trūkumo. 
\par 26 Jis kirpdavosi plaukus kiekvienų metų pabaigoje, nes plaukai būdavo jam sunkūs. Jo nukirpti galvos plaukai svėrė du šimtus šekelių pagal karaliaus svorį. 
\par 27 Abšalomas turėjo tris sūnus ir vieną dukterį, vardu Tamara; ji buvo graži moteris. 
\par 28 Abšalomas išgyveno Jeruzalėje dvejus metus, nematęs karaliaus veido. 
\par 29 Jis kvietė Joabą pas save, norėdamas jį pasiųsti pas karalių, bet Joabas neatėjo. Antrą kartą pakviestas, Joabas taip pat neatėjo. 
\par 30 Tuomet Abšalomas kalbėjo savo tarnams: “Jūs žinote, kad Joabo laukas yra šalia mano ir jame auga miežiai. Eikite ir juos padekite”. Abšalomo tarnai padegė miežius. 
\par 31 Tada Joabas, atėjęs pas Abšalomą, klausė: “Kodėl tavo tarnai padegė mano miežius?” 
\par 32 Abšalomas atsakė Joabui: “Aš tave kviečiau ateiti pas mane, nes norėjau, kad tu paklaustum karaliaus, kodėl turėjau grįžti iš Gešūro. Man būtų buvę geriau, jei ten būčiau pasilikęs. Aš noriu pamatyti karalių. Jei aš kaltas, tegul nužudo mane”. 
\par 33 Joabas, atėjęs pas karalių, jam viską pranešė. Tada karalius pasikvietė Abšalomą. Jis atėjo pas karalių ir nusilenkė veidu iki žemės, o karalius pabučiavo Abšalomą.



\chapter{15}

\par 1 Abšalomas įsigijo vežimų bei žirgų ir penkiasdešimt vyrų, kurie bėgdavo pirma jo. 
\par 2 Jis keldavosi anksti ir atsistodavo šalia kelio prie vartų. Jei koks žmogus su savo byla ateidavo į karaliaus teismą, Abšalomas pasišaukdavo jį ir klausdavo: “Iš kurio tu miesto?” Kai tas atsakydavo: “Tavo tarnas yra iš tokios Izraelio giminės”, 
\par 3 Abšalomas jam sakydavo: “Tavo reikalas yra geras ir teisingas, bet nėra pas karalių kas tave išklausytų. 
\par 4 Jei mane paskirtų krašto teisėju, tai pas mane galėtų ateiti kiekvienas, turįs bylą ar skundą, ir aš apginčiau jo teises”. 
\par 5 Jei kas norėdavo Abšalomui nusilenkti, jis ištiesęs ranką tą apkabindavo ir pabučiuodavo. 
\par 6 Taip Abšalomas elgėsi su visais izraelitais, kurie ateidavo į karaliaus teismą. Abšalomas tokiu būdu pavogė izraelitų širdis. 
\par 7 Praėjus keturiasdešimčiai metų, Abšalomas tarė karaliui: “Prašau, leisk man eiti į Hebroną ir įvykdyti įžadą, kurį daviau Viešpačiui. 
\par 8 Gyvendamas Gešūre, Sirijoje, daviau įžadą: ‘Jei Viešpats mane parves atgal į Jeruzalę, aš tarnausiu Jam’ ”. 
\par 9 Karalius jam tarė: “Eik ramybėje”. Abšalomas pakilo ir išėjo į Hebroną. 
\par 10 Jis slapta išsiuntė pasiuntinius į visas Izraelio gimines su pranešimu: “Išgirdę trimito garsą, sakykite: ‘Abšalomas karaliauja Hebrone’ ”. 
\par 11 Iš Jeruzalės su Abšalomu išvyko du šimtai pakviestų vyrų. Jie ėjo nieko pikto nemanydami ir nieko nežinodami. 
\par 12 Abšalomas aukodamas dar pasikvietė iš Gilojo miesto Ahitofelį, Dovydo patarėją. Taip sąmokslas išsiplėtė, nes žmonės nuolat ėjo pas Abšalomą. 
\par 13 Pasiuntinys atėjo pas Dovydą ir pranešė: “Izraelio žmonių širdys prisirišo prie Abšalomo”. 
\par 14 Dovydas sakė visiems savo tarnams Jeruzalėje: “Bėkime, nes nėra kito išsigelbėjimo nuo Abšalomo. Skubėkime, kad jis mūsų neužkluptų, nepadarytų mums pikta ir neišžudytų miesto kardu”. 
\par 15 Karaliaus tarnai sakė karaliui: “Štai tavo tarnai vykdys visa, ką karalius įsakys”. 
\par 16 Karalius išėjo kartu su visais namiškiais, liko tik dešimt sugulovių namų saugoti. 
\par 17 Karalius išėjo, ir visi žmonės sekė paskui jį, ir jie sustojo toli nuo namų. 
\par 18 Visi jo tarnai ėjo šalia jo, o keretai, peletai ir gatiečiai, šeši šimtai vyrų, kurie buvo atėję su juo iš Gato, ėjo pirma karaliaus. 
\par 19 Dovydas klausė gatietį Itają: “Kodėl tu eini su mumis? Grįžk ir pasilik pas karalių. Tu juk esi svetimšalis, atėjęs čia iš savo krašto. 
\par 20 Vakar atėjai, o šiandien ar turėtum eiti su mumis? Aš turiu eiti, o tu grįžk su savo broliais. Gailestingumas ir tiesa tebūna su tavimi”. 
\par 21 Itajas atsakė karaliui: “Kaip Viešpats gyvas ir mano valdovas karalius gyvas! Kur bus mano karalius, mirtyje ar gyvenime, ten bus ir jo tarnas”. 
\par 22 Dovydas tarė Itajui: “Tai eik su manimi”. Itajas perėjo su visais savo vyrais ir jų šeimomis. 
\par 23 Visas kraštas garsiai verkė, ir visi žmonės perėjo į kitą pusę, ir karalius perėjo per Kedrono upelį ir traukė dykumos link. 
\par 24 Atėjo Cadokas ir levitai, nešdami Dievo Sandoros skrynią. Jie pastatė Dievo skrynią, ir Abjataras ėjo aukštyn, kol visi pasitraukė iš miesto. 
\par 25 Karalius Cadokui tarė: “Nešk Dievo skrynią atgal į miestą. Jei rasiu malonę Viešpaties akyse, Jis mane parves ir parodys man ją ir savo buveinę. 
\par 26 Jei Jis man sakys: ‘Aš nepatenkintas tavimi’, tai tegul daro su manimi, kaip Jam patinka”. 
\par 27 Tada karalius kalbėjo kunigui Cadokui: “Tu esi regėtojas. Tu, tavo sūnus Ahimaacas ir Abjataro sūnus Jehonatanas grįžkite ramybėje į miestą. 
\par 28 Štai aš pasiliksiu dykumoje, kol gausiu iš jūsų pranešimą”. 
\par 29 Cadokas ir Abjataras parnešė Dievo skrynią į Jeruzalę ir ten pasiliko. 
\par 30 Dovydas ėjo aukštyn Alyvų kalno šlaitu verkdamas, apsigaubęs galvą ir basas. Visi su juo buvę žmonės, apsigaubę galvas, ėjo aukštyn taip pat verkdami. 
\par 31 Dovydas, sužinojęs, kad Ahitofelis yra tarp sąmokslininkų pas Abšalomą, meldėsi: “Viešpatie, meldžiu, paversk Ahitofelio patarimą niekais”. 
\par 32 Kalno viršūnėje, kur Dovydas garbino Dievą, jį pasitiko arkietis Hušajas su perplėštais drabužiais ir žemėmis apibarstyta galva. 
\par 33 Dovydas jam tarė: “Jei eisi su manimi, būsi man našta, 
\par 34 o jei grįši į miestą ir prisidėsi prie Abšalomo, sakydamas jam: ‘Aš noriu būti tavo tarnas, karaliau. Buvau tavo tėvo tarnas praeityje, o dabar noriu būti tavo tarnas’, tai Ahitofelio patarimą galėsi niekais paversti. 
\par 35 Su tavimi bus kunigai Cadokas ir Abjataras. Pranešk jiems visa, ką nugirsi karaliaus namuose. 
\par 36 Pas juos yra abu jų sūnūs: Cadoko sūnus Ahimaacas ir Abjataro­Jehonatanas; per juos perduok man viską, ką išgirsi”. 
\par 37 Dovydo draugas Hušajas sugrįžo į Jeruzalę, Abšalomui įeinant į ją.



\chapter{16}


\par 1 Dovydui paėjus kiek toliau nuo kalno viršūnės, jį sutiko Mefi Bošeto tarnas Ciba su pora asilų, apkrautų dviem šimtais duonos kepalų, šimtu ryšulių džiovintų vynuogių, šimtu ryšulių vasaros vaisių ir odine vyno. 
\par 2 Karalius klausė Cibą: “Kam tau visa tai?” Ciba atsakė: “Asilai­ karaliaus namiškiams joti, duona ir vaisiai­jaunuoliams valgyti, o vynas­nusilpusiems dykumoje atgaivinti”. 
\par 3 Karalius klausė: “Kur yra tavo valdovo sūnus?” Ciba atsakė karaliui: “Jis pasiliko Jeruzalėje, sakydamas: ‘Šiandien Izraelis man sugrąžins mano tėvo karalystę’ ”. 
\par 4 Karalius tarė Cibai: “Kas buvo Mefi Bošeto, dabar priklauso tau”. Ciba atsakė: “Nuolankiai maldauju tave, mano valdove karaliau, leisk man surasti malonę tavo akyse”. 
\par 5 Karaliui Dovydui atvykus į Bahurimą, Sauliaus giminės vyras Šimis, Geros sūnus, išėjo ir keikė eidamas. 
\par 6 Jis mėtė akmenimis Dovydą bei visus jo tarnus; karalius buvo apsuptas žmonių ir karių. 
\par 7 Šimis keikdamas sakė: “Išeik, išeik, tu kraugery, Belialo žmogau. 
\par 8 Viešpats atmokėjo tau už visą Sauliaus namų kraują, kurio vietoje tu pasidarei karaliumi. Viešpats atidavė karalystę tavo sūnui Abšalomui. Dabar tu patekai nemalonėn, nes esi kraugerys”. 
\par 9 Cerujos sūnus Abišajas paklausė karaliaus: “Kodėl šitas pastipęs šuo keikia mano valdovą karalių? Leisk man nueiti ir nuimti jam galvą”. 
\par 10 Karalius atsakė: “Kas man ir jums, Cerujos sūnūs! Jis keikia, kadangi Viešpats jam liepė: ‘Keik Dovydą!’ Kas gali klausti: ‘Kodėl taip darai?’ ” 
\par 11 Dovydas kalbėjo Abišajui ir visiems tarnams: “Štai mano paties sūnus ieško mano gyvybės. Tuo labiau šis benjaminas. Palikite jį ramybėje, tegul jis keikia, nes jam taip liepė Viešpats. 
\par 12 Gal Viešpats pažvelgs į mano sielvartą ir man atmokės geru už jo šiandieninius keiksmus”. 
\par 13 Dovydas ėjo keliu su savo vyrais, o Šimis­kalno šlaitu šalia jo keikdamas, mėtydamas akmenimis ir barstydamas dulkes. 
\par 14 Pagaliau karalius ir visi jo žmonės pavargo ir ten ilsėjosi. 
\par 15 Abšalomas su visais žmonėmis atėjo į Jeruzalę; Achitofelis buvo su juo. 
\par 16 Arkietis Hušajas, Dovydo draugas, atėjęs pas Abšalomą, tarė: “Tegyvuoja karalius! Tegyvuoja karalius!” 
\par 17 Abšalomas paklausė jo: “Ar toks tavo dėkingumas draugui? Kodėl nėjai su savo draugu?” 
\par 18 Hušajas atsakė Abšalomui: “Aš priklausau tam, kurį išrinko Viešpats, Izraelio vyrai ir šitie žmonės. 
\par 19 Kam gi aš galėčiau tarnauti, jei ne jo sūnui. Kaip tarnavau tavo tėvui, taip noriu tarnauti tau”. 
\par 20 Abšalomas tarė Ahitofeliui: “Patark, ką mums daryti?” 
\par 21 Ahitofelis tarė Abšalomui: “Įeik pas savo tėvo suguloves, kurias jis paliko namų saugoti. Kai izraelitai sužinos, kad tapai pasibjaurėtinas savo tėvui, visų tavo šalininkų rankos sustiprės”. 
\par 22 Jie pastatė Abšalomui palapinę ant stogo, o jis, visiems izraelitams matant, įėjo pas savo tėvo suguloves. 
\par 23 Tais laikais Ahitofelio patarimas būdavo labai vertinamas, lyg jį duotų pats Dievas. Jo patarimų klausė Dovydas ir Abšalomas.



\chapter{17}

\par 1 Ahitofelis tarė Abšalomui: “Leisk man išsirinkti dvylika tūkstančių vyrų ir vytis Dovydą šią naktį. 
\par 2 Aš jį užpulsiu, kol jis dar nuvargęs ir nuleidęs rankas, ir jį išgąsdinsiu. Visi jo vyrai pabėgs, o karalių nužudysiu. 
\par 3 Po to aš atvesiu visus žmones pas tave. Kai nebus to, kurio gyvybės ieškai, visa tauta gyvens ramybėje”. 
\par 4 Tas pasiūlymas patiko Abšalomui ir visiems Izraelio vyresniesiems. 
\par 5 Abšalomas tarė: “Pakvieskite arkietį Hušają. Paklausysime, ką jis pasakys”. 
\par 6 Hušajui atėjus, Abšalomas jam tarė: “Taip kalbėjo Ahitofelis. Ar mums elgtis pagal jo patarimą? Jei ne, tai tu kalbėk”. 
\par 7 Hušajas atsakė Abšalomui: “Šitas Ahitofelio patarimas nėra geras. 
\par 8 Tu žinai, kad tavo tėvas ir jo vyrai yra karžygiai ir jie yra taip įnirtę, kaip lokė lauke, netekusi jauniklių. Be to, tavo tėvas yra patyręs karys ir nenakvos kartu su žmonėmis. 
\par 9 Jis dabar yra pasislėpęs kokiame nors urve ar kitoje vietoje. Jei juos užpulsime ir keli iš mūsų žus, visi, išgirdę tai, sakys: ‘Sumušti žmonės, kurie sekė Abšalomą’. 
\par 10 Tada ir narsiausias, kurio širdis yra kaip liūto, drebės iš baimės, nes visas Izraelis žino, kad tavo tėvas yra galingas vyras ir su juo esantys yra narsūs vyrai. 
\par 11 Todėl aš patariu sušaukti visus izraelitus nuo Dano iki Beer Šebos, kad jų būtų kaip smilčių pajūryje, ir tu pats eisi į mūšį su jais. 
\par 12 Suradę jį, užpulsime, kaip rasa krintanti ant žemės, ir nepaliksime gyvo nė vieno iš vyrų, kurie yra su juo. 
\par 13 O jei jis pasitrauks į miestą, tai izraelitai atsineš virvių ir jį su visu miestu nuvilksime į upę, nė akmenėlio nepaliksime”. 
\par 14 Abšalomas ir visi izraelitai tarė: “Arkiečio Hušajo patarimas yra geresnis už Ahitofelio”. Nes Viešpats padarė, kad geras Ahitofelio patarimas nueitų niekais ir Viešpats galėtų užtraukti nelaimę Abšalomui. 
\par 15 Po to Hušajas pranešė kunigams Cadokui ir Abjatarui apie savo ir Ahitofelio patarimus Abšalomui bei Izraelio vyresniesiems. 
\par 16 Jis sakė: “Skubiai siųskite ir pasakykite Dovydui nenakvoti šiąnakt dykumoje, bet tuojau persikelti kitur, kad nepražūtų karalius ir visi su juo esantys žmonės”. 
\par 17 Jehonatanas ir Ahimaacas buvo apsistoję prie En Rogelio šaltinio. Tarnaitė nuėjusi jiems pranešė, ir jie išėjo pasakyti karaliui Dovydui. Jie patys negalėjo pasirodyti mieste. 
\par 18 Tačiau vienas jaunuolis juos pastebėjo ir pranešė Abšalomui. Juodu skubiai nuėjo pas vieną žmogų Bahurime, kuris turėjo šulinį savo kieme, ir pasislėpė šulinyje. 
\par 19 Šeimininkė uždengė šulinį dangčiu ir pabarstė ant jo miežinių kruopų, kad nebūtų įtarimo. 
\par 20 Abšalomo tarnai, atėję į tuos namus, klausė šeimininkės: “Kur yra Ahimaacas ir Jehonatanas?” Moteris jiems atsakė: “Jie perėjo upelį”. Paieškoję ir neradę jų, jie sugrįžo į Jeruzalę. 
\par 21 Abšalomo tarnams nuėjus, Ahimaacas ir Jehonatanas išlipo iš šulinio ir nuėję pranešė karaliui Dovydui: “Skubiai pasitrauk už Jordano, nes štai ką patarė prieš tave Ahitofelis”. 
\par 22 Dovydas ir jo žmonės perėjo Jordaną. Rytui išaušus, visi buvo kitoje Jordano pusėje. 
\par 23 Ahitofelis, matydamas, kad jo patarimo neklausoma, pasibalnojo asilą ir parvyko namo į savo miestą. Jis, sutvarkęs savo namų reikalus, pasikorė ir buvo palaidotas savo tėvo kape. 
\par 24 Dovydui atvykus į Machanaimą, Abšalomas su savo vyrais perėjo per Jordaną. 
\par 25 Abšalomas paskyrė Amasą kariuomenės vadu Joabo vieton. Amasa buvo sūnus izraelito Itros, kuris buvo vedęs Abigailę, Nahašo dukterį, Joabo motinos Cerujos seserį. 
\par 26 Abšalomas su izraelitais pasistatė stovyklą Gileado krašte. 
\par 27 Dovydui atvykus į Machanaimą, Nahašo sūnus Šobis iš amonitų Rabos, Amielio sūnus Machyras iš Lo Debaro ir Barzilajas iš Gileado Rogelimo 
\par 28 atgabeno antklodžių, dubenių, molinių indų, kviečių, miežių, miltų, paskrudintų grūdų, pupų, lęšių, 
\par 29 medaus, sviesto ir sūrių. Visa tai jie atnešė Dovydui ir jo žmonėms, sakydami: “Žmonės dykumoje yra išalkę, nuvargę ir ištroškę”.



\chapter{18}

\par 1 Dovydas apžiūrėjo savo žmones ir paskyrė jiems tūkstantininkus ir šimtininkus. 
\par 2 Dovydas paskirstė karius į tris dalis; jiems vadovavo Joabas, Cerujos sūnus Abišajias, Joabo brolis, ir gatietis Itajas. Tuomet karalius tarė kariams: “Aš eisiu su jumis”. 
\par 3 Žmonės atsakė: “Tu neisi su mumis! Jei mes turėsime bėgti arba pusė mūsų žus, jie mūsų nežiūrės. Tu vertas dešimt tūkstančių mūsų. Būtų geriau, jei teiktum mums pagalbą iš miesto”. 
\par 4 Karalius jiems atsakė: “Darysiu, kas jums atrodo geriausia”. Karalius stovėjo prie vartų, o visi žmonės išėjo šimtais ir tūkstančiais. 
\par 5 Karalius įsakė Joabui, Abišajui ir Itajui: “Saugokite mano vaiką Abšalomą!” Visi žmonės girdėjo tą karaliaus įsakymą vadams. 
\par 6 Žmonės išėjo prieš Izraelį. Mūšis įvyko Efraimo miške. 
\par 7 Izraelio žmonės buvo nugalėti Dovydo tarnų, ir buvo nužudyta tą dieną dvidešimt tūkstančių vyrų. 
\par 8 Mūšis išsiplėtė visoje apylinkėje. Miške žuvo daugiau žmonių, negu nuo kardo. 
\par 9 Abšalomas susitiko su Dovydo tarnais. Abšalomas jojo ant mulo. Kai jo mulas bėgo po dideliu ąžuolu, Abšalomo galva įstrigo tarp ąžuolo šakų, ir jis liko kaboti tarp dangaus ir žemės, o jo mulas, ant kurio jis jojo, nubėgo. 
\par 10 Vienas vyras, tai pamatęs, pranešė Joabui: “Aš mačiau Abšalomą, kabantį ąžuole”. 
\par 11 Joabas jam tarė: “Jei jį matei, kodėl jo neužmušei. Aš būčiau tau davęs dešimt sidabrinių ir diržą”. 
\par 12 Vyras atsakė Joabui: “Jei duotum ir tūkstantį sidabrinių, nepakelčiau savo rankos prieš karaliaus sūnų; juk karalius mums girdint įsakė tau, Abišajui ir Itajui: ‘Kas bebūtų, saugokite mano sūnų Abšalomą’. 
\par 13 Jei aš būčiau taip padaręs, mano gyvybė būtų pavojuje, nes nuo karaliaus nieko nėra paslėpta, ir tu pats būtum prieš mane”. 
\par 14 Joabas tarė: “Aš negaišiu čia su tavimi”. Paėmęs tris ietis į rankas, įsmeigė jas Abšalomui į širdį, kuris dar gyvas kabojo ąžuole. 
\par 15 Po to atėję dešimt jaunuolių ginklanešių nužudė Abšalomą. 
\par 16 Joabui sutrimitavus, žmonės nustojo vytis izraelitus, nes Joabas juos sulaikė. 
\par 17 Paėmę Abšalomą, įmetė jį į gilią duobę miške ir ant jo sukrovė didelę akmenų krūvą. Izraelitai pabėgo į savo palapines. 
\par 18 Abšalomas, dar gyvas būdamas, pasistatė sau paminklą Karaliaus slėnyje, galvodamas: “Neturiu sūnaus, kuris išlaikytų mano vardą”. Jis pavadino paminklą savo vardu. Ir iki šios dienos jis vadinamas Abšalomo paminklu. 
\par 19 Cadoko sūnus Ahimaacas prašė Joabo: “Leisk man nubėgti ir pranešti karaliui, kaip Viešpats atlygino jo priešams”. 
\par 20 Joabas atsakė jam: “Ne šiandien! Kitą dieną galėsi pranešti žinią, bet šiandien tau neleidžiu, nes žuvo karaliaus sūnus”. 
\par 21 Joabas įsakė Kušui eiti ir pranešti karaliui, ką matė. Kušas nusilenkė Joabui ir nubėgo. 
\par 22 Cadoko sūnus Ahimaacas dar kartą prašė Joabą: “Kas bebūtų, leisk man bėgti paskui Kušą”. Joabas tarė: “Kodėl tu nori bėgti, mano sūnau? Tu nenuneši geros žinios”. 
\par 23 Bet Ahimaacas tarė: “Kas bebūtų, aš bėgsiu”. Joabas jam atsakė: “Bėk!” Ahimaacas bėgo lygumos keliu ir pralenkė Kušą. 
\par 24 Dovydas sėdėjo tarpuvartėje, o sargybinis stovėjo ant vartų stogo ir mūro sienos. Jis, pakėlęs akis, pamatė bėgantį vieną vyrą. 
\par 25 Sargybinis pranešė tai karaliui. Karalius tarė: “Jei jis vienas, tai su žinia”. O tas vis artėjo. 
\par 26 Tuo tarpu sargybinis pamatė kitą bėgantį vyrą ir vėl pranešė vartininkui: “Štai kitas vyras atbėga!” Karalius tarė: “Ir šitas atneša žinią”. 
\par 27 Sargybinis tarė: “Pirmasis vyras bėga panašiai kaip Cadoko sūnus Ahimaacas”. Karalius atsiliepė: “Tai geras vyras, jis atneša gerą žinią”. 
\par 28 Ahimaacas priartėjęs tarė karaliui: “Ramybė tau!” Parpuolęs veidu į žemę prieš karalių, kalbėjo: “Palaimintas Viešpats, tavo Dievas, kuris atidavė mums vyrus, kurie buvo pakėlę ranką prieš tave, karaliau”. 
\par 29 Karalius paklausė: “Ar Abšalomas gyvas?” Ahimaacas atsakė: “Mačiau didelį sąmyšį, prieš išbėgdamas pas tave, bet nežinau, kas atsitiko”. 
\par 30 Karalius įsakė: “Pasitrauk ir atsistok čia”. Jis pasitraukė ir stovėjo. 
\par 31 Atbėgo ir Kušas. Jis tarė: “Gera žinia mano valdovui karaliui. Šiandien Viešpats atlygino visiems, kurie prieš tave sukilo”. 
\par 32 Karalius paklausė: “Ar Abšalomas gyvas?” Kušas atsakė: “Tegul visiems karaliaus priešams taip atsitinka, kaip tam jaunuoliui”. 
\par 33 Karalius labai susijaudino. Nuėjęs į aukštutinį kambarį, jis verkė: “Mano sūnau Abšalomai! Mano sūnau, mano sūnau Abšalomai! O kad aš būčiau miręs tavo vietoje, Abšalomai! Mano sūnau, mano sūnau!”



\chapter{19}


\par 1 Joabui buvo pranešta, kad karalius verkia ir gedi Abšalomo. 
\par 2 Tos dienos pergalė virto gedulu visiems žmonėms, nes žmonės girdėjo kalbant, kad karalius sielojasi dėl savo sūnaus. 
\par 3 Žmonės grįžo į miestą tylomis, kaip grįžta susigėdę tie, kurie pabėgo iš mūšio. 
\par 4 Karalius, apsidengęs veidą, šaukė garsiu balsu: “Mano sūnau Abšalomai! Abšalomai, mano sūnau, mano sūnau!” 
\par 5 Joabas, nuėjęs pas karalių, tarė: “Tu šiandien sugėdinai visus savo tarnus, kurie išgelbėjo tave bei tavo sūnų, tavo dukterų, tavo žmonų ir sugulovių gyvybes. 
\par 6 Tu myli tuos, kurie tavęs nekenčia, ir nekenti tų, kurie tave myli. Tu šiandien parodei, kad kunigaikščiai ir tauta tau nieko nereiškia. Dabar matau, kad jei Abšalomas būtų gyvas, o mes visi būtume žuvę, tu būtum patenkintas. 
\par 7 Taigi dabar eik ir kalbėk savo žmonėms. Prisiekiu Viešpačiu, jei neišeisi, tai rytoj neturėsi jų nė vieno; tai bus tau didžiausia nelaimė iš visų, kurias patyrei nuo pat savo jaunystės iki šios dienos”. 
\par 8 Karalius atsikėlė ir atsisėdo vartuose. Ir visiems žmonėms buvo pranešta: “Karalius sėdi vartuose”. Tuomet visi žmonės ėjo pas karalių. Izraelitai išsibėgiojo kiekvienas į savo palapines. 
\par 9 Tarp izraelitų giminių kilo ginčai ir nesutarimai: “Karalius mus išlaisvino iš mūsų priešų, jis mus išgelbėjo iš filistinų, o dabar turėjo bėgti iš krašto nuo Abšalomo. 
\par 10 Abšalomas, kurį buvome patepę valdovu, žuvo mūšyje. Kodėl dabar delsiame parsivesti karalių atgal?” 
\par 11 Dovydas pasiuntė pas kunigus Cadoką ir Abjatarą ir prašė kalbėti Judo vyresniesiems: “Kodėl jūs norite būti paskutiniai parvedant karalių į Jeruzalę? Nes tos kalbos Izraelyje pasiekė net karaliaus namus. 
\par 12 Jūs esate mano broliai, mano kūnas ir kaulai. Kodėl jūs norite būti paskutiniai parvedant karalių?” 
\par 13 O Amasai kalbėjo: “Argi tu nesi mano kūnas ir kaulai? Tegul Dievas padaro man tai ir dar daugiau, jei tu netapsi mano kariuomenės vadu vietoj Joabo”. 
\par 14 Tuo būdu Dovydas laimėjo visų Judo žmonių širdis; jis buvo pakviestas grįžti su visais savo tarnais. 
\par 15 Karalius grįžo ir pasiekė Jordaną. Judo žmonės atėjo į Gilgalą sutikti karalių ir perkelti jį per Jordaną. 
\par 16 Gero sūnus Šimis, benjaminas iš Bahurimo, taip pat atskubėjo su Judo vyrais karaliaus Dovydo pasitikti. 
\par 17 Su juo buvo tūkstantis vyrų iš Benjamino, taip pat Ciba, Sauliaus namų tarnas, su penkiolika sūnų bei dvidešimt tarnų. Jie perėjo per Jordaną pirma karaliaus 
\par 18 ir padėjo jo šeimynai persikelti per Jordaną, patarnaudami jam. Gero sūnus Šimis puolė ant žemės prieš karalių, kai jis buvo persikėlęs per Jordaną, 
\par 19 ir tarė: “Neįskaityk mano kaltės, mano valdove, ir neprisimink to, ką tavo tarnas darė tau išeinant iš Jeruzalės, ir nelaikyk to savo širdyje. 
\par 20 Žinau, kad esu kaltas. Šiandien atvykau pirmas iš visų Juozapo namų savo valdovo karaliaus pasitikti”. 
\par 21 Cerujos sūnus Abišajas tarė: “Ar Šimį nereikėtų bausti mirtimi už tai, kad jis keikė Viešpaties pateptąjį?” 
\par 22 Dovydas atsakė: “Kas man ir jums, Cerujos sūnūs, kad jūs šiandien man prieštaraujate. Ar šiandien reikėtų ką nors bausti mirtimi Izraelyje? Juk aš šiandien žinau, kad esu Izraelio karalius”. 
\par 23 Karalius pasakė Šimiui: “Tu nemirsi”, ir prisiekė jam. 
\par 24 Sauliaus sūnus Mefi Bošetas atvyko karaliaus pasitikti. Jis nebuvo plovęs savo kojų, kirpęs barzdos ir plovęs drabužių nuo tos dienos, kai karalius išėjo, iki jis sugrįžo. 
\par 25 Jam atvykus iš Jeruzalės karaliaus pasitikti, karalius klausė: “Mefi Bošetai, kodėl nėjai su manimi?” 
\par 26 Mefi Bošetas atsakė: “Mano valdove karaliau, mano tarnas apgavo mane. Tavo tarnas buvo jam įsakęs pabalnoti asilą, kad raitas galėčiau vykti su karaliumi, nes aš esu luošas, 
\par 27 o jis apšmeižė mane tau. Tačiau karalius kaip Dievo angelas. Taigi daryk, kas tau patinka. 
\par 28 Visi mano tėvo namai buvo tarsi mirę žmonės mano valdovo karaliaus akyse. O tu pasodinai savo tarną valgyti prie savo stalo. Argi dar daugiau galėčiau tikėtis iš karaliaus?” 
\par 29 Karalius jam atsakė: “Kam tu visa tai kalbi. Aš pasakiau, kad tu ir Ciba pasidalintumėte žemę”. 
\par 30 Mefi Bošetas atsakė karaliui: “Tegul Ciba viską ima, man gana to, kad mano karalius sveikas sugrįžo į savo namus”. 
\par 31 Gileadietis Barzilajas atėjo iš Rogelimo palydėti karalių per Jordaną. 
\par 32 Jis buvo labai senas žmogus, aštuoniasdešimties metų amžiaus. Karaliui gyvenant Machanaime, jis aprūpino karalių maistu, nes Barzilajas buvo labai turtingas. 
\par 33 Karalius sakė Barzilajui: “Eime su manimi į Jeruzalę, ten aš tave viskuo aprūpinsiu”. 
\par 34 Barzilajas atsakė karaliui: “Kiek man beliko gyventi, kad aš eičiau su karaliumi į Jeruzalę? 
\par 35 Man šiandien jau aštuoniasdešimt metų. Ar aš atskirsiu, kas gera ir kas bloga? Ar pajusiu tavo valgių ir gėrimų skonį? Ar beišgirsiu dainuojančių vyrų ar moterų balsus? Tai kodėl tavo tarnas turėtų būti našta mano valdovui karaliui? 
\par 36 Tavo tarnas tik truputį palydės tave už Jordano. Kodėl tad karalius turėtų man taip atsilyginti? 
\par 37 Prašau, leisk man grįžti į savo miestą, kad galėčiau mirti ten, kur mano tėvo ir motinos kapas. Tegul tavo tarnas Kimhamas eina su tavimi, karaliau, ir jam daryk tai, kas tau atrodys tinkama”. 
\par 38 Karalius atsakė: “Kimhamas eis su manimi, aš jam atlyginsiu pagal tavo norą ir, ko prašysi, padarysiu”. 
\par 39 Visi žmonės ir karalius persikėlė per Jordaną. Karalius pabučiavo Barzilają ir jį palaimino. Barzilajas sugrįžo į savo namus. 
\par 40 Karaliui keliaujant į Gilgalą, Kimhamas ir visi Judo žmonės bei pusė Izraelio žmonių lydėjo karalių. 
\par 41 Visi Izraelio vyrai atėjo pas karalių ir tarė: “Kodėl tave pasisavino mūsų broliai iš Judo ir perkėlė per Jordaną tave su šeimyna ir visais tavo žmonėmis?” 
\par 42 Judo vyrai atsakė izraelitams: “Todėl, kad karalius yra mums artimesnis. Kodėl pykstate? Ar karalius mus maitino? Ar jis davė mums dovanų?” 
\par 43 Izraelitai atsakė Judo vyrams: “Mes turime dešimt dalių pas karalių ir mes turime daugiau teisių į Dovydą negu jūs. Kodėl mus paniekinote? Argi ne mes pirmieji turėjome sugrąžinti karalių?” Judo vyrai kalbėjo griežčiau už izraelitus.



\chapter{20}


\par 1 Tuo laiku ten buvo Belialo žmogus, vardu Šeba, Bichrio sūnus, benjaminas. Sutrimitavęs jis šaukė: “Mes neturime dalies Dovyde nė Jesės sūnaus paveldėjime. Izraelitai, kiekvienas į savo palapines!” 
\par 2 Visi izraelitai, atsiskyrę nuo Dovydo, sekė Bichrio sūnų Šebą, o Judo vyrai liko ištikimi Dovydui nuo Jordano iki Jeruzalės. 
\par 3 Dovydas sugrįžo į Jeruzalę. Karalius dešimt savo sugulovių, kurias buvo palikęs namų saugoti, apgyvendino atskiruose namuose; jis jas išlaikė, tačiau neįėjo pas jas. Taip jos buvo uždarytos iki savo mirties ir gyveno kaip našlės. 
\par 4 Karalius įsakė Amasai per tris dienas sušaukti visus Judo vyrus ir pačiam atvykti. 
\par 5 Amasa išėjo surinkti Judo vyrų, tačiau užtruko ilgiau, negu buvo nustatyta. 
\par 6 Tuomet Dovydas tarė Abišajui: “Bichrio sūnus Šeba padarys mums daugiau žalos negu Abšalomas. Imk savo valdovo tarnus ir vykis jį. Kitaip jis susiras sustiprintų miestų ir paspruks nuo mūsų”. 
\par 7 Joabo kariai, keretai, peletai ir visi karžygiai iš Jeruzalės vijosi Bichrio sūnų Šebą. 
\par 8 Kai jie atėjo į Gibeoną prie didžiojo akmens, juos pasitiko Amasa. Joabas vilkėjo kario rūbais, susijuosęs diržu, ant kurio buvo pritvirtintas kardas makštyje; kardas buvo lengvai ištraukiamas. 
\par 9 Joabas klausė Amasos: “Ar tu esi sveikas, broli?” ir paėmė dešine ranka jam už barzdos, lyg norėdamas jį pabučiuoti. 
\par 10 Amasa nepastebėjo kardo Joabo rankoje; tuo metu Joabas dūrė jam į pilvą, ir visi jo viduriai išvirto ant žemės. Antro smūgio nereikėjo, ir Amasa mirė. Joabas ir jo brolis Abišajas vijosi Bichrio sūnų Šebą. 
\par 11 Vienas Joabo vyrų, pasilikęs prie Amasos, sakė: “Kas su Joabu ir už Dovydą, sekite Joabą”. 
\par 12 Amasa gulėjo kruvinas vieškelio viduryje. Tas vyras, matydamas žmones sustojant, patraukė Amasos kūną nuo vieškelio ir užmetė ant jo drabužį. 
\par 13 Kai jis buvo patrauktas nuo vieškelio, visi žmonės kartu su Joabu vijosi Bichrio sūnų Šebą. 
\par 14 Šeba perėjo visą Izraelį iki Abel Bet Maakos, ir visi beritai susirinkę sekė jį. 
\par 15 Joabas su savo žmonėmis atėjo ir apgulė Abel Bet Maaką, supylė pylimą aplink ir ruošėsi griauti miesto sienas. 
\par 16 Viena išmintinga moteris iš miesto šaukė: “Klausykite! Klausykite! Pasakykite Joabui, kad jis prieitų ir aš galėčiau jam kai ką pasakyti”. 
\par 17 Jam atėjus, moteris klausė: “Ar tu Joabas?” Jis atsakė: “Taip”. Ji sakė: “Paklausyk savo tarnaitės žodžių”. Jis tarė: “Klausau”. 
\par 18 Ji kalbėjo: “Seniau sakydavo: ‘Tepasiklausia Abelyje’. Ir taip išspręsdavo bylą. 
\par 19 Mes esame taikingi ir ištikimi Izraeliui. Tu sieki sunaikinti miestą, kuris yra motina Izraeliui. Kodėl nori praryti Viešpaties paveldėjimą?” 
\par 20 Joabas atsakė: “Tebūna tai toli nuo manęs, kad aš praryčiau ar sugriaučiau. 
\par 21 Čia ne toks reikalas! Vienas vyras iš Efraimo kalnų, vardu Šeba, Bichrio sūnus, pakėlė ranką prieš karalių Dovydą. Atiduokite jį, ir aš atsitrauksiu nuo miesto”. Moteris atsakė Joabui: “Jo galvą tau numes per sieną”. 
\par 22 Ta moteris ėjo pas žmones ir kalbėjo jiems išmintingai. Jie nukirto Bichrio sūnui Šebai galvą ir ją numetė Joabui. Tuomet Joabas trimitavo, jie atsitraukė nuo miesto, ir kiekvienas nuėjo į savo palapines. O Joabas sugrįžo į Jeruzalę pas karalių. 
\par 23 Joabas buvo visos Izraelio kariuomenės vadas, Jehojados sūnus Benajas buvo keretų bei peletų viršininkas, 
\par 24 Adoramas buvo mokesčių rinkėjas, Ahiludo sūnus Juozapatas buvo metraštininkas, 
\par 25 Ševa buvo raštininkas, Cadokas bei Abjataras­kunigai, 
\par 26 o Ira, jajiritas, buvo vyriausiasis Dovydo valdytojas.



\chapter{21}

\par 1 Dovydo dienomis buvo badas trejus metus iš eilės. Dovydas klausė Viešpaties, ir Viešpats atsakė: “Tai dėl Sauliaus ir jo kraugeriškų namų, nes jis išžudė gibeoniečius”. 
\par 2 Gibeoniečiai buvo ne izraelitų kilmės, bet iš amoritų likučio. Izraelitai buvo jiems prisiekę jų neišnaikinti, tačiau Saulius iš savo uolumo dėl Izraelio ir Judo siekė juos išžudyti. 
\par 3 Dovydas klausė gibeoniečių: “Ką aš galėčiau jums padaryti ir kaip sutaikinti jus, kad jūs palaimintumėte Viešpaties paveldėjimą?” 
\par 4 Gibeoniečiai jam atsakė: “Mes nereikalaujame nei sidabro, nei aukso iš Sauliaus ir jo namų ir dėl mūsų nieko nežudyk Izraelyje”. Dovydas sakė: “Kaip jūs pasakysite, taip padarysiu dėl jūsų”. 
\par 5 Jie atsakė karaliui: “Tas žmogus, kuris mus žudė ir siekė visus išnaikinti Izraelio krašte, 
\par 6 tegul duoda mums septynis vyrus iš jo sūnų, kuriuos mes pakarsime Sauliaus Gibėjoje, Viešpaties akivaizdoje”. Karalius atsakė: “Aš duosiu juos”. 
\par 7 Tačiau karalius gailėjo Mefi Bošeto, sūnaus Jehonatano, sūnaus Sauliaus, dėl Viešpaties priesaikos, kuri saistė Dovydą su Jehonatanu. 
\par 8 Karalius ėmė abu Ajos dukters Ricpos ir Sauliaus sūnus Armonį ir Mefi Bošetą ir penkis Sauliaus dukters Mikalės ir meholiečio Barzilajaus sūnaus Adrielio sūnus 
\par 9 ir juos atidavė gibeoniečiams. Jie juos pakorė ant kalvos Viešpaties akivaizdoje. Taip tie septyni mirė kartu. Jie buvo nužudyti pirmosiomis pjūties dienomis, pradėjus pjauti miežius. 
\par 10 Ajos duktė Ricpa pasitiesė ašutinę ant uolos pjūties pradžioje ir buvo, kol pradėjo lyti. Ji sėdėjo ir neleido padangių paukščiams nutūpti ant jų dieną nė laukiniams žvėrims prieiti naktį. 
\par 11 Kai Dovydui pranešė, ką padarė Ajos duktė Ricpa, Sauliaus sugulovė, 
\par 12 Dovydas nuėjęs paėmė iš Jabeš Gileado gyventojų Sauliaus ir jo sūnaus Jehonatano kaulus. Jie jų kūnus buvo pavogę iš Bet Šano, kur filistinai juos buvo pakorę tą dieną, kai jie nukovė Saulių Gilbojoje. 
\par 13 Iš ten Dovydas atgabeno Sauliaus ir jo sūnaus Jehonatano kaulus; be to, jie surinko ir pakartųjų kaulus. 
\par 14 Ir jie palaidojo Sauliaus ir jo sūnaus Jehonatano kaulus Celoje, Benjamino krašte, jo tėvo Kišo kape. Jie įvykdė visa, ką karalius įsakė. Po to Dievas išklausė maldą dėl krašto. 
\par 15 Kilo karas tarp filistinų ir Izraelio. Dovydas ir jo vyrai išėjo kariauti su filistinais. Ir Dovydas pavargo. 
\par 16 Išbi Benobas, milžinų palikuonis, kurio ietis svėrė tris šimtus šekelių vario, dėvėjo naujus šarvus ir norėjo nukauti Dovydą. 
\par 17 Bet į pagalbą atėjo Cerujos sūnus Abišajas ir nukovė filistiną. Tuomet Dovydo vyrai prisiekė: “Tu nebeisi su mumis į karą, kad neužgestų Izraelio žiburys”. 
\par 18 Po to vėl kilo karas su filistinais. Prie Gobo hušatitas Sibechajas nukovė milžinų palikuonį Safą. 
\par 19 Vėl kilus karui su filistinais prie Gobo, Jaare Oregimo sūnus Elhananas iš Betliejaus nukovė gatietį, Galijoto brolį, kurio ieties kotas buvo kaip audėjo staklių riestuvas. 
\par 20 Kartą mūšyje prie Gato dalyvavo aukšto ūgio vyras, kurio rankos ir kojos turėjo po šešis pirštus, iš viso dvidešimt keturis; jis irgi buvo kilęs iš milžinų. 
\par 21 Jam keikiant Izraelį, jį nukovė Jehonatanas, Dovydo brolio Šimos sūnus. 
\par 22 Šitie keturi buvo Gato milžinų palikuonys; juos nukovė Dovydas ir jo tarnai.



\chapter{22}

\par 1 Dovydas kalbėjo Viešpačiui šios giesmės žodžius tą dieną, kai Viešpats jį išgelbėjo iš visų jo priešų ir iš Sauliaus rankų. 
\par 2 Jis sakė: “Viešpats yra mano uola, tvirtovė ir išlaisvintojas. 
\par 3 Dievas yra mano uola, Juo pasitikėsiu; Jis mano skydas ir išgelbėjimo ragas, mano apsauga ir mano aukštas bokštas, mano gelbėtojas. Tu gelbsti mane iš smurto. 
\par 4 Šauksiuosi Viešpaties, kuris vertas gyriaus, ir taip būsiu išgelbėtas nuo priešų. 
\par 5 Mirties bangos supo mane, bedievių antplūdis gąsdino mane. 
\par 6 Pragaro kančios apraizgė mane, mirties pinklės laukė manęs. 
\par 7 Sielvarte šaukiausi Viešpaties, savo Dievo. Jis išgirdo savo šventykloje mano balsą, ir mano šauksmas pasiekė Jo ausis. 
\par 8 Susvyravo, sudrebėjo žemė, ir dangaus pamatai sujudėjo ir drebėjo, nes Jis užsirūstino. 
\par 9 Iš Jo šnervių kilo dūmai, iš burnos veržėsi naikinančios liepsnos, įkaitusios žarijos skraidė. 
\par 10 Jis palenkė dangų ir nužengė, tamsa buvo po Jo kojomis. 
\par 11 Jis skrido ant cherubo ir lėkė vėjo sparnais; 
\par 12 Jį supo tamsūs skliautai, vandenys, tiršti debesys. 
\par 13 Nuo spindesio prieš Jį užsidegė žarijos. 
\par 14 Viešpats sugriaudė iš dangaus, Aukščiausiojo balsas pasigirdo. 
\par 15 Jis laidė strėles ir išsklaidė juos, žaibais sunaikino juos. 
\par 16 Iškilo jūros dugnas, atsivėrė žemės pamatai nuo Viešpaties balso ir Jo rūstybės kvapo. 
\par 17 Iš aukštybės Jis ištiesė ranką ir paėmė mane, ištraukė iš gausių vandenų. 
\par 18 Jis išgelbėjo mane iš galingo priešo, iš tų, kurie nekentė manęs, nes jie buvo stipresni už mane. 
\par 19 Jie puolė mane aną pražūtingąją dieną, bet mano atrama buvo Viešpats. 
\par 20 Jis išvedė mane į platybes ir išlaisvino mane, nes Jis pamėgo mane. 
\par 21 Viešpats atlygino man pagal mano teisumą, pagal mano rankų švarumą atmokėjo man. 
\par 22 Aš laikiausi Viešpaties kelio ir neatsitraukiau nuo savo Dievo nusikalsdamas. 
\par 23 Jo įsakymai buvo priešais mane, ir nuo Jo nuostatų neatsitraukiau. 
\par 24 Prieš Jį buvau tyras ir saugojausi, kad nenusikalsčiau. 
\par 25 Todėl atlygino Viešpats pagal mano teisumą, pagal mano švarumą Jo akyse. 
\par 26 Gailestingam Tu pasirodai gailestingas, tobulam­tobulas, 
\par 27 tyram Tu pasirodai tyras, o su sukčiumi elgiesi suktai. 
\par 28 Tu gelbsti prispaustuosius, bet išdidžiuosius pažemini. 
\par 29 Viešpatie, Tu esi mano žiburys. Viešpats šviečia man tamsumoje. 
\par 30 Su Tavimi galiu pulti priešą, su Dievu­peršokti sieną. 
\par 31 Dievo kelias tobulas, Viešpaties žodis ugnimi valytas. Jis yra skydas visiems, kurie Juo pasitiki. 
\par 32 Kas yra Dievas, jei ne Viešpats? Kas yra uola, jei ne mūsų Dievas? 
\par 33 Dievas yra stiprybė ir jėga. Jis padaro mano kelią tobulą. 
\par 34 Jis padaro mano kojas kaip stirnos ir iškelia mane į aukštumas. 
\par 35 Jis moko mano rankas kovoti, kad mano rankos sulaužytų plieninį lanką. 
\par 36 Tu man davei išgelbėjimo skydą, Tavo gerumas išaukštino mane. 
\par 37 Tu praplatinai mano žingsnius, kad mano koja nepaslystų. 
\par 38 Aš persekiojau savo priešus ir sunaikinau, nepasukau atgal, jų neišnaikinęs. 
\par 39 Aš naikinau juos ir sužeidžiau, kad jie nebegalėjo atsikelti; jie krito man po kojomis. 
\par 40 Tu apjuosei mane jėga kovai, atidavei man tuos, kurie kilo prieš mane. 
\par 41 Tu palenkei prieš mane priešus, kad galėčiau sunaikinti tuos, kurie manęs nekenčia. 
\par 42 Jie ieškojo pagalbos, tačiau jos nebuvo, kreipėsi į Viešpatį, bet Jis neatsiliepė. 
\par 43 Aš sutrypiau juos į žemės dulkes, kaip gatvių purvą sumyniau juos. 
\par 44 Tu išgelbėjai mane tautos kovose, man skyrei valdyti pagonis. Tautos, kurių nepažinau, tarnaus man. 
\par 45 Svetimšaliai pasiduos man; kai tik išgirs apie mane, jie paklus man. 
\par 46 Svetimšaliai išblykš, drebėdami išeis iš savo pilių. 
\par 47 Viešpats yra gyvas! Palaiminta tebūna mano uola, išaukštintas tebūna Dievas­mano išgelbėjimo uola. 
\par 48 Dievas atkeršija už mane ir pajungia man tautas, 
\par 49 Jis išgelbsti mane iš mano priešų. Tu iškėlei mane aukščiau tų, kurie sukyla prieš mane, ir išlaisvinai iš žiauraus žmogaus. 
\par 50 Todėl dėkosiu Tau, Viešpatie, tarp pagonių ir giedosiu gyrių Tavo vardui. 
\par 51 Jis išgelbėjimo bokštas karaliui; Jis parodo gailestingumą savo pateptajam Dovydui ir jo palikuonims per amžius”.



\chapter{23}


\par 1 Tai yra paskutiniai Dovydo žodžiai. Dovydas, Jesės sūnus, vyras aukštai iškeltas, pateptas Jokūbo Dievo, geriausias giesmininkas Izraelyje, sakė: 
\par 2 “Viešpaties Dvasia kalbėjo per mane, Jo žodžius aš tariau. 
\par 3 Izraelio Dievas pasakė, Izraelio uola man kalbėjo: ‘Kas valdo žmones teisingai, bijodamas Dievo, 
\par 4 yra lyg giedro ryto šviesa, saulei tekant, ir po lietaus dygstanti švelni žolė iš žemės’. 
\par 5 Tokie yra mano namai su Dievu. Jis su manimi sudarė amžiną sandorą, tvirtą ir nekintamą. Mano išgelbėjimas ir mano troškimai kyla iš Jo. 
\par 6 Belialo vaikai yra lyg išmėtyti erškėčiai, niekas jų neima į rankas. 
\par 7 O kas juos nori paliesti, turi apsiginkluoti. Jie bus visi sudeginti”. 
\par 8 Šitie yra Dovydo karžygių vardai. Tachemonitas, žymiausias iš visų, kuris išžudė ietimi aštuonis šimtus vienu metu. 
\par 9 Po jo buvo Dodojo sūnus Eleazaras, ahohitas, vienas iš trijų karžygių, kurie buvo su Dovydu, kai jie plūdo filistinus, susirinkusius mūšiui; Izraelio vyrai atsitraukė, 
\par 10 o jis mušė filistinus, kol jo ranka pavargo ir prilipo prie kardo. Viešpats tą dieną suteikė didelę pergalę, ir žmonės ėjo paskui jį tik plėšti užmuštųjų. 
\par 11 Trečiasis yra Agės sūnus Šama, hararitas. Kartą filistinai susirinko prie lęšių lauko. Žmonės pabėgo nuo filistinų, 
\par 12 bet Šama, atsistojęs lauko viduryje, gynė jį ir žudė filistinus. Taip Viešpats suteikė didelę pergalę. 
\par 13 Tie trys vyrai iš trisdešimties vyresniųjų atėjo pjūčiai prasidėjus pas Dovydą į Adulamo olą; filistinų kariai buvo Rafaimų slėnyje. 
\par 14 Dovydas tada buvo olos tvirtovėje, o filistinų būrys buvo Betliejuje. 
\par 15 Kartą Dovydas tarė: “Kas man atneš vandens iš Betliejaus šulinio, esančio prie vartų?” 
\par 16 Tuomet tie trys karžygiai prasilaužė pro filistinų stovyklą, pasėmė vandens iš Betliejaus šulinio ir atnešė Dovydui. Tačiau jis jo negėrė, bet išliejo Viešpačiui, 
\par 17 sakydamas: “Tebūna tai toli nuo manęs, Viešpatie, kad tai daryčiau ir gerčiau kraują vyrų, kurie statė savo gyvybes pavojun”. Jis atsisakė gerti tą vandenį. Tai padarė tie trys karžygiai. 
\par 18 Abišajas, Joabo brolis, Cerujos sūnus, buvo žymiausias tarp trijų. Jis pakėlė ietį prieš tris šimtus, nukovė juos ir pagarsėjo tarp trijų. 
\par 19 Tarp trijų jis buvo žymiausias ir tapo jų vadu, tačiau aniems trims neprilygo. 
\par 20 Jehojados sūnus Benajas, narsus vyras iš Kabceelio, įvykdė daug žygdarbių. Jis nukovė du žymius Moabo karžygius. Kartą pasnigus nuėjęs užmušė liūtą duobėje. 
\par 21 Jis nužudė egiptietį, augalotą vyrą. Egiptietis laikė rankoje ietį, o Benajas, nuėjęs prie jo su lazda, išplėšė ietį iš egiptiečio rankų ir jį nukovė jo paties ietimi. 
\par 22 Tuo Jehojados sūnus Benajas pagarsėjo tarp trijų karžygių. 
\par 23 Tarp trisdešimties jis buvo žymiausias, tačiau pirmiems trims neprilygo. Dovydas jį paskyrė savo sargybos viršininku. 
\par 24 Prie tų trisdešimties priklausė Joabo brolis Asaelis, Dodojo sūnus Elhananas iš Betliejaus, 
\par 25 harodietis Šama, harodietis Elika, 
\par 26 peletietis Helecas, tekojiečio Ikešo sūnus Ira, 
\par 27 anatotietis Abiezeras, hušietis Mebunajas, 
\par 28 ahohitas Calmonas, netofietis Mahrajas, 
\par 29 netofietis Helebas, Baanos sūnus, Ribajo sūnus Itajas iš benjaminų Gibėjos, 
\par 30 piratonietis Benajas, Hidajas iš Gaašo klonių, 
\par 31 arabietis Abi Albonas, bahurimietis Azmavetas, 
\par 32 šaalbonietis Eljachba, Jašeno sūnus Jehonatanas, 
\par 33 hararitas Šama, hararito Šararo sūnus Ahiamas, 
\par 34 Ahasbajo sūnus Elifeletas iš Maako, Ahitofelio sūnus gilojietis Eliamas, 
\par 35 karmelietis Hecrajas, arabietis Paarajas, 
\par 36 Natano sūnus Igalas iš Cobos, gadas Banis, 
\par 37 amonitas Celekas, beerotietis Nacharajas, Cerujos sūnaus Joabo ginklanešys, 
\par 38 itras Garebas, itritas Ira, 
\par 39 hetitas Ūrija; iš viso trisdešimt septyni.



\chapter{24}

\par 1 Viešpaties rūstybė vėl užsidegė prieš Izraelį. Jis paragino Dovydą suskaičiuoti Izraelio ir Judo gyventojus. 
\par 2 Karalius įsakė Joabui, kariuomenės vadui: “Suskaičiuok izraelitus nuo Dano iki Beer Šebos, kad žinočiau jų skaičių”. 
\par 3 Joabas atsakė karaliui: “Viešpats, tavo Dievas, tepadaugina žmones šimtą kartų valdovui matant. Bet kodėl, mano valdove karaliau, užsigeidei šitokio dalyko?” 
\par 4 Tačiau Joabas ir kariuomenės vadai turėjo paklusti karaliaus žodžiams, ir Joabas su kariuomenės vadais išėjo skaičiuoti Izraelio žmonių. 
\par 5 Jie, perėję Jordaną, pradėjo nuo Aroero miesto, esančio slėnyje, Gado ir Jazero kryptimi. 
\par 6 Atėję į Gileado kraštą iki Kadešo, pasuko į Sidoną. 
\par 7 Jie atėjo iki sutvirtintojo Tyro miesto, apėjo visus hivių ir kanaaniečių miestus ir pasiekė Beer Šebą Judo pietuose. 
\par 8 Apvaikščioję visą šalį, po devynių mėnesių ir dvidešimties dienų jie sugrįžo į Jeruzalę. 
\par 9 Joabas įteikė karaliui tautos skaičiavimo rezultatus. Izraelyje buvo aštuoni šimtai tūkstančių kariuomenei tinkamų vyrų, o Jude­penki šimtai tūkstančių. 
\par 10 Ir Dovydo širdis sudrebėjo, kai jis buvo suskaičiavęs žmones. Dovydas tarė Viešpačiui: “Labai nusidėjau, taip darydamas. Viešpatie, prašau, atleisk savo tarno kaltę, nes labai kvailai pasielgiau”. 
\par 11 Kai Dovydas rytą atsikėlė, Viešpaties žodis atėjo pranašui Gadui, Dovydo regėtojui: 
\par 12 “Eik ir sakyk Dovydui, kad siūlau jam tris dalykus. Tegul pasirenka vieną iš trijų”. 
\par 13 Gadas atėjo pas Dovydą, pasakė jam visa tai ir klausė: “Ar nori, kad septyni bado metai būtų tavo krašte, ar kad tris mėnesius turėtum bėgti nuo savo priešų, ar kad tris dienas maras siaustų krašte? Dabar apsigalvok ir nuspręsk, ką turiu atsakyti mane siuntusiam”. 
\par 14 Dovydas atsakė Gadui: “Patekau į didelę bėdą. Bet geriau pakliūti į Viešpaties rankas, nes Jis gailestingas, negu pakliūti man į žmonių rankas”. 
\par 15 Viešpats siuntė marą Izraeliui tą rytą ir jis tęsėsi iki nustatyto laiko. Nuo Dano iki Beer Šebos mirė septyniasdešimt tūkstančių vyrų. 
\par 16 Kai angelas ištiesė ranką į Jeruzalę, kad ją sunaikintų, Viešpačiui pagailo žmonių ir Jis tarė angelui, naikinusiam žmones: “Užteks! Nuleisk savo ranką!” Viešpaties angelas buvo prie jebusiečio Araunos klojimo. 
\par 17 Dovydas, pamatęs angelą, kuris žudė žmones, tarė Viešpačiui: “Aš nusidėjau ir piktai pasielgiau! O šios avys, ką jos padarė? Tebūna Tavo ranka prieš mane ir mano tėvo namus”. 
\par 18 Tą dieną Gadas atėjo pas Dovydą ir jam tarė: “Eik, pastatyk Viešpačiui aukurą jebusiečio Araunos klojime”. 
\par 19 Dovydas paklausė Gado ir nuėjo, kaip Viešpats įsakė. 
\par 20 Arauna, pamatęs karalių ir jo tarnus ateinančius pas jį, išėjo ir nusilenkė veidu iki žemės. 
\par 21 Arauna paklausė: “Kodėl mano valdovas karalius atėjo pas savo tarną?” Dovydas atsakė: “Atėjau pirkti tavo klojimą, kad galėčiau Viešpačiui aukurą pastatyti ir maras liautųsi tautoje”. 
\par 22 Tuomet Arauna atsakė Dovydui: “Mano valdove karaliau, imk ir aukok, ko tau reikia. Štai jaučiai deginamajai aukai, kūlimo įrankiai ir galvijų pakinktai kurui. 
\par 23 Visa tai, karaliau, aš dovanoju. Viešpats, tavo Dievas, tebūna tau malonus”. 
\par 24 Karalius atsakė: “Ne! Aš tikrai pirksiu tai iš tavęs už deramą kainą. Aš nenoriu Viešpačiui, savo Dievui, aukoti deginamųjų aukų, kurios man nieko nekainuoja”. Dovydas nupirko galvijus ir klojimą už penkiasdešimt šekelių sidabro, 
\par 25 pastatė ten aukurą Viešpačiui ir aukojo deginamąsias bei padėkos aukas. Viešpats išklausė jo maldų, ir maras liovėsi Izraelyje.




\end{document}