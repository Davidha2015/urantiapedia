\begin{document}

\title{Pirmoji Karalių knyga}

\chapter{1}

\par 1 Karalius Dovydas paseno ir sulaukė daug metų. Gerai apklotas, jis nebesušildavo. 
\par 2 Jo tarnai jam tarė: “Leisk mums, mūsų valdove karaliau, paieškoti jaunos mergaitės, kuri tau patarnautų, tave slaugytų, su tavimi miegotų ir tave sušildytų”. 
\par 3 Jie ieškojo gražios mergaitės visame Izraelio krašte; suradę šunemietę Abišagą, atvedė ją pas karalių. 
\par 4 Mergaitė buvo labai graži; ji patarnavo karaliui ir slaugė jį, bet karalius jos nepažino. 
\par 5 Hagitos sūnus Adonijas didžiavosi, sakydamas: “Aš būsiu karalius”. Jis įsigijo vežimų bei žirgų ir penkiasdešimt vyrų, kurie bėgtų pirma jo. 
\par 6 Tėvas nieko jam nedraudė ir neperspėjo taip nedaryti. Be to, jis buvo labai gražus ir gimęs po Abšalomo. 
\par 7 Cerujos sūnus Joabas ir kunigas Abjataras sekė Adoniją ir jam padėjo. 
\par 8 Bet kunigas Cadokas ir Jehojados sūnus Benajas, pranašas Natanas, Šimis, Rėjas ir Dovydo karžygiai nepalaikė Adonijo. 
\par 9 Kartą Adonijas aukojo avis, jaučius ir nupenėtus galvijus prie Cokeleto akmens, šalia En Rogelio versmės. Ta proga jis pasikvietė visus savo brolius, karaliaus sūnus, ir visus Judo vyrus, kurie tarnavo karaliui, 
\par 10 bet pranašo Natano, Benajo, karžygių ir savo brolio Saliamono jis nepakvietė. 
\par 11 Natanas kalbėjo Saliamono motinai Batšebai: “Ar girdėjai, kad, mūsų valdovui Dovydui nežinant, Hagitos sūnus Adonijas tapo karaliumi? 
\par 12 Aš noriu tau patarti, kaip išgelbėti save ir savo sūnaus Saliamono gyvybę. 
\par 13 Nuėjusi pas karalių Dovydą, paklausk jį: ‘Ar mano valdovas karalius neprisiekė man, savo tarnaitei, kad mano sūnus Saliamonas karaliaus po tavęs ir sėdės tavo soste? Kodėl Adonijas dabar tapo karaliumi?’ 
\par 14 Tau tebekalbant su karaliumi, aš ateisiu ir patvirtinsiu tavo žodžius”. 
\par 15 Batšeba atėjo į karaliaus kambarį. Karalius buvo labai pasenęs, ir šunemietė Abišaga patarnavo jam. 
\par 16 Batšeba nusilenkė ir išreiškė pagarbą karaliui. Karalius klausė: “Ko nori?” 
\par 17 Ji jam atsakė: “Mano valdove, tu prisiekei Viešpačiu, savo Dievu, savo tarnaitei, sakydamas: ‘Tavo sūnus Saliamonas karaliaus po manęs ir jis sėdės mano soste’. 
\par 18 O dabar Adonijas, mano valdovui karaliui nežinant, tapo karaliumi! 
\par 19 Jis aukojo daugybę jaučių, nupenėtų galvijų bei avių ir pasikvietė karaliaus sūnus, kunigą Abjatarą ir kariuomenės vadą Joabą, bet tavo tarno Saliamono nepakvietė. 
\par 20 Dabar, mano valdove karaliau, viso Izraelio akys yra nukreiptos į tave, ką tu paskirsi į savo sostą po savęs. 
\par 21 Kitaip, mano valdovui karaliui atsigulus prie savo tėvų, aš ir mano sūnus Saliamonas būsime laikomi nusikaltėliais”. 
\par 22 Dar jai tebekalbant su karaliumi, atėjo pranašas Natanas. 
\par 23 Buvo pranešta karaliui: “Atėjo pranašas Natanas”. Atėjęs jis nusilenkė prieš karalių veidu iki žemės 
\par 24 ir tarė: “Mano valdove karaliau, ar tu sakei, kad Adonijas karaliaus po tavęs ir sėdės tavo soste? 
\par 25 Jis šiandien aukojo daugybę jaučių, nupenėtų galvijų bei avių ir pakvietė visus karaliaus sūnus, kariuomenės vadus ir kunigą Abjatarą į vaišes. Jie valgo, geria ir šaukia: ‘Tegyvuoja karalius Adonijas!’ 
\par 26 Bet manęs, tavo tarno, kunigo Cadoko, Jehojados sūnaus Benajo ir tavo sūnaus Saliamono jis nepakvietė. 
\par 27 Jei iš tikrųjų tai padaryta mano valdovo karaliaus paliepimu, tai kodėl nepranešei savo tarnui, kas sėdės tavo soste po tavęs?” 
\par 28 Karalius Dovydas tarė: “Pašaukite Batšebą!” Jai įėjus ir atsistojus prieš karalių, 
\par 29 jis tarė: “Kaip gyvas Viešpats, kuris išvadavo mane iš visokių bėdų, 
\par 30 tavo sūnus Saliamonas karaliaus po manęs, kaip tau prisiekiau Viešpačiu, Izraelio Dievu. Jis sėdės soste vietoje manęs, ir tai aš padarysiu šiandien”. 
\par 31 Tada Batšeba nusilenkė veidu iki žemės, išreiškė karaliui pagarbą ir tarė: “Tegyvuoja mano valdovas karalius Dovydas per amžius!” 
\par 32 Karalius Dovydas sakė: “Pašaukite kunigą Cadoką, pranašą Nataną ir Jehojados sūnų Benają!” Jiems atėjus, 
\par 33 karalius sakė: “Imkite savo valdovo tarnus, užsodinkite mano sūnų Saliamoną ant mano mulo ir palydėkite jį į Gihoną. 
\par 34 Kunigas Cadokas ir pranašas Natanas ten jį tepatepa Izraelio karaliumi. Po to trimituokite ir šaukite: ‘Tegyvuoja karalius Saliamonas!’ 
\par 35 Jūs lydėkite jį, kad jis ateitų ir atsisėstų mano soste. Jis karaliaus mano vietoje, aš jį paskyriau Izraelio ir Judo valdovu”. 
\par 36 Tuomet Jehojados sūnus Benajas tarė karaliui: “Amen! Taip sako Viešpats, mano valdovo karaliaus Dievas! 
\par 37 Kaip Viešpats buvo su mano valdovu karaliumi, taip jis tebūna su Saliamonu ir tepadaro jo sostą didingesnį už mano valdovo karaliaus Dovydo sostą!” 
\par 38 Kunigas Cadokas, pranašas Natanas, Jehojados sūnus Benajas, keretai ir peletai nuėję užsodino Saliamoną ant karaliaus Dovydo mulo ir nulydėjo jį į Gihoną. 
\par 39 Ten kunigas Cadokas ėmė ragą su aliejumi iš Dievo palapinės ir patepė Saliamoną. Po to jie trimitavo ir visi žmonės šaukė: “Tegyvuoja karalius Saliamonas!” 
\par 40 Visi žmonės ėjo paskui jį pūsdami vamzdžius ir labai džiaugdamiesi, net žemė drebėjo nuo jų triukšmo. 
\par 41 Adonijas ir jo svečiai, baigdami valgyti, išgirdo tą triukšmą. Joabas, išgirdęs trimito garsą, paklausė: “Kodėl toks triukšmas mieste?” 
\par 42 Jam kalbant, atėjo kunigo Abjataro sūnus Jehonatanas. Adonijas tarė: “Užeik! Tu esi ištikimas vyras ir atnešei gerą žinią”. 
\par 43 Jehonatanas atsakė Adonijui: “Mūsų karalius Dovydas paskyrė karaliumi Saliamoną. 
\par 44 Karalius pasiuntė kunigą Cadoką, pranašą Nataną, Jehojados sūnų Benają, keretus ir peletus ir jie užsodino Saliamoną ant karaliaus mulo. 
\par 45 Kunigas Cadokas bei pranašas Natanas patepė jį karaliumi Gihone. Iš ten jie ėjo džiaugdamiesi ir šūkaudami, net visas miestas drebėjo. Tą triukšmą jūs ir girdėjote. 
\par 46 Saliamonas jau sėdi karaliaus soste. 
\par 47 Be to, karaliaus tarnai atėjo pasveikinti mūsų karalių Dovydą, sakydami: ‘Tepadaro tavo Dievas Saliamono vardą garsesnį už tavo vardą ir jo sostą didingesnį už tavo sostą!’ Karalius nusilenkė lovoje 
\par 48 ir kalbėjo: ‘Palaimintas Viešpats, Izraelio Dievas, kuris dar mano akims matant pasodino jį į mano sostą!’ ” 
\par 49 Tuomet visi Adonijo svečiai išsigandę išėjo kiekvienas savo keliu. 
\par 50 Adonijas taip išsigando Saliamono, kad nuėjo ir nusitvėrė už aukuro ragų. 
\par 51 Saliamonui buvo pranešta: “Adonijas bijo karaliaus Saliamono. Jis nusitvėrė už aukuro ragų ir sako: ‘Teprisiekia man karalius Saliamonas, kad nenužudys savo tarno’ ”. 
\par 52 Saliamonas atsakė: “Jei jis bus ištikimas, tai nė vienas jo plaukas nenukris žemėn, bet jei nusikals­mirs!” 
\par 53 Karalius Saliamonas liepė atvesti Adoniją nuo aukuro pas jį. Kai Adonijas įėjo ir nusilenkė prieš karalių Saliamoną, karalius tarė: “Eik namo!”


\chapter{2}

\par 1 Prieš mirtį Dovydas tarė savo sūnui Saliamonui: 
\par 2 “Aš einu visos žemės keliu. Būk stiprus ir tikras vyras. 
\par 3 Saugok Viešpaties, savo Dievo, nurodymus, vaikščiok Jo keliais, laikykis Jo nuostatų, įsakymų, įstatymų ir įspėjimų, surašytų Mozės įstatyme, kad tau sektųsi visur ir visuomet, 
\par 4 kad Viešpats ištesėtų savo žodį, kurį Jis man kalbėjo: ‘Jei tavo sūnūs ištikimai vaikščios mano keliais, visa širdimi ir siela pasitikės manimi, tai tavo palikuonys valdys Izraelį’. 
\par 5 Be to, tu žinai, ką padarė man Cerujos sūnus Joabas, kaip jis nužudė abu Izraelio kariuomenės vadus: Nero sūnų Abnerą ir Jetero sūnų Amasą. Jis nužudė juos, taikos metu praliedamas karo kraują, ir sutepė nekaltu krauju savo diržą ir kurpes. 
\par 6 Pasielk su juo pagal savo išmintį ir neleisk jam ramiai numirti. 
\par 7 Būk malonus gileadito Barzilajaus sūnums, leisk jiems valgyti prie savo stalo, nes jie mane pasitiko, kai bėgau nuo tavo brolio Abšalomo. 
\par 8 Gero sūnus Šimis, benjaminas iš Bahurimo, mane smarkiai keikė tą dieną, kai ėjau į Machanaimą. Bet jis atėjo manęs pasitikti prie Jordano, ir aš jam prisiekiau Viešpačiu, sakydamas: ‘Aš nežudysiu tavęs’. 
\par 9 Nepalik jo nenubaudęs. Tu esi išmintingas vyras ir žinosi, kaip su juo pasielgti, kad jo žili plaukai kruvini į kapą nueitų”. 
\par 10 Dovydas užmigo prie savo tėvų ir buvo palaidotas Dovydo mieste. 
\par 11 Dovydas valdė Izraelį keturiasdešimt metų. Septynerius metus jis karaliavo Hebrone ir trisdešimt trejus metus Jeruzalėje. 
\par 12 Saliamonas atsisėdo į savo tėvo Dovydo sostą, ir jo karalystė labai sustiprėjo. 
\par 13 Kartą Hagitos sūnus Adonijas atėjo pas Saliamono motiną Batšebą. Ji klausė: “Ar atėjai taikingai?” Jis atsakė: “Taikingai”. 
\par 14 Ir pridūrė: “Aš turiu tau kai ką pasakyti”. Ji tarė: “Sakyk”. 
\par 15 Adonijas tarė: “Tu žinai, kad aš turėjau valdyti Izraelį ir visas Izraelis to laukė. Tačiau viskas kitaip išėjo ir karalystė teko mano broliui, nes taip buvo Viešpaties skirta. 
\par 16 Neatmesk mano prašymo”. Ji tarė: “Sakyk”. 
\par 17 Jis sakė: “Prašau, pakalbėk su karaliumi Saliamonu. Jis tikrai išklausys tave ir leis man vesti šunemietę Abišagą”. 
\par 18 Batšeba atsakė: “Gerai, aš pakalbėsiu su karaliumi dėl tavęs”. 
\par 19 Batšeba nuėjo pas karalių Saliamoną pakalbėti už Adoniją. Karalius atsistojęs pasitiko ją ir nusilenkė jai. Tada atsisėdo į savo sostą, o karaliaus motinai buvo skirta vieta jo dešinėje. 
\par 20 Ji tarė: “Aš atėjau pas tave su mažu prašymu, karaliau. Neatsakyk man!” Karalius jai tarė: “Prašyk, motin! Aš neatsakysiu tau”. 
\par 21 Ji tarė: “Leisk Adonijui, savo broliui, vesti šunemietę Abišagą”. 
\par 22 Karalius Saliamonas atsakė savo motinai: “Kodėl tu prašai Adonijui tik šunemietės Abišagos? Prašyk jam ir karalystės, nes jis yra mano vyresnysis brolis ir su juo yra kunigas Abjataras bei Cerujos sūnus Joabas”. 
\par 23 Karalius Saliamonas prisiekė Viešpačiu: “Tegul Dievas padaro man tai ir dar daugiau, jei šitas jo prašymas nekainuos Adonijui gyvybės. 
\par 24 Kaip gyvas Viešpats, kuris man suteikė karalystę, pasodino mane į mano tėvo Dovydo sostą ir įkūrė man namus, kaip buvo pažadėjęs, šiandien Adonijas mirs”. 
\par 25 Karalius Saliamonas pasiuntė Jehojados sūnų Benają, kuris nužudė Adoniją. 
\par 26 Kunigui Abjatarui karalius įsakė: “Eik į Anatotą, į savo žemę. Nors esi nusipelnęs mirties, bet šiandien nebausiu tavęs mirtimi, nes tu nešiojai Viešpaties Dievo skrynią mano tėvo Dovydo laikais ir kentėjai su juo visus vargus, kuriuos jis kentėjo”. 
\par 27 Saliamonas pašalino Abjatarą iš kunigo tarnystės. Taip išsipildė Viešpaties žodis, kurį Jis kalbėjo apie Elio giminę Šilojuje. 
\par 28 Kai ta žinia pasiekė Joabą, jis nubėgo į Viešpaties palapinę ir nusitvėrė už aukuro ragų. Joabas buvo išvien su Adoniju, nors Abšalomo jis nerėmė. 
\par 29 Kai karaliui Saliamonui pranešė, kad Joabas pabėgo į Viešpaties palapinę ir stovi šalia aukuro, Saliamonas pasiuntė Jehojados sūnų Benają ir įsakė nužudyti Joabą. 
\par 30 Benajas, atėjęs prie Viešpaties palapinės, jam tarė: “Karalius įsako: ‘Išeik!’ ” Bet Joabas atsakė: “Neisiu, bet mirsiu čia”. Tada Benajas pranešė karaliui, ką Joabas atsakė. 
\par 31 Karalius jam atsakė: “Daryk, kaip jis sakė: nužudyk jį ir palaidok, kad būtų pašalintas Joabo nekaltai pralietas kraujas nuo manęs ir mano tėvo namų. 
\par 32 Viešpats jo kraują grąžins ant jo galvos, nes jis, mano tėvui nežinant, nužudė du vyrus, teisesnius ir geresnius už save: Nero sūnų Abnerą, Izraelio kariuomenės vadą, ir Jetero sūnų Amasą, Judo kariuomenės vadą. 
\par 33 Jų kraujas sugrįš ant Joabo galvos ir ant jo palikuonių per amžius, bet Dovydui, jo palikuonims, namams ir sostui tebūna nuo Viešpaties amžina ramybė” . 
\par 34 Jehojados sūnus Benajas nuėjo, nužudė jį ir palaidojo jam priklausančioje žemėje dykumoje. 
\par 35 Karalius paskyrė jo vieton kariuomenės vadu Jehojados sūnų Benają, o kunigą Cadoką­į Abjataro vietą. 
\par 36 Po to karalius pasikvietė Šimį ir jam sakė: “Pasistatyk namus Jeruzalėje, čia gyvenk ir niekur neišeik. 
\par 37 Jei kurią dieną išeisi ir pereisi Kidrono upelį, tikrai mirsi. Tavo kraujas bus ant tavo paties galvos”. 
\par 38 Šimis atsakė karaliui: “Gerai pasakyta. Kaip mano valdovas karalius įsakė, taip aš darysiu”. Šimis ilgai gyveno Jeruzalėje. 
\par 39 Po trejų metų du Šimio tarnai pabėgo pas Maakos sūnų Achišą, Gato karalių. Kai Šimiui pranešė, kad jo tarnai Gate, 
\par 40 jis, pasibalnojęs asilą, nuvyko į Gatą pas Achišą savo tarnų ieškoti; ir Šimis, juos suradęs, parsivedė iš Gato. 
\par 41 Saliamonui buvo pranešta, kad Šimis buvo nuvykęs iš Jeruzalės į Gatą ir grįžo. 
\par 42 Karalius pasikvietė Šimį ir jam tarė: “Ar aš neprisaikdinau tavęs Viešpačiu, kad tą dieną, kai tu iš Jeruzalės išeisi ir kur nors išvyksi, tikrai mirsi? Tu man sakei: ‘Gerai pasakyta’. 
\par 43 Kodėl nesilaikei Viešpaties priesaikos ir mano įsakymo, kurį tau buvau davęs? 
\par 44 Tu žinai nedorybes, kurias padarei mano tėvui Dovydui. Viešpats sugrąžins tavo nedorybes ant tavo galvos. 
\par 45 Karalius Saliamonas bus palaimintas ir Dovydo sostas bus įtvirtintas Viešpaties akivaizdoje per amžius”. 
\par 46 Karalius įsakė Jehojados sūnui Benajui, ir jis išėjęs užmušė Šimį. Karalystė buvo įtvirtinta Saliamono rankose.



\chapter{3}

\par 1 Saliamonas susigiminiavo su faraonu, Egipto karaliumi. Jis vedė jo dukterį ir parsivežė ją į Dovydo miestą. Ji čia gyveno, kol Saliamonas baigė statyti sau namus, Viešpaties namus ir sieną apie Jeruzalę. 
\par 2 Tačiau žmonės vis dar aukojo aukštumose, nes nebuvo pastatyti namai Viešpaties vardui. 
\par 3 Saliamonas mylėjo Viešpatį, laikydamasis savo tėvo Dovydo nurodymų, tačiau ir jis aukojo bei degino smilkalus aukštumose. 
\par 4 Karalius nuėjo į Gibeoną aukoti, nes tai buvo garsi aukštuma. Jis aukojo ant to aukuro tūkstantį deginamųjų aukų. 
\par 5 Gibeone Viešpats pasirodė Saliamonui naktį sapne ir tarė: “Prašyk, ką Aš tau galėčiau duoti”. 
\par 6 Saliamonas atsakė: “Tu parodei savo tarnui, mano tėvui Dovydui, didelį gailestingumą, nes jis vaikščiojo prieš Tave tiesoje, teisume ir širdies tyrume. Tu parodei jam savo gerumą, duodamas jam sūnų, kuris sėdėtų jo soste, kaip tai yra šiandien. 
\par 7 Viešpatie, mano Dieve, Tu padarei savo tarną karaliumi mano tėvo Dovydo vietoje, nors aš dar labai jaunas ir nežinau, kaip įeiti ir kaip išeiti. 
\par 8 Tavo tarnas yra vidury Tavo tautos, kurią Tu išsirinkai, didelės tautos, kurios neįmanoma suskaičiuoti nei aprėpti dėl daugumo. 
\par 9 Duok savo tarnui išmintingą širdį, kad galėčiau teisti Tavo tautą ir skirti gera nuo blogo. Kas galėtų valdyti tokią didelę Tavo tautą?” 
\par 10 Viešpačiui patiko toks Saliamono prašymas. 
\par 11 Dievas jam tarė: “Kadangi prašei išminties teisingai nuspręsti, o ne ilgo amžiaus sau, turtų sau ar savo priešų gyvybės, 
\par 12 tai padariau pagal tavo prašymą: Aš daviau tau išmintingą ir sumanią širdį, kad panašaus į tave nebuvo iki tavęs ir po tavęs nebus tokio, kaip tu. 
\par 13 Taip pat daviau tau, ir ko neprašei­turtų ir garbės; niekas tau neprilygs tarp karalių per visas tavo dienas. 
\par 14 Ir jei vaikščiosi mano keliais, laikydamasis mano nuostatų ir įsakymų, kaip tavo tėvas Dovydas, Aš prailginsiu tavo dienas”. 
\par 15 Saliamonas pabudo ir suprato, kad tai buvo sapnas. Po to jis sugrįžo į Jeruzalę ir priešais Viešpaties Sandoros skrynią aukojo deginamąsias bei padėkos aukas. Jis iškėlė puotą visiems savo tarnams. 
\par 16 Kartą dvi paleistuvės atėjo pas karalių ir stojo jo akivaizdon. 
\par 17 Viena moteris sakė: “Mano valdove! Aš ir šita moteris gyvename vienuose namuose; aš pagimdžiau sūnų, būdama su ja tuose namuose. 
\par 18 O po trijų dienų ir ji pagimdė. Mes buvome vienos ir nieko iš pašalinių nebuvo su mumis, tik mes dvi. 
\par 19 Šitos moters sūnus naktį mirė, nes ji nugulė jį. 
\par 20 Vidurnaktį atsikėlus ji paėmė mano sūnų, kai aš miegojau, nuo mano pašonės ir pasiguldė jį prie savęs, o savo mirusį sūnų paguldė prie manęs. 
\par 21 Rytą atsikėlusi norėjau maitinti savo sūnų, bet pamačiau, kad jis miręs. Gerai įsižiūrėjusi, pažinau, kad tai nebuvo mano sūnus”. 
\par 22 Antroji moteris tarė: “Netiesa, mano sūnus yra gyvas, o tavo sūnus miręs”. Bet šita atsakė: “Ne, tavo sūnus miręs, o mano sūnus gyvas”. Taip jos kalbėjo karaliaus akivaizdoje. 
\par 23 Karalius tarė: “Viena sako: ‘Mano sūnus yra gyvas, o tavo sūnus miręs’, ir antroji tvirtina: ‘Tavo sūnus yra miręs, o mano sūnus gyvas’ ”. 
\par 24 Ir karalius pasakė: “Atneškite kardą”. Ir atnešė kardą karaliui. 
\par 25 Ir karalius pasakė: “Padalykite gyvąjį kūdikį ir duokite vieną pusę vienai, o kitą pusę kitai”. 
\par 26 Tuomet moteris, kurios sūnus buvo gyvas, gailėdamasi jo, tarė karaliui: “Mano valdove! Atiduok jai gyvąjį kūdikį ir nežudyk jo!” O antroji sakė: “Gerai, nebus nei man, nei tau. Padalykite jį!” 
\par 27 Tada karalius atsakė: “Atiduokite gyvąjį kūdikį jai ir jokiu būdu jo nežudykite. Ji yra jo motina”. 
\par 28 Visas Izraelis, išgirdęs tą karaliaus sprendimą, ėmė bijoti karaliaus, nes jie matė, kad Dievo išmintis buvo jame, kad jis teistų.



\chapter{4}

\par 1 Karalius Saliamonas valdė visą Izraelį. 
\par 2 Jo kunigaikščiai buvo: kunigo Cadoko sūnus Azarijas, 
\par 3 Šišos sūnūs Elihorefas ir Ahija­ raštininkai, Ahiludo sūnus Juozapatas­metraštininkas, 
\par 4 Jehojados sūnus Benajas­kariuomenės vadas, Cadokas ir Abjataras­kunigai, 
\par 5 Natano sūnus Azarijas buvo valdininkų viršininkas, o Natano sūnus Zabudas­vyriausias valdytojas ir karaliaus draugas, 
\par 6 Ahišaras­karaliaus namų viršininkas, o Abdos sūnus Adoniramas­duoklės prižiūrėtojas. 
\par 7 Saliamonas turėjo dvylika valdininkų visame Izraelyje, kurie aprūpindavo maistu karalių ir jo namiškius; kiekvienas aprūpindavo vieną mėnesį per metus. 
\par 8 Tai jų vardai: Ben-Huras Efraimo kalnyne; 
\par 9 Ben-Dekeris Makace, Šaalbime, Bet Šemeše, Elon Bet Hanane; 
\par 10 Ben-Hesedas Arubote, jo žinioje buvo Sokas ir visas Hefero kraštas; 
\par 11 Ben-Abinadabo žinioje buvo visa Doro sritis; jis buvo vedęs Saliamono dukterį Tafatą; 
\par 12 Ahiludo sūnaus Baanos žinioje buvo Tanaachas, Megidas, visas Bet Šeanas, esąs šalia Caretano ir žemiau Jezreelio, nuo Bet Šeano iki Abel Meholos ir iki anapus Jokneamo; 
\par 13 Ben-Geberas Ramot Gileade, jo žinioje buvo Manaso sūnaus Jayro miestai, esą Gileade, Argobo sritis, esanti Bašane­šešiasdešimt didelių miestų su sienomis ir variniais užkaiščiais; 
\par 14 Idojo sūnus Ahinadabas Machanaime; 
\par 15 Ahimaacas Naftalyje; jis buvo vedęs Saliamono dukterį Basmatą; 
\par 16 Hušajo sūnus Baana Ašere ir Bealote; 
\par 17 Paruacho sūnus Juozapatas Isachare; 
\par 18 Elos sūnus Šimis Benjamine; 
\par 19 Ūrio sūnus Geberas Gileade, amoritų karaliaus Sihono ir Basano karaliaus Ogo šalyje. 
\par 20 Judo ir Izraelio žmonių buvo kaip smilčių pajūryje; jie valgė, gėrė ir linksminosi. 
\par 21 Saliamonas valdė visas karalystes nuo upės iki filistinų šalies ir Egipto sienos; jos mokėdavo duoklę ir tarnavo Saliamonui per visas jo dienas. 
\par 22 Saliamonui kasdien pristatydavo: trisdešimt homerų smulkių ir šešiasdešimt homerų paprastų miltų, 
\par 23 dešimt nupenėtų jaučių, dvidešimt jaučių iš ganyklos ir šimtą avių, neskaičiuojant elnių, gazelių, stirnų ir nupenėtų paukščių. 
\par 24 Saliamonas valdė visą sritį į vakarus nuo upės, nuo Tifsacho iki Gazos. Jis gyveno taikiai su visais kaimynais. 
\par 25 Judas ir Izraelis, nuo Dano iki Beer Šebos, gyveno saugiai, kiekvienas žmogus po savo vynmedžiu ir figmedžiu per visas Saliamono dienas. 
\par 26 Saliamonas turėjo keturiasdešimt tūkstančių arklidžių kovos vežimų žirgams ir dvylika tūkstančių raitelių. 
\par 27 Dvylika valdininkų aprūpindavo maistu karalių Saliamoną ir visus, kurie valgė prie karaliaus Saliamono stalo; kiekvienas rūpinosi, kad maisto netrūktų vienam mėnesiui. 
\par 28 Miežių ir pašaro žirgams ir mulams jie atgabendavo į tą vietą, kur būdavo karalius, kaip jiems būdavo įsakyta. 
\par 29 Dievas suteikė Saliamonui išmintį ir nepaprastą sumanumą bei tokį širdies platumą, kaip smėlis jūros pakrantėje. 
\par 30 Saliamono išmintis pranoko visų rytiečių ir egiptiečių išmintį. 
\par 31 Jis buvo išmintingesnis už visus žmones, net už ezrahitą Etaną ir Hemaną, Kalkolą bei Dardą, Maholo sūnus. Jis buvo garsus visose aplinkinėse tautose. 
\par 32 Jis sukūrė tris tūkstančius patarlių ir tūkstantį penkias giesmes. 
\par 33 Jis kalbėjo apie medžius, nuo Libano kedrų iki yzopų, kurie auga ant sienų, apie žvėris, paukščius, roplius ir žuvis. 
\par 34 Iš visų tautų ateidavo pasiklausyti Saliamono išminties, nuo visų žemės karalių, kurie girdėjo apie jo išmintį.



\chapter{5}

\par 1 Tyro karalius Hiramas, išgirdęs, kad Saliamonas pateptas karaliumi savo tėvo vietoje, siuntė pas jį pasiuntinius, nes Hiramas draugavo su Dovydu. 
\par 2 Ir Saliamonas siuntė pas Hiramą, sakydamas: 
\par 3 “Tu žinai, kad mano tėvas Dovydas negalėjo pastatyti namų Viešpaties, savo Dievo, vardui dėl karų su tautomis, kurios buvo prieš jį iš visų pusių, kol Viešpats padėjo jas po jo kojų padais. 
\par 4 Dabar Viešpats, mano Dievas, man suteikė ramybę; nėra nei priešų, nei trukdymų. 
\par 5 Aš galvoju statyti namus Viešpaties, savo Dievo, vardui, kaip Viešpats kalbėjo mano tėvui Dovydui: ‘Tavo sūnus, kurį Aš pasodinsiu į sostą tavo vietoje, pastatys namus mano vardui’. 
\par 6 Taigi įsakyk kirsti kedrus Libane. Mano tarnai tegu dirba su tavo tarnais. Už darbą tavo tarnams mokėsiu, kiek nustatysi. Tu žinai, kad tarp mūsų nėra nė vieno tokio medžių kirtėjo, kaip sidoniečiai”. 
\par 7 Hiramas, išgirdęs Saliamono žodžius, labai apsidžiaugė ir tarė: “Palaimintas Viešpats, kuris davė Dovydui išmintingą sūnų, kad valdytų tą didelę tautą”. 
\par 8 Hiramas pranešė Saliamonui: “Gavau žinią, kurią man siuntei. Aš viską padarysiu pagal tavo norą dėl medžių kirtimo. 
\par 9 Mano tarnai juos nugabens nuo Libano kalnų į jūrą; jie sieliais bus nuplukdyti iki vietos, kurią man nurodysi; ten juos sukraus ir tu juos atsiimsi. Tu turėsi patenkinti mano norą ir tiekti maisto mano namams”. 
\par 10 Hiramas siuntė Saliamonui kedro ir kipariso medžių, kiek tik jis norėjo. 
\par 11 O Saliamonas davė Hiramo namams dvidešimt tūkstančių homerų kviečių ir dvidešimt homerų tyriausio aliejaus. Tokį kiekį Saliamonas duodavo Hiramui kiekvienais metais. 
\par 12 Viešpats suteikė Saliamonui išmintį, kaip Jis jam buvo pažadėjęs. Hiramas ir Saliamonas sudarė taikos sutartį. 
\par 13 Karalius Saliamonas parinko iš viso Izraelio trisdešimt tūkstančių vyrų darbams. 
\par 14 Jis juos siųsdavo pamainomis, kas mėnesį po dešimt tūkstančių: vieną mėnesį jie būdavo Libane, o du mėnesius namie. Adoniramas buvo darbininkų viršininkas. 
\par 15 Saliamonas turėjo septyniasdešimt tūkstančių nešikų ir aštuoniasdešimt tūkstančių akmenskaldžių kalnuose, 
\par 16 neskaičiuojant trijų tūkstančių trijų šimtų Saliamono vyresniųjų valdininkų, kurie vadovavo darbams, ir prižiūrėjo žmones, atliekančius darbą. 
\par 17 Karalius įsakė paruošti didelius ir brangius akmenis, nutašytus namų pamatams. 
\par 18 Saliamono darbininkai ir Hiramo darbininkai iš Gebali tašė juos. Taip jie paruošė rąstus ir akmenis namų statybai.



\chapter{6}


\par 1 Keturi šimtai aštuoniasdešimtaisiais metais po izraelitų išėjimo iš Egipto, ketvirtaisiais Saliamono valdymo Izraelyje metais, Zivo mėnesį, kuris yra antras mėnuo, jis pradėjo statyti Viešpaties namus. 
\par 2 Namai, kuriuos karalius Saliamonas statė Viešpačiui, buvo šešiasdešimties uolekčių ilgio, dvidešimties uolekčių pločio ir trisdešimties uolekčių aukščio. 
\par 3 Prieangis priešais šventyklą buvo dvidešimties uolekčių ilgio, lygus namų pločiui, ir dešimties uolekčių pločio priešais namus. 
\par 4 Namams jis padarė siaurus langus. 
\par 5 Pastatas turėjo šonuose ir užpakalyje priestatus, kur buvo įrengti šoniniai kambariai. 
\par 6 Apatinio aukšto šoniniai kambariai buvo penkių uolekčių pločio, vidurinio­šešių uolekčių ir trečiojo­septynių uolekčių pločio. Pastato išorėje buvo padaryti išsikišimai, kad nereikėtų rąstų įleisti į sienas. 
\par 7 Namai buvo statomi iš paruoštų akmenų; statybos darbams vykstant, nesigirdėjo nei kūjų, nei kirvių, nei kitų statybos įrankių garso. 
\par 8 Įėjimas į vidurinį aukštą buvo dešiniajame pastato šone: suktiniai laiptai vedė į vidurinį, o iš vidurinio į trečiąjį aukštą. 
\par 9 Taip jis pastatė namus ir užbaigė juos. Ir padengė namus kedro rąstais ir lentomis. 
\par 10 Šventyklos priestato kambariai buvo penkių uolekčių aukščio, sujungti su pagrindiniu pastatu kedro rąstais. 
\par 11 Viešpaties žodis buvo Saliamonui: 
\par 12 “Dėl namų, kuriuos man statai: jei vaikščiosi pagal mano nuostatus, vykdysi mano sprendimus ir laikysies mano įsakymų, kad vaikščiotum pagal juos, tai Aš išpildysiu tau savo žodį, kurį kalbėjau tavo tėvui Dovydui: 
\par 13 ‘Aš gyvensiu tarp izraelitų ir neapleisiu savo tautos­Izraelio’ ”. 
\par 14 Saliamonas pastatė namus ir užbaigė juos. 
\par 15 Namų vidaus sienos buvo iškaltos kedro lentomis nuo apačios iki lubų, o grindys išklotos kiparisų lentomis. 
\par 16 Pastato gale kedro medžio lentomis nuo grindų iki lubų buvo atitverta dvidešimties uolekčių ilgio ir tokio pat pločio patalpa­ Šventų švenčiausioji. 
\par 17 Namų, tai yra šventyklos, ilgis iki jos buvo keturiasdešimt uolekčių. 
\par 18 Visas namų vidus buvo iškaltas kedro medžio lentomis su išraižytais pumpurais ir išsiskleidusiomis gėlėmis taip, kad akmens visai nesimatė. 
\par 19 Namų viduje paruošė Šventų švenčiausiąją Viešpaties Sandoros skryniai. 
\par 20 Švenčiausioji buvo dvidešimties uolekčių ilgio, pločio ir aukščio; jos vidus buvo padengtas grynu auksu. Taip pat padengtas buvo ir aukuras iš kedro medžio. 
\par 21 Taip Saliamonas visą namų vidų padengė grynu auksu; Šventų švenčiausiosios, kuri buvo padengta auksu, priekyje kabojo auksinės grandinės. 
\par 22 Visas namų vidus, taip pat ir aukuras prieš Šventų švenčiausiąją buvo padengti auksu. 
\par 23 Šventų švenčiausiojoje jis padarė du cherubus iš alyvmedžio, kiekvieną dešimties uolekčių aukščio. 
\par 24 Vienas cherubo sparnas buvo penkių uolekčių ir kitas sparnas penkių uolekčių; dešimt uolekčių nuo vieno sparno galo iki kito. 
\par 25 Taip pat dešimties uolekčių buvo ir antrasis cherubas; abu cherubai buvo tokio pat dydžio ir taip pat atrodė. 
\par 26 Vienas cherubas buvo dešimties uolekčių aukščio, taip pat ir kitas. 
\par 27 Cherubai stovėjo vidinėje patalpoje. Jų sparnai buvo taip ištiesti, kad vieno sparnas siekė vieną sieną, o kito­kitą sieną; antrieji jų sparnai siekė vienas kitą Šventų švenčiausiosios viduryje. 
\par 28 Ir jis aptraukė cherubus auksu. 
\par 29 Visos namų sienos buvo išraižytos cherubų atvaizdais, palmėmis ir gėlėmis tiek viduje, tiek išorėje. 
\par 30 Namų grindis padengė auksu išorėje ir viduje. 
\par 31 Durys į Šventų švenčiausiąją buvo iš alyvmedžio, staktos buvo penkiakampės. 
\par 32 Abejos durys buvo išpuoštos cherubų, palmių bei gėlių raižiniais ir padengtos auksu. 
\par 33 Šventyklos įėjimo staktos buvo iš alyvmedžio, keturkampės. 
\par 34 Dvejos durys buvo iš kipariso medžio. Vienos pusės durys buvo iš dviejų dalių ir kitos pusės durys buvo iš dviejų dalių. 
\par 35 Ant durų išraižė cherubų, palmių bei gėlių atvaizdus ir padengė jas auksu. 
\par 36 Jis pastatė vidinį kiemą iš trijų tašyto akmens eilių ir vienos eilės kedro rąstų. 
\par 37 Ketvirtaisiais metais Zivo mėnesį buvo padėti Viešpaties namų pamatai, 
\par 38 o vienuoliktaisiais metais Bulo mėnesį, kuris yra aštuntas mėnuo, namai buvo baigti su visais priestatais pagal visus jų brėžinius. Saliamonas juos statė septynerius metus.



\chapter{7}

\par 1 O savo namus Saliamonas statė trylika metų, kol juos pabaigė. 
\par 2 Jis statė namus iš Libano kedrų. Namai buvo šimto uolekčių ilgio, penkiasdešimties pločio ir trisdešimties aukščio; jie stovėjo ant keturių eilių kedro medžio stulpų, ant kurių buvo kedro rąstai. 
\par 3 Rąstai, kurie buvo ant keturiasdešimt penkių stulpų, po penkiolika stulpų vienoje eilėje, buvo apkalti kedro medžio lentomis. 
\par 4 Langai buvo trimis eilėmis, langas priešais langą trijuose aukštuose. 
\par 5 Visos durys ir jų staktos buvo keturkampės kaip ir langai; langas priešais langą trijuose aukštuose. 
\par 6 Jis pastatė ir prieangį iš stulpų, penkiasdešimties uolekčių ilgio ir trisdešimties uolekčių pločio, ir kitą prieangį priešais jį su stulpais ir storais skersiniais rąstais. 
\par 7 Tada jis padarė prieangį, kuriame stovėtų sostas ir jis galėtų teisti­teismo prieangį, ir iškalė jį kedro lentomis nuo grindų iki lubų. 
\par 8 Karaliaus gyvenamieji namai turėjo kitą kiemą su prieangiu, panašiai pastatytu. Saliamonas panašius namus pastatė ir savo žmonai, faraono dukteriai. 
\par 9 Visi šie pastatai buvo pastatyti iš brangių akmenų nuo pamato iki stogo. Akmenys statybai buvo keturkampiai, nutašyti ir nulyginti iš vidaus ir lauko pusės. 
\par 10 Pamatai buvo iš brangių, didelių, dešimties ir aštuonių uolekčių akmenų. 
\par 11 Virš jų buvo brangūs tašyti akmenys ir kedrai. 
\par 12 Didysis kiemas, Viešpaties namų vidinis kiemas ir namų prieangis buvo aptverti aplinkui trimis eilėmis tašytų akmenų ir viena eile kedro rąstų. 
\par 13 Karalius Saliamonas pasikvietė iš Tyro Hiramą. 
\par 14 Jis buvo našlės iš Naftalio giminės sūnus, o jo tėvas buvo iš Tyro. Jis buvo variakalys­sumanus, gabus ir išmintingas, galintis atlikti įvairius darbus iš vario. Atvykęs pas karalių Saliamoną, jis dirbo įvairius darbus. 
\par 15 Hiramas nuliejo dvi varines kolonas, kurių kiekviena buvo aštuoniolikos uolekčių aukščio ir dvylikos uolekčių apimties. 
\par 16 Jis padarė du varinius kapitelius kolonų viršui. Kiekvienas kapitelis buvo penkių uolekčių aukščio. 
\par 17 Be to, jis padarė groteles ir pynes iš grandinėlių kapiteliams, kurie buvo ant kolonų­septynias vienam kapiteliui ir septynias kitam. 
\par 18 Jis padarė kolonas ir po dvi eiles granato vaisių ant visų kapitelius dengiančių grotelių. 
\par 19 Kapiteliai, kurie buvo ant kolonų, atrodė kaip lelijos, keturių uolekčių. 
\par 20 Ant dviejų kolonų kapitelių buvo granato vaisiai, virš išlinkimo prie grotelių, ir aplinkui visą kapitelį buvo du šimtai granato vaisių. 
\par 21 Hiramas pastatė kolonas prie šventyklos įėjimo; dešinę koloną pavadino Jachinu, o kairę Boazu. 
\par 22 Lelijų pavidalo pagražinimai puošė kolonų viršų. Taip jis užbaigė kolonas. 
\par 23 Jis taip pat nuliejo baseiną dešimties uolekčių skersmens, apskritą, penkių uolekčių aukščio, o jo apimtis buvo trisdešimt uolekčių. 
\par 24 Dvi eilės pumpurų, nulietų išvien su baseinu, buvo po briauna aplinkui jį, po dešimt pumpurų uolektyje. 
\par 25 Jis stovėjo ant dvylikos jaučių. Trys buvo atsigręžę į šiaurę, trys į vakarus, trys į pietus ir trys į rytus. Baseinas buvo jų viršuje; ir jų užpakalinės dalys buvo po baseinu. 
\par 26 Jis buvo plaštakos storio, jo briauna buvo panaši į taurės briauną, į lelijos žiedą; jame tilpo du tūkstančiai batų vandens. 
\par 27 Hiramas dar padarė dešimt varinių stovų. Kiekvienas stovas buvo keturių uolekčių ilgio, keturių pločio ir trijų aukščio. 
\par 28 Stovas buvo toks: jis turėjo keturias šonines plokštes, kurios buvo įtvirtintos rėmuose. 
\par 29 Tose plokštėse buvo išraižyti liūtų, jaučių ir cherubų atvaizdai, o taip pat ir ant rėmų; po jaučiais bei liūtais buvo kabantys vainikėliai. 
\par 30 Be to, kiekvienas stovas turėjo keturis varinius ratus su ašimis. Visuose keturiuose kampuose buvo atramos, ant kurių stovėjo praustuvė. Atramų šonus puošė išlieti vainikai. 
\par 31 Stovo skylė buvo apvali, vienos uolekties skersmens iš vidaus ir pusantros uolekties skersmens išorėje. Jo šonai buvo išraižyti ir sudarė keturkampį, o ne apskritimą. 
\par 32 Keturi ratai buvo stovo šonuose, jų ašys pritvirtintos prie stovo. Kiekvieno rato aukštis buvo pusantros uolekties. 
\par 33 Ratai buvo padaryti panašiai kaip vežimų ratai; jų ratlankiai, stipinai, stebulės ir ašys buvo nulietos. 
\par 34 Keturios atramos prie kiekvieno stovo kampo buvo nulietos išvien su stovu. 
\par 35 Stovo viršuje buvo pusės uolekties aukščio apskritas lankas; rėmai ir šoninės plokštės buvo išlietos išvien. 
\par 36 Hiramas išraižė ant rėmų ir šoninių plokščių­visur, kur buvo galima, cherubų, liūtų ir palmių atvaizdus bei pridėjo aplinkui vainikėlių. 
\par 37 Taip jis padarė dešimt stovų, kurie buvo nulieti vienodai, to paties dydžio ir taip pat atrodantys. 
\par 38 Jis taip pat padarė dešimt varinių praustuvių. Kiekvienoje tilpo keturiasdešimt batų vandens. Kiekvienos praustuvės skersmuo buvo keturios uolektys. Ant kiekvieno stovo buvo po praustuvę. 
\par 39 Penki stovai buvo pastatyti viename ir penki­kitame šventyklos šone; baseinas stovėjo šventyklos dešinėje, pietryčiuose. 
\par 40 Hiramas dar padarė puodus, semtuvėlius bei dubenis. Taip Hiramas baigė visą darbą, kurį jis darė karaliui Saliamonui dėl Viešpaties šventyklos: 
\par 41 dvi kolonas, du kapitelius, kurie buvo ant kolonų, groteles apdengti kapiteliams, kurie buvo ant kolonų, 
\par 42 keturis šimtus granato vaisių abiems grotelėms­po dvi eiles granato vaisių abiems kapiteliams, kurie buvo ant kolonų, 
\par 43 dešimt stovų ir dešimt praustuvių ant stovų, 
\par 44 baseiną ir dvylika jaučių po baseinu, 
\par 45 puodus, semtuvėlius bei dubenis. Visus šituos reikmenis Hiramas padarė Saliamonui dėl Viešpaties šventyklos iš skaistaus vario. 
\par 46 Karalius juos nuliejo Jordano lygumos molingoje žemėje tarp Sukoto ir Cartano. 
\par 47 Visi šitie daiktai liko nepasverti, nes buvo sunaudota tiek daug vario, kad buvo neįmanoma jo pasverti. 
\par 48 Saliamonas padarė Viešpaties šventyklai visus reikmenis: auksinį aukurą, auksinį stalą padėtinei duonai laikyti, 
\par 49 penkias žvakides iš gryno aukso dešinėje ir penkias kairėje priešais Šventų švenčiausiąją, su gėlėmis, lempomis ir žnyplėmis iš aukso, 
\par 50 šakutes, samčius, dubenis, lėkštes, smilkytuvus iš gryno aukso, taip pat auksinius vyrius vidaus durims į Šventų švenčiausiąją ir durims į šventyklą. 
\par 51 Taip buvo padaryti visi darbai, kuriuos karalius Saliamonas padarė Viešpaties namams. Po to Saliamonas atgabeno į šventyklą tai, ką jo tėvas Dovydas buvo pašventęs: sidabrą, auksą bei indus, ir sudėjo šventyklos sandėliuose.



\chapter{8}

\par 1 Saliamonas sušaukė į Jeruzalę Izraelio vyresniuosius, giminių vadus ir izraelitų šeimų galvas Viešpaties Sandoros skryniai perkelti iš Dovydo miesto Siono. 
\par 2 Visi Izraelio vyrai susirinko pas karalių Saliamoną Etanimo mėnesį, kuris yra septintasis mėnuo, šventės metu. 
\par 3 Atėjo visi Izraelio vyresnieji, ir kunigai paėmė skrynią. 
\par 4 Ir jie nešė Viešpaties skrynią, Susitikimo palapinę ir visus šventus indus, kurie buvo palapinėje; kunigai ir levitai nešė juos. 
\par 5 Karalius Saliamonas ir visi izraelitai, susirinkę pas jį, ėjo priešais skrynią ir aukojo tiek avių ir galvijų, kad jų neįmanoma buvo suskaityti. 
\par 6 Kunigai įnešė Viešpaties Sandoros skrynią į Šventų švenčiausiąją po cherubų sparnais. 
\par 7 Cherubų sparnai buvo išskleisti virš skrynios ir cherubai dengė skrynią bei jos kartis. 
\par 8 Kartys buvo tokios ilgos, kad jų galai buvo matomi šventykloje, tačiau iš lauko jie nebuvo matomi. Jos ten pasiliko iki šios dienos. 
\par 9 Skrynioje buvo tik dvi akmeninės plokštės, kurias Mozė įdėjo Horebe, kai Viešpats padarė sandorą su izraelitais, jiems išėjus iš Egipto krašto. 
\par 10 Kunigams išėjus iš šventyklos, debesis pripildė Viešpaties namus 
\par 11 taip, kad kunigai negalėjo tarnauti dėl debesies, nes Viešpaties šlovė pripildė Viešpaties namus. 
\par 12 Tuomet Saliamonas tarė: “Viešpats kalbėjo, kad nori gyventi tirštoje tamsoje. 
\par 13 Aš pastačiau Tau namus, kuriuose gyventum, vietą, kurioje pasiliktum per amžius”. 
\par 14 Karalius atsisukęs palaimino visus susirinkusius izraelitus, o visi izraelitai tuo tarpu stovėjo. 
\par 15 Jis sakė: “Palaimintas Viešpats, Izraelio Dievas, kuris įvykdė, ką pažadėjo mano tėvui Dovydui, sakydamas: 
\par 16 ‘Nuo tos dienos, kai išvedžiau savo tautą Izraelį iš Egipto, Aš nepasirinkau jokio miesto iš visų Izraelio giminių statyti namams, kur būtų mano vardas, bet išsirinkau Dovydą, kad jis valdytų mano tautą­Izraelį’. 
\par 17 Mano tėvas Dovydas norėjo pastatyti namus Viešpaties, Izraelio Dievo, vardui. 
\par 18 Viešpats kalbėjo mano tėvui Dovydui: ‘Gerai, kad tu norėjai pastatyti namus mano vardui, 
\par 19 tačiau ne tu juos pastatysi, bet tavo sūnus, kuris tau gims, pastatys namus mano vardui’. 
\par 20 Viešpats išpildė savo žodį, kurį kalbėjo. Aš užėmiau savo tėvo Dovydo vietą ir Izraelio sostą, kaip Viešpats buvo pažadėjęs, ir pastačiau namus Viešpaties, Izraelio Dievo, vardui. 
\par 21 Ten paruošiau vietą skryniai, kurioje yra Viešpaties Sandora, padaryta su mūsų tėvais, kai Jis išvedė juos iš Egipto krašto”. 
\par 22 Saliamonas atsistojo priešais Viešpaties aukurą viso Izraelio akivaizdoje, iškėlė rankas į dangų 
\par 23 ir sakė: “Viešpatie, Izraelio Dieve, nei danguje, nei žemėje nėra dievo, kuris būtų lygus Tau, kuris būtų gailestingas ir laikytųsi sandoros su savo tarnais, kurie Tavimi visa širdimi pasitiki. 
\par 24 Tu ištesėjai savo tarnui Dovydui, mano tėvui, duotą pažadą. Tu kalbėjai savo lūpomis ir įvykdei tai savo rankomis šiandien. 
\par 25 Dabar, Viešpatie, Izraelio Dieve, įvykdyk tai, ką pažadėjai savo tarnui Dovydui, mano tėvui, sakydamas: ‘Izraelio soste nepritrūksi vyro mano akivaizdoje, jei tik tavo sūnūs saugos savo kelius ir vaikščios priešais mane, kaip tu vaikščiojai’. 
\par 26 Izraelio Dieve, meldžiu: įvykdyk pažadą, kurį davei savo tarnui Dovydui, mano tėvui. 
\par 27 Bet argi Dievas iš tiesų gyvens žemėje? Štai dangus ir dangų dangūs nepajėgia Tavęs sutalpinti, juo labiau šitie namai, kuriuos pastačiau. 
\par 28 Atsižvelk į savo tarno maldą ir prašymą, Viešpatie, mano Dieve, išklausyk šauksmą ir maldą, kuria Tavo tarnas meldžiasi Tavo akivaizdoje šiandien. 
\par 29 Tebūna Tavo akys atvertos link šių namų dieną ir naktį, link vietos, kurią pasirinkai, kad Tavo vardas ten būtų. Išklausyk savo tarno maldą, kai jis melsis šioje vietoje. 
\par 30 Išgirsk savo tarno ir savo tautos maldavimą, kai jie melsis šioje vietoje. Išgirsk danguje, kur Tu gyveni, ir atleisk. 
\par 31 Jei kas nusikals prieš savo artimą ir ateis į šituos namus prie Tavo aukuro prisiekti, 
\par 32 išgirsk danguje ir teisk savo tarnus: pasmerk kaltąjį pagal jo nusikaltimą ir išteisink teisųjį, atlygindamas jam pagal jo teisumą. 
\par 33 Jei Tavo tauta Izraelis bus nugalėta priešo dėl to, kad Tau nusidėjo, ir jei ji atsigręš į Tave, išpažins Tavo vardą, melsis ir maldaus Tavęs šituose namuose, 
\par 34 tai išgirsk danguje, atleisk savo tautai Izraeliui nuodėmę ir sugrąžink ją į žemę, kurią davei jų tėvams. 
\par 35 Jei dangus bus uždarytas ir nebus lietaus dėl to, kad jie nusidėjo, ir jei jie melsis šitoje vietoje, išpažins Tavo vardą ir nusigręš nuo savo nuodėmės, už kurią juos baudi, 
\par 36 išgirsk danguje ir atleisk savo tarnų, Izraelio tautos, nuodėmę, parodyk jiems gerą kelią, kuriuo jie turi eiti, ir duok lietaus kraštui, kurį jiems davei paveldėti. 
\par 37 Jei badas ar maras siaus krašte, jei pūs karštas pietų vėjas, jei pelėsiai ir skėriai naikins derlių, jei žmones miestuose apsups priešas, užeis vargai ar ligos 
\par 38 ir visa Izraelio tauta ar atskiras žmogus melsis, ištiesę rankas šitų namų link, 
\par 39 išklausyk danguje, atleisk jiems ir atlygink kiekvienam pagal jo kelius, kaip Tu matai jo širdyje, nes Tu vienas pažįsti kiekvieno žmogaus širdį, 
\par 40 kad jie Tavęs bijotų, kol gyvens žemėje, kurią davei mūsų tėvams. 
\par 41 Jei svetimšalis, ne Tavo tautos Izraelio žmogus, ateitų iš tolimo krašto dėl Tavo vardo 
\par 42 (nes jie išgirs apie Tavo didingą vardą, Tavo stiprią ranką ir ištiestą ranką), kai jis ateis ir melsis prie šitų namų, 
\par 43 išklausyk jį danguje ir įvykdyk, ko jis Tavęs prašys, kad visos žemės tautos pažintų Tavo vardą, bijotų Tavęs kaip Tavo tauta Izraelis ir patirtų, jog šitie namai, kuriuos pastačiau, vadinami Tavo vardu. 
\par 44 Jei Tavo tauta išeitų kariauti su priešais, nepaisant kur Tu juos pasiųsi, ir melstųsi atsigręžę į šį miestą ir šiuos namus, kuriuos pastačiau Tavo vardui, 
\par 45 išgirsk danguje jų maldą ir prašymą ir apgink jų teises. 
\par 46 Jei izraelitai Tau nusidės,­juk nėra žmogaus, kuris nenusidėtų,­ir Tu užsirūstinęs juos atiduosi priešui, kuris paims juos nelaisvėn ir išves į tolimą priešo šalį, 
\par 47 ir jei jie ten būdami susipras, gailėsis, atsivers ir Tavęs maldaus, sakydami: ‘Mes nusidėjome, elgėmės neteisingai ir padarėme piktadarystę’, 
\par 48 ir gręšis į Tave visa širdimi bei visa siela priešų šalyje, į kurią jie buvo išvesti, ir melsis Tau, atsigręžę į šalį, kurią davei jų tėvams, į miestą, kurį išsirinkai, ir į namus, kuriuos pastačiau Tavo vardui, 
\par 49 išklausyk danguje jų maldas ir maldavimus ir apgink jų teises, 
\par 50 ir atleisk savo tautai, kuri Tau nusidėjo, visus jos nusikaltimus. Sukelk gailestį priešams, kurie juos laiko nelaisvėje, kad tie jų pasigailėtų, 
\par 51 nes jie yra Tavo tauta ir Tavo nuosavybė, kurią Tu išvedei iš Egipto­iš geležinės krosnies vidurio. 
\par 52 Viešpatie, pažvelk į savo tarno ir Izraelio tautos maldavimus, išklausyk juos, kai jie šaukiasi Tavęs. 
\par 53 Viešpatie Dieve, Tu juos išskyrei iš visų žemės tautų ir pasirinkai sau, kaip kalbėjai per savo tarną Mozę, išvesdamas mūsų tėvus iš Egipto”. 
\par 54 Kai Saliamonas baigė šitą maldą, jis atsikėlė nuo Viešpaties aukuro, kur jis klūpojo iškeltomis rankomis, 
\par 55 ir stovėdamas laimino visus susirinkusius, garsiu balsu sakydamas: 
\par 56 “Palaimintas Viešpats, kuris davė ramybę savo tautai Izraeliui, kaip buvo pažadėjęs; neliko neišpildytas nė vienas žodis iš viso gerojo pažado, kurį Jis davė per savo tarną Mozę. 
\par 57 Viešpats, mūsų Dievas, tebūna su mumis, kaip Jis buvo su mūsų tėvais, tegul nepalieka ir neapleidžia mūsų, 
\par 58 bet tepalenkia mūsų širdis prie savęs, kad vaikščiotume Jo keliais, vykdytume Jo įsakymus, nuostatus ir potvarkius, kuriuos Jis davė mūsų tėvams. 
\par 59 Šitie mano maldos žodžiai, tarti Viešpaties akivaizdoje, tebūna šalia Viešpaties, mūsų Dievo, dieną ir naktį, kad Jis apgintų savo tarno ir savo tautos Izraelio teises bet kuriuo metu, kai to reikia, 
\par 60 kad visos žemės tautos žinotų, jog Viešpats yra Dievas ir nėra kito. 
\par 61 Tebūna jūsų širdys tobulos prieš Viešpatį, mūsų Dievą, kad jūs gyventumėte pagal Jo nuostatus ir vykdytumėte Jo įsakymus”. 
\par 62 Po to karalius ir visas Izraelis aukojo aukas Viešpačiui. 
\par 63 Saliamonas aukojo dvidešimt du tūkstančius galvijų ir šimtą dvidešimt tūkstančių avių kaip padėkos auką Viešpačiui. Taip karalius ir visi izraelitai pašventino Viešpaties namus. 
\par 64 Tą pačią dieną karalius pašventino vidurinį kiemą prieš Viešpaties namus, nes ten jis aukojo deginamąsias aukas, duonos aukas ir padėkos aukų taukus, kadangi varinis Viešpaties aukuras buvo per mažas sutalpinti deginamąsias aukas, duonos aukas ir padėkos aukų taukus. 
\par 65 Saliamonas su visu Izraeliu nuo Hamato iki Egipto upės šventė Viešpaties, mūsų Dievo, akivaizdoje septynias dienas ir kitas septynias, iš viso keturiolika dienų. 
\par 66 Aštuntą dieną jis paleido žmones; jie palaimino karalių ir nuėjo į savo palapines, džiaugdamiesi dėl to gero, kurį Viešpats padarė savo tarnui Dovydui ir visai Izraelio tautai.



\chapter{9}

\par 1 Kai Saliamonas baigė statyti Viešpaties namus, karaliaus namus ir visa, ką jis buvo sumanęs, 
\par 2 Viešpats pasirodė Saliamonui antrą kartą taip, kaip buvo pasirodęs Gibeone, 
\par 3 ir tarė: “Aš išklausiau tavo maldą ir prašymą. Aš pašventinau šituos namus, kuriuos pastatei, kad juose amžinai būtų mano vardas. Ir mano akys bei mano širdis visuomet bus ten. 
\par 4 Jei tu vaikščiosi priešais mane, kaip vaikščiojo tavo tėvas Dovydas, tyra ir neklastinga širdimi vykdydamas tai, ką tau įsakau, ir laikysies mano nuostatų bei potvarkių, 
\par 5 tai įtvirtinsiu tavo karalystės sostą Izraelyje amžiams, kaip pažadėjau tavo tėvui Dovydui, sakydamas: ‘Tu nepritrūksi vyro, kuris sėdėtų Izraelio soste’. 
\par 6 Bet jei jūs ar jūsų vaikai nuo manęs nusigręšite ir nesilaikysite mano įsakymų bei mano nuostatų, kuriuos jums daviau, ir nuėję tarnausite kitiems dievams ir juos garbinsite, 
\par 7 tai išnaikinsiu Izraelį iš žemės, kurią jiems daviau, ir atmesiu šituos namus, kuriuos pašventinau savo vardui; Izraelis taps patarle ir priežodžiu visose tautose. 
\par 8 Ir dėl šitų namų, kurie yra aukšti, kiekvienas praeivis stebėsis, švilps ir sakys: ‘Kodėl Viešpats taip padarė šitai žemei ir šitiems namams?’ 
\par 9 Ir atsakys: ‘Kadangi jie paliko Viešpatį, savo Dievą, kuris išvedė jų tėvus iš Egipto, ir garbino kitus dievus ir jiems tarnavo, tai Viešpats siuntė jiems šitą nelaimę’ ”. 
\par 10 Dvidešimt metų, kol Saliamonas statė Viešpaties namus ir karaliaus namus, 
\par 11 Tyro karalius Hiramas aprūpino Saliamoną kedrų ir kiparisų medžiais, ir auksu, kiek jam reikėjo. Už tai karalius Saliamonas davė Hiramui dvidešimt miestų Galilėjos žemėje. 
\par 12 Hiramas atvyko iš Tyro pažiūrėti Saliamono jam dovanotų miestų ir jie jam nepatiko. 
\par 13 Jis sakė: “Kokie tai miestai, kuriuos man davei, mano broli?” Jis pavadino juos Kabulo šalimi, ir taip jie vadinami iki šiol. 
\par 14 Ir Hiramas pasiuntė karaliui šimtą dvidešimt talentų aukso. 
\par 15 Karalius Saliamonas įvedė darbo prievolę, statydamas Viešpaties namus, savo namus, Milojų Jeruzalės sieną, Hacoro, Megido ir Gezero miestus. 
\par 16 Faraonas, Egipto karalius, užėmė Gezerą, jį sudegino, išžudė kanaaniečius, gyvenusius mieste, ir atidavė kaip kraitį savo dukteriai, Saliamono žmonai. 
\par 17 Saliamonas atstatė Gezerą, žemutinį Bet Horoną, 
\par 18 Baalatą, Tadmorą dykumoje, 
\par 19 pastatė sandėlių, kovos vežimų ir raitelių miestus; jis statė tai, ką norėjo Jeruzalėje, Libane ir visame krašte, kurį valdė. 
\par 20 Neizraelitų kilmės gyventojų iš amonitų, hetitų, periziečių, hevitų ir jebusiečių tautų, 
\par 21 kurių izraelitai neįstengė visiškai sunaikinti, palikuonis Saliamonas apdėjo prievolėmis, ir taip liko iki šios dienos. 
\par 22 Bet nė vieno izraelito Saliamonas nepavergė. Jie buvo kariai, tarnai, kunigaikščiai, vadai ir kovos vežimų bei raitelių viršininkai. 
\par 23 Saliamonas turėjo penkis šimtus penkiasdešimt prižiūrėtojų darbininkams. 
\par 24 Kai faraono duktė persikėlė iš Dovydo miesto į savo namus, kuriuos Saliamonas jai pastatė, jis pradėjo statyti Milojų. 
\par 25 Tris kartus per metus Saliamonas aukodavo deginamąsias bei padėkos aukas ant aukuro, kurį jis padarė Viešpačiui. Jis degino ir smilkalus ant aukuro, kuris buvo Viešpaties akivaizdoje. Taip jis pabaigė statyti namus. 
\par 26 Karalius Saliamonas pasistatė laivų Ecjon Geberyje, prie Elato, Raudonosios jūros pakrantėje, Edomo šalyje. 
\par 27 Hiramas atsiuntė tiems laivams savo tarnų, patyrusių jūrininkų, kurie plaukė kartu su Saliamono tarnais. 
\par 28 Tie, nuplaukę į Ofyrą, iš ten parvežė keturis šimtus dvidešimt talentų aukso karaliui Saliamonui.



\chapter{10}

\par 1 Šebos karalienė, išgirdusi apie Saliamoną, pagarsėjusį dėl Viešpaties, atvyko jį išmėginti sunkiais klausimais. 
\par 2 Ji atkeliavo į Jeruzalę su labai didele palyda; kupranugariai nešė kvepalų, labai daug aukso ir brangių akmenų. Atėjusi pas Saliamoną ji kalbėjo su juo apie visa, kas buvo jos širdyje. 
\par 3 Saliamonas atsakė jai į visus klausimus. Nebuvo nieko, ko karalius nebūtų galėjęs jai atsakyti. 
\par 4 Šebos karalienė, pamačiusi Saliamono išmintį ir namus, kuriuos jis pastatė, 
\par 5 jo stalo valgius, tarnų būstus ir patarnautojų laikyseną bei apdarus, vyno pilstytojus ir užėjimą į Viešpaties namus, nebegalėjo susilaikyti 
\par 6 ir tarė karaliui: “Ką girdėjau savo krašte apie tavo darbus ir išmintį, yra tiesa. 
\par 7 Aš netikėjau tais žodžiais, kol neatvykau ir savo akimis nepamačiau. Iš tikrųjų nė pusės man nebuvo pasakyta. Tavo išmintis ir turtai viršija tai, ką apie tave girdėjau. 
\par 8 Laimingi tavo žmonės ir laimingi šie tavo tarnai, kurie nuolat yra priešais tave ir girdi tavo išmintį. 
\par 9 Palaimintas Viešpats, tavo Dievas, kuris pamėgo tave ir pasodino Izraelio soste. Viešpats pamilo Izraelį amžiams ir todėl paskyrė tave karaliumi teismui ir teisingumui vykdyti”. 
\par 10 Ji padovanojo karaliui šimtą dvidešimt talentų aukso, labai daug kvepalų ir brangiųjų akmenų. Niekad daugiau nebuvo atgabenta tiek kvepalų, kiek Šebos karalienė padovanojo karaliui Saliamonui. 
\par 11 Hiramo laivai parvežė iš Ofyro aukso, labai daug raudonmedžio ir brangiųjų akmenų. 
\par 12 Karalius pagamino iš to medžio stulpus Viešpaties namams ir karaliaus namams, psalterių bei arfų. Tiek raudonmedžio nebuvo nei atgabenta, nei matyta iki šios dienos. 
\par 13 Karalius Saliamonas davė Šebos karalienei visa, ko ji norėjo ir prašė, be to, ką Saliamonas jai davė iš karališko dosnumo. Po to ji grįžo į savo šalį su visa palyda. 
\par 14 Auksas, kurį kas metai atgabendavo Saliamonui, svėrė šešis šimtus šešiasdešimt šešis talentus, 
\par 15 neskaičiuojant to, ką gaudavo iš prekybininkų, keliaujančių pirklių bei visų Arabijos karalių ir šalies valdytojų. 
\par 16 Karalius Saliamonas padarė du šimtus didžiųjų skydų iš kalto aukso; šeši šimtai šekelių aukso buvo sunaudota vienam skydui; 
\par 17 be to, buvo padaryti trys šimtai mažų skydų, taip pat iš kalto aukso; vienam skydui sunaudojo tris minas aukso. Karalius juos visus laikė namuose iš Libano medžio. 
\par 18 Karalius padirbdino didelį sostą iš dramblio kaulo ir padengė jį geriausiu auksu. 
\par 19 Sostas turėjo šešis laiptus, jo viršus užpakalinėje dalyje buvo apvalus, atramos rankoms buvo abiejose sosto pusėse, o du liūtai stovėjo šalia atramų. 
\par 20 Dvylika liūtų stovėjo ant šešių laiptų, po vieną iš abiejų pusių. Nieko panašaus nebuvo padaryta jokioje karalystėje. 
\par 21 Visi karaliaus Saliamono geriamieji indai buvo auksiniai; visi namų iš Libano medžių reikmenys buvo gryno aukso. Nieko nebuvo iš sidabro, nes jis neturėjo vertės Saliamono dienomis. 
\par 22 Karaliaus laivai drauge su Hiramo laivais kas treji metai grįždavo iš Taršišo ir atgabendavo aukso, sidabro, dramblio kaulo, beždžionių bei povų. 
\par 23 Karalius Saliamonas savo turtais ir išmintimi pranoko visus žemės karalius. 
\par 24 Visas pasaulis norėjo pamatyti Saliamoną ir išgirsti jo išmintį, kurią Dievas buvo įdėjęs į jo širdį. 
\par 25 Kiekvienas atgabendavo dovanų: sidabrinių ir auksinių daiktų, rūbų, ginklų, kvepalų, žirgų ir mulų; taip buvo metai po metų. 
\par 26 Saliamonas turėjo tūkstantį keturis šimtus kovos vežimų ir dvylika tūkstančių raitelių, kuriuos jis buvo paskirstęs kovos vežimų miestuose ir pas save Jeruzalėje. 
\par 27 Karalius padarė, kad sidabro Jeruzalėje buvo kaip akmenų ir kedrų kaip figmedžių, kurių gausiai auga slėniuose. 
\par 28 Saliamonas parsigabendavo žirgų iš Egipto ir Kevės. Karaliaus pirkliai pirkdavo juos Kevėje už pinigus. 
\par 29 Iš Egipto pirkdavo kovos vežimą už šešis šimtus šekelių sidabro, o žirgą­už šimtą penkiasdešimt. Taip pat jie pristatydavo žirgus visiems hetitų ir Sirijos karaliams.



\chapter{11}


\par 1 Karalius Saliamonas mylėjo daug svetimšalių moterų. Be faraono dukters, jis turėjo moabičių, amoničių, edomičių, sidoniečių, hetičių­ 
\par 2 moterų iš tautų, apie kurias Viešpats buvo sakęs izraelitams: “Neveskite jų ir neleiskite savo dukterų už jų, nes jie tikrai nukreips jūsų širdis į savo dievus”. Tačiau Saliamonas įsimylėjo jas. 
\par 3 Jis turėjo septynis šimtus žmonų ir tris šimtus sugulovių; ir jo žmonos nukreipė jo širdį. 
\par 4 Saliamonui pasenus, jo žmonos nukreipė jo širdį į kitus dievus, ir jo širdis nebuvo tobula prieš Viešpatį, jo Dievą, kaip jo tėvo Dovydo širdis. 
\par 5 Saliamonas sekė sidoniečių deivę Astartę ir amonitų pabaisą Milkomą. 
\par 6 Saliamonas darė pikta Viešpaties akyse ir nesekė iki galo Viešpačiu, kaip jo tėvas Dovydas. 
\par 7 Saliamonas pastatė ant kalno ties Jeruzale aukštumą Moabo pabaisai Chemošui ir amonitų pabaisai Molechui. 
\par 8 Taip jis padarė visoms svetimšalėms savo žmonoms, kurios smilkydavo ir aukodavo savo dievams. 
\par 9 Viešpats užsirūstino ant Saliamono, nes jis nusigręžė nuo savo Viešpaties, Izraelio Dievo, kuris jam buvo pasirodęs du kartus 
\par 10 ir įsakęs jam, kad nesektų paskui svetimus dievus. Bet jis nesilaikė Viešpaties įsakymo. 
\par 11 Tada Viešpats tarė Saliamonui: “Kadangi tu taip pasielgei ir nesilaikei mano sandoros bei mano nuostatų, kuriuos tau daviau, tai Aš atimsiu iš tavęs karalystę ir ją atiduosiu tavo tarnui. 
\par 12 Tačiau tau gyvam esant to nedarysiu dėl tavo tėvo Dovydo, bet atimsiu ją iš tavo sūnaus. 
\par 13 Visos karalystės neatimsiu, vieną giminę duosiu tavo sūnui dėl savo tarno Dovydo ir dėl Jeruzalės, kurią išsirinkau”. 
\par 14 Viešpats sukurstė Hadadą iš Edomo karaliaus giminės prieš Saliamoną. 
\par 15 Kai Dovydas nugalėjo Edomą ir kariuomenės vadas Joabas laidojo žuvusius, Joabas išžudė visus vyrus Edome. 
\par 16 Jis ten buvo pasilikęs šešis mėnesius su visu Izraeliu, kol išnaikino visus vyrus Edome. 
\par 17 Tuo metu Hadadas su kai kuriais edomitais, jo tėvo tarnais, pabėgo iš Edomo, norėdami patekti į Egiptą. Tada Hadadas buvo dar vaikas. 
\par 18 Iš Midiano jie atėjo į Paraną. Čia prie jų prisidėjo daugiau žmonių. Jie visi atėjo pas faraoną, Egipto karalių, kuris Hadadui davė namus, žemės ir aprūpinimą. 
\par 19 Hadadas įsigijo tokį didelį faraono palankumą, kad tas leido jam vesti savo žmonos, karalienės Tachpenesės, seserį. 
\par 20 Tachpenesės sesuo pagimdė sūnų Genubatą. Genubatas augo Tachpenesės priežiūroje faraono namuose drauge su faraono sūnumis. 
\par 21 Kai Hadadas, būdamas Egipte, sužinojo, kad Dovydas ir kariuomenės vadas Joabas mirę, jis prašė faraoną: “Leisk man eiti į savo šalį”. 
\par 22 Faraonas klausė jo: “Ko tau trūksta pas mane, kad nori grįžti į savo kraštą?” Jis atsakė: “Nieko man netrūksta, bet išleisk mane”. 
\par 23 Dievas sukėlė prieš jį ir kitą priešą Eljado sūnų Razoną, kuris buvo pabėgęs nuo savo valdovo Hadadezerio, Cobos karaliaus. 
\par 24 Tas surinko būrį vyrų ir tapo jų vadu, kai Dovydas sumušė juos Coboje. Jis su vyrais nuėjo į Damaską, ten apsigyveno ir tapo Damasko karaliumi. 
\par 25 Per visas Saliamono dienas jis buvo Izraelio priešas, jam kenkė ir nekentė jo lygiai taip pat, kaip Hadadas. 
\par 26 Taip pat ir Nebato sūnus Jeroboamas, efraimitas iš Ceredos, Saliamono tarnas, kurio motina buvo našlė, vardu Ceruva, pakėlė savo ranką prieš karalių. 
\par 27 Priežastis, dėl ko jis pakėlė savo ranką prieš karalių, buvo tokia: Saliamonas, statydamas Milojų, taisė savo tėvo Dovydo miesto sienas. 
\par 28 Jeroboamas buvo sumanus vyras. Saliamonas, matydamas, kad jaunuolis išradingas, pavedė jam prižiūrėti visos Juozapo giminės darbą. 
\par 29 Kartą Jeroboamas, išėjęs iš Jeruzalės, kelyje sutiko pranašą Ahiją iš Šilojo. Jis buvo apsisiautęs nauju apsiaustu. Juodu buvo vieni laukuose. 
\par 30 Ahija, nutvėręs savo naują apsiaustą, suplėšė jį į dvyliką dalių 
\par 31 ir tarė Jeroboamui: “Pasiimk dešimt dalių, nes Viešpats, Izraelio Dievas, sako: ‘Aš atimsiu karalystę iš Saliamono ir tau duosiu dešimt giminių, 
\par 32 bet vieną giminę jam paliksiu dėl savo tarno Dovydo ir Jeruzalės miesto, kurį išsirinkau iš visų Izraelio giminių. 
\par 33 Tai dėl to, kad jie paliko mane ir garbino sidoniečių deivę Astartę, Moabo dievą Chemošą bei amonitų dievą Milkomą, nevaikščiojo mano keliais, nedarė, kas teisinga mano akyse, ir nesilaikė mano nuostatų bei potvarkių, kaip jo tėvas Dovydas. 
\par 34 Tačiau Aš neatimsiu karalystės iš jo rankos. Jį paliksiu valdovu, kol jis gyvas, dėl savo tarno Dovydo, kurį išsirinkau ir kuris laikėsi mano įsakymų bei nuostatų. 
\par 35 Aš atimsiu karalystę iš Saliamono sūnaus ir tau duosiu dešimt giminių. 
\par 36 Jo sūnui duosiu vieną giminę, kad visą laiką mano tarno Dovydo žiburys būtų priešais mane Jeruzalėje, kurią išsirinkau savo vardui. 
\par 37 Tave paimsiu, ir tu karaliausi, kaip trokš tavo siela, ir būsi karaliumi visam Izraeliui. 
\par 38 Jei paklusi mano įstatymams, vaikščiosi mano keliais, darysi, kas teisinga mano akivaizdoje, laikysiesi mano nuostatų ir įsakymų, kaip darė mano tarnas Dovydas, tai Aš būsiu su tavimi, tau pastatysiu tikrus namus, kaip pastačiau Dovydui, ir tau duosiu Izraelį. 
\par 39 Tuo Aš pažeminsiu Dovydo palikuonis, tačiau ne visam laikui’ ”. 
\par 40 Saliamonas dėl to norėjo nužudyti Jeroboamą, bet Jeroboamas pabėgo į Egiptą pas karalių Šišaką ir pasiliko ten iki Saliamono mirties. 
\par 41 Kiti Saliamono darbai ir jo poelgiai bei išmintis yra aprašyta Saliamono darbų knygoje. 
\par 42 Saliamonas karaliavo visam Izraeliui Jeruzalėje keturiasdešimt metų. 
\par 43 Saliamonas užmigo prie savo tėvų ir buvo palaidotas savo tėvo Dovydo mieste; jo sūnus Roboamas pradėjo karaliauti jo vietoje.



\chapter{12}

\par 1 Roboamas nuėjo į Sichemą, kur buvo susirinkę visi izraelitai paskelbti jį karaliumi. 
\par 2 Nebato sūnus Jeroboamas, kuris buvo pabėgęs nuo karaliaus Saliamono į Egiptą, išgirdo apie tai dar būdamas Egipte. 
\par 3 Jie pasiuntė ir pasikvietė jį. Jeroboamas ir visas Izraelis atėjo ir kalbėjo Roboamui: 
\par 4 “Tavo tėvas uždėjo mums sunkų jungą. Dabar palengvink savo tėvo mums uždėtą naštą, tai mes tau tarnausime”. 
\par 5 Jis jiems atsakė: “Eikite ir po trijų dienų sugrįžkite pas mane”. Ir žmonės nuėjo. 
\par 6 Karalius Roboamas tarėsi su senesniaisiais, kurie stovėdavo priešais jo tėvą Saliamoną, kai jis dar buvo gyvas: “Patarkite man, ką atsakyti tautai”. 
\par 7 Tie jam kalbėjo: “Jei šiandien būsi tarnas šitiems žmonėms, tarnausi jiems ir kalbėsi švelniais žodžiais, jie visados bus tavo tarnai”. 
\par 8 Bet jis atmetė senesniųjų duotą patarimą ir tarėsi su jaunesniaisiais, kurie užaugo kartu su juo ir stovėjo priešais jį. 
\par 9 Jis jiems tarė: “Ką jūs patariate man atsakyti šitiems žmonėms, kurie man kalbėjo: ‘Palengvink jungą, kurį mums uždėjo tavo tėvas’?” 
\par 10 Jaunieji, kurie užaugo kartu su juo, jam kalbėjo: “Šitiems žmonėms, kurie tau sakė: ‘Tavo tėvas padarė mūsų jungą sunkų, o tu jį mums palengvink’, taip atsakyk: ‘Mano mažasis pirštas storesnis už mano tėvo strėnas. 
\par 11 Mano tėvas jums uždėjo sunkų jungą, bet aš jį jums dar pasunkinsiu. Mano tėvas jus plakė botagais, o aš jus plaksiu dygliuotais rimbais’ ”. 
\par 12 Kai Jeroboamas ir visa tauta trečią dieną atėjo pas Roboamą, kaip karalius buvo paskyręs, sakydamas: “Sugrįžkite pas mane trečią dieną”, 
\par 13 karalius, atmetęs senesniųjų patarimą, kalbėjo tautai griežtai, 
\par 14 kaip patarė jaunieji: “Mano tėvas uždėjo jums sunkų jungą, o aš jį jums dar pasunkinsiu. Mano tėvas jus plakė botagais, o aš jus plaksiu dygliuotais rimbais”. 
\par 15 Karalius nepaklausė tautos, nes tai buvo nuo Viešpaties, kad Viešpats ištesėtų savo žodį, kurį Jis kalbėjo per Ahiją iš Šilojo Nebato sūnui Jeroboamui. 
\par 16 Izraelitai, pamatę, kad karalius nenori jų išklausyti, atsakė jam: “Mes neturime dalies Dovyde nei paveldėjimo Jesės sūnuje. Izraeli, į savo palapines! Dovydai, rūpinkis savo namais”. Ir Izraelis išsiskirstė į savo palapines. 
\par 17 Izraelitams, gyvenantiems Judo miestuose, karaliavo Roboamas. 
\par 18 Jis pasiuntė Adoramą, mokesčių rinkėją, pas izraelitus, bet jie užmušė jį akmenimis. Karalius Roboamas skubiai įšoko į vežimą ir pabėgo į Jeruzalę. 
\par 19 Taip Izraelis atsiskyrė nuo Dovydo namų iki šios dienos. 
\par 20 Visas Izraelis, išgirdęs, kad Jeroboamas grįžęs, pasikvietė jį į susirinkimą ir paskelbė jį viso Izraelio karaliumi. Niekas nebesekė Dovydo namais, išskyrus Judo giminę. 
\par 21 Roboamas, sugrįžęs į Jeruzalę, surinko visus Judo ir Benjamino giminių vyrus, šimtą aštuoniasdešimt tūkstančių rinktinių karių, karui su Izraeliu, kad sugrąžintų karalystę Roboamui, Saliamono sūnui. 
\par 22 Bet Dievo žodis atėjo Dievo vyrui Šemajai: 
\par 23 “Kalbėk Saliamono sūnui Roboamui, Judo karaliui, visiems Judo ir Benjamino namams ir likusiai tautai, sakydamas: 
\par 24 ‘Taip sako Viešpats: ‘Neikite ir nekariaukite su savo broliais izraelitais. Kiekvienas grįžkite į savo namus, nes tai atėjo iš manęs’ ”. Jie pakluso Viešpaties žodžiui ir grįžo, kaip Viešpats liepė. 
\par 25 Jeroboamas pastatydino Sichemą Efraimo kalnyne ir ten apsigyveno. Iš ten išėjęs jis pastatė Penuelio miestą. 
\par 26 Jeroboamas sakė savo širdyje: “Karalystė gali sugrįžti Dovydo namams. 
\par 27 Jei šita tauta eis aukoti į Viešpaties namus Jeruzalėje, tai žmonių širdys atsigręš į jų valdovą, į Judo karalių Roboamą, ir jie, nužudę mane, sugrįš pas Judo karalių Roboamą”. 
\par 28 Karalius pasitaręs padirbdino du auksinius veršius ir tarė tautai: “Per toli jums eiti į Jeruzalę. Izraeli, štai tavo dievai, kurie tave išvedė iš Egipto žemės”. 
\par 29 Jis pastatė vieną Betelyje, o kitą­ Dane. 
\par 30 Tai tapo nuodėme, nes tauta eidavo net į Daną jų garbinti. 
\par 31 Jeroboamas pastatydino šventyklą aukštumose ir paskyrė kunigų iš žmonių, kurie nebuvo Levio sūnūs. 
\par 32 Aštuntojo mėnesio penkioliktą dieną Jeroboamas paskelbė šventę, panašią į tą, kurią švęsdavo Jude, ir aukojo ant aukuro. Taip jis darė Betelyje aukodamas veršiams, kuriuos padirbdino, ir Betelio aukštumoms, kurias įrengė, paskyrė kunigus. 
\par 33 Jis aukojo ant aukuro, kurį pastatė Betelyje, aštunto mėnesio penkioliktą dieną­dieną, kurią jis sumanė savo širdyje, ir paskelbė šventę izraelitams. Ir jis aukojo ant aukuro bei degino smilkalus.



\chapter{13}

\par 1 Viešpaties siųstas Dievo vyras atėjo iš Judo į Betelį, kai Jeroboamas stovėjo prie aukuro, norėdamas smilkyti. 
\par 2 Jis šaukė prieš aukurą Viešpaties žodžius, sakydamas: “Aukure, aukure! Taip sako Viešpats: ‘Dovydo namams užgims sūnus, vardu Jozijas; jis aukos ant tavęs aukštumų kunigus, kurie čia smilko, ir žmonių kaulus sudegins ant tavęs’ ”. 
\par 3 Ir jis davė ženklą tą dieną, sakydamas: “Jūs matysite ženklą, kad Viešpats tikrai taip kalbėjo. Štai aukuras sugrius ir pelenai išbyrės”. 
\par 4 Karalius Jeroboamas, išgirdęs Dievo vyro žodžius, kuriuos jis kalbėjo prieš aukurą Betelyje, ištiesė savo ranką ir liepė suimti jį. Jo ranka, kurią jis buvo ištiesęs, padžiūvo ir jis nebegalėjo jos prie savęs pritraukti. 
\par 5 Aukuras sugriuvo ir pelenai išbyrėjo pagal ženklą, kurį Dievo vyras buvo paskelbęs nuo Viešpaties. 
\par 6 Tada karalius tarė Dievo vyrui: “Maldauk Viešpatį, savo Dievą, kad mano ranka būtų atstatyta”. Dievo vyras meldėsi, ir karaliaus ranka buvo atstatyta ir pasidarė, kokia buvo anksčiau. 
\par 7 Karalius sakė Dievo vyrui: “Eime pas mane į namus pasistiprinti, ir aš tau atsilyginsiu”. 
\par 8 Dievo vyras atsakė karaliui: “Jei man duotum pusę savo namų, aš neičiau su tavimi, nevalgyčiau duonos ir negerčiau vandens šitoje vietoje, 
\par 9 nes Viešpats man taip įsakė: ‘Tau nevalia nei duonos valgyti, nei vandens gerti, nei grįžti keliu, kuriuo atėjai’ ”. 
\par 10 Taip jis nuėjo kitu keliu ir negrįžo tuo, kuriuo atėjo į Betelį. 
\par 11 Betelyje gyveno senas pranašas. Jo sūnūs parėję pasakojo jam viską, ką Dievo vyras buvo padaręs tą dieną Betelyje ir ką jis kalbėjo karaliui. 
\par 12 Tada tėvas klausė: “Kuriuo keliu jis nuėjo?” Sūnūs parodė kelią, kuriuo nuėjo Dievo vyras. 
\par 13 Tėvas liepė pabalnoti asilą. Jie pabalnojo asilą, ir jis užsėdęs 
\par 14 nujojo paskui Dievo vyrą. Radęs jį sėdintį po ąžuolu, klausė: “Ar tu esi Dievo vyras, atėjęs iš Judo?” Jis atsakė: “Taip, aš”. 
\par 15 Tuomet jis sakė: “Eime pas mane į namus ir užvalgyk duonos”. 
\par 16 Jis atsakė: “Negaliu grįžti su tavimi, nei valgyti duonos, nei gerti vandens su tavimi šioje vietoje, 
\par 17 nes man Viešpaties pasakyta: ‘Tau nevalia nei duonos valgyti, nei vandens gerti, nei grįžti tuo keliu, kuriuo atėjai’ ”. 
\par 18 Jis sakė jam: “Aš irgi esu pranašas kaip ir tu; angelas kalbėjo man Viešpaties žodžius, sakydamas: ‘Parsivesk jį į savo namus, kad jis galėtų valgyti ir gerti’ ”. Bet jis jam melavo. 
\par 19 Jie sugrįžo, valgė duonos ir gėrė vandens jo namuose. 
\par 20 Jiems tebesėdint prie stalo, Viešpaties žodis atėjo pranašui, kuris buvo jį parsivedęs. 
\par 21 Ir jis šaukė Dievo vyrui, kuris buvo atėjęs iš Judo: “Taip sako Viešpats: ‘Kadangi neklausei Viešpaties ir nesilaikei įsakymo, kurį tau davė Viešpats, tavo Dievas, 
\par 22 bet sugrįžai ir valgei duonos bei gėrei vandens vietoje, apie kurią Jis tau kalbėjo, kad nevalia joje nei duonos valgyti, nei vandens gerti, tavo lavonas nebus palaidotas tavo tėvų kape’ ”. 
\par 23 Kai tas pavalgė ir atsigėrė, jis pabalnojo asilą pranašui, kurį buvo parsivedęs. 
\par 24 Jam keliaujant, jį sutiko liūtas ir nužudė. Jo lavonas gulėjo ant kelio, o asilas stovėjo šalia jo; taip pat ir liūtas stovėjo šalia lavono. 
\par 25 Žmonės praeidami ant kelio matė gulintį lavoną ir liūtą, stovintį šalia lavono. Atėję į miestą, kuriame gyveno senasis pranašas, pasakojo, ką buvo matę. 
\par 26 Tai išgirdęs, pranašas, kuris jį buvo sugrąžinęs iš kelio, tarė: “Tai Dievo vyras, kuris buvo nepaklusnus Viešpaties žodžiui; todėl Viešpats jį atidavė liūtui, kuris jį sudraskė ir nužudė, kaip Viešpats buvo jam kalbėjęs”. 
\par 27 Savo sūnums jis tarė: “Pabalnokite man asilą”. Ir jie pabalnojo. 
\par 28 Nuvykęs jis rado lavoną, gulintį ant kelio, ir asilą su liūtu, stovinčius šalia lavono. Liūtas nelietė nei lavono, nei asilo. 
\par 29 Pranašas pakėlė Dievo vyro lavoną, uždėjo jį ant asilo ir pargabeno atgal į miestą, kad apraudotų jį ir palaidotų. 
\par 30 Jis paguldė lavoną į savo paties kapą ir raudojo: “Ak, mano broli!” 
\par 31 Jį palaidojęs, jis tarė savo sūnums: “Kai numirsiu, palaidokite mane kape, kuriame palaidotas Dievo vyras; šalia jo kaulų padėkite mano kaulus, 
\par 32 nes žodis, kurį jis, Viešpačiui įsakius, šaukė prieš aukurą Betelyje ir prieš visas Samarijos miestų aukštumas, tikrai išsipildys”. 
\par 33 Po viso to Jeroboamas neatsisakė savo pikto kelio, bet toliau skyrė kunigus aukštumoms iš prasčiausių žmonių. Kas norėdavo, tą jis įšventindavo aukštumų kunigu. 
\par 34 Tai buvo Jeroboamo namų nuodėmė, ir jie buvo sunaikinti bei pašalinti nuo žemės paviršiaus.



\chapter{14}

\par 1 Tuo metu Jeroboamo sūnus Abija susirgo. 
\par 2 Jeroboamas tarė savo žmonai: “Persirenk, kad neatpažintų, jog esi Jeroboamo žmona, ir eik į Šilojų. Ten gyvena pranašas Ahija, kuris pasakė, kad aš tapsiu šitos tautos karaliumi. 
\par 3 Pasiimk dešimt duonos kepalų, pyragaičių bei medaus ąsotį ir eik pas jį. Jis pasakys, kas atsitiks vaikui”. 
\par 4 Jeroboamo žmona taip ir padarė. Ji nuėjo į Šilojų pas Ahiją. Ahija nebematė dėl senatvės. 
\par 5 Ir Viešpats pasakė Ahijai: “Ateina Jeroboamo žmona sužinoti iš tavęs apie savo sūnų, kuris serga. Aš pasakysiu, ką jai sakyti. Nes atėjusi ji dėsis kita moterimi”. 
\par 6 Kai Ahija išgirdo įeinančios žingsnius, jis tarė: “Įeik, Jeroboamo žmona! Kodėl dediesi esanti kita? Turiu tau blogų žinių. 
\par 7 Pasakyk Jeroboamui, kad Viešpats, Izraelio Dievas, sako: ‘Aš tave išaukštinau ir paskyriau kunigaikščiu savo tautai, Izraeliui. 
\par 8 Atėmęs karalystę iš Dovydo namų, ją tau daviau. Bet tu nebuvai kaip mano tarnas Dovydas, kuris laikėsi mano įsakymų ir sekė mane visa savo širdimi, darydamas tai, kas buvo teisinga mano akyse. 
\par 9 Tu elgeisi blogiau už visus, pirma tavęs buvusius, pasidirbdinai kitų dievų ir lietų atvaizdų, sukėlei mano pyktį ir atsukai man nugarą. 
\par 10 Todėl aš bausiu Jeroboamo namus ir išnaikinsiu visus Jeroboamo vyrus, laisvus ir pavergtuosius, išvalysiu Jeroboamo namus, kaip žmogus išvalo mėšlą, kol nė vieno nebeliks. 
\par 11 Kas iš Jeroboamo mirs mieste, tą suės šunys, kas mirs lauke, tą les padangių paukščiai, nes taip pasakė Viešpats’. 
\par 12 O tu eik namo. Tau įžengus į miestą, vaikas mirs. 
\par 13 Jį apraudos visas Izraelis ir palaidos. Jis vienintelis iš Jeroboamo bus palaidotas kape, nes tik jis patiko Viešpačiui, Izraelio Dievui, iš Jeroboamo namų. 
\par 14 Viešpats pakels Izraeliui karalių, kuris sunaikins Jeroboamo namus tą dieną ir netgi dabar. 
\par 15 Viešpats ištiks Izraelį, kad jis siūbuos kaip nendrė vandenyje; išraus Izraelį iš šitos geros žemės, kurią Jis davė jų tėvams ir išsklaidys juos anapus upės dėl to, kad jie pasidarė alkų, sukeldami Viešpaties pyktį. 
\par 16 Jis apleis Izraelį dėl Jeroboamo nuodėmių, nes jis pats nusidėjo ir įvedė Izraelį į nuodėmę”. 
\par 17 Jeroboamo žmona sugrįžo į Tircą. Jai įžengus į namus, berniukas mirė. 
\par 18 Jis buvo palaidotas, ir visas Izraelis apraudojo jį, kaip paskelbė Viešpats per savo tarną pranašą Ahiją. 
\par 19 Visi kiti Jeroboamo darbai, kaip jis kariavo ir karaliavo, yra surašyti Izraelio karalių metraščių knygoje. 
\par 20 Jeroboamas valdė Izraelį dvidešimt dvejus metus. Jam mirus, jo vietą užėmė jo sūnus Nadabas. 
\par 21 Saliamono sūnus Roboamas karaliavo Jude. Pradėdamas valdyti kraš-tą, jis buvo keturiasdešimt vienerių metų amžiaus. Septyniolika metų jis karaliavo Jeruzalėje, mieste, kurį Viešpats išsirinko iš visų Izraelio giminių. Jo motina buvo amonitė Naama. 
\par 22 Judo žmonės darė pikta Viešpaties akyse, sukeldami Jo pavydą savo nuodėmėmis, kurios buvo sunkesnės, negu jų tėvų. 
\par 23 Jie įrengė sau aukštumas, pasistatė atvaizdus, pasidarė alkus ant kiekvienos aukštos kalvos ir po kiekvienu žaliuojančiu medžiu. 
\par 24 Krašte buvo ir iškrypėlių. Jie darė visus bjaurius darbus tautų, kurias Viešpats išnaikino prieš Izraeliui užimant kraštą. 
\par 25 Penktaisiais karaliaus Roboamo valdymo metais Egipto karalius Šišakas atėjęs užpuolė Jeruzalę, 
\par 26 paėmė Viešpaties namų bei karaliaus namų turtus ir viską išvežė. Jis paėmė ir visus Saliamono padirbdintus auksinius skydus. 
\par 27 Karalius Roboamas padirbdino jų vietoje varinių skydų ir juos pavedė karaliaus namų sargybos viršininkams. 
\par 28 Sargybiniai juos nešdavo karaliui einant į Viešpaties namus; po to juos padėdavo atgal į sargybinių patalpą. 
\par 29 Visi kiti Roboamo darbai surašyti Judo karalių metraščių knygoje. 
\par 30 Karas tarp Roboamo ir Jeroboamo tęsėsi per visas jų dienas. 
\par 31 Roboamas užmigo prie savo tėvų ir buvo palaidotas prie savo tėvų Dovydo mieste. Jo motina buvo amonitė Naama. Jo sūnus Abijamas karaliavo jo vietoje.



\chapter{15}


\par 1 Aštuonioliktaisiais karaliaus Jeroboamo, Nebato sūnaus, valdymo metais Abijamas pradėjo karaliauti Jude. 
\par 2 Trejus metus jis valdė Judą, gyvendamas Jeruzalėje. Jo motina buvo Abšalomo duktė Maaka. 
\par 3 Abijamas vaikščiojo visose savo tėvo nuodėmėse, kurias tas darė iki jo. Jo širdis nebuvo tobula prieš Viešpatį, jo Dievą, kaip jo tėvo Dovydo širdis. 
\par 4 Tačiau Dovydo dėlei Viešpats, jo Dievas, davė jam žiburį Jeruzalėje, pakeldamas jo sūnų po jo ir įtvirtindamas Jeruzalę. 
\par 5 Nes Dovydas darė tai, kas teisinga Viešpaties akyse ir nenukrypo nuo viso to, ką Jis įsakė, per visas savo dienas, išskyrus atsitikimą su hetitu Ūrija. 
\par 6 Karas tarp Roboamo ir Jeroboamo tęsėsi per visas jo gyvenimo dienas. 
\par 7 Visi kiti Abijamo darbai surašyti Judo karalių metraščių knygoje. Tarp Abijamo ir Jeroboamo vyko karas. 
\par 8 Abijamas užmigo prie savo tėvų ir buvo palaidotas Dovydo mieste, o jo vietoje pradėjo karaliauti jo sūnus Asa. 
\par 9 Dvidešimtaisiais Izraelio karaliaus Jeroboamo valdymo metais Judą pradėjo valdyti karalius Asa. 
\par 10 Keturiasdešimt vienerius metus jis karaliavo Jeruzalėje. Jo motina buvo Abšalomo duktė Maaka. 
\par 11 Asa darė tai, kas teisinga Viešpaties akyse, kaip jo tėvas Dovydas. 
\par 12 Jis pašalino iš krašto iškrypėlius ir visus stabus, kuriuos buvo padarę jo tėvai. 
\par 13 Net savo motiną Maaką jis pašalino iš karalienės vietos, nes ji buvo padirbdinusi giraitėje stabą, kurį Asa sukapojo ir sudegino Kidrono slėnyje. 
\par 14 Bet aukštumų jis nepanaikino. Tačiau Asos širdis buvo tobula prieš Viešpatį per visas jo dienas. 
\par 15 Jis atnešė į Viešpaties namus savo tėvo ir savo paskirtas dovanas: sidabro, aukso ir indų. 
\par 16 Karas tarp Asos ir Izraelio karaliaus Baašos tęsėsi per visas jų dienas. 
\par 17 Izraelio karalius Baaša išėjo prieš Judą ir statė Ramą, kad niekam neleistų įeiti ar išeiti iš Asos, Judo karaliaus. 
\par 18 Tuomet Asa, paėmęs visą sidabrą ir auksą, likusį Viešpaties namų ir karaliaus namų ižde, pasiuntė per savo tarnus į Damaską Sirijos karaliui Ben Hadadui, Hezjono sūnaus Tabrimono sūnui, sakydamas: 
\par 19 “Padarykime sąjungą tarp manęs ir tavęs, kaip buvo tarp mūsų tėvų. Siunčiu tau dovanų sidabro ir aukso ir prašau: sulaužyk sąjungą su Izraelio karaliumi Baaša, kad jis atsitrauktų nuo manęs”. 
\par 20 Ben Hadadas paklausė karaliaus Asos ir pasiuntė savo kariuomenės vadus prieš Izraelio miestus, ir užėmė Ijoną, Daną, Abel Bet Maachą, visą Kinerotą ir Naftalio kraštą. 
\par 21 Baasa, tai išgirdęs, liovėsi statyti Ramą ir sugrįžo į Tircą. 
\par 22 Karalius Asa sušaukė visą Judą, nieko neaplenkdamas, ir jie paėmė Ramos akmenis bei rąstus, kuriuos buvo pastatęs Baaša, ir jais karalius Asa sutvirtino Benjamino Gebą bei Micpą. 
\par 23 Visi kiti Asos darbai, jo galybė, miestai, kuriuos jis pastatė, yra surašyti Judo karalių metraščių knygoje. Senatvėje jo kojos buvo nesveikos. 
\par 24 Asa užmigo prie savo tėvų ir buvo palaidotas prie savo tėvų Dovydo mieste. Jo sūnus Juozapatas pradėjo karaliauti jo vietoje. 
\par 25 Jeroboamo sūnus Nadabas pradėjo karaliauti Izraelyje antraisiais karaliaus Asos valdymo metais ir karaliavo dvejus metus. 
\par 26 Jis darė pikta Viešpaties akyse, vaikščiojo savo tėvo keliais ir jo nuodėmėje, į kurią tas įtraukė Izraelį. 
\par 27 Ahijos sūnus Baaša iš Isacharo giminės surengė sąmokslą prieš Nadabą ir jį nužudė Gibetone, kuris priklausė filistinams, tuo metu, kai Nadabas su visa kariuomene buvo apgulęs Gibetoną. 
\par 28 Tai įvyko trečiaisiais Judo karaliaus Asos valdymo metais, ir tuomet Baaša užėmė Nadabo sostą. 
\par 29 Tapęs karaliumi, jis išžudė visą Jeroboamo giminę ir nepaliko gyvo nė vieno pagal Viešpaties žodžius, kurie buvo paskelbti per Jo tarną Ahiją iš Šilojo. 
\par 30 Tai įvyko dėl Jeroboamo nuodėmių, kuriomis jis nusidėjo ir į kurias įtraukė Izraelį, sukeldamas Viešpaties, Izraelio Dievo, rūstybę. 
\par 31 Visi kiti Nadabo darbai yra surašyti Izraelio karalių metraščių knygoje. 
\par 32 Karas tarp Asos ir Izraelio karaliaus Baašos truko per visas jų dienas. 
\par 33 Trečiaisiais Judo karaliaus Asos valdymo metais Ahijos sūnus Baaša pradėjo karaliauti Izraeliui Tircoje ir karaliavo dvidešimt ketverius metus. 
\par 34 Jis darė pikta Viešpaties akyse ir vaikščiojo Jeroboamo keliais ir jo nuodėmėje, į kurią tas įtraukė Izraelį.



\chapter{16}


\par 1 Viešpats kalbėjo Hananio sūnui Jehuvui apie Baašą: 
\par 2 “Aš tave pakėliau iš dulkių, išaukštinau ir padariau kunigaikščiu Izraelio tautai, o tu vaikščiojai Jeroboamo keliu ir įtraukei į nuodėmę mano tautą Izraelį, sukeldamas mano pyktį jų nusikaltimais. 
\par 3 Aš atimsiu Baašos palikuonis ir jo namų palikuonis; su jo namais padarysiu taip, kaip padariau su Nebato sūnaus Jeroboamo namais. 
\par 4 Kas iš Baašos mirs mieste, tą suės šunys, o kas mirs lauke, tą les padangių paukščiai”. 
\par 5 Visi kiti Baašos darbai, jo veikla ir jo galybė yra aprašyti Izraelio karalių metraščių knygoje. 
\par 6 Baaša užmigo prie savo tėvų ir buvo palaidotas Tircoje, o jo vietoje pradėjo karaliauti jo sūnus Ela. 
\par 7 Viešpats kalbėjo per pranašą Jehuvą, Hananio sūnų, prieš Baašą ir jo namus už tai, kad jis darė pikta Viešpaties akivaizdoje, sukeldamas Jo pyktį savo darbais, kaip Jeroboamo namai, ir dėl to, kad išžudė juos. 
\par 8 Dvidešimt šeštaisiais Judo karaliaus Asos metais Baašos sūnus Ela tapo Izraelio karaliumi Tircoje ir valdė dvejus metus. 
\par 9 Zimris, pusės kovos vežimų viršininkas, sukėlė maištą prieš jį. Elai puotaujant Tircoje pas Arcą, kuris buvo karaliaus rūmų prievaizdas, 
\par 10 Zimris įėjo ir jį nužudė dvidešimt septintais Judo karaliaus Asos metais ir karaliavo jo vietoje. 
\par 11 Tapęs karaliumi, Zimris išžudė visus Baašos namus ir nepaliko gyvo nei vieno vyro, nei giminaičių, nei draugų. 
\par 12 Taip Zimris sunaikino Baašos namus pagal Viešpaties žodį, kurį Jis kalbėjo prieš Baašą per pranašą Jehuvą. 
\par 13 Visa tai įvyko dėl visų Baašos ir jo sūnaus Elos nuodėmių, kurias jie darė ir į kurias įtraukė Izraelį, sukeldami Viešpaties, Izraelio Dievo, pyktį savo tuštybėmis. 
\par 14 Visi kiti Elos darbai ir veikla yra surašyta Izraelio karalių metraščių knygoje. 
\par 15 Dvidešimt septintaisiais Judo karaliaus Asos metais Zimris karaliavo Izraelyje, Tircoje, septynias dienas. Žmonės tuo laiku buvo apsupę Gibetoną, priklausantį filistinams. 
\par 16 Sužinoję, kad Zimris sukėlė maištą ir nužudė karalių Elą, jie tą pačią dieną išrinko Izraelio karaliumi kariuomenės vadą Omrį. 
\par 17 Omris su visu Izraeliu ėjo iš Gibetono ir apgulė Tircą. 
\par 18 Zimris, matydamas, kad miestas paimtas, nuėjo į karaliaus namus, juos padegė ir žuvo liepsnose 
\par 19 dėl savo nuodėmių, kuriomis nusidėjo, darydamas pikta Viešpaties akivaizdoje ir vaikščiodamas Jeroboamo keliais ir jo nuodėmėje, kurią šis padarė, įtraukdamas Izraelį į nuodėmę. 
\par 20 Visi kiti Zimrio darbai ir jo sąmokslas prieš karalių yra surašyti Izraelio karalių metraščių knygoje. 
\par 21 Tada Izraelio tauta suskilo: pusė tautos palaikė Ginato sūnų Tibnį, norėdami paskelbti jį karaliumi, o kita pusė­Omrį. 
\par 22 Žmonės, palaikę Omrį, laimėjo prieš tuos, kurie sekė Tibnį, Ginato sūnų. Tibnis mirė, o Omris tapo karaliumi. 
\par 23 Trisdešimt pirmaisiais Judo karaliaus Asos metais Omris pradėjo valdyti Izraelį ir karaliavo dvylika metų. Tircoje jis karaliavo šešerius metus. 
\par 24 Jis nupirko iš Šemero Samarijos kalną už du talentus sidabro, pastatė miestą, jį sutvirtino ir pavadino Samarija, buvusio savininko Šemero vardu. 
\par 25 Omris darė pikta Viešpaties akivaizdoje, elgdamasis blogiau už visus savo pirmtakus. 
\par 26 Jis vaikščiojo Nebato sūnaus Jeroboamo keliais ir jo nuodėmėje, į kurią tas įtraukė Izraelį, kad sukeltų Viešpaties, Izraelio Dievo, pyktį savo tuštybėmis. 
\par 27 Visi kiti Omrio darbai ir jo galia, kurią jis parodė, yra surašyti Izraelio karalių metraščių knygoje. 
\par 28 Karalius Omris užmigo prie savo tėvų ir buvo palaidotas Samarijoje. Jo vietoje pradėjo karaliauti jo sūnus Ahabas. 
\par 29 Omrio sūnus Ahabas pradėjo karaliauti Izraelyje trisdešimt aštuntaisiais Judo karaliaus Asos metais ir karaliavo Samarijoje dvidešimt dvejus metus. 
\par 30 Ahabas, Omrio sūnus, darė pikta Viešpaties akivaizdoje labiau už visus savo pirmtakus. 
\par 31 Negana to, kad jis vaikščiojo Nebato sūnaus Jeroboamo nuodėmėje, jis dar vedė sidoniečių karaliaus Etbaalo dukterį Jezabelę ir tarnavo Baalui bei garbino jį. 
\par 32 Jis pastatė Samarijoje Baalo namus ir juose aukurą Baalui. 
\par 33 Ahabas dar įrengė alką. Ahabas labiau negu visi prieš jį buvę Izraelio karaliai darė tai, kas sukėlė Viešpaties pyktį. 
\par 34 Ahabui valdant, Hielis iš Betelio atstatė Jerichą. Jis padėjo jo pamatus ant savo pirmagimio Abiramo ir įstatė vartus ant jauniausiojo sūnaus Segubo, kaip buvo paskelbęs Viešpats per Nūno sūnų Jozuę.



\chapter{17}

\par 1 Elijas iš Gileado Tišbos tarė Ahabui: “Kaip gyvas Viešpats, Izraelio Dievas, kuriam aš tarnauju, ateinančiais metais nebus nei rasos, nei lietaus, nebent man paliepus”. 
\par 2 Viešpats kalbėjo jam, sakydamas: 
\par 3 “Eik iš čia ir pasislėpk prie Kerito upelio priešais Jordaną. 
\par 4 Gerk iš upelio, o varnams Aš įsakiau aprūpinti tave maistu”. 
\par 5 Jis nuėjo ir darė, ką Viešpats buvo jam įsakęs. Jis apsistojo prie Kerito upelio priešais Jordaną. 
\par 6 Varnai atnešdavo jam duonos ir mėsos kas rytą ir vakarą, o iš upelio jis atsigerdavo. 
\par 7 Po kurio laiko upelis išdžiūvo, nes krašte nebuvo lietaus. 
\par 8 Viešpats kalbėjo jam, sakydamas: 
\par 9 “Eik į Sareptą Sidono krašte ir ten pasilik. Aš įsakiau vienai našlei aprūpinti tave”. 
\par 10 Jis nuėjo į Sareptą. Prie miesto vartų jis pamatė našlę, rankiojančią malkas. Elijas kreipėsi į ją: “Atnešk man truputį vandens atsigerti”. 
\par 11 Jai einant, jis dar šūktelėjo: “Atnešk ir duonos kąsnelį!” 
\par 12 Ji atsakė: “Kaip gyvas Viešpats, tavo Dievas, nieko neturiu, tik saują miltų statinaitėje ir truputį aliejaus puodelyje. Štai renku truputį malkų. Parėjusi paruošiu sau ir savo sūnui valgį ir, suvalgę jį, numirsime”. 
\par 13 Elijas jai atsakė: “Nebijok! Parėjus padaryk, kaip sakei, tik iškepk man pirma mažą paplotėlį ir atnešk jį man, o sau ir sūnui paskui padarysi. 
\par 14 Nes Viešpats, Izraelio Dievas, sako: ‘Miltai statinaitėje nesibaigs ir puodelyje aliejaus nesumažės iki tos dienos, kol Viešpats žemei duos lietaus’ ”. 
\par 15 Parėjusi namo, ji padarė, kaip Elijas sakė. Ir valgė ji, jis ir jos namai kasdien. 
\par 16 Statinaitėje miltai nesibaigė ir aliejaus puodelyje nesumažėjo, kaip Viešpats pasakė per Eliją. 
\par 17 Po to susirgo tos moters, šeimininkės, sūnus. Jo liga buvo tokia sunki, kad jis liovėsi kvėpavęs. 
\par 18 Tuomet ji tarė Elijui: “Kas man ir tau, Dievo vyre? Ar tu atėjai priminti mano kaltes ir numarinti mano sūnų?” 
\par 19 Jis jai atsakė: “Duok man savo sūnų”. Paėmęs jį iš jos, užnešė į aukštutinį kambarį, kuriame gyveno, ir paguldė savo lovoje. 
\par 20 Jis šaukėsi Viešpaties, sakydamas: “Viešpatie, mano Dieve, argi našlei, pas kurią aš gyvenu, Tu siųsi nelaimę, numarindamas jos sūnų?” 
\par 21 Po to jis tris kartus išsitiesė ant vaiko ir meldėsi: “Viešpatie, mano Dieve, meldžiu, tesugrįžta šio vaiko siela pas jį”. 
\par 22 Viešpats išklausė Elijo maldą, vaiko siela sugrįžo pas jį, ir jis atgijo. 
\par 23 Elijas, paėmęs vaiką, jį nuvedė iš aukštutinio kambario žemyn ir, atiduodamas jo motinai, tarė: “Žiūrėk, tavo sūnus gyvas!” 
\par 24 Moteris atsakė Elijui: “Dabar žinau, kad tu esi Dievo vyras ir kad Viešpaties žodis tavo lūpose yra tiesa”.



\chapter{18}

\par 1 Trečiaisiais metais Viešpats vėl kalbėjo Elijui: “Eik ir pasirodyk Ahabui; Aš duosiu lietaus žemei”. 
\par 2 Elijas išėjo pas Ahabą. Samarijoje buvo didelis badas. 
\par 3 Ahabas pasišaukė Abdiją, rūmų valdytoją, kuris buvo labai dievobaimingas žmogus. 
\par 4 Jezabelei naikinant Viešpaties pranašus, Abdijas paslėpė šimtą pranašų, po penkiasdešimt vienoje oloje, ir aprūpino juos duona bei vandeniu. 
\par 5 Ahabas tarė rūmų valdytojui Abdijui: “Eikime per kraštą ir apžiūrėkime vandens šaltinius ir upelius: gal rasime žolės žirgams ir mulams, kad neprarastume visų gyvulių”. 
\par 6 Jie pasidalino kraštą, kur kiekvienas eis. Ahabas ėjo vienu keliu, o Abdijas kitu. 
\par 7 Abdijas keliaudamas netikėtai sutiko Eliją. Jį atpažinęs, jis puolė ant kelių ir tarė: “Ar tai tu, Elijau, mano viešpatie?” 
\par 8 Tas jam atsakė: “Aš. Eik ir pasakyk savo valdovui: ‘Elijas čia’ ”. 
\par 9 Jis sakė: “Kuo nusidėjau, kad tu atiduodi savo tarną Ahabui nužudyti? 
\par 10 Kaip gyvas Viešpats, tavo Dievas, nėra tautos nė karalystės, kur mano valdovas nebūtų siuntęs tavęs ieškoti. O atsakius: ‘Jo čia nėra’, jis prisaikdindavo tą karalystę ar tautą, kad tu nerastas. 
\par 11 O dabar tu sakai: ‘Eik ir sakyk savo valdovui, kad Elijas čia’. 
\par 12 Man nuėjus, Viešpaties Dvasia nuneš tave į man nežinomą vietą. Kai aš kalbėsiu Ahabui ir jis tavęs neras, jis nužudys mane, nors tavo tarnas bijo Viešpaties nuo pat savo jaunystės. 
\par 13 Argi nebuvo pranešta mano viešpačiui, ką aš, Jezabelei žudant Dievo pranašus, padariau, kaip paslėpiau šimtą Viešpaties pranašų, po penkiasdešimt vienoje oloje, ir juos aprūpinau duona bei vandeniu? 
\par 14 O dabar tu sakai: ‘Eik ir sakyk savo valdovui, kad Elijas čia’. Juk jis mane nužudys”. 
\par 15 Elijas atsakė: “Kaip gyvas kareivijų Viešpats, kuriam aš tarnauju, šiandien Ahabui pasirodysiu”. 
\par 16 Abdijas nuėjo pasitikti Ahabo ir pasakė jam. Tada Ahabas nuėjo sutikti Eliją. 
\par 17 Ahabas, pamatęs Eliją, tarė: “Ar tai tu, kuris vargini Izraelį?” 
\par 18 Elijas atsakė: “Ne aš varginu Izraelį, bet tu ir tavo tėvo namai, nes apleidote Viešpaties įsakymus ir sekate Baalą 
\par 19 Dabar surink prie manęs visą Izraelį ant Karmelio kalno, taip pat keturis šimtus penkiasdešimt Baalo ir keturis šimtus alkų pranašų, valgančių nuo Jezabelės stalo”. 
\par 20 Ahabas surinko visus izraelitus ir pranašus ant Karmelio kalno. 
\par 21 Elijas, atsistojęs prieš juos, tarė: “Ar ilgai jūs svyruosite? Jei Viešpats yra Dievas, sekite Jį, o jei Baalas, sekite Baalą”. Tauta neatsakė nė žodžio. 
\par 22 Elijas sakė tautai: “Aš likau vienas Viešpaties pranašas, o Baalo pranašų yra keturi šimtai penkiasdešimt. 
\par 23 Duokite mums du jaučius; jie tepasirenka vieną jautį ir, supjaustę jį į gabalus, tepadeda ant malkų, bet neuždega, o aš paruošiu kitą jautį ir uždėsiu ant malkų, bet neuždegsiu. 
\par 24 Tegul jie šaukiasi savo dievų, o aš šauksiuosi Viešpaties vardo. Tas dievas, kuris atsakys ugnimi, yra Dievas”. Žmonės atsakė: “Gerai pasakyta”. 
\par 25 Elijas tarė Baalo pranašams: “Pasirinkite vieną jautį ir jį paruoškite pirma, nes jūsų yra daug; šaukitės savo dievų vardo, bet neuždekite ugnies”. 
\par 26 Jie paruošė jautį ir šaukėsi Baalo nuo ryto iki vidudienio: “Baalai, išklausyk mus!” Bet atsakymo nebuvo. Jie šokinėjo aplink aukurą, kurį buvo pasidarę. 
\par 27 Vidudienį Elijas, tyčiodamasis iš jų, tarė: “Šaukite garsiau! Juk jis dievas! Gal jis šnekasi, užsiėmęs ar kelionėje? Gal jis miega ir jį reikia pažadinti?” 
\par 28 Jie šaukė garsiai ir, kaip buvo įpratę, raižėsi peiliais iki kraujo. 
\par 29 Praėjus vidudieniui, jie vis dar šaukė iki aukojimo laiko, bet nei balso, nei atsakymo nesulaukė. 
\par 30 Tada Elijas tarė žmonėms: “Ateikite prie manęs!” Žmonėms priartėjus, jis atstatė Viešpaties aukurą, kuris buvo sugriautas. 
\par 31 Elijas ėmė dvylika akmenų, pagal Jokūbo, kuriam Viešpats buvo sakęs: “Izraelis bus tavo vardas”, sūnų giminių skaičių. 
\par 32 Jis pastatė aukurą iš akmenų Viešpaties vardui ir iškasė aplink jį griovį, į kurį tilptų du saikai sėklos. 
\par 33 Sudėjęs tvarkingai malkas, jis supjaustė jautį į gabalus, uždėjo juos ant malkų ir tarė: “Pripildykite keturis kibirus vandens ir užpilkite ant deginamosios aukos ir ant malkų”. 
\par 34 Po to jis tarė: “Pakartokite”. Jiems pakartojus, jis vėl tarė: “Darykite tai trečią kartą”. Jie padarė ir trečią kartą. 
\par 35 Vanduo tekėjo nuo aukuro ir pripildė griovį. 
\par 36 Vakarinės aukos metu pranašas Elijas priėjęs meldėsi: “Viešpatie, Abraomo, Izaoko ir Izraelio Dieve, tebūna šiandien žinoma, kad Tu esi Dievas Izraelyje, o aš Tavo tarnas, ir kad visa tai Tavo paliepimu padariau. 
\par 37 Išklausyk mane, Viešpatie! Išklausyk mane, kad šita tauta žinotų, jog Tu, Viešpatie, esi Dievas, ir gręžtųsi savo širdimis į Tave”. 
\par 38 Tada Viešpaties ugnis krito ant aukuro ir sudegino auką, malkas, akmenis bei dulkes, ir sulaižė griovyje buvusį vandenį. 
\par 39 Visi žmonės, tai matydami, puolė veidais į žemę ir šaukė: “Viešpats yra Dievas! Viešpats yra Dievas!” 
\par 40 Elijas jiems tarė: “Suimkite Baalo pranašus! Nepaleiskite nė vieno!” Jiems juos suėmus, Elijas nuvedė juos prie Kišono upelio ir ten nužudė. 
\par 41 Po to Elijas tarė Ahabui: “Valgyk ir gerk, nes jau girdžiu lietaus šniokštimą”. 
\par 42 Ahabas nuėjo valgyti ir gerti, o Elijas, užlipęs ant Karmelio viršūnės, pasilenkė ir paslėpė savo veidą tarp kelių. 
\par 43 Jis tarė savo tarnui: “Eik ir pažiūrėk link jūros”. Tas nuėjęs ir pažiūrėjęs sakė: “Nieko nėra!” Elijas tarė: “Padaryk tai septynis kartus”. 
\par 44 Septintąjį kartą tarnas sakė: “Debesėlis vyro plaštakos pločio kyla iš jūros”. Elijas atsakė: “Eik ir sakyk Ahabui: ‘Pasikinkyk arklius ir skubėk, kad tavęs neužkluptų lietus’ ”. 
\par 45 Tuo tarpu dangus apsiniaukė ir užėjo smarkus lietus. Ahabas nuskubėjo į Jezreelį. 
\par 46 Viešpaties ranka buvo ant Elijo; jis susijuosė strėnas ir bėgo Ahabo priekyje į Jezreelį.



\chapter{19}

\par 1 Ahabas papasakojo Jezabelei visa, ką padarė Elijas ir kaip jis išžudė visus Baalo pranašus. 
\par 2 Jezabelė siuntė pasiuntinį pas Eliją, sakydama: “Tegul dievai man padaro tai ir dar daugiau, jei rytoj apie šitą laiką aš nepadarysiu tau taip, kaip tu padarei Baalo pranašams”. 
\par 3 Tai pamatęs, jis pakilo ir išėjo, kad išgelbėtų savo gyvybę. Atėjęs į Judo Beer Šebą ir palikęs ten savo tarną, 
\par 4 pats ėjo į dykumą visą dieną. Atėjęs jis atsisėdo po vienu kadagiu ir meldė sau mirties: “Viešpatie, gana, pasiimk mano gyvenimą, aš nesu geresnis už savo tėvus”. 
\par 5 Jis atsigulė ir užmigo po kadagiu. Angelas palietė jį ir tarė: “Kelkis ir valgyk”. 
\par 6 Elijas pažiūrėjo ir pamatė galvūgalyje paplotį ir indą su vandeniu. Pavalgęs ir atsigėręs jis vėl atsigulė. 
\par 7 Viešpaties angelas atėjo antrą kartą ir, jį palietęs, tarė: “Kelkis ir valgyk, nes tavęs laukia ilgas kelias”. 
\par 8 Pavalgęs ir pasistiprinęs tuo maistu, jis ėjo keturiasdešimt parų iki Dievo kalno Horebo. 
\par 9 Jam apsinakvojus vienoje oloje, Viešpats jam tarė: “Ką čia veiki, Elijau?” 
\par 10 Jis atsakė: “Aš buvau labai uolus dėl Viešpaties, kareivijų Dievo, nes izraelitai sulaužė Tavo sandorą, išgriovė Tavo aukurus ir išžudė Tavo pranašus. Aš vienas likau, ir jie ieško mano gyvybės”. 
\par 11 Viešpats tarė: “Išeik ir atsistok ant kalno prieš Viešpatį”. Viešpats praėjo, ir didelė bei smarki audra, ardanti kalnus ir trupinanti uolas, buvo priešais Viešpatį. Bet audroje nebuvo Viešpaties. Po audros drebėjo žemė, bet Viešpaties nebuvo žemės drebėjime. 
\par 12 Žemės drebėjimui praėjus, pakilo liepsnos, bet ir liepsnose Viešpaties nebuvo. Tada pasigirdo tylus ramus balsas. 
\par 13 Elijas, jį išgirdęs, apsigaubė veidą apsiaustu ir išėjęs atsistojo olos angoje. Pasigirdo balsas: “Ką čia veiki, Elijau?” 
\par 14 Jis atsakė: “Aš buvau labai uolus dėl Viešpaties, kareivijų Dievo, nes izraelitai sulaužė Tavo sandorą, išgriovė Tavo aukurus ir išžudė Tavo pranašus. Aš vienas likau, ir jie ieško mano gyvybės”. 
\par 15 Viešpats jam tarė: “Grįžk savo keliu per dykumą į Damaską ir nuvykęs ten patepk Hazaelį Sirijos karaliumi, 
\par 16 Nimšio sūnų Jehuvą­Izraelio karaliumi ir Šafato sūnų Eliziejų iš Abel Meholos patepk pranašu savo vieton. 
\par 17 Kas paspruks nuo Hazaelio kardo, tą nužudys Jehuvas, o kas paspruks nuo Jehuvo kardo, tą nužudys Eliziejus. 
\par 18 Tačiau Aš pasilikau Izraelyje septynis tūkstančius, kurie nesulenkė kelių prieš Baalą ir nebučiavo jo”. 
\par 19 Elijas, išėjęs iš ten, surado Šafato sūnų Eliziejų ariantį lauką. Dvylika jungų jaučių ėjo pirma jo, jis pats arė su dvyliktuoju. Elijas, praeidamas pro jį, užmetė ant jo savo apsiaustą. 
\par 20 Tas, palikęs jaučius ir bėgdamas paskui Eliją, tarė: “Leisk man pabučiuoti savo tėvą ir motiną, tuomet aš seksiu tave”. Elijas jam atsakė: “Eik ir sugrįžk, nes ką aš tau padariau?” 
\par 21 Jis sugrįžo, papjovė jungo jaučius ir, išviręs juos ant pakinktų medžių, mėsą išdalino žmonėms, ir jie valgė. Po to jis sekė Eliją ir jam tarnavo.



\chapter{20}

\par 1 Sirijos karalius Ben Hadadas surinko visą savo kariuomenę: su juo buvo trisdešimt du karaliai, žirgai ir kovos vežimai, ir puolė Samariją. 
\par 2 Ben Hadadas per pasiuntinius pranešė Izraelio karaliui Ahabui: 
\par 3 “Tavo sidabras ir auksas yra mano, taip pat tavo žmonos ir geriausieji vaikai”. 
\par 4 Izraelio karalius atsakė: “Mano valdove karaliau, kaip tu sakei, aš ir visa, kas man priklauso, esame tavo”. 
\par 5 Pasiuntiniai vėl atėjo pas Ahabą ir sakė: “Taip sako Ben Hadadas: ‘Nors aš sakiau tau, kad tu atiduosi man savo sidabrą, auksą, žmonas ir vaikus, 
\par 6 bet rytoj šituo laiku atsiųsiu pas tave savo tarnus, jie iškratys tavo bei tavo tarnų namus ir visa, kas brangu tavo akyse, surinkę išsigabens’ ”. 
\par 7 Tada Izraelio karalius, sušaukęs visus krašto vyresniuosius, jiems kalbėjo: “Klausykite ir stebėkite, kaip Ben Hadadas siekia pikto. Jis atsiuntė pas mane pasiuntinius, reikalaudamas mano žmonų, vaikų, sidabro ir aukso, ir aš jam neprieštaravau”. 
\par 8 Visi vyresnieji ir tauta jam atsakė: “Neklausyk ir nesutik!” 
\par 9 Ahabas atsakė Ben Hadado pasiuntiniams: “Sakykite mano valdovui karaliui, kad aš visa, ko jis reikalavo pradžioje, darysiu, bet paskutinio jo reikalavimo nevykdysiu”. Pasiuntiniai viską pranešė Ben Hadadui. 
\par 10 Karalius vėl siuntė pas Ahabą pasiuntinius su tokiu pranešimu: “Tegul dievai padaro man tai ir dar daugiau, jei Samarijos dulkių bent po saują užtektų žmonėms, kurie seka mane”. 
\par 11 Izraelio karalius atsakė: “Tenesigiria tas, kuris susijuosia, tarsi jau nusijuostų”. 
\par 12 Kai išgirdo šitą atsakymą, Ben Hadadas puotavo su karaliais palapinėje. Jis įsakė savo tarnams pasiruošti, ir tie išsirikiavo kovai prieš miestą. 
\par 13 Pranašas priėjo prie Izraelio karaliaus Ahabo ir tarė: “Taip sako Viešpats: ‘Ar matai visą šitą daugybę? Aš ją šiandien atiduosiu į tavo rankas, kad žinotum, jog Aš esu Viešpats’ ”. 
\par 14 Ahabas klausė: “Per ką?” Pranašas atsakė: “Taip sako Viešpats: ‘Per sričių kunigaikščių jaunuolius’ ”. Karalius vėl klausė: “Kas turi pradėti mūšį?” Jis atsakė: “Tu”. 
\par 15 Sričių kunigaikščių jaunuolių buvo du šimtai trisdešimt du, o iš viso izraelitų­septyni tūkstančiai. 
\par 16 Jis išėjo vidudienį; Ben Hadadas buvo nusigėręs, gerdamas palapinėje su trisdešimt dviem karaliais, jo sąjungininkais. 
\par 17 Sričių kunigaikščių jaunuoliai išėjo pirmieji. Ben Hadadas pasiuntė žvalgus, kurie jam pranešė: “Vyrai išėjo iš Samarijos”. 
\par 18 Jis įsakė: “Jei jie eina taikos prašyti, suimkite juos gyvus, o jei eina kariauti, taip pat suimkite juos gyvus”. 
\par 19 Sričių kunigaikščių jaunuoliai išėjo iš miesto, o kariuomenė sekė juos. 
\par 20 Jie žudė kiekvieną, kuris išėjo prieš juos, ir sirai bėgo, o izraelitai juos vijosi. Sirijos karalius Ben Hadadas pabėgo ant žirgo su raiteliais. 
\par 21 Izraelio karalius išėjo ir paėmė kovos vežimų ir žirgų, ir smarkiai sumušė sirus. 
\par 22 Tada pranašas, atėjęs pas Izraelio karalių, jam tarė: “Eik, sustiprėk ir žiūrėk, ką darai, nes po metų Sirijos karalius vėl puls tave”. 
\par 23 Sirijos karaliaus tarnai kalbėjo karaliui: “Jų dievai yra kalnų dievai, todėl jie mus nugalėjo. Bet jei kariautume su jais lygumoje, mes juos nugalėtume. 
\par 24 Padaryk šitaip: pašalink visus karalius iš jų vietų ir jų vieton paskirk vadus. 
\par 25 Surink kariuomenę, kokią praradai, taip pat atstatyk žirgų ir kovos vežimų skaičių. Tada kariausime su jais lygumoje ir tikrai juos nugalėsime”. Karalius paklausė jų patarimo ir taip padarė. 
\par 26 Metams praėjus, Ben Hadadas su savo kariuomene atėjo į Afeką kariauti su Izraeliu. 
\par 27 Izraelitai susirinko ir išėjo prieš juos. Izraelitai sustojo priešais juos lyg dvi mažos ožkų kaimenės, o sirų buvo pilnas kraštas. 
\par 28 Tuomet Dievo vyras, priėjęs prie Izraelio karaliaus, tarė: “Taip sako Viešpats: ‘Kadangi sirai sakė, kad Viešpats yra kalnų Dievas, bet ne slėnių Dievas, tai Aš atiduosiu visą šitą daugybę į tavo rankas, kad jūs žinotumėte, jog Aš esu Viešpats’ ”. 
\par 29 Taip jie stovėjo pasiruošę vieni prieš kitus septynias dienas, ir septintą dieną prasidėjo kova. Izraelitai nužudė tą dieną šimtą tūkstančių sirų pėstininkų. 
\par 30 Likusieji pabėgo į Afeko miestą, kur ant likusių dvidešimt septynių tūkstančių užgriuvo mūras. Ben Hadadas irgi pabėgo į miestą ir pasislėpė vidiniame kambaryje. 
\par 31 Jo tarnai kalbėjo jam: “Mes girdėjome, kad Izraelio karaliai yra gailestingi. Apsijuoskime ašutinėmis, užsidėkime virves ant kaklo ir išeikime pas Izraelio karalių. Gal jis paliks tave gyvą”. 
\par 32 Jie, apsijuosę ašutinėmis, užsidėjo virves ir, atėję pas Izraelio karalių, tarė: “Tavo tarnas Ben Hadadas prašo: ‘Palik mane gyvą’ ”. Ahabas atsakė: “Jei jis dar gyvas, jis yra mano brolis”. 
\par 33 Vyrai tai palaikė geru ženklu ir pakartojo: “Ben Hadadas yra tavo brolis”. Karalius tarė: “Atveskite jį!” Ben Hadadui atėjus, Ahabas pasisodino jį į savo vežimą. 
\par 34 Ben Hadadas tarė jam: “Miestus, kuriuos mano tėvas atėmė iš tavo tėvo, grąžinu; tu įsteik prekyviečių Damaske, kaip mano tėvas buvo įsteigęs Samarijoje”. Ahabas atsakė: “Šitomis sąlygomis aš tave paleidžiu”. Juodu sudarė sutartį, ir Ahabas paleido Sirijos karalių. 
\par 35 Vienas iš pranašų, Viešpačiui paliepus, tarė savo draugui: “Mušk mane”. Bet tas atsisakė jį mušti. 
\par 36 Tada jis jam sakė: “Kadangi nepaklausei Viešpaties žodžio, kai pasitrauksi nuo manęs, tave nužudys liūtas”. Kai jis pasitraukė, jį sutiko liūtas ir sudraskė. 
\par 37 Sutikęs kitą vyrą, jis vėl sakė: “Mušk mane”. Tas mušė jį ir sužeidė. 
\par 38 Po to pranašas nuėjęs atsistojo prie kelio ir laukė karaliaus. Jis užsirišo raištį ant akių. 
\par 39 Karaliui praeinant, jis sušuko: “Tavo tarnas dalyvavo mūšyje. Vienas žmogus atvedė pas mane vyrą ir įsakė: ‘Saugok jį! Jei jis pabėgs, tu atsakysi savo gyvybe arba turėsi sumokėti talentą sidabro’. 
\par 40 Kai tavo tarnas buvo užsiėmęs šiuo bei tuo, jis pabėgo”. Izraelio karalius jam tarė: “Teismo sprendimą tu pats sau pasakei”. 
\par 41 Jis skubiai nusiėmė raištį, ir Izraelio karalius atpažino, kad jis pranašas. 
\par 42 Pranašas jam tarė: “Taip sako Viešpats: ‘Kadangi tu paleidai vyrą, kurį Aš buvau paskyręs visiškam sunaikinimui, tai tavo gyvybė bus už jo gyvybę ir tavo tauta už jo tautą’ ”. 
\par 43 Izraelio karalius vyko namo į Samariją paniuręs ir nepatenkintas.



\chapter{21}

\par 1 Jezreelietis Nabotas turėjo vynuogyną šalia Samarijos karaliaus Ahabo rūmų. 
\par 2 Ahabas kalbėjo Nabotui: “Duok man savo vynuogyną. Aš pasidarysiu daržą iš jo, nes jis yra arti mano namų, o aš tau duosiu vietoje jo geresnį vynuogyną, arba jei tau geriau, sumokėsiu, kiek jis vertas”. 
\par 3 Nabotas atsakė karaliui: “Taip nebus, kad atiduočiau savo tėvų palikimą”. 
\par 4 Ahabas sugrįžo į namus paniuręs ir nepatenkintas dėl jezreeliečio Naboto atsakymo: “Neatiduosiu savo tėvų palikimo”. Jis atsigulė į lovą, nusisuko ir nevalgė. 
\par 5 Jo žmona Jezabelė, atėjusi pas jį, klausė: “Kodėl tavo dvasia taip nuliūdusi, kad nevalgai?” 
\par 6 Jis jai atsakė: “Aš prašiau jezreeliečio Naboto parduoti man savo vynuogyną arba išmainyti į kitą vynuogyną, bet jis nesutiko”. 
\par 7 Jo žmona Jezabelė tarė jam: “Ar ne tu valdai Izraelį? Kelkis, valgyk ir būk geros nuotaikos! Aš atiduosiu tau jezreeliečio Naboto vynuogyną”. 
\par 8 Ji parašė Ahabo vardu laiškų, juos užantspaudavo jo antspaudu ir pasiuntė miesto vyresniesiems bei didžiūnams, kurie gyveno su Nabotu. 
\par 9 Laiškuose ji rašė: “Paskelbkite pasninką ir pasodinkite Nabotą į pirmą vietą tarp žmonių, 
\par 10 o prieš jį pasodinkite du vyrus, Belialo vaikus, kurie paliudytų, kad Nabotas keikė Dievą ir karalių. Po to išveskite jį ir užmuškite akmenimis”. 
\par 11 Jo miesto vyrai, vyresnieji ir didžiūnai, gyvenantieji mieste, padarė taip, kaip buvo parašyta Jezabelės laiškuose, kuriuos ji jiems pasiuntė. 
\par 12 Jie paskelbė pasninką ir pasodino Nabotą į pirmąją vietą tarp žmonių. 
\par 13 Po to atėjo du vyrai, Belialo vaikai, ir atsisėdo prieš jį. Tie liudijo prieš Nabotą, visiems žmonėms girdint: “Nabotas keikė Dievą ir karalių”. Po to jie, išvedę jį už miesto, užmušė akmenimis 
\par 14 ir pranešė Jezabelei: “Nabotas užmuštas akmenimis”. 
\par 15 Jezabelė, išgirdusi, kad Nabotas užmuštas akmenimis, tarė Ahabui: “Pasisavink jezreeliečio Naboto vynuogyną, kurį jis atsisakė tau parduoti už pinigus, nes Nabotas jau nebegyvas”. 
\par 16 Ahabas, sužinojęs, kad Nabotas miręs, tuojau išėjo į jezreeliečio Naboto vynuogyną, kad jį pasisavintų. 
\par 17 Tuomet Viešpats kalbėjo Elijui: 
\par 18 “Eik į Samariją, kur sutiksi Izraelio karalių Ahabą. Jį rasi Naboto vynuogyne, kurio jis nuėjo pasisavinti. 
\par 19 Sakyk jam: ‘Taip sako Viešpats: ‘Tu nužudei žmogų ir pagrobei jo nuosavybę. Toje vietoje, kurioje šunys laižė Naboto kraują, šunys laižys ir tavo kraują’ ”. 
\par 20 Ahabas, pamatęs Eliją, tarė: “Tai suradai mane, mano prieše?” Tas atsakė: “Aš tave suradau, nes tu parsidavei, kad darytum pikta Viešpaties akyse. 
\par 21 Todėl Viešpats sako: ‘Aš bausiu tave ir išnaikinsiu visus Ahabo šeimos vyrus, nė vieno nepaliksiu gyvo. 
\par 22 Padarysiu tau, kaip padariau Nebato sūnui Jeroboamui ir Ahijos sūnui Baašai, nes tu įtraukei Izraelį į nuodėmę’. 
\par 23 Ir apie Jezabelę Viešpats taip pat kalbėjo: ‘Šunys ės Jezabelę Jezrahelio mieste. 
\par 24 Kas iš Ahabo mirs mieste, tą ės šunys, kas mirs laukuose, tą les padangių paukščiai’ ”. 
\par 25 Nebuvo kito tokio, kuris būtų taip parsidavęs daryti pikta Viešpaties akyse, kaip Ahabas, nes jį suvedžiojo jo žmona Jezabelė. 
\par 26 Jis bjauriai elgėsi sekdamas stabus, kaip darė amoritai, kuriuos Viešpats išvarė prieš izraelitams įsikuriant. 
\par 27 Ahabas, išgirdęs šituos žodžius, perplėšė savo drabužius ir, užsidėjęs ašutinę, pasninkavo, gulėjo ašutinėje ir vaikščiojo nusiminęs. 
\par 28 Tada Viešpats kalbėjo pranašui Elijui: 
\par 29 “Ar matai, kaip Ahabas nusižemino prieš mane? Kadangi jis taip padarė, Aš nesiųsiu jam nelaimių iki jo mirties, bet jo sūnaus dienomis įvykdysiu bausmę jo namams”.



\chapter{22}


\par 1 Trejus metus nebuvo karo tarp Sirijos ir Izraelio. 
\par 2 Trečiaisiais metais Judo karalius Juozapatas atvyko pas Izraelio karalių. 
\par 3 Izraelio karalius kalbėjo savo tarnams: “Ar žinote, kad Ramot Gileadas mums priklauso, o mes delsiame jį atsiimti iš Sirijos karaliaus?” 
\par 4 Ahabas klausė Juozapato: “Ar eisi su manimi kariauti į Ramot Gileadą?” Juozapatas atsakė Izraelio karaliui: “Kaip tu, taip ir aš, mano tauta­kaip tavoji ir mano žirgai­kaip tavo”. 
\par 5 Juozapatas sakė Izraelio karaliui: “Pasiklausk, ką Viešpats sako”. 
\par 6 Izraelio karalius sušaukė apie keturis šimtus pranašų ir jų klausė: “Ar man eiti kariauti prieš Ramot Gileadą, ar ne?” Jie atsakė: “Eik! Viešpats jį atiduos į karaliaus rankas”. 
\par 7 Juozapatas klausė: “Ar čia nėra Viešpaties pranašo, kad jo galėtume pasiklausti?” 
\par 8 Izraelio karalius atsakė Juozapatui: “Yra vienas­Imlos sūnus Michėjas, per kurį būtų galima paklausti Viešpaties, bet aš jo nekenčiu, nes jis niekad nepranašauja apie mane gerai, tik blogai”. Juozapatas atsakė: “Nekalbėk taip, karaliau”. 
\par 9 Izraelio karalius pasišaukė vieną rūmų valdininką ir įsakė: “Skubiai atvesk Imlos sūnų Michėją!” 
\par 10 Izraelio karalius Ahabas ir Judo karalius Juozapatas sėdėjo savo sostuose apsivilkę karališkais drabužiais aikštėje prie Samarijos vartų, ir visi pranašai pranašavo priešais juos. 
\par 11 Kenaanos sūnus Sedekijas pasidarė geležinius ragus ir sakė: “Taip sako Viešpats: ‘Jais badysi sirus, kol juos pribaigsi’ ”. 
\par 12 Ir visi pranašai taip pranašavo: “Eik į Ramot Gileadą ir laimėk! Viešpats jį atiduos į karaliaus rankas”. 
\par 13 Pasiuntinys, nuėjęs pakviesti Michėjo, jam sakė: “Štai pranašų žodžiai vienbalsiai skelbia gerą žinią karaliui. Tebūna ir tavo žodis panašus į jų, kalbėk tai, kas gera”. 
\par 14 Michėjas atsakė: “Kaip Viešpats gyvas, ką Viešpats man sakys, tą kalbėsiu”. 
\par 15 Jam atėjus, karalius klausė: “Michėjau, ar mums eiti kariauti prieš Ramot Gileadą?” Tas jam atsakė: “Eik ir laimėk! Viešpats jį atiduos į karaliaus rankas”. 
\par 16 Karalius jam tarė: “Kiek kartų reikės tave saikdinti, kad man nieko kito nekalbėtum, tik tiesą Viešpaties vardu?” 
\par 17 Tada Michėjas atsakė: “Mačiau visą Izraelį išsklaidytą kalnuose kaip avis be piemens. O Viešpats tarė: ‘Šitie neturi valdovo, tegul kiekvienas grįžta ramybėje į savo namus’ ”. 
\par 18 Izraelio karalius sakė Juozapatui: “Ar tau nesakiau, kad jis nepranašauja apie mane gera, tik bloga?” 
\par 19 Michėjas tęsė: “Klausyk Viešpaties žodžio. Mačiau Viešpatį, sėdintį savo soste, ir visą dangaus kareiviją, stovinčią Jo dešinėje ir kairėje. 
\par 20 Viešpats klausė: ‘Kas įtikins Ahabą, kad jis eitų ir žūtų Ramot Gileade?’ Vieni sakė taip, kiti­ kitaip. 
\par 21 Pagaliau išėjo dvasia ir atsistojusi Viešpaties akivaizdoje tarė: ‘Aš jį įtikinsiu’. 
\par 22 Viešpats ją klausė: ‘Kaip?’ Ji atsakė: ‘Aš eisiu ir būsiu melo dvasia visų karaliaus Ahabo pranašų lūpose’. Viešpats tarė: ‘Tau pavyks įtikinti. Eik ir daryk tai’. 
\par 23 Taigi Viešpats įdėjo melo dvasią į visų tavo pranašų lūpas, nes Viešpats kalbėjo prieš tave pikta”. 
\par 24 Tada Keenanos sūnus Sedekijas, priėjęs prie Michėjo, trenkė jam į veidą ir tarė: “Kuriuo keliu Viešpaties Dvasia pasitraukė nuo manęs, kad kalbėtų tau?” 
\par 25 Michėjas atsakė: “Tu pamatysi tai tą dieną, kai bėgsi slėptis į vidinį kambarį”. 
\par 26 Izraelio karalius įsakė sulaikyti Michėją, nuvesti jį pas miesto valdytoją Amoną ir pas karaliaus sūnų Jehoašą, 
\par 27 sakydamas: “Laikykite jį kalėjime ir maitinkite sielvarto duona bei vandeniu, kol ramybėje sugrįšiu”. 
\par 28 Michėjas atsakė: “Jei tu sugrįši ramybėje, tai Viešpats nekalbėjo per mane”. Ir jis sakė: “Klausykite, visi žmonės!” 
\par 29 Izraelio ir Judo karaliai išėjo į Ramot Gileadą. 
\par 30 Izraelio karalius kalbėjo Juozapatui: “Aš persirengsiu ir eisiu į mūšį, o tu apsirenk savo drabužiais”. Izraelio karalius persirengė ir išėjo į mūšį. 
\par 31 Sirijos karalius buvo įsakęs savo trisdešimt dviem kovos vežimų viršininkams nepulti nieko, tik Izraelio karalių. 
\par 32 Kovos vežimų viršininkai, pamatę Juozapatą, palaikė jį Izraelio karaliumi ir puolė. Juozapatas šaukėsi pagalbos. 
\par 33 Kovos vežimų viršininkai, supratę, kad jis ne Izraelio karalius, liovėsi jį puolę. 
\par 34 Vienas vyras netaikydamas įtempė lanką ir iššovė. Strėlė pataikė Izraelio karaliui tarp šarvų. Tada jis tarė savo vežikui: “Išvežk mane iš kovos lauko, nes esu sužeistas”. 
\par 35 Tą dieną mūšis sustiprėjo, karalius stovėjo vežime prieš sirus ir vakare mirė. Kraujas tekėjo iš žaizdos į vežimą. 
\par 36 Saulei leidžiantis, kariuomenei buvo pranešta: “Kiekvienas į savo kraštą ir į savo miestą”. 
\par 37 Karalius mirė ir buvo parvežtas į Samariją. Ir jie palaidojo karalių Samarijoje. 
\par 38 Jiems plaunant vežimą prie Samarijos tvenkinio, šunys laižė jo kraują, o paleistuvės prausėsi tvenkinyje­viskas įvyko pagal Viešpaties žodį. 
\par 39 Visi kiti Ahabo darbai, dramblio kaulo namai, kuriuos jis pastatė, miestai, kuriuos sutvirtino, yra surašyta Izraelio karalių metraščių knygoje. 
\par 40 Ahabas užmigo prie savo tėvų, o jo vietoje pradėjo karaliauti jo sūnus Ahazijas. 
\par 41 Asos sūnus Juozapatas pradėjo karaliauti Jude ketvirtaisiais Izraelio karaliaus Ahabo metais. 
\par 42 Jis, pradėdamas karaliauti, buvo trisdešimt penkerių metų amžiaus ir karaliavo dvidešimt penkerius metus Jeruzalėje. Jo motina buvo Šilhio duktė Azuba. 
\par 43 Jis vaikščiojo savo tėvo Asos keliais ir nenukrypo nuo jų, darydamas tai, kas teisinga Viešpaties akyse. Bet aukojimo aukštumų nepanaikino, žmonės tebeaukojo ir tebesmilkė aukštumose. 
\par 44 Juozapatas padarė taiką su Izraelio karaliumi. 
\par 45 Visi kiti Juozapato darbai, jo galia, kurią jis parodė, jo kovos yra surašyta Judo karalių metraščių knygoje. 
\par 46 Jis išnaikino per jo tėvo Asos dienas krašte užsilikusius iškrypėlius. 
\par 47 Tuomet Edome nebuvo karaliaus ir jį valdė vietininkas. 
\par 48 Juozapatas pastatė laivų auksui gabenti iš Ofyro krašto. Bet jie nenuplaukė, nes laivai sudužo Ecion Geberyje. 
\par 49 Ahabo sūnus Ahazijas pasiūlė Juozapatui: “Mano tarnai teplaukioja laivuose su tavo tarnais”, bet Juozapatas pasiūlymo nepriėmė. 
\par 50 Juozapatas užmigo prie savo tėvų ir buvo palaidotas prie savo tėvų savo tėvo Dovydo mieste, o jo vietoje pradėjo karaliauti jo sūnus Jehoramas. 
\par 51 Ahabo sūnus Ahazijas pradėjo karaliauti Izraelyje, Samarijoje, septynioliktais Judo karaliaus Juozapato metais ir karaliavo Izraelyje dvejus metus. 
\par 52 Jis darė pikta Viešpaties akyse ir vaikščiojo savo tėvo, motinos ir Nebato sūnaus Jeroboamo, kuris įtraukė Izraelį į nuodėmę, keliais. 
\par 53 Jis tarnavo Baalui, jį garbino ir sukėlė Viešpaties, Izraelio Dievo, pyktį kaip ir jo tėvas.


\end{document}