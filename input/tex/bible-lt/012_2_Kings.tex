\begin{document}

\title{
\par 2 Kings}

\chapter{1}

\par 1 Ahabui mirus, Moabas sukilo prieš Izraelį. 
\par 2 Ahazijas nukrito Samarijoje nuo viršutinio aukšto pro groteles ir susižeidė. Jis siuntė pasiuntinius pas Ekrono dievą Baal Zebubą paklausti, ar jis pasveiks iš šitos ligos. 
\par 3 Viešpaties angelas įsakė Elijui: “Eik pasitikti Samarijos karaliaus pasiuntinių ir jiems sakyk: ‘Ar nėra Dievo Izraelyje, kad einate klausti Ekrono dievo Baal Zebubo?’ 
\par 4 Todėl taip sako Viešpats: ‘Iš lovos, į kurią atsigulei, nebeatsikelsi, bet tikrai mirsi!’ ” Ir Elijas nuėjo. 
\par 5 Pasiuntiniai sugrįžo pas karalių. Jis klausė jų: “Kodėl sugrįžote?” 
\par 6 Jie jam atsakė: “Vienas vyras pasitiko mus ir tarė: ‘Grįžkite pas karalių, kuris jus siuntė, ir jam sakykite: ‘Taip sako Viešpats: ‘Ar nėra Dievo Izraelyje, kad siunti klausti Ekrono dievo Baal Zebubo? Todėl iš lovos, į kurią atsigulei, nebeatsikelsi, bet tikrai mirsi!’ ” 
\par 7 Karalius jų klausė: “Kaip atrodė tas vyras, kuris jus pasitiko ir jums taip kalbėjo?” 
\par 8 Jie jam atsakė: “Tas vyras buvo ilgais plaukais ir susijuosęs odiniu diržu”. Ahazijas tarė: “Tai tišbietis Elijas”. 
\par 9 Karalius siuntė pas Eliją penkiasdešimtininką su jo vyrais. Tas, užlipęs į kalno viršūnę, kur Elijas sėdėjo, kreipėsi į jį: “Dievo vyre, karalius įsako tau lipti žemyn!” 
\par 10 Elijas atsakė penkiasdešimtininkui: “Jei aš Dievo vyras, tai tegul ugnis krinta iš dangaus ir praryja tave bei tavo penkiasdešimt vyrų”. Ugnis krito iš dangaus ir prarijo jį bei jo visus vyrus. 
\par 11 Karalius siuntė pas jį kitą penkiasdešimtininką su jo vyrais. Tas, užėjęs į kalną, tarė jam: “Dievo vyre, karalius įsako tau skubiai lipti žemyn!” 
\par 12 Elijas atsakė: “Jei aš Dievo vyras, tai tegul krinta iš dangaus ugnis ir praryja tave bei tavo penkiasdešimt vyrų”. Dievo ugnis krito iš dangaus ir prarijo jį ir visus jo vyrus. 
\par 13 Karalius vėl siuntė penkiasdešimtininką su jo vyrais. Trečiasis penkiasdešimtininkas, atėjęs pas Eliją, atsiklaupė ant kelių prieš jį ir maldavo: “Dievo vyre, meldžiu, pasigailėk manęs ir šitų penkiasdešimties vyrų gyvybės. 
\par 14 Štai krito ugnis iš dangaus ir prarijo abu pirmuosius penkiasdešimtininkus ir jų vyrus. Prašau, pasigailėk manęs”. 
\par 15 Viešpaties angelas sakė Elijui: “Leiskis su juo žemyn. Nebijok”. Elijas nuėjo su juo pas karalių 
\par 16 ir jam kalbėjo: “Taip sako Viešpats: ‘Kadangi tu siuntei pasiuntinius klausti pas Ekrono dievą Baal Zebubą, tarsi Izraelyje nebūtų Dievo, kurį galima pasiklausti, tai iš lovos, į kurią atsigulei, nebeatsikelsi, bet tikrai mirsi’ ”. 
\par 17 Jis mirė, kaip paskelbė Viešpats per Eliją. Ir Jehoramas karaliavo jo vietoje antraisiais Judo karaliaus Jehoramo, Juozapato sūnaus, metais, nes Ahazijas neturėjo sūnaus. 
\par 18 Visi kiti Ahazijo darbai yra surašyti Izraelio karalių metraščių knygoje.


\chapter{2}


\par 1 Elijui ir Eliziejui einant iš Gilgalo, Viešpats buvo numatęs paimti Eliją viesulu į dangų. 
\par 2 Elijas tarė Eliziejui: “Pasilik čia, nes Viešpats mane siunčia į Betelį”. Bet Eliziejus atsakė: “Kaip gyvas Viešpats ir gyva tavo siela, aš tavęs nepaliksiu”. Jie nuėjo į Betelį. 
\par 3 Pranašų sūnūs, kurie buvo Betelyje, atėję pas Eliziejų, klausė: “Ar žinai, kad šiandien Viešpats paims tavo valdovą nuo tavęs?” Tas atsakė: “Aš žinau, tylėkite”. 
\par 4 Elijas sakė jam: “Eliziejau, pasilik čia, nes Viešpats mane siunčia į Jerichą”. Tas atsakė: “Kaip gyvas Viešpats ir gyva tavo siela, aš tavęs nepaliksiu”. Jie atėjo į Jerichą. 
\par 5 Pranašų mokiniai, kurie buvo Jeriche, atėję pas Eliziejų, klausė: “Ar žinai, kad šiandien Viešpats paims tavo valdovą nuo tavęs?” Tas atsakė: “Aš žinau, tylėkite!” 
\par 6 Tada Elijas jam tarė: “Pasilik čia, nes Viešpats mane siunčia prie Jordano”. Tas atsakė: “Kaip gyvas Viešpats ir gyva tavo siela, aš tavęs nepaliksiu”. Juodu ėjo toliau. 
\par 7 Penkiasdešimt vyrų, pranašų sūnų, ėjo ir sustoję stebėjo iš tolo, o juodu nuėjo iki Jordano. 
\par 8 Elijas, paėmęs savo apsiaustą, suvyniojo jį ir juo sudavė per vandenį. Vanduo persiskyrė, ir juodu perėjo sausuma. 
\par 9 Kitoje Jordano pusėje Elijas tarė Eliziejui: “Ko norėtum, kad padaryčiau prieš tai, kai būsiu paimtas nuo tavęs?” Eliziejus atsakė: “Prašau, kad tavo dvasios būtų dvigubai ant manęs”. 
\par 10 Elijas atsakė: “Prašai sunkaus dalyko. Jei matysi, kai mane paima, bus tai tau, bet jei nematysi­nebus”. 
\par 11 Jiems besikalbant ir einant toliau, pasirodė ugninis vežimas bei ugniniai žirgai ir perskyrė juodu. Taip Elijas buvo paimtas viesulu į dangų. 
\par 12 Eliziejus tai matė ir šaukė: “Mano tėve, mano tėve! Izraelio kovos vežimai ir jo raiteliai!” Ir jis daugiau jo nebematė. Tuomet jis persiplėšė drabužius. 
\par 13 Eliziejus, pakėlęs Elijo apsiaustą, kuris nuo jo nukrito, sugrįžęs atsistojo ant Jordano kranto. 
\par 14 Jis paėmė nuo Elijo nukritusį apsiaustą ir, sudavęs juo per vandenį, tarė: “Kur yra Viešpats, Elijo Dievas?” Kai jis sudavė per vandenį, vanduo persiskyrė, ir Eliziejus perėjo. 
\par 15 Pranašų sūnūs, kurie buvo kitoje pusėje, Jeriche, jį pamatę, sakė: “Elijo dvasia nužengė ant Eliziejaus”. Atėję jo pasitikti, jie nusilenkė prieš jį iki žemės 
\par 16 ir jam tarė: “Pas mus yra penkiasdešimt tvirtų vyrų, tavo tarnų. Tegu jie nueina ieškoti Elijo. Gal Viešpaties Dvasia, paėmusi jį, numetė ant kokio kalno ar į slėnį?” Jis atsakė: “Nesiųskite”. 
\par 17 Jiems primygtinai prašant, Eliziejus sutiko. Jie išsiuntė penkiasdešimt vyrų, kurie ieškojo tris dienas, bet jo nerado. 
\par 18 Jiems sugrįžus į Jerichą, Eliziejus jiems tarė: “Argi aš jums nesakiau: ‘Neikite’?” 
\par 19 Jericho gyventojai sakė Eliziejui: “Miestas geroje vietoje, kaip tu pats matai, bet vanduo blogas ir žemė nederlinga”. 
\par 20 Eliziejus atsakė: “Atneškite man naują dubenį ir įberkite į jį druskos”. Ir jie atnešė jam. 
\par 21 Nuėjęs prie vandens šaltinio, jis įmetė į jį druskos ir sakė: “Taip sako Viešpats: ‘Aš pagydžiau šitą vandenį. Jis nebebus priežastimi nei mirties, nei nevaisingumo’ ”. 
\par 22 Vanduo yra sveikas iki šios dienos, kaip Eliziejus pasakė. 
\par 23 Iš Jericho jis ėjo į Betelį. Jam einant keliu, maži vaikai, išėję iš miesto, tyčiojosi iš jo ir šaukė: “Eik aukštyn, pliki! Eik aukštyn, pliki!” 
\par 24 Atsigręžęs jis pažiūrėjo į juos ir prakeikė juos Viešpaties vardu. Dvi lokės atbėgo iš miško ir sudraskė keturiasdešimt du vaikus. 
\par 25 Iš čia jis nuėjo į Karmelio kalną, o iš ten grįžo į Samariją.



\chapter{3}

\par 1 Ahabo sūnus Jehoramas pradėjo valdyti Izraelį Samarijoje aštuonioliktaisiais Judo karaliaus Juozapato metais ir karaliavo dvylika metų. 
\par 2 Jis darė pikta Viešpaties akyse, tačiau ne taip, kaip jo tėvas ir motina. Jis pašalino Baalo atvaizdą, kurį buvo padaręs jo tėvas. 
\par 3 Tačiau jis laikėsi nuodėmių Nebato sūnaus Jeroboamo, kuris įtraukė Izraelį į nuodėmę; jis neatsitraukė nuo jų. 
\par 4 Moabo karalius Meša buvo avių augintojas. Jis kasmet duodavo Izraelio karaliui duoklę šimtą tūkstančių ėriukų ir šimto tūkstančių avių vilnų. 
\par 5 Ahabui mirus, Moabo karalius sukilo prieš Izraelio karalių. 
\par 6 Tada karalius Jehoramas išėjo iš Samarijos ir suskaičiavo visą Izraelį. 
\par 7 Jis siuntė pas Judo karalių Juozapatą, sakydamas: “Moabo karalius sukilo prieš mane. Ar eisi su manimi kariauti prieš Moabą?” Juozapatas sakė: “Eisiu. Aš­kaip ir tu, mano tauta­kaip ir tavo tauta, mano žirgai­kaip ir tavo žirgai”. 
\par 8 Jis klausė: “Kuriuo keliu eisime?” Tas atsakė: “Edomo dykumos keliu”. 
\par 9 Izraelio, Judo ir Edomo karaliai išėjo kartu. Septintą dieną jiems pritrūko vandens kariuomenei ir gyvuliams. 
\par 10 Izraelio karalius sakė: “Vargas! Viešpats sukvietė šituos tris karalius, kad atiduotų į Moabo rankas”. 
\par 11 Juozapatas paklausė: “Ar čia nėra Viešpaties pranašo, per kurį galėtume paklausti Viešpaties?” Vienas Izraelio karaliaus tarnas atsakė: “Čia yra Šafato sūnus Eliziejus, kuris pildavo Elijui ant rankų vandenį”. 
\par 12 Juozapatas atsakė: “Viešpaties žodis yra su juo”. Izraelio karalius, Juozapatas ir Edomo karalius nuėjo pas jį. 
\par 13 Eliziejus kalbėjo Izraelio karaliui: “Ką aš turiu bendro su tavimi? Eik pas savo tėvo ir motinos pranašus!” Izraelio karalius jam atsakė: “Ne! Juk Viešpats sukvietė šituos tris karalius, kad juos atiduotų į Moabo rankas”. 
\par 14 Tuomet Eliziejus tarė: “Kaip gyvas kareivijų Viešpats, kuriam tarnauju, jei ne Judo karalius Juozapatas, aš nežiūrėčiau ir nekreipčiau dėmesio į tave. 
\par 15 Dabar atveskite man arfininką”. Arfininkui skambinant, Viešpaties ranka prisilietė prie jo 
\par 16 ir jis kalbėjo: “Taip sako Viešpats: ‘Iškaskite šiame slėnyje daug griovių’. 
\par 17 Nes taip sako Viešpats: ‘Jūs nepastebėsite nei vėjo, nei lietaus, tačiau šitas slėnis prisipildys vandens; galėsite gerti jūs ir jūsų gyvuliai’. 
\par 18 Negana to, Viešpats dar atiduos Moabą į jūsų rankas. 
\par 19 Jūs užimsite visus žymesniuosius ir sustiprintus miestus, iškirsite geriausius medžius, užversite vandens šaltinius ir derlingą žemę užteršite akmenimis”. 
\par 20 Rytą, kai paprastai aukojama duonos auka, pasirodė vanduo iš Edomo pusės ir pripildė visą slėnį. 
\par 21 Moabitai, išgirdę, kad karaliai atėjo kariauti prieš juos, surinko visus vyrus, jaunus ir senus, galinčius nešioti ginklą, ir sustojo krašto pasienyje. 
\par 22 Anksti rytą atsikėlę, saulei šviečiant, moabitai pamatė vandenį, raudoną kaip kraujas, 
\par 23 ir tarė: “Tai kraujas! Tikrai karaliai bus susikovę tarp savęs ir vienas kitą išžudę. Dabar prie grobio, Moabai!” 
\par 24 Bet, jiems atėjus prie Izraelio stovyklos, izraelitai taip sumušė moabitus, kad jie bėgo, o tie vijosi ir mušė juos. 
\par 25 Jie sugriovė miestus ir ant derlingos žemės kiekvienas numetė po akmenį, vandens šaltinius užvertė ir geriausius medžius iškirto; išliko tik akmenys Kir Haresete. Mėtytojai apsupo jį ir sugriovė. 
\par 26 Kai Moabo karalius pamatė, kad karas jam per sunkus, pasiėmė septynis šimtus vyrų, ginkluotų kardais, ir norėjo prasilaužti pas Edomo karalių, bet neįstengė. 
\par 27 Tada jis paėmė savo pirmagimį sūnų, kuris turėjo karaliauti jo vietoje, ir aukojo kaip deginamąją auką ant miesto sienos. Kilo didelis pasipiktinimas prieš Izraelį. Ir jie atsitraukė nuo jo ir grįžo į savo kraštą.



\chapter{4}


\par 1 Vieno pranašo žmona šaukė Eliziejui: “Tavo tarnas, mano vyras, mirė. Tu žinai, kad jis bijojo Viešpaties, tačiau skolintojas atėjo už skolą paimti vergais abu mano sūnus”. 
\par 2 Eliziejus klausė: “Kuo galiu padėti? Sakyk, ką turi namuose?” Ji atsakė: “Nieko kito neturiu, tik ąsotėlį aliejaus”. 
\par 3 Jis jai tarė: “Eik ir pasiskolink tuščių indų iš visų savo kaimynų. Nemažai pasiskolink. 
\par 4 Sugrįžusi užsirakink duris su savo sūnumis ir pilk aliejų į visus tuščius indus, o pilnus padėk į šalį”. 
\par 5 Parėjusi ji su savo sūnumis užsirakino duris, ir jie jai padavinėjo indus, o ji pylė. 
\par 6 Kai visi indai buvo pilni, ji tarė sūnui: “Paduok man dar vieną indą”. Jis atsakė: “Nebėra indų”. Tuomet aliejus nustojo tekėjęs. 
\par 7 Ji, nuėjusi pas Dievo vyrą, viską jam papasakojo. Eliziejus tarė: “Parduok aliejų ir užmokėk skolą, o kas liks, iš to gyvenkite”. 
\par 8 Kartą Eliziejus ėjo į Šunemą, kur gyveno turtinga moteris. Ji kvietė jį užeiti pavalgyti. Nuo to laiko kiekvieną kartą praeidamas jis užsukdavo pavalgyti. 
\par 9 Kartą ji tarė savo vyrui: “Aš suprantu, kad tas žmogus, kuris nuolat užeina pas mus, yra šventas Dievo vyras. 
\par 10 Padarykime mažą kambarėlį ant aukšto ir padėkime jam ten lovą, stalą, kėdę ir lempą, kad užsukęs jis turėtų kur pailsėti”. 
\par 11 Kartą Eliziejus atėjęs užėjo į tą kambarį ir ten ilsėjosi. 
\par 12 Jis liepė savo tarnui Gehaziui pakviesti šunemietę. Ji atėjo. 
\par 13 Jai atėjus, jis tarė tarnui: “Paklausk jos, kuo galėčiau atsilyginti už visą jos rūpestį. Gal ji norėtų, kad pakalbėčiau už ją su karaliumi ar kariuomenės vadu?” Ji atsakė: “Aš gyvenu tarp savo žmonių”. 
\par 14 Eliziejus vėl klausė: “Ką jai padaryti?” Gehazis atsiliepė: “Ji neturi sūnaus, o jos vyras senas”. 
\par 15 Tuomet Eliziejus liepė pakviesti ją. Atėjusi ji atsistojo tarpduryje. 
\par 16 Jis tarė jai: “Už metų, apie šitą laiką, tu glamonėsi sūnų!” Ji sakė: “O ne, mano viešpatie, Dievo vyre, nemeluok savo tarnaitei”. 
\par 17 Moteris pastojo ir po metų pagimdė sūnų, kaip Eliziejus jai buvo sakęs. 
\par 18 Vaikas paaugo ir kartą išėjo pas tėvą prie pjovėjų. 
\par 19 Jis tarė savo tėvui: “Mano galva, mano galva!” Tėvas liepė tarnui parnešti jį namo pas motiną. 
\par 20 Kai jį parnešė pas motiną, pasėdėjęs ant jos kelių iki vidudienio, jis mirė. 
\par 21 Ji užnešė jį, paguldė į Dievo vyro lovą, užrakino duris ir išėjo. 
\par 22 Pašaukusi savo vyrą, ji tarė: “Atsiųsk tarną ir asilę, kad galėčiau skubiai nuvykti pas Dievo vyrą ir tuojau sugrįžti”. 
\par 23 Jis paklausė: “Kodėl tu eini pas jį šiandien? Juk nei jaunas mėnulis, nei sabatas”. Bet ji atsakė: “Viskas gerai”. 
\par 24 Pabalnojusi asilę, ji sakė savo tarnui: “Varyk ir skubėk. Nesustok, kol pasakysiu”. 
\par 25 Ji atėjo pas Dievo vyrą prie Karmelio kalno. Dievo vyras, pamatęs ją iš tolo, tarė savo tarnui Gehaziui: “Šunemietė čia! 
\par 26 Bėk ją pasitikti ir paklausk, kaip sekasi jai, jos vyrui ir vaikui”. Ji atsakė: “Gerai”. 
\par 27 Atėjusi pas Dievo vyrą, ji apkabino jo kojas. Gehazis norėjo atitraukti, bet Dievo vyras tarė: “Palik ją, ji labai nusiminusi. Viešpats nuslėpė tai nuo manęs ir nepasakė”. 
\par 28 Ji tarė: “Argi aš tavęs, mano viešpatie, prašiau sūnaus? Argi nesakiau: ‘Neapgauk manęs’?” 
\par 29 Tada Eliziejus liepė Gehaziui susijuosti, pasiimti jo lazdą ir eiti: “Jei ką sutiksi, nesveikink jo, o jei kas tave sveikins, neatsakyk jam. Nuėjęs uždėk mano lazdą ant berniuko veido”. 
\par 30 Vaiko motina atsakė Eliziejui: “Kaip gyvas Viešpats ir gyva tavo siela, aš tavęs nepaliksiu”. Jis nuėjo kartu su ja. 
\par 31 Gehazis, nuėjęs pirma jų, uždėjo lazdą ant berniuko, tačiau nebuvo nei balso, nei atsakymo. Tada jis grįžo ir, kelyje susitikęs juos, pasakė jam: “Berniukas nepabudo”. 
\par 32 Eliziejus, įėjęs į namus, pamatė negyvą berniuką jo lovoje. 
\par 33 Jis užrakino duris paskui save ir meldėsi Viešpačiui. 
\par 34 Po to jis atsigulė ant vaiko, uždėjo savo burną ant jo burnos, savo akis ant jo akių ir savo rankas ant jo rankų. Vaiko kūnas sušilo. 
\par 35 Pasivaikščiojęs Eliziejus vėl išsitiesė ant vaiko. Tuomet berniukas sučiaudėjo septynis kartus ir atsimerkė. 
\par 36 Pasišaukęs Gehazį, jis tarė: “Pašauk šunemietę”. Jai atėjus, jis sakė: “Pasiimk savo sūnų”. 
\par 37 Įėjusi ji parpuolė prie jo kojų, nusilenkė iki žemės ir, pasiėmusi sūnų, išėjo. 
\par 38 Eliziejus sugrįžo į Gilgalą. Badas buvo krašte. Pranašų sūnums sėdint prie jo, jis liepė savo tarnui užkaisti didelį puodą ir išvirti visiems viralo. 
\par 39 Vienas iš jų išėjo į lauką pasirinkti daržovių; jis rado laukinį augalą, nuo kurio prisirinko laukinių moliūgų pilną apsiaustą. Parėjęs juos supjaustė į viralo puodą, nes jų nepažino. 
\par 40 Vyrai, paragavę įpilto viralo, šaukė: “Mirtis puode, Dievo vyre!” Jie negalėjo valgyti. 
\par 41 Eliziejus liepė atnešti miltų. Supylęs juos į puodą, jis tarė: “Pilkite žmonėms, kad jie valgytų”. Puode nebuvo nieko kenksmingo. 
\par 42 Vienas vyras atėjo iš Baal Šališos ir atnešė Dievo vyrui pirmavaisių: dvidešimt miežinių kepalų ir šviežių varpų maiše. Eliziejus liepė savo tarnui duoti žmonėms valgyti, 
\par 43 bet jis atsakė: “Ar tuo maistu galiu pamaitinti šimtą vyrų?” Eliziejus pakartojo: “Duok žmonėms valgyti, nes Viešpats sako: ‘Jie valgys ir dar paliks’ ”. 
\par 44 Jis padavė jiems maistą; jie valgė ir dar liko pagal Viešpaties žodį.



\chapter{5}


\par 1 Sirijos karaliaus kariuomenės vadas Naamanas buvo didis žmogus pas savo valdovą ir gerbiamas, nes Viešpats per jį suteikė išlaisvinimą Sirijai. Jis buvo narsus karys, tačiau sirgo raupsais. 
\par 2 Sirai, kartą išėję būriais, parsivedė iš Izraelio į nelaisvę jauną mergaitę; ji tarnavo Naamano žmonai. 
\par 3 Kartą ji tarė savo šeimininkei: “Jei mano valdovas būtų Samarijoje, tai pranašas pagydytų jį nuo raupsų”. 
\par 4 Atėjęs pas savo valdovą, Naamanas pasakė, ką mergaitė iš Izraelio kalbėjo jo žmonai. 
\par 5 Sirijos karalius jam tarė: “Gerai, eik! O aš pasiųsiu Izraelio karaliui laišką”. Jis pasiėmė dešimtį talentų sidabro, šešis tūkstančius šekelių aukso, dešimt išeiginių drabužių ir išėjo. 
\par 6 Jis nunešė Izraelio karaliui laišką, kuriame buvo parašyta: “Kartu su šituo laišku siunčiu pas tave savo tarną Naamaną, kad jį pagydytum nuo raupsų”. 
\par 7 Izraelio karalius, perskaitęs laišką, perplėšė savo drabužius ir tarė: “Argi aš Dievas, kad gyvybė ir mirtis būtų mano rankose? Jis siunčia pas mane, kad pagydyčiau vyrą nuo raupsų! Pagalvokite ir matykite, kaip jis ieško priekabių prie manęs”. 
\par 8 Dievo vyras Eliziejus, išgirdęs, kad Izraelio karalius perplėšė savo drabužius, siuntė pas karalių, sakydamas: “Kodėl perplėšei savo drabužius? Tegul jis ateina pas mane ir jis sužinos, kad Izraelyje yra pranašas”. 
\par 9 Naamanas atvyko su žirgais bei vežimais ir sustojo prie Eliziejaus namų durų. 
\par 10 Eliziejus siuntė pas Naamaną savo tarną, sakydamas: “Eik ir apsiplauk septynis kartus Jordane, tavo kūnas atsinaujins ir bus švarus”. 
\par 11 Naamanas supyko ir nuėjo, sakydamas: “Aš galvojau, kad jis išeis, atsistos, šauksis Viešpaties, savo Dievo, vardo, uždės ranką ant raupsuotų vietų ir taip pagydys nuo raupsų. 
\par 12 Argi Damasko upių Abanos ir Parparo vanduo ne geresnis už visus Izraelio vandenis? Argi negalėčiau juose apsiplauti ir tapti švarus?” Jis apsigręžė ir nuėjo labai supykęs. 
\par 13 Bet jo tarnai priartėję jam kalbėjo: “Jei pranašas tau būtų įsakęs atlikti kokį didelį dalyką, argi nebūtum daręs? Tuo labiau, kai jis sakė: ‘Apsiplauk ir būsi švarus’ ”. 
\par 14 Jis nuėjo prie Jordano ir pasinėrė jame septynis kartus, kaip Dievo vyras buvo liepęs. Jo kūnas pasidarė kaip mažo vaiko, ir jis tapo švarus. 
\par 15 Jis ir visi jo palydovai sugrįžo pas Dievo vyrą. Atsistojęs prieš pranašą, jis tarė: “Dabar tikrai žinau, kad niekur kitur žemėje nėra Dievo, tik Izraelyje. Prašau, priimk dovaną iš savo tarno”. 
\par 16 Eliziejus atsakė: “Kaip gyvas Viešpats, kuriam aš tarnauju, nieko neimsiu”. Nors Naamanas jį įkalbinėjo paimti, tačiau jis atsisakė. 
\par 17 Tuomet Naamanas prašė: “Jeigu ne, tai, prašau, duok savo tarnui tiek žemės, kiek gali panešti pora mulų, nes tavo tarnas daugiau nebeaukos deginamųjų ir kitų aukų jokiam dievui, tik Viešpačiui. 
\par 18 Tik norėčiau, kad Viešpats man atleistų nusikaltimą, kai aš vesiu savo valdovą į Rimono namus pagarbinti ir kartu su juo nusilenksiu Rimonui”. 
\par 19 Eliziejus jam tarė: “Eik ramybėje”. Jam paėjus nuo jo gerą kelio galą, 
\par 20 Dievo vyro Eliziejaus tarnas Gehazis pagalvojo: “Mano viešpats pasigailėjo siro Naamano ir nepriėmė iš jo to, ką jis atnešė. Kaip gyvas Viešpats, aš bėgsiu paskui jį ir paimsiu iš jo ką nors”. 
\par 21 Gehazis nusivijo Naamaną. Naamanas, pamatęs atbėgantį vyrą jam iš paskos, nušoko nuo vežimo ir klausė: “Ar viskas gerai?” 
\par 22 Tas atsakė: “Viskas gerai! Mano viešpats pasiuntė mane, sakydamas: ‘Ką tik atėjo du pranašų sūnūs iš Efraimo kalnų. Prašau, duok jiems talentą sidabro ir du išeiginius drabužius’ ”. 
\par 23 Naamanas atsakė: “Paimk du talentus”. Jis įkalbinėjo, įrišo du talentus sidabro į du maišus, taip pat du išeiginius drabužius ir padavė dviem tarnams, kurie nunešė pirma jo. 
\par 24 Atėjęs prie kalvos, Gehazis paėmė iš jų maišus ir padėjo juos savo namuose. Po to jis paleido vyrus, ir jie grįžo. 
\par 25 Gehazis atėjo pas Eliziejų, kuris paklausė: “Iš kur ateini, Gehazi?” Tas atsakė: “Tavo tarnas niekur nebuvo išėjęs”. 
\par 26 Bet jis jam atsakė: “Argi mano širdis nebuvo su tavimi, kai vyras iššoko iš vežimo tavęs pasitikti? Ar dabar metas paimti pinigus ir drabužius, alyvų sodus ir vynuogynus, avis ir galvijus, tarnus ir tarnaites? 
\par 27 Todėl ir Naamano raupsai prilips tau ir tavo palikuonims per amžius”. Jis išėjo nuo jo raupsuotas kaip sniegas.



\chapter{6}

\par 1 Pranašų sūnūs sakė Eliziejui: “Vieta, kurioje gyvename, kaip matai, mums per ankšta. 
\par 2 Leisk mums eiti prie Jordano, pasikirsti ten medžių ir pasistatyti namus”. Jis atsakė: “Eikite”. 
\par 3 Vienas jų prašė pranašą eiti su jais. Jis atsakė: “Gerai, aš eisiu”. 
\par 4 Jis išėjo su jais. Atėję prie Jordano, jie kirto medžius. 
\par 5 Vienam kertant medį, kirvio geležis įkrito į vandenį. Jis sušuko: “Vargas, šeimininke! Jį buvau pasiskolinęs!” 
\par 6 Dievo vyras paklausė: “Kur jis įkrito?” Kai tas parodė vietą, pranašas nusipjovė lazdą, įmetė ją į tą vietą, ir kirvis iškilo. 
\par 7 Tada tarė jis: “Pasiimk jį”. Vyras ištiesė ranką ir pasiėmė kirvį. 
\par 8 Sirijos karalius kariavo su Izraeliu. Jis tarėsi su savo tarnais, sakydamas: “Tokioje ir tokioje vietoje bus mano stovykla”. 
\par 9 Dievo vyras pasiuntė pas Izraelio karalių, sakydamas: “Saugokis, neik pro tą vietą, nes ten sirai”. 
\par 10 Izraelio karalius pasiuntė į tą vietą, kurią jam nurodė Dievo vyras, ir saugojo ją. Taip jis saugojosi ne kartą ir ne du. 
\par 11 Sirijos karalius dėl to labai nerimavo. Jis, pasišaukęs savo tarnus, tarė: “Sakykite, kas iš mūsų su Izraelio karaliumi?” 
\par 12 Vienas iš jo tarnų tarė: “Nė vienas, mano valdove karaliau. Bet pranašas Eliziejus, kuris gyvena Izraelyje, praneša Izraelio karaliui tai, ką kalbi savo miegamajame”. 
\par 13 Jis įsakė savo vyrams: “Eikite ir sužinokite, kur jis yra, kad jį suimčiau”. Jam buvo pranešta: “Jis Dotane”. 
\par 14 Jis pasiuntė raitelių, kovos vežimų ir didelę kariuomenę, kurie nakčia apsupo miestą. 
\par 15 Dievo vyro tarnas, anksti atsikėlęs ir išėjęs, pamatė kariuomenę su raiteliais ir kovos vežimais, kurie buvo apsupę miestą. Tarnas sakė jam: “Vargas, mano šeimininke! Ką dabar darysime?” 
\par 16 Eliziejus atsakė: “Nebijok, nes su mumis yra daugiau negu su jais”. 
\par 17 Eliziejus meldėsi: “Viešpatie, atverk jo akis, kad matytų”. Viešpats atvėrė tarno akis, ir jis pamatė: štai kalnas buvo pilnas ugninių žirgų ir vežimų, supančių Eliziejų. 
\par 18 Sirams artinantis prie jo, Eliziejus meldėsi: “Viešpatie, apakink juos”. Ir Jis apakino juos pagal Eliziejaus žodį. 
\par 19 Tada Eliziejus tarė jiems: “Čia ne tas kelias ir ne tas miestas. Sekite mane, ir aš jus nuvesiu pas tą vyrą, kurio ieškote”. Jis juos nuvedė į Samariją. 
\par 20 Jiems atėjus į Samariją, Eliziejus meldėsi: “Viešpatie, atverk jų akis, kad jie vėl matytų”. Viešpats atvėrė jų akis, ir jie pamatė esą Samarijos viduryje. 
\par 21 Izraelio karalius, juos pamatęs, klausė Eliziejų: “Mano tėve, ar juos užmušti?” 
\par 22 Eliziejus atsakė: “Neužmušk. Argi tu žudai tuos, kuriuos paimi į nelaisvę savo kardu ir lanku? Duok jiems duonos ir vandens, kad pavalgytų ir atsigertų, o tada tegul eina pas savo valdovą”. 
\par 23 Jis jiems paruošė dideles vaišes, ir, jiems pavalgius bei atsigėrus, paleido juos grįžti pas savo valdovą. Nuo to laiko Sirijos būriai daugiau nebepuldinėjo Izraelio. 
\par 24 Kartą Sirijos karalius Ben Hadadas, surinkęs visą savo kariuomenę, atėjo ir apgulė Samariją. 
\par 25 Tada kilo didelis badas Samarijoje. Jie laikė ją apgulę, kol asilo galva kainavo aštuoniasdešimt šekelių sidabro ir ketvirtis kabo balandžių mėšlo penkis šekelius sidabro. 
\par 26 Izraelio karaliui einant mūro siena, viena moteris sušuko: “Padėk, mano valdove karaliau!” 
\par 27 Jis atsakė: “Jei Viešpats tau nepadeda, kaip aš tau padėsiu? Iš klojimo ar iš vyno spaustuvo?” 
\par 28 Karalius jos paklausė: “Ko norėtum?” Ji atsakė: “Šita moteris man sakė: ‘Suvalgykime tavo sūnų šiandien, o mano sūnų rytoj suvalgysime’. 
\par 29 Išvirėme mano sūnų ir jį suvalgėme. O kai kitą dieną jai pasakiau, kad duotų savo sūnų suvalgyti, tai ji paslėpė jį”. 
\par 30 Karalius, išgirdęs moters žodžius, persiplėšė drabužius. Jis ėjo per sieną, ir žmonės matė, kad jis po rūbais vilkėjo ašutinę ant kūno. 
\par 31 Jis tarė: “Tegul Dievas padaro man tai ir dar daugiau, jei Šafato sūnaus Eliziejaus galva šiandien pasiliks ant jo pečių”. 
\par 32 Karalius pasiuntė vyrą pas Eliziejų, kuris tuo laiku sėdėjo savo namuose su vyresniaisiais. Eliziejus tarė vyresniesiems, prieš pasiuntiniui ateinant: “Ar matote, šitas žmogžudžio sūnus pasiuntė vyrą, kad nukirstų man galvą? Pamatę pasiuntinį ateinant, užrakinkite duris ir tvirtai laikykite, nes ir jo valdovą tuoj pamatysite”. 
\par 33 Jam kalbant su jais, atėjo pasiuntinys ir tarė: “Šita nelaimė yra iš Viešpaties! Tai ko man dar laukti iš Viešpaties?”



\chapter{7}

\par 1 Tada Eliziejus sakė: “Klausykite Viešpaties žodžio: ‘Rytoj tuo laiku Samarijos vartuose smulkių miltų sykelis kainuos vieną šekelį ir du miežių sykeliai vieną šekelį’ ”. 
\par 2 Tuomet vyras, į kurio ranką karalius remdavosi, sakė Dievo vyrui: “Jei Viešpats atidarytų dangaus langus, ar galėtų taip įvykti?” Jis atsakė: “Tu tai matysi savo akimis, bet nevalgysi to”. 
\par 3 Keturi raupsuoti vyrai buvo prie miesto vartų. Jie kalbėjosi: “Ar mes čia sėdėsime, kol mirsime? 
\par 4 Jei eisime į miestą, ten siaučia badas ir mes mirsime ten, o jei čia sėdėsime, taip pat mirsime. Taigi eikime į sirų stovyklą. Jei jie paliks mus gyvus, gyvensime, o jei jie mus nužudys, mirsime”. 
\par 5 Prieblandoje jie pakilo eiti į sirų stovyklą. Priėję prie stovyklos, jie pamatė, kad ten nebuvo nė vieno žmogaus. 
\par 6 Viešpats padarė, kad sirų kariuomenė girdėjo vežimų ir žirgų kanopų bildesį, didelės kariuomenės garsus. Sirai kalbėjosi: “Tikrai, Izraelio karalius pasamdė prieš mus hetitų ir egiptiečių karalius”. 
\par 7 Jie skubėdami pakilo prieblandoje ir, palikę palapines, žirgus, asilus ir visa, kas stovykloje buvo, bėgo, gelbėdami savo gyvybes. 
\par 8 Raupsuotieji atėjo į stovyklą. Įėję į vieną palapinę, valgė, gėrė ir, pasiėmę sidabro, aukso bei drabužių, paslėpė. Po to jie sugrįžo, įėjo į kitą palapinę ir, išnešę iš ten, taip pat paslėpė. 
\par 9 Tada jie sakė vienas kitam: “Negerai darome, nes šita diena yra geros naujienos diena. Jei mes delsime ir lauksime iki aušros, ištiks mus nelaimė. Tad dabar eikime ir praneškime karaliaus namiškiams”. 
\par 10 Sugrįžę jie pranešė miesto vartų sargybai: “Buvome nuėję į sirų stovyklą; ten nėra nė vieno žmogaus, tik pririšti žirgai ir asilai, ir palapinėse viskas kaip buvo”. 
\par 11 Vartų sargyba pranešė tą žinią karaliaus namams. 
\par 12 Karalius, naktį atsikėlęs, sakė savo tarnams: “Aš jums pasakysiu, ką sirai padarė. Jie žino, kad mes alkani. Taigi jie išėjo iš stovyklos ir pasislėpė atvirame lauke, sakydami: ‘Jie išeis iš miesto, tada mes juos gyvus suimsime ir įsiveršime į miestą’ ”. 
\par 13 Vienas iš jo tarnų tarė karaliui: “Leisk paimti likusius mieste penkis žirgus, nes jie yra likę mieste, kaip ir visa Izraelio daugybė, pasiųskime vyrus ir ištirkime”. 
\par 14 Jie pakinkė žirgus į du vežimus, ir karalius pasiuntė juos paskui sirų kariuomenę, sakydamas: “Eikite ir pažiūrėkite”. 
\par 15 Jie sekė juos iki Jordano; visas kelias buvo pilnas drabužių ir ginklų, kuriuos sirai skubėdami išmėtė. Pasiuntiniai sugrįžo ir pranešė karaliui. 
\par 16 Tada žmonės ėjo ir plėšė sirų stovyklą. Sykelis smulkių miltų arba du sykeliai miežių kainavo vieną šekelį, kaip Viešpats buvo sakęs. 
\par 17 Karalius pavedė vyrui, į kurio ranką jis remdavosi, prižiūrėti vartus, bet žmonės jį sumindžiojo vartuose ir jis mirė, kaip Dievo vyras buvo sakęs, kai pas jį buvo atėjęs karalius. 
\par 18 Kai Dievo vyras sakė karaliui: “Du sykeliai miežių arba vienas sykelis smulkių miltų kainuos vieną šekelį rytoj apie šitą laiką Samarijos vartuose”, 
\par 19 tas vyras sakė Dievo vyrui: “Jei Viešpats atidarytų dangaus langus, ar galėtų taip įvykti?” Pranašas atsakė: “Tu tai matysi savo akimis, tačiau nevalgysi”. 
\par 20 Taip ir atsitiko. Žmonės jį mirtinai sumindžiojo vartuose.



\chapter{8}

\par 1 Eliziejus kalbėjo moteriai, kurios sūnų jis atgaivino: “Eik iš čia su savo namiškiais ir apsigyvenk kur nors svetur, nes Viešpats pašaukė badą į šitą šalį septyneriems metams”. 
\par 2 Moteris padarė pagal Dievo vyro žodį. Ji nuvyko su savo namiškiais į filistinų kraštą ir septynerius metus gyveno ten kaip ateivė. 
\par 3 Septyneriems metams praėjus, moteris sugrįžo iš filistinų šalies ir nuėjusi kreipėsi į karalių dėl savo namų ir žemės. 
\par 4 Karalius kalbėjosi su Dievo vyro tarnu Gehaziu, sakydamas: “Papasakok apie visus didelius darbus, kuriuos padarė Eliziejus”. 
\par 5 Gehaziui pasakojant, kaip Eliziejus prikėlė mirusįjį, atėjo ta moteris, kurios sūnų jis buvo atgaivinęs, ir kreipėsi dėl savo namų ir žemės. Gehazis tarė: “Mano valdove karaliau, štai ta moteris ir jos sūnus, kurį prikėlė Eliziejus”. 
\par 6 Karalius klausinėjo moterį, ir ji jam viską papasakojo. Po to karalius paskyrė jai vieną rūmų valdininką ir įsakė: “Sugrąžink jai visa, kas jai priklauso, ir visą lauko derlių nuo tos dienos, kai ji paliko šalį, iki šiol”. 
\par 7 Kartą Eliziejus atėjo į Damaską. Sirijos karalius Ben Hadadas sirgo. Kai jam pranešė, kad atėjo Dievo vyras, 
\par 8 karalius sakė Hazaeliui: “Imk dovanų ir eik Dievo vyro pasitikti. Per jį paklausk Viešpatį, sakydamas: ‘Ar pasveiksiu iš šitos ligos?’ ” 
\par 9 Hazaelis, pasiėmęs dovanų, visokių Damasko gėrybių, kiek keturiasdešimt kupranugarių gali panešti, ėjo susitikti su Eliziejumi. Sutikęs jį, Hazaelis tarė: “Tavo sūnus Ben Hadadas, Sirijos karalius, atsiuntė mane pas tave paklausti, ar jis pasveiks iš šitos ligos”. 
\par 10 Eliziejus jam atsakė: “Eik ir pasakyk jam: ‘Tu tikrai pagysi’, bet Viešpats man parodė, kad jis tikrai mirs”. 
\par 11 Jis ilgai žiūrėjo į jį, kol tas susigėdo. Ir Dievo vyras pradėjo verkti. 
\par 12 Hazaelis klausė: “Kodėl tu verki, viešpatie?” Jis atsakė: “Dėl to, kad aš žinau, ką tu darysi izraelitams. Jų sutvirtintus miestus tu sudeginsi, jaunuolius išžudysi kardu, kūdikius sutraiškysi ir nėščias moteris perskrosi”. 
\par 13 Hazaelis atsakė: “Kas gi yra tavo tarnas? Ar aš šuo, kad galėčiau padaryti tokius didelius dalykus?” Eliziejus atsakė: “Viešpats man parodė, kad tu tapsi Sirijos karaliumi”. 
\par 14 Išsiskyręs su Eliziejumi, jis atėjo pas savo valdovą, kuris jį klausė: “Ką sakė Eliziejus?” Jis atsakė: “Jis sakė, kad tu tikrai pagysi”. 
\par 15 Kitą dieną jis ėmė antklodę, pamirkė ją vandenyje ir uždėjo karaliui ant veido, ir tas mirė. Hazaelis ėmė karaliauti jo vietoje. 
\par 16 Penktaisiais Izraelio karaliaus Jehoramo, Ahabo sūnaus, metais Jehoramas, Judo karaliaus Juozapato sūnus, pradėjo karaliauti Jeruzalėje. 
\par 17 Jis buvo trisdešimt dvejų metų, kai pradėjo karaliauti, ir karaliavo aštuonerius metus Jeruzalėje. 
\par 18 Jis vaikščiojo Izraelio karalių keliais kaip Ahabo namai, nes Ahabo duktė buvo jo žmona. Ir jis darė pikta Viešpaties akyse. 
\par 19 Tačiau Viešpats nenorėjo Judo sunaikinti dėl savo tarno Dovydo, nes jam buvo pažadėjęs duoti žiburį iš jo vaikų. 
\par 20 Jehoramui valdant, Edomas sukilo, atsiskyrė nuo Judo ir paskyrė sau karalių. 
\par 21 Jehoramas nuvyko į Cayrą su kovos vežimais. Naktį jis užpuolė edomitus, kurie buvo apsupę jį ir kovos vežimų viršininkus, ir žmonės pabėgo į savo palapines. 
\par 22 Taip Edomas išsilaisvino iš Judo valdžios. Tuo pačiu laiku atsiskyrė ir Libna. 
\par 23 Visi kiti Jehoramo darbai surašyti Judo karalių metraščių knygoje. 
\par 24 Jehoramas užmigo prie savo tėvų ir buvo palaidotas prie savo tėvų Dovydo mieste, o jo sūnus Ahazijas pradėjo karaliauti jo vietoje. 
\par 25 Dvyliktaisiais Izraelio karaliaus Jehoramo, Ahabo sūnaus, metais Jeruzalėje pradėjo karaliauti Judo karaliaus Jehoramo sūnus Ahazijas. 
\par 26 Jis buvo dvidešimt dvejų metų, kai pradėjo karaliauti, ir karaliavo Jeruzalėje vienerius metus. Jo motina buvo vardu Atalija, Izraelio karaliaus Omrio duktė. 
\par 27 Jis vaikščiojo Ahabo namų keliais ir darė pikta Viešpaties akyse, nes buvo Ahabo namų žentas. 
\par 28 Jis su Ahabo sūnumi Jehoramu kariavo prieš Sirijos karalių Hazaelį Ramot Gileade, ir sirai sužeidė Jehoramą. 
\par 29 Karalius Jehoramas sugrįžo gydytis į Jezreelį. Ahazijas, Judo karaliaus Jehoramo sūnus, aplankė Jezreelyje sužeistą karalių Jehoramą, Ahabo sūnų.



\chapter{9}

\par 1 Pranašas Eliziejus, pasišaukęs vieną iš pranašų sūnų, jam tarė: “Susijuosk, pasiimk indą aliejaus ir eik į Ramot Gileadą. 
\par 2 Ten surask Nimšio sūnaus Juozapato sūnų Jehuvą. Pasikviesk jį iš jo brolių ir, įsivedęs į vidinį kambarį, 
\par 3 užpilk jam iš indo aliejaus ant galvos, sakydamas: ‘Taip sako Viešpats: ‘Aš tave patepiau Izraelio karaliumi’. Po to, atidaręs duris, išbėk nelaukdamas”. 
\par 4 Jaunasis pranašas atėjo į Ramot Gileadą. 
\par 5 Čia jis rado kariuomenės vadus besėdinčius. Jis tarė: “Turiu tau, vade, ką pasakyti”. Jehuvas paklausė: “Kuriam iš mūsų?” Jis atsakė: “Tau, vade”. 
\par 6 Atsikėlęs Jehuvas įėjo į namus. Tuomet jis išpylė aliejų jam ant galvos ir tarė: “Taip sako Viešpats, Izraelio Dievas: ‘Aš tave patepiau Viešpaties tautos Izraelio karaliumi. 
\par 7 Tu sunaikinsi savo valdovo Ahabo namus ir atkeršysi Jezabelei už pranašų ir visų Viešpaties tarnų kraują. 
\par 8 Visi Ahabo namai pražus. Aš išnaikinsiu Izraelyje visus Ahabo giminės vyrus, kaip vergus, taip ir likusius Izraelyje. 
\par 9 Padarysiu su Ahabo namais, kaip padariau su Nebato sūnaus Jeroboamo ir su Ahijos sūnaus Baašos namais. 
\par 10 Jezabelę suės šunys Jezreelyje ir niekas jos nepalaidos’ ”. Po to jis atidarė duris ir išbėgo. 
\par 11 Jehuvui įėjus pas savo valdovo tarnus, jie klausė: “Kas atsitiko? Ko tas beprotis buvo atėjęs pas tave?” Jis jiems atsakė: “Jūs žinote tą žmogų ir jo kalbą”. 
\par 12 Jie sakė: “Netiesa, pasakyk mums dabar”. Tuomet jis pasakė, ką pranašas kalbėjo, sakydamas: “Taip sako Viešpats: ‘Aš tave patepiau Izraelio karaliumi’ ”. 
\par 13 Tada kiekvienas iš jų skubiai ėmė savo apsiaustą, patiesė jį po jo kojomis ant plikų laiptų ir trimitavo, skelbdami: “Jehuvas yra karalius!” 
\par 14 Taip Nimšio sūnaus Juozapato sūnus Jehuvas surengė sąmokslą prieš Jehoramą, kuris su visu Izraeliu buvo Ramot Gileade prieš Sirijos karalių Hazaelį. 
\par 15 Karalius Jehoramas buvo sugrįžęs į Jezreelį gydytis, nes buvo sužeistas, kai kariavo su Sirijos karaliumi Hazaeliu. Jehuvas sakė: “Jei jūs pritariate, tai tegul niekas neišeina iš miesto, kad nepraneštų Jezreelyje”. 
\par 16 Jehuvas sėdo į vežimą ir išvažiavo į Jezreelį, nes Jehoramas ten gydėsi, o Judo karalius Ahazijas buvo atvykęs Jehoramo lankyti. 
\par 17 Jezreelio bokšte sargybinis, pamatęs Jehuvo būrį atvažiuojant, pranešė: “Matau būrį”. Jehoramas įsakė pasiųsti raitelį ir jų paklausti: “Ar su taika?” 
\par 18 Raitelis, susitikęs su Jehuvo būriu, klausė: “Ar su taika?” Jehuvas atsakė: “Kam tau rūpi taika? Apsisuk ir sek paskui mane”. Sargybinis pranešė: “Pasiuntinys su jais susitiko, bet negrįžta”. 
\par 19 Jehoramas išsiuntė kitą raitelį. Tas, susitikęs su atvykstančiais, klausė: “Ar su taika?” Jehuvas atsakė: “Kam tau rūpi taika? Apsisuk ir sek paskui mane”. 
\par 20 Sargybinis pranešė: “Jis juos pasiekė, bet negrįžta. Važiavimas panašus į Nimšio sūnaus Jehuvo važiavimą, nes važiuoja kaip padūkęs”. 
\par 21 Jehoramas įsakė paruošti vežimą. Izraelio karalius Jehoramas ir Judo karalius Ahazijas išvažiavo kiekvienas savo vežime. Jie sutiko Jehuvą prie jezreeliečio Naboto žemės sklypo. 
\par 22 Jehoramas, pamatęs Jehuvą, paklausė: “Ar su taika, Jehuvai?” Tas atsakė: “Kokia gali būti taika, kol vyksta tavo motinos Jezabelės paleistuvystės ir žyniavimai?” 
\par 23 Jehoramas pasuko vežimą ir norėjo pabėgti, šaukdamas: “Išdavystė, Ahazijau!” 
\par 24 Tačiau Jehuvas įtempė lanką ir iššovė. Strėlė pataikė Jehoramui į širdį ir jis susmuko vežime. 
\par 25 Jehuvas įsakė savo vadui Bidkarui: “Išmesk jį ant jezreeliečio Naboto lauko! Aš prisimenu, kaip mudu važiavome paskui jo tėvą Ahabą ir Viešpats paskelbė prieš jį šį sprendimą: 
\par 26 ‘Taip, kaip Aš vakar čia mačiau Naboto ir jo sūnų kraują, taip Aš tau atlyginsiu ant šito žemės sklypo’. Dabar išmesk jį ant šito sklypo, kaip Viešpats pasakė”. 
\par 27 Judo karalius Ahazijas, tai pamatęs, bėgo Bet Gano link. Jehuvas jį vijosi ir šaukė: “Nušaukite jį!” Jie peršovė jį vežime Gūro įkalnėje, prie Ibleamo. Jis pabėgo į Megidą ir ten mirė. 
\par 28 Jo tarnai parvežė jį į Jeruzalę ir palaidojo prie jo tėvų Dovydo mieste. 
\par 29 Vienuoliktaisiais Ahabo sūnaus Jehoramo metais Ahazijas pradėjo karaliauti Jude. 
\par 30 Jehuvas atvyko į Jezreelį. Jezabelė, tai išgirdusi, išsidažė veidą, pasipuošė galvą ir žiūrėjo pro langą. 
\par 31 Jehuvui įeinant pro vartus, ji klausė: “Ar turėjo Zimris ramybę, nužudęs savo valdovą?” 
\par 32 Jis pažvelgė aukštyn į langą ir sakė: “Kas už mane? Kas?” Du ar trys eunuchai pažiūrėjo žemyn. 
\par 33 Jis įsakė: “Numeskite ją žemyn!” Jie numetė ją. Jos kraujas aptaškė sienas ir žirgus, kurie ją sutrypė. 
\par 34 Po to jis įėjo į vidų, valgė, gėrė ir įsakė: “Eikite ir pažiūrėkite, kad ta prakeiktoji būtų palaidota, nes ji buvo karaliaus duktė”. 
\par 35 Nuėję jos laidoti, jie rado tik kaukolę, kojas ir rankų plaštakas. 
\par 36 Jie sugrįžo ir jam pranešė. Jehuvas tarė: “Tai Viešpaties žodis, kurį Jis kalbėjo per savo tarną tišbietį Eliją: ‘Jezreelyje šunys suės Jezabelės kūną 
\par 37 ir jos lavonas gulės kaip mėšlas Jezreelio laukuose. Ir niekas nesakys, kad tai Jezabelė’ ”.



\chapter{10}


\par 1 Ahabas turėjo septyniasdešimt sūnų Samarijoje. Jehuvas nusiuntė Samarijos miesto vyresniesiems ir Ahabo sūnų globėjams laiškus, kuriuose rašė: 
\par 2 “Gavę šitą laišką, visi, kurie turite savo valdovo sūnų, sustiprintą miestą, kovos vežimų ir ginklų, 
\par 3 pasirinkite geriausią ir tinkamiausią iš savo valdovo sūnų, pasodinkite jį į tėvo sostą ir kariaukite už savo valdovo namus”. 
\par 4 Bet jie labai nusigando ir sakė: “Du karaliai neatsilaikė prieš jį, kaip tad mes atsilaikysime?” 
\par 5 Namų valdytojas, miesto valdytojas, vyresnieji ir sūnų globėjai pasiuntė Jehuvui tokį atsakymą: “Mes esame tavo tarnai ir visa, ką mums įsakysi, darysime. Nė vieno karaliumi nepaskirsime. Daryk, kaip tau patinka”. 
\par 6 Tada Jehuvas parašė jiems antrą laišką: “Jei jūs esate manieji ir klausysite manęs, paimkite savo valdovo sūnų galvas ir rytoj apie šitą laiką atneškite jas man į Jezreelį!” Karaliaus sūnūs, septyniasdešimt vyrų, buvo miesto didžiūnų užauginti. 
\par 7 Gavę šitą laišką, jie nužudė visus karaliaus sūnus, septyniasdešimt vyrų, ir, sudėję jų galvas į pintines, nusiuntė jas Jehuvui į Jezreelį. 
\par 8 Pasiuntinys jam pranešė: “Atnešė karaliaus sūnų galvas”. Jis įsakė: “Sukraukite jas į dvi krūvas vartų įėjime ir palikite iki ryto”. 
\par 9 Rytą Jehuvas, išėjęs ir atsistojęs prie vartų, kalbėjo visiems žmonėms: “Jūs esate nekalti. Aš padariau sąmokslą prieš savo valdovą ir jį nužudžiau. Bet kas nužudė šituos? 
\par 10 Žinokite, kad nė vienas Viešpaties žodis nekris į žemę, ką Viešpats kalbėjo apie Ahabo namus. Dabar Viešpats įvykdė, ką kalbėjo per savo tarną Eliją”. 
\par 11 Jehuvas išžudė visus, kas liko iš Ahabo namų Jezreelyje, visus jo didžiūnus, patikėtinius ir kunigus; nė vieno nepaliko gyvo. 
\par 12 Po to jis ėjo į Samariją. Pakelyje, prie piemenų namų, 
\par 13 Jehuvas sutiko Judo karaliaus Ahazijo brolius ir klausė: “Kas jūs esate?” Jie atsakė: “Mes esame Ahazijo broliai, atėjome aplankyti karaliaus sūnų ir karalienės sūnų”. 
\par 14 Jehuvas įsakė suimti juos gyvus. Jie suėmė juos ir nužudė prie Bet Ekedo šulinio keturiasdešimt du vyrus; nė vieno nepaliko gyvo. 
\par 15 Pasitraukęs iš ten, jis sutiko Rechabo sūnų Jehonadabą. Jis jį pasveikino ir klausė: “Ar tavo širdis nusiteikusi, kaip ir mano širdis dėl tavęs?” Jehonadabas atsakė: “Taip”. Jehuvas tarė: “Jei taip, tai duok ranką”. Jam padavus ranką, Jehuvas padėjo jam įlipti į savo vežimą 
\par 16 ir tarė: “Eime su manimi ir pamatysi, koks aš uolus dėl Viešpaties”. 
\par 17 Atvykęs į Samariją, jis išžudė visus likusius Ahabo gimines Samarijoje pagal Viešpaties žodį, kurį Jis kalbėjo per Eliją. 
\par 18 Jehuvas, sušaukęs visus žmones, jiems tarė: “Ahabas mažai tarnavo Baalui, Jehuvas tarnaus jam daugiau. 
\par 19 Dabar sušaukite pas mane visus Baalo pranašus, visus jo garbintojus ir visus kunigus; žiūrėkite, kad nė vieno netrūktų, nes turiu aukoti Baalui didelę auką. Kas neatvyks, neliks gyvas”. Jehuvas tai darė klastingai, norėdamas išžudyti Baalo garbintojus. 
\par 20 Jis įsakė suruošti Baalui iškilmingą šventę. 
\par 21 Jehuvas apie tai paskelbė visam Izraeliui. Atvyko visi iki vieno Baalo garbintojai, ir Baalo namai buvo pilni nuo vieno krašto iki kito. 
\par 22 Jis įsakė drabužių saugotojui išdalinti drabužius visiems Baalo garbintojams. 
\par 23 Jehuvas, įėjęs su Rechabo sūnumi Jehonadabu į Baalo namus, tarė Baalo garbintojams: “Rūpestingai patikrinkite ir pažiūrėkite, kad čia nebūtų su jumis Viešpaties tarnų, tik vieni Baalo garbintojai”. 
\par 24 Jie įėjo aukoti deginamąsias ir kitas aukas. Jehuvas buvo pastatęs lauke aštuoniasdešimt vyrų ir įsakęs: “Tas vyras, kuris leis ištrūkti bent vienam iš žmonių, kuriuos atiduodu į jūsų rankas, savo gyvybe atsakys už jį”. 
\par 25 Baigus aukoti deginamąją auką, Jehuvas įsakė sargybai ir kariams: “Išžudykite juos! Niekas tenelieka gyvas!” Jie išžudė juos kardu, sargyba bei vyresnieji juos išmetė ir, įėję į Baalo namus, 
\par 26 išnešė atvaizdus iš Baalo namų, juos sudegino, 
\par 27 sudaužė Baalo atvaizdą, sugriovė šventyklą ir toje vietoje padarė išvietes, išlikusias iki šios dienos. 
\par 28 Taip Jehuvas išnaikino Baalą iš Izraelio. 
\par 29 Tačiau nuo nuodėmių Nebato sūnaus Jeroboamo, kuris įtraukė Izraelį į nuodėmę, nuo auksinių veršių Betelyje ir Dane, Jehuvas nepasitraukė. 
\par 30 Viešpats tarė Jehuvui: “Kadangi tu gerai padarei įvykdydamas, kas teisinga mano akyse, ir padarei Ahabo namams visa, kas buvo mano širdyje, tai tavo sūnūs iki ketvirtos kartos sėdės Izraelio soste”. 
\par 31 Tačiau Jehuvas nesilaikė Viešpaties, Izraelio Dievo, įstatymo visa širdimi ir neatsitraukė nuo nuodėmių Jeroboamo, kuris įtraukė Izraelį į nuodėmę. 
\par 32 Tuo laiku Viešpats pradėjo mažinti Izraelį. Hazaelis juos nugalėjo visame pasienyje, 
\par 33 nuo Jordano rytuose, visą Gileado šalį, gadus, rubenus ir manasus, iki Aroero miesto prie Arnono upės, ir Bašaną. 
\par 34 Visi kiti Jehuvo darbai ir jo galia yra aprašyta Izraelio karalių metraščių knygoje. 
\par 35 Jehuvas užmigo prie savo tėvų ir jį palaidojo Samarijoje; jo sūnus Jehoachazas karaliavo jo vietoje. 
\par 36 Jehuvas karaliavo Izraeliui Samarijoje dvidešimt aštuonerius metus.



\chapter{11}

\par 1 Ahazijo motina Atalija, sužinojusi, kad jos sūnus miręs, išžudė visus karaliaus palikuonis. 
\par 2 Bet karaliaus Joramo duktė Jehošeba, Ahazijo sesuo, slapčia paėmė Ahazijo sūnų Jehoašą iš karaliaus sūnų, kurie turėjo būti nužudyti, ir jį paslėpė nuo Atalijos su jo aukle miegamajame, ir jis išliko gyvas. 
\par 3 Jis su ja buvo paslėptas Viešpaties namuose šešerius metus. Atalija valdė kraštą. 
\par 4 Septintais metais Jehojada, pasikvietęs šimtininkus ir sargybos viršininkus, įsivedė juos į Viešpaties namus ir padarė su jais sutartį. Prisaikdinęs Viešpaties namuose, jis parodė jiems karaliaus sūnų. 
\par 5 Po to jis įsakė: “Trečdalis jūsų, kurie įeinate sabato dieną, saugosite karaliaus namus; 
\par 6 trečdalis būsite prie Sūro vartų ir trečdalis prie vartų už sargybos. Taip jūs saugosite namus, kad jie nebūtų sugriauti. 
\par 7 O dvi jūsų dalys iš tų, kurie išeinate sabato dieną, saugos Viešpaties namus ir karalių. 
\par 8 Kiekvienas iš jūsų, laikydamas rankoje paruoštą ginklą, apsupkite karalių ir, jei kas artinsis prie jo įeinant ar išeinant, nužudykite”. 
\par 9 Šimtininkai padarė taip, kaip įsakė kunigas Jehojada: kiekvienas su savo vyrais, kurie sabato dieną įeidavo arba išeidavo, atėjo pas kunigą Jehojadą. 
\par 10 Kunigas padavė šimtininkams karaliaus Dovydo ietis ir skydus, buvusius Viešpaties šventykloje. 
\par 11 Sargybiniai stovėjo su ginklais rankose aplink karalių nuo dešiniojo šventyklos šono iki kairiojo, prie aukuro ir prie šventyklos. 
\par 12 Jis išvedė karaliaus sūnų, uždėjo jam karūną, įteikė liudijimą, patepė jį ir paskelbė karaliumi. Jie plojo rankomis ir šaukė: “Tegyvuoja karalius!” 
\par 13 Atalija, išgirdusi sargybos ir minios šauksmą, atėjo į Viešpaties šventyklą. 
\par 14 Pamačiusi karalių, stovintį ant paaukštinimo prie kolonos, kunigaikščius, trimitininkus prie karaliaus ir krašto žmones, besidžiaugiančius ir trimituojančius, Atalija perplėšė savo drabužius ir šaukė: “Sąmokslas! Sąmokslas!” 
\par 15 Kunigas Jehojada įsakė šimtininkams: “Išveskite ją pro eiles! Kas ją seks, nužudykite kardu”. Nes kunigas sakė: “Kad ji nebūtų nužudyta Viešpaties namuose”. 
\par 16 Jie išvedė ją į karaliaus namus pro Žirgų vartus ir nužudė. 
\par 17 Jehojada padarė sandorą tarp Viešpaties ir karaliaus bei tautos, kad jie bus Viešpaties tauta, ir tarp karaliaus ir tautos. 
\par 18 Po to visi žmonės nuėjo į Baalo namus, juos sugriovė, aukurus ir atvaizdus sudaužė ir Baalo kunigą Mataną nužudė prie aukurų. Kunigas paskyrė prižiūrėtojus Viešpaties namams. 
\par 19 Jis kartu su šimtininkais, kariuomenės vadais, sargybiniais ir visais žmonėmis nuvedė karalių iš Viešpaties namų pro sargybos vartus į karaliaus namus, kur jis atsisėdo į karaliaus sostą. 
\par 20 Visi krašto žmonės džiaugėsi, ir mieste buvo ramu. Atalija buvo kardu nužudyta karaliaus namuose. 
\par 21 Pradėdamas karaliauti, Jehoašas buvo septynerių metų.



\chapter{12}

\par 1 Septintaisiais Jehuvo metais Jehoašas tapo karaliumi ir valdė Judą keturiasdešimt metų, gyvendamas Jeruzalėje. Jo motina buvo vardu Cibija iš Beer Šebos. 
\par 2 Jehoašas darė tai, kas teisinga Viešpaties akyse, per visas dienas, kol jį mokė kunigas Jehojada. 
\par 3 Tik aukštumos nebuvo pašalintos, žmonės vis dar aukojo ir smilkė aukštumose. 
\par 4 Jehoasas įsakė kunigams: “Visus pašvęstus pinigus, atneštus į Viešpaties namus, praeinančiųjų pinigus, asmens mokesčius, savo noru duodamas šventyklai aukas 
\par 5 turi surinkti kunigai, kiekvienas iš savo pažįstamų. Tegul jie užtaiso visas spragas namuose, kur tik suras kokių spragų”. 
\par 6 Dvidešimt trečiaisiais karaliaus Jehoašo metais kunigai dar nebuvo užtaisę namų spragų. 
\par 7 Todėl karalius Jehoašas, pasišaukęs kunigą Jehojadą ir kitus kunigus, jiems tarė: “Kodėl jūs neužtaisote namų spragų? Nuo šiolei nebeimkite pinigų iš savo pažįstamų, bet atiduokite juos namų spragoms užtaisyti”. 
\par 8 Kunigai nuo to laiko neberinko iš žmonių pinigų ir nebetaisė namų spragų. 
\par 9 Kunigas Jehojada paėmė dėžę, išgręžė jos dangtyje skylę ir ją pastatė šalia aukuro, dešinėje pusėje prie įėjimo į Viešpaties namus. Kunigai, budėję prie įėjimo, įmesdavo į ją visus pinigus, atneštus į Viešpaties namus. 
\par 10 Pamatę, kad dėžėje daug pinigų, karaliaus raštininkas ir vyriausiasis kunigas atėję suskaičiuodavo Viešpaties namuose rastus pinigus ir, užrišę maišuose, paimdavo. 
\par 11 Suskaičiuotus pinigus atiduodavo darbų prižiūrėtojams Viešpaties namuose, jie sumokėdavo dailidėms ir statybininkams, dirbusiems Viešpaties namuose, 
\par 12 mūrininkams bei akmens tašytojams, o taip pat pirkdavo rąstus bei tašytus akmenis Viešpaties namų spragoms užtaisyti ir apmokėdavo visas remonto išlaidas. 
\par 13 Už surenkamus Viešpaties namuose pinigus nebuvo daromi reikmenys Viešpaties namams: sidabriniai dubenys, gnybtuvai, šlakstytuvai, trimitai ir auksiniai bei sidabriniai indai. 
\par 14 Pinigus atiduodavo darbininkams ir už juos atlikdavo Viešpaties namų remontą. 
\par 15 Nebuvo reikalaujama ataskaitos iš tų, kurie pinigus išmokėdavo, nes jie elgėsi ištikimai. 
\par 16 Piniginės aukos už nusikaltimus ir nuodėmes nebuvo įnešamos į Viešpaties namus, bet priklausė kunigams. 
\par 17 Kartą Sirijos karalius Hazaelis kariavo prieš Gatą ir jį užėmė. Užėmęs Gatą, Hazaelis atsigręžė prieš Jeruzalę. 
\par 18 Judo karalius Jehoašas paėmė visas pašvęstas dovanas, kurias pašventė jo tėvai Judo karaliai Juozapatas ir Jehoramas, bei savo paties pašvęstas dovanas ir visą auksą iš Viešpaties bei karaliaus namų iždų ir nusiuntė Sirijos karaliui Hazaeliui. Tada jis pasitraukė nuo Jeruzalės. 
\par 19 Visi kiti Jehoašo darbai yra surašyti Judo karalių metraščių knygoje. 
\par 20 Jo tarnai padarė sąmokslą, sukilo ir nužudė Jehoašą Milojo namuose, pakelėje į Silą. 
\par 21 Jį nužudė jo tarnai: Šimato sūnus Jozakaras ir Šomero sūnus Jehozabadas. Ir jis buvo palaidotas prie savo tėvų Dovydo mieste, o jo sūnus Amacijas karaliavo jo vietoje.



\chapter{13}

\par 1 Dvidešimt trečiaisiais Judo karaliaus Jehoašo, Ahazijo sūnaus, metais Jehuvo sūnus Jehoachazas pradėjo karaliauti Izraelyje, Samarijoje, ir karaliavo septyniolika metų. 
\par 2 Jis darė pikta Viešpaties akyse ir sekė nuodėmėmis Nebato sūnaus Jeroboamo, kuris įtraukė į nuodėmę Izraelį, ir neatsitraukė nuo jų. 
\par 3 Viešpaties rūstybė užsidegė prieš Izraelį, ir Jis juos atidavė Sirijos karaliui Hazaeliui ir jo sūnui Ben Hadadui per visas jų dienas. 
\par 4 Jehoachazas meldėsi, ir Viešpats jį išklausė, nes matė, kaip Izraelis vargo Sirijos priespaudoje. 
\par 5 Viešpats davė Izraeliui gelbėtoją, ir jie buvo išlaisvinti iš Sirijos rankos; tuomet izraelitai gyveno savo palapinėse kaip anksčiau. 
\par 6 Tačiau jie neatsitraukė nuo nuodėmių Jeroboamo namų, kuris įtraukė Izraelį į nuodėmę, ir gyveno jose, ir neišnaikino giraitės Samarijoje. 
\par 7 Sirijos karalius paliko Jehoachazui tik penkiasdešimt raitelių, dešimt kovos vežimų ir dešimt tūkstančių pėstininkų, nes jis juos sunaikino ir padarė kaip dulkes kūlimo metu. 
\par 8 Visi kiti Jehoachazo darbai ir jo galia yra aprašyti Izraelio karalių metraščių knygoje. 
\par 9 Jehoachazas užmigo prie savo tėvų ir buvo palaidotas Samarijoje, o jo sūnus Jehoašas karaliavo jo vietoje. 
\par 10 Trisdešimt septintaisiais Judo karaliaus Jehoašo metais Jehoachazo sūnus Jehoašas pradėjo karaliauti Izraelyje, Samarijoje, ir valdė kraštą šešiolika metų. 
\par 11 Jis darė pikta Viešpaties akyse ir neatsitraukė nuo nuodėmių Nebato sūnaus Jeroboamo, kuris įtraukė Izraelį į nuodėmę, ir vaikščiojo jose. 
\par 12 Visi kiti Jehoašo darbai ir jo galia, kariaujant su Judo karaliumi Amaciju, yra aprašyti Izraelio karalių metraščių knygoje. 
\par 13 Jehoašas užmigo prie savo tėvų ir jo sūnus Jeroboamas atsisėdo į jo sostą. Jehoašą palaidojo Samarijoje prie Izraelio karalių. 
\par 14 Eliziejus susirgo liga, nuo kurios vėliau jis numirė. Izraelio karalius Jehoašas atėjo pas jį, verkė prie jo ir sakė: “Mano tėve, mano tėve, Izraelio kovos vežimai ir jo raiteliai!” 
\par 15 Eliziejus sakė jam: “Paimk lanką ir strėlių”. Jehoašas paėmė lanką ir strėlių. 
\par 16 Tuomet Eliziejus sakė Izraelio karaliui: “Uždėk savo ranką ant lanko”. Jis uždėjo ranką, o Eliziejus uždėjo savo rankas ant karaliaus rankų 
\par 17 ir tarė: “Atidaryk langą į rytus ir šauk”. Jis atidarė ir šovė. Eliziejus sakė: “Tai Viešpaties išlaisvinimo strėlė, išlaisvinimo iš Sirijos strėlė. Tu nugalėsi sirus Afeke ir juos pribaigsi”. 
\par 18 Po to jis sakė: “Paimk strėles ir suduok į žemę”. Karalius paėmė ir tris kartus sudavė strėlėmis į žemę. 
\par 19 Dievo vyras supykęs tarė: “Reikėjo suduoti penkis ar šešis kartus, tada būtum nugalėjęs Siriją ir juos pribaigęs. O dabar tik tris kartus ją nugalėsi”. 
\par 20 Eliziejus mirė ir buvo palaidotas. Tais metais Moabo būriai įsiveržė į Izraelį. 
\par 21 Kartą, laidojant vyrą, žmonės pamatė karių būrį ir įmetė tą vyrą į Eliziejaus kapą. Kai tas vyras kape palietė Eliziejaus kaulus, atgijo ir atsistojo. 
\par 22 Sirijos karalius Hazaelis spaudė Izraelį per visas Jehoachazo dienas. 
\par 23 Tačiau Viešpats buvo jiems maloningas, jų gailėjosi ir pažvelgė į juos dėl savo sandoros su Abraomu, Izaoku ir Jokūbu. Jis nenorėjo jų sunaikinti ir iki šiol jų neatmetė nuo savęs. 
\par 24 Sirijos karalius Hazaelis mirė, ir jo sūnus Ben Hadadas pradėjo karaliauti jo vietoje. 
\par 25 Jehoachazo sūnus Jehoašas atsiėmė iš Hazaelio sūnaus Ben Hadado miestus, kuriuos jis buvo karu paėmęs iš jo tėvo Jehoachazo. Tris kartus Jehoašas jį nugalėjo ir atgavo Izraelio miestus.



\chapter{14}


\par 1 Antraisiais Izraelio karaliaus Jehoašo, Jehoachazo sūnaus, metais Jehoašo sūnus Amacijas karaliavo Jude. 
\par 2 Pradėdamas karaliauti, jis buvo dvidešimt penkerių metų amžiaus ir karaliavo Jeruzalėje dvidešimt devynerius metus. Jo motina buvo vardu Jehoadina, iš Jeruzalės. 
\par 3 Amacijas darė tai, kas teisinga Viešpaties akyse, tačiau ne taip, kaip jo tėvas Dovydas. Jis visur elgėsi taip, kaip jo tėvas Jehoašas. 
\par 4 Aukštumos nebuvo panaikintos, ir žmonės vis dar aukojo ir smilkė jose. 
\par 5 Kai karalystė įsitvirtino jo rankose, jis nužudė tuos tarnus, kurie nužudė jo tėvą karalių. 
\par 6 Tačiau jų vaikų jis nenužudė, laikydamasis to, kas parašyta Mozės įstatymo knygoje, kur Viešpats sako: “Nežudykite tėvų už vaikus, nė vaikų už tėvus, bet kiekvienas turi mirti už savo nuodėmę”. 
\par 7 Jis nugalėjo Druskos slėnyje edomitus, nukaudamas jų dešimt tūkstančių, ir užkariavo Selą. Jis ją pavadino Jokteeliu, ir taip ji vadinama iki šios dienos. 
\par 8 Tuomet Amacijas siuntė pasiuntinius pas Jehuvo sūnaus Jehoachazo sūnų Jehoasą, Izraelio karalių, ir sakė jam: “Išeik, kad susitiktume veidas į veidą”. 
\par 9 Izraelio karalius Jehoašas siuntė Judo karaliui Amacijui tokį atsakymą: “Libane auganti usnis siuntė pas Libano kedrą, sakydama: ‘Leisk savo dukterį už mano sūnaus’. Bet Libano laukinis žvėris prabėgdamas sutrypė usnį. 
\par 10 Tu nugalėjai edomitus, todėl pasididžiavo tavo širdis. Džiaukis tuo ir pasilik namie. Kodėl nori prisišaukti nelaimę ir žūti kartu su Judu?” 
\par 11 Amacijas nepaklausė, todėl Izraelio karalius Jehoasas atėjo. Jis ir Judo karalius Amacijas susitiko veidas į veidą Bet Šemeše, kuris priklauso Judui. 
\par 12 Izraelis nugalėjo Judą, ir jo vyrai pabėgo į savo palapines. 
\par 13 Judo karalių Amaciją, Ahazijo sūnaus Jehoašo sūnų, Izraelio karalius Jehoašas paėmė į nelaisvę Bet Šemeše; tada atėjo į Jeruzalę ir nugriovė Jeruzalės sieną nuo Efraimo vartų iki Kampo vartų, keturis šimtus uolekčių. 
\par 14 Jis pasiėmė visą auksą, sidabrą, visus indus, rastus Viešpaties namuose ir karaliaus namų ižde, bei įkaitus ir sugrįžo į Samariją. 
\par 15 Visi kiti Jehoašo darbai, jo galia ir jo kova su Judo karaliumi Amaciju yra aprašyti Izraelio karalių metraščių knygoje. 
\par 16 Jehoašas užmigo prie savo tėvų ir buvo palaidotas Samarijoje prie Izraelio karalių, o jo sūnus Jeroboamas karaliavo jo vietoje. 
\par 17 Jehoašo sūnus Amacijas, Judo karalius, mirus Jehoachazo sūnui Jehoašui, Izraelio karaliui, dar gyveno penkiolika metų. 
\par 18 Visi kiti Amacijo darbai surašyti Judo karalių metraščių knygoje. 
\par 19 Jeruzalėje prieš jį kilo sąmokslas, ir Amacijas pabėgo į Lachišą. Bet jie pasiuntė į Lachišą ir jį ten nužudė. 
\par 20 Jo kūną pargabeno ant žirgų Jeruzalėn ir palaidojo prie jo tėvų Dovydo mieste. 
\par 21 Tada visi Judo žmonės ėmė Azariją, kuriam buvo šešiolika metų, ir padarė savo karaliumi vietoje jo tėvo Amacijo. 
\par 22 Jis sutvirtino Elatą ir sugrąžino jį Judui po to, kai karalius užmigo prie savo tėvų. 
\par 23 Jeroboamas, Izraelio karaliaus Jehoašo sūnus, pradėjo karaliauti Samarijoje penkioliktaisiais Judo karaliaus Amacijo, Jehoašo sūnaus, metais. Jis valdė keturiasdešimt vienerius metus. 
\par 24 Jis darė pikta Viešpaties akyse ir neatsitraukė nuo nuodėmių Nebato sūnaus Jeroboamo, kuris įtraukė Izraelį į nuodėmę. 
\par 25 Jis atstatė Izraelio ribas nuo Hamato iki lygumos jūros pagal Viešpaties, Izraelio Dievo, žodį, kurį Jis paskelbė per savo tarną pranašą Joną, Amitajo sūnų, iš Get Hefero. 
\par 26 Viešpats matė didelį Izraelio vargą, kad nebeliko nei vergų, nei laisvųjų ir kad nebuvo nė vieno, kas padėtų Izraeliui. 
\par 27 Viešpats nesakė, kad išnaikins Izraelio vardą nuo žemės paviršiaus, bet Jis išgelbėjo juos per Jehoašo sūnų Jeroboamą. 
\par 28 Visi kiti Jeroboamo darbai, jo galia, kovos, kaip jis sugrąžino Izraeliui Damaską ir Hamatą, yra aprašyta Izraelio karalių metraščių knygoje. 
\par 29 Jeroboamas užmigo prie savo tėvų, Izraelio karalių, ir jo sūnus Zacharija karaliavo jo vietoje.



\chapter{15}

\par 1 Dvidešimt septintaisiais Izraelio karaliaus Jeroboamo metais Jude pradėjo karaliauti karaliaus Amacijo sūnus Azarija. 
\par 2 Pradėdamas valdyti, jis buvo šešiolikos metų ir penkiasdešimt dvejus metus karaliavo Jeruzalėje. Jo motina buvo vardu Jecholija, iš Jeruzalės. 
\par 3 Jis darė tai, kas teisinga Viešpaties akyse, kaip ir jo tėvas Amacijas. 
\par 4 Tik aukštumos nebuvo panaikintos, žmonės vis dar aukojo ir smilkė aukštumose. 
\par 5 Viešpats ištiko karalių raupsais. Iki mirties jis gyveno atskiruose namuose, o jo sūnus Jotamas valdė jo namus ir teisė krašto žmones. 
\par 6 Visi kiti Azarijos darbai surašyti Judo karalių metraščių knygoje. 
\par 7 Azarija užmigo prie savo tėvų ir buvo palaidotas prie savo tėvų Dovydo mieste; jo sūnus Jotamas karaliavo jo vietoje. 
\par 8 Trisdešimt aštuntaisiais Judo karaliaus Azarijos metais Jeroboamo sūnus Zacharija šešis mėnesius karaliavo Izraelyje, Samarijoje. 
\par 9 Jis darė pikta Viešpaties akyse kaip jo tėvai ir neatsitraukė nuo nuodėmių Nebato sūnaus Jeroboamo, kuris įtraukė Izraelį į nuodėmę. 
\par 10 Jabešo sūnus Šalumas surengė sąmokslą, užpuolė žmonių akivaizdoje Zachariją, nužudė jį ir karaliavo jo vietoje. 
\par 11 Visi kiti Zacharijos darbai surašyti Izraelio karalių metraščių knygoje. 
\par 12 Toks buvo Viešpaties žodis, kurį Jis kalbėjo Jehuvui: “Iki ketvirtosios kartos tavo sūnūs sėdės Izraelio soste”. Ir taip įvyko. 
\par 13 Jabešo sūnus Šalumas pradėjo karaliauti Izraelyje trisdešimt devintaisiais Judo karaliaus Uzijo metais ir karaliavo Samarijoje vieną mėnesį. 
\par 14 Gadžio sūnus Menahemas, išėjęs iš Tircos, atėjo į Samariją; jis, užpuolęs Jabešo sūnų Šalumą, nužudė jį ir karaliavo jo vietoje. 
\par 15 Visi kiti Šalumo darbai ir jo surengtas sąmokslas yra aprašyta Izraelio karalių metraščių knygoje. 
\par 16 Menahemas sunaikino Tifsacho miestą ir visą kraštą nuo Tircos; gyventojus išžudė, nes jie neatidarė jam miesto vartų, o nėščias moteris perskrodė. 
\par 17 Trisdešimt devintaisiais Judo karaliaus Azarijos metais Izraelyje pradėjo karaliauti Gadžio sūnus Menahemas ir karaliavo Samarijoje dešimt metų. 
\par 18 Jis darė pikta Viešpaties akyse ir neatsitraukė nuo nuodėmių Nebato sūnaus Jeroboamo, kuris įtraukė Izraelį į nuodėmę. 
\par 19 Asirijos karalius Pulas užpuolė kraštą. Menahemas davė jam tūkstantį talentų sidabro, kad jo ranka būtų su juo ir padėtų įsitvirtinti valdžioje. 
\par 20 Menahemas visus pasiturinčius gyventojus apdėjo mokesčiais, kiekvieną po penkiasdešimt šekelių sidabro, kad galėtų sumokėti Asirijos karaliui. Taip Asirijos karalius apsisuko ir pasitraukė iš krašto. 
\par 21 Visi kiti Menahemo darbai surašyti Izraelio karalių metraščių knygoje. 
\par 22 Menahemas užmigo prie savo tėvų, ir jo sūnus Pekachija karaliavo jo vietoje. 
\par 23 Penkiasdešimtaisiais Judo karaliaus Azarijos metais Izraelio karaliumi Samarijoje tapo Menahemo sūnus Pekachija ir karaliavo dvejus metus. 
\par 24 Jis darė pikta Viešpaties akyse ir nepasitraukė nuo nuodėmių Nebato sūnaus Jeroboamo, kuris įtraukė Izraelį į nuodėmę. 
\par 25 Prieš jį surengė sąmokslą Remalijo sūnus Pekachas, jo karo vadas. Jis susitarė su Argobu, Arjė bei penkiasdešimt gileadiečių ir nužudė jį Samarijoje, karaliaus rūmuose. Ir jis karaliavo jo vietoje. 
\par 26 Visi kiti Pekachijos darbai surašyti Izraelio karalių metraščių knygoje. 
\par 27 Penkiasdešimt antraisiais Judo karaliaus Azarijos metais Izraelyje ir Samarijoje pradėjo karaliauti Remalijo sūnus Pekachas ir karaliavo dvidešimt metų. 
\par 28 Pekachas darė pikta Viešpaties akyse ir neatsitraukė nuo nuodėmių Nebato sūnaus Jeroboamo, kuris įtraukė Izraelį į nuodėmę. 
\par 29 Izraelio karaliaus Pekacho dienomis Asirijos karalius Tiglat Pileseras įsiveržė ir paėmė Ijoną, Abel Bet Maaką, Janoachą, Kedešą, Hasorą, Gileadą, Galilėją ir visą Naftalio kraštą, o žmones išsivedė į Asiriją. 
\par 30 Elos sūnus Ozėjas surengė sąmokslą prieš Remalijo sūnų Pekachą, nužudė jį ir karaliavo jo vietoje dvidešimtaisiais Uzijo sūnaus Jotamo metais. 
\par 31 Visi kiti Pekacho darbai surašyti Izraelio karalių metraščių knygoje. 
\par 32 Antraisiais Remalijo sūnaus Pekacho, Izraelio karaliaus, metais pradėjo karaliauti Judo karaliaus Uzijo sūnus Jotamas. 
\par 33 Jis pradėjo karaliauti dvidešimt penkerių metų ir karaliavo Jeruzalėje šešiolika metų. Jo motina buvo Cadoko duktė Jeruša. 
\par 34 Jis darė tai, kas teisinga Viešpaties akyse, kaip ir jo tėvas Uzijas. 
\par 35 Tačiau aukštumos nebuvo panaikintos, žmonės vis dar smilkė ir aukojo aukštumose. Jis pastatė aukštutinius Viešpaties namų vartus. 
\par 36 Visi kiti Jotamo darbai yra surašyti Judo karalių metraščių knygoje. 
\par 37 Tomis dienomis Viešpats pradėjo siųsti prieš Judą Sirijos karalių Reciną ir Remalijo sūnų Pekachą. 
\par 38 Jotamas užmigo prie savo tėvų ir buvo palaidotas prie savo tėvų savo tėvo Dovydo mieste; jo sūnus Ahazas pradėjo karaliauti jo vietoje.



\chapter{16}

\par 1 Septynioliktaisiais Remalijo sūnaus Pekacho metais pradėjo karaliauti Judo karaliaus Jotamo sūnus Ahazas. 
\par 2 Pradėdamas karaliauti, jis buvo dvidešimties metų ir karaliavo Jeruzalėje šešiolika metų. Jis nedarė to, kas teisinga Viešpaties, jo Dievo, akyse, kaip darė jo tėvas Dovydas, 
\par 3 bet vaikščiojo Izraelio karalių keliais, net leido savo sūnui eiti per ugnį, mėgdžiodamas bjaurystes pagonių, kuriuos Viešpats išvarė prieš izraelitams užimant tą kraštą. 
\par 4 Jis aukojo ir smilkė aukštumose, ant kalvų ir po kiekvienu žaliuojančiu medžiu. 
\par 5 Sirijos karalius Recinas ir Remalijo sūnus Pekachas, Izraelio karalius, atėję prieš Jeruzalę, apgulė miestą, bet neįstengė jo paimti. 
\par 6 Sirijos karalius Recinas sugrąžino Elatą Sirijai ir, išvaręs žydus iš Elato, jį apgyvendino edomitais, kurie ten tebegyvena iki šios dienos. 
\par 7 Tada Ahazas siuntė pasiuntinius pas Asirijos karalių Tiglat Pileserą, sakydamas: “Aš esu tavo tarnas ir tavo sūnus. Ateik ir išgelbėk mane iš Sirijos ir Izraelio karalių rankų”. 
\par 8 Ahazas paėmė auksą ir sidabrą, kuriuos rado Viešpaties namuose ir karaliaus namų ižde, ir pasiuntė dovaną Asirijos karaliui. 
\par 9 Asirijos karalius paklausė jo ir, užpuolęs Damaską, paėmė jį, gyventojus išvedė į Kyrą, o Reciną nužudė. 
\par 10 Karalius Ahazas nuvyko į Damaską pasitikti Asirijos karaliaus Tiglat Pilesero. Pamatęs Damaske aukurą, karalius Ahazas pasiuntė kunigui Ūrijai aukuro atvaizdą, jo pavyzdį ir brėžinius su visomis detalėmis. 
\par 11 Kunigas Ūrija pastatė aukurą pagal pavyzdį, kurį karalius Ahazas atsiuntė iš Damasko, prieš Ahazui grįžtant iš Damasko. 
\par 12 Karalius, sugrįžęs iš Damasko ir pamatęs aukurą, priėjo prie jo ir aukojo ant jo. 
\par 13 Jis aukojo deginamąją bei duonos auką, išliejo geriamąją auką ir šlakstė padėkos aukų kraują ant aukuro. 
\par 14 Varinį aukurą, kuris buvo priešais Viešpatį, perkėlė į kitą vietą, į šiaurę nuo naujo aukuro. 
\par 15 Karalius Ahazas įsakė kunigui Ūrijai: “Ant naujojo aukuro aukok deginamąją auką rytą ir duonos auką vakare, karaliaus deginamąją bei duonos auką ir krašto gyventojų deginamąsias, duonos ir geriamąsias aukas, o aukų kraują šlakstyk ant jo. Varinį aukurą palik mano nuožiūrai”. 
\par 16 Kunigas Ūrija padarė, kaip jam įsakė karalius Ahazas. 
\par 17 Karalius Ahazas supjaustė stovų rėmus ir nuėmė nuo jų praustuves. Jis taip pat nukėlė baseiną nuo varinių jaučių, kurie buvo po juo, ir padėjo jį ant akmeninio grindinio. 
\par 18 Be to, sabato pastogę, kuri buvo pastatyta namuose, ir karaliaus įėjimą jis pašalino iš Viešpaties namų dėl Asirijos karaliaus. 
\par 19 Visi kiti Ahazo darbai yra surašyti Judo karalių metraščių knygoje. 
\par 20 Ahazas užmigo prie savo tėvų ir buvo palaidotas Dovydo mieste; jo sūnus Ezekijas karaliavo jo vietoje.



\chapter{17}

\par 1 Dvyliktaisiais Judo karaliaus Ahazo metais Izraelyje ir Samarijoje pradėjo karaliauti Elos sūnus Ozėjas ir karaliavo devynerius metus. 
\par 2 Jis darė pikta Viešpaties akyse, bet ne taip, kaip Izraelio karaliai, buvę prieš jį. 
\par 3 Asirijos karalius Šalmaneseras kariavo prieš jį, ir Ozėjas, tapęs jo tarnu, mokėjo jam duoklę. 
\par 4 Asirijos karalius pastebėjo, kad Ozėjas rengia sąmokslą, nes pasiuntė pasiuntinius pas Egipto karalių Soją ir nesumokėjo duoklės kaip ankstesniais metais. Todėl Asirijos karalius suėmė jį ir uždarė į kalėjimą. 
\par 5 Asirijos karalius užėmė visą kraštą, pasiekė Samariją ir laikė ją apgulęs trejus metus. 
\par 6 Devintaisiais Ozėjo metais Asirijos karalius paėmė Samariją ir išvedė izraelitus į Asiriją, juos apgyvendino Halache, Habore prie Gozano upės ir medų miestuose. 
\par 7 Taip įvyko dėl to, kad izraelitai nusidėjo Viešpačiui, savo Dievui, kuris juos išvedė iš Egipto žemės ir faraono priespaudos, ir garbino svetimus dievus. 
\par 8 Jie elgėsi pagal papročius pagonių, kuriuos Viešpats išvarė prieš izraelitams užimant kraštą, ir pagal Izraelio karalių įvestus papročius. 
\par 9 Izraelitai slapta darė tai, kas nebuvo teisinga Viešpaties, jų Dievo, akyse. Jie pasidarė aukštumas visuose savo miestuose, pradedant sargybų bokštais ir baigiant sutvirtintais miestais. 
\par 10 Jie pasistatė atvaizdus ir pasidarė alkus kiekvienoje kalvoje ir po kiekvienu žaliuojančiu medžiu. 
\par 11 Ten jie smilkė aukštumose kaip pagonys, kuriuos Viešpats išvarė priešais juos, ir darė nedorybes, sukeldami Viešpaties pyktį. 
\par 12 Jie tarnavo stabams, apie kuriuos Viešpats buvo jiems pasakęs: “Nedarykite tokių”. 
\par 13 Viešpats įspėjo Izraelį ir Judą per savo pranašus ir regėtojus, sakydamas: “Nusisukite nuo savo piktų kelių ir laikykitės mano įsakymų bei nuostatų pagal įsakymą, kurį daviau jūsų tėvams ir siunčiau per savo tarnus pranašus”. 
\par 14 Tačiau jie neklausė, bet sukietino savo sprandus, kaip ir jų tėvai, kurie netikėjo Viešpačiu, savo Dievu. 
\par 15 Jie atmetė Jo nuostatus ir sandorą, kurią Jis buvo padaręs su jų tėvais, ir Jo įspėjimus, kuriais juos įspėdavo. Jie sekė tuštybe ir patys tapo tušti, nuėjo paskui aplinkines tautas, ko Viešpats buvo įsakęs nedaryti. 
\par 16 Jie paliko visus Viešpaties, savo Dievo, įsakymus ir pasidarė nulietus veršius ir alkus, garbino dangaus kareiviją ir tarnavo Baalui. 
\par 17 Jie leido savo sūnus bei dukteris per ugnį, žyniavo ir kerėjo, ir parsidavė daryti pikta Viešpaties akyse, sukeldami Jo pyktį. 
\par 18 Todėl Viešpats labai užsirūstino ant Izraelio ir pašalino juos iš savo akivaizdos. Nieko neliko, išskyrus Judo giminę. 
\par 19 Taip pat ir Judas nesilaikė Viešpaties, savo Dievo, įsakymų, bet elgėsi pagal Izraelyje priimtus papročius. 
\par 20 Viešpats atmetė visus Izraelio palikuonis ir bausdamas atidavė juos į plėšikų rankas, kol jie buvo pašalinti iš Jo akivaizdos. 
\par 21 Jis atplėšė Izraelį nuo Dovydo namų, ir jie paskelbė savo karaliumi Nebato sūnų Jeroboamą. Jeroboamas nukreipė Izraelį nuo Viešpaties ir juos įtraukė į didelę nuodėmę. 
\par 22 Izraelitai vaikščiojo visose Jeroboamo nuodėmėse ir nepaliko jų, 
\par 23 kol Viešpats atstūmė Izraelį nuo savęs, kaip kalbėjo per savo tarnus pranašus. Taip Izraelis buvo perkeltas iš savo žemės į Asiriją, kur jie pasiliko iki šios dienos. 
\par 24 Asirijos karalius atkėlė į Samarijos miestus izraelitų vieton žmonių iš Babilono, Kuto, Avos, Hamato ir Sefarvaimo. Jie įsikūrė Samarijoje ir gyveno jos miestuose. 
\par 25 Ten apsigyvenę, jie nebijojo Viešpaties, todėl Viešpats siuntė tarp jų liūtų, kurie juos žudė. 
\par 26 Asirijos karaliui buvo pranešta: “Tautos, kurias atkėlei į Samarijos miestus, nežino apie to krašto Dievą, todėl Jis siuntė liūtų, kurie žudo juos, kadangi jie nežino tos šalies Dievo”. 
\par 27 Asirijos karalius įsakė: “Nusiųskite ten vieną jų kunigą, kurį atvedėte iš ten, kad jis nuvykęs ten apsigyventų ir juos pamokytų apie to krašto Dievą”. 
\par 28 Vienas kunigas iš tų, kurie buvo išvesti iš Samarijos, sugrįžęs apsigyveno Betelyje ir mokė juos, kaip reikia bijoti Viešpaties. 
\par 29 Tačiau kiekviena tauta pasidarė savo dievus ir pasistatė juos samariečių aukštumų namuose, kiekviena tauta tame mieste, kur ji gyveno. 
\par 30 Babiloniečiai garbino Sukot Benotą, Kuto žmonės­Nergalą, Hamato žmonės­Ašimą, 
\par 31 aviečiai­Nibhazą ir Tartaką, o Sefarvaimo žmonės degindavo savo vaikus Sefarvaimo dievams Adramelechui ir Anamelechui. 
\par 32 Tačiau jie bijojo ir Viešpaties. Aukštumų kunigais jie paskyrė prasčiausius iš savųjų, kurie aukodavo už juos aukštumų namuose. 
\par 33 Jie bijojo Viešpaties ir tarnavo savo dievams, kaip buvo pratę savame krašte. 
\par 34 Iki šios dienos jie laikosi savo senų įpročių. Jie nebijo Viešpaties ir nevykdo Jo nuostatų ir potvarkių, įstatymų ir įsakymų, kuriuos Viešpats davė sūnums Jokūbo, kurį Jis pavadino Izraeliu. 
\par 35 Viešpats padarė su jais sandorą ir jiems įsakė: “Nebijokite kitų dievų, nesilenkite prieš juos, netarnaukite ir neaukokite jiems. 
\par 36 Bet bijokite Viešpaties, kuris jus išvedė iš Egipto šalies savo galinga jėga ir ištiesta ranka; Jį garbinkite ir Jam aukokite. 
\par 37 Nuostatus ir potvarkius, įstatymus ir įsakymus, kuriuos jums surašiau, saugokite ir vykdykite per amžius, ir nebijokite kitų dievų. 
\par 38 Nepamirškite sandoros, kurią padariau su jumis, ir nebijokite kitų dievų. 
\par 39 Viešpaties, savo Dievo, bijokite, ir Jis išgelbės jus iš visų jūsų priešų”. 
\par 40 Tačiau jie neklausė, bet elgėsi pagal savo įpročius. 
\par 41 Tos tautos bijojo Viešpaties, bet tarnavo ir savo drožtiems atvaizdams. Taip darė jų vaikai ir vaikų vaikai, taip jie tebedaro iki šios dienos.



\chapter{18}

\par 1 Trečiaisiais Elos sūnaus Ozėjo, Izraelio karaliaus, metais pradėjo karaliauti Judo karaliaus Ahazo sūnus Ezekijas. 
\par 2 Pradėdamas karaliauti, jis buvo dvidešimt penkerių metų ir valdė Jeruzalėje dvidešimt devynerius metus. Jo motina buvo vardu Abija, Zacharijos duktė. 
\par 3 Jis darė tai, kas teisinga Viešpaties akyse, kaip ir jo tėvas Dovydas. 
\par 4 Ezekijas panaikino aukštumas, sudaužė atvaizdus, iškirto giraites, į gabalus sudaužė varinę gyvatę, kurią padarė Mozė. Nes iki to laiko izraelitai dar tebesmilkė jai ir vadino ją Nehuštanu. 
\par 5 Ezekijas pasitikėjo Viešpačiu, Izraelio Dievu. Tokio karaliaus Jude nebuvo nei iki jo, nei po jo. 
\par 6 Jis glaudėsi prie Viešpaties ir nepaliovė sekti Jį bei vykdė įsakymus, kuriuos Viešpats davė Mozei. 
\par 7 Viešpats buvo su juo, ir visur, kur jis ėjo, jam sekėsi. Jis sukilo prieš Asirijos karalių ir netarnavo jam. 
\par 8 Ezekijas nugalėjo filistinus iki Gazos ir jos apylinkių, nuo sargybų bokštų iki sutvirtintų miestų. 
\par 9 Ketvirtaisiais karaliaus Ezekijo metais, kurie buvo septintieji Izraelio karaliaus Ozėjo, Elos sūnaus, metai, Asirijos karalius Šalmaneseras atėjo prieš Samariją ir ją apgulė. 
\par 10 Trečiųjų metų pabaigoje ją paėmė. Tai įvyko šeštaisiais Ezekijo ir devintaisiais Izraelio karaliaus Ozėjo metais. 
\par 11 Asirijos karalius išvedė izraelitus į Asiriją ir juos apgyvendino Halache, Habore prie Gozano upės ir medų miestuose, 
\par 12 nes jie nepakluso Viešpaties, savo Dievo, balsui ir sulaužė Jo sandorą; jie neklausė ir nevykdė, ką Viešpaties tarnas Mozė buvo įsakęs. 
\par 13 Keturioliktais karaliaus Ezekijo metais Asirijos karalius Senheribas puolė visus sutvirtintus Judo miestus ir juos paėmė. 
\par 14 Tada Judo karalius Ezekijas siuntė pas Asirijos karalių į Lachišą, sakydamas: “Nusikaltau, pasitrauk nuo manęs. Ko reikalausi, padarysiu”. Asirijos karalius uždėjo Judo karaliui Ezekijui tris šimtus talentų sidabro ir trisdešimt talento aukso duoklę. 
\par 15 Ezekijas atidavė visą sidabrą, kurį surado Viešpaties namuose ir karaliaus rūmų ižde. 
\par 16 Ezekijas nuplėšė auksą nuo Viešpaties šventyklos durų ir staktų, kurias jis buvo padengęs, ir atidavė Asirijos karaliui. 
\par 17 Asirijos karalius siuntė iš Lachišo Tartaną, Rabsarį ir Rabšakę su didele kariuomene prieš Jeruzalę. Jie atėjo į Jeruzalę ir sustojo prie aukštutinio vandentiekio tvenkinio, vėlėjo lauke. 
\par 18 Jie pašaukė karalių. Išėjo pas juos rūmų viršininkas Eljakimas, Hilkijo sūnus, raštininkas Šebna ir metraštininkas Joahas, Asafo sūnus. 
\par 19 Rabšakė jiems tarė: “Taip sakykite Ezekijui: ‘Taip sako didysis karalius, Asirijos karalius: ‘Kuo remiasi tavo pasitikėjimas? 
\par 20 Tu kalbi tuščius žodžius, o karui reikalingas patarimas ir jėga. Kuo pasitiki, kad sukilai prieš mane? 
\par 21 Ar ketini atsiremti į Egiptą, šitą sulūžusią nendrę? Pasirėmus į ją, ji įsminga į ranką ir ją perduria. Toks yra faraonas, Egipto karalius, visiems, kurie juo pasitiki. 
\par 22 O jei sakysite: ‘Mes pasitikime Viešpačiu, savo Dievu’, tai ar ne Jo aukštumas ir aukurus pašalino Ezekijas ir paliepė Judui bei Jeruzalei: ‘Jūs garbinsite prie šito aukuro Jeruzalėje’? 
\par 23 Taigi dabar lenktyniauk su mano valdovu, Asirijos karaliumi; aš tau duosiu du tūkstančius žirgų, jei tu surinksi tiek raitelių ant jų joti. 
\par 24 Ar gali pasipriešinti silpniausiam mano valdovo tarnų būriui, nors ir pasitiki Egipto vežimais ir raiteliais? 
\par 25 Ar aš be Viešpaties ėjau į šitą vietą, kad ją sunaikinčiau? Viešpats man pasakė: ‘Eik ir sunaikink tą kraštą’ ”. 
\par 26 Tuomet Hilkijo sūnus Eljakimas, Joahas ir Šebna tarė Rabšakei: “Kalbėk su savo tarnais aramėjiškai, mes suprantame; nekalbėk su mumis žydiškai, girdint žmonėms ant sienų”. 
\par 27 Bet Rabšakė atsakė: “Ar mano valdovas siuntė mane tik pas tavo valdovą ir tave kalbėti šituos žodžius? Ar ne pas vyrus, kurie sėdi ant sienos, kad valgytų su jumis savo išmatas bei gertų savo šlapimą?” 
\par 28 Rabšakė atsistojo ir garsiai šaukė žydiškai: “Klausykite didžiojo karaliaus, Asirijos karaliaus, žodžių! 
\par 29 Taip sako karalius: ‘Nesiduokite Ezekijo suvedžiojami, nes jis neišgelbės jūsų iš mano rankos! 
\par 30 Teneįtikina jūsų Ezekijas pasitikėti Viešpačiu, sakydamas: ‘Viešpats tikrai mus išgelbės ir neatiduos šito miesto į Asirijos karaliaus rankas’. 
\par 31 Neklausykite Ezekijo, nes taip sako Asirijos karalius: ‘Padarykite su manimi sutartį ir išeikite pas mane. Kiekvienas valgysite nuo savo vynmedžio, nuo savo figmedžio ir gersite vandenį iš savo šulinio, 
\par 32 kol aš ateisiu ir išvesiu jus į žemę, panašią į jūsų žemę, pilną javų, vyno, duonos, vynuogių, alyvmedžių ir medaus, kad galėtumėte gyventi ir nemirti. Neklausykite Ezekijo, kai jis jus įtikinėja, sakydamas: ‘Viešpats mus išgelbės’. 
\par 33 Argi kuris nors iš tautų dievų išgelbėjo savo kraštą iš Asirijos karaliaus rankos? 
\par 34 Kur yra Hamato ir Arpado dievai? Kur Sefarvaimo, Henos ir Ivos dievai? Ar jie išgelbėjo Samariją iš mano rankos? 
\par 35 Kuris iš dievų išgelbėjo savo kraštą iš mano rankos, kad Viešpats išgelbėtų Jeruzalę iš mano rankos?’ ” 
\par 36 Žmonės tylėjo ir neatsakė jam nė žodžio, nes toks buvo karaliaus įsakymas: “Neatsakykite jam”. 
\par 37 Rūmų viršininkas Hilkijo sūnus Eljakimas, raštininkas Šebna ir Asafo sūnus Joahas, metraštininkas, atėjo pas Ezekiją perplėštais drabužiais ir jam perdavė Rabšakės žodžius.



\chapter{19}

\par 1 Tai išgirdęs, karalius Ezekijas perplėšė savo drabužius, apsirengė ašutine ir nuėjo į Viešpaties namus. 
\par 2 Jis nusiuntė rūmų viršininką Eljakimą, raštininką Šebną ir vyresniuosius kunigus, apsirengusius ašutinėmis, pas pranašą Izaiją, Amoco sūnų. 
\par 3 Jie sakė jam: “Taip sako Ezekijas: ‘Šita diena yra bausmės, pažeminimo ir gėdos diena; atėjo laikas gimdyti, o jėgų nėra. 
\par 4 Gal Viešpats, tavo Dievas, išgirs žodžius Rabšakės, kurį Asirijos karalius pasiuntė niekinti gyvąjį Dievą, ir sudraus už žodžius, kuriuos Viešpats, tavo Dievas, girdėjo. Melskis už tuos, kurie yra likę’ ”. 
\par 5 Karaliaus Ezekijo tarnai atėjo pas Izaiją. 
\par 6 Ir Izaijas sakė jiems: “Sakykite savo valdovui: ‘Taip sako Viešpats: ‘Neišsigąsk girdėtų žodžių, kuriais Asirijos karaliaus tarnai man piktžodžiavo. 
\par 7 Štai Aš pasiųsiu jam dvasią, ir jis, išgirdęs žinią, grįš į savo šalį ir ten bus nužudytas’ ”. 
\par 8 Rabšakė sugrįžęs rado Asirijos karalių kovojantį prieš Libną, nes jis girdėjo, kad šis pasitraukė nuo Lachišo. 
\par 9 Asirijos karalius išgirdo, kad Etiopijos karalius Tirhaka ateina kariauti prieš jį. Tuomet jis dar kartą siuntė pasiuntinius pas Ezekiją, sakydamas: 
\par 10 “Taip kalbėkite Judo karaliui Ezekijui: ‘Tegu Dievas, kuriuo tu pasitiki, neapgauna tavęs, sakydamas: ‘Jeruzalė nepateks į Asirijos karaliaus rankas’. 
\par 11 Tu girdėjai, ką Asirijos karaliai padarė visose šalyse, jas visiškai sunaikindami. Argi tu būsi išgelbėtas? 
\par 12 Argi tų tautų, kurias mano tėvai sunaikino, dievai išgelbėjo Gozaną, Charaną, Recefą ir Edeno vaikus, gyvenusius Telasare? 
\par 13 Kur yra Hamato, Arpado, Sefarvaimo, Henos ir Ivos miestų karaliai?” 
\par 14 Ezekijas paėmė laišką iš pasiuntinių ir jį perskaitė. Po to, nuėjęs į Viešpaties namus, karalius jį išskleidė priešais Viešpatį. 
\par 15 Ezekijas meldėsi prieš Viešpatį ir sakė: “Viešpatie, Izraelio Dieve, kuris gyveni tarp cherubų. Tu vienas esi visų žemės karalysčių Dievas. Tu sukūrei dangų ir žemę. 
\par 16 Palenk, Viešpatie, savo ausį ir išgirsk. Atverk, Viešpatie, savo akis ir pamatyk. Išgirsk Sanheribo žodžius, kuriais jis niekino gyvąjį Dievą. 
\par 17 Tai tiesa, Viešpatie, kad Asirijos karaliai išnaikino tautas ir jų šalis. 
\par 18 Jie sudegino jų dievus, nes jie nebuvo dievai, tik žmonių rankų darbas­medis ir akmuo­todėl jie sunaikino juos. 
\par 19 Dabar, Viešpatie, mūsų Dieve, išgelbėk mus iš jo rankų, kad visos žemės karalystės žinotų, jog Tu vienas, Viešpatie, esi Dievas!” 
\par 20 Amoco sūnus Izaijas siuntė pas Ezekiją, sakydamas: “Taip sako Viešpats, Izraelio Dievas: ‘Tavo maldą dėl Asirijos karaliaus Sanheribo Aš išgirdau’. 
\par 21 Štai Viešpaties žodis, kurį Jis kalbėjo apie jį: ‘Mergelė, Siono dukra, paniekino tave ir pasityčiojo iš tavęs. Jeruzalės dukra kraipo galvą dėl tavęs. 
\par 22 Ką tu paniekinai ir prieš ką piktžodžiavai? Prieš ką išdidžiai pakėlei balsą ir akis? Prieš Izraelio Šventąjį! 
\par 23 Per savo pasiuntinius tu niekinai Viešpatį ir sakei: ‘Su daugybe kovos vežimų aš pasikėliau į kalnų aukštumas, Libano aukščiausias vietas. Aš iškirsiu jo aukštuosius kedrus, gražiausius kiparisus. Aš pasieksiu tolimiausią vietą­ Karmelio mišką. 
\par 24 Aš kasiau šulinius ir gėriau svetimus vandenis; išdžiovinau savo kojų padais visas upes apsiausties vietose’. 
\par 25 Argi negirdėjai? Jau seniai Aš tai padariau, labai seniai tai paruošiau, tik dabar įvykdžiau, kad tu galėtum sustiprintus miestus paversti griuvėsių krūvomis. 
\par 26 Todėl jų gyventojai bejėgiai, jie nusigando ir susigėdo, jie tapo kaip lauko žolė, kaip gležna žolė ant stogų, kuri nudžiūna, dar neužaugusi. 
\par 27 Aš žinau, kaip tu gyveni, kaip tu įeini ir išeini, kaip tu siautėji prieš mane. 
\par 28 Kadangi tavo siautėjimas prieš mane ir tavo pasipūtimas pasiekė mano ausis, Aš įversiu savo grandį į tavo šnerves ir tave pažabosiu, ir vesiu tave atgal keliu, kuriuo atėjai’. 
\par 29 Tai bus ženklas tau, Ezekijau. Šiemet valgyk, ką randi, kitais metais­kas užaugs savaime; trečiaisiais metais sėkite ir pjaukite, sodinkite vynuogynus ir valgykite jų vaisius. 
\par 30 Judo namų likutis vėl leis šaknis apačioje ir neš vaisių viršuje. 
\par 31 Iš Jeruzalės išeis išlikusieji, iš Siono kalno išgelbėtieji. Viešpaties uolumas tai padarys! 
\par 32 Todėl Viešpats taip sako apie Asirijos karalių: ‘Jis neįeis į šitą miestą ir nepaleis į jį nė vienos strėlės, neateis prieš jį su skydais ir nesupils pylimo. 
\par 33 Jis sugrįš tuo pačiu keliu, kuriuo atėjo, ir į šitą miestą neįeis,­sako Viešpats.­ 
\par 34 Aš apginsiu šitą miestą ir jį išgelbėsiu dėl savęs ir dėl mano tarno Dovydo’ ”. 
\par 35 Tą naktį Viešpaties angelas atėjo į Asirijos stovyklą ir išžudė šimtą aštuoniasdešimt penkis tūkstančius. Jie atsikėlė antksti rytą, ir štai­aplinkui gausu lavonų. 
\par 36 Asirijos karalius Sanheribas pasitraukė ir sugrįžo į Ninevę. 
\par 37 Kai jis garbino savo dievo Nisrocho namuose, jo sūnūs Adramelechas ir Sareceras užmušė jį kardu ir pabėgo į Armėnijos kraštą. Jo sūnus Asarhadonas karaliavo jo vietoje.



\chapter{20}

\par 1 Tomis dienomis Ezekijas mirtinai susirgo. Pranašas Izaijas, Amoco sūnus, atėjęs pas jį, tarė: “Taip sako Viešpats: ‘Sutvarkyk savo namus, nes tu nebepasveiksi, bet mirsi’ ”. 
\par 2 Tada Ezekijas nusigręžė į sieną ir meldėsi: 
\par 3 “Viešpatie, atsimink, kad aš vaikščiojau prieš Tave teisingai ir tobula širdimi ir dariau gera Tavo akyse”. Ir Ezekijas graudžiai verkė. 
\par 4 Izaijui dar neperėjus kiemo, Viešpats kalbėjo jam: 
\par 5 “Grįžk ir sakyk mano tautos vadui Ezekijui: ‘Taip sako Viešpats, tavo tėvo Dovydo Dievas: ‘Aš girdėjau tavo maldą ir mačiau tavo ašaras. Aš tave pagydysiu, ir trečią dieną tu eisi į Viešpaties namus. 
\par 6 Aš pridėsiu prie tavo dienų penkiolika metų, be to, išgelbėsiu tave ir šitą miestą iš Asirijos karaliaus rankų ir apginsiu miestą dėl savęs ir dėl savo tarno Dovydo’ ”. 
\par 7 Po to Izaijas sakė: “Paimkite gabalėlį figos”. Jie paėmė, uždėjo ant voties, ir jis pasveiko. 
\par 8 Ezekijas paklausė Izaiją: “Koks yra ženklas, kad Viešpats išgydys mane ir trečią dieną aš eisiu į Viešpaties namus?” 
\par 9 Izaijas atsakė: “Tai bus ženklas iš Viešpaties, kad Viešpats išpildys savo žodį: ar nori, kad saulės laikrodžio šešėlis nueitų dešimt laipsnių pirmyn ar kad grįžtų dešimt laipsnių atgal?” 
\par 10 Ezekijas atsakė: “Lengva šešėliui pailgėti dešimt laipsnių.Tegul jis grįžta dešimt laipsnių atgal”. 
\par 11 Pranašas Izaijas šaukėsi Viešpaties, ir Jis grąžino Ahazo saulės laikrodžio šešėlį dešimt laipsnių atgal. 
\par 12 Tuo metu Babilono karalius Merodach Baladanas, Baladano sūnus, atsiuntė Ezekijui laišką ir dovanų, nes jis sužinojo, kad Ezekijas serga. 
\par 13 Ezekijas išklausė juos ir aprodė jiems visus savo turtus: sidabrą, auksą, kvepalus, brangųjį aliejų, ginklų sandėlius ir visa, kas buvo jo ižde. Nebuvo nieko, ko Ezekijas nebūtų jiems parodęs savo namuose ir savo valdose. 
\par 14 Pranašas Izaijas, atėjęs pas karalių Ezekiją, paklausė: “Iš kur tie vyrai atėjo pas tave ir ką jie tau pasakė?” Ezekijas atsakė: “Jie atėjo iš tolimos šalies, iš Babilono”. 
\par 15 Ir jis sakė: “Ką jie matė tavo namuose?” Ezekijas atsakė: “Jie matė viską, kas yra mano namuose; nėra nieko, ko nebūčiau jiems parodęs”. 
\par 16 Tada Izaijas tarė Ezekijui: “Klausyk Viešpaties žodžio: 
\par 17 ‘Ateis diena, kai į Babiloną išgabens viską, kas yra tavo namuose ir ką iki šiol yra sukrovę tavo tėvai, ir nieko nebeliks. 
\par 18 Jie išves tavo sūnus, kurie tau gims, ir jie bus eunuchais Babilono karaliaus rūmuose’ ”. 
\par 19 Ezekijas atsakė Izaijui: “Viešpaties žodis, kurį tu kalbėjai, yra geras. Taika ir saugumas bus mano dienomis”. 
\par 20 Visi kiti Ezekijo darbai ir jo galia, kaip jis padarė tvenkinį ir vandentiekį, atvesdamas į miestą vandenį, yra aprašyta Judo karalių metraščių knygoje. 
\par 21 Ezekijas užmigo prie savo tėvų, o jo sūnus Manasas karaliavo jo vietoje.



\chapter{21}

\par 1 Pradėdamas karaliauti Manasas buvo dvylikos metų ir karaliavo Jeruzalėje penkiasdešimt penkerius metus. Jo motina buvo vardu Hefciba. 
\par 2 Jis darė pikta Viešpaties akyse, mėgdžiodamas bjaurius papročius pagonių, kuriuos Viešpats išvarė izraelitams užimant kraštą. 
\par 3 Jis vėl atstatė aukštumas, kurias jo tėvas Ezekijas buvo išnaikinęs, pastatė aukurą Baalui, pasodino giraitę, kaip Izraelio karalius Ahabas, garbino dangaus kareiviją ir jiems tarnavo. 
\par 4 Manasas pastatė aukurų net Viešpaties namuose, apie kuriuos Viešpats buvo pasakęs: “Jeruzalėje bus mano vardas”. 
\par 5 Dviejuose Viešpaties namų kiemuose jis pastatė aukurus dangaus kareivijai. 
\par 6 Jis leido savo sūnų per ugnį, kerėdavo ir būrė iš ženklų, bendravo su mirusiųjų dvasių iššaukėjais bei žyniais. Jis darė daug piktadarysčių Viešpaties akyse, sukeldamas Jo pyktį. 
\par 7 Jis pastatė Ašeros drožtą atvaizdą namuose, apie kuriuos Viešpats pasakė Dovydui ir jo sūnui Saliamonui: “Iš visų Izraelio giminių išsirinkau Jeruzalę ir šituos namus; čia amžinai bus mano vardas. 
\par 8 Nebeleisiu Izraeliui iškelti kojos iš žemės, kurią daviau jų tėvams, jei jie rūpestingai laikysis mano įstatymų ir įsakymų, kuriuos jiems davė mano tarnas Mozė”. 
\par 9 Tačiau jie neklausė, ir karalius Manasas juos suvedžiojo elgtis pikčiau už tautas, kurias Viešpats išnaikino prieš Izraelio vaikus. 
\par 10 Viešpats kalbėjo per savo tarnus pranašus, sakydamas: 
\par 11 “Kadangi Judo karalius Manasas darė šitas bjaurystes ir elgėsi pikčiau negu amoritai, gyvenę pirma jo, ir įtraukė į nuodėmę Judą dėl savo stabų, 
\par 12 todėl taip sako Viešpats, Izraelio Dievas: ‘Štai Aš užvesiu Jeruzalei ir Judui tokias nelaimes, kad suspengs ausyse, išgirdus apie tai. 
\par 13 Aš ištiesiu virš Jeruzalės Samarijos matavimo virvę ir Ahabo namų svambalą; Jeruzalę išvalysiu, kaip yra išvalomas indas ir apverčiamas iššluosčius. 
\par 14 Aš apleisiu savo paveldo likutį ir atiduosiu į jų priešų rankas, kurie juos pavergs ir apiplėš, 
\par 15 kadangi jie darė pikta mano akyse ir rūstino mane nuo tos dienos, kai jų tėvai išėjo iš Egipto, iki šios dienos’ ”. 
\par 16 Be savo nuodėmės, į kurią jis įtraukė Judą, darydamas tai, kas pikta Viešpaties akyse, Manasas praliejo tiek nekalto kraujo, kad užtvindė Jeruzalę nuo vieno galo iki kito. 
\par 17 Visi kiti Manaso darbai ir jo nuodėmės, kurias jis padarė, yra surašyta Judo karalių metraščių knygoje. 
\par 18 Manasas užmigo prie savo tėvų ir buvo palaidotas savo namų sode, Uzos sode; jo sūnus Amonas karaliavo jo vietoje. 
\par 19 Pradėdamas karaliauti, Amonas buvo dvidešimt dvejų metų ir karaliavo Jeruzalėje dvejus metus. Jo motina buvo Mešulemeta, Haruco duktė, iš Jotbos. 
\par 20 Jis darė pikta Viešpaties akyse, kaip ir jo tėvas Manasas. 
\par 21 Jis vaikščiojo visais savo tėvo keliais, tarnavo stabams ir juos garbino. 
\par 22 Jis apleido Viešpatį, savo tėvų Dievą, ir nevaikščiojo Viešpaties keliais. 
\par 23 Amono tarnai surengė sąmokslą ir nužudė karalių jo namuose. 
\par 24 Tačiau krašto žmonės nužudė visus, kurie dalyvavo sąmoksle prieš karalių Amoną, ir paskelbė karaliumi jo sūnų Joziją jo vietoje. 
\par 25 Visi kiti Amono darbai yra surašyti Judo karalių metraščių knygoje. 
\par 26 Jis buvo palaidotas savo kape, Uzos sode, ir jo sūnus Jozijas karaliavo jo vietoje.



\chapter{22}

\par 1 Pradėdamas karaliauti, Jozijas buvo aštuonerių metų ir karaliavo Jeruzalėje trisdešimt vienerius metus. Jo motina buvo vardu Jedida, Adajos duktė, iš Bockato. 
\par 2 Jis darė tai, kas teisinga Viešpaties akyse, ir vaikščiojo savo tėvo Dovydo keliais, nenukrypdamas nei į kairę, nei į dešinę. 
\par 3 Aštuonioliktais karaliaus Jozijo metais karalius siuntė raštininką Šafaną, Mešulamo sūnaus Acalijo sūnų, į Viešpaties namus 
\par 4 pas vyriausiąjį kunigą Hilkiją sužinoti, kiek pinigų buvo surinkta prie Viešpaties namų įėjimo. 
\par 5 Tuos pinigus jis įsakė perduoti darbų prižiūrėtojams Viešpaties namuose, kad tie juos išdalintų darbininkams, kurie dirba Viešpaties namuose, užtaisydami jų spragas, 
\par 6 dailidėms, statybininkams bei mūrininkams ir pirkti rąstus bei tašytus akmenis namų remontui. 
\par 7 Tačiau jiems nereikėjo atsiskaityti už pinigus, kuriuos gaudavo, nes jie buvo ištikimi. 
\par 8 Vyriausiasis kunigas Hilkijas sakė raštininkui Šafanui: “Viešpaties namuose radau įstatymo knygą!” Hilkijas padavė tą knygą Šafanui, ir šis ją skaitė. 
\par 9 Šafanas atėjo pas karalių ir pranešė jam, kad jo tarnai suskaičiavo paaukotus pinigus ir juos perdavė darbų prižiūrėtojams Viešpaties namuose. 
\par 10 Raštininkas Šafanas pranešė karaliui: “Kunigas Hilkijas davė man knygą”. Ir Šafanas skaitė ją karaliui. 
\par 11 Karalius, išgirdęs įstatymo knygos žodžius, perplėšė savo drabužius. 
\par 12 Jis įsakė kunigui Hilkijui, Šafano sūnui Ahikamui, Mikajos sūnui Achborui, raštininkui Šafanui ir karaliaus tarnui Asajai, sakydamas: 
\par 13 “Eikite ir pasiklauskite Viešpatį už mane, už tautą ir už visą Judą dėl šitos knygos žodžių. Didelė Viešpaties rūstybė užsidegusi prieš mus, kadangi mūsų tėvai neklausė šitos knygos žodžių ir nesielgė taip, kaip joje parašyta”. 
\par 14 Kunigas Hilkijas, Ahikamas, Achboras, Šafanas ir Asaja nuėjo pas pranašę Huldą, žmoną Šalumo, Tikvos sūnaus, Harhaso sūnaus, drabužių sargo, kuri gyveno Jeruzalės antroje dalyje, ir kalbėjo su ja. 
\par 15 Ji jiems sakė: “Taip sako Viešpats, Izraelio Dievas: ‘Sakykite vyrui, kuris jus siuntė pas mane, 
\par 16 kad Aš, Viešpats, bausiu šitą vietą ir jos gyventojus, kaip pasakyta knygoje, kurią skaitė Judo karalius. 
\par 17 Jie paliko mane ir degino smilkalus kitiems dievams, sukeldami mano pyktį savo rankų darbais, todėl mano rūstybė užsidegs prieš šitą vietą ir neužges’. 
\par 18 Judo karaliui, kuris jus siuntė pasiklausti Viešpaties, sakykite: ‘Taip sako Viešpats, Izraelio Dievas: ‘Dėl žodžių, kuriuos tu girdėjai, 
\par 19 tavo širdis buvo minkšta ir tu nusižeminai prieš Viešpatį, klausydamas, ką Aš kalbėjau prieš šitą vietą ir jos gyventojus, kad ji taps griuvėsiais ir prakeikimu, ir kadangi tu perplėšei savo drabužius bei verkei mano akivaizdoje, Aš išklausiau tave. 
\par 20 Todėl Aš paimsiu tave prie tavo tėvų, ir tu būsi paguldytas į kapą ramybėje; tavo akys nebematys visų tų nelaimių, kurias Aš siųsiu šitai vietai’ ”. Jie pranešė šituos žodžius karaliui.



\chapter{23}

\par 1 Karalius sukvietė pas save visus Judo ir Jeruzalės vyresniuosius. 
\par 2 Karalius su kunigais, pranašais ir Judo bei Jeruzalės gyventojais nuo mažo iki didelio nuėjo į Viešpaties namus. Karalius garsiai skaitė visus žodžius iš sandoros knygos, atrastos Viešpaties namuose. 
\par 3 Karalius atsistojo prie kolonos ir pakartojo sandorą Viešpaties akivaizdoje: sekti Viešpatį ir laikytis Jo įsakymų, įspėjimų ir nuostatų visa širdimi ir visa siela, kad būtų įvykdyti visi sandoros žodžiai, užrašyti šitoje knygoje. Visa tauta pasižadėjo laikytis sandoros. 
\par 4 Karalius įsakė vyriausiajam kunigui Hilkijui, antros eilės kunigams ir vartininkams išnešti iš Viešpaties šventyklos visus daiktus, skirtus Baalui, Ašerai bei dangaus kareivijai. Jie buvo sudeginti už Jeruzalės sienų, Kidrono slėnyje, o jų pelenus išnešė į Betelį. 
\par 5 Jis išnaikino stabų kunigus, kuriuos buvo paskyrę Judo karaliai deginti smilkalus aukštumose, Judo miestuose ir Jeruzalės apylinkėse; taip pat ir tuos, kurie smilkė Baalui, saulei, mėnuliui, planetoms ir visai dangaus kareivijai. 
\par 6 Jis išnešė Ašeros statulą iš Viešpaties namų už Jeruzalės sienų, į Kidrono upelio slėnį; ten ją sudegino, sutrynė į dulkes ir išbarstė kapuose. 
\par 7 Be to, jis sugriovė paleistuvių namus, buvusius prie Viešpaties namų, kur moterys ausdavo apdangalus Ašerai. 
\par 8 Jis sušaukė visus kunigus iš Judo miestų ir išniekino aukštumas, kuriose kunigai smilkė, nuo Gebos iki Beer Šebos. Karalius sunaikino aukštumas prie valdytojo Jozuės vartų, miesto vartų kairėje pusėje. 
\par 9 Aukštumų kunigai neprieidavo prie Viešpaties aukuro Jeruzalėje, bet valgydavo neraugintą duoną su savo broliais. 
\par 10 Jozijas išniekino Tofetą Hinomo vaikų slėnyje, kad niekas savo sūnaus ar dukters neleistų per ugnį Molechui. 
\par 11 Taip pat pašalino Judo karalių saulei pašvęstus žirgus, kurie buvo prie Viešpaties namų įėjimo, prie eunucho Netan Melecho kambario, o saulės vežimus sudegino. 
\par 12 Judo karalių ant Ahazo namų stogo pastatytus aukurus ir Manaso pastatytus aukurus dviejuose Viešpaties namų kiemuose karalius nugriovė, sudaužė ir jų dulkes sumetė į Kidrono upelį. 
\par 13 Karalius išniekino aukštumas, kurios buvo priešais Jeruzalę, į dešinę nuo Sugedimo kalno, kurias Izraelio karalius Saliamonas buvo padaręs sidoniečių pabaisai Astartei, Moabo pabaisai Kemošui ir amonitų pabaisai Milkomui. 
\par 14 Jis sudaužė atvaizdus, iškirto giraites ir jų vietas užpildė žmonių kaulais. 
\par 15 Aukurą Betelio aukštumoje, kurį pastatė Nebato sūnus Jeroboamas, įtraukęs į nuodėmę Izraelį, ir pačią aukštumą jis nugriovė, sudegino ir sutrynė į dulkes, ir taip pat sudegino giraitę. 
\par 16 Jozijas, pamatęs kapus ant kalno, pasiuntė ir paėmė kaulus iš kapų, juos sudegino ant aukuro, jį išniekindamas pagal Viešpaties žodį, paskelbtą Dievo vyro. 
\par 17 Tada jis sakė: “Kieno tas antkapis, kurį aš matau?” Miesto žmonės jam atsakė: “Tai kapas Dievo vyro, kuris buvo atėjęs iš Judo ir paskelbė, ką tu dabar įvykdei ant Betelio aukuro”. 
\par 18 Jis sakė: “Palikite jį. Niekas tenepajudina jo kaulų”. Jie paliko jo ir pranašo, kuris buvo iš Samarijos, kaulus nepaliestus. 
\par 19 Visus aukštumų namus, buvusius Samarijos miestuose, kuriuos pastatė Izraelio karaliai, sukeldami Viešpaties pyktį, Jozijas pašalino, padarydamas su jais kaip Betelyje. 
\par 20 Jis išžudė visus ten buvusius aukštumų kunigus ir ant aukurų degino žmonių kaulus. Po to jis sugrįžo į Jeruzalę. 
\par 21 Karalius įsakė visai tautai: “Švęskite Paschą Viešpačiui, savo Dievui, kaip parašyta šitoje sandoros knygoje!” 
\par 22 Tokia Pascha nebuvo švenčiama nuo teisėjų laikų per visas Izraelio bei Judo karalių dienas. 
\par 23 Aštuonioliktais karaliaus Jozijo metais vėl šventė tokią Paschą Viešpačiui Jeruzalėje. 
\par 24 Jozijas pašalino mirusiųjų dvasių iššaukėjus, žynius, atvaizdus, stabus ir visas baisenybes, kurias surado Judo žemėje ir Jeruzalėje, kad įvykdytų įstatymo žodžius, užrašytus knygoje, kurią vyriausiasis kunigas Hilkijas rado Viešpaties namuose. 
\par 25 Nebuvo iki jo tokio karaliaus, kuris būtų atsigręžęs į Viešpatį visa širdimi, visa siela ir visomis jėgomis pagal visą Mozės įstatymą, ir po jo nebuvo tokio. 
\par 26 Tačiau Viešpats neatsisakė savo didelės rūstybės, kuri buvo užsidegusi prieš Judą dėl visų Manaso nusikaltimų, kuriais jis supykdė Viešpatį. 
\par 27 Viešpats tarė: “Aš atstumsiu Judą, kaip atstūmiau Izraelį, ir atmesiu Jeruzalę­šitą miestą, kurį išsirinkau, ir namus, apie kuriuos sakiau: ‘Ten bus mano vardas’ ”. 
\par 28 Visi kiti Jozijo darbai surašyti Judo karalių metraščių knygoje. 
\par 29 Jo dienomis Egipto faraonas Nekojas išėjo prieš Asirijos karalių Eufrato upės link. Karalius Jozijas išėjo priešais jį, ir tas nužudė jį Megide, kai jie susitiko. 
\par 30 Jo tarnai parvežė jį vežime mirusį iš Megido į Jeruzalę ir palaidojo jo kape. Krašto žmonės paėmė Jehoachazą, Jozijo sūnų, ir patepė jį karaliumi jo tėvo vietoje. 
\par 31 Pradėdamas karaliauti, Jehoachazas buvo dvidešimt trejų metų ir karaliavo Jeruzalėje tris mėnesius. Jo motina buvo Hamutalė, Jeremijo duktė, iš Libnos. 
\par 32 Jis darė pikta Viešpaties akyse, kaip ir jo tėvai. 
\par 33 Faraonas Nekojas suėmė jį Ribloje, Hamato krašte, kad jis negalėtų karaliauti Jeruzalėje, o kraštui uždėjo duoklę: šimtą talentų sidabro ir talentą aukso. 
\par 34 Faraonas Nekojas paskyrė Judo karaliumi Jozijo sūnų Eljakimą jo tėvo Jozijo vietoje ir pakeitė jo vardą į Jehojakimą, o Jehoachazą nusivedė į Egiptą; ten jis ir mirė. 
\par 35 Jehojakimas atidavė sidabrą ir auksą faraonui. Jis apdėjo kraštą mokesčiais, kad galėtų sumokėti faraono uždėtą duoklę. Kiekvienas krašto gyventojas turėjo sumokėti jam paskirtą mokestį sidabru ir auksu. 
\par 36 Pradėdamas karaliauti, Jehojakimas buvo dvidešimt penkerių metų ir karaliavo Jeruzalėje vienuolika metų. Jo motina buvo vardu Zebida, Pedajo duktė, iš Rumos. 
\par 37 Jis darė pikta Viešpaties akyse, kaip ir jo tėvai.



\chapter{24}

\par 1 Jo dienomis atėjo Babilono karalius Nebukadnecaras. Jehojakimas buvo jo tarnu trejus metus, o po to sukilo prieš jį. 
\par 2 Tuomet Viešpats siuntė chaldėjų, sirų, moabitų bei amonitų būrius prieš Judą, kad jį sunaikintų, kaip Jis buvo paskelbęs per savo tarnus pranašus. 
\par 3 Viešpaties įsakymu tai atsitiko Judui, kad jis būtų pašalintas iš Jo akivaizdos dėl Manaso nuodėmių. 
\par 4 Taip pat Viešpats neatleido nekaltai pralieto kraujo, kuriuo šis pripildė Jeruzalę. 
\par 5 Visi kiti Jehojakimo darbai yra surašyti Judo karalių metraščių knygoje. 
\par 6 Jehojakimas užmigo prie savo tėvų, ir jo sūnus Jehojachinas karaliavo jo vietoje. 
\par 7 Egipto karalius daugiau nebeišėjo iš savo šalies, nes Babilono karalius užėmė tai, kas priklausė Egipto karaliui, nuo Egipto upės iki Eufrato upės. 
\par 8 Pradėdamas karaliauti, Jehojachinas buvo aštuoniolikos metų ir karaliavo Jeruzalėje tris mėnesius. Jo motina buvo Nehušta, Elnatano duktė, iš Jeruzalės. 
\par 9 Jis darė pikta Viešpaties akyse, kaip ir jo tėvas. 
\par 10 Tuo laiku Babilono karaliaus Nebukadnecaro tarnai užpuolė ir apgulė Jeruzalę. 
\par 11 Ir Babilono karalius Nebukadnecaras atėjo prieš miestą, kai jo tarnai buvo apgulę jį. 
\par 12 Tada Judo karalius Jehojachinas, jo motina, tarnai, kunigaikščiai ir valdininkai išėjo pas Babilono karalių. Ir Babilono karalius suėmė jį aštuntaisiais savo valdymo metais. 
\par 13 Jis išgabeno visus Viešpaties namų ir karaliaus namų turtus ir sukapojo visus auksinius daiktus, kuriuos Izraelio karalius Saliamonas buvo padaręs Viešpaties šventykloje, kaip Viešpats buvo sakęs. 
\par 14 Jis išvedė į nelaisvę visą Jeruzalę, visus kunigaikščius ir visus narsius karius, iš viso dešimt tūkstančių belaisvių, taip pat visus amatininkus bei kalvius. Nieko neliko, išskyrus vargingiausius krašto žmones. 
\par 15 Jehojachiną, karaliaus motiną, jo žmonas, jo valdininkus ir krašto galinguosius jis išvedė iš Jeruzalės į nelaisvę Babilone. 
\par 16 Visus karius, iš viso septynis tūkstančius, taip pat amatininkus ir kalvius, iš viso tūkstantį, ir visus vyrus, tinkančius kariuomenei, Babilono karalius išsivedė belaisviais į Babiloną. 
\par 17 Judo karaliumi jis paskyrė Jehojachino dėdę Mataniją ir jo vardą pakeitė Zedekiju. 
\par 18 Zedekijas buvo dvidešimt vienerių metų, kai tapo karaliumi. Jis karaliavo Jeruzalėje vienuolika metų. Jo motina buvo Hamutalė, Jeremijo duktė, iš Libnos. 
\par 19 Jis darė pikta Viešpaties akyse, kaip ir Jehojakimas. 
\par 20 Dėl Viešpaties rūstybės taip atsitiko Jeruzalei ir Judui, kad galiausiai Jis pašalino juos iš savo akių. Ir Zedekijas sukilo prieš Babilono karalių.



\chapter{25}

\par 1 Devintaisiais jo karaliavimo metais, dešimto mėnesio dešimtą dieną prieš Jeruzalę atėjo Babilono karalius Nebukadnecaras su visa kariuomene, apgulė ją ir supylė aplinkui pylimą. 
\par 2 Miestas buvo apgultas iki vienuoliktų karaliaus Zedekijo metų. 
\par 3 Ketvirto mėnesio devintą dieną mieste taip sustiprėjo badas, kad žmonės nebeturėjo ko valgyti. 
\par 4 Pralaužę miesto sieną, karalius su visais kariais pabėgo naktį taku, esančiu tarp dviejų miesto sienų, prie karaliaus sodo. Chaldėjai buvo išsidėstę aplinkui miestą. Jie traukėsi lygumos keliu. 
\par 5 Chaldėjų kariuomenė vijosi karalių ir sugavo Jericho lygumoje. Visa jo kariuomenė buvo išsklaidyta. 
\par 6 Jie suėmė karalių ir atgabeno į Riblą pas Babilono karalių, ir jie teisė jį. 
\par 7 Jie nužudė Zedekijo sūnus jo akyse, o pačiam Zedekijui išlupo akis, sukaustė grandinėmis ir išsivedė į Babiloną. 
\par 8 Devynioliktų Babilono karaliaus Nebukadnecaro metų penkto mėnesio septintą dieną į Jeruzalę atėjo Babilono karaliaus tarnas Nebuzaradanas, sargybos viršininkas, 
\par 9 ir sudegino Viešpaties namus, karaliaus namus ir visus didelius miesto pastatus. 
\par 10 Chaldėjų kariuomenė, kuri buvo su sargybos viršininku, išgriovė aplink Jeruzalę esančias sienas. 
\par 11 Išlikusius miesto gyventojus ir perbėgusius pas Babilono karalių sargybos viršininkas Nebuzaradanas išvedė į nelaisvę. 
\par 12 Bet jis paliko kai kuriuos krašto beturčius, kad prižiūrėtų vynuogynus ir dirbtų žemę. 
\par 13 Chaldėjai sulaužė varines kolonas, stovus ir varinį baseiną, buvusius Viešpaties namuose, ir jų varį išgabeno į Babiloną. 
\par 14 Jie paėmė ir puodus, semtuvus, gnybtuvus, dubenis bei visus varinius indus, kurie buvo naudojami tarnavimo metu. 
\par 15 Sargybos viršininkas pasiėmė indus smilkalams, taures ir visa, kas buvo iš aukso ir sidabro. 
\par 16 Dviejų kolonų, baseino ir stovų, kuriuos Saliamonas padirbo Viešpaties namams, vario buvo tiek, kad nebuvo įmanoma pasverti. 
\par 17 Viena kolona buvo aštuoniolikos uolekčių aukščio, ir ant jos buvo varinis kapitelis trijų uolekčių aukščio; grotelės ir granato vaisiai aplinkui buvo iš vario. Tokia pat buvo ir antroji kolona su grotelėmis. 
\par 18 Sargybos viršininkas paėmė vyriausiąjį kunigą Serają, antrąjį kunigą Sofoniją, tris durininkus, 
\par 19 miesto valdininką, kuris buvo karo vyrų viršininkas, penkis vyrus, karaliaus patarėjus, kuriuos rado mieste, kariuomenės vyriausiąjį raštininką, kuris šaukdavo kariuomenėn vyrus, ir šešiasdešimt krašto vyrų, kurie buvo mieste. 
\par 20 Ir Nebuzaradanas, sargybos viršininkas, nuvedė juos pas Babilono karalių į Riblą. 
\par 21 Karalius nužudė juos Rebloje, Emato krašte. Taip Judas buvo ištremtas iš savo krašto. 
\par 22 Valdytoju Judo krašte likusiems žmonėms Babilono karalius Nebukadnecaras paskyrė Šafano sūnaus Ahikamo sūnų Gedaliją. 
\par 23 Kariuomenės vadai ir jų vyrai išgirdo, kad Babilono karalius paskyrė Goedaliją Judo valdytoju. Ir atėjo pas jį į Micpą Netanijos sūnus Izmaelis, Kareacho sūnus Johananas, netofiečio Tanhumeto sūnus Seraja, maako sūnus Jaazanijas ir jų vyrai. 
\par 24 Gedalijas prisiekė jiems: “Nebijokite chaldėjų tarnų, gyvenkite krašte, tarnaukite Babilono karaliui ir bus jums gerai”. 
\par 25 Septintą mėnesį į Miscpą atėjo Elišamo sūnaus Netanijos sūnus Izmaelis, kilęs iš karališkos giminės, su dešimčia vyrų ir užmušė Gedaliją, žydus ir chaldėjus, buvusius su juo. 
\par 26 Tada visi žmonės, dideli ir maži, bei kariuomenės vadai išėjo į Egiptą, nes jie bijojo chaldėjų. 
\par 27 Trisdešimt septintaisiais Judo karaliaus Jehojachino tremties metais, dvylikto mėnesio dvidešimt septintą dieną Babilono karalius Evil Merodachas pirmaisiais savo karaliavimo metais išleido Judo karalių Jehojachiną iš kalėjimo. 
\par 28 Jis draugiškai su juo kalbėjo ir davė jam sostą, aukštesnį negu kitų karalių, kurie buvo su juo Babilone. 
\par 29 Jis pakeitė Jehojachino kalėjimo drabužius, ir tas valgė karaliaus akivaizdoje per visas savo gyvenimo dienas. 
\par 30 Karalius jam paskyrė nuolatinį išlaikymą, kurį jis gaudavo kiekvieną dieną, per visas savo gyvenimo dienas.




\end{document}