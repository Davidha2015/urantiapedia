\begin{document}

\title{Pirmoji Kronikų knyga}

\chapter{1}

\par 1 Adomas, Setas, Enas, 
\par 2 Kainamas, Maleleelis, Jaretas, 
\par 3 Henochas, Matūzalis, Lamechas, 
\par 4 Nojus, Semas, Chamas ir Jafetas. 
\par 5 Jafeto sūnūs: Gomeras, Magogas, Madajas, Javanas, Tubalas, Mešechas ir Tyras. 
\par 6 Gomero sūnūs: Aškenazas, Rifatas ir Togarmas. 
\par 7 Javano sūnūs: Eliša, Taršišas, Kitimas ir Dodanimas. 
\par 8 Chamo sūnūs: Kušas, Micraimas, Putas ir Kanaanas. 
\par 9 Kušo sūnūs: Seba, Havila, Sabta, Ramair Sabtecha. Ramo sūnūs: Šeba ir Dedanas. 
\par 10 Kušas buvo tėvas Nimrodo, kuris tapo galingas žemėje. 
\par 11 Mizraimas buvo Ludo, Anamimo, Lehabo, Naftoacho, 
\par 12 Patroso, Kasluho, iš kurių kilo filistinai, ir Kaftoro tėvas. 
\par 13 Kanaanui gimė pirmagimis Sidonas, Hetas, 
\par 14 jebusiečiai, amoritai, girgašai, 
\par 15 hivai, arkai, sinai, 
\par 16 arvadiečiai, cemarai ir hamatiečiai. 
\par 17 Semo sūnūs: Elamas, Asūras, Arfaksadas, Aramas, Ludas, Ucas, Hulas, Geteras ir Mešechas. 
\par 18 Arfaksadas buvo Salos tėvas, o Sala buvo Ebero tėvas. 
\par 19 Eberas turėjo du sūnus: vienas buvo vardu Falekas, nes jo dienomis buvo padalinta žemė, o jo brolis buvo vardu Joktanas. 
\par 20 Joktanas buvo Almodado, Šelefo, Hazarmaveto, Jeracho, 
\par 21 Hadoramo, Uzalio, Diklo, 
\par 22 Obalio, Abimaelio, Šebo, 
\par 23 Ofyro, Havilos ir Jobabo tėvas; tie visi buvo Joktano sūnūs. 
\par 24 Semas, Arfaksadas, Sala, 
\par 25 Eberas, Falekas, Ragaujas, 
\par 26 Seruchas, Nachoras, Tara, 
\par 27 Abramas, jis taip pat Abraomas. 
\par 28 Abraomo sūnūs: Izaokas ir Izmaelis. 
\par 29 Šitie yra jų palikuonys: Izmaelio pirmagimis Nebajotas, po jo Kedaras, Adbeeis, Mibsamas, 
\par 30 Mišma, Dūma, Masa, Hadadas, Tema, 
\par 31 Jetūras, Nafišas ir Kedma; tai Izmaelio sūnūs. 
\par 32 Abraomo sugulovės Ketūros sūnūs: Zimranas, Jokšanas, Medanas, Midjanas, Išbakas, Šuachas. Jokšano sūnūs: Šeba ir Dedanas. 
\par 33 Midjano sūnūs: Efa, Eferas, Henochas, Abida ir Eldava. Tie visi buvo Ketūros sūnūs. 
\par 34 Abraomui gimė Izaokas. Izaoko sūnūs: Ezavas ir Izraelis. 
\par 35 Ezavo sūnūs: Elifazas, Reuelis, Jeušas, Jalamas ir Korachas. 
\par 36 Elifazo sūnūs: Temanas, Omaras, Cefojas, Gatamas, Kenazas, Timna ir Amalekas. 
\par 37 Reuelio sūnūs: Nahatas, Zerachas, Šama, Miza. 
\par 38 Seyro sūnūs: Lotanas, Šobalas, Cibeonas, Ana, Dišonas, Eceras ir Dišanas. 
\par 39 Lotano sūnūs: Horis ir Homamas. Lotano sesuo buvo Timna. 
\par 40 Šobalio sūnūs: Aljanas, Manahatas, Ebalas, Šefis ir Onamas. Cibeono sūnūs: Aja ir Ana. 
\par 41 Ano sūnus­Dišonas. Dišono sūnūs: Hamranas, Ešbanas, Itranas, Keranas. 
\par 42 Ecerio sūnūs: Bilhanas, Zavanas ir Akanas. Dišano sūnūs: Ucas ir Aranas. 
\par 43 Šitie karaliai karaliavo Edomo šalyje, kai izraelitai dar neturėjo karaliaus: Beoro sūnus Bela, kurio miestas vadinosi Dinhaba. 
\par 44 Belai mirus, jo vietoje viešpatavo Zeracho sūnus Jobabas iš Bocros. 
\par 45 Jobabui mirus, jo vietoje viešpatavo Hušamas iš Temano šalies. 
\par 46 Hušamui mirus, jo vietoje viešpatavo Bedado sūnus Hadadas, kuris nugalėjo Midjaną Moabo laukuose; jo miestas vadinosi Avitas. 
\par 47 Hadadui mirus, jo vietoje viešpatavo Samla iš Masreko. 
\par 48 Samlai mirus, jo vietoje viešpatavo Saulius iš Rehoboto, esančio prie Eufrato. 
\par 49 Sauliui mirus, jo vietoje viešpatavo Achboro sūnus Baal Hananas. 
\par 50 Baal Hananui mirus, jo vietoje viešpatavo Hadadas, kurio miestas buvo Pajas; jo žmona buvo vardu Mehetabelė, duktė Me Zahabo, duktė Matredo. 
\par 51 Hadadas taip pat mirė. Edomo kunigaikščiai buvo Timna, Alija, Jetetas, 
\par 52 Oholibama, Ela, Pinonas, 
\par 53 Kenazas, Temanas, Mibcaras, 
\par 54 Magdielis, Iramas.


\chapter{2}

\par 1 Izraelio sūnūs: Rubenas, Simeonas, Levis, Judas, Isacharas, Zabulonas, 
\par 2 Danas, Juozapas, Benjaminas, Neftalis, Gadas ir Ašeras. 
\par 3 Judo sūnūs: Eras, Onanas ir Šela; tie trys buvo kanaanietės Šūvos vaikai. Judo pirmagimis Eras buvo nedoras Viešpaties akyse, todėl Viešpats siuntė jam mirtį. 
\par 4 Su savo marčia Tamara Judas turėjo Perecą ir Zerachą. Iš viso buvo penki Judo sūnūs. 
\par 5 Pereco sūnūs: Esromas ir Hamulas. 
\par 6 Zeracho sūnūs: Zimris, Etanas, Hemanas, Kalkolas ir Dara, iš viso penki. 
\par 7 Karmio sūnus Achanas užtraukė nelaimę Izraeliui, nes jis pavogė, kas buvo skirta sunaikinti. 
\par 8 Etano sūnus buvo Azarija. 
\par 9 Esromo sūnūs: Jerachmeelis, Aramas ir Kelubajas. 
\par 10 Aramas buvo Aminadabo tėvas; Aminadabas buvo Judo kunigaikščio Naasono tėvas. 
\par 11 Naasonas buvo Salmono tėvas, Salmonas­Boozo tėvas, 
\par 12 Boozas­Jobedo tėvas, o Jobedas­ Jesės tėvas. 
\par 13 Jesės sūnūs: pirmagimis­Eliabas, antras­Abinadabas, trečias­ Šima, 
\par 14 ketvirtas­Netanelis, penktas­ Radajas, 
\par 15 šeštas­Ocemas, septintas­Dovydas. 
\par 16 Jų seserys: Ceruja ir Abigailė. Cerujos sūnūs: Abšajas, Joabas ir Asaelis­trys. 
\par 17 Abigailės sūnus­Amasa, jo tėvas buvo izmaelitas Jeteras. 
\par 18 Esromo sūnaus Kalebo sūnūs iš jo žmonų Azubos ir Jerijotos buvo Ješeras, Šobabas ir Ardonas. 
\par 19 Azubai mirus, Kalebas vedė Efratą, kuri pagimdė sūnų Hūrą. 
\par 20 Hūras buvo Ūrio tėvas, o Ūris­ Becalelio tėvas. 
\par 21 Vėliau Esromas, būdamas šešiasdešimties metų amžiaus, vedė Gileado tėvo Machyro dukterį; ji jam pagimdė Segubą. 
\par 22 Segubas buvo tėvas Jayro, kuriam priklausė dvidešimt trys miestai Gileado krašte. 
\par 23 Gešūras ir Aramas paėmė iš Jayro Kenato miestus ir jų kaimus­šešiasdešimt vietovių. Tie visi buvo Gileado tėvo Machyro palikuonys. 
\par 24 Esromui mirus Kalebo Efratoje, jo žmona Abija pagimdė nuo jo Tekojos tėvą Ašhūrą. 
\par 25 Esromo pirmagimio Jerachmelio sūnūs buvo pirmagimis Ramas, kiti­Būna, Orenas, Ocemas ir Ahija. 
\par 26 Jerachmelis dar turėjo kitą žmoną, kuri buvo vardu Atara; ji buvo Onamo motina. 
\par 27 Jerachmelio pirmagimio Ramo sūnūs buvo Maacas, Jaminas ir Ekeras. 
\par 28 Onamo sūnūs­Šamajas ir Jada, o Šamajo sūnūs­Nadabas ir Abišūras. 
\par 29 Abišūro žmona buvo vardu Abihailė; ji pagimdė Achbaną ir Molidą. 
\par 30 Nadabo sūnūs: Seledas ir Apaimas. Seledas mirė bevaikis. 
\par 31 Apajimo sūnus­Išis; Išio sūnus­Šešanas; Šešanui gimė Achlajas. 
\par 32 Jada, Šamajo brolis, turėjo sūnus Jeterą ir Jehonataną. Jeteras mirė bevaikis. 
\par 33 Jehonatano sūnūs: Peletas ir Zaza. Šitie buvo Jerachmelio palikuonys. 
\par 34 Šešanas neturėjo sūnų, tik dukteris. Šešanas turėjo egiptietį vergą, vardu Jarha. 
\par 35 Šešanas atidavė savo dukterį Jarhai į žmonas, o ji jam pagimdė Atają. 
\par 36 Atajas buvo Natano tėvas, Natanas­Zabado tėvas, 
\par 37 Zabadas­Eflalio tėvas, Eflalas­Jobedo tėvas, 
\par 38 Jobedas­Jehuvo tėvas, Jehuvas­Azarijo tėvas, 
\par 39 Azarija­Heleco tėvas, Helecas­Eleasos tėvas, 
\par 40 Eleasa­Sismajo tėvas, Sismajas­Šalumo tėvas, 
\par 41 Šalumas­Jekamijos tėvas, o Jekamija­Elišamos tėvas. 
\par 42 Jerachmelio brolio Kalebo pirmagimis sūnus Meša buvo Zifo tėvas, o Marešos sūnus buvo Hebronas. 
\par 43 Hebrono sūnūs: Korachas, Tapuachas, Rekemas ir Šema. 
\par 44 Šema buvo Rahamo tėvas, o Rahamas­Jorkoamo. Rekemas buvo Šamajo tėvas. 
\par 45 Šamajo sūnus buvo Maonas, o Maonas buvo Bet Cūro tėvas. 
\par 46 Efa, Kalebo sugulovė, pagimdė Haraną, Mocą ir Gazezą, ir Haranui gimė Gazezas. 
\par 47 Jahdojo sūnūs: Regemas, Joatamas, Gešanas, Peletas, Efa ir Šaafas. 
\par 48 Kalebo sugulovė Maaka pagimdė Šeberą ir Tirhaną. 
\par 49 Ji dar pagimdė Safą, Madmanos tėvą, ir Ševą, Machbenos ir Gibėjos tėvą. Kalebo duktė buvo Achsa. 
\par 50 Šitie buvo Kalebo palikuonys. Efratos pirmagimio Hūro sūnūs: Šobalas­Kirjat Jearimo tėvas, 
\par 51 Salma­Betliejaus tėvas, Harefas­Bet Gaderio tėvas. 
\par 52 Šobalas turėjo sūnų Haroję, ir pusė manahatiečių buvo kilę iš jo. 
\par 53 Kirjat Jearimo šeimos­itrai, putai, šumatai ir mišrai; iš šitų kilo coriečiai ir eštaoliečiai. 
\par 54 Salmos palikuonys: Betliejus, netofiečiai, Atarotas, Joabo namai, pusė manahatiečių ir coriečiai. 
\par 55 Raštininkų šeimos, kurios gyveno Jabece: tiratai, šimatai ir suchatai; jie yra kainitai, kilę iš Hamato, Rechabo namų tėvo.



\chapter{3}


\par 1 Hebrone gimę Dovydo sūnūs: pirmagimis­Amnonas iš jezreelietės Ahinoamos, antras­ Danielius iš karmelietės Abigailės, 
\par 2 trečias­Abšalomas, Gešūro karaliaus Talmajo dukters Maakos sūnus, ketvirtas­Adonijas, Hagitos sūnus, 
\par 3 penktas­Šefatija iš Abitalės, šeštas­Itramas, Eglos sūnus. 
\par 4 Šeši sūnūs gimė Hebrone, kur jis karaliavo septynerius metus ir šešis mėnesius. Trisdešimt trejus metus jis karaliavo Jeruzalėje. 
\par 5 Ten gimė Šima, Šobabas, Natanas ir Saliamonas­keturi iš Amielio dukters Batšebos; 
\par 6 Ibharas, Elišama, Elifeletas, 
\par 7 Nogahas, Nefegas, Jafija, 
\par 8 Elišama, Eljada ir Elifeletas­ devyni. 
\par 9 Tai visi Dovydo sūnūs, neskaičiuojant sugulovių sūnų; Tamara buvo jų sesuo. 
\par 10 Saliamono sūnus buvo Roboamas, jo sūnus buvo Abija, jo sūnus­Asa, jo sūnus­Juozapatas, 
\par 11 jo sūnus­Joramas, jo sūnus­ Ahazijas, jo sūnus­Jehoašas, 
\par 12 jo sūnus­Amacijas, jo sūnus­ Azarija, jo sūnus­Joatamas, 
\par 13 jo sūnus­Achazas, jo sūnus­ Ezekijas, jo sūnus­Manasas, 
\par 14 jo sūnus­Amonas, jo sūnus­ Jozijas. 
\par 15 Jozijo pirmagimis­Johananas, antras­Jehojakimas, trečias­Zedekijas, ketvirtas­Šalumas. 
\par 16 Jehojakimo sūnūs: Jechonijas ir Zedekijas. 
\par 17 Jechonijo sūnūs: Asiras, Salatielis, 
\par 18 Malkiramas, Pedaja, Šenacaras, Jekamija, Hošama ir Nedabija. 
\par 19 Pedajos sūnūs: Zorobabelis ir Šimis. Zorobabelio sūnūs: Mešulamas ir Hananija; Šelomita buvo jų sesuo; 
\par 20 Hašuba, Ohelis, Berechija, Hasadija ir Jušab Hesedas­penki. 
\par 21 Hananijos sūnūs: Pelatija ir Izaja. Refajos, Arnano, Abdijos ir Šechanijos sūnūs. 
\par 22 Šechanija turėjo sūnų Šemają, Šemajos sūnūs: Hatušas, Igalas, Bariachas, Nearija ir Šafatas, iš viso šeši. 
\par 23 Nearijos sūnūs: Eljoenajas, Ezekijas ir Azrikamas­trys. 
\par 24 Eljoenajo sūnūs: Hodavijas, Eljašibas, Pelaja, Akubas, Joananas, Delaja ir Ananis­septyni.



\chapter{4}

\par 1 Judo palikuonys: Perecas, Esromas, Karmis, Hūras ir Šobalas. 
\par 2 Šobalo sūnus Reaja buvo Jahato tėvas, o Jahatas­Ahumajo ir Lahado tėvas. Tai coriečių giminės. 
\par 3 Etamo sūnūs: Jezreelis, Išma, Idbašas, jų sesuo buvo vardu Haclelponė. 
\par 4 Gedoro tėvas­Penuelis, Hušos­ Ezeras. Šitie buvo Efratos pirmagimio Hūro, Betliejaus tėvo, palikuonys. 
\par 5 Tekojos tėvas Ašhūras turėjo dvi žmonas: Helą ir Naarą. 
\par 6 Su Naara jis turėjo Ahuzamą, Heferą, Temaną ir Ahaštarą. 
\par 7 Su Hela­Ceretą, Coharą ir Etnaną. 
\par 8 Kocas buvo Anubo ir Hacobebo tėvas ir Harumo sūnaus Aharhelio giminės protėvis. 
\par 9 Jabecas pasižymėjo tarp savo brolių. Jo motina jį praminė Jabecu, nes jo gimdymas buvo sunkus. 
\par 10 Jabecas šaukėsi Izraelio Dievo: “Norėčiau, kad Tu mane laimintum ir praplėstum mano krašto sienas, kad būtum su manimi ir saugotum mane nuo pikto, kad nepatirčiau vargo”. Dievas suteikė jam tai, ko jis prašė. 
\par 11 Šuho brolis Kelubas buvo Mehyro tėvas, Mehyras­Eštono tėvas, 
\par 12 Eštonui gimė Bet Rafa, Paseachas ir Tehina, Ir Nahašo tėvas. Šitie vyrai gyveno Rechos mieste. 
\par 13 Kenazo sūnūs: Otnielis ir Seraja. Otnielio sūnus­Hatatas. 
\par 14 Meonotajas buvo Ofros tėvas. Seraja iš Amatininkų slėnio buvo Joabo tėvas; jie buvo amatininkai. 
\par 15 Jefunės sūnaus Kalebo sūnūs: Iruvas, Ela ir Naamas. Elos sūnus buvo Kenazas. 
\par 16 Jehalėlelio sūnūs: Zifa, Zifas, Tirija ir Asarelis. 
\par 17 Ezro sūnūs: Jeteras, Meredas, Eferas ir Jalonas; be to, jam gimė Mirjama, Šamajas ir Išbachas, Eštemojos tėvas. 
\par 18 Jo žmona Jahudija pagimdė Jaredą­Gedoro tėvą, Heberą­Sochojo tėvą ir Jekutielį­Zanoacho tėvą. Šitie yra sūnūs Bitijos, faraono dukters, kurią paėmė Meredas. 
\par 19 Jo žmona Hodija buvo sesuo Nahamo, kuris buvo garmito Keilos ir maakatito Eštemojo tėvas. 
\par 20 Šimono sūnūs: Amnonas, Rina, Ben Hananas ir Tilonas. Išio sūnūs: Zohetas ir Benzohetas. 
\par 21 Judo sūnaus Šelos palikuonys: Lechos tėvas Eras, Marešos tėvas Lada, drobės audėjų giminė iš Bet Ašbėjos namų, 
\par 22 Jokimas ir Kozebos gyventojai, taip pat Jehoašas ir Sarafas, kuris viešpatavo Moabe, ir Jasubilehemas (pagal senus užrašus). 
\par 23 Jie buvo puodžiai, Netaimo bei Gederos gyventojai; jie gyveno pas karalių ir jam dirbo. 
\par 24 Simeono palikuonys: Nemuelis, Jaminas, Jaribas, Zerachas ir Saulius. 
\par 25 Jo sūnus­Šalumas, jo sūnus­ Mibsamas, jo sūnus­Mišma. 
\par 26 Mišmos sūnūs: Hamuelis, jo sūnus­Zakūras, o jo sūnus­Šimis. 
\par 27 Šimis turėjo šešiolika sūnų bei šešias dukteris; bet jo broliai neturėjo daug vaikų ir jų giminė nedaugėjo kaip Judo. 
\par 28 Jie gyveno Beer Šeboje, Moladoje, Hazar Šuale, 
\par 29 Baloje, Ezeme, Tolade, 
\par 30 Betuelyje, Hormoje, Ciklage, 
\par 31 Bet Markabote, Hazar Susime, Bet Biryje ir Šaaraime. Tai buvo jų miestai, iki pradėjo karaliauti Dovydas. 
\par 32 Jų kaimai buvo: Etamas, Ainas, Rimonas, Tochenas ir Ašanas; 
\par 33 jiems priklausė kaimai, esą prie šitų miestų, iki Baalo. Tai buvo jų gyvenvietės ir jų kilmė. 
\par 34 Mešobabas, Jamlechas, Amacijos sūnus Joša, 
\par 35 Joelis, Asielio sūnaus Serajos sūnaus Jošibijos sūnus Jehuvas, 
\par 36 Eljoenajas, Jaakoba, Ješohaja, Asaja, Adielis, Jesimielis, Benaja 
\par 37 ir Šemajos sūnaus Šimrio sūnaus Jedajos sūnaus Alono sūnaus Šifio sūnus Ziza. 
\par 38 Čia paminėti kunigaikščių vardai, kurie buvo giminių vadai. Jų šeimos labai augo; 
\par 39 jie su visa manta traukėsi į Gedoro apylinkes, į rytus nuo slėnio, norėdami susirasti ganyklų savo bandoms. 
\par 40 Šalis buvo plati ir rami, jie susirado labai gerų ganyklų. Ten pirmiau gyveno chamitai. 
\par 41 Visi čia išvardinti atsikėlė į tą kraštą Judo karaliaus Ezekijo laikais. Jie nugalėjo meunus, jų palapines sunaikino ir, juos visiškai pavergę, apsigyveno jų vietoje ir tebegyvena iki šių dienų, nes ten buvo geros ganyklos jų bandoms. 
\par 42 Be to, dalis tų simeonitų, penki šimtai vyrų, nužygiavo, vadovaujant Pelatijai, Nearijai, Refajai ir Uzieliui, Išio sūnums, į Seyro aukštumas. 
\par 43 Nugalėję ir išnaikinę amalekiečių likutį, ten apsigyveno ir tebegyvena iki šios dienos.



\chapter{5}

\par 1 Rubenas buvo Izraelio pirmagimis. Kadangi jis sutepė savo tėvo patalą, jo pirmagimio teisė buvo atiduota Izraelio sūnaus Juozapo sūnums; Rubenas nebuvo įrašytas į sąrašą pirmagimio teisėmis. 
\par 2 Nors Judas buvo galingiausias tarp savo brolių ir iš jo kilo kunigaikštis, bet pirmagimio teisė atiteko Juozapui. 
\par 3 Izraelio pirmagimio Rubeno sūnūs: Henochas, Paluvas, Hecronas ir Karmis. 
\par 4 Joelio palikuonys: Šemaja, jo sūnus­Gogas, jo sūnus­Šimis, 
\par 5 jo sūnus­Michėjas, jo sūnus­ Reaja, jo sūnus­Baalas, 
\par 6 jo sūnus­Beera, kurį ištrėmė Asirijos karalius Tiglat Pileseras; jis buvo rubenų kunigaikštis. 
\par 7 Rubeno palikuonių šeimų sąrašas: Jejelis, Zacharijas 
\par 8 ir Bela, sūnus Azazo, sūnaus Šemos, sūnaus Joelio. Jo žemės tęsėsi nuo Aroerio iki Nebojo ir Baal Meono. 
\par 9 Į rytus jis išsiplėtė iki dykumos ir Eufrato upės, nes turėjo daug galvijų Gileado šalyje. 
\par 10 Sauliaus laikais jis kariavo su hagarais. Juos nugalėjęs, jis apsigyveno jų palapinėse visoje rytinėje Gileado šalyje. 
\par 11 Gado sūnūs gyveno greta jų Bašano šalyje iki Salchos: 
\par 12 vyriausiasis Joelis, po jo Šafamas, Jakanas ir Šafatas Bašane. 
\par 13 Jų broliai: Mykolas, Mešulamas, Šeba, Jorajas, Jakanas, Zija ir Eberas; 
\par 14 šitie buvo vaikai Abihailio, sūnaus Hūrio, sūnaus Jaroacho, sūnaus Gileado, sūnaus Mykolo, sūnaus Ješišajo, sūnaus Jachdajo, sūnaus Būzo. 
\par 15 Ahis, Gūnio sūnaus Abdielio sūnus, buvo vyriausiasis jų šeimose. 
\par 16 Jie gyveno Gileade, Bašane ir jiems priklausančiuose miestuose bei visuose Šarono priemiesčiuose iki savo ribų. 
\par 17 Jie visi buvo surašyti giminėmis Judo karaliaus Joatamo ir Izraelio karaliaus Jeroboamo laikais. 
\par 18 Rubenų, gadų ir pusėje Manaso giminės buvo karingų vyrų, galinčių nešioti skydą, kardą ir įtempti lanką; įgudusių kovoje buvo keturiasdešimt keturi tūkstančiai septyni šimtai šešiasdešimt vyrų. 
\par 19 Jie kovojo su hagarais: Jetūru, Nafišu ir Nodabu. 
\par 20 Jie susilaukė pagalbos ir nugalėjo hagarus bei visus jų sąjungininkus. Jie šaukėsi Dievo mūšyje, ir Jis išklausė juos, nes jie Juo pasitikėjo. 
\par 21 Jie paėmė visa, ką priešai turėjo: penkiasdešimt tūkstančių kupranugarių, du šimtus penkiasdešimt tūkstančių avių, du tūkstančius asilų ir šimtą tūkstančių žmonių. 
\par 22 Daug buvo nužudyta, nes mūšis buvo Dievo. Jie gyveno jų vietoje iki ištrėmimo. 
\par 23 Pusė Manaso giminės gyveno Bašano krašte; jie išsiplėtė iki Baal Hermono, Senyro ir Hermono kalnų. 
\par 24 Jų šeimų galvos buvo Eferas, Išis, Elielis, Azrielis, Jeremija, Hodavija ir Jachdielis; galingi kariai, žymūs vyrai. 
\par 25 Jie nusikalto savo tėvų Dievui, paleistuvaudami su dievais svetimų tautų, kurias Dievas buvo išnaikinęs tuose kraštuose. 
\par 26 Izraelio Dievas pakėlė Asirijos karalių Pulą, kuris yra Tiglat Palasaras, ir jis ištrėmė rubenus, gadus ir pusę Manaso giminės į Halachą, Haborą ir Harą prie Gozano upės, kur jie tebegyvena iki šios dienos.



\chapter{6}

\par 1 Levio sūnūs: Geršonas, Kehatas ir Meraris. 
\par 2 Kehato sūnūs: Amramas, Iccharas, Hebronas ir Uzielis. 
\par 3 Amramo vaikai: Aaronas, Mozė ir Mirjama. Aarono sūnūs: Nadabas, Abihuvas, Eleazaras ir Itamaras. 
\par 4 Eleazaro palikuonys: Eleazaras buvo Finehaso tėvas, Finehasas­ Abišūvos, 
\par 5 Abišūvas­Bukio, Bukis­Ucio, 
\par 6 Ucis­Zerachijos, Zerachija­ Merajoto, 
\par 7 Merajotas­Amarijos, Amarija­Ahitubo, 
\par 8 Ahitubas­Cadoko, Cadokas­ Ahimaaco, 
\par 9 Ahimaacas­Azarijos, Azarija­ Johanano, 
\par 10 Johananas­Azarijos, kuris ėjo kunigo tarnystę šventykloje, kurią Saliamonas pastatė Jeruzalėje. 
\par 11 Azarija buvo Amarijos tėvas, Amarija—Ahitubo, 
\par 12 Ahitubas—Cadoko, Cadokas­ Šalumo, 
\par 13 Šalumas­Hilkijos, Hilkija­ Azarijos, 
\par 14 Azarija­Serajos, Seraja­Jehocadako, 
\par 15 Jehocadakas buvo išvestas, kai Viešpats ištrėmė Judą ir Jeruzalę karaliaus Nebukadnecaro rankomis. 
\par 16 Levio sūnūs: Geršomas, Kehatas ir Meraris. 
\par 17 Geršomo sūnų vardai buvo Libnis ir Šimis. 
\par 18 Kehato sūnūs: Amramas, Iccharas, Hebronas ir Uzielis. 
\par 19 Merario sūnūs: Machlis ir Mušis. Šitos yra Levio giminės šeimos. 
\par 20 Geršomo palikuonys: Libnis, jo sūnus­Jahatas, jo sūnus­Zima, 
\par 21 jo sūnus­Joachas, jo sūnus­ Idojas, jo sūnus­Zerachas, jo sūnus­Jeotrajas. 
\par 22 Kehato palikuonys: jo sūnus­ Aminadabas, jo sūnus­Korachas, jo sūnus­Asiras, 
\par 23 jo sūnus­Elkana, jo sūnus­ Ebijasafas, jo sūnus­Asiras, 
\par 24 jo sūnus­Tahatas, jo sūnus­ Ūrielis, jo sūnus­Uzija, jo sūnus­Saulius. 
\par 25 Elkanos sūnūs: Amasajas ir Ahimotas; 
\par 26 jo sūnus­Elkana, jo sūnus­ Cofajas, jo sūnus­Nahatas, 
\par 27 jo sūnus­Eliabas, jo sūnus­Jerohamas, jo sūnus­Elkana. 
\par 28 Samuelio sūnūs: pirmagimis­ Joelis, antrasis­Abija. 
\par 29 Merario palikuonys: jo sūnus­ Machlis, jo sūnus­Libnis, jo sūnus­Šimis, jo sūnus­Uza, 
\par 30 jo sūnus Šima, jo sūnus­Hagija, jo sūnus­Asaja. 
\par 31 Šitie vyrai buvo Dovydo paskirti tarnauti giesmėmis Viešpaties namuose, kai skrynia buvo padėta į jos vietą. 
\par 32 Kol Saliamonas pastatė Viešpaties namus Jeruzalėje, jie tarnavo giedodami prie Susitikimo palapinės ir atlikdami tarnystę pagal savo eilę. 
\par 33 Iš Kehato giminės buvo giedotojas Hemanas, sūnus Joelio, sūnaus Samuelio, 
\par 34 sūnaus Elkanos, sūnaus Jerohamo, sūnaus Elielio, sūnaus Toacho, 
\par 35 sūnaus Cūfo, sūnaus Elkanos, sūnaus Mahato, sūnaus Amasajo, 
\par 36 sūnaus Elkanos, sūnaus Joelio, sūnaus Azarijos, sūnaus Sofonijos, 
\par 37 sūnaus Tahato, sūnaus Asiro, sūnaus Ebjasafo, sūnaus Koracho, 
\par 38 sūnaus Iccharo, sūnaus Kehato, sūnaus Levio, sūnaus Izraelio. 
\par 39 Jo brolis Asafas buvo jam iš dešinės. Asafas buvo sūnus Berechijo, sūnaus Šimos, 
\par 40 sūnaus Mykolo, sūnaus Baasėjos, sūnaus Malkijos, 
\par 41 sūnaus Etnio, sūnaus Zeracho, sūnaus Adajos, 
\par 42 sūnaus Etano, sūnaus Zimo, sūnaus Šimio, 
\par 43 sūnaus Jahato, sūnaus Geršomo, sūnaus Levio. 
\par 44 Jų broliai iš Merario sūnų buvo jiems iš kairės: Etanas, sūnus Kišio, sūnaus Abdžio, sūnaus Malucho, 
\par 45 sūnaus Hašabijos, sūnaus Amacijos, sūnaus Hilkijos, 
\par 46 sūnaus Amcio, sūnaus Banio, sūnaus Šemero, 
\par 47 sūnaus Machlio, sūnaus Mušio, sūnaus Merario, sūnaus Levio. 
\par 48 Jų broliai levitai buvo paskirti įvairiems darbams Dievo namų palapinėje. 
\par 49 Aaronas ir jo sūnūs aukodavo aukas ant deginamųjų aukų aukuro ir ant smilkymo aukuro; jie turėdavo tarnauti Šventų švenčiausiojoje ir sutaikinti Izraelį, kaip Dievo tarnas Mozė buvo įsakęs. 
\par 50 Aarono palikuonys: jo sūnus­ Eleazaras, jo sūnus­Finehasas, jo sūnus­Abišūva, 
\par 51 jo sūnus­Bukis, jo sūnus­Uzis, jo sūnus­Zerachija, 
\par 52 jo sūnus­Merajotas, jo sūnus­ Amarija, jo sūnus­Ahitubas, 
\par 53 jo sūnus­Cadokas, jo sūnus­ Ahimaacas. 
\par 54 Šitos yra Aarono palikuonių iš Kehato giminės gyvenamos vietos, kaip jiems krito burtas. 
\par 55 Jiems davė Hebroną Judo žemėje ir ganyklas aplink jį, 
\par 56 tačiau miesto laukus su jo kaimais gavo Jefunės sūnus Kalebas. 
\par 57 Aarono palikuonims iš Judo žemių davė prieglaudos miestą Hebroną su ganyklomis, Libną su ganyklomis, Jatyrą su ganyklomis, Eštemoją su ganyklomis, 
\par 58 Hilezą su ganyklomis, Debyrą su ganyklomis, 
\par 59 Ašaną su ganyklomis, Bet Šemešą su ganyklomis; 
\par 60 iš Benjamino giminės­Gebą su ganyklomis, Alemetą su ganyklomis ir Anatotą su ganyklomis. Iš viso buvo trylika miestų visoms jų šeimoms. 
\par 61 Likusieji Kehato palikuonys gavo savo žemes iš pusės Manaso giminės­dešimt miestų. 
\par 62 Geršomo sūnų palikuonys gavo trylika miestų iš Isacharo, Ašero, Naftalio ir pusės Manaso giminės Bašane. 
\par 63 Merario sūnų palikuonys gavo žemes iš Rubeno, Gado ir Zabulono giminių­dvylika miestų. 
\par 64 Izraelitai davė levitams miestų su ganyklomis 
\par 65 burtų keliu iš Judo, Simeono ir Benjamino giminių. 
\par 66 Kai kurie Kehato palikuonys gavo miestus su ganyklomis iš Efraimo giminės: 
\par 67 prieglaudos miestą Sichemą Efraimo aukštumose, Gezerą, 
\par 68 Jokmeamą, Bet Horoną, 
\par 69 Ajaloną ir Gat Rimoną; 
\par 70 iš pusės Manaso giminės­Anerą ir Bileamą. 
\par 71 Gersomitai gavo žemės: iš pusės Manaso giminės­Golaną Bašane ir Aštarotą; 
\par 72 iš Isacharo giminės­Kedešą, Daberatą, 
\par 73 Ramotą ir Anemą; 
\par 74 iš Ašero giminės­Mašalą, Abdoną, 
\par 75 Hukoką ir Rehobą; 
\par 76 iš Neftalio giminės­Kedešą Galilėjoje, Hamoną ir Kirjataimą; visus minėtus miestus jie gavo su ganyklomis. 
\par 77 Meraritai gavo: iš Zabulono giminės Rimoną ir Taborą; 
\par 78 anapus Jordano ties Jerichu iš Rubeno giminės­Becero miestą dykumoje, Jahcos miestą, 
\par 79 Kedemoto ir Mefaato miestus; 
\par 80 iš Gado giminės­Ramotą Gileade ir Machanaimą, 
\par 81 Hešboną ir Jazerą. Visus miestus jie gavo su ganyklomis.



\chapter{7}


\par 1 Isacharo palikuonys: Tola, Pūva, Jašubas ir Šimronas­keturi sūnūs. 
\par 2 Tolos sūnūs: Uzis, Refaja, Jerielis, Jachmajas, Ibsamas ir Samuelis­šeimų vadai, narsūs kovotojai. Jų giminės palikuonių Dovydo laikais buvo dvidešimt du tūkstančiai šeši šimtai. 
\par 3 Uzio sūnaus Izrachijos sūnūs: Mykolas, Abdija, Joelis ir Išija; jie visi­šeimų vadai. 
\par 4 Jų giminės palikuonių, skaičiuojant šeimomis ir kariuomenės būriais, buvo trisdešimt šeši tūkstančiai, nes jie turėjo daug žmonų ir vaikų. 
\par 5 Visoje Isacharo giminėje buvo aštuoniasdešimt septyni tūkstančiai narsių karių. 
\par 6 Benjamino sūnūs: Bela, Becheras ir Jediaelis. 
\par 7 Belos sūnūs: Ecbonas, Uzis, Uzielis, Jerimotas ir Iris; jie­šeimų vadai, narsūs kariai. Jų šeimų palikuonių buvo dvidešimt du tūkstančiai trisdešimt keturi. 
\par 8 Bechero sūnūs: Cemyra, Joašas, Eliezeras, Eljoenajas, Omris, Jeremotas, Abija, Anatotas ir Alemetas. 
\par 9 Šeimų palikuonių sąraše buvo dvidešimt tūkstančių du šimtai narsių karių. 
\par 10 Jediaelio sūnus­Bilhanas; Bilhano sūnūs: Jeušas, Benjaminas, Ehudas, Kenaana, Zetanas, Taršišas ir Ahišaharas. 
\par 11 Jie visi buvo giminės šeimų vadai; narsių karių, tinkančių karui, buvo septyniolika tūkstančių du šimtai vyrų. 
\par 12 Iro sūnūs: Šupimas ir Hupimas; Ahero sūnus­Hušimas. 
\par 13 Naftalio sūnūs: Jahacielis, Gūnis, Jeceras ir Šalumas, Bilhos palikuonys. 
\par 14 Manaso sūnūs, kuriuos pagimdė jo sugulovė aramėjė: Asrielis ir Machiras, Gileado tėvas. 
\par 15 Machiras paėmė į žmonas Hupimo ir Šupimo seserį, vardu Maaka; Machiro antrasis sūnus buvo Celofhadas, kuris turėjo tik dukteris. 
\par 16 Machiro žmona Maaka turėjo du sūnus: Perešą ir Šerešą; Šerešo sūnūs buvo Ulamas ir Rekemas. 
\par 17 Ulamo sūnus­Bedanas. Šitie buvo Gileado, Machiro sūnaus, Manaso anūko, palikuonys. 
\par 18 Jo sesuo Hamolecheta pagimdė Išhodą, Abiezerą ir Machlą. 
\par 19 Šemidos sūnūs: Achjanas, Šechemas, Likhis ir Aniamas. 
\par 20 Efraimo sūnus­Šutelachas, jo sūnus­Beredas, jo sūnus­Tahatas, jo sūnus­Eleadas, jo sūnus­ Tahatas, 
\par 21 jo sūnus­Zabadas, jo sūnus­ Šutelachas, taip pat Ezeras ir Eleadas. Juos nužudė Gato vyrai, kai jie norėjo nuvaryti jų galvijus. 
\par 22 Jų tėvas Efraimas ilgai gedėjo savo sūnų. Jo broliai atėjo jį paguosti. 
\par 23 Jo žmona pagimdė dar vieną sūnų, kurį pavadino Berija, nes vaikas gimė šeimos nelaimės metu. 
\par 24 Jo duktė Šeera įkūrė žemutinį ir aukštutinį Bet Horoną ir Uzen Šeerą. 
\par 25 Refachas buvo jo sūnus, taip pat Rešefas, jo sūnus­Telachas, jo sūnus­Tahanas, 
\par 26 jo sūnus­Ladanas, jo sūnus­ Amihudas, jo sūnus­Elišama, 
\par 27 jo sūnus­Nūnas, jo sūnus­Jozuė. 
\par 28 Jų nuosavybė ir gyvenvietės buvo Betelis su miesteliais, rytuose Naaranas, vakaruose Gazeras su miesteliais, taip pat ir Sichemo bei Gazos miestai su miesteliais. 
\par 29 Manaso palikuonys gyveno Bet Šeane, Taanache, Megide, Dore ir visuose tų miestų apylinkių kaimuose. Tai buvo Izraelio sūnaus Juozapo palikuonys. 
\par 30 Ašero sūnūs: Imna, Išva, Išvis, Berija ir jų sesuo Seracha. 
\par 31 Berijos sūnūs: Heberas ir Malkielis, kuris buvo Birzajo tėvas. 
\par 32 Heberas buvo Jafleto, Šomero, Hotamo ir jų sesers Šuvos tėvas. 
\par 33 Jafleto sūnūs: Pasachas, Bimhalas ir Ašvatas. 
\par 34 Jo brolio Šemero sūnūs: Ahis, Rohga, Jehuba ir Aramas. 
\par 35 Jo brolio Helemo sūnūs: Cofachas, Imna, Šelešas ir Amalas. 
\par 36 Cofacho sūnūs: Suachas, Harneferas, Šualas, Beris, Imra, 
\par 37 Beceras, Hodas, Šama, Šilša, Itranas ir Beera. 
\par 38 Jeterio sūnūs: Jefunė, Pispa ir Ara. 
\par 39 Ulos sūnūs: Arachas, Hanielis ir Ricija. 
\par 40 Šitie buvo Ašero palikuonys, šeimų galvos, narsūs kariai, vyriausieji kunigaikščiai. Jų skaičius buvo dvidešimt šeši tūkstančiai karo tarnybai tinkamų vyrų.



\chapter{8}


\par 1 Benjamino pirmagimis buvo Bela, kiti­Ašbelis, Achrachas, 
\par 2 Noha ir Rafa. 
\par 3 Belos palikuonys: Adaras, Gera, Abihudas, 
\par 4 Abišūva, Naamanas, Ahoachas, 
\par 5 Gera, Šefufanas ir Huramas. 
\par 6 Ehudo palikuonys buvo Gebos gyventojai, šeimų vadai; jie buvo ištremti į Manahatą: 
\par 7 Naamanas, Ahija ir Gera, kuris buvo Uzos ir Ahihudo tėvas. 
\par 8 Šaharaimas susilaukė sūnų Moabo krašte, atleidęs savo žmonas Hušimą ir Baarą. 
\par 9 Jis vedė Hodešą ir susilaukė septynių sūnų: Jobabo, Cibijo, Mešo, Malkamo, 
\par 10 Jeuco, Sachijos ir Mirmos. Šitie jo sūnūs buvo šeimų vadai. 
\par 11 Su Hušima jis turėjo Abitubą ir Elpaalį. 
\par 12 Elpaalio sūnūs: Eberas, Mišamas ir Šemedas, kuris pastatė Onojo ir Lodo miestus bei jų miestelius. 
\par 13 Berija ir Šema buvo Ajalono gyventojų šeimų vadai; jie privertė pasitraukti Gato gyventojus. 
\par 14 Berijos sūnūs: Achjojas, Šašakas, Jeremotas, 
\par 15 Zebadija, Aradas, Ederas, 
\par 16 Mykolas, Išpa ir Joha. 
\par 17 Elpaalio sūnūs: Zebadija, Mešulamas, Hizkis, Heberas, 
\par 18 Išmerajas, Izlija ir Jobabas. 
\par 19 Šimio sūnūs: Jakimas, Zichris, Zabdis, 
\par 20 Elienajas, Ciletajas, Elielis, 
\par 21 Adaja, Beraja ir Šimratas. 
\par 22 Šašako sūnūs: Išpanas, Eberas, Elielis, 
\par 23 Abdonas, Zichris, Hananas, 
\par 24 Hananija, Elamas, Antotija, 
\par 25 Ifdėja ir Penuelis. 
\par 26 Jerohamo sūnūs: Šamšerajas, Šeharija, Atalija, 
\par 27 Jaarešija, Elija ir Zichris. 
\par 28 Šitie buvo šeimų vadai. Jie gyveno Jeruzalėje. 
\par 29 Gibeone gyveno Gibeono tėvas su žmona Maaka. 
\par 30 Jų pirmagimis sūnus buvo Abdonas, kiti—Cūras, Kišas, Baalas, Nadabas, 
\par 31 Gedoras, Achjojas ir Zecheris. 
\par 32 Miklotui gimė Šima. Jie gyveno šalia savo brolių Jeruzalėje. 
\par 33 Neras buvo Kišo tėvas, Kišas­ Sauliaus, Saulius­Jehonatano, Malkišūvos, Abinadabo ir Ešbaalo tėvas. 
\par 34 Jehonatano sūnus buvo Merib Baalas, o Merib Baalo sūnus­ Michėjas. 
\par 35 Michėjo sūnūs: Pitonas, Melechas, Tarėja ir Ahazas. 
\par 36 Ahazas buvo Jehoados tėvas, Jehoada­Alemeto, Azmaveto ir Zimrio, Zimris­Mocos, 
\par 37 Moca buvo Binėjos tėvas, Binėja­Rafos, Rafa­Eleasos, o Eleasa­Acelio. 
\par 38 Acelis turėjo šešis sūnus: Azrikamą, Bochruvą, Izmaelį, Šeariją, Abdiją ir Hananą. 
\par 39 Jo brolio Ešeko sūnūs: pirmagimis­Ulamas, kiti­Jeušas ir Elifeletas. 
\par 40 Ulamo sūnūs buvo narsūs kariai ir geri šauliai. Ulamas turėjo šimtą penkiasdešimt palikuonių­ sūnų ir anūkų. Visi šie yra Benjamino sūnūs.



\chapter{9}

\par 1 Visi izraelitai buvo surašyti giminėmis Izraelio ir Judo karalių knygose. Jie buvo ištremti į Babiloną dėl savo nusikaltimų. 
\par 2 Pirmieji, apsigyvenę Izraelio miestuose, buvo izraelitai, kunigai, levitai ir šventyklos tarnai. 
\par 3 Jeruzalėje gyveno dalis Judo, Benjamino, Efraimo ir Manaso giminių palikuonių. 
\par 4 Judo sūnaus Pereco palikuonys: Amihudas, Utajas, Omris, Imris ir Banis su šeimomis. 
\par 5 Šilojiečių palikuonys: pirmagimis­Asaja ir jo sūnūs. 
\par 6 Zeracho palikuonys: Jeuelis ir jų giminės­šeši šimtai devyniasdešimt žmonių. 
\par 7 Benjaminai: Saluvas­sūnus Mešulamo, sūnaus Hodavijos, sūnaus Hasenuvos. 
\par 8 Jerohamo sūnus Ibnėja, Michrio sūnaus Uzio sūnus Ela ir Mešulamas­sūnus Šefatijos, sūnaus Reuelio, sūnaus Ibnijos. 
\par 9 Jų giminės žmonių skaičius buvo devyni šimtai penkiasdešimt šeši. Visi minimi asmenys buvo šeimų vadai. 
\par 10 Gyvenę Jeruzalėje kunigai: Jedaja, Jehojaribas, Jachinas 
\par 11 ir Azarija, sūnus Hilkijos, kuris buvo sūnus Mešulamo, sūnaus Cadoko, sūnaus Merajoto, sūnaus Ahitubo; jis buvo Dievo namų valdytojas. 
\par 12 Jerohamo sūnaus Adajos palikuonys buvo Pašhūras, Malkija, Masajas, Adielis, Jachzera, Mešulamas, Mešilemitas ir Imeras. 
\par 13 Kunigų, kurie buvo giminės šeimų vyresnieji, buvo tūkstantis septyni šimtai šešiasdešimt pajėgių vyrų šventyklos tarnybai. 
\par 14 Levitai: Šemajas­sūnus Hašubo, sūnaus Azrikamo, sūnaus Hašabijo iš merarių; 
\par 15 Bakbakaras, Herešas bei Galalas ir Matanija­sūnus Michėjo, sūnaus Zichrio, sūnaus Asafo; 
\par 16 Abdija­sūnus Semajos, sūnaus Galalo, sūnaus Jedutūno, ir Elkanos sūnaus Asos sūnus Berechija, kuris gyveno netofiečių kaimuose. 
\par 17 Vartininkai buvo Šalumas, Akubas, Talmonas, Ahimanas ir jų broliai. Šalumas buvo jų viršininkas. 
\par 18 Jie ėjo sargybą karaliaus vartų rytų pusėje. Šie levitai buvo vartų sargai. 
\par 19 Šalumas­sūnus Korės, sūnus Ebjasafo, sūnus Koracho, ir jo broliai iš jo tėvo namų, korachai, buvo Viešpaties palapinės durų sargai. Jų tėvai ėjo sargų pareigas prie Viešpaties palapinės įėjimo. 
\par 20 Eleazaro sūnus Finehasas anksčiau buvo jų viršininkas; Viešpats buvo su juo. 
\par 21 Mešelemijos sūnus Zacharija buvo sargas prie Susitikimo palapinės įėjimo. 
\par 22 Durų sargų buvo du šimtai dvylika. Jie buvo įtraukti į giminių sąrašus savo vietovėse. Juos paskyrė toms pareigoms Dovydas ir regėtojas Samuelis, nes jais pasitikėjo. 
\par 23 Jie ir jų sūnūs buvo įpareigoti eiti sargybą prie Viešpaties palapinės. 
\par 24 Sargyba budėdavo visose keturiose palapinės pusėse: rytų, vakarų, šiaurės ir pietų. 
\par 25 Jų broliai, gyvenantieji kaimuose, kas septintą dieną privalėjo juos pakeisti ir septynias dienas eiti tas pareigas. 
\par 26 Keturi vyresnieji vartininkai buvo levitai. Jie buvo paskirti toms pareigoms ir atsakingi už Dievo namų patalpas bei turtus. 
\par 27 Jų pareiga buvo kas naktį eiti sargybą prie Dievo namų, o rytą juos atidaryti. 
\par 28 Kai kuriems iš jų buvo pavesta tvarkyti tarnavimo reikmenis. Jie tikrindavo juos įnešant ir išnešant. 
\par 29 Kiti iš jų buvo paskirti prižiūrėti daiktams, visiems šventyklos reikmenims: smulkiems miltams, vynui, aliejui, smilkalams ir kvepalams. 
\par 30 Kunigų pareigos buvo paruošti kvepalų mišinį. 
\par 31 Levitas Matitija, Šalumo pirmagimis, korachas, prižiūrėjo kepimą. 
\par 32 Kai kuriems Kehato giminės levitams buvo pavesta rūpintis padėtine duona, ją paruošti sabatui. 
\par 33 Kai kurie giesmininkai, levitų šeimų vadai, buvo laisvi nuo kitų pareigų. Jie gyveno prie šventyklos, nes dieną ir naktį jie turėjo tarnauti. 
\par 34 Šitie buvo savo kartos levitų šeimų vyriausieji ir gyveno Jeruzalėje. 
\par 35 Gibeone gyveno Gibeono tėvas Jejelis, kurio žmona buvo vardu Maaka. 
\par 36 Jo pirmagimis sūnus buvo Abdonas, kiti­Cūras, Kišas, Baalas, Neras, Nadabas, 
\par 37 Gedoras, Achjojas, Zacharija ir Miklotas. 
\par 38 Miklotas buvo Šimamo tėvas; jie gyveno šalia savo brolių Jeruzalėje. 
\par 39 Neras buvo Kišo tėvas, Kišas­ Sauliaus, Saulius­Jehonatano, Malkišuvos, Abinadabo ir Ešbaalo tėvas. 
\par 40 Jehonatano sūnus buvo Merib Baalas, o Merib Baalas buvo Michėjo tėvas. 
\par 41 Michėjo sūnūs: Pitonas, Melechas, Tarėja ir Ahazas. 
\par 42 Ahazas buvo Jaros tėvas, Jara­ Alemeto, Azmaveto ir Zimrio, Zimris­Mocos, 
\par 43 o Moca­Binėjos. Jo sūnus buvo Refaja, jo sūnus­Eleasa, jo sūnus­Azelis. 
\par 44 Azelis turėjo šešis sūnus: Azrikamą, Bochruvą, Izmaelį, Šeariją, Abdiją ir Hananą.



\chapter{10}


\par 1 Filistinai kariavo su Izraeliu. Izraelio vyrai bėgo nuo filistinų ir krito nužudyti ant Gilbojos kalno. 
\par 2 Filistinai persekiojo Saulių bei jo sūnus ir nužudė Jehonataną, Abinadabą ir Malkišūvą. 
\par 3 Vyko smarki kova prieš Saulių, šauliai pataikė į Saulių ir jį sužeidė. 
\par 4 Tada Saulius tarė savo ginklanešiui: “Išsitrauk kardą ir juo perverk mane, kad šitie neapipjaustytieji atėję neišniekintų manęs”. Bet jo ginklanešys nesutiko, nes jis labai bijojo. Tada Saulius, paėmęs savo kardą, krito ant jo. 
\par 5 Jo ginklanešys, pamatęs, kad Saulius miręs, irgi puolė ant savo kardo ir mirė kartu. 
\par 6 Taip mirė Saulius, jo trys sūnūs ir visi jo namai. 
\par 7 Izraelitai, kurie gyveno slėnyje, pamatę, kad jie pabėgo, o Saulius bei jo sūnūs mirę, paliko savo miestus ir bėgo. Atėję filistinai apsigyveno juose. 
\par 8 Kitą dieną filistinai, atėję apiplėšti užmuštųjų, rado Saulių ir tris jo sūnus žuvusius ant Gilbojos kalno. 
\par 9 Jie išrengė jį, paėmė jo galvą bei ginklus ir nešiojo po filistinų kraštą, skelbdami apie pergalę savo stabams ir tautai. 
\par 10 Jo ginklus jie padėjo savo dievų namuose, o jo galvą prikalė Dagono šventykloje. 
\par 11 Jabeš Gileado gyventojai išgirdo, ką filistinai padarė Sauliui. 
\par 12 Jų visi narsūs vyrai pakilo, paėmė Sauliaus bei jo sūnų lavonus, parnešė juos į Jabešą ir, palaidoję po ąžuolu, pasninkavo septynias dienas. 
\par 13 Taip mirė Saulius dėl savo nusikaltimo Viešpačiui ir Jo žodžiui, kurio jis nesilaikė, ir dėl to, kad ieškojo patarimo pas mirusiųjų dvasių iššaukėją. 
\par 14 Jis neieškojo Viešpaties, todėl Viešpats nužudė jį ir atidavė karalystę Jesės sūnui Dovydui.



\chapter{11}


\par 1 Visi izraelitai susirinko pas Dovydą į Hebroną ir tarė: “Mes esame tavo kūnas ir kaulas. 
\par 2 Anksčiau, kai Saulius buvo mūsų karalius, tu išvesdavai ir įvesdavai Izraelį; Viešpats, tavo Dievas, tau pažadėjo: ‘Tu ganysi mano tautą Izraelį ir būsi jo kunigaikščiu’ ”. 
\par 3 Visi Izraelio vyresnieji atėjo pas karalių į Hebroną. Dovydas padarė su jais sandorą Hebrone. Jie patepė Dovydą Izraelio karaliumi, kaip Viešpats buvo paskelbęs per Samuelį. 
\par 4 Dovydas ir visas Izraelis ėjo į Jeruzalę (kitaip Jebusą), kur gyveno jebusiečiai. 
\par 5 Jebuso gyventojai sakė Dovydui: “Tu neįeisi į miestą”. Tačiau Dovydas paėmė Siono tvirtovę, tai yra Dovydo miestą. 
\par 6 Dovydas tarė: “Kas nugalės jebusiečius, tas taps kariuomenės vadu”. Pirmasis miestą puolė Cerujos sūnus Joabas, jį paėmė ir tapo kariuomenės vadu. 
\par 7 Dovydas apsigyveno tvirtovėje, todėl ją pavadino Dovydo miestu. 
\par 8 Jis statė miestą aplinkui, pradėdamas nuo Milojo. Joabas atstatė likusią miesto dalį. 
\par 9 Dovydas vis daugiau įsigalėjo, nes kareivijų Viešpats buvo su juo. 
\par 10 Šie yra žymiausi karžygiai, kurie drauge su visu Izraeliu rėmė Dovydą užimant karaliaus sostą, kaip Viešpats buvo kalbėjęs apie Izraelį. 
\par 11 Hachmonis Jašobamas­vyriausiasis iš trijų; jis pakėlė savo ietį prieš tris šimtus ir nukovė juos visus vienu kartu. 
\par 12 Antras pasižymėjęs buvo Dodojo sūnus Eleazaras, ahoachas. 
\par 13 Jis buvo su Dovydu prie Pas Damimo, kur filistinai susirinko prieš juos mūšiui miežių laukuose. Žmonėms pradėjus bėgti nuo filistinų, 
\par 14 jie atsistojo lauko viduryje, kovojo ir nugalėjo filistinus. Taip Viešpats suteikė jiems didelį išgelbėjimą. 
\par 15 Trys vyrai iš trisdešimties vyresniųjų atėjo pas Dovydą į Adulamo olą; tuo metu filistinų kariai buvo pasistatę stovyklą Refajų slėnyje. 
\par 16 Dovydas tuo laiku buvo tvirtovėje, o filistinų būrys­Betliejuje. 
\par 17 Tuomet Dovydas, ilgesio kankinamas, tarė: “Kas man atneš vandens iš Betliejaus šulinio, esančio prie vartų?” 
\par 18 Tie trys prasilaužė pro filistinų stovyklą, pasėmė vandens iš Betliejaus šulinio, esančio prie vartų, ir atnešė Dovydui. Tačiau Dovydas negėrė jo, bet išliejo jį Viešpačiui 
\par 19 ir tarė: “Taip nebus, kad gerčiau šitų vyrų kraują! Juk jie, statydami savo gyvybę pavojun, man jo atnešė”. Tai padarė tie trys karžygiai. 
\par 20 Joabo brolis Abišajias buvo žymiausias iš trijų. Jis pakėlė savo ietį prieš tris šimtus ir, juos nugalėjęs, pagarsėjo tarp trijų. 
\par 21 Iš trijų jis buvo garsiausias ir tapo jų vadu, tačiau aniems trims neprilygo. 
\par 22 Jehojados sūnus Benaja iš Kabcelio buvo narsus vyras. Jis padarė daug žygdarbių: nukovė du žymius Moabo karžygius, sningant duobėje užmušė liūtą. 
\par 23 Be to, nukovė egiptietį, vyrą penkių uolekčių aukščio. Egiptiečio rankoje ietis buvo kaip audėjo staklių riestuvas. Nuėjęs prie jo su lazda, jis atėmė ietį iš egiptiečio rankos ir jį nukovė jo paties ietimi. 
\par 24 Tuo Jehojados sūnus Benaja pagarsėjo tarp trijų karžygių. 
\par 25 Ir jis pagarsėjo tarp tų trisdešimties, tačiau pirmiems trims neprilygo. Dovydas jį paskyrė savo sargybos viršininku. 
\par 26 Kariuomenės karžygiai buvo Joabo brolis Asaelis, Dodojo sūnus Elhananas iš Betliejaus, 
\par 27 harodietis Šamotas, pelojietis Helecas, 
\par 28 tekojiečio Ikešo sūnus Ira, anatotietis Abiezeras, 
\par 29 hušietis Sibechajas, ahohitas Ilajas, 
\par 30 netofietis Mahrajas, netofietis Baanos sūnus Heledas, 
\par 31 Ribajo sūnus Itajas iš Benjamino Gibėjos, piratonietis Benaja, 
\par 32 Hurajas iš Gaašo klonių, arabietis Abielis, 
\par 33 baharumietis Azmavetas, šaalbonietis Eljachba, 
\par 34 gizojietis Hašemas, hararas Šagės sūnus Jehonatanas, 
\par 35 hararas Sacharo sūnus Ahiamas, Ūro sūnus Elifalas, 
\par 36 mecherietis Heferas, pelojietis Ahija, 
\par 37 karmelietis Hezrojas, Ezbajo sūnus Naarajas, 
\par 38 Natano brolis Joelis, Hagrio sūnus Mibharas, 
\par 39 amonitas Celekas, beerotietis Nachrajas, Cerujos sūnaus Joabo ginklanešys, 
\par 40 itritai Garebas ir Ira, 
\par 41 hetitas Ūrija, Achlajo sūnus Zabadas, 
\par 42 rubenas Šizos sūnus Adina, rubenų vadas ir trisdešimties viršininkas, 
\par 43 Maakos sūnus Hananas ir mitnietis Juozapatas, 
\par 44 aštarotietis Uzija, aroeriečio Hotamo sūnūs Šama ir Jejelis, 
\par 45 Šimrio sūnus Jediaelis ir jo brolis ticietis Joha, 
\par 46 mahavietis Elielis, Elnaamo sūnūs Jeribajas ir Jošavija, moabitas Itma, 
\par 47 Elielis, Jobedas ir mezobaitas Jaasielis.



\chapter{12}


\par 1 Šitie atėjo pas Dovydą į Ciklagą, kai jis dar turėjo slėptis nuo Kišo sūnaus Sauliaus; jie priklausė prie karžygių ir padėjo jam kovose. 
\par 2 Jie buvo ginkluoti lankais ir sugebėjo tiek dešiniąja, tiek kairiąja ranka svaidyti akmenis bei šaudyti strėlėmis; jie buvo Sauliaus giminaičiai iš Benjamino giminės. 
\par 3 Ahiezeras buvo jų vadas, po to Jehoašas, abu Šemavos iš Gibėjos sūnūs; Azmaveto sūnūs­Jezielis ir Peletas, Beracha ir anatotietis Jehuvas; 
\par 4 gibeonietis Išmaja­karžygys, trisdešimties viršininkas; gederiečiai Jeremija, Jahazielis, Johananas ir Jehozabadas; 
\par 5 harifai Eluzajas, Jerimotas, Bealija, Šemarijas ir Šefatijas; 
\par 6 koritai Elkana, Išijas, Azarelis, Joezeras ir Jašobamas; 
\par 7 Jerohamo sūnūs­Joela ir Zebadija iš Gedoro. 
\par 8 Dovydui besislapstant dykumoje, jo pusėn perėjo Gado giminės narsūs vyrai, patyrę kariai, tinkami karo žygiui, sugebą vartoti skydą ir ietį. Savo narsumu jie prilygo liūtui, o eiklumu­kalnų gazelei. 
\par 9 Vyriausiasis iš jų buvo Ezeras, antras­Abdija, trečias­Eliabas, 
\par 10 ketvirtas­Mišmana, penktas­ Jeremija, 
\par 11 šeštas­Atajas, septintas­Elielis, 
\par 12 aštuntas­Johananas, devintas­Elzabadas, 
\par 13 dešimtas­Jeremijas, vienuoliktas­Machbanajas. 
\par 14 Šitie buvo gadų kariuomenės vadai, vadovavę nuo šimto iki tūkstančio kareivių. 
\par 15 Jie persikėlė per Jordaną, kai upė buvo išsiliejusi iš savo krantų pirmą metų mėnesį, ir privertė trauktis visus iš slėnių į rytus ir į vakarus. 
\par 16 Kai kurie iš Benjamino ir Judo karių atėjo pas Dovydą į tvirtovę. 
\par 17 Dovydas išėjo jų pasitikti ir kalbėjo jiems: “Jei atėjote pas mane taikingai, norėdami man padėti, aš nuoširdžiai priimsiu jus, o jei atėjote mane išduoti priešams, nors aš jums nieko blogo nepadariau, tegul mūsų tėvų Dievas mato ir teisia”. 
\par 18 Tuomet dvasia nužengė ant Amasajo, jų vyriausiojo, ir jis tarė: “Dovydai, Jesės sūnau, mes tavo ir su tavimi! Ramybė tau ir ramybė tavo pagalbininkams, nes tau padeda tavo Dievas!” Tuomet Dovydas juos priėmė ir paskyrė savo kariuomenės būrių viršininkais. 
\par 19 Ir iš Manaso giminės kai kurie perėjo į Dovydo pusę, kai jis atžygiavo su filistinais prieš Saulių. Tačiau jis nekovojo drauge su jais, nes filistinų kunigaikščiai pasitarė ir pasiuntė jį atgal, sakydami: “Ant mūsų galvų jis pereis į savo valdovo Sauliaus pusę”. 
\par 20 Dovydui žygiuojant į Ciklagą, iš Manaso pusės perėjo Adnachas, Jehozabadas, Jediaelis, Mykolas, Jehozabadas, Elihuvas ir Ciletajas, Manaso tūkstantininkai. 
\par 21 Jie padėjo Dovydui prieš užpuolikus, nes jie visi buvo narsūs vyrai ir kariuomenės vadai. 
\par 22 Kas dieną žmonės ateidavo pas Dovydą padėti jam; susidarė didelė kariuomenė, lyg Dievo kariuomenė. 
\par 23 Dovydui esant Hebrone, didelis skaičius ginkluotų karių atvyko jam atiduoti Sauliaus karalystę pagal Viešpaties pažadą. 
\par 24 Judo ginkluotų skydais ir ietimis vyrų buvo šeši tūkstančiai aštuoni šimtai; 
\par 25 Simeono­septyni tūkstančiai šimtas, 
\par 26 Levio­keturi tūkstančiai šeši šimtai. 
\par 27 Be to, aaronitų vado Jehojados­ trys tūkstančiai septyni šimtai, 
\par 28 Cadoko, narsaus ir pasižymėjusio jaunuolio, ir jo tėvo namų­ dvidešimt du vadai; 
\par 29 Benjamino, Sauliaus giminaičių,­trys tūkstančiai; iki to laiko dauguma iš jų buvo ištikimi Sauliui. 
\par 30 Efraimo­dvidešimt tūkstančių aštuoni šimtai narsių, pagarsėjusių vyrų. 
\par 31 Iš pusės Manaso giminės­aštuoniolika tūkstančių, pašauktų vardais, atvyko paskelbti Dovydą karaliumi. 
\par 32 Isacharo vyrų, kurie suprato laikus ir žinojo, ką Izraelis turi daryti, atvyko du šimtai viršininkų su visais savo kariais; 
\par 33 Zabulono patyrusių, patikimų ir ginkluotų karių­penkiasdešimt tūkstančių, pasiryžusių padėti Dovydui. 
\par 34 Naftalio­tūkstantis vadų ir trisdešimt septyni tūkstančiai ginkluotų skydais ir ietimis vyrų; 
\par 35 Dano patyrusių karių­dvidešimt aštuoni tūkstančiai šeši šimtai; 
\par 36 Ašero tinkamų karo žygiui karių­ keturiasdešimt tūkstančių; 
\par 37 Iš kitos Jordano pusės rubenų, gadų ir pusės Manaso giminės tinkamai ginkluotų karių­šimtas dvidešimt tūkstančių. 
\par 38 Visi šitie kariai, galintys eiti rikiuotėje, atėjo į Hebroną, pasiryžę paskelbti Dovydą viso Izraelio karaliumi. Visi kiti izraelitai taip pat buvo vieningai nusiteikę paskelbti Dovydą karaliumi. 
\par 39 Pas Dovydą jie buvo tris dienas, valgydami ir gerdami, nes jų broliai buvo aprūpinę juos. 
\par 40 Net Isacharo, Zabulono ir Naftalio kaimynai asilais, kupranugariais, mulais ir jaučiais gabeno jiems maisto: figų papločių, džiovintų vynuogių, vyno, aliejaus ir daugybę avių; džiaugsmas buvo visame Izraelyje.



\chapter{13}


\par 1 Dovydas, pasitaręs su tūkstantininkais, šimtininkais ir visais vadais, 
\par 2 visiems susirinkusiems tarė: “Jei jūs sutinkate ir jei tai Viešpaties, mūsų Dievo, valia, siųskime pasiuntinius pas savo brolius izraelitus ir tarp jų gyvenančius kunigus bei levitus į miestus ir į jų ganyklas, kad susirinktų pas mus, 
\par 3 ir parsigabenkime Dievo skrynią, nes Sauliaus dienomis mes nesikreipėme į Jį”. 
\par 4 Visi susirinkusieji jam pritarė, nes visai tautai patiko tas pasiūlymas. 
\par 5 Dovydas sušaukė visą Izraelį nuo Egipto Sichoro iki Hemato pargabenti Dievo skrynią iš Kirjat Jearimo. 
\par 6 Dovydas ir visas Izraelis nuvyko į Baalą­Kirjat Jearimą, kuris priklausė Judui, atgabenti Dievo skrynią, kuri vadinama Viešpaties, gyvenančio tarp cherubų, vardu. 
\par 7 Paėmę Dievo skrynią iš Abinadabo namų, vežė ją naujame vežime, Uza ir Achjojas varė vežimą. 
\par 8 Dovydas bei visi izraelitai džiūgavo ir garbino Dievą giesmėmis, psalterių, cimbolų, arfų, būgnų ir trimitų garsais. 
\par 9 Jiems pasiekus Kidono klojimą, Uza ištiesė ranką, norėdamas prilaikyti skrynią, nes jaučiai suklupo. 
\par 10 Tuomet Viešpaties rūstybė užsidegė prieš Uzą ir Dievas ištiko jį, nes jis palietė skrynią. Jis mirė ten pat Dievo akivaizdoje. 
\par 11 Dovydas liūdėjo, kad Viešpats ištiko Uzą; ta vieta iki šios dienos vadinama Perec-Uza. 
\par 12 Dovydas išsigando tą dieną Dievo, sakydamas: “Kaip aš galiu pargabenti Dievo skrynią pas save?” 
\par 13 Dovydas nevežė skrynios į Dovydo miestą; ji buvo nuvežta į gatiečio Obed Edomo namus. 
\par 14 Dievo skrynia pasiliko Obed Edomo namuose tris mėnesius. Viešpats laimino Obed Edomo namus ir visa, ką jis turėjo.



\chapter{14}


\par 1 Tyro karalius Hiramas siuntė pasiuntinių pas Dovydą su kedro medžiais, dailidžių bei mūrininkų, kad jie pastatytų Dovydui namus. 
\par 2 Dovydas suprato, kad Viešpats patvirtino jį Izraelio karaliumi, nes išaukštino jo karalystę dėl savo tautos Izraelio. 
\par 3 Jeruzalėje Dovydas vedė daugiau žmonų ir susilaukė daugiau sūnų bei dukterų. 
\par 4 Šitie vardai jo vaikų, kurie gimė Jeruzalėje: Šamūva, Šobabas, Natanas, Saliamonas, 
\par 5 Ibharas, Elišūva, Elpaletas, 
\par 6 Nogahas, Nefegas, Jafija, 
\par 7 Elišama, Beeljada ir Elifeletas. 
\par 8 Filistinai, išgirdę, kad Dovydas pateptas Izraelio karaliumi, pakilo Dovydo ieškoti. Dovydas, tai išgirdęs, išėjo prieš juos. 
\par 9 Filistinai atėję sustojo Rafajų slėnyje. 
\par 10 Dovydas klausė Dievo, sakydamas: “Ar man eiti prieš filistinus? Ar atiduosi juos į mano rankas?” Viešpats atsakė: “Eik, nes Aš atiduosiu juos į tavo rankas”. 
\par 11 Jis, nuėjęs į Baal Peracimus, juos nugalėjo. Dovydas tarė: “Dievas nušlavė mano priešus mano rankomis, kaip vanduo pralaužęs pylimą”. Todėl ta vieta pavadinta Baal Peracimu. 
\par 12 Filistinai ten paliko savo stabus, kuriuos Dovydas įsakė sudeginti. 
\par 13 Filistinai dar kartą atėjo ir sustojo slėnyje. 
\par 14 Dovydas vėl klausė Dievo. Dievas jam atsakė: “Neik tiesiai prieš juos! Apeik juos ir pulk iš šilkmedžių pusės. 
\par 15 Išgirdęs šlamesį šilkmedžių viršūnėse, užpulk juos. Dievas eis pirma tavęs ir naikins filistinų kariuomenę”. 
\par 16 Dovydas padarė, kaip Dievas jam įsakė. Jis mušė filistinus nuo Gibeono iki Gazero. 
\par 17 Dovydo vardas išgarsėjo visose šalyse; Viešpats sukėlė baimę prieš jį visose tautose.



\chapter{15}

\par 1 Dovydas pasistatė namus Dovydo mieste. Paruošęs vietą ir ištiesęs palapinę Dievo skryniai, 
\par 2 Dovydas sakė: “Niekam nevalia nešti Dievo skrynios, tik levitams, nes juos Viešpats išsirinko Dievo skryniai nešioti ir Jam amžinai tarnauti”. 
\par 3 Dovydas sušaukė visą Izraelį į Jeruzalę, kad atgabentų Viešpaties skrynią į jai paruoštą vietą. 
\par 4 Dovydas surinko Aarono palikuonis ir levitus: 
\par 5 Kehato palikuonių buvo vyriausiasis Ūrielis ir jo broliai­šimtas dvidešimt; 
\par 6 Merario palikuonių­vyriausiasis Asaja ir jo broliai­du šimtai dvidešimt; 
\par 7 Geršono palikuonių­vyriausiasis Joelis ir jo broliai­šimtas trisdešimt; 
\par 8 Elicafano palikuonių­vyriausiasis Šemaja ir jo broliai­du šimtai; 
\par 9 Hebrono palikuonių­vyriausiasis Elielis ir jo broliai­aštuoniasdešimt; 
\par 10 Uzielio palikuonių­vyriausiasis Aminadabas ir jo broliai­šimtas dvylika. 
\par 11 Po to Dovydas pasikvietė kunigus Cadoką ir Abjatarą bei levitus Ūrielį, Asają, Joelį, Šemają, Elielį ir Aminadabą 
\par 12 ir jiems tarė: “Jūs esate levitų šeimų vadai; pasišventinkite jūs ir jūsų broliai, kad galėtumėte atgabenti Viešpaties, Izraelio Dievo, skrynią į vietą, kurią jai paruošiau. 
\par 13 Kadangi pirmą kartą jūsų nebuvo, Viešpats, mūsų Dievas, ištiko mus, nes mes ieškojome Jo ne taip, kaip turėjome”. 
\par 14 Kunigai ir levitai pasišventino, kad galėtų atgabenti Viešpaties, Izraelio Dievo, skrynią. 
\par 15 Levitai nešė Dievo skrynią ant savo pečių, kaip Mozė buvo įsakęs pagal Viešpaties jam duotą žodį. 
\par 16 Dovydas įsakė levitų vyresniesiems paskirti brolius giesmininkus su instrumentais: arfomis, psalteriais, cimbolais, kad skambindami keltų džiaugsmingą triukšmą. 
\par 17 Levitai paskyrė Joelio sūnų Hemaną, Berechijo sūnų Asafą, iš merarių­Kušajo sūnų Etaną, 
\par 18 su jais kitos eilės tarnautojus: Zachariją, Jaazielį, Šemiramotą, Jehielį, Unį, Eliabą, Benają, Maasėją, Matitiją, Elifelehuvą, Miknėją bei vartininkus Obed Edomą ir Jejelį. 
\par 19 Giedotojai Hemanas, Asafas ir Etanas skambino variniais cimbolais, 
\par 20 Zacharija, Azielis, Šemiramotas, Jehielis, Unis, Eliabas, Maasėjas ir Benajas skambino psalteriais. 
\par 21 Matitijas, Elifelehuvas, Miknėjas, Obed Edomas, Jejelis ir Azazijas skambino arfomis seminitų gaida. 
\par 22 Kenanijas buvo levitų giedojimo mokytojas, nes buvo įgudęs giedoti. 
\par 23 Berechija ir Elkana buvo durininkai prie skrynios. 
\par 24 Kunigai Šebanijas, Juozapatas, Netanelis, Amasajas, Zacharijas, Benajas ir Eliezeras trimitavo priešais Dievo skrynią. Obed Edomas ir Jehija buvo durininkai prie skrynios. 
\par 25 Dovydas, Izraelio vyresnieji ir tūkstantininkai su džiaugsmu ėjo pargabenti Viešpaties Sandoros skrynios iš Obed Edomo namų. 
\par 26 Kadangi Dievas padėjo levitams, nešusiems Viešpaties Sandoros skrynią, tai jie aukojo septynis jaučius ir septynis avinus. 
\par 27 Dovydas vilkėjo plonos drobės drabužiais kaip visi levitai, kurie nešė skrynią, giesmininkai ir Kenanija, kuris vadovavo giedojimui; Dovydas vilkėjo dar ir lininį efodą. 
\par 28 Taip izraelitai gabeno Viešpaties Sandoros skrynią džiūgaudami, rago, trimitų, cimbolų, arfų ir psalterių garsams palydint. 
\par 29 Kai Viešpaties Sandoros skrynia pasiekė Dovydo miestą, Sauliaus duktė Mikalė, žiūrėdama pro langą, pamatė grojantį bei šokantį karalių Dovydą ir paniekino jį savo širdyje.



\chapter{16}


\par 1 Taip jie atnešė Dievo skrynią ir, padėję ją į palapinę, kurią Dovydas jai paruošė, aukojo deginamąsias bei padėkos aukas Dievo akivaizdoje. 
\par 2 Dovydas, baigęs aukoti deginamąsias ir padėkos aukas, palaimino tautą Viešpaties vardu 
\par 3 ir išdalino visiems izraelitams, vyrams bei moterims, kiekvienam po duonos kepalą, mėsos gabalą ir vynuogių pyragaitį. 
\par 4 Dalį levitų Dovydas paskyrė tarnauti prieš Viešpaties skrynią, kad garbintų, dėkotų ir šlovintų Viešpatį, Izraelio Dievą. 
\par 5 Asafą paskyrė vyriausiuoju, po jo­Zachariją, Jejelį, Šemiramotą, Jehielį, Matitiją, Eliabą, Benają, Obed Edomą; Jejelį paskyrė groti arfomis ir psalteriais, o Asafas skambino cimbolais. 
\par 6 Kunigą Benają ir Jahazielį paskyrė nuolat trimituoti prie Dievo Sandoros skrynios. 
\par 7 Tą dieną Dovydas pirmą kartą pamokė Asafą su broliais dėkoti Viešpačiui šia giesme: 
\par 8 “Dėkokite Viešpačiui, šaukitės Jo vardo. Skelbkite tautose Jo darbus. 
\par 9 Giedokite Jam, skambinkite Jam. Garsinkite visus Jo stebuklus. 
\par 10 Didžiuokitės Jo šventu vardu. Tegul džiaugiasi širdis tų, kurie ieško Viešpaties. 
\par 11 Ieškokite Viešpaties ir Jo jėgos. Ieškokite nuolat Jo veido. 
\par 12 Atsiminkite Jo nuostabius darbus, kuriuos Jis yra padaręs, Jo stebuklus ir Jo lūpų tartus sprendimus. 
\par 13 Jūs, Jo tarno Izraelio palikuonys, Jokūbo vaikai, Jo išrinktieji. 
\par 14 Jis yra Viešpats, mūsų Dievas, visoje žemėje galioja Jo sprendimai. 
\par 15 Atsiminkite per amžius Jo sandorą, žodį, kurį Jis įsakė tūkstančiui kartų, 
\par 16 sandorą, kurią Jis padarė su Abraomu, ir priesaiką, duotą Izaokui. 
\par 17 Jis patvirtino ją Jokūbui įstatymu ir Izraeliui amžina sandora, 
\par 18 sakydamas: ‘Aš tau duosiu Kanaano šalį, tavo paveldėjimo dalį’. 
\par 19 Jie buvo negausūs skaičiumi, tik ateiviai joje. 
\par 20 Jie keliavo iš tautos į tautą, iš vienos karalystės į kitą. 
\par 21 Jis niekam neleido jų skriausti, sudrausdavo karalius dėl jų: 
\par 22 ‘Nelieskite mano pateptųjų ir mano pranašams nedarykite pikto’. 
\par 23 Visos šalys, giedokite Viešpačiui, kiekvieną dieną skelbkite Jo išgelbėjimą, 
\par 24 apsakykite pagonims Jo garbę ir Jo stebuklus visoms tautoms. 
\par 25 Didis yra Viešpats ir didžiai girtinas, bijotinas labiausiai iš visų dievų. 
\par 26 Visi tautų dievai yra stabai, bet Viešpats sukūrė dangų. 
\par 27 Šlovė ir garbė Jo akivaizdoje, galia ir džiaugsmas su Juo. 
\par 28 Pripažinkite Viešpačiui, tautų giminės, pripažinkite Viešpačiui garbę ir galybę! 
\par 29 Atiduokite Viešpačiui šlovę, priderančią Jo vardui, atneškite auką ir ateikite pas Jį. Garbinkite Viešpatį šventumo grožyje. 
\par 30 Visa žemė tesudreba prieš Jį! Tvirtai stovi pasaulis. 
\par 31 Tesilinksmina dangūs ir tedžiūgauja žemė. Tegul skamba tautose: ‘Viešpats karaliauja!’ 
\par 32 Tegul jūra šniokščia ir visa, kas joje! Tegul linksminasi laukai ir visa, kas juose! 
\par 33 Tada miško medžiai giedos Viešpaties akivaizdoje, nes Jis ateina žemės teisti. 
\par 34 Dėkokite Viešpačiui, nes Jis geras ir Jo gailestingumas amžinas. 
\par 35 Sakykite: ‘Išvaduok mus, Dieve, mūsų gelbėtojau! Surink mus ir išlaisvink iš pagonių, kad dėkotume Tavo šventam vardui ir girtumėmės Tavo šlove’. 
\par 36 Garbė Viešpačiui, Izraelio Dievui, per amžių amžius”. Visa tauta tarė: “Amen”, ir šlovino Viešpatį. 
\par 37 Taigi jis paliko Asafą ir jo brolius nuolat tarnauti priešais Viešpaties Sandoros skrynią, atliekant kasdienę tarnystę, 
\par 38 taip pat Obed Edomą bei jo brolius, šešiasdešimt aštuonis; Obed Edomas, Jedutūno sūnus ir Hosa buvo vartininkai. 
\par 39 Kunigą Cadoką ir jo brolius kunigus paskyrė prie Viešpaties palapinės Gibeono aukštumoje 
\par 40 nuolat, rytą ir vakare, aukoti Viešpačiui deginamąsias aukas ant deginamųjų aukų aukuro ir daryti visa, kas parašyta Viešpaties įstatyme, kurį Jis davė Izraeliui. 
\par 41 Hemaną, Jedutūną ir kitus, pašauktus vardais, paskyrė dėkoti Viešpačiui, nes Jo gailestingumas amžinas. 
\par 42 Hemanas ir Jedutūnas turėjo trimitus, cimbolus ir kitus instrumentus giesmėms pritarti. Jedutūno sūnūs buvo paskirti vartininkais. 
\par 43 Po to visi išsiskirstė į savo namus; Dovydas sugrįžo palaiminti savo namiškių.



\chapter{17}


\par 1 Gyvendamas savo namuose, Dovydas tarė pranašui Natanui: “Aš gyvenu kedro namuose, o Viešpaties Sandoros skrynia­palapinėje”. 
\par 2 Natanas atsakė Dovydui: “Daryk visa, kas yra tavo širdyje, nes Dievas yra su tavimi”. 
\par 3 Tą pačią naktį Dievo žodis atėjo Natanui: 
\par 4 “Eik ir kalbėk mano tarnui Dovydui: ‘Taip sako Viešpats: ‘Tu nepastatysi man namų, kuriuose gyvenčiau. 
\par 5 Aš negyvenau namuose nuo tos dienos, kai išvedžiau Izraelį iš Egipto, iki šios dienos, bet keliavau iš palapinės į palapinę, iš pastogės į pastogę. 
\par 6 Ar Aš, keliaudamas su Izraeliu, esu sakęs kuriam Izraelio teisėjui, kam pavesdavau ganyti mano tautą: ‘Kodėl man nepastatote kedro namų?’ 
\par 7 Sakyk mano tarnui Dovydui: ‘Aš tave paėmiau iš ganyklos, nuo avių, kad būtum vadas mano tautai, Izraeliui. 
\par 8 Aš buvau su tavimi visur, kur tu ėjai; išnaikinau visus tavo priešus priešais tave; tavo vardą padariau garsų kaip žemės didžiūnų vardą. 
\par 9 Aš paskirsiu vietą savo tautai Izraeliui ir jį įsodinsiu, kad jis gyventų savo vietoje ir nebeklajotų ir nedorybės vaikai nespaustų jų, kaip pradžioje, 
\par 10 ir nuo to laiko, kai įsakiau teisėjams valdyti Izraelį. Aš pažeminsiu visus tavo priešus; be to, Aš sakau tau, kad Viešpats pastatys tau namus. 
\par 11 Tavo dienoms pasibaigus, kai tu išeisi pas savo tėvus, Aš pakelsiu vieną tavo palikuonį po tavęs iš tavo sūnų ir įtvirtinsiu jo karalystę. 
\par 12 Jis pastatys man namus, o Aš įtvirtinsiu jo sostą amžiams. 
\par 13 Aš būsiu jam tėvas, o jis bus man sūnus; mano gailestingumas nepaliks jo, kaip paliko tą, kuris buvo prieš tave. 
\par 14 Aš įstatysiu jį savo namuose ir savo karalystėje per amžius, jo sostas bus amžinas’ ”. 
\par 15 Visus šiuos žodžius ir regėjimą Natanas persakė Dovydui. 
\par 16 Karalius Dovydas įėjo, atsisėdo Viešpaties akivaizdoje ir tarė: “Kas aš ir mano namai, Viešpatie Dieve, kad mane iki čia atvedei. 
\par 17 Ir tai pasirodė dar per maža Tavo akyse, Dieve. Tu kalbėjai apie savo tarno namus tolimoje ateityje ir pasielgei su manimi kaip su žymiu žmogumi, Viešpatie Dieve! 
\par 18 Ką gi daugiau Dovydas gali, Tau taip pagerbus Tavo tarną? Nes Tu žinai savo tarną. 
\par 19 Dėl savo tarno ir pagal savo širdį Tu padarei šitą didybę, pranešdamas savo tarnui tuos didžius dalykus. 
\par 20 Viešpatie, nėra nė vieno Tau lygaus ir nėra kito Dievo šalia Tavęs, kaip mes girdėjome savo ausimis. 
\par 21 Ar yra kita tokia tauta ant žemės, kuri prilygtų Tavo tautai, Izraeliui, pas kurią Dievas būtų atėjęs išpirkti jos sau ir išgarsinti savo vardo, didingo ir baisaus, išvarydamas kitas tautas priešais savo tautą, kurią Tu išpirkai iš Egipto vergijos. 
\par 22 Tu, Viešpatie, padarei Izraelį savo tauta visiems laikams ir tapai jos Dievu. 
\par 23 Dabar, Viešpatie, tegul tai, ką Tu kalbėjai apie savo tarną ir jo namus, būna įtvirtinta amžiams ir padaryk, kaip pasakei. 
\par 24 Įtvirtink tai, kad Tavo vardas būtų aukštinamas per amžius, sakant: ‘Kareivijų Viešpats yra Izraelio Dievas’. Tegul Tavo tarno Dovydo namai būna įtvirtinti Tavo akivaizdoje. 
\par 25 Tu, Dieve, pasakei savo tarnui, kad pastatysi jam namus, todėl Tavo tarnas išdrįso savo širdyje melstis Tavo akivaizdoje. 
\par 26 Viešpatie, Tu esi Dievas ir pažadėjai savo tarnui šitą gerovę. 
\par 27 Dabar teikis laiminti savo tarno namus, kad jie per amžius būtų Tavo akivaizdoje. Nes jeigu Tu, Viešpatie, palaimini, tai bus palaiminta amžinai”.



\chapter{18}


\par 1 Dovydas, nugalėjęs filistinus, atėmė iš jų Gatą ir nuo jo priklausomus miestelius. 
\par 2 Nugalėjęs Moabą, jis padarė moabitus savo pavaldiniais ir privertė mokėti jam duoklę. 
\par 3 Po to Dovydas sumušė Hemato krašte Cobos karalių Hadadezerą, kai jis kariavo, norėdamas išplėsti savo valdžią iki Eufrato upės. 
\par 4 Dovydas atėmė iš jo tūkstantį kovos vežimų, paėmė nelaisvėn septynis šimtus raitelių ir dvidešimt tūkstančių pėstininkų. Jis pakirto žirgams kojas ir pasilaikė sau žirgų tik dėl šimto kovos vežimų. 
\par 5 Kai Damasko sirai atėjo į pagalbą Cobos karaliui Hadadezerui, Dovydas nukovė dvidešimt du tūkstančius sirų. 
\par 6 Dovydas paskyrė įgulas Damaske. Sirai tapo Dovydo tarnais ir mokėjo jam duoklę. Viešpats saugojo Dovydą visur, kur jis ėjo. 
\par 7 Dovydas paėmė Hadadezero tarnų auksinius skydus ir juos parsigabeno į Jeruzalę. 
\par 8 Iš Tibhato ir Kūno, Hadadezero miestų, Dovydas parsigabeno labai daug vario. Iš jo Saliamonas padirbdino baseiną, kolonas ir varinius indus. 
\par 9 Hemato karalius Tojas, išgirdęs, kad Dovydas sumušė Sobos karaliaus Hadadezerio kariuomenę, 
\par 10 siuntė savo sūnų Adoramą pasveikinti karalių Dovydą, laimėjusį karą prieš Hadadezerą. Tojas dažnai kariaudavo su Hadadezeru. Jis atsiuntė auksinių, sidabrinių ir varinių indų. 
\par 11 Tuos daiktus karalius Dovydas paskyrė Viešpačiui kartu su sidabru ir auksu iš edomitų, moabitų, amonitų, filistinų bei amalekiečių. 
\par 12 Cerujos sūnus Abišajas sumušė Druskos slėnyje aštuoniolika tūkstančių edomitų. 
\par 13 Jis paskyrė įgulas Edome. Edomitai tapo Dovydo tarnais. Viešpats saugojo Dovydą visur, kur jis ėjo. 
\par 14 Dovydas karaliavo visame Izraelyje ir vykdė teisingumą bei teismą visai tautai. 
\par 15 Cerujos sūnus Joabas buvo kariuomenės vadas, Ahiludo sūnus Juozapatas­metraštininkas, 
\par 16 Ahitubo sūnus Cadokas ir Abjataro sūnus Abimelechas buvo kunigai, Šavša­raštininkas, 
\par 17 Jehojados sūnus Benaja buvo keretų ir peletų viršininkas, o Dovydo sūnūs buvo aukšti pareigūnai prie karaliaus.



\chapter{19}

\par 1 Amonitų karalius Nahašas mirė, ir jo sūnus pradėjo karaliauti jo vietoje. 
\par 2 Dovydas sakė: “Aš būsiu geras Nahašo sūnui Hanūnui, kaip jo tėvas buvo man”. Ir Dovydas siuntė pasiuntinius paguosti jo dėl tėvo. Dovydo tarnai atėjo į Amono žemę pas Hanūną jo paguosti. 
\par 3 Bet Amono kunigaikščiai tarė Hanūnui: “Ar manai, kad Dovydas, pagerbdamas tavo tėvą, atsiuntė pas tave guodėjus? Ar ne apžiūrėti, išžvalgyti ir sunaikinti kraštą?” 
\par 4 Hanūnas paėmė Dovydo tarnus, nuskuto jiems barzdas, nukirpo jų drabužius iki pusės, iki pat juostos, ir išsiuntė. 
\par 5 Dovydas, sužinojęs, kas įvyko, pasiuntė vyrus jų pasitikti, nes jie buvo labai sugėdinti. Karalius sakė: “Pasilikite Jeriche, kol ataugs jūsų barzdos, o tada sugrįžkite”. 
\par 6 Kai amonitai suprato, kad tapo nepakenčiami Dovydui, Hanūnas siuntė tūkstantį talentų sidabro pasamdyti iš Mesopotamijos, Sirijos ir Cobos kovos vežimų ir raitelių. 
\par 7 Jie pasamdė trisdešimt du tūkstančius kovos vežimų ir Maakos karalių su jo kariuomene. Jie atvykę pasistatė stovyklą ties Medeba, o amonitai susirinko iš savo miestų ir išėjo į karą. 
\par 8 Dovydas, tai išgirdęs, pasiuntė Joabą su visa stiprių vyrų kariuomene. 
\par 9 Amonitai išsirikiavo kovai prie miesto vartų, o karaliai, kurie atėjo padėti, stovėjo atvirame lauke. 
\par 10 Joabas pamatė, kad prieš jį ruošiamas puolimas iš priekio ir iš užnugario; jis išrinko Izraelio geriausius karius ir išrikiavo prieš sirus. 
\par 11 Likusius žmones jis pavedė savo broliui Abišajui, kuris išrikiavo juos prieš amonitus. 
\par 12 Joabas sakė broliui: “Jei sirai bus per stiprūs man, tu ateisi man į pagalbą, o jei amonitai bus per stiprūs tau, tai aš tau padėsiu. 
\par 13 Būk drąsus, narsiai kovokime už savo tautą ir už savo Dievo miestus. O Viešpats tedaro, kaip jam atrodo tinkama”. 
\par 14 Joabas ir su juo buvę žmonės pradėjo kovą prieš sirus, ir tie pabėgo nuo jo. 
\par 15 Amonitai, pamatę, kad sirai pabėgo, irgi bėgo nuo jo brolio Abišajo ir užsidarė mieste. Tuomet Joabas sugrįžo į Jeruzalę. 
\par 16 Sirai, pamatę, kad Izraelis juos nugalėjo, siuntė pasiuntinius ir pasikvietė sirus, gyvenusius anapus upės. Hadadezero kariuomenės vadas Šofachas vadovavo kariuomenei. 
\par 17 Kai tai buvo pranešta Dovydui, jis surinko visus izraelitus, persikėlė per Jordaną ir išsirikiavo mūšiui. Kai Dovydas išsirikiavo prieš sirus, jie kovojo su juo. 
\par 18 Bet sirai bėgo nuo Izraelio; Dovydas sunaikino septynis tūkstančius kovos vežimų ir keturiasdešimt tūkstančių pėstininkų; žuvo ir kariuomenės vadas Šofachas. 
\par 19 Hadadezero tarnai, pamatę, kad jie Izraelio nugalėti, padarė taiką su Dovydu ir tapo jo tarnais. Nuo to laiko sirai nebepadėdavo amonitams.



\chapter{20}


\par 1 Praėjus metams, tuo laiku, kai karaliai eina į karą, Joabas, surinkęs kariuomenę, nusiaubė amonitų šalį ir apgulė Rabą; Dovydas buvo pasilikęs Jeruzalėje. Joabas nugalėjo Rabą ir ją sugriovė. 
\par 2 Dovydas nuėmė jų karaliui karūną su brangiais akmenimis, sveriančią talentą aukso, ir užsidėjo ją ant galvos. Be to, jis išgabeno iš miesto labai daug grobio. 
\par 3 Miesto gyventojus jis pristatė prie pjūklų, geležinių akėčių ir kirvių. Taip Dovydas pasielgė su visais užimtais amonitų miestais. Vėliau Dovydas su visa kariuomene sugrįžo į Jeruzalę. 
\par 4 Po to prie Gezero kilo karas su filistinais. Tuomet hušietis Sibechajas nukovė Sipają, milžinų palikuonį, ir jie buvo pažeminti. 
\par 5 Vėl kilus karui su filistinais, Jayro sūnus Elhananas nukovė Lachmį, gatiečio Galijoto brolį, kurio ietis buvo kaip audėjo staklių riestuvas. 
\par 6 Po to dar kartą įvyko kova prie Gato. Ten buvo aukšto ūgio vyras, kuris turėjo po šešis pirštus ant kiekvienos rankos ir kojos; jis buvo kilęs iš milžinų. 
\par 7 Jam keikiant Izraelį, jį nukovė Jehonatanas, Dovydo brolio Šimos sūnus. 
\par 8 Šitie buvo kilę iš Gato milžinų; juos nužudė Dovydas ir jo tarnai.



\chapter{21}

\par 1 Šėtonas pakilo prieš Izraelį ir sukurstė karalių Dovydą suskaičiuoti tautą. 
\par 2 Dovydas įsakė Joabui ir kitiems kariuomenės vadams: “Eikite, suskaičiuokite izraelitus nuo Beer Šebos iki Dano ir praneškite man, kad žinočiau jų skaičių”. 
\par 3 Joabas atsakė: “Tegul Viešpats prideda prie savo žmonių šimtą kartų tiek, kiek jų yra! Argi jie visi, mano valdove karaliau, nėra tavo tarnai? Kodėl, mano valdove, reikalauji to? Kodėl nori užtraukti nusikaltimą ant Izraelio?” 
\par 4 Tačiau Joabas turėjo paklusti karaliaus žodžiui. Todėl Joabas išėjo ir, perėjęs visą Izraelį, sugrįžo į Jeruzalę. 
\par 5 Jis įteikė Dovydui tautos skaičiavimo rezultatus. Izraelyje buvo milijonas ir šimtas tūkstančių vyrų, tinkamų karui, o Jude­keturi šimtai septyniasdešimt tūkstančių. 
\par 6 Tačiau Levio ir Benjamino giminių jis neskaičiavo, nes Joabui karaliaus žodis buvo pasibjaurėjimas. 
\par 7 Dievui tai nepatiko, ir Jis baudė Izraelį. 
\par 8 Dovydas tarė Dievui: “Labai nusidėjau taip darydamas. Maldauju, atleisk savo tarnui kaltę, nes labai kvailai pasielgiau”. 
\par 9 Viešpats kalbėjo Gadui, Dovydo regėtojui: 
\par 10 “Eik ir sakyk Dovydui, ką Viešpats sako: ‘Tris dalykus tau siūlau, pasirink vieną iš jų, kurį įvykdysiu’ ”. 
\par 11 Gadas, atėjęs pas Dovydą, jam tarė: “Pasirink, ką nori: 
\par 12 trejus bado metus, bėgti nuo savo priešų tris mėnesius, kai tavo priešų kardas persekioja tave, arba Viešpaties kardą­marą, kuris siaustų krašte tris dienas, Viešpaties angelui naikinant izraelitus. Nuspręsk, ką turiu atsakyti mane siuntusiam”. 
\par 13 Dovydas atsakė Gadui: “Patekau į didelę bėdą. Bet geriau pakliūti į Viešpaties rankas, nes Jo gailestingumas begalinis, negu į žmogaus rankas”. 
\par 14 Viešpats siuntė marą Izraeliui, ir mirė Izraelyje septyniasdešimt tūkstančių vyrų. 
\par 15 Dievas siuntė angelą sunaikinti Jeruzalę. Angelui pradėjus naikinti, Viešpats gailėjosi dėl tos nelaimės ir tarė angelui naikintojui: “Užteks! Nuleisk savo ranką!” Viešpaties angelas tuo metu buvo prie jebusiečio Ornano klojimo. 
\par 16 Dovydas, pakėlęs savo akis, pamatė Viešpaties angelą, stovintį tarp dangaus ir žemės su nuogu kardu, iškeltu virš Jeruzalės. Tada Dovydas ir Izraelio vyresnieji, apsirengę ašutinėmis, puolė veidais į žemę. 
\par 17 Dovydas tarė: “Viešpatie, aš įsakiau suskaičiuoti tautą! Aš nusikaltau ir piktai pasielgiau, o šios avys, ką jos padarė? Tebūna Tavo ranka ant manęs ir mano tėvo namų, Viešpatie, mano Dieve, o ne ant Tavo tautos, kad juos pražudytų!” 
\par 18 Tada Viešpaties angelas liepė Gadui, kad jis sakytų Dovydui eiti ir pastatyti Viešpačiui aukurą jebusiečio Ornano klojime. 
\par 19 Dovydas nuėjo pagal Gado žodį, kurį jis kalbėjo Viešpaties vardu. 
\par 20 Ornanas atsisuko ir pamatė angelą. Jo keturi sūnūs kartu su juo pasislėpė. Tuo metu Ornanas kūlė kviečius. 
\par 21 Dovydas atėjo pas Ornaną. Jis, pamatęs ateinantį Dovydą, išėjo iš klojimo ir nusilenkė prieš jį veidu iki žemės. 
\par 22 Dovydas sakė Ornanui: “Parduok man klojimą, kad pastatyčiau aukurą Viešpačiui. Parduok man jį už deramą kainą, kad nelaimė liautųsi tautoje”. 
\par 23 Ornanas atsakė Dovydui: “Mano valdove karaliau, imk ir daryk, kaip tau atrodo tinkama. Jaučius deginamosioms aukoms, kūlimo įrankius malkoms ir kviečius valgomajai aukai­visa tau duodu”. 
\par 24 Karalius Dovydas atsakė Ornanui: “Ne, aš noriu pirkti už deramą kainą. Aš negaliu imti Viešpačiui, kas tau priklauso, ir aukoti deginamąją auką, kuri man nieko nekainuoja”. 
\par 25 Dovydas sumokėjo už tą vietą šešis šimtus šekelių aukso. 
\par 26 Po to Dovydas pastatė aukurą Viešpačiui, aukojo deginamąsias bei padėkos aukas ir šaukėsi Viešpaties. Jis atsakė, pasiųsdamas iš dangaus ugnį ant deginamosios aukos aukuro. 
\par 27 Tada Viešpats įsakė angelui, ir jis paslėpė savo kardą. 
\par 28 Kai Dovydas pamatė, kad Viešpats jį išklausė jebusiečio Ornano klojime, jis ten aukojo. 
\par 29 Viešpaties palapinė, kurią Mozė padirbdino dykumoje, ir deginamųjų aukų aukuras tuo metu buvo Gibeono aukštumoje. 
\par 30 Dovydas, bijodamas Viešpaties angelo kardo, negalėjo ten pasiklausti Dievo.



\chapter{22}


\par 1 Dovydas tarė: “Čia Viešpaties Dievo namai ir aukuras Izraelio deginamajai aukai”. 
\par 2 Dovydas įsakė surinkti svetimšalius, kurie buvo Izraelio krašte, ir paskyrė akmenskaldžius paruošti tinkamų akmenų Dievo namams statyti. 
\par 3 Dovydas paruošė daug geležies durų vinims bei apkaustymams ir tiek daug vario, kad negalėjo jo pasverti; 
\par 4 taip pat ir kedro rąstų be skaičiaus, nes Sidono ir Tyro gyventojai atgabeno Dovydui daug kedro rąstų. 
\par 5 Dovydas tai darė, galvodamas, kad jo sūnus Saliamonas yra jaunas ir neprityręs, o Viešpačiui statomi namai privalo būti nepaprastai didingi, kad garsas apie juos pasiektų visas šalis; todėl Dovydas prieš mirdamas šventyklos statybai atliko daugybę paruošiamųjų darbų. 
\par 6 Dovydas, pasišaukęs savo sūnų Saliamoną, įsakė jam pastatyti namus Viešpačiui, Izraelio Dievui. 
\par 7 Jis tarė Saliamonui: “Mano sūnau, aš buvau sumanęs statyti namus Viešpaties, savo Dievo, vardui, 
\par 8 bet Viešpats kalbėjo man: ‘Tu praliejai daug kraujo, vedei didelius karus. Tu nestatysi namų mano vardui, nes praliejai daug kraujo žemėje mano akivaizdoje. 
\par 9 Tau gims sūnus, jis bus ramus vyras, nes Aš jam duosiu poilsį nuo visų aplinkinių priešų. Jo vardas bus Saliamonas, ir Aš suteiksiu Izraeliui taiką ir ramybę jo dienomis. 
\par 10 Jis pastatys namus mano vardui ir jis bus mano sūnus, o Aš būsiu jo tėvas; Aš įtvirtinsiu jo karalystės sostą Izraelyje amžinai’. 
\par 11 Mano sūnau, Viešpats tebūna su tavimi, kad sėkmingai pastatytum Viešpaties, savo Dievo, namus, kaip Jis kalbėjo. 
\par 12 Tegul Viešpats suteikia tau išminties bei supratimo ir paskiria tave Izraelio valdovu; tik laikykis Viešpaties, savo Dievo, įstatymų. 
\par 13 Tau seksis, jei atidžiai vykdysi nuostatus ir įsakymus, kuriuos Viešpats davė Izraeliui per Mozę. Būk drąsus ir stiprus, nebijok ir nepasiduok baimei. 
\par 14 Aš iš savo neturto paruošiau Viešpaties namams šimtą tūkstančių talentų aukso, milijoną talentų sidabro, o vario ir geležies nepasveriamą kiekį, taip pat rąstų ir akmenų. Prie viso to tu galėsi dar pridėti. 
\par 15 Be to, tavo žinioje yra daug amatininkų: akmenskaldžių, mūrininkų, statybininkų ir visokių meistrų bet kuriam darbui atlikti. 
\par 16 Auksui, sidabrui, variui ir geležiai nėra skaičiaus. Imkis darbo, ir Viešpats bus su tavimi”. 
\par 17 Dovydas taip pat įsakė visiems Izraelio kunigaikščiams padėti jo sūnui Saliamonui: 
\par 18 “Juk Viešpats, jūsų Dievas, buvo su jumis ir suteikė jums ramybę visame krašte. Jis atidavė krašto gyventojus į mano rankas, ir visas kraštas nusilenkė Viešpačiui ir Jo tautai. 
\par 19 Nukreipkite savo širdis ir sielas ieškoti Viešpaties, savo Dievo. Pastatykite Viešpaties Dievo šventyklą, kad Viešpaties Sandoros skrynia ir šventi Dievo indai būtų įnešti į namus, kurie bus pastatyti Viešpaties vardui”.



\chapter{23}


\par 1 Kai Dovydas paseno ir sulaukė daug metų, jis savo vieton paskyrė savo sūnų Saliamoną Izraelio karaliumi. 
\par 2 Dovydas sukvietė visus Izraelio kunigaikščius ir levitus. 
\par 3 Buvo suskaičiuoti levitai, sulaukę trisdešimties metų ir vyresni. Jų skaičius buvo trisdešimt aštuoni tūkstančiai vyrų. 
\par 4 Iš jų tarnystei Viešpaties namuose buvo paskirta dvidešimt keturi tūkstančiai, šeši tūkstančiai­vyresniaisiais ir teisėjais, 
\par 5 keturi tūkstančiai­vartininkais, o keturi tūkstančiai šlovino Viešpatį instrumentais, kuriuos padirbdino Dovydas. 
\par 6 Dovydas suskirstė juos pagal tris Levio sūnus: Geršoną, Kehatą ir Merarį. 
\par 7 Geršonai buvo Ladanas ir Šimis. 
\par 8 Ladano sūnūs: pirmasis­Jehielis, kiti­Zetamas ir Joelis. 
\par 9 Šimio sūnūs: Šelomitas, Hazielis ir Haranas, iš viso trys. Šitie buvo Ladano šeimų vyresnieji. 
\par 10 Šimio sūnūs: Jahatas, Ziza, Jeušas ir Berija, iš viso keturi. 
\par 11 Jahatas buvo pirmutinis, o Ziza­antrasis. Jeušas ir Berija neturėjo daug sūnų; jie buvo laikomi viena šeima. 
\par 12 Kehato sūnūs: Amramas, Iccharas, Hebronas ir Uzielis, iš viso keturi. 
\par 13 Amramo sūnūs: Aaronas ir Mozė. Aaronas ir jo sūnūs buvo paskirti tarnauti Švenčiausioje, smilkyti Viešpaties akivaizdoje, Jam tarnauti ir laiminti Jo vardu. 
\par 14 Dievo vyro Mozės sūnūs buvo priskirti prie Levio giminės. 
\par 15 Mozės sūnūs­Geršomas ir Eliezeras. 
\par 16 Geršomo sūnus buvo Šebuelis. 
\par 17 Eliezero sūnus buvo Rehabija; Eliezeras neturėjo kitų vaikų, o Rehabijos sūnų buvo nepaprastai daug. 
\par 18 Iccharo sūnus Šelomitas buvo vyriausias. 
\par 19 Hebrono sūnūs: pirmasis­Jerijas, antrasis­Amarija, trečiasis­ Jahazielis ir ketvirtasis­Jekamamas. 
\par 20 Uzielio sūnūs: pirmasis­Michėjas ir antrasis­Išija. 
\par 21 Merario sūnūs­Machlis ir Mušis. Machlio sūnūs­Eleazaras ir Kišas. 
\par 22 Eleazaras mirė, neturėdamas sūnų, tik dukteris, kurias vedė Kišo sūnūs, jų pusbroliai. 
\par 23 Mušio sūnūs: Machlis, Ederas ir Jerimotas, trys. 
\par 24 Šitie buvo Levio palikuonys, vyresnieji savo tėvų namuose, surašyti pavieniui vardais; dvidešimties metų amžiaus ir vyresni, kurie atliko tarnystę Viešpaties namuose. 
\par 25 Dovydas sakė: “Viešpats, Izraelio Dievas, suteikė ramybę savo tautai, kad jie gyventų Jeruzalėje per amžius. 
\par 26 Levitams nebereikės nešioti palapinės ir indų, reikalingų tarnavimui šventykloje”. 
\par 27 Pagal paskutinius Dovydo žodžius levitai, dvidešimties metų amžiaus ir vyresni, buvo suskaičiuoti. 
\par 28 Jų tarnavimas buvo padėti aaronitams Viešpaties namų tarnyboje, tvarkyti kiemus ir kambarius, valyti bei prižiūrėti šventyklos reikmenis ir atlikti visus darbus prie Dievo namų; 
\par 29 parūpinti padėtinę duoną, miltus valgomajai aukai, neraugintus papločius, keptas ir maišytas su aliejumi aukas, svarsčius bei saikus; 
\par 30 kas rytą ir vakarą dėkoti ir šlovinti Viešpatį, 
\par 31 aukojant deginamąsias aukas per sabatą, jauną mėnulį ir kitomis šventėmis, kaip buvo nustatyta. 
\par 32 Jie turėjo saugoti Susitikimo palapinę, šventyklą ir savo brolius Aarono sūnus, tarnaudami Viešpaties namuose.



\chapter{24}

\par 1 Toks yra Aarono palikuonių suskirstymas. Aarono sūnūs: Nadabas ir Abihuvas, Eleazaras ir Itamaras. 
\par 2 Nadabas ir Abihuvas mirė pirma savo tėvo ir nepaliko sūnų; kunigų tarnystę ėjo Eleazaras ir Itamaras. 
\par 3 Dovydas paskyrė Cadoką iš Eleazaro sūnų ir Ahimelechą iš Itamaro sūnų atlikti tarnystę pakaitomis. 
\par 4 Kadangi Eleazaro sūnų buvo daugiau negu Itamaro sūnų, tai juos taip suskirstė, kad iš Eleazaro sūnų buvo šešiolika vyresniųjų savo tėvų namuose, o iš Itamaro­aštuoni. 
\par 5 Juos paskirstė mesdami burtą, nes šventyklos vyresniųjų ir vyresniųjų prieš Dievą buvo ir iš Eleazaro sūnų, ir iš Itamaro. 
\par 6 Juos surašė raštininkas Netanelio sūnus Šemaja, levitas, prižiūrint Dovydui, kunigaikščiams, kunigui Cadokui, Abjataro sūnui Ahimelechui ir kunigų bei levitų šeimų vyresniesiems. Po vieną burtą metė pakaitomis Eleazaro ir Itamaro šeimos. 
\par 7 Dvidešimt keturių skyrių paskyrimas burtų keliu buvo padarytas tokia eile: Jehojaribas, Jedaja, 
\par 8 Harimas, Seorimas, 
\par 9 Malkija, Mijaminas, 
\par 10 Hakocas, Abija, 
\par 11 Ješūva, Šechanijas, 
\par 12 Eljašibas, Jakimas, 
\par 13 Hupa, Ješebabas, 
\par 14 Bilga, Imeras, 
\par 15 Hezyras, Hapicecas, 
\par 16 Petachija, Ezechielis, 
\par 17 Jachinas, Gamulas, 
\par 18 Delajas ir Maazijas. 
\par 19 Tokia tvarka jie įeidavo tarnauti į Viešpaties namus pagal jų tėvo Aarono nuostatus, kaip jam buvo įsakęs Viešpats, Izraelio Dievas. 
\par 20 Kiti Levio palikuonys: Amramo­Šubaelis, Šubaelio­Jechdijas. 
\par 21 Rehabijo vyriausias sūnus buvo Išija. 
\par 22 Iccharų­Šelomotas, Šelomoto­Jahatas. 
\par 23 Iš Hebrono­Jerijas, Amarijas, Jahazielis, Jekamamas. 
\par 24 Uzielio­Michėjas, Michėjo­ Šamyras. 
\par 25 Michėjo brolis­Išija, Išijos sūnus­Zacharijas. 
\par 26 Merario sūnūs­Machlis ir Mušis. Jaazijo sūnus­Benas. 
\par 27 Merario palikuonys iš jo sūnaus Jaazijo: Benas, Šohamas, Zakūras ir Ibris. 
\par 28 Machlio­Eleazaras, kuris neturėjo sūnų. 
\par 29 Kišo sūnus­Jerachmeelis. 
\par 30 Mušio sūnūs: Machlis, Ederas ir Jerimotas. Šitie buvo levitų palikuonys. 
\par 31 Taip pat burtų keliu jie, tiek šeimos vyriausias, tiek jauniausias, nustatė eilę kaip ir jų broliai aaronitai karaliaus Dovydo, kunigų Cadoko ir Ahimelecho ir levitų šeimų vyresniųjų akivaizdoje.



\chapter{25}


\par 1 Dovydas su vyresniaisiais paskyrė Asafo, Hemano ir Jedutūno sūnus skambinti psalteriais, arfomis ir cimbolais. Šiam tarnavimui buvo paskirti muzikantai. 
\par 2 Asafo palikuonių buvo keturi: Zakūras, Juozapas, Netanija ir Asarela. Jiems vadovavo Asafas, laikydamasis karaliaus nurodymų. 
\par 3 Jedutūno palikuonys: Gedalijas, Ceris, Izaijas, Šimis, Hašabija ir Matitijas. Jiems vadovavo jų tėvas Jedutūnas; jie skambino arfomis, garbindami Viešpatį ir dėkodami Jam. 
\par 4 Hemano palikuonys: Bukijas, Matanijas, Uzielis, Šebuelis ir Jerimotas, Hananija, Hananis, Elijata, Gidaltis ir Romamti Ezeras, Jošbekaša, Malotis, Hotyras, Mahazijotas. 
\par 5 Visi jie buvo Hemano, karaliaus regėtojo, sūnūs pagal Dievo žodį, kad išaukštintų jo ragą. Dievas davė Hemanui keturiolika sūnų ir tris dukteris. 
\par 6 Jie visi, tėvui vadovaujant, giedojo Viešpaties namuose, pritariant cimbolams, arfoms ir psalteriams pagal karaliaus nurodymą Asafui, Jedutūnui ir Hemanui. 
\par 7 Jų skaičius kartu su broliais buvo du šimtai aštuoniasdešimt aštuoni. Jie buvo išmokyti giesmių Viešpačiui ir įgudę. 
\par 8 Jie metė burtus dėl tarnavimo tvarkos: kaip mažas, taip ir didelis, kaip mokytojas, taip ir mokinys. 
\par 9 Pirmasis burtas teko Juozapui iš Asafo giminės. Antras­Gedalijui su sūnumis ir broliais, jų buvo dvylika. 
\par 10 Trečias­Zakūrui su sūnumis ir broliais, jų buvo dvylika. 
\par 11 Ketvirtas­Icriui su sūnumis ir broliais, jų buvo dvylika. 
\par 12 Penktas­Netanijui su sūnumis ir broliais, jų buvo dvylika. 
\par 13 Šeštas­Bukijui su sūnumis ir broliais, jų buvo dvylika. 
\par 14 Septintas­Jesarelai su sūnumis ir broliais, jų buvo dvylika. 
\par 15 Aštuntas­Izaijui su sūnumis ir broliais, jų buvo dvylika. 
\par 16 Devintas­Matanijui su sūnumis ir broliais, jų buvo dvylika. 
\par 17 Dešimtas­Šimiui su sūnumis ir broliais, jų buvo dvylika. 
\par 18 Vienuoliktas­Azareliui su sūnumis ir broliais, jų buvo dvylika. 
\par 19 Dvyliktas­Hašabijai su sūnumis ir broliais, jų buvo dvylika. 
\par 20 Tryliktas­Subaeliui su sūnumis ir broliais, jų buvo dvylika. 
\par 21 Keturioliktas­Matitijui su sūnumis ir broliais, jų buvo dvylika. 
\par 22 Penkioliktas­Jeremotui su sūnumis ir broliais, jų buvo dvylika. 
\par 23 Šešioliktas­Hananijai su sūnumis ir broliais, jų buvo dvylika. 
\par 24 Septynioliktas­Jošbekašai su sūnumis ir broliais, jų buvo dvylika. 
\par 25 Aštuonioliktas­Hananiui su sūnumis ir broliais, jų buvo dvylika. 
\par 26 Devynioliktas­Maločiui su sūnumis ir broliais, jų buvo dvylika. 
\par 27 Dvidešimtas­Elijatai su sūnumis ir broliais, jų buvo dvylika. 
\par 28 Dvidešimt pirmas­Hotyrui su sūnumis ir broliais, jų buvo dvylika. 
\par 29 Dvidešimt antras­Gidalčiui su sūnumis ir broliais, jų buvo dvylika. 
\par 30 Dvidešimt trečias­Mahazijotui su sūnumis ir broliais, jų buvo dvylika. 
\par 31 Dvidešimt ketvirtas­Romamti Ezerui su sūnumis ir broliais, jų buvo dvylika.



\chapter{26}

\par 1 Vartininkų paskirstymas. Korės sūnaus Mešelemijos iš Asafo giminės sūnūs: 
\par 2 pirmagimis­Zacharijas, kiti­ Jedijaelis, Zebadijas, Jatnielis, 
\par 3 Elamas, Johananas, Eljehoenajas. 
\par 4 Obed Edomo sūnūs: pirmagimis­Šemaja, kiti­Jehozabadas, Joachas, Sacharas, Netanelis, 
\par 5 Amielis, Isacharas, Peuletajas­ aštuoni, nes Dievas jį laimino. 
\par 6 Šemaja turėjo sūnų; jie buvo šeimų vyresnieji, karžygiai. 
\par 7 Šemajos sūnūs: Otnis, Refaelis, Jobedas, Elzabadas ir jo broliai Elihuvas ir Semachijas, kurie buvo stiprūs vyrai. 
\par 8 Obed Edomo palikuonių su sūnumis ir broliais, tinkančių tarnybai vyrų, iš viso šešiasdešimt du. 
\par 9 Mešelemijos šeimos buvo aštuoniolika stiprių vyrų. 
\par 10 Merario sūnaus Hosos sūnūs: vyriausias buvo Šimris, nors jis nebuvo pirmagimis, jo tėvas jį paskyrė vyriausiuoju, 
\par 11 Hilkijas, Tebalijas ir Zacharijas. Hosos sūnų ir brolių buvo trylika vyrų. 
\par 12 Vartininkų vyresnieji pasiskirstė tarnauti prie Viešpaties namų pagal eilę, 
\par 13 mesdami burtus kaip mažas, taip ir didelis, pagal savo tėvo namus prie kiekvienų vartų. 
\par 14 Rytų vartai teko Šelemijai, šiauriniai­jo sūnui Zacharijui, išmintingam patarėjui. 
\par 15 Obed Edomui­pietų vartai, jo sūnums­sandėliai. 
\par 16 Šupimui ir Hosai­vakarų ir Šalecheto vartai prie kalvos vieškelio, kur sargybinis prie sargybinio stovėdavo. 
\par 17 Rytinėje pusėje buvo šeši levitai, šiaurinėje­keturi, pietinėje­ keturi, o prie sandėlių­po du. 
\par 18 Vakarinėje pusėje buvo keturi prie vieškelio, du­prie priestato. 
\par 19 Šitie vyrai buvo vartininkų grupių vyresnieji, korachų ir Merario palikuonys. 
\par 20 Levitui Ahijai buvo pavesta saugoti šventyklos turtus ir pašvęstus daiktus. 
\par 21 Geršono Ladano palikuonis­ Jehielis. 
\par 22 Jehielio sūnūs: Zetamas ir jo brolis Joelis; juodu buvo paskirti šventyklos turtų prižiūrėtojais. 
\par 23 Iš amramų, iccharų, hebronų, uzielitų buvo 
\par 24 Mozės sūnaus Geršomo sūnus Šebuelis, vyriausiasis turtų prižiūrėtojas. 
\par 25 Jo brolio Eliezero sūnus­Rehabijas, jo sūnus­Izaijas, jo sūnus­Jehoramas, jo sūnus­Zichris ir jo sūnus­Šelomitas. 
\par 26 Šelomitas ir jo broliai buvo paskirti prižiūrėti dovanoms, kurias karalius Dovydas, šeimų vyresnieji, tūkstantininkai, šimtininkai ir kariuomenės vadai buvo paaukoję. 
\par 27 Karuose laimėto grobio dalis, paskirta Viešpaties namų reikalams, 
\par 28 ir visi turtai, gauti iš regėtojo Samuelio, Kišo sūnaus Sauliaus, Nero sūnaus Abnero ir Cerujos sūnaus Joabo, buvo pavesti Šelomito ir jo brolių priežiūrai. 
\par 29 Iccharitas Kenanijas ir jo sūnūs buvo paskirti Izraelyje valdytojais ir teisėjais. 
\par 30 Hebronas Hašabijas ir jo giminės, tūkstantis septyni šimtai narsių vyrų, buvo paskirti Jordano vakaruose esantiems izraelitams viršininkais Viešpaties reikalams ir karaliaus tarnybai. 
\par 31 Vyriausias tarp hebronų buvo Jerija. Jie buvo suskaityti šeimomis keturiasdešimtaisiais Dovydo valdymo metais. Buvo rasta narsių vyrų, gyvenančių Gileado Jazeryje, 
\par 32 du tūkstančiai septyni šimtai. Karalius Dovydas pavedė jiems valdyti rubenus, gadus ir pusę Manaso giminės visuose Dievo ir karaliaus reikaluose.



\chapter{27}

\par 1 Šie yra Izraelio šeimų vyresnieji, kariuomenės būrių vadai ir karaliaus valdininkai. Tie būriai keisdavosi kas mėnesį. Kiekviename būryje buvo dvidešimt keturi tūkstančiai vyrų. 
\par 2 Pirmojo mėnesio būrio vyresnysis­Jašobamas, Zabdielio sūnus. Jo būryje buvo dvidešimt keturi tūkstančiai. 
\par 3 Jis buvo iš Pereco vaikų, vyriausiasis iš visų pirmo mėnesio vadų. 
\par 4 Antrojo mėnesio būrio vyresnysis­ahoachas Dodajas. Jo būryje vadas buvo ir Miklotas. Iš viso būryje buvo dvidešimt keturi tūkstančiai. 
\par 5 Trečiojo mėnesio kariuomenės vyresnysis­vyriausiojo kunigo Jehojados sūnus Benaja. Jo būryje buvo dvidešimt keturi tūkstančiai. 
\par 6 Jis buvo vyriausiasis tarp trisdešimties karžygių. Jo būryje buvo jo sūnus Amizabadas. 
\par 7 Ketvirtojo mėnesio­Joabo brolis Asaelis, o po jo­Zebadija, jo sūnus. Jo būryje buvo dvidešimt keturi tūkstančiai. 
\par 8 Penktojo mėnesio­izrachas Šamhutas. Jo būryje buvo dvidešimt keturi tūkstančiai. 
\par 9 Šeštojo mėnesio­Ikešo sūnus Ira, tekojietis. Jo būryje buvo dvidešimt keturi tūkstančiai. 
\par 10 Septintojo mėnesio­efraimas Helecas iš Pelojos. Jo būryje buvo dvidešimt keturi tūkstančiai. 
\par 11 Aštuntojo mėnesio­hušietis Sibechajas iš zerachų. Jo būryje buvo dvidešimt keturi tūkstančiai. 
\par 12 Devintojo mėnesio­anatotietis Abiezeras iš Benjamino. Jo būryje buo dvidešimt keturi tūkstančiai. 
\par 13 Dešimtojo mėnesio­netofietis Mahrajas iš zerachų. Jo būryje buvo dvidešimt keturi tūkstančiai. 
\par 14 Vienuoliktojo mėnesio­Benaja iš Piratono, efraimas. Jo būryje buvo dvidešimt keturi tūkstančiai. 
\par 15 Dvyliktojo mėnesio­netofietis Heldajas iš Otnielio. Jo būryje buvo dvidešimt keturi tūkstančiai. 
\par 16 Izraelio giminių vyresnieji: rubenams­Zichrio sūnus Eliezeras; simeonams­Maakos sūnus Šefatijas; 
\par 17 levitams­Kemuelio sūnus Hašabija, aaronitams­Cadokas; 
\par 18 Judui­Elihuvas, vienas iš Dovydo brolių; Isacharui­Mykolo sūnus Omris; 
\par 19 Zabulonui­Abdijo sūnus Isšmajas; Naftaliui­Azrielio sūnus Jerimotas; 
\par 20 efraimams­Azazijo sūnus Ozėjas; pusei Manaso giminės­Pedajo sūnus Joelis; 
\par 21 pusei Manaso giminės Gileade­Zacharijo sūnus Idojas; Benjaminui­Abnero sūnus Jaasielis; 
\par 22 Danui­Jerohamo sūnus Azarelis. Šitie buvo Izraelio giminių kunigaikščiai. 
\par 23 Dovydas neskaičiavo tų, kurie buvo dvidešimties metų ir jaunesni, nes Viešpats buvo pažadėjęs padauginti Izraelį kaip žvaigždes danguje. 
\par 24 Cerujos sūnus Joabas buvo pradėjęs skaičiuoti izraelitus, bet nebaigė, nes Izraelis už tai buvo nubaustas. Šis skaičiavimas nepateko į karaliaus Dovydo metraščius. 
\par 25 Karaliaus turtus saugojo Adielio sūnus Azmavetas. Tuos gi turtus, kurie buvo miestuose, kaimuose ir bokštuose­Uzijo sūnus Jehonatanas; 
\par 26 lauko darbus prižiūrėjo Kelubo sūnus Ezris; 
\par 27 vynuogynus­Šimis iš Ramos; vyno atsargas­šefamietis Zabdis; 
\par 28 alyvmedžius ir figmedžius, augusius Šefeloje,­gederietis Baal Hananas; aliejaus sandėlius­Joasas; 
\par 29 galvijus, kurie ganėsi Šarone,­ šaronietis Šitrajas; galvijus slėniuose­Adlajo sūnus Šafatas; 
\par 30 kupranugarius­izmaelitas Obilas; asilus­meronotietis Jechdijas; 
\par 31 avis ir ožkas­hagaras Jazizas. Šitie buvo karaliaus Dovydo nuosavybės vyriausieji prižiūrėtojai. 
\par 32 Dovydo dėdė Jehonatanas, išmintingas ir raštingas vyras, buvo patarėjas, o Hachmonio sūnus Jehielis mokė karaliaus sūnus. 
\par 33 Ahitofelis buvo karaliaus patarėjas, o archas Hušajas buvo karaliaus draugas. 
\par 34 Po Ahitofelio buvo Benajos sūnus Jehojada ir Abjataras, o vyriausiasis karaliaus kariuomenės vadas buvo Joabas.



\chapter{28}

\par 1 Dovydas sušaukė į Jeruzalę visus Izraelio ir giminių kunigaikščius, būrių viršininkus, tūkstantininkus, šimtininkus, karaliaus bei jo sūnų nuosavybės ir gyvulių vyriausiuosius prievaizdus, rūmų valdininkus, narsius vyrus ir karžygius. 
\par 2 Karalius atsistojęs tarė: “Paklausykite, mano broliai, mano tauta. Aš norėjau statyti namus Viešpaties Sandoros skryniai, pakojį mūsų Dievui ir buvau pasiruošęs statybai, 
\par 3 bet Dievas man tarė: ‘Tu nestatysi namų mano vardui, nes tu esi karys ir praliejai kraują’. 
\par 4 Tačiau Viešpats, Izraelio Dievas, pasirinko mane iš visų mano tėvo namų, kad būčiau karalius viso Izraelio. Jis išrinko iš Judo giminės mano tėvo namus ir iš mano tėvo sūnų mane panorėjo paskirti viso Izraelio karaliumi. 
\par 5 Viešpats davė man daug sūnų ir iš visų Jis pasirinko Saliamoną, kad jis sėdėtų Viešpaties karalystės soste ir valdytų Izraelį. 
\par 6 Jis man pasakė: ‘Tavo sūnus Saliamonas pastatys mano namus ir mano kiemus; Aš jį išsirinkau savo sūnumi ir Aš būsiu jam tėvas. 
\par 7 Be to, Aš įtvirtinsiu jo karalystę per amžius, jei jis vykdys mano įsakymus ir nurodymus kaip iki šiolei’. 
\par 8 Taigi dabar viso Izraelio ir mūsų Dievo akivaizdoje jums sakau: klausykite ir vykdykite visus Viešpaties, savo Dievo, įsakymus, kad gyventumėte šioje geroje žemėje ir ji liktų jūsų vaikams per amžius. 
\par 9 O tu, mano sūnau Saliamonai, pažink savo tėvo Dievą ir Jam tarnauk tobula širdimi ir su tikru noru, nes Viešpats ištiria visų širdis ir siekius. Jei Jo ieškosi, rasi Jį, o jei Jį paliksi, Jis atmes tave amžiams. 
\par 10 Dabar įsidėmėk! Viešpats tave išsirinko, kad pastatytum namus šventyklai. Būk stiprus ir daryk tai”. 
\par 11 Dovydas davė savo sūnui Saliamonui šventyklos planą: prieangių, pastatų, sandėlių, aukštutinių ir vidinių patalpų ir Švenčiausiosios; 
\par 12 taip pat ir kitus brėžinius, kurie buvo jo širdyje: Viešpaties namų kiemo, visų aplinkui esančių patalpų, Dievo namų turtų saugyklos ir pašvęstų daiktų saugyklos, 
\par 13 patalpų kunigų bei levitų padaliniams ir įvairiems šventyklos darbams, ir visų Viešpaties namų tarnystės indų. 
\par 14 Jis davė Saliamonui aukso ir sidabro auksiniams ir sidabriniams indams nulieti, nustatydamas atskirų indų svorį: 
\par 15 auksinėms žvakidėms ir jų lempoms nustatė kiekvienos žvakidės ir lempos svorį, taip pat sidabrinėms žvakidėms ir jų lempoms nustatė kiekvienos žvakidės ir lempos svorį pagal jų paskirtį. 
\par 16 Taip pat davė aukso padėtinės duonos stalams iš aukso ir sidabro sidabriniams stalams, 
\par 17 gryno aukso šakutėms, puodams, taurėms ir auksiniams dubenims, kiekvienam pagal svorį, sidabriniams dubenims, kiekvienam sidabro pagal svorį; 
\par 18 smilkomajam aukurui gryniausio aukso pagal svorį ir cherubams, kurie laikė išskleidę sparnus ir dengė Viešpaties Sandoros skrynią. 
\par 19 Visa tai buvo užrašyta pagal Viešpaties duotą apreiškimą, kaip Jis nurodė atlikti darbus. 
\par 20 Dovydas tarė savo sūnui Saliamonui: “Būk stiprus ir drąsus bei imkis darbo! Nebijok ir nenusigąsk, nes Viešpats Dievas, mano Dievas, bus su tavimi. Jis nepaliks tavęs, kol baigsi visus Viešpaties namų statybos darbus. 
\par 21 Kunigai ir levitai yra pasiruošę bet kokiai Dievo namų tarnystei; kiekviename darbe tau padės patyrę meistrai ir visa tauta bei kunigaikščiai vykdys tavo įsakymus”.



\chapter{29}

\par 1 Karalius Dovydas tarė izraelitams: “Mano sūnus Saliamonas, kurį Dievas išsirinko, dar jaunas ir nepatyręs, o darbas­didelis, nes rūmai ne žmogui, bet Viešpačiui Dievui. 
\par 2 Iš visų jėgų paruošiau aukso Dievo namų auksiniams reikmenims, sidabro­sidabriniams, vario­variniams, geležies­geležiniams, medžio­mediniams; taip pat onikso ir spalvotų akmenų mozaikai, įvairių brangių akmenų ir gausybę marmuro. 
\par 3 Be to, Dievo namams aš pridedu iš savo nuosavybės aukso ir sidabro: 
\par 4 tris tūkstančius talentų Ofyro aukso, septynis tūkstančius talentų valyto sidabro sienoms aptraukti, 
\par 5 aukso auksiniams ir sidabro sidabriniams dirbiniams, kuriuos padarys meistrai. Kas iš jūsų šiandien laisva valia norėtų aukoti Viešpačiui?” 
\par 6 Izraelio giminių ir šeimų vadai, tūkstantininkai, šimtininkai bei karaliaus darbų vyriausieji prievaizdai aukojo dosniai 
\par 7 ir davė Dievo namų statybai penkis tūkstančius talentų aukso ir dešimt tūkstančių auksinių monetų; dešimt tūkstančių talentų sidabro, aštuoniolika tūkstančių talentų vario ir šimtą tūkstančių talentų geležies. 
\par 8 Kas turėjo brangių akmenų, atidavė juos Viešpaties namų iždui, geršono Jehielio globon. 
\par 9 Tauta džiaugėsi, nes jie aukojo Viešpačiui dosniai ir tobula širdimi, ir karalius Dovydas tuo taip pat labai džiaugėsi. 
\par 10 Dovydas garbino Viešpatį susirinkusiųjų akivaizdoje ir tarė: “Garbė Tau per amžius, Viešpatie, mūsų tėvo Izraelio Dieve! 
\par 11 Tavo, Viešpatie, yra didybė ir galybė, ir pergalė, ir šlovė, ir išaukštinimas, nes Tau priklauso visa danguje ir žemėje. Viešpatie, Tavo yra karalystė ir Tu esi valdovas virš visko. 
\par 12 Iš Tavęs ateina turtai ir garbė, Tu viskam karaliauji. Tavo rankoje yra jėga ir galybė, Tu gali išaukštinti ir suteikti stiprybės. 
\par 13 Mūsų Dieve, mes dėkojame Tau ir giriame Tavo šlovingą vardą. 
\par 14 Kas esu aš ir kas mano tauta, kad mes galėtume taip dosniai aukoti Tau? Juk iš Tavęs ateina viskas, o mes tik atiduodame, ką gavome iš Tavęs. 
\par 15 Mes esame svetimi ir ateiviai Tavo akivaizdoje, kaip buvo ir mūsų tėvai. Mūsų dienos žemėje kaip šešėlis ir nė vienas nepasiliekame. 
\par 16 Viešpatie, mūsų Dieve, visa, ką mes šiandien aukojame Tavo namams, Tavo šventajam vardui, ateina iš Tavęs ir yra Tavo! 
\par 17 Mano Dieve, žinau, kad Tu ištiri žmonių širdis ir mėgsti nuoširdumą. Aš visa tai paaukojau iš tyros širdies ir matau, kad čia esanti tauta aukoja Tau su džiaugsmu ir noriai. 
\par 18 Viešpatie, mūsų tėvų Abraomo, Izaoko ir Jokūbo Dieve, išlaikyk savo tautos širdžių nusistatymą tokį, koks jis šiandien, kad jie liktų Tau ištikimi. 
\par 19 O mano sūnui Saliamonui duok tobulą širdį, kad jis laikytųsi Tavo įsakymų, nuostatų bei paliepimų ir pastatytų šventyklą, kuriai paruošiau reikalingų medžiagų”. 
\par 20 Tuomet Dovydas tarė susirinkusiems: “Garbinkite Viešpatį, savo Dievą”. Susirinkusieji parpuolė ant žemės prieš Viešpatį bei karalių ir garbino Viešpatį, savo tėvų Dievą. 
\par 21 Kitą dieną jie aukojo Viešpačiui deginamąsias aukas: tūkstantį jaučių, tūkstantį avinų, tūkstantį ėriukų su priklausančiomis geriamosiomis ir padėkos aukomis, kad maisto užtektų visiems izraelitams. 
\par 22 Tą dieną jie valgė ir gėrė Viešpaties akivaizdoje su pakilia nuotaika ir antrą kartą paskelbė karaliumi Dovydo sūnų Saliamoną; jie patepė jį kunigaikščiu Viešpačiui, o Cadoką­kunigu. 
\par 23 Saliamonas atsisėdo Viešpaties soste kaip karalius savo tėvo Dovydo vieton; jam sekėsi, ir visas Izraelis jo klausė. 
\par 24 Visi kunigaikščiai, karžygiai ir visi karaliaus Dovydo sūnūs pakluso karaliui Saliamonui. 
\par 25 Viešpats padarė Saliamoną labai didį viso Izraelio akyse ir jam suteikė tokią karališką didybę, kokios neturėjo joks Izraelio karalius iki jo. 
\par 26 Taip Jesės sūnus Dovydas karaliavo Izraeliui. 
\par 27 Jis karaliavo Izraeliui keturiasdešimt metų; Hebrone jis karaliavo septynerius metus, o Jeruzalėje­trisdešimt trejus metus. 
\par 28 Jis mirė sulaukęs žilos senatvės, turtingas ir gerbiamas; jo sūnus Saliamonas užėmė jo vietą. 
\par 29 Karaliaus Dovydo darbai nuo pradžios iki galo yra užrašyti regėtojo Samuelio, pranašo Natano ir regėtojo Gado knygose. 
\par 30 Ten aprašyta jo viešpatavimas, galybė ir visa, kas atsitiko jam ir Izraeliui bei kitoms karalystėms.



\end{document}