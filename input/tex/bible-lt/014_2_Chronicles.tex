\begin{document}

\title{
\par 2 Chronicles}

\chapter{1}

\par 1 Dovydo sūnus Saliamonas įsitvirtino karalystėje, o Viešpats, jo Dievas, buvo su juo ir labai išaukštino jį. 
\par 2 Saliamonas kalbėjo visam Izraeliui: tūkstantininkams, šimtininkams, teisėjams, visiems Izraelio valdytojams ir šeimų vyresniesiems. 
\par 3 Ir Saliamonas su visais izraelitais ėjo į Gibeono aukštumą, kur buvo Dievo Susitikimo palapinė, Viešpaties tarno Mozės padaryta dykumoje. 
\par 4 Bet Dievo skrynią Dovydas buvo atgabenęs iš Kirjat Jearimo į vietą, kurią jis paruošė Jeruzalėje, pastatęs jai ten palapinę. 
\par 5 Be to, varinis aukuras, kurį padirbo Hūro sūnaus Ūrio sūnus Becalelis, buvo ten prie Viešpaties palapinės; Saliamonas ir izraelitai susirinko toje vietoje. 
\par 6 Saliamonas aukojo tūkstantį deginamųjų aukų ant varinio aukuro prie Susitikimo palapinės, Viešpaties akivaizdoje. 
\par 7 Tą pačią naktį Dievas pasirodė Saliamonui ir jam tarė: “Prašyk, ką Aš galėčiau duoti”. 
\par 8 Saliamonas atsakė Dievui: “Tu parodei mano tėvui Dovydui didelį gailestingumą ir mane padarei karaliumi jo vietoje. 
\par 9 Viešpatie Dieve, įtvirtink savo pažadą, kurį davei mano tėvui Dovydui. Nes Tu padarei mane karaliumi žmonių, kurių yra kaip žemės dulkių. 
\par 10 Duok man išminties ir pažinimo, kad galėčiau įeiti ir išeiti prieš šitą tautą. Kas galėtų teisti šią Tavo tautą, kuri yra tokia didelė?” 
\par 11 Dievas atsakė Saliamonui: “Kadangi tai buvo tavo širdyje ir tu neprašei turtų, garbės ar savo priešų gyvybės ir net ilgo amžiaus, bet prašei išminties ir pažinimo, kad galėtum teisti mano tautą, kurios karaliumi tave padariau, 
\par 12 tai tau yra suteikta išmintis ir pažinimas. Ir Aš duosiu tau turtų bei garbės, kokių neturėjo joks karalius iki tavęs ir neturės po tavęs”. 
\par 13 Saliamonas sugrįžo nuo Gibeono aukštumos, nuo Susitikimo palapinės į Jeruzalę ir karaliavo Izraelyje. 
\par 14 Saliamonas turėjo tūkstantį keturis šimtus kovos vežimų ir dvylika tūkstančių raitelių, kuriuos jis paskirstė kovos vežimams skirtuose miestuose ir Jeruzalėje. 
\par 15 Karalius padarė, kad sidabro ir aukso Jeruzalėje buvo kaip akmenų, o kedro medžių buvo kaip figmedžių, kurių gausiai auga slėniuose. 
\par 16 Saliamonas parsigabendavo žirgų iš Egipto ir Kevės; karaliaus pirkliai pirkdavo juos Kevėje už pinigus. 
\par 17 Iš Egipto pirkdavo kovos vežimą už šešis šimtus šekelių sidabro, o žirgą­už šimtą penkiasdešimt. Taip pat jie pristatydavo žirgus visiems hetitų ir Sirijos karaliams.


\chapter{2}


\par 1 Saliamonas užsimojo statyti namus Viešpaties vardui ir karališkus namus sau. 
\par 2 Jis paskyrė septyniasdešimt tūkstančių vyrų nešikais ir aštuoniasdešimt tūkstančių akmenskaldžiais kalnuose, o jų prižiūrėtojais­tris tūkstančius šešis šimtus. 
\par 3 Saliamonas siuntė pas Tyro karalių Hiramą, sakydamas: “Kaip tu siuntei kedrų mano tėvui Dovydui, kai jis statėsi namus, taip daryk ir man. 
\par 4 Aš statau namus Viešpaties, savo Dievo, vardui, kurie bus pašvęsti Jam. Ten bus smilkomi kvapnūs smilkalai Jo akivaizdoje, nuolat laikoma padėtinė duona, aukojamos deginamosios aukos kas rytą ir vakarą, taip pat sabatais, per jauną mėnulį ir mūsų Viešpaties šventėmis; taip įsakyta Izraeliui daryti per amžius. 
\par 5 Namai, kuriuos aš statau, bus dideli, nes mūsų Dievas didesnis už visus dievus. 
\par 6 Kas galėtų Jam pastatyti tinkamus namus? Dangus ir dangų dangūs negali Jo sutalpinti. Tai kas aš esu, kad statyčiau Jam namus? Tebūna jie bent vieta aukoms deginti Jo akivaizdoje. 
\par 7 Atsiųsk man vyrą, kuris būtų įgudęs daryti iš aukso, sidabro, vario, geležies, raudono, mėlyno ir violetinio audeklo, be to, mokantį atlikti raižymo darbus. Jis dirbs kartu su mano amatininkais iš Judo ir Jeruzalės, kuriuos paruošė mano tėvas Dovydas. 
\par 8 Atsiųsk man taip pat Libano kedrų, kiparisų ir kadagio medžių. Aš žinau, kad tavo tarnai moka kirsti medžius Libane. Mano tarnai atvyks ir dirbs kartu su tavo tarnais, 
\par 9 kad paruoštų man daugybę rąstų, nes namai, kuriuos rengiuosi statyti, bus dideli ir nuostabūs. 
\par 10 Aš aprūpinsiu tavo medžių kirtėjus maistu, duosiu dvidešimt tūkstančių homerų maltų kviečių, dvidešimt tūkstančių homerų miežių, dvidešimt tūkstančių batų vyno ir dvidešimt tūkstančių batų aliejaus”. 
\par 11 Tyro karalius Hiramas atsakė Saliamonui laišku: “Kadangi Viešpats myli savo tautą, Jis tave padarė jos karaliumi. 
\par 12 Palaimintas Viešpats, Izraelio Dievas, kuris sukūrė dangų ir žemę, kad davė karaliui Dovydui išmintingą sūnų, apdovanotą sumanumu ir protu, kuris rengiasi statyti namus Viešpačiui ir karališkus namus sau. 
\par 13 Aš siunčiu įgudusį, išmintingą vyrą Hiramą, 
\par 14 sūnų moters iš Dano dukterų, kurio tėvas yra vyras iš Tyro, mokantį gaminti įvairius daiktus iš aukso, sidabro, vario, geležies, akmens, medžio, raudono, mėlyno ir violetinio audeklo bei plonos drobės; mokantį raižyti įvairius raižinius ir daryti kiekvieną darbą, kurie jam pavedami. Jis galės dirbti kartu su tavo amatininkais, su mano valdovo, tavo tėvo Dovydo, amatininkais. 
\par 15 Kviečių, miežių, aliejaus ir vyno, kuriuos pažadėjo mano valdovas, tegul atsiunčia savo tarnams. 
\par 16 Mes prikirsime medžių Libane, kiek tau reikės, ir, surišę į sielius, nuplukdysime jūra į Jopę, iš kur galėsi juos parsigabenti į Jeruzalę”. 
\par 17 Saliamonas suskaičiavo visus svetimšalius, gyvenančius Izraelio krašte, kaip buvo padaręs jo tėvas Dovydas. Jų buvo šimtas penkiasdešimt trys tūkstančiai šeši šimtai. 
\par 18 Iš jų jis paskyrė septyniasdešimt tūkstančių nešikais, aštuoniasdešimt tūkstančių akmenskaldžiais kalnuose ir tris tūkstančius šešis šimtus darbo prižiūrėtojais.



\chapter{3}

\par 1 Saliamonas pradėjo statyti Viešpaties namus Jeruzalėje ant Morijos kalno, kur Viešpats pasirodė jo tėvui Dovydui, vietoje, kurią Dovydas paruošė jebusiečio Ornano klojime. 
\par 2 Saliamonas pradėjo juos statyti ketvirtųjų savo karaliavimo metų antrojo mėnesio antrąją dieną. 
\par 3 Dievo namų, kuriuos Saliamonas ruošėsi statyti, dydis buvo šešiasdešimt uolekčių ilgio senuoju mastu ir dvidešimt uolekčių pločio. 
\par 4 Priešakinio prieangio ilgis buvo lygus pastato pločiui­dvidešimt uolekčių, o jo aukštis buvo šimtas dvidešimt uolekčių. Jis aptraukė jį iš vidaus grynu auksu. 
\par 5 Pagrindinį pastatą jis iškalė kipariso lentomis, aptraukė jas grynu auksu ir išraižė ant jų palmes bei grandinėles. 
\par 6 Sienos buvo papuoštos brangakmeniais ir Parvaimo auksu. 
\par 7 Namų sijas, slenksčius, sienas ir duris aptraukė auksu ir ant sienų išraižė cherubus. 
\par 8 Jis padarė Šventų švenčiausiąją, kuri buvo dvidešimties uolekčių ilgio pagal namo plotį ir dvidešimties uolekčių pločio, visas jos vidus buvo aptrauktas grynu auksu, sveriančiu šešis šimtus talentų. 
\par 9 Vinys svėrė penkiasdešimt šekelių aukso. Viršutinius kambarius jis taip pat aptraukė auksu. 
\par 10 Švenčiausioje padarė du cherubus ir aptraukė juos auksu. 
\par 11 Cherubų sparnai buvo dvidešimties uolekčių ilgio: vienas penkių uolekčių ilgio sparnas siekė sieną, kitas penkių uolekčių ilgio sparnas lietė kito cherubo sparną. 
\par 12 Antro cherubo penkių uolekčių ilgio sparnas siekė priešingą sieną, o kitas jo penkių uolekčių ilgio sparnas lietė pirmojo cherubo sparną. 
\par 13 Cherubų ištiesti sparnai buvo dvidešimties uolekčių ilgio. Jie stovėjo ant kojų, o jų veidai buvo nukreipti į šventyklos vidų. 
\par 14 Jis padarė uždangą iš raudonų, mėlynų, violetinių ir plonų lininių siūlų ir ant jos išsiuvinėjo cherubus. 
\par 15 Šventyklos priekyje jis padarė dvi kolonas, kiekvieną trisdešimt penkių uolekčių aukščio, o kapiteliai ant jų viršaus buvo penkių uolekčių. 
\par 16 Jis padarė ir grandinėles kaip Švenčiausioje ir pritvirtino kolonų viršuje. Ant kiekvienos grandinėlės buvo po šimtą granato vaisių. 
\par 17 Kolonos stovėjo šventyklos priekyje, viena­dešinėje, o antra­kairėje; dešiniąją pavadino Jachinu, o kairiąją Boazu.



\chapter{4}


\par 1 Be to, jis padarė varinį aukurą dvidešimties uolekčių ilgio, dvidešimties uolekčių pločio ir dešimties uolekčių aukščio. 
\par 2 Ir nuliejo baseiną dešimties uolekčių skersmens nuo briaunos iki briaunos ir penkių uolekčių aukščio. Jo apimtis buvo trisdešimt uolekčių. 
\par 3 Dvi eilės jaučių pavidalų buvo aplinkui jį po dešimt kiekvienoje uolektyje, nulietų išvien su baseinu. 
\par 4 Jis stovėjo ant dvylikos jaučių. Trys buvo atsigręžę į šiaurę, trys į vakarus, trys į pietus ir trys į rytus. Baseinas buvo jų viršuje ir jų užpakalinės dalys buvo po baseinu. 
\par 5 Jo storis buvo per plaštaką, briauna buvo kaip taurės briauna, kaip lelijos žiedas; jame tilpo trys tūkstančiai batų vandens. 
\par 6 Jis padarė ir dešimt praustuvių, kurios buvo pastatytos po penkias dešinėje ir kairėje pusėje. Jose buvo plaunama tai, ką aukodavo deginamajai aukai; o baseine prausdavosi kunigai. 
\par 7 Jis padarė dešimt auksinių žvakidžių pagal duotus nurodymus ir pastatė šventykloje, penkias kairėje ir penkias dešinėje. 
\par 8 Ir dešimt stalų padarė ir pastatė taip pat po penkis vienoje ir kitoje pusėje. Taip pat padarė šimtą auksinių taurių. 
\par 9 Taip pat padarė kiemą kunigams ir didįjį kiemą; jų vartai buvo variniai. 
\par 10 Baseiną pastatė dešinėje, pietryčių pusėje. 
\par 11 Hiramas dar padirbo puodus, semtuvėlius ir dubenis. Ir Hiramas baigė darbą, kurį turėjo padaryti karaliui Saliamonui dėl Dievo namų: 
\par 12 dvi kolonas, du kapitelius ant kolonų viršaus, du tinklus apdengti abiems kapiteliams, kurie buvo ant kolonų, 
\par 13 taip pat keturis šimtus granato vaisių abiems tinklams, po dvi eiles granato vaisių kiekvienam tinklui, dengiančiam kapitelius, 
\par 14 stovus ir praustuves ant stovų, 
\par 15 baseiną ir dvylika jaučių, ant kurių jis stovėjo, 
\par 16 puodus, semtuvėlius, šakutes. Visus reikmenis Hiramas padirbo karaliui Saliamonui dėl Viešpaties namų iš tikro vario. 
\par 17 Karalius juos nuliejo Jordano lygumoje, molingoje žemėje tarp Sukoto ir Ceredos. 
\par 18 Saliamonas padarė tiek daug daiktų, kad nebuvo įmanoma apskaičiuoti vario svorio. 
\par 19 Saliamonas padarė Dievo namams visus reikmenis: auksinį aukurą, stalus padėtinei duonai, 
\par 20 žvakides su lempomis iš gryno aukso, 
\par 21 gėles, lempas ir žnyples iš gryno aukso, 
\par 22 šakutes, samčius, dubenis ir smilkytuvus­visa iš gryno aukso. Šventyklos vidinės durys į Švenčiausiąją ir durys į šventyklą buvo padarytos iš aukso.



\chapter{5}


\par 1 Taip buvo pabaigti visi darbai, kuriuos Saliamonas padarė dėl Viešpaties namų, ir Saliamonas atgabeno tai, ką jo tėvas Dovydas buvo pašventinęs: auksą, sidabrą bei indus, ir sudėjo Dievo namų sandėliuose. 
\par 2 Saliamonas sušaukė Izraelio vyresniuosius, giminių vadus ir izraelitų šeimų galvas į Jeruzalę, kad atgabentų Viešpaties Sandoros skrynią iš Dovydo miesto, Siono. 
\par 3 Todėl visi Izraelio vyrai susirinko pas karalių į šventę, kuri vyko septintąjį mėnesį. 
\par 4 Atėjo visi Izraelio vyresnieji, ir levitai paėmė skrynią. 
\par 5 Jie nešė skrynią, Susitikimo palapinę ir visus šventus indus, kurie buvo palapinėje, kunigai ir levitai nešė juos. 
\par 6 Karalius Saliamonas ir visas Izraelis, susirinkęs pas jį priešais Sandoros skrynią, aukojo tiek avių ir galvijų, kad jų nebuvo įmanoma suskaičiuoti. 
\par 7 Kunigai įnešė Viešpaties Sandoros skrynią į Šventų švenčiausiąją po cherubų sparnais. 
\par 8 Cherubų sparnai buvo išskleisti virš skrynios ir dengė ją bei kartis. 
\par 9 Kartys buvo tokios ilgos, kad jų galai buvo matomi priešais Švenčiausiąją, tačiau iš lauko jų nesimatė. Taip ten yra iki šios dienos. 
\par 10 Skrynioje buvo tik dvi plokštės, kurias Mozė įdėjo Horebe, kai Viešpats padarė sandorą su izraelitais, jiems išėjus iš Egipto. 
\par 11 Kunigams išėjus iš šventyklos, nes visi ten buvę kunigai buvo pasišventinę, nepaisant jų tarnavimo eilės, 
\par 12 levitai giesmininkai Asafas, Hemanas, Jedutūnas, jų sūnūs ir broliai, vilkėdami plonos drobės rūbus, su cimbolais, psalteriais ir arfomis stovėjo į rytus nuo aukuro ir prie jų šimtas dvidešimt kunigų su trimitais. 
\par 13 Trimitininkai ir giedotojai buvo kaip vienas, šlovindami ir dėkodami Viešpačiui vienu balsu. Kai jie pakėlė savo balsus su trimitais, cimbolais ir kitais muzikos instrumentais, šlovindami Viešpatį ir sakydami: “Jis yra geras ir Jo gailestingumas amžinas”, debesis pripildė Viešpaties namus 
\par 14 taip, kad kunigai negalėjo tarnauti dėl debesies, nes Viešpaties šlovė pripildė Dievo namus.



\chapter{6}

\par 1 Tuomet Saliamonas tarė: “Viešpats kalbėjo, kad nori gyventi tirštoje tamsoje. 
\par 2 Bet aš pastačiau Tau namus, vietą, kur Tu gyventum per amžius”. 
\par 3 Karalius atsisukęs palaimino visus susirinkusius izraelitus, o visi izraelitai tuo metu stovėjo. 
\par 4 Jis sakė: “Palaimintas Viešpats, Izraelio Dievas, kuris įvykdė, ką pažadėjo mano tėvui Dovydui, sakydamas: 
\par 5 ‘Nuo tos dienos, kai išvedžiau savo tautą iš Egipto, neišsirinkau kito miesto tarp visų Izraelio giminių statyti namams, kur būtų mano vardas, ir jokio kito vyro, kuris būtų mano tautos Izraelio valdovu, 
\par 6 bet Aš išsirinkau Jeruzalę, kad mano vardas būtų joje, ir Dovydą, kad valdytų mano tautą Izraelį’. 
\par 7 Mano tėvas Dovydas norėjo pastatyti namus Viešpaties, Izraelio Dievo, vardui, 
\par 8 tačiau Viešpats kalbėjo mano tėvui Dovydui: ‘Gerai, kad tu norėjai pastatyti namus mano vardui, 
\par 9 tačiau ne tu juos pastatysi, bet tavo sūnus, kuris tau gims, pastatys namus mano vardui’. 
\par 10 Viešpats išpildė savo žodį, kurį kalbėjo. Aš užėmiau savo tėvo Dovydo vietą ir atsisėdau Izraelio soste, kaip Viešpats žadėjo, ir pastačiau namus Viešpaties, Izraelio Dievo, vardui. 
\par 11 Ten padėjau Sandoros skrynią, kurioje yra Viešpaties Sandora, padaryta su izraelitais”. 
\par 12 Saliamonas atsistojo priešais Viešpaties aukurą Izraelio akivaizdoje ir iškėlė rankas. 
\par 13 Saliamonas buvo padirbdinęs varinį paaukštinimą, penkių uolekčių ilgio bei tiek pat uolekčių pločio ir trijų uolekčių aukščio ir jį pastatęs kiemo viduryje. Užlipęs ant jo, jis atsiklaupė izraelitų akivaizdoje ir, iškėlęs rankas, 
\par 14 sakė: “Viešpatie, Izraelio Dieve, nei danguje, nei žemėje nėra dievo, kuris būtų lygus Tau, kuris laikytųsi sandoros ir būtų gailestingas savo tarnams, kurie visa širdimi pasitiki Tavimi. 
\par 15 Tu ištesėjai savo tarnui Dovydui, mano tėvui, duotą pažadą. Tu kalbėjai savo lūpomis ir įvykdei tai savo rankomis šiandien. 
\par 16 Dabar, Viešpatie, Izraelio Dieve, įvykdyk tai, ką pažadėjai savo tarnui Dovydui, mano tėvui, sakydamas: ‘Tu nepritrūksi vyro Izraelio soste mano akivaizdoje, jei tavo sūnūs vaikščios pagal mano įstatymą, kaip tu vaikščiojai’. 
\par 17 Viešpatie, Izraelio Dieve, įvykdyk savo pažadą, kurį davei savo tarnui Dovydui. 
\par 18 Bet argi Dievas iš tiesų gyvens su žmonėmis žemėje? Dangus ir dangų dangūs nepajėgia Tavęs sutalpinti, tuo labiau šitie namai, kuriuos pastačiau. 
\par 19 Atsižvelk į savo tarno maldą ir prašymą, Viešpatie, mano Dieve, išklausyk šauksmą ir maldą, kuria Tavo tarnas meldžiasi Tavo akivaizdoje. 
\par 20 Dieną ir naktį tebūna atviros Tavo akys tiems namams ir tai vietai, apie kurią sakei, kad joje bus Tavo vardas. Išklausyk savo tarno maldą, kai jis melsis šioje vietoje. 
\par 21 Išgirsk savo tarno ir savo tautos maldavimus, kai jie melsis šitoje vietoje. Išgirsk danguje, kur Tu gyveni, ir atleisk. 
\par 22 Jei kas nusidės savo artimui ir ateis į šituos namus prie Tavo aukuro prisiekti, 
\par 23 išgirsk danguje ir teisk savo tarnus: nubausk kaltąjį pagal jo nusikaltimą ir išteisink teisųjį, atlygindamas jam pagal jo teisumą. 
\par 24 Jei Tavo tauta Izraelis bus nugalėta priešų dėl to, kad jie Tau nusikalto, ir jei jie atgailaus, išpažins Tavo vardą, melsis ir maldaus Tavęs šiuose namuose, 
\par 25 tai išgirsk danguje, atleisk savo tautai Izraeliui nuodėmę ir sugrąžink juos į žemę, kurią davei jiems ir jų tėvams. 
\par 26 Jei dangus bus uždarytas ir nebus lietaus dėl to, kad jie Tau nusidėjo, ir jei jie melsis šitoje vietoje, išpažins Tavo vardą ir nusigręš nuo savo nuodėmės, už kurią juos baudi, 
\par 27 išgirsk danguje ir atleisk savo tarnų, Izraelio tautos, nuodėmę; pamokyk juos eiti geru keliu ir duok lietaus kraštui, kurį davei jiems paveldėti. 
\par 28 Jei badas siaus krašte, kils maras, jei sausra, pelėsiai ir skėriai naikins derlių, jei priešai apsups žmones miestuose, užeis vargai ir ligos 
\par 29 ir jei atskiras žmogus ar visa tauta, pajutę vargą, melsis ištiesę rankas šitų namų link, 
\par 30 išgirsk danguje, atleisk jiems ir atlygink kiekvienam pagal jo kelius, kaip Tu matai jo širdyje, nes tik Tu pažįsti kiekvieno žmogaus širdį, 
\par 31 kad jie Tavęs bijotų ir vaikščiotų Tavo keliais, kol gyvens žemėje, kurią davei mūsų tėvams. 
\par 32 Jei svetimšalis, neizraelitas, ateis iš tolimo krašto dėl Tavo didingo vardo, Tavo galingos ir ištiestos rankos ir jei jis atėjęs melsis šiuose namuose, 
\par 33 išgirsk jį danguje ir įvykdyk, ko jis Tavęs prašys, kad visos žemės tautos pažintų Tavo vardą ir bijotų Tavęs kaip Tavo tauta Izraelis, ir patirtų, jog šitie namai, kuriuos pastačiau, vadinami Tavo vardu. 
\par 34 Jei Tavo tauta išeis kariauti su priešais, kur Tu juos pasiųsi, ir melsis atsigręžę į šį miestą, kurį Tu išsirinkai, ir šituos namus, kuriuos pastačiau Tavo vardui, 
\par 35 išgirsk danguje jų maldą ir prašymą ir apgink jų teises. 
\par 36 Jei jie Tau nusidės,­juk nėra žmogaus, kuris nenusidėtų,­ir Tu, užsirūstinęs ant jų, atiduosi juos priešui, kuris paims juos nelaisvėn ir išves į tolimą šalį, 
\par 37 ir jei jie, ten būdami, supras, atsivers ir Tavęs maldaus, sakydami: ‘Mes nusidėjome, elgėmės neteisingai ir padarėme piktadarystę’, 
\par 38 ir gręšis į Tave visa širdimi bei visa siela priešų šalyje, į kurią jie buvo išvesti, ir melsis atsigręžę į šalį, kurią davei jų tėvams, į miestą, kurį išsirinkai, ir šituos namus, kuriuos pastačiau Tavo vardui, 
\par 39 tai išklausyk danguje jų maldas bei maldavimus, apgink jų teises ir atleisk savo tautai, kuri Tau nusidėjo. 
\par 40 Mano Dieve, meldžiu, kad Tavo akys ir ausys būtų atviros maldai šioje vietoje. 
\par 41 Viešpatie Dieve, būk šitoje vietoje su savo galybės skrynia. Viešpatie Dieve, tegu Tavo kunigai būna apsirengę išgelbėjimu ir Tavo šventieji tesidžiaugia Tavo gerumu. 
\par 42 Viešpatie Dieve, nenusisuk nuo savo pateptojo, atsimink gailestingumą savo tarnui Dovydui”.



\chapter{7}


\par 1 Saliamonui baigus melstis, ugnis nusileido iš dangaus ir prarijo deginamąsias bei kitas aukas, ir Viešpaties šlovė pripildė namus. 
\par 2 Kunigai negalėjo įeiti į Viešpaties namus, nes Viešpaties šlovė buvo pripildžiusi Viešpaties namus. 
\par 3 Visi izraelitai, pamatę nusileidžiančią ugnį ir Viešpaties šlovę ant šventyklos, krito veidais į žemę, pagarbino ir šlovino Viešpatį, sakydami: “Jis yra geras ir Jo gailestingumas amžinas”. 
\par 4 Karalius ir visa tauta aukojo aukas Viešpaties akivaizdoje. 
\par 5 Saliamonas aukojo dvidešimt du tūkstančius jaučių ir šimtą dvidešimt tūkstančių avių. Taip karalius ir visa tauta pašventino Dievo namus. 
\par 6 Kunigai stovėjo savo vietose, taip pat ir levitai su Viešpaties muzikos instrumentais, kuriuos padarė karalius Dovydas Viešpačiui šlovinti už Jo amžiną gailestingumą, kai Dovydas šlovindavo per jų tarnavimą; kunigai trimitavo priešais juos ir visas Izraelis stovėjo. 
\par 7 Saliamonas pašventino ir kiemo vidurį, kuris buvo priešais Viešpaties namus, nes ten jis aukojo deginamąsias aukas ir padėkos aukų taukus, kadangi varinis aukuras, kurį Saliamonas buvo padaręs, nesutalpino deginamųjų bei duonos aukų ir taukų. 
\par 8 Saliamonas su visu Izraeliu, nuo Hamato apylinkių iki Egipto upės, šventė septynias dienas. 
\par 9 Aštuntą dieną jie padarė iškilmingą susirinkimą, nes jie šventino aukurą septynias dienas ir dar septynias dienas vyko šventė. 
\par 10 Septinto mėnesio dvidešimt trečią dieną Saliamonas paleido žmones, besidžiaugiančius dėl to gero, kurį Viešpats padarė Dovydui, Saliamonui ir visai Izraelio tautai, grįžti į savo palapines. 
\par 11 Saliamonui pabaigus statyti Viešpaties namus bei karaliaus namus, sėkmingai atlikus viską, ką jis buvo numatęs padaryti Viešpaties namuose ir savo namuose, 
\par 12 naktį jam pasirodė Viešpats ir tarė: “Aš išklausiau tavo maldą ir pasirinkau šitą vietą aukoms aukoti. 
\par 13 Jei nebus lietaus arba jei leisiu skėriams naikinti krašto augalus, arba siųsiu marą savo tautai 
\par 14 ir jei mano tauta, kuri vadinasi mano vardu, nusižemins, melsis, ieškos mano veido ir atsisakys savo blogų kelių, Aš išgirsiu danguje, atleisiu jos nuodėmes ir išgydysiu jų žemę. 
\par 15 Mano akys ir ausys bus atviros maldai šioje vietoje, 
\par 16 nes Aš išsirinkau ir pašventinau šituos namus, kad juose amžinai būtų mano vardas; mano akys ir širdis bus ten visada. 
\par 17 O jei tu vaikščiosi prieš mane, kaip tavo tėvas Dovydas vaikščiojo, ir vykdysi, ką tau įsakiau, bei laikysies mano nuostatų ir potvarkių, 
\par 18 tai įtvirtinsiu tavo karalystės sostą, kaip pažadėjau tavo tėvui Dovydui, sakydamas: ‘Tu nepritrūksi vyro, kuris valdytų Izraelį’. 
\par 19 Bet jei jūs nusigręšite ir nesilaikysite mano nuostatų ir įsakymų, kuriuos jums daviau, nuėję tarnausite kitiems dievams ir juos garbinsite, 
\par 20 tai Aš jus išrausiu su šaknimis iš žemės, kurią jums daviau, o šituos namus, kuriuos pašventinau savo vardui, pašalinsiu iš savo akių ir padarysiu juos patarle ir priežodžiu visose tautose. 
\par 21 Šitie aukšti namai kels nusistebėjimą kiekvienam praeiviui, ir jis sakys: ‘Kodėl Viešpats taip padarė šitai žemei ir šitiems namams?’ 
\par 22 Ir jam atsakys: ‘Kadangi jie paliko Viešpatį, savo tėvų Dievą, kuris juos išvedė iš Egipto šalies, ir sekė kitus dievus, juos garbino ir jiems tarnavo, tai Jis siuntė jiems šią nelaimę’ ”.



\chapter{8}


\par 1 Po dvidešimties metų, per kuriuos Saliamonas pastatė Viešpaties namus ir savo namus, 
\par 2 jis sutvirtino tuos miestus, kuriuos Hiramas atidavė Saliamonui, ir juose apgyvendino izraelitus. 
\par 3 Tuomet Saliamonas nuėjo į Hamat Cobą ir jį paėmė. 
\par 4 Jis pastatė Tadmorą dykumoje ir visus sandėlių miestus Hamate. 
\par 5 Jis taip pat sutvirtino Aukštutinį Bet Horoną ir Žemutinį Bet Horoną, apvesdamas juos sienomis, įstatydamas vartus ir skląsčius; 
\par 6 Baalatą ir visus sandėlių miestus, kurie priklausė Saliamonui, visus kovos vežimų ir raitelių miestus ir visa, ką Saliamonas sumanė statyti Jeruzalėje, Libane ir visame krašte, kurį valdė. 
\par 7 Visus hetitus, amoritus, perizus, hivus ir jebusiečius, kilusius ne iš Izraelio, 
\par 8 bet palikuonis likusiųjų krašte, kurių izraelitai nesunaikino, Saliamonas apdėjo prievolėmis, ir taip jie liko iki šios dienos. 
\par 9 Bet izraelitų Saliamonas neapkrovė darbais; jie buvo kareiviai, vadai ir kovos vežimų bei raitelių viršininkai. 
\par 10 Karalius Saliamonas turėjo du šimtus penkiasdešimt prižiūrėtojų, kurie valdė žmones. 
\par 11 Saliamonas perkėlė faraono dukterį iš Dovydo miesto į namus, kuriuos jis jai pastatė, nes sakė: “Mano žmona negyvens Izraelio karaliaus Dovydo namuose, nes jie šventi, kadangi juose buvo Viešpaties skrynia”. 
\par 12 Saliamonas aukojo Viešpačiui deginamąsias aukas ant Viešpaties aukuro, kurį jis pastatė priešais prieangį, 
\par 13 pagal Mozės įstatymo reikalavimus: kasdienes aukas, taip pat aukas sabatais, per jauną mėnulį ir per tris metines šventes­Neraugintos duonos, Savaičių ir Palapinių. 
\par 14 Jis paskyrė, kaip jo tėvas Dovydas buvo nustatęs, kunigų skyrius ir levitus jų pareigoms, kad jie tarnavimo metu giedotų ir patarnautų kunigams pagal kasdienes pareigas, ir vartininkus, kaip jie buvo paskirti skyriais kiekvieniems vartams; nes toks buvo Dievo vyro Dovydo įsakymas. 
\par 15 Kunigai ir levitai nenukrypo nuo karaliaus įsakymo jokiame dalyke nei prižiūrėdami turtus. 
\par 16 Taip buvo baigtas visas Saliamono darbas nuo Viešpaties namų pamatų dėjimo iki užbaigimo ir galutinio Viešpaties namų įrengimo. 
\par 17 Saliamonas nuvyko į Ecjon Geberą ir Elatą, prie jūros kranto Edomo šalyje. 
\par 18 Hiramas atsiuntė jam laivų su savo tarnais, patyrusiais jūrininkais; tie, nuplaukę su Saliamono tarnais į Ofyrą, iš ten pargabeno keturis šimtus penkiasdešimt talentų aukso karaliui Saliamonui.



\chapter{9}

\par 1 Šebos karalienė, išgirdusi apie Saliamoną ir norėdama jį išmėginti sunkiais klausimais, atvyko į Jeruzalę su labai didele palyda ir kupranugariais, nešančiais kvepalus, gausybę aukso ir brangiųjų akmenų. Atėjusi pas Saliamoną, ji kalbėjosi su juo apie viską, kas buvo jos širdyje. 
\par 2 Saliamonas atsakė jai į visus klausimus. Nebuvo nieko, ko Saliamonas nebūtų galėjęs jai atsakyti. 
\par 3 Šebos karalienė, pamačiusi Saliamono išmintį ir namus, kuriuos jis pasistatė, 
\par 4 jo stalo valgius, tarnų būstus, patarnautojų laikyseną bei apdarus, vyno pilstytojus bei jų apdarus, įėjimą į Viešpaties namus, nebegalėjo susilaikyti 
\par 5 ir tarė karaliui: “Ką girdėjau savo krašte apie tavo darbus ir išmintį, yra tiesa. 
\par 6 Aš netikėjau žodžiais, kol neatvykau ir savo akimis nepamačiau. Iš tikrųjų nė pusės apie tavo išmintį man nebuvo papasakota. Tu viršiji tai, ką apie tave girdėjau. 
\par 7 Laimingi tavo žmonės ir laimingi tavo tarnai, kurie nuolat yra priešais tave ir girdi tavo išmintį. 
\par 8 Palaimintas Viešpats, tavo Dievas, kuris pamėgo tave ir pasodino į savo sostą, kad būtum Viešpaties, savo Dievo, karaliumi. Kadangi Dievas myli Izraelį ir nori jį išlaikyti per amžius, Jis padarė tave jų karaliumi teismui ir teisingumui vykdyti”. 
\par 9 Šebos karalienė padovanojo karaliui šimtą dvidešimt talentų aukso, labai daug kvepalų ir brangiųjų akmenų; niekad nebuvo tokių kvepalų, kokius Šebos karalienė padovanojo karaliui Saliamonui. 
\par 10 Hiramo ir Saliamono tarnai pargabeno aukso iš Ofyro, raudonmedžio ir brangiųjų akmenų. 
\par 11 Iš raudonmedžio buvo padaryti laiptai Viešpaties namams, psalteriai ir arfos giesmininkams; anksčiau tokių niekas nebuvo matęs Judo žemėje. 
\par 12 Karalius Saliamonas davė Šebos karalienei viską, ko ji norėjo ir prašė. Jis jai dovanojo daugiau, negu ji jam atgabeno. Po to ji su savo palyda grįžo į savo kraštą. 
\par 13 Auksas, kurį kas metai atgabendavo Saliamonui, svėrė šešis šimtus šešiasdešimt šešis talentus, 
\par 14 neskaičiuojant to, ką atgabendavo karaliui Saliamonui prekybininkai ir keliaujantieji pirkliai, Arabijos karaliai ir krašto valdytojai. 
\par 15 Karalius Saliamonas padarė du šimtus didžiųjų skydų iš kalto aukso, sunaudodamas kiekvienam skydui po šešis šimtus šekelių aukso, 
\par 16 ir tris šimtus mažųjų skydų iš kalto aukso, sunaudodamas vienam po tris šimtus šekelių aukso. Karalius juos sudėjo namuose iš Libano medžio. 
\par 17 Karalius padarė didžiulį sostą iš dramblio kaulo ir padengė jį grynu auksu. 
\par 18 Sostas turėjo šešis laiptus ir auksinį pakojį. Abiejose sosto pusėse buvo rankoms atramos, du liūtai stovėjo šalia tų atramų. 
\par 19 Be to, dvylika liūtų stovėjo ant šešių laiptų, po vieną iš abiejų pusių. Nieko panašaus nebuvo padaryta jokioje karalystėje. 
\par 20 Visi karaliaus Saliamono geriamieji indai ir namų iš Libano medžio indai buvo gryno aukso. Nieko nebuvo iš sidabro, nes jis neturėjo vertės Saliamono laikais. 
\par 21 Karaliaus laivai plaukiojo į Taršišą kartu su Hiramo tarnais. Kas treji metai Taršišo laivai grįždavo su auksu, sidabru, dramblio kaulu, beždžionėmis ir povais. 
\par 22 Karalius Saliamonas pranoko visus žemės karalius savo turtais ir išmintimi. 
\par 23 Visi žemės karaliai siekė išvysti Saliamoną, norėdami išgirsti jo išmintį, kurią Dievas buvo įdėjęs į jo širdį. 
\par 24 Kiekvienas atgabendavo jam dovanų: sidabrinių ir auksinių daiktų, rūbų, ginklų, kvepalų, žirgų ir mulų; taip buvo kasmet. 
\par 25 Saliamonas turėjo keturis tūkstančius pastatų žirgams bei kovos vežimams ir dvylika tūkstančių raitelių, kuriuos jis paskirstė kovos vežimų miestuose ir pas save Jeruzalėje. 
\par 26 Jis buvo valdovas visų karalių nuo upės iki filistinų krašto ir Egipto sienos. 
\par 27 Karalius padarė, kad sidabro Jeruzalėje buvo kaip akmenų, o kedrų tiek, kiek figmedžių, kurie gausiai auga slėniuose. 
\par 28 Saliamonas pirkdavo žirgų iš Egipto ir kitų šalių. 
\par 29 Kiti Saliamono darbai nuo pradžios iki galo yra surašyti pranašo Natano knygoje, Ahijos iš Šilojo pranašystėje ir regėtojo Idojo regėjimuose apie Nebato sūnų Jeroboamą. 
\par 30 Saliamonas karaliavo Jeruzalėje visam Izraeliui keturiasdešimt metų. 
\par 31 Saliamonas užmigo prie savo tėvų ir buvo palaidotas savo tėvo Dovydo mieste; jo sūnus Roboamas pradėjo karaliauti jo vietoje.



\chapter{10}


\par 1 Roboamas nuėjo į Sichemą, kur buvo susirinkę visi izraelitai paskelbti jį karaliumi. 
\par 2 Nebato sūnus Jeroboamas, kuris buvo pabėgęs nuo karaliaus Saliamono į Egiptą, išgirdęs apie tai, sugrįžo iš Egipto. 
\par 3 Jie pasiuntė ir pakvietė jį. Jeroboamas ir visas Izraelis atėjo ir kalbėjo Roboamui: 
\par 4 “Tavo tėvas uždėjo mums sunkų jungą. Dabar palengvink savo tėvo mums uždėtą naštą, tai mes tau tarnausime”. 
\par 5 Jis jiems tarė: “Po trijų dienų sugrįžkite pas mane”. Ir žmonės nuėjo. 
\par 6 Karalius Roboamas tarėsi su senesniaisiais, kurie buvo priešais jo tėvą Saliamoną, kai jis dar buvo gyvas: “Patarkite man, ką atsakyti tautai”. 
\par 7 Tie jam kalbėjo: “Jei būsi malonus šitiems žmonėms, patiksi jiems ir kalbėsi švelniais žodžiais, jie visados bus tavo tarnai”. 
\par 8 Bet jis atmetė senesniųjų duotą patarimą ir tarėsi su jaunesniaisiais, kurie užaugo kartu su juo ir stovėjo priešais jį. 
\par 9 Jis jiems tarė: “Ką jūs patariate man atsakyti šitiems žmonėms, kurie man kalbėjo: ‘Palengvink jungą, kurį mums uždėjo tavo tėvas’?” 
\par 10 Jaunieji, užaugę kartu su juo, jam kalbėjo: “Šitiems žmonėms, kurie tau sakė: ‘Tavo tėvas padarė mūsų jungą sunkų, o tu mums jį palengvink’, atsakyk taip: ‘Mano mažasis pirštas storesnis už mano tėvo strėnas. 
\par 11 Mano tėvas jums uždėjo sunkų jungą, bet aš jį jums dar pasunkinsiu. Mano tėvas jus plakė botagais, o aš plaksiu jus dygliuotais rimbais’ ”. 
\par 12 Kai Jeroboamas ir visa tauta trečią dieną atėjo pas Roboamą, kaip karalius buvo sakęs: “Sugrįžkite pas mane trečią dieną”, 
\par 13 karalius, atmetęs senesniųjų patarimą, kalbėjo tautai griežtai, 
\par 14 kaip jaunesnieji jam buvo patarę: “Mano tėvas uždėjo jums sunkų jungą, o aš jį jums dar pasunkinsiu. Mano tėvas jus plakė botagais, o aš plaksiu jus dygliuotais rimbais”. 
\par 15 Karalius nepaklausė tautos, nes tai buvo nuo Dievo, kad Viešpats ištesėtų savo žodį, kurį Jis kalbėjo per Ahiją iš Šilojo Nebato sūnui Jeroboamui. 
\par 16 Izraelitai, pamatę, kad karalius nenori jų išklausyti, atsakė jam: “Mes neturime dalies Dovyde nė paveldėjimo Jesės sūnuje. Visi Izraelio vyrai, į savo palapines! Dovydai, rūpinkis savo namais”. Visi izraelitai nuėjo į savo palapines. 
\par 17 Izraelitams, gyvenantiems Judo miestuose, karaliavo Roboamas. 
\par 18 Karalius Roboamas pasiuntė Adoramą, mokesčių rinkėją, pas izraelitus, bet jie užmušė jį akmenimis. Karalius Roboamas skubiai įšoko į vežimą ir pabėgo į Jeruzalę. 
\par 19 Taip Izraelis atsiskyrė nuo Dovydo namų iki šios dienos.



\chapter{11}

\par 1 Roboamas, sugrįžęs į Jeruzalę, surinko iš Judo ir Benjamino namų šimtą aštuoniasdešimt tūkstančių rinktinių karių karui su Izraeliu, kad sugrąžintų karalystę Roboamui. 
\par 2 Bet Viešpaties žodis atėjo Dievo vyrui Šemajai: 
\par 3 “Kalbėk Saliamono sūnui Roboamui, Judo karaliui, ir visiems izraelitams Jude ir Benjamine ir sakyk: 
\par 4 ‘Taip sako Viešpats: ‘Neikite ir nekariaukite su savo broliais! Kiekvienas grįžkite į savo namus, nes tai atėjo iš manęs’ ”. Jie pakluso Viešpaties žodžiams ir sugrįžo, atsisakę eiti prieš Jeroboamą. 
\par 5 Roboamas gyveno Jeruzalėje; jis sustiprino šiuos Judo miestus: 
\par 6 Betliejų, Etamą, Tekoją, 
\par 7 Bet Cūrą, Sochoją, Adulamą, 
\par 8 Gatą, Marešą, Zifą, 
\par 9 Adorajimą, Lachišą, Azeką, 
\par 10 Corą, Ajaloną ir Hebroną, kurie yra Jude ir Benjamine. 
\par 11 Sustiprinęs tvirtoves, jis paskyrė joms viršininkus ir įrengė jose maisto, aliejaus ir vyno sandėlius. 
\par 12 Kiekviename mieste jis laikė skydų bei iečių ir labai juos sutvirtino. Taip Judas ir Benjaminas pasiliko su juo. 
\par 13 Kunigai ir levitai, gyvenantys Izraelyje, iš viso krašto susirinko į Judą. 
\par 14 Levitai, palikę savo priemiesčius ir nuosavybę, atėjo į Judą ir Jeruzalę, nes Jeroboamas ir jo sūnūs pašalino juos iš Viešpaties kunigų tarnystės 
\par 15 ir paskyrė kunigus aukoti aukštumose demonams ir veršiams, kuriuos jis padarė. 
\par 16 Paskui juos ir visi, kurie nukreipė savo širdis, kad ieškotų Viešpaties, Izraelio Dievo, iš visų Izraelio giminių atėjo į Jeruzalę, norėdami aukoti Viešpačiui, savo tėvų Dievui. 
\par 17 Jie sustiprino Judo karalystę ir padarė Saliamono sūnų Roboamą stiprų trejus metus; nes tuos metus jie vaikščiojo Dovydo ir Saliamono keliais. 
\par 18 Roboamas vedė Mahalatą, Dovydo sūnaus Jerimoto ir Abihailės, kurios tėvas buvo Eliabas, Jesės sūnus, dukterį. 
\par 19 Mahalata pagimdė jam Jeušą, Šemariją ir Zahamą. 
\par 20 Po jos Roboamas dar vedė Abšalomo dukterį Maaką, kuri pagimdė Abiją, Atają, Zizą ir Šelomitą. 
\par 21 Roboamas mylėjo Abšalomo dukterį Maaką labiau už visas kitas savo žmonas ir suguloves. Jis turėjo aštuoniolika žmonų ir šešiasdešimt sugulovių, dvidešimt aštuonis sūnus ir šešiasdešimt dukterų. 
\par 22 Roboamas paskyrė Maakos sūnų Abiją brolių vyresniuoju, nes norėjo padaryti jį karaliumi. 
\par 23 Jis elgėsi išmintingai, išskirstydamas visus savo sūnus po Judą ir Benjamino žemių sustiprintus miestus, teikdamas jiems gausiai maisto ir parūpindamas daug žmonų.



\chapter{12}

\par 1 Kai Roboamas įtvirtino karalystę ir sustiprėjo pats, jis ir su juo visas Izraelis apleido Viešpaties įstatymą. 
\par 2 Penktaisiais karaliaus Roboamo valdymo metais Egipto karalius Šešonkas puolė Jeruzalę, nes jie nusikalto Viešpačiui. 
\par 3 Jis atėjo su tūkstančiu dviem šimtais kovos vežimų ir šešiasdešimt tūkstančių raitelių, o žmonių, kurie atėjo su juo iš Egipto, Libijos, Sukimo ir Etiopijos, buvo nesuskaitoma daugybė. 
\par 4 Jis paėmė Judo sutvirtintus miestus ir pasiekė Jeruzalę. 
\par 5 Pranašas Šemaja, atėjęs pas Roboamą ir Judo kunigaikščius, kurie dėl Šešonko susirinko Jeruzalėje, kalbėjo jiems: “Taip sako Viešpats: ‘Jūs palikote mane, todėl Aš jus paliksiu Šešonko rankose’ ”. 
\par 6 Izraelio kunigaikščiai ir karalius nusižemino ir tarė: “Teisus yra Viešpats!” 
\par 7 Viešpats, matydamas, kad jie nusižemino, kalbėjo Šemajai: “Kadangi jie nusižemino, Aš jų nesunaikinsiu, bet greitai išgelbėsiu. Mano rūstybė neišsilies ant Jeruzalės per Šešonką. 
\par 8 Bet jie taps jo tarnais, kad žinotų, kuo skiriasi tarnavimas man ir tarnavimas svetimai karalystei”. 
\par 9 Egipto karalius Šešonkas užpuolė Jeruzalę ir, paėmęs Viešpaties namų ir karaliaus rūmų turtus, viską išvežė. Jis paėmė ir auksinius skydus, kuriuos Saliamonas buvo padaręs. 
\par 10 Karalius Roboamas padirbdino jų vietoje varinių skydų ir juos pavedė sargybos viršininkams, kurie buvo prie karaliaus rūmų. 
\par 11 Karaliui einant į Viešpaties namus, sargybiniai juos nešdavo ir po to vėl juos padėdavo į sargybinių patalpą. 
\par 12 Kadangi jis nusižemino, Viešpaties rūstybė nusisuko nuo jo ir nesunaikino jo; ir Judui taip pat gerai sekėsi. 
\par 13 Karalius Roboamas įsitvirtino Jeruzalėje ir karaliavo. Pradėdamas karaliauti, Roboamas buvo keturiasdešimt vienerių metų amžiaus ir karaliavo septyniolika metų Jeruzalėje, kurią Viešpats išsirinko iš visų Izraelio giminių, kad ten būtų Jo vardas. Jo motina buvo amonitė, vardu Naama. 
\par 14 Jis darė pikta, nes neparuošė savo širdies, kad ieškotų Viešpaties. 
\par 15 Roboamo darbai nuo pradžios iki galo yra surašyti pranašo Šemajos ir regėtojo Idojo raštuose. Karas tarp Roboamo ir Jeroboamo tęsėsi per visas jų dienas. 
\par 16 Roboamas užmigo prie savo tėvų ir buvo palaidotas Dovydo mieste; jo sūnus Abija karaliavo jo vietoje.



\chapter{13}

\par 1 Aštuonioliktaisiais karaliaus Jeroboamo metais Abija pradėjo karaliauti Jude. 
\par 2 Trejus metus jis karaliavo Jeruzalėje. Jo motina buvo vardu Mikaja, duktė Urielio iš Gibėjos. Kilo karas tarp Abijos ir Jeroboamo. 
\par 3 Abija išėjo į kovą su keturių šimtų tūkstančių rinktinių vyrų kariuomene, o Jeroboamas išsirikiavo prieš jį su aštuoniais šimtais tūkstančių rinktinių karių. 
\par 4 Abija, atsistojęs ant Cemaraimo kalno Efraimo aukštumose, tarė: “Tu, Jeroboamai, ir visas Izraeli, pasiklausykite manęs! 
\par 5 Argi jūs nežinote, kad Viešpats, Izraelio Dievas, amžiams atidavė karališkąją Izraelio valdžią Dovydui ir jo sūnums druskos sandora? 
\par 6 Bet Nebato sūnus Jeroboamas, Dovydo sūnaus Saliamono tarnas, sukilo prieš savo valdovą. 
\par 7 Pas jį susirinko netikę žmonės, Belialo vaikai, ir įsitvirtino prieš Saliamono sūnų Roboamą, kai jis buvo jaunas, nepatyręs ir neatsilaikė prieš juos. 
\par 8 Dabar jūs manote galėsią atsilaikyti prieš Viešpaties karalystę, esančią Dovydo sūnų rankose, kadangi jūsų yra didelė daugybė ir su jumis yra auksiniai veršiai, kuriuos jums padirbdino Jeroboamas, kad būtų jūsų dievai. 
\par 9 Jūs išvarėte Viešpaties kunigus, Aarono sūnus, ir levitus bei pasidarėte kunigų kaip kitų kraštų tautos. Kas tik ateina pasišvęsti su jaučiu ir septyniais avinais, tampa kunigu tų, kurie nėra dievai. 
\par 10 Bet mūsų Dievas yra Viešpats, mes Jo nepalikome. Viešpačiui tarnauja kunigai, Aarono sūnūs, taip pat levitai atlieka savo tarnystę. 
\par 11 Jie kas rytą ir kas vakarą aukoja Viešpačiui deginamąsias aukas ir smilko smilkalus, padeda padėtinę duoną ant auksinio stalo ir kas vakarą uždega lempas auksinėje žvakidėje. Mes laikomės Viešpaties, mūsų Dievo, nurodymų, bet jūs Jį palikote. 
\par 12 Pats Dievas veda mus, yra su mumis ir Jo kunigai su trimitais, kad trimituotų prieš jus. Izraelitai, nekovokite su Viešpačiu, savo tėvų Dievu, nes neturėsite sėkmės”. 
\par 13 Jeroboamas pasiuntė pasalą, kad užeitų jiems iš užnugario. Jo kariai buvo priešais Judą, o pasala­už jų. 
\par 14 Judas, pamatęs, kad kova bus iš priekio ir iš užpakalio, šaukėsi Viešpaties, o kunigai trimitavo. 
\par 15 Judo vyrai sušuko, ir kai jie šaukė, Dievas sumušė Jeroboamą ir Izraelį priešais Abiją ir Judą. 
\par 16 Izraelitai bėgo nuo Judo, ir Dievas atidavė juos į šių rankas. 
\par 17 Abija ir jo žmonės smarkiai sumušė juos, ir žuvo izraelitų penki šimtai tūkstančių rinktinių vyrų. 
\par 18 Taip izraelitai pralaimėjo tą kartą, o Judo vaikai nugalėjo, nes jie pasitikėjo Viešpačiu, savo tėvų Dievu. 
\par 19 Abija persekiojo Jeroboamą ir atėmė iš jo šiuos miestus: Betelį, Ješaną ir Efroną su jų kaimais. 
\par 20 Jeroboamas nebeatgavo savo galios Abijos dienomis. Viešpats ištiko jį, ir jis mirė. 
\par 21 Bet Abija sustiprėjo. Jis vedė keturiolika žmonų ir turėjo dvidešimt du sūnus ir šešiolika dukterų. 
\par 22 Visi kiti Abijos darbai, jo keliai ir jo kalbos surašyti pranašo Idojo knygoje.



\chapter{14}


\par 1 Abija užmigo prie savo tėvų ir jį palaidojo Dovydo mieste. Jo sūnus Asa karaliavo jo vietoje. Jo dienomis krašte buvo ramu dešimt metų. 
\par 2 Asa darė tai, kas gera ir teisinga Viešpaties, jo Dievo, akyse. 
\par 3 Jis pašalino svetimų dievų aukurus aukštumose, sudaužė atvaizdus, iškirto giraites 
\par 4 ir įsakė Judo žmonėms ieškoti Viešpaties, jų tėvų Dievo, ir vykdyti Jo įstatymus bei įsakymus. 
\par 5 Jis pašalino visuose Judo miestuose aukštumas ir atvaizdus, ir jam valdant buvo ramu. 
\par 6 Jis statė sutvirtintus miestus Jude, nes kraštas ilsėjosi, ir su niekuo nekariavo tais metais, kadangi Viešpats suteikė jam poilsį. 
\par 7 Asa kalbėjo Judui: “Pastatykime šiuos miestus, aptverkime juos sienomis su bokštais, vartais ir skląsčiais, nes kraštas priklauso mums. Kadangi mes ieškojome Viešpaties, mūsų Dievo, Jis suteikė mums ramybę iš visų pusių”. Jie statė, ir jiems sekėsi. 
\par 8 Asos kariuomenėje buvo trys šimtai tūkstančių Judo vyrų, ginkluotų skydais ir ietimis, ir du šimtai aštuoniasdešimt tūkstančių Benjamino vyrų, ginkluotų skydais ir lankais. Jie visi buvo narsūs kariai. 
\par 9 Prieš juos išėjo etiopas Zerachas su tūkstančiu tūkstančių kareivių, trimis šimtais kovos vežimų ir pasiekė Marešą. 
\par 10 Asa išėjo prieš jį ir išsirikiavo kautynėms Cefatos slėnyje prie Marešos. 
\par 11 Asa šaukėsi Viešpaties, savo Dievo: “Viešpatie, Tu gali padėti ir galingam, ir bejėgiui. Padėk mums, Viešpatie, mūsų Dieve, nes mes šliejamės prie Tavęs ir Tavo vardu einame prieš šitą daugybę! Viešpatie, Tu esi mūsų Dievas, tenenugali Tavęs žmogus”. 
\par 12 Viešpats sumušė etiopus priešais Asą ir Judą, ir etiopai bėgo. 
\par 13 Asa su žmonėmis juos vijosi iki Gerara. Etiopai buvo nugalėti ir nebeatsigavo. Jie buvo sumušti Viešpaties ir Jo kariuomenės. Asos kariai paėmė labai daug grobio. 
\par 14 Jie užėmė visus miestus Geraros apylinkėje, nes juos buvo apėmusi Viešpaties baimė. Iš visų miestų jie išsigabeno daug grobio. 
\par 15 Be to, jie sugriovė gyvulių pastoges ir, išsivarę daugybę avių bei kupranugarių, sugrįžo į Jeruzalę.



\chapter{15}

\par 1 Dievo Dvasia nužengė ant Odedo sūnaus Azarijo. 
\par 2 Jis išėjo pasitikti Asos ir jam tarė: “Karaliau Asa, visas Judai ir Benjaminai, paklausykite manęs! Viešpats yra su jumis, kai jūs esate su Juo, ir jei Jo ieškosite, surasite Jį. Bet jei Jį apleisite, Jis jus apleis. 
\par 3 Ilgą laiką Izraelis gyveno be tikro Dievo, be pamokančio kunigo ir be įstatymo. 
\par 4 Bet kai jie savo varge atsigręžė į Viešpatį, Izraelio Dievą, ir ieškojo Jo, Jis leidosi jų surandamas. 
\par 5 Anais laikais nebuvo saugu įeinančiam ir išeinančiam, nes didelis sumišimas visose šalyse vargino gyventojus. 
\par 6 Tauta naikino tautą ir miestas miestą, nes Dievas padarė sumaištį tarp jų visomis nelaimėmis. 
\par 7 O jūs būkite stiprūs ir tenepailsta jūsų rankos. Jums bus atlyginta už jūsų darbus”. 
\par 8 Asa, išgirdęs šituos pranašystės žodžius, įsidrąsino ir pašalino pasibjaurėtinus stabus iš Judo ir Benjamino krašto ir iš miestų, kuriuos jis paėmė Efraimo aukštumose. Jis atnaujino Viešpaties aukurą prie Viešpaties šventyklos. 
\par 9 Jis surinko visus Judo ir Benjamino gyventojus bei ateivius, kurie apsigyveno pas juos, iš Efraimo, Manaso ir Simeono; daugelis iš Izraelio perbėgo pas Asą, pamatę, kad Viešpats, jo Dievas, yra su juo. 
\par 10 Jie susirinko į Jeruzalę penkioliktais Asos karaliavimo metais trečią mėnesį. 
\par 11 Tą dieną jie aukojo Viešpačiui iš paimto grobio septynis šimtus jaučių ir septynis tūkstančius avių. 
\par 12 Jie padarė sandorą, kad ieškos Viešpaties, savo tėvų Dievo, visa širdimi ir visa siela. 
\par 13 Kas neieškos Viešpaties, Izraelio Dievo, bus baudžiamas mirtimi, nepaisant ar jis mažas, ar didelis, ar vyras, ar moteris. 
\par 14 Jie prisiekė Viešpačiui garsiu balsu su šauksmais, trimitams ir ragams aidint. 
\par 15 Visas Judas džiaugėsi priesaika, nes jie prisiekė iš visų savo širdžių ir ieškojo Jo su dideliu troškimu. Ir Jis leidosi jų surandamas, ir Viešpats jiems suteikė ramybę iš visų pusių. 
\par 16 Karalius Asa pašalino iš karalienės vietos net savo motiną Maaką, nes ji buvo padariusi stabą giraitėje. Asa sukapojo tą stabą ir sutrupinęs sudegino Kedrono slėnyje. 
\par 17 Bet aukštumos nebuvo sunaikintos Izraelyje. Tačiau Asos širdis buvo tobula per visas jo dienas. 
\par 18 Jis atnešė į Dievo namus savo tėvo ir savo paskirtas dovanas: sidabrą, auksą ir indus. 
\par 19 Karo nebuvo iki trisdešimt penktų Asos karaliavimo metų.



\chapter{16}


\par 1 Trisdešimt šeštaisiais Asos karaliavimo metais Izraelio karalius Baša užpuolė Judą ir statė Ramą, kad niekam neleistų įeiti ar išeiti iš Asos, Judo karaliaus. 
\par 2 Asa, paėmęs sidabrą ir auksą iš Viešpaties namų ir karaliaus namų, pasiuntė Sirijos karaliui Ben Hadadui į Damaską, sakydamas: 
\par 3 “Padarykime sąjungą tarp manęs ir tavęs, kaip buvo tarp mūsų tėvų. Siunčiu tau sidabro ir aukso; sulaužyk sąjungą su Izraelio karaliumi Baša, kad jis atsitrauktų nuo manęs”. 
\par 4 Ben Hadadas paklausė karaliaus Asos ir pasiuntė savo kariuomenės vadus prieš Izraelio miestus. Jie užėmė Ijoną, Daną, Abel Maimą ir visus Neftalio sandėlių miestus. 
\par 5 Izraelio karalius Baša, tai išgirdęs, liovėsi statęs Ramą ir pasitraukė. 
\par 6 Karalius Asa, surinkęs visą Judą, paėmė akmenis ir rąstus, kuriuos Baša naudojo statybai, ir jais sutvirtino Gebą ir Micpą. 
\par 7 Tuo metu regėtojas Hananis, atėjęs pas Judo karalių Asą, jam tarė: “Kadangi tu pasitikėjai Sirijos karaliumi, o ne Viešpačiu, savo Dievu, Sirijos karaliaus kariuomenė ištrūko iš tavo rankų. 
\par 8 Ar etiopai ir libiai neturėjo didžiulės kariuomenės, kovos vežimų ir raitelių? Kadangi pasitikėjai Viešpačiu, tai Jis juos atidavė į tavo rankas. 
\par 9 Nes Viešpaties akys stebi visą žemę, kad parodytų savo galią dėl tų, kurių širdis yra tobula prieš Jį. Tu pasielgei neišmintingai, todėl nuo šiol turėsi daug karų”. 
\par 10 Asa, užsirūstinęs ant regėtojo, įmetė jį į kalėjimą, nes labai supyko ant jo dėl to dalyko. Tuo pačiu metu Asa išnaudojo ir kitus savo tautos žmones. 
\par 11 Visi Asos darbai yra surašyti Judo ir Izraelio karalių knygoje. 
\par 12 Trisdešimt devintaisiais karaliavimo metais Asa susirgo sunkia kojų liga. Tačiau jis ir sirgdamas neieškojo Viešpaties, bet kreipėsi į gydytojus. 
\par 13 Asa užmigo prie savo tėvų ir mirė keturiasdešimt pirmaisiais savo karaliavimo metais. 
\par 14 Jis buvo palaidotas savame kape, kurį buvo pasidaręs Dovydo mieste. Jo kūnas buvo balzamuotas kvepalais ir įvairiausių rūšių tepalais. Jo garbei buvo sukurtas didelis laužas.



\chapter{17}


\par 1 Jo sūnus Juozapatas pradėjo karaliauti jo vietoje ir sustiprėjo prieš Izraelį. 
\par 2 Jis pasiuntė kariuomenės būrius į visus sutvirtintus Judo miestus ir paskyrė įgulas Judo krašte bei Efraimo miestuose, kuriuos buvo paėmęs jo tėvas Asa. 
\par 3 Viešpats buvo su Juozapatu, nes jis vaikščiojo ankstesniais savo tėvo Dovydo keliais ir negarbino Baalo. 
\par 4 Jis ieškojo savo tėvų Dievo ir vykdė Jo įsakymus, ir nesekė Izraelio pavyzdžiu. 
\par 5 Viešpats įtvirtino karalystę jo rankoje, ir visi Judo gyventojai nešė Juozapatui dovanų. Jis turėjo daug turtų ir buvo gerbiamas. 
\par 6 Jo širdis buvo išaukštinta Viešpaties keliuose ir jis sunaikino aukštumas bei giraites Jude. 
\par 7 Trečiaisiais karaliavimo metais jis siuntė į Judo miestus mokyti savo kunigaikščius: Ben Hailą, Abdiją, Zachariją, Netanelį ir Mikają. 
\par 8 Su jais buvo pasiųsti levitai: Šemajas, Netanijas, Zebadijas, Asaelis, Šemiramotas, Jehonatanas, Adonijas, Tobijas, Tob Adonija ir du kunigai­Elišama bei Jehoramas. 
\par 9 Jie mokė Jude žmones iš Viešpaties įstatymų knygos visuose miestuose. 
\par 10 Viešpaties baimė apėmė visus Judo kaimynus taip, kad jie nedrįso kariauti su Juozapatu. 
\par 11 Kai kurie filistinai nešė Juozapatui dovanų ir sidabrą kaip duoklę. O arabai atvedė jam septynis tūkstančius septynis šimtus avinų ir tokį pat skaičių ožių. 
\par 12 Juozapato galybė augo. Jis pastatė Jude pilis ir sandėlių miestus, 
\par 13 kuriuose jis laikė daug atsargų. Jeruzalėje buvo narsūs kariai. 
\par 14 Štai jų skaičius pagal jų tėvų namus. Iš Judo buvo šie tūkstantininkai. Adna­vadas trijų šimtų tūkstančių rinktinių karių. 
\par 15 Po jo buvo vadas Johananas su dviem šimtais aštuoniasdešimt tūkstančių karių. 
\par 16 Šalia jo Zichrio sūnus Amasija, savanoriškai pasišventęs Viešpačiui; su juo buvo du šimtai tūkstančių rinktinių karių. 
\par 17 Iš Benjamino­karžygys Eljada su dviem šimtais tūkstančių karių, ginkluotų lankais ir skydais. 
\par 18 Po jo­Jehozabadas su šimtu aštuoniasdešimt tūkstančių ginkluotų karių. 
\par 19 Šitie tarnavo karaliui, neskaičiuojant tų, kurie buvo karaliaus paskirstyti sutvirtintuose miestuose visame Jude.



\chapter{18}

\par 1 Juozapatas buvo labai turtingas ir gerbiamas; jis susigiminiavo su Ahabu. 
\par 2 Keleriems metams praėjus, jis nuvyko pas Ahabą į Samariją. Ahabas jam ir jo žmonėms papjovė daug avių bei galvijų ir įtikinėjo jį užpulti Ramot Gileadą. 
\par 3 Izraelio karalius Ahabas klausė Judo karalių Juozapatą: “Ar tu eisi su manimi į Ramot Gileadą?” Tas jam atsakė: “Kaip tu, taip ir aš; mano tauta, kaip ir tavo tauta; aš eisiu su tavimi į karą”. 
\par 4 Juozapatas sakė Izraelio karaliui: “Sužinok, ką Viešpats sako”. 
\par 5 Izraelio karalius, sušaukęs keturis šimtus pranašų, klausė: “Ar man eiti į Ramot Gileadą kariauti, ar ne?” Jie atsakė: “Eik! Dievas jį atiduos į karaliaus rankas”. 
\par 6 Juozapatas klausė: “Ar čia nėra Viešpaties pranašo, kad jo galėtume pasiklausti?” 
\par 7 Izraelio karalius atsakė Juozapatui: “Yra vienas, per kurį būtų galima paklausti Viešpaties, bet aš jo nekenčiu, nes jis niekuomet nepranašauja apie mane gerai, visuomet tik blogai; tai Imlos sūnus Michėjas”. Juozapatas atsakė: “Nekalbėk taip, karaliau”. 
\par 8 Izraelio karalius pasišaukė vieną valdininką ir įsakė jam skubiai atvesti Imlos sūnų Michėją. 
\par 9 Izraelio karalius Ahabas ir Judo karalius Juozapatas, apsivilkę karališkais drabužiais, sėdėjo savo sostuose aikštėje prie Samarijos vartų, ir visi pranašai pranašavo priešais juos. 
\par 10 Kenaanos sūnus Zedekijas pasidarė geležinius ragus ir sakė: “Taip sako Viešpats: ‘Jais badysi sirus, kol juos pribaigsi’ ”. 
\par 11 Ir visi pranašai taip pranašavo: “Eik į Ramot Gileadą ir laimėk! Viešpats jį atiduos į karaliaus rankas”. 
\par 12 Pasiuntinys, nuėjęs pakviesti Michėjo, jam sakė: “Štai pranašų žodžiai vienbalsiai skelbia gerą žinią karaliui. Tebūna ir tavo žodis panašus į jų; kalbėk tai, kas gera”. 
\par 13 Michėjas atsakė: “Kaip gyvas Viešpats, ką mano Dievas man sakys, tą kalbėsiu”. 
\par 14 Jam atėjus pas karalių, šis paklausė: “Michėjau, ar mums eiti kariauti į Ramot Gileadą, ar ne?” Jis atsakė: “Eikite ir laimėkite! Jie bus atiduoti į jūsų rankas”. 
\par 15 Karalius jam atsakė: “Kiek kartų reikės tave saikdinti, kad man nieko kito nekalbėtum, tik tiesą Viešpaties vardu”. 
\par 16 Tada Michėjas atsakė: “Mačiau visą Izraelį, išsklaidytą kalnuose kaip avis be piemens. O Viešpats tarė: ‘Šitie neturi valdovo, tegul kiekvienas grįžta ramybėje į savo namus’ ”. 
\par 17 Izraelio karalius tarė Juozapatui: “Ar tau nesakiau, kad jis nepranašauja apie mane gera, tik pikta?” 
\par 18 Michėjas tęsė: “Klausykitės Viešpaties žodžio! Mačiau Viešpatį, sėdintį savo soste, ir visą dangaus kareiviją, stovinčią Jo dešinėje ir kairėje. 
\par 19 Viešpats klausė: ‘Kas suvedžios Izraelio karalių Ahabą, kad jis eitų ir žūtų Ramot Gileade?’ Vienas sakė taip, kitas­kitaip. 
\par 20 Pagaliau išėjo dvasia, kuri, atsistojusi Viešpaties akivaizdoje, tarė: ‘Aš jį suklaidinsiu’. Viešpats paklausė: ‘Kaip?’ 
\par 21 Ji atsakė: ‘Aš eisiu ir būsiu melo dvasia visų jo pranašų lūpose’. Viešpats tarė: ‘Tau pavyks jį suvedžioti. Eik ir daryk taip’. 
\par 22 Taigi Viešpats įdėjo melo dvasią į visų tavo pranašų lūpas, nes Viešpats kalbėjo prieš tave pikta”. 
\par 23 Tada priėjo Kenaanos sūnus Zedekijas, trenkė Michėjui į veidą ir tarė: “Kuriuo keliu Viešpaties Dvasia pasitraukė nuo manęs, kad kalbėtų tau?” 
\par 24 Michėjas atsakė: “Tai pamatysi tą dieną, kai bėgsi slėptis į vidinį kambarį”. 
\par 25 Izraelio karalius įsakė: “Suimkite Michėją, nuveskite jį pas miesto valdytoją Amoną ir pas karaliaus sūnų Jehoašą 
\par 26 ir pasakykite: ‘Taip sako karalius: ‘Įmeskite jį į kalėjimą ir maitinkite sielvarto duona bei vandeniu, kol aš ramybėje sugrįšiu’ ”. 
\par 27 Michėjas atsakė: “Jei tu sugrįši ramybėje, tai Viešpats nekalbėjo per mane”. Ir jis sakė: “Klausykite, visi žmonės!” 
\par 28 Izraelio ir Judo karaliai išėjo į Ramot Gileadą. 
\par 29 Izraelio karalius tarė Juozapatui: “Aš persirengęs eisiu į mūšį, o tu apsirenk savo drabužiais”. Izraelio karalius persirengė, ir jie išėjo į mūšį. 
\par 30 Sirijos karalius buvo įsakęs savo kovos vežimų viršininkams: “Nekovokite su nieku kitu, tik su Izraelio karaliumi”. 
\par 31 Kovos vežimų viršininkai, pamatę Juozapatą, sakė: “Jis yra Izraelio karalius”. Jie apsupo jį, norėdami kautis. Bet Juozapatas šaukė, ir Viešpats jam padėjo, ir Dievas nukreipė priešus nuo jo. 
\par 32 Kovos vežimų viršininkai, supratę, kad jis ne Izraelio karalius, liovėsi jį puolę. 
\par 33 Vienas vyras netaikydamas įtempė lanką ir iššovė; strėlė pataikė Izraelio karaliui tarp šarvų. Tada jis tarė savo vežikui: “Apsisuk ir išvežk mane iš kovos lauko, nes esu sužeistas”. 
\par 34 Tą dieną mūšis sustiprėjo, ir Izraelio karalius stovėjo vežime prieš sirus iki vakaro. Saulei leidžiantis, jis mirė.



\chapter{19}


\par 1 Judo karalius Juozapatas sugrįžo ramybėje į savo namus Jeruzalėje. 
\par 2 Hananio sūnus Jehuvas, regėtojas, išėjo jo pasitikti ir kalbėjo karaliui Juozapatui: “Ar turėtum padėti bedieviui ir mylėti tuos, kurie nekenčia Viešpaties? Dėl to užsitraukei Viešpaties rūstybę. 
\par 3 Tačiau ir gerų dalykų rasta tavyje, nes tu išnaikinai krašte giraites ir paruošei savo širdį ieškoti Dievo”. 
\par 4 Juozapatas gyveno Jeruzalėje, tačiau jis ėjo per tautą nuo Beer Šebos iki Efraimo aukštumų, grąžindamas žmones prie Viešpaties, savo tėvų Dievo. 
\par 5 Jis paskyrė krašte teisėjus kiekvienam sutvirtintam Judo miestui 
\par 6 ir įsakė jiems: “Žiūrėkite, ką darote! Jūs teisiate ne dėl žmonių, bet dėl Viešpaties, kuris yra su jumis teismo metu. 
\par 7 Tebūna Viešpaties baimė ant jūsų ir būkite atidūs. Viešpats, mūsų Dievas, nėra neteisingas, neatsižvelgia į asmenis ir neima kyšių”. 
\par 8 Be to, Jeruzalėje Juozapatas paskyrė Viešpaties teismui ir ginčams spręsti po kelis levitus, kunigus ir Izraelio šeimų vyresniuosius, kai jie sugrįžo į Jeruzalę. 
\par 9 Karalius jiems įsakė: “Taip darykite Viešpaties baimėje, ištikimai ir tobula širdimi. 
\par 10 Kiekvienoje byloje, kuri jums bus pavesta jūsų brolių, gyvenančių savo miestuose, ar tai būtų dėl kraujo praliejimo, įstatymo, įsakymo, nuostatų ar potvarkių laužymo, mokykite juos, kad jie nenusikalstų Viešpačiui ir kad Viešpaties bausmė nepaliestų jūsų ir jūsų brolių. Taip darykite ir nenusikalsite. 
\par 11 Vyriausiasis kunigas Amarijas skiriamas jūsų vyresniuoju visuose Viešpaties reikaluose, o Izmaelio sūnus Zebadijas, Judo giminės vyresnysis,­visuose karaliaus reikaluose. Levitai bus jums valdininkais. Elkitės drąsiai, ir Viešpats bus su gerai besielgiančiais”.



\chapter{20}


\par 1 Moabitai, amonitai ir su jais kiti amonitų sąjungininkai išėjo kariauti prieš Juozapatą. 
\par 2 Juozapatui buvo pranešta: “Didelė daugybė iš anapus jūros ateina prieš tave. Jie jau Haceczon Tamaroje, kuris yra En Gedyje”. 
\par 3 Juozapatas nusigandęs atsidavė ieškoti Viešpaties ir paskelbė pasninką visame Jude. 
\par 4 Judo gyventojai susirinko prašyti Viešpaties pagalbos, iš visų miestų jie atėjo ieškoti Viešpaties. 
\par 5 Juozapatas atsistojo Viešpaties namuose, priešais naują kiemą, Judo ir Jeruzalės susirinkime, 
\par 6 ir sakė: “Viešpatie, mūsų tėvų Dieve, argi ne Tu esi Dievas danguje ir argi ne Tu valdai visas pagonių karalystes? Tavo rankose yra jėga ir galybė, ir niekas negali atsilaikyti prieš Tave. 
\par 7 Argi ne Tu, mūsų Dieve, išvarei šitos šalies gyventojus prieš Izraeliui užimant šį kraštą ir jį atidavei savo draugo Abraomo palikuonims amžiams? 
\par 8 Jie apsigyveno jame ir pastatė Tau šventyklą, kurioje būtų Tavo vardas, sakydami: 
\par 9 ‘Jei mus užpuls nelaimės, kardas, maras ar badas, tai mes, atsistoję ties šitais namais, Tavo akivaizdoje,­nes Tavo vardas yra šituose namuose,­šauksimės Tavęs savo suspaudime, o Tu mus išgirsi ir išgelbėsi’. 
\par 10 Amonitai, moabitai ir Seyro aukštumų gyventojai, kurių Tu neleidai izraelitams užpulti, jiems išėjus iš Egipto šalies, ir kuriuos izraelitai aplenkė ir jų nesunaikino, 
\par 11 dabar atmoka mums tuo, kad ateina mūsų išvaryti iš Tavo mums duotos nuosavybės. 
\par 12 Dieve, argi neteisi jų? Mes esame bejėgiai prieš šitą daugybę, kuri išėjo prieš mus, ir nežinome, ką mums daryti. Bet mūsų akys nukreiptos į Tave”. 
\par 13 Visi Judo gyventojai stovėjo Viešpaties akivaizdoje su kūdikiais, žmonomis ir vaikais. 
\par 14 Tada ant Jahazielio, sūnaus Zacharijo, sūnaus Benajos, sūnaus Jejelio, sūnaus Matanijos, levito iš Asafo sūnų, nužengė Viešpaties Dvasia, jam stovint tarp susirinkusiųjų, 
\par 15 ir jis tarė: “Klausykite, Judo ir Jeruzalės gyventojai ir tu, karaliau Juozapatai! Taip sako Viešpats: ‘Nebijokite ir neišsigąskite šitos daugybės, nes kova yra ne jūsų, bet Dievo. 
\par 16 Rytoj išeikite prieš juos. Jie eis Zizo įkalne ir jūs sutiksite juos slėnio pabaigoje, ties Jeruelio dykuma. 
\par 17 Jums nereikės kovoti. Išsirikiuokite, stovėkite ir stebėkite, kaip Viešpats jus išgelbės. Judo ir Jeruzalės gyventojai, nenusigąskite ir nebijokite! Rytoj išeikite prieš juos, nes Viešpats bus su jumis!’ ” 
\par 18 Juozapatas nusilenkė iki žemės, ir visi Judo bei Jeruzalės gyventojai krito prieš Viešpatį, garbindami Jį. 
\par 19 Levitai, Kehato ir Koracho palikuonys, garsiai šlovino Viešpatį, Izraelio Dievą. 
\par 20 Anksti rytą atsikėlę, jie išėjo į Tekojos dykumą. Jiems išeinant, Juozapatas tarė: “Paklausykite manęs, Judo ir Jeruzalės gyventojai. Tikėkite Viešpačiu, savo Dievu, tai būsite įtvirtinti. Tikėkite Jo pranašais, tai klestėsite”. 
\par 21 Pasitaręs su tauta, jis paskyrė giedotojus Viešpačiui, kad jie eitų kariuomenės priekyje, girdami šventumo grožį, ir sakytų: “Dėkokite Viešpačiui, nes Jo gailestingumas amžinas!” 
\par 22 Kai jie pradėjo giedoti ir girti, Viešpats sukėlė paniką tarp amonitų, moabitų ir Seyro aukštumų gyventojų, kurie buvo išėję prieš Judą, ir jie vieni kitus sunaikino. 
\par 23 Amonitai ir moabitai sukilo prieš Seyro aukštumų gyventojus, žudydami juos ir naikindami. Išžudę Seyro gyventojus, jie ėmė naikinti vieni kitus. 
\par 24 Kai Judas atėjo į vietą, iš kur buvo matoma dykuma, jie pamatė žemę, nuklotą lavonais. 
\par 25 Juozapatas su žmonėmis atėjo surinkti grobio ir rado daugybę turtų, brangių daiktų ir kitų gėrybių, kurių prisirinko daugiau negu galėjo panešti. Tris dienas jie rinko grobį, nes jo buvo tiek daug. 
\par 26 Ketvirtą dieną jie susirinko į Berako slėnį ir laimino Viešpatį. Todėl ta vieta vadinama Berako slėniu iki šios dienos. 
\par 27 Po to visi Judo ir Jeruzalės vyrai su Juozapatu priekyje sugrįžo su džiaugsmu į Jeruzalę, nes Viešpats suteikė jiems džiaugsmo dėl jų priešų. 
\par 28 Jie atėjo Jeruzalėje prie Viešpaties namų su arfomis, psalteriais ir trimitais. 
\par 29 Dievo baimė apėmė aplinkines karalystes, kai jos išgirdo, kad Viešpats kovojo prieš Izraelio priešus. 
\par 30 Juozapato karaliavimas tapo ramus, nes Dievas suteikė jam ramybę iš visų pusių. 
\par 31 Juozapatas, pradėdamas valdyti Judą, buvo trisdešimt penkerių metų amžiaus. Jeruzalėje jis karaliavo dvidešimt penkerius metus. Jo motina buvo vardu Azuba, Silio duktė. 
\par 32 Jis vaikščiojo savo tėvo Asos keliais ir nenukrypo nuo jų, darydamas, kas teisinga Viešpaties akyse. 
\par 33 Tačiau aukštumos nebuvo sunaikintos, nes tauta dar nebuvo paruošusi širdžių savo tėvų Dievui. 
\par 34 Visi kiti Juozapato darbai yra surašyti knygoje Hananio sūnaus Jehuvo, kuris yra minimas Izraelio karalių knygoje. 
\par 35 Vėliau Judo karalius Juozapatas susidėjo su Izraelio karaliumi Ahaziju, kuris elgėsi labai nedorai. 
\par 36 Jie abu kartu statė laivus Ecjon Gebere, kad plauktų į Taršišą. 
\par 37 Dodavahuvo sūnus Eliezeras iš Marešos pranašavo prieš Juozapatą, sakydamas: “Kadangi tu susidėjai su Ahaziju, Viešpats sudaužė tavo darbą”. Laivai sudužo ir negalėjo plaukti į Taršišą.



\chapter{21}

\par 1 Juozapatas užmigo prie savo tėvų ir buvo palaidotas prie savo tėvų Dovydo mieste. Jo sūnus Joramas karaliavo jo vietoje. 
\par 2 Jo broliai buvo Juozapato sūnūs: Azarija, Jahielis, Zacharijas, Azarijas, Mykolas ir Šefatijas. Visi jie buvo Izraelio karaliaus sūnūs. 
\par 3 Jų tėvas jiems davė daug dovanų: aukso, sidabro ir kitokių brangių daiktų, be to, sutvirtintų Judo miestų, bet karalystę jis atidavė Joramui, nes jis buvo pirmagimis. 
\par 4 Joramas, perėmęs savo tėvo karalystę ir sustiprėjęs, išžudė visus savo brolius ir kai kuriuos Judo kunigaikščius. 
\par 5 Joramas pradėjo karaliauti trisdešimt dvejų metų ir aštuonerius metus karaliavo Jeruzalėje. 
\par 6 Jis vaikščiojo Izraelio karalių keliais kaip Ahabo namai, nes Ahabo duktė buvo jo žmona. Jis darė pikta Viešpaties akyse. 
\par 7 Tačiau Viešpats nenorėjo sunaikinti Dovydo namų dėl sandoros, kurią Jis padarė su Dovydu, pažadėdamas duoti žiburį jam ir jo sūnums per amžius. 
\par 8 Joramui valdant, edomitai sukilo, atsiskyrė nuo Judo ir paskyrė sau karalių. 
\par 9 Tuomet Joramas su savo vadais ir kovos vežimais naktį puolė ir sumušė edomitus, kurie buvo apsupę jį ir kovos vežimų viršininkus. 
\par 10 Tačiau Edomas atsiskyrė nuo Judo iki šios dienos. Tuo pačiu metu ir Libna sukilo prieš Joramą, nes jis apleido Viešpatį, savo tėvų Dievą. 
\par 11 Jis padarė aukštumas Judo kalnuose, suvedžiojo Jeruzalės gyventojus ir visą Judą. 
\par 12 Pranašas Elijas atsiuntė jam laišką: “Taip sako Viešpats, tavo tėvo Dovydo Dievas: ‘Kadangi tu nevaikščiojai savo tėvo Juozapato ir Judo karaliaus Asos keliais, 
\par 13 bet pasirinkai Izraelio karalių kelius ir vedei į paleistuvystę Judo ir Jeruzalės gyventojus kaip Ahabas, be to, išžudei savo brolius, savo tėvo namiškius, geresnius už tave, 
\par 14 Viešpats baus tavo tautą, tavo vaikus, žmonas ir sunaikins visą tavo nuosavybę, 
\par 15 o tu pats susirgsi sunkia vidurių liga, kuri tave kankins kasdien’ ”. 
\par 16 Viešpats sukėlė prieš Joramą filistinus ir arabus, gyvenančius etiopų kaimynystėje. 
\par 17 Jie užpuolė Judo žemę, įsiveržė į ją ir išsigabeno visą turtą, kurį rado karaliaus namuose, jo vaikus ir žmonas; išliko tik jauniausias sūnus Jehoachazas. 
\par 18 Po to Viešpats ištiko jį nepagydoma vidurių liga. 
\par 19 Po dvejų metų jo viduriai išvirto, ir jis mirė baisiuose skausmuose. Jo tauta nesukūrė jam laužo, kaip darydavo jo tėvams. 
\par 20 Joramas buvo trisdešimt dvejų metų, pradėdamas karaliauti, ir karaliavo Jeruzalėje aštuonerius metus. Jis mirė ir buvo palaidotas Dovydo mieste, tačiau ne karalių kapinėse, nes jis buvo visų nemėgstamas.



\chapter{22}


\par 1 Jeruzalės gyventojai paskelbė karaliumi jo jauniausiąjį sūnų Ahaziją, nes visus vyresniuosius išžudė būriai, atėję su arabais. Taip Judo karaliumi tapo karaliaus Joramo sūnus Ahazijas. 
\par 2 Pradėdamas karaliauti, Ahazijas buvo keturiasdešimt dvejų metų ir karaliavo Jeruzalėje vienerius metus. Jo motina buvo vardu Atalija, Omrio duktė. 
\par 3 Jis vaikščiojo Ahabo namų keliais, nes jo motina buvo jam patarėja piktuose darbuose. 
\par 4 Jis darė pikta Viešpaties akyse kaip Ahabo namai, nes jie, jo tėvui mirus, buvo jo patarėjais jo paties pražūčiai. 
\par 5 Jų patariamas, jis ėjo su Ahabo sūnumi Joramu, Izraelio karaliumi, kariauti prieš Sirijos karalių Hazaelį į Ramot Gileadą. Sirams sužeidus Joramą, 
\par 6 jis grįžo į Jezreelį gydytis, nes buvo sužeistas, kariaudamas su Sirijos karaliumi Hazaeliu. Joramo sūnus Ahazijas, Judo karalius, vyko į Jezreelį aplankyti Ahabo sūnų Joramą. 
\par 7 Tai buvo nuo Dievo, kad Ahazijas žūtų, lankydamas Joramą. Jam atvykus pas Joramą, jie susitiko su Jehuvu, Nimšio sūnumi, kurį Viešpats buvo patepęs sunaikinti Ahabo namus. 
\par 8 Kai Jehuvas vykdė teismą Ahabo namams, jis sutiko Judo kunigaikščius ir Ahazijo brolių sūnus, tarnavusius Ahazijui, ir juos išžudė. 
\par 9 Jis ieškojo Ahazijo ir rado jį besislapstantį Samarijoje. Jis buvo atvestas pas Jehų ir nužudytas. Jie palaidojo jį, sakydami: “Jis yra sūnus Juozapato, kuris ieškojo Viešpaties visa širdimi”. Ahazijo namuose nebuvo nė vieno vyro, tinkamo užimti karaliaus sostą. 
\par 10 Ahazijo motina Atalija, sužinojusi, kad jos sūnus miręs, išžudė visus Judo karališkuosius palikuonis. 
\par 11 Bet karaliaus duktė Jehošabata slapčia paėmė Ahazijo sūnų Jehoašą iš karaliaus sūnų, kurie turėjo būti nužudyti, ir paslėpė su jo aukle miegamajame. Jehošabata buvo Joramo duktė, kunigo Jehojados žmona, Ahazijo sesuo. Taip jis išliko gyvas. 
\par 12 Jis buvo paslėptas Dievo namuose šešerius metus. Tuo metu Atalija valdė kraštą.



\chapter{23}

\par 1 Septintaisiais metais Jehojada išdrįso pasikviesti sąjungininkais šimtininkus: Joramo sūnų Azariją, Johanano sūnų Izmaelį, Jobedo sūnų Azariją, Adajo sūnų Maasėją ir Zichrio sūnų Elišafatą. 
\par 2 Jie apėjo Judą ir surinko levitus iš visų Judo miestų bei Izraelio šeimų vadus, ir atėjo į Jeruzalę. 
\par 3 Visas susirinkimas padarė sandorą su karaliumi Viešpaties namuose. Jehojada jiems kalbėjo: “Štai karaliaus sūnus bus karaliumi, kaip Viešpats kalbėjo apie Dovydo palikuonis. 
\par 4 Padarykite štai ką: trečdalis kunigų ir levitų, kurie įeina sabate, saugos visas duris, 
\par 5 trečdalis budės prie karaliaus namų ir trečdalis­prie pagrindinių vartų, o visi žmonės susirinks Viešpaties namų kiemuose. 
\par 6 Niekas teneįeina į Viešpaties namus, tik kunigai ir tie levitai, kurie atlieka tarnystę, jie įeis, nes yra šventi, o visi žmonės pasiliks Viešpaties sargyboje. 
\par 7 Levitai, kiekvienas su ginklu rankoje, apsups karalių ir bus su juo, kai jis įeis ir išeis. Kas įeis į namus, bus nužudytas”. 
\par 8 Levitai ir visas Judas darė, kaip kunigas Jehojada buvo įsakęs: kiekvienas atėjo su savo vyrais, kurie turėjo įeiti sabate ir kurie turėjo išeiti sabate, nes kunigas Jehojada nepaleido sargybą baigusių skyrių. 
\par 9 Kunigas Jehojada padalino šimtininkams karaliaus Dovydo ietis, didžiuosius ir mažuosius skydus, buvusius Dievo namuose, 
\par 10 ir sustatė visus žmones su ginklais rankose nuo dešiniojo šventyklos šono iki kairiojo, prie aukuro ir prie šventyklos aplinkui karalių. 
\par 11 Tada jie išvedė karaliaus sūnų, uždėjo jam karūną, įteikė liudijimą ir paskelbė jį karaliumi. Kunigas Jehojada ir jo sūnūs patepė jį ir visi šaukė: “Tegyvuoja karalius!” 
\par 12 Atalija, išgirdusi triukšmą ir sveikinimus karaliui, pamačiusi bėgančius žmones, atėjo į Viešpaties namus. 
\par 13 Čia ji pamatė karalių, stovintį šalia kolonos prie įėjimo, apsuptą kunigaikščių ir trimitininkų, ir visus žmones besidžiaugiančius. Trimitininkai trimitavo, giesmininkai, pritardami muzikos instrumentais, giedojo gyriaus giesmę. Tada Atalija perplėšė savo drabužius, šaukdama: “Sąmokslas! Sąmokslas!” 
\par 14 Kunigas Jehojada įsakė šimtininkams, kariuomenės vadams: “Išveskite ją pro sargybų eiles, o kas ją seks, nužudykite kardu”. Kunigas įsakė nežudyti jos Viešpaties namuose. 
\par 15 Jie atvedė ją iki Arklių vartų prie karaliaus rūmų ir ten nužudė. 
\par 16 Jehojada padarė sandorą su karaliumi ir visais žmonėmis, kad jie bus Viešpaties tauta. 
\par 17 Po to visi žmonės nuėjo į Baalo namus, sugriovė juos, aukurus ir atvaizdus sudaužė, o Baalo kunigą Mataną nužudė prie aukuro. 
\par 18 Jehojada pavedė Viešpaties namų priežiūrą kunigams ir levitams, suskirstydamas kunigus ir levitus skyriais, kaip Dovydas buvo juos suskirstęs. Jie aukojo Viešpačiui deginamąsias aukas, kaip parašyta Mozės įstatyme, giedodami ir džiaugdamiesi, kaip nurodė Dovydas. 
\par 19 Jis taip pat pastatė vartininkus prie Viešpaties namų vartų, kad neįeitų niekas, susitepęs kokiu nors būdu. 
\par 20 Jis kartu su šimtininkais, kilmingaisiais, valdytojais ir visais žmonėmis nuvedė karalių iš Viešpaties namų pro aukštutinius vartus į karaliaus namus ir pasodino jį į karaliaus sostą. 
\par 21 Visi krašto žmonės džiaugėsi, mieste buvo ramu, kai Atalija buvo nužudyta kardu.



\chapter{24}

\par 1 Jehoašas pradėjo karaliauti būdamas septynerių metų ir keturiasdešimt metų karaliavo Jeruzalėje. Jo motina buvo vardu Cibija iš Beer Šebos. 
\par 2 Jehoašas darė tai, kas teisinga Viešpaties akyse, per visas kunigo Jehojados dienas. 
\par 3 Jehojada parinko jam dvi žmonas, ir jis susilaukė sūnų bei dukterų. 
\par 4 Po to Jehoašas sumanė atnaujinti Viešpaties namus. 
\par 5 Sukvietęs kunigus ir levitus, jis jiems kalbėjo: “Eikite į Judo miestus ir rinkite iš viso Izraelio pinigus jūsų Dievo namams pataisyti kiekvienais metais. Ir darykite tai skubiai”. Bet levitai neskubėjo. 
\par 6 Karalius pasišaukė vyriausiąjį kunigą Jehojadą ir jam tarė: “Kodėl tu nereikalauji iš levitų, kad jie surinktų iš Judo ir Jeruzalės mokestį, kurį Viešpaties tarnas Mozė įsakė mokėti visam Izraeliui dėl Liudijimo palapinės? 
\par 7 Piktadarė Atalija ir jos sūnūs sunaikino Dievo namus ir visas Viešpaties namams pašvęstas dovanas panaudojo Baalui”. 
\par 8 Karaliui įsakius, buvo padaryta dėžė ir pastatyta prie Viešpaties namų durų lauko pusėje. 
\par 9 Ir jie paskelbė Jude ir Jeruzalėje, kad neštų Viešpačiui mokestį, kurį Dievo tarnas Mozė įsakė Izraeliui dykumoje. 
\par 10 Kunigaikščiai bei visa tauta džiaugėsi ir, atnešę mokestį, metė į dėžę, kol ją pripildė. 
\par 11 Dėžei prisipildžius, ją atnešdavo pas karalių. Karaliaus raštininkas ir vyriausiojo kunigo įgaliotinis išimdavo pinigus, o dėžę nunešdavo atgal į jos vietą. Taip jie darė kiekvieną dieną ir surinko daug pinigų. 
\par 12 Karalius bei Jehojada juos atiduodavo Viešpaties namų darbų prižiūrėtojams, o tie pasamdydavo mūrininkų ir dailidžių, geležies ir vario kalvių Viešpaties namams pataisyti bei atnaujinti. 
\par 13 Darbininkai dirbo ir jiems sekėsi. Jie atstatė Dievo namus ir juos sutvirtino. 
\par 14 Pabaigę darbą, jie atnešė likusius pinigus ir grąžino karaliui bei Jehojadai. Už juos buvo padaryti indai Viešpaties namams: indai reikalingi tarnaujant, aukojimo indai, taurės, auksiniai bei sidabriniai indai. Jie nuolat aukojo deginamąsias aukas Viešpaties namuose per visas Jehojados dienas. 
\par 15 Bet Jehojada paseno ir mirė sulaukęs šimto trisdešimties metų. 
\par 16 Jį palaidojo Dovydo mieste prie karalių, nes jis darė gera Izraelyje Dievui ir Jo namams. 
\par 17 Kunigui Jehojadai mirus, Judo kunigaikščiai atėjo pas karalių ir jam nusilenkė. Tada jis ėmė jų klausyti. 
\par 18 Jie paliko Viešpaties, savo tėvų Dievo, namus ir tarnavo alkams ir stabams. Tas nusikaltimas sukėlė Viešpaties rūstybę prieš Judą ir Jeruzalę. 
\par 19 Viešpats siuntė jiems pranašų, norėdamas juos susigrąžinti, kurie įspėjo juos, tačiau jie neklausė. 
\par 20 Dievo Dvasia nužengė ant kunigo Jehojados sūnaus Zacharijos, kuris atsistojęs kalbėjo susirinkusiems: “Taip sako Viešpats: ‘Kodėl jūs laužote Viešpaties įsakymus ir nenorite, kad jums sektųsi? Kadangi jūs palikote Viešpatį, Jis paliko jus’ ”. 
\par 21 Jie susitarė prieš jį ir, karaliui įsakius, užmušė akmenimis Viešpaties namų kieme. 
\par 22 Karalius Jehoašas neatsiminė to gero, kurį Zacharijos tėvas Jehojada buvo jam padaręs, bet nužudė jo sūnų. Mirdamas jis sakė: “Viešpats mato ir atlygins”. 
\par 23 Metų gale Sirijos kariuomenė atėjo prieš jį, įsiveržė į Judą bei Jeruzalę, išžudė tautos kunigaikščius ir paimtą grobį išsiuntė Damasko karaliui. 
\par 24 Nors sirų kariuomenėje buvo mažai žmonių, bet Viešpats atidavė į jų rankas labai didelę kariuomenę, kadangi jie paliko Viešpatį, savo tėvų Dievą. Taip jie įvykdė teismą Jehoašui. 
\par 25 Atsitraukdami jie paliko jį sunkiai sergantį. Jehoašo tarnai susitarė prieš jį dėl pralieto kunigo Jehojados sūnų kraujo ir nužudė karalių lovoje. Jis mirė ir buvo palaidotas Dovydo mieste, tačiau ne karalių kapinėse. 
\par 26 Amonitės Šimeatos sūnus Zabadas ir moabitės Šimritos sūnus Jehozabadas surengė sąmokslą prieš jį. 
\par 27 Apie jo sūnus, naštų, kurios buvo jam užkrautos, didumą ir Dievo namų atnaujinimą yra parašyta karalių knygoje. Jo sūnus Amacijas karaliavo jo vietoje.



\chapter{25}

\par 1 Amacijas pradėjo karaliauti būdamas dvidešimt penkerių metų ir karaliavo Jeruzalėje dvidešimt devynerius metus. Jo motina buvo vardu Jehoadana iš Jeruzalės. 
\par 2 Jis darė tai, kas teisinga Viešpaties akyse, tačiau ne tobula širdimi. 
\par 3 Įsitvirtinęs karalystėje, jis nužudė tuos savo tarnus, kurie nužudė jo tėvą karalių. 
\par 4 Jų vaikų jis nenužudė, kaip Mozės įstatymo knygoje parašyta, kur Viešpats įsako: “Tėvai neturi mirti dėl vaikų ir vaikai dėl tėvų, bet kiekvienas mirs dėl savo paties nuodėmės”. 
\par 5 Amacijas sušaukė Judo ir Benjamino gyventojus ir paskyrė jiems tūkstantininkus bei šimtininkus pagal jų tėvų namus. Jis suskaičiavo dvidešimties metų ir vyresnius ir rado tris šimtus tūkstančių vyrų, tinkančių karui, galinčių naudoti ietį ir skydą. 
\par 6 Be to, jis pasisamdė iš Izraelio šimtą tūkstančių rinktinių karių už šimtą talentų sidabro. 
\par 7 Dievo vyras, atėjęs pas jį, tarė: “Karaliau, tegul neina su tavimi Izraelio kariuomenė, nes Viešpats nėra su Izraeliu, su Efraimo sūnumis. 
\par 8 O jeigu eisi, pasiruošk mūšiui, tačiau Dievas tave parklupdys prieš priešus, nes Dievas turi galią padėti ir parklupdyti”. 
\par 9 Amacijas klausė Dievo vyro: “O ką daryti su šimtu talentų, kuriuos daviau Izraelio kariuomenei?” Dievo vyras atsakė: “Dievas tau gali duoti daug daugiau negu tiek”. 
\par 10 Amacijas atskyrė iš Efraimo atėjusią kariuomenę ir paleido juos namo. Jie, degdami pykčiu Judui, grįžo į savo kraštą. 
\par 11 Amacijas įsidrąsino, išvedė savo žmones ir, atėjęs į Druskos slėnį, išžudė dešimt tūkstančių Seyro vaikų. 
\par 12 Dešimt tūkstančių jie paėmė į nelaisvę ir, užvedę juos ant uolos viršūnės, nustūmė žemyn, ir jie visi užsimušė. 
\par 13 Ta kariuomenė, kurią Amacijas pasiuntė atgal, kad jie neitų su juo, užpuolė Judo miestus nuo Samarijos iki Bet Horono, išžudė tris tūkstančius gyventojų ir prisiplėšė daug grobio. 
\par 14 Amacijas, nugalėjęs edomitus, grįždamas parsigabeno Seyro vaikų dievus, kuriuos pasistatė sau dievais, jiems aukojo ir juos garbino. 
\par 15 Viešpaties rūstybė užsidegė prieš Amaciją. Jis siuntė pranašą, kuris sakė: “Kodėl tu ieškai tos tautos dievų, kurie negalėjo išgelbėti savo tautos iš tavo rankų?” 
\par 16 Pranašui kalbant, karalius jam tarė: “Ar tu paskirtas karaliaus patarėju? Nutilk, kad nenužudyčiau tavęs”. Pranašas nutilo, pasakęs: “Žinau, kad Dievas nusprendė pražudyti tave, nes tu taip pasielgei ir neklausei mano patarimo”. 
\par 17 Judo karalius Amacijas pasitarė ir siuntė pas Jehuvo sūnaus Jehoachazo sūnų Jehoašą, Izraelio karalių, sakydamas: “Išeik, kad susitiktume veidas į veidą”. 
\par 18 Izraelio karalius Jehoašas atsakė Judo karaliui Amacijui: “Libano usnis siuntė pas Libano kedrą, sakydama: ‘Leisk savo dukterį už mano sūnaus’. Bet Libano laukinis žvėris prabėgdamas sutrypė usnį. 
\par 19 Tu didžiuojiesi nugalėjęs Edomą ir keliesi puikybėn besigirdamas. Lik namuose. Kodėl nori prisišaukti nelaimę ir žūti kartu su Judu?” 
\par 20 Amacijas nepaklausė, nes tai buvo iš Dievo, kad Jis galėtų atiduoti juos į priešų rankas, nes jie ieškojo Edomo dievų. 
\par 21 Izraelio karalius Jehoašas atėjo. Jis ir Judo karalius Amacijas susitiko Bet Šeme, kuris priklauso Judui. 
\par 22 Izraelis nugalėjo Judą, ir šio vyrai pabėgo į savo palapines. 
\par 23 Izraelio karalius Jehoašas paėmė į nelaisvę Judo karalių Amaciją, Jehoachazo sūnaus Jehoašo sūnų, Bet Šemeše ir, atvedęs jį į Jeruzalę, nugriovė Jeruzalės sieną nuo Efraimo vartų iki Kampo vartų, keturis šimtus uolekčių. 
\par 24 Po to Izraelio karalius pasiėmė visą auksą bei sidabrą ir visus indus, rastus Dievo namuose Obed Edomo priežiūroje, taip pat karaliaus namų turtus, įkaitus ir grįžo į Samariją. 
\par 25 Jehoašo sūnus Amacijas, Judo karalius, mirus Jehoachazo sūnui Jehoašui, Izraelio karaliui, dar gyveno penkiolika metų. 
\par 26 Visi kiti Amacijo darbai, pirmieji ir paskutinieji, yra surašyti Judo ir Izraelio karalių knygoje. 
\par 27 Amacijui pasitraukus nuo Viešpaties, prieš jį kilo sąmokslas Jeruzalėje, ir jis pabėgo į Lachišą. Bet jie pasiuntė į Lachišą ir jį ten nužudė. 
\par 28 Jo kūną pargabeno ant žirgų ir palaidojo prie jo tėvų Judo mieste.



\chapter{26}

\par 1 Tada visi Judo žmonės ėmė Oziją, kuriam buvo šešiolika metų, ir padarė savo karaliumi jo tėvo Amacijo vietoje. 
\par 2 Jis sutvirtino Elatą ir sugrąžino jį Judui po to, kai karalius užmigo prie savo tėvų. 
\par 3 Pradėdamas valdyti, Ozijas buvo šešiolikos metų ir penkiasdešimt dvejus metus karaliavo Jeruzalėje. Jo motina buvo vardu Jecholija iš Jeruzalės. 
\par 4 Jis darė tai, kas teisinga Viešpaties akyse, kaip ir jo tėvas Amacijas. 
\par 5 Jis ieškojo Dievo, kol Zacharijas, kuris suprasdavo Dievo regėjimus, buvo gyvas, ir kol jis ieškojo Viešpaties, Dievas davė jam sėkmę. 
\par 6 Jis kariavo su filistinais ir sugriovė Gato, Jabnės bei Ašdodo sienas. Jis pastatė miestus aplink Ašdodą ir filistinų krašte. 
\par 7 Dievas padėjo jam prieš filistinus ir arabus, gyvenusius Gūr Baale, ir prieš meunus. 
\par 8 Amonitai mokėjo Ozijui duoklę. Jo vardas pagarsėjo net iki Egipto sienos, nes jis nepaprastai sustiprėjo. 
\par 9 Ozijas pastatė Jeruzalėje bokštus prie Kampo vartų, Slėnio vartų bei prie kampų ir juos sutvirtino. 
\par 10 Jis taip pat pastatė bokštų dykumoje ir iškasė daug šulinių, nes turėjo daug gyvulių žemumose ir slėnyje bei žemdirbių ir vynuogynų prižiūrėtojų įkalnėse ir Karmelyje; jis mėgo žemdirbystę. 
\par 11 Ozijas turėjo kariuomenę, kuri išeidavo į karą būriais, kuri buvo suskaičiuota raštininko Jejelio ir valdininko Maasėjo, vadovaujant vienam iš karaliaus vadų Hananijui. 
\par 12 Šeimų vadų ir karžygių buvo du tūkstančiai šeši šimtai vyrų. 
\par 13 Jie vadovavo trims šimtams septyniems tūkstančiams penkiems šimtams kariuomenės vyrų, kurie galingai kariaudavo, padėdami karaliui prieš priešą. 
\par 14 Ozijas parūpino kariuomenei skydų, iečių, šalmų, lankų ir svaidyklių akmenims svaidyti. 
\par 15 Jis turėjo Jeruzalėje naujai išrastų karinių mašinų strėlėms ir dideliems akmenims svaidyti, kurias įstatė bokštuose ir kampuose. Jo vardas plačiai išgarsėjo, nes nuostabi pagalba lydėjo jį ir jis tapo galingas. 
\par 16 Kai jis buvo galingas, jo širdis pasididžiavo jo pražūčiai. Jis nusikalto Viešpačiui, savo Dievui, eidamas į Viešpaties šventyklą smilkyti ant smilkymo aukuro. 
\par 17 Kunigas Azarijas įėjo paskui jį su aštuoniasdešimt drąsių Viešpaties kunigų. 
\par 18 Jie pasipriešino karaliui Ozijui ir sakė: “Ozijau, tu neturi teisės smilkyti Viešpačiui. Tai kunigų, Aarono sūnų, pareiga. Išeik iš šventyklos! Tu nusikaltai ir Viešpats Dievas nepriskaitys to tavo garbei”. 
\par 19 Ozijas jau laikė rankoje smilkytuvą, pasiruošęs smilkyti. Jis labai supyko ant kunigų. Kai jis supyko, raupsai atsirado jo kaktoje Viešpaties namuose prie smilkymo aukuro, kunigams matant. 
\par 20 Vyriausiasis kunigas Azarijas ir visi kunigai žiūrėjo į jį, ir jis buvo raupsuotas. Jie išstūmė jį iš ten, o ir jis pats skubėjo išeiti, nes Viešpats jį ištiko. 
\par 21 Karalius Ozijas liko raupsuotas iki savo mirties. Jis gyveno atskiruose namuose ir buvo atskirtas nuo Viešpaties namų. Jo sūnus Joatamas valdė karaliaus namus ir teisė krašto žmones. 
\par 22 Visus kitus Ozijo darbus, pirmus ir paskutinius, užrašė Amoco sūnus pranašas Izaijas. 
\par 23 Ozijas užmigo prie savo tėvų ir buvo palaidotas prie savo tėvų karalių kapinių lauke, nes jis buvo raupsuotas. Jo sūnus Joatamas karaliavo jo vietoje.



\chapter{27}


\par 1 Pradėdamas karaliauti, Joatamas buvo dvidešimt penkerių metų ir šešiolika metų karaliavo Jeruzalėje. Jo motina buvo vardu Jeruša, Cadoko duktė. 
\par 2 Jis darė tai, kas teisinga Viešpaties akyse, kaip ir jo tėvas Ozijas, tačiau jis neįėjo į Viešpaties šventyklą. Tauta vis dar elgėsi netikusiai. 
\par 3 Joatamas pastatė aukštutinius vartus Viešpaties namuose ir daug statė ant Ofelio sienos. 
\par 4 Be to, jis statė miestus Judo aukštumose ir pilis bei bokštus miškuose. 
\par 5 Jis kariavo su amonitų karaliumi ir jį nugalėjo. Amonitai davė jam šimtą talentų sidabro, dešimt tūkstančių saikų kviečių ir tiek pat miežių pirmaisiais metais, tiek pat antraisiais ir trečiaisiais metais. 
\par 6 Joatamas sustiprėjo, nes vaikščiojo priešais Viešpatį, savo Dievą. 
\par 7 Visi kiti Joatamo darbai ir visi jo karai surašyti Izraelio ir Judo karalių knygoje. 
\par 8 Pradėdamas karaliauti, jis buvo dvidešimt penkerių metų ir šešiolika metų karaliavo Jeruzalėje. 
\par 9 Joatamas užmigo prie savo tėvų ir buvo palaidotas Dovydo mieste, o jo sūnus Achazas karaliavo jo vietoje.



\chapter{28}


\par 1 Achazas, būdamas dvidešimties metų, pradėjo karaliauti ir šešiolika metų karaliavo Jeruzalėje. Jis nedarė to, kas teisinga Viešpaties akyse, kaip darė jo tėvas Dovydas, 
\par 2 bet vaikščiojo Izraelio karalių keliais ir nuliedino Baalų atvaizdus. 
\par 3 Jis smilkė Ben Hinomo slėnyje ir sudegino savo sūnus Baalui, mėgdžiodamas bjaurystes pagonių, kuriuos Viešpats išvarė, prieš izraelitams užimant kraštą. 
\par 4 Jis aukojo ir smilkė aukštumose, ant kalvų ir po kiekvienu žaliuojančiu medžiu. 
\par 5 Todėl Viešpats, jo Dievas, atidavė jį į Sirijos karaliaus rankas. Sirai paėmė daug belaisvių ir nusivedė juos į Damaską. Jis taip pat buvo atiduotas į Izraelio karaliaus rankas, kuris daugelį iš jų nužudė. 
\par 6 Remalijo sūnus Pekachas išžudė Jude per vieną dieną šimtą dvidešimt tūkstančių karių, nes jie apleido Viešpatį, savo tėvų Dievą. 
\par 7 Zichris, Efraimo karžygys, nukovė karaliaus sūnų Maasėją, namų valdytoją Azrikamą ir Elkaną, kuris buvo antras po karaliaus. 
\par 8 Be to, izraelitai išsivedė nelaisvėn iš savo brolių du šimtus tūkstančių moterų, sūnų ir dukterų ir, prisiplėšę daug grobio, parsigabeno į Samariją. 
\par 9 Bet ten gyveno Viešpaties pranašas Odedas, kuris pasitiko grįžtančią į Samariją kariuomenę ir jiems tarė: “Viešpats, jūsų tėvų Dievas, supykęs ant Judo, atidavė juos į jūsų rankas, bet jūs žudėte taip žiauriai, kad jūsų žiaurumas pasiekė dangų. 
\par 10 Jūs nutarėte Judo ir Jeruzalės gyventojus paversti vergais ir vergėmis. Argi jūs patys nenusikaltote Viešpačiui, savo Dievui? 
\par 11 Klausykite manęs ir sugrąžinkite belaisvius, kuriuos atsivedėte iš Judo, nes didelė Viešpaties rūstybė užsidegė prieš jus”. 
\par 12 Efraimitų vyresnieji: Johanano sūnus Azarijas, Mešilemoto sūnus Berechijas, Šalumo sūnus Jehizkijas ir Hadlajo sūnus Amasa atsistojo prieš tuos, kurie grįžo iš karo, 
\par 13 sakydami: “Neįvesite čia belaisvių, nes jūs tuo prisidėsite prie mūsų nusikaltimų Viešpačiui ir juos mums dar padauginsite; nes mūsų nusikaltimas didelis ir Viešpaties rūstybė užsidegusi prieš Izraelį”. 
\par 14 Tuomet kariai paliko belaisvius ir grobį kunigaikščiams bei susirinkusiems. 
\par 15 Vardais paminėti vyrai pakilo, paėmė belaisvius, iš grobio jie juos apvilko, apavė, pavalgydino, pagirdė ir patepė aliejumi; silpnesniuosius užkėlė ant asilų ir nuvedė visus į Jerichą, palmių miestą, pas jų brolius. Po to jie sugrįžo į Samariją. 
\par 16 Tuo metu karalius Achazas prašė Asirijos karaliaus pagalbos, 
\par 17 nes edomitai buvo įsiveržę ir nugalėję Judą bei išsivedę belaisvius. 
\par 18 Ir filistinai užpuolė pietų Judą žemumoje ir, paėmę Bet Šemešą, Ajaloną, Gederotą, Sochoją, Timną ir Gimzoją su jų kaimais, juose apsigyveno. 
\par 19 Nes Viešpats pažemino Judą dėl Izraelio karaliaus Achazo, kuris apnuogino Judą ir labai nusikalto Viešpačiui. 
\par 20 Atėjęs pas jį Tiglat Pileseras, Asirijos karalius, vargino jį, o ne sustiprino. 
\par 21 Achazas, paėmęs Viešpaties namų, karaliaus namų ir kunigaikščių auksą, atidavė Asirijos karaliui, tačiau tai jam nepadėjo. 
\par 22 Karalius Achazas savo vargų laikotarpiu dar labiau nusikalto Viešpačiui. 
\par 23 Jis aukojo Damasko dievams, kurie jį nugalėjo, sakydamas: “Sirijos karalių dievai padeda jiems; aš jiems aukosiu, kad jie ir man padėtų”. Bet jie tapo pražūtimi jam ir visam Izraeliui. 
\par 24 Achazas surinko Dievo namų indus, supjaustė juos, užrakino Viešpaties namų duris ir pristatė aukurų kiekviename Jeruzalės kampe. 
\par 25 Jis pristeigė kiekviename Judo mieste aukštumų svetimiems dievams smilkyti, sukeldamas Viešpaties, savo tėvų Dievo, rūstybę. 
\par 26 Visi kiti jo darbai yra surašyti Judo ir Izraelio karalių knygoje. 
\par 27 Achazas užmigo prie savo tėvų, ir jį palaidojo Jeruzalės mieste, bet ne Izraelio karalių kapuose. Jo sūnus Ezekijas karaliavo jo vietoje.



\chapter{29}

\par 1 Ezekijas pradėjo karaliauti dvidešimt penkerių metų ir dvidešimt devynerius metus karaliavo Jeruzalėje. Jo motina buvo vardu Abija, Zacharijo duktė. 
\par 2 Jis darė tai, kas teisinga Viešpaties akyse, kaip ir jo tėvas Dovydas. 
\par 3 Pirmaisiais savo karaliavimo metais, pirmąjį mėnesį jis atidarė Viešpaties namų duris ir juos atnaujino. 
\par 4 Sukvietęs kunigus ir levitus į rytinę aikštę, 
\par 5 jiems tarė: “Levitai, paklausykite manęs! Pasišventinkite ir pašventinkite Viešpaties, savo tėvų Dievo, namus, pašalindami nešvarumus iš šventyklos. 
\par 6 Mūsų tėvai nusikalto ir darė pikta Viešpaties, mūsų Dievo, akyse. Jie paliko Jį ir nusigręžė nuo Viešpaties buveinės, atsukdami Jam nugaras. 
\par 7 Jie užrakino šventyklos duris, užgesino lempas, nebesmilkė smilkalų ir nebeaukojo deginamųjų aukų Izraelio Dievui šventoje vietoje. 
\par 8 Todėl Viešpats užsirūstino ant Judo ir Jeruzalės. Jis atidavė juos vargui, pasibaisėjimui ir pajuokai, kaip patys matote savo akimis. 
\par 9 Nes štai mūsų tėvai žuvo nuo kardo ir mūsų sūnūs, dukterys ir žmonos pateko nelaisvėn. 
\par 10 Dabar mano širdyje yra noras padaryti sandorą su Viešpačiu, Izraelio Dievu, kad Jo didžioji rūstybė nusisuktų nuo mūsų. 
\par 11 Mano sūnūs, neatidėliokite, nes jus išsirinko Viešpats, kad prieš Jį stovėtumėte, Jam tarnautumėte ir smilkytumėte”. 
\par 12 Tuomet pakilo levitai. Iš kehatų­Amasajo sūnus Mahatas ir Azarijo sūnus Joelis, iš merarių­ Abdžio sūnus Kišas ir Jehalėlelio sūnus Azarijas, iš geršonų­Zimos sūnus Joachas ir Joacho sūnus Edenas, 
\par 13 iš Elicafano palikuonių­Šimris ir Jejelis, iš Asafo palikuonių­ Zacharijas ir Matanijas, 
\par 14 iš Hemano palikuonių­Jehielis ir Šimis, o iš Jedutūno palikuonių­Šemaja ir Uzielis. 
\par 15 Jie sušaukė savo brolius, pasišventino ir ėjo Viešpaties namų valyti, kaip karalius buvo įsakęs pagal Viešpaties žodį. 
\par 16 Kunigai, įėję į Viešpaties namų vidų, išnešė visus nešvarumus, kuriuos rado Viešpaties šventykloje, į kiemą, o levitai nešė juos į Kedrono upelį. 
\par 17 Pirmojo mėnesio pirmą dieną jie pradėjo šventinti, o mėnesio aštuntą dieną pasiekė Viešpaties namų prieangį, per aštuonias dienas jie pašventino Viešpaties namus ir baigė pirmo mėnesio šešioliktą dieną. 
\par 18 Po to, atėję pas karalių Ezekiją, pranešė: “Išvalėme visus Viešpaties namus: deginamųjų aukų aukurą su visais jo reikmenimis ir padėtinės duonos stalą su visais jo indais. 
\par 19 Visus reikmenis, kuriuos karalius Achazas karaliaudamas išmetė savo nusikaltimo metu, mes nuvalėme ir pašventinome, ir štai jie yra prie Viešpaties aukuro”. 
\par 20 Karalius Ezekijas, atsikėlęs anksti, sušaukė miesto vyresniuosius ir nuėjo į Viešpaties namus. 
\par 21 Jie atsivedė septynis jaučius, septynis avinus, septynis ėriukus ir septynis ožius aukai už nuodėmę,­už karalystę, šventyklą ir Judą. Jis įsakė kunigams, Aarono sūnums, aukoti ant Viešpaties aukuro. 
\par 22 Papjovę jaučius, kunigai ėmė kraujo ir juo šlakstė aukurą; papjovę avinus, taip pat šlakstė krauju aukurą ir, papjovę ėriukus, šlakstė jų krauju aukurą. 
\par 23 Po to privedė už nuodėmę aukojamus ožius prie karaliaus ir žmonių, kurie uždėjo ant jų rankas. 
\par 24 Kunigai juos papjovė ir atnešė prie aukuro jų kraują kaip sutaikinimo auką už visą Izraelį, nes karalius buvo įsakęs aukoti deginamąją auką už visą Izraelį. 
\par 25 Ezekijas pastatė prie Viešpaties namų levitus su cimbolais, arfomis ir psalteriais pagal karaliaus Dovydo, regėtojo Gado ir pranašo Natano nurodymus. Taip buvo įsakęs Viešpats per savo pranašus. 
\par 26 Levitai stovėjo su Dovydo instrumentais, o kunigai­su trimitais. 
\par 27 Ezekijas įsakė aukoti ant aukuro deginamąją auką. Prasidėjus deginamosios aukos aukojimui, suskambėjo Viešpaties giesmės ir trimitai, pritariant karaliaus Dovydo muzikiniams instrumentams. 
\par 28 Visi susirinkusieji garbino, giedotojai giedojo ir trimitai skardeno, kol pabaigė aukoti deginamąją auką. 
\par 29 Aukojimui pasibaigus, karalius ir visi, kurie buvo su juo, nusilenkę garbino Dievą. 
\par 30 Karalius Ezekijas ir kunigaikščiai liepė levitams šlovinti Viešpatį Dovydo ir regėtojo Asafo giesmėmis. Jie džiaugsmingai giedojo ir nusilenkę garbino. 
\par 31 Ezekijas tarė: “Dabar jūs esate pasišventinę Viešpačiui! Atgabenkite aukas bei padėkos aukas į Viešpaties namus”. Susirinkusieji atgabeno minėtas aukas ir kas norėjo­deginamąsias aukas. 
\par 32 Deginamųjų aukų, kurias susirinkusieji aukojo, buvo septyniasdešimt jaučių, šimtas avinų, du šimtai ėriukų; visi jie buvo paskirti deginamajai aukai Viešpačiui. 
\par 33 Pašvęstųjų aukų buvo šeši šimtai jaučių ir trys tūkstančiai avių. 
\par 34 Kunigų buvo per mažai, jie nespėjo lupti visų deginamųjų aukų odų; tad jiems padėjo jų broliai levitai, kol kiti kunigai pasišventino; levitai stropiau rūpinosi pasišventinti negu kunigai. 
\par 35 Be to, dar reikėjo aukoti daug deginamųjų aukų, padėkos aukų taukus ir geriamąsias aukas prie kiekvienos deginamosios aukos. Taip buvo atstatytas tarnavimas Viešpaties namuose. 
\par 36 Ezekijas ir visa tauta džiaugėsi, kad Dievas paruošė tautą, nes tai įvyko staiga.



\chapter{30}

\par 1 Ezekijas parašė ir išsiuntinėjo laiškus Izraeliui, Judui, Efraimui bei Manasui, kad atvyktų į Viešpaties namus Jeruzalėje švęsti Paschos Viešpačiui, Izraelio Dievui. 
\par 2 Karalius, kunigaikščiai ir Jeruzalės gyventojai sutarė švęsti Paschą antrąjį mėnesį. 
\par 3 Jie negalėjo jos švęsti laiku dėl to, kad dar nebuvo pasišventinęs pakankamas kunigų skaičius ir Jeruzalėje nebuvo susirinkę žmonės. 
\par 4 Tas sumanymas patiko karaliui ir visiems žmonėms. 
\par 5 Jie nutarė pranešti visam Izraeliui nuo Beer Šebos iki Dano, kad ateitų švęsti Paschos Viešpačiui, Izraelio Dievui, į Jeruzalę, nes jie seniai nebuvo šventę Paschos taip, kaip parašyta. 
\par 6 Karaliaus ir kunigaikščių paskirti šaukliai vaikščiojo su laiškais po visą Izraelį ir Judą, kaip karalius buvo įsakęs, skelbdami: “Izraelitai, gręžkitės į Viešpatį, Abraomo, Izaoko ir Jokūbo Dievą, ir Jis grįš pas jus, išlikusius iš Asirijos karalių rankos. 
\par 7 Nebūkite kaip jūsų tėvai ir broliai, kurie nusikalto Viešpačiui, savo tėvų Dievui, todėl Jis juos atidavė sunaikinti, kaip patys matote. 
\par 8 Nebūkite kietasprandžiai kaip jūsų tėvai! Paveskite save Viešpačiui ir ateikite į Jo šventyklą, kurią Jis pašventino amžiams, ir tarnaukite Viešpačiui, savo Dievui, kad Jo didžioji rūstybė nusisuktų nuo jūsų. 
\par 9 Jei gręšitės į Viešpatį, jūsų broliai ir sūnūs ras pasigailėjimą akyse tų, kurie juos išvedė, ir sugrįš į šitą šalį; nes maloningas ir gailestingas yra Viešpats, jūsų Dievas. Jis jūsų neatstums, jei sugrįšite pas Jį”. 
\par 10 Šaukliai ėjo iš miesto į miestą per Efraimo ir Manaso žemes iki Zabulono, bet šie juos išjuokė ir tyčiojosi iš jų. 
\par 11 Tačiau kai kurie iš Ašero, Manaso ir Zabulono nusižemino ir atėjo į Jeruzalę. 
\par 12 Jude buvo Viešpaties ranka, kad duotų jiems vieną širdį vykdyti karaliaus ir kunigaikščių įsakymą pagal Viešpaties žodį. 
\par 13 Į Jeruzalę susirinko labai daug žmonių švęsti Neraugintos duonos šventės antrąjį mėnesį. 
\par 14 Jie pašalino aukurus Jeruzalėje, visus smilkymo aukurus sudaužė ir sumetė juos į Kedrono upelį. 
\par 15 Jie pjovė Paschos avinėlį antrojo mėnesio keturioliktą dieną. Kunigai ir levitai susigėdę pasišventino ir atgabeno deginamųjų aukų į Viešpaties namus. 
\par 16 Jie ėjo savo tarnystę jiems įprasta tvarka, laikydamiesi Dievo vyro Mozės įstatymo; kunigai šlakstė kraują, paėmę jį iš levitų rankų. 
\par 17 Daugelis iš susirinkusiųjų nebuvo pasišventinę, todėl levitai papjaudavo Paschos avinėlį už tuos, kurie buvo susitepę, kad pašventintų juos Viešpačiui. 
\par 18 Daug žmonių, ypač iš Efraimo, Manaso, Isacharo ir Zabulono, nebuvo apsivalę ir valgė Paschą ne taip, kaip pasakyta įstatyme. Tačiau Ezekijas meldėsi už juos: “Gerasis Viešpatie, atleisk kiekvienam, 
\par 19 kuris paruošė savo širdį ieškoti Dievo­Viešpaties, savo tėvų Dievo, nors ir nėra apsivalęs pagal šventyklos reikalavimus”. 
\par 20 Viešpats išklausė Ezekiją ir išgydė žmones. 
\par 21 Izraelitai, susirinkę į Jeruzalę, šventė Neraugintos duonos šventę septynias dienas su dideliu džiaugsmu, o kunigai ir levitai kasdien garsiai šlovino Viešpatį muzikos instrumentais. 
\par 22 Ezekijas padrąsino levitus, kurie žinojo, kaip tinkamai tarnauti Viešpačiui. Jie valgė septynias šventės dienas, aukojo padėkos aukas ir garbino Viešpatį, savo tėvų Dievą. 
\par 23 Visi sutarė švęsti dar septynias dienas. Taip jie šventė džiaugsmingai kitas septynias dienas. 
\par 24 Judo karalius Ezekijas davė susirinkusiems tūkstantį jaučių ir septynis tūkstančius avių, o kunigaikščiai davė tūkstantį jaučių ir dešimt tūkstančių avių. Daug kunigų pasišventino. 
\par 25 Džiaugėsi visi Judo žmonės, kunigai, levitai, visi susirinkusieji iš Izraelio ir svetimtaučiai, atėję iš Izraelio, ir gyvenantieji Jude. 
\par 26 Didelis džiaugsmas buvo Jeruzalėje, nes nuo Izraelio karaliaus Dovydo sūnaus Saliamono laikų nieko panašaus nebuvo buvę Jeruzalėje. 
\par 27 Kunigai ir levitai pakilę laimino žmones, ir jų balsas buvo išgirstas, jų malda pasiekė Jo šventą buveinę danguje.



\chapter{31}


\par 1 Kai visa tai pasibaigė, visi susirinkę izraelitai, išėję į Judo miestus, sudaužė stabus, iškirto giraites, nugriovė aukštumas ir aukurus visame Jude, Benjamine, Efraime ir Manase. Po to visi izraelitai sugrįžo į savo miestus prie savo nuosavybės. 
\par 2 Ezekijas suskirstė kunigus ir levitus skyriais, kiekvieną pagal jo tarnystę, aukoti deginamąsias ir padėkos aukas, tarnauti, giedoti ir šlovinti Viešpaties šventyklos vartuose. 
\par 3 Karalius davė savo turtų dalį deginamosioms aukoms rytais, vakarais, sabatais, per jauną mėnulį ir metinėmis šventėmis, kaip parašyta Viešpaties įstatyme. 
\par 4 Jis įsakė tautai ir Jeruzalės gyventojams duoti dalį kunigams ir levitams, kad jie būtų padrąsinti Viešpaties įstatyme. 
\par 5 Įsakymą paskelbus, izraelitai gausiai atgabeno pirmavaisių javų, vyno, aliejaus ir medaus, taip pat atnešė gausiai dešimtinių. 
\par 6 Izraelitai ir Judo miestų gyventojai taip pat davė dešimtinę nuo galvijų ir avių, ir laisvos valios aukų Viešpačiui, savo Dievui, jie sukrovė krūvas. 
\par 7 Trečią mėnesį jie pradėjo krauti tas krūvas, o septintą­užbaigė. 
\par 8 Ezekijas ir kunigaikščiai, atėję ir pamatę krūvas, šlovino Viešpatį ir laimino Jo tautą Izraelį. 
\par 9 Ezekijas paklausė kunigų ir levitų apie krūvas. 
\par 10 Vyriausiasis kunigas Azarijas iš Cadoko namų atsakė: “Nuo to laiko, kai pradėjo gabenti aukas Viešpaties namams, mes sočiai valgome ir dar daugiau atlieka, nes Viešpats palaimino savo tautą”. 
\par 11 Ezekijas įsakė paruošti kambarius Viešpaties namuose. Juos paruošus, 
\par 12 jie ištikimai nešė aukas šventyklai, dešimtines ir dovanas. Jų vyriausiasis prižiūrėtojas buvo levitas Konanijas, o jo brolis Šimis buvo po jo. 
\par 13 Jehielis, Azazijas, Nahatas, Asaelis, Jerimotas, Jehozabadas, Elielis, Išmakijas, Mahatas ir Benajas buvo prižiūrėtojai Konanijo ir jo brolio Šimio žinioje, karaliaus Ezekijo ir Azarijo, kuris buvo vyriausiasis Dievo namuose, įsakymu. 
\par 14 Imnos sūnui Korei, levitui, rytinių vartų sargui, buvo paskirta prižiūrėti Dievui laisva valia atnešamas dovanas, aukas Viešpačiui ir šventas dovanas. 
\par 15 Korei vadovaujant, Edenas, Minjaminas, Ješūva, Šemajas, Amarijas ir Šechanijas kunigų miestuose paskirstydavo dalis savo broliams, dideliems ir mažiems, 
\par 16 surašytiems giminių sąrašuose, vyrams nuo trejų metų ir vyresniems, kurie kasdien eidavo į Viešpaties namus atlikti tarnystės, laikydamiesi nustatytos eilės, 
\par 17 kunigams, kurie buvo surašyti šeimomis, ir levitams, dvidešimties metų ir vyresniems, pagal jų pareigas skyriuose, 
\par 18 ir visiems, kurie buvo surašyti: kūdikiams, žmonoms, sūnums ir dukterims, nes jie ištikimai pasišventė šventai tarnystei. 
\par 19 Kunigai aaronitai, gyvenantieji jų miestams priklausančiuose priemiesčiuose, buvo įtraukti į sąrašus vardais. Pagal juos duodavo dalį kiekvienam kunigui ir kiekvienam levitui, esančiam sąraše. 
\par 20 Taip darė Ezekijas visame Jude. Jis darė tai, kas teisinga, gera ir tinkama Viešpačiui, jo Dievui. 
\par 21 Kiekvieną darbą, kurį jis darė Dievo namų reikalams, vykdydamas įsakymus ar įstatymus, ieškodamas savo Dievo, jis darė iš visos savo širdies, ir jam sekėsi.



\chapter{32}


\par 1 Po šitų įvykių ir darbų Asirijos karalius Sanheribas įsibrovė į Judą ir apgulė sutvirtintus miestus, tikėdamasis juos paimti. 
\par 2 Kai Ezekijas pamatė, kad Sanheribas rengiasi pulti Jeruzalę, 
\par 3 jis pasitarė su kunigaikščiais bei karžygiais ir nutarė užversti už miesto esančius šaltinius. 
\par 4 Buvo sušaukta daug žmonių, kurie užvertė visus šaltinius ir upelį, tekantį per kraštą, kad Asirijos kariai, apgulę miestą, neturėtų vandens. 
\par 5 Be to, karalius ryžtingai ėmėsi darbo ir atstatė visą apgriuvusią miesto sieną; pastatė ant jos bokštų, jos lauko pusėje pastatė kitą sieną ir sutvirtino Miloją Dovydo mieste. Jis pagamino daug iečių ir skydų. 
\par 6 Ezekijas, paskyręs karo vadus, sušaukė juos aikštėje prie miesto vartų ir jiems kalbėjo padrąsinančiai: 
\par 7 “Būkite drąsūs ir stiprūs. Nebijokite ir nenusigąskite Asirijos karaliaus ir tos daugybės, kuri yra su juo, nes su mumis yra daugiau, negu su juo: 
\par 8 su juo yra kūno ranka, o su mumis­Viešpats, mūsų Dievas, kad padėtų mums ir kovotų mūsų kovas”. Žmones įkvėpė Judo karaliaus Ezekijo žodžiai. 
\par 9 Asirijos karalius Sanheribas buvo apgulęs Lachišą. Jis siuntė pasiuntinius į Jeruzalę pas Judo karalių Ezekiją ir pas visus Judo ir Jeruzalės gyventojus, sakydamas: 
\par 10 “Taip sako Asirijos karalius Sanheribas: ‘Kuo jūs pasitikite, kad sėdite apgultame mieste, Jeruzalėje? 
\par 11 Ar ne Ezekijas įtikinėja jus atsiduoti mirčiai nuo bado ir troškulio, sakydamas: ‘Viešpats, mūsų Dievas, mus išgelbės iš Asirijos karaliaus rankos?’ 
\par 12 Argi ne tas pats Ezekijas pašalino Jo aukštumas bei Jo aukurus ir liepė Judo ir Jeruzalės gyventojams: ‘Tik prie to vieno aukuro turite garbinti Viešpatį ir Jam smilkyti?’ 
\par 13 Argi nežinote, ką aš ir mano tėvai padarėme visoms kitų kraštų tautoms? Ar tų tautų dievai išgelbėjo savo kraštus iš mano rankos? 
\par 14 Kur yra dievai tų tautų, kurias visiškai sunaikino mano tėvai? Ar nors vienas dievas išgelbėjo savo kraštą iš mano rankos? Kaip tad jūsų Dievas galėtų jus išgelbėti? 
\par 15 Tegul Ezekijas neapgaudinėja ir neįtikinėja jūsų! Netikėkite juo, nes nei vienos tautos, nei karalystės dievas negalėjo išgelbėti savo tautos iš mano ir iš mano tėvų rankos. Ir jūsų Dievas neišgelbės jūsų’ ”. 
\par 16 Asirijos pasiuntiniai dar daugiau kalbėjo prieš Viešpatį Dievą ir prieš Jo tarną Ezekiją. 
\par 17 Jų karalius rašė laiškus, įžeidžiančius Viešpatį, Izraelio Dievą: “Kaip kitų tautų dievai neišgelbėjo savo tautų iš mano rankų, taip ir Ezekijo Dievas neišgelbės savo tautos”. 
\par 18 Pasiuntiniai garsiai šaukė žydų kalba žmonėms, buvusiems ant Jeruzalės sienos, norėdami juos įbauginti, kad galėtų paimti miestą. 
\par 19 Taip kalbėdami apie Jeruzalės Dievą, jie Jį lygino su kitų žemės tautų dievais, kurie yra žmogaus rankų darbas. 
\par 20 Karalius Ezekijas ir pranašas Izaijas, Amoco sūnus, meldėsi ir šaukėsi dangaus. 
\par 21 Viešpats siuntė angelą, kuris išnaikino karžygius, vadus ir karininkus Asirijos karaliaus stovykloje. Karalius sugėdintas turėjo grįžti į savo šalį. Jam įėjus į savo dievo namus, jo paties sūnūs jį ten nužudė. 
\par 22 Taip Viešpats išgelbėjo Ezekiją ir Jeruzalės gyventojus iš Asirijos karaliaus Sanheribo rankos ir kitų priešų. Viešpats saugojo juos iš visų pusių. 
\par 23 Daugelis nešė dovanas į Jeruzalę Viešpačiui ir brangenybių Judo karaliui Ezekijui, kuris pagarsėjo aplinkinėse tautose. 
\par 24 Tuo laiku Ezekijas mirtinai susirgo. Jis meldėsi, Viešpats išklausė jį ir davė jam ženklą. 
\par 25 Tačiau Ezekijas nebuvo dėkingas už jam suteiktą Dievo malonę; jis pasididžiavo, todėl rūstybė užsidegė prieš jį, Judą ir Jeruzalę. 
\par 26 Bet Ezekijas nusižemino dėl savo išdidumo ir visi Jeruzalės gyventojai su juo, todėl Viešpaties rūstybė neatėjo ant jų Ezekijo dienomis. 
\par 27 Ezekijas įsigijo garbės ir daug turtų: sidabro, aukso, brangiųjų akmenų, kvepiančių aliejų, skydų ir visokių brangių indų. 
\par 28 Jam valdant, buvo pastatyta daug sandėlių javams, vyno ir aliejaus atsargoms ir tvartų visokiems gyvuliams, 
\par 29 nes jis laikė dideles bandas galvijų ir avių. Jis pastatė daug miestų, nes Dievas jam suteikė labai daug turtų. 
\par 30 Ezekijas užtvenkė aukštutinę Gihono šaltinio vandens ištaką ir jos vandenį nuvedė kanalu į vakarinę Dovydo miesto dalį. Ezekijui sekėsi visuose jo darbuose. 
\par 31 Tačiau kai Babilono kunigaikščių pasiuntiniai atvyko pas Ezekiją teirautis apie stebuklą, kuris įvyko krašte, Dievas paliko jį, norėdamas išmėginti ir sužinoti, kas yra jo širdyje. 
\par 32 Visi kiti Ezekijo darbai ir jo geradarystės yra surašyti pranašo Izaijo, Amoco sūnaus, regėjime ir Judo bei Izraelio karalių knygoje. 
\par 33 Ezekijas užmigo prie savo tėvų ir buvo palaidotas geriausioje Dovydo sūnų kapų vietoje; visi Judo ir Jeruzalės gyventojai pagerbė mirusįjį. Jo sūnus Manasas karaliavo jo vietoje.



\chapter{33}


\par 1 Manasas, pradėdamas karaliauti, buvo dvylikos metų ir karaliavo Jeruzalėje penkiasdešimt penkerius metus. 
\par 2 Jis darė pikta Viešpaties akyse, mėgdžiodamas bjaurius papročius pagonių, kuriuos Viešpats išvarė, atiduodamas izraelitams kraštą. 
\par 3 Jis atstatė aukštumas, kurias jo tėvas Ezekijas buvo nugriovęs, pastatė aukurų Baalui, pasodino giraičių ir garbino visą dangaus kareiviją, ir jiems tarnavo. 
\par 4 Jis pastatė aukurų net Viešpaties namuose, apie kuriuos Viešpats buvo pasakęs: “Jeruzalėje mano vardas bus per amžius”. 
\par 5 Buvo pastatyti aukurai dangaus kareivijai garbinti dviejuose Viešpaties namų kiemuose. 
\par 6 Jis leido savo vaikus per ugnį Ben Hinomo slėnyje. Be to, jis žyniavo, būrė iš ženklų, kerėjo ir laikė mirusiųjų dvasių iššaukėjus bei žynius. Jis darė daug pikto Viešpaties akyse, sukeldamas Jo rūstybę. 
\par 7 Jis pastatė drožtą atvaizdą, stabą Dievo namuose, apie kuriuos Dievas kalbėjo Dovydui ir jo sūnui Saliamonui: “Šituose namuose ir Jeruzalėje, kurią išsirinkau iš visų Izraelio giminių, per amžius bus mano vardas. 
\par 8 Aš nepašalinsiu Izraelio tautos iš krašto, kurį daviau jūsų tėvams, jei jie rūpestingai laikysis mano įstatymų, įsakymų ir nuostatų, duotų jiems per Mozę”. 
\par 9 Bet Manasas suvedžiojo Judo ir Jeruzalės gyventojus taip, kad jie elgėsi blogiau negu pagonys, kuriuos Viešpats išnaikino izraelitų akivaizdoje. 
\par 10 Viešpats kalbėjo Manasui ir jo tautai, tačiau jie nekreipė dėmesio. 
\par 11 Viešpats leido Asirijos karaliaus kariuomenės vadams užimti kraštą; jie sugavo Manasą, sukaustė jį grandinėmis ir nuvedė į Babiloną. 
\par 12 Būdamas nelaisvėje, jis nusižemino prieš savo tėvų Dievą ir maldavo Viešpatį, savo Dievą. 
\par 13 Viešpats išgirdo jo prašymą, išklausė jo maldavimą ir leido jam grįžti į Jeruzalę, į jo karalystę. Tada Manasas suprato, kad Viešpats yra Dievas. 
\par 14 Po to jis pastatė labai aukštą išorinę sieną Dovydo miestui Gihono šaltinio vakaruose, slėnyje, iki Žuvų vartų ir aplink Ofelio kalvą. Jis taip pat paskyrė kariuomenės vadus kiekvienam sutvirtintam Judo miestui. 
\par 15 Jis pašalino svetimus dievus ir stabą iš Viešpaties namų bei visus aukurus, kuriuos buvo pastatęs Viešpaties namų kalne bei Jeruzalėje, ir išmetė juos už miesto vartų. 
\par 16 Jis atstatė Viešpaties aukurą, aukojo ant jo sutaikinimo ir padėkos aukas ir įsakė Judui tarnauti Viešpačiui, Izraelio Dievui. 
\par 17 Žmonės vis dar aukojo aukštumose, tačiau tik Viešpačiui, savo Dievui. 
\par 18 Kiti Manaso darbai, jo malda į savo Dievą ir žodžiai regėtojų, kurie jam kalbėjo Viešpaties, Izraelio Dievo, vardu, yra surašyti Izraelio karalių knygoje. 
\par 19 Jo malda, Viešpaties atsakymas, nusikaltimas, neištikimybė ir vietos, kuriose jis prieš nusižemindamas įkūrė giraičių, aukštumų ir drožtų atvaizdų, yra surašyta regėtojų raštuose. 
\par 20 Manasas užmigo prie savo tėvų ir jį palaidojo jo namuose; jo sūnus Amonas karaliavo jo vietoje. 
\par 21 Amonas pradėjo karaliauti, būdamas dvidešimt dvejų metų, ir karaliavo Jeruzalėje dvejus metus. 
\par 22 Jis, kaip ir jo tėvas Manasas, darė pikta Viešpaties akyse, aukodamas ir tarnaudamas visiems stabams, kuriuos padarė jo tėvas Manasas. 
\par 23 Jis nenusižemino prieš Viešpatį kaip jo tėvas Manasas, bet nusikalto labiau ir labiau. 
\par 24 Jo tarnai surengė sąmokslą ir nužudė jį jo namuose. 
\par 25 Krašto žmonės nužudė visus, kurie dalyvavo sąmoksle prieš karalių Amoną, ir paskelbė karaliumi jo sūnų Joziją.



\chapter{34}

\par 1 Jozijas pradėjo karaliauti aštuonerių metų ir trisdešimt vienerius metus karaliavo Jeruzalėje. 
\par 2 Jis darė tai, kas teisinga Viešpaties akyse, ir vaikščiojo savo tėvo Dovydo keliais, nenukrypdamas nei į kairę, nei į dešinę. 
\par 3 Aštuntaisiais savo karaliavimo metais, dar būdamas jaunuolis, jis pradėjo ieškoti savo tėvo Dovydo Dievo, o dvyliktaisiais metais jis pradėjo valyti Judą ir Jeruzalę nuo aukštumų, giraičių, drožtų ir lietų atvaizdų. 
\par 4 Jie nugriovė aukurus Baalui ir atvaizdus, kurie buvo virš jų, iškirto giraites, drožtus bei lietus atvaizdus sutrupino į gabalus ir dulkes išbarstė ant kapų tų, kurie jiems aukojo. 
\par 5 Kunigų kaulus jis sudegino ant jų aukurų ir apvalė Judą bei Jeruzalę. 
\par 6 Tą patį jis padarė Manaso, Efraimo ir Simeono miestuose iki Neftalio: 
\par 7 sugriovė visus aukurus, iškirto giraites, drožtus atvaizdus sudaužė į dulkes ir sunaikino visus stabus Izraelyje. Po to jis grįžo į Jeruzalę. 
\par 8 Aštuonioliktaisiais savo karaliavimo metais, baigęs valyti kraštą ir namus, jis siuntė Acalijo sūnų Šafaną, miesto valdytoją Maasėją ir metraštininką Joachą, Jehoachazo sūnų, taisyti Viešpaties namų. 
\par 9 Jie, atėję pas vyriausiąjį kunigą Helkiją, atnešė jam Dievo namams paaukotus pinigus, kuriuos levitai, durų sargai, surinko iš Manaso, Efraimo ir viso Izraelio, taip pat iš Judo, Benjamino ir Jeruzalės gyventojų. 
\par 10 Jie perdavė juos darbų prižiūrėtojams Viešpaties namuose, o tie atidavė darbininkams, kurie tvarkė Viešpaties namus­ 
\par 11 dailidėms ir statybininkams pirkti tašytus akmenis bei medžius sijoms ir perdangoms, nes Judo karaliai buvo apleidę Viešpaties namus. 
\par 12 Tie vyrai dirbo ištikimai. Jų prižiūrėtojais buvo levitai iš Merario sūnų Jahatas ir Abdijas, iš kehatų­Zacharija ir Mešulamas. Levitai, kurie grojo muzikos instrumentais, 
\par 13 vadovavo nešikams ir prižiūrėjo visus darbininkus; iš levitų buvo raštininkai, tvarkytojai ir vartų sargai. 
\par 14 Jiems išimant Viešpaties namams paaukotus pinigus, kunigas Helkijas rado Viešpaties įstatymo, duoto per Mozę, knygą. 
\par 15 Helkijas tarė raštininkui Šafanui: “Radau įstatymo knygą Viešpaties namuose”. Helkijas padavė knygą Šafanui. 
\par 16 Šafanas, nunešęs knygą pas karalių, pranešė, kad viskas daroma, kas buvo jiems pavesta, 
\par 17 o pinigus, rastus Viešpaties namuose, jie atidavė prižiūrėtojams ir darbininkams. 
\par 18 Tada raštininkas Šafanas pranešė karaliui: “Kunigas Helkijas davė man knygą”. Ir Šafanas skaitė ją karaliui. 
\par 19 Karalius, išgirdęs įstatymo žodžius, perplėšė savo drabužius 
\par 20 ir įsakė Helkijui, Šafano sūnui Ahikamui, Michėjo sūnui Abdonui, raštininkui Šafanui ir karaliaus tarnui Asajai: 
\par 21 “Eikite ir pasiklauskite Viešpatį už mane ir visus likusius Izraelyje bei Jude dėl šitos knygos žodžių; didelė Viešpaties rūstybė bus išlieta ant mūsų, nes mūsų tėvai nesilaikė Viešpaties žodžio ir nevykdė, kas parašyta šitoje knygoje”. 
\par 22 Helkijas ir tie, kuriems karalius buvo įsakęs, nuėjo pas pranašę Huldą (jos vyras Šalumas, Hasros sūnaus Tikvo sūnus, buvo drabužių sargas), kuri gyveno Jeruzalės antroje dalyje, ir kalbėjo su ja apie tai. 
\par 23 Ji atsakė: “Taip sako Viešpats, Izraelio Dievas: ‘Sakykite vyrui, kuris jus atsiuntė pas mane: 
\par 24 ‘Taip sako Viešpats: ‘Aš bausiu šitą vietą ir jos gyventojus visais prakeikimais, kurie surašyti knygoje, kuri buvo perskaityta Judo karaliui; 
\par 25 jie paliko mane ir degino smilkalus kitiems dievams, sukeldami mano pyktį savo rankų darbais. Mano rūstybė bus išlieta ant šitos vietos ir neužges’. 
\par 26 Judo karaliui, kuris jus siuntė pasiklausti Viešpaties, pasakykite: ‘Taip sako Viešpats, Izraelio Dievas: ‘Dėl žodžių, kuriuos tu girdėjai, 
\par 27 tavo širdis buvo minkšta ir tu nusižeminai prieš Dievą, klausydamasis žodžių prieš šitą vietą ir jos gyventojus. Kadangi nusižeminai prieš mane, perplėšei savo drabužius ir verkei, tai Aš išklausiau tave. 
\par 28 Aš paimsiu tave prie tavo tėvų, tu nueisi ramybėje į kapus ir tavo akys nematys visų tų nelaimių, kurias siųsiu šitai vietai ir jos gyventojams’ ”. Jie visa tai pranešė karaliui. 
\par 29 Karalius sukvietė visus Judo ir Jeruzalės vyresniuosius. 
\par 30 Karalius nuėjo į Viešpaties namus, ir visi Judo bei Jeruzalės gyventojai nuo mažiausio iki didžiausio, ir kunigai bei levitai. Karalius garsiai skaitė visus žodžius iš sandoros knygos, atrastos Viešpaties namuose. 
\par 31 Karalius, stovėdamas savo vietoje, padarė sandorą Viešpaties akivaizdoje: sekti Viešpatį ir laikytis Jo įsakymų, įspėjimų ir nuostatų; visa širdimi ir visa siela vykdyti sandoros žodžius, užrašytus šitoje knygoje. 
\par 32 Jis įsakė laikytis sandoros visiems gyvenantiems Jeruzalėje ir Benjamine. Jeruzalės gyventojai laikėsi savo tėvų Dievo sandoros. 
\par 33 Jozijas pašalino visus stabus iš visų Izraelio kraštų ir įsakė visiems Izraelyje tarnauti Viešpačiui, jų tėvų Dievui. Kol jis buvo gyvas, jie nepasitraukė nuo Viešpaties.



\chapter{35}


\par 1 Jozijas šventė Jeruzalėje Paschą Viešpaties garbei. Jie papjovė Paschos avinėlį pirmojo mėnesio keturioliktą dieną. 
\par 2 Jis paskirstė kunigams pareigas ir juos padrąsino tarnauti Viešpaties namuose. 
\par 3 Jis sakė levitams, kurie mokė izraelitus ir buvo pasišventę Viešpačiui: “Įneškite šventąją skrynią į namus, kuriuos pastatė Dovydo sūnus Saliamonas, Izraelio karalius; jums nebereikia jos nešioti ant pečių. Tarnaukite Viešpačiui, savo Dievui, ir Jo tautai Izraeliui. 
\par 4 Pasiruoškite pagal savo tėvų namus ir skyrius, kaip nustatyta Izraelio karaliaus Dovydo ir jo sūnaus Saliamono. 
\par 5 Stovėkite šventykloje ten, kur sustojusios jūsų brolių tėvų šeimos, kad kiekvienai šeimai tektų levitų šeimos skyrius; 
\par 6 papjaukite Paschos avinėlį, pasišventinkite ir paruoškite jį savo broliams, kad jie galėtų išpildyti tai, ką Viešpats kalbėjo Mozei”. 
\par 7 Jozijas davė tautai trisdešimt tūkstančių avinėlių Paschos aukai ir tris tūkstančius jaučių iš karaliaus turto. 
\par 8 Kunigaikščiai noriai davė žmonėms, kunigams ir levitams. Dievo namų vyresnieji: Helkijas, Zacharijas ir Jehielis davė kunigams du tūkstančius šešis šimtus avių ir tris šimtus jaučių aukojimui. 
\par 9 Konanijas ir jo broliai Šemajas bei Netanelis, taip pat Hašabijas, Jejelis ir Jehozabadas, levitų viršininkai, davė levitams penkis tūkstančius avinėlių Paschos aukai ir penkis šimtus jaučių. 
\par 10 Tarnavimo metu kunigai sustojo savo vietose ir levitai pagal savo skyrius, kaip karalius įsakė. 
\par 11 Jie papjovė Paschai avinėlį, ir kunigai ėmė kraują iš jų, šlakstė, o levitai nulupo kailį. 
\par 12 Kas buvo skirta deginamajai aukai, buvo atidėta į šalį ir paskirstyta tautos šeimoms, kad jie aukotų Viešpačiui, kaip parašyta Mozės knygoje; taip pat paskirstė ir galvijus. 
\par 13 Jie kepė Paschos avinėlį ugnyje, kaip reikalavo nuostatai, o kitas šventas aukas virė puoduose bei katiluose ir skubiai išdalino visiems žmonėms. 
\par 14 Po to jie paruošė maistą sau ir kunigams, nes kunigai, aaronitai, iki nakties degino deginamąsias aukas ir taukus. 
\par 15 Giedotojai, Asafo sūnūs, stovėjo savo vietoje, kaip buvo įsakę Dovydas, Asafas, Hemanas ir Jedutūnas, karaliaus regėtojas; vartininkai stovėjo prie jiems skirtų vartų, jiems nereikėjo palikti tarnavimo, nes jų broliai levitai jiems viską paruošė. 
\par 16 Taip tą dieną buvo atliktas visas tarnavimas Viešpačiui: Paschos šventimas ir deginamųjų aukų aukojimas ant Viešpaties aukuro, laikantis karaliaus Jozijo įsakymo. 
\par 17 Izraelitai, kurie buvo susirinkę, šventė Paschą ir Neraugintos duonos šventę septynias dienas. 
\par 18 Tokios Paschos Izraelis nebuvo šventęs nuo pranašo Samuelio dienų; joks Izraelio karalius nebuvo šventęs Paschos taip, kaip Jozijas­su kunigais, levitais, visu Judu, Izraeliu ir Jeruzalės gyventojais. 
\par 19 Ta Pascha buvo švęsta aštuonioliktaisiais Jozijo karaliavimo metais. 
\par 20 Jozijui atstačius šventyklą, Egipto karalius Nechas išėjo prieš Karchemišą prie Eufrato. Jozijas išėjo priešais jį. 
\par 21 Nekojas siuntė pas jį pasiuntinių, sakydamas: “Kas man ir tau, Judo karaliau? Ne prieš tave einu šiandien, bet prieš tą, su kuriuo kariauju. Dievas įsakė man skubėti. Nesipriešink Dievui, kuris yra su manimi, kad Jis tavęs nepražudytų”. 
\par 22 Tačiau Jozijas nekreipė dėmesio į jo kalbą ir pasiruošė kovai. Jis nepaklausė Nekojo žodžių, nors jie buvo iš Dievo lūpų, ir atėjo kariauti į Megido lygumą. 
\par 23 Šauliai pataikė į karalių Joziją, ir jis įsakė savo tarnams: “Išvežkite mane iš kovos lauko, nes aš esu sunkiai sužeistas”. 
\par 24 Jo tarnai, perkėlę jį iš kovos vežimo į kitą, nugabeno į Jeruzalę, kur jis mirė. Jį palaidojo jo tėvų kapuose. Visas Judas ir Jeruzalė apraudojo Joziją. 
\par 25 Ir Jeremijas apraudojo Joziją, ir visi giedotojai ir giedotojos mini Joziją savo raudose iki šios dienos, ir tai tapo nuostatu Izraeliui. Jos yra surašytos raudų knygoje. 
\par 26 Kiti Jozijo darbai ir jo geradarystės vykdant Viešpaties įstatymus 
\par 27 nuo pradžios iki galo surašyti Izraelio ir Judo karalių knygoje.



\chapter{36}

\par 1 Tada krašto žmonės paskelbė karaliumi Jozijo sūnų Jehoachazą jo tėvo vieton Jeruzalėje. 
\par 2 Jehoachazas, pradėdamas karaliauti, buvo dvidešimt trejų metų ir tris mėnesius karaliavo Jeruzalėje. 
\par 3 Egipto karalius pašalino jį nuo sosto Jeruzalėje ir uždėjo kraštui piniginę duoklę: šimtą talentų sidabro ir talentą aukso. 
\par 4 Egipto karalius padarė Jehoachazo brolį Eliakimą Judo karaliumi Jeruzalėje ir pakeitė jo vardą į Jehojakimą. Jo brolį Jehoachazą Nekojas nusivedė į Egiptą. 
\par 5 Jehojakimas, pradėdamas karaliauti, buvo dvidešimt penkerių metų ir vienuolika metų karaliavo Jeruzalėje. Jis darė pikta Viešpaties, savo Dievo, akyse. 
\par 6 Prieš jį atėjo Babilono karalius Nebukadnecaras, sukaustė jį grandinėmis ir nusivedė į Babiloną. 
\par 7 Nebukadnecaras parsigabeno į Babiloną ir Viešpaties namų indus ir padėjo juos savo šventykloje. 
\par 8 Visi kiti Jehojakimo darbai ir jo nusikaltimai yra surašyti Izraelio ir Judo karalių knygoje. Jo sūnus Joachinas karaliavo jo vietoje. 
\par 9 Jehojachinas, pradėdamas karaliauti, buvo aštuoniolikos metų ir karaliavo Jeruzalėje tris mėnesius ir dešimt dienų. Jis darė pikta Viešpaties akyse. 
\par 10 Pavasarį karalius Nebukadnecaras atsiuntė ir išvedė jį į Babiloną kartu su brangiausiais Viešpaties namų indais. Jo brolį Zedekiją paskyrė Judo karaliumi. 
\par 11 Zedekijas pradėjo karaliauti, būdamas dvidešimt vienerių metų, ir karaliavo Jeruzalėje vienuolika metų. 
\par 12 Jis darė pikta Viešpaties, savo Dievo, akyse ir nenusižemino prieš pranašą Jeremiją, kuris kalbėjo Viešpaties žodžius. 
\par 13 Be to, jis dar sukilo prieš karalių Nebukadnecarą, kuris jį buvo prisaikdinęs Dievo vardu. Jis tapo kietasprandis ir kietaširdis ir nesigręžė į Viešpatį, Izraelio Dievą. 
\par 14 Visi vyresnieji kunigai ir tauta labai nusikalto. Jie mėgdžiojo pagonių bjaurystes ir suteršė Viešpaties namus, kuriuos Jis buvo pašventinęs Jeruzalėje. 
\par 15 Viešpats, jų tėvų Dievas, nuo ankstaus ryto siuntė pas juos savo pasiuntinius, nes Jis gailėjosi savo tautos ir savo buveinės. 
\par 16 Bet jie tyčiojosi iš Dievo ir Jo pasiuntinių, niekino Jo pranašus ir Jo žodžius, kol pagaliau Viešpaties rūstybė išsiliejo tautai ir nebebuvo išsigelbėjimo. 
\par 17 Jis atvedė prieš juos chaldėjų karalių, kuris išžudė Judo jaunuolius šventyklos namuose ir nepagailėjo nei jaunuolio, nei mergaitės, nei seno; visus Jis atidavė į jo rankas. 
\par 18 Taip pat visus Dievo namų reikmenis, didelius ir mažus, ir Viešpaties namų, karaliaus bei jo vyresniųjų turtus jis išgabeno į Babiloną. 
\par 19 Po to jie sudegino Dievo namus, sugriovė Jeruzalės sienas, sudegino visus rūmus, o brangius daiktus sunaikino. 
\par 20 Išlikusius gyvus išsivedė į Babiloną; jie tapo jo ir jo sūnų vergais iki persų karalystės įsigalėjimo, 
\par 21 kad įvyktų Viešpaties žodis, paskelbtas Jeremijo. Visą tą laiką žemė buvo negyvenama ir ilsėjosi septyniasdešimt metų. 
\par 22 Kad įvyktų Viešpaties žodis, paskelbtas Jeremijo, pirmaisiais persų karaliaus Kyro metais Viešpats paragino persų karalių Kyrą, kad jis visoje savo karalystėje paskelbtų žodžiu ir raštu: 
\par 23 “Taip sako persų karalius Kyras: ‘Visas žemės karalystes man atidavė Viešpats, dangaus Dievas; Jis man pavedė atstatyti Jo namus Jeruzalėje, kuri yra Jude. Kas iš jūsų yra iš Jo tautos, Viešpats, jo Dievas, tebūna su juo ir jis teeina’ ”.



\end{document}