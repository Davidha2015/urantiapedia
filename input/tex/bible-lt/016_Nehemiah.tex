\begin{document}

\title{Nehemijo knyga}

\chapter{1}


\par 1 Nehemijo, Hachalijos sūnaus, žodžiai. Dvidešimtaisiais metais, Kislevo mėnesį, aš buvau sostinėje Sūzuose. 
\par 2 Vienas iš mano brolių Hananis atvyko su keliais vyrais iš Judo. Aš juos klausinėjau apie žydus, kurie nebuvo ištremti, ir apie Jeruzalę. 
\par 3 Jie man atsakė: “Išvengę tremties, jie gyvena dideliame varge ir yra niekinami. Jeruzalės sienos sugriautos ir vartai sudeginti”. 
\par 4 Išgirdęs šituos žodžius, atsisėdau, verkiau ir liūdėjau ištisas dienas; pasninkavau ir meldžiausi prieš dangaus Dievą: 
\par 5 “Viešpatie, didis ir galingas dangaus Dieve, kuris laikaisi sandoros ir gailestingumo su tais, kurie Tave myli ir vykdo Tavo įsakymus. 
\par 6 Pažvelk į mane ir išgirsk savo tarno maldą. Meldžiuosi dieną ir naktį Tavo akivaizdoje už izraelitus, Tavo tarnus, išpažįstu jų nuodėmes, kuriomis mes visi­aš ir mano tėvai­Tau nusidėjome. 
\par 7 Mes elgėmės netinkamai prieš Tave ir nesilaikėme Tavo įsakymų, nuostatų ir potvarkių, kuriuos Tu davei savo tarnui Mozei. 
\par 8 Meldžiu, prisimink žodžius, kuriuos sakei savo tarnui Mozei: ‘Jei jūs nusikalsite, Aš jus išsklaidysiu tarp tautų, 
\par 9 o jei sugrįšite pas mane, laikysitės mano įsakymų ir juos vykdysite, tai Aš jus iš ten surinksiu ir parvesiu į vietą, kurią išrinkau savo vardui, nors jūsų išsklaidytieji būtų toliausiame žemės pakraštyje’. 
\par 10 Jie yra Tavo tarnai ir Tavo tauta, kurią išgelbėjai savo didele jėga ir galinga ranka. 
\par 11 Viešpatie, meldžiu, tebūna Tavo ausis atidi mano maldai, išklausyk maldas savo tarnų, kurie bijosi Tavo vardo, siųsk šiandien savo tarnui sėkmę ir suteik jam malonę to žmogaus akyse”. Aš tuo laiku buvau karaliaus vyno pilstytojas.


\chapter{2}

\par 1 Dvidešimtaisiais karaliaus Artakserkso metais, Nisano mėnesį, aš pakėliau vyno taurę ir ją padaviau karaliui. Kadangi aš niekada nebūdavau nuliūdęs jo akivaizdoje, 
\par 2 karalius klausė manęs: “Kodėl tavo veidas nusiminęs? Juk nesergi. Tai ne kas kita, kaip širdies skausmas”. Aš labai nusigandau 
\par 3 ir tariau karaliui: “Tegyvuoja karalius amžinai! Kaip man neliūdėti, kai miestas, kuriame yra mano tėvų kapai, apleistas ir jo vartai ugnies sunaikinti?” 
\par 4 Karalius klausė: “Ko norėtum?” Aš meldžiausi dangaus Dievui 
\par 5 ir tariau karaliui: “Jei karaliui patiktų ir tavo tarnas surastų malonę tavo akyse, siųsk mane į Judą, į mano tėvų kapų miestą, kad galėčiau jį atstatyti”. 
\par 6 Karalius, karalienei sėdint šalia jo, sakė man: “Kiek užtruksi kelionėje ir kada sugrįši?” Karaliui patiko mane pasiųsti, ir aš nurodžiau jam laiką. 
\par 7 Be to, aš sakiau karaliui: “Jei karaliui patiktų, tegul duoda man laiškų valdytojams anapus upės, kad jie mane praleistų vykti į Judą, 
\par 8 ir laišką Asafui, karaliaus girininkui, kad jis man duotų medžių rūmų vartams, miesto sienai ir namams, kuriuose apsigyvensiu”. Karalius davė man, ko prašiau, nes gera Dievo ranka buvo ant manęs. 
\par 9 Nuvykęs pas valdytojus anapus upės, įteikiau jiems karaliaus laiškus; kartu su manimi buvo karaliaus pasiųsti kariuomenės vadai ir raiteliai. 
\par 10 Sanbalatas iš Horono ir Tobija, tarnas iš Amono, sužinoję, kad atvyko besirūpinąs izraelitų gerove žmogus, buvo labai nepatenkinti. 
\par 11 Atvykęs į Jeruzalę ir išbuvęs tris dienas, 
\par 12 atsikėliau naktį, su manimi atsikėlė ir keli vyrai. Niekam nesakiau, ką Dievas buvo įdėjęs man į širdį daryti Jeruzalėje. Su savimi turėjau tik gyvulį, ant kurio jojau. 
\par 13 Naktį aš išjojau pro Slėnio vartus priešais Slibino versmę ir jojau Šiukšlių vartų link, apžiūrėdamas Jeruzalės sugriautas sienas ir ugnies sunaikintus vartus. 
\par 14 Prijojus prie Versmės vartų ir Karaliaus tvenkinio, nebuvo vietos, kur galėtų praeiti gyvulys, kuriuo jojau. 
\par 15 Aš ėjau palei upelį, apžiūrėdamas sieną ir grįžau atgal pro Slėnio vartus. 
\par 16 Valdininkai nežinojo, kur aš buvau ir ką dariau. Aš iki šiol niekam nebuvau sakęs: nei žydams, nei kunigams, nei kilmingiesiems, nei valdininkams, nei tiems, kurie turėjo dirbti. 
\par 17 Tada jiems tariau: “Jūs matote, kokiame varge esame; Jeruzalė apleista ir jos vartai sudeginti. Atstatykime Jeruzalės sieną, kad nebūtų mums pažeminimo”. 
\par 18 Ir papasakojau jiems, kaip mano Dievo ranka buvo su manimi; taip pat papasakojau, ką karalius man kalbėjo. Jie sakė: “Pakilkime ir statykime”. Taip jie sutvirtino savo rankas geram darbui. 
\par 19 Kai Sanbalatas iš Horono, Tobija, tarnas iš Amono, ir arabas Gešemas apie tai išgirdo, jie tyčiojosi iš mūsų ir niekino mus, sakydami: “Ką jūs čia dirbate? Ar norite sukilti prieš karalių?” 
\par 20 Jiems atsakiau: “Dangaus Dievas duos mums sėkmę, o mes, Jo tarnai, statysime. Bet jūs neturite dalies, teisės nė atminimo Jeruzalėje”.



\chapter{3}

\par 1 Tuomet vyriausiasis kunigas Eljašibas su savo broliais kunigais pirmasis pradėjo darbą ir atstatė Avių vartus; jie juos atnaujino, įstatė jiems duris ir pašventino juos iki Mejos bokšto ir toliau iki Hananelio bokšto. 
\par 2 Šalia jo statė Jericho vyrai, o po jų­Imrio sūnus Zakūras. 
\par 3 Žuvų vartus statė Hasenajo sūnūs; jie sudėjo sijas, įstatė duris, padarė užraktus ir užkaiščius. 
\par 4 Šalia jų sienos dalį taisė Hakoco sūnaus Ūrijos sūnus Meremotas. Už jo taisė Mešezabelio sūnaus Berechijos sūnus Mešulamas, toliau­Baanos sūnus Cadokas. 
\par 5 Šalia jų dirbo tekojiečiai, atstatydami sieną, tačiau jų kilmingieji nepalenkė savo sprandų prie Viešpaties darbo. 
\par 6 Senuosius vartus statė Paseacho sūnus Jehojada ir Besodijos sūnus Mešulamas; jie sudėjo sijas, įstatė duris, padarė užraktus ir užkaiščius. 
\par 7 Greta jų dirbo šios upės pusės valdytojui pavaldūs: gibeonietis Melatija ir meronotietis Jadonas su Gibeono ir Micpos vyrais. 
\par 8 Šalia jų­Harhajos sūnus Uzielis, auksakalys, ir Hananija, vaistininko sūnus. Jie sutvirtino Jeruzalę iki Plačiosios sienos. 
\par 9 Po jų­Hūro sūnus Refaja, pusės Jeruzalės viršininkas. 
\par 10 Po jų­Harumafo sūnus Jedaja iki savo namų. Šalia jo­Hasabnėjo sūnus Hatušas. 
\par 11 Kitą dalį, taip pat Krosnių bokštą atstatė Harimo sūnus Malkija ir toliau Pahat Moabo sūnus Hašubas. 
\par 12 Po jų­Ha Lohešo sūnus Šalumas, pusės Jeruzalės viršininkas; jis dirbo su dukterimis. 
\par 13 Slėnio vartus sutvarkė Hanūnas ir Zanoacho gyventojai; jie atstatė juos, įstatė duris, sudėjo užraktus ir užkaiščius. Be to, jie dar atstatė tūkstantį uolekčių sienos iki Šiukšlių vartų. 
\par 14 Šiukšlių vartus atstatė, sudėjo užraktus ir užkaiščius Rechabo sūnus Malkija, Bet Keremo srities viršininkas. 
\par 15 Šaltinio vartus sutvarkė Kol Hozės sūnus Šalumas, Micpos srities viršininkas. Jis apdengė stogą, įstatė duris, sudėjo užraktus ir užkaiščius ir atstatė Siloamo tvenkinio sieną prie karaliaus sodo iki laiptų, nusileidžiančių iš Dovydo miesto. 
\par 16 Už jo­Azbuko sūnus Nehemija, pusės Bet Cūro srities viršininkas; jis atstatė iki Dovydo kapų, dirbtinio tvenkinio ir iki karžygių namų. 
\par 17 Už jo dirbo levitai: Banio sūnus Rehumas ir šalia jo Hašabija, pusės Keilos viršininkas. 
\par 18 Po jų statė jų broliai: Henadado sūnus Bavajis, pusės Keilos viršininkas, 
\par 19 ir Ješūvos sūnus Ezeras, Micpos viršininkas; jie taisė sieną iki tako į ginklų sandėlį prie kampo. 
\par 20 Už jo Zabajo sūnus Baruchas uoliai taisė kitą dalį, nuo kampo iki vyriausiojo kunigo Eljašibo namų vartų. 
\par 21 Už jo­Meremotas, Hakoco sūnaus Ūrijos sūnus, nuo Eljašibo namų vartų iki Eljašibo namų galo. 
\par 22 Po jų taisė kunigai, lygumos vyrai. 
\par 23 Už jų­Benjaminas ir Hašubas priešais savo namus. Už jų­Azarijas, Ananijos sūnaus Maasėjos sūnus, greta savo namų. 
\par 24 Už jo­Henadado sūnus Binujis nuo Azarijos namų iki sienos pasisukimo, iki kampo. 
\par 25 Uzajo sūnus Palalas­nuo pasisukimo ir bokšto, kuris išsikiša iš karaliaus aukštutinių namų ties sargybos kiemu. Už jo­Parošo sūnus Pedaja 
\par 26 ir šventyklos tarnai, gyvenantieji Ofelyje; jie atstatė iki Vandens vartų rytuose ir iki išsikišusio bokšto. 
\par 27 Už jo tekojiečiai statė kitą dalį priešingoje išsikišusio didžiojo bokšto pusėje iki Ofelio sienos. 
\par 28 Nuo Arklių vartų statė kunigai, kiekvienas ties savo namais. 
\par 29 Už jų­Imerio sūnus Cadokas ties savo namais. Už jo­Šechanijo sūnus Šemajas, rytinių vartų sargas. 
\par 30 Už jo­Šelemijo sūnus Hananija ir Zalafo šeštasis sūnus Hanūnas. Už jo­Berechijo sūnus Mešulamas ties savo kambariu. 
\par 31 Už jo­auksakalio sūnus Malkija iki šventyklos tarnų ir pirklių namo ties Sargybos vartais ir iki kampo kambario. 
\par 32 O tarp kampo kambario ir Avių vartų statė auksakaliai ir pirkliai.



\chapter{4}

\par 1 Sanbalatas, išgirdęs, kad statoma siena, labai pyko ir tyčiojosi iš žydų, 
\par 2 kalbėdamas su savo broliais ir Samarijos kariuomene: “Ką tie bejėgiai žydai daro? Bene jie Jeruzalę atstatys? Ar jie aukos? Ar jie užbaigs darbą vieną dieną? Ar jie iš sudegusių griuvėsių padarys tinkamus statybai akmenis?” 
\par 3 O amonitas Tobija, stovėdamas šalia jo, sakė: “Tegul stato! Kai lapė užlips, sugrius jų akmeninė siena”. 
\par 4 Dieve, ar girdi, kaip mes esame niekinami? Nukreipk jų priekaištus ant jų galvų, atiduok juos į priešų nelaisvę. 
\par 5 Neatleisk jiems nuodėmių ir nepamiršk jų nusikaltimų, nes jie įžeidė dirbančius. 
\par 6 Mes statėme sieną ir pusę jos jau pastatėme, nes žmonės noriai dirbo. 
\par 7 Sanbalatas, Tobija, arabai, amonitai ir ašdodiečiai, išgirdę, kad Jeruzalės sienos atstatomos ir spragos užtaisomos, labai supyko 
\par 8 ir nutarė kartu kovoti prieš Jeruzalę bei trukdyti darbą. 
\par 9 Mes meldėme Dievą ir pastatėme sargybinius budėti dieną ir naktį. 
\par 10 Judas sakė: “Nešikų jėgos senka, o griuvėsių dar daug. Mes neįstengsime atstatyti sienų”. 
\par 11 O mūsų priešai kalbėjo: “Jie nesužinos ir nepastebės mūsų, kai mes, atsiradę tarp jų, išžudysime juos ir sustabdysime darbą”. 
\par 12 Jų kaimynystėje gyveną žydai ateidavo ir dešimt kartų sakė mums, kad jie ateis iš visų pusių prieš mus. 
\par 13 Žemesnėse ir atvirose vietose už sienos aš pastačiau žmones, ginkluotus kardais, ietimis ir lankais. 
\par 14 Apžiūrėjęs tariau kilmingiesiems, viršininkams ir visiems žmonėms: “Nebijokite jų! Atsiminkite Viešpatį, didingą ir baisų, ir kovokite už savo brolius, sūnus, dukteris, žmonas ir savo namus”. 
\par 15 Mūsų priešai išgirdo, kad tai mums žinoma, ir Dievas pavertė niekais jų planus; o mes visi grįžome prie statybos, kiekvienas prie savo darbo. 
\par 16 Nuo tos dienos pusė mano tarnų dirbo darbą, o kita pusė ėjo sargybą, apsiginklavę ietimis, skydais, lankais ir šarvais; vyresnieji buvo sustoję už visų žydų. 
\par 17 Tie, kurie statė sieną, taip pat ir naštų nešėjai buvo ginkluoti; viena ranka dirbo, o kitoje rankoje laikė ginklą. 
\par 18 Kiekvienas darbininkas nešiojo prisijuosęs kardą, o trimitininkas stovėjo šalia manęs. 
\par 19 Aš pranešiau kilmingiesiems, viršininkams ir visiems žmonėms: “Darbas yra didelis ir platus. Mes pasiskirstę ant sienos toli vienas nuo kito. 
\par 20 Išgirdę trimito garsą, tuojau susirinkite prie mūsų. Mūsų Dievas kariaus už mus!” 
\par 21 Taip mes dirbome, o pusė iš jų laikė rankose ietį nuo aušros iki sutemų. 
\par 22 Tuo metu aš įsakiau žmonėms: “Kiekvienas su savo tarnu tegul nakvoja Jeruzalėje, kad naktį jie eitų sargybą, o dieną dirbtų”. 
\par 23 Nei aš pats, nei mano broliai, nei mano tarnai, nei sargybiniai, lydėję mane, nenusivilkdavo savo drabužių; nusivilkdavo kiekvienas tik tada, kai reikėdavo apsiprausti.



\chapter{5}

\par 1 Kilo didelis žmonių ir jų moterų šauksmas prieš savo brolius žydus. 
\par 2 Vieni sakė: “Mūsų su sūnumis ir dukterimis yra daug. Pirkime grūdus, kad turėtume ką valgyti ir išliktume gyvi!” 
\par 3 Kiti sakė: “Savo laukus, vynuogynus ir namus užstatėme už grūdus, kad apsigintume nuo bado”. 
\par 4 Dar kiti sakė: “Mes turime skolintis pinigų iš karaliaus, užstatydami savo laukus ir vynuogynus. 
\par 5 Mes esame tokie pat, kaip ir mūsų broliai; mūsų vaikai yra tokie pat, kaip ir jų vaikai. Tačiau mes turime atiduoti savo sūnus ir dukteris vergais, ir kai kurių mūsų dukterys jau yra vergės. Mes negalime jų išpirkti, nes mūsų laukai ir vynuogynai priklauso kitiems”. 
\par 6 Išgirdęs tą šauksmą ir tuos žodžius, labai supykau. 
\par 7 Apsvarsčiau reikalą ir sudraudžiau kilminguosius ir viršininkus, sakydamas: “Jūs kiekvienas lupate palūkanas iš savo brolio”. Sušaukęs visuotinį susirinkimą, 
\par 8 kalbėjau: “Kiek leido mūsų išgalės, mes išpirkome savo brolius žydus, kurie buvo parduoti pagonims, o jūs verčiate savo brolius parsiduoti jums!” Jie tylėjo ir nieko neatsakė. 
\par 9 Aš tęsiau: “Negerai darote! Argi neturėtumėte bijoti Dievo ir neduoti progos pagonims mūsų gėdinti? 
\par 10 Aš, mano broliai ir mano tarnai taip pat skolinome pinigų ir grūdų. Dovanokime jiems šitą skolą! 
\par 11 Šiandien pat grąžinkite jiems dirvas, vynuogynus, alyvų sodus bei namus ir dalį skolų: pinigus, javus, vyną ir aliejų, ką ėmėte palūkanų”. 
\par 12 Tuomet jie atsakė: “Viską grąžinsime ir skolų iš jų nereikalausime, darysime, kaip sakei”. Pasišaukęs kunigus, prisaikdinau juos žmonėms girdint, kad vykdytų savo pažadą. 
\par 13 Aš iškračiau savo antį ir tariau: “Tegul Dievas taip pat iškrato kiekvieną, kuris neištesės šito pažado, iš jo namų ir iš įsigytos nuosavybės ir tegul jis lieka tuščias”. Visi susirinkusieji tarė: “Amen”. Ir šlovino Viešpatį. Žmonės vykdė, ką buvo pasižadėję. 
\par 14 Nuo tos dienos, kai buvau paskirtas valdytoju Judo krašte, nuo dvidešimtųjų iki trisdešimt antrųjų karaliaus Artakserkso metų, dvylika metų aš su savo broliais nevalgiau valdytojo duonos. 
\par 15 Valdytojai, buvę iki manęs, skriaudė žmones, imdami iš jų duonos ir vyno ir keturiasdešimt šekelių sidabro; taip pat ir jų tarnai išnaudojo tautą, bet aš taip nedariau, bijodamas Dievo. 
\par 16 Aš taip pat tęsiau darbą prie sienos ir neįsigijau jokios nuosavybės; visi mano tarnai irgi dirbo prie statybos. 
\par 17 Be to, šimtas penkiasdešimt žydų ir viršininkų valgė prie mano stalo, neskaičiuojant svetimų tautų žmonių, kurie apsilankydavo pas mus. 
\par 18 Kasdien man paruošdavo vieną jautį, šešias rinktines avis ir paukščių; kas dešimtą dieną gausiai pristatydavo visokio vyno; aš nereikalaudavau valdytojo duonos, nes skurdas sunkiai slėgė tautą. 
\par 19 Mano Dieve, priskaityk mano naudai visa, ką padariau šiai tautai!



\chapter{6}

\par 1 Kai Sanbalatui, Tobijai, arabui Gešemui ir visiems kitiems mūsų priešams buvo pranešta, kad siena atstatyta, spragos užtaisytos, tik dar nebuvau įstatęs durų vartuose, 
\par 2 Sanbalatas ir Gešemas kvietė mane susitikti su jais viename kaime Onojo lygumoje. Tačiau jie buvo sugalvoję prieš mane pikta. 
\par 3 Aš jiems pranešiau: “Dirbu didelį darbą ir negaliu ateiti. Darbas susitrukdytų, jei eičiau pas jus, palikdamas jį”. 
\par 4 Jie darė taip keturis kartus; aš jiems atsakiau tą patį. 
\par 5 Sanbalatas penktą kartą siuntė pas mane savo tarną su atviru laišku rankoje. 
\par 6 Jame buvo parašyta: “Tautose sklinda gandas ir Gešemas tai tvirtina, kad tu ir žydai ruošiatės maištui; todėl tu pastatei sieną, kad galėtum tapti jų karaliumi. 
\par 7 Taip pat jau esi paskyręs ir pranašų, kad jie skelbtų Jeruzalėje tave esant Judo karaliumi. Šitos kalbos bus praneštos karaliui. Taigi dabar ateik pasitarti”. 
\par 8 Aš pasiunčiau jam atsakymą: “Nieko panašaus nėra, tai tik tavo prasimanymai”. 
\par 9 Jie norėjo mus įbauginti, manydami: “Jie nuleis rankas, ir darbas nebus užbaigtas”. Tačiau tas mane dar labiau sustiprino. 
\par 10 Kartą aš atėjau į Mehetabelio sūnaus Delajos sūnaus Šemajos namus. Jis užsirakino ir kalbėjo man: “Eikime į Dievo namus, į šventyklą, ir užrakinkime duris, nes nakčia jie ateis tavęs nužudyti”. 
\par 11 Aš jam atsakiau: “Ar aš turėčiau bėgti? Kas aš esu, kad eičiau į šventyklą gelbėtis? Neisiu”. 
\par 12 Supratau, kad Dievas jo nesiuntė; jis man kalbėjo taip, nes buvo Tobijos ir Sanbalato papirktas, 
\par 13 kad mane įbaugintų ir aš taip padarydamas nusidėčiau. Jie tai būtų panaudoję man sugėdinti. 
\par 14 Mano Dieve, prisimink Tobiją ir Sanbalatą pagal jų darbus, taip pat pranašę Noadiją ir kitus pranašus, norėjusius įbauginti mane. 
\par 15 Siena buvo baigta Elulo mėnesio dvidešimt penktą dieną, per penkiasdešimt dvi dienas. 
\par 16 Mūsų priešai ir apylinkėje gyveną pagonys, pamatę mūsų darbą, pasijuto labai pažeminti, nes suprato, kad šitas darbas buvo padarytas mūsų Dievo. 
\par 17 Tomis dienomis Judo kilmingieji dažnai susirašinėjo su Tobija. 
\par 18 Daugelis Jude buvo prisiekę jam, nes jis buvo Aracho sūnaus Šechanijos žentas; be to, Tobijo sūnus Johananas buvo vedęs Berechijos sūnaus Mešulamo dukterį. 
\par 19 Jie kalbėjo apie jo gerus darbus man girdint, o mano žodžius pranešdavo jam. Tobijas siuntė laiškus, norėdamas mane įbauginti.



\chapter{7}

\par 1 Kai buvo baigta siena, įstatytos durys ir paskirti vartininkai, giedotojai ir levitai, 
\par 2 daviau savo broliui Hananiui ir rūmų viršininkui Hananijai paliepimą dėl Jeruzalės, nes jie buvo ištikimi ir dievobaimingi vyrai. 
\par 3 Įsakiau jiems neatidaryti Jeruzalės vartų iki saulės kaitros, o uždaryti bei užsklęsti juos prieš sutemstant. Sargybas statyti iš Jeruzalės gyventojų, kiekvieną arti jo namų. 
\par 4 Miestas buvo platus ir didelis, bet žmonių ir namų jame buvo mažai. 
\par 5 Dievas įdėjo į mano širdį sukviesti kilminguosius, viršininkus ir tautą ir surašyti giminėmis. Suradau sąrašus pirmųjų, grįžusių iš nelaisvės. 
\par 6 Tie yra krašto žmonės, kurie grįžo iš nelaisvės, iš tų, kuriuos Babilono karalius Nabuchodonosaras buvo ištrėmęs į Babiloną. Jie sugrįžo į Jeruzalę bei Judą, kiekvienas į savo miestą. 
\par 7 Jiems vadovavo Zorobabelis, Jozuė, Nehemija, Azarija, Raamija, Nahamanis, Mordechajas, Bilšanas, Misperetas, Bigvajas, Nehumas ir Baana. Izraelio tautos vyrų skaičius: 
\par 8 Parošo palikuonių buvo du tūkstančiai šimtas septyniasdešimt du; 
\par 9 Šefatijos­trys šimtai septyniasdešimt du; 
\par 10 Aracho­šeši šimtai penkiasdešimt du; 
\par 11 Pahat Moabo palikuonių iš Ješūvos ir Joabo giminės­du tūkstančiai aštuoni šimtai aštuoniolika; 
\par 12 Elamo­tūkstantis du šimtai penkiasdešimt keturi; 
\par 13 Zatuvo­aštuoni šimtai keturiasdešimt penki; 
\par 14 Zakajo­septyni šimtai šešiasdešimt; 
\par 15 Binujo­šeši šimtai keturiasdešimt aštuoni; 
\par 16 Bebajo­šeši šimtai dvidešimt aštuoni; 
\par 17 Azgado­du tūkstančiai trys šimtai dvidešimt du; 
\par 18 Adonikamo­šeši šimtai šešiasdešimt septyni; 
\par 19 Bigvajo­du tūkstančiai šešiasdešimt septyni; 
\par 20 Adino­šeši šimtai penkiasdešimt penki; 
\par 21 Atero palikuonių iš Ezekijo­ devyniasdešimt aštuoni; 
\par 22 Hašumo­trys šimtai dvidešimt aštuoni; 
\par 23 Becajo­trys šimtai dvidešimt keturi; 
\par 24 Harifo­šimtas dvylika; 
\par 25 Gibeono­devyniasdešimt penki; 
\par 26 Betliejaus ir Netofos vyrų­šimtas aštuoniasdešimt aštuoni; 
\par 27 Anatoto vyrų­šimtas dvidešimt aštuoni; 
\par 28 Bet Azmaveto vyrų­keturiasdešimt du; 
\par 29 Kirjat Jearimo, Kefyros ir Beeroto vyrų­septyniasdešimt trys; 
\par 30 Ramos ir Gebos vyrų­šeši šimtai dvidešimt vienas; 
\par 31 Michmašo vyrų­šimtas dvidešimt du; 
\par 32 Betelio ir Ajo vyrų­šimtas dvidešimt trys; 
\par 33 Kito Nebojo vyrų­penkiasdešimt du; 
\par 34 Kito Elamo palikuonių­tūkstantis du šimtai penkiasdešimt keturi; 
\par 35 Harimo­trys šimtai dvidešimt; 
\par 36 Jericho­trys šimtai keturiasdešimt penki; 
\par 37 Lodo, Hadido ir Onojo­septyni šimtai dvidešimt vienas; 
\par 38 Senavos­trys tūkstančiai devyni šimtai trisdešimt. 
\par 39 Kunigų: Jedajos palikuonių iš Ješūvos namų­devyni šimtai septyniasdešimt trys; 
\par 40 Imero­tūkstantis penkiasdešimt du; 
\par 41 Pašhūro­tūkstantis du šimtai keturiasdešimt septyni; 
\par 42 Harimo­tūkstantis septyniolika. 
\par 43 Levitų: Jozuės ir Kadmielio palikuonių iš Hodvos sūnų­septyniasdešimt keturi. 
\par 44 Giedotojų: Asafo palikuonių­ šimtas keturiasdešimt aštuoni. 
\par 45 Vartininkų: Šalumo, Atero, Talmono, Akubo, Hatitos ir Šobajo palikuonių­šimtas trisdešimt aštuoni. 
\par 46 Šventyklos tarnai: Cihos, Hasufos, Tabaoto, 
\par 47 Keroso, Sijos, Padono, 
\par 48 Lebanos, Hagabos, Šalmajo, 
\par 49 Hanano, Gidelio, Gaharo, 
\par 50 Reajos, Recino, Nekodos, 
\par 51 Gazamo, Uzos, Paseacho, 
\par 52 Besajo, Meunimo, Nefišsos, 
\par 53 Bakbuko, Hakufos, Harhūro, 
\par 54 Baclito, Mehidos, Haršos, 
\par 55 Barkoso, Siseros, Temacho, 
\par 56 Neciacho ir Hatifos palikuonys. 
\par 57 Saliamono tarnų palikuonys: Sotajo, Sofereto, Peridos, 
\par 58 Jaalos, Darkono, Gidelio, 
\par 59 Šefatijos, Hatilo, Pocheret Cebaimo ir Amono palikuonys. 
\par 60 Šventyklos ir Saliamono tarnų palikuonių buvo trys šimtai devyniasdešimt du. 
\par 61 Šitie atvyko iš Tel Melacho, Tel Haršos, Kerub Adono ir Imero, bet negalėjo įrodyti savo tėvų ir savo kilmės, ar jie kilę iš Izraelio: 
\par 62 Delajos, Tobijos ir Nekodos palikuonių­šeši šimtai keturiasdešimt du. 
\par 63 Iš kunigų: Hobajos, Hakoco, Barzilajaus (kuris buvo vedęs gileadito Barzilajaus dukterį ir buvo vadinamas jų vardu) palikuonys. 
\par 64 Jie ieškojo savo vardų giminių sąrašuose, tačiau nerado; todėl jie buvo atskirti nuo kunigystės kaip susitepę. 
\par 65 Tiršata jiems uždraudė valgyti labai šventą maistą, kol atsiras kunigas su Urimu ir Tumimu. 
\par 66 Iš viso žmonių buvo keturiasdešimt du tūkstančiai trys šimtai šešiasdešimt, 
\par 67 neskaičiuojant jų tarnų ir tarnaičių, kurių buvo septyni tūkstančiai trys šimtai trisdešimt septyni. Be to, jie turėjo giedotojų vyrų ir moterų­du šimtus keturiasdešimt penkis. 
\par 68 Arklių buvo septyni šimtai trisdešimt šeši, mulų­du šimtai keturiasdešimt penki, 
\par 69 kupranugarių­keturi šimtai trisdešimt penki, asilų­šeši tūkstančiai septyni šimtai dvidešimt. 
\par 70 Kai kurie šeimų vadai aukojo darbui. Tiršata davė tūkstantį drachmų aukso, penkiasdešimt šlakstytuvų, penkis šimtus trisdešimt kunigų apdarų. 
\par 71 Kai kurie šeimų vadai­dvidešimt tūkstančių drachmų aukso ir du tūkstančius du šimtus minų sidabro. 
\par 72 Visų kitų dovanos buvo dvidešimt tūkstančių drachmų aukso, du tūkstančiai minų sidabro ir šešiasdešimt septyni kunigų apdarai. 
\par 73 Kunigai, levitai, giedotojai, vartininkai, dalis tautos, šventyklos tarnai ir visas Izraelis apsigyveno savo miestuose. Septintą mėnesį izraelitai buvo savo miestuose.



\chapter{8}


\par 1 Septintą mėnesį visi žmonės iki vieno susirinko Vandens vartų aikštėje ir sakė Rašto žinovui Ezrai atnešti knygą Mozės įstatymo, kurį Viešpats buvo davęs Izraeliui. 
\par 2 Septinto mėnesio pirmą dieną kunigas Ezra atnešė įstatymo knygą į susirinkimą, kur buvo vyrai, moterys ir kiti, galintys suprasti tai, ką girdi. 
\par 3 Jis skaitė iš knygos aikštėje prie Vandens vartų nuo ankstyvo ryto iki vidudienio vyrams, moterims ir visiems suprantantiems; visų dėmesys buvo nukreiptas į įstatymą. 
\par 4 Rašto žinovas Ezra stovėjo ant medinio paaukštinimo, kuris buvo padarytas tam reikalui. Šalia jo, dešinėje, stovėjo Matitija, Šema, Anaja, Ūrija, Hilkija ir Maasėja, o jo kairėje­Pedaja, Mišaelis, Malkija, Hašumas, Hašbadana, Zacharija ir Mešulamas. 
\par 5 Ezra atvyniojo knygą visiems matant, nes jis stovėjo aukščiau. Jam atvyniojus knygą, visi žmonės atsistojo. 
\par 6 Ezra palaimino Viešpatį, didį Dievą, o visi žmonės pakėlė rankas ir atsakė: “Amen, amen” ir, parpuolę žemėn, garbino Viešpatį. 
\par 7 Jozuė, Banis, Šerebija, Jaminas, Akubas, Šabetajas, Hodija, Maasėja, Kelita, Azarija, Jehozabadas, Hananas, Pelaja ir levitai aiškino žmonėms įstatymą, o žmonės stovėjo savo vietose. 
\par 8 Jie skaitė iš Dievo įstatymų knygos, aiškindami juos, ir žmonės suprato, kas buvo skaitoma. 
\par 9 Nehemijas, tai yra Tiršata, Rašto žinovas kunigas Ezra ir levitai, kurie mokė žmones, kalbėjo žmonėms: “Šita diena yra šventa Viešpačiui, jūsų Dievui. Neliūdėkite ir neverkite!” Nes visi žmonės verkė, klausydamiesi įstatymo žodžių. 
\par 10 Nehemijas tarė: “Eikite, valgykite riebius valgius, gerkite saldų vyną ir pasiųskite dalį tiems, kurie nieko neturi paruošto, nes ši diena yra šventa mūsų Viešpačiui; nesisielokite, nes Viešpaties džiaugsmas­jūsų stiprybė”. 
\par 11 Levitai ramino žmones: “Nusiraminkite, nes ši diena yra šventa, ir nesisielokite”. 
\par 12 Visi žmonės nuėję valgė, gėrė ir dalinosi su vargšais, nes jie suprato žodžius, kurie jiems buvo paskelbti. 
\par 13 Kitą dieną susirinko pas Rašto žinovą Ezrą visos tautos šeimų vadai, kunigai ir levitai, norėdami suprasti įstatymo žodžius. 
\par 14 Įstatyme, kurį Viešpats davė per Mozę, jie rado parašyta, kad izraelitai septinto mėnesio šventės metu turi gyventi palapinėse. 
\par 15 Jie paskelbė Jeruzalėje ir visuose miestuose: “Išeikite į kalnus ir atsineškite alyvmedžių, laukinių alyvmedžių, mirtų, palmių ir kitokių šakų palapinėms pasidaryti, kaip parašyta”. 
\par 16 Žmonės išėję atsinešė šakų ir pasistatė kiekvienas sau palapines: vieni ant savo namų stogo, kiti kieme, dar kiti Dievo namų kieme arba Vandens vartų ir Efraimo vartų aikštėse. 
\par 17 Grįžę iš nelaisvės žydai pasistatė kiekvienas sau palapines ir gyveno jose. Nuo Nūno sūnaus Jozuės laikų iki tos dienos izraelitai nebuvo to darę. Visi buvo labai patenkinti. 
\par 18 Kasdien buvo skaitoma iš Dievo įstatymo knygos. Taip šventė septynias dienas, o aštuntą dieną įvyko iškilmingas susirinkimas, kaip buvo nustatyta.
Online Parallel Study Bible



\chapter{9}

\par 1 Šio mėnesio dvidešimt ketvirtą dieną izraelitai susirinko pasninkaudami, apsivilkę ašutinėmis ir apsibarstę galvas žemėmis. 
\par 2 Izraelio palikuonys atsiskyrė nuo visų svetimtaučių ir sustoję išpažino savo nuodėmes bei tėvų nusikaltimus. 
\par 3 Stovėdami savo vietose, jie ketvirtį dienos skaitė iš Viešpaties, savo Dievo, įstatymo knygos, o antrą dienos ketvirtį išpažino savo nuodėmes ir, parpuolę ant žemės, garbino Viešpatį, savo Dievą. 
\par 4 Ant paaukštinimo stovėjo levitai: Ješūva, Banis, Kadmielis, Šebanija, Būnis, Šerebija, Banis ir Kenanis­ir garsiai šaukėsi Viešpaties, savo Dievo. 
\par 5 Levitai Ješūva, Kadmielis, Banis, Hašabnėja, Šerebija, Hodija, Šebanija ir Petachija sakė: “Atsistokite ir šlovinkite Viešpatį, savo Dievą, per amžių amžius. Tebūna palaimintas Tavo šlovingas vardas, kuris išaukštintas virš visokio palaiminimo ir gyriaus”. 
\par 6 Ezra meldėsi: “Tu, Viešpatie, esi vienintelis. Tu sutvėrei dangų, dangaus dangų ir visą jų kareiviją, žemę, jūras ir visa, kas jose. Tu visa tai palaikai, ir dangaus kareivijos garbina Tave. 
\par 7 Tu, Viešpatie, esi Dievas, kuris išsirinkai Abramą, jį išvedei iš chaldėjų Ūro ir davei jam Abraomo vardą. 
\par 8 Patyręs, kad jis Tau ištikimas, padarei su juo sandorą, kad jo palikuonims duosi kraštą kanaaniečių, hetitų, amoritų, perizų, jebusiečių ir girgašų. Tu ištesėjai savo žodį, nes Tu esi teisus. 
\par 9 Tu matei mūsų tėvų priespaudą Egipte ir išgirdai jų šauksmą prie Raudonosios jūros. 
\par 10 Tu darei ženklų ir stebuklų faraonui, jo tarnams ir visiems jo krašto žmonėms, nes žinojai, kaip jie didžiavosi prieš juos. Taip Tu įsigijai vardą, kaip yra ir iki šios dienos. 
\par 11 Tu perskyrei jūrą prieš juos, ir izraelitai perėjo sausuma per jūrą, o jų persekiotojus atidavei gelmėms kaip akmenį šėlstančioms bangoms. 
\par 12 Debesies stulpu vedei juos dieną ir ugnies stulpu nušvietei naktį jiems kelią, kuriuo jie turėjo eiti. 
\par 13 Tu nužengei ant Sinajaus kalno ir kalbėjaisi su jais iš dangaus; davei jiems teisingus potvarkius, tikrus įstatymus, gerus nuostatus ir įsakymus. 
\par 14 Tu paskelbei jiems savo šventą sabatą; įsakymus, nuostatus bei įstatymus jiems davei per savo tarną Mozę. 
\par 15 Kai jie badavo, davei jiems duonos iš dangaus, kai troško, leidai vandeniui tekėti iš uolos. Tu pažadėjai jiems, kad jie įeis į žemę, kurią Tu prisiekei jiems atiduoti. 
\par 16 Bet jie ir mūsų tėvai elgėsi išdidžiai, sukietino savo sprandus ir neklausė Tavo įsakymų. 
\par 17 Jie atsisakė paklusti ir neprisiminė Tavo stebuklų, kuriuos tarp jų padarei. Jie sukietino savo sprandus ir maištaudami išsirinko vadą, kuris juos parvestų vergijon. Bet Tu esi Dievas, pasiruošęs atleisti, maloningas ir gailestingas, lėtas pykti ir didžiai geras, todėl neapleidai jų. 
\par 18 Kai jie nusiliejo veršį ir sakė: ‘Tai yra tavo dievas, kuris tave išvedė iš Egipto’, jie Tave labai supykdė. 
\par 19 Tačiau Tu, Dieve, būdamas didžiai gailestingas, jų nepalikai dykumoje; debesies stulpas neatsitraukė nuo jų dieną nė ugnies stulpas naktį, rodydamas jiems kelią, kuriuo jie turėjo eiti. 
\par 20 Tu davei savo gerąją dvasią ir mokei juos, maitinai mana ir girdei vandeniu. 
\par 21 Keturiasdešimt metų aprūpinai juos dykumoje. Jiems nieko netrūko. Jų drabužiai nenusidėvėjo ir jų kojos neištino. 
\par 22 Karalystes ir tautas Tu atidavei jiems ir paskirstei jas dalimis. Jie užėmė Hešbono karaliaus Sihono kraštą ir Bašano karaliaus Ogo žemę. 
\par 23 Jų vaikus padauginai kaip dangaus žvaigždes ir jiems davei kraštą, kurį pažadėjai jų tėvams. 
\par 24 Jų vaikai įėjo ir užėmė tą žemę. Tu pajungei jiems krašto gyventojus, kanaaniečius, ir atidavei juos į jų rankas su jų karaliais ir krašto žmonėmis, kad jie pasielgtų su jais, kaip jiems patinka. 
\par 25 Jie paėmė sutvirtintus miestus, derlingą žemę, pasisavino jų namus, pilnus gėrybių, ir iškastus šulinius, vynuogynus, alyvmedžių sodus ir vaismedžių daugybę. Jie valgė, pasisotino ir pralobo. Jie džiaugėsi Tavo didžiu gerumu. 
\par 26 Bet jie tapo neklusnūs, maištavo prieš Tave, atmetė Tavo įstatymus ir žudė Tavo pranašus, kurie įspėjo juos sugrįžti prie Tavęs; jie Tau labai nusikalto. 
\par 27 Todėl Tu atidavei juos jų priešams, kurie juos užėmė ir pavergė. Priespaudos metu jie šaukėsi Tavęs. Tu juos išklausei ir, didžiai gailėdamasis jų, siuntei jiems gelbėtojų, kurie juos išgelbėjo iš jų priešų rankų. 
\par 28 Sulaukę ramybės, jie vėl darydavo pikta prieš Tave. Tuomet Tu vėl juos atiduodavai į priešų rankas. Jiems atgailaujant, Tu daug kartų išklausei juos ir išvadavai iš priešų, gailėdamasis jų. 
\par 29 Tu perspėdavai juos grįžti prie Tavo įstatymų, tačiau jie elgdavosi išdidžiai ir nepaklusdavo Tavo įsakymams, bet laužė Tavo nuostatus, kuriuos vykdydamas žmogus yra gyvas. Jie atsuko Tau nugarą ir pasiliko užkietėję. 
\par 30 Daugelį metų Tu pakentei juos, liudydamas prieš juos per savo dvasią savo pranašuose. Jie neklausė Tavęs, todėl Tu atidavei juos į žemės tautų rankas. 
\par 31 Tačiau dėl savo didžio gailestingumo Tu jų nesunaikinai visiškai ir neatstūmei, nes Tu esi maloningas ir gailestingas Dievas. 
\par 32 O dabar, didis, galingas ir baisus Dieve, kuris laikaisi sandoros ir gailestingumo, pažvelk į mūsų vargus, kurie ištiko mūsų karalius, kunigaikščius, kunigus, pranašus, tėvus ir visą mūsų tautą nuo Asirijos karalių laikų iki šios dienos. 
\par 33 Tu teisingai baudei mus. Tu buvai mums ištikimas, o mes nusikaltome Tau. 
\par 34 Mūsų karaliai, kunigaikščiai, kunigai ir mūsų tėvai nevykdė įstatymų ir nepaisė Tavo įsakymų ir įspėjimų. 
\par 35 Gyvendami savo karalystėje, jie naudojosi gėrybėmis, kurias jiems davei, Tavo jiems duotame derlingame ir plačiame krašte. Tačiau jie netarnavo Tau ir neatsisakė savo piktų darbų. 
\par 36 Šiandien esame vergai šalyje, kurią davei mūsų tėvams ir leidai naudotis jos vaisiais ir gėrybėmis. 
\par 37 Dabar jos derlius priklauso karaliams, kuriuos paskyrei mums už mūsų nuodėmes. Mūsų kūnai ir mūsų gyvuliai yra jų nuosavybė, su kuria jie elgiasi, kaip jiems patinka; mes papuolę į vargą. 
\par 38 Todėl mes darome tvirtą sandorą ir ją užrašome, o mūsų kunigaikščiai, levitai ir kunigai ją užantspauduoja”.



\chapter{10}

\par 1 Užantspaudavo šie: Nehemijas, Hachalijos sūnus, tai yra Tiršata, Zedekija, 
\par 2 Seraja, Azarija, Jeremija, 
\par 3 Pašhūras, Amarija, Malkija, 
\par 4 Hatušas, Šebanija, Maluchas, 
\par 5 Harimas, Meremotas, Abdija, 
\par 6 Danielius, Ginetonas, Baruchas, 
\par 7 Mešulamas, Abija, Mijaminas, 
\par 8 Maazija, Bilgajas ir Šemaja; jie visi yra kunigai. 
\par 9 Levitai: Azanijos sūnus Ješūva, Binujas iš Henadado palikuonių, Kadmielis, 
\par 10 Šebanija, Hodija, Kelita, Pelaja, Hananas, 
\par 11 Michėjas, Rehobas, Hašabija, 
\par 12 Zakūras, Šerebija, Šebanija, 
\par 13 Hodija, Banis ir Beninuvas. 
\par 14 Tautos kunigaikščiai: Parošas, Pahat Moabas, Elamas, Zatuvas, Banis, 
\par 15 Būnis, Azgadas, Bebajas, 
\par 16 Adonija, Bigvajas, Adinas, 
\par 17 Ateras, Ezekijas, Azūras, 
\par 18 Hodija, Hašumas, Becajas, 
\par 19 Harifas, Anatotas, Nebajas, 
\par 20 Magpiašas, Mešulamas, Hezyras, 
\par 21 Mešezabelis, Cadokas, Jadūva, 
\par 22 Pelatija, Hananas, Anaja, 
\par 23 Ozėjas, Hananija, Hašubas, 
\par 24 Ha Lohešas, Pilha, Šobekas, 
\par 25 Rehumas, Hašabna, Maasėja, 
\par 26 Ahija, Hananas, Ananas, 
\par 27 Maluchas, Harimas, Baana. 
\par 28 Visi kiti kunigai, levitai, vartininkai, giedotojai, šventyklos tarnai ir tie, kurie atsiskyrė nuo krašto tautų dėl Dievo įstatymo, jų žmonos, sūnūs ir dukterys, kurie suprato, 
\par 29 prisijungė prie savo brolių, savo kilmingųjų, ir įsipareigojo vykdyti Dievo įstatymą, kuris buvo duotas per Dievo tarną Mozę, ir laikytis visų Viešpaties potvarkių bei nuostatų. 
\par 30 Pasižadėjome neduoti savo dukterų krašto žmonėms ir neimti jų dukterų. 
\par 31 Jei krašto žmonės sabato dieną atgabentų prekių ar grūdų parduoti, nieko iš jų nepirksime nei sabatą, nei šventadieniais. Taip pat pasižadėjome septintais metais nesėti ir atleisti visas skolas. 
\par 32 Mes pasižadėjome kas metai duoti trečdalį šekelio Dievo namų reikalams: 
\par 33 padėtinei duonai, nuolatinei duonos aukai, nuolatinei deginamajai aukai, sabatų aukoms, jauno mėnulio ir iškilmingų švenčių aukoms, šventiems daiktams, aukoms už nuodėmę Izraeliui sutaikinti ir kitiems Dievo namų reikalams. 
\par 34 Kunigai, levitai ir žmonės burtų keliu nustatė malkų pristatymo Dievo namams eilę, kad būtų atnešama šeimomis, kas metai nustatytu laiku Viešpaties, mūsų Dievo, aukurui, kaip parašyta įstatyme. 
\par 35 Taip pat įsipareigojome kas metai pristatyti Viešpaties namams mūsų laukų derliaus ir vaisių pirmavaisius; 
\par 36 be to, mūsų sūnų pirmagimius ir galvijų bei avių pirmgimius pristatyti į Dievo namus kunigams, tarnaujantiems Dievo namuose, kaip parašyta įstatyme. 
\par 37 Kas geriausia iš mūsų valgių, įvairių medžių pirmųjų vaisių, vyno ir aliejaus pasižadėjome pristatyti kunigams į Dievo namų sandėlius, o mūsų laukų dešimtinę­levitams; levitai patys ims dešimtines visuose mūsų miestuose. 
\par 38 Aarono sūnus kunigas lydės levitus, jiems renkant dešimtinę, o levitai atiduos dešimtą dalį dešimtinės į Dievo namų sandėlius. 
\par 39 Izraelitai ir levitai sugabens javus, vyną ir aliejų į sandėlius, kur yra šventyklos indai, tarnaujantys kunigai, vartininkai ir giedotojai, kad neapleistume Dievo namų.



\chapter{11}


\par 1 Tautos kunigaikščiai gyveno Jeruzalėje, o kiti metė burtą, kad vienas dešimtadalis gyventų Jeruzalėje, šventajame mieste, o devyni dešimtadaliai liktų gyventi kituose miestuose. 
\par 2 Tauta palaimino kiekvieną, kuris savanoriškai sutiko apsigyventi Jeruzalėje. 
\par 3 Šie krašto valdytojai gyveno Jeruzalėje. Izraelitai, kunigai, levitai, šventyklos tarnai ir Saliamono tarnų palikuonys gyveno Judo miestuose, kiekvienas savo mieste prie savo nuosavybės. 
\par 4 Jeruzalėje gyveno Judo ir Benjamino palikuonys. Judo sūnaus Faro palikuonys: Ataja, sūnus Uzijo, sūnaus Zacharijos, sūnaus Amarijo, sūnaus Šefatijos, sūnaus Mahalalelio, 
\par 5 ir Maasėja, sūnus Barucho, sūnaus Kol Hozės, sūnaus Hazajos, sūnaus Adajos, sūnaus Jehojaribo, sūnaus Zacharijos, sūnaus Šilono. 
\par 6 Iš viso Faro palikuonių, gyvenusių Jeruzalėje, buvo keturi šimtai šešiasdešimt aštuoni vyrai. 
\par 7 Benjamino palikuonys: Saluvas, sūnus Mešulamo, sūnaus Joedo, sūnaus Pedajos, sūnaus Kolajos, sūnaus Maasėjos, sūnaus Itielio, sūnaus Jesajo. 
\par 8 Po jo Gabajas ir Salajas­devyni šimtai dvidešimt aštuoni. 
\par 9 Zichrio sūnus Joelis buvo jų viršininkas, o Ha Senūvos sūnus Judas buvo miesto viršininko padėjėjas. 
\par 10 Kunigai: Jehojaribo sūnus Jedaja, Jachinas. 
\par 11 Serajas, sūnus Hilkijos, sūnaus Mešulamo, sūnaus Cadoko, sūnaus Merajoto, sūnaus Ahitubo, buvo vyriausiasis Dievo namuose. 
\par 12 Jų brolių, kurie dirbo šventykloje, buvo aštuoni šimtai dvidešimt du. Adaja, sūnus Jerohamo, sūnaus Pelalijo, sūnaus Amcio, sūnaus Zacharijos, sūnaus Pašhūro, sūnaus Malkijos. 
\par 13 Jo brolių, šeimos vyresniųjų, buvo du šimtai keturiasdešimt du. Amašsajas­sūnus Azarelio, sūnaus Achzajo, sūnaus Mešilemoto, sūnaus Imero. 
\par 14 Jų brolių, drąsių vyrų, buvo šimtas dvidešimt aštuoni. Jų viršininkas buvo Zabdielis, Ha Gedolimo sūnus. 
\par 15 Levitai: Šemajas, sūnus Hašubo, sūnaus Azrikamo, sūnaus Hašabijos, sūnaus Būnio; 
\par 16 Šabetajas ir Jehozabadas­levitų viršininkai, kurie prižiūrėjo Dievo namų išorinius darbus; 
\par 17 Matanija, sūnus Michėjo, sūnaus Zabdžio, sūnaus Asafo, vadovavo padėkos maldoms, Bakbukija, antras tarp savo brolių, ir Abda, sūnus Šamūvos, sūnaus Galalo, sūnaus Jedutūno. 
\par 18 Iš viso šventajame mieste levitų buvo du šimtai aštuoniasdešimt keturi. 
\par 19 Vartininkai: Akubas, Talmonas ir jų broliai, kurie ėjo sargybą prie vartų, iš viso šimtas septyniasdešimt du. 
\par 20 Likusieji izraelitai, kunigai ir levitai gyveno visuose Judo miestuose, kiekvienas savo pavelde. 
\par 21 Šventyklos tarnai gyveno Ofelyje; Ciha ir Gišpa buvo šventyklos tarnų viršininkai. 
\par 22 Levitų viršininkas Jeruzalėje buvo Uzis, sūnus Banio, sūnaus Hašabijos, sūnaus Matanijos, sūnaus Michėjo; Dievo namų giedotojai buvo Asafo palikuonys. 
\par 23 Karaliaus įsakymu giedotojai gavo išlaikymą kiekvieną dieną. 
\par 24 Mešezabelio sūnus Petachija iš Judo sūnaus Zaros palikuonių buvo karaliaus paskirtas visiems tautos reikalams. 
\par 25 Kai kurie Judo palikuonys gyveno Kirjat Arboje, Dibone, Jekabceelyje, 
\par 26 Ješūve, Moladoje, Bet Pelete, 
\par 27 Hacar Šuale, Beer Šeboje, 
\par 28 Ciklage, Mechonoje, 
\par 29 En Rimone, Coroje, Jarmute, 
\par 30 Zanoache, Adulame, Lachiše, Azekoje ir tų miestų kaimuose. Jie gyveno nuo Beer Šebos iki Hinomo slėnio. 
\par 31 O benjaminitai gyveno Geboje, Michmaše, Ajoje, Betelyje, 
\par 32 Anatote, Nobe, Ananijoje, 
\par 33 Hacore, Ramoje, Gitaimuose, 
\par 34 Hadide, Ceboime, Nebalate, 
\par 35 Lode, Onojuje, jų kaimuose ir Amatininkų slėnyje. 
\par 36 Kai kurie levitai buvo pasiskirstę Jude ir Benjamine.
Online Parallel Study Bible



\chapter{12}


\par 1 Kunigai ir levitai, atvykę su Salatielio sūnumi Zorobabeliu ir Jozue, buvo šie: Seraja, Jeremija, Ezra, 
\par 2 Amarija, Maluchas, Hatušas, 
\par 3 Šechanija, Rehumas, Meremotas, 
\par 4 Idojas, Ginetonas, Abija, 
\par 5 Mijaminas, Maadija, Bilga, 
\par 6 Šemaja, Jehojaribas, Jedaja, 
\par 7 Saluvas, Amokas, Hilkija ir Jedajas. Šitie buvo kunigų ir jų brolių viršininkai Ješūvos laikais. 
\par 8 Levitai: Ješūva, Binujas, Kadmielis, Šerebija, Judas, Matanija, kuris vadovavo dėkojimui, ir jo broliai. 
\par 9 Bakbukija, Unis ir jų broliai budėjo priešais juos. 
\par 10 Jozuė buvo Jehojakimo tėvas, Jehojakimas­Eljašibo, Eljašibas­ Jehojados, 
\par 11 Jehojada­Jehonatano, Jehonatanas­Jadūvos. 
\par 12 Jehojakimo laikais šie kunigai buvo šeimų vyresnieji: Meraja­ Serajo, Hananija­Jeremijo, 
\par 13 Mešulamas­Ezro, Johananas­ Amarijo, 
\par 14 Jehonatanas­Melichuvo, Juozapas­Šebanijo, 
\par 15 Adna­Harimo, Helkajas­Merajoto, 
\par 16 Zacharija­Idojo, Mešulamas­ Ginetono, 
\par 17 Zichris­Abijo, Piltajas­Minjamino ir Moadijos, 
\par 18 Šamūvos­Bilgo, Jehonatanas­ Šemajo, 
\par 19 Matenajas­Jehojaribo, Uzis­ Jedajos, 
\par 20 Kalajas­Salajo, Eberas­Amoko, 
\par 21 Hašabija­Hilkijos, Netanelis­ Jedajos. 
\par 22 Eljašibo, Jehojados, Joanano ir Jaduvo laikais buvo surašyti levitai, taip pat ir kunigų šeimų vyresnieji iki persų karaliaus Darijaus laikų. 
\par 23 Levitų šeimų vyresnieji buvo surašyti metraščių knygoje iki Eljašibo sūnaus Joanano laikų. 
\par 24 Vadovaujant levitams Hašabijai, Serebijui ir Jozuei, Kadmielio sūnui, giedotojai stovėjo prieš juos ir giedojo padėkos giesmes, kaip karalius, Dievo vyras Dovydas, buvo nustatęs. 
\par 25 Matanijas, Bakbukijas, Abdija, Mešulamas, Talmonas ir Akubas buvo vartininkai, jie ėjo sargybą prie vartų sandėlių. 
\par 26 Jie gyveno Jehocadako anūko, Ješūvos sūnaus Jehojakimo dienomis ir dienomis valdytojo Nehemijo bei Rašto žinovo kunigo Ezro. 
\par 27 Jeruzalės sienoms pašventinti buvo sušaukti levitai iš visų vietovių; jie turėjo atvykti Jeruzalėn, kad pašventinimas vyktų su džiaugsmu, padėkos giesmėmis ir su cimbolų, arfų ir psalterių muzika. 
\par 28 Giedotojų sūnūs susirinko iš Jeruzalės ir Netofos apylinkių, 
\par 29 iš Gilgalo, Gebos ir Azmaveto vietovių; nes giedotojai buvo pasistatę kaimus Jeruzalės apylinkėse. 
\par 30 Kunigai ir levitai apsivalė patys ir apvalė tautą, vartus ir sienas. 
\par 31 Aš užvedžiau Judo kunigaikščius ant sienos ir pavedžiau jiems vadovauti dviems dėkojančiųjų grupėms. Viena ėjo sienos viršumi į dešinę, link Šiukšlių vartų. 
\par 32 Juos sekė Hošaja ir pusė Judo kunigaikščių: 
\par 33 Azarija, Ezra, Mešulamas, 
\par 34 Judas, Benjaminas, Šemaja ir Jeremija; 
\par 35 iš kunigų sūnų su trimitais sekė Zacharija, sūnus Jehonatano, sūnaus Šemajos, sūnaus Matanijos, sūnaus Mikajos, sūnaus Zakūro, sūnaus Asafo, 
\par 36 ir jo broliai: Šemaja, Azarelis, Milalajas, Gilalajas, Maajas, Netanelis, Judas ir Hananis su Dievo vyro Dovydo muzikos instrumentais; Rašto žinovas Ezra ėjo jų priekyje. 
\par 37 Praėję pro Šaltinio vartus, jie lipo aukštyn Dovydo miesto laiptais, vedančiais pro Dovydo namus, iki rytinių Vandens vartų. 
\par 38 Antra grupė ėjo kairėn; paskui juos aš sekiau su puse tautos pro Krosnių bokštą iki Plačiosios sienos, 
\par 39 toliau pro Efraimo vartus, Senuosius vartus, Žuvų vartus, Hananelio bokštą ir Mejos bokštą iki Avių vartų, ir jie sustojo prie Sargybos vartų. 
\par 40 Abi grupės sustojo prie Dievo namų, taip pat aš, pusė viršininkų su manimi 
\par 41 ir kunigai: Eljakimas, Maasėja, Minjaminas, Mikaja, Eljoenajas, Zacharija, Hananija su trimitais, 
\par 42 Maasėja, Šemaja, Eleazaras, Uzis, Johananas, Malkija, Elamas ir Ezeras. Giedotojai, kuriems vadovavo Izrachija, garsiai giedojo. 
\par 43 Tą dieną jie daug aukojo ir džiaugėsi, nes Dievas jiems suteikė didelį džiaugsmą. Moterys ir vaikai taip pat džiaugėsi, ir Jeruzalės džiaugsmas buvo toli girdimas. 
\par 44 Tuo metu buvo paskirti žmonės prižiūrėti sandėliams, kurie buvo įrengti atsargoms, vaisių pirmienoms ir dešimtinėms, kas buvo surinkta iš miestų ir kaimų, pagal įstatymą išlaikyti kunigus ir levitus. Judo gyventojai džiaugėsi kunigais ir levitais. 
\par 45 Giedotojai ir vartininkai atliko savo tarnystę Dievui ir apvalymo tarnystę, kaip buvo įsakęs Dovydas ir jo sūnus Saliamonas. 
\par 46 Jau Dovydo ir Asafo laikais buvo paskirti vadovai giedotojams, kurie giedojo šlovinimo ir padėkos giesmes Dievui. 
\par 47 Zorobabelio ir Nehemijo dienomis izraelitai pristatydavo giedotojams ir vartininkams skirtą kasdieninę dalį. Jie duodavo šventas dovanas levitams, o levitai duodavo tai Aarono vaikams.



\chapter{13}


\par 1 Skaitant susirinkusiems Mozės knygą, buvo rasta parašyta: “Joks amonitas ar moabitas negali priklausyti Dievo tautai, 
\par 2 nes jie nepasitiko izraelitų su duona ir vandeniu, bet pasamdė Balaamą, kad juos prakeiktų. Tačiau mūsų Dievas pakeitė prakeikimą į palaiminimą”. 
\par 3 Išgirdę įstatymą, jie atskyrė visus svetimtaučius iš Izraelio. 
\par 4 Prieš tai kunigas Eljašibas, Tobijos giminaitis, Dievo namų sandėlių prižiūrėtojas, 
\par 5 buvo parūpinęs Tobijai didelį kambarį, kuriame anksčiau laikydavo duonos aukas, smilkalus, indus ir javų, vyno bei aliejaus dešimtinę, priklausančią levitams, giedotojams, vartininkams ir kunigams. 
\par 6 Tuo laiku aš nebuvau Jeruzalėje, nes trisdešimt antraisiais Babilono karaliaus Artakserkso metais buvau pas karalių. Tačiau kuriam laikui praėjus, aš pasiprašiau karaliaus išleidžiamas. 
\par 7 Sugrįžęs į Jeruzalę, sužinojau, kad Eljašibas blogai pasielgė, parūpindamas Tobijai kambarį Dievo namų kiemuose. 
\par 8 Man tai labai nepatiko, todėl aš išmečiau iš kambario laukan visus Tobijo sdaiktus. 
\par 9 Po to įsakiau išvalyti kambarius ir atgal sunešti Dievo namų reikmenis, duonos aukas ir smilkalus. 
\par 10 Aš taip pat sužinojau, kad levitams nebuvo atiduodama jiems priklausanti dalis ir kad levitai ir giedotojai neatliko savo tarnystės, bet sugrįžo kiekvienas į savo laukus. 
\par 11 Tuomet aš bariau vyresniuosius ir sakiau: “Kodėl Dievo namai apleisti?” Surinkęs visus, grąžinau juos į jų vietą. 
\par 12 Tada visas Judas atnešė į sandėlius grūdų, vyno ir aliejaus dešimtinę. 
\par 13 Aš paskyriau sandėlių prižiūrėtojais kunigą Šelemiją, raštininką Cadoką ir Pedają iš levitų, o jų padėjėju­Matanijos sūnaus Zakūro sūnų Hananą, nes jais buvo galima pasitikėti. Jie buvo įpareigoti išduoti broliams jiems priklausančią dalį. 
\par 14 Mano Dieve, atsimink mane ir nepamiršk mano gerų darbų, kuriuos padariau Dievo namams ir tarnavimui juose. 
\par 15 Tuomet aš mačiau Jude tuos, kurie per sabatą mynė vyno spaustuvus ir, suvalydami javus, krovė juos ant asilų; vyno, vynuogių, figų ir visokių naštų sabato dieną gabeno į Jeruzalę. Aš įspėjau juos, kad tą dieną nepardavinėtų maisto. 
\par 16 Jeruzalėje gyveną Tyro gyventojai atgabendavo žuvies bei visokių prekių ir per sabatą parduodavo Judo žmonėms. 
\par 17 Aš bariau Judo kilminguosius: “Kodėl taip piktai elgiatės ir išniekinate sabato dieną? 
\par 18 Taip darė mūsų tėvai, todėl mūsų Dievas užleido mums ir šitam miestui tokią nelaimę. Jūs užtrauksite rūstybę Izraeliui, paniekindami sabatą”. 
\par 19 Aš įsakiau sabato išvakarėse, pradėjus temti, užrakinti Jeruzalės vartus ir atidaryti juos tik sabatui pasibaigus ir pastačiau savo tarnus prie vartų, kad nebūtų įnešamos naštos per sabatą. 
\par 20 Pirkliai ir visokių prekių pardavėjai vieną ir kitą kartą pasiliko už Jeruzalės vartų. 
\par 21 Aš klausiau jų: “Kodėl pasiliekate už vartų? Jei padarysite tai dar kartą, aš panaudosiu prieš jus jėgą”. Nuo to laiko jie nebeateidavo sabate. 
\par 22 Levitams įsakiau apsivalyti, eiti sargybą prie vartų ir švęsti sabatą. Todėl, Viešpatie, mano Dieve, atsimink ir pasigailėk manęs dėl savo didžio gailestingumo. 
\par 23 Tais laikais aš mačiau ir žydus, imančius žmonas iš Ašdodo, Amono ir Moabo. 
\par 24 Jų vaikai kalbėjo pusiau Ašdodo gyventojų kalba ir nemokėjo žydiškai. 
\par 25 Aš subariau juos, iškeikiau, kai kuriuos sumušiau, roviau jų plaukus ir prisaikdinau, kad neduotų savo dukterų jų sūnums, neleistų savo sūnums vesti jų dukterų ir patys nevestų jų: 
\par 26 “Argi ne tuo nusidėjo Izraelio karalius Saliamonas? Juk jam nebuvo lygių žemėje. Jis buvo Dievo mylimas ir Dievas jį buvo padaręs karaliumi visam Izraeliui, tačiau net jį svetimtautės moterys įtraukė į nuodėmę. 
\par 27 Argi ir mes neklausysime, padarysime tą didelę piktenybę ir nusikalsime prieš Dievą, vesdami svetimtautes?” 
\par 28 Vienas vyriausiojo kunigo Eljašibo sūnaus Jehojados sūnus buvo horoniečio Sanbalato žentas; aš jį išvariau nuo savęs. 
\par 29 Viešpatie, mano Dieve, prisimink tuos, kurie išniekino kunigystę, kunigų ir levitų sandorą. 
\par 30 Taip aš apvaliau juos nuo svetimtaučių ir atstačiau kunigų bei levitų tarnavimus, kiekvieną jam skirtoje vietoje, 
\par 31 ir įsakiau skirtu laiku atgabenti malkų ir pirmavaisių. Mano Dieve, atsimink mane mano labui.



\end{document}