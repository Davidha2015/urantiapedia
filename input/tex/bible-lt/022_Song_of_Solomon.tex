\begin{document}

\title{Giesmių giesmės knyga}


\chapter{1}


\par 1 Giesmių giesmė, Saliamono sukurta. 
\par 2 Tepabučiuoja jis mane savo burnos pabučiavimu, nes tavo meilė yra geriau už vyną. 
\par 3 Kvapas tavo tepalų yra malonus. Kaip išlietas brangus aliejus yra tavo vardas, todėl mergaitės myli tave. 
\par 4 Nusivesk mane! Mes bėgsime paskui tave. Karalius parsivedė mane į savo kambarius! Mes džiaugsimės ir linksminsimės dėl tavęs, mes prisiminsime tavo meilę labiau negu vyną. Jos tikrai tave myli. 
\par 5 Jeruzalės dukros, aš esu tamsi, bet graži kaip Kedaro palapinės, kaip Saliamono užuolaidos. 
\par 6 Nežiūrėkite į mane, kad aš tamsi, nes saulė nudegino mane. Mano broliai pyko ant manęs ir liepė man saugoti vynuogynus, bet savo vynuogyno aš nesaugojau. 
\par 7 Pasakyk man tu, kurį mano siela myli, kur tu ganai, kur tavo banda pietų metu ilsisi. Kodėl aš turėčiau klaidžioti prie tavo draugų bandų? 
\par 8 Jei tu nežinai, o gražiausioji, sek bandos pėdomis ir ganyk savo ožiukus šalia piemenų palapinių. 
\par 9 Mano mylimoji, aš tave lyginu su faraono kovos vežimų žirgais. 
\par 10 Tavo skruostai gražūs tarp papuošalų, tavo kaklas papuoštas karoliais. 
\par 11 Mes padarysime tau auksinę grandinėlę su sidabro gėlytėmis. 
\par 12 Karaliui sėdint prie stalo, mano nardas skleidė savo kvapą. 
\par 13 Mano mylimasis yra man kaip miros ryšulėlis, kabąs man ant krūtinės. 
\par 14 Mano mylimasis yra man kaip kamparo žiedai En Gedžio vynuogyne. 
\par 15 Tu graži, mano mylimoji, tu graži. Tavo akys kaip balandėlės. 
\par 16 Tu graži, mano mylimoji, tikrai maloni! Mūsų guolis­žalumynuose, 
\par 17 mūsų namų rąstai­kedrai, o lubos­kiparisai.


\chapter{2}


\par 1 Aš esu rožė iš Sarono, slėnių lelija. 
\par 2 Kaip lelija tarp erškėčių, taip mano mylimoji tarp dukterų. 
\par 3 Kaip obelis tarp miško medžių, taip mano mylimasis tarp sūnų. Su džiaugsmu sėdžiu jo ūksmėje, jo vaisius man saldus. 
\par 4 Jis atvedė mane į puotos namus, jo vėliava virš manęs­meilė. 
\par 5 Atgaivinkite mane vynuogėmis, sustiprinkite obuoliais, nes aš alpstu iš meilės. 
\par 6 Jo kairė ranka po mano galva, o dešinė apkabina mane. 
\par 7 Saikdinu jus, Jeruzalės dukros, laukų stirnomis ir elnėmis, nežadinkite ir nekelkite mano mylimosios, kol ji pati nenorės. 
\par 8 Mano mylimojo balsas! Jis ateina šokinėdamas per kalnus ir kalnelius. 
\par 9 Mano mylimasis yra lyg stirna ar jaunas briedis. Štai jis jau stovi už sienos ir žiūri pro lango groteles. 
\par 10 Mano mylimasis man kalba: “Kelkis, mano mylimoji, mano gražuole, ateik! 
\par 11 Žiema jau praėjo, lietus pasibaigė ir liovėsi. 
\par 12 Gėlės jau pasirodė žemėje; atėjo giedojimo metas, ir balandžių balsai girdimi krašte. 
\par 13 Figmedžio pumpurai sprogsta, vynuogynai žydi ir kvepia. Kelkis, mano mylimoji, mano gražuole, ateik! 
\par 14 Mano balandėle, gyvenanti uolų plyšiuose, parodyk savo veidą! Leisk išgirsti tavo balsą, nes tavo balsas gražus ir veidas žavus”. 
\par 15 Sugaukite lapes, mažas laputes, kurios gadina vynuogynus, nes mūsų vynuogynas žydi. 
\par 16 Mano mylimasis yra mano, o aš jo. Jis gano tarp lelijų. 
\par 17 Kol diena aušta ir šešėliai pabėga, skubėk pas mane, mano mylimasai, kaip stirna, kaip jaunas briedis per Beterio kalnus.



\chapter{3}


\par 1 Naktį savo guolyje ieškojau to, kurį myliu; aš ieškojau jo, bet neradau. 
\par 2 Aš kelsiuos, eisiu į miestą, vaikštinėsiu miesto gatvėmis ir aikštėmis, ieškosiu savo mylimojo. Aš ieškojau jo, bet neradau. 
\par 3 Mane sutiko miesto sargybiniai, vaikščiodami miesto gatvėmis. Aš paklausiau, ar jie nematė mano mylimojo. 
\par 4 Praėjusi pro juos, radau tą, kurį myliu. Pagavau jį ir nepaleidau, kol neįsivedžiau į savo motinos namus, į kambarį savo gimdytojos. 
\par 5 Saikdinu jus, Jeruzalės dukros, laukų stirnomis ir elnėmis, nežadinkite ir nekelkite mano mylimosios, kol ji pati nenorės. 
\par 6 Kas ten lyg dūmų stulpas ateina iš dykumos; iškvėpintas mira, kvepalais ir visokiais pirklių milteliais. 
\par 7 Tai Saliamono neštuvai! Šešiasdešimt karių, Izraelio karžygių, jį lydi. 
\par 8 Jie visi ginkluoti kardais, įgudę kovotojai; kiekvieno kardas prie juosmens, paruoštas nakties pavojui. 
\par 9 Karaliaus Saliamono neštuvai padaryti iš Libano medžių. 
\par 10 Neštuvų atramos sidabrinės, atlošas auksinis, sėdynė purpuru aptraukta, o vidus su meile papuoštas Jeruzalės dukterų. 
\par 11 Siono dukros, išeikite pamatyti karalių Saliamoną su karūna, kuria jo motina karūnavo jį vestuvių dieną, jo linksmybių dieną.



\chapter{4}


\par 1 Tu graži, mano mylimoji, tu graži. Tavo akys kaip balandėlės spindi iš po garbanų. Tavo plaukai kaip ožkų kaimenė, besileidžianti nuo Gileado kalnų. 
\par 2 Tavo dantys kaip banda nukirptų avių, išeinančių iš maudyklės; jos visos turi dvynukus, nė vienos nėra bergždžios. 
\par 3 Tavo lūpos kaip raudonos juostelės, o kalba meili. Tavo skruostai kaip granato vaisiaus šonai po tavo garbanomis. 
\par 4 Tavo kaklas kaip Dovydo bokštas, pastatytas ginklams, ant kurio kabo tūkstantis karžygių skydų. 
\par 5 Tavo dvi krūtys kaip stirnos dvynukai, besiganantys tarp lelijų. 
\par 6 Kol diena aušta ir šešėliai pabėga, aš eisiu prie miros kalno ir smilkalų kalvos. 
\par 7 Tu graži, mano mylimoji, ir nėra tavyje dėmės. 
\par 8 Ateik nuo Libano, mano sužadėtine, ateik pas mane! Nusileisk nuo Amanos, Senyro ir Hermono kalnų viršūnių; išeik iš liūtų guolių, nuo leopardų kalnų. 
\par 9 Tu pagrobei mano širdį, mano sesuo, mano sužadėtine! Tu pagrobei mano širdį vienu savo akių žvilgsniu, vienu savo karolių perlu. 
\par 10 Kaip maloni yra tavo meilė, mano sesuo, mano sužadėtine! Tavo meilė yra malonesnė už vyną, o tavo tepalų kvapas­už visus brangiausius kvepalus. 
\par 11 Mano sužadėtine, tavo lūpos varva kaip korys. Po tavo liežuviu yra medus ir pienas. Tavo rūbų kvapas kaip kvapas smilkalų iš Libano. 
\par 12 Mano sesuo, mano sužadėtine, tu esi užrakintas sodas, užantspauduotas šaltinis. 
\par 13 Tavo sode auga granatai ir rinktiniai vaismedžiai, kamparas ir nardas; 
\par 14 nardas ir šafranas, kvepiančios nendrės ir cinamonas bei įvairūs smilkalų medžiai; mira ir alavijas su geriausiais aromatais; 
\par 15 sodo šaltinis­gyvybės vanduo ir tekantys upeliai nuo Libano. 
\par 16 Pakilk šiaury, pūsk, pietų vėjau, į mano sodą, kad jo kvapas sklistų toli! Tegul mano mylimasis ateina į savo sodą ir valgo jo skanių vaisių.



\chapter{5}


\par 1 Aš atėjau į savo sodą, mano sesuo ir sužadėtine; rinkau mirą ir kvepalus, valgiau medaus iš korių, gėriau pieno ir vyno. Draugai, valgykite ir gerkite, mano mylimieji! 
\par 2 Aš miegu, bet mano širdis budi. Tai balsas mano mylimojo, kuris beldžia į duris: “Atidaryk, mano sesuo, mano mylimoji, mano balandėle, mano tyroji! Mano galva pilna rasos, o mano garbanos­ nakties lašų”. 
\par 3 Aš nusivilkau savo drabužius, kaip aš juos vėl apsivilksiu? Nusiploviau kojas, kaip vėl jas sutepsiu? 
\par 4 Mano mylimasis įkišo ranką pro duris, ir mano širdis suvirpėjo nuo jo. 
\par 5 Aš atsikėliau atidaryti savo mylimajam. Nuo mano rankų ir pirštų varvėjo mira ant durų skląsčio. 
\par 6 Aš atidariau duris savo mylimajam, bet jis buvo nuėjęs. Mano siela alpo, kai jis kalbėjo. Aš ieškojau jo, bet neradau; šaukiau, bet jis neatsiliepė. 
\par 7 Mane pastebėjo miesto sargai. Jie mušė ir sužeidė mane; sienų sargai nuplėšė mano apsiaustą. 
\par 8 Saikdinu jus, Jeruzalės dukros, jei rasite mano mylimąjį, pasakykite jam, kad alpstu iš meilės. 
\par 9 Gražiausioji tarp moterų, kuo tavo mylimasis skiriasi iš kitų, kad mus taip saikdini? 
\par 10 Mano mylimasis skaistus ir įraudęs, geriausias iš dešimties tūkstančių. 
\par 11 Jo galva yra lyg brangiausias auksas, jo plaukai banguoti ir juodi kaip varnas. 
\par 12 Jo akys kaip balandėliai prie vandens upelių, nuplauti piene, labai jam tinka. 
\par 13 Jo skruostai kaip lysvės, kuriose auga kvepiančios žolės. Jo lūpos kaip lelijos, varvančios geriausia mira. 
\par 14 Jo rankos lyg aukso žiedai su berilio akmenimis. Jo liemuo lyg iš dramblio kaulo, pagražintas safyrais. 
\par 15 Jo kojos lyg marmuro kolonos, stovinčios ant auksinių papėdžių. Jis atrodo kaip Libanas, puikus kaip kedras. 
\par 16 Jo burna labai saldi, jis visas žavingas. Jeruzalės dukros, toks yra mano mylimasis, mano draugas.



\chapter{6}


\par 1 Pasakyk, gražiausioji tarp moterų, kur nuėjo tavo mylimasis? Kuriuo keliu pasuko jis, kad galėtume kartu su tavimi jo ieškoti? 
\par 2 Mano mylimasis nuėjo į savo sodą, prie kvepiančių augalų lysvių; ten jis gano avis ir skina lelijas. 
\par 3 Aš esu mylimojo, o jis yra mano; jis gano tarp lelijų. 
\par 4 Graži tu, mano mylimoji, kaip Tirca, puošni kaip Jeruzalė, bauginanti kaip kariuomenė su vėliavomis. 
\par 5 Nežiūrėk į mane, nes tavo akys mane nugalėjo. Tavo plaukai kaip ožkų kaimenė, besileidžianti nuo Gileado. 
\par 6 Tavo dantys kaip avių banda, išeinanti iš maudyklės; jos visos turi dvynukus, nė vienos nėra bergždžios. 
\par 7 Tavo skruostai kaip granato vaisiaus šonai po tavo garbanomis. 
\par 8 Yra šešiasdešimt karalienių, aštuoniasdešimt sugulovių ir mergaičių be skaičiaus. 
\par 9 Bet tik viena yra mano balandėlė, mano tyroji, vienintelė pas savo motiną, išskirtinė tai, kuri ją pagimdė. Mergaitės matė ir laimino ją. Karalienės ir sugulovės taip pat ją gyrė. 
\par 10 Kas yra toji, kuri pasirodo lyg ryto aušra, graži kaip mėnulis, šviesi kaip saulė, bauginanti kaip kariuomenė su vėliavomis? 
\par 11 Aš nuėjau pažiūrėti į riešutų sodą, ar pražydo vynuogynas, ar išsprogo granato medžiai. 
\par 12 Net nesuvokiau, kaip mano siela nusinešė mane tarsi Aminadabo kovos vežimai. 
\par 13 Sugrįžk, sugrįžk, šulamiete! Sugrįžk, kad galėtume pamatyti tave! Ką gi matysime? Tik dvi pasiruošusias kariuomenes.



\chapter{7}


\par 1 Kokios gražios tavo kojos su kurpėmis, kunigaikščio dukra! Tavo strėnų išlenkimai kaip papuošalai, kaip menininko rankų darbas. 
\par 2 Tavo liemuo lyg ištekinta taurė, kurioje niekada nestinga gėrimo, tavo pilvas lyg kviečių krūva, apkaišiota lelijomis. 
\par 3 Tavo dvi krūtys lyg stirnos dvynukai. 
\par 4 Tavo kaklas lyg dramblio kaulo bokštas. Tavo akys lyg Hešbono tvenkiniai prie Batrabimo vartų. Tavo nosis lyg Libano bokštas, nukreiptas Damasko link. 
\par 5 Tavo galva lyg Karmelis. Tavo galvos plaukai lyg purpuras, karalius sužavėtas tavo garbanomis. 
\par 6 Kokia graži ir miela pažiūrėti tu esi, mano mylimoji! 
\par 7 Tavo stuomuo yra panašus į palmę, tavo krūtys­į vynuogių kekes. 
\par 8 Aš tariau: “Įkopsiu į palmę, įsikibsiu į jos šakas”. Tavo krūtys bus kaip vynuogių kekės, tavo burnos kvapas­kaip obuolių. 
\par 9 Tavo lūpos yra lyg geriausias vynas, kuris švelniai slenka gomuriu ir prakalbina apsnūdusį. 
\par 10 Aš priklausau savo mylimajam, o jis geidžia manęs. 
\par 11 Ateik, mylimasis, eikime į laukus, nakvokime kaime. 
\par 12 Anksti rytą apžiūrėkime vynuogynus, ar jau sprogsta vynmedžiai, ar skleidžiasi žiedai, ar pražydo granato medžiai. Ten aš tau atiduosiu savo meilę. 
\par 13 Mandragoros kvepia, prie mūsų durų yra įvairiausių vaisių, šviežių ir senų, kuriuos laikiau tau, mano mylimasai.



\chapter{8}


\par 1 O kad tu būtum mano brolis, kurį maitino mano motina! Tada, sutikus tave lauke, galėčiau bučiuoti, niekas manęs neniekintų. 
\par 2 Aš paimčiau ir vesčiau tave į savo motinos namus, į kambarius, kur gimiau. Aš duočiau tau gerti kvepiančio vyno ir granato sulčių. 
\par 3 Jo kairė ranka po mano galva, o dešinė apkabina mane. 
\par 4 Saikdinu jus, Jeruzalės dukros, nežadinkite ir nekelkite mano mylimosios, kol ji pati nenorės. 
\par 5 Kas yra ta, kuri ateina iš dykumos, pasiremdama į savo mylimąjį? Po obelim aš pažadinau tave, kur motina tave pagimdė. 
\par 6 Laikyk mane kaip antspaudą prie savo širdies, kaip apyrankę ant rankos. Meilė yra stipri kaip mirtis, pavydas žiaurus kaip mirusiųjų buveinė. Jos karštis yra ugnies karštis, stipriausia liepsna. 
\par 7 Daugybė vandenų neužgesins meilės ir srovės nepaskandins jos. Jei žmogus duotų už meilę visus savo turtus, būtų visiškai paniekintas. 
\par 8 Mes turime mažą seserį, ji neturi krūtų. Ką darysime su ja, kai ateis jai laikas ištekėti? 
\par 9 Jei ji būtų mūras, pastatytume ant jos sidabrinių bokštų, jei ji būtų durys, apkaltume ją kedro lentomis. 
\par 10 Aš esu mūras, ir mano krūtys kaip bokštai. Aš atradau palankumą jo akyse. 
\par 11 Saliamonas turėjo vynuogyną Baal Hamone, kurį išnuomojo. Kiekvienas už vaisius privalėjo jam mokėti tūkstantį šekelių sidabro. 
\par 12 Mano vynuogynas priklauso tik man. Tau, Saliamonai, duodu tūkstantį, o jo prižiūrėtojams už vaisius­du šimtus. 
\par 13 Tu, kuri gyveni soduose, draugai tavęs klausosi. Leisk man išgirsti tavo balsą. 
\par 14 Skubėk, mano mylimasai, būk kaip stirna ar jaunas briedis, bėgąs kvepiančiais kalnais.



\end{document}