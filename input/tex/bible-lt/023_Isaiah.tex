\begin{document}

\title{Izaijo knyga}

\chapter{1}


\par 1 Izaijo, Amoco sūnaus, regėjimas, kurį jis matė apie Judą ir Jeruzalę Judo karalių Uzijo, Jotamo, Ahazo ir Ezekijo dienomis. 
\par 2 Dangūs, klausykite, žeme, išgirsk! Viešpats kalba: “Aš užauginau ir išaukštinau vaikus, bet jie sukilo prieš mane. 
\par 3 Jautis pažįsta savo savininką ir asilas­savo šeimininko ėdžias, bet Izraelis nepažįsta, mano tauta nesupranta”. 
\par 4 Vargas nuodėmingai giminei, nuo nedorybių apsunkusiai tautai, piktadarių palikuonims, nepaklusniems vaikams! Jie paliko Viešpatį, supykdė Izraelio Šventąjį, nusigręžė nuo Jo. 
\par 5 Ar verta jus dar daugiau bausti, kurie ir toliau maištaujate? Jau visa galva nesveika, širdis alpsta. 
\par 6 Nuo kojų padų iki viršugalvio nebėra nieko sveiko: sumušimai, randai, pūliuojančios žaizdos­nevalytos, neaprištos, aliejumi nepateptos. 
\par 7 Jūsų šalis ištuštėjusi, miestai sudeginti, laukų vaisius svetimieji ryja jūsų akivaizdoje. Visa yra sunaikinta svetimųjų. 
\par 8 Siono dukra palikta kaip palapinė vynuogyne, kaip sargo būda agurkyne, kaip apgultas miestas. 
\par 9 Jei kareivijų Viešpats nebūtų palikęs mums mažo likučio, mes būtume kaip Sodoma, panašūs į Gomorą. 
\par 10 Klausykite Viešpaties žodžio, jūs Sodomos valdovai! Išgirsk mūsų Dievo įstatymą, Gomoros tauta! 
\par 11 “Kam man daugybė jūsų aukų?­sako Viešpats.­Man jau užtenka jūsų avinų deginamųjų aukų ir penimų gyvulių taukų. Aš nemėgstu jaučių, avių ir ožių kraujo. 
\par 12 Kas reikalauja iš jūsų viso to, kai ateinate pasirodyti man ir mindote mano kiemus? 
\par 13 Neaukokite man tuščių aukų, smilkalai man yra pasibjaurėjimas. Aš negaliu pakęsti jauno mėnulio, sabatų ir kitų švenčių. Netgi jūsų iškilmingi susirinkimai yra nedorybė. 
\par 14 Jūsų jauno mėnulio ir kitų iškilmių mano siela nekenčia. Jos mane slegia ir Aš pavargau nuo jų. 
\par 15 Kai jūs iškeliate rankas, Aš nusisuku nuo jūsų, kai kalbate daug maldų, neklausau, nes jūsų rankos kruvinos. 
\par 16 Nusiplaukite, apsivalykite; pašalinkite savo piktus darbus iš mano akių, liaukitės darę pikta! 
\par 17 Mokykitės daryti gera, ieškokite teisingumo, padėkite prispaustiesiems, apginkite našlaičius, užstokite našles. 
\par 18 Tada ateikite ir kartu pasvarstysime,­sako Viešpats.­Nors jūsų nuodėmės būtų skaisčiai raudonos, taps baltos kaip sniegas; nors būtų raudonos kaip purpuras, taps kaip vilna. 
\par 19 Jei jūs noriai klausysite manęs, valgysite krašto gėrybes. 
\par 20 Bet jei jūs priešinsitės ir maištausite, jus praris kardas”. Taip kalbėjo Viešpats. 
\par 21 Kaip ištikimas miestas tapo paleistuve? Jis buvo pilnas teisingumo, teisumas gyveno jame, o dabar­žmogžudžiai. 
\par 22 Tavo sidabras pavirto nuodegomis, vynas sumaišytas su vandeniu. 
\par 23 Tavo kunigaikščiai tapo maištininkais ir vagių draugais; visi mėgsta kyšius ir laukia dovanų. Jie nebegina našlaičių, našlės byla nepasiekia jų. 
\par 24 Taip sako Viešpats, kareivijų Dievas, Izraelio Galingasis: “Aš atlyginsiu ir atkeršysiu savo priešams! 
\par 25 Aš pakelsiu savo ranką prieš tave, visiškai nuvalysiu nuo tavęs nuodegas, pašalinsiu iš tavęs priemaišas. 
\par 26 Aš grąžinsiu tau teisėjus, kokie jie buvo, ir patarėjus kaip pradžioje. Tuomet tu būsi teisumo ir ištikimybės miestas. 
\par 27 Sionas bus išgelbėtas teisingumu ir jo atgailaujantys­teisumu. 
\par 28 Nusikaltėliai ir nusidėjėliai bus sunaikinti, kurie apleidžia Viešpatį, pražus. 
\par 29 Jūs gėdysitės ąžuolų, kuriais džiaugėtės, ir jūs raudonuosite dėl sodų, kuriuos buvote pasirinkę. 
\par 30 Jūs būsite kaip ąžuolas suvytusiais lapais, kaip sodas be vandens. 
\par 31 Stiprusis taps lyg pakulos, jo darbas lyg kibirkštis; jie sudegs kartu, ir niekas jų neužgesins”.


\chapter{2}


\par 1 Regėjimas, kurį matė Amoco sūnus Izaijas, apie Judą ir Jeruzalę. 
\par 2 Paskutinėmis dienomis Viešpaties namų kalnas iškils aukščiau už visus kalnus ir kalnelius. Į jį plūs visos tautos. 
\par 3 Daug tautų ateis ir sakys: “Eikime prie Viešpaties kalno, į Jokūbo Dievo namus, kad Jis mokytų mus savo kelių ir mes vaikščiotume Jo takais”. Nes iš Siono išeis įstatymas, o iš Jeruzalės­Viešpaties žodis. 
\par 4 Jis teis tautas ir sudraus daugelį tautų. Jie perkals savo kardus į noragus ir ietis į pjautuvus. Tautos nebekariaus tarpusavyje ir nebesimokys kariauti. 
\par 5 Jokūbo namai, ateikite ir visi vaikščiokime Viešpaties šviesoje! 
\par 6 Todėl Tu atmetei savo tautą, Jokūbo namus, kad jie pasidarė kaip rytiečiai, jie turi burtininkų kaip filistinai. Jie susideda su svetimaisiais. 
\par 7 Jų kraštas yra pilnas sidabro ir aukso, ir nėra galo jų turtams. Jų šalis pilna žirgų ir kovos vežimų be skaičiaus. 
\par 8 Jų kraštas taip pat pilnas stabų; jie garbina savo rankų darbus, kuriuos pagamino jų pačių pirštai. 
\par 9 Ten lenkiasi žmogus, žeminasi vyras. Neatleisk jiems! 
\par 10 Eik į uolą ir slėpkis dulkėse nuo Viešpaties baimės ir Jo didingumo šlovės. 
\par 11 Išdidūs žmonių žvilgsniai bus pažeminti ir jų puikybė palaužta. Tik vienas Viešpats bus išaukštintas tą dieną. 
\par 12 Nes kareivijų Viešpaties diena užgrius kiekvieną, kuris išdidus ir pasipūtęs, tą kuris didžiuojasi, ir jie bus pažeminti: 
\par 13 visi Libano kedrai, aukšti ir išdidūs, ir visi Bašano ąžuolai; 
\par 14 visi aukšti kalnai ir visos aukštai iškilusios kalvos; 
\par 15 kiekvienas aukštas bokštas ir kiekviena sustiprinta siena; 
\par 16 visi Taršišo laivai ir visa, kas gražiai atrodo. 
\par 17 Žmogaus išdidumas bus pažemintas, žmonių puikybė bus palaužta. Tik vienas Viešpats bus išaukštintas tą dieną. 
\par 18 Stabus Jis visai sunaikins. 
\par 19 Jie slėpsis uolų plyšiuose ir žemės urvuose nuo Viešpaties baimės ir Jo didingumo šlovės, kai Jis pakils sudrebinti žemės. 
\par 20 Tą dieną žmogus išmes savo sidabrinius ir auksinius stabus, kuriuos pasidaręs garbino, kurmiams ir šikšnosparniams, 
\par 21 kad pasislėptų uolų urvuose ir akmenų plyšiuose nuo Viešpaties baimės ir Jo didingumo šlovės, kai Jis pakils sudrebinti žemės. 
\par 22 Atsitraukite nuo žmogaus, kurio kvėpavimas šnervėse; kokia yra jo vertė?



\chapter{3}


\par 1 Štai Viešpats, kareivijų Dievas, atims iš Jeruzalės ir Judo ramstį ir lazdą, duonos ir vandens atsargas, 
\par 2 karžygį ir karį, teisėją ir pranašą, žynį ir vyresnįjį, 
\par 3 penkiasdešimtininką ir kilmingąjį, patarėją, įgudusį amatininką ir iškalbingą kalbėtoją. 
\par 4 Vaikai bus jų kunigaikščiai, juos valdys kūdikiai. 
\par 5 Žmonės skriaus vienas kitą, net savo artimą. Vaikai bus pasipūtę prieš vyresniuosius, niekam tikę prieš kilminguosius. 
\par 6 Tada žmogus, nutvėręs savo brolį iš tėvo namų, sakys: “Tu dar turi apsiaustą, būk mūsų vadas. Šie griuvėsiai tebūna tavo valdžioje”. 
\par 7 Jis atsakys: “Aš nesu gydytojas, mano namuose nėra nei duonos, nei drabužių. Nedarykite manęs tautos vadu”. 
\par 8 Jeruzalė ir Judas kris, nes jų žodžiai ir darbai priešingi Viešpačiui; jie erzina Jo šlovės akis. 
\par 9 Jų veido išraiška liudija prieš juos. Jie nesislėpdami nuodėmiauja kaip Sodomos gyventojai. Vargas jiems! Jie patys sau užtraukia nelaimę. 
\par 10 Sakykite teisiesiems, kad jiems bus gerai, nes jie naudosis savo darbų vaisiais. 
\par 11 Vargas nedorėliui! Jam bus atlyginta pagal jo darbus. 
\par 12 Mano tauta! Vaikai juos spaudžia, moterys jiems viešpatauja. O mano tauta! Tavo vedliai tave suvedžioja ir veda klaidingu taku. 
\par 13 Viešpats keliasi daryti teismo ir teisti tautas. 
\par 14 Viešpats teis vyresniuosius ir savo tautos kunigaikščius. “Jūs jau prarijote vynuogyną. Beturčių nuosavybė yra jūsų namuose. 
\par 15 Ką galvojate, skriausdami mano tautą ir daužydami beturčiams per veidus?”­klausia Viešpats, kareivijų Dievas. 
\par 16 Viešpats sako: “Kadangi Siono dukterys išdidžiai vaikščioja, ištiesusios kaklus, mėto geidulingus žvilgsnius, eidamos mažais žingsneliais, skambina kojų žiedais, 
\par 17 Viešpats apdengs Siono dukterų galvas šašais ir apnuogins jas”. 
\par 18 Tą dieną Viešpats nuplėš jų papuošalus: kojų grandinėles, kaktos juosteles ir mėnuliukus; 
\par 19 grandinėles, apyrankes ir šydus; 
\par 20 vainikėlius, kojų papuošalus, puošnias juostas, kvepalų dėžutes ir auskarus; 
\par 21 pirštų ir nosies žiedus, 
\par 22 išeiginius rūbus ir apsiaustus; skareles ir pinigines, 
\par 23 veidrodėlius, plono lino drabužius, galvos kaspinus ir skraistes. 
\par 24 Vietoje kvepalų bus smarvė, vietoje juostos­virvė; vietoje gražios šukuosenos­plikė, vietoje apsiausto­ašutinė; vietoje grožio­įdeginta žymė. 
\par 25 Tavo vyrai kris nuo kardo, o tavo karžygiai­kovoje. 
\par 26 Jos vartai aimanuos ir gedės, apleista ji sėdės ant žemės.



\chapter{4}


\par 1 Tą dieną septynios moterys, nutvėrusios vieną vyrą, sakys: “Mes valgysime savo duoną ir vilkėsime savais drabužiais, tik leisk mums vadintis tavo vardu, nuimk nuo mūsų panieką”. 
\par 2 Tą dieną Viešpaties atžala bus graži ir šlovinga, o žemės vaisius bus pasididžiavimas ir garbė išlikusiųjų Izraelyje. 
\par 3 Išlikęs Sione ir gyvenąs Jeruzalėje bus vadinamas šventu­kiekvienas, kuris bus įrašytas gyventi Jeruzalėje. 
\par 4 Kai Viešpats nuplaus Siono dukters nešvarumą ir išvalys Jeruzalės kraują teismo ir ugnies dvasia, 
\par 5 tada Viešpats padarys ant viso Siono kalno ir ant susirinkusių debesį ir dūmus dieną, o naktį skaisčiai liepsnojančią ugnį. Viešpaties šlovė bus apsauga visiems. 
\par 6 Tai bus ūksmė dienos metu nuo karščio ir priedanga bei užuovėja nuo audros ir lietaus.



\chapter{5}


\par 1 Aš giedosiu apie savo mylimojo draugo vynuogyną. Mano mylimasis turėjo vynuogyną derlingame kalnelyje. 
\par 2 Jis aptvėrė jį ir išrinko iš jo akmenis, prisodino jame rinktinių vynmedžių. Jo viduryje pastatė bokštą ir padirbo spaustuvą. Jis laukė užaugant vynuogių, bet užaugo rūgščios uogos. 
\par 3 “Dabar, Jeruzalės ir Judo gyventojai, darykite teismą tarp manęs ir mano vynuogyno. 
\par 4 Ką galima dar padaryti mano vynuogynui, ko Aš nepadariau? Kodėl, kai Aš laukiau vynuogių, jis išaugino rūgščias uogas? 
\par 5 Dabar Aš jums pasakysiu, ką darysiu su savo vynuogynu: Aš pašalinsiu tvorą, ir jį nuganys, nugriausiu sieną, ir jį išmindžios. 
\par 6 Aš apleisiu jį: nebepurensiu ir nebeapkarpysiu jo. Jis apaugs usnimis ir erškėčiais. Aš įsakysiu nebelyti ant jo”. 
\par 7 Kareivijų Viešpaties vynuogynas yra Izraelio namai, o Judo žmonės­Jo mėgiamiausi daigai. Jis laukė teisingumo­ir štai liejamas kraujas, teisumo­ir štai šauksmas! 
\par 8 Vargas tiems, kurie jungia namą prie namo ir lauką prie lauko, kol nebelieka vietos kitiems, lyg jie vieni čia gyventų. 
\par 9 Kareivijų Viešpats kalbėjo man: “Daugybė didelių ir gražių namų bus tušti ir be gyventojų. 
\par 10 Dešimt akrų vynuogyno teduos vieną batą vyno, o iš homero sėklų išaugs efa derliaus”. 
\par 11 Vargas tiems, kurie keliasi anksti rytą girtuokliauti ir sėdi iki vėlaus vakaro, kol įkaista nuo vyno. 
\par 12 Jų pokyliuose yra arfa ir styginiai, būgnas, vamzdis ir vynas. Bet jie nekreipia dėmesio į Viešpaties veiksmus nei į Jo darbus. 
\par 13 Todėl mano tauta bus ištremta, nes neturi pažinimo; jos kilmingieji mirs badu, minia neturės net vandens. 
\par 14 Todėl mirusiųjų buveinė plačiai atvers savo nasrus. Tautos šlovė ir visa minia, visi triukšmautojai ir lėbautojai žengs į ją. 
\par 15 Žmogus bus parblokštas, karžygys pažemintas, išdidūs nuleis savo akis. 
\par 16 Kareivijų Viešpats bus išaukštintas teisme, šventas Dievas parodys savo šventumą teisumu. 
\par 17 Tada ėriukai ganysis kaip savo ganykloje ir svetimieji maitinsis tuo, kas liks po turtuolių. 
\par 18 Vargas tiems, kurie traukia neteisybę tuštybės virvėmis ir nuodėmę lyg vežimėlio vadžiomis. 
\par 19 Kurie sako: “Tepaskuba Jis, tepagreitina savo darbą, kad mes matytume, teįvykdo Izraelio Šventasis savo sprendimą, kad mes žinotume”. 
\par 20 Vargas tiems, kurie vadina pikta geru ir gera piktu, kurie paverčia tamsą šviesa ir šviesą tamsa, kurie padaro kartų saldžiu ir saldų karčiu. 
\par 21 Vargas tiems, kurie išmintingi savo pačių akyse ir sumanūs savo pačių manymu. 
\par 22 Vargas tiems, kurie yra karžygiai gerti vyną ir drąsūs vyrai maišyti stiprius gėrimus, 
\par 23 kurie išteisina nedorėlį už kyšius, o nekaltą pasmerkia. 
\par 24 Kaip liepsna sudegina ražienas ir sausa žolė pranyksta ugnyje, taip jų šaknys supus ir žiedai dulkėmis pavirs, nes jie atmetė kareivijų Viešpaties įstatymą ir paniekino Izraelio Šventojo žodį. 
\par 25 Todėl Viešpaties rūstybė užsidegė prieš Jo tautą. Jis pakėlė savo ranką ir ištiko juos. Kalnai sudrebėjo, jų lavonai kaip sąšlavos gulėjo gatvėse. Dėl viso to Jo rūstybė dar neatsileido, Jo ranka dar pakelta. 
\par 26 Jis duos ženklą toli esančiai tautai, sušvilps ir pašauks ją iš žemės pakraščių. Ir ji ateis skubėdama. 
\par 27 Tarp jų nebus nusilpusių nei nuvargusių, snaudžiančių nei miegančių. Nė vienam iš jų neatsiriš strėnų juosta ir neatsileis apavo dirželis. 
\par 28 Jų strėlės aštrios, visų lankai įtempti. Jų žirgų kanopos kaip titnagas, jų vežimų ratai kaip viesulas. 
\par 29 Jie riaumoja kaip liūtas, kaip jaunas liūtas. Jie, pagriebę grobį, nusineša jį, ir niekas jo iš jų neatims. 
\par 30 Tą dieną jie riaumos prieš juos lyg šėlstanti jūra. Pažvelgus į kraštą, bus gąsdinanti tamsa, šviesą uždengs debesys.



\chapter{6}


\par 1 Karaliaus Uzijo mirties metais mačiau Viešpatį aukštame ir pakeltame soste. Jo rūbas pripildė šventyklą. 
\par 2 Serafai stovėjo ties Juo. Kiekvienas iš jų turėjo po šešis sparnus: dviem jie dengė savo veidą, dviem kojas ir dviem skrido. 
\par 3 Jie šaukė vienas kitam, sakydami: “Šventas, šventas, šventas, kareivijų Viešpats; visa žemė pilna Jo šlovės”. 
\par 4 Durys sudrebėjo nuo šauksmo, o namai prisipildė dūmų. 
\par 5 Aš tariau: “Vargas man, aš esu žuvęs! Aš esu žmogus nešvariomis lūpomis ir gyvenu tarp žmonių nešvariomis lūpomis. Aš savo akimis mačiau Karalių, kareivijų Viešpatį!” 
\par 6 Vienas serafas atskrido prie manęs. Jis laikė rankoje žėruojančią anglį, kurią buvo pasiėmęs replėmis nuo aukuro. 
\par 7 Jis palietė ja mano lūpas ir tarė: “Štai ji palietė tavo lūpas, tavo kaltė panaikinta, nuodėmė apvalyta!” 
\par 8 Aš girdėjau Viešpaties balsą. Jis klausė: “Ką man pasiųsti? Kas eis už mus?” Aš atsiliepiau: “Aš čia, siųsk mane!” 
\par 9 Jis atsakė: “Eik ir sakyk šiai tautai: ‘Ir toliau girdėkite, bet nesupraskite, ir toliau žiūrėkite, bet nematykite’. 
\par 10 Sukietink širdį šitos tautos, apkurtink jos ausis ir aptemdyk akis, kad nematytų akimis, negirdėtų ausimis, nesuprastų širdimi, neatsiverstų ir nebūtų pagydyta”. 
\par 11 Aš klausiau: “Ar ilgai, Viešpatie?” Jis atsakė: “Kol miestai pasidarys tušti ir be gyventojų, namai be žmonių, o kraštas pavirs dykyne. 
\par 12 Viešpats ištrems žmones toli; kraštas visiškai ištuštės. 
\par 13 Jei iš jų liks dešimtoji dalis ir sugrįš, jie vėl bus naikinami. Tačiau kaip nukirtus uosį ar ąžuolą lieka kelmas, taip šventoji sėkla bus jų kelmas”.



\chapter{7}


\par 1 Judo karaliaus Achazo, sūnaus Jotamo, sūnaus Uzijo, dienomis Sirijos karalius Recinas ir Izraelio karalius Pekachas, Remalijo sūnus, atžygiavo kovoti prieš Jeruzalę, bet neįstengė jos paimti. 
\par 2 Kai Dovydo namams pranešė: “Sirija susidėjo su Efraimu”, karaliaus ir jo tautos širdis taip drebėjo, kaip dreba nuo vėjo miško medžiai. 
\par 3 Viešpats tarė Izaijui: “Tu ir tavo sūnus Šear Jašubas eikite pasitikti Achazą prie aukštutinio tvenkinio vandentiekio keliu, kuris veda į vėlėjo lauką. 
\par 4 Tu jam sakyk: ‘Stebėk ir būk ramus; nebijok ir tavo širdis teneišsigąsta šitų dviejų smilkstančių nuodėgulių: nirštančio Sirijos karaliaus Recino ir Remalijo sūnaus. 
\par 5 Kadangi Sirija, Efraimas ir Remalijo sūnus surengė prieš tave sąmokslą, galvodami: 
\par 6 ‘Žygiuokime prieš Judą, išgąsdinkime jį, nugalėkime ir paskirkime karaliumi Tabeelio sūnų’, 
\par 7 Viešpats Dievas taip sako: ‘Taip nebus ir tai neįvyks! 
\par 8 Sirijos galva yra Damaskas, o Damasko galva yra Recinas. Dar šešiasdešimt penkeri metai, ir Efraimas bus sunaikintas­tautos nebebus. 
\par 9 Efraimo galva yra Samarija, o Samarijos galva­Remalijo sūnus. Jei jūs netikite, neišliksite!’ ” 
\par 10 Viešpats toliau kalbėjo Achazui: 
\par 11 “Prašyk Viešpatį, savo Dievą, ženklo iš požemio gilybių ar dangaus aukštybių”. 
\par 12 Achazas atsakė: “Aš neprašysiu ir negundysiu Viešpaties”. 
\par 13 Tuomet Izaijas sakė: “Klausykitės, Dovydo namai, ar jums neužtenka varginti žmonių, ar varginsite ir Dievą? 
\par 14 Pats Viešpats duos jums ženklą. Štai mergelė taps nėščia, pagimdys sūnų ir pavadins jį Emanueliu. 
\par 15 Jis maitinsis pienu ir medumi, kol išmoks atmesti, kas pikta, ir pasirinkti, kas gera. 
\par 16 Prieš berniukui išmokstant atmesti pikta ir pasirinkti gera, šalys, kurių karalių bijaisi, bus sunaikintos. 
\par 17 Viešpats leis tau, tavo tautai ir tavo tėvų namams patirti tokių dienų per Asirijos karalių, kokių nebuvo nuo Efraimo ir Judo atsiskyrimo laikų. 
\par 18 Tuomet Viešpats sušvilps musėms Egipto upės gale ir bitėms Asirijos krašte 
\par 19 ir jos atskris ir nusileis slėnių tarpekliuose, uolų plyšiuose, ant visų krūmokšnių ir ganyklose. 
\par 20 Tą dieną Viešpats iš anapus upės pasisamdytu skustuvu­Asirijos karaliumi­nuskus galvos ir kojų plaukus, taipogi ir barzdą pašalins. 
\par 21 Tuo pačiu metu atsitiks, jog žmogus laikys vieną karvę ir dvi avis 
\par 22 ir jos duos tiek pieno, kad jis valgys sviestą; ir kiekvienas, kuris bus išlikęs šalyje, valgys sviestą ir medų. 
\par 23 Tuo metu vieta, kur augo tūkstantis vynmedžių, verta tūkstančio šekelių sidabro, pavirs erškėtynu ir usnynu. 
\par 24 Žmogus ten eis su strėlėmis ir lanku, nes usnių ir erškėčių bus pilnas kraštas. 
\par 25 Visi kalnai, kurie dabar kapliu kasami, taps neįžengiami dėl usnių ir erškėčių, juose ganysis galvijai ir vaikštinės avys”.



\chapter{8}


\par 1 Viešpats įsakė man: “Imk didelę lentelę ir įrašyk aiškiomis raidėmis: ‘Imk grobį, skubėk plėšti’ ”. 
\par 2 Aš pasiėmiau ištikimus liudytojus, kunigą Ūriją ir Jeberechijo sūnų Zachariją. 
\par 3 Aš įėjau pas pranašę, ji pastojo ir pagimdė sūnų. O Viešpats man tarė: “Duok jam vardą: ‘Imk grobį, skubėk plėšti’. 
\par 4 Prieš vaikui išmokstant pašaukti tėvą ir motiną, Damasko turtai ir Samarijos grobis bus Asirijos karaliaus išvežti”. 
\par 5 Viešpats man toliau kalbėjo: 
\par 6 “Kadangi ši tauta paniekino ramiai tekantį Siloamo vandenį ir pamėgo Reciną ir Remalijo sūnų, 
\par 7 Viešpats atves prieš jus daugybės galingų upių vandenis­Asirijos karalių ir visą jo garbę. Jis išsilies iš savo vagos ir užtvindys visus laukus. 
\par 8 Jis tekės per Judą, patvins ir sieks iki kaklo; išties savo sparnus ir apdengs visą tavo šalį, Emanueli!” 
\par 9 Susirinkite, tautos, ir jūs būsite nugalėtos! Išgirskite, visos tolimos šalys, susijuoskite, būsite nugalėtos! Susijuoskite, būsite nugalėtos! 
\par 10 Susitarkite tarpusavyje, iš to nieko neišeis. Tarkite žodį, bet jis nebus įvykdytas, nes Dievas yra su mumis! 
\par 11 Viešpats kalbėjo man, stipria ranka laikė ir įspėjo mane neiti šitos tautos keliu, sakydamas: 
\par 12 “Nevadinkite sąmokslu to, ką šita tauta vadina sąmokslu. Nebijokite to, ko ji bijo, ir neišsigąskite. 
\par 13 Kareivijų Viešpatį laikykite šventu, Jo bijokite ir prieš Jį drebėkite. 
\par 14 Jis bus pašventinimas, suklupimo akmuo ir papiktinimo uola abiems Izraelio namams, spąstai bei kilpa Jeruzalės gyventojams. 
\par 15 Daugelis suklups, kris ir suduš; įsipainios ir bus pagauti”. 
\par 16 Saugok liudijimą, užantspauduok įstatymą tarp mano mokinių. 
\par 17 Aš lauksiu Viešpaties, kuris paslėpė savo veidą nuo Jokūbo namų, aš pasitikėsiu Juo. 
\par 18 Aš ir mano vaikai, kuriuos man davė Viešpats, esame kareivijų Viešpaties, kuris gyvena Siono kalne, ženklas ir įspėjimas Izraeliui. 
\par 19 Jie jums sako: “Klauskite mirusiųjų dvasių iššaukėjus ir būrėjus, kurie jums murma ir šnibžda”. Argi tauta neturėtų klausti savo Dievo? Argi reikia klausti mirusiųjų gyvųjų reikalais? 
\par 20 Kreipkitės į įstatymą ir liudijimą. Jeigu jie taip nesako, nėra juose šviesos. 
\par 21 Jie klaidžios suvargę ir išalkę. Būdami alkani, jie pyks ir keiks savo karalių ir Dievą, žiūrėdami aukštyn. 
\par 22 Kai jie pažiūrės į žemę, visur bus sielvartas ir tamsa, tamsybė ir priespauda. Jie pateks į nakties tamsą.



\chapter{9}


\par 1 Kas patyrė vargą, tiems nebus tamsu. Seniau Zabulono žemė ir Neftalio žemė buvo paniekintos, bet ateity paežerio kraštas ir kraštas anapus Jordano, pagonių Galilėja, bus išaukštintas. 
\par 2 Tauta, gyvenusi tamsybėje, išvydo didelę šviesą. Gyvenusiems mirties šešėlio krašte nušvito šviesa. 
\par 3 Tu padauginai tautą ir suteikei jiems džiaugsmo. Jie džiaugiasi Tavo akivaizdoje, kaip džiaugiamasi pjūties metu arba dalijantis grobį. 
\par 4 Tu sulaužei tautą slėgusį jungą, juos plakusią rykštę, jos prispaudėjo lazdą kaip Midjano dieną. 
\par 5 Visų karių apavas ir krauju sutepti rūbai bus ugnimi sunaikinti. 
\par 6 Kūdikis mums gimė, sūnus mums duotas. Ant jo peties viešpatavimas, jis bus vadinamas Nuostabusis, Patarėjas, Galingasis Dievas, Amžinasis Tėvas, Ramybės Kunigaikštis. 
\par 7 Jo viešpatavimas plėsis ir taikai nebus galo. Dovydo sostą ir jo karalystę jis sustiprins ir įtvirtins teisingumu ir teisybe per amžius. Kareivijų Viešpaties uolumas tai padarys. 
\par 8 Viešpats pasiuntė žodį Jokūbui, jis krito į Izraelį. 
\par 9 Tai žinos visa tauta, Efraimas ir Samarijos gyventojai, kurie puikybėje ir išdidume kalbėjo: 
\par 10 “Plytų mūrai sugriuvo, mes juos atstatysime iš tašytų akmenų; nukirto laukinius figmedžius, mes juos pakeisime kedrais”. 
\par 11 Todėl Viešpats sukels prieš juos Recino priešus ir sukurstys priešininkus. 
\par 12 Sirija rytuose, o filistinai vakaruose ryja Izraelį, plačiai atvėrę nasrus. Dėl viso to Jo rūstybė dar neatsileido, Jo ranka dar pakelta. 
\par 13 Bet tauta nesigręžia į Tą, kuris juos baudžia, jie neieško kareivijų Viešpaties. 
\par 14 Todėl Viešpats nukirs Izraelio galvą ir uodegą, palmę ir nendrę tą pačią dieną. 
\par 15 Vyresnieji ir kilmingieji­tai galva, o melą kalbantys pranašai­ uodega. 
\par 16 Šios tautos vadai yra suvedžiotojai, o jų vedamieji pražus. 
\par 17 Todėl Viešpats nesidžiaugs jų jaunimu ir nepagailės nei jų našlaičių, nei našlių. Jie visi yra veidmainiai ir nedorėliai, jie kalba kvailystes. Dėl viso to Jo rūstybė dar neatsileido, Jo ranka dar pakelta. 
\par 18 Nedorybė kaip ugnis praryja usnis ir erškėčius, padega miško tankumynus, o dūmai kyla kaip stulpai. 
\par 19 Kareivijų Viešpaties rūstybė padegė šalį. Tauta bus lyg kuras ugniai. Žmogus nesigailės savo brolio. 
\par 20 Jis griebs iš dešinės, bet alks, ris iš kairės, bet nepasisotins. Kiekvienas valgys savo kūną. 
\par 21 Manasas­Efraimą, Efraimas­ Manasą, abu drauge­Judą. Dėl viso to Jo rūstybė dar neatsileido, Jo ranka dar pakelta.



\chapter{10}


\par 1 Vargas tiems, kurie leidžia neteisingus įstatymus ir užrašo neteisėtus sprendimus, 
\par 2 kad pašalintų beturtį iš teismo ir paniekintų mano tautos vargšų teises, kad našlės taptų jų grobiu ir galėtų apiplėšti našlaičius. 
\par 3 Ką jūs darysite aplankymo dieną, kai nelaimės jus užklups? Pas ką bėgsite ieškoti pagalbos, kur paliksite savo turtus? 
\par 4 Be manęs jie pateks į nelaisvę, žus nuo kardo. Dėl viso to Jo rūstybė dar neatsileido, Jo ranka dar pakelta. 
\par 5 Vargas Asirijai, mano rūstybės rykštei ir lazdai. 
\par 6 Aš ją pasiųsiu prieš veidmainių tautą, prieš savo rūstybės žmones ją nukreipsiu, įsakysiu plėšti, grobti ir mindyti ją kaip gatvių purvą. 
\par 7 Bet ji ne taip galvoja: ji svajoja sunaikinti daug tautų. 
\par 8 Ji sako: “Ar mano valdovai nėra karaliai? 
\par 9 Ar Kalnė ne kaip Karkemišas? Ar Hamatas ne kaip Arpadas? Ar Samarija ne kaip Damaskas? 
\par 10 Ar mano ranka nepasiekė stabmeldžių karalysčių, kurių stabų buvo daugiau negu Jeruzalėje ir Samarijoje? 
\par 11 Argi negaliu padaryti Jeruzalei ir jos stabams, kaip padariau Samarijai ir jos stabams?” 
\par 12 Todėl, kai Viešpats užbaigs savo darbą Siono kalne ir Jeruzalėje, Jis nubaus pasipūtusį Asirijos karalių, jo išdidžiai pakeltas akis. 
\par 13 Nes jis sako: “Aš tai padariau savo galia, savo išmintimi tai pasiekiau. Aš panaikinau tautų ribas, išplėšiau jų turtus, kaip karžygys pažeminau jų gyventojus. 
\par 14 Mano ranka surado tautų turtus kaip lizdus, kaip surenka paliktus kiaušinius, taip surinkau visą žemę. Nė vienas nepajudino sparno, nepravėrė snapo ir nesučirškė”. 
\par 15 Ar kirvis giriasi prieš tą, kuris juo kerta? Ar pjūklas didžiuojasi prieš tą, kuris jį traukia? Ar lazda pakyla prieš tą, kas nėra medis? Ar rykštė plaka tą, kuris ją laiko? 
\par 16 Viešpats, kareivijų Dievas, užleis sunkias ligas ant jo riebiųjų ir jo garbę sunaikins lyg ugnimi. 
\par 17 Izraelio Šviesa taps ugnimi, jo Šventasis­liepsna. Per vieną dieną Jis sudegins ir sunaikins usnis ir erškėčius. 
\par 18 Jis sunaikins jo miško garbę ir sodo sielą bei kūną. Jie sunyks kaip sunkiai sergantis ligonis. 
\par 19 Jo miško medžių taip maža beliks, kad vaikas galės juos suskaičiuoti. 
\par 20 Tuomet Izraelio ir Jokūbo namų likutis nebesirems juos nugalėjusiais, bet tiesoje vilsis Viešpačiu, Izraelio Šventuoju. 
\par 21 Jokūbo likutis sugrįš prie galingojo Dievo. 
\par 22 Izraeli, jei tavo tauta ir būtų gausi kaip pajūrio smiltys, tik likutis sugrįš. Sunaikinimas numatytas ir pelnytai užplūs. 
\par 23 Viešpats, kareivijų Dievas, įvykdys numatytą sunaikinimą visoje žemėje. 
\par 24 Todėl Viešpats, kareivijų Dievas, taip sako: “Mano tauta, gyvenanti Sione, neišsigąsk asirų, kai jie plaks tave rykštėmis ir muš lazdomis kaip egiptiečiai. 
\par 25 Nes dar trumpa valandėlė, ir mano rūstybė bei pyktis pasibaigs jų sunaikinimu”. 
\par 26 Tada kareivijų Viešpats pakels prieš juos rimbą ir smogs jiems, kaip smogė Midjanui prie Orebo uolos. Jis išties savo lazdą virš jūros, kaip tai padarė Egipte. 
\par 27 Tą dieną jų našta bus pašalinta nuo tavo pečių ir jungas nuo tavo kaklo, ir patepimas sudaužys jungą. 
\par 28 Jie pakilo iš Rimono, atžygiavo į Ajatą pro Migroną ir paliko atsargas Michmaše. 
\par 29 Jie perėjo tarpeklį, Geboje nakvojo. Rama dreba, Sauliaus Gibėja išbėgiojo. 
\par 30 Šauk balsiai, Galimų dukra! Teišgirsta tave Laišoje, vargšas Anatote. 
\par 31 Madmena ištuštėjo, Gebimų gyventojai ieško išsigelbėjimo. 
\par 32 Dar šiandien jie pasieks Nobą, pakels grasinantį kumštį prieš Siono dukterį, prieš Jeruzalės aukštumą. 
\par 33 Štai Viešpats, kareivijų Dievas, nulauš su triukšmu šakas, aukštai iškilusius nukirs, aukštieji bus pažeminti, 
\par 34 miško tankumynus iškapos kirviu, Libanas kris nuo Galingojo.



\chapter{11}


\par 1 Iš Jesės kelmo išdygs atžala ir iš jo šaknies išaugs šaka. 
\par 2 Viešpaties dvasia bus ant jo: dvasia išminties ir supratimo, dvasia patarimo ir galybės, dvasia pažinimo ir Viešpaties baimės. 
\par 3 Bijoti Viešpaties jam bus džiaugsmas. Jis teis ne kaip akys mato ir pasmerks ne kaip ausys girdi. 
\par 4 Jis teisingai teis beturčius ir bešališkai krašto romiuosius. Jis ištiks žemę savo burnos lazda, nužudys nedorėlį savo pūstelėjimu. 
\par 5 Teisumas bus jo strėnų raištis, ištikimybe jis susijuos juosmenį. 
\par 6 Vilkas gyvens su avinėliu, leopardas gulės su ožiuku, veršis, jautis ir jaunas liūtas bus drauge, juos ganys mažas vaikas. 
\par 7 Karvė ir lokys ganysis, jų jaunikliai guls drauge ir liūtas ės šiaudus kaip jautis. 
\par 8 Motinos maitinamas kūdikis žais prie angies urvo, ir mažas vaikas kiš ranką į gyvatės olą. 
\par 9 Niekas nekenks ir nežudys mano šventame kalne. Kaip vanduo pripildo jūrą, taip žemė bus pilna Viešpaties pažinimo. 
\par 10 Tą dieną Jesės šaknis bus vėliava tautoms; pagonys ieškos jo, nes jo poilsis bus šlovingas. 
\par 11 Tuomet Viešpats vėl pakels ranką gelbėti savo tautos likutį, kuris bus likęs Asirijoje, Egipte, Patrose, Kuše, Elame, Senaare, Emate ir jūros salose. 
\par 12 Jis iškels vėliavą tautoms, surinks Izraelio pabėgėlius ir Judo išsklaidytuosius iš keturių žemės kraštų. 
\par 13 Tuomet Efraimo pavydas išnyks, ir Judo priešai bus sunaikinti, Efraimas nepavydės Judui, Judas nebekovos prieš Efraimą. 
\par 14 Jie skubės jūros link prieš filistinus ir kartu apiplėš rytiečius. Jie pasieks ir valdys edomitus ir moabitus, amonitai jiems paklus. 
\par 15 Viešpats išdžiovins Egipto jūros liežuvį. Jis taip pat, pakėlęs savo ranką, padalins upę į septynis upelius; tuomet bus galima pereiti per ją sausuma. 
\par 16 Ir bus platus kelias mano tautos likučiui, kuris išliko Asirijoje, kaip buvo Izraeliui išeinant iš Egipto.



\chapter{12}


\par 1 Tuomet tu sakysi: “Aš šlovinu Tave, Viešpatie. Nors buvai užsirūstinęs ant manęs, bet Tavo pyktis atslūgo, ir Tu paguodei mane. 
\par 2 Dievas yra mano išgelbėjimas. Aš Juo pasitikiu ir nebijau, nes mano stiprybė ir giesmė yra Viešpats. Jis yra mano gelbėtojas”. 
\par 3 Jūs su džiaugsmu semsite vandenį iš išgelbėjimo šaltinių, 
\par 4 sakydami: “Girkite Viešpatį, šaukitės Jo vardo, skelbkite tautose Jo darbus, kalbėkite, kad Jo vardas didingas. 
\par 5 Giedokite Viešpačiui, nes Jis padarė didingų darbų; tebūna tai žinoma visoje žemėje. 
\par 6 Džiūgaukite ir šūkaukite, Siono gyventojai, nes Izraelio Šventasis yra didis tarp jūsų”.



\chapter{13}


\par 1 Regėjimas apie Babiloną, kurį matė Izaijas, Amoco sūnus. 
\par 2 Iškelkite vėliavą kalno viršūnėje, šaukite garsiai, mokite ranka, kad jie įeitų pro kilmingųjų vartus. 
\par 3 Aš įsakiau savo pašvęstiesiems ir pašaukiau savo karžygius, kad įvykdytų mano rūstybę,­tuos, kurie džiaugiasi mano išaukštinimu. 
\par 4 Kalnuose minios triukšmas tarsi gausios tautos. Tai sujudimas karalysčių ir susirinkusių tautų. Kareivijų Viešpats ruošia kariuomenę kovai. 
\par 5 Jie ateina iš tolimos šalies, nuo padangių pakraščių; tai Viešpaties rūstybės įrankiai žemei sunaikinti. 
\par 6 Verkite, nes Viešpaties diena arti; ji ateina kaip sunaikinimas nuo Visagalio. 
\par 7 Dėl to visos rankos nusvirs, kiekvieno žmogaus širdis sutirps. 
\par 8 Jie išsigąs, juos apims skausmai ir kentėjimai. Juos suims skausmai kaip gimdyves. Jie apstulbę žiūrės vienas į kitą: jų veidai degs ugnimi. 
\par 9 Artėja Viešpaties diena, žiauri ir pilna rūstybės bei keršto, kad paverstų žemę tyrlaukiais ir sunaikintų jos nusidėjėlius. 
\par 10 Dangaus žvaigždės ir žvaigždynai nebešvies, saulė, tik patekėjus, aptems ir mėnulis nebešvies. 
\par 11 Aš bausiu pasaulį dėl jo piktybės, nedorėlius­dėl jų kalčių. Aš sustabdysiu puikuolių pasipūtimą ir pažeminsiu baisiųjų išdidumą. 
\par 12 Aš padarysiu, kad žmogus bus brangesnis už auksą, o vyras­už gryną Ofyro auksą. 
\par 13 Aš supurtysiu dangus, o žemė pajudės iš savo vietos dėl kareivijų Viešpaties rūstybės Jo degančio pykčio dieną. 
\par 14 Kaip išgąsdintos stirnos arba avys, kurių niekas nebegano, kiekvienas grįš prie savo tautos, į savo kraštą. 
\par 15 Ką suras, perdurs, ką sugaus, nužudys. 
\par 16 Jų kūdikius sutraiškys jų pačių akivaizdoje, išplėš namus ir moteris išprievartaus. 
\par 17 Aš sukurstysiu prieš Babiloną medus, kurie nevertina sidabro ir negeidžia aukso. 
\par 18 Jie strėlėmis žudys jaunuolius, nepagailės nėščių moterų nei jų kūdikių. 
\par 19 Babilonas, karalysčių šlovė, chaldėjų pasididžiavimas, taps kaip Sodoma ir Gomora, kai Viešpats jas sunaikino. 
\par 20 Jis niekad nebebus apgyvendintas ir niekas jame negyvens per kartų kartas. Arabas ten nebeties savo palapinės ir piemenys nebeganys savo bandos. 
\par 21 Laukiniai žvėrys ten šeimininkaus, jų namai bus pilni pelėdų. Stručiai ten bėgios ir šokinės satyrai. 
\par 22 Hienos kauks bokštuose ir šakalai puikiuose rūmuose. Babilono galas arti, jo dienos suskaitytos.



\chapter{14}


\par 1 Nes Viešpats pasigailės Jokūbo ir vėl išsirinks Izraelį, apgyvendins juos jų pačių krašte. Prie jų jungsis ateiviai ir glausis prie Jokūbo. 
\par 2 Pagonys paims juos ir parves į jų vietą. Izraelitai Viešpaties žemėje juos apgyvendins kaip tarnus ir tarnaites. Jie padarys belaisvius tuos, kurių belaisviai jie buvo, ir pavergs savo prispaudėjus. 
\par 3 Kai Viešpats po visų vargų, baimės ir kietos vergijos, kurioje vargote, jums suteiks ramybę, 
\par 4 tada dainuosite pasityčiojimo dainą apie Babilono karalių, sakydami: “Kaip nurimo prispaudėjas, auksinio miesto nebeliko! 
\par 5 Viešpats sulaužė nedorėlio lazdą, jo valdovo skeptrą, 
\par 6 kuris be paliovos įtūžęs plakė tautas, jas pavergė ir žiauriai persekiojo. 
\par 7 Visa žemė nurimo ir ilsisi, linksmai dainuoja. 
\par 8 Džiaugiasi kiparisai ir Libano kedrai, sakydami: ‘Kai tu kritai, kirtėjai nebeateina kirsti mūsų’. 
\par 9 Pragaras sujudo pasitikti tavęs ateinančio; dėl tavęs pažadino mirusiuosius, visus žemės valdovus, pakėlė nuo sostų visus tautų karalius. 
\par 10 Visi jie kalbės tau: ‘Tu nusilpai kaip ir mes, tapai mums lygus!’ 
\par 11 Tavo didybė nugarmėjo į pragarą su tavo styginių skambesiu. Kandys yra tavo paklodė ir kirminai užklojo tave. 
\par 12 Kaip tu iškritai iš dangaus, Liuciferi, ryto aušros sūnau? Kaip tu kritai žemėn, kuris buvai pamynęs tautas? 
\par 13 Tu sakei savo širdyje: ‘Aš pakilsiu į dangų, iškelsiu savo sostą aukščiau Dievo žvaigždžių, sėdėsiu dievų kalne tolimiausioje šiaurėje. 
\par 14 Aš pakilsiu aukščiau debesų, būsiu lygus Aukščiausiajam!’ 
\par 15 Bet tu esi nublokštas į pragarą, į giliausią bedugnę. 
\par 16 Kurie tave mato, įsižiūri ir galvoja: ‘Ar tai žmogus, prieš kurį drebėjo žemė ir karalystės? 
\par 17 Kuris pavertė pasaulį dykuma, sugriovė miestus ir nepaleido savo belaisvių?’ 
\par 18 Visų tautų karaliai garbingai guli savo kapuose. 
\par 19 Tu gi išmestas iš karsto kaip bjauri šaka. Esi apdengtas kardu nužudytųjų kūnais, guli duobėje tarp akmenų, kojomis mindomas. 
\par 20 Tu nebūsi palaidotas su kitais, nes sunaikinai savo kraštą, išžudei tautą. Niekada nebus minimi piktadario palikuonys. 
\par 21 Išžudykite jo vaikus už tėvo nusikaltimus, kad jie nebepakiltų, neapgyvendintų krašto ir nepripildytų žemės savo miestų”. 
\par 22 “Aš eisiu prieš juos,­sako kareivijų Viešpats,­sunaikinsiu Babilono vardą ir likutį­sūnų ir vaikaitį”,­sako Viešpats. 
\par 23 “Paversiu jį ežių buveine ir vandens liūnu; visa iššluosiu sunaikinimo šluota”,­sako kareivijų Viešpats. 
\par 24 Kareivijų Viešpats prisiekė: “Kaip Aš sumaniau, taip įvyks, kaip nusprendžiau, taip ir bus. 
\par 25 Aš sutraiškysiu asirus savo žemėje ir sumindžiosiu juos kalnuose. Jų jungas bus pašalintas ir jų našta nuimta nuo Izraelio pečių. 
\par 26 Tai Aš sumaniau padaryti visoje žemėje ir pakėliau ranką prieš visas tautas”. 
\par 27 Kareivijų Viešpats taip nusprendė, kas tai pakeis? Jis ranką pakėlė, kas ją sulaikys? 
\par 28 Karaliaus Ahazo mirties metais buvo paskelbta: 
\par 29 “Nedžiūgauk, filistinų žeme, kad rykštė, kuri tave plakė, yra sulaužyta. Iš gyvatės šaknų augs angis, o iš angies kiaušinių­drakonas. 
\par 30 Beturčių pirmagimiai bus maitinami, neturtingieji gulės saugiai. Tavo šaknį pražudysiu badu, jis išžudys tavo likutį. 
\par 31 Vartai, dejuokite! Mieste, šauk! Išsigąskite visi filistinai! Iš šiaurės atslenka dūmai, jų gretose nėra nė vieno atsiliekančio”. 
\par 32 Ką atsakyti tautos pasiuntiniams? Viešpats įkūrė Sioną, Jo tautos vargšai ras čia prieglaudą.



\chapter{15}


\par 1 Regėjimas apie Moabą. Tą naktį, kai Ar Moabas ir Kir Moabas buvo sunaikinti, Moabas žuvo. 
\par 2 Dibono duktė pakilo į aukštumas raudoti; moabitai rauda dėl Nebojo ir Medebos. Jų visų galvos plaukai nukirpti ir barzdos nuskustos. 
\par 3 Gatvėse jie vilki ašutinėmis, ant namų stogų ir aikštėse visi rauda ir dejuoja. 
\par 4 Hešbone ir Elealėje jie taip rauda, kad jų balsas girdėti Jahace; net Moabo kariai šaukia ir dreba. 
\par 5 Mano širdis liūdi dėl Moabo. Žmonės iš jo bėga į Coarą. Kiti eina taku verkdami į Luhitą. Kelyje į Horonaimą jie rauda, nes visa sunaikinta. 
\par 6 Nimrimų vandenys išdžiūvo, žolė nuvyto, nauja nebeželia, ir nėra jokios žalumos. 
\par 7 Visa, kas dar liko, jie nešasi per Karklų upelį. 
\par 8 Šauksmas girdėti visame Moabo krašte; jis girdimas Eglaimuose ir Berelime. 
\par 9 Dimono vandenys pilni kraujo. Bet Aš siųsiu Dimonui dar daugiau nelaimių: liūtų Moabo pabėgėliams ir krašte likusiems.



\chapter{16}


\par 1 Siųskite krašto kunigaikščiui avinėlį iš Selos į dykumą Siono dukros kalnui. 
\par 2 Kaip iš lizdo išmesti paukščiukai yra Moabo dukterys prie Arnono brastų. 
\par 3 Sušaukite pasitarimą, įvykdykite teisingumą. Tegu tavo šešėlis būna kaip naktis vidudienį. Išsklaidytus paslėpk, neišduok pabėgėlių. 
\par 4 Priimk, Moabai, mano išsklaidytuosius ir pabėgėlius, tegul jie gyvena pas tave. Būk jiems apsauga nuo sunaikintojo. Prispaudėjo nebėra, plėšimas pasibaigė, niokotojai pašalinti iš krašto. 
\par 5 Gailestingumu sostas bus įtvirtintas. Jis teisėtai sėdės jame, Dovydo palapinėje, teisdamas, ieškodamas teisingumo ir skubėdamas įgyvendinti teisybę. 
\par 6 Mes girdėjome apie Moabą, koks išdidus jis yra! Jo išdidumas, akiplėšiškumas ir įžūlumas tėra tuščias pasigyrimas. 
\par 7 Visi moabitai verks ir raudos. Jie verks ir liūdės dėl Kir Hareseto pamatų, kurie bus sujudinti. 
\par 8 Hešbono laukai išdžiūvo ir Sibmos vynmedžiai, kurių vyną gerdavo tautų viešpačiai, yra sunaikinti, vynuogynai, kurie tęsėsi iki Jazerio, siekė dykumą ir anapus jūros. 
\par 9 Aš verkiu drauge su Jazeru dėl Sibmos vynmedžių, laistau ašaromis Hešboną ir Elealę. Tavo vynuogyno derliumi džiaugiasi priešai. 
\par 10 Linksmybė ir džiaugsmas dingo iš derlingųjų laukų. Vynuogynuose nebėra nei džiaugsmo, nei linksmybės. Spaudyklose niekas nebespaudžia vyno. Aš nutildžiau dirbančiųjų šauksmus. 
\par 11 Mano siela dėl Moabo skamba lyg arfa, o širdis liūdi Kir Hareseto. 
\par 12 Jei Moabas pasirodys aukštumoje aukoti ir eis į savo šventyklą melstis, jis nieko nelaimės. 
\par 13 Tai yra Viešpaties žodis, kurį Jis kalbėjo seniau apie Moabą. 
\par 14 Bet dabar Viešpats sako: “Per trejus metus Moabo didybė sunyks, iš gausios tautos liks tik mažas ir silpnas likutis”.



\chapter{17}

\par 1 Pranašavimas apie Damaską. “Damaskas nebebus miestas, jis taps griuvėsių krūva. 
\par 2 Aroero miestai ištuštės, jie taps ganyklomis. Čia ilsėsis kaimenės ir niekas jų nebaidys. 
\par 3 Dings Efraimo tvirtovės ir Damasko karalystė. Sirija susilauks to paties, ko susilaukė Izraelio šlovė”,­sako kareivijųViešpats. 
\par 4 “Tą dieną Jokūbo šlovė sumenkės ir jo kūnas suliesės. 
\par 5 Bus taip, kaip pjovėjui pjūties metu nuimant derlių arba kaip renkančiam varpas Rafaimų slėnyje. 
\par 6 Iš jo liks tiek, kaip nurinkus alyvmedį: dvi ar trys alyvos medžio viršūnėje ir keturios ar penkios ant šakų galų”,­sako Viešpats, Izraelio Dievas. 
\par 7 Tą dieną žmogus žiūrės į savo Kūrėją, jo akys žvelgs į Izraelio Šventąjį. 
\par 8 Jis nebežiūrės į aukurus, kuriuos padarė jo ranka, ir nebevertins to, kas jo padirbta,­giraičių ir atvaizdų. 
\par 9 Tada kaip apleista šaka ar viršūnė bus jo sutvirtinti miestai, kuriuos jie paliks dėl izraelitų. Čia bus tyrlaukiai. 
\par 10 Kadangi tu užmiršai savo išgelbėjimo Dievą ir neatsiminei savo stiprybės uolos, tu sodinsi savo mėgiamus sodus ir svetimų vynmedžių daigus. 
\par 11 Tą dieną, kai sodinsi, jie sužaliuos ir kitą rytą žydės, tačiau derliaus jie neneš­tik vargą, nelaimes ir skausmus. 
\par 12 Vargas daugeliui tautų, kurios šėlsta lyg jūrų bangos ir ūžia kaip putojantys vandenys. 
\par 13 Viešpats sudraus jas, ir jos bėgs lyg dulkės, nešamos vėjo kalnuose, kaip audros sūkurio blaškomi lapai jos dings. 
\par 14 Vakare­siaubas! Rytą jų jau nebėra! Tokia dalia mūsų naikintojų, likimas tų, kurie mus plėšia!



\chapter{18}


\par 1 Vargas kraštui, esančiam anapus Etiopijos upių, kuriame girdimas sparnų ūžesys. 
\par 2 Jis siunčia pasiuntinius jūra, vandens keliais nendriniuose laiveliuose. Greitieji pasiuntiniai, skubėkite pas aukšto ūgio ir išdidžią tautą, pas tautą, kurios bijosi arti ir toli gyvenantys, kurios kraštas upėmis išraižytas. 
\par 3 Viso pasaulio gyventojai! Stebėkite, kai vėliava bus pakelta kalnuose! Klausykite, kai išgirsite trimito balsą! 
\par 4 Štai ką Viešpats man kalbėjo: “Aš ramiai viską seksiu iš savo vietos kaip karštis giedroje, kaip rūko debesis pjūties įkarštyje”. 
\par 5 Prieš pjūtį, žydėjimui praėjus, vynuogėms pradėjus nokti, Jis išpjaustys visas atžalas ir iškapos šakeles. 
\par 6 Jie bus palikti kalnų paukščiams ir žemės žvėrims. Paukščiai tame krašte praleis vasarą ir žvėrys gyvens žiemą. 
\par 7 Tuo metu aukšto ūgio, nenugalima ir išdidi tauta, kurios bijo arti ir toli gyvenantys, kurios kraštas upėmis išraižytas, atneš dovanų kareivijų Viešpačiui į Siono kalną.



\chapter{19}


\par 1 Pranašavimas apie Egiptą. Štai Viešpats ateina į Egiptą ant lengvo debesies. Egipto stabai dreba Jo akivaizdoje, ir egiptiečių širdys tirpsta juose. 
\par 2 “Aš sukurstysiu egiptiečius prieš egiptiečius, brolis kovos prieš brolį, draugas prieš draugą, miestas prieš miestą ir karalystė prieš karalystę. 
\par 3 Egiptiečiai gyvens sąmyšyje, ir Aš paversiu niekais jų planus. Jie klaus savo stabų ir burtininkų, mirusiųjų dvasių iššaukėjų ir žynių. 
\par 4 Aš atiduosiu egiptiečius į žiauraus valdovo rankas; nuožmus karalius juos valdys”,­sako Viešpats, kareivijų Dievas. 
\par 5 Jūros vanduo nuseks, ir upė išdžius. 
\par 6 Upės ir kanalai dvoks, nendrės ir meldai suvys. 
\par 7 Papirusai upės pakrantėse ir pasėliai prie upės išdžius, sunyks ir nieko nebeliks. 
\par 8 Žvejai liūdės ir dejuos visi, kurie meta meškeres į upę ir kurie vandenyje tiesia tinklus. 
\par 9 Linų apdirbėjai ir verpėjos bei audėjos neteks vilties. 
\par 10 Visų, kurie laiko tvenkinius žuvims, planai sužlugs. 
\par 11 Coano kunigaikščiai yra kvaili. Faraono išmintingieji patarėjai duos kvailus patarimus. Kaip galite sakyti faraonui: “Aš esu išminčiaus sūnus, karalių palikuonis?” 
\par 12 Kur tavo išminčiai? Jie tepraneša ir tepaskelbia, ką kareivijų Viešpats nusprendė dėl Egipto. 
\par 13 Coano kunigaikščiai taps kvaili, Nofo kunigaikščiai klys. Giminių vadai suklaidins Egiptą. 
\par 14 Viešpats pasiuntė jiems svaigulio dvasią, kuri klaidina egiptiečius visuose jų darbuose; jie kaip girtas žmogus, svyrinėjantis savo vėmaluose. 
\par 15 Jokio darbo Egipte negalės daryti nei galva, nei uodega, nei palmė, nei nendrė. 
\par 16 Tą dieną egiptiečiai bus lyg moteris: jie drebės ir išsigąs kareivijų Viešpaties rankos, kurią Jis pakels prieš Egiptą. 
\par 17 Judo žemė bus siaubas Egiptui. Kiekvienas išsigąs, vos jį paminėjęs, dėl to, ką Viešpats nusprendė dėl Egipto. 
\par 18 Tada Egipto šalyje bus penki miestai, kurie kalbės Kanaano kalba ir prisieks kareivijų Viešpačiui. Vienas bus vadinamas Saulės miestu. 
\par 19 Egipto žemės viduryje stovės aukuras Viešpačiui, o krašto pasienyje­paminklas Jam. 
\par 20 Tai bus ženklas ir liudijimas apie Viešpatį Egipto krašte, nes jie šauksis Viešpaties pavojui iškilus. Jis siųs jiems gelbėtoją, kuris kovos ir išlaisvins juos. 
\par 21 Viešpats apsireikš egiptiečiams, ir tą dieną egiptiečiai pažins Viešpatį. Jie aukos Jam deginamąsias ir duonos aukas, duos įžadus Viešpačiui ir laikysis jų. 
\par 22 Viešpats užgaus Egiptą ir išgydys. Jie gręšis į Viešpatį. Jis leisis permaldaujamas ir pagydys juos. 
\par 23 Tuomet bus vieškelis iš Egipto į Asiriją. Asirai keliaus į Egiptą ir egiptiečiai į Asiriją. Asirai ir egiptiečiai drauge tarnaus Viešpačiui. 
\par 24 Izraelis bus trečias su Egiptu ir Asirija­palaiminimas krašto viduryje. 
\par 25 Juos kareivijų Viešpats laimins, sakydamas: “Tebūna palaimintas Egiptas, mano tauta, ir Asirija, mano rankų darbas, ir Izraelis, mano paveldas”.



\chapter{20}


\par 1 Asirijos karalius Sargonas siuntė Tartaną, ir tas atėjo į Ašdodą, kariavo prieš jį ir paėmė. 
\par 2 Tuo metu Viešpats kalbėjo Amoco sūnui Izaijui: “Eik, nusivilk ašutinę ir nusiauk kurpes”. Jis taip padarė ir vaikščiojo nuogas ir basas. 
\par 3 Tada Viešpats tarė: “Kaip mano tarnas Izaijas vaikščiojo nuogas ir basas trejus metus (tai buvo ženklas Egiptui ir Etiopijai), 
\par 4 taip Asirijos karalius ves Egipto belaisvius ir Etiopijos tremtinius, jaunus ir senus, nuogus ir basus, neapdengtomis šlaunimis Egipto gėdai. 
\par 5 Jie išsigąs ir drebės dėl Etiopijos, kuri buvo jų viltis, ir dėl Egipto, kuriuo didžiavosi. 
\par 6 Tą dieną šito jūros pakraščio gyventojai sakys: ‘Štai kas atsitiko tiems, pas kuriuos bėgome ieškoti pagalbos, kad mus išlaisvintų nuo Asirijos karaliaus! Kaip dabar išsigelbėsime?’ ”



\chapter{21}


\par 1 Pranašavimas apie pajūrio dykumą. Viesulas kyla iš pietų, ateina iš dykumos, iš baimę keliančio krašto. 
\par 2 Aš turėjau bauginantį regėjimą: plėšikas plėšia, naikintojas naikina. Pakilk, Elamai! Medija, apgulk! Aš padarysiu galą visiems vaitojimams. 
\par 3 Man strėnas skauda; skausmai suėmė mane lyg gimdyvę, aš nebegaliu girdėti nė matyti. 
\par 4 Mano širdis dreba, baimė ima mane; mano malonumų naktis Jis pavertė siaubu. 
\par 5 Stalas padengtas, kilimai ištiesti, jie valgo ir geria. Kunigaikščiai, pakilkite, patepkite skydus! 
\par 6 Viešpats man įsakė: “Eik, pastatyk sargą, ką jis matys, tepraneša!” 
\par 7 Jis pamatė vežimą, traukiamą poros žirgų, raitelį ant asilo ir raitelį ant kupranugario, ir jis labai atidžiai stebėjo. 
\par 8 Jis šaukė kaip liūtas: “Viešpatie, aš stoviu sargyboje dieną ir budžiu naktimis. 
\par 9 Štai, artėja vežimas ir pora raitelių”. Jis atsakė ir tarė: “Krito, krito Babilonas, visi jo dievų atvaizdai sutrupinti guli ant žemės!” 
\par 10 O mano klojimo iškultieji grūdai! Ką girdėjau iš kareivijų Viešpaties, Izraelio Dievo, tą pranešiau jums. 
\par 11 Pranašavimas apie Dūmą. Jis šaukia man iš Seyro: “Sarge, kiek dar naktis tęsis? Sarge, kiek dar naktis tęsis?” 
\par 12 Sargas atsakė: “Rytas artėja ir naktis. Jei norite klausti, klauskite vėl; sugrįžkite ir klauskite!” 
\par 13 Pranašavimas apie Arabiją. Jūs nakvosite Arabijos miškuose, Dedano karavanai. 
\par 14 Trokštančiam duokite vandens, pabėgėlius pasitikite su duona, Temos krašto gyventojai. 
\par 15 Jie bėga nuo kardo, nuo iškelto kardo, nuo įtempto lanko, nuo baisaus karo. 
\par 16 Viešpats man sako: “Dar vieneri metai, ir visa Kedaro didybė pranyks. 
\par 17 Kedaro sūnų, drąsiųjų šaulių, išliks tik mažas likutis, nes Viešpats, Izraelio Dievas, taip kalbėjo”.



\chapter{22}


\par 1 Pranašavimas apie Regėjimo slėnį. Kas gi atsitiko, kad visi sulipote ant stogų? 
\par 2 Tu, pilnas triukšmo ir neramumo, linksmasis mieste! Tavo kritusieji nekrito nuo kardo ir nemirė kovoje. 
\par 3 Visi tavo valdovai bėgo, bet buvo surišti šaulių. Visi, kurie surasti tavyje, buvo surišti, nors ir toli bėgo. 
\par 4 Aš sakiau: “Atsitraukite nuo manęs, aš graudžiai verksiu. Neguoskite manęs, nes mano tauta sunaikinta”. 
\par 5 Tai buvo paniekos ir sąmyšio diena Regėjimo slėnyje, kurią mums siuntė kareivijų Viešpats. 
\par 6 Elamas atėjo su strėlinėmis, kovos vežimais ir raiteliais, o Kyras­su skydais. 
\par 7 Tavo mėgiamieji slėniai buvo pilni kovos vežimų, raiteliai išsirikiavo prie vartų. 
\par 8 Judo apsauga sugriuvo. Tą dieną jūs apžiūrėjote ginklus Miško namuose. 
\par 9 Jūs matėte Dovydo miesto sienų plyšius, kurių buvo daug. Jūs sėmėte vandenį iš žemutinio tvenkinio. 
\par 10 Jūs apžiūrėjote Jeruzalės namus, juos griovėte ir jais tvirtinote miesto sieną. 
\par 11 Jūs padarėte saugyklą tarp abiejų sienų senojo tvenkinio vandeniui. Bet jūs pamiršote Tą, kuris visa tai darė, ir nekreipėte dėmesio į Jį. 
\par 12 Kareivijų Viešpats ragino tą dieną verkti ir raudoti, skustis galvos plaukus ir apsirengti ašutine. 
\par 13 Bet štai džiaugsmas ir linksmybė, veršių ir avių pjovimas, mėsos valgymas ir vyno gėrimas. “Valgykime ir gerkime, nes rytoj mirsime”. 
\par 14 Kareivijų Viešpats apsireiškė man, sakydamas: “Šitas nusikaltimas jums nebus atleistas iki mirties,­sako kareivijų Viešpats”. 
\par 15 Kareivijų Viešpats sakė: “Eik pas rūmų prievaizdą Šebną ir jam sakyk: 
\par 16 ‘Ką tu čia veiki? Kodėl išsikirtai sau kapą? Rūpestingai aukštoje vietoje išsikaldinai sau buveinę uoloje. 
\par 17 Štai kareivijų Viešpats tvirtai nutvers tave ir pašalins iš čia. 
\par 18 Jis sukte pasuks tave ir mes kaip kamuolį į didelę šalį; ten tu mirsi ir ten tavo garbės vežimas taps gėda tavo valdovo namams! 
\par 19 Aš pašalinsiu tave iš tavo vietos ir tarnybos. 
\par 20 Tą dieną aš pašauksiu savo tarną Eljakimą, Hilkijo sūnų, 
\par 21 apvilksiu jį tavo rūbu, apjuosiu tavo juosta ir perduosiu jam tavo valdžią. Jis bus kaip tėvas Jeruzalės gyventojams ir Judo namams. 
\par 22 Aš uždėsiu jam ant peties Dovydo namų raktą. Jis atidarys, ir niekas nebeuždarys, jis užrakins, ir niekas nebeatrakins. 
\par 23 Aš įkalsiu jį kaip vinį tikroje vietoje, ir jis taps savo tėvo namuose šlovingu sostu. 
\par 24 Ant jo kabos visa jo tėvo namų šlovė: visos atžalos ir ataugos, maži ir dideli indai nuo puodukų iki puodų. 
\par 25 Ateis diena, kai tikroje vietoje įkalta vinis nulūš, visa, kas ant jos kabojo, nukris ir suduš, nes taip Viešpats pasakė’ ”.



\chapter{23}


\par 1 Pranašavimas apie Tyrą. Raudokite Taršišo laivai, nes sunaikintas jūsų miestas, į kurį galėtumėte grįžti. Tą žinią gavome iš Kitimų. 
\par 2 Nutilkite, gyventojai jūros pakraščių, kuriuos buvo pripildę Sidono pirkliai, plaukiojantys po jūrą. 
\par 3 Per plačius vandenis atgabendavo Sichoro grūdus, derlių nuo upės, ir jūs buvote tautų prekyvietė. 
\par 4 Susigėsk, Sidone, nes jūra kalba, jūros stiprybė sako: “Aš nebuvau nėščia ir negimdžiau, neauginau nei jaunikaičių, nei mergaičių”. 
\par 5 Kai žinia apie Tyrą pasieks Egiptą, jie išsigandę drebės. 
\par 6 Jūros pakraščių gyventojai, plaukite į Taršišą ir raudokite! 
\par 7 Ar tai ne linksmasis senų senovėje įkurtas miestas? Jo kojos nuneš jį į tolimą šalį būti ateiviu. 
\par 8 Kas tą padarė Tyrui? Jis juk buvo karališkas miestas, jo pirkliai buvo kunigaikščiai, visoje žemėje gerbiami. 
\par 9 Kareivijų Viešpats taip nusprendė, kad suvaldytų jo puikybę ir visą jo garbę, kad sugėdintų visus žemės garbinguosius. 
\par 10 Taršišo dukterie, vaikščiok savo žemėje lyg upė, užliejanti kraštą; niekas tavęs nebevaržo. 
\par 11 Viešpats ištiesė savo ranką virš jūros, Jis drebino karalystes. Viešpats davė įsakymą sunaikinti Kanaano tvirtoves. 
\par 12 Jis sakė: “Nebesidžiauk, prispausta mergaite, Sidono dukterie! Pakilk ir plauk į Kitimus! Ir ten nerasi ramybės”. 
\par 13 Štai chaldėjų šalis. Šitos tautos anksčiau nebuvo, asirai įkūrė ją dykumos gyventojams. Jie stato savo bokštus, griauna jo rūmus, paverčia jį griuvėsiais. 
\par 14 Vaitokite, Taršišo laivai, nes jūsų tvirtovė sunaikinta. 
\par 15 Tyras bus užmirštas septyniasdešimt metų. Toks yra žmogaus amžius. Po septyniasdešimties metų atsitiks Tyrui kaip dainoje apie paleistuvę: 
\par 16 “Pasiimk arfą, vaikščiok po miestą, užmiršta paleistuve, dainuok meiliai ir gražiai, kad tave atsimintų”. 
\par 17 Po septyniasdešimties metų Viešpats aplankys Tyrą. Miestas vėl grįš prie savo pelno, ištvirkaus su visomis pasaulio karalystėmis visoje žemėje. 
\par 18 Jo prekyba ir pelnas bus pašvęsta Viešpačiui. Jie nekraus atsargų sandėliuose, bet jas atiduos tarnaujantiems Viešpačiui, kad jie valgytų ir apsirengtų.



\chapter{24}


\par 1 Štai Viešpats daro žemę tuščią lyg dykumą, pakeičia jos paviršių, išsklaido jos gyventojus. 
\par 2 Taip bus ir tautai, ir kunigui; ir vergui, ir šeimininkui; ir tarnaitei, ir jos šeimininkei; ir perkančiam, ir parduodančiam; ir skolininkui, ir skolintojui. 
\par 3 Kraštas bus nušluotas ir nualintas, nes taip Viešpats patvarkė. 
\par 4 Žemė liūdi ir nyksta, pasaulis visiškai sumenkėjo ir nyksta; žemės išdidieji nusilpsta. 
\par 5 Žemė suteršta jos gyventojų, nes jie nesilaikė įstatymų, iškreipė teisingumą ir sulaužė amžinąją sandorą. 
\par 6 Todėl prakeikimas ryja žemę ir naikina jos gyventojus, todėl žemės gyventojai sudeginti ir žmonių mažai belikę. 
\par 7 Vynas gedi, vynuogės nyksta, visi, kieno buvo linksmos širdys, dūsauja. 
\par 8 Linksmybės su būgneliais liovėsi, linksmųjų triukšmas pasibaigė, švelnūs arfos garsai nutilo. 
\par 9 Nebeskamba vyną geriančiųjų dainos, stiprus gėrimas apkarto jo gėrėjams. 
\par 10 Tuštybės miestas bus sugriautas, namai užrakinti, niekas durų nebevarstys. 
\par 11 Verkia dėl vyno gatvėse; išnyko įvairios linksmybės, džiaugsmas žemėje dingo. 
\par 12 Miestas ištuštėjo, vartai guli griuvėsiuose. 
\par 13 Visas kraštas bus kaip alyvmedis po vaisių nurinkimo, kaip vynuogynas po vynuogių nurinkimo. 
\par 14 Jie pakels balsus, giedos apie Viešpaties didybę vakaruose, 
\par 15 šlovins Viešpatį rytuose ir jūros pakraščiuose Izraelio Dievo vardą. 
\par 16 Nuo žemės pakraščių girdime giedant: “Šlovė Teisiajam”. Bet aš sakau: “Vargas man, aš nebepakelsiu! Išdavikai be sąžinės­apgaudinėja, suvedžioja”. 
\par 17 Siaubas, duobė ir žabangai tau, žemės gyventojau! 
\par 18 Kas bėgs nuo siaubo garso, įkris į duobę; išlipęs iš duobės, pateks į žabangus; debesys atsivėrę, žemės pamatai dreba. 
\par 19 Žemė draskyte draskoma, trinte trinama, kratyte kratoma. 
\par 20 Žemė svyruote svyruos kaip girtas, siūbuos kaip palapinė. Ją slėgs jos nusikaltimas, ji grius ir nebeatsikels. 
\par 21 Tą dieną Viešpats baus dangaus kareiviją aukštybėse ir žemės karalius žemėje. 
\par 22 Jie bus surinkti kaip belaisviai į duobę ir užrakinti; po daugelio dienų jie bus aplankyti. 
\par 23 Mėnulis paraus ir saulė susigės, nes kareivijų Viešpats šlovingai viešpataus Siono kalne, Jeruzalėje ir savo vyresniųjų akivaizdoje.



\chapter{25}


\par 1 Viešpatie, Tu esi mano Dievas, aš aukštinsiu Tave ir girsiu Tavo vardą, nes Tu padarei nuostabių dalykų, ištikimai įvykdei, ką seniai buvai pažadėjęs. 
\par 2 Tu pavertei miestą akmenų krūva, tvirtoves­griuvėsiais; svetimųjų rūmų nebėra ir jie niekad nebebus atstatyti. 
\par 3 Stipri tauta šlovins Tave; galingų tautų miestai bijos Tavęs. 
\par 4 Tu buvai stiprybė beturčiui, apsauga vargšui nelaimėje, priebėga nuo audros, ūksmė karštyje; baisiųjų rūstybė atsimušė kaip audra į sieną. 
\par 5 Kaip karštis sausoje vietoje, taip Tu sudrausi svetimųjų siautimą; kaip kaitra, nuslopinama debesų šešėliu, taip baisiųjų pergalės bus nutildytos. 
\par 6 Kareivijų Viešpats šitame kalne suruoš pokylį visoms tautoms; pokylį su geriausiu maistu ir senu bei nusistovėjusiu vynu. 
\par 7 Jis pašalins šitame kalne šydą, kuris gaubia visas tautas ir gimines. 
\par 8 Jis sunaikins mirtį amžiams. Viešpats Dievas visiems nušluostys ašaras ir pašalins panieką nuo savo tautos visoje žemėje, nes taip pasakė Viešpats. 
\par 9 Tuomet jie sakys: “Štai Jis, mūsų Dievas; mes laukėme Jo, ir Jis išgelbės mus. Jis yra Viešpats, kurio laukėme; džiaukimės ir džiūgaukime dėl Jo išgelbėjimo!” 
\par 10 Viešpaties ranka ilsėsis šitame kalne, o Moabas bus sumintas, kaip suminami šiaudai duobėje. 
\par 11 Jis išties savo rankas, kaip plaukikas ištiesia plaukti, ir Jis pažemins jų išdidumą kartu su jų rankų grobiu. 
\par 12 Aukštas tvirtovių sienas Jis pažemins, sugriaus ir pavers dulkėmis.



\chapter{26}


\par 1 Tuomet Judo šalyje giedos šią giesmę: “Mūsų miestas yra tvirtas; Jis suteiks išgelbėjimą už jo sienų ir pylimų. 
\par 2 Atkelkite vartus ir teįeina teisioji tauta, kuri saugo tiesą. 
\par 3 Tu suteiksi tobulą ramybę tiems, kurie pasitiki Tavimi. 
\par 4 Pasitikėkite Viešpačiu visados, nes Viešpats Jahvė yra amžina stiprybė. 
\par 5 Jis pažemina išpuikusius išdidaus miesto gyventojus. Jis pažemina juos iki žemės, dulkėmis paversdamas miestą. 
\par 6 Mindžioja jį beturčių kojos, vargšų žingsniai”. 
\par 7 Teisiojo kelias yra tiesus, Tu jo taką išlygini. 
\par 8 Tavo teismų kelyje, Viešpatie, mes laukėme; mūsų sielos atsimena Tavo vardą ir ilgisi Tavęs. 
\par 9 Mano siela naktį ilgisi Tavęs, mano dvasia ieško Tavęs. Kai Tavo teismai pasireiškia žemėje, pasaulio gyventojai pasimoko teisumo. 
\par 10 Jei nedorėlio bus pasigailėta, jis nepasimokys teisumo; teisiųjų šalyje jis darys pikta ir nekreips dėmesio į Viešpaties didybę. 
\par 11 Viešpatie, Tavo ranka yra pakelta, bet jie nemato jos. Tavo uolumą dėl tautos jie tepamato ir tesusigėsta. Ugnis tesunaikina Tavo priešus. 
\par 12 Viešpatie, Tu suteiksi mums ramybės, nes Tu juk viską padarei dėl mūsų. 
\par 13 Viešpatie, mūsų Dieve, kiti valdovai viešpatavo mums, bet mes tik Tavo vardą pripažįstame ir Jį garbiname. 
\par 14 Jie mirė ir nebeatgis, jie yra šešėliai ir nebeatsikels; Tu aplankei juos ir sunaikinai, jų atminimą išdildei. 
\par 15 Viešpatie, tu padidinai tautą; šlovė Tau už tai; Tu išplėtei krašto sienas. 
\par 16 Viešpatie, jie ieškojo Tavęs nelaimės metu; jie meldėsi, kai juos baudei. 
\par 17 Kai gimdyvės laikas priartėja, ji šaukia iš skausmo. Viešpatie, tokie mes esame Tavo akivaizdoje. 
\par 18 Mes lyg gimdančios kentėjome, bet pagimdėme tik vėją. Mes neatnešėme kraštui išlaisvinimo ir pasaulio gyventojai nekrito. 
\par 19 Tavo mirusieji bus gyvi, jų kūnai kelsis kartu su mano. Dulkėse esantieji, pabuskite ir giedokite. Kaip rasa gaivina augalus, taip Viešpats prikels mirusiuosius. 
\par 20 Mano tauta, eik, įeik į savo kambarius; pasislėpk valandėlę, kol praeis Jo rūstybė. 
\par 21 Viešpats ateina iš savo buveinės bausti žemės gyventojų už jų nusikaltimus. Žemė atidengs ant jos pralietą kraują ir nebeslėps nužudytųjų.



\chapter{27}


\par 1 Tą dieną Viešpats nubaus kietu, dideliu ir stipriu kardu leviataną, šliaužiančią ir besiraitančią gyvatę, ir nukaus jūros slibiną. 
\par 2 Tą dieną giedokite apie vynuogyną: 
\par 3 “Aš, Viešpats, esu jo sargas. Aš nuolatos laistau jį; kad jam kas nepakenktų, saugau jį dieną ir naktį. 
\par 4 Nėra manyje rūstybės. Jei kas jame pasėtų erškėčių ir usnių, Aš įeičiau, kovočiau ir sudeginčiau juos. 
\par 5 Tegu jie laikosi mano stiprybės, kad galėtų susitaikyti su manimi; ir tada jie susitaikys su manimi”. 
\par 6 Ateityje Jokūbas vėl įsišaknys, o Izraelis žaliuos ir žydės; jie pripildys visą pasaulį savo vaisių. 
\par 7 Ar Jis baudė juos, kaip jų priešai buvo baudžiami? Ar jų krito tiek, kiek jų priešų? 
\par 8 Tu baudei juos nuosaikiai, kai atmetei. Jis pašalino juos savo stipriu pūstelėjimu rytų vėjo dieną. 
\par 9 Jokūbo kaltė bus atleista ir nuodėmė pašalinta, kai Jis aukurų akmenis sutrupins kaip kalkakmenius, giraičių bei atvaizdų nebeliks. 
\par 10 Sustiprintas miestas bus tuščias, be gyventojų, kaip dykuma. Galvijai ganysis jame, gulės ir apgrauš jo šakas. 
\par 11 Kai jo šakos nudžius, atėjusios moterys jas nulauš ir sudegins. Tai tauta, neturinti supratimo. Todėl jos Kūrėjas nepasigailės ir nesuteiks jai malonės. 
\par 12 Tomis dienomis Viešpats kuls nuo Eufrato upės iki Egipto upelio, o jūs, izraelitai, būsite surinkti kaip grūdai. 
\par 13 Tą dieną pasigirs didysis trimitas, paliktieji Asirijos krašte bei išsklaidytieji Egipto šalyje sugrįš ir pagarbins Viešpatį šventajame Jeruzalės kalne.



\chapter{28}


\par 1 Vargas puikybės vainikui, Efraimo girtuokliams, vystančiai šlovingo grožio gėlei derlingame vyno įveiktųjų slėnyje. 
\par 2 Štai Viešpaties stiprusis ir galingasis kaip smarki kruša, kaip laužantis viesulas, kaip baisus lietus, kaip plūstantis vanduo užtvindys žemę. 
\par 3 Jis sunaikins Efraimo girtuoklių puikybę, 
\par 4 šlovingo grožio vystančiai gėlei viduryje derlingo slėnio bus kaip ankstyvam figos vaisiui: kas jį pamato, nuskina ir suvalgo. 
\par 5 Tą dieną kareivijų Viešpats bus šlovės karūna ir gražus vainikas savo tautos likučiui; 
\par 6 teisingumo dvasia teisėjui ir stiprybė kariams nugalėti priešą. 
\par 7 Tačiau šie apsvaigo nuo vyno, svyruoja nuo girtuokliavimo. Kunigas ir pranašas, apsvaigę nuo girtuokliavimo, nežino, ką darą. Jie klysta regėjimuose, suklumpa sprendimuose. 
\par 8 Visi jų stalai apvemti, nėra švarios vietos. 
\par 9 Ką Jis pamokys ir kam paaiškins girdėtą pranešimą? Ką tik nujunkytiems kūdikiams? 
\par 10 Taisyklė po taisyklės, taisyklė po taisyklės, eilutė po eilutės, eilutė po eilutės, čia truputį ir ten truputį. 
\par 11 Viešpats kalbės mikčiojančiomis lūpomis ir svetima kalba šitai tautai, 
\par 12 kuriai sakė: “Tai poilsis, kur gali pailsėti pavargę, tai atgaiva”. Bet jie neklausė. 
\par 13 Jiems buvo Viešpaties žodis: “Taisyklė po taisyklės, taisyklė po taisyklės, eilutė po eilutės, eilutė po eilutės, čia truputį ir ten truputį”, kad jie eitų, svyrinėtų, sukluptų, įsipainiotų ir patektų į nelaisvę. 
\par 14 Išgirskite Viešpaties žodį, jūs pasityčiotojai, kurie viešpataujate mano tautai Jeruzalėje. 
\par 15 Jūs sakote: “Mes padarėme sandorą su mirtimi ir susitarimą su mirusiųjų buveine. Atūžiantis tvanas nelies mūsų, nes melas yra mūsų priebėga ir apgaulė mus dengia”. 
\par 16 Todėl taip sako Viešpats Dievas: “Štai Aš dedu Sione pamatui ištirtą akmenį, brangų pamato kertinį akmenį. Kas tiki, nesielgs skubotai. 
\par 17 Teisingumas bus mano virvė, teisumas­mano svambalas. Kruša sunaikins melo priebėgą, ir vanduo užlies slėptuvę. 
\par 18 Tada jūsų sandora su mirtimi bus panaikinta ir susitarimas su mirusiųjų buveine nustos galioti. Atūžiantis tvanas parblokš jus. 
\par 19 Kai jis praeis, nusineš jus. Jis užeis kas rytą ir kas dieną, ir kas naktį. Tai bus siaubinga žinia”. 
\par 20 Guolis per trumpas išsitiesti, ir antklodė per siaura įsivynioti. 
\par 21 Viešpats pakils kaip Peracimų kalne, kaip Gibeono slėnyje. Jis padarys darbą, savo bauginantį darbą, kaip yra nusprendęs. 
\par 22 Nebesityčiokite, kad jūsų pančiai nebūtų stipriau suveržti! Iš Viešpaties, kareivijų Dievo, aš girdėjau apie numatytą sunaikinimą visoje žemėje. 
\par 23 Išgirskite mano balsą, klausykite ir supraskite mano kalbą. 
\par 24 Argi artojas kas dieną aria ir akėja, ruošdamas dirvą sėjai? 
\par 25 Argi, sulyginęs žemės paviršių, jis nesėja krapų, kmynų, miežių ir rugių? 
\par 26 Dievas pamokė jį, kad išmanytų. 
\par 27 Juk krapų ir kmynų niekas nekulia velenais. Krapus iškulia lazda ir kmynus lazdele. 
\par 28 Javus duonai reikia sumalti, todėl ant jų nevažinėja velenais visą laiką ir nemindo gyvulių kanopomis. 
\par 29 Taip patvarkė kareivijų Viešpats; Jo patarimas yra nuostabus ir išmintis didinga.



\chapter{29}


\par 1 Vargas Arieliui, miestui, kuriame gyveno Dovydas. Kasmet švęskite šventes, pjaukite aukas. 
\par 2 Aš užleisiu priespaudą Arieliui. Jame bus verksmas ir vaitojimas; jis bus tikras Arielis. 
\par 3 Apgulsiu tave, apkasiu grioviais, apstatysiu apsiausties bokštais. 
\par 4 Tu būsi labai pažemintas ir iš dulkių prislopintu balsu kalbėsi. Tavo balsas bus girdimas kaip mirusiųjų dvasių iššaukėjo balsas, kaip šnabždesys iš po žemių. 
\par 5 Tavo priešų bus daugybė kaip dulkių ir tavo prispaudėjų gausu­kaip vėjo nešamų pelų. Tai įvyks visai nelauktai ir ūmai. 
\par 6 Kareivijų Viešpats aplankys tave griausmu, žemės drebėjimu, audros viesulu ir ryjančia ugnies liepsna. 
\par 7 Kaip sapnas, kaip nakties regėjimas bus gausybė tautų, kariaujančių prieš Arielį, jį apgulusių grioviais ir bokštais. 
\par 8 Alkanas sapnuoja, kad jis valgo, bet pabudęs tebėra alkanas; arba ištroškęs sapnuoja, kad geria, o pabudęs tebėra ištroškęs. Taip bus tautoms, kurios kariaus prieš Siono kalną. 
\par 9 Nusistebėkite, pasibaisėkite ir šaukite! Jie girti, bet ne nuo vyno; svyruoja, bet ne nuo stipraus gėrimo. 
\par 10 Viešpats siuntė jums kieto miego dvasią, užmerkė jūsų akis­pranašus, uždengė jūsų galvas­regėtojus. 
\par 11 Visi regėjimai bus jums kaip užantspauduota knyga. Jei kas paduotų ją mokančiam skaityti ir sakytų: “Paskaityk!”, tas atsakytų: “Negaliu, nes ji užantspauduota”. 
\par 12 Jei knygą paduotų nemokančiam skaityti ir jam sakytų: “Skaityk!”, jis atsakytų: “Aš nemoku skaityti”. 
\par 13 Viešpats tarė: “Kadangi ši tauta artinasi prie manęs savo burna ir pagerbia mane savo lūpomis, bet jų širdis yra toli nuo manęs ir jie mokosi bijoti manęs, klausydami žmonių priesakų, 
\par 14 tai Aš nustebinsiu šią tautą savo nuostabiu darbu. Jų išminčių išmintis pranyks, gudriųjų sumanumas pražus”. 
\par 15 Vargas tiems, kurie savo planus slepia nuo Viešpaties ir darbus daro tamsoje, galvodami: “Kas mus mato ir kas mus žino?” 
\par 16 Jūs iškreipiate dalykus! Argi puodžius gali būti laikomas lygiu moliui? Ar kūrinys sako apie savo kūrėją: “Jis nesukūrė manęs”? Ar daiktas kalba apie tą, kuris jį padarė: “Jis nieko nesupranta”? 
\par 17 Netrukus ir Libanas taps ariama dirva, o dirva­mišku. 
\par 18 Tą dieną kurtieji išgirs knygos žodžius ir aklųjų akys praregės. 
\par 19 Romieji dar labiau džiaugsis Viešpačiu ir beturčiai­Izraelio Šventuoju. 
\par 20 Prispaudėjai dings ir pasityčiotojai žus; bus sunaikinti, kurie elgiasi neteisingai, 
\par 21 kurie apšmeižia žmogų, kurie vartuose kaltintojui spendžia spąstus ir teisųjį laiko nieku. 
\par 22 Todėl Viešpats, kuris išgelbėjo Abraomą, taip sako Jokūbo namams: “Jokūbas nebebus pažemintas, jo veidas nebeišblykš. 
\par 23 Jis matys savo vaikus, mano rankų darbą, tarp savųjų; jie pripažins šventu mano vardą ir Jokūbo Šventąjį ir bijos Izraelio Dievo. 
\par 24 Kurie klydo dvasioje, susipras, kurie buvo nepatenkinti, priims pamokymą”.



\chapter{30}


\par 1 Viešpats sako: “Vargas maištaujantiems vaikams, kurie priimate patarimą, bet ne mano duotą, sudarote sąjungą, bet be mano dvasios. Taip kaupiate nuodėmes. 
\par 2 Einate į Egiptą, manęs nepasiklausę, ieškote faraono pagalbos ir pasitikite Egipto šešėliu. 
\par 3 Faraono pagalba taps jums gėda ir pasitikėjimas Egipto šešėliu­ negarbe. 
\par 4 Nors tavo kunigaikščiai yra Coane ir jų pasiuntiniai pasiekė Hanesą, 
\par 5 sulauksite gėdos ir pajuokos dėl tautos, kuri jūsų negali nei išgelbėti, nei suteikti pagalbos, nei naudos”. 
\par 6 Regėjimas apie žvėris pietų krašto, kuriame yra vargas ir priespauda; kur liūtai, gyvatės ir skrendanti angis gyvena. Žmonės gabena savo turtus ant asilų ir savo lobius ant kupranugarių pas tautą, kuri negali padėti. 
\par 7 Egipto pagalba yra bevertė ir betikslė. Todėl šaukiau: “Jų jėga ramiai sėdėti”. 
\par 8 Tad eik, užrašyk lentoje ir įrašyk knygoje, kad vėlesniam laikui išliktų įspėjimas. 
\par 9 Ši tauta yra maištinga, melagiai vaikai, kurie neklauso Viešpaties įstatymo. 
\par 10 Jie sako regėtojams: “Neregėkite!”, ir pranašams: “Nepranašaukite, kas tiesa; kalbėkite, kas mums patinka, pranašaukite mums apgaules. 
\par 11 Pasitraukite iš kelio, šalinkitės nuo mūsų tako, tepasitraukia mums iš akių Izraelio Šventasis”. 
\par 12 Štai ką sako Izraelio Šventasis: “Kadangi jūs niekinate žodį, viliatės priespauda ir remiatės skriauda, 
\par 13 jums šis nusikaltimas bus pavojingas, kaip aukštoje sienoje didėjantis plyšys: siena ūmai ir nelauktai sugrius; 
\par 14 ji subyrės, kaip puodžiaus indas, kuris taip sutrupinamas, kad nebelieka šukės ugniai paimti iš židinio ar pasisemti truputį vandens iš duobės”. 
\par 15 Taip sako Viešpats, Izraelio Šventasis: “Jei atsigręšite ir nusiraminsite, būsite išgelbėti. Ramume ir pasitikėjime yra jūsų stiprybė”. Bet jūs nenorite. 
\par 16 Jūs sakote: “Ne, mes bėgsime ant žirgų”. Todėl jūs bėgsite. “Mes josime ant eikliųjų”. Ir jūsų persekiotojai bus eiklūs. 
\par 17 Tūkstantis bėgsite, išsigandę vieno, o, gąsdinant penkiems, bėgsite, kol liksite kaip pušies stuobrys kalno viršūnėje, kaip vėliava kalvoje. 
\par 18 Viešpats nori jūsų pasigailėti ir suteikti jums malonę. Viešpats yra teisingumo Dievas; palaiminti, kurie Jo laukia. 
\par 19 Siono tauta, gyvenanti Jeruzalėje, tu daugiau nebeverksi! Jis tikrai pasigailės tavęs. Tavo šauksmą Jis išgirs ir į jį atsakys. 
\par 20 Viešpats maitino jus vargo duona ir girdė priespaudos vandeniu. Dabar tavo mokytojai nebesislėps, tavo akys visada juos matys. 
\par 21 Tavo ausys girdės žodžius, sakomus tau už nugaros: “Tas yra kelias, eikite juo”, jei būsite nukrypę į dešinę ar į kairę. 
\par 22 Tu pašalinsi sidabrinius atvaizdus ir auksinius stabus, kaip šiukšles juos išmesi, sakydamas: “Lauk iš čia!” 
\par 23 Jis duos lietaus tavo sėklai, kuria apsėsi dirvą. Javai gausiai užderės derlingoje žemėje. Tada tavo banda ganysis plačioje ganykloje. 
\par 24 Tavo jaučiai ir asilai, kuriais įdirbama žemė, ės sūdytą pašarą, šakėmis sukratytą. 
\par 25 Nuo kiekvieno aukštesnio kalno ir iškilusios kalvos tekės vandens upeliai žudymo dienoje, kai bokštai grius. 
\par 26 Mėnulis švies kaip saulė; saulė bus septynis kartus šviesesnė, kaip septynių dienų šviesa, kai Viešpats perriš savo tautos žaizdą ir pagydys kirčių padarytas žaizdas. 
\par 27 Viešpaties vardas ateina iš tolo. Jo rūstybė deganti ir sunki, Jo lūpos pilnos įtūžio, liežuvis­ryjanti ugnis. 
\par 28 Jo kvapas kaip patvinusios upės srovė, kuri siekia iki kaklo ir naikina tautas. Jis pažaboja tautas ir paklaidina jas. 
\par 29 Tada jūs giedosite kaip šventų iškilmių naktį ir nuoširdžiai džiaugsitės, lyg eidami į Viešpaties kalną, pas Izraelio Galingąjį, palydint fleitai. 
\par 30 Viešpats leis išgirsti savo šlovingą balsą ir pajusti nusileidžiančios rankos smūgį, užsidegusios rūstybės liepsnojančią ir ryjančią ugnį, smarkų lietų ir krušą. 
\par 31 Viešpaties balsas išgąsdins asirus, kai Jis jiems smogs lazda. 
\par 32 Kiekvienas lazdos smūgis, kuriuo Viešpats smogs jiems, bus palydimas būgneliais ir arfomis. Jis kovos su jais grasindamas. 
\par 33 Jau seniai yra paruoštas Tofetas; taip, karaliui jis paruoštas, gilus ir platus, jame bus gausu ugnies ir malkų; Viešpaties kvapas kaip sieros srovė uždegs jį.



\chapter{31}


\par 1 Vargas einantiems į Egiptą pagalbos, kurie pasitiki žirgais ir kovos vežimais. Kadangi jų tiek daug ir raiteliai tokie gausūs ir stiprūs, jie nesikreipia į Izraelio Šventąjį ir neieško Viešpaties. 
\par 2 Jis yra išmintingas, užleidžia nelaimę ir neatsiima žodžių. Jis pakils prieš nedorėlių namus ir nusikaltėlių padėjėjus. 
\par 3 Egiptiečiai yra žmonės, ne Dievas. Jų žirgai yra kūnas, ne dvasia. Kai Viešpats išties ranką, padėjėjas suklups, ir tas, kuriam teikiama pagalba, kris; taip jie abu drauge žus. 
\par 4 Taip Viešpats pasakė man: “Kaip liūtas ar jaunas liūtukas urzgia prie savo grobio ir neišsigąsta, kai prie jo artėja piemenų būrys šūkaudamas, ir jų nebijo, taip kareivijų Viešpats nusileis kovoti dėl Siono kalno ir jo aukštumos. 
\par 5 Kaip paukščiai skraido, taip kareivijų Viešpats gins Jeruzalę; gins, išlaisvins ir išgelbės”. 
\par 6 Izraelio vaikai, grįžkite prie To, nuo kurio buvote visai nutolę. 
\par 7 Tą dieną kiekvienas išmes sidabro ir aukso stabus, kuriuos savo rankomis pasidarė ir jais nusidėjo. 
\par 8 “Asirija kris ir žus ne nuo žmogaus kardo. Jie bėgs nuo kardo, ir jų jaunuoliai paklius į vergystę. 
\par 9 Jie pabėgs į savo tvirtovę iš baimės ir jų kunigaikščiai išsigąs vėliavos”,­sako Viešpats, kurio ugnis Sione ir krosnis Jeruzalėje.



\chapter{32}


\par 1 Štai karalius karaliaus teisume ir kunigaikščiai valdys teisingai. 
\par 2 Kiekvienas iš jų bus kaip prieglauda nuo vėjo ar apsauga audroje: kaip vandens upeliai dykumoje ir kaip didelės uolos šešėlis tyruose. 
\par 3 Reginčiųjų akys matys ir girdinčiųjų ausys atidžiai klausysis. 
\par 4 Lengvabūdžiai supras pažinimą ir mikčiojančiųjų kalba bus aiški. 
\par 5 Kvailio nebevadins kilniu, apgaviko­garbingu. 
\par 6 Kvailys kalba kvailystes, jo širdis siekia neteisybės. Jis veidmainiauja ir kalba neteisingai apie Viešpatį. Jis nepavalgydina alkano ir nepagirdo ištroškusio. 
\par 7 Apgaviko sumanymai yra pikti, jis galvoja pakenkti vargšui teisme savo melais, nors beturtis ir kalba teisybę. 
\par 8 Kilnus galvoja kilniai ir gina teisingumą. 
\par 9 Jūs, nerūpestingos moterys, klausykite mano balso; jūs, savimi pasitikinčios dukros, išgirskite mano kalbą. 
\par 10 Metams praėjus, jūs, savimi pasitikinčios, išsigąsite, nes niekas vynuogių neberinks ir sodų vaisių nebeskins. 
\par 11 Išsigąskite ir drebėkite. Nusirenkite ir apsisiauskite ašutinėmis. 
\par 12 Muškitės į krūtinę ir dejuokite, apgailėdamos gražius laukus ir derlingus vynuogynus. 
\par 13 Mano tautos derlingoje dirvoje ir gyvybės pilno miesto namų vietoje žels erškėčiai ir augs usnys. 
\par 14 Rūmai ištuštės, miesto triukšmas nutils, kalva ir stebėjimo bokštas virs lauku, kuriuo džiaugsis laukiniai asilai ir ganysis bandos. 
\par 15 Taip pasiliks, iki dvasia iš aukšto bus išlieta. Tada dykumos taps derlingais laukais ir miškais; 
\par 16 teisingumas gyvens dykumoje ir teisumas pasiliks derlinguose laukuose. 
\par 17 Teisumo darbas bus taika, jo pasekmė­ramybė ir pasitikėjimas. 
\par 18 Mano tauta gyvens ramioje vietoje ir saugiuose namuose, 
\par 19 kai kruša kris ant miško ir miestas nusileis į žemumą. 
\par 20 Palaiminti jūs sėsite prie vandenų, laisvai ten ganysis asilai ir jaučiai.



\chapter{33}


\par 1 Vargas tau, naikintojau, kurio niekas nenaikina. Tu apgaulingai elgiesi, o tavęs niekas neapgaudinėja. Kai baigsi naikinti, pats būsi sunaikintas, ir kai baigsi apgaudinėti, tave apgaus. 
\par 2 Viešpatie, būk mums maloningas, mes laukiame Tavęs. Būk mūsų ranka kas rytą, išgelbėk nelaimės metu. 
\par 3 Tautos bėga nuo triukšmo; kai Tu pakyli­išsisklaido giminės. 
\par 4 Jų grobis sunyks lyg skėrių sunaikintas derlius. 
\par 5 Viešpats yra išaukštintas, nes Jis gyvena aukštybėje. Jis pripildė Sioną teisingumu ir teisumu. 
\par 6 Jis duos laikus, saugius išgelbėjimo jėga, išmintimi, pažinimu. Viešpaties baimė bus jų turtas. 
\par 7 Štai karžygiai rauda lauke, taikos pasiuntiniai graudžiai verkia. 
\par 8 Keliai ištuštėjo, keliautojai išnyko. Jis sulaužė sandorą, miestus sunaikino, žmonėmis nesirūpina. 
\par 9 Kraštas liūdi ir nyksta: Libanas apleistas ir sunykęs, Saronas pavirto dykuma, Bašano ir Karmelio lapai nukrito. 
\par 10 Viešpats sako: “Dabar Aš kelsiuos ir būsiu išaukštintas. 
\par 11 Jūs pastojote šienu ir pagimdysite šiaudus. Jūsų kvapas sunaikins jus kaip ugnis. 
\par 12 Tautos bus kaip išdegtos kalkės, kaip nupjauti erškėčiai bus sudegintos ugnimi. 
\par 13 Išgirskite toli esantys, ką Aš padariau, arti esantys supraskite mano galią”. 
\par 14 Siono nusidėjėliai išsigandę, veidmainiai dreba: “Kas gali gyventi prie viską ryjančios ugnies? Kas gali gyventi prie amžino karščio?” 
\par 15 Kas vaikšto teisiai ir kalba tiesą, kas paniekina priespaudos pelną, kas neima kyšių, kas užsikemša ausis ir nesiklauso kraują praliejančių, kas užmerkia akis ir nesigėri piktybėmis, 
\par 16 tas gyvens aukštumose; jo apsaugos pilis bus aukštose uolose, jis turės duonos ir jo vanduo neišseks. 
\par 17 Tavo akys matys karalių jo grožybėje, jos matys tolimą šalį. 
\par 18 Tu prisiminsi siaubo laikus: “Kur dingo mokesčių skaičiuotojas ir svėrėjas? Kur yra skaičiavęs bokštus?” 
\par 19 Tu nebematysi žiaurios tautos, tautos, kurios kalbos nesupranti, mikčiojančio liežuvio, kuris tau svetimas. 
\par 20 Žvelk į Sioną, mūsų iškilmių miestą. Tavo akys tesidžiaugia Jeruzale: tvirtais pastatais, nesugriaunama palapine, kurios stulpai nepašalinami ir virvės nesutraukomos. 
\par 21 Šlovingas Viešpats yra mūsų gyrius ir plačių srovių versmė. Čia nepasirodys irkluotojų valtis ir neplauks išdidus laivas. 
\par 22 Viešpats yra mūsų teisėjas, Viešpats­mūsų valdovas, Viešpats­ mūsų karalius; Jis išgelbės mus. 
\par 23 Tavo virvės atsileidusios; jos tvirtai nebelaiko stiebo savo vietoje nei išpūstų burių. Tada bus dalinamas didelis grobis, net raišieji gaus dalį. 
\par 24 Nė vienas iš gyventojų nesakys: “Aš sergu”. Žmonėms, kurie čia gyvens, bus atleistos kaltės.



\chapter{34}


\par 1 Priartėkite, tautos, klausykitės, pagonys, teišgirsta žemė ir visa, kas joje, pasaulis ir visa, kas iš jo kyla. 
\par 2 Viešpats rūstauja ant tautų, grasina jų kariuomenėms. Jis pasmerkė jas, paskyrė jas sunaikinti. 
\par 3 Užmuštieji bus išmesti, jų lavonai dvoks, kalnai mirks jų kraujyje. 
\par 4 Dangaus kareivijos sutirps, dangus bus suvyniotas kaip knyga. Lyg vynmedžio ir figmedžio suvytę lapai krinta jo pulkai. 
\par 5 Mano kardas bus nuplautas danguje, jis nusileis Idumėjos teismui, ant mano prakeiktos tautos. 
\par 6 Viešpaties kardas suteptas krauju, aptekęs taukais. Jis suteptas avinėlių ir ožių krauju, aptekęs avinų inkstų taukais. Viešpaties auka Bocroje, didelės skerdynės Idumėjoje. 
\par 7 Čia krinta stumbrai drauge su jaučiais ir buliais. Žemė yra permirkus krauju, dirva pilna taukų. 
\par 8 Tai Viešpaties keršto diena, atlyginimo metai už Siono skriaudą. 
\par 9 Upeliai pavirs derva, žemė­siera, kraštas degs kaip derva. 
\par 10 Dieną ir naktį ji degs, dūmai rūks amžinai. Kartų kartoms kraštas liks tyrlaukiu, niekas juo nekeliaus. 
\par 11 Ten gyvens vanagas ir ežys, įsikurs pelėda ir varnas. Viešpats išties sumaišties virvę ir ištuštėjimo svambalą. 
\par 12 Nebeliks kilmingųjų, kurie galėtų karaliauti, visi kunigaikščiai taps niekas. 
\par 13 Rūmuose augs erškėčiai, tvirtovėse­dilgėlės ir usnys. Čia gyvens šakalai ir stručiai. 
\par 14 Ten susitiks vilkai su hienomis, satyrai šauks vienas kitam. Nakties šmėklos ten susiras sau poilsio vietą. 
\par 15 Gyvatės turės ten lizdus ir saugos savo vaikus; ten rinksis ir būriuosis maitvanagiai. 
\par 16 Skaitykite ir tyrinėkite Viešpaties raštus­visi susirinks, kaip pasakyta. Viešpats taip pasakė, ir Jo dvasia juos surinks. 
\par 17 Jis metė burtą, matavimo virve padalino jiems kraštą. Jie paveldės jį amžiams ir per kartų kartas gyvens tame krašte.



\chapter{35}


\par 1 Tyrai ir dykumos džiūgaus, stepės pražys pievų gėlėmis; 
\par 2 ji žydės, bus linksma, džiaugsis ir giedos. Jai teks Libano šlovė, Karmelio ir Sarono grožis. Taip, jie matys Viešpaties šlovę ir Dievo didybę. 
\par 3 Sustiprinkite suglebusias rankas ir klumpančius kelius. 
\par 4 Sakykite išsigandusioms širdims: “Būkite drąsūs ir nebijokite. Štai jūsų Dievas atkeršys ir atlygins. Jis ateis ir išgelbės jus”. 
\par 5 Tada aklųjų akys atsimerks ir kurčiųjų ausys atsivers. 
\par 6 Tada raišas šokinės kaip briedis ir nebylys giedos. Vanduo trykš dykumoje ir upeliai tekės stepėse. 
\par 7 Sausoje žemėje ežerai, išdžiūvusioje­vandens šaltiniai. Kur šakalai gyveno, augs nendrės ir meldai. 
\par 8 Ten eis kelias, vadinamas Šventu keliu, kuriuo nevaikščios niekas nešvarus. 
\par 9 Čia nesimaišys liūtas, ir joks plėšrus žvėris nevaikštinės juo; išpirktieji keliaus juo. 
\par 10 Viešpaties išlaisvintieji sugrįš ir eis į Sioną giedodami; amžina linksmybė vainikuos jų galvas. Jie džiaugsis ir linksminsis, vaitojimo ir dūsavimo nebebus.



\chapter{36}


\par 1 Keturioliktais karaliaus Ezekijo metais Asirijos karalius Sanheribas puolė visus sustiprintus Judo miestus ir juos paėmė. 
\par 2 Asirijos karalius siuntė Rabšakę iš Lachišo į Jeruzalę pas karalių Ezekiją su didele kariuomene. Jis sustojo prie aukštutinio vandentiekio tvenkinio, vėlėjo lauke. 
\par 3 Pas jį išėjo Hilkijo sūnus Eljakimas, rūmų viršininkas, raštininkas Šebna ir metraštininkas Joachas, Asafo sūnus. 
\par 4 Rabšakė jiems tarė: “Taip sakykite Ezekijui: ‘Taip sako didysis karalius, Asirijos karalius: ‘Kuo remiasi tavo pasitikėjimas? 
\par 5 Tu kalbi tuščius žodžius, o karui reikalingas patarimas ir jėga. Kuo pasitiki, kad maištauji prieš mane? 
\par 6 Ar ketini atsiremti į Egiptą, šitą sulūžusią nendrę? Kas į ją atsiremia, tam ji įsminga į ranką ir perduria ją. Toks yra faraonas, Egipto karalius, visiems, kurie juo pasitiki. 
\par 7 Jei tu sakysi: ‘Mes pasitikime Viešpačiu, savo Dievu’, tai ar ne Jo aukštumas ir aukurus Ezekijas pašalino ir įsakė Judui ir Jeruzalei: ‘Jūs garbinkite prie šito aukuro’. 
\par 8 Taigi dabar lenktyniauk su mano valdovu, Asirijos karaliumi; aš tau duosiu du tūkstančius žirgų, jei tu surinksi tiek raitelių ant jų joti. 
\par 9 Ar gali pasipriešinti silpniausiam mano valdovo tarnų būriui, nors ir pasitiki Egipto kovos vežimais ir raiteliais? 
\par 10 Argi aš be Viešpaties atėjau į šį kraštą jo sunaikinti? Viešpats man įsakė: ‘Eik ir sunaikink tą kraštą’ ”. 
\par 11 Tada Eljakimas, Šebna ir Joachas sakė Rabšakei: “Kalbėk su savo tarnais aramėjiškai, mes suprantame; nekalbėk su mumis žydiškai, girdint žmonėms ant sienų”. 
\par 12 Rabšakė atsakė: “Ar mano valdovas siuntė mane tik pas tavo valdovą ir tave kalbėti šiuos žodžius? Ar ne pas tuos vyrus, kurie sėdi ant sienos, kad valgytų su jumis savo pačių išmatas bei gertų savo šlapimą?” 
\par 13 Tada Rabšakė atsistojo ir garsiai šaukė žydiškai: “Klausykite didžiojo karaliaus, Asirijos karaliaus žodžių! 
\par 14 Taip sako karalius: ‘Nesiduokite Ezekijo suvedžiojami, nes jis neišgelbės jūsų. 
\par 15 Teneįtikina Ezekijas jūsų pasitikėti Viešpačiu, sakydamas: ‘Viešpats tikrai išgelbės mus ir šito miesto neatiduos į Asirijos karaliaus rankas’. 
\par 16 Neklausykite Ezekijo, nes taip sako Asirijos karalius: ‘Padarykite su manimi sutartį ir išeikite pas mane, tuomet kiekvienas valgysite nuo savo vynmedžio ir nuo savo figmedžio bei gersite vandenį iš savo šulinio, 
\par 17 kol aš ateisiu ir išvesiu jus į kraštą, panašų į jūsų, pilną javų ir vyno, duonos ir vynuogių. 
\par 18 Nesiduokite Ezekijo suklaidinami, kai jis kalba: ‘Viešpats išgelbės mus’. Argi kuris nors iš tautų dievų išgelbėjo savo kraštą iš Asirijos karaliaus rankos? 
\par 19 Kur yra Hamato ir Arpado dievai? Kur Sefarvaimų dievai? Argi jie išgelbėjo Samariją iš mano rankos? 
\par 20 Kuris visų šitų kraštų dievas išgelbėjo savo kraštą iš mano rankos, kad Viešpats išgelbėtų Jeruzalę?’ ” 
\par 21 Jie visi tylėjo ir neatsakė nė žodžio, nes karalius buvo įsakęs: “Neatsakykite jam”. 
\par 22 Eljakimas, Hilkijo sūnus, rūmų viršininkas, raštininkas Šebna ir Joachas, Asafo sūnus, metraštininkas, atėjo pas Ezekiją su perplėštais rūbais ir perdavė jam Rabšakės žodžius.



\chapter{37}


\par 1 Tai išgirdęs, karalius Ezekijas perplėšė savo rūbus, apsirengė ašutine ir nuėjo į Viešpaties namus. 
\par 2 Jis nusiuntė rūmų viršininką Eljakimą, raštininką Šebną ir vyresniuosius kunigus, apsirengusius ašutinėmis, pas pranašą Izaiją, Amoco sūnų. 
\par 3 Jie kalbėjo jam: “Taip sako Ezekijas: ‘Šita diena yra bausmės, pažeminimo ir gėdos diena. Atėjo laikas gimdyti, bet nėra jėgų. 
\par 4 Gal Viešpats, tavo Dievas, išgirs žodžius Rabšakės, kurį Asirijos karalius siuntė niekinti gyvąjį Dievą, ir sudraus už žodžius, kuriuos Viešpats, tavo Dievas, girdėjo. Melskis už tuos, kurie yra likę’ ”. 
\par 5 Karaliaus Ezekijo tarnai atėjo pas Izaiją. 
\par 6 Ir Izaijas kalbėjo jiems: “Sakykite savo valdovui: ‘Taip sako Viešpats: ‘Neišsigąsk girdėtų žodžių, kuriais Asirijos karaliaus tarnai man piktžodžiavo. 
\par 7 Štai Aš pasiųsiu jam dvasią, ir jis, išgirdęs žinią, grįš į savo šalį ir ten bus nužudytas’ ”. 
\par 8 Rabšakė sugrįžęs rado Asirijos karalių kariaujantį prieš Libną, nes jis girdėjo, kad šis pasitraukė nuo Lachišo. 
\par 9 Jis išgirdo sakant apie Etiopijos karalių Tirhaką: “Jis ateina kovoti prieš tave”. Tai išgirdęs, jis siuntė pasiuntinius pas Ezekiją ir įsakė: 
\par 10 “Taip sakykite Ezekijui, Judo karaliui: ‘Teneapgauna tavęs tavo Dievas, kuriuo pasitiki, sakydamas: ‘Jeruzalė nebus atiduota į Asirijos karaliaus rankas’. 
\par 11 Tu girdėjai, ką Asirijos karaliai padarė visose šalyse, jas visai sunaikindami. Nejaugi tu būsi išgelbėtas? 
\par 12 Argi tų tautų, kurias mano tėvai sunaikino, dievai išgelbėjo Gozaną, Charaną, Recefą ir Edeno vaikus, gyvenusius Telasare? 
\par 13 Kur yra Hamato, Arpado, Sefarvaimų, Henos ir Ivos miestų karaliai?’ ” 
\par 14 Ezekijas paėmė laišką iš pasiuntinių ir perskaitė jį. Po to, nuėjęs į Viešpaties namus, išskleidė jį Viešpaties akivaizdoje. 
\par 15 Ezekijas meldėsi, sakydamas: 
\par 16 “Kareivijų Viešpatie, Izraelio Dieve, kuris gyveni tarp cherubų. Tu vienas esi visų žemės karalysčių Dievas, Tu sukūrei dangų ir žemę. 
\par 17 Palenk, Viešpatie, savo ausį ir išgirsk. Atverk, Viešpatie, savo akis ir pamatyk. Išgirsk visus Sanheribo žodžius, kuriais jis niekino gyvąjį Dievą. 
\par 18 Tai tiesa, Viešpatie, kad Asirijos karaliai išnaikino tautas ir jų šalis. 
\par 19 Jie sudegino jų dievus, nes jie nebuvo dievai, tik žmonių rankų darbas­medis ir akmuo,­todėl jie juos sunaikino. 
\par 20 Dabar, Viešpatie, mūsų Dieve, išgelbėk mus iš jo rankų, kad visos žemės karalystės žinotų, jog Tu vienas esi Viešpats”. 
\par 21 Tada Izaijas, Amoco sūnus, pasiuntė pas Ezekiją, sakydamas: “Taip sako Viešpats, Izraelio Dievas, apie Sanheribą, Asirijos karalių. 
\par 22 Štai Viešpaties žodis, kurį Jis kalbėjo apie jį: ‘Mergelė, Siono dukra, paniekino tave ir pasityčiojo iš tavęs. Jeruzalės dukra kraipo galvą dėl tavęs. 
\par 23 Ką tu niekinai ir prieš ką piktžodžiavai? Prieš ką išdidžiai pakėlei balsą ir savo akis? Prieš Izraelio Šventąjį! 
\par 24 Per savo tarnus tu niekinai Viešpatį ir sakei: ‘Su daugybe kovos vežimų aš pasikėliau į kalnų aukštumas, Libano aukščiausias vietas. Aš iškirsiu jo aukštuosius kedrus, gražiausius kiparisus. Aš pasieksiu pačią aukštumą, Karmelio mišką. 
\par 25 Aš kasiau šulinius ir gėriau vandenį; savo kojų padais išdžiovinau upes apsiausties vietose’. 
\par 26 Argi negirdėjai? Jau seniai Aš tai padariau, labai seniai tai paruošiau, tik dabar įvykdžiau, kad tu galėtum sustiprintus miestus paversti griuvėsių krūvomis. 
\par 27 Todėl jų gyventojai bejėgiai, jie nusigando ir susigėdo, jie tapo kaip lauko žolė, kaip gležna žolė ant stogų, kuri nudžiūna neužaugusi. 
\par 28 Aš žinau, kaip tu gyveni, kaip tu įeini ir išeini, kaip tu siautėji prieš mane. 
\par 29 Kadangi tavo siautėjimas prieš mane ir tavo pasipūtimas pasiekė mano ausis, Aš įversiu savo grandį į tavo šnerves ir tave pažabosiu, ir vesiu tave atgal keliu, kuriuo atėjai’. 
\par 30 Tau bus toks ženklas: šiais metais valgyk, ką randi, kitais metais maitinkis tuo, kas užaugs savaime; o trečiais metais sėkite ir pjaukite, sodinkite vynuogynus ir valgykite jų vaisus. 
\par 31 Judo namų likutis vėl leis šaknis apačioje ir neš vaisių viršuje. 
\par 32 Iš Jeruzalės išeis išlikusieji, iš Siono kalno išgelbėtieji. Tai darys kareivijų Viešpaties uolumas. 
\par 33 Todėl taip sako Viešpats apie Asirijos karalių: ‘Jis neįeis į šį miestą ir nepaleis į jį nė vienos strėlės, neateis prieš jį su skydais ir nesupils pylimo aplink jį. 
\par 34 Jis sugrįš tuo pačiu keliu, kuriuo atėjo, ir į miestą neįsiverš,­ sako Viešpats.­ 
\par 35 Aš apginsiu šitą miestą ir išgelbėsiu jį dėl savęs ir dėl mano tarno Dovydo’ ”. 
\par 36 Viešpaties angelas išėjo ir išžudė asirų stovykloje šimtą aštuoniasdešimt penkis tūkstančius. Kai jie atsikėlė anksti rytą, visur buvo pilna lavonų. 
\par 37 Asirijos karalius Sanheribas pasitraukė ir grįžo į Ninevę. 
\par 38 Kai jis garbino savo dievo Nisrocho namuose, jo sūnūs Adramelechas ir Sareceras nužudė jį kardu ir pabėgo į Armėnijos kraštą. Jo sūnus Asarhadonas karaliavo jo vietoje.



\chapter{38}


\par 1 Tomis dienomis Ezekijas mirtinai susirgo. Pas jį atėjo pranašas Izaijas, Amoco sūnus, ir jam tarė: “Taip sako Viešpats: ‘Sutvarkyk savo namus, nes tu nebepasveiksi, bet mirsi’ ”. 
\par 2 Ezekijas nusigręžė į sieną ir meldėsi: 
\par 3 “Viešpatie, meldžiu Tave, atsimink, kad aš teisingai ir tobula širdimi vaikščiojau prieš Tave ir dariau gera Tavo akyse”. Ir Ezekijas graudžiai verkė. 
\par 4 Viešpaties žodis buvo suteiktas Izaijui: 
\par 5 “Eik ir sakyk Ezekijui: ‘Taip sako Viešpats, tavo tėvo Dovydo Dievas: ‘Aš girdėjau tavo maldą ir mačiau tavo ašaras. Aš pridėsiu tau dar penkiolika metų 
\par 6 ir išgelbėsiu tave ir šitą miestą iš Asirijos karaliaus rankų. 
\par 7 Tau bus šis Viešpaties ženklas, kad Viešpats įvykdys, ką pažadėjo. 
\par 8 Aš pastumsiu Ahazo saulės laikrodžio šešėlį atgal dešimčia laipsnių’ ”. Saulė sugrįžo dešimčia laipsnių atgal, kuriuos buvo nuėjusi. 
\par 9 Ezekijas, Judo karalius, pagijęs iš savo ligos, parašė: 
\par 10 “Aš maniau, jog, įpusėjus mano amžiui, turėsiu eiti į mirusiųjų buveinę. Aš pasigendu savo likusių metų. 
\par 11 Aš tariau: ‘Nebematysiu Viešpaties gyvųjų žemėje, nebepamatysiu daugiau pasaulio gyventojų’. 
\par 12 Mano metai praėjo ir pašalinti nuo manęs kaip piemenų palapinė; mano gyvenimas kaip audėjo rietimas suriestas ir iš staklių išimtas. Dieną ir naktį Tu vedi mane į pabaigą. 
\par 13 Aš laukiau iki ryto. Jis kaip liūtas sutrupino visus mano kaulus pirmiau, negu diena pasibaigė ir atėjo naktis. 
\par 14 Aš čirškiu kaip kregždė, burkuoju kaip balandis. Mano akys nusilpo, bežiūrėdamos aukštyn. Viešpatie, aš prislėgtas, padėk man. 
\par 15 Ką aš galiu bekalbėti ir besakyti? Jis davė man pažadą ir jį įvykdė. Aš tyliai gyvensiu visus likusius savo metus savo sielos apkartime. 
\par 16 Viešpatie, tuo žmogus yra gyvas ir tai yra mano dvasios gyvybė. Tu išgydysi mane ir suteiksi man gyvenimą. 
\par 17 Štai apkartimas man pavirto ramybe. Tu išgelbėjai mano gyvybę nuo duobės ir sunaikinimo, visas mano nuodėmes numetei sau už nugaros. 
\par 18 Mirusiųjų pasaulis Tau nedėkoja ir mirtis nešlovina Tavęs. Kurie žengia į duobę, nebesitiki Tavo ištikimybės. 
\par 19 Gyvieji giria Tave, kaip ir aš šiandien. Tėvas pasakoja vaikams apie Tavo ištikimybę. 
\par 20 Viešpatie, gelbėk mane, tuomet aš visą savo gyvenimą Viešpaties šventykloje giedosiu ir skambinsiu”. 
\par 21 Izaijas liepė uždėti gabalėlį figos ant voties, kad karalius pasveiktų. 
\par 22 Ezekijas klausė: “Koks yra ženklas, kad įeisiu į Viešpaties namus?”



\chapter{39}


\par 1 Tuo metu Merodach Baladanas, Baladano sūnus, Babilono karalius, atsiuntė pasiuntinius su laišku ir dovanomis Ezekijui, nes jis girdėjo, kad tas sirgo ir pasveiko. 
\par 2 Ezekijas džiaugėsi ir parodė jiems visus savo namų turtus: sidabrą, auksą, kvepalus, brangius aliejus, ginklus ir visa, kas buvo sandėliuose. Nebuvo nieko jo namuose, ko Ezekijas nebūtų jiems parodęs. 
\par 3 Pranašas Izaijas atėjo pas karalių Ezekiją ir klausė: “Ką sakė tie vyrai? Iš kur jie atėjo?” Ezekijas atsakė: “Jie atėjo iš tolimo krašto, iš Babilono”. 
\par 4 Jis klausė: “Ką jie matė tavo namuose?” Ezekijas atsakė: “Jie matė viską, kas yra mano namuose; nėra nieko, ko nebūčiau jiems parodęs”. 
\par 5 Tada pranašas Izaijas tarė: “Klausykis kareivijų Viešpaties žodžio: 
\par 6 ‘Ateis dienos, kai visa, kas yra tavo namuose, ką tavo tėvai sukrovė iki šios dienos, bus išgabenta į Babiloną. Nieko nebus palikta,­sako Viešpats.­ 
\par 7 Net kai kuriuos iš tavo sūnų paims, išsives ir jie bus eunuchais Babilono karaliaus rūmuose’ ”. 
\par 8 Ezekijas atsakė Izaijui: “Viešpaties žodis, kurį tu kalbėjai, yra geras”. Ir jis pridūrė: “Kad tik mano dienomis būtų taika ir saugumas”.



\chapter{40}


\par 1 “Guoskite, guoskite mano tautą,­sako jūsų Dievas.­ 
\par 2 Kalbėkite paguodą Jeruzalei, praneškite jai, kad jos kovos pasibaigė, nusikaltimas atleistas. Ji gavo iš Viešpaties dvigubai už savo nuodėmes”. 
\par 3 Dykumoje šaukiančiojo balsas: “Paruoškite kelią Viešpačiui, tiesų darykite Jam vieškelį dykumoje. 
\par 4 Slėnius užpilkite, kalnus ir kalvas pažeminkite, kas kreiva, ištiesinkite, kas nelygu, išlyginkite. 
\par 5 Viešpaties šlovė bus apreikšta, visi kūnai tai matys, nes Viešpats taip kalbėjo”. 
\par 6 Balsas tarė: “Šauk!” Aš klausiau: “Ką šaukti?”­“Kiekvienas kūnas yra žolė, visas jo grožis yra lyg lauko gėlės žiedas. 
\par 7 Kai Viešpats pūsteli, žolė nuvysta ir žiedas nukrinta. Taip ir tauta yra žolė. 
\par 8 Žolė nuvysta, žiedas nukrinta, bet mūsų Dievo žodis išlieka per amžius”. 
\par 9 Pakilk į aukštą kalną, geros žinios nešėjau Sione! Pakelk galingai balsą, geros žinios nešėja Jeruzale! Pakelk balsą, nebijok! Sakyk Judo miestams: “Štai jūsų Dievas!” 
\par 10 Viešpats Dievas ateina su galia, Jo ranka valdo visa. Jo atpildas yra su Juo ir Jo darbas priešais Jį. 
\par 11 Jis ganys savo bandą kaip piemuo, surankios avinėlius, juos neš prie krūtinės, o avis su jaunikliais vedžios švelniai. 
\par 12 Kas išsėmė vandenis sauja ir išmatavo dangų sprindžiais? Kas žemės dulkes saiku seikėjo, pasvėrė kalnus ir kalvas svarstyklėmis? 
\par 13 Kas nukreipė Viešpaties Dvasią ir buvo Jo patarėjas? 
\par 14 Kas davė Jam patarimą, kas mokė Jį teisingumo ir pažinimo, kas parodė Jam supratimo kelią? 
\par 15 Tautos yra kaip lašas kibire, kaip grūdelis svarstyklėse. Jam salos lyg dulkės. 
\par 16 Libano kedrų neužtektų ugniai kūrenti, gyvulių nepakaktų Jo deginimo aukai. 
\par 17 Visos tautos Jo akivaizdoje bevertės, jos vertinamos mažiau už nieką ir tuštybę. 
\par 18 Su kuo tad palyginsite Dievą? Į ką panašų darysite Jo atvaizdą? 
\par 19 Amatininkas nulieja atvaizdą, auksakalys aptraukia jį auksu ir papuošia sidabrinėmis grandinėlėmis. 
\par 20 Kas neturtingas, pasirenka nepūvantį medį, susiranda išmanų amatininką, tas padaro drožinį ir pastato jį, kad nejudėtų. 
\par 21 Argi jūs nežinote? Argi negirdėjote? Argi nebuvo pranešta nuo pradžios? Argi nesuprantate iš pasaulio sutvėrimo? 
\par 22 Jis sėdi virš žemės skliauto, jos gyventojai atrodo lyg skėriai. Jis ištiesia dangus lyg užuolaidas, išskleidžia juos lyg palapinę gyventi. 
\par 23 Jis kunigaikščius paverčia nieku ir žemės teisėjus padaro kaip tuštybę. 
\par 24 Jie bus ką tik pasėti, ką tik jų kamienai bus išleidę šaknis, kai Jis pūstelės, ir jie nuvys, viesulas nuneš juos kaip šiaudus. 
\par 25 “Su kuo mane palyginsite ir į ką Aš panašus?”­klausia Šventasis. 
\par 26 Pakelkite akis ir pažiūrėkite aukštyn, kas visa tai sutvėrė? Jis suskaitęs veda jų pulkus ir kiekvieną vadina vardu. Jo galia ir jėga yra tokia didelė, kad nė vieno netrūksta. 
\par 27 Kodėl sakai, Jokūbai, kodėl taip kalbi, Izraeli: “Viešpačiui mano keliai nežinomi ir mano teisių Dievas nemato”. 
\par 28 Ar nežinai? Ar negirdėjai? Viešpats, amžinasis Dievas, kuris sutvėrė žemę, niekada nepailsta ir nepavargsta, Jo išmintis neišsemiama. 
\par 29 Jis duoda pavargusiam jėgų ir bejėgį atgaivina. 
\par 30 Net jaunuoliai pavargsta ir pailsta, jauni vyrai krinta išsekę. 
\par 31 Bet tie, kurie laukia Viešpaties, įgaus naujų jėgų. Jie pakils ant sparnų kaip ereliai, bėgs ir nepavargs, eis ir nepails.



\chapter{41}


\par 1 “Salos, nutilkite prieš mane. Tautos teįgauna naujų jėgų, tepriartėja ir tekalba; eikime kartu į teismą. 
\par 2 Kas pašaukė teisųjį iš rytų, liepė jam sekti paskui save? Jis atidavė tautas jam, pavergė karalius, atidavė juos kardui kaip dulkes, lankui­kaip vėjo nešiojamus šiaudus. 
\par 3 Jis vejasi juos, saugiai pereina keliu, kuriuo prieš tai nebuvo ėjęs. 
\par 4 Kas tai nuveikė ir padarė? Kas pašaukė kartas pradžioje? Aš, Viešpats, pirmasis ir paskutinysis. 
\par 5 Salos tai matė ir išsigando, drebėdami iš baimės atėjo žemės pakraščiai. 
\par 6 Jie vienas kitam padėjo ir vienas kitą drąsino. 
\par 7 Amatininkas padrąsina auksakalį, kuris su plaktuku dirba, giria ant priekalo kalantį: ‘Taip, gerai!’ Jie prikala vinimis stabą, kad nejudėtų. 
\par 8 Bet tu, Izraeli, mano tarne Jokūbai, kurį išsirinkau, mano draugo Abraomo palikuoni. 
\par 9 Aš pašaukiau tave iš žemės pakraščių, susivadinau iš tolimiausių kampų ir tau sakiau: ‘Tu esi mano tarnas, Aš tave išsirinkau ir neatmesiu tavęs. 
\par 10 Nebijok, nes Aš esu su tavimi; nepasiduok baimei, nes Aš esu tavo Dievas. Aš sustiprinsiu tave ir padėsiu tau, Aš palaikysiu tave savo teisumo dešine’. 
\par 11 Visi, kurie tau priešinasi, bus sugėdinti ir raudonuos; kurie kovoja prieš tave, sunyks ir žus. 
\par 12 Tu ieškosi ir nerasi tų, kurie sukyla prieš tave; tie, kurie kovoja prieš tave, bus kaip niekas, kaip tuščia vieta. 
\par 13 Nes Aš, Viešpats, tavo Dievas, laikysiu tavo dešinę ranką ir sakysiu: ‘Nebijok, Aš tau padėsiu!’ 
\par 14 Nebijok, tu kirmėle Jokūbai, jūs Izraelio žmonės! Aš padėsiu tau”,­sako Viešpats, tavo atpirkėjas, Izraelio Šventasis. 
\par 15 “Aš padarysiu tave lyg naują kuliamąjį veleną su aštriais krumpliais. Tu kulsi ir sutriuškinsi kalnus, kalvas sutrinsi kaip pelus. 
\par 16 Tu vėtysi juos, vėjas nuneš ir viesulas išblaškys juos, o tu džiaugsies Viešpačiu ir didžiuosies Izraelio Šventuoju. 
\par 17 Kai vargšai ir beturčiai ieškos vandens, bet jo neras ir jų liežuvis sudžius iš troškulio, Aš, Viešpats, Izraelio Dievas, išklausysiu juos, nepaliksiu jų. 
\par 18 Aš atidarysiu upelius ant kalvų ir šaltinius slėniuose; dykumą paversiu ežerais ir išdžiūvusią žemę­vandens versmėmis. 
\par 19 Dykumoje pasodinsiu kedrų ir akacijų, mirtų ir alyvmedžių; kartu augs pušys, kiparisai ir jovarai, 
\par 20 kad jie matytų ir suprastų, apsvarstytų ir žinotų, jog Viešpaties ranka tai padarė, Izraelio Šventasis tai sukūrė. 
\par 21 Ateikite, ginkite savo bylą, pateikite įrodymus,­sako Viešpats, Jokūbo karalius.­ 
\par 22 Priartėkite ir praneškite, kas įvyks. Pasakykite, kas seniau buvo, kad apsvarstytume ir suprastume, kas įvyks toliau. 
\par 23 Praneškite, kas ateityje įvyks, kad žinotume, jog jūs esate dievai. Darykite gera ar bloga, nustebinkite ar išgąsdinkite. 
\par 24 Jūs esate niekas ir jūsų darbas yra niekas. Pasibjaurėjimas, kas jus pasirenka. 
\par 25 Aš pažadinau jį iš šiaurės, ir jis ateis. Iš rytų ateis tas, kuris šauksis manęs. Jis valdovus sumins kaip kalkių skiedinį, kaip puodžius mina molį. 
\par 26 Kas paskelbė nuo pat pradžios, kad žinotume, ir iš senovės, kad sakytume: ‘Jis teisus’. Nė vienas nepranešė ir nepaskelbė, nė vienas neišgirdo jūsų žodžių. 
\par 27 Aš pirmasis pranešiau Sionui ir Jeruzalei, pasiunčiau tą, kuris skelbia gerą žinią. 
\par 28 Aš apsidairiau, bet nebuvo nė vieno, kuris patartų ar paklaustas atsakytų. 
\par 29 Jie yra tuštybė, jų darbai­niekas, jų lieti atvaizdai­vėjas ir niekai”.



\chapter{42}


\par 1 “Štai mano tarnas, kurį palaikau, mano išrinktasis, kurį pamėgau. Aš suteikiau jam savo dvasią, jis įgyvendins teisingumą pagonyse. 
\par 2 Jis nešūkaus ir nepakels balso; jo balso nesigirdės gatvėse. 
\par 3 Jis nenulauš įlūžusios nendrės ir neužgesins gruzdančio dagčio; Jis įvykdys teisingumą iki galo. 
\par 4 Jis nepails ir nenusivils, kol įgyvendins teisingumą žemėje, ir visi kraštai lauks jo įstatymo”. 
\par 5 Taip sako Viešpats Dievas, kuris sutvėrė dangus ir juos ištiesė, sutvirtino žemę ir kas ant jos auga, kuris duoda gyvybę ir dvasią tautoms, gyvenančioms žemėje: 
\par 6 “Aš, Viešpats, pašaukiau tave tiesoje. Aš laikysiu tave už rankos ir padarysiu tave sandora tautai, šviesa pagonims, 
\par 7 kad atvertum akis akliems, išvestum iš kalėjimo belaisvius, sėdinčius tamsiuose kalėjimuose. 
\par 8 Aš esu Viešpats­tai mano vardas. Aš neduosiu savo garbės kitam ir savo šlovės stabams. 
\par 9 Ankstesnieji dalykai įvyko, skelbiu jums dabar naujus, pranešu anksčiau, negu tai įvyks”. 
\par 10 Giedokite Viešpačiui naują giesmę, gyrius Jam teskamba iki žemės pakraščių! Girkite Jį, plaukiantys jūra ir visa, kas joje, salos ir jų gyventojai! 
\par 11 Tedžiūgauja dykuma ir jos miestai, kaimai, kedariečių apgyventi; uolų gyventojai linksmai šūkaukite nuo kalnų viršūnių. 
\par 12 Atiduokite Viešpačiui šlovę ir skelbkite Jo garbę salose! 
\par 13 Viešpats išeis kaip karžygys, kaip karo vyras pažadins savo įniršį. Jis šauks, šauks garsiai, Jis nugalės savo priešus. 
\par 14 “Aš ilgai tylėjau, buvau tylus ir kantrus. Dabar šauksiu kaip gimdyvė, iš karto viską sugriausiu ir sunaikinsiu. 
\par 15 Aš ištuštinsiu kalnus ir kalvas, upes padarysiu salomis, išdžiovinsiu tvenkinius ir žolę. 
\par 16 Aš vesiu akluosius jiems nežinomais keliais, nepažįstamais takais; pakeisiu jiems tamsą šviesa, kreivus takus­tiesiais. Aš tai įvykdysiu ir jų nepaliksiu. 
\par 17 Trauksis atgal ir bus visai sugėdinti tie, kurie pasitiki drožtais atvaizdais ir sako nulietiems stabams: ‘Jūs esate mūsų dievai’. 
\par 18 Kurtieji, klausykite; aklieji, žiūrėkite ir matykite. 
\par 19 Kas toks aklas kaip mano tarnas ir kurčias kaip mano pasiuntinys, kurį siunčiu? Kas toks aklas kaip ištikimasis, kaip Viešpaties tarnas? 
\par 20 Jis žiūri, bet nepastebi. Jo ausys atviros, bet jis negirdi”. 
\par 21 Viešpats panorėjo dėl savo teisumo išaukštinti ir išgarsinti įstatymą. 
\par 22 Tauta yra apiplėšta ir apvogta, visi klasta suimti ir sugrūsti kalėjimuose. Jie tapo grobiu, ir nėra gelbėtojo. Niekas nereikalauja paleisti aukos. 
\par 23 Kas iš jūsų atkreipia dėmesį į tai ir rūpinasi tuo, kas bus ateityje? 
\par 24 Kas leido Jokūbą apiplėšti ir Izraelį apvogti? Ar ne Viešpats, kuriam mes nusikaltome? Jie nenorėjo vaikščioti Jo keliais ir paklusti įstatymui. 
\par 25 Jis išliejo ant Izraelio savo degančią rūstybę ir užleido karo baisumą. Visur siautė ugnis, bet jis nesuprato, ji degino jį, bet jis neėmė to į širdį.



\chapter{43}


\par 1 Taip sako Viešpats, kuris sutvėrė tave, Jokūbai, ir padarė tave, Izraeli: “Nebijok! Aš atpirkau tave ir pašaukiau tave vardu; tu esi mano. 
\par 2 Kai eisi per vandenį, Aš būsiu su tavimi ir upės nepaskandins tavęs. Kai eisi per ugnį, nesudegsi ir liepsna nesunaikins tavęs. 
\par 3 Aš esu Viešpats, tavo Dievas, Izraelio Šventasis, tavo gelbėtojas. Už tave atidaviau Egiptą, Etiopiją ir Sebą kaip išpirką. 
\par 4 Kadangi tu esi brangus mano akyse, Aš vertinu ir myliu tave. Aš atiduosiu žmones už tave ir tautas už tavo gyvybę. 
\par 5 Nebijok, nes Aš esu su tavimi, Aš parvesiu tavo palikuonis iš rytų ir surinksiu tavuosius iš vakarų. 
\par 6 Aš įsakysiu šiaurei: ‘Atiduok’ ir pietums: ‘Nesulaikyk’. Atvesk mano sūnus iš tolimų šalių ir mano dukteris nuo žemės pakraščių. 
\par 7 Kiekvieną, kuris vadinasi mano vardu, Aš sukūriau savo šlovei, Aš sutvėriau ir padariau jį. 
\par 8 Išvesk aklą tautą, turinčią akis, ir kurčią tautą, turinčią ausis. 
\par 9 Tesusirenka visos tautos ir giminės. Kas iš jų gali pranešti ir paskelbti praeities įvykius? Tepastato liudytojus ir tepasiteisina, kad visi matytų ir girdėtų, jog tai tiesa”. 
\par 10 Viešpats sako: “Jūs esate mano liudytojai ir mano tarnas, kurį pasirinkau, kad žinotumėte, tikėtumėte ir suprastumėte, kad Aš Tas, kuris esu. Pirma manęs nebuvo jokio dievo ir po manęs nebus. 
\par 11 Aš esu Viešpats, ir be manęs nėra kito gelbėtojo. 
\par 12 Aš paskelbiau, išgelbėjau ir pranešiau, kai tarp jūsų nebuvo kito dievo. Jūs esate mano liudytojai,­ sako Viešpats,­kad Aš esu Dievas. 
\par 13 Aš esu nuo laikų pradžios; nėra nė vieno, kuris jus iš mano rankos išplėštų. Aš darau, kas tai sulaikys?” 
\par 14 Taip sako Viešpats, jūsų atpirkėjas, Izraelio Šventasis: “Dėl jūsų sulaužiau Babilono skląsčius, chaldėjų džiaugsmą paverčiau vaitojimu. 
\par 15 Aš esu Viešpats, jūsų Šventasis, Izraelio Kūrėjas, jūsų karalius”. 
\par 16 Taip sako Viešpats, kuris padarė kelią jūroje ir taką giliuose vandenyse, 
\par 17 kuris išvedė kovos vežimą ir žirgą, kariuomenę ir karžygį, ir jie nebeatsikėlė, užgeso kaip dagtis: 
\par 18 “Nebegalvokite apie buvusius dalykus, nekreipkite dėmesio į praeitį. 
\par 19 Štai Aš darau nauja­jau pasirodo, ar nepastebite? Aš padarysiu kelią tyruose, upės tekės dykumose. 
\par 20 Laukiniai žvėrys, šakalai ir stručiai gerbs mane. Aš duosiu vandens tyruose ir dykumose gerti mano išrinktajai tautai. 
\par 21 Tauta, kurią sutvėriau, skelbs mano šlovę. 
\par 22 Jokūbai, tu nesišaukei manęs, Aš nusibodau tau, Izraeli. 
\par 23 Tu neaukojai man avių, deginamosios aukos, ir nepagerbei manęs aukomis. Aš nereikalavau iš tavęs duonos aukų nė smilkalų. 
\par 24 Tu nepirkai man už sidabrą kvepiančių sakų ir nedžiuginai manęs aukų taukais. Bet tu varginai mane savo nuodėmėmis ir nusikaltimais. 
\par 25 Aš panaikinu tavo neteisybes dėl savęs ir neatsiminsiu tavo nusikaltimų. 
\par 26 Primink man, bylinėkimės drauge. Kalbėk, kad galėtum pasiteisinti. 
\par 27 Tavo protėvis nusidėjo, tavo mokytojai nusikalto, 
\par 28 todėl suteršiau tavo šventyklos kunigaikščius ir atidaviau Jokūbą prakeikimui, Izraelį­paniekai”.



\chapter{44}


\par 1 “Klausyk, Jokūbai, mano tarne, ir Izraeli, kurį išsirinkau. 
\par 2 Taip sako Viešpats, kuris tave padarė ir sukūrė tave dar įsčiose: ‘Nebijok, mano tarne Jokūbai ir Ješurūnai, kurį išsirinkau. 
\par 3 Aš išliesiu vandens sroves ant išdžiūvusios žemės ir ant dykumų; išliesiu savo dvasios ant tavo palikuonių ir savo palaiminimą ant tavo vaikų, 
\par 4 jie žels kaip žolė prie vandens, kaip gluosniai prie tekančio upelio. 
\par 5 Vienas sakys: ‘Aš esu Viešpaties’, kitas vadinsis Jokūbo vardu, trečiasis rašys savo ranka: ‘Viešpaties’ ir pasivadins Izraeliu’ ”. 
\par 6 Bet Viešpats, Izraelio karalius, jo atpirkėjas, kareivijų Viešpats, sako: “Aš esu pirmasis ir paskutinysis, be manęs nėra kito dievo. 
\par 7 Kas yra toks, kaip Aš? Tegul jis pasirodo, praneša, paskelbia ir išdėsto, kas buvo praeityje ir kas įvyks. 
\par 8 Nebijokite ir nenusigąskite! Aš jums jau seniai sakiau ir pranešiau: jūs esate mano liudytojai. Be manęs nėra kito dievo, Aš nežinau jokio”. 
\par 9 Visi stabų dirbėjai yra niekas, jų darbas jiems nenaudingas. Jie patys sau yra liudytojai, jie nieko nesupranta ir nemato, todėl jie bus sugėdinti. 
\par 10 Padarytas dievas ir nulietas atvaizdas nieko nevertas. 
\par 11 Visiems jų dirbėjams bus gėda. Visi stabų dirbėjai yra žmonės. Jie visi susirinkę išsigąs ir gėdysis savo darbo. 
\par 12 Kalvis savo stipriomis rankomis dirba stabą: anglių karštyje įkaitinęs geležį, jį daro, kūjais lygina. Jis dirbdamas alksta, trokšta ir pavargsta. 
\par 13 Dailidė dirbdamas naudoja virvę, pieštuką, kampainį, oblių; jais matuoja, nustato formas ir padaro gražaus žmogaus pavidalo stabą, tinkantį gyventi namuose. 
\par 14 Jam pagaminti jis nusikerta kedrą, uosį, ąžuolą ar pušį, užaugusius miške tarp medžių. 
\par 15 Jie tinka žmonėms kurui: jais šildosi ir ant jų gamina maistą. Iš likusio medžio padirbęs dievą, parpuolęs prieš savo drožinį, garbina jį. 
\par 16 Dalį jis sudegino: išsikepė mėsos, išsivirė viralo. Pasisotinęs ir sušilęs džiaugiasi: “Man šilta!” 
\par 17 Iš likusios dalies pasidirbo stabą ir, atsiklaupęs prieš jį, garbino jį ir maldavo: “Išlaisvink mane, nes tu esi mano dievas”. 
\par 18 Jie nežino ir nesupranta, nes jie apakę ir širdimi nesuvokia. 
\par 19 Nė vienas nepagalvoja, kad pusę sudegino, ant anglių išsikepė duonos, išsivirė mėsos ir pavalgė. Iš likusio medžio pasidirbę stabą, parpuola prieš medžio gabalą. 
\par 20 Jis maitinasi pelenais, jo protas aptemęs, jo širdis klaidina, jis nepajėgia išsilaisvinti ir pasakyti: “Ar aš neapgaudinėju pats savęs?” 
\par 21 “Atsimink tai, Jokūbai ir Izraeli, tu esi mano tarnas! Aš tave padariau, Izraeli. Aš neužmiršiu tavęs! 
\par 22 Aš pašalinau tavo nusikaltimus kaip debesį ir tavo nuodėmes­ kaip miglą. Grįžk pas mane, nes Aš atpirkau tave”. 
\par 23 Džiaukitės, dangūs, nes Viešpats tai įvykdė! Šūkaukite, žemės gelmės, linksmai giedokite, miškai ir visi medžiai! Viešpats atpirko Jokūbą ir Izraelyje apsireikš Jo šlovė. 
\par 24 Taip sako Viešpats, tavo atpirkėjas, kuris sukūrė tave: “Aš esu Viešpats, kuris visa darau; Aš vienas ištiesiau dangus ir sutvirtinau žemę savo jėga”. 
\par 25 Jis išniekina žynių ženklus ir burtininkus padaro kvailiais; Jis pašalina išminčius, o jų išmintį padaro kvailyste. 
\par 26 Jis patvirtina savo tarno žodį ir įvykdo savo pasiuntinio pranešimą. Jis sako Jeruzalei: “Tu būsi apgyvendinta”, ir Judo miestams: “Jūs būsite atstatyti”. 
\par 27 Jis sako gelmei: “Išsek! Aš išdžiovinsiu tavo upes!” 
\par 28 Viešpats kalba apie Kyrą: “Jis yra mano tarnas ir įvykdys mano valią!” Jis sako Jeruzalei: “Tu būsi atstatyta”, o šventyklai: “Tavo pamatai bus padėti”.



\chapter{45}


\par 1 Taip sako Viešpats savo pateptajam Kyrui, kurį Jis pasirinko, kad atiduotų jam tautas ir nuginkluotų karalius, atidarytų duris ir vartų niekas jam neuždarytų: 
\par 2 “Aš eisiu pirma tavęs, sulyginsiu kalnus, sutrupinsiu varines duris ir sulaužysiu geležinius užkaiščius. 
\par 3 Aš duosiu tau tamsos lobius ir paslėptus turtus, kad žinotum, jog Aš esu Viešpats, Izraelio Dievas, kuris šaukiu tave vardu. 
\par 4 Dėl mano tarno Jokūbo ir mano išrinktojo Izraelio pašaukiau tave vardu ir suteikiau tau vardą, nors manęs nepažinai. 
\par 5 Aš esu Viešpats ir kito nėra; nėra kito dievo šalia manęs. Aš sujuosiau tave, nors nepažinai manęs, 
\par 6 kad žmonės visame pasaulyje žinotų, jog kito nėra šalia manęs. Aš esu Viešpats ir niekas kitas. 
\par 7 Aš darau šviesą ir sukuriu tamsą, duodu ramybę ir sukuriu pikta. Aš, Viešpats, visa tai darau. 
\par 8 Terasoja dangūs iš aukštybių, telieja debesys teisumą! Žemė teželdina išgelbėjimą ir teišsprogsta teisumas! Aš, Viešpats, tai sukūriau. 
\par 9 Vargas tam, kuris vaidijasi su savo Kūrėju, molio šukė su puodžiumi. Ar sako molis puodžiui: ‘Ką darai?’ arba dirbinys savo dirbėjui: ‘Tu esi neišmanėlis!’? 
\par 10 Vargas tam, kuris sako savo tėvui: ‘Kodėl mane pagimdei?’ arba moteriai: ‘Kodėl gimdai?’ ” 
\par 11 Taip sako Viešpats, Izraelio Šventasis, jo Kūrėjas: “Ar norite klausti apie ateitį, apie mano vaikus ir mano darbus? 
\par 12 Aš sukūriau žemę ir ant jos sutvėriau žmogų; mano rankos ištiesė dangus ir Aš tvarkau visą jų kareiviją. 
\par 13 Aš pašaukiau jį teisybei ir nukreipsiu jo kelius. Jis atstatys mano miestą ir paleis belaisvius ne už pinigus ir ne už atlygį. 
\par 14 Egipto turtai ir Etiopijos pelnas tau atiteks, ir sebiečiai, aukšto ūgio vyrai, tau pasiduos. Jie seks tave grandinėse, parpuolę maldaus, sakydami: ‘Tik pas tave yra Dievas, niekur kitur Dievo nėra’ ”. 
\par 15 Tikrai Tu esi Dievas, kuris slepiesi, Izraelio Dieve, gelbėtojau. 
\par 16 Jie visi bus sugėdinti, visi stabų darytojai sumiš. 
\par 17 Izraelį Viešpats išgelbės amžinu išgelbėjimu, jūs nerausite per amžius ir nesigėdysite. 
\par 18 Nes taip pasakė Viešpats, dangaus Kūrėjas, Dievas, kuris sutvėrė žemę ir nepaliko jos tuščios, bet padarė ją tinkamą gyventi: “Aš esu Viešpats, nėra kito šalia manęs. 
\par 19 Aš nekalbėjau slaptai nei tamsoje. Nesakiau Jokūbo palikuonims, kad veltui manęs ieškotų. Aš, Viešpats, kalbu tiesą ir skelbiu, kas teisinga. 
\par 20 Susirinkite, ateikite ir priartėkite, tautų išlikusieji! Neišmanėliai nešiojasi medinius stabus ir meldžiasi dievams, kurie negali išgelbėti. 
\par 21 Skubėkite, ateikite ir tarkitės! Kas apie tai pranešė pirmiau ir paskelbė iš anksto? Ar ne Aš, Viešpats? Šalia manęs nėra kito. Aš­ teisus Dievas ir gelbėtojas. 
\par 22 Pažiūrėkite į mane ir būkite išgelbėti, visi žemės kraštai,­Aš esu Dievas ir nėra kito. 
\par 23 Aš savimi prisiekiau, mano žodis yra tiesa ir tas žodis pasiliks. Prieš mane suklups kiekvienas kelis ir kiekvienas liežuvis man prisieks. 
\par 24 Bus sakoma: ‘Viešpatyje aš turiu teisumą ir jėgą’. Visi, kurie Jam priešinasi, ateis pas Jį susigėdę. 
\par 25 Visi Izraelio palikuonys bus išteisinti Viešpatyje ir džiūgaus”.



\chapter{46}


\par 1 “Belis sulaužytas, Nebojas guli sutrupintas. Stabai išgabenami pakrauti ant gyvulių. Jie sunki našta juos nešantiems. 
\par 2 Jie sulaužyti ir sutrupinti, negali išsigelbėti, bet patys nešami į nelaisvę. 
\par 3 Klausykite manęs, Jokūbo namai ir Izraelio likuti. Aš jus nešioju ir globoju nuo užgimimo. 
\par 4 Aš esu, kuris nešiosiu jus ir jūsų senatvėje. Tai dariau ir toliau darysiu: nešiosiu ir globosiu jus. 
\par 5 Su kuo mane palyginsite, į ką mane darysite panašų? 
\par 6 Jie auksą iškrato iš maišelių, sidabrą pasveria svarstyklėmis, pasamdo auksakalį, kad padarytų dievą, parpuolę prieš jį, meldžiasi, 
\par 7 užsideda stabą ant pečių, nuneša į prirengtą jam vietą ir pastato. Jis stovi ir nejuda. Kai jie šaukiasi jo, jis negirdi ir neišgelbsti jų iš nelaimės. 
\par 8 Apmąstykite ir supraskite! Imkite tai į širdį, jūs atskalūnai! 
\par 9 Atsiminkite praeitį, kad Aš esu Dievas ir kito dievo, panašaus į mane, nėra. 
\par 10 Aš skelbiu dalykus nuo pat pradžios ir pasakau, kas dar nėra įvykę. Mano nutarimas pasiliks ir Aš padarysiu, ką esu numatęs. 
\par 11 Aš šaukiu iš rytų plėšrų paukštį, tolimo krašto vyrą, mano paskirtą. Aš tai paskelbiau ir padarysiu, nusprendžiau ir įvykdysiu. 
\par 12 Klausykite manęs, sukietėjusios širdies žmonės, esantys toli nuo teisumo. 
\par 13 Aš priartinu savo teisumą, ir jis nebus toli. Mano išgelbėjimas neuždels. Aš duosiu išgelbėjimą Sione dėl savo šlovės­Izraelio”.



\chapter{47}


\par 1 “Nuženk ir sėskis dulkėse, mergele, Babilono dukra! Čia nėra sosto, todėl sėskis ant žemės, chaldėjų dukra! Tavęs nebevadins švelnia ar gražuole. 
\par 2 Stok prie girnų ir malk! Nusiimk šydą, nusisiausk apsiaustą, atidengusi blauzdas brisk per upę! 
\par 3 Tu būsi apnuoginta ir tavo gėda bus matoma. Aš išliesiu savo kerštą ir niekas jo nesustabdys”. 
\par 4 Mūsų atpirkėjas­kareivijų Viešpats, Izraelio Šventasis yra Jo vardas. 
\par 5 “Chaldėjų dukra, sėdėk tylėdama ir eik į tamsybę, nes tu nebebūsi vadinama karalysčių valdove. 
\par 6 Aš buvau įpykęs ant savo tautos, suteršiau savo paveldą, padaviau juos į tavo ranką. Tu neparodei jiems gailestingumo, net žilagalviams uždėjai labai sunkų jungą. 
\par 7 Tu sakei: ‘Aš išliksiu per amžius, būsiu valdovė visados’. Tu neėmei to į širdį ir nepagalvojai apie savo galą. 
\par 8 Dabar išgirsk, mėgstanti prabangiai ir nerūpestingai gyventi. Tu galvoji: ‘Aš esu ir šalia manęs nėra kitos. Aš nesėdėsiu kaip našlė ir mano vaikai nebus iš manęs atimti’. 
\par 9 Ūmai, vieną dieną, tau šie du dalykai atsitiks: tapsi našlė ir bevaikė. Taip įvyks, nepaisant tavo burtų ir daugybės kerų. 
\par 10 Tu pasitikėjai savo nedorybe ir sakei: ‘Manęs niekas nemato’. Tavo išmintis ir supratimas tave suvedžiojo. Tu galvojai: ‘Aš esu ir kitos nėra šalia manęs’. 
\par 11 Tave užklups nelaimė ir nežinosi, iš kur ji; užguls priespauda, iš kurios neišsipirksi; staiga ateis sunaikinimas, kurio nenujauti. 
\par 12 Pakilk su savo burtais bei daugybe kerų, kuriais vadovavaisi nuo pat savo jaunystės. Gal jie bus tau naudingi ir tu nugalėsi. 
\par 13 Tu esi bejėgė dėl savo daugybės patarėjų. Tegu astrologai, žvaigždžių stebėtojai, dangaus ženklų tyrinėtojai atsistoja ir išgelbsti tave nuo to, kas tave užgrius. 
\par 14 Jie yra kaip ražienos, ugnis sudegins juos. Jie negali išgelbėti savo gyvybės iš liepsnos, kuri nėra žarijos pasišildyti ar židinys, prie kurio galima sėdėti. 
\par 15 Tokie yra tavo burtininkai, su kuriais prekiavai nuo pat jaunystės. Jie nuklys kiekvienas į savo pusę, nė vienas negelbės tavęs”.



\chapter{48}


\par 1 “Klausykite, Jokūbo namai, vadinami Izraelio vardu ir kilę iš Judo šaltinio, kurie prisiekiate Viešpaties vardu ir kalbate apie Izraelio Dievą, bet ne tiesoje ir teisume. 
\par 2 Jie vadinasi šventojo miesto vardu ir remiasi Izraelio Dievu. Jo vardas­kareivijų Viešpats. 
\par 3 Aš praeities įvykius paskelbiau seniai. Jie išėjo iš mano lūpų ir juos Aš atvėriau. Staiga Aš tai padariau, ir jie įpyko. 
\par 4 Kadangi Aš žinojau, kad tu kietasprandis, tavo sprandas geležinis ir tavo kakta varinė, 
\par 5 Aš pranešiau tai tau iš anksto, pirma, negu įvyko, kad nesakytum: ‘Mano stabas tai įvykdė, mano drožti ir lieti atvaizdai taip įsakė’. 
\par 6 Tu girdėjai ir matei visa tai, ar nenori to pripažinti? Dabar skelbiu tai, ko dar nežinai­tai nauji ir paslėpti dalykai. 
\par 7 Tai padaryta dabar, ne pradžioje. Pirmiau apie tai negirdėjai ir negali sakyti: ‘Aš žinojau’. 
\par 8 Tu to negirdėjai ir nežinojai, tai nepasiekė tavo ausų. Bet Aš žinojau, kad būsi neištikimas ir nusikalsi, ir vadinau tave neklaužada nuo pat gimimo dienos. 
\par 9 Dėl savo vardo Aš sulaikysiu savo rūstybę ir dėl savo šlovės susivaldysiu, kad tavęs nesunaikinčiau. 
\par 10 Aš apvaliau tave ugnimi, bet ne kaip sidabrą, Aš išbandžiau tave vargų krosnyje. 
\par 11 Dėl savęs, dėl savęs Aš tai darysiu, kad mano vardas nebūtų suterštas. Savo šlovės Aš neduosiu kitam. 
\par 12 Klausyk manęs, Jokūbai ir Izraeli, kurį pašaukiau. Aš, Aš esu pirmasis ir paskutinysis. 
\par 13 Aš sukūriau žemę ir ištiesiau dangus. Aš juos pašaukiau, ir jie stovi čia. 
\par 14 Susirinkite visi ir pasiklausykite. Kas iš jų tai paskelbė? Tas, kurį Viešpats myli, įvykdys Jo sprendimą Babilonui ir chaldėjams. 
\par 15 Aš, Aš tai kalbėjau, pašaukiau jį, atvedžiau ir jam seksis. 
\par 16 Priartėkite prie manęs ir išgirskite; nuo pat pradžios Aš nekalbėjau slaptai, Aš buvau anksčiau, negu tai įvyko. Dabar Viešpats Dievas ir Jo Dvasia siuntė mane”. 
\par 17 Taip sako Viešpats, tavo atpirkėjas, Izraelio Šventasis: “Aš, Viešpats, tavo Dievas, mokau tave, kas naudinga, ir vedu keliu, kuriuo turėtum eiti. 
\par 18 Jei būtum klausęs mano įsakymų, tai tavo ramybė būtų kaip upė ir tavo teisumas kaip jūros bangos. 
\par 19 Tavo palikuonys būtų buvę kaip smiltys ir tavo ainiai kaip smėlio grūdeliai. Jų vardas nebūtų išnykęs ir nebūtų žuvęs mano akyse”. 
\par 20 Išeikite iš Babilono, bėkite nuo chaldėjų! Džiūgaudami skelbkite tą žinią! Teskamba tai iki žemės pakraščių! Sakykite: “Viešpats atpirko savo tarną Jokūbą”. 
\par 21 Jie netroško dykumoje, kai Jis juos vedė; Jis perskėlė uolą, ir vanduo išsiveržė. 
\par 22 Viešpats sako: “Nedorėlis neturi ramybės”.



\chapter{49}


\par 1 Klausykitės, salos, stebėkite, toli esančios tautos! Viešpats pašaukė mane dar man negimus, Jis minėjo mano vardą, kai tebebuvau motinos įsčiose. 
\par 2 Jis padarė mano žodžius lyg aštrų kardą ir pridengė mane savo rankos šešėliu; Jis padarė mane tinkama strėle ir paslėpė savo strėlinėje. 
\par 3 Jis sakė: “Tu, Izraeli, esi mano tarnas, per tave būsiu pašlovintas”. 
\par 4 Aš sakiau: “Veltui dirbau, be reikalo ir tuščiai eikvojau savo jėgas; bet pas Viešpatį yra mano teisingumas, pas Dievą mano atlyginimas”. 
\par 5 Dabar kalba Viešpats, kuris nuo gimimo padarė mane savo tarnu, kad sugrąžinčiau Jam Jokūbą ir surinkčiau Izraelį. Aš būsiu šlovingas Viešpaties akyse, Dievas yra mano stiprybė. 
\par 6 Jis sako: “Negana to, kad tu būsi mano tarnu Jokūbo giminėms sugrąžinti ir Izraelio likučiui surinkti. Aš tave padarysiu šviesa pagonims, kad tu būtum mano išgelbėjimas iki žemės pakraščių”. 
\par 7 Taip sako Viešpats, Izraelio atpirkėjas, jo Šventasis, valdovų vergui, kurį žmonės paniekino ir tautos bjaurisi: “Karaliai pamatys ir atsistos, kunigaikščiai pagarbins dėl Viešpaties, kuris ištikimas, ir Izraelio Šventojo, kuris tave išsirinko”. 
\par 8 Taip sako Viešpats: “Tinkamu metu išklausiau tave ir išgelbėjimo dieną padėjau tau; padarysiu tave sandora tautai, kad atkurtum šalį ir apgyvendintum apleistą nuosavybę. 
\par 9 Kad sakytum belaisviams: ‘Išeikite’, ir tamsoje esantiems: ‘Pasirodykite’. Jie ganysis pakelėse, aukštumos bus jų ganykla. 
\par 10 Jie nealks ir netrokš, jų neištiks saulė nė karštis; Tas, kuris jų pasigailėjo, ves juos prie tyro vandens šaltinių. 
\par 11 Aš pravesiu kelius kalnuose ir vieškelius aukštumose. 
\par 12 Štai šitie ateis iš toli, tie­iš šiaurės ir vakarų, o anie­iš Sinimo krašto”. 
\par 13 Giedokite, dangūs! Džiūgauk, žeme! Kalnai, linksmai giedokite! Nes Viešpats paguodė savo tautą, pasigailėjo savo prispaustųjų! 
\par 14 Bet Sionas sakė: “Viešpats paliko mane, Viešpats užmiršo mane”. 
\par 15 “Ar gali moteris užmiršti savo kūdikį, nepasigailėti savo sūnaus? Jei ji ir užmirštų, tačiau Aš neužmiršiu tavęs. 
\par 16 Aš įrašiau tave į savo rankos delną, tavo sienos visada yra mano akyse. 
\par 17 Tavo sūnūs atskuba, o tavo ardytojai ir naikintojai pasitraukia. 
\par 18 Pakelk akis ir apsidairyk aplinkui: jie visi renkasi prie tavęs. Kaip Aš gyvas,­sako Viešpats,­tu juos priimsi kaip papuošalus ir jais pasidabinsi lyg sužadėtinė. 
\par 19 Tavo griuvėsiuose, ištuštėjusiose vietose ir sunaikintoje šalyje bus ankšta gyventojams, o tie, kurie tave naikino, bus toli. 
\par 20 Tavo vaikai, gimę priespaudoje, sakys: ‘Mums ankšta, padaryk daugiau vietos gyventi’. 
\par 21 Tu galvosi: ‘Kada man gimė šitie? Juk aš buvau nevaisinga ir bevaikė, tremtinė ir belaisvė. Kas juos užaugino? Aš buvau viena palikta. Iš kur šitie?’ ” 
\par 22 Taip sako Viešpats Dievas: “Aš ištiesiu savo ranką į tautas ir duosiu ženklą pagonims; jie atneš tavo sūnus glėbyje ir tavo dukteris ant pečių. 
\par 23 Karaliai bus tavo sargai ir karalienės­auklės. Jie puls prieš tave veidu į žemę ir laižys dulkes nuo tavo kojų. Tuomet suprasi, kad Aš esu Viešpats; nebus sugėdinti, kurie laukia manęs. 
\par 24 Argi iš galingojo galima atimti grobį? Ar belaisvius galima išgelbėti iš galiūno? 
\par 25 Bet taip sako Viešpats: ‘Belaisviai bus atimti iš galiūno ir grobis išplėštas iš galingojo. Aš kovosiu su tais, kurie kovojo prieš tave, ir tavo vaikus išgelbėsiu. 
\par 26 Aš maitinsiu tavo prispaudėjus jų pačių kūnu, ir jie gers savo kraują kaip vyną. Tada kiekvienas kūnas žinos, kad Aš esu Viešpats, tavo gelbėtojas ir atpirkėjas, Jokūbo Galingasis’ ”.



\chapter{50}


\par 1 Taip sako Viešpats: “Kur yra jūsų motinos skyrybų raštas, kuriuo Aš ją atleidau? Kur yra skolintojas, kuriam už skolą jus atidaviau? Dėl savo nusikaltimų jūs pardavėte save, dėl jūsų neištikimybės jūsų motina atleista. 
\par 2 Kodėl atėjęs Aš neradau žmogaus? Aš šaukiau ir niekas neatsiliepė. Argi mano ranka sutrumpėjo gelbėti? Argi neturiu jėgos išlaisvinti? Savo sudraudimu išdžiovinu jūrą, upes paverčiu dykuma. Žuvys trokšta ir gaišta, nes nebėra vandens. 
\par 3 Aš aptemdau dangų, apsupu jį ašutine”. 
\par 4 Viešpats Dievas davė man miklų liežuvį paguosti nuvargusį. Kiekvieną rytą Jis žadina mane, kad klausyčiau Jo mokymo; 
\par 5 Viešpats Dievas atvėrė man ausis, aš nesipriešinau ir nesitraukiau. 
\par 6 Aš leidausi mušamas ir tąsomas, neslėpiau veido nuo mane plūstančių ir į mane spjaudančių. 
\par 7 Viešpats Dievas padės man, todėl aš nebūsiu sugėdintas. Aš padariau savo veidą kietą kaip titnagas ir žinau, kad nebūsiu pažemintas. 
\par 8 Arti yra Tas, kuris mane išteisina, kas dabar su manimi ginčysis? Stokite čia. Kas nori su manimi bylinėtis, teateina. 
\par 9 Viešpats Dievas padės man. Kas drįs mane pasmerkti? Jie visi pasens kaip drabužis, kandys juos suės. 
\par 10 Kas iš jūsų bijo Viešpaties, teklauso Jo tarno balso. Kas vaikščioja tamsoje, tepasitiki Viešpačiu, savo Dievu. 
\par 11 Jūs visi, kurie įžiebiate ugnį ir apsupate save žiežirbomis, vaikščiokite savo ugnies šviesoje ir tarp savo žiežirbų. Tai padarys mano ranka, jūs atsigulsite skausmuose.



\chapter{51}


\par 1 Klausykite manęs, kurie sekate teisumą ir ieškote Viešpaties. Pažvelkite į uolą, iš kurios jūs iškirsti, ir į duobę, iš kurios jūs iškasti. 
\par 2 Pažvelkite į savo tėvą Abraomą ir į Sarą, kuri jus pagimdė. Aš pašaukiau jį vieną, palaiminau ir padauginau. 
\par 3 Viešpats paguos Sioną ir atgaivins visus jo griuvėsius, Jis padarys jo tyrus kaip Edeną, dykumas kaip Viešpaties sodą. Linksmybė ir džiaugsmas bus ten, padėkos giesmės skambės. 
\par 4 Mano tauta, klausyk manęs, mano giminės, išgirskite mane. Įstatymas išeis iš manęs; mano teisingumas bus šviesa tautoms. 
\par 5 Mano teisumas arti ir išgelbėjimas greitai ateis. Aš teisiu tautas. Salos lauks manęs, jos pasitikės manimi. 
\par 6 Pakelkite savo akis į dangų, pažvelkite į žemę! Dangūs praeis kaip dūmai, žemė susidėvės kaip drabužis, jos gyventojai taip pat išmirs. Bet mano išgelbėjimas liks amžinai, mano teisumas nesibaigs. 
\par 7 Klausykite manęs, kurie pažįstate teisumą, tauta, kurios širdyje yra mano įstatymas. Nebijokite žmonių pajuokos, neišsigąskite jų plūdimo. 
\par 8 Kaip vilnonį drabužį juos sugrauš kandys; bet mano teisumas liks amžinai, mano išgelbėjimas­ kartų kartoms. 
\par 9 Gelbėk, Viešpatie, prašau, padėk, kaip anomis dienomis senosioms kartoms! Ar ne Tu sukapojai Rahabą ir perdūrei slibiną? 
\par 10 Ar ne Tu išdžiovinai jūrą ir jos gelmėse padarei kelią pereiti išpirktiesiems? 
\par 11 Viešpaties išpirktieji sugrįš, ateis į Sioną džiūgaudami, amžinas džiaugsmas lydės juos. Džiaugsmas ir linksmybė sugrįš, o skausmai ir vaitojimai išnyks. 
\par 12 “Aš, Aš guodžiu tave. Kas gi tu, kad bijotum mirtingo žmogaus, žmogaus sūnaus, kuris yra lyg žolė? 
\par 13 Tu užmiršai Viešpatį, savo Kūrėją, kuris ištiesė dangus ir sukūrė žemę. Tu nuolatos drebėjai dėl prispaudėjo žiaurumo, kuris siekė tave sunaikinti! Kur dingo prispaudėjo žiaurumas? 
\par 14 Prispaustasis bus greitai išlaisvintas; jis nemirs, nežengs į duobę ir nebadaus. 
\par 15 Aš esu Viešpats, tavo Dievas, kuris perskyrė šėlstančios jūros bangas. Kareivijų Viešpats yra mano vardas. 
\par 16 Aš sukūriau dangų ir žemę, pridengiau tave savo ranka, įdėjau savo žodžius į tavo lūpas ir sakiau Sionui: ‘Tu­mano tauta’ ”. 
\par 17 Pabusk, pabusk, kelkis, Jeruzale! Tu gėrei iš Viešpaties rankos Jo rūstybės taurę; svaiginančią rūstybę išgėrei iki pat dugno. 
\par 18 Nė vienas iš Jeruzalės sūnų nepakėlė jos ir nepadėjo jai. 
\par 19 Tave ištiko sunaikinimas ir sugriovimas, badas ir kardas, bet kas liūdi tavęs, kas paguos tave? 
\par 20 Tavo vaikai neteko jėgų ir gulėjo pakelėse kaip pagautas gyvulys tinkle. Jie pilni Viešpaties rūstybės, tavo Dievo pabarimo. 
\par 21 Tad išgirskite, jūs, suvargusieji ir girti, bet ne nuo vyno. 
\par 22 Taip sako Viešpats, tavo valdovas, tavo Dievas, kuris gina savo tautą: “Aš paėmiau iš tavo rankos apsvaigimo taurę; tu nebegersi daugiau mano rūstybės. 
\par 23 Aš ją paduosiu į rankas tavo kankintojų, kurie sakė tau: ‘Pasilenk, kad liptume per tave!’ Taip, tavo nugara buvo padaryta lyg žemė, lyg kelias, kuriuo jie ėjo”.



\chapter{52}


\par 1 Pabusk, pabusk, apsirenk stiprybe, Sione! Jeruzale, šventasis mieste, apsirenk savo gražius drabužius! Į tave nebeįeis nei neapipjaustytas, nei nešvarus. 
\par 2 Jeruzale, kelkis, nusikratyk dulkes, atsirišk nuo kaklo pančius, belaisve Siono dukra! 
\par 3 Taip sako Viešpats: “Jūs pardavėte save už nieką ir būsite išpirkti be pinigų”. 
\par 4 Štai ką sako Viešpats Dievas: “Pradžioje mano tauta nuėjo į Egiptą viešėti, ir Asirija ją pavergė be priežasties”. 
\par 5 Viešpats sako: “Ką Aš dabar turiu, kai mano tauta yra veltui išvesta. Jos pavergėjai kankina juos,­sako Viešpats,­ir nuolatos piktžodžiauja mano vardui. 
\par 6 Todėl mano tauta pažins mano vardą. Todėl pažins tą dieną, kad Aš esu Tas, kuris sako: ‘Aš esu’ ”. 
\par 7 Kokios gražios kalnuose kojos to, kuris atneša gerą žinią ir skelbia ramybę, kuris atneša linksmą naujieną, skelbia išgelbėjimą ir sako Sionui: “Tavo Dievas viešpatauja”. 
\par 8 Sargyba šauks, ir visi drauge džiūgaus, nes aiškiai matys savo akimis, kaip Viešpats sugrąžina Sioną. 
\par 9 Džiaukitės, Jeruzalės griuvėsiai! Viešpats paguodė savo tautą ir išlaisvino Jeruzalę. 
\par 10 Viešpats apnuogino savo šventą ranką visų tautų akivaizdoje. Visi žemės pakraščiai matys mūsų Dievo išgelbėjimą. 
\par 11 Traukitės, traukitės! Išeikite iš ten! Nelieskite nieko sutepto. Išeikite iš jų būrio, apsivalykite, Viešpaties indų nešėjai! 
\par 12 Jūs išeisite neskubėdami ir nebėgsite. Viešpats eis pirma jūsų, Izraelio Dievas bus jūsų apsauga. 
\par 13 Štai mano tarnas elgsis išmintingai, jis bus išaukštintas, išgirtas ir labai kilnus. 
\par 14 Kaip daugelis baisėjosi tavimi, taip jo veidas buvo nežmoniškai sudarkytas, jis buvo nebepanašus į žmogų. 
\par 15 Jis daugelį tautų nustebins, karaliai jo akivaizdoje užsičiaups: apie ką jiems niekada nebuvo pasakota, jie matys ir, ko jie niekad negirdėjo, jie supras.



\chapter{53}


\par 1 Kas patikėjo mūsų skelbimu? Ir kam buvo apreikšta Viešpaties ranka? 
\par 2 Jis išaugs Jo akivaizdoje kaip gležnas augalas, kaip šaknis iš sausos žemės. Neturi jis nei išvaizdos, nei patrauklumo, kai žiūrime į jį, nėra jokio grožio, kuris mus prie jo trauktų. 
\par 3 Jis paniekintas ir žmonių atmestas, skausmų vyras, negalią pažinęs; mes slėpėme nuo jo savo veidus, jis buvo paniekintas, ir mes jį nieku laikėme. 
\par 4 Tikrai jis nešė mūsų negalias ir sau pasiėmė mūsų skausmus. O mes laikėme jį nubaustu, Dievo ištiktu ir pažemintu. 
\par 5 Jis buvo sužeistas už mūsų kaltes ir sumuštas už mūsų nuodėmes. Bausmė dėl mūsų ramybės krito ant jo; jo žaizdomis esame išgydyti. 
\par 6 Mes visi buvome paklydę kaip avys, kiekvienas ėjome savo keliu. Bet Viešpats uždėjo ant jo visus mūsų nusikaltimus. 
\par 7 Jis buvo kankinamas ir žeminamas, bet neatvėrė savo burnos. Kaip avinėlis, vedamas pjauti, ir kaip avis, kuri tyli prieš kirpėjus, jis neatvėrė savo burnos. 
\par 8 Jis buvo paimtas iš kalėjimo ir teismo. Kas paskelbs jo giminę? Nes jis buvo atskirtas nuo gyvųjų šalies ir varginamas dėl mano tautos nusikaltimų. 
\par 9 Jam paruošė kapą su nedorėliais, su turtingais po jo mirties, nors jis nepadarė nieko blogo ir jo lūpose nebuvo melo. 
\par 10 Bet Viešpats panorėjo jį sumušti, Jis atidavė jį skausmui. Kai Tu padarysi jo sielą auka už nuodėmę, jis matys savo palikuonis, jo dienos bus prailgintos ir jo rankomis Viešpaties valia bus įvykdyta. 
\par 11 Jis matys savo sielos vargą ir bus patenkintas. Per savo pažinimą mano teisusis tarnas nuteisins daugelį, nes jis neš jų nusikaltimus. 
\par 12 Todėl Aš duosiu jam dalį su didžiaisiais, ir jis dalinsis grobį su stipriaisiais, kadangi jis atidavė savo sielą mirčiai ir buvo priskirtas prie nusikaltėlių. Jis nešė daugelio nuodėmes ir užtarė nusidėjėlius.



\chapter{54}


\par 1 “Giedok, nevaisingoji, kuri negimdei! Džiūgauk linksmai, kuri nepažinai gimdymo skausmų! Nes apleistoji turi daugiau vaikų negu turinti vyrą”,­sako Viešpats. 
\par 2 “Išplėsk savo palapinės vietą, ištempk savo buveinės uždangalus, netaupyk! Pailgink virves ir sustiprink kuolelius! 
\par 3 Tu išsiplėsi į dešinę ir į kairę, tavo palikuonys valdys tautas ir apgyvendins tuščius miestus. 
\par 4 Nesibijok, nes nebūsi sugėdinta! Nerausk, nes nebūsi niekinama! Jaunystės gėdą tu užmirši ir našlystės pajuokos neatsiminsi. 
\par 5 Tavo Kūrėjas yra tavo vyras, kareivijų Viešpats­Jo vardas. Tavo atpirkėjas yra Izraelio Šventasis, Jis bus vadinamas pasaulio Dievu. 
\par 6 Viešpats pašaukė tave kaip atskirtą, dvasioje suvargusią moterį, kaip atstumtą jaunystės žmoną”,­sako Dievas. 
\par 7 “Trumpai valandėlei palikau tave, bet, didžiai gailėdamasis, surinksiu tavuosius. 
\par 8 Rūstybės valandą paslėpiau savo veidą akimirkai, bet amžina malone pasigailėsiu tavęs”,­sako Viešpats, tavo atpirkėjas. 
\par 9 “Tai kaip Nojaus vandenys man. Aš prisiekiau, kad Nojaus vandenys nebeužlies žemės, taip ir dabar prisiekiu, kad nebekeršysiu ir nebebausiu tavęs. 
\par 10 Kalnai gali griūti ir kalvos drebėti, bet mano malonė nuo tavęs nebepasitrauks ir mano ramybės sandora tavęs nepaliks”,­sako Viešpats. 
\par 11 “Tu varginta, audrų mėtyta, nepaguosta; Aš tavo akmenis klosiu eilėmis ir dėsiu safyrų pamatą. 
\par 12 Aš tavo langus padarysiu iš agato, vartus statysiu iš rubinų ir sienas iš brangakmenių. 
\par 13 Visi tavo vaikai bus Viešpaties mokomi ir gyvens didelėje ramybėje. 
\par 14 Tu įsitvirtinsi teisume, priespaudos nebepatirsi. Nebijok, nes siaubas prie tavęs nepriartės. 
\par 15 Štai jie rinksis kartu prieš tave, bet be manęs. Kas susirinks prieš tave, kris dėl tavęs. 
\par 16 Aš sutvėriau kalvį, kuris įpučia anglis ir padaro reikiamą įrankį. Aš taip pat sutvėriau naikintoją, kad griautų. 
\par 17 Joks prieš tave nukaltas ginklas neturės galios. Kiekvienas liežuvis, kuris priešinsis tau teisme, bus pasmerktas. Tai yra Viešpaties tarnų paveldėjimas ir jų teisumas yra iš manęs”,­sako Viešpats.



\chapter{55}


\par 1 “Visi, kurie trokštate, ateikite prie vandens, visi, kurie neturite pinigų! Pirkite ir valgykite; pirkite vyno ir pieno be pinigų, veltui. 
\par 2 Kodėl jūs leidžiate pinigus už tai, kas nėra maistas, ir dirbate už tai, kas jūsų nepatenkina? Klausykite manęs atidžiai ir valgykite tai, kas jūsų sielą atgaivins ir sustiprins. 
\par 3 Palenkite savo ausį ir ateikite pas mane, klausykite, ir jūsų siela bus gyva! Aš padarysiu amžiną sandorą su jumis, kaip suteikiau amžiną malonę Dovydui. 
\par 4 Aš jį paskyriau liudytoju tautoms, kunigaikščiu ir vadu giminėms. 
\par 5 Tu pašauksi tautas, kurių nepažįsti; tautos, kurios tavęs nepažįsta, bėgs pas tave dėl Viešpaties, tavo Dievo, Izraelio Šventojo, nes Jis pašlovino tave. 
\par 6 Ieškokite Viešpaties, kol galima Jį rasti, šaukitės Jo, kol Jis arti. 
\par 7 Nedorėlis tepalieka savo kelią, o neteisusis­savo mintis; tegrįžta jis pas Viešpatį, mūsų Dievą, ir Jis pasigailės jo, nes yra gailestingas. 
\par 8 Nes mano mintys yra ne jūsų mintys ir mano keliai­ne jūsų keliai,­sako Viešpats.­ 
\par 9 Nes kiek dangūs yra aukščiau už žemę, tiek mano keliai aukštesni už jūsų kelius ir mano mintys­už jūsų mintis. 
\par 10 Kaip lietus ir sniegas krinta iš dangaus ir nesugrįžta, bet sudrėkina žemę, padaro ją derlingą ir duoda sėklos sėjėjui bei duonos valgytojui, 
\par 11 taip ir mano žodis, kuris išeina iš mano burnos, negrįš tuščias, bet įvykdys mano valią ir atliks tai, kam yra siųstas. 
\par 12 Jūs išeisite linksmi, ir ramybė lydės jus. Kalnai ir kalvos džiūgaus jūsų akivaizdoje ir laukų medžiai plos rankomis. 
\par 13 Erškėčių vietoje augs kiparisai ir usnių vietoje­mirtos. Tai bus atminimas, amžinas ženklas Viešpaties garbei”.



\chapter{56}


\par 1 Taip sako Viešpats: “Laikykitės teisingumo ir vykdykite teisybę, nes mano išgelbėjimas arti, mano teisumas tuoj bus apreikštas. 
\par 2 Palaimintas žmogus, kuris tai vykdo, ir žmogaus sūnus, kuris to laikosi: saugo sabatą nesuterštą ir savo rankas sulaiko nuo pikto”. 
\par 3 Svetimtautis, kuris prisiglaudė prie Viešpaties, tenesako: “Viešpats atskyrė mane nuo savo tautos”, ir eunuchas tenesako: “Aš kaip padžiūvęs medis”. 
\par 4 Nes Viešpats taip sako: “Eunuchams, kurie švenčia mano sabatą, pasirenka, kas man patinka, ir laikosi mano sandoros, 
\par 5 duosiu savo namuose vietos ir padarysiu jų vardą garsesnį už sūnų bei dukterų; duosiu jiems amžiais nežūstantį vardą. 
\par 6 Svetimtaučius, kurie prisijungs prie Viešpaties, Jam tarnaus, Jo vardą mylės, bus Jo tarnai, švęs sabatą ir jo nesuteps bei laikysis mano sandoros, 
\par 7 atvesiu į savo šventąjį kalną ir pradžiuginsiu savo maldos namuose; jų aukos ir deginamosios aukos bus mėgiamos ant mano aukuro, nes mano namai bus vadinami maldos namais visoms tautoms”. 
\par 8 Taip sako Viešpats Dievas, kuris surinko Izraelio išsklaidytuosius: “Prie tų, kurie jau surinkti, Aš surinksiu ir kitus”. 
\par 9 Visi laukiniai ir miško žvėrys, ateikite ėsti! 
\par 10 Jo sargai yra akli, neturi supratimo; jie yra lyg šunys, kurie neloja. Jie mėgsta gulėti, snausti ir sapnuoti. 
\par 11 Jie yra godūs šunys, kurie niekada nepasisotina. Jie yra ganytojai, neturintys supratimo. Jie žiūri į savo kelią ir ieško sau naudos. 
\par 12 Jie sako: “Ateikite, duosime jums vyno, pasigerkime! Taip darykime šiandien, o rytoj bus dar geriau!”



\chapter{57}


\par 1 Teisusis žūva, ir nė vienas į tai nekreipia dėmesio; gailestingasis miršta, ir niekas nepastebi, kad teisusis paimamas iš vargo 
\par 2 ir eina ramybėn. Kas elgiasi dorai, ilsėsis savo guoliuose. 
\par 3 “Jūs, kerėtojos sūnūs, svetimautojo ir paleistuvės palikuonys, ateikite ir klausykite! 
\par 4 Iš ko jūs tyčiojatės? Kam rodote liežuvį? Argi jūs ne nusikaltėlių vaikai, melagių palikuonys? 
\par 5 Jūs degate aistra stabams po lapuotais medžiais; žudote kūdikius paupiuose, uolų plyšiuose. 
\par 6 Jūsų dalis tarp glotnių upelio akmenėlių; toks jūsų likimas; stabams jūs aukojate gėrimus ir duonos auką. Argi Aš turėčiau tai pakęsti? 
\par 7 Kalnuose patiesęs savo guolį, tu eini aukoti aukų. 
\par 8 Ant durų staktų tu pasidarei sau ženklą. Nudengusi ir praplatinusi lovą, tu palikai mane; susitarus su jais, pamėgai jų lovą. 
\par 9 Tu pasitepusi išėjai pas karalių; savo pasiuntinius pasiuntei toli, kritai iki pragaro. 
\par 10 Tu pavargai besiblaškydama, bet nesakei: ‘Nėra vilties’. Tu pamėgai tokį gyvenimą ir taip toliau darei. 
\par 11 Ko tu bijojai, kad melavai ir neprisiminei manęs, ir neėmei į širdį? Ar ne dėl to, kad taip ilgai tylėjau, tu lioveisi manęs bijoti? 
\par 12 Aš paskelbsiu tavo teisumą ir tavo darbus; jie tau nebus naudingi. 
\par 13 Kai tu šauksi, tegelbsti tave tavo surinktieji. Vėjo dvelkimas juos visus nuneš. O kas pasitiki manimi, tas gaus žemę ir paveldės mano šventąjį kalną”. 
\par 14 Ir sakys: “Nutieskite kelią, nutieskite kelią mano tautai ir pašalinkite kliūtis”. 
\par 15 Taip sako Aukščiausiasis ir Prakilnusis, kuris gyvena amžinybėje, kurio vardas šventas: “Aš gyvenu aukštybėje ir šventoje vietoje su tais, kurie turi atgailaujančią ir nuolankią dvasią, kad atgaivinčiau nuolankiųjų dvasią ir atgailaujančiųjų širdį. 
\par 16 Aš neamžinai bylinėsiuos ir nevisada rūstausiu, nes tada gyvybės dvasia, kurią Aš įkvėpiau, sunyktų. 
\par 17 Aš užsirūstinau dėl nuodėmingo tautos godumo, smogiau jai ir savo veidą paslėpiau, bet ji ėjo toliau klaidingu savo širdies keliu. 
\par 18 Aš mačiau jos kelius ir Aš gydysiu bei vedžiosiu ją, suteiksiu jai ramybės ir paguodos. 
\par 19 Ramybė ir taika arti ir toli esantiems,­sako Viešpats.­Aš ją pagydysiu. 
\par 20 Nedorėliai yra lyg sujudinta jūra, kuri negali nurimti, jos vanduo išmeta purvą ir dumblą. 
\par 21 Nėra ramybės nedorėliui”,­sako mano Dievas.



\chapter{58}


\par 1 “Šauk garsiai, nesivaržyk! Pakelk savo balsą kaip trimitą; pranešk mano tautai jos kaltę ir Jokūbo namams jų nuodėmes. 
\par 2 Jie ieško manęs kas dieną ir nori žinoti mano kelius kaip tauta, kuri elgiasi teisiai ir nėra palikusi savo Dievo. Jie prašo manęs teisingo sprendimo, jie mėgsta ateiti prie Dievo. 
\par 3 ‘Kodėl mums pasninkaujant, Tu nematai? Kodėl mums varginant savo sielas, Tu nepaisai?’ Pasninko dieną jūs ieškote malonumų ir išnaudojate savo darbininkus. 
\par 4 Jūs pasninkaujate dėl vaidų ir kivirčų, smogiate nedorybės kumščiu. Jūs nepasninkaujate šiandien taip, kad jūsų balsas būtų išgirstas aukštybėse. 
\par 5 Ar tai pasninkas, kokį Aš pasirinkau, kad žmogus vargintų per dieną savo sielą, nulenktų galvą kaip nendrę, klotų po savimi ašutinę ir pasibarstytų pelenų? Ir tai jūs vadinate pasninku, Viešpačiui priimtina diena? 
\par 6 Ar ne toks pasninkas, kurį Aš pasirinkau: pašalinkite nedorybės pančius, išlaisvinkite pavergtuosius, suteikite laisvę prislėgtiesiems, sulaužykite kiekvieną jungą. 
\par 7 Pasidalink maistą su alkanu, benamius ir vargšus parsivesk į namus, pamatęs nuogą, aprenk jį, nesislėpk nuo savo paties kūno. 
\par 8 Tada tavo šviesa nušvis kaip aušra, tavo sveikata išsiskleis greitai, tavo teisumas eis pirma tavęs ir Viešpaties šlovė lydės tave. 
\par 9 Tu šauksies, ir Viešpats atsakys, prašysi pagalbos, ir Jis tars: ‘Aš čia!’ Jei pašalinsi iš savo tarpo priespaudą, negrasinsi ir nekalbėsi tuštybių, 
\par 10 jei alkstantį pasotinsi kaip pats save ir prispaustajam padėsi, tada tavo šviesa spindės tamsoje ir tavo tamsybė taps šviesa. 
\par 11 Tada Viešpats vedžios tave, pasotins tavo sielą sausros metu ir sustiprins tavo kūną; tu būsi kaip palaistytas sodas, kaip neišsenkantis vandens šaltinis. 
\par 12 Bus atstatyta tavyje tai, kas seniai paversta griuvėsiais. Tu dėsi pamatus kartų kartoms. Tave vadins spragų užtaisytoju ir gatvių tiesėju. 
\par 13 Jei tu savo kojas sulaikysi sabato dieną, nepramogausi mano šventoje dienoje, vadinsi sabatą pasimėgimu, šventa ir garbinga Viešpaties diena, gerbsi Jį, nevaikščiosi savo keliais, neieškosi pramogų ir nekalbėsi tuščiai, 
\par 14 tada tu gėrėsies Viešpačiu, ir Aš tave vesiu ant aukštų kalnų ir leisiu naudotis tavo tėvo Jokūbo paveldu”,­taip kalbėjo Viešpats.



\chapter{59}


\par 1 Viešpaties ranka nesutrumpėjo gelbėti ir Jo ausis neapkurto girdėti, 
\par 2 bet jūsų nusikaltimai atskyrė jus nuo jūsų Dievo ir jūsų nuodėmės paslėpė Jo veidą, kad Jis nebegirdėtų. 
\par 3 Jūsų rankos suteptos krauju ir pirštai nusikaltimu; jūsų lūpos kalba melą ir liežuvis kartoja neteisybę. 
\par 4 Nė vienas neieško teisingumo ir negina tiesos. Jie pasitiki tuštybe ir kalba melus, pradeda vargą ir pagimdo nusikaltimą. 
\par 5 Jie peri gyvačių kiaušinius ir audžia voratinklius. Kas valgo jų kiaušinių, tas miršta. Jei kas tokį kiaušinį sudaužo, iš jo iššoka angis. 
\par 6 Jų audiniai netinka drabužiams, jais neįmanoma prisidengti. Jų darbai yra pikti ir jų rankose smurto veiksmai. 
\par 7 Jų kojos bėga į pikta, jie skuba nekaltą kraują pralieti; jų mintys yra pilnos nedorybės, jų keliuose sunaikinimas ir griuvėsiai. 
\par 8 Jie nepažįsta ramybės kelio, nėra teisingumo jų poelgiuose, jų takai klaidingi; kas jais eina, nepažins ramybės. 
\par 9 Todėl teisingumas yra toli nuo mūsų, teisybė nepasiekia mūsų; mes laukiame šviesos, o čia tamsa, laukiame aušros, o vaikščiojame sutemose. 
\par 10 Mes ieškome kaip akli sienos, suklumpame vidudienį kaip naktį, esame tamsoje lyg mirusieji. 
\par 11 Mes visi riaumojame kaip lokiai, vaitojame liūdnai kaip burkuojantys balandžiai; mes laukiame tiesos, bet jos nėra, laukiame išgelbėjimo, bet jis toli nuo mūsų. 
\par 12 Mūsų neteisybės padaugėjo Tavo akivaizdoje ir mūsų nuodėmės liudija prieš mus; mes žinome savo neteisybes ir mūsų nusikaltimai yra su mumis. 
\par 13 Mes nusidedame ir išsiginame Viešpaties, nusigręžiame ir nesekame paskui savo Dievą; kalbame apie priespaudą ir maištaujame, priimame melus ir jais nuoširdžiai džiaugiamės. 
\par 14 Teisingumas yra atmestas ir teisybė pašalinta; tiesos nebėra miesto aikštėse, bešališkumas negali įeiti. 
\par 15 Tiesos nėra; kas šalinasi pikto, pats yra apiplėšiamas. Viešpats tai matė, ir Jam nepatiko, kad nesilaikome teisingumo. 
\par 16 Viešpats matė ir stebėjosi, kad nebuvo žmogaus, kuris užtartų; Jo ranka Jį išgelbėjo, Jo teisumas Jį palaikė. 
\par 17 Jis užsidėjo teisumą kaip šarvą ir išgelbėjimo šalmą ant savo galvos. Jis apsirengė keršto drabužiais, apsisiautė uolumu kaip apsiaustu. 
\par 18 Jis atlygins savo priešams pagal jų darbus rūstybe ir kerštu. Jis atsilygins saloms ir savo priešams. 
\par 19 Tada vakaruose esantys bijos Viešpaties vardo, o rytuose­matys Jo šlovę. Kai priešas ateis kaip srovė, Viešpaties Dvasia iškels savo vėliavą. 
\par 20 “Atpirkėjas ateis į Sioną ir pas tuos, kurie atsitrauks nuo nusikaltimų Jokūbe”,­sako Viešpats. 
\par 21 “Tai mano sandora su jais,­sako Viešpats.­Mano dvasia, kuri yra ant tavęs, ir mano žodžiai, kuriuos tau daviau, nepaliks tavęs nei tavo vaikų, nei vaikaičių nuo dabar ir per amžius”,­sako Viešpats.



\chapter{60}


\par 1 Kelkis ir šviesk, Jeruzale, nes tavo šviesa ateina ir Viešpaties šlovė tau šviečia. 
\par 2 Štai tamsa padengs žemę ir tautas, bet Viešpats pakils virš tavęs ir Jo šlovė bus matoma tavyje. 
\par 3 Tautos ir karaliai ateis prie tavo šviesos spindesio. 
\par 4 Pakelk savo akis ir apsižvalgyk aplinkui! Jie visi renkasi pas tave: tavo sūnūs ateina iš toli, tavo dukterys atnešamos ant rankų. 
\par 5 Tu tai matysi ir spindėsi iš džiaugsmo; tavo širdis sudrebės ir išsiplės, kai jūros pilnybės gręšis į tave ir pagonių stiprybė ateis pas tave. 
\par 6 Daugybė kupranugarių ateis pas tave iš Midjano, Efos ir Šebos, atnešdami aukso bei smilkalų, ir skelbs didingus Viešpaties darbus. 
\par 7 Visos Kedaro bandos bus surinktos pas tave, taip pat ir Nebajoto avinai. Jie bus aukojami prie aukuro. Aš pašlovinsiu savo šlovės namus. 
\par 8 Kas tie, kurie kaip debesys plaukia, kaip balandžiai skrenda į savo lizdus? 
\par 9 Salos laukia manęs, Taršišo laivai skuba, kad tavo vaikus atgabentų iš toli; sidabrą ir auksą drauge su jais Viešpaties, tavo Dievo, vardui ir Izraelio Šventajam, nes Jis tave pašlovino. 
\par 10 Svetimtaučiai statys tavo sienas, ir jų karaliai tarnaus tau. Savo rūstybėje smogiau tau, bet dėl savo malonės tavęs pasigailėjau. 
\par 11 Tavo vartai bus nuolatos atdari, dieną ir naktį nebus užrakinti, kad pagonių stiprybę tau atgabentų ir jų karalius atvestų. 
\par 12 Tauta ir karalystė, kuri tau netarnaus, žus; tokios tautos bus visai sunaikintos. 
\par 13 Libano šlovė atiteks tau, kiparisai, platanai ir pušys puoš tavo šventąją vietą, Aš pašlovinsiu savo pakojo vietą. 
\par 14 Tavo prispaudėjų ir niekintojų sūnūs ateis nuolankiai nusižeminę pas tave ir puls prie tavo kojų; jie vadins tave Viešpaties miestu, Izraelio Šventojo Sionu. 
\par 15 Tu buvai paliktas ir nekenčiamas taip, kad nė vienas nenorėjo pas tave ateiti. Aš padarysiu tave puikybe amžiams ir džiaugsmu kartų kartoms. 
\par 16 Tu čiulpsi pagonių pieną ir žįsi karalių krūtis. Tuomet tu suprasi, kad Aš, Viešpats, esu tavo gelbėtojas, atpirkėjas ir Jokūbo Galingasis. 
\par 17 Vietoj vario Aš atgabensiu aukso, vietoj geležies­sidabro, vietoj medžių­vario ir vietoj akmenų­ geležies. Tavo valdytojas bus taika ir tavo prižiūrėtojas­teisumas. 
\par 18 Tavo šalyje nebebus smurto ir tavo krašte­sunaikinimo bei griovimo. Tu vadinsi savo miesto sienas išgelbėjimu ir vartus­šlove. 
\par 19 Tau nebereikės saulės dieną ir mėnulio šviesos naktį; Viešpats bus tavo amžina šviesa ir Dievas bus tavo šlovė. 
\par 20 Tavo saulė nebenusileis ir mėnulis nebepasikeis, nes Viešpats bus tavo amžina šviesa­tavo gedulo dienos pasibaigs. 
\par 21 Visi tavo žmonės bus teisūs, jie amžiams paveldės šalį. Jie bus mano pasėlio atžala, mano rankų darbas, kad Aš būčiau pašlovintas. 
\par 22 Mažiausias taps tūkstančiu ir menkas­galinga tauta. Aš, Viešpats, skubiai tai įvykdysiu reikiamu metu.



\chapter{61}


\par 1 Viešpaties Dievo Dvasia ant manęs, nes Viešpats patepė mane skelbti gerą žinią vargšams, Jis siuntė mane gydyti tų, kurių širdys sudužusios, skelbti belaisviams išlaisvinimą ir kaliniams atidaryti kalėjimą; 
\par 2 skelbti Viešpaties malonės metus ir mūsų Dievo keršto dieną, paguosti visus liūdinčius, 
\par 3 įteikti liūdintiems Sione grožybę vietoj pelenų, džiaugsmo aliejaus vietoj gedulo, gyriaus drabužį vietoj apsunkusios dvasios. Juos vadins teisumo medžiais, Viešpaties sodiniais Jo šlovei. 
\par 4 Jie atstatys griuvėsius, kurie seniai buvo paversti dykyne, atnaujins sunaikintus miestus, kurie kartų kartas buvo ištuštėję. 
\par 5 Ateiviai ganys jų bandas, svetimtaučių sūnūs ars laukus ir prižiūrės vynuogynus. 
\par 6 Jūs būsite vadinami Viešpaties kunigais, jus vadins mūsų Dievo tarnais. Jūs naudositės pagonių turtais ir didžiuositės jų šlove. 
\par 7 Už savo gėdą jūs gausite dvigubai. Vietoj paniekos jie džiaugsis ir paveldės savo šalyje dvigubai; jie turės amžiną džiaugsmą. 
\par 8 Nes Aš, Viešpats, mėgstu teisingumą, nekenčiu plėšikavimo. Aš jiems atlyginsiu teisingai ir sudarysiu amžiną sandorą su jais. 
\par 9 Pagonys pažins jų palikuonis, jų atžala bus žinoma tarp tautų. Visi, kurie matys juos, pripažins, kad jie yra Viešpaties palaiminti palikuonys. 
\par 10 Aš labai džiaugsiuos Viešpačiu, mano siela džiūgaus Dieve! Nes Jis apvilko mane išgelbėjimo rūbais, apsiautė teisumo apsiaustu kaip sužadėtinį, kuris užsideda papuošalą ant galvos, kaip sužadėtinę, pasipuošusią papuošalais. 
\par 11 Kaip žemė išaugina želmenis ir kaip darže dygsta tai, kas pasėta, taip Viešpats Dievas įželdins teisumą ir gyrių visų tautų akivaizdoje.



\chapter{62}


\par 1 Aš netylėsiu dėl Siono ir nenurimsiu dėl Jeruzalės, kol pasirodys jų teisumas lyg ryto aušra ir išgelbėjimas lyg degantis žibintas. 
\par 2 Tautos matys tavo teisumą ir karaliai­tavo šlovę. Tave vadins nauju vardu, kurį Viešpats tau duos. 
\par 3 Tu būsi šlovės karūna ir karališkas vainikas Viešpaties, tavo Dievo, rankoje. 
\par 4 Tavęs nebevadins apleistąja ir tavo žemės­dykyne. Tave vadins: “Mano pasimėgimas”, o tavo šalį­ištekėjusiąja, nes Viešpats pamėgo tave, ir šalis bus sutuokta. 
\par 5 Kaip jaunuolis veda mergaitę, taip tavo sūnūs ves tave. Kaip jaunikis džiaugiasi jaunąja, taip tavo Dievas džiaugsis tavimi. 
\par 6 Jeruzale, ant tavo sienų Aš pastatysiu sargus; jie niekada­dieną ir naktį­nenurims. Jūs, kurie tariate Viešpaties vardą, netylėkite, 
\par 7 neduokite Jam poilsio, kol Jis atstatys Jeruzalę ir jos šlovę žemėje. 
\par 8 Viešpats prisiekė savo tvirta ranka: “Aš nebeduosiu tavo grūdų tavo priešams maistui ir svetimi nebegers tavo vyno, kurį pagaminai. 
\par 9 Kas surenka, valgys ir girs Viešpatį; kas užaugina, gers vyną mano šventuose kiemuose”. 
\par 10 Įeikite, įeikite pro vartus, paruoškite kelią tautai; nutieskite, nutieskite kelią, pašalinkite akmenis, pakelkite vėliavą žmonėms. 
\par 11 Viešpats paskelbė iki žemės pakraščių: “Sakykite Siono dukrai: ‘Tavo išgelbėjimas, užmokestis ir atlyginimas artėja’ ”. 
\par 12 Juos vadins: “Šventa tauta, Viešpaties atpirktieji”; tu būsi vadinama: “Ieškotasis, neapleistasis miestas”.



\chapter{63}


\par 1 Kas ateina raudonais drabužiais iš Edomo miesto Bocros, apsirengęs šlovingais rūbais eina savo jėgos pilnatvėje? “Aš, kuris skelbiu teisumą ir turiu galią išgelbėti”. 
\par 2 Kodėl tavo apsiaustas raudonas ir tavo drabužiai tarsi minančių vyno spaustuvą? 
\par 3 “Aš vienas myniau spaustuvą, nė vieno iš žmonių nebuvo su manimi. Aš mindžiojau juos įtūžęs ir trypiau savo rūstybėje. Jų kraujas aptaškė mano drabužius ir sutepė juos. 
\par 4 Keršto diena buvo mano širdyje, ir išgelbėjimo metas atėjo. 
\par 5 Aš apsidairiau­ir nebuvo, kas padėtų, Aš stebėjausi, kad niekas nepalaikė. Todėl mano paties ranka atnešė man išgelbėjimą ir mano įtūžis palaikė mane. 
\par 6 Aš įnirtęs mindžiojau tautas, daviau joms paragauti savo rūstybės ir nubloškiau žemėn jų galybę”. 
\par 7 Atsimenu Viešpaties mums parodytą malonę ir Jo šlovę, ir visa, ką Jis mums suteikė; Viešpats labai gailestingas Izraelio namams. Jis visa tai darė iš savo malonės, pasigailėdamas mūsų. 
\par 8 Viešpats tarė: “Jie yra mano tauta, mano vaikai, kurie nemeluos”. Jis buvo jiems gelbėtoju. 
\par 9 Jis patyrė visą jų vargą ir angelas, esantis Jo akivaizdoje, jiems padėjo. Iš meilės ir pasigailėjimo jiems Jis atpirko juos, pakėlė juos ir nešė per visas praeities dienas. 
\par 10 Jie maištavo ir vargino Jo šventą Dvasią, todėl Jis apsigręžė ir tapo jų priešu, Jis kovojo prieš juos. 
\par 11 Jie atsiminė senas praeities dienas, Jo tarno Mozės laikus, kai Jis vedė savo tautą. Kur yra Tas, kas išvedė iš jūros savo bandą kartu su jų ganytoju? Kur yra Tas, kas davė jam savo šventą Dvasią? 
\par 12 Kas savo galinga dešine vedė Mozę, perskyrė vandenį pirma jų ir įsigijo amžiną vardą? 
\par 13 Kas vedė juos per gelmes lyg žirgą dykumoje, kad jie nesukluptų? 
\par 14 Kaip kaimenė eina į slėnį, taip Viešpaties Dvasia davė jiems poilsį. Taip Tu vedei savo tautą ir įsigijai šlovingą vardą. 
\par 15 Pažvelk iš dangaus, iš savo šventos ir šlovingos buveinės. Kur Tavo uolumas ir galia? Kur Tavo širdies ilgesys ir gailestingumas? Ar jie paliovė? 
\par 16 Tik Tu esi mūsų tėvas. Abraomas nepažįsta mūsų, Izraelis nieko nežino apie mus. Tu, Viešpatie, esi mūsų tėvas, mūsų atpirkėjas. Tavo vardas yra amžinas. 
\par 17 Viešpatie, kodėl leidai mums nuklysti nuo Tavo kelių, sukietinai mūsų širdį, kad Tavęs nebijotume? Sugrįžk dėl savo tarnų, savo paveldėtų giminių. 
\par 18 Tavo šventa tauta tik trumpai čia gyveno; mūsų priešai mindžiojo Tavo šventyklą. 
\par 19 Mes esame tokie, lyg niekad nebūtume Tau priklausę ir Tavo vardu nesivadinę.



\chapter{64}


\par 1 O kad Tu praplėštum dangų ir nužengtum, kad Tavo akivaizdoje kalnai drebėtų! 
\par 2 Kaip ugnis sudegina malkas ir užvirina vandenį, taip Tavo vardas tegul tampa žinomas priešams ir tautos tegul dreba Tavo akivaizdoje. 
\par 3 Tu padarei baisių dalykų, kurių mes nelaukėme. Tu nužengei, ir kalnai drebėjo Tavo akivaizdoje. 
\par 4 Nuo amžių ausis negirdėjo ir akis neregėjo kito dievo, be Tavęs, kuris tiek padarytų Jo laukiantiems. 
\par 5 Tu sutinki tą, kuris su džiaugsmu elgiasi teisiai, atsimena Tave ir Tavo kelius. Tu buvai užsirūstinęs, nes mes nuolat nusidėdavome. Ar dar išgelbėsi mus? 
\par 6 Mes visi esame kaip nešvarūs, mūsų teisumas kaip purvini skarmalai. Mes visi vystame kaip lapai, mūsų piktadarystės blaško mus kaip vėjas. 
\par 7 Nė vienas nesišaukia Tavo vardo, nepakyla, kad įsikibtų į Tave. Tu paslėpei savo veidą nuo mūsų, palikai mus dėl mūsų nusikaltimų. 
\par 8 Viešpatie, Tu esi mūsų tėvas; mes­molis, o Tu­puodžius; mes visi­Tavo rankų darbas. 
\par 9 Viešpatie, nerūstauk, neminėk amžinai mūsų kalčių. Pažvelk į mus, maldaujame Tavęs, nes mes esame Tavo tauta. 
\par 10 Tavo šventieji miestai sunaikinti. Sionas paverstas dykuma, Jeruzalė sugriauta. 
\par 11 Mūsų šventi ir didingi namai, kuriuose mūsų tėvai girdavo Tave, sudeginti. Visa, ką mes mėgome, tapo griuvėsiais. 
\par 12 Ar dėl viso to dar susilaikysi, Viešpatie? Ar dar tylėsi ir varginsi mus be saiko?



\chapter{65}


\par 1 “Manęs ieško tie, kurie apie mane neklausinėjo, ir suranda tie, kurie manęs neieškojo. Aš sakau: ‘Aš čia, Aš čia!’ tautai, kuri nesivadino mano vardu. 
\par 2 Visą dieną laikiau ištiestas rankas į maištaujančią tautą, kuri eina neteisingu keliu, sekdama savo mintis. 
\par 3 Tai tauta, kuri mane nuolat rūstina. Ji aukoja aukas soduose ir degina smilkalus ant aukurų iš plytų. 
\par 4 Jie pasilieka kapinėse ir miega olose; jie valgo kiaulieną, ir bjaurus viralas yra jų puoduose. 
\par 5 Jie sako: ‘Nesiartink prie manęs, aš esu šventesnis už tave’. Jie yra dūmai mano nosyje, ugnis, deganti visą dieną. 
\par 6 Tai užrašyta mano akivaizdoje; Aš netylėsiu, bet atlyginsiu. Atlyginsiu jiems į antį 
\par 7 už jūsų kaltes ir kaltes jūsų tėvų­sako Viešpats,­kurie aukojo kalnuose ir mane niekino aukštumose; iki galo jiems atseikėsiu už jų darbus”. 
\par 8 Taip sako Viešpats: “Apie vynuogių kekę sakoma: ‘Nesunaikink jos, nes joje yra palaiminimas’, taip Aš darysiu su savo tarnais ir jų visų nesunaikinsiu. 
\par 9 Jokūbo ir Judo palikuonys paveldės mano kalnus, mano išrinktieji ir mano tarnai apgyvendins juos. 
\par 10 Šarono lyguma bus ganykla avims ir Achoro slėnis­vieta sugulti bandai, mano tautai, kuri ieškojo manęs. 
\par 11 Bet jums, kurie paliekate Viešpatį, užmirštate mano šventąjį kalną, dengiate stalą laimės deivei ir liejate geriamąją auką lemties dievui, 
\par 12 Aš lemsiu kardą, ir jūs nusilenksite išžudymui. Nes kai Aš šaukiau, jūs neatsiliepėte, kai kalbėjau, neklausėte; jūs darėte pikta mano akivaizdoje ir pasirinkote, kas man nepatiko”. 
\par 13 Todėl Viešpats Dievas taip sako: “Štai mano tarnai valgys, o jūs alksite, mano tarnai gers, o jūs trokšite, mano tarnai džiūgaus, o jūs būsite sugėdinti, 
\par 14 mano tarnai giedos iš širdies džiaugsmo, o jūs verksite iš širdies skausmo ir dejuosite iš dvasios suspaudimo. 
\par 15 Mano išrinktieji tars jūsų vardą kaip keiksmažodį, nes Viešpats Dievas nužudys tave, o savo tarnus pavadins kitu vardu. 
\par 16 Kas laimins krašte, laimins tiesos Dievo vardu, ir kas prisieks šalyje, prisieks tiesos Dievo vardu, nes senoji priespauda bus užmiršta ir nebeminima mano akivaizdoje. 
\par 17 Štai Aš kuriu naują dangų ir naują žemę. Senųjų nebeatsimins ir apie juos nebegalvos. 
\par 18 Jūs džiaugsitės ir džiūgausite amžinai tuo, ką Aš sukursiu, nes Jeruzalę sukursiu džiūgauti ir jos žmones džiaugsmui. 
\par 19 Aš gėrėsiuos Jeruzale ir džiaugsiuosi savo tauta; nebesigirdės joje verksmo nė dejonių. 
\par 20 Nebus joje kūdikio, kuris išgyventų kelias dienas, nė seno žmogaus, kuris nesulauktų dienų pilnatvės; nes vaikas mirs, nugyvenęs šimtą metų, bet šimtametis nusidėjėlis bus prakeiktas. 
\par 21 Jie statys namus ir gyvens juose, įsiveis vynuogynų ir valgys jų vaisių. 
\par 22 Jie nieko nestatys, kas atitektų kitiems. Jie nesodins, kad kiti suvalgytų. Mano žmonių amžius prilygs medžio amžiui, gyvendami jie ilgai džiaugsis savo rankų darbais. 
\par 23 Jie nedirbs veltui ir negimdys vargui; jie ir jų vaikai bus Viešpaties palaimintųjų palikuonys. 
\par 24 Prieš jiems šaukiant, Aš atsakysiu; jiems dar tebekalbant, Aš išgirsiu. 
\par 25 Vilkas ir avinėlis ganysis drauge, liūtas ės šiaudus kaip jautis ir gyvatė maitinsis dulkėmis. Jie nekenks ir nenaikins mano šventajame kalne,­sako Viešpats”.



\chapter{66}


\par 1 Taip sako Viešpats: “Dangus yra mano sostas ir žemė­mano pakojis. Kur yra namai, kuriuos jūs man norite statyti, ir kur mano poilsio vieta? 
\par 2 Mano ranka visa tai sukūrė, ir taip visa atsirado,­sako Viešpats.­Aš pažvelgsiu į žmogų, kuris yra vargšas bei turi atgailaujančią dvasią ir dreba prieš mano žodį. 
\par 3 Pjaunantis jautį yra kaip žudantis žmogų, aukojantis avį­kaip nusukantis sprandą šuniui, aukojantis duonos auką­kaip aukojantis kiaulės kraują, deginantis smilkalus­kaip garbinantis stabą. Taip jie pasirinko savo kelius, jų siela mėgsta jų bjaurystes. 
\par 4 Tad ir Aš parinksiu ir užleisiu ant jų vargą, kurio jie bijo. Nes Aš šaukiau, bet nė vienas neatsiliepė, Aš kalbėjau, bet jie nesiklausė. Jie darė mano akivaizdoje pikta ir pasirinko, kas man nepatinka”. 
\par 5 Išgirskite, ką sako Viešpats, jūs, kurie drebate prieš Jo žodį: “Jūsų broliai, kurie neapkenčia jūsų ir jus atmeta dėl mano vardo, sako: ‘Teapreiškia Viešpats savo šlovę, kad matytume jūsų džiaugsmą!’ Bet jie bus sugėdinti”. 
\par 6 Tai balsas mieste! Tai balsas iš šventyklos! Tai Viešpaties balsas, kai Jis atlygina savo priešams. 
\par 7 Dar skausmų nepajutus, ji pagimdė, dar skausmams neprasidėjus, pagimdė sūnų. 
\par 8 Kas tai girdėjo, ar ką panašaus matė? Ar gali kraštas užgimti per vieną dieną? Ar gali tauta atsirasti per vieną akimirką? Tik skausmams prasidėjus, Sionas pagimdė savo vaikus. 
\par 9 “Argi Aš, atvedęs iki gimdymo, neleisiu pagimdyti?”­sako Viešpats. “Argi Aš, leidęs pradėti gimdymą, uždarysiu įsčias?”­sako tavo Dievas. 
\par 10 Džiaukitės kartu su Jeruzale, kurie ją mylite. Džiaukitės, kurie liūdėjote dėl jos. 
\par 11 Kad galėtumėte maitintis ir pasisotinti jos paguodos krūtimis, kad gaivintumėtės ir mėgautumėtės jos šlovės gausumu. 
\par 12 Taip sako Viešpats: “Aš užliesiu ją ramybe kaip upe, tautų lobiai plauks į ją kaip neišsenkanti srovė. Jūs būsite kaip kūdikiai maitinami, ant rankų nešiojami ir supami ant jos kelių. 
\par 13 Kaip motina paguodžia kūdikį, taip Aš jus paguosiu; Jeruzalėje jūs būsite paguosti”. 
\par 14 Jūs tai matysite ir jūsų širdis džiaugsis, jūsų kaulai žaliuos kaip jauna žolė. Viešpaties ranka bus apreikšta Jo tarnams, o Jo rūstybė­priešams. 
\par 15 Viešpats ateis su ugnimi, Jo vežimai kaip viesulas, kad išlietų savo rūstybę ir įtūžį, nubaustų ugnies liepsnomis. 
\par 16 Ugnimi ir kardu Viešpats padarys teismą kiekvienam kūnui. Viešpaties užmuštųjų bus daug. 
\par 17 “Visi, kurie pasišvenčia ir apsivalo soduose, valgo kiaulieną, peles ir kitus pasibjaurėtinus dalykus, bus sunaikinti”,­sako Viešpats. 
\par 18 “Aš žinau jūsų darbus ir mintis; surinksiu įvairių kalbų tautas, ir jos matys mano šlovę. 
\par 19 Aš duosiu ženklą jiems ir pasiųsiu dalį išlikusiųjų į visas tautas: į Taršišą, Pulą ir Ludą, pas šaudančius strėlėmis, į Tubalą ir Javaną, į tolimas salas, kuriose negirdėjo mano darbų ir nematė mano šlovės. Jie paskelbs mano šlovę tautoms. 
\par 20 Jie atgabens jūsų brolius iš visų tautų, kaip yra atgabenamos aukos, ant žirgų, vežimuose, neštuvuose, ant mulų ir kupranugarių į šventąjį Jeruzalės kalną, kaip Izraelio sūnūs atneša duonos auką į Viešpaties namus švariame inde”,­sako Viešpats. 
\par 21 “Kai kuriuos iš jų padarysiu kunigais ir levitais”,­sako Viešpats. 
\par 22 “Kaip nauji dangūs ir nauja žemė, kuriuos sukursiu, pasiliks mano akivaizdoje, taip jūsų palikuonys ir vardas išliks”,­sako Viešpats. 
\par 23 “Nuo vieno jauno mėnulio iki kito ir nuo vieno sabato iki kito visi ateis ir parpuls mano akivaizdoje,­sako Viešpats.­ 
\par 24 Jie išeis ir matys lavonus žmonių, kurie maištavo prieš mane. Jų kirminas nemirs ir ugnis neužges. Jie bus pasibaisėjimu visai žmonijai”.




\end{document}