\begin{document}

\title{Jeremijo knyga}


\chapter{1}


\par 1 Žodžiai Jeremijo, kunigo Hilkijo sūnaus iš Anatoto, iš Benjamino žemių. 
\par 2 Viešpats kalbėjo Jeremijui tryliktaisiais Jozijo, Amono sūnaus, Judo karaliaus, viešpatavimo metais; 
\par 3 taip pat viešpataujant Jozijo sūnui Jehojakimui ir iki Zedekijo, Jozijo sūnaus, Judo karaliaus, vienuoliktųjų metų penkto mėnesio, tai yra Jeruzalės išvedimo į nelaisvę. 
\par 4 Viešpats kalbėjo man, sakydamas: 
\par 5 “Pažinau tave prieš tau gimstant, pašventinau tave ir paskyriau pranašu tautoms”. 
\par 6 Aš atsakiau: “Ak, Viešpatie Dieve, aš nemoku kalbėti, nes esu tik vaikas”. 
\par 7 Bet Viešpats man sakė: “Nesakyk: ‘Aš esu vaikas’, bet eik, pas ką tave siųsiu, ir kalbėk, ką tau liepsiu. 
\par 8 Nebijok jų, nes Aš esu su tavimi ir tave išgelbėsiu,­sako Viešpats”. 
\par 9 Tada Viešpats ištiesė savo ranką ir, palietęs mano lūpas, tarė man: “Štai Aš įdėjau savo žodžius į tavo lūpas. 
\par 10 Šiandien paskyriau tave virš tautų ir karalysčių rauti, griauti, nuversti, naikinti, statyti ir sodinti”. 
\par 11 Viešpats klausė manęs: “Ką matai, Jeremijau?” Aš atsakiau: “Matau migdolo šaką”. 
\par 12 Viešpats atsakė: “Gerai matei, nes Aš budėsiu, kad mano žodis būtų įvykdytas”. 
\par 13 Viešpats klausė manęs antrą kartą: “Ką matai?” Aš atsakiau: “Matau verdantį katilą, krypstantį iš šiaurės”. 
\par 14 Viešpats atsakė: “Iš šiaurės ateis nelaimė ant visų šalies gyventojų. 
\par 15 Aš pašauksiu visas šiaurės karalystes, ir jos ateis, ir kiekviena pastatys savo sostą prie Jeruzalės vartų, prieš visas jos sienas iš visų pusių ir prieš visus Judo miestus. 
\par 16 Tada įvykdysiu teismo sprendimą dėl visų jų nusikaltimų: jie paliko mane, aukojo svetimiems dievams ir garbino savo rankų darbus. 
\par 17 O tu susijuosk, kelkis ir kalbėk jiems, ką tau liepsiu. Nenusigąsk jų, kad neišgąsdinčiau tavęs jų akivaizdoje. 
\par 18 Aš šiandien padariau tave sustiprintu miestu, geležiniu stulpu ir varine siena prieš visą šalį: prieš Judo karalius, kunigaikščius, kunigus ir visą tautą. 
\par 19 Jie kovos prieš tave, bet nenugalės, nes Aš esu su tavimi ir tave išgelbėsiu,­sako Viešpats”.


\chapter{2}


\par 1 Viešpats kalbėjo: 
\par 2 “Eik ir šauk Jeruzalei, sakydamas: ‘Taip sako Viešpats: ‘Aš atsimenu tavo jaunystės ištikimybę, tavo meilę, kai buvai sužadėtinė, kaip tu mane sekei dykumoje, neapsėtoje žemėje; 
\par 3 Izraelis buvo šventas Viešpačiui, Jo derliaus pirmasis vaisius. Kas jį skriaudė, nusikalto, nelaimė juos ištiko’. 
\par 4 Klausykite Viešpaties žodžio, Jokūbo namai ir visos Izraelio giminės: 
\par 5 ‘Kokią neteisybę rado manyje jūsų tėvai, kad jie atitolo nuo manęs, sekė tuštybę ir patys tapo tušti. 
\par 6 Jie neklausė: ‘Kur yra Viešpats, kuris mus išvedė iš Egipto šalies, vedė dykumoje, stepių ir daubų krašte, išdžiūvusioje ir tamsioje žemėje; krašte, per kurį niekas nekeliauja ir kuriame joks žmogus negyvena?’ 
\par 7 Juk Aš jus atvedžiau į derlingą šalį ir leidau naudotis jos vaisiais ir gėrybėmis. Bet jūs atėję sutepėte mano kraštą, mano nuosavybę padarėte bjaurią. 
\par 8 Kunigai neklausė: ‘Kur yra Viešpats?’, įstatymo sargai nepažino manęs, ganytojai sukilo prieš mane, pranašai pranašavo Baalo vardu ir sekė tai, kas neturi vertės. 
\par 9 Todėl bylinėsiuosi su jumis ir jūsų vaikų vaikais. 
\par 10 Eikite į Kitimų salas ir pažiūrėkite arba pasiųskite į Kedarą ir atidžiai ieškokite, ar ten yra kas nors panašaus? 
\par 11 Ar yra kuri tauta pakeitusi savo dievus, kurie nėra dievai? Bet mano tauta pakeitė savo šlovę į tai, kas yra be vertės. 
\par 12 Stebėkitės dėl to, dangūs, išsigąskite ir būkite sukrėsti. 
\par 13 Dvi piktybes padarė mano tauta: ji paliko mane, gyvojo vandens šaltinį, ir išsikasė skylėtų šulinių, kurie nesulaiko vandens. 
\par 14 Argi Izraelis vergas? Argi jis gimęs vergu? Kodėl jis tapo grobiu? 
\par 15 Liūtai riaumojo prieš jį ir staugė. Jie pavertė šalį dykyne, miestai sudeginti ir be gyventojų. 
\par 16 Nofo ir Tachpanheso sūnūs sudaužė tavo galvos karūną. 
\par 17 Argi ne pats tai sau užsitraukei, palikdamas Viešpatį, savo Dievą, kai Jis tave vedė keliu? 
\par 18 O dabar kodėl bėgi į Egiptą Šihoro vandens gerti ar į Asiriją gerti upės vandens? 
\par 19 Tavo paties nedorybė nubaus tave ir tavo nuklydimas pamokys tave. Todėl pažink ir pamatyk, kaip negera ir kartu yra tai, kad tu palikai Viešpatį ir nėra tavyje mano baimės’,­sako Viešpats, kareivijų Dievas. 
\par 20 ‘Aš jau seniai sulaužiau tavo jungą, sutraukiau pančius, ir tu sakei: ‘Nenusikalsiu’. Tačiau kiekvienoje aukštoje kalvoje ir po kiekvienu žaliu medžiu tu klaidžiojai paleistuvaudamas. 
\par 21 Aš tave pasodinau kaip taurų vynmedį, tikrą želmenį. Kaip tu išsigimei į laukinį vynmedį? 
\par 22 Nors ir nusipraustum šarmu ir muilu, tačiau liksi suteptas savo kalte mano akivaizdoje’,­sako Viešpats Dievas. 
\par 23 ‘Kaip tu gali sakyti: ‘Aš nesusitepiau, nesekiojau paskui Baalą’? Pažvelk į savo kelius slėnyje, pagalvok, ką darei. Tu bėgi kaip laukinė kupranugarė, pasileidusi savo keliais. 
\par 24 Laukinė asilė, pripratusi prie dykumos, geisdama uosto orą. Kas gali ją sulaikyti? Jos ieškantieji nenuvargs, bet suras ją jos mėnesį. 
\par 25 Saugok savo koją, kad nebūtų basa, ir savo gerklę nuo troškulio. Bet tu sakai: ‘Nėra vilties. Aš myliu svetimuosius ir seksiu paskui juos’. 
\par 26 Kaip gėdisi vagis, kai jį pagauna, taip bus sugėdintas Izraelis: karaliai, kunigaikščiai, kunigai ir pranašai, 
\par 27 kurie sako medžiui: ‘Tu mano tėvas’, ir akmeniui: ‘Tu mane pagimdei’. Jie atgręžė man nugarą. Bet savo nelaimės metu jie sako: ‘Kelkis ir gelbėk mus’. 
\par 28 Kurgi tavo dievai, kuriuos pasidarei? Jie teatsikelia ir, jei gali, tepadeda tau nelaimės metu. Judai, kiek tavo miestų, tiek dievų. 
\par 29 Ar jūs bylinėsitės su manimi? Jūs visi nusikaltote man’,­sako Viešpats. 
\par 30 ‘Veltui Aš baudžiau jūsų vaikus, jie nepriėmė pamokymo. Jūsų pačių kardas prarijo pranašus kaip draskąs liūtas. 
\par 31 O karta, stebėkite Viešpaties žodį. Argi Aš buvau dykuma Izraeliui arba tamsi šalis? Kodėl mano tauta sako: ‘Mes esame viešpačiai, nebegrįšime pas Tave’? 
\par 32 Ar mergaitė pamiršta savo papuošalus, nuotaka­savo apdarą? Bet mano tauta jau seniai pamiršo mane. 
\par 33 Kaip puikiai tu moki ieškoti meilės! Tu išmokei nedoras moteris savo kelių. 
\par 34 Ant tavo drabužių kraštų randamas nekaltai nužudytųjų kraujas. Jo netgi nereikia ieškoti, jis aiškiai matomas ant tavęs. 
\par 35 Tačiau tu sakai: ‘Aš nekalta, Jo rūstybė nusigręš nuo manęs’. Aš apkaltinsiu tave, nes sakei: ‘Nenusidėjau’. 
\par 36 Ko tu blaškaisi, keisdama savo kelius? Ir Egiptas apvils tave, kaip Asirija apvylė. 
\par 37 Iš ten tu irgi išeisi, užsidėjusi rankas ant galvos, nes Viešpats atmetė tuos, kuriais pasitikėjai, ir nieko iš jų nelaimėsi’ ”.



\chapter{3}


\par 1 “Jie sako: ‘Jei vyras paleidžia žmoną ir ji, atsiskyrusi nuo jo, išteka už kito vyro, ar ji gali sugrįžti pas jį? Argi tai nesuteptų krašto?’ O tu paleistuvavai su daugeliu meilužių, tačiau sugrįžti pas mane,­sako Viešpats.­ 
\par 2 Pakelk savo akis į aukštumas ir pažvelk, kur tik tu nesi paleistuvavusi? Pakelėse tu sėdėjai laukdama kaip arabas dykumoje. Tu sutepei šalį savo paleistuvystėmis ir nedorybėmis. 
\par 3 Nebuvo ankstyvojo nė vėlyvojo lietaus. Bet tu turėjai paleistuvės kaktą ir nesigėdijai. 
\par 4 Dabar tu šauki: ‘Mano tėve! Mano jaunystės vadove! 
\par 5 Ar Tu pyksi amžinai ir visados rūstausi?’ Tu darei ir kalbėjai pikta, kaip tik galėjai”. 
\par 6 Viešpats man tarė karaliaus Jozijo dienomis: “Ar matei, ką darė nuklydusi Izraelio tauta? Ji paleistuvavo ant kiekvieno aukšto kalno ir po kiekvienu žaliuojančiu medžiu. 
\par 7 Aš sakiau po viso to: ‘Sugrįžk pas mane’, bet ji nesugrįžo. Neištikimoji jos sesuo, Judo tauta, matė, 
\par 8 kad Aš ją atstūmiau ir daviau jai skyrybų raštą, nes Izraelio tauta nuklydusi svetimavo. Tačiau neištikimoji jos sesuo, Judo tauta, nepabūgo, bet taip pat nuėjusi paleistuvavo. 
\par 9 Savo paleistuvystėmis ji sutepė kraštą ir svetimavo su akmenimis ir medžiais. 
\par 10 Tačiau, nors visa tai įvyko, neištikimoji jos sesuo, Judo tauta, taip pat nesugrįžo pas mane visa širdimi, bet tik veidmainiavo,­sako Viešpats”. 
\par 11 Viešpats kalbėjo man: “Nuklydusi Izraelio tauta buvo teisesnė už neištikimąją Judo tautą. 
\par 12 Eik šiaurės link, skelbdamas: ‘Sugrįžk, nuklydusi Izraelio tauta. Aš neberūstausiu, nes esu gailestingas, nepyksiu ant jūsų amžinai. 
\par 13 Prisipažink, kad Viešpačiui, savo Dievui, buvai neištikima ir nusikaltai, svetimaudama po žaliuojančiais medžiais ir mano balso neklausydama,­sako Viešpats.­ 
\par 14 Sugrįžkite, nuklydę vaikai, nes Aš esu jus vedęs. Aš paimsiu jus po vieną iš miesto ir po du iš giminės ir atvesiu į Sioną. 
\par 15 Aš jums duosiu ganytojų pagal savo širdį: jie jus ganys išmintingai ir sumaniai. 
\par 16 Tuomet jūs tapsite gausia tauta ir niekas nebeklaus, kur yra Viešpaties Sandoros skrynia. Ji niekam neberūpės, jos nei prisimins, nei pasiges ir naujos nebedarys. 
\par 17 Tuomet Jeruzalę vadins Viešpaties sostu ir į ją susirinks visos tautos prie Viešpaties vardo Jeruzalėje. Jie nebevaikščios pagal savo piktos širdies sumanymus. 
\par 18 Judas su Izraeliu ateis iš šiaurės krašto į šalį, kurią daviau jūsų tėvams paveldėti. 
\par 19 Galvojau padaryti tave sūnumi ir duoti tau gražią šalį, gerą paveldą tarp tautų. Tikėjausi, kad tu mane vadinsi savo tėvu ir seksi paskui mane visados. 
\par 20 Tačiau kaip moteris sulaužo ištikimybę savo vyrui, taip jūs, izraelitai, sulaužėte ištikimybę man,­ sako Viešpats.­ 
\par 21 Aukštumose girdimas Izraelio vaikų maldavimas ir verksmas; jie elgėsi neištikimai, pamiršo Viešpatį, savo Dievą. 
\par 22 Sugrįžkite, nuklydusieji vaikai. Aš atleisiu jums’ ”. Mes ateiname pas Tave, nes Tu esi Viešpats, mūsų Dievas. 
\par 23 Iš tiesų veltui mes vylėmės aukštumomis ir kalnais.Viešpatyje, mūsų Dieve, yra Izraelio išgelbėjimas. 
\par 24 Gėda prarijo mūsų tėvų jaunystės triūsą: avis, galvijus, sūnus ir dukteris. 
\par 25 Gėdykimės ir prisipažinkime nusidėję Viešpačiui, savo Dievui; mes ir mūsų tėvai nuo jaunystės iki šios dienos neklausėme Viešpaties, savo Dievo.



\chapter{4}


\par 1 “Izraeli, jei nori sugrįžti,­sako Viešpats,­grįžk pas mane. Jei pašalinsi savo bjaurystes, tau nereikės bėgti nuo manęs. 
\par 2 Tada prisieksi: ‘Kaip gyvas Viešpats!’ tiesoje, teisingume ir teisume. Ir Jis bus palaiminimas tautoms, ir tautos girsis Juo”. 
\par 3 Taip sako Viešpats Judo žmonėms ir Jeruzalei: “Arkite dirvonus ir nesėkite erškėtynuose. 
\par 4 Apsipjaustykite Viešpačiui ir pašalinkite nuo savo širdžių plėvelę, Judo žmonės ir Jeruzalės gyventojai, kad neišsiveržtų kaip ugnis mano rūstybė ir neužsidegtų neužgesinamai dėl jūsų piktų darbų. 
\par 5 Skelbkite Jude ir praneškite Jeruzalėje, pūskite ragą krašte, garsiai šaukite ir sakykite: ‘Susirinkite ir eikime į sutvirtintus miestus!’ 
\par 6 Iškelkite vėliavą Siono link! Bėkite! Nestovėkite! Nelaimę ir sunaikinimą Aš užleisiu iš šiaurės. 
\par 7 Liūtas pakilo iš tankumyno, tautų naikintojas yra kelyje. Jis pakilo iš savo vietos paversti tavo šalį dykuma; tavo miestai pavirs griuvėsiais ir ištuštės. 
\par 8 Todėl apsivilkite ašutinėmis, raudokite ir dejuokite: ‘Nenusigręžė nuo mūsų Viešpaties rūstybė’ ”. 
\par 9 Viešpats sako: “Ateis diena, kada pradings karaliaus drąsa ir kunigaikščių narsumas, kunigai išsigąs ir pranašai pastirs iš baimės. 
\par 10 Jie sakys: ‘Ak, Viešpatie Dieve, iš tikrųjų Tu smarkiai apgavai šią tautą ir Jeruzalę, sakydamas: ‘Jūs turėsite taiką’, o dabar kardas siekia mūsų sielas!’ 
\par 11 Tuomet bus pasakyta šiai tautai ir Jeruzalei: ‘Karštas vėjas pučia nuo plikų aukštumų iš dykumos mano tautos link, bet ne vėtyti ir ne valyti jos. 
\par 12 Smarkus vėjas iš ten ateina pas mane. Dabar Aš pats paskelbsiu jiems nuosprendį’ ”. 
\par 13 Kaip debesis Jis pakyla, kaip audra Jo kovos vežimai, greitesni už erelius Jo žirgai! Vargas mums, mes žuvę! 
\par 14 Jeruzale, apvalyk nuo nedorybių savo širdį, kad būtum išgelbėta! Ar ilgai piktos mintys pasiliks tavyje? 
\par 15 Klausykis, iš Dano ir iš Efraimo kalnyno balsas skelbia blogą žinią: 
\par 16 “Įspėkite tautas, praneškite Jeruzalei: priešai ateina iš tolimos šalies ir pakels balsą prieš Judo miestus! 
\par 17 Kaip lauko sargai jie apstojo ją, nes ji buvo sukilusi prieš mane,­sako Viešpats.­ 
\par 18 Tavo keliai ir darbai tau tai užtraukė! Tavo nedorybė yra karti ir pasiekė tavo širdį”. 
\par 19 Ak, mano siela, mano siela! Skausmas pasiekė mano širdį. Mano širdis nerimsta, negaliu tylėti. Nes tu, mano siela, girdi trimito garsą, karo pavojų. 
\par 20 Sunaikinimas po sunaikinimo! Visa šalis nuniokota! Staiga sunaikinamos mano pastogės, ūmai­ mano palapinės! 
\par 21 Ar ilgai matysiu vėliavas, girdėsiu trimito garsą? 
\par 22 Mano tauta kvaila, ji manęs nepažįsta. Jie neprotingi vaikai, neturintys supratimo. Jie išmintingi daryti pikta, bet daryti gera jie nesugeba. 
\par 23 Aš pažvelgiau į žemę­ji buvo be pavidalo ir tuščia, pažvelgiau į dangus­ten nebuvo šviesos. 
\par 24 Pažvelgiau į kalnus­jie drebėjo, visos aukštumos siūbavo. 
\par 25 Aš žvalgiaus, ir nebuvo nė vieno žmogaus, visi padangių paukščiai buvo nuskridę. 
\par 26 Pažvelgiau į derlingą žemę­ji buvo dykuma, visi miestai buvo sunaikinti Viešpaties akivaizdoje, Jo rūstybės įkarštyje. 
\par 27 Viešpats tarė: “Visa šalis liks tuščia, tačiau jos nesunaikinsiu visiškai. 
\par 28 Dėl to gedės žemė ir aptems aukštai dangūs. Aš tai pasakiau ir nesigailėsiu, nutariau ir įvykdysiu!” 
\par 29 Nuo raitelių ir šaulių šauksmo pabėgs visas miestas: jie sulįs į olas, pasislėps tankynėse ir lips ant uolų. Miestai ištuštės ir liks be gyventojų. 
\par 30 Ką dabar darysi tu, apiplėštoji? Nors apsivilktum purpuru, pasidabintum auksiniais papuošalais, išsidažytum veidą­veltui tu puoštumeisi. Tavo meilužiai paniekins tave, jie ieškos tavo gyvybės. 
\par 31 Girdžiu tarsi gimdančios riksmą, tarsi pirmagimį gimdančios šauksmą­balsą Siono dukters, gaudančios kvapą, tiesiančios rankas ir sakančios: “Vargas man, mano sielą nuvargino žudikai”.



\chapter{5}


\par 1 Pereikite Jeruzalės gatves, žiūrėkite ir stebėkite, ieškokite aikštėse; jei rasite nors vieną, kuris elgiasi teisingai ir siekia tiesos, tada Aš jai atleisiu. 
\par 2 Nors jie sako: “Kaip gyvas Viešpats”, iš tikrųjų jie melagingai prisiekia. 
\par 3 Viešpatie, argi tavo akys nenukreiptos į tiesą? Tu baudei juos, bet jie neatgailavo; prispaudei juos, bet jie nepriėmė pataisymo. Jie pasidarė kietesni už uolą, atsisakė atsiversti. 
\par 4 Todėl aš sakiau: “Tai vargšai; jie kvaili, nes nežino Viešpaties kelių, nepažįsta Dievo įstatymo. 
\par 5 Eisiu pas kilminguosius ir su jais kalbėsiu, nes jie žino Viešpaties kelius ir Jo įstatymą”. Tačiau jie taip pat sulaužė jungą, sutraukė pančius. 
\par 6 Todėl sudraskys juos liūtas, suplėšys stepių vilkas. Leopardas tykos prie jų miestų; kas tik išeis, bus sudraskytas, nes gausu jų nusikaltimų, jų paklydimas didelis. 
\par 7 “Kaip Aš galėčiau tau atleisti? Tavo vaikai paliko mane ir prisiekė tuo, kas nėra dievai. Aš juos pasotinau, o jie svetimavo ir lankė paleistuvių namus. 
\par 8 Jie kaip nupenėti eržilai geidė savo artimo žmonos. 
\par 9 Ar neturėčiau tokių nubausti?­ sako Viešpats.­Ar šitokiai tautai neturėčiau atkeršyti? 
\par 10 Eikite į vynuogynus ir naikinkite, bet ne iki galo. Pašalinkite atžalas, nes jos nepriklauso Viešpačiui! 
\par 11 Labai neištikimi buvo Izraelis ir Judas”,­sako Viešpats. 
\par 12 Jie išsigynė Viešpaties ir sakė: “Tai ne Jis. Nesulauksime nelaimės, nematysime nei kardo, nei bado”. 
\par 13 Pranašai taps vėjais, jie neturės žodžių­taip jiems atsitiks. 
\par 14 Todėl Viešpats, kareivijų Dievas, sako: “Kaip jie kalbėjo, taip jiems įvyks. Aš padarysiu savo žodžius tavo burnoje ugnimi, o šitą tautą­malkomis, kurias ugnis sudegins. 
\par 15 ‘Izraeli, Aš atvesiu prieš jus tautą iš toli,­sako Viešpats,­galingą tautą, labai seną tautą, kurios kalbos tu nemoki ir nesupranti, ką ji kalba. 
\par 16 Jų strėlinė kaip atviras kapas; jie visi yra karžygiai! 
\par 17 Jie suvalgys tavo derlių ir duoną, skirtą tavo sūnums ir dukterims, ir tavo avis bei galvijus; nurinks tavo vynmedžius ir figmedžius; jie sunaikins sustiprintus miestus, kuriais pasitikėjai. 
\par 18 Tačiau tada nesunaikinsiu jūsų visų’. 
\par 19 Kai jie klaus: ‘Už ką Viešpats, mūsų Dievas, mums taip padarė?’, tu jiems atsakysi: ‘Kaip jūs palikote mane ir tarnavote svetimiems dievams savo krašte, taip tarnausite svetimiesiems ne savo šalyje’ ”. 
\par 20 Skelbkite Jokūbo namams ir praneškite Judui: 
\par 21 “Klausykitės, kvaili ir neprotingi žmonės, kurie turite akis, bet nematote, kurie turite ausis, bet negirdite! 
\par 22 Argi nebijote manęs?­sako Viešpats.­Argi nedrebėsite prieš mane? Aš sulaikiau jūrą smėlio riba kaip amžina užtvara, kurios ji neperžengs, nors ir siaus. Nors bangos ir daužytų ją, nepralauš jos. 
\par 23 Bet šita tauta yra užsispyrusi ir kietos širdies, ji pasitraukė ir nuėjo. 
\par 24 Ji nepagalvojo: ‘Reikia bijotis Viešpaties, savo Dievo, kuris tinkamu laiku duoda ankstyvą ir vėlyvą lietų ir išsaugo mums mūsų derlių!’ 
\par 25 Jūsų kaltės nukreipė tai, ir nuodėmės patraukė nuo jūsų gėrybes. 
\par 26 Mano tautoje yra nedorėlių, kurie susilenkę tyko kaip paukštgaudžiai, spendžia žabangus žmonėms pagauti. 
\par 27 Kaip krepšys pilnas paukščių, taip jų namai pilni apgaulės; tokiu būdu jie tapo žymūs ir pralobo. 
\par 28 Jie taip nutuko, kad net blizga, jų nedorybei nėra ribų. Jie negina našlaičių teisme, tačiau klesti. Jie nežiūri vargšo teisių. 
\par 29 Ar neturėčiau tokių bausti?­sako Viešpats.­Ar šitokiai tautai neturėčiau atkeršyti? 
\par 30 Siaubingų ir bjaurių dalykų vyksta šalyje: 
\par 31 pranašai pranašauja melus, kunigai moko, kaip jiems naudinga, o mano tauta tai mėgsta! Bet ką darysite, kai ateis galas?”



\chapter{6}


\par 1 Bėkite, benjaminai, iš Jeruzalės; Tekojoje pūskite trimitą ir Bet Kereme uždekite ugnį kaip ženklą, nes nelaimė ir sunaikinimas ateina iš šiaurės. 
\par 2 Siono duktė panaši į gražią ir lepią moteriškę. 
\par 3 Piemenys su bandomis ateis prieš ją. Jie pasistatys aplink ją palapines, kiekvienas nuganys savo dalį. 
\par 4 Pradėkim prieš ją kovą! Kelkitės ir pradėkim kovą vidudienį! Vargas mums, nes baigiasi diena ir ilgėja vakaro šešėliai! 
\par 5 Kelkitės ir pulkime nakčia, sunaikinkime jos rūmus! 
\par 6 Taip sako kareivijų Viešpats: “Kirskite medžius ir supilkite prieš Jeruzalę pylimą! Ji yra baudžiamas miestas, nes vien tik priespauda jame. 
\par 7 Kaip iš šaltinio trykšta vanduo, taip iš jo trykšta nedorybės. Smurtas ir sunaikinimas jame, skriauda ir žaizdos nuolat mano akivaizdoje. 
\par 8 Leiskis pamokoma, Jeruzale, kad kartais Aš neatsitraukčiau nuo tavęs ir nepaversčiau tavęs dykyne, negyvenama šalimi!” 
\par 9 Taip sako kareivijų Viešpats: “Jie nurinks Izraelio likutį kaip vynmedį, dar kartą ranka perbrauks šakeles kaip vynuogių skynėjas!” 
\par 10 Kam turiu kalbėti ir ką įspėti, kad jie klausytų? Jų ausys neapipjaustytos, jos negali išgirsti. Viešpaties žodis jiems tapo pajuoka, jie jo nemėgsta. 
\par 11 Aš esu pilnas Viešpaties rūstybės, nebeįstengiu susilaikyti. Išliesiu ją ant vaikų gatvėje ir ant susirinkusių jaunuolių! Vyras ir žmona bus sugauti, pagyvenęs ir senas. 
\par 12 “Jų namai, laukai ir žmonos atiteks kitiems, nes Aš ištiesiu savo ranką prieš šalies gyventojus”,­ sako Viešpats. 
\par 13 “Jie visi, nuo mažiausio iki didžiausio, pasidavė godumui, nuo pranašo iki kunigo jie visi elgiasi klastingai. 
\par 14 Jie gydo mano tautos žaizdas tik paviršutiniškai, sakydami: ‘Taika! Taika!’ Tačiau taikos nėra. 
\par 15 Jie turėtų gėdytis, nes elgėsi bjauriai, tačiau nei jie gėdijasi, nei parausta. Todėl jie kris tarp krintančių, sukniubs, kai juos aplankysiu”,­sako Viešpats. 
\par 16 Taip sako Viešpats: “Eikite į kelius ir ieškokite senovinių takų; suradę gerą kelią, juo eikite. Taip rasite atilsį savo sieloms. Bet jie atsakė: ‘Neisime!’ 
\par 17 Aš pastačiau jiems sargų, sakydamas: ‘Klausykite trimito garso’. Bet jie atsakė: ‘Nesiklausysime!’ 
\par 18 Todėl klausykite, tautos, ir supraskite, kas tarp jų vyksta. 
\par 19 Klausyk, žeme! Aš užleisiu nelaimę šitai tautai kaip jų minčių vaisių, nes jie neklausė mano žodžių ir mano įstatymo, bet jį atmetė. 
\par 20 Kam man smilkalai, nešami iš Šebos, ir kvepianti nendrė iš tolimos šalies? Jūsų deginamosios aukos man nepatinka ir jūsų kraujo aukos man nemielos”. 
\par 21 Todėl taip sako Viešpats: “Štai Aš dedu šitai tautai ant kelio atsitrenkimo akmenų. Tėvai ir vaikai suklups ant jų, kaimynas ir draugas kartu pražus”. 
\par 22 Taip sako Viešpats: “Ateina tauta iš šiaurės, didelė tauta kyla nuo žemės pakraščių. 
\par 23 Jie lankais ir ietimis ginkluoti, žiaurūs ir negailestingi. Jų triukšmas kaip ūžianti jūra. Jie joja ant žirgų, kiekvienas pasirengęs kovai prieš tave, Sione!” 
\par 24 Kai tik išgirdome pranešimą, nusviro mūsų rankos, baimė ir skausmai apėmė mus. 
\par 25 Neikite į laukus ir nevaikščiokite keliu, nes priešo kardas kelia baimę aplinkui! 
\par 26 Mano tautos dukterie, apsivilk ašutine ir voliokis pelenuose. Gedėk kaip vienintelio sūnaus, graudžiai raudok; staiga ateis naikintojas ant mūsų! 
\par 27 Aš pastačiau tave bokštu ir tvirtove tarp mano žmonių, kad pažintum ir ištirtum jų kelius. 
\par 28 Jie visi kietasprandžiai ir šmeižikai, jie visi kaip varis ir geležis, visi sugedę. 
\par 29 Dumplės sugedo, ugnis išlydė šviną, bet veltui dirbo liejėjas­ nedorėliai neatsiskyrė nuo jų. 
\par 30 Jie bus vadinami atmestu sidabru, nes Viešpats atmetė juos.



\chapter{7}


\par 1 Viešpats kalbėjo Jeremijui: 
\par 2 “Atsistok Viešpaties šventyklos vartuose ir skelbk šitą žodį: ‘Klausykitės Viešpaties žodžio, visi Judo gyventojai, kurie įeinate pro vartus Viešpatį garbinti. 
\par 3 Taip sako Viešpats: ‘Pagerinkite savo kelius ir darbus, tada gyvensite šitoje vietoje. 
\par 4 Nepasitikėkite melagingais žodžiais: ‘Tai Viešpaties šventykla’. 
\par 5 Jei iš tikrųjų pagerinsite savo kelius ir darbus, jei teisingai spręsite bylas tarpusavyje, 
\par 6 svetimtaučio, našlaičio ir našlės neskriausite, nekalto kraujo nepraliesite šitoje vietoje ir svetimų dievų nesekiosite savo pačių nelaimei, 
\par 7 tada Aš amžiams paliksiu jus gyventi krašte, kurį daviau jūsų tėvams. 
\par 8 Jūs pasitikite melagingais žodžiais, kurie nieko nepadeda. 
\par 9 Argi manote, kad galite vogti, žudyti, svetimauti, melagingai prisiekti, aukoti Baalui ir sekioti svetimus dievus, 
\par 10 o po to ateiti ir atsistoti mano akivaizdoje šituose namuose, kurie pavadinti mano vardu, ir sakyti: ‘Mes saugūs ir galime daryti šias bjaurystes’. 
\par 11 Argi mano vardu pavadinti namai jūsų akyse tapo plėšikų lindyne? Aš mačiau tai’,­sako Viešpats. 
\par 12 ‘Eikite į mano buvusią vietą Šilojuje, kur prieš tai buvo mano vardas, ir pažiūrėkite, ką Aš ten padariau dėl mano tautos Izraelio nedorybių. 
\par 13 Kadangi dabar jūs darote tuos pačius darbus,­sako Viešpats,­ ir Aš kalbėjau jums anksti atsikeldamas, bet jūs nesiklausėte, jus šaukiau, bet jūs neatsiliepėte, 
\par 14 tai namams, kurie vadinami mano vardu ir kuriais jūs pasitikite, ir vietai, kurią daviau jums ir jūsų tėvams, Aš padarysiu taip, kaip padariau Šilojui. 
\par 15 Aš pašalinsiu jus iš savo akivaizdos, kaip pašalinau jūsų brolius, Efraimo palikuonis’. 
\par 16 Nesimelsk už šitą tautą ir nemaldauk manęs, nes Aš tavęs neišklausysiu. 
\par 17 Ar nematai, ką jie daro Judo miestuose ir Jeruzalės gatvėse? 
\par 18 Vaikai prirenka malkų, tėvai sukuria ugnį, moterys minko tešlą, kepa pyragaičius ir aukoja juos dangaus karalienei. Jie aukoja geriamąsias aukas svetimiems dievams, sukeldami mano rūstybę. 
\par 19 Ar jie sukelia mano rūstybę?­ sako Viešpats.­Ar ne sau gėdą jie užsitraukia? 
\par 20 Todėl taip sako Viešpats: ‘Aš išliesiu savo rūstybę ir įtūžį ant šitos vietos: ant žmonių, gyvulių, lauko medžių ir žemės vaisių; tai bus lyg negęstanti ugnis’. 
\par 21 Taip sako kareivijų Viešpats, Izraelio Dievas: ‘Pridėkite savo deginamąsias aukas prie padėkos aukų ir valgykite jų mėsą. 
\par 22 Aš nekalbėjau jūsų tėvams ir jiems nedaviau jokių įsakymų apie deginamąsias ir padėkos aukas, kai juos išvedžiau iš Egipto šalies. 
\par 23 Bet jiems įsakiau štai ką: ‘Pakluskite mano balsui, ir Aš būsiu jūsų Dievas, o jūs būsite mano tauta. Vaikščiokite keliu, kurį jums nurodžiau, kad jums gerai sektųsi’. 
\par 24 Tačiau jie neklausė manęs ir nekreipė dėmesio, bet sekė savo piktos širdies norais; jie ėjo ne pirmyn, bet atgal. 
\par 25 Nuo tos dienos, kai jūsų tėvai išėjo iš Egipto krašto, iki šios dienos Aš nuolat siunčiau pas jus savo tarnus, pranašus, keldamas juos anksti rytą. 
\par 26 Tačiau jie neklausė manęs ir nekreipė dėmesio, bet buvo užsispyrę. Jie elgėsi blogiau už savo tėvus’. 
\par 27 Kai tu kalbėsi šituos žodžius, jie tavęs nesiklausys, kai juos šauksi, jie tau neatsilieps. 
\par 28 Turėsi jiems sakyti: ‘Tai tauta, kuri nepakluso Viešpaties, savo Dievo, balsui ir nepriėmė pamokymo; tiesa pražuvo, ir niekas apie ją nekalbėjo. 
\par 29 Nusikirpk galvos plaukus ir numesk, raudok aukštumose, nes Viešpats atmetė ir paliko Jį užrūstinusią kartą. 
\par 30 Judo vaikai darė pikta,­sako Viešpats.­Jie pastatė savo bjaurystes namuose, kurie pavadinti mano vardu, ir juos sutepė. 
\par 31 Jie pastatė Tofeto aukurą Ben Hinomo slėnyje savo sūnums ir dukterims deginti, ko Aš neįsakiau ir kas man niekada neatėjo į širdį. 
\par 32 Ateis diena, kai Tofetas ir Ben Hinomo slėnis bus vadinamas žudynių slėniu: laidos Tofete, kol nebeliks jame vietos. 
\par 33 Šitos tautos lavonai bus maistas padangių paukščiams ir žemės žvėrims, ir niekas jų nenubaidys. 
\par 34 Iš Judo miestų ir iš Jeruzalės gatvių dings džiaugsmo ir linksmybės garsai, balsai jaunojo ir jaunosios, nes šalis pavirs dykyne’ ”.



\chapter{8}


\par 1 “Tuo metu,­sako Viešpats,­paims iš kapų Judo karalių, kunigaikščių, kunigų, pranašų ir Jeruzalės gyventojų kaulus. 
\par 2 Juos išbarstys priešais saulę, mėnulį ir visą dangaus kareiviją, kuriuos jie mylėjo ir sekė, kuriems tarnavo, ieškojo ir garbino. Jie nebus surinkti ir palaidoti, jie bus mėšlas dirvai tręšti. 
\par 3 Visi likę gyvieji iš šitos piktos kartos, kurie yra mano išsklaidyti, labiau norės mirti negu gyventi. 
\par 4 Sakyk jiems: ‘Taip sako Viešpats: ‘Jei kas krinta, ar jis nebeatsikels? Jei kas nusigręžia, ar jis nebeatsigręš? 
\par 5 Kodėl šita tauta užsispyrusiai laikosi savo paklydimo? Jie įsikibę į apgaulę ir nesutinka atsiversti. 
\par 6 Aš klausiausi ir supratau, kad jie kalba netiesą. Nė vienas neatgailauja dėl savo nedorybės, sakydamas : ‘Ką aš padariau?’ Jie visi eina savais keliais kaip žirgas, puoląs į kovą. 
\par 7 Net gandras, balandis, kregždė ir strazdas žino savo sugrįžimo laiką, bet mano tauta nežino Viešpaties nuostatų. 
\par 8 Kaip jūs galite sakyti: ‘Mes išmintingi ir Viešpaties įstatymas yra pas mus’? Iš tikrųjų mano įstatymą raštininkų plunksna padarė bevertį. 
\par 9 Išmintingieji bus sugėdinti, išgąsdinti ir pagauti. Jie atmetė Viešpaties žodį, tai kur jų išmintis? 
\par 10 Todėl Aš atiduosiu jų žmonas kitiems, jų laukus svetimiems, nes jie visi, nuo mažiausio iki didžiausio, pasidavę godumui, nuo pranašo iki kunigo jie visi elgiasi klastingai. 
\par 11 Jie gydo mano tautos žaizdas tik paviršutiniškai, sakydami: ‘Taika! Taika!’ Tačiau taikos nėra. 
\par 12 Jie turėtų gėdytis, nes elgėsi bjauriai, tačiau jie nei gėdijasi, nei parausta. Todėl jie kris tarp krintančių sukniubę, kai juos aplankysiu’,­sako Viešpats. 
\par 13 ‘Aš tikrai juos sunaikinsiu,­sako Viešpats.­Neliks vynuogių ant vynmedžių ir figų ant figmedžių, o lapai nuvys. Tai, ką jiems esu davęs, pasitrauks nuo jų’ ”. 
\par 14 Kodėl mes čia sėdime? Susirinkime, skubėkime į sustiprintus miestus ir ten sėdėkime tyliai, nes Viešpats, mūsų Dievas, mus nutildė ir girdo karčiu vandeniu, kadangi Jam nusidėjome. 
\par 15 Mes laukėme taikos, bet nieko gero nesulaukėme. Laukėme sveikatos, o štai­sunaikinimas! 
\par 16 Nuo Dano girdisi žirgų prunkštimas, nuo jų žvengimo dreba visas kraštas. Priešas užima šalį ir ryja visa: jos turtus, miestus ir gyventojus. 
\par 17 “Aš siunčiu jums nuodingų gyvačių, kurių negalima užkerėti, ir jos jus įgels”,­sako Viešpats. 
\par 18 Kada aš būsiu paguostas savo skausme, mano širdis alpsta manyje. 
\par 19 Mano tautos pagalbos šauksmas girdisi visame krašte. Argi nebėra Viešpaties Sione? Argi jis nebekaraliauja? Kodėl jie pykdė mane savo drožiniais ir svetimais stabais? 
\par 20 Praėjo pjūtis, pasibaigė vasara, o mes nesame išgelbėti. 
\par 21 Dėl savo tautos nelaimės aš kenčiu ir gedžiu, siaubas apėmė mane. 
\par 22 Argi nėra balzamo Gileade, argi nėra ten gydytojo? Kodėl neužgyja mano tautos žaizda?



\chapter{9}


\par 1 O kad mano galva būtų vandens šaltinis, ir akys ašarų versmė, kad dieną ir naktį galėčiau apraudoti savo tautos užmuštuosius! 
\par 2 Jei dykumoje turėčiau pastogę, palikčiau savo tautą ir eičiau ten! Nes jie visi svetimauja­neištikimųjų gauja. 
\par 3 “Jie sulenkia savo liežuvį melui kaip lanką. Jie nepakyla už tiesą žemėje. Jie eina iš pikto į piktą, o manęs nepažįsta”,­sako Viešpats. 
\par 4 “Saugokitės artimo, nepasitikėkite broliu, nes brolis elgiasi klastingai, o artimas skleidžia šmeižtus. 
\par 5 Vienas apgauna kitą, jie nekalba tiesos. Jie išmokė savo liežuvius kalbėti melą, elgiasi suktai, kol pavargsta. 
\par 6 Tu gyveni apsuptas apgaulių. Dėl savo apgaulių jie atsisako pažinti mane”,­sako Viešpats. 
\par 7 Todėl taip sako kareivijų Viešpats: “Aš juos išlydysiu ir mėginsiu, nes kaip kitaip galėčiau pasielgti su tokia tauta? 
\par 8 Jų liežuvis yra mirtina strėlė, jų žodžiai­klasta. Jie taikiai kalba su savo artimu, bet savo širdyje spendžia jam žabangus. 
\par 9 Ar dėl to neturėčiau jų bausti?­ sako Viešpats,­ar šitokiai tautai neturėčiau atkeršyti?” 
\par 10 Dėl kalnų verksiu ir dejuosiu, dėl ganyklų raudosiu, nes jos sunaikintos; niekas nebekeliauja per juos, nebesigirdi ir galvijų balso. Net padangių paukščiai ir žvėrys pasitraukė. 
\par 11 Aš padarysiu Jeruzalę akmenų krūva, šakalų lindyne ir Judo miestus paversiu negyvenama dykuma. 
\par 12 Kas yra toks išmintingas, kad tai suprastų? Kam Viešpats kalbėjo, kad jis tai paskelbtų, kodėl šalis sunaikinta lyg dykuma, kad niekas nebekeliauja per ją? 
\par 13 Viešpats tarė: “Kadangi jie atmetė mano įstatymą, kurį jiems daviau, nepakluso mano balsui ir nesielgė pagal jį, 
\par 14 bet sekė savo širdies užgaidas ir Baalą, kaip jų tėvai juos mokė, 
\par 15 todėl Aš juos valgydinsiu metėlėmis, girdysiu karčiu vandeniu 
\par 16 ir išsklaidysiu tarp tautų, kurių nepažino nei jie, nei jų tėvai. Iš paskos pasiųsiu kardą, kol juos visai sunaikinsiu”. 
\par 17 Taip sako kareivijų Viešpats: “Pašaukite raudotojų, kad jos ateitų, 
\par 18 kad jos atskubėtų ir giedotų raudą apie mus ir mes susigraudinę verktume”. 
\par 19 Rauda girdėti iš Siono: “Mes sunaikinti, visai sugėdinti! Mes turime palikti gimtinę, mūsų gyvenvietės sunaikintos”. 
\par 20 Klausykitės jūs, moterys, Viešpaties žodžio ir priimkite Jo pamokymą. Mokykite savo dukteris raudoti ir kaimynes apverkti. 
\par 21 Mirtis įėjo pro langus, įsilaužė į mūsų namus. Ji žudo vaikus gatvėse, jaunuolius aikštėse. 
\par 22 Taip sako Viešpats: “Žmonių lavonai bus išmėtyti kaip mėšlas laukuose, kaip pėdai užpakalyje pjovėjo. Ir niekas jų nesurinks”. 
\par 23 Taip sako Viešpats: “Išmintingasis tenesigiria savo išmintimi, stiprusis­savo stiprybe, o turtingasis­savo turtais. 
\par 24 Kas nori girtis, tegul giriasi, kad supranta ir pažįsta mane, kad Aš­Viešpats, kuris vykdau malonę, teismą ir teisingumą žemėje, nes tai man patinka,­sako Viešpats”. 
\par 25 “Ateis diena, kai Aš bausiu visus apipjaustytuosius ir neapipjaustytuosius: 
\par 26 Egiptą, Judą, Edomą, Amoną, Moabą ir visus, gyvenančius dykumoje. Nes visos tautos yra neapipjaustytos ir Izraelis yra neapipjaustytas širdyje”,­sako Viešpats.



\chapter{10}


\par 1 Izraeli, klausykis Viešpaties žodžio! 
\par 2 Taip sako Viešpats: “Nesimokykite pagonių kelių ir nebijokite dangaus ženklų, nors pagonys jų bijo. 
\par 3 Pagonių papročiai yra tuštybė. Jie nusikerta medį miške, amatininkas kirviu aptašo jį, 
\par 4 sidabru ir auksu pagražina, pritvirtina plaktuku ir vinimis, kad nejudėtų. 
\par 5 Jie yra tiesūs kaip palmė, negali kalbėti nei vaikščioti, juos reikia nešti. Nebijokite jų, nes jie negali nei pakenkti, nei gera daryti”. 
\par 6 Viešpatie, Tau nėra lygaus, Tu esi didis ir Tavo vardas galingas! 
\par 7 Kas Tavęs nebijotų, tautų Karaliau? Tau priklauso tai, nes tarp visų tautų išminčių ir visų jų valdovų nėra nė vieno Tau lygaus. 
\par 8 Jie visi yra kvaili ir neišmintingi. Medis yra tuštybės mokymas, 
\par 9 padengtas sidabrine skarda, atgabenta iš Taršišo ir Ufazo. Jis yra amatininko ir auksakalio dirbinys, apvilktas violetiniu ir raudonu drabužiu. Jie visi yra menininkų dirbiniai. 
\par 10 Tačiau Viešpats yra tikrasis Dievas, gyvasis Dievas ir amžinasis Karalius. Nuo Jo rūstybės dreba žemė ir Jo grūmojimo nepakelia tautos. 
\par 11 Taip turite jiems sakyti: “Dievai, kurie nepadarė nei dangaus, nei žemės, turi pradingti nuo žemės ir iš šios padangės”. 
\par 12 Jis sukūrė žemę savo jėga, savo išmintimi padėjo pasaulio pamatą ir savo supratimu ištiesė dangus. 
\par 13 Jo balso klauso vandenys danguose, Jis pakelia garus nuo žemės pakraščių. Jis siunčia žaibus su lietumi, paleidžia vėją iš savo sandėlių. 
\par 14 Žmogus neturi pažinimo ir yra neišmintingas. Amatininkai bus sugėdinti dėl savo drožinių, jų lieti atvaizdai yra apgaulė, juose nėra kvapo. 
\par 15 Jie yra tuštybė, paklydimo darbas. Aplankymo metu jie pražus. 
\par 16 Visai kitokia yra Jokūbo dalis. Jis yra visa ko Kūrėjas, Izraelis yra Jo nuosavybė. Kareivijų Viešpats yra Jo vardas! 
\par 17 Susirinkite savo daiktus nuo žemės, tvirtovės gyventojai! 
\par 18 Nes taip sako Viešpats: “Šį kartą Aš išmesiu krašto gyventojus, prispausiu juos taip, kad jie pajus”. 
\par 19 Vargas man, esu sužeistas! Mano žaizda nepagydoma! Aš tariau: “Tai yra sielvartas, kurį turiu pakelti”. 
\par 20 Mano palapinė apiplėšta, visos virvės sutraukytos. Mano vaikai paliko mane, jų nebėra. Nėra kas ištiestų ir pakeltų mano palapinę. 
\par 21 Ganytojai tapo kvailiais ir neieškojo Viešpaties. Todėl jiems nesisekė, visa jų kaimenė bus išsklaidyta. 
\par 22 Štai žinia, kad artėja didelis pavojus iš šiaurės paversti Judo miestus griuvėsiais, šakalų buveinėmis. 
\par 23 Viešpatie, aš žinau, kad ne žmogaus rankose yra jo keliai. Nė vienas negali pakreipti savo žingsnių, kaip jis nori. 
\par 24 Bausk mane, Viešpatie, bet teisingai, nerūstaudamas, kad nesunaikintum manęs visiškai! 
\par 25 Išliek savo rūstybę ant pagonių, kurie Tavęs nepažįsta, ir ant giminių, kurios nesišaukia Tavęs, nes jos ėdė ir prarijo Jokūbą, dykuma pavertė jo gyvenvietes.



\chapter{11}


\par 1 Viešpats kalbėjo Jeremijui: 
\par 2 “Klausyk sandoros žodžių ir kalbėk Judo žmonėms bei Jeruzalės gyventojams. 
\par 3 Sakyk jiems: ‘Taip sako Viešpats: ‘Prakeiktas žmogus, kuris nepaklūsta šitos sandoros žodžiams, 
\par 4 kuriuos Aš įsakiau jūsų tėvams, kai juos išvedžiau iš Egipto šalies, iš geležinės krosnies. Pakluskite mano balsui ir elkitės taip, kaip jums įsakiau, tai būsite mano tauta, o Aš būsiu jūsų Dievas, 
\par 5 kad ištesėčiau priesaiką, kurią daviau jūsų tėvams paveldėti šalį, tekančią pienu ir medumi’ ”. Tada aš atsiliepiau: “Tebūna taip, Viešpatie!” 
\par 6 Paskui Viešpats man tarė: “Skelbk visus šiuos žodžius Judo miestuose ir Jeruzalės gatvėse, sakydamas: ‘Klausykite šitos sandoros žodžių ir juos vykdykite! 
\par 7 Aš įsakmiai raginau jūsų tėvus, kai juos vedžiau iš Egipto, ir iki šios dienos, sakydamas: ‘Pakluskite mano balsui’. 
\par 8 Tačiau jie neklausė, bet sekė savo piktų širdžių užgaidas. Todėl Aš užvesiu jiems visus sandoros žodžius, kuriuos buvau jiems įsakęs, bet jie nesilaikė’ ”. 
\par 9 Viešpats vėl man kalbėjo: “Judo žmonės ir Jeruzalės gyventojai daro sąmokslą; 
\par 10 jie nusikalsta kaip jų protėviai, kurie atsisakė klausyti mano žodžių: jie seka svetimus dievus ir tarnauja jiems. Izraelis ir Judas sulaužė mano sandorą, kurią padariau su jų tėvais. 
\par 11 Todėl taip sako Viešpats: ‘Aš užleisiu jiems nelaimę, iš kurios jie negalės ištrūkti, ir, kai jie šauksis manęs, Aš jų neišklausysiu. 
\par 12 Tada Judo miestai ir Jeruzalės gyventojai šauksis dievų, kuriems jie smilkė, bet tie niekuo jiems nepadės nelaimės metu. 
\par 13 Judai, kiek tavo miestų, tiek dievų; kiek Jeruzalėje gatvių, tiek pasistatėte gėdingų aukurų smilkyti Baalui! 
\par 14 Nesimelsk už šitą tautą. Aš jų neišklausysiu, kai jie šauksis manęs suspaudimo metu. 
\par 15 Ko nori, mano mylimoji, mano namuose po savo begėdysčių su daugeliu? Ar šventa mėsa nukreips nuo tavęs nelaimę?’ ” 
\par 16 Žaliuojančiu, gražiu alyvmedžiu Viešpats vadino tave; dabar Jis ūžiančia ugnimi sudegina jo lapus ir šakas! 
\par 17 Kareivijų Viešpats, kuris pasodino tave, nusprendė tau nelaimę užleisti dėl Izraelio ir Judo nusikaltimų, kuriais jie užsitraukė Viešpaties rūstybę, smilkydami Baalui. 
\par 18 Viešpats man apreiškė, ir aš supratau; tada Tu parodei man jų piktus darbus. 
\par 19 Aš buvau kaip paklusnus avinėlis, vedamas pjauti, nežinojau, kad jie ruošė planus prieš mane: “Sunaikinkime medį ir jo vaisius! Išraukime jį iš gyvųjų žemės, kad užmirštų jo vardą!” 
\par 20 Kareivijų Viešpatie, kuris teisi teisingai ir ištiri žmogaus širdį ir inkstus, leisk man matyti Tavo kerštą jiems, nes aš Tau pavedžiau savo bylą. 
\par 21 Taip sako Viešpats apie Anatoto vyrus, kurie ieško mano gyvybės, sakydami: “Nepranašauk Viešpaties vardu, kad mes tavęs nenužudytume!” 
\par 22 Todėl kareivijų Viešpats sako: “Aš juos nubausiu; jų jaunuoliai žus nuo kardo, o jų sūnūs ir dukterys mirs badu. 
\par 23 Iš jų niekas neišliks, kai užleisiu nelaimę ant Anatoto žmonių jų aplankymo metu”.



\chapter{12}


\par 1 Viešpatie, Tu liksi teisus, jei bylinėsiuosi su Tavimi! Tačiau leisk man kalbėti su Tavimi apie Tavo sprendimus. Kodėl nedorėliams sekasi ir kodėl be rūpesčių gyvena visi, kurie elgiasi klastingai? 
\par 2 Tu juos įsodinai, jie įleido šaknis, užaugo, nešė vaisių. Tu esi arti jų burnos, bet toli nuo jų širdies. 
\par 3 Viešpatie, Tu pažįsti mane, ištyrei mano širdį ir žinai, kokia ji yra prieš Tave. Atskirk juos kaip avis pjauti, paruošk juos žudymo dienai! 
\par 4 Ar ilgai liūdės šalis ir džius lauko augalai? Dėl jos gyventojų nedorybių išnyko žvėrys ir paukščiai. Jie sakė: “Jis nematys mūsų galo”. 
\par 5 Jei tu bėgai su pėstininkais ir jie nuvargino tave, tai kaip lenktyniausi su žirgais? Ir jei ramioje šalyje nesi saugus, tai ką darysi, Jordanui išsiliejus? 
\par 6 Net tavo broliai, tavo tėvo namai nėra tau ištikimi. Jie sušaukė minią tau už nugaros. Netikėk jais, nors jie kalba gražius žodžius. 
\par 7 Aš palikau savo namus, atsisakiau nuosavybės; kas mano sielai miela, atidaviau į priešo rankas. 
\par 8 Mano paveldėjimas pasidarė man kaip liūtas miške ir pakėlė savo balsą prieš mane, todėl ėmiau nekęsti jo. 
\par 9 Argi mano paveldėjimas margas paukštis, kad paukščiai apspitę puola jį? Ateikite, susirinkite, visi laukiniai žvėrys, ateikite ėsti! 
\par 10 Daug ganytojų naikino mano vynuogyną, mindžiojo mano paveldėjimą, pavertė mano puikųjį lauką tuščia dykuma. 
\par 11 Jie padarė jį dykuma, kuri verkia mano akivaizdoje. Visa šalis virto dykuma ir niekas dėl to nesisieloja. 
\par 12 Visoms dykumos kalvoms atėjo naikintojai. Viešpaties kardas ryja nuo vieno šalies krašto iki kito, joks žmogus neturi ramybės. 
\par 13 Jie sėjo kviečius, bet pjaus erškėčius, jie vargo, bet be naudos. Jie bus sugėdinti savo derliumi dėl Viešpaties rūstybės. 
\par 14 Taip sako Viešpats: “Visus piktus kaimynus, besikėsinančius į mano tautos, Izraelio, paveldėjimą, Aš išrausiu iš jų žemės, o Judo namus išrausiu iš jų tarpo. 
\par 15 Tačiau, kai būsiu juos išrovęs, vėl jų pasigailėsiu ir juos visus sugrąžinsiu į jų paveldėjimą ir į jų kraštą. 
\par 16 Jei jie rūpestingai mokysis mano tautos kelių ir prisieks mano vardu: ‘Kaip Viešpats gyvas!’, kaip jie buvo išmokę mano tautą prisiekti Baalu, tai jiems leisiu įsikurti mano tautoje. 
\par 17 Bet jei jie neklausys, tai tokią tautą visiškai išrausiu ir sunaikinsiu,­sako Viešpats”.



\chapter{13}


\par 1 Viešpats tarė man: “Eik, nusipirk drobinį diržą ir juo susijuosk strėnas, bet nesušlapink jo”. 
\par 2 Aš nusipirkau diržą, kaip Viešpats liepė, ir susijuosiau. 
\par 3 Viešpats man antrą kartą sakė: 
\par 4 “Imk diržą, kurį nusipirkai ir kuris yra ant tavo strėnų, nueik prie Eufrato ir ten jį paslėpk uolos plyšyje”. 
\par 5 Aš nuėjau ir paslėpiau jį prie Eufrato, kaip Viešpats man liepė. 
\par 6 Praėjus daugeliui dienų, Viešpats man tarė: “Nueik prie Eufrato ir iš ten pasiimk diržą, kurį tau liepiau paslėpti”. 
\par 7 Aš nuėjau prie Eufrato, iškasiau diržą iš tos vietos, kur jį buvau paslėpęs, ir radau jį supuvusį, niekam nebetinkamą. 
\par 8 Tada Viešpats man kalbėjo: 
\par 9 “Taip Aš supūdysiu Judo ir Jeruzalės puikybę. 
\par 10 Šita pikta tauta, kuri atsisako klausyti mano žodžių, vaikšto paskui savo širdies užgaidas ir seka svetimus dievus, jiems tarnauja ir juos garbina, taps kaip šitas diržas, kuris niekam nebetinka. 
\par 11 Kaip diržas prisiglaudžia prie vyro strėnų, taip Aš norėjau, kad Izraelis ir Judas prisiglaustų prie manęs,­sako Viešpats.­Kad jie būtų mano tauta, mano garbė ir pasididžiavimas ir mano vardu vadintųsi, bet jie neklausė. 
\par 12 Sakyk jiems šiuos žodžius: ‘Taip sako Viešpats, Izraelio Dievas: ‘Kiekvienas ąsotis pripilamas vyno’. Jie tau sakys: ‘Argi mes nežinome, kad ąsotis skirtas supilti vynui?’ 
\par 13 Tada jiems atsakyk: ‘Taip sako Viešpats: ‘Aš nugirdysiu visus šitos šalies gyventojus: karalius, kurie sėdi Dovydo soste, kunigus, pranašus ir visus Jeruzalės gyventojus. 
\par 14 Aš sudaužysiu juos vienas į kitą, tėvus kartu su vaikais,­sako Viešpats.­Aš jų nepasigailėsiu, neužjausiu, bet sunaikinsiu’ ”. 
\par 15 Klausykitės ir supraskite, nebūkite išdidūs, nes Viešpats kalba! 
\par 16 Duokite Viešpačiui, savo Dievui, šlovę, kol nesutemo, prieš atsitrenkiant jūsų kojoms į tamsoje esančius kalnus! Jūs laukiate šviesos, bet Jis užleis tamsą, visišką tamsybę. 
\par 17 Jei jūs neklausysite, mano siela graudžiai verks dėl jūsų išdidumo ir mano akys ašaros, nes Viešpaties kaimenė vedama į nelaisvę. 
\par 18 Sakykite karaliui ir karalienei, kad jie nusižemintų ir sėstųsi žemai, nes nuo jų galvos nukris šlovės karūna. 
\par 19 Pietų krašto miestai bus užrakinti, ir nebus kas juos atidarytų; visas Judas bus išvestas į nelaisvę. 
\par 20 Pakelk savo akis ir žiūrėk, kaip jie ateina iš šiaurės! Kur yra tau patikėta kaimenė, tavo gražiosios avys? 
\par 21 Ką darysi, kai Jis tave aplankys? Tu pati pripratinai juos vadovauti tau? Argi nesuims tavęs skausmai kaip gimdyvės? 
\par 22 Jei sakysi savo širdy: “Kodėl man taip atsitiko?”­Dėl tavo nuodėmių daugybės tavo sijonas pakeltas, tavo kulnai apnuoginti. 
\par 23 Jei etiopas galėtų pakeisti savo odos spalvą ir leopardas savo kailio dėmes, tai ir jūs galėtumėte daryti gera, kurie esate įpratę daryti pikta. 
\par 24 “Aš išsklaidysiu jus kaip pelus, kuriuos dykumos vėjas išnešioja. 
\par 25 Tai yra tau kritęs burtas ir tavo dalis nuo manęs,­sako Viešpats,­kadangi pamiršai mane ir pasitikėjai melu, 
\par 26 tai Aš užversiu tavo sijoną tau ant veido, kad pasirodytų tavo gėda. 
\par 27 Aš mačiau tavo svetimavimus, gašlumą ir paleistuvystę. Vargas tau, Jeruzale. Ar ilgai dar neapsivalysi?”



\chapter{14}


\par 1 Viešpats kalbėjo Jeremijui apie sausrą: 
\par 2 “Judas liūdi, jo vartai svyruoja. Žmonės pajuodę guli ant žemės, ir Jeruzalės šauksmas kyla aukštyn. 
\par 3 Didikai siunčia tarnus vandens; tie nueina prie šulinių, bet, neradę vandens, sugrįžta tuščiais indais. Jie sugėdinti ir nusiminę, apdengtomis galvomis. 
\par 4 Nėra lietaus, žemė išdžiūvo. Artojai stovi susigėdę, galvas apsidengę. 
\par 5 Net elnė palieka savo jauniklį, nes nėra žolės. 
\par 6 Laukiniai asilai, stovėdami ant nuplikusių aukštumų, uosto vėją kaip šakalai; jų akys aptemę, nes nėra žolės”. 
\par 7 Viešpatie, nors mūsų nusikaltimai liudija prieš mus, gelbėk mus dėl savo vardo. Savo daugybe paklydimų Tau nusidėjome. 
\par 8 Izraelio viltie, gelbėtojau nelaimės metu! Kodėl Tu esi kaip svetimšalis, kaip keleivis, kuris užsuka tik nakvoti? 
\par 9 Kodėl Tu esi kaip bejėgis žmogus, kaip karžygys, kuris negali išgelbėti? Tačiau Tu, Viešpatie, esi tarp mūsų, mes vadinami Tavo vardu! Nepalik mūsų! 
\par 10 Taip sako Viešpats šiai tautai: “Jie mėgsta klajoti, nesulaiko savo kojų. Todėl Viešpats nepriims jų, Jis atsimins jų kaltę ir aplankys juos už jų nuodėmes”. 
\par 11 Viešpats man tarė: “Nesimelsk už šitą tautą, kad jai gerai sektųsi. 
\par 12 Nors jie pasninkaus, Aš neišklausysiu jų šauksmo; nors jie aukos deginamąsias ir duonos aukas, Aš nepriimsiu jų; Aš sunaikinsiu juos kardu, badu ir maru”. 
\par 13 Aš atsakiau: “Ak, Viešpatie Dieve, pranašai jiems skelbia: ‘Nematysite kardo ir nepajusite bado, bet jums duosiu tikrą ramybę šioje vietoje’ ”. 
\par 14 Viešpats man tarė: “Pranašai melagingai pranašauja mano vardu. Aš jų nesiunčiau, nepaliepiau ir nekalbėjau jiems. Jie jums pranašauja išgalvotus regėjimus, žyniavimus ir niekam tikusią, pačių prasimanytą apgaulę. 
\par 15 Todėl apie pranašus, kurie mano vardu pranašauja, sakydami: ‘Nebus nei kardo, nei bado šitoje šalyje’, nors Aš jų nesiunčiau, Viešpats taip sako: ‘Nuo kardo ir bado žus tie pranašai. 
\par 16 Ir žmonės, kuriems jie pranašauja, žus nuo kardo ir bado ir gulės Jeruzalės gatvėse­jie ir jų žmonos, sūnūs ir dukterys,­nes nebus kam palaidoti. Aš išliesiu jų nedorybes ant jų’ ”. 
\par 17 Viešpats liepė man sakyti jiems: “Mano akys plūsta ašaromis dieną ir naktį, negaliu nurimti, nes mano tauta nepagydomai sužeista. 
\par 18 Laukuose guli žuvusieji, mieste badas ir kančios. Pranašai ir kunigai be nuovokos vaikščioja po kraštą”. 
\par 19 Ar Tu visai išsižadėjai Judo, ar Sionas nusibodo Tavo sielai? Kodėl mus nepagydomai sumušei? Mes tikėjomės ramybės, tačiau nėra nieko gero, laukėme išgydymo, o štai­sunaikinimas! 
\par 20 Viešpatie, mes pripažįstame savo nedorybę ir mūsų tėvų kaltes, nes mes tau nusidėjome. 
\par 21 Neatstumk mūsų dėl savo vardo; nepaniekink savo šlovės sosto; atsimink sandorą su mumis ir nesulaužyk jos! 
\par 22 Ar pagonių tuštybės gali duoti lietaus, ar dangūs patys siunčia lietų? Ar ne Tu, Viešpatie, mūsų Dieve? Mes laukiame Tavęs, nes Tu visa padarei!



\chapter{15}


\par 1 Tada Viešpats man tarė: “Net jei Mozė ir Samuelis stotų mano akivaizdoje, mano širdies nepalenktų, nepagailėčiau šitos tautos. Pašalink juos iš mano akių, kad pasitrauktų. 
\par 2 Jei jie klaus tavęs: ‘Kur mums eiti?’, atsakyk jiems: ‘Taip sako Viešpats: ‘Kas paskirtas mirčiai, teeina į mirtį, kas kardui­į žūtį, kas badui­į badą ir kas nelaisvei­į nelaisvę’. 
\par 3 Aš juos naikinsiu keturiais būdais: kardas žudys, šunys draskys, padangių paukščiai ir laukiniai žvėrys ris ir naikins. 
\par 4 Už tai, ką Judo karalius Manasas, Ezekijo sūnus, padarė Jeruzalėje, Aš atiduosiu juos vargui į visas žemės karalystes. 
\par 5 ‘Kas gailėsis tavęs, Jeruzale, kas tave užjaus? Ir kas teirausis apie tavo gerovę? 
\par 6 Tu mane palikai,­sako Viešpats,­ir atsukai man nugarą. Todėl Aš ištiesiu prieš tave savo ranką ir tave sunaikinsiu, Aš pavargau gailėtis’. 
\par 7 Aš juos vėtysiu visoje šalyje. Aš atimsiu jų vaikus ir naikinsiu juos, kadangi jie neatsisako savo kelių. 
\par 8 Jų našlių yra daugiau negu jūros smilčių. Aš atvedžiau jų jaunuoliams naikintoją, kuris staiga užpuolė, ir miestą apėmė siaubas. 
\par 9 Septynių sūnų motina nusilpo, ji atidavė dvasią; jai nusileido saulė dienos metu, ji buvo apvilta ir pažeminta. O likusius atiduosiu priešų kardui,­sako Viešpats”. 
\par 10 Motina, vargas man! Kam tu mane pagimdei vaidams ir kivirčams visame krašte? Nei aš skolinau, nei skolinausi, tačiau jie visi keikia mane. 
\par 11 Viešpats sako: “Tikrai su tavimi bus viskas gerai. Aš padarysiu, kad priešai užstos tave nelaimės ir pavojaus metu. 
\par 12 Ar gali geležis sulaužyti geležį iš šiaurės ir varį? 
\par 13 Tavo turtą ir lobius leisiu grobti už tavo nuodėmes visame krašte. 
\par 14 Aš pasiųsiu tave su priešais į svetimą kraštą. Mano rūstybės ugnis užsidegė, ji degins jus”. 
\par 15 Viešpatie, atsimink mane ir aplankyk mane. Atkeršyk už mane mano persekiotojams. Nesunaikink manęs, būdamas maloningas. Tu žinai, kad aš dėl Tavęs kenčiu priekaištus! 
\par 16 Tavo žodžiais maitinuosi. Tavo žodis džiugina mane, jis yra mano širdies linksmybė; aš vadinuosi Tavo vardu, Viešpatie, kareivijų Dieve! 
\par 17 Aš niekada nesėdėjau pašaipūnų būryje ir nesidžiaugiau. Aš sėdėjau vienišas dėl Tavo rankos, nes Tu pripildei mane apmaudo. 
\par 18 Kodėl mano skausmas nepraeina, žaizda nepagydoma? Nejaugi Tu būsi man kaip apgaulingas upelis, kaip nepatikimas vanduo? 
\par 19 Taip sako Viešpats: “Jei grįši, Aš vėl priimsiu tave, tu būsi mano tarnas. Tu atskirsi, kas brangu ir kas menka, būsi kaip mano lūpos. Jie kreipsis į tave, o ne tu į juos. 
\par 20 Aš padarysiu tave tvirta varine siena: nors jie kovos prieš tave, tačiau nenugalės­Aš būsiu su tavimi, tau padėsiu ir išgelbėsiu tave,­sako Viešpats,­ 
\par 21 Aš išgelbėsiu tave iš priešo rankos ir išvaduosiu iš prispaudėjo rankos”.



\chapter{16}


\par 1 Viešpats kalbėjo man: 
\par 2 “Tau nevalia vesti žmonos ir turėti sūnų ar dukterų šioje vietoje, 
\par 3 nes Viešpats taip sako apie sūnus ir dukteris, gimstančius šitoje vietoje, ir apie jų motinas ir tėvus: 
\par 4 ‘Baisia mirtimi jie mirs! Jų niekas neapraudos ir nepalaidos, jie bus kaip mėšlas laukuose! Nuo kardo ir bado jie žus, jų lavonai bus ėdesiu padangių paukščiams ir laukiniams žvėrims’. 
\par 5 Taip sako Viešpats: ‘Neik į namus, kuriuose gedulas, neraudok ir neužjausk jų, nes Aš atėmiau iš šitos tautos savo ramybę, malonę ir pasigailėjimą. 
\par 6 Dideli ir maži mirs šitoje šalyje; jų nepalaidos ir neapraudos, nesiraižys dėl jų ir plikai nenusiskus. 
\par 7 Nelauš duonos gedinčiam, nepaguos dėl mirusio ir neduos gerti paguodos taurės dėl tėvo ar motinos. 
\par 8 Tau nevalia eiti į puotos namus ir ten sėdėti, valgyti ir gerti su jais. 
\par 9 Nes taip sako kareivijų Viešpats, Izraelio Dievas: ‘Aš pašalinsiu tada šitoje vietoje, jums matant, džiaugsmo ir linksmybės balsą, jaunikio ir jaunosios balsą’. 
\par 10 Kai tu praneši šitai tautai visus tuos žodžius ir jie klaus: ‘Kodėl Viešpats paskelbė mums šitą didelę nelaimę? Kokia yra mūsų nuodėmė, kuo nusikaltome Viešpačiui, mūsų Dievui?’, 
\par 11 tada jiems atsakyk: ‘Nes jūsų tėvai paliko mane, sekė svetimus dievus, jiems tarnavo ir juos garbino, o mano įstatymo nesilaikė. 
\par 12 Jūs elgiatės dar blogiau negu jūsų tėvai, jūs sekate savo piktos širdies užgaidas, o manęs neklausote. 
\par 13 Aš išmesiu jus iš šitos šalies į šalį, kurios nežinote nei jūs, nei jūsų tėvai, ir ten dieną bei naktį tarnausite svetimiems dievams, nes Aš neberodysiu jums savo malonės. 
\par 14 Ateina dienos,­sako Viešpats,­kai nebesakys: ‘Kaip gyvas Viešpats, kuris išvedė Izraelio tautą iš Egipto žemės!’ 
\par 15 Bet sakys: ‘Kaip gyvas Viešpats, kuris išvedė izraelitus iš šiaurės ir iš visų šalių, kuriose jie buvo išsklaidyti’. Aš juos sugrąžinsiu į jų žemę, kurią daviau jų tėvams. 
\par 16 Aš pasiųsiu daug žvejų, kurie žvejos juos; po to pasiųsiu daug medžiotojų, kurie juos medžios kalnuose ir kalvose bei uolų plyšiuose. 
\par 17 Mano akys seka juos visuose keliuose, jie negali pasislėpti nuo manęs, mano akys mato jų kaltę. 
\par 18 Aš visų pirma dvigubai atlyginsiu jiems už jų nusikaltimus, nes jie suteršė mano žemę ir mano paveldėjimą pripildė savo šlykščiais ir bjauriais stabais’ ”. 
\par 19 Viešpatie, Tu esi mano stiprybė ir mano pilis, mano priebėga nelaimės metu! Pas Tave ateis tautos nuo žemės pakraščių ir sakys: “Mūsų tėvai paveldėjo tik apgaulę, tuštybę ir stabus, kurie jiems nieko nepadėjo”. 
\par 20 Ar gali žmogus pasidaryti dievų? Juk jie ne dievai! 
\par 21 “Aš leisiu jiems patirti savo galią ir jie žinos, kad mano vardas yra Viešpats”.



\chapter{17}


\par 1 Judo nuodėmė užrašyta geležiniu rašikliu, aštriu deimantu įrėžta į jų širdį ir ant aukurų ragų, 
\par 2 kol jų vaikai atsimena aukurus ir alkus po žaliuojančiais medžiais ant aukštų kalvų. 
\par 3 O mano kalne laukuose! Tavo turtą ir visus lobius leisiu išplėšti visame krašte už tavo nuodėmes. 
\par 4 Tu turėsi palikti savo paveldėjimą, kurį tau daviau; tave padarysiu vergu priešų šalyje, kurios nepažįsti, nes mano rūstybės ugnis degs amžinai. 
\par 5 Taip sako Viešpats: “Prakeiktas žmogus, kuris pasitiki žmogumi ir laiko kūną savo stiprybe, kurio širdis nutolsta nuo Viešpaties. 
\par 6 Jis bus kaip krūmokšnis dykumoje ir nieko gero nematys. Jis gyvens sausoje, druskingoje ir negyvenamoje šalyje. 
\par 7 Palaimintas žmogus, kuris pasitiki Viešpačiu, kurio viltis yra Viešpats! 
\par 8 Jis bus kaip medis, pasodintas prie vandens, leidžiantis šaknis prie upelio. Jis nebijos ateinančios kaitros, jo lapai visada žaliuos. Jis nesirūpins sausros metu, bet nuolat neš vaisių. 
\par 9 Širdis yra labai klastinga ir be galo nedora. Kas ją supras! 
\par 10 Aš, Viešpats, ištiriu širdį, išbandau inkstus ir atlyginu kiekvienam pagal jo kelius ir jo darbų vaisius”. 
\par 11 Kaip kurapka, tupinti ant kiaušinių, bet neišperinti jų, yra tas, kuris neteisingai įgyja turtą. Gyvenimui įpusėjus turtai paliks jį, ir galiausiai jis liks kaip kvailys. 
\par 12 Mūsų šventykla yra šlovingas sostas, išaukštintas nuo pradžios. 
\par 13 Viešpatie, Izraelio viltie! Visi, kurie palieka Tave, bus sugėdinti. Kurie nutolsta nuo Tavęs, bus įrašyti į žemės dulkes, nes jie paliko Viešpatį, gyvojo vandens versmę. 
\par 14 Viešpatie, pagydyk mane, tai būsiu sveikas; gelbėk mane, tai būsiu išgelbėtas, nes Tu esi mano gyrius! 
\par 15 Jie man sako: “Kur Viešpaties žodis? Teateina dabar!” 
\par 16 Aš niekados nevengiau būti ganytoju pas Tave ir nelaukiau nelaimės. Tu žinai visa, ką kalbėjau. 
\par 17 Negąsdink manęs, Tu esi mano viltis piktą dieną. 
\par 18 Sugėdink mano persekiotojus ir išgąsdink juos, bet ne mane! Siųsk jiems piktą dieną ir sunaikink juos dvigubu sunaikinimu! 
\par 19 Viešpats įsakė man: “Eik ir atsistok šventyklos vartuose, pro kuriuos įeina ir išeina Judo karaliai, ir kituose Jeruzalės vartuose, 
\par 20 sakydamas jiems: ‘Klausykitės Viešpaties žodžio, Judo karaliai ir visi Judo bei Jeruzalės gyventojai, kurie įeinate pro šiuos vartus! 
\par 21 Taip sako Viešpats: ‘Saugokitės ir neneškite nieko sabato dieną pro Jeruzalės vartus. 
\par 22 Neišneškite jokios naštos iš savo namų sabato dieną ir nedirbkite jokio darbo, bet švęskite sabatą, kaip įsakiau jūsų tėvams. 
\par 23 Tačiau jie neklausė ir nekreipė dėmesio, bet užsispyrė, kad negirdėtų ir nepriimtų pamokymo. 
\par 24 Jei jūs manęs paklausysite,­sako Viešpats,­ir nenešite jokios naštos į šitą miestą pro vartus sabato dieną, bet švęsite sabatą ir nedirbsite jokio darbo, 
\par 25 tai pro šito miesto vartus įeis karaliai ir kunigaikščiai, sėdintieji Dovydo soste; važiuos vežimais ir jos ant žirgų jie ir jų kunigaikščiai, Judo ir Jeruzalės gyventojai; ir šitas miestas pasiliks per amžius. 
\par 26 Jie ateis iš Judo miestų, iš Jeruzalės apylinkių, iš Benjamino krašto, iš lygumos, iš kalnyno ir pietų šalies, atnešdami deginamąsias, padėkos ir duonos aukas bei smilkalus į Viešpaties namus. 
\par 27 Bet jei manęs neklausysite, nešvęsite sabato, nešite naštas pro Jeruzalės vartus sabato dieną, tai Aš uždegsiu jos vartus ir sunaikinsiu Jeruzalės rūmus’ ”.



\chapter{18}


\par 1 Viešpats kalbėjo Jeremijui: 
\par 2 “Eik į puodžiaus namus, ten tu išgirsi mano žodžius!” 
\par 3 Aš, nuėjęs į puodžiaus namus, radau jį bedirbantį prie žiestuvo. 
\par 4 Indas, kurį jis darė iš molio, nepavyko, ir jis iš naujo nužiedė kitą indą, kokį panorėjo. 
\par 5 Tada Viešpats tarė man: 
\par 6 “Izraelitai, ar Aš negaliu su jumis taip daryti, kaip šitas puodžius? Kaip molis puodžiaus rankoje, taip jūs, izraelitai, mano rankoje. 
\par 7 Kartais Aš grasinu tautai ar karalystei ją išrauti, sužlugdyti ir sunaikinti, 
\par 8 bet jei tauta, kuriai grasinu, nusisuka nuo savo piktų darbų, tai Aš susilaikau nuo to pikto, kurį buvau jai numatęs. 
\par 9 Kartais Aš pažadu tautai ar karalystei ją statyti ir įtvirtinti, 
\par 10 bet jei ji daro pikta ir neklauso manęs, tai Aš susilaikau nuo to gero, kurį buvau pažadėjęs padaryti. 
\par 11 Taigi dabar sakyk Judo ir Jeruzalės gyventojams, kad Aš ketinu juos bausti, jeigu jie neatsisakys savo piktų kelių ir nedarys to, kas gera”. 
\par 12 Bet jie atsakė: “Jokių vilčių. Mes vykdysime savo pačių sumanymus ir elgsimės, kaip mums patinka”. 
\par 13 Todėl taip sako Viešpats: “Klauskite pagonių, ar kas praeityje girdėjo apie tokius dalykus, kokius padarė Izraelis! 
\par 14 Ar pradingsta nuo Libano uolų baltas sniegas? Ar išsenka šaltinio šaltas, tekantis vanduo? 
\par 15 Tačiau mano tauta pamiršo mane, tuštybėms jie smilko smilkalus! Jie nukrypo nuo savo seno kelio ir eina takais, o ne vieškeliu, 
\par 16 kad padarytų savo šalį dykuma, amžina pajuoka. Kiekvienas praeivis ja baisėsis ir kraipys galvą. 
\par 17 Kaip rytų vėjas Aš juos išsklaidysiu priešo akivaizdoje, nugarą atsuksiu jiems jų nelaimės dieną!” 
\par 18 Jie tarėsi: “Susirinkime ir surenkime prieš Jeremiją sąmokslą! Įstatymas tebėra pas kunigą, patarimas­pas išminčių, žodis­pas pranašą. Eikime ir pulkime jį liežuviu, nekreipkime dėmesio į jo kalbas”. 
\par 19 Viešpatie, saugok mane ir išgirsk, ką mano priešininkai kalba. 
\par 20 Argi atmokama už gera piktu? Jie kasa duobę mano gyvybei. Atsimink, kaip aš stovėjau Tavo akivaizdoje ir kalbėjau už juos, kad nukreipčiau nuo jų Tavo rūstybę! 
\par 21 Jų vaikai tebadauja, atiduok juos kardo valiai. Jų žmonos tenetenka vaikų ir tetampa našlėmis, jų vyrai­tepražūna, o jaunuoliai tekrinta kovoje. 
\par 22 Tegul pasigirsta šauksmas iš jų namų, kai juos staiga užpuls priešai! Jie kasė duobę, norėdami mane pagauti, ir slapta dėjo man spąstus. 
\par 23 Viešpatie, tu žinai visą jų sąmokslą nužudyti mane. Neatleisk jų kalčių ir nuodėmių nepašalink! Tebūna jie parblokšti Tavo akivaizdoje! Nubausk juos savo rūstybės metu!



\chapter{19}


\par 1 Viešpats sako: “Eik, nusipirk iš puodžiaus molinį ąsotį ir paimk su savimi kelis vyresniuosius kunigus ir kelis tautos vyresniuosius. 
\par 2 Nuėjęs į Ben Hinomo slėnį, kuris yra prie Šukių vartų, skelbk žodžius, kuriuos tau kalbėsiu, 
\par 3 ir sakyk: ‘Klausykitės Viešpaties žodžio, Judo karaliai ir Jeruzalės gyventojai! Taip sako kareivijų Viešpats, Izraelio Dievas: ‘Aš užleisiu ant šitos vietos tokią nelaimę, kad kiekvienam, kuris išgirs apie ją, suspengs ausyse. 
\par 4 Jie paliko mane ir sutepė šitą vietą, smilkė joje kitiems dievams, kurių nepažino nei jie, nei jų tėvai, nei Judo karaliai. Jie pripildė šitą vietą nekalto kraujo. 
\par 5 Jie statė Baalui aukurus, ant kurių degino savo vaikus kaip aukas Baalui; to Aš neįsakiau ir tai man niekada neatėjo į galvą. 
\par 6 Todėl ateis diena, kai šita vieta nesivadins Tofetu ar Ben Hinomo slėniu, bet žudynių slėniu. 
\par 7 Aš nieku paversiu Judo ir Jeruzalės išmintį, jie kris nuo kardo priešų akivaizdoje ir nuo rankos tų, kurie siekia jų gyvybės. Aš atiduosiu jų lavonus padangių paukščiams ir laukiniams žvėrims. 
\par 8 Aš padarysiu šitą miestą dykuma ir pajuoka: kiekvienas, praeinantis pro jį, baisėsis jo nelaimėmis. 
\par 9 Jie valgys kūnus savo vaikų ir vienas kito, kai priešai juos apguls ir prispaus tie, kurie siekia jų gyvybės’. 
\par 10 Sudaužyk ąsotį su tavimi atėjusių vyrų akivaizdoje 
\par 11 ir jiems sakyk: ‘Taip sako kareivijų Viešpats: ‘Aš sudaužysiu šitą tautą ir miestą lygiai taip, kaip sudaužomas puodžiaus indas, ir jo nebegalima sulipdyti. Jie laidos Tofete, kol nebeliks jame vietos. 
\par 12 Aš padarysiu su šita vieta ir su jos gyventojais taip, kad šitas miestas bus panašus į Tofetą. 
\par 13 Jeruzalės ir Judo karalių namai bus suteršti kaip Tofeto vieta, nes ant stogų jie smilkė dangaus kareivijai ir aukojo geriamąsiais aukas svetimiems dievams’ ”. 
\par 14 Jeremijas, sugrįžęs iš Tofeto, kur Viešpats jį buvo siuntęs pranašauti, atsistojo Viešpaties namų kieme ir kalbėjo visai tautai: 
\par 15 “Taip sako kareivijų Viešpats, Izraelio Dievas: ‘Aš užleisiu ant šito miesto ir ant visų jo miestų nelaimę, kurią paskelbiau, nes jie yra kietasprandžiai ir neklauso mano žodžio’ ”.



\chapter{20}


\par 1 Kunigas Pašhūras, Imero sūnus, vyriausiasis Viešpaties namų prižiūrėtojas, girdėjo Jeremiją pranašaujant. 
\par 2 Tada, nuplakdinęs pranašą Jeremiją, įtvėrė jį į šiekštą viršutiniuose Benjamino vartuose, prie Viešpaties namų. 
\par 3 Kai rytą Pašhūras išlaisvino Jeremiją iš šiekšto, Jeremijas jam tarė: “Ne Pašhūru tave pavadino Viešpats, bet siaubu. 
\par 4 Nes taip sako Viešpats: ‘Aš padarysiu tave siaubu tau ir visiems tavo draugams; jie kris nuo priešų kardo tau matant, o visą Judą atiduosiu į Babilono karaliaus rankas, kuris juos ištrems ir žudys. 
\par 5 Aš atiduosiu visus šito miesto turtus, visas jo atsargas, brangenybes ir visus Judo karalių lobius į jų priešų rankas, kurie juos pasiims ir išgabens į Babiloną. 
\par 6 O tu, Pašhūrai, ir visi tavo namai būsite išvesti į nelaisvę. Į Babiloną tave nuvarys, kur tu ir tavo draugai, kuriems pranašavai melus, mirsite ir būsite palaidoti’ ”. 
\par 7 Viešpatie, Tu suklaidinai mane, ir aš esu suklaidintas. Tu stipresnis už mane ir nugalėjai. Aš esu pajuokiamas kasdien, kiekvienas tyčiojasi iš manęs. 
\par 8 Kiekvieną kartą, kai kalbu, turiu šaukti: “Plėšimas, smurtas!” Viešpaties žodis tapo man plūdimu ir kasdienėmis patyčiomis. 
\par 9 Aš galvojau: “Nebeminėsiu Jo ir nebekalbėsiu Jo vardu”. Tačiau Jo žodis mano širdyje buvo tarsi ugnis, uždaryta mano kauluose, aš stengiausi susilaikyti, bet negalėjau. 
\par 10 Aš girdėjau daugelį šnibždant: “Įskųskime jį!” Visi mano artimi draugai laukia mano suklupimo: “Gal jis leisis suviliojamas, tada jį nugalėsime ir jam atkeršysime!” 
\par 11 Bet Viešpats yra su manimi kaip galingas karžygys! Todėl mano persekiotojai suklups ir nieko nelaimės. Jie bus labai sugėdinti ir jiems nepavyks, jų gėda nebus pamiršta per amžius. 
\par 12 Kareivijų Viešpatie, kuris mėgini teisųjį, matai jo inkstus ir širdį, leisk man matyti Tavo kerštą jiems, nes aš Tau patikėjau savo bylą! 
\par 13 Giedokite Viešpačiui, girkite Viešpatį, nes Jis išgelbėjo vargšo gyvybę iš piktadarių rankų! 
\par 14 Prakeikta diena, kurią gimiau. Diena, kurią mane pagimdė motina, tenebūna palaiminta! 
\par 15 Prakeiktas žmogus, kuris pranešė mano tėvui žinią: “Tau gimė sūnus!”, ir jį labai pradžiugino. 
\par 16 Tebūna tas žmogus kaip miestai, kuriuos Viešpats nesigailėdamas sunaikino. Tegirdi jis šauksmą rytą ir vaitojimą vidudienį 
\par 17 dėl to, kad nenužudė manęs dar įsčiose, kad mano motina būtų man kapu ir būtų likusi nėščia amžinai! 
\par 18 Kodėl turėjau gimti, patirti vargą bei sielvartą ir praleisti savo dienas gėdoje?



\chapter{21}


\par 1 Viešpats kalbėjo Jeremijui, kai karalius Zedekijas siuntė pas jį Malkijos sūnų Pašhūrą ir kunigą Sofoniją, Maasėjos sūnų: 
\par 2 “Babilono karalius Nebukadnecaras kariauja prieš mus; paklausk už mus Viešpatį, gal Jis padarys stebuklą kaip seniau ir priešas turės atsitraukti nuo mūsų?” 
\par 3 Jeremijas jiems atsakė: “Taip sakykite Zedekijui: 
\par 4 ‘Viešpats, Izraelio Dievas, sako: ‘Aš apgręšiu kovos ginklus jūsų rankose, kuriais kariaujate prieš Babilono karalių ir chaldėjus, jus apgulusius, ir juos surinksiu viduryje šio miesto. 
\par 5 Aš pats, įpykęs, užsirūstinęs ir įnirtęs, kariausiu prieš jus ištiesta galinga ranka 
\par 6 ir ištiksiu šito miesto žmones bei gyvulius. Jie mirs nuo didelio maro’. 
\par 7 Viešpats sako: ‘Po to Aš atiduosiu Zedekiją, Judo karalių, jo tarnus ir miesto gyventojus, išlikusius nuo maro, kardo ir bado, į Babilono karaliaus Nebukadnecaro rankas ir rankas tų, kurie siekia jų gyvybės. Jis žudys juos kardu, nieko nesigailėdamas ir neužjausdamas’ ”. 
\par 8 Šitai tautai sakyk: “Taip sako Viešpats: ‘Aš leidžiu jums pasirinkti gyvenimo kelią ar mirties kelią. 
\par 9 Kas pasiliks šitame mieste, mirs nuo kardo, bado ir maro, o kas išeis ir pasiduos chaldėjams, jus apgulusiems, išliks gyvas ir išgelbės savo gyvybę. 
\par 10 Aš atsigręžiau į šitą miestą daryti pikta, o ne gera,­sako Viešpats,­į Babilono karaliaus rankas jis bus atiduotas, o tas jį sudegins’ ”. 
\par 11 Judo karalių namams sakyk: “Klausykite Viešpaties žodžio, 
\par 12 Dovydo namai: ‘Taip sako Viešpats: ‘Elkitės kasdien teisingai, ginkite skriaudžiamuosius, kad mano rūstybė neužsidegtų neužgesinama ugnimi dėl jūsų piktadarysčių’ ”. 
\par 13 “Aš esu prieš tave, slėnio gyventoja, lygumos uola,­sako Viešpats.­Jūs sakote: ‘Kas galėtų mus užpulti ir įsibrauti į mūsų buveines?’ 
\par 14 Aš nubausiu jus pagal jūsų darbus,­sako Viešpats,­įžiebsiu ugnį jūsų miške, kad sudegintų visą apylinkę”.



\chapter{22}


\par 1 Taip sako Viešpats: “Eik į Judo karaliaus rūmus ir ten kalbėk šiuos žodžius: 
\par 2 ‘Judo karaliau, kuris sėdi Dovydo soste, tavo tarnai ir žmonės, kurie įeinate pro šituos vartus, klausykite Viešpaties žodžio: 
\par 3 ‘Elkitės teisiai ir teisingai, ginkite skriaudžiamuosius, o svetimšalio, našlaičio ir našlės neskriauskite ir nenaudokite prieš juos prievartos, nepraliekite nekalto kraujo šitoje vietoje. 
\par 4 Jei jūs tikrai taip darysite, pro šitų namų vartus įeis karaliai, sėdintieji Dovydo soste, jie, jų tarnai ir tauta važiuos vežimais ir jos ant žirgų. 
\par 5 Bet jei neklausysite šitų žodžių, tai prisiekiu savimi,­sako Viešpats,­kad šitie namai pavirs griuvėsiais’ ”. 
\par 6 Judo karaliaus namams Viešpats sako: “Tu esi man Gileado kraštas, aukščiausia Libano viršūnė, bet Aš padarysiu tave dykuma ir negyvenamais miestais. 
\par 7 Aš siųsiu naikintojų prieš tave, kurie savo įrankiais nukirs tavo rinktinius kedrus ir sudegins juos. 
\par 8 Daugelis tautų eis pro šitą miestą, klausinėdami vienas kito: ‘Kodėl Viešpats taip padarė šitam dideliam miestui?’ 
\par 9 Jiems atsakys: ‘Kadangi jie paliko Viešpaties, savo Dievo, sandorą, garbino svetimus dievus ir jiems tarnavo’ ”. 
\par 10 Neverkite mirusio ir neraudokite dėl jo. Verkite to, kuris išėjo, nes jis nebesugrįš ir nebematys savo gimtojo krašto. 
\par 11 Nes taip sako Viešpats apie Jozijo sūnų Šalumą, Judo karalių, karaliavusį savo tėvo Jozijo vietoje: “Jis išėjo iš šitos vietos ir nebesugrįš; 
\par 12 ten jis ir mirs savo nelaisvės vietoje, neišvydęs savo šalies. 
\par 13 Vargas statančiam namus neteisybe ir suktybėmis, kuris verčia artimą dirbti jam ir neatlygina už darbą, 
\par 14 kuris sako: ‘Aš pasistatysiu didelius namus erdviais kambariais’, išsikerta langus, apkala sienas kedro lentomis ir nudažo raudonai. 
\par 15 Ar tu karaliausi dėl to, kad pasistatei kedro namus? Ar tavo tėvas nevalgė ir negėrė, nebuvo teisus ir teisingas, ir ar ne dėl to jam sekėsi? 
\par 16 Jis gynė vargšo ir beturčio teises, ir jam sekėsi. Tai ir yra mano pažinimas,­sako Viešpats.­ 
\par 17 Tavo akys ir širdis linkusios į godumą ir trokšta pralieti nekaltą kraują, vykdyti priespaudą bei prievartą”. 
\par 18 Todėl apie Jozijo sūnų Jehojakimą, Judo karalių, Viešpats sako: “Jo neapraudos, sakydami: ‘O mano broli! O sesuo!’, arba: ‘O valdove! O jūsų didenybe!’ 
\par 19 Jis bus palaidotas kaip asilas, nuvilktas ir numestas už Jeruzalės vartų”. 
\par 20 “Eik į Libaną ir šauk, garsiai dejuok Bašane, verk Abarimo kalnyne, nes sunaikinti visi tavo meilužiai. 
\par 21 Aš kalbėjau tau, kai tu dar klestėjai, bet tu neklausei. Taip tu elgeisi nuo pat savo jaunystės, neklausydama mano balso. 
\par 22 Visus tavo ganytojus nuneš vėjas, visi tavo meilužiai bus ištremti. Tada tu gėdysies ir rausi dėl savo nedorybių. 
\par 23 Tu susikrovei lizdą Libano kedruose. Kaip tu vaitosi, kai tave suims gimdymo skausmai! 
\par 24 Kaip Aš gyvas,­sako Viešpats,­ nors Jehojakimo sūnus Konijas, Judo karalius, būtų antspaudo žiedas ant mano dešinės, Aš jį nutraukčiau. 
\par 25 Aš atiduosiu tave į rankas tų, kurie siekia tavo gyvybės, kurių tu bijai, į Babilono karaliaus Nebukadnecaro ir į chaldėjų rankas. 
\par 26 Aš išmesiu tave ir tavo motiną, kuri tave pagimdė, į šalį, kuri nėra jūsų gimtinė, ir ten jūs mirsite. 
\par 27 Bet į šalį, kur jūs norėsite sugrįžti, nebesugrįšite”. 
\par 28 Argi tas žmogus Konijas yra paniekintas stabas? Argi jis indas, kuris niekam nepatinka? Kodėl jis ir jo vaikai ištremti į šalį, apie kurią jie nieko nežinojo? 
\par 29 O žeme, žeme! Klausykis Viešpaties žodžio! 
\par 30 Taip sako Viešpats: “Užrašykite šitą vyrą kaip bevaikį, kuriam nesiseka gyvenime, nes iš jo palikuonių nė vienas nesėdės Dovydo soste ir nevaldys Judo”.



\chapter{23}


\par 1 “Vargas ganytojams, kurie mano ganyklos avis naikina ir išsklaido!”­sako Viešpats. 
\par 2 Todėl taip sako Viešpats, Izraelio Dievas, apie savo tautos ganytojus: “Jūs išsklaidėte mano avis ir jų neaplankėte. Todėl Aš jus aplankysiu dėl jūsų piktų darbų. 
\par 3 Aš pats surinksiu savo bandos likutį iš visų šalių, į kurias Aš jas išvariau, ir sugrąžinsiu į ganyklą; ten jos bus vaisingos ir dauginsis. 
\par 4 Aš paskirsiu joms ganytojų, kurie tikrai jas ganys. Jos nebebijos ir nesibaimins, nė vienos iš jų netrūks,­sako Viešpats”. 
\par 5 “Ateis diena,­sako Viešpats,­ kai Aš išauginsiu teisią atžalą iš Dovydo giminės. Jis viešpataus kaip karalius, elgsis išmintingai, vykdys teisingumą bei teismą krašte. 
\par 6 Jo dienomis Judas bus išgelbėtas ir Izraelis gyvens saugiai. Jo vardas bus: ‘Viešpats, mūsų teisumas’. 
\par 7 Tuomet nebesakys: ‘Kaip gyvas Viešpats, kuris išvedė izraelitus iš Egipto krašto’, 
\par 8 bet sakys: ‘Kaip gyvas Viešpats, kuris išvedė ir parvedė Izraelio palikuonis iš šiaurės ir iš visų kraštų, į kuriuos Aš juos buvau išvaręs’. Ir jie gyvens savo žemėje”. 
\par 9 Mano širdis plyšta krūtinėje dėl pranašų, visi mano kaulai dreba. Aš esu kaip girtas vyras, kaip įveiktas vyno vyras dėl Viešpaties ir Jo šventų žodžių. 
\par 10 “Šalis pilna svetimautojų; dėl jų prakeikimo gedi kraštas, išdžiūvusios stepių ganyklos. Jų siekiai yra pikti, jų jėga­neteisybė. 
\par 11 Jų pranašai ir kunigai yra bedieviai, net savo namuose aptinku jų nedorybę”,­sako Viešpats. 
\par 12 “Jų kelias yra slidus kelias tamsoje, kuriuo eidami, jie paslys ir grius. Aš siųsiu sunaikinimą ir bausiu juos”,­sako Viešpats. 
\par 13 Tarp Samarijos pranašų mačiau kvailystę: jie pranašavo Baalo vardu ir klaidino mano tautą, Izraelį. 
\par 14 Tarp Jeruzalės pranašų mačiau baisių dalykų: svetimavimą ir melagystę. Jie padrąsina piktadarius, kad jie nenusigręžtų nuo savo nedorybių. Visi yra kaip Sodomos ir Gomoros gyventojai. 
\par 15 Todėl kareivijų Viešpats taip sako apie pranašus: “Aš juos valgydinsiu metėlėmis ir girdysiu karčiu vandeniu, nes nuo Jeruzalės pranašų bedievystė pasklido visoje šalyje”. 
\par 16 Taip sako Viešpats: “Neklausykite pranašų, jums pranašaujančių. Jie tik mulkina jus, kalbėdami prasimanytus regėjimus, o ne tai, kas ateina iš Viešpaties. 
\par 17 Jie sako tiems, kurie niekina mane: ‘Taip kalba Viešpats: ‘Jūs turėsite ramybę’, ir kiekvienam, kuris seka savo širdies užgaidas sako: ‘Jums neatsitiks nieko pikto’ ”. 
\par 18 Kas buvo Viešpaties pasitarime, kas Jį matė ir girdėjo Jo žodį? Kas klausėsi Jo žodžio ir Jį išgirdo? 
\par 19 Ateina Viešpaties pykčio audra, viesulas užgrius ant nedorėlių galvų. 
\par 20 Nenusigręš Viešpaties rūstybė. Jis įvykdys visa, ką savo širdyje sumanė. Paskutinėmis dienomis jūs tai aiškiai suprasite. 
\par 21 “Aš nesiunčiau šitų pranašų, jie patys bėgo! Nekalbėjau jiems, tačiau jie pranašavo! 
\par 22 Jei jie būtų buvę mano pasitarime ir skelbtų mano žodžius tautai, jie būtų nukreipę juos nuo jų pikto kelio ir nuo jų piktų darbų. 
\par 23 Argi Aš esu Dievas tiktai arti, o toli nebesu Dievas? 
\par 24 Argi gali kas taip pasislėpti, kad Aš jo nematyčiau?­sako Viešpats.­Argi Aš nepripildau dangaus ir žemės? 
\par 25 Aš girdėjau, ką kalbėjo pranašai, pranašaudami mano vardu melą ir sakydami: ‘Sapnavau, sapnavau!’ 
\par 26 Ar ilgai tai bus širdyse pranašų, kurie pranašauja melą ir apgaulę? 
\par 27 Jie siekia, kad mano tauta pamirštų mane dėl jų sapnų, kuriuos jie pasakoja vieni kitiems, kaip jų tėvai pamiršo mane dėl Baalo. 
\par 28 Pranašas, kuris turi sapną, tepasakoja sapną, o kas turi mano žodį, teskelbia jį ištikimai. Ką šiaudai turi bendro su kviečių grūdais?”­sako Viešpats. 
\par 29 “Argi mano žodis nėra kaip ugnis? Argi jis nėra kaip kūjis, sutrupinantis uolą? 
\par 30 Aš esu prieš pranašus, kurie vagia mano žodžius vienas nuo kito! 
\par 31 Aš esu prieš pranašus, kurie savo liežuviu sako: ‘Jis pasakė’. 
\par 32 Aš esu prieš tuos, kurie pranašauja melagingus sapnus. Jie pasakoja ir klaidina mano tautą savo lengvabūdiškais melais. Aš nesiunčiau jų ir jiems neįsakiau; jie visi yra nenaudingi šitai tautai”,­sako Viešpats. 
\par 33 “Jei tave klaus šita tauta, pranašas ar kunigas: ‘Kokia Viešpaties našta?’, atsakyk jiems: ‘Kokia našta? Viešpats sako: ‘Aš jus atmesiu’. 
\par 34 Bet jei pranašas, kunigas ar kitas žmogus sakytų: ‘Viešpaties našta’,­tai Aš nubausiu jį ir jo namus. 
\par 35 Taip jūs privalote sakyti savo artimui ar savo broliui: ‘Ką atsakė Viešpats?’, arba: ‘Ką kalbėjo Viešpats?’ 
\par 36 Neminėkite daugiau Viešpaties naštos, nes kiekvienas žmogaus žodis bus jam našta; jūs iškraipote gyvojo Dievo, kareivijų Viešpaties, žodžius. 
\par 37 Jūs turite klausti pranašo: ‘Ką tau atsakė Viešpats?’, arba: ‘Ką kalbėjo Viešpats?’ 
\par 38 Bet kadangi jūs sakote: ‘Viešpaties našta’, nors Aš įsakiau nesakyti: ‘Viešpaties našta’, 
\par 39 Aš visiškai pamiršiu jus ir apleisiu jus ir miestą, kurį daviau jums ir jūsų tėvams, ir pašalinsiu iš savo akivaizdos. 
\par 40 Aš užtrauksiu jums amžiną pajuoką ir nepamirštamą gėdą”.



\chapter{24}


\par 1 Viešpats man parodė dvi figų pintines, padėtas prie Viešpaties šventyklos. Tai buvo po to, kai Babilono karalius Nebukadnecaras ištrėmė iš Jeruzalės į Babiloną Jehojakimo sūnų Jechoniją, Judo karalių, bei Judo kunigaikščius, kalvius ir amatininkus. 
\par 2 Vienoje pintinėje buvo labai gerų, ankstyvųjų figų, o kitoje pintinėje­labai blogų figų, kurių nebegalima valgyti. 
\par 3 Viešpats klausė manęs: “Ką matai, Jeremijau?” Aš atsakiau: “Figų. Gerosios figos labai geros, o blogosios tokios blogos, kad jų nebegalima valgyti”. 
\par 4 Tada Viešpats man kalbėjo: 
\par 5 “Taip sako Viešpats, Izraelio Dievas: ‘Kokios šitos gerosios figos, tokie man bus Judo tremtiniai, kuriuos pasiunčiau iš šitos vietos į chaldėjų šalį. 
\par 6 Aš palankiai žiūrėsiu į juos ir parvesiu juos atgal į šitą šalį, juos atstatysiu ir nebegriausiu, juos įsodinsiu ir neišrausiu. 
\par 7 Aš duosiu jiems širdis, kurios pažintų mane, kad Aš esu Viešpats. Jie bus mano tauta, o Aš jiems būsiu jų Dievas, nes jie grįš prie manęs visa širdimi. 
\par 8 Aš padarysiu su Judo karaliumi Zedekiju ir Jeruzalės gyventojais, pasilikusiais šitoje šalyje ir apsigyvenusiais Egipto krašte, kaip su blogosiomis figomis, kurių negalima valgyti. 
\par 9 Aš juos padarysiu baidykle visoms žemės karalystėms, gėda ir patarle, pašaipa ir keiksmažodžiu visose vietose, kur juos išvysiu. 
\par 10 Aš siųsiu jiems kardą, badą ir marą, kol jie bus išnaikinti krašte, kurį daviau jiems ir jų tėvams”.



\chapter{25}


\par 1 Žodis Jeremijui apie visą Judo tautą ketvirtaisiais Jozijo sūnaus Jehojakimo, Judo karaliaus, viešpatavimo metais. Tai buvo pirmieji Babilono karaliaus Nebukadnecaro metai. 
\par 2 Pranašas Jeremijas kalbėjo visai Judo tautai ir visiems Jeruzalės gyventojams: 
\par 3 “Nuo tryliktųjų Amono sūnaus Jozijo, Judo karaliaus, metų iki šios dienos, dvidešimt trejus metus, Viešpats kalbėjo man, ir aš skelbiau tai jums, keldamasis anksti rytą, bet jūs neklausėte. 
\par 4 Viešpats siuntė pas jus savo tarnus, pranašus, bet jūs neklausėte ir nekreipėte dėmesio. 
\par 5 Jie sakė: ‘Palikite kiekvienas savo piktą kelią ir piktus darbus, tai liksite gyventi žemėje, kurią Viešpats davė jums ir jūsų tėvams amžinai. 
\par 6 Nesekiokite svetimų dievų, netarnaukite jiems ir negarbinkite jų, nerūstinkite manęs savo rankų darbais, ir Aš jūsų nebausiu’. 
\par 7 ‘Bet jūs neklausėte manęs,­sako Viešpats,­ir savo rankų darbais užrūstinote mane savo nenaudai. 
\par 8 Todėl taip sako kareivijų Viešpats: ‘Kadangi jūs neklausėte mano žodžių, 
\par 9 tai Aš surinksiu visas šiaurės tautas pas Babilono karalių Nebukadnecarą, mano tarną, ir jas atvesiu į šitą šalį prieš jus ir visų šitų aplinkinių tautų gyventojus; Aš sunaikinsiu juos ir padarysiu pasibaisėjimu, pašaipa ir amžina dykyne. 
\par 10 Aš atimsiu iš jūsų džiaugsmo ir linksmybės balsus, jaunikio ir jaunosios balsus, girnų ūžesį ir žibintų šviesą. 
\par 11 Visas kraštas pavirs baisia dykuma, ir šios tautos tarnaus Babilono karaliui septyniasdešimt metų. 
\par 12 Po septyniasdešimties metų Aš nubausiu Babilono karalių, jo tautą ir chaldėjų šalį už jų nedorybes ir ją paversiu amžina dykyne. 
\par 13 Aš įvykdysiu viską, ką kalbėjau, kas parašyta šitoje knygoje, ką Jeremijas pranašavo apie tautas. 
\par 14 Jie tarnaus galingoms tautoms ir jų karaliams. Aš jiems atlyginsiu pagal jų nusikaltimus’ ”. 
\par 15 Viešpats, Izraelio Dievas, man įsakė: “Imk šitą rūstybės vyno taurę iš mano rankos ir girdyk juo visas tautas, pas kurias Aš tave siunčiu. 
\par 16 Jos gers, svyrinės ir bus pamišusios dėl kardo, kurį siųsiu jiems!” 
\par 17 Aš ėmiau taurę iš Viešpaties rankos ir girdžiau visas tautas, pas kurias Viešpats mane siuntė: 
\par 18 Jeruzalę, Judo miestus, jo karalius ir kunigaikščius, kad jie taptų apleisti, pašaipa ir keiksmu iki šios dienos. 
\par 19 Faraoną, Egipto karalių, jo tarnus, kunigaikščius, visą tautą 
\par 20 ir visus ten gyvenančius svetimtaučius; visus Uco šalies ir filistinų šalies karalius, Aškeloną, Gazą, Ekroną ir Ašdodo likutį; 
\par 21 Edomą, Moabą ir amonitus; 
\par 22 visus Tyro ir Sidono karalius, taip pat salų ir pajūrio karalius; 
\par 23 Dedaną, Temą, Būzą ir visus esančius tolimiausiuose pakraščiuose; 
\par 24 visus Arabijos karalius, kurie gyvena dykumoje; 
\par 25 visus Zimrio, Elamo ir medų karalius; 
\par 26 visus šiaurės karalius, artimuosius ir tolimuosius, vieną po kito, ir visas karalystes, kurios yra visoje žemėje. O Šešacho karalius gers paskutinis. 
\par 27 “Tu jiems sakyk: ‘Taip sako kareivijų Viešpats, Izraelio Dievas: ‘Gerkite, pasigerkite ir vemkite! Kriskite nuo kardo, kurį siunčiu jums, ir nebeatsikelkite!’ 
\par 28 Jei jie nesutiks imti taurės iš tavo rankos ir gerti, tai jiems sakyk: ‘Taip sako kareivijų Viešpats: ‘Jūs tikrai gersite!’ 
\par 29 Mieste, kuris vadinamas mano vardu, Aš pradedu vykdyti bausmę, o jūs tikitės likti nenubausti? Jūs neliksite nenubausti, nes Aš pašauksiu kardą visiems žemės gyventojams’. 
\par 30 Pranašauk prieš juos visus šiuos žodžius: ‘Viešpats grūmoja iš aukštybės, iš Jo šventos buveinės skamba Jo balsas. Jis grūmoja savo kaimenei, šūkauja kaip vynuogių mynėjai. 
\par 31 Visus žemės pakraščius pasieks garsas, nes Viešpats teis visas tautas, visą žmoniją ir nedorėlius jis atiduos kardui’ ”,­sako Viešpats. 
\par 32 Taip sako kareivijų Viešpats: “Nelaimė eina iš tautos į tautą, didelė audra kyla nuo žemės pakraščių. 
\par 33 Viešpaties užmuštųjų tą dieną bus pilna žemė. Jų neapraudos, nesurinks ir nepalaidos­jie bus mėšlas dirvai tręšti. 
\par 34 Kaimenės vedliai ir ganytojai, šaukite ir voliokitės pelenuose! Atėjo metas skersti ir išsklaidyti jus, jūs krisite kaip gražūs indai. 
\par 35 Kaimenės ganytojai ir vedliai neturės galimybės pabėgti nė išsigelbėti. 
\par 36 Štai ganytojų šauksmas ir kaimenės vedlių verksmas, nes Viešpats sunaikina jų ganyklą. 
\par 37 Sunaikintos yra ramios buveinės dėl Viešpaties rūstybės užsidegimo. 
\par 38 Jis paliko savo namus kaip liūtas, jų šalis virto dykyne dėl prispaudėjų žiaurumo, dėl Jo rūstybės užsidegimo”.



\chapter{26}


\par 1 Jozijo sūnaus Jehojakimo, Judo karaliaus, karaliavimo pradžioje Viešpats kalbėjo: 
\par 2 “Atsistok Viešpaties namų kieme ir kalbėk visiems Judo miestams, atėjusiems į Viešpaties namus pagarbinti. Visa, ką tau įsakiau, pasakyk jiems, nepraleisk nė žodžio! 
\par 3 Gal jie klausysis ir atsisakys savo piktų kelių, kad galėčiau atsisakyti bausmės, kurią esu jiems numatęs. 
\par 4 Jiems sakyk: ‘Taip sako Viešpats: ‘Jei manęs neklausysite ir nesilaikysite mano įstatymo, kurį jums daviau, 
\par 5 neklausysite mano tarnų pranašų, kuriuos, anksti keldamas, jums siunčiu, nors neklausote jų, 
\par 6 tai Aš padarysiu šituos namus kaip Šiloją ir šitą miestą keiksmažodžiu visoms tautoms’ ”. 
\par 7 Kunigai, pranašai ir visa tauta girdėjo Jeremiją kalbant šiuos žodžius Viešpaties namuose. 
\par 8 Kai Jeremijas baigė kalbėti, ką Viešpats jam buvo įsakęs pranešti visai tautai, jį nutvėrė kunigai, pranašai ir visa tauta, šaukdami: “Tu turi mirti! 
\par 9 Kodėl pranašavai Viešpaties vardu, kad šitie namai bus kaip Šilojas ir miestas bus nebegyvenamas?” Visa tauta susibūrė prieš Jeremiją prie Viešpaties namų. 
\par 10 Kai Judo kunigaikščiai išgirdo apie tai, jie nuėjo iš karaliaus namų į Viešpaties namus ir atsisėdo Viešpaties namų Naujuosiuose vartuose. 
\par 11 Kunigai ir pranašai kalbėjo kunigaikščiams ir visai tautai: “Šitas vyras nusipelnė mirties, nes jis pranašavo prieš šitą miestą, kaip jūs girdėjote savo ausimis”. 
\par 12 Jeremijas atsakė kunigaikščiams ir visai tautai: “Viešpats siuntė mane pranašauti prieš šituos namus ir šitą miestą tais žodžiais, kuriuos jūs girdėjote. 
\par 13 Taigi dabar pakeiskite savo kelius bei darbus ir klausykite Viešpaties, savo Dievo, tai Jis pasigailės jūsų ir nesiųs sunaikinimo. 
\par 14 O aš esu jūsų rankose, darykite su manimi, kaip jums atrodo tinkama. 
\par 15 Bet žinokite, jei jūs mane nužudysite, nekaltą kraują užsitrauksite ant savęs, ant šito miesto ir jo gyventojų, nes Viešpats tikrai mane siuntė kalbėti jums visus šituos žodžius”. 
\par 16 Kunigaikščiai ir visa tauta atsakė kunigams ir pranašams: “Šitas vyras nenusipelnė mirties, nes jis mums kalbėjo Viešpaties, mūsų Dievo, vardu”. 
\par 17 Kai kurie krašto vyresnieji kalbėjo tautos susirinkimui: 
\par 18 “Michėjas iš Morešeto pranašavo Judo karaliaus Ezekijo dienomis visai Judo tautai Viešpaties vardu, kad Sionas bus ariamas kaip laukas, Jeruzalė pavirs griuvėsiais, o šventyklos kalnas apaugs mišku. 
\par 19 Argi karalius Ezekijas ir visi Judo žmonės jį nužudė? Jie pabūgo Viešpaties ir maldavo Jį taip, kad Viešpats susilaikė nuo bausmės, kurią Jis jiems buvo paskelbęs. O mes siekiame užsitraukti didelę nelaimę!” 
\par 20 Tuo laiku buvo dar vienas vyras, Šemajo sūnus Ūrijas iš Kirjat Jarimo, kuris pranašavo Viešpaties vardu prieš šitą miestą ir kraštą taip, kaip Jeremijas. 
\par 21 Ir kai karalius Jehojakimas, visi jo kariai ir kunigaikščiai išgirdo jo žodžius, karalius ieškojo jo, kad nužudytų. Tai išgirdęs, Ūrijas nusigando ir pabėgo į Egiptą. 
\par 22 Karalius Jehojakimas siuntė Achboro sūnų Elnataną su palyda į Egiptą. 
\par 23 Tie atvedė Ūriją iš Egipto pas karalių Jehojakimą, o tas, jį nužudęs kardu, išmetė jo lavoną į prastuomenės kapines. 
\par 24 Tačiau Safano sūnaus Ahikamo ranka Jeremijas buvo apgintas ir neatiduotas miniai nužudyti.



\chapter{27}


\par 1 Jozijo sūnaus Jehojakimo, Judo karaliaus, karaliavimo pradžioje Viešpats įsakė Jeremijui: 
\par 2 “Pasidaryk pančių bei jungų, užsidėk juos ant kaklo 
\par 3 ir pasiųsk juos Edomo, Moabo, Amono, Tyro ir Sidono karaliams per jų pasiuntinius, atvykusius į Jeruzalę pas Judo karalių Zedekiją. 
\par 4 Jiems įsakyk pranešti jų valdovams kareivijų Viešpaties, Izraelio Dievo, žodžius: ‘Taip sakykite savo valdovams: 
\par 5 ‘Aš sukūriau žemę su žmonėmis ir gyvuliais, esančiais ant žemės, savo didžia galia ir ištiesta ranka. Ir Aš duodu ją tam, kuriam Aš noriu. 
\par 6 Dabar Aš atidaviau visas šalis mano tarnui, Babilono karaliui Nebukadnecarui, net laukinius žvėris jam atidaviau, kad jam tarnautų. 
\par 7 Jam tarnaus visos tautos, jo sūnui ir jo sūnaus sūnui, kol ateis jo šalies metas; daug tautų ir galingų karalių tarnaus jam. 
\par 8 Tautą ir karalystę, kuri Babilono karaliui netarnaus ir nepalenks sprando po jo jungu, Aš bausiu kardu, badu ir maru,­sako Viešpats,­kol išnaikinsiu jo rankomis. 
\par 9 Todėl neklausykite savo pranašų, žynių, sapnuotojų, burtininkų ir kerėtojų, kurie jums sako: ‘Jūs netarnausite Babilono karaliui’. 
\par 10 Jie jus apgaudinėja, kad pašalintų jus iš jūsų žemės, Aš jus ištremčiau ir jūs žūtumėte. 
\par 11 O tautas, kurios palenks sprandus po Babilono karaliaus jungu ir tarnaus jam, Aš paliksiu jų žemėje, kad jos ją dirbtų ir joje gyventų,­sako Viešpats’ ”. 
\par 12 Judo karaliui Zedekijui aš kalbėjau tais pačiais žodžiais: “Palenkite sprandus po Babilono karaliaus jungu ir tarnaukite jam bei jo tautai, tada išliksite gyvi! 
\par 13 Kodėl tau, karaliau, ir tavo tautai mirti nuo kardo, bado ir maro, kaip Viešpats grasino tautai, kuri netarnaus Babilono karaliui? 
\par 14 Neklausykite pranašų, kurie jums sako: ‘Jūs netarnausite Babilono karaliui’. Jie jums meluoja. 
\par 15 ‘Aš jų nesiunčiau,­sako Viešpats,­tačiau jie pranašauja melus mano vardu, kad Aš jus išvaryčiau ir jūs žūtumėte, jūs ir pranašai, kurie jums pranašauja’ ”. 
\par 16 Kunigams ir visai tautai aš kalbėjau: “Taip sako Viešpats: ‘Neklausykite savo pranašų, kurie jums pranašauja, kad Viešpaties namų indai bus netrukus pargabenti iš Babilono. Jie pranašauja jums melą. 
\par 17 Neklausykite jų! Tarnaukite Babilono karaliui, tai išliksite gyvi. Kodėl šitas miestas turėtų pavirsti griuvėsiais? 
\par 18 Jei jie pranašai ir žino Viešpaties žodį, tai tegul užtaria prieš kareivijų Viešpatį, kad indai, pasilikę Viešpaties namuose, Judo karaliaus namuose ir Jeruzalėje, nepatektų į Babiloną’. 
\par 19 Kareivijų Viešpats taip sako apie kolonas, baseiną, stovus ir indus, pasilikusius šitame mieste, 
\par 20 kurių Babilono karalius Nebukadnecaras, išvesdamas į nelaisvę Jehojakimo sūnų Jechoniją, Judo karalių, ir visus Judo bei Jeruzalės kilminguosius, nepaėmė iš Jeruzalės į Babiloną. 
\par 21 Taip sako kareivijų Viešpats, Izraelio Dievas, apie indus, pasilikusius Viešpaties namuose, Judo karaliaus namuose ir Jeruzalėje: 
\par 22 ‘Jie bus nugabenti į Babiloną ir ten liks, kol juos aplankysiu,­sako Viešpats.­Tada Aš juos atgabensiu ir sugrąžinsiu į šitą vietą’ ”.



\chapter{28}


\par 1 Ketvirtaisiais Judo karaliaus Zedekijo karaliavimo metais, penktąjį mėnesį, Azūro sūnus Hananija, pranašas, kilęs iš Gibeono, kalbėjo man Viešpaties namuose kunigų ir visos tautos akivaizdoje: 
\par 2 “Taip sako kareivijų Viešpats, Izraelio Dievas: ‘Aš sulaužau Babilono karaliaus jungą. 
\par 3 Po dvejų metų Aš sugrąžinsiu į šitą vietą visus Viešpaties namų indus, kuriuos Babilono karalius Nebukadnecaras paėmė iš šitos vietos ir nugabeno į Babiloną. 
\par 4 Taip pat ir Jehojakimo sūnų Jechoniją, Judo karalių, ir visus Judo tremtinius iš Babilono Aš sugrąžinsiu į šitą vietą, nes Aš sulaužysiu Babilono karaliaus jungą’ ”. 
\par 5 Pranašas Jeremijas atsakė pranašui Hananijai, kunigams ir visiems žmonėms, stovėjusiems Viešpaties namuose: 
\par 6 “Tebūna taip! Tegul padaro taip Viešpats! Teįvykdo Viešpats tavo žodžius, kuriuos pranašavai, tesugrąžina Viešpaties namų indus ir visus tremtinius iš Babilono į šitą vietą. 
\par 7 Dabar klausykis mano žodžio, kurį kalbėsiu tau visai tautai girdint: 
\par 8 ‘Pranašai, kurie buvo pirma manęs ir pirma tavęs, pranašavo galingoms šalims ir didelėms karalystėms apie karą, nelaimes ir marą. 
\par 9 Pranašas, kuris pranašaudavo apie taiką, tik jo žodžiams įvykus būdavo pripažįstamas kaip tikrai Viešpaties siųstas’ ”. 
\par 10 Tada pranašas Hananija nuėmė jungą nuo pranašo Jeremijo kaklo, sulaužė 
\par 11 ir visos tautos akivaizdoje tarė: “Taip sako Viešpats: ‘Po dvejų metų Aš taip sulaužysiu Babilono karaliaus Nebukadnecaro jungą ir jį nuimsiu nuo visų tautų kaklo’ ”. Pranašas Jeremijas nuėjo savo keliu. 
\par 12 Kai pranašas Hananija sulaužė jungą, nuėmęs jį nuo pranašo Jeremijo kaklo, Viešpats kalbėjo Jeremijui: 
\par 13 “Eik ir sakyk Hananijai: ‘Taip sako Viešpats: ‘Medinį jungą tu sulaužei, bet jo vietoje tu uždėsi jiems geležinį jungą. 
\par 14 Nes taip sako kareivijų Viešpats, Izraelio Dievas: ‘Aš uždėjau geležinį jungą visoms šioms tautoms, kad jos tarnautų Babilono karaliui Nebukadnecarui, ir jos tarnaus jam. Ir Aš taip pat atidaviau jam laukinius žvėris’ ”. 
\par 15 Pranašas Jeremijas tarė pranašui Hananijai: “Klausyk, Hananija! Viešpats tavęs nesiuntė, bet tu suvedžiojai šitą tautą. 
\par 16 Todėl Viešpats sako: ‘Aš tave pašalinsiu nuo žemės paviršiaus. Dar šiais metais tu mirsi, nes mokei sukilti prieš Viešpatį’ ”. 
\par 17 Pranašas Hananija mirė tų pačių metų septintą mėnesį.



\chapter{29}


\par 1 Tai yra žodžiai laiško, kurį Jeremijas pasiuntė iš Jeruzalės tremtinių vyresniesiems, kunigams ir visai tautai, kuriuos Nebukadnecaras išvedė iš Jeruzalės į Babiloną 
\par 2 (po to, kai karalius Jechonijas, karalienė, eunuchai, Judo ir Jeruzalės kunigaikščiai, kalviai ir amatininkai paliko Jeruzalę), 
\par 3 per Šafano sūnų Eleasą ir Hilkijos sūnų Gemariją, kuriuos Judo karalius Zedekijas siuntė pas Babilono karalių Nebukadnecarą. 
\par 4 “Kareivijų Viešpats, Izraelio Dievas, visiems tremtiniams, ištremtiems iš Jeruzalės į Babiloną, sako: 
\par 5 ‘Statykite namus ir gyvenkite juose, sodinkite sodus ir valgykite jų vaisius. 
\par 6 Veskite žmonas ir gimdykite sūnus bei dukteris; imkite savo sūnums žmonas ir savo dukteris išleiskite už vyrų; jos tegimdo sūnus ir dukteris, kad ten jūsų padaugėtų, o ne sumažėtų. 
\par 7 Siekite gerovės miestui, į kurį jus ištrėmiau, melskitės už jį, nes jo gerovėje ir jūs turėsite ramybę. 
\par 8 Nesiduokite suvedžiojami savo pranašų, kurie yra tarp jūsų, nė žynių, nekreipkite dėmesio į savo sapnus, kuriuos jūs sapnuojate. 
\par 9 Jie meluoja, pranašaudami mano vardu! Aš jų nesiunčiau,­sako Viešpats.­ 
\par 10 Tik išbuvus jums Babilone septyniasdešimtį metų, Aš aplankysiu jus ir ištesėsiu savo pažadą, parvesiu jus atgal į šitą vietą. 
\par 11 Aš žinau, kokius sumanymus turiu dėl jūsų,­sako Viešpats.­ Sumanymus jūsų gerovei, o ne nelaimėms, ir ateitį su viltimi. 
\par 12 Tada jūs šauksitės manęs ir melsitės, ir Aš jus išklausysiu. 
\par 13 Kai manęs ieškosite visa širdimi, rasite. 
\par 14 Jūs surasite mane, ir Aš sugrąžinsiu jus iš nelaisvės, surinksiu jus iš visų tautų ir visų vietų, kuriose jūs buvote ištremti, ir jus parvesiu į vietą, iš kurios jus ištrėmiau,­sako Viešpats’. 
\par 15 Jei jūs sakote: ‘Viešpats mums davė pranašą Babilone’, 
\par 16 tai paklausykite, ką Viešpats sako apie karalių, sėdintį Dovydo soste, apie tautą, gyvenančią šitame mieste, ir apie jūsų brolius, kurie nebuvo ištremti su jumis: 
\par 17 ‘Aš siunčiu jiems kardą, badą ir marą ir juos darau panašius į blogas figas, kurių negalima valgyti. 
\par 18 Aš juos persekiosiu kardu, badu ir maru ir juos padarysiu siaubu visoms žemės karalystėms, prakeikimu, pasibaisėjimu, pašaipa ir pajuoka visose tautose, į kurias juos išvariau, 
\par 19 nes jie neklausė mano žodžių, kuriuos jiems siunčiau per savo tarnus pranašus, keldamas juos anksti rytą. Bet jie nenorėjo klausyti,­sako Viešpats.­ 
\par 20 Visi tremtiniai, kuriuos Aš pasiunčiau iš Jeruzalės į Babiloną, klausykite mano žodžio. 
\par 21 Apie Ahabą, Kolajos sūnų ir apie Maasėjos sūnų Zedekiją, pranašaujančius jums melą Viešpaties vardu, kareivijų Viešpats, Izraelio Dievas, sako: ‘Aš juos atiduosiu į Babilono karaliaus Nebukadnecaro rankas, jis juos nužudys jums matant. 
\par 22 Judo tremtiniams Babilone šie bus keiksmažodžiu, ir jie sakys: ‘Viešpats tepadaro tau kaip Zedekijui ir Ahabui, kuriuos Babilono karalius iškepė ugnyje’. 
\par 23 Jie blogai elgėsi Izraelyje: svetimavo su savo artimų žmonomis ir kalbėjo melus mano vardu, ko jiems neliepiau. Aš tai žinau ir esu liudytojas,­sako Viešpats’ ”. 
\par 24 Nehelamiečiui Šemajui sakyk: 
\par 25 “Taip sako kareivijų Viešpats, Izraelio Dievas: ‘Kadangi tu siuntei savo vardu visai tautai Jeruzalėje, kunigui Sofonijai, Maasėjos sūnui, ir visiems kunigams tokio turinio raštus: 
\par 26 ‘Viešpats tave paskyrė kunigu Jehojados vietoje, kad prižiūrėtum Viešpaties namus ir kiekvieną pamišėlį, kuris apsimeta pranašu, pasodintum į kalėjimą ir įtvertum į šiekštą. 
\par 27 Kodėl tad nesudraudei anatotiečio Jeremijo, kuris jums pranašavo? 
\par 28 Jis juk pasiuntė mums į Babiloną tokią žinią: ‘Tremtis bus ilga! Statykite namus ir gyvenkite juose, sodinkite sodus ir valgykite jų vaisius’ ”. 
\par 29 Kunigas Sofonija perskaitė šitą laišką pranašui Jeremijui. 
\par 30 Tuomet Viešpats tarė Jeremijui: 
\par 31 “Pasiųsk tokią žinią visiems tremtiniams, kad Viešpats apie nehelamietį Šemają sako: ‘Kadangi Šemajas jums pranašavo, nors Aš jo nesiunčiau, ir įtikino jus savo melais, 
\par 32 tai Aš nubausiu nehelamietį Šemają ir jo palikuonis. Jis neturės nė vieno, kuris gyventų šioje tautoje ir matytų jos gerovę, kurią duosiu savo tautai, nes jis mokė sukilti prieš Viešpatį’ ”.



\chapter{30}


\par 1 Viešpats, Izraelio Dievas, kalbėjo Jeremijui: 
\par 2 “Užrašyk į knygą visus žodžius, kuriuos tau kalbėjau. 
\par 3 Ateina dienos, kai Aš sugrąžinsiu savo tautą, Izraelį ir Judą, iš nelaisvės. Aš juos parvesiu į šalį, kurią daviau jų tėvams, ir jie ją paveldės”. 
\par 4 Šitie yra Viešpaties žodžiai apie Izraelį ir Judą. 
\par 5 Viešpats sako: “Mes girdėjome išgąsčio ir baimės šūksnius, o ne taikos. 
\par 6 Klausykite ir apsvarstykite, ar vyras gali gimdyti? Kodėl matau vyrus, kurių rankos ant strėnų kaip gimdančios moters? Kodėl kiekvieno veidas išblyškęs? 
\par 7 Ateina didinga ir baisi diena Jokūbui, jai nėra lygios, tačiau jis bus išgelbėtas. 
\par 8 Tuomet Aš nuimsiu jungą nuo jo kaklo, sulaužysiu jį bei sutraukysiu pančius; jis nebevergaus svetimiems. 
\par 9 Jie tarnaus Viešpačiui, savo Dievui, ir karaliui Dovydui, kurį Aš jiems išugdysiu. 
\par 10 Nebijok, mano tarne Jokūbai, ir neišsigąsk, Izraeli! Aš išgelbėsiu tave ir tavo palikuonis iš tolimo krašto. Jokūbas sugrįš ir ramiai gyvens, bus saugus, ir niekas jo negąsdins. 
\par 11 Nes Aš esu su tavimi,­sako Viešpats,­kad išgelbėčiau tave. Aš padarysiu galą visoms tautoms, į kurias tave ištrėmiau, bet tavęs visai nesunaikinsiu. Tave nuplaksiu ir nepaliksiu tavęs nenubausto”. 
\par 12 Nes taip sako Viešpats: “Tu esi baisiai sužeistas, tavo žaizda nepagydoma. 
\par 13 Nėra kas rūpintųsi tavimi ir aprištų tavo žaizdas, tu neturi jokių vaistų. 
\par 14 Tavo meilužės pamiršo tave, nesiteirauja apie tave. Aš baudžiau tave kaip priešą dėl tavo nusikaltimų daugybės, dėl tavo nuodėmių gausos. 
\par 15 Ko šauki iš skausmo, dėl nepagydomų žaizdų? Už tavo daugybę nusikaltimų Aš tave nubaudžiau. 
\par 16 Visi, kurie ėdė tave, bus ėdami, visi tavo prispaudėjai eis į nelaisvę. Kas tave apiplėšė, bus apiplėštas, kurie grobė iš tavęs, tuos atiduosiu kaip grobį. 
\par 17 Aš atstatysiu tavo sveikatą, išgydysiu tavo žaizdas. Atstumtuoju vadino tave, sakydami: ‘Tai Sionas, kuriuo niekas nesirūpina’ ”. 
\par 18 Taip sako Viešpats: “Aš atstatysiu Jokūbo palapines ir pasigailėsiu jo gyvenviečių. Miestas bus atstatytas ant savo kalvos, ir rūmai stovės įprastoje vietoje. 
\par 19 Iš jų skambės padėkos giesmės ir linksmas klegesys. Aš juos padauginsiu, jų nebebus keletas. Aš pašlovinsiu juos, jie nebebus niekinami. 
\par 20 Jų vaikai bus kaip ankstesniais laikais, jų bendruomenė įsitvirtins mano akivaizdoje, ir Aš nubausiu visus jų prispaudėjus. 
\par 21 Jų kilmingieji bus iš jų pačių, ir kunigaikštis kils iš jų tarpo. Aš patrauksiu jį, ir jis ateis pas mane. Kas gi pats išdrįstų ateiti pas mane?­sako Viešpats.­ 
\par 22 Tada jūs būsite mano tauta, o Aš būsiu jūsų Dievas”. 
\par 23 Štai Viešpaties audra kyla, rūstybės sūkurys atūžia ant nedorėlių galvos! 
\par 24 Viešpaties rūstybė nenusigręš, kol Jis įvykdys, ką savo širdyje sumanė. Paskutinėmis dienomis jūs tai suprasite.



\chapter{31}


\par 1 “Anuo metu,­sako Viešpats,­Aš būsiu Dievas visoms Izraelio giminėms, ir jos bus mano tauta. 
\par 2 Tauta, išlikusi nuo kardo, rado malonę dykumoje, Izraelis, kai Aš ėjau suteikti jam poilsį”. 
\par 3 Dar prieš tai man pasirodė Viešpats, sakydamas: “Aš amžina meile tave pamilau, todėl esu tau ištikimas. 
\par 4 Aš atstatysiu tave, Izraelio mergaite. Tu vėl pasipuoši būgneliais ir linksmai šoksi su besidžiaugiančiais. 
\par 5 Tu vėl sodinsi vynuogynus Samarijos kalnuose. Kas juos sodins, tas ir naudosis jais. 
\par 6 Ateina diena, kai sargai šauks Efraimo kalnyne: ‘Kelkitės ir eikime į Sioną pas Viešpatį, mūsų Dievą!’ ” 
\par 7 Nes taip sako Viešpats: “Linksmai giedokite dėl Jokūbo, šaukite prieš tautų valdovą. Skelbkite, girkite ir sakykite: ‘Viešpatie, išgelbėk savo tautos Izraelio likutį!’ 
\par 8 Aš juos parvesiu iš šiaurės šalies ir surinksiu iš tolimiausių kraštų kartu su aklais ir raišais, nėščiomis ir gimdančiomis, didžiulė minia jų sugrįš! 
\par 9 Verkdami jie pareis. Aš paguosiu juos ir vesiu prie vandens upelių tiesiu keliu, kad jie nesukluptų. Aš esu Izraelio tėvas, o Efraimas yra mano pirmagimis”. 
\par 10 Tautos, klausykite Viešpaties žodžio ir skelbkite jį tolimiausiuose pajūriuose: “Kas išsklaidė Izraelį, Tas jį surinks ir saugos kaip ganytojas bandą. 
\par 11 Viešpats atpirko Izraelį, išgelbėjo jį iš stiprių rankų. 
\par 12 Jie ateis džiūgaudami į Siono kalną, susirinks prie Viešpaties gėrybių: javų, vyno, aliejaus, avių ir galvijų jauniklių. Jie bus kaip drėkinamas sodas ir nebepatirs skausmo. 
\par 13 Mergaitės iš džiaugsmo šoks, o jaunimas bei seneliai bus laimingi. Aš pakeisiu jų liūdesį džiaugsmu, paguosiu juos ir palinksminsiu po jų kančių. 
\par 14 Aš su pertekliumi pasotinsiu kunigų sielas ir mano tauta bus pasotinta mano gėrybėmis,­sako Viešpats”. 
\par 15 Taip sako Viešpats: “Ramoje girdėti dejavimai, graudūs verksmai. Tai Rachelė verkia savo vaikų ir nesileidžia paguodžiama, nes jų nebėra. 
\par 16 Nešauk balsu ir neliek ašarų, nes tavo vargas bus atlygintas,­ sako Viešpats.­Jie sugrįš iš priešo šalies. 
\par 17 Tavo ateičiai yra vilties. Tavo vaikai sugrįš į savo kraštą. 
\par 18 Aš girdžiu, kaip Efraimas dejuoja: ‘Nuplakei mane kaip nesuvaldomą veršį. Sugrąžink mane, ir aš sugrįšiu, nes Tu esi Viešpats, mano Dievas! 
\par 19 Kai aš buvau nuklydęs­atgailavau, kai buvau pamokytas­mušiausi į krūtinę. Aš gėdijuosi ir raustu dėl savo jaunystės darbų’. 
\par 20 Argi Efraimas ne mano mielas sūnus, argi jis ne mano mylimas vaikas? Nors kalbėjau prieš jį, Aš su meile jį prisimenu. Mano širdis ilgisi jo ir Aš tikrai jo pasigailėsiu,­sako Viešpats”. 
\par 21 Pasistatyk kelrodžių, paženklink kelius! Įsidėmėk kelią, kuriuo ėjai! Sugrįžk, Izraelio mergaite, į savo miestus! 
\par 22 Ar ilgai delsi, nuklydusi dukra? Nes Viešpats sukūrė naują dalyką žemėje­moteris apglėbs vyrą. 
\par 23 Taip sako kareivijų Viešpats, Izraelio Dievas. “Kai Aš parvesiu juos iš nelaisvės, Judo šalyje ir jo miestuose jie sakys: ‘Viešpats telaimina tave, teisybės buveine, šventasis kalne!’ 
\par 24 Jude ir jo žemėse drauge gyvens žemdirbiai ir gyvulių augintojai. 
\par 25 Aš atgaivinsiu pailsusią sielą, o suvargusią­pastiprinsiu”. 
\par 26 Aš pabudau iš miego, kuris man buvo saldus. 
\par 27 “Ateis laikas,­sako Viešpats,­ kai Izraelio ir Judo namus užsėsiu žmonių ir gyvulių sėkla. 
\par 28 Kaip Aš budėjau prie jų, kad išraučiau ir sugriaučiau, sunaikinčiau bei bausčiau, taip dabar budėsiu, kad statyčiau ir sodinčiau. 
\par 29 Tada nebesakys: ‘Tėvai valgė rūgščių vynuogių, o vaikams atšipo dantys’. 
\par 30 Kiekvienas mirs už savo paties kaltes. Kas valgys rūgščių vynuogių, to dantys atšips. 
\par 31 Ateina dienos, kai Aš padarysiu su Izraeliu ir Judu naują sandorą,­sako Viešpats,­ 
\par 32 ne tokią sandorą, kokią padariau su jų tėvais, kai juos, paėmęs už rankos, išvedžiau iš Egipto žemės. Jie sulaužė mano sandorą, nors Aš buvau jų valdovas. 
\par 33 Bet tokią sandorą Aš padarysiu su Izraelio namais: toms dienoms praėjus, Aš įdėsiu savo įstatymą į jų vidų ir įrašysiu į jų širdis. Aš būsiu jų Dievas, o jie bus mano tauta. 
\par 34 Tada nė vienas nebemokys savo artimo ir nebesakys savo broliui: ‘Pažink Viešpatį!’ Jie visi, nuo mažiausiojo iki didžiausiojo, mane pažins. Aš atleisiu jų kaltę ir jų nuodėmės nebeprisiminsiu”. 
\par 35 Taip sako Viešpats, kuris davė saulę šviesti dieną ir paskyrė mėnulį bei žvaigždes spindėti naktį, kuris sukelia jūroje šėlstančias bangas­kareivijų Viešpats yra Jo vardas: 
\par 36 “Jeigu šitie įstatai liausis veikę mano akivaizdoje, tai ir Izraelio palikuonys paliaus būti mano tauta. 
\par 37 Jeigu galima išmatuoti dangaus aukštybę ir ištirti žemės gelmes, tai Aš atmesiu Izraelio palikuonis už tai, ką jie padarė,­sako Viešpats.­ 
\par 38 Ateina dienos, kai Viešpaties miestas bus atstatytas nuo Hananelio bokšto iki Kampinių vartų 
\par 39 ir toliau eis tiesiai į Garebo kalną, paskui pasuks į Goją. 
\par 40 Visas lavonų ir aukų pelenų slėnis bei visi laukai iki Kidrono upelio ir Žirgų vartų kampo rytuose bus šventa vieta Viešpačiui. Ji nebebus griaunama ir ardoma per amžius,­sako Viešpats”.



\chapter{32}


\par 1 Dešimtaisiais Judo karaliaus Zedekijo metais Viešpats kalbėjo Jeremijui. Tai buvo aštuonioliktieji Nebukadnecaro metai. 
\par 2 Tuomet Babilono karaliaus kariuomenė buvo apgulusi Jeruzalę, o pranašas Jeremijas buvo uždarytas sargybos kieme prie Judo karaliaus namų. 
\par 3 Judo karalius Zedekijas jį uždarė, sakydamas: “Kodėl tu pranašauji ir sakai: ‘Taip sako Viešpats: ‘Aš atiduosiu šitą miestą į Babilono karaliaus rankas, ir šis jį paims. 
\par 4 Ir Judo karalius Zedekijas neištrūks iš chaldėjų rankų, bet tikrai pateks į Babilono karaliaus rankas ir kalbės su juo veidas į veidą, o jo akys matys ano akis. 
\par 5 Nebukadnecaras nuves Zedekiją į Babiloną, ir ten šis pasiliks, kol jį aplankysiu. Nors kovosite prieš chaldėjus, neturėsite pasisekimo’?” 
\par 6 Jeremijas sakė: “Toks buvo man Viešpaties žodis: 
\par 7 ‘Hanamelis, tavo dėdės Selumo sūnus, ateis pas tave ir sakys: ‘Pirk mano žemę Anatote, nes tu turi teisę pirmas ją įsigyti’ ”. 
\par 8 Hanamelis, mano dėdės sūnus, atėjo pas mane, kaip Viešpats buvo sakęs, į sargybos kiemą ir man tarė: “Pirk mano žemę Anatote, Benjamino krašte, nes tu turi teisę ją pirkti”. Aš supratau, kad tai buvo Viešpaties žodis. 
\par 9 Taip aš nupirkau iš savo dėdės sūnaus Hanamelio žemę Anatote ir jam atsvėriau septyniolika šekelių sidabro. 
\par 10 Aš parašiau raštą, jį užantspaudavau, pakviečiau liudytojų ir pasvėriau sidabrą svarstyklėmis. 
\par 11 Aš paėmiau užantspauduotąjį pirkimo raštą, kaip reikalauja įstatymas ir papročiai, ir atvirąjį pirkimo raštą 
\par 12 ir padaviau Machsėjos sūnaus Nerijo sūnui Baruchui, matant mano pusbroliui Hanameliui ir liudytojams, pasirašiusiems pirkimo raštą, ir matant visiems žmonėms, buvusiems sargybos kieme. 
\par 13 Jų akivaizdoje pasakiau Baruchui: 
\par 14 “Taip sako kareivijų Viešpats, Izraelio Dievas: ‘Imk šituos pirkimo raštus, užantspauduotąjį ir atvirąjį, ir juos įdėk į molinį indą, kad jie ilgai išliktų’. 
\par 15 Nes taip sako kareivijų Viešpats, Izraelio Dievas: ‘Namai, žemė ir vynuogynai vėl bus perkami šiame krašte’ ”. 
\par 16 Padavęs pirkimo raštą Nerijo sūnui Baruchui, aš meldžiau Viešpatį: 
\par 17 “Viešpatie Dieve, Tu sukūrei dangų ir žemę savo didele galia ir ištiesta ranka. Nieko Tau nėra negalimo. 
\par 18 Tu rodai malonę tūkstančiams ir baudi vaikus už jų tėvų nusikaltimus. Tu didis ir galingas Dievas, kurio vardas­kareivijų Viešpats. 
\par 19 Didis patarimu ir galingas darbais. Tavo akys mato visus žmonių kelius, kad atlygintų kiekvienam pagal jo kelius ir darbus. 
\par 20 Tu darei ženklų ir stebuklų Egipto šalyje ir darai iki šios dienos Izraelyje ir tarp žmonių, ir įsigijai vardą, kaip yra šiandien. 
\par 21 Tu išvedei Izraelio tautą iš Egipto šalies su ženklais ir stebuklais, savo galinga ir ištiesta ranka, ir su dideliu siaubu. 
\par 22 Jiems davei šią šalį, plūstančią pienu ir medumi, kurią duoti buvai prisiekęs jų tėvams. 
\par 23 Įėję ir apsigyvenę šalyje, jie neklausė Tavęs, nesilaikė įstatymo ir nedarė, ką jiems buvai įsakęs. Todėl baudei juos, užleisdamas šitą nelaimę. 
\par 24 Štai pylimai jau prie miesto sienų, ir miestas per kardą, badą ir marą yra atiduodamas į chaldėjų rankas, kurie jį puola. Tavo žodis išsipildė, ir Tu pats tai matai. 
\par 25 O Tu, Viešpatie Dieve, man liepei pirkti žemę už pinigus ir pasiimti liudytojus, nors miestas atiduotas į chaldėjų rankas”. 
\par 26 Viešpats tarė Jeremijui: 
\par 27 “Aš Viešpats, kiekvieno kūno Dievas. Ar yra man kas nors per sunku? 
\par 28 Aš atiduosiu šitą miestą į chaldėjų ir Babilono karaliaus Nebukadnecaro rankas. 
\par 29 Chaldėjai, kurie puola šitą miestą, įsiverš į jį, užims, padegs ir sudegins jį su visais namais, ant kurių stogų jie smilkė aukas Baalui ir aukojo geriamąsias aukas svetimiems dievams, mane užrūstindami. 
\par 30 Izraelitai ir Judo gyventojai nuo pat jaunystės darė pikta mano akivaizdoje ir pykdė mane savo rankų darbais,­sako Viešpats.­ 
\par 31 Šitas miestas pykdė ir rūstino mane nuo pat jo įkūrimo dienos, ir Aš nusprendžiau jį sunaikinti 
\par 32 dėl izraelitų piktybių, kurias jie darė, mane supykdydami, kartu su savo karaliais, kunigaikščiais, kunigais, pranašais ir Jeruzalės gyventojais. 
\par 33 Jie atgręžė man nugarą, nors Aš juos mokiau nuo ankstaus ryto; tačiau jie nesiklausė ir nepriėmė mano pamokymo. 
\par 34 Jie pastatė savo bjaurystes namuose, kurie vadinami mano vardu, kad juos suteptų. 
\par 35 Jie įrengė Baalui aukštumą Ben Hinomo slėnyje, kad leistų savo sūnus ir dukteris per ugnį Molechui. To Aš jiems neįsakiau ir tai man net į galvą neatėjo, kad jie darytų tokį nusikaltimą ir įtrauktų Judą į nuodėmę. 
\par 36 Todėl dabar taip sako Viešpats, Izraelio Dievas, apie šį miestą, apie kurį jūs sakote, kad jis bus atiduotas į Babilono karaliaus rankas kardu, badu ir maru: 
\par 37 ‘Aš juos surinksiu iš visų šalių, po kurias užsirūstinęs ir supykęs išsklaidžiau. Juos sugrąžinsiu į šitą vietą ir leisiu jiems saugiai gyventi. 
\par 38 Jie bus mano tauta, o Aš būsiu jų Dievas. 
\par 39 Aš jiems duosiu vieną širdį ir vieną kelią, kad jie bijotų manęs, savo pačių ir savo vaikų labui. 
\par 40 Aš sudarysiu su jais amžiną sandorą, nesiliausiu jiems gera daręs. Aš įdėsiu į jų širdis savo baimę, kad jie nepaliktų manęs. 
\par 41 Aš džiaugsiuosi jais, darydamas jiems gera visa savo širdimi ir siela, ir tikrai juos įtvirtinsiu šitoje žemėje. 
\par 42 Kaip Aš juos baudžiau, taip Aš duosiu jiems visas gėrybes, kurias esu pažadėjęs. 
\par 43 Jie vėl pirks žemę šioje šalyje, apie kurią sakoma: ‘Ji dykuma be žmonių ir gyvulių, ji atiduota į chaldėjų rankas’. 
\par 44 Žmonės pirks žemę už pinigus, rašys pirkimo raštus, juos užantspauduos ir kvies liudytojų Benjamino krašte, Jeruzalės apylinkėse, Judo miestuose, kalnyno, lygumos ir pietų krašto miestuose, nes Aš parvesiu jų belaisvius’ ”.



\chapter{33}


\par 1 Viešpats vėl kalbėjo Jeremijui, kai jis tebebuvo uždarytas sargybos kieme: 
\par 2 “Taip sako Viešpats, kuris sukūrė žemę, sutvėrė ją ir padėjo pamatus, Viešpats yra Jo vardas: 
\par 3 ‘Šaukis manęs, tai išklausysiu tave ir parodysiu tau didelių bei nesuvokiamų dalykų, apie kuriuos nieko nežinai’. 
\par 4 Nes taip sako Viešpats, Izraelio Dievas, apie šį miestą ir Judo karaliaus namus, kurie buvo sugriauti, kad pastatytų pylimus ir įtvirtinimus: 
\par 5 ‘Chaldėjai įsiverš ir pripildys gatves lavonų. Aš savo rūstybėje juos išžudysiu ir nusigręšiu nuo šito miesto dėl jo piktybių. 
\par 6 Bet Aš juos vėl išgelbėsiu ir išgydysiu, atversiu jiems taikos ir tiesos gausybę. 
\par 7 Aš parvesiu Judo ir Izraelio belaisvius ir atstatysiu juos, kaip buvo pradžioje; 
\par 8 nuplausiu jų nuodėmes ir atleisiu nusikaltimus, kuriais jie man nusikalto. 
\par 9 Šis miestas bus man džiaugsmas, pasigyrimas ir garbė visose žemės tautose. Kai jos išgirs apie gerovę ir perteklių, kurį jiems duosiu, jos išsigandusios drebės’. 
\par 10 Taip sako Viešpats: ‘Šioje vietoje, apie kurią jūs sakote, kad ji yra tuščia, be žmonių ir gyvulių, Judo miestuose ir Jeruzalės gatvėse, kurios yra tuščios, be gyventojų, be žmonių ir gyvulių, vėl girdėsis 
\par 11 džiaugsmo ir linksmybės balsai, jaunikio ir jaunosios balsas ir balsai tų, kurie, nešdami gyriaus aukas į Viešpaties namus, sakys: ‘Girkite kareivijų Viešpatį, nes Viešpats yra geras ir Jo gailestingumas amžinas’. Aš atstatysiu kraštą, koks jis buvo,­sako Viešpats’. 
\par 12 Taip sako kareivijų Viešpats: ‘Šioje tuščioje vietoje, kuri yra be žmonių ir gyvulių, ir kituose jos miestuose vėl bus gyvuliams ganyklų, o ganytojams ir jų bandoms poilsio vietų. 
\par 13 Kalnų, lygumų ir pietų krašto miestuose, Benjamino krašte, apie Jeruzalę ir Judo miestuose bandos praeis pro rankas to, kuris jas skaičiuos,­sako Viešpats’. 
\par 14 ‘Ateina dienos,­sako Viešpats,­kai Aš įvykdysiu pažadą, duotą Izraeliui ir Judui. 
\par 15 Tuomet Aš išauginsiu teisumo atžalą iš Dovydo palikuonių. Jis vykdys krašte teisumą ir teisingumą. 
\par 16 Tuo laiku Judas bus išgelbėtas ir Jeruzalė gyvens saugiai. Ji bus vadinama: ‘Viešpats­mūsų teisumas’. 
\par 17 Nes taip sako Viešpats: ‘Dovydas nepritrūks vyro, kuris sėdėtų Izraelio soste, 
\par 18 ir Levio giminės kunigai nepritrūks vyrų, aukojančių deginamąsias aukas, deginančių duonos aukas ir pjaunančių aukas mano akivaizdoje,­sako kareivijų Viešpats’ ”. 
\par 19 Viešpats kalbėjo Jeremijui: 
\par 20 “Jei jūs galite pakeisti mano sandorą su diena ir naktimi, kad naktis ir diena neateitų savo metu, 
\par 21 tai galėtų būti pakeista ir mano sandora su mano tarnu Dovydu, kad vienas iš jo sūnų viešpataus jo soste, ir sandora su kunigais levitų kilmės, mano tarnais. 
\par 22 Kaip dangaus žvaigždės ir jūros smiltys nesuskaitomos, taip Aš padauginsiu Dovydo palikuonis ir levitus, man tarnaujančius”. 
\par 23 Viešpats vėl kalbėjo Jeremijui: 
\par 24 “Ar nepastebėjai, kaip žmonės kalba: ‘Viešpats atmetė abi gimines, kurias buvo išsirinkęs’? Taip jie niekina mano tautą. Ji jų akyse nebėra tauta. 
\par 25 Kaip Aš sukūriau dieną ir naktį, dangui ir žemei daviau įstatus, 
\par 26 taip Aš neatmesiu Jokūbo giminės ir mano tarno Dovydo palikuonių. Abraomo, Izaoko ir Jokūbo palikuonims Aš paskirsiu valdovus iš Dovydo giminės. Nes Aš parvesiu juos iš nelaisvės ir pasigailėsiu jų”.



\chapter{34}


\par 1 Viešpats kalbėjo Jeremijui, kai Nebukadnecaras, Babilono karalius, su visa kariuomene, visomis jo valdžioje esančiomis karalystėmis ir tautomis kariavo prieš Jeruzalę ir Judo miestus: 
\par 2 “Eik ir pranešk Zedekijui, Judo karaliui, kad taip sako Viešpats: ‘Aš atiduosiu šį miestą į Babilono karaliaus rankas, ir šis jį sudegins. 
\par 3 Ir tu neištrūksi iš jo rankų, būsi sugautas ir jam atiduotas. Tu matysi Babilono karalių ir kalbėsi su juo veidas į veidą, ir tave nugabens į Babiloną. 
\par 4 Tačiau klausykis Viešpaties žodžio, Zedekijau, Judo karaliau. Tu nežūsi nuo kardo, 
\par 5 bet mirsi ramybėje. Kaip tavo tėvams, kurie karaliavo pirma tavęs, sukurdavo laidotuvių ugnį, taip tau padarys ir tave apraudos: ‘O mūsų valdove’. Aš taip pasakiau,­sako Viešpats’ ”. 
\par 6 Pranašas Jeremijas kalbėjo šituos žodžius karaliui Zedekijui Jeruzalėje. 
\par 7 Babilono karaliaus kariuomenė kariavo prieš Jeruzalę ir Judo miestus­Lachišą ir Azeką, nes tie sutvirtinti Judo miestai dar buvo nepaimti. 
\par 8 Viešpats kalbėjo Jeremijui, kai karalius Zedekijas padarė sandorą su Jeruzalės gyventojais ir paskelbė, 
\par 9 kad kiekvienas suteiktų laisvę vergams hebrajams, kad nelaikytų vergais žydų. 
\par 10 Kunigaikščiai ir visa tauta, kurie padarė sandorą suteikti laisvę vergams ir vergėms, kad jie nebevergautų, pakluso ir paleido juos. 
\par 11 Bet po to jie persigalvojo ir susigrąžino vergus bei verges, ir privertė juos toliau vergauti. 
\par 12 Todėl Viešpats kalbėjo Jeremijui: 
\par 13 “Aš sudariau sandorą su jūsų tėvais, kai juos išvedžiau iš Egipto vergijos, ir įsakiau: 
\par 14 ‘Baigiantis septyneriems metams, kiekvienas privalo paleisti savo brolį hebrają, kuris jam vergavo. Šešerius metus jis tedirba, o po to suteikite jam laisvę’. Bet jūsų tėvai neklausė manęs ir nepalenkė savo ausies girdėti. 
\par 15 Dabar jūs buvote atsivertę ir padarę tai, kas teisinga mano akyse, paskelbdami laisvę savo broliams. Jūs sudarėte sandorą mano akivaizdoje, namuose, vadinamuose mano vardu. 
\par 16 Bet jūs persigalvojote ir sutepėte mano vardą, susigrąžindami vergus ir verges. Jie buvo laisvi, galėjo gyventi, kaip norėjo, o jūs juos vėl pavergėte. 
\par 17 Kadangi jūs neklausėte manęs ir nesuteikėte laisvės broliui ir artimui, Aš išlaisvinu jus kardui, marui ir badui. Aš išsklaidysiu jus visose žemės karalystėse. 
\par 18 Žmones, nesilaikiusius sandoros, sudarytos mano akivaizdoje, sandoros, kurią sudarėte, perpjaudami veršį ir praeidami tarp jo dalių, 
\par 19 Judo ir Jeruzalės kunigaikščius, valdininkus, kunigus ir visus žmones, kurie praėjo tarp veršio dalių, 
\par 20 Aš atiduosiu į jų priešų, kurie siekia jų gyvybės, rankas. Jų lavonai bus maistu padangių paukščiams ir laukiniams žvėrims. 
\par 21 Taip pat Zedekijas, Judo karalius, ir jo kunigaikščiai pateks į priešų, siekiančių jų gyvybės, rankas, į rankas Babilono karaliaus kariuomenės, kuri šiuo metu yra pasitraukusi. 
\par 22 Aš įsakysiu,­sako Viešpats,­ir sugrąžinsiu ją prieš šitą miestą. Jie kariaus, paims ir sudegins jį. Judo miestus Aš padarysiu dykyne be gyventojų”.



\chapter{35}


\par 1 Jehojakimo, Jozijo sūnaus, Judo karaliaus, dienomis Viešpats kalbėjo Jeremijui: 
\par 2 “Eik į rechabų namus ir kalbėk su jais, nuvesk juos į vieną Viešpaties namų kambarį ir duok jiems gerti vyno”. 
\par 3 Aš atvedžiau Habacinijo sūnaus Jeremijo sūnų Jaazaniją, jo brolius, vaikus ir visą rechabų šeimą, 
\par 4 įvedžiau į Viešpaties namus, į Dievo vyro Igdalijo sūnaus Hanano sūnų kambarį, kuris buvo prie kunigaikščių kambario, virš Šalumo sūnaus Maasėjos patalpos. 
\par 5 Aš pastačiau priešais rechabus pilnus ąsočius vyno bei taures ir pasiūliau gerti vyną. 
\par 6 Jie atsakė: “Mes negersime vyno, nes mūsų tėvas Jehonadabas, Rechabo sūnus, mums įsakė: ‘Niekados negerkite vyno nei jūs, nei jūsų vaikai. 
\par 7 Nesistatykite namų, nesėkite, nesodinkite vynuogynų ir neturėkite jų, gyvenkite palapinėse per visas savo dienas, kad ilgai gyventumėte krašte, kuriame esate ateiviai’. 
\par 8 Mes laikomės Rechabo sūnaus, mūsų tėvo Jehonadabo, įsakymo. Visą gyvenimą nei mes, nei mūsų žmonos, nei sūnūs, nei dukterys negeriame vyno. 
\par 9 Mes nesistatome namų, neturime vynuogynų nei laukų sėjame; 
\par 10 gyvename palapinėse ir laikomės viso, ką mūsų tėvas Jehonadabas įsakė. 
\par 11 Bet kai Babilono karalius Nebukadnecaras puolė šitą kraštą, mes nutarėme pasitraukti nuo chaldėjų ir sirų kariuomenės į Jeruzalę. Taip ir gyvename Jeruzalėje”. 
\par 12 Tada Viešpats tarė Jeremijui: 
\par 13 “Eik ir sakyk Jeruzalės žmonėms ir Judo gyventojams: ‘Ar neklausysite ir nepasimokysite iš mano žodžių? 
\par 14 Jehonadabas, Rechabo sūnus, įsakė savo sūnums negerti vyno. Jie laikosi to iki šios dienos, negeria vyno ir vykdo savo tėvo įsakymą. Aš nuolat kalbėjau jums, bet jūs neklausėte manęs. 
\par 15 Aš siunčiau savo pranašus, sakydamas: ‘Atsisakykite piktų savo kelių ir pasitaisykite savo elgesiu, nesekite svetimų dievų ir netarnaukite jiems, tai jūs ilgai gyvensite krašte, kurį daviau jums ir jūsų tėvams’. Bet jūs nekreipėte dėmesio ir neklausėte manęs. 
\par 16 Rechabo sūnaus Jehonadabo sūnūs laikosi savo tėvo įsakymo, bet mano tauta neklauso manęs. 
\par 17 Todėl Aš užleisiu ant Judo ir Jeruzalės gyventojų paskelbtą nelaimę, nes Aš kalbėjau jiems, bet jie neklausė, Aš šaukiau, bet jie neatsakė,­sako Viešpats’ ”. 
\par 18 Jeremijas sakė rechabams: “Taip sako kareivijų Viešpats: ‘Kadangi jūs Jehonadabo, savo tėvo, įsakymą vykdote, 
\par 19 tai Jehonadabui, Rechabo sūnui, niekada nepristigs vyro, stovinčio mano akivaizdoje,­sako kareivijų Viešpats, Izraelio Dievas’ ”.



\chapter{36}


\par 1 Jehojakimo, Jozijo sūnaus, Judo karaliaus, ketvirtaisiais metais, Viešpats tarė Jeremijui: 
\par 2 “Imk ritinį ir užrašyk visus mano tau kalbėtus žodžius apie Izraelį, Judą ir visas tautas, apie kurias tau kalbėjau nuo Jozijo dienų iki šiandien. 
\par 3 Gal Judo namai, girdėdami, kaip ketinu juos bausti, atsisakys pikto kelio, ir Aš jiems jų nusikaltimus atleisiu”. 
\par 4 Jeremijas pasišaukė Baruchą, Nerijos sūnų, kuris iš Jeremijo lūpų į ritinį užrašė visus Viešpaties kalbėtus žodžius. 
\par 5 Jeremijas tarė Baruchui: “Aš uždarytas ir negaliu eiti į Viešpaties namus. 
\par 6 Todėl tu eik į Viešpaties namus pasninko dieną ir paskaityk Viešpaties žodžius, kuriuos užrašei iš mano lūpų į ritinį, visų žmonių akivaizdoje. Skaityk, tegul girdi visi Judo miestų gyventojai, atėję į Jeruzalę. 
\par 7 Gal jie nusižeminę maldaus Viešpatį ir atsisakys savo piktų kelių, nes didelė Viešpaties rūstybė ir bausmė paskelbta šitam miestui”. 
\par 8 Baruchas, Nerijos sūnus, padarė, kaip pranašas Jeremijas jam įsakė, ir perskaitė Viešpaties žodžius Viešpaties namuose. 
\par 9 Penktaisiais Judo karaliaus Jehojakimo, Jozijo sūnaus, metais, devintą mėnesį, Jeruzalėje buvo pasninkas visai tautai ir visiems žmonėms, kurie buvo atėję iš Judo miestų į Jeruzalę. 
\par 10 Baruchas perskaitė iš ritinio Jeremijo žodžius girdint visai tautai Viešpaties namuose, raštininko Gemarijo, Šafano sūnaus, kambaryje, kuris buvo viršutiniame kieme, prie Viešpaties namų Naujųjų vartų įėjimo. 
\par 11 Šafano sūnaus Gemarijo sūnus Mikajas, išklausęs visus Viešpaties žodžius, skaitytus iš ritinio, 
\par 12 nuėjo į karaliaus namus, į raštininko kambarį. Ten buvo susirinkę kunigaikščiai: raštininkas Elišama, Šemajo sūnus Delajas, Achboro sūnus Elnatanas, Šafano sūnus Gemarijas, Hananijo sūnus Zedekijas ir kiti valdovai. 
\par 13 Mikajas pranešė jiems girdėtus žodžius, kuriuos Baruchas skaitė tautai iš ritinio. 
\par 14 Kunigaikščiai pasiuntė Jehudį, sūnų Netanijo, sūnaus Šelemijo, sūnaus Kušio, pas Baruchą ir jam įsakė: “Pasiimk ritinį, iš kurio skaitei tautai ir ateik”. Baruchas, Nerijos sūnus, pasiėmė ritinį ir atėjo pas juos. 
\par 15 Jie tarė jam: “Sėskis ir skaityk mums”. Baruchas skaitė, jiems girdint. 
\par 16 Išgirdę visus žodžius, jie išsigandę žiūrėjo vienas į kitą ir kalbėjo: “Mes privalome visus šiuos žodžius pranešti karaliui”. 
\par 17 Jie paklausė Baruchą: “Sakyk, kaip tu surašei visus šituos žodžius iš jo lūpų?” 
\par 18 Baruchas atsakė: “Jis skelbė šituos žodžius man savo lūpomis, ir aš užrašiau į ritinį rašalu”. 
\par 19 Kunigaikščiai liepė jam: “Tu ir Jeremijas eikite pasislėpti, kad niekas nežinotų, kur jūs esate”. 
\par 20 Jie nuėjo pas karalių į rūmų kiemą ir visa pranešė karaliui. Ritinį jie paliko raštininko Elišamos kambaryje. 
\par 21 Karalius pasiuntė Jehudį atnešti ritinį. Jis atnešė jį iš raštininko Elišamos kambario ir skaitė jį karaliaus ir visų kunigaikščių, stovinčių šalia karaliaus, akivaizdoje. 
\par 22 Tai buvo devintas mėnuo, ir karalius sėdėjo žiemos rūmuose, o priešais jį židinyje degė ugnis. 
\par 23 Jehudžiui perskaičius tris ar keturis skyrius, karalius nupjaudavo juos raštininko peiliu, įmesdavo į ugnį židinyje ir sudegindavo, kol visas ritinys buvo sunaikintas židinio ugnyje. 
\par 24 Nei karalius, nei jo tarnai, girdėdami tuos žodžius, neišsigando ir nepersiplėšė savo rūbų. 
\par 25 Nors Elnatanas, Delajas ir Gemarijas maldavo karalių nedeginti ritinio, bet jis nekreipė dėmesio į juos. 
\par 26 Be to, karalius įsakė Jerachmeeliui, karaliaus sūnui, Serajui, Azrielio sūnui, ir Šelemijui, Abdeelio sūnui, suimti raštininką Baruchą ir pranašą Jeremiją, bet Viešpats paslėpė juos. 
\par 27 Kai karalius sudegino ritinį su žodžiais, kuriuos Baruchas iš Jeremijo lūpų buvo užrašęs, Viešpats tarė Jeremijui: 
\par 28 “Imk kitą ritinį ir užrašyk į jį visus žodžius, buvusius ritinyje, kurį Jehojakimas, Judo karalius, sudegino. 
\par 29 O Jehojakimui, Judo karaliui, sakyk: ‘Tu sudeginai ritinį, sakydamas: ‘Kodėl tu užrašei, kad Babilono karalius ateis ir sunaikins šitame krašte žmones ir gyvulius?’ 
\par 30 Todėl Viešpats apie Jehojakimą, Judo karalių, sako: ‘Jo palikuonys nesėdės Dovydo soste, o jo lavonas bus išmestas ir gulės dieną karštyje ir naktį šaltyje. 
\par 31 Aš nubausiu jį, jo palikuonis bei tarnus ir užleisiu visas nelaimes ant Jeruzalės gyventojų ir Judo žmonių, kurias paskelbiau prieš juos, bet jie neklausė’ ”. 
\par 32 Jeremijas ėmė kitą ritinį, padavė jį raštininkui Baruchui, Nerijos sūnui, o tas iš Jeremijo lūpų užrašė į ritinį visus žodžius, kuriuos Jehojakimas, Judo karalius, sudegino, ir pridėjo dar daugiau panašių žodžių.



\chapter{37}


\par 1 Vietoje Konijo, Jehojakimo sūnaus, Babilono karalius Nebukadnecaras paskyrė Judo karaliumi Jozijo sūnų Zedekiją. 
\par 2 Bet nei jis, nei jo tarnai, nei krašto gyventojai neklausė Viešpaties žodžių, kalbėtų per pranašą Jeremiją. 
\par 3 Karalius Zedekijas pasiuntė Jehuchalą, Šelemijo sūnų, ir kunigą Sofoniją, Maasėjos sūnų, pas pranašą Jeremiją ir prašė jo: “Melsk už mus Viešpatį, mūsų Dievą”. 
\par 4 Jeremijas tada dar nebuvo įmestas į kalėjimą ir vaikščiojo savo tautoje. 
\par 5 Chaldėjai, kurie buvo apgulę Jeruzalę, išgirdę, kad ateina Egipto faraono kariuomenė, pasitraukė nuo Jeruzalės. 
\par 6 Tada Viešpats kalbėjo pranašui Jeremijui: 
\par 7 “Taip sako Viešpats, Izraelio Dievas: ‘Sakyk Judo karaliaus pasiuntiniams: ‘Faraono kariuomenė, kuri ateina jums padėti, grįš į savo kraštą Egiptą, 
\par 8 ir chaldėjai vėl ateis, kariaus prieš šitą miestą, paims jį ir sudegins. 
\par 9 Neapgaudinėkite patys savęs, manydami, kad chaldėjai pasitrauks ir nebesugrįš. Jie nepasitrauks. 
\par 10 Jei jūs chaldėjų kariuomenę, prieš jus kariaujančią, taip sumuštumėte, jog liktų tik sužeistieji, tai jie, pakilę iš savo palapinių, sudegintų šitą miestą!’ ” 
\par 11 Kai chaldėjų kariuomenė, išsigandusi Egipto kariuomenės, pasitraukė nuo Jeruzalės, 
\par 12 Jeremijas norėjo eiti iš Jeruzalės į Benjamino kraštą savo nuosavybės reikalus sutvarkyti su savo giminaičiais. 
\par 13 Jam atėjus prie Benjamino vartų, sargybinis Irija, sūnus Šelemijos, sūnaus Hananijos, sulaikė pranašą Jeremiją ir tarė: “Tu bėgi pas chaldėjus!” 
\par 14 Jeremijas atsakė: “Tai netiesa, aš nebėgu pas chaldėjus”. Irija nepatikėjo. Jis sulaikė Jeremiją ir nuvedė pas kunigaikščius. 
\par 15 Kunigaikščiai supyko ir sumušė Jeremiją bei įmetė į kalėjimą raštininko Jehonatano namuose, nes jie buvo padaryti kalėjimu. 
\par 16 Jeremijas pateko į požemį ir išbuvo ten daug dienų. 
\par 17 Vieną dieną karalius Zedekijas pasiuntė atvesti Jeremiją. Karalius slaptai savo rūmuose klausė jį: “Ar yra žodis iš Viešpaties?” Jeremijas atsakė: “Taip, tu būsi atiduotas į Babilono karaliaus rankas”. 
\par 18 Be to, Jeremijas sakė karaliui Sedekijui: “Kuo aš nusikaltau tau, tavo tarnams ir tautai, kad mane laikote kalėjime? 
\par 19 Kur dabar yra jūsų pranašai, pranašavę jums, kad Babilono karalius neateis prieš jus ir prieš šitą kraštą? 
\par 20 Mano valdove karaliau, išklausyk mano prašymą, nebesiųsk manęs atgal į raštininko Jehonatano namus, kad ten nemirčiau!” 
\par 21 Karalius Zedekijas įsakė, kad Jeremijas būtų nuvestas į sargybos kiemą ir kad jam duotų gabalą duonos iš Kepėjų gatvės, kol visa duona mieste pasibaigs. Taip Jeremijas liko sargybos kieme.



\chapter{38}


\par 1 Šefatija, Matano sūnus, Gedolija, Pašhūro sūnus, Jehuchalas, Šelemijo sūnus, ir Pašhūras, Malkijos sūnus, girdėjo Jeremijo žodžius, kuriuos jis kalbėjo tautai: 
\par 2 “Taip sakoViešpats: ‘Kas liks šitame mieste, mirs nuo kardo, bado ir maro, bet kas išeis pas chaldėjus­išliks. Jis gaus savo gyvybę kaip grobį. 
\par 3 Šitas miestas tikrai pateks į Babilono karaliaus kariuomenės rankas,­sako Viešpats’ ”. 
\par 4 Kunigaikščiai kreipėsi į karalių Zedekiją: “Šitą vyrą reikia nužudyti, nes jis silpnina mieste likusius karius ir savo kalbomis atima drąsą visai tautai. Jis neieško tautos gerovės, bet neša nelaimę jai”. 
\par 5 Karalius Zedekijas atsakė: “Jis yra jūsų rankose, aš negaliu jums prieštarauti”. 
\par 6 Jie suėmė Jeremiją ir įmetė į karaliaus sūnaus Malkijos šulinį prie sargybos kiemo. Jie nuleido jį virvėmis. Šulinyje nebuvo vandens, tik dumblas. Jeremijas įklimpo dumble. 
\par 7 Etiopas Ebed Melechas, karaliaus namų eunuchas, išgirdęs, kad Jeremiją įmetė į šulinį, 
\par 8 išėjo iš rūmų, nuėjo pas karalių, kuris tuo metu buvo prie Benjamino vartų, ir tarė: 
\par 9 “Mano valdove karaliau, tie vyrai piktai pasielgė su pranašu Jeremiju. Jie įmetė jį į šulinį ir jis ten mirs badu, nes mieste nebėra duonos”. 
\par 10 Karalius įsakė etiopui Ebed Melechui: “Imk iš čia trisdešimt vyrų ir ištrauk pranašą Jeremiją iš šulinio, kol nenumirė”. 
\par 11 Ebed Melechas su vyrais nuėjo į karaliaus namų drabužių sandėlį, paėmė iš ten skarmalų ir suplyšusių drabužių ir nuleido juos virvėmis į šulinį Jeremijui. 
\par 12 Etiopas Ebed Melechas pasakė Jeremijui: “Skudurus ir drabužius pasikišk į pažastis po virvėmis”. Jeremijas taip ir padarė. 
\par 13 Jie ištraukė Jeremiją virvėmis iš šulinio. Jis liko sargybos kieme. 
\par 14 Karalius Zedekijas pasiuntė tarnus, kurie pranašą Jeremiją atvedė pas jį. Šis jį pasitiko Viešpaties namų trečiame įėjime. Karalius sakė Jeremijui: “Aš noriu tave kai ko paklausti; nieko neslėpk nuo manęs”. 
\par 15 Jeremijas atsakė Zedekijui: “Jei aš tau pasakysiu, ar nežudysi manęs? Jei aš tau patarsiu, ar paklausysi manęs?” 
\par 16 Karalius Zedekijas slaptai prisiekė Jeremijui: “Kaip gyvas Viešpats, kuris davė mums gyvybę, aš nežudysiu tavęs ir neatiduosiu į rankas vyrų, siekančių tavo gyvybės”. 
\par 17 Jeremijas atsakė Zedekijui: “Jei tu pasiduosi Babilono karaliaus kunigaikščiams, išliksi gyvas ir šis miestas nebus sudegintas. Liksi gyvas tu ir tavo namai. 
\par 18 O jei tu nepasiduosi Babilono karaliaus kunigaikščiams, miestas atiteks chaldėjams; šie sudegins jį ir tu neištrūksi iš jų rankų”. 
\par 19 Karalius Zedekijas atsakė Jeremijui: “Aš bijau pas chaldėjus perbėgusių žydų, kad nebūčiau atiduotas jų valiai ir jie nepasityčiotų iš manęs”. 
\par 20 Bet Jeremijas sakė: “Tavęs neišduos jiems. Prašau, paklusk Viešpaties balsui, tai tau bus gerai ir liksi gyvas. 
\par 21 Jei nepasiduosi, štai Viešpaties žodis: 
\par 22 ‘Visos Judo karaliaus namų moterys bus išvestos pas Babilono karaliaus kunigaikščius ir jos sakys: ‘Tavo draugai suvedžiojo ir įklampino tave, kai tu į dumblą patekai, jie paliko tave’. 
\par 23 Tavo žmonas ir vaikus išves pas chaldėjus, o tu pats neištrūksi iš jų rankų. Tave suims Babilono karalius ir šitą miestą sudegins’ ”. 
\par 24 Zedekijas atsakė Jeremijui: “Šį pokalbį laikyk paslaptyje ir tada tu nemirsi! 
\par 25 Jei kunigaikščiai išgirs, kad aš kalbėjau su tavimi, ateis ir klaus: ‘Pasakyk, ką kalbėjai su karaliumi ir ką karalius tau sakė; nieko nenuslėpk nuo mūsų, ir mes tavęs nežudysime’, 
\par 26 tada jiems atsakyk: ‘Aš maldavau karalių, kad jis manęs nesiųstų į Jehonatano namus ir aš ten nemirčiau’ ”. 
\par 27 Tada kunigaikščiai atėjo pas Jeremiją ir klausė. Jis atsakė jiems, kaip karalius liepė. Jie paliko jį ramybėje, nes niekas nenugirdo jo pokalbio su karaliumi. 
\par 28 Jeremijas liko sargybos kieme iki tos dienos, kai Jeruzalė buvo paimta.



\chapter{39}


\par 1 Judo karaliaus Sedekijo devintųjų metų dešimtą mėnesį Babilono karalius Nebukadnecaras ir visa jo kariuomenė atėjo ir apgulė Jeruzalę. 
\par 2 Vienuoliktaisiais Zedekijo metais, ketvirto mėnesio devintą dieną miesto siena buvo pralaužta. 
\par 3 Visi Babilono karaliaus kunigaikščiai suėjo ir susėdo prie viduriniųjų vartų: Nergal Sareceras, Samgar Nebas, Sarsechimas, Rabsaris, Nergal Sareceras, Rabmagas ir visi kiti Babilono karaliaus kunigaikščiai. 
\par 4 Judo karalius Zedekijas ir visi kariai, pamatę juos, bėgo pro karaliaus sodą tarp dviejų sienų, palikdami miestą nakties metu. Jie traukėsi lygumos keliu. 
\par 5 Chaldėjų kariuomenė sekė juos ir pasivijo Zedekiją Jericho lygumose. Jie suėmė jį ir atgabeno pas Babilono karalių Nebukadnecarą į Riblą Hamato krašte. Karalius čia jį teisė. 
\par 6 Jis nužudė Zedekijo sūnus tėvo akivaizdoje, taip pat visus Judo kilminguosius. 
\par 7 Be to, išlupo Zedekijui akis, surakino grandinėmis ir išgabeno jį į Babiloną. 
\par 8 Chaldėjai sudegino karaliaus namus, žmonių namus ir nugriovė Jeruzalės sienas. 
\par 9 Nebuzaradanas, sargybos viršininkas, ištrėmė į Babiloną mieste likusius gyventojus, tuos, kurie perbėgo pas jį, ir amatininkus. 
\par 10 Bet Nebuzaradanas, sargybos viršininkas, paliko krašte kai kuriuos beturčius ir davė jiems vynuogynų ir žemės. 
\par 11 Nebukadnecaras, Babilono karalius, davė įsakymą sargybos viršininkui Nebuzaradanui apie Jeremiją: 
\par 12 “Paimk jį, prižiūrėk ir nedaryk jam nieko blogo, bet padaryk jam, kaip jis pats panorės”. 
\par 13 Nebuzaradanas, sargybos viršininkas, Nebušazbanas, Rabsaris, Nergal Sareceras, Rabmagas ir visi kiti Babilono karaliaus kunigaikščiai 
\par 14 pasiuntė ir atgabeno Jeremiją iš sargybos kiemo. Jie patikėjo jį Gedolijui, sūnui Ahikamo, sūnaus Šafano, kad jį globotų. Taip Jeremijas liko gyventi savo tautoje. 
\par 15 Jeremijui dar esant sargybos kieme, Viešpats tarė: 
\par 16 “Eik ir kalbėk etiopui Ebed Melechui: ‘Aš išpildysiu savo žodžius šito miesto nelaimei, o ne gerovei, ir tu pats savo akimis matysi juos išsipildant. 
\par 17 Bet tą dieną Aš tave išgelbėsiu ir tu nebūsi atiduotas į rankas tų žmonių, kurių bijai. 
\par 18 Tu nekrisi nuo kardo ir tavo gyvybė atiteks tau kaip grobis, nes pasitikėjai manimi,­sako Viešpats’ ”.



\chapter{40}


\par 1 Viešpats kalbėjo Jeremijui, kai Nebuzaradanas, sargybos viršininkas, paleido jį Ramoje. Tas jį išlaisvino iš grandinių, nes Jeremijas buvo vedamas drauge su kitais iš Jeruzalės į Babiloną. 
\par 2 Sargybos viršininkas įsakė pašaukti Jeremiją ir tarė jam: “Viešpats, tavo Dievas, paskelbė apie ateinančią nelaimę šitai vietai. 
\par 3 Dabar Viešpats padarė taip, kaip buvo kalbėjęs, nes jūs nusikaltote Viešpačiui, neklausydami Jo. 
\par 4 Taigi šiandien aš nuimu grandines nuo tavo rankų. Jei tu nori eiti drauge su manimi į Babiloną, eik, ir aš tavimi pasirūpinsiu. O jei tu nenori eiti į Babiloną, neik! Visas kraštas yra tau prieš akis. Eik ten, kur tau patinka. 
\par 5 Eik pas Gedoliją, Šafano sūnaus Ahikamo sūnų, kurį Babilono karalius paskyrė Judo valdytoju, gyvenk ten arba pasirink kitą vietą”. Sargybos viršininkas davė jam maisto bei dovanų ir paleido jį. 
\par 6 Jeremijas nuėjo pas Goedoliją, Ahikamo sūnų, į Micpą ir gyveno tarp žmonių, kurie buvo likę krašte. 
\par 7 Kai išlikę Judo karo vadai ir jų žmonės išgirdo, jog Babilono karalius paskyrė krašto valdytoju Gedoliją, Ahikamo sūnų, ir pavedė jam likusius krašte beturčius, vyrus, moteris ir vaikus, neištremtus į Babiloną, 
\par 8 atėjo pas Gedoliją į Micpą Izmaelis, Netanijo sūnus, Johananas ir Jehonatanas, Kareacho sūnūs, Seraja, Tanhumeto sūnus, sūnūs Efajo iš Netofos ir Jezanijas, maako sūnus, su savo žmonėmis. 
\par 9 Gedolijas, sūnus Ahikamo, sūnaus Šafano, su priesaika sakė jiems: “Nebijokite chaldėjų, likite krašte, tarnaukite Babilono karaliui ir jums bus gerai. 
\par 10 Aš lieku Micpoje ir tarnausiu chaldėjams, kurie čia atvyks. O jūs rinkite vynuogynų, sodų ir alyvmedžių vaisius, kaupkite jų atsargas ir gyvenkite tuose miestuose, kuriuos pasirinkote”. 
\par 11 Visi žydai, kurie buvo Moabe, Edome ir kituose kraštuose, išgirdę, kad Babilono karalius paliko krašte likutį ir Gedoliją, Šafano sūnaus Ahikamo sūnų, paskyrė krašto valdytoju, 
\par 12 sugrįžo iš visų vietovių, kuriose jie buvo išblaškyti, į Judo kraštą pas Gedoliją į Micpą. Jie surinko labai daug vynuogių ir sodų vaisių. 
\par 13 Johananas, Kareacho sūnus, ir kiti karo vadai, kurie buvo krašte, atėjo pas Gedoliją į Micpą. 
\par 14 Jie jam sakė: “Ar žinai, kad Baalis, amonitų karalius, atsiuntė Izmaelį, Netanijo sūnų, tavęs nužudyti?” Bet Gedolijas, Ahikamo sūnus, jais netikėjo. 
\par 15 Johananas, Kareacho sūnus, slaptai pasisiūlė Gedolijui Micpoje: “Leisk man eiti ir nužudyti Izmaelį, Netanijo sūnų. Niekas apie tai nesužinos. Kodėl jis turėtų nužudyti tave? Tuomet Judo žmonės, kurie yra čia susirinkę, būtų išblaškyti ir Judo likutis pražūtų”. 
\par 16 Gedolijas, Ahikamo sūnus, atsakė Johananui, Kareacho sūnui: “Nedaryk taip, nes netiesą sakai apie Izmaelį”.



\chapter{41}


\par 1 Izmaelis, sūnus Netanijo, sūnaus Elišamos, iš karališkos giminės, karaliaus kunigaikštis, septintą mėnesį atėjo su dešimčia vyrų pas Gedoliją, Ahikamo sūnų, į Micpą. Jiems kartu Micpoje valgant, 
\par 2 pakilo Izmaelis, Netanijo sūnus, ir dešimt su juo buvusių vyrų ir nužudė kardu Gedoliją, Šafano sūnaus Ahikamo sūnų, kurį Babilono karalius buvo paskyręs krašto valdytoju. 
\par 3 Izmaelis nužudė taip pat visus su Gedoliju buvusius žydus ir chaldėjų karius. 
\par 4 Kitą dieną, Gedolijui žuvus, dar niekam nežinant apie tai, 
\par 5 atėjo aštuoniasdešimt vyrų iš Sichemo, Šilojo ir Samarijos. Jų barzdos buvo nuskustos, drabužiai perplėšti ir oda suraižyta. Jie atsinešė aukų bei smilkalų aukoti Viešpaties namuose. 
\par 6 Izmaelis, Netanijo sūnus, išėjo iš Micpos jų pasitikti. Eidamas jis verkė ir, juos sutikęs, sakė: “Ateikite pas Gedoliją, Ahikamo sūnų”. 
\par 7 Jiems įėjus į miestą, Izmaelis, Netanijo sūnus, ir jo vyrai nužudė juos ir sumetė į duobę. 
\par 8 Dešimt atėjusių vyrų sakė Izmaeliui: “Nežudyk mūsų, mes turime laukuose paslėpę kviečių, miežių, aliejaus ir medaus atsargų”. Izmaelis nenužudė jų kartu su kitais. 
\par 9 Duobė, į kurią Izmaelis sumetė nužudytųjų lavonus, buvo karaliaus Asos padaryta, kai jis išsigando Izraelio karaliaus Bašos. Izmaelis, Netanijo sūnus, pripildė ją užmuštųjų. 
\par 10 Jis tautos likutį, kuris buvo Micpoje, padarė belaisvius. Karaliaus dukteris ir visus Micpos žmones, kuriuos Naebuzaradanas, sargybos viršininkas, buvo pavedęs Gedolijui, Ahikamo sūnui, Izmaelis, Netanijo sūnus, suėmė ir vedėsi į amonitų kraštą. 
\par 11 Kai Johananas, Kareacho sūnus, ir visi karo vadai išgirdo apie tai, ką Izmaelis, Netanijo sūnus, padarė, 
\par 12 surinko visus savo vyrus ir išėjo kovoti prieš Izmaelį, Netanijo sūnų. Jie užtiko jį prie didelių vandenų Gibeone. 
\par 13 Kai su Izmaeliu esantys žmonės pamatė Johananą, Kareacho sūnų, ir visus karo vadus su juo, nudžiugo. 
\par 14 Visi žmonės, kuriuos Izmaelis išvedė kaip belaisvius iš Micpos, atsisuko ir nuėjo pas Johananą, Kareacho sūnų. 
\par 15 Bet Izmaelis, Netanijo sūnus, paspruko su aštuoniais vyrais ir pabėgo pas amonitus. 
\par 16 Tada Johananas, Kareacho sūnus, ir visi karo vadai, esantys su juo, ėmė tautos likutį, kurį Izmaelis, Netanijo sūnus, nužudęs Gedoliją, Ahikamo sūnų, išvedė iš Micpos, ir sugrąžino iš Gibeono vyrus, moteris, vaikus ir eunuchus. 
\par 17 Keliaudami jie apsistojo Kimhame, prie Betliejaus. Jie norėjo pabėgti į Egiptą, 
\par 18 nes bijojo chaldėjų, kadangi Izmaelis, Netanijo sūnus, nužudė Gedoliją, Ahikamo sūnų, kurį Babilono karalius buvo paskyręs krašto valdytoju.



\chapter{42}


\par 1 Visi karo vadai, Johananas, Kareacho sūnus, Jezanijas, Hošajos sūnus, ir visi žmonės, nuo didžiausio iki mažiausio, atėjo ir sakė pranašui Jeremijui: 
\par 2 “Išklausyk mūsų maldavimą! Melsk už mus Viešpatį, savo Dievą, dėl šio likučio, nes iš daugelio likome mažas būrelis, kaip pats matai savo akimis, 
\par 3 kad Viešpats, tavo Dievas, nurodytų, ką turime daryti”. 
\par 4 Pranašas Jeremijas jiems atsakė: “Aš išklausiau jus ir melsiu, kaip prašėte, Viešpatį, jūsų Dievą. Ką Viešpats, jūsų Dievas, atsakys, pranešiu jums”. 
\par 5 Jie vėl sakė Jeremijui: “Viešpats tebūna liudytojas, jei nevykdysime to žodžio, su kuriuo Viešpats, tavo Dievas, tave atsiųs pas mus. 
\par 6 Ar tai bus gera, ar bloga, paklusime Viešpaties, mūsų Dievo, pas kurį tave siunčiame, balsui, kad mums gerai sektųsi”. 
\par 7 Po dešimties dienų Viešpats kalbėjo Jeremijui. 
\par 8 Jis pasišaukė Johananą, Kareacho sūnų, visus karo vadus, esančius su juo, ir visus žmones, nuo mažiausio iki didžiausio, 
\par 9 ir pranešė jiems: “Taip sako Viešpats, Izraelio Dievas, pas kurį siuntėte mane maldauti už jus: 
\par 10 ‘Jei jūs liksite šiame krašte, Aš jus statysiu ir negriausiu, sodinsiu ir neišrausiu, nes Aš gailiuosi dėl to pikto, kurį jums padariau. 
\par 11 Nebijokite Babilono karaliaus, nes Aš būsiu su jumis ir išgelbėsiu jus iš jo rankų. 
\par 12 Aš būsiu jums gailestingas, kad jis jūsų pasigailėtų ir paliktų jus gyventi jūsų krašte’. 
\par 13 Bet jei jūs sakysite: ‘Mes neliksime šiame krašte’, ir neklausysite Viešpaties, savo Dievo, balso, 
\par 14 sakydami: ‘Ne! Mes eisime į Egipto kraštą, kur nematysime karo, negirdėsime trimito balso ir nebadausime. Ten mes apsigyvensime!’, 
\par 15 tada klausyk, Judo likuti, Viešpaties žodžio: ‘Jei jūs eisite į Egiptą gyventi, 
\par 16 tai kardas, kurio bijote, pavys jus ir badas, kuris jus baugina, seks paskui jus į Egiptą. Ten jūs ir mirsite. 
\par 17 Žmonės, kurie pasiryžę eiti į Egiptą gyventi, žus nuo kardo ir mirs nuo bado ir maro, nė vienas neištrūks nuo pikto, kurį ant jų užleisiu. 
\par 18 Kaip Aš savo rūstybę išliejau ant Jeruzalės gyventojų, taip išliesiu savo rūstybę ant einančių į Egiptą. Jūs ten būsite keiksmu, pasibaisėjimu, pasityčiojimu bei pajuoka ir savo krašto daugiau nematysite’. 
\par 19 Tai yra Viešpaties žodis tau, Judo likuti. Neikite į Egiptą! Žinokite, kad šiandien aš jus įspėjau! 
\par 20 Jūs veidmainiavote, kai siuntėte mane pas Viešpatį, jūsų Dievą, sakydami: ‘Melsk už mus Viešpatį, mūsų Dievą! Ką Viešpats, mūsų Dievas, pasakys, paskelbk mums, ir mes darysime!’ 
\par 21 Aš šiandien jums paskelbiau, bet jūs neklausote Viešpaties, savo Dievo. 
\par 22 Dabar tikrai žinokite, kad žūsite nuo kardo ir mirsite nuo bado bei maro vietoje, į kurią norite eiti ir gyventi”.



\chapter{43}


\par 1 Kai Jeremijas baigė kalbėti žmonėms Viešpaties, jų Dievo, žodžius, kuriuos Viešpats, jų Dievas, jiems siuntė, 
\par 2 Azarija, Hošajos sūnus, Johananas, Kareacho sūnus, ir visi išdidūs žmonės atsakė Jeremijui: “Netiesą kalbi! Viešpats, mūsų Dievas, nesiuntė tavęs mums pasakyti: ‘Neikite į Egiptą gyventi’. 
\par 3 Baruchas, Nerijos sūnus, kursto tave prieš mus, kad mes patektume į chaldėjų rankas ir jie mus nužudytų arba ištremtų į Babiloną!” 
\par 4 Johananas, Kareacho sūnus, visi karo vadai ir visa tauta nepaklausė Viešpaties įsakymo likti Judo krašte. 
\par 5 Johananas, Kareacho sūnus, ir visi karo vadai paėmė žmones, sugrįžusius iš visų aplinkinių tautų, kur jie buvo išsklaidyti: 
\par 6 vyrus, moteris, vaikus ir karaliaus dukteris­visus, kuriuos Nebuzaradanas, sargybos viršininkas, buvo palikęs pas Gedoliją, Šafano sūnaus Ahikamo sūnų, taip pat pranašą Jeremiją ir Baruchą, Nerijos sūnų. 
\par 7 Jie nepakluso Viešpaties balsui ir atėjo į Tachpanhesą. 
\par 8 Viešpats kalbėjo Jeremijui Tachpanhese: 
\par 9 “Imk didelių akmenų ir įkask juos žemėje prie faraono namų vartų Tachpanhese, izraelitams matant, 
\par 10 ir sakyk: ‘Taip sako Viešpats: ‘Aš pasiųsiu Nebukadnecarą, Babilono karalių, mano tarną, ir pastatysiu jo sostą ant šitų paslėptų akmenų. Ir jis išties savo palapinę virš šitų akmenų. 
\par 11 Jis ateis ir užims Egipto žemę. Tada kas skirtas mirčiai­mirs, kas skirtas nelaisvei­eis į nelaisvę, kas skirtas kardui­atiteks kardui. 
\par 12 Jis padegs Egipto dievų namus, o juos išgabens į nelaisvę. Jis apsirengs Egipto kraštu, kaip piemuo apsirengia apsiaustu, ir ramiai išeis iš ten. 
\par 13 Jis sudaužys atvaizdus Egipto Bet Šemeše ir sudegins Egipto dievų namus’ ”.



\chapter{44}


\par 1 Viešpats kalbėjo per Jeremiją visiems izraelitams, gyvenantiems Egipto žemėje: Migdole, Tachpanhese, Nofe ir Patroso krašte: 
\par 2 “Jūs patys matote, kaip nubaudžiau Jeruzalę ir Judo miestus. Šiandien jie yra apleisti griuvėsiai. 
\par 3 Tai įvyko dėl jų nusikaltimų, kuriais jie sukėlė mano rūstybę, smilkydami svetimiems dievams, kurių nepažino nei jie, nei jūs, nei jūsų tėvai. 
\par 4 Aš nuolat siunčiau savo tarnus, pranašus, ir sakiau: ‘Nedarykite to pasibjaurėtino dalyko, kurio Aš nekenčiu’. 
\par 5 Bet jie nekreipė dėmesio, neklausė ir nesiliovė aukoti svetimiems dievams. 
\par 6 Dėl to mano rūstybė ir įtūžis išsiliejo ir užsiliepsnojo Judo miestuose ir Jeruzalės gatvėse­jie pavirto griuvėsiais, kaip yra šiandien”. 
\par 7 Todėl dabar taip sako Viešpats: “Kodėl patys sau darote bloga, sunaikindami vyrus, moteris, kūdikius ir vaikus, kol nebeliks Jude nė likučio? 
\par 8 Kodėl rūstinate mane savo darbais, smilkydami svetimiems dievams Egipto krašte, į kurį atvykote gyventi? Patys sau rengiate sunaikinimą, kol tapsite visoms tautoms prakeikimu ir pajuoka. 
\par 9 Ar užmiršote visas savo tėvų, Judo karalių, jų žmonų, savo pačių ir savo žmonų nedorybes, kurias darėte Judo krašte ir Jeruzalės gatvėse? 
\par 10 Jūs iki šios dienos nenusižeminote, nebijojote manęs ir nesilaikėte mano įstatymo bei įsakymų, kuriuos daviau jums ir jūsų tėvams”. 
\par 11 Todėl taip sako kareivijų Viešpats, Izraelio Dievas: “Aš nusisuksiu nuo jūsų ir visai sunaikinsiu Judą. 
\par 12 Aš sunaikinsiu jūsų likutį, sumaniusį vykti į Egiptą gyventi. Jūs žūsite nuo kardo ir bado Egipto žemėje, būsite sunaikinti nuo mažiausio iki didžiausio, tapsite pasibaisėjimu, keiksmu ir pajuoka. 
\par 13 Aš bausiu gyvenančius Egipte, kaip baudžiau Jeruzalę­kardu, badu ir maru. 
\par 14 Iš jūsų, atvykusių gyventi į Egipto kraštą, nė vienas neišsigelbės, neištrūks ir nesugrįš į Judo kraštą, kurio jūs ilgitės ir į kurį norėtumėte sugrįžti, nebent kas pabėgtų”. 
\par 15 Tuomet visi vyrai, kurie žinojo, kad jų žmonos smilko svetimiems dievams, ir didelis būrys moterų, ir visi gyvenatys Patrose, Egipto žemėje, atsakė Jeremijui: 
\par 16 “Žodžių, kuriuos mums kalbėjai Viešpaties vardu, mes neklausysime. 
\par 17 Mes darysime tai, ką pasižadėjome: smilkysime dangaus karalienei ir liesime jai geriamąsias aukas, kaip mūsų tėvai, karaliai ir kunigaikščiai darė Judo miestuose ir Jeruzalės gatvėse. Tada turėjome pakankamai maisto ir nepatyrėme pikto. 
\par 18 Kai liovėmės smilkyti dangaus karalienei ir nebeaukojame geriamųjų aukų, kenčiame nepriteklių ir žūstame nuo kardo ir bado. 
\par 19 Argi mes smilkėme ir liejome aukas dangaus karalienei be mūsų vyrų žinios? Argi be jų sutikimo kepėme jai pyragaičius, kad ją pagarbintume, ir liejome geriamąsias aukas?” 
\par 20 Jeremijas atsakė vyrams, moterims ir visiems žmonėms: 
\par 21 “Smilkalus, kuriuos Judo miestuose ir Jeruzalės gatvėse deginote jūs, jūsų tėvai, karaliai, kunigaikščiai ir visi žmonės, Viešpats prisiminė. 
\par 22 Jis nebegalėjo daugiau pakęsti jūsų piktų darbų ir jūsų daromų bjaurysčių, todėl jūsų kraštas tapo dykyne, pasibaisėjimu bei keiksmažodžiu ir liko be gyventojų, kaip matome šiandien. 
\par 23 Kadangi jūs tomis aukomis nusikaltote Viešpačiui, neklausėte Jo, nesilaikėte įstatymo, įsakymų ir nuostatų, todėl jus ištiko šios nelaimės, kaip yra šiandien”. 
\par 24 Be to, Jeremijas kalbėjo visiems žmonėms ir visoms moterims: “Klausykite Viešpaties žodžio, visi žydai, gyvenantys Egipte: 
\par 25 ‘Taip sako kareivijų Viešpats, Izraelio Dievas: ‘Jūs ir jūsų žmonos, kaip pažadėjote, taip įvykdėte, sakydami: ‘Ištesėsime savo įžadus, smilkysime ir liesime geriamąsias aukas dangaus karalienei’. Ištesėkite savo įžadus ir įvykdykite, ką nusprendėte. 
\par 26 Bet išgirskite ir Viešpaties žodį, visi žydai, gyvenantys Egipto žemėje: ‘Aš prisiekiau savo vardu, kad nė vienas iš Judo žmonių nebesakys visoje Egipto žemėje: ‘Kaip Viešpats Dievas gyvas!’ 
\par 27 Aš darysiu jiems pikta, o ne gera; visi Judo žmonės Egipto žemėje žus nuo kardo ir bado, kol bus visai sunaikinti. 
\par 28 Tik labai mažas skaičius paspruks nuo kardo ir sugrįš iš Egipto į Judą. Tada Judo likutis, esantis Egipto krašte, pamatys, kieno žodis išsipildys­mano ar jų. 
\par 29 Štai jums ženklas, kad nubausiu jus šioje vietoje, kad žinotumėte, jog mano žodžiai prieš jus išsipildys. 
\par 30 Aš atiduosiu faraoną Hofrą, Egipto karalių, į rankas jo priešų, kurie siekia jo gyvybės, kaip atidaviau Zedekiją, Judo karalių, į jo priešo Nebukadnecaro, Babilono karaliaus, kuris siekė jo gyvybės, rankas,­sako kareivijų Viešpats, Izraelio Dievas’ ”.



\chapter{45}


\par 1 Pranašo Jeremijo žodžiai Baruchui, Nerijos sūnui, kai jis šiuos žodžius iš Jeremijo lūpų užrašė į knygą ketvirtais Johakimo, Jozijo sūnaus, Judo karaliaus, metais: 
\par 2 “Taip sako Viešpats, Izraelio Dievas, tau, Baruchai: 
\par 3 ‘Tu sakei: ‘Vargas man! Viešpats prideda man skausmų prie mano sielvarto. Pavargau vaitodamas, neturiu ramybės’. 
\par 4 Todėl taip kalbėjo Viešpats: ‘Ką Aš stačiau­nugriausiu ir ką sodinau­išrausiu. 
\par 5 Tu prašai sau didelių dalykų. Neprašyk, nes štai Aš bausiu visą kraštą, bet tavo gyvybę duosiu tau kaip grobį visose vietose, kur tik eisi’ ”.



\chapter{46}


\par 1 Viešpats kalbėjo pranašui Jeremijui apie pagonių tautas. 
\par 2 Apie Egiptą, apie faraono Nekojo, Egipto karaliaus, kariuomenę, buvusią prie Eufrato upės Karkemiše, kurią Nebukadnecaras, Babilono karalius, nugalėjo kervirtaisiais Johakimo, Jozijo sūnaus, Judo karaliaus, metais. 
\par 3 “Paruoškite mažąjį ir didįjį skydą ir traukite į kovą. 
\par 4 Balnokite žirgus ir sėskite ant jų. Rikiuokitės užsidėję šalmus, galąskite ietis, apsivilkite šarvais. 
\par 5 Kodėl Aš mačiau juos išsigandusius ir besitraukiančius atgal? Jų karžygiai išblaškyti bėga be atodairos, visur išgąstis,­sako Viešpats.­ 
\par 6 Eiklieji nepabėgs ir stiprieji neišsigelbės. Šiaurėje, prie Eufrato, jie susvyruos ir parkris. 
\par 7 Kas kyla lyg potvynis, lyg upės vanduo nerimsta? 
\par 8 Egiptas kyla kaip potvynis, kaip upės vanduo nerimsta. Jis sako: ‘Aš pakilsiu, apdengsiu žemę, sunaikinsiu miestus ir jų gyventojus!’ 
\par 9 Pirmyn, žirgai, skubėkite, kovos vežimai! Karžygiai pirmyn! Etiopai ir libiai su skydais, Lidijos gyventojai su įtemptais lankais! 
\par 10 Tai yra Viešpaties, kareivijų Dievo, keršto diena. Jis atkeršys savo priešams. Kardas ris ir bus sotus bei pasigers krauju. Tai bus auka Viešpačiui, kareivijų Dievui, šiaurės krašte, prie Eufrato. 
\par 11 Eik į Gileadą, atsinešk balzamo, mergele, Egipto dukra! Veltui vartoji vaistų daugybę, tu nepagysi! 
\par 12 Tautos išgirdo apie tavo gėdą, pilna žemė tavo šauksmo. Galiūnas susidūrė su galiūnu, ir abu kartu krito!” 
\par 13 Viešpats kalbėjo pranašui Jeremijui, kad Nebukadnecaras, Babilono karalius, ateis ir užpuls Egipto žemę. 
\par 14 “Praneškite Migdole, paskelbkite Tachpanhese ir Nofe! Sakykite: ‘Atsistokite ir pasiruoškite, nes kardas jau ryja aplinkui!’ 
\par 15 Kodėl tavo karžygiai išvaikyti? Jie neatsilaikė, nes Viešpats juos parklupdė. 
\par 16 Tavo daugybė susvyravo ir krito vienas po kito. Jie sakė: ‘Grįžkime pas savo tautą, į savo gimtinę, bėkime nuo žudančio kardo!’ 
\par 17 Jie šaukė apie faraoną, Egipto karalių: ‘Jis yra triukšmas po laiko’. 
\par 18 Kaip Aš gyvas,­sako Karalius, kurio vardas­kareivijų Viešpats,­kaip Taboras yra tarp kalnų ir Karmelis prie jūros, taip tikrai jis ateis. 
\par 19 Pasiruoškite tremčiai, Egipto gyventojai! Nofas pavirs dykyne be gyventojų. 
\par 20 Egiptas yra puiki telyčia, bet iš šiaurės ateina sunaikinimas. 
\par 21 Jo samdyti kariai yra kaip nupenėti veršiai. Jie visi apsigręžė ir pabėgo. Jie neatsilaikė, nes atėjo pražūties diena, priartėjo aplankymo metas. 
\par 22 Jų balsas yra kaip gyvatės šnypštimas. Jie ateina su kirviais tarytum miško kirtėjai. 
\par 23 Jie iškirs mišką, kuris buvo nepereinamas. Jų yra nesuskaitoma daugybė kaip skėrių. 
\par 24 Egipto duktė bus sugėdinta, ji bus atiduota į šiaurės tautos rankas”. 
\par 25 Kareivijų Viešpats, Izraelio Dievas, sako: “Aš nubausiu Noją, faraoną, Egiptą su jų dievais ir karaliais; faraoną ir visus, kurie juo pasitiki. 
\par 26 Aš atiduosiu juos į rankas tų, kurie siekia jų gyvybės, į Nebukadnecaro, Babilono karaliaus, ir jo tarnų rankas. Po daug metų kraštas vėl bus apgyvendintas kaip senais laikais,­sako Viešpats.­ 
\par 27 Tu, mano tarne Jokūbai, nebijok, Izraeli, neišsigąsk! Aš tave išgelbėsiu ir tavo palikuonis parvesiu iš nelaisvės, iš tolimo krašto. Jokūbas sugrįš, turės ramybę, gyvens saugiai, ir niekas jo negąsdins. 
\par 28 Tu, mano tarne Jokūbai, nebijok,­sako Viešpats.­Aš esu su tavimi. Aš visiškai sunaikinsiu tautas, į kurias tave ištrėmiau, bet tavęs iki galo nesunaikinsiu. Aš bausiu tave saikingai, bet be bausmės nepaleisiu”.



\chapter{47}


\par 1 Viešpats kalbėjo pranašui Jeremijui apie filistinus, prieš faraonui užimant Gazą: 
\par 2 “Vandenys kyla iš šiaurės ir tampa patvinusia upe, apsemia kraštą ir visa, kas jame, miestus ir jų gyventojus. Žmonės šauks, visi krašto gyventojai dejuos 
\par 3 nuo žirgų trypimo, kovos vežimų dundėjimo, jų ratų dardėjimo. Tėvai nebesirūpins vaikais, jų rankos nusvirs. 
\par 4 Apiplėšimo diena ateina visam filistinų kraštui, kad sunaikintų paskutinius Tyro ir Sidono padėjėjus. Viešpats sunaikins filistinus, Kaftoro krašto likutį. 
\par 5 Gaza nupliko, Aškelonas sunaikintas, kartu ir jų slėnių liekanos. Ar ilgai tu raižysi save? 
\par 6 Viešpaties karde, ar ilgai tu nenurimsi? Sugrįžk atgal į makštį, liaukis ir nurimk! 
\par 7 Kaip jis gali nurimti, kai Viešpats jį pasiuntė prieš Aškeloną ir pajūrį?”



\chapter{48}


\par 1 Kareivijų Viešpats, Izraelio Dievas, apie Moabą sako: “Vargas Nebojui, jis apiplėštas; Kirjataimai paimti, tvirtovė sugėdinta ir sunaikinta. 
\par 2 Moabo garbė praėjo. Priešai Hešbone galvojo tave sunaikinti: ‘Pulkime, sunaikinkime Moabą ir pašalinkime jį iš tautų tarpo!’ Tu, Madmeno mieste, irgi nutilsi, kardas sunaikins tave! 
\par 3 Šauksmas girdimas Horonaimuose, plėšimas ir didelis sunaikinimas. 
\par 4 Moabas sunaikintas, verkia jo kūdikiai. 
\par 5 Jie kyla Luhito šlaitu verkdami, nusileidžia į Horonaimus, jų priešai girdi verksmą dėl sunaikinimo. 
\par 6 Bėkite, būkite kaip kadagys dykumoje. 
\par 7 Kadangi pasitikėjai savo darbais ir turtais, tu būsi paimtas. Kemošas išeis į nelaisvę kartu su kunigais ir kunigaikščiais. 
\par 8 Sunaikinimas pasieks kiekvieną miestą, nė vienas neišsigelbės. Slėniai ir lygumos bus sunaikintos. 
\par 9 Duokite Moabui sparnus, kad jis galėtų pabėgti! Jis bus visai sunaikintas, miestai ištuštės. 
\par 10 Prakeiktas, kas Viešpaties įsakymą nenoriai vykdo ir kas saugo savo kardą nuo kraujo. 
\par 11 Moabas gyveno be rūpesčių nuo pat savo jaunystės, jo mielės nusėdo; jis nebuvo perpilamas iš indo į indą ir nebuvo ištremtas. Todėl jo skonis liko tas pats ir kvapas nepasikeitė. 
\par 12 Ateis laikas, kai Aš siųsiu jam pilstytojų, kurie jį perpils, jo ąsočius ištuštins ir sudaužys. 
\par 13 Moabas gėdysis Kemošo, kaip Izraelis gėdijosi Betelio, kuriuo pasitikėjo. 
\par 14 Kaip galite sakyti: ‘Mes esame galingi vyrai, karžygiai!’ 
\par 15 Moabas apiplėštas, jo miestai sunaikinti, rinktiniai jaunuoliai išėjo į pražūtį,­sako Karalius, kareivijų Viešpats.­ 
\par 16 Moabo sunaikinimas artėja. 
\par 17 Apverkite jį, kaimynai ir visi, kurie žinote jo vardą. Sulaužytas jo stiprusis skeptras, puikioji lazda! 
\par 18 Nusileisk iš savo šlovės sosto į purvą, Dibono dukra! Moabo naikintojas ateis ir sugriaus tavo tvirtoves. 
\par 19 Sustok pakelėje, Aroero gyventoja, ir klausk pabėgėlį, kas atsitiko? 
\par 20 Moabas yra sumuštas ir nugalėtas! Šaukite ir dejuokite! Praneškite Arnone, kad Moabas apiplėštas. 
\par 21 Bausmė atėjo lygumos kraštui: Holonui, Jahacui, Mefaatui, 
\par 22 Dibonui, Nebojui, Bet Diblataimams, 
\par 23 Kirjataimams, Bet Gamului, Bet Meonui, 
\par 24 Kerijotams, Bocrai ir visiems Moabo miestams. 
\par 25 Nukirstas Moabo ragas ir jo petys sutriuškintas,­sako Viešpats.­ 
\par 26 Nugirdykite jį, nes jis didžiavosi prieš Viešpatį. Moabas voliosis savo vėmaluose ir taps pajuoka. 
\par 27 Ar nesityčiojai iš Izraelio, lyg jis būtų vagis? 
\par 28 Moabo gyventojai, pasitraukite iš miestų ir gyvenkite uolose kaip balandžiai, susikrovę lizdą aukštai skardžiuose. 
\par 29 Mes girdėjome apie Moabo išdidumą, puikybę, akiplėšiškumą ir pasididžiavimą. 
\par 30 Aš žinau, kad jo pasigyrimas yra tuščias ir darbai niekam tikę. 
\par 31 Aš verkiu ir dejuoju Moabo ir Kir Hereso žmonių. 
\par 32 Aš verkiu dėl tavęs, Sibmos vynuogyne, daugiau negu dėl Jazero. Tavo atžalos nusidriekė per jūrą ir pasiekė Jazerą. Naikintojas užpuolė tavo vasaros vaisius ir vynuogyno derlių. 
\par 33 Džiaugsmas ir linksmybė dingo iš derlingų Moabo laukų. Aš pašalinau vyną iš spaustuvo, vyno mynėjas nebemina jo, džiaugsmo šūksnių negirdėti. 
\par 34 Šauksmas iš Hešbono pasiekia Elealę, jų aimanos girdimos iki Jahaco, o iš Coaro­iki Horonaimų ir Eglat Šelišijos. Ir Nimrimų vandenys išseks. 
\par 35 Aš sustabdysiu Moabe aukojimą ir smilkymą jų dievams aukštumose. 
\par 36 Mano širdis dejuoja kaip fleita dėl Moabo ir Kir Hereso žmonių. Jie neteko visų savo turtų, kuriuos turėjo. 
\par 37 Visų galvos nuskustos ir barzdos nukirptos; rankos suraižytos ir strėnos padengtos ašutinėmis. 
\par 38 Ant visų Moabo stogų ir aikštėse girdisi tik dejavimas. Aš sudaužiau Moabą kaip netinkamą indą,­sako Viešpats.­ 
\par 39 Jie dejuos, sakydami: ‘Kaip sudaužytas, kaip sugėdintas Moabas!’ Jis taps pajuoka ir pasibaisėjimu visoms aplinkinėms tautoms. 
\par 40 Jis atskrenda kaip erelis ir ištiesia sparnus virš Moabo. 
\par 41 Jis paims tvirtoves ir miestus. Tą dieną Moabo kariai bus nuliūdę ir išsigandę kaip moterys. 
\par 42 Moabo tauta bus sunaikinta, nes ji didžiavosi prieš Viešpatį. 
\par 43 Išgąstis, duobė ir spąstai laukia jūsų, Moabo gyventojai! 
\par 44 Kas pabėgs nuo išgąsčio, įkris į duobę, kas išlips iš duobės, pateks į spąstus. Tai ištiks Moabą jų aplankymo metu. 
\par 45 Bėgantys ir netekę jėgų sustos Hešbono pavėsyje. Bet ugnis išeis iš Hešbono ir liepsna iš Sihono ir praris Moabo kaktą ir triukšmadarių galvos vainiką. 
\par 46 Vargas tau, Moabai! Tu žuvai, Kemošo tauta! Tavo sūnūs yra ištremti, tavo dukterys pateko į nelaisvę. 
\par 47 Bet Aš parvesiu Moabo ištremtuosius paskutinėmis dienomis,­ sako kareivijų Viešpats”. Toks yra Moabo teismas.



\chapter{49}


\par 1 Viešpats sako amonitams: “Argi Izraelis neturi paveldėtojų? Kodėl Milkomas paveldėjo Gadą ir jo žmonės apsigyveno miestuose? 
\par 2 Ateis diena, kai Aš leisiu amonitų Rabai patirti karą. Ji pavirs griuvėsiais, o aplinkiniai miestai bus sudeginti; Izraelis atgaus savo nuosavybę. 
\par 3 Vaitok, Hešbone, nes Ajas sunaikintas. Šaukite, Rabos dukterys, apsivilkite ašutinėmis, raudokite, bėgiodamos patvoriais, nes Milkomas ir jo kunigai bei kunigaikščiai eis į nelaisvę. 
\par 4 Ko tu giries savo slėniais, nuklydusioji dukra? Pasitiki savo turtais, sakydama: ‘Kas drįs eiti prieš mane?’ 
\par 5 Aš atvesiu prieš tave siaubą,­ sako Viešpats, kareivijų Dievas.­ Apsupę priešai taip jus išsklaidys, kad visi išbėgios ir niekas jų nebesurinks. 
\par 6 Bet po to Aš išlaisvinsiu amonitus,­sako kareivijų Viešpats”. 
\par 7 Kareivijų Viešpats sako Edomui: “Ar nebėra išminties Temane? Ar išminčiai nebeduoda patarimų, ar jų išmintis išseko? 
\par 8 Bėkite ir pasislėpkite slėptuvėse, Dedano gyventojai! Aš sunaikinsiu Ezavą, aplankydamas jį. 
\par 9 Ar vynuogių skynėjai nepaliktų kiek vynuogių? Jei vagys įsilaužtų nakčia, ar jie nepavogtų tik tiek, kiek jiems reikia? 
\par 10 Bet Aš apnuoginau Ezavą, Aš atidengiau jo paslaptis, ir jis nebeturės, kur pasislėpti. Jo vaikai, broliai ir kaimynai bus sunaikinti. 
\par 11 Palik savo našlaičius, Aš išsaugosiu juos, tavo našlės tepasitiki manimi. 
\par 12 Jei nekaltieji turės gerti taurę, tai ar tu liksi nebaustas? Tu neišvengsi bausmės! 
\par 13 Aš prisiekiau pats savimi, kad Bocra taps siaubu, virs dykuma bei keiksmažodžiu; visi jos miestai bus amžini griuvėsiai,­sako Viešpats”. 
\par 14 Aš sužinojau iš Viešpaties, kad pasiuntinys yra pasiųstas tautoms. Jis kviečia visus į kovą prieš jį. 
\par 15 “Tu būsi mažas tarp tautų ir paniekintas žmonėse. 
\par 16 Tavo smarkumas ir širdies išdidumas apgavo tave. Tu gyveni uolų plyšiuose, laikaisi kalvų viršūnėse. Jei tu susikrautum savo lizdą taip aukštai kaip erelis, Aš nustumčiau tave žemyn,­sako Viešpats.­ 
\par 17 Edomas taps pasibaisėjimu; kiekvienas praeivis švilps ir baisėsis jo likimu. 
\par 18 Kaip sunaikinta Sodoma ir Gomora bei jų apylinkės, taip ir čia nepasiliks ir negyvens joks žmogus. 
\par 19 Kaip liūtas iš Jordano tankynės jis pakyla ir ateina prieš stipriųjų buveines, bet Aš jį nuvysiu ir paskirsiu tą, kurį išsirinksiu. Kas yra man lygus ir kas gali man pasipriešinti? 
\par 20 Štai Viešpaties nutarimas Edomui ir sprendimas Temano gyventojams. Iš tiesų net menkiausi iš bandos juos ištrems ir jų buveinės liks apleistos. 
\par 21 Nuo jų griuvimo trenksmo sudrebės žemė, ir jų šauksmo balsas bus girdimas prie Raudonosios jūros. 
\par 22 Jis pakils kaip erelis, atskris ir išskės sparnus virš Bocros. Tą dieną Edomo kariai bus nuliūdę ir išsigandę kaip moterys”. 
\par 23 Apie Damaską: “Sąmyšis Hamate ir Arpade, nes bloga žinia pasiekė juos. Jie išsigandę nerimauja kaip neramios jūros bangos. 
\par 24 Damaskas nusiminęs bėga baimės apimtas; baimė ir skausmai užgriuvo jį kaip gimdyvę. 
\par 25 Garsusis miestas ištuštėjo, mano džiaugsmo miestas. 
\par 26 Jo jaunuoliai ir kariai žus gatvėse tą dieną,­sako kareivijų Viešpats,­ 
\par 27 Aš įžiebsiu ugnį Damasko sienose, ji praris Ben Hadado rūmus”. 
\par 28 Apie Kedarą ir Hacoro karalystes, kurias nugalėjo Nebukadnecaras, Babilono karalius, Viešpats sako: “Pasiruoškite ir žygiuokite prieš Kedarą, apiplėškite rytų tautas. 
\par 29 Jie paims jų palapines ir bandas, apdangalus, indus ir kupranugarius. Ir jiems šauks: ‘Siaubas iš visų pusių!’ 
\par 30 Skubiai bėkite ir pasislėpkite, Hacoro gyventojai. Nebukadnecaras, Babilono karalius, tarėsi prieš jus ir nusprendė dėl jūsų. 
\par 31 Pakilk prieš šitą turtingą, ramiai gyvenančią tautą,­sako Viešpats.­Jie neturi nei vartų, nei užkaiščių, gyvena atsiskyrę. 
\par 32 Jų kupranugariai ir didžiulės bandos taps grobiu. Aš išsklaidysiu į visas šalis vyrus, kurie kerpasi plaukus, ir bausiu juos iš visų pusių,­sako Viešpats,­ 
\par 33 Hacore gyvens šakalai, jis liks amžina dykuma. Jame negyvens joks žmogus”. 
\par 34 Viešpats kalbėjo pranašui Jeremijui apie Elamą, pradedant karaliauti Zedekijui, Judo karaliui: 
\par 35 “Aš sulaužysiu Elamo lanką, jo didžiausią stiprybę. 
\par 36 Aš pasiųsiu ant Elamo keturis vėjus iš keturių pasaulio šalių ir juos išsklaidysiu į visus pasaulio kraštus taip, kad nebus tautos, kurioje Elamo išsklaidytųjų nebūtų. 
\par 37 Aš sukelsiu Elame baimę priešų, kurie siekia jo gyvybės, ir atvesiu prieš juos nelaimę, savo rūstybę, siųsiu paskui juos kardą ir sunaikinsiu juos. 
\par 38 Aš pastatysiu Elame savo sostą ir pašalinsiu jų karalius bei kunigaikščius. 
\par 39 Bet paskutinėmis dienomis Aš išvaduosiu Elamą,­sako Viešpats”.



\chapter{50}


\par 1 Viešpats kalbėjo pranašui Jeremijui apie Babiloną ir chaldėjų kraštą: 
\par 2 “Paskelbkite tautoms, iškelkite vėliavas, neslėpkite, kad Babilonas paimtas, Belis nebegarbinamas, Merodachas sunaikintas! Jų stabai išniekinti, atvaizdai sudaužyti. 
\par 3 Iš šiaurės prieš jį ateina tauta. Ji pavers kraštą dykyne; žmonės ir gyvuliai jame nebegyvens, pabėgs iš jo. 
\par 4 Tuomet sugrįš Izraelio ir Judo vaikai,­sako Viešpats.­Jie eis verkdami ir ieškos Viešpaties, savo Dievo. 
\par 5 Jie klaus kelio į Sioną ir keliaus, sakydami: ‘Eikime, glauskimės prie Viešpaties amžina sandora, kuri nebus užmiršta!’ 
\par 6 Mano tauta tapo paklydusia banda. Ganytojai ją suvedžiojo ir paklaidino kalnuose. Jie ėjo per kalnus bei kalvas ir užmiršo savo poilsio vietą. 
\par 7 Kas juos sutiko, rijo juos. Jų priešai sakė: ‘Mes tuo nenusikaltome. Izraelitai nusikalto Viešpačiui, teisingumo buveinei, ir Viešpačiui, savo tėvų vilčiai’. 
\par 8 Skubėkite iš Babilono, traukitės iš chaldėjų krašto! Būkite kaip ožiai bandos priekyje. 
\par 9 Aš sukelsiu prieš Babiloną daug galingų tautų ir atvesiu jas iš šiaurės. Jos išsirikiuos ir nugalės jį. Visos jų strėlės įgudusio kario rankose, jos pasiekia tikslą. 
\par 10 Chaldėja taps grobiu, jos priešai prisiplėš turto užtektinai,­sako Viešpats,­ 
\par 11 nes jūs džiaugėtės ir didžiavotės, mano paveldo grobėjai, šokinėjote kaip veršiai ant žolės ir baubėte kaip jaučiai. 
\par 12 Jūsų motina bus išniekinta ir sugėdinta. Ji bus paskutinė tarp tautų, virs dykyne, sausa žeme, dykuma. 
\par 13 Dėl Viešpaties rūstybės ji bus negyvenama. Kiekvienas, praeinantis pro Babiloną, stebėsis ir švilps dėl jo nelaimės. 
\par 14 Išsirikiuokite prieš Babiloną, įtempkite lankus prieš jį, šaukite, negailėkite strėlių, nes jis nusikalto Viešpačiui. 
\par 15 Skelbkite visur, kad jis paimtas. Jo apsaugos pylimas krito, sienos nugriautos. Tai Viešpaties kerštas jam už jo darbus. 
\par 16 Išnaikinkite Babilone sėjėją ir pjovėją. Karui siaučiant, kiekvienas bėgs į savo kraštą, pas savo tautą. 
\par 17 Izraelis yra kaip išsklaidytos avys, kurias išvaikė liūtai. Pirmasis jį rijo Asirijos karalius, o po to Nebukadnecaras, Babilono karalius, sutraiškė jo kaulus”. 
\par 18 Todėl taip sako kareivijų Viešpats, Izraelio Dievas: “Aš nubausiu Babilono karalių ir jo kraštą, kaip nubaudžiau Asirijos karalių. 
\par 19 Izraelį Aš parvesiu atgal į savo kraštą. Jis ganysis Karmelyje ir Bašane, pasisotins Efraimo kalnyne bei Gileade. 
\par 20 Tuo metu ieškos Izraelio kaltės ir Judo nuodėmės, bet jų neras, nes Aš atleisiu jiems ir jų nesunaikinsiu,­sako kareivijų Viešpats.­ 
\par 21 Žygiuok prieš Merataimų ir Pekodo kraštų gyventojus! Žudyk ir naikink,­sako Viešpats,­daryk taip, kaip tau įsakiau! 
\par 22 Krašte girdėti šauksmai kovos ir didelio sunaikinimo. 
\par 23 Visos žemės kūjis pats sudaužytas ir sutrupėjęs. Babilonas tapo dykyne tarp tautų. 
\par 24 Babilone, Aš stačiau tau spąstus ir sugavau tave. Tu to nepastebėjai, bet buvai surastas ir sugautas, nes kovojai prieš Viešpatį. 
\par 25 Viešpats atidarė savo ginklų sandėlį ir ištraukė savo rūstybės ginklus, nes tai yra Viešpaties, kareivijų Dievo, darbas chaldėjų krašte. 
\par 26 Pakilkite prieš jį, visi kraštai, atidarykite jo grūdų sandėlius, supilkite viską į krūvas ir sunaikinkite­tenelieka nieko. 
\par 27 Išžudykite jo veršius, teeina jie į skerdyklą. Vargas jiems! Atėjo jų aplankymo diena. 
\par 28 Štai pabėgėliai iš Babilono krašto! Jie praneša Sione apie Viešpaties kerštą, apie mūsų Dievo kerštą dėl Jo šventyklos. 
\par 29 Surinkite šaulius prieš Babiloną. Apsupkite jį taip, kad nė vienas neištrūktų! Atmokėkite jam pagal jo darbus; ką jis darė, jam darykite, nes jis didžiavosi prieš mane, Izraelio Šventąjį. 
\par 30 Jo jaunuoliai kris aikštėse ir visi jo kariai bus sunaikinti tą dieną,­sako Viešpats.­ 
\par 31 Aš esu prieš tave, tu išdidusis! Atėjo tavo aplankymo metas. 
\par 32 Išdidusis suklups ir kris, nė vienas jo nepakels. Aš įžiebsiu jo miestuose ugnį, kuri suris viską aplinkui”. 
\par 33 Kareivijų Viešpats sako: “Prispausti yra Izraelio ir Judo vaikai. Tie, kurie juos išvedė į nelaisvę, laiko juos ir nė nemano jų paleisti. 
\par 34 Jų Atpirkėjas yra stiprus, kareivijų Viešpats yra Jo vardas. Jis rūpinsis jų byla ir suteiks kraštui ramybę, bet privers drebėti Babilono gyventojus. 
\par 35 Kardas chaldėjams, Babilono gyventojams, jo kunigaikščiams ir išminčiams! 
\par 36 Kardas jo žyniams, kurie taps kvaili, ir kariams, kad išsigąstų. 
\par 37 Kardas žirgams, kovos vežimams ir samdytiems kariams, kurie taps kaip moterys. Kardas jo turtams, kurie taps grobiu! 
\par 38 Sausra išdžiovins jo vandenis. Tai drožtų atvaizdų kraštas, per savo stabus jie sukvailėjo. 
\par 39 Todėl ten gyvens laukiniai žvėrys, šakalai ir stručiai; Babilonas niekados nebus apgyvendintas nė atstatytas. 
\par 40 Kaip Dievas sunaikino Sodomą, Gomorą ir jų aplinkinius miestus, taip ir Babilonas bus sunaikintas, niekas jame negyvens. 
\par 41 Galinga tauta ateina iš šiaurės ir daug karalių iš žemės pakraščių. 
\par 42 Jie ginkluoti lankais ir ietimis, žiaurūs bei negailestingi. Jie atūžia kaip jūra, joja ant žirgų, pasirengę kovai prieš tave, Babilono dukra! 
\par 43 Babilono karalius, išgirdęs apie juos, nuleido rankas; jį apėmė baimė ir skausmai tarsi gimdyvę. 
\par 44 Kaip liūtas iš Jordano tankynės jis pakyla ir ateina prieš stipriųjų buveines, bet Aš jį nuvysiu ir paskirsiu tą, kurį išsirinksiu. Kas yra man lygus ir kas gali man pasipriešinti? Koks valdovas galėtų man prieštarauti? 
\par 45 Šai Viešpaties nutarimas Babilonui ir sprendimas chaldėjų kraštui. Iš tiesų net menkiausi iš bandos juos ištrems ir jų buveinės liks apleistos. 
\par 46 Babilono paėmimo triukšmas sudrebins žemę, ir jų šauksmą išgirs visos tautos”.



\chapter{51}


\par 1 Taip sako Viešpats: “Aš pažadinsiu prieš Babilono ir Chaldėjos gyventojus naikinantį vėją. 
\par 2 Aš siųsiu į Babiloną vėtytojus. Jie vėtys jį ir ištuštins šalį. Ji bus apsupta iš visų pusių. 
\par 3 Šauliai, įtempkite lankus, šarvuotieji, pakilkite. Nesigailėkite jos jaunuolių ir sunaikinkite visą jos kariuomenę. 
\par 4 Žuvusieji ir sunkiai sužeisti chaldėjų krašte gulės gatvėse. 
\par 5 Izraelio ir Judo neatstūmė Viešpats, jų Dievas, nors jų kraštas yra pilnas nusikaltimų prieš Izraelio Šventąjį. 
\par 6 Bėkite iš Babilono, gelbėkitės, nežūkite dėl jo nusikaltimų. Tai Viešpaties keršto diena, ir Jis atlygins jam už jo darbus. 
\par 7 Babilonas buvo auksinė taurė Viešpaties rankoje, visa žemė pasigėrė iš jos. Jos vyno gėrė tautos, todėl jos išprotėjo. 
\par 8 Babilonas krito, jis sutriuškintas. Apraudokite jį, atneškite balzamo jo žaizdoms, gal jis pagis?” 
\par 9 Mes gydėme Babiloną, bet jis nepagijo. Palikime jį ir grįžkime kiekvienas į savo kraštą. Jo teismas pasiekė dangų. 
\par 10 Viešpats iškėlė mūsų teisumą. Eikime ir pasakokime Sione, ką Viešpats, mūsų Dievas, padarė. 
\par 11 Galąskite strėles, imkite skydus! Viešpats sukėlė medų karalius, nes Jo sumanymas yra sunaikinti Babiloną. Tai Viešpaties kerštas dėl šventyklos. 
\par 12 Iškelkite vėliavą prieš Babilono sienas, sustiprinkite sargybą, paruoškite pasalas. Ką Viešpats nusprendė, tai ir padarys Babilono gyventojams. 
\par 13 Tu, kuris gyveni prie gausių vandenų ir turi gausybę turtų. Atėjo tavo galas, tavo godumo saikas. 
\par 14 Kareivijų Viešpats prisiekė: “Tave užplūs žmonės kaip skėriai ir pakels prieš tave savo balsus”. 
\par 15 Jis savo jėga sukūrė žemę, savo išmintimi padėjo pasaulio pamatą ir savo supratimu ištiesė dangų. 
\par 16 Jo balso klauso vandenys danguose, Jis pakelia garus nuo žemės pakraščių. Jis siunčia žaibus su lietumi, paleidžia vėją iš savo sandėlių. 
\par 17 Žmogus neturi pažinimo ir yra neišmintingas. Amatininkai bus sugėdinti dėl savo drožinių, jų lieti atvaizdai yra apgaulė, juose nėra kvapo. 
\par 18 Jie yra tuštybė, paklydimo darbai. Jie pražus aplankymo dieną. 
\par 19 Visai kitokia yra Jokūbo dalis. Jis yra visa ko Kūrėjas, Izraelis yra Jo nuosavybė. Kareivijų Viešpats yra Jo vardas. 
\par 20 “Tu esi mano kūjis ir kovos ginklas. Tavimi sudaužysiu tautas ir sunaikinsiu karalystes. 
\par 21 Tavimi sunaikinsiu žirgą ir raitelį, kovos vežimą ir jame esantį. 
\par 22 Tavimi sunaikinsiu vyrą ir moterį, seną ir jauną, jaunuolį ir mergaitę, 
\par 23 piemenį ir bandą, artoją ir jungą su gyvuliais, kunigaikščius ir valdovus. 
\par 24 Bet Aš atlyginsiu Babilonui ir visiems Chaldėjos gyventojams jūsų akivaizdoje už jų piktybes, padarytas Sione,­sako Viešpats.­ 
\par 25 Tu buvai naikinantis kalnas, sugadinęs visą žemę. Aš ištiesiu savo ranką prieš tave, sulyginsiu tave su žeme ir paversiu pelenais. 
\par 26 Tavo akmenų nenaudos nei kampams, nei pamatams. Tu būsi amžina dykyne. 
\par 27 Iškelkite vėliavą, trimituokite trimitais, kad išgirstų tautos. Sušaukite prieš jį Ararato, Minio ir Aškenazo karalystes. Paskirkite kariuomenei vadą ir surinkite tiek karių kaip skėrių laukuose. 
\par 28 Pasiruoškite kovai kartu su medų karaliais, valdovais, kunigaikščiais ir visais jų valdžioje esančiais kraštais. 
\par 29 Žemė pajudės ir drebės, nes Babilonui bus įvykdytas Viešpaties sprendimas. Babilono kraštas taps tuščias ir negyvenamas”. 
\par 30 Babilono kariai nebeina į kovą. Jie sėdi tvirtovėse netekę drąsos. Jų gyvenvietės dega, vartai išlaužti. 
\par 31 Pasiuntinys sutinka pasiuntinį. Jie neša žinią Babilono karaliui, kad jo miestas paimtas iš visų pusių: 
\par 32 brastos užimtos, įtvirtinimai dega, kariai apimti panikos. 
\par 33 Kareivijų Viešpats, Izraelio Dievas, sako: “Babilonas yra kaip klojimas kūlimo metu. Dar valandėlė, ir derliaus metas ateis”. 
\par 34 “Nebukadnecaras, Babilono karalius, ėdė mane ir naikino, paliko mane kaip tuščią indą. Jis prarijo mane kaip slibinas. Pripildęs savo pilvą mano gardumynais, mane išstūmė. 
\par 35 Man ir mano žmonėms padaryta skriauda tekrinta ant Babilono”,­sakys Sionas. “Mano kraujas tekrinta ant Chaldėjos”,­sakys Jeruzalė. 
\par 36 Viešpats sako: “Aš ginsiu tavo bylą ir atkeršysiu už tave. Aš išdžiovinsiu Babilono vandenis, jo šaltiniai išseks. 
\par 37 Babilonas pavirs griuvėsių krūva šakalams gyventi, vieta pasibaisėjimo ir pajuokos, be gyventojų. 
\par 38 Jie riaumos kaip liūtai, staugs kaip liūtų jaunikliai. 
\par 39 Aš jiems paruošiu puotą: jie nusigers ir užmigs amžinu miegu. 
\par 40 Aš juos nuvesiu į skerdyklą kaip avinėlius, avinus ir ožius. 
\par 41 Krito Šešachas, pasaulio puošmena! Babilonas tapo siaubu tautoms! 
\par 42 Jūra įsiveržė į Babiloną, daugybė bangų užliejo jį. 
\par 43 Jo miestai virto dykyne, išdžiūvusia žeme, kurioje niekas negyvena ir joks žmogus per ją nekeliauja. 
\par 44 Aš nubausiu Belį Babilone ir išplėšiu iš jo gerklės, ką jis prarijo. Tautos nebeplauks pas jį. Babilono sienos krito. 
\par 45 Išeik iš jo, mano tauta! Kiekvienas gelbėkite savo gyvybę nuo degančios Viešpaties rūstybės. 
\par 46 Nenusiminkite, neišsigąskite gandų, kurie kas metai sklis krašte apie neramumus ir valdovų tarpusavio kovas. 
\par 47 Ateina laikas, kai Aš teisiu Babilono atvaizdus; visas kraštas susigės, o jo gyventojai bus išžudyti. 
\par 48 Tada dangus, žemė ir visa, kas juose yra, džiaugsis žuvimu Babilono, kurį užims iš šiaurės atėjęs naikintojas. 
\par 49 Kaip Babilono ranka žudė Izraelyje, taip Babilone kris nužudytieji. 
\par 50 Jūs, kurie ištrūkote nuo kardo, eikite, nestovėkite vietoje, ir, toli būdami, atsiminkite Viešpatį ir Jeruzalę”. 
\par 51 Mes susigėdome, girdėdami pajuokas; gėda apdengė mūsų veidus, kai svetimi atėjo į šventąją vietą Viešpaties namuose. 
\par 52 “Ateis diena, kai Aš nuteisiu jų drožinius; tuomet visame krašte vaitos sužeistieji. 
\par 53 Jei Babilonas pakiltų iki dangaus ir savo pilis pastatytų iki debesų, mano siųstas naikintojas užklups jį”,­sako Viešpats. 
\par 54 Šauksmas kyla iš Babilono, baisus sunaikinimas Chaldėjos krašte, 
\par 55 nes Viešpats plėšia Babiloną ir tildo jo galingą balsą, nors jo bangos šniokščia kaip galingi vandenys, girdimi jų triukšmingi balsai. 
\par 56 Priešas veržiasi į Babiloną; jo kariai patenka į nelaisvę, jų lankai sulaužyti. Viešpats, atlygio Dievas, tikrai atlygins. 
\par 57 “Aš nugirdysiu Babilono kunigaikščius, išminčius, valdovus, karo vadus ir karius. Jie užmigs amžinu miegu ir nepabus”,­sako Karalius, kareivijų Viešpats. 
\par 58 “Plačioji Babilono siena bus sulyginta su žeme ir aukštieji vartai sudeginti. Tautos vargo veltui, giminės dirbo ir statė ugniai”,­sako Viešpats. 
\par 59 Pranašo Jeremijo žodis Serajai, Machsėjos sūnaus Nerijos sūnui, kai jis lydėjo Zedekiją, Judo karalių, jo ketvirtais karaliavimo metais į Babiloną. Seraja buvo žymus kunigaikštis. 
\par 60 Jeremijas užrašė į knygą visas nelaimes, kurios ištiks Babiloną, visus žodžius, kurie parašyti prieš Babiloną. 
\par 61 Jeremijas sakė Serajai: “Nuvykęs į Babiloną, perskaityk visus šiuos žodžius 
\par 62 ir sakyk: ‘Viešpatie, Tu grasinai šitą vietą taip sunaikinti, kad čia nebebūtų nieko: nei žmonių, nei gyvulių, ir ji liktų amžina dykyne’. 
\par 63 Perskaitęs šią knygą, pririšk prie jos akmenį ir įmesk ją į Eufrato upę, 
\par 64 sakydamas: ‘Taip įvyks su Babilonu. Jis paskęs ir nebepakils dėl visų nelaimių, kurias Viešpats jam užves’ ”. Tiek Jeremijo žodžių.



\chapter{52}


\par 1 Zedekijas buvo dvidešimt vienerių metų, kai tapo karaliumi. Jis karaliavo vienuolika metų Jeruzalėje. Jo motina buvo Hamutalė, Jeremijo duktė iš Libnos. 
\par 2 Jis darė pikta Viešpaties akyse kaip ir Jehojakimas. 
\par 3 Dėl Viešpaties rūstybės taip atsitiko Jeruzalei ir Judui, kad galiausiai Jis pašalino juos iš savo akių. Ir Zedekijas sukilo prieš Babilono karalių. 
\par 4 Devintaisiais Zedekijo karaliavimo metais, dešimto mėnesio dešimtą dieną, Babilono karalius Nebukadnecaras atėjo su visa kariuomene prieš Jeruzalę, ją apgulė ir supylė aplinkui pylimą. 
\par 5 Miestas buvo apgultas iki vienuoliktų karaliaus Zedekijo metų. 
\par 6 Ketvirto mėnesio devintą dieną mieste taip sustiprėjo badas, kad žmonės nebeturėjo ko valgyti. 
\par 7 Pralaužę miesto sieną, karalius ir visi jo kariai pabėgo naktį taku, esančiu tarp dviejų miesto sienų, pro karaliaus sodą. Chaldėjai buvo apsupę miestą. Jie traukėsi lygumos keliu. 
\par 8 Chaldėjų kariuomenė vijosi karalių ir sugavo Zedekiją lygumoje prie Jericho. Visa jo kariuomenė buvo išsklaidyta. 
\par 9 Jie suėmė karalių ir atgabeno pas Babilono karalių į Riblą Hamato krašte, ir tas jį teisė. 
\par 10 Babilono karalius nužudė Zedekijo sūnus jo akyse, taip pat ir visus Judo kunigaikščius Ribloje. 
\par 11 Tada Zedekijui išlupo akis, sukaustė grandinėmis, išsivedė į Babiloną ir įmetė į kalėjimą, kuriame jis išbuvo iki mirties. 
\par 12 Devynioliktais Nebukadnecaro, Babilono karaliaus, metais, penkto mėnesio dešimtą dieną, sargybos viršininkas Nebuzaradanas, kuris tarnavo Babilono karaliui, atėjo į Jeruzalę. 
\par 13 Jis sudegino Viešpaties namus, karaliaus namus ir visus didelius Jeruzalės pastatus. 
\par 14 Chaldėjų kariai, buvę su sargybos viršininku, išgriovė aplinkui Jeruzalę esančias sienas. 
\par 15 Nebuzaradanas, sargybos viršininkas, išvedė į nelaisvę dalį tautos beturčių, mieste likusius gyventojus, tuos, kurie perbėgo pas babiloniečius, ir visus amatininkus. 
\par 16 Bet Nebuzaradanas paliko kai kuriuos krašto beturčius, kad prižiūrėtų vynuogynus ir dirbtų žemę. 
\par 17 Chaldėjai sulaužė varines kolonas, stovus ir varinį baseiną, buvusius Viešpaties namuose, ir visą jų varį išgabeno į Babiloną. 
\par 18 Jie paėmė ir puodus, šakutes, gnybtuvus, dubenis, semtuvus bei visus varinius indus, kurie buvo naudojami tarnavimo metu. 
\par 19 Sargybos viršininkas pasiėmė indus smilkalams, taures, praustuves, žvakides, lėkštes­visa, kas buvo iš aukso ir sidabro. 
\par 20 Dviejų kolonų, baseino, dvylikos varinių jaučių, buvusių po baseinu, ir stovų, kuriuos karalius Saliamonas buvo padaręs Viešpaties namams, vario buvo neįmanoma pasverti. 
\par 21 Viena kolona buvo aštuoniolikos uolekčių aukščio, dvylikos uolekčių apimties; kolonos sienos buvo keturių pirštų storio, o vidus tuščias. 
\par 22 Ant kolonos buvo varinis kapitelis, penkių uolekčių aukščio; jį supo grotelės ir granato vaisiai­viskas buvo iš vario. Tokia pat buvo ir antroji kolona. 
\par 23 Devyniasdešimt šeši granato vaisiai buvo iš vienos pusės; iš viso aplinkui groteles jų buvo šimtas. 
\par 24 Sargybos viršininkas paėmė vyriausiąjį kunigą Serają, antrąjį kunigą Sofoniją, tris durininkus, 
\par 25 miesto valdininką, kuris buvo karo vyrų viršininkas, septynis vyrus, karaliaus patarėjus, kuriuos rado mieste, kariuomenės vyriausiąjį raštininką, kuris šaukdavo į kariuomenę, ir šešiasdešimt krašto vyrų, kurie buvo mieste. 
\par 26 Nebuzaradanas, sargybos viršininkas, paėmęs juos, nuvedė pas Babilono karalių į Riblą. 
\par 27 Karalius nužudė juos Ribloje, Hamato krašte. Taip Judas buvo ištremtas iš savo krašto. 
\par 28 Nebukadnecaras septintaisiais savo karaliavimo metais iš Jeruzalės ištrėmė tris tūkstančius dvidešimt tris žydus; 
\par 29 aštuonioliktaisiais metais­aštuonis šimtus trisdešimt du asmenis; 
\par 30 dvidešimt trečiaisiais metais sargybos viršininkas Nebuzaradanas ištrėmė septynis šimtus keturiasdešimt penkis žydus. Iš viso keturis tūkstančius šešis šimtus žmonių. 
\par 31 Trisdešimt septintaisiais Judo karaliaus Jehojachino tremties metais, dvylikto mėnesio dvidešimt penktą dieną, Babilono karalius Evil Merodachas pirmaisiais savo karaliavimo metais dovanojo bausmę Judo karaliui Jehojachinui ir išleido jį iš kalėjimo. 
\par 32 Jis kalbėjo su juo draugiškai ir davė jam sostą, aukštesnį negu kitų karalių, kurie buvo su juo Babilone. 
\par 33 Jis pakeitė Jehojachino kalėjimo drabužius, ir tas valgė karaliaus akivaizdoje per visas savo gyvenimo dienas. 
\par 34 Karalius jam paskyrė nuolatinį išlaikymą, kurį jis gaudavo kiekvieną dieną per visas savo gyvenimo dienas.



\end{document}