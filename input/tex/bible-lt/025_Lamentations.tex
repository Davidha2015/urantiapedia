\begin{document}

\title{Raudų knyga}

\chapter{1}


\par 1 Koks vienišas miestas, kuris pirma buvo pilnas žmonių! Jis buvo garsus tarp tautų, o dabar lyg našlė. Buvo lyg kunigaikštienė, o dabar lyg vergė. 
\par 2 Ji verkia graudžiai nakties metu, skruostai pasruvę ašaromis. Nė vienas jos meilužis neguodžia jos. Visi draugai tapo išdavikais ir priešais. 
\par 3 Judas ištremtas vargsta ir vergauja, gyvena tarp pagonių tautų, neturi ramybės. Jo persekiotojai pasivijo jį siaurose vietose. 
\par 4 Siono keliai tušti, niekas jais nekeliauja į šventes. Miesto vartai sunaikinti, kunigai dūsauja. Jo mergaitės liūdi, jis apimtas kartėlio. 
\par 5 Prispaudėjai viešpatauja, priešai džiaugiasi. Viešpats siuntė tą bausmę Jeruzalei dėl daugybės jos kalčių. Jos vaikai priešų išvesti nelaisvėn. 
\par 6 Siono dukters grožis­praeitis. Jos kunigaikščiai lyg elniai, nerandantys ganyklos, bejėgiai jie eina savo priešų priekyje. 
\par 7 Jeruzalė, pavergta ir pamiršta, prisimena laimingus praeities laikus. Ji pateko į priešo rankas, niekas jai nepadėjo. Prispaudėjai tyčiojasi iš jos sabatų. 
\par 8 Jeruzalės nusikaltimas didelis, todėl ji atmesta. Kas ją gerbė, dabar niekina, nes matė jos nuogumą. Ji pati vaitoja ir sukasi į šalį. 
\par 9 Jos rūbas suteptas. Ji nepagalvojo, kas jos laukia, todėl ji visko neteko, nėra kas ją paguostų. Viešpatie, pažvelk į mano vargą, nes mano priešai iškilo! 
\par 10 Jos brangenybės priešų rankose. Net pagonių tautos, kurioms Tu buvai uždraudęs įeiti į ją, įsilaužė į šventyklą. 
\par 11 Visa tauta dūsauja, trūksta maisto. Jie keičia brangenybes į maistą gyvybei palaikyti. Viešpatie, pažvelk į mano paniekinimą! 
\par 12 Ar tai nieko nereiškia jums, einantiems pro šalį? Pažvelkite ir pagalvokite, ar kas kenčia tokį vargą kaip aš? Viešpats baudžia mane savo rūstybės įkarštyje. 
\par 13 Jis siuntė iš aukštybės ugnį ir degino mane. Jis ištiesė tinklą, pagavo mano kojas ir paklupdė mane. Jis padarė mane apleistą ir besikamuojančią. 
\par 14 Jis uždėjo man jungą už mano nusikaltimus. Jo ranka uždėtas jungas slegia mano pečius. Viešpats palaužė mano jėgas, atidavė mane į galingesnių rankas. 
\par 15 Viešpats sunaikino visus mano stipriuosius; Jis sušaukė daugybę, kad sunaikintų mano jaunuolius. Viešpats mynė mergelę, Judo dukterį, kaip vynuogių spaustuve. 
\par 16 Aš verkiu, mano skruostais rieda ašaros. Neturiu, kas mane nuramintų, kas atgaivintų mano sielą. Mano vaikus išsklaidė galingas priešas. 
\par 17 Sionas tiesia savo rankas, bet nėra, kas jį paguostų. Viešpats sukėlė prieš Jokūbą jo kaimynus, jie tapo jo priešais. Jeruzalė tapo jiems kaip moteris mėnesinių metu. 
\par 18 Viešpats yra teisus, nes aš neklausiau Jo įsakymų. Tautos, išgirskite! Pamatykite mano vargą! Mano jaunuoliai ir mergaitės išvesti į nelaisvę. 
\par 19 Aš ieškojau pagalbos tarp meilužių, bet jie apvylė mane. Kunigai ir vyresnieji mirė iš bado mieste, nerasdami maisto gyvybei palaikyti. 
\par 20 Viešpatie, pažvelk, kokia aš nelaiminga ir nerami. Mano širdis nerimsta krūtinėje, nes aš neklausiau Tavęs. Lauke siaučia kardas, o viduje­mirtis. 
\par 21 Jie išgirdo mano vaitojimą, bet nėra kas mane paguostų. Mano priešai, išgirdę, kad Tu mane baudi, džiaugiasi. Tavo keršto diena teateina ir jiems, kaip ji atėjo man. 
\par 22 Teiškyla ir jų nusikaltimai Tavo akivaizdoje. Atlygink jiems taip, kaip man atlyginai už mano kaltes. Mano dūsavimams nėra galo, mano širdis alpsta.


\chapter{2}


\par 1 Viešpats savo rūstybės debesimi aptemdė Siono dukterį. Jis nusviedė Izraelio didybę iš dangaus į dulkes. Rūstybės dieną Jis užmiršo savo pakojį. 
\par 2 Viešpats negailestingai sunaikino visas Jokūbo gyvenvietes, nugriovė Judo dukters tvirtoves, nusviedė į dulkes; karalystė ir kunigaikščiai neteko savo garbės. 
\par 3 Jis savo rūstybėje nulaužė Izraelio ragą, atitraukė nuo jo savo dešinę priešų akivaizdoje ir degė Jokūbe kaip viską ryjanti ugnis. 
\par 4 Jis savo dešine įtempė lanką kaip priešas ir sunaikino visą Siono dukters išdidumą. Jo įtūžis išsiliejo kaip ugnis. 
\par 5 Viešpats tapo priešu: prarijo Izraelį, prarijo jo rūmus, pavertė griuvėsiais tvirtoves ir padaugino Judo dukrai vargų ir kančių. 
\par 6 Savo palapinę Jis sugriovė kaip sodo pastogę, susirinkimų vietą sunaikino. Viešpats pašalino Sione švenčių ir sabatų iškilmes, paniekino savo įtūžyje karalius bei kunigus. 
\par 7 Viešpats atmetė savo aukurą, atsisakė šventyklos, o rūmus atidavė į priešų rankas. Jie šūkavo Viešpaties namuose kaip anksčiau iškilmių metu. 
\par 8 Viešpats nusprendė Siono dukters sienas sunaikinti. Jis, ištempęs matavimo virvę, ištiesė ranką galutiniam sunaikinimui. Įtvirtinimai ir sienos sunaikinti. 
\par 9 Vartai sulindo į žemę, užkaiščiai sulinko ir sulūžo, karaliai ir kunigaikščiai išsklaidyti pagonių tautose; įstatymo nebeliko, pranašai nebegauna regėjimų iš Viešpaties. 
\par 10 Siono dukters vyresnieji sėdi tylėdami ant žemės, galvas apsibarstę pelenais ir apsisiautę ašutinėmis. Jeruzalės mergaitės stovi, nuleidusios galvas. 
\par 11 Mano akys paraudo nuo ašarų, siela nerimsta, širdis plyšta iš skausmo dėl tautos sunaikinimo; kūdikiai ir vaikai alpsta miesto gatvėse. 
\par 12 Jie sako motinoms: “Kur grūdai ir vynas?”, kai alpsta, lyg būtų sunkiai sužeisti, ir miršta motinų glėbyje. 
\par 13 Jeruzale, kuo paguosiu ir kam prilyginsiu tavo kančias? Siono dukterie, kuo pastiprinsiu tave? Tavo žaizda yra didelė kaip jūra. Kas gali pagydyti tave? 
\par 14 Tavo pranašai pranašavo tuštybes ir kvailystes. Jie neatidengė tavo kalčių, kad apsaugotų nuo tremties. Jie matė melagingų regėjimų dėl tavęs ir apgaulę. 
\par 15 Dabar praeiviai ploja rankomis, švilpia ir kraipo galvas, žiūrėdami į Jeruzalę: “Ar taip atrodo miestas, kurį vadino grožio tobulybe ir visos žemės džiaugsmu?” 
\par 16 Tavo priešai atvėrė burnas prieš tave, švilpia, griežia dantimis. Jie sako: “Mes prarijome ją! Tai diena, kurios laukėme. Mes sulaukėme ir matome tai!” 
\par 17 Ką Viešpats nusprendė, tą įvykdė. Jo seniai paskelbti žodžiai išsipildė. Jis griovė ir nesigailėjo, o tavo priešams leido džiaugtis ir išaukštino jų ragus. 
\par 18 Jų širdys šaukėsi Viešpaties: “Siono dukters siena!” Tavo ašaros tegul teka srovėmis dieną ir naktį! Nesudėk akių ir nesiilsėk! 
\par 19 Maldauk vakare ir nakčia! Išliek savo širdį kaip vandenį Viešpaties akivaizdoje. Pakelk rankas į Jį dėl savo vaikų gyvybės, kurie alpsta iš bado gatvėse. 
\par 20 Viešpatie, pažvelk! Argi esi ką panašaus matęs? Nejaugi motinos turi valgyti savo vaisių, kūdikius, kuriuos glamonėjo? Argi kunigai ir pranašai turi būti žudomi Viešpaties šventykloje? 
\par 21 Gatvių dulkėse guli jauni ir seni. Jaunuoliai ir mergaitės krito nuo kardo. Pykčio metu juos nužudei, išžudei nesigailėdamas. 
\par 22 Tu pašaukei iš visų kampų lyg į šventas iškilmes mano priešus taip, kad Viešpaties įtūžio dieną nė vienas neištrūko ir nepabėgo. Kuriuos auginau ir auklėjau, sunaikino priešas.



\chapter{3}


\par 1 Aš­žmogus, patyręs vargą nuo Jo rūstybės lazdos. 
\par 2 Jis atvedė mane į tamsybę, o ne į šviesą. 
\par 3 Jis laiko ištiesęs savo ranką prieš mane visą dieną. 
\par 4 Jis pasendino mano kūną ir odą, sulaužė kaulus. 
\par 5 Jis apsupo mane kartybe ir vargu, 
\par 6 perkėlė į tamsą kaip mirusį. 
\par 7 Jis uždarė man duris ir apkalė mane sunkiomis grandinėmis. 
\par 8 Aš šaukiu ir meldžiuosi, bet Jis neatsako į mano maldą. 
\par 9 Jis užtvėrė mano kelius tašytais akmenimis ir mano takus iškraipė. 
\par 10 Jis tykojo manęs kaip lokys, kaip liūtas lindynėje. 
\par 11 Jis mane paklaidino, sudraskė ir paliko vienišą. 
\par 12 Įtempęs lanką, Jis pastatė mane taikiniu 
\par 13 ir pervėrė mano širdį strėlėmis. 
\par 14 Aš tapau pajuoka visai savo tautai, apie mane jie dainuoja per dieną. 
\par 15 Jis pasotino mane kartybėmis ir girdė metėlėmis. 
\par 16 Jis išlaužė mano dantis į žvyrą, užpylė mane pelenais. 
\par 17 Neturiu ramybės ir nežinau, kas yra gerovė. 
\par 18 Aš tariau: “Mano stiprybė ir viltis Viešpatyje pražuvo”. 
\par 19 Atsimink mano vargą, kartybę, metėlę ir tulžį. 
\par 20 Mano siela nuolat tai atsimena ir yra pažeminta manyje. 
\par 21 Nors aš viso to neužmirštu, visgi dar turiu vilties. 
\par 22 Viešpaties malonė nepranyko, Jo gailestingumas dar nepasibaigė. 
\par 23 Tai atsinaujina kas rytą, ir didelė yra Jo ištikimybė. 
\par 24 Viešpats yra mano dalis, todėl vilsiuosi Juo. 
\par 25 Viešpats yra geras Jo laukiantiems ir ieškantiems. 
\par 26 Gera yra turėti viltį ir kantriai laukti Viešpaties išgelbėjimo, 
\par 27 gera žmogui nešti jungą nuo pat jaunystės. 
\par 28 Jis sėdi atsiskyręs ir tyli, nes tai Viešpaties uždėta našta. 
\par 29 Jis paliečia dulkes savo burna: “Galbūt dar yra vilties”. 
\par 30 Jis atsuka skruostą jį mušančiam, sotinasi panieka, 
\par 31 nes Viešpats neatstumia amžiams. 
\par 32 Jis siunčia sielvartą, bet vėl pasigaili dėl savo malonės gausos. 
\par 33 Jis nenori varginti žmonių ir sukelti jiems sielvarto. 
\par 34 Kai mindžioja kojomis belaisvius, 
\par 35 kai Aukščiausiojo akivaizdoje pamina žmogaus teises, 
\par 36 kai iškraipo žmogaus bylą, Viešpats tam nepritaria. 
\par 37 Kas gali sakyti, kad įvyksta ir tai, ko Viešpats neįsako? 
\par 38 Ar ne iš Aukščiausiojo burnos ateina, kas gera ir kas pikta? 
\par 39 Kodėl žmogus skundžiasi, baudžiamas dėl savo nuodėmių? 
\par 40 Patikrinkime savo kelius ir grįžkime prie Viešpaties. 
\par 41 Kelkime savo širdis ir rankas į Dievą danguose. 
\par 42 Mes nusikaltome ir maištavome, ir Tu mums neatleidai. 
\par 43 Tu apsisiautei rūstybe ir persekiojai mus, Tu žudei mus nesigailėdamas. 
\par 44 Tu apsigaubei debesiu taip, kad maldos nepasiektų Tavęs. 
\par 45 Tu padarei mus sąšlavomis ir atmatomis tarp tautų. 
\par 46 Mūsų priešai atvėrė savo burnas prieš mus. 
\par 47 Baimė ir žabangai užgriuvo mus, griovimas ir sunaikinimas. 
\par 48 Mano akys pasruvo ašaromis dėl tautos sunaikinimo. 
\par 49 Mano ašaros plūs nesulaikomai, be perstojo, 
\par 50 kol Viešpats pažvelgs iš dangaus į mus. 
\par 51 Aš liūdžiu dėl savo miesto dukterų. 
\par 52 Priešai pagavo mane kaip paukštį be priežasties, 
\par 53 įmetė mane gyvą į duobę, mėtė akmenimis. 
\par 54 Vanduo pakilo iki mano galvos; maniau, esu žuvęs. 
\par 55 Viešpatie, iš duobės gilybės šaukiausi Tavęs. 
\par 56 Tu išgirdai mano balsą. Nenukreipk savo ausies nuo mano šauksmo. 
\par 57 Tu priartėjai, kai šaukiausi Tavo pagalbos, ir tarei: “Nebijok”. 
\par 58 Viešpatie, Tu atėjai man į pagalbą ir išgelbėjai mano gyvybę. 
\par 59 Viešpatie, Tu matei man daromą skriaudą, išspręsk mano bylą. 
\par 60 Tu matei jų įniršį ir visus jų sumanymus prieš mane; 
\par 61 Tu girdėjai jų patyčias ir visus jų sumanymus prieš mane. 
\par 62 Mano priešininkų lūpos visą laiką planuoja pikta prieš mane. 
\par 63 Ar jie sėdi, ar keliasi, aš esu jų daina. 
\par 64 Viešpatie, atlygink jiems pagal jų darbus. 
\par 65 Suteik jų širdims skausmo. Prakeikimas tekrinta ant jų. 
\par 66 Viešpatie, persekiok juos ir nušluok nuo žemės paviršiaus.



\chapter{4}


\par 1 Kaip grynas auksas patamsėjo, kaip jis pasikeitė! Šventyklos akmenys guli išmėtyti visur gatvėse. 
\par 2 Brangūs Siono sūnūs, prilygstantys auksui, dabar laikomi moliniais indais, puodžiaus rankų darbu! 
\par 3 Net jūrų pabaisos maitina ir žindo savo vaikus, o mano tautos duktė tapo žiauri lyg strutis dykumoje. 
\par 4 Nuo troškulio kūdikių liežuvis prilipo prie gomurio. Vaikai prašo duonos, bet niekas jiems neatlaužia jos. 
\par 5 Valgę skanumynus, dabar miršta gatvėse badu; išauginti purpuro drabužiuose, dabar guli dulkėse. 
\par 6 Bausmė už mano tautos kaltę yra didesnė už Sodomos bausmę, kuri buvo sunaikinta per akimirką be žmogaus rankos. 
\par 7 Jos nazarėnai buvo švaresni už sniegą, baltesni už pieną, jų kūnas rausvesnis už koralus, išvaizda gražesnė už safyrą. 
\par 8 Dabar jų veidai juodesni už anglį, jie nebeatpažįstami gatvėje, jų oda prilipusi prie kaulų, sudžiūvusi kaip medis. 
\par 9 Laimingesni kritę kovoje negu mirę badu, nes jų kūnas seko pamažu ir, netekę maisto, jie mirė. 
\par 10 Gailestingosios moterys virė ir valgė savo pačių vaikus; tai buvo jų maistas mano tautos sunaikinimo metu. 
\par 11 Viešpaties rūstybė pasireiškė, Jis išliejo savo įtūžį. Jis įžiebė Sione ugnį, kuri sunaikino jo pamatus. 
\par 12 Žemės karaliai ir pasaulio gyventojai netikėjo, kad priešas galėtų įžengti į Jeruzalę. 
\par 13 Tas įvyko dėl pranašų nuodėmių ir kunigų kalčių, kurie praliejo teisiųjų kraują miesto viduryje. 
\par 14 Jie vaikščiojo gatvėmis lyg akli, taip susitepę nekaltųjų krauju, kad buvo baisu juos paliesti. 
\par 15 Apie juos buvo sakoma: “Pasitraukite, jie nešvarūs, nepalieskite jų!” Jie pabėgo ir klajojo aplinkui, bet net pagonys sakė: “Jų neturi būti tarp mūsų!” 
\par 16 Pats Viešpats juos išsklaidė ir nekreipė dėmesio į juos; nebuvo pagarbos nei kunigams, nei vyresniesiems. 
\par 17 Mes pavargome, belaukdami pagalbos, bet jos nesulaukėme. Laukėme pagalbos iš tautos, kuri negalėjo mums padėti. 
\par 18 Mūsų žingsniai buvo sekami, negalėjome net gatvėje pasirodyti. Mūsų galas artėjo, dienos baigėsi. 
\par 19 Persekiotojai buvo greitesni už padangių erelius. Jie gaudė mus kalnuose, tykojo dykumose. 
\par 20 Viešpaties pateptąjį jie sugavo, o mes tikėjome, kad jo ūksmėje gyvensime tarp tautų. 
\par 21 Edomo dukra, gyvenanti Uco krašte, džiaukis ir būk linksma! Ir tave pasieks keršto taurė, tu taip pat būsi nugirdyta ir apsinuoginsi. 
\par 22 Siono dukra, tavo bausmė baigta! Jis tavęs nebeištrems. Jis aplankys tavo kaltę, Edomo dukra, ir iškels tavo nuodėmes.



\chapter{5}


\par 1 Viešpatie, atsimink, kas įvyko. Pažvelk, atkreipk dėmesį į mūsų vargus. 
\par 2 Mūsų paveldas ir namai svetimųjų rankose. 
\par 3 Mes esame našlaičiai, mūsų motinos­našlės; 
\par 4 privalome pirkti savo vandenį ir mokėti pinigus už savo malkas. 
\par 5 Mus vargina sunkiais darbais ir pavargus neleidžia atsikvėpti. 
\par 6 Prašėme pagalbos egiptiečių ir asirų, kad bent duonos gautume. 
\par 7 Mūsų tėvai nusikalto, o mes turime nešti jų kaltę. 
\par 8 Vergai viešpatauja mums, ir niekas negali mūsų išvaduoti iš jų rankos. 
\par 9 Bijodami kardo dykumoje, mes parsigabenome duonos. 
\par 10 Mūsų oda pajuodusi kaip krosnis nuo siaučiančio bado. 
\par 11 Moterys ir mergaitės prievartaujamos Sione ir Judo miestuose. 
\par 12 Kunigaikščius jie pakorė, o vyresniųjų negerbia. 
\par 13 Jauni vyrai verčiami girnomis malti, vaikai klumpa po sunkiomis naštomis. 
\par 14 Vyresnieji nebesirodo prie miesto vartų, ir vaikai nebesusirenka žaisti. 
\par 15 Mūsų širdies džiaugsmas dingo, žaidimai virto liūdesiu. 
\par 16 Karūna nuo galvos nukrito. Vargas mums, nes mes nusikaltome. 
\par 17 Todėl mūsų širdis alpsta, akys aptemo. 
\par 18 Siono kalnas apleistas, lapės gyvena jame. 
\par 19 Bet Tu, Viešpatie, pasilieki per amžius. Tavo sostas lieka kartų kartoms. 
\par 20 Kodėl Tu mus taip ilgai užmiršai ir palikai? 
\par 21 Viešpatie, sugrąžink mus pas save, ir mes sugrįšime. Atnaujink mus kaip anomis dienomis! 
\par 22 Argi Tu mus visiškai atstūmei ir rūstausi amžinai?




\end{document}