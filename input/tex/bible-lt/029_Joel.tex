\begin{document}

\title{Joelio knyga}

\chapter{1}


\par 1 Viešpaties žodis buvo suteiktas Joeliui, Petuelio sūnui. 
\par 2 Pasiklausykite, vyresnieji, išgirskite savo ausimis, visi krašto gyventojai! Ar taip buvo jūsų ar jūsų tėvų dienomis? 
\par 3 Pasakykite tai savo vaikams, jie tepapasakoja savo vaikams, o anie­kitai kartai. 
\par 4 Ką paliko vikšrai, nuėdė skėriai; kas paliko nuo skėrių, suėdė vabalai; kas liko nuo vabalų, sunaikino amaras. 
\par 5 Girtuokliai, pabuskite. Verkite ir dejuokite visi, kurie geriate vyną, nes jo nebebus jums. 
\par 6 Tauta užpuolė mano kraštą, galinga ir gausi tauta. Jos dantys yra kaip liūto, krūminiai kaip galingo liūto. 
\par 7 Ji nusiaubė mano vynuogyną, sunaikino figmedžius, nuėdė jų lapus bei nugraužė žievę; jų šakos liko baltos. 
\par 8 Raudok apsisiautusi ašutine kaip mergaitė dėl savo jaunystės vyro. 
\par 9 Duonos aukos ir geriamosios aukos dingo iš Viešpaties namų, gedi kunigai, Viešpaties tarnai. 
\par 10 Laukai sunaikinti, dirvos liūdi. Sunaikinti javai, vyno nebėra, aliejus išseko. 
\par 11 Vynininkai, susigėskite, žemdirbiai, dejuokite netekę kviečių ir miežių, nes laukų derlius žuvo. 
\par 12 Vynmedis nudžiūvo, figmedis nuvyto. Granatmedis, palmė, obelis ir visi laukų medžiai nudžiūvo. Todėl ir džiaugsmas dingo tarp žmonių. 
\par 13 Kunigai, apsisiauskite ašutine ir liūdėkite! Aukuro tarnai, verkite ir raudokite! Mano Dievo tarnai, apsisiautę ašutine, gulėkite per naktį ir raudokite, nes nebėra duonos aukų ir geriamųjų aukų jūsų Dievo namuose. 
\par 14 Paskelbkite pasninką, sušaukite iškilmingą susirinkimą, sukvieskite vyresniuosius ir visus krašto gyventojus į Viešpaties, jūsų Dievo, namus ir šaukitės Viešpaties. 
\par 15 Viešpaties diena arti! Kaip sunaikinimas ji ateis nuo Visagalio. 
\par 16 Maistas atimtas, mums matant, džiaugsmas bei džiūgavimas išnyko iš Dievo namų. 
\par 17 Pasėti grūdai žemėje sudžiūvo, sandėliai tušti, klojimai nenaudojami, nes javų nėra. 
\par 18 Vargsta gyvuliai! Nerimsta galvijų bandos, nes nebėra joms ganyklų, kenčia ir avys. 
\par 19 Tavęs, Viešpatie, šauksiuosi, nes ugnis sunaikino pievas ir ganyklas, liepsna nusvilino visus laukų medžius. 
\par 20 Net laukiniai žvėrys šaukiasi Tavęs, nes išdžiūvo visi upeliai ir ugnis nuėdė pievas bei ganyklas.


\chapter{2}


\par 1 Trimituokite Sione, skelbkite pavojų mano šventajame kalne. Tedreba krašto gyventojai, nes ateina Viešpaties diena, nes ji arti. 
\par 2 Tamsi, niūri ir debesuota diena. Kaip ryto migla apgaubia kalnų viršūnes, taip galinga ir didelė tauta ateina. Tokios nebuvo iki šiol ir nebebus per kartų kartas. 
\par 3 Jų priekyje­visa ryjanti ugnis, užpakalyje­siaučianti liepsna. Kraštas prieš juos kaip Edeno sodas, jiems praėjus­tuščia dykuma; niekas neišsigelbės nuo jų. 
\par 4 Jie atrodo kaip žirgai ir puls kaip kariuomenės raiteliai. 
\par 5 Jų garsas lyg kovos vežimų, riedančių kalnų viršūnėmis, kaip ūžimas liepsnos, ryjančios ražienas, kaip kariuomenės, kuri išsirikiavusi kovai. 
\par 6 Prieš juos drebės tautos, visų veidai pabals. 
\par 7 Jie bėgs kaip karžygiai, lips sienomis kaip kariai; kiekvienas eis savo keliu, nesuardydamas gretų. 
\par 8 Jie nesistumdys, bet eis kiekvienas savo taku. Jie kris ant kardo, bet nesusižeis. 
\par 9 Jie lakstys po miestą, bėgs sienomis, lips į namus, lįs pro langus kaip vagys. 
\par 10 Prieš juos drebės žemė ir svyruos dangus, saulė ir mėnulis aptems, žvaigždės nebespindės. 
\par 11 Viešpats pakels balsą priešais savo kariuomenę. Jo pulkai yra labai gausūs, ir stiprus tas, kuris vykdo Jo žodį. Viešpaties diena yra didi ir labai baisi, kas galės ją iškęsti? 
\par 12 “Todėl dabar,­sako Viešpats,­ gręžkitės į mane visa širdimi, pasninkaudami, verkdami ir dejuodami”. 
\par 13 Persiplėškite savo širdis, ne rūbus! Gręžkitės į Viešpatį, savo Dievą, nes Jis yra maloningas ir gailestingas, lėtas pykti ir didžiai geras bei susilaikantis nuo bausmės. 
\par 14 Kas žino, gal Jis sugrįš, pasigailės ir paliks palaiminimą­duonos auką ir geriamąją auką Viešpačiui, jūsų Dievui? 
\par 15 Trimituokite Sione, paskelbkite pasninką, sušaukite iškilmingą susirinkimą, 
\par 16 sukvieskite tautą ir vyresniuosius, atgabenkite vaikus ir žindomus kūdikius. Jaunavedžiai teišeina iš savo kambarių. 
\par 17 Tarp prieangio ir aukuro tegul rauda kunigai, Viešpaties tarnai, sakydami: “Viešpatie, pasigailėk savo tautos, neatiduok savo paveldo pagonių paniekai ir pajuokai! Kodėl tautos turėtų sakyti: ‘Kur yra jų Dievas’?” 
\par 18 Tada Viešpatį apims pavydas dėl savo krašto ir Jis pasigailės savo tautos. 
\par 19 Viešpats atsakys ir tars savo tautai: “Štai Aš duosiu jums javų, vyno ir aliejaus, ir jūs būsite sotūs; Aš nebeleisiu tautoms jūsų niekinti. 
\par 20 Atėjusį iš šiaurės pašalinsiu toli nuo jūsų ir nusviesiu jį į tuščią dykumą, jo veidas bus į Rytų jūrą ir užpakalinė dalis­į Vakarų jūrą. Nuo jo smarvė pakils ir bjaurus kvapas, nes jis baisių dalykų padarė”. 
\par 21 Nebijok, žeme, džiaukis ir džiūgauk, nes Viešpats padarys didelių dalykų. 
\par 22 Laukiniai žvėrys, nebijokite! Ganyklos vėl žaliuoja, medžiai neša vaisių, figmedis ir vynmedis duoda derlių. 
\par 23 Siono vaikai, džiaukitės Viešpačiu, savo Dievu. Jis ištikimai duodavo jums ankstyvą lietų ir Jis siųs jums lietų, ankstyvąjį ir vėlyvąjį lietų, kaip ir anksčiau. 
\par 24 Klojimai bus pilni javų ir statinės sklidinos vyno ir aliejaus. 
\par 25 “Aš atlyginsiu jums už metus, kuriuos sunaikino vikšrai, skėriai, vabalai ir amaras­mano didžioji kariuomenė, kurią siunčiau prieš jus. 
\par 26 Jūs gausiai valgysite ir būsite sotūs, ir girsite Viešpaties, savo Dievo, vardą, kuris padarė jums nuostabių dalykų. Mano tauta nebus sugėdinta per amžius. 
\par 27 Jūs žinosite, kad Aš esu Izraelyje, Aš, Viešpats, jūsų Dievas, ir nėra kito. Mano tauta nebus sugėdinta per amžius. 
\par 28 Po to Aš išliesiu savo dvasios ant kiekvieno kūno. Jūsų sūnūs ir dukterys pranašaus, seniai sapnuos sapnus ir jaunuoliai matys regėjimus. 
\par 29 Taip pat ant vergų ir vergių tomis dienomis išliesiu savo dvasios. 
\par 30 Aš danguje ir žemėje parodysiu stebuklų­kraujo, ugnies bei rūkstančių dūmų. 
\par 31 Saulė pavirs tamsa ir mėnulis­ krauju, prieš ateinant didingai ir baisiai Viešpaties dienai. 
\par 32 Tada kiekvienas, kuris šauksis Viešpaties vardo, bus išgelbėtas. Siono kalne ir Jeruzalėje bus išgelbėjimas, kaip Viešpats sakė, likučiui, kurį Jis pašauks”.



\chapter{3}


\par 1 “Tuo metu, kai parvesiu Judo ir Jeruzalės ištremtuosius, 
\par 2 Aš surinksiu visas tautas ir atvesiu jas į Juozapato slėnį. Ten juos teisiu dėl mano tautos ir mano paveldo Izraelio, kurį jos buvo išsklaidžiusios ir pasidalinusios mano žemę. 
\par 3 Jie metė burtus dėl mano tautos, už paleistuvę atidavė berniuką, už vyną pardavė mergaitę ir girtuokliavo. 
\par 4 Ką jūs turite bendro su manimi, Tyre, Sidone ir visi filistinų kraštai? Ar jūs norite man atlyginti už mano darbus ir man atkeršyti? Aš greitai ir lengvai sugrąžinsiu jūsų kerštą ant jūsų pačių galvų. 
\par 5 Jūs paėmėte mano sidabrą bei auksą ir mano brangenybes nugabenote į savo šventyklas. 
\par 6 Jūs pardavėte Judo ir Jeruzalės gyventojus graikams, išvarėte juos toli iš jų krašto. 
\par 7 Aš pakelsiu juos toje vietoje, kurion jūs juos pardavėte, ir sugrąžinsiu jūsų kerštą ant jūsų pačių galvų. 
\par 8 Aš atiduosiu jūsų sūnus ir dukteris Judo vaikams, o jie parduos juos šebiečiams, toli gyvenančiai tautai, nes taip pasakė Viešpats”. 
\par 9 Paskelbkite tai tautoms, pasiruoškite karui, žadinkite karžygius! Kariai tegul ateina artyn. 
\par 10 Perkaldinkite savo žagres į kardus ir pjautuvus į ietis. Silpnasis tesako: “Aš stiprus”. 
\par 11 Skubiai susirinkusios, ateikite, aplinkinės tautos! Viešpatie, ten atvesk savo karžygius! 
\par 12 “Tepakyla ir tesusirenka tautos į Juozapato slėnį: ten Aš teisiu jas. 
\par 13 Imkite pjautuvą, nes atėjo pjūties metas. Spaustuvas pilnas, ateikite ir minkite. Indai bus sklidini, nes jų nedorybė didelė”. 
\par 14 Minių minios sprendimo slėnyje! Viešpaties diena priartėjo. 
\par 15 Saulė ir mėnulis aptems, žvaigždės neteks spindesio. 
\par 16 Viešpats suriaumos iš Siono, Jo balsas pasigirs iš Jeruzalės, dangus ir žemė sudrebės. Viešpats yra savo tautos viltis ir Izraelio vaikų stiprybė. 
\par 17 “Tada jūs žinosite, kad Aš esu Viešpats, jūsų Dievas, kuris gyvena Sione, savo šventame kalne. Jeruzalė bus šventa, svetimieji nebemindžios jos. 
\par 18 Tą dieną iš kalnų lašės saldus vynas, iš kalvų tekės pienas. Judo upės bus pilnos vandens, iš Viešpaties namų tekės šaltinis ir drėkins Šitimo slėnį. 
\par 19 Egiptas taps dykyne, Edomas virs tuščia dykuma, nes jie žiauriai praliejo nekaltą Judo vaikų kraują. 
\par 20 Judas gyvens per amžius ir Jeruzalė per kartų kartas. 
\par 21 Aš apvalysiu jų kraują, kurio dar neapvaliau, nes Viešpats gyvena Sione!”



\end{document}