\begin{document}

\title{Jonos knyga}

\chapter{1}


\par 1 Viešpats tarė Amitajo sūnui Jonai: 
\par 2 “Eik į Ninevę, didį miestą, ir šauk prieš jį, nes jų nedorybės pasiekė mane”. 
\par 3 Jona pakilo, kad bėgtų nuo Viešpaties į Taršišą. Jis pasiekė Jopę, kur rado laivą, plaukiantį į Taršišą, sumokėjo už kelionę ir, įsėdęs į jį, su kitais plaukė į Taršišą nuo Viešpaties akivaizdos. 
\par 4 Viešpats pasiuntė smarkų vėją. Kilo didelė audra jūroje, ir laivui grėsė pavojus sudužti. 
\par 5 Jūrininkai išsigando, ir kiekvienas šaukėsi savo dievo. Jie išmetė laive esantį krovinį į jūrą, kad laivas palengvėtų. Tuo metu Jona buvo nusileidęs į laivo vidų ir kietai miegojo. 
\par 6 Kapitonas atėjo pas jį ir klausė: “Kaip tu gali miegoti? Kelkis, šaukis savo Dievo! Gal Jis prisimins mus ir mes nežūsime?” 
\par 7 Po to jie kalbėjosi: “Eikime, meskime burtą, kuris iš mūsų kaltas dėl šitos nelaimės”. Jie metė burtą, ir burtas krito Jonai. 
\par 8 Tada jie klausė jį: “Pasakyk, dėl ko mums šita nelaimė? Kuo tu užsiimi? Iš kur keliauji? Iš kokio krašto ir iš kurios tautos esi?” 
\par 9 Jis jiems atsakė: “Aš esu hebrajas, garbinu Viešpatį, dangaus Dievą, kuris sukūrė jūrą ir sausumą”. 
\par 10 Tie vyrai labai išsigando ir klausė jo: “Kodėl taip padarei?” Jie žinojo, kad jis bėgo nuo Viešpaties, nes jis jiems tai buvo papasakojęs. 
\par 11 Tada jie sakė jam: “Ką turime daryti su tavimi, kad jūra mums nurimtų?” Nes jūra siautė vis smarkiau. 
\par 12 Jis jiems atsakė: “Imkite mane ir meskite į jūrą! Tada jūra jums nurims; nes aš žinau, kad dėl manęs kilo ši baisi audra”. 
\par 13 Vyrai yrėsi visomis jėgomis, kad grįžtų prie kranto, bet neįstengė, nes jūra nesiliovė siautusi. 
\par 14 Jie šaukėsi Viešpaties: “Maldaujame Tave, Viešpatie, neleisk mums žūti dėl šito žmogaus, nepriskaityk mums nekalto kraujo, nes Tu, Viešpatie, darai, ką nori”. 
\par 15 Tada jie paėmė Joną ir išmetė jį į jūrą. Jūra nurimo. 
\par 16 Tie vyrai labai išsigando Viešpaties, aukojo Viešpačiui aukas ir davė įžadus. 
\par 17 Viešpats paruošė didelę žuvį Jonai praryti. Jis išbuvo žuvies pilve tris dienas ir tris naktis.


\chapter{2}


\par 1 Tada Jona iš žuvies pilvo meldėsi Viešpačiui, savo Dievui: 
\par 2 “Aš šaukiausi Viešpaties savo varguose, ir Jis išklausė mane. Kai iš mirusiųjų buveinės pilvo šaukiau, Tu išgirdai mano balsą. 
\par 3 Tu įmetei mane į gelmę, į jūrų širdį. Vandenys apsupo mane, visos Tavo bangos ir vilnys ritosi per mane. 
\par 4 Tada galvojau: ‘Esu atstumtas nuo Tavęs, nebepamatysiu Tavo šventyklos!’ 
\par 5 Vandenys apsupo mane ir grėsė mano gyvybei, gelmės apgaubė mane, jūros žolės vyniojosi apie mano galvą. 
\par 6 Aš nugrimzdau į jūros gelmes, žemė uždarė mane savo skląsčiais. Bet Tu, Viešpatie, mano Dieve, išvedei mane gyvą iš gelmės. 
\par 7 Mano sielai alpstant, prisiminiau Viešpatį ir mano malda pasiekė Tave Tavo šventykloje. 
\par 8 Apgaulingų tuštybių garbintojai apleidžia savo Gailestingąjį. 
\par 9 Bet aš, garsiai dėkodamas, atnešiu Tau auką; ką pažadėjau­ištesėsiu. Išgelbėjimas ateina iš Viešpaties!” 
\par 10 Viešpats įsakė, ir žuvis išspjovė Joną ant kranto.



\chapter{3}


\par 1 Viešpats tarė Jonai antrą kartą: 
\par 2 “Eik į Ninevę, didį miestą, ir skelbk jiems, ką tau liepiau”. 
\par 3 Jona pakluso Viešpaties žodžiui ir ėjo į Ninevę, kuri buvo labai didelis miestas­reikėjo trijų dienų jį pereiti. 
\par 4 Jona ėjo per miestą vieną dieną. Jis skelbė: “Dar keturiasdešimt dienų, ir Ninevė bus sunaikinta”. 
\par 5 Ninevės žmonės patikėjo Dievu ir paskelbė pasninką ir nuo mažiausio iki didžiausio apsisiautė ašutinėmis. 
\par 6 Kai ta žinia pasiekė Ninevės karalių, jis pakilo iš savo sosto, nusivilko savo drabužius, apsisiautė ašutine ir atsisėdo pelenuose. 
\par 7 Ir jis išleido Ninevėje tokį įsakymą: “Karalius ir jo didžiūnai skelbia: žmonės ir gyvuliai, galvijai ir avys privalo pasninkauti. Teneragauja jie nei maisto, nei vandens. 
\par 8 Žmonės ir gyvuliai turi apsidengti ašutinėmis ir garsiai šauktis Dievo, atsisakyti piktų kelių ir smurto. 
\par 9 Kas žino, gal Dievas pasigailės mūsų, atsileis Jo rūstybė ir mes nepražūsime?” 
\par 10 Dievas, pamatęs, kad jie atsisakė savo piktų kelių, gailėjosi jų ir neįvykdė to, ką buvo jiems sakęs.



\chapter{4}


\par 1 Jonai tai labai nepatiko, ir jis supyko. 
\par 2 Jona meldėsi: “Ak, Viešpatie, argi aš taip nesakiau, kai dar buvau savo krašte? Dėl to aš ir norėjau bėgti į Taršišą, žinodamas, kad esi maloningas Dievas, gailestingas, lėtas pykti, didžiai geras ir susilaikantis nuo bausmės. 
\par 3 Todėl, Viešpatie, meldžiu, paimk mano gyvybę, nes man yra geriau mirti negu gyventi”. 
\par 4 Viešpats klausė: “Ar manai, kad teisingai pyksti?” 
\par 5 Jona išėjo iš miesto ir atsisėdo rytų pusėje. Jis, pasistatęs ten stoginę, sėdėjo jos pavėsyje ir laukė, kas atsitiks miestui. 
\par 6 Viešpats Dievas išaugino augalą, kad jo šešėlis dengtų Jonos galvą ir išvaduotų iš sielvarto. Jona labai džiaugėsi augalu. 
\par 7 Bet kitą dieną, aušrai brėkštant, Dievas paruošė kirmėlę, kuri pakando augalą, ir tas nudžiūvo. 
\par 8 Saulei patekėjus, Dievas paruošė svilinantį rytų vėją. Saulė spigino Jonos galvą, jis alpo ir geidė sau mirties, sakydamas: “Geriau man mirti negu gyventi!” 
\par 9 Ir Dievas tarė Jonai: “Ar tu teisingai pyksti dėl augalo?” Jis atsakė: “Taip, teisingai pykstu, netgi iki mirties”. 
\par 10 Tada Viešpats tarė: “Tau gaila augalo, dėl kurio tu nei vargai, nei jo auginai. Jis per vieną naktį užaugo ir per vieną naktį pražuvo. 
\par 11 Argi Aš turėčiau nesigailėti Ninevės, šio didelio miesto, kuriame gyvena daugiau negu šimtas dvidešimt tūkstančių žmonių, kurie nemoka atskirti dešinės nuo kairės, ir, be to, daug gyvulių?”




\end{document}