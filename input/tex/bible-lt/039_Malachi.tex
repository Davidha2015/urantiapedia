\begin{document}

\title{Malachijo knyga}

\chapter{1}


\par 1 Viešpats kalbėjo Izraeliui per Malachiją. 
\par 2 “Aš pamilau jus,­sako Viešpats.­O jūs sakote: ‘Kaip gi Tu mus pamilai?’ Argi Ezavas nebuvo Jokūbo brolis?­sako Viešpats.­ Bet Aš pamilau Jokūbą, 
\par 3 o Ezavo nekenčiau. Aš paverčiau jo kalnus dykyne, o jo paveldėtą žemę atidaviau dykumų šakalams. 
\par 4 Kadangi Edomas sako: ‘Mus sunaikino, bet mes atstatysime miestus’, tai kareivijų Viešpats sako: ‘Jie testato, o Aš juos vėl griausiu. Juos vadins nedorybės kraštu, tauta, kurios Viešpats neapkenčia. 
\par 5 Jūsų akys tai matys, ir jūs sakysite: ‘Galingas yra Viešpats net už Izraelio sienų!’ ” 
\par 6 “Sūnus gerbia tėvą, o tarnas­ savo šeimininką. Jei Aš tėvas, kur derama man pagarba? O jei Aš šeimininkas, kur mano baimė?­ sako kareivijų Viešpats jums, kunigai, kurie niekinate mano vardą.­Jūs klausiate: ‘Kaip mes niekiname Tavo vardą?’ 
\par 7 Aukodami ant mano aukuro suteptą maistą! Jūs klausiate: ‘Kaip mes sutepame jį?’ Negerbdami Viešpaties stalo. 
\par 8 Jei aukojate aklą, raišą ar ligotą gyvulį, ar tai nėra blogai? Įteik tokį savo valdovui! Ar jam patiks, ar jis maloniai tave priims?”­sako kareivijų Viešpats. 
\par 9 “Taip jūs norite permaldauti Dievą, kad Jis būtų jums malonus! Jei tokią dovaną duodate, ar Jis priims jus?”­sako kareivijų Viešpats. 
\par 10 “Verčiau uždarykite duris ir nekurkite mano aukuro veltui. Aš nemėgstu jūsų ir nepriimsiu aukos iš jūsų rankos. 
\par 11 Nuo saulės užtekėjimo iki nusileidimo didis yra mano vardas tautose, ir visur smilkoma bei aukojama tyra auka mano vardui”,­sako kareivijų Viešpats. 
\par 12 “Bet jūs sutepate jas, sakydami: ‘Viešpaties stalas yra suterštas ir ant jo padėta auka neverta pagarbos’. 
\par 13 Ir kai sakote: ‘Koks vargas!’, tuo jūs supykinate mane,­sako kareivijų Viešpats.­Jei nešate luošą ir ligotą auką, ar Aš turiu tai priimti iš jūsų?”­sako Viešpats. 
\par 14 “Prakeiktas apgavikas, kuris turi bandoje sveiką patiną ir jį pažada, tačiau aukoja Viešpačiui sužalotą! Aš esu didis Karalius,­sako kareivijų Viešpats,­ir mano vardo bijosi tautos”.


\chapter{2}


\par 1 Kunigai, jums skirtas šitas įspėjimas. 
\par 2 “Jei neklausysite, nesidėsite į širdį ir nešlovinsite mano vardo, Aš siųsiu jums prakeikimą. Aš jau prakeikiau jus, nes jūs nesidedate to į širdį. 
\par 3 Aš sudrausiu jūsų palikuonis, drėbsiu jūsų aukų mėšlą jums į veidą ir jus pačius pašalinsiu. 
\par 4 Tada žinosite, jog Aš daviau jums šį įspėjimą, laikydamasis sandoros su Levio palikuonimis. 
\par 5 Mano sandora su juo buvo gyvenimas ir ramybė, kad manęs bijotų ir gerbtų mano vardą. 
\par 6 Tiesos įstatymas buvo jo burnoje, netiesos nebuvo jo lūpose. Jis su manimi vaikščiojo ramybėje ir tiesoje ir daugelį nukreipė nuo nuodėmės. 
\par 7 Kunigo lūpos turėtų teikti žinių ir iš jo burnos bus ieškoma įstatymo, nes jis yra kareivijų Viešpaties pasiuntinys. 
\par 8 Bet jūs palikote tą kelią, suklaidinote daugelį savo mokymu, sulaužėte Levio sandorą,­sako kareivijų Viešpats.­ 
\par 9 Todėl ir Aš jus paniekinsiu ir menkaverčius padarysiu visos tautos akivaizdoje, nes nesilaikėte mano kelių ir buvote šališki įstatymo reikaluose”. 
\par 10 Argi visi nesame vieno tėvo vaikai? Argi ne tas pats Dievas mus sutvėrė? Kodėl klastingai elgiamės vienas su kitu ir laužome savo tėvų sandorą? 
\par 11 Neištikimai ir bjauriai elgėsi Judas, Izraelis ir Jeruzalė. Jie išniekino Viešpaties šventyklą, kurią Jis mylėjo, vesdami svetimų dievų dukteris. 
\par 12 Viešpats sunaikins tuos žmones iš Jokūbo palapinių, kurie taip daro, nors jie aukoja kareivijų Viešpačiui! 
\par 13 Viešpaties aukurą jūs užliejote ašaromis, verksmais ir dejavimais, nes Jis nebežiūri į jūsų aukas ir nebepriima jų iš jūsų rankų. 
\par 14 Jūs klausiate: “Kodėl?” Todėl, kad Viešpats yra liudytojas tarp tavęs ir tavo jaunystės žmonos, kuriai tu buvai neištikimas, nors ji yra tavo draugė ir sandoros žmona. 
\par 15 Argi jūs nebuvote padaryti vienas kūnas ir viena dvasia? Kodėl vienas? Jis nori matyti gerą sėklą. Todėl saugokite savo dvasias ir būkite ištikimi savo žmonoms. 
\par 16 “Aš nekenčiu skyrybų, nekenčiu smurtu sutepto drabužio”,­sako kareivijų Viešpats, Izraelio Dievas. 
\par 17 Jūs nuvarginote Viešpatį savo žodžiais. Jūs sakote: “Kaip mes Jį nuvarginome?” Sakydami: “Piktadarys yra geras Viešpaties akyse, tokie Jam patinka”, arba: “Kur yra teisingasis Dievas?”



\chapter{3}


\par 1 “Aš siunčiu savo pasiuntinį paruošti kelią pirma manęs. Viešpats, kurio jūs laukiate, ateis netikėtai į savo šventyklą. Štai Jis ateina, sandoros pasiuntinys, kuris jums patinka”,­sako kareivijų Viešpats. 
\par 2 Kas išsilaikys tą dieną ir kas ištvers, Jam pasirodžius? Jis yra kaip lydytojo ugnis ir kaip skalbėjų šarmas. 
\par 3 Jis ateis kaip sidabro ir aukso lydytojas ir apvalys Levio palikuonis, išgrynins juos kaip auksą ir sidabrą, kad aukotų Viešpačiui teisumo auką. 
\par 4 Tada Viešpačiui patiks Judo ir Jeruzalės aukos kaip praeityje. 
\par 5 “Aš ateisiu jūsų teisti ir nedelsdamas pasmerksiu žynius, svetimautojus, melagingai prisiekiančius, nusukančius samdinių algą, skriaudžiančius našles bei našlaičius, laužančius ateivių teises ir manęs nebijančius. 
\par 6 Aš esu Viešpats, Aš nesikeičiu, todėl jūs, Jokūbo vaikai, nebuvote sunaikinti. 
\par 7 Jūs jau nuo savo tėvų dienų palikote mano nuostatus ir jų nesilaikėte. Sugrįžkite pas mane, tai Aš sugrįšiu pas jus. Jūs sakote: ‘Kaip mums sugrįžti?’ 
\par 8 Ar gali žmogus apiplėšti Dievą? Tačiau jūs apiplėšėte mane. Jūs sakote: ‘Kaip mes apiplėšėme Tave?’ Dešimtinėmis ir aukomis! 
\par 9 Prakeikimas krinta ant jūsų, kadangi jūs, visa tauta, apiplėšėte mane. 
\par 10 Atneškite visas dešimtines į sandėlius, kad būtų maisto mano namuose, ir tuo išmėginkite mane,­sako kareivijų Viešpats.­Ar Aš neatversiu jums dangaus langų ir neišliesiu jums apsčiai palaiminimų? 
\par 11 Ar nesudrausiu kenkėjo, kad jis nesunaikintų jūsų laukų derliaus ir vynmedis neliktų be vaisių?­ sako kareivijų Viešpats.­ 
\par 12 Tada visos tautos vadins jus palaimintais, nes būsite graži šalis,­ sako kareivijų Viešpats.­ 
\par 13 Įžūlios yra jūsų kalbos prieš mane, tačiau jūs sakote: ‘Ką gi sakėme prieš Tave?’ 
\par 14 Jūs sakėte: ‘Jokios naudos iš tarnavimo Dievui. Kas iš to, kad laikėmės Jo nuostatų ir vaikščiojome nuleidę galvas Viešpaties akivaizdoje? 
\par 15 Dabar mes vadiname laimingais išdidžiuosius. Jiems sekasi, nors jie elgiasi nedorai. Jie gundo Dievą, bet nebaudžiami’ ”. 
\par 16 Dievo bijantys kalbėjosi vienas su kitu. Viešpats stebėjo ir klausė. Buvo parašyta Jo akivaizdoje atminimo knyga apie tuos, kurie bijo Viešpaties ir gerbia Jo vardą. 
\par 17 “Jie bus mano,­sako kareivijų Viešpats.­Tą dieną Aš padarysiu juos savo brangenybėmis. Aš jų gailėsiuos, kaip gailisi žmogus savo sūnaus, kuris jam tarnauja. 
\par 18 Tada jūs matysite skirtumą tarp teisiojo ir nedorėlio, tarp to, kuris tarnauja Dievui, ir to, kuris Jam netarnauja”.



\chapter{4}


\par 1 “Štai ateina diena, deganti kaip krosnis, kurioje visi išdidieji ir visi nedorėliai sudegs kaip ražienos,­sako kareivijų Viešpats.­Jiems neliks nei šaknies, nei šakos. 
\par 2 Bet jums, bijantiems mano vardo, užtekės teisumo saulė su išgydymu po jos sparnais. Jūs išeisite ir šokinėsite kaip išleisti iš tvarto veršiukai. 
\par 3 Jūs mindžiosite nedorėlius, ir jie bus pelenai po jūsų kojomis tą dieną, kai Aš tai padarysiu,­sako kareivijų Viešpats.­ 
\par 4 Atsiminkite mano tarno Mozės įstatymą, kurį daviau Horebe visam Izraeliui, jo nuostatus ir potvarkius. 
\par 5 Aš siųsiu jums pranašą Eliją, prieš ateinant didingai ir baisiai Viešpaties dienai. 
\par 6 Jis atgręš tėvų širdis į vaikus ir vaikų širdis į tėvus, kad neateičiau ir neištikčiau žemės prakeikimu”.



\end{document}