\begin{document}

\title{Evangelija pagal Morkų}

\chapter{1}


\par 1 Jėzaus Kristaus, Dievo Sūnaus, Gerosios naujienos pradžia, 
\par 2 kaip pranašų parašyta: “Štai Aš siunčiu pirma Tavęs savo pasiuntinį, kuris nuties Tau kelią. 
\par 3 Dykumoje šaukiančiojo balsas: ‘Paruoškite Viešpačiui kelią! Ištiesinkite Jam takus!’ ” 
\par 4 Jonas pasirodė dykumoje, krikštijo ir skelbė atgailos krikštą nuodėmėms atleisti. 
\par 5 Pas jį traukė visa Judėjos šalis ir Jeruzalės gyventojai. Jie išpažindavo savo nuodėmes ir visi buvo krikštijami Jordano upėje. 
\par 6 Jonas vilkėjo kupranugario vilnų apdaru, o strėnas buvo susijuosęs odiniu diržu. Jis maitinosi skėriais ir lauko medumi. 
\par 7 Jis skelbė: “Po manęs ateina galingesnis už mane­aš nevertas nusilenkęs atrišti Jo sandalų dirželio. 
\par 8 Aš jus krikštijau vandeniu, o Jis krikštys jus Šventąja Dvasia”. 
\par 9 Tomis dienomis atėjo Jėzus iš Galilėjos Nazareto, ir Jonas pakrikštijo Jį Jordane. 
\par 10 Vos tik išbridęs iš vandens, Jis pamatė prasiveriantį dangų ir Dvasią tarsi balandį, nusileidžiančią ant Jo. 
\par 11 Ir iš dangaus pasigirdo balsas: “Tu esi mano mylimasis Sūnus, kuriuo Aš gėriuosi”. 
\par 12 Ir tuojau Dvasia Jį nuvedė į dykumą. 
\par 13 Jis praleido dykumoje keturiasdešimt dienų šėtono gundomas, buvo kartu su žvėrimis, ir angelai Jam tarnavo. 
\par 14 Kai Jonas buvo suimtas, Jėzus atėjo į Galilėją ir skelbė Dievo karalystės Evangeliją: 
\par 15 “Atėjo metas, prisiartino Dievo karalystė. Atgailaukite ir tikėkite Evangelija!” 
\par 16 Eidamas palei Galilėjos ežerą, Jėzus pamatė Simoną ir jo brolį Andriejų, metančius tinklą į ežerą; mat jie buvo žvejai. 
\par 17 Jėzus jiems tarė: “Sekite paskui mane, ir Aš jus padarysiu žmonių žvejais”. 
\par 18 Ir tuojau, palikę tinklus, jie nusekė paskui Jį. 
\par 19 Paėjęs truputį toliau, Jis pamatė Zebediejaus sūnų Jokūbą ir jo brolį Joną, valtyje betaisančius tinklus. 
\par 20 Tuoj pat Jis pašaukė ir juos. Palikę savo tėvą Zebediejų valtyje su samdiniais, jie nusekė paskui Jį. 
\par 21 Jie atėjo į Kafarnaumą, ir iškart, sabato dieną, Jis nuėjo į sinagogą ir ėmė mokyti. 
\par 22 Žmonės stebėjosi Jo mokymu, nes Jis mokė kaip turintis valdžią, o ne kaip Rašto žinovai. 
\par 23 Jų sinagogoje buvo žmogus, turintis netyrąją dvasią. Jis sušuko: 
\par 24 “Palik mus! Ko Tau iš mūsų reikia, Jėzau Nazarieti? Gal atėjai mūsų pražudyti? Aš žinau, kas Tu esi: Dievo Šventasis!” 
\par 25 Jėzus sudraudė jį: “Nutilk ir išeik iš jo!” 
\par 26 Tada netyroji dvasia pradėjo jį tąsyti ir, baisiai šaukdama, išėjo iš jo. 
\par 27 Visi apstulbo ir klausinėjo vienas kitą: “Kas tai? Koks čia naujas mokymas? Jis įsakinėja su valdžia net netyrosioms dvasioms, ir tos Jam paklūsta!” 
\par 28 Garsas apie Jį kaipmat pasklido po visą Galilėjos kraštą. 
\par 29 Išėję iš sinagogos, jie su Jokūbu ir Jonu atėjo į Simono ir Andriejaus namus. 
\par 30 Simono uošvė gulėjo karščiuodama, ir jie tuojau apie tai Jam pasakė. 
\par 31 Jis priėjęs paėmė ją už rankos ir pakėlė; karštis tučtuojau paliovė, ir ji galėjo jiems patarnauti. 
\par 32 Vakare, saulei nusileidus, pas Jėzų sugabeno visus ligonius ir demonų apsėstuosius. 
\par 33 Visas miestas buvo susirinkęs prie durų. 
\par 34 Jis pagydė daug sergančių įvairiomis ligomis, išvarė daug demonų ir neleido demonams kalbėti, nes jie pažino Jį. 
\par 35 Anksti rytą, gerokai prieš aušrą, Jėzus atsikėlęs nuėjo į nuošalią vietą ir ten meldėsi. 
\par 36 Simonas ir buvę su juo nusekė iš paskos 
\par 37 ir, suradę Jį, pasakė: “Visi Tavęs ieško”. 
\par 38 Jis atsakė: “Eikime kitur, į gretimus miestelius, kad ir ten pamokslaučiau, nes tam esu atėjęs”. 
\par 39 Ir Jis pamokslavo jų sinagogose po visą Galilėją ir išvarinėjo demonus. 
\par 40 Pas Jį atėjo raupsuotasis ir atsiklaupęs maldavo: “Jeigu nori, gali mane padaryti švarų”. 
\par 41 Jėzus, apimtas gailesčio, ištiesė ranką, palietė jį ir tarė: “Noriu, būk švarus!” 
\par 42 Jam tai ištarus, raupsai iškart pranyko, ir jis tapo švarus. 
\par 43 Jėzus liepė jam tuojau pasišalinti ir griežtai įspėjo: 
\par 44 “Žiūrėk, kad niekam nieko nepasakotum! Eik, pasirodyk kunigui ir aukok už apvalymą Mozės įsakytą atnašą jiems paliudyti”. 
\par 45 Bet šis išėjęs pradėjo taip plačiai skelbti ir skleisti tą įvykį, jog Jėzus nebegalėjo viešai pasirodyti mieste. Jis laikėsi už miesto, negyvenamose vietose, bet žmonės iš visur rinkosi pas Jį.


\chapter{2}


\par 1 Po kelių dienų Jėzus vėl atėjo į Kafarnaumą. Žmonės, išgirdę Jį esant namuose, 
\par 2 nedelsiant susirinko, ir taip gausiai, jog nė prie durų nebeliko vietos. O Jis skelbė jiems žodį. 
\par 3 Tada keturi vyrai atnešė pas Jį paralyžiuotą žmogų. 
\par 4 Negalėdami dėl minios prinešti jo prie Jėzaus, jie praplėšė stogą namo, kur Jis buvo, ir, padarę skylę, nuleido žemyn neštuvus, ant kurių gulėjo paralyžiuotasis. 
\par 5 Išvydęs jų tikėjimą, Jėzus tarė paralyžiuotajam: “Sūnau, atleidžiamos tau tavo nuodėmės!” 
\par 6 Tenai sėdėjo keletas Rašto žinovų ir svarstė savo širdyse: 
\par 7 “Kodėl Jis taip piktžodžiauja? Kas gali atleisti nuodėmes, jei ne vienas Dievas?!” 
\par 8 Jėzus, iš karto savo dvasia supratęs jų mintis, tarė: “Kam taip svarstote savo širdyse? 
\par 9 Kas lengviau­ar pasakyti paralyžiuotam: ‘Tau atleidžiamos nuodėmės’, ar liepti: ‘Kelkis, imk neštuvus ir vaikščiok’? 
\par 10 Bet, kad žinotumėte Žmogaus Sūnų turintį valdžią atleisti žemėje nuodėmes,­čia Jis tarė paralyžiuotajam,­ 
\par 11 sakau tau: kelkis, imk savo neštuvus ir eik namo!” 
\par 12 Šis tuojau atsikėlė ir, visiems matant, pasiėmęs neštuvus, išėjo. Visi stebėjosi ir garbino Dievą, sakydami: “Tokių dalykų mes niekad nesame matę”. 
\par 13 Jėzus vėl nuėjo į paežerę. Didelė minia rinkosi prie Jo, ir Jis juos mokė. 
\par 14 Praeidamas Jis pamatė Levį, Alfiejaus sūnų, sėdintį muitinėje, ir tarė jam: “Sek paskui mane!” Šis atsikėlė ir nusekė paskui Jį. 
\par 15 Kai Jėzus Levio namuose valgė, drauge prie stalo su Jėzumi ir mokiniais sėdėjo daug muitininkų bei nusidėjėlių, nes tokių buvo daug ir jie sekė paskui Jį. 
\par 16 Rašto žinovai ir fariziejai, išvydę Jį valgantį su muitininkais ir nusidėjėliais, klausė Jo mokinių: “Kodėl Jis valgo su muitininkais ir nusidėjėliais?” 
\par 17 Tai išgirdęs, Jėzus atsiliepė: “Ne sveikiesiems reikia gydytojo, bet ligoniams! Aš atėjau šaukti ne teisiųjų, bet nusidėjėlių atgailai”. 
\par 18 Jono mokiniai ir fariziejai pasninkavo. Jie atėję klausė Jėzų: “Kodėl Jono ir fariziejų mokiniai pasninkauja, o Tavieji ne?” 
\par 19 Jėzus atsakė: “Argi gali vestuvininkai pasninkauti, kol su jais yra jaunikis? Kol jie turi su savim jaunikį, jie negali pasninkauti. 
\par 20 Bet ateis dienos, kai jaunikis bus iš jų atimtas, ir tada, tomis dienomis, jie pasninkaus. 
\par 21 Niekas nesiuva lopo iš naujo audinio ant sudėvėto drabužio; antraip naujasis atplėštų nuo senojo gabalą, ir pasidarytų dar didesnė skylė. 
\par 22 Taip pat niekas nepila jauno vyno į senus vynmaišius. Antraip vynas suplėšytų vynmaišius, ir nueitų niekais ir vynas, ir vynmaišiai. Jaunam vynui būtini nauji vynmaišiai!” 
\par 23 Sabato dieną Jėzus ėjo per javų lauką, ir Jo mokiniai eidami skabė varpas. 
\par 24 Fariziejai Jam sakė: “Žiūrėk, kodėl jie daro per sabatą tai, kas draudžiama?” 
\par 25 Jėzus atsakė: “Nejaugi niekad neskaitėte, ką prireikus padarė Dovydas, kai jis ir jo palydovai neturėjo maisto ir buvo išalkę? 
\par 26 Prie vyriausiojo kunigo Abjataro jis įėjo į Dievo namus ir valgė padėtinės duonos ir davė savo palydovams, nors niekam neleistina jos valgyti, tik kunigams”. 
\par 27 Ir pridūrė: “Sabatas padarytas žmogui, o ne žmogus sabatui; 
\par 28 taigi Žmogaus Sūnus yra ir sabato Viešpats”.



\chapter{3}


\par 1 Jis vėl atėjo į sinagogą, o ten buvo žmogus su padžiūvusia ranka. 
\par 2 Jie stebėjo Jį, ar Jis gydys šį sabato dieną, kad galėtų Jėzų apkaltinti. 
\par 3 Jėzus tarė žmogui su padžiūvusia ranka: “Stok į vidurį!” 
\par 4 O juos paklausė: “Ar sabato dieną leistina daryti gera, ar bloga? Gelbėti gyvybę ar žudyti?” Bet anie tylėjo. 
\par 5 Tada, rūsčiai juos apžvelgęs ir nuliūdęs dėl jų širdies kietumo, tarė tam žmogui: “Ištiesk ranką!” Šis ištiesė, ir ranka tapo sveika kaip ir kita. 
\par 6 Išėję fariziejai tuojau pradėjo tartis su erodininkais, kaip Jėzų pražudyti. 
\par 7 O Jėzus su savo mokiniais pasitraukė prie ežero. Jį sekė didelė minia iš Galilėjos. Ir iš Judėjos, 
\par 8 Jeruzalės ir Idumėjos, iš anapus Jordano bei Tyro ir Sidono šalies atvyko daugybė žmonių, kurie buvo girdėję apie Jo didelius darbus. 
\par 9 Jėzus liepė mokiniams laikyti Jam paruoštą nedidelę valtį, kad minia Jo nesuspaustų. 
\par 10 Mat Jis buvo daugelį išgydęs, ir visi, kuriuos kankino ligos, veržėsi prie Jo, norėdami Jį paliesti. 
\par 11 Taip pat netyrosios dvasios, vos tik Jį pamačiusios, parpuldavo priešais Jį ir šaukdavo: “Tu esi Dievo Sūnus!” 
\par 12 Bet Jėzus griežtai jas drausdavo, kad Jo negarsintų. 
\par 13 Jėzus užkopė ant kalno ir pasišaukė, kuriuos pats norėjo, ir jie atėjo pas Jį. 
\par 14 Jis paskyrė dvylika, kad jie būtų kartu su Juo ir kad galėtų siųsti juos pamokslauti 
\par 15 ir jie turėtų valdžią gydyti ligas ir išvarinėti demonus: 
\par 16 Simoną, pavadinęs jį Petru; 
\par 17 Zebediejaus sūnų Jokūbą ir Jokūbo brolį Joną (juos pavadinęs Boanerges, tai yra “griaustinio vaikai”); 
\par 18 Andriejų, Pilypą, Baltramiejų, Matą, Tomą, Alfiejaus sūnų Jokūbą, Tadą, Simoną Kananietį 
\par 19 ir Judą Iskarijotą, kuris Jį ir išdavė. Ir jie grįžo namo. 
\par 20 Vėl susirinko tiek žmonių, kad jie nebegalėjo nė pavalgyti. 
\par 21 Saviškiai, apie tai išgirdę, ėjo sulaikyti Jo, sakydami, kad Jis išėjęs iš proto. 
\par 22 Atvykę iš Jeruzalės Rašto žinovai sakė: “Jis turi Belzebulą”, ir: “Demonų kunigaikščio jėga Jis išvaro demonus”. 
\par 23 O Jėzus, pasivadinęs juos, kalbėjo palyginimais: “Kaip gali šėtonas išvaryti šėtoną? 
\par 24 Jei karalystė suskilusi, tokia karalystė negali išsilaikyti. 
\par 25 Ir jei namai suskilę, tokie namai negali išsilaikyti. 
\par 26 Ir jei šėtonas sukyla pats prieš save ir tampa susiskaldęs, jis negali išsilaikyti ir žlunga. 
\par 27 Niekas negali įeiti į galiūno namus ir pasigrobti jo turto, pirmiau nesurišęs galiūno. Tik tada jis apiplėš jo namus. 
\par 28 Iš tiesų sakau jums: bus atleistos žmonių vaikams visos nuodėmės ir piktžodžiavimai, kaip jie bepiktžodžiautų; 
\par 29 bet jei kas piktžodžiautų Šventajai Dvasiai, tam niekada nebus atleista, ir jis amžiams bus pasmerktas”. 
\par 30 Mat jie sakė: “Jis turi netyrąją dvasią”. 
\par 31 Atėjo Jėzaus motina ir broliai ir, lauke sustoję, prašė Jį pakviesti. 
\par 32 Aplink Jį sėdėjo minia, kai Jam pranešė: “Štai Tavo motina ir broliai bei seserys lauke stovi ir ieško Tavęs”. 
\par 33 O Jis atsakė: “Kas yra mano motina ir mano broliai?” 
\par 34 Ir, apžvelgęs aplinkui sėdinčius, tarė: “Štai mano motina ir mano broliai! 
\par 35 Kas vykdo Dievo valią, tas mano brolis, ir sesuo, ir motina”.



\chapter{4}


\par 1 Jis vėl ėmė mokyti paežerėje. Prie Jo susirinko didžiausia minia, kad Jis net įlipo į valtį ir sėdėjo ant vandens, o visa minia liko ant kranto palei ežerą. 
\par 2 Jis mokė juos daugelio dalykų palyginimais. Mokydamas sakė: 
\par 3 “Paklausykite! Štai sėjėjas išėjo sėti. 
\par 4 Jam besėjant, dalis grūdų nukrito palei kelią, ir atskridę padangių paukščiai juos sulesė. 
\par 5 Kiti nukrito į uolėtą dirvą, kur buvo nedaug žemės, ir greit sudygo, nes neturėjo gilesnio žemės sluoksnio. 
\par 6 Saulei patekėjus, daigai išdegė ir, neturėdami šaknų, sudžiūvo. 
\par 7 Kiti nukrito tarp erškėčių. Erškėčiai išaugo ir nusmelkė juos, ir jie nedavė derliaus. 
\par 8 Dar kiti nukrito į gerą žemę, sudygo, užaugo ir davė derlių: vieni trisdešimteriopą, kiti šešiasdešimteriopą, treti šimteriopą”. 
\par 9 Jis pasakė jiems: “Kas turi ausis klausyti­teklauso!” 
\par 10 Kai Jėzus pasiliko vienas, aplink Jį esantys kartu su dvylika paklausė apie palyginimą. 
\par 11 Jis atsakė: “Jums duota suprasti Dievo karalystės paslaptis, o pašaliniams viskas sakoma palyginimais, 
\par 12 kad jie ‘regėti regėtų, bet nematytų, girdėti girdėtų, bet nesuprastų, kad neatsiverstų ir nuodėmės nebūtų jiems atleistos’ ”. 
\par 13 Ir Jis paklausė: “Nejaugi nesuprantate šito palyginimo? Tai kaip suprasite visus kitus palyginimus? 
\par 14 Sėjėjas sėja žodį. 
\par 15 Palei kelią sėjamas žodis­tai tie, kuriems vos išgirdus, tuoj pat ateina šėtonas ir išplėšia jų širdyse pasėtąjį žodį. 
\par 16 Panašiai ir su tais, kurie pasėti uolėtoje dirvoje. Išgirdę žodį, jie tuojau su džiaugsmu jį priima. 
\par 17 Bet jie neturi savyje šaknų ir išsilaiko neilgai. Ištikus kokiam sunkumui ar persekiojimui dėl žodžio, jie tuoj pat pasipiktina. 
\par 18 Kiti sėjami tarp erškėčių. Jie išgirsta žodį, 
\par 19 bet pasaulio rūpesčiai, turtų apgaulė ir sukilę geismai kitiems dalykams nusmelkia žodį ir jis tampa nevaisingas. 
\par 20 Geroje žemėje pasėta sėkla­tai tie, kurie išgirsta žodį, priima jį ir duoda vaisių: kas trisdešimteriopą, kas šešiasdešimteriopą, o kas šimteriopą”. 
\par 21 Jis kalbėjo jiems: “Argi žiburys atnešamas pakišti po indu ar po lova? Argi ne įstatyti į žibintuvą? 
\par 22 Juk nėra nieko paslėpta, kas nebūtų apreikšta, ir nieko paslaptyje laikomo, kas neišeitų aikštėn. 
\par 23 Jei kas turi ausis klausyti­teklauso!” 
\par 24 Jis sakė jiems: “Įsidėmėkite, ką girdite: kokiu saiku seikėjate, tokiu ir jums bus atseikėta, ir jums, kurie girdite, bus dar pridėta. 
\par 25 Kas turi, tam bus duota, o iš neturinčio bus atimta net ir tai, ką turi”. 
\par 26 Jis kalbėjo: “Su Dievo karalyste yra kaip su žmogumi, beriančiu dirvon sėklą. 
\par 27 Ar jis miega ar keliasi, ar naktį ar dieną, sėkla dygsta ir auga, jam nežinant kaip. 
\par 28 Juk žemė savaime duoda vaisių: pradžioje želmenį, paskui varpą, pagaliau pribrendusį grūdą varpoje. 
\par 29 Derliui prinokus, žmogus tuojau ima pjautuvą, nes pjūtis atėjo”. 
\par 30 Jėzus dar sakė: “Su kuo galima palyginti Dievo karalystę? Arba kokiu palyginimu ją pavaizduosime? 
\par 31 Ji kaip garstyčios grūdelis, kuris sėjamas dirvon. Jis yra mažiausias iš visų sėklų žemėje, 
\par 32 bet pasėtas užauga, tampa didesnis už visus augalus ir išleidžia plačias šakas, kad jo pavėsyje gali susisukti lizdą padangių sparnuočiai”. 
\par 33 Daugeliu tokių palyginimų Jėzus skelbė jiems žodį, kiek jie sugebėjo suprasti. 
\par 34 Be palyginimų Jis jiems nekalbėdavo, bet kai pasilikdavo vienas su savo mokiniais, viską jiems išaiškindavo. 
\par 35 Tą pačią dieną, atėjus vakarui, Jis tarė mokiniams: “Irkimės į aną pusę!” 
\par 36 Atleidę žmones, jie Jį pasiėmė, nes Jis tebebuvo valtyje. Kartu plaukė ir kitos mažos valtys. 
\par 37 Pakilo didžiulė vėtra ir bangos daužėsi į valtį taip, kad ją jau sėmė. 
\par 38 Jėzus buvo valties gale ir miegojo ant pagalvės. Mokiniai pažadino Jį, šaukdami: “Mokytojau, Tau nerūpi, kad mes žūvame?!” 
\par 39 Pabudęs Jis sudraudė vėją ir įsakė ežerui: “Nutilk, nusiramink!” Tuojau pat vėjas nutilo, ir pasidarė visiškai ramu. 
\par 40 Jis tarė jiems: “Kodėl jūs tokie bailūs? Ar vis dar neturite tikėjimo?” 
\par 41 Juos apėmė didelė baimė, ir jie kalbėjo vienas kitam: “Kas gi Jis toks? Net vėjas ir ežeras Jo klauso!”



\chapter{5}


\par 1 Jie priplaukė ežero krantą gadariečių krašte. 
\par 2 Jam išlipus iš valties, tuojau prieš Jį iš kapinių atbėgo vyras, turintis netyrąją dvasią. 
\par 3 Jis gyveno kapų rūsiuose, ir niekas negalėjo nė grandinėmis jo surakinti. 
\par 4 Nors jis jau daug kartų buvo pančiojamas ir grandinėmis rakinamas, bet sutrupindavo grandines, nutraukydavo pančius, ir niekas negalėdavo jo suvaldyti. 
\par 5 Per kiauras naktis ir dienas jis bastydavosi po kalnus ir kapines, klykdamas ir daužydamas save akmenimis. 
\par 6 Iš tolo pamatęs Jėzų, atbėgo, parpuolė prieš Jį 
\par 7 ir ėmė garsiai šaukti: “Ko Tau reikia iš manęs, Jėzau, aukščiausiojo Dievo Sūnau? Saikdinu tave Dievu, nekankink manęs!” 
\par 8 Jėzus mat buvo paliepęs: “Išeik, netyroji dvasia, iš žmogaus!” 
\par 9 Jėzus dar paklausė: “O kaip tu vadiniesi?” Ji atsakė: “Mano vardas­Legionas, nes mūsų daug”. 
\par 10 Ir pradėjo labai prašytis nevaryti jų iš to krašto. 
\par 11 Ten pat, atkalnėje, ganėsi didžiulė kiaulių banda. 
\par 12 Visi demonai maldavo Jį, sakydami: “Pasiųsk mus į tas kiaules, kad į jas sueitume!” 
\par 13 Jėzus iškart jiems leido. Išėjusios netyrosios dvasios apniko kiaules, ir visa banda, apie du tūkstančius kiaulių, metėsi nuo skardžio į ežerą ir prigėrė. 
\par 14 Tie, kurie jas ganė, išbėgiojo ir pranešė apie įvykį mieste ir kaimuose. Žmonės išėjo pažiūrėti, kas atsitiko. 
\par 15 Atėję prie Jėzaus, pamatė sėdintį demonų apsėstąjį­tą, kuriame buvo Legionas,­apsirengusį ir sveiko proto, ir juos apėmė baimė. 
\par 16 Mačiusieji papasakojo jiems, kas nutiko su apsėstuoju, ir apie kiaules. 
\par 17 Tada žmonės ėmė Jėzų maldauti, kad Jis pasišalintų iš jų krašto. 
\par 18 Jėzui lipant į valtį, buvęs demonų apsėstasis prašė leisti pasilikti su Juo, 
\par 19 bet Jėzus nesutiko ir pasakė: “Eik namo pas saviškius ir papasakok, kokių didžių dalykų Viešpats tau padarė ir kaip tavęs pasigailėjo”. 
\par 20 Tada jis nuėjo savo keliu ir Dekapolyje ėmė skelbti, kokių didžių dalykų Jėzus jam padarė, ir visi stebėjosi. 
\par 21 Jėzui vėl persikėlus valtimi į kitą pusę, prie Jo susirinko didžiulė minia, ir Jis buvo paežerėje. 
\par 22 Štai ateina vienas iš sinagogos vyresniųjų, vardu Jayras, ir, pamatęs Jį, puola Jam po kojų, 
\par 23 karštai maldaudamas: “Mano dukrelė miršta! Ateik ir uždėk ant jos rankas, kad pagytų ir gyventų”. 
\par 24 Jėzus nuėjo su juo. Paskui Jį sekė didžiulė minia ir Jį spauste spaudė. 
\par 25 Ten buvo viena moteris, dvylika metų serganti kraujoplūdžiu. 
\par 26 Nemaža iškentėjusi nuo daugelio gydytojų ir išleidusi visa, ką turėjo, ji nė kiek nepasitaisė, bet ėjo vis blogyn. 
\par 27 Išgirdusi apie Jėzų, ji prasispraudė iš minios galo ir prisilietė prie Jo apsiausto. 
\par 28 Mat ji kalbėjo: “Jeigu paliesiu bent Jo drabužį­išgysiu!” 
\par 29 Tuojau kraujas nustojo jai plūdęs, ir ji pajuto kūnu, kad yra pasveikusi nuo savo ligos. 
\par 30 Ir Jėzus iš karto pajuto, kad iš Jo išėjo jėga, ir, atsigręžęs į minią, paklausė: “Kas prisilietė prie mano apsiausto?” 
\par 31 Jo mokiniai Jam atsakė: “Matai, kaip minia Tave spaudžia, o Tu klausi: ‘Kas mane palietė?’ ” 
\par 32 Bet Jis dairėsi tos, kuri taip buvo padariusi. 
\par 33 Moteris išėjo į priekį išsigandusi ir virpėdama, nes žinojo, kas jai atsitiko, ir, puolusi prieš Jį, papasakojo visą tiesą. 
\par 34 O Jis tarė jai: “Dukra, tavo tikėjimas išgydė tave, eik rami ir būk sveika nuo savo ligos”. 
\par 35 Jam dar tebekalbant, atėjo sinagogos vyresniojo žmonės ir pranešė: “Tavo duktė numirė, kam dar vargini Mokytoją?” 
\par 36 Išgirdęs tuos žodžius, Jėzus tarė sinagogos vyresniajam: “Nebijok, vien tik tikėk!” 
\par 37 Ir Jis niekam neleido eiti kartu, išskyrus Petrą, Jokūbą ir Jokūbo brolį Joną. 
\par 38 Atėjęs į sinagogos vyresniojo namus, Jėzus pamatė sujudimą ir garsiai verkiančius bei raudančius. 
\par 39 Įžengęs vidun, Jis tarė: “Kam tas triukšmas ir verksmas?! Vaikas nėra miręs, o miega”. 
\par 40 Žmonės šaipėsi iš Jo. Tada, išvaręs juos visus, Jis pasiėmė vaiko tėvą ir motiną, taip pat savo palydovus ir įėjo ten, kur vaikas gulėjo. 
\par 41 Paėmęs mergaitę už rankos, pasakė jai: “Talitį kum”; išvertus reiškia: “Mergaite, sakau tau, kelkis!” 
\par 42 Mergaitė tuojau atsikėlė ir ėmė vaikščioti. Jai buvo dvylika metų. Jie nustėro iš nuostabos. 
\par 43 Jis griežtai įsakė, kad niekas to nežinotų, ir liepė duoti mergaitei valgyti.



\chapter{6}


\par 1 Išvykęs iš ten, Jis, mokinių lydimas, parkeliavo į savo tėviškę. 
\par 2 Atėjus sabatui, Jis pradėjo mokyti sinagogoje. Daugelis girdėdami stebėjosi: “Iš kur Jam tai? Kas per išmintis Jam suteikta, kad net tokie stebuklai daromi Jo rankomis? 
\par 3 Argi Jis ne dailidė, ne Marijos sūnus, Jokūbo, Jozės, Judo ir Simono brolis?! Argi Jo seserys negyvena čia, pas mus?!” Ir jie piktinosi Juo. 
\par 4 O Jėzus jiems tarė: “Pranašas nebūna be pagarbos, nebent savo tėviškėje tarp savo giminių ir savo namuose”. 
\par 5 Ir Jis ten negalėjo padaryti jokio stebuklo, tik keliems ligoniams uždėjo rankas ir juos išgydė. 
\par 6 Ir Jis stebėjosi jų netikėjimu. Jis ėjo per apylinkės kaimus ir mokė. 
\par 7 Jėzus pasišaukė pas save dvylika, ėmė juos siuntinėti po du ir davė jiems valdžią netyrosioms dvasioms. 
\par 8 Liepė, be lazdos, nieko neimti į kelionę­nei krepšio, nei duonos, nei pinigų dirže,­ 
\par 9 tik apsiauti sandalais, bet neapsivilkti dviejų tunikų. 
\par 10 Ir sakė jiems: “Į kuriuos namus užeisite, ten ir pasilikite, kol išvyksite. 
\par 11 Jei kurioje vietoje jūsų nepriimtų ir nesiklausytų, išeidami iš ten nusikratykite dulkes nuo kojų, kaip liudijimą prieš juos”. 
\par 12 Jie išėjo ir skelbė atgailą, 
\par 13 išvarė daug demonų, daugelį ligonių tepė aliejumi ir išgydė. 
\par 14 Karalius Erodas išgirdęs,­nes Jėzaus vardas tapo plačiai žinomas,­kalbėjo: “Jonas Krikštytojas prisikėlė iš numirusių ir todėl jame veikia stebuklingos jėgos”. 
\par 15 Kiti tvirtino: “Jis­Elijas!” Dar kiti sakė: “Jis pranašas, kaip ir kiti pranašai”. 
\par 16 Tai išgirdęs, Erodas sakė: “Tai Jonas, kuriam aš nukirsdinau galvą,­jis prisikėlė!” 
\par 17 Pats Erodas buvo įsakęs suimti Joną ir laikė jį sukaustytą kalėjime dėl savo brolio Pilypo žmonos Erodiados, kurią buvo vedęs. 
\par 18 Jonas mat sakė Erodui: “Nevalia tau gyventi su brolio žmona”. 
\par 19 Erodiada neapkentė Jono ir troško jį nužudyti, tačiau negalėjo, 
\par 20 nes Erodas bijojo Jono, žinodamas jį esant teisų ir šventą vyrą, ir todėl jį saugojo. Girdėdamas Joną kalbant, jis daug ką darydavo ir mielai jo klausydavosi. 
\par 21 Proga pasitaikė, kai Erodas, švęsdamas savo gimimo dieną, iškėlė pokylį savo didžiūnams, kariuomenės vadams ir Galilėjos kilmingiesiems. 
\par 22 Erodiados duktė ten įėjusi šoko ir patiko Erodui bei jo svečiams. Karalius tarė mergaitei: “Prašyk iš manęs, ko tik nori, ir aš tau duosiu”. 
\par 23 Ir jis prisiekė: “Ko tik prašysi, aš tau duosiu, kad ir pusę savo karalystės!” 
\par 24 Tada ji nuėjusi paklausė savo motiną: “Ko man prašyti?” O ši tarė: “Jono Krikštytojo galvos!” 
\par 25 Toji, skubiai atbėgusi pas karalių, paprašė: “Noriu, kad man tuojau duotum dubenyje Jono Krikštytojo galvą”. 
\par 26 Karalius labai nuliūdo, tačiau dėl priesaikos ir svečių nesiryžo jai atsakyti. 
\par 27 Jis tuoj pat pasiuntė budelį ir įsakė jam atnešti Jono galvą. Šis nuėjęs nukirto jam kalėjime galvą 
\par 28 ir, atnešęs ją dubenyje, padavė mergaitei, o mergaitė atidavė ją savo motinai. 
\par 29 Tai išgirdę, Jono mokiniai atėjo, paėmė jo kūną ir palaidojo kape. 
\par 30 Apaštalai susirinko pas Jėzų ir papasakojo Jam viską, ką buvo nuveikę ir ko mokę. 
\par 31 O Jis tarė jiems: “Eikite vieni į nuošalią vietą ir truputį pailsėkite”. Mat daugybė žmonių ateidavo ir išeidavo, ir jiems nebūdavo laiko nė pavalgyti. 
\par 32 Ir jie išplaukė valtimi į negyvenamą vietą. 
\par 33 Bet žmonės pastebėjo juos išplaukiant, ir daugelis pažino Jį. Iš visų miestų subėgo ten pėsti, pralenkdami mokinius, ir susirinko pas Jį. 
\par 34 Išlipęs į krantą, Jėzus pamatė didžiulę minią, ir Jam pagailo žmonių, nes jie buvo tarsi avys be piemens; ir pradėjo juos mokyti daugelio dalykų. 
\par 35 Dienai baigiantis, priėjo prie Jo mokiniai ir kalbėjo: “Čia dykvietė, ir jau vėlyvas laikas. 
\par 36 Paleisk juos, kad, pasklidę po aplinkinius kiemus bei kaimus, nusipirktų duonos, nes jie neturi ko valgyti”. 
\par 37 Bet Jėzus tarė: “Jūs duokite jiems valgyti”. Mokiniai tada klausia: “Ar mums eiti ir nupirkti duonos už du šimtus denarų ir duoti jiems valgyti?” 
\par 38 Jis sako: “Kiek turite kepalų? Eikite ir pažiūrėkite”. Patikrinę jie atsakė: “Penkis ir dvi žuvis”. 
\par 39 Jis įsakė susodinti žmones būriais ant žalios žolės. 
\par 40 Ir tie susėdo būrys prie būrio, po šimtą ir po penkiasdešimt žmonių. 
\par 41 Jėzus paėmė penkis kepalus ir dvi žuvis, pažvelgė į dangų, palaimino, laužė duoną ir davė savo mokiniams, kad padalintų žmonėms. Taip pat Jis padalino visiems tas dvi žuvis. 
\par 42 Ir visi valgė ir pasisotino. 
\par 43 Jie pririnko dvylika pilnų pintinių duonos trupinių ir žuvies likučių. 
\par 44 O valgytojų buvo apie penkis tūkstančius vyrų. 
\par 45 Tuojau Jis privertė savo mokinius sėsti į valtį ir pirma Jo irtis į kitą krantą prie Betsaidos, kol Jis paleisiąs žmones. 
\par 46 Juos paleidęs, Jis nuėjo į kalną melstis. 
\par 47 Atėjus vakarui, valtis buvo ežero viduryje, o Jis vienas sausumoje. 
\par 48 Matydamas, kad mokiniai vargsta besiirdami,­nes vėjas buvo jiems priešingas,­apie ketvirtą nakties sargybą Jis ateina pas juos, žengdamas ežero paviršiumi, ir buvo bepraeinąs pro šalį. 
\par 49 Šie, pamatę Jį einantį ežeru, pamanė, jog tai šmėkla, ir pradėjo šaukti. 
\par 50 Mat visi Jį matė ir išsigando. Bet Jis tuojau juos prakalbino: “Drąsos! Tai Aš. Nebijokite!” 
\par 51 Tada Jis įlipo pas juos į valtį, ir vėjas liovėsi. Mokiniai buvo didžiai apstulbinti ir be galo stebėjosi, 
\par 52 nes nebuvo supratę duonos stebuklo, kadangi jų širdis tebebuvo užkietėjusi. 
\par 53 Persiyrę per ežerą, jie pasiekė Genezareto kraštą ir čia lipo į krantą. 
\par 54 Jiems išlipus iš valties, žmonės tuojau pažino Jį, 
\par 55 apibėgo visą apylinkę ir pradėjo gabenti neštuvais ligonius ten, kur girdėjo Jį esant. 
\par 56 Ir kur Jėzus užeidavo į kaimus, miestus ar kiemus, jie aikštėse guldydavo ligonius ir maldaudavo Jį, kad leistų jiems palytėti bent savo drabužio apvadą. Ir visi, kas tik Jį paliesdavo, išgydavo.



\chapter{7}


\par 1 Pas Jį susirinko fariziejų ir keli Rašto žinovai, atvykę iš Jeruzalės. 
\par 2 Jie, pamatę kai kuriuos Jo mokinius valgant duoną suterštomis (tai yra neplautomis) rankomis, pradėjo priekaištauti. 
\par 3 Mat fariziejai ir visi žydai, laikydamiesi prosenių tradicijos, valgo tik nusiplovę rankas. 
\par 4 Taip pat sugrįžę iš turgaus, jie nevalgo neapsiplovę. Be to, yra daug kitų nuostatų, kurių jie laikosi, sekdami tradicija, pavyzdžiui, taurių, puodelių bei varinių indų plovimo. 
\par 5 Taigi fariziejai ir Rašto žinovai Jį klausė: “Kodėl Tavo mokiniai nesilaiko prosenių tradicijos ir valgo duoną neplautomis rankomis?” 
\par 6 Jis jiems atsakė: “Gerai apie jus, veidmainius, pranašavo Izaijas, kaip parašyta: ‘Ši tauta gerbia mane lūpomis, bet jų širdis toli nuo manęs. 
\par 7 Veltui jie mane garbina, žmogiškus priesakus paversdami mokymu’. 
\par 8 Palikdami Dievo įsakymą, jūs įsikibę laikotės žmonių tradicijų­ puodelių ir taurių plovimo, ir daug kitų panašių dalykų darote”. 
\par 9 Ir Jis pridūrė: “Puikiai jūs paverčiate niekais Dievo įsakymą, kad tik išsaugotumėte savo tradicijas! 
\par 10 Antai Mozė įsakė: ‘Gerbk savo tėvą ir motiną’, ir: ‘Kas keiktų tėvą ar motiną, mirtimi temiršta’. 
\par 11 O jūs sakote: ‘Jei žmogus pasako savo tėvui ar motinai: Viskas, kas tau būtų naudinga iš manęs, tebūnie Korbįn (tai yra: dovana Dievui),­ 
\par 12 tada jūs nebeleidžiate jam padėti tėvui ar motinai, 
\par 13 savo perduodama tradicija Dievo žodį darydami negaliojantį. Ir daug panašių dalykų darote”. 
\par 14 Sušaukęs visus žmones, Jėzus kalbėjo: “Paklausykite manęs visi ir supraskite: 
\par 15 nėra nieko, kas, iš išorės patekęs į žmogų, galėtų jį suteršti. Žmogų suteršia vien tai, kas iš žmogaus išeina. 
\par 16 Kas turi ausis klausyti­teklauso”. 
\par 17 Kai sugrįžo nuo minios į namus, Jo mokiniai paklausė apie palyginimą. 
\par 18 Jis jiems sako: “Nejaugi ir jūs nesuprantate? Argi neaišku jums, kad visa, kas patenka į žmogų iš lauko, negali jo suteršti, 
\par 19 nes nepatenka į jo širdį, bet į vidurius ir išeina laukan, ir taip išvalomas visas maistas?” 
\par 20 Ir Jis pasakė: “Žmogų suteršia tai, kas iš jo išeina. 
\par 21 Iš vidaus, iš žmonių širdies, išeina pikti sumanymai, svetimavimai, paleistuvystės, vagystės, žmogžudystės, 
\par 22 godumas, piktumas, klasta, nesusilaikymas, pavydas, piktžodžiavimai, išdidumas, kvailystė. 
\par 23 Visos tos blogybės išeina iš vidaus ir suteršia žmogų”. 
\par 24 Išvykęs iš ten, Jis nukeliavo į Tyro ir Sidono sritis. Užėjęs į vienus namus, Jis norėjo, kad niekas apie tai nežinotų, bet Jam nepavyko to nuslėpti. 
\par 25 Išgirdus apie Jį, moteris, kurios duktė turėjo netyrąją dvasią, atėjo ir puolė Jam po kojų. 
\par 26 Moteris buvo graikė, kilimo sirofinikietė. Ji maldavo, kad Jis išvarytų iš jos dukrelės demoną. 
\par 27 Jėzus jai tarė: “Leisk pirmiau pasisotinti vaikams. Juk nedera imti vaikų duoną ir mesti šunyčiams”. 
\par 28 Moteris tarė: “Taip, Viešpatie! Bet ir šunyčiai po stalu ėda vaikų trupinius”. 
\par 29 Tada Jis tarė: “Dėl šitų žodžių eik namo,­demonas jau išėjęs iš tavo dukters”. 
\par 30 Parėjusi namo, ji rado dukrelę gulinčią patale ir demoną išėjusį. 
\par 31 Palikęs Tyro ir Sidono sritis, Jis vėl atėjo prie Galilėjos ežero, į Dekapolio krašto vidurį. 
\par 32 Ten atvedė Jam kurčią nebylį ir prašė uždėti ant jo ranką. 
\par 33 Jis pasivedė jį nuošaliau nuo minios, įkišo savo pirštus į jo ausis, palietė seilėmis jo liežuvį, 
\par 34 pažvelgė į dangų, atsiduso ir tarė jam: “Efatį!”, tai yra: “Atsiverk!” 
\par 35 Ir tuojau atsivėrė jo klausa, atsipalaidavo liežuvio ryšys, ir jis kalbėjo kaip reikia. 
\par 36 Jėzus jiems įsakė niekam šito nepasakoti. Bet kuo labiau Jis draudė, tuo jie plačiau Jį garsino. 
\par 37 Žmonės labai stebėjosi ir kalbėjo: “Jis visa gerai padarė! Jis daro, kad kurtieji girdi ir nebyliai kalba!”



\chapter{8}


\par 1 Anomis dienomis, susirinkus gausiai miniai ir žmonėms neturint ko valgyti, Jėzus, pasišaukęs savo mokinius, tarė: 
\par 2 “Gaila man minios! Jau trys dienos žmonės yra su manimi ir neturi ko valgyti. 
\par 3 Jei paleisiu juos namo alkanus, jie nusilps kelyje, nes kai kurie yra atėję iš toli”. 
\par 4 Mokiniai Jam atsakė: “Iš kur dykumoje gauti duonos jiems pavalgydinti?” 
\par 5 Jėzus paklausė: “Kiek kepalų turite?” Jie atsakė: “Septynis”. 
\par 6 Tada Jis liepė žmonėms susėsti ant žemės. Paėmęs septynis kepalus, palaimino, laužė ir davė mokiniams dalyti, ir tie padalijo miniai. 
\par 7 Jie dar turėjo kelias žuveles. Jis palaimino jas ir taip pat liepė dalyti. 
\par 8 Žmonės pavalgė iki soties, ir jie surinko dar septynias pintines likučių. 
\par 9 O valgytojų buvo apie keturis tūkstančius. Jėzus paleido juos 
\par 10 ir tuojau, įsėdęs su mokiniais į valtį, nuplaukė į Dalmanutos sritį. 
\par 11 Čia priėjo fariziejų ir pradėjo su Juo ginčytis. Mėgindami Jį, jie reikalavo ženklo iš dangaus. 
\par 12 Atsidusęs iš dvasios gilumos, Jis tarė: “Ir kam šita karta reikalauja ženklo? Iš tiesų sakau jums: ženklo šiai kartai nebus duota!” 
\par 13 Ir, palikęs juos, Jis vėl sėdo į valtį ir nuplaukė į kitą krantą. 
\par 14 Mokiniai buvo pamiršę pasiimti duonos. Jie teturėjo su savim valtyje vieną kepalą. 
\par 15 Jėzus juos įspėjo: “Žiūrėkite, saugokitės fariziejų raugo ir Erodo raugo”. 
\par 16 Jie svarstė tarpusavy, sakydami: “Tai todėl, kad nepasiėmėme duonos”. 
\par 17 Tai supratęs, Jėzus tarė: “Kam jūs tariatės neturį duonos? Argi vis dar neišmanote ir nesuprantate ir jūsų širdys vis dar užkietėjusios? 
\par 18 Turite akis, ir nematote; turite ausis, ir negirdite? Argi neatsimenate, 
\par 19 jog penkis kepalus Aš sulaužiau penkiems tūkstančiams? O kiek pilnų pintinių trupinių pririnkote?” Jie atsakė: “Dvylika”. 
\par 20 “O kai septynis kepalus sulaužiau keturiems tūkstančiams, kiek pilnų pintinių trupinių pririnkote?” Jie atsakė: “Septynias”. 
\par 21 Tada Jis tarė: “Tai kaipgi vis dar nesuprantate?!” 
\par 22 Jie ateina į Betsaidą. Ten atveda pas Jėzų neregį ir prašo jį palytėti. 
\par 23 Jis paėmė neregį už rankos ir nusivedė už kaimo. Ten spjovė jam į akis, uždėjo ant jo rankas ir paklausė: “Ar ką nors matai?” 
\par 24 Šis apsižvalgęs tarė: “Regiu žmones. Lyg kokius medžius matau juos vaikščiojančius”. 
\par 25 Jis vėl rankomis palietė jo akis ir liepė apsižvalgyti. Ir šis tapo sveikas ir viską aiškiai matė. 
\par 26 Jėzus išsiuntė jį namo, sakydamas: “Neužeik į kaimą ir niekam ten nepasakok!” 
\par 27 Jėzus su savo mokiniais išėjo į Pilypo Cezarėjos kaimus. Kelyje klausė mokinius: “Kuo mane žmonės laiko?” 
\par 28 Jie atsakė: “Vieni­Jonu Krikštytoju, kiti­Eliju, treti­vienu iš pranašų”. 
\par 29 Tada Jis paklausė: “O jūs kuo mane laikote?” Petras Jam atsakė: “Tu esi Kristus”. 
\par 30 Tada Jėzus griežtai įsakė niekam apie Jį nekalbėti. 
\par 31 Jis pradėjo juos mokyti, jog Žmogaus Sūnus turės daug iškentėti, būti vyresniųjų, aukštųjų kunigų bei Rašto žinovų atmestas, nužudytas ir po trijų dienų prisikelti. 
\par 32 Jis tai kalbėjo visiškai atvirai. Tada Petras, pasivadinęs Jį į šalį, ėmė Jį drausti. 
\par 33 Jėzus atsigręžęs pažiūrėjo į mokinius ir sudraudė Petrą: “Eik šalin, šėtone, nes tu mąstai ne apie tai, kas Dievo, o kas žmonių!” 
\par 34 Pasišaukęs minią ir savo mokinius, Jėzus prabilo: “Jei kas nori eiti paskui mane, teišsižada pats savęs, teima savo kryžių ir teseka manimi. 
\par 35 Kas nori išgelbėti savo gyvybę, tas ją praras; o kas praras savo gyvybę dėl manęs ir dėl Evangelijos, tas ją išgelbės. 
\par 36 Kokia gi žmogui nauda, jeigu jis laimėtų visą pasaulį, o pakenktų savo sielai? 
\par 37 Arba kuo žmogus galėtų išsipirkti savo sielą?! 
\par 38 Jei kas gėdisi manęs ir mano žodžių šios svetimaujančios ir nuodėmingos kartos akivaizdoje, to gėdysis ir Žmogaus Sūnus, kai Jis ateis savo Tėvo šlovėje su šventaisiais angelais”.



\chapter{9}


\par 1 Jis jiems kalbėjo: “Iš tiesų sakau jums: tarp čia stovinčių yra tokių, kurie neragaus mirties, kol išvys Dievo karalystę, ateinančią su galybe”. 
\par 2 Po šešių dienų Jėzus pasiėmė Petrą, Jokūbą ir Joną ir užsivedė juos vienus nuošaliai į aukštą kalną. Ten Jis atsimainė jų akivaizdoje. 
\par 3 Jo drabužiai pradėjo spindėti tarsi sniegas, kaip jų išbalinti negalėtų joks skalbėjas žemėje. 
\par 4 Jiems pasirodė Elijas ir Mozė, ir jie kalbėjosi su Jėzumi. 
\par 5 Petras tarė Jėzui: “Rabi, gera mums čia būti. Pastatykime tris palapines: vieną Tau, kitą Mozei, trečią Elijui”. 
\par 6 Jis nežinojo, ką sakyti, nes jie buvo labai persigandę. 
\par 7 Užėjo debesis ir uždengė juos, o iš debesies nuskambėjo balsas: “Šitas yra mano mylimasis Sūnus. Jo klausykite!” 
\par 8 Ir tuojau apsižvalgę, jie nieko prie savęs nebematė, tik vieną Jėzų. 
\par 9 Besileidžiant nuo kalno, Jėzus liepė niekam nepasakoti, ką jie matė, kol Žmogaus Sūnus neprisikels iš numirusių. 
\par 10 Jie įsidėmėjo šį pasakymą ir svarstė, ką reiškia “prisikelti iš numirusių”. 
\par 11 Jie klausė Jį: “Kodėl Rašto žinovai sako, jog pirmiau turįs ateiti Elijas?” 
\par 12 Jėzus jiems atsakė: “Tikrai, Elijas ateina pirmas ir viską atstato. Bet kaipgi parašyta apie Žmogaus Sūnų, jog Jis daug iškentėsiąs ir būsiąs paniekintas? 
\par 13 Todėl sakau jums: Elijas buvo atėjęs, ir jie pasielgė su juo kaip norėjo­taip, kaip apie jį parašyta”. 
\par 14 Sugrįžęs pas mokinius, Jis pamatė apie juos susirinkusią didelę minią ir besiginčijančius su jais Rašto žinovus. 
\par 15 Vos pastebėjusi Jėzų, minia labai nustebo, ir visi bėgo Jį pasveikinti. 
\par 16 Jis paklausė Rašto žinovų: “Apie ką ginčijatės su jais?” 
\par 17 Vienas iš minios Jam atsakė: “Mokytojau, aš atvedžiau pas Tave savo sūnų, kuris turi nebylę dvasią. 
\par 18 Kur tik sugriebusi, dvasia jį tąso, iš burnos jam eina putos, jis griežia dantimis ir pastyra. Aš prašiau Tavo mokinius išvaryti dvasią, bet jie nepajėgė”. 
\par 19 Tada Jėzus tarė: “O netikinti karta! Kiek dar man reikės su jumis būti? Kaip ilgai jus kęsti? Atveskite jį pas mane!” 
\par 20 Jie atvedė. Vos Jį pamačiusi, dvasia pradėjo tąsyti berniuką; šis parpuolė ant žemės ir apsiputojęs raičiojosi. 
\par 21 Jėzus paklausė tėvą: “Ar seniai jam taip darosi?” Šis atsakė: “Nuo pat vaikystės. 
\par 22 Dvasia dažnai įstumdavo jį į vandenį ir į ugnį, norėdama nužudyti. Bet, jei ką gali padaryti, pasigailėk mūsų ir padėk mums!” 
\par 23 Jėzus jam atsakė: “Jei gali tikėti, viskas įmanoma tikinčiam!” 
\par 24 Tučtuojau vaiko tėvas verkdamas sušuko: “Tikiu, Viešpatie! Padėk mano netikėjimui!” 
\par 25 Matydamas susibėgančią minią, Jėzus sudraudė netyrąją dvasią, sakydamas jai: “Nebyle ir kurčia dvasia, įsakau tau, išeik iš jo ir daugiau nebegrįžk!” 
\par 26 Dvasia, klykdama ir smarkiai jį purtydama, išėjo. O jis liko tarsi negyvas, ir daugelis sakė: “Jis mirė”. 
\par 27 Bet Jėzus paėmė jį už rankos, pakėlė, ir šis atsistojo. 
\par 28 Kai Jis sugrįžo namo, mokiniai, pasilikę su Juo vieni, klausė: “Kodėl mes negalėjome jos išvaryti?” 
\par 29 Jis atsakė: “Ta veislė neišvaroma nieku kitu, tik malda ir pasninku”. 
\par 30 Iš ten išėję, jie keliavo per Galilėją. Jėzus nenorėjo, kad kas apie tai žinotų. 
\par 31 Mokydamas savo mokinius, Jis sakė jiems: “Žmogaus Sūnus bus atiduotas į žmonių rankas, ir jie nužudys Jį. Nužudytas Jis po trijų dienų prisikels”. 
\par 32 Mokiniai nesuprato tų žodžių, bet bijojo Jį klausti. 
\par 33 Jie atėjo į Kafarnaumą. Namie Jėzus juos paklausė: “Apie ką kalbėjotės kelyje?” 
\par 34 Jie tylėjo. Mat kelyje jie ginčijosi, kuris iš jų didžiausias. 
\par 35 Atsisėdęs Jis pasišaukė dvylika ir tarė: “Jei kas trokšta būti pirmas, tebūnie paskutinis ir visų tarnas!” 
\par 36 Paėmęs mažą vaiką, pastatė tarp jų ir, apsikabinęs jį, pasakė jiems: 
\par 37 “Kas priima tokį vaiką mano vardu, tas priima mane, o kas priima mane, tas ne mane priima, bet Tą, kuris mane siuntė”. 
\par 38 Jonas Jam tarė: “Mokytojau, mes matėme vieną, kuris nevaikščioja su mumis, bet Tavo vardu išvarinėja demonus. Mes jam draudėme, nes jis mūsų neseka”. 
\par 39 Jėzus atsakė: “Nedrauskite jam! Nėra tokio, kuris mano vardu darytų stebuklus ir galėtų čia pat blogai apie mane kalbėti. 
\par 40 Kas ne prieš mus, tas už mus. 
\par 41 Kas duos jums atsigerti taurę vandens mano vardu dėl to, kad priklausote Kristui, iš tiesų sakau jums, tas nepraras savo atlygio”. 
\par 42 “Kas pastūmėtų į nuodėmę vieną iš šitų mažutėlių, kurie tiki manimi, tam būtų geriau, jeigu jam užkabintų ant kaklo girnų akmenį ir įmestų į jūrą. 
\par 43 Jei tavo ranka traukia tave nusidėti,­nukirsk ją! Tau geriau sužalotam įeiti į amžinąjį gyvenimą, negu su abiem rankom patekti į pragarą, į negęstančią ugnį, 
\par 44 kur ‘jų kirminas nemiršta ir ugnis negęsta’. 
\par 45 Ir jei tavo koja traukia tave į nuodėmę, nukirsk ją, nes tau geriau luošam įžengti į amžinąjį gyvenimą, negu su abiem kojom būti įmestam į pragarą, į negęstančią ugnį, 
\par 46 kur ‘jų kirminas nemiršta ir ugnis negęsta’. 
\par 47 Ir jei tavo akis traukia tave nusidėti,­išlupk ją, nes geriau tau vienakiui įeiti į Dievo karalystę, negu su abiem akim būti įmestam į ugnies pragarą, 
\par 48 kur ‘jų kirminas nemiršta ir ugnis negęsta’. 
\par 49 Kiekvienas bus pasūdytas ugnimi, ir kiekviena auka bus druska pasūdyta. 
\par 50 Druska­geras daiktas, bet jeigu ji netektų sūrumo, kuo ją pasūdyti? Turėkite savyje druskos ir taikiai gyvenkite tarpusavy”.



\chapter{10}


\par 1 Jėzus, iškeliavęs iš ten, atvyksta į Judėją ir Užjordanę. Žmonių būriai vėl susirinko pas Jį, ir Jis vėl mokė, kaip buvo pratęs. 
\par 2 Tada atėjo fariziejų, kurie, mėgindami Jį, klausė, ar galima vyrui atleisti žmoną. 
\par 3 Jis jiems atsakė: “O ką jums įsakė Mozė?” 
\par 4 Jie tarė: “Mozė leido parašyti skyrybų raštą ir atleisti”. 
\par 5 Tuomet Jėzus pasakė: “Dėl jūsų širdies kietumo parašė jums Mozė tokį nuostatą. 
\par 6 O nuo sutvėrimo pradžios Dievas ‘sukūrė juos, vyrą ir moterį’. 
\par 7 ‘Todėl vyras paliks savo tėvą ir motiną ir susijungs su savo žmona, 
\par 8 ir du taps vienu kūnu’. Taigi jie jau nebėra du, o vienas kūnas. 
\par 9 Todėl ką Dievas sujungė, žmogus teneperskiria”. 
\par 10 Namie mokiniai vėl klausė Jį apie tai. 
\par 11 Jis atsakė: “Kas atleidžia savo žmoną ir veda kitą, tas nusikalsta pirmajai svetimavimu. 
\par 12 Ir jei moteris palieka savo vyrą ir išteka už kito, ji svetimauja”. 
\par 13 Jam nešė vaikučius, kad juos palytėtų, bet mokiniai jiems draudė. 
\par 14 Tai pamatęs, Jėzus pyktelėjo ir tarė jiems: “Leiskite mažutėliams ateiti pas mane ir netrukdykite, nes tokių yra Dievo karalystė. 
\par 15 Iš tiesų sakau jums: kas nepriims Dievo karalystės kaip mažas vaikas,­niekaip neįeis į ją”. 
\par 16 Ir Jis laimino juos, apkabindamas ir dėdamas ant jų rankas. 
\par 17 Jėzui išeinant į kelią, vienas žmogus pribėgęs puolė prieš Jį ant kelių ir klausė: “Gerasis Mokytojau, ką turiu daryti, kad paveldėčiau amžinąjį gyvenimą?” 
\par 18 Jėzus jam tarė: “Kam vadini mane geru? Nė vieno nėra gero, tik vienas Dievas. 
\par 19 Žinai įsakymus: ‘Nesvetimauk, nežudyk, nevok, melagingai neliudyk, neapgaudinėk, gerbk savo tėvą ir motiną’ ”. 
\par 20 Tas atsakė: “Mokytojau, aš viso to laikausi nuo savo jaunystės”. 
\par 21 Jėzus, pažvelgęs į jį, jį pamilo ir tarė: “Vieno dalyko tau trūksta: eik, parduok visa, ką turi, išdalink vargšams ir turėsi turtą danguje. Tada ateik, paimk kryžių ir sek paskui mane”. 
\par 22 Po šitų žodžių jis nuliūdo ir nusiminęs pasitraukė, nes turėjo daug turto. 
\par 23 Jėzus apsidairė ir prabilo į mokinius: “Kaip sunkiai turtingi įeis į Dievo karalystę!” 
\par 24 Mokiniai buvo priblokšti Jo žodžių. Tada Jėzus vėl jiems tarė: “Vaikeliai, kaip sunku tiems, kurie pasitiki turtais, įeiti į Dievo karalystę! 
\par 25 Lengviau kupranugariui išlįsti pro adatos ausį, negu turtingam įeiti į Dievo karalystę”. 
\par 26 Mokiniai dar labiau nustebo ir kalbėjosi: “Kas tada gali būti išgelbėtas?” 
\par 27 Jėzus pažvelgė į juos ir tarė: “Žmonėms tai neįmanoma, bet ne Dievui, nes Dievui viskas įmanoma”. 
\par 28 Tada Petras sakė Jam: “Štai mes viską palikome ir sekame paskui Tave!” 
\par 29 Jėzus tarė: “Iš tiesų sakau jums: nėra nė vieno, kuris paliktų namus ar brolius, ar seseris, ar motiną, ar tėvą, ar vaikus, ar laukus dėl manęs ir dėl Evangelijos 
\par 30 ir kuris jau dabar, šiuo metu, negautų šimteriopai namų, brolių, seserų, motinų, vaikų ir laukų kartu su persekiojimais ir būsimajame amžiuje­amžinojo gyvenimo. 
\par 31 Tačiau daug pirmųjų bus paskutiniai, ir paskutiniai­pirmi”. 
\par 32 Jiems bekeliaujant į Jeruzalę, Jėzus ėjo priekyje, o mokiniai stebėjosi. Sekdami iš paskos, jie nuogąstavo. Vėl pasišaukęs dvylika, pradėjo jiems sakyti, kas Jo laukia: 
\par 33 “Štai einame į Jeruzalę, ir Žmogaus Sūnus bus išduotas aukštiesiems kunigams ir Rašto žinovams. Jie nuteis Jį mirti ir atiduos pagonims, 
\par 34 tie tyčiosis iš Jo, apspjaudys, nuplaks ir nužudys, ir trečią dieną Jis prisikels”. 
\par 35 Prie Jėzaus priėjo Zebediejaus sūnūs Jokūbas ir Jonas ir kreipėsi: “Mokytojau, mes norime, kad padarytum mums viską, ko tik prašysime”. 
\par 36 Jis atsakė: “Ko norite, kad jums padaryčiau?” 
\par 37 Jie tarė: “Leisk mums sėdėti vienam Tavo dešinėje, kitam­kairėje, Tavo šlovėje!” 
\par 38 Jėzus atsakė: “Nežinote, ko prašote. Ar galite gerti taurę, kurią Aš geriu, ir būti pakrikštyti krikštu, kuriuo Aš krikštijamas?” 
\par 39 Jie sakė: “Galime”. Jėzus jiems tarė: “Tiesa, taurę, kurią Aš geriu, jūs gersite, ir krikštu, kuriuo Aš krikštijamas, jūs irgi būsite pakrikštyti. 
\par 40 Bet vietą savo dešinėje ar kairėje ne Aš duodu,­tai bus tiems, kuriems paskirta”. 
\par 41 Tai išgirdę, kiti dešimt labai supyko ant Jokūbo ir Jono. 
\par 42 O Jėzus, pasikvietęs juos, tarė: “Jūs žinote, kad tie, kurie laikomi pagonių valdovais, viešpatauja jiems, ir jų didieji juos valdo. 
\par 43 Bet tarp jūsų taip neturi būti. Kas iš jūsų nori būti didžiausias, bus jūsų tarnas, 
\par 44 ir kas nori tarp jūsų būti pirmas, bus visų vergas. 
\par 45 Juk ir Žmogaus Sūnus atėjo, ne kad Jam tarnautų, bet pats tarnauti ir savo gyvybės atiduoti kaip išpirkos už daugelį”. 
\par 46 Jie atėjo į Jerichą. O iškeliaujant Jam su mokiniais ir gausia minia iš Jericho, neregys Bartimiejus, Timiejaus sūnus, sėdėjo šalikelėje elgetaudamas. 
\par 47 Išgirdęs, jog čia Jėzus iš Nazareto, jis pradėjo garsiai šaukti: “Jėzau, Dovydo Sūnau, pasigailėk manęs!” 
\par 48 Daugelis jį draudė, kad nutiltų, bet jis dar garsiau šaukė: “Dovydo Sūnau, pasigailėk manęs!” 
\par 49 Jėzus sustojo ir tarė: “Pašaukite jį”. Žmonės pašaukė neregį, sakydami: “Pasitikėk! Kelkis, Jis tave šaukia”. 
\par 50 Tas, nusimetęs apsiaustą, pašoko ir atėjo prie Jėzaus. 
\par 51 Jėzus jo paklausė: “Ko nori, kad tau padaryčiau?” Neregys atsakė: “Rabuni, kad praregėčiau!” 
\par 52 Tada Jėzus jam tarė: “Eik, tavo tikėjimas išgydė tave”. Jis tuoj pat praregėjo ir nusekė paskui Jėzų keliu.



\chapter{11}


\par 1 Prisiartinęs prie Jeruzalės pro Betfagę ir Betaniją, ties Alyvų kalnu, Jėzus pasiuntė du savo mokinius, 
\par 2 liepdamas: “Eikite į priešais esantį kaimą ir, vos įžengę į jį, rasite pririštą asilaitį, kuriuo dar niekas iš žmonių nėra jojęs. Atriškite jį ir atveskite”. 
\par 3 Jeigu kas nors paklaustų: ‘Ką čia darote?’, atsakykite: ‘Jo reikia Viešpačiui’, ir iš karto jį paleis. 
\par 4 Nuėję jiedu rado pakelėje asilaitį, pririštą prie vartų, ir atrišo jį. 
\par 5 Kai kurie iš ten stovinčių jų paklausė: “Ką darote, kam atrišate asilaitį?” 
\par 6 O jie atsakė taip, kaip Jėzus buvo jiems liepęs, ir tie leido jį vestis. 
\par 7 Jie atvedė asilaitį pas Jėzų, apdengė jį savo apsiaustais, ir Jėzus užsėdo ant jo. 
\par 8 Daugelis tiesė ant kelio savo drabužius, kiti kirto ir klojo ant kelio medžių šakas. 
\par 9 Iš priekio ir iš paskos einantys šaukė: “Osana! Palaimintas, kuris ateina Viešpaties vardu! 
\par 10 Palaiminta mūsų tėvo Dovydo karalystė, ateinanti Viešpaties vardu! Osana aukštybėse!” 
\par 11 Taip Jėzus įžengė į Jeruzalę ir į šventyklą. Viską apžiūrėjęs,­ kadangi buvo jau vakaro valanda,­Jis su dvylika išėjo į Betaniją. 
\par 12 Rytojaus dieną, jiems keliaujant iš Betanijos, Jėzus išalko. 
\par 13 Pamatęs iš tolo sulapojusį figmedį, Jis priėjo pažiūrėti, gal ką ant jo ras. Tačiau, atėjęs prie medžio, Jis nerado nieko, tiktai lapus, nes dar nebuvo figų metas. 
\par 14 Atsakydamas Jėzus tarė medžiui: “Tegul per amžius niekas nebevalgys tavo vaisiaus!” Jo mokiniai tai girdėjo. 
\par 15 Jie atėjo į Jeruzalę. Įėjęs į šventyklą, Jėzus ėmė varyti laukan parduodančius ir perkančius šventykloje. Jis išvartė pinigų keitėjų stalus bei karvelių pardavėjų suolus 
\par 16 ir neleido nešti prekių per šventyklą. 
\par 17 Jis mokė, sakydamas jiems: “Argi neparašyta: ‘Mano namai vadinsis maldos namai visoms tautoms’? O jūs pavertėte juos ‘plėšikų lindyne’ ”. 
\par 18 Tai išgirdę, aukštieji kunigai ir Rašto žinovai tarėsi, kaip Jį pražudyti. Jie mat bijojo Jėzaus, nes visi žmonės buvo didžiai nustebinti Jo mokymo. 
\par 19 Atėjus vakarui, Jėzus su mokiniais išėjo iš miesto. 
\par 20 Rytą eidami pro šalį, jie pamatė, kad figmedis nudžiūvęs iš pat šaknų. 
\par 21 Prisiminęs Petras tarė Jėzui: “Rabi, žiūrėk! Figmedis, kurį prakeikei, nudžiūvo!” 
\par 22 Jėzus, atsakydamas jiems, tarė: “Turėkite tikėjimą Dievu! 
\par 23 Iš tiesų sakau jums: kas pasakytų šitam kalnui: ‘Pasikelk ir meskis į jūrą’, ir savo širdyje neabejotų, bet tikėtų, kad įvyks tai, ką sako,­jis turės, ką besakytų. 
\par 24 Todėl sakau jums: ko tik prašote melsdamiesi, tikėkite, kad gaunate, ir jūs turėsite. 
\par 25 Kai stovite melsdamiesi, atleiskite, jei turite ką nors prieš kitus, kad ir jūsų Tėvas, kuris danguje, galėtų jums atleisti jūsų nusižengimus. 
\par 26 Bet jeigu jūs neatleisite, nė jūsų Tėvas, kuris danguje, neatleis jūsų nusižengimų”. 
\par 27 Jie vėl sugrįžo į Jeruzalę. Jam vaikščiojant po šventyklą, prie Jo priėjo aukštųjų kunigų, Rašto žinovų bei vyresniųjų 
\par 28 ir klausė: “Kokią teisę turi taip daryti? Kas Tau davė valdžią tai daryti?” 
\par 29 Jėzus jiems atsakė: “Aš irgi paklausiu jus vieno dalyko, o jūs atsakykite man, tada ir Aš jums pasakysiu, kokia valdžia tai darau. 
\par 30 Jono krikštas buvo iš dangaus ar iš žmonių? Atsakykite man!” 
\par 31 Jie samprotavo tarpusavy: “Jei pasakysime­iš dangaus, Jis mums sakys: ‘Tai kodėl juo netikėjote?’ 
\par 32 O jei pasakysime­iš žmonių...” Jie bijojo minios, nes visi buvo įsitikinę, kad Jonas tikrai buvo pranašas. 
\par 33 Todėl jie atsakė Jėzui: “Mes nežinome”. Tada Jėzus tarė: “Tai ir Aš jums nesakysiu, kokia valdžia tai darau”.



\chapter{12}


\par 1 Jėzus pradėjo kalbėti jiems palyginimais: “Vienas žmogus pasodino vynuogyną, aptvėrė jį tvora, įrengė spaustuvą, pastatė bokštą, išnuomojo vynininkams ir iškeliavo į tolimą šalį. 
\par 2 Atėjus metui, jis nusiuntė pas vynininkus tarną atsiimti iš vynininkų savo vaisių. 
\par 3 Tie pačiupo jį, sumušė ir paleido tuščiomis. 
\par 4 Tada jis vėl nusiuntė pas juos kitą tarną, o tie, apmėtę akmenimis, sužeidė jį į galvą ir išniekinę paleido. 
\par 5 Jis pasiuntė dar vieną, bet tą jie nužudė; ir dar daugelį kitų tarnų, kurių vienus jie primušė, kitus nužudė. 
\par 6 Dar vieną turėjo­mylimąjį sūnų. Jį nusiuntė pas juos paskutinį, sakydamas sau: ‘Jie gerbs mano sūnų’. 
\par 7 Bet vynininkai tarėsi: ‘Tai paveldėtojas. Eikime, užmuškime jį, ir jo palikimas bus mūsų’. 
\par 8 Nutvėrę nužudė jį ir išmetė laukan iš vynuogyno. 
\par 9 Ką tada darys vynuogyno šeimininkas? Jis ateis, nužudys vynininkus ir atiduos vynuogyną kitiems. 
\par 10 Ar neskaitėte, kas parašyta Raštuose: ‘Akmuo, kurį statytojai atmetė, tapo kertiniu akmeniu. 
\par 11 Tai Viešpaties padaryta ir nuostabu mūsų akyse’?” 
\par 12 Anie norėjo Jį suimti, tačiau bijojo minios. Mat suprato, kad palyginimas buvo jiems taikomas. Tad, palikę Jį, pasitraukė. 
\par 13 Tada jie siunčia pas Jėzų fariziejų ir erodininkų sugauti Jo kalboje. 
\par 14 Šitie atėję sako Jam: “Mokytojau, žinome, jog Tu esi tiesus ir niekam nepataikauji. Tu neatsižvelgi į asmenis ir mokai Dievo kelio, kaip reikalauja tiesa. Ar reikia mokėti ciesoriui mokesčius, ar ne? 
\par 15 Mokėti ar nemokėti?” Žinodamas jų veidmainystę, Jis tarė: “Kam spendžiate man pinkles? Atneškite man pažiūrėti denarą”. 
\par 16 Jie padavė. Jis jų klausia: “Kieno čia atvaizdas ir įrašas?” Jie atsakė: “Ciesoriaus”. 
\par 17 Tuomet Jėzus jiems tarė: “Kas ciesoriaus, atiduokite ciesoriui, o kas Dievo­Dievui”. Ir jie labai Juo stebėjosi. 
\par 18 Pas Jėzų ateina sadukiejų, kurie neigia mirusiųjų prisikėlimą, ir klausia: 
\par 19 “Mokytojau, Mozė mums parašė: ‘Jei kieno brolis mirtų ir paliktų žmoną, o nepaliktų vaikų, tuomet jo brolis tegul veda našlę ir pažadina savo broliui palikuonių’. 
\par 20 Ir štai buvo septyni broliai. Pirmasis vedė žmoną ir mirdamas nepaliko vaikų. 
\par 21 Vedė ją antrasis, bet ir šis mirė bevaikis. Taip atsitiko ir su trečiuoju, 
\par 22 ir visi septyni nepaliko vaikų. Po jų visų numirė ir ta moteris. 
\par 23 Taigi prisikėlime, kai jie prisikels, kurio iš jų žmona ji bus? Juk ji buvo visų septynių žmona”. 
\par 24 Jėzus jiems atsakė: “Argi ne todėl klystate, kad nepažįstate nei Raštų, nei Dievo jėgos? 
\par 25 Kai prisikels iš numirusiųjų, jie nei ves, nei tekės, bet bus kaip angelai danguje. 
\par 26 O kad mirusieji keliasi,­ar neskaitėte Mozės knygoje, kaip Mozei Dievas pasakė iš krūmo: ‘Aš esu Abraomo Dievas, Izaoko Dievas ir Jokūbo Dievas!’? 
\par 27 Jis nėra mirusiųjų Dievas, bet gyvųjų Dievas. Taigi jūs labai klystate”. 
\par 28 Vienas iš Rašto žinovų, girdėjęs juos besiginčijant ir supratęs, kaip puikiai Jėzus jiems atsakinėjo, priėjo ir paklausė Jį: “Koks yra visų pirmasis įsakymas?” 
\par 29 Jėzus jam atsakė: “Pirmasis yra šis: ‘Klausyk, Izraeli,­Viešpats, mūsų Dievas, yra vienintelis Viešpats; 
\par 30 tad mylėk Viešpatį, savo Dievą, visa savo širdimi, visa savo siela, visu savo protu ir visomis savo jėgomis’,­tai pirmasis įsakymas. 
\par 31 Antrasis panašus į jį: ‘Mylėk savo artimą kaip save patį’. Nėra jokio kito įsakymo, didesnio už šiuodu”. 
\par 32 Tada Rašto žinovas Jam atsakė: “Gerai, Mokytojau, Tu tiesą pasakei: yra vienas Dievas ir nėra kito, tik Jis; 
\par 33 o mylėti Jį visa širdimi, visu protu ir visomis jėgomis bei mylėti savo artimą kaip save patį yra daugiau negu visos deginamosios atnašos ir aukos”. 
\par 34 Matydamas, kaip išmintingai jis atsakė, Jėzus jam tarė: “Tu netoli nuo Dievo karalystės!” Ir niekas daugiau nebedrįso Jo klausti. 
\par 35 Mokydamas šventykloje, Jėzus kalbėjo: “Kaip Rašto žinovai gali sakyti, jog Kristus yra Dovydo Sūnus? 
\par 36 Juk pats Dovydas Šventąja Dvasia pasakė: ‘Viešpats tarė mano Viešpačiui: sėskis mano dešinėje, kol patiesiu Tavo priešus tarsi pakojį po Tavo kojų’. 
\par 37 Pats Dovydas vadina jį Viešpačiu, tai kaip Jis gali būti jo Sūnus?” Didelė minia džiugiai Jo klausėsi. 
\par 38 Mokydamas Jis kalbėjo: “Saugokitės Rašto žinovų, kurie mėgsta vaikščioti su ilgais drabužiais ir būti sveikinami aikštėse, 
\par 39 užimti pirmuosius krėslus sinagogose ir garbingas vietas pokyliuose. 
\par 40 Jie suryja našlių namus ir dedasi kalbą ilgas maldas. Jie gaus dar didesnį pasmerkimą”. 
\par 41 Atsisėdęs ties iždine, Jėzus stebėjo, kaip žmonės metė į ją pinigus. Daugelis turtingųjų aukojo gausiai. 
\par 42 Atėjo viena beturtė našlė ir įmetė du pinigėlius, tai yra skatiką. 
\par 43 Pasišaukęs savo mokinius, Jėzus tarė jiems: “Iš tiesų sakau jums: ši beturtė našlė įmetė daugiausia iš visų, kurie dėjo į iždinę. 
\par 44 Visi aukojo iš savo pertekliaus, o ji iš savo nepritekliaus įmetė visa, ką turėjo, visą savo pragyvenimą”.



\chapter{13}


\par 1 Jam išeinant iš šventyklos, vienas iš mokinių Jam sako: “Mokytojau, pažvelk, kokie akmenys ir kokie pastatai!” 
\par 2 Jėzus jam atsakė: “Matai šituos didžiulius pastatus? Čia neliks akmens ant akmens, viskas bus išgriauta”. 
\par 3 Kai Jis sėdėjo Alyvų kalne, priešais šventyklą, Petras, Jokūbas, Jonas ir Andriejus atskirai nuo kitų klausė Jį: 
\par 4 “Pasakyk mums, kada tai įvyks ir koks bus ženklas, kai visa tai pradės pildytis?” 
\par 5 Jėzus, jiems atsakydamas, pradėjo kalbėti: “Žiūrėkite, kad niekas jūsų nesuklaidintų. 
\par 6 Daug kas ateis mano vardu ir sakys: ‘Tai Aš’, ir daugelį suklaidins. 
\par 7 Išgirdę apie karus ir karų gandus, neišsigąskite. Tai turi įvykti, bet dar ne galas. 
\par 8 Tauta sukils prieš tautą ir karalystė prieš karalystę. Įvairiose vietose bus žemės drebėjimų, bus badmečių ir neramumų. Tai gimdymo skausmų pradžia. 
\par 9 Jūs saugokitės, nes atidavinės jus teismams, plaks sinagogose, ir jūs turėsite dėl manęs stoti prieš valdytojus ir karalius jiems liudyti. 
\par 10 Ir Evangelija pirmiau turės būti paskelbta visoms tautoms. 
\par 11 Kai suėmę jus ves, nesirūpinkite ir negalvokite iš anksto, ką kalbėsite. Kalbėkite tai, kas tą valandą bus jums duota, nes kalbėsite ne jūs, o Šventoji Dvasia. 
\par 12 Brolis išduos nužudyti brolį, o tėvas­savo vaiką. Vaikai pakels ranką prieš savo gimdytojus ir juos žudys. 
\par 13 Jūs būsite visų nekenčiami dėl mano vardo. Bet kas ištvers iki galo, tas bus išgelbėtas”. 
\par 14 “Kai pamatysite per pranašą Danielių paskelbtą naikinimo bjaurastį, stovinčią ten, kur jos neturi būti (kas skaito­teišmano), tada, kas bus Judėjoje, tebėga į kalnus; 
\par 15 kas bus ant stogo, tenelipa žemyn į namus ir tegul neina ko nors pasiimti iš savo namų; 
\par 16 o kas laukuose, tenegrįžta pasiimti apsiausto. 
\par 17 Vargas nėščioms ir žindančioms tomis dienomis! 
\par 18 Melskitės, kad jums netektų bėgti žiemą! 
\par 19 Tomis dienomis bus toks suspaudimas, kokio nėra buvę nuo pradžios pasaulio, kurį Dievas sutvėrė, iki šiol, ir daugiau nebebus. 
\par 20 Ir, jeigu Viešpats nebūtų sutrumpinęs tų dienų, neišsigelbėtų nė vienas kūnas. Tačiau dėl išrinktųjų, kuriuos išsirinko, Jis sutrumpino tas dienas. 
\par 21 Jei tada kas nors jums sakys: ‘Štai čia Kristus’, arba: ‘Jis tenai!’,­netikėkite, 
\par 22 nes atsiras netikrų kristų ir netikrų pranašų. Jie darys ženklų ir stebuklų, kad suklaidintų, jei įmanoma, net išrinktuosius. 
\par 23 Todėl būkite atidūs; štai Aš jums iš anksto visa tai pasakiau”. 
\par 24 “Tomis dienomis, po ano suspaudimo, saulė užtems, mėnulis nebeduos šviesos, 
\par 25 dangaus žvaigždės kris ir dangaus jėgos bus sudrebintos. 
\par 26 Tada jie pamatys Žmogaus Sūnų, ateinantį debesyse su didžia jėga ir šlove. 
\par 27 Jis pasiųs savo angelus, ir tie surinks Jo išrinktuosius iš keturių žemės pusių, nuo žemės pakraščių iki dangaus tolybių. 
\par 28 Pasimokykite iš palyginimo su figmedžiu: kai jo šaka suminkštėja ir sprogsta lapai, žinote, jog artėja vasara. 
\par 29 Taip pat jūs, išvydę visa tai dedantis, žinokite, jog Jis jau arti, prie durų. 
\par 30 Iš tiesų sakau jums: ši karta nepraeis, iki visa tai įvyks. 
\par 31 Dangus ir žemė praeis, o mano žodžiai nepraeis. 
\par 32 Tačiau tos dienos ir valandos niekas nežino, nei angelai danguje, nei Sūnus, tik Tėvas”. 
\par 33 “Žiūrėkite, budėkite ir melskitės, nes nežinote, kada ateis laikas! 
\par 34 Bus kaip su žmogumi, kuris iškeliavo toli, paliko namus, suteikė tarnams valdžią, kiekvienam paskyrė darbą, o durininkui įsakė budėti. 
\par 35 Taigi budėkite, nes nežinote, kada grįš namų šeimininkas: ar vakare, ar vidurnaktyje, ar gaidžiui giedant, ar rytmety, 
\par 36 kad, netikėtai sugrįžęs, nerastų jūsų miegančių. 
\par 37 Ką sakau jums, sakau ir visiems: budėkite!”



\chapter{14}


\par 1 Iki Paschos ir Neraugintos duonos šventės buvo likę dvi dienos. Aukštieji kunigai ir Rašto žinovai ieškojo būdo klasta suimti Jėzų ir nužudyti. 
\par 2 Bet jie sakė: “Tik ne per šventes, kad žmonėse nekiltų sąmyšio”. 
\par 3 Jėzui esant Betanijoje, Simono Raupsuotojo namuose, ir sėdint prie stalo, atėjo moteris su alebastriniu labai brangaus gryno nardo tepalo indu. Sudaužiusi indą, ji išpylė tepalą Jam ant galvos. 
\par 4 Kai kurie ten esantys pasipiktino ir kalbėjo vienas kitam: “Kam toks tepalo eikvojimas? 
\par 5 Juk jį buvo galima parduoti daugiau negu už tris šimtus denarų ir pinigus išdalyti vargšams!” Ir jie murmėjo prieš tą moterį. 
\par 6 Bet Jėzus atsiliepė: “Palikite ją ramybėje! Kam ją skaudinate? Ji man padarė gerą darbą. 
\par 7 Vargšų jūs visuomet turite su savimi ir, kada tik panorėję, galėsite jiems gera daryti, o mane ne visuomet turėsite. 
\par 8 Ji padarė, ką galėjo. Ji iš anksto patepė mano kūną laidotuvėms. 
\par 9 Iš tiesų sakau jums: visame pasaulyje, kur tik bus skelbiama ši Evangelija, jos atminimui bus pasakojama ir tai, ką ji padarė”. 
\par 10 Judas Iskarijotas, vienas iš dvylikos, nuėjo pas aukštuosius kunigus išduoti Jėzų. 
\par 11 Tai išgirdę, jie apsidžiaugė ir pažadėjo jam pinigų. Jis ėmė ieškoti progos Jėzų išduoti. 
\par 12 Pirmąją Neraugintos duonos dieną, kada pjaunamas Paschos avinėlis, mokiniai klausė Jėzų: “Kur nori, kad Tau paruoštume valgyti Paschą?” 
\par 13 Jis pasiunčia du mokinius, tardamas: “Eikite į miestą. Ten jus sutiks žmogus, vandens ąsočiu nešinas. Sekite paskui jį 
\par 14 ir, kur jis įeis, sakykite namų šeimininkui: ‘Mokytojas liepė paklausti: kur yra svečių kambarys, kuriame su savo mokiniais galėčiau valgyti Paschą?’ 
\par 15 Jis parodys jums didelį apstatytą aukštutinį kambarį. Ten ir paruoškite mums”. 
\par 16 Mokiniai išėjo ir nuvyko į miestą. Jie rado visa, kaip Jis sakė, ir paruošė Paschą. 
\par 17 Vakare Jis atėjo su dvylika. 
\par 18 Jiems sėdint už stalo ir valgant, Jėzus tarė: “Iš tiesų sakau jums: vienas iš jūsų, valgančių su manimi, išduos mane”. 
\par 19 Jie labai nuliūdo ir vienas paskui kitą ėmė Jo klausinėti: “Nejaugi aš?”, “Nejaugi aš?” 
\par 20 O Jis jiems tarė: “Vienas iš dvylikos, kuris dažo su manimi dubenyje. 
\par 21 Žmogaus Sūnus, tiesa, eina, kaip apie Jį parašyta, bet vargas tam žmogui, per kurį Žmogaus Sūnus išduodamas. Geriau būtų buvę tam žmogui negimti”. 
\par 22 Jiems bevalgant, Jėzus paėmė duoną, palaimino, laužė ir davė mokiniams, sakydamas: “Imkite ir valgykite: tai yra mano kūnas!” 
\par 23 Po to paėmė taurę, padėkojo, davė jiems, ir visi gėrė iš jos. 
\par 24 Jis jiems tarė: “Tai yra mano kraujas, Naujosios Sandoros kraujas, kuris išliejamas už daugelį. 
\par 25 Iš tiesų sakau jums: Aš daugiau nebegersiu vynmedžio vaisiaus iki tos dienos, kada gersiu jį naują Dievo karalystėje”. 
\par 26 Pagiedoję himną, jie išėjo į Alyvų kalną. 
\par 27 Jėzus jiems tarė: “Šią naktį jūs visi manimi pasipiktinsite, nes parašyta: ‘Ištiksiu piemenį, ir avys išsisklaidys’. 
\par 28 Bet prisikėlęs Aš pirma jūsų nueisiu į Galilėją”. 
\par 29 Petras atsiliepė: “Jei ir visi pasipiktintų, tai tik ne aš!” 
\par 30 Jėzus jam atsakė: “Iš tiesų sakau tau: dar šiandien, jau šią naktį, gaidžiui nė dukart nepragydus, tu tris kartus manęs išsiginsi”. 
\par 31 Bet Petras dar atkakliau tvirtino: “Jei man reikėtų net mirti su Tavimi, aš vis tiek Tavęs neišsiginsiu”. Tą patį kalbėjo ir visi kiti. 
\par 32 Jie atėjo į vietą, vadinamą Getsemane. Jėzus sako savo mokiniams: “Pasėdėkite čia, kol Aš melsiuosi”. 
\par 33 Pasiėmęs su savimi Petrą, Jokūbą ir Joną, Jis pradėjo nuogąstauti ir sielvartauti. 
\par 34 Jis jiems sakė: “Mano siela mirtinai nuliūdusi. Pasilikite čia ir budėkite!” 
\par 35 Paėjęs truputį toliau, sukniubo ant žemės ir meldėsi, kad, jei įmanoma, Jį aplenktų toji valanda. 
\par 36 Jis sakė: “Aba, Tėve, Tau viskas įmanoma. Atitolink nuo manęs šitą taurę! Tačiau ne kaip Aš noriu, bet kaip Tu”. 
\par 37 Po to grįžta, randa juos miegančius ir taria Petrui: “Simonai, tu miegi? Nepajėgei nė vienos valandos pabudėti? 
\par 38 Budėkite ir melskitės, kad nepatektumėte į pagundymą, nes dvasia ryžtinga, bet kūnas silpnas”. 
\par 39 Jis vėl nuėjo ir meldėsi tais pačiais žodžiais. 
\par 40 Sugrįžęs Jis vėl rado juos miegančius­jų akys buvo apsunkusios, ir jie nežinojo, ką atsakyti. 
\par 41 Jis atėjo trečią kartą ir tarė jiems: “Vis dar tebemiegate ir ilsitės? Gana! Atėjo valanda: štai Žmogaus Sūnus išduodamas į nusidėjėlių rankas. 
\par 42 Kelkitės, eime! Štai mano išdavėjas čia pat”. 
\par 43 Ir tuojau, dar Jam tebekalbant, pasirodė vienas iš dvylikos­Judas, o kartu su juo didelė minia, ginkluota kalavijais ir vėzdais, pasiųsta aukštųjų kunigų, Rašto žinovų ir vyresniųjų. 
\par 44 Išdavėjas buvo jiems nurodęs ženklą: “Kurį pabučiuosiu, tai Tas. Suimkite Jį ir veskite saugodami!” 
\par 45 Atėjęs jis tuojau prisiartino prie Jėzaus ir tarė: “Rabi!”, ir pabučiavo Jį. 
\par 46 O kiti čiupo Jėzų rankomis ir suėmė. 
\par 47 Vienas iš ten stovinčiųjų, išsitraukęs kalaviją, smogė vyriausiojo kunigo tarnui ir nukirto jam ausį. 
\par 48 O Jėzus jiems tarė: “Kaip prieš plėšiką išėjote su kalavijais ir vėzdais suimti manęs. 
\par 49 Aš kasdien buvau su jumis šventykloje ir mokiau, ir jūs manęs nesuėmėte. Tačiau turi išsipildyti Raštai”. 
\par 50 Tada visi paliko Jį ir pabėgo. 
\par 51 Vienas jaunuolis sekė Jį iš paskos, susisupęs vien į drobulę. Jie čiupo jį, 
\par 52 bet šis išsinėrė iš drobulės ir nuogas pabėgo. 
\par 53 Jėzų nuvedė pas vyriausiąjį kunigą, kur buvo susirinkę visi aukštieji kunigai, vyresnieji ir Rašto žinovai. 
\par 54 Petras sekė Jį iš tolo iki vyriausiojo kunigo rūmų kiemo. Ten jis atsisėdo su tarnais ir šildėsi prie ugnies. 
\par 55 Aukštieji kunigai ir visas sinedrionas ieškojo prieš Jėzų liudijimo, kad galėtų nuteisti Jį mirti, bet nerado. 
\par 56 Nors daugelis melagingai liudijo prieš Jį, tačiau jų liudijimai nesutapo. 
\par 57 Kai kurie atsistoję melagingai kaltino Jį, teigdami: 
\par 58 “Mes girdėjome Jį sakant: ‘Aš sugriausiu šitą rankomis pastatytą šventyklą ir per tris dienas pastatysiu kitą, ne rankų darbo’ ”. 
\par 59 Bet ir šie kaltinimai nesutapo. 
\par 60 Tada vyriausiasis kunigas, atsistojęs viduryje, paklausė Jėzų: “Tu nieko neatsakai į šituos kaltinimus?” 
\par 61 Tačiau Jis tylėjo ir nieko neatsakė. Tada vyriausiasis kunigas vėl Jį paklausė: “Ar Tu esi Kristus, Palaimintojo Sūnus?!” 
\par 62 Ir Jėzus pasakė: “Aš Esu. Ir jūs išvysite Žmogaus Sūnų, sėdintį Galybės dešinėje ir ateinantį dangaus debesyse”. 
\par 63 Tada vyriausiasis kunigas persiplėšė drabužius ir sušuko: “Kam dar mums liudytojai? 
\par 64 Jūs girdėjote piktžodžiavimą! Kaip jums atrodo?” Ir jie visi nusprendė Jį esant vertą mirties. 
\par 65 Kai kurie pradėjo į Jį spjaudyti, dangstė Jam veidą, mušė kumščiais ir sakė: “Pranašauk!” O tarnai daužė Jį per veidą. 
\par 66 Petrui esant žemai, kieme, atėjo viena vyriausiojo kunigo tarnaitė 
\par 67 ir, pamačiusi besišildantį Petrą, įsižiūrėjo į jį ir tarė: “Ir tu buvai su šituo Nazariečiu Jėzumi”. 
\par 68 Bet Petras išsigynė, sakydamas: “Nei žinau, nei suprantu, ką sakai”. Jis išėjo į prieškiemį, ir pragydo gaidys. 
\par 69 Pamačiusi jį, tarnaitė vėl pradėjo sakyti aplink stovėjusiems: “Šitas yra iš jų!” 
\par 70 Jis vėl išsigynė. Kiek vėliau šalia stovintieji sakė Petrui: “Tu tikrai vienas iš jų, juk tu irgi galilėjietis, ir tarmė tavo tokia”. 
\par 71 Tada jis pradėjo keiktis ir prisiekinėti: “Aš nepažįstu to žmogaus, apie kurį jūs kalbate!” 
\par 72 Gaidys pragydo antrą kartą. Petras atsiminė, ką jam sakė Jėzus: “Gaidžiui nė dukart nepragydus, tu tris kartus manęs išsiginsi”. Tai prisiminęs, jis pravirko.



\chapter{15}


\par 1 Tuojau iš ryto aukštieji kunigai pasitarė su vyresniaisiais ir Rašto žinovais bei visu sinedrionu ir, surišę Jėzų, jie Jį nuvedė ir perdavė Pilotui. 
\par 2 Pilotas paklausė Jį: “Ar Tu esi žydų karalius?” Jis atsakė: “Taip yra, kaip sakai”. 
\par 3 Aukštieji kunigai Jį daug kuo kaltino, bet Jis nieko neatsakinėjo. 
\par 4 Pilotas vėl klausė Jį: “Tu nieko neatsakai? Žiūrėk, kiek daug kaltinimų jie Tau pateikia”. 
\par 5 Tačiau Jėzus nieko nebeatsakinėjo, ir Pilotas labai stebėjosi. 
\par 6 Per šventes jis paleisdavo vieną kalinį, kurio žmonės prašydavo. 
\par 7 Tada buvo vienas kalinys, vardu Barabas, suimtas kartu su maištininkais, kurie maišto metu nužudė žmogų. 
\par 8 Susirinkusi minia, garsiai šaukdama, pradėjo prašyti to, ką Pilotas visuomet darydavo. 
\par 9 Pilotas paklausė: “Ar norite, kad jums paleisčiau žydų karalių?” 
\par 10 Nes jis žinojo, kad aukštieji kunigai Jį įskundė iš pavydo. 
\par 11 Tačiau aukštieji kunigai sukurstė minią reikalauti, kad geriau paleistų Barabą. 
\par 12 Tada Pilotas vėl kreipėsi į juos: “O kaip jūs norite, kad aš pasielgčiau su Tuo, kurį vadinate žydų karaliumi?” 
\par 13 Tie šaukė: “Nukryžiuok Jį!” 
\par 14 Pilotas jų klausė: “Ką bloga Jis padarė?” Tada jie pradėjo dar garsiau rėkti: “Nukryžiuok Jį!” 
\par 15 Norėdamas įtikti miniai, Pilotas paleido Barabą, o Jėzų nuplakdino ir atidavė nukryžiuoti. 
\par 16 Kareiviai nusivedė Jį į rūmų kiemą, tai yra pretorijų, ir ten sušaukė visą kuopą. 
\par 17 Jie apvilko Jį purpuriniu apsiaustu, nupynę uždėjo Jam erškėčių vainiką 
\par 18 ir pradėjo Jį sveikinti: “Sveikas, žydų karaliau!” 
\par 19 Jie daužė Jam per galvą nendrine lazda, spjaudė ir priklaupdami garbino Jį. 
\par 20 Pasityčioję nuvilko Jam purpurinį apsiaustą, apvilko Jo paties drabužiais ir išvedė nukryžiuoti. 
\par 21 Jie privertė vieną grįžtantį iš lauko praeivį­Simoną Kirėnietį, Aleksandro ir Rufo tėvą,­nešti Jo kryžių. 
\par 22 Ir jie nuvedė Jį į Golgotos vietą; išvertus tai reiškia: “Kaukolės vieta”. 
\par 23 Ten davė Jam mira atmiešto vyno, bet Jis negėrė. 
\par 24 Nukryžiavę Jį, jie pasidalijo Jo drabužius, mesdami burtą, kas kuriam turi tekti. 
\par 25 Buvo trečia valanda, kai Jį nukryžiavo. 
\par 26 Taip pat buvo užrašytas Jo kaltinimas: “Žydų karalius”. 
\par 27 Kartu su Juo nukryžiavo du plėšikus: vieną dešinėje, kitą kairėje. 
\par 28 Taip išsipildė Rašto žodis: “Jis buvo priskaitytas prie piktadarių”. 
\par 29 Einantys pro šalį plūdo Jėzų, kraipydami galvas ir sakydami: “Še Tau, kuris sugriauni šventyklą ir per tris dienas ją atstatai. 
\par 30 Išgelbėk save, nuženk nuo kryžiaus!” 
\par 31 Panašiai tyčiojosi ir aukštieji kunigai su Rašto žinovais, kalbėdami tarp savęs: “Kitus gelbėdavo, o savęs negali išgelbėti. 
\par 32 Tegul Kristus, Izraelio karalius, dabar nužengia nuo kryžiaus, kad pamatytume ir įtikėtume”. Kartu nukryžiuotieji irgi užgauliojo Jį. 
\par 33 Šeštai valandai atėjus, visą kraštą apgaubė tamsa iki devintos valandos. 
\par 34 Devintą valandą Jėzus garsiu balsu sušuko: “Elojí, Elojí, lemá sabachtáni?” Tai reiškia: “Mano Dieve, mano Dieve, kodėl mane palikai?!” 
\par 35 Kai kurie ten stovintys išgirdę sakė: “Žiūrėk, Jis šaukiasi Elijo”. 
\par 36 Tada vienas nubėgęs primirkė kempinę rūgštaus vyno, užmovė ją ant nendrės ir davė Jam gerti, sakydamas: “Palaukite, pažiūrėsime, ar ateis Elijas Jo nuimti”. 
\par 37 Bet Jėzus, garsiai sušukęs, atidavė dvasią. 
\par 38 Ir šventyklos uždanga perplyšo pusiau nuo viršaus iki apačios. 
\par 39 Šimtininkas, stovėjęs priešais ir matęs, kaip Jis šaukdamas mirė, tarė: “Tikrai šitas žmogus buvo Dievo Sūnus!” 
\par 40 Ten taip pat buvo moterų, kurios žiūrėjo iš toli; tarp jų ir Marija Magdalietė, Marija­Jokūbo Jaunesniojo ir Jozės motina­ir Salomė. 
\par 41 Kai Jėzus dar buvo Galilėjoje, jos Jį lydėjo ir Jam tarnavo. Ten buvo ir daug kitų moterų, kartu su Juo atvykusių į Jeruzalę. 
\par 42 Vakarui atėjus, kadangi buvo Prisirengimas,­sabato išvakarės,­ 
\par 43 atvyko Juozapas iš Arimatėjos, garbingas teismo tarybos narys, kuris irgi laukė Dievo karalystės. Jis drąsiai nuėjo pas Pilotą ir paprašė Jėzaus kūno. 
\par 44 Pilotas nustebo, argi jau būtų miręs? Jis pasišaukė šimtininką ir paklausė, ar Jėzus jau miręs. 
\par 45 Sužinojęs tai iš šimtininko, jis atidavė Juozapui kūną. 
\par 46 Šis nupirko drobulę, nuėmė Jėzų nuo kryžiaus, įvyniojo į drobulę, paguldė Jį kape, kuris buvo iškaltas uoloje, ir užritino angą akmeniu. 
\par 47 Marija Magdalena ir Marija, Jozės motina, matė, kur Jis buvo palaidotas.



\chapter{16}


\par 1 Sabatui praėjus, Marija Magdalietė, Marija, Jokūbo motina, ir Salomė nupirko kvepalų, kad nuėjusios galėtų Jėzų patepti. 
\par 2 Labai anksti, pirmąją savaitės dieną, saulei tekant, jos atėjo prie kapo 
\par 3 ir kalbėjosi tarp savęs: “Kas mums nuritins akmenį nuo kapo angos?” 
\par 4 Bet pažvelgusios pamatė, kad akmuo nuristas. O jis buvo labai didelis. 
\par 5 Įėjusios į kapo rūsį, išvydo dešinėje sėdintį jaunuolį ilgais baltais drabužiais ir nustėro. 
\par 6 Jis joms tarė: “Neišsigąskite! Jūs ieškote nukryžiuotojo Jėzaus Nazariečio. Jis prisikėlė, Jo čia nebėra. Štai vieta, kur Jį buvo paguldę. 
\par 7 Eikite, pasakykite Jo mokiniams ir Petrui: Jis eina pirma jūsų į Galilėją. Ten Jį pamatysite, kaip Jis yra jums sakęs”. 
\par 8 Jos skubiai išėjo ir nubėgo nuo kapo, nes drebėjo ir buvo apstulbusios. Persigandusios jos niekam nieko nesakė. 
\par 9 Prisikėlęs anksti rytą, pirmąją savaitės dieną, Jėzus pirmiausia pasirodė Marijai Magdalenai, iš kurios buvo išvaręs septynis demonus. 
\par 10 Ji nuėjusi pranešė Jo bičiuliams, kurie liūdėjo ir verkė. 
\par 11 Tie, išgirdę, kad Jis gyvas ir kad ji pati Jį mačiusi, netikėjo. 
\par 12 Po to Jis pasirodė dviem iš jų kelyje į kaimą, tačiau kitokiu pavidalu. 
\par 13 Ir šitie sugrįžę pranešė visiems kitiems, bet ir jais anie netikėjo. 
\par 14 Pagaliau Jėzus pasirodė visiems vienuolikai, kai jie sėdėjo už stalo. Jis barė juos už jų netikėjimą ir širdies kietumą, kad netikėjo tais, kurie buvo matę Jį prisikėlusį. 
\par 15 Jis tarė jiems: “Eikite į visą pasaulį ir skelbkite Evangeliją visai kūrinijai. 
\par 16 Kas įtikės ir krikštysis, bus išgelbėtas, o kas netikės, bus pasmerktas. 
\par 17 Ir kurie tikės, tuos lydės šie ženklai: mano vardu jie išvarinės demonus, kalbės naujomis kalbomis, 
\par 18 ims plikomis rankomis gyvates ir, jei išgertų mirtinų nuodų, jiems nepakenks. Jie dės rankas ant ligonių, ir tie pasveiks”. 
\par 19 Baigęs jiems kalbėti, Viešpats buvo paimtas į dangų ir atsisėdo Dievo dešinėje. 
\par 20 O jie ėjo ir visur pamokslavo, Viešpačiui drauge veikiant ir patvirtinant žodį lydinčiais stebuklais. Amen.



\end{document}