\begin{document}

\title{Apaštalų darbai}

\chapter{1}


\par 1 Pirmoje knygoje, Teofiliau, aš parašiau apie viską, ką Jėzus pradėjo daryti ir mokyti 
\par 2 iki tos dienos, kurią buvo paimtas į dangų, pirmiau per Šventąją Dvasią davęs savo išrinktiesiems apaštalams įsakymų. 
\par 3 Po savo kančios Jis pateikė jiems daugelį įrodymų, kad yra gyvas, per keturiasdešimt dienų jiems rodydamasis ir kalbėdamas apie Dievo karalystę. 
\par 4 Kartą, būdamas kartu su jais, liepė jiems nepasišalinti iš Jeruzalės, bet laukti Tėvo pažado,­“apie kurį,­pasakė Jis,­esate girdėję iš manęs; 
\par 5 nes Jonas krikštijo vandeniu, o jūs po kelių dienų būsite pakrikštyti Šventąja Dvasia”. 
\par 6 Susirinkusieji paklausė Jį: “Viešpatie, gal Tu šiuo metu atkursi Izraelio karalystę?” 
\par 7 Jis jiems tarė: “Ne jūsų reikalas žinoti laiką ir metą, kuriuos Tėvas nustatė savo valdžia. 
\par 8 Kai ant jūsų nužengs Šventoji Dvasia, jūs gausite jėgos ir tapsite mano liudytojais Jeruzalėje ir visoje Judėjoje bei Samarijoje ir lig pat žemės pakraščių”. 
\par 9 Tai pasakęs, jiems bežiūrint, Jėzus pakilo aukštyn, ir debesis Jį paslėpė nuo jų akių. 
\par 10 Kai jie, akių nenuleisdami, žiūrėjo į žengiantį dangun Jėzų, štai prie jų atsirado du vyrai baltais drabužiais 
\par 11 ir tarė: “Vyrai galilėjiečiai, ko stovite, žiūrėdami į dangų? Tas pats Jėzus, paimtas nuo jūsų į dangų, sugrįš taip pat, kaip Jį matėte žengiantį į dangų”. 
\par 12 Tuomet jie sugrįžo į Jeruzalę iš vadinamojo Alyvų kalno, esančio netoli Jeruzalės, tokiu atstumu, koks leidžiamas nueiti per sabatą. 
\par 13 Parėję jie susirinko aukštutiniame kambaryje, kur buvo apsistoję,­Petras ir Jokūbas, Jonas ir Andriejus, Pilypas ir Tomas, Baltramiejus ir Matas, Alfiejaus sūnus Jokūbas, Simonas Uolusis ir Judas, Jokūbo brolis. 
\par 14 Jie visi ištvermingai ir vieningai atsidėjo maldai ir prašymui kartu su moterimis ir Jėzaus motina Marija bei Jo broliais. 
\par 15 Vieną dieną, atsistojęs tarp brolių,­ten buvo susirinkę apie šimtą dvidešimt asmenų,­Petras tarė: 
\par 16 “Vyrai broliai, turėjo išsipildyti Rašto žodžiai, kuriuos Šventoji Dvasia pranašiškai išsakė Dovydo lūpomis apie Judą, tapusį Jėzaus suėmėjų vadovu. 
\par 17 Jis juk buvo priskaitytas prie mūsų ir turėjo dalį šioje tarnystėje. 
\par 18 Bet jis nedorybės kaina įsigijo sklypą, paskui stačia galva puolė žemyn, perplyšo pusiau, ir visi jo viduriai išvirto. 
\par 19 Tai pasidarė žinoma visiems Jeruzalės gyventojams, ir anas sklypas jų kalba buvo pavadintas Hakeldamachu,­tai reiškia: ‘Kraujo sklypas’. 
\par 20 Nes Psalmių knygoje parašyta: ‘Jo kiemas tepavirsta dykviete, ir tegul niekas ten nebegyvena’, ir: ‘Tegul kitas perima jo tarnystę’. 
\par 21 Taigi vienam iš vyrų, kurie drauge su mumis vaikščiojo visą laiką, kol Viešpats Jėzus buvo tarp mūsų, 
\par 22 pradedant Jono krikštu ir baigiant ta diena, kai Jis buvo paimtas iš mūsų aukštyn, reikia kartu su mumis tapti Jo prisikėlimo liudytoju”. 
\par 23 Ir jie išskyrė du: Juozapą, vadinamą Barsabu, pravarde Justas, ir Motiejų. 
\par 24 Po to meldėsi, sakydami: “Tu, Viešpatie, kuris pažįsti visų širdis, parodyk, kurį iš šių dviejų pasirenki, 
\par 25 kad gautų dalį šioje tarnystėje ir apaštalystėje, nuo kurios nuklydo Judas, nueidamas į savąją vietą”. 
\par 26 Tada metė burtus, ir burtas krito Motiejui, ir jis buvo priskaičiuotas prie vienuolikos apaštalų.


\chapter{2}


\par 1 Atėjus Sekminių dienai, visi mokiniai vieningai buvo vienoje vietoje. 
\par 2 Staiga iš dangaus pasigirdo ūžesys, tarsi pūstų smarkus vėjas. Jis pripildė visą namą, kur jie sėdėjo. 
\par 3 Jiems pasirodė tarsi ugnies liežuviai, kurie pasidaliję nusileido ant kiekvieno iš jų. 
\par 4 Visi pasidarė pilni Šventosios Dvasios ir pradėjo kalbėti kitomis kalbomis, kaip Dvasia jiems davė prabilti. 
\par 5 Jeruzalėje gyveno žydų ir dievobaimingų žmonių iš visų tautų po dangumi. 
\par 6 Kilus tam ūžesiui, subėgo daugybė žmonių. Jie buvo priblokšti, kiekvienas girdėdamas savo kalba juos kalbant. 
\par 7 Lyg nesavi ir nustebę, jie vienas kitam kalbėjo: “Argi štai šitie kalbantys nėra galilėjiečiai? 
\par 8 Tai kaipgi mes kiekvienas juos girdime savo gimtąja kalba?! 
\par 9 Mes, partai, medai, elamitai, Mesopotamijos, Judėjos ir Kapadokijos, Ponto ir Azijos, 
\par 10 Frygijos ir Pamfilijos, Egipto bei Libijos pakraščio ties Kirėne gyventojai, ateiviai romiečiai, žydai ir prozelitai, 
\par 11 kretiečiai ir arabai,­girdime juos skelbiant didingus Dievo darbus mūsų kalbomis”. 
\par 12 Visi labai stebėjosi ir, nieko nesuprasdami, klausinėjo: “Ką tai reiškia?” 
\par 13 O kiti šaipydamiesi kalbėjo: “Jie prisigėrė jauno vyno”. 
\par 14 Tada stojo Petras su vienuolika ir, pakėlęs balsą, prabilo į juos: “Jūs, vyrai judėjiečiai bei visi Jeruzalės gyventojai, tebūnie jums žinoma,­įsidėmėkite mano žodžius! 
\par 15 Šitie nėra, kaip jūs manote, prisigėrę,­juk dabar vos trečia dienos valanda,­ 
\par 16 bet čia išsipildė, kas buvo pasakyta per pranašą Joelį: 
\par 17 ‘Paskutinėmis dienomis,­sako Dievas,­Aš išliesiu savo Dvasios kiekvienam kūnui, ir jūsų sūnūs bei dukterys pranašaus, jūsų jaunuoliai matys regėjimus, o senieji sapnuos sapnus. 
\par 18 Ir savo tarnams bei tarnaitėms tomis dienomis Aš išliesiu savo Dvasios, ir jie pranašaus. 
\par 19 Aš darysiu stebuklų aukštai danguje ir apačioje, žemėje, parodysiu ženklų: kraujo, ugnies bei rūkstančių dūmų. 
\par 20 Saulė pavirs tamsybe, o mėnulis­krauju, prieš ateinant didingai ir šlovingai Viešpaties dienai. 
\par 21 Ir kiekvienas, kuris šauksis Viešpaties vardo, bus išgelbėtas’. 
\par 22 Izraelio vyrai! Klausykite šitų žodžių: Jėzų iš Nazareto­vyrą, Dievo jums patvirtintą galingais darbais, stebuklais ir ženklais, kuriuos per Jį nuveikė Dievas tarp jūsų,­kaip ir patys žinote,­ 
\par 23 Jį, išankstiniu Dievo sprendimu bei numatymu atiduotą, jūs piktadarių rankomis nužudėte, prikaldami prie kryžiaus. 
\par 24 Dievas Jį prikėlė, išvaduodamas iš mirties skausmų, nes buvo neįmanoma, kad Jis liktų jos galioje. 
\par 25 Juk Dovydas apie Jį sako: ‘Aš visuomet matau Viešpatį priešais save. Jis mano dešinėje, kad Aš nesusvyruočiau. 
\par 26 Todėl džiūgavo mano širdis, krykštavo mano lūpos, ir mano kūnas ilsėsis viltyje, 
\par 27 nes Tu nepaliksi mano sielos pragare ir neduosi savo Šventajam matyti supuvimo. 
\par 28 Tu man atvėrei gyvenimo kelius ir pripildysi mane džiaugsmu prieš savo veidą’. 
\par 29 Vyrai broliai! Noriu jums drąsiai pasakyti apie patriarchą Dovydą. Jis mirė, buvo palaidotas, ir jo kapas tebėra pas mus iki šios dienos. 
\par 30 Būdamas pranašas ir žinodamas, jog Dievas jam prisiekdamas pažadėjo, kad iš jo palikuonių pagal kūną pakils Kristus užimti jo sosto, 
\par 31 Dovydas tai numatė ir kalbėjo apie Kristaus prisikėlimą, kad ‘Jo siela neliks pragare ir Jo kūnas nematys supuvimo’. 
\par 32 Tą Jėzų Dievas prikėlė, ir mes visi esame šito liudytojai. 
\par 33 Dievo dešinės išaukštintas, Jis gavo iš Tėvo Šventosios Dvasios pažadą ir išliejo tai, ką dabar matote ir girdite. 
\par 34 Juk Dovydas neįžengė į dangų. Jis pats kalba: ‘Viešpats tarė mano Viešpačiui: sėskis mano dešinėje, 
\par 35 kol Aš patiesiu Tavo priešus, tarsi pakojį po Tavo kojų’. 
\par 36 Tad tvirtai žinokite, visi Izraelio namai: Dievas padarė Viešpačiu ir Kristumi tą Jėzų, kurį jūs nukryžiavote”. 
\par 37 Tie žodžiai vėrė jiems širdį, ir jie klausė Petrą bei kitus apaštalus: “Ką mums daryti, vyrai broliai?” 
\par 38 Petras jiems tarė: “Atgailaukite, ir kiekvienas tepasikrikštija Jėzaus Kristaus vardu, kad būtų atleistos jūsų nuodėmės, ir jūs gausite Šventosios Dvasios dovaną. 
\par 39 Juk jums skirtas pažadas, taip pat ir jūsų vaikams ir visiems toli esantiems, kuriuos tik pasišauks Viešpats, mūsų Dievas”. 
\par 40 Dar daugeliu kitų žodžių jis liudijo ir ragino juos, sakydamas: “Gelbėkitės iš šios iškrypusios kartos!” 
\par 41 Kurie mielai priėmė jo žodį, buvo pakrikštyti, ir tą dieną prisidėjo prie jų apie tris tūkstančius sielų. 
\par 42 Jie ištvermingai laikėsi apaštalų mokymo, bendravimo, duonos laužymo ir maldų. 
\par 43 Baimė apėmė kiekvieną, nes per apaštalus vyko daug stebuklų ir ženklų. 
\par 44 Visi tikintieji laikėsi drauge ir turėjo visa bendra. 
\par 45 Nuosavybę bei turtą jie parduodavo ir, ką gavę, padalydavo visiems, kiek kam reikėdavo. 
\par 46 Jie kasdien vieningai rinkdavosi šventykloje, o savo namuose tai vienur, tai kitur laužydavo duoną, vaišindavosi su džiugia ir tauria širdimi, 
\par 47 šlovindami Dievą, ir turėjo malonę visų žmonių akyse. O Viešpats kasdien gausino bažnyčią išgelbėtaisiais.


\chapter{3}


\par 1 Kartą Petras ir Jonas devintą maldos valandą ėjo drauge į šventyklą. 
\par 2 Ten buvo nešamas ir vienas žmogus, luošas nuo motinos įsčių. Jį kasdien sodindavo prie šventyklos vartų, vadinamų Gražiaisiais, kad prašytų išmaldos iš ateinančių į šventyklą. 
\par 3 Pastebėjęs beįeinančius į šventyklą Petrą ir Joną, jis paprašė išmaldos. 
\par 4 Petras, įdėmiai pažvelgęs į jį drauge su Jonu, tarė: 
\par 5 “Pažiūrėk į mudu”. Jis pažvelgė į juos, tikėdamasis ką nors iš jų gauti. 
\par 6 Bet Petras pasakė: “Sidabro nei aukso neturiu, bet ką turiu, tą duodu. Jėzaus Kristaus iš Nazareto vardu kelkis ir vaikščiok!” 
\par 7 Ir, paėmęs už dešinės rankos, pakėlė jį. Jo pėdos ir keliai bematant sustiprėjo. 
\par 8 Jis pašokęs atsistojo, pradėjo vaikščioti ir kartu su apaštalais įėjo į šventyklą. Ten vaikščiodamas ir pasišokinėdamas šlovino Dievą. 
\par 9 Visi žmonės pamatė jį vaikščiojant ir šlovinant Dievą. 
\par 10 Jie pažino, kad tai tas pats, kuris sėdėdavo elgetaudamas prie Gražiųjų vartų. Visi nustėro ir nustebo dėl to, kas buvo jam atsitikę. 
\par 11 Kadangi išgydytas luošys laikėsi Petro ir Jono, prie jų į vadinamąją Saliamono stoginę didžiai nustebinti susibėgo visi žmonės. 
\par 12 Tai matydamas, Petras kreipėsi į žmones: “Izraelio vyrai! Ko stebitės tuo ir ko taip žiūrite į mudu, tarsi mes savo jėga ar savo šventumu būtume padarę, kad šis vaikščiotų?! 
\par 13 Abraomo, Izaoko ir Jokūbo Dievas, mūsų tėvų Dievas, pašlovino savo tarną Jėzų, kurį jūs išdavėte ir kurio išsižadėjote Piloto akivaizdoje, kai tas buvo nusprendęs Jį paleisti. 
\par 14 Jūs išsižadėjote Šventojo ir Teisiojo, o pareikalavote atiduoti jums žmogžudį. 
\par 15 Jūs nužudėte gyvybės Kūrėją, kurį Dievas prikėlė iš numirusių, ir mes esame to liudytojai. 
\par 16 Jėzaus vardas­dėl tikėjimo Jo vardu­tvirtą padarė tą, kurį jūs matote ir pažįstate. Iš Jėzaus kylantis tikėjimas suteikė jam visišką sveikatą jūsų visų akyse. 
\par 17 O dabar, broliai, aš žinau, kad jūs taip padarėte iš nežinojimo, kaip ir jūsų vadai. 
\par 18 Taip Dievas įvykdė, ką iš anksto buvo paskelbęs visų savo pranašų lūpomis, būtent, kad Kristus kentėsiąs. 
\par 19 Tad atgailaukite ir atsiverskite, kad būtų panaikintos jūsų nuodėmės, kad nuo Viešpaties veido ateitų atgaivos laikai 
\par 20 ir Jis atsiųstų jums iš anksto paskelbtąjį Jėzų Kristų. 
\par 21 Jį turi priimti dangus iki visų dalykų atnaujinimo meto. Dievas tai nuo amžių paskelbė visų savo šventųjų pranašų lūpomis. 
\par 22 Juk Mozė tėvams pasakė: ‘Viešpats, mūsų Dievas, iš jūsų brolių pažadins jums Pranašą kaip mane. Klausykite Jo visame kame, ką tik Jis jums sakys. 
\par 23 O kiekviena siela, kuri to Pranašo neklausys, bus išnaikinta iš tautos’. 
\par 24 Ir visi pranašai, kurie tik kalbėjo nuo Samuelio laikų, vienas po kito skelbė šias dienas. 
\par 25 Jūs esate vaikai pranašų ir tos sandoros, kurią Dievas sudarė su jūsų tėvais, tardamas Abraomui: ‘Tavo palikuonyse bus palaimintos visos žemės giminės’. 
\par 26 Dievas pirmiausia jums prikėlęs pasiuntė savo tarną Jėzų, kad Jis atneštų jums palaiminimą, nukreipdamas kiekvieną nuo jo nusikaltimų”.


\chapter{4}


\par 1 Jiems dar kalbant miniai, prie jų priėjo kunigų, šventyklos apsaugos viršininkas ir sadukiejų, 
\par 2 kurie buvo labai pasipiktinę, kad apaštalai mokė žmones ir skelbė mirusiųjų prisikėlimą Jėzuje. 
\par 3 Jie suėmė juos ir įkalino iki kitos dienos, nes jau buvo vakaras. 
\par 4 Bet daugelis, girdėjusių žodį, įtikėjo, ir tikinčiųjų skaičius padidėjo maždaug iki penkių tūkstančių. 
\par 5 Rytojaus dieną Jeruzalėje susirinko tautos vadovai, vyresnieji ir Rašto žinovai, 
\par 6 taip pat vyriausiasis kunigas Anas, Kajafas, Jonas, Aleksandras ir kiti vyriausiojo kunigo giminės. 
\par 7 Pasistatę apaštalus viduryje, jie paklausė: “Kokia jėga ar kieno vardu jūs tai padarėte?” 
\par 8 Tada Petras, kupinas Šventosios Dvasios, jiems atsakė: “Tautos vadovai ir Izraelio vyresnieji! 
\par 9 Jeigu dėl gero darbo ligotam žmogui šiandien mus klausinėjate, kaip jis buvo išgydytas, 
\par 10 tai tebūnie jums visiems ir visai Izraelio tautai žinoma: vardu Jėzaus Kristaus iš Nazareto, kurį jūs nukryžiavote ir kurį Dievas prikėlė iš numirusių! Jo dėka šis vyras jūsų akivaizdoje stovi sveikas. 
\par 11 Jėzus yra ‘akmuo, kurį jūs, statytojai, atmetėte ir kuris tapo kertiniu akmeniu’. 
\par 12 Ir nėra niekame kitame išgelbėjimo, nes neduota žmonėms po dangumi kito vardo, kuriuo turime būti išgelbėti”. 
\par 13 Matydami Petro ir Jono drąsą ir patyrę, kad tai nemokyti, paprasti žmonės, jie labai stebėjosi; be to, atpažino juos buvus kartu su Jėzumi. 
\par 14 Bet, matydami stovintį su jais pagydytąjį, jie neturėjo ką pasakyti prieš. 
\par 15 Tada liepė jiems išeiti iš sinedriono ir ėmė tartis: 
\par 16 “Ką daryti su šitais žmonėmis? Juk visiems Jeruzalės gyventojams žinomas jų padarytas akivaizdus stebuklas, ir mes to negalime nuneigti. 
\par 17 Bet kad dalykas neišplistų žmonėse, griežtai uždrauskime, kad jie nė vienam žmogui daugiau nekalbėtų tuo vardu”. 
\par 18 Ir vėl juos pasišaukę, įsakė jiems išvis neskelbti ir nemokyti Jėzaus vardu. 
\par 19 Tačiau Petras ir Jonas jiems atsakė: “Spręskite patys, ar teisinga Dievo akivaizdoje jūsų klausyti labiau negu Dievo! 
\par 20 Juk mes negalime nekalbėti apie tai, ką matėme ir girdėjome”. 
\par 21 Prigrasinę jie paleido juos, nerasdami už ką bausti,­dėl žmonių, nes visi garbino Dievą už tai, kas buvo įvykę. 
\par 22 Vyras, kuris patyrė tą išgydymo stebuklą, buvo daugiau nei keturiasdešimties metų. 
\par 23 Paleisti jie atėjo pas saviškius ir išpasakojo, ką jiems sakė aukštieji kunigai ir vyresnieji. 
\par 24 Išklausę visi vieningai pakėlė balsus į Dievą ir sakė: “Valdove, Tu esi Dievas, sutvėręs dangų, žemę, jūrą ir visa, kas juose yra. 
\par 25 Tu kalbėjai savo tarno Dovydo lūpomis: ‘Kodėl niršta pagonys, kam veltui tautos sąmokslus rengia? 
\par 26 Žemės karaliai sukilo, valdovai susibūrė draugėn prieš Viešpatį ir Jo Kristų’. 
\par 27 Prieš Tavo šventąjį Sūnų Jėzų, kurį Tu patepei, iš tiesų susibūrė Erodas, Poncijus Pilotas su pagonimis ir Izraelio tauta, 
\par 28 kad įvykdytų, ką Tavo ranka ir sprendimas iš anksto buvo nulėmę įvykti. 
\par 29 O dabar, Viešpatie, pažvelk į jų grasinimus ir duok savo tarnams su didžia drąsa skelbti Tavo žodį, 
\par 30 ištiesdamas savo ranką išgydymams, ir kad būtų daromi ženklai bei stebuklai Tavo šventojo Sūnaus Jėzaus vardu”. 
\par 31 Jiems pasimeldus, sudrebėjo susirinkimo vieta, visi prisipildė Šventosios Dvasios ir drąsiai skelbė Dievo žodį. 
\par 32 Visa tikinčiųjų daugybė buvo vienos širdies ir vienos sielos. Ką turėjo, nė vienas nevadino savo nuosavybe, bet visa jiems buvo bendra. 
\par 33 Apaštalai su didžiule jėga liudijo apie Viešpaties Jėzaus prisikėlimą, ir gausi malonė buvo su jais visais. 
\par 34 Tarp jų nebuvo stokojančių, nes visi, kurie turėjo žemės sklypus ar namus, juos parduodavo, o už tai gautus pinigus atnešdavo 
\par 35 ir sudėdavo prie apaštalų kojų, ir kiekvienam buvo dalijama, kiek kam reikėjo. 
\par 36 Jozė, apaštalų pramintas Barnabu (išvertus tai reiškia: “Paguodos sūnus”), levitas, kilęs iš Kipro, 
\par 37 pardavęs savo žemės sklypą, atnešė pinigus ir padėjo apaštalams po kojų.


\chapter{5}


\par 1 O vienas vyras, vardu Ananijas, su savo žmona Sapfyra pardavė nuosavybę 
\par 2 ir, žmonai pritariant, pasiliko dalį gautų pinigų, o kitą dalį atnešė ir padėjo prie apaštalų kojų. 
\par 3 Petras paklausė: “Ananijau, kodėl šėtonas užvaldė tavo širdį, kad tu pamelavai Šventajai Dvasiai, pasilikdamas dalį už žemę gautų pinigų? 
\par 4 Argi nebuvo tavo tai, ką turėjai, ir ką gavai pardavęs, argi nebuvo tavo žinioje? Tai kodėl gi taip sumanei savo širdyje? Tu pamelavai ne žmonėms, bet Dievui!” 
\par 5 Išgirdęs tuos žodžius, Ananijas krito ant žemės ir mirė. Didelė baimė apėmė visus tai girdėjusius. 
\par 6 Keli jaunuoliai pakilo, suvyniojo jį marškomis, išnešė ir palaidojo. 
\par 7 Po kokių trijų valandų atėjo jo žmona, kuri nežinojo, kas buvo atsitikę. 
\par 8 Petras ją paklausė: “Pasakyk man, ar už tiek pardavėte sklypą?” “Taip, už tiek”,­atsakė ji. 
\par 9 Tada Petras jai tarė: “Kodėl susitarėte mėginti Viešpaties Dvasią? Štai ties durimis skamba žingsniai tų, kurie palaidojo tavo vyrą. Jie ir tave išneš”. 
\par 10 Bematant ji susmuko jam po kojų ir mirė. Atėję jaunuoliai rado ją negyvą, nunešė ir palaidojo šalia vyro. 
\par 11 Didelė baimė apėmė visą bažnyčią ir visus, kurie apie tai išgirdo. 
\par 12 Per apaštalų rankas žmonėse vyko daug ženklų ir stebuklų. Visi jie vieningai rinkdavosi Saliamono stoginėje. 
\par 13 Iš kitų nė vienas neišdrįsdavo prie jų prisidėti. Žmonės juos labai gerbė. 
\par 14 Ir vis labiau augo būrys vyrų ir moterų, įtikėjusių Viešpatį. 
\par 15 Net į gatves nešdavo ligonius ir guldydavo ant neštuvų bei lovų, kad, Petrui praeinant pro šalį, bent jo šešėlis kristų ant gulinčiųjų. 
\par 16 Taip pat ir iš aplinkinių miestų daugybė žmonių keliaudavo į Jeruzalę, gabendami sergančius ir netyrųjų dvasių kankinamus, ir visi jie būdavo pagydomi. 
\par 17 Tuomet sujudo vyriausiasis kunigas ir visi jo šalininkai iš sadukiejų partijos. Degdami pavydu, 
\par 18 jie suėmė apaštalus ir įmetė juos į viešą kalėjimą. 
\par 19 Bet Viešpaties angelas naktį atidarė kalėjimo vartus, išvedė juos ir tarė: 
\par 20 “Eikite ir, atsistoję šventykloje, skelbkite žmonėms visus šio Gyvenimo žodžius”. 
\par 21 Tai išgirdę, jie auštant nuėjo į šventyklą ir ėmė mokyti. Tuo metu vyriausiasis kunigas ir jo šalininkai sukvietė sinedrioną bei visą Izraelio tautos vyresniųjų tarybą ir nusiuntė į kalėjimą tarnus atvesti apaštalų. 
\par 22 Bet tarnai nuėję nerado jų kalėjime ir sugrįžę pranešė: 
\par 23 “Kalėjimą radome saugiai užrakintą ir sargybinius stovinčius prie vartų. Bet atidarę nieko viduje neradome!” 
\par 24 Išgirdę tokį pranešimą, vyriausiasis kunigas, šventyklos apsaugos viršininkas ir aukštieji kunigai suglumo, nesuprasdami, ką tai galėtų reikšti. 
\par 25 Tuomet kažkas atėjęs pranešė: “Tie vyrai, kuriuos buvote uždarę kalėjime, stovi šventykloje ir moko žmones”. 
\par 26 Tada viršininkas su tarnais nuėjo ir atsivedė juos be prievartos, nes bijojo žmonių, kad neužmėtytų akmenimis. 
\par 27 Taigi atsivedę apaštalus, pastatė juos sinedrione. Vyriausiasis kunigas jiems tarė: 
\par 28 “Argi mes jums drauste neuždraudėme mokyti tuo vardu, o štai jūs užtvindėte Jeruzalę savo mokymu ir dar norite ant mūsų užtraukti to žmogaus kraują”. 
\par 29 Petras ir apaštalai atsakė: “Dievo reikia klausyti labiau negu žmonių. 
\par 30 Mūsų tėvų Dievas prikėlė Jėzų, kurį jūs nužudėte, pakabindami ant medžio. 
\par 31 Dievas išaukštino Jį savo dešine kaip Karalių ir Išgelbėtoją, kad suteiktų Izraeliui atgailą ir nuodėmių atleidimą. 
\par 32 Mes esame Jo ir tų įvykių liudytojai, taip pat ir Šventoji Dvasia, kurią Dievas suteikė tiems, kurie Jam paklūsta”. 
\par 33 Girdėdami šituos žodžius, jie baisiai įtūžo ir ketino juos užmušti. 
\par 34 Tuomet sinedrione pakilo vienas fariziejus, vardu Gamalielis, visos tautos gerbiamas Įstatymo mokytojas. Jis įsakė trumpam išvesti apaštalus 
\par 35 ir tarė susirinkusiems: “Vyrai izraelitai! Gerai pagalvokite, kaip pasielgti su šitais žmonėmis. 
\par 36 Juk prieš kiek laiko buvo iškilęs Teudas, kuris laikė save kažkuo nepaprastu. Prie jo prisidėjo apie keturis šimtus vyrų, bet jis buvo užmuštas, visi šalininkai išsisklaidė ir nuėjo niekais. 
\par 37 Po jo, gyventojų surašymo dienomis, atsirado Judas Galilėjietis ir patraukė nemažai žmonių paskui save. Jis taip pat žuvo, o visi jo sekėjai buvo išblaškyti. 
\par 38 Todėl dabar jums sakau: palikite šituos žmones ramybėje ir paleiskite juos. Jei šis sumanymas ir ši veikla iš žmonių,­jie žlugs savaime, 
\par 39 o jeigu tai iš Dievo, tai jūs nepajėgsite jų sunaikinti. Žiūrėkite, kad nepasirodytumėte kovojantys prieš Dievą!” Jie paklausė jo patarimo. 
\par 40 Pasišaukę apaštalus, nuplakdino juos, uždraudė kalbėti Jėzaus vardu ir paleido. 
\par 41 O tie ėjo iš sinedriono džiaugdamiesi, kad buvo palaikyti vertais dėl Jėzaus vardo iškęsti paniekinimą. 
\par 42 Kiekvieną dieną šventykloje ir po namus jie nesiliovė mokyti bei skelbti Jėzų Kristumi.


\chapter{6}


\par 1 Tomis dienomis, mokinių skaičiui augant, tarp graikiškai kalbančiųjų kilo nepasitenkinimas vietiniais žydais, nes kasdieniniame aprūpinime būdavo aplenkiamos jų našlės. 
\par 2 Tuomet dvylika sušaukė mokinių susirinkimą ir tarė: “Nedera mums palikus Dievo žodį tarnauti prie stalų. 
\par 3 Todėl, broliai, išsirinkite iš savųjų septynis vyrus, turinčius gerą vardą, kupinus Šventosios Dvasios ir išminties. Mes juos paskirsime tam darbui, 
\par 4 o patys toliau atsidėsime maldai ir žodžio tarnavimui”. 
\par 5 Šis pasiūlymas patiko visam susirinkimui, ir jie išsirinko Steponą, vyrą pilną tikėjimo ir Šventosios Dvasios, Pilypą, Prochorą, Nikanorą, Timoną, Parmeną ir Mikalojų, prozelitą iš Antiochijos. 
\par 6 Juos pastatė prieš apaštalus, o šie melsdamiesi uždėjo ant jų rankas. 
\par 7 Dievo žodis klestėjo, ir mokinių skaičius Jeruzalėje smarkiai augo. Ir didelis kunigų būrys pakluso tikėjimui. 
\par 8 Steponas, pilnas tikėjimo ir jėgos, darė žmonėse didžių stebuklų ir ženklų. 
\par 9 Tuomet pakilo kai kurie iš vadinamosios libertinų sinagogos, iš kirėniečių, aleksandriečių ir iš Kilikijos bei Azijos ir ėmė ginčytis su Steponu. 
\par 10 Tačiau jie negalėjo atsispirti išminčiai ir Dvasiai, kurios įkvėptas jis kalbėjo. 
\par 11 Tada jie papirko keletą vyrų, kad tie sakytų: “Mes girdėjome jį piktžodžiaujant Mozei ir Dievui”. 
\par 12 Taip jie sukurstė minią, vyresniuosius bei Rašto žinovus, užpuolę sučiupo jį ir nusivedė į sinedrioną. 
\par 13 Ten pastatė melagingus liudytojus, kurie tvirtino: “Šitas žmogus nesiliauja kalbėjęs prieš šventąją vietą ir Įstatymą. 
\par 14 Mes girdėjome jį sakant, kad Jėzus iš Nazareto išgriaus šią vietą ir pakeis Mozės perduotus mums papročius”. 
\par 15 Visi sėdintys sinedrione įsmeigė į jį akis ir matė jo veidą tarytum angelo veidą.


\chapter{7}


\par 1 Vyriausiasis kunigas paklausė: “Ar tikrai taip yra?” 
\par 2 O Steponas prabilo: “Vyrai broliai ir tėvai, paklausykite! Šlovės Dievas apsireiškė mūsų tėvui Abraomui Mesopotamijoje, kai jis dar nebuvo persikėlęs į Charaną, 
\par 3 ir pasakė jam: ‘Išeik iš savo krašto, nuo savo giminių, ir keliauk į šalį, kurią tau parodysiu’. 
\par 4 Tada jis paliko chaldėjų kraštą ir apsigyveno Charane. Iš ten, jo tėvui mirus, Dievas atvedė jį į šitą šalį, kurioje jūs dabar gyvenate. 
\par 5 Bet čia jam nedavė paveldėti nė pėdos žemės, tik pažadėjo ją atiduoti jo ir jo palikuonių nuosavybėn, nors jis dar buvo bevaikis. 
\par 6 Dievas pasakė, kad jo palikuonys gyvens kaip ateiviai svetimoje žemėje ir keturis šimtus metų bus pavergti ir skriaudžiami. 
\par 7 ‘Tačiau,­tarė Dievas,­Aš nuteisiu tautą, kuriai jie vergaus; tada jie išeis ir tarnaus man šitoje vietoje’. 
\par 8 Ir tuomet Jis davė jam apipjaustymo sandorą. Tada Abraomui gimė Izaokas, kurį jis apipjaustė aštuntąją dieną. Izaokui gimė Jokūbas, o Jokūbui gimė dvylika patriarchų. 
\par 9 Šie patriarchai iš pavydo pardavė Juozapą į Egiptą. Bet Dievas buvo su juo, 
\par 10 išgelbėjo Juozapą iš visų jo vargų ir suteikė jam malonės bei išminties Egipto karaliaus­faraono akivaizdoje. Tas jį paskyrė Egipto ir visų savo namų valdytoju. 
\par 11 Tuomet visą Egipto žemę ir Kanaaną ištiko badas bei didis vargas, ir mūsų tėvai negalėjo rasti sau maisto. 
\par 12 Išgirdęs, kad Egipte yra javų, Jokūbas išsiuntė ten mūsų tėvus pirmą kartą, 
\par 13 o antrą kartą Juozapas leidosi savo brolių atpažįstamas, ir faraonas sužinojo apie Juozapo giminę. 
\par 14 Tada Juozapas pasiuntė pakviesti savo tėvo Jokūbo ir visų giminaičių­septyniasdešimt penkių sielų. 
\par 15 Šitaip Jokūbas iškeliavo į Egiptą. Ten mirė jis ir mūsų tėvai. 
\par 16 Juos pargabeno į Sichemą ir palaidojo kape, kurį Abraomas už pinigus buvo nupirkęs iš sichemiečio Emoro sūnų. 
\par 17 Artėjant metui pažado, dėl kurio Dievas buvo prisiekęs Abraomui, tauta išaugo ir išsiplėtė Egipte, 
\par 18 kol pakilo kitas karalius, kuris nieko nežinojo apie Juozapą. 
\par 19 Jis klastingai elgėsi su mūsų gimine, versdamas mūsų tėvus išmesti savo kūdikius, kad jie neliktų gyvi. 
\par 20 Tuo metu gimė Mozė, kuris buvo mielas Dievui. Tris mėnesius jis buvo maitinamas savo tėvo namuose, 
\par 21 o kai buvo išmestas, jį pasiėmė faraono duktė ir augino kaip savo pačios sūnų. 
\par 22 Mozė buvo išmokytas visos Egipto išminties ir tapo galingas žodžiais ir darbais. 
\par 23 Kai jam sukako keturiasdešimt metų, jo širdyje kilo troškimas aplankyti savo brolius, Izraelio vaikus. 
\par 24 Pamatęs vieną iš jų skriaudžiamą, stojo ginti jo ir keršydamas užmušė egiptietį. 
\par 25 Mozė tikėjosi, kad jo broliai suprasią, jog Dievas jo ranka suteiks jiems išgelbėjimą, bet jie šito nesuprato. 
\par 26 Kitą dieną jis priėjo prie besimušančių tautiečių ir mėgino juos sutaikinti, sakydamas: ‘Vyrai, jūs esate broliai! Kodėl skriaudžiate vienas kitą?’ 
\par 27 Bet tasai, kuris skriaudė artimą, atstūmė jį, sakydamas: ‘Kas tave paskyrė mūsų valdovu ir teisėju? 
\par 28 Gal nori ir mane nužudyti, kaip vakar nužudei egiptietį?’ 
\par 29 Tai išgirdęs, Mozė pabėgo ir buvo ateivis Madiano krašte, kur jam gimė du sūnūs. 
\par 30 Po keturiasdešimties metų Sinajaus kalno dykumoje jam pasirodė angelas degančio erškėčių krūmo liepsnose. 
\par 31 Tai pamatęs, Mozė labai nusistebėjo reginiu. Kai jis artinosi, norėdamas geriau įsižiūrėti, jam pasigirdo Viešpaties balsas: 
\par 32 ‘Aš esu tavo tėvų Dievas, Abraomo, Izaoko ir Jokūbo Dievas’. Tada Mozė ėmė drebėti ir nebedrįso toliau žiūrėti. 
\par 33 O Viešpats jam tarė: ‘Nusiauk apavą nuo kojų, nes ta vieta, kur tu stovi, yra šventa žemė. 
\par 34 Aš regėte regėjau savo tautos priespaudą Egipte, išgirdau jos dejones ir nužengiau jos išvaduoti. Tad ateik čia, Aš pasiųsiu tave į Egiptą’. 
\par 35 Tą Mozę, kurio jie atsižadėjo, sakydami: ‘Kas tave paskyrė valdovu ir teisėju?’, Dievas pasiuntė kaip vadovą ir gelbėtoją ranka angelo, kuris pasirodė jam erškėčių krūme. 
\par 36 Jis išvedė juos, darydamas stebuklus ir ženklus Egipto žemėje, prie Raudonosios jūros ir per keturiasdešimt metų dykumoje. 
\par 37 Tai šis Mozė pasakė Izraelio vaikams: ‘Dievas pažadins jums iš jūsų brolių Pranašą, panašiai kaip mane’. 
\par 38 Tai jis susirinkimo dieną dykumoje tarpininkavo tarp angelo, kalbėjusio jam Sinajaus kalne, ir mūsų tėvų. Ten jis gavo mums skirtuosius gyvenimo žodžius. 
\par 39 Jam mūsų tėvai nenorėjo paklusti, atstūmė jį ir savo širdis nukreipė į Egiptą. 
\par 40 Jie tarė Aaronui: ‘Padaryk mums dievų, kurie žygiuotų mūsų priekyje, nes mes nežinome, kas nutiko tam Mozei, kuris išvedė mus iš Egipto žemės’. 
\par 41 Taip anomis dienomis jie pasidirbo veršį, atnašavo stabui auką ir džiaugėsi savo rankų darbu. 
\par 42 Tada Dievas nusigręžė nuo jų ir paliko juos garbinti dangaus kariauną, kaip parašyta pranašų knygoje: ‘Ar jūs, Izraelio namai, man atnešėte aukų bei atnašų, keturiasdešimt metų būdami dykumoje? 
\par 43 Jūs pasiėmėte Molocho palapinę ir dievo Refano žvaigždę­stabus, kuriuos pasidirbote garbinti. Todėl ištremsiu jus anapus Babilono’. 
\par 44 Mūsų tėvai dykumoje turėjo Liudijimo palapinę, kaip įsakė Tas, kuris kalbėjo Mozei, kad padarytų ją pagal regėtą jos vaizdą. 
\par 45 Mūsų tėvai su Jozue pasiėmė ją ir atsigabeno į žemes pagonių, kuriuos Dievas išvijo nuo mūsų tėvų akių. Taip pasiliko iki Dovydo dienų. 
\par 46 Šis susilaukė Dievo akyse malonės ir meldė, kad rastų buveinę Jokūbo Dievui. 
\par 47 Ir Saliamonas pastatė Jam namus. 
\par 48 Bet Aukščiausiasis negyvena rankų darbo šventyklose, kaip sako ir pranašas: 
\par 49 ‘Dangus­mano sostas, o žemė­pakojis po mano kojomis. Kokius gi namus man statysite?­ klausia Viešpats,­ar kokia mano poilsio vieta? 
\par 50 Argi ne mano ranka visa tai padarė?’ 
\par 51 Jūs, kietasprandžiai, neapipjaustytomis širdimis ir ausimis! Jūs, kaip ir jūsų tėvai, visuomet priešinatės Šventajai Dvasiai. 
\par 52 Kurio iš pranašų nepersekiojo jūsų tėvai? Jie užmušė iš anksto skelbusius Teisiojo atėjimą. Jo išdavėjais ir žudikais dabar esate jūs. 
\par 53 Jūs, kurie gavote Įstatymą, paskelbtą per angelus, bet jo nesilaikėte”. 
\par 54 Tie žodžiai jiems draskė širdį, ir jie griežė ant jo dantimis. 
\par 55 O Steponas, kupinas Šventosios Dvasios, žvelgė į dangų ir išvydo Dievo šlovę ir Jėzų, stovintį Dievo dešinėje. 
\par 56 Jis tarė: “Štai regiu atsivėrusį dangų ir Žmogaus Sūnų, stovintį Dievo dešinėje”. 
\par 57 Tada, baisiai rėkdami, jie užsikimšo ausis ir visi kaip vienas puolė jį, 
\par 58 išsitempė už miesto ir užmėtė akmenimis. Liudytojai padėjo savo drabužius prie kojų vieno jauno vyro, vardu Sauliaus. 
\par 59 Taip jie mušė akmenimis Steponą, o jis šaukė: “Viešpatie Jėzau, priimk mano dvasią!” 
\par 60 Pagaliau suklupęs jis galingu balsu sušuko: “Viešpatie, neįskaityk jiems šios nuodėmės!” Ir, tai ištaręs, užmigo.


\chapter{8}


\par 1 Saulius pritarė Stepono nužudymui. Tomis dienomis prasidėjo didelis Jeruzalės bažnyčios persekiojimas. Visi, išskyrus apaštalus, išsisklaidė po Judėjos ir Samarijos apylinkes. 
\par 2 Dievobaimingi vyrai palaidojo Steponą ir labai jį apraudojo. 
\par 3 O Saulius niokojo bažnyčią, naršydamas po namus, tempdamas iš jų vyrus ir moteris ir siųsdamas juos į kalėjimą. 
\par 4 Tuo tarpu išblaškytieji keliaudami skelbė žodį. 
\par 5 Pilypas, nuvykęs į Samarijos miestą, ėmė skelbti gyventojams Kristų. 
\par 6 Minios vieningai klausėsi Pilypo žodžių, girdėdamos ir matydamos, kokius jis darė stebuklus. 
\par 7 Iš daugelio apsėstųjų, baisiai šaukdamos, išeidavo netyrosios dvasios, išgydavo daug paralyžiuotųjų ir luošų. 
\par 8 Ir didelis džiaugsmas pasklido po tą miestą. 
\par 9 Mieste buvo vienas vyras, vardu Simonas, kuris nuo seno užsiiminėjo magija, stebindamas Samarijos gyventojus ir sakydamas esąs nepaprastas. 
\par 10 Visi­nuo mažo iki didelio­jį gerbė ir sakė: “Jis yra didelė Dievo jėga”. 
\par 11 Gerbė jį todėl, kad ilgą laiką šis stebino juos savo kerais. 
\par 12 Bet patikėję Pilypu, kuris skelbė Dievo karalystę ir Jėzaus Kristaus vardą, ėmė krikštytis vyrai ir moterys. 
\par 13 Ir pats Simonas įtikėjo ir pasikrikštijęs nesitraukė nuo Pilypo. Jis buvo apstulbintas, matydamas daromus ženklus ir stebuklus. 
\par 14 Apaštalai Jeruzalėje, išgirdę, jog Samarija priėmė Dievo žodį, nusiuntė ten Petrą ir Joną, 
\par 15 kurie atvykę ėmė melstis už samariečius, kad jie priimtų Šventąją Dvasią. 
\par 16 Mat Ji dar nebuvo nė vienam jų nužengusi, ir jie tebuvo pakrikštyti Viešpaties Jėzaus vardu. 
\par 17 Tuomet apaštalai dėjo ant jų rankas, ir tie gavo Šventąją Dvasią. 
\par 18 Pamatęs, kad apaštalų rankų uždėjimu teikiama Šventoji Dvasia, Simonas pasiūlė jiems pinigų, 
\par 19 sakydamas: “Duokite ir man tą jėgą, kad, kam tik uždėsiu rankas, gautų Šventąją Dvasią”. 
\par 20 Bet Petras jam tarė: “Kad tu pražūtum kartu su savo pinigais, jei sumanei už pinigus gauti Dievo dovaną! 
\par 21 Šiame dalyke tu negali turėti nė mažiausios dalies, nes tavo širdis neteisi prieš Dievą. 
\par 22 Taigi atgailauk dėl šio savo nedorumo ir melsk Dievą­gal Jis atleis tavo širdies sumanymą. 
\par 23 Matau, tu pilnas karčios tulžies ir esi nedorybės pančiuose”. 
\par 24 Simonas atsakė: “Melskite už mane Viešpatį, kad manęs neištiktų tai, ką jūs sakėte”. 
\par 25 O jie, paliudiję ir apsakę Viešpaties žodį, pasuko atgal į Jeruzalę, skelbdami Evangeliją daugelyje Samarijos kaimų. 
\par 26 Viešpaties angelas prabilo į Pilypą, sakydamas: “Kelkis ir eik pietų link ant kelio, kuris eina iš Jeruzalės į Gazą. Jis visiškai tuščias”. 
\par 27 Jis pakilo ir iškeliavo. Ir štai važiuoja etiopas eunuchas, aukštas Etiopijos karalienės Kandakės dvariškis, viso jos iždo valdytojas. Jis buvo atvykęs į Jeruzalę pagarbinti, 
\par 28 o dabar keliavo namo ir, sėdėdamas savo vežime, skaitė pranašą Izaiją. 
\par 29 Dvasia pasakė Pilypui: “Prieik ir laikykis greta šito vežimo”. 
\par 30 Pribėgęs Pilypas išgirdo jį skaitant pranašą Izaiją ir paklausė: “Ar supranti, ką skaitai?” 
\par 31 Šis atsiliepė: “Kaip galiu suprasti, jei man niekas nepaaiškina?!” Ir jis pakvietė Pilypą lipti į vežimą ir sėstis šalia. 
\par 32 Rašto vieta, kurią jis skaitė, buvo ši: “Tarsi avį vedė Jį į pjovyklą, ir kaip ėriukas, kuris tyli kerpamas, Jis neatvėrė savo lūpų. 
\par 33 Jis buvo pažemintas ir neteisingai nuteistas. Kas apsakys Jo giminę, jeigu Jo gyvenimas žemėje buvo nutrauktas?!” 
\par 34 Eunuchas paklausė Pilypą: “Prašom paaiškinti, apie ką čia pranašas kalba? Apie save ar apie ką kitą?” 
\par 35 Atvėręs lūpas ir pradėjęs nuo tos Rašto vietos, Pilypas jam paskelbė Gerąją naujieną apie Jėzų. 
\par 36 Keliaudami toliau, jie privažiavo vandenį. “Štai vanduo,­tarė eunuchas,­kas gi kliudo man pasikrikštyti?” 
\par 37 Pilypas tarė: “Jei tiki iš visos širdies, tai galima”. Tas atsakė: “Tikiu, kad Jėzus Kristus yra Dievo Sūnus”. 
\par 38 Jis liepė sustabdyti vežimą, ir jie abu­Pilypas ir eunuchas­išlipę įbrido į vandenį, ir Pilypas jį pakrikštijo. 
\par 39 Išėjus iš vandens, Viešpaties Dvasia pagavo Pilypą. Eunuchas daugiau jo nebematė, tik džiūgaudamas traukė savo keliu. 
\par 40 O Pilypas atsidūrė Azote; iš ten keliaudamas skelbė Evangeliją visuose miestuose ir taip atvyko į Cezarėją.


\chapter{9}


\par 1 Tuo tarpu Saulius, tebealsuodamas grasinimais ir žudynėmis prieš Viešpaties mokinius, nuėjo pas vyriausiąjį kunigą 
\par 2 ir išgavo raštus Damasko sinagogoms, kad, užtikęs to Kelio sekėjus, vyrus ir moteris, galėtų juos suiminėti ir gabenti į Jeruzalę. 
\par 3 Kai keliaudamas priartėjo prie Damasko, staiga jį apšvietė iš dangaus šviesa. 
\par 4 Jis krito ant žemės ir išgirdo balsą, sakantį jam: “Sauliau, Sauliau, kam mane persekioji?” 
\par 5 Jis tarė: “Kas Tu esi, Viešpatie?” Viešpats atsakė: “Aš esu Jėzus, kurį tu persekioji. Sunku tau spyriotis prieš akstiną! 
\par 6 Kelkis, eik į miestą, ir tau bus pasakyta, ką turi daryti”. 
\par 7 Su juo keliavę vyrai stovėjo be žado: jie girdėjo balsą, tačiau nieko nematė. 
\par 8 Saulius atsikėlė nuo žemės, bet, atmerkęs akis, nieko nebematė. Laikydami už rankos, jie nuvedė jį į Damaską. 
\par 9 Jis tris dienas išbuvo neregintis, nieko nevalgė ir negėrė. 
\par 10 Damaske buvo vienas mokinys, vardu Ananijas. Viešpats tarė jam regėjime: “Ananijau!” Šis atsakė: “Štai aš, Viešpatie”. 
\par 11 Viešpats tęsė: “Kelkis, eik į gatvę, vadinamą Tiesiąja, ir Judo namuose teiraukis tarsiečio, vardu Sauliaus. Štai, jis meldžiasi 
\par 12 ir regėjime pamatė vyrą, vardu Ananiją, ateinantį ir uždedantį ant jo rankas, kad praregėtų”. 
\par 13 Ananijas atsakė: “Viešpatie, iš daugelio esu girdėjęs apie tą vyrą, kiek daug pikto jis yra padaręs Tavo šventiesiems Jeruzalėje. 
\par 14 Ir čia jis turi aukštųjų kunigų įgaliojimus suiminėti visus, kurie šaukiasi Tavo vardo”. 
\par 15 Bet Viešpats jam tarė: “Eik, nes jis yra mano išrinktas indas, kuris neš mano vardą pagonims, karaliams ir Izraelio vaikams. 
\par 16 Aš jam parodysiu, kiek daug jis turės iškentėti dėl mano vardo”. 
\par 17 Ananijas nuėjo į tuos namus, uždėjo ant jo rankas ir tarė: “Broli Sauliau! Viešpats Jėzus, kuris tau pasirodė kelyje, kai keliavai, atsiuntė mane, kad tu vėl regėtum ir taptum pilnas Šventosios Dvasios”. 
\par 18 Ir bematant jam nuo akių lyg žvynai nukrito. Jis kaipmat atgavo regėjimą ir buvo pakrikštytas. 
\par 19 Paskui užvalgęs sustiprėjo. Pabuvęs kelias dienas su Damasko mokiniais, 
\par 20 Saulius bematant pradėjo pamokslauti sinagogose, kad Kristus yra Dievo Sūnus. 
\par 21 Visi, kurie tai girdėjo, be galo stebėjosi ir klausinėjo: “Ar čia ne tas pats, kuris Jeruzalėje persekiojo visus, kurie šaukiasi šito vardo?! Argi jis nėra čia atvykęs jų suimti ir gabenti pas aukštuosius kunigus?!” 
\par 22 O Saulius vis labiau stiprėjo ir kėlė sąmyšį tarp Damaske gyvenančių žydų, įrodinėdamas, kad Jėzus yra Kristus. 
\par 23 Praslinkus nemažai laiko, žydai nutarė jį nužudyti. 
\par 24 Tačiau Saulius sužinojo apie jų sąmokslą. Dieną ir naktį jie tykojo prie vartų, norėdami jį užmušti, 
\par 25 bet mokiniai nakčia nuleido jį per sieną ant virvių pintinėje. 
\par 26 Saulius nuvyko į Jeruzalę ir mėgino prisidėti prie mokinių, tačiau visi jo bijojo, netikėdami jį esant mokinį. 
\par 27 Bet Barnabas, pasiėmęs jį, nusivedė pas apaštalus ir jiems papasakojo, kaip kelionėje Saulius regėjęs Viešpatį ir šis kalbėjęs su juo, ir kaip Damaske jis drąsiai pamokslavęs Jėzaus vardu. 
\par 28 Taip jis liko su jais Jeruzalėje ir drąsiai kalbėjo Viešpaties Jėzaus vardu. 
\par 29 Jis kalbėdavosi ir ginčydavosi su helenistais, o tie kėsinosi jį nužudyti. 
\par 30 Tai sužinoję, broliai jį nugabeno į Cezarėją ir iš ten išsiuntė į Tarsą. 
\par 31 Tuo tarpu bažnyčios visoje Judėjoje, Galilėjoje ir Samarijoje džiaugėsi ramybe, statydinosi ir augo, gyvendamos su Viešpaties baime ir guodžiamos Šventosios Dvasios. 
\par 32 Kartą Petras, lankydamas jas visas, atėjo pas šventuosius, gyvenančius Lydoje. 
\par 33 Ten jis užtiko žmogų, vardu Enėją, kuris paralyžiuotas jau aštuoneri metai gulėjo patale. 
\par 34 Petras jam tarė: “Enėjau, Jėzus Kristus išgydo tave. Kelkis ir užklok savo patalą!” Šis bematant atsikėlė. 
\par 35 Visi Lydos ir Sarono gyventojai pamatė jį ir atsivertė į Viešpatį. 
\par 36 Jopėje gyveno viena mokinė, vardu Tabita, išvertus Dorkadė. Ji garsėjo gerais darbais ir gailestingumo aukomis. 
\par 37 Tomis dienomis ji susirgo ir mirė. Ją nuprausė ir paguldė aukštutiniame kambaryje. 
\par 38 Kadangi Lyda netoli Jopės, mokiniai, išgirdę ten esant Petrą, pasiuntė pas jį du vyrus, prašydami, kad jis nedelsdamas pas juos atvyktų. 
\par 39 Petras pakilęs iškeliavo su jais. Kai tik atėjo, jį nuvedė į aukštutinį kambarį. Ten jį apstojo visos našlės, verkdamos ir rodydamos jam tunikas bei viršutinius drabužius, kuriuos, tebegyvendama su jomis, buvo pasiuvusi Dorkadė. 
\par 40 Petras liepė visiems išeiti. Atsiklaupęs pasimeldė ir, atsisukęs į lavoną, tarė: “Tabita, kelkis!” Ši atmerkė akis ir, išvydusi Petrą, atsisėdo. 
\par 41 Jis ištiesė jai ranką ir ją pakėlė. Tada, pašaukęs vidun šventuosius ir našles, parodė jiems ją gyvą. 
\par 42 Žinia apie tai pasklido visoje Jopėje, ir daugelis įtikėjo Viešpatį. 
\par 43 Jis dar ilgesnį laiką pasiliko Jopėje pas vieną odininką, vardu Simoną.


\chapter{10}


\par 1 Cezarėjoje gyveno vyras, vardu Kornelijus, vadinamosios italų kohortos šimtininkas. 
\par 2 Jis buvo dievotas ir drauge su savo namiškiais bijojo Dievo, gausiai šelpdavo žmones ir nuolat melsdavosi Dievui. 
\par 3 Kartą, apie devintą valandą, jis regėjime aiškiai išvydo pas jį ateinantį Dievo angelą, kuris jam tarė: “Kornelijau!” 
\par 4 Tas, pažvelgęs į jį, išsigando ir paklausė: “Kas yra, viešpatie?” Šis jam atsakė: “Tavo maldos ir gailestingumo aukos pakilo Dievo akivaizdon, ir Jis tave prisiminė. 
\par 5 Dabar siųsk vyrus į Jopę ir pasikviesk Simoną, vadinamą Petru. 
\par 6 Jis svečiuojasi pas vieną odininką, Simoną, kurio namai stovi prie jūros. Petras pasakys tau, ką turi daryti”. 
\par 7 Kai su juo kalbėjęs angelas pasitraukė, Kornelijus pasišaukė du tarnus ir vieną pamaldų kareivį iš nuolat šalia esančių valdinių 
\par 8 ir, viską jiems išaiškinęs, išsiuntė į Jopę. 
\par 9 Rytojaus dieną, kai šitie keliaudami artinosi prie miesto, Petras užlipo ant plokščiastogio melstis. Buvo apie šeštą valandą. 
\par 10 Jis pasijuto labai išalkęs ir norėjo užkąsti. Kol jam ruošė valgį, jį ištiko dvasios pagava. 
\par 11 Jis išvydo atsivėrusį dangų, iš kurio jam leidosi žemyn kažkoks padėklas, lyg didelė marška, pririšta už keturių kampų ir nuleidžiama žemėn. 
\par 12 Jame buvo įvairiausių žemės keturkojų, laukinių žvėrių, roplių ir padangės paukščių. 
\par 13 Ir jam pasigirdo balsas: “Kelkis, Petrai, pjauk ir valgyk!” 
\par 14 Bet Petras atsakė: “Jokiu būdu, Viešpatie! Aš niekuomet nesu valgęs nieko sutepto ar nešvaraus”. 
\par 15 Balsas antrą kartą tarė: “Ką Dievas apvalė, tu nevadink suteptu!” 
\par 16 Taip pasikartojo tris kartus, ir tuojau padėklas vėl pakilo į dangų. 
\par 17 Petrui tebesvarstant, ką galėtų reikšti matytas regėjimas, štai Kornelijaus pasiuntiniai, išklausinėję, kur Simono namai, sustojo prie vartų. 
\par 18 Jie sušukę paklausė: “Ar čia vieši Simonas, vadinamas Petru?” 
\par 19 Petrui, vis tebemąstančiam apie regėjimą, Dvasia tarė: “Štai trys vyrai ieško tavęs. 
\par 20 Kelkis, lipk žemyn ir keliauk su jais nė kiek nedvejodamas, nes Aš juos atsiunčiau”. 
\par 21 Taigi Petras nulipo žemyn pas Kornelijaus siųstus vyrus ir tarė: “Štai aš, kurio ieškote. Su kokiu reikalu atėjote?” 
\par 22 Tie atsakė: “Šimtininkas Kornelijus, teisus ir dievobaimingas vyras, turintis gerą vardą visoje žydų tautoje, gavo iš šventojo angelo nurodymą pakviesti tave į savo namus ir išklausyti tavo žodžių”. 
\par 23 Tada Petras paprašė juos į vidų ir svetingai priėmė. Kitą rytą iškeliavo su jais. Jį lydėjo keli broliai iš Jopės. 
\par 24 Rytojaus dieną jie atvyko į Cezarėją. Kornelijus jų laukė, susikvietęs savo gimines ir artimiausius draugus. 
\par 25 Įeinantį Petrą Kornelijus pasitiko ir, puldamas po kojų, išreiškė jam pagarbą. 
\par 26 Bet Petras pakėlė Kornelijų, tardamas: “Kelkis! Juk ir aš esu žmogus!” 
\par 27 Paskui, kalbėdamasis su juo, įėjo vidun ir, radęs daug susirinkusių, 
\par 28 prabilo: “Jūs žinote, kad žydui nevalia bendrauti ar užeiti pas svetimtautį. Bet Dievas parodė man, jog negalima jokio žmogaus laikyti suteptu ar netyru. 
\par 29 Štai kodėl pakviestas aš nesipriešindamas atvykau. Taigi klausiu dabar, kokiu reikalu mane pakvietėte?” 
\par 30 Kornelijus atsakė: “Kaip tik dabar sukanka lygiai keturios dienos, kai aš savo namuose pasninkavau ir meldžiausi devintą valandą. Ir štai prieš mane stojo spindinčiais drabužiais vyras 
\par 31 ir prabilo: ‘Kornelijau, tavo maldos išklausytos, ir Dievas prisiminė tavo gailestingumo aukas. 
\par 32 Tad siųsk pasiuntinius į Jopę ir pasikviesk Simoną, vadinamą Petru. Jis yra apsistojęs odininko Simono namuose prie jūros. Atėjęs jis tau viską pasakys’. 
\par 33 Taigi aš iš karto pasiunčiau pas tave, o tu gerai padarei, čia atvykdamas. Dabar mes visi esame susirinkę Dievo akivaizdoje, kad išgirstume visa, ką Dievas tau pavedė”. 
\par 34 Tada, atvėręs lūpas, Petras pasakė: “Iš tiesų dabar suprantu, kad Dievas neatsižvelgia į asmenis. 
\par 35 Jam priimtinas kiekvienos tautos žmogus, kuris Jo bijo ir teisiai gyvena. 
\par 36 Jis pasiuntė savo žodį Izraelio vaikams ir per Jėzų Kristų paskelbė taikos Evangeliją. Jis yra visų Viešpats. 
\par 37 Jūs žinote, kas yra įvykę visoje Judėjoje, pradedant nuo Galilėjos, po Jono skelbtojo krikšto,­ 
\par 38 kaip Dievas patepė Šventąja Dvasia ir jėga Jėzų iš Nazareto, ir Jis vaikščiojo, darydamas gera ir gydydamas visus velnio pavergtuosius, nes Dievas buvo su Juo. 
\par 39 Mes esame liudytojai visko, ką Jis yra padaręs žydų šalyje ir Jeruzalėje. Jį nužudė, pakabindami ant medžio. 
\par 40 Tačiau trečią dieną Dievas Jį prikėlė ir leido Jam pasirodyti, 
\par 41 beje, ne visai tautai, o Dievo iš anksto išrinktiems liudytojams, būtent mums, kurie su Juo valgėme ir gėrėme, Jam prisikėlus iš numirusių. 
\par 42 Jis mums įsakė skelbti žmonėms ir liudyti, kad Jis yra Dievo paskirtasis gyvųjų ir mirusiųjų teisėjas. 
\par 43 Apie Jį visi pranašai liudija, kad kiekvienas, kuris Jį tiki, gauna Jo vardu nuodėmių atleidimą”. 
\par 44 Petrui dar tebekalbant šiuos dalykus, Šventoji Dvasia nužengė ant visų, kurie klausėsi žodžio. 
\par 45 Su Petru atvykę žydų kilmės tikintieji labai stebėjosi, kad ir pagonims buvo išlieta Šventosios Dvasios dovana. 
\par 46 Jie girdėjo juos kalbant kalbomis ir aukštinant Dievą. 
\par 47 Tuomet Petras tarė: “Ar kas galėtų uždrausti pasikrikštyti jiems vandeniu­šiems, kurie, kaip ir mes, gavo Šventąją Dvasią?” 
\par 48 Ir jis liepė juos pakrikštyti Viešpaties vardu. Po to jie prašė jį pasilikti dar kelias dienas.


\chapter{11}


\par 1 Apaštalai ir broliai, kurie buvo Judėjoje, išgirdo, kad ir pagonys priėmė Dievo žodį. 
\par 2 Todėl, kai Petras atvyko į Jeruzalę, žydų kilmės tikintieji ėmė ginčytis su juo, sakydami: 
\par 3 “Tu nuėjai pas neapipjaustytus vyrus ir su jais valgei!” 
\par 4 Tada Petras pradėjo jiems iš eilės aiškinti: 
\par 5 “Aš kartą meldžiausi Jopės mieste ir Dvasios pagavoje mačiau regėjimą. Kažkoks indas, tarsi didžiulė marška, už keturių kampų leidžiama iš dangaus, nusileido prie manęs. 
\par 6 Atidžiai įsižiūrėjęs, pamačiau jame keturkojų žemės gyvių, laukinių žvėrių, roplių ir padangės paukščių. 
\par 7 Ir išgirdau balsą, kuris man sakė: ‘Kelkis, Petrai, pjauk ir valgyk!’ 
\par 8 Aš atsakiau: ‘Jokiu būdu, Viešpatie! Dar niekada suteptas ir nešvarus maistas nebuvo mano burnoje’. 
\par 9 Bet balsas iš dangaus prabilo antrą kartą: ‘Ką Dievas apvalė, tu nevadink suteptu!’ 
\par 10 Taip atsitiko tris kartus, ir vėl viskas pakilo į dangų. 
\par 11 Ir štai prie namų, kuriuose aš buvau, tuojau atėjo trys vyrai. Jie buvo siųsti pas mane iš Cezarėjos. 
\par 12 Dvasia man pasakė nė kiek nedvejojant keliauti su jais. Su manimi ėjo ir šitie šeši broliai, ir mes kartu atvykome į vieno vyro namus. 
\par 13 Jis mums papasakojo savo namuose regėjęs stovintį angelą, kuris jam liepė: ‘Nusiųsk į Jopę žmones ir pasikviesk Simoną, vadinamą Petru. 
\par 14 Jis pasakys tau žodžius, kuriais išsigelbėsi tu ir visi tavo namai’. 
\par 15 Kai pradėjau kalbėti, Šventoji Dvasia nužengė ant jų, kaip ir pradžioje ant mūsų. 
\par 16 Tada prisiminiau Viešpaties žodžius: ‘Jonas krikštijo vandeniu, o jūs būsite pakrikštyti Šventąja Dvasia’. 
\par 17 Jeigu tad Dievas suteikė jiems tokią pačią dovaną, kaip ir mums, įtikėjusiems Viešpatį Jėzų Kristų, tai kas gi aš toks, kad galėčiau trukdyti Dievui?!” 
\par 18 Tai išgirdę, jie nusiramino ir šlovino Dievą, sakydami: “Vadinasi, Dievas ir pagonims suteikė atgailą, kad jie gyventų”. 
\par 19 Išblaškytieji persekiojimo, kuris buvo kilęs dėl Stepono, nukeliavo į Finikiją, Kiprą ir Antiochiją. Jie pamokslavo žodį vien tik žydams. 
\par 20 Kai kurie iš jų, būtent kipriečiai ir kirėniečiai, atvykę į Antiochiją, kreipėsi ir į graikus, skelbdami Viešpatį Jėzų. 
\par 21 Viešpaties ranka buvo su jais: didelis žmonių skaičius įtikėjo ir atsivertė į Viešpatį. 
\par 22 Žinia apie juos pasiekė Jeruzalės bažnyčios ausis, ir ji išsiuntė Barnabą į Antiochiją. 
\par 23 Atvykęs ir pamatęs Dievo malonę, jis apsidžiaugė ir visus ragino ryžtinga širdimi pasilikti su Viešpačiu. 
\par 24 Mat jis buvo geras vyras, pilnas Šventosios Dvasios ir tikėjimo. Ir Viešpačiui prisidėjo didelis būrys. 
\par 25 Tada Barnabas nukeliavo į Tarsą ieškoti Sauliaus 
\par 26 ir, radęs jį, atsivedė į Antiochiją. Jiedu ištisus metus darbavosi bažnyčioje ir mokė gausų būrį. Antiochijoje pirmą kartą imta vadinti mokinius “krikščionimis”. 
\par 27 Tomis dienomis iš Jeruzalės į Antiochiją atvyko pranašų. 
\par 28 Vienas iš jų, vardu Agabas, Dvasios įkvėptas, išpranašavo didelį badą, kuris ištiksiąs visą pasaulį. Ir badas atėjo, Klaudijui valdant. 
\par 29 Tada mokiniai, kiekvienas pagal savo išteklius, nusprendė nusiųsti paramą Jeruzalės broliams. 
\par 30 Jie taip ir padarė, per Barnabą bei Saulių nusiųsdami tai vyresniesiems.


\chapter{12}


\par 1 Tuo metu karalius Erodas pakėlė ranką prieš kai kuriuos bažnyčios žmones. 
\par 2 Jis nukirsdino kalaviju Jokūbą, Jono brolį. 
\par 3 Pamatęs, kad tai patinka žydams, įsakė suimti ir Petrą. Buvo Neraugintos duonos dienos. 
\par 4 Suėmęs jį, įmesdino į kalėjimą ir pavedė saugoti keturgubai sargybai po keturis kareivius, o po Paschos ketino išvesti jį prieš minią. 
\par 5 Taigi Petras buvo uždarytas kalėjime. O bažnyčia nepaliaujamai meldėsi už jį Dievui. 
\par 6 Paskutinę naktį prieš Erodui išvedant Petrą, tas, supančiotas dviem grandinėmis, miegojo tarp dviejų kareivių. Prie durų kalėjimą saugojo sargybiniai. 
\par 7 Ir štai ten atsirado Viešpaties angelas, ir kamerą nutvieskė šviesa. Jis sudavė Petrui į šoną ir žadindamas tarė: “Kelkis greičiau!” Ir nukrito jam grandinės nuo rankų. 
\par 8 Angelas kalbėjo toliau: “Susijuosk ir apsiauk sandalus!” Jis taip ir padarė. Angelas tęsė: “Užsimesk apsiaustą ir eik paskui mane!” 
\par 9 Petras išėjo ir sekė paskui jį, nesuvokdamas, kad angelo veiksmai tikri, nes jis tarėsi matąs regėjimą. 
\par 10 Praėję pro pirmą ir antrą sargybą, jie prisiartino prie geležinių vartų į miestą, kurie savaime atsidarė. Išėję pro juos, jie leidosi tolyn viena gatve. Staiga šalia ėjęs angelas nuo jo pasitraukė. 
\par 11 Petras atsipeikėjęs tarė: “Dabar tikrai žinau, kad Viešpats atsiuntė savo angelą ir išvadavo mane iš Erodo rankų ir nuo viso to, ko tikėjosi žydų minia”. 
\par 12 Tai supratęs, jis atėjo prie Morkumi vadinamo Jono motinos Marijos namų, kuriuose daug susirinkusiųjų meldėsi. 
\par 13 Petrui beldžiantis į vartų duris, tarnaitė, vardu Rodė, atėjo paklausti, kas ten. 
\par 14 Pažinusi Petro balsą, ji iš džiaugsmo pamiršo atidaryti vartus, bet nubėgo vidun ir pranešė, jog Petras stovįs už vartų. 
\par 15 Jie jai sakė: “Tu pakvaišai!” Bet ji tvirtino savo. Tada jie tarė: “Tai jo angelas”. 
\par 16 Tuo tarpu Petras toliau beldė. Atidarę jie pamatė Petrą ir nustėro. 
\par 17 Ranka davęs ženklą laikytis tyliai, jis jiems papasakojo, kaip Viešpats išvedė jį iš kalėjimo. Jis dar pridūrė: “Praneškite apie tai Jokūbui ir kitiems broliams”. Ir išėjęs jis nuėjo į kitą vietą. 
\par 18 Išaušus dienai tarp kareivių kilo nemenkas sąmyšis dėl to, kas galėję nutikti Petrui. 
\par 19 Erodas, paieškojęs jo ir neradęs, ištardė sargybinius ir įsakė nubausti juos mirtimi. Po to iš Judėjos nuvyko į Cezarėją ir ten pasiliko. 
\par 20 Erodas nirto ant Tyro ir Sidono gyventojų. Bet jie susitarę atvyko pas jį ir, palenkę savo pusėn karaliaus rūmininką Blastą, prašė taikos, nes jų kraštas maitinosi iš karaliaus. 
\par 21 Nustatytą dieną Erodas, apsivilkęs karališkais drabužiais, atsisėdo į sostą ir sakė jiems prakalbą. 
\par 22 Liaudis ėmė šaukti: “Tai dievo, ne žmogaus balsas!” 
\par 23 Ir beregint jį ištiko Viešpaties angelas, kad neatidavė Dievui garbės. Ir jis mirė, kirminų suėstas. 
\par 24 O Viešpaties žodis augo ir plito. 
\par 25 Barnabas ir Saulius, atlikę savo uždavinį ir paėmę su savimi Joną, vadinamą Morkumi, sugrįžo iš Jeruzalės.


\chapter{13}


\par 1 Antiochijos bažnyčioje buvo pranašų ir mokytojų: Barnabas, Simeonas, pravarde Juodasis, Lucijus Kirėnietis, Manaenas, augęs kartu su tetrarchu Erodu, ir Saulius. 
\par 2 Kartą, kai jie tarnavo Viešpačiui ir pasninkavo, Šventoji Dvasia pasakė: “Išskirkite man Barnabą ir Saulių darbui, kuriam Aš juos pašaukiau”. 
\par 3 Tuomet jie pasninkavo ir meldėsi, ir, uždėję ant jų rankas, išleido. 
\par 4 Šventosios Dvasios pasiųsti, jie nukeliavo į Seleukiją, o iš ten laivu pasiekė Kiprą. 
\par 5 Atvykę į Salaminą, jie pamokslavo Dievo žodį žydų sinagogose. Jiems talkino Jonas. 
\par 6 Perėję visą salą iki Pafo, jie susitiko vieną magą, netikrą pranašą žydą, vardu Barjėzų, 
\par 7 kuris buvo su prokonsulu Sergijumi Pauliumi, išmintingu vyru. Šis, pasikvietęs Barnabą ir Saulių, norėjo pasiklausyti Dievo žodžio. 
\par 8 Bet Elimas­magas (toks šito žodžio vertimas)­jiems priešinosi, stengdamasis atitraukti prokonsulą nuo tikėjimo. 
\par 9 Tada Saulius, kitaip vadinamas Pauliumi, Šventosios Dvasios kupinas, įdėmiai pažvelgė į jį 
\par 10 ir tarė: “Ak tu, visokių klastų bei piktybių pilnas velnio vaike! Tu, teisumo prieše! Ar nesiliausi kraipęs tiesių Viešpaties kelių? 
\par 11 Dabar štai tave ištiks Viešpaties ranka: tu tapsi aklas ir kurį laiką neregėsi saulės”. Ir tučtuojau užgulė jį migla ir tamsa, ir jis ėmė graibytis aplink, ieškodamas, kas jam ištiestų ranką. 
\par 12 Tai pamatęs, prokonsulas įtikėjo, apstulbintas Viešpaties mokslo. 
\par 13 Išplaukę iš Pafo, Paulius ir jo palydovai atvyko į Pergę, Pamfilijoje. Čia Jonas pasitraukė nuo jų ir sugrįžo į Jeruzalę. 
\par 14 O jie, išvykę iš Pergės, atkeliavo į Pisidijos Antiochiją. Sabato dieną nuėjo į sinagogą ir ten susėdo. 
\par 15 Po Įstatymo ir Pranašų skaitymo sinagogos vadovai kreipėsi į juos: “Vyrai broliai, jei turite paskatinimo žodį tautiečiams, tarkite!” 
\par 16 Tada Paulius atsistojo ir, davęs ranka ženklą nutilti, prabilo: “Izraelio vyrai ir Dievo bijantys žmonės, paklausykite! 
\par 17 Šios Izraelio tautos Dievas išsirinko mūsų tėvus, išaukštino juos, gyvenančius kaip ateivius Egipto šalyje, ir iškelta ranka išvedė juos iš Egipto. 
\par 18 Apie keturiasdešimt metų Jis pakentė jų elgesį dykumoje, 
\par 19 paskui, išnaikinęs septynias tautas Kanaane, išdalino jiems anų žemę. 
\par 20 Po to per keturis šimtus penkiasdešimt metų davė teisėjus iki pranašo Samuelio. 
\par 21 Tuo laiku jie ėmė prašyti karaliaus, ir Dievas jiems davė Saulių, Kišo sūnų, Benjamino giminės vyrą, keturiasdešimčiai metų. 
\par 22 Pašalinęs jį, Jis pakėlė jiems karaliumi Dovydą, apie kurį liudydamas pasakė: ‘Radau Dovydą, Jesės sūnų, vyrą pagal savo širdį, kuris įvykdys visus mano norus’. 
\par 23 Iš jo palikuonių, kaip buvo žadėjęs, Dievas iškėlė Izraeliui Gelbėtoją Jėzų. 
\par 24 Prieš Jam ateinant, Jonas skelbė atgailos krikštą visai Izraelio tautai. 
\par 25 Baigdamas gyvenimo kelią, Jonas pasakė: ‘Kuo jūs mane laikote? Aš nesu Jis. Bet štai po manęs ateina Tas, kuriam aš nevertas atrišti kojų apavo’. 
\par 26 Vyrai broliai, Abraomo giminės sūnūs ir čia esantys Dievo bijantys žmonės! Tai jums atsiųstas šis išgelbėjimo žodis. 
\par 27 Nes Jeruzalės gyventojai ir jų vyresnieji nepažino Jo ir pasmerkdami įvykdė pranašų žodžius, skaitomus kiekvieną sabatą. 
\par 28 Nors nerado Jame jokios mirties vertos kaltės, jie reikalavo iš Piloto, kad Jis būtų nužudytas. 
\par 29 Išpildę visa, kas buvo apie Jį parašyta, nuėmė Jį nuo medžio ir paguldė į kapą. 
\par 30 Bet Dievas Jį prikėlė iš numirusių. 
\par 31 Jis per daugelį dienų rodėsi tiems, kurie buvo su Juo atėję iš Galilėjos į Jeruzalę. Jie yra Jo liudytojai žmonėms. 
\par 32 Ir mes jums skelbiame Gerąją naujieną: tėvams duotąjį pažadą 
\par 33 Dievas įvykdė mums, jų vaikams, prikeldamas Jėzų, kaip ir parašyta antroje psalmėje: ‘Tu esi mano Sūnus, šiandien Aš pagimdžiau Tave!’ 
\par 34 O kad Jį prikėlė iš numirusių ir kad Jis neturėjo supūti, Dievas buvo taip nusakęs: ‘Aš ištesėsiu šventus bei tikrus pažadus Dovydui’. 
\par 35 Todėl ir kitoje vietoje sakoma: ‘Tu neleisi savo Šventajam matyti supuvimo’. 
\par 36 Juk Dovydas, patarnavęs savajai kartai pagal Dievo valią, užmigo, buvo palaidotas prie savo tėvų ir supuvo. 
\par 37 O Tas, kurį Dievas prikėlė, nesupuvo. 
\par 38 Taigi tebūnie jums žinoma, vyrai broliai, kad per Jį jums skelbiamas nuodėmių atleidimas. 
\par 39 Ir kiekvienas, kuris tiki, išteisinamas Juo nuo viso to, nuo ko nepajėgė jūsų išteisinti Mozės Įstatymas. 
\par 40 Tad saugokitės, kad jums nepritaptų, kas pasakyta Pranašuose: 
\par 41 ‘Žiūrėkite, niekintojai, stebėkitės ir pranykite! Nes Aš darau darbą jūsų dienomis, darbą, kuriuo nepatikėsite, jei kas jums pasakos’ ”. 
\par 42 Žydams išeinant iš sinagogos, pagonys prašė, kad ir kitą sabatą būtų kalbama apie tuos dalykus. 
\par 43 Sinagogos susirinkimui pasibaigus, daugelis žydų ir pamaldžių prozelitų sekė Paulių ir Barnabą, kurie, kalbėdami jiems, įtikinėjo pasilikti Dievo malonėje. 
\par 44 Kitą sabatą beveik visas miestas susirinko pasiklausyti Dievo žodžio. 
\par 45 Išvydusius tokias minias žydus apėmė pavydas, ir jie piktžodžiaudami ėmė prieštarauti tam, ką kalbėjo Paulius. 
\par 46 Bet Paulius ir Barnabas drąsiai pasakė: “Pirmiausia jums turėjo būti skelbiamas Dievo žodis. Bet kadangi jūs Jį atmetate ir patys laikote save nevertais amžinojo gyvenimo, tai štai mes kreipiamės į pagonis. 
\par 47 Nes taip mums liepė Viešpats: ‘Paskyriau Tave, kad būtum šviesa pagonims, kad būtumei išgelbėjimu iki pat žemės pakraščių’ ”. 
\par 48 Tai girdėdami, pagonys džiaugėsi ir šlovino Viešpaties žodį; ir įtikėjo visi skirtieji amžinajam gyvenimui. 
\par 49 Ir Viešpaties žodis pasklido po visą kraštą. 
\par 50 Bet žydai, sukurstę pamaldžias bei gerbiamas moteris ir įtakingus miesto piliečius, sukėlė Pauliaus ir Barnabo persekiojimą ir išvijo juos iš savo žemių. 
\par 51 O tie, nusikratę prieš juos nuo kojų dulkes, atvyko į Ikonijų. 
\par 52 Mokiniai buvo pilni džiaugsmo ir Šventosios Dvasios.


\chapter{14}


\par 1 Taip pat atsitiko ir Ikonijuje: jie nuėjo į žydų sinagogą ir kalbėjo taip, kad įtikėjo didelė minia žydų ir graikų. 
\par 2 Bet tikėjimui nepaklusę žydai sukurstė pagonis prieš brolius. 
\par 3 Tačiau jie ten išbuvo daug laiko ir drąsiai kalbėjo Viešpatyje, kuris liudijo savo malonės žodžius ir per jų rankas darė ženklus ir stebuklus. 
\par 4 Miesto gyventojai suskilo: vieni buvo už žydus, kiti už apaštalus. 
\par 5 Kai pagonys ir žydai su savo vyresnybe susiruošė juos išniekinti ir užmėtyti akmenimis, 
\par 6 jie sužinoję pabėgo į Likaonijos miestus Listrą ir Derbę bei jų apylinkes. 
\par 7 Ten jie skelbė Evangeliją. 
\par 8 Listroje gyveno vienas vyras nesveikomis kojomis. Jis buvo luošas nuo pat gimimo, niekada nė žingsnio nežengęs. 
\par 9 Jis klausėsi Pauliaus kalbant, o šis įdėmiai pažvelgė į jį ir, pamatęs jį turint tikėjimą, kad būtų pagydytas, 
\par 10 garsiu balsu pasakė: “Atsistok tiesiai ant savo kojų!” Tas pašoko ir ėmė vaikščioti. 
\par 11 Minia, pamačiusi, ką Paulius padarė, pradėjo garsiai likaoniškai šaukti: “Dievai, pasivertę žmonėmis, nužengė pas mus!” 
\par 12 Barnabą jie vadino Dzeusu, o Paulių­Hermiu, nes jis vadovavo kalbai. 
\par 13 Priešais jų miestą esančios Dzeuso šventyklos kunigas atvarė prie vartų jaučių su vainikais ir norėjo kartu su minia juos paaukoti. 
\par 14 Sužinoję apaštalai Barnabas ir Paulius perplėšė savo drabužius ir puolė į minią, 
\par 15 šaukdami: “Vyrai, ką darote?! Juk mes tokie patys žmonės kaip ir jūs. Ir mes jums skelbiame Gerąją naujieną, kad nuo šių tuštybių atsiverstumėte į gyvąjį Dievą, ‘kuris sutvėrė dangų ir žemę, jūrą ir visa, kas juose yra’. 
\par 16 Praėjusiais amžiais Jis leido eiti visoms tautoms savais keliais. 
\par 17 Tačiau Jis nepaliko savęs nepaliudyto, darydamas gera, duodamas mums lietaus iš dangaus ir vaisingų metų, pripildydamas mūsų širdis maisto ir džiaugsmo”. 
\par 18 Tai dėstydami, jiedu šiaip ne taip sulaikė minią, kad jiems neaukotų. 
\par 19 Bet iš Antiochijos ir Ikonijaus atvyko žydai ir, perkalbėję žmones, užmėtė Paulių akmenimis ir išvilko už miesto, palaikę jį mirusiu. 
\par 20 Susirinkus aplink jį mokiniams, jis atsikėlė ir parėjo į miestą. O rytojaus dieną kartu su Barnabu iškeliavo į Derbę. 
\par 21 Šitame mieste jie skelbė Evangeliją ir išmokė daugelį. Paskui grįžo atgal į Listrą, Ikonijų ir Antiochiją. 
\par 22 Ten jie stiprino mokinių sielas, ragino juos pasilikti tikėjime ir sakė: “Per daugelį išmėginimų mes turime įeiti į Dievo karalystę”. 
\par 23 Kiekvienoje bažnyčioje su malda ir pasninku, uždėdami rankas, jie paskyrė jiems vyresniuosius ir pavedė juos Viešpačiui, kurį šie buvo įtikėję. 
\par 24 Apkeliavę Pisidiją, jie atvyko į Pamfiliją. 
\par 25 Paskelbę žodį Pergėje, leidosi žemyn į Ataliją. 
\par 26 Iš ten išplaukė į Antiochiją, kur buvo pavesti Dievo malonei, kad nuveiktų darbą, kurį dabar pabaigė. 
\par 27 Sugrįžę jie sušaukė bažnyčią ir papasakojo, kokius darbus nuveikęs Dievas per juos ir kaip atvėręs pagonims tikėjimo vartus. 
\par 28 Ir jie išbuvo su mokiniais netrumpą laiką.


\chapter{15}


\par 1 Kai kurie, atvykę iš Judėjos, ėmė mokyti brolius: “Jei nesiduosite apipjaustomi pagal Mozės paprotį, negalėsite būti išgelbėti”. 
\par 2 Kilo nemažas vaidas ir ginčas tarp jų ir Pauliaus bei Barnabo. Buvo nutarta, kad Paulius, Barnabas ir kai kurie kiti iš anų nuvyktų dėl šio klausimo į Jeruzalę pas apaštalus ir vyresniuosius. 
\par 3 Bažnyčios aprūpinti, jie iškeliavo per Finikiją ir Samariją, pasakodami apie pagonių atsivertimą, ir tuo padarė didelį džiaugsmą visiems broliams. 
\par 4 Atvykę į Jeruzalę, jie buvo priimti bažnyčios, apaštalų bei vyresniųjų ir pranešė jiems, ką Dievas kartu su jais nuveikė. 
\par 5 Tuomet pakilo kai kurie įtikėjusieji iš fariziejų partijos ir tarė: “Juos reikia apipjaustyti ir įsakyti, kad laikytųsi Mozės Įstatymo”. 
\par 6 Apaštalai ir vyresnieji susirinko šio klausimo apsvarstyti. 
\par 7 Įsiliepsnojus ilgam ginčui, Petras pakilo ir kreipėsi į juos: “Vyrai broliai, jūs žinote, kad Dievas jau nuo senų dienų išsirinko mane iš jūsų, kad pagonys iš mano lūpų išgirstų Evangelijos žodį ir įtikėtų. 
\par 8 Ir Dievas, kuris pažįsta žmonių širdis, paliudijo jų naudai, duodamas jiems Šventąją Dvasią kaip ir mums. 
\par 9 Jis nepadarė skirtumo tarp mūsų ir jų, tikėjimu nuskaistindamas jų širdis. 
\par 10 Tad kodėl gundote Dievą ir kraunate ant mokinių sprando jungą, kurio nei mūsų tėvai, nei mes patys negalėjome panešti? 
\par 11 Juk mes tikime, kad Viešpaties Jėzaus Kristaus malone būsime išgelbėti kaip ir jie”. 
\par 12 Tada visas susirinkimas nutilo ir ėmė klausytis Barnabo bei Pauliaus pasakojimo, kokių ženklų ir stebuklų per juos Dievas padarė tarp pagonių. 
\par 13 Kai jie nutilo, atsiliepė Jokūbas ir tarė: “Vyrai broliai, paklausykite manęs! 
\par 14 Simonas papasakojo, kaip Dievas pirmą kartą aplankė pagonis, kad išsirinktų iš jų savo vardui žmones. 
\par 15 Čia dera pranašų žodžiai, kaip parašyta: 
\par 16 ‘Paskui sugrįšiu ir vėl atstatysiu sugriuvusią Dovydo palapinę. Aš prikelsiu ją iš griuvėsių ir vėl ją išskleisiu, 
\par 17 kad ieškotų Viešpaties ir visi kiti žmonės, visi pagonys, kuriems skelbiamas mano vardas,­sako Viešpats, visa tai darantis’. 
\par 18 Nuo amžių Dievui žinomi visi Jo darbai. 
\par 19 Todėl aš manau, kad į Dievą atsivertusių pagonių nereikia apsunkinti, 
\par 20 o tik jiems parašyti, jog susilaikytų nuo susiteršimo stabais, nuo ištvirkavimo, pasmaugtų gyvulių mėsos ir kraujo. 
\par 21 Juk Mozė kiekviename mieste nuo senų laikų turi savo skelbėjų ir kas sabatą yra skaitomas sinagogose”. 
\par 22 Tada apaštalai ir vyresnieji kartu su visa bažnyčia nutarė pasiųsti į Antiochiją iš savųjų išrinktus vyrus kartu su Pauliumi ir Barnabu: Judą, vadinamą Barsabu, ir Silą, kurie buvo vadovaujantys tarp brolių. 
\par 23 Jiems įteikė tokį raštą: “Apaštalai, vyresnieji ir broliai siunčia sveikinimą broliams, kilusiems iš pagonių Antiochijoje, Sirijoje bei Kilikijoje. 
\par 24 Sužinoję, kad kai kurie iš mūsų nuvykę savo kalbomis sukėlė jums nerimo ir sujaukė jūsų sielas, liepdami apsipjaustyti ir laikytis Įstatymo, 
\par 25 mes susirinkę vieningai nusprendėme pasiųsti pas jus išrinktus vyrus su mūsų mylimaisiais Barnabu ir Pauliumi, 
\par 26 kurie už mūsų Viešpatį Jėzų Kristų yra guldę savo galvas. 
\par 27 Taigi siunčiame su jais Judą ir Silą, kurie jums tą patį praneš žodžiu. 
\par 28 Šventajai Dvasiai ir mums pasirodė teisinga neužkrauti jums daugiau naštų, išskyrus tai, kas būtina: 
\par 29 susilaikyti nuo aukų stabams, kraujo, pasmaugtų gyvulių mėsos ir ištvirkavimo. Jūs gerai elgsitės, saugodamiesi šitų dalykų. Likite sveiki!” 
\par 30 Išsiųstieji, atvykę į Antiochiją, sukvietė bendruomenę ir įteikė laišką. 
\par 31 Skaitydami šie džiaugėsi paguoda. 
\par 32 Judas ir Silas, būdami pranašai, gausiais žodžiais skatino ir stiprino brolius. 
\par 33 Pabuvę ten kurį laiką, jie buvo brolių išleisti ramybėje pas apaštalus. 
\par 34 Bet Silas nusprendė ten pasilikti. 
\par 35 Paulius su Barnabu taip pat pasiliko Antiochijoje, kartu su daugeliu kitų mokydami ir pamokslaudami Viešpaties žodį. 
\par 36 Po kiek laiko Paulius tarė Barnabui: “Grįžkime, aplankykime brolius visuose miestuose, kur skelbėme Viešpaties žodį, ir pažiūrėkime, kaip jiems sekasi”. 
\par 37 Barnabas norėjo pasiimti kartu ir Joną, vadinamą Morkumi, 
\par 38 bet Paulius manė būsiant geriau neimti tokio, kuris Pamfilijoje buvo nuo jų pasitraukęs ir nesidarbavo su jais. 
\par 39 Kilo toks smarkus ginčas, kad jie vienas nuo kito atsiskyrė. Barnabas, pasiėmęs Morkų, išplaukė į Kiprą, 
\par 40 o Paulius, pasirinkęs Silą, iškeliavo, brolių patikėtas Viešpaties malonei. 
\par 41 Jis leidosi per Siriją ir Kilikiją, stiprindamas bažnyčias.


\chapter{16}


\par 1 Paulius atkeliavo į Derbę ir Listrą. Ir štai ten buvo vienas mokinys, vardu Timotiejus, kurio motina buvo įtikėjusi žydė, o tėvas­graikas. 
\par 2 Apie jį gerai liudijo Listros ir Ikonijaus broliai. 
\par 3 Paulius panorėjo jį pasiimti su savimi. Jis apipjaustydino jį dėl žydų, gyvenančių tame krašte. Mat visi žinojo jo tėvą esant graiką. 
\par 4 Keliaudami per miestus, jie liepdavo tikintiesiems laikytis Jeruzalės apaštalų bei vyresniųjų priimtų nutarimų. 
\par 5 Taip bažnyčios stiprėjo tikėjimu ir kasdien augo skaičiumi. 
\par 6 Jiems perėjus Frygiją ir Galatijos šalį, Šventoji Dvasia draudė jiems skelbti žodį Azijoje. 
\par 7 Atvykę iki Myzijos, jie mėgino eiti į Bitiniją, tačiau Dvasia jų neleido. 
\par 8 Perėję Myziją, jie nuėjo į Troadę. 
\par 9 Čia Paulius naktį turėjo regėjimą: jam pasirodė makedonietis, kuris maldavo: “Ateik į Makedoniją ir padėk mums!” 
\par 10 Po šio regėjimo mes nedelsdami pasistengėme išvykti į Makedoniją, įsitikinę, kad Viešpats mus pašaukė skelbti jiems Evangeliją. 
\par 11 Išplaukę iš Troadės, leidomės tiesiog į Samotrakę ir rytojaus dieną į Neapolį. 
\par 12 Iš čia atvykome į Filipus­pirmąjį šios Makedonijos dalies ir kolonijos miestą. Šiame mieste užtrukome kelias dienas. 
\par 13 Sabato dieną išėjome už miesto prie upės, kur pagal paprotį buvo maldos vieta, ir atsisėdę kalbėjome susirinkusioms moterims. 
\par 14 Viena dievobaiminga moteris, vardu Lidija, prekiaujanti purpuro drabužiais, kilusi iš Tiatyrų miesto, klausėsi, ir Viešpats atvėrė jos širdį tam, ką kalbėjo Paulius. 
\par 15 Kai ji su savo namiškiais buvo pakrikštyta, ėmė mūsų prašyti: “Jei mane laikote Viešpaties tikinčiąja, ateikite ir pasilikite mano namuose”. Ji tiesiog mus privertė. 
\par 16 Kartą, einančius į maldos vietą, mus pasitiko viena tarnaitė, turinti spėjimo dvasią. Spėdama ji daug uždirbdavo savo šeimininkams. 
\par 17 Ji ėmė sekti paskui Paulių bei mus, šaukdama: “Šitie vyrai yra aukščiausiojo Dievo tarnai ir skelbia mums išgelbėjimo kelią”. 
\par 18 Taip ji darė daugelį dienų. Nebeapsikęsdamas Paulius atsigręžė ir paliepė dvasiai: “Jėzaus Kristaus vardu įsakau tau iš jos išeiti!” Ir dvasia tučtuojau išėjo. 
\par 19 Šeimininkai, pamatę, kad jų pasipelnymo viltys žlugo, sugriebė Paulių bei Silą ir nutempė į miesto aikštę pas vyresnybę. 
\par 20 Nuvedę pas pretorius, jie tarė: “Šitie žmonės mūsų mieste kelia sąmyšį. Jie yra žydai 
\par 21 ir skelbia papročius, kurių mums, romėnams, nevalia nei priimti, nei laikytis”. 
\par 22 Prieš juos sukilo ir minia. Pretoriai nuplėšė nuo jų drabužius ir įsakė juos mušti lazdomis. 
\par 23 Davę daug kirčių, įmetė juos į kalėjimą, įsakydami kalėjimo prižiūrėtojui rūpestingai saugoti. 
\par 24 Gavęs tokį įsakymą, kalėjimo prižiūrėtojas įgrūdo juos į vidinę kamerą, o jų kojas įtvėrė į šiekštą. 
\par 25 Apie vidurnaktį Paulius ir Silas meldėsi ir giedojo Dievui himnus. Kiti kaliniai jų klausėsi. 
\par 26 Staiga kilo toks stiprus žemės drebėjimas, jog kalėjimo pamatai susvyravo. Bematant atsivėrė visos durys, ir visiems nukrito pančiai. 
\par 27 Kalėjimo prižiūrėtojas nubudo ir, pamatęs atviras kalėjimo duris, išsitraukė kalaviją norėdamas nusižudyti: jis pamanė, kad kaliniai pabėgo. 
\par 28 Bet Paulius garsiai sušuko: “Nedaryk sau pikto! Mes visi esame čia”. 
\par 29 Paprašęs šviesos, jis šoko vidun ir, visas drebėdamas, puolė Pauliui ir Silui po kojų. 
\par 30 Po to išvedė juos laukan ir paklausė: “Gerbiamieji, ką turiu daryti, kad būčiau išgelbėtas?” 
\par 31 Jie atsakė: “Tikėk Viešpatį Jėzų Kristų ir būsi išgelbėtas tu ir tavo namai”. 
\par 32 Ir jie skelbė Viešpaties žodį jam ir jo namiškiams. 
\par 33 Tą pačią nakties valandą jis pasiėmė juos, nuplovė jų žaizdas ir nedelsiant kartu su visais saviškiais buvo pakrikštytas. 
\par 34 Nusivedęs į savo namus, jis padengė jiems stalą ir su visais namiškiais džiūgavo, įtikėjęs Dievą. 
\par 35 Dienai išaušus, pretoriai pasiuntė liktorius pranešti kalėjimo viršininkui: “Paleisk tuos žmones!” 
\par 36 Tas perdavė tuos žodžius Pauliui: “Pretoriai liepia jus paleisti. Galite eiti ir ramiai keliauti”. 
\par 37 Bet Paulius jiems atsakė: “Mus, Romos piliečius, nenuteistus viešai mušė ir įmetė į kalėjimą, o dabar nori slapta paleisti?! Šito nebus! Tegul patys ateina ir išveda!” 
\par 38 Liktoriai perdavė pretoriams šį atsakymą. Išgirdę, kad tai Romos piliečiai, pretoriai išsigando. 
\par 39 Jie atėjo, atsiprašė ir išsivedę prašė, kad paliktų miestą. 
\par 40 Išėję iš kalėjimo, Paulius ir Silas užsuko pas Lidiją, pasimatė su broliais, padrąsino juos ir iškeliavo toliau.


\chapter{17}


\par 1 Perėję Amfipolį ir Apoloniją, jie atvyko į Tesaloniką, kur buvo žydų sinagoga. 
\par 2 Pagal savo įprotį Paulius užėjo pas juos ir tris sabatus aiškinosi su jais Raštus, 
\par 3 dėstydamas ir įrodinėdamas, kad Kristus turėjo kentėti ir prisikelti iš numirusių ir kad: “Kristus­tai Jėzus, kurį aš jums skelbiu”. 
\par 4 Kai kurie iš jų įtikėjo ir prisidėjo prie Pauliaus ir Silo, taip pat daugybė pamaldžių graikų ir nemaža aukštos kilmės moterų. 
\par 5 Bet neįtikėję žydai, apimti pavydo, surinko iš gatvės piktų žmonių, sukurstė minią ir sukėlė mieste sąmyšį. Jie užpuolė Jasono namus ir ėmė ten ieškoti Pauliaus ir Silo, norėdami išvesti juos prieš minią. 
\par 6 Jų neradę, nusitempė Jasoną ir kelis brolius pas miesto vadovus, šaukdami: “Tie žmonės, kurie verčia visą pasaulį aukštyn kojom, atvyko ir čia, 
\par 7 o Jasonas juos priglaudė. Visi šitie laužo ciesoriaus įsakymus, tvirtindami, jog esąs kitas karalius, būtent Jėzus”. 
\par 8 Taip jie sukėlė nerimo jų klausiusiai miniai ir miesto vadovams. 
\par 9 Šie, gavę iš Jasono ir kitų užstatą, paleido juos. 
\par 10 Broliai tuojau pat, nakčia, išsiuntė Paulių ir Silą į Berėją. Ten atvykę, jie užėjo į žydų sinagogą. 
\par 11 Tenykščiai pasirodė esą kilnesni už tesalonikiečius. Jie noriai priėmė žodį ir kasdien tyrinėjo Raštus, ar taip esą iš tikrųjų. 
\par 12 Daugelis iš jų įtikėjo, taip pat nemažai kilmingų graikų moterų ir vyrų. 
\par 13 Sužinoję, kad Paulius jau Berėjoje skelbia Dievo žodį, Tesalonikos žydai atskubėjo ir čia drumsti ir kurstyti žmonių. 
\par 14 Tada broliai skubiai išsiuntė Paulių prie jūros, o Silas ir Timotiejus pasiliko ten. 
\par 15 Pauliaus palydovai nuvedė jį iki Atėnų. Gavę įsakymą pranešti Silui ir Timotiejui, kad kuo greičiau atvyktų pas jį, grįžo atgal. 
\par 16 Belaukdamas Atėnuose jų atvykstant ir matydamas pilną miestą stabų, Paulius dvasioje netvėrė pykčiu. 
\par 17 Jis aiškinosi sinagogoje su žydais ir pamaldžiaisiais, o aikštėje kasdien su tais, kurie ten būdavo. 
\par 18 Kai kurie epikūrininkų ir stoikų filosofai bandė su juo ginčytis. Vieni klausė: “Ką šis plepys nori pasakyti?” O kiti: “Atrodo, kad jis svetimų demonų skelbėjas”. Mat jis skelbė Jėzų ir prisikėlimą. 
\par 19 Jie paėmė, nusivedė jį į Areopagą ir tarė jam: “Ar galėtume sužinoti, kokį naują mokslą tu skelbi? 
\par 20 Nes tu kalbi mūsų ausims negirdėtus dalykus. Taigi norėtume sužinoti, kas tai būtų”. 
\par 21 Mat visi atėniečiai ir ten gyvenantys ateiviai leisdavo laiką ne kaip kitaip, o tik pasakodami arba klausydami ką nors nauja. 
\par 22 Tada Paulius, atsistojęs Areopago viduryje, prabilo: “Atėnų vyrai, matau, kad jūs visais atžvilgiais labai religingi. 
\par 23 Vaikščiodamas ir apžiūrinėdamas jūsų šventenybes, radau aukurą su užrašu: ‘Nežinomam dievui’. Taigi Tą, kurį nepažindami garbinate, aš jums ir skelbiu. 
\par 24 Dievas, pasaulio ir visko, kas jame yra, Kūrėjas, būdamas dangaus ir žemės Viešpats, negyvena žmonių rankomis statytose šventyklose 
\par 25 ir nėra žmonių rankomis aptarnaujamas, tarsi Jam ko nors trūktų. Jis gi pats visiems duoda gyvybę, alsavimą ir visa kita. 
\par 26 Iš vieno kraujo Jis išvedė visas žmonių tautas, kad šios gyventų visoje žemėje. Jis nustatė iš anksto paskirtus laikus ir apsigyvenimo ribas, 
\par 27 kad žmonės ieškotų Viešpaties ir tartum apgraibomis Jį atrastų, nors Jis netoli nuo kiekvieno iš mūsų. 
\par 28 Juk mes Jame gyvename, judame ir esame, kaip yra pasakę ir kai kurie jūsų poetai: ‘Mes irgi esame kilę iš Jo’. 
\par 29 Todėl, būdami dieviškos kilmės, neturime manyti, jog Dievybė yra panaši į auksą, sidabrą ar akmenį, įgavusį pavidalą žmogaus sumanymų ir sugebėjimų dėka. 
\par 30 Ir štai Dievas nebežiūri anų neišmanymo laikų ir dabar įsako visur visiems žmonėms atgailauti, 
\par 31 nes Jis nustatė dieną, kada teisingai teis visą pasaulį per žmogų, kurį paskyrė ir visiems už Jį laidavo, prikeldamas Jį iš numirusių”. 
\par 32 Išgirdę apie prisikėlimą iš numirusių, vieni ėmė šaipytis, o kiti sakė: “Apie tai tavęs paklausysime kitą kartą”. 
\par 33 Šitaip Paulius paliko jų būrį. 
\par 34 Vis dėlto kai kurie vyrai prisidėjo prie jo ir įtikėjo. Iš jų Dionizas, Areopago narys, viena moteris, vardu Damaridė, ir jų draugai.


\chapter{18}


\par 1 Po viso šito Paulius iškeliavo iš Atėnų ir nuvyko į Korintą. 
\par 2 Čia jis sutiko vieną iš Ponto kilusį žydą, vardu Akvilą, su žmona Priscile, neseniai atsikėlusius iš Italijos. Mat Klaudijus buvo išleidęs įsakymą visiems žydams išsikraustyti iš Romos. Jis nuėjo pas juos 
\par 3 ir, kadangi mokėjo bendrą amatą, pasiliko ten ir dirbo. Jie vertėsi palapinių audimu. 
\par 4 Kiekvieną sabatą Paulius kalbėdavo sinagogoje, įtikinėdamas žydus ir graikus. 
\par 5 Atvykus iš Makedonijos Silui ir Timotiejui, Paulius buvo Dvasios paragintas ir liudijo žydams, kad Jėzus yra Kristus. 
\par 6 Bet jiems prieštaraujant ir piktžodžiaujant, jis nusipurtė drabužius ir tarė: “Jūsų kraujas tekrinta ant jūsų pačių galvų! Aš nekaltas ir nuo šiol eisiu pas pagonis”. 
\par 7 Ir, pasitraukęs iš ten, persikėlė į vieno dievobaimingo vyro, vardu Ticijaus Justo, namus, buvusius šalia sinagogos. 
\par 8 Sinagogos vyresnysis Krispas su visais savo namiškiais įtikėjo Viešpatį. Ir daugelis korintiečių klausydamiesi įtikėjo ir buvo pakrikštyti. 
\par 9 Naktį Viešpats regėjime prabilo Pauliui: “Nebijok! Kalbėk ir netylėk! 
\par 10 Aš esu su tavimi ir niekas nesikėsins tau kenkti, nes šiame mieste daugel žmonių­manieji”. 
\par 11 Ir jis ten pasiliko metus ir šešis mėnesius, mokydamas juos Dievo žodžio. 
\par 12 Galionui būnant Achajos prokonsulu, žydai visi kaip vienas sukilo prieš Paulių, nusitempė jį į teismo vietą 
\par 13 ir pareiškė: “Šitas įtikinėja žmones garbinti Dievą Įstatymui priešingu būdu”. 
\par 14 Pauliui rengiantis prabilti, Galionas tarė žydams: “Jei tai būtų koks nusikaltimas ar piktas darbas, tuomet, žydų vyrai, būtų dėl ko jus išklausyti. 
\par 15 Bet kai kyla ginčas apie mokymą, vardus bei jūsų Įstatymą, spręskite patys; tokių dalykų teisėjas būti nenoriu”. 
\par 16 Ir pavarė juos nuo teismo krasės. 
\par 17 Tada visi graikai užpuolė sinagogos vyresnįjį Sosteną ir sumušė jį prie teismo krasės. Galionas į tai nekreipė dėmesio. 
\par 18 Paulius, išbuvęs Korinte dar netrumpą laiką, atsisveikino su broliais ir kartu su Priscile bei Akvilu išplaukė į Siriją. Kenchrėjoje jis nusiskuto galvą, nes padarė įžadą. 
\par 19 Atvykęs į Efezą, jis paliko juos, o pats nuėjęs į sinagogą, ėmė kalbėtis su žydais. 
\par 20 Jie prašė jį pasilikti ilgesnį laiką, bet jis nesutiko 
\par 21 ir atsisveikindamas tarė: “Aš būtinai turiu praleisti ateinančią šventę Jeruzalėje. Jei Dievas panorės, vėl atvyksiu pas jus”. Ir iškeliavo iš Efezo. 
\par 22 Atplaukęs į Cezarėją, jis patraukė aukštyn, pasveikino bažnyčią ir išvyko į Antiochiją. 
\par 23 Pabuvęs ten kiek laiko, vėl leidosi į kelionę ir ėjo per Galatijos kraštą bei Frygiją, stiprindamas visus mokinius. 
\par 24 Vienas žydas, vardu Apolas, kilęs iš Aleksandrijos, iškalbingas ir puikiai išmanantis Raštus vyras, atvyko į Efezą. 
\par 25 Jis buvo pamokytas Viešpaties kelio ir, degdamas dvasia, kalbėjo ir uoliai mokė apie Viešpatį, nors tepažinojo Jono krikštą. 
\par 26 Jis ėmė drąsiai skelbti sinagogoje. Išgirdę jį, Priscilė ir Akvilas pasikvietė pas save ir nuodugniau išaiškino jam Dievo kelią. 
\par 27 Kai jis panoro vykti į Achają, broliai parašė mokiniams, ragindami jį priimti. Nuvykęs tenai, jis buvo labai naudingas per malonę įtikėjusiesiems. 
\par 28 Mat jis viešumoje smarkiai sukirsdavo žydus, iš Raštų įrodydamas Jėzų esant Kristų.


\chapter{19}


\par 1 Apolui būnant Korinte, Paulius, perėjęs aukštutines sritis, atvyko į Efezą. Užtikęs ten kelis mokinius, 
\par 2 paklausė: “Ar įtikėję gavote Šventąją Dvasią?” Jie atsakė: “Mes nė girdėti negirdėjome, kad yra Šventoji Dvasia”. 
\par 3 Jis klausė toliau: “Kokiu tad krikštu jūs buvote pakrikštyti?” Jie atsakė: “Jono krikštu”. 
\par 4 Tada Paulius tarė: “Jonas krikštijo atgailos krikštu, ragindamas žmones tikėti Tą, kuris ateis po jo, būtent Kristų Jėzų”. 
\par 5 Tai išgirdę, jie buvo pakrikštyti Viešpaties Jėzaus vardu. 
\par 6 Kai Paulius jiems uždėjo rankas, ant jų nužengė Šventoji Dvasia ir jie ėmė kalbėti kalbomis ir pranašauti. 
\par 7 Iš viso jų buvo apie dvylika vyrų. 
\par 8 Paulius nuėjo į sinagogą ir ten per tris mėnesius drąsiai kalbėjo, įtikinėdamas ir aiškindamas apie Dievo karalystę. 
\par 9 Kai kurie užkietėjo ir netikėjo, piktžodžiaudami tam Keliui daugumos akyse, tad Paulius pasitraukė nuo jų, atskyrė mokinius ir kasdien juos mokė Tirano mokykloje. 
\par 10 Tai truko dvejus metus, ir visi Azijos gyventojai, tiek žydai, tiek graikai, išgirdo Viešpaties Jėzaus žodį. 
\par 11 Pauliaus rankomis Dievas darė ypatingų stebuklų. 
\par 12 Žmonės net dėdavo ligoniams jo kūną lietusias skepetėles bei prijuostes, ir nuo jų pasitraukdavo ligos, išeidavo piktosios dvasios. 
\par 13 Panašiai ir kai kurie keliaujantys žydų egzorcistai mėgino turėjusiems piktųjų dvasių prišaukti Viešpaties Jėzaus vardą, sakydami: “Mes jus saikdiname Jėzaus vardu, kurį skelbia Paulius”. 
\par 14 Taip darė vieno žydų vyresniojo kunigo Skėvo septyni sūnūs. 
\par 15 Bet piktoji dvasia jiems atsakė: “Pažįstu Jėzų ir žinau Paulių. O kas jūs esate?” 
\par 16 Ir žmogus, turįs piktąją dvasią, užpuolė juos ir nugalėjo su tokia jėga, kad jie nuogi ir sužaloti išbėgo iš anų namų. 
\par 17 Apie tai sužinojo visi žydai ir graikai, gyvenantys Efeze. Visus apėmė baimė, o Viešpaties Jėzaus vardas buvo išaukštintas. 
\par 18 Daug įtikėjusiųjų atėję išpažindavo ir pasisakydavo, ką buvo darę. 
\par 19 Daugelis, užsiiminėjusių magija, sunešė savo knygas ir visų akyse sudegino. Apskaičiuota, kad jos buvo vertos penkiasdešimties tūkstančių sidabro drachmų. 
\par 20 Šitaip galingai Viešpaties žodis plito ir ėmė viršų. 
\par 21 Atlikęs tuos darbus, Paulius dvasioje nusprendė keliauti per Makedoniją ir Achają į Jeruzalę. “Pabuvęs ten,­sakė jis,­turiu pamatyti ir Romą”. 
\par 22 Jis nusiuntė į Makedoniją du savo pagalbininkus, Timotiejų ir Erastą, o pats kurį laiką pasiliko Azijoje. 
\par 23 Tuo metu kilo nemažas sąmyšis dėl Kelio. 
\par 24 Vienas sidabrakalys, vardu Demetrijas, gaminęs sidabrines Artemidės šventyklos miniatiūras ir duodavęs gerai uždirbti amatininkams, 
\par 25 susivadino juos bei kitus to amato darbininkus ir kalbėjo: “Vyrai, jūs žinote, kad šiuo amatu remiasi mūsų gerovė. 
\par 26 Bet jūs matote ir girdite, kad tas Paulius ne tik Efeze, bet ir beveik visoje Azijoje įtikino ir nukreipė į šalį daug žmonių, tvirtindamas, esą tai ne dievai, kurie pagaminami žmonių rankomis. 
\par 27 Gresia pavojus, kad ne tik nusmuks mūsų amatas, bet kad bus paniekinta didžiosios deivės Artemidės šventykla ir žlugs didybė tos, kurią garbina visa Azija ir visas pasaulis”. 
\par 28 Tai išgirdę, jie labai įniršo ir ėmė šaukti: “Didi efeziečių Artemidė!” 
\par 29 Sąmyšis apėmė visą miestą, ir jie, pagriebę Pauliaus palydovus, makedoniečius Gajų ir Aristarchą, visi kaip vienas pasileido bėgti į teatrą. 
\par 30 Pauliui panorus įsimaišyti į minią, mokiniai jo neleido. 
\par 31 Taip pat ir keli jo bičiuliai, Azijos valdininkai, per kitus prašė, kad jis neitų į teatrą. 
\par 32 O ten vieni rėkavo vienaip, kiti­kitaip. Susirinkimas buvo toks pakrikęs, jog daugumas net nežinojo, kuriam galui susirinkta. 
\par 33 Iš minios ištempė Aleksandrą, kurį žydai stūmė į priekį. Aleksandras, pamojęs ranka, norėjo pasiaiškinti miniai. 
\par 34 Bet, pažinę jį esant žydą, visi vienu balsu ėmė rėkti ir kokias dvi valandas šūkavo: “Didi efeziečių Artemidė!” 
\par 35 Galiausiai miesto raštininkas, nuraminęs minią, pasakė: “Efezo vyrai! Kur rasi žmogų, kuris nežinotų, jog efeziečių miestas garbina didžiąją deivę Artemidę ir iš dangaus nukritusią jos statulą?! 
\par 36 Šito niekas negali paneigti. Taigi nusiraminkite ir nieko galvotrūkčiais nedarykite. 
\par 37 Jūs atitempėte čionai šiuos vyrus, kurie nėra nei šventyklos vagys, nei piktžodžiautojai mūsų deivei. 
\par 38 Jei Demetrijas ir jo bendrai amatininkai turi prieš ką nors skundą, tai tam yra teismo dienos ir prokonsulai, tegul sau bylinėjasi. 
\par 39 O jeigu jūs turite kokį kitą apsvarstytiną reikalą, tai jį bus galima išspręsti teisėtame susirinkime. 
\par 40 Šiaip jau yra pavojus būti apkaltintiems maištu dėl to, kas šiandien čia dėjosi, nes nėra jokio pagrindo, kuriuo galėtume pateisinti šį sambrūzdį”. Tai pasakęs, jis paleido susirinkimą.


\chapter{20}


\par 1 Sąmyšiui pasibaigus, Paulius susikvietė mokinius ir atsisveikinęs iškeliavo į Makedoniją. 
\par 2 Keliaudamas per anuos kraštus ir skatindamas mokinius gausiais žodžiais, jis atvyko į Graikiją, 
\par 3 kur pasiliko tris mėnesius. Kai jam besirengiant plaukti į Siriją žydai ėmė ruošti prieš jį sąmokslą, jis nutarė grįžti atgal per Makedoniją. 
\par 4 Iki Azijos jį lydėjo Sopatras iš Berėjos, Aristarchas ir Sekundas iš Tesalonikos, Gajus iš Derbės ir Timotiejus, Tichikas ir Trofimas iš Azijos. 
\par 5 Šie iškeliavo pirmiau ir laukė mūsų Troadėje. 
\par 6 Po Neraugintos duonos dienų mes išplaukėme iš Filipų ir per penkias dienas atvykome pas juos į Troadę; ten išbuvome septynias dienas. 
\par 7 Pirmą savaitės dieną, mokiniams susirinkus laužyti duonos, Paulius mokė juos ir, kadangi žadėjo rytojaus dieną iškeliauti, tai užtęsė savo kalbą iki vidurnakčio. 
\par 8 Aukštutiniame kambaryje, kur jie buvo susirinkę, degė daug žibintų. 
\par 9 Vienas jaunuolis, vardu Eutichas, sėdėjęs ant lango, giliai įmigo ir, Pauliui ilgiau bedėstant, miego įveiktas, iš trečio aukšto nukrito žemėn. Kai jį pakėlė, jis buvo nebegyvas. 
\par 10 Nulipęs žemyn, Paulius pasilenkė ir, apglėbęs jį, tarė: “Nekelkite triukšmo, gyvybė tebėra jame”. 
\par 11 Vėl užlipęs aukštyn, jis laužė ir valgė duoną. Dar ilgai jis kalbėjo, lig pat aušros, ir tada iškeliavo. 
\par 12 O jaunuolį atvedė gyvą, ir tai buvo nemaža paguoda. 
\par 13 Mes, nuėję ir įsėdę į laivą, išplaukėme į Asą, kur ketinome paimti Paulių, nes jis buvo taip patvarkęs, pats norėdamas ten nuvykti pėsčias. 
\par 14 Kai jis susitiko su mumis Ase, paėmę jį, nuplaukėme į Mitilėnę. 
\par 15 Iš ten plaukdami toliau, kitą dieną atsidūrėme priešais Chiją. Rytojaus dieną pasiekėme Samą ir, pabuvę Trogilione, dar po dienos atvykome į Miletą. 
\par 16 Kad netektų gaišti Azijoje, Paulius nutarė aplenkti Efezą, nes jis skubėjo, norėdamas, jei įmanoma, Sekminių dieną būti Jeruzalėje. 
\par 17 Iš Mileto jis pasiuntė į Efezą pakviesti bažnyčios vyresniųjų. 
\par 18 Kai jie pas jį atvyko, Paulius prabilo: “Jūs žinote, kaip nuo pirmosios dienos, kurią įžengiau į Aziją, visą laiką pas jus elgiausi, 
\par 19 tarnaudamas Viešpačiui su visu nusižeminimu, ašaromis ir išbandymais, kurie ištiko mane dėl žydų pinklių. 
\par 20 Kaip nieko nenutylėjau, kas naudinga, bet kalbėjau ir mokiau jus tiek viešumoje, tiek po namus, 
\par 21 liudydamas žydams ir graikams apie atgailą prieš Dievą ir tikėjimą mūsų Viešpačiu Jėzumi Kristumi. 
\par 22 Ir štai dabar aš, dvasios kalinys, keliauju į Jeruzalę, nežinodamas, kas man ten nutiks, 
\par 23 tiktai Šventoji Dvasia kiekviename mieste man liudija, sakydama, kad manęs laukia pančiai ir suspaudimai. 
\par 24 Bet tai man nesvarbu, ir aš nebranginu savo gyvybės. Svarbu, kad tik su džiaugsmu baigčiau savo bėgimą ir tarnavimą, kurį gavau iš Viešpaties Jėzaus: liudyti Dievo malonės Evangeliją. 
\par 25 Ir štai dabar aš žinau, kad jūs visi, su kuriais buvau skelbdamas Dievo karalystę, daugiau nebematysite mano veido. 
\par 26 Todėl šiandien jums liudiju, jog esu švarus nuo visų kraujo. 
\par 27 Aš nevengiau jums paskelbti visų Dievo nutarimų. 
\par 28 Būkite rūpestingi sau ir visai kaimenei, kuriai Šventoji Dvasia jus paskyrė prižiūrėtojais, kad ganytumėte Dievo bažnyčią, kurią Jis įsigijo savo krauju. 
\par 29 Nes aš žinau, kad, man pasitraukus, įsibraus pas jus žiaurių vilkų, kurie nepagailės kaimenės. 
\par 30 Net iš jūsų atsiras tokių, kurie kreivomis kalbomis stengsis patraukti paskui save mokinius. 
\par 31 Todėl budėkite ir nepamirškite, kad aš per trejus metus dieną ir naktį nepaliaudamas, su ašaromis įspėjinėjau kiekvieną. 
\par 32 O dabar, broliai, pavedu jus Dievui ir Jo malonės žodžiui, kuris turi galią jus išugdyti ir duoti jums paveldėjimą tarp visų pašventintųjų. 
\par 33 Nė iš vieno negeidžiau nei sidabro, nei aukso, nei drabužio. 
\par 34 Jūs žinote, kad mano ir buvusiųjų su manimi reikalams tarnavo šitos va mano rankos. 
\par 35 Ir aš jums visur rodydavau, kad, šitaip triūsiant, reikia paremti silpnuosius ir atminti Viešpaties Jėzaus pasakytus žodžius: ‘Labiau palaiminta duoti negu imti’ ”. 
\par 36 Tai pasakęs, jis atsiklaupė ir kartu su visais pasimeldė. 
\par 37 Visi pradėjo graudžiai verkti ir, puldami Pauliui ant kaklo, jį bučiavo. 
\par 38 Jie ypač nuliūdo dėl žodžių, kad daugiau nebematysią jo veido. Po to jie palydėjo jį į laivą.


\chapter{21}


\par 1 Išsiplėšę iš jų glėbio, išplaukėme ir tiesiu keliu atvykome į Kosą, o kitą dieną į Rodą ir iš ten į Patarą. 
\par 2 Radę laivą, plaukiantį į Finikiją, įlipome ir išplaukėme. 
\par 3 Išvydę artėjant Kiprą, palikome jį kairėje ir leidomės link Sirijos, kur išlipome Tyre. Ten laivas turėjo iškrauti savo krovinius. 
\par 4 Suradę mokinių, išbuvome su jais septynias dienas. Dvasios paskatinti, jie sakė Pauliui nevykti į Jeruzalę. 
\par 5 Praėjus toms dienoms, mes išėjome ir leidomės į kelionę, o jie visi kartu su žmonomis ir vaikais palydėjo mus iki miesto ribos. Pajūryje suklaupę visi pasimeldėme. 
\par 6 Vieni su kitais atsisveikinę, sulipome į laivą, o jie sugrįžo namo. 
\par 7 Keliaudami toliau iš Tyro atplaukėme į Ptolemaidę. Ten pasveikinome brolius ir pasilikome pas juos vienai dienai. 
\par 8 Kitą dieną Paulius ir mes, buvę su juo, vėl iškeliavome ir atvykome į Cezarėją. Ten nuėjome į evangelisto Pilypo, vieno iš septynių, namus ir pas jį pasilikome. 
\par 9 Jis turėjo keturias netekėjusias dukteris, kurios pranašaudavo. 
\par 10 Mums bebūnant ten daugiau dienų, iš Judėjos atėjo vienas pranašas, vardu Agabas. 
\par 11 Atėjęs pas mus, jis paėmė Pauliaus juostą ir, susirišęs ja rankas ir kojas, tarė: “Taip sako Šventoji Dvasia: ‘Vyrą, kuriam priklauso ši juosta, taip Jeruzalėje supančios žydai ir atiduos į pagonių rankas’ ”. 
\par 12 Tai išgirdę, ir mes, ir tenykščiai prašėme, kad jis neitų į Jeruzalę. 
\par 13 Bet Paulius atsakė: “Kodėl raudate ir draskote man širdį? Aš pasirengęs Jeruzalėje dėl Viešpaties Jėzaus ne tik būti supančiotas, bet ir numirti!” 
\par 14 Jam nesiduodant perkalbamam, mes nurimome ir tarėme: “Tebūnie Viešpaties valia!” 
\par 15 Po tų dienų, susiruošę į kelią, išvykome į Jeruzalę. 
\par 16 Mus lydėjo kai kurie mokiniai iš Cezarėjos ir nuvedė apsistoti pas vieną kiprietį, seną mokinį Mnasoną. 
\par 17 Kai atvykome į Jeruzalę, broliai mus džiaugsmingai priėmė. 
\par 18 Kitą dieną Paulius kartu su mumis nuėjo pas Jokūbą, kur buvo susirinkę visi vyresnieji. 
\par 19 Juos pasveikinęs, jis smulkiai išdėstė visa, ką Dievas padarė pagonijoje per jo tarnystę. 
\par 20 Jie išklausę šlovino Viešpatį ir tarė Pauliui: “Tu matai, broli, kiek daug tūkstančių žydų tapo tikinčiaisiais, ir jie visi uoliai laikosi Įstatymo. 
\par 21 Bet jiems prikalbėta apie tave, kad tu mokai visus žydus, gyvenančius tarp pagonių, atsižadėti Mozės, sakydamas, jog jiems nereikia apipjaustyti vaikų nei laikytis papročių. 
\par 22 Ką daryti? Jie būtinai susirinks, išgirdę, kad esi atvykęs. 
\par 23 Todėl padaryk, kaip tau sakome. Pas mus yra keturi vyrai, padarę įžadą. 
\par 24 Pasiimk juos, apsivalyk kartu su jais ir prisiimk jų išlaidas, kad jie galėtų nusikirpti plaukus. Tada visi pamatys, jog gandai apie tave nieko verti ir tu nenukrypdamas laikaisi Įstatymo. 
\par 25 O dėl tikinčiųjų iš pagonių, tai mes išsiuntėme nurodymus, kad jie saugotųsi to, kas paaukota stabams, kraujo, pasmaugtų gyvulių mėsos ir ištvirkavimo”. 
\par 26 Tada Paulius pasiėmė tuos vyrus ir kitą dieną su jais apsivalęs įėjo į šventyklą. Ten jis nurodė apsivalymo dienų pabaigą, kada už kiekvieną iš jų turės būti paaukota auka. 
\par 27 Toms septynioms dienoms baigiantis, žydai iš Azijos, pastebėję Paulių šventykloje, sukurstė visą minią ir, nutvėrę jį, 
\par 28 ėmė šaukti: “Izraelio vyrai, padėkite! Šitas žmogus visur visus moko prieš tautą, Įstatymą ir šią vietą. Jis net atsivedė šventyklon graikus ir suteršė šią šventą vietą”. 
\par 29 Mat jie prieš tai matė su juo mieste Trofimą iš Efezo ir manė, kad Paulius jį atsivedęs į šventyklą. 
\par 30 Sujudo visas miestas, subėgo žmonės, nutvėrė Paulių ir ištempė jį iš šventyklos. Durys bematant buvo uždarytos. 
\par 31 Jiems ketinant jį užmušti, įgulos viršininkui­tribūnui buvo duota žinia, kad visoje Jeruzalėje kyla sąmyšis. 
\par 32 Jis nedelsdamas su kareiviais ir šimtininkais atskubėjo ten. Jie, pamatę tribūną ir kareivius, liovėsi mušę Paulių. 
\par 33 Prisiartinęs tribūnas suėmė Paulių ir įsakė surišti dviem grandinėmis. Paskui paklausė, kas jis esąs ir ką padaręs. 
\par 34 Tuo tarpu iš minios šaukė tai šį, tai tą. Negalėdamas dėl triukšmo nieko tikro apie jį sužinoti, tribūnas įsakė nuvesti Paulių į kareivines. 
\par 35 Atėjus prie laiptų, dėl minios įsisiautėjimo kareiviams teko jį nešte nešti. 
\par 36 Nes žmonių minia sekė iš paskos, šaukdama: “Mirtis jam!” 
\par 37 Vedamas į kareivines, Paulius kreipėsi į tribūną: “Ar galiu tau šį tą pasakyti?” Šis atsiliepė: “Tu moki graikiškai? 
\par 38 Tai tu­ne anas egiptietis, kuris neseniai sukėlė sąmyšį ir išsivedė į dykumą keturis tūkstančius vyrų žudikų?” 
\par 39 Paulius atsakė: “Aš esu žydas iš Kilikijos Tarso, taigi žymaus miesto pilietis. Prašau tavęs, leisk man prabilti žmonėms”. 
\par 40 Tribūnas sutiko. Paulius atsistojo ant laiptų ir pamojo ranka miniai. Visiems nutilus, jis prabilo hebrajų kalba:


\chapter{22}


\par 1 “Vyrai broliai ir tėvai! Paklausykite, ką dabar pasakysiu sau apginti”. 
\par 2 Išgirdę jį kreipiantis į juos hebrajiškai, jie dar labiau nurimo. O jis kalbėjo toliau: 
\par 3 “Aš esu žydas, gimęs Tarse, Kilikijoje, bet išauklėtas šitame mieste, prie Gamalielio kojų, tobulai išmokytas pagal mūsų protėvių Įstatymą, ir buvau ypatingai uolus dėl Dievo, kaip ir jūs visi šiandien. 
\par 4 Todėl iki mirties persekiojau šį Kelią, pančiodamas ir mesdamas į kalėjimą vyrus ir moteris. 
\par 5 Tai gali paliudyti ir vyriausiasis kunigas, ir visa vyresniųjų taryba. Iš jų buvau gavęs laiškus broliams ir keliavau į Damaską, kad tenykščius surištus atgabenčiau į Jeruzalę nubausti. 
\par 6 Man keliaujant ir artinantis prie Damasko, apie vidurdienį staiga mane apšvietė ryški šviesa iš dangaus. 
\par 7 Aš kritau ant žemės ir išgirdau balsą, man sakantį: ‘Sauliau, Sauliau, kam mane persekioji?’ 
\par 8 Aš paklausiau: ‘Kas Tu esi, Viešpatie?’ Jis man atsakė: ‘Aš esu Jėzus iš Nazareto, kurį tu persekioji’. 
\par 9 Buvusieji su manimi matė šviesą ir išsigando, bet negirdėjo man kalbančiojo balso. 
\par 10 Aš dar paklausiau: ‘Ką man daryti, Viešpatie?’ O Viešpats man tarė: ‘Kelkis, eik į Damaską, tenai tau bus pasakyta visa, ką tau reikia daryti’. 
\par 11 Kadangi tos šviesos šlovės apakintas nieko nebemačiau, palydovų už rankos vedamas pasiekiau Damaską. 
\par 12 Toksai Ananijas, Įstatymo požiūriu dievotas vyras, apie kurį gerai liudijo visi aplinkiniai žydai, 
\par 13 atėjęs stojo prieš mane ir tarė: ‘Broli Sauliau, praregėk!’ Ir tą pačią akimirką aš jį pamačiau. 
\par 14 O jis kalbėjo: ‘Mūsų protėvių Dievas išsirinko tave, kad pažintum Jo valią, išvystum Teisųjį ir išgirstum balsą iš Jo lūpų, 
\par 15 nes tu būsi Jo liudytojas visiems žmonėms, skelbdamas, ką girdėjai ir regėjai. 
\par 16 Tad ko lauki? Kelkis, pasikrikštyk ir nusiplauk nuodėmes, šaukdamasis Viešpaties vardo!’ 
\par 17 Vėliau, sugrįžęs į Jeruzalę ir melsdamasis šventykloje, patyriau Dvasios pagavą 
\par 18 ir išvydau Jėzų. Jis pasakė: ‘Skubiai pasitrauk iš Jeruzalės, nes jie nepriims tavo liudijimo apie mane’. 
\par 19 Aš atsiliepiau: ‘Viešpatie, juk jie žino, kad Tavo tikinčiuosius iš visų sinagogų mesdavau į kalėjimą ir plakdavau. 
\par 20 O kai buvo pralietas tavo liudytojo Stepono kraujas, aš ten stovėjau pritardamas jo nužudymui ir saugodamas žudikų drabužius’. 
\par 21 Bet Jis tarė man: ‘Eik, nes Aš siųsiu tave toli, pas pagonis’ ”. 
\par 22 Jie klausėsi jo iki šitų žodžių, o čia ėmė garsiai šaukti: “Nušluoti nuo žemės jį! Tokiam nevalia gyventi!” 
\par 23 Jie klykė, mosavo drabužiais ir svaidė į orą smėlį. 
\par 24 Tribūnas įsakė nuvesti Paulių į kareivines ir liepė tardyti plakant, kad išsiaiškintų, kodėl žmonės prieš jį taip rėkė. 
\par 25 Kai jau buvo pririštas diržais, Paulius kreipėsi į šalia stovintį šimtininką: “Ar jums leista plakti Romos pilietį, ir dar nenuteistą?” 
\par 26 Tai išgirdęs, šimtininkas priėjo prie tribūno ir pasakė: “Žiūrėk, ką darai! Tas žmogus yra Romos pilietis”. 
\par 27 Tribūnas atėjęs paklausė: “Pasakyk man, ar tu Romos pilietis?” Paulius atsakė: “Taip”. 
\par 28 Tribūnas tarė: “Aš šitą pilietybę įsigijau už didelius pinigus”. Paulius atsiliepė: “O aš turiu ją nuo gimimo”. 
\par 29 Bematant pasitraukė nuo jo tie, kurie turėjo jį tardyti. Tribūnas išsigando, sužinojęs, kad Paulius yra Romos pilietis, ir kad jį surišo. 
\par 30 Rytojaus dieną, norėdamas tiksliau išsiaiškinti, kuo jis žydų kaltinamas, tribūnas išlaisvino jį iš grandinių, liepė sušaukti aukštuosius kunigus bei visą sinedrioną ir, atvedęs Paulių, pastatė jų akivaizdoje.


\chapter{23}


\par 1 Skvarbiu žvilgsniu permetęs sinedrioną, Paulius prabilo: “Vyrai broliai! Iki šios dienos elgiausi Dievo akivaizdoje visiškai gryna sąžine”. 
\par 2 Bet vyriausiasis kunigas Ananijas įsakė šalia stovintiesiems smogti jam per burną. 
\par 3 Tuomet Paulius jam tarė: “Tau smogs Dievas, tu, pabaltinta siena! Tu čia sėdi, kad mane teistum pagal Įstatymą, o liepi mane mušti prieš Įstatymą?” 
\par 4 Šalia esantys tarė: “Tu keiki vyriausiąjį Dievo kunigą?!” 
\par 5 Paulius atsiliepė: “Broliai, aš nežinojau, kad jis vyriausiasis kunigas. Juk parašyta: ‘Nepiktžodžiauk savo tautos vadovui’ ”. 
\par 6 Supratęs, kad viena dalis buvo sadukiejai, o kita­fariziejai, Paulius sušuko sinedrionui: “Vyrai broliai, aš fariziejus, fariziejaus sūnus, ir esu teisiamas už mirusiųjų prisikėlimo viltį!” 
\par 7 Jam tai pasakius, tarp fariziejų ir sadukiejų kilo barnis, ir susirinkimas suskilo. 
\par 8 Mat sadukiejai sako, kad nėra nei prisikėlimo, nei angelų, nei dvasios, o fariziejai tuos dalykus pripažįsta. 
\par 9 Todėl kilo didelis triukšmas. Atsistoję fariziejų pusės Rašto žinovai griežtai prieštaravo, sakydami: “Mes nerandame nieko blogo šitame žmoguje. O jeigu jam kalbėjo dvasia arba angelas, nesipriešinkime Dievui!” 
\par 10 Įsisiautėjus smarkiam ginčui, tribūnas, bijodamas, kad jie nesudraskytų Pauliaus, įsakė kareivių daliniui nusileisti žemyn, išplėšti Paulių iš jų ir nuvesti į kareivines. 
\par 11 Kitą naktį šalia jo stojo Viešpats ir tarė: “Būk drąsus, Pauliau! Kaip liudijai apie mane Jeruzalėje, taip turėsi liudyti ir Romoje”. 
\par 12 Dieną slaptai susirinko žydų būrys ir prisiekė nei valgyti, nei gerti, kol nužudys Paulių. 
\par 13 Tokią sankalbą padarė daugiau negu keturiasdešimt žmonių. 
\par 14 Jie nuėjo pas aukštuosius kunigus bei vyresniuosius ir pasakė: “Mes siekte prisiekėme nieko neragauti, kol nužudysime Paulių. 
\par 15 Todėl dabar jūs kartu su sinedrionu prašykite tribūną, kad jis pristatytų jį jums rytoj, lyg norėtumėte tiksliau ištirti jo bylą. Tuo tarpu mes būsime pasiruošę užmušti jį kelyje”. 
\par 16 Apie šį suokalbį nugirdo Pauliaus sesers sūnus. Jis atėjo į kareivines ir pranešė Pauliui. 
\par 17 Tada Paulius, pasivadinęs vieną šimtininką, paprašė: “Nuvesk šį jaunuolį pas tribūną. Jis turi jam kai ką pranešti”. 
\par 18 Tas, paėmęs jį, nuvedė pas tribūną ir paaiškino: “Kalinys Paulius pasišaukė mane ir paprašė, kad atvesčiau pas tave šitą jaunuolį. Jis turįs tau kai ką pasakyti”. 
\par 19 Tribūnas, paėmęs jį už rankos, pasivedė į šalį ir paklausė: “Ką turi man pranešti?” 
\par 20 Tas atsakė: “Žydai susitarė prašyti tave, kad rytoj nuvestum Paulių į sinedrioną, neva norėdami tiksliau ištirti jo bylą. 
\par 21 Netikėk jais! Nes jo tyko daugiau negu keturiasdešimt vyrų, kurie prisiekė nei valgyti, nei gerti, kol jį nužudys. Jie jau dabar pasiruošę ir laukia tavo sutikimo”. 
\par 22 Tribūnas atleido jaunuolį ir griežtai įsakė: “Niekam nesakyk, kad tu man tai pranešei”. 
\par 23 Pasišaukęs du šimtininkus, tribūnas pasakė: “Nuo trečios valandos nakties laikykite parengtyje žygiuoti į Cezarėją du šimtus kareivių, septyniasdešimt raitelių ir du šimtus ietininkų. 
\par 24 Parūpinkite ir arklių, kad raitą Paulių saugiai nugabentų pas valdytoją Feliksą”. 
\par 25 Ir jis parašė tokio turinio laišką: 
\par 26 “Klaudijus Lisijas kilniausiajam valdytojui Feliksui siunčia sveikinimą. 
\par 27 Šitą vyrą žydai buvo nutvėrę ir norėjo nužudyti. Aš, atėjęs su kareiviais, jį išgelbėjau, sužinojęs esant Romos pilietį. 
\par 28 Norėdamas patirti jo apkaltinimo priežastį, nuvedžiau jį į jų sinedrioną. 
\par 29 Radau, kad jis kaltinamas dėl jų Įstatymo klausimų, o ne dėl kokio nusikaltimo, baustino mirtimi ar kalėjimu. 
\par 30 Kadangi man buvo pranešta apie žydų ruošiamą pasikėsinimą į šį vyrą, tai nedelsdamas siunčiu jį pas tave, nurodęs ir kaltintojams, kad jie tau pateiktų prieš jį turimus kaltinimus. Lik sveikas”. 
\par 31 Kareiviai, vykdydami įsakymą, paėmė ir nugabeno nakčia Paulių į Antipatridę. 
\par 32 Rytojaus dieną jie pasiuntė su juo raitelius, o patys sugrįžo į kareivines. 
\par 33 Anie, atvykę į Cezarėją, įteikė valdytojui laišką ir pristatė Paulių. 
\par 34 Perskaitęs laišką, jis pasiteiravo, iš kokios provincijos tas esąs. Sužinojęs, kad iš Kilikijos, 
\par 35 jis pasakė: “Aš tave išklausysiu, kai atvyks tavo kaltintojai”. Ir liepė jį saugoti Erodo pretorijuje.


\chapter{24}


\par 1 Po penkių dienų atvyko vyriausiasis kunigas Ananijas su vyresniaisiais ir oratoriumi Tertulu. Jie pateikė valdytojui kaltinimus prieš Paulių. 
\par 2 Kai pastarasis buvo iškviestas, Tertulas pradėjo savo kaltinamąją kalbą: “Tavo rūpesčiu turėdami tikrą taiką ir didelę gerovę šitoje tautoje, 
\par 3 mes, prakilnusis Feliksai, visuomet ir visur tai pripažįstame su didžiu dėkingumu. 
\par 4 Nenorėdamas tavęs ilgiau gaišinti, prašau tave trumpai mūsų paklausyti su savo įprastu maloningumu. 
\par 5 Mes nustatėme šį žmogų esant tarsi marą. Jis kursto maištą viso pasaulio žydijoje ir yra nazariečių sektos vadeiva. 
\par 6 Jis net mėgino išniekinti šventyklą. Štai kodėl jį suėmėme ir norėjome nuteisti pagal savąjį Įstatymą. 
\par 7 Bet tribūnas Lisijas su didele prievarta išplėšė jį iš mūsų rankų 
\par 8 ir liepė jo kaltintojams atvykti pas tave. Apie tai, kuo jį kaltiname, pats galėsi viską sužinoti, jį išklausinėjęs”. 
\par 9 Tiems žodžiams pritarė žydai, tvirtindami, kad taip esą iš tikrųjų. 
\par 10 Valdytojui davus ženklą, Paulius atsakė: “Žinodamas tave jau daug metų esant šios tautos teisėju, drąsiai ginsiu savo bylą. 
\par 11 Tau nesunku nustatyti, jog praėjo ne daugiau kaip dvylika dienų, kai atvykau į Jeruzalę pagarbinti. 
\par 12 Ir niekas manęs nematė su kuo nors besiginčijant šventykloje nei suburiant minią sinagogose ar mieste. 
\par 13 Jie negali įrodyti to, kuo dabar mane kaltina. 
\par 14 Bet aš tau išpažįstu, jog tarnauju savo tėvų Dievui pagal Kelią, jų vadinamą sekta, tikėdamas visa, kas parašyta Įstatyme ir Pranašuose, 
\par 15 ir turiu viltį Dieve, kurią jie patys irgi pripažįsta, jog bus prisikėlimas iš numirusių­tiek teisiųjų, tiek neteisiųjų. 
\par 16 Todėl stengiuosi visuomet turėti tyrą sąžinę prieš Dievą ir prieš žmones. 
\par 17 Aš po daugelio metų atkeliavau savo tautai atiduoti gailestingumo dovanų ir aukų. 
\par 18 Todėl kai kurie Azijos žydai rado mane apsivaliusį šventykloje be jokios minios ir be jokio triukšmo. 
\par 19 Tai jiems reikėtų čia būti ir, jei ką turi prieš mane, kaltinti tavo akyse. 
\par 20 Pagaliau tegul ir šitie pasako, kokį nusikaltimą jie man įrodė, kai stovėjau prieš sinedrioną? 
\par 21 Nebent tik žodžius, kuriuos šaukiau, stovėdamas tarp jų: ‘Šiandien jūs mane teisiate už mirusiųjų prisikėlimą!’ ” 
\par 22 Feliksas, gerai nusimanydamas apie tą Kelią, tai išklausęs, atidėjo skundo svarstymą, sakydamas: “Kai atvyks tribūnas Lisijas, tuomet ir išaiškinsiu jūsų klausimą”. 
\par 23 Jis davė nurodymą šimtininkui saugoti Paulių, tačiau daryti jam lengvatų ir niekam iš jo žmonių nedrausti jam patarnauti ar pas jį ateiti. 
\par 24 Po kelių dienų Feliksas atėjo su savo žmona Druzila, kuri buvo žydė. Jis liepė pakviesti Paulių ir išklausė jo apie tikėjimą Kristumi. 
\par 25 Pauliui dėstant apie teisumą, susilaikymą ir būsimąjį teismą, Feliksas išsigando ir tarė: “Šiam kartui užtenka. Gali eiti. Kai turėsiu laiko, tave pasišauksiu”. 
\par 26 Be to, jis tikėjosi, kad Paulius duos jam pinigų, jog jį išleistų, todėl dažniau jį kviesdavosi ir su juo kalbėdavosi. 
\par 27 Prabėgus dvejiems metams, Feliksą pakeitė įpėdinis Porcijus Festas. Norėdamas padaryti žydams malonumą, Feliksas paliko Paulių kalėjime.


\chapter{25}


\par 1 Atvykęs į provinciją, Festas po trijų dienų nukeliavo iš Cezarėjos į Jeruzalę. 
\par 2 Ten į jį kreipėsi vyriausiasis kunigas ir žydų didikai, kaltindami Paulių, įkalbinėjo 
\par 3 ir prašė malonės atsiųsti jį į Jeruzalę, klastingai galvodami kelyje jį nužudyti. 
\par 4 Bet Festas atsakė, kad Paulius turįs būti saugomas Cezarėjoje. Be to, jis pats ketinąs netrukus ten grįžti. 
\par 5 “Tegul jūsų įgaliotiniai,­tęsė jis,­keliauja kartu ir, jei tas vyras kuo nors nusikaltęs, tekaltina jį”. 
\par 6 Pabuvęs tarp jų daugiau kaip dešimt dienų, jis sugrįžo į Cezarėją. Kitą dieną, atsisėdęs į teismo krasę, liepė atvesti Paulių. 
\par 7 Vos tik pasirodžiusį apstojo jį iš Jeruzalės atvykę žydai, primesdami daug sunkių kaltinimų, kurių neįstengė įrodyti. 
\par 8 Paulius gynėsi: “Aš nieku nenusikaltęs nei žydų Įstatymui, nei šventyklai, nei ciesoriui”. 
\par 9 Norėdamas parodyti palankumą žydams, Festas paklausė Paulių: “Ar nori keliauti į Jeruzalę ir ten mano akivaizdoje būti teisiamas dėl šių dalykų?” 
\par 10 Paulius atsakė: “Aš stoviu prieš ciesoriaus teismą ir ten privalau būti teisiamas. Žydams aš nepadariau nieko pikto, kaip tu pats puikiai žinai. 
\par 11 Jei esu nusikaltęs ir padaręs ką nors verta mirties, neatsisakau mirti. Bet jeigu jų metami kaltinimai nepagrįsti, niekas negali jiems manęs išduoti. Aš šaukiuosi ciesoriaus!” 
\par 12 Tada Festas, pasitaręs su savo taryba, paskelbė: “Šaukiesi ciesoriaus­eisi pas ciesorių!” 
\par 13 Praslinkus kelioms dienoms, karalius Agripa ir Berenikė atvyko į Cezarėją Festo pasveikinti. 
\par 14 Jiems ten būnant nemažai dienų, Festas supažindino karalių su Pauliaus byla ir papasakojo: “Feliksas paliko įkalintą vieną vyrą. 
\par 15 Kai buvau Jeruzalėje, aukštieji kunigai ir žydų vyresnieji kreipėsi į mane, kaltindami jį ir reikalaudami pasmerkti. 
\par 16 Aš jiems atsakiau, kad romėnai neturi papročio pasmerkti kokį nors žmogų, nedavę kaltinamajam galimybės stoti savo kaltintojų akivaizdoje ir gintis nuo kaltinimų. 
\par 17 Jiems čia atvykus, aš nedelsdamas rytojaus dieną atsisėdau į teismo krasę ir liepiau atvesti tą vyrą. 
\par 18 Prieš jį atsistoję kaltintojai nenurodė jokio nusikaltimo, kokio buvau tikėjęsis. 
\par 19 Jie vien tik ginčijosi ir jam prikaišiojo dėl kai kurių savo religijos klausimų ir dėl kažkokio mirusio Jėzaus, kurį Paulius tvirtino esant gyvą. 
\par 20 Dvejodamas, kaip išspręsti tokius klausimus, paklausiau, ar jis nenorėtų vykti į Jeruzalę ir ten būti dėl šito teisiamas. 
\par 21 Bet Paulius pareikalavo, kad jo byla būtų perduota Augusto žiniai. Todėl įsakiau jį toliau kalinti, kol išsiųsiu pas ciesorių”. 
\par 22 Agripa tarė Festui: “Aš ir pats norėčiau pasiklausyti to žmogaus”. Anas atsakė: “Galėsi rytoj pasiklausyti”. 
\par 23 Rytojaus dieną, kai Agripa ir Berenikė atėjo su didele iškilme ir įžengė į salę kartu su tribūnais ir miesto didžiūnais, Festui įsakius, buvo atvestas Paulius. 
\par 24 Festas prabilo: “Karaliau Agripa ir visi čia esantys vyrai! Jūs matote žmogų, dėl kurio visa daugybė žydų kreipėsi į mane Jeruzalėje ir čia, šaukdami, kad jo negalima palikti gyvo. 
\par 25 Bet aš nustačiau, kad jis nėra padaręs nieko, kas būtų baustina mirtimi. Jam šaukiantis Augusto, nusprendžiau jį ten pasiųsti. 
\par 26 Tačiau nežinau nieko tikro, ką turėčiau parašyti valdovui. Todėl iškviečiau jį jūsų akivaizdon, ypač tavo, karaliau Agripa, kad, jį apklausęs, žinočiau, ką turiu parašyti. 
\par 27 Mat, man rodos, neprotinga siųsti kalinį, nenurodant kaltinimų”.


\chapter{26}


\par 1 Tada Agripa tarė Pauliui: “Tau leidžiama paaiškinti savo bylą”. Paulius, pamojęs ranka, pradėjo gynimosi kalbą: 
\par 2 “Karaliau Agripa! Esu laimingas, galėdamas šiandien tavo akivaizdoje gintis nuo viso to, kuo esu žydų kaltinamas, 
\par 3 juo labiau, kad tau žinomi visi žydų papročiai ir ginčijami klausimai. Todėl prašau kantriai manęs išklausyti. 
\par 4 Visi žydai žino apie mano gyvenimą nuo jaunystės, kuris nuo pradžių prabėgo mano tautoje, Jeruzalėje. 
\par 5 Jie pažįsta mane nuo seno ir, jei norėtų, galėtų paliudyti, kad aš, būdamas fariziejus, gyvenau, laikydamasis griežčiausios mūsų religijos krypties. 
\par 6 Dabar aš čia stoviu ir esu teisiamas už tai, kad viliuosi pažadu, kurį Dievas yra davęs mūsų tėvams. 
\par 7 Jo išsipildant tikisi sulaukti mūsų dvylika giminių, uoliai tarnaudamos Dievui dieną ir naktį. Dėl šitos vilties, karaliau Agripa, aš ir esu žydų kaltinamas. 
\par 8 Kodėl jums atrodo neįtikėtina, kad Dievas prikelia numirusius? 
\par 9 Tiesa, ir aš maniau, kad privalau visais būdais kovoti su Jėzaus Nazariečio vardu. 
\par 10 Aš taip ir dariau Jeruzalėje. Gavęs iš aukštųjų kunigų įgaliojimus, daugybę šventųjų uždariau į kalėjimus, o kai jie buvo žudomi, pritardavau. 
\par 11 Visose sinagogose dažnai juos bausdavau, versdavau piktžodžiauti ir, be saiko prieš juos įtūžęs, persekiojau net svetimuose miestuose. 
\par 12 Tais pačiais tikslais keliavau į Damaską, turėdamas aukštųjų kunigų įgaliojimus ir leidimą. 
\par 13 Kelyje vidurdienį, karaliau, aš staiga išvydau, kaip mane ir keliavusius su manimi apšvietė šviesa iš dangaus, skaistesnė už saulę. 
\par 14 Mes visi parpuolėme žemėn, ir aš išgirdau balsą, kuris man sakė hebrajiškai: ‘Sauliau, Sauliau, kam mane persekioji? Sunku tau spyriotis prieš akstiną!’ 
\par 15 Aš paklausiau: ‘Kas Tu esi, Viešpatie?’ Jis atsakė: ‘Aš esu Jėzus, kurį tu persekioji. 
\par 16 Kelkis ir stokis ant kojų! Aš tau apsireiškiau, kad paskirčiau tave tarnu bei liudytoju tų dalykų, kuriuos matei ir kuriuos tau dar apreikšiu. 
\par 17 Aš tave gelbėsiu nuo tautiečių ir pagonių, pas kuriuos tave dabar siunčiu, 
\par 18 kad atvertum jų akis ir jie iš tamsybių gręžtųsi į šviesą, nuo šėtono valdžios­į Dievą ir, tikėdami mane, gautų nuodėmių atleidimą bei paveldėjimą su pašventintaisiais’. 
\par 19 Todėl, karaliau Agripa, nebuvau nepaklusnus dangiškam regėjimui 
\par 20 ir iš pradžių Damaske ir Jeruzalėje, o paskui visame Judėjos krašte ir pagonijoje skelbiau, kad žmonės atgailautų, gręžtųsi į Dievą ir imtųsi atgailos vertų darbų. 
\par 21 Už tai žydai sugriebė mane šventykloje ir mėgino užmušti. 
\par 22 Bet, Dievui padedant, iki šios dienos tebeliudiju mažam ir dideliam, neskelbdamas nieko viršaus, o tik tai, ką skelbė įvyksiant pranašai ir Mozė, 
\par 23 būtent, kad Kristus kentės, jog pirmutinis prisikels iš numirusių ir paskelbs šviesą tautai ir pagonims”. 
\par 24 Jam tai kalbant, Festas garsiai sušuko: “Pauliau, tu iš galvos kraustaisi! Iš didelio rašto išėjai iš krašto”. 
\par 25 Paulius atsiliepė: “Ne, nesikraustau iš galvos, prakilnusis Festai, bet skelbiu tiesos ir sveiko proto žodžius. 
\par 26 Šiuos dalykus žino karalius, kuriam taip atvirai kalbu. Esu tikras, kad iš viso to jam nieko nėra nežinomo, nes šitai dėjosi ne kur nors užkampyje. 
\par 27 Karaliau Agripa, ar tiki pranašais? Žinau, kad tiki”. 
\par 28 Agripa atsakė: “Ko gero įtikinsi mane krikščionimi tapti...” 
\par 29 Paulius tarė: “Meldžiu Dievą, kad ar per trumpą, ar per ilgą laiką ne tik tu, bet ir visi, šiandien girdėję mane kalbant, taptų tokie, kaip aš, tik be šitų pančių!” 
\par 30 Jam tai pasakius, pakilo karalius, valdytojas, Berenikė ir visi sėdėję su jais. 
\par 31 Išeidami jie tarpusavyje kalbėjosi: “Šitas žmogus nepadarė nieko baustino mirtimi ar kalėjimu”. 
\par 32 Agripa Festui pareiškė: “Būtų galima paleisti šitą žmogų, jeigu jis nebūtų šaukęsis ciesoriaus”.


\chapter{27}


\par 1 Kai buvo nuspręsta, kad mes turime išplaukti į Italiją, Paulių ir kelis kitus kalinius perdavė Augusto kohortos šimtininkui Julijui. 
\par 2 Mes įlipome į Adramitijos laivą, kuris turėjo plaukti Azijos pakrantėmis, ir leidomės į kelionę. Su mumis buvo makedonietis Aristarchas iš Tesalonikos. 
\par 3 Kitą dieną atplaukėme į Sidoną. Julijus kilniai elgėsi su Pauliumi ir leido jam aplankyti bičiulius, kad galėtų pasidžiaugti jų globa. 
\par 4 Išvykę iš ten, plaukėme Kipro užuovėja, nes pūtė priešingi vėjai. 
\par 5 Perplaukę jūrą ties Kilikija ir Pamfilija, atvykome į Lykijos miestą Myrą. 
\par 6 Tenai šimtininkas surado Aleksandrijos laivą, kuris plaukė į Italiją, ir persodino mus į jį. 
\par 7 Daug dienų plaukėme palengva ir vargais negalais atsiradome ties Knidu. Kadangi vėjas kliudė, plaukėme Kretos priedangoje, link Salmonės. 
\par 8 Šiaip ne taip ją aplenkę, atvykome į vietovę, kuri vadinosi Dailioji Prieplauka, netoli Lasėjos miesto. 
\par 9 Prabėgo daug laiko, ir laivyba tapo pavojinga, nes jau buvo pasibaigęs rudens pasninko metas. Paulius juos įspėjo, 
\par 10 sakydamas: “Vyrai, numatau, jog šis plaukimas bus grėsmingas ir pražūtingas ne tik kroviniams bei laivui, bet ir mūsų gyvybėms”. 
\par 11 Tačiau šimtininkas labiau tikėjo vairininku ir laivo savininku negu tuo, ką kalbėjo Paulius. 
\par 12 Kadangi uostas buvo netinkamas žiemoti, dauguma nusprendė iš ten plaukti toliau, kaip nors pasiekti Feniksą, Kretos uostą, atvirą į pietvakarius ir šiaurės vakarus, ir ten žiemoti. 
\par 13 Kai ėmė pūsti lengvas pietų vėjas, jie tikėjosi įvykdyti savo sumanymą ir, pakėlę inkarą, leidosi pirmyn palei Kretos krantus. 
\par 14 Bet netrukus nuo salos pusės pakilo viesulas, vadinamas Šiauriaryčiu. 
\par 15 Jis pagavo laivą, nepajėgusį jam priešintis, ir mes turėjome jį leisti pavėjui. 
\par 16 Kai mus nešė pro salelę, vardu Klaudą, mes tik vargais negalais išsaugojome gelbėjimo valtį. 
\par 17 Kai ji buvo užkelta, jūreiviai ėmė stiprinti laivą­apjuosė jį virvėmis. Bijodami pakliūti ant Sirtės seklumų, jie nuleido bures ir taip plaukė toliau. 
\par 18 Mus baisiai vargino audra. Todėl rytojaus dieną teko išmesti į jūrą dalį krovinio. 
\par 19 Trečią dieną savo rankomis išmetėme kai kuriuos laivo įrengimus. 
\par 20 Ilgą laiką nematydami nei saulės, nei žvaigždžių, smarkios audros blaškomi, galiausiai praradome bet kokią viltį išsigelbėti. 
\par 21 Žmonės jau ilgą laiką nieko nevalgė. Paulius, atsistojęs tarp jų, tarė: “Vyrai! Reikėjo paklausyti manęs ir neplaukti nuo Kretos į jūrą. Būtų išvengta šių pavojų ir nuostolių. 
\par 22 Tačiau ir dabar raginu laikytis drąsiai. Niekas iš jūsų nežus, vien tik laivas. 
\par 23 Šią naktį mane aplankė angelas Dievo, kuriam aš priklausau ir tarnauju, 
\par 24 ir pasakė: ‘Nebijok, Pauliau! Tu privalai stoti prieš ciesorių. Ir štai Dievas tau dovanoja visus, plaukiančius su tavimi’. 
\par 25 Tad, vyrai, drąsiau! Aš tikiu Dievu, kad taip ir įvyks, kaip man pasakyta. 
\par 26 Mus išmes į kokią nors salą”. 
\par 27 Kai buvome svaidomi Adrijos jūroje keturioliktą naktį, apie vidurnaktį jūreiviams pasirodė, kad artėjame prie kažkokios sausumos. 
\par 28 Jie išmetė grimzlę ir nustatė dvidešimties sieksnių gylį; po kiek laiko išmatavo dar kartą ir berado penkiolika. 
\par 29 Bijodami užšokti ant povandeninių uolų, jie išmetė iš paskuigalio keturis inkarus ir laukė aušros. 
\par 30 Jūreiviai mėgino pabėgti iš laivo. Jie nuleido į jūrą gelbėjimo valtį, neva norėdami išmesti inkarus laivo priešakyje. 
\par 31 Paulius tarė šimtininkui ir kareiviams: “Jeigu šitie nepasiliks laive, jūs neišsigelbėsite”. 
\par 32 Tada kareiviai nukapojo valties virves ir leido jai nukristi į jūrą. 
\par 33 Prieš rytą Paulius paragino visus valgyti, sakydamas: “Šiandien jau keturiolikta diena, kai jūs laukiate nevalgę, nieko burnoje neturėję. 
\par 34 Todėl aš jus prašau valgyti. To reikia jūsų išsigelbėjimui. Nė vienam iš jūsų nenukris nė plaukas nuo galvos!” 
\par 35 Tai pasakęs, jis paėmė duonos, visų akivaizdoje padėkojo Dievui, sulaužė ir pradėjo valgyti. 
\par 36 Tada visi pralinksmėjo ir ėmėsi valgio. 
\par 37 Laive iš viso buvome du šimtai septyniasdešimt šeši žmonės. 
\par 38 Pavalgę jie palengvino laivą, išmesdami jūron javus. 
\par 39 Rytui išaušus, jūreiviai negalėjo pažinti žemės, tiktai pastebėjo nedidelę įlanką lėkštais krantais, į kurią, jei bus įmanoma, jie nutarė pasukti laivą. 
\par 40 Nupjovė inkarus, paliko juos jūroje, atleido vairo diržus ir, iškėlę prieš vėją priekinę burę, leidosi į krantą. 
\par 41 Atsidūrę prie seklumos, jie užšokdino ant jos laivą. Priekis liko tvirtai įsmigęs, o paskuigalis ėmė irti nuo smarkios bangų mūšos. 
\par 42 Kareiviai sumanė išžudyti kalinius, kad kuris išplaukęs nepaspruktų. 
\par 43 Bet šimtininkas, gelbėdamas Paulių, sutrukdė jų sumanymą. Jis įsakė, kad mokantys plaukti pirmi šoktų į jūrą ir plauktų į krantą, 
\par 44 o kiti tai padarytų kas ant lentų, kas ant laivo nuolaužų. Šitaip visi išsigelbėjo ir pasiekė žemę.


\chapter{28}


\par 1 Išsigelbėję jie sužinojo, kad sala vadinasi Melitė. 
\par 2 Barbarai su mumis elgėsi labai draugiškai. Užkūrė ugnį ir pakvietė mus visus prie jos, nes lijo ir buvo šalta. 
\par 3 Paulius pririnko glėbį sausų šakų ir metė į ugnį. Čia nuo kaitros iš laužo iššoko gyvatė ir įsikirto jam į ranką. 
\par 4 Pamatę prie rankos prikibusią gyvatę, barbarai ėmė vienas kitam kalbėti: “Tas žmogus tikriausiai žmogžudys: išsigelbėjo iš jūros, o keršto deivė vis tiek neduoda jam gyventi”. 
\par 5 Tačiau jis nupurtė šliužą į ugnį, nepatirdamas nieko blogo. 
\par 6 Barbarai laukė, kada jis ištins ar staiga kris negyvas. Laukę nesulaukę ir pamatę, kad jam nesidaro nieko blogo, jie pakeitė nuomonę ir ėmė kalbėti, jog jis esąs dievas. 
\par 7 Netoli tos vietos buvo vyriausiojo salos valdininko, vardu Publijus, valdos. Jis mus priėmė ir tris dienas bičiuliškai globojo. 
\par 8 Tuo metu Publijaus tėvas susirgo karštine ir viduriavimu. Paulius užėjo pas jį, pasimeldė, uždėjęs ant jo rankas, ir išgydė. 
\par 9 Po šito įvykio ir kiti salos gyventojai, turėję ligų, ėjo pas Paulių ir buvo pagydomi. 
\par 10 Už tai jie mus didžiai gerbė, o išvykstant aprūpino viskuo, ko mums reikėjo. 
\par 11 Po trijų mėnesių mes išplaukėme žiemojusiu saloje Aleksandrijos laivu, kuris turėjo Dvynių ženklą. 
\par 12 Atplaukę į Sirakūzus, prastovėjome tris dienas. 
\par 13 Iš ten plaukdami, esant nepalankiam vėjui, pasiekėme Regijų ir dar po dienos, ėmus pūsti pietų vėjui, kitą dieną atvykome į Puteolus. 
\par 14 Ten susitikome su broliais; jie pakvietė mus septynioms dienoms paviešėti. Taip mes atvykome į Romą. 
\par 15 Tenykščiai broliai, išgirdę apie mus, atėjo pasitikti iki Apijaus aikštės ir Trijų tavernų. Juos išvydęs, Paulius dėkojo Dievui ir įgavo naujo pasitikėjimo. 
\par 16 Kai atvykome į Romą, šimtininkas perdavė kalinius sargybos viršininkui, o Pauliui buvo leista apsigyventi vienam su saugojančiu jį kareiviu. 
\par 17 Po trijų dienų Paulius pakvietė pas save žydų vadovus. Jiems susirinkus, jis prabilo: “Vyrai broliai! Nors aš nieku nesu nusikaltęs nei tautai, nei mūsų tėvų papročiams, buvau Jeruzalėje suimtas ir atiduotas į romėnų rankas. 
\par 18 Tie ištardę, norėjo mane paleisti, nes nerado jokio mirties verto nusižengimo. 
\par 19 Kadangi žydai prieštaravo, turėjau šauktis ciesoriaus, tiktai ne tam, kad apkaltinčiau savo tautą. 
\par 20 Dėl šios priežasties ir pakviečiau jus, kad su jumis pasimatyčiau ir pasikalbėčiau; nes dėl Izraelio vilties esu surakintas šita grandine!” 
\par 21 Jie atsakė jam: “Mes nesame gavę apie tave iš Judėjos laiškų, ir nė vienas iš atvykusių brolių nepranešė ir nekalbėjo nieko blogo apie tave. 
\par 22 Vis dėlto norėtume išgirsti tavo pažiūras. Mat apie šitą sektą tiek težinome, jog jai visur prieštaraujama”. 
\par 23 Paskyrę jam dieną, jie gausiai susirinko pas jį svečių kambaryje. Nuo ryto iki vakaro jis aiškino jiems ir liudijo apie Dievo karalystę, įrodinėdamas jiems Jėzų iš Mozės Įstatymo ir Pranašų. 
\par 24 Vieni jo žodžiais patikėjo, kiti ne. 
\par 25 Nesutardami tarpusavyje, jie ėmė skirstytis, o Paulius tepasakė jiems viena: “Teisingai Šventoji Dvasia yra mūsų tėvams pasakiusi per pranašą Izaiją: 
\par 26 ‘Eik pas šitą tautą ir sakyk: girdėti girdėsite, bet nesuprasite, žiūrėti žiūrėsite, bet nematysite. 
\par 27 Šitų žmonių širdis aptuko, jie prastai girdėjo ausimis ir užmerkė akis, kad kartais nepamatytų akimis, neišgirstų ausimis, nesuprastų širdimi ir neatsiverstų, ir Aš jų nepagydyčiau’. 
\par 28 Tebūnie tad jums žinoma: šis Dievo išgelbėjimas yra pasiųstas pagonims, ir jie išgirs”. 
\par 29 Jam tai pasakius, žydai išėjo, smarkiai ginčydamiesi tarpusavyje. 
\par 30 Paulius gyveno savo išsinuomotame name ištisus dvejus metus ir priiminėdavo visus, kurie pas jį ateidavo. 
\par 31 Jis skelbė Dievo karalystę ir labai drąsiai, netrukdomas mokė apie Viešpatį Jėzų Kristų.



\end{document}