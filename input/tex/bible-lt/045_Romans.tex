\begin{document}

\title{Laiškas romiečiams}

\chapter{1}


\par 1 Paulius, Jėzaus Kristaus tarnas, pašauktas apaštalu, išrinktas skelbti Dievo Evangeliją, 
\par 2 kurią Jis iš anksto pažadėjo per savo pranašus Šventuosiuose Raštuose, 
\par 3 apie Jo Sūnų, kūnu kilusį iš Dovydo palikuonių, 
\par 4 šventumo Dvasia per prisikėlimą iš numirusių pristatytą galingu Dievo Sūnumi,­Jėzų Kristų, mūsų Viešpatį. 
\par 5 Per Jį gavome malonę ir apaštalystę, kad Jo vardu padarytume klusnias tikėjimui visas tautas, 
\par 6 iš kurių ir jūs esate Jėzaus Kristaus pašaukti. 
\par 7 Visiems Dievo numylėtiesiems, esantiems Romoje, pašauktiesiems šventiesiems: tebūna jums malonė bei ramybė nuo mūsų Dievo Tėvo ir Viešpaties Jėzaus Kristaus! 
\par 8 Pirmiausia dėkoju savo Dievui per Jėzų Kristų už jus visus, nes visame pasaulyje kalbama apie jūsų tikėjimą. 
\par 9 Man liudytojas Dievas, kuriam tarnauju dvasia, skelbdamas Jo Sūnaus Evangeliją, jog be paliovos jus prisimenu savo maldose, 
\par 10 prašydamas, kad Dievo valia man pavyktų kaip nors atvykti pas jus. 
\par 11 Trokštu jus pamatyti, kad galėčiau perduoti šiek tiek dvasinių dovanų jums sustiprinti, 
\par 12 tai yra drauge pasiguosti bendru jūsų ir mano tikėjimu. 
\par 13 Noriu, broliai, kad žinotumėte, jog ne kartą ketinau atvykti pas jus,­bet iki šiol vis pasitaikydavo kliūčių,­kad ir tarp jūsų turėčiau šiek tiek vaisių kaip ir tarp kitų pagonių. 
\par 14 Juk aš skolingas graikams ir barbarams, mokytiems ir nemokytiems. 
\par 15 Štai kodėl mano širdį traukia ir jums Romoje skelbti Evangeliją. 
\par 16 Aš nesigėdiju Evangelijos, nes ji yra Dievo jėga išgelbėti kiekvienam, kuris tiki, pirma žydui, paskui graikui. 
\par 17 Joje apsireiškia Dievo teisumas iš tikėjimo į tikėjimą, kaip parašyta: “Teisusis gyvens tikėjimu”. 
\par 18 Dievo rūstybė apsireiškia iš dangaus už visokią žmonių bedievystę ir neteisybę, kai teisybę jie užgniaužia neteisumu. 
\par 19 Juk tai, kas gali būti žinoma apie Dievą, jiems aišku, nes Dievas jiems tai apreiškė. 
\par 20 Jo neregimosios ypatybės­Jo amžinoji galybė ir dievystė­nuo pat pasaulio sukūrimo aiškiai suvokiamos iš Jo kūrinių, todėl jie nepateisinami. 
\par 21 Pažinę Dievą, jie negarbino Jo kaip Dievo ir Jam nedėkojo, bet tuščiai mąstydami paklydo, ir neišmani jų širdis aptemo. 
\par 22 Vadindami save išmintingais, tapo kvaili. 
\par 23 Jie išmainė nenykstančiojo Dievo šlovę į nykstančius žmogaus, paukščių, keturkojų bei šliužų atvaizdus. 
\par 24 Todėl Dievas per jų širdies geidulius atidavė juos neskaistumui, kad jie patys terštų savo kūnus. 
\par 25 Jie Dievo tiesą iškeitė į melą ir garbino kūrinius bei tarnavo jiems, o ne Kūrėjui, kuris palaimintas per amžius. Amen! 
\par 26 Todėl Dievas paliko juos gėdingų aistrų valiai. Jų moterys prigimtinius santykius pakeitė priešingais prigimčiai. 
\par 27 Panašiai ir vyrai, pametę prigimtinius santykius su moterimis, užsigeidė vienas kito, ištvirkavo vyrai su vyrais, ir gaudavo už savo paklydimą vertą atpildą. 
\par 28 Kadangi jie nesirūpino pažinti Dievą, tai Dievas leido jiems vadovautis netikusiu išmanymu ir daryti, kas nepridera. 
\par 29 Todėl jie pilni visokio neteisumo, netyrumo, piktybių, godulystės ir piktumo, pilni pavydo, žudynių, nesantaikos, klastingumo, paniekos, apkalbų. 
\par 30 Tai šmeižikai, nekenčiantys Dievo, akiplėšos, išpuikėliai, pagyrūnai, išradingi piktadariai, neklausantys tėvų, 
\par 31 neprotingi, nepatikimi, nemylintys, neatlaidūs, negailestingi. 
\par 32 Nors jie žino teisingą Dievo nuosprendį, kad visa tai darantys verti mirties, jie ne tik patys taip daro, bet ir palaiko taip darančius.


\chapter{2}


\par 1 Esi nepateisinamas, kas bebūtum, žmogau, kuris teisi kitą. Juk teisdamas kitą, pasmerki save, nes ir pats tai darai, už ką teisi. 
\par 2 Mes žinome, kad Dievas teisingai teis tuos, kurie tokius nusikaltimus daro. 
\par 3 Nejaugi manai, žmogau, pats taip darydamas ir teisdamas taip darančius, išvengsiąs Dievo teismo?! 
\par 4 Kaip drįsti niekinti Jo gerumo, pakantumo ir kantrumo turtus? Ar nesupranti, kad Dievo gerumas skatina tave atgailauti? 
\par 5 Deja, savo užkietėjimu bei neatgailaujančia širdimi tu pats sau kaupi rūstybę Dievo rūstybės ir Jo teisingo teismo apsireiškimo dienai. 
\par 6 Jis kiekvienam atmokės pagal jo darbus: 
\par 7 tiems, kurie, ištvermingai darydami gera, ieško šlovės, garbingumo ir nemirtingumo,­amžinuoju gyvenimu, 
\par 8 o išpuikėliams, kurie nepaklūsta tiesai, bet yra pasidavę neteisumui,­pykčiu ir rūstybe. 
\par 9 Sielvartas ir suspaudimas sielai kiekvieno žmogaus, kuris daro bloga, pirma žydo, paskui graiko. 
\par 10 Ir šlovė, pagarba bei ramybė kiekvienam, kuris daro gera, pirma žydui, paskui graikui. 
\par 11 Juk Dievas nėra šališkas. 
\par 12 Visi, kurie nusidėjo, neturėdami įstatymo, pražus be įstatymo, o visi, kurie nusidėjo, turėdami įstatymą, bus nuteisti pagal įstatymą. 
\par 13 Ne įstatymo klausytojai teisūs Dievo akyse, bet įstatymo vykdytojai bus išteisinti. 
\par 14 Kai jokio įstatymo neturintys pagonys iš prigimties vykdo įstatymo reikalavimus, tada jie­neturintys įstatymo­patys sau yra įstatymas. 
\par 15 Jie parodo, kad įstatymo reikalavimai įrašyti jų širdyse, ir tai liudija jų sąžinė bei mintys, kurios tai kaltina, tai teisina viena kitą. 
\par 16 Aną dieną Dievas per Jėzų Kristų teis žmonių slėpinius, kaip sako mano skelbiama Evangelija. 
\par 17 Štai tu vadiniesi žydas, pasikliauji įstatymu ir giriesi Dievu. 
\par 18 Tu žinai Jo valią ir, įstatymo pamokytas, išmanai, kas geriau. 
\par 19 Įsitikinęs, kad esi aklųjų vadovas, šviesa tamsoje esantiems, 
\par 20 neišmanančių mokytojas, kūdikių auklėtojas, turįs įstatyme išreikštą pažinimą ir tiesą. 
\par 21 Tai kodėl tu, mokydamas kitus, nepamokai pats savęs? Kodėl, liepdamas nevogti, pats vagi? 
\par 22 Sakydamas nesvetimauti, pats svetimauji? Bjaurėdamasis stabais, pats apiplėši šventyklas? 
\par 23 Giriesi įstatymu, o paniekini Dievą, laužydamas įstatymą? 
\par 24 Juk parašyta: “Dėl jūsų piktžodžiauja Dievo vardui pagonys”. 
\par 25 Apipjaustymas, tiesa, naudingas, jei vykdai įstatymą. O jeigu esi įstatymo laužytojas, tavo apipjaustymas tampa neapipjaustymu. 
\par 26 Taigi, kai neapipjaustytas žmogus laikosi įstatymo reikalavimų, ar jo neapipjaustymas nebus įskaitytas apipjaustymu? 
\par 27 Ar nuo gimimo neapipjaustytas kūne, bet vykdantis įstatymą nepasmerks tavęs, kuris laužai įstatymą, turėdamas jo raidę ir apipjaustymą? 
\par 28 Ne tas yra žydas, kuris išoriškai laikomas žydu, ir ne tas apipjaustymas, kuris išoriškai atliktas kūne. 
\par 29 Bet tas yra žydas, kuris toks viduje, ir tada yra apipjaustymas, kai širdis apipjaustyta dvasioje, o ne pagal raidę. Tokiam ir šlovė ne iš žmonių, bet iš Dievo.


\chapter{3}


\par 1 Koks tada pranašumas būti žydu arba kokia nauda iš apipjaustymo? 
\par 2 Visokeriopas! Pirmiausia tas, kad jiems buvo patikėtas Dievo žodis. 
\par 3 Jei kai kurie tapo netikintys,­negi jų netikėjimas panaikins Dievo ištikimybę? 
\par 4 Jokiu būdu! Dievas išlieka teisingas, o kiekvienas žmogus­melagis, kaip parašyta: “Kad Tu būtum pripažintas teisus savo žodžiuose ir laimėtum, kai esi teisiamas”. 
\par 5 Jei mūsų neteisumas iškelia Dievo teisumą,­ką gi sakysime? Gal Dievas neteisus, rūsčiai bausdamas? Kalbu, kaip žmonėms įprasta. 
\par 6 Jokiu būdu! Kaip tada Dievas galėtų teisti pasaulį? 
\par 7 Bet jeigu Dievo tiesa per mano melagystę tik dar labiau iškilo Jo šlovei, tai kam dar teisti mane kaip nusidėjėlį? 
\par 8 Tai gal “darykime bloga, kad išeitų gera”,­kaip esame šmeižiami ir kaip kai kurie sako mus skelbiant? Tokie pasmerkti vertai. 
\par 9 Tai ką gi? Ar mes turime pirmenybę? Visai ne! Juk jau įrodėme, kad žydai ir pagonys­visi yra nuodėmės valdžioje, 
\par 10 kaip parašyta: “Nėra teisaus, nėra nė vieno. 
\par 11 Nėra išmanančio, nėra kas Dievo ieškotų. 
\par 12 Visi paklydo ir tapo netikusiais; nėra kas darytų gera, nėra nė vieno! 
\par 13 Jų gerklė­atviras kapas; savo liežuviais klastas jie raizgė, gyvačių nuodai jų lūpose. 
\par 14 Jų burna pilna keiksmų ir kartumo, 
\par 15 jų kojos eiklios kraujo pralieti, 
\par 16 jų keliuose griuvimas ir vargas. 
\par 17 Jie nepažino taikos kelio, 
\par 18 ir prieš jų akis nestovi Dievo baimė”. 
\par 19 Mes gi žinome, kad, ką besakytų įstatymas, jis kalba tiems, kurie yra įstatymo valdžioje, kad visos burnos užsičiauptų ir visas pasaulis pasirodytų kaltas prieš Dievą, 
\par 20 nes įstatymo darbais Jo akivaizdoje nebus išteisintas nė vienas žmogus. Per įstatymą tik pažįstame nuodėmę. 
\par 21 Bet dabar, nepriklausomai nuo įstatymo, yra apreikštas Dievo teisumas, kurį paliudijo Įstatymas ir Pranašai,­ 
\par 22 Dievo teisumas, tikėjimu į Jėzų Kristų duodamas visiems, kurie tiki. Nėra jokio skirtumo, 
\par 23 nes visi nusidėjo ir stokoja Dievo šlovės, 
\par 24 o išteisinami dovanai Jo malone dėl atpirkimo, kuris yra Jėzuje Kristuje. 
\par 25 Dievas Jį paskyrė permaldavimo auka, veikiančia per tikėjimą Jo kraujo galia. Jis parodė savo teisumą tuo, kad, būdamas kantrus, nenubaudė už nuodėmes, padarytas anksčiau, 
\par 26 ir parodė savo teisumą dabartiniu metu, pasirodydamas esąs teisus ir išteisinantis tą, kuris tiki Jėzų. 
\par 27 Kur tada pagrindas girtis? Jis atmestas. Kokiu įstatymu? Darbų? Ne, tik tikėjimo įstatymu. 
\par 28 Mes įsitikinę, kad žmogus išteisinamas tikėjimu, be įstatymo darbų. 
\par 29 Argi Dievas­tiktai žydų Dievas? Ar Jis nėra ir pagonių? Taip, ir pagonių, 
\par 30 nes tėra vienas Dievas, kuris per tikėjimą išteisins apipjaustytus ir per tikėjimą išteisins neapipjaustytus. 
\par 31 O gal tikėjimu panaikiname įstatymą? Jokiu būdu! Priešingai, mes įstatymą įtvirtiname.


\chapter{4}


\par 1 O ką pasakysime gavus Abraomą­mūsų protėvį pagal kūną? 
\par 2 Jei Abraomas būtų buvęs išteisintas darbais, jis turėtų kuo pasigirti, tik ne prieš Dievą. 
\par 3 Bet ką sako Raštas? “Abraomas patikėjo Dievu, ir tai jam buvo įskaityta teisumu”. 
\par 4 Tam, kuris dirba, atlyginimas nelaikomas malone, bet skola. 
\par 5 O tam, kuris nedirba, bet tiki Tuo, kuris išteisina bedievį, jo tikėjimas įskaitomas jam teisumu. 
\par 6 Taip ir Dovydas skelbia palaiminimą žmogui, kuriam Dievas be darbų įskaito teisumą: 
\par 7 “Palaiminti, kurių nusikaltimai atleisti, kurių nuodėmės uždengtos; 
\par 8 palaimintas žmogus, kuriam Viešpats nuodėmės neįskaito!” 
\par 9 Ar šis palaiminimas taikomas tik apipjaustytiesiems, ar taip pat ir neapipjaustytiesiems? Mes sakėme, kad Abraomui tikėjimas buvo įskaitytas teisumu. 
\par 10 Kokiu būdu? Jam esant apipjaustytam ar neapipjaustytam? Ne po apipjaustymo, bet prieš apipjaustymą. 
\par 11 Jis gavo apipjaustymo žymę kaip antspaudą tikėjimo teisumo, kurį turėjo, būdamas dar neapipjaustytas. Taip jis tapo tėvu visiems tikintiesiems iš neapipjaustytųjų, kad ir jiems būtų įskaitytas teisumas, 
\par 12 ir apipjaustymo tėvu tiems, kurie ne tik apipjaustyti, bet ir vaikšto mūsų tėvo Abraomo tikėjimo pėdomis, kurį jis turėjo, būdamas dar neapipjaustytas. 
\par 13 Ne įstatymu rėmėsi Abraomui arba jo palikuonims duotas pažadas, kad paveldės pasaulį, bet tikėjimo teisumu. 
\par 14 Jei paveldėtojai būtų tie, kurie remiasi įstatymu, tai tikėjimas būtų tuščias, o pažadas netektų vertės. 
\par 15 Juk įstatymą lydi baudžianti rūstybė, o kur nėra įstatymo, ten nėra ir nusižengimo. 
\par 16 Taigi paveldėjimas priklauso nuo tikėjimo, kad būtų iš malonės ir pažadas būtų tikras visiems palikuonims, ne tik tiems, kurie remiasi įstatymu, bet ir tiems, kurie turi tikėjimą Abraomo, kuris yra mūsų visų tėvas, 
\par 17 kaip parašyta: “Aš padariau tave daugelio tautų tėvu”;­tėvas prieš Dievą, kuriuo jis tikėjo, kuris atgaivina mirusius, ir tai, ko nėra, vadina taip, lyg būtų. 
\par 18 Nesant jokios vilties, Abraomas patikėjo viltimi ir taip tapo daugelio tautų tėvu, kaip jam buvo pasakyta: “Tokie bus tavo palikuonys”. 
\par 19 Jis nepavargo tikėti ir nelaikė savo kūno apmirusiu (nors jam buvo apie šimtą metų) ir Saros įsčių apmirusiomis. 
\par 20 Jis nepasidavė netikėjimui Dievo pažadu, bet buvo tvirtas tikėjime, teikdamas Dievui šlovę 
\par 21 ir būdamas visiškai įsitikinęs, jog, ką Jis pažadėjo, įstengs ir įvykdyti. 
\par 22 Todėl jam tai buvo įskaityta teisumu. 
\par 23 Tačiau ne vien apie jį parašyta, kad “buvo įskaityta”, 
\par 24 bet ir apie mus,­nes turės būti įskaityta ir mums, jei tikime Tą, kuris prikėlė iš numirusių Jėzų, mūsų Viešpatį, 
\par 25 paaukotą dėl mūsų nusikaltimų ir prikeltą mums išteisinti.


\chapter{5}


\par 1 Taigi, išteisinti tikėjimu, turime ramybę su Dievu per mūsų Viešpatį Jėzų Kristų, 
\par 2 per kurį tikėjimu pasiekiame tą malonę, kurioje stovime ir džiaugiamės Dievo šlovės viltimi. 
\par 3 Ir ne vien tuo. Mes taip pat džiaugiamės sielvartais, žinodami, kad sielvartas ugdo ištvermę, 
\par 4 ištvermė­patirtį, patirtis­viltį. 
\par 5 O viltis neapvilia, nes Dievo meilė yra išlieta mūsų širdyse per Šventąją Dvasią, kuri mums duota. 
\par 6 Mums dar esant silpniems, Kristus savo metu numirė už bedievius. 
\par 7 Vargu ar kas sutiktų mirti už teisųjį; nebent kas ryžtųsi mirti už geradarį. 
\par 8 O Dievas mums parodė savo meilę tuo, kad Kristus mirė už mus, kai tebebuvome nusidėjėliai. 
\par 9 Tad dar tikriau dabar, kai esame išteisinti Jo krauju, mes būsime per Jį išgelbėti nuo rūstybės. 
\par 10 Jeigu, kai dar buvome priešai, mus sutaikė su Dievu Jo Sūnaus mirtis, tai tuo labiau mus išgelbės Jo gyvybė, kai jau esame sutaikinti. 
\par 11 Negana to,­mes džiaugiamės Dieve per mūsų Viešpatį Jėzų Kristų, kuriuo esame sutaikyti. 
\par 12 Todėl, kaip per vieną žmogų nuodėmė įėjo į pasaulį, o per nuodėmę mirtis, taip ir mirtis pasiekė visus žmones, nes visi nusidėjo. 
\par 13 Nuodėmė buvo pasaulyje ir iki įstatymo, bet, nesant įstatymo, nuodėmė neįskaitoma. 
\par 14 Vis dėlto nuo Adomo iki Mozės viešpatavo mirtis net tiems, kurie nebuvo padarę nuodėmių, panašių į nusikaltimą Adomo, kuris buvo Būsimojo provaizdis. 
\par 15 Bet su dovana yra ne taip kaip su kalte. Jei dėl vieno žmogaus nusikaltimo mirė daugelis, tai tuo labiau Dievo malonė ir malonės dovana per vieną Žmogų, Jėzų Kristų, gausiai atiteko daugybei. 
\par 16 Ne taip yra su dovana kaip su vieno žmogaus nusikaltimu. Juk teismas vieno nusikaltimą pasmerkė, bet laisva dovana iš daugybės nusikaltimų atvedė į išteisinimą. 
\par 17 Jei dėl vieno žmogaus nusikaltimo mirtis įsiviešpatavo per tą vieną, tai nepalyginti labiau tie, kurie su perteklium gauna malonės bei teisumo dovaną, viešpataus gyvenime per vieną Jėzų Kristų. 
\par 18 Todėl kaip vieno žmogaus nusikaltimas visiems žmonėms užtraukė teismą ir pasmerkimą, taip vieno Žmogaus teisumas visiems pelnė išteisinimą, kad gyventų. 
\par 19 Kaip vieno žmogaus neklusnumu daugelis tapo nusidėjėliais, taip ir vieno klusnumu daugelis taps teisūs. 
\par 20 Be to, įstatymas įsiterpė, kad nusikaltimas dar labiau padidėtų. Bet kur buvo apstu nuodėmės, ten dar apstesnė tapo malonė, 
\par 21 kad kaip nuodėmė viešpatavo mirtimi, taip malonė viešpatautų teisumu amžinajam gyvenimui per Jėzų Kristų, mūsų Viešpatį.


\chapter{6}


\par 1 Ką gi sakysime? Gal mums pasilikti nuodėmėje, kad gausėtų malonė? 
\par 2 Jokiu būdu! Mirę nuodėmei, kaipgi gyvensime joje? 
\par 3 Argi nežinote, jog mes, pakrikštyti Jėzuje Kristuje, buvome pakrikštyti Jo mirtyje? 
\par 4 Taigi krikštu mes esame kartu su Juo palaidoti mirtyje, kad kaip Kristus buvo prikeltas iš numirusių Tėvo šlove, taip ir mes gyventume naują gyvenimą. 
\par 5 Jei esame suaugę su Jo mirties paveikslu, būsime suaugę ir su prisikėlimo, 
\par 6 žinodami, jog mūsų senasis žmogus buvo nukryžiuotas kartu su Juo, kad būtų sunaikintas nuodėmės kūnas ir kad mes daugiau nebevergautume nuodėmei. 
\par 7 Juk kas miręs, tas išlaisvintas iš nuodėmės. 
\par 8 Jeigu esame mirę su Kristumi, tikime, kad ir gyvensime su Juo. 
\par 9 Žinome, kad, prisikėlęs iš numirusių, Kristus daugiau nebemiršta; mirtis jau nebeturi Jam galios. 
\par 10 Kad Jis mirė, tai mirė nuodėmei kartą visiems laikams, o kad gyvena­gyvena Dievui. 
\par 11 Taip ir jūs laikykite save mirusiais nuodėmei, o gyvais Dievui Kristuje Jėzuje, mūsų Viešpatyje. 
\par 12 Todėl neleiskite nuodėmei viešpatauti jūsų mirtingame kūne, kad nepasiduotumėte jo geiduliams. 
\par 13 Neduokite nuodėmei savo kūno narių kaip neteisumo ginklų, bet paveskite save Dievui, kaip iš numirusiųjų atgijusius, ir savo narius Jam­kaip teisumo ginklus. 
\par 14 Nuodėmė neturi jums viešpatauti: jūs ne įstatymo, bet malonės valdžioje. 
\par 15 Tai ką? Gal darysime nuodėmes, jei esame ne įstatymo, bet malonės valdžioje? Jokiu būdu! 
\par 16 Argi nežinote, kad pasiduodami kam nors vergauti, jūs iš tiesų tampate vergais to, kuriam paklūstate: ar tai būtų nuodėmė, vedanti į mirtį, ar paklusnumas, vedantis į teisumą. 
\par 17 Bet ačiū Dievui, kad, nors buvote nuodėmės vergais, jūs iš širdies paklusote tam mokymo pavyzdžiui, kuriam buvote pavesti; 
\par 18 ir, išlaisvinti iš nuodėmės, tapote teisumo tarnais. 
\par 19 Kalbu grynai žmogiškai dėl jūsų kūniško silpnumo. Kaip buvote atidavę savo kūno narius vergauti netyrumui ir nedorybei, kad elgtumėtės nedorai, taip pat atiduokite savo narius vergauti teisumui, kad taptumėte šventi. 
\par 20 Būdami nuodėmės vergai, jūs buvote nepriklausomi nuo teisumo. 
\par 21 Bet kokį vaisių turėjote tuomet iš to, ko dabar gėdijatės? Juk tų dalykų galas­mirtis. 
\par 22 O dabar, išlaisvinti iš nuodėmės ir tapę Dievo tarnais, turite kaip vaisių­šventumą, ir kaip baigtį­amžinąjį gyvenimą. 
\par 23 Atpildas už nuodėmę­mirtis, o Dievo dovana­amžinasis gyvenimas per Jėzų Kristų, mūsų Viešpatį.


\chapter{7}


\par 1 Ar nežinote, broliai,­aš kalbu žinantiems įstatymą,­kad įstatymas galioja žmogui, kol jis gyvas? 
\par 2 Pavyzdžiui, ištekėjusi moteris įstatymo surišta su vyru, kol jis gyvas. Jei vyras miršta, ji tampa laisva nuo įstatymo, rišusio ją su vyru. 
\par 3 Ji vadinsis svetimautoja, jei, vyrui gyvam esant, bus žmona kitam. Bet jei vyras miršta, ji tampa laisva nuo įstatymo ir, ištekėdama už kito, nebėra svetimautoja. 
\par 4 Taip ir jūs, mano broliai, per Kristaus kūną esate mirę įstatymui, kad priklausytumėte kitam­ prikeltam iš numirusių­ir kad mes neštume vaisių Dievui. 
\par 5 Kol gyvenome pagal kūną, mumyse veikė įstatymo pažadintos nuodėmingos aistros, ir mes nešėme vaisių mirčiai. 
\par 6 Bet dabar, numirę įstatymui, kuriuo buvome surišti, esame išlaisvinti iš jo, kad tarnautume nauja dvasia, o ne pasenusia raide. 
\par 7 Ką tad pasakysime? Gal įstatymas yra nuodėmė? Jokiu būdu! Bet aš nebūčiau pažinęs nuodėmės, jei nebūtų įstatymo. Nebūčiau suvokęs geismo, jei įstatymas nebūtų pasakęs: “Negeisk!” 
\par 8 Bet nuodėmė, per įsakymus gavusi progą, pažadino manyje visokius geismus, o be įstatymo nuodėmė negyva. 
\par 9 Kadaise be įstatymo aš buvau gyvas. Bet, atėjus įsakymui, atgijo nuodėmė, ir aš numiriau. 
\par 10 Taip man paaiškėjo, kad įsakymas, skirtas gyvenimui, nuvedė mane į mirtį. 
\par 11 Nes įsakymo paskatinta nuodėmė mane suvedžiojo ir juo mane nužudė. 
\par 12 Todėl įstatymas šventas; įsakymas taip pat šventas, ir teisingas, ir geras. 
\par 13 Vadinasi, geras dalykas tapo man mirtimi? Jokiu būdu! Bet nuodėmė pasirodė nuodėme tuo, kad atnešė man mirtį, pasinaudodama geru dalyku,­kad per įsakymą nuodėmė taptų be galo nuodėminga. 
\par 14 Nes mes žinome, kad įstatymas yra dvasiškas, o aš esu kūniškas, parduotas nuodėmei. 
\par 15 Aš net neišmanau, ką darąs, nes darau ne tai, ko noriu, bet tai, ko nekenčiu. 
\par 16 O jei darau tai, ko nenoriu, tada sutinku, kad įstatymas geras. 
\par 17 Tada jau nebe aš tai darau, bet manyje gyvenanti nuodėmė. 
\par 18 Aš juk žinau, kad manyje, tai yra mano kūne, nėra jokio gėrio. Gero trokšti sugebu, o padaryti­ ne. 
\par 19 Aš nedarau gėrio, kurio trokštu, o darau blogį, kurio nenoriu. 
\par 20 O jeigu darau, ko nenoriu, tada nebe aš tai darau, bet manyje gyvenanti nuodėmė. 
\par 21 Taigi aš randu tokį įstatymą: kai trokštu padaryti gera, prie manęs prilimpa bloga. 
\par 22 Juk kaip vidinis žmogus aš gėriuosi Dievo įstatymu. 
\par 23 Bet savo nariuose matau kitą įstatymą, kovojantį su mano proto įstatymu, ir paverčiantį mane belaisviu nuodėmės įstatymo, kuris yra mano nariuose. 
\par 24 Vargšas aš žmogus! Kas išlaisvins mane iš šito mirties kūno! 
\par 25 Bet ačiū Dievui­per mūsų Viešpatį Jėzų Kristų! Taigi aš pats protu tarnauju Dievo įstatymui, o kūnu­nuodėmės įstatymui.


\chapter{8}


\par 1 Taigi dabar nebėra pasmerkimo tiems, kurie yra Kristuje Jėzuje, kurie gyvena ne pagal kūną, bet pagal Dvasią. 
\par 2 Nes gyvenimo Kristuje Jėzuje Dvasios įstatymas išlaisvino mane iš nuodėmės ir mirties įstatymo. 
\par 3 Ko įstatymas nepajėgė, būdamas silpnas dėl kūno, tai įvykdė Dievas. Jis atsiuntė savo Sūnų nuodėmingo kūno pavidalu kaip auką už nuodėmę ir pasmerkė nuodėmę kūne, 
\par 4 kad įstatymo teisumas išsipildytų mumyse, gyvenančiuose ne pagal kūną, bet pagal Dvasią. 
\par 5 Kurie gyvena pagal kūną, tie mąsto kūniškai, o kurie gyvena pagal Dvasią­dvasiškai. 
\par 6 Kūniškas mąstymas­tai mirtis, o dvasiškas­gyvenimas ir ramybė. 
\par 7 Kūniškas mąstymas priešiškas Dievui; jis nepaklūsta Dievo įstatymui ir net negali paklusti. 
\par 8 Ir todėl gyvenantys pagal kūną negali patikti Dievui. 
\par 9 Tačiau jūs negyvenate pagal kūną, bet pagal Dvasią, jei tik Dievo Dvasia gyvena jumyse. O kas neturi Kristaus Dvasios, tas nėra Jo. 
\par 10 Jeigu Kristus yra jumyse, tai kūnas yra miręs dėl nuodėmės, bet dvasia­gyva dėl teisumo. 
\par 11 Jei jumyse gyvena Dvasia To, kuris Jėzų prikėlė iš numirusių, tai Jis­prikėlęs iš numirusių Kristų­atgaivins ir jūsų mirtingus kūnus savo Dvasia, gyvenančia jumyse. 
\par 12 Taigi, broliai, mes nesame skolingi kūnui, kad gyventume pagal kūną. 
\par 13 Jei jūs gyvenate pagal kūną­ mirsite. Bet jei dvasia marinate kūniškus darbus­gyvensite. 
\par 14 Visi, vedami Dievo Dvasios, yra Dievo vaikai. 
\par 15 Jūs gi gavote ne vergystės dvasią, kad vėl bijotumėte, bet gavote įsūnystės Dvasią, kuria šaukiame: “Aba, Tėve!” 
\par 16 Pati Dvasia liudija mūsų dvasiai, kad esame Dievo vaikai. 
\par 17 O jei esame vaikai, tai ir paveldėtojai. Dievo paveldėtojai ir Kristaus bendrapaveldėtojai, jeigu tik su Juo kenčiame, kad su Juo būtume pašlovinti. 
\par 18 Aš manau, jog šio laiko kentėjimai nieko nereiškia, lyginant juos su būsimąja šlove, kuri mumyse bus apreikšta. 
\par 19 Kūrinija su ilgesiu laukia, kada bus apreikšti Dievo sūnūs. 
\par 20 Mat kūrinija buvo pajungta tuštybei,­ne savo noru, bet pavergėjo valia,­su viltimi, 
\par 21 kad ir pati kūrinija bus išlaisvinta iš suirimo vergijos ir įgis šlovingą Dievo vaikų laisvę. 
\par 22 Juk mes žinome, kad visa kūrinija iki šiol dejuoja ir tebėra gimdymo skausmuose. 
\par 23 Ir ne tik ji, bet ir mes patys, turintys pirmuosius Dvasios vaisius,­ir mes dejuojame, kantriai laukdami įsūnijimo, mūsų kūno atpirkimo. 
\par 24 Šia viltimi mes esame išgelbėti, bet regima viltis nėra viltis. Jeigu kas mato, tai kam jam viltis? 
\par 25 Bet jei viliamės to, ko nematome, tada laukiame ištvermingai. 
\par 26 Taip pat ir Dvasia padeda mūsų silpnumui. Nes mes nežinome, ko turėtume melsti, bet pati Dvasia užtaria mus neišsakomom dejonėm. 
\par 27 Širdžių Tyrėjas žino Dvasios mintis, nes Ji užtaria šventuosius pagal Dievo valią. 
\par 28 Be to, mes žinome, kad mylintiems Dievą viskas išeina į gera, būtent Jo tikslu pašauktiesiems. 
\par 29 O kuriuos Jis iš anksto numatė, tuos iš anksto ir paskyrė tapti panašius į Jo Sūnaus atvaizdą, kad šis būtų pirmagimis tarp daugelio brolių. 
\par 30 O kuriuos Jis iš anksto paskyrė, tuos ir pašaukė; kuriuos pašaukė, tuos ir išteisino; kuriuos išteisino, tuos ir pašlovino. 
\par 31 Tai ką dėl viso šito pasakysime? Jei Dievas už mus, tai kas gi prieš mus?! 
\par 32 Tas, kuris nepagailėjo savo Sūnaus, bet atidavė Jį už mus visus,­kaipgi Jis ir visko nedovanotų kartu su Juo? 
\par 33 Kas kaltins Dievo išrinktuosius? Juk Dievas išteisina! 
\par 34 Kas pasmerks? Kristus mirė, bet buvo prikeltas ir yra Dievo dešinėje, ir užtaria mus. 
\par 35 Kas gi mus atskirs nuo Kristaus meilės? Ar sielvartas? ar nelaimė? ar persekiojimas? ar badas? ar nuogumas? ar pavojus? ar kalavijas? 
\par 36 Parašyta: “Dėl Tavęs mes žudomi ištisą dieną, laikomi avimis skerdimui”. 
\par 37 Tačiau visuose šiuose dalykuose mes esame daugiau negu nugalėtojai per Tą, kuris mus pamilo. 
\par 38 Ir aš įsitikinęs, kad nei mirtis, nei gyvenimas, nei angelai, nei kunigaikštystės, nei galybės, nei dabartis, nei ateitis, 
\par 39 nei aukštumos, nei gelmės, nei jokie kiti kūriniai negalės mūsų atskirti nuo Dievo meilės, kuri yra Kristuje Jėzuje, mūsų Viešpatyje.


\chapter{9}


\par 1 Sakau tiesą Kristuje, nemeluoju,­tai liudija ir mano sąžinė Šventojoje Dvasioje, 
\par 2 kad man labai sunku ir nuolat liūdi mano širdis. 
\par 3 Man mieliau būtų pačiam būti prakeiktam ir atskirtam nuo Kristaus vietoj savo brolių, tautiečių pagal kūną, 
\par 4 kurie yra izraelitai, turintys įsūnystę, šlovę, sandoras, įstatymą, tarnavimą Dievui ir pažadus. 
\par 5 Iš jų­tėvai, ir iš jų kūno atžvilgiu yra kilęs Kristus­visiems viešpataujantis Dievas, palaimintas per amžius. Amen! 
\par 6 Netiesa, kad gali neišsipildyti Dievo žodis. Ne visi, kilę iš Izraelio, priklauso Izraeliui. 
\par 7 Ir ne visi Abraomo palikuonys yra jo vaikai, bet kaip pasakyta: “Iš Izaoko tau bus pašaukti palikuonys”. 
\par 8 Tai reiškia, kad ne vaikai pagal kūną yra Dievo vaikai, bet vaikai pagal pažadą laikomi palikuonimis. 
\par 9 O pažado žodis toks: “Apie tą laiką Aš ateisiu, ir Sara turės sūnų”. 
\par 10 Ir ne tik tai, bet taip pat ir Rebekai, pradėjusiai iš vieno, mūsų tėvo Izaoko 
\par 11 (dar jos dvyniams negimus ir jiems dar nepadarius nei gero, nei blogo,­kad Dievo nutarimas įvyktų pagal pasirinkimą, ne dėl darbų, bet šaukiančiojo valia), 
\par 12 buvo pasakyta: “Vyresnysis tarnaus jaunesniajam”, 
\par 13 kaip ir parašyta: “Jokūbą pamilau, o Ezavo nekenčiau”. 
\par 14 Ką gi pasakysime? Gal Dievas neteisingai daro? Jokiu būdu! 
\par 15 Jis Mozei kalba: “Aš pasigailėsiu to, kurio norėsiu pasigailėti, ir būsiu gailestingas tam, kuriam norėsiu gailestingas būti”. 
\par 16 Taigi viskas priklauso ne nuo to, kuris trokšta ar kuris bėga, bet nuo gailestingojo Dievo. 
\par 17 Juk Raštas faraonui sako: “Aš iškėliau tave, kad parodyčiau savo jėgą tau ir kad mano vardas būtų skelbiamas visoje žemėje”. 
\par 18 Vadinasi, ko Jis nori, to pasigaili, ir kurį nori, tą užkietina. 
\par 19 Gal man pasakysi: “O už ką tada Jis kaltina? Kas gi galėtų atsispirti Jo valiai?” 
\par 20 Ak, žmogau! Kas gi, tiesą sakant, tu toks esi, kad drįsti prieštarauti Dievui? Argi dirbinys klausia meistro: “Kodėl mane tokį padarei?” 
\par 21 Ar puodžius neturi galios moliui, kad iš to paties minkalo pagamintų vieną indą garbingam panaudojimui, o kitą negarbingam? 
\par 22 O jeigu Dievas, norėdamas parodyti savo rūstybę ir apreikšti savo jėgą, didžiu kantrumu pakentė pražūčiai nužiestus rūstybės indus, 
\par 23 kad apreikštų ir savo šlovės turtus gailestingumo indams, kuriuos iš anksto paruošė šlovei,­ 
\par 24 ir mus pašaukė ne tik iš žydų, bet ir iš pagonių? 
\par 25 Jis kalba per Ozėją: “Ne savo tautą pavadinsiu savąja tauta ir nemylimą­mylima. 
\par 26 Ir toje vietoje, kur jiems buvo sakyta: ‘Jūs ne manoji tauta’, ten jie bus vadinami gyvojo Dievo vaikais”. 
\par 27 O Izaijas šaukia apie Izraelį: “Nors Izraelio vaikų skaičius būtų kaip jūros smiltys, tik likutis bus išgelbėtas. 
\par 28 Nes Jis pabaigs darbą, greitai įvykdydamas teisumą, skubiai Viešpats atliks darbą žemėje”. 
\par 29 Izaijas nusakė iš anksto: “Jei kareivijų Viešpats nebūtų mums palikuonių palikęs, būtume tapę kaip Sodoma, būtume į Gomorą panašūs”. 
\par 30 Tai ką gi pasakysime? Kad pagonys, kurie neieškojo teisumo, gavo teisumą, būtent teisumą iš tikėjimo. 
\par 31 O Izraelis, ieškojęs teisumo įstatyme, nepasiekė teisumo įstatymo. 
\par 32 Kodėl? Todėl, kad ieškojo jo ne tikėjimu, bet įstatymo darbais. Jie užkliuvo už suklupimo akmens, 
\par 33 kaip parašyta: “Štai dedu Sione suklupimo akmenį, papiktinimo uolą; bet kas Juo tiki, nebus sugėdintas”.


\chapter{10}


\par 1 Broliai, mano širdies troškimas ir malda Dievui yra už Izraelį,­kad jie išsigelbėtų. 
\par 2 Aš jiems liudiju, kad jie turi uolumo Dievui, tačiau be pažinimo. 
\par 3 Nesuprasdami Dievo teisumo ir bandydami įtvirtinti savąjį teisumą, jie nepakluso Dievo teisumui. 
\par 4 Nes įstatymo pabaiga­Kristus, išteisinimui kiekvieno, kuris tiki. 
\par 5 Mozė rašo apie teisumą iš įstatymo: “Jį vykdydamas žmogus juo gyvens”. 
\par 6 Bet teisumas iš tikėjimo kalba taip: “Nesakyk savo širdyje: ‘Kas įžengs į dangų?’­tai yra Kristaus atsivesti; 
\par 7 arba: ‘Kas nusileis į bedugnę?’­ tai yra Kristaus iš numirusių susigrąžinti”. 
\par 8 Bet ką jis sako?­“Arti tavęs yra žodis­tavo burnoje ir tavo širdyje”,­tai yra mūsų skelbiamas tikėjimo žodis. 
\par 9 Jeigu lūpomis išpažinsi Viešpatį Jėzų ir širdimi tikėsi, kad Dievas Jį prikėlė iš numirusių, būsi išgelbėtas. 
\par 10 Nes širdimi tikima, ir taip įgyjamas teisumas, o lūpomis išpažįstama, ir taip įgyjamas išgelbėjimas. 
\par 11 Raštas juk sako: “Kiekvienas, kuris Jį tiki, nebus sugėdintas”. 
\par 12 Nėra skirtumo tarp žydo ir graiko, nes tas pats Viešpats visiems, turtingas kiekvienam, kuris Jo šaukiasi, 
\par 13 juk “kiekvienas, kuris šaukiasi Viešpaties vardo, bus išgelbėtas”. 
\par 14 Kaip žmonės šauksis To, kurio neįtikėjo? Ir kaip jie įtikės Tą, apie kurį negirdėjo? Kaip išgirs be skelbėjo? 
\par 15 Ir kaip jie skelbs, jei nebus pasiųsti? Kaip parašyta: “Kokios puikios kojos tų, kurie skelbia ramybės Evangeliją, kurie neša geras žinias!” 
\par 16 Bet ne visi pakluso Evangelijai. Nes Izaijas sako: “Viešpatie, kas patikėjo mūsų skelbimu?” 
\par 17 Taigi tikėjimas­iš klausymo, klausymas­iš Dievo žodžio. 
\par 18 Bet aš klausiu: argi jie negirdėjo? Kaipgi ne! “Po visą žemę pasklido jų garsas, ir jų žodžiai­iki pasaulio pakraščių”. 
\par 19 Klausiu toliau: ar Izraelis nežinojo? Bet Mozė pirmas sako: “Aš sukelsiu jums pavydą per netautą, sukelsiu pyktį per neišmanančią tautą”. 
\par 20 Izaijas labai drąsiai sako: “Mane atrado tie, kurie manęs neieškojo, apsireiškiau tiems, kurie apie mane neklausinėjo”. 
\par 21 Bet Izraeliui sako: “Ištisą dieną Aš laikiau ištiesęs savo rankas į neklusnią ir prieštaraujančią tautą”.


\chapter{11}


\par 1 Tad aš klausiu: ar Dievas atstūmė savo tautą? Jokiu būdu! Juk ir aš izraelitas, iš Abraomo palikuonių, iš Benjamino giminės. 
\par 2 Dievas neatstūmė savosios tautos, kurią iš anksto numatė. Ar nežinote, ką sako Raštas apie Eliją, kai šis skundžiasi Izraeliu: 
\par 3 “Viešpatie, jie išžudė Tavo pranašus, išgriovė Tavo aukurus; aš vienas belikau, ir jie tyko mano gyvybės”. 
\par 4 O kaip skamba Dievo atsakymas? “Aš pasilaikiau septynis tūkstančius vyrų, kurie nesulenkė kelių prieš Baalį”. 
\par 5 Ir dabartiniu metu yra malonės išrinktas likutis. 
\par 6 Ir jei malone, tai ne dėl darbų, nes tada malonė nebūtų malonė. Bet jeigu darbais, tai jau nebus malonė; kitaip darbas nebūtų darbas. 
\par 7 Tai ką gi? Izraelis nepasiekė to, ko ieškojo. Pasiekė tiktai išrinktoji dalis. Kiti buvo apakinti, 
\par 8 kaip parašyta: “Dievas jiems siuntė snaudulio dvasią, kad akys neregėtų ir ausys negirdėtų iki šios dienos”. 
\par 9 Ir Dovydas sako: “Jų stalas tepavirsta jiems spąstais, žabangais, suklupimo akmeniu ir atpildu. 
\par 10 Tegul aptemsta jų akys, kad neregėtų, ir jų nugarą laikyk nuolat sulenktą”. 
\par 11 Tad aš klausiu: negi izraelitai taip suklupo, kad pargriūtų? Jokiu būdu! Tik per jų suklupimą pagonims atėjo išgelbėjimas, kad juos paimtų pavydas. 
\par 12 Bet jeigu jų suklupimas yra pasauliui praturtinimas ir jų sumažėjimas­pagonims praturtinimas, tai ką duos jų visuma? 
\par 13 Jums, pagonims, sakau: būdamas pagonių apaštalas, aš gerbiu savo tarnavimą: 
\par 14 gal kaip nors man pavyks sukelti savo tautiečių pavydą ir bent kai kuriuos išgelbėti. 
\par 15 Jeigu jų atmetimas reiškia pasauliui sutaikinimą, tai ką gi reikštų jų priėmimas, jei ne gyvenimą iš numirusių? 
\par 16 Jei pirmieji vaisiai šventi, tai šventa ir visuma. Jei šaknis šventa, tai ir šakos. 
\par 17 Jeigu kai kurios šakos buvo nulaužtos, o tu­laukinis alyvmedis­esi tarp jų įskiepytas ir tapęs šaknies bei alyvmedžio syvų dalininku, 
\par 18 tai nesididžiuok prieš anas šakas! O jeigu didžiuojiesi, tai žinok, kad ne tu išlaikai šaknį, bet šaknis tave. 
\par 19 Gal pasakysi: “Šakos nulaužtos tam, kad aš būčiau įskiepytas?” 
\par 20 Gerai! Jos nulaužtos dėl netikėjimo, o tu stovi tikėjimu. Nesididžiuok, bet bijok! 
\par 21 Jei Dievas nepagailėjo prigimtinių šakų, gali nepagailėti ir tavęs. 
\par 22 Taigi matai Dievo gerumą ir griežtumą: nupuolusiems­griežtumas, o tau­gerumas, jei pasiliksi Jo gerume, kitaip­ir tu būsi iškirstas! 
\par 23 Bet ir anie, jei nepasiliks netikėjime, bus priskiepyti, nes Dievas turi galią ir vėl juos priskiepyti. 
\par 24 Tad jeigu buvai iškirstas iš prigimtojo laukinio alyvmedžio ir prieš prigimtį įskiepytas tauriajame alyvmedyje, tai juo labiau jie­ tikrosios šakos­bus priskiepyti savajame alyvmedyje. 
\par 25 Aš nenoriu, broliai, palikti jus nežinioje dėl šios paslapties,­kad jūs per aukštai apie save nemanytumėte: dalis Izraelio užkietėjo, kol įeis pagonių visuma, 
\par 26 o tada bus išgelbėtas visas Izraelis, kaip parašyta: “Iš Siono ateis Gelbėtojas ir nukreips bedievystes nuo Jokūbo. 
\par 27 Tokia bus jiems mano sandora, kai nuimsiu jų nuodėmes”. 
\par 28 Žiūrint Evangelijos, jie yra Dievo priešai jūsų naudai; bet pagal išrinkimą jie numylėtiniai dėl savųjų tėvų. 
\par 29 Juk Dievo dovanos ir pašaukimas­neatšaukiami. 
\par 30 Kaip jūs kadaise netikėjote Dievu, o dabar per jų netikėjimą patyrėte gailestingumą, 
\par 31 taip ir jie dabar netiki, kad dėl jums suteikto pasigailėjimo ir jie susilauktų gailestingumo. 
\par 32 Dievas juos visus uždarė netikėjime, kad visų pasigailėtų. 
\par 33 O Dievo turtų, išminties ir pažinimo gelme! Kokie neištiriami Jo teismai ir nesusekami Jo keliai! 
\par 34 “Ir kas gi pažino Viešpaties mintį? Ir kas buvo Jo patarėju?” 
\par 35 “Arba kas Jam yra davęs pirmas, kad jam būtų atmokėta?” 
\par 36 Iš Jo, per Jį ir Jam yra visa. Jam šlovė per amžius! Amen.


\chapter{12}


\par 1 gailestingumu aš prašau jus, broliai, aukoti savo kūnus kaip gyvą, šventą, Dievui patinkančią auką,­tai jūsų sąmoningas tarnavimas. 
\par 2 Ir neprisitaikykite prie šio pasaulio, bet pasikeiskite, atnaujindami savo protą, kad galėtumėte ištirti, kas yra gera, priimtina ir tobula Dievo valia. 
\par 3 Iš man suteiktos malonės raginu kiekvieną iš jūsų nemanyti apie save geriau negu dera manyti, bet manyti apie save blaiviai, pagal kiekvienam Dievo duotąjį tikėjimo saiką. 
\par 4 Juk kaip viename kūne turime daug narių, bet ne visi nariai atlieka tą patį uždavinį, 
\par 5 taip ir mūsų daugybė yra vienas kūnas Kristuje, o pavieniui­vieni kitų nariai. 
\par 6 Pagal mums suteiktą malonę turime įvairių dovanų. Jei kas turi pranašavimą, tepranašauja pagal tikėjimo saiką; 
\par 7 jei kas turi tarnavimą­tetarnauja; kas mokymą­temoko; 
\par 8 kas skatinimą­teskatina; kas duoda­tedaro tai iš atviros širdies; kas vadovauja­tevadovauja uoliai; kas daro gailestingumo darbus­tedaro tai su džiaugsmu. 
\par 9 Meilė tebūna neveidmainiška. Venkite pikto, laikykitės gero. 
\par 10 Švelniai mylėkite vienas kitą broliška meile; pagarbiai vertinkite kitus aukščiau nei save. 
\par 11 Uolumu nebūkite tingūs; būkite liepsnojančios dvasios, tarnaukite Viešpačiui. 
\par 12 Džiaukitės viltimi, būkite kantrūs išmėginimuose, nepaliaujamai melskitės. 
\par 13 Dalinkitės šventųjų poreikiais, puoselėkite svetingumą. 
\par 14 Laiminkite savo persekiotojus, laiminkite, o ne keikite. 
\par 15 Džiaukitės su besidžiaugiančiais, verkite su verkiančiais. 
\par 16 Būkite vienminčiai tarpusavyje. Negalvokite apie didelius dalykus, bet sekite nuolankiaisiais. Nebūkite išmintingi savo akyse. 
\par 17 Niekam neatmokėkite piktu už pikta, rūpinkitės tuo, kas dora visų žmonių akyse. 
\par 18 Kiek įmanoma ir kiek nuo jūsų priklauso, gyvenkite taikingai su visais žmonėmis. 
\par 19 Nekeršykite patys, mylimieji, bet palikite tai rūstybei, nes parašyta: “Mano kerštas, Aš atmokėsiu”,­sako Viešpats. 
\par 20 Todėl, jei tavo priešininkas alkanas, pavalgydink jį, jei trokšta, pagirdyk jį. Taip darydamas, tu sukrausi žarijas ant jo galvos. 
\par 21 Nesiduok pikto nugalimas, bet nugalėk pikta gerumu.


\chapter{13}


\par 1 Kiekviena siela tebūna klusni aukštesnėms valdžioms, nes nėra valdžios, kuri nebūtų iš Dievo. Esančios valdžios yra Dievo nustatytos. 
\par 2 Todėl kas priešinasi valdžiai, priešinasi Dievo tvarkai. Kurie priešinasi, užsitraukia sau teismą. 
\par 3 Nes valdininkų bijoma ne gera darant, o bloga. Nori nebijoti valdžios? Daryk gera, ir susilauksi iš jos pagyrimo. 
\par 4 Juk valdininkas yra Dievo tarnas tavo labui. Bet jei darai bloga­bijok, nes jis ne veltui nešioja kardą. Jis yra Dievo tarnas ir baudžia, įvykdydamas rūstybę darantiems pikta. 
\par 5 Todėl reikia paklusti ne tik dėl rūstybės, bet ir dėl sąžinės. 
\par 6 Juk todėl ir mokesčius mokate, nes anie yra Dievo tarnai, nuolatos užsiimantys tais dalykais. 
\par 7 Atiduokite visiems, ką privalote: kam mokestį­mokestį, kam muitą­muitą, kam baimę­baimę, kam pagarbą­pagarbą. 
\par 8 Niekam nebūkite ką nors skolingi, išskyrus meilę vienas kitam, nes kas myli, tas įvykdo įstatymą. 
\par 9 Juk įsakymai: “Nesvetimauk, nežudyk, nevok, neteisingai neliudyk, negeisk” ir kiti, yra sutraukti į šį posakį: “Mylėk savo artimą kaip save patį”. 
\par 10 Meilė nedaro blogo artimui. Todėl meilė­įstatymo išpildymas. 
\par 11 Taip elkitės, suprasdami, koks dabar laikas. Išmušė valanda mums pabusti iš miego. Dabar mūsų išgelbėjimas arčiau negu tada, kai įtikėjome. 
\par 12 Naktis nuslinko, diena prisiartino. Todėl nusimeskime tamsos darbus, apsiginkluokime šviesos ginklais! 
\par 13 Kaip dieną, elkimės dorai: nepasiduokime apsirijimui ir girtavimui, gašlavimui ir paleistuvavimui, vaidams ir pavydui, 
\par 14 bet apsirenkite Viešpačiu Jėzumi Kristumi ir netenkinkite kūno geidulių.


\chapter{14}


\par 1 Silpno tikėjimo žmogų priimkite, bet venkite ginčų dėl skirtingų nuomonių. 
\par 2 Vienas įsitikinęs, kad galima viską valgyti, o silpnas valgo tik daržoves. 
\par 3 Kuris valgo, teneniekina nevalgančio, o kuris nevalgo, teneteisia valgančio, nes Dievas jį priėmė. 
\par 4 Kas tu toks, kad drįsti teisti kito tarną?! Ar jis stovi, ar krenta­tai savajam Viešpačiui. Bet jis stovės, nes Dievas turi galią jį išlaikyti. 
\par 5 Vienas išskiria vieną dieną iš kitų dienų, o kitam jos visos vienodos. Kiekvienas tebūna įsitikinęs pagal savo išmanymą. 
\par 6 Tas, kuris išskiria dieną, daro tai Viešpačiui, ir tas, kuris neišskiria dienos, nesilaiko jos Viešpačiui. Tas, kuris valgo­valgo Viešpačiui, nes jis dėkoja Dievui, o tas, kuris nevalgo­nevalgo Viešpačiui ir dėkoja Dievui. 
\par 7 Nė vienas iš mūsų negyvena sau ir nė vienas sau nemiršta. 
\par 8 Jei mes gyvename, gyvename Viešpačiui, ir jeigu mirštame, Viešpačiui mirštame. Todėl, ar mes gyvename, ar mirštame,­ esame Viešpaties. 
\par 9 Nes dėl to Kristus ir mirė, ir prisikėlė, ir atgijo, kad būtų ir mirusiųjų, ir gyvųjų Viešpats. 
\par 10 Tai kodėl gi tu teisi savo brolį? Arba kodėl niekini savo brolį? Juk mes visi stosime prieš Kristaus teismo krasę. 
\par 11 Parašyta: “Kaip Aš gyvas,­sako Viešpats,­prieš mane suklups kiekvienas kelis, ir kiekvienos lūpos išpažins Dievą”. 
\par 12 Taigi kiekvienas iš mūsų duos Dievui apyskaitą už save. 
\par 13 Tad liaukimės teisti vieni kitus. Verčiau nuspręskime neduoti broliui akstino nupulti ar pasipiktinti. 
\par 14 Žinau ir esu įsitikinęs Viešpatyje Jėzuje, kad nieko nėra savaime netyro. Bet tam, kas mano esant netyrų dalykų, tam jie netyri. 
\par 15 Jei tavo brolis įsižeidžia dėl maisto, tu jau nebesielgi iš meilės. Savo maistu nežlugdyk to, už kurį mirė Kristus! 
\par 16 Jūsų gėris tegul nebūna akstinas piktžodžiauti. 
\par 17 Dievo karalystė nėra valgymas ir gėrimas, bet teisumas, ramybė ir džiaugsmas Šventojoje Dvasioje. 
\par 18 Kas taip Kristui tarnauja, tas priimtinas Dievui ir vertas žmonių pritarimo. 
\par 19 Tad siekime to, kas atneša ramybę ir pasitarnauja tarpusavio ugdymui. 
\par 20 Negriauk Dievo darbo dėl maisto! Nors viskas tyra, bet yra bloga žmogui, kuris valgo kitų papiktinimui. 
\par 21 Gera yra nevalgyti mėsos, negerti vyno ir vengti visko, kas tavo brolį piktina, žeidžia ar silpnina. 
\par 22 Turi tikėjimą? Turėk jį sau, prieš Dievą. Laimingas tas, kuris nesmerkia savęs už tai, ką pasirenka. 
\par 23 O kas valgo abejodamas, tas smerktinas, nes valgo ne iš įsitikinimo. Visa, kas daroma ne iš įsitikinimo, yra nuodėmė.


\chapter{15}


\par 1 Mes, stiprieji, turime pakęsti silpnųjų silpnybes ir ne sau pataikauti. 
\par 2 Kiekvienas iš mūsų tebūna malonus artimui jo labui ir pažangai. 
\par 3 Nes ir Kristus gyveno ne savo malonumui, bet, kaip parašyta: “Tave keikiančiųjų keiksmai krito ant manęs”. 
\par 4 O visa, kas anksčiau parašyta, mums pamokyti parašyta, kad ištverme ir Raštų paguoda turėtume viltį. 
\par 5 Ištvermės ir paguodos Dievas teduoda jums tarpusavyje būti vienos minties, Kristaus Jėzaus pavyzdžiu, 
\par 6 kad sutartinai vienu balsu šlovintumėte Dievą, mūsų Viešpaties Jėzaus Kristaus Tėvą. 
\par 7 Todėl priimkite vienas kitą, kaip ir Kristus jus priėmė į Dievo šlovę. 
\par 8 Aš sakau: Kristus tapo apipjaustytųjų tarnas dėl Dievo tiesos, kad patvirtintų tėvams suteiktus pažadus 
\par 9 ir kad pagonys šlovintų Dievą už Jo gailestingumą, kaip parašyta: “Todėl išpažinsiu Tave tarp pagonių, Tavo vardui giedosiu”. 
\par 10 Ir vėl sakoma: “Džiaukitės, pagonys, kartu su Jo tauta”. 
\par 11 Ir dar: “Girkite Viešpatį, visi pagonys, šlovinkite Jį visos tautos”. 
\par 12 Ir Izaijas vėl sako: “Bus Jesės šaknis, Tas, kuris pakils valdyti pagonių, ir Juo vilsis pagonys”. 
\par 13 Tegul vilties Dievas pripildo jus dideliu džiaugsmu ir tikėjimo ramybe, kad Šventosios Dvasios jėga būtumėte pertekę vilties. 
\par 14 Aš, mano broliai, esu įsitikinęs, kad jūs esate kupini gerumo, pilni visokio pažinimo ir galite vieni kitus perspėti. 
\par 15 Parašiau jums, broliai, kai kur kiek per drąsiai, norėdamas jums priminti, kad dėl Dievo man suteiktos malonės 
\par 16 turiu būti pagonims Jėzaus Kristaus tarnas ir skelbti Dievo Evangeliją, kad pagonių auka taptų priimtina, Šventosios Dvasios pašventinta. 
\par 17 Taigi Kristuje Jėzuje aš galiu pasigirti Dievo darbais. 
\par 18 Juk nedrįsčiau ko nors pasakoti, ko Kristus nebūtų per mane nuveikęs, kad pagonys paklustų žodžiu ir darbu; 
\par 19 galingais ženklais ir stebuklais, Dievo Dvasios jėga nuo Jeruzalės ir aplinkui, iki Ilyrijos, aš iki galo paskelbiau Kristaus Evangeliją. 
\par 20 Be to, kad nestatyčiau ant svetimų pamatų, aš stengiausi skelbti Evangeliją ne ten, kur Kristaus vardas jau pagarsintas, 
\par 21 bet, kaip parašyta: “Pamatys tie, kuriems nebuvo apie Jį skelbta, ir supras tie, kurie nebuvo girdėję”. 
\par 22 Todėl aš ir buvau daug kartų sutrukdytas pas jus atvykti. 
\par 23 Bet dabar, neturėdamas vietos šiuose kraštuose ir daugel metų trokšdamas atvykti pas jus, 
\par 24 aš apsilankysiu, kai keliausiu į Ispaniją. Turiu vilties, keliaudamas pro šalį, išvysti jus ir tikiuosi, kad jūs palydėsite mane, prieš tai nors kiek pasimėgavusį bendravimu su jumis. 
\par 25 Bet dabar vykstu į Jeruzalę šventiesiems patarnauti. 
\par 26 Mat Makedonijai ir Achajai buvo malonu padaryti rinkliavą Jeruzalės šventųjų beturčiams. 
\par 27 Joms tai labai patiko, ir iš tikro jos yra jų skolininkės. Jeigu pagonys tapo jų dvasinių gėrybių dalininkais, tai jų pareiga­patarnauti medžiaginėmis gėrybėmis. 
\par 28 Taigi, baigęs šį reikalą ir jiems saugiai atidavęs, kas surinkta, keliausiu pro jus į Ispaniją. 
\par 29 Ir aš esu tikras, kad atvyksiu pas jus Kristaus Evangelijos palaiminimo pilnatvėje. 
\par 30 Todėl maldauju jus, broliai, dėl Viešpaties Jėzaus Kristaus ir Dvasios meilės, kad jūs kartu su manimi kovotumėte už mane maldose Dievui, 
\par 31 kad būčiau išgelbėtas nuo netikinčiųjų Judėjoje, kad mano paslauga Jeruzalei būtų priimtina šventiesiems, 
\par 32 ir kad pagal Dievo valią atvykčiau pas jus su džiaugsmu ir atsigaivinčiau drauge su jumis. 
\par 33 Ramybės Dievas tebūnie su jumis visais! Amen.


\chapter{16}


\par 1 Pavedu jums mūsų sesę Febę, kuri yra Kenchrėjos bažnyčios tarnautoja; 
\par 2 priimkite ją Viešpatyje, kaip pridera šventiesiems, ir padėkite jai, prireikus jūsų paramos, nes ir ji yra padėjusi daug kam ir man pačiam. 
\par 3 Sveikinkite Priscilę ir Akvilą, mano bendradarbius Kristuje Jėzuje, 
\par 4 kurie guldė galvas, gelbėdami mano gyvybę. Jiems dėkoju ne aš vienas, bet ir visos pagonių bažnyčios. 
\par 5 Taip pat pasveikinkite bažnyčią, kuri yra jų namuose. Sveikinkite mano mylimąjį Epenetą­pirmąjį vaisių Kristui Azijoje. 
\par 6 Sveikinkite Mariją, kuri daug darbavosi dėl mūsų. 
\par 7 Sveikinkite Androniką ir Juniją, mano tautiečius, kalėjusius su manimi; jie žymūs tarp apaštalų ir anksčiau už mane yra Kristuje. 
\par 8 Sveikinkite mano mylimąjį Viešpatyje Ampliatą. 
\par 9 Sveikinkite mūsų bendradarbį Kristuje Urboną ir mano mylimąjį Stachį. 
\par 10 Sveikinkite Apelį, išbandytą Kristuje. Sveikinkite tuos, kurie iš Aristobulo namų. 
\par 11 Sveikinkite mano tautietį Erodioną. Sveikinkite tuos iš Narcizo šeimynos, kurie yra Viešpatyje. 
\par 12 Sveikinkite Trifeną ir Trifosą, besidarbuojančias Viešpatyje. Sveikinkite mieląją Persidę, kuri daug darbavosi Viešpatyje. 
\par 13 Sveikinkite išrinktąjį Viešpatyje Rufą, taip pat ir jo bei mano motiną. 
\par 14 Sveikinkite Asinkritą, Flegontą, Hermį, Patrobą, Hermą ir su jais esančius brolius. 
\par 15 Sveikinkite Filologą ir Juliją, Nerėją ir jo seserį, Olimpą ir visus šventuosius, kurie yra su jais. 
\par 16 Sveikinkite vieni kitus šventu pabučiavimu. Jus sveikina visos Kristaus bažnyčios. 
\par 17 Prašau jus, broliai, įsidėmėti tuos, kurie kelia susiskaldymus bei papiktinimus, prieštaraudami mokymui, kurio jūs išmokote. Venkite jų! 
\par 18 Tokie žmonės netarnauja mūsų Viešpačiui Jėzui Kristui, bet savo pilvui. Saldžiomis ir pataikūniškomis kalbomis jie apgaudinėja paprastųjų širdis. 
\par 19 Jūsų klusnumas žinomas visur. Todėl džiaugiuosi jumis, ir linkiu, kad būtumėte išmintingi gėriui ir neišmanūs blogiui. 
\par 20 O ramybės Dievas sutryps šėtoną po jūsų kojomis greitu laiku. Mūsų Viešpaties Jėzaus malonė tebūnie su jumis! Amen. 
\par 21 Jus sveikina mano bendradarbis Timotiejus ir mano tautiečiai Liucijus, Jasonas ir Sosipatras. 
\par 22 Aš, Tercijus, šio laiško surašytojas, sveikinu jus Viešpatyje. 
\par 23 Jus sveikina Gajus, mano ir visos bažnyčios šeimininkas. Jus sveikina Erastas, miesto iždininkas, ir mano brolis Kvartas. 
\par 24 Mūsų Viešpaties Jėzaus Kristaus malonė tebūnie su jumis visais. Amen. 
\par 25 Tam, kuris gali jus sustiprinti pagal mano Evangeliją ir Jėzaus Kristaus skelbimą, pagal apreiškimo paslaptį, nutylėtą nuo pasaulio pradžios, 
\par 26 bet dabar atskleistą ir pranašų raštais amžinojo Dievo įsakymu paskelbtą visoms tautoms, kad jos paklustų tikėjimui,­ 
\par 27 vienam išmintingajam Dievui per Jėzų Kristų šlovė per amžius! Amen.




\end{document}