\begin{document}

\title{Pirmasis laiškas korintiečiams}

\chapter{1}


\par 1 Paulius, Dievo valia pašauktas Kristaus Jėzaus apaštalas, ir brolis Sostenas­ 
\par 2 Dievo bažnyčiai Korinte, pašventintiems Kristuje Jėzuje, pašauktiesiems šventiesiems su visais, kurie šaukiasi mūsų Viešpaties Jėzaus Kristaus vardo kiekvienoje vietoje pas jus ir pas mus. 
\par 3 Malonė ir ramybė jums nuo Dievo, mūsų Tėvo, ir Viešpaties Jėzaus Kristaus! 
\par 4 Aš nuolat dėkoju dėl jūsų Dievui už Dievo malonę, suteiktą jums per Jėzų Kristų, 
\par 5 kad per Jį praturtėjote viskuo­ visokiu žodžiu ir visokiu pažinimu,­ 
\par 6 ir Kristaus liudijimas jumyse tapo tvirtas. 
\par 7 Todėl jums nestinga jokios dovanos laukiant, kada apsireikš mūsų Viešpats Jėzus Kristus, 
\par 8 kuris ir sutvirtins jus iki galo, kad būtumėte nekalti mūsų Viešpaties Jėzaus Kristaus dieną. 
\par 9 Ištikimas yra Dievas, kuris jus pašaukė į savo Sūnaus, mūsų Viešpaties Jėzaus Kristaus, bendravimą. 
\par 10 Broliai, Viešpaties Jėzaus Kristaus vardu maldauju jus, kad visi vienaip kalbėtumėte ir nebūtų tarp jūsų susiskaldymų, bet kad būtumėte tobulai suvienyti vienos minties ir vieno sprendimo. 
\par 11 Mat Chlojės namiškiai pranešė man apie jus, mano broliai, kad tarp jūsų esama kivirčų. 
\par 12 Turiu omenyje tai, kad iš jūsų yra tokių, kurie sako: “Aš esu Pauliaus”, “aš­Apolo”, “aš­Kefo”, “o aš­Kristaus”. 
\par 13 Argi Kristus padalytas? Argi Paulius buvo už jus nukryžiuotas? Argi vardan Pauliaus buvote pakrikštyti? 
\par 14 Ačiū Dievui, kad pas jus nieko daugiau nekrikštijau, išskyrus Krispą ir Gajų, 
\par 15 kad niekas nesakytų, jog aš krikštiju savo paties vardu. 
\par 16 Tiesa, pakrikštijau taip pat Stepono namus, o daugiau nežinau, ar ką pakrikštijau. 
\par 17 Kristus nesiuntė manęs krikštyti, bet Evangelijos skelbti,­ne žodžių išmintimi, kad Kristaus kryžius netaptų bereikšmis. 
\par 18 Mat žodis apie kryžių tiems, kurie žūsta, yra kvailystė, bet mums, išgelbėtiems, jis yra Dievo jėga. 
\par 19 Nes parašyta: “Sunaikinsiu išmintingųjų išmintį, niekais paversiu protingųjų išmanymą”. 
\par 20 Kur išminčius? Kur Rašto žinovas? Kur šio amžiaus tyrinėtojas? Argi Dievas nepavertė šio pasaulio išminties kvailyste? 
\par 21 Kadangi pasaulis išmintimi nepažino Dievo pagal Jo išmintį, tai Dievui patiko skelbimo kvailumu išgelbėti tuos, kurie tiki. 
\par 22 Žydai reikalauja ženklų, graikai ieško išminties, 
\par 23 bet mes skelbiame Kristų nukryžiuotąjį, kuris žydams yra papiktinimas, o pagonims­kvailystė. 
\par 24 Bet pašauktiesiems­tiek žydams, tiek graikams­skelbiame Kristų­Dievo jėgą ir Dievo išmintį. 
\par 25 Dievo kvailystė išmintingesnė už žmones ir Dievo silpnybė stipresnė už žmones. 
\par 26 Juk jūs matote savo pašaukimą, broliai,­nedaug tarp jūsų kūno atžvilgiu išmintingų, nedaug galingų, nedaug kilmingų. 
\par 27 Bet Dievas išsirinko, kas pasaulyje kvaila, kad sugėdintų išminčius. Dievas išsirinko, kas pasaulyje silpna, kad sugėdintų, kas galinga. 
\par 28 Ir kas pasaulyje žemos kilmės, kas paniekinta, kas yra niekas, Dievas išsirinko, kad niekais paverstų tai, kas laikoma kažkuo,­ 
\par 29 kad joks kūnas nesigirtų prieš Dievą. 
\par 30 Jo dėka jūs esate Kristuje Jėzuje, kuris mums tapo išmintimi iš Dievo, teisumu, pašventinimu ir atpirkimu, 
\par 31 kad, kaip parašyta: “Kas giriasi, tesigiria Viešpačiu”.


\chapter{2}


\par 1 Ir aš, broliai, kai pas jus lankiausi, atėjau ne su gražbyliavimu ar išmintimi skelbti jums Dievo liudijimo. 
\par 2 Nes aš nusprendžiau tarp jūsų nežinoti nieko, išskyrus Jėzų Kristų, ir Tą nukryžiuotą. 
\par 3 Aš buvau pas jus silpnas, išsigandęs ir labai drebėjau. 
\par 4 Mano kalba ir skelbimas pasižymėjo ne įtikinančiais žmogiškos išminties žodžiais, bet Dvasios ir jėgos parodymu, 
\par 5 kad ir jūsų tikėjimas remtųsi ne žmogiška išmintimi, bet Dievo jėga. 
\par 6 Tiesa, tarp subrendusiųjų mes skelbiame išmintį, tačiau tai išmintis ne šio pasaulio ir ne šio pasaulio valdovų, kurie pranyks. 
\par 7 Mes skelbiame paslaptingą ir paslėptą Dievo išmintį, kurią Dievas nuo amžių paskyrė mums išaukštinti, 
\par 8 kurios nepažino jokie šio pasaulio valdovai, nes, jei būtų pažinę, nebūtų šlovės Viešpaties nukryžiavę. 
\par 9 Bet skelbiame, kaip parašyta: “Ko akis neregėjo, ko ausis negirdėjo, kas žmogui į širdį neatėjo, tai paruošė Dievas tiems, kurie Jį myli”. 
\par 10 Dievas mums tai apreiškė per savo Dvasią, nes Dvasia visa ištiria, net Dievo gelmes. 
\par 11 Kas iš žmonių žino, kas yra žmogaus, jei ne paties žmogaus dvasia? Taip pat niekas nežino, kas yra Dievo, tik Dievo Dvasia. 
\par 12 O mes gavome ne pasaulio dvasią, bet Dvasią iš Dievo, kad suvoktume, kas mums Dievo dovanota. 
\par 13 Apie tai ir kalbame ne žodžiais, kurių moko žmogiškoji išmintis, bet tais, kurių moko Šventoji Dvasia,­dvasinius dalykus gretindami su dvasiniais. 
\par 14 Bet sielinis žmogus nepriima to, kas yra iš Dievo Dvasios, nes jam tai kvailystė; ir negali suprasti, nes tai dvasiškai vertinama. 
\par 15 O dvasinis žmogus gali spręsti apie viską, bet niekas negali spręsti apie jį. 
\par 16 “Kas gi suvokė Viešpaties mintį, kad galėtų Jį pamokyti?” O mes turime Kristaus protą.


\chapter{3}


\par 1 Aš, broliai, negalėjau kalbėti jums kaip dvasiniams, bet kaip kūniškiems, kaip kūdikiams Kristuje. 
\par 2 Maitinau jus pienu, ne tvirtu maistu, kurio jūs negalėjote priimti. Net ir dabar negalite, 
\par 3 nes tebesate kūniški. Jeigu tarp jūsų pavydas, nesantaika ir susiskaldymai,­argi nesate kūniški? Argi nesielgiate grynai žmogiškai? 
\par 4 Kol vienas sako: “Aš­Pauliaus”, kitas: “Aš­Apolo”,­argi nesate kūniški? 
\par 5 Kas yra Paulius? Kas yra Apolas? Tarnai, kurių dėka įtikėjote ir kurie tarnavo, kiek Viešpats kiekvienam skyrė. 
\par 6 Aš sodinau, Apolas laistė, o Dievas augino. 
\par 7 Todėl nieko nereiškia nei sodintojas, nei laistytojas, bet Dievas­augintojas. 
\par 8 Kas sodina ir kas laisto, yra viena, ir kiekvienas gaus savąjį užmokestį pagal savo triūsą. 
\par 9 Mes juk esame Dievo bendradarbiai, o jūs­Dievo dirva, Dievo statinys. 
\par 10 Pagal Dievo man suteiktą malonę aš, kaip išmintingas statybos vadovas, padėjau pamatą, o kitas stato ant jo. Tegul kiekvienas žiūri, kaip stato. 
\par 11 Juk niekas negali dėti kito pamato, kaip tik tą, kuris jau padėtas, kuris yra Jėzus Kristus. 
\par 12 Jei kas stato ant šio pamato iš aukso, sidabro, brangakmenių, medžio, šieno ar šiaudų,­ 
\par 13 kiekvieno darbas išaiškės. Todėl, kad diena jį atskleis, nes tai bus atskleista ugnimi ir ugnis ištirs, koks kieno darbas. 
\par 14 Jei kieno statybos darbas išliks, tas gaus užmokestį. 
\par 15 O kieno darbas sudegs, tas turės nuostolį, bet jis pats bus išgelbėtas, tačiau kaip per ugnį. 
\par 16 Ar nežinote, kad jūs esate Dievo šventykla ir Dievo Dvasia gyvena jumyse? 
\par 17 Jei kas Dievo šventyklą niokoja, tą Dievas sunaikins, nes Dievo šventykla šventa, ir toji šventykla­tai jūs! 
\par 18 Tegul niekas savęs neapgaudinėja. Jei kas tarp jūsų tariasi esąs išmintingas šiame pasaulyje, tepasidaro kvailas, kad būtų išmintingas. 
\par 19 Šio pasaulio išmintis Dievo akyse yra kvailystė, nes parašyta: “Jis sugauna protinguosius jų gudrybėje”. 
\par 20 Ir vėl: “Viešpats žino išminčių mintis, kad jos tuščios”. 
\par 21 Tad niekas tenesididžiuoja žmonėmis! Nes viskas yra jūsų: 
\par 22 ar Paulius, ar Apolas, ar Kefas, ar pasaulis, ar gyvenimas, ar mirtis, ar dabartis, ar ateitis,­viskas yra jūsų, 
\par 23 bet jūs patys­Kristaus, o Kristus­Dievo.


\chapter{4}


\par 1 Tegul kiekvienas laiko mus Kristaus tarnais ir Dievo paslapčių tvarkytojais. 
\par 2 O iš tvarkytojų reikalaujama, kad būtų ištikimi. 
\par 3 Man mažai rūpi, ką jūs ar žmonių teismas spręstų apie mane. Ir aš pats savęs neteisiu. 
\par 4 Nors nematau nieko netinkamo savyje, bet tuo dar nesu išteisintas. Mano teisėjas yra Viešpats. 
\par 5 Todėl neteiskite nieko prieš laiką, iki ateis Viešpats, kuris nušvies, kas tamsoje paslėpta, ir atskleis širdžių sumanymus. Tada kiekvienam teks pagyrimas iš Dievo. 
\par 6 Visa tai, broliai, jūsų labui pritaikiau sau ir Apolui, kad iš mūsų pasimokytumėte negalvoti daugiau negu parašyta ir kad nepasipūstumėte vienas prieš kitą. 
\par 7 Kas gi tave išskiria iš kitų? Ir ką gi turi, ko nebūtum gavęs? O jei esi gavęs, tai ko didžiuojies, lyg nebūtum gavęs? 
\par 8 Jūs jau esate sotūs, jau turtingi, jau pradėjote be mūsų karaliauti! O, kad jūs iš tikrųjų karaliautumėte, kad ir mes galėtume kartu karaliauti! 
\par 9 Man atrodo, kad Dievas mums, apaštalams, paskyrė paskutiniąją vietą, tarsi mirčiai pasmerktiems. Mes tapome reginys pasauliui, angelams ir žmonėms. 
\par 10 Mes kvaili dėl Kristaus, o jūs išmintingi Kristuje. Mes silpni, o jūs stiprūs; jūs gerbiami, o mes niekinami. 
\par 11 Iki šios valandos alkstame ir trokštame, esame nuogi ir mušami, be pastogės 
\par 12 ir vargstame, darbuodamiesi savo rankomis. Keikiami­laiminame, persekiojami­kenčiame, 
\par 13 piktžodžiaujami­maloniai atsakome. Iki šiol esame laikomi pasaulio sąšlavomis, visų atmatomis. 
\par 14 Tai rašau, ne norėdamas jus gėdinti, bet įspėdamas kaip mylimus vaikus. 
\par 15 Nors turėtumėte tūkstančius auklėtojų Kristuje, bet neturėsite daug tėvų, nes Evangelija aš pagimdžiau jus Kristuje Jėzuje. 
\par 16 Todėl raginu jus: būkite mano sekėjai! 
\par 17 Tuo tikslu ir pasiunčiau pas jus Timotiejų, kuris yra mano mylimas sūnus ir ištikimas Viešpatyje. Jis jums primins mano kelius Kristuje, kaip aš mokau visur, kiekvienoje bažnyčioje. 
\par 18 Kai kurie pasipūtė, tartum neketinčiau pas jus atvykti. 
\par 19 Jei Viešpats panorės, veikiai atvyksiu pas jus ir patikrinsiu ne pasipūtusių kalbas, bet jėgą. 
\par 20 Nes Dievo karalystė yra ne kalboje, bet jėgoje. 
\par 21 Ko norite? Ar kad ateičiau pas jus su lazda, ar su meile ir romumo dvasia?


\chapter{5}


\par 1 Tenka girdėti apie ištvirkimą tarp jūsų ir net apie tokį ištvirkimą, kokio nesigirdi net pas pagonis: būtent kažkas gyvena su savo tėvo žmona! 
\par 2 Ir jūs dar esate pasipūtę! Užuot nuliūdę ir tai padariusį išmetę ir savo tarpo?! 
\par 3 Aš, nebūdamas pas jus kūnu, tačiau būdamas dvasia, jau nuteisiau tą nusikaltimą padariusį, lyg būdamas tarp jūsų. 
\par 4 Jums susirinkus mūsų Viešpaties Jėzaus Kristaus vardu, dalyvaujant mano dvasiai, mūsų Viešpaties Jėzaus Kristaus jėga 
\par 5 atiduokite tokį šėtonui, kad sužlugdytų kūną, o dvasia būtų išgelbėta Viešpaties Jėzaus dieną. 
\par 6 Jūsų gyrimasis niekam tikęs. Argi nežinote, jog truputis raugo suraugina visą maišymą? 
\par 7 Todėl išmeskite senąjį raugą, kad taptumėte nauju maišymu. Jūs juk esate nerauginti, nes jau paskerstas mūsų Paschos Avinėlis, Kristus. 
\par 8 Tad švęskime šventes ne su senu raugu, ne su blogybės ir nusikaltimo raugu, bet su nerauginta tyrumo ir tiesos duona. 
\par 9 Jums rašiau savo laiške, kad nebendrautumėte su ištvirkėliais. 
\par 10 Suprantama, ne su visais šio pasaulio ištvirkėliais, gobšais, plėšikais ar stabmeldžiais, nes tada reikėtų pasitraukti iš šio pasaulio. 
\par 11 Bet jums rašiau, kad nebendrautumėte su tuo, kuris vadinasi brolis, o yra ištvirkėlis, gobšas, stabmeldys, keikūnas, girtuoklis ar plėšikas. Su tokiu net nevalgykite kartu. 
\par 12 Kam man teisti svetimus? Argi ir jūs ne savuosius teisiate? 
\par 13 Tuos, kurie ne mūsiškiai, Dievas teis. Todėl “pašalinkite piktadarį iš savo pačių tarpo”.


\chapter{6}


\par 1 Kaip kai kurie iš jūsų, turėdami ginčų, drįsta bylinėtis vieni su kitais pas neteisiuosius, o ne pas šventuosius? 
\par 2 Ar nežinote, kad šventieji teis pasaulį? O jeigu teisite pasaulį, tai nejaugi esate neverti išspręsti menkų bylų? 
\par 3 Ar nežinote, kad mes teisime angelus, tad juo labiau­kasdieninius dalykus? 
\par 4 Taigi, kai turite bylų kasdieniais reikalais, negi savo teisėjais paskirsite neturinčius balso bažnyčioje? 
\par 5 Tai sakau, norėdamas jus sugėdinti. Nejaugi pas jus nėra nė vieno išmintingo, kuris sugebėtų išspręsti tarp brolių iškilusią bylą? 
\par 6 Bet brolis bylinėjasi su broliu ir tai daro pas netikinčius. 
\par 7 Ir iš viso jums didelis pažeminimas, kad tarpusavyje bylinėjatės. Ar ne geriau pakęsti skriaudą? Ar ne geriau pakęsti apgaulę? 
\par 8 Deja, jūs patys skriaudžiate ir apgaudinėjate, ir dar brolius! 
\par 9 Ar nežinote, kad neteisieji nepaveldės Dievo karalystės? Neapsigaukite! Nei ištvirkėliai, nei stabmeldžiai, nei svetimautojai, nei homoseksualistai, 
\par 10 nei vagys, nei gobšai, nei girtuokliai, nei keikūnai, nei plėšikai nepaveldės Dievo karalystės. 
\par 11 Kai kurie iš jūsų buvote tokie, bet dabar esate nuplauti, pašventinti, išteisinti Viešpaties Jėzaus Kristaus vardu ir mūsų Dievo Dvasia. 
\par 12 Viskas man leistina, bet ne viskas naudinga. Viskas man leistina, bet aš nesiduosiu niekieno pavergiamas! 
\par 13 Valgis yra pilvui ir pilvas­valgiui, bet Dievas sunaikins ir vieną, ir kitą. Tačiau kūnas skirtas ne ištvirkavimui, bet Viešpačiui, o Viešpats­kūnui. 
\par 14 Kaip Dievas prikėlė Viešpatį­prikels ir mus savo jėga. 
\par 15 Argi nežinote, kad jūsų kūnai yra Kristaus nariai? Tad nejaugi aš, ėmęs Kristaus narius, paversiu juos paleistuvės nariais? Jokiu būdu! 
\par 16 Ar nežinote, kad tas, kuris susijungia su paleistuve, tampa vienas kūnas su ja? Nes “du taps,­sako Raštas,­vienu kūnu”. 
\par 17 Taip pat, kas susijungia su Viešpačiu, tampa viena dvasia su Juo. 
\par 18 Saugokitės ištvirkimo! Bet kokia kita žmogaus daroma nuodėmė nepaliečia kūno, o ištvirkėlis nusideda savo kūnui. 
\par 19 Ar nežinote, kad jūsų kūnas yra šventykla jumyse gyvenančios Šventosios Dvasios, kurią gavote iš Dievo, ir kad jūs nebepriklausote patys sau? 
\par 20 Jūs esate nupirkti už didelę kainą. Tad šlovinkite Dievą savo kūnu ir savo dvasia, kurie yra Dievo.


\chapter{7}


\par 1 Atsakau į jūsų laišką. Gerai daro vyras, neliesdamas moters. 
\par 2 Tačiau ištvirkavimui išvengti kiekvienas teturi sau žmoną ir kiekviena sau vyrą. 
\par 3 Vyras teatlieka pareigą žmonai, o žmona vyrui. 
\par 4 Žmona neturi valios savo kūnui, bet vyras. Panašiai ir vyras neturi valios savo kūnui, bet žmona. 
\par 5 Nesitraukite vienas nuo kito, nebent abiems susitarus kuriam laikui, kad atsidėtumėte pasninkui ir maldai, paskui vėl būkite drauge, kad šėtonas negundytų jūsų nesusilaikymu. 
\par 6 Tai sakau leisdamas, o ne įsakydamas. 
\par 7 Norėčiau, kad visi žmonės būtų tokie kaip aš. Bet kiekvienas turi iš Dievo savo dovaną, vienas tokią, kitas kitokią. 
\par 8 Nesusituokusiems ir našlėms sakau: jie gerai darys, pasilikdami tokie kaip aš. 
\par 9 Bet, jei negali susilaikyti, tegul tuokiasi. Geriau tuoktis negu degti. 
\par 10 Susituokusiems įsakau ne aš, bet Viešpats, kad žmona nesiskirtų nuo vyro, 
\par 11 o jei atsiskirtų, kad liktų netekėjusi arba susitaikytų su vyru;­ taip pat ir vyras tenepalieka žmonos. 
\par 12 Kitiems sakau aš, ne Viešpats: jei kuris brolis turi netikinčią žmoną ir ji sutinka gyventi su juo, tenesiskiria su ja. 
\par 13 Taip pat ir moteris, turinti netikintį vyrą, kuris sutinka su ja gyventi, tenesiskiria su juo. 
\par 14 Mat netikintis vyras pašventinamas žmona, o netikinti žmona pašventinama vyru. Kitaip jūsų vaikai būtų netyri, o dabar jie šventi. 
\par 15 Bet, jei netikintis nori skirtis, tesiskiria. Tokiais atvejais brolis ar sesuo nėra surišti, nes Dievas mus pašaukė ramybei. 
\par 16 Iš kur žinai, žmona, kad išgelbėsi vyrą? Arba iš kur tau žinoma, vyre, kad išgelbėsi žmoną? 
\par 17 Todėl, kaip Viešpats kuriam paskyrė, kokį Dievas kurį pašaukė, to jis ir toliau tesilaiko. Taip aš mokau visose bažnyčiose. 
\par 18 Jei kas buvo pašauktas apipjaustytas, tenesistengia nuslėpti apipjaustymo. Jei buvo pašauktas neapipjaustytas, tenedaro apipjaustymo. 
\par 19 Apipjaustymas yra niekas ir neapipjaustymas yra niekas, tik Dievo įsakymų laikymasis yra viskas. 
\par 20 Kiekvienas tepasilieka toks, koks buvo pašauktas. 
\par 21 Jei buvai pašauktas, būdamas vergas, nesijaudink dėl to, bet jei gali tapti laisvas, pasinaudok proga. 
\par 22 Viešpatyje pašauktas vergas yra Jo išlaisvintas. Panašiai ir pašauktas laisvasis yra Kristaus vergas. 
\par 23 Jūs esate nupirkti už didelę kainą, todėl nepasidarykite žmonių vergais. 
\par 24 Kiekvienas, broliai, kokie buvote pašaukti, tokie ir pasilikite prieš Dievą. 
\par 25 Dėl nesusituokusių neturiu Viešpaties įsakymo, bet duodu savo patarimą kaip tas, kuris iš Viešpaties gailestingumo vertas pasitikėjimo. 
\par 26 Taigi manau, jog yra gerai, atsižvelgiant į šių laikų suspaudimus­gerai žmogui būti tokiam. 
\par 27 Jei esi susietas su žmona, neieškok skyrybų. Likai be žmonos­ neieškok žmonos. 
\par 28 Jei vedi, nenusidedi, ir jei mergina išteka, nenusideda. Tačiau šitokie žmonės turės kūno vargų, o aš norėčiau jus apsaugoti nuo jų. 
\par 29 Sakau jums, broliai: laikas trumpas! Belieka tiems, kurie turi žmonas, gyventi, tarsi jų neturėtų, 
\par 30 ir kurie verkia, tarsi neverktų, ir kurie džiaugiasi, tarsi nesidžiaugtų, ir kurie perka, tarsi neįsigytų, 
\par 31 ir kurie naudojasi šiuo pasauliu,­tarsi nesinaudotų. Nes šio pasaulio pavidalas praeina. 
\par 32 Norėčiau, kad jūs gyventumėte be rūpesčių. Nesusituokęs rūpinasi Viešpaties reikalais­stengiasi patikti Viešpačiui. 
\par 33 O susituokęs rūpinasi pasaulio reikalais­stengiasi patikti žmonai. 
\par 34 Yra skirtumas tarp žmonos ir mergaitės. Netekėjusi moteris rūpinasi Viešpaties reikalais, kad būtų šventa kūnu ir dvasia, o ištekėjusi rūpinasi pasaulio reikalais­kaip patikti vyrui. 
\par 35 Tai kalbu jūsų pačių labui, ne norėdamas užnerti jums kilpą, bet dėl to, kad tai yra tinkama ir kad jūs neblaškomi galėtumėte atsiduoti Viešpačiui. 
\par 36 Jei kas mano būsiant jam negarbinga, kad jo mergaitė liks senmergė, ir jei taip reikia, tedaro kaip nori,­nenusidės,­tegul jie susituokia. 
\par 37 Bet jei kas savo širdyje yra tvirtai apsisprendęs, ne iš prievartos, bet būdamas savo valios šeimininkas, ir savo širdyje nutaręs išsaugoti savo mergaitę, tas gerai padarys. 
\par 38 Taigi, kas išleidžia savo mergaitę, gerai daro, o kas neišleidžia savo mergaitės, geriau daro. 
\par 39 Žmona surišta įstatymu, kol jos vyras gyvas. Vyrui mirus, ji laisva ir gali tekėti už ko nori, tik Viešpatyje. 
\par 40 Bet, mano nuomone, ji bus laimingesnė netekėdama. Manau, kad ir aš turiu Dievo Dvasią.


\chapter{8}


\par 1 Dėl aukų stabams mums aišku: mes visi turime pažinimą. Pažinimas išpučia, bet meilė ugdo. 
\par 2 Jei kas mano ką nors žinąs, tai jis dar nieko nežino, kaip turi žinoti. 
\par 3 Bet kas myli Dievą, tas yra Jo pažintas. 
\par 4 Taigi dėl stabams paaukotų dalykų valgymo mes žinome, kad stabas pasaulyje yra niekas ir kad nėra jokių kitų dievų, kaip tik vienas Dievas. 
\par 5 Ir nors yra vadinamųjų dievų danguje ar žemėje,­daug tų dievų ir daug viešpačių,­ 
\par 6 tai mes turime tik vieną Dievą, Tėvą, iš kurio yra visa ir Jam esame mes, ir vieną Viešpatį, Jėzų Kristų, per kurį yra visa ir mes per Jį. 
\par 7 Bet ne visi turi tokį pažinimą. Kai kurie su sąžine, pripažįstančia stabus, iki šiol valgo maistą, kaip stabams paaukotą, ir jų silpna sąžinė susitepa. 
\par 8 Maistas nepriartina mūsų prie Dievo. Kai valgome, nieko nelaimime, ir kai nevalgome, nieko neprarandame. 
\par 9 Bet žiūrėkite, kad ši jūsų laisvė netaptų papiktinimu silpniesiems. 
\par 10 Antai, jei kas pamatytų tave, turintį pažinimą ir valgantį stabų šventykloje, argi silpno žmogaus sąžinė nebus paskatinta valgyti stabams paaukoto maisto? 
\par 11 Ar dėl tavo pažinimo nežus silpnas brolis, už kurį mirė Kristus? 
\par 12 Šitaip nusidėdami broliams ir sužeisdami jų silpnas sąžines, nusidedate Kristui. 
\par 13 Todėl jei valgis piktina mano brolį, aš nevalgysiu mėsos per amžius, kad nepapiktinčiau savo brolio.


\chapter{9}

\par 1 Ar aš ne apaštalas? Ar aš ne laisvas? Ar nesu regėjęs Jėzaus Kristaus, mūsų Viešpaties? Ar jūs ne mano darbo vaisius Viešpatyje? 
\par 2 Jei kitiems ir nesu apaštalas, tai jums, be abejo, esu, nes mano apaštalystės antspaudas esate jūs Viešpatyje. 
\par 3 Štai mano pasiteisinimas prieš tuos, kurie mane kaltina. 
\par 4 Argi mes neturime teisės valgyti ir gerti? 
\par 5 Ar neturime teisės vaikščioti kartu su seserimi, žmona, kaip kiti apaštalai ir Viešpaties broliai bei Kefas? 
\par 6 O gal tik aš ir Barnabas neturime teisės nedirbti? 
\par 7 Kas gi tarnauja kariuomenėje už savo paties pinigus? Kas sodina vynuogyną ir nevalgo jo vaisių? Arba kas gano bandą ir negeria bandos pieno? 
\par 8 Ar aš tai sakau vien kaip žmogus? Argi to paties nesako ir Įstatymas? 
\par 9 Mozės Įstatyme parašyta: “Neužrišk nasrų kuliančiam jaučiui!” Bet ar Dievui terūpi jaučiai? 
\par 10 O gal Jis iš tiesų taiko tai mums? Juk dėl mūsų parašyta, kad artojas artų su viltimi ir kūlėjas kultų su viltimi, jog gaus savo dalį. 
\par 11 Jei mes jums pasėjome dvasinių gėrybių, tai ar didelis dalykas, kad pjausime pas jus medžiaginių? 
\par 12 Jei kiti turi teisių į jus, tai ar ne juo labiau mes? Vis dėlto mes nesinaudojome savo teise, bet viską pakeliame, kad tik Kristaus Evangelijai nedarytume kliūčių. 
\par 13 Ar nežinote, kad tie, kurie tarnauja šventykloje, valgo iš šventyklos pajamų ir kad aukuro tarnai gauna dalį aukų? 
\par 14 Taip ir Viešpats patvarkė, kad Evangelijos skelbėjai gyventų iš Evangelijos. 
\par 15 Bet aš nesinaudojau šiomis teisėmis. Ir tai rašau ne tam, kad sau pritaikyčiau. Man geriau mirti, negu leisti, kad kas atimtų iš manęs šį pasigyrimą. 
\par 16 Jeigu skelbiu Evangeliją, neturiu pagrindo girtis, nes tai mano būtina pareiga, ir vargas man, jei Evangelijos neskelbčiau! 
\par 17 Jeigu tai darau savo valia, turiu atlygį; bet jei darau ne savo valia, tai atlieku man patikėtą tarnavimą. 
\par 18 Koks tada mano atlygis? Ogi kad, skelbdamas Kristaus Evangeliją, pateikiu ją veltui ir nesinaudoju savo teise, kurią man duoda Evangelijos skelbimas. 
\par 19 Būdamas nuo nieko nepriklausomas, pasidariau visų vergas, kad tik daugiau jų laimėčiau. 
\par 20 Žydams buvau kaip žydas, kad laimėčiau žydus. Tiems, kurie laikosi įstatymo, tapau besilaikančiu įstatymo, kad laimėčiau besilaikančius įstatymo, nors pats nesu jam pavaldus. 
\par 21 Tiems, kurie neturi įstatymo, buvau kaip neturintis įstatymo,­ pats būdamas ne be Dievo įstatymo, bet surištas Kristaus įstatymu,­kad laimėčiau tuos, kurie neturi įstatymo. 
\par 22 Silpniesiems pasidariau kaip silpnas, kad laimėčiau silpnuosius. Visiems tapau viskuo, kad vienaip ar kitaip kai kuriuos išgelbėčiau. 
\par 23 Visa tai darau dėl Evangelijos, kad būčiau jos dalininkas. 
\par 24 Argi nežinote, kad lenktynėse bėga visi, bet tik vienas gauna laimėtojo apdovanojimą? Taigi bėkite taip, kad laimėtumėte! 
\par 25 Kiekvienas varžybų dalyvis nuo visko susilaiko; jie taip daro, norėdami gauti vystantį vainiką, o mes­nevystantį. 
\par 26 Todėl aš bėgu nedvejodamas ir grumiuosi ne kaip į orą smūgiuodamas, 
\par 27 bet tramdau savo kūną ir darau jį klusnų, kad, kitiems skelbdamas, pats netapčiau atmestinas.


\chapter{10}


\par 1 Aš nenoriu, kad liktumėte nežinioje, broliai,­visi mūsų tėvai buvo po debesim ir visi perėjo jūrą. 
\par 2 Ir visi buvo pakrikštyti į Mozę, debesyje ir jūroje; 
\par 3 visi valgė tą patį dvasinį maistą 
\par 4 ir visi gėrė tą patį dvasinį gėrimą. Jie gėrė iš dvasinės juos lydinčios uolos, o ta uola buvo Kristus. 
\par 5 Vis dėlto daugumas iš jų nepatiko Dievui, ir “jų kūnai liko gulėti dykumoje”. 
\par 6 Tie įvykiai yra mums pavyzdžiai, kad negeistume blogio, kaip anie geidė. 
\par 7 Nebūkite stabmeldžiai, kaip kai kurie iš jų,­kaip parašyta: “Tauta sėdosi valgyti ir gerti ir kėlėsi žaisti”. 
\par 8 Neištvirkaukime, kaip kai kurie iš jų ištvirkavo ir žuvo vieną dieną dvidešimt trys tūkstančiai. 
\par 9 Negundykime Kristaus, kaip kai kurie iš jų gundė ir mirė nuo gyvačių. 
\par 10 Nemurmėkite, kaip kai kurie iš jų murmėjo ir žuvo nuo naikintojo. 
\par 11 Visa tai jiems atsitiko kaip pavyzdžiai, ir užrašyti pamokyti mums, gyvenantiems amžių pabaigoje. 
\par 12 Todėl, kas tariasi stovįs, težiūri, kad nekristų. 
\par 13 Jums tekęs pagundymas tėra tik žmogiškas. Bet Dievas ištikimas. Jis neleis jūsų gundyti daugiau nei jūsų jėgos leidžia, bet kartu su pagundymu duos ir išeitį, kad sugebėtumėte jį atlaikyti. 
\par 14 Todėl, mano mylimieji, bėkite nuo stabmeldystės! 
\par 15 Kalbu kaip išmintingiems; apsvarstykite patys, ką sakau. 
\par 16 Argi laiminimo taurė, kurią laiminame, nėra bendravimas Kristaus kraujyje? Argi duona, kurią laužome, nėra bendravimas Kristaus kūne? 
\par 17 Nors mūsų daug, mes visi esame viena duona ir vienas kūnas: juk mes visi dalijamės viena duona. 
\par 18 Pažiūrėkite į Izraelį pagal kūną: argi tie, kurie valgo aukas, nėra aukuro bendrininkai? 
\par 19 Ką gi sakau? Ar stabas ką nors reiškia? Ar ką nors reiškia auka stabams? 
\par 20 Ne, bet pagonys, aukodami aukas, aukoja demonams, o ne Dievui. Ir aš nenoriu, kad jūs būtumėte demonų bendrininkai. 
\par 21 Jūs negalite gerti Viešpaties taurės ir demonų taurės. Negalite sėdėti prie Viešpaties stalo ir prie demonų stalo. 
\par 22 Nejaugi kurstysime Viešpaties pavydą? Ar mes už Jį stipresni? 
\par 23 Viskas man leistina, bet ne viskas naudinga. Viskas man leistina, bet ne viskas ugdo! 
\par 24 Niekas teneieško, kaip jam geriau, bet kaip kitam. 
\par 25 Valgykite visa, kas parduodama mėsos prekyvietėje, sąžinės labui nieko neklausinėdami. 
\par 26 Juk “Viešpaties yra žemė ir visa, ko ji pilna”. 
\par 27 Jeigu jus pasikviečia netikintis žmogus ir jūs norite jį aplankyti, valgykite visa, kas jums padedama, sąžinės labui nieko neklausinėdami. 
\par 28 Bet jei kas jums pasakytų: “Tai auka stabams”, tada nevalgykite dėl žmogaus, kuris tai pasakė, ir dėl sąžinės,­juk “Viešpaties yra žemė ir visa, ko ji pilna”. 
\par 29 Čia aš kalbu ne apie tavo paties sąžinę, bet apie kito sąžinę. Kodėl gi mano laisvė turėtų būti teisiama kitos sąžinės? 
\par 30 Jeigu aš dėkodamas valgau, kodėl man priekaištaujama dėl maisto, už kurį dėkoju? 
\par 31 Todėl ar valgote, ar geriate, ar šiaip ką darote, visa darykite Dievo šlovei. 
\par 32 Nepiktinkite nei žydų, nei graikų, nei Dievo bažnyčios, 
\par 33 kaip ir aš stengiuosi visiems viskuo patikti, neieškodamas sau naudos, bet to, kas naudinga daugeliui, kad jie būtų išgelbėti.


\chapter{11}

\par 1 Sekite manimi, kaip ir aš seku Kristumi. 
\par 2 Aš jus giriu, kad visame kame prisimenate mane ir laikotės nurodymų, kuriuos jums daviau. 
\par 3 Noriu, kad žinotumėte, jog kiekvieno vyro galva yra Kristus, moters galva­vyras, o Kristaus galva­Dievas. 
\par 4 Kiekvienas vyras, kuris meldžiasi ar pranašauja apdengta galva, paniekina savo galvą. 
\par 5 Ir kiekviena moteris, kuri meldžiasi ar pranašauja neapdengta galva, paniekina savo galvą: tai tas pats, kaip būti nuskustai. 
\par 6 Jei moteris neapsigaubia, tai tegul ir nusikerpa! O jei moteriai gėda nusikirpti ar nusiskusti plaukus, teapsigaubia. 
\par 7 Vyrui nereikia gaubti galvos, nes jis yra Dievo atvaizdas ir šlovė. O moteris yra vyro šlovė. 
\par 8 Juk ne vyras iš moters, bet moteris iš vyro, 
\par 9 taip pat ne vyras buvo sukurtas moteriai, o moteris vyrui. 
\par 10 Todėl moteris privalo turėti ant galvos pavaldumo ženklą dėl angelų. 
\par 11 Bet Viešpatyje nei vyras be moters, nei moteris be vyro. 
\par 12 Kaip moteris iš vyro, taip vyras per moterį, bet visa­iš Dievo. 
\par 13 Spręskite patys: argi tinka moteriai melstis Dievui neapsigaubusiai? 
\par 14 Argi pati prigimtis jūsų nemoko, jog vyrui gėda nešioti ilgus plaukus? 
\par 15 Tuo tarpu moteriai garbė turėti ilgus plaukus. Nes plaukai jai duoti kaip dangalas. 
\par 16 Bet jei kas norėtų ginčytis,­težino, jog nei mes, nei Dievo bažnyčios neturime tokio papročio. 
\par 17 Duodamas šiuos nurodymus, aš jūsų negiriu, nes jūsų susirinkimai išeina ne į gera, bet į bloga. 
\par 18 Pirmiausia man teko girdėti, kad, kai jūs susirenkate bažnyčioje, tarp jūsų būna susiskaldymų, ir iš dalies aš tuo tikiu. 
\par 19 Juk pas jus turi būti atskalų, kad išaiškėtų tie, kurie yra patikimi. 
\par 20 Jūs susirenkate kartu, bet ne Viešpaties vakarienės valgyti, 
\par 21 nes kiekvienas paskuba suvalgyti savo maistą, ir vienas lieka alkanas, o kitas nusigeria. 
\par 22 Argi neturite savo namų valgyti ir gerti? O gal norite paniekinti Dievo bažnyčią ir sugėdinti stokojančius? Kas man belieka sakyti? Pagirti? Ne! Už tai nepagirsiu. 
\par 23 Aš tai gavau iš Viešpaties ir perdaviau jums, kad Viešpats Jėzus tą naktį, kurią buvo išduotas, paėmė duoną 
\par 24 ir padėkojęs sulaužė ir tarė: “Imkite ir valgykite; tai yra mano kūnas, kuris už jus sulaužomas. Tai darykite mano atminimui”. 
\par 25 Taip pat po vakarienės jis paėmė taurę ir tarė: “Ši taurė yra Naujoji Sandora mano kraujyje. Kiek kartų gersite, darykite tai mano atminimui”. 
\par 26 Taigi, kada tik valgote šitą duoną ir geriate šitą taurę, jūs skelbiate Viešpaties mirtį, kol Jis ateis. 
\par 27 Todėl kas nevertai valgo tos duonos ir geria iš Viešpaties taurės, tas bus kaltas prieš Viešpaties kūną ir kraują. 
\par 28 Teištiria žmogus pats save ir tada tevalgo tos duonos ir tegeria iš tos taurės. 
\par 29 Nes kas valgo ir geria nevertai, Viešpaties kūno neišskirdamas, tas valgo ir geria sau pasmerkimą. 
\par 30 Todėl tarp jūsų daug silpnų bei ligotų ir daug užmigusių. 
\par 31 Jei mes patys save teistume, nebūtume teisiami. 
\par 32 Bet kai Viešpats mus teisia, tai ir sudraudžia, kad nebūtume pasmerkti kartu su pasauliu. 
\par 33 Todėl, mano broliai, kai susirenkate valgyti, palaukite vieni kitų. 
\par 34 O jeigu kas išalkęs, tepavalgo namie, kad nesirinktumėte pasmerkimui. Kitus reikalus sutvarkysiu atvykęs.


\chapter{12}


\par 1 Aš nenoriu, broliai, kad jūs neišmanytumėte apie dvasines dovanas. 
\par 2 Jūs žinote, kad, kai buvote pagonys, ėjote prie nebylių stabų kaip vedami. 
\par 3 Todėl aš jums aiškinu, kad nė vienas, kuris kalba Dievo Dvasia, neprakeikia Jėzaus. Ir nė vienas negali sakyti, kad Jėzus yra Viešpats, kaip tik Šventąja Dvasia. 
\par 4 Yra skirtingų dovanų, tačiau ta pati Dvasia. 
\par 5 Yra skirtingų tarnavimų, tačiau tas pats Viešpats. 
\par 6 Ir yra skirtingi veiksmai, tačiau tas pats Dievas, kuris visa veikia visame kame. 
\par 7 Bet kiekvienam suteikiamas Dvasios pasireiškimas bendram labui. 
\par 8 Vienam Dvasia suteikiamas išminties žodis, kitam ta pačia Dvasia­pažinimo žodis, 
\par 9 kitam­tikėjimas ta pačia Dvasia, kitam­išgydymų dovanos ta pačia Dvasia, 
\par 10 kitam­stebuklų darymas, kitam­pranašavimas, kitam­dvasių atpažinimas, kitam­skirtingos kalbos, kitam­kalbų aiškinimas. 
\par 11 Bet visa tai daro viena ir ta pati Dvasia, kuri dalija kiekvienam atskirai, kaip Jai patinka. 
\par 12 Nes kaip kūnas yra vienas ir turi daug narių, o visi to vieno kūno nariai, nepaisant daugumo, sudaro vieną kūną, taip ir Kristus. 
\par 13 Nes viena Dvasia mes visi esame pakrikštyti į vieną kūną,­žydai ar graikai, vergai ar laisvieji; ir visi buvome pagirdyti viena Dvasia. 
\par 14 Juk kūnas nėra sudėtas iš vieno nario, bet iš daugelio. 
\par 15 Jei koja sakytų: “Kadangi nesu ranka, todėl nepriklausau kūnui”, argi dėl to ji nepriklausytų kūnui? 
\par 16 O jeigu ausis sakytų: “Kadangi nesu akis, todėl nepriklausau kūnui”, argi dėl to ji nepriklausytų kūnui? 
\par 17 Jeigu visas kūnas būtų akis, tai kur būtų klausa? Jeigu visas kūnas būtų klausa, tai kur būtų uoslė? 
\par 18 Bet dabar Dievas sudėliojo kūne narius ir kiekvieną iš jų, kaip panorėjo. 
\par 19 Ir jei visi būtų vienas narys, kur beliktų kūnas? 
\par 20 Bet dabar narių daug, o kūnas vienas. 
\par 21 Akis negali pasakyti rankai: “Man tavęs nereikia”, ar galva kojoms: “Man jūsų nereikia”. 
\par 22 Priešingai, tie kūno nariai, kurie atrodo silpnesni, yra būtini. 
\par 23 Tuos kūno narius, kuriuos laikome mažiau garbingais, mes apsupame didesne pagarba, ir mūsų gėdingi nariai gaubiami didesnio padorumo, 
\par 24 kurio nereikia mūsų padoriesiems nariams. Taigi, tvarkydamas kūną, Dievas skyrė daugiau pagarbos tiems kūno nariams, kurie jos stokojo, 
\par 25 kad kūne nebūtų susiskaldymų, bet patys nariai rūpintųsi vieni kitais. 
\par 26 Todėl, jei kenčia vienas narys, su juo kenčia ir visi nariai. Ir jei kuris narys pagerbiamas, su juo džiaugiasi visi nariai. 
\par 27 Jūs esate Kristaus kūnas, o pavieniui­nariai. 
\par 28 Ir šiuos Dievas paskyrė bažnyčioje: pirma­apaštalais, antra­pranašais, trečia­mokytojais; po to­ stebuklai, paskui­išgydymų dovanos, visokia pagalba, vadovavimai, įvairios kalbos. 
\par 29 Ar visi apaštalai? Ar visi pranašai? Ar visi mokytojai? Ar visi stebukladariai? 
\par 30 Ar visi turi išgydymų dovanas? Ar visi kalba kalbomis? Ar visi aiškina? 
\par 31 Taigi karštai trokškite aukštesniųjų dovanų! Ir visgi rodau jums dar pranašesnį kelią.


\chapter{13}


\par 1 Jeigu aš kalbu žmonių ir angelų kalbomis, bet neturiu meilės, esu kaip skambantis varis ar žvangantys cimbolai. 
\par 2 Ir jei turiu pranašavimo dovaną ir suprantu visas paslaptis, ir turiu visą pažinimą; jei turiu visą tikėjimą, kad galiu kalnus perkelti, tačiau neturiu meilės, esu niekas. 
\par 3 Ir jei išdalinu vargšams pamaitinti visa, ką turiu, ir jeigu atiduodu savo kūną sudeginti, bet neturiu meilės,­man nėra iš to jokios naudos. 
\par 4 Meilė kantri ir maloni, meilė nepavydi; meilė nesigiria ir neišpuiksta. 
\par 5 Ji nesielgia nepadoriai, neieško savo naudos, nepasiduoda piktumui, nemąsto piktai, 
\par 6 nesidžiaugia neteisybe, džiaugiasi tiesa; 
\par 7 visa pakenčia, visa tiki, viskuo viliasi ir visa ištveria. 
\par 8 Meilė niekada nesibaigia. Baigsis pranašystės, paliaus kalbos, išnyks pažinimas, 
\par 9 nes mes žinome iš dalies ir mes pranašaujame iš dalies. 
\par 10 Bet kai ateis tobulumas, tai, kas iš dalies, pasibaigs. 
\par 11 Kai buvau vaikas, kalbėjau kaip vaikas, supratau kaip vaikas, mąsčiau kaip vaikas, bet tapęs vyru, palikau tai, kas vaikiška. 
\par 12 Dabar mes matome kaip per stiklą, miglotai, bet tada­veidas į veidą. Dabar žinau iš dalies, bet tada pažinsiu, kaip ir pats esu pažintas. 
\par 13 Taigi dabar pasilieka tikėjimas, viltis ir meilė­šis trejetas, bet didžiausia iš jų yra meilė.


\chapter{14}


\par 1 Siekite meilės ir trokškite dvasinių dovanų, ypač, kad pranašautumėte. 
\par 2 Kas kalba kalbomis, ne žmonėms kalba, bet Dievui; niekas jo nesupranta, nes jis dvasioje kalba paslaptis. 
\par 3 Bet kas pranašauja, tas kalba žmonių ugdymui, paraginimui ir paguodai. 
\par 4 Kas kalba kalbomis, pats save ugdo, o kas pranašauja­ugdo bažnyčią. 
\par 5 Aš norėčiau, kad jūs visi kalbėtumėte kalbomis, bet būtų geriau, jei pranašautumėte, nes kas pranašauja, yra didesnis už tą, kuris kalba kalbomis, nebent pastarasis ir aiškintų, kad būtų ugdoma bažnyčia. 
\par 6 Bet dabar, broliai, jei ateičiau pas jus, kalbėdamas kalbomis, kokia jums būtų nauda, jeigu neskelbčiau jums apreiškimo, pažinimo, pranašystės ar mokymo? 
\par 7 Taip pat ir negyvi daiktai, skleidžiantys garsus, ar tai būtų fleita, ar kanklės, jei neduotų skirtingų garsų, iš ko pažintume, kas grojama ar skambinama? 
\par 8 Ir jeigu trimitas duotų neaiškų garsą, kas ruoštųsi į mūšį? 
\par 9 Tas pat ir su jumis. Jei kalbėsite nesuprantamus žodžius, kaip bus galima suprasti, ką sakote? Jūs kalbėsite vėjams! 
\par 10 Kas žino, kiek daug yra įvairių kalbų pasaulyje, bet nė viena iš jų nėra bereikšmė. 
\par 11 Todėl jei nesuprantu kalbos prasmės, būsiu kalbėtojui svetimšalis, ir kalbėtojas bus man svetimšalis. 
\par 12 Taigi ir jūs, karštai trokštantys dvasinių dovanų, siekite jų bažnyčios ugdymui, kad gausiai jų turėtumėte. 
\par 13 Todėl, kas kalba kalbomis, tesimeldžia, kad galėtų aiškinti. 
\par 14 Nes jei meldžiuosi kalbomis, meldžiasi mano dvasia, bet protas lieka bevaisis. 
\par 15 Ką gi tada daryti? Melsiuosi dvasia ir melsiuosi protu; giedosiu dvasia ir giedosiu protu. 
\par 16 Be to, jei tu laimini dvasia, kaip neišmanantis pasakys tavo padėkai “amen”, nesuprasdamas, ką tu kalbi? 
\par 17 Juk tu gražiai dėkoji, tačiau kitas nėra ugdomas. 
\par 18 Dėkui mano Dievui, aš kalbu kalbomis daugiau už jus visus, 
\par 19 vis dėlto bažnyčioje geriau pasakysiu penkis žodžius savo protu, kad pamokyčiau ir kitus, negu tūkstančius žodžių kalbomis. 
\par 20 Broliai! Nebūkite vaikai išmanymu. Verčiau blogybe būkite kūdikiai, bet išmanymu­subrendę. 
\par 21 Įstatyme parašyta: “Svetimomis kalbomis ir svetimųjų lūpomis Aš kalbėsiu šiai tautai, bet ir tada jie nepaklausys manęs”,­sako Viešpats. 
\par 22 Todėl kalbos yra ženklas ne tikintiems, bet netikintiems. O pranašavimas­ne netikintiems, bet tikintiems. 
\par 23 Juk jei susirinktų visa bažnyčia ir visi imtų kalbėti kalbomis, ir įeitų neišmanantys ar netikintys,­argi jie nesakytų, kad jūs išėję iš proto? 
\par 24 Bet jeigu visi pranašautų ir įeitų netikintis ar neišmanantis, jis būtų visų apkaltintas ir visų atpažintas. 
\par 25 Jo širdies paslaptys būtų atskleistos, ir jis, puolęs veidu žemėn, pagarbintų Dievą ir išpažintų: “Dievas iš tiesų yra tarp jūsų!” 
\par 26 Tad kaip bus, broliai? Kai susirenkate, kiekvienas turi giesmę ar pamokymą, ar kalbą, ar apreiškimą, ar aiškinimą. Tegul viskas tarnauja ugdymui. 
\par 27 Jei kalba kas kalbomis, tekalba du, daugiausia trys, paeiliui, o vienas tegul aiškina. 
\par 28 Bet jeigu nėra aiškintojo, tegul tas bažnyčioje tyli ir tekalba tik sau ir Dievui. 
\par 29 Ir pranašai tekalba du ar trys, o kiti teapsvarsto. 
\par 30 Bet jei kitam šalia sėdinčiam kas nors apreiškiama, pirmasis tenutyla. 
\par 31 Nes jūs visi galite vienas po kito pranašauti, kad visi pasimokytų ir visi būtų paguosti. 
\par 32 Pranašų dvasios yra paklusnios pranašams, 
\par 33 nes Dievas nėra sumaišties, bet ramybės Dievas,­kaip ir visose šventųjų bažnyčiose. 
\par 34 Jūsų moterys bažnyčiose tetyli, nes joms neleidžiama kalbėti, jos turi būti klusnios, kaip sako ir įstatymas. 
\par 35 Ir jeigu jos nori ko nors išmokti, tepasiklausia namie savo vyro, nes moterims gėdinga bažnyčioje kalbėti. 
\par 36 Argi iš jūsų išėjo Dievo žodis? Ar tik jus vienus pasiekė? 
\par 37 Jei kas mano esąs pranašas ar dvasinis žmogus, tegu pripažįsta, kad tai, ką jums rašau, yra Viešpaties įsakymai. 
\par 38 Bet jei kas neišmano, tegul neišmano. 
\par 39 Todėl, broliai, karštai trokškite pranašauti ir nedrauskite kalbėti kalbomis. 
\par 40 Tebūnie viskas daroma padoriai ir tvarkingai.


\chapter{15}


\par 1 Broliai, aiškinu jums Evangeliją, kurią jums paskelbiau, kurią jūs ir priėmėte ir kurioje stovite, 
\par 2 ir kuria esate išgelbėti,­jeigu jūs laikotės to žodžio, kurį jums paskelbiau; kitaip jūs įtikėjote veltui. 
\par 3 Pirmiausia jums perdaviau tai, ką pats gavau: kad Kristus numirė už mūsų nuodėmes pagal Raštus; 
\par 4 ir kad Jis buvo palaidotas, ir kad prisikėlė trečią dieną pagal Raštus; 
\par 5 ir kad Jis pasirodė Kefui, po to dvylikai. 
\par 6 Po to Jis pasirodė iš karto daugiau nei penkiems šimtams brolių, kurių daugumas tebegyvena iki šiol, o kai kurie yra užmigę. 
\par 7 Po to Jis pasirodė Jokūbui, paskui visiems apaštalams. 
\par 8 O visų paskiausiai, lyg ne laiku gimusiam, Jis pasirodė ir man. 
\par 9 Juk aš esu mažiausias iš apaštalų, nevertas vadintis apaštalu, nes persekiojau Dievo bažnyčią. 
\par 10 Bet Dievo malone esu, kas esu, ir Jo malonė man neliko bergždžia, bet aš darbavausi daug daugiau už juos visus, nors ne aš, bet Dievo malonė, esanti su manimi. 
\par 11 Taigi ar aš, ar jie,­taip mes skelbiame, ir taip jūs įtikėjote. 
\par 12 Jeigu apie Kristų skelbiama, kad Jis buvo prikeltas iš numirusių, tad kaip kai kurie iš jūsų sako, kad nėra mirusiųjų prisikėlimo?! 
\par 13 Jeigu nėra mirusiųjų prisikėlimo, tai Kristus nebuvo prikeltas. 
\par 14 O jei Kristus nebuvo prikeltas, tai tuščias mūsų skelbimas ir tuščias jūsų tikėjimas. 
\par 15 Ir mes tada liekame melagingi Dievo liudytojai, nes liudijome apie Dievą, kad Jis prikėlė Kristų, kurio Jis nėra prikėlęs, jeigu mirusieji neprisikelia. 
\par 16 Nes jei mirusieji neprisikelia, tada ir Kristus nebuvo prikeltas. 
\par 17 O jei Kristus nebuvo prikeltas, tai jūsų tikėjimas tuščias; jūs dar tebesate savo nuodėmėse. 
\par 18 Tuomet ir užmigusieji Kristuje yra žuvę. 
\par 19 Ir jei vien tik šiame gyvenime mes viliamės Kristumi, tai mes esame labiausiai apgailėtini iš visų žmonių. 
\par 20 Bet dabar Kristus yra prikeltas iš numirusių­pirmasis iš užmigusiųjų. 
\par 21 Kaip per žmogų­mirtis, taip per žmogų ir mirusiųjų prisikėlimas. 
\par 22 Kaip Adome visi miršta, taip Kristuje visi bus atgaivinti, 
\par 23 tačiau kiekvienas pagal savo eilę: Kristus­pirmasis, vėliau­priklausantys Kristui Jo atėjimo metu. 
\par 24 Po to bus galas, kai Jis perduos karalystę Dievui Tėvui, sunaikinęs visas kunigaikštystes, visas valdžias ir jėgas. 
\par 25 Nes Jis turi valdyti, kol paguldys visus priešus po savo kojomis. 
\par 26 Kaip paskutinis priešas bus sunaikinta mirtis. 
\par 27 Nes “Jis visa paklojo Jam po kojų”. Bet kai Jis sako, kad visa paklota, tai savaime suprantama, kad išskyrus Tą, kuris Jam visa paklojo. 
\par 28 Kai Jam bus visa pajungta, tada ir pats Sūnus nusilenks Tam, kuris viską Jam pajungė, kad Dievas būtų viskas visame kame. 
\par 29 Kita vertus, ką darys tie, kuriuos krikštija už mirusiuosius? Jei iš viso mirusieji neprisikels, tai kam gi jie krikštijami už mirusiuosius? 
\par 30 Ir kodėl ir mes kas valandą esame pavojuje? 
\par 31 Prisiekiu savo pasididžiavimu­ jumis, broliai, mūsų Viešpatyje Jėzuje Kristuje, jog aš kasdien mirštu! 
\par 32 Jei grynai kaip žmogus kovojau Efeze su laukiniais žvėrimis, tai kokia man iš to nauda? Jeigu mirusieji neprisikels, tai “valgykime ir gerkime, nes rytoj mirsime”. 
\par 33 Neapsirikite: “Blogos draugijos gadina gerus papročius!” 
\par 34 Pabuskite teisumui ir nenuodėmiaukite, nes kai kurie iš jūsų nepažįsta Dievo. Tai sakau jūsų gėdai. 
\par 35 Bet gal kas paklaus: “Kaip bus prikelti mirusieji? Su kokiu kūnu jie pasirodys?” 
\par 36 Kvaily! Ką tu pasėji, neatgyja, jei prieš tai nenumiršta. 
\par 37 Ir ką besėtum, tu sėji ne būsimąjį kūną, bet pliką grūdą, sakysime, kviečių ar kitokių javų. 
\par 38 Tuo tarpu Dievas duoda jam kūną tokį, koks Jam patinka, ir kiekvienai sėklai jos kūną. 
\par 39 Ne visi kūnai yra vienodi. Vienoks žmonių kūnas, kitoks gyvulių, kitoks žuvų ir kitoks paukščių. 
\par 40 Taip pat yra dangaus kūnai ir žemės kūnai, bet vienokia dangaus kūnų šlovė ir kitokia žemės kūnų. 
\par 41 Vienokia yra saulės šlovė, kitokia šlovė mėnulio ir dar kitokia šlovė žvaigždžių. Ir žvaigždė nuo žvaigždės skiriasi šlove. 
\par 42 Taip ir su mirusiųjų prisikėlimu. Sėjamas gendantis kūnas, prikeliamas negendantis. 
\par 43 Sėjamas negarbingas, prikeliamas šlovingas. Sėjamas silpnas, prikeliamas galingas. 
\par 44 Sėjamas sielinis kūnas, prikeliamas dvasinis kūnas. Yra sielinis kūnas ir yra dvasinis kūnas. 
\par 45 Taip ir parašyta: “Pirmasis žmogus Adomas tapo gyva siela”; paskutinysis Adomas­gyvybę teikiančia dvasia. 
\par 46 Bet ne dvasinis pirmiau, o sielinis, ir tik po to dvasinis. 
\par 47 Pirmasis žmogus­iš žemės, žemiškas; antrasis žmogus­Viešpats iš dangaus. 
\par 48 Koks buvo žemiškasis, tokie yra ir žemiškieji, o koks yra dangiškasis, tokie yra ir dangiškieji. 
\par 49 Ir kaip nešiojame žemiškojo atvaizdą, taip nešiosime ir dangiškojo atvaizdą. 
\par 50 Bet aš jums, broliai, sakau, kad kūnas ir kraujas nepaveldės Dievo karalystės, ir kas genda, nepaveldės to, kas negenda. 
\par 51 Aš jums atskleidžiu paslaptį: ne visi užmigsime, bet visi būsime pakeisti,­ 
\par 52 staiga, viena akimirka, skambant paskutiniam trimitui. Trimitas nuskambės, ir mirusieji bus prikelti negendantys, o mes būsime pakeisti. 
\par 53 Nes šis gendantis turi apsivilkti negendamybe, ir šis marus apsivilkti nemarybe. 
\par 54 Kada šis gendantis apsivilks negendamybe ir šis marusis apsivilks nemarybe, tada išsipildys užrašytas žodis: “Pergalė prarijo mirtį! 
\par 55 Kurgi, mirtie, tavo geluonis? Kurgi, mirtie, tavo pergalė?” 
\par 56 Mirties geluonis yra nuodėmė, o nuodėmės jėga­įstatymas. 
\par 57 Bet dėkui Dievui, kuris duoda mums pergalę per mūsų Viešpatį Jėzų Kristų! 
\par 58 Todėl, mano mylimieji broliai, būkite tvirti, nepajudinami, visada gausūs Viešpaties darbais, žinodami, kad jūsų triūsas ne veltui Viešpatyje.


\chapter{16}


\par 1 Su rinkliava šventiesiems darykite taip, kaip nurodžiau Galatijos bažnyčioms. 
\par 2 Pirmąją savaitės dieną kiekvienas iš jūsų teatideda pagal tai, kiek turi, kad rinkliavos neprasidėtų man atvykus. 
\par 3 Atvykęs pasiųsiu į Jeruzalę žmones, kuriuos jūs nutarsite esant tinkamus, kad jie su palydimaisiais laiškais nugabentų jūsų dovaną. 
\par 4 Jei pasirodytų tinkama ir man keliauti, tai jie vyks kartu su manimi. 
\par 5 Atvyksiu pas jus, perėjęs Makedoniją. Mat per Makedoniją eisiu, 
\par 6 o pas jus, galimas daiktas, pabūsiu arba ir peržiemosiu, kad jūs mane palydėtumėte, kai vyksiu toliau. 
\par 7 Nenoriu dabar su jumis susitikti prabėgomis. Tikiuosi, jei Viešpats leis, kurį laiką pasilikti pas jus. 
\par 8 Bet Efeze išbūsiu iki Sekminių. 
\par 9 Mat man yra atvertos plačios durys našiam darbui, ir priešininkų daug. 
\par 10 Ir jeigu atvyks Timotiejus, žiūrėkite, kad jis gyventų pas jus be baimės, nes jis dirba Viešpaties darbą kaip ir aš. 
\par 11 Taigi tegul niekas jo neniekina. Išlydėkite jį su ramybe, kad atvyktų pas mane, nes laukiu jo su broliais. 
\par 12 O dėl brolio Apolo, tai aš didžiai troškau, kad jis su broliais keliautų pas jus, bet jis niekaip nenori šiuo metu. Bet atvyks, radęs tinkamą laiką. 
\par 13 Budėkite, tvirtai stovėkite tikėjime, elkitės vyriškai, būkite stiprūs! 
\par 14 Viską, ką darote, darykite su meile. 
\par 15 Prašau jūsų, broliai (jūs pažįstate Stepono namus: jie yra Achajos pirmatikiai ir atsidavę tarnauti šventiesiems), 
\par 16 kad jūs taip pat paklustumėt tokiems žmonėms ir visiems, kurie mums padeda ir su mumis dirba. 
\par 17 Aš džiaugiuosi Stepono, Fortunato ir Achaiko apsilankymu. Jie man atstojo jus, 
\par 18 atgaivino mano ir jūsų dvasią. Todėl pripažinkite tokius žmones! 
\par 19 Jus sveikina Azijos bažnyčios. Karštai sveikina Viešpatyje Akvilas ir Priscilė kartu su bažnyčia, kuri jų namuose. 
\par 20 Jus sveikina visi broliai. Pasveikinkite vienas kitą šventu pabučiavimu. 
\par 21 Sveikinu jus savo, Pauliaus, ranka. 
\par 22 Jei kas nemyli Viešpaties Jėzaus Kristaus, tebūna prakeiktas! Mūsų Viešpatie, ateik! 
\par 23 Viešpaties Jėzaus Kristaus malonė tebūna su jumis. 
\par 24 Mano meilė jums visiems Kristuje Jėzuje. Amen.



\end{document}