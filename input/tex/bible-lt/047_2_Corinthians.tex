\begin{document}

\title{Antrasis laiškas korintiečiams}

\chapter{1}


\par 1 Paulius, Dievo valia Kristaus Jėzaus apaštalas, ir brolis Timotiejus­Dievo bažnyčiai Korinte ir visiems šventiesiems, gyvenantiems visoje Achajoje. 
\par 2 Malonė jums ir ramybė nuo Dievo, mūsų Tėvo, ir Viešpaties Jėzaus Kristaus! 
\par 3 Tebūna palaimintas Dievas, mūsų Viešpaties Jėzaus Kristaus Tėvas, gailestingumo Tėvas ir visokios paguodos Dievas, 
\par 4 kuris guodžia mus kiekviename mūsų sielvarte, kad mes galėtume paguosti bet kokio sielvarto ištiktuosius ta paguoda, kuria patys Dievo guodžiami. 
\par 5 Kaip gausėja mumyse Kristaus kentėjimai, taip gausėja per Kristų ir mūsų paguoda. 
\par 6 Jeigu patiriam sielvartą, tai jūsų paguodai ir išgelbėjimui, kuris padeda iškęsti tokius pačius kentėjimus, kokius mes kenčiame; o jeigu mes guodžiami, tai jūsų nusiraminimui ir išgelbėjimui. 
\par 7 Ir mūsų viltis jumis tvirta, nes žinome, kad, būdami kentėjimų dalininkai, būsite ir paguodos dalininkai. 
\par 8 Mes nenorime, broliai, palikti jus nežinioje apie Azijoje mus ištikusią negandą. Mes buvome prislėgti daugiau nei mūsų jėgos leidžia­taip, kad nebesitikėjome išliksią gyvi. 
\par 9 Bet patys savyje patyrėme mirties nuosprendį, kad pasitikėtume ne savimi, o Dievu, kuris prikelia mirusius. 
\par 10 Jis išgelbėjo mus iš mirties nasrų ir tebegelbsti. Juo mes pasitikime, kad Jis mus gelbės toliau, 
\par 11 jums padedant malda už mus, kad daugelis dėl mūsų dėkotų už dovaną, per daugelį suteiktą mums. 
\par 12 Nes mūsų pasigyrimas yra toks: liudija mūsų sąžinė, kad pasaulyje, o ypač pas jus, mes elgėmės paprastai ir su dievišku nuoširdumu,­ne kūniška išmintimi, bet Dievo malone. 
\par 13 Nerašome jums nieko kito, kaip tik tai, ką jūs skaitote ir suprantate. Tikiuosi, kad ir iki galo suprasite, 
\par 14 kaip iš dalies jau mus supratote, kad mes esame jūsų pasigyrimas, kaip ir jūs mūsų­Viešpaties Jėzaus dieną. 
\par 15 Tokio pasitikėjimo kupinas, aš norėjau anksčiau pas jus atvykti, kad antrąkart gautumėt malonę, 
\par 16 ir pro jus keliauti į Makedoniją, o iš Makedonijos vėl grįžti pas jus, kad jūs mane išlydėtumėte į Judėją. 
\par 17 Ar taip manydamas, elgiausi lengvabūdiškai? O gal mano sumanymai buvo pagal kūną, kad mano “taip, taip” yra ir “ne, ne”? 
\par 18 Bet kaip Dievas ištikimas, mūsų žodis jums nebuvo “taip” ir “ne”. 
\par 19 Nes Dievo Sūnus, Jėzus Kristus, kurį jums paskelbėme aš, Silvanas ir Timotiejus, nebuvo “taip” ir “ne”, bet Jame buvo “taip”. 
\par 20 Nes visi Dievo pažadai Jame yra “taip” ir Jame “amen” Dievo šlovei per mus. 
\par 21 Tas, kuris sutvirtina mus su jumis Kristuje ir mus patepė, yra Dievas, 
\par 22 kuris ir užantspaudavo mus ir davė kaip užstatą Dvasią į mūsų širdis. 
\par 23 Šaukiuosi Dievą liudytoju savo sielai, kad jūsų gailėdamas aš neatvykau į Korintą. 
\par 24 Mes juk neviešpataujame jūsų tikėjimui, bet esame jūsų džiaugsmo talkininkai, nes tikėjimu jūs stovite tvirtai.


\chapter{2}


\par 1 Taigi aš nusprendžiau, kad neatvyksiu pas jus vėl su liūdesiu. 
\par 2 O jei aš jus liūdinu, tai kas gi mane pralinksmins, jeigu ne mano nuliūdintasis? 
\par 3 Aš tai parašiau, kad atvykęs neturėčiau liūdėti dėl tų, kuriais derėtų džiaugtis, pasitikėdamas jumis visais, kad mano džiaugsmas yra ir visų jūsų džiaugsmas. 
\par 4 Nes aš jums rašiau iš didelio skausmo ir širdies sielvarto, su gausiomis ašaromis, ne tam, kad jus nuliūdinčiau, bet kad pažintumėte meilę, kurios taip apsčiai jums turiu. 
\par 5 Bet jeigu kas nuliūdino, tai nuliūdino ne mane, bet,—kad neperdėčiau,—bent iš dalies­jus visus. 
\par 6 Tokiam žmogui pakanka daugumos paskirtos bausmės. 
\par 7 Net atvirkščiai, jūs turėtumėte jam atleisti ir jį paguosti, kad jo neapimtų pernelyg didelis liūdesys. 
\par 8 Todėl aš prašau jus parodyti jam meilę. 
\par 9 Nes tuo tikslu jums ir parašiau,­ kad patikrinčiau, ar jūs visais atžvilgiais esate klusnūs. 
\par 10 Kam jūs atleidžiate, tam atleidžiu ir aš; nes, jei kam atleidau, tai padariau dėl jūsų Kristaus akivaizdoje, 
\par 11 kad neapgautų mūsų šėtonas. Nes mums nėra nežinomi jo kėslai. 
\par 12 Kai atvykau į Troadę skelbti Kristaus Evangelijos, durys man buvo Viešpaties atvertos, 
\par 13 bet aš neturėjau savo dvasioje ramybės, kadangi nesutikau savo brolio Tito; tad atsisveikinęs iškeliavau į Makedoniją. 
\par 14 Dėkui Dievui, kuris visada veda mus Kristuje triumfo eisenoje ir per mus kiekvienoje vietoje skleidžia mielą Jo pažinimo kvapą. 
\par 15 Juk mes esame Kristaus malonus kvapas Dievui tarp išgelbėtų ir tarp žūstančių. 
\par 16 Vieniems­mirties kvapas mirčiai, kitiems­gyvenimo kvapas gyvenimui. O kas gi yra tam tinkamas? 
\par 17 Mes neprekiaujame, kaip daugelis, Dievo žodžiu, bet nuoširdžiai, kaip iš Dievo, kalbame Kristuje, Dievo akivaizdoje.


\chapter{3}


\par 1 Ar pradėsime iš naujo jums prisistatinėti? Gal mums reikia, kaip kai kuriems, palydimųjų laiškų jums ir iš jūsų? 
\par 2 Jūs esate mūsų laiškas, įrašytas mūsų širdyse, visų žmonių suprantamas ir skaitomas. 
\par 3 Jūs pasirodote esą Kristaus laiškas, mūsų tarnavimu parašytas ne rašalu, bet gyvojo Dievo Dvasia, ne akmens plokštėse, bet gyvų širdžių plokštėse. 
\par 4 Tokį pasitikėjimą Dievu mes turime per Kristų. 
\par 5 Ne todėl, kad būtume patys tinkami ką nors sumanyti tarytum iš savęs, bet mūsų tinkamumas iš Dievo, 
\par 6 kuris padarė mus tinkamus būti Naujosios Sandoros tarnais­ne raidės, bet Dvasios, nes raidė žudo, o Dvasia teikia gyvybę. 
\par 7 Jeigu mirties tarnavimas, išraižytas raidėmis akmenyse, buvo toks šlovingas, kad Izraelio vaikai negalėjo pažvelgti Mozei į veidą dėl jo veido šlovės, kuri buvo praeinanti, 
\par 8 tai kiek šlovingesnis bus Dvasios tarnavimas? 
\par 9 Jeigu pasmerkimo tarnavimas šlovingas, tai daug daugiau šlovingesnis teisumo tarnavimas. 
\par 10 Tai, kas buvo šlovinga, visai nešlovinga, lyginant su visa pranokstančia šlove. 
\par 11 Jeigu praeinantis dalykas buvo šlovingas, tai kur kas šlovingesnis pasiliekantis. 
\par 12 Taigi, turėdami tokią viltį, mes kalbame labai tiesiai ir drąsiai, 
\par 13 ne kaip Mozė, kuris ant veido užsileisdavo gaubtuvą, kad Izraelio vaikai nepamatytų to, kas praeina. 
\par 14 Bet jų protai buvo apakinti. Iki šios dienos tas pats gaubtuvas lieka nenuimtas skaitant Senąjį Testamentą, nes jis nuimamas Kristuje. 
\par 15 Net iki šios dienos, kai skaitomas Mozė, gaubtuvas tebedengia jų širdį. 
\par 16 Bet kai žmogus atsigręžia į Viešpatį, gaubtuvas nuimamas. 
\par 17 Viešpats yra Dvasia. O kur Viešpaties Dvasia, ten laisvė. 
\par 18 Mes visi, atidengtu veidu lyg veidrodyje regėdami Viešpaties šlovę, esame keičiami į tą patį atvaizdą iš šlovės į šlovę, veikiami Viešpaties, kuris yra Dvasia.


\chapter{4}


\par 1 Todėl, suprasdami, kad turime šį tarnavimą iš gailestingumo, mes nepailstame. 
\par 2 Atsisakę slaptų begėdysčių, nesileidžiame į gudravimus ir nekraipome Dievo žodžio, bet, atskleisdami tiesą, prisistatome kiekvieno žmogaus sąžinei Dievo akivaizdoje. 
\par 3 O jeigu Evangelija yra paslėpta, tai ji paslėpta žūstantiems, 
\par 4 kuriems šio amžiaus dievas apakino protus; netikintiems, kad jiems nesušvistų Kristaus, kuris yra Dievo atvaizdas, šlovės Evangelijos šviesa. 
\par 5 Mes ne save pačius skelbiame, bet Kristų Jėzų kaip Viešpatį, o mes­jūsų tarnai dėl Jėzaus. 
\par 6 Nes tai Dievas, kuris įsakė iš tamsos nušvisti šviesai, sušvito mūsų širdyse, kad suteiktų mums Dievo šlovės pažinimo šviesą Kristaus veide. 
\par 7 Bet šitą turtą mes laikome moliniuose induose, kad būtų aišku, jog visa viršijanti jėgos apstybė iš Dievo, o ne iš mūsų. 
\par 8 Mes visaip spaudžiami, bet nesugniuždyti; suglumę, bet nenusivylę; 
\par 9 persekiojami, bet nepalikti; parblokšti, bet nežuvę. 
\par 10 Visada nešiojame savo kūne Viešpaties Jėzaus mirtį, kad ir Jėzaus gyvybė apsireikštų mūsų kūne. 
\par 11 Nes mes, gyvieji, dėl Jėzaus nuolat atiduodami mirčiai, kad ir Jėzaus gyvybė apsireikštų mūsų mirtingame kūne. 
\par 12 Taigi mumyse veikia mirtis, o jumyse gyvybė. 
\par 13 Turėdami tą pačią tikėjimo dvasią, apie kurią parašyta: “Aš įtikėjau, todėl prakalbėjau”, mes irgi tikime ir todėl kalbame, 
\par 14 žinodami, kad Tas, kuris prikėlė Viešpatį Jėzų, taip pat ir mus prikels per Jėzų ir pastatys kartu su jumis. 
\par 15 Nes viskas yra dėl jūsų, kad per daugelį pagausėjusi malonė pagausintų dėkojimą Dievo šlovei. 
\par 16 Todėl mes nepailstame. Nors mūsų išorinis žmogus ir nyksta, vidinis diena iš dienos atsinaujina. 
\par 17 Mūsų trumpalaikis lengvas sielvartas ruošia mums visa pranokstančią amžinąją šlovę. 
\par 18 Tuo tarpu mes nežiūrime į tai, kas regima, bet į tai, kas neregima, nes kas regima, yra laikina, o kas neregima­amžina.


\chapter{5}


\par 1 Mes žinome, kad, mūsų žemiškajam namui, šiai palapinei, suirus, turime pastatą iš Dievo, ne rankomis statytus amžinus namus danguje. 
\par 2 Todėl mes dejuojame, karštai trokšdami apsivilkti savo buveinę iš dangaus, 
\par 3 nes aprengti nebūsime nuogi. 
\par 4 Juk mes, esantys šioje palapinėje, dejuojame prislėgti, norėdami ne nusirengti, bet apsirengti, kad tai, kas maru, būtų gyvenimo praryta. 
\par 5 Tam mus paruošęs yra Dievas, kuris davė mums Dvasią kaip užstatą. 
\par 6 Todėl mes visada užtikrinti, žinodami, kad, kol gyvename namuose­kūne, mes nesame su Viešpačiu,— 
\par 7 mes gyvename tikėjimu, o ne regėjimu,— 
\par 8 tačiau esame užtikrinti ir norėtume verčiau palikti kūną ir būti kartu su Viešpačiu. 
\par 9 Todėl mes siekiame,­tiek nebūdami, tiek būdami su Juo,­Jam patikti. 
\par 10 Nes mums visiems reikės stoti prieš Kristaus teismo krasę, kad kiekvienas atsiimtų pagal tai, ką jis, gyvendamas kūne, darė­gera ar bloga. 
\par 11 Todėl, pažindami Viešpaties baimę, mes stengiamės įtikinti žmones. Dievui mes esame atviri, tikiuosi, kad esame atviri ir jūsų sąžinėms. 
\par 12 Neprisistatome jums iš naujo, bet duodame progą pasigirti mumis, kad turėtumėte, ką sakyti žmonėms, besigiriantiems savo išore, o ne širdimi. 
\par 13 Jei mes išprotėję­tai Dievui; jei esame sveiko proto­tai dėl jūsų. 
\par 14 Nes Kristaus meilė valdo mus, įsitikinusius, kad jei vienas mirė už visus, tai ir visi yra mirę. 
\par 15 O Jis mirė už visus, kad gyvieji nebe sau gyventų, bet Tam, kuris už juos mirė ir prisikėlė. 
\par 16 Todėl nuo šiol mes nė vieno nebepažįstame pagal kūną. Jei mes ir pažinome Kristų pagal kūną, tai dabar taip Jo nebepažįstame. 
\par 17 Taigi, jei kas yra Kristuje, tas yra naujas kūrinys. Kas sena­ praėjo, štai visa tapo nauja. 
\par 18 O visa tai iš Dievo, kuris per Jėzų Kristų sutaikino mus su savimi ir davė mums sutaikinimo tarnavimą. 
\par 19 Tai Dievas Kristuje sutaikino su savimi pasaulį, nebeįskaitydamas žmonėms jų nusikaltimų, ir davė mums sutaikinimo žodį. 
\par 20 Taigi mes esame Kristaus pasiuntiniai, tarsi pats Dievas prašytų per mus. Kristaus vardu maldaujame: “Susitaikinkite su Dievu!” 
\par 21 Nes Tą, kuris nepažino nuodėmės, Jis padarė nuodėme dėl mūsų, kad mes Jame taptume Dievo teisumu.


\chapter{6}


\par 1 Būdami Jo bendradarbiai, jūsų taip pat prašome, kad nepriimtumėte Dievo malonės veltui! 
\par 2 Nes Jis sako: “Aš išklausiau tave priimtinu metu ir išgelbėjimo dieną Aš padėjau tau”. Štai dabar yra priimtinas metas, štai dabar išgelbėjimo diena! 
\par 3 Mes niekuo neduodame akstino pasipiktinti, kad mūsų tarnavimas nebūtų peiktinas. 
\par 4 Bet visame kame pasirodome Dievo tarnai: su didele kantrybe, skausmuose, sunkumuose, suspaudimuose, 
\par 5 plakimuose, įkalinimuose, sąmyšiuose, darbuose, budėjimuose, pasninkuose; 
\par 6 tyrumu, pažinimu, pakantumu, gerumu, Šventąja Dvasia, neveidmainiška meile, 
\par 7 tiesos žodžiu, Dievo jėga, teisumo ginklais iš dešinės ir kairės; 
\par 8 gerbiami ir negerbiami, šmeižiami ir giriami, laikomi apgavikais ir teisiais, 
\par 9 nepažįstamais ir gerai žinomais, laikomi mirštančiais­bet štai mes gyvi; esame baudžiami, bet nenužudomi, 
\par 10 mus liūdina, bet mes visada džiaugiamės, esame skurdžiai, bet daugelį praturtiname, neturime nieko­ir valdome viską. 
\par 11 O korintiečiai! Mūsų lūpos atvirai jums prabilo, mūsų širdis plačiai atverta. 
\par 12 Ne mumyse jums ankšta; ankšta jūsų pačių širdyse. 
\par 13 Tad atsimokėkite tuo pačiu,­ kalbu kaip vaikams,­ir taip pat atsiverkite. 
\par 14 Nevilkite svetimo jungo su netikinčiais. Kas gi bendro tarp teisumo ir nusikaltimo? Ir kas bendro tarp šviesos ir tamsos? 
\par 15 Kaipgi galima gretinti Kristų su Beliaru? Arba kokia dalis tikinčio su netikinčiu? 
\par 16 Ir kaip suderinti Dievo šventyklą su stabais? Juk jūs esate gyvojo Dievo šventykla, kaip Dievas yra pasakęs: “Aš gyvensiu juose ir vaikščiosiu tarp jų; būsiu jų Dievas, ir jie bus manoji tauta”. 
\par 17 Todėl: “Išeikite iš jų ir atsiskirkite,­sako Viešpats,­ir nelieskite to, kas netyra, ir Aš jus priimsiu 
\par 18 ir būsiu jums Tėvas, o jūs būsite mano sūnūs ir dukterys,­sako visagalis Viešpats”.


\chapter{7}


\par 1 Taigi, mylimieji, turėdami tokius pažadus, apsivalykime nuo visos kūno ir dvasios nešvaros, tobulindami šventumą Dievo baimėje. 
\par 2 Priimkite mus. Mes nė vieno nenuskriaudėme, nė vienam nepakenkėm, nė vieno neapgavome. 
\par 3 Tai sakau, ne norėdamas jus smerkti. Nes jau esu sakęs, jog esate mūsų širdyse, kad kartu mirtume ir kartu gyventume. 
\par 4 Aš labai pasitikiu jumis ir labai jumis didžiuojuosi. Esu kupinas paguodos ir džiaugsmo visuose mūsų sunkumuose. 
\par 5 Kai atvykome į Makedoniją, mūsų kūnui neteko patirti nė kiek ramybės; mes buvome visokeriopai varginami: iš išorės­kovos, viduje­baimė. 
\par 6 Bet Dievas, pažemintųjų guodėjas, paguodė mus Tito atvykimu. 
\par 7 Ir ne vien jo atvykimu, bet ir ta paguoda, kuria jis buvo paguostas pas jus. Jis pranešė mums apie jūsų karštą troškimą, dejones, jūsų uolumą man. Taigi aš pradžiugau dar labiau. 
\par 8 Todėl jeigu nuliūdinau jus laišku, tai šito nesigailiu, nors ir apgailestavau. Nes matau, kad tas laiškas jus nuliūdino, bet tik kuriam laikui. 
\par 9 Dabar aš džiaugiuosi, žinoma, ne todėl, kad jums teko nuliūsti, bet kad jūsų nuliūdimas atvedė jus į atgailą. Jūs buvote nuliūdę dievišku liūdesiu, todėl iš mūsų pusės nebuvo jums jokios skriaudos. 
\par 10 Dieviškas liūdesys gimdo atgailą išgelbėjimui, dėl kurio nereikia gailėtis; o pasaulio liūdesys gimdo mirtį. 
\par 11 Ir štai kaip tik tas dieviškas nuliūdimas pagimdė jumyse tokį susirūpinimą, tokį teisinimąsi, apmaudą, baimę, tokį stiprų troškimą, uolumą, tokį atpildą! Jūs visais atžvilgiais pasirodėte švarūs šiame reikale. 
\par 12 Todėl, jeigu jums parašiau, tai ne dėl įžeidėjo ir ne dėl įžeistojo, bet kad jums paaiškėtų mūsų rūpestis jumis Dievo akivaizdoje. 
\par 13 Todėl mes pasiguodėme jūsų paguoda ir dar labiau nudžiugome dėl Tito džiaugsmo, nes jo dvasia buvo jūsų visų atgaivinta. 
\par 14 Taigi nelikau sugėdintas, kad jam buvau pasigyręs jumis, bet kaip ir viską jums teisingai kalbėjome, taip ir mūsų pasigyrimas Titui pasirodė teisingas. 
\par 15 Ir jo širdis dar labiau prisirišusi prie jūsų, nes jis prisimena jūsų paklusnumą ir kaip priėmėte jį su baime ir drebėdami. 
\par 16 Todėl džiaugiuosi, kad visais atžvilgiais galiu jumis pasitikėti.


\chapter{8}


\par 1 Be to, pranešame jums, broliai, apie Dievo malonę, suteiktą Makedonijos bažnyčioms. 
\par 2 Nors dideli vargai jas bandė, jos pasirodė kupinos džiaugsmo, ir jų gilus skurdas išsiliejo ypatingo dosnumo turtais. 
\par 3 Aš liudiju, kad jie pagal išgales ir viršydami išgales, savo noru, 
\par 4 prašyte prašė mus, kad mes priimtume dovaną ir jų dalyvavimą tarnavime šventiesiems. 
\par 5 Ir ne tik taip, kaip mes tikėjomės, bet visų pirma atsidavė Viešpačiui, o paskui Dievo valia ir mums. 
\par 6 Todėl mes paprašėme Titą, kad taip, kaip pradėjo, taip ir baigtų pas jus šią malonę. 
\par 7 Tad, būdami visa ko pertekę­tikėjimo, žodžio, pažinimo, visokeriopo uolumo ir meilės mums,­ būkite pertekę ir šios malonės. 
\par 8 Tai sakau ne įsakydamas, bet norėdamas kitų uolumu patikrinti jūsų meilės nuoširdumą. 
\par 9 Nes jūs pažįstate mūsų Viešpaties Jėzaus Kristaus malonę, jog Jis, būdamas turtingas, dėl jūsų tapo vargšu, kad jūs per Jo neturtą taptumėte turtingi. 
\par 10 Ir čia aš duodu patarimą, nes tai naudinga jums, kurie ne tik pradėjote daryti tai anksčiau, bet ir troškote jau nuo pernai. 
\par 11 Taigi dabar pabaikite šį darbą, kad jūsų nuovokus troškimas būtų įvykdytas iš to, ką turite. 
\par 12 Nes jei yra sąmoningas noras, jis priimtinas pagal tai, ką žmogus turi, o ne pagal tai, ko neturi. 
\par 13 Aš nenoriu, kad kitiems tektų lengvatos, o jums naštos, 
\par 14 bet kad būtų lygybė: kad šiuo metu jūsų perteklius patenkintų jų nepriteklių, o vėliau jų perteklius taip pat patenkintų jūsų nepriteklių ir būtų lygybė, 
\par 15 kaip parašyta: “Kas daug surinko, neturėjo atliekamo, o kas mažai­nekentėjo nepritekliaus”. 
\par 16 Dėkui Dievui, įdėjusiam į Tito širdį tą patį uolų rūpestį jumis. 
\par 17 Jis ne tik paklausė mano raginimo, bet, būdamas labai uolus, nuvyko pas jus savo noru. 
\par 18 Kartu su juo pasiuntėme brolį, giriamą dėl Evangelijos skelbimo visose bažnyčiose. 
\par 19 Dar daugiau, jis yra bažnyčių paskirtas mūsų kelionių palydovu šiai malonei, kurią mes vykdome paties Viešpaties šlovei ir jūsų paslaugumui parodyti, 
\par 20 sergėdamiesi, kad niekas mūsų nekaltintų dėl šios gausos, kurią mes tvarkome. 
\par 21 Mes rūpinamės tuo, kas gerbtina ne tik Viešpaties, bet ir žmonių akyse. 
\par 22 Taigi su jais pasiuntėme savo brolį, kurio uolumą daugel kartų įvairiai esame patikrinę ir kuris dabar yra dar uolesnis labai jumis pasitikėdamas. 
\par 23 Titas‰tai mano bendražygis ir jūsų pagalbininkas, o tie mūsų broliai‰bažnyčių pasiuntiniai, Kristaus šlovė. 
\par 24 Todėl jų ir visų bažnyčių akyse įrodykite savo meilę ir mūsų pasigyrimą jumis.


\chapter{9}


\par 1 Apie tarnavimą šventiesiems man nebereikia jums rašyti: 
\par 2 aš žinau jūsų pasiryžimą ir giruosi jumis makedoniečiams, sakydamas, kad Achaja pasiruošusi nuo pereitų metų. Ir jūsų uolumas yra daugelį paskatinęs. 
\par 3 O brolius siunčiu tam, kad mano pasigyrimas jumis nepasirodytų šiuo atveju tuščias ir kad jūs, kaip sakiau, būtumėte pasiruošę. 
\par 4 Kad, jei kartais makedoniečiai atvyktų kartu su manimi ir rastų jus nepasiruošusius, mes (nekalbant apie jus) nebūtume sugėdinti dėl tokio tvirto pasigyrimo. 
\par 5 Todėl maniau esant reikalinga paprašyti brolius iš anksto nuvykti pas jus ir pasirūpinti, kad anksčiau viešai paskelbtas palaiminimas būtų paruoštas ir kad liudytų jūsų dosnumą, o ne godumą. 
\par 6 Štai ką pasakysiu: kas šykščiai sėja, šykščiai ir pjaus, o kas dosniai sėja, dosniai ir pjaus. 
\par 7 Kiekvienas tegul aukoja, kaip yra širdyje nutaręs, ne gailėdamas ar verčiamas, nes Dievas myli linksmą davėją. 
\par 8 O Dievas gali jus gausiai apdovanoti visokeriopomis malonėmis, kad visada ir visais atžvilgiais būtumėme aprūpinti ir turtingi kiekvienam geram darbui, 
\par 9 kaip parašyta: “Jis pažėrė, Jis davė vargšams; Jo teisumas išlieka per amžius”. 
\par 10 Tas, kuris parūpina sėklos sėjėjui ir duonos valgytojui, padaugins jūsų pasėtą sėklą ir subrandins jūsų teisumo vaisius. 
\par 11 O jūs ir taip esate visokeriopai turtingi ir dosnūs, dėl ko mes ir dėkojame Dievui. 
\par 12 Nes šis tarnavimas ne tik patenkina šventųjų poreikius, bet ir gausina daugelio dėkojimus Dievui. 
\par 13 Patyrę tokias paslaugas, jie šlovins Dievą už jūsų paklusnumą išpažįstamai Kristaus Evangelijai ir už jūsų dosnų pasidalijimą su jais ir visais kitais. 
\par 14 Jie melsis už jus ir ilgėsis jūsų dėl visa pranokstančios Dievo malonės jumyse. 
\par 15 Ačiū Dievui už neapsakomą Jo dovaną!


\chapter{10}


\par 1 Aš pats, Paulius, jus maldauju Kristaus romumu ir švelnumu­aš, kuris “akyse su jumis esu toks nusižeminęs, o už akių toks drąsus”. 
\par 2 Aš jus maldauju, kad atvykęs neturėčiau pasirodyti smarkuoliu, pasiryžusiu griežtai sudrausti kai kuriuos, manančius, jog mes elgėmės pagal kūną. 
\par 3 Nors mes gyvename kūne, kovojame ne pagal kūną. 
\par 4 Mūsų kovos ginklai ne kūniški, bet galingi Dieve griauti tvirtoves. 
\par 5 Jais mes nugalime samprotavimus ir bet kokią puikybę, kuri sukyla prieš Dievo pažinimą, ir paimame nelaisvėn kiekvieną mintį, kad paklustų Kristui, 
\par 6 esame pasiruošę nubausti kiekvieną neklusnumą, kai tik jūsų klusnumas taps tobulas. 
\par 7 Negi viską vertinate pagal išorę? Jei kas pasitiki, kad yra Kristaus, tegul pamąsto dar kartą: kaip jis Kristaus, taip ir mes. 
\par 8 O jei panorėčiau daugiau pasigirti ta valdžia, kurią Viešpats mums suteikė jūsų ugdymui, o ne griovimui, tai nebūtų man gėdos. 
\par 9 Beje, nenorėčiau pasirodyti bauginąs jus laiškais. 
\par 10 Nes “jo laiškai,­sako,­yra svarūs ir stiprūs, bet kūno išvaizda menka ir iškalba prasta”. 
\par 11 Kas taip mano, teįsidėmi, jog kokie esame iš tolo laiško žodžiais, tokie būsime ir vietoje darbais. 
\par 12 Mes nedrįstame savęs išskirti ar lyginti su kai kuriais žmonėmis, kurie patys save giria. Juk jie, matuodami save pagal save pačius ir lygindami save su savimi, elgiasi neprotingai. 
\par 13 Mes nesigiriame be saiko, o tik iki tų ribų, kurias mums Dievas užbrėžė ir kurios siekia net ligi jūsų. 
\par 14 Mes nepersistengiame,­tarytum nebūtume pasiekę jūsų,­nes Kristaus Evangelija pasiekėme ir jus. 
\par 15 Nesigiriame be saiko svetimo darbo vaisiais, bet turime viltį, jūsų tikėjimui augant, su jumis peraugti ligšiolines ribas, 
\par 16 kad galėtume skelbti Evangeliją už jūsų ribų ir nesigirti kito atliktu darbu svetimoje srityje. 
\par 17 “Kas giriasi, tesigiria Viešpačiu!” 
\par 18 Ne tas pagirtinas, kuris pats save giria, bet tas, kurį Viešpats pagiria.


\chapter{11}


\par 1 O, kad jūs pakęstumėte truputėlį mano kvailumo! Betgi jūs ir pakenčiate. 
\par 2 Aš pavyduliauju dėl jūsų Dievo pavydu, nes sužiedavau jus su vienu vyru, kad nuvesčiau jus Kristui kaip skaisčią mergelę. 
\par 3 Bet bijau, kad kaip gyvatė savo gudrumu suvedžiojo Ievą, taip ir jūsų mintys nesugestų be paprastumo Kristuje. 
\par 4 Mat jei kas užklydęs ima skelbti kitą Jėzų, kurio mes neskelbėme, arba jei jūs priimate kitą dvasią, kurios nebuvote priėmę, ar kitą evangeliją, kurios nebuvote gavę, jūs ramiausiai tai pakenčiate. 
\par 5 Bet aš manau nesąs prastesnis už pačius didžiausius apaštalus. 
\par 6 Ir jei man trūksta iškalbingumo, tai anaiptol ne pažinimo. Ir tai jums esame aiškiai įrodę visais atžvilgiais. 
\par 7 Nejaugi nusidėjau, kad pažeminau save ir išaukštinau jus, paskelbdamas jums Dievo Evangeliją už dyką? 
\par 8 Apiplėšiau kitas bažnyčias, imdamas iš jų atlyginimą, kad galėčiau tarnauti jums. 
\par 9 O būdamas pas jus ir stokodamas, nė vieno neapsunkinau, nes ko man trūko, parūpino iš Makedonijos atvykę broliai. Aš saugojausi ir ateityje saugosiuos tapti jums našta bet kuria prasme. 
\par 10 Sakau jums vardan Kristaus tiesos, esančios manyje, kad šio pasididžiavimo Achajos srityse niekas iš manęs neatims. 
\par 11 Kodėl? Ar todėl, kad jūsų nemyliu? Dievas žino! 
\par 12 Ką darau, darysiu ir toliau, kad negalėtų pasiteisinti norintys pasiteisinti, kad tuo, kuo giriasi, jie pasirodytų esą tokie kaip ir mes. 
\par 13 Juk tokie yra netikri apaštalai, apgaulingi darbininkai, besidedantys Kristaus apaštalais. 
\par 14 Ir nenuostabu. Juk pats šėtonas apsimeta šviesos angelu. 
\par 15 Tad nieko ypatingo, jei jo tarnai apsimeta teisumo tarnais. Bet jų galas bus pagal jų darbus. 
\par 16 Kartoju: tegu nei vienas nelaiko manęs kvailiu! O jeigu jau laikote, tai pasiklausykite manęs kaip kvailo, kad irgi galėčiau truputį pasigirti. 
\par 17 Ką pasakysiu, pasakysiu ne pagal Viešpatį, bet tarytum kvailiodamas ir manydamas galįs pasigirti. 
\par 18 Matydamas, kad daug kas giriasi pagal kūną, tai pasigirsiu ir aš. 
\par 19 Juk, būdami protingi, mokate mielai pakęsti kvailius. 
\par 20 Nes jūs pakenčiate, kai jus pavergia, kai apryja, kai atima, kai didžiuojasi, kai smogia per veidą. 
\par 21 Mūsų gėdai pasakysiu, kad buvome tam per silpni. Bet jei kas kuo nors drąsus,­tai sakau iš kvailumo,­ aš drąsus taip pat. 
\par 22 Jie žydai? Ir aš. Jie izraelitai? Ir aš. Jie Abraomo palikuonys? Ir aš. 
\par 23 Jie Kristaus tarnai? Kalbu kaip kvailys: aš juo labiau! Aš daug daugiau darbavausi, gavau rykščių be saiko, daugiau kalėjau ir daugel kartų buvau mirties pavojuje. 
\par 24 Nuo žydų gavau penkis kartus po keturiasdešimt be vieno kirčio. 
\par 25 Tris kartus buvau muštas lazdomis, vienąkart buvau užmėtytas akmenimis. Tris kartus pergyvenau laivo sudužimą, ištisą parą plūduriavau jūroje. 
\par 26 Dažnai buvau kelionėse, upių pavojuose, pavojuose nuo plėšikų, pavojuose nuo tautiečių, pavojuose nuo pagonių, pavojuose mieste, dykumos pavojuose, pavojuose jūroje, pavojuose nuo netikrų brolių. 
\par 27 Kenčiau nuovargį ir skausmą, dažnai budėjau naktimis, alkau ir troškau, dažnai pasninkavau, kenčiau šaltį ir nuogumą. 
\par 28 Neminint viso kito, kas atsitinka kasdien, rūpinuosi visomis bažnyčiomis. 
\par 29 Jei kas silpsta, ar aš nesilpstu? Jei kas piktinasi, ar aš nedegu apmaudu? 
\par 30 Jei reikia girtis, girsiuosi savo silpnumu. 
\par 31 Dievas, mūsų Viešpaties Jėzaus Kristaus Tėvas, kuris palaimintas per amžius, žino, kad nemeluoju. 
\par 32 Damaske karaliaus Areto valdytojas saugojo damaskiečių miestą, norėdamas mane suimti, 
\par 33 bet buvau pro langą nuleistas pintinėje per sieną ir taip ištrūkau iš jo rankų.


\chapter{12}

\par 1 Jei reikia girtis (nors iš to jokios naudos), eisiu prie Viešpaties regėjimų ir apreiškimų. 
\par 2 Pažįstu žmogų Kristuje, kuris prieš keturiolika metų,­ar kūne, ar be kūno­nežinau, Dievas žino,­buvo paimtas iki trečiojo dangaus. 
\par 3 Ir žinau, kad šitas žmogus,­ar kūne, ar be kūno­nežinau, Dievas žino,­ 
\par 4 buvo paimtas į rojų ir girdėjo neišreiškiamus žodžius, kurių nevalia žmogui ištarti. 
\par 5 Tokiu pasigirsiu, o ne savimi, nebent savo negaliomis. 
\par 6 Jei norėčiau girtis, nebūčiau kvailys, nes kalbėčiau tiesą. Bet susilaikau, kad kas nors apie mane nepagalvotų daugiau negu tai, ką manyje mato ar iš manęs girdi. 
\par 7 Ir kad perdėm neišpuikčiau dėl gausybės apreiškimų, man duotas dyglys kūne, šėtono pasiuntinys, kad mane smūgiuotų ir aš neišpuikčiau. 
\par 8 Dėl to tris kartus meldžiau Viešpatį, kad tai nuo manęs atitrauktų. 
\par 9 Bet Viešpats man pasakė: “Pakanka tau mano malonės, nes mano stiprybė tampa tobula silpnume”. Todėl mieliausiai girsiuosi savo silpnumais, kad Kristaus jėga ilsėtųsi ant manęs. 
\par 10 Patenkintas tad silpnumu, paniekinimais, sunkumais, persekiojimais ir priespauda dėl Kristaus, nes, būdamas silpnas, esu galingas. 
\par 11 Jūsų verčiamas, tapau kvailiu besigirdamas. Iš tikro tai jūs turėtumėte mane girti, nes aš ne prastesnis už pačius didžiausius apaštalus, nors esu niekas. 
\par 12 Iš tiesų jūsų akyse pasitvirtino apaštalo ženklai: visokeriopa kantrybė, ženklai, stebuklai ir galingi darbai. 
\par 13 Tad ko jums trūksta palyginti su kitomis bažnyčiomis? Vien to, kad nebuvau jums našta. Atleiskite man šitą nusikaltimą! 
\par 14 Esu pasiruošęs trečią kartą atvykti pas jus, tačiau neapsunkinsiu jūsų. Ieškau ne to, kas jūsų, bet jūsų pačių. Juk ne vaikai privalo krauti turtą tėvams, bet tėvai vaikams. 
\par 15 Todėl mielai išleisiu tai, kas mano, ir dar save pridėsiu jūsų sielų labui; nors, kuo daugiau jus myliu, tuo mažiau esu mylimas. 
\par 16 Tebūnie ir taip,­sakysite,­aš jūsų neapsunkinau, bet, būdamas gudrus, klasta jus pagavau. 
\par 17 Bet ar kaip nors išnaudojau jus per kurį savo pasiuntinį? 
\par 18 Paprašiau Titą keliauti ir pasiunčiau su juo vieną brolį. Ar Titas jus išnaudojo? Ar mes veikėme ne ta pačia dvasia? Ar nevaikščiojome tomis pačiomis pėdomis? 
\par 19 Vėl jūs manote, kad prieš jus teisinamės. Mes kalbame prieš Dievą Kristuje: viską darome, mylimieji, jūsų ugdymui. 
\par 20 Mat bijau, kad atvykęs nerasčiau jūsų tokių, kokių nenoriu, ir kad pats nebūčiau toks, kokio jūs nenorite; kad nebūtų nesantaikos, pavydo, piktumo, barnių, šmeižtų, apkalbų, pasididžiavimo, netvarkos, 
\par 21 kad, kai vėl atvyksiu, Dievas nepažemintų manęs prieš jus ir man netektų liūdėti dėl daugelio, kurie pirmiau buvo nusidėję ir dar neatgailavo dėl padarytų netyrumo, ištvirkavimo ir gašlumo darbų.


\chapter{13}


\par 1 Jau trečią kartą keliauju pas jus. “Dviejų ar trijų liudytojų lūpomis bus patvirtintas kiekvienas žodis”. 
\par 2 Jau sakiau anksčiau ir sakau, kaip ir antrąkart lankydamasis pas jus, ir, būdamas atstu, dabar rašau tiems, kurie anksčiau nusidėjo, ir visiems kitiems,­jeigu atvyksiu vėl, būsiu negailestingas. 
\par 3 Jūs reikalaujate įrodymo, kad manyje kalba Kristus, o Jis nėra silpnas prieš jus, bet galingas jumyse. 
\par 4 Nors Jis buvo nukryžiuotas dėl silpnumo, bet dabar gyvas Dievo jėga. Nors mes irgi silpni Jame, bet jums gyvensime su Juo Dievo galybe. 
\par 5 Patikrinkite patys save, ar esate tikėjime. Ištirkite save! Ar nepažįstate savęs ir nežinote, kad jumyse yra Jėzus Kristus, jeigu tik nesate atmestini? 
\par 6 Bet turiu vilties, kad jūs suprasite, jog mes nesame atmestini. 
\par 7 Aš meldžiu Dievą, kad jūs nedarytumėte nieko blogo,­ne tam, kad pasirodytume tinkami, bet kad jūs darytumėte gera, o mes būtume tartum atmestini. 
\par 8 Juk nieko negalime daryti prieš tiesą, bet tik už tiesą. 
\par 9 Mes džiaugiamės, kai esame silpni, o jūs stiprūs. Ir taip pat norime jūsų tobulumo. 
\par 10 Todėl, būdamas toli, tai rašau, kad atvykęs neturėčiau griežtai elgtis, naudodamas valdžią, kurią man Viešpats suteikė tam, kad ugdyčiau, o ne kad griaučiau. 
\par 11 Galiausiai, broliai, likite sveiki. Būkite tobuli, džiaukitės paguoda, būkite vienos minties, gyvenkite taikiai, ir meilės bei ramybės Dievas bus su jumis. 
\par 12 Pasveikinkite vieni kitus šventu pabučiavimu. 
\par 13 Jus sveikina visi šventieji. Viešpaties Jėzaus Kristaus malonė, Dievo meilė ir Šventosios Dvasios bendravimas tebūna su jumis visais! Amen.



\end{document}