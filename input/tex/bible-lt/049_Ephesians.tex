\begin{document}

\title{Laiškas efeziečiams}

\chapter{1}


\par 1 Paulius, Dievo valia Jėzaus Kristaus apaštalas, šventiesiems, gyvenantiems Efeze, ir ištikimiesiems Kristuje Jėzuje. 
\par 2 Malonė jums ir ramybė nuo Dievo, mūsų Tėvo, ir Viešpaties Jėzaus Kristaus! 
\par 3 Tebūna palaimintas Dievas, mūsų Viešpaties Jėzaus Kristaus Tėvas, kuris palaimino mus Kristuje visais dvasiniais palaiminimais danguje, 
\par 4 mus išrinkdamas Jame prieš pasaulio sutvėrimą, kad būtume šventi ir nekalti meile Jo akivaizdoje. 
\par 5 Geros valios nutarimu Jis iš anksto paskyrė mus įsūnyti per Jėzų Kristų 
\par 6 Jo malonės šlovės gyriui, kuria padarė mus priimtinus Mylimajame. 
\par 7 Jame turime atpirkimą per Jo kraują ir nuodėmių atleidimą pagal turtus Jo malonės, 
\par 8 kurią Jis dosniai suteikė mums su visa išmintimi ir supratimu, 
\par 9 paskelbdamas mums savo valios paslaptį pagal savo palankumą, kaip nusprendė savyje, 
\par 10 kad, laikų pilnatvei atėjus, galėtų suvienyti Kristuje visa, kas yra tiek danguje, tiek ir žemėje. 
\par 11 Jame ir gavome palikimą, iš anksto paskirtą sutvarkymu To, kuris visa veikia pagal savo valios nutarimą, 
\par 12 kad pasitarnautume Jo šlovės gyriui mes, kurie nuo seno turėjome viltį Kristuje. 
\par 13 Jame ir jūs, išgirdę tiesos žodį­ jūsų išgelbėjimo Evangeliją­ir įtikėję Juo, esate užantspauduoti pažadėtąja Šventąja Dvasia, 
\par 14 kuri yra mūsų paveldėjimo užstatas iki nuosavybės atpirkimo Jo šlovės gyriui. 
\par 15 Todėl ir aš, išgirdęs apie jūsų tikėjimą Viešpačiu Jėzumi ir apie jūsų meilę visiems šventiesiems, 
\par 16 nesiliauju dėkojęs už jus, prisimindamas jus savo maldose, 
\par 17 kad mūsų Viešpaties Jėzaus Kristaus Dievas, šlovės Tėvas, duotų jums išminties ir apreiškimo dvasią Jo pažinimui 
\par 18 ir apšviestų jūsų širdies akis, kad pažintumėte, kokia yra Jo pašaukimo viltis, kokie Jo palikimo šlovės turtai šventuosiuose 
\par 19 ir kokia beribė Jo jėgos didybė mums, kurie tikime, veikiant jo galingai jėgai. 
\par 20 Ja Jis veikė Kristuje, prikeldamas Jį iš numirusių ir pasodindamas savo dešinėje danguose, 
\par 21 aukščiau už kiekvieną kunigaikštystę, valdžią, jėgą, viešpatystę ir už kiekvieną vardą, tariamą ne tik šiame amžiuje, bet ir būsimajame, 
\par 22 ir visa paklojo po Jo kojomis, o Jį patį pastatė viršum visko, kad būtų galva bažnyčios, 
\par 23 kuri yra Jo kūnas, pilnatvė To, kuris visa visame pripildo.


\chapter{2}


\par 1 Ir jūs buvote mirę nusikaltimais ir nuodėmėmis, 
\par 2 kuriuose kadaise gyvenote pagal šio pasaulio būdą, paklusdami kunigaikščiui, viešpataujančiam ore, dvasiai, kuri dabar veikia neklusnumo vaikuose. 
\par 3 Tarp jų kadaise ir mes visi gyvenome, sekdami savo kūno geiduliais, vykdydami kūno ir minčių troškimus, ir iš prigimties buvome rūstybės vaikai kaip ir kiti. 
\par 4 Bet Dievas, apstus gailestingumo, iš savo didžios meilės, kuria mus pamilo, 
\par 5 mus, mirusius nusikaltimais, atgaivino kartu su Kristumi,­malone jūs esate išgelbėti,­ 
\par 6 kartu prikėlė ir pasodino danguje Kristuje Jėzuje, 
\par 7 kad ateinančiais amžiais savo gerumu parodytų mums beribius savo malonės turtus Kristuje Jėzuje. 
\par 8 Nes jūs esate išgelbėti malone per tikėjimą, ir tai ne iš jūsų­tai Dievo dovana, 
\par 9 ir ne dėl darbų, kad kas nors nesigirtų. 
\par 10 Mes esame Jo kūrinys, sukurti Kristuje Jėzuje geriems darbams, kuriuos Dievas iš anksto paskyrė mums atlikti. 
\par 11 Todėl atsiminkite, kad jūs kadaise buvote kūnu pagonys, kuriuos vadino neapipjaustytais vadinamieji apipjaustytieji, apipjaustyti kūne rankomis. 
\par 12 Tuo metu jūs buvote be Kristaus, atskirti nuo Izraelio bendruomenės, svetimi pažado sandoroms, be vilties ir be Dievo pasaulyje. 
\par 13 Bet dabar Kristuje Jėzuje jūs, kadaise buvusieji toli, per Kristaus kraują tapote artimi. 
\par 14 Nes mūsų sutaikinimas yra Jis, iš abejų padaręs viena ir sugriovęs mus skyrusią sieną, 
\par 15 savo kūnu panaikinęs priešybę­ įsakymų Įstatymą su jo potvarkiais,­kad iš dviejų sutvertų savyje naują žmogų ir atneštų taiką. 
\par 16 Jis viename kūne abejus sutaikino su Dievu per kryžių, kuriuo ir sugriovė priešiškumą. 
\par 17 Atėjęs Jis skelbė taiką jums, kurie buvote toli, ir tiems, kurie buvo arti, 
\par 18 nes per Jį vieni ir kiti galime prieiti prie Tėvo viena Dvasia. 
\par 19 Todėl jūs jau nebesate pašaliniai nei svetimšaliai, bet šventųjų bendrapiliečiai ir Dievo namiškiai, 
\par 20 pastatyti ant apaštalų ir pranašų pamato, turintys kertiniu akmeniu patį Jėzų Kristų, 
\par 21 ant kurio darniai auga visas pastatas į šventą šventyklą Viešpatyje, 
\par 22 ant kurio ir jūs esate drauge statomi kaip Dievo buveinė Dvasioje.


\chapter{3}


\par 1 Todėl aš, Paulius, esu Kristaus Jėzaus kalinys dėl jūsų­ pagonių. 
\par 2 Jūs esate girdėję apie Dievo malonės tvarkymą, man suteiktą jūsų labui. 
\par 3 Apreiškimu man buvo atskleista paslaptis, kaip aš ką tik trumpai aprašiau. 
\par 4 Skaitydami galite įsitikinti, kad suvokiu Kristaus paslaptį, 
\par 5 kuri ankstesnėms žmonių kartoms nebuvo paskelbta taip, kaip ji dabar Dvasios atskleista Jo šventiesiems apaštalams ir pranašams: 
\par 6 pagonys yra bendrapaveldėtojai, priklauso vienam kūnui ir yra pažado dalininkai Kristuje per Evangeliją, 
\par 7 kurios tarnu tapau pagal Dievo malonės dovaną, kuri man buvo duota Jo jėgos veikimu. 
\par 8 Man, visų šventųjų mažiausiajam, atiteko malonė skelbti pagonims nesuvokiamus Kristaus turtus 
\par 9 ir atskleisti visiems, kaip turi išsipildyti šita paslaptis, nuo amžių uždengta Dieve­viską sukūrusiame per Jėzų Kristų,­ 
\par 10 kad dabar per bažnyčią taptų žinoma kunigaikštystėms ir valdžioms danguje visokeriopa Dievo išmintis. 
\par 11 Tai atitinka amžinąjį nutarimą, padarytą Kristuje Jėzuje, mūsų Viešpatyje, 
\par 12 kuriame mes turime drąsą ir užtikrintą priėjimą per tikėjimą Juo. 
\par 13 Todėl prašau nenusiminti dėl mano vargų jūsų dėlei, nes jie yra jūsų šlovė. 
\par 14 Dėl to aš klaupiuosi prieš mūsų Viešpaties Jėzaus Kristaus Tėvą, 
\par 15 iš kurio visa šeima danguje ir žemėje turi vardą, 
\par 16 kad iš savo šlovės turtų duotų jums sustiprėti Jo jėga per Dvasią vidiniame žmoguje, 
\par 17 kad Kristus per tikėjimą gyventų jūsų širdyse ir jūs, įsišakniję ir įsitvirtinę meilėje, 
\par 18 galėtumėte suvokti kartu su visais šventaisiais, koks yra plotis, ir ilgis, ir gylis, ir aukštis, 
\par 19 ir pažinti Kristaus meilę, kuri pranoksta pažinimą, kad būtumėte pripildyti visos Dievo pilnatvės. 
\par 20 O Tam, kuris savo jėga, veikiančia mumyse, gali padaryti nepalyginamai daugiau, negu mes prašome ar suprantame, 
\par 21 Jam tebūna šlovė bažnyčioje Kristuje Jėzuje per visas kartas amžių amžiais! Amen.


\chapter{4}


\par 1 Taigi aš, kalinys Viešpatyje, raginu jus elgtis, kaip dera jūsų pašaukimui, į kurį esate pašaukti. 
\par 2 Su visu nuolankumu bei romumu, su ištverme pakęsdami vienas kitą meilėje, 
\par 3 siekite išsaugoti Dvasios vienybę taikos ryšiais. 
\par 4 Vienas kūnas ir viena Dvasia, kaip ir esate pašaukti vienai pašaukimo vilčiai. 
\par 5 Vienas Viešpats, vienas tikėjimas, vienas krikštas. 
\par 6 Vienas Dievas ir visų Tėvas, kuris virš visų, per visus ir visuose. 
\par 7 Bet kiekvienam iš mūsų duota malonė pagal Kristaus dovanos saiką. 
\par 8 Todėl sakoma: “Pakilęs aukštyn, nusivedė belaisvius ir davė žmonėms dovanų”. 
\par 9 Ką reiškia “Jis pakilo”, jeigu ne tai, kad Jis pirma ir nusileido į žemesniąsias žemės vietas. 
\par 10 Tas, kuris nužengė, yra ir Tas, kuris iškilo aukščiau už visus dangus, kad visa užpildytų. 
\par 11 Ir Jis paskyrė vienus apaštalais, kitus pranašais, evangelistais, ganytojais ir mokytojais, 
\par 12 kad išlavintų šventuosius tarnavimo darbui, Kristaus kūno ugdymui, 
\par 13 kol mes visi pasieksime tikėjimo vienybę ir Dievo Sūnaus pažinimą, tobulai subręsime iki Kristaus amžiaus pilnatvės saiko, 
\par 14 kad daugiau nebebūtume kūdikiai, siūbuojami ir nešiojami bet kokio mokymo vėjo, žmonių apgaulės, gudrumo, vedančio į paklydimą, 
\par 15 bet, kalbėdami tiesą meilėje, augtume visame kame į Jį, kuris yra galva­Kristus. 
\par 16 Iš Jo visas kūnas, suderintas ir stipriai sujungtas įvairių raiščių, pagal savo saiką veikiant kiekvienai daliai, auga, kad ugdytų save meilėje. 
\par 17 Taigi aš liepiu ir įspėju Viešpatyje, kad jūs nebesielgtumėte, kaip elgiasi pagonys dėl savo proto tuštybės. 
\par 18 Jų protas aptemęs, jie atskirti nuo Dievo gyvenimo dėl savo neišmanymo bei širdies užkietėjimo. 
\par 19 Jie sustabarėję, pasidavę gašlumui, nepasotinamai daro visus nešvarius darbus. 
\par 20 Bet jūs ne taip pažinote Kristų! 
\par 21 Juk iš Jo girdėjote ir Jame išmokote,­nes tiesa yra Jėzuje,­ 
\par 22 kad privalu atsižadėti senojo žmogaus ankstesnio gyvenimo būdo, žlugdančio apgaulingais geismais, 
\par 23 atsinaujinti savo proto dvasioje 
\par 24 ir apsirengti nauju žmogumi, sutvertu pagal Dievą teisume ir tiesos šventume. 
\par 25 Tad, atmetę melą, “kiekvienas tekalba tiesą savo artimui”, nes esame vieni kitų nariai. 
\par 26 “Rūstaukite ir nenusidėkite”. Tegul saulė nenusileidžia ant jūsų rūstybės! 
\par 27 Ir neduokite vietos velniui. 
\par 28 Kas vogdavo, tegu daugiau nebevagia, bet dirba, darydamas savo rankomis gerus darbus, kad turėtų iš ko padėti stokojančiam. 
\par 29 Joks bjaurus žodis teneišeina iš jūsų lūpų; bet tik tai, kas gera, kas tinka ugdymui ir suteikia malonę klausytojams. 
\par 30 Ir neliūdinkite Šventosios Dievo Dvasios, kuria esate užantspauduoti atpirkimo dienai. 
\par 31 Tebūna toli nuo jūsų visoks kartėlis, piktumas, rūstybė, riksmai ir keiksmai su visomis piktybėmis. 
\par 32 Būkite malonūs, gailestingi, atlaidūs vieni kitiems, kaip ir Dievas Kristuje atleido jums.


\chapter{5}


\par 1 Taigi būkite Dievo sekėjai, kaip mylimi vaikai, 
\par 2 ir gyvenkite mylėdami, kaip ir Kristus pamilo mus ir atidavė už mus save kaip atnašą ir kvapią auką Dievui. 
\par 3 Todėl ištvirkavimas, visoks netyrumas ar godumas tenebūna net minimi pas jus, kaip pridera šventiesiems; 
\par 4 taip pat nešvankumas, kvaila šneka ar juokų krėtimas jums netinka, verčiau tebūna dėkojimas. 
\par 5 Nes jūs žinote, kad joks ištvirkėlis, netyras ar gobšas, kuris yra stabmeldys, nepaveldės Kristaus ir Dievo karalystės. 
\par 6 Tegul niekas neapgauna jūsų tuščiais plepalais; už tokius dalykus Dievo rūstybė ištinka neklusnumo vaikus. 
\par 7 Todėl nebūkite jų bendrai! 
\par 8 Juk kadaise buvote tamsa, o dabar esate šviesa Viešpatyje. Elkitės kaip šviesos vaikai,­ 
\par 9 nes Dvasios vaisius reiškiasi visokeriopu gerumu, teisumu ir tiesa,­ 
\par 10 ištirdami, kas patinka Viešpačiui. 
\par 11 Ir neprisidėkite prie nevaisingų tamsos darbų, o verčiau atskleiskite juos. 
\par 12 Nes ką jie slapčia daro, gėda net sakyti. 
\par 13 Bet viskas, kas atskleidžiama, tampa šviesos apšviesta, o kas tik apšviesta, yra šviesa. 
\par 14 Todėl sakoma: “Pabusk, kuris miegi, kelkis iš numirusių, ir apšvies tave Kristus”. 
\par 15 Todėl rūpestingai žiūrėkite, kaip elgiatės: kad nebūtumėte kaip kvailiai, bet kaip išmintingi, 
\par 16 branginantys laiką, nes dienos yra piktos. 
\par 17 Nebūkite tad neprotingi, bet supraskite, kokia yra Viešpaties valia. 
\par 18 Ir nepasigerkite vynu, kuriame pasileidimas, bet būkite pilni Dvasios, 
\par 19 kalbėdami vieni kitiems psalmėmis, himnais bei dvasinėmis giesmėmis, giedodami ir šlovindami savo širdyse Viešpatį, 
\par 20 visada ir už viską dėkodami Dievui Tėvui mūsų Viešpaties Jėzaus Kristaus vardu, 
\par 21 paklusdami vieni kitiems Dievo baimėje. 
\par 22 Jūs, žmonos, būkite klusnios savo vyrams lyg Viešpačiui, 
\par 23 nes vyras yra žmonos galva, kaip ir Kristus yra galva bažnyčios,­Jis kūno gelbėtojas. 
\par 24 Todėl kaip bažnyčia paklūsta Kristui, taip ir žmonos visame kame teklauso savo vyrų. 
\par 25 Jūs, vyrai, mylėkite savo žmonas, kaip ir Kristus pamilo bažnyčią ir atidavė už ją save, 
\par 26 kad ją pašventintų, apvalydamas vandens nuplovimu ir žodžiu, 
\par 27 kad pristatytų sau šlovingą bažnyčią, neturinčią dėmės nei raukšlės, nei nieko tokio, bet šventą ir nesuteptą. 
\par 28 Taip ir vyrai turi mylėti savo žmonas kaip savo kūnus. Kas myli savo žmoną, myli save patį. 
\par 29 Juk niekas niekada nėra nekentęs savo kūno, bet jį maitina ir globoja kaip ir Kristus bažnyčią. 
\par 30 Mes gi esame Jo kūno nariai, iš Jo kūno ir kaulų. 
\par 31 “Todėl žmogus paliks tėvą bei motiną ir susijungs su savo žmona, ir du taps vienu kūnu”. 
\par 32 Tai didelė paslaptis,­aš tai sakau, žvelgdamas į Kristų ir bažnyčią. 
\par 33 Taigi kiekvienas iš jūsų tegul myli savo žmoną taip, kaip save patį, o žmona tegerbia savo vyrą.


\chapter{6}


\par 1 Jūs, vaikai, klausykite Viešpatyje savo tėvų, nes tai teisinga. 
\par 2 “Gerbk savo tėvą ir motiną”,­ tai pirmasis įsakymas su pažadu: 
\par 3 “Kad tau gerai sektųsi ir ilgai gyventum žemėje”. 
\par 4 Ir jūs, tėvai, neerzinkite savo vaikų, bet auklėkite juos drausmindami ir mokydami Viešpatyje. 
\par 5 Jūs, vergai, klausykite savo žemiškųjų šeimininkų su baime ir drebėjimu, nuoširdžiai kaip Kristaus, 
\par 6 ne dėl akių tarnaudami, lyg žmonėms įtinkantys, bet kaip Kristaus vergai, iš širdies vykdantys Dievo valią. 
\par 7 Noriai tarnaukite kaip Viešpačiui, o ne žmonėms, 
\par 8 žinodami, jog kiekvienas, tiek vergas, tiek laisvasis, jei daro ką gera, gaus atlyginimą iš Viešpaties. 
\par 9 Ir jūs, šeimininkai, tą patį darykite jiems. Liaukitės grasinę, žinodami, kad ir jiems, ir jums yra Viešpats danguje ir Jis nedaro skirtumo tarp asmenų. 
\par 10 Pagaliau, mano broliai, būkite stiprūs Viešpatyje ir Jo galybės jėga. 
\par 11 Apsirenkite visa Dievo ginkluote, kad galėtumėte pasipriešinti prieš velnio klastas. 
\par 12 Nes mes grumiamės ne su kūnu ir krauju, bet su kunigaikštystėmis, valdžiomis, šio amžiaus tamsybių valdovais ir dvasinėmis blogio jėgomis danguje. 
\par 13 Todėl imkitės visų Dievo ginklų, kad galėtumėte piktą dieną pasipriešinti ir, visa atlaikę, išstovėti. 
\par 14 Tad stovėkite susijuosę savo strėnas tiesa, apsivilkę teisumo šarvais 
\par 15 ir apsiavę kojas pasiruošimu skelbti taikos Evangeliją. 
\par 16 O svarbiausia, pasiimkite tikėjimo skydą, su kuriuo užgesinsite visas liepsnojančias piktojo strėles. 
\par 17 Pasiimkite ir išgelbėjimo šalmą bei Dvasios kalaviją, tai yra Dievo žodį, 
\par 18 visada melsdamiesi Dvasioje visokeriopomis maldomis ir prašymu. Ištvermingai budėkite, malda užtardami visus šventuosius 
\par 19 ir mane, kad, man atvėrus lūpas, būtų duotas žodis drąsiai atskleisti Evangelijos paslaptį, 
\par 20 kurios pasiuntinys esu ir būdamas kalinys,­kad turėčiau drąsos kalbėti taip, kaip privalau kalbėti. 
\par 21 Kad ir jūs sužinotumėte, kaip man einasi ir ką veikiu, visa papasakos jums Tichikas, mylimas brolis ir ištikimas tarnas Viešpatyje. 
\par 22 Aš tam jį ir pasiunčiau, kad jūs sužinotumėte apie mus ir kad jis paguostų jūsų širdis. 
\par 23 Broliams ramybė ir meilė su tikėjimu nuo Dievo Tėvo ir Viešpaties Jėzaus Kristaus. 
\par 24 Malonė visiems, kurie nuoširdžiai myli mūsų Viešpatį Jėzų Kristų! Amen.


\end{document}