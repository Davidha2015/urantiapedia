\begin{document}

\title{Laiškas filipiečiams}

\chapter{1}


\par 1 Paulius ir Timotiejus, Jėzaus Kristaus tarnai, visiems šventiesiems Kristuje Jėzuje, gyvenantiems Filipuose, kartu su vyskupais ir diakonais. 
\par 2 Malonė jums ir ramybė nuo mūsų Dievo Tėvo ir Viešpaties Jėzaus Kristaus! 
\par 3 Aš dėkoju savo Dievui, kada tik jus prisimenu, 
\par 4 visada kiekvienoje savo maldoje su džiaugsmu už jus visus besimelsdamas, 
\par 5 už jūsų dalyvavimą skelbiant Evangeliją nuo pirmosios dienos iki šiandien. 
\par 6 Esu tikras, kad Tas, kuris jumyse pradėjo gerą darbą, jį ir pabaigs iki Jėzaus Kristaus dienos. 
\par 7 Šitaip manyti apie jus visus teisinga, nes turiu savo širdyje jus, kurie, tiek man esant surakintam, tiek ginant ir įtvirtinant Evangeliją, visi tebesate mano malonės dalininkai. 
\par 8 Dievas man liudytojas, kaip aš jūsų visų pasiilgau Kristaus Jėzaus nuoširdumu. 
\par 9 Ir meldžiu, kad jūsų meilė vis augtų ir augtų pažinimu ir visokeriopu įžvalgumu, 
\par 10 kad jūs mokėtumėte pasirinkti, kas tobuliau, kad būtumėte tyri ir be priekaišto iki Kristaus dienos, 
\par 11 pilni teisumo vaisių per Jėzų Kristų Dievo šlovei ir gyriui. 
\par 12 Broliai, aš noriu, kad jūs žinotumėte, jog mano būklė pasitarnavo Evangelijos plitimui. 
\par 13 Mat mano kalinimas dėl Kristaus išgarsėjo visame pretorijuje ir tarp visų kitų, 
\par 14 ir daugumas brolių Viešpatyje, mano pančių paakinti, imasi daug drąsiau, be baimės skelbti žodį. 
\par 15 Beje, kai kurie skelbia Kristų iš pavydo ir rungtyniaudami, kiti gera valia; 
\par 16 anie skelbia Kristų varžydamiesi, ne iš gryno nusistatymo, tardamiesi pasunkinsią mano pančius, 
\par 17 o šitie iš meilės, suprasdami, kad aš paskirtas ginti Evangelijos. 
\par 18 Tesižinai! Kad visokiais būdais, apsimetant ar iš tikrųjų, yra skelbiamas Kristus,­štai kuo džiaugiuosi ir toliau džiaugsiuos! 
\par 19 Nes aš žinau, kad tai pasitarnaus mano išlaisvinimui dėl jūsų maldos ir Jėzaus Kristaus Dvasios pagalbos. 
\par 20 Aš karštai laukiu ir turiu viltį, jog niekuo neliksiu sugėdintas, bet kaip visada, taip ir dabar Kristus bus viešai išaukštintas mano kūne­ar gyvenimu, ar mirtimi. 
\par 21 Man gyvenimas­tai Kristus, o mirtis­tai laimėjimas. 
\par 22 Bet jei aš, gyvendamas kūne, dar galiu vaisingai pasidarbuoti, tuomet nebežinau, ką pasirinkti. 
\par 23 Mane traukia ir viena, ir kita, nors verčiau man iškeliauti ir būti su Kristumi, nes tai visų geriausia. 
\par 24 O mano pasilikimas kūne reikalingesnis jums. 
\par 25 Taip įsitikinęs, aš žinau, jog liksiu ir būsiu su jumis visais jūsų pažangai ir tikėjimo džiaugsmui, 
\par 26 kad jūs dar labiau galėtumėte pasigirti manimi Jėzuje Kristuje, kai aš vėl atvyksiu pas jus. 
\par 27 Tiktai jūsų elgesys tebūna vertas Kristaus Evangelijos, kad atvykęs matyčiau, o jei neatvyksiu­išgirsčiau, kad gyvenate vienoje dvasioje, viena siela kartu kovojate už Evangelijos tikėjimą 
\par 28 ir niekuo nesiduodate priešininkų išgąsdinami. Jiems tai žlugimo ženklas, o jums­išgelbėjimo, ir jis Dievo duotas. 
\par 29 Nes jums duota dėl Kristaus ne tik Jį tikėti, bet ir dėl Jo kentėti, 
\par 30 kovojant tokią pat kovą, kokią matėte mane kovojant ir apie kokią dabar girdite, jog aš kovoju.


\chapter{2}


\par 1 Taigi, jeigu esama Kristuje kokio padrąsinimo, meilės paguodos, jei esama kokio Dvasios bendravimo, nuoširdumo ir gailestingumo, 
\par 2 tai padarykite mano džiaugsmą tobulą, laikydamiesi vienos minties, turėdami vienokią meilę, būdami vieningi ir to paties nusistatymo. 
\par 3 Nedarykite nieko varžydamiesi ar iš tuščios puikybės, bet nuolankiai vienas kitą laikykite aukštesniu už save 
\par 4 ir žiūrėkite kiekvienas ne savo naudos, bet kitų. 
\par 5 Būkite tokio nusistatymo kaip Kristus Jėzus, 
\par 6 kuris, esybe būdamas Dievas, nesilaikė pasiglemžęs savo lygybės su Dievu, 
\par 7 bet apiplėšė save ir esybe tapo tarnu ir panašus į žmones. 
\par 8 Ir išore tapęs kaip žmogus, Jis nusižemino, tapdamas paklusnus iki mirties, iki kryžiaus mirties. 
\par 9 Todėl Dievas Jį labai išaukštino ir suteikė Jam vardą aukščiau visų kitų vardų, 
\par 10 kad Jėzaus vardui priklauptų kiekvienas kelis danguje, žemėje ir po žeme 
\par 11 ir kiekvienos lūpos Dievo Tėvo šlovei išpažintų, kad Jėzus Kristus yra Viešpats. 
\par 12 Taigi, mano mylimieji, kaip visada paklusdavote, kai būdavau tarp jūsų, tai juo labiau dabar, man nesant tarp jūsų,—atbaikite savo išgelbėjimą su baime ir drebėdami, 
\par 13 nes tai Dievas, veikiantis jumyse, suteikia ir troškimą, ir darbą iš savo palankumo. 
\par 14 Visa darykite be murmėjimų ir svyravimų, 
\par 15 kad būtumėte nepeiktini, nekalti ir nesutepti Dievo vaikai sugedusioje ir iškrypusioje žmonių kartoje, kur jūs spindite tarsi žiburiai pasaulyje. 
\par 16 Tvirtai laikykitės gyvenimo žodžio, kad Kristaus dieną galėčiau pasigirti ne veltui bėgęs ir ne veltui dirbęs. 
\par 17 O jei aš atnašaujamas ant jūsų tikėjimo aukos ir tarnavimo, esu linksmas ir džiaugiuosi kartu su jumis visais. 
\par 18 Taip pat ir jūs būkite linksmi ir džiaukitės kartu su manimi. 
\par 19 Aš turiu Viešpatyje Jėzuje viltį netrukus pasiųsti pas jus Timotiejų, kad būčiau paguostas, sužinojęs, kaip jums sekasi. 
\par 20 Mat aš neturiu nė vieno kito tokio, kuris taip nuoširdžiai jumis rūpintųsi. 
\par 21 Visi kiti ieško ne Jėzaus Kristaus, bet savo naudos. 
\par 22 O apie jo ištikimybę jūs žinote, nes skelbiant Evangeliją jis man tarnavo kaip sūnus tėvui. 
\par 23 Taigi, kai tik paaiškės mano byla, viliuosi tučtuojau jį pasiųsti. 
\par 24 Be to, turiu Viešpatyje viltį ir pats netrukus atvykti pas jus. 
\par 25 Aš dar nusprendžiau siųsti pas jus Epafroditą, mano brolį, bendradarbį ir kovų draugą, o jūsų pasiuntinį ir pagalbininką mano reikmėse. 
\par 26 Jis labai jūsų išsiilgo ir sielojosi, kad jūs išgirdote apie jo ligą. 
\par 27 Tikrai jis sirgo ir buvo arti mirties, tačiau Dievas jo pasigailėjo, ir ne vien tik jo, bet ir manęs, kad manęs neužgriūtų sielvartas po sielvarto. 
\par 28 Taigi aš jį skubiai siunčiu, kad, jį pamatę, pradžiugtumėte ir aš taip pat neliūdėčiau. 
\par 29 Tad priimkite jį Viešpatyje su tikru džiaugsmu ir gerbkite tokius žmones, 
\par 30 nes dėl Kristaus darbo jis buvo atsidūręs prie mirties, nebrangindamas savo gyvybės, kad užpildytų spragą jūsų patarnavime man.


\chapter{3}


\par 1 Pagaliau, mano broliai, džiaukitės Viešpatyje! Man rašyti vis tą patį nesunku, o jums tai pastiprinimas. 
\par 2 Saugokitės šunų, saugokitės piktų darbininkų, saugokitės apsipjaustėlių! 
\par 3 Nes apipjaustymas esame mes, kurie garbinam Dievą Dvasia, giriamės Kristumi Jėzumi ir pasitikime ne kūnu. 
\par 4 Nors aš galėčiau pasitikėti ir kūnu! Jei kas nors mano galįs pasitikėti kūnu, tai aš juo labiau; 
\par 5 aštuntą dieną apipjaustytas, iš Izraelio tautos, Benjamino giminės, žydas iš žydų, įstatymu­fariziejus, 
\par 6 uolumu­bažnyčios persekiotojas, įstatymo teisumu­nepeiktinas. 
\par 7 Bet tai, kas man buvo laimėjimas, dėl Kristaus palaikiau nuostoliu. 
\par 8 O taip! Aš iš tikrųjų visa laikau nuostoliu dėl Kristaus Jėzaus, mano Viešpaties, pažinimo didybės. Dėl Jo aš praradau viską ir viską laikau sąšlavomis, kad laimėčiau Kristų 
\par 9 ir būčiau atrastas Jame, nebeturėdamas savo teisumo iš įstatymo, bet turėdamas teisumą per tikėjimą Kristumi, teisumą iš Dievo, paremtą tikėjimu, 
\par 10 kad pažinčiau Jį, Jo prisikėlimo jėgą ir bendravimą Jo kentėjimuose, suaugčiau su Jo mirtimi, 
\par 11 kad pasiekčiau prisikėlimą iš numirusių. 
\par 12 Nesakau, kad jau esu šitai gavęs ar tapęs tobulas, bet vejuosi, norėdamas pagauti, nes jau esu Kristaus Jėzaus pagautas. 
\par 13 Broliai, aš nemanau, kad jau būčiau tai pasiekęs. Tik viena tikra: pamiršdamas, kas už manęs, ir siekdamas to, kas priešakyje, 
\par 14 veržiuosi į tikslą aukštybėse, siekiu apdovanojimo už Dievo pašaukimą Kristuje Jėzuje. 
\par 15 Taigi visi, kurie esame subrendę, taip mąstykime. O jeigu jūs apie ką nors manote kitaip, tai Dievas jums ir tai apreikš. 
\par 16 Kiek bebūtumėme pasiekę, vadovaukimės ta pačia taisykle ir taip mąstykime. 
\par 17 Broliai, būkite mano sekėjai ir žiūrėkite į tuos, kurie elgiasi pagal pavyzdį, kurį matote mumyse. 
\par 18 Daugelis­apie juos ne kartą esu jums kalbėjęs ir dabar net su ašaromis kalbu­elgiasi kaip Kristaus kryžiaus priešai. 
\par 19 Jų galas­pražūtis, jų dievas­ pilvas ir jų garbė—gėda. Jie temąsto apie žemiškus dalykus. 
\par 20 Tuo tarpu mūsų tėvynė danguje, ir iš ten mes karštai laukiame Gelbėtojo, Viešpaties Jėzaus Kristaus, 
\par 21 kuris pakeis mūsų gėdingą kūną ir padarys jį panašų į savo šlovingą kūną ta jėga, kuria Jis visa palenkia sau.


\chapter{4}


\par 1 Taigi, mano broliai, mano mylimieji ir išsiilgtieji, mano džiaugsme ir mano vainike,­tvirtai stovėkite Viešpatyje, mylimieji! 
\par 2 Aš raginu Evodiją ir raginu Sintichę būti vienos minties Viešpatyje. 
\par 3 Taip pat raginu tave, tikrasis bendradarbi, padėk toms moterims, kurios darbavosi su manimi Evangelijos labui kartu su Klemensu ir kitais mano bendradarbiais, kurių vardai gyvenimo knygoje. 
\par 4 Džiaukitės Viešpatyje visuomet! Ir vėl kartoju: džiaukitės! 
\par 5 Jūsų romumas tebūna žinomas visiems. Viešpats yra arti! 
\par 6 Niekuo nesirūpinkite, bet visuose reikaluose malda ir prašymu su padėka jūsų troškimai tesidaro žinomi Dievui. 
\par 7 Ir Dievo ramybė, pranokstanti visokį supratimą, saugos jūsų širdis ir mintis Kristuje Jėzuje. 
\par 8 Pagaliau, broliai, mąstykite apie tai, kas tikra, garbinga, teisinga, tyra, mylima, giriama,­apie visa, kas dora ir šlovinga. 
\par 9 Darykite, ką tik iš manęs išmokote, ką gavote, ką girdėjote ir matėte manyje, ir ramybės Dievas bus su jumis. 
\par 10 Labai nudžiugau Viešpatyje, kad pagaliau vėl pražydo jūsų rūpinimasis manimi. Jūs ir seniau rūpindavotės, bet stigo progų tai parodyti. 
\par 11 Kalbu tai ne todėl, kad stokoju, nes išmokau būti patenkintas savo būkle. 
\par 12 Esu patyręs ir skurdą, ir perteklių. Visa ko esu ragavęs: buvau sotus ir alkanas, turtingas ir beturtis. 
\par 13 Aš visa galiu Kristuje, kuris mane stiprina. 
\par 14 Vis dėlto jūs gerai padarėte, dalyvaudami mano varge. 
\par 15 Jūs, filipiečiai, taip pat žinote: kai, pradėjęs skelbti Evangeliją, išvykau iš Makedonijos, jokia bažnyčia neužmezgė su manimi davimo ir gavimo santykių, tik jūs vieni. 
\par 16 Jūs mano reikalams pasiuntėte aukų vieną ir kitą kartą į Tesaloniką. 
\par 17 Aš netrokštu dovanos, bet trokštu vaisiaus, kuris augtų jūsų sąskaiton. 
\par 18 Aš esu visko gavęs ir turiu su pertekliumi. Esu visiškai aprūpintas, per Epafroditą gavęs iš jūsų kvapią dovaną, Dievui priimtiną ir patinkančią auką. 
\par 19 O mano Dievas patenkins visas jūsų reikmes iš savo šlovės turtų Kristuje Jėzuje. 
\par 20 Mūsų Dievui ir Tėvui šlovė per amžių amžius! Amen. 
\par 21 Sveikinkite kiekvieną šventąjį Kristuje Jėzuje. Jus sveikina su manimi esantys broliai. 
\par 22 Jus sveikina visi šventieji, o ypač iš ciesoriaus namiškių. 
\par 23 Viešpaties Jėzaus Kristaus malonė tebūna su jumis visais! Amen.


\end{document}