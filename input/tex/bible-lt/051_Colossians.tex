\begin{document}

\title{Laiškas kolosiečiams}

\chapter{1}


\par 1 Paulius, Dievo valia Kristaus Jėzaus apaštalas, ir brolis Timotiejus,­ 
\par 2 šventiesiems ir ištikimiesiems broliams Kristuje, gyvenantiems Kolosuose: malonė ir ramybė jums nuo Dievo, mūsų Tėvo, ir mūsų Viešpaties Jėzaus Kristaus! 
\par 3 Nuolat melsdamiesi už jus, dėkojame Dievui, mūsų Viešpaties Jėzaus Kristaus Tėvui, 
\par 4 nes girdime apie jūsų tikėjimą Kristumi Jėzumi ir jūsų meilę visiems šventiesiems 
\par 5 dėl vilties, kuri jums paruošta danguje. Apie ją jūs esate girdėję Evangelijos tiesos žodyje, 
\par 6 kuris pasiekė jus ir panašiai kaip visame pasaulyje, taip ir pas jus neša vaisių ir auga nuo tos dienos, kada išgirdote ir pažinote Dievo malonę tiesoje. 
\par 7 To jūs išmokote iš mūsų mylimojo bendradarbio Epafro, kuris jums yra ištikimas Kristaus tarnas. 
\par 8 Jis ir davė žinią mums apie jūsų meilę Dvasioje. 
\par 9 Todėl ir mes nuo tos dienos, kada tai išgirdome, nesiliaujame už jus meldę ir prašę, kad jūs būtumėte pilni Dievo valios pažinimo su visa išmintimi ir dvasiniu supratimu, 
\par 10 kad elgtumėtės, kaip verta Viešpaties, ir Jam tobulai patiktumėte, nešdami vaisių gerais darbais ir augdami Dievo pažinimu; 
\par 11 kad, sustiprinti visokeriopa jėga iš Jo šlovės galios didžiai kantrybei ir ištvermei, su džiaugsmu 
\par 12 dėkotumėte Tėvui, kuris padarė mus tinkamus paveldėti šventųjų dalį šviesoje, 
\par 13 kuris išlaisvino mus iš tamsybių valdžios ir perkėlė į savo mylimojo Sūnaus karalystę. 
\par 14 Jame mes turime atpirkimą Jo krauju ir nuodėmių atleidimą. 
\par 15 Jis yra neregimojo Dievo atvaizdas, visos kūrinijos pirmagimis, 
\par 16 nes Juo sutverta visa, kas yra danguje ir žemėje, kas regima ir neregima; ar sostai, ar viešpatystės, ar kunigaikštystės, ar valdžios,­visa sutverta per Jį ir Jam. 
\par 17 Jis yra pirma visų daiktų, ir visa Juo laikosi. 
\par 18 Ir Jis yra kūno­bažnyčios galva. Jis­pradžia, pirmagimis iš mirusiųjų, kad visame kame turėtų pirmenybę. 
\par 19 Nes Tėvui patiko Jame apgyvendinti visą pilnatvę 
\par 20 ir per Jį visa sutaikinti su savimi, darant Jo kryžiaus krauju taiką,­per Jį sutaikinti visa, kas yra žemėje ir danguje. 
\par 21 Taip pat ir jus, kurie kadaise buvote svetimi ir priešiški savo protuose piktais darbais, Dievas dabar sutaikino 
\par 22 mirtimi Jo žemiškajame kūne, kad pasirodytumėte Jo akyse šventi, tyri ir nekalti, 
\par 23 jei tik pasiliekate tikėjime įsitvirtinę bei nepajudinami ir neatsitraukiate nuo Evangelijos vilties, kurią išgirdote, kuri paskelbta visai kūrinijai po dangumi ir kurios tarnu aš, Paulius, tapau. 
\par 24 Dabar aš džiaugiuosi savo kentėjimais už jus ir savo kūne papildau, ko dar trūksta Kristaus vargams dėl Jo kūno­bažnyčios. 
\par 25 Jos tarnu aš tapau Dievo patvarkymu, kuris duotas man dėl jūsų, kad išpildyčiau Dievo žodį, 
\par 26 tą paslaptį, kuri buvo paslėpta amžiams ir kartoms, o dabar apreikšta Jo šventiesiems. 
\par 27 Jiems Dievas panorėjo atskleisti, kokie šios paslapties šlovės turtai skirti pagonims, būtent Kristus jumyse­šlovės viltis. 
\par 28 Mes Jį skelbiame, įspėdami kiekvieną žmogų ir mokydami kiekvieną žmogų su visokeriopa išmintimi, kad kiekvieną žmogų padarytume tobulą Kristuje. 
\par 29 Dėl to aš ir darbuojuos, grumdamasis Jo suteikta jėga, kuri galingai veikia manyje.


\chapter{2}


\par 1 Tad noriu, kad jūs žinotumėte, kaip įnirtingai kovoju už jus, už laodikiečius ir visus, kurie nėra matę mano kūno veido, 
\par 2 kad būtų paguostos visų širdys, kad, meile suvienyti, visi pasiektų pažinimo pilnatvės turtus ir pažintų paslaptį Dievo—Tėvo ir Kristaus, 
\par 3 kuriame slypi visi išminties ir pažinimo turtai. 
\par 4 Jums tai sakau, kad niekas jūsų nesuvedžiotų įtikinančia kalba. 
\par 5 Nors kūnu esu toli nuo jūsų, tačiau dvasia su jumis ir džiaugiuosi, matydamas jūsų tvarką ir jūsų tikėjimo Kristumi tvirtumą. 
\par 6 Taigi, kaip esate priėmę Viešpatį Jėzų Kristų, taip ir gyvenkite Jame, 
\par 7 įsišakniję bei statydindamiesi Jame ir įsitvirtinę tikėjime, kaip esate išmokyti, kupini dėkingumo. 
\par 8 Žiūrėkite, kad kas jūsų nepavergtų filosofija ir tuščia apgaule, kuri remiasi žmonių tradicijomis ir pasaulio pradmenimis, o ne Kristumi. 
\par 9 Jame kūniškai gyvena visa Dievybės pilnatvė, 
\par 10 ir jūs esate tobuli Jame, kuris yra kiekvienos kunigaikštystės ir valdžios galva. 
\par 11 Jame jūs taip pat esate apipjaustyti ne rankomis atliktu apipjaustymu, bet kūno nuodėmių, kūniškumo nusirengimu­Kristaus apipjaustymu. 
\par 12 Su Juo palaidoti krikšte, kuriame jūs buvote ir prikelti, tikėdami jėga Dievo, prikėlusio Jį iš numirusių. 
\par 13 Ir jus, mirusius nusikaltimais ir jūsų kūno neapipjaustymu, Jis atgaivino kartu su Juo, atleisdamas visus nusikaltimus. 
\par 14 Jis ištrynė skolos raštą su mus kaltinančiais reikalavimais, raštą, kuris buvo prieš mus, ir panaikino jį, prikaldamas prie kryžiaus. 
\par 15 Jis nuginklavo kunigaikštystes bei valdžias ir viešai jas pažemino, triumfuodamas prieš jas ant kryžiaus. 
\par 16 Taigi niekas tenesmerkia jūsų dėl valgio ar gėrimo, dėl švenčių, jauno mėnulio ar sabato dienų. 
\par 17 Visa tai tėra būsimųjų dalykų šešėlis, o kūnas yra Kristaus. 
\par 18 Niekas teneatima jūsų atlygio, pamėgęs tariamą nusižeminimą ir angelų garbinimą, pasinėręs į tai, ko nėra matęs, be pagrindo pasipūtęs savo kūniškais samprotavimais, 
\par 19 nesijungdamas su Galva, iš kurios visas kūnas, sąnariais ir raiščiais aprūpinamas bei jungiamas vienybėn, auga Dievo teikiamu ūgiu. 
\par 20 Jei su Kristumi mirėte pasaulio pradmenims, tai kodėl gi, tarsi tebegyvendami pasaulyje, pasiduodate nuostatoms 
\par 21 (“Neliesk! Neragauk! Neimk!”­ 
\par 22 visa tai vartojama dingsta.) pagal žmonių priesakus bei doktrinas? 
\par 23 Tiesa, tai atrodo išmintingai dėl susikurto pamaldumo, tariamo nusižeminimo ir kūno varginimo, tačiau neturi jokios vertės ir pasotina kūniškumą.


\chapter{3}


\par 1 Jeigu esate su Kristumi prikelti, siekite to, kas aukštybėse, kur Kristus sėdi Dievo dešinėje. 
\par 2 Mąstykite apie tai, kas aukštybėse, o ne apie tai, kas žemėje. 
\par 3 Jūs juk esate mirę, ir jūsų gyvenimas su Kristumi yra paslėptas Dieve. 
\par 4 Kai pasirodys Kristus­mūsų gyvenimas, tada su Juo ir jūs pasirodysite šlovėje. 
\par 5 Todėl marinkite tuos savo narius, kurie yra žemėje: ištvirkavimą, netyrumą, aistringumą, piktą pageidimą, taip pat godumą, kuris yra stabmeldystė. 
\par 6 Dėl šių dalykų ateina Dievo rūstybė neklusnumo vaikams. 
\par 7 Jūs irgi kadaise taip elgėtės, gyvendami tarp jų. 
\par 8 Bet dabar jūs visa tai nusivilkite: rūstybę, nirtulį, nelabumą, pyktį, piktžodžiavimą, nešvarias kalbas nuo savo lūpų. 
\par 9 Nebemeluokite vienas kitam, nusivilkę senąjį žmogų su jo darbais 
\par 10 ir apsivilkę nauju, kuris atnaujinamas pažinimu pagal atvaizdą To, kuris jį sukūrė. 
\par 11 Čia jau nebėra nei graiko, nei žydo, nei apipjaustyto, nei neapipjaustyto, nei barbaro, nei skito, nei vergo, nei laisvojo, bet visa ir visuose­Kristus. 
\par 12 Todėl, kaip Dievo išrinktieji, šventieji ir numylėtiniai, apsivilkite nuoširdžiu gailestingumu, gerumu, nuolankumu, romumu ir ištverme. 
\par 13 Būkite vieni kitiems pakantūs ir atleiskite vieni kitiems, jei vienas prieš kitą turite skundą. Kaip Kristus atleido, taip ir jūs atleiskite. 
\par 14 O virš viso šito apsivilkite meile, kuri yra tobulumo raištis. 
\par 15 Jūsų širdyse teviešpatauja Dievo ramybė, į kurią esate pašaukti viename kūne. Ir būkite dėkingi. 
\par 16 Kristaus žodis tegul tarpsta jumyse. Mokykite ir įspėkite vieni kitus visokeriopa išmintimi, su dėkinga širdimi giedokite Viešpačiui psalmes, himnus ir dvasines giesmes. 
\par 17 Ir visa, ką bedarytumėte žodžiu ar darbu, visa darykite Viešpaties Jėzaus vardu, per Jį dėkodami Dievui Tėvui. 
\par 18 Jūs, žmonos, būkite atsidavusios savo vyrams, kaip dera Viešpatyje. 
\par 19 O jūs, vyrai, mylėkite savo žmonas ir nebūkite joms šiurkštūs. 
\par 20 Jūs, vaikai, visuose dalykuose klausykite savo tėvų, nes tai patinka Viešpačiui. 
\par 21 O jūs, tėvai, neerzinkite savo vaikų, kad jie nepasidarytų baukštūs. 
\par 22 Jūs, vergai, visame kame pakluskite savo šeimininkams pagal kūną, ne dėl akių tarnaudami, kaip žmonėms patikti norėdami, bet iš tyros širdies, bijodami Dievo. 
\par 23 Ir ką tik darytumėte, darykite iš širdies, kaip Viešpačiui, o ne žmonėms, 
\par 24 žinodami, kad iš Viešpaties gausite palikimą kaip atlyginimą,­ nes jūs tarnaujate Viešpačiui Kristui. 
\par 25 O kas daro neteisybę, susilauks atlygio už tai, ką padarė, ir nebus žiūrima asmens.


\chapter{4}


\par 1 O jūs, šeimininkai, duokite vergams, kas teisinga ir kas dera, atsimindami, kad ir jūs turite Šeimininką danguje. 
\par 2 Nuolat melskitės, budėdami ir dėkodami; 
\par 3 melskitės taip pat ir už mus, kad Dievas mums atvertų žodžio duris skelbti Kristaus paslaptį, dėl kurios aš surakintas, 
\par 4 kad sugebėčiau ją atskleisti taip, kaip privalau ją skelbti. 
\par 5 Elkitės protingai su pašaliniais, išnaudodami laiką. 
\par 6 Jūsų kalba visuomet tebūna maloni ir druska pasūdyta, kad sugebėtumėte kiekvienam atsakyti. 
\par 7 Apie mano reikalus jums viską praneš Tichikas, mylimas brolis, ištikimas tarnas ir bendradarbis Viešpatyje. 
\par 8 Aš tam jį ir siunčiu, kad sužinotų, kaip jums sekasi, ir paguostų jūsų širdis. 
\par 9 Jis su Onesimu, ištikimu ir mylimu broliu, kuris yra iš jūsų, papasakos jums visa, kas čia dedasi. 
\par 10 Jus sveikina mano kalėjimo draugas Aristarchas ir Barnabo pusbrolis Morkus­dėl jo jau gavote nurodymų; jei jis atvyks pas jus, priimkite jį. 
\par 11 Dar jus sveikina Jėzus, vadinamas Justu. Iš apipjaustytųjų jie yra vieninteliai mano bendradarbiai dėl Dievo karalystės, tapę mano paguoda. 
\par 12 Jus sveikina jūsiškis Epafras, Kristaus tarnas, kuris visada grumiasi už jus maldose, kad jūs būtumėt tobuli ir visiškai įsitikinę, kas yra Dievo valia. 
\par 13 Aš galiu paliudyti, kad jis labai uolus dėl jūsų, laodikiečių ir hierapoliečių. 
\par 14 Sveikina jus mylimasis gydytojas Lukas ir Demas. 
\par 15 Pasveikinkite brolius Laodikėjoje ir Nimfą bei bažnyčią, kuri jo namuose. 
\par 16 Kai šitas laiškas bus perskaitytas pas jus, pasirūpinkite, kad jis būtų perskaitytas ir laodikiečių bažnyčioje, o jūs perskaitykite laišką, kuris ateis iš Laodikėjos. 
\par 17 Taip pat pasakykite Archipui: “Žiūrėk, kad vykdytum tarnavimą, kurį gavai Viešpatyje!” 
\par 18 Sveikinimas, parašytas mano, Pauliaus, ranka. Prisiminkite mano pančius. Malonė su jumis! Amen.


\end{document}