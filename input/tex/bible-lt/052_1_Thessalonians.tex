\begin{document}

\title{Pirmasis laiškas tesalonikiečiams}

\chapter{1}


\par 1 Paulius, Silvanas ir Timotiejus tesalonikiečių bažnyčiai Dieve Tėve ir Viešpatyje Jėzuje Kristuje. Malonė jums ir ramybė nuo Dievo, mūsų Tėvo, ir Viešpaties Jėzaus Kristaus! 
\par 2 Mes visada dėkojame Dievui už jus visus, prisimindami jus savo maldose, 
\par 3 nuolat minėdami jūsų tikėjimo darbus, meilės triūsą bei vilties ištvermingumą mūsų Viešpatyje Jėzuje Kristuje mūsų Dievo ir Tėvo akivaizdoje, 
\par 4 žinodami, Dievo numylėtieji broliai, jūsų išrinkimą. 
\par 5 Nes mūsų Evangelija neatėjo pas jus vien tik žodžiais, bet su jėga ir Šventąja Dvasia, ir su tvirtu įsitikinimu. Jūs žinote, kaip mes pas jus elgėmės jūsų labui. 
\par 6 Ir jūs pasidarėte mūsų ir Viešpaties pasekėjai, priėmę žodį didžiame suspaudime su Šventosios Dvasios džiaugsmu, 
\par 7 ir taip jūs tapote pavyzdžiu visiems Makedonijos bei Achajos tikintiesiems. 
\par 8 Mat iš jūsų Viešpaties žodis nuskambėjo ne tik Makedonijoje bei Achajoje, bet jūsų tikėjimas Dievu pasklido visur, ir mums jau nebereikia nieko kalbėti. 
\par 9 Jie patys pasakoja apie mūsų atvykimą pas jus, ir kaip jūs nuo stabų atsivertėte prie Dievo tarnauti gyvajam bei tikrajam Dievui 
\par 10 ir laukti iš dangaus Jo Sūnaus, kurį Jis prikėlė iš numirusių,­Jėzaus, gelbstinčio mus nuo ateinančios rūstybės.


\chapter{2}


\par 1 Jūs patys, broliai, žinote, kad mūsų apsilankymas pas jus nebuvo veltui. 
\par 2 Prieš tai, kaip patys žinote, nukentėję ir paniekinti Filipuose, buvome drąsūs Dieve ir skelbėme jums Evangeliją esant dideliam pasipriešinimui. 
\par 3 Juk mūsų skelbimas plaukia ne iš klaidos, nei iš netyro nusistatymo, nei iš klastingumo. 
\par 4 Dievo pripažinti tinkami, kad mums būtų patikėta Evangelija, ją ir skelbiame taip, kad patiktume ne žmonėms, bet Dievui, kuris tiria mūsų širdis. 
\par 5 Kaip žinote, niekada nepasižymėjome pataikaujančiomis kalbomis ir paslėptu godumu,­Dievas yra liudytojas,­ 
\par 6 niekada neieškojome žmonių garbės, nei tarp jūsų, nei kitur. Būdami Kristaus apaštalai, galėjome būti jums našta, 
\par 7 vis dėlto pas jus buvome švelnūs, tarsi maitinanti motina, globojanti savo kūdikius. 
\par 8 Taip jus mylėdami, troškome pasidalyti su jumis ne tik Dievo Evangelija, bet ir savo gyvybe, nes tapote mums brangūs. 
\par 9 Jūs, broliai, atsimenate mūsų triūsą ir pastangas: dirbdami per dienas ir naktis, kad neapsunkintume nė vieno iš jūsų, skelbėme jums Dievo Evangeliją. 
\par 10 Ir jūs, ir Dievas gali paliudyti, kaip šventai, teisingai ir nepriekaištingai elgėmės su jumis, įtikėjusiais. 
\par 11 Jūs žinote, kaip kiekvieną iš jūsų raginome, kalbinome, maldavome, tarsi tėvas savo vaikus, 
\par 12 kad elgtumėtės kaip dera prieš Dievą, kuris jus šaukia į savo karalystę ir šlovę. 
\par 13 Todėl ir mes be paliovos dėkojame Dievui, kad, priėmę Dievo žodį, kurį girdėjote iš mūsų, priėmėte jį ne kaip žmonių žodį, bet, kas jis iš tikro yra,­kaip Dievo žodį, kuris ir veikia jumyse, tikinčiuosiuose. 
\par 14 Jūs, broliai, tapote sekėjais Dievo bažnyčių Kristuje Jėzuje, kurios yra Judėjoje. Jūs tą patį iškentėjote nuo savo tautiečių, kaip ir jos nuo žydų, 
\par 15 kurie nužudė Viešpatį Jėzų ir savo pranašus ir persekiojo mus. Jie nepatinka Dievui ir yra priešiški visiems žmonėms, 
\par 16 nes draudžia mums skelbti Evangeliją pagonims, kad šie būtų išgelbėti. Taip jie nuolat pildo savo nuodėmių saiką, ir jiems artinasi galutinė Dievo rūstybė. 
\par 17 O mes, broliai, kuriam laikui atskirti nuo jūsų,­žinoma, tik kūnu, ne širdimi,­be galo pasiilgę, labai troškome išvysti jūsų veidus. 
\par 18 Todėl ruošėmės atvykti pas jus,­bent jau aš, Paulius, ruošiausi kartą ir kitą,­tačiau mums sutrukdė šėtonas. 
\par 19 Kas gi yra mūsų viltis, džiaugsmas ir pasigyrimo vainikas? Argi ne jūs prieš mūsų Viešpatį Jėzų Kristų Jo atėjimo metu? 
\par 20 Taip, jūs esate mūsų šlovė ir džiaugsmas!


\chapter{3}


\par 1 Todėl, ilgiau nebeiškęsdami, nutarėme vieni pasilikti Atėnuose 
\par 2 ir pasiuntėme Timotiejų, savo brolį, Dievo tarną ir bendradarbį Kristaus Evangelijoje, kad sustiprintų jus ir padrąsintų jus tikėjime, 
\par 3 kad nė vienas nesvyruotų šiuose suspaudimuose, nes jūs patys žinote, kad tam esame skirti. 
\par 4 Dar būdami tarp jūsų, iš anksto sakėme, kad turėsime kentėti priespaudą, ir, kaip žinote, taip ir įvyko. 
\par 5 Todėl, ilgiau nebeiškęsdamas, pasiunčiau pasiuntinį, norėdamas sužinoti apie jūsų tikėjimą, ar kartais jūsų nesugundė gundytojas ir ar nėra niekais pavirtęs mūsų triūsas. 
\par 6 Dabar Timotiejus iš jūsų grįžo pas mus ir atnešė gerą žinią apie jūsų tikėjimą ir jūsų meilę: kad jūs nuolat mus maloniai prisimenate ir karštai trokštate mus pamatyti kaip ir mes jus. 
\par 7 Taigi, broliai, jūs savo tikėjimu paguodėte mus visuose mūsų suspaudimuose ir negandose. 
\par 8 Dabar mes tikrai gyvuojame, nes jūs tvirtai stovite Viešpatyje. 
\par 9 Ir kaip atsidėkosime Dievui už jus, už visus džiaugsmus, kuriuos dėl jūsų patiriame savo Dievo akivaizdoje? 
\par 10 Dieną ir naktį be saiko meldžiamės, kad išvystume jūsų veidus ir galėtume papildyti, ko dar stinga jūsų tikėjimui. 
\par 11 Jis pats­mūsų Dievas ir Tėvas­ir mūsų Viešpats Jėzus Kristus tenutiesia mums kelią pas jus. 
\par 12 Viešpats teaugina jus ir gausiai tepraturtina meile vienų kitiems ir visiems, kaip ir mes jus mylime,­ 
\par 13 tesustiprina jūsų širdis ir padaro nepeiktinas šventume prieš Dievą, mūsų Tėvą, Viešpaties Jėzaus Kristaus ir visų Jo šventųjų atėjimo metu.


\chapter{4}


\par 1 Pagaliau, broliai, prašome ir raginame jus Viešpatyje Jėzuje: jeigu išmokote iš mūsų, kaip privalote elgtis ir patikti Dievui­taip ir elkitės, darydami vis daugiau pažangos! 
\par 2 Jūs juk žinote, kokių nurodymų jums davėme Viešpaties Jėzaus vardu. 
\par 3 Nes tokia Dievo valia­jūsų šventėjimas; kad susilaikytumėte nuo ištvirkavimo 
\par 4 ir kiekvienas iš jūsų mokėtų saugoti savąjį indą šventume ir pagarboje, 
\par 5 o ne aistringame geiduly, kaip pagonys, kurie nepažįsta Dievo; 
\par 6 kad nė vienas neperžengtų ribų ir šituo dalyku neapgaudinėtų savo brolio, nes Dievas už visa tai keršija, kaip jau esame įspėję ir patvirtinę. 
\par 7 Dievas nepašaukė mūsų nedorybei, bet šventumui. 
\par 8 Todėl, kas tai atmeta, ne žmogų atmeta, bet Dievą, kuris ir davė mums savo Šventąją Dvasią. 
\par 9 Apie brolišką meilę nebereikia jums rašyti, nes jūs patys esate Dievo išmokyti mylėti vienas kitą, 
\par 10 ir jūs taip darote visiems broliams visoje Makedonijoje. Mes tik raginame jus, broliai, daryti vis daugiau pažangos, 
\par 11 stengtis gyventi ramiai, atsidėti saviesiems reikalams ir dirbti savo rankomis, kaip jums įsakėme. 
\par 12 Taip jūs deramai elgsitės pašalinių akyse ir jums nieko netrūks. 
\par 13 Aš nenoriu, kad jūs, broliai, liktumėte nežinioje dėl užmigusiųjų ir nusimintumėte kaip kiti, kurie neturi vilties. 
\par 14 Jeigu tikime, kad Jėzus mirė ir prisikėlė, tai Dievas ir tuos, kurie užmigo su Jėzumi, atves kartu su Juo. 
\par 15 Ir tai jums sakome Viešpaties žodžiu, jog mes, gyvieji, išlikusieji iki Viešpaties atėjimo, nepralenksime užmigusiųjų. 
\par 16 Nes pats Viešpats nužengs iš dangaus, nuskambėjus paliepimui, arkangelo balsui ir Dievo trimitui, ir mirusieji Kristuje prisikels pirmiausia, 
\par 17 paskui mes, gyvieji, išlikusieji, kartu su jais būsime pagauti į debesis susitikti su Viešpačiu ore ir taip visuomet pasiliksime su Viešpačiu. 
\par 18 Todėl guoskite vieni kitus šiais žodžiais.


\chapter{5}


\par 1 Dėl valandos ir laiko, broliai, nėra reikalo jums rašyti. 
\par 2 Jūs patys gerai žinote, kad Viešpaties diena užklups lyg vagis naktį. 
\par 3 Kai žmonės kalbės: “Ramybė ir saugumas”, tada juos ir ištiks netikėtas žlugimas, tarytum gimdymo skausmai nėščią moterį, ir jie niekur nepaspruks. 
\par 4 Bet jūs, broliai, nesate tamsoje, kad toji diena jus užkluptų lyg vagis. 
\par 5 Juk jūs visi esate šviesos ir dienos vaikai. Mes nepriklausome nei nakčiai, nei tamsai. 
\par 6 Todėl nemiegokime kaip kiti, bet budėkime ir būkime blaivūs! 
\par 7 Mat kas miega, miega naktį, ir kas pasigeria, pasigeria naktį. 
\par 8 O mes, priklausydami dienai, būkime blaivūs ir apsirengę tikėjimo bei meilės šarvus ir užsidėję išgelbėjimo vilties šalmą. 
\par 9 Juk Dievas mus paskyrė ne rūstybei, bet kad įgytume išgelbėjimą per mūsų Viešpatį Jėzų Kristų, 
\par 10 kuris mirė už mus, kad mes ar budėdami, ar miegodami gyventume kartu su Juo. 
\par 11 Todėl guoskite ir statydinkite vieni kitus, ką jūs ir darote. 
\par 12 Prašome jus, broliai, gerbti tuos, kurie darbuojasi tarp jūsų, vadovauja jums Viešpatyje ir teikia jums pamokymų. 
\par 13 Be galo branginkite juos, mylėdami dėl jų darbo! Taikiai sugyvenkite tarpusavyje! 
\par 14 Raginame jus, broliai: įspėkite nedrausminguosius, padrąsinkite liūdinčiuosius, palaikykite silpnuosius, būkite kantrūs su visais! 
\par 15 Žiūrėkite, kad kas neatsimokėtų kam nors blogu už bloga, bet visada siekite daryti gera vieni kitiems ir visiems. 
\par 16 Visada džiaukitės, 
\par 17 be paliovos melskitės! 
\par 18 Už viską dėkokite, nes tokia Dievo valia jums Kristuje Jėzuje. 
\par 19 Negesinkite Dvasios! 
\par 20 Neniekinkite pranašavimų! 
\par 21 Visa ištirkite ir to, kas gera, laikykitės! 
\par 22 Susilaikykite nuo visokio blogio! 
\par 23 Pats ramybės Dievas iki galo jus tepašventina ir teišlaiko jūsų dvasią, sielą ir kūną nepeiktiną mūsų Viešpaties Jėzaus atėjimui. 
\par 24 Ištikimas yra Tas, kuris jus šaukia, Jis ir įvykdys! 
\par 25 Broliai, melskitės už mus! 
\par 26 Sveikinkite visus brolius šventu pabučiavimu. 
\par 27 Saikdinu jus Viešpačiu, kad šis laiškas būtų perskaitytas visiems šventiesiems broliams. 
\par 28 Mūsų Viešpaties Jėzaus Kristaus malonė tebūna su jumis! Amen.



\end{document}