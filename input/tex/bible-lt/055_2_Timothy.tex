\begin{document}

\title{Antrasis laiškas Timotiejui}

\chapter{1}


\par 1 Paulius, Dievo valia Jėzaus Kristaus apaštalas pagal gyvenimo pažadą Kristuje Jėzuje,— 
\par 2 mylimam sūnui Timotiejui: malonė, gailestingumas ir ramybė nuo Dievo Tėvo ir Jėzaus Kristaus, mūsų Viešpaties! 
\par 3 Dėkoju Dievui, kuriam tarnauju kaip ir mano protėviai su tyra sąžine, be paliovos tave prisimindamas savo maldose dieną ir naktį. 
\par 4 Menu tavo ašaras ir trokštu tave matyti, kad būčiau kupinas džiaugsmo. 
\par 5 Aš vis prisimenu tavo neveidmainišką tikėjimą, kuris pirma gyveno tavo senelėje Loidėje, tavo motinoje Eunikėje ir, esu įsitikinęs, gyvena ir tavyje. 
\par 6 Todėl tau primenu, kad vėl uždegtum Dievo dovaną, esančią tavyje mano rankų uždėjimu. 
\par 7 Nes Dievas davė mums ne baimės dvasią, bet jėgos, meilės ir savitvardos dvasią. 
\par 8 Todėl nesigėdyk mūsų Viešpaties liudijimo, nei manęs, Jo kalinio, bet drauge su manimi kentėk dėl Evangelijos jėga Dievo, 
\par 9 kuris išgelbėjo mus bei pašaukė šventu pašaukimu, ne pagal mūsų darbus, bet savo paties nutarimu bei malone, kuri buvo suteikta mums Kristuje Jėzuje prieš amžinuosius laikus, 
\par 10 o dabar apreikšta, pasirodžius mūsų Gelbėtojui Jėzui Kristui, kuris sunaikino mirtį ir nušvietė gyvenimą bei nemirtingumą savo Evangelija; 
\par 11 jai aš esu paskirtas pamokslininku, apaštalu ir pagonių mokytoju. 
\par 12 Dėl šios priežasties aš ir kenčiu, bet nesigėdiju, nes žinau, kuo įtikėjau, ir esu įsitikinęs, kad Jis gali išlaikyti iki anos dienos tai, ką Jam patikėjau. 
\par 13 Sau pavyzdžiu laikyk sveikus žodžius, kuriuos girdėjai iš manęs tikėjime ir meilėje Kristuje Jėzuje. 
\par 14 Saugok tą gera, kas tau patikėta, Šventąja Dvasia, kuri gyvena mumyse. 
\par 15 Tu žinai, kad nuo manęs nusigręžė visi, kurie yra Azijoje, tarp jų Figelas ir Hermogenas. 
\par 16 Viešpats tebūna gailestingas Onesiforo namams, nes jis dažnai mane atgaivindavo ir nesigėdijo mano pančių, 
\par 17 bet, atvykęs į Romą, labai uoliai manęs ieškojo ir surado. 
\par 18 Viešpats tesuteikia jam aną dieną rasti Viešpaties pasigailėjimą. Be to, tu gerai žinai, kiek daug jis yra tarnavęs man Efeze.


\chapter{2}


\par 1 Taigi tu, mano sūnau, būk stiprus malone, kuri yra Kristuje Jėzuje, 
\par 2 ir, ką iš manęs girdėjai prie daugelio liudytojų, patikėk ištikimiems žmonėms, kurie sugebės ir kitus mokyti. 
\par 3 Iškęsk sunkumus kaip geras Kristaus Jėzaus karys. 
\par 4 Nė vienas kareivis neįsivelia į gyvenimo reikalus, norėdamas patikti jį pašaukusiam. 
\par 5 Ir kas stoja į rungtynes, negaus vainiko, jei nebus grūmęsis pagal taisykles. 
\par 6 Sunkiai dirbantis žemdirbys turi pirmas pasiimti vaisių. 
\par 7 Suprask, ką sakau; Viešpats teduoda tau išmanymo apie visus dalykus. 
\par 8 Prisimink prikeltąjį iš numirusių Jėzų Kristų iš Dovydo palikuonių, kaip skelbiama mano Evangelijoje, 
\par 9 dėl kurios aš kenčiu ir net esu surakintas lyg piktadarys. Bet Dievo žodis nesurakinamas! 
\par 10 Todėl aš visa pakenčiu dėl išrinktųjų, kad ir jie įgytų išgelbėjimą Kristuje Jėzuje su amžinąja šlove. 
\par 11 Štai patikimas žodis: jei mes su Juo mirėme, su Juo ir gyvensime. 
\par 12 Jei kenčiame, su Juo ir valdysime. Jeigu mes Jo išsižadėsime, ir Jis mūsų išsižadės. 
\par 13 Jeigu esame neištikimi, Jis lieka ištikimas, nes savęs Jis negali išsižadėti. 
\par 14 Primink tai, liepdamas Viešpaties akivaizdoje, kad nekovotų žodžiais, nes iš to jokios naudos, vien tik žala klausytojams. 
\par 15 Uoliai stenkis pasirodyti Dievui tinkamu darbininku, neturinčiu ko gėdytis, tiksliai perteikiančiu tiesos žodį. 
\par 16 Venk bedieviškų ir tuščių tauškalų, kurie vis giliau ves į bedievystę. 
\par 17 Jų kalba, tarsi gangrena, vis plis. Tokie yra Himenėjas ir Filetas, 
\par 18 kurie nuklydo nuo tiesos, tvirtindami, kad prisikėlimas jau įvykęs, ir tuo griaudami kai kurių tikėjimą. 
\par 19 Bet tvirtai stovi Dievo pamatas, turintis tokį antspaudą: “Viešpats pažįsta savuosius” ir: “Tepasitraukia nuo neteisybės kiekvienas, kuris šaukiasi Kristaus vardo”. 
\par 20 O dideliuose namuose yra ne tik auksinių ir sidabrinių indų, bet ir medinių bei molinių. Vieni tarnauja garbingiems reikalams, kiti­negarbingiems. 
\par 21 Jeigu kas apsivalys nuo minėtų dalykų, bus indas, skirtas garbei, pašventintas, tinkamas Šeimininkui, pasiruošęs kiekvienam geram darbui. 
\par 22 Bėk nuo jaunystės geidulių ir siek teisumo, tikėjimo, meilės ir ramybės su tais, kurie iš tyros širdies šaukiasi Viešpaties. 
\par 23 O kvailų ir nemokšiškų ginčų venk, žinodamas, kad jie sukelia kovas. 
\par 24 Viešpaties tarnas neturi kivirčytis, bet būti malonus su visais, gabus pamokyti, kantrus, 
\par 25 romiai aiškinti prieštaraujantiems,­rasi Dievas duos jiems atgailauti, kad pažintų tiesą 
\par 26 ir atsipeikėtų nuo pinklių velnio, kuris pavergęs juos savo valiai.


\chapter{3}


\par 1 Žinok, kad paskutinėmis dienomis užeis sunkūs laikai, 
\par 2 nes žmonės bus savimylos, pinigų mylėtojai, pagyrūnai, išdidūs, piktžodžiautojai, neklusnūs tėvams, nedėkingi, nešventi, 
\par 3 nemylintys, nesutaikomi, šmeižikai, nesusivaldantys, šiurkštūs, nekenčiantys to, kas gera, 
\par 4 išdavikai, užsispyrę, pasipūtėliai, labiau mylintys malonumus negu Dievą, 
\par 5 turintys dievotumo išvaizdą, bet atsižadėję jo jėgos. Šalinkis tokių žmonių! 
\par 6 Iš jų yra tie, kurie įsiskverbia į namus ir pavergia silpnas moterėles, pilnas nuodėmių, geidulių vedžiojamas, 
\par 7 nuolat besimokančias ir vis nesugebančias pasiekti tiesos pažinimo. 
\par 8 Kaip Janas ir Jambras priešinosi Mozei, taip ir jie priešinasi tiesai. Tai žmonės sugedusio proto, netikusio tikėjimo. 
\par 9 Bet jie toli nenužengs, nes jų kvailumas, kaip ir anų, bus visiems regimas. 
\par 10 Bet tu stropiai pasekei mano mokymu, gyvenimo būdu, tikslu, tikėjimu, ištverme, meile, kantrybe, 
\par 11 persekiojimais, sunkumais, kurie mane ištiko Antiochijoje, Ikonijuje, Listroje. O kokių tik persekiojimų man neteko iškęsti! Bet iš visų išgelbėjo mane Viešpats. 
\par 12 Taip ir visi, kurie trokšta dievotai gyventi Kristuje Jėzuje, bus persekiojami. 
\par 13 Pikti žmonės ir suvedžiotojai eis blogyn, klaidindami ir klysdami. 
\par 14 O tu pasilik prie to, ką išmokai ir įtikėjai, žinodamas, iš ko išmokai. 
\par 15 Tu nuo vaikystės žinai šventuosius Raštus, galinčius tave pamokyti išgelbėjimui per tikėjimą, kuris yra Kristuje Jėzuje. 
\par 16 Visas Raštas yra Dievo įkvėptas ir naudingas mokyti, barti, taisyti, auklėti teisumui, 
\par 17 kad Dievo žmogus taptų tobulas, pasiruošęs kiekvienam geram darbui.


\chapter{4}


\par 1 Aš primygtinai prašau prieš Dievą ir Viešpatį Jėzų Kristų, kuris teis gyvuosius ir mirusiuosius, Jam ir Jo karalystei atėjus: 
\par 2 skelbk žodį, veik laiku ir nelaiku, bark, drausk, ragink su didžia ištverme ir pamokymu. 
\par 3 Nes ateis laikas, kai žmonės nebepakęs sveiko mokslo, bet, pasidavę savo įgeidžiams, pasikvies sau mokytojus, kad tie dūzgentų ausyse; 
\par 4 jie nukreips ausis nuo tiesos ir atvers pasakoms. 
\par 5 Bet tu būk visame kame apdairus, iškęsk sunkumus, dirbk evangelisto darbą, atlik savo tarnavimą. 
\par 6 Nes aš jau atnašaujamas, ir mano iškeliavimo laikas jau čia pat. 
\par 7 Aš kovojau gerą kovą, baigiau bėgimą, išlaikiau tikėjimą. 
\par 8 Nuo šiol manęs laukia teisumo vainikas, kurį aną dieną man duos Viešpats, teisingasis Teisėjas,­ir ne tik man, bet ir visiems, kurie pamilo Jo pasirodymą. 
\par 9 Pasistenk greitai atvykti pas mane, 
\par 10 nes Demas, pamilęs šį pasaulį, paliko mane ir iškeliavo į Tesaloniką, Krescentas­į Galatiją, Titas­į Dalmatiją. 
\par 11 Vienas Lukas tėra su manimi. Pasiimk ir atvesk su savimi Morkų, jis man naudingas tarnavimui. 
\par 12 Tichiką pasiunčiau į Efezą. 
\par 13 Atvykdamas atgabenk apsiaustą, kurį palikau Troadėje pas Karpą, taip pat ir knygas, ypač pergamentus. 
\par 14 Kalvis Aleksandras man padarė daug bloga, Viešpats jam teatmoka pagal jo darbus. 
\par 15 Ir tu saugokis jo, nes jis labai priešinosi mūsų žodžiams. 
\par 16 Mano pirmajame apsigynime nė vieno nebuvo su manimi, visi mane paliko. Tenebus jiems tai palaikyta nusikaltimu! 
\par 17 Bet Viešpats stovėjo su manimi ir sustiprino mane, kad toliau skelbčiau Evangeliją ir išgirstų visi pagonys; aš buvau išgelbėtas iš liūto nasrų. 
\par 18 Ir Viešpats išgelbės mane iš visų piktų kėslų ir išsaugos savo dangiškajai karalystei. Jam šlovė per amžių amžius! Amen. 
\par 19 Pasveikink Priską, Akvilą ir Onesiforo namiškius. 
\par 20 Erastas pasiliko Korinte, o Trofimą palikau Milete sergantį. 
\par 21 Pasistenk atvykti dar prieš žiemą! Tave sveikina Eubulas, Pudentas, Linas ir Klaudija bei visi broliai. 
\par 22 Viešpats Jėzus Kristus tebūna su tavo dvasia! Malonė teesie su jumis! Amen.



\end{document}