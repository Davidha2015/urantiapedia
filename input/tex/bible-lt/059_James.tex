\begin{document}

\title{Jokūbo laiškas}


\chapter{1}


\par 1 Jokūbas, Dievo ir Viešpaties Jėzaus Kristaus tarnas, siunčia sveikinimus dvylikai pasklidusių giminių. 
\par 2 Mano broliai, laikykite didžiausiu džiaugsmu, kai patenkate į visokius išbandymus. 
\par 3 Žinokite, kad jūsų tikėjimo išbandymas ugdo ištvermę, 
\par 4 o ištvermė tesubręsta iki galo, kad būtumėte tobuli, subrendę ir nieko nestokotumėte. 
\par 5 Jei kuriam iš jūsų trūksta išminties, teprašo Dievą, kuris visiems dosniai duoda ir nepriekaištauja, ir jam bus suteikta. 
\par 6 Bet tegul prašo tikėdamas, nė kiek neabejodamas, nes abejojantis panašus į jūros bangą, varinėjamą ir blaškomą vėjo. 
\par 7 Toksai žmogus tenemano ką nors gausiąs iš Viešpaties,­ 
\par 8 toks svyruojantis, visuose savo keliuose nepastovus žmogus. 
\par 9 Tesigiria pažemintas brolis savo išaukštinimu, 
\par 10 o turtuolis savo pažeminimu, nes jis išnyks kaip lauko gėlė. 
\par 11 Juk pakyla saulė, jos kaitra išdžiovina žolyną, ir jo žiedas nubyra, jo išvaizdos grožis pranyksta. Taip ir turtuolis sunyks savo keliuose. 
\par 12 Palaimintas žmogus, kuris ištveria pagundymą, nes, kai bus išbandytas, jis gaus gyvenimo vainiką, kurį Viešpats pažadėjo Jį mylintiems. 
\par 13 Nė vienas gundomas tenesako: “Esu Dievo gundomas”. Dievas negali būti gundomas blogiu ir pats nieko negundo. 
\par 14 Kiekvienas yra gundomas, savo paties geismo pagrobtas ir suviliotas. 
\par 15 Paskui užsimezgęs geismas pagimdo nuodėmę, o užbaigta nuodėmė gimdo mirtį. 
\par 16 Neapsigaukite, mano mylimi broliai! 
\par 17 Kiekvienas geras davinys ir kiekviena tobula dovana yra iš aukštybių, nužengia nuo šviesybių Tėvo, kuriame nėra permainų ir nė šešėlio keitimosi. 
\par 18 Savo valia Jis pagimdė mus tiesos žodžiu, kad būtume tarsi Jo kūrinių pirmieji vaisiai. 
\par 19 Žinokite, mano mylimi broliai: kiekvienas žmogus tebūna greitas klausyti, lėtas kalbėti, lėtas pykti. 
\par 20 Žmogaus rūstybė nedaro Dievo teisumo. 
\par 21 Todėl, atmetę visą nešvarą bei piktybės gausą, su romumu priimkite įdiegtąjį žodį, kuris gali išgelbėti jūsų sielas. 
\par 22 Būkite žodžio vykdytojai, o ne vien klausytojai, apgaudinėjantys patys save. 
\par 23 Jei kas tėra žodžio klausytojas, o ne vykdytojas, tai jis panašus į žmogų, kuris žiūri į savo gimtąjį veidą veidrodyje. 
\par 24 Pasižiūrėjo į save ir nuėjo, ir bematant pamiršo, koks buvo. 
\par 25 Bet kas įsižiūri į tobuląjį laisvės įstatymą ir jį vykdo, kas tampa nebe klausytojas užuomarša, bet darbo vykdytojas, tas bus palaimintas savo darbuose. 
\par 26 Jei kas iš jūsų mano esąs pamaldus ir nepažaboja savo liežuvio, bet apgaudinėja savo širdį, to pamaldumas tuščias. 
\par 27 Tyras ir nesuteptas pamaldumas prieš Dievą ir Tėvą yra: lankyti našlaičius ir našles jų sielvarte ir saugoti save nesuterštą šiuo pasauliu.


\chapter{2}


\par 1 Mano broliai, turėkite mūsų Jėzaus Kristaus, šlovės Viešpaties, tikėjimą, neatsižvelgdami į asmenis. 
\par 2 Štai į jūsų susirinkimą ateina žmogus su auksiniu žiedu, puikiais drabužiais, taip pat įžengia beturtis apskurusiu apdaru. 
\par 3 Jūs žiūrite į tą, kuris puikiai apsirengęs, ir sakote: “Atsisėsk čia į gerą vietą”, o beturčiui tariate: “Stovėk ten”, arba: “Sėskis prie mano kojų”. 
\par 4 Argi jūs nesate šališki, argi netampate piktais sumetimais besivadovaujantys teisėjai? 
\par 5 Paklausykite, mano mylimieji broliai: ar Dievas neišsirinko pasaulio vargšų, kad jie būtų turtingi tikėjimu ir paveldėtų karalystę, pažadėtą Jį mylintiems? 
\par 6 O jūs paniekinote beturtį! Argi ne turtuoliai jus spaudžia, ar ne jie tampo jus po teismus? 
\par 7 Ar ne jie niekina tą kilnų vardą, kuriuo jūs vadinatės? 
\par 8 Jeigu tik įvykdote karališkąjį įstatymą, kaip sako Raštas: “Mylėk savo artimą kaip save patį”, jūs gerai darote; 
\par 9 bet, jeigu atsižvelgiate į asmenis, darote nuodėmę ir esate įstatymo kaltinami kaip nusižengėliai. 
\par 10 Mat, kas laikosi viso įstatymo, bet nusižengia vienu dalyku, tas kaltas dėl visų. 
\par 11 Juk Tas, kuris pasakė: “Nesvetimauk!”, taip pat pasakė ir: “Nežudyk”. Tad jeigu tu nesvetimauji, bet žudai,­vis tiek esi įstatymo laužytojas. 
\par 12 Taip kalbėkite ir taip darykite, kaip tie, kurie bus teisiami pagal laisvės įstatymą. 
\par 13 Teismas negailestingas tam, kuris neparodė gailestingumo. O gailestingumas triumfuoja prieš teismą. 
\par 14 Kokia nauda, mano broliai, jei kas sakosi turįs tikėjimą, bet neturi darbų? Ar gali jį išgelbėti toks tikėjimas? 
\par 15 Jei brolis ar sesuo neturi drabužių ir stokoja kasdienio maisto, 
\par 16 ir kas nors iš jūsų jiems tartų: “Eikite ramybėje, sušilkite ir pasisotinkite”, o neduotų, ko reikia jų kūnui,­kokia iš to nauda? 
\par 17 Taip ir tikėjimas: jei neturi darbų, jis savyje miręs. 
\par 18 Bet kažkas pasakys: “Tu turi tikėjimą, o aš turiu darbus”. Parodyk man savo tikėjimą be darbų, o aš tau darbais parodysiu savo tikėjimą. 
\par 19 Tu tiki, kad yra vienas Dievas? Gerai darai. Ir demonai tiki ir dreba! 
\par 20 Ar nori žinoti, neišmintingas žmogau, kad tikėjimas be darbų miręs? 
\par 21 Argi ne darbais buvo išteisintas mūsų tėvas Abraomas, aukodamas savo sūnų Izaoką ant aukuro? 
\par 22 Ar matai, kad tikėjimas veikė kartu su jo darbais, ir darbais tikėjimas buvo atbaigtas? 
\par 23 Taip išsipildė Rašto žodžiai: “Abraomas patikėjo Dievu, ir tai buvo jam įskaityta teisumu”, o jis buvo pramintas Dievo draugu. 
\par 24 Jūs matote, kad žmogus išteisinamas darbais, o ne vienu tikėjimu. 
\par 25 Taip pat ir paleistuvė Rahaba: argi ji ne darbais buvo išteisinta, kai priėmė pasiuntinius ir kitu keliu juos išleido? 
\par 26 Kaip kūnas be dvasios yra miręs, taip ir tikėjimas be darbų negyvas.


\chapter{3}


\par 1 Mano broliai, ne visi būkite mokytojais. Žinokite, kad mūsų laukia griežtesnis teismas. 
\par 2 Juk mes visi daug kur nusižengiame. Kas nenusižengia žodžiu, tas yra tobulas žmogus; jis sugeba pažaboti ir visą kūną. 
\par 3 Jei mes įbrukame žąslus arkliams į nasrus, kad mums paklustų, mes suvaldome visą jų kūną. 
\par 4 Štai kad ir laivai: nors jie tokie dideli ir smarkių vėjų varomi, mažytis vairas juos pakreipia, kur nori vairininkas. 
\par 5 Taip pat ir liežuvis yra mažas narys, bet giriasi didžiais dalykais. Žiūrėkite, kokia maža ugnelė padega didžiausią girią; 
\par 6 ir liežuvis yra ugnis­nedorybės pasaulis. Liežuvis yra vienas iš mūsų narių, kuris suteršia visą kūną, padega gyvenimo eigą, pats pragaro padegtas. 
\par 7 Kiekviena žvėrių, paukščių, šliužų ir jūros gyvūnų veislė buvo sutramdyta ir yra sutramdoma žmogaus prigimties jėga. 
\par 8 O liežuvio joks žmogus nepajėgia suvaldyti; jis lieka nerimstanti blogybė, pilna mirtinų nuodų. 
\par 9 Juo laiminame mūsų Dievą Tėvą ir juo keikiame žmones, kurie sutverti pagal Dievo atvaizdą. 
\par 10 Iš tų pačių lūpų plaukia ir laiminimas, ir prakeikimas. Bet taip, mano broliai, neturi būti! 
\par 11 Nejaugi šaltinis iš tos pačios versmės lieja saldų ir kartų vandenį? 
\par 12 Argi gali, mano broliai, figmedis išauginti alyvas, o vynmedis figas? Taip pat ir šaltinis negali duoti sūraus vandens ir saldaus. 
\par 13 Kas tarp jūsų išmintingas ir sumanus? Teparodo geru elgesiu savo darbus su išmintingu romumu. 
\par 14 Bet jeigu jūs savo širdyje puoselėjate kartų pavydą ir savanaudiškumą, tuomet nesigirkite ir nemeluokite tiesai. 
\par 15 Tai nėra išmintis, nužengusi iš aukštybių, bet žemiška, sielinė ir demoniška. 
\par 16 Kur pavydas ir savanaudiškumas, ten netvarka bei įvairūs pikti darbai. 
\par 17 Bet išmintis, kilusi iš aukštybių, pirmiausia yra tyra, paskui taikinga, švelni, klusni, pilna gailestingumo ir gerų vaisių, bešališka ir neveidmainiška. 
\par 18 O teisumo vaisius su ramybe sėjamas tų, kurie neša ramybę.


\chapter{4}


\par 1 Iš kur tarp jūsų atsiranda karai ir kivirčai? Ar ne iš jūsų užgaidų, kurios nerimsta jūsų nariuose? 
\par 2 Geidžiate ir neturite; žudote ir pavydite­ir negalite pasiekti; kovojate ir kariaujate; neturite, nes neprašote. 
\par 3 Jūs prašote ir negaunate, nes blogo prašote­savo užgaidoms išleisti. 
\par 4 Paleistuviai ir paleistuvės! Ar nežinote, kad draugystė su pasauliu yra priešiškumas Dievui? Taigi kas nori būti pasaulio bičiulis, tas tampa Dievo priešu. 
\par 5 Gal manote, kad Raštas veltui sako: “Pavydžiai trokšta Dvasia, kuri gyvena mumyse”. 
\par 6 Bet Jis duoda dar didesnę malonę, ir todėl sako: “Dievas išdidiems priešinasi, o nuolankiesiems teikia malonę”. 
\par 7 Todėl atsiduokite Dievui; priešinkitės velniui, ir jis bėgs nuo jūsų. 
\par 8 Artinkitės prie Dievo, ir Jis artinsis prie jūsų. Nusiplaukite rankas, nusidėjėliai, nusivalykite širdis, dvejojantys. 
\par 9 Dejuokite, liūdėkite ir raudokite! Jūsų juokas tepavirsta gedulu, o džiaugsmas­liūdesiu. 
\par 10 Nusižeminkite prieš Viešpatį, ir Jis jus išaukštins. 
\par 11 Broliai, neapkalbinėkite vieni kitų! Kas apkalba ir teisia savo brolį, tas apkalba ir teisia įstatymą. O jeigu teisi įstatymą, vadinasi, nesi įstatymo vykdytojas, bet teisėjas. 
\par 12 Yra vienas įstatymo leidėjas, kuris gali išgelbėti ir pražudyti. O kas gi tu toks, kuris teisi kitą? 
\par 13 Nagi jūs, kurie sakote: “Šiandien arba rytoj keliausime į tą ir tą miestą, tenai išbūsime metus, versimės prekyba ir pasipelnysime”,­ 
\par 14 jūs nežinote, kas atsitiks rytoj! Ir kas gi jūsų gyvybė? Garas, kuris trumpam pasirodo ir paskui išnyksta. 
\par 15 Verčiau sakytumėte: “Jei Viešpats panorės, gyvensime ir darysime šį bei tą”. 
\par 16 O dabar jūs giriatės iš savo pasipūtimo, ir kiekvienas toks gyrimasis yra blogas. 
\par 17 Kas moka daryti gera ir nedaro, tas nusideda.


\chapter{5}


\par 1 Nagi dabar jūs, turtuoliai, verkite ir raudokite dėl jums artėjančių negandų! 
\par 2 Jūsų turtai supuvę ir jūsų drabužiai kandžių sukapoti. 
\par 3 Jūsų auksas ir sidabras surūdijo, ir jų rūdys prieš jus liudys ir ės jūsų kūnus kaip ugnis. Jūs susikrovėte turtų paskutinėmis dienomis. 
\par 4 Štai šaukia jūsų laukus nuvaliusių darbininkų užmokestis, kurį jūs nusukote, ir pjovėjų šauksmai pasiekė kareivijų Viešpaties ausis. 
\par 5 Jūs gašliai gyvenote žemėje ir mėgavotės; jūs nupenėjote savo širdis tarsi skerdimo dienai. 
\par 6 Jūs pasmerkėte ir nužudėte teisųjį: jis jums nesipriešina. 
\par 7 Tad būkite kantrūs, broliai, iki Viešpaties atėjimo. Antai žemdirbys ilgai ir kantriai laukia brangaus žemės vaisiaus, kol šis gauna ankstyvojo ir vėlyvojo lietaus. 
\par 8 Ir jūs būkite kantrūs, sustiprinkite savo širdis, nes Viešpaties atėjimas arti. 
\par 9 Nemurmėkite, broliai, vieni prieš kitus, kad nebūtumėte teisiami. Štai teisėjas jau stovi prie durų. 
\par 10 Imkite, broliai, kentėjimo ir ištvermės pavyzdžiu pranašus, kurie kalbėjo Viešpaties vardu. 
\par 11 Štai mes laikome palaimintais ištvėrusius. Jūs girdėjote apie Jobo ištvermę ir matėte, kokia buvo jam Viešpaties skirta pabaiga, nes Viešpats kupinas užuojautos ir gailestingumo. 
\par 12 Bet pirmiausia, mano broliai, neprisiekite nei dangumi, nei žeme, nei kitokia priesaika. Tebūnie jūsų “taip”­ taip ir jūsų “ne”­ ne, kad nepakliūtumėte į teismą. 
\par 13 Kenčia kas iš jūsų? Tesimeldžia. Kas nors džiaugiasi? Tegul gieda psalmes. 
\par 14 Kas nors pas jus serga? Tepasikviečia bažnyčios vyresniuosius, ir jie tesimeldžia už jį, patepdami aliejumi Viešpaties vardu. 
\par 15 Ir tikėjimo malda išgelbės ligonį, ir Viešpats jį pakels, o jeigu jis būtų nusidėjęs, jam bus atleista. 
\par 16 Išpažinkite vieni kitiems savo nusižengimus ir melskitės vieni už kitus, kad būtumėte išgydyti. Daug pajėgia veiksminga, karšta teisiojo malda. 
\par 17 Elijas buvo toks pat žmogus kaip ir mes. Jis meldė, kad nelytų, ir nelijo žemėje trejus metus ir šešis mėnesius; 
\par 18 ir jis vėl meldė, ir dangus davė lietaus, o žemė užaugino savo vaisių. 
\par 19 Mano broliai, jeigu kas iš jūsų nuklystų nuo tiesos ir kas nors jį atverstų,­ 
\par 20 težino, kad, sugrąžindamas nusidėjėlį iš jo klystkelio, išgelbės sielą nuo mirties ir uždengs daugybę nuodėmių.



\end{document}