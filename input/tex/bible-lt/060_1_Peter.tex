\begin{document}

\title{Pirmasis Petro laiškas}


\chapter{1}


\par 1 Petras, Jėzaus Kristaus apaštalas, ateiviams, pasklidusiems Ponte, Galatijoje, Kapadokijoje, Azijoje ir Bitinijoje, 
\par 2 išrinktiems išankstiniu Dievo Tėvo numatymu, Dvasios pašventinimu, kad būtų klusnūs ir apšlakstyti Jėzaus Kristaus krauju. Tepadaugėja jums malonė ir ramybė! 
\par 3 Tebūnie palaimintas Dievas, mūsų Viešpaties Jėzaus Kristaus Tėvas, kuris iš savo didžio gailestingumo Jėzaus Kristaus prikėlimu iš numirusių atgimdė mus gyvai vilčiai, 
\par 4 nenykstančiam, nesuteptam ir nevystančiam palikimui, kuris paruoštas jums danguje. 
\par 5 Dievo jėga per tikėjimą jūs esate saugomi išgelbėjimui, kuris parengtas apsireikšti paskutiniu laiku. 
\par 6 Tuo džiaukitės, nors dabar, jei reikia, trumpai kenčiate įvairiuose išmėginimuose, 
\par 7 kad jūsų išbandytas tikėjimas, brangesnis už pragaištantį auksą, nors ir ugnimi ištirtą, būtų pripažintas vertas gyriaus, garbės ir šlovės, kai apsireikš Jėzus Kristus. 
\par 8 Jūs mylite Jį, nors ir nesate Jo matę; tikėdami Jį, nors ir neregėdami, džiaugiatės neišsakomu ir šlovingiausiu džiaugsmu, 
\par 9 gaudami jūsų tikėjimo tikslą­ sielų išgelbėjimą. 
\par 10 Šito išgelbėjimo ieškojo ir kruopščiai jį tyrinėjo pranašai, kurie pranašavo apie jums skirtąją malonę. 
\par 11 Jie tyrinėjo, kurį ir kokį laiką skelbė juose esanti Kristaus Dvasia, iš anksto nurodžiusi Kristaus kentėjimus ir juos lydinčią šlovę. 
\par 12 Jiems buvo apreikšta, kad jie ne sau, bet mums tarnavo tuo, kas dabar pranešta jums per tuos, kurie paskelbė Evangeliją Šventąja Dvasia, pasiųsta iš dangaus; į tai trokšta pažvelgti angelai. 
\par 13 Todėl, susijuosę savo proto strėnas, būkite blaivūs ir visiškai pasitikėkite malone, kuri bus jums suteikta, kai apsireikš Jėzus Kristus. 
\par 14 Kaip klusnūs vaikai, nepasiduokite ankstesniems savo neišmaningumo laikų geiduliams, 
\par 15 bet, kaip šventas yra Tas, kuris jus pašaukė, taip ir jūs būkite šventi visu savo elgesiu, 
\par 16 nes parašyta: “Būkite šventi, nes Aš esu šventas”. 
\par 17 Ir jei kaip Tėvo šaukiatės To, kuris nešališkai teisia pagal kiekvieno darbą, su baime elkitės savo viešnagės metu, 
\par 18 žinodami, kad esate atpirkti nuo betikslio iš protėvių paveldėto gyvenimo būdo ne nykstančiais turtais, sidabru ar auksu, 
\par 19 bet brangiuoju krauju Kristaus, to avinėlio be kliaudos ir dėmės. 
\par 20 Jis buvo numatytas dar prieš pasaulio sutvėrimą, o apreikštas šiais paskutiniais laikais jums, 
\par 21 per Jį įtikėjusiems Dievą, kuris prikėlė Jį iš numirusių ir suteikė Jam šlovę, kad jūs tikėtumėte ir viltumėtės Dievu. 
\par 22 Nuskaidrinę savo sielas Dvasia klusnumu tiesai dėl neveidmainiškos brolių meilės, karštai iš tyros širdies mylėkite vieni kitus. 
\par 23 Jūs esate atgimę ne iš pranykstančios, bet iš nenykstančios sėklos gyvu ir amžinai pasiliekančiu Dievo žodžiu. 
\par 24 Mat “kiekvienas kūnas­tartum žolynas, ir visa žmogaus garbė tarsi žolyno žiedas. Žolynas sudžiūsta, ir jo žiedas nubyra, 
\par 25 bet Viešpaties žodis išlieka per amžius”. Toks yra jums paskelbtas Evangelijos žodis.


\chapter{2}


\par 1 Taigi, atmetę visokį blogį, visokią klastą ir veidmainystes, pavyduliavimus ir visokias apkalbas, 
\par 2 lyg naujagimiai trokškite tyro žodžio pieno, kad nuo jo augtumėte išgelbėjimui, 
\par 3 jeigu tikrai paragavote, koks Viešpats yra maloningas. 
\par 4 Ženkite prie Jo, gyvojo akmens, tiesa, žmonių atmesto, bet Dievo išrinkto, brangaus, 
\par 5 ir patys, kaip gyvieji akmenys, statydinkitės į dvasinius namus, kad būtumėte šventa kunigystė ir atnašautumėte dvasines aukas, priimtinas Dievui per Jėzų Kristų. 
\par 6 Todėl Rašte pasakyta: “Štai dedu Sione kertinį akmenį, rinktinį, brangų; ir kas tiki Jį, nebus sugėdintas”. 
\par 7 Tad jums, kurie tikite, Jis yra brangus, o nepaklūstantiems “tas statytojų atmestas akmuo tapo kertiniu akmeniu, 
\par 8 suklupimo akmeniu ir papiktinimo uola”. Jie suklumpa, neklausydami žodžio; tam jie ir skirti. 
\par 9 O jūs esate “išrinktoji giminė, karališkoji kunigystė, šventoji tauta, ypatingi žmonės, kad skelbtumėte dorybes” To, kuris pašaukė jus iš tamsybių į savo nuostabią šviesą. 
\par 10 Seniau ne tauta, dabar Dievo tauta, seniau neradę gailestingumo, dabar jį suradę. 
\par 11 Mylimieji, maldauju jus kaip svečius ir ateivius: susilaikykite nuo kūno geidulių, kurie kovoja prieš sielą. 
\par 12 Jūsų elgesys tarp pagonių tebūna kilnus, kad jie už tai, už ką šmeižia jus kaip piktadarius, pamatę jūsų gerus darbus, imtų šlovinti Dievą aplankymo dieną. 
\par 13 Būkite klusnūs kiekvienai žmonių valdžiai dėl Viešpaties: tiek karaliui, kaip vyriausiajam, 
\par 14 tiek valdytojams, kaip jo pasiųstiems bausti piktadarių ir pagirti geradarių. 
\par 15 Mat tokia Dievo valia, kad, darydami gera, nutildytumėte neprotingų žmonių neišmanymą. 
\par 16 Elkitės kaip laisvi; ne kaip tie, kurie laisve pridengia blogį, bet kaip Dievo tarnai. 
\par 17 Gerbkite visus, mylėkite broliją, bijokite Dievo, gerbkite karalių. 
\par 18 Tarnai, su visa baime būkite klusnūs šeimininkams, ne tik geriems ir švelniems, bet ir rūstiems. 
\par 19 Girtina, jeigu kas dėl Dievo pažinimo pakelia skausmus, nekaltai kentėdamas. 
\par 20 Menka garbė, jei jūs kantrūs, kai esate plakami už nusikaltimus. Bet kai esate kantrūs, darydami gera ir kentėdami, Dievo akyse tai verta pagyrimo. 
\par 21 Juk jūs tam pašaukti; ir Kristus kentėjo už mus, palikdamas mums pavyzdį, kad eitumėte Jo pėdomis. 
\par 22 Jis “nepadarė nuodėmės, ir Jo lūpose nerasta klastos”. 
\par 23 Šmeižiamas neatsakė tuo pačiu, kentėdamas negrasino, bet pavedė save Tam, kuris teisia teisingai. 
\par 24 Jis pats savo kūne užnešė mūsų nuodėmes ant medžio, kad mirę nuodėmėms, gyventume teisumui. “Jo žaizdomis jūs buvote išgydyti”. 
\par 25 Jūs buvote tarsi klaidžiojančios avys, o dabar sugrįžote pas savo sielų Ganytoją ir Sargą.


\chapter{3}


\par 1 Jūs, žmonos, būkite klusnios savo vyrams, kad tie, kurie neklauso žodžio, ir be žodžio būtų laimėti savo žmonų elgesiu, 
\par 2 matydami jūsų gyvenimo skaistumą ir dievobaimingumą. 
\par 3 Tegu puošia jus ne išorė­supinti plaukai, aukso papuošalai ar ištaigingi drabužiai,­ 
\par 4 bet paslėptas širdies žmogus: nenykstančia, romia ir taikinga dvasia, kuri labai brangi Dievo akyse. 
\par 5 Juk kadaise taip ir puošdavosi šventos moterys, kurios turėjo viltį Dieve, būdamos klusnios savo vyrams. 
\par 6 Taip Sara klausė Abraomo ir vadino jį viešpačiu. Jūs tapote jos dukterimis, darydamos gera ir nebijodamos jokių bauginimų. 
\par 7 Ir jūs, vyrai, gyvenkite su žmonomis supratingai, kaip su silpnesniu indu, gerbdami jas kaip gyvenimo malonės bendrapaveldėtojas, kad jūsų maldos nebūtų trukdomos. 
\par 8 Galiausiai visi būkite vienminčiai, užjaučiantys kitus, mylintys brolius, gailestingi, draugiški. 
\par 9 Neatsilyginkite piktu už pikta ar keiksmu už keiksmą, bet, priešingai, laiminkite, žinodami, kad ir patys esate pašaukti paveldėti palaiminimo. 
\par 10 “Kas trokšta mylėti gyvenimą ir matyti gerų dienų, tepažaboja liežuvį nuo pikto ir lūpas nuo klastingų kalbų. 
\par 11 Tegu jis nusigręžia nuo pikto ir daro gera, teieško ramybės ir tesiveja ją. 
\par 12 Viešpaties žvilgsnis lydi teisiuosius, ir Jo ausys girdi jų maldas, bet Viešpaties veidas­prieš darančius pikta”. 
\par 13 Kas gi jums pakenks, jei darysite gera? 
\par 14 Bet jei jums ir tektų kentėti už teisumą,­jūs palaiminti! “Jų gąsdinimo neišsigąskite ir nesutrikite”. 
\par 15 Šventu laikykite Viešpatį Dievą savo širdyse, visada pasiruošę atsakyti kiekvienam klausiančiam apie jumyse esančią viltį romiai ir pagarbiai, 
\par 16 turėdami tyrą sąžinę, kad šmeižiantys jūsų gerą elgesį Kristuje liktų sugėdinti. 
\par 17 Geriau, jei tokia būtų Dievo valia, kentėti už gerus darbus negu už piktus. 
\par 18 Ir Kristus vieną kartą kentėjo už nuodėmes, teisusis už neteisiuosius, kad mus nuvestų pas Dievą, beje, kūnu numarintas, bet atgaivintas Dvasia. 
\par 19 Ja Jis nužengė žemyn ir skelbė kalėjime esančioms dvasioms, 
\par 20 kurios kadaise buvo neklusnios, kai Nojaus dienomis Dievo kantrybė laukė, bestatant arką, kuria nedaugelis, tai yra aštuonios sielos, buvo išgelbėtos vandeniu. 
\par 21 Ir mus dabar gelbsti tų įvykių vaizdinys­krikštas. Jis nėra kūno nešvaros nuplovimas, bet grynos sąžinės atsakas Dievui per prisikėlimą Jėzaus Kristaus, 
\par 22 kuris, įžengęs į dangų, yra Dievo dešinėje; Jam yra pavaldūs angelai, valdžios ir jėgos.


\chapter{4}


\par 1 Kadangi Kristus kentėjo kūnu už mus, tai ir jūs apsiginkluokite ta pačia mintimi,­nes kas kenčia kūnu, tas pametė nuodėmę, 
\par 2 kad likusį laiką kūne gyventų nebe žmonių aistromis, o Dievo valia. 
\par 3 Gana, kad praėjusį laiką buvome pasidavę pagonių valiai ir gyvenome gašliai, geidulingai, girtuokliavome, ūžavome, puotavome ir pasiduodavome bjaurioms stabmeldystėms. 
\par 4 Todėl jiems stebėtina, kad jūs nebebėgate kartu su jais pasinerti į tą patį ištvirkimo potvynį, ir jie piktžodžiauja. 
\par 5 Jie turės duoti apyskaitą Tam, kuris pasiruošęs teisti gyvuosius ir mirusiuosius. 
\par 6 Todėl buvo paskelbta Evangelija ir mirusiems, kad jie, nors ir nuteisti kūne kaip žmonės, gyventų dvasia kaip Dievas. 
\par 7 Visų dalykų galas arti. Todėl būkite blaivūs ir budėkite maldose. 
\par 8 Visų pirma karštai mylėkite vienas kitą, nes meilė uždengia gausybę nuodėmių. 
\par 9 Būkite tarpusavyje svetingi be murmėjimo. 
\par 10 Tarnaukite vieni kitiems kaip geri visokeriopos Dievo malonės tvarkytojai, sulig kiekvieno gautąja dovana. 
\par 11 Jei kas kalba, tekalba kaip Dievo žodžius; jei kas tarnauja, tegul tarnauja pagal Dievo teikiamus sugebėjimus, kad visuose dalykuose per Jėzų Kristų būtų pašlovintas Dievas. Jam šlovė ir valdžia per amžių amžius! Amen. 
\par 12 Mylimieji, nesistebėkite, kad jus degina ugnis, lyg jums būtų atsitikę kas keista, nes taip darosi jums išbandyti. 
\par 13 Verčiau džiaukitės, dalyvaudami Kristaus kentėjimuose, kad ir tada, kai Jo šlovė apsireikš, galėtumėte džiūgauti dideliu džiaugsmu. 
\par 14 Jei jus užgaulioja dėl Jėzaus vardo,­jūs palaiminti, nes šlovės ir Dievo Dvasia ilsisi ant jūsų. Jų Ji keikiama, o jūsų­šlovinama. 
\par 15 Tik tegul niekas iš jūsų nekenčia kaip žmogžudys, vagis, piktadarys ar įkyruolis, besikišąs į kitų reikalus. 
\par 16 Bet jei kenčia kaip krikščionis, tegul nesigėdija, o tešlovina dėl to Dievą. 
\par 17 Nes jau metas prasidėti teismui nuo Dievo namų; ir jeigu jis pirmiausia prasideda nuo mūsų, tai koks galas laukia tų, kurie neklauso Dievo Evangelijos?! 
\par 18 Ir “jeigu teisusis vos ne vos išsigelbės, tai kur pasidės bedievis ir nusidėjėlis!” 
\par 19 Todėl tie, kurie kenčia pagal Dievo valią, tepaveda savo sielas Jam, ištikimajam Kūrėjui, darydami gera.


\chapter{5}


\par 1 Jūsų vyresniuosius raginu aš, irgi vyresnysis, Kristaus kentėjimų liudytojas ir dalyvis šlovės, kuri bus apreikšta: 
\par 2 ganykite pas jus esančią Dievo kaimenę, prižiūrėdami ją ne iš prievartos, bet noriai, ne dėl nešvaraus pelno, bet uoliai, 
\par 3 ne kaip viešpataujantys jums patikėtiems, bet būdami pavyzdžiu kaimenei. 
\par 4 O kai pasirodys Vyriausiasis Ganytojas, jūs gausite nevystantį šlovės vainiką. 
\par 5 Taip pat jūs, jaunesnieji, būkite klusnūs vyresniesiems. Ir visi, paklusdami vieni kitiems, apsivilkite nuolankumu, nes “Dievas išdidiems priešinasi, o nuolankiesiems teikia malonę”. 
\par 6 Tad nusižeminkite po galinga Dievo ranka, kad Jis išaukštintų jus metui atėjus. 
\par 7 Meskite ant Jo savo rūpesčius, nes Jis jumis rūpinasi. 
\par 8 Būkite blaivūs ir budrūs, nes jūsų priešas velnias slankioja aplinkui kaip riaumojantis liūtas, tykodamas kurį nors praryti. 
\par 9 Pasipriešinkite jam tvirtu tikėjimu, žinodami, kad tokius pačius kentėjimus patiria jūsų broliai pasaulyje. 
\par 10 O visokeriopos malonės Dievas, pašaukęs mus į savo amžinąją šlovę Kristuje Jėzuje, pats jus, trumpai pakentėjusius, ištobulins, sutvirtins, sustiprins ir pastatys ant tvirto pagrindo. 
\par 11 Jam šlovė ir valdžia per amžių amžius! Amen. 
\par 12 Per Silvaną, ištikimąjį brolį,­tokiu jį laikau,­aš jums trumpai parašiau, ragindamas ir liudydamas, kad tai yra tikroji Dievo malonė, kurioje jūs stovite. 
\par 13 Jus sveikina kartu išrinktoji Babilone ir mano sūnus Morkus. 
\par 14 Sveikinkite vieni kitus meilės pabučiavimu! Ramybė jums visiems, kurie esate Kristuje Jėzuje! Amen.


\end{document}