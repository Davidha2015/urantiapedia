\begin{document}

\title{Antrasis Petro laiškas}


\chapter{1}


\par 1 Simonas Petras, Jėzaus Kristaus tarnas ir apaštalas, gavusiems su mumis lygiai brangų tikėjimą mūsų Dievo ir Gelbėtojo Jėzaus Kristaus teisumu. 
\par 2 Malonė ir ramybė tepadaugėja jums Dievo ir mūsų Viešpaties Jėzaus pažinimu. 
\par 3 Jo dieviška jėga padovanojo mums viską, ko reikia gyvenimui ir dievotumui, per pažinimą To, kuris pašaukė mus šlove ir dorybe. 
\par 4 Drauge Jis davė mums be galo didžius bei brangius pažadus, kad per juos taptume dieviškosios prigimties dalininkais, ištrūkę iš sugedimo, kuris sklinda pasaulyje geiduliais. 
\par 5 Todėl, parodydami visą stropumą, praturtinkite savo tikėjimą dorybe, dorybę—pažinimu, 
\par 6 pažinimą­susivaldymu, susivaldymą­ištverme, ištvermę­maldingumu, 
\par 7 maldingumą­brolybe, brolybę­ meile. 
\par 8 Jei šie dalykai jumyse gyvuoja ir tarpsta, jie neduoda jums apsileisti ir likti bevaisiams mūsų Viešpaties Jėzaus Kristaus pažinime. 
\par 9 O kam jų trūksta, tas aklas ir trumparegis, užmiršęs, kad yra apvalytas nuo savo senųjų nuodėmių. 
\par 10 Todėl, broliai, dar uoliau stenkitės sutvirtinti savąjį pašaukimą ir išrinkimą. Tai darydami, jūs niekada nesuklupsite. 
\par 11 Šitaip dar plačiau atsivers jums įėjimas į amžinąją mūsų Viešpaties ir Gelbėtojo Jėzaus Kristaus karalystę. 
\par 12 Todėl aš niekad nesiliausiu jums priminęs šiuos dalykus, nors jūs ir žinote juos ir esate sutvirtinti turimoje tiesoje. 
\par 13 Manau, kad teisinga, kol esu šioje palapinėje, žadinti jus paraginimais. 
\par 14 Žinau, kad greitai ateis mano palapinės nugriovimo metas, kaip ir mūsų Viešpats Jėzus Kristus man apreiškė. 
\par 15 Bet aš pasistengsiu, kad ir man iškeliavus, jūs kiekvienu metu galėtumėte tai prisiminti. 
\par 16 Juk mes skelbėme jums mūsų Viešpaties Jėzaus Kristaus galybę ir atėjimą, ne mėgdžiodami gudriai išgalvotas pasakas, bet kaip savo akimis matę Jo didybę liudytojai. 
\par 17 Jis gavo iš Dievo Tėvo garbę ir šlovę, kai iš tobulybės šlovės nuskambėjo Jam balsas: “Šitas yra mano mylimasis Sūnus, kuriuo Aš gėriuosi”. 
\par 18 Tą balsą mes girdėjome aidint iš dangaus, kai buvome su Juo ant šventojo kalno. 
\par 19 Taip pat mes turime dar tvirtesnį pranašų žodį. Jūs gerai darote, laikydamiesi jo tarsi žiburio, šviečiančio tamsioje vietoje, kol išauš diena ir jūsų širdyse užtekės aušrinė žvaigždė. 
\par 20 Pirmiausia žinokite, kad jokia Rašto pranašystė negali būti savavališkai aiškinama, 
\par 21 nes pranašystė niekada nėra atėjusi žmogaus valia, bet kalbėjo Šventosios Dvasios įkvėpti šventi Dievo žmonės.


\chapter{2}


\par 1 Buvo tautoje ir netikrų pranašų, kaip ir tarp jūsų bus netikrų mokytojų, kurie paslapčia įves pražūtingų erezijų, išsigindami net juos atpirkusio Viešpaties, ir užsitrauks greitą žlugimą. 
\par 2 Daugelis paseks jų pražūtingais keliais, ir dėl jų bus piktžodžiaujama tiesos keliui. 
\par 3 Iš godumo jie išnaudos jus suktais žodžiais. Bet nuo seno pasmerkimas jų laukia ir žuvimas nesnaudžia. 
\par 4 Dievas nepagailėjo nusidėjusių angelų, bet surišo juos tamsos raiščiais giliausiose pragaro gelmėse, kur laiko juos teismui. 
\par 5 Jis nepagailėjo senojo pasaulio, tik išsaugojo teisumo šauklį Nojų ir dar septynetą, kai siuntė bedievių pasauliui tvaną. 
\par 6 Paversdamas pelenais Sodomos ir Gomoros miestus, pasmerkė juos žlugti ir taip davė pavyzdį ateities bedieviams. 
\par 7 Išgelbėjo teisųjį Lotą, varginamą nedorėlių palaido elgesio, 
\par 8 nes tarp jų gyvenantis šis teisusis diena iš dienos vargino savo teisią sielą, matydamas ir girdėdamas nedorus darbus. 
\par 9 Viešpats žino, kaip išgelbėti pamaldžiuosius nuo išbandymo ir kaip išlaikyti nedoruosius teismo dienai ir bausmei, 
\par 10 o ypač tuos, kurie pasiduoda nešvariems kūno geismams ir niekina viešpatystę. Įžūlūs ir savavaliai! Jie nesudreba, piktžodžiaudami šlovingiesiems, 
\par 11 tuo tarpu angelai, aukštesni jėga ir galia, Viešpaties akivaizdoje neištaria jiems piktžodiško kaltinimo. 
\par 12 Bet jie, kaip neprotingi gyvuliai, gimę sugavimui ir užmušimui, piktžodžiauja tam, ko nesupranta, ir pražus savo sugedime, 
\par 13 gaudami atpildą už nusikaltimus. Jie laiko pramoga lėbauti dienos metu. Jie susitepę ir iškrypę, smarkaudami savo apgaulėmis vaišinasi su jumis. 
\par 14 Jų akys kupinos svetimavimo, nepasotinamos nuodėmės. Jie suvedžioja svyruojančias sielas. Jų širdis išlavinta godumo. Jie prakeikimo vaikai. 
\par 15 Palikę teisingą kelią, jie nuklydo ir pasuko Bosoro sūnaus Balaamo keliu, kuris pamėgo neteisumo atlygį, 
\par 16 tačiau buvo subartas dėl savo nedorybės: nebylus asilas prabilo žmogaus balsu ir sutrukdė pranašo beprotystę. 
\par 17 Jie yra šaltiniai be vandens, audros genami debesys; jiems skirta juodžiausia tamsybė per amžius. 
\par 18 Skelbdami išpūstas ir tuščias kalbas, kūno geismais ir pasileidimu jie suvilioja tuos, kurie yra vos pasprukę nuo gyvenančių paklydime. 
\par 19 Jie žada šiems laisvę, patys būdami sugedimo vergai: juk nugalėtasis tampa nugalėjusiojo vergu. 
\par 20 Bet jeigu, ištrūkę iš pasaulio purvyno Viešpaties ir Gelbėtojo Jėzaus Kristaus pažinimu, jie ir vėl jame įklimpę pralaimi, tai jiems paskui darosi blogiau negu pirma. 
\par 21 Jiems būtų buvę geriau iš viso nepažinti teisumo kelio, negu, jį pažinus, nusigręžti nuo jiems duoto švento įsakymo. 
\par 22 Jiems nutiko, kaip sako teisinga patarlė: “Šuo sugrįžta prie savo vėmalo”, ir: “Išmaudyta kiaulė vėl voliojasi purvyne”.


\chapter{3}


\par 1 Mylimieji, tai jau antras laiškas, kurį jums rašau. Šiuose laiškuose žadinu jūsų tyras mintis prisiminimais, 
\par 2 kad atsimintumėte šventųjų pranašų iš anksto paskelbtus žodžius ir mūsų­Viešpaties ir Gelbėtojo apaštalų­įsakymą. 
\par 3 Pirmiausia žinokite, kad paskutinėmis dienomis pasirodys šaipūnai, gyvenantys savo geiduliais 
\par 4 ir kalbantys: “Kur Jo atėjimo pažadas? Juk nuo to laiko, kai užmigo protėviai, visa pasilieka kaip buvę nuo sutvėrimo pradžios”. 
\par 5 Mat jiems, to norintiems, yra paslėpta, kad nuo seno buvo dangūs ir žemė, iš vandens ir per vandenį sutvarkyta Dievo žodžiu. 
\par 6 Todėl ir ano meto pasaulis žuvo, vandeniu užtvindytas. 
\par 7 O dabartiniai dangūs ir žemė tuo pačiu žodžiu palaikomi ugniai, saugomi teismo dienai ir bedievių žmonių žuvimui. 
\par 8 Tačiau, mylimieji, vienas dalykas neturi likti jūsų nepastebėtas: viena diena pas Viešpatį yra kaip tūkstantis metų, ir tūkstantis metų­kaip viena diena. 
\par 9 Viešpats nedelsia ištesėti savo pažado, kaip kai kurie mano, bet kantriai elgiasi su mumis, nenorėdamas, kad kuris pražūtų, bet kad visi atsiverstų. 
\par 10 O Viešpaties diena ateis kaip vagis naktį. Tada dangūs praeis su smarkiu ūžesiu, elementai sutirps karštyje, ir žemė su savo kūriniais sudegs. 
\par 11 Jeigu visa taip suirs, tai kaip reikėtų pasižymėti šventu elgesiu ir dievotumu jums, 
\par 12 laukiantiems ir skubinantiems Dievo dienos atėjimą, kai dangūs suirs liepsnose ir elementai sutirps iš karščio! 
\par 13 Tačiau mes pagal Jo pažadą laukiame naujo dangaus ir naujos žemės, kuriuose gyvena teisumas. 
\par 14 Todėl, mylimieji, šito laukdami, stenkitės, kad Jis rastų jus taikoje, nesuteptus ir nepeiktinus. 
\par 15 Mūsų Viešpaties kantrumą laikykite išgelbėjimu, kaip jums parašė ir mūsų mylimas brolis Paulius pagal jam duotą išmintį; 
\par 16 jis taip kalba apie šituos dalykus visuose laiškuose. Juose esama sunkiai suprantamų dalykų, kuriuos neišmokyti ir svyruojantys iškraipo, aiškindami, kaip ir kitus Raštus, savo pačių pražūčiai. 
\par 17 Tad jūs, mylimieji, iš anksto tai žinodami, saugokitės, kad, nedorėlių paklydimo traukiami, nenupultumėte nuo savo stiprybės. 
\par 18 Aukite malone ir mūsų Viešpaties ir Gelbėtojo Jėzaus Kristaus pažinimu. Jam šlovė dabar ir per amžius! Amen.



\end{document}