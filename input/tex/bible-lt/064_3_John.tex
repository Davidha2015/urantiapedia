\begin{document}

\title{Trečiasis Jono laiškas}

\chapter{1}


\par 1 Vyresnysis mylimajam Gajui, kurį myliu tiesoje. 
\par 2 Mylimasis, aš meldžiu, kad tau visame kame gerai sektųsi, kad būtum sveikas,­taip, kaip gerai sekasi tavo sielai. 
\par 3 Aš labai apsidžiaugiau, kai atvykę broliai paliudijo apie tavo tiesą, kaip tu vaikščioji tiesoje. 
\par 4 Neturiu didesnio džiaugsmo, kaip girdėti, jog mano vaikai gyvena tiesoje. 
\par 5 Mielasis, tu ištikimai elgiesi, pagelbėdamas broliams ir atvykstantiems iš kitur. 
\par 6 Jie paliudijo bažnyčiai apie tavo meilę. Tu puikiai padarysi, išruošdamas juos į kelionę taip, kaip Dievui patinka, 
\par 7 nes jie išvyko Jo vardo labui, nieko neimdami iš pagonių. 
\par 8 Mes turime tokius priimti, kad taptume tiesos bendradarbiais. 
\par 9 Aš parašiau bažnyčiai, bet mėgstantis jiems vadovauti Diotrefas nepriima mūsų. 
\par 10 Todėl, jei atvyksiu, priminsiu jo darbus, kuriuos jis daro, skleisdamas apie mus piktas kalbas; maža to, nei pats nepriima brolių, nei kitiems, kurie norėtų priimti, neleidžia ir išmeta juos iš bažnyčios. 
\par 11 Mielasis, nesek tuo, kas pikta, bet tuo, kas gera. Kuris daro gera, yra iš Dievo, o kuris pikta, nėra matęs Dievo. 
\par 12 Apie Demetriją gerai liudija visi ir pati tiesa. Ir mes liudijame, o tu žinai, kad mūsų liudijimas tikras. 
\par 13 Dar daug ką turėčiau tau parašyti, bet nenoriu rašyti rašalu ir plunksna. 
\par 14 Tikiuosi greitai pamatyti tave ir pasikalbėti iš lūpų į lūpas. Ramybė tau! Sveikina tave bičiuliai. Sveikink draugus pavardžiui!



\end{document}