\begin{document}

\title{Apreiškimas Jonui}

\chapter{1}


\par 1 Jėzaus Kristaus apreiškimas, kurį Dievas Jam davė, kad Jis atskleistų savo tarnams, kas turi greitai įvykti. Per savo pasiųstą angelą Jis padarė jį žinomą savajam tarnui Jonui, 
\par 2 kuris paliudijo Dievo žodį bei Jėzaus Kristaus liudijimą­visa, ką buvo matęs. 
\par 3 Palaimintas, kas skaito bei klauso šios pranašystės žodžių ir laikosi, kas joje parašyta, nes laikas yra arti. 
\par 4 Jonas septynioms Azijos bažnyčioms: malonė ir ramybė jums nuo To, kuris yra, kuris buvo ir kuris ateina, ir nuo septynių dvasių, esančių prieš Jo sostą, 
\par 5 ir nuo Jėzaus Kristaus, ištikimojo Liudytojo, mirusiųjų Pirmagimio, žemės karalių Valdovo. Tam, kuris pamilo mus ir nuplovė savo krauju mūsų nuodėmes, 
\par 6 ir padarė iš mūsų karalystę bei kunigus savo Dievui ir Tėvui,­ Jam šlovė ir galybė per amžių amžius! Amen. 
\par 7 Štai Jis ateina su debesimis, ir išvys Jį kiekviena akis, net ir tie, kurie Jį perdūrė; ir dėl Jo raudos visos žemės giminės. Taip, amen! 
\par 8 “Aš esu Alfa ir Omega, Pradžia ir Pabaiga”,­sako Viešpats, kuris yra, kuris buvo ir kuris ateina, Visagalis. 
\par 9 Aš, Jonas, jūsų brolis ir sielvarto, karalystės ir Jėzaus Kristaus kantrybės bendrininkas, buvau saloje, vardu Patmas, dėl Dievo žodžio ir Jėzaus Kristaus liudijimo. 
\par 10 Aš buvau Dvasioje Viešpaties dieną ir išgirdau už savo nugaros galingą balsą, tarsi trimitą, 
\par 11 sakantį: “Aš esu Alfa ir Omega, Pirmasis ir Paskutinysis. Ką matai, surašyk į knygą ir pasiųsk septynioms bažnyčioms Azijoje: į Efezą, į Smirną, į Pergamą, į Tiatyrus, į Sardus, į Filadelfiją ir į Laodikėją”. 
\par 12 Tuomet aš atsigręžiau pažiūrėti balso, kalbančio su manimi, ir atsigręžęs išvydau septynis aukso žibintuvus, 
\par 13 o septynių žibintuvų viduryje­ panašų į Žmogaus Sūnų, apsivilkusį ilga mantija ir persijuosusį per krūtinę aukso juosta. 
\par 14 Jo galva ir plaukai buvo balti kaip balčiausia vilna ar sniegas, Jo akys tarsi ugnies liepsna, 
\par 15 Jo kojos panašios į krosnyje įkaitintą skaistvarį, ir Jo balsas buvo tarytum daugybės vandenų šniokštimas. 
\par 16 Dešinėje rankoje Jis laikė septynias žvaigždes, iš Jo burnos ėjo aštrus dviašmenis kalavijas, o Jo veidas buvo tarytum saulė, spindinti visu skaistumu. 
\par 17 Jį išvydęs, aš puoliau Jam po kojų tarsi negyvas. Bet Jis uždėjo ant manęs savo dešinę ir prabilo: “Nebijok! Aš esu Pirmasis, ir Paskutinysis, 
\par 18 ir Gyvasis. Aš buvau numiręs ir štai esu gyvas per amžius. Amen. Aš turiu mirties ir pragaro raktus. 
\par 19 Tad užrašyk tai, ką regėjai, kas yra ir kas vėliau įvyks. 
\par 20 Štai septynių žvaigždžių, kurias matei mano dešinėje, ir septynių aukso žibintuvų paslaptis: septynios žvaigždės­tai septynių bažnyčių angelai, o septyni žibintuvai, kuriuos matei,­tai septynios bažnyčios”.


\chapter{2}


\par 1 “Efezo bažnyčios angelui rašyk: ‘Tai sako Tas, kuris laiko savo dešinėje septynias žvaigždes, kuris vaikščioja tarp septynių aukso žibintuvų. 
\par 2 Aš žinau tavo darbus, tavo triūsą ir tavo kantrybę. Žinau, kad tu negali pakęsti piktųjų ir ištyrei tuos, kurie sakosi esą apaštalai, bet tokie nėra, ir radai juos esant melagius. 
\par 3 Tu ištvėrei, esi kantrus ir dėl mano vardo triūsei ir nepailsai. 
\par 4 Bet Aš turiu prieš tave tai, kad palikai savo pirmąją meilę. 
\par 5 Taigi prisimink, nuo kur nupuolei, atgailauk ir vėl imkis pirmykščių darbų, o jeigu ne,­jei neatgailausi,­greitai ateisiu ir patrauksiu iš vietos tavo žibintuvą. 
\par 6 Savo naudai tu turi, kad nekenti nikolaitų darbų, kurių ir Aš nekenčiu’. 
\par 7 Kas turi ausis, teklauso, ką Dvasia sako bažnyčioms: ‘Nugalėtojui Aš duosiu valgyti nuo gyvybės medžio, esančio Dievo rojaus viduryje’ ”. 
\par 8 “Smirnos bažnyčios angelui rašyk: ‘Tai sako Pirmasis ir Paskutinysis, kuris buvo miręs ir vėl grįžo į gyvenimą. 
\par 9 Aš žinau tavo darbus ir priespaudą, ir tavo skurdą,­o vis dėlto tu turtingas!­ir kaip tau piktžodžiauja tie, kurie sakosi esą žydai, bet nėra tokie, o tik šėtono sinagoga. 
\par 10 Nebijok būsimųjų kentėjimų. Štai velnias įmes kai kuriuos jūsiškius į kalėjimą, kad būtumėte išbandyti. Jūsų laukia dešimties dienų priespauda. Būk ištikimas iki mirties, ir Aš tau duosiu gyvenimo vainiką’. 
\par 11 Kas turi ausis, teklauso, ką Dvasia sako bažnyčioms: ‘Nugalėtojas nenukentės nuo antrosios mirties’ ”. 
\par 12 “Pergamo bažnyčios angelui rašyk: ‘Tai sako Tas, kuris turi aštrų dviašmenį kalaviją. 
\par 13 Aš žinau tavo darbus ir kur tu gyveni: ten, kur šėtono sostas. Bet tu tvirtai laikaisi mano vardo ir neišsigynei mano tikėjimo net tomis dienomis, kada pas jus,­kur gyvena šėtonas,­buvo nužudytas mano ištikimasis liudytojas Antipas. 
\par 14 Vis dėlto turiu šį tą prieš tave: tu tenai turi besilaikančių Balaamo mokslo, kuris mokė Balaką suvedžioti Izraelio sūnus, kad šie valgytų stabams aukojamas aukas ir ištvirkautų. 
\par 15 Ir tu taip pat turi besilaikančių nikolaitų mokslo, kurio Aš nekenčiu. 
\par 16 Tad atgailauk! O jeigu ne, Aš greitai ateisiu ir kovosiu su jais savo burnos kalaviju’. 
\par 17 Kas turi ausis, teklauso, ką Dvasia sako bažnyčioms: ‘Nugalėtojui Aš duosiu paslėptos manos ir baltą akmenėlį, o ant akmenėlio bus įrašytas naujas vardas, kurio niekas nežino, tiktai gavėjas’ ”. 
\par 18 “Tiatyrų bažnyčios angelui rašyk: ‘Tai sako Dievo Sūnus, kurio akys tarytum ugnies liepsna ir kurio kojos panašios į skaistvarį. 
\par 19 Žinau tavo darbus, meilę, tikėjimą, tarnavimą, kantrybę ir kad tavo paskutinieji darbai didesni už pirmuosius. 
\par 20 Bet Aš turiu šį tą prieš tave: tu leidi moteriškei Jezabelei, kuri sakosi esanti pranašė, mokyti bei suvedžioti mano tarnus, kad jie ištvirkautų ir valgytų stabams paaukotas aukas. 
\par 21 Aš jai daviau laiko atgailauti dėl ištvirkavimo, bet ji neatgailavo. 
\par 22 Štai Aš ją nublokšiu į ligos patalą, o ištvirkavusius su ja­į didelį sielvartą, jeigu jie neatgailaus dėl savo darbų. 
\par 23 Jos vaikus išžudysiu, ir visos bažnyčios sužinos, kad Aš esu Tas, kuris ištiria protus ir širdis; Aš atsilyginsiu jums kiekvienam pagal jūsų darbus. 
\par 24 O jums ir kitiems tiatyriečiams, kurie nesilaiko ano mokslo, kurie, kaip sakosi, nėra pažinę šėtono gelmių, Aš sakau: neužkrausiu jums kitokios naštos, 
\par 25 tiktai tvirtai laikykitės to, ką turite, iki ateisiu. 
\par 26 Tam, kuris nugali ir iki galo laikosi mano darbų, Aš ‘duosiu valdžią tautoms: 
\par 27 jis valdys jas geležine lazda, ir jos bus sudaužytos tarsi moliniai indai’,­kaip ir Aš esu gavęs valdžią iš savo Tėvo;­ 
\par 28 ir jam duosiu aušrinę žvaigždę’. 
\par 29 Kas turi ausis, teklauso, ką Dvasia sako bažnyčioms!”


\chapter{3}


\par 1 “Sardų bažnyčios angelui rašyk: ‘Tai sako Tas, kuris turi septynias Dievo Dvasias ir septynias žvaigždes. Aš žinau tavo darbus: tave vadina gyvu, o tu esi miręs. 
\par 2 Budėk ir stiprink, kas dar yra ir merdi! Aš neradau tavo darbų, kurie būtų pabaigti Dievo akivaizdoje. 
\par 3 Todėl prisimink, kaip priėmei ir išgirdai; laikykis to ir atgailauk! Jeigu nebudėsi, ateisiu kaip vagis, ir nežinosi, kurią valandą tave užklupsiu. 
\par 4 Vis dėlto tu Sarduose turi keletą vardų, kurie nesutepė savo drabužių. Jie vaikščios su manimi, apsirengę baltai, nes jie to verti. 
\par 5 Nugalėtojas bus aprengtas baltais drabužiais, ir jo vardo neištrinsiu iš gyvenimo knygos. Aš išpažinsiu jo vardą savo Tėvo ir Jo angelų akivaizdoje’. 
\par 6 Kas turi ausis, teklauso, ką Dvasia sako bažnyčioms”. 
\par 7 “Filadelfijos bažnyčios angelui rašyk: ‘Tai kalba Šventasis, Tikrasis, turintis Dovydo raktą,­Tas, kuris atidaro, ir niekas negali uždaryti, uždaro, ir niekas negali atidaryti. 
\par 8 Aš žinau tavo darbus. Štai Aš atvėriau prieš tave duris, ir niekas nebegali jų uždaryti; nedaug turi jėgų, bet išsaugojai mano žodį ir neatsižadėjai mano vardo. 
\par 9 Štai Aš tau duodu tuos iš šėtono sinagogos, kurie tvirtina, jog jie žydai, bet nėra, nes jie meluoja. Taigi padarysiu, kad jie ateitų, pultų tau po kojų ir suprastų, jog Aš pamilau tave. 
\par 10 Kadangi tu išlaikei mano kantrybės žodį, tai ir Aš tave apsaugosiu nuo išbandymo valandos, kuri ištiks visą pasaulį, kad būtų išmėginti žemės gyventojai. 
\par 11 Štai Aš veikiai ateinu. Tvirtai laikyk, ką turi, kad niekas neatimtų tavo vainiko. 
\par 12 Nugalėtoją Aš padarysiu ramsčiu savo Dievo šventykloje, ir jis jau nebeišeis laukan; užrašysiu ant jo savo Dievo vardą ir vardą savo Dievo miesto, naujosios Jeruzalės, nužengiančios iš dangaus nuo mano Dievo, ir savo naująjį vardą’. 
\par 13 Kas turi ausis, teklauso, ką Dvasia sako bažnyčioms!” 
\par 14 “Laodikėjos bažnyčios angelui rašyk: ‘Tai skelbia Amen, ištikimasis ir tikrasis Liudytojas, Dievo kūrinijos pradžia. 
\par 15 Žinau tavo darbus, jog esi nei šaltas, nei karštas. O, kad būtum arba šaltas, arba karštas! 
\par 16 Bet kadangi esi drungnas ir nei karštas, nei šaltas, Aš išspjausiu tave iš savo burnos. 
\par 17 Tu gi sakai: ‘Aš esu turtingas ir pralobęs, ir nieko man nebereikia’,­o nežinai, kad esi skurdžius, apgailėtinas, beturtis, aklas ir nuogas. 
\par 18 Aš tau patariu pirkti iš manęs išgryninto ugnyje aukso, kad pralobtum, baltus drabužius, kad apsirengtum ir nebūtų matoma tavo nuogumo gėda, ir tepalo pasitepti akims, kad praregėtum. 
\par 19 Tuos, kuriuos myliu, Aš baru ir drausminu; būk tad uolus ir atgailauk! 
\par 20 Štai Aš stoviu prie durų ir beldžiu: jei kas išgirs mano balsą ir atvers duris, Aš pas jį užeisiu ir vakarieniausiu su juo, o jis su manimi. 
\par 21 Nugalėtojui Aš duosiu atsisėsti šalia savęs, savo soste, kaip ir Aš nugalėjau ir atsisėdau šalia savo Tėvo, Jo soste’. 
\par 22 Kas turi ausis, teklauso, ką Dvasia sako bažnyčioms!”


\chapter{4}


\par 1 Paskui aš pažiūrėjau, ir štai atvertos durys danguje, ir pirmasis balsas, kurį girdėjau gaudžiant tarsi trimitą, kalbėjo: “Užženk čionai, ir tau parodysiu, kas toliau turi įvykti”. 
\par 2 Bematant mane ištiko Dvasios pagava. Ir štai danguje buvo sostas, o soste­Sėdintysis. 
\par 3 Jo išvaizda buvo panaši į jaspio ir sardžio brangakmenius, o vaivorykštė, juosianti sostą, buvo panaši į smaragdą. 
\par 4 Aplinkui sostą regėjau dvidešimt keturis sostus ir tuose sostuose sėdinčius dvidešimt keturis vyresniuosius baltais drabužiais, o jų galvas puošė aukso vainikai. 
\par 5 Nuo sosto ėjo žaibai, aidėjo balsai ir griaustiniai; septyni deglai liepsnojo prieš sostą, o tai yra septynios Dievo Dvasios. 
\par 6 Prieš sostą tviskėjo tarsi stiklo jūra, panaši į krištolą; sosto viduryje ir aplinkui sostą buvo keturios būtybės, pilnos akių iš priekio ir iš užpakalio. 
\par 7 Pirmoji būtybė buvo panaši į liūtą, antroji būtybė panaši į veršį, trečioji būtybė turinti tartum žmogaus veidą, ketvirtoji būtybė panaši į skrendantį erelį. 
\par 8 Kiekviena iš keturių būtybių turėjo po šešis sparnus; aplinkui ir viduryje jos buvo pilnos akių. Ir be paliovos, dieną ir naktį, jos šaukė: “Šventas, šventas, šventas, Viešpats, visagalis Dievas, kuris buvo, kuris yra ir kuris ateina!” 
\par 9 Ir kiekvieną kartą, kai būtybės teikė Sėdinčiajam soste, Gyvenančiajam per amžių amžius, šlovę, pagarbą ir padėką, 
\par 10 dvidešimt keturi vyresnieji parpuldavo prieš Sėdintįjį soste, pagarbindavo Gyvenantįjį per amžių amžius ir numesdavo savo vainikus priešais sostą, sakydami: 
\par 11 “Vertas esi, o Viešpatie, priimti šlovę, pagarbą ir jėgą, nes Tu visa sutvėrei­Tavo valia visa yra ir buvo sutverta”.


\chapter{5}


\par 1 Ir aš mačiau soste Sėdinčiojo dešinėje knygos ritinį, prirašytą iš vidaus ir iš viršaus, užantspauduotą septyniais antspaudais. 
\par 2 Ir pamačiau galingą angelą, skelbiantį garsiu balsu: “Kas yra vertas atversti knygą ir nuplėšti nuo jos antspaudus?” 
\par 3 Bet niekas nei danguje, nei žemėje, nei po žeme negalėjo atversti knygos nei pažiūrėti į ją. 
\par 4 Ir aš smarkiai verkiau, kad neatsirado verto atversti ir skaityti knygą ar pažiūrėti į ją. 
\par 5 Tada vienas iš vyresniųjų man tarė: “Neverk! Štai nugalėjo liūtas iš Judo giminės, Dovydo atžala, kad atverstų knygą ir nuplėštų septynis jos antspaudus”. 
\par 6 Aš pažvelgiau, ir štai sosto ir keturių būtybių bei vyresniųjų viduryje stovėjo Avinėlis, tarytum užmuštas, turintis septynis ragus ir septynias akis, kurios yra septynios Dievo Dvasios, siųstos į visą žemę. 
\par 7 Jis priėjo ir paėmė knygą iš soste Sėdinčiojo dešinės. 
\par 8 Kai Jis paėmė knygą, keturios būtybės ir dvidešimt keturi vyresnieji parpuolė prieš Avinėlį, kiekvienas laikydamas rankose arfą ir aukso indus, pilnus smilkalų, kas yra šventųjų maldos. 
\par 9 Ir jie giedojo naują giesmę, skelbdami: “Vertas esi paimti knygą ir atverti jos antspaudus, nes buvai užmuštas ir atpirkai Dievui savo krauju mus iš visų genčių, kalbų, tautų ir giminių. 
\par 10 Ir iš mūsų padarei mūsų Dievui karalystę bei kunigus, ir mes viešpatausime žemėje”. 
\par 11 Aš pažvelgiau ir išgirdau balsą daugybės angelų aplinkui sostą, būtybių ir vyresniųjų; jų skaičius buvo miriadų miriadai ir tūkstančių tūkstančiai. 
\par 12 Jie skelbė skambiu balsu: “Vertas Avinėlis, kuris buvo užmuštas, priimti galybę, ir turtus, ir išmintį, ir stiprybę, ir pagarbą, ir šlovę, ir palaiminimą”. 
\par 13 Ir girdėjau, kaip visi tvariniai, esantys danguje, žemėje, po žeme ir jūroje, ir visa, kas juose yra, skelbė: “Sėdinčiajam soste ir Avinėliui tebūnie palaiminimas, ir pagarba, ir šlovė, ir valdžia per amžių amžius!” 
\par 14 Keturios būtybės sakė: “Amen!”, o dvidešimt keturi vyresnieji puolė ant žemės ir pagarbino Gyvenantįjį per amžių amžius.


\chapter{6}


\par 1 Ir aš mačiau, kaip Avinėlis atplėšė pirmąjį iš antspaudų, ir išgirdau vieną iš keturių būtybių tartum griaustinio balsu šaukiant: “Ateik ir žiūrėk!” 
\par 2 Aš pažvelgiau, ir štai pasirodė baltas žirgas ir ant jo raitelis, turintis lanką. Jam buvo duotas vainikas, ir jis išjojo kaip nugalėtojas, kad dar nugalėtų. 
\par 3 Kai Jis atplėšė antrąjį antspaudą, aš išgirdau antrąją būtybę sakant: “Ateik ir žiūrėk!” 
\par 4 Ir pasirodė kitas žirgas, ugniaspalvis, ir jo raiteliui buvo duota atimti iš žemės taiką, kad žmonės vieni kitus žudytų; jam buvo duotas didelis kalavijas. 
\par 5 Kai Avinėlis atplėšė trečiąjį antspaudą, išgirdau trečiąją būtybę sakant: “Ateik ir žiūrėk!” Aš pažiūrėjau, ir štai pasirodė juodas žirgas, o raitelis turėjo savo rankoje svarstykles. 
\par 6 Ir aš girdėjau balsą keturių būtybių viduryje sakantį: “Kviečių saikas už denarą, trys miežių saikai už denarą, bet aliejui ir vynui nedaryk skriaudos!” 
\par 7 Kai atplėšė ketvirtąjį antspaudą, išgirdau ketvirtosios būtybės balsą sakant: “Ateik ir žiūrėk!” 
\par 8 Aš pažvelgiau, ir štai pasirodė palšas žirgas, o jo raitelio vardas buvo Mirtis, ir paskui jį sekė Pragaras. Jiems buvo duota valdžia ketvirtadalyje žemės žudyti kardu, badu, mirtimi ir žemės žvėrimis. 
\par 9 Kai Avinėlis atplėšė penktąjį antspaudą, pamačiau po aukuru sielas nužudytųjų dėl Dievo žodžio ir dėl liudijimo, kurio jie laikėsi. 
\par 10 Jie šaukė garsiu balsu, sakydami: “Kaip ilgai, Šventasis ir Teisusis Valdove, neteisi ir nekeršysi už mūsų kraują žemės gyventojams?!” 
\par 11 Tada kiekvienam iš jų buvo duotas baltas drabužis, ir jiems buvo pasakyta, kad dar truputį palūkėtų, kol skaičių papildys jų draugai ir broliai, kurie bus nužudyti kaip ir jie. 
\par 12 Aš mačiau, kai Jis atplėšė šeštąjį antspaudą. Ir štai kilo didelis žemės drebėjimas, saulė pasidarė juoda kaip ašutinis maišas, mėnulis tapo lyg kraujas, 
\par 13 o dangaus žvaigždės ėmė kristi žemėn, tarytum stipraus vėjo purtomas figmedis mestų dar neprinokusius vaisius. 
\par 14 Dangus nutolo tarsi suvyniojamas knygos rietimas, ir kiekvienas kalnas ir sala buvo išjudinti iš savo vietų. 
\par 15 Ir tada žemės karaliai, didžiūnai, karo vadai, turtuoliai, galiūnai, visi vergai ir visi laisvieji pasislėpė urvuose ir tarp kalnų uolų. 
\par 16 Jie šaukė kalnams ir uoloms: “Griūkite ant mūsų ir paslėpkite mus nuo Sėdinčiojo soste veido ir nuo Avinėlio rūstybės, 
\par 17 nes atėjo didi Jo rūstybės diena, ir kas gali išstovėti?!”


\chapter{7}


\par 1 Po to aš regėjau keturis angelus, stovinčius keturiuose žemės kampuose, laikančius keturis žemės vėjus, kad vėjas nepūstų nei žemėje, nei jūroje, nei į medžius. 
\par 2 Ir išvydau kitą angelą, pakylantį nuo saulėtekio, turintį gyvojo Dievo antspaudą. Jis šaukė skardžiu balsu keturiems angelams, kuriems buvo duota kenkti žemei ir jūrai: 
\par 3 “Nekenkite nei žemei, nei jūrai, nei medžiams, kol paženklinsime antspaudu savo Dievo tarnų kaktas!” 
\par 4 Ir aš išgirdau paženklintųjų skaičių: šimtas keturiasdešimt keturi tūkstančiai paženklintųjų iš visų Izraelio vaikų giminių: 
\par 5 Iš Judo giminės dvylika tūkstančių paženklintųjų, iš Rubeno giminės dvylika tūkstančių, iš Gado giminės dvylika tūkstančių, 
\par 6 iš Asero giminės dvylika tūkstančių, iš Neftalio giminės dvylika tūkstančių, iš Manaso giminės dvylika tūkstančių, 
\par 7 iš Simeono giminės dvylika tūkstančių, iš Levio giminės dvylika tūkstančių, iš Isacharo giminės dvylika tūkstančių, 
\par 8 iš Zabulono giminės dvylika tūkstančių, iš Juozapo giminės dvylika tūkstančių, iš Benjamino giminės dvylika tūkstančių paženklintųjų. 
\par 9 Paskui regėjau: štai milžiniška minia, kurios niekas negalėjo suskaičiuoti, iš visų giminių, genčių, tautų ir kalbų. Visi stovėjo priešais sostą ir Avinėlį, apsisiautę baltais apsiaustais, su palmių šakomis rankose. 
\par 10 Jie šaukė skambiu balsu: “Išgelbėjimas iš mūsų Dievo, sėdinčio soste, ir Avinėlio!” 
\par 11 Visi angelai, stovintys aplink sostą, vyresniuosius ir keturias būtybes, parpuolė prieš sostą veidais žemėn ir pagarbino Dievą, 
\par 12 sakydami: “Amen! Palaiminimas, ir šlovė, ir išmintis, ir padėka, ir garbė, ir jėga, ir galybė mūsų Dievui per amžių amžius! Amen!” 
\par 13 Tuomet vienas iš vyresniųjų tarė man: “Kas tokie yra ir iš kur atėjo tie, apsivilkę baltais apsiaustais?” 
\par 14 Aš jam atsakiau: “Viešpatie, tu žinai”. Ir jis man tarė: “Jie atėjo iš didelio suspaudimo. Jie išplovė savo apsiaustus ir juos išbaltino Avinėlio kraujyje. 
\par 15 Todėl jie yra prieš Dievo sostą ir tarnauja Jam dieną ir naktį Jo šventykloje, o Sėdintysis soste išskleis ant jų savo palapinę. 
\par 16 Jie daugiau nebealks ir nebetrokš, nebekepins jų saulė nei jokia kaitra. 
\par 17 Nes Avinėlis, kuris sosto viduryje, juos ganys ir vedžios prie gyvųjų vandens šaltinių, ir Dievas nušluostys kiekvieną ašarą nuo jų akių”.


\chapter{8}


\par 1 Kai Avinėlis atplėšė septintąjį antspaudą, danguje pusvalandžiui pasidarė tylu. 
\par 2 Ir aš išvydau septynis angelus, stovinčius Dievo akivaizdoje, ir jiems buvo įteikti septyni trimitai. 
\par 3 Atėjo dar vienas angelas ir atsistojo prie aukuro, laikydamas aukso smilkytuvą. Jam buvo duota daug smilkalų, kad jis aukotų juos su visų šventųjų maldomis ant auksinio aukuro priešais sostą. 
\par 4 Ir pakilo smilkalų dūmai su šventųjų maldomis iš angelo rankų Dievo akivaizdon. 
\par 5 Po to angelas paėmė smilkytuvą, pripildė jį aukuro ugnies ir sviedė į žemę. Sugriaudė griausmai, pakilo šauksmai, sublyksėjo žaibai, sudrebėjo žemė. 
\par 6 Septyni angelai, turintys septynis trimitus, pasiruošė trimituoti. 
\par 7 Sutrimitavo pirmasis angelas. Ir radosi kruša ir ugnis, sumišę su krauju, ir tai buvo numesta žemėn. Ir išdegė trečdalis medžių ir visa žaliuojanti žolė. 
\par 8 Sutrimitavo antrasis angelas. Ir tarytum didžiulis kalnas, liepsnojantis ugnimi, buvo sviestas į jūrą, ir trečdalis jūros pavirto krauju. 
\par 9 Trečdalis jūros padarų, turinčių gyvybę, išdvėsė, ir trečdalis laivų buvo sunaikinti. 
\par 10 Ir sutrimitavo trečiasis angelas. Tada iš dangaus nukrito didelė žvaigždė, liepsnodama tarsi deglas, ir nukrito ant trečdalio upių ir ant vandens šaltinių. 
\par 11 Žvaigždės vardas yra Metėlė. Ir pavirto trečdalis vandenų į metėlę, ir daugybė žmonių mirė nuo vandens, nes jis pasidarė kartus. 
\par 12 Ir sutrimitavo ketvirtasis angelas. Tuomet buvo užgautas trečdalis saulės, trečdalis mėnulio ir trečdalis žvaigždžių taip, kad jų trečdalis užtemo. Diena bei naktis prarado trečdalį šviesumo. 
\par 13 Aš regėjau ir girdėjau dangaus viduriu lekiantį angelą, kuris garsiu balsu skelbė: “Vargas, vargas, vargas žemės gyventojams nuo besirengiančių užtrimituoti kitų trijų angelų trimitų garsų!”


\chapter{9}


\par 1 Ir sutrimitavo penktasis angelas. Aš išvydau žvaigždę, nukritusią iš dangaus žemėn. Jai buvo duotas raktas nuo bedugnės šulinio. 
\par 2 Ji atidarė bedugnės šulinį, ir išsiveržė dūmai iš šulinio, tarytum iš milžiniškos krosnies. Ir aptemo saulė ir oras nuo šulinio dūmų, 
\par 3 o iš dūmų pasipylė žemėn skėriai, kuriems buvo duota galia, kaip turi galią žemės skorpionai. 
\par 4 Jiems buvo įsakyta nekenkti žemės žolei, nei jokiam žalumynui, nei jokiam medžiui, o vien tik žmonėms, kurie neturi savo kaktose Dievo antspaudo. 
\par 5 Ir jiems buvo leista ne žudyti žmones, bet kankinti penkis mėnesius; jų kankinimas it kankinimas skorpiono, kai jis įgelia žmogų. 
\par 6 Anomis dienomis žmonės ieškos mirties ir jos neras, trokš numirti, bet mirtis bėgs nuo jų. 
\par 7 Skėrių išvaizda panaši į žirgų, parengtų kautynėms. Ant jų galvų tartum vainikai, panašūs į auksą, o jų veidai­tartum žmonių veidai; 
\par 8 jie turėjo plaukus, panašius į moterų plaukus, o jų dantys buvo lyg liūtų dantys. 
\par 9 Jie turėjo šarvus tarsi geležinius krūtinšarvius, o jų sparnų garsas buvo kaip bildesys daugybės vežimų ir arklių, bėgančių į mūšį. 
\par 10 Jie turi uodegas, panašias į skorpionų, ir geluonis uodegose. Jie turi valdžią kenkti žmonėms per penkis mėnesius. 
\par 11 Jie turi sau karalių, bedugnės angelą, kurio vardas hebrajiškai Abadonas, o graikiškai tas vardas Apolionas. 
\par 12 Pirmoji neganda praėjo; štai iš paskos eina dar dvi negandos. 
\par 13 Ir sutrimitavo šeštasis angelas. Aš išgirdau balsą nuo keturių ragų auksinio aukuro, stovinčio Dievo akivaizdoje. 
\par 14 Jis sakė šeštajam angelui, turinčiam trimitą: “Paleisk keturis angelus, kurie yra surišti prie didžiosios Eufrato upės!” 
\par 15 Ir buvo atrišti keturi angelai, paruošti nustatytai valandai, dienai, mėnesiui ir metams išžudyti trečdalį žmonių. 
\par 16 Jų kariuomenės skaičius,­aš išgirdau skaičių,­buvo du miriadai miriadų. 
\par 17 Taigi aš mačiau regėjime žirgus ir raitelius su krūtinšarviais, ugniaspalviais, violetiniais ir geltonais; žirgų galvos atrodė kaip liūtų galvos, o iš jų nasrų veržėsi ugnis, dūmai ir siera. 
\par 18 Trečdalis žmonių žuvo nuo šitų trijų piktenybių­nuo ugnies, dūmų ir sieros, besiveržiančių iš jų nasrų. 
\par 19 Mat jų galia buvo jų nasruose ir jų uodegose. Jų uodegos panašios į gyvates ir turi galvas, kuriomis kenkia. 
\par 20 Bet likusieji žmonės, kurie nebuvo šitų piktenybių išžudyti, neatgailavo dėl savo rankų darbų, kad nebegarbintų demonų ir auksinių, sidabrinių, žalvarinių, akmeninių ir medinių stabų, kurie negali nei matyti, nei girdėti, nei vaikščioti. 
\par 21 Jie taip pat neatgailavo dėl savo žmogžudysčių, žyniavimų, ištvirkavimų ir vagysčių.


\chapter{10}


\par 1 Aš išvydau dar vieną galingą angelą, nužengiantį iš dangaus, apsisiautusį debesimi. Jo galvą supo vaivorykštė, veidas švytėjo kaip saulė ir kojos­tarsi ugnies stulpai. 
\par 2 Jis laikė rankoje išvyniotą knygelę. Ir Jis atsistojo dešiniąja koja ant jūros, o kairiąja ant sausumos, 
\par 3 ir ėmė šaukti galingu balsu tartum riaumojantis liūtas. Kai jis sušuko, atsiliepė septyni griaustiniai savais balsais. 
\par 4 Septyniems griaustiniams prabilus savais balsais, puoliausi rašyti, bet išgirdau iš dangaus balsą, sakantį man: “Užantspauduok, ką pasakė septyni griaustiniai, ir to nerašyk!” 
\par 5 O angelas, kurį mačiau stovint ant jūros ir ant sausumos, pakėlė savo ranką į dangų 
\par 6 ir prisiekė Gyvenančiuoju per amžių amžius, kuris sutvėrė dangų ir visa, kas jame, žemę ir visa, kas joje, bei jūrą ir visa, kas joje,­kad laiko daugiau nebebus, 
\par 7 bet septintojo angelo trimitavimo dienomis bus baigta Dievo paslaptis, kaip Jis yra paskelbęs Gerąją naujieną savo tarnams pranašams. 
\par 8 Tuomet balsas, kurį girdėjau iš dangaus, vėl ėmė kalbėti man ir tarė: “Eik, paimk atvyniotą knygelę iš angelo rankos, stovinčio ant jūros ir sausumos”. 
\par 9 Aš nuėjau pas angelą ir paprašiau, kad duotų man knygelę. O jis man tarė: “Imk ir suvalgyk ją! Ji bus karti viduriuose, bet burnoje ji bus saldi kaip medus”. 
\par 10 Ir aš paėmiau knygelę iš angelo rankos ir ją suvalgiau. Ji buvo mano burnoje saldi tarytum medus, bet kai prarijau, mano viduriuose ji apkarto. 
\par 11 Ir jis pasakė man: “Tu turi vėl pranašauti apie daugelį žmonių, tautų, kalbų ir karalių”.


\chapter{11}


\par 1 Man buvo duotas nendrinis matas, panašus į lazdą, ir angelas pasakė: “Kelkis ir išmatuok Dievo šventyklą, aukurą ir tuos, kurie tenai garbina. 
\par 2 O išorinį šventyklos kiemą praleisk, palik, nematuok jo, nes jis skirtas pagonims. Jie tryps šventąjį miestą keturiasdešimt du mėnesius. 
\par 3 Aš duosiu galią savo dviem liudytojams, ir jie, apsivilkę ašutinėmis, pranašaus tūkstantį du šimtus šešiasdešimt dienų”. 
\par 4 Jie yra du alyvmedžiai ir du žibintuvai, stovintys žemės Dievo akivaizdoje. 
\par 5 Ir jei kas panorės juos skriausti, iš jų burnos išsiverš ugnis ir praris jų priešininkus. Jei kas norės jiems kenkti, tas irgi taip žus. 
\par 6 Jiedu turi valdžią užrakinti dangų, kad jų pranašavimo dienomis nelytų lietus, ir turi valdžią vandenis paversti krauju ir ištikti žemę bet kokia neganda, kada tik panorės. 
\par 7 Kai jie baigs liudyti, žvėris, išlindęs iš bedugnės, kovos su jais, nugalės ir nužudys juos. 
\par 8 Jų lavonai gulės gatvėje didžiojo miesto, kuris dvasine prasme vadinamas Sodoma ir Egiptu, kur ir mūsų Viešpats buvo nukryžiuotas. 
\par 9 Tada įvairių tautų, genčių, kalbų ir giminių žmonės matys jų lavonus pusketvirtos dienos ir neleis jų lavonų palaidoti kapuose. 
\par 10 Žemės gyventojai džiūgaus dėl jų ir linksminsis, ir siųs vieni kitiems dovanas, nes tiedu pranašai vargino žemės gyventojus. 
\par 11 O po pusketvirtos dienos gyvybės dvasia nuo Dievo įžengė į juodu. Jie pašoko ant kojų, ir didžiulė baimė pagavo tuos, kurie į juos žiūrėjo. 
\par 12 Ir jie išgirdo galingą balsą iš dangaus, kuris jiems šaukė: “Užženkite šen!” Ir juodu užžengė į dangų debesyje, o jų priešai žiūrėjo į juodu. 
\par 13 Tą pačią valandą kilo didelis žemės drebėjimas, ir dešimta dalis miesto sugriuvo. Nuo žemės drebėjimo žuvo septyni tūkstančiai žmonių, o likusieji, baimės apimti, atidavė šlovę dangaus Dievui. 
\par 14 Antroji neganda praėjo. Štai greit artinasi trečioji neganda. 
\par 15 Sutrimitavo septintasis angelas. Danguje pasigirdo galingi balsai, kurie skelbė: “Šio pasaulio karalystės tapo mūsų Viešpaties ir Jo Kristaus karalystėmis, ir Jis valdys per amžių amžius!” 
\par 16 O dvidešimt keturi vyresnieji, sėdintys savo sostuose Dievo akivaizdoje, parpuolė veidais žemėn ir garbino Dievą, 
\par 17 sakydami: “Dėkojame Tau, o Viešpatie, visagalis Dieve, kuris esi ir kuris buvai, ir kuris ateini, kad pasiėmei savo didžiąją galią ir pradėjai viešpatauti. 
\par 18 Tautos įširdo, ir atėjo Tavo rūstybė, metas teisti mirusius ir atlyginti Tavo tarnams, pranašams ir šventiesiems, ir visiems bijantiems Tavojo vardo, mažiems ir dideliems, ir metas sunaikinti tuos, kurie niokoja žemę”. 
\par 19 Ir atsidarė danguje Dievo šventykla, ir pasirodė Jo šventykloje Jo Sandoros skrynia. Ir radosi žaibai, balsai, griaustiniai, žemės drebėjimas ir didelė kruša.


\chapter{12}


\par 1 Ir pasirodė danguje didingas ženklas: moteris, apsisiautusi saule, po jos kojų mėnulis, o ant galvos dvylikos žvaigždžių vainikas. 
\par 2 Ji buvo nėščia ir šaukė, kentėdama sąrėmius bei gimdymo skausmus. 
\par 3 Pasirodė ir kitas ženklas danguje: štai milžiniškas ugniaspalvis slibinas su septyniomis galvomis, su dešimčia ragų ir su septyniomis diademomis ant galvų. 
\par 4 Jo uodega nušlavė trečdalį dangaus žvaigždžių ir nužėrė jas žemėn. Slibinas stojo priešais moterį, kad, jai pagimdžius, prarytų jos kūdikį. 
\par 5 Ji pagimdė Sūnų, berniuką, kuriam skirta ganyti visas tautas geležine lazda. Ir jos kūdikis buvo paimtas pas Dievą, prie Jo sosto. 
\par 6 O moteris pabėgo į dykumą, kur buvo jai Dievo paruošta vieta, kad tenai ji būtų maitinama tūkstantį du šimtus šešiasdešimt dienų. 
\par 7 Ir kilo danguje kova. Mykolas ir jo angelai kovojo su slibinu. Ir kovėsi slibinas ir jo angelai, 
\par 8 bet jie nenugalėjo, ir nebeliko daugiau jiems vietos danguje. 
\par 9 Taip buvo išmestas didysis slibinas, senoji gyvatė, vadinamas Velniu ir Šėtonu, kuris suvedžioja visą pasaulį. Jis buvo išmestas žemėn, ir kartu su juo buvo išmesti jo angelai. 
\par 10 Aš girdėjau danguje galingą balsą, sakantį: “Dabar atėjo mūsų Dievo išgelbėjimas, galybė, karalystė ir Jo Kristaus valdžia, nes išmestas mūsų brolių kaltintojas, skundęs juos mūsų Dievui dieną ir naktį. 
\par 11 Ir jie nugalėjo jį Avinėlio krauju ir savo liudijimo žodžiu. Jie nebrangino savo gyvybės, net iki mirties. 
\par 12 Todėl džiūgaukite, dangūs ir jų gyventojai! Bet vargas gyvenantiems žemėje ir jūrai, nes pas jus numestas velnias, kupinas didelio įniršio, žinodamas mažai beturįs laiko”. 
\par 13 Slibinas, pamatęs, kad yra nutrenktas žemėn, ėmė persekioti moterį, kuri buvo pagimdžiusi berniuką. 
\par 14 Bet moteriai buvo duoti du didžiojo erelio sparnai skristi nuo gyvatės į dykumą, į savo vietą, kur bus maitinama per laiką, laikus ir pusę laiko. 
\par 15 Gyvatė išliejo iš savo nasrų paskui moterį vandenį lyg upę, kad nuplukdytų ją bangomis. 
\par 16 Bet žemė pagelbėjo moteriai: žemė atvėrė savo žiotis ir sugėrė upę, kurią slibinas buvo paliejęs iš savo nasrų. 
\par 17 Ir slibinas įnirto prieš moterį, ir metėsi kautis su kitais jos palikuonimis, kurie laikosi Dievo įsakymų ir turi Jėzaus Kristaus liudijimą.


\chapter{13}


\par 1 Aš stovėjau ant jūros kranto. Ir išvydau iš jūros išnyrant žvėrį, turintį septynias galvas ir dešimt ragų, o ant jo ragų buvo dešimt diademų ir ant jo galvų piktžodžiavimo vardas. 
\par 2 Žvėris, kurį mačiau, buvo panašus į leopardą; jo kojos tarytum lokio kojos, o jo snukis­lyg liūto snukis. Slibinas davė jam savo jėgą, savo sostą ir didelę valdžią. 
\par 3 Ir aš mačiau vieną iš jo galvų mirtinai sužeistą, tačiau jos mirštamoji žaizda užgijo. Ir visa žemė stebėdamasi nusekė paskui žvėrį. 
\par 4 Žmonės garbino slibiną, kuris atidavė valdžią žvėriui. Jie garbino žvėrį, sakydami: “Kas galėtų lygintis su žvėrimi, ir kas galėtų kovoti su juo!” 
\par 5 Jam buvo duotas snukis kalbėti išdidžiai ir piktžodžiauti, ir duota valdžia taip daryti per keturiasdešimt du mėnesius. 
\par 6 Ir jis atvėrė nasrus piktžodžiauti Dievui, piktžodžiauti Jo vardui, Jo buveinei ir dangaus gyventojams. 
\par 7 Jam buvo duota kovoti su šventaisiais ir juos nugalėti. Jam buvo suteikta valdžia visoms gentims, kalboms ir tautoms. 
\par 8 Ir jį garbins visi tie žemės gyventojai, kurių vardai neįrašyti nuo pasaulio sutvėrimo nužudytojo Avinėlio gyvenimo knygoje. 
\par 9 Kas turi ausis, teklauso. 
\par 10 Jei kas veda į nelaisvę, pats į nelaisvę eis. O kas žudo kalaviju, turės nuo kalavijo žūti. Čia šventųjų ištvermė ir tikėjimas! 
\par 11 Ir aš pamačiau kitą žvėrį, ateinantį iš žemės. Jis turėjo du ragus, panašius į Avinėlio, ir kalbėjo kaip slibinas. 
\par 12 Jis naudojasi visa pirmojo žvėries valdžia jo akivaizdoje ir verčia žemę bei jos gyventojus garbinti pirmąjį žvėrį, kurio mirštama žaizda užgijo. 
\par 13 Jis daro didžius ženklus, netgi, žmonėms matant, nuleidžia ugnį iš dangaus į žemę. 
\par 14 Jis suvedžioja žemės gyventojus tais ženklais, kuriuos jam buvo duota daryti žvėries akyse, sakydamas žemės gyventojams padaryti žvėries atvaizdą, kuris gavo žaizdą nuo kalavijo ir išgijo. 
\par 15 Jam buvo duota suteikti žvėries atvaizdui dvasią, kad žvėries atvaizdas imtų kalbėti, ir padaryti taip, kad visi, kurie atsisakys garbinti žvėries atvaizdą, būtų nužudyti. 
\par 16 Jis vertė visus, mažus ir didelius, turtuolius ir vargšus, laisvuosius ir vergus, pasidaryti ženklą ant dešinės rankos arba ant kaktos, 
\par 17 kad nė vienas negalėtų nei pirkti, nei parduoti, jei neturės to ženklo ar žvėries vardo, ar jo vardo skaičiaus. 
\par 18 Čia slypi išmintis! Kas turi išmanymą, teapskaičiuoja žvėries skaičių, nes tai žmogaus skaičius; jo skaičius yra šeši šimtai šešiasdešimt šeši.


\chapter{14}


\par 1 Ir aš išvydau: štai Avinėlis, bestovįs ant Siono kalno, o su Juo šimtas keturiasdešimt keturi tūkstančiai, turintys Jo Tėvo vardą, įrašytą savo kaktose. 
\par 2 Aš išgirdau iš dangaus garsą, tarsi daugybės vandenų šniokštimą ir tarsi galingo griaustinio dundėjimą. Garsas, kurį girdėjau, buvo tarytum arfininkų, skambinančių savo arfomis. 
\par 3 Jie giedojo naują giesmę priešais sostą, keturias būtybes ir vyresniuosius, ir niekas negalėjo išmokti tos giesmės, išskyrus tuos šimtą keturiasdešimt keturis tūkstančius, atpirktus iš žemės. 
\par 4 Tai tie, kurie nesusitepė su moterimis, nes jie mergelės. Tai tie, kurie lydi Avinėlį, kur tik Jis eina. Jie yra atpirkti iš žmonių, pirmieji vaisiai Dievui ir Avinėliui. 
\par 5 Jų lūpose nerasta apgaulės; jie be dėmės prieš Dievo sostą. 
\par 6 Ir aš pamačiau kitą angelą, lekiantį dangaus viduriu, turintį amžinąją Evangeliją, kad ją paskelbtų žemės gyventojams, kiekvienai giminei, genčiai, kalbai ir tautai. 
\par 7 Jis šaukė galingu balsu: “Bijokite Dievo ir atiduokite Jam šlovę, nes atėjo Jo teismo valanda; šlovinkite Tą, kuris sutvėrė dangų ir žemę, jūrą ir vandens šaltinius!” 
\par 8 Paskui jį skrido antras angelas, kuris šaukė: “Krito, krito Babelė, didis miestas, kuris savo paleistuvystės įniršio vynu nugirdė visas tautas!” 
\par 9 Ir trečias angelas lydėjo juos, šaukdamas skardžiu balsu: “Kas garbina žvėrį ir jo atvaizdą bei priima ant savo kaktos ar rankos ženklą, 
\par 10 tas gers Dievo įniršio vyno, įpilto ir neatmiešto Jo rūstybės taurėje, ir bus kankinamas ugnimi ir siera šventųjų angelų ir Avinėlio akivaizdoje. 
\par 11 Jų kankinimo dūmai kils per amžių amžius, ir jie neturės atilsio nei dieną, nei naktį­tie, kurie garbina žvėrį bei jo atvaizdą ir ima jo vardo ženklą”. 
\par 12 Čia pasirodo šventųjų ištvermė, kurie laikosi Dievo įsakymų ir Jėzaus tikėjimo. 
\par 13 Ir aš išgirdau iš dangaus balsą, kuris man sakė: “Rašyk: ‘Nuo šiol palaiminti mirusieji, kurie miršta Viešpatyje. Taip,­sako Dvasia,­ kad atilsėtų nuo savo vargų; ir jų darbai seka juos’ ”. 
\par 14 Ir aš regėjau: štai baltas debesis, o ant debesies sėdėjo panašus į Žmogaus Sūnų. Ant galvos Jis turėjo aukso vainiką, o rankoje­aštrų pjautuvą. 
\par 15 Iš šventyklos išėjo dar vienas angelas, kuris šaukė galingu balsu sėdinčiajam ant debesies: “Paleisk darban savo pjautuvą ir pjauk; atėjo Tau valanda pjauti, nes žemės derlius prinoko”. 
\par 16 Tuomet sėdintysis ant debesies nusviedė savo pjautuvą žemėn, ir žemės derlius buvo nupjautas. 
\par 17 Dar kitas angelas išėjo iš dangaus šventyklos, taip pat turintis aštrų pjautuvą. 
\par 18 Ir dar vienas angelas išėjo nuo aukuro, turintis valdžią ugniai. Jis stipriu balsu sušuko turinčiajam aštrų pjautuvą: “Paleisk darban savo aštrųjį pjautuvą ir nurink žemės vynmedžio kekes, nes uogos jau prinoko”. 
\par 19 Tada angelas numetė savo pjautuvą žemėn, nuskynė žemės vynmedį ir supylė vynuoges į didįjį Dievo rūstybės spaustuvą. 
\par 20 Spaustuvas buvo minamas už miesto, ir išsiveržė iš spaustuvo kraujas, pakildamas arkliams iki žąslų tūkstančio šešių šimtų stadijų atstumu.


\chapter{15}


\par 1 Ir aš pamačiau danguje dar vieną didį ir įspūdingą ženklą: septynis angelus, turinčius septynias paskutines negandas, nes jomis išsibaigia Dievo rūstybė. 
\par 2 Aš išvydau tarsi stiklo jūrą, sumaišytą su ugnimi, ir nugalėjusiuosius žvėrį, jo atvaizdą, jo ženklą ir jo vardo skaičių, stovinčius su Dievo arfomis ant stiklo jūros. 
\par 3 Jie giedojo Dievo tarno Mozės giesmę ir Avinėlio giesmę: “Didingi ir nuostabūs Tavo darbai, Viešpatie, visagali Dieve! Teisingi ir tikri Tavo keliai, šventųjų Karaliau! 
\par 4 Kas gi nesibijotų Tavęs, Viešpatie, ir nešlovintų Tavojo vardo?! Juk Tu vienas šventas! Visos tautos ateis ir pagarbins Tave, nes apreikšti Tavo teisūs sprendimai”. 
\par 5 Paskui aš regėjau: štai atsidarė Liudijimo palapinės šventykla danguje, 
\par 6 ir išėjo iš šventyklos septyni angelai, turintys septynias negandas. Jie buvo apsivilkę tyra, spindinčia drobe ir persijuosę per krūtines aukso juostomis. 
\par 7 Viena iš keturių būtybių padavė septyniems angelams septynis aukso dubenis, pilnus Gyvenančiojo per amžių amžius Dievo rūstybės. 
\par 8 Šventykla prisipildė dūmų nuo Dievo šlovės ir galybės, ir niekas negalėjo įeiti šventyklon, kol nesibaigs septynių angelų septynios negandos.


\chapter{16}


\par 1 Ir išgirdau iš šventyklos galingą balsą, sakantį septyniems angelams: “Eikite ir išpilkite septynis Dievo rūstybės dubenis žemėn!” 
\par 2 Ir nuėjo pirmasis, ir išliejo savo dubenį žemėn. Ir apniko piktos ir skaudžios votys žmones, turinčius žvėries ženklą ir garbinančius jo atvaizdą. 
\par 3 Antrasis angelas išpylė savo dubenį jūron; ji tapo lyg numirėlio kraujas, ir visi gyviai jūroje išgaišo. 
\par 4 Trečiasis angelas išliejo savo dubenį į upes ir vandens šaltinius, ir jie pavirto krauju. 
\par 5 Ir išgirdau vandenų angelą sakant: “Teisus Tu, o Viešpatie, kuris esi ir kuris buvai, šventas, kad taip teisi. 
\par 6 Nes jie praliejo šventųjų ir pranašų kraują, todėl duodi jiems gerti kraują. Taip! Jie to verti!” 
\par 7 Aš dar išgirdau kitą, nuo aukuro sakant: “Taip, visagali Viešpatie Dieve, tavo nuosprendžiai tikri ir teisingi!” 
\par 8 Ketvirtasis angelas išpylė savo dubenį saulėn, ir jai buvo duota svilinti žmones ugnimi. 
\par 9 Žmones degino baisi kaitra, o jie keikė vardą Dievo, kuris turi valdžią šitoms negandoms. Ir jie neatgailavo, kad atiduotų Jam šlovę. 
\par 10 Penktasis angelas išpylė savo dubenį ant žvėries sosto, ir jo karalystė paskendo tamsoje, o žmonės krimto savo liežuvius iš skausmo. 
\par 11 Jie piktžodžiavo dangaus Dievui dėl savo skausmų ir vočių, bet neatgailavo dėl savo darbų. 
\par 12 Šeštasis angelas išliejo savo dubenį į didžiąją Eufrato upę, ir jos vanduo išdžiūvo, kad pasidarytų kelias karaliams iš rytų. 
\par 13 Tada pamačiau iš slibino nasrų, iš žvėries snukio ir iš netikrojo pranašo burnos išeinant tris netyrąsias dvasias, tartum varles. 
\par 14 O tai yra demonų dvasios, darančios ženklus; jos išeina pas žemės ir viso pasaulio karalius, kad juos suburtų didžiosios visagalio Dievo dienos kovai. 
\par 15 “Štai Aš ateinu kaip vagis. Palaimintas, kas budi ir saugo savo drabužius, kad netektų vaikščioti nuogam ir jie nematytų jo gėdos!” 
\par 16 Ir jis subūrė juos į vietovę, kuri hebrajiškai vadinasi Harmagedonas. 
\par 17 Septintasis angelas išpylė savo dubenį į orą, ir nuskambėjo iš šventyklos, nuo sosto, galingas balsas: “Įvyko!” 
\par 18 Ir radosi žaibai, griaustiniai, garsai, ir kilo didžiulis žemės drebėjimas, kokio nebuvo, kiek žmogus gyvena žemėje,­toks smarkus, toks baisus žemės drebėjimas! 
\par 19 Didysis miestas suskilo į tris dalis, ir tautų miestai sugriuvo. Ir Dievas atsiminė didžiąją Babelę, kad jai duotų savo rūstybės ir įniršio vyno taurę. 
\par 20 Ir pabėgo visos salos, ir nebeliko kalnų. 
\par 21 Ledo gabalai, talento svorio, krito iš dangaus ant žmonių. Žmonės keikė Dievą dėl ledų negandos, nes siaubinga buvo ši neganda.


\chapter{17}


\par 1 Tuomet atėjo vienas iš septynių angelų, turėjusių septynis dubenis, ir prakalbino mane, sakydamas: “Eikš, aš parodysiu tau teismą didžiosios paleistuvės, sėdinčios ant daugybės vandenų. 
\par 2 Su ja ištvirkavo žemės karaliai, ir jos ištvirkavimo vynu pasigėrė žemės gyventojai”. 
\par 3 Ir jis Dvasia nunešė mane į dykumą. Ir aš išvydau moterį, sėdinčią ant skaisčiai raudono žvėries, pilno piktžodžiavimo vardų, turinčio septynias galvas ir dešimt ragų. 
\par 4 Moteris buvo apsivilkusi purpuru ir škarlatu, išsipuošusi auksu, brangakmeniais ir perlais. Ji laikė rankoje aukso taurę, pilną savo ištvirkavimo šlykštybių ir nešvarumų. 
\par 5 Ant jos kaktos buvo užrašytas vardas: “Paslaptis, didžioji Babelė, ištvirkėlių ir žemės šlykštybių motina”. 
\par 6 Mačiau tą moterį, girtą nuo šventųjų kraujo ir nuo Jėzaus liudytojų kraujo. Ją matydamas, aš stebėte stebėjausi. 
\par 7 O angelas man tarė: “Ko stebiesi? Aš tau pasakysiu paslaptį moters ir ją nešančio žvėries, kuris turi septynias galvas ir dešimt ragų”. 
\par 8 “Žvėris, kurį regėjai, buvo, bet jo nebėra; jis ruošiasi išlipti iš bedugnės, tačiau eina į pražūtį. Žemės gyventojai, kurių vardai nėra įrašyti gyvenimo knygoje nuo pasaulio sutvėrimo, stebėsis, žiūrėdami į žvėrį, kad jis buvo ir jo nebėra, ir jis vėl pasirodys. 
\par 9 Čia reikia proto, turinčio išmintį! Septynios galvos reiškia septynis kalnus, ant kurių sėdi moteris. 
\par 10 Taip pat yra septyni karaliai; penki žlugo, vienas tebėra, o vienas dar neatėjo; kai jis ateis, turės trumpam pasilikti. 
\par 11 O žvėris, kuris buvo ir kurio nebėra,­tai aštuntasis, bet vienas iš septynių, ir jis eina į pražūtį. 
\par 12 Tie dešimt ragų, kuriuos matei, yra dešimt karalių, kurie dar negavo karalystės, bet jie gaus valdžią kaip karaliai vienai valandai kartu su žvėrimi. 
\par 13 Jie turi vieną tikslą ir savo jėgą bei valdžią atiduos žvėriui. 
\par 14 Jie kovos su Avinėliu, bet Avinėlis juos nugalės, nes Jis yra viešpačių Viešpats ir karalių Karalius, ir su juo visi pašauktieji, išrinktieji ir ištikimieji”. 
\par 15 Angelas toliau man sako: “Vandenys, kuriuos regėjai, kur sėdi paleistuvė, yra žmonės, minios, tautos ir kalbos. 
\par 16 Tie dešimt ragų, kuriuos matei ant žvėries­jie ims nekęsti paleistuvės, apiplėš ją ir paliks ją nuogą, ės jos kūną ir ją pačią sudegins ugnyje. 
\par 17 Nes Dievas įkvėpė jų širdis vykdyti Jo tikslą, vykdyti vieną tikslą,­kad jie atiduotų savo karalystę žvėriui, kol išsipildys Dievo žodžiai. 
\par 18 Ta moteris, kurią regėjai, yra didysis miestas, valdantis žemės karalius”.


\chapter{18}


\par 1 Paskui aš išvydau kitą angelą, nužengiantį iš dangaus ir turintį didžią valdžią. Žemė nušvito nuo jo šlovės. 
\par 2 O jis šaukė galingai, stipriu balsu: “Krito, krito didžioji Babelė! Ji pavirto demonų buveine, visų netyrųjų dvasių pastoge, visų nešvarių ir nekenčiamų paukščių narvu. 
\par 3 Nuo jos ištvirkimo įniršio vyno buvo girtos visos tautos; žemės karaliai su ja ištvirkavo, o žemės pirkliai pralobo iš jos nežabotos prabangos”. 
\par 4 Ir aš išgirdau iš dangaus kitą balsą, skelbiantį: “Išeikite iš jos, mano žmonės, kad nedalyvautumėte jos nuodėmėse ir nepatirtumėte jos negandų! 
\par 5 Nes jos nuodėmės pasiekė dangų, ir Dievas prisiminė jos piktadarystes. 
\par 6 Atmokėkite jai, kaip ji jums atlygino, ir atiduokite dvigubai pagal jos darbus. Dvigubai sumaišykite taurę, kurią ji jums maišė. 
\par 7 Kiek ji puikavo ir lėbavo, tiek jai paruoškite kentėjimų ir nuliūdimo, nes ji savo širdyje kalba: ‘Aš sėdžiu kaip karalienė, nesu našlė ir liūdesio nematysiu’. 
\par 8 Todėl vieną dieną ją apniks negandos: mirtis, gedulas, badas, ir ji bus sudeginta ugnyje, nes galingas yra Viešpats Dievas, kuris nuteisė ją. 
\par 9 Jos verks ir raudos žemės karaliai, kurie su ja ištvirkavo bei lėbavo, kai pamatys jos gaisro dūmus. 
\par 10 Jie stovės iš tolo ir, jos kankinimų išgąsdinti, sakys: ‘Vargas, vargas, didysis mieste! Babele, galingasis mieste, per vieną valandą įvyko tavo teismas!’ 
\par 11 Žemės pirkliai verks ir gedės jos, nes niekas jau nebepirks jų atplukdytų prekių: 
\par 12 aukso, sidabro, brangakmenių, perlų, švelnios drobės, purpuro, šilko, škarlato, jokios kvapios medienos, jokių dramblio kaulo dirbinių, jokių brangmedžio rakandų, nei vario, geležies, marmuro, 
\par 13 cinamono, kvapių augalų, miros, smilkalų, vyno, aliejaus, smulkių miltų, kviečių, galvijų, avių, arklių, vežimų, žmonių kūnų ir sielų. 
\par 14 Vaisiai, kurių taip geidė tavo siela, nutolo nuo tavęs; visas puošnumas ir spindesys tau pražuvo, ir niekad jų neberasi. 
\par 15 Tų daiktų pirkliai, iš jos pralobę, stovės iš tolo, išsigandę jos kankinimų, verks, gedės 
\par 16 ir sakys: ‘Vargas, vargas didžiajam miestui, vilkėjusiam ploniausia drobe, purpuru, škarlatu, išsipuošusiam auksu, brangakmeniais ir perlais. 
\par 17 Per vieną valandą niekais virto šitokie turtai!’ Visi laivų vairininkai, visi pakrančių laivininkai, visi jūreiviai ir visi, kurie jūroje darbuojasi, iš tolo sustoję 
\par 18 ir, stebėdami jos gaisro dūmus, šaukė: ‘Kas galėtų lygintis su didžiuoju miestu?!’ 
\par 19 Jie bėrė žemes ant galvų, šaukė, verkė ir aimanavo: ‘Vargas, vargas didžiajam miestui, iš kurio prabangos pralobo, kurie jūroje turėjo laivų. Per vieną valandą jis virto dykyne!’ 
\par 20 Džiūgauk dėl jo, dangau, ir jūs, šventieji, ir apaštalai, ir pranašai, nes dėl jūsų Dievas nubaudė jį teismu!” 
\par 21 Tuomet vienas galingas angelas iškėlė akmenį, tarsi didelį girnakmenį, ir sviedė jį į jūrą, sakydamas: “Tokiu smarkumu bus nublokštas didysis Babelės miestas, ir jo nebebus galima rasti. 
\par 22 Nebesigirdės daugiau tavyje arfininkų, giesmininkų, vamzdininkų, trimitininkų balsų. Niekas neberas tavyje nė vieno jokio meno kūrėjo, ir nebesigirdės tavyje malūno dūzgimo. 
\par 23 Tavyje nebešvies žiburio spindulys, niekas nebegirdės jaunikio ir nuotakos balso. Nes tavo pirkliai buvo tapę žemės didžiūnais, nes tavo burtais buvo suvedžiotos visos tautos 
\par 24 ir tavyje buvo rastas pranašų ir šventųjų ir visų žemėje nužudytųjų kraujas”.


\chapter{19}


\par 1 Paskui aš girdėjau danguje galingą didžiulės minios balsą, skelbiantį: “Aleliuja! Išgelbėjimas, galybė, šlovė ir garbė priklauso Viešpačiui, mūsų Dievui, 
\par 2 nes tikri ir teisingi Jo teismai! Jis nuteisė didžiąją paleistuvę, kuri suteršė žemę savo ištvirkavimu; Jis atkeršijo už savo tarnų kraują, pralietą jos rankomis”. 
\par 3 Ir dar kartą jie skelbė: “Aleliuja! Jos dūmai rūks per amžių amžius!” 
\par 4 Dvidešimt keturi vyresnieji ir keturios būtybės parpuolė ir pagarbino Dievą, sėdintį soste, sakydami: “Amen! Aleliuja!” 
\par 5 O nuo sosto nuskambėjo balsas: “Šlovinkite mūsų Dievą visi Jo tarnai ir tie, kurie bijote Jo: maži ir dideli!” 
\par 6 Ir išgirdau gausios minios balsą, lyg daugybės vandenų šniokštimą ar galingų griaustinių dundėjimą, sakantį: “Aleliuja! Užviešpatavo Viešpats Dievas, Visagalis. 
\par 7 Džiūgaukime ir linksminkimės, ir duokime Jam šlovę! Nes atėjo Avinėlio vestuvės ir Jo nuotaka pasiruošė”. 
\par 8 Jai buvo duota apsirengti spindinčia, tyra drobe, o ta drobė­tai šventųjų teisūs darbai. 
\par 9 Ir angelas sako man: “Rašyk: ‘Palaiminti, kurie pakviesti į Avinėlio vestuvių pokylį’ ”. Jis pridūrė: “Šie žodžiai yra tikri Dievo žodžiai!” 
\par 10 Aš puoliau jam po kojų, norėdamas jį pagarbinti, bet jis pasakė: “Žiūrėk, kad to nedarytum! Aš esu tarnas, kaip tu ir broliai, kurie turi Jėzaus liudijimą. Dievą garbink! Nes Jėzaus liudijimas yra pranašystės dvasia”. 
\par 11 Ir aš išvydau atvirą dangų, ir štai pasirodė baltas žirgas. Ant jo sėdėjo raitelis, vardu Ištikimasis ir Teisusis; Jis teisingai teisia ir kovoja. 
\par 12 Jo akys švietė kaip ugnies liepsna, o ant Jo galvos daug diademų ir įrašytas vardas, kurio niekas nežino, tik Jis pats. 
\par 13 Jis apsirengęs kraujyje pamirkytu drabužiu ir Jo vardas­Dievo žodis. 
\par 14 Paskui Jį sekė dangaus kariaunos pulkai ant baltų žirgų, apsivilkę tyros baltos drobės drabužiais. 
\par 15 Iš Jo burnos ėjo aštrus dviašmenis kalavijas, kuriuo Jis ištiks tautas. Jis valdys jas geležine lazda. Jis mina visagalio Dievo įniršio ir rūstybės vyno spaustuvą. 
\par 16 Ant Jo drabužio ir ant strėnų užrašytas vardas: “Karalių Karalius ir viešpačių Viešpats”. 
\par 17 Aš regėjau angelą, stovintį saulėje. Jis garsiai šaukė, kviesdamas visus paukščius, skrendančius dangaus viduriu: “Skriskite šen, į didžiojo Dievo pokylį, 
\par 18 ir leskite kūnus karalių, karo vadų, galiūnų, žirgų, raitelių, visų laisvųjų ir vergų, mažų ir didelių!” 
\par 19 Ir išvydau žvėrį ir žemės karalius bei jų kariuomenes, susirinkusias kovoti su sėdinčiuoju ant žirgo ir Jo kariuomene. 
\par 20 Žvėris buvo sugautas, o kartu su juo netikrasis pranašas, jo akyse daręs ženklus ir jais klaidinęs žmones, kurie buvo priėmę žvėries ženklą ir garbino jo atvaizdą. Jiedu gyvi buvo įmesti į ugnies ežerą, degantį siera. 
\par 21 O visi kiti buvo užmušti kalaviju, einančiu iš raitelio burnos. Ir visi paukščiai prisilesė jų lavonų.


\chapter{20}


\par 1 Ir išvydau angelą, nužengiantį iš dangaus, laikantį rankoje bedugnės raktą ir didžiulę grandinę. 
\par 2 Jis nutvėrė slibiną­senąją gyvatę, kuri yra Velnias ir Šėtonas,­ surišo jį tūkstančiui metų 
\par 3 ir įmetė į bedungę, užrakino ją ir iš viršaus užantspaudavo, kad nebegalėtų daugiau suvedžioti tautų, kol pasibaigs tūkstantis metų. Po to jis turės būti atrištas trumpam laikui. 
\par 4 Aš pamačiau sostus ir juose sėdinčiuosius, kuriems buvo pavesta teisti. Taip pat regėjau sielas tų, kuriems buvo nukirstos galvos dėl Jėzaus liudijimo ir dėl Dievo žodžio, kurie negarbino žvėries, nei jo atvaizdo, ir neėmė ženklo sau ant kaktos ar rankos. Jie atgijo ir viešpatavo su Kristumi tūkstantį metų. 
\par 5 O visi kiti mirusieji neatgijo, iki pasibaigiant tūkstančiui metų. Šis yra pirmasis prisikėlimas. 
\par 6 Palaimintas ir šventas, kas turi dalį pirmajame prisikėlime. Šitiems antroji mirtis neturi galios; jie bus Dievo ir Kristaus kunigai ir valdys su Juo tūkstantį metų. 
\par 7 Kai pasibaigs tūkstantis metų, šėtonas bus išleistas iš savo kalėjimo 
\par 8 ir išeis suvedžioti tautų, gyvenančių keturiuose žemės kampuose, Gogo ir Magogo, ir surinkti jų kovai. Jų skaičius kaip pajūrio smiltys. 
\par 9 Jie išėjo ant žemės platumos ir apsupo šventųjų stovyklą ir mylimąjį miestą. Ir nužengė ugnis iš dangaus nuo Dievo ir juos prarijo, 
\par 10 o jų suvedžiotojas velnias buvo įmestas į ugnies ir sieros ežerą, kur jau yra žvėris ir netikrasis pranašas. Jie bus kankinami dieną ir naktį per amžių amžius. 
\par 11 Paskui mačiau didelį baltą sostą ir jame Sėdintįjį, nuo kurio veido pabėgo žemė ir dangus, ir nebeliko jiems vietos. 
\par 12 Ir mačiau mirusius, didelius ir mažus, stovinčius priešais Dievą. Buvo atskleistos knygos. Ir buvo atversta dar viena, būtent gyvenimo knyga. Mirusieji buvo teisiami iš užrašų knygose pagal jų darbus. 
\par 13 Jūra atidavė savo mirusiuosius, o mirtis ir pragaras atidavė savuosius. Ir kiekvienas buvo teisiamas pagal savo darbus. 
\par 14 Mirtis ir pragaras buvo įmesti į ugnies ežerą. Tai yra antroji mirtis. 
\par 15 Kas tik nebuvo rastas įrašytas gyvenimo knygoje, buvo įmestas į ugnies ežerą.


\chapter{21}


\par 1 Ir aš pamačiau naują dangų ir naują žemę, nes pirmasis dangus ir pirmoji žemė praėjo ir jūros daugiau nebebuvo. 
\par 2 Ir aš, Jonas, išvydau šventąjį miestą­naująją Jeruzalę, nužengiančią iš dangaus nuo Dievo; ji buvo pasiruošusi kaip nuotaka, pasipuošusi savo sužadėtiniui. 
\par 3 Ir išgirdau galingą balsą, skambantį iš dangaus: “Štai Dievo buveinė tarp žmonių. Jis apsigyvens pas juos, ir jie bus Jo tauta, ir pats Dievas, jų Dievas, bus su jais. 
\par 4 Jis nušluostys kiekvieną ašarą nuo jų akių; nebebus daugiau mirties, nei liūdesio, nei dejonės, nei skausmo daugiau nebebus, nes kas buvo pirmiau­praėjo”. 
\par 5 Ir Sėdintysis soste tarė: “Štai Aš visa darau nauja!” Jis pasakė man: “Rašyk, nes šitie žodžiai patikimi ir tikri”. 
\par 6 Ir Jis man pasakė: “Įvyko! Aš esu Alfa ir Omega, Pradžia ir Pabaiga. Trokštančiam Aš duosiu dovanai gerti iš gyvenimo vandens šaltinio. 
\par 7 Nugalėtojas paveldės viską, ir Aš būsiu jo Dievas, o jis bus mano sūnus. 
\par 8 Bet bailiams, netikintiems, nešvankėliams, žudikams, ištvirkėliams, burtininkams, stabmeldžiams ir visiems melagiams skirta dalis ežere, kuris dega ugnimi ir siera; tai yra antroji mirtis”. 
\par 9 Tada prie manęs priėjo vienas iš septynių angelų, turėjusių septynis dubenis, pilnus septynių paskutinių negandų, ir pasakė man: “Eikš, aš tau parodysiu nuotaką, Avinėlio sužadėtinę”. 
\par 10 Ir jis nunešė mane dvasioje ant didelio ir aukšto kalno, ir parodė man miestą, šventąją Jeruzalę, nusileidžiančią iš dangaus nuo Dievo, 
\par 11 žėrinčią Dievo šlove. Jos švytėjimas tarsi brangakmenio, tarsi jaspio akmens, tviskančio kaip krištolas. 
\par 12 Ji apjuosta didele ir aukšta siena su dvylika vartų, o ant vartų dvylika angelų ir užrašyti dvylikos Izraelio giminių vardai. 
\par 13 Nuo rytų pusės treji vartai, nuo šiaurės treji vartai, nuo pietų treji vartai ir nuo vakarų treji vartai. 
\par 14 Miesto sienos turi dvylika pamatų, ant kurių dvylikos Avinėlio apaštalų vardai. 
\par 15 Kalbantysis su manimi turėjo matą­auksinę nendrę išmatuoti miestui, jo vartams ir sienoms. 
\par 16 Miestas išdėstytas keturkampiu: jo ilgis ir plotis lygūs. Jis išmatavo miestą nendre ir rado dvylika tūkstančių stadijų. Jo ilgis, plotis ir aukštis lygūs. 
\par 17 Jis išmatavo jo sienas ir rado šimtą keturiasdešimt keturias uolektis žmonių mastu, tiek pat ir angelo mastu. 
\par 18 Jo sienos sukrautos iš jaspio, o pats miestas iš gryno aukso, panašaus į vaiskų stiklą. 
\par 19 Miesto sienų pamatai papuošti visokiais brangakmeniais. Pirmas pamatas yra jaspio, antras safyro, trečias chalcedono, ketvirtas smaragdo, 
\par 20 penktas sardonikso, šeštas sardžio, septintas chrizolito, aštuntas berilio, devintas topazo, dešimtas chrizoprazo, vienuoliktas hiacinto, dvyliktas ametisto. 
\par 21 Dvylika vartų­dvylika perlų, kiekvieni vartai iš vieno perlo. Ir miesto gatvės­grynas auksas, tarsi vaiskus stiklas. 
\par 22 Bet jame nemačiau šventyklos, nes Viešpats, visagalis Dievas, ir Avinėlis yra jo šventykla. 
\par 23 Miestui apšviesti nereikia nei saulės, nei mėnulio, nes jame šviečia Dievo šlovė ir jo žiburys yra Avinėlis. 
\par 24 Ir išgelbėtos tautos vaikščios jo šviesoje, ir žemės karaliai atsineš į jį savo šlovę ir garbę. 
\par 25 Jo vartai nebus uždaromi dieną,­nes tenai nebus nakties,­ 
\par 26 ir į jį bus atnešta tautų šlovė ir garbė. 
\par 27 Ir ten niekados nepateks, kas netyra, joks nešvankėlis ar melagis, o tiktai tie, kurie įrašyti Avinėlio gyvenimo knygoje.


\chapter{22}


\par 1 Jis parodė man tyrą gyvenimo vandens upę, tvaskančią tarsi krištolas, ištekančią nuo Dievo ir Avinėlio sosto. 
\par 2 Viduryje miesto gatvės, abejose upės pusėse, augo gyvenimo medis, duodantis dvylika derlių, kiekvieną mėnesį vedantis vaisių, o to medžio lapai­tautoms gydyti. 
\par 3 Ir nebus daugiau jokio prakeikimo. Mieste stovės Dievo ir Avinėlio sostas, ir Jo tarnai tarnaus Jam. 
\par 4 Jie regės Jo veidą, ir jų kaktose bus Jo vardas. 
\par 5 Ten nebebus nakties, jiems nereikės nei žiburio, nei saulės šviesos, nes Viešpats Dievas jiems švies, ir jie viešpataus per amžių amžius. 
\par 6 Tuomet jis man pasakė: “Šie žodžiai patikimi ir tikri. Viešpats, šventų pranašų Dievas, atsiuntė savo angelą parodyti savo tarnams, kas turi įvykti netrukus”. 
\par 7 “Štai Aš veikiai ateinu! Palaimintas, kas laikosi šios knygos pranašystės žodžių!” 
\par 8 Aš, Jonas, visa tai mačiau ir girdėjau. Išgirdęs ir pamatęs, puoliau po kojų angelui, kuris man visa tai parodė, norėdamas jį pagarbinti. 
\par 9 Bet jis man pasakė: “Žiūrėk, kad to nedarytum! Juk ir aš esu tarnas, kaip tu ir tavo broliai pranašai, ir visi, kurie laikosi šios knygos žodžių. Dievą garbink!” 
\par 10 Jis sako man: “Neužantspauduok pranašiškų šios knygos žodžių, nes laikas trumpas. 
\par 11 Neteisusis toliau tesielgia neteisiai, kas susitepęs, ir toliau tebūna susitepęs, teisusis toliau tevykdo teisumą, ir šventasis dar tepašventėja”. 
\par 12 “Štai Aš veikiai ateinu, ir mano atlygis su manimi, kad kiekvienam atlyginčiau pagal jo darbus. 
\par 13 Aš esu Alfa ir Omega, Pradžia ir Pabaiga, Pirmasis ir Paskutinysis”. 
\par 14 “Palaiminti, kurie vykdo Jo įsakymus, kad įgytų teisę į gyvenimo medį ir galėtų įžengti pro vartus į miestą. 
\par 15 O lauke lieka šunys, burtininkai, ištvirkėliai, žudikai, stabmeldžiai ir visi, kurie mėgsta melą ir jį daro”. 
\par 16 “Aš, Jėzus, pasiunčiau savo angelą jums tai paliudyti apie bažnyčias. Aš esu Dovydo šaknis ir palikuonis, žėrinti aušrinė žvaigždė!” 
\par 17 Ir Dvasia, ir sužadėtinė kviečia: “Ateik!” Ir kas girdi, teatsiliepia: “Ateik!” Ir kas trokšta, teateina, ir kas nori, tesisemia dovanai gyvenimo vandens. 
\par 18 Aš sakau kiekvienam, kuris girdi šios knygos pranašystės žodžius: “Jeigu kas prie jų ką pridės­Dievas jam pridės aprašytų šioje knygoje negandų. 
\par 19 Ir jeigu kas atims ką nors nuo šios pranašystės knygos žodžių­ Dievas atims jo dalį iš gyvenimo knygos ir šventojo miesto, kurie aprašyti šitoje knygoje”. 
\par 20 Tas, kuris šitai liudija, sako: “Taip, Aš veikiai ateinu!” Amen. Taip, ateik, Viešpatie Jėzau! 
\par 21 Mūsų Viešpaties Jėzaus Kristaus malonė su jumis visais! Amen!



\end{document}