\begin{document}

\title{1 Samuel}



\chapter{1}

\par 1 Daar was een man van Ramathaim-zofim, van het gebergte van Efraim, wiens naam was Elkana, een zoon van Jerocham, den zoon van Elihu, den zoon van Tochu, den zoon van Zuf, een Efrathiet.
\par 2 En hij had twee vrouwen; de naam van de ene was Hanna, en de naam van de andere was Peninna. Peninna nu had kinderen, maar Hanna had geen kinderen.
\par 3 Deze man nu ging opwaarts uit zijn stad van jaar tot jaar om te aanbidden, en om te offeren den HEERE der heirscharen te Silo; en aldaar waren priesters des HEEREN, Hofni, en Pinehas, de twee zonen van Eli.
\par 4 En het geschiedde op dien dag, als Elkana offerde, zo gaf hij aan Peninna, zijn huisvrouw, en aan al haar zonen en haar dochteren, delen.
\par 5 Maar aan Hanna gaf hij een aanzienlijk deel, want hij had Hanna lief; doch de HEERE had haar baarmoeder toegesloten.
\par 6 En haar tegenpartijdige tergde haar ook met terging, om haar te vergrimmen, omdat de HEERE haar baarmoeder toegesloten had.
\par 7 En alzo deed hij jaar op jaar; van dat zij opging tot het huis des HEEREN, zo tergde zij haar alzo; daarom weende zij en at niet.
\par 8 Toen zeide Elkana, haar man: Hanna, waarom weent gij, en waarom eet gij niet, en waarom is uw hart kwalijk gesteld? Ben ik u niet beter dan tien zonen?
\par 9 Toen stond Hanna op, nadat hij gegeten, en nadat hij gedronken had te Silo. En Eli, de priester, zat op een stoel bij een post van den tempel des HEEREN.
\par 10 Zij dan viel bitterlijk bedroefd zijnde, zo bad zij tot den HEERE, en zij weende zeer.
\par 11 En zij beloofde een gelofte, en zeide: HEERE der heirscharen, zo Gij eenmaal de ellende Uwer dienstmaagd aanziet, en mijner gedenkt, en Uw dienstmaagd niet vergeet, maar geeft aan Uw dienstmaagd een mannelijk zaad, zo zal ik dat den HEERE geven al de dagen zijns levens, en er zal geen scheermes op zijn hoofd komen.
\par 12 Het geschiedde nu, als zij evenzeer bleef biddende voor het aangezicht des HEEREN, zo gaf Eli acht op haar mond.
\par 13 Want Hanna sprak in haar hart; alleenlijk roerden zich haar lippen, maar haar stem werd niet gehoord; daarom hield Eli haar voor dronken.
\par 14 En Eli zeide tot haar: Hoe lang zult gij u dronken aanstellen? Doe uw wijn van u.
\par 15 Doch Hanna antwoordde en zeide: Neen, mijn heer! ik ben een vrouw, bezwaard van geest; ik heb noch wijn, noch sterken drank gedronken; maar ik heb mijn ziel uitgegoten voor het aangezicht des HEEREN.
\par 16 Acht toch uw dienstmaagd niet voor een dochter Belials; want ik heb tot nu toe gesproken uit de veelheid van mijn gedachten en van mijn verdriet.
\par 17 Toen antwoordde Eli en zeide: Ga heen in vrede, en de God Israels zal uw bede geven, die gij van Hem gebeden hebt.
\par 18 En zij zeide: Laat uw dienstmaagd genade vinden in uw ogen! Alzo ging die vrouw haars weegs; en zij at, en haar aangezicht was haar zodanig niet meer.
\par 19 En zij stonden des morgens vroeg op, en zij aanbaden voor het aangezicht des HEEREN, en zij keerden weder, en kwamen tot hun huis te Rama. En Elkana bekende zijn huisvrouw Hanna, en de HEERE gedacht aan haar.
\par 20 En het geschiedde, na verloop van dagen, dat Hanna bevrucht werd, en baarde een zoon, en zij noemde zijn naam Samuel: Want, zeide zij, ik heb hem van den HEERE gebeden.
\par 21 En die man, Elkana toog op met zijn ganse huis, om den HEERE te offeren het jaarlijkse offer, en zijn gelofte.
\par 22 Doch Hanna toog niet op; maar zij zeide tot haar man: Als de jongen gespeend is, dan zal ik hem brengen, dat hij voor het aangezicht des HEEREN verschijne, en blijve daar tot in eeuwigheid.
\par 23 En Elkana, haar man, zeide tot haar: Doe, wat goed is in uw ogen; blijf, totdat gij hem zult gespeend hebben; de HEERE bevestige naar Zijn woord! Alzo bleef de vrouw, en zoogde haar zoon, totdat zij hem speende.
\par 24 Daarna, als zij hem gespeend had, bracht zij hem met zich opwaarts, met drie varren, en een efa meels, en een fles met wijn; en zij bracht hem in het huis des HEEREN te Silo; en het jongsken was zeer jong.
\par 25 En zij slachtten een var; alzo brachten zij het kind tot Eli.
\par 26 En zij zeide: Och, mijn heer! zo waarachtig als uw ziel leeft, mijn heer! Ik ben die vrouw, die hier bij u stond, om den HEERE te bidden.
\par 27 Ik bad om dit kind, en de HEERE heeft mij mijn bede gegeven, die ik van Hem gebeden heb.
\par 28 Daarom heb ik hem ook den HEERE overgegeven al de dagen, die hij wezen zal; hij is van den HEERE gebeden. En hij bad aldaar den HEERE aan.

\chapter{2}

\par 1 Toen bad Hanna en zeide: Mijn hart springt van vreugde op in den HEERE; mijn hoorn is verhoogd in den HEERE; mijn mond is wijd opengedaan over mijn vijanden; want ik verheug mij in Uw heil.
\par 2 Er is niemand heilig, gelijk de HEERE; want er is niemand dan Gij, en er is geen rotssteen, gelijk onze God!
\par 3 Maakt het niet te veel, dat gij hoog, hoog zoudt spreken, dat iets hards uit uw mond zou gaan; want de HEERE is een God der wetenschappen, en Zijn daden zijn recht gedaan.
\par 4 De boog der sterken is gebroken; en die struikelden, zijn met sterkte omgord.
\par 5 Die verzadigd waren, hebben zich verhuurd om brood, en die hongerig waren, zijn het niet meer; totdat de onvruchtbare zeven heeft gebaard, en die vele kinderen had, krachteloos is geworden.
\par 6 De HEERE doodt en maakt levend; Hij doet ter helle nederdalen, en Hij doet weder opkomen.
\par 7 De HEERE maakt arm en maakt rijk; Hij vernedert, ook verhoogt Hij.
\par 8 Hij verheft den geringe uit het stof, en den nooddruftige verhoogt Hij uit den drek, om te doen zitten bij de vorsten, dat Hij hen den stoel der ere doe beerven; want de grondvesten des aardrijks zijn des HEEREN, en Hij heeft de wereld daarop gezet.
\par 9 Hij zal de voeten Zijner gunstgenoten bewaren; maar de goddelozen zullen zwijgen in duisternis; want een man vermag niet door kracht.
\par 10 Die met den HEERE twisten, zullen verpletterd worden; Hij zal in den hemel over hen donderen; de HEERE zal de einden der aarde richten, en zal Zijn Koning sterkte geven, en den hoorn Zijns Gezalfden verhogen.
\par 11 Daarna ging Elkana naar Rama in zijn huis; maar de jongeling was den HEERE dienende voor het aangezicht van den priester Eli.
\par 12 Doch de zonen van Eli waren kinderen Belials; zij kenden den HEERE niet.
\par 13 Want de wijze dier priesters met het volk was, dat, wanneer iemand een offerande offerde, des priesters jongen kwam, terwijl het vlees kookte, met een drietandigen krauwel in zijn hand;
\par 14 En sloeg in de teile, of in den ketel, of in de pan, of in den pot; al wat de krauwel optrok, dat nam de priester voor zich. Alzo deden zij aan al de Israelieten, die te Silo kwamen.
\par 15 Ook eer zij het vet aanstaken, kwam des priesters jongen, en zeide tot den man, die offerde: Geef dat vlees om te braden voor den priester; want hij zal geen gekookt vlees van u nemen, maar rauw.
\par 16 Wanneer nu die man tot hem zeide: Zij zullen dat vet als heden ganselijk aansteken, zo neem dan voor u, gelijk als het uw ziel lusten zal; zo zeide hij tot hem: Nu zult gij het immers geven, en zo niet, ik zal het met geweld nemen.
\par 17 Alzo was de zonde dezer jongelingen zeer groot voor het aangezicht des HEEREN; want de lieden verachtten het spijsoffer des HEEREN.
\par 18 Doch Samuel diende voor het aangezicht des HEEREN, zijnde een jongeling, omgord met den linnen lijfrok.
\par 19 En zijn moeder maakte hem een kleinen rok, en bracht hem dien van jaar tot jaar, als zij opkwam met haar man, om het jaarlijkse offer te offeren.
\par 20 En Eli zegende Elkana, en zijn huisvrouw, en zeide: De HEERE geve u zaad uit deze vrouw voor de bede, die zij den HEERE afgebeden heeft. En zij gingen naar zijn plaats.
\par 21 Want de HEERE bezocht Hanna, en zij werd bevrucht, en baarde drie zonen en twee dochters; en de jongeling Samuel werd groot bij den HEERE.
\par 22 Doch Eli was zeer oud, en hoorde al, wat zijn zonen aan gans Israel deden, en dat zij sliepen bij de vrouwen, die met hopen samenkwamen aan de deur van de tent der samenkomst.
\par 23 En hij zeide tot hen: Waarom doet gij al zulke dingen, dat ik deze uw boze stukken hore van dit ganse volk?
\par 24 Niet, mijn zonen; want dit is geen goed gerucht, dat ik hoor; gij maakt, dat het volk des HEEREN overtreedt.
\par 25 Wanneer een mens tegen een mens zondigt, zo zullen de goden hem oordelen; maar wanneer een mens tegen den HEERE zondigt, wie zal voor hem bidden? Doch zij hoorden de stem huns vaders niet, want de HEERE wilde hen doden.
\par 26 En de jongeling Samuel nam toe, en werd groot en aangenaam beide bij den HEERE en ook bij de mensen.
\par 27 En er kwam een man Gods tot Eli, en zeide tot hem: Zo zegt de HEERE: Heb Ik Mij klaarlijk geopenbaard aan het huis uws vaders, toen zij in Egypte waren, in het huis van Farao?
\par 28 En Ik heb hem uit alle stammen van Israel Mij ten priester verkoren, om te offeren op Mijn altaar, om het reukwerk aan te steken, om den efod voor Mijn aangezicht te dragen; en heb aan het huis uws vaders gegeven al de vuurofferen van de kinderen Israels.
\par 29 Waarom slaat gijlieden achteruit tegen Mijn slachtoffer, en tegen Mijn spijsoffer, hetwelk Ik geboden heb in de woning; en eert uw zonen meer dan Mij, dat gijlieden u mest van het voornaamste van alle spijsoffers van Mijn volk Israel?
\par 30 Daarom spreekt de HEERE, de God Israels: Ik had wel klaarlijk gezegd: Uw huis en uws vaders huis zouden voor Mijn aangezicht wandelen tot in eeuwigheid; maar nu spreekt de HEERE: Dat zij verre van Mij; want die Mij eren, zal Ik eren, maar die Mij versmaden, zullen licht geacht worden.
\par 31 Zie, de dagen komen, dat Ik uw arm zal afhouwen, en den arm van uws vaders huis, dat er geen oud man in uw huis wezen zal.
\par 32 En gij zult aanschouwen de benauwdheid der woning Gods, in plaats van al het goede, dat Hij Israel zou gedaan hebben; en er zal te genen dage een oud man in uw huis zijn.
\par 33 Doch de man, dien Ik u niet zal uitroeien van Mijn altaar, zou zijn om uw ogen te verteren, en om uw ziel te bedroeven; en al de menigte uws huizes zal sterven, mannen geworden zijnde.
\par 34 Dit nu zal u een teken zijn, hetwelk over uw beide zonen, over Hofni en Pinehas, komen zal: op een dag zullen zij beiden sterven.
\par 35 En Ik zal Mij een getrouwen priester verwekken; die zal doen, gelijk als in Mijn hart en in Mijn ziel zijn zal; dien zal Ik een bestendig huis bouwen, en hij zal altijd voor het aangezicht Mijns Gezalfden wandelen.
\par 36 En het zal geschieden, dat al wie van uw huis zal overig zijn, zal komen, om zich voor hem neder te buigen voor een stukje gelds, en een bolle broods, en zal zeggen: Neem mij toch aan tot enige priesterlijke bediening, dat ik een bete broods moge eten.

\chapter{3}

\par 1 En de jongeling Samuel diende den HEERE voor het aangezicht van Eli; en het woord des HEEREN was dierbaar in die dagen; er was geen openbaar gezicht.
\par 2 En het geschiedde te dien dage, als Eli op zijn plaats nederlag (en zijn ogen begonnen donker te worden, dat hij niet zien kon),
\par 3 En Samuel zich ook nedergelegd had, eer de lampe Gods uitgedaan werd, in den tempel des HEEREN, waar de ark Gods was,
\par 4 Dat de HEERE Samuel riep; en hij zeide: Zie, hier ben ik.
\par 5 En hij liep tot Eli en zeide: Zie, hier ben ik, want gij hebt mij geroepen. Doch hij zeide: Ik heb niet geroepen, keer weder, leide u neder. En hij ging heen en legde zich neder.
\par 6 Toen riep de HEERE Samuel wederom; en Samuel stond op; en ging tot Eli, en zeide: Zie, hier ben ik, want gij hebt mij geroepen. Hij dan zeide: Ik heb u niet geroepen, mijn zoon; keer weder, leg u neder.
\par 7 Doch Samuel kende den HEERE nog niet; en het woord des HEEREN was aan hem nog niet geopenbaard.
\par 8 Toen riep de HEERE Samuel wederom, ten derden maal; en hij stond op, en ging tot Eli, en zeide: Zie, hier ben ik, want gij hebt mij geroepen. Toen verstond Eli, dat de HEERE den jongeling riep.
\par 9 Daarom zeide Eli tot Samuel: Ga heen, leg u neder, en het zal geschieden, zo Hij u roept, zo zult gij zeggen: Spreek, HEERE, want Uw knecht hoort. Toen ging Samuel heen en leide zich aan zijn plaats.
\par 10 Toen kwam de HEERE, en stelde Zich daar, en riep gelijk de andere malen: Samuel, Samuel! En Samuel zeide: Spreek, want Uw knecht hoort.
\par 11 En de HEERE zeide tot Samuel: Zie, Ik doe een ding in Israel, dat al wie het horen zal, dien zullen zijn beide oren klinken.
\par 12 Te dienzelven dage zal Ik verwekken over Eli alles, wat Ik tegen zijn huis gesproken heb; Ik zal het beginnen en voleinden.
\par 13 Want Ik heb hem te kennen gegeven, dat Ik zijn huis rechten zal tot in eeuwigheid, om der ongerechtigheids wil, die hij geweten heeft; want als zijn zonen zich hebben vervloekt gemaakt, zo heeft hij hen niet eens zuur aangezien.
\par 14 Daarom dan heb Ik het huis van Eli gezworen: Zo de ongerechtigheid van het huis van Eli tot in eeuwigheid zal verzoend worden door slachtoffer of door spijsoffer!
\par 15 Samuel nu lag tot aan den morgen; toen deed hij de deuren van het huis des HEEREN open; doch Samuel vreesde dit gezicht aan Eli te kennen te geven.
\par 16 Toen riep Eli Samuel, en zeide: Mijn zoon Samuel! Hij dan zeide: Zie, hier ben ik.
\par 17 En hij zeide: Wat is het woord, dat Hij tot u gesproken heeft? Verberg het toch niet voor mij; God doe u zo, en zo doe Hij daartoe, indien gij een woord voor mij verbergt van al de woorden, die Hij tot u gesproken heeft!
\par 18 Toen gaf hem Samuel te kennen al die woorden, en verborg ze voor hem niet. En hij zeide: Hij is de HEERE; Hij doe, wat goed is in Zijn ogen!
\par 19 Samuel nu werd groot; en de HEERE was met hem, en liet niet een van al Zijn woorden op de aarde vallen.
\par 20 En gans Israel, van Dan tot Ber-seba toe, bekende, dat Samuel bevestigd was tot een profeet des HEEREN.
\par 21 En de HEERE voer voort te verschijnen te Silo; want de HEERE openbaarde Zich aan Samuel te Silo, door het woord des HEEREN.

\chapter{4}

\par 1 En het woord van Samuel geschiedde aan gans Israel. En Israel toog uit, den Filistijnen tegemoet, ten strijde, en legerde zich bij Eben-haezer, maar de Filistijnen legerden zich bij Afek.
\par 2 En de Filistijnen stelden zich in slagorden, om Israel te ontmoeten; en als zich de strijd uitspreidde, zo werd Israel voor der Filistijnen aangezicht geslagen; want zij sloegen in de slagorden in het veld omtrent vier duizend man.
\par 3 Als het volk wederom in het leger gekomen was, zo zeiden de oudsten van Israel: Waarom heeft ons de HEERE heden geslagen voor het aangezicht der Filistijnen? Laat ons van Silo tot ons nemen de ark des verbonds des HEEREN, en laat die in het midden van ons komen, opdat zij ons verlosse van de hand onzer vijanden.
\par 4 Het volk dan zond naar Silo, en men bracht van daar de ark des verbonds des HEEREN der heirscharen, die tussen de cherubim woont; en de twee zonen van Eli, Hofni en Pinehas, waren daar met de ark des verbonds van God.
\par 5 En het geschiedde, als de ark des verbonds des HEEREN in het leger kwam, zo juichte gans Israel met een groot gejuich, alzo dat de aarde dreunde.
\par 6 Als nu de Filistijnen de stem van het juichen hoorden, zo zeiden zij: Wat is de stem van dit grote juichen in het leger der Hebreen? Toen vernamen zij, dat de ark des HEEREN in het leger gekomen was.
\par 7 Daarom vreesden de Filistijnen, want zij zeiden: God is in het leger gekomen. En zij zeiden: Wee ons, want diergelijke is gisteren en eergisteren niet geschied!
\par 8 Wee ons, wie zal ons redden uit de hand van deze heerlijke goden? Dit zijn dezelfde goden, die de Egyptenaars met alle plagen geplaagd hebben, bij de woestijn.
\par 9 Zijt sterk, en weest mannen, gij Filistijnen, opdat gij de Hebreen niet misschien dient, gelijk als zij ulieden gediend hebben; zo zijt mannen, en strijdt.
\par 10 Toen streden de Filistijnen, en Israel werd geslagen, en zij vloden een iegelijk in zijn tenten; en er geschiedde een zeer grote nederlaag, zodat er van Israel vielen dertig duizend voetvolks.
\par 11 En de ark Gods werd genomen, en de twee zonen van Eli, Hofni en Pinehas, stierven.
\par 12 Toen liep er een Benjaminiet uit de slagorden, en kwam te Silo denzelfden dag; en zijn klederen waren gescheurd, en er was aarde op zijn hoofd.
\par 13 En als hij kwam, ziet, zo zat Eli op een stoel aan de zijde van den weg, uitziende; want zijn hart was sidderende vanwege de ark Gods. Als die man kwam, om zulks te verkondigen in de stad, toen schreeuwde de ganse stad.
\par 14 En als Eli de stem des geroeps hoorde, zo zeide hij: Wat is de stem dezer beroerte? Toen haastte zich die man, en hij kwam en boodschapte het aan Eli.
\par 15 (Eli nu was een man van acht en negentig jaren, en zijn ogen stonden stijf, dat hij niet zien kon.)
\par 16 En die man zeide tot Eli: Ik ben het, die uit de slagorden kom, en ik ben heden uit de slagorden gevloden. Hij dan zeide: Wat is er geschied, mijn zoon?
\par 17 Toen antwoordde hij, die de boodschap bracht, en zeide: Israel is gevloden voor het aangezicht der Filistijnen, en er is ook een grote nederlaag onder het volk geschied; daarenboven zijn uw twee zonen, Hofni en Pinehas, gestorven, en de ark Gods is genomen.
\par 18 En het geschiedde, als hij van de ark Gods vermeldde, zo viel hij achterwaarts van den stoel af, aan de zijde der poort, en brak den nek, en stierf; want de man was oud en zwaar; en hij richtte Israel veertig jaren.
\par 19 En zijn schoondochter, de huisvrouw van Pinehas, was bevrucht, zij zou baren; als deze de tijding hoorde, dat de ark Gods genomen was, en haar schoonvader gestorven was, en haar man, zo kromde zij zich, en baarde; want haar weeen overvielen haar.
\par 20 En omtrent den tijd van haar sterven, zo spraken de vrouwen, die bij haar stonden: Vrees niet, want gij hebt een zoon gebaard. Doch zij antwoordde niet, en nam het niet ter harte.
\par 21 En zij noemde het jongsken Ikabod, zeggende: De eer is weggevoerd uit Israel! Omdat de ark Gods gevankelijk weggevoerd was, en om haars schoonvaders en haars mans wil.
\par 22 En zij zeide: De eer is gevankelijk weggevoerd uit Israel, want de ark Gods is genomen.

\chapter{5}

\par 1 De Filistijnen nu namen de ark Gods, en zij brachten ze van Eben-haezer tot Asdod.
\par 2 En de Filistijnen namen de ark Gods, en zij brachten ze in het huis van Dagon, en stelden ze bij Dagon.
\par 3 Maar als die van Asdod des anderen daags vroeg opstonden, ziet, zo was Dagon op zijn aangezicht ter aarde gevallen voor de ark des HEEREN. En zij namen Dagon en zetten hem weder op zijn plaats.
\par 4 Toen zij nu des anderen daags des morgens vroeg opstonden, ziet, Dagon lag op zijn aangezicht ter aarde gevallen voor de ark des HEEREN; maar het hoofd van Dagon, en de beide palmen zijner handen afgehouwen, aan den dorpel; alleenlijk was Dagon daarop overgebleven.
\par 5 Daarom treden de priesters van Dagon, en allen, die in het huis van Dagon komen, niet op den dorpel van Dagon te Asdod, tot op dezen dag.
\par 6 Doch de hand des HEEREN was zwaar over die van Asdod, en verwoestte hen; en Hij sloeg ze met spenen, Asdod en haar landpalen.
\par 7 Toen nu de mannen te Asdod zagen, dat het alzo toeging, zo zeiden zij: Dat de ark des Gods van Israel bij ons niet blijve; want Zijn hand is hard over ons, en over Dagon, onzen god.
\par 8 Daarom zonden zij heen, en verzamelden tot zich al de vorsten der Filistijnen, en zij zeiden: Wat zullen wij met de ark des Gods van Israel doen? En die zeiden: Dat de ark des Gods van Israel rondom Gath ga. Alzo droegen zij de ark des Gods van Israel rondom.
\par 9 En het geschiedde, nadat zij die hadden rondom gedragen, zo was de hand des HEEREN tegen die stad met een zeer grote kwelling; want Hij sloeg de lieden dier stad van den kleine tot den grote, en zij hadden spenen in de verborgene plaatsen.
\par 10 Toen zonden zij de ark Gods naar Ekron; maar het geschiedde, als de ark Gods te Ekron kwam, zo riepen die van Ekron, zeggende: Zij hebben de ark des Gods van Israel tot mij rondom gebracht, om mij en mijn volk te doden.
\par 11 En zij zonden heen, en vergaderden al de vorsten der Filistijnen, en zeiden: Zendt de ark des Gods van Israel heen, dat zij wederkere tot haar plaats, opdat zij mij en mijn volk niet dode; want er was een dodelijke kwelling in de ganse stad, en de hand Gods was er zeer zwaar.
\par 12 En de mensen, die niet stierven, werden geslagen met spenen, zodat het geschrei der stad opklom naar den hemel.

\chapter{6}

\par 1 Als nu de ark des HEEREN zeven maanden in het land der Filistijnen geweest was,
\par 2 Zo riepen de Filistijnen de priesters en de waarzeggers, zeggende: Wat zullen wij met de ark des HEEREN doen? Laat ons weten, waarmede wij ze aan haar plaats zenden zullen.
\par 3 Zij dan zeiden: Indien gij de ark des Gods van Israel wegzendt, zendt haar niet ledig weg, maar vergeldt Hem ganselijk een schuldoffer; dan zult gij genezen worden, en ulieden zal bekend worden, waarom Zijn hand van u niet afwijkt.
\par 4 Toen zeiden zij: Welk is dat schuldoffer, dat wij Hem vergelden zullen? En zij zeiden: Vijf gouden spenen, en vijf gouden muizen, naar het getal van de vorsten der Filistijnen; want het is enerlei plaag over u allen, en over uw vorsten.
\par 5 Zo maakt dan beelden uwer spenen, en beelden uwer muizen, die het land verderven, en geeft den God van Israel de eer; misschien zal Hij Zijn hand verlichten van over ulieden, en van over uw god, en van over uw land.
\par 6 Waarom toch zoudt gijlieden uw hart verzwaren, gelijk de Egyptenaars en Farao hun hart verzwaard hebben? Hebben zij niet, toen Hij wonderlijk met hen gehandeld had, hen laten trekken, dat zij heengingen?
\par 7 Nu dan, neemt en maakt een nieuwen wagen, en twee zogende koeien, op dewelke geen juk gekomen is; spant de koeien aan den wagen, en brengt haar kalveren van achter haar weder naar huis.
\par 8 Neemt dan de ark des HEEREN, en zet ze op den wagen, en legt de gouden kleinoden, die gij Hem ten schuloffer vergelden zult, in een koffertje aan haar zijde; en zendt ze weg, dat zij heenga.
\par 9 Ziet dan toe, indien zij den weg van haar landpale opgaat naar Beth-semes, zo heeft Hij ons dit groot kwaad gedaan; maar zo niet, zo zullen wij weten, dat Zijn hand ons niet geraakt heeft; het is ons een toeval geweest.
\par 10 En die lieden deden alzo, en namen twee zogende koeien, en spanden ze aan den wagen, en haar kalveren sloten zij in huis.
\par 11 En zij zetten de ark des HEEREN op den wagen, en het koffertje met de gouden muizen, en de beelden hunner spenen.
\par 12 De koeien nu gingen recht in dien weg, op den weg naar Beth-semes op een straat; zij gingen steeds voort, al loeiende, en weken noch ter rechter hand noch ter linkerhand; en de vorsten der Filistijnen gingen achter dezelve tot aan de landpale van Beth-semes.
\par 13 En die van Beth-semes maaiden den tarweoogst in het dal, en als zij hun ogen ophieven, zagen zij de ark en verblijdden zich, als zij die zagen.
\par 14 En de wagen kwam op den akker van Jozua, den Beth-semiet, en bleef daar staande; en daar was een grote steen, en zij kloofden het hout van den wagen, en offerden de koeien den HEERE ten brandoffer.
\par 15 En de Levieten namen de ark des HEEREN af en het koffertje, dat daarbij was, waarin de gouden kleinoden waren, en zetten ze op dien groten steen; en die lieden van Beth-semes offerden brandofferen, en slachtten slachtofferen den HEERE, op denzelven dag.
\par 16 En als de vijf vorsten der Filistijnen zulks gezien hadden, zo keerden zij weder op denzelven dag naar Ekron.
\par 17 Dit nu zijn de gouden spenen, die de Filistijnen aan den HEERE ten schuldoffer vergolden hebben: Voor Asdod een voor Gaza een, voor Askelon een, voor Gath een, voor Ekron een.
\par 18 Ook gouden muizen, naar het getal van alle steden der Filistijnen, onder de vijf vorsten, van de vaste steden af tot aan de landvlekken; en tot aan Abel, den groten steen, op denwelken zij de ark des HEEREN nedergesteld hadden, die tot op dezen dag is op den akker van Jozua, den Beth-semiet.
\par 19 En de Heere sloeg onder die lieden van Beth-semes, omdat zij in de ark des HEEREN gezien hadden; ja, Hij sloeg van het volk zeventig mannen, en vijftig duizend mannen. Toen bedreef het volk rouw, omdat de HEERE een groten slag onder het volk geslagen had.
\par 20 Toen zeiden de lieden van Beth-semes: Wie zou kunnen bestaan voor het aangezicht van den HEERE, dezen heiligen God? En tot wien van ons zal Hij optrekken?
\par 21 Zo zonden zij boden tot de inwoners van Kirjath-jearim, zeggende: De Filistijnen hebben de ark des HEEREN wedergebracht; komt af, haalt ze opwaarts tot u.

\chapter{7}

\par 1 Toen kwamen de mannen van Kirjath-jearim, en haalden de ark des HEEREN op, en zij brachten ze in het huis van Abinadab, op den heuvel; en zij heiligden zijn zoon Eleazar, dat hij de ark des HEEREN bewaarde.
\par 2 En het geschiedde, van dien dag af, dat de ark des Heeren te Kirjath-jearim bleef, en de dagen werden twintig jaren; en het ganse huis van Israel klaagde den HEERE achterna.
\par 3 Toen sprak Samuel tot het ganse huis van Israel, zeggende: Indien gijlieden u met uw ganse hart tot den HEERE bekeert, zo doet de vreemde goden uit het midden van u weg, ook de Astharoths; en richt uw hart tot den HEERE, en dient Hem alleen, zo zal Hij u uit de hand der Filistijnen rukken.
\par 4 De kinderen Israels nu deden de Baals en de Astharoths weg, en zij dienden den HEERE alleen.
\par 5 Verder zeide Samuel: Vergadert het ganse Israel naar Mizpa, en ik zal den HEERE voor u bidden.
\par 6 En zij werden vergaderd te Mizpa, en zij schepten water, en goten het uit voor het aangezicht des HEEREN; en zij vastten te dien dage, en zeiden aldaar: Wij hebben tegen den HEERE gezondigd. Alzo richtte Samuel de kinderen Israels te Mizpa.
\par 7 Toen de Filistijnen hoorden, dat de kinderen Israels zich vergaderd hadden te Mizpa, zo kwamen de oversten der Filistijnen op tegen Israel. Als de kinderen Israels dat hoorden, zo vreesden zij voor het aangezicht der Filistijnen.
\par 8 En de kinderen Israels zeiden tot Samuel: Zwijg niet van onzentwege, dat gij niet zoudt roepen tot den HEERE, onzen God, opdat Hij ons verlosse uit de hand der Filistijnen.
\par 9 Toen nam Samuel een melklam, en hij offerde het geheel den HEERE ten brandoffer; en Samuel riep tot den HEERE voor Israel; en de HEERE verhoorde hem.
\par 10 En het geschiedde, toen Samuel dat brandoffer offerde, zo kwamen de Filistijnen aan ten strijde tegen Israel; en de HEERE donderde te dien dage met een groten donder over de Filistijnen, en Hij verschrikte hen, zodat zij verslagen werden voor het aangezicht van Israel.
\par 11 En de mannen van Israel togen uit van Mizpa, en vervolgden de Filistijnen, en zij sloegen hen tot onder Beth-kar.
\par 12 Samuel nu nam een steen, en stelde dien tussen Mizpa en tussen Sen, en hij noemde diens naam Eben-haezer; en hij zeide: Tot hiertoe heeft de HEERE ons geholpen.
\par 13 Alzo werden de Filistijnen vernederd, en kwamen niet meer in de landpalen van Israel; want de hand des HEEREN was tegen de Filistijnen al de dagen van Samuel.
\par 14 En de steden, welke de Filistijnen van Israel genomen hadden kwamen weder aan Israel, van Ekron tot Gath toe; ook rukte Israel derzelver landpale uit de hand der Filistijnen; en er was vrede tussen Israel en tussen de Amorieten.
\par 15 Samuel nu richtte Israel al de dagen zijns levens.
\par 16 En hij toog van jaar tot jaar, en ging rondom naar Beth-el, en Gilgal, en Mizpa; en hij richtte Israel in al die plaatsen.
\par 17 Doch hij keerde weder naar Rama; want daar was zijn huis, en daar richtte hij Israel; en hij bouwde aldaar den HEERE een altaar.

\chapter{8}

\par 1 Het geschiedde nu, toen Samuel oud geworden was, zo stelde hij zijn zonen tot richters over Israel.
\par 2 De naam van zijn eerstgeborenen zoon nu was Joel, en de naam van zijn tweeden was Abia; zij waren richters te Ber-seba.
\par 3 Doch zijn zonen wandelden niet in zijn wegen; maar zij neigden zich tot de gierigheid, en namen geschenken, en bogen het recht.
\par 4 Toen vergaderden zich alle oudsten van Israel, en zij kwamen tot Samuel te Rama;
\par 5 En zij zeiden tot hem: Zie, gij zijt oud geworden, en uw zonen wandelen niet in uw wegen; zo zet nu een koning over ons, om ons te richten, gelijk al de volken hebben.
\par 6 Maar dit woord was kwaad in de ogen van Samuel, als zij zeiden: Geef ons een koning, om ons te richten. En Samuel bad den HEERE aan.
\par 7 Doch de HEERE zeide tot Samuel: Hoor naar de stem des volks in alles, wat zij tot u zeggen zullen; want zij hebben u niet verworpen, maar zij hebben Mij verworpen, dat Ik geen Koning over hen zal zijn.
\par 8 Naar de werken, die zij gedaan hebben, van dien dag af, toen Ik hen uit Egypte geleid heb, tot op dezen dag toe, en hebben Mij verlaten en andere goden gediend; alzo doen zij u ook.
\par 9 Hoor dan nu naar hun stem; doch als gij hen op het hoogste zult betuigd hebben, zo zult gij hen te kennen geven de wijze des konings, die over hen regeren zal.
\par 10 Samuel nu zeide al de woorden des HEEREN het volk aan, hetwelk een koning van hem begeerde.
\par 11 En zeide: Dit zal des konings wijze zijn, die over u regeren zal: hij zal uw zonen nemen, dat hij hen zich stelle tot zijn wagen, en tot zijn ruiteren, dat zij voor zijn wagen henen lopen;
\par 12 En dat hij hen zich stelle tot oversten der duizenden, en tot oversten der vijftigen; en dat zij zijn akker ploegen, en dat zij zijn oogst oogsten, en dat zij zijn krijgswapenen maken, mitsgaders zijn wapentuig.
\par 13 En uw dochteren zal hij nemen tot apothekeressen, en tot keukenmaagden, en tot baksters.
\par 14 En uw akkers, en uw wijngaarden, en uw olijfgaarden, die de beste zijn, zal hij nemen, en zal ze aan zijn knechten geven.
\par 15 En uw zaad, en uw wijngaarden zal hij vertienen, en hij zal ze aan zijn hovelingen, en aan zijn knechten geven.
\par 16 En hij zal uw knechten, en uw dienstmaagden, en uw beste jongelingen, en uw ezelen nemen, en hij zal zijn werk daarmede doen.
\par 17 Hij zal uw kudden vertienen; en gij zult hem tot knechten zijn.
\par 18 Gij zult wel te dien dage roepen, vanwege uw koning, dien gij u zult verkoren hebben, maar de HEERE zal u te dien dage niet verhoren.
\par 19 Doch het volk weigerde Samuels stem te horen; en zij zeiden: Neen, maar er zal een koning over ons zijn.
\par 20 En wij zullen ook zijn gelijk al de volken; en onze koning zal ons richten, en hij zal voor onze aangezichten uitgaan, en hij zal onze krijgen voeren.
\par 21 Als Samuel al de woorden des volks gehoord had, zo sprak hij dezelve voor de oren des HEEREN.
\par 22 De HEERE nu zeide tot Samuel: Hoor naar hun stem, en stel hun een koning. Toen zeide Samuel tot de mannen van Israel: Gaat heen, een iegelijk naar zijn stad.

\chapter{9}

\par 1 Er was nu een man van Benjamin, wiens naam was Kis, een zoon van Abiel, den zoon van Zeror, den zoon van Bechorath, den zoon van Afiah, den zoon eens mans van Jemini, een dapper held.
\par 2 Die had een zoon, wiens naam was Saul, een jongeling, en schoon, ja, er was geen schoner man dan hij onder de kinderen Israels; van zijn schouderen en opwaarts was hij hoger dan al het volk.
\par 3 De ezelinnen nu van Kis, den vader van Saul, waren verloren; daarom zeide Kis tot zijn zoon Saul: Neem nu een van de jongens met u, en maak u op, ga heen, zoek de ezelinnen.
\par 4 Hij dan ging door het gebergte van Efraim, en hij ging door het land van Salisa, maar zij vonden ze niet; daarna gingen zij door het land van Sahalim, maar zij waren er niet; verder ging hij door het land van Jemini, doch zij vonden ze niet.
\par 5 Toen zij in het land van Zuf kwamen, zeide Saul tot zijn jongen, die bij hem was: Kom en laat ons wederkeren; dat niet misschien mijn vader van de ezelinnen aflate, en voor ons bekommerd zij.
\par 6 Hij daarentegen zeide tot hem: Zie toch, er is een man Gods in deze stad, en hij is een geeerd man; al wat hij spreekt, dat komt zekerlijk; laat ons nu derwaarts gaan, misschien zal hij ons onzen weg aanwijzen, op denwelken wij gaan zullen.
\par 7 Toen zeide Saul tot zijn jongen: Maar zie, zo wij gaan, wat zullen wij toch dien man brengen? Want het brood is weg uit onze vaten, en wij hebben geen gaven, om den man Gods te brengen; wat hebben wij?
\par 8 En de jongen antwoordde Saul verder en zeide: Zie, er vindt zich in mijn hand het vierendeel eens zilveren sikkels; dat zal ik den man Gods geven, opdat hij ons onzen weg wijze.
\par 9 (Eertijds zeide een ieder aldus in Israel, als hij ging om God te vragen: Komt en laat ons gaan tot den ziener; want die heden een profeet genoemd wordt, die werd eertijds een ziener genoemd.)
\par 10 Toen zeide Saul tot zijn jongen: Uw woord is goed, kom, laat ons gaan. En zij gingen naar de stad, waar de man Gods was.
\par 11 Als zij opklommen door den opgang der stad, zo vonden zij maagden, die uitgingen om water te putten; en zij zeiden tot haar: Is de ziener hier?
\par 12 Toen antwoordden zij hun, en zeiden: Ziet, hij is voor uw aangezicht; haast u nu, want hij is heden in de stad gekomen, dewijl het volk heden een offerande heeft op de hoogte.
\par 13 Wanneer gijlieden in de stad komt, zo zult gij hem vinden, eer hij opgaat op de hoogte om te eten; want het volk zal niet eten, totdat hij komt, want hij zegent het offer, daarna eten de genodigden; daarom gaat nu op, want hem, als heden zult gij hem vinden.
\par 14 Alzo gingen zij op in de stad. Toen zij in het midden der stad kwamen, ziet, zo ging Samuel uit hun tegemoet, om op te gaan naar de hoogte.
\par 15 Want de HEERE had het voor Samuels oor geopenbaard, een dag eer Saul kwam, zeggende:
\par 16 Morgen omtrent dezen tijd zal Ik tot u zenden een man uit het land van Benjamin, dien zult gij ten voorganger zalven over Mijn volk Israel; en hij zal Mijn volk verlossen uit der Filistijnen hand, want Ik heb Mijn volk aangezien, dewijl deszelfs geroep tot Mij gekomen is.
\par 17 Toen Samuel Saul aanzag, zo antwoordde hem de HEERE: Zie, dit is de man, van welken Ik u gezegd heb: Deze zal over Mijn volk heersen.
\par 18 En Saul naderde tot Samuel in het midden der poort, en zeide: Wijs mij toch, waar is hier het huis des zieners?
\par 19 En Samuel antwoordde Saul en zeide: Ik ben de ziener; ga op voor mijn aangezicht op de hoogte, dat gijlieden heden met mij eet; zo zal ik u morgen vroeg laten gaan, en alles, wat in uw hart is, zal ik u te kennen geven.
\par 20 Want de ezelinnen aangaande, die gij heden den derden dag verloren hebt, zet uw hart daarop niet, want zij zijn gevonden; en wiens zal zijn al het gewenste, dat in Israel is? Is het niet van u, en van het ganse huis uws vaders?
\par 21 Toen antwoordde Saul, en zeide: Ben ik niet een zoon van Jemini, van den kleinsten der stammen van Israel? en mijn geslacht is het niet het kleinste van al de geslachten van den stam van Benjamin? Waarom spreekt gij mij dan aan met zulke woorden?
\par 22 Samuel dan nam Saul en zijn jongen, en hij bracht ze in de kamer; en hij gaf hun plaats aan het opperste der genodigden; die nu waren omtrent dertig man.
\par 23 Toen zeide Samuel tot den kok: Lang dat stuk, hetwelk Ik u gegeven heb, waarvan ik tot u zeide: Zet het bij u weg.
\par 24 De kok nu bracht een schouder op, met wat daaraan was, en zette het voor Saul; en hij zeide: Zie, dit is het overgeblevene; zet het voor u, eet, want het is ter bestemder tijd voor u bewaard, als ik zeide: Ik heb het volk genodigd. Alzo at Saul met Samuel op dien dag.
\par 25 Daarna gingen zij af van de hoogte in de stad; en hij sprak met Saul op het dak.
\par 26 En zij stonden vroeg op; en het geschiedde, omtrent den opgang des dageraads, zo riep Samuel Saul op het dak, zeggende: Sta op, dat ik u gaan late. Toen stond Saul op, en zij beiden gingen uit, hij en Samuel, naar buiten.
\par 27 Toen zij afgegaan waren aan het einde der stad, zo zeide Samuel tot Saul: Zeg den jongen, dat hij voor onze aangezichten heenga; toen ging hij heen; maar sta gij als nu stil, en ik zal u Gods woord doen horen.

\chapter{10}

\par 1 Toen nam Samuel een oliekruik, en goot ze uit op zijn hoofd, en kuste hem, en zeide: Is het niet alzo, dat de HEERE u tot een voorganger over Zijn erfdeel gezalfd heeft?
\par 2 Als gij heden van mij gaat, zo zult gij twee mannen vinden bij het graf van Rachel, aan de landpale van Benjamin, te Zelzah; die zullen tot u zeggen: De ezelinnen zijn gevonden, die gij zijt gaan zoeken, en zie, uw vader heeft de zaken der ezelinnen verlaten, en hij is bekommerd voor ulieden, zeggende: Wat zal ik om mijn zoon doen?
\par 3 Als gij u van daar en verder aan begeeft, en zult komen tot aan Elon-thabor, daar zullen u drie mannen vinden, opgaande tot God naar Beth-el; een, dragende drie bokjes, en een, dragende drie bollen broods, en een, dragende een fles wijn.
\par 4 En zij zullen u naar uw welstand vragen, en zij zullen u twee broden geven; die zult gij van hun hand nemen.
\par 5 Daarna zult gij komen op den heuvel Gods, waar der Filistijnen bezettingen zijn; en het zal geschieden, als gij aldaar in de stad komt, zo zult gij ontmoeten een hoop profeten, van de hoogte afkomende, en voor hun aangezichten luiten, en trommelen, en pijpen, en harpen, en zij zullen profeteren.
\par 6 En de Geest des HEEREN zal vaardig worden over u, en gij zult met hen profeteren; en gij zult in een anderen man veranderd worden.
\par 7 En het zal geschieden, als u deze tekenen zullen komen, doe gij, wat uw hand vinden zal, want God zal met u zijn.
\par 8 Gij nu zult voor mijn aangezicht afgaan naar Gilgal, en zie, ik zal tot u afkomen, om brandofferen te offeren, om te offeren offeranden der dankzegging; zeven dagen zult gij daar beiden, totdat ik tot u kome, en u bekend make, wat gij doen zult.
\par 9 Het geschiedde nu, toen hij zijn schouder keerde, om van Samuel te gaan, veranderde God hem het hart in een ander; en al die tekenen kwamen ten zelven dage.
\par 10 Toen zij daar aan den heuvel kwamen, zie, zo kwam hem een hoop profeten tegemoet; en de Geest des HEEREN werd vaardig over hem, en hij profeteerde in het midden van hen.
\par 11 En het geschiedde, als een iegelijk, die hem van te voren gekend had, zag, dat hij, ziet, profeteerde met de profeten, zo zeide het volk, een ieder tot zijn metgezel: Wat is dit, dat den zoon van Kis geschied is? Is Saul ook onder de profeten?
\par 12 Toen antwoordde een man van daar, en zeide: Wie is toch hun vader? Daarom is het tot een spreekwoord geworden: Is Saul ook onder de profeten?
\par 13 Toen hij nu voleind had te profeteren, zo kwam hij op de hoogte.
\par 14 En Sauls oom zeide tot hem en tot zijn jongen: Waar zijt gijlieden heengegaan? Hij nu zeide: Om de ezelinnen te zoeken; toen wij zagen, dat zij er niet waren, zo kwamen wij tot Samuel.
\par 15 Toen zeide Sauls oom: Geef mij toch te kennen, wat heeft Samuel ulieden gezegd?
\par 16 Saul nu zeide tot zijn oom: Hij heeft ons voorzeker te kennen gegeven, dat de ezelinnen gevonden waren; maar de zaak des koninkrijks, waarvan Samuel gezegd had, gaf hij hem niet te kennen.
\par 17 Doch Samuel riep het volk te zamen tot den HEERE, te Mizpa.
\par 18 En hij zeide tot de kinderen Israels: Alzo heeft de HEERE, de God Israel, gesproken: Ik heb Israel uit Egypte opgebracht, en Ik heb ulieden van de hand der Egyptenaren gered, en van de hand van alle koninkrijken, die u onderdrukten.
\par 19 Maar gijlieden hebt heden uw God verworpen, Die u uit al uw ellenden en uw noden verlost heeft, en hebt tot Hem gezegd: Zet een koning over ons; nu dan, stelt u voor het aangezicht des HEEREN, naar uw stammen en naar uw duizenden.
\par 20 Toen nu Samuel al de stammen van Israel had doen naderen, zo is de stam van Benjamin geraakt.
\par 21 Toen hij den stam van Benjamin deed aankomen naar zijn geslachten, zo werd het geslacht van Matri geraakt; en Saul, de zoon van Kis, werd geraakt. En zij zochten hem, maar hij werd niet gevonden.
\par 22 Toen vraagden zij verder den HEERE, of die man nog derwaarts komen zou? De HEERE dan zeide: Ziet, hij heeft zich tussen de vaten verstoken.
\par 23 Zij nu liepen, en namen hem van daar, en hij stelde zich in het midden des volks; en hij was hoger dan al het volk, van zijn schouder en opwaarts.
\par 24 Toen zeide Samuel tot het ganse volk: Ziet gij, dien de HEERE verkoren heeft? Want gelijk hij, is er niemand onder het ganse volk. Toen juichte het ganse volk, en zij zeiden: de koning leve!
\par 25 Samuel nu sprak tot het volk het recht des koninkrijks, en schreef het in een boek, en leide het voor het aangezicht des HEEREN. Toen liet Samuel het ganse volk gaan, elk naar zijn huis.
\par 26 En Saul ging ook naar zijn huis te Gibea, en van het heir gingen met hem, welker hart God geroerd had.
\par 27 Doch de kinderen Belials zeiden: Wat zou ons deze verlossen? en zij verachtten hem, en brachten hem geen geschenk. Doch hij was als doof.

\chapter{11}

\par 1 Toen toog Nahas, de Ammoniet, op, en belegerde Jabes in Gilead. En al de mannen van Jabes zeiden tot Nahas: Maak een verbond met ons, zo zullen wij u dienen.
\par 2 Doch Nahas, de Ammoniet, zeide tot hen: Mits dezen zal ik een verbond met ulieden maken, dat ik u allen het rechteroog uitsteke; en dat ik deze schande op gans Israel legge.
\par 3 Toen zeiden tot hem de oudsten Jabes: Laat zeven dagen van ons af, dat wij boden zenden in al de landpalen van Israel; is er dan niemand, die ons verlost, zo zullen wij tot u uitgaan.
\par 4 Als de boden te Gibea-sauls kwamen, zo spraken zij deze woorden voor de oren van het volk. Toen hief al het volk zijn stem op, en weende.
\par 5 En ziet, Saul kwam achter de runderen uit het veld, en Saul zeide: Wat is den volke, dat zij wenen? Toen vertelden zij hem de woorden der mannen van Jabes.
\par 6 Toen werd de Geest Gods vaardig over Saul, als hij deze woorden hoorde; en zijn toorn ontstak zeer.
\par 7 En hij nam een paar runderen, en hieuw ze in stukken, en hij zond ze in alle landpalen van Israel door de hand der boden, zeggende: Die niet zelf uittrekt achter Saul en achter Samuel, alzo zal men zijn runderen doen. Toen viel de vreze des HEEREN op het volk, en zij gingen uit als een enig man.
\par 8 En hij telde hen te Bezek; en van de kinderen Israels waren driehonderd duizend, en van de mannen van Juda dertig duizend.
\par 9 Toen zeiden zij tot de boden, die gekomen waren: Aldus zult gijlieden den mannen te Jabes in Gilead zeggen: Morgen zal u verlossing geschieden, als de zon heet worden zal. Als de boden kwamen, en verkondigden dat aan de mannen te Jabes, zo werden zij verblijd.
\par 10 En de mannen van Jabes zeiden: Morgen zullen wij tot ulieden uitgaan, en gij zult ons doen naar alles, wat goed is in uw ogen.
\par 11 Het geschiedde nu des anderen daags, dat Saul het volk stelde in drie hopen, en zij kwamen in het midden des legers, in de morgenwake, en zij sloegen Ammon, totdat de dag heet werd; en het geschiedde, dat de overigen alzo verstrooid werden, dat er onder hen geen twee te zamen bleven.
\par 12 Toen zeide het volk tot Samuel: Wie is hij, die zeide: Zou Saul over ons regeren? Geeft hier die mannen, dat wij hen doden.
\par 13 Maar Saul zeide: Er zal te dezen dage geen man gedood worden, want de HEERE heeft heden een verlossing in Israel gedaan.
\par 14 Verder zeide Samuel tot het volk: Komt en laat ons naar Gilgal gaan, en het koninkrijk aldaar vernieuwen.
\par 15 Toen ging al het volk naar Gilgal, en maakte Saul aldaar koning voor het aangezicht des HEEREN te Gilgal; en zij offerden aldaar dankofferen voor het aangezicht des HEEREN; en Saul verheugde zich aldaar gans zeer, met al de mannen van Israel.

\chapter{12}

\par 1 Toen zeide Samuel tot gans Israel: Ziet, ik heb naar ulieder stem gehoord in alles, wat gij mij gezegd hebt, en ik heb een koning over u gezet.
\par 2 En nu, ziet, daar trekt de koning voor uw aangezicht heen, en ik ben oud en grijs geworden, en ziet, mijn zonen zijn bij ulieden; en ik heb voor uw aangezichten gewandeld van mijn jeugd af tot dezen dag toe.
\par 3 Ziet, hier ben ik, betuigt tegen mij voor den HEERE, en voor Zijn gezalfde, wiens os ik genomen heb, en wiens ezel ik genomen heb, en wien ik verongelijkt heb, wien ik onderdrukt heb, en van wiens hand ik een geschenk genomen heb, dat ik mijn ogen van hem zou verborgen hebben; zo zal ik het ulieden wedergeven.
\par 4 Toen zeiden zij: Gij hebt ons niet verongelijkt, en gij hebt ons niet onderdrukt, en gij hebt van niemands hand iets genomen.
\par 5 Toen zeide hij tot hen: De HEERE zij een Getuige tegen ulieden, en Zijn gezalfde zij te dezen dage getuige, dat gij in mijn hand niets gevonden hebt! En het volk zeide: Hij zij Getuige!
\par 6 Verder zeide Samuel tot het volk: Het is de HEERE, Die Mozes en Aaron gemaakt heeft, en Die uw vaders uit Egypteland opgebracht heeft.
\par 7 En nu, stelt u hier, dat ik met ulieden rechte, voor het aangezicht des HEEREN, over al de gerechtigheden des HEEREN, die Hij aan u en aan uw vaderen gedaan heeft.
\par 8 Nadat Jakob in Egypte gekomen was, zo riepen uw vaders tot den HEERE; en de HEERE zond Mozes en Aaron, en zij leidden uw vaders uit Egypte, en deden hen aan deze plaats wonen.
\par 9 Maar zij vergaten den HEERE, hun God; zo verkocht Hij hen in de hand van Sisera, den krijgsoverste, te Hazor, en in de hand der Filistijnen, en in de hand van den koning der Moabieten, die tegen hen streden.
\par 10 En zij riepen tot den HEERE, en zeiden: Wij hebben gezondigd, dewijl wij den HEERE verlaten, en de Baals en Astharoths gediend hebben; en nu, ruk ons uit de hand onzer vijanden, en wij zullen U dienen.
\par 11 En de HEERE zond Jerubbaal, en Bedan, en Jeftha, en Samuel, en Hij rukte u uit de hand uwer vijanden rondom, alzo dat gij zeker woondet.
\par 12 Als gij nu zaagt, dat Nahas, de koning van de kinderen Ammons, tegen u kwam, zo zeidet gij tot mij: Neen, maar een koning zal over ons regeren; zo toch de HEERE, uw God, uw Koning was.
\par 13 En nu, ziet daar den koning, dien gij verkoren hebt, dien gij begeerd hebt; en ziet, de HEERE heeft een koning over ulieden gezet.
\par 14 Zo gij den HEERE zult vrezen, en Hem dienen, en naar Zijn stem horen, en den mond des HEEREN niet wederspannig zijt, zo zult gijlieden, zowel gij als de koning, die over u regeren zal, achter den HEERE, uw God, zijn.
\par 15 Doch zo gij naar de stem des HEEREN niet zult horen, maar den mond des HEEREN wederspannig zijn, zo zal de hand des HEEREN, tegen u zijn, als tegen uw vaders.
\par 16 Ook stelt u nu hier, en ziet die grote zaak, die de HEERE voor uw ogen doen zal.
\par 17 Is het niet vandaag de tarweoogst? Ik zal tot den HEERE roepen, en Hij zal donder en regen geven; zo weet dan, en ziet, dat uw kwaad groot is, dat gij voor de ogen des HEEREN gedaan hebt, dat gij een koning voor u begeerd hebt.
\par 18 Toen Samuel den HEERE aanriep, zo gaf de HEERE donder en regen te dien dage; daarom vreesde al het volk zeer den HEERE en Samuel.
\par 19 En al het volk zeide tot Samuel: Bid voor uw knechten den HEERE, uw God, dat wij niet sterven; want boven al onze zonden hebben wij dit kwaad daartoe gedaan, dat wij voor ons een koning begeerd hebben.
\par 20 Toen zeide Samuel tot het volk: Vreest niet, gij hebt al dit kwaad gedaan; doch wijkt niet van achter den HEERE af, maar dient den HEERE met uw ganse hart.
\par 21 En wijkt niet af; want gij zoudt de ijdelheden na volgen, die niet bevorderlijk zijn, noch verlossen, want zij zijn ijdelheden.
\par 22 Want de HEERE zal Zijn volk niet verlaten, om Zijns groten Naams wil, dewijl het den HEERE beliefd heeft, ulieden Zich tot een volk te maken.
\par 23 Wat ook mij aangaat, het zij verre van mij, dat ik tegen den HEERE zou zondigen, dat ik zou aflaten voor ulieden te bidden; maar ik zal u den goeden en rechten weg leren.
\par 24 Vreest slechts den HEERE, en dient Hem trouwelijk met uw ganse hart; want ziet, hoe grote dingen Hij bij ulieden gedaan heeft!
\par 25 Maar indien gij voortaan kwaad doet, zo zult gijlieden, als ook uw koning, omkomen.

\chapter{13}

\par 1 Saul was een jaar in zijn regering geweest, en het tweede jaar regeerde hij over Israel.
\par 2 Toen verkoos zich Saul drie duizend mannen uit Israel; en er waren bij Saul twee duizend te Michmas en op het gebergte van Beth-el, en duizend waren er bij Jonathan te Gibea-benjamins; en het overige des volks liet hij gaan, een iegelijk naar zijn tent.
\par 3 Doch Jonathan sloeg de bezetting der Filistijnen, die te Geba was, hetwelk de Filistijnen hoorden. Daarom blies Saul met de bazuin in het ganse land, zeggende: Laat het de Hebreen horen.
\par 4 Toen hoorde het ganse Israel zeggen: Saul heeft de bezetting der Filistijnen geslagen, en ook is Israel stinkende geworden bij de Filistijnen. Toen werd het volk samengeroepen achter Saul, naar Gilgal.
\par 5 En de Filistijnen werden verzameld om te strijden tegen Israel, dertig duizend wagens, en zes duizend ruiters, en volk in menigte als het zand, dat aan den oever der zee is; en zij togen op, en legerden zich te Michmas, tegen het oosten van Beth-aven.
\par 6 Toen de mannen van Israel zagen, dat zij in nood waren (want het volk was benauwd), zo verborg zich het volk in de spelonken, en in de doornbossen, en in de steenklippen, en in de vestingen, en in de putten.
\par 7 De Hebreen nu gingen over de Jordaan in het land van Gad en Gilead. Toen Saul nog zelf te Gilgal was, zo kwam al het volk bevende achter hem.
\par 8 En hij vertoefde zeven dagen, tot den tijd, dien Samuel bestemd had. Als Samuel te Gilgal niet opkwam, zo verstrooide het volk van hem.
\par 9 Toen zeide Saul: Brengt tot mij herwaarts een brandoffer, en dankofferen; en hij offerde brandoffer.
\par 10 En het geschiedde, toen hij geeindigd had het brandoffer te offeren, ziet, zo kwam Samuel; en Saul ging uit hem tegemoet, om hem te zegenen.
\par 11 Toen zeide Samuel: Wat hebt gij gedaan? Saul nu zeide: Omdat ik zag, dat zich het volk van mij verstrooide, en gij op den bestemden tijd der dagen niet kwaamt, en de Filistijnen te Michmas vergaderd waren,
\par 12 Zo zeide ik: Nu zullen de Filistijnen tot mij afkomen te Gilgal, en ik heb het aangezicht des HEEREN niet ernstelijk aangebeden, zo dwong ik mijzelven, en heb brandoffer geofferd.
\par 13 Toen zeide Samuel tot Saul: Gij hebt zottelijk gedaan; gij hebt het gebod van den HEERE, uw God, niet gehouden, dat Hij u geboden heeft; want de HEERE zou nu uw rijk over Israel bevestigd hebben tot in eeuwigheid.
\par 14 Maar nu zal uw rijk niet bestaan. De HEERE heeft Zich een man gezocht naar Zijn hart, en de HEERE heeft hem geboden een voorganger te zijn over Zijn volk, omdat gij niet gehouden hebt, wat u de HEERE geboden had.
\par 15 Toen maakte zich Samuel op, en hij ging op van Gilgal naar Gibea-benjamins; en Saul telde het volk, dat bij hem gevonden werd, omtrent zeshonderd man.
\par 16 En Saul en zijn zoon Jonathan, en het volk, dat bij hen gevonden was, bleven te Gibea-benjamins; maar de Filistijnen waren te Michmas gelegerd.
\par 17 En de verdervers gingen uit het leger der Filistijnen, in drie hopen; de ene hoop keerde zich op den weg naar Ofra, naar het land Sual;
\par 18 En een hoop keerde zich naar den weg van Beth-horon; en een hoop keerde zich naar den weg der landpale, die naar het dal Zeboim naar de woestijn uitziet.
\par 19 En er werd geen smid gevonden in het ganse land van Israel; want de Filistijnen hadden gezegd: Opdat de Hebreen geen zwaard noch spies maken.
\par 20 Daarom moest gans Israel tot de Filistijnen aftrekken, opdat een iegelijk zijn ploegijzer, of zijn spade, of zijn bijl, of zijn houweel scherpen liet.
\par 21 Maar zij hadden tandige vijlen tot hun houwelen, en tot hun spaden, en tot de drietandige vorken, en tot de bijlen, en tot het stellen der prikkelen.
\par 22 En het geschiedde ten dage des strijds, dat er geen zwaard noch spies gevonden werd in de hand van het ganse volk, dat bij Saul en bij Jonathan was; doch bij Saul en bij Jonathan, zijn zoon, werden zij gevonden.
\par 23 En der Filistijnen leger toog naar den doortocht van Michmas.

\chapter{14}

\par 1 Het geschiedde nu op een dag, dat Jonathan, de zoon van Saul, tot den jongen, die zijn wapenen droeg, zeide: Kom, en laat ons tot de bezetting der Filistijnen overgaan, welke aan gene zijde is; doch hij gaf het zijn vader niet te kennen.
\par 2 Saul nu zat aan het uiterste van Gibea onder den granatenboom, die te Migron was; en het volk, dat bij hem was, was omtrent zeshonderd man.
\par 3 En Ahia, de zoon van Ahitub, den broeder van Ikabod, den zoon van Pinehas, den zoon van Eli, was priester des HEEREN, te Silo, dragende den efod; doch het volk wist niet, dat Jonathan heengegaan was.
\par 4 Er was nu tussen de doortochten, waar Jonathan zocht door te gaan tot der Filistijnen bezetting, een scherpte van een steenklip aan deze zijde, en een scherpte van een steenklip aan gene zijde; en de naam der ene was Bozes, en de naam der andere Sene.
\par 5 De ene tand was gelegen tegen het noorden, tegenover Michmas, en de andere tegen het zuiden, tegenover Geba.
\par 6 Jonathan nu zeide tot den jongen, die zijn wapenen droeg: Kom, en laat ons tot de bezetting dezer onbesnedenen overgaan; misschien zal de HEERE voor ons werken; want bij den HEERE is geen verhindering, om te verlossen door velen of door weinigen.
\par 7 Toen zeide zijn wapendrager tot hem: Doe al, wat in uw hart is; wend u, zie ik ben met u, naar uw hart.
\par 8 Jonathan nu zeide: Zie, wij zullen overgaan tot die mannen, en wij zullen ons aan hen ontdekken.
\par 9 Indien zij aldus tot ons zeggen: Staat stil, totdat wij aan ulieden komen; zo zullen wij blijven staan aan onze plaats, en tot hen niet opklimmen.
\par 10 Maar zeggen zij aldus: Klimt tot ons op; zo zullen wij opklimmen, want de HEERE heeft hen in onze hand gegeven; en dit zal ons een teken zijn.
\par 11 Toen zij beiden zich aan der Filistijnen bezetting ontdekten, zo zeiden de Filistijnen: Ziet, de Hebreen zijn uit de holen uitgegaan, waarin zij zich verstoken hadden.
\par 12 Verder antwoordden de mannen der bezetting aan Jonathan en zijn wapendrager, en zeiden: Klimt op tot ons, en wij zullen het u wijs maken. En Jonathan zeide tot zijn wapendrager: Klim op achter mij, want de HEERE heeft hen gegeven in de hand van Israel.
\par 13 Toen klom Jonathan op zijn handen en op zijn voeten, en zijn wapendrager hem na; en zij vielen voor Jonathans aangezicht, en zijn wapendrager doodde ze achter hem.
\par 14 Deze eerste slag nu, waarmede Jonathan en zijn wapendrager omtrent twintig mannen versloegen, geschiedde omtrent in de helft eens bunders, zijnde een juk ossen lands.
\par 15 En er was een beving in het leger, op het veld en onder het ganse volk; de bezetting en de verdervers beefden ook zelven; ja, het land werd beroerd, want het was een beving Gods.
\par 16 Als nu de wachters van Saul te Gibea-benjamins zagen, dat, ziet, de menigte versmolt, en doorging, en geklopt werd;
\par 17 Toen zeide Saul tot het volk, dat bij hem was: Telt toch, en beziet, wie van ons weggegaan zijn. En zij telden, en ziet, Jonathan en zijn wapendrager waren daar niet.
\par 18 Toen zeide Saul tot Ahia: Breng de ark Gods herwaarts. Want de ark Gods was te dien dage bij de kinderen Israels.
\par 19 En het geschiedde, toen Saul nog tot den priester sprak, dat het rumoer, hetwelk in der Filistijnen leger was, zeer toenam en vermenigvuldigde; zo zeide Saul tot den priester: Haal uw hand in.
\par 20 Saul nu, en al het volk, dat bij hem was, werd samengeroepen, en zij kwamen ten strijde; en ziet, het zwaard des enen was tegen den anderen, er was een zeer groot gedruis.
\par 21 Er waren ook Hebreen bij de Filistijnen, als eertijds, die met hen in het leger opgetogen waren rondom; dezen nu vervoegden zich ook met de Israelieten, die bij Saul en Jonathan waren.
\par 22 Als alle mannen van Israel, die zich verstoken hadden in het gebergte van Efraim, hoorden, dat de Filistijnen vluchtten, zo kleefden zij ook hen achteraan in den strijd.
\par 23 Alzo verloste de HEERE Israel te dien dage; en het leger trok over naar Beth-aven.
\par 24 En de mannen van Israel werden mat te dien dage; want Saul had het volk bezworen, zeggende: Vervloekt zij de man, die spijze eet tot aan den avond, opdat ik mij aan mijn vijanden wreke! Daarom proefde dat ganse volk geen spijs.
\par 25 En het ganse volk kwam in een woud; en daar was honig op het veld.
\par 26 Toen het volk in het woud kwam, ziet, zo was er een honigvloed; maar niemand raakte met zijn hand aan zijn mond, want het volk vreesde de bezwering.
\par 27 Maar Jonathan had het niet gehoord, toen zijn vader het volk bezworen had, en hij reikte het einde van den staf uit, die in zijn hand was, en hij doopte denzelven in een honigraat; als hij nu zijn hand tot zijn mond wendde, zo werden zijn ogen verlicht.
\par 28 Toen antwoordde een man uit het volk, en zeide: Uw vader heeft het volk zwaarlijk bezworen, zeggende: Vervloekt zij de man, die heden brood eet! Daarom bezwijkt het volk.
\par 29 Toen zeide Jonathan: Mijn vader heeft het land beroerd; zie toch, hoe mijn ogen verlicht zijn, omdat ik een weinig van dezen honig gesmaakt heb;
\par 30 Hoe veel meer, indien het volk heden had mogen vrijelijk eten van den buit zijner vijanden, dien het gevonden heeft! Maar nu is die slag niet groot geweest over de Filistijnen.
\par 31 Doch zij sloegen te dien dage de Filistijnen van Michmas tot Ajalon; en het volk was zeer moede.
\par 32 Toen maakte zich het volk aan den buit, en zij namen schapen, en runderen, en kalveren, en zij slachtten ze tegen de aarde; en het volk at ze met het bloed.
\par 33 En men boodschapte het Saul, zeggende: Zie, het volk verzondigt zich aan den HEERE, etende met het bloed. En hij zeide: Gij hebt trouwelooslijk gehandeld; wentelt heden een groten steen tot mij.
\par 34 Verder sprak Saul: Verstrooit u onder het volk, en zegt tot hen: Brengt tot mij een iegelijk zijn os, en een iegelijk zijn schaap, en slacht het hier, en eet, en bezondigt u niet aan den HEERE, die etende met het bloed. Toen bracht al het volk een iegelijk zijn os met zijn hand, des nachts, en slachtte ze aldaar.
\par 35 Toen bouwde Saul den HEERE een altaar; dit was het eerste altaar, dat hij den HEERE bouwde.
\par 36 Daarna zeide Saul: Laat ons aftrekken de Filistijnen na, bij nacht, en laat ons dezelve beroven, totdat het morgen licht worde, en laat ons niet een man onder hen overig laten. Zij nu zeiden: Doe al wat goed is in uw ogen; maar de priester zeide: Laat ons herwaarts tot God naderen.
\par 37 Toen vraagde Saul God: Zal ik aftrekken de Filistijnen na? Zult Gij ze in de hand van Israel overgeven? Doch Hij antwoordde hem niet te dien dage.
\par 38 Toen zeide Saul: Komt herwaarts uit alle hoeken des volks, en verneemt, en ziet, waarin deze zonde heden geschied zij.
\par 39 Want zo waarachtig als de HEERE leeft, Die Israel verlost, al ware het in mijn zoon Jonathan, zo zal hij den dood sterven; en niemand uit het ganse volk antwoordde hem.
\par 40 Verder zeide hij tot het ganse Israel: Gijlieden zult aan de ene zijde zijn, en ik en mijn zoon Jonathan zullen aan de andere zijde zijn. Toen zeide het volk tot Saul: Doe, wat goed is in uw ogen.
\par 41 Saul nu sprak tot den HEERE, den God Israels: Toon den onschuldige. Toen werd Jonathan en Saul geraakt, en het volk ging vrij uit.
\par 42 Toen zeide Saul: Werpt het lot tussen mij en tussen mijn zoon Jonathan. Toen werd Jonathan geraakt.
\par 43 Saul dan zeide tot Jonathan: Geef mij te kennen, wat gij gedaan hebt. Toen gaf het Jonathan hem te kennen, en zeide: Ik heb maar een weinig honigs geproefd, met het uiterste des stafs, dien ik in mijn hand had; zie hier ben ik, moet ik sterven?
\par 44 Toen zeide Saul: Zo doe mij God, en zo doe Hij daartoe, Jonathan! gij moet den dood sterven.
\par 45 Maar het volk zeide tot Saul: Zou Jonathan sterven, die deze grote verlossing in Israel gedaan heeft? Dat zij verre! zo waarachtig als de HEERE leeft, als er een haar van zijn hoofd op de aarde vallen zal; want hij heeft dit heden met God gedaan. Alzo verloste het volk Jonathan, dat hij niet stierf.
\par 46 Saul nu toog op van achter de Filistijnen, en de Filistijnen trokken aan hun plaats.
\par 47 Toen nam Saul het koninkrijk over Israel in; en hij streed rondom tegen al zijn vijanden, tegen Moab, en tegen de kinderen Ammons, en tegen Edom, en tegen de koningen van Zoba, en tegen de Filistijnen; en overal, waar hij zich wendde, oefende hij straf.
\par 48 En hij handelde dapper, en hij sloeg de Amalekieten, en hij redde Israel uit de hand desgenen, die hem beroofde.
\par 49 De zonen van Saul nu waren: Jonathan, en Isvi, en Malchi-sua; en de namen zijner twee dochteren waren deze: de naam der eerstgeborene was Merab, en de naam der kleinste Michal.
\par 50 En de naam van Sauls huisvrouw was Ahinoam, een dochter van Ahimaaz; en de naam van zijn krijgsoverste was Abi-ner, een zoon van Ner, Sauls oom.
\par 51 En Kis was Sauls vader, en Ner, Abners vader, was een zoon van Abiel.
\par 52 En er was een sterke krijg tegen de Filistijnen al de dagen van Saul; daarom alle helden en alle kloeke mannen, die Saul zag, die vergaderde hij tot zich.

\chapter{15}

\par 1 Toen zeide Samuel tot Saul: de HEERE heeft mij gezonden, dat ik u ten koning zalfde over Zijn volk, over Israel; hoor dan nu de stem van de woorden des HEEREN.
\par 2 Alzo zegt de HEERE der heirscharen: Ik heb bezocht, hetgeen Amalek aan Israel gedaan heeft, hoe hij zich tegen hem gesteld heeft op den weg, toen hij uit Egypte opkwam.
\par 3 Ga nu heen, en sla Amalek, en verban alles, wat hij heeft, en verschoon hem niet; maar dood van den man af tot de vrouw toe, van de kinderen tot de zuigelingen, van de ossen tot de schapen, van de kemelen tot de ezelen toe.
\par 4 Dit verkondigde Saul het volk, en hij telde hen te Telaim, tweehonderd duizend voetvolks, en tien duizend mannen van Juda.
\par 5 Als Saul tot aan de stad Amalek kwam, zo leide hij een achterlage in het dal.
\par 6 En Saul liet den Kenieten zeggen: Gaat weg, wijkt, trekt af uit het midden der Amalekieten, opdat ik u met hen niet wegruime; want gij hebt barmhartigheid gedaan aan al de kinderen Israels, toen zij uit Egypte opkwamen. Alzo weken de Kenieten uit het midden der Amalekieten.
\par 7 Toen sloeg Saul de Amalekieten van Havila af, tot daar gij komt te Sur, dat voor aan Egypte is.
\par 8 En hij ving Agag, den koning der Amalekieten, levend; maar al het volk verbande hij door de scherpte des zwaards.
\par 9 Doch Saul en het ganse volk verschoonde Agag, en de beste schapen, en runderen, en de naast beste, en de lammeren, en al wat best was, en zij wilden ze niet verbannen; maar alle ding, dat verachtzaam, en dat verdwijnende was, verbanden zij.
\par 10 Toen geschiedde het woord des HEEREN tot Samuel, zeggende:
\par 11 Het berouwt Mij, dat Ik Saul tot koning gemaakt heb, dewijl hij zich van achter Mij afgekeerd heeft, en Mijn woorden niet bevestigd heeft. Toen ontstak Samuel, en hij riep tot den HEERE den gansen nacht.
\par 12 Daarna maakte zich Samuel des morgens vroeg op, Saul tegemoet; en het werd Samuel geboodschapt, zeggende: Saul is te Karmel gekomen, en zie, hij heeft zich een pilaar gesteld; daarna is hij omgetogen, en doorgetrokken, en naar Gilgal afgekomen.
\par 13 Samuel nu kwam tot Saul, en Saul zeide tot hem: Gezegend zijt gij den HEERE! Ik heb des HEEREN woord bevestigd.
\par 14 Toen zeide Samuel: Wat is dan dit voor een stem der schapen in mijn oren, en een stem der runderen, die ik hoor?
\par 15 Saul nu zeide: Zij hebben ze van de Amalekieten gebracht, want het volk heeft de beste schapen en runderen verschoond, om den HEERE, uw God, te offeren; maar het overige hebben wij verbannen.
\par 16 Toen zeide Samuel tot Saul: Houd op, zo zal ik u te kennen geven, wat de HEERE vannacht tot mij gesproken heeft. Hij dan zeide tot hem: Spreek.
\par 17 En Samuel zeide: Is het niet alzo, toen ge klein waart in uw ogen, dat gij het hoofd der stammen van Israel geworden zijt, en dat u de HEERE tot koning over Israel gezalfd heeft?
\par 18 En de HEERE heeft u op den weg gezonden, en gezegd: Ga heen en verban de zondaars, de Amalekieten, en strijd tegen hen, totdat gij dezelve te niet doet.
\par 19 Waarom toch hebt gij naar de stem des HEEREN niet gehoord, maar zijt tot den roof gevlogen, en hebt gedaan dat kwaad was in de ogen des HEEREN?
\par 20 Toen zeide Saul tot Samuel: Ik heb immers naar de stem des HEEREN gehoord, en heb gewandeld op den weg, op denwelken mij de HEERE gezonden heeft; en ik heb Agag, den koning der Amalekieten, mede gebracht, maar de Amalekieten heb ik verbannen.
\par 21 Het volk nu heeft genomen van den roof, schapen en runderen, het voornaamste van het verbannene, om den HEERE, uw God, op te offeren te Gilgal.
\par 22 Doch Samuel zeide: Heeft de HEERE lust aan brandofferen, en slachtofferen, als aan het gehoorzamen van de stem des HEEREN? Zie, gehoorzamen is beter dan slachtoffer, opmerken dan het vette der rammen.
\par 23 Want wederspannigheid is een zonde der toverij, en wederstreven is afgoderij en beeldendienst. Omdat gij des HEEREN woord verworpen hebt, zo heeft Hij u verworpen, dat gij geen koning zult zijn.
\par 24 Toen zeide Saul tot Samuel: Ik heb gezondigd, omdat ik des HEEREN bevel en uw woorden overtreden heb; want ik heb het volk gevreesd en naar hun stem gehoord.
\par 25 Nu dan, vergeef mij toch mijn zonde, en keer met mij wederom, dat ik den HEERE aanbidde.
\par 26 Doch Samuel zeide tot Saul: Ik zal met u niet wederkeren; omdat gij het woord des HEEREN verworpen hebt, zo heeft u de HEERE verworpen, dat gij geen koning over Israel zult zijn.
\par 27 Als zich Samuel omkeerde om weg te gaan, zo greep hij een slip van zijn mantel en zij scheurde.
\par 28 Toen zeide Samuel tot hem: De HEERE heeft heden het koninkrijk van Israel van u afgescheurd, en heeft het aan uw naaste gegeven, die beter is dan gij.
\par 29 En ook liegt Hij, Die de Overwinning van Israel is, niet, en het berouwt Hem niet; want Hij is geen mens, dat Hem iets berouwen zou.
\par 30 Hij dan zeide: Ik heb gezondigd; eer mij toch nu voor de oudsten mijns volks, en voor Israel; en keer wederom met mij, dat ik den HEERE, uw God, aanbidde.
\par 31 Toen keerde Samuel wederom Saul na; en Saul aanbad den HEERE.
\par 32 Toen zeide Samuel: Breng Agag, den koning der Amalekieten, hier tot mij; Agag nu ging tot hem weeldelijk; en Agag zeide: Voorwaar, de bitterheid des doods is geweken!
\par 33 Maar Samuel zeide: Gelijk als uw zwaard de vrouwen van haar kinderen beroofd heeft, alzo zal uw moeder van haar kinderen beroofd worden onder de vrouwen. Toen hieuw Samuel Agag in stukken, voor het aangezicht des HEEREN te Gilgal.
\par 34 Daarna ging Samuel naar Rama; en Saul ging op naar zijn huis te Gibea-sauls.
\par 35 En Samuel zag Saul niet meer tot den dag zijns doods toe; evenwel droeg Samuel leed om Saul; en het berouwde den HEERE, dat Hij Saul tot koning over Israel gemaakt had.

\chapter{16}

\par 1 Toen zeide de HEERE tot Samuel: Hoe lang draagt gij leed om Saul, dien Ik toch verworpen heb, dat hij geen koning zij over Israel? Vul uw hoorn met olie, en ga heen; Ik zal u zenden tot Isai, den Bethlehemiet; want Ik heb Mij een koning onder zijn zonen uitgezien.
\par 2 Maar Samuel zeide: Hoe zou ik heengaan? Saul zal het toch horen en mij doden. Toen zeide de HEERE: Neem een kalf van de runderen met u, en zeg: Ik ben gekomen, om den HEERE offerande te doen.
\par 3 En gij zult Isai ten offer nodigen, en Ik zal u te kennen geven, wat gij doen zult, en gij zult Mij zalven, dien Ik u zeggen zal.
\par 4 Samuel nu deed, hetgeen de HEERE gesproken had, en hij kwam te Bethlehem. Toen kwamen de oudsten der stad bevende hem tegemoet, en zeiden: Is uw komst met vrede?
\par 5 Hij dan zeide: Met vrede; ik ben gekomen om den HEERE offerande te doen; heiligt u, en komt met mij ten offer; en hij heiligde Isai en zijn zonen, en hij nodigde hen ten offer.
\par 6 En het geschiedde, toen zij inkwamen, zo zag hij Eliab aan, en dacht: Zekerlijk, is deze voor den HEERE, Zijn gezalfde.
\par 7 Doch de HEERE zeide tot Samuel: Zie zijn gestalte niet aan, noch de hoogte zijner statuur, want Ik heb hem verworpen; want het is niet gelijk de mens ziet; want de mens ziet aan, wat voor ogen is, maar de HEERE ziet het hart aan.
\par 8 Toen riep Isai Abinadab, en hij deed hem voorbij het aangezicht van Samuel gaan; doch hij zeide: Dezen heeft de HEERE ook niet verkoren.
\par 9 Daarna liet Isai Samma voorbijgaan; doch hij zeide: Dezen heeft de HEERE ook niet verkoren.
\par 10 Alzo liet Isai zijn zeven zonen voorbij het aangezicht van Samuel gaan; doch Samuel zeide tot Isai: De HEERE heeft dezen niet verkoren.
\par 11 Voorts zeide Samuel tot Isai: Zijn dit al de jongelingen? En hij zeide: De kleinste is nog overig, en zie, hij weidt de schapen. Samuel nu zeide tot Isai: Zend heen en laat hem halen; want wij zullen niet rondom aanzitten, totdat hij hier zal gekomen zijn.
\par 12 Toen zond hij heen, en bracht hem in; hij nu was roodachtig, mitsgaders schoon van ogen en schoon van aanzien; en de HEERE zeide: Sta op, zalf hem, want deze is het.
\par 13 Toen nam Samuel den oliehoorn, en hij zalfde hem in het midden zijner broederen. En de Geest des HEEREN werd vaardig over David van dien dag af en voortaan. Daarna stond Samuel op, en hij ging naar Rama.
\par 14 En de Geest des HEEREN week van Saul; en een boze geest van den HEERE verschrikte hem.
\par 15 Toen zeiden Sauls knechten tot hem: Zie toch, een boze geest Gods verschrikt u.
\par 16 Onze heer zegge toch tot uw knechten, die voor uw aangezicht staan, dat zij een man zoeken, die op de harp spelen kan; en het zal geschieden, als de boze geest Gods op u is, dat hij met zijn hand spele, dat het beter met u worde.
\par 17 Toen zeide Saul tot zijn knechten: Ziet mij toch naar een man uit, die wel spelen kan, en brengt hem tot mij.
\par 18 Toen antwoordde een van de jongelingen, en zeide: Zie, ik heb gezien een zoon van Isai, den Bethlehemiet, die spelen kan en hij is een dapper held, en een krijgsman, en verstandig in zaken, en een schoon man, en de HEERE is met hem.
\par 19 Saul nu zond boden tot Isai, en zeide: Zend uw zoon David tot mij, die bij de schapen is.
\par 20 Toen nam Isai een ezel met brood, en een lederen zak met wijn, en een geitenbokje; en hij zond ze door de hand van zijn zoon David aan Saul.
\par 21 Alzo kwam David tot Saul, en hij stond voor zijn aangezicht; en hij beminde hem zeer, en hij werd zijn wapendrager.
\par 22 Daarna zond Saul tot Isai, om te zeggen: Laat toch David voor mijn aangezicht staan, want hij heeft genade in mijn ogen gevonden.
\par 23 En het geschiedde, als de geest Gods over Saul was, zo nam David de harp, en hij speelde met zijn hand; dat was voor Saul een verademing, en het werd beter met hem, en de boze geest week van hem.

\chapter{17}

\par 1 En de Filistijnen verzamelden hun heir ten strijde, en verzamelden zich te Socho, dat in Juda is; en zij legerden zich tussen Socho en tussen Azeka, aan het einde van Dammim.
\par 2 Doch Saul en de mannen van Israel verzamelden zich, en legerden zich in het eikendal; en stelden de slagorde tegen de Filistijnen aan.
\par 3 De Filistijnen nu stonden aan een berg aan gene, en de Israelieten stonden aan een berg aan deze zijde; en de vallei was tussen hen.
\par 4 Toen ging er een kampvechter uit, uit het leger der Filistijnen; zijn naam was Goliath, van Gath; zijn hoogte was zes ellen en een span.
\par 5 En hij had een koperen helm op zijn hoofd, en hij had een schubachtig pantsier aan; en het gewicht van het pantsier was vijf duizend sikkelen kopers;
\par 6 En een koperen scheenharnas boven zijn voeten, en een koperen schild tussen zijn schouders;
\par 7 En de schacht zijner spies was als een weversboom, en het lemmer zijner spies was van zeshonderd sikkelen ijzers; en de schilddrager ging voor zijn aangezicht.
\par 8 Deze nu stond, en riep tot de slagorden van Israel, en zeide tot hen: Waarom zoudt gijlieden uittrekken, om de slagorde te stellen? Ben ik niet een Filistijn, en gijlieden knechten van Saul? Kiest een man onder u, die tot mij afkome.
\par 9 Indien hij tegen mij strijden en mij verslaan kan, zo zullen wij ulieden tot knechten zijn; maar indien ik hem overwin en hem sla, zo zult gij ons tot knechten zijn, en ons dienen.
\par 10 Verder zeide de Filistijn: Ik heb heden de slagorden van Israel gehoond, zeggende: Geeft mij een man, dat wij te zamen strijden!
\par 11 Toen Saul en het ganse Israel deze woorden van den Filistijn hoorden, zo ontzetten zij zich, en vreesden zeer.
\par 12 David nu was de zoon van den Efrathischen man van Bethlehem-juda, wiens naam was Isai, en die acht zonen had, en in de dagen van Saul was hij een man, oud, afgaande onder de mannen.
\par 13 En de drie grootste zonen van Isai gingen heen; zij volgden Saul na in den krijg. De namen nu zijner drie zonen, die in den krijg gingen, waren: Eliab, de eerstgeborene, en zijn tweede Abinadab, en de derde Samma.
\par 14 En David was de kleinste; en de drie grootsten waren Saul nagevolgd.
\par 15 Doch David ging henen, en kwam weder van Saul, om zijns vaders schapen te weiden te Bethlehem.
\par 16 De Filistijn nu trad toe, des morgens vroeg en des avonds. Alzo stelde hij zich daar veertig dagen lang.
\par 17 En Isai zeide tot zijn zoon David: Neem toch voor uw broeders een efa van dit geroost koren, en deze tien broden, en breng ze terloops in het leger tot uw broederen.
\par 18 Maar breng deze tien melkkazen aan de oversten over duizend; en gij zult uw broederen bezoeken, of het hun welga, en gij zult van hen pand medenemen.
\par 19 Saul nu, en zij, en alle mannen van Israel waren bij het eikendal met de Filistijnen strijdende.
\par 20 Toen maakte zich David des morgens vroeg op, en hij liet de schapen bij den hoeder, en hij nam het op, en ging henen, gelijk als Isai hem bevolen had; en hij kwam aan den wagenburg, als het heir in slagorde uittoog, en men ten strijde riep.
\par 21 En de Israelieten en Filistijnen stelden slagorde tegen slagorde.
\par 22 David nu liet de vaten van zich, onder de hand van den bewaarder der vaten, en hij liep ter slagorde; en hij kwam en vraagde zijn broederen naar hun welstand.
\par 23 Toen hij met hen sprak, ziet, zo kwam de kampvechter op; zijn naam was Goliath, de Filistijn van Gath, uit het heir der Filistijnen, en hij sprak achtereenvolgens die woorden; en David hoorde ze.
\par 24 Doch alle mannen in Israel, als zij dien man zagen, zo vluchtten zij voor zijn aangezicht, en zij vreesden zeer.
\par 25 En de mannen Israels zeiden: Hebt gijlieden dien man wel gezien, die opgekomen is? Want hij is opgekomen, om Israel te honen; en het zal geschieden, dat de koning dien man, die hem slaat, met groten rijkdom verrijken zal, en hij zal hem zijn dochter geven, en hij zal zijns vaders huis vrijmaken in Israel.
\par 26 Toen zeide David tot de mannen, die bij hem stonden, zeggende: Wat zal men dien man doen, die dezen Filistijn slaat, en den smaad van Israel wendt? Want wie is deze onbesneden Filistijn, dat hij de slagorden van den levenden God zou honen?
\par 27 Wederom zeide hem het volk achtervolgens dat woord, zeggende: Alzo zal men den man doen, die hem slaat.
\par 28 Als Eliab, zijn grootste broeder, hem tot die mannen hoorde spreken, zo ontstak de toorn van Eliab tegen David, en hij zeide: Waarom zijt gij nu afgekomen, en onder wien hebt gij de weinige schapen in de woestijn gelaten? Ik ken uw vermetelheid, en de boosheid uws harten wel; want gij zijt afgekomen, opdat gij den strijd zaagt.
\par 29 Toen zeide David: Wat heb ik nu gedaan? Is er geen oorzaak?
\par 30 En hij wendde zich af van dien naar een anderen toe, en hij zeide achtereenvolgens dat woord; en het volk gaf hem weder antwoord, achtervolgens de eerste woorden.
\par 31 Toen die woorden gehoord werden, die David gesproken had, en in de tegenwoordigheid van Saul verkondigd werden, zo liet hij hem halen.
\par 32 En David zeide tot Saul: Aan geen mens ontvalle het hart, om zijnentwil. Uw knecht zal heengaan en hij zal met dezen Filistijn strijden.
\par 33 Maar Saul zeide tot David: Gij zult niet kunnen heengaan tot dezen Filistijn, om met hem te strijden; want gij zijt een jongeling, en hij is een krijgsman van zijn jeugd af.
\par 34 Toen zeide David tot Saul: Uw knecht weidde de schapen zijns vaders, en er kwam een leeuw en een beer, en nam een schaap van de kudde weg.
\par 35 En ik ging uit hem na, en ik sloeg hem, en redde het uit zijn mond; toen hij tegen mij opstond, zo vatte ik hem bij zijn baard, en sloeg hem, en doodde hem.
\par 36 Uw knecht heeft zo den leeuw als den beer geslagen; alzo zal deze onbesneden Filistijn zijn, gelijk een van die, omdat hij de slagorden van den levenden God gehoond heeft.
\par 37 Verder zeide David: De HEERE, Die mij van de hand des leeuws gered heeft, en uit de hand des beers, Die zal mij redden uit de hand van dezen Filistijn. Toen zeide Saul tot David: Ga heen, en de HEERE zij met u!
\par 38 En Saul kleedde David met zijn klederen, en zette een koperen helm op zijn hoofd, en kleedde hem met een pantsier.
\par 39 En David gordde zijn zwaard aan over zijn klederen, en wilde gaan; want hij had het nooit verzocht. Toen zeide David tot Saul: Ik kan in deze niet gaan, want ik heb het nooit verzocht; en David leide ze van zich.
\par 40 En hij nam zijn staf in zijn hand, en hij koos zich vijf gladde stenen uit de beek, en leide ze in de herderstas, die hij had, te weten in den zak, en zijn slinger was in zijn hand; alzo naderde hij tot den Filistijn.
\par 41 De Filistijn ging ook heen, gaande en naderende tot David, en zijn schilddrager ging voor zijn aangezicht.
\par 42 Toen de Filistijn opzag, en David zag, zo verachtte hij hem; want hij was een jongeling, roodachtig, mitsgaders schoon van aanzien.
\par 43 De Filistijn nu zeide tot David: Ben ik een hond, dat gij tot mij komt met stokken? En de Filistijn vloekte David bij zijn goden.
\par 44 Daarna zeide de Filistijn tot David: Kom tot mij, zo zal ik uw vlees aan de vogelen des hemels geven, en aan de dieren des velds.
\par 45 David daarentegen zeide tot den Filistijn: Gij komt tot mij met een zwaard, en met een spies, en met een schild; maar ik kom tot u in den Naam van den HEERE der heirscharen, den God der slagorden van Israel, Dien gij gehoond hebt.
\par 46 Te dezen dage zal de HEERE u besluiten in mijn hand, en ik zal u slaan, en ik zal uw hoofd van u wegnemen, en ik zal de dode lichamen van der Filistijnen leger dezen dag aan de vogelen des hemels, en aan de beesten des velds geven; en de ganse aarde zal weten, dat Israel een God heeft.
\par 47 En deze ganse vergadering zal weten, dat de HEERE niet door het zwaard, noch door de spies verlost; want de krijg is des HEEREN, Die zal ulieden in onze hand geven.
\par 48 En het geschiedde, toen de Filistijn zich opmaakte, en heenging, en David tegemoet naderde, zo haastte David, en liep naar de slagorde toe, den Filistijn tegemoet.
\par 49 En David stak zijn hand in de tas, en hij nam een steen daaruit, en hij slingerde, en trof den Filistijn in zijn voorhoofd; zodat de steen zonk in zijn voorhoofd, en hij viel op zijn aangezicht ter aarde.
\par 50 Alzo overweldigde David den Filistijn met een slinger en met een steen; en hij versloeg den Filistijn, en doodde hem; doch David had geen zwaard in de hand.
\par 51 Daarom liep David, en stond op den Filistijn, en nam zijn zwaard, en hij trok het uit zijn schede, en hij doodde hem, en hij hieuw hem het hoofd daarmede af. Toen de Filistijnen zagen, dat hun geweldigste dood was, zo vluchtten zij.
\par 52 Toen maakten zich de mannen van Israel en van Juda op, en juichten, en vervolgden de Filistijnen, tot daar men komt aan de vallei, en tot aan de poorten van Ekron; en de verwonden der Filistijnen vielen op den weg van Saaraim, en tot aan Gath, en tot aan Ekron.
\par 53 Daarna keerden de kinderen Israels om, van het hittig najagen der Filistijnen, en zij beroofden hun legers.
\par 54 Daarna nam David het hoofd van den Filistijn, en bracht het naar Jeruzalem; maar zijn wapenen leide hij in zijn tent.
\par 55 Toen Saul David zag uitgaan den Filistijn tegemoet, zeide hij tot Abner, den krijgsoverste: Wiens zoon is deze jongeling, Abner? En Abner zeide: Zo waarachtig als uw ziel leeft, o koning! ik weet het niet.
\par 56 De koning nu zeide: Vraag gij het, wiens zoon deze jongeling is.
\par 57 Als David wederkeerde van het slaan des Filistijns, zo nam hem Abner, en hij bracht hem voor het aangezicht van Saul, en het hoofd van den Filistijn was in zijn hand.
\par 58 En Saul zeide tot hem: Wiens zoon zijt gij, jongeling? En David zeide: Ik ben een zoon van uw knecht Isai, den Bethlehemiet.

\chapter{18}

\par 1 Het geschiedde nu, als hij geeindigd had tot Saul te spreken, dat de ziel van Jonathan verbonden werd aan de ziel van David; en Jonathan beminde hem als zijn ziel.
\par 2 En Saul nam hem te dien dage, en liet hem niet werderkeren tot zijns vaders huis.
\par 3 Jonathan nu en David maakten een verbond, dewijl hij hem liefhad als zijn ziel.
\par 4 En Jonathan deed zijn mantel af, dien hij aan had, en gaf hem aan David, ook zijn klederen, ja, tot zijn zwaard toe, en tot zijn boog toe, en tot zijn gordel toe.
\par 5 En David toog uit, overal, waar Saul hem zond; hij gedroeg zich voorzichtiglijk, en Saul zette hem over de krijgslieden; en hij was aangenaam in de ogen des gansen volks, en ook in de ogen der knechten van Saul.
\par 6 Het geschiedde nu, toen zij kwamen, en David wederkeerde van het slaan der Filistijnen, dat de vrouwen uitgingen uit al de steden van Israel, met gezang en reien, den koning Saul tegemoet, met trommelen, met vreugde en met muziekinstrumenten.
\par 7 En de vrouwen, spelende, antwoordden elkander en zeiden: Saul heeft zijn duizenden verslagen, maar David zijn tienduizenden!
\par 8 Toen ontstak Saul zeer, en dat woord was kwaad in zijn ogen, en hij zeide: Zij hebben David tien duizend gegeven, doch mij hebben zij maar duizend gegeven; en voorzeker zal het koninkrijk nog voor hem zijn.
\par 9 En Saul had het oog op David, van dien dag af en voortaan.
\par 10 En het geschiedde des anderen daags, dat de boze geest Gods over Saul vaardig werd, en hij profeteerde midden in het huis, en David speelde op snarenspel met zijn hand, als van dag tot dag; Saul nu had een spies in zijn hand.
\par 11 En Saul schoot de spies, en zeide: Ik zal David aan den wand spitten; maar David wendde zich tweemaal van zijn aangezicht af.
\par 12 En Saul vreesde voor David, want de HEERE was met hem, en Hij was van Saul geweken.
\par 13 Daarom deed hem Saul van zich weg, en hij zette hem zich tot een overste van duizend; en hij ging uit en hij ging in voor het aangezicht des volks.
\par 14 En David gedroeg zich voorzichtiglijk op al zijn wegen; en de HEERE was met hem.
\par 15 Toen nu Saul zag, dat hij zich zeer voorzichtiglijk gedroeg, vreesde hij voor zijn aangezicht.
\par 16 Doch gans Israel en Juda had David lief; want hij ging uit en hij ging in voor hun aangezicht.
\par 17 Derhalve zeide Saul tot David: Zie, mijn grootste dochter Merab zal ik u tot een vrouw geven; alleenlijk, wees mij een dapper zoon, en voer den krijg des HEEREN. Want Saul zeide: Dat mijn hand niet tegen hem zij, maar dat de hand der Filistijnen tegen hem zij.
\par 18 Doch David zeide tot Saul: Wie ben ik, en wat is mijn leven, en mijns vaders huisgezin in Israel, dat ik des konings schoonzoon zou worden?
\par 19 Het geschiedde nu ten tijde als men Merab, de dochter van Saul, aan David geven zou, zo is zij aan Adriel, den Meholathiet, ter vrouw gegeven.
\par 20 Doch Michal, de dochter van Saul, had David lief. Toen dat Saul te kennen werd gegeven, zo was die zaak recht in zijn ogen.
\par 21 En Saul zeide: Ik zal haar hem geven, dat zij hem tot een valstrik zij, en dat de hand der Filistijnen tegen hem zij. Daarom zeide Saul tot David: Met de andere zult gij heden mijn schoonzoon worden.
\par 22 En Saul gebood zijn knechten: Spreekt met David in het heimelijke, zeggende: Zie, de koning heeft lust aan u, en al zijn knechten hebben u lief; word dan nu des konings schoonzoon.
\par 23 En de knechten van Saul spraken deze woorden voor de oren van David. Toen zeide David: Is dat licht in ulieder ogen, des konings schoonzoon te worden, daar ik een arm en verachtzaam man ben?
\par 24 En de knechten van Saul boodschapten het hem, zeggende: Zulke woorden heeft David gesproken.
\par 25 Toen zeide Saul: Aldus zult gijlieden tot David zeggen: De koning heeft geen lust aan den bruidschat, maar aan honderd voorhuiden der Filistijnen, opdat men zich wreke aan des konings vijanden. Want Saul dacht David te vellen door de hand der Filistijnen.
\par 26 Zijn knechten nu boodschapten David deze woorden. En die zaak was recht in de ogen van David, dat hij des konings schoonzoon zou worden; maar de dagen waren nog niet vervuld.
\par 27 Toen maakte zich David op, en hij en zijn mannen gingen heen, en zij sloegen onder de Filistijnen tweehonderd mannen, en David bracht hun voorhuiden, en men leverde ze den koning volkomenlijk, opdat hij schoonzoon des konings worden zou. Toen gaf Saul hem zijn dochter Michal ter vrouw.
\par 28 En Saul zag en merkte, dat de HEERE met David was; en Michal, de dochter van Saul, had hem lief.
\par 29 Toen vreesde zich Saul nog meer voor David; en Saul was David een vijand al zijn dagen.
\par 30 Als de vorsten der Filistijnen uittogen, zo geschiedde het, als zij uittogen, dat David kloeker was, dan al de knechten van Saul; zodat zijn naam zeer geacht was.

\chapter{19}

\par 1 Derhalve sprak Saul tot zijn zoon Jonathan en tot al zijn knechten, om David te doden. Doch Jonathan, Sauls zoon, had groot welgevallen aan David.
\par 2 En Jonathan verkondigde het David, zeggende: Mijn vader Saul zoekt u te doden; nu dan, wacht u toch des morgens, en blijf in het verborgene, en versteek u.
\par 3 Doch ik zal uitgaan, en aan de hand mijns vaders staan op het veld, waar gij zult zijn; en ik zal van u tot mijn vader spreken, en zal zien wat het zij; dat zal ik u verkondigen.
\par 4 Zo sprak dan Jonathan goed van David tot zijn vader Saul; en hij zeide tot hem: De koning zondige niet tegen zijn knecht David, omdat hij tegen u niet gezondigd heeft, en omdat zijn daden voor u zeer goed zijn.
\par 5 Want hij heeft zijn ziel in zijn hand gezet, en hij heeft den Filistijn geslagen, en de HEERE heeft een groot heil aan het ganse Israel gedaan; gij hebt het gezien, en gij zijt verblijd geweest; waarom zoudt gij dan tegen onschuldig bloed zondigen, David zonder oorzaak dodende?
\par 6 Saul nu hoorde naar de stem van Jonathan; en Saul zwoer: zo waarachtig als de HEERE leeft, hij zal niet gedood worden!
\par 7 En Jonathan riep David, en Jonathan gaf hem al deze woorden te kennen; en Jonathan bracht David tot Saul, en hij was voor zijn aangezicht als gisteren en eergisteren.
\par 8 En er werd wederom krijg; en David toog uit, en streed tegen de Filistijnen, en hij sloeg hen met een groten slag, en zij vloden voor zijn aangezicht.
\par 9 Doch de boze geest des HEEREN was over Saul, en hij zat in zijn huis, en zijn spies was in zijn hand; en David speelde op snarenspel met de hand;
\par 10 Saul nu zocht met de spies David aan den wand te spitten, doch hij ontweek van het aangezicht van Saul, die met de spies in den wand sloeg. Toen vlood David, en ontkwam in dienzelfden nacht.
\par 11 Maar Saul zond boden heen tot Davids huis, dat zij hem bewaarden, en dat zij hem des morgens doodden. Dit gaf Michal, zijn huisvrouw, David te kennen, zeggende: Indien gij uw ziel dezen nacht niet behoedt, zo zult gij morgen gedood worden.
\par 12 En Michal liet David door een venster neder, en hij ging heen, en vluchtte, en ontkwam.
\par 13 En Michal nam een beeld, en zij leide het in het bed, en zij leide een geitenvel aan zijn hoofdpeluw, en dekte het met een kleed toe.
\par 14 Saul nu zond boden, om David te halen. Zij dan zeide: Hij is ziek.
\par 15 Toen zond Saul boden, om David te bezien, zeggende: Breng hem op het bed tot mij op, dat men hem dode.
\par 16 Als de boden kwamen, zo ziet, er was een beeld in het bed, en er was een geitenvel aan zijn hoofdpeluw.
\par 17 Toen zeide Saul tot Michal: Waarom hebt gij mij alzo bedrogen en hebt mijn vijand laten gaan, dat hij ontkomen is? Michal nu zeide tot Saul: Hij zeide tot mij: Laat mij gaan, waarom zou ik u doden?
\par 18 Alzo vluchtte David en ontkwam, en hij kwam tot Samuel te Rama, en hij gaf hem te kennen al wat Saul hem gedaan had; en hij en Samuel gingen heen, en zij bleven te Najoth.
\par 19 En men boodschapte Saul, zeggende: Zie, David is te Najoth, bij Rama.
\par 20 Toen zond Saul boden heen, om David te halen; die zagen een vergadering van profeten, profeterende, en Samuel staande, over hen gesteld; en de Geest Gods was over Sauls boden, en die profeteerden ook.
\par 21 Toen men het Saul boodschapte, zo zond hij andere boden, en die profeteerden ook; toen voer Saul voort en zond de derde boden, en die profeteerden ook.
\par 22 Daarna ging hij ook zelf naar Rama, en hij kwam tot den groten waterput, die te Sechu was, en hij vraagde en zeide: Waar is Samuel, en David? Toen werd hem gezegd: Zie, zij zijn te Najoth bij Rama.
\par 23 Toen ging hij derwaarts naar Najoth bij Rama; en dezelfde Geest Gods was ook op hem, en hij, al voortgaande, profeteerde, totdat hij te Najoth in Rama kwam.
\par 24 En hij toog zelf ook zijn klederen uit, en hij profeteerde zelf ook, voor het aangezicht van Samuel; en hij viel bloot neder dienzelfden gansen dag, en den gansen nacht. Daarom zegt men: Is Saul ook onder de profeten?

\chapter{20}

\par 1 Toen vluchtte David van Najoth bij Rama, en hij kwam, en zeide voor het aangezicht van Jonathan: Wat heb ik gedaan, wat is mijn misdaad, en wat is mijn zonde voor het aangezicht uws vaders, dat hij mijn ziel zoekt?
\par 2 Hij daarentegen zeide tot hem: Dat zij verre, gij zult niet sterven. Zie, mijn vader doet geen grote zaak, en geen kleine zaak, die hij voor mijn oor niet openbaart; waarom zou dan mijn vader deze zaak van mij verbergen? Dat is niet.
\par 3 Toen zwoer David verder, en zeide: Uw vader weet zeer wel, dat ik genade in uw ogen gevonden heb; daarom heeft hij gezegd: Dat Jonathan dit niet wete, opdat hij zich niet bekommere; en zekerlijk, zo waarachtig als de HEERE leeft, en uw ziel leeft, er is maar als een schrede tussen mij en tussen den dood!
\par 4 Jonathan nu zeide tot David: Wat uw ziel zegt, dat zal ik u doen.
\par 5 En David zeide tot Jonathan: Zie, morgen is de nieuwe maan, dat ik zekerlijk met den koning zou aanzitten om te eten; zo laat mij gaan, dat ik mij op het veld verberge tot aan den derden avond.
\par 6 Indien uw vader mij gewisselijk mist, zo zult gij zeggen: David heeft van mij zeer begeerd, dat hij tot zijn stad Bethlehem mocht lopen; want aldaar is een jaarlijks offer voor het ganse geslacht.
\par 7 Indien hij aldus zegt: Het is goed, zo heeft uw knecht vrede; maar indien hij gans ontstoken is, zo weet, dat het kwaad bij hem ten volle besloten is.
\par 8 Doe dan barmhartigheid aan uw knecht, want gij hebt uw knecht in een verbond des HEEREN met u gebracht; maar is er een misdaad in mij, zo dood gij mij; waarom zoudt gij mij toch tot uw vader brengen?
\par 9 Toen zeide Jonathan: Dat zij verre van u! Maar indien ik zekerlijk merkte, dat dit kwaad bij mijn vader ten volle besloten ware, dat het u zou overkomen, zou ik dat u dan niet te kennen geven?
\par 10 David nu zeide tot Jonathan: Wie zal het mij te kennen geven, indien uw vader u wat hards antwoordt?
\par 11 Toen zeide Jonathan tot David: Kom, laat ons toch uitgaan in het veld; en die beiden gingen uit in het veld.
\par 12 En Jonathan zeide tot David: De HEERE, de God Israels, indien ik mijn vader onderzocht zal hebben omtrent dezen tijd, morgen of overmorgen, en zie, het is goed voor David, en ik dan tot u niet zende, en voor uw oor openbare;
\par 13 Alzo doe de HEERE aan Jonathan, en alzo doe Hij daartoe! Als mijn vader het kwaad over u behaagt, zo zal ik het voor uw oor ontdekken, en ik zal u trekken laten, dat gij in vrede heengaat; en de HEERE zij met u, gelijk als Hij met mijn vader geweest is.
\par 14 En zult gij niet, indien ik dan nog leve, ja, zult gij niet de weldadigheid des HEEREN aan mij doen, dat ik niet sterve?
\par 15 Ook zult gij uw weldadigheid niet afsnijden van mijn huis tot in eeuwigheid; ook niet wanneer de HEERE een iegelijk der vijanden van David van den aardbodem zal afgesneden hebben.
\par 16 Alzo maakte Jonathan een verbond met het huis van David, zeggende: Dat het de HEERE eise van de hand der vijanden Davids!
\par 17 En Jonathan voer voort, met David te doen zweren, omdat hij hem liefhad; want hij had hem lief met de liefde zijner ziel.
\par 18 Daarna zeide Jonathan tot hem: Morgen is de nieuwe maan; dan zal men u missen, want uw zitplaats zal ledig gevonden worden.
\par 19 En als gij de drie dagen zult uitgebleven zijn, kom haastig af, en ga tot die plaats, waar gij u verborgen hadt ten dage dezer handeling; en blijf bij den steen Ezel.
\par 20 Zo zal ik drie pijlen ter zijde schieten, als of ik naar een teken schoot.
\par 21 En zie, ik zal den jongen zenden, zeggende: Ga heen, zoek de pijlen, indien ik uitdrukkelijk tot den jongen zeg: Zie, de pijlen zijn van u af en herwaarts, neem hem; en kom gij, want er is vrede voor u, en er is geen ding, zo waarlijk de HEERE leeft!
\par 22 Maar indien ik tot den jongen alzo zeg: Zie, de pijlen zijn van u af en verder; ga heen, want de HEERE heeft u laten gaan.
\par 23 En aangaande de zaak, waarvan ik en gij gesproken hebben, zie, de HEERE zij tussen mij en tussen u, tot in eeuwigheid!
\par 24 David nu verborg zich in het veld; en als het nieuwe maan was, zat de koning bij de spijze, om te eten.
\par 25 Toen zich de koning gezet had op zijn zitplaats, op dit maal gelijk de andere maal, aan de stede bij den wand, zo stond Jonathan op, en Abner zat aan Sauls zijde, en Davids plaats werd ledig gevonden.
\par 26 En Saul sprak te dien dage niets, want hij zeide: Hem is wat voorgevallen, dat hij niet rein is; voorzeker, hij is niet rein.
\par 27 Het geschiedde nu des anderen daags, den tweeden der nieuwe maan, als Davids plaats ledig gevonden werd, zo zeide Saul tot zijn zoon Jonathan: Waarom is de zoon van Isai noch gisteren noch heden tot de spijze gekomen?
\par 28 En Jonathan antwoordde Saul: David begeerde van mij ernstelijk naar Bethlehem te mogen gaan.
\par 29 En hij zeide: Laat mij toch gaan; want ons geslacht heeft een offer in de stad, en mijn broeder heeft het mij zelfs geboden; heb ik nu genade in uw ogen gevonden, laat mij toch ontslagen zijn, dat ik mijn broeders zie; hierom is hij aan des konings tafel niet gekomen.
\par 30 Toen ontstak de toorn van Saul tegen Jonathan, en hij zeide tot hem: Gij, zoon der verkeerde in wederspannigheid, weet ik het niet, dat gij den zoon van Isai verkoren hebt tot uw schande, en tot schande van de naaktheid uwer moeder?
\par 31 Want al de dagen, die de zoon van Isai op den aardbodem leven zal, zo zult gij noch uw koninkrijk bevestigd worden; nu dan, schik heen, en haal hem tot mij, want hij is een kind des doods.
\par 32 Toen antwoordde Jonathan Saul, zijn vader, en zeide tot hem: Waarom zal hij gedood worden? Wat heeft hij gedaan?
\par 33 Toen schoot Saul de spies op hem, om hem te slaan. Alzo merkte Jonathan, dat dit ten volle bij zijn vader besloten was, David te doden.
\par 34 Daarom stond Jonathan van de tafel op in hittigheid des toorns; en hij at op den tweeden dag der nieuwe maan geen brood, want hij was bekommerd om David, omdat zijn vader hem gesmaad had.
\par 35 En het geschiedde des morgens, dat Jonathan in het veld ging, op den tijd, die David bestemd was; en er was een kleine jongen bij hem.
\par 36 En hij zeide tot zijn jongen: Loop, zoek nu de pijlen, die ik schieten zal. De jongen liep heen, en hij schoot een pijl, dien hij deed over hem vliegen.
\par 37 Toen de jongen tot aan de plaats des pijls, dien Jonathan geschoten had, gekomen was, zo riep Jonathan den jongen na, en zeide: Is niet de pijl van u af en verder?
\par 38 Wederom riep Jonathan den jongen na: Haast u, spoed u, sta niet stil! De jongen van Jonathan nu raapte den pijl op, en hij kwam tot zijn heer.
\par 39 Doch de jongen wist er niets van; Jonathan en David alleen wisten van de zaak.
\par 40 Toen gaf Jonathan zijn gereedschap aan den jongen, dien hij had; en hij zeide tot hem: Ga heen, breng het in de stad.
\par 41 Als de jongen heenging, zo stond David op van de zuidzijde, en hij viel op zijn aangezicht ter aarde, en hij boog zich driemaal; en zij kusten elkander, en weenden met elkander, totdat het David gans veel maakte.
\par 42 Toen zeide Jonathan tot David: Ga in vrede; hetgeen wij beiden in den Naam des HEEREN gezworen hebben, zeggende: De HEERE zij tussen mij en tussen u, en tussen mijn zaad en tussen uw zaad, zij tot in eeuwigheid! Daarna stond hij op, en ging heen; en Jonathan kwam in de stad.

\chapter{21}

\par 1 Toen kwam David te Nob, tot den priester Achimelech; en Achimelech kwam bevende David tegemoet, en hij zeide tot hem: Waarom zijt gij alleen, en geen man met u?
\par 2 En David zeide tot den priester Achimelech: De koning heeft mij een zaak bevolen, en zeide tot mij: Laat niemand iets van de zaak weten, om dewelke ik u gezonden heb, en die ik u geboden heb; den jongelingen nu heb ik de plaats van zulk een te kennen te kennen gegeven.
\par 3 En nu wat is er onder uw hand? Geef mij vijf broden in mijn hand, of wat er gevonden wordt.
\par 4 En de priester antwoordde David, en zeide: Er is geen gemeen brood onder mijn hand; maar er is heilig brood, wanneer zich de jongelingen slechts van de vrouwen onthouden hebben.
\par 5 David nu antwoordde den priester, en zeide tot hem: Ja trouwens, de vrouwen zijn ons onthouden geweest gisteren en eergisteren, toen ik uitging, en de vaten der jongelingen zijn heilig; en het is enigerwijze gemeen brood, te meer dewijl heden ander in de vaten zal geheiligd worden.
\par 6 Toen gaf de priester hem dat heilige brood, dewijl er geen brood was dan de toonbroden, die van voor het aangezicht des HEEREN weggenomen waren, dat men er warm brood leide, ten dage als dat weggenomen werd.
\par 7 Daar was nu een man van de knechten van Saul, te dienzelven dage opgehouden voor het aangezicht des HEEREN, en zijn naam was Doeg, een Edomiet, de machtigste onder de herderen, die Saul had.
\par 8 En David zeide tot Achimelech: Is hier onder uw hand geen spies of zwaard? Want ik heb noch mijn zwaard noch ook mijn wapenen in mijn hand genomen, dewijl de zaak des konings haastig was.
\par 9 Toen zeide de priester: Het zwaard van Goliath, den Filistijn, denwelken gij sloegt in het eikendal, zie, dat is hier, gewonden in een kleed, achter den efod; indien gij u dat nemen wilt, zo neem het, want hier is geen ander dan dit. David nu zeide: Er is zijns gelijke niet; geef het mij.
\par 10 En David maakte zich op, en vluchtte te dien dage van het aangezicht van Saul; en hij kwam tot Achis, den koning van Gath.
\par 11 Doch de knechten van Achis zeiden tot hem: Is deze niet David, de koning des lands? Zong men niet van dezen in de reien, zeggende: Saul heeft zijn duizenden verslagen, maar David zijn tienduizenden?
\par 12 En David leide deze woorden in zijn hart; en hij was zeer bevreesd voor het aangezicht van Achis, den koning van Gath.
\par 13 Daarom veranderde hij zijn gelaat voor hun ogen, en hij maakte zichzelven gek onder hun handen; en hij bekrabbelde de deuren der poort, en hij liet zijn zever in zijn baard aflopen.
\par 14 Toen zeide Achis tot zijn knechten: Ziet, gij ziet, dat de man razende is, waarom hebt gij hem tot mij gebracht?
\par 15 Heb ik razenden gebrek, dat gij dezen gebracht hebt, om voor mij te razen? Zal deze in mijn huis komen?

\chapter{22}

\par 1 Toen ging David van daar, en ontkwam in de spelonk van Adullam. En zijn broeders hoorden het, en het ganse huis zijns vaders, en kwamen derwaarts tot hem af.
\par 2 En tot hem vergaderde alle man, die benauwd was, en alle man, die een schuldeiser had, en alle man, wiens ziel bitterlijk bedroefd was, en hij werd tot overste over hen; zodat bij hem waren omtrent vierhonderd mannen.
\par 3 En David ging van daar naar Mizpa der Moabieten; en hij zeide tot den koning der Moabieten: Laat toch mijn vader en mijn moeder bij ulieden uitgaan, totdat ik weet, wat God mij doen zal.
\par 4 En hij bracht hen voor het aangezicht van den koning der Moabieten; en zij bleven bij hem al de dagen, die David in de vesting was.
\par 5 Doch de profeet Gad zeide tot David: Blijf in de vesting niet, ga heen, en ga in het land van Juda. Toen ging David heen, en hij kwam in het woud Chereth.
\par 6 En Saul hoorde, dat David bekend geworden was, en de mannen, die bij hem waren. Saul nu zat op een heuvel onder het geboomte te Rama, en hij had zijn spies in zijn hand, en al zijn knechten stonden bij hem.
\par 7 Toen zeide Saul tot zijn knechten, die bij hem stonden: Hoort toch, gij, zonen van Jemini, zal ook de zoon van Isai u altegader akkers en wijnbergen geven? Zal hij u allen tot oversten van duizenden, en oversten van honderden stellen?
\par 8 Dat gij u allen tegen mij verbonden hebt, en niemand voor mijn oor openbaart, dat mijn zoon een verbond gemaakt heeft met den zoon van Isai; en niemand is onder ulieden, dien het wee doet van mijnentwege, en die het voor mijn oor openbaart; want mijn zoon heeft mijn knecht tegen mij opgewekt, tot een lagenlegger, gelijk het te dezen dage is.
\par 9 Toen antwoordde Doeg, de Edomiet, die bij de knechten van Saul stond, en zeide: Ik zag den zoon van Isai, komende te Nob, tot Achimelech, den zoon van Ahitub;
\par 10 Die den HEERE voor hem vraagde, en gaf hem teerkost; hij gaf hem ook het zwaard van Goliath, den Filistijn.
\par 11 Toen zond de koning heen, om den priester Achimelech, den zoon van Ahitub, te roepen, en zijns vaders ganse huis, de priesters, die te Nob waren; en zij kwamen allen tot den koning.
\par 12 En Saul zeide: Hoor nu, gij, zoon van Ahitub! En hij zeide: Zie, hier ben ik, mijn heer!
\par 13 Toen zeide Saul tot hem: Waarom hebt gijlieden samen u tegen mij verbonden, gij en de zoon van Isai, mits dat gij hem gegeven hebt brood en het zwaard, en God voor hem gevraagd, dat hij zou opstaan tegen mij tot een lagenlegger, gelijk het te dezen dage is?
\par 14 En Achimelech antwoordde den koning en zeide: Wie is toch onder al uw knechten getrouw als David, en des konings schoonzoon, en voortgaande in uw gehoorzaamheid, en is eerlijk in uw huis?
\par 15 Heb ik heden begonnen God voor hem te vragen? Dat zij verre van mij, de koning legge op zijn knecht geen ding, noch op het ganse huis mijns vader; want uw knecht heeft van al deze dingen niet geweten, klein noch groot.
\par 16 Doch de koning zeide: Achimelech, gij moet den dood sterven, gij en het ganse huis uws vaders.
\par 17 En de koning zeide tot de trawanten, die bij hem stonden: Wendt u, en doodt de priesters des HEEREN, omdat hun hand ook met David is, en omdat zij geweten hebben, dat hij vluchtte, en hebben het voor mijn oren niet geopenbaard. Doch de knechten des konings wilden hun hand niet uitsteken, om op de priesters des HEEREN aan te vallen.
\par 18 Toen zeide de koning tot Doeg: Wend gij u, en val aan op de priesters. Toen wendde zich Doeg, de Edomiet, en hij viel aan op de priesters, en doodde te dien dage vijf en tachtig mannen, die den linnen lijfrok droegen.
\par 19 Hij sloeg ook Nob, de stad dezer priesters, met de scherpte des zwaards, van den man tot de vrouw, van de kinderen tot de zuigelingen, zelfs de ossen en ezels, en de schapen, sloeg hij met de scherpte des zwaards.
\par 20 Doch een der zonen van Achimelech, den zoon van Ahitub, ontkwam, wiens naam was Abjathar; die vluchtte David na.
\par 21 En Abjathar boodschapte het David, dat Saul de priesteren des HEEREN gedood had.
\par 22 Toen zeide David tot Abjathar: Ik wist wel te dien dage, toen Doeg, de Edomiet, daar was, dat hij het voorzeker Saul zou te kennen geven; ik heb oorzaak gegeven tegen al de zielen van uws vaders huis.
\par 23 Blijf bij mij; vrees niet; want wie mijn ziel zoeken zal, die zal uw ziel zoeken; maar gij zult met mij in bewaring zijn.

\chapter{23}

\par 1 En men boodschapte David, zeggende: Zie, de Filistijnen strijden tegen Kehila, en zij beroven de schuren.
\par 2 En David vraagde den HEERE, zeggende: Zal ik heengaan en deze Filistijnen slaan? En de HEERE zeide tot David: Ga heen, en gij zult de Filistijnen slaan en Kehila verlossen.
\par 3 Doch de mannen Davids zeiden tot hem: Zie, wij vrezen hier in Juda; hoeveel te meer, als wij naar Kehila tegen der Filistijnen slagorden gaan zullen.
\par 4 Toen vraagde David den HEERE nog verder; en de HEERE antwoordde hem en zeide: Maak u op, trek af naar Kehila; want Ik geef de Filistijnen in uw hand.
\par 5 Alzo toog David en zijn mannen naar Kehila, en hij streed tegen de Filistijnen, en dreef hun vee weg, en hij sloeg onder hen een groten slag; alzo verloste David de inwoners van Kehila.
\par 6 En het geschiedde, toen Abjathar, de zoon van Achimelech, tot David vluchtte naar Kehila, dat hij afkwam met den efod in zijn hand.
\par 7 Als aan Saul te kennen gegeven werd, dat David te Kehila gekomen was, zo zeide Saul: God heeft hem in mijn hand overgegeven, want hij is besloten, komende in een stad met poorten en grendelen.
\par 8 Toen liet Saul al het volk ten strijde roepen, dat zij aftogen naar Kehila, om David en zijn mannen te belegeren.
\par 9 Als nu David verstond, dat Saul dit kwaad tegen hem heimelijk voorhad, zeide hij tot den priester Abjathar: Breng den efod herwaarts.
\par 10 En David zeide: HEERE, God van Israel! Uw knecht heeft zekerlijk gehoord, dat Saul zoekt naar Kehila te komen, en de stad te verderven om mijnentwil.
\par 11 Zullen mij ook de burgers van Kehila in zijn hand overgeven? Zal Saul afkomen, gelijk als Uw knecht gehoord heeft? O HEERE, God van Israel, geef het toch Uw knecht te kennen! De HEERE nu zeide: Hij zal afkomen.
\par 12 Daarna zeide David: Zouden de burgers van Kehila mij en mijn mannen overgeven in de hand van Saul? En de HEERE zeide: Zij zouden u overgeven.
\par 13 Toen maakte zich David en zijn mannen op, omtrent zeshonderd man, en zij gingen uit Kehila, en zij gingen heen, waar zij konden gaan. Toen aan Saul geboodschapt werd, dat David uit Kehila ontkomen was, zo hield hij op uit te trekken.
\par 14 David nu bleef in de woestijn in de vestingen, en hij bleef op den berg in de woestijn Zif; en Saul zocht hem alle dagen, doch God gaf hem niet over in zijn hand.
\par 15 Als David zag, dat Saul uitgetogen was, om zijn ziel te zoeken, zo was David in de woestijn Zif in een woud.
\par 16 Toen maakte zich Jonathan, de zoon van Saul, op, en hij ging tot David in het woud; en hij versterkte zijn hand in God.
\par 17 En hij zeide tot hem: Vrees niet, want de hand van Saul, mijn vader, zal u niet vinden, maar gij zult koning worden over Israel, en ik zal de tweede bij u zijn; ook weet mijn vader Saul zulks wel.
\par 18 En die beiden maakten een verbond voor het aangezicht des HEEREN; en David bleef in het woud, maar Jonathan ging naar zijn huis.
\par 19 Toen togen de Zifieten op tot Saul naar Gibea, zeggende: Heeft zich niet David bij ons verborgen in de vestingen in het woud, op den heuvel van Hachila, die aan de rechterhand der wildernis is?
\par 20 Nu dan, o koning, kom spoedig af naar al de begeerte uwer ziel; en het komt ons toe hem over te geven in de hand des konings.
\par 21 Toen zeide Saul: Gezegend zijt gijlieden den HEERE, dat gij u over mij ontfermd hebt!
\par 22 Gaat toch heen, en bereidt de zaak nog meer, dat gij weet en beziet zijn plaats, waar zijn gang is, wie hem daar gezien heeft; want hij heeft tot mij gezegd, dat hij zeer listiglijk pleegt te handelen.
\par 23 Daarom ziet toe, en verneemt naar alle schuilplaatsen, in dewelke hij schuilt; komt dan weder tot mij met vast bescheid, zo zal ik met ulieden gaan; en het zal geschieden, zo hij in het land is, zo zal ik hem naspeuren onder alle duizenden van Juda.
\par 24 Toen maakten zij zich op, en zij gingen naar Zif voor het aangezicht van Saul. David nu en zijn mannen waren in de woestijn van Maon, in het vlakke veld, aan de rechterhand der wildernis.
\par 25 Saul en zijn mannen gingen ook om te zoeken. Dat werd David geboodschapt, die van dien rotssteen afgegaan was, en bleef in de woestijn van Maon. Toen Saul dat hoorde, jaagde hij David na in de woestijn van Maon.
\par 26 En Saul ging aan deze zijde des bergs, en David en zijn mannen aan gene zijde des bergs. Het geschiedde nu, dat zich David haastte, om te ontgaan van het aangezicht van Saul; en Saul en zijn mannen omsingelden David en zijn mannen, om die te grijpen.
\par 27 Doch daar kwam een bode tot Saul, zeggende: Haast u, en kom, want de Filistijnen zijn in het land gevallen.
\par 28 Toen keerde zich Saul van David na te jagen, en hij toog den Filistijnen tegemoet; daarom noemde men die plaats Sela-machlekoth.

\chapter{24}

\par 1 En David toog van daar op, en hij bleef in de vestingen van En-gedi.
\par 2 En het geschiedde, nadat Saul wedergekeerd was van achter de Filistijnen, zo gaf men hem te kennen, zeggende: Zie, David is in de woestijn van En-gedi.
\par 3 Toen nam Saul drie duizend uitgelezen mannen uit gans Israel, en hij toog heen, om David en zijn mannen te zoeken boven op de rotsstenen der steenbokken.
\par 4 En hij kwam tot de schaapskooien aan den weg, waar een spelonk was; en Saul ging daarin, om zijn voeten te dekken. David nu en zijn mannen zaten aan de zijden der spelonk.
\par 5 Toen zeiden de mannen van David tot hem: Zie den dag, in welken de HEERE tot u zegt: Zie, Ik geef uw vijand in uw hand, en gij zult hem doen, gelijk als het goed zal zijn in uw ogen. En David stond op, en sneed stilletjes een slip van Sauls mantel.
\par 6 Doch het geschiedde daarna, dat Davids hart hem sloeg, omdat hij de slip van Saul afgesneden had.
\par 7 En hij zeide tot zijn mannen: Dat late de HEERE ver van mij zijn, dat ik die zaak doen zou aan mijn heer, den gezalfde des HEEREN, dat ik mijn hand tegen hem uitsteken zou; want hij is de gezalfde des HEEREN!
\par 8 En David scheidde zijn mannen met woorden, en liet hun niet toe, dat zij opstonden tegen Saul. En Saul maakte zich op uit de spelonk, en ging op den weg.
\par 9 Daarna maakte zich David ook op, en ging uit de spelonk, en hij riep Saul achterna, zeggende: Mijn heer koning! Toen zag Saul achter zich om, en David boog zich met het aangezicht ter aarde en neigde zich.
\par 10 En David zeide tot Saul: Waarom hoort gij de woorden der mensen, zeggende: Zie, David zoekt uw kwaad?
\par 11 Zie, te dezen dage hebben uw ogen gezien, dat de HEERE u heden in mijn hand gegeven heeft in deze spelonk, en men zeide, dat ik u doden zou; doch mijn hand verschoonde u, want ik zeide: Ik zal mijn hand niet uitsteken tegen mijn heer, want hij is de gezalfde des HEEREN.
\par 12 Zie toch, mijn vader, ja, zie de slip uws mantels in mijn hand; want als ik de slip uws mantels afgesneden heb, zo heb ik u niet gedood; beken en zie, dat er in mijn hand geen kwaad, noch overtreding is, en ik tegen u niet gezondigd heb; nochtans jaagt gij mijn ziel, dat gij ze wegneemt.
\par 13 De HEERE zal richten tussen mij en tussen u, en de HEERE zal mij wreken aan u; maar mijn hand zal niet tegen u zijn.
\par 14 Gelijk als het spreekwoord der ouden zegt: Van de goddelozen komt goddeloosheid voort; maar mijn hand zal niet tegen u zijn.
\par 15 Naar wien is de koning van Israel uitgegaan? Wien jaagt gij na? Naar een doden hond, naar een enige vlo!
\par 16 Doch de HEERE zal zijn tot Rechter, en richten tussen mij en tussen u, en zien daarin, en twisten mijn twist, en richten mij van uw hand.
\par 17 En het geschiedde, toen David geeindigd had al deze woorden tot Saul te spreken, zo zeide Saul: Is dit uw stem, mijn zoon David? Toen hief Saul zijn stem op en weende.
\par 18 En hij zeide tot David: Gij zijt rechtvaardiger dan ik; want gij hebt mij goed vergolden, en ik heb u kwaad vergolden.
\par 19 En gij hebt mij heden aangewezen, dat gij mij goed gedaan hebt; want de HEERE had mij in uw hand besloten, en gij hebt mij niet gedood.
\par 20 Zo wanneer iemand zijn vijand gevonden heeft, zal hij hem op een goeden weg laten gaan? De HEERE nu vergelde u het goede, voor dezen dag, dien gij mij heden gemaakt hebt.
\par 21 En nu, zie, ik weet, dat gij voorzeker koning worden zult, en dat het koninkrijk van Israel in uw hand bestaan zal.
\par 22 Zo zweer mij dan nu bij den HEERE, zo gij mijn zaad na mij zult uitroeien, en mijn naam zult uitdelgen van mijns vaders huis!
\par 23 Toen zwoer David aan Saul; en Saul ging in zijn huis, maar David en zijn mannen gingen op in de vesting.

\chapter{25}

\par 1 En Samuel stierf; en gans Israel vergaderde zich, en zij bedreven rouw over hem, en begroeven hem in zijn huis te Rama. En David maakte zich op, en toog af naar de woestijn Paran.
\par 2 En er was een man te Maon, en zijn bedrijf was te Karmel; en die man was zeer groot, en hij had drie duizend schapen, en duizend geiten; en hij was in het scheren zijner schapen te Karmel.
\par 3 En de naam des mans was Nabal, en de naam zijner huisvrouw was Abigail; en de vrouw was goed van verstand, en schoon van gedaante; maar de man was hard en boos van daden, en hij was een Kalebiet.
\par 4 Als David hoorde in de woestijn, dat Nabal zijn schapen schoor,
\par 5 Zo zond David tien jongelingen; en David zeide tot de jongelingen: Gaat op naar Karmel, en als gij tot Nabal komt, zo zult gij hem in mijn naam naar den welstand vragen;
\par 6 En zult alzo zeggen tot dien welvarende: Vrede zij u, en uw huize zij vrede, en alles, wat gij hebt, zij vrede!
\par 7 En nu, ik heb gehoord, dat gij scheerders hebt; nu, de herders, die gij hebt, zijn bij ons geweest; wij hebben hun geen smaadheid aangedaan, en zij hebben ook niets gemist al de dagen, die zij te Karmel geweest zijn.
\par 8 Vraag het uw jongelingen, en zij zullen het u te kennen geven. Laat dan deze jongelingen genade vinden in uw ogen, want wij zijn op een goeden dag gekomen; geef toch uw knechten, en uw zoon David, hetgeen uw hand vinden zal.
\par 9 Toen de jongelingen van David gekomen waren, en in Davids naam naar al die woorden tot Nabal gesproken hadden, zo hielden zij stil.
\par 10 En Nabal antwoordde den knechten van David, en zeide: Wie is David, en wie is de zoon van Isai? Er zijn heden vele knechten, die zich afscheuren, elk van zijn heer.
\par 11 Zou ik dan mijn brood, en mijn water, en mijn geslacht vlees nemen, dat ik voor mijn scheerders geslacht heb, en zou ik het den mannen geven, die ik niet weet, van waar zij zijn?
\par 12 Toen keerden zich de jongelingen van David naar hun weg; en zij keerden weder, en kwamen, en boodschapten hem achtervolgens al deze woorden.
\par 13 David dan zeide tot zijn mannen: Een iegelijk gorde zijn zwaard aan. Toen gordde een iegelijk zijn zwaard aan, en David gordde ook zijn zwaard aan; en zij togen op achter David, omtrent vierhonderd man, en daar bleven er tweehonderd bij het gereedschap.
\par 14 Doch een jongeling uit de jongelingen boodschapte het aan Abigail, de huisvrouw van Nabal, zeggende: Zie, David heeft boden gezonden uit de woestijn, om onzen heer te zegenen; maar hij is tegen hen uitgevaren.
\par 15 Nochtans zijn zij ons zeer goede mannen geweest; en wij hebben geen smaadheid geleden, en wij hebben niets gemist al de dagen, die wij met hen verkeerd hebben, toen wij op het veld waren.
\par 16 Zij zijn een muur om ons geweest, zo bij nacht als bij dag, al de dagen, die wij bij hen geweest zijn, weidende de schapen.
\par 17 Weet dan nu, en zie, wat gij doen zult; want het kwaad is ten volle over onzen heer besloten, en over zijn ganse huis; en hij is een zoon Belials, dat men hem niet mag aanspreken.
\par 18 Toen haastte zich Abigail, en nam tweehonderd broden, en twee lederzakken wijns, en vijf toebereide schapen, en vijf maten geroost koren, en honderd stukken rozijnen, en tweehonderd klompen vijgen, en leide die op ezelen.
\par 19 En zij zeide tot haar jongelingen: Trekt heen voor mijn aangezicht; ziet, ik kom achter ulieden; doch haar man Nabal gaf zij het niet te kennen.
\par 20 Het geschiedde nu, toen zij op den ezel reed, en dat zij afkwam in het verborgene des bergs, en ziet, David en zijn mannen kwamen af haar tegemoet, en zij ontmoette hen.
\par 21 David nu had gezegd: Trouwens ik heb te vergeefs bewaard al wat deze in de woestijn heeft, alzo dat er niets van alles, wat hij heeft, gemist is; en hij heeft mij kwaad voor goed vergolden.
\par 22 Zo doe God aan de vijanden van David, en zo doe Hij daartoe, indien ik van allen, die hij heeft, iets tot morgen overlaat, dat mannelijk is!
\par 23 Toen nu Abigail David zag, zo haastte zij zich, en kwam van den ezel af, en zij viel voor het aangezicht van David op haar aangezicht, en zij boog zich ter aarde.
\par 24 En zij viel aan zijn voeten en zeide: Och, mijn heer, mijn zij de misdaad, en laat toch uw dienstmaagd voor uw oren spreken, en hoor de woorden uwer dienstmaagd.
\par 25 Mijn heer stelle toch zijn hart niet aan dezen Belials man, aan Nabal; want gelijk zijn naam is, alzo is hij; zijn naam is Nabal, en dwaasheid is bij hem; en ik, uw dienstmaagd, heb de jongelingen van mijn heer niet gezien, die gij gezonden hebt.
\par 26 En nu, mijn heer! zo waarachtig als de HEERE leeft, en uw ziel leeft, het is de HEERE, Die u verhinderd heeft van te komen met bloedstorting, dat uw hand u zou verlossen; en nu, dat als Nabal worden uw vijanden, en die tegen mijn heer kwaad zoeken!
\par 27 En nu, dit is de zegen, dien uw dienstmaagd mijn heer toegebracht heeft, dat hij gegeven worde den jongelingen, die mijns heren voetstappen nawandelen.
\par 28 Vergeef toch aan uw dienstmaagd de overtreding, want de HEERE zal zekerlijk mijn heer een bestendig huis maken, dewijl mijn heer de oorlogen des HEEREN oorloogt, en geen kwaad bij u gevonden is van uw dagen af.
\par 29 Wanneer een mens opstaan zal om u te vervolgen, en om uw ziel te zoeken, zo zal de ziel mijns heren ingebonden zijn in het bundeltje der levenden bij den HEERE, uw God; maar de ziel uwer vijanden zal Hij slingeren uit het midden van de holligheid des slingers.
\par 30 En het zal geschieden, als de HEERE mijn heer naar al het goede doen zal, dat Hij over u gesproken heeft, en Hij u gebieden zal een voorganger te zijn over Israel;
\par 31 Zo zal dit u, mijn heer, niet zijn tot wankeling, noch aanstoot des harten, te weten, dat gij bloed zonder oorzaak zoudt vergoten hebben, en dat mijn heer zichzelven zou verlost hebben; en als de HEERE mijn heer weldoen zal, zo zult gij uwer dienstmaagd gedenken.
\par 32 Toen zeide David tot Abigail: Gezegend zij de HEERE, de God Israels, Die u te dezen dage mij tegemoet gezonden heeft!
\par 33 En gezegend zij uw raad en gezegend zijt gij, dat gij mij te dezen dage geweerd hebt, van te komen met bloedstorting, dat mijn hand mij verlost zou hebben!
\par 34 Want voorzeker, het is zo waarachtig als de HEERE, de God Israels, leeft, Die mij verhinderd heeft, van u kwaad te doen, dat, ten ware dat gij u gehaast hadt, en mij tegemoet gekomen waart, zo ware van Nabal niemand, die mannelijk is, overgebleven tot het morgenlicht!
\par 35 Toen nam David uit haar hand, wat zij hem gebracht had; en hij zeide tot haar: Trek met vrede op naar uw huis; zie, ik heb naar uw stem gehoord, en heb uw aangezicht aangenomen.
\par 36 Toen nu Abigail tot Nabal kwam, ziet, zo had hij een maaltijd in zijn huis, als eens konings maaltijd; en het hart van Nabal was vrolijk op denzelven, en hij was zeer dronken; daarom gaf zij hem niet een woord, klein noch groot, te kennen, tot aan het morgenlicht.
\par 37 Het geschiedde nu in den morgen, toen de wijn van Nabal gegaan was, zo gaf hem zijn huisvrouw die woorden te kennen. Toen bestierf zijn hart in het binnenste van hem, en hij werd als een steen.
\par 38 En het geschiedde omtrent na tien dagen, zo sloeg de HEERE Nabal, dat hij stierf.
\par 39 Toen David hoorde, dat Nabal dood was, zo zeide hij: Gezegend zij de HEERE, Die den twist mijner smaadheid getwist heeft van de hand van Nabal, en heeft zijn knecht onthouden van het kwade, en dat de HEERE het kwaad van Nabal op zijn hoofd heeft doen wederkeren! En David zond heen, en liet met Abigail spreken, dat hij ze zich ter vrouwe nam.
\par 40 Als nu de knechten van David tot Abigail gekomen waren te Karmel, zo spraken zij tot haar, zeggende: David heeft ons tot u gezonden, dat hij zich u ter vrouwe neme.
\par 41 Toen stond zij op, en neigde zich met het aangezicht ter aarde, en zij zeide: Ziet, uw dienstmaagd zij tot een dienares, om de voeten der knechten mijns heren te wassen.
\par 42 Abigail nu haastte, en maakte zich op, en zij reed op een ezel, met haar vijf jonge maagden, die haar voetstappen nawandelden; zij dan volgde de boden van David na, en zij werd hem ter huisvrouw.
\par 43 Ook nam David Ahinoam van Jizreel; alzo waren ook die beiden hem tot vrouwen.
\par 44 Want Saul had zijn dochter Michal, de huisvrouw van David, gegeven aan Palti, den zoon van Lais, die van Gallim was.

\chapter{26}

\par 1 De Zifieten nu kwamen tot Saul te Gibea, zeggende: Houdt zich David niet verborgen op den heuvel van Hachila, voor aan de wildernis?
\par 2 Toen maakte zich Saul op, en toog af naar de woestijn Zif, en met hem drie duizend man, uitgelezenen van Israel, om David te zoeken in de woestijn Zif.
\par 3 En Saul legerde zich op den heuvel van Hachila, die voor aan de wildernis is aan den weg, maar David bleef in de woestijn, en zag, dat Saul achter hem kwam naar de woestijn.
\par 4 Want David had verspieders gezonden, en hij vernam, dat Saul voorzeker kwam.
\par 5 En David maakte zich op, en kwam aan de plaats, waar Saul zich gelegerd had, en David bezag de plaats, waar Saul lag, met Abner, den zoon van Ner, zijn krijgsoverste. En Saul lag in den wagenburg, en het volk was rondom hem gelegerd.
\par 6 Toen antwoordde David, en sprak tot Achimelech, den Hethiet, en tot Abisai, den zoon van Zeruja, den broeder van Joab, zeggende: Wie zal met mij tot Saul in het leger afgaan? Toen zeide Abisai: Ik zal met u afgaan.
\par 7 Alzo kwamen David en Abisai tot het volk des nachts; en ziet, Saul lag te slapen in den wagenburg, en zijn spies stak in de aarde aan zijn hoofdeinde, en Abner, en het volk lag rondom hem.
\par 8 Toen zeide Abisai tot David: God heeft heden uw vijand in uw hand besloten; laat mij toch hem nu met de spies op eenmaal ter aarde slaan, en ik zal het hem niet ten tweeden male doen.
\par 9 David daarentegen zeide tot Abisai: Verderf hem niet; want wie heeft zijn hand aan den gezalfde des HEEREN gelegd, en is onschuldig gebleven?
\par 10 Verder zeide David: Zo waarachtig als de HEERE leeft, maar de HEERE zal hem slaan, of zijn dag zal komen, dat hij zal sterven, of hij zal in een strijd trekken, dat hij omkome.
\par 11 De HEERE late het verre van mij zijn, dat ik mijn hand legge aan den gezalfde des HEEREN! zo neem toch nu de spies, die aan zijn hoofdeinde is, en de waterfles, en laat ons gaan.
\par 12 Zo nam David de spies en de waterfles van Sauls hoofdeinde, en zij gingen heen; en er was niemand, die het zag, en niemand, die het merkte, ook niemand, die ontwaakte; want zij sliepen allen; want er was een diepe slaap des HEEREN op hen gevallen.
\par 13 Toen David over aan gene zijde gekomen was, zo stond hij op de hoogte des bergs van verre, dat er een grote plaats tussen hen was.
\par 14 En David riep tot het volk, en tot Abner, den zoon van Ner, zeggende: Zult gij niet antwoorden, Abner? Toen antwoordde Abner en zeide: Wie zijt gij, die tot den koning roept?
\par 15 Toen zeide David tot Abner: Zijt gij niet een man, en wie is u gelijk in Israel? Waarom dan hebt gij over uw heer, den koning, geen wacht gehouden? Want daar is een van het volk gekomen, om den koning, uw heer, te verderven.
\par 16 Deze zaak, die gij gedaan hebt, is niet goed; zo waarachtig als de HEERE leeft, gijlieden zijt kinderen des doods, die over uw heer, den gezalfde des HEEREN, geen wacht gehouden hebt! En nu, zie, waar de spies des konings is, en de waterfles, die aan zijn hoofdeinde was.
\par 17 Saul nu kende de stem van David, en zeide: Is dit uw stem, mijn zoon David? David zeide: Het is mijn stem, mijn heer koning!
\par 18 Hij zeide verder: Waarom vervolgt mijn heer zijn knecht alzo achterna, want wat heb ik gedaan, en wat kwaad is er in mijn hand?
\par 19 En nu, mijn heer de koning hore toch naar de woorden zijns knechts. Indien de HEERE u tegen mij aanport, laat Hem het spijsoffer rieken; maar indien het mensenkinderen zijn, zo zijn zij vervloekt voor het aangezicht des HEEREN, dewijl zij mij heden verstoten, dat ik niet mag vastgehecht blijven in het erfdeel des HEEREN, zeggende: Ga heen, dien andere goden.
\par 20 En nu, mijn bloed valle niet op de aarde van voor het aangezicht des HEEREN; want de koning van Israel is uitgegaan om een enige vlo te zoeken, gelijk als men een veldhoen op de bergen najaagt.
\par 21 Toen zeide Saul: Ik heb gezondigd; keer weder, mijn zoon David, want ik zal u geen kwaad meer doen, voor dat mijn ziel dezen dag dierbaar in uw ogen geweest is; zie, ik heb dwaselijk gedaan, en ik heb zeer grotelijks gedwaald.
\par 22 Toen antwoordde David, en zeide: Zie, de spies des konings; zo laat een van de jongelingen overkomen, en halen ze.
\par 23 De HEERE dan vergelde aan een iegelijk zijn gerechtigheid en zijn getrouwheid; want de HEERE had u heden in mijn hand gegeven; maar ik heb mijn hand niet willen uitsteken, aan den gezalfde des HEEREN.
\par 24 En zie, gelijk als te dezen dage uw ziel in mijn ogen is groot geacht geweest, alzo zij mijn ziel in de ogen des HEEREN groot geacht, en Hij verlosse mij uit allen nood.
\par 25 Toen zeide Saul tot David: Gezegend zijt gij, mijn zoon David; gij zult het ja gewisselijk doen, en gij zult ook gewisselijk de overhand hebben. Toen ging David op zijn weg, en Saul keerde weder naar zijn plaats.

\chapter{27}

\par 1 David nu zeide in zijn hart: Nu zal ik een der dagen door Sauls hand omkomen; mij is niet beter, dan dat ik haastelijk ontkome in het land der Filistijnen, opdat Saul van mij de hoop verlieze, om mij meer te zoeken in de ganse landpale van Israel; zo zal ik ontkomen uit zijn hand.
\par 2 Toen maakte zich David op, en hij ging door, hij en de zeshonderd mannen, die bij hem waren, tot Achis, den zoon van Maoch, den koning van Gath.
\par 3 En David bleef bij Achis te Gath, hij en zijn mannen, een iegelijk met zijn huis; David met zijn beide vrouwen, Ahinoam, en Jizreelietische, en Abigail, de huisvrouw van Nabal, de Karmelietische.
\par 4 Toen aan Saul geboodschapt werd, dat David gevlucht was naar Gath, zo voer hij niet meer voort hem te zoeken.
\par 5 En David zeide tot Achis: Indien ik nu genade in uw ogen gevonden heb, men geve mij een plaats in een van de steden des lands, dat ik daar wone; want waarom zou uw knecht in de koninklijke stad bij u wonen?
\par 6 Toen gaf Achis te dien dage Ziklag; daarom is Ziklag van de koningen van Juda geweest tot op dezen dag.
\par 7 Het getal nu der dagen, die David in het land der Filistijnen woonde, was een jaar en vier maanden.
\par 8 David nu toog op met zijn mannen, en zij overvielen de Gesurieten, en de Girzieten, en de Amalekieten (want deze zijn vanouds geweest de inwoners des lands), dat gij gaat naar Sur, en tot aan Egypteland.
\par 9 En David sloeg dat land, en liet noch man noch vrouw leven; ook nam hij de schapen en runderen, en de ezelen, en kemels, en klederen, en keerde weder en kwam tot Achis.
\par 10 Als Achis zeide: Waar zijt gijlieden heden ingevallen? zo zeide David: Tegen het zuiden van Juda, en tegen het zuiden der Jerahmeelieten, en tegen het zuiden der Kenieten.
\par 11 En David liet noch man noch vrouw leven, om te Gath te brengen, zeggende: Dat zij misschien van ons niet boodschappen, zeggende: Alzo heeft David gedaan! En alzo was zijn wijze al de dagen, die hij in der Filistijnen land gewoond heeft.
\par 12 En Achis geloofde David, zeggende: Hij heeft zich ten enenmaal stinkende gemaakt bij zijn volk, in Israel; daarom zal hij eeuwiglijk mij tot een knecht zijn.

\chapter{28}

\par 1 En het geschiedde in die dagen, als de Filistijnen hun legers vergaderden tot den strijd, om tegen Israel te strijden, zo zeide Achis tot David: Gij zult zekerlijk weten, dat gij met mij in het leger zult uittrekken, gij en uw mannen.
\par 2 Toen zeide David tot Achis: Aldus zult gij weten, wat uw knecht doen zal. En Achis zeide tot David: Daarom zal ik u ten bewaarder mijns hoofds zetten, te allen dage.
\par 3 Samuel nu was gestorven, en gans Israel had rouw over hem bedreven; en zij hadden hem begraven te Rama, te weten in zijn stad. En Saul had uit het land weggedaan de waarzeggers en duivelskunstenaars.
\par 4 En de Filistijnen kwamen en vergaderden zich, en zij legerden zich te Sunem; en Saul vergaderde gans Israel, en zij legerden zich op Gilboa.
\par 5 Toen Saul het leger der Filistijnen zag, zo vreesde hij, en zijn hart beefde zeer.
\par 6 En Saul vraagde den HEERE; maar de HEERE antwoordde hem niet; noch door dromen, noch door de urim, noch door de profeten.
\par 7 Toen zeide Saul tot zijn knechten: Zoekt mij een vrouw, die een waarzeggenden geest heeft, dat ik tot haar ga, en door haar onderzoeke. Zijn knechten nu zeiden tot hem: Zie, te Endor is een vrouw, die een waarzeggenden geest heeft.
\par 8 En Saul verstelde zich, en trok andere klederen aan, en ging heen, en twee mannen met hem, en zij kwamen des nachts tot de vrouw, en hij zeide: Voorzeg mij toch door den waarzeggenden geest, en doe mij opkomen, dien ik tot u zeggen zal.
\par 9 Toen zeide de vrouw tot hem: Zie, gij weet, wat Saul gedaan heeft, hoe hij de waarzegsters en de duivelskunstenaars uit dit land heeft uitgeroeid; waarom stelt gij dan mijn ziel een strik, om mij te doden?
\par 10 Saul nu zwoer haar bij den HEERE, zeggende: Zo waarachtig als de HEERE leeft, indien u een straf om deze zaak zal overkomen!
\par 11 Toen zeide de vrouw: Wien zal ik u doen opkomen? En hij zeide: Doe mij Samuel opkomen.
\par 12 Toen nu de vrouw Samuel zag, zo riep zij met luider stem, en de vrouw sprak tot Saul, zeggende: Waarom hebt gij mij bedrogen? Want gij zijt Saul.
\par 13 En de koning zeide tot haar: Vrees niet; maar wat ziet gij? Toen zeide de vrouw tot Saul: Ik zie goden, uit de aarde opkomende.
\par 14 Hij dan zeide tot haar: Hoe is zijn gedaante? En zij zeide: Er komt een oud man op, en hij is met een mantel bekleed. Toen Saul vernam, dat het Samuel was, zo neigde hij zich met het aangezicht ter aarde, en hij boog zich.
\par 15 En Samuel zeide tot Saul: Waarom hebt gij mij onrustig gemaakt, mij doende opkomen? Toen zeide Saul: Ik ben zeer beangstigd, want de Filistijnen krijgen tegen mij, en God is van mij geweken, en antwoordt mij niet meer, noch door den dienst der profeten, noch door dromen; daarom heb ik u geroepen, dat gij mij te kennen geeft, wat ik doen zal.
\par 16 Toen zeide Samuel: Waarom vraagt gij mij toch, dewijl de HEERE van u geweken en uw vijand geworden is?
\par 17 Want de HEERE heeft voor Zich gedaan, gelijk als Hij door mijn dienst gesproken heeft; en heeft het koninkrijk van uw hand gescheurd, en Hij heeft dat gegeven aan uw naaste, aan David.
\par 18 Gelijk als gij naar de stem des HEEREN niet gehoord hebt, en de hittigheid Zijns toorns niet uitgericht hebt tegen Amalek; daarom heeft de HEERE u deze zaak gedaan te dezen dage.
\par 19 En de HEERE zal ook Israel met u in de hand der Filistijnen geven, en morgen zult gij en uw zonen bij mij zijn; ook zal de HEERE het leger van Israel in de hand der Filistijnen geven.
\par 20 Toen viel Saul haastelijk ter aarde, zo lang als hij was, en hij vreesde zeer vanwege de woorden van Samuel; ook was er geen kracht in hem; want hij had den gehelen dag en den gehelen nacht geen brood gegeten.
\par 21 De vrouw nu kwam tot Saul, en zag, dat hij zeer verbaasd was; en zij zeide tot hem: Zie, uw dienstmaagd heeft naar uw stem gehoord, en ik heb mijn ziel in mijn hand gesteld, en ik heb uw woorden gehoord, die gij tot mij gesproken hebt.
\par 22 Zo hoor toch gij nu ook naar de stem uwer dienstmaagd, en laat mij een bete broods voor u zetten, en eet; zo zal er kracht in u zijn, dat gij over weg gaat.
\par 23 Doch hij weigerde het, en zeide: Ik zal niet eten. Maar zijn knechten, en ook de vrouw, hielden bij hem aan. Toen hoorde hij naar hun stem, en hij stond op van de aarde, en zette zich op het bed.
\par 24 En de vrouw had een gemest kalf in het huis; en zij haastte zich en slachtte het; en zij nam meel, en kneedde het, en bakte daar ongezuurde koeken van.
\par 25 En zij bracht ze voor Saul en voor zijn knechten, en zij aten; daarna stonden zij op, en gingen weg in dienzelfden nacht.

\chapter{29}

\par 1 De Filistijnen nu hadden al hun legers vergaderd te Afek; en de Israelieten legerden zich bij de fontein, die bij Jizreel is.
\par 2 En de vorsten der Filistijnen togen daarheen met honderden, en met duizenden; doch David met zijn mannen togen met Achis in den achtertocht.
\par 3 Toen zeiden de oversten der Filistijnen: Wat zullen deze Hebreen? Zo zeide Achis tot de oversten der Filistijnen: Is deze niet David, de knecht van Saul, den koning van Israel, die deze dagen of deze jaren bij mij geweest is? En ik heb in hem niets gevonden van dien dag af, dat hij afgevallen is tot dezen dag toe.
\par 4 Doch de oversten der Filistijnen werden zeer toornig op hem, en de oversten der Filistijnen zeiden tot hem: Doe den man wederkeren, dat hij tot zijn plaats wederkere, waar gij hem besteld hebt, en dat hij niet met ons aftrekke in den strijd, opdat hij ons niet tot een tegenpartijder worde in den strijd; want waarmede zou deze zich bij zijn heer aangenaam maken? Is het niet met de hoofden dezer mannen?
\par 5 Is dit niet die David, van denwelken zij in den rei elkander antwoordden, zeggende: Saul heeft zijn duizenden geslagen, maar David zijn tienduizenden?
\par 6 Toen riep Achis David, en zeide tot hem: Het is zo waarachtig als de HEERE leeft, dat gij oprecht zijt, en uw uitgang en uw ingang met mij in het leger is goed in mijn ogen; want ik heb geen kwaad bij u gevonden, van dien dag af, dat gij tot mij zijt gekomen, tot dezen dag toe; maar gij zijt niet aangenaam in de ogen der vorsten.
\par 7 Zo keer nu om, en ga in vrede, opdat gij geen kwaad doet in de ogen van de vorsten der Filistijnen.
\par 8 Toen zeide David tot Achis: Maar wat heb ik gedaan? Of wat hebt gij in uw knecht gevonden, van dien dag af, dat ik voor uw aangezicht geweest ben, tot dezen dag toe, dat ik niet zal gaan en strijden tegen de vijanden van mijn heer, den koning?
\par 9 Achis nu antwoordde en zeide tot David: Ik weet het; voorwaar, gij zijt aangenaam in mijn ogen, als een engel Gods; maar de oversten der Filistijnen hebben gezegd: Laat hem met ons in dezen strijd niet optrekken.
\par 10 Nu dan, maak u morgen vroeg op met de knechten uws heren, die met u gekomen zijn; en als gijlieden u morgen vroeg zult opgemaakt hebben, en het ulieden licht geworden is, zo gaat heen.
\par 11 Toen maakte zich David vroeg op, hij en zijn mannen, dat zij des morgens weggingen, om weder te keren in het land der Filistijnen; de Filistijnen daarentegen togen op naar Jizreel.

\chapter{30}

\par 1 Het geschiedde nu, als David en zijn mannen den derden dag te Ziklag kwamen, dat de Amalekieten in het zuiden en te Ziklag ingevallen waren, en Ziklag geslagen, en dezelve met vuur verbrand hadden;
\par 2 En dat zij de vrouwen, die daarin waren, gevankelijk weggevoerd hadden; doch zij hadden niemand doodgeslagen, van den kleinste tot den grootste, maar hadden ze weggevoerd en waren huns weegs gegaan.
\par 3 En David en zijn mannen kwamen aan de stad, en ziet, zij was met vuur verbrand; en hun vrouwen, en hun zonen en hun dochteren waren gevankelijk weggevoerd.
\par 4 Toen hief David en het volk, dat bij hem was, hun stem op, en weenden, tot dat er geen kracht meer in hen was om te wenen.
\par 5 Davids beide vrouwen waren ook gevankelijk weggevoerd, Ahinoam, de Jizreelietische, en Abigail, de huisvrouw van Nabal, den Karmeliet.
\par 6 En David werd zeer bang, want het volk sprak van hem te stenigen; want de zielen van het ganse volk waren verbitterd, een iegelijk over zijn zonen en over zijn dochteren; doch David sterkte zich in den HEERE, zijn God.
\par 7 En David zeide tot den priester Abjathar, den zoon van Achimelech: Breng mij toch den efod hier. En Abjathar bracht den efod tot David.
\par 8 Toen vraagde David den HEERE, zeggende: Zal ik deze bende achternajagen? Zal ik ze achterhalen? En Hij zeide tot hem: Jaag na, want gij zult gewisselijk achterhalen, en gij zult gewisselijk verlossen.
\par 9 David dan ging heen, hij en de zes honderd mannen, die bij hem waren; en als zij kwamen aan de beek Besor, zo bleven de overigen staan.
\par 10 En David vervolgde hen, hij en die vierhonderd mannen; en tweehonderd mannen bleven staan, die zo moede waren, dat zij over de beek Besor niet konden gaan.
\par 11 En zij vonden een Egyptischen man op het veld, en zij brachten hem tot David; en zij gaven hem brood, en hij at, en zij gaven hem water te drinken.
\par 12 Zij gaven hem ook een stuk van een klomp vijgen, en twee stukken rozijnen; en hij at, en zijn geest kwam weder in hem; want hij had in drie dagen en drie nachten geen brood gegeten, noch water gedronken.
\par 13 Daarna zeide David tot hem: Wiens zijt gij? En van waar zijt gij? Toen zeide de Egyptische jongen: Ik ben de knecht van een Amalekietischen man, en mijn heer heeft mij verlaten, omdat ik voor drie dagen krank geworden ben.
\par 14 Wij waren ingevallen tegen het zuiden van de Cherethieten, en op hetgeen van Juda is, en tegen het zuiden van Kaleb; en wij hebben Ziklag met vuur verbrand.
\par 15 Toen zeide David tot hem: Zoudt gij mij wel henen afleiden tot deze bende? Hij dan zeide: Zweer mij bij God, dat gij mij niet zult doden, en dat gij mij niet zult overleveren in de hand mijns heren! Zo zal ik u tot deze bende afleiden.
\par 16 En hij leidde hem af, en ziet, zij lagen verstrooid over de ganse aarde, etende, en drinkende, en dansende, om al den groten buit, dien zij genomen hadden uit het land der Filistijnen, en uit het land van Juda.
\par 17 En David sloeg hen van de schemering tot aan den avond van hunlieder anderen dag; en er ontkwam niet een man van hen, behalve vierhonderd jonge mannen, die op kemelen reden en vloden.
\par 18 Alzo redde David al wat de Amalekieten genomen hadden; ook redde David zijn twee vrouwen.
\par 19 En onder hen werd niet gemist van den kleinste tot aan den grootste, en tot aan de zonen en dochteren; en van den buit, ook tot alles, wat zij van hen genomen hadden; David bracht het altemaal weder.
\par 20 David nam ook al de schapen en de runderen; zij dreven ze voor datzelve vee heen, en zeiden: Dit is Davids buit.
\par 21 Als David tot de tweehonderd mannen kwam, die zo moede waren geweest, dat zij David niet hadden kunnen navolgen, en die zij aan de beek Besor hadden laten blijven, die gingen David tegemoet, en het volk, dat bij hem was, tegemoet; en David trad tot het volk, en hij vraagde hen naar den welstand.
\par 22 Toen antwoordde een ieder boos en Belials man onder de mannen, die met David getogen waren, en zij zeiden: Omdat zij met ons niet getogen zijn, zullen wij hun van den buit, dien wij gered hebben, niet geven, maar aan een iegelijk zijn vrouw en zijn kinderen; laat hen die heenleiden, en weggaan.
\par 23 Maar David zeide: Alzo zult gij niet doen, mijn broeders, met hetgeen ons de HEERE gegeven heeft, en Hij heeft ons bewaard, en heeft de bende, die tegen ons kwam, in onze hand gegeven.
\par 24 Wie zou toch ulieden in deze zaak horen? Want gelijk het deel dergenen is, die in den strijd mede afgetogen zijn, alzo zal ook het deel dergenen zijn, die bij het gereedschap gebleven zijn; zij zullen gelijkelijk delen.
\par 25 En dit is van dien dag af en voortaan alzo geweest; want hij heeft het tot een inzetting en tot een recht gesteld in Israel, tot op dezen dag.
\par 26 Als nu David te Ziklag kwam, zo zond hij tot de oudsten van Juda, zijn vrienden, van den buit, zeggende: Ziet, daar is een zegen voor ulieden, van den buit der vijanden des HEEREN.
\par 27 Namelijk tot die te Beth-el, en tot die te Ramoth tegen het zuiden, en tot die te Jather,
\par 28 En tot die te Aroer, en tot die te Sifmoth, en tot die te Esthemoa,
\par 29 En tot die te Rachel, en tot die, welke in de steden der Jerahmeelieten waren, en tot die, welke in de steden der Kenieten waren,
\par 30 En tot die te Horma, en tot die te Chor-asan, en tot die te Atach,
\par 31 En tot die te Hebron, en tot al de plaatsen, waar David gewandeld had, hij en zijn mannen.

\chapter{31}

\par 1 De Filistijnen dan steden tegen Israel; en de mannen Israels vloden voor het aangezicht der Filistijnen, en vielen verslagen op het gebergte Gilboa.
\par 2 En de Filistijnen hielden dicht op Saul en zijn zonen; en de Filistijnen sloegen Jonathan, en Abinadab, en Malchisua, de zonen van Saul.
\par 3 En de strijd werd zwaar tegen Saul; en de mannen, die met den boog schieten, troffen hem aan, en hij vreesde zeer voor de schutters.
\par 4 Toen zeide Saul tot zijn wapendrager: Trek uw zwaard uit, en doorsteek mij daarmede, dat misschien deze onbesnedenen niet komen, en mij doorsteken, en met mij den spot drijven. Maar zijn wapendrager wilde niet, want hij vreesde zeer. Toen nam Saul het zwaard, en viel daarin.
\par 5 Toen zijn wapendrager zag, dat Saul dood was, zo viel hij ook in zijn zwaard en stierf met hem.
\par 6 Alzo stierf Saul, en zijn drie zonen, en zijn wapendrager, ook al zijn mannen, te dienzelven dage te gelijk.
\par 7 Als de mannen van Israel, die aan deze zijde van het dal waren, en die aan deze zijde der Jordaan waren, zagen, dat de mannen van Israel gevloden waren, en dat Saul en zijn zonen dood waren, zo verlieten zij de steden, en zij vloden. Toen kwamen de Filistijnen en woonden daarin.
\par 8 Het geschiedde nu des anderen daags, als de Filistijnen kwamen, om de verslagenen te plunderen, zo vonden zij Saul en zijn drie zonen, liggende op het gebergte Gilboa.
\par 9 En zij hieuwen zijn hoofd af, en zij togen zijn wapenen uit, en zij zonden ze in der Filistijnen land rondom, om te boodschappen in het huis hunner afgoden, en onder het volk.
\par 10 En zij leiden zijn wapenen in het huis van Astharoth; en zijn lichaam hechtten zij aan den muur te Beth-san.
\par 11 Als de inwoners van Jabes in Gilead daarvan hoorden, wat de Filistijnen Saul gedaan hadden;
\par 12 Zo maakten zich op alle strijdbare mannen, en gingen den gehelen nacht, en zij namen het lichaam van Saul, en de lichamen zijner zonen, van den muur te Beth-san; en zij kwamen te Jabes, en brandden ze aldaar.
\par 13 En zij namen hun beenderen, en begroeven ze onder het geboomte te Jabes; en zij vastten zeven dagen.



\end{document}