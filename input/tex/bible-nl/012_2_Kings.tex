\begin{document}

\title{2 Kings}



\chapter{1}

\par 1 En Moab viel van Israel af, na Achabs dood.
\par 2 En Ahazia viel door een tralie in zijn opperzaal, die te Samaria was, en werd krank. En hij zond boden, en zeide tot hen: Gaat heen, vraagt Baal-zebub, den god van Ekron, of ik van deze krankheid genezen zal.
\par 3 Maar de Engel des HEEREN sprak tot Elia, den Thisbiet: Maak u op, ga op, den boden des konings van Samaria tegemoet, en spreek tot hen: Is het, omdat er geen God in Israel is, dat gijlieden heengaat, om Baal-zebub, den god van Ekron, te vragen?
\par 4 Daarom nu zegt de HEERE alzo: Gij zult niet afkomen van dat bed, waarop gij geklommen zijt, maar gij zult den dood sterven. En Elia ging weg.
\par 5 Zo kwamen de boden weder tot hem; en hij zeide tot hen: Wat is dit, dat gij wederkomt?
\par 6 En zij zeiden tot hem: Een man kwam op, ons tegemoet, en zeide tot ons: Gaat heen, keert weder tot den koning die u gezonden heeft, en spreekt tot hem: Zo zegt de HEERE: Is het, omdat er geen God in Israel is, dat gij zendt, om Baal-zebub, den god van Ekron, te vragen? Daarom zult gij van dat bed, waarop gij geklommen zijt, niet afkomen, maar gij zult den dood sterven.
\par 7 En hij sprak tot hen: Hoedanig was de gestalte des mans, die u tegemoet opgekomen is, en deze woorden tot u gesproken heeft?
\par 8 En zij zeiden tot hem: Hij was een man met een harig kleed, en met een lederen gordel gegord om zijn lenden. Toen zeide hij: Het is Elia, de Thisbiet.
\par 9 En hij zond tot hem een hoofdman van vijftig met zijn vijftigen. En als hij tot hem opkwam (want ziet, hij zat op de hoogte eens bergs), zo sprak hij tot hem: Gij, man Gods! de koning zegt: Kom af.
\par 10 Maar Elia antwoordde en sprak tot den hoofdman van vijftigen: Indien ik dan een man Gods ben, zo dale vuur van den hemel, en vertere u en uw vijftigen. Toen daalde vuur van den hemel, en verteerde hem en zijn vijftigen.
\par 11 En hij zond wederom tot hem een anderen hoofdman van vijftig met zijn vijftigen. Deze antwoordde en sprak tot hem: Gij, man Gods! zo zegt de koning: Kom haastelijk af.
\par 12 En Elia antwoordde en sprak tot hem: Ben ik een man Gods, zo dale vuur van den hemel, en vertere u en uw vijftigen. Toen daalde vuur Gods van den hemel en verteerde hem en zijn vijftigen.
\par 13 En wederom zond hij een hoofdman van de derde vijftigen met zijn vijftigen. Zo ging de derde hoofdman van vijftigen op, en kwam en boog zich op zijn knieen, voor Elia, en smeekte hem, en sprak tot hem: Gij, man Gods, laat toch mijn ziel en de ziel van uw knechten, van deze vijftigen, dierbaar zijn in uw ogen!
\par 14 Zie, het vuur is van den hemel gedaald, en heeft die twee eerste hoofdmannen van vijftigen met hun vijftigen verteerd; maar nu, laat mijn ziel dierbaar zijn in uw ogen!
\par 15 Toen sprak de Engel des HEEREN tot Elia: Ga af met hem; vrees niet voor zijn aangezicht. En hij stond op, en ging met hem af tot den koning.
\par 16 En hij sprak tot hem: Zo zegt de HEERE: Daarom, dat gij boden gezonden hebt, om Baal-zebub, den god van Ekron, te vragen (is het, omdat er geen God in Israel is, om Zijn woord te vragen?); daarom, van dat bed, waarop gij geklommen zijt, zult gij niet afkomen, maar gij zult den dood sterven.
\par 17 Alzo stierf hij, naar het woord des HEEREN, dat Elia gesproken had; en Joram werd koning in zijn plaats, in het tweede jaar van Joram, den zoon van Josafat, den koning van Juda; want hij had geen zoon.
\par 18 Het overige nu der zaken van Ahazia, die hij gedaan heeft, is dat niet geschreven in het boek der kronieken der koningen van Israel?

\chapter{2}

\par 1 Het geschiedde nu, als de HEERE Elia met een onweder ten hemel opnemen zou, dat Elia met Elisa ging van Gilgal.
\par 2 En Elia zeide tot Elisa: Blijf toch hier, want de HEERE heeft mij naar Beth-el gezonden. Maar Elisa zeide: Zo waarachtig als de HEERE leeft en uw ziel leeft, ik zal u niet verlaten! Alzo gingen zij af naar Beth-el.
\par 3 Toen gingen de zonen der profeten, die te Beth-el waren, tot Elisa uit, en zeiden tot hem: Weet gij, dat de HEERE heden uw heer van uw hoofd wegnemen zal? En hij zeide: Ik weet het ook wel, zwijgt gij stil.
\par 4 En Elia zeide tot hem: Elisa, blijf toch hier, want de HEERE heeft mij naar Jericho gezonden. Maar hij zeide: Zo waarachtig als de HEERE leeft en uw ziel leeft, ik zal u niet verlaten! Alzo kwamen zij te Jericho.
\par 5 Toen traden de zonen der profeten, die te Jericho waren, naar Elisa toe, en zeiden tot hem: Weet gij, dat de HEERE heden uw heer van uw hoofd wegnemen zal? En hij zeide: Ik weet het ook wel, zwijgt gij stil.
\par 6 En Elia zeide tot hem: Blijf toch hier, want de HEERE heeft mij naar de Jordaan gezonden. Maar hij zeide: Zo waarachtig als de HEERE leeft en uw ziel leeft, ik zal u niet verlaten! En zij beiden gingen henen.
\par 7 En vijftig mannen van de zonen der profeten gingen henen, en stonden tegenover van verre; en die beiden stonden aan de Jordaan.
\par 8 Toen nam Elia zijn mantel, en wond hem samen, en sloeg het water, en het werd herwaarts en derwaarts verdeeld; en zij beiden gingen er door op het droge.
\par 9 Het geschiedde nu, als zij overgekomen waren, dat Elia zeide tot Elisa: Begeer wat ik u doen zal, eer ik van bij u weggenomen worde. En Elisa zeide: Dat toch twee delen van uw geest op mij zijn!
\par 10 En hij zeide: Gij hebt een harde zaak begeerd; indien gij mij zult zien, als ik van bij u weggenomen worde, het zal u alzo geschieden; doch zo niet, het zal niet geschieden.
\par 11 En het gebeurde, als zij voortgingen, gaande en sprekende, ziet, zo was er een vurige wagen met vurige paarden, die tussen hen beiden scheiding maakten. Alzo voer Elia met een onweder ten hemel.
\par 12 En Elisa zag het, en hij riep: Mijn vader, mijn vader, wagen Israels en zijn ruiteren! En hij zag hem niet meer; en hij vatte zijn klederen en scheurde ze in twee stukken.
\par 13 Hij hief ook Elia's mantel op, die van hem afgevallen was, en keerde weder, en stond aan den oever van de Jordaan.
\par 14 En hij nam den mantel van Elia, die van hem afgevallen was, en sloeg het water, en zeide: Waar is de HEERE, de God van Elia? Ja, Dezelve? En hij sloeg het water, en het werd herwaarts en derwaarts verdeeld, en Elisa ging er door.
\par 15 Als nu de kinderen der profeten, die tegenover te Jericho waren, hem zagen, zo zeiden zij: De geest van Elia rust op Elisa; en zij kwamen hem tegemoet, en bogen zich voor hem neder ter aarde.
\par 16 En zij zeiden tot hem: Zie nu, er zijn bij uw knechten vijftig dappere mannen; laat hen toch heengaan, en uw heer zoeken, of niet misschien de Geest des HEEREN hem opgenomen, en op een der bergen, of in een der dalen hem geworpen heeft. Doch hij zeide: Zendt niet.
\par 17 Maar zij hielden bij hem aan tot schamens toe; en hij zeide: Zendt. En zij zonden vijftig mannen, die drie dagen zochten, doch hem niet vonden.
\par 18 Toen kwamen zij weder tot hem, daar hij te Jericho gebleven was; en hij zeide tot hen: Heb ik tot ulieden niet gezegd: Gaat niet?
\par 19 En de mannen der stad zeiden tot Elisa: Zie toch, de woning dezer stad is goed, gelijk als mijn heer ziet; maar het water is kwaad, en het land onvruchtbaar.
\par 20 En hij zeide: Brengt mij een nieuwe schaal, en legt er zout in. En zij brachten ze tot hem.
\par 21 Toen ging hij uit tot de waterwel, en wierp het zout daarin, en zeide: Zo zegt de HEERE: Ik heb dit water gezond gemaakt, er zal geen dood noch onvruchtbaarheid meer van worden.
\par 22 Alzo werd dat water gezond, tot op dezen dag, naar het woord van Elisa, dat hij gesproken had.
\par 23 En hij ging van daar op naar Beth-el. Als hij nu den weg opging, zo kwamen kleine jongens uit de stad; die bespotten hem, en zeiden tot hem: Kaalkop, ga op, kaalkop, ga op!
\par 24 En hij keerde zich achterom, en hij zag ze, en vloekte hen, in den Naam des HEEREN. Toen kwamen twee beren uit het woud, en verscheurden van dezelve twee en veertig kinderen.
\par 25 En hij ging van daar naar den berg Karmel; en van daar keerde hij weder naar Samaria.

\chapter{3}

\par 1 Joram nu, de zoon van Achab, werd koning over Israel te Samaria, in het achttiende jaar van Josafat, den koning van Juda, en hij regeerde twaalf jaren.
\par 2 En hij deed dat kwaad was in de ogen des HEEREN, doch niet gelijk zijn vader en gelijk zijn moeder; want hij deed dat opgerichte beeld van Baal weg, hetwelk zijn vader gemaakt had.
\par 3 Evenwel hing hij de zonden van Jerobeam, den zoon van Nebat, aan, die Israel deed zondigen; hij week daarvan niet af.
\par 4 Mesa nu, de koning der Moabieten, was een veehandelaar, en bracht op aan den koning van Israel honderd duizend lammeren, en honderd duizend rammen met de wol.
\par 5 Maar het geschiedde, als Achab gestorven was, dat de koning der Moabieten van den koning van Israel afviel.
\par 6 Zo toog de koning Joram ter zelfder tijd uit Samaria, en monsterde gans Israel.
\par 7 En hij ging heen, en zond tot Josafat, den koning van Juda, zeggende: De koning der Moabieten is van mij afgevallen, zult gij met mij trekken in den oorlog tegen de Moabieten? En hij zeide: Ik zal opkomen; zo zal ik zijn, gelijk gij zijt, zo mijn volk als uw volk, zo mijn paarden als uw paarden.
\par 8 En hij zeide: Door welken weg zullen wij optrekken? Hij dan zeide: Door den weg der woestijn van Edom.
\par 9 Alzo toog de koning van Israel heen, en de koning van Juda, en de koning van Edom; en als zij zeven dagreizen omgetogen waren, zo had het leger en het vee, dat hen navolgde, geen water.
\par 10 Toen zeide de koning van Israel: Ach, dat de HEERE deze drie koningen geroepen heeft, om die in der Moabieten hand te geven!
\par 11 En Josafat zeide: Is hier geen profeet des HEEREN, dat wij door hem den HEERE mochten vragen? Toen antwoordde een van de knechten des konings van Israel, en zeide: Hier is Elisa, de zoon van Safat, die water op Elia's handen goot.
\par 12 En Josafat zeide: Des HEEREN woord is bij hem. Zo togen tot hem af de koning van Israel, en Josafat, en de koning van Edom.
\par 13 Maar Elisa zeide tot den koning van Israel: Wat heb ik met u te doen? Ga heen tot de profeten uws vaders, en tot de profeten uwer moeder. Doch de koning van Israel zeide tot hem: Neen, want de HEERE heeft deze drie koningen geroepen, om die in der Moabieten hand te geven.
\par 14 En Elisa zeide: Zo waarachtig als de HEERE der heirscharen leeft, voor Wiens aangezicht ik sta, zo ik niet het aangezicht van Josafat, den koning van Juda, opnam, ik zou u niet aanschouwen, noch u aanzien!
\par 15 Nu dan, brengt mij een speelman. En het geschiedde, als de speelman op de snaren speelde, dat de hand des HEEREN op hem kwam.
\par 16 En hij zeide: Zo zegt de HEERE: Maakt in dit dal vele grachten.
\par 17 Want zo zegt de HEERE: Gijlieden zult geen wind zien, en gij zult geen regen zien; nochtans zal dit dal met water vervuld worden, zodat gij zult drinken, gij en uw vee, en uw beesten.
\par 18 Daartoe is dat slecht in de ogen des HEEREN, Hij zal ook de Moabieten in ulieder hand geven.
\par 19 En gij zult alle vaste steden, en alle uitgelezene steden slaan, en zult alle goede bomen vellen, en zult alle waterfonteinen stoppen; en alle goede stukken lands zult gij met stenen verderven.
\par 20 En het geschiedde des morgens, als men het spijsoffer offert, dat er, ziet, water door den weg van Edom kwam, en het land met water vervuld werd.
\par 21 Toen nu al de Moabieten hoorden, dat koningen opgetogen waren, om tegen hen te strijden, zo werden zij samen geroepen, van al degenen af, die den gordel aangordden en daarboven, en zij stonden aan de landpale.
\par 22 En toen zij zich des morgens vroeg opmaakten, en de zon over dat water oprees, zagen de Moabieten dat water tegenover rood, gelijk bloed.
\par 23 En zij zeiden: Dit is bloed; de koningen hebben voorzeker zich met het zwaard verdorven, en hebben de een den ander verslagen; nu dan aan den buit, gij Moabieten!
\par 24 Maar als zij aan het leger van Israel kwamen, maakten zich de Israelieten op, en sloegen de Moabieten; en zij vloden van hun aangezicht; ja, zij kwamen in het land, slaande ook de Moabieten.
\par 25 De steden nu braken zij af, en een iegelijk wierp zijn steen op alle goede stukken lands, en zij vulden ze, en stopten alle waterfonteinen, en velden alle goede bomen, totdat zij in Kir-hareseth alleen de stenen daarvan lieten overblijven; en de slingeraars omsingelden en sloegen hen.
\par 26 Doch als de koning der Moabieten zag, dat hem de strijd te sterk was, nam hij tot zich zevenhonderd mannen, die het zwaard uittogen, om door te breken tegen den koning van Edom; maar zij konden niet.
\par 27 Toen nam hij zijn eerstgeboren zoon, die in zijn plaats koning zou worden, en offerde hem ten brandoffer op den muur. Daaruit werd een zeer grote toorn in Israel; daarom trokken zij van hem af, en keerden weder in hun land.

\chapter{4}

\par 1 Een vrouw nu uit de vrouwen van de zonen der profeten riep tot Elisa, zeggende: Uw knecht, mijn man, is gestorven, en gij weet, dat uw knecht den HEERE was vrezende; nu is de schuldheer gekomen, om mijn beide kinderen voor zich tot knechten te nemen.
\par 2 En Elisa zeide tot haar: Wat zal ik u doen? Geef mij te kennen, wat gij in het huis hebt. En zij zeide: Uw dienstmaagd heeft niet met al in het huis, dan een kruik met olie.
\par 3 Toen zeide hij: Ga, eis voor u vaten van buiten, van al uw naburen ledige vaten; maak er niet weinig te hebben.
\par 4 Kom dan in, en sluit de deur voor u en voor uw zonen toe; daarna giet in al die vaten, en zet weg, dat vol is.
\par 5 Zo ging zij van hem, en sloot de deur voor zich en voor haar zonen toe; die brachten haar de vaten toe, en zij goot in.
\par 6 En het geschiedde, als die vaten vol waren, dat zij tot haar zoon zeide: Breng mij nog een vat aan; maar hij zeide tot haar: Er is geen vat meer. En de olie stond stil.
\par 7 Toen kwam zij, en gaf het den man Gods te kennen; en hij zeide: Ga heen, verkoop de olie, en betaal uw schuldheer; gij dan met uw zonen, leef bij het overige.
\par 8 Het geschiedde ook op een dag, als Elisa naar Sunem doortrok, dat aldaar een grote vrouw was, dewelke hem aanhield om brood te eten. Voorts geschiedde het, zo dikwijls hij doortrok, week hij daarin, om brood te eten.
\par 9 En zij zeide tot haar man: Zie nu, ik heb gemerkt, dat deze man Gods heilig is, die bij ons altoos doortrekt.
\par 10 Laat ons toch een kleine opperkamer van een wand maken, en laat ons daar voor hem zetten een bed, en tafel, en stoel, en kandelaar; zo zal het geschieden, wanneer hij tot ons komt, dat hij daar inwijke.
\par 11 En het geschiedde op een dag, dat hij daar kwam; en hij week in die opperkamer, en leide zich daar neder.
\par 12 Toen zeide hij tot zijn jongen Gehazi: Roep deze Sunamietische. En als hij ze geroepen had, stond zij voor zijn aangezicht.
\par 13 (Want hij had hem gezegd: Zeg nu tot haar: Zie, gij zijt zorgvuldig voor ons geweest, met al deze zorgvuldigheid; wat is er voor u te doen? Is er iets om voor u te spreken tot den koning, of tot den krijgsoverste? En zij had gezegd: Ik woon in het midden mijns volks.
\par 14 Toen had hij gezegd: Wat is er dan voor haar te doen? En Gehazi had gezegd: Zij heeft toch geen zoon, en haar man is oud.
\par 15 Daarom had hij gezegd: Roep haar. En als hij ze geroepen had, stond zij in de deur.)
\par 16 En hij zeide: Op dezen gezetten tijd, omtrent dezen tijd des levens zult gij een zoon omhelzen. En zij zeide: Neen, mijn heer, gij, man Gods, lieg tegen uw dienstmaagd niet.
\par 17 En de vrouw werd zwanger, en baarde een zoon op dien gezette tijd, omtrent den tijd des levens, dien Elisa tot haar gesproken had.
\par 18 Toen nu het kind groot werd, geschiedde het op een dag, dat het uitging tot zijn vader, tot de maaiers.
\par 19 En het zeide tot zijn vader: Mijn hoofd, mijn hoofd! Hij dan zeide tot een jongen: Draag hem tot zijn moeder.
\par 20 En hij droeg hem, en bracht hem tot zijn moeder. En hij zat op haar knieen tot aan den middag toe; toen stierf hij.
\par 21 En zij ging op, en leide hem op het bed van den man Gods; daarna sloot zij voor hem toe, en ging uit.
\par 22 En zij riep om haar man, en zeide: Zend mij toch een van de jongens, en een van de ezelinnen, dat ik tot den man Gods lope, en wederkome.
\par 23 En hij zeide: Waarom gaat gij heden tot hem? Het is geen nieuwe maan, noch sabbat. En zij zeide: Het zal wel zijn.
\par 24 Toen zadelde zij de ezelin, en zeide tot haar jongen: Drijf, en ga voort; houd mij niet op voort te rijden, tenzij dan dat ik het u zegge.
\par 25 Alzo toog zij heen, en kwam tot den man Gods, tot den berg Karmel. En het geschiedde, als de man Gods haar van tegenover zag, dat hij tot Gehazi, zijn jongen, zeide: Zie, daar is de Sunamietische.
\par 26 Nu loop toch haar tegemoet, en zeg tot haar: Is het wel met u? Is het wel met uw man? Is het wel met uw kind? En zij zeide: Het is wel.
\par 27 Toen zij nu tot den man Gods op den berg kwam, vatte zij zijn voeten. Maar Gehazi trad toe, om haar af te stoten. Doch de man Gods zeide: Laat ze geworden; want haar ziel is in haar bitterlijk bedroefd, en de HEERE heeft het voor mij verborgen, en mij niet verkondigd.
\par 28 En zij zeide: Heb ik een zoon van mijn heer begeerd? Zeide ik niet: Bedrieg mij niet?
\par 29 En hij zeide tot Gehazi: Gord uw lenden, en neem mijn staf in uw hand, en ga henen; zo gij iemand vindt, groet hem niet; en zo u iemand groet, antwoord hem niet; en leg mijn staf op het aangezicht van den jongen.
\par 30 Doch de moeder van den jongen zeide: Zo waarachtig als de HEERE leeft en uw ziel leeft, ik zal u niet verlaten! Hij stond dan op, en volgde haar na.
\par 31 Gehazi nu was voor hun aangezicht doorgegaan; en hij leide den staf op het aangezicht van den jongen; doch er was geen stem, noch opmerking. Zo keerde hij weder hem tegemoet, en bracht hem boodschap, zeggende: De jongen is niet ontwaakt.
\par 32 En toen Elisa in het huis kwam, ziet, zo was de jongen dood, zijnde gelegd op zijn bed.
\par 33 Zo ging hij in, en sloot de deur voor hen beiden toe, en bad tot den HEERE.
\par 34 En hij klom op, en leide zich neder op het kind, en leggende zijn mond op deszelfs mond, en zijn ogen op zijn ogen, en zijn handen op zijn handen, breidde zich over hem uit. En het vlees des kinds werd warm.
\par 35 Daarna kwam hij weder, en wandelde in het huis eens herwaarts, en eens derwaarts, en klom weder op, en breidde zich over hem uit; en de jongen niesde tot zevenmaal toe; daarna deed de jongen zijn ogen open.
\par 36 En hij riep Gehazi, en zeide: Roep deze Sunamietische. En hij riep ze, en zij kwam tot hem; en hij zeide: Neem uw zoon op.
\par 37 Zo kwam zij, en viel voor zijn voeten, en boog zich ter aarde, en zij nam haar zoon op, en ging uit.
\par 38 Als nu Elisa weder te Gilgal kwam, zo was er honger in dat land, en de zonen der profeten zaten voor zijn aangezicht; en hij zeide tot zijn jongen: Zet den groten pot aan, en zied moes voor de zonen der profeten.
\par 39 Toen ging er een uit in het veld, om moeskruiden te lezen, en hij vond een wilden wijnstok, en las daarvan, zijn kleed vol wilde kolokwinten, en kwam, en sneed ze in den moespot; want zij kenden ze niet.
\par 40 Daarna schepten zij voor de mannen op om te eten; en het geschiedde, als zij aten van dat moes, dat zij riepen en zeiden: Man Gods, de dood is in den pot! En zij konden het niet eten.
\par 41 Maar hij zeide: Brengt dan meel; en hij wierp het in den pot; en hij zeide: Schep voor het volk op, dat zij eten. Toen was er niets kwaads in den pot.
\par 42 En er kwam een man van Baal-salisa, en bracht den man Gods broden der eerstelingen, twintig gerstebroden, en groene aren in haar hulzen; en hij zeide: Geef aan het volk, dat zij eten.
\par 43 Doch zijn dienaar zeide: Wat zou ik dat aan honderd mannen voorzetten? En hij zeide: Geef aan het volk, dat zij eten; want alzo zegt de HEERE: Men zal eten en overhouden.
\par 44 Zo zette hij het hun voor, en zij aten, en zij hielden over, naar het woord des HEEREN.

\chapter{5}

\par 1 Naaman nu, de krijgsoverste van den koning van Syrie, was een groot man voor het aangezicht zijns heren, en van hoog aanzien; want door hem had de HEERE den Syriers verlossing gegeven; zo was deze man een strijdbaar held, doch melaats.
\par 2 En er waren benden uit Syrie getogen, en hadden een kleine jonge dochter uit het land van Israel gevankelijk gebracht, die in den dienst der huisvrouw van Naaman was.
\par 3 Deze zeide tot haar vrouw: Och, of mijn heer ware voor het aangezicht van den profeet, die te Samaria is, dan zou hij hem van zijn melaatsheid ontledigen.
\par 4 Toen ging hij in en gaf het zijn heer te kennen, zeggende: Zo en zo heeft de jonge dochter gesproken, die uit het land van Israel is.
\par 5 Toen zeide de koning van Syrie: Ga heen, kom, en ik zal een brief aan den koning van Israel zenden. En hij ging heen, en nam in zijn hand tien talenten zilvers, en zes duizend sikkelen gouds, en tien wisselklederen.
\par 6 En hij bracht den brief tot den koning van Israel, zeggende: Zo wanneer nu deze brief tot u zal gekomen zijn, zie, ik heb mijn knecht Naaman tot u gezonden, dat gij hem ontledigt van zijn melaatsheid.
\par 7 En het geschiedde, als de koning van Israel den brief gelezen had, dat hij zijn klederen scheurde, en zeide: Ben ik dan God, om te doden en levend te maken, dat deze tot mij zendt, om een man van zijn melaatsheid te ontledigen? Want voorwaar, merkt toch, en ziet, dat hij oorzaak tegen mij zoekt.
\par 8 Maar het geschiedde, als Elisa, de man Gods, gehoord had, dat de koning van Israel zijn klederen gescheurd had, dat hij tot den koning zond, om te zeggen: Waarom hebt gij uw klederen gescheurd? Laat hem nu tot mij komen, zo zal hij weten, dat er een profeet in Israel is.
\par 9 Alzo kwam Naaman met zijn paarden en met zijn wagen, en stond voor de deur van het huis van Elisa.
\par 10 Toen zond Elisa tot hem een bode, zeggende: Ga heen en was u zevenmaal in de Jordaan, en uw vlees zal u wederkomen, en gij zult rein zijn.
\par 11 Maar Naaman werd zeer toornig, en toog weg, en zeide: Zie, ik zeide bij mij zelven: Hij zal zekerlijk uitkomen, en staan, en den Naam des HEEREN, Zijns Gods, aanroepen, en zijn hand over de plaats strijken, en den melaatse ontledigen.
\par 12 Zijn niet Abana en Farpar, de rivieren van Damaskus, beter dan alle wateren van Israel; zou ik mij in die niet kunnen wassen en rein worden? Zo wendde hij zich, en toog weg met grimmigheid.
\par 13 Toen traden zijn knechten toe, en spraken tot hem, en zeiden: Mijn vader, zo die profeet tot u een grote zaak gesproken had, zoudt gij ze niet gedaan hebben? Hoeveel te meer, naardien hij tot u gezegd heeft: Was u, en gij zult rein zijn?
\par 14 Zo klom hij af, en doopte zich in de Jordaan zevenmaal, naar het woord van den man Gods; en zijn vlees kwam weder, gelijk het vlees van een kleinen jongen; en hij werd rein.
\par 15 Toen keerde hij weder tot den man Gods, hij en zijn ganse heir, en kwam, en stond voor zijn aangezicht en zeide: Zie, nu weet ik, dat er geen God is op de ganse aarde, dan in Israel! Nu dan, neem toch een zegen van uw knecht.
\par 16 Maar hij zeide: Zo waarachtig als de HEERE leeft, voor Wiens aangezicht ik sta, indien ik het neme! En hij hield bij hem aan, opdat hij het nam, doch hij weigerde het.
\par 17 En Naaman zeide: Zo niet; laat toch uw knecht gegeven worden een last aarde van een juk muildieren; want uw knecht zal niet meer brandoffer of slachtoffer aan andere goden doen, maar den HEERE.
\par 18 In deze zaak vergeve de HEERE uw knecht: wanneer mijn heer in het huis van Rimmon zal gaan, om zich daar neder te buigen, en hij op mijn hand leunen zal en ik mij in het huis van Rimmon nederbuigen zal; als ik mij alzo nederbuigen zal in het huis van Rimmon, de HEERE vergeve toch uw knecht in deze zaak.
\par 19 En hij zeide tot hem: Ga in vrede. En hij ging van hem een kleine streek lands.
\par 20 Gehazi nu, de jongen van Elisa, den man Gods, zeide: Zie, mijn heer heeft Naaman, dien Syrier belet, dat men uit zijn hand niet genomen heeft, wat hij gebracht had; maar zo waarachtig als de HEERE leeft, ik zal hem nalopen, en zal wat van hem nemen!
\par 21 Zo volgde Gehazi Naaman achterna. En toen Naaman zag, dat hij hem naliep, viel hij van den wagen af, hem tegemoet, en hij zeide: Is het wel?
\par 22 En hij zeide: Het is wel; mijn heer heeft mij gezonden, om te zeggen: Zie, nu straks zijn tot mij twee jongelingen uit de zonen der profeten, van het gebergte van Efraim gekomen; geef hun toch een talent zilvers en twee wisselklederen.
\par 23 En Naaman zeide: Belieft het u, neem twee talenten. En hij hield aan bij hem, en bond twee talenten zilvers in twee buidels, met twee wisselklederen, en hij leide ze op twee van zijn jongens, die ze voor zijn aangezicht droegen.
\par 24 Als hij nu op de hoogte kwam, nam hij ze van hun hand, en bestelde ze in een huis; en hij liet de mannen gaan, en zij togen heen.
\par 25 Daarna kwam hij in, en stond voor zijn heer. En Elisa zeide tot hem: Van waar, Gehazi? En hij zeide: Uw knecht is noch herwaarts noch derwaarts gegaan.
\par 26 Maar hij zeide tot hem: Ging niet mijn hart mede, als die man zich omkeerde van op zijn wagen u tegemoet? Was het tijd, om dat zilver te nemen, en om klederen te nemen, en olijfbomen, en wijngaarden, en schapen, en runderen, en knechten, en dienstmaagden?
\par 27 Daarom zal u de melaatsheid van Naaman aankleven, en uw zaad in eeuwigheid! Toen ging hij uit van voor zijn aangezicht, melaats, wit als de sneeuw.

\chapter{6}

\par 1 En de kinderen der profeten zeiden tot Elisa: Zie nu, de plaats, waar wij wonen voor uw aangezicht, is voor ons te eng.
\par 2 Laat ons toch tot aan de Jordaan gaan, en elk van daar een timmerhout halen, dat wij ons daar een plaats maken, om er te wonen. En hij zeide: Gaat heen.
\par 3 En er zeide een: Het believe u toch te gaan met uw knechten. En hij zeide: Ik zal gaan.
\par 4 Zo ging hij met hen. Als zij nu aan de Jordaan gekomen waren, hieuwen zij hout af.
\par 5 En het geschiedde, als een het timmerhout velde, dat het ijzer in het water viel; en hij riep, en zeide: Ach, mijn heer, want het was geleend.
\par 6 En de man Gods zeide: Waar is het gevallen? En toen hij hem de plaats gewezen had, sneed hij een hout af, en wierp het daarhenen, en deed het ijzer boven zwemmen.
\par 7 En hij zeide: Neem het tot u op. Toen stak hij zijn hand uit, en nam het,
\par 8 En de koning van Syrie voerde krijg tegen Israel, en beraadslaagde zich met zijn knechten, zeggende: Mijn legering zal zijn in de plaats van zulk een.
\par 9 Maar de man Gods zond henen tot den koning van Israel, zeggende: Wacht u, dat gij door die plaats niet trekt, want de Syriers zijn daarhenen afgekomen.
\par 10 Daarom zond de koning van Israel henen aan die plaats, waarvan hem de man Gods gezegd en hem gewaarschuwd had, en wachtte zich aldaar, niet eenmaal, noch tweemaal.
\par 11 Toen werd het hart des konings van Syrie onstuimig over dezen handel; en hij riep zijn knechten, en zeide tot hen: Zult gij mij dan niet te kennen geven, wie van de onzen zij voor den koning van Israel?
\par 12 En een van zijn knechten zeide: Neen, mijn heer koning! Maar Elisa, de profeet, die in Israel is, geeft den koning van Israel te kennen de woorden, die gij in uw binnenste slaapkamer spreekt.
\par 13 En hij zeide: Gaat heen, en ziet, waar hij is, dat ik zende en hem halen late. En hem werd te kennen gegeven, zeggende: Zie, hij is te Dothan.
\par 14 Toen zond hij daarhenen paarden, en wagenen, en een zwaar heir; welke des nachts kwamen, en omsingelden de stad.
\par 15 En de dienaar van den man Gods stond zeer vroeg op, en ging uit; en ziet, een heir omringde de stad met paarden en wagenen. Toen zeide zijn jongen tot hem: Ach, mijn heer, hoe zullen wij doen.
\par 16 En hij zeide: Vrees niet; want die bij ons zijn, zijn meer, dan die bij hen zijn.
\par 17 En Elisa bad, en zeide: HEERE, open toch zijn ogen, dat hij zie! En de HEERE opende de ogen van den jongen, dat hij zag; en ziet, de berg was vol vurige paarden en wagenen rondom Elisa.
\par 18 Als zij nu tot hem afkwamen, bad Elisa tot den HEERE, en zeide: Sla toch dit volk met verblindheden. En Hij sloeg hen met verblindheden, naar het woord van Elisa.
\par 19 Toen zeide Elisa tot hen: Dit is de weg niet, en dit is de stad niet; volgt mij na, en ik zal u leiden tot den man, dien gij zoekt; en hij leidde hen naar Samaria.
\par 20 En het geschiedde, als zij te Samaria gekomen waren, dat Elisa zeide: HEERE, open de ogen van dezen, dat zij zien! En de HEERE opende hun ogen, dat zij zagen; en ziet, zij waren in het midden van Samaria.
\par 21 En de koning van Israel zeide tot Elisa, als hij hen zag: Zal ik hen slaan? Zal ik hen slaan, mijn vader?
\par 22 Doch hij zeide: Gij zult hen niet slaan; zoudt gij ook slaan, die gij met uw zwaard en met uw boog gevangen hadt? Zet hun brood en water voor, dat zij eten en drinken, en tot hun heer trekken.
\par 23 En hij bereidde hun een grote maaltijd, dat zij aten en dronken; daarna liet hij hen gaan, en zij trokken tot hun heer. Zo kwamen de benden der Syriers niet meer in het land van Israel.
\par 24 En het geschiedde daarna, dat Benhadad, de koning van Syrie, zijn gehele leger verzamelde, en optoog, en Samaria belegerde.
\par 25 En er werd grote honger in Samaria; want ziet, zij belegerden ze, totdat een ezelskop voor tachtig zilverlingen was verkocht, en een vierendeel van een kab duivenmest voor vijf zilverlingen.
\par 26 En het geschiedde, als de koning op den muur voorbijging, dat een vrouw tot hem riep, zeggende: Help mij, heer koning!
\par 27 En hij zeide: De HEERE helpt u niet; waarvan zou ik u helpen? Van den dorsvloer of van de wijnpers?
\par 28 Verder zeide de koning tot haar: Wat is u? En zij zeide: Deze vrouw heeft tot mij gezegd: Geef uw zoon, dat wij hem heden eten, en morgen zullen wij mijn zoon eten.
\par 29 Zo hebben wij mijn zoon gezoden, en hebben hem gegeten; maar als ik des anderen daags tot haar zeide: Geef uw zoon, dat wij hem eten, zo heeft zij haar zoon verstoken.
\par 30 En het geschiedde, als de koning de woorden dezer vrouw gehoord had, dat hij zijn klederen scheurde, alzo hij op den muur voortging; en het volk zag, dat, ziet, een zak van binnen over zijn vlees was.
\par 31 En hij zeide: Zo doe mij God, en doe zo daartoe, indien het hoofd van Elisa den zoon van Safat, heden op hem zal blijven staan!
\par 32 (Elisa nu zat in zijn huis, en de oudsten zaten bij hem.) En hij zond een man van voor zijn aangezicht; maar eer de bode tot hem gekomen was, had hij gezegd tot de oudsten: Hebt gijlieden gezien, hoe die zoon des moordenaars gezonden heeft, om mijn hoofd af te nemen? Ziet toe, als die bode komt, sluit de deur toe, en dringt hem uit met de deur; is niet het geruis der voeten van zijn heer achter hem?
\par 33 Als hij nog met hen sprak, ziet, zo kwam de bode tot hem af; en hij zeide: Zie, dat kwaad is van den HEERE; wat zou ik verder op den HEERE wachten?

\chapter{7}

\par 1 Toen zeide Elisa: Hoort het woord des HEEREN; zo zegt de HEERE: Morgen omtrent dezen tijd zal een maat meelbloem verkocht worden voor een sikkel, en twee maten gerst voor een sikkel, in de poort van Samaria.
\par 2 Maar een hoofdman, op wiens hand de koning leunde, antwoordde den man Gods, en zeide: Zie, zo de HEERE vensteren in den hemel maakte, zou die zaak kunnen geschieden? En hij zeide: Zie, gij zult het met uw ogen zien, doch daarvan niet eten.
\par 3 Er waren nu vier melaatse mannen voor de deur der poort; die zeiden, de een tot den ander: Wat blijven wij hier, totdat wij sterven?
\par 4 Indien wij zeggen: Laat ons in de stad komen, zo is de honger in de stad, en wij zullen daar sterven, en indien wij hier blijven, wij zullen ook sterven; nu dan, komt, en laat ons in het leger der Syriers vallen; indien zij ons laten leven, wij zullen leven; en indien zij ons doden, wij zullen maar sterven.
\par 5 En zij stonden op in de schemering, om in het leger der Syriers te komen. Toen zij aan het uiterste van het leger der Syriers kwamen, ziet, toen was er niemand.
\par 6 Want de HEERE had het heir der Syriers doen horen een geluid van wagenen, en een geluid van paarden, het geluid ener grote heirkracht; zodat zij zeiden de een tot den ander: Zie, de koning van Israel heeft tegen ons gehuurd de koningen der Hethieten, en de koningen der Egyptenaren, om tegen ons te komen.
\par 7 Derhalve hadden zij zich opgemaakt, en waren in de schemering gevloden, en hadden hun tenten gelaten, en hun paarden, en hun ezelen, het leger gelijk als het was; en waren gevloden om huns levens wil.
\par 8 Als nu deze melaatsen aan het uiterste des legers kwamen, zo gingen zij in een tent, en aten en dronken, en namen van daar zilver, en goud, en klederen, en gingen henen, en verborgen het; daarna keerden zij weder, en kwamen in een andere tent, namen van daar ook, en gingen henen, en verborgen het.
\par 9 Toen zeiden zij, de een tot den ander: Wij doen niet recht; deze dag is een dag van goede boodschap, en wij zwijgen stil. Indien wij vertoeven tot den lichten morgen, zo zal ons de ongerechtigheid vinden; daarom nu, komt, laat ons gaan, en dit aan het huis des konings boodschappen.
\par 10 Zo kwamen zij, en riepen tot den poortier der stad, en boodschapten hun, zeggende: Wij zijn gekomen tot het leger der Syriers, en ziet, niemand was daar, noch eens mensen stem; maar paarden aangebonden, en ezels aangebonden, en tenten, gelijk als zij waren.
\par 11 En hij riep de poortiers; en zij deden de boodschap binnen in het huis des konings.
\par 12 En de koning stond op in den nacht, en zeide tot zijn knechten: Ik zal u nu te kennen geven, wat de Syriers ons gedaan hebben; zij weten, dat wij hongerig zijn; daarom zijn zij uit het leger gegaan, om zich in het veld te versteken, zeggende: Als zij uit de stad gegaan zullen zijn, dan zullen wij hen levend grijpen, en wij zullen in de stad komen.
\par 13 Toen antwoordde een van zijn knechten, en zeide: Dat men toch neme vijf van de overige paarden, die hierbinnen overgebleven zijn (zie, zij zijn als de gehele menigte der Israelieten, die hierbinnen overgebleven zijn; zie, zij zijn als de gehele menigte der Israelieten, die vergaan zijn), laat ons die zenden, en zien.
\par 14 Zij namen dan twee wagenpaarden. En de koning zond het leger der Syriers achterna, zeggende: Gaat henen, en ziet.
\par 15 En zij volgden hen na tot de Jordaan toe; en ziet, de ganse weg was vol van klederen en gereedschap, die de Syriers in hun verhaasten weggeworpen hadden. De boden nu keerden weder, en boodschapten het den koning.
\par 16 Toen ging het volk uit, en beroofde het leger der Syriers; en een maat meelbloem werd verkocht voor een sikkel, en twee maten gerst voor een sikkel, naar het woord des HEEREN.
\par 17 De koning nu had den hoofdman, op wiens hand hij leunde, over die poort gesteld; en het volk vertrad hem in de poort, dat hij stierf, gelijk de man Gods gesproken had, die het sprak, als de koning tot hem afgekomen was.
\par 18 Want het was geschied, gelijk de man Gods gesproken had tot den koning, zeggende: Morgen omtrent dezen tijd zullen twee maten gerst voor een sikkel, en een maat meelbloem voor een sikkel verkocht worden, in de poort van Samaria.
\par 19 En die hoofdman had den man Gods geantwoord en gezegd: Zie, zo de HEERE vensteren in den hemel maakte, zou het ook naar dit woord geschieden kunnen? En hij had gezegd: Zie, gij zult het met uw ogen zien, doch daarvan niet eten.
\par 20 Even alzo geschiedde hem, want het volk vertrad hem in de poort, dat hij stierf.

\chapter{8}

\par 1 Elisa nu had gesproken tot die vrouw, welker zoon hij levend gemaakt had, zeggende: Maak u op, en ga heen, gij en uw huisgezin, en verkeer als vreemdeling, waar gij verkeren kunt; want de HEERE heeft een honger geroepen, die ook in het land zeven jaren komen zal.
\par 2 En de vrouw had zich opgemaakt, en had gedaan naar het woord van den man Gods; want zij was gegaan met haar huisgezin, en had als vreemdeling verkeerd in het land der Filistijnen, zeven jaren.
\par 3 En het geschiedde met het einde der zeven jaren, dat de vrouw uit het land der Filistijnen wederkeerde; en zij ging uit, dat zij tot den koning riep, om haar huis en om haar akker.
\par 4 De koning nu sprak tot Gehazi, den jongen van den man Gods, zeggende: Vertel mij toch al de grote dingen, die Elisa gedaan heeft.
\par 5 En het geschiedde, als hij den koning vertelde, hoe hij een dode had levend gemaakt, ziet, zo riep de vrouw, welker zoon hij levend gemaakt had, tot den koning, om haar huis en om haar akker. Toen zeide Gehazi: Mijn heer koning! Dit is de vrouw, en dit is haar zoon, dien Elisa heeft levend gemaakt.
\par 6 En de koning ondervraagde de vrouw, en zij vertelde het hem. Toen gaf de koning haar een kamerling, zeggende: Doe haar wederhebben alles, wat het hare was, daartoe alle inkomsten des akkers, van den dag af, dat zij het land verlaten heeft, tot nu toe.
\par 7 Daarna kwam Elisa te Damaskus, als Benhadad, de koning van Syrie, krank was; en men boodschapte hem, zeggende: De man Gods is herwaarts gekomen.
\par 8 Toen zeide de koning tot Hazael: Neem een geschenk in uw hand, en ga den man Gods tegemoet; en vraag door hem den HEERE, zeggende: Zal ik van deze krankheid genezen?
\par 9 Zo ging Hazael hem tegemoet, en nam een geschenk in zijn hand, te weten, alle goed van Damaskus, een last van veertig kemelen; en hij kwam, en stond voor zijn aangezicht, en zeide: Uw zoon Benhadad, de koning van Syrie, heeft mij tot u gezonden, om te zeggen: Zal ik van deze krankheid genezen?
\par 10 En Elisa zeide tot hem: Ga, zeg, gij zult ganselijk niet genezen; want de HEERE heeft mij getoond, dat hij den dood sterven zal.
\par 11 En hij hield zijn gezicht staande, en zette het vast tot schamens toe; en de man Gods weende.
\par 12 Toen zeide Hazael: Waarom weent mijn heer? En hij zeide: omdat ik weet, wat kwaad gij den kinderen Israels doen zult; gij zult hun sterkten in het vuur zetten, en hun jonge manschap met het zwaard doden, en hun jonge kinderen verpletteren, en hun zwangere vrouwen opensnijden.
\par 13 En Hazael zeide: Maar wat is uw knecht, die een hond is, dat hij deze grote zaak doen zou? En Elisa zeide: De HEERE heeft mij getoond, dat gij koning zijn zult over Syrie.
\par 14 Zo ging hij weg van Elisa, en kwam tot zijn heer, die tot hem zeide: Wat heeft Elisa tot u gezegd? En hij zeide: Hij heeft tot mij gezegd: Gij zult zekerlijk genezen.
\par 15 En het geschiedde des anderen daags, dat hij een deken nam, en in het water doopte, en over zijn aangezicht uitspreidde, dat hij stierf; en Hazael werd koning in zijn plaats.
\par 16 In het vijfde jaar nu van Joram, den zoon van Achab, den koning van Israel, toen Josafat koning was van Juda, begon Jehoram, de zoon van Josafat, den koning van Juda, te regeren.
\par 17 Hij was twee en dertig jaren oud, toen hij koning werd, en hij regeerde acht jaren te Jeruzalem.
\par 18 En hij wandelde op den weg der koningen van Israel, gelijk als het huis van Achab deed; want de dochter van Achab was hem ter vrouw geworden; en hij deed dat kwaad was in de ogen des HEEREN.
\par 19 Doch de HEERE wilde Juda niet verderven, om Davids Zijns knechts wil; gelijk als Hij hem gezegd had, dat Hij hem te allen tijde voor zijn zonen een lamp zou geven.
\par 20 In zijn dagen vielen de Edomieten van onder het gebied van Juda af, en maakten een koning over zich.
\par 21 Daarom toog Joram over naar Zair, en al de wagenen met hem; en hij maakte zich des nachts op, en sloeg de Edomieten, die rondom hem waren, daartoe de oversten der wagenen; en het volk vlood in zijn hutten.
\par 22 De Edomieten evenwel vielen van onder het gebied van Juda af, tot op dezen dag; toen viel Libna af in denzelfden tijd.
\par 23 Het overige nu der geschiedenissen van Joram, en alles wat hij gedaan heeft, is dat niet geschreven in het boek der kronieken der koningen van Juda?
\par 24 En Joram ontsliep met zijn vaderen, en werd begraven bij zijn vaderen, in de stad Davids; en Ahazia, zijn zoon, werd koning in zijn plaats.
\par 25 In het twaalfde jaar van Joram, den zoon van Achab, den koning van Israel, begon Ahazia, de zoon van Jeroham, den koning van Juda, te regeren.
\par 26 Twee en twintig jaren was Ahazia oud, als hij koning werd, en regeerde een jaar te Jeruzalem; en de naam zijner moeder was Athalia, de dochter van Omri, den koning van Israel.
\par 27 En hij wandelde in den weg van het huis van Achab, en deed dat kwaad was in de ogen des HEEREN, gelijk het huis van Achab; want hij was een schoonzoon van het huis van Achab.
\par 28 En hij toog met Joram, den zoon van Achab, naar den strijd, te Ramoth in Gilead, tegen Hazael, den koning van Syrie; en de Syriers sloegen Joram.
\par 29 Toen keerde Joram, de koning, wederom, opdat hij zich te Jizreel helen liet van de slagen, die hem de Syriers te Rama geslagen hadden, als hij streed tegen Hazael, den koning van Syrie; en Ahazia, de zoon van Jehoram, de koning van Juda, kwam af, om Joram, den zoon van Achab, te Jizreel te bezien, want hij was krank.

\chapter{9}

\par 1 Toen riep de profeet Elisa een van de zonen der profeten, en hij zeide tot hem: Gord uw lenden, en neem deze oliekruik in uw hand, en ga heen naar Ramoth in Gilead.
\par 2 Als gij daar zult gekomen zijn, zo zie, waar Jehu, de zoon van Josafat, den zoon van Nimsi, is; en ga in, en doe hem opstaan uit het midden zijner broederen, en breng hem in een binnenste kamer.
\par 3 En neem de oliekruik, en giet ze uit op zijn hoofd, en zeg: Zo zegt de HEERE: Ik heb u tot koning gezalfd over Israel. Doe daarna de deur open, en vlied, en vertoef niet.
\par 4 Zo ging de jongeling, die jongeling van den profeet, naar Ramoth in Gilead.
\par 5 En toen hij inkwam, ziet, daar zaten de hoofdmannen van het heir, en hij zeide: Ik heb een woord aan u, o hoofdman! En Jehu zeide: Tot wien van ons allen? En hij zeide: Tot u, o hoofdman!
\par 6 Toen stond hij op, en ging in huis; hij dan goot de olie op zijn hoofd, en hij zeide tot hem: Zo zegt de HEERE, de God Israels: Ik heb u gezalfd tot koning over het volk des HEEREN, over Israel.
\par 7 En gij zult het huis van Achab, uw heer, slaan, opdat Ik het bloed van Mijn knechten, de profeten, en het bloed van alle knechten des HEEREN, wreke van de hand van Izebel.
\par 8 En het ganse huis van Achab zal omkomen; en Ik zal van Achab uitroeien, wat mannelijk is, ook den beslotene en verlatene in Israel.
\par 9 Want Ik zal het huis van Achab maken als het huis van Jerobeam, den zoon van Nebat, en als het huis van Baesa, den zoon van Ahia.
\par 10 Ook zullen de honden Izebel eten op het stuk lands van Jizreel, en er zal niemand zijn, die haar begrave. Toen deed hij de deur open en vlood.
\par 11 En als Jehu uitging tot de knechten zijns heren, zeide men tot hem: Is het al wel? Waarom is deze onzinnige tot u gekomen? En hij zeide tot hen: Gij kent den man en zijn spraak.
\par 12 Maar zij zeiden: Het is leugen; geef het ons nu te kennen. En hij zeide: Zo en zo heeft hij tot mij gesproken, zeggende: Zo zegt de HEERE: Ik heb u gezalfd tot koning over Israel.
\par 13 Toen haastten zij zich, en een iegelijk nam zijn kleed, en leide het onder hem, op den hoogsten trap; en zij bliezen met de bazuin, en zeiden: Jehu is koning geworden!
\par 14 Alzo maakte Jehu, de zoon van Josafat, den zoon van Nimsi, een verbintenis tegen Joram. (Joram nu had Ramoth in Gilead bewaard, hij en gans Israel, uit oorzake van Hazael, den koning van Syrie;
\par 15 Maar de koning Joram was wedergekeerd, opdat hij zich te Jizreel helen liet van de slagen, die hem de Syriers geslagen hadden, als hij streed tegen Hazael, den koning van Syrie.) En Jehu zeide: Zo het ulieder wil is, laat niemand van de stad uittrekken, die ontkome, om dit in Jizreel te gaan verkondigen.
\par 16 Toen reed Jehu, en toog naar Jizreel; want Joram lag aldaar; en Ahazia, de koning van Juda, was afgekomen, om Joram te bezien.
\par 17 De wachter nu stond op den toren te Jizreel, en zag den hoop van Jehu, als hij aankwam, en zeide: Ik zie een hoop. Toen zeide Joram: Neem een ruiter, en zend dien hunlieden tegemoet, en dat hij zegge: Is het vrede?
\par 18 En de ruiter te paard toog heen hem tegemoet, en zeide: Zo zegt de koning: Is het vrede? En Jehu zeide: Wat hebt gij met den vrede te doen? Keer om naar achter mij. En de wachter gaf het te kennen, zeggende: De bode is tot hen gekomen, maar hij komt niet weder.
\par 19 Toen zond hij een anderen ruiter te paard; en als deze tot hen gekomen was, zeide hij: Zo zegt de koning: Is het vrede? En Jehu zeide: Wat hebt gij met den vrede te doen? Keer om naar achter mij.
\par 20 En de wachter gaf dit te kennen, zeggende: Hij is tot aan hen gekomen, maar hij komt niet weder; en het drijven is als het drijven van Jehu, den zoon van Nimsi, want hij drijft onzinniglijk.
\par 21 Toen zeide Joram: Span aan. En men spande zijn wagen aan. Zo toog Joram, de koning van Israel, uit, en Ahazia, de koning van Juda, een ieder op zijn wagen; en zij togen uit Jehu tegemoet, en vonden hem op het stuk lands van Naboth, den Jizreeliet.
\par 22 Het geschiedde nu, als Joram Jehu zag, dat hij zeide: Is het ook vrede, Jehu? Maar hij zeide: Wat vrede, zo lang als de hoererijen van uw moeder Izebel, en haar toverijen zo vele zijn?
\par 23 Toen keerde Joram zijn hand, en vlood, en zeide tot Ahazia: Het is bedrog, Ahazia!
\par 24 Maar Jehu spande den boog met volle kracht, en schoot Joram tussen zijn armen, dat de pijl door zijn hart uitging; en hij kromde zich in zijn wagen.
\par 25 Toen zeide Jehu tot Bidkar, zijn hoofdman: Neem, werp hem op dat stuk lands van Naboth, den Jizreeliet; want gedenk, als ik en gij nevens elkander achter zijn vader Achab reden, dat hem de HEERE dezen last opleide, zeggende:
\par 26 Zo Ik gisteravond niet gezien heb het bloed van Naboth, en het bloed zijner zonen, zegt de HEERE, en Ik u dat niet vergelde op dit stuk lands, zegt de HEERE. Nu dan, neem, werp hem op dat stuk lands, naar het woord des HEEREN.
\par 27 Als Ahazia, de koning van Juda, dat zag, zo vlood hij door den weg van het huis des hofs; doch Jehu vervolgde hem achterna, en zeide: Slaat hem ook op den wagen, aan den opgang naar Gur, die bij Jibleam is; en hij vlood naar Megiddo, en stierf aldaar.
\par 28 En zijn knechten voerden hem naar Jeruzalem, en zij begroeven hem in zijn graf, bij zijn vaderen in de stad Davids.
\par 29 In het elfde jaar nu van Joram, den zoon van Achab, was Ahazia koning geworden over Juda.
\par 30 En Jehu kwam te Jizreel. Als Izebel dat hoorde, zo blankette zij haar aangezicht, en versierde haar hoofd, en keek ten venster uit.
\par 31 Toen nu Jehu ter poorte inkwam, zeide zij: Is het wel, o Zimri, doodslager van zijn heer?
\par 32 En hij hief zijn aangezicht op naar het venster, en zeide: Wie is met mij? Wie? Toen zagen op hem twee, drie kamerlingen.
\par 33 En hij zeide: Stoot ze van boven neder. En zij stieten haar van boven neder, zodat van haar bloed aan den wand en aan de paarden gesprengd werd; en hij vertrad haar.
\par 34 Als hij nu ingekomen was, en gegeten en gedronken had, zeide hij: Ziet nu naar die vervloekte, en begraaf ze; want zij is eens konings dochter.
\par 35 En zij gingen heen om haar te begraven; doch zij vonden niet van haar, dan het bekkeneel, en de voeten, en de palmen harer handen.
\par 36 Toen kwamen zij weder, en gaven het hem te kennen, en hij zeide: Dit is het woord des HEEREN, dat Hij gesproken heeft door den dienst van Zijn knecht Elia, den Thisbiet, zeggende: Op het stuk lands van Jizreel zullen de honden het vlees van Izebel eten.
\par 37 En het dode lichaam van Izebel zal zijn gelijk mest op het veld, in het stuk lands van Jizreel, dat men niet zal kunnen zeggen: Dit is Izebel.

\chapter{10}

\par 1 Achab nu had zeventig zonen te Samaria; en Jehu schreef brieven, dewelke hij zond naar Samaria, tot de oversten van Jizreel, de oudsten, en tot de voedsterheren van Achab, zeggende:
\par 2 Zo wanneer nu deze brief tot u zal gekomen zijn, dewijl de zonen van uw heer bij u zijn, ook de wagenen en de paarden bij u zijn, mitsgaders een vaste stad, en wapenen;
\par 3 Zo ziet naar den beste en gerechtigste van de zonen uws heren, zet dien op zijns vaders troon; en strijdt voor het huis uws heren.
\par 4 Doch zij vreesden gans zeer, en zeiden: Ziet, twee koningen bestonden niet voor zijn aangezicht, hoe zouden wij dan bestaan?
\par 5 Die dan over het huis was, en die over de stad was, en de oudsten, en de voedsterheren zonden tot Jehu, zeggende: Wij zijn uw knechten, en al wat gij tot ons zeggen zult, zullen wij doen; wij zullen niemand koning maken; doe wat goed is in uw ogen.
\par 6 Toen schreef hij ten tweeden male tot hen een brief, zeggende: Zo gij mijn zijt, en gij naar mijn stem hoort, neemt de hoofden van de mannen, de zonen uws heren, en komt tot mij morgen omtrent dezen tijd naar Jizreel. (De zonen nu de konings, zeventig mannen, waren bij de groten stad, die hen opvoedden.)
\par 7 Het geschiedde dan, als die brief tot hen kwam, dat zij de zonen des konings namen, en zeventig mannen sloegen; en zij leiden hun hoofden in korven, die zij tot hem zonden naar Jizreel.
\par 8 En er kwam een bode, en boodschapte hem, zeggende: Zij hebben de hoofden van de zonen des konings gebracht. En hij zeide: Legt ze in twee hopen, aan de deur der poort, tot morgen.
\par 9 En het geschiedde des morgens, toen hij uitging, dat hij stil stond, en tot al het volk zeide: Gij zijt rechtvaardig. Ziet, ik heb een verbintenis gemaakt tegen mijn heer, en heb hem doodgeslagen; en wie heeft alle deze geslagen?
\par 10 Weet nu, dat niets van het woord des HEEREN, hetwelk de HEERE tegen het huis van Achab gesproken heeft, zal op de aarde vallen; want de HEERE heeft gedaan, wat Hij door den dienst van Zijn knecht Elia gesproken heeft.
\par 11 Daartoe sloeg Jehu al de overgeblevenen van het huis van Achab te Jizreel, en al zijn groten, en zijn bekenden, en zijn priesteren; totdat hij hem geen overigen liet overblijven.
\par 12 En hij maakte zich op, en toog heen en ging naar Samaria; en zijnde te Beth-heked der herderen, op den weg,
\par 13 Vond Jehu de broederen van Ahazia, den koning van Juda, en hij zeide: Wie zijt gijlieden? En zij zeiden: Wij zijn de broederen van Ahazia, en zijn afgekomen, om de zonen des konings en de zonen der koningin te groeten.
\par 14 Toen zeide hij: Grijpt hen levend. En zij grepen hen levend; en zij sloegen hen bij den bornput van Beth-heked, twee en veertig mannen, en hij liet niet een van hen over.
\par 15 En van daar gegaan zijnde, zo vond hij Jonadab, den zoon van Rechab, hem tegemoet; die hem groette; en hij zeide tot hem: Is uw hart recht, gelijk als mijn hart met uw hart is? En Jonadab zeide: Het is, ja, het is; geef uw hand. En hij gaf zijn hand, en hij deed hem tot zich op den wagen klimmen.
\par 16 En hij zeide: Ga met mij, en zie mijn ijver aan voor den HEERE. Zo deden zij hem rijden op zijn wagen.
\par 17 En toen hij te Samaria kwam, sloeg hij allen, die aan Achab te Samaria overgebleven waren, totdat hij hem verdelgd had, naar het woord des HEEREN, dat Hij tot Elia gesproken had.
\par 18 En Jehu verzamelde al het volk, en zeide tot hen: Achab heeft Baal een weinig gediend; Jehu zal hem veel dienen.
\par 19 Nu daarom roept alle profeten van Baal, al zijn dienaren, en al zijn priesteren tot mij, dat niemand gemist worde; want ik heb een grote offerande aan Baal; al wie gemist wordt, zal niet leven. Doch Jehu deed dat door listigheid, opdat hij de dienaren van Baal ombracht.
\par 20 Verder zeide Jehu: Heiligt Baal een verbods dag. en zij riepen dien uit.
\par 21 Ook zond Jehu in het ganse Israel; en alle Baalsdienaren kwamen, dat niet een man overbleef, die niet kwam; en zij kwamen in het huis van Baal, dat het huis van Baal vervuld werd van het ene einde tot het andere einde.
\par 22 Toen zeide hij tot dengene, die over het klederhuis was: Breng voor alle dienaren van Baal de kleding uit. En hij bracht voor hen de kleding uit.
\par 23 En Jehu kwam met Jonadab, den zoon van Rechab, in het huis van Baal; en hij zeide tot de dienaren van Baal: Onderzoekt, en ziet toe, dat hier misschien bij u niemand zij van de dienaren des HEEREN, maar van de dienaren van Baal alleen.
\par 24 Toen zij nu inkwamen, om slachtofferen en brandofferen te doen, bestelde zich Jehu daarbuiten tachtig mannen, en hij zeide: Zo iemand van de mannen, die ik in uw handen gebracht heb, ontkomt, zijn ziel zal voor deszelfs ziel zijn.
\par 25 En het geschiedde, als hij voleind had het brandoffer te doen, dat Jehu zeide tot de trawanten en tot de hoofdmannen: Komt in, slaat hen, dat niemand uitkome. En zij sloegen hen met de scherpte des zwaard; en de trawanten en hoofdmannen wierpen hen weg; daarna kwamen zij tot de stad in het huis van Baal;
\par 26 En zij brachten de opgerichte beelden uit het huis van Baal, en verbrandden ze.
\par 27 Zij braken ook het opgerichte beeld van Baal af; daartoe braken zij het huis van Baal af, en maakten dat tot heimelijke gemakken, tot op dezen dag.
\par 28 Alzo verdelgde Jehu Baal uit Israel.
\par 29 Maar van de zonden van Jerobeam, den zoon van Nebat, die Israel zondigen deed, na te volgen, week Jehu niet af, te weten, van de gouden kalveren, die te Beth-el en die te Dan waren.
\par 30 De HEERE dan zeide tot Jehu: Daarom dat gij welgedaan hebt, doende wat recht is in Mijn ogen, en hebt aan het huis van Achab gedaan, naar alles, wat in Mijn hart was, zullen u zonen tot het vierde gelid op den troon van Israel zitten.
\par 31 Maar Jehu nam niet waar te wandelen in de wet des HEEREN, des Gods van Israel, met zijn ganse hart; hij week niet van de zonden van Jerobeam, die Israel zondigen deed.
\par 32 In die dagen begon de HEERE Israel af te korten, want Hazael sloeg ze in alle landpalen van Israel:
\par 33 Van de Jordaan af, tegen den opgang der zon, het ganse land van Gilead, der Gadieten, en der Rubenieten, en der Manassieten; van Aroer, dat aan de beek van Arnon is, en Gilead, en Basan.
\par 34 Het overige nu der geschiedenissen van Jehu, en al wat hij gedaan heeft, en al zijn macht, zijn die niet geschreven in het boek der kronieken der koningen van Israel?
\par 35 En Jehu ontsliep met zijn vaderen, en zij begroeven hem te Samaria, en zijn zoon Joahaz werd koning in zijn plaats.
\par 36 En de dagen, die Jehu over Israel geregeerd heeft in Samaria, zijn acht en twintig jaren.

\chapter{11}

\par 1 Toen nu Athalia, de moeder van Ahazia, zag, dat haar zoon dood was, zo maakte zij zich op, en bracht al het koninklijke zaad om.
\par 2 Maar Joseba, de dochter van den koning Joram, de zuster van Ahazia, nam Joas, den zoon van Ahazia, en stal hem uit het midden van des konings zonen, die gedood werden, zettende hem en zijn voedster in een slaapkamer; en zij verborgen hem voor Athalia, dat hij niet gedood werd.
\par 3 En hij was met haar verstoken in het huis des HEEREN zes jaren; en Athalia regeerde over het land.
\par 4 In het zevende jaar nu zond Jojada, en nam de oversten van honderd met de hoofdmannen, en met de trawanten, en hij bracht hen tot zich, in het huis des HEEREN; en hij maakte een verbond met hen, en hij beedigde hen in het huis des HEEREN, en hij toonde hun den zoon des konings.
\par 5 En hij gebood hun, zeggende: Dit is de zaak, die gij doen zult: een derde deel van u, die op den sabbat ingaan, zullen de wacht waarnemen van het huis des konings;
\par 6 En een derde deel zal zijn aan de poort Sur; en een derde deel aan de poort achter de trawanten; zo zult gij waarnemen de wacht van dit huis, tegen inbreking.
\par 7 En de twee delen van ulieden, allen, die op den sabbat uitgaan, zullen de wacht van het huis des HEEREN waarnemen bij den koning.
\par 8 En gij zult den koning rondom omsingelen, een ieder met zijn wapenen in zijn hand, en hij, die tussen de ordeningen intreedt, zal gedood worden; en zijt gij bij den koning, als hij uitgaat, en als hij inkomt.
\par 9 De oversten dan van honderd deden naar al wat de priester Jojada geboden had, en namen ieder zijn mannen, die op den sabbat ingingen, met degenen, die op den sabbat uitgingen; en zij kwamen tot den priester Jojada.
\par 10 En de priester gaf aan de oversten van honderd de spiesen en de schilden, die van den koning David geweest waren, die in het huis des HEEREN geweest waren.
\par 11 En de trawanten stonden, ieder met zijn wapenen in zijn hand, van de rechterzijde van het huis, tot de linkerzijde van het huis, naar het altaar en naar het huis toe, bij den koning rondom.
\par 12 Daarna bracht hij des konings zoon voor, en zette hem de kroon op, en gaf hem de getuigenis; en zij maakten hem koning, en zalfden hem; daartoe klapten zij met de handen, en zeiden: De koning leve!
\par 13 Toen Athalia hoorde de stem der trawanten en des volks, zo kwam zij tot het volk in het huis des HEEREN.
\par 14 En zij zag toe, en ziet, de koning stond bij den pilaar, naar de wijze, en de oversten en de trompetten bij den koning; en al het volk des lands was blijde, en blies met trompetten. Toen verscheurde Athalia haar klederen, en zij riep: Verraad, verraad!
\par 15 Maar de priester Jojada gebood aan de oversten van honderd, die over het heir gesteld waren, en zeide tot hen: Brengt haar uit tot buiten de ordeningen, en doodt, wie haar volgt, met het zwaard; want de priester had gezegd: Laat ze in het huis des HEEREN niet gedood worden.
\par 16 En zij leiden de handen aan haar; en zij ging den weg van den ingang der paarden naar het huis des konings, en zij werd daar gedood.
\par 17 En Jojada maakte een verbond tussen den HEERE en tussen den koning, en tussen het volk, dat het den HEERE tot een volk zou zijn; mitsgaders tussen den koning en tussen het volk.
\par 18 Daarna ging al het volk des lands in het huis van Baal, en braken dat af; zijn altaren en zijn beelden verbraken zij recht wel; en Mattan, den priester van Baal, sloegen zij dood voor de altaren. De priester nu bestelde de ambten in het huis des HEEREN.
\par 19 En hij nam de oversten van honderd, en de hoofdmannen, en de trawanten, en al het volk des lands; en zij brachten den koning af uit het huis des HEEREN, en kwamen door den weg van de poort der trawanten tot het huis des konings, en hij zat op den troon der koningen.
\par 20 En al het volk des lands was blijde, en de stad werd stil, nadat zij Athalia met het zwaard gedood hadden bij des konings huis.
\par 21 Joas was zeven jaren oud, toen hij koning werd.

\chapter{12}

\par 1 In het zevende jaar van Jehu werd Joas koning, en regeerde veertig jaren te Jeruzalem; en de naam zijner moeder was Zibja van Ber-seba.
\par 2 En Joas deed dat recht was in de ogen des HEEREN, al zijn dagen, in dewelke de priester Jojada hem onderwees.
\par 3 Alleenlijk werden de hoogten niet weggenomen; het volk offerde en rookte nog op de hoogten.
\par 4 En Joas zeide tot de priesteren: Al het geld der geheiligde dingen, dat gebracht zal worden in het huis des HEEREN, te weten het geld desgenen, die overgaat tot de getelden, het geld van een ieder der personen naar zijn schatting, en al het geld, dat in ieders hart komt, om dat te brengen in het huis des HEEREN,
\par 5 Zullen de priesters tot zich nemen, een ieder van zijn bekende; en zij zullen de breuken van het huis verbeteren, naar alles wat er voor breuk bevonden zal worden.
\par 6 Maar het geschiedde in het drie en twintigste jaar van den koning Joas, dat de priesters de breuken van het huis niet gebeterd hadden.
\par 7 Toen riep de koning Joas den priester Jojada en de andere priesteren, en zeide tot hen: Waarom betert gijlieden niet de breuken van het huis? Nu dan, neemt geen geld van uw bekenden, dat gij het zoudt geven voor de breuken van het huis.
\par 8 En de priesters bewilligden van het volk geen geld te nemen, noch de breuken van het huis te verbeteren.
\par 9 Maar de priester Jojada nam een kist, en boorde een gat in haar deksel, en zette die bij het altaar ter rechterhand, als iemand inkwam in het huis des HEEREN; en de priesters, die den dorpel bewaarden, staken daarin al het geld, dat ten huize des HEEREN gebracht werd.
\par 10 Het geschiedde nu, als zij zagen, dat veel gelds in de kist was, dat des konings schrijver met den hogepriester opkwam, en zij bonden het samen, en telden het geld, dat in het huis des HEEREN gevonden werd.
\par 11 En zij gaven het geld wel gewogen in handen der verzorgers van dat werk, die gesteld waren over het huis des HEEREN; en zij besteedden het uit aan de timmerlieden en aan de bouwlieden, die het huis des HEEREN vermaakten;
\par 12 En aan de metselaren, en aan de steenhouwers, en om hout en gehouwen stenen te kopen, om de breuken van het huis des HEEREN te verbeteren, en voor al wat uitgegeven werd voor het huis, om dat te beteren.
\par 13 Evenwel werden niet gemaakt voor het huis des HEEREN zilveren schalen, gaffelen, sprengbekkens, trompetten, noch enig gouden vat, of zilveren vat, van het geld, dat ten huize des HEEREN gebracht werd.
\par 14 Maar zij gaven dat aan degenen, die het werk deden; en zij verbeterden daarmede het huis des HEEREN.
\par 15 Daartoe eisten zij geen rekening van de mannen, wien zij dat geld in hun handen gaven, om aan degenen, die het werk deden, te geven; want zij handelden trouwelijk.
\par 16 Het geld van schuldoffer, en het geld van zondofferen werd ten huize des HEEREN niet gebracht; het was voor de priesteren.
\par 17 Toen trok Hazael, de koning van Syrie op, en krijgde tegen Gath, en nam haar in; daarna stelde Hazael zijn aangezicht, om tegen Jeruzalem op te trekken.
\par 18 Maar Joas, de koning van Juda, nam al de geheiligde dingen, die Josafat, en Joram, en Ahazia, zijn vaderen, de koningen van Juda, geheiligd hadden, en zijn geheiligde dingen, en al het goud, dat gevonden werd in de schatten van het huis des HEEREN, en van het huis des konings, en zond het tot Hazael, den koning van Syrie; toen trok hij op van Jeruzalem.
\par 19 Het overige nu der geschiedenissen van Joas, en al wat hij gedaan heeft, is dat niet geschreven in het boek der kronieken der koningen van Juda?
\par 20 En zijn knechten stonden op, en maakten een verbintenis, en sloegen Joas, in het huis van Millo, dat afgaat naar Silla;
\par 21 Want Jozacar, de zoon van Simeath, en Jozabad, de zoon van Somer, zijn knechten, sloegen hem, dat hij stierf; en zij begroeven hem met zijn vaderen in de stad Davids; en Amazia, zijn zoon, werd koning in zijn plaats.

\chapter{13}

\par 1 In het drie en twintigste jaar van Joas, den zoon van Ahazia, den koning van Juda, werd Joahaz, de zoon van Jehu, koning over Israel, te Samaria, en regeerde zeventien jaren.
\par 2 En hij deed dat kwaad was in de ogen des HEEREN; want hij wandelde na de zonden van Jerobeam, den zoon van Nebat, die Israel zondigen deed; hij week daarvan niet af.
\par 3 Daarom ontstak des HEEREN toorn tegen Israel; en Hij gaf hen in de hand van Hazael, den koning van Syrie, en in de hand van Benhadad, den zoon van Hazael, al die dagen.
\par 4 Doch Joahaz bad des HEEREN aangezicht ernstelijk aan; en de HEERE verhoorde hem; want Hij zag de verdrukking van Israel, dat de koning van Syrie hen verdrukte.
\par 5 (Zo gaf de HEERE Israel een verlosser, dat zij van onder de hand der Syriers uitkwamen; en de kinderen Israels woonden in hun tenten, als te voren.
\par 6 Nochtans weken zij niet af van de zonden van het huis van Jerobeam, die Israel zondigen deed; maar hij wandelde daarin; en het bos bleef ook staan te Samaria.)
\par 7 Want hij had Joahaz geen volk laten overblijven dan vijftig ruiteren en tien wagenen, en tien duizend voetvolks; want de koning van Syrie had hen omgebracht, en had hen dorsende gemaakt als stof.
\par 8 Het overige nu der geschiedenissen van Joahaz, en al wat hij gedaan heeft, en zijn macht, zijn die niet geschreven in het boek der kronieken der koningen van Israel?
\par 9 En Joahaz ontsliep met zijn vaderen, en zij begroeven hem te Samaria; en Joas, zijn zoon, regeerde in zijn plaats.
\par 10 In het zeven en dertigste jaar van Joas, den koning van Juda, werd Joas, de zoon van Joahaz, koning over Israel, te Samaria, en regeerde zestien jaren.
\par 11 En hij deed dat kwaad was in de ogen des HEEREN; hij week niet af van al de zonden van Jerobeam, den zoon van Nebat, die Israel zondigen deed, maar hij wandelde daarin.
\par 12 Het overige nu der geschiedenissen van Joas, en al wat hij gedaan heeft, en zijn macht, waarmede hij gestreden heeft tegen Amazia, den koning van Juda, zijn die niet geschreven in het boek der kronieken der koningen van Israel?
\par 13 En Joas ontsliep met zijn vaderen, en Jerobeam zat op zijn troon. En Joas werd begraven te Samaria, bij de koningen van Israel.
\par 14 Elisa nu was krank geweest van zijn krankheid, van dewelke hij stierf; en Joas, de koning van Israel, was tot hem afgekomen, en had geweend over zijn aangezicht, en gezegd: Mijn vader, mijn vader, wagen Israels en zijn ruiteren!
\par 15 En Elisa zeide tot hem: Neem een boog en pijlen. En hij nam tot zich een boog en pijlen.
\par 16 En hij zeide tot den koning van Israel: Leg uw hand aan den boog, en hij leide zijn hand daaraan; en Elisa leide zijn handen op des konings handen.
\par 17 En hij zeide: Doe het venster open tegen het oosten. En hij deed het open. Toen zeide Elisa: Schiet. En hij schoot. En hij zeide: Het is een pijl der verlossing des HEEREN, en een pijl der verlossing tegen de Syriers; want gij zult de Syriers slaan in Afek, tot verdoens toe.
\par 18 Daarna zeide hij: Neem de pijlen. En hij nam ze. Toen zeide hij tot den koning van Israel: Sla tegen de aarde. En hij sloeg driemaal; daarna stond hij stil.
\par 19 Toen werd de man Gods zeer toornig op hem, en zeide: Gij zoudt vijf maal of zesmaal geslagen hebben; dan zoudt gij de Syriers tot verdoens toe geslagen hebben; doch nu zult gij de Syriers driemaal slaan.
\par 20 Daarna stierf Elisa, en zij begroeven hem. De benden nu der Moabieten kwamen in het land met het ingaan des jaars.
\par 21 En het geschiedde, als zij een man begroeven, dat zij, ziet, een bende zagen; zo wierpen zij den man in het graf van Elisa; en toen de man daarin kwam, en het gebeente van Elisa aanroerde, werd hij levend, en rees op zijn voeten.
\par 22 Hazael nu, de koning van Syrie, verdrukte Israel, al de dagen van Joahaz.
\par 23 Doch de HEERE was hun genadig, en ontfermde Zich hunner, en wendde Zich tot hen, om Zijns verbonds wil met Abraham, Izak en Jakob; en Hij wilde hen niet verderven, en heeft hen niet verworpen van Zijn aangezicht, tot nu toe.
\par 24 En Hazael, de koning van Syrie, stierf, en zijn zoon Benhadad werd koning in zijn plaats.
\par 25 Joas nu, de zoon van Joahaz, nam de steden weder in, uit de hand van Benhadad, den zoon van Hazael, die hij uit de hand van Joahaz, zijn vader, met krijg genomen had; Joas sloeg hem driemaal, en bracht de steden aan Israel weder.

\chapter{14}

\par 1 In het tweede jaar van Joas, den zoon van Joahaz, den koning van Israel, werd Amazia koning, de zoon van Joas, den koning van Juda.
\par 2 Vijf en twintig jaren was hij oud, toen hij koning werd, en regeerde negen en twintig jaren te Jeruzalem; en de naam zijner moeder was Joaddan van Jeruzalem.
\par 3 En hij deed dat recht was in de ogen des HEEREN, nochtans niet als zijn vader David; hij deed naar alles, wat zijn vader Joas gedaan had.
\par 4 Alleenlijk werden de hoogten niet weggenomen; het volk offerde en rookte nog op de hoogten.
\par 5 Het geschiedde nu, als het koninkrijk in zijn hand versterkt was, dat hij zijn knechten sloeg, die den koning, zijn vader, geslagen hadden,
\par 6 Doch de kinderen der doodslagers doodde hij niet; gelijk geschreven is in het wetboek van Mozes, waar de HEERE geboden heeft, zeggende: De vaders zullen voor de kinderen niet gedood worden, en de kinderen zullen voor de vaders niet gedood worden; maar een ieder zal om zijn zonde gedood worden.
\par 7 Hij sloeg de Edomieten in het Zoutdal tien duizend, en nam Sela in met krijg, en noemde haar naam Jokteel, tot op dezen dag.
\par 8 Toen zond Amazia boden tot Joas, den zoon van Joahaz, den zoon van Jehu, den koning van Israel, zeggende: Kom, laat ons elkanders aangezicht zien.
\par 9 Maar Joas, de koning van Israel, zond tot Amazia, den koning van Juda, zeggende: De distel, die op den Libanon is, zond tot den ceder, die op den Libanon is, zeggende: Geef uw dochter mijn zoon ter vrouw; maar het gedierte des velds, dat op den Libanon is, ging voorbij, en vertrad den distel.
\par 10 Gij hebt de Edomieten dapper geslagen, daarom heeft uw hart u verheven; heb de eer, en blijf in uw huis; want waarom zoudt gij u in het kwade mengen, dat gij vallen zoudt, gij en Juda met u?
\par 11 Doch Amazia hoorde niet; daarom toog Joas, de koning van Israel, op, zodat hij en Amazia, de koning van Juda, elkanders aangezicht zagen te Beth-semes, dat in Juda is.
\par 12 En Juda werd geslagen voor het aangezicht van Israel, en zij vloden, een iegelijk in zijn tenten.
\par 13 En Joas, de koning van Israel, greep Amazia, den koning van Juda, den zoon van Joas, den zoon van Ahazia, te Beth-semes, en kwam te Jeruzalem; en hij brak aan den muur van Jeruzalem, van de poort van Efraim tot aan de Hoekpoort, vierhonderd ellen.
\par 14 En hij nam al het goud, en het zilver, en al de vaten, die gevonden werden in het huis des HEEREN, en in de schatten van des konings huis, mitsgaders gijzelaars; en hij keerde weder naar Samaria.
\par 15 Het overige nu der geschiedenissen van Joas, wat hij gedaan heeft, en zijn macht, en hoe hij gestreden heeft tegen Amazia, den koning van Juda, zijn die niet geschreven in het boek der kronieken der koningen van Israel?
\par 16 En Joas ontsliep met zijn vaderen, en werd te Samaria begraven bij de koningen van Israel; en zijn zoon Jerobeam werd koning in zijn plaats.
\par 17 Amazia nu, de zoon van Joas, koning van Juda, leefde na den dood van Joas, den zoon van Joahaz, den koning van Israel, vijftien jaren.
\par 18 Het overige nu der geschiedenissen van Amazia, is dat niet geschreven in het boek der kronieken der koningen van Juda?
\par 19 En zij maakten een verbintenis tegen hem te Jeruzalem, dat hij vluchtte naar Lachis; maar zij zonden hem na tot Lachis, en doodden hem aldaar.
\par 20 En zij brachten hem op paarden; en hij werd te Jeruzalem begraven, bij zijn vaderen, in de stad Davids.
\par 21 En het ganse volk van Juda nam Azaria (die nu zestien jaren oud was), en maakten hem koning in plaats van zijn vader Amazia.
\par 22 Die bouwde Elath, en bracht haar weder aan Juda, nadat de koning met zijn vaderen ontslapen was.
\par 23 In het vijftiende jaar van Amazia, den zoon van Joas, den koning van Juda, werd te Samaria koning, Jerobeam, de zoon van Joas, koning van Israel, en regeerde een en veertig jaren.
\par 24 En hij deed dat kwaad was in de ogen des HEEREN; hij week niet van alle zonden van Jerobeam, den zoon van Nebat, die Israel zondigen deed.
\par 25 Hij bracht ook weder de landpale van Israel van den ingang van Hamath, tot aan de zee van het vlakke veld; naar het woord des HEEREN, des Gods van Israel, dat Hij gesproken had door den dienst van Zijn knecht Jona, den zoon van Amitthai, den profeet, die van Gath-hefer was.
\par 26 Want de HEERE zag, dat de ellende van Israel zeer bitter was, en dat er geen opgeslotenen noch verlatenen waren, en dat Israel geen helper had.
\par 27 En de HEERE had niet gesproken, dat Hij den naam van Israel van onder den hemel verdelgen zou; maar Hij verloste hen door de hand van Jerobeam, den zoon van Joas.
\par 28 Het overige nu der geschiedenissen van Jerobeam, en al wat hij gedaan heeft, en zijn macht, hoe hij gekrijgd heeft, en hoe hij Damaskus en Hamath, tot Juda behorende, aan Israel wedergebracht heeft, zijn die niet geschreven in het boek der kronieken der koningen van Israel?
\par 29 En Jerobeam ontsliep met zijn vaderen, met de koningen van Israel; en zijn zoon Zacharia werd koning in zijn plaats.

\chapter{15}

\par 1 In het zeven en twintigste jaar van Jerobeam, den koning van Israel, werd koning Azaria, de zoon van Amazia, den koning van Juda.
\par 2 Hij was zestien jaren oud, toen hij koning werd, en hij regeerde twee en vijftig jaren te Jeruzalem; en de naam zijner moeder was Jecholia van Jeruzalem.
\par 3 En hij deed dat recht was in de ogen des HEEREN, naar al wat zijn vader Amazia gedaan had.
\par 4 Alleenlijk werden de hoogten niet weggenomen; het volk offerde en rookte nog op de hoogten.
\par 5 En de HEERE plaagde den koning, dat hij melaats werd tot den dag zijns doods; en hij woonde in een afgezonderd huis; doch Jotham, de zoon des konings, was over het huis, richtende het volk des lands.
\par 6 Het overige nu der geschiedenissen van Azaria, en al wat hij gedaan heeft, zijn die niet geschreven in het boek der kronieken der koningen van Juda?
\par 7 En Azaria ontsliep met zijn vaderen, en zij begroeven hem bij zijn vaderen, in de stad Davids; en zijn zoon Jotham werd koning in zijn plaats.
\par 8 In het acht en dertigste jaar van Azaria, den koning van Juda, regeerde Zacharia, de zoon van Jerobeam, over Israel te Samaria, zes maanden.
\par 9 En hij deed dat kwaad was in de ogen des HEEREN, gelijk als zijn vaderen gedaan hadden; hij week niet af van de zonden van Jerobeam, den zoon van Nebat, die Israel zondigen deed.
\par 10 En Sallum, de zoon van Jabes, maakte een verbintenis tegen hem, en sloeg hem voor het volk, en doodde hem; en hij werd koning in zijn plaats.
\par 11 Het overige nu der geschiedenissen van Zacharia, ziet, dat is geschreven in het boek der kronieken der koningen van Israel.
\par 12 Dit was het woord des HEEREN, dat Hij gesproken had tot Jehu, zeggende: U zullen zonen van het vierde gelid op den troon van Israel zitten; en het is alzo geschied.
\par 13 Sallum, de zoon van Jabes, werd koning, in het negen en dertigste jaar van Uzzia, den koning van Juda; en hij regeerde een volle maand te Samaria.
\par 14 Want Menahem, de zoon van Gadi, toog op van Thirza, en kwam te Samaria, en sloeg Sallum, den zoon van Jabes, te Samaria, en doodde hem, en werd koning in zijn plaats.
\par 15 Het overige nu der geschiedenissen van Sallum, en zijn verbintenis, die hij maakte, ziet, die zijn geschreven in het boek der kronieken der koningen van Israel.
\par 16 Toen sloeg Menahem Tifsah, met allen, die daarin waren, ook haar landpalen van Thirza af; omdat men niet voor hem had opengedaan, zo sloeg hij hen; al haar bevruchte vrouwen hieuw hij in stukken.
\par 17 In het negen en dertigste jaar van Azaria, den koning van Juda, werd Menahem, de zoon van Gadi, koning over Israel, en regeerde tien jaren te Samaria.
\par 18 En hij deed dat kwaad was in de ogen des HEEREN; hij week al zijn dagen niet af van de zonden van Jerobeam, den zoon van Nebat, die Israel zondigen deed.
\par 19 Toen kwam Pul, de koning van Assyrie, tegen het land; en Menahem gaf aan Pul duizend talenten zilvers, opdat zijn hand met hem zoude zijn, om het koninkrijk in zijn hand te sterken.
\par 20 Menahem nu bracht dit geld op van Israel, van alle geweldigen van vermogen, om den koning van Assyrie te geven, voor elk man vijftig zilveren sikkels; alzo keerde de koning van Assyrie weder, en bleef daar niet in het land.
\par 21 Het overige nu der geschiedenissen van Menahem, en al wat hij gedaan heeft, is dat niet geschreven in het boek der kronieken der koningen van Israel?
\par 22 Daarna ontsliep Menahem met zijn vaderen; en zijn zoon Pekahia werd koning in zijn plaats.
\par 23 In het vijftigste jaar van Azaria, den koning van Juda, werd Pekahia, de zoon van Menahem, koning over Israel, en regeerde twee jaren te Samaria.
\par 24 En hij deed dat kwaad was in de ogen des HEEREN; hij week niet af van de zonden van Jerobeam, den zoon van Nebat, die Israel zondigen deed.
\par 25 En Pekah, de zoon van Remalia, zijn hoofdman, maakte een verbintenis tegen hem, en sloeg hem te Samaria, in het paleis van het huis des konings, met Argob en met Arje, en met hem vijftig mannen van de kinderen der Gileadieten; alzo doodde hij hem, en werd koning in zijn plaats.
\par 26 Het overige nu der geschiedenissen van Pekahia, en al wat hij gedaan heeft, ziet, dat is geschreven in het boek der kronieken der koningen van Israel.
\par 27 In het twee en vijftigste jaar van Azaria, den koning van Juda, werd Pekah, de zoon van Remalia, koning over Israel, en regeerde twintig jaren te Samaria.
\par 28 En hij deed dat kwaad was in de ogen des HEEREN; hij week niet af van de zonden van Jerobeam, den zoon van Nebat, die Israel zondigen deed.
\par 29 In de dagen Pekah, den koning van Israel, kwam Tiglath-pilezer, de koning van Assyrie, en nam Ijon in, en Abel-beth-maacha, en Janoah, en Kedes, en Hazor, en Gilead, en Galilea, het ganse land van Nafthali; en hij voerde hen weg naar Assyrie.
\par 30 En Hosea, de zoon van Ela, maakte een verbintenis tegen Pekah, den zoon van Remalia, en sloeg hem, en doodde hem, en werd koning in zijn plaats; in het twintigste jaar van Jotham, den zoon van Uzzia.
\par 31 Het overige nu der geschiedenissen van Pekah, en al wat hij gedaan heeft, ziet, dat is geschreven in het boek der kronieken der koningen van Israel.
\par 32 In het tweede jaar van Pekah, den zoon van Remalia, den koning van Israel, werd Jotham koning, de zoon van Uzzia, den koning van Juda.
\par 33 Vijf en twintig jaren was hij oud, als hij koning werd, en regeerde zestien jaren te Jeruzalem; en de naam zijner moeder was Jerusa, de dochter van Zadok.
\par 34 En hij deed dat recht was in de ogen des HEEREN; naar alles, wat zijn vader Uzzia gedaan had, deed hij.
\par 35 Alleenlijk werden de hoogten niet weggenomen; het volk offerde en rookte nog op de hoogten; dezelve bouwde de hoge poort aan het huis des HEEREN.
\par 36 Het overige nu der geschiedenissen van Jotham, en al wat hij gedaan heeft, is dat niet geschreven in het boek der kronieken der koningen van Juda?
\par 37 In die dagen begon de HEERE in Juda te zenden Rezin, den koning van Syrie, en Pekah, den zoon van Remalia.
\par 38 En Jotham ontsliep met zijn vaderen, en werd begraven bij zijn vaderen in de stad van zijn vader David; en zijn zoon Achaz werd koning in zijn plaats.

\chapter{16}

\par 1 In het zeventiende jaar van Pekah, den zoon van Remalia, werd Achaz koning, de zoon van Jotham, den koning van Juda.
\par 2 Twintig jaren was Achaz oud, toen hij koning werd, en hij regeerde zestien jaren te Jeruzalem; en hij deed niet dat recht was in de ogen des HEEREN zijns Gods, als zijn vader David.
\par 3 Want hij wandelde in den weg der koningen van Israel; ja, hij deed ook zijn zoon door het vuur gaan, naar de gruwelen der heidenen, die de HEERE voor de kinderen Israels verdreven had.
\par 4 Hij offerde ook en rookte op de hoogten en op de heuvelen, ook onder alle groen geboomte.
\par 5 Toen toog Rezin, de koning van Syrie, op, met Pekah, den zoon van Remalia, den koning van Israel, naar Jeruzalem ten strijde; en zij belegerden Achaz, maar zij vermochten niet met strijden.
\par 6 Te dierzelfder tijd bracht Rezin, de koning van Syrie, Elath weder aan Syrie, en wierp de Joden uit Elath; en de Syriers kwamen te Elath, en hebben daar gewoond tot op dezen dag.
\par 7 Achaz nu zond boden tot Tiglath-pilezer, den koning van Assyrie, zeggende: Ik ben uw knecht en uw zoon; kom op, en verlos mij uit de hand van den koning van Syrie, en uit de hand van den koning van Israel, die zich tegen mij opmaken.
\par 8 En Achaz nam het zilver en het goud, dat in het huis des HEEREN, en in de schatten van het huis des konings gevonden werd, en hij zond den koning van Assyrie een geschenk.
\par 9 Zo hoorde de koning van Assyrie naar hem; want de koning van Assyrie toog op tegen Damaskus, en nam haar in, en voerde hen gevankelijk naar Kir, en hij doodde Rezin.
\par 10 Toen toog de koning Achaz Tiglath-pilezer, den koning van Assyrie, tegemoet, naar Damaskus; en gezien hebbende een altaar, dat te Damaskus was, zo zond de koning Achaz aan den priester Uria de gelijkenis van het altaar, en zijn afbeelding, naar zijn ganse maaksel.
\par 11 En Uria, de priester, bouwde een altaar, naar alles, wat de koning Achaz van Damaskus ontboden had; alzo deed de priester Uria, tegen dat de koning Achaz van Damaskus kwam.
\par 12 Als nu de koning van Damaskus gekomen was, zag de koning het altaar en de koning naderde tot het altaar, en offerde daarop.
\par 13 En hij stak zijn brandoffer aan, en zijn spijsoffer, en goot zijn drankoffer en sprengde het bloed zijner dankofferen op dat altaar.
\par 14 Maar het koperen altaar, dat voor het aangezicht des HEEREN was, dat bracht hij van het voorste deel van het huis, van tussen zijn altaar, en van tussen het huis des HEEREN, en hij zette het aan de zijde zijns altaars noordwaarts.
\par 15 En de koning Achaz gebood Uria, den priester, zeggende: Steek op het grote altaar aan het morgenbrandoffer, en het avondspijsoffer, en des konings brandoffer, en zijn spijsoffer, en het brandoffer van al het volk des lands, en hun spijsoffer, en hun drankofferen; en spreng daarop al het bloed des brandoffers, en al het bloed des slachtoffers; maar het koperen altaar zal mij zijn, om te onderzoeken.
\par 16 En Uria, de priester, deed naar alles, wat de koning Achaz geboden had.
\par 17 En de koning Achaz sneed de lijsten der stellingen af, en nam die van boven het wasvat weg, en deed de zee af van de koperen runderen, die daaronder waren; en hij zette die op een stenen vloer.
\par 18 Daartoe het deksel des sabbats, dat zij in het huis gebouwd hadden, en den buitensten ingang des konings nam hij weg van het huis des HEEREN, vanwege den koning van Assyrie.
\par 19 Het overige nu der geschiedenissen van Achaz, wat hij gedaan heeft, is dat niet geschreven in het boek der kronieken der koningen van Juda?
\par 20 En Achaz ontsliep met zijn vaderen, en werd begraven bij zijn vaderen, in de stad Davids; en Hizkia, zijn zoon, werd koning in zijn plaats.

\chapter{17}

\par 1 In het twaalfde jaar van Achaz, den koning van Juda, werd Hosea, de zoon van Ela, koning over Israel te Samaria, en regeerde negen jaren.
\par 2 En hij deed dat kwaad was in de ogen des HEEREN; evenwel niet, als de koningen van Israel, die voor hem geweest waren.
\par 3 Tegen hem toog op Salmaneser, koning van Assyrie; en Hosea werd zijn knecht, dat hij hem een geschenk gaf.
\par 4 Maar de koning van Assyrie bevond een verbintenis in Hosea, dat hij tot So, den koning van Egypte, boden gezonden had, en het geschenk aan den koning van Assyrie niet als te voren van jaar tot jaar opbracht; zo besloot hem de koning van Assyrie, en bond hem in het gevangenhuis.
\par 5 Want de koning van Assyrie toog op in het ganse land; ja, hij kwam op naar Samaria, en hij belegerde haar drie jaren.
\par 6 In het negende jaar van Hosea, nam de koning van Assyrie Samaria in, en voerde Israel weg in Assyrie, en deed ze wonen in Halah, en in Habor, aan de rivier Gozan, en in de steden der Meden.
\par 7 Want het was geschied, dat de kinderen Israels gezondigd hadden tegen den HEERE, hun God, Die hen uit Egypteland opgebracht had, van onder de hand van Farao, den koning van Egypte; en hadden andere goden gevreesd;
\par 8 En hadden gewandeld in de inzettingen der heidenen, die de HEERE voor het aangezicht der kinderen Israels verdreven had, en der koningen van Israel, die ze gemaakt hadden.
\par 9 En de kinderen Israels hadden de zaken, die niet recht zijn, tegen den HEERE, hun God, bemanteld; en hadden zich hoogten gebouwd in al hun steden, van den wachttoren af tot de vaste steden toe.
\par 10 En zij hadden zich staande beelden opgericht en bossen, op allen hogen heuvel en onder alle groen geboomte.
\par 11 En zij hadden daar gerookt op alle hoogten, gelijk de heidenen, die de HEERE van hun aangezichten weggevoerd had; en zij hadden kwade dingen gedaan, om den HEERE tot toorn te verwekken.
\par 12 En zij hadden de drekgoden gediend, waarvan de HEERE tot hen gezegd had: Gij zult deze zaak niet doen.
\par 13 Als nu de HEERE tegen Israel en tegen Juda, door den dienst van alle profeten, van alle zieners, betuigd had, zeggende: Bekeert u van uw boze wegen en houdt Mijn geboden, en Mijn inzettingen, naar al de wet, die Ik uw vaderen geboden heb, en die Ik tot u door de hand van Mijn knechten, de profeten, gezonden heb;
\par 14 Zo hoorden zij niet, maar zij verhardden hun nek, gelijk de nek hunner vaderen geweest was, die aan den HEERE, hun God, niet geloofd hadden.
\par 15 Daartoe verwierpen zij Zijn inzettingen, en Zijn verbond, dat Hij met hun vaderen gemaakt had, en Zijn getuigenissen, die Hij tegen hen betuigd had, en wandelden de ijdelheid na, dat zij ijdel werden, en achter de heidenen, die rondom hen waren, van dewelke de HEERE hun geboden had, dat zij niet zouden doen gelijk die.
\par 16 Ja, zij verlieten al de geboden des HEEREN, huns Gods, en maakten zich gegoten beelden, twee kalveren; en maakten bossen, en bogen zich voor alle heir des hemels, en dienden Baal.
\par 17 Ook deden zij hun zonen en hun dochteren door het vuur gaan, en gebruikten waarzeggerijen, en gaven op vogelgeschrei acht, en verkochten zich, om te doen dat kwaad was in de ogen des HEEREN, om Hem tot toorn te verwekken.
\par 18 Daarom vertoornde zich de HEERE zeer over Israel, dat Hij hen wegdeed van Zijn aangezicht; er bleef niets over, behalve de stam van Juda alleen.
\par 19 Zelfs hield Juda de geboden des HEEREN, huns Gods, niet; maar zij wandelden in de inzettingen van Israel, die zij gemaakt hadden.
\par 20 Zo verwierp de HEERE het ganse zaad van Israel, en bedrukte hen, en gaf ze in de hand der rovers, totdat Hij hen van Zijn aangezicht weggeworpen had.
\par 21 Want Hij scheurde Israel van het huis van David af, en zij maakten Jerobeam, den zoon van Nebat, koning; en Jerobeam dreef Israel af van achter den HEERE, en hij deed ze een grote zonde zondigen.
\par 22 Alzo wandelden de kinderen Israels in alle zonden van Jerobeam die hij gedaan had; zij weken daarvan niet af;
\par 23 Totdat de HEERE Israel van Zijn aangezicht wegdeed, gelijk als Hij gesproken had door den dienst van al Zijn knechten, de profeten; alzo werd Israel weggevoerd uit zijn land naar Assyrie, tot op dezen dag.
\par 24 De koning nu van Assyrie bracht volk van Babel, en van Chuta, en van Avva, en van Hamath, en Sefarvaim, en deed hen wonen in de steden van Samaria, in de plaats der kinderen Israels; en zij namen Samaria erfelijk in, en woonden in haar steden.
\par 25 En het geschiedde in het begin hunner woning aldaar, dat zij den HEERE niet vreesden; zo zond de HEERE leeuwen onder hen, die enigen van hen doodden.
\par 26 Daarom spraken zij tot den koning van Assyrie, zeggende: De volken, die gij vervoerd hebt, en hebt doen wonen in de steden van Samaria, weten de wijze des Gods van het land niet; daarom heeft Hij leeuwen onder hen gezonden, en ziet, zij doden hen, dewijl zij niet weten de wijze des Gods van het land.
\par 27 Toen gebood de koning van Assyrie, zeggende: Brengt een der priesteren daarheen, die gijlieden van daar weggevoerd hebt, dat zij henentrekken, en wonen aldaar; en dat hij hun lere de wijze des Gods van het land.
\par 28 Zo kwam een uit de priesteren, die zij van Samaria weggevoerd hadden, en woonde te Beth-el; en hij leerde hun, hoe zij den HEERE vrezen zouden.
\par 29 Maar elk volk maakte zijn goden; en zij stelden ze in de huizen der hoogten, die de Samaritanen gemaakt hadden, elk volk in hun steden, waarin zij woonachtig waren.
\par 30 Want de lieden van Babel maakten Sukkoth Benoth, en de lieden van Chut maakten Nergal, en de lieden van Hamath maakten Asima,
\par 31 En de Avieten maakten Nibhaz en Tartak, en de Sefarvieten verbrandden hun zonen voor Adramelech en Anamelech, de goden van Sefarvaim, met vuur.
\par 32 Ook vreesden zij den HEERE, en maakten zich van hun geringsten priesteren der hoogten, dewelke voor hen dienst deden in de huizen der hoogten.
\par 33 Zij vreesden den HEERE, en dienden ook hun goden, naar de wijze der volken, van dewelke zij die weggevoerd hadden.
\par 34 Tot op dezen dag toe doen die naar de eerste wijzen; zij vrezen den HEERE niet, en zij doen niet naar hun inzettingen, en naar hun rechten, en naar de wet, en naar het gebod, dat de HEERE geboden heeft aan de kinderen van Jakob, dien Hij den naam Israel gaf.
\par 35 Nochtans had de HEERE een verbond met hen gemaakt, en had hun geboden, zeggende: Gij zult geen andere goden vrezen, noch u voor hen nederbuigen, noch hen dienen, noch hun offerande doen.
\par 36 Maar den HEERE, Die u uit Egypteland met grote kracht en met een uitgestrekten arm opgevoerd heeft, Dien zult gij vrezen, en voor Hem zult gij u buigen, en Hem zult gij offerande doen;
\par 37 En de inzettingen, en de rechten, en de wet, en het gebod, die Hij u geschreven heeft, zult gij waarnemen te doen te allen dag; en gij zult andere goden niet vrezen.
\par 38 En het verbond, dat Ik met u gemaakt heb, zult gij niet vergeten; en gij zult andere goden niet vrezen.
\par 39 Maar den HEERE, uw God, zult gij vrezen; en Hij zal u redden uit de hand van al uw vijanden.
\par 40 Doch zij hoorden niet, maar zij deden naar hun eerste wijze.
\par 41 Maar deze volken vreesden den HEERE, en dienden hun gesneden beelden; ook doen hun kinderen en hun kindskinderen, gelijk als hun vaders gedaan hebben, tot op dezen dag.

\chapter{18}

\par 1 Het geschiedde nu in het derde jaar van Hosea, den zoon van Ela, den koning van Israel, dat Hizkia koning werd, de zoon van Achaz, koning van Juda.
\par 2 Vijf en twintig jaren was hij oud, toen hij koning werd, en hij regeerde negen en twintig jaren te Jeruzalem, en de naam zijner moeder was Abi, een dochter van Zacharia.
\par 3 En hij deed dat recht was in de ogen des HEEREN, naar alles, wat zijn vader David gedaan had.
\par 4 Hij nam de hoogten weg, en brak de opgerichte beelden, en roeide de bossen uit; en hij verbrijzelde de koperen slang, die Mozes gemaakt had, omdat de kinderen Israels tot die dagen toe haar gerookt hadden; en hij noemde haar Nehustan.
\par 5 Hij betrouwde op den HEERE, den God Israels, zodat na hem zijns gelijke niet was onder alle koningen van Juda, noch die voor hem geweest waren.
\par 6 Want hij kleefde den HEERE aan; hij week niet van Hem na te volgen, en hij hield Zijn geboden, die de HEERE aan Mozes geboden had.
\par 7 Zo was de HEERE met hem; overal, waar hij henen uittrok, handelde hij kloekelijk; daartoe viel hij af van den koning van Assyrie, dat hij hem niet diende.
\par 8 Hij sloeg de Filistijnen tot Gaza toe, en haar landpalen, van den wachttoren af tot de vaste steden toe.
\par 9 Het geschiedde nu in het vierde jaar van den koning Hizkia (hetwelk was het zevende jaar van Hosea, den zoon van Ela, den koning van Israel) dat Salmaneser, de koning van Assyrie, opkwam tegen Samaria, en haar belegerde.
\par 10 En zij namen haar in ten einde van drie jaren, in het zesde jaar van Hizkia; het was het negende jaar van Hosea, den koning van Israel, als Samaria ingenomen werd.
\par 11 En de koning van Assyrie voerde Israel weg naar Assyrie, en deed hen leiden in Halah, en in Habor, bij de rivier Gozan, en in de steden der Meden.
\par 12 Daarom dat zij de stem des HEEREN, huns Gods, niet waren gehoorzaam geweest, maar Zijn verbond overtreden hadden; en al wat Mozes, de knecht des HEEREN, geboden had, dat hadden zij niet gehoord, noch gedaan.
\par 13 Maar in het veertiende jaar van den koning Hizkia kwam Sanherib, de koning van Assyrie, op tegen alle vaste steden van Juda, en nam ze in.
\par 14 Toen zond Hizkia, de koning van Juda, tot den koning van Assyrie, naar Lachis, zeggende: Ik heb gezondigd, keer af van mij, wat gij mij opleggen zult, zal ik dragen. Toen leide de koning van Assyrie Hizkia, den koning van Juda, driehonderd talenten zilvers, en dertig talenten gouds op.
\par 15 Alzo gaf Hizkia al het zilver, dat gevonden werd in het huis des HEEREN, en in de schatten van het huis des konings.
\par 16 Te dier tijd sneed Hizkia het goud af van de deuren van den tempel des HEEREN, en van de posten, die Hizkia, de koning van Juda, had laten overtrekken, en gaf dat aan de koning van Assyrie.
\par 17 Evenwel zond de koning van Assyrie Tartan, en Rabsaris, en Rabsake, van Lachis tot den koning Hizkia, met een zwaar heir naar Jeruzalem; en zij togen op, en kwamen naar Jeruzalem. En als zij optogen en gekomen waren, bleven zij staan bij den watergang des oppersten vijvers, welke is bij den hogen weg van het veld des vollers.
\par 18 En zij riepen tot den koning; zo ging tot hen uit Eljakim, de zoon van Hilkia, de hofmeester, en Sebna, de schrijver, en Joah, de zoon van Asaf, de kanselier.
\par 19 En Rabsake zeide tot hen: Zegt nu tot Hizkia: Zo zegt de grote koning, de koning van Assyrie: Wat vertrouwen is dit, waarmede gij vertrouwt?
\par 20 Gij zegt (doch het is een woord der lippen): Er is raad en macht tot den oorlog; op wien vertrouwt gij nu, dat gij tegen mij rebelleert?
\par 21 Zie nu, vertrouwt gij u op dien gebroken rietstaf, op Egypte, op denwelken zo iemand leunt, zo zal hij in zijn hand gaan, en die doorboren; alzo is Farao, de koning van Egypte, al dengenen, die op hem vertrouwen.
\par 22 Maar zo gij tot mij zegt: Wij vertrouwen op den HEERE, onzen God; is Hij die niet, Wiens hoogten en Wiens altaren Hizkia weggenomen heeft, en tot Juda en tot Jeruzalem gezegd heeft: Voor dit altaar zult gij u buigen te Jeruzalem?
\par 23 Nu dan, wed toch met mijn heer, den koning van Assyrie; en ik zal u twee duizend paarden geven, zo gij voor u de ruiters daarop zult kunnen geven.
\par 24 Hoe zoudt gij dan het aangezicht van een enigen vorst van de geringste knechten mijns heren afkeren? Maar gij vertrouwt op Egypte, om de wagenen en om de ruiteren.
\par 25 Nu, ben ik zonder den HEERE opgetogen tegen deze plaats, om die te verderven? De HEERE heeft tot mij gezegd: Trek op tegen dat land, en verderf het.
\par 26 Toen zeide Eljakim, de zoon van Hilkia, en Sebna, en Joah tot Rabsake: Spreek toch tot uw knechten in het Syrisch, want wij verstaan het wel; en spreek met ons niet in het Joods, voor de oren des volks, dat op den muur is.
\par 27 Maar Rabsake zeide tot hen: Heeft mijn heer mij tot uw heer en tot u gezonden, om deze woorden te spreken? Is het niet tot de mannen, die op den muur zitten, dat zij met ulieden hun drek eten, en hun water drinken zullen?
\par 28 Alzo stond Rabsake, en riep met luider stem in het Joods; en hij sprak en zeide: Hoort het woord des groten konings, des konings van Assyrie!
\par 29 Zo zegt de koning: Dat Hizkia u niet bedriege: want hij zal u niet kunnen redden uit zijn hand.
\par 30 Daartoe dat Hizkia u niet doe vertrouwen op den HEERE, zeggende: De HEERE zal ons zekerlijk redden, en deze stad zal niet in de hand van den koning van Assyrie gegeven worden.
\par 31 Hoort naar Hizkia niet; want zo zegt de koning van Assyrie: Handelt met mij door een geschenk, en komt tot mij uit, en eet, een ieder van zijn wijnstok, en een ieder van zijn vijgeboom; en drinkt een ieder het water zijns bornputs;
\par 32 Totdat ik kom, en u haal in een land, als ulieder land, een land van koren en van most, een land van brood en van wijngaarden, een land van olijven, van olie en van honig; zo zult gij leven en niet sterven; en hoort niet naar Hizkia, want hij hitst u op, zeggende: De HEERE zal ons redden.
\par 33 Hebben de goden der volken, ieder zijn land, enigszins gered uit de hand van den koning van Assyrie?
\par 34 Waar zijn de goden van Hamath, en van Arpad? Waar zijn de goden van Sefarvaim, Hena en Ivva? Ja, hebben zij Samaria uit mijn hand gered?
\par 35 Welke zijn ze onder alle goden der landen, die hun land uit mijn hand gered hebben, dat de HEERE Jeruzalem uit mijn hand redden zou?
\par 36 Doch het volk zweeg stil en antwoordde hem niet een woord; want het gebod des konings was, zeggende: Gij zult hem niet antwoorden.
\par 37 Toen kwam Eljakim, de zoon van Hilkia, de hofmeester, en Sebna, de schrijver, en Joah, de zoon van Asaf, de kanselier, tot Hizkia, met gescheurde klederen; en zij gaven hem de woorden van Rabsake te kennen.

\chapter{19}

\par 1 En het geschiedde, als de koning Hizkia dat hoorde, zo scheurde hij zijn klederen, en bedekte zich met een zak, en ging in het huis des HEEREN.
\par 2 Daarna zond hij Eljakim, den hofmeester, en Sebna, den schrijver, en de oudsten der priesteren, met zakken bedekt, tot Jesaja, den profeet, den zoon van Amoz;
\par 3 En zij zeiden tot hem: Alzo zegt Hizkia: Deze dag is een dag der benauwdheid, en der schelding, en der lastering; want de kinderen zijn gekomen tot aan de geboorte, en er is geen kracht om te baren.
\par 4 Misschien zal de HEERE, uw God, horen al de woorden van Rabsake, denwelken zijn heer, de koning van Assyrie, gezonden heeft, om den levenden God te honen, en te schelden, met woorden, die de HEERE, uw God, gehoord heeft; hef dan een gebed op voor het overblijfsel, dat gevonden wordt.
\par 5 En de knechten van den koning Hizkia kwamen tot Jesaja.
\par 6 En Jesaja zeide tot hen: Zo zult gij tot uw heer zeggen: Zo zegt de HEERE: Vrees niet voor de woorden, die gij gehoord hebt, waarmede Mij de dienaars van den koning van Assyrie gelasterd hebben.
\par 7 Zie, Ik zal een geest in hem geven, dat hij een gerucht horen zal, en weder in zijn land keren; en Ik zal hem door het zwaard in zijn land vellen.
\par 8 Zo kwam Rabsake weder, en vond den koning van Assyrie, strijdende tegen Libna; want hij had gehoord, dat hij van Lachis vertrokken was.
\par 9 Als hij nu hoorde van Tirhaka, den koning van Cusch, zeggen: Ziet, hij is uitgetogen om tegen u te strijden, zond hij weder boden tot Hizkia, zeggende:
\par 10 Zo zult gij spreken tot Hizkia, den koning van Juda, zeggende: Laat u uw God niet bedriegen, op welken gij vertrouwt, zeggende: Jeruzalem zal in de hand des konings van Assyrie niet gegeven worden.
\par 11 Zie, gij hebt gehoord, wat de koningen van Assyrie aan alle landen gedaan hebben, die verbannende; en zoudt gij gered worden?
\par 12 Hebben de goden der volken, die mijn vaders verdorven hebben, dezelve gered, als Gozan, en Haran, en Rezef, en de kinderen van Eden, die in Telasser waren?
\par 13 Waar is de koning van Hamath, en de koning van Arpad, en de koning der stad Sefarvaim, Hena en Ivva?
\par 14 Als nu Hizkia de brieven uit der boden hand ontvangen, en die gelezen had, ging hij op in het huis des HEEREN, en Hizkia breidde die uit voor het aangezicht des HEEREN.
\par 15 En Hizkia bad voor het aangezicht des HEEREN, en zeide: O HEERE, God Israels, Die tussen de cherubim woont! Gij zelf, Gij alleen zijt de God van alle koninkrijken der aarde, Gij hebt den hemel en de aarde gemaakt.
\par 16 O, HEERE! neig Uw oor en hoor, doe, HEERE! Uw ogen open en zie, en hoor de woorden van Sanherib, die dezen gezonden heeft, om den levenden God te honen.
\par 17 Waarlijk, HEERE, hebben de koningen van Assyrie die heidenen en hun land verwoest;
\par 18 En hebben hun goden in het vuur geworpen; want zij waren geen goden, maar het werk van mensenhanden, hout en steen; daarom hebben zij die verdorven.
\par 19 Nu dan, HEERE, onze God, verlos ons toch uit zijn hand; zo zullen alle koninkrijken der aarde weten, dat Gij, HEERE, alleen God zijt.
\par 20 Toen zond Jesaja, de zoon van Amoz, tot Hizkia, zeggende: Zo spreekt de HEERE, de God Israels: Dat gij tot Mij gebeden hebt tegen Sanherib, den koning van Assyrie, heb Ik gehoord.
\par 21 Dit is het woord, dat de HEERE over hem gesproken heeft: De jonkvrouw, de dochter van Sion, veracht u, zij bespot u, de dochter van Jeruzalem schudt het hoofd achter u.
\par 22 Wien hebt gij gehoond en gelasterd? en tegen Wien hebt gij de stem verheven, en uw ogen omhoog opgeheven? Tegen den Heilige Israels!
\par 23 Door middel uwer boden hebt gij den HEERE gehoond, en gezegd: Ik heb met de menigte mijner wagenen beklommen de hoogten der bergen, de zijden van den Libanon; en ik zal zijn hoge cederbomen, en zijn uitgelezen dennebomen afhouwen; en zal komen in zijn uiterste herberg, in het woud zijns schonen velds.
\par 24 Ik heb gegraven en heb gedronken vreemde wateren; en ik heb met mijn voetzolen alle rivieren der belegerde plaatsen verdroogd.
\par 25 Hebt gij niet gehoord, dat Ik zulks lang te voren gedaan heb en dat van oude dagen af geformeerd heb? Nu heb Ik dat doen komen, dat gij zoudt zijn, om de vaste steden te verstoren tot woeste hopen.
\par 26 Daarom waren haar inwoners handeloos; zij waren verslagen en beschaamd; zij waren als het gras des velds, en de groene grasscheutjes, het hooi der daken, en het brandkoren, eer het over einde staat.
\par 27 Maar Ik weet uw zitten, en uw uitgaan, en uw inkomen, en uw woeden tegen Mij.
\par 28 Om uw woeden tegen Mij, en dat uw woeling voor Mijn oren opgekomen is, zo zal Ik Mijn haak in uw neus leggen, en Mijn gebit in uw lippen, en Ik zal u doen wederkeren door dien weg, door denwelken gij gekomen zijt.
\par 29 En dat zij u een teken, dat men in dit jaar eten zal, wat van zelf gewassen is; en in het tweede jaar, wat daarvan weder uitspruit; maar zaait in het derde jaar, en maait, en plant wijngaarden, en eet hun vruchten.
\par 30 Want het ontkomene, dat overgebleven is van het huis van Juda, zal wederom nederwaarts wortelen, en zal opwaarts vrucht dragen.
\par 31 Want van Jeruzalem zal het overblijfsel uitgaan, en het ontkomene van den berg Sion; de ijver van den HEERE der heirscharen zal dit doen.
\par 32 Daarom zo zegt de HEERE van den koning van Assyrie: Hij zal in deze stad niet komen, noch daar een pijl inschieten; ook zal hij met geen schild daarvoor komen, en zal geen wal daartegen opwerpen.
\par 33 Door den weg, dien hij gekomen is, door dien zal hij wederkeren; maar in deze stad zal hij niet komen, zegt de HEERE.
\par 34 Want Ik zal deze stad beschermen, om die te verlossen, om Mijnentwil, en om Davids, Mijns knechts wil.
\par 35 Het geschiedde dan in dienzelven nacht, dat de Engel des HEEREN uitvoer, en sloeg in het leger van Assyrie honderd vijf en tachtig duizend. En toen zij zich des morgens vroeg opmaakten, ziet, die allen waren dode lichamen.
\par 36 Zo vertrok Sanherib, de koning van Assyrie, en toog henen, en keerde weder; en hij bleef te Nineve.
\par 37 Het geschiedde nu, als hij in het huis van Nisroch, zijn god, zich nederboog, dat Adramelech en Sarezer, zijn zonen, hem met het zwaard versloegen; doch zij ontkwamen in het land van Ararat; en Esar-haddon, zijn zoon, werd koning in zijn plaats.

\chapter{20}

\par 1 In die dagen werd Hizkia krank tot stervens toe; en de profeet Jesaja, de zoon van Amoz, kwam tot hem, en zeide tot hem: Zo zegt de HEERE: Geef bevel aan uw huis, want gij zult sterven, en niet leven.
\par 2 Toen keerde hij zijn aangezicht om naar den wand, en hij bad tot den HEERE, zeggende:
\par 3 Och, HEERE, gedenk toch, dat ik voor Uw aangezicht in waarheid en met een volkomen hart gewandeld, en wat goed in Uw ogen is, gedaan heb. En Hizkia weende gans zeer.
\par 4 Het gebeurde nu, als Jesaja uit het middelvoorhof nog niet gegaan was, dat het woord des HEEREN tot hem geschiedde, zeggende:
\par 5 Keer weder en zeg tot Hizkia, den voorganger Mijns volks: Zo zegt de HEERE, de God van uw vader David: Ik heb uw gebed gehoord, Ik heb uw tranen gezien; zie, Ik zal u gezond maken; aan den derden dag zult gij opgaan in het huis des HEEREN;
\par 6 En Ik zal vijftien jaren tot uw dagen toedoen, en zal u uit de hand des konings van Assyrie verlossen, mitsgaders deze stad; en Ik zal deze stad beschermen om Mijnentwil, en om Mijns knechts Davids wil.
\par 7 Daarna zeide Jesaja: Neemt een klomp vijgen; en zij namen ze, en leiden ze op de zweer, en hij werd genezen.
\par 8 Hizkia nu had gezegd tot Jesaja: Welk is het teken, dat de HEERE mij gezond maken zal, en dat ik den derden dag in des HEEREN huis zal opgaan?
\par 9 En Jesaja zeide: Dit zal u een teken van den HEERE zijn, dat de HEERE het woord, dat Hij gesproken heeft, doen zal: Zal de schaduw tien graden voorwaarts gaan, of tien graden achterwaarts keren?
\par 10 Toen zeide Hizkia: Het is der schaduwe licht, tien graden nederwaarts te gaan; neen, maar dat de schaduw tien graden achterwaarts kere.
\par 11 En Jesaja, de profeet, riep den HEERE aan; en Hij deed de schaduw tien graden achterwaarts keren in de graden, dewelke zij nederwaarts gegaan was, in de graden van Achaz' zonnewijzer.
\par 12 Te dier tijd zond Berodach Baladan de zoon van Baladan, de koning van Babel, brieven en een geschenk aan Hizkia; want hij had gehoord, dat Hizkia krank geweest was.
\par 13 En Hizkia hoorde naar hen, en hij toonde hun zijn ganse schathuis, het zilver, en het goud, en de specerijen, en de beste olie, en zijn wapenhuis, en al wat gevonden werd in zijn schatten; er was geen ding in zijn huis, noch in zijn ganse heerschappij, dat hij hun niet toonde.
\par 14 Toen kwam de profeet Jesaja tot den koning Hizkia, en zeide tot hem: Wat hebben die mannen gezegd, en van waar zijn zij tot u gekomen? En Hizkia zeide: Zij zijn uit verren lande gekomen, uit Babel.
\par 15 En hij zeide: Wat hebben zij gezien in uw huis? En Hizkia zeide: Zij hebben alles gezien, wat in mijn huis is; geen ding is er in mijn schatten, dat ik hun niet getoond heb.
\par 16 Toen zeide Jesaja tot Hizkia: Hoor des HEEREN woord.
\par 17 Zie, de dagen komen, dat al wat in uw huis is, en wat uw vaderen tot dezen dage toe opgelegd hebben, naar Babel weggevoerd zal worden; er zal niets overgelaten worden, zegt de HEERE.
\par 18 Daartoe zullen zij van uw zonen, die uit u zullen voortkomen, die gij gewinnen zult, nemen, dat zij hovelingen zijn in het paleis des konings van Babel.
\par 19 Maar Hizkia zeide tot Jesaja: Het woord des HEEREN, dat gij gesproken hebt, is goed. Ook zeide hij: Zou het niet, naardien vrede en waarheid in mijn dagen wezen zal?
\par 20 Het overige nu der geschiedenissen van Hizkia, en al zijn macht, en hoe hij den vijver en den watergang gemaakt heeft, en water in de stad gebracht heeft, zijn die niet geschreven in het boek der kronieken der koningen van Juda?
\par 21 En Hizkia ontsliep met zijn vaderen; en zijn zoon Manasse werd koning in zijn plaats.

\chapter{21}

\par 1 Manasse was twaalf jaren oud, toen hij koning werd, en hij regeerde vijf en vijftig jaren te Jeruzalem; en de naam zijner moeder was Hefzi-bah.
\par 2 En hij deed dat kwaad was in de ogen des HEEREN, naar de gruwelen der heidenen, die de HEERE voor het aangezicht der kinderen Israels uit de bezitting verdreven had.
\par 3 Want hij bouwde de hoogten weder op, die Hizkia, zijn vader, verdorven had; en hij richtte Baal altaren op, en maakte een bos, gelijk als Achab, de koning van Israel, gemaakt had, en boog zich neder voor het heir des hemels, en diende ze.
\par 4 En hij bouwde altaren in het huis des HEEREN, waarvan de HEERE gezegd had: Te Jeruzalem zal Ik Mijn Naam zetten.
\par 5 Daartoe bouwde hij altaren voor al het heir des hemels, in beide de voorhoven van het huis des HEEREN.
\par 6 Ja, hij deed zijn zoon door het vuur gaan, en pleegde guichelarij en gaf op vogelgeschrei acht; en hij stelde waarzeggers en duivelskunstenaren; hij deed zeer veel kwaads in de ogen des HEEREN, om Hem tot toorn te verwekken.
\par 7 Hij stelde ook een gesneden beeld van het bos, dat hij gemaakt had, in het huis, waarvan de HEERE gezegd had tot David, en tot zijn zoon Salomo: In dit huis, en in Jeruzalem, die Ik uit alle stammen van Israel verkoren heb, zal Ik Mijn Naam zetten in eeuwigheid.
\par 8 En Ik zal niet voortvaren den voet van Israel te bewegen uit dit land, dat Ik hun vaderen gegeven heb; alleenlijk, zo zij waarnemen te doen, naar alles, wat Ik hun geboden heb, en naar de ganse wet, die Mijn knecht Mozes hun geboden heeft.
\par 9 Maar zij hoorden niet; want Manasse deed hen dwalen, dat zij erger deden dan de heidenen, die de HEERE voor het aangezicht der kinderen Israels verdelgd had.
\par 10 Toen sprak de HEERE door den dienst van Zijn knechten, de profeten, zeggende:
\par 11 Dewijl dat Manasse, de koning van Juda, deze gruwelen gedaan heeft, erger doende dan al wat de Amorieten gedaan hebben, die voor hem geweest zijn, ja, ook Juda door zijn drekgoden heeft doen zondigen;
\par 12 Daarom, alzo zegt de HEERE, de God Israels: Ziet, Ik zal een kwaad over Jeruzalem en Juda brengen, dat een ieder, die het hoort, beide zijn oren klinken zullen.
\par 13 En Ik zal over Jeruzalem het meetsnoer van Samaria trekken, mitsgaders het paslood van het huis van Achab; en Ik zal Jeruzalem uitwissen, gelijk als men een schotel uitwist; men wist dien uit, en men keert hem om op zijn holligheid.
\par 14 En Ik zal het overblijfsel Mijns erfdeels verlaten, en zal ze in de hand hunner vijanden geven; en zij zullen tot een roof en plundering worden al hun vijanden.
\par 15 Daarom, dat zij gedaan hebben dat kwaad was in Mijn ogen, en Mij tot toorn verwekt hebben, van dien dag, dat hun vaderen van Egypte uitgegaan zijn, ook tot op dezen dag toe.
\par 16 Daartoe vergoot Manasse ook zeer veel onschuldig bloed, totdat hij Jeruzalem van het ene einde tot het andere vervuld had; behalve zijn zonde, die hij Juda zondigen deed, doende wat kwaad was in de ogen des HEEREN.
\par 17 Het overige der geschiedenissen van Manasse, en al wat hij gedaan heeft, en zijn zonde, die hij gezondigd heeft, zijn die niet geschreven in het boek der kronieken der koningen van Juda?
\par 18 En Manasse ontsliep met zijn vaderen, en werd begraven in den hof van zijn huis, in den hof van Uzza; en zijn zoon Amon werd koning in zijn plaats.
\par 19 Amon was twee en twintig jaren oud, toen hij koning werd, en hij regeerde twee jaren te Jeruzalem; en de naam zijner moeder was Mesullemet, een dochter van Haruz van Jotba.
\par 20 En hij deed dat kwaad was in de ogen des HEEREN; gelijk als zijn vader Manasse gedaan had.
\par 21 Want hij wandelde in al den weg, dien zijn vader gewandeld had, en hij diende de drekgoden, die zijn vader gediend had, en hij boog zich voor die neder.
\par 22 Zo verliet hij den HEERE, den God zijner vaderen, en hij wandelde niet in den weg des HEEREN.
\par 23 En de knechten van Amon maakten een verbintenis tegen hem, en zij doodden den koning in zijn huis.
\par 24 Maar het volk des lands versloeg allen, die tegen den koning Amon een verbintenis gemaakt hadden; en het volk des lands maakte zijn zoon Josia koning in zijn plaats.
\par 25 Het overige nu der geschiedenissen van Amon, wat hij gedaan heeft, zijn die niet geschreven in het boek der kronieken der koningen van Juda?
\par 26 En men begroef hem in zijn graf, in den hof van Uzza; en zijn zoon Josia werd koning in zijn plaats.

\chapter{22}

\par 1 Josia was acht jaren oud, toen hij koning werd, en regeerde een en dertig jaren te Jeruzalem; en de naam zijner moeder was Jedida, een dochter van Adaja, van Bozkath.
\par 2 En hij deed dat recht was in de ogen des HEEREN; en hij wandelde in al den weg van zijn vader David, en week niet af ter rechter hand noch ter linkerhand.
\par 3 Het geschiedde nu in het achttiende jaar van den koning Josia, dat de koning den schrijver Safan, den zoon van Azalia, den zoon van Mesullam, zond in het huis des HEEREN, zeggende:
\par 4 Ga op tot Hilkia, den hogepriester, opdat hij het geld opsomme, dat in het huis des HEEREN gebracht is, hetwelk de wachters des dorpels van het volk verzameld hebben;
\par 5 En dat zij dat geven in de hand der verzorgers van het werk, die besteld zijn over het huis des HEEREN; opdat zij het geven aan degenen, die het werk doen, dat in het huis des HEEREN is, om de breuken van het huis te beteren;
\par 6 Aan de timmerlieden en de bouwlieden, en de metselaars, en om hout en gehouwene stenen te kopen, om het huis te beteren.
\par 7 Doch er werd met hen geen rekening gehouden van het geld, dat in hun hand geleverd was, want zij handelden trouwelijk.
\par 8 Toen zeide de hogepriester Hilkia tot Safan, den schrijver: Ik heb het wetboek in het huis des HEEREN gevonden; en Hilkia gaf dat boek aan Safan, die las het.
\par 9 Daarna kwam Safan, de schrijver, tot den koning, en bracht den koning bescheid weder, en hij zeide: Uw knechten hebben het geld, dat in het huis gevonden was, samengebracht, en hebben het gegeven in de hand der verzorgers van het werk, die besteld waren over het huis des HEEREN.
\par 10 Ook gaf Safan, de schrijver, den koning te kennen, zeggende: De priester Hilkia heeft mij een boek gegeven. En Safan las dat voor het aangezicht des konings.
\par 11 Het geschiedde nu, als de koning de woorden des wetboeks hoorde, dat hij zijn klederen scheurde.
\par 12 En de koning gebood Hilkia, den priester, en Ahikam, den zoon van Safan, en Achbor, den zoon van Michaja, en Safan, den schrijver, en Asaja, den knecht des konings, zeggende:
\par 13 Gaat henen, vraagt den HEERE voor mij, en voor het volk, en voor het ganse Juda, over de woorden dezes boeks, dat gevonden is; want de grimmigheid des HEEREN is groot, dewelke tegen ons aangestoken is, omdat onze vaderen niet gehoord hebben naar de woorden dezes boeks, om te doen naar al wat voor ons geschreven is.
\par 14 Toen ging de priester Hilkia, en Ahikam, en Achbor, en Safan, en Asaja henen tot de profetes Hulda, de huisvrouw van Sallum, den zoon van Tikva, den zoon van Harhas, den klederbewaarder (zij nu woonde te Jeruzalem, in het tweede deel), en zij spraken tot haar.
\par 15 En zij zeide tot hen: Zo zegt de HEERE, de God Israels: Zegt tot den man, die u tot mij gezonden heeft:
\par 16 Zo zegt de HEERE: Zie, Ik zal kwaad over deze plaats brengen, en over haar inwoners, namelijk al de woorden des boeks, dat de koning van Juda gelezen heeft.
\par 17 Daarom dat zij Mij verlaten, en anderen goden gerookt hebben, opdat zij Mij tot toorn verwekten met al het werk hunner handen, zo zal Mijn grimmigheid aangestoken worden, tegen deze plaats, en niet uitgeblust worden.
\par 18 Maar tot den koning van Juda, die u gezonden heeft, om den HEERE te vragen, alzo zult gij tot hem zeggen: Zo zegt de HEERE, de God Israels: Aangaande de woorden, die gij gehoord hebt;
\par 19 Omdat uw hart week geworden is, en gij u voor het aangezicht des HEEREN vernederd hebt, als gij hoordet, wat Ik gesproken heb tegen deze plaats en derzelver inwoners, dat zij tot een verwoesting en vloek zullen worden, en dat gij uw klederen gescheurd en voor Mijn aangezicht geweend hebt; zo heb Ik u ook verhoord, spreekt de HEERE.
\par 20 Daarom zie, Ik zal u verzamelen tot uw vaderen, en gij zult met vrede in uw graf verzameld worden, en uw ogen zullen al het kwaad niet zien, dat Ik over deze plaats brengen zal. En zij brachten den koning het antwoord weder.

\chapter{23}

\par 1 Toen zond de koning henen, en tot hem verzamelden al die oudsten van Juda en Jeruzalem.
\par 2 En de koning ging op in het huis des HEEREN, en met hem alle man van Juda, en alle inwoners van Jeruzalem, en de priesters en de profeten, en al het volk, van den minste tot den meeste; en hij las voor hun oren al de woorden van het boek des verbonds, dat in het huis des HEEREN gevonden was.
\par 3 De koning nu stond aan den pilaar, en maakte een verbond voor des HEEREN aangezicht, om den HEERE na te wandelen, en Zijn geboden, en Zijn getuigenissen, en Zijn inzettingen met ganser harte en met ganser ziele te houden, bevestigende de woorden dezes verbonds, die in dit boek geschreven zijn. En het ganse volk stond in dit verbond.
\par 4 En de koning gebood den hogepriester Hilkia, en den priesteren der tweede ordening, en den dorpelbewaarders, dat zij uit den tempel des HEEREN alle gereedschap, dat voor Baal, en voor het beeld van het bos, en voor al het heir des hemels gemaakt was, uitbrengen zouden; en hij verbrandde dat buiten Jeruzalem in de velden van Kidron, en liet het stof daarvan naar Beth-el dragen.
\par 5 Daartoe schafte hij de Chemarim af, die de koningen van Juda gesteld hadden, opdat men roken zou op de hoogten, in de steden van Juda, en rondom Jeruzalem, mitsgaders, die voor Baal, de zon, en de maan, en de andere planeten, en al het heir des hemels rookten.
\par 6 Hij bracht ook het beeld van het bos uit het huis des HEEREN weg, buiten Jeruzalem, tot de beek Kidron, en verbrandde het aan de beek Kidron, en vergruisde het tot stof; en hij wierp het stof daarvan op de graven der kinderen des volks.
\par 7 Daartoe brak hij de huizen der schandjongens af, die aan het huis des HEEREN waren, alwaar de vrouwen huisjes voor het beeld van het bos weefden.
\par 8 En hij bracht al de priesters uit de steden van Juda, en verontreinigde de hoogten, alwaar die priesters gerookt hadden, van Geba af tot Ber-seba toe; en hij brak de hoogten der poorten af, ook die aan de deur der poort van Jozua, den overste der stad, was, welke aan iemands linkerhand was, in de stadspoort gaande.
\par 9 Doch de priesters der hoogten offerden niet op het altaar des HEEREN te Jeruzalem; maar zij aten ongezuurde broden in het midden van hun broederen.
\par 10 Hij verontreinigde ook Thofeth, dat in het dal der kinderen van Hinnom is, opdat niemand zijn zoon of zijn dochter voor den Molech door het vuur deed gaan.
\par 11 En hij schafte de paarden af, die de koningen van Juda voor de zon gesteld hadden, van den ingang van het huis des HEEREN, tot de kamer van Nathan-melech, den hoveling, die in Parvarim was; en de wagenen der zon verbrandde hij met vuur.
\par 12 Verder de altaren die op het dak der opperzaal van Achaz waren, die de koningen van Juda gemaakt hadden, mitsgaders de altaren, die Manasse in de twee voorhoven van het huis des HEEREN gemaakt had, brak de koning af; en hij verbrijzelde ze van daar, en wierp het stof daarvan in de beek Kidron.
\par 13 De hoogten ook, die vooraan Jeruzalem waren, dewelke waren ter rechterhand van den berg Mashith, die Salomo, de koning van Israel, voor Astoreth, het verfoeisel der Sidoniers, en voor Kamos, het verfoeisel der Moabieten, en voor Milchom, den gruwel der kinderen Ammons, gebouwd had, verontreinigde de koning.
\par 14 Insgelijks brak hij de opgerichte beelden, en roeide de bossen uit; en hij vervulde hun plaats met mensenbeenderen.
\par 15 Daartoe ook het altaar, dat te Beth-el was, en de hoogte, die Jerobeam, de zoon van Nebat, dewelke Israel zondigen deed, gemaakt had; te zamen dat altaar en die hoogte brak hij af; ja, hij verbrandde de hoogte, hij vergruisde ze tot stof, en hij verbrandde het bos.
\par 16 En als Josia zich omkeerde, zag hij de graven, die daar op den berg waren, en zond henen, en nam de beenderen uit de graven, en verbrandde ze op dat altaar, en verontreinigde dat; naar het woord des HEEREN, dat de man Gods uitgeroepen had, die deze woorden uitriep.
\par 17 Verder zeide hij: Wat is dat voor een grafteken, dat ik zie? En de lieden der stad zeiden tot hem: Het is het graf van den man Gods, die uit Juda kwam, en deze dingen, die gij tegen dit altaar van Beth-el gedaan hebt, uitgeroepen heeft.
\par 18 En hij zeide: Laat hem liggen, dat niemand zijn beenderen verroere. Zo bevrijdden zij zijn beenderen, met de beenderen van den profeet, die uit Samaria gekomen was.
\par 19 Daartoe nam Josia ook weg al de huizen der hoogten, die in de steden van Samaria waren, die de koningen van Israel gemaakt hadden, om den HEERE tot toorn te verwekken; en hij deed dezelve naar al de daden, die hij te Beth-el gedaan had.
\par 20 En hij slachtte al de priesteren der hoogten, die daar waren, op de altaren, en verbrandde mensenbeenderen op dezelve. Daarna keerde hij weder naar Jeruzalem.
\par 21 En de koning gebood het ganse volk, zeggende: Houdt den HEERE, uw God, pascha, gelijk in dit boek des verbonds geschreven is.
\par 22 Want gelijk dit pascha was er geen gehouden, van de dagen der richteren af, die Israel gericht hadden, noch in al de dagen der koningen van Israel, noch der koningen van Juda.
\par 23 Maar in het achttiende jaar van den koning Josia, werd dit pascha den HEERE te Jeruzalem gehouden.
\par 24 En ook deed Josia weg de waarzeggers, en de duivelskunstenaars, en de terafim, en de drekgoden, en alle verfoeiselen, die in het land van Juda en in Jeruzalem gezien werden; opdat hij bevestigde de woorden der wet, die geschreven waren in het boek, dat de priester Hilkia in het huis des HEEREN gevonden had.
\par 25 En voor hem was geen koning zijns gelijke, die zich tot den HEERE, met zijn ganse hart, en met zijn ganse ziel, en met zijn ganse kracht, naar al de wet van Mozes, bekeerd had; en na hem stond zijns gelijke niet op.
\par 26 Nochtans keerde zich de HEERE van den brand Zijns groten toorns niet af, waarmede Zijn toorn brandde tegen Juda, om al de tergingen, waarmede Manasse Hem getergd had.
\par 27 En de HEERE zeide: Ik zal Juda ook van Mijn aangezicht wegdoen, gelijk als Ik Israel weggedaan heb; en Ik zal deze stad Jeruzalem verwerpen, die Ik verkoren heb, en het huis, waarvan Ik gezegd heb: Mijn Naam zal daar wezen.
\par 28 Het overige nu der geschiedenissen van Josia, en al wat hij gedaan heeft, zijn die niet geschreven in het boek der kronieken der koningen van Juda?
\par 29 In zijn dagen toog Farao Necho, de koning van Egypte, op tegen den koning van Assyrie, naar de rivier Frath; en de koning Josia toog hem tegemoet, en hij doodde hem te Megiddo, als hij hem gezien had.
\par 30 En zijn knechten voerden hem dood op een wagen van Megiddo, en brachten hem te Jeruzalem, en begroeven hem in zijn graf; en het volk des lands nam Joahaz, den zoon Josia, en zalfden hem, en maakten hem koning in zijns vaders plaats.
\par 31 Drie en twintig jaren was Joahaz oud, toen hij koning werd, en hij regeerde drie maanden te Jeruzalem; en de naam zijner moeder was Hamutal, de dochter van Jeremia, van Libna.
\par 32 En hij deed dat kwaad was in de ogen des HEEREN, naar alles, wat zijn vaderen gedaan hadden.
\par 33 Doch Farao Necho liet hem binden te Ribla in het land van Hamath, opdat hij te Jeruzalem niet regeren zou; en hij leide het land een boete op van honderd talenten zilvers en een talent gouds.
\par 34 Ook maakte Farao Necho Eljakim, den zoon van Josia, koning, in de plaats van zijn vader Josia, en veranderde zijn naam in Jojakim; maar Joahaz nam hij mede, en hij kwam in Egypte, en stierf aldaar.
\par 35 En Jojakim gaf dat zilver en dat goud aan Farao; doch hij schatte het land, om dat geld naar het bevel van Farao te geven; een ieder naar zijn schatting eiste hij het zilver en goud af van het volk des lands, om aan Farao Necho te geven.
\par 36 Vijf en twintig jaren was Jojakim oud, toen hij koning werd, en regeerde elf jaren te Jeruzalem; en de naam zijner moeder was Zebudda, een dochter van Pedaja, van Ruma.
\par 37 En hij deed dat kwaad was in de ogen des HEEREN, naar alles, wat zijn vaders gedaan hadden.

\chapter{24}

\par 1 In zijn dagen toog Nebukadnezar, de koning van Babel, op, en Jojakim werd zijn knecht drie jaren; daarna keerde hij zich om, en rebelleerde tegen hem.
\par 2 En de HEERE zond tegen hem de benden der Chaldeen, en de benden der Syriers, en de benden der Moabieten, en de benden der kinderen Ammons, en zond hen tegen Juda, om dat te verderven, naar het woord des HEEREN, dat Hij gesproken had door den dienst Zijner knechten, de profeten.
\par 3 Zekerlijk geschiedde dit naar het bevel des HEEREN tegen Juda, dat Hij hen van Zijn aangezicht wegdeed, om de zonden van Manasse, naar alles, wat hij gedaan had;
\par 4 Als ook om het onschuldig bloed, dat hij vergoten had, zodat hij Jeruzalem met onschuldig bloed vervuld had; daarom wilde de HEERE niet vergeven.
\par 5 Het overige nu der geschiedenissen van Jojakim, en al wat hij gedaan heeft, is dat niet geschreven in het boek der kronieken der koningen van Juda?
\par 6 En Jojakim ontsliep met zijn vaderen; en zijn zoon Jojachin werd koning in zijn plaats.
\par 7 De koning nu van Egypte toog voortaan niet meer uit zijn land; want de koning van Babel had, van de rivier van Egypte af tot aan de rivier Frath, ingenomen al wat van den koning van Egypte was.
\par 8 Jojachin was achttien jaren oud, toen hij koning werd, en regeerde drie maanden te Jeruzalem; en de naam zijner moeder was Nehusta, een dochter van Elnathan, van Jeruzalem.
\par 9 En hij deed dat kwaad was in de ogen des HEEREN, naar alles, wat zijn vader gedaan had.
\par 10 Te dier tijd togen de knechten van Nebukadnezar, den koning van Babel, naar Jeruzalem; en de stad werd belegerd.
\par 11 Zelfs kwam Nebukadnezar, de koning van Babel, tegen de stad, als zijn knechten die belegerden.
\par 12 Toen ging Jojachin, de koning van Juda, uit tot den koning van Babel, hij, en zijn moeder, en zijn knechten, en zijn vorsten, en zijn hovelingen; en de koning van Babel nam hem gevangen in het achtste jaar zijner regering.
\par 13 En hij bracht van daar uit al de schatten van het huis des HEEREN, en de schatten van het huis des konings; en hij hieuw alle gouden vaten af, die Salomo, de koning van Israel, in den tempel des HEEREN gemaakt had, gelijk als de HEERE gesproken had.
\par 14 En hij voerde gans Jeruzalem weg, mitsgaders al de vorsten, en alle strijdbare helden, tien duizend gevangen, en alle timmerlieden en smeden; niemand werd overgelaten, dan het arme volk des lands.
\par 15 Zo voerde hij Jojachin weg naar Babel, mitsgaders des konings moeder, en des konings vrouwen, en zijn hovelingen; daartoe de machtigen des lands bracht hij gevankelijk van Jeruzalem naar Babel;
\par 16 En alle kloeke mannen tot zeven duizend, en timmerlieden en smeden tot een duizend, en alle helden, die ten oorlog geoefend waren; dezen bracht de koning van Babel gevankelijk naar Babel.
\par 17 En de koning van Babel maakte Mattanja, deszelfs oom, koning in plaats van hem, en veranderde zijn naam in Zedekia.
\par 18 Zedekia was een en twintig jaren oud, als hij koning werd, en hij regeerde elf jaren te Jeruzalem; en de naam zijner moeder was Hamutal, een dochter van Jeremia, van Libna.
\par 19 En hij deed dat kwaad was in de ogen des HEEREN, naar alles, wat Jojakim gedaan had.
\par 20 Want het geschiedde, om den toorn des HEEREN tegen Jeruzalem en tegen Juda, totdat Hij hen van Zijn aangezicht weggeworpen had. En Zedekia rebelleerde tegen den koning van Babel.

\chapter{25}

\par 1 En het geschiedde in het negende jaar zijner regering, in de tiende maand, op den tienden der maand, dat Nebukadnezar, de koning van Babel, kwam tegen Jeruzalem, hij en zijn ganse heir, en legerde zich tegen haar; en zij bouwden tegen haar sterkten rondom.
\par 2 Zo kwam de stad in belegering, tot in het elfde jaar van den koning Zedekia.
\par 3 Op den negenden der vierde maand, als de honger in de stad sterk werd, en het volk des lands geen brood had,
\par 4 Toen werd de stad doorgebroken, en al de krijgslieden vloden des nachts door den weg der poort, tussen de twee muren, die aan des konings hof waren (de Chaldeen nu waren tegen de stad rondom), en de koning trok door den weg des vlakken velds.
\par 5 Doch het heir der Chaldeen jaagde den koning na, en zij achterhaalden hem in de vlakke velden van Jericho, en al zijn heir werd van bij hem verstrooid.
\par 6 Zij dan grepen den koning, en voerden hem opwaarts tot den koning van Babel, naar Ribla; en zij spraken een oordeel tegen hem.
\par 7 En zij slachtten de zonen van Zedekia voor zijn ogen, en men verblindde Zedekia's ogen, en zij bonden hem met twee koperen ketenen, en voerden hem naar Babel.
\par 8 Daarna in de vijfde maand, op den zevenden der maand (dit was het negentiende jaar van Nebukadnezar, den koning van Babel) kwam Nebuzaradan, de overste der trawanten, de knecht des konings van Babel, te Jeruzalem.
\par 9 En hij verbrandde het huis des HEEREN, en het huis des konings, mitsgaders alle huizen van Jeruzalem; en alle huizen der groten verbrandde hij met vuur.
\par 10 En het ganse heir de Chaldeen, dat met den overste der trawanten was, brak de muren van Jeruzalem rondom af.
\par 11 Het overige nu des volks, die in de stad overgelaten waren, en de afvalligen, die tot den koning van Babel gevallen waren, en het overige der menigte, voerde Nebuzaradan, de overste der trawanten, gevankelijk weg.
\par 12 Maar van de armsten des lands liet de overste der trawanten enigen overig tot wijngaardeniers en tot akkerlieden.
\par 13 Verder braken de Chaldeen de koperen pilaren, die in het huis des HEEREN waren, en de stellingen, en de koperen zee, die in het huis des HEEREN was; en zij voerden het koper daarvan naar Babel.
\par 14 Zij namen ook de potten, en de schoffelen, en de gaffelen, en de rookschalen, en al de koperen vaten, daar men den dienst mede deed.
\par 15 En de overste der trawanten nam weg de wierookvaten en de sprengbekkens, wat geheel goud en wat geheel zilver was.
\par 16 De twee pilaren, de ene zee, en de stellingen, die Salomo voor het huis des HEEREN gemaakt had; het koper van al deze vaten was zonder gewicht.
\par 17 De hoogte van een pilaar was achttien ellen, en het kapiteel daarop was koper; en de hoogte des kapiteels was drie ellen; en het net, en de granaatappelen op het kapiteel rondom, waren alle van koper; en dezen gelijk had de andere pilaar, met het net.
\par 18 Ook nam de overste der trawanten Seraja, den hoofdpriester, en Zefanja, den tweeden priester, en de drie dorpelbewaarders.
\par 19 En uit de stad nam hij een hoveling, die over de krijgslieden gesteld was, en vijf mannen uit degenen, die des konings aangezicht zagen, die in de stad gevonden werden, mitsgaders den oversten schrijver des heirs, die het volk des lands ten oorlog opschreef, en zestig mannen van het volk des lands, die in de stad gevonden werden.
\par 20 Als Nebuzaradan, de overste der trawanten, dezen genomen had, zo bracht hij hen tot den koning van Babel, naar Ribla.
\par 21 En de koning van Babel sloeg hen, en doodde hen te Ribla, in het land van Hamath. Alzo werd Juda uit zijn land gevankelijk weggevoerd.
\par 22 Maar aangaande het volk, dat in het land van Juda overgebleven was, dat Nebukadnezar, de koning van Babel, had laten overblijven, daarover stelde hij Gedalia, den zoon van Ahikam, den zoon van Safan.
\par 23 Toen nu al de oversten der heiren, zij en hun mannen, hoorden, dat de koning van Babel Gedalia tot overste gesteld had, kwamen zij tot Gedalia naar Mizpa; namelijk, Ismael, de zoon van Nethanja, en Johanan, de zoon van Kareah, en Seraja, de zoon van Tanhumeth, de Netofathiet, en Jaazanja, de zoon van den Maachathiet, zij en hun mannen.
\par 24 En Gedalia zwoer hun en hun mannen, en zeide tot hen: Vreest niet van te zijn knechten der Chaldeen, blijft in het land, en dient den koning van Babel, zo zal het u wel gaan.
\par 25 Maar het geschiedde in de zevende maand, dat Ismael, de zoon van Nethanja, den zoon van Elisama, van koninklijk zaad, kwam, en tien mannen met hem; en zij sloegen Gedalia, dat hij stierf; mitsgaders de Joden en de Chaldeen, die met hem te Mizpa waren.
\par 26 Toen maakte zich al het volk op, van de minste tot den meeste, en de oversten der heiren, en kwamen in Egypte; want zij vreesden voor de Chaldeen.
\par 27 Het geschiedde daarna in het zeven en dertigste jaar der wegvoering van Jojachin, den koning van Juda, in de twaalfde maand, op den zeven en twintigsten der maand, dat Evilmerodach, de koning van Babel, in het jaar, als hij koning werd, het hoofd van Jojachin, den koning van Juda, uit het gevangenhuis, verhief.
\par 28 En hij sprak vriendelijk met hem, en stelde zijn stoel boven den stoel der koningen, die bij hem te Babel waren.
\par 29 En hij veranderde de klederen zijner gevangenis, en hij at geduriglijk brood voor zijn aangezicht, al de dagen zijns levens.
\par 30 En aangaande zijn tering, een gedurige tering werd hem van den koning gegeven, elk dagelijks bestemde deel op zijn dag, al de dagen zijns levens.




\end{document}