\begin{document}

\title{Prediker}



\chapter{1}

\par 1 De woorden van den prediker, den zoon van David, den koning te Jeruzalem.
\par 2 Ijdelheid der ijdelheden, zegt de prediker; ijdelheid der ijdelheden, het is al ijdelheid.
\par 3 Wat voordeel heeft de mens van al zijn arbeid, dien hij arbeidt onder de zon?
\par 4 Het ene geslacht gaat, en het andere geslacht komt; maar de aarde staat in der eeuwigheid.
\par 5 Ook rijst de zon op, en de zon gaat onder, en zij hijgt naar haar plaats, waar zij oprees.
\par 6 Zij gaat naar het zuiden, en zij gaat om naar het noorden; de wind gaat steeds omgaande, en de wind keert weder tot zijn omgangen.
\par 7 Al de beken gaan in de zee, nochtans wordt de zee niet vol; naar de plaats, waar de beken heengaan, derwaarts gaande keren zij weder.
\par 8 Al deze dingen worden zo moede, dat het niemand zou kunnen uitspreken; het oog wordt niet verzadigd met zien; en het oor wordt niet vervuld van horen.
\par 9 Hetgeen er geweest is, hetzelve zal er zijn, en hetgeen er gedaan is, hetzelve zal er gedaan worden; zodat er niets nieuws is onder de zon.
\par 10 Is er enig ding, waarvan men zou kunnen zeggen: Ziet dat, het is nieuw? Het is alreeds geweest in de eeuwen, die voor ons geweest zijn.
\par 11 Er is geen gedachtenis van de voorgaande dingen; en van de navolgende dingen, die zijn zullen, van dezelve zal ook geen gedachtenis zijn bij degenen, die namaals wezen zullen.
\par 12 Ik, prediker, was koning over Israel te Jeruzalem.
\par 13 En ik begaf mijn hart om met wijsheid te onderzoeken, en na te speuren al wat er geschiedt onder den hemel. Deze moeilijke bezigheid heeft God den kinderen der mensen gegeven, om zich daarin te bekommeren.
\par 14 Ik zag al de werken aan, die onder de zon geschieden; en ziet, het was al ijdelheid en kwelling des geestes.
\par 15 Het kromme kan niet recht gemaakt worden; en hetgeen ontbreekt, kan niet geteld worden.
\par 16 Ik sprak met mijn hart, zeggende: Zie, ik heb wijsheid vergroot en vermeerderd, boven allen, die voor mij te Jeruzalem geweest zijn; en mijn hart heeft veel wijsheid en wetenschap gezien.
\par 17 En ik begaf mijn hart om wijsheid en wetenschap te weten, onzinnigheden en dwaasheid; ik ben gewaar geworden, dat ook dit een kwelling des geestes is.
\par 18 Want in veel wijsheid is veel verdriet; en die wetenschap vermeerdert, vermeerdert smart.

\chapter{2}

\par 1 Ik zeide in mijn hart: Nu, welaan, ik zal u beproeven door vreugde; derhalve zie het goede aan; maar zie, ook dat was ijdelheid.
\par 2 Tot het lachen zeide ik: Gij zijt onzinnig, en tot de vreugde: Wat maakt deze?
\par 3 Ik heb in mijn hart nagespeurd, om mijn vlees op te houden in den wijn, (nochtans leidende mijn hart in wijsheid) en om de dwaasheid vast te houden, totdat ik zou zien wat den kinderen der mensen het best ware, dat zij doen zouden onder den hemel, gedurende het getal der dagen huns levens.
\par 4 Ik maakte mij grote werken, ik bouwde mij huizen, ik plantte mij wijngaarden.
\par 5 Ik maakte mij hoven en lusthoven, en ik plantte bomen in dezelve, van allerlei vrucht.
\par 6 Ik maakte mij vijvers van wateren, om daarmede te bewateren het woud, dat met bomen groende.
\par 7 Ik kreeg knechten en maagden, en ik had kinderen des huizes; ook had ik een groot bezit van runderen en schapen, meer dan allen, die voor mij te Jeruzalem geweest waren.
\par 8 Ik vergaderde mij ook zilver en goud, en kleinoden der koningen en der landschappen; ik bestelde mij zangers en zangeressen, en wellustigheden der mensenkinderen, snarenspel, ja, allerlei snarenspel.
\par 9 En ik werd groot, en nam toe, meer dan iemand, die voor mij te Jeruzalem geweest was; ook bleef mijn wijsheid mij bij.
\par 10 En al wat mijn ogen begeerden, dat onttrok ik hun niet; ik wederhield mijn hart niet van enige blijdschap, maar mijn hart was verblijd vanwege al mijn arbeid; en dit was mijn deel van al mijn arbeid.
\par 11 Toen wendde ik mij tot al mijn werken, die mijn handen gemaakt hadden, en tot den arbeid, dien ik werkende gearbeid had; ziet, het was al ijdelheid en kwelling des geestes, en daarin was geen voordeel onder de zon.
\par 12 Daarna wendde ik mij, om te zien wijsheid, ook onzinnigheden en dwaasheid; want hoe zou een mens, die den koning nakomen zal, doen hetgeen alrede gedaan is?
\par 13 Toen zag ik, dat de wijsheid uitnemendheid heeft boven de dwaasheid, gelijk het licht uitnemendheid heeft boven de duisternis.
\par 14 De ogen des wijzen zijn in zijn hoofd, maar de zot wandelt in de duisternis. Toen bemerkte ik ook, dat enerlei geval hun allen bejegent.
\par 15 Dies zeide ik in mijn hart: Gelijk het den dwaze bejegent, zal het ook mijzelven bejegenen; waarom heb ik dan toen meer naar wijsheid gestaan? Toen sprak ik in mijn hart, dat ook hetzelve ijdelheid was.
\par 16 Want er zal in eeuwigheid niet meer gedachtenis van een wijze, dan van een dwaas zijn; aangezien hetgeen nu is, in de toekomende dagen altemaal vergeten wordt; en hoe sterft de wijze met den zot?
\par 17 Daarom haatte ik dit leven, want dit werk dacht mij kwaad, dat onder de zon geschiedt; want het is al ijdelheid en kwelling des geestes.
\par 18 Ik haatte ook al mijn arbeid, dien ik bearbeid had onder de zon, dat ik dien zou achterlaten aan een mens, die na mij wezen zal.
\par 19 Want wie weet, of hij wijs zal zijn, of dwaas? Evenwel zal hij heersen over al mijn arbeid, dien ik bearbeid heb en dien ik wijselijk beleid heb onder de zon. Dat is ook ijdelheid.
\par 20 Daarom keerde ik mij om, om mijn hart te doen wanhopen over al den arbeid, dien ik bearbeid heb onder de zon.
\par 21 Want er is een mens, wiens arbeid in wijsheid, en in wetenschap, en in geschiktheid is; nochtans zal hij dien overgeven tot zijn deel, aan een mens, die daaraan niet gearbeid heeft. Dit is ook ijdelheid en een groot kwaad.
\par 22 Wat heeft toch die mens van al zijn arbeid, en van de kwellingen zijns harten, dien hij is bearbeidende onder de zon?
\par 23 Want al zijn dagen zijn smarten, en zijn bezigheid is verdriet; zelfs des nachts rust zijn hart niet. Datzelve is ook ijdelheid.
\par 24 Is het dan niet goed voor den mens, dat hij ete en drinke, en dat hij zijn ziel het goede doe genieten in zijn arbeid? Ik heb ook gezien, dat zulks van de hand Gods is.
\par 25 (Want wie zou er van eten, of wie zou zich daartoe haasten, meer dan ik zelf?)
\par 26 Want Hij geeft wijsheid, en wetenschap, en vreugde den mens, die goed is voor Zijn aangezicht; maar den zondaar geeft Hij bezigheid om te verzamelen en te vergaderen, opdat Hij het geve dien, die goed is voor Gods aangezicht. Dit is ook ijdelheid en kwelling des geestes.

\chapter{3}

\par 1 Alles heeft een bestemden tijd, en alle voornemen onder den hemel heeft zijn tijd.
\par 2 Er is een tijd om geboren te worden, en een tijd om te sterven; een tijd om te planten, en een tijd om het geplante uit te roeien;
\par 3 Een tijd om om te doden, en een tijd om te genezen; een tijd om af te breken, en een tijd om te bouwen;
\par 4 Een tijd om te wenen, en een tijd om te lachen; een tijd om te kermen, en een tijd om op te springen;
\par 5 Een tijd om stenen weg te werpen, en een tijd om stenen te vergaderen; een tijd om te omhelzen, en een tijd om verre te zijn van omhelzen;
\par 6 Een tijd om te zoeken, en een tijd om verloren te laten gaan; een tijd om te bewaren, en een tijd om weg te werpen;
\par 7 Een tijd om te scheuren, en een tijd om toe te naaien; een tijd om te zwijgen, en een tijd om te spreken;
\par 8 Een tijd om lief te hebben, en een tijd om te haten; een tijd van oorlog, en een tijd van vrede.
\par 9 Wat voordeel heeft hij, die werkt, van hetgeen hij bearbeidt?
\par 10 Ik heb gezien de bezigheid, die God den kinderen der mensen gegeven heeft, om zichzelven daarmede te bekommeren.
\par 11 Hij heeft ieder ding schoon gemaakt op zijn tijd; ook heeft Hij de eeuw in hun hart gelegd, zonder dat een mens het werk, dat God gemaakt heeft, kan uitvinden, van het begin tot het einde toe.
\par 12 Ik heb gemerkt, dat er niets beters voor henlieden is, dan zich te verblijden, en goed te doen in zijn leven.
\par 13 Ja ook, dat ieder mens ete en drinke, en het goede geniete van al zijn arbeid, Dit is een gave Gods.
\par 14 Ik weet, dat al wat God doet, dat zal in der eeuwigheid zijn, en er is niet toe te doen, noch is er af te doen; en God doet dat, opdat men vreze voor Zijn aangezicht.
\par 15 Hetgeen geweest is, dat is nu, en wat wezen zal, dat is alrede geweest; en God zoekt het weggedrevene;
\par 16 Verder heb ik ook gezien onder de zon, ter plaatse des gerichts, aldaar was goddeloosheid; en ter plaatse der gerechtigheid, aldaar was goddeloosheid.
\par 17 Ik zeide in mijn hart: God zal den rechtvaardige en den goddeloze oordelen; want aldaar is de tijd voor alle voornemen, en over alle werk.
\par 18 Ik zeide in mijn hart van de gelegenheid der mensenkinderen, dat God hen zal verklaren, en dat zij zullen zien, dat zij als de beesten zijn aan zichzelven.
\par 19 Want wat den kinderen der mensen wedervaart, dat wedervaart ook den beesten; en enerlei wedervaart hun beiden; gelijk die sterft, alzo sterft deze, en zij allen hebben enerlei adem, en de uitnemendheid der mensen boven de beesten is geen; want allen zijn zij ijdelheid.
\par 20 Zij gaan allen naar een plaats; zij zijn allen uit het stof, en zij keren allen weder tot het stof.
\par 21 Wie merkt, dat de adem van de kinderen der mensen opvaart naar boven, en de adem der beesten nederwaarts vaart in de aarde?
\par 22 Dies heb ik gezien, dat er niets beters is, dan dat de mens zich verblijde in zijn werken, want dat is zijn deel; want wie zal hem daarhenen brengen, dat hij ziet, hetgeen na hem geschieden zal?

\chapter{4}

\par 1 Daarna wende ik mij, en zag aan al de onderdrukkingen, die onder de zon geschieden; en ziet, er waren de tranen der verdrukten, en dergenen, die geen trooster hadden; en aan de zijde hunner verdrukkers was macht, zij daarentegen hadden geen vertrooster.
\par 2 Dies prees ik de doden, die alrede gestorven waren, boven de levenden, die tot nog toe levend zijn.
\par 3 Ja, hij is beter dan die beiden, die nog niet geweest is, die niet gezien heeft het boze werk, dat onder de zon geschiedt.
\par 4 Verder zag ik al den arbeid en alle geschikkelijkheid des werks, dat het den mens nijd van zijn naaste aanbrengt. Dat is ook ijdelheid en kwelling des geestes.
\par 5 De zot vouwt zijn handen samen, en eet zijn eigen vlees.
\par 6 Een hand vol met rust is beter, dan beide de vuisten vol met arbeid en kwelling des geestes.
\par 7 Ik wendde mij wederom, en ik zag een ijdelheid onder de zon;
\par 8 Daar is er een, en geen tweede; hij heeft ook geen kind, noch broeder; nochtans is van al zijn arbeid geen einde; ook wordt zijn oog niet verzadigd van den rijkdom, en zegt niet: Voor wien arbeide ik toch, en doe mijn ziel gebrek hebben van het goede? Dit is ook ijdelheid, en het is een moeilijke bezigheid.
\par 9 Twee zijn beter dan een; want zij hebben een goede beloning van hun arbeid;
\par 10 Want indien zij vallen, de een richt zijn metgezel op; maar wee den ene, die gevallen is, want er is geen tweede om hem op te helpen.
\par 11 Ook, indien twee te zamen liggen, zo hebben zij warmte; maar hoe zou een alleen warm worden?
\par 12 En indien iemand den een mocht overweldigen, zo zullen de twee tegen hem bestaan; en een drievoudig snoer wordt niet haast gebroken.
\par 13 Beter is een arm en wijs jongeling, dan een oud en zot koning, die niet weet van meer vermaand te worden.
\par 14 Want een komt uit het gevangenhuis, om koning te zijn; daar ook een, die in zijn koninkrijk geboren is, verarmt.
\par 15 Ik zag al de levenden wandelen onder de zon, met den jongeling, den tweede, die in diens plaats staan zal.
\par 16 Er is geen einde van al het volk, van allen, die voor hen geweest zijn; de nakomelingen zullen zich ook over hem niet verblijden; gewisselijk, dat is ook ijdelheid en kwelling des geestes.
\par 17 Bewaar uw voet, als gij tot het huis Gods ingaat, en zijt liever nabij om te horen, dan om der zotten slachtoffer te geven; want zij weten niet, dat zij kwaad doen.

\chapter{5}

\par 1 Wees niet te snel met uw mond, en uw hart haaste niet een woord voort te brengen voor Gods aangezicht; want God is in den hemel, en gij zijt op de aarde; daarom laat uw woorden weinig zijn.
\par 2 Want gelijk de droom komt door veel bezigheid, alzo de stem des zots door de veelheid der woorden.
\par 3 Wanneer gij een gelofte aan God zult beloofd hebben, stel niet uit dezelve te betalen; want Hij heeft geen lust aan zotten; wat gij zult beloofd hebben, betaal het.
\par 4 Het is beter, dat gij niet belooft, dan dat gij belooft en niet betaalt.
\par 5 Laat uw mond niet toe, dat hij uw vlees zou doen zondigen; en zeg niet voor het aangezicht des engels, dat het een dwaling was; waarom zou God grotelijks toornen, om uwer stemme wille, en verderven het werk uwer handen?
\par 6 Want gelijk in de veelheid der dromen ijdelheden zijn, alzo in veel woorden; maar vrees gij God!
\par 7 Indien gij de onderdrukking des armen, en de beroving des gerichts en der gerechtigheid ziet in een landschap, verwonder u niet over zulk een voornemen; want die hoger is dan de hoge, neemt er acht op; en daar zijn hogen boven henlieden.
\par 8 Het voordeel des aardrijks is voor allen: de koning zelfs wordt van het veld gediend.
\par 9 Die het geld liefheeft, wordt van het geld niet zat; en wie den overvloed liefheeft, wordt van het inkomen niet zat. Dit is ook ijdelheid.
\par 10 Waar het goed vermenigvuldigt, daar vermenigvuldigen ook die het eten; wat nuttigheid hebben dan de bezitters daarvan, dan het gezicht hunner ogen?
\par 11 De slaap des arbeiders is zoet, hij hebbe weinig of veel gegeten; maar de zatheid des rijken laat hem niet slapen.
\par 12 Er is een kwaad, dat krankheid aanbrengt, hetwelk ik zag onder de zon: rijkdom van zijn bezitters bewaard tot hun eigen kwaad.
\par 13 Of de rijkdom zelf vergaat door een moeilijke bezigheid; en hij gewint een zoon, en er is niet met al in zijn hand.
\par 14 Gelijk als hij voortgekomen is uit zijner moeders buik, alzo zal hij naakt wederkeren, gaande gelijk hij gekomen was; en hij zal niet medenemen van zijn arbeid, dat hij met zijn hand zou wegdragen.
\par 15 Daarom is dit ook een kwaad, dat krankheid aanbrengt; dat hij in alle manier, gelijk hij gekomen is, alzo heen gaat; en wat voordeel is het hem, dat hij in den wind gearbeid heeft?
\par 16 Dat hij ook al zijn dagen in duisternis gegeten heeft; en dat hij veel verdriets gehad heeft, ook zijn krankheid, en onstuimigen toorn?
\par 17 Ziet, wat ik gezien heb, een goede zaak, die schoon is: te eten en te drinken, en te genieten het goede van al zijn arbeid, dien hij bearbeid heeft onder de zon, gedurende het getal der dagen zijns levens, hetwelk God hem geeft; want dat is zijn deel.
\par 18 Ook een iegelijk mens, aan denwelken God rijkdom en goederen gegeven heeft, en Hij geeft hem de macht, om daarvan te eten, en om zijn deel te nemen, en om zich te verheugen van zijn arbeid, datzelve is een gave van God.
\par 19 Want hij zal niet veel gedenken aan de dagen zijns levens, dewijl hem God verhoort in de blijdschap zijns harten.

\chapter{6}

\par 1 Er is een kwaad, dat ik gezien heb onder de zon, en het is veel onder de mensen:
\par 2 Een man, denwelken God gegeven heeft rijkdom, en goederen, en eer; en hij heeft voor zijn ziel aan geen ding gebrek, van alles wat hij begeert; en God geeft hem de macht niet, om daarvan te eten, maar dat een vreemd man dat opeet. Dit is ook ijdelheid en een kwade smart.
\par 3 Indien een man honderd kinderen gewon, en vele jaren leefde, zodat de dagen zijner jaren veel waren, doch zijn ziel niet verzadigd werd van het goed, en hij ook geen begrafenis had; ik zeg, dat een misdracht beter is dan hij.
\par 4 Want met ijdelheid komt zij, en in duisternis gaat zij weg, en met duisternis wordt haar naam bedekt.
\par 5 Ook heeft zij de zon niet gezien, noch bekend; zij heeft meer rust, dan hij.
\par 6 Ja, al leefde hij schoon tweemaal duizend jaren, en het goede niet zag; gaan zij niet allen naar een plaats?
\par 7 Al de arbeid des mensen is voor zijn mond; en nochtans wordt de begeerlijkheid niet vervuld.
\par 8 Want wat heeft de wijze meer dan de zot? Wat heeft de arme meer, die voor de levenden weet te wandelen?
\par 9 Beter is het aanzien der ogen, dan het wandelen der begeerlijkheid. Dit is ook ijdelheid en kwelling des geestes.
\par 10 Wat ook iemand zij, alrede is zijn naam genoemd, en het is bekend, dat hij een mens is; en dat hij niet kan rechten met dien, die sterker is dan hij.
\par 11 Voorwaar, er zijn veel dingen, die de ijdelheid vermeerderen; wat heeft de mens te meer daarvan?
\par 12 Want wie weet, wat goed is voor den mens in dit leven, gedurende het getal der dagen van het leven zijner ijdelheid, welke hij doorbrengt als een schaduw? Want wie kan den mens aanzeggen, wat na hem wezen zal onder de zon?

\chapter{7}

\par 1 Beter is een goede naam, dan goede olie, en de dag des doods, dan de dag, dat iemand geboren wordt.
\par 2 Het is beter te gaan in het klaaghuis, dan te gaan in het huis des maaltijds; want in hetzelve is het einde aller mensen, en de levende legt het in zijn hart.
\par 3 Het treuren is beter dan het lachen; want door de droefheid des aangezichts wordt het hart gebeterd.
\par 4 Het hart der wijzen is in het klaaghuis; maar het hart der zotten in het huis der vreugde.
\par 5 Het is beter te horen het bestraffen des wijzen, dan dat iemand hore het gezang der dwazen.
\par 6 Want gelijk het geluid der doornen onder een pot is, alzo is het lachen eens zots. Dit is ook ijdelheid.
\par 7 Voorwaar, de onderdrukking zou wel een wijze dol maken; en het geschenk verderft het hart.
\par 8 Het einde van een ding is beter dan zijn begin; de lankmoedige is beter dan de hoogmoedige.
\par 9 Zijt niet haastig in uw geest om te toornen; want de toorn rust in den boezem der dwazen.
\par 10 Zeg niet: Wat is er, dat de vorige dagen beter geweest zijn, dan deze? Want gij zoudt naar zulks niet uit wijsheid vragen.
\par 11 De wijsheid is goed met een erfdeel; en degenen, die de zon aanschouwen, hebben voordeel daarvan.
\par 12 Want de wijsheid is tot een schaduw, en het geld is tot een schaduw; maar de uitnemendheid der wetenschap is, dat de wijsheid haar bezitters het leven geeft.
\par 13 Aanmerk het werk Gods; want wie kan recht maken, dat Hij krom gemaakt heeft?
\par 14 Geniet het goede ten dage des voorspoeds, maar ten dage des tegenspoeds, zie toe; want God maakt ook den een tegenover den ander, ter oorzake dat de mens niet zou vinden iets, dat na hem zal zijn.
\par 15 Dit alles heb ik gezien in de dagen mijner ijdelheid; er is een rechtvaardige, die in zijn gerechtigheid omkomt; daarentegen is er een goddeloze, die in zijn boosheid zijn dagen verlengt.
\par 16 Wees niet al te rechtvaardig, noch houd uzelven al te wijs; waarom zoudt gij verwoesting over u brengen?
\par 17 Wees niet al te goddeloos, noch wees al te dwaas; waarom zoudt gij sterven buiten uw tijd?
\par 18 Het is goed, dat gij daaraan vasthoudt, en trek ook uw hand van dit niet af; want die God vreest, dien ontgaat dat al.
\par 19 De wijsheid versterkt den wijze meer dan tien heerschappers, die in een stad zijn.
\par 20 Voorwaar, er is geen mens rechtvaardig op aarde, die goed doet, en niet zondigt.
\par 21 Geef ook uw hart niet tot alle woorden, die men spreekt, opdat gij niet hoort, dat uw knecht u vloekt.
\par 22 Want uw hart heeft ook veelmalen bekend, dat gij ook anderen gevloekt hebt.
\par 23 Dit alles heb ik met wijsheid verzocht; ik zeide: Ik zal wijsheid bekomen, maar zij was nog verre van mij.
\par 24 Hetgeen verre af is, en zeer diep, wie zal dat vinden?
\par 25 Ik keerde mij om, en mijn hart, om te weten, en om na te sporen, en te zoeken wijsheid en een sluitrede; en om te weten de goddeloosheid der zotheid, en de dwaasheid der onzinnigheden.
\par 26 En ik vond een bitterder ding, dan de dood: een vrouw, welker hart netten en garen, en haar handen banden zijn; wie goed is voor Gods aangezicht, zal van haar ontkomen; daarentegen de zondaar zal van haar gevangen worden.
\par 27 Ziet, dit heb ik gevonden, zegt de prediker, het ene bij het andere, om de sluitrede te vinden;
\par 28 Dewelke mijn ziel nog zoekt, maar ik heb haar niet gevonden: een man uit duizend heb ik gevonden; maar een vrouw onder die allen heb ik niet gevonden.
\par 29 Alleenlijk ziet, dit heb ik gevonden, dat God den mens recht gemaakt heeft, maar zij hebben veel vonden gezocht.

\chapter{8}

\par 1 Wie is gelijk de wijze, en wie weet de uitlegging der dingen? De wijsheid der mensen verlicht zijn aangezicht, en de stuursheid zijns aangezichts wordt daardoor veranderd.
\par 2 Ik zeg: Neem acht op den mond des konings; doch naar de gelegenheid van den eed Gods.
\par 3 Haast u niet weg te gaan van zijn aangezicht; blijf niet staande in een kwade zaak; want al wat hem lust, doet hij.
\par 4 Waar het woord des konings is, daar is heerschappij; en wie zal tot hem zeggen: Wat doet gij?
\par 5 Wie het gebod onderhoudt, zal niets kwaads gewaar worden; en het hart eens wijzen zal tijd en wijze weten.
\par 6 Want een ieder voornemen heeft tijd en wijze, dewijl het kwaad des mensen veel is over hem.
\par 7 Want hij weet niet, wat er geschieden zal; want wie zal het hem te kennen geven, wanneer het geschieden zal?
\par 8 Er is geen mens, die heerschappij heeft over den geest, om den geest in te houden; en hij heeft geen heerschappij over den dag des doods; ook geen geweer in dezen strijd; ook zal de goddeloosheid haar meesters niet verlossen.
\par 9 Dit alles heb ik gezien, toen ik mijn hart begaf tot alle werk, dat onder de zon geschiedt: er is een tijd, dat de ene mens over den anderen mens heerst, hem ten kwade.
\par 10 Alzo heb ik ook gezien de goddelozen, die begraven waren, en degenen, die kwamen, en uit de plaats des Heiligen gingen, die werden vergeten in die stad, in dewelke zij recht gedaan hadden. Dit is ook ijdelheid.
\par 11 Omdat niet haastelijk het oordeel over de boze daad geschiedt, daarom is het hart van de kinderen der mensen in hen vol om kwaad te doen.
\par 12 Hoewel een zondaar honderd maal kwaad doet, en God hem de dagen verlengt; zo weet ik toch, dat het dien zal welgaan, die God vrezen, die voor Zijn aangezicht vrezen.
\par 13 Maar den goddeloze zal het niet welgaan, en hij zal de dagen niet verlengen; hij zal zijn gelijk een schaduw, omdat hij voor Gods aangezicht niet vreest.
\par 14 Er is nog een ijdelheid, die op aarde geschiedt: dat er zijn rechtvaardigen, dien het wedervaart naar het werk der goddelozen, en er zijn goddelozen, dien het wedervaart naar het werk der rechtvaardigen. Ik zeg, dat dit ook ijdelheid is.
\par 15 Daarom prees ik de blijdschap, dewijl de mens niets beters heeft onder de zon, dan te eten, en te drinken, en blijde te zijn; want dat zal hem aankleven van zijn arbeid, de dagen zijns levens, die hem God geeft onder de zon.
\par 16 Als ik mijn hart begaf, om wijsheid te weten, en om aan te zien de bezigheid, die op de aarde geschiedt, dat men ook, des daags of des nachts, den slaap niet ziet met zijne ogen;
\par 17 Toen zag ik al het werk Gods, dat de mens niet kan uitvinden, het werk, dat onder de zon geschiedt, om hetwelk een mens arbeidt om te zoeken, maar hij zal het niet uitvinden; ja, indien ook een wijze zeide, dat hij het zou weten, zo zal hij het toch niet kunnen uitvinden.

\chapter{9}

\par 1 Zekerlijk, dit alles heb ik in mijn hart gelegd, opdat ik dit alles klaarlijk mocht verstaan, dat de rechtvaardigen, en de wijzen, en hun werken in de hand Gods zijn; ook liefde, ook haat, weet de mens niet uit al hetgeen voor zijn aangezicht is.
\par 2 Alle ding wedervaart hun, gelijk aan alle anderen; enerlei wedervaart den rechtvaardige en den goddeloze, den goede en den reine, als den onreine; zo dien, die offert, als dien, die niet offert; gelijk den goede, alzo ook den zondaar, dien, die zweert, gelijk dien, die den eed vreest.
\par 3 Dit is een kwaad onder alles, wat onder de zon geschiedt, dat enerlei ding allen wedervaart, en dat ook het hart der mensenkinderen vol boosheid is, en dat er in hun leven onzinnigheden zijn in hun hart; en daarna moeten zij naar de doden toe.
\par 4 Want voor dengene, die vergezelschapt is bij alle levenden, is er hoop; want een levende hond is beter dan een dode leeuw.
\par 5 Want de levenden weten, dat zij sterven zullen, maar de doden weten niet met al; zij hebben ook geen loon meer, maar hun gedachtenis is vergeten.
\par 6 Ook is alrede hun liefde, ook hun haat, ook hun nijdigheid vergaan; en zij hebben geen deel meer in deze eeuw in alles, wat onder de zon geschiedt.
\par 7 Ga dan heen, eet uw brood met vreugde, en drink uw wijn van goeder harte; want God heeft alrede een behagen aan uw werken.
\par 8 Laat uw klederen te allen tijd wit zijn, en laat op uw hoofd geen olie ontbreken.
\par 9 Geniet het leven met de vrouw, die gij liefhebt, al de dagen uws ijdelen levens, welke God u gegeven heeft onder de zon, al uw ijdele dagen; want dit is uw deel in dit leven, en van uw arbeid, dien gij arbeidt onder de zon.
\par 10 Alles, wat uw hand vindt om te doen, doe dat met uw macht; want er is geen werk, noch verzinning, noch wetenschap, noch wijsheid in het graf, daar gij heengaat.
\par 11 Ik keerde mij, en zag onder de zon, dat de loop niet is der snellen, noch de strijd der helden, noch ook de spijs der wijzen, noch ook de rijkdom der verstandigen, noch ook de gunst der welwetenden, maar dat tijd en toeval aan alle dezen wedervaart;
\par 12 Dat ook de mens zijn tijd niet weet, gelijk de vissen, die gevangen worden met het boze net; en gelijk de vogelen, die gevangen worden met den strik; gelijk die, alzo worden de kinderen der mensen verstrikt, ter bozer tijd, wanneer derzelve haastelijk over hen valt.
\par 13 Ook heb ik onder de zon deze wijsheid gezien, en zij was groot bij mij;
\par 14 Er was een kleine stad, en weinig lieden waren daarin; en een groot koning kwam tegen haar, en hij omsingelde ze, en hij bouwde grote vastigheden tegen haar.
\par 15 En men vond daar een armen wijzen man in, die de stad verloste door zijn wijsheid; maar geen mens gedacht denzelven armen man.
\par 16 Toen zeide ik: Wijsheid is beter dan kracht, hoewel de wijsheid des armen veracht, en zijn woorden niet waren gehoord geweest.
\par 17 De woorden der wijzen moeten in stilheid aangehoord worden, meer dan het geroep desgenen, die over de zotten heerst.
\par 18 De wijsheid is beter dan de krijgswapenen, maar een enig zondaar verderft veel goeds.

\chapter{10}

\par 1 Een dode vlieg doet de zalf des apothekers stinken en opwellen; alzo een weinig dwaasheid een man, die kostelijk is van wijsheid en van eer.
\par 2 Het hart des wijzen is tot zijn rechter hand, maar het hart eens zots is tot zijn linkerhand.
\par 3 En ook wanneer de dwaas op den weg wandelt, zijn hart ontbreekt hem, en hij zegt tot een iegelijk, dat hij dwaas is.
\par 4 Als de geest des heersers tegen u oprijst, verlaat uw plaats niet; want het is medicijn, het stilt grote zonden.
\par 5 Er is nog een kwaad, dat ik gezien heb onder de zon, als een dwaling, die van het aangezicht des oversten voortkomt.
\par 6 Een dwaas wordt gezet in grote hoogheden, maar de rijken zitten in de laagte.
\par 7 Ik heb knechten te paard gezien, en vorsten, gaande als knechten op de aarde.
\par 8 Wie een kuil graaft, zal daarin vallen; en wie een muur doorbreekt, een slang zal hem bijten.
\par 9 Wie stenen wegdraagt, zal smart daardoor lijden; wie hout klieft, zal daardoor in gevaar zijn.
\par 10 Indien hij het ijzer heeft stomp gemaakt, en hij slijpt de snede niet, dan moet hij meerder kracht te werk stellen; maar de wijsheid is een uitnemende zaak, om iets recht te maken.
\par 11 Indien de slang gebeten heeft, eer der bezwering geschied is, dan is er geen nuttigheid voor den allerwelsprekendsten bezweerder.
\par 12 De woorden van een wijzen mond zijn aangenaam; maar de lippen van een zot verslinden hemzelven.
\par 13 Het begin der woorden zijns monds is dwaasheid, en het einde zijns monds is boze dolligheid.
\par 14 De dwaas maakt wel veel woorden; maar de mens weet niet, wat het zij, dat geschieden zal; en wat na hem geschieden zal, wie zal het hem te kennen geven?
\par 15 De arbeid der zotten maakt een iegelijk van hen moede; dewijl zij niet weten naar de stad te gaan.
\par 16 Wee u, land! welks koning een kind is, en welks vorsten tot in den morgenstond eten!
\par 17 Welgelukzalig zijt gij, land! welks koning een zoon der edelen is, en welks vorsten ter rechter tijd eten, tot sterkte en niet tot drinkerij.
\par 18 Door grote luiheid verzwakt het gebint, en door slapheid der handen wordt het huis doorlekkende.
\par 19 Men maakt maaltijden om te lachen, en de wijn verheugt de levenden, en het geld verantwoordt alles.
\par 20 Vloek den koning niet, zelfs in uw gedachten, en vloek den rijke niet in het binnenste uwer slaapkamer; want het gevogelte des hemels zou de stem wegvoeren, en het gevleugelde zou het woord te kennen geven.

\chapter{11}

\par 1 Werp uw brood uit op het water, want gij zult het vinden na vele dagen.
\par 2 Geef een deel aan zeven, ja, ook aan acht; want gij weet niet, wat kwaad op de aarde wezen zal.
\par 3 Als de wolken vol geworden zijn, zo storten zij plasregen uit op de aarde; en als de boom naar het zuiden, of als hij naar het noorden valt, in de plaats, waar de boom valt, daar zal hij wezen.
\par 4 Wie op den wind acht geeft, die zal niet zaaien, en wie op de wolken ziet, die zal niet maaien.
\par 5 Gelijk gij niet weet, welke de weg des winds zij, of hoedanig de beenderen zijn in den buik van een zwangere vrouw, alzo weet gij het werk Gods niet, Die het alles maakt.
\par 6 Zaai uw zaad in den morgenstond, en trek uw hand des avonds niet af; want gij weet niet, wat recht wezen zal, of dit of dat, of dat die beide te zamen goed zijn zullen.
\par 7 Verder, het licht is zoet, en het is den ogen goed de zon te aanschouwen;
\par 8 Maar indien de mens veel jaren heeft, en verblijdt zich in die allen, zo laat hem ook gedenken aan de dagen der duisternis, want die zullen veel zijn; en al wat gekomen is, is ijdelheid.
\par 9 Verblijd u, o jongeling! in uw jeugd, en laat uw hart zich vermaken in de dagen uwer jongelingschap, en wandel in de wegen uws harten, en in de aanschouwingen uwer ogen; maar weet, dat God, om al deze dingen, u zal doen komen voor het gericht.
\par 10 Zo doe dan de toornigheid wijken van uw hart, en doe het kwade weg van uw vlees, want de jeugd, en de jonkheid is ijdelheid.

\chapter{12}

\par 1 En gedenk aan uw Schepper in de dagen uwer jongelingschap, eer dat de kwade dagen komen, en de jaren naderen, van dewelke gij zeggen zult: Ik heb geen lust in dezelve.
\par 2 Eer dan de zon, en het licht, en de maan, en de sterren verduisterd worden, en de wolken wederkomen na den regen.
\par 3 In den dag, wanneer de wachters des huizes zullen beven, en de sterke mannen zichzelven zullen krommen, en de maalsters zullen stilstaan, omdat zij minder geworden zijn, en die door de vensteren zien, verduisterd zullen worden;
\par 4 En de twee deuren naar de straat zullen gesloten worden, als er is een nederig geluid der maling, en hij opstaat op de stem van het vogeltje, en al de zangeressen nedergebogen zullen worden.
\par 5 Ook wanneer zij voor de hoogte zullen vrezen, en dat er verschrikkingen zullen zijn op den weg, en de amandelboom zal bloeien, en dat de sprinkhaan zichzelven een last zal wezen, en dat de lust zal vergaan; want de mens gaat naar zijn eeuwig huis, en de rouwklagers zullen in de straat omgaan.
\par 6 Eer dat het zilveren koord ontketend wordt, en de gulden schaal in stukken gestoten wordt, en de kruik aan de springader gebroken wordt, en het rad aan den bornput in stukken gestoten wordt;
\par 7 En dat het stof wederom tot aarde keert, als het geweest is; en de geest weder tot God keert, Die hem gegeven heeft.
\par 8 Ijdelheid der ijdelheden, zegt de prediker; het is al ijdelheid!
\par 9 En voorts, dewijl de prediker wijs geweest is, zo leerde hij het volk nog wetenschap, en merkte op, en onderzocht; hij stelde vele spreuken in orde.
\par 10 De prediker zocht aangename woorden uit te vinden, en het geschrevene is recht, woorden der waarheid.
\par 11 De woorden der wijzen zijn gelijk prikkelen, en gelijk nagelen, diep ingeslagen van de meesters der verzamelingen, die gegeven zijn van den enigen Herder.
\par 12 En wat boven dezelve is, mijn zoon! wees gewaarschuwd; van vele boeken te maken is geen einde, en veel lezens is vermoeiing des vleses.
\par 13 Van alles, wat gehoord is, is het einde van de zaak: Vrees God, en houd Zijn geboden, want dit betaamt allen mensen.
\par 14 Want God zal ieder werk in het gericht brengen, met al wat verborgen is, hetzij goed, of hetzij kwaad.



\end{document}