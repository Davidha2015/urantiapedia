\begin{document}

\title{Daniel}



\chapter{1}

\par 1 In het derde jaar des koninkrijks van Jojakim, den koning van Juda, kwam Nebukadnezar, de koning van Babel, te Jeruzalem, en belegerde haar.
\par 2 En de HEERE gaf Jojakim, den koning van Juda, in zijn hand, en een deel der vaten van het huis Gods; en hij bracht ze in het land van Sinear, in het huis zijns gods; en de vaten bracht hij in het schathuis zijns gods.
\par 3 En de koning zeide tot Aspenaz, den overste zijner kamerlingen, dat hij voorbrengen zou enigen uit de kinderen Israels, te weten, uit het koninklijk zaad, en uit de prinsen;
\par 4 Jongelingen, aan dewelke geen gebrek ware, maar schoon van aangezicht, en vernuftig in alle wijsheid, en ervaren in wetenschap, en kloek van verstand, en in dewelke bekwaamheid ware, om te staan in des konings paleis; en dat men hen onderwees in de boeken en spraak der Chaldeen.
\par 5 En de koning verordende hun, wat men ze dag bij dag geven zou van de stukken der spijs des konings, en van den wijn zijns dranks, en dat men hen drie jaren alzo optoog, en dat zij ten einde derzelve zouden staan voor het aangezicht des konings.
\par 6 Onder dezelve nu waren uit de kinderen van Juda: Daniel, Hananja, Misael en Azarja.
\par 7 En de overste der kamerlingen gaf hun andere namen, en Daniel noemde hij Beltsazar, en Hananja Sadrach, en Misael Mesach, en Azarja Abed-nego.
\par 8 Daniel nu nam voor in zijn hart, dat hij zich niet zou ontreinigen met de stukken van de spijs des konings, noch met den wijn zijns dranks; daarom verzocht hij van den overste der kamerlingen, dat hij zich niet mocht ontreinigen.
\par 9 En God gaf Daniel genade en barmhartigheid voor het aangezicht van den overste der kamerlingen.
\par 10 Want de overste der kamerlingen zeide tot Daniel: Ik vreze mijn heer, den koning, die ulieder spijs, en ulieder drank verordend heeft; want waarom zou hij ulieder aangezichten droeviger zien, dan der jongelingen, die in gelijkheid met ulieden zijn? Alzo zoudt gij mijn hoofd bij den koning schuldig maken.
\par 11 Toen zeide Daniel tot Melzar, dien de overste der kamerlingen gesteld had over Daniel, Hananja, Misael en Azarja:
\par 12 Beproef toch uw knechten tien dagen lang, en men geve ons van het gezaaide te eten, en water te drinken.
\par 13 En men zie voor uw aangezicht onze gedaanten, en de gedaante der jongelingen, die de stukken van de spijs des konings eten; en doe met uw knechten, naar dat gij zien zult.
\par 14 Toen hoorde hij hen in deze zaak, en hij beproefde ze tien dagen.
\par 15 Ten einde nu der tien dagen, zag men, dat hun gedaanten schoner waren, en zij vetter waren van vlees dan al de jongelingen, die de stukken van de spijze des konings aten.
\par 16 Toen geschiedde het, dat Melzar de stukken hunner spijs wegnam, mitsgaders den wijn huns dranks, en hij gaf hun van het gezaaide.
\par 17 Aan deze vier jongelingen nu gaf God wetenschap en verstand in alle boeken, en wijsheid; maar Daniel gaf Hij verstand in allerlei gezichten en dromen.
\par 18 Ten einde nu der dagen, waarvan de koning gezegd had, dat men hen zou inbrengen, zo bracht ze de overste der kamerlingen in voor het aangezicht van Nebukadnezar,
\par 19 En de koning sprak met hen; doch er werd uit hen allen niemand gevonden, gelijk Daniel, Hananja, Misael en Azarja; en zij stonden voor het aangezicht des konings.
\par 20 En in alle zaken van verstandige wijsheid, die de koning hun afvroeg, zo vond hij hen tienmaal boven al de tovenaars en sterrekijkers, die in zijn ganse koninkrijk waren.
\par 21 En Daniel bleef tot het eerste jaar van den koning Kores toe.

\chapter{2}

\par 1 In het tweede jaar nu des koninkrijks van Nebukadnezar, droomde Nebukadnezar dromen; daarvan werd zijn geest verslagen, en zijn slaap werd in hem gebroken.
\par 2 Toen zeide de koning, dat men roepen zou de tovenaars, en de sterrekijkers, en de guichelaars, en de Chaldeen, om den koning zijn dromen te kennen te geven; zij nu kwamen, en stonden voor het aangezicht des konings.
\par 3 En de koning zeide tot hen: Ik heb een droom gedroomd; en mijn geest is ontsteld om dien droom te weten.
\par 4 Toen spraken de Chaldeen, tot den koning in het Syrisch: O koning, leef in eeuwigheid! Zeg uw knechten den droom, zo zullen wij de uitlegging te kennen geven.
\par 5 De koning antwoordde en zeide tot de Chaldeen: De zaak is mij ontgaan; indien gij mij den droom en zijn uitlegging niet bekend maakt, gij zult in stukken gehouwen worden, en uw huizen zullen tot een drekhoop gemaakt worden.
\par 6 Maar indien gijlieden den droom en zijn uitlegging te kennen geeft, zo zult gij geschenken en gaven, en grote eer van mij ontvangen; daarom geeft mij den droom en zijn uitlegging te kennen.
\par 7 Zij antwoordden ten tweeden male, en zeiden: De koning zegge zijn knechten den droom, dan zullen wij de uitlegging te kennen geven.
\par 8 De koning antwoordde en zeide: Ik weet vastelijk, dat gijlieden den tijd uitkoopt, dewijl gij ziet, dat de zaak mij ontgaan is.
\par 9 Indien gijlieden mij dien droom niet te kennen geeft, ulieder vonnis is enerlei; daarom hebt gij een leugenachtig en verdicht woord voor mij te zeggen bereid, totdat de tijd verandere; daarom zegt mij den droom, dan zal ik weten, dat gij mij deszelfs uitlegging zult te kennen geven.
\par 10 De Chaldeen antwoordden voor den koning, en zeiden: Er is geen mens op den aardbodem, die des konings woord zou kunnen te kennen geven; daarom is er geen koning, grote of heerser, die zulk een zaak begeerd heeft van enigen tovenaar, of sterrekijker, of Chaldeer.
\par 11 Want de zaak die de koning begeert, is te zwaar; en er is niemand anders, die dezelve voor den koning te kennen kan geven, dan de goden, welker woning bij het vlees niet is.
\par 12 Daarom werd de koning toornig en zeer verbolgen, en zeide, dat men al de wijzen te Babel zou ombrengen.
\par 13 Die wet dan ging uit, en de wijzen werden gedood; men zocht ook Daniel en zijn metgezellen, om gedood te worden.
\par 14 Toen bracht Daniel een raad en oordeel in, aan Arioch, den overste der trawanten des konings, die uitgetogen was, om de wijzen van Babel te doden.
\par 15 Hij antwoordde en zeide tot Arioch, den bevelhebber des konings: Waarom zou de wet van 's konings wege zo verhaast worden? Toen gaf Arioch aan Daniel de zaak te kennen.
\par 16 En Daniel ging in, en verzocht van den koning, dat hij hem een bestemden tijd wilde geven, dat hij den koning de uitlegging te kennen gave.
\par 17 Toen ging Daniel naar zijn huis, en hij gaf de zaak zijn metgezellen, Hananja, Misael, en Azarja te kennen;
\par 18 Opdat zij van den God des hemels barmhartigheden verzochten over deze verborgenheid, dat Daniel en zijn metgezellen met de overige wijzen van Babel niet omkwamen.
\par 19 Toen werd aan Daniel in een nachtgezicht de verborgenheid geopenbaard; toen loofde Daniel den God des hemels.
\par 20 Daniel antwoordde en zeide: De Naam Gods zij geloofd van eeuwigheid tot in eeuwigheid, want Zijn is de wijsheid en de kracht.
\par 21 Want Hij verandert de tijden en stonden; Hij zet de koningen af, en Hij bevestigt de koningen; Hij geeft den wijzen wijsheid, en wetenschap dengenen, die verstand hebben;
\par 22 Hij openbaart diepe en verborgen dingen; Hij weet, wat in het duister is, want het licht woont bij Hem.
\par 23 Ik dank en ik loof U, o God mijner vaderen! omdat Gij mij wijsheid en kracht gegeven hebt, en mij nu bekend gemaakt hebt, wat wij van U verzocht hebben, want Gij hebt ons des konings zaak bekend gemaakt.
\par 24 Daarom ging Daniel in tot Arioch, dien de koning gesteld had om de wijzen van Babel om te brengen; hij ging henen en zeide aldus tot hem: Breng de wijzen van Babel niet om, maar breng mij in voor den koning, en ik zal den koning de uitlegging te kennen geven.
\par 25 Toen bracht Arioch met haast Daniel voor den koning, en hij sprak alzo tot hem: Ik heb een man van de gevankelijk weggevoerden van Juda gevonden, die den koning de uitlegging zal bekend maken.
\par 26 De koning antwoordde en zeide tot Daniel, wiens naam Beltsazar was: Zijt gij machtig mij bekend te maken den droom, dien ik gezien heb, en zijn uitlegging?
\par 27 Daniel antwoordde voor den koning, en zeide: De verborgenheid, die de koning eist, kunnen de wijzen, de sterrekijkers, de tovenaars, en de waarzeggers den koning niet te kennen geven;
\par 28 Maar er is een God in den hemel, Die verborgenheden openbaart, Die heeft den koning Nebukadnezar bekend gemaakt, wat er geschieden zal in het laatste der dagen; uw droom, en de gezichten uws hoofds op uw leger, zijn deze:
\par 29 Gij, o koning! op uw leger zijnde, klommen uw gedachten op, wat hierna geschieden zou; en Hij, Die verborgen dingen openbaart, heeft u te kennen gegeven, wat er geschieden zal.
\par 30 Mij nu, mij is de verborgenheid geopenbaard, niet door wijsheid, die in mij is boven alle levenden; maar daarom opdat men den koning de uitlegging zou bekend maken, en opdat gij de gedachten uws harten zoudt weten.
\par 31 Gij, o koning! zaagt, en ziet, er was een groot beeld (dit beeld was treffelijk, en deszelfs glans was uitnemend), staande tegen u over; en zijn gedaante was schrikkelijk.
\par 32 Het hoofd van dit beeld was van goed goud; zijn borst en zijn armen van zilver; zijn buik en zijn dijen van koper;
\par 33 Zijn schenkelen van ijzer; zijn voeten eensdeels van ijzer, en eensdeels van leem.
\par 34 Dit zaagt gij, totdat er een steen afgehouwen werd zonder handen, die sloeg dat beeld aan zijn voeten van ijzer en leem, en vermaalde ze.
\par 35 Toen werden te zamen vermaald het ijzer, leem, koper, zilver en goud, en zij werden gelijk kaf van de dorsvloeren des zomers, en de wind nam ze weg, en er werd geen plaats voor dezelve gevonden; maar de steen, die het beeld geslagen heeft, werd tot een groten berg, alzo dat hij de gehele aarde vervulde.
\par 36 Dit is de droom; zijn uitlegging nu zullen wij voor den koning zeggen.
\par 37 Gij, o koning! zijt een koning der koningen; want de God des hemels heeft u een koninkrijk, macht, en sterkte, en eer gegeven;
\par 38 En overal, waar mensenkinderen wonen, heeft Hij de beesten des velds en de vogelen des hemels in uw hand gegeven; en Hij heeft u gesteld tot een heerser over al dezelve; gij zijt dat gouden hoofd.
\par 39 En na u zal een ander koninkrijk opstaan, lager dan het uwe; daarna een ander, het derde koninkrijk van koper, hetwelk heersen zal over de gehele aarde.
\par 40 En het vierde koninkrijk zal hard zijn, gelijk ijzer; aangezien het ijzer alles vermaalt en verzwakt; gelijk nu het ijzer, dat zulks alles verbreekt, alzo zal het vermalen en verbreken.
\par 41 En dat gij gezien hebt de voeten en de tenen, ten dele van pottenbakkersleem, en ten dele van ijzer, dat zal een gedeeld koninkrijk zijn, doch daar zal van des ijzers vastigheid in zijn, ten welken aanzien gij gezien hebt ijzer vermengd met modderig leem;
\par 42 En de tenen der voeten, ten dele ijzer, en ten dele leem; dat koninkrijk zal ten dele hard zijn, en ten dele broos.
\par 43 En dat gij gezien hebt ijzer vermengd met modderig leem, zij zullen zich wel door menselijk zaad vermengen, maar zij zullen de een aan den ander niet hechten, gelijk als zich ijzer met leem niet vermengt.
\par 44 Doch in de dagen van die koningen zal de God des hemels een Koninkrijk verwekken, dat in der eeuwigheid niet zal verstoord worden; en dat Koninkrijk zal aan geen ander volk overgelaten worden; het zal al die koninkrijken vermalen, en te niet doen, maar zelf zal het in alle eeuwigheid bestaan.
\par 45 Daarom hebt gij gezien, dat uit den berg een steen zonder handen afgehouwen is geworden, die het ijzer, koper, leem, zilver en goud vermaalde; de grote God heeft den koning bekend gemaakt, wat hierna geschieden zal; de droom nu is gewis, en zijn uitlegging is zeker.
\par 46 Toen viel de koning Nebukadnezar op zijn aangezicht, en aanbad Daniel; en hij zeide, dat men hem met spijsoffer en liefelijk reukwerk een drankoffer doen zou.
\par 47 De koning antwoordde Daniel en zeide: Het is de waarheid, dat ulieder God een God der goden is, en een Heere der koningen, en Die de verborgenheden openbaart, dewijl gij deze verborgenheid hebt kunnen openbaren.
\par 48 Toen maakte de koning Daniel groot, en hij gaf hem vele grote geschenken, en hij stelde hem tot een heerser over het ganse landschap van Babel, en een overste der overheden over al de wijzen van Babel.
\par 49 Toen verzocht Daniel van den koning; en hij stelde Sadrach, Mesach en Abed-nego over de bediening van het landschap van Babel; maar Daniel bleef aan de poort des konings.

\chapter{3}

\par 1 De koning Nebukadnezar maakte een beeld van goud, welks hoogte was zestig ellen, zijn breedte zes ellen; hij richtte het op in het dal Dura, in het landschap van Babel.
\par 2 En de koning Nebukadnezar zond henen, om te verzamelen, de stadhouders, de overheden, en de landvoogden, de wethouders, de schatmeesters, de raadsheren, de ambtlieden, en al de heerschappers der landschappen, dat zij komen zouden tot de inwijding van het beeld, hetwelk de koning Nebukadnezar had opgericht.
\par 3 Toen verzamelden zich de stadhouders, de overheden, de landvoogden, de wethouders, de schatmeesters, de raadsheren, de ambtlieden, en al de heerschappers der landschappen, tot inwijding van het beeld, hetwelk de koning Nebukadnezar had opgericht; en zij stonden voor het beeld, dat Nebukadnezar had opgericht.
\par 4 En een heraut riep met kracht: Men zegt u aan, gij volken, gij natien, en tongen!
\par 5 Ten tijde als gij horen zult het geluid des hoorns, der pijp, der citer, der vedel, der psalteren, des akkoordgezangs, en allerlei soorten van muziek, zo zult gijlieden nedervallen, en aanbidden het gouden beeld, hetwelk de koning Nebukadnezar heeft opgericht;
\par 6 En wie niet nedervalt en aanbidt, die zal te dierzelfder ure in het midden van den oven des brandenden vuurs geworpen worden.
\par 7 Daarom te dier tijd, als al die volken hoorden het geluid des hoorns, der pijp, der citer, der vedel, der psalteren, en allerlei soorten der muziek, alle volken, natien, en tongen nedervallende, aanbaden het gouden beeld, hetwelk de koning Nebukadnezar had opgericht.
\par 8 Daarom naderden even ter zelfder tijd Chaldeeuwse mannen, die de Joden openlijk beschuldigden;
\par 9 Zij antwoordden en zeiden tot den koning Nebukadnezar: O koning! leef in der eeuwigheid!
\par 10 Gij, o koning! hebt een bevel gegeven, dat alle mensen, die horen zouden het geluid des hoorns, der pijp, der citer, der vedel, der psalteren, en des akkoordgezangs, en allerlei soorten van muziek, nedervallen, en het gouden beeld aanbidden zouden;
\par 11 En wie niet nederviel, en aanbad, die zou in het midden van den oven des brandenden vuurs geworpen worden.
\par 12 Er zijn Joodse mannen, die gij over de bediening van het landschap van Babel gesteld hebt, Sadrach, Mesach en Abed-nego; deze mannen hebben, o koning! op u geen acht gesteld; uw goden eren zij niet, en zij bidden het gouden beeld niet aan, hetwelk gij opgericht hebt.
\par 13 Toen zeide Nebukadnezar in toorn en grimmigheid, dat men Sadrach, Mesach en Abed-nego voorbrengen zou; toen werden die mannen voor den koning gebracht.
\par 14 Nebukadnezar antwoordde en zeide tot hen: Is het met opzet, Sadrach, Mesach en Abed-nego, dat gijlieden mijn goden niet eert, en het gouden beeld, dat ik opgericht heb, niet aanbidt?
\par 15 Nu dan, zo gijlieden gereed zijt, dat gij ten tijde, als gij horen zult het geluid des hoorns, der pijp, der citer, der vedel, der psalteren, en des akkoordgezangs, en allerlei soort der muziek, nedervalt, en aanbidt het beeld, dat ik gemaakt heb, zo is het wel; maar zo gijlieden het niet aanbidt; ter zelfder ure zult gijlieden geworpen worden in het midden van den oven des brandenden vuurs; en wie is de God, Die ulieden uit mijn handen verlossen zou?
\par 16 Sadrach, Mesach en Abed-nego antwoordden en zeiden tot den koning Nebukadnezar: Wij hebben niet nodig u op deze zaak te antwoorden.
\par 17 Zal het zo zijn, onze God, Dien wij eren, is machtig ons te verlossen uit den oven des brandenden vuurs, en Hij zal ons uit uw hand, o koning! verlossen.
\par 18 Maar zo niet, u zij bekend, o koning! dat wij uw goden niet zullen eren, noch het gouden beeld, dat gij hebt opgericht, zullen aanbidden.
\par 19 Toen werd Nebukadnezar vol grimmigheid, en de gedaante zijns aangezichts veranderde tegen Sadrach, Mesach en Abed-nego; hij antwoordde en zeide, dat men den oven zevenmaal meer heet zou maken dan men dien pleegt heet te maken.
\par 20 En tot de sterkste mannen van kracht, die in zijn heir waren, zeide hij, dat zij Sadrach, Mesach en Abed-nego binden zouden, om te werpen in den oven des brandenden vuurs.
\par 21 Toen werden die mannen gebonden in hun mantels, hun broeken, en hun hoeden, en hun andere klederen, en zij wierpen hen in het midden van den oven des brandenden vuurs.
\par 22 Daarom dan, dewijl het woord des konings aandreef, en de oven zeer heet was, zo hebben de vonken des vuurs die mannen, die Sadrach, Mesach en Abed-nego opgeheven hadden, gedood.
\par 23 Maar als die drie mannen, Sadrach, Mesach en Abed-nego, in het midden van den oven des brandenden vuurs, gebonden zijnde, gevallen waren,
\par 24 Toen ontzette zich de koning Nebukadnezar, en hij stond op in der haast, antwoordde en zeide tot zijn raadsheren: Hebben wij niet drie mannen in het midden des vuurs, gebonden zijnde, geworpen? Zij antwoordden en zeiden tot den koning: Het is gewis, o koning!
\par 25 Hij antwoordde en zeide: Ziet, ik zie vier mannen, los wandelende in het midden des vuurs, en er is geen verderf aan hen; en de gedaante des vierden is gelijk eens zoons der goden.
\par 26 Toen naderde Nebukadnezar tot de deur van den oven des brandenden vuurs, antwoordde en sprak: Gij Sadrach, Mesach en Abed-nego, gij knechten des allerhoogsten Gods! gaat uit en komt hier! Toen gingen Sadrach, Mesach en Abed-nego uit het midden des vuurs.
\par 27 Toen vergaderden de stadhouders, de overheden, en de landvoogden, en de raadsheren des konings, deze mannen beziende, omdat het vuur over hun lichamen niet geheerst had, en dat het haar huns hoofds niet verbrand was, en hun mantels niet veranderd waren, ja, dat de reuk des vuurs daardoor niet gegaan was.
\par 28 Nebukadnezar antwoordde en zeide: Geloofd zij de God van Sadrach, Mesach en Abed-nego, Die Zijn engel gezonden, en Zijn knechten verlost heeft, die op Hem vertrouwd hebben, en des konings woord veranderd, en hun lichamen overgegeven hebben, opdat zij geen god eerden noch aanbaden, dan hun God.
\par 29 Daarom wordt van mij een bevel gegeven, dat alle volk, natie en tong, die lastering spreekt tegen den God van Sadrach, Mesach en Abed-nego, in stukken gehouwen worde, en zijn huis tot een drekhoop gesteld worde; want er is geen ander God, Die alzo verlossen kan.
\par 30 Toen maakte de koning Sadrach, Mesach en Abed-nego voorspoedig in het landschap van Babel.

\chapter{4}

\par 1 De koning Nebukadnezar aan alle volken, natien en tongen, die op den gansen aardbodem wonen: uw vrede worde vermenigvuldigd!
\par 2 Het behaagt mij te verkondigen de tekenen en wonderen, die de allerhoogste God aan mij gedaan heeft.
\par 3 Hoe groot zijn Zijn tekenen! en hoe machtig Zijn wonderen! Zijn Rijk is een eeuwig Rijk, en Zijn heerschappij is van geslacht tot geslacht.
\par 4 Ik, Nebukadnezar, gerust zijnde in mijn huis, en in mijn paleis groenende,
\par 5 Zag een droom, die mij vervaarde, en de gedachten, die ik op mijn bed had, en de gezichten mijns hoofds beroerden mij.
\par 6 Daarom is er een bevel van mij gesteld, dat men voor mij zou inbrengen al de wijzen van Babel, opdat zij mij de uitlegging van dien droom zouden bekend maken.
\par 7 Toen kwamen in de tovenaars, de sterrekijkers, de Chaldeen en de waarzeggers; en ik zeide den droom voor hen; maar zij maakten mij zijn uitlegging niet bekend;
\par 8 Totdat ten laatste Daniel voor mij inkwam, wiens naam Beltsazar is, naar den naam mijns gods, in wien ook de geest der heilige goden is; en ik vertelde den droom voor hem, zeggende:
\par 9 Beltsazar, gij overste der tovenaars! dewijl ik weet, dat de geest der heilige goden in u is, en geen verborgenheid u zwaar is, zo zeg de gezichten mijns drooms, dien ik gezien heb, te weten zijn uitlegging.
\par 10 De gezichten nu mijns hoofds op mijn leger waren deze: Ik zag, en ziet, er was een boom in het midden der aarde, en zijn hoogte was groot.
\par 11 De boom werd groot en sterk; en zijn hoogte reikte aan den hemel, en hij werd gezien tot aan het einde der ganse aarde;
\par 12 Zijn loof was schoon, en zijn vruchten vele, en er was spijze aan dezelve voor allen; onder hem vond het gedierte des velds schaduw, en de vogelen des hemels woonden in zijn takken, en alle vlees werd daarvan gevoed.
\par 13 Ik zag verder in de gezichten mijns hoofds, op mijn leger; en ziet, een wachter, namelijk een heilige, kwam af van den hemel,
\par 14 Roepende met kracht, en aldus zeggende: Houwt dien boom af, en kapt zijn takken af; stroopt zijn loof af, en verstrooit zijn vruchten, dat de dieren van onder hem wegzwerven, en de vogelen van zijn takken;
\par 15 Doch laat den stam met zijn wortelen in de aarde, en met een ijzeren en koperen band in het tedere gras des velds; en laat hem in den dauw des hemels nat gemaakt worden, en zijn deel zij met het gedierte in het kruid der aarde.
\par 16 Zijn hart worde veranderd, dat het geens mensen hart meer zij, en hem worde eens beesten hart gegeven, en laat zeven tijden over hem voorbijgaan.
\par 17 Deze zaak is in het besluit der wachters, en deze begeerte is in het woord der heiligen; opdat de levenden bekennen, dat de Allerhoogste heerschappij heeft over de koninkrijken der mensen, en geeft ze aan wien Hij wil, ja, zet daarover den laagste onder de mensen.
\par 18 Dezen droom heb ik, koning Nebukadnezar gezien; gij nu, Beltsazar! zeg de uitlegging van dien, dewijl als de wijzen mijns koninkrijks mij de uitlegging niet hebben kunnen bekend maken; maar gij kunt wel, dewijl de geest der heilige goden in u is.
\par 19 Toen ontzette zich Daniel, wiens naam Beltsazar is, bij een uur lang, en zijn gedachten beroerden hem. De koning antwoordde en zeide: Beltsazar! laat u de droom en zijn uitlegging niet beroeren. Beltsazar antwoordde en zeide: Mijn heer! de droom wedervare uw hateren, en zijn uitlegging uw wederpartijders!
\par 20 De boom, dien gij gezien hebt, die groot en sterk geworden was, en wiens hoogte tot aan den hemel reikte, en die over het ganse aardrijk gezien werd;
\par 21 En wiens loof schoon, en wiens vruchten vele waren, en waar spijze aan was voor allen, onder wien het gedierte des velds woonde, en in wiens takken de vogelen des hemels nestelden;
\par 22 Dat zijt gij, o koning! die groot en sterk zijt geworden; want uw grootheid is zo gewassen, dat zij reikt aan den hemel, en uw heerschappij aan het einde des aardrijks.
\par 23 Dat nu de koning, een wachter, namelijk een heilige gezien heeft, van den hemel afkomende, die zeide: Houwt dezen boom af, en verderft hem; doch laat den stam met zijn wortelen in de aarde, en met een ijzeren en koperen band in het tedere gras des velds, en in de dauw des hemels nat gemaakt worden, en dat zijn deel zij met het gedierte des velds, totdat er zeven tijden over hem voorbijgaan;
\par 24 Dit is de beduiding, o koning! en dit is een besluit des Allerhoogsten, hetwelk over mijn heer, den koning, komen zal:
\par 25 Te weten, men zal u van de mensen verstoten, en met het gedierte des velds zal uw woning zijn, en men zal u het kruid, als den ossen, te smaken geven; en gij zult van den dauw des hemels nat gemaakt worden, en er zullen zeven tijden over u voorbijgaan, totdat gij bekent, dat de Allerhoogste heerschappij heeft over de koninkrijken der mensen, en geeft ze, wien Hij wil.
\par 26 Dat er ook gezegd is, dat men den stam met de wortelen van dien boom laten zou; uw koninkrijk zal u bestendig zijn, nadat gij zult bekend hebben, dat de Hemel heerst.
\par 27 Daarom, o koning! laat mijn raad u behagen, en breek uw zonden af door gerechtigheid, en uw ongerechtigheden door genade te bewijzen aan de ellendigen, of er verlenging van uw vrede mocht wezen.
\par 28 Dit alles overkwam den koning Nebukadnezar.
\par 29 Want op het einde van twaalf maanden, toen hij op het koninklijk paleis van Babel wandelde,
\par 30 Sprak de koning, en zeide: Is dit niet het grote Babel, dat ik gebouwd heb tot een huis des koninkrijks, door de sterkte mijner macht, en ter ere mijner heerlijkheid!
\par 31 Dit woord nog zijnde in des konings mond, viel er een stem uit den hemel: U, o koning Nebukadnezar! wordt gezegd: Het koninkrijk is van u gegaan.
\par 32 En men zal u van de mensen verstoten, en uw woning zal bij de beesten des velds zijn; men zal u gras te smaken geven, als den ossen, en er zullen zeven tijden over u voorbijgaan, totdat gij bekent, dat de Allerhoogste over de koninkrijken der mensen heerschappij heeft, en dat Hij ze geeft, aan wien Hij wil.
\par 33 Ter zelfder ure werd dat woord volbracht over Nebukadnezar, want hij werd uit de mensen verstoten, en hij at gras als de ossen, en zijn lichaam werd van den dauw des hemels nat gemaakt, totdat zijn haar wies als der arenden vederen, en zijn nagelen als der vogelen.
\par 34 Ten einde dezer dagen nu, hief ik, Nebukadnezar, mijn ogen op ten hemel, want mijn verstand kwam weer in mij; en ik loofde den Allerhoogste, en ik prees en verheerlijkte den Eeuwiglevende, omdat Zijn heerschappij is een eeuwige heerschappij, en Zijn Koninkrijk is van geslacht tot geslacht;
\par 35 En al de inwoners der aarde zijn als niets geacht, en Hij doet naar Zijn wil met het heir des hemels en de inwoners der aarde, en er is niemand, die Zijn hand afslaan, of tot Hem zeggen kan: Wat doet Gij?
\par 36 Ter zelfder tijd kwam mijn verstand weder in mij; ook kwam de heerlijkheid mijns koninkrijks, mijn majesteit en mijn glans weder op mij; en mijn raadsheren en mijn geweldigen zochten mij, en ik werd in mijn koninkrijk bevestigd; en mij werd groter heerlijkheid toegevoegd.
\par 37 Nu prijs ik, Nebukadnezar, en verhoog, en verheerlijk den Koning des hemels, omdat al Zijn werken waarheid, en Zijn paden gerichten zijn; en Hij is machtig te vernederen degenen, die in hoogmoed wandelen.

\chapter{5}

\par 1 De koning Belsazar maakte een groten maaltijd voor zijn duizend geweldigen, en hij dronk wijn voor die duizend.
\par 2 Als Belsazar den wijn geproefd had, zeide hij, dat men de gouden en zilveren vaten voorbrengen zou, die zijn vader Nebukadnezar uit den tempel, die te Jeruzalem geweest was, weggevoerd had; opdat de koning en zijn geweldigen, zijn vrouwen en zijn bijwijven uit dezelve dronken.
\par 3 Toen bracht men voor de gouden vaten, die men uit den tempel van het huis Gods, die te Jeruzalem geweest was, weggevoerd had; en de koning en zijn geweldigen, zijn vrouwen, en zijn bijwijven dronken daaruit.
\par 4 Zij dronken den wijn, en prezen de gouden, en de zilveren, de koperen, de ijzeren, de houten en de stenen goden.
\par 5 Ter zelfder ure kwamen er vingeren van eens mensen hand voort, die schreven tegenover den kandelaar, op de kalk van den wand van het koninklijk paleis, en de koning zag het deel der hand, die daar schreef.
\par 6 Toen veranderde zich de glans des konings, en zijn gedachten verschrikten hem; en de banden zijner lendenen werden los, en zijn knieen stieten tegen elkander aan.
\par 7 Zodat de koning met kracht riep dat men de sterrekijkers, de Chaldeen en de waarzeggers inbrengen zou; en de koning antwoordde en zeide tot de wijzen van Babel: Alle man, die dit schrift lezen, en deszelfs uitlegging mij te kennen zal geven, die zal met purper gekleed worden, met een gouden keten om zijn hals, en hij zal de derde heerser in dit koninkrijk zijn.
\par 8 Toen kwamen al de wijzen des konings in; maar zij konden dit schrift niet lezen, noch den koning deszelfs uitlegging bekend maken.
\par 9 Toen verschrikte de koning Belsazar zeer, en zijn glans werd aan hem veranderd, en zijn geweldigen werden verbaasd.
\par 10 Om deze woorden des konings en zijner geweldigen, ging de koningin in het huis des maaltijds. De koningin sprak en zeide: O koning, leef in eeuwigheid! laat u uw gedachten niet verschrikken, en uw glans niet veranderd worden.
\par 11 Er is een man in uw koninkrijk, in wien de geest der heilige goden is, want in de dagen uws vaders is bij hem gevonden licht, en verstand, en wijsheid, gelijk de wijsheid der goden is; daarom stelde hem de koning Nebukadnezar, uw vader, tot een overste der tovenaars, der sterrekijkers, der Chaldeen, en der waarzeggers, uw vader, o koning!
\par 12 Omdat een voortreffelijke geest, en wetenschap, en verstand van een, die dromen uitlegt, en der aanwijzing van raadselen, en van een, die knopen ontbindt, gevonden werd in hem, in Daniel, dien de koning den naam van Beltsazar gaf; laat nu Daniel geroepen worden, die zal de uitlegging te kennen geven.
\par 13 Toen werd Daniel voor den koning ingebracht. De koning antwoordde en zeide tot Daniel: Zijt gij die Daniel, een uit de gevankelijk weggevoerden van Juda, die de koning, mijn vader, uit Juda gebracht heeft?
\par 14 Ik heb toch van u gehoord, dat de geest der goden in u is, en dat er licht, en verstand, en voortreffelijke wijsheid in u gevonden wordt.
\par 15 Nu, zo zijn voor mij ingebracht de wijzen en de sterrekijkers, om dit schrift te lezen, en deszelfs uitlegging mij bekend te maken; maar zij kunnen de uitlegging dezer woorden niet te kennen geven.
\par 16 Doch van u heb ik gehoord, dat gij uitleggingen kunt geven, en knopen ontbinden; nu, indien gij dit schrift zult kunnen lezen, en deszelfs uitlegging mij bekend maken, gij zult met purper bekleed worden, met een gouden keten om uw hals, en gij zult de derde heerser in dit koninkrijk zijn.
\par 17 Toen antwoordde Daniel, en zeide voor den koning: Heb uw gaven voor uzelven, en geef uw vereringen aan een ander; ik zal nochtans het schrift voor den koning lezen, en de uitlegging zal ik hem bekend maken.
\par 18 Wat u aangaat, o koning! de allerhoogste God heeft uw vader Nebukadnezar het koninkrijk, en grootheid, en eer, en heerlijkheid gegeven;
\par 19 En vanwege de grootheid, die Hij hem gegeven had, beefden en sidderden alle volken, natien en tongen voor hem; dien hij wilde, doodde hij, en dien hij wilde, behield hij in het leven, en dien hij wilde, verhoogde hij, en dien hij wilde, vernederde hij.
\par 20 Maar toen zich zijn hart verhief, en zijn geest verstijfd werd ter hovaardij, werd hij van den troon zijns koninkrijks afgestoten, en men nam de eer van hem weg.
\par 21 En hij werd van de kinderen der mensen verstoten, en zijn hart werd den beesten gelijk gemaakt, en zijn woning was bij de woudezelen; men gaf hem gras te smaken gelijk den ossen; en zijn lichaam werd van den dauw des hemels nat gemaakt, totdat hij bekende, dat God, de Allerhoogste, Heerser is over de koninkrijken der mensen, en over dezelve stelt, wien Hij wil.
\par 22 En gij, Belsazar, zijn zoon! hebt uw hart niet vernederd, alhoewel gij dit alles wel geweten hebt.
\par 23 Maar gij hebt u verheven tegen den Heere des hemels, en men heeft de vaten van Zijn huis voor u gebracht, en gij, en uw geweldigen, uw vrouwen, en uw bijwijven hebben wijn uit dezelve gedronken, en de goden van zilver en goud, koper, ijzer, hout en steen, die niet zien, noch horen, noch weten, hebt gij geprezen; maar dien God, in Wiens hand uw adem is, en bij Wien al uw paden zijn, hebt gij niet verheerlijkt.
\par 24 Toen is dat deel der hand van Hem gezonden, en dit schrift getekend geworden.
\par 25 Dit nu is het schrift, dat daar getekend is: MENE, MENE, TEKEL, UPHARSIN.
\par 26 Dit is de uitlegging dezer woorden: MENE; God heeft uw koninkrijk geteld, en Hij heeft het voleind.
\par 27 TEKEL; gij zijt in weegschalen gewogen; en gij zijt te licht gevonden.
\par 28 PERES; uw koninkrijk is verdeeld, en het is den Meden en den Perzen gegeven.
\par 29 Toen beval Belsazar, en zij bekleedden Daniel met purper, met een gouden keten om zijn hals, en zij riepen overluid van hem, dat hij de derde heerser in dat koninkrijk was.
\par 30 In dienzelfden nacht, werd Belsazar, der Chaldeen koning, gedood.

\chapter{6}

\par 1 Darius, de Meder nu, ontving het koninkrijk, omtrent twee en zestig jaren oud zijnde.
\par 2 En het dacht Darius goed, dat hij over het koninkrijk stelde honderd en twintig stadhouders, die over het ganse koninkrijk zijn zouden;
\par 3 En over dezelve drie vorsten, van dewelke Daniel de eerste zou zijn, denwelken die stadhouders zelfs zouden rekenschap geven, opdat de koning geen schade leed.
\par 4 Toen overtrof deze Daniel die vorsten en die stadhouders, daarom dat een voortreffelijke geest in hem was; en de koning dacht hem te stellen over het gehele koninkrijk.
\par 5 Toen zochten de vorsten en de stadhouders gelegenheid te vinden, tegen Daniel vanwege het koninkrijk; maar zij konden geen gelegenheid noch misdaad vinden, dewijl hij getrouw was, en geen vergrijping noch misdaad in hem gevonden werd.
\par 6 Toen zeiden die mannen: Wij zullen tegen dezen Daniel geen gelegenheid vinden, tenzij wij tegen hem iets vinden in de wet zijns Gods.
\par 7 Zo kwamen deze vorsten en de stadhouders met hopen tot den koning, en zeiden aldus tot hem: O koning Darius, leef in eeuwigheid!
\par 8 Al de vorsten des rijks, de overheden en stadhouders, de raadsheren en landvoogden hebben zich beraadslaagd een koninklijke ordonnantie te stellen, en een sterk gebod te maken, dat al wie in dertig dagen een verzoek zal doen van enigen god of mens, behalve van u, o koning! die zal in den kuil der leeuwen geworpen worden.
\par 9 Nu, o koning! gij zult een gebod bevestigen, en een schrift tekenen, dat niet veranderd worde, naar de wet der Meden en der Perzen, die niet mag wederroepen worden.
\par 10 Daarom tekende de koning Darius dat schrift en gebod.
\par 11 Toen nu Daniel verstond, dat dit schrift getekend was, ging hij in zijn huis (hij nu had in zijn opperzaal open vensters tegen Jeruzalem aan), en hij knielde drie tijden 's daags op zijn knieen, en hij bad, en deed belijdenis voor zijn God, ganselijk gelijk hij voor dezen gedaan had.
\par 12 Toen kwamen die mannen met hopen, en zij vonden Daniel biddende en smekende voor zijn God.
\par 13 Toen kwamen zij nader, en spraken voor den koning van het gebod des konings: Hebt gij niet een gebod getekend, dat alle man, die in dertig dagen van enigen god of mens iets verzoeken zou, behalve van u, o koning! in den kuil der leeuwen zou geworpen worden? De koning antwoordde en zeide: Het is een vaste rede, naar de wet der Meden en Perzen, die niet mag herroepen worden.
\par 14 Toen antwoordden zij, en zeiden voor den koning: Daniel, een van de gevankelijk weggevoerden uit Juda heeft, o koning! op u geen acht gesteld, noch op het gebod dat gij getekend hebt; maar hij bidt op drie tijden 's daags zijn gebed.
\par 15 Toen de koning deze rede hoorde, was hij zeer bedroefd bij zichzelven, en hij stelde het hart op Daniel om hem te verlossen; ja, tot den ondergang der zon toe bemoeide hij zich, om hem te redden.
\par 16 Toen kwamen die mannen met hopen tot den koning, en zij zeiden tot den koning: Weet, o koning! dat der Meden en der Perzen wet is, dat geen gebod noch ordonnantie, die de koning verordend heeft, mag veranderd worden.
\par 17 Toen beval de koning, en zij brachten Daniel voor, en wierpen hem in den kuil der leeuwen; en de koning antwoordde en zeide tot Daniel: Uw God, Dien gij geduriglijk eert, Die verlosse u!
\par 18 En er werd een steen gebracht, en op den mond des kuils gelegd: en de koning verzegelde denzelven met zijn ring, en met den ring zijner geweldigen, opdat de wil aangaande Daniel niet zou veranderd worden.
\par 19 Toen ging de koning naar zijn paleis, en overnachtte nuchteren, en liet geen vreugdespel voor zich brengen; en zijn slaap week verre van hem.
\par 20 Toen stond de koning in den vroegen morgenstond met het licht op, en hij ging met haast henen tot den kuil der leeuwen.
\par 21 Als hij nu tot den kuil genaderd was, riep hij tot Daniel met een droeve stem; de koning antwoordde en zeide tot Daniel: O Daniel, gij knecht des levenden Gods! heeft ook uw God, Dien gij geduriglijk eert, u van de leeuwen kunnen verlossen?
\par 22 Toen sprak Daniel tot den koning: O koning, leef in eeuwigheid!
\par 23 Mijn God heeft Zijn engel gezonden, en Hij heeft den muil der leeuwen toegesloten, dat zij mij niet beschadigd hebben, omdat voor Hem onschuld in mij gevonden is; ook heb ik, o koning! tegen u geen misdaad gedaan.
\par 24 Toen werd de koning bij zichzelven zeer vrolijk, en zeide, dat men Daniel uit den kuil trekken zou. Toen Daniel uit den kuil opgetrokken was, zo werd er geen schade aan hem gevonden, dewijl hij in zijn God geloofd had.
\par 25 Toen beval de koning, en zij brachten die mannen voor, die Daniel overluid beschuldigd hadden, en zij wierpen in den kuil der leeuwen hen, hun kinderen, en hun vrouwen; en zij kwamen niet op den grond des kuils, of de leeuwen heersten over hen, zij vermorzelden ook al hun beenderen.
\par 26 Toen schreef de koning Darius aan alle volken, natien en tongen, die op de ganse aarde woonden: Uw vrede worde vermenigvuldigd!
\par 27 Van mij is een bevel gegeven, dat men in de ganse heerschappij mijns koninkrijks beve en siddere voor het aangezicht van den God van Daniel; want Hij is de levende God, en bestendig in eeuwigheden, en Zijn koninkrijk is niet verderfelijk, en Zijn heerschappij is tot het einde toe.
\par 28 Hij verlost en redt, en Hij doet tekenen en wonderen in den hemel en op de aarde; Die heeft Daniel uit het geweld der leeuwen verlost.
\par 29 Deze Daniel nu had voorspoed in het koninkrijk van Darius, en in het koninkrijk van Kores, den Perziaan.

\chapter{7}

\par 1 In het eerste jaar van Belsazar, den koning van Babel, zag Daniel een droom, en gezichten zijns hoofds, op zijn leger; toen schreef hij dien droom, en hij zeide de hoofdsom der zaken.
\par 2 Daniel antwoordde en zeide: Ik zag in mijn gezicht bij nacht, en ziet, de vier winden des hemels braken voort op de grote zee.
\par 3 En er klommen vier grote dieren op uit de zee, het ene van het andere verscheiden.
\par 4 Het eerste was als een leeuw, en het had arendsvleugelen; ik zag toe, totdat zijn vleugelen uitgeplukt waren, en het werd van de aarde opgeheven, en op de voeten gesteld, als een mens, en aan hetzelve werd eens mensen hart gegeven.
\par 5 Daarna, ziet, het andere dier, het tweede, was gelijk een beer, en stelde zich aan de ene zijde, en het had drie ribben in zijn muil tussen zijn tanden; en men zeide aldus tot hetzelve: Sta op, eet veel vlees.
\par 6 Daarna zag ik, en ziet, er was een ander dier, gelijk een luipaard, en het had vier vleugels eens vogels op zijn rug; ook had hetzelve dier vier hoofden, en aan hetzelve werd de heerschappij gegeven.
\par 7 Daarna zag ik in de nachtgezichten, en ziet, het vierde dier was schrikkelijk en gruwelijk, en zeer sterk; en het had grote ijzeren tanden, het at, en verbrijzelde, en vertrad het overige met zijn voeten; en het was verscheiden van al de dieren, die voor hetzelve geweest waren; en het had tien hoornen.
\par 8 Ik nam acht op de hoornen, en ziet, een andere kleine hoorn kwam op tussen dezelve, en drie uit de vorige hoornen werden uitgerukt voor denzelven; en ziet, in dienzelven hoorn waren ogen als mensenogen, en een mond, grote dingen sprekende.
\par 9 Dit zag ik, totdat er tronen gezet werden, en de Oude van dagen Zich zette, Wiens kleed wit was als de sneeuw, en het haar Zijns hoofds als zuivere wol; Zijn troon was vuurvonken, deszelfs raderen een brandend vuur.
\par 10 Een vurige rivier vloeide, en ging van voor Hem uit, duizendmaal duizenden dienden Hem, en tien duizendmaal tien duizenden stonden voor Hem; het gericht zette zich, en de boeken werden geopend.
\par 11 Toen zag ik toe vanwege de stem der grote woorden, welke die hoorn sprak; ik zag toe, totdat het dier gedood, en zijn lichaam verdaan werd, en overgegeven om van het vuur verbrand te worden.
\par 12 Aangaande ook de overige dieren, men nam hun heerschappij weg, want verlenging van het leven was hun gegeven tot tijd en stonde toe.
\par 13 Verder zag ik in de nachtgezichten, en ziet, er kwam Een met de wolken des hemels, als eens mensen zoon, en Hij kwam tot den Oude van dagen, en zij deden Hem voor Denzelven naderen.
\par 14 En Hem werd gegeven heerschappij, en eer, en het Koninkrijk, dat Hem alle volken, natien en tongen eren zouden; Zijn heerschappij is een eeuwige heerschappij, die niet vergaan zal, en Zijn Koninkrijk zal niet verdorven worden.
\par 15 Mij Daniel werd mijn geest doorstoken in het midden van het lichaam, en de gezichten mijns hoofds verschrikten mij.
\par 16 Ik naderde tot een dergenen, die daar stonden, en verzocht van hem de zekerheid over dit alles; en hij zeide ze mij, en gaf mij de uitlegging dezer zaken te kennen.
\par 17 Deze grote dieren, die vier zijn, zijn vier koningen, die uit de aarde opstaan zullen.
\par 18 Maar de heiligen der hoge plaatsen zullen dat Koninkrijk ontvangen, en zij zullen het Rijk bezitten tot in der eeuwigheid, ja, tot in eeuwigheid der eeuwigheden.
\par 19 Toen wenste ik naar de waarheid van het vierde dier, hetwelk verscheiden was van al de andere, zeer gruwelijk, welks tanden van ijzer waren, en zijn klauwen van koper; het at, het verbrijzelde, en vertrad het overige met zijn voeten.
\par 20 En aangaande de tien hoornen die op zijn hoofd waren, en den anderen, die opkwam, en voor denwelken drie afgevallen waren, namelijk dien hoorn, die ogen had, en een mond, die grote dingen sprak, en wiens aanzien groter was, dan van zijn metgezellen.
\par 21 Ik had gezien, dat diezelve hoorn krijg voerde tegen de heiligen, en dat hij die overmocht,
\par 22 Totdat de Oude van dagen kwam, en het gericht gegeven werd aan de heiligen der hoge plaatsen, en dat de bestemde tijd kwam, dat de heiligen het Rijk bezaten.
\par 23 Hij zeide aldus: Het vierde dier zal het vierde rijk op aarde zijn, dat verscheiden zal zijn van al die rijken, en het zal de ganse aarde opeten, en het zal dezelve vertreden, en het zal ze verbrijzelen.
\par 24 Belangende nu de tien hoornen: uit dat koninkrijk zullen tien koningen opstaan, en een ander zal na hen opstaan; en dat zal verscheiden zijn van de vorigen, en het zal drie koningen vernederen.
\par 25 En het zal woorden spreken tegen den Allerhoogsten, en het zal de heiligen der hoge plaatsen verstoren, en het zal menen de tijden en de wet te veranderen, en zij zullen in deszelfs hand overgegeven worden tot een tijd, en tijden, en een gedeelte eens tijds.
\par 26 Daarna zal het gericht zitten, en men zal zijn heerschappij wegnemen, hem verdelgende en verdoende, tot het einde toe.
\par 27 Maar het rijk, en de heerschappij, en de grootheid der koninkrijken onder den gansen hemel, zal gegeven worden den volke der heiligen der hoge plaatsen, welks Rijk een eeuwig Rijk zijn zal; en alle heerschappijen zullen Hem eren en gehoorzamen.
\par 28 Tot hiertoe is het einde dezer rede. Wat mij Daniel aangaat, mijn gedachten verschrikten mij zeer, en mijn glans veranderde aan mij; doch ik bewaarde dat woord in mijn hart.

\chapter{8}

\par 1 In het derde jaar des koninkrijks van den koning Belsazar, verscheen mij een gezicht, mij Daniel, na hetgeen mij in het eerste verschenen was.
\par 2 En ik zag een gezicht, (het geschiedde nu, toen ik het zag, dat ik in den burg Susan was, welke in het landschap Elam is) ik zag dan in een gezicht, dat ik aan den vloed Ulai was.
\par 3 En ik hief mijn ogen op, en ik zag, en ziet, een ram stond voor dien vloed, die had twee hoornen, en die twee hoornen waren hoog, en de een was hoger dan de andere, en de hoogste kwam in het laatste op.
\par 4 Ik zag, dat de ram met de hoornen tegen het westen stiet, en tegen het noorden, en tegen het zuiden, en geen dieren konden voor zijn aangezicht bestaan, en er was niemand, die uit zijn hand verloste; maar hij deed naar zijn welgevallen, en hij maakte zich groot.
\par 5 Toen ik dit overlegde, ziet, er kwam een geitenbok van het westen over den gansen aardbodem, en roerde de aarde niet aan; en die bok had een aanzienlijken hoorn tussen zijn ogen.
\par 6 En hij kwam tot den ram, die de twee hoornen had, dien ik had zien staan voor den vloed; en hij liep op hem aan in de grimmigheid zijner kracht.
\par 7 En ik zag hem, nakende aan den ram, en hij verbitterde zich tegen hem, en hij stiet den ram, en hij brak zijn beide hoornen; en in den ram was geen kracht, om voor zijn aangezicht te bestaan; en hij wierp hem ter aarde, en hij vertrad hem, en er was niemand, die den ram uit zijn hand verloste.
\par 8 En de geitenbok maakte zich uitermate groot; maar toen hij sterk geworden was, brak die grote hoorn, en er kwamen op aan deszelfs plaats vier aanzienlijke, naar de vier winden des hemels.
\par 9 En uit een van die kwam voort een kleine hoorn, welke uitnemend groot werd, tegen het zuiden, en tegen het oosten, en tegen het sierlijke land.
\par 10 En hij werd groot tot aan het heir des hemels; en hij wierp er sommigen van dat heir, namelijk van de sterren, ter aarde neder, en hij vertrad ze.
\par 11 Ja, hij maakte zich groot tot aan den Vorst diens heirs, en van Denzelven werd weggenomen het gedurig offer, en de woning Zijns heiligdoms werd nedergeworpen.
\par 12 En het heir werd in den afval overgegeven tegen het gedurig offer; en hij wierp de waarheid ter aarde; en deed het, en het gelukte wel.
\par 13 Daarna hoorde ik een heilige spreken; en de heilige zeide tot den onbenoemde, die daar sprak: Tot hoelang zal dat gezicht van het gedurig offer en van den verwoestenden afval zijn, dat zo het heiligdom als het heir ter vertreding zal overgegeven worden?
\par 14 En hij zeide tot mij: Tot twee duizend en driehonderd avonden en morgens; dan zal het heiligdom gerechtvaardigd worden.
\par 15 En het geschiedde, toen ik dat gezicht zag, ik Daniel, zo zocht ik het verstand deszelven, en ziet, er stond voor mij als de gedaante eens mans.
\par 16 En ik hoorde tussen Ulai eens mensen stem, die riep en zeide: Gabriel! geef dezen het gezicht te verstaan.
\par 17 En hij kwam nevens waar ik stond; en als hij kwam, verschrikte ik, en viel op mijn aangezicht. Toen zeide hij tot mij: Versta, gij mensenkind! want dit gezicht zal zijn tot den tijd van het einde.
\par 18 Als hij nu met mij sprak, viel ik in een diepen slaap op mijn aangezicht ter aarde; toen roerde hij mij aan, en hij stelde mij op mijn standplaats.
\par 19 En hij zeide: Zie, ik zal u te kennen geven, wat er geschieden zal ten einde dezer gramschap; want ter bestemder tijd zal het einde zijn.
\par 20 De ram met de twee hoornen, dien gij gezien hebt, zijn de koningen der Meden en der Perzen.
\par 21 Die harige bok nu, is de koning van Griekenland; en de grote hoorn, welke tussen zijn ogen is, is de eerste koning.
\par 22 Dat er nu vier aan zijn plaats stonden, toen hij verbroken was; vier koninkrijken zullen uit dat volk ontstaan, doch niet met zijn kracht.
\par 23 Doch op het laatste huns koninkrijks, als het de afvalligen op het hoogste gebracht zullen hebben, zo zal er een koning staan, stijf van aangezicht, en raadselen verstaande;
\par 24 En zijn kracht zal sterk worden, doch niet door zijn kracht; en hij zal het wonderlijk verderven, en zal geluk hebben, en zal het doen; en hij zal de sterken, mitsgaders het heilige volk verderven;
\par 25 En door zijn kloekheid zo zal hij de bedriegerij doen gedijen in zijn hand; en hij zal zich in zijn hart verheffen; en in stille rust zal hij er velen verderven, en zal staan tegen den Vorst der vorsten, doch hij zal zonder hand verbroken worden.
\par 26 Het gezicht nu van avond en morgen, dat er gezegd is, is de waarheid; en gij, sluit dit gezicht toe, want er zijn nog vele dagen toe.
\par 27 Toen werd ik, Daniel, zwak, en was enige dagen krank; daarna stond ik op, en deed des konings werk; en ik was ontzet over dit gezicht; maar niemand merkte het.

\chapter{9}

\par 1 In het eerste jaar van Darius, den zoon van Ahasveros, uit het zaad der Meden, die koning gemaakt was over het koninkrijk der Chaldeen;
\par 2 In het eerste jaar zijner regering, merkte ik, Daniel, in de boeken, dat het getal der jaren, van dewelke het woord des HEEREN tot den profeet Jeremia geschied was, in het vervullen der verwoestingen van Jeruzalem, zeventig jaren was.
\par 3 En ik stelde mijn aangezicht tot God, den Heere, om Hem te zoeken met het gebed, en smekingen, met vasten, en zak, en as.
\par 4 Ik bad dan tot den HEERE, mijn God, en deed belijdenis, en zeide: Och Heere! Gij grote en verschrikkelijke God, Die het verbond en de weldadigheid houdt dien, die Hem liefhebben en Zijn geboden houden.
\par 5 Wij hebben gezondigd, en hebben onrecht gedaan, en goddelooslijk gehandeld, en gerebelleerd, met af te wijken van Uw geboden, en van Uw rechten.
\par 6 En wij hebben niet gehoord naar Uw dienstknechten, de profeten, die in Uw Naam spraken tot onze koningen, onze vorsten en onze vaders, en tot al het volk des lands.
\par 7 Bij U, o Heere! is de gerechtigheid, maar bij ons de beschaamdheid der aangezichten, gelijk het is te dezen dage; bij de mannen van Juda, en de inwoners van Jeruzalem, en geheel Israel, die nabij en die verre zijn, in al de landen, waar Gij ze henengedreven hebt, om hun overtreding, waarmede zij tegen U overtreden hebben.
\par 8 O Heere! bij ons is de beschaamdheid der aangezichten, bij onze koningen, bij onze vorsten, en bij onze vaders, omdat wij tegen U gezondigd hebben.
\par 9 Bij den Heere, onzen God, zijn de barmhartigheden en vergevingen, alhoewel wij tegen Hem gerebelleerd hebben.
\par 10 En wij hebben der stem des HEEREN, onzes Gods, niet gehoorzaamd, dat wij in Zijn wetten wandelen zouden, die Hij gegeven heeft voor onze aangezichten, door de hand van Zijn knechten, de profeten.
\par 11 Maar geheel Israel heeft Uw wet overtreden, met af te wijken, dat zij Uwer stem niet gehoorzaamden; daarom is over ons uitgestort die vloek, en die eed, die geschreven is in de wet van Mozes, den knecht Gods, dewijl wij tegen Hem gezondigd hebben.
\par 12 En Hij heeft Zijn woorden bevestigd, die Hij gesproken heeft tegen ons, en tegen onze richters, die ons richtten, brengende over ons een groot kwaad, hetwelk niet geschied is onder den gansen hemel, gelijk aan Jeruzalem geschied is.
\par 13 Gelijk als in de wet van Mozes geschreven is, alzo is al dat kwaad over ons gekomen; en wij smeekten het aangezicht des HEEREN, onzes Gods, niet, afkerende van onze ongerechtigheden, en verstandelijk acht gevende op Uw waarheid.
\par 14 Daarom heeft de HEERE over het kwade gewaakt, en Hij heeft het over ons gebracht; want de HEERE, onze God, is rechtvaardig in al Zijn werken, die Hij gedaan heeft, dewijl wij Zijner stem niet gehoorzaamden.
\par 15 En nu, o Heere, onze God! Die Uw volk uit Egypteland gevoerd hebt, met een sterke hand, en hebt U een Naam gemaakt, gelijk hij is te dezen dage; wij hebben gezondigd, wij zijn goddeloos geweest.
\par 16 O Heere! naar al Uw gerechtigheden, laat toch Uw toorn en Uw grimmigheid afgekeerd worden van Uw stad Jeruzalem, Uw heiligen berg; want om onzer zonden wil en om onzer vaderen ongerechtigheden, zijn Jeruzalem en Uw volk tot versmaadheid bij allen, die rondom ons zijn.
\par 17 En nu, o onze God! hoor naar het gebed Uws knechts, en naar zijn smekingen; en doe Uw aangezicht lichten over Uw heiligdom, dat verwoest is; om des Heeren wil.
\par 18 Neig Uw oor, mijn God! en hoor, doe Uw ogen op, en zie onze verwoestingen, en de stad, die naar Uw Naam genoemd is; want wij werpen onze smekingen voor Uw aangezicht niet neder op onze gerechtigheden, maar op Uw barmhartigheden, die groot zijn.
\par 19 O Heere, hoor! o Heere, vergeef! o Heere, merk op en doe het, vertraag het niet! Om Uws Zelfs wil, o mijn God! Want Uw stad, en Uw volk is naar Uw Naam genoemd.
\par 20 Als ik nog sprak, en bad, en beleed mijn zonde, en de zonde mijns volks van Israel, en mijn smeking nederwierp voor het aangezicht des HEEREN, mijns Gods, om des heiligen bergs wil mijns Gods;
\par 21 Als ik nog sprak in het gebed, zo kwam de man Gabriel, dien ik in het begin in een gezicht gezien had, snellijk gevlogen, mij aanrakende, omtrent den tijd des avondoffers.
\par 22 En hij onderrichtte mij en sprak met mij, en zeide: Daniel! nu ben ik uitgegaan, om u den zin te doen verstaan.
\par 23 In het begin uwer smekingen is het woord uitgegaan, en ik ben gekomen, om u dat te kennen te geven; want gij zijt een zeer gewenst man; versta dan dit woord, en merk op dit gezicht.
\par 24 Zeventig weken zijn bestemd over uw volk, en over uw heilige stad, om de overtreding te sluiten, en om de zonden te verzegelen, en om de ongerechtigheid te verzoenen, en om een eeuwige gerechtigheid aan te brengen, en om het gezicht, en den profeet te verzegelen, en om de heiligheid der heiligheden te zalven.
\par 25 Weet dan, en versta: van den uitgang des woords, om te doen wederkeren, en om Jeruzalem te bouwen, tot op Messias, den Vorst, zijn zeven weken, en twee en zestig weken; de straten, en de grachten zullen wederom gebouwd worden, doch in benauwdheid der tijden.
\par 26 En na die twee en zestig weken zal de Messias uitgeroeid worden, maar het zal niet voor Hem zelven zijn; en een volk des vorsten, hetwelk komen zal, zal de stad en het heiligdom verderven, en zijn einde zal zijn met een overstromenden vloed, en tot het einde toe zal er krijg zijn, en vastelijk besloten verwoestingen.
\par 27 En hij zal velen het verbond versterken een week; en in de helft der week zal hij het slachtoffer en het spijsoffer doen ophouden, en over den gruwelijken vleugel zal een verwoester zijn, ook tot de voleinding toe, die vastelijk besloten zijnde, zal uitgestort worden over den verwoeste.

\chapter{10}

\par 1 In het derde jaar van Kores, den koning van Perzie, werd aan Daniel, wiens naam genoemd werd Beltsazar, een zaak geopenbaard, en die zaak is de waarheid, doch in een gezetten groten tijd; en hij verstond die zaak, en hij had verstand van het gezicht.
\par 2 In die dagen was ik, Daniel, treurende drie weken der dagen.
\par 3 Begeerlijke spijze at ik niet, en vlees of wijn kwam in mijn mond niet; ook zalfde ik mij gans niet, totdat die drie weken der dagen vervuld waren.
\par 4 En op den vier en twintigsten dag der eerste maand, zo was ik aan den oever der grote rivier, welke is Hiddekel.
\par 5 En ik hief mijn ogen op, en zag, en ziet, er was een Man met linnen bekleed, en Zijn lenden waren omgord met fijn goud van Ufaz.
\par 6 En Zijn lichaam was gelijk een turkoois, en Zijn aangezicht gelijk de gedaante des bliksems, en Zijn ogen gelijk vurige fakkelen, en Zijn armen en Zijn voeten gelijk de verf van gepolijst koper; en de stem Zijner woorden was gelijk de stem ener menigte.
\par 7 En ik, Daniel, alleen zag dat gezicht, maar de mannen, die bij mij waren, zagen dat gezicht niet; doch een grote verschrikking viel op hen, en zij vloden, om zich te versteken.
\par 8 Ik dan werd alleen overgelaten, en zag dit grote gezicht, en er bleef in mij geen kracht overig; en mijn sierlijkheid werd aan mij veranderd in een verderving, zodat ik geen kracht behield.
\par 9 En ik hoorde de stem Zijner woorden; en toen ik de stem Zijner woorden hoorde, zo viel ik in een diepen slaap op mijn aangezicht, met mijn aangezicht ter aarde.
\par 10 En ziet, een hand roerde mij aan, en maakte, dat ik mij bewoog op mijn knieen, en de palmen mijner handen.
\par 11 En Hij zeide tot mij: Daniel, gij zeer gewenste man! merk op de woorden, die Ik tot u spreken zal, en sta op uw standplaats, want Ik ben alnu tot u gezonden; en toen Hij dat woord tot mij sprak, stond ik bevende.
\par 12 Toen zeide Hij tot mij: Vrees niet, Daniel! want van den eersten dag aan, dat gij uw hart begaaft, om te verstaan en om uzelven te verootmoedigen, voor het aangezicht uws Gods, zijn uw woorden gehoord, en om uwer woorden wil ben Ik gekomen.
\par 13 Doch de vorst des koninkrijks van Perzie stond tegenover Mij een en twintig dagen; en ziet, Michael, een van de eerste vorsten, kwam om Mij te helpen, en Ik werd aldaar gelaten bij de koningen van Perzie.
\par 14 Nu ben Ik gekomen, om u te doen verstaan, hetgeen uw volk bejegenen zal in het vervolg der dagen, want het gezicht is nog voor vele dagen.
\par 15 En toen Hij deze woorden met mij sprak, sloeg ik mijn aangezicht ter aarde, en ik werd stom.
\par 16 En ziet, Een, den mensenkinderen gelijk, raakte mijn lippen aan, toen deed ik mijn mond open, en ik sprak, en zeide tot Dien, Die tegenover mij stond: Mijn Heere! om des gezichts wil keren zich mijn weeen over mij, zodat ik geen kracht behoude.
\par 17 En hoe kan de knecht van dezen mijn Heere spreken met dien mijn Heere? Want wat mij aangaat, van nu af bestaat geen kracht in mij, en geen adem is in mij overgebleven.
\par 18 Toen raakte mij wederom aan Een, als in de gedaante van een mens; en Hij versterkte mij.
\par 19 En Hij zeide: Vrees niet, gij zeer gewenste man! vrede zij u, wees sterk, ja, wees sterk! En terwijl Hij met mij sprak, werd ik versterkt, en zeide: Mijn Heere spreke, want Gij hebt mij versterkt.
\par 20 Toen zeide Hij: Weet gij, waarom dat Ik tot u gekomen ben? Doch nu zal Ik wederkeren om te strijden tegen den vorst der Perzen; en als Ik zal uitgegaan zijn, ziet, zo zal de vorst van Griekenland komen.
\par 21 Doch Ik zal u te kennen geven, hetgeen getekend is in het geschrift der waarheid; en er is niet een, die zich met Mij versterkt tegen dezen, dan uw vorst Michael.

\chapter{11}

\par 1 Ik nu, ik stond in het eerste jaar van Darius, den Meder, om hem te versterken en te stijven.
\par 2 En nu, ik zal u de waarheid te kennen geven; ziet, er zullen nog drie koningen in Perzie staan, en de vierde zal verrijkt worden met groten rijkdom, meer dan al de anderen; en nadat hij zich in zijn rijkdom zal versterkt hebben, zal hij ze allen verwekken tegen het koninkrijk van Griekenland.
\par 3 Daarna zal er een geweldig koning opstaan, die met grote heerschappij heersen zal, en hij zal doen naar zijn welgevallen.
\par 4 En als hij zal staan, zal zijn rijk gebroken, en in de vier winden des hemels verdeeld worden, maar niet aan zijn nakomelingen, ook niet naar zijn heerschappij, waarmede hij heerste; want zijn rijk zal uitgerukt worden, en dat voor anderen, dan deze.
\par 5 En de koning van het Zuiden, die een van zijn vorsten is, zal sterk worden; doch een ander zal sterker worden dan hij, en hij zal heersen; zijn heerschappij zal een grote heerschappij zijn.
\par 6 Op het einde nu van sommige jaren, zullen zij zich met elkander bevrienden, en de dochter des konings van het Zuiden zal komen tot de koning van het Noorden, om billijke voorwaarden te maken; doch zij zal de macht des arms niet behouden, daarom zal hij, noch zijn arm, niet bestaan; maar zij zal overgegeven worden, en die haar gebracht hebben, en die haar gegenereerd heeft, en die haar gesterkt heeft in die tijden.
\par 7 Doch uit de spruit van haar wortelen zal er een opstaan in zijn staat, die zal met heirkracht komen, en hij zal komen tegen die sterke plaatsen des konings van het Noorden, en hij zal tegen dezelve doen, en hij zal ze bemachtigen.
\par 8 Ook zal hij hun goden, met hun vorsten, met hun gewenste vaten van zilver en goud, in de gevangenis naar Egypte brengen; en hij zal enige jaren staande blijven boven den koning van het Noorden.
\par 9 Alzo zal de koning van het Zuiden in het koninkrijk komen, en hij zal wederom in zijn land trekken.
\par 10 Doch zijn zonen zullen zich in strijd mengen, en zij zullen een menigte van grote heiren verzamelen; en een van hen zal snellijk komen, en als een vloed overstromen en doortrekken; en hij zal wederom komen, en zich in den strijd mengen, tot aan zijn sterke plaats toe.
\par 11 En de koning van het Zuiden zal verbitterd worden, en hij zal uittrekken, en strijden tegen hem, tegen den koning van het Noorden, die ook een grote menigte oprichten zal, doch die menigte zal in zijn hand gegeven worden.
\par 12 Als die menigte zal weggenomen zijn, zal zijn hart zich verheffen, en hij zal er enige tien duizenden nedervellen; evenwel zal hij niet gesterkt worden.
\par 13 Want de koning van het Noorden zal wederkeren, en hij zal een groter menigte dan de eerste was, oprichten; en aan het einde van de tijden der jaren, zal hij snellijk komen met een grote heirkracht, en met groot goed.
\par 14 Ook zullen er in die tijden velen opstaan tegen den koning van het Zuiden; en de scheurmakers uws volks zullen verheven worden, om het gezicht te bevestigen, doch zij zullen vallen.
\par 15 En de koning van het Noorden zal komen, en een wal opwerpen, en vaste steden innemen; en de armen van het Zuiden zullen niet bestaan, noch zijn uitgelezen volk, ja, er zal geen kracht zijn om te bestaan.
\par 16 Maar hij, die tegen hem komt, zal doen naar zijn welgevallen, en niemand zal voor zijn aangezicht bestaan; hij zal ook staan in het land des sieraads, en de verderving zal in zijn hand wezen.
\par 17 En hij zal zijn aangezicht stellen, om met de kracht zijns gansen rijks te komen, en hij zal billijke voorwaarden medebrengen, en hij zal het doen; want hij zal hem een dochter der vrouwen geven, om haar te verderven, maar zij zal niet vast staan, en zij zal voor hem niet zijn.
\par 18 Daarna zal hij zijn aangezicht tot de eilanden keren, en hij zal er vele innemen; doch een overste zal zijn smaad tegen hem doen ophouden, behalve dat hij zijn smaad op hem zal doen wederkeren.
\par 19 En hij zal zijn aangezicht keren naar de sterkten zijns lands, en hij zal aanstoten, en vallen, en niet gevonden worden.
\par 20 En in zijn staat zal er een opstaan, doende een geldeiser doortrekken, in koninklijke heerlijkheid; maar hij zal in enige dagen gebroken worden, nochtans niet door toornigheden, noch door oorlog.
\par 21 Daarna zal er een verachte in zijn staat staan, denwelken men de koninklijke waardigheid niet zal geven; doch hij zal in stilheid komen, en het koninkrijk door vleierijen bemachtigen.
\par 22 En de armen der overstroming zullen overstroomd worden van voor zijn aangezicht, en zij zullen gebroken worden, en ook de vorst des verbonds.
\par 23 En na de vereniging met hem zal hij bedrog plegen, en hij zal optrekken, en hij zal met weinig volks gesterkt worden.
\par 24 Met stilheid zal hij ook in de vette plaatsen des landschaps komen, en hij zal doen, dat zijn vaders, of de vaders zijner vaderen, niet gedaan hebben; roof, en buit, en goederen, zal hij onder hen uitstrooien, en hij zal tegen de vastigheden zijn gedachten denken, doch tot een zekeren tijd toe.
\par 25 En hij zal zijn kracht en zijn hart verwekken tegen den koning van het Zuiden, met een grote heirkracht; en de koning van het Zuiden zal zich in den strijd mengen met een grote en zeer machtige heirkracht; doch hij zal niet bestaan, want zij zullen gedachten tegen hem denken.
\par 26 En die de stukken zijner spijze zullen eten, zullen hem breken, en de heirkracht deszelven zal overstromen, en vele verslagenen zullen vallen.
\par 27 En het hart van beide deze koningen zal wezen om kwaad te doen, en aan een tafel zullen zij leugen spreken; en het zal niet gelukken, want het zal nog een einde hebben ter bestemder tijd.
\par 28 En hij zal in zijn land wederkeren met groot goed, en zijn hart zal zijn tegen het heilig verbond; en hij zal het doen, en wederkeren in zijn land.
\par 29 Ter bestemder tijd zal hij wederkeren, en tegen het Zuiden komen, doch het zal niet zijn gelijk de eerste, noch gelijk de laatste reize.
\par 30 Want er zullen schepen van Chittim tegen hem komen, daarom zal hij met smart bevangen worden, en hij zal wederkeren, en gram worden tegen het heilig verbond, en hij zal het doen; want wederkerende zal hij acht geven op de verlaters des heiligen verbonds.
\par 31 En er zullen armen uit hem ontstaan, en zij zullen het heiligdom ontheiligen, en de sterkte, en zij zullen het gedurige offer wegnemen, en een verwoestenden gruwel stellen.
\par 32 En die goddelooslijk handelen tegen het verbond, zal hij doen huichelen door vleierijen; maar het volk, die hun God kennen, zullen zij grijpen, en zullen het doen.
\par 33 En de leraars des volks zullen er velen onderwijzen, en zij zullen vallen door het zwaard en door vlam, door gevangenis en door beroving, vele dagen.
\par 34 Als zij nu zullen vallen, zullen zij met een kleine hulp geholpen worden; doch velen zullen zich door vleierijen tot hen vervoegen.
\par 35 En van de leraars zullen er sommigen vallen, om hen te louteren en te reinigen, en wit te maken, tot den tijd van het einde toe; want het zal nog zijn voor een bestemden tijd.
\par 36 En die koning zal doen naar zijn welgevallen, en hij zal zichzelven verheffen, en groot maken boven allen God, en hij zal tegen den God der goden wonderlijke dingen spreken; en hij zal voorspoedig zijn, totdat de gramschap voleind zij, want het is vastelijk besloten, het zal geschieden.
\par 37 En op de goden zijner vaderen zal hij geen acht geven, noch op de begeerte der vrouwen; hij zal ook op geen God acht geven, maar hij zal zich boven alles groot maken.
\par 38 En hij zal den god Mauzzim in zijn standplaats eren; namelijk den god, welken zijn vaders niet gekend hebben, zal hij eren met goud, en met zilver, en met kostelijk gesteente, en met gewenste dingen.
\par 39 En hij zal de vastigheden der sterkten maken met den vreemden god; dengenen, die hij kennen zal, zal hij de eer vermenigvuldigen, en hij zal ze doen heersen over velen, en hij zal het land uitdelen om prijs.
\par 40 En op den tijd van het einde, zal de koning van het Zuiden tegen hem met hoornen stoten; en de koning van het Noorden zal tegen hem aanstormen, met wagenen, en met ruiteren, en met vele schepen; en hij zal in de landen komen, en hij zal ze overstromen en doortrekken.
\par 41 En hij zal komen in het land des sieraads, en vele landen zullen ter nedergeworpen worden; doch deze zullen zijn hand ontkomen, Edom en Moab, en de eerstelingen der kinderen Ammons.
\par 42 En hij zal zijn hand aan de landen leggen, ook zal het land van Egypte niet ontkomen.
\par 43 En hij zal heersen over de verborgen schatten des gouds en des zilvers, en over al de gewenste dingen van Egypte; en die van Libye, en de Moren zullen in zijn gangen wezen.
\par 44 Maar de geruchten van het Oosten en van het Noorden zullen hem verschrikken; daarom zal hij uittrekken met grote grimmigheid om velen te verdelgen en te verbannen.
\par 45 En hij zal de tenten van zijn paleis planten tussen de zeeen aan den berg des heiligen sieraads; en hij zal tot zijn einde komen, en zal geen helper hebben.

\chapter{12}

\par 1 En te dier tijd zal Michael opstaan, die grote vorst, die voor de kinderen uws volks staat, als het zulk een tijd der benauwdheid zijn zal, als er niet geweest is, sinds dat er een volk geweest is, tot op dienzelven tijd toe; en te dier tijd zal uw volk verlost worden, al wie gevonden wordt geschreven te zijn in het boek.
\par 2 En velen van die, die in het stof der aarde slapen, zullen ontwaken, dezen ten eeuwigen leven, en genen tot versmaadheden, en tot eeuwige afgrijzing.
\par 3 De leraars nu zullen blinken, als de glans des uitspansels, en die er velen rechtvaardigen, gelijk de sterren, altoos en eeuwiglijk.
\par 4 En gij, Daniel! sluit deze woorden toe, en verzegel dit boek, tot den tijd van het einde; velen zullen het naspeuren, en de wetenschap zal vermenigvuldigd worden.
\par 5 En ik, Daniel, zag, en ziet, er stonden twee anderen, de een aan deze zijde van den oever der rivier, en de ander aan gene zijde van den oever der rivier.
\par 6 En hij zeide tot den Man, bekleed met linnen, Die boven op het water der rivier was: Tot hoe lang zal het zijn, dat er een einde van deze wonderen zal wezen?
\par 7 En ik hoorde dien Man, bekleed met linnen, Die boven op het water van de rivier was, en Hij hief Zijn rechter hand en Zijn linkerhand op naar den hemel, en zwoer bij Dien, Die eeuwiglijk leeft, dat na een bestemden tijd, bestemde tijden, en een helft, en als Hij zal voleind hebben te verstrooien de hand des heiligen volks, al deze dingen voleind zullen worden.
\par 8 Dit hoorde ik, doch ik verstond het niet; en ik zeide: Mijn Heere! wat zal het einde zijn van deze dingen?
\par 9 En Hij zeide: Ga henen, Daniel! want deze woorden zijn toegesloten en verzegeld tot den tijd van het einde.
\par 10 Velen zullen er gereinigd en wit gemaakt, en gelouterd worden; doch de goddelozen zullen goddelooslijk handelen, en geen van de goddelozen zullen het verstaan, maar de verstandigen zullen het verstaan.
\par 11 En van dien tijd af, dat het gedurig offer zal weggenomen, en de verwoestende gruwel zal gesteld zijn, zullen zijn duizend tweehonderd en negentig dagen.
\par 12 Welgelukzalig is hij, die verwacht en raakt tot duizend driehonderd vijf en dertig dagen.
\par 13 Maar gij, ga henen tot het einde, want gij zult rusten, en zult opstaan in uw lot, in het einde der dagen.



\end{document}