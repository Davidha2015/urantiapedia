\begin{document}

\title{Mark}



\chapter{1}

\par 1 Het begin des Evangelies van JEZUS CHRISTUS, den Zoon van God.
\par 2 Gelijk geschreven is in de profeten: Ziet, Ik zend Mijn engel voor Uw aangezicht, die Uw weg voor U heen bereiden zal.
\par 3 De stem des roependen in de woestijn: Bereidt den weg des Heeren, maakt Zijn paden recht.
\par 4 Johannes was dopende in de woestijn, en predikende den doop der bekering tot vergeving der zonden.
\par 5 En al het Joodse land ging tot hem uit, en die van Jeruzalem; en werden allen van hem gedoopt in de rivier de Jordaan, belijdende hun zonden.
\par 6 En Johannes was gekleed met kemelshaar, en met een lederen gordel om zijn lenden, en at sprinkhanen en wilde honig.
\par 7 En hij predikte, zeggende: Na mij komt, Die sterker is dan ik, Wien ik niet waardig ben, nederbukkende, den riem Zijner schoenen te ontbinden.
\par 8 Ik heb ulieden wel gedoopt met water, maar Hij zal u dopen met den Heiligen Geest.
\par 9 En het geschiedde in diezelfde dagen, dat Jezus kwam van Nazareth, gelegen in Galilea, en werd van Johannes gedoopt in de Jordaan.
\par 10 En terstond, als Hij uit het water opklom, zag Hij de hemelen opengaan, en den Geest, gelijk een duif, op Hem nederdalen.
\par 11 En er geschiedde een stem uit de hemelen: Gij zijt Mijn geliefde Zoon, in Denwelken Ik Mijn welbehagen heb!
\par 12 En terstond dreef Hem de Geest uit in de woestijn.
\par 13 En Hij was aldaar in de woestijn veertig dagen, verzocht van den satan; en was bij de wilde gedierten; en de engelen dienden Hem.
\par 14 En nadat Johannes overgeleverd was, kwam Jezus in Galilea, predikende het Evangelie van het Koninkrijk Gods.
\par 15 En zeggende: De tijd is vervuld, en het Koninkrijk Gods nabij gekomen; bekeert u, en gelooft het Evangelie.
\par 16 En wandelende bij de Galilese zee, zag Hij Simon en Andreas, zijn broeder, werpende het net in de zee (want zij waren vissers);
\par 17 En Jezus zeide tot hen: Volgt Mij na, en Ik zal maken, dat gij vissers der mensen zult worden.
\par 18 En zij, terstond hun netten verlatende, zijn Hem gevolgd.
\par 19 En van daar een weinig voortgegaan zijnde, zag Hij Jakobus, den zoon van Zebedeus, en Johannes, zijn broeder, en dezelven in het schip hun netten vermakende.
\par 20 En terstond riep Hij hen; en zij, latende hun vader Zebedeus in het schip, met de huurlingen, zijn Hem nagevolgd.
\par 21 En zij kwamen binnen Kapernaum; en terstond op den sabbatdag in de synagoge gegaan zijnde, leerde Hij.
\par 22 En zij versloegen zich over Zijn leer; want Hij leerde hen, als machthebbende, en niet als de Schriftgeleerden.
\par 23 En er was in hun synagoge een mens, met een onreinen geest, en hij riep uit,
\par 24 Zeggende: Laat af, wat hebben wij met U te doen, Gij Jezus Nazarener, zijt Gij gekomen om ons te verderven? Ik ken U, wie Gij zijt, namelijk de Heilige Gods.
\par 25 En Jezus bestrafte hem, zeggende: Zwijg stil, en ga uit van hem.
\par 26 En de onreine geest, hem scheurende, en roepende met een grote stem, ging uit van hem.
\par 27 En zij werden allen verbaasd, zodat zij onder elkander vraagden, zeggende: Wat is dit? Wat nieuwe leer is deze, dat Hij met macht ook den onreinen geesten gebiedt, en zij Hem gehoorzaam zijn!
\par 28 En Zijn gerucht ging terstond uit, in het gehele omliggende land van Galilea.
\par 29 En van stonde aan uit de synagoge gegaan zijnde, kwamen zij in het huis van Simon en Andreas, met Jakobus en Johannes.
\par 30 En Simons vrouws moeder lag met de koorts; en terstond zeiden zij Hem van haar.
\par 31 En Hij, tot haar gaande, vatte haar hand, en richtte ze op; en terstond verliet haar de koorts, en zij diende henlieden.
\par 32 Als het nu avond geworden was, toen de zon onderging, brachten zij tot Hem allen, die kwalijk gesteld, en van den duivel bezeten waren.
\par 33 En de gehele stad was bijeenvergaderd omtrent de deur.
\par 34 En Hij genas er velen, die door verscheidene ziekten kwalijk gesteld waren; en wierp vele duivelen uit, en liet de duivelen niet toe te spreken, omdat zij Hem kenden.
\par 35 En des morgens vroeg, als het nog diep in den nacht was, opgestaan zijnde, ging Hij uit, en ging henen in een woeste plaats, en bad aldaar.
\par 36 En Simon, en die met hem waren, zijn Hem nagevolgd.
\par 37 En zij Hem gevonden hebbende, zeiden tot Hem: Zij zoeken U allen.
\par 38 En Hij zeide tot hen: Laat ons in de bijliggende vlekken gaan, opdat Ik ook daar predike; want daartoe ben Ik uitgegaan.
\par 39 En Hij predikte in hun synagogen, door geheel Galilea, en wierp de duivelen uit.
\par 40 En tot Hem kwam een melaatse, biddende Hem, en vallende voor Hem op de knieen, en tot Hem zeggende: Indien Gij wilt, Gij kunt mij reinigen.
\par 41 En Jezus, met barmhartigheid innerlijk bewogen zijnde, strekte de hand uit, en raakte hem aan, en zeide tot hem: Ik wil, word gereinigd!
\par 42 En als Hij dit gezegd had, ging de melaatsheid terstond van hem, en hij werd gereinigd.
\par 43 En als Hij hem strengelijk verboden had, deed Hij hem terstond van Zich gaan;
\par 44 En zeide tot hem: Zie, dat gij niemand iets zegt; maar ga heen en vertoon uzelven den priester, en offer voor uw reiniging, hetgeen Mozes geboden heeft, hun tot een getuigenis.
\par 45 Maar hij uitgegaan zijnde, begon vele dingen te verkondigen, en dat woord te verbreiden, alzo dat Hij niet meer openbaar in de stad kon komen, maar was buiten in de woeste plaatsen; en zij kwamen tot Hem van alle kanten.

\chapter{2}

\par 1 En na sommige dagen is Hij wederom binnen Kapernaum gekomen; en het werd gehoord, dat Hij in huis was.
\par 2 En terstond vergaderden daar velen, alzo dat ook zelfs de plaatsen omtrent de deur hen niet meer konden bevatten; en Hij sprak het woord tot hen.
\par 3 En er kwamen sommigen tot Hem, brengende een geraakte, die van vier gedragen werd.
\par 4 En niet kunnende tot Hem genaken, overmits de schare, ontdekten zij het dak, waar Hij was; en dat opgebroken hebbende, lieten zij het beddeken neder, daar de geraakte op lag.
\par 5 En Jezus, hun geloof ziende, zeide tot den geraakte: Zoon, uw zonden zijn u vergeven.
\par 6 En sommigen van de Schriftgeleerden zaten aldaar, en overdachten in hun harten:
\par 7 Wat spreekt Deze aldus gods lasteringen? Wie kan de zonden vergeven, dan alleen God?
\par 8 En Jezus, terstond in Zijn geest bekennende, dat zij alzo in zichzelven overdachten, zeide tot hen: Wat overdenkt gij deze dingen in uw harten?
\par 9 Wat is lichter, te zeggen tot den geraakte: De zonden zijn u vergeven, of te zeggen: Sta op, en neem uw beddeken op, en wandel?
\par 10 Doch opdat gij moogt weten, dat de Zoon des mensen macht heeft, om de zonden op de aarde te vergeven (zeide Hij tot den geraakte):
\par 11 Ik zeg u: Sta op, en neem uw beddeken op, en ga heen naar uw huis.
\par 12 En terstond stond hij op, en het beddeken opgenomen hebbende, ging hij uit in aller tegenwoordigheid; zodat zij zich allen ontzetten en verheerlijkten God, zeggende: Wij hebben nooit zulks gezien!
\par 13 En Hij ging wederom uit naar de zee; en de gehele schare kwam tot Hem, en Hij leerde hen.
\par 14 En voorbijgaande zag Hij Levi, den zoon van Alfeus zitten in het tolhuis, en zeide tot hem: Volg Mij. En hij opstaande, volgde Hem.
\par 15 En het geschiedde, als Hij aanzat in deszelfs huis, dat ook vele tollenaren en zondaren aanzaten met Jezus en Zijn discipelen; want zij waren velen, en waren Hem gevolgd.
\par 16 En de Schriftgeleerden en de Farizeen, ziende Hem eten met de tollenaren en zondaren, zeiden tot Zijn discipelen: Wat is het, dat Hij met de tollenaren en zondaren eet en drinkt?
\par 17 En Jezus, dat horende, zeide tot hen: Die gezond zijn, hebben den medicijnmeester niet van node, maar die ziek zijn. Ik ben niet gekomen, om te roepen rechtvaardigen, maar zondaars tot bekering.
\par 18 En de discipelen van Johannes en der Farizeen vastten; en zij kwamen en zeiden tot Hem: Waarom vasten de discipelen van Johannes en der Farizeen, en Uw discipelen vasten niet?
\par 19 En Jezus zeide tot hen: Kunnen ook de bruiloftskinderen vasten, terwijl de Bruidegom bij hen is? Zo langen tijd zij den Bruidegom bij zich hebben, kunnen zij niet vasten.
\par 20 Maar de dagen zullen komen, wanneer de Bruidegom van hen zal weggenomen zijn, en alsdan zullen zij vasten in dezelve dagen.
\par 21 En niemand naait een lap ongevold laken op een oud kleed; anders scheurt deszelfs nieuwe aangenaaide lap iets af van het oude kleed, en er wordt een ergere scheur.
\par 22 En niemand doet nieuwen wijn in oude lederzakken; anders doet de nieuwe wijn de leder zakken bersten en de wijn wordt uitgestort, en de leder zakken verderven; maar nieuwen wijn moet men in nieuwe leder zakken doen.
\par 23 En het geschiedde, dat Hij op een sabbatdag door het gezaaide ging, en Zijn discipelen begonnen, al gaande, aren te plukken.
\par 24 En de Farizeen zeiden tot Hem: Zie, waarom doen zij op den sabbatdag, wat niet geoorloofd is?
\par 25 En Hij zeide tot hen: Hebt gij nooit gelezen, wat David gedaan heeft, als hij nood had, en hem hongerde, en dengenen, die met hem waren?
\par 26 Hoe hij ingegaan is in het huis Gods, ten tijde van Abjathar, den hogepriester, en de toonbroden gegeten heeft, die niemand zijn geoorloofd te eten, dan den priesteren, en ook gegeven heeft dengenen, die met hem waren?
\par 27 En Hij zeide tot hen: De sabbat is gemaakt om den mens, niet de mens om den sabbat.
\par 28 Zo is dan de Zoon des mensen een Heere ook van den sabbat.

\chapter{3}

\par 1 En Hij ging wederom in de synagoge; en aldaar was een mens, hebbende een verdorde hand.
\par 2 En zij namen Hem waar, of Hij op den sabbat hem genezen zou, opdat zij Hem beschuldigen mochten.
\par 3 En Hij zeide tot den mens, die de verdorde hand had: Sta op in het midden.
\par 4 En Hij zeide tot hen: Is het geoorloofd op sabbatdagen goed te doen, of kwaad te doen, een mens te behouden, of te doden? En zij zwegen stil.
\par 5 En als Hij hen met toorn rondom aangezien had, meteen bedroefd zijnde over de verharding van hun hart, zeide Hij tot den mens: Strek uw hand uit. En hij strekte ze uit; en zijn hand werd hersteld, gezond gelijk de andere.
\par 6 En de Farizeen, uitgegaan zijnde, hebben terstond met de Herodeanen te zamen raad gehouden tegen Hem, hoe zij Hem doden zouden.
\par 7 En Jezus vertrok met Zijn discipelen naar de zee; en Hem volgde een grote menigte van Galilea, en van Judea,
\par 8 en van Jeruzalem, en van Idumea, en van over de Jordaan; en die van omtrent Tyrus en Sidon, een grote menigte, gehoord hebbende, hoe grote dingen Hij deed, kwamen tot Hem.
\par 9 En Hij zeide tot Zijn discipelen, dat een scheepje steeds omtrent Hem blijven zou, om der schare wil, opdat zij Hem niet zouden verdringen.
\par 10 Want Hij had er velen genezen, alzo dat Hem al degenen, die enige kwalen hadden, overvielen, opdat zij Hem mochten aanraken.
\par 11 En de onreine geesten, als zij Hem zagen, vielen voor Hem neder en riepen, zeggende: Gij zijt de Zone Gods!
\par 12 En Hij gebood hun scherpelijk dat zij Hem niet zouden openbaar maken.
\par 13 En Hij klom op den berg, en riep tot Zich, die Hij wilde; en zij kwamen tot Hem.
\par 14 En Hij stelde er twaalf, opdat zij met Hem zouden zijn, en opdat Hij dezelve zou uitzenden om te prediken;
\par 15 En om macht te hebben, de ziekten te genezen, en de duivelen uit te werpen.
\par 16 En Simon gaf Hij den toe naam Petrus;
\par 17 En Jakobus, den zoon van Zebedeus, en Johannes, den broeder van Jakobus; en gaf hun toe namen, Boanerges, hetwelk is, zonen des donders;
\par 18 En Andreas, en Filippus, en Bartholomeus, en Mattheus, en Thomas, en Jakobus, den zoon van Alfeus, en Thaddeus, en Simon Kananites,
\par 19 En Judas Iskariot, die Hem ook verraden heeft.
\par 20 En zij kwamen in huis; en daar vergaderde wederom een schare, alzo dat zij ook zelfs niet konden brood eten.
\par 21 En als degenen, die Hem bestonden, dit hoorden, gingen zij uit, om Hem vast te houden; want zij zeiden: Hij is buiten Zijn zinnen.
\par 22 En de Schriftgeleerden, die van Jeruzalem afgekomen waren, zeiden: Hij heeft Beelzebul, en door den overste der duivelen werpt Hij de duivelen uit.
\par 23 En hen tot Zich geroepen hebbende, zeide Hij tot hen in gelijkenissen: Hoe kan de satan den satan uitwerpen?
\par 24 En indien een koninkrijk tegen zichzelf verdeeld is, zo kan dat koninkrijk niet bestaan.
\par 25 En indien een huis tegen zichzelf verdeeld is, zo kan dat huis niet bestaan.
\par 26 En indien de satan tegen zichzelven opstaat, en verdeeld is, zo kan hij niet bestaan, maar heeft een einde.
\par 27 Er kan niemand in het huis eens sterken ingaan en zijn vaten ontroven, indien hij niet eerst den sterke bindt; en alsdan zal hij zijn huis beroven.
\par 28 Voorwaar, Ik zeg u, dat al de zonden den kinderen der mensen zullen vergeven worden, en allerlei lasteringen, waarmede zij zullen gelasterd hebben;
\par 29 Maar zo wie zal gelasterd hebben tegen den Heiligen Geest, die heeft geen vergeving in der eeuwigheid, maar hij is schuldig des eeuwigen oordeels.
\par 30 Want zij zeiden: Hij heeft een onreinen geest.
\par 31 Zo kwamen dan Zijn broeders en Zijn moeder; en buiten staande, zonden zij tot Hem, en riepen Hem.
\par 32 En de schare zat rondom Hem; en zij zeiden tot Hem: Zie, Uw moeder en Uw broeders daar buiten zoeken U.
\par 33 En Hij antwoordde hun, zeggende: Wie is Mijn moeder, of Mijn broeders?
\par 34 En rondom overzien hebbende, die om Hem zaten, zeide Hij: Ziet, Mijn moeder en Mijn broeders.
\par 35 Want zo wie den wil van God doet, die is Mijn broeder, en Mijn zuster, en moeder.

\chapter{4}

\par 1 En Hij begon wederom te leren omtrent de zee; en er vergaderde een grote schare bij Hem, alzo dat Hij, in het schip gegaan zijnde, nederzat op de zee; en de gehele schare was op het land aan de zee.
\par 2 En Hij leerde hun veel dingen door gelijkenissen, en Hij zeide in Zijn lering tot hen:
\par 3 Hoort toe: ziet, een zaaier ging uit om te zaaien.
\par 4 En het geschiedde in het zaaien, dat het ene deel zaads viel bij den weg; en de vogelen des hemels kwamen, en aten het op.
\par 5 En het andere viel op het steenachtige, waar het niet veel aarde had; en het ging terstond op, omdat het geen diepte van aarde had.
\par 6 Maar als de zon opgegaan was, zo is het verbrand geworden, en omdat het geen wortel had, zo is het verdord.
\par 7 En het andere viel in de doornen, en de doornen wiesen op, en verstikten hetzelve, en het gaf geen vrucht.
\par 8 En het andere viel in de goede aarde, en gaf vrucht, die opging en wies; en het ene droeg dertig-,en het andere zestig-,en het andere honderd voud.
\par 9 En Hij zeide tot hen: Wie oren heeft om te horen, die hore.
\par 10 En als Hij nu alleen was, vraagden Hem degenen, die omtrent Hem waren, met de twaalven, naar de gelijkenis.
\par 11 En Hij zeide tot hen: Het is u gegeven te verstaan de verborgenheid van het Koninkrijk Gods; maar dengenen, die buiten zijn, geschieden al deze dingen door gelijkenissen;
\par 12 Opdat zij ziende zien, en niet bemerken, en horende horen, en niet verstaan; opdat zij zich niet te eniger tijd, bekeren en hun de zonden vergeven worden.
\par 13 En Hij zeide tot hen: Weet gij deze gelijkenis niet, en hoe zult gij al de gelijkenissen verstaan?
\par 14 De zaaier is, die het Woord zaait.
\par 15 En dezen zijn, die bij den weg bezaaid worden, waarin het Woord gezaaid wordt; en als zij het gehoord hebben, zo komt de satan terstond, en neemt het Woord weg, hetwelk in hun harten gezaaid was.
\par 16 En dezen zijn desgelijks, die op de steenachtige plaatsen bezaaid worden; welke, als zij het Woord gehoord hebben, terstond hetzelve met vreugde ontvangen.
\par 17 En hebben geen wortel in zichzelven, maar zijn voor een tijd; daarna, als verdrukking of vervolging komt om des Woords wil, zo worden zij terstond geergerd.
\par 18 En dezen zijn, die in de doornen bezaaid worden; namelijk degenen, die het Woord horen;
\par 19 En de zorgvuldigheden dezer wereld, en de verleiding des rijkdoms en de begeerlijkheden omtrent de andere dingen, inkomende, verstikken het Woord, en het wordt onvruchtbaar.
\par 20 En dezen zijn, die in de goede aarde bezaaid zijn, welke het Woord horen en aannemen, en dragen vruchten, het ene dertig-,en het andere zestig-,en het andere honderd voud.
\par 21 En Hij zeide tot hen: Komt ook de kaars, opdat zij onder de koornmaat of onder het bed gezet worde? Is het niet, opdat zij op den kandelaar gezet worde?
\par 22 Want er is niets verborgen, dat niet geopenbaard zal worden; en er is niets geschied, om verborgen te zijn, maar opdat het in het openbaar zou komen.
\par 23 Zo iemand oren heeft om te horen, die hore.
\par 24 En Hij zeide tot hen: Ziet, wat gij hoort. Met wat mate gij meet, zal u gemeten worden, en u, die hoort, zal meer toegelegd worden.
\par 25 Want zo wie heeft, dien zal gegeven worden; en wie niet heeft, van dien zal genomen worden, ook dat hij heeft.
\par 26 En Hij zeide: Alzo is het Koninkrijk Gods, gelijk of een mens het zaad in de aarde wierp;
\par 27 En voorts sliep, en opstond, nacht en dag; en het zaad uitsproot en lang werd, dat hij zelf niet wist, hoe.
\par 28 Want de aarde brengt van zelve vruchten voort: eerst het kruid, daarna de aar, daarna het volle koren in de aar.
\par 29 En als de vrucht zich voordoet, terstond zendt hij de sikkel daarin, omdat de oogst daar is.
\par 30 En Hij zeide: Waarbij zullen wij het Koninkrijk Gods vergelijken, of met wat gelijkenis zullen wij hetzelve vergelijken?
\par 31 Namelijk bij een mosterdzaad, hetwelk, wanneer het in de aarde gezaaid wordt, het minste is van al de zaden, die op de aarde zijn.
\par 32 En wanneer het gezaaid is, gaat het op, en wordt het meeste van al de moeskruiden, en maakt grote takken, alzo dat de vogelen des hemels onder zijn schaduw kunnen nestelen.
\par 33 En door vele zulke gelijkenissen sprak Hij tot hen het Woord, naardat zij het horen konden.
\par 34 En zonder gelijkenis sprak Hij tot hen niet; maar Hij verklaarde alles Zijn discipelen in het bijzonder.
\par 35 En op denzelfden dag, als het nu avond geworden was, zeide Hij tot hen: Laat ons overvaren aan de andere zijde.
\par 36 En zij, de schare gelaten hebbende, namen Hem mede, gelijk Hij in het schip was; en er waren nog andere scheepjes met Hem.
\par 37 En er werd een grote storm van wind, en de baren sloegen over in het schip, alzo dat het nu vol werd.
\par 38 En Hij was in het achterschip, slapende op een oorkussen; en zij wekten Hem op, en zeiden tot Hem: Meester, bekommert het U niet, dat wij vergaan?
\par 39 En Hij opgewekt zijnde, bestrafte den wind, en zeide tot de zee: Zwijg, wees stil! En de wind ging liggen, en er werd grote stilte.
\par 40 En Hij zeide tot hen: Wat zijt gij zo vreesachtig? Hebt gij geen geloof?
\par 41 En zij vreesden met grote vreze, en zeiden tot elkander: Wie is toch Deze, dat ook de wind en de zee Hem gehoorzaam zijn?

\chapter{5}

\par 1 En zij kwamen over op de andere zijde der zee, in het land der Gadarenen.
\par 2 En zo Hij uit het schip gegaan was, terstond ontmoette Hem, uit de graven, een mens met een onreinen geest;
\par 3 Dewelke zijn woning in de graven had, en niemand kon hem binden, ook zelfs niet met ketenen.
\par 4 Want hij was menigmaal met boeien en ketenen gebonden geweest, en de ketenen waren van hem in stukken getrokken, en de boeien verbrijzeld, en niemand was machtig om hem te temmen.
\par 5 En hij was altijd, nacht en dag, op de bergen en in de graven, roepende en slaande zichzelven met stenen.
\par 6 Als hij nu Jezus van verre zag, liep hij toe, en aanbad Hem.
\par 7 En met een grote stem roepende, zeide hij: Wat heb ik met U te doen, Jezus, Gij Zone Gods, des Allerhoogsten? Ik bezweer U bij God, dat Gij mij niet pijnigt!
\par 8 (Want Hij zeide tot hem: Gij onreine geest, ga uit van den mens!)
\par 9 En Hij vraagde hem: Welke is uw naam? En hij antwoordde, zeggende: Mijn naam is Legio; want wij zijn velen.
\par 10 En hij bad Hem zeer, dat Hij hen buiten het land niet wegzond.
\par 11 En aldaar aan de bergen was een grote kudde zwijnen, weidende.
\par 12 En al de duivelen baden Hem, zeggende: Zend ons in die zwijnen, opdat wij in dezelve mogen varen.
\par 13 En Jezus liet het hun terstond toe. En de onreine geesten, uitgevaren zijnde, voeren in de zwijnen; en de kudde stortte van de steilte af in de zee (daar waren er nu omtrent twee duizend), en versmoorden in de zee.
\par 14 En die de zwijnen weidden zijn gevlucht, en boodschapten zulks in de stad en op het land. En zij gingen uit, om te zien, wat het was, dat er geschied was.
\par 15 En zij kwamen tot Jezus, en zagen den bezetene zittende, en gekleed, en wel bij zijn verstand, namelijk die het legioen gehad had, en zij werden bevreesd.
\par 16 En die het gezien hadden, vertelden hun, wat den bezetene geschied was, en ook van de zwijnen.
\par 17 En zij begonnen Hem te bidden, dat Hij van hun landpalen wegging.
\par 18 En als Hij in het schip ging, bad Hem degene, die bezeten was geweest, dat hij met Hem mocht zijn.
\par 19 Doch Jezus liet hem dat niet toe, maar zeide tot hem: Ga heen naar uw huis tot de uwen, en boodschap hun, wat grote dingen u de Heere gedaan heeft, en hoe Hij Zich uwer ontfermd heeft.
\par 20 En hij ging heen, en begon te verkondigen in het land van Dekapolis, wat grote dingen hem Jezus gedaan had; en zij verwonderden zich allen.
\par 21 En als Jezus wederom in het schip overgevaren was aan de andere zijde, vergaderde een grote schare bij Hem; en Hij was bij de zee.
\par 22 En ziet, er kwam een van de oversten der synagoge, met name Jairus; en Hem ziende, viel hij aan Zijn voeten,
\par 23 En bad Hem zeer, zeggende: Mijn dochtertje is in haar uiterste; ik bid U, dat Gij komt en de handen op haar legt, opdat zij behouden worde, en zij zal leven.
\par 24 En Hij ging met hem; en een grote schare volgde Hem, en zij verdrongen Hem.
\par 25 En een zekere vrouw, die twaalf jaren den vloed des bloeds gehad had,
\par 26 En veel geleden had van vele medicijnmeesters, en al het hare daaraan ten koste gelegd en geen baat gevonden had, maar met welke het veeleer erger geworden was;
\par 27 Deze van Jezus horende, kwam onder de schare van achteren, en raakte Zijn kleed aan;
\par 28 Want zij zeide: Indien ik maar Zijn klederen mag aanraken, zal ik gezond worden.
\par 29 En terstond is de fontein haars bloeds opgedroogd, en zij gevoelde aan haar lichaam, dat zij van die kwaal genezen was.
\par 30 En terstond Jezus, bekennende in Zichzelven de kracht, die van Hem uitgegaan was, keerde Zich om in de schare, en zeide: Wie heeft Mijn klederen aangeraakt?
\par 31 En Zijn discipelen zeiden tot Hem: Gij ziet, dat de schare U verdringt, en zegt Gij: Wie heeft Mij aangeraakt?
\par 32 En Hij zag rondom om haar te zien, die dat gedaan had.
\par 33 En de vrouw, vrezende en bevende, wetende, wat aan haar geschied was, kwam en viel voor Hem neder, en zeide Hem al de waarheid.
\par 34 En Hij zeide tot haar: Dochter, uw geloof heeft u behouden; ga heen in vrede, en zijt genezen van deze uw kwaal.
\par 35 Terwijl Hij nog sprak, kwamen enigen van het huis des oversten der synagoge, zeggende: Uw dochter is gestorven; wat zijt gij den Meester nog moeilijk?
\par 36 En Jezus, terstond gehoord hebbende het woord, dat er gesproken werd, zeide tot den overste der synagoge: Vrees niet; geloof alleenlijk.
\par 37 En Hij liet niemand toe Hem te volgen, dan Petrus, en Jakobus, en Johannes, den broeder van Jakobus;
\par 38 En kwam in het huis des oversten der synagoge; en zag de beroerte en degenen, die zeer weenden en huilden.
\par 39 En ingegaan zijnde, zeide Hij tot hen: Wat maakt gij beroerte, en wat weent gij? Het kind is niet gestorven, maar het slaapt.
\par 40 En zij belachten Hem; maar Hij, als Hij hen allen had uitgedreven, nam bij Zich den vader en de moeder des kinds, en degenen die met Hem waren, en ging binnen, waar het kind lag.
\par 41 En Hij vatte de hand des kinds, en zeide tot haar: Talitha kumi! hetwelk is, zijnde overgezet: Gij dochtertje (Ik zeg u), sta op.
\par 42 En terstond stond het dochtertje op, en wandelde; want het was twaalf jaren oud; en zij ontzetten zich met grote ontzetting.
\par 43 En Hij gebood hun zeer, dat niemand datzelve zou weten; en zeide, dat men haar zou te eten geven.

\chapter{6}

\par 1 En Hij ging van daar weg, en kwam in Zijn vaderland, en Zijn discipelen volgden Hem.
\par 2 En als het sabbat geworden was, begon Hij in de synagoge te leren; en velen, die Hem hoorden, ontzetten zich, zeggende: Van waar komen Dezen deze dingen, en wat wijsheid is dit, die Hem gegeven is, dat ook zulke krachten door Zijn handen geschieden?
\par 3 Is deze niet de timmerman, de zoon van Maria, en de broeder van Jakobus en Joses, en van Judas en Simon, en zijn Zijn zusters niet hier bij ons? En zij werden aan Hem geergerd.
\par 4 En Jezus zeide tot hen: Een profeet is niet ongeeerd dan in zijn vaderland en onder zijn magen, en in zijn huis.
\par 5 En Hij kon aldaar geen kracht doen; dan Hij legde weinigen zieken de handen op, en genas hen.
\par 6 En Hij verwonderde Zich over hun ongeloof, en omging de vlekken daar rondom, lerende.
\par 7 En Hij riep tot Zich de twaalven, en begon hen uit te zenden twee en twee, en gaf hun macht over de onreine geesten.
\par 8 En Hij gebood hun, dat zij niets zouden nemen tot den weg, dan alleenlijk een staf, geen male, geen brood, geen geld in den gordel;
\par 9 Maar dat zij schoenzolen zouden aanbinden, en met geen twee rokken gekleed zijn.
\par 10 En Hij zeide tot hen: Zo waar gij in een huis zult ingaan, blijft daar, totdat gij van daar uitgaat.
\par 11 En zo wie u niet zullen ontvangen, noch u horen, vertrekkende van daar, schudt het stof af, dat onder aan uw voeten is, hun tot een getuigenis. Voorwaar zeg Ik u: Het zal Sodom en Gomorra verdragelijker zijn in den dag des oordeels dan dezelve stad.
\par 12 En uitgegaan zijnde, predikten zij, dat zij zich zouden bekeren.
\par 13 En zij wierpen vele duivelen uit, en zalfden vele kranken met olie, en maakten hen gezond.
\par 14 En de koning Herodes hoorde het (want Zijn Naam was openbaar geworden), en zeide: Johannes, die daar doopte, is van de doden opgewekt, en daarom werken die krachten in Hem.
\par 15 Anderen zeiden: Hij is Elias; en anderen zeiden: Hij is een profeet, of als een der profeten.
\par 16 Maar als het Herodes hoorde, zeide hij: Deze is Johannes, dien ik onthoofd heb; die is van de doden opgewekt.
\par 17 Want dezelve Herodes, enigen uitgezonden hebbende, had Johannes gevangen genomen, en hem in de gevangenis gebonden, uit oorzaak van Herodias, de huisvrouw van zijn broeder Filippus, omdat hij haar getrouwd had.
\par 18 Want Johannes zeide tot Herodes: Het is u niet geoorloofd de huisvrouw uws broeders te hebben.
\par 19 En Herodias legde op hem toe; en wilde hem doden, en konde niet;
\par 20 Want Herodes vreesde Johannes, wetende, dat hij een rechtvaardig en heilig man was, en hield hem in waarde; en als hij hem hoorde, deed hij vele dingen, en hoorde hem gaarne.
\par 21 En als er een welgelegen dag gekomen was, toen Herodes, op den dag zijner geboorte, een maaltijd aanrichtte, voor zijn groten, en de oversten over duizend, en de voornaamsten van Galilea;
\par 22 En als de dochter van dezelve Herodias inkwam, en danste, en Herodes en dengenen, die mede aanzaten, behaagde, zo zeide de koning tot het dochtertje: Eis van mij, wat gij ook wilt, en ik zal het u geven.
\par 23 En hij zwoer haar: Zo wat gij van mij zult eisen, zal ik u geven, ook tot de helft mijns koninkrijks!
\par 24 En zij, uitgegaan zijnde, zeide tot haar moeder: Wat zal ik eisen? En die zeide: Het hoofd van Johannes den Doper.
\par 25 En zij, terstond met haast ingaande tot den koning, heeft het geeist, zeggende: Ik wil, dat gij mij nu terstond, in een schotel, geeft het hoofd van Johannes den Doper.
\par 26 En de koning, zeer bedroefd geworden zijnde, nochtans om de eden, en degenen, die mede aanzaten, wilde hij haar hetzelve niet afslaan.
\par 27 En de koning zond terstond een scherprechter, en gebood zijn hoofd te brengen. Deze nu ging heen, en onthoofdde hem in de gevangenis;
\par 28 En bracht zijn hoofd in een schotel, en gaf hetzelve het dochtertje, en het dochtertje gaf hetzelve harer moeder.
\par 29 En als zijn discipelen dit hoorden, gingen zij en namen zijn dood lichaam weg, en legden dat in een graf.
\par 30 En de apostelen kwamen weder tot Jezus, en boodschapten Hem alles, beide wat zij gedaan hadden, en wat zij geleerd hadden.
\par 31 En Hij zeide tot hen: Komt gijlieden in een woeste plaats hier alleen, en rust een weinig; want er waren velen, die kwamen en die gingen, en zij hadden zelfs geen gelegen tijd om te eten.
\par 32 En zij vertrokken in een schip, naar een woeste plaats, alleen.
\par 33 En de scharen zagen hen heenvaren, en velen werden Hem kennende, en liepen gezamenlijk te voet van alle steden derwaarts, en kwamen hun voor, en gingen samen tot Hem.
\par 34 En Jezus, uitgaande, zag een grote schare, en werd innerlijk met ontferming bewogen over hen; want zij waren als schapen, die geen herder hebben; en Hij begon hun vele dingen te leren.
\par 35 En als het nu laat op den dag geworden was, kwamen Zijn discipelen tot Hem, en zeiden: Deze plaats is woest, en het is nu laat op den dag;
\par 36 Laat ze van U, opdat zij heengaan in de omliggende dorpen en vlekken, en broden voor zichzelven mogen kopen; want zij hebben niet, wat zij eten zullen.
\par 37 Maar Hij, antwoordende, zeide tot hen: Geeft gij hun te eten. En zij zeiden tot Hem: Zullen wij heengaan, en kopen voor tweehonderd penningen brood, en hun te eten geven?
\par 38 En Hij zeide tot hen: Hoeveel broden hebt gij? Gaat heen en beziet het. En toen zij het vernomen hadden, zeiden zij: Vijf, en twee vissen.
\par 39 En Hij gebood hun, dat zij hen allen zouden doen nederzitten bij waardschappen, op het groene gras.
\par 40 En zij zaten neder in gedeelten bij honderd te zamen, en bij vijftig te zamen.
\par 41 En als Hij de vijf broden en de twee vissen genomen had, zag Hij op naar den hemel, zegende en brak de broden, en gaf ze Zijn discipelen, opdat zij ze hun zouden voorleggen, en de twee vissen deelde Hij voor allen.
\par 42 En zij aten allen, en zijn verzadigd geworden.
\par 43 En zij namen op twaalf volle korven brokken, en van de vissen.
\par 44 En die daar de broden gegeten hadden, waren omtrent vijf duizend mannen.
\par 45 En terstond dwong Hij Zijn discipelen in het schip te gaan, en voor henen te varen aan de andere zijde tegen over Bethsaida, terwijl Hij de schare van Zich zou laten.
\par 46 En als Hij aan dezelve hun afscheid gegeven had, ging Hij op den berg om te bidden.
\par 47 En als het nu avond was geworden, zo was het schip in het midden van de zee, en Hij was alleen op het land.
\par 48 En Hij zag, dat zij zich zeer pijnigden, om het schip voort te krijgen; want de wind was hun tegen; en omtrent de vierde wake des nachts, kwam Hij tot hen, wandelende op de zee, en wilde hen voorbijgaan.
\par 49 En zij, ziende Hem wandelen op de zee, meenden, dat het een spooksel was, en schreeuwden zeer;
\par 50 Want zij zagen Hem allen, en werden ontroerd; en terstond sprak Hij met hen, en zeide tot hen: Zijt welgemoed, Ik ben het; vreest niet.
\par 51 En Hij klom tot hen in het schip, en de wind stilde; en zij ontzetten zich bovenmate zeer in zichzelven, en waren verwonderd.
\par 52 Want zij hadden niet gelet op het wonder der broden; want hun hart was verhard.
\par 53 En als zij overgevaren waren, kwamen zij in het land Gennesareth, en havenden aldaar.
\par 54 En als zij uit het schip gegaan waren, terstond werden zij Hem kennende.
\par 55 En het gehele omliggende land doorlopende, begonnen zij op beddekens degenen, die kwalijk gesteld waren, om te dragen, ter plaatse, waar zij hoorden, dat Hij was.
\par 56 En zo waar Hij kwam, in vlekken, of steden, of dorpen, daar leiden zij de kranken op de markten, en baden Hem, dat zij maar den zoom Zijns kleeds aanraken mochten; en zovelen, als er Hem aanraakten, werden gezond.

\chapter{7}

\par 1 En tot Hem vergaderden de Farizeen, en sommigen der Schriftgeleerden, die van Jeruzalem gekomen waren;
\par 2 En ziende, dat sommigen van Zijn discipelen met onreine, dat is, met ongewassen handen brood aten, berispten zij hen.
\par 3 Want de Farizeen en al de Joden eten niet, tenzij dat zij eerst de handen dikmaals wassen, houdende de inzettingen der ouden.
\par 4 En van de markt komende, eten zij niet, tenzij dat zij eerst gewassen zijn. En vele andere dingen zijn er, die zij aangenomen hebben te houden, als namelijk de wassingen der drinkbekers, en kannen, en koperen vaten, en bedden.
\par 5 Daarna vraagden Hem de Farizeen en de Schriftgeleerden: Waarom wandelen Uw discipelen niet naar de inzetting der ouden, maar eten het brood met ongewassen handen?
\par 6 Maar Hij antwoordde en zeide tot hen: Wel heeft Jesaja, van u, geveinsden, geprofeteerd, gelijk geschreven is: Dit volk eert Mij met de lippen, maar hun hart houdt zich verre van Mij.
\par 7 Doch tevergeefs eren zij Mij, lerende leringen, die geboden zijn der mensen;
\par 8 Want, nalatende het gebod Gods, houdt gij de inzettingen der mensen, als namelijk wassingen der kannen en drinkbekers; en andere dergelijke dingen doet gij vele.
\par 9 En Hij zeide tot hen: Gij doet zeker Gods gebod wel te niet, opdat gij uw inzettingen zoudt onderhouden.
\par 10 Want Mozes heeft gezegd: Eer uw vader en uw moeder; en: wie vader of moeder vloekt, die zal den dood sterven.
\par 11 Maar gijlieden zegt: Zo een mens tot vader of moeder zegt: Het is korban (dat is te zeggen, een gave), zo wat u van mij zou kunnen ten nutte komen, die voldoet.
\par 12 En gij laat hem niet meer toe, iets aan zijn vader of zijn moeder te doen;
\par 13 Makende alzo Gods woord krachteloos door uw inzetting, die gij ingezet hebt; en vele dergelijke dingen doet gij.
\par 14 En tot Zich de ganse schare geroepen hebbende, zeide Hij tot hen: Hoort Mij allen en verstaat.
\par 15 Er is niets van buiten den mens in hem ingaande, hetwelk hem kan ontreinigen; maar de dingen, die van hem uitgaan, die zijn het, welke den mens ontreinigen.
\par 16 Zo iemand oren heeft om te horen, die hore.
\par 17 En toen Hij van de schare in huis gekomen was, vraagden Hem Zijn discipelen van de gelijkenis.
\par 18 En Hij zeide tot hen: Zijt ook gij alzo onwetende? Verstaat gij niet, dat al wat van buiten in den mens ingaat, hem niet kan ontreinigen?
\par 19 Want het gaat niet in zijn hart, maar in den buik, en gaat in de heimelijkheid uit, reinigende al de spijzen.
\par 20 En Hij zeide: Hetgeen uitgaat uit den mens, dat ontreinigt den mens.
\par 21 Want van binnen uit het hart der mensen komen voort kwade gedachten, overspelen, hoererijen, doodslagen,
\par 22 Dieverijen, gierigheden, boosheden, bedrog, ontuchtigheid, een boos oog, lastering, hovaardij, onverstand.
\par 23 Al deze boze dingen komen voort van binnen, en ontreinigen den mens.
\par 24 En van daar opstaande, ging Hij weg naar de landpalen van Tyrus en Sidon; en in een huis gegaan zijnde, wilde Hij niet, dat het iemand wist, en Hij kon nochtans niet verborgen zijn.
\par 25 Want een vrouw, welker dochtertje een onreinen geest had, van Hem gehoord hebbende, kwam en viel neder aan Zijn voeten.
\par 26 Deze nu was een Griekse vrouw, van geboorte uit Syro-fenicie; en zij bad Hem, dat Hij den duivel uitwierp uit haar dochter.
\par 27 Maar Jezus zeide tot haar: Laat eerst de kinderen verzadigd worden; want het is niet betamelijk dat men het brood der kinderen neme, en den hondekens voor werpe.
\par 28 Maar zij antwoordde en zeide tot Hem: Ja, Heere, doch ook de hondekens eten onder de tafel van de kruimkens der kinderen.
\par 29 En Hij zeide tot haar: Om dezes woords wil ga heen, de duivel is uit uw dochter uitgevaren.
\par 30 En als zij in haar huis kwam, vond zij, dat de duivel uitgevaren was, en de dochter liggende op het bed.
\par 31 En Hij wederom weggegaan zijnde van de landpalen van Tyrus en Sidon, kwam aan de zee van Galilea, door het midden der landpalen van Dekapolis.
\par 32 En zij brachten tot Hem een dove, die zwaarlijk sprak, en baden Hem, dat Hij de hand op hem legde.
\par 33 En hem van de schare alleen genomen hebbende, stak Hij Zijn vingeren in zijn oren, en gespogen hebbende, raakte Hij zijn tong aan;
\par 34 En opwaarts ziende naar den hemel, zuchtte Hij, en zeide tot hem: Effatha! dat is: wordt geopend!
\par 35 En terstond werden zijn oren geopend, en de band zijner tong werd los, en hij sprak recht.
\par 36 En Hij gebood hunlieden, dat zij het niemand zeggen zouden; maar wat Hij hun ook gebood, zo verkondigden zij het des te meer.
\par 37 En zij ontzetten zich bovenmate zeer, zeggende: Hij heeft alles wel gedaan, en Hij maakt, dat de doven horen, en de stommen spreken.

\chapter{8}

\par 1 In dezelfde dagen, als er een geheel grote schare was, en zij niet hadden, wat zij eten zouden, riep Jezus Zijn discipelen tot Zich, en zeide tot hen:
\par 2 Ik word innerlijk met ontferming bewogen over de schare; want zij zijn nu drie dagen bij Mij gebleven, en hebben niet, wat zij eten zouden.
\par 3 En indien Ik hen nuchteren naar hun huis laat gaan, zo zullen zij op den weg bezwijken; want sommigen van hen komen van verre.
\par 4 En Zijn discipelen antwoordden Hem: Van waar zal iemand dezen met broden hier in de woestijn kunnen verzadigen?
\par 5 En Hij vraagde hun: Hoeveel broden hebt gij? En zij zeiden: Zeven.
\par 6 En Hij gebood de schare neder te zitten op de aarde, en Hij nam de zeven broden, en gedankt hebbende, brak Hij ze, en gaf ze Zijn discipelen, opdat zij ze zouden voorleggen; en zij leiden ze der schare voor.
\par 7 En zij hadden weinige visjes; en als Hij gezegend had, zeide Hij, dat zij ook die zouden voorleggen.
\par 8 En zij hebben gegeten, en zijn verzadigd geworden, en zij namen het overschot der brokken op, zeven manden.
\par 9 Die nu gegeten hadden, waren omtrent vier duizend; en Hij liet hen gaan.
\par 10 En terstond in het schip gegaan zijnde met Zijn discipelen, is Hij gekomen in de delen van Dalmanutha.
\par 11 En de Farizeen gingen uit, en begonnen met Hem te twisten, begerende van Hem een teken van den hemel, Hem verzoekende.
\par 12 En Hij, zwaarlijk zuchtende in Zijn geest, zeide: Wat begeert dit geslacht een teken? Voorwaar, Ik zeg u: Zo aan dit geslacht een teken gegeven zal worden!
\par 13 En Hij verliet hen, en wederom in het schip gegaan zijnde, voer Hij weg naar de andere zijde.
\par 14 En Zijn discipelen hadden vergeten brood mede te nemen, en hadden niet dan een brood met zich in het schip.
\par 15 En Hij gebood hun, zeggende: Ziet toe, wacht u van den zuurdesem der Farizeen, en van den zuurdesem van Herodes.
\par 16 En zij overleiden onder elkander, zeggende: Het is, omdat wij geen broden hebben.
\par 17 En Jezus, dat bekennende, zeide tot hen: Wat overlegt gij, dat gij geen broden hebt? Bemerkt gij nog niet, en verstaat gij niet, hebt gij nog uw verharde hart?
\par 18 Ogen hebbende, ziet gij niet? En oren hebbende, hoort gij niet?
\par 19 En gedenkt gij niet, toen Ik de vijf broden brak onder de vijf duizend mannen, hoeveel volle korven met brokken gij opnaamt? Zij zeggen Hem: Twaalf.
\par 20 En toen Ik de zeven brak onder de vier duizend mannen, hoeveel volle manden met brokken gij opnaamt? En zij zeiden: Zeven.
\par 21 En Hij zeide tot hen: Hoe verstaat gij niet?
\par 22 En Hij kwam te Bethsaida; en zij brachten tot Hem een blinde, en baden Hem, dat Hij hem aanraakte.
\par 23 En de hand des blinden genomen hebbende, leidde Hij hem uit buiten het vlek, en spoog in zijn ogen, en leide de handen op hem, en vraagde hem, of hij iets zag.
\par 24 En hij, opziende, zeide: Ik zie de mensen, want ik zie hen, als bomen, wandelen.
\par 25 Daarna leide Hij de handen wederom op zijn ogen, en deed hem opzien. En hij werd hersteld, en zag hen allen ver en klaar.
\par 26 En Hij zond hem naar zijn huis, zeggende: Ga niet in het vlek, en zeg het niemand in het vlek.
\par 27 En Jezus ging uit en Zijn discipelen naar de vlekken van Cesarea Filippi. En op den weg vraagde Hij Zijn discipelen, zeggende tot hen: Wie zeggen de mensen, dat Ik ben?
\par 28 En zij antwoordden: Johannes de Doper; en anderen: Elias; en anderen: Een van de profeten.
\par 29 En Hij zeide tot hen: Maar gijlieden, wie zegt gij dat Ik ben? En Petrus, antwoordende, zeide tot Hem: Gij zijt de Christus.
\par 30 En Hij gebood hun scherpelijk, dat zij het niemand zouden zeggen van Hem.
\par 31 En Hij begon hun te leren, dat de Zoon des mensen veel moest lijden, en verworpen worden van de ouderlingen, en overpriesters, en Schriftgeleerden, en gedood worden, en na drie dagen wederom opstaan.
\par 32 En dit woord sprak Hij vrij uit; en Petrus, Hem tot zich genomen hebbende, begon Hem te bestraffen;
\par 33 Maar Hij, Zich omkerende, en Zijn discipelen aanziende, bestrafte Petrus, zeggende: Ga heen, achter Mij, satanas, want gij verzint niet de dingen, die Gods zijn, maar die der mensen zijn.
\par 34 En tot Zich geroepen hebbende de schare met Zijn discipelen, zeide Hij tot hen: Zo wie achter Mij wil komen, die verloochene zichzelven, en neme zijn kruis op, en volge Mij.
\par 35 Want zo wie zijn leven zal willen behouden, die zal hetzelve verliezen; maar zo wie zijn leven zal verliezen, om Mijnentwil, en om des Evangelies wil, die zal hetzelve behouden.
\par 36 Want wat zou het den mens baten zo hij de gehele wereld won, en zijner ziele schade leed?
\par 37 Of wat zal een mens geven, tot lossing van zijn ziel?
\par 38 Want zo wie zich Mijns en Mijner woorden zal geschaamd hebben, in dit overspelig en zondig geslacht, diens zal Zich de Zoon des mensen ook schamen, wanneer Hij zal komen in de heerlijkheid Zijns Vaders, met de heilige engelen.

\chapter{9}

\par 1 En Hij zeide tot hen: Voorwaar, Ik zeg u, dat er sommigen zijn van degenen, die hier staan, die den dood niet zullen smaken, totdat zij zullen hebben gezien, dat het Koninkrijk Gods met kracht gekomen is.
\par 2 En na zes dagen nam Jezus met Zich Petrus, en Jakobus, en Johannes, en bracht hen op een hogen berg bezijden alleen; en Hij werd voor hen van gedaante veranderd.
\par 3 En Zijn klederen werden blinkende, zeer wit als sneeuw, hoedanige geen voller op aarde zo wit maken kan.
\par 4 En van hen werd gezien Elias met Mozes, en zij spraken met Jezus.
\par 5 En Petrus, antwoordende, zeide tot Jezus: Rabbi, het is goed, dat wij hier zijn, en laat ons drie tabernakelen maken, voor U een, en voor Mozes een, en voor Elias een.
\par 6 Want hij wist niet, wat hij zeide; want zij waren zeer bevreesd.
\par 7 En er kwam een wolk, die hen overschaduwde, en een stem kwam uit de wolk, zeggende: Deze is Mijn geliefde Zoon, hoort Hem!
\par 8 En haastelijk rondom ziende, zagen zij niemand meer, dan Jezus alleen bij zich.
\par 9 En als zij van den berg afkwamen, gebood Hij hun, dat zij niemand verhalen zouden, hetgeen zij gezien hadden, dan wanneer de Zoon des mensen uit de doden zou opgestaan zijn.
\par 10 En zij behielden dit woord bij zichzelven, vragende onder elkander, wat het was, uit de doden opstaan.
\par 11 En zij vraagden Hem, zeggende: Waarom zeggen de Schriftgeleerden, dat Elias eerst komen moet?
\par 12 En Hij, antwoordende, zeide tot hen: Elias zal wel eerst komen, en alles weder oprichten; en het zal geschieden, gelijk geschreven is van den Zoon des mensen, dat Hij veel lijden zal en veracht worden.
\par 13 Maar Ik zeg u, dat ook Elias gekomen is, en zij hebben hem gedaan al wat zij gewild hebben, gelijk van hem geschreven is.
\par 14 En als Hij bij de discipelen gekomen was, zag Hij een grote schare rondom hen, en enige Schriftgeleerden met hen twistende.
\par 15 En terstond de gehele schare Hem ziende, werd verbaasd, en toelopende groetten zij Hem.
\par 16 En Hij vraagde den Schriftgeleerden: Wat twist gij met dezen?
\par 17 En een uit de schare, antwoordende, zeide: Meester, ik heb mijn zoon tot U gebracht, die een stommen geest heeft.
\par 18 En waar hij hem ook aangrijpt, zo scheurt hij hem, en schuimt, en knerst met zijn tanden, en verdort; en ik heb Uw discipelen gezegd dat zij hem zouden uitwerpen, en zij hebben niet gekund.
\par 19 En Hij antwoordden hem, en zeide: O ongelovig geslacht, hoe lang zal Ik nog bij ulieden zijn, hoe lang zal Ik u nog verdragen? Brengt hem tot Mij.
\par 20 En zij brachten denzelven tot Hem; en als hij Hem zag, scheurde hem terstond de geest; en hij vallende op de aarde, wentelde zich al schuimende.
\par 21 En Hij vraagde zijn vader: Hoe langen tijd is het, dat hem dit overkomen is? En hij zeide: Van zijn kindsheid af.
\par 22 En menigmaal heeft hij hem ook in het vuur en in het water geworpen, om hem te verderven; maar zo Gij iets kunt, wees met innerlijke ontferming over ons bewogen, en help ons.
\par 23 En Jezus zeide tot hem: Zo gij kunt geloven, alle dingen zijn mogelijk dengene, die gelooft.
\par 24 En terstond de vader des kinds, roepende met tranen, zeide: Ik geloof, Heere! kom mijn ongelovigheid te hulp.
\par 25 En Jezus ziende, dat de schare gezamenlijk toeliep, bestrafte den onreinen geest, zeggende tot hem: Gij stomme en dove geest! Ik beveel u, ga uit van hem, en kom niet meer in hem.
\par 26 En hij, roepende en hem zeer scheurende, ging uit; en het kind werd als dood, alzo dat velen zeiden, dat het gestorven was.
\par 27 En Jezus, hem bij de hand grijpende, richtte hem op; en hij stond op.
\par 28 En als Hij in huis gegaan was, vraagden Hem Zijn discipelen alleen: Waarom hebben wij hem niet kunnen uitwerpen?
\par 29 En Hij zeide tot hen: Dit geslacht kan nergens door uitgaan, dan door bidden en vasten.
\par 30 En van daar weggaande, reisden zij door Galilea; en Hij wilde niet, dat het iemand wist.
\par 31 Want Hij leerde Zijn discipelen, en zeide tot hen: De Zoon des mensen zal overgeleverd worden in de handen der mensen, en zij zullen Hem doden, en gedood zijnde, zal Hij ten derden dage wederopstaan.
\par 32 Maar zij verstonden dat woord niet, en zij vreesden Hem te vragen.
\par 33 En Hij kwam te Kapernaum, en in het huis gekomen zijnde, vraagde Hij hun: Waarvan hadt gij woorden onder elkander op den weg?
\par 34 Doch zij zwegen; want zij waren onder elkander in woorden geweest op den weg, wie de meeste zou zijn.
\par 35 En nedergezeten zijnde, riep Hij de twaalven, en zeide tot hen: Indien iemand wil de eerste zijn, die zal de laatste van allen zijn, en aller dienaar.
\par 36 En nemende een kindeken, stelde Hij dat midden onder hen, en omving het met Zijn armen, en zeide tot hen:
\par 37 Zo wie een van zodanige kinderkens zal ontvangen in Mijn Naam, die ontvangt Mij; en zo wie Mij zal ontvangen, die ontvangt Mij niet, maar Dien, Die Mij gezonden heeft.
\par 38 En Johannes antwoordde Hem, zeggende: Meester! wij hebben een gezien, die de duivelen uitwierp in Uw Naam, welke ons niet volgt; en wij hebben het hem verboden, omdat hij ons niet volgt.
\par 39 Doch Jezus zeide: Verbiedt hem niet; want er is niemand, die een kracht doen zal in Mijn Naam, en haastelijk van Mij zal kunnen kwalijk spreken.
\par 40 Want wie tegen ons niet is, die is voor ons.
\par 41 Want zo wie ulieden een beker water zal te drinken geven in Mijn Naam, omdat gij discipelen van Christus zijt, voorwaar zeg Ik u, hij zal zijn loon geenszins verliezen.
\par 42 En zo wie een van deze kleinen, die in Mij geloven, ergert, het ware hem beter, dat een molensteen om zijn hals gedaan ware, en dat hij in de zee geworpen ware.
\par 43 En indien uw hand u ergert, houwt ze af; het is u beter verminkt tot het leven in te gaan, dan de twee handen hebbende, heen te gaan in de hel, in het onuitblusselijk vuur;
\par 44 Waar hun worm niet sterft, en het vuur niet uitgeblust wordt.
\par 45 En indien uw voet u ergert, houwt hem af; het is u beter kreupel tot het leven in te gaan, dan de twee voeten hebbende, geworpen te worden in de hel, in het onuitblusselijk vuur;
\par 46 Waar hun worm niet sterft, en het vuur niet uitgeblust wordt.
\par 47 En indien uw oog u ergert, werpt het uit; het is u beter maar een oog hebbende in het Koninkrijk Gods in te gaan, dan twee ogen hebbende, in het helse vuur geworpen te worden;
\par 48 Waar hun worm niet sterft, en het vuur niet uitgeblust wordt.
\par 49 Want een ieder zal met vuur gezouten worden, en iedere offerande zal met zout gezouten worden.
\par 50 Het zout is goed; maar indien het zout onzout wordt, waarmede zult gij dat smakelijk maken? Hebt zout in uzelven, en houdt vrede onder elkander.

\chapter{10}

\par 1 En van daar opgestaan zijnde, ging Hij naar de landpalen van Judea, door de overzijde van de Jordaan; en de scharen kwamen wederom samen bij Hem, en gelijk Hij gewoon was, leerde Hij hen wederom.
\par 2 En de Farizeen, tot Hem komende, vraagden Hem, of het een man geoorloofd is, zijn vrouw te verlaten, Hem verzoekende.
\par 3 Maar Hij antwoordende, zeide tot hen: Wat heeft u Mozes geboden?
\par 4 En zij zeiden: Mozes heeft toegelaten een scheidbrief te schrijven, en haar te verlaten.
\par 5 En Jezus, antwoordende, zeide tot hen: Vanwege de hardigheid uwer harten heeft hij ulieden dat gebod geschreven.
\par 6 Maar van het begin der schepping heeft ze God man en vrouw gemaakt.
\par 7 Daarom zal een mens zijn vader en zijn moeder verlaten, en zal zijn vrouw aanhangen;
\par 8 En die twee zullen tot een vlees zijn, alzo dat zij niet meer twee zijn, maar een vlees.
\par 9 Hetgeen dan God samengevoegd heeft, scheide de mens niet.
\par 10 En in het huis vraagden Hem Zijn discipelen wederom van hetzelve.
\par 11 En Hij zeide tot hen: Zo wie zijn vrouw verlaat, en een andere trouwt, die doet overspel tegen haar.
\par 12 En indien een vrouw haar man zal verlaten, en met een anderen trouwen, die doet overspel.
\par 13 En zij brachten kinderkens tot Hem, opdat Hij ze aanraken zou; en de discipelen bestraften degenen, die ze tot Hem brachten.
\par 14 Maar Jezus, dat ziende, nam het zeer kwalijk, en zeide tot hen: Laat de kinderkens tot Mij komen, en verhindert ze niet; want derzulken is het Koninkrijk Gods.
\par 15 Voorwaar zeg Ik u: Zo wie het Koninkrijk Gods niet ontvangt, gelijk een kindeken, die zal in hetzelve geenszins ingaan.
\par 16 En Hij omving ze met Zijn armen, en de handen op hen gelegd hebbende, zegende Hij dezelve.
\par 17 En als Hij uitging op den weg, liep een tot Hem, en voor Hem op de knieen vallende, vraagde Hem: Goede Meester! wat zal ik doen, opdat ik het eeuwige leven beerve?
\par 18 En Jezus zeide tot hem: Wat noemt gij Mij goed? Niemand is goed, dan Een, namelijk God.
\par 19 Gij weet de geboden: Gij zult geen overspel doen; gij zult niet doden; gij zult niet stelen; gij zult geen valse getuigenis geven; gij zult niemand te kort doen; eer uw vader en uw moeder.
\par 20 Doch hij, antwoordende, zeide tot Hem: Meester! al deze dingen heb ik onderhouden van mijn jonkheid af.
\par 21 En Jezus, hem aanziende, beminde hem, en zeide tot hem: Een ding ontbreekt u; ga heen, verkoop alles, wat gij hebt, en geef het den armen, en gij zult een schat hebben in den hemel; en kom herwaarts, neem het kruis op, en volg Mij.
\par 22 Maar hij, treurig geworden zijnde over dat woord, ging bedroefd weg; want hij had vele goederen.
\par 23 En Jezus rondom ziende, zeide tot Zijn discipelen: Hoe bezwaarlijk zullen degenen, die goed hebben, in het Koninkrijk Gods inkomen!
\par 24 En de discipelen werden verbaasd over deze Zijn woorden. Maar Jezus wederom antwoordende, zeide tot hen: Kinderen! Hoe zwaar is het, dat degenen, die op het goed hun betrouwen zetten, in het Koninkrijk Gods ingaan!
\par 25 Het is lichter, dat een kemel ga door het oog van een naald, dan dat een rijke in het Koninkrijk Gods inga.
\par 26 En zij werden nog meer verslagen, zeggende tot elkander: Wie kan dan zalig worden?
\par 27 Doch Jezus, hen aanziende, zeide: Bij de mensen is het onmogelijk, maar niet bij God; want alle dingen zijn mogelijk bij God.
\par 28 En Petrus begon tot Hem te zeggen: Zie, wij hebben alles verlaten, en zijn U gevolgd.
\par 29 En Jezus, antwoordende, zeide: Voorwaar zeg Ik ulieden: Er is niemand, die verlaten heeft huis, of broeders, of zusters, of vader, of moeder, of vrouw, of kinderen, of akkers, om Mijnentwil en des Evangelies wil,
\par 30 Of hij ontvangt honderdvoud, nu in dezen tijd, huizen, en broeders, en zusters, en moeders, en kinderen, en akkers, met de vervolgingen, en in de toekomende eeuw het eeuwige leven.
\par 31 Maar vele eersten zullen de laatsten zijn, en velen, die de laatsten zijn, de eersten.
\par 32 En zij waren op den weg, gaande op naar Jeruzalem; en Jezus ging voor hen; en zij waren verbaasd, en Hem volgende, waren zij bevreesd. En de twaalven wederom tot Zich nemende, begon Hij hun te zeggen de dingen, die Hem overkomen zouden;
\par 33 Zeggende: Ziet, wij gaan op naar Jeruzalem, en de Zoon des mensen zal den overpriesteren, en den Schriftgeleerden overgeleverd worden, en zij zullen Hem ter dood veroordelen, en Hem den heidenen overleveren;
\par 34 En zij zullen Hem bespotten, en Hem geselen, en Hem bespuwen, en Hem doden; en ten derden dage zal Hij weder opstaan.
\par 35 En tot Hem kwamen Jakobus en Johannes, de zonen van Zebedeus, zeggende: Meester! wij wilden wel, dat Gij ons deedt, zo wat wij begeren zullen.
\par 36 En Hij zeide tot hen: Wat wilt gij, dat Ik u doe?
\par 37 En zij zeiden tot Hem: Geef ons, dat wij mogen zitten, de een aan Uw rechter-,en de ander aan Uw linker hand in Uw heerlijkheid.
\par 38 Maar Jezus zeide tot hen: Gij weet niet, wat gij begeert. Kunt gij den drinkbeker drinken, dien Ik drink, en met den doop gedoopt worden, daar Ik mede gedoopt word?
\par 39 En zij zeiden tot Hem: Wij kunnen. Doch Jezus zeide tot hen: Den drinkbeker, dien Ik drink, zult gij wel drinken, en met den doop gedoopt worden, daar Ik mede gedoopt word;
\par 40 Maar het zitten tot Mijn rechter-en tot Mijn linker hand staat bij Mij niet te geven; maar het zal gegeven worden dien het bereid is.
\par 41 En als de andere tien dit hoorden, begonnen zij het van Jakobus en Johannes zeer kwalijk te nemen.
\par 42 Maar Jezus, het tot Zich geroepen hebbende, zeide tot hen: Gij weet, dat degenen, die geacht worden oversten te zijn der volken, heerschappij voeren over hen, en hun groten gebruiken macht over hen.
\par 43 Doch alzo zal het onder u niet zijn; maar zo wie onder u groot zal willen worden, die zal uw dienaar zijn.
\par 44 En zo wie van u de eerste zal willen worden, die zal aller dienstknecht zijn.
\par 45 Want ook de Zoon des mensen is niet gekomen om gediend te worden, maar om te dienen, en Zijn ziel te geven tot een rantsoen voor velen.
\par 46 En zij kwamen te Jericho. En als Hij en Zijn discipelen, en een grote schare van Jericho uitging, zat de zoon van Timeus, Bar-timeus, de blinde, aan den weg, bedelende.
\par 47 En horende, dat het Jezus de Nazarener was, begon hij te roepen en te zeggen: Jezus, Gij Zone Davids! ontferm U mijner.
\par 48 En velen bestraften hem, opdat hij zwijgen zou; maar hij riep zoveel temeer: Gij Zone Davids! ontferm U mijner.
\par 49 En Jezus, stil staande, zeide, dat men hem roepen zou; en zij riepen den blinde, zeggende tot hem: Heb goeden moed; sta op; Hij roept u.
\par 50 En hij, zijn mantel afgeworpen hebbende, stond op, en kwam tot Jezus.
\par 51 En Jezus, antwoordende, zeide tot hem: Wat wilt gij, dat Ik u doen zal? En de blinde zeide tot Hem: Rabboni! dat ik ziende mag worden.
\par 52 En Jezus zeide tot hem: Ga heen, uw geloof heeft u behouden. En terstond werd hij ziende, en volgde Jezus op den weg.

\chapter{11}

\par 1 En toen zij Jeruzalem genaakten, te Beth-fage en Bethanie, aan den Olijfberg, zond Hij twee van Zijn discipelen uit,
\par 2 En zeide tot hen: Gaat heen in het vlek, dat tegen u over is; en terstond als gij in hetzelve komt, zult gij vinden een veulen gebonden, op hetwelk geen mens gezeten heeft, ontbindt het, en brengt het.
\par 3 En indien iemand tot u zegt: Waarom doet gij dat? Zo zegt, dat de Heere hetzelve van node heeft; en hij zal het terstond herwaarts zenden.
\par 4 En zij gingen heen, en vonden het veulen gebonden bij de deur, buiten aan de wegscheiding, en zij ontbonden hetzelve.
\par 5 En sommigen van degenen, die aldaar stonden, zeiden tot hen: Wat doet gij, dat gij het veulen ontbindt?
\par 6 Doch zij zeiden tot hen, gelijk Jezus bevolen had; en zij lieten ze gaan.
\par 7 En zij brachten het veulen tot Jezus, en wierpen hun klederen daarop; en Hij zat op hetzelve.
\par 8 En velen spreidden hun klederen op den weg, en anderen hieuwen meien van de bomen, en spreidden ze op den weg.
\par 9 En die voorgingen en die volgden riepen, zeggende: Hosanna! gezegend is Hij, Die komt in den Naam des Heeren!
\par 10 Gezegend zij het Koninkrijk van onzen vader David, hetwelk komt in den Naam des Heeren! Hosanna in de hoogste hemelen!
\par 11 En Jezus kwam binnen Jeruzalem, en in den tempel; en als Hij alles rondom bezien had, en het nu avondstond was, ging Hij uit naar Bethanie met de twaalven.
\par 12 En des anderen daags, als zij uit Bethanie gingen, hongerde Hem.
\par 13 En ziende van verre een vijgeboom, die bladeren had, ging Hij om te zien, of Hij ook iets op denzelven zou vinden; en daarbij gekomen zijnde, vond Hij niet dan bladeren; want het was de tijd der vijgen niet.
\par 14 En Jezus, antwoordende, zeide tot denzelven: Niemand ete enige vrucht meer van u in der eeuwigheid! En Zijn discipelen hoorden het.
\par 15 En zij kwamen te Jeruzalem; en Jezus, in den tempel gegaan zijnde, begon degenen, die in den tempel verkochten en kochten, uit te drijven; en de tafelen der wisselaars, en de zitstoelen dergenen, die de duiven verkochten, keerde Hij om;
\par 16 En liet niet toe, dat iemand enig vat door den tempel droeg.
\par 17 En Hij leerde, zeggende tot hen: Is er niet geschreven: Mijn huis zal een huis des gebeds genaamd worden allen volken? Maar gij hebt dat tot een kuil der moordenaren gemaakt.
\par 18 En de Schriftgeleerden en de overpriesters hoorden dat, en zochten, hoe zij Hem doden zouden; want zij vreesden Hem, omdat de ganse schare ontzet was over Zijn leer.
\par 19 En als het nu laat geworden was, ging Hij uit buiten de stad.
\par 20 En des morgens vroeg voorbijgaande, zagen zij, dat de vijgeboom verdord was, van de wortelen af.
\par 21 En Petrus, zulks indachtig geworden zijnde, zeide tot Hem: Rabbi! zie, de vijgeboom, dien Gij vervloekt hebt, is verdord.
\par 22 En Jezus, antwoordende, zeide tot hen: Hebt geloof op God.
\par 23 Want voorwaar zeg Ik u, dat, zo wie tot dezen berg zal zeggen: Word opgeheven en in de zee geworpen; en niet zal twijfelen in zijn hart, maar zal geloven, dat hetgeen hij zegt, geschieden zal, het zal hem geworden, zo wat hij zegt.
\par 24 Daarom zeg Ik u: Alle dingen, die gij biddende begeert, gelooft, dat gij ze ontvangen zult, en zij zullen u geworden.
\par 25 En wanneer gij staat om te bidden, vergeeft, indien gij iets hebt tegen iemand; opdat ook uw Vader, Die in de hemelen is, ulieden uw misdaden vergeve.
\par 26 Maar indien gij niet vergeeft, zo zal uw Vader, Die in de hemelen is, ook uw misdaden niet vergeven.
\par 27 En zij kwamen wederom te Jeruzalem. En als Hij in den tempel wandelde, kwamen tot Hem de overpriesters, en de Schriftgeleerden, en de ouderlingen.
\par 28 En zeiden tot Hem: Door wat macht doet Gij deze dingen? En wie heeft U deze macht gegeven, dat Gij deze dingen doen zoudt?
\par 29 Maar Jezus, antwoordende, zeide tot hen: Ik zal u ook een woord vragen; antwoordt Mij ook, en zo zal Ik u zeggen, door wat macht Ik deze dingen doe:
\par 30 De doop van Johannes, was die uit den hemel, of uit de mensen? Antwoordt Mij.
\par 31 En zij overlegden onder zich, zeggende: Indien wij zeggen: Uit den hemel, zo zal Hij zeggen: Waarom hebt gij hem dan niet geloofd?
\par 32 Maar indien wij zeggen: Uit de mensen; zo vrezen wij het volk; want zij hielden allen van Johannes, dat hij waarlijk een profeet was.
\par 33 En, antwoordende, zeiden zij tot Jezus: Wij weten het niet. En Jezus, antwoordende, zeide tot hen: Zo zeg Ik u ook niet, door wat macht Ik deze dingen doe.

\chapter{12}

\par 1 En Hij begon door gelijkenissen tot hen te zeggen: Een mens plantte een wijngaard, en zette een tuin daarom, en groef een wijnpersbak, en bouwde een toren, en verhuurde dien aan de landlieden, en reisde buiten 's lands.
\par 2 En als het de tijd was, zond hij een dienstknecht tot de landlieden, opdat hij van de landlieden ontving van de vrucht des wijngaards.
\par 3 Maar zij namen en sloegen hem, en zonden hem ledig heen.
\par 4 En hij zond wederom een anderen dienstknecht tot hen, en dien stenigden zij, en wondden hem het hoofd, en zonden hem henen, schandelijk behandeld zijnde.
\par 5 En wederom zond hij een anderen, en dien doodden zij; en vele anderen, waarvan zij sommigen sloegen, en sommigen doodden.
\par 6 Als hij dan nog een zoon had, die hem lief was, zo heeft hij ook dien ten laatste tot hen gezonden, zeggende: Zij zullen immers mijn zoon ontzien.
\par 7 Maar die landlieden zeiden onder elkander: Deze is de erfgenaam; komt, laat ons hem doden, en de erfenis zal onze zijn.
\par 8 En zij namen en doodden hem, en wierpen hem uit, buiten den wijngaard.
\par 9 Wat zal dan de heer des wijngaards doen? Hij zal komen, en de landlieden verderven, en den wijngaard aan anderen geven.
\par 10 Hebt gij ook deze Schrift niet gelezen: De steen, dien de bouwlieden verworpen hebben, deze is geworden tot een hoofd des hoeks;
\par 11 Van den Heere is dit geschied, en het is wonderlijk in onze ogen?
\par 12 En zij zochten Hem te vangen, maar zij vreesden de schare; want zij verstonden, dat Hij die gelijkenis op hen sprak; en zij verlieten Hem en gingen weg.
\par 13 En zij zonden tot Hem enigen der Farizeen en der Herodianen, opdat zij Hem in Zijn rede vangen zouden.
\par 14 Dezen nu kwamen en zeiden tot Hem: Meester! wij weten, dat Gij waarachtig zijt, en naar niemand vraagt; want Gij ziet den persoon der mensen niet aan, maar Gij leert den weg Gods in der waarheid; is het geoorloofd, den keizer schatting te geven, of niet? Zullen wij geven, of niet geven?
\par 15 En Hij, wetende hun geveinsdheid, zeide tot hen: Wat verzoekt gij Mij? Brengt Mij een penning, dat Ik hem zie.
\par 16 En zij brachten een. En Hij zeide tot hen: Wiens is dit beeld, en het opschrift? en zij zeiden tot Hem: Des keizers.
\par 17 En Jezus, antwoordende, zeide tot hen: Geeft dan den keizer, dat des keizers is, en Gode, dat Gods is. En zij verwonderden zich over Hem.
\par 18 En de Sadduceen kwamen tot Hem, welke zeggen, dat er geen opstanding is, en vraagden Hem, zeggende:
\par 19 Meester! Mozes heeft ons geschreven: Indien iemands broeder sterft, en een vrouw achterlaat, en geen kinderen nalaat, dat zijn broeder deszelfs vrouw nemen zal en zijn broeder zaad verwekken.
\par 20 Er waren nu zeven broeders, en de eerste nam een vrouw, en stervende liet geen zaad na.
\par 21 De tweede nam haar ook, en is gestorven, en ook deze liet geen zaad na; en de derde desgelijks.
\par 22 En al de zeven namen dezelve, en lieten geen zaad na; de laatste van allen is ook de vrouw gestorven.
\par 23 In de opstanding dan, wanneer zij zullen opgestaan zijn, wiens vrouw zal zij van dezen zijn? Want die zeven hebben haar tot een vrouw gehad.
\par 24 En Jezus, antwoordende, zeide tot hen: Dwaalt gij niet, daarom, dat gij de Schriften niet weet, noch de kracht Gods?
\par 25 Want als zij uit de doden zullen opgestaan zijn, zo trouwen zij niet, noch worden ten huwelijk gegeven; maar zij zijn gelijk engelen, die in de hemelen zijn.
\par 26 Doch aangaande de doden, dat zij opgewekt zullen worden, hebt gij niet gelezen in het boek van Mozes, hoe God in het doornenbos tot hem gesproken heeft, zeggende: Ik ben de God Abrahams, en de God Izaks, en de God Jakobs?
\par 27 God is niet een God der doden, maar een God der levenden. Gij dwaalt dan zeer.
\par 28 En een der Schriftgeleerden horende, dat zij te zamen in woorden waren, en wetende, dat Hij hun wel geantwoord had, kwam tot Hem, en vraagde Hem: Welk is het eerste gebod van alle?
\par 29 En Jezus antwoordde hem: Het eerste van al de geboden is: Hoor, Israel! de Heere, onze God, is een enig Heere.
\par 30 En gij zult den Heere, uw God, liefhebben uit geheel uw hart, en uit geheel uw ziel, en uit geheel uw verstand, en uit geheel uw kracht. Dit is het eerste gebod.
\par 31 En het tweede aan dit gelijk, is dit: Gij zult uw naaste liefhebben als uzelven. Er is geen ander gebod, groter dan deze.
\par 32 En de schriftgeleerde zeide tot Hem: Meester, Gij hebt wel in der waarheid gezegd, dat er een enig God is, en er is geen ander dan Hij;
\par 33 En Hem lief te hebben uit geheel het hart, en uit geheel het verstand, en uit geheel de ziel, en uit geheel de kracht; en den naaste lief te hebben als zichzelven, is meer dan al de brandofferen en de slachtofferen.
\par 34 En Jezus ziende, dat hij verstandelijk geantwoord had, zeide tot hem: Gij zijt niet verre van het Koninkrijk Gods. En niemand durfde Hem meer vragen.
\par 35 En Jezus antwoordde en zeide, lerende in den tempel: Hoe zeggen de Schriftgeleerden, dat de Christus een Zoon van David is?
\par 36 Want David zelf heeft door den Heiligen Geest gezegd: De Heere heeft gezegd tot mijn Heere: Zit aan Mijn rechter hand, totdat Ik Uw vijanden zal gezet hebben tot een voetbank Uwer voeten.
\par 37 David dan zelf noemt Hem zijn Heere, en hoe is Hij zijn Zoon? En de menigte der schare hoorde Hem gaarne.
\par 38 En Hij zeide tot hen in Zijn leer: Wacht u voor de Schriftgeleerden, die daar gaarne willen wandelen in lange klederen, en gegroet zijn op de markten;
\par 39 En de voorgestoelten hebben in de synagogen, en de vooraanzittingen in de maaltijden;
\par 40 Welke de huizen der weduwen opeten, en dat onder den schijn van lang te bidden. Dezen zullen zwaarder oordeel ontvangen.
\par 41 En Jezus, gezeten zijnde tegenover de schatkist, zag, hoe de schare geld wierp in de schatkist; en vele rijken wierpen veel daarin.
\par 42 En er kwam een arme weduwe, die twee kleine penningen daarin wierp, hetwelk is een oort.
\par 43 En Jezus, Zijn discipelen tot Zich geroepen hebbende, zeide tot hen: Voorwaar, Ik zeg u, dat deze arme weduwe meer ingeworpen heeft, dan allen, die in de schatkist geworpen hebben.
\par 44 Want zij allen hebben van hun overvloed daarin geworpen; maar deze heeft van haar gebrek, al wat zij had, daarin geworpen, haar gansen leeftocht.

\chapter{13}

\par 1 En als Hij uit den tempel ging, zeide een van Zijn discipelen tot Hem: Meester, zie, hoedanige stenen, en hoedanige gebouwen!
\par 2 En Jezus, antwoordende, zeide tot hem: Ziet gij deze grote gebouwen? Er zal niet een steen op den anderen steen gelaten worden, die niet afgebroken zal worden.
\par 3 En als Hij gezeten was op den Olijfberg, tegen den tempel over, vraagden Hem Petrus, en Jakobus, en Johannes, en Andreas, alleen:
\par 4 Zeg ons, wanneer zullen deze dingen zijn? En welk is het teken, wanneer deze dingen allen voleindigd zullen worden?
\par 5 En Jezus, hun antwoordende, begon te zeggen: Ziet toe, dat u niemand verleide.
\par 6 Want velen zullen komen onder Mijn Naam, zeggende: Ik ben de Christus; en zullen velen verleiden.
\par 7 En wanneer gij zult horen van oorlogen, en geruchten van oorlogen, zo wordt niet verschrikt; want dit moet geschieden; maar nog is het einde niet.
\par 8 Want het ene volk zal tegen het andere volk opstaan, en het ene koninkrijk tegen het andere koninkrijk; en er zullen aardbevingen zijn in verscheidene plaatsen, en er zullen hongersnoden wezen, en beroerten. Deze dingen zijn maar beginselen der smarten.
\par 9 Maar ziet gij voor uzelven toe; want zij zullen u overleveren in de raadsvergaderingen, en in de synagogen; gij zult geslagen worden, en voor stadhouders en koningen zult gij gesteld worden, om Mijnentwil, hun tot een getuigenis.
\par 10 En het Evangelie moet eerst gepredikt worden onder al de volken.
\par 11 Doch wanneer zij u leiden zullen, om u over te leveren, zo zijt te voren niet bezorgd, wat gij spreken zult, en bedenkt het niet; maar zo wat u in die ure gegeven zal worden, spreekt dat; want gij zijt het niet, die spreekt, maar de Heilige Geest.
\par 12 En de ene broeder zal den anderen overleveren tot den dood, en de vader het kind; en de kinderen zullen opstaan tegen de ouders, en zullen hen doden.
\par 13 En gij zult gehaat worden van allen, om Mijns Naams wil; maar wie volharden zal tot het einde, die zal zalig worden.
\par 14 Wanneer gij dan zult zien den gruwel der verwoesting, waarvan door den profeet Daniel gesproken is, staande waar het niet behoort, (die het leest, die merke daarop!) alsdan, die in Judea zijn, dat zij vlieden op de bergen.
\par 15 En die op het dak is, kome niet af in het huis, en ga niet in, om iets uit zijn huis weg te nemen.
\par 16 En die op den akker is, kere niet weder terug, om zijn kleed te nemen.
\par 17 Maar wee den bevruchten en den zogenden vrouwen in die dagen!
\par 18 Doch bidt, dat uw vlucht niet geschiede des winters.
\par 19 Want die dagen zullen zulke verdrukking zijn, welker gelijke niet geweest is van het begin der schepselen, die God geschapen heeft, tot nu toe, en ook niet zijn zal.
\par 20 En indien de Heere de dagen niet verkort had, geen vlees zou behouden worden; maar om der uitverkorenen wil, die Hij heeft uitverkoren, heeft Hij de dagen verkort.
\par 21 En alsdan, zo iemand tot ulieden zal zeggen: Ziet, hier is de Christus; of ziet, Hij is daar; gelooft het niet.
\par 22 Want er zullen valse christussen, en valse profeten opstaan, en zullen tekenen en wonderen doen, om te verleiden, indien het mogelijk ware, ook de uitverkorenen.
\par 23 Maar gijlieden ziet toe; ziet, Ik heb u alles voorzegd!
\par 24 Maar in die dagen, na die verdrukking, zal de zon verduisterd worden, en de maan zal haar schijnsel niet geven.
\par 25 En de sterren des hemels zullen daaruit vallen, en de krachten, die in de hemelen zijn, zullen bewogen worden.
\par 26 En alsdan zullen zij den Zoon des mensen zien, komende in de wolken, met grote kracht en heerlijkheid.
\par 27 En alsdan zal Hij Zijn engelen uitzenden, en zal Zijn uitverkorenen bijeenvergaderen uit de vier winden, van het uiterste der aarde, tot het uiterste des hemels.
\par 28 En leert van den vijgeboom deze gelijkenis; wanneer nu zijn tak teder wordt, en de bladeren uitspruiten, zo weet gij, dat de zomer nabij is.
\par 29 Alzo ook gij, wanneer gij deze dingen zult zien geschieden, zo weet, dat het nabij, voor de deur is.
\par 30 Voorwaar, Ik zeg u, dat dit geslacht niet zal voorbijgaan, totdat al deze dingen zullen geschied zijn.
\par 31 De hemel en de aarde zullen voorbijgaan; maar Mijn woorden zullen geenszins voorbijgaan.
\par 32 Maar van dien dag en die ure weet niemand, noch de engelen, die in den hemel zijn, noch de Zoon, dan de Vader.
\par 33 Ziet toe, waakt en bidt; want gij weet niet, wanneer de tijd is.
\par 34 Gelijk een mens, buiten 's lands reizende, zijn huis verliet, en zijn dienstknechten macht gaf, en elk zijn werk, en den deurwachter gebood, dat hij zou waken;
\par 35 Zo waakt dan (want gij weet niet, wanneer de heer des huizes komen zal, des avonds laat, of ter middernacht, of met het hanengekraai, of in den morgenstond);
\par 36 Opdat hij niet onvoorziens kome, en u slapende vinde.
\par 37 En hetgeen Ik u zeg, dat zeg Ik allen: Waakt.

\chapter{14}

\par 1 En het pascha, en het feest der ongehevelde broden was na twee dagen. En de overpriesters en de Schriftgeleerden zochten, hoe zij Hem met listigheid vangen en doden zouden.
\par 2 Maar zij zeiden: Niet in het feest, opdat niet misschien oproer onder het volk worde.
\par 3 En als Hij te Bethanie was, in het huis van Simon, den melaatse, daar Hij aan tafel zat, kwam een vrouw, hebbende een albasten fles met zalf van onvervalsten nardus, van groten prijs; en de albasten fles gebroken hebbende, goot die op Zijn hoofd.
\par 4 En er waren sommigen, die dat zeer kwalijk namen bij zichzelven, en zeiden: Waartoe is dit verlies der zalf geschied?
\par 5 Want dezelve had kunnen boven de driehonderd penningen verkocht, en die den armen gegeven worden; en zij vergrimden tegen haar.
\par 6 Maar Jezus zeide: Laat af van haar; wat doet gij haar moeite aan? Zij heeft een goed werk aan Mij gewrocht.
\par 7 Want de armen hebt gij altijd met u, en wanneer gij wilt, kunt gij hun weldoen; maar Mij hebt gij niet altijd.
\par 8 Zij heeft gedaan, hetgeen zij konde; zij is voorgekomen, om Mijn lichaam te zalven, tot een voorbereiding ter begrafenis.
\par 9 Voorwaar zeg Ik u: Alwaar dit Evangelie gepredikt zal worden in de gehele wereld, daar zal ook tot haar gedachtenis gesproken worden, van hetgeen zij gedaan heeft.
\par 10 En Judas Iskariot, een van de twaalven, ging heen tot de overpriesters, opdat hij Hem hun zou overleveren.
\par 11 En zij, dat horende, waren verblijd, en beloofden hem geld te geven; en hij zocht, hoe hij Hem bekwamelijk overleveren zou.
\par 12 En op den eersten dag der ongehevelde broden, wanneer zij het pascha slachtten, zeiden Zijn discipelen tot Hem: Waar wilt Gij, dat wij heengaan, en bereiden, dat Gij het pascha eet?
\par 13 En Hij zond twee van Zijn discipelen uit, en zeide tot hen: Gaat henen in de stad, en u zal een mens ontmoeten, dragende een kruik water, volgt dien;
\par 14 En zo waar hij ingaat, zegt tot den heer des huizes: De Meester zegt: Waar is de eetzaal, daar Ik het pascha met Mijn discipelen eten zal?
\par 15 En hij zal u wijzen een grote opperzaal, toegerust en gereed; bereidt het ons aldaar.
\par 16 En Zijn discipelen gingen uit, en kwamen in de stad, en vonden het, gelijk Hij hun gezegd had, en bereidden het pascha.
\par 17 En als het avond geworden was, kwam Hij met de twaalven.
\par 18 En als zij aanzaten en aten, zeide Jezus: Voorwaar, Ik zeg u, dat een van u, die met Mij eet, Mij zal verraden.
\par 19 En zij begonnen bedroefd te worden, en de een na den ander tot Hem te zeggen: Ben ik het? En een ander: Ben ik het?
\par 20 Maar Hij antwoordde en zeide tot hen: Het is een uit de twaalven, die met Mij in den schotel indoopt.
\par 21 De Zoon des mensen gaat wel heen, gelijk van Hem geschreven is; maar wee dien mens, door welken de Zoon des mensen verraden wordt! Het ware hem goed, zo die mens niet geboren ware geweest.
\par 22 En als zij aten, nam Jezus brood, en als Hij gezegend had, brak Hij het, en gaf het hun, en zeide: Neemt, eet, dat is Mijn lichaam.
\par 23 En Hij nam den drinkbeker, en gedankt hebbende, gaf hun dien; en zij dronken allen uit denzelven.
\par 24 En Hij zeide tot hen: Dat is Mijn bloed, het bloed des Nieuwen Testaments, hetwelk voor velen vergoten wordt.
\par 25 Voorwaar, Ik zeg u, dat Ik niet meer zal drinken van de vrucht des wijnstoks, tot op dien dag, wanneer Ik dezelve nieuw zal drinken in het Koninkrijk Gods.
\par 26 En als zij den lofzang gezongen hadden, gingen zij uit naar den Olijfberg.
\par 27 En Jezus zeide tot hen: Gij zult in dezen nacht allen aan Mij geergerd worden; want er is geschreven: Ik zal den Herder slaan, en de schapen zullen verstrooid worden.
\par 28 Maar nadat Ik zal opgestaan zijn, zal Ik u voorgaan naar Galilea.
\par 29 En Petrus zeide tot Hem: Of zij ook allen geergerd wierden, zo zal ik toch niet geergerd worden.
\par 30 En Jezus zeide tot hem: Voorwaar, Ik zeg u, dat heden in dezen nacht, eer de haan tweemaal gekraaid zal hebben, gij Mij driemaal zult verloochenen.
\par 31 Maar hij zeide nog des te meer: Al moest ik met U sterven, zo zal ik U geenszins verloochenen! En insgelijks zeiden zij ook allen.
\par 32 En zij kwamen in een plaats, welker naam was Gethsemane, en Hij zeide tot Zijn discipelen: Zit hier neder, totdat Ik gebeden zal hebben.
\par 33 En Hij nam met Zich Petrus, en Jakobus, en Johannes, en begon verbaasd en zeer beangst te worden;
\par 34 En zeide tot hen: Mijn ziel is geheel bedroefd tot den dood toe; blijft hier, en waakt.
\par 35 En een weinig voortgegaan zijnde, viel Hij op de aarde, en bad, zo het mogelijk ware, dat die ure van Hem voorbijginge.
\par 36 En Hij zeide: Abba, Vader! alle dingen zijn U mogelijk; neem dezen drinkbeker van Mij weg, doch niet wat Ik wil, maar wat Gij wilt.
\par 37 En Hij kwam, en vond hen slapende, en zeide tot Petrus: Simon! slaapt gij? Kunt gij niet een uur waken?
\par 38 Waakt en bidt, opdat gij niet in verzoeking komt; de geest is wel gewillig, maar het vlees is zwak.
\par 39 En wederom heengegaan zijnde, bad Hij, sprekende dezelfde woorden.
\par 40 En wedergekeerd zijnde, vond Hij hen wederom slapende, want hun ogen waren bezwaard; en zij wisten niet, wat zij Hem antwoorden zouden.
\par 41 En Hij kwam ten derden male, en zeide tot hen: Slaapt nu voort, en rust; het is genoeg, de ure is gekomen; ziet, de Zoon des mensen wordt overgeleverd in de handen der zondaren.
\par 42 Staat op, laat ons gaan; ziet, die Mij verraadt, is nabij.
\par 43 En terstond, als Hij nog sprak, kwam Judas aan, die een was van de twaalven, en met hem een grote schare, met zwaarden en stokken, gezonden van de overpriesters, en de Schriftgeleerden, en de ouderlingen.
\par 44 En die Hem verried, had hun een gemeen teken gegeven, zeggende: Dien ik kussen zal, Die is het, grijpt Hem, en leidt Hem zekerlijk henen.
\par 45 En als hij gekomen was, ging hij terstond tot Hem, en zeide: Rabbi, en kuste Hem.
\par 46 En zij sloegen hun handen aan Hem, en grepen Hem.
\par 47 En een dergenen, die daarbij stonden, het zwaard trekkende, sloeg den dienstknecht des hogepriesters, en hieuw hem zijn oor af.
\par 48 En Jezus, antwoordende, zeide tot hen: Zijt gij uitgegaan, met zwaarden en stokken, als tegen een moordenaar, om Mij te vangen?
\par 49 Dagelijks was Ik bij ulieden in den tempel, lerende, en gij hebt Mij niet gegrepen; maar dit geschiedt, opdat de Schriften vervuld zouden worden.
\par 50 En zij, Hem verlatende, zijn allen gevloden.
\par 51 En een zeker jongeling volgde Hem, hebbende een lijnwaad omgedaan over het naakte lijf, en de jongelingen grepen hem.
\par 52 En hij, het lijnwaad verlatende, is naakt van hen gevloden.
\par 53 En zij leidden Jezus henen tot den hogepriester; en bij hem vergaderden al de overpriesters, en de ouderlingen, en de Schriftgeleerden.
\par 54 En Petrus volgde Hem van verre, tot binnen in de zaal des hogepriesters, en hij was mede zittende met de dienaren, en zich warmende bij het vuur.
\par 55 En de overpriesters, en de gehele raad, zochten getuigenis tegen Jezus, om Hem te doden, en vonden niet.
\par 56 Want velen getuigden valselijk tegen Hem, en de getuigenissen waren niet eenparig.
\par 57 En enigen, opstaande, getuigden valselijk tegen Hem, zeggende:
\par 58 Wij hebben Hem horen zeggen: Ik zal dezen tempel, die met handen gemaakt is, afbreken, en in drie dagen een anderen, zonder handen gemaakt, bouwen.
\par 59 En ook alzo was hun getuigenis niet eenparig.
\par 60 En de hogepriester, in het midden opstaande, vraagde Jezus, zeggende: Antwoordt Gij niets? Wat getuigen dezen tegen U;
\par 61 Maar Hij zweeg stil, en antwoordde niets. Wederom vraagde Hem de hogepriester, en zeide tot Hem: Zijt Gij de Christus, de Zoon des gezegenden Gods?
\par 62 En Jezus zeide: Ik ben het. En gijlieden zult den Zoon des mensen zien zitten ter rechter hand der kracht Gods, en komen met de wolken des hemels.
\par 63 En de hogepriester, verscheurende zijn klederen, zeide: Wat hebben wij nog getuigen van node?
\par 64 Gij hebt de gods lastering gehoord; wat dunkt ulieden? En zij allen veroordeelden Hem, des doods schuldig te zijn.
\par 65 En sommigen begonnen Hem te bespuwen, en Zijn aangezicht te bedekken, en met vuisten te slaan, en tot Hem te zeggen: Profeteer! En de dienaars gaven Hem kinnebakslagen.
\par 66 En als Petrus beneden in de zaal was, kwam een van de dienstmaagden des hogepriesters;
\par 67 En ziende Petrus zich warmende, zag zij hem aan, en zeide: Ook gij waart met Jezus den Nazarener.
\par 68 Maar hij heeft het geloochend, zeggende: Ik ken Hem niet, en ik weet niet wat gij zegt. En hij ging buiten in de voorzaal, en de haan kraaide.
\par 69 En de dienstmaagd, hem wederom ziende, begon te zeggen tot degenen, die daarbij stonden: Deze is een van die.
\par 70 Maar hij loochende het wederom. En een weinig daarna, die daarbij stonden, zeiden wederom tot Petrus: Waarlijk, gij zijt een van die; want gij zijt ook een Galileer, en uw spraak gelijkt.
\par 71 En hij begon zichzelven te vervloeken en te zweren: Ik ken dezen Mens niet, Dien gij zegt.
\par 72 En de haan kraaide de tweede maal; en Petrus werd indachtig het woord, hetwelk Jezus tot hem gezegd had: Eer de haan tweemaal gekraaid zal hebben, zult gij Mij driemaal verloochenen. En hij, zich van daar makende, weende.

\chapter{15}

\par 1 En terstond, des morgens vroeg, hielden de overpriesters te zamen raad, met de ouderlingen en Schriftgeleerden, en den gehelen raad, en Jezus gebonden hebbende, brachten zij Hem heen, en gaven Hem aan Pilatus over.
\par 2 En Pilatus vraagde Hem: Zijt Gij de Koning der Joden? En Hij antwoordende, zeide tot hem: Gij zegt het.
\par 3 En de overpriesters beschuldigden Hem van vele zaken; maar Hij antwoordde niets.
\par 4 En Pilatus vraagde Hem wederom, zeggende: Antwoordt Gij niet? Zie, hoe vele zaken zij tegen U getuigen!
\par 5 En Jezus heeft niet meer geantwoord, zodat Pilatus zich verwonderde.
\par 6 En op het feest liet hij hun een gevangene los, wien zij ook begeerden.
\par 7 En er was een, genaamd Bar-abbas, gevangen met andere medeoproermakers, die in het oproer een doodslag gedaan had.
\par 8 En de schare riep uit, en begon te begeren, dat hij deed, gelijk hij hun altijd gedaan had.
\par 9 En Pilatus antwoordde hun, zeggende: Wilt gij, dat ik u den Koning der Joden loslate?
\par 10 (Want hij wist, dat de overpriesters Hem door nijd overgeleverd hadden.)
\par 11 Maar de overpriesters bewogen de schare, dat hij hun liever Bar-abbas zou loslaten.
\par 12 En Pilatus, antwoordende, zeide wederom tot hen: Wat wilt gij dan, dat ik met Hem doen zal, Dien gij een Koning der Joden noemt?
\par 13 En zij riepen wederom: Kruis Hem.
\par 14 Doch Pilatus zeide tot hen: Wat heeft Hij dan kwaads gedaan? En zij riepen te meer: Kruis Hem!
\par 15 Pilatus nu, willende der schare genoeg doen, heeft hun Bar-abbas losgelaten, en gaf Jezus over, als hij Hem gegeseld had, om gekruist te worden.
\par 16 En de krijgsknechten leidden Hem binnen in de zaal, welke is het rechthuis, en riepen de ganse bende samen;
\par 17 En deden Hem een purperen mantel aan, en een doornenkroon gevlochten hebbende, zetten Hem die op;
\par 18 En begonnen Hem te groeten, zeggende: Wees gegroet, Gij Koning der Joden!
\par 19 En sloegen Zijn hoofd met een rietstok, en bespogen Hem, en vallende op de knieen, aanbaden Hem.
\par 20 En als zij Hem bespot hadden, deden zij Hem den purperen mantel af, en deden Hem Zijn eigen klederen aan, en leidden Hem uit, om Hem te kruisigen.
\par 21 En zij dwongen een Simon van Cyrene, die daar voorbijging, komende van den akker, den vader van Alexander en Rufus, dat hij Zijn kruis droeg.
\par 22 En zij brachten Hem tot de plaats Golgotha, hetwelk is, overgezet zijnde, Hoofdschedelplaats.
\par 23 En zij gaven Hem gemirreden wijn te drinken; maar Hij nam dien niet.
\par 24 En als zij Hem gekruisigd hadden, verdeelden zij Zijn klederen, werpende het lot over dezelve, wat een iegelijk wegnemen zou.
\par 25 En het was de derde ure, en zij kruisigden Hem.
\par 26 En het opschrift Zijner beschuldiging was boven Hem geschreven: DE KONING DER JODEN.
\par 27 En zij kruisigden met Hem twee moordenaars, een aan Zijn rechter-,en een aan Zijn linker zijde.
\par 28 En de Schrift is vervuld geworden, die daar zegt: En Hij is met de misdadigers gerekend.
\par 29 En die voorbijgingen, lasterden Hem, schuddende hun hoofden, en zeggende: Ha! Gij, die den tempel afbreekt, en in drie dagen opbouwt,
\par 30 Behoud Uzelven, en kom af van het kruis.
\par 31 En insgelijks ook de overpriesters, met de Schriftgeleerden, zeiden tot elkander, al spottende: Hij heeft anderen verlost; Zichzelven kan Hij niet verlossen.
\par 32 De Christus, de Koning Israels, kome nu af van het kruis, opdat wij het zien en geloven mogen. Ook die met Hem gekruist waren, smaadden Hem.
\par 33 En als de zesde ure gekomen was, werd er duisternis over de gehele aarde, tot de negende ure toe.
\par 34 En ter negender ure, riep Jezus met een grote stem, zeggende: ELOI, ELOI, LAMMA SABACHTANI, hetwelk is, overgezet zijnde: Mijn God, Mijn God! Waarom hebt Gij Mij verlaten?
\par 35 En sommigen van die daarbij stonden, dit horende, zeiden: Ziet, Hij roept Elias.
\par 36 En er liep een, en vulde een spons met edik, en stak ze op een rietstok, en gaf Hem te drinken, zeggende: Houdt stil, laat ons zien, of Elias komt, om Hem af te nemen.
\par 37 En Jezus, een grote stem van Zich gegeven hebbende, gaf den geest.
\par 38 En het voorhangsel des tempels scheurde in tweeen, van boven tot beneden.
\par 39 En de hoofdman over honderd, die daarbij tegenover Hem stond, ziende, dat Hij alzo roepende den geest gegeven had, zeide: Waarlijk, deze Mens was Gods Zoon!
\par 40 En er waren ook vrouwen, van verre dit aanschouwende, onder welke ook was Maria Magdalena, en Maria, de moeder van Jakobus, den kleine, en van Joses, en Salome;
\par 41 Welke ook, toen Hij in Galilea was, Hem waren gevolgd, en Hem gediend hadden; en vele andere vrouwen, die met Hem naar Jeruzalem opgekomen waren.
\par 42 En als het nu avond was geworden, dewijl het de voorbereiding was, welke is de voorsabbat;
\par 43 Kwam Jozef, die van Arimathea was, een eerlijk raadsheer, die ook zelf het Koninkrijk Gods was verwachtende, en zich verstoutende, ging hij in tot Pilatus, en begeerde het lichaam van Jezus.
\par 44 En Pilatus verwonderde zich, dat Hij alrede gestorven was; en den hoofdman over honderd tot zich geroepen hebbende, vraagde hem, of Hij lang gestorven was.
\par 45 En als hij het van den hoofdman over honderd verstaan had, schonk hij Jozef het lichaam.
\par 46 En hij kocht fijn lijnwaad, en Hem afgenomen hebbende, wond Hem in dat fijne lijnwaad, en leide Hem in een graf, hetwelk uit een steenrots gehouwen was; en hij wentelde een steen tegen de deur des grafs.
\par 47 En Maria Magdalena, en Maria, de moeder van Joses, aanschouwden, waar Hij gelegd werd.

\chapter{16}

\par 1 En als de sabbat voorbijgegaan was, hadden Maria Magdalena, en Maria, de moeder van Jakobus, en Salome specerijen gekocht, opdat zij kwamen en Hem zalfden.
\par 2 En zeer vroeg op den eersten dag der week, kwamen zij tot het graf, als de zon opging;
\par 3 En zeiden tot elkander: Wie zal ons den steen van de deur des grafs afwentelen?
\par 4 (En opziende zagen zij, dat de steen afgewenteld was) want hij was zeer groot.
\par 5 En in het graf ingegaan zijnde, zagen zij een jongeling, zittende ter rechter zijde, bekleed met een wit lang kleed, en werden verbaasd.
\par 6 Maar hij zeide tot haar: Zijt niet verbaasd; gij zoekt Jezus den Nazarener, Die gekruist was; Hij is opgestaan; Hij is hier niet; ziet de plaats, waar zij Hem gelegd hadden.
\par 7 Doch gaat heen, zegt Zijnen discipelen, en Petrus, dat Hij u voorgaat naar Galilea; aldaar zult gij Hem zien, gelijk Hij ulieden gezegd heeft.
\par 8 En zij, haastelijk uitgegaan zijnde, vloden van het graf, en beving en ontzetting had haar bevangen; en zij zeiden niemand iets; want zij waren bevreesd.
\par 9 En als Jezus opgestaan was, des morgens vroeg, op den eersten dag der week, verscheen Hij eerst aan Maria Magdalena, uit welke Hij zeven duivelen uitgeworpen had.
\par 10 Deze, heengaande, boodschapte het dengenen, die met Hem geweest waren, welke treurden en weenden.
\par 11 En als dezen hoorden, dat Hij leefde, en van haar gezien was, geloofden zij het niet.
\par 12 En na dezen is Hij geopenbaard in een andere gedaante, aan twee van hen, daar zij wandelden, en in het veld gingen.
\par 13 Dezen, ook heengaande, boodschapten het aan de anderen; maar zij geloofden ook die niet.
\par 14 Daarna is Hij geopenbaard aan de elven, daar zij aanzaten, en verweet hun hun ongelovigheid en hardigheid des harten, omdat zij niet geloofd hadden degenen, die Hem gezien hadden, nadat Hij opgestaan was.
\par 15 En Hij zeide tot hen: Gaat heen in de gehele wereld, predikt het Evangelie aan alle kreaturen.
\par 16 Die geloofd zal hebben, en gedoopt zal zijn, zal zalig worden; maar die niet zal geloofd hebben, zal verdoemd worden.
\par 17 En degenen, die geloofd zullen hebben, zullen deze tekenen volgen: in Mijn Naam zullen zij duivelen uitwerpen; met nieuwe tongen zullen zij spreken.
\par 18 Slangen zullen zij opnemen; en al is het, dat zij iets dodelijks zullen drinken, dat zal hun niet schaden; op kranken zullen zij de handen leggen, en zij zullen gezond worden.
\par 19 De Heere dan, nadat Hij tot hen gesproken had, is opgenomen in den hemel, en is gezeten aan de rechter hand Gods.
\par 20 En zij, uitgegaan zijnde, predikten overal, en de Heere wrocht mede, en bevestigde het Woord door tekenen, die daarop volgden. Amen.




\end{document}