\begin{document}

\title{Lukas}



\chapter{1}

\par 1 Nademaal velen ter hand genomen hebben, om in orde te stellen een verhaal van de dingen, die onder ons volkomen zekerheid hebben;
\par 2 Gelijk ons overgeleverd hebben, die van den beginne zelven aanschouwers en dienaars des Woords geweest zijn;
\par 3 Zo heeft het ook mij goed gedacht, hebbende alles van voren aan naarstiglijk onderzocht, vervolgens aan u te schrijven, voortreffelijke Theofilus!
\par 4 Opdat gij moogt kennen de zekerheid der dingen, waarvan gij onderwezen zijt.
\par 5 In de dagen van Herodes, den koning van Judea, was een zeker priester, met name Zacharias, van de dagorde van Abia; en zijn vrouw was uit de dochteren van Aaron, en haar naam Elizabet.
\par 6 En zij waren beiden rechtvaardig voor God, wandelende in al de geboden en rechten des Heeren, onberispelijk.
\par 7 En zij hadden geen kind, omdat Elizabet onvruchtbaar was, en zij beiden verre op hun dagen gekomen waren.
\par 8 En het geschiedde, dat, als hij het priesterambt bediende voor God, in de beurt zijner dagorde.
\par 9 Naar de gewoonte der priesterlijke bediening, hem te lote was gevallen, dat hij zoude ingaan in den tempel des Heeren om te reukofferen.
\par 10 En al de menigte des volks was buiten, biddende, ten ure des reukoffers.
\par 11 En van hem werd gezien een engel des Heeren, staande ter rechter zijde van het altaar des reukoffers.
\par 12 En Zacharias, hem ziende, werd ontroerd, en vreze is op hem gevallen.
\par 13 Maar de engel zeide tot hem: Vrees niet, Zacharias! want uw gebed is verhoord, en uw vrouw Elizabet zal u een zoon baren, en gij zult zijn naam heten Johannes.
\par 14 En u zal blijdschap en verheuging zijn, en velen zullen zich over zijn geboorte verblijden.
\par 15 Want hij zal groot zijn voor den Heere; noch wijn, noch sterken drank zal hij drinken, en hij zal met den Heiligen Geest vervuld worden, ook van zijner moeders lijf aan.
\par 16 En hij zal velen der kinderen Israels bekeren tot den Heere, hun God.
\par 17 En hij zal voor Hem heengaan, in den geest en de kracht van Elias, om te bekeren de harten der vaderen tot de kinderen, en de ongehoorzamen tot de voorzichtigheid der rechtvaardigen, om den Heere te bereiden een toegerust volk.
\par 18 En Zacharias zeide tot den engel: Waarbij zal ik dat weten? Want ik ben oud, en mijn vrouw is verre op haar dagen gekomen.
\par 19 En de engel antwoordde en zeide tot hem: Ik ben Gabriel, die voor God sta, en ben uitgezonden, om tot u te spreken, en u deze dingen te verkondigen.
\par 20 En zie, gij zult zwijgen, en niet kunnen spreken, tot op den dag, dat deze dingen geschied zullen zijn; om dies wil, dat gij mijn woorden niet geloofd hebt, welke vervuld zullen worden op hun tijd.
\par 21 En het volk was wachtende op Zacharias, en zij waren verwonderd, dat hij zo lang vertoefde in den tempel.
\par 22 En als hij uitkwam, kon hij tot hen niet spreken; en zij bekenden, dat hij een gezicht in den tempel gezien had. En hij wenkte hun toe, en bleef stom.
\par 23 En het geschiedde, als de dagen zijner bediening vervuld waren, dat hij naar zijn huis ging.
\par 24 En na die dagen werd Elizabet, zijn vrouw, bevrucht; en zij verborg zich vijf maanden, zeggende:
\par 25 Alzo heeft mij de Heere gedaan, in de dagen, in welke Hij mij aangezien heeft, om mijn versmaadheid onder de mensen weg te nemen.
\par 26 En in de zesde maand werd de engel Gabriel van God gezonden naar een stad in Galilea, genaamd Nazareth;
\par 27 Tot een maagd, die ondertrouwd was met een man, wiens naam was Jozef, uit den huize Davids; en de naam der maagd was Maria.
\par 28 En de engel tot haar ingekomen zijnde, zeide: Wees gegroet, gij begenadigde; de Heere is met u; gij zijt gezegend onder de vrouwen.
\par 29 En als zij hem zag, werd zij zeer ontroerd over dit zijn woord, en overleide, hoedanig deze groetenis mocht zijn.
\par 30 En de engel zeide tot haar: Vrees niet, Maria, want gij hebt genade bij God gevonden.
\par 31 En zie, gij zult bevrucht worden, en een Zoon baren, en zult Zijn naam heten JEZUS.
\par 32 Deze zal groot zijn, en de Zoon des Allerhoogsten genaamd worden; en God, de Heere, zal Hem den troon van Zijn vader David geven.
\par 33 En Hij zal over het huis Jakobs Koning zijn in der eeuwigheid, en Zijns Koninkrijks zal geen einde zijn.
\par 34 En Maria zeide tot den engel: Hoe zal dat wezen, dewijl ik geen man bekenne?
\par 35 En de engel, antwoordende, zeide tot haar: De Heilige Geest zal over u komen, en de kracht des Allerhoogsten zal u overschaduwen; daarom ook, dat Heilige, Dat uit u geboren zal worden, zal Gods Zoon genaamd worden.
\par 36 En zie, Elizabet, uw nicht, is ook zelve bevrucht, met een zoon, in haar ouderdom; en deze maand is haar, die onvruchtbaar genaamd was, de zesde.
\par 37 Want geen ding zal bij God onmogelijk zijn.
\par 38 En Maria zeide: Zie, de dienstmaagd des Heeren; mij geschiede naar uw woord. En de engel ging weg van haar.
\par 39 En Maria, opgestaan zijnde in diezelfde dagen, reisde met haast naar het gebergte, in een stad van Juda;
\par 40 En kwam in het huis van Zacharias, en groette Elizabet.
\par 41 En het geschiedde, als Elizabet de groetenis van Maria hoorde, zo sprong het kindeken op in haar buik; en Elizabet werd vervuld met den Heiligen Geest;
\par 42 En riep uit met een grote stem, en zeide: Gezegend zijt gij onder de vrouwen, en gezegend is de vrucht uws buiks!
\par 43 En van waar komt mij dit, dat de moeder mijns Heeren tot mij komt?
\par 44 Want zie, als de stem uwer groetenis in mijn oren geschiedde, zo sprong het kindeken van vreugde op in mijn buik.
\par 45 En zalig is zij, die geloofd heeft; want de dingen, die haar van den Heere gezegd zijn, zullen volbracht worden.
\par 46 En Maria zeide: Mijn ziel maakt groot den Heere;
\par 47 En mijn geest verheugt zich in God, mijn Zaligmaker;
\par 48 Omdat Hij de nederheid Zijner dienstmaagd heeft aangezien; want zie, van nu aan zullen mij zalig spreken al de geslachten.
\par 49 Want grote dingen heeft aan mij gedaan Hij, Die machtig is, en heilig is Zijn Naam.
\par 50 En Zijn barmhartigheid is van geslacht tot geslacht over degenen, die Hem vrezen.
\par 51 Hij heeft een krachtig werk gedaan door Zijn arm; Hij heeft verstrooid de hoogmoedigen in de gedachten hunner harten.
\par 52 Hij heeft machtigen van de tronen afgetrokken, en nederigen heeft Hij verhoogd.
\par 53 Hongerigen heeft Hij met goederen vervuld; en rijken heeft Hij ledig weggezonden.
\par 54 Hij heeft Israel, Zijn knecht, opgenomen, opdat Hij gedachtig ware der barmhartigheid.
\par 55 (Gelijk Hij gesproken heeft tot onze vaderen, namelijk tot Abraham, en zijn zaad) in eeuwigheid.
\par 56 En Maria bleef bij haar omtrent drie maanden, en keerde weder tot haar huis.
\par 57 En de tijd van Elizabet werd vervuld, dat zij baren zoude, en zij baarde een zoon.
\par 58 En die daar rondom woonden, en haar magen hoorden, dat de Heere Zijn barmhartigheid grotelijks aan haar bewezen had, en waren met haar verblijd.
\par 59 En het geschiedde, dat zij op den achtsten dag kwamen, om het kindeken te besnijden, en noemden het Zacharias, naar den naam zijns vaders.
\par 60 En zijn moeder antwoordde en zeide: Niet alzo, maar hij zal Johannes heten.
\par 61 En zij zeiden tot haar: Er is niemand in uw maagschap, die met dien naam genaamd wordt.
\par 62 En zij wenkten zijn vader, hoe hij wilde, dat hij genaamd zou worden.
\par 63 En als hij een schrijftafeltje geeist had, schreef hij, zeggende: Johannes is zijn naam. En zij verwonderden zich allen.
\par 64 En terstond werd zijn mond geopend, en zijn tong losgemaakt; en hij sprak, God lovende.
\par 65 En er kwam vrees over allen, die rondom hen woonden; en in het gehele gebergte van Judea werd veel gesproken van al deze dingen.
\par 66 En allen, die het hoorden, namen het ter harte, zeggende: Wat zal toch dit kindeken wezen? En de hand des Heeren was met hem.
\par 67 En Zacharias, zijn vader, werd vervuld met den Heiligen Geest, en profeteerde, zeggende:
\par 68 Geloofd zij de Heere, de God Israels, want Hij heeft bezocht, en verlossing te weeg gebracht Zijn volke;
\par 69 En heeft een hoorn der zaligheid ons opgericht, in het huis van David, Zijn knecht;
\par 70 Gelijk Hij gesproken heeft door den mond Zijner heilige profeten, die van het begin der wereld geweest zijn;
\par 71 Namelijk een verlossing van onze vijanden, en van de hand al dergenen, die ons haten;
\par 72 Opdat Hij barmhartigheid deed aan onze vaderen, en gedachtig ware aan Zijn heilig verbond;
\par 73 En aan den eed, dien Hij Abraham, onzen vader, gezworen heeft, om ons te geven.
\par 74 Dat wij, verlost zijnde uit de hand onzer vijanden, Hem dienen zouden zonder vreze.
\par 75 In heiligheid en gerechtigheid voor Hem, al de dagen onzes levens.
\par 76 En gij, kindeken, zult een profeet des Allerhoogsten genaamd worden; want gij zult voor het aangezicht des Heeren heengaan, om Zijn wegen te bereiden;
\par 77 Om Zijn volk kennis der zaligheid te geven, in vergeving hunner zonden,
\par 78 door de innerlijke bewegingen der barmhartigheid onzes Gods, met welke ons bezocht heeft de Opgang uit de hoogte;
\par 79 Om te verschijnen dengenen, die gezeten zijn in duisternis en schaduw des doods; om onze voeten te richten op den weg des vredes.
\par 80 En het kindeken wies op, en werd gesterkt in den geest, en was in de woestijnen, tot den dag zijner vertoning aan Israel.

\chapter{2}

\par 1 En het geschiedde in diezelfde dagen, dat er een gebod uitging van den Keizer Augustus, dat de gehele wereld beschreven zou worden.
\par 2 Deze eerste beschrijving geschiedde, als Cyrenius over Syrie stadhouder was.
\par 3 En zij gingen allen om beschreven te worden, een iegelijk naar zijn eigen stad.
\par 4 En Jozef ging ook op van Galilea, uit de stad Nazareth, naar Judea, tot de stad Davids, die Bethlehem genaamd wordt, (omdat hij uit het huis en geslacht van David was);
\par 5 Om beschreven te worden met Maria, zijn ondertrouwde vrouw, welke bevrucht was.
\par 6 En het geschiedde, als zij daar waren, dat de dagen vervuld werden, dat zij baren zoude.
\par 7 En zij baarde haar eerstgeboren Zoon, en wond Hem in doeken, en leide Hem neder in de kribbe, omdat voor henlieden geen plaats was in de herberg.
\par 8 En er waren herders in diezelfde landstreek, zich houdende in het veld, en hielden de nachtwacht over hun kudde.
\par 9 En ziet, een engel des Heeren stond bij hen, en de heerlijkheid des Heeren omscheen hen, en zij vreesden met grote vreze.
\par 10 En de engel zeide tot hen: Vreest niet, want, ziet, ik verkondig u grote blijdschap, die al den volke wezen zal;
\par 11 Namelijk dat u heden geboren is de Zaligmaker, welke is Christus, de Heere, in de stad Davids.
\par 12 En dit zal u het teken zijn: gij zult het Kindeken vinden in doeken gewonden, en liggende in de kribbe.
\par 13 En van stonde aan was er met den engel een menigte des hemelsen heirlegers, prijzende God en zeggende:
\par 14 Ere zij God in de hoogste hemelen, en vrede op aarde, in de mensen een welbehagen.
\par 15 En het geschiedde, als de engelen van hen weggevaren waren naar den hemel, dat de herders tot elkander zeiden: Laat ons dan heengaan naar Bethlehem, en laat ons zien het woord, dat er geschied is, hetwelk de Heere ons heeft verkondigd.
\par 16 En zij kwamen met haast, en vonden Maria en Jozef, en het Kindeken liggende in de kribbe.
\par 17 En als zij Het gezien hadden, maakten zij alom bekend het woord, dat hun van dit Kindeken gezegd was.
\par 18 En allen, die het hoorden, verwonderden zich over hetgeen hun gezegd werd van de herders.
\par 19 Doch Maria bewaarde deze woorden alle te zamen, overleggende die in haar hart.
\par 20 En de herders keerde wederom, verheerlijkende en prijzende God over alles, wat zij gehoord en gezien hadden, gelijk tot hen gesproken was.
\par 21 En als acht dagen vervuld waren, dat men het Kindeken besnijden zou, zo werd Zijn Naam genaamd JEZUS, welke genaamd was van den engel, eer Hij in het lichaam ontvangen was.
\par 22 En als de dagen harer reiniging vervuld waren, naar de wet van Mozes, brachten zij Hem te Jeruzalem, opdat zij Hem den Heere voorstelden;
\par 23 (Gelijk geschreven is in de wet des Heeren: Al wat mannelijk is, dat de moeder opent, zal den Heere heilig genaamd worden.)
\par 24 En opdat zij offerande gaven, naar hetgeen in de wet des Heeren gezegd is, een paar tortelduiven, of twee jonge duiven.
\par 25 En ziet, er was een mens te Jeruzalem, wiens naam was Simeon; en deze mens was rechtvaardig en godvrezende; verwachtende de vertroosting Israels, en de Heilige Geest was op hem.
\par 26 En hem was een Goddelijke openbaring gedaan door den Heiligen Geest, dat hij den dood niet zien zoude, eer hij den Christus des Heeren zou zien.
\par 27 En hij kwam door den Geest in den tempel. En als de ouders het Kindeken Jezus inbrachten, om naar de gewoonte der wet met Hem te doen;
\par 28 Zo nam hij Hetzelve in zijn armen, en loofde God, en zeide:
\par 29 Nu laat Gij, Heere! Uw dienstknecht gaan in vrede naar Uw woord;
\par 30 Want mijn ogen hebben Uw zaligheid gezien,
\par 31 Die Gij bereid hebt voor het aangezicht van al de volken:
\par 32 Een Licht tot verlichting der heidenen, en tot heerlijkheid van Uw volk Israel.
\par 33 En Jozef en Zijn moeder verwonderden zich over hetgeen van Hem gezegd werd.
\par 34 En Simeon zegende henlieden, en zeide tot Maria, Zijn moeder: Zie, Deze wordt gezet tot een val en opstanding veler in Israel, en tot een teken, dat wedersproken zal worden.
\par 35 (En ook een zwaard zal door uw eigen ziel gaan) opdat de gedachten uit vele harten geopenbaard worden.
\par 36 En er was Anna, een profetesse, een dochter van Fanuel, uit den stam van Aser; deze was tot groten ouderdom gekomen, welke met haar man zeven jaren had geleefd van haar maagdom af.
\par 37 En zij was een weduwe van omtrent vier en tachtig jaren, dewelke niet week uit den tempel, met vasten en bidden, God dienende nacht en dag.
\par 38 En deze, te dierzelfder ure daarbij komende, heeft insgelijks den Heere beleden, en sprak van Hem tot allen, die de verlossing in Jeruzalem verwachtten.
\par 39 En als zij alles voleindigd hadden, wat naar de wet des Heeren te doen was, keerden zij weder naar Galilea, tot hun stad Nazareth.
\par 40 En het Kindeken wies op, en werd gesterkt in den geest, en vervuld met wijsheid; en de genade Gods was over Hem.
\par 41 En Zijn ouders reisden alle jaar naar Jeruzalem, op het feest van pascha.
\par 42 En toen Hij twaalf jaren oud geworden was, en zij naar Jeruzalem opgegaan waren, naar de gewoonte van den feestdag;
\par 43 En de dagen aldaar voleindigd hadden, toen zij wederkeerden, bleef het Kind Jezus te Jeruzalem, en Jozef en Zijn moeder wisten het niet.
\par 44 Maar menende, dat Hij in het gezelschap op den weg was, gingen zij een dagreize, en zochten Hem onder de magen, en onder de bekenden.
\par 45 En als zij Hem niet vonden, keerden zij wederom naar Jeruzalem, Hem zoekende.
\par 46 En het geschiedde, na drie dagen, dat zij Hem vonden in den tempel, zittende in het midden der leraren, hen horende, en hen ondervragende.
\par 47 En allen, die Hem hoorden, ontzetten zich over Zijn verstand en antwoorden.
\par 48 En zij, Hem ziende, werden verslagen; en Zijn moeder zeide tot Hem: Kind! waarom hebt Gij ons zo gedaan? Zie, Uw vader en ik hebben U met angst gezocht.
\par 49 En Hij zeide tot hen: Wat is het, dat gij Mij gezocht hebt? Wist gij niet, dat Ik moet zijn in de dingen Mijns Vaders?
\par 50 En zij verstonden het woord niet, dat Hij tot hen sprak.
\par 51 En Hij ging met hen af, en kwam te Nazareth, en was hun onderdanig. En Zijn moeder bewaarde al deze dingen in haar hart.
\par 52 En Jezus nam toe in wijsheid, en in grootte, en in genade bij God en de mensen.

\chapter{3}

\par 1 En in het vijftiende jaar der regering van den keizer Tiberius, als Pontius Pilatus stadhouder was over Judea, en Herodes een viervorst over Galilea, en Filippus, zijn broeder, een viervorst over Iturea en over het land Trachonitis, en Lysanias een viervorst over Abilene;
\par 2 Onder de hogepriesters Annas en Kajafas, geschiedde het woord Gods tot Johannes, den zoon van Zacharias, in de woestijn.
\par 3 En hij kwam in al het omliggende land der Jordaan, predikende den doop der bekering tot vergeving der zonden.
\par 4 Gelijk geschreven is in het boek der woorden van Jesaja, den profeet, zeggende: De stem des roependen in de woestijn: Bereidt den weg des Heeren, maakt Zijn paden recht!
\par 5 Alle dal zal gevuld worden, en alle berg en heuvel zal vernederd worden, en de kromme wegen zullen tot een rechten weg worden, en de oneffen tot effen wegen.
\par 6 En alle vlees zal de zaligheid Gods zien.
\par 7 Hij zeide dan tot de scharen, die uitkwamen, om van hem gedoopt te worden: Gij adderengebroedsels, wie heeft u aangewezen te vlieden van den toekomenden toorn?
\par 8 Brengt dan vruchten voort der bekering waardig; en begint niet te zeggen bij uzelven: Wij hebben Abraham tot een vader; want ik zeg u, dat God zelfs uit deze stenen Abraham kinderen kan verwekken.
\par 9 En de bijl ligt ook alrede aan den wortel der bomen; alle boom dan, die geen goede vrucht voortbrengt, wordt uitgehouwen, en in het vuur geworpen.
\par 10 En de scharen vraagden hem, zeggende: Wat zullen wij dan doen?
\par 11 En hij, antwoordende, zeide tot hen: Die twee rokken heeft, dele hem mede, die geen heeft; en die spijze heeft, doe desgelijks.
\par 12 En er kwamen ook tollenaars om gedoopt te worden, en zeiden tot hem: Meester! wat zullen wij doen?
\par 13 En hij zeide tot hen: Eist niet meer, dan hetgeen u gezet is.
\par 14 En hem vraagden ook de krijgslieden, zeggende: En wij, wat zullen wij doen? En hij zeide tot hen: Doet niemand overlast, en ontvreemdt niemand het zijne met bedrog, en laat u vergenoegen met uw bezoldigingen.
\par 15 En als het volk verwachtte, en allen in hun harten overleiden van Johannes, of hij niet mogelijk de Christus ware;
\par 16 Zo antwoordde Johannes aan allen, zeggende: Ik doop u wel met water; maar Hij komt, Die sterker is dan ik, Wien ik niet waardig ben den riem van Zijn schoenen te ontbinden; Deze zal u dopen met den Heiligen Geest en met vuur;
\par 17 Wiens wan in Zijn hand is, en Hij zal Zijn dorsvloer doorzuiveren, en de tarwe zal Hij in Zijn schuur samenbrengen; maar het kaf zal Hij met onuitblusselijk vuur verbranden.
\par 18 Hij dan, ook nog vele andere dingen vermanende, verkondigde den volke het Evangelie.
\par 19 Maar als Herodes, de viervorst van hem bestraft werd, om Herodias' wil, de vrouw van Filippus, zijn broeder, en over alle boze stukken, die Herodes deed,
\par 20 Zo heeft hij ook dit nog boven alles daar toegedaan, dat hij Johannes in de gevangenis gesloten heeft.
\par 21 En het geschiedde, toen al het volk gedoopt werd, en Jezus ook gedoopt was, en bad, dat de hemel geopend werd;
\par 22 En dat de Heilige Geest op Hem nederdaalde, in lichamelijke gedaante, gelijk een duif; en dat er een stem geschiedde uit den hemel, zeggende: Gij zijt Mijn geliefde Zoon, in U heb Ik Mijn welbehagen!
\par 23 En Hij, Jezus, begon omtrent dertig jaren oud te wezen, zijnde (alzo men meende) de zoon van Jozef, den zoon van Heli,
\par 24 Den zoon van Matthat, den zoon van Levi, den zoon van Melchi, den zoon van Janna, den zoon van Jozef,
\par 25 Den zoon van Mattathias, den zoon van Amos, den zoon van Naum, den zoon van Esli, den zoon van Naggai,
\par 26 Den zoon van Maath, den zoon van Mattathias, den zoon van Semei, den zoon van Jozef, den zoon van Juda,
\par 27 Den zoon van Johannes, den zoon van Rhesa, den zoon van Zorobabel, den zoon van Salathiel, den zoon van Neri,
\par 28 Den zoon van Melchi, den zoon van Addi, den zoon van Kosam, den zoon van Elmodam, den zoon van Er,
\par 29 Den zoon van Joses, den zoon van Eliezer, den zoon van Jorim, den zoon van Matthat, den zoon van Levi,
\par 30 Den zoon van Simeon, den zoon van Juda, den zoon van Jozef, den zoon van Jonan, den zoon van Eljakim,
\par 31 Den zoon van Meleas, den zoon van Mainan, den zoon van Mattatha, den zoon van Nathan, den zoon van David,
\par 32 Den zoon van Jesse, den zoon van Obed, den zoon van Booz, den zoon van Salmon, den zoon van Nahasson,
\par 33 Den zoon van Aminadab, den zoon van Aram, den zoon van Esrom, den zoon van Fares, den zoon van Juda,
\par 34 Den zoon van Jakob, den zoon van Izak, den zoon van Abraham, den zoon van Thara, den zoon van Nachor,
\par 35 Den zoon van Saruch, den zoon van Ragau, den zoon van Falek, den zoon van Heber, den zoon van Sala,
\par 36 Den zoon van Kainan, den zoon van Arfaxad, den zoon van Sem, den zoon van Noe, den zoon van Lamech,
\par 37 Den zoon van Mathusala, den zoon van Enoch, den zoon van Jared, den zoon van Malaleel, den zoon van Kainan,
\par 38 Den zoon van Enos, den zoon van Seth, den zoon van Adam, den zoon van God.

\chapter{4}

\par 1 En Jezus, vol des Heiligen Geestes, keerde wederom van de Jordaan, en werd door den Geest geleid in de woestijn;
\par 2 En werd veertig dagen verzocht van den duivel; en at gans niet in die dagen, en als dezelve geeindigd waren, zo hongerde Hem ten laatste.
\par 3 En de duivel zeide tot Hem: Indien Gij Gods Zoon zijt, zeg tot dezen steen, dat hij brood worde.
\par 4 En Jezus antwoordde hem, zeggende: Er is geschreven, dat de mens bij brood alleen niet zal leven, maar bij alle woord Gods.
\par 5 En als Hem de duivel geleid had op een hogen berg, toonde hij Hem al de koninkrijken der wereld, in een ogenblik tijds.
\par 6 En de duivel zeide tot Hem: Ik zal U al deze macht, en de heerlijkheid derzelver koninkrijken geven; want zij is mij overgegeven, en ik geef ze, wien ik ook wil;
\par 7 Indien Gij dan mij zult aanbidden, zo zal het alles Uw zijn.
\par 8 En Jezus, antwoordende, zeide tot hem: Ga weg van Mij, satan, want er is geschreven: Gij zult den Heere, uw God, aanbidden, en Hem alleen dienen.
\par 9 En hij leidde Hem naar Jeruzalem, en stelde Hem op de tinne des tempels, en zeide tot Hem: Indien Gij de Zoon Gods zijt, werp Uzelven van hier nederwaarts;
\par 10 Want er is geschreven, dat Hij Zijn engelen van U bevelen zal, dat zij U bewaren zullen;
\par 11 En dat zij U op de handen nemen zullen, opdat Gij Uw voet niet te eniger tijd aan een steen stoot.
\par 12 En Jezus, antwoordende, zeide tot hem: Er is gezegd: Gij zult den Heere, uw God, niet verzoeken.
\par 13 En als de duivel alle verzoeking voleindigd had, week hij van Hem voor een tijd.
\par 14 En Jezus keerde wederom, door de kracht des Geestes, naar Galilea; en het gerucht van Hem ging uit door het gehele omliggende land.
\par 15 En Hij leerde in hun synagogen, en werd van allen geprezen.
\par 16 En Hij kwam te Nazareth, daar Hij opgevoed was, en ging, naar Zijn gewoonte, op den dag des sabbats in de synagoge; en stond op om te lezen.
\par 17 En Hem werd gegeven het boek van den profeet Jesaja; en als Hij het boek opengedaan had, vond Hij de plaats, daar geschreven was:
\par 18 De Geest des Heeren is op Mij, daarom heeft Hij Mij gezalfd; Hij heeft Mij gezonden, om den armen het Evangelie te verkondigen, om te genezen, die gebroken zijn van hart;
\par 19 Om den gevangenen te prediken loslating, en den blinden het gezicht, om de verslagenen heen te zenden in vrijheid; om te prediken het aangename jaar des Heeren.
\par 20 En als Hij het boek toegedaan en den dienaar wedergegeven had, zat Hij neder; en de ogen van allen in de synagoge waren op Hem geslagen.
\par 21 En Hij begon tot hen te zeggen: Heden is deze Schrift in uw oren vervuld.
\par 22 En zij gaven Hem allen getuigenis, en verwonderden zich over de aangename woorden, die uit Zijn mond voortkwamen; en zeiden: Is deze niet de Zoon van Jozef?
\par 23 En Hij zeide tot hen: Gij zult zonder twijfel tot Mij dit spreekwoord zeggen: Medicijnmeester! genees Uzelven; al wat wij gehoord hebben, dat in Kapernaum geschied is, doe dat ook hier in Uw vaderland.
\par 24 En Hij zeide: Voorwaar Ik zeg u, dat geen profeet aangenaam is in zijn vaderland.
\par 25 Maar Ik zeg u in der waarheid: Er waren vele weduwen in Israel in de dagen van Elias, toen de hemel drie jaren en zes maanden gesloten was, zodat er grote hongersnood werd over het gehele land.
\par 26 En tot geen van haar werd Elias gezonden, dan naar Sarepta Sidonis, tot een vrouw, die weduwe was.
\par 27 En er waren vele melaatsen in Israel, ten tijde van den profeet Elisa; en geen van hen werd gereinigd, dan Naaman, de Syrier.
\par 28 En zij werden allen in de synagoge met toorn vervuld, als zij dit hoorden.
\par 29 En opstaande, wierpen zij Hem uit, buiten de stad, en leidden Hem op den top des bergs, op denwelken hun stad gebouwd was, om Hem van de steilte af te werpen.
\par 30 Maar Hij, door het midden van hen doorgegaan zijnde, ging weg.
\par 31 En Hij kwam af te Kapernaum, een stad van Galilea, en leerde hen op de sabbatdagen.
\par 32 En zij versloegen zich over Zijn leer, want Zijn woord was met macht.
\par 33 En in de synagoge was een mens, hebbende een geest eens onreinen duivels; en hij riep uit met grote stemme,
\par 34 Zeggende: Laat af, wat hebben wij met U te doen, Gij Jezus Nazarener? Zijt Gij gekomen, om ons te verderven? Ik ken U, wie Gij zijt, namelijk de Heilige Gods.
\par 35 En Jezus bestrafte hem, zeggende: Zwijg stil, en ga van hem uit. En de duivel, hem in het midden geworpen hebbende, voer van hem uit, zonder hem iets te beschadigen.
\par 36 En er kwam een verbaasdheid over allen; en zij spraken samen tot elkander, zeggende: Wat woord is dit, dat Hij met macht en kracht den onreinen geesten gebiedt, en zij varen uit?
\par 37 En het gerucht van Hem ging uit in alle plaatsen des omliggenden lands.
\par 38 En Jezus, opgestaan zijnde uit de synagoge, ging in het huis van Simon; en Simons vrouws moeder was met een grote koorts bevangen, en zij baden Hem voor haar.
\par 39 En staande boven haar, bestrafte Hij de koorts, en de koorts verliet haar; en zij van stonde aan opstaande, diende henlieden.
\par 40 En als de zon onderging, brachten allen, die kranken hadden, met verscheidene ziekten bevangen, die tot Hem, en Hij leide een iegelijk van hen de handen op, en genas dezelve.
\par 41 En er voeren ook duivelen uit van velen, roepende en zeggende: Gij zijt de Christus, de Zone Gods! En hen bestraffende, liet Hij die niet spreken, omdat zij wisten, dat Hij de Christus was.
\par 42 En als het dag werd, ging Hij uit, en trok naar een woeste plaats; en de scharen zochten Hem, en kwamen tot bij Hem, en hielden Hem op, dat Hij van hen niet zou weggaan.
\par 43 Maar Hij zeide tot hen: Ik moet ook anderen steden het Evangelie van het Koninkrijk Gods verkondigen; want daartoe ben Ik uitgezonden.
\par 44 En Hij predikte in de synagogen van Galilea.

\chapter{5}

\par 1 En het geschiedde, als de schare op Hem aandrong, om het Woord Gods te horen, dat Hij stond bij het meer Gennesareth.
\par 2 En Hij zag twee schepen aan den oever van het meer liggende, en de vissers waren daaruit gegaan, en spoelden de netten.
\par 3 En Hij ging in een van die schepen, hetwelk van Simon was, en bad hem, dat hij een weinig van het land afstak; en nederzittende, leerde Hij de scharen uit het schip.
\par 4 En als Hij afliet van spreken, zeide Hij tot Simon: Steek af naar de diepte, en werp uw netten uit om te vangen.
\par 5 En Simon antwoordde en zeide tot Hem: Meester, wij hebben den gehelen nacht over gearbeid, en niet gevangen; doch op Uw woord zal ik het net uitwerpen.
\par 6 En als zij dat gedaan hadden, besloten zij een grote menigte vissen, en hun net scheurde.
\par 7 En zij wenkten hun medegenoten, die in het andere schip waren, dat zij hen zouden komen helpen. En zij kwamen, en vulden beide de schepen, zodat zij bijna zonken.
\par 8 En Simon Petrus, dat ziende, viel neder aan de knieen van Jezus, zeggende: Heere! ga uit van mij; want ik ben een zondig mens.
\par 9 Want verbaasdheid had hem bevangen, en allen, die met hem waren, over de vangst der vissen, die zij gevangen hadden;
\par 10 En desgelijks ook Jakobus en Johannes, de zonen van Zebedeus, die medegenoten van Simon waren. En Jezus zeide tot Simon: Vrees niet; van nu aan zult gij mensen vangen.
\par 11 En als zij de schepen aan land gestuurd hadden, verlieten zij alles, en volgden Hem.
\par 12 En het geschiedde, als Hij in een dier steden was, ziet, er was een man vol melaatsheid; en Jezus ziende, viel hij op het aangezicht, en bad Hem, zeggende: Heere! zo Gij wilt, Gij kunt mij reinigen.
\par 13 En Hij, de hand uitstrekkende, raakte hem aan; en zeide: Ik wil, word gereinigd! En terstond ging de melaatsheid van hem.
\par 14 En Hij gebood hem, dat hij het niemand zeggen zou; maar ga heen, zeide Hij, vertoon uzelven den priester, en offer voor uw reiniging, gelijk Mozes geboden heeft, hun tot een getuigenis.
\par 15 Maar het gerucht van Hem ging te meer voort; en vele scharen kwamen samen om Hem te horen, en door Hem genezen te worden van hun krankheden.
\par 16 Maar Hij vertrok in de woestijnen, en bad aldaar.
\par 17 En het geschiedde in een dier dagen, dat Hij leerde, en er zaten Farizeen en leraars der wet, die van alle vlekken van Galilea, en Judea, en Jeruzalem gekomen waren; en de kracht des Heeren was er om hen te genezen.
\par 18 En ziet, enige mannen brachten op een bed een mens, die geraakt was, en zochten hem in te brengen, en voor Hem te leggen.
\par 19 En niet vindende, waardoor zij hem inbrengen mochten, overmits de schare, zo klommen zij op het dak, en lieten hem door de tichelen neder met het beddeken, in het midden, voor Jezus.
\par 20 En Hij ziende hun geloof, zeide tot hem: Mens, uw zonden zijn u vergeven.
\par 21 En de Schriftgeleerden en de Farizeen begonnen te overdenken, zeggende: Wie is Deze, Die gods lastering spreekt? Wie kan de zonden vergeven, dan God alleen?
\par 22 Maar Jezus, hun overdenkingen bekennende, antwoordde en zeide tot hen: Wat overdenkt gij in uw harten?
\par 23 Wat is lichter te zeggen: Uw zonden zijn u vergeven, of te zeggen: Sta op en wandel?
\par 24 Doch opdat gij moogt weten, dat de Zoon des mensen macht heeft op de aarde, de zonde te vergeven (zeide Hij tot den geraakte): Ik zeg u, sta op, en neem uw beddeken op, en ga heen naar uw huis.
\par 25 En hij, terstond voor Hem opstaande, en opgenomen hebbende hetgeen, daar hij op gelegen had, ging heen naar zijn huis, God verheerlijkende.
\par 26 En ontzetting heeft hen allen bevangen, en zij verheerlijkten God, en werden vervuld met vreze, zeggende: Wij hebben heden ongelofelijke dingen gezien.
\par 27 En na dezen ging Hij uit, en zag een tollenaar, met name Levi, zitten in het tolhuis, en zeide tot hem: Volg Mij.
\par 28 En hij, alles verlatende, stond op en volgde Hem.
\par 29 En Levi richtte Hem een groten maaltijd aan, in zijn huis; en er was een grote schare van tollenaren, en van anderen, die met hen aanzaten.
\par 30 En hun Schriftgeleerden en de Farizeen murmureerden tegen Zijn discipelen, zeggende: Waarom eet en drinkt gij met tollenaren en zondaren?
\par 31 En Jezus, antwoordende, zeide tot hen: Die gezond zijn, hebben den medicijnmeester niet van node, maar die ziek zijn.
\par 32 Ik ben niet gekomen om te roepen rechtvaardigen, maar zondaren tot bekering.
\par 33 En zij zeiden tot Hem: Waarom vasten de discipelen van Johannes dikmaals, en doen gebeden, desgelijks ook de discipelen der Farizeen, maar de Uwe eten en drinken?
\par 34 Doch Hij zeide tot hen: Kunt gij de bruiloftskinderen, terwijl de Bruidegom bij hen is, doen vasten?
\par 35 Maar de dagen zullen komen, wanneer de Bruidegom van hen zal weggenomen zijn, dan zullen zij vasten in die dagen.
\par 36 En Hij zeide ook tot hen een gelijkenis: Niemand zet een lap van een nieuw kleed op een oud kleed; anders zo scheurt ook dat nieuwe het oude, en de lap van het nieuwe komt met het oude niet overeen.
\par 37 En niemand doet nieuwen wijn in oude leder zakken; anders zo zal de nieuwe wijn de leder zakken doen bersten, en de wijn zal uitgestort worden, en de leder zakken zullen verderven.
\par 38 Maar nieuwen wijn moet men in nieuwe leder zakken doen, en zij worden beide te zamen behouden.
\par 39 En niemand, die ouden drinkt, begeert terstond nieuwen; want hij zegt: De oude is beter.

\chapter{6}

\par 1 En het geschiedde op den tweeden eersten sabbat, dat Hij door het gezaaide ging; en Zijn discipelen plukten aren, en aten ze, die wrijvende met de handen.
\par 2 En sommigen der Farizeen zeiden tot hen: Waarom doet gij, wat niet geoorloofd is te doen op de sabbatten?
\par 3 En Jezus, hun antwoordende, zeide: Hebt gij ook dat niet gelezen, hetwelk David deed, wanneer hem hongerde, en dengenen, die met hem waren?
\par 4 Hoe hij ingegaan is in het huis Gods, en de toonbroden genomen en gegeten heeft, en ook gegeven dengenen, die met hem waren, welke niet zijn geoorloofd te eten, dan alleen den priesteren.
\par 5 En Hij zeide tot hen: De Zoon des mensen is een Heere ook van den sabbat.
\par 6 En het geschiedde ook op een anderen sabbat, dat Hij in de synagoge ging, en leerde. En daar was een mens, en zijn rechterhand was dor.
\par 7 En de Schriftgeleerden en de Farizeen namen Hem waar, of Hij op den sabbat genezen zou; opdat zij enige beschuldiging tegen Hem mochten vinden.
\par 8 Doch Hij kende hun gedachten, en zeide tot den mens, die de dorre hand had: Rijs op, en sta in het midden. En hij opgestaan zijnde, stond over einde.
\par 9 Zo zeide dan Jezus tot hen: Ik zal u vragen: Wat is geoorloofd op de sabbatten, goed te doen, of kwaad te doen, een mens te behouden, of te verderven?
\par 10 En hen allen rondom aangezien hebbende, zeide Hij tot den mens: Strek uw hand uit. En hij deed alzo; en zijn hand werd hersteld, gezond gelijk de andere.
\par 11 En zij werden vervuld met uitzinnigheid, en spraken samen met elkander, wat zij Jezus doen zouden.
\par 12 En het geschiedde in die dagen, dat Hij uitging naar den berg, om te bidden, en Hij bleef den nacht over in het gebed tot God.
\par 13 En als het dag was geworden, riep Hij Zijn discipelen tot Zich, en verkoos er twaalf uit hen, die Hij ook apostelen noemde:
\par 14 Namelijk Simon, welken Hij ook Petrus noemde; en Andreas zijn broeder, Jakobus en Johannes, Filippus en Bartholomeus;
\par 15 Mattheus en Thomas, Jakobus, den zoon van Alfeus, en Simon genaamd Zelotes;
\par 16 Judas Jakobi, en Judas Iskariot, die ook de verrader geworden is.
\par 17 En met hen afgekomen zijnde, stond Hij op een vlakke plaats, en met Hem de schare Zijner discipelen, en een grote menigte des volks van geheel Judea en Jeruzalem, en van den zeekant van Tyrus en Sidon;
\par 18 Die gekomen waren, om Hem te horen, en om van hun ziekten genezen te worden, en die van onreine geesten gekweld waren; en zij werden genezen.
\par 19 En al de schare zocht Hem aan te raken; want er ging kracht van Hem uit, en Hij genas ze allen.
\par 20 En Hij, Zijn ogen opslaande over Zijn discipelen, zeide: Zalig zijt gij, armen, want uwer is het Koninkrijk Gods.
\par 21 Zalig zijt gij, die nu hongert; want gij zult verzadigd worden. Zalig zijt gij, die nu weent; want gij zult lachen.
\par 22 Zalig zijt gij, wanneer u de mensen haten, en wanneer zij u afscheiden, en smaden, en uw naam als kwaad verwerpen, om des Zoons des mensen wil.
\par 23 Verblijdt u in dien dag, en zijt vrolijk; want, ziet, uw loon is groot in den hemel; want hun vaders deden desgelijks den profeten.
\par 24 Maar wee u, gij rijken, want gij hebt uw troost weg.
\par 25 Wee u, die verzadigd zijt, want gij zult hongeren. Wee u, die nu lacht, want gij zult treuren en wenen.
\par 26 Wee u, wanneer al de mensen wel van u spreken, want hun vaders deden desgelijks den valsen profeten.
\par 27 Maar Ik zeg ulieden, die dit hoort: Hebt uw vijanden lief; doet wel dengenen, die u haten.
\par 28 Zegent degenen, die u vervloeken, en bidt voor degenen, die u geweld doen.
\par 29 Dengene, die u aan de wang slaat, biedt ook de andere; en dengene, die u den mantel neemt, verhindert ook den rok niet te nemen.
\par 30 Maar geeft een iegelijk, die van u begeert; en van dengene, die het uwe neemt, eist niet weder.
\par 31 En gelijk gij wilt, dat u de mensen doen zullen, doet gij hun ook desgelijks.
\par 32 En indien gij liefhebt, die u liefhebben, wat dank hebt gij? Want ook de zondaars hebben lief degenen, die hen liefhebben.
\par 33 En indien gij goed doet dengenen, die u goed doen, wat dank hebt gij? Want ook de zondaars doen hetzelfde.
\par 34 En indien gij leent dengenen, van welke gij hoopt weder te ontvangen, wat dank hebt gij? Want ook de zondaars lenen den zondaren, opdat zij evengelijk weder mogen ontvangen.
\par 35 Maar hebt uw vijanden lief, en doet goed, en leent, zonder iets weder te hopen; en uw loon zal groot zijn, en gij zult kinderen des Allerhoogsten zijn; want Hij is goedertieren over de ondankbaren en bozen.
\par 36 Weest dan barmhartig, gelijk ook uw Vader barmhartig is.
\par 37 En oordeelt niet, en gij zult niet geoordeeld worden; verdoemt niet, en gij zult niet verdoemd worden; laat los, en gij zult losgelaten worden.
\par 38 Geeft, en u zal gegeven worden; een goede, neergedrukte, en geschudde en overlopende maat zal men in uw schoot geven; want met dezelfde maat, waarmede gijlieden meet, zal ulieden wedergemeten worden.
\par 39 En Hij zeide tot hen een gelijkenis: Kan ook wel een blinde een blinde op den weg leiden? Zullen zij niet beiden in de gracht vallen?
\par 40 De discipel is niet boven zijn meester; maar een iegelijk volmaakt discipel zal zijn gelijk zijn meester.
\par 41 En wat ziet gij den splinter, die in uws broeders oog is, en den balk, die in uw eigen oog is, merkt gij niet?
\par 42 Of hoe kunt gij tot uw broeder zeggen: Broeder, laat toe, dat ik den splinter, die in uw oog is, uitdoe; daar gij zelf den balk, die in uw oog is, niet ziet? Gij geveinsde! doe eerst den balk uit uw oog, en dan zult gij bezien, om den splinter uit te doen, die in uws broeders oog is.
\par 43 Want het is geen goede boom, die kwade vrucht voortbrengt, en geen kwade boom, die goede vrucht voortbrengt;
\par 44 Want ieder boom wordt uit zijn eigen vrucht gekend; want men leest geen vijgen van doornen, en men snijdt geen druif van bramen.
\par 45 De goede mens brengt het goede voort uit den goeden schat zijns harten; en de kwade mens brengt het kwade voort uit den kwaden schat zijns harten; want uit den overvloed des harten spreekt zijn mond.
\par 46 En wat noemt gij Mij, Heere, Heere! en doet niet hetgeen Ik zeg?
\par 47 Een iegelijk, die tot Mij komt, en Mijn woorden hoort, en dezelve doet, Ik zal u tonen, wien hij gelijk is.
\par 48 Hij is gelijk een mens, die een huis bouwde, en groef, en verdiepte, en leide het fondament op een steenrots; als nu de hoge vloed kwam, zo sloeg de waterstroom tegen dat huis aan, en kon het niet bewegen; want het was op de steenrots gegrond.
\par 49 Maar die ze gehoord, en niet gedaan zal hebben, is gelijk een mens, die een huis bouwde op de aarde zonder fondament; tegen hetwelk de waterstroom aansloeg, en het viel terstond, en de val van datzelve huis was groot.

\chapter{7}

\par 1 Nadat Hij nu al Zijn woorden voleindigd had, ten aanhore des volks, ging Hij in te Kapernaum.
\par 2 En een dienstknecht van een zeker hoofdman over honderd, die hem zeer waard was, krank zijnde, lag op zijn sterven.
\par 3 En van Jezus gehoord hebbende, zond hij tot Hem de ouderlingen der Joden, Hem biddende, dat Hij wilde komen, en zijn dienstknecht gezond maken.
\par 4 Dezen nu, tot Jezus gekomen zijnde, baden Hem ernstelijk, zeggende: Hij is waardig, dat Gij hem dat doet;
\par 5 Want hij heeft ons volk lief, en heeft zelf ons de synagoge gebouwd.
\par 6 En Jezus ging met hen. En als Hij nu niet verre van het huis was, zond de hoofdman over honderd tot Hem enige vrienden, en zeide tot Hem: Heere, neem de moeite niet; want ik ben niet waardig, dat Gij onder mijn dak zoudt inkomen.
\par 7 Daarom heb ik ook mijzelven niet waardig geacht, om tot U te komen; maar zeg het met een woord, en mijn knecht zal genezen worden.
\par 8 Want ik ben ook een mens, onder de macht van anderen gesteld, hebbende krijgsknechten onder mij, en ik zeg tot dezen: Ga, en hij gaat; en tot den anderen: Kom! en hij komt; en tot mijn dienstknecht: Doe dat! en hij doet het.
\par 9 En Jezus, dit horende, verwonderde Zich over hem; en Zich omkerende, zeide tot de schare, die Hem volgde: Ik zeg ulieden: Ik heb zo groot een geloof zelfs in Israel niet gevonden.
\par 10 En die gezonden waren, wedergekeerd zijnde in het huis, vonden den kranken dienstknecht gezond.
\par 11 En het geschiedde op den volgenden dag, dat Hij ging naar een stad, genaamd Nain, en met Hem gingen velen van Zijn discipelen, en een grote schare.
\par 12 En als Hij de poort der stad genaakte, zie daar, een dode werd uitgedragen, die een eniggeboren zoon zijner moeder was, en zij was weduwe en een grote schare van de stad was met haar.
\par 13 En de Heere, haar ziende, werd innerlijk met ontferming over haar bewogen, en zeide tot haar: Ween niet.
\par 14 En Hij ging toe, en raakte de baar aan; (de dragers nu stonden stil) en Hij zeide: Jongeling, Ik zeg u, sta op!
\par 15 En de dode zat overeind, en begon te spreken. En Hij gaf hem aan zijn moeder.
\par 16 En vreze beving hen allen, en zij verheerlijkten God, zeggende: Een groot Profeet is onder ons opgestaan, en God heeft Zijn volk bezocht.
\par 17 En dit gerucht van Hem ging uit in geheel Judea, en in al het omliggende land.
\par 18 En de discipelen van Johannes boodschapten hem van al deze dingen.
\par 19 En Johannes, zekere twee van zijn discipelen tot zich geroepen hebbende, zond hen tot Jezus, zeggende: Zijt Gij Degene, Die komen zou, of verwachten wij een anderen?
\par 20 En als de mannen tot Hem gekomen waren, zeiden zij: Johannes de Doper heeft ons tot U afgezonden, zeggende: Zijt Gij, Die komen zou, of verwachten wij een anderen?
\par 21 En in dezelfde ure genas Hij er velen van ziekten en kwalen, en boze geesten; en velen blinden gaf Hij het gezicht.
\par 22 En Jezus, antwoordende, zeide tot hen: Gaat heen, en boodschapt Johannes weder de dingen, die gij gezien en gehoord hebt, namelijk dat de blinden ziende worden, de kreupelen wandelen, de melaatsen gereinigd worden, de doven horen, de doden opgewekt worden, den armen het Evangelie verkondigd wordt.
\par 23 En zalig is hij, die aan Mij niet zal geergerd worden.
\par 24 Als nu de boden van Johannes weggegaan waren, begon Hij tot de scharen van Johannes te zeggen: Wat zijt gij uitgegaan in de woestijn te aanschouwen? Een riet, dat van den wind ginds en weder bewogen wordt?
\par 25 Maar wat zijt gij uitgegaan te zien? Een mens, met zachte klederen bekleed? Ziet, die in heerlijke kleding en wellust zijn, die zijn in de koninklijke hoven.
\par 26 Maar wat zijt gij uitgegaan te zien? Een profeet? Ja, Ik zeg u, ook veel meer dan een profeet.
\par 27 Deze is het, van welken geschreven is: Ziet, Ik zende Mijn engel voor uw aangezicht, die Uw weg voor U heen bereiden zal.
\par 28 Want Ik zeg ulieden: Onder die van vrouwen geboren zijn, is niemand meerder profeet, dan Johannes de Doper; maar de minste in het Koninkrijk Gods is meerder dan hij.
\par 29 En al het volk, Hem horende, en de tollenaars, die met den doop van Johannes gedoopt waren, rechtvaardigden God.
\par 30 Maar de Farizeen en de wetgeleerden hebben den raad Gods tegen zichzelven verworpen, van hem niet gedoopt zijnde.
\par 31 En de Heere zeide: Bij wien zal Ik dan de mensen van dit geslacht vergelijken, en wien zijn zij gelijk?
\par 32 Zij zijn gelijk aan de kinderen, die op de markt zitten, en elkander toeroepen, en zeggen: Wij hebben u op de fluit gespeeld, en gij hebt niet gedanst; wij hebben u klaagliederen gezongen, en gij hebt niet geweend.
\par 33 Want Johannes de Doper is gekomen, noch brood etende, noch wijn drinkende; en gij zegt: Hij heeft den duivel.
\par 34 De Zoon des mensen is gekomen, etende en drinkende, en gij zegt: Ziet daar, een Mens, Die een vraat en wijnzuiper is, een Vriend van tollenaren en zondaren.
\par 35 Doch de wijsheid is gerechtvaardigd geworden van al haar kinderen.
\par 36 En een der Farizeen bad Hem, dat Hij met hem ate; en ingegaan zijnde in des Farizeers huis, zat Hij aan.
\par 37 En ziet, een vrouw in de stad, welke een zondares was, verstaande, dat Hij in des Farizeers huis aanzat, bracht een albasten fles met zalf.
\par 38 En staande achter Zijn voeten, wenende, begon zij Zijn voeten nat te maken met tranen, en zij droogde ze af met het haar van haar hoofd, en kuste Zijn voeten, en zalfde ze met de zalf.
\par 39 En de Farizeer, die Hem genood had, zulks ziende, sprak bij zichzelven, zeggende: Deze, indien Hij een profeet ware, zou wel weten, wat en hoedanige vrouw deze is, die Hem aanraakt; want zij is een zondares.
\par 40 En Jezus antwoordende, zeide tot hem: Simon! Ik heb u wat te zeggen. En hij sprak: Meester! zeg het.
\par 41 Jezus zeide: Een zeker schuldheer had twee schuldenaars; de een was schuldig vijfhonderd penningen, en de andere vijftig;
\par 42 En als zij niet hadden om te betalen, schold hij het hun beiden kwijt. Zeg dan, wie van dezen zal hem meer liefhebben?
\par 43 En Simon, antwoordende, zeide: Ik acht, dat hij het is, dien hij het meeste kwijtgescholden heeft. En Hij zeide tot hem: Gij hebt recht geoordeeld.
\par 44 En Hij, Zich omkerende naar de vrouw, zeide tot Simon: Ziet gij deze vrouw? Ik ben in uw huis gekomen; water hebt gij niet tot Mijn voeten gegeven; maar deze heeft Mijn voeten met tranen nat gemaakt, en met het haar van haar hoofd afgedroogd.
\par 45 Gij hebt Mij geen kus gegeven; maar deze, van dat zij ingekomen is, heeft niet afgelaten Mijn voeten te kussen.
\par 46 Met olie hebt gij Mijn hoofd niet gezalfd; maar deze heeft Mijn voeten met zalf gezalfd.
\par 47 Daarom zeg Ik u: Haar zonden zijn haar vergeven, die vele waren; want zij heeft veel liefgehad; maar dien weinig vergeven wordt, die heeft weinig lief.
\par 48 En Hij zeide tot haar: Uw zonden zijn u vergeven.
\par 49 En die mede aanzaten, begonnen te zeggen bij zichzelven: Wie is Deze, Die ook de zonden vergeeft?
\par 50 Maar Hij zeide tot de vrouw: Uw geloof heeft u behouden; ga heen in vrede.

\chapter{8}

\par 1 En het geschiedde daarna, dat Hij reisde van de ene stad en vlek tot de andere, predikende en verkondigende het Evangelie van het Koninkrijk Gods; en de twaalven waren met Hem;
\par 2 En sommige vrouwen, die van boze geesten en krankheden genezen waren, namelijk Maria, genaamd Magdalena, van welke zeven duivelen uitgegaan waren;
\par 3 En Johanna, de huisvrouw van Chusas, den rentmeester van Herodes, en Susanna, en vele anderen, die Hem dienden van haar goederen.
\par 4 Als nu een grote schare bijeenvergaderde, en zij van alle steden tot Hem kwamen, zo zeide Hij door gelijkenis:
\par 5 Een zaaier ging uit, om zijn zaad te zaaien; en als hij zaaide, viel het ene bij den weg, en werd vertreden, en de vogelen des hemels aten dat op.
\par 6 En het andere viel op een steenrots, en opgewassen zijnde, is het verdord, omdat het geen vochtigheid had.
\par 7 En het andere viel in het midden van de doornen, en de doornen mede opwassende, verstikten hetzelve.
\par 8 En het andere viel op de goede aarde, en opgewassen zijnde, bracht het honderdvoudige vrucht voort. Dit zeggende, riep Hij: Wie oren heeft, om te horen, die hore.
\par 9 En Zijn discipelen vraagden Hem, zeggende: Wat mag deze gelijkenis wezen?
\par 10 En Hij zeide: U is het gegeven, de verborgenheden van het Koninkrijk Gods te verstaan; maar tot de anderen spreek Ik in gelijkenissen, opdat zij ziende niet zien, en horende niet verstaan.
\par 11 Dit is nu de gelijkenis: Het zaad is het Woord Gods.
\par 12 En die bij den weg bezaaid worden, zijn dezen, die horen; daarna komt de duivel, en neemt het Woord uit hun hart weg, opdat zij niet zouden geloven, en zalig worden.
\par 13 En die op de steenrots bezaaid worden, zijn dezen, die, wanneer zij het gehoord hebben, het Woord met vreugde ontvangen; en dezen hebben geen wortel, die maar voor een tijd geloven, en in den tijd der verzoeking wijken zij af.
\par 14 En dat in de doornen valt, zijn dezen, die gehoord hebben, en heengaande verstikt worden door de zorgvuldigheden, en rijkdom, en wellusten des levens, en voldragen geen vrucht.
\par 15 En dat in de goede aarde valt, zijn dezen, die, het Woord gehoord hebbende, hetzelve in een eerlijk en goed hart bewaren, en in volstandigheid vruchten voortbrengen.
\par 16 En niemand, die een kaars ontsteekt, bedekt dezelve met een vat, of zet ze onder een bed; maar zet ze op een kandelaar, opdat degenen, die inkomen, het licht zien mogen.
\par 17 Want er is niets verborgen, dat niet openbaar zal worden; noch heimelijk, dat niet bekend zal worden, en in het openbaar komen.
\par 18 Ziet dan, hoe gij hoort; want zo wie heeft, dien zal gegeven worden; en zo wie niet heeft, ook hetgeen hij meent te hebben, zal van hem genomen worden.
\par 19 En Zijn moeder en Zijn broeders kwamen tot Hem, en konden bij Hem niet komen, vanwege de schare.
\par 20 En Hem werd geboodschapt van enigen, die zeiden: Uw moeder en Uw broeders staan daar buiten, begerende U te zien.
\par 21 Maar Hij antwoordde en zeide tot hen: Mijn moeder en Mijn broeders zijn dezen, die Gods Woord horen, en datzelve doen.
\par 22 En het geschiedde in een van die dagen, dat Hij in een schip ging, en Zijn discipelen met Hem; en Hij zeide tot hen: Laat ons overvaren aan de andere zijde van het meer. En zij staken af.
\par 23 En als zij voeren, viel Hij in slaap; en er kwam een storm van wind op het meer, en zij werden vol waters, en waren in nood.
\par 24 En zij gingen tot Hem, en wekten Hem op, zeggende: Meester, Meester, wij vergaan! en Hij, opgestaan zijnde, bestrafte den wind en de watergolven, en zij hielden op, en er werd stilte.
\par 25 En Hij zeide tot hen: Waar is uw geloof? Maar zij, bevreesd zijnde, verwonderden zich, zeggende tot elkander: Wie is toch Deze, dat Hij ook de winden en het water gebiedt, en zij zijn Hem gehoorzaam?
\par 26 En zij voeren voort naar het land der Gadarenen, hetwelk is tegenover Galilea.
\par 27 En als Hij aan het land uitgegaan was, ontmoette Hem een zeker man uit de stad, die van over langen tijd met duivelen was bezeten geweest; en was met geen klederen gekleed, en bleef in geen huis, maar in de graven.
\par 28 En hij, Jezus ziende, en zeer roepende, viel voor Hem neder, en zeide met een grote stem: Wat heb ik met U te doen, Jezus, Gij Zone Gods, des Allerhoogsten, ik bid U, dat Gij mij niet pijnigt!
\par 29 Want Hij had den onreinen geest geboden, dat hij van den mens zou uitvaren; want hij had hem menigen tijd bevangen gehad; en hij werd met ketenen en met boeien gebonden, om bewaard te zijn; en hij verbrak de banden, en werd van den duivel gedreven in de woestijnen.
\par 30 En Jezus vraagde hem, zeggende: Welke is uw naam? En hij zeide: Legio. Want vele duivelen waren in hem gevaren.
\par 31 En zij baden Hem, dat Hij hun niet gebieden zou in den afgrond heen te varen.
\par 32 En aldaar was een kudde veler zwijnen, weidende op den berg; en zij baden Hem, dat Hij hun wilde toelaten in dezelve te varen. En Hij liet het hun toe.
\par 33 En de duivelen, uitvarende van den mens, voeren in de zwijnen; en de kudde stortte van de steilte af in het meer; en versmoorde.
\par 34 En die ze weidden, ziende hetgeen geschied was, zijn gevlucht; en heengaande boodschapten het in de stad, en op het land.
\par 35 En zij gingen uit, om te zien hetgeen geschied was, en kwamen tot Jezus, en vonden den mens, van welken de duivelen uitgevaren waren, zittend aan de voeten van Jezus, gekleed en wel bij zijn verstand; en zij werden bevreesd.
\par 36 En ook, die het gezien hadden, verhaalden hun, hoe de bezetene was verlost geworden.
\par 37 En de gehele menigte van het omliggende land der Gadarenen baden Hem, dat Hij van hen wegging; want zij waren met grote vreze bevangen. En Hij, in het schip gegaan zijnde, keerde wederom.
\par 38 En de man, van welken de duivelen uitgevaren waren, bad Hem, dat hij mocht bij Hem zijn. Maar Jezus liet hem van Zich gaan, zeggende:
\par 39 Keer weder naar uw huis, en vertel, wat grote dingen u God gedaan heeft. En hij ging heen door de gehele stad, verkondigende, wat grote dingen Jezus hem gedaan had.
\par 40 En het geschiedde, als Jezus wederkeerde, dat Hem de schare ontving; want zij waren allen Hem verwachtende.
\par 41 En ziet, er kwam een man, wiens naam was Jairus, en hij was een overste der synagoge; en hij viel aan de voeten van Jezus, en bad Hem, dat Hij in zijn huis wilde komen.
\par 42 Want hij had een enige dochter, van omtrent twaalf jaren, en deze lag op haar sterven. En als Hij heenging, zo verdrongen Hem de scharen.
\par 43 En een vrouw, die twaalf jaren lang den vloed des bloeds gehad had, welke al haar leeftocht aan medicijnmeesters ten koste gelegd had; en van niemand had kunnen genezen worden,
\par 44 Van achteren tot Hem komende, raakte den zoom Zijns kleeds aan; en terstond stelpte de vloed haars bloeds.
\par 45 En Jezus zeide: Wie is het, die Mij heeft aangeraakt? En als zij het allen miszaakten, zeide Petrus en die met hem waren: Meester, de scharen drukken en verdringen U, en zegt Gij: Wie is het, die Mij aangeraakt heeft?
\par 46 En Jezus zeide: Iemand heeft Mij aangeraakt; want Ik heb bekend, dat kracht van Mij uitgegaan is.
\par 47 De vrouw nu, ziende, dat zij niet verborgen was, kwam bevende, en voor Hem nedervallende, verklaarde Hem voor al het volk, om wat oorzaak zij Hem aangeraakt had, en hoe zij terstond genezen was.
\par 48 En Hij zeide tot haar: Dochter, wees welgemoed, uw geloof heeft u behouden; ga heen in vrede.
\par 49 Als Hij nog sprak, kwam er een van het huis des oversten der synagoge, zeggende tot hem: Uw dochter is gestorven; zijt den Meester niet moeielijk.
\par 50 Maar Jezus, dat horende, antwoordde hem, zeggende: Vrees niet, geloof alleenlijk, en zij zal behouden worden.
\par 51 En als Hij in het huis kwam, liet Hij niemand inkomen, dan Petrus, en Jakobus, en Johannes, en den vader en de moeder des kinds.
\par 52 En zij schreiden allen, en maakten misbaar over hetzelve. En Hij zeide: Schreit niet; zij is niet gestorven; maar zij slaapt.
\par 53 En zij belachten Hem, wetende, dat zij gestorven was.
\par 54 Maar als Hij ze allen uitgedreven had, greep Hij haar hand en riep, zeggende: Kind, sta op!
\par 55 En haar geest keerde weder, en zij is terstond opgestaan; en Hij gebood, dat men haar te eten geven zoude.
\par 56 En haar ouders ontzetten zich; en Hij beval hun, dat zij niemand zouden zeggen hetgeen geschied was.

\chapter{9}

\par 1 En Zijn twaalf discipelen samengeroepen hebbende, gaf Hij hun kracht en macht over al de duivelen, en om ziekten te genezen.
\par 2 En Hij zond hen heen, om te prediken het Koninkrijk Gods, en de kranken gezond te maken.
\par 3 En Hij zeide tot hen: Neemt niets mede tot den weg, noch staven, noch male, noch brood, noch geld; noch iemand van u zal twee rokken hebben.
\par 4 En in wat huis gij ook zult ingaan, blijft aldaar, en gaat van daar uit.
\par 5 En zo wie u niet zullen ontvangen, uitgaande van die stad, schudt ook het stof af van uw voeten, tot een getuigenis tegen hen.
\par 6 En zij, uitgaande, doorgingen al de vlekken, verkondigende het Evangelie, en genezende de zieken overal.
\par 7 En Herodes, de viervorst, hoorde al de dingen, die van Hem geschiedden; en was twijfelmoedig, omdat van sommigen gezegd werd, dat Johannes van de doden was opgestaan.
\par 8 En van sommigen, dat Elias verschenen was; en van anderen, dat een profeet van de ouden was opgestaan.
\par 9 En Herodes zeide: Johannes heb ik onthoofd; wie is nu Deze, van Welken ik zulke dingen hoor? En hij zocht Hem te zien.
\par 10 En de apostelen, wedergekeerd zijnde, verhaalden Hem al wat zij gedaan hadden. En Hij nam hen mede en vertrok alleen in een woeste plaats der stad, genaamd Bethsaida.
\par 11 En de scharen, dat verstaande, volgden Hem; en Hij ontving ze, en sprak tot hen van het Koninkrijk Gods; en die genezing van node hadden, maakte Hij gezond.
\par 12 En de dag begon te dalen; en de twaalven, tot Hem komende, zeiden tot Hem: Laat de schare van U, opdat zij, heengaande in de omliggende vlekken en in de dorpen, herberg nemen mogen, en spijze vinden; want wij zijn hier in een woeste plaats.
\par 13 Maar Hij zeide tot hen: Geeft gij hun te eten. En zij zeiden: Wij hebben niet meer dan vijf broden, en twee vissen; tenzij dan dat wij heengaan en spijs kopen voor al dit volk;
\par 14 Want er waren omtrent vijf duizend mannen. Doch Hij zeide tot Zijn discipelen: Doet hen nederzitten bij zaten, elk van vijftig.
\par 15 En zij deden alzo, en deden hen allen nederzitten.
\par 16 En Hij, de vijf broden en de twee vissen genomen hebbende, zag op naar den hemel, en zegende die, en brak ze, en gaf ze den discipelen, om der schare voor te leggen.
\par 17 En zij aten en werden allen verzadigd; en er werd opgenomen, hetgeen hun van de brokken overgeschoten was, twaalf korven.
\par 18 En het geschiedde, als Hij alleen was biddende, dat de discipelen met Hem waren, en Hij vraagde hen, zeggende: Wie zeggen de scharen, dat Ik ben?
\par 19 En zij, antwoordende, zeiden: Johannes de Doper; en anderen: Elias; en anderen: Dat enig profeet van de ouden opgestaan is.
\par 20 En Hij zeide tot hen: Maar gijlieden, wie zegt gij, dat Ik ben? En Petrus, antwoordende, zeide: De Christus Gods.
\par 21 En Hij gebood hun scherpelijk en beval, dat zij dit niemand zeggen zouden;
\par 22 Zeggende: De Zoon des mensen moet veel lijden, en verworpen worden van de ouderlingen, en overpriesters, en Schriftgeleerden, en gedood en ten derden dage opgewekt worden.
\par 23 En Hij zeide tot allen: Zo iemand achter Mij wil komen, die verloochene zichzelven, en neme zijn kruis dagelijks op, en volge Mij.
\par 24 Want zo wie zijn leven behouden wil, die zal het verliezen; maar zo wie zijn leven verliezen zal, om Mijnentwil, die zal het behouden.
\par 25 Want wat baat het een mens, die de gehele wereld zou winnen, en zichzelven verliezen, of schade zijns zelfs lijden?
\par 26 Want zo wie zich Mijns en Mijner woorden zal geschaamd hebben, diens zal de Zoon des mensen Zich schamen, wanneer Hij komen zal in Zijn heerlijkheid, en in de heerlijkheid des Vaders, en der heilige engelen.
\par 27 En Ik zeg u waarlijk: Er zijn sommigen dergenen, die hier staan, die den dood niet zullen smaken, totdat zij het Koninkrijk Gods zullen gezien hebben.
\par 28 En het geschiedde, omtrent acht dagen na deze woorden, dat Hij medenam Petrus, en Johannes, en Jakobus, en klom op den berg, om te bidden.
\par 29 En als Hij bad, werd de gedaante Zijns aangezichts veranderd, en Zijn kleding wit en zeer blinkende.
\par 30 En ziet, twee mannen spraken met Hem, welke waren Mozes en Elias.
\par 31 Dewelke, gezien zijnde in heerlijkheid, zeiden Zijn uitgang, dien Hij zoude volbrengen te Jeruzalem.
\par 32 Petrus nu, en die met hem waren, waren met slaap bezwaard; en ontwaakt zijnde, zagen zij Zijn heerlijkheid, en de twee mannen, die bij Hem stonden.
\par 33 En het geschiedde, als zij van Hem afscheidden, zo zeide Petrus tot Jezus: Meester, het is goed, dat wij hier zijn; en laat ons drie tabernakelen maken, voor U een, en voor Mozes een, en voor Elias een; niet wetende, wat hij zeide.
\par 34 Als hij nu dit zeide, kwam een wolk, en overschaduwde hen; en zij werden bevreesd, als die in de wolk ingingen.
\par 35 En er geschiedde een stem uit de wolk, zeggende: Deze is Mijn geliefde Zoon; hoort Hem!
\par 36 En als de stem geschiedde, zo werd Jezus alleen gevonden. En zij zwegen stil, en verhaalden in die dagen niemand iets van hetgeen zij gezien hadden.
\par 37 En het geschiedde des daags daaraan, als zij van den berg afkwamen, dat Hem een grote schare in het gemoet kwam.
\par 38 En ziet, een man van de schare riep uit, zeggende: Meester, ik bid U, zie toch mijn zoon aan; want hij is mij een eniggeborene.
\par 39 En zie, een geest neemt hem, en van stonde aan roept hij, en hij scheurt hem, dat hij schuimt, en wijkt nauwelijks van hem, en verplettert hem.
\par 40 En ik heb Uw discipelen gebeden, dat zij hem zouden uitwerpen, en zij hebben niet gekund.
\par 41 En Jezus, antwoordende, zeide: O ongelovig en verkeerd geslacht, hoe lang zal Ik nog bij ulieden zijn, en ulieden verdragen? Breng uw zoon hier.
\par 42 En nog, als hij naar Hem toekwam, scheurde hem de duivel, en verscheurde hem; maar Jezus bestrafte den onreinen geest, en maakte het kind gezond, en gaf hem zijn vader weder.
\par 43 En zij werden allen verslagen over de grootdadigheid Gods. En als zij allen zich verwonderden over al de dingen, die Jezus gedaan had, zeide Hij tot Zijn discipelen:
\par 44 Legt gij deze woorden in uw oren: Want de Zoon des mensen zal overgeleverd worden in der mensen handen.
\par 45 Maar zij verstonden dit woord niet, en het was voor hen verborgen, alzo dat zij het niet begrepen; en zij vreesden van dat woord Hem te vragen.
\par 46 En er rees een overlegging onder hen, namelijk, wie van hen de meeste ware.
\par 47 Maar Jezus, ziende de overlegging hunner harten, nam een kindeken, en stelde dat bij Zich;
\par 48 En zeide tot hen: Zo wie dit kindeken ontvangen zal in Mijn Naam, die ontvangt Mij; en zo wie Mij ontvangen zal, ontvangt Hem, Die Mij gezonden heeft. Want die de minste onder u allen is, die zal groot zijn.
\par 49 En Johannes antwoordde en zeide: Meester! wij hebben een gezien, die in Uw Naam de duivelen uitwierp, en wij hebben het hem verboden, omdat hij U met ons niet volgt.
\par 50 En Jezus zeide tot hem: Verbied het niet; want wie tegen ons niet is, die is voor ons.
\par 51 En het geschiedde, als de dagen Zijner opneming vervuld werden, zo richtte Hij Zijn aangezicht, om naar Jeruzalem te reizen.
\par 52 En Hij zond boden uit voor Zijn aangezicht; en zij, heengereisd zijnde, kwamen in een vlek der Samaritanen, om voor Hem herberg te bereiden.
\par 53 En zij ontvingen Hem niet, omdat Zijn aangezicht was als reizende naar Jeruzalem.
\par 54 Als nu Zijn discipelen, Jakobus en Johannes, dat zagen, zeiden zij: Heere, wilt Gij, dat wij zeggen, dat vuur van den hemel nederdale, en dezen verslinde, gelijk ook Elias gedaan heeft?
\par 55 Maar Zich omkerende, bestrafte Hij hen, en zeide: Gij weet niet van hoedanigen geest gij zijt.
\par 56 Want de Zoon des mensen is niet gekomen om der mensen zielen te verderven, maar om te behouden. En zij gingen naar een ander vlek.
\par 57 En het geschiedde op den weg, als zij reisden, dat een tot Hem zeide: Heere, ik zal U volgen, waar Gij ook heengaat.
\par 58 En Jezus zeide tot hem: De vossen hebben holen, en de vogelen des hemels nesten; maar de Zoon des mensen heeft niet, waar Hij het hoofd nederlegge.
\par 59 En Hij zeide tot een anderen: Volg Mij. Doch hij zeide: Heere, laat mij toe, dat ik heenga, en eerst mijn vader begrave.
\par 60 Maar Jezus zeide tot hem: Laat de doden hun doden begraven; doch gij, ga heen en verkondig het Koninkrijk Gods.
\par 61 En ook een ander zeide: Heere, ik zal U volgen; maar laat mij eerst toe, dat ik afscheid neme van degenen, die in mijn huis zijn.
\par 62 En Jezus zeide tot hem: Niemand, die zijn hand aan den ploeg slaat, en ziet naar hetgeen achter is, is bekwaam tot het Koninkrijk Gods.

\chapter{10}

\par 1 En na dezen stelde de Heere nog andere zeventig, en zond hen heen voor Zijn aangezicht, twee en twee, in iedere stad en plaats, daar Hij komen zou.
\par 2 Hij zeide dan tot hen: De oogst is wel groot, maar de arbeiders zijn weinige; daarom, bidt den Heere des oogstes, dat Hij arbeiders in Zijn oogst uitstote.
\par 3 Gaat henen; ziet, Ik zend u als lammeren in het midden der wolven.
\par 4 Draagt geen buidel, noch male, noch schoenen; en groet niemand op den weg.
\par 5 En in wat huis gij zult ingaan, zegt eerst: Vrede zij dezen huize!
\par 6 En indien aldaar een zoon des vredes is, zo zal uw vrede op hem rusten; maar indien niet, zo zal uw vrede tot u wederkeren.
\par 7 En blijft in datzelve huis, etende en drinkende, hetgeen van hen voorgezet wordt; want de arbeider is zijn loon waardig; gaat niet over van het ene huis in het andere huis.
\par 8 En in wat stad gij zult ingaan, en zij u ontvangen, eet hetgeen ulieden voorgezet wordt.
\par 9 En geneest de kranken, die daarin zijn, en zegt tot hen: Het Koninkrijk Gods is nabij u gekomen.
\par 10 Maar in wat stad gij zult ingaan, en zij u niet ontvangen, uitgaande op haar straten, zo zegt:
\par 11 Ook het stof, dat uit uw stad aan ons kleeft, schudden wij af op ulieden; nochtans zo weet dit, dat het Koninkrijk Gods nabij u gekomen is.
\par 12 En Ik zeg u, dat het dien van Sodom verdragelijker wezen zal in dien dag, dan dezelve stad.
\par 13 Wee u, Chorazin, wee u, Bethsaida, want zo in Tyrus en Sidon de krachten geschied waren, die in u geschied zijn, zij zouden eertijds, in zak en as zittende, zich bekeerd hebben.
\par 14 Doch het zal Tyrus en Sidon verdragelijker zijn in het oordeel, dan ulieden.
\par 15 En gij, Kapernaum, die tot den hemel toe verhoogd zijt, gij zult tot de hel toe nedergestoten worden.
\par 16 Wie u hoort, die hoort Mij; en wie u verwerpt, die verwerpt Mij; en wie Mij verwerpt, die verwerpt Dengene, Die Mij gezonden heeft.
\par 17 En de zeventigen zijn wedergekeerd met blijdschap, zeggende: Heere, ook de duivelen zijn ons onderworpen, in Uw Naam.
\par 18 En Hij zeide tot hen: Ik zag den satan, als een bliksem, uit den hemel vallen.
\par 19 Ziet, Ik geve u de macht, om op slangen en schorpioenen te treden, en over alle kracht des vijands; en geen ding zal u enigszins beschadigen.
\par 20 Doch verblijdt u daarin niet, dat de geesten u onderworpen zijn; maar verblijdt u veel meer, dat uw namen geschreven zijn in de hemelen.
\par 21 Te dier ure verheugde Zich Jezus in den geest, en zeide: Ik dank U, Vader! Heere des hemels en der aarde; dat Gij deze dingen voor de wijzen en verstandigen verborgen hebt, en hebt dezelve den kinderkens geopenbaard; ja, Vader, want alzo is geweest het welbehagen voor U.
\par 22 Alle dingen zijn Mij van Mijn Vader overgegeven; en niemand weet, wie de Zoon is, dan de Vader; en wie de Vader is, dan de Zoon, en dien het de Zoon zal willen openbaren.
\par 23 En Zich kerende naar de discipelen, zeide Hij tot hen alleen: Zalig zijn de ogen, die zien, hetgeen gij ziet.
\par 24 Want Ik zeg u, dat vele profeten en koningen hebben begeerd te zien, hetgeen gij ziet, en hebben het niet gezien; en te horen, hetgeen gij hoort, en hebben het niet gehoord.
\par 25 En ziet, een zeker wetgeleerde stond op, Hem verzoekende, en zeggende: Meester, wat doende zal ik het eeuwige leven beerven?
\par 26 En Hij zeide tot hem: Wat is in de wet geschreven? Hoe leest gij?
\par 27 En hij, antwoordende, zeide: Gij zult den Heere, uw God, liefhebben, uit geheel uw hart, en uit geheel uw ziel, en uit geheel uw kracht, en uit geheel uw verstand; en uw naaste als uzelven.
\par 28 En Hij zeide tot hem: Gij hebt recht geantwoord; doe dat, en gij zult leven.
\par 29 Maar hij, willende zichzelven rechtvaardigen, zeide tot Jezus: En wie is mijn naaste?
\par 30 En Jezus, antwoordende, zeide: Een zeker mens kwam af van Jeruzalem naar Jericho, en viel onder de moordenaars, welke, hem ook uitgetogen, en daartoe zware slagen gegeven hebbende, heengingen, en lieten hem half dood liggen.
\par 31 En bij geval kwam een zeker priester denzelven weg af, en hem ziende, ging hij tegenover hem voorbij.
\par 32 En desgelijks ook een Leviet, als hij was bij die plaats, kwam hij, en zag hem, en ging tegenover hem voorbij.
\par 33 Maar een zeker Samaritaan, reizende, kwam omtrent hem, en hem ziende, werd hij met innerlijke ontferming bewogen.
\par 34 En hij, tot hem gaande, verbond zijn wonden, gietende daarin olie en wijn; en hem heffende op zijn eigen beest, voerde hem in de herberg en verzorgde hem.
\par 35 En des anderen daags weggaande, langde hij twee penningen uit, en gaf ze den waard, en zeide tot hem: Draag zorg voor hem: en zo wat gij meer aan hem ten koste zult leggen, dat zal ik u wedergeven, als ik wederkom.
\par 36 Wie dan van deze drie dunkt u de naaste geweest te zijn desgenen, die onder de moordenaars gevallen was?
\par 37 En hij zeide: Die barmhartigheid aan hem gedaan heeft. Zo zeide dan Jezus tot hem: Ga heen, en doe gij desgelijks.
\par 38 En het geschiedde, als zij reisden, dat Hij kwam in een vlek; en een zekere vrouw, met name Martha, ontving Hem in haar huis.
\par 39 En deze had een zuster, genaamd Maria, welke ook, zittende aan de voeten van Jezus, Zijn woord hoorde.
\par 40 Doch Martha was zeer bezig met veel dienens, en daarbij komende, zeide zij: Heere, trekt Gij U dat niet aan, dat mijn zuster mij alleen laat dienen? Zeg dan haar, dat zij mij helpe.
\par 41 En Jezus, antwoordende, zeide tot haar: Martha, Martha, gij bekommert en ontrust u over vele dingen;
\par 42 Maar een ding is nodig; doch Maria heeft het goede deel uitgekozen, hetwelk van haar niet zal weggenomen worden.

\chapter{11}

\par 1 En het geschiedde, toen Hij in een zekere plaats was biddende, als Hij ophield, dat een van Zijn discipelen tot Hem zeide: Heere, leer ons bidden, gelijk ook Johannes zijn discipelen geleerd heeft.
\par 2 En Hij zeide tot hen: Wanneer gij bidt, zo zegt: Onze Vader, Die in de hemelen zijt! Uw Naam worde geheiligd. Uw Koninkrijk kome. Uw wil geschiede, gelijk in den hemel, alzo ook op de aarde.
\par 3 Geef ons elken dag ons dagelijks brood.
\par 4 En vergeef ons onze zonden; want ook wij vergeven aan een iegelijk, die ons schuldig is. En leid ons niet in verzoeking, maar verlos ons van den boze.
\par 5 En Hij zeide tot hen: Wie van u zal een vriend hebben, en zal ter middernacht tot hem gaan, en tot hem zeggen: Vriend! leen mij drie broden;
\par 6 Overmits mijn vriend van de reis tot mij gekomen is, en ik heb niet, dat ik hem voorzette;
\par 7 En dat die van binnen, antwoordende, zou zeggen: Doe mij geen moeite aan; de deur is nu gesloten, en mijn kinderen zijn met mij in de slaapkamer; ik kan niet opstaan, om u te geven.
\par 8 Ik zeg ulieden: Hoewel hij niet zou opstaan en hem geven, omdat hij zijn vriend is, nochtans om zijner onbeschaamdheid wil, zal hij opstaan, en hem geven zoveel als hij er behoeft.
\par 9 En Ik zeg ulieden: Bidt, en u zal gegeven worden; zoekt, en gij zult vinden; klopt, en u zal opengedaan worden.
\par 10 Want een iegelijk, die bidt, die ontvangt; en die zoekt, die vindt; en die klopt, dien zal opengedaan worden.
\par 11 En wat vader onder u, dien de zoon om brood bidt, zal hem een steen geven, of ook om een vis, zal hem voor een vis een slang geven?
\par 12 Of zo hij ook om een ei zou bidden, zal hij hem een schorpioen geven?
\par 13 Indien dan gij, die boos zijt, weet uw kinderen goede gaven te geven, hoeveel te meer zal de hemelse Vader den Heiligen Geest geven dengenen, die Hem bidden?
\par 14 En Hij wierp een duivel uit, en die was stom. En het geschiedde, als de duivel uitgevaren was, dat de stomme sprak; en de scharen verwonderden zich.
\par 15 Maar sommigen van hen zeiden: Hij werpt de duivelen uit door Beelzebul, den overste der duivelen.
\par 16 En anderen, Hem verzoekende, begeerden van Hem een teken uit den hemel.
\par 17 Maar Hij, kennende hun gedachten, zeide tot hen: Een ieder koninkrijk, dat tegen zichzelf verdeeld is, wordt verwoest; en een huis, tegen zichzelf verdeeld zijnde, valt.
\par 18 Indien nu ook de satan tegen zichzelven verdeeld is, hoe zal zijn rijk bestaan? Dewijl gij zegt, dat Ik door Beelzebul de duivelen uitwerp.
\par 19 En indien Ik door Beelzebul de duivelen uitwerp, door wien werpen ze uw zonen uit? Daarom zullen dezen uw rechters zijn.
\par 20 Maar indien Ik door den vinger Gods de duivelen uitwerp, zo is dan het Koninkrijk Gods tot u gekomen.
\par 21 Wanneer een sterke gewapende zijn hof bewaart, zo is al wat hij heeft in vrede.
\par 22 Maar als een daarover komt, die sterker is dan hij, en hem overwint, die neemt zijn gehele wapenrusting, daar hij op vertrouwde, en deelt zijn roof uit.
\par 23 Wie met Mij niet is, die is tegen Mij; en wie met Mij niet vergadert, die verstrooit.
\par 24 Wanneer de onreine geest van den mens uitgevaren is, zo gaat hij door dorre plaatsen, zoekende rust; en die niet vindende, zegt hij: Ik zal wederkeren in mijn huis, daar ik uitgevaren ben.
\par 25 En komende, vindt hij het met bezemen gekeerd en versierd.
\par 26 Dan gaat hij heen, en neemt met zich zeven andere geesten, bozer dan hij zelf is, en ingegaan zijnde, wonen zij aldaar; en het laatste van dien mens wordt erger dan het eerste.
\par 27 En het geschiedde, als Hij deze dingen sprak, dat een zekere vrouw, de stem verheffende uit de schare, tot Hem zeide: Zalig is de buik, die U gedragen heeft, en de borsten, die Gij hebt gezogen.
\par 28 Maar Hij zeide: Ja, zalig zijn degenen, die het Woord Gods horen, en hetzelve bewaren.
\par 29 En als de scharen dicht bijeenvergaderden, begon Hij te zeggen: Dit is een boos geslacht; het verzoekt een teken, en hetzelve zal geen teken gegeven worden, dan het teken van Jonas, den profeet.
\par 30 Want gelijk Jonas den Ninevieten een teken geweest is, alzo zal ook de Zoon des mensen zijn dezen geslachte.
\par 31 De koningin van het Zuiden zal opstaan in het oordeel met de mannen van dit geslacht, en zal ze veroordelen; want zij is gekomen van de einden der aarde, om te horen de wijsheid van Salomo; en ziet, meer dan Salomo is hier.
\par 32 De mannen van Nineve, zullen opstaan in het oordeel met dit geslacht, en zullen hetzelve veroordelen; want zij hebben zich bekeerd op de prediking van Jonas; en ziet, meer dan Jonas is hier!
\par 33 En niemand, die een kaars ontsteekt, zet die in het verborgen, noch onder een koornmaat, maar op een kandelaar, opdat degenen, die inkomen, het licht zien mogen.
\par 34 De kaars des lichaams is het oog: wanneer dan uw oog eenvoudig is, zo is ook uw gehele lichaam verlicht; maar zo het boos is, zo is ook uw gehele lichaam duister.
\par 35 Zie dan toe, dat niet het licht, hetwelk in u is, duisternis zij.
\par 36 Indien dan uw lichaam geheel verlicht is, niet hebbende enig deel, dat duister is, zo zal het geheel verlicht zijn, gelijk wanneer de kaars met het schijnsel u verlicht.
\par 37 Als Hij nu dit sprak, bad Hem een zeker Farizeer, dat Hij bij hem het middagmaal wilde eten; en ingegaan zijnde, zat Hij aan.
\par 38 En de Farizeer, dat ziende, verwonderde zich, dat Hij niet eerst, voor het middagmaal, Zich gewassen had.
\par 39 En de Heere zeide tot hem: Nu gij Farizeen, gij reinigt het buitenste des drinkbekers en des schotels; maar het binnenste van u is vol van roof en boosheid.
\par 40 Gij onverstandigen! Die het buitenste heeft gemaakt, heeft Hij ook niet het binnenste gemaakt?
\par 41 Doch geeft tot aalmoes, hetgeen daarin is; en ziet, alles is u rein.
\par 42 Maar wee u, Farizeen, want gij vertient munte, en ruite, en alle moeskruid, en gij gaat voorbij het oordeel en de liefde Gods. Dit moest men doen, en het andere niet nalaten.
\par 43 Wee u, Farizeen, want gij bemint het voorgestoelte in de synagogen, en de begroetingen op de markten.
\par 44 Wee u, gij Schriftgeleerden en Farizeen, gij geveinsden, want gij zijt gelijk de graven, die niet openbaar zijn, en de mensen, die daarover wandelen, weten het niet.
\par 45 En een van de wetgeleerden, antwoordende, zeide tot Hem: Meester! als Gij deze dingen zegt, zo doet Gij ook ons smaadheid aan.
\par 46 Doch Hij zeide: Wee ook u, wetgeleerden! want gij belast de mensen met lasten, zwaar om te dragen, en zelven raakt gij die lasten niet aan met een van uw vingeren.
\par 47 Wee u, want gij bouwt de graven der profeten, en uw vaders hebben dezelve gedood.
\par 48 Zo getuigt gij dan, dat gij mede behagen hebt aan de werken uwer vaderen; want zij hebben ze gedood, en gij bouwt hun graven.
\par 49 Waarom ook de wijsheid Gods zegt: Ik zal profeten en apostelen tot hen zenden, en van die zullen zij sommigen doden, en sommigen zullen zij uitjagen;
\par 50 Opdat van dit geslacht afgeeist worde het bloed van al de profeten, dat vergoten is van de grondlegging der wereld af.
\par 51 Van het bloed van Abel, tot het bloed van Zacharia, die gedood is tussen het altaar en het huis Gods; ja, zeg Ik u, het zal afgeeist worden van dit geslacht!
\par 52 Wee u, gij wetgeleerden, want gij hebt den sleutel der kennis weggenomen; gijzelven zijt niet ingegaan, en die ingingen, hebt gij verhinderd.
\par 53 En als Hij deze dingen tot hen zeide, begonnen de Schriftgeleerden en Farizeen hard aan te houden, en Hem van vele dingen te doen spreken;
\par 54 Hem lagen leggende, en zoekende iets uit Zijn mond te bejagen, opdat zij Hem beschuldigen mochten.

\chapter{12}

\par 1 Daarentussen als vele duizenden der schare bijeenvergaderd waren, zodat zij elkander vertraden, begon Hij te zeggen tot Zijn discipelen: Vooreerst wacht uzelven voor den zuurdesem der Farizeen, welke is geveinsdheid.
\par 2 En er is niets bedekt, dat niet zal ontdekt worden, en verborgen, dat niet zal geweten worden.
\par 3 Daarom, al wat gij in de duisternis gezegd hebt, zal in het licht gehoord worden; en wat gij in het oor gesproken hebt, in de binnenkamers, zal op de daken gepredikt worden.
\par 4 En Ik zeg u, Mijn vrienden: Vreest u niet voor degenen, die het lichaam doden, en daarna niet meer kunnen doen.
\par 5 Maar Ik zal u tonen, Wien gij vrezen zult: vreest Dien, Die, nadat Hij gedood heeft, ook macht heeft in de hel te werpen; ja, Ik zeg u, vreest Dien!
\par 6 Worden niet vijf musjes verkocht voor twee penningskens? En niet een van die is voor God vergeten.
\par 7 Ja, ook de haren uws hoofds zijn alle geteld. Vreest dan niet; gij gaat vele musjes te boven.
\par 8 En Ik zeg u: Een iegelijk, die Mij belijden zal voor de mensen, dien zal ook de Zoon des mensen belijden voor de engelen Gods.
\par 9 Maar wie Mij verloochenen zal voor de mensen, die zal verloochend worden voor de engelen Gods.
\par 10 En een iegelijk, die enig woord spreken zal tegen den Zoon des mensen, het zal hem vergeven worden; maar wie tegen den Heiligen Geest gelasterd zal hebben, dien zal het niet vergeven worden.
\par 11 En wanneer zij u heenbrengen zullen in de synagogen, en tot de overheden en de machten, zo zijt niet bezorgd, hoe of wat gij tot verantwoording zeggen, of wat gij spreken zult;
\par 12 Want de Heilige Geest zal u in dezelve ure leren, hetgeen gij spreken moet.
\par 13 En een uit de schare zeide tot Hem: Meester, zeg mijn broeder, dat hij met mij de erfenis dele.
\par 14 Maar Hij zeide tot hem: Mens, wie heeft Mij tot een rechter of scheidsman over ulieden gesteld?
\par 15 En Hij zeide tot hen: Ziet toe en wacht u van de gierigheid; want het is niet in den overvloed gelegen, dat iemand leeft uit zijn goederen.
\par 16 En Hij zeide tot hen een gelijkenis, en sprak: Eens rijken mensen land had wel gedragen;
\par 17 En hij overleide bij zichzelven, zeggende: Wat zal ik doen, want ik heb niet, waarin ik mijn vruchten zal verzamelen.
\par 18 En hij zeide: Dit zal ik doen; ik zal mijn schuren afbreken, en grotere bouwen, en zal aldaar verzamelen al dit mijn gewas, en deze mijn goederen;
\par 19 En ik zal tot mijn ziel zeggen: Ziel! gij hebt vele goederen, die opgelegd zijn voor vele jaren, neem rust, eet, drink, wees vrolijk.
\par 20 Maar God zeide tot hem: Gij dwaas! in dezen nacht zal men uw ziel van u afeisen; en hetgeen gij bereid hebt, wiens zal het zijn?
\par 21 Alzo is het met dien, die zichzelven schatten vergadert, en niet rijk is in God.
\par 22 En Hij zeide tot Zijn discipelen: Daarom zeg Ik u: Zijt niet bezorgd voor uw leven, wat gij eten zult, noch voor het lichaam, waarmede gij u kleden zult.
\par 23 Het leven is meer dan het voedsel, en het lichaam dan de kleding.
\par 24 Aanmerkt de raven, dat zij niet zaaien, noch maaien, welke geen spijskamer noch schuur hebben, en God voedt dezelve; hoeveel gaat gij de vogelen te boven?
\par 25 Wie toch van u kan, met bezorgd te zijn, een el tot zijn lengte toedoen?
\par 26 Indien gij dan ook het minste niet kunt, wat zijt gij voor de andere dingen bezorgd?
\par 27 Aanmerkt de lelien, hoe zij wassen; zij arbeiden niet, en spinnen niet; en Ik zeg u: ook Salomo in al zijn heerlijkheid is niet bekleed geweest als een van deze.
\par 28 Indien nu God het gras dat heden op het veld is, en morgen in den oven geworpen wordt, alzo bekleedt, hoeveel meer u, gij kleingelovigen!
\par 29 En gijlieden, vraagt niet, wat gij eten, of wat gij drinken zult; en weest niet wankelmoedig.
\par 30 Want al deze dingen zoeken de volken der wereld; maar uw Vader weet, dat gij deze dingen behoeft.
\par 31 Maar zoekt het Koninkrijk Gods, en al deze dingen zullen u toegeworpen worden.
\par 32 Vreest niet, gij klein kuddeken, want het is uws Vaders welbehagen, ulieden het Koninkrijk te geven.
\par 33 Verkoopt hetgeen gij hebt, en geeft aalmoes. Maakt uzelven buidels, die niet verouden, een schat, die niet afneemt, in de hemelen, daar de dief niet bijkomt, noch de mot verderft.
\par 34 Want waar uw schat is, aldaar zal ook uw hart zijn.
\par 35 Laat uw lendenen omgord zijn, en de kaarsen brandende.
\par 36 En zijt gij den mensen gelijk, die op hun heer wachten, wanneer hij wederkomen zal van de bruiloft, opdat, als hij komt en klopt, zij hem terstond mogen opendoen.
\par 37 Zalig zijn die dienstknechten, welke de heer, als hij komt, zal wakende vinden. Voorwaar, Ik zeg u, dat hij zich zal omgorden, en zal hen doen aanzitten, en bijkomende, zal hij hen dienen.
\par 38 En zo hij komt in de tweede nacht wake, en komt in de derde wake, en vindt hen alzo, zalig zijn dezelve dienstknechten.
\par 39 Maar weet dit, dat, indien de heer des huizes geweten had, in welke ure de dief zou komen, hij zou gewaakt hebben, en zou zijn huis niet hebben laten doorgraven.
\par 40 Gij dan, zijt ook bereid; want in welke ure gij het niet meent, zal de Zoon des mensen komen.
\par 41 En Petrus zeide tot Hem: Heere! zegt Gij deze gelijkenis tot ons, of ook tot allen?
\par 42 En de Heere zeide: Wie is dan de getrouwe en voorzichtige huisbezorger, dien de heer over zijn dienstboden zal zetten, om hun ter rechter tijd het bescheiden deel spijze te geven?
\par 43 Zalig is de dienstknecht, welken zijn heer, als hij komt, zal vinden, alzo doende.
\par 44 Waarlijk, Ik zeg ulieden, dat hij hem over al zijn goederen zetten zal.
\par 45 Maar indien dezelve dienstknecht in zijn hart zou zeggen: Mijn heer vertoeft te komen; en zou beginnen de knechten en de dienstmaagden te slaan, en te eten en te drinken, en dronken te worden;
\par 46 Zo zal de heer deszelven dienstknechts komen ten dage, in welken hij hem niet verwacht, en ter ure, die hij niet weet; en zal hem afscheiden, en zal zijn deel zetten met de ontrouwen.
\par 47 En die dienstknecht, welke geweten heeft den wil zijns heeren, en zich niet bereid, noch naar zijn wil gedaan heeft, die zal met vele slagen geslagen worden.
\par 48 Maar die denzelven niet geweten heeft, en gedaan heeft dingen, die slagen waardig zijn, die zal met weinige slagen geslagen worden. En een iegelijk, wien veel gegeven is, van dien zal veel geeist worden; en wien men veel vertrouwd heeft, van dien zal men overvloediger eisen.
\par 49 Ik ben gekomen, om vuur op de aarde te werpen; en wat wil Ik, indien het alrede ontstoken is?
\par 50 Maar Ik moet met een doop gedoopt worden; en hoe worde Ik geperst, totdat het volbracht zij!
\par 51 Meent gij, dat Ik gekomen ben, om vrede te geven op de aarde? Neen, zeg Ik u, maar veeleer verdeeldheid.
\par 52 Want van nu aan zullen er vijf in een huis verdeeld zijn, drie tegen twee, en twee tegen drie.
\par 53 De vader zal tegen den zoon verdeeld zijn, en de zoon tegen den vader; de moeder tegen de dochter; en de dochter tegen de moeder; de schoonmoeder tegen haar schoondochter, en de schoondochter tegen haar schoonmoeder.
\par 54 En Hij zeide ook tot de scharen: Wanneer gij een wolk ziet opgaan van het westen, terstond zegt gijlieden: Er komt regen; en het geschiedt alzo.
\par 55 En wanneer gij den zuidenwind ziet waaien, zo zegt gij: Er zal hitte zijn; en het geschiedt.
\par 56 Gij geveinsden, het aanschijn der aarde en des hemels weet gij te beproeven; en hoe beproeft gij dezen tijd niet?
\par 57 En waarom oordeelt gij ook van uzelven niet, hetgeen recht is?
\par 58 Want als gij heengaat met uw wederpartij voor de overheid, zo doet naarstigheid op den weg, om van hem verlost te worden; opdat hij misschien u niet voor den rechter trekke, en de rechter u den gerechtsdienaar overlevere, en de gerechtsdienaar u in de gevangenis werpe.
\par 59 Ik zeg u: Gij zult van daar geenszins uitgaan, totdat gij ook het laatste penningsken betaald zult hebben.

\chapter{13}

\par 1 En er waren te dierzelfder tijd enigen tegenwoordig, die Hem boodschapten van de Galileers, welker bloed Pilatus met hun offeranden gemengd had.
\par 2 En Jezus antwoordde, en zeide tot hen: Meent gij, dat deze Galileers zondaars zijn geweest boven al de Galileers, omdat zij zulks geleden hebben?
\par 3 Ik zeg u: Neen zij; maar indien gij u niet bekeert, zo zult gij allen desgelijks vergaan.
\par 4 Of die achttien, op welke de toren in Siloam viel, en doodde ze; meent gij, dat deze schuldenaars zijn geweest, boven alle mensen, die in Jeruzalem wonen?
\par 5 Ik zeg u: Neen zij; maar indien gij u niet bekeert, zo zult gij allen insgelijks vergaan.
\par 6 En Hij zeide deze gelijkenis: Een zeker man had een vijgeboom, geplant in zijn wijngaard; en hij kwam en zocht vrucht daarop, en vond ze niet.
\par 7 En hij zeide tot den wijngaardenier: Zie, ik kome nu drie jaren, zoekende vrucht op dezen vijgeboom, en vind ze niet; houw hem uit; waartoe beslaat hij ook onnuttelijk de aarde?
\par 8 En hij, antwoordende, zeide tot hem: Heer, laat hem ook nog dit jaar, totdat ik om hem gegraven en mest gelegd zal hebben;
\par 9 En indien hij vrucht zal voortbrengen, laat hem staan; maar indien niet, zo zult gij hem namaals uithouwen.
\par 10 En Hij leerde op den sabbat in een der synagogen.
\par 11 En ziet, er was een vrouw, die een geest der krankheid achttien jaren lang gehad had, en zij was samengebogen, en kon zich ganselijk niet oprichten.
\par 12 En Jezus, haar ziende, riep haar tot Zich, en zeide tot haar: Vrouw, gij zijt verlost van uw krankheid.
\par 13 En Hij legde de handen op haar; en zij werd terstond weder recht, en verheerlijkte God.
\par 14 En de overste der synagoge, kwalijk nemende, dat Jezus op den sabbat genezen had, antwoordde en zeide tot de schare: Er zijn zes dagen, in welke men moet werken; komt dan in dezelve, en laat u genezen, en niet op den dag des sabbats.
\par 15 De Heere dan antwoordde hem en zeide: Gij geveinsde, maakt niet een iegelijk van u op den sabbat zijn os of ezel van de kribbe los, en leidt hem heen om te doen drinken?
\par 16 En deze, die een dochter Abrahams is, welke de satan, ziet, nu achttien jaren gebonden had, moest die niet losgemaakt worden van dezen band, op den dag des sabbats?
\par 17 En als Hij dit zeide, werden zij allen beschaamd, die zich tegen Hem stelden; en al de schare verblijdde zich over al de heerlijke dingen, die van Hem geschiedden.
\par 18 En Hij zeide: Wien is het Koninkrijk Gods gelijk, en waarbij zal Ik hetzelve vergelijken?
\par 19 Het is gelijk aan een mostaardzaad, hetwelk een mens genomen en in zijn hof geworpen heeft; en het wies op, en werd tot een groten boom, en de vogelen des hemels nestelden in zijn takken.
\par 20 En Hij zeide wederom: Waarbij zal Ik het Koninkrijk Gods vergelijken?
\par 21 Het is gelijk aan een zuurdesem, welken een vrouw nam, en verborg in drie maten meels, totdat het geheel gezuurd was.
\par 22 En Hij reisde van de ene stad en vlek tot de andere, lerende, en richtende Zijn reis naar Jeruzalem.
\par 23 En er zeide een tot Hem: Heere, zijn er ook weinigen, die zalig worden? En Hij zeide tot hen:
\par 24 Strijdt om in te gaan door de enge poort; want velen, zeg Ik u, zullen zoeken in te gaan, en zullen niet kunnen;
\par 25 Namelijk nadat de Heer des huizes zal opgestaan zijn, en de deur zal gesloten hebben, en gij zult beginnen buiten te staan, en aan de deur te kloppen, zeggende: Heere, Heere, doe ons open! en Hij zal antwoorden en tot u zeggen: Ik ken u niet, van waar gij zijt.
\par 26 Alsdan zult gij beginnen te zeggen: Wij hebben in Uw tegenwoordigheid gegeten en gedronken, en Gij hebt in onze straten geleerd.
\par 27 En Hij zal zeggen: Ik zeg u, Ik ken u niet, van waar gij zijt; wijkt van Mij af, alle gij werkers der ongerechtigheid!
\par 28 Aldaar zal zijn wening en knersing der tanden, wanneer gij zult zien Abraham, en Izak, en Jakob, en al de profeten in het Koninkrijk Gods, maar ulieden buiten uitgeworpen.
\par 29 En daar zullen er komen van Oosten en Westen, en van Noorden en Zuiden, en zullen aanzitten in het Koninkrijk Gods.
\par 30 En ziet, er zijn laatsten, die de eersten zullen zijn; en er zijn eersten, die de laatsten zullen zijn.
\par 31 Te dienzelfden dage kwamen er enige Farizeen, zeggende tot Hem: Ga weg, en vertrek van hier; want Herodes wil U doden.
\par 32 En Hij zeide tot hen: Gaat heen, en zegt dien vos: Zie, Ik werp duivelen uit, en maak gezond, heden en morgen, en ten derden dage worde Ik voleindigd.
\par 33 Doch Ik moet heden, en morgen, en den volgenden dag reizen; want het gebeurt niet, dat een profeet gedood wordt buiten Jeruzalem.
\par 34 Jeruzalem, Jeruzalem! gij, die de profeten doodt, en stenigt, die tot u gezonden zijn, hoe menigmaal heb Ik uw kinderen willen bijeenvergaderen, gelijkerwijs een hen haar kiekens onder de vleugelen vergadert; en gijlieden hebt niet gewild?
\par 35 Ziet, uw huis wordt ulieden woest gelaten. En voorwaar, Ik zeg u, dat gij Mij niet zult zien, totdat de tijd zal gekomen zijn, als gij zult zeggen: Gezegend is Hij, Die komt in den Naam des Heeren!

\chapter{14}

\par 1 En het geschiedde, als Hij gekomen was in het huis van een der oversten der Farizeen, op den sabbat, om brood te eten, dat zij Hem waarnamen.
\par 2 En ziet, er was een zeker waterzuchtig mens voor Hem.
\par 3 En Jezus, antwoordende, zeide tot de wetgeleerden en Farizeen, en sprak: Is het ook geoorloofd op den sabbat gezond te maken?
\par 4 Maar zij zwegen stil. En Hij nam hem, en genas hem, en liet hem gaan.
\par 5 En Hij, hun antwoordende, zeide: Wiens ezel of os van ulieden zal in een put vallen, en die hem niet terstond zal uittrekken op den dag des sabbats?
\par 6 En zij konden Hem daarop niet weder antwoorden.
\par 7 En Hij zeide tot de genoden een gelijkenis, aanmerkende, hoe zij de vooraanzittingen verkozen; zeggende tot hen:
\par 8 Wanneer gij van iemand ter bruiloft genood zult zijn, zo zet u niet in de eerste zitplaats; opdat niet misschien een waardiger dan gij van hem genood zij;
\par 9 En hij, komende, die u en hem genood heeft, tot u zegge: Geef dezen plaats; en gij alsdan zoudt beginnen met schaamte de laatste plaats te houden.
\par 10 Maar wanneer gij genood zult zijn, ga heen en zet u in de laatste plaats; opdat, wanneer hij komt, die u genood heeft, hij tot u zegge: Vriend, ga hoger op. Alsdan zal het u eer zijn voor degenen, die met u aanzitten.
\par 11 Want een iegelijk, die zichzelven verhoogt, zal vernederd worden; en die zichzelven vernedert, zal verhoogd worden.
\par 12 En Hij zeide ook tot dengene, die Hem genood had: Wanneer gij een middagmaal of avondmaal zult houden, zo roep niet uw vrienden, noch uw broeders, noch uw magen, noch uw rijke geburen; opdat ook dezelve u niet te eniger tijd wedernoden, en u vergelding geschiede.
\par 13 Maar wanneer gij een maaltijd zult houden, zo nood armen, verminkten, kreupelen, blinden;
\par 14 En gij zult zalig zijn, omdat zij niet hebben, om u te vergelden; want het zal u vergolden worden in de opstanding der rechtvaardigen.
\par 15 En als een van degenen, die mede aanzaten, deze dingen hoorde, zeide hij tot Hem: Zalig is hij, die brood eet in het Koninkrijk Gods.
\par 16 Maar Hij zeide tot hem: Een zeker mens bereidde een groot avondmaal, en hij noodde er velen.
\par 17 En hij zond zijn dienstknecht uit ten ure des avondmaals, om den genoden te zeggen: Komt, want alle dingen zijn nu gereed.
\par 18 En zij begonnen allen zich eendrachtelijk te ontschuldigen. De eerste zeide tot hem: Ik heb een akker gekocht, en het is nodig, dat ik uitga, en hem bezie; ik bid u, houd mij voor verontschuldigd.
\par 19 En een ander zeide: Ik heb vijf juk ossen gekocht, en ik ga heen, om die te beproeven; ik bid u, houd mij voor verontschuldigd.
\par 20 En een ander zeide: Ik heb een vrouw getrouwd, en daarom kan ik niet komen.
\par 21 En dezelve dienstknecht weder gekomen zijnde, boodschapte deze dingen zijn heer. Toen werd de heer des huizes toornig, en zeide tot zijn dienstknecht: Ga haastelijk uit in de straten en wijken der stad, en breng de armen, en verminkten, en kreupelen, en blinden hier in.
\par 22 En de dienstknecht zeide: Heere, het is geschied, gelijk gij bevolen hebt, en nog is er plaats.
\par 23 En de heer zeide tot den dienstknecht: Ga uit in de wegen en heggen; en dwing ze in te komen, opdat mijn huis vol worde;
\par 24 Want ik zeg ulieden, dat niemand van die mannen, die genood waren, mijn avondmaal smaken zal.
\par 25 En vele scharen gingen met Hem; en Hij, Zich omkerende, zeide tot hen:
\par 26 Indien iemand tot Mij komt en niet haat zijn vader, en moeder, en vrouw, en kinderen, en broeders, en zusters, ja, ook zelfs zijn eigen leven, die kan Mijn discipel niet zijn.
\par 27 En wie zijn kruis niet draagt, en Mij navolgt, die kan Mijn discipel niet zijn.
\par 28 Want wie van u, willende een toren bouwen, zit niet eerst neder, en overrekent de kosten, of hij ook heeft, hetgeen tot volmaking nodig is?
\par 29 Opdat niet misschien, als hij het fondament gelegd heeft, en niet kan voleindigen, allen, die het zien, hem beginnen te bespotten.
\par 30 Zeggende: Deze mens heeft begonnen te bouwen, en heeft niet kunnen voleindigen.
\par 31 Of wat koning, gaande naar den krijg, om tegen een anderen koning te slaan, zit niet eerst neder, en beraadslaagt, of hij machtig is met tien duizend te ontmoeten dengene, die met twintig duizend tegen hem komt?
\par 32 Anderszins zendt hij gezanten uit, terwijl degene nog verre is, en begeert, hetgeen tot vrede dient.
\par 33 Alzo dan een iegelijk van u, die niet verlaat alles, wat hij heeft, die kan Mijn discipel niet zijn.
\par 34 Het zout is goed; maar indien het zout smakeloos geworden is, waarmede zal het smakelijk gemaakt worden?
\par 35 Het is noch tot het land, noch tot den mesthoop bekwaam; men werpt het weg. Wie oren heeft, om te horen, die hore.

\chapter{15}

\par 1 En al de tollenaars en de zondaars naderden tot Hem, om Hem te horen.
\par 2 En de Farizeen en de Schriftgeleerden murmureerden, zeggende: Deze ontvangt de zondaars, en eet met hen.
\par 3 En Hij sprak tot hen deze gelijkenis, zeggende:
\par 4 Wat mens onder u, hebbende honderd schapen; en een van die verliezende, verlaat niet de negen en negentig in de woestijn, en gaat naar het verlorene, totdat hij hetzelve vinde?
\par 5 En als hij het gevonden heeft, legt hij het op zijn schouders, verblijd zijnde.
\par 6 En te huis komende, roept hij de vrienden en de geburen samen, zeggende tot hen: Weest blijde met mij; want ik heb mijn schaap gevonden, dat verloren was.
\par 7 Ik zeg ulieden, dat er alzo blijdschap zal zijn in den hemel over een zondaar, die zich bekeert, meer dan over negen en negentig rechtvaardigen, die de bekering niet van node hebben.
\par 8 Of wat vrouw, hebbende tien penningen, indien zij een penning verliest, ontsteekt niet een kaars, en keert het huis met bezemen, en zoekt naarstiglijk, totdat zij dien vindt?
\par 9 En als zij dien gevonden heeft, roept zij de vriendinnen en de geburinnen samen, zeggende: Weest blijde met mij; want ik heb den penning gevonden, dien ik verloren had.
\par 10 Alzo, zeg Ik ulieden, is er blijdschap voor de engelen Gods over een zondaar, die zich bekeert.
\par 11 En Hij zeide: Een zeker mens had twee zonen.
\par 12 En de jongste van hen zeide tot den vader: Vader, geef mij het deel des goeds, dat mij toekomt. En hij deelde hun het goed.
\par 13 En niet vele dagen daarna, de jongste zoon, alles bijeenvergaderd hebbende, is weggereisd in een ver gelegen land, en heeft aldaar zijn goed doorgebracht, levende overdadiglijk.
\par 14 En als hij het alles verteerd had, werd er een grote hongersnood in datzelve land, en hij begon gebrek te lijden.
\par 15 En hij ging heen, en voegde zich bij een van de burgers deszelven lands; en die zond hem op zijn land om de zwijnen te weiden.
\par 16 En hij begeerde zijn buik te vullen met den draf, dien de zwijnen aten; en niemand gaf hem dien.
\par 17 En tot zichzelven gekomen zijnde, zeide hij: Hoe vele huurlingen mijns vaders hebben overvloed van brood, en ik verga van honger!
\par 18 Ik zal opstaan en tot mijn vader gaan, en ik zal tot hem zeggen: Vader, ik heb gezondigd tegen den Hemel, en voor u;
\par 19 En ik ben niet meer waardig uw zoon genaamd te worden; maak mij als een van uw huurlingen.
\par 20 En opstaande ging hij naar zijn vader. En als hij nog ver van hem was, zag hem zijn vader, en werd met innerlijke ontferming bewogen; en toe lopende, viel hem om zijn hals, en kuste hem.
\par 21 En de zoon zeide tot hem: Vader, ik heb gezondigd tegen den Hemel, en voor u, en ben niet meer waardig uw zoon genaamd te worden.
\par 22 Maar de vader zeide tot zijn dienstknechten: Brengt hier voor het beste kleed, en doet het hem aan, en geeft hem een ring aan zijn hand, en schoenen aan de voeten;
\par 23 En brengt het gemeste kalf, en slacht het; en laat ons eten en vrolijk zijn.
\par 24 Want deze mijn zoon was dood, en is weder levend geworden; en hij was verloren, en is gevonden! En zij begonnen vrolijk te zijn.
\par 25 En zijn oudste zoon was in het veld; en als hij kwam, en het huis genaakte, hoorde hij het gezang en het gerei,
\par 26 En tot zich geroepen hebbende een van de knechten, vraagde, wat dat mocht zijn.
\par 27 En deze zeide tot hem: Uw broeder is gekomen, en uw vader heeft het gemeste kalf geslacht, omdat hij hem gezond weder ontvangen heeft.
\par 28 Maar hij werd toornig, en wilde niet ingaan. Zo ging dan zijn vader uit, en bad hem.
\par 29 Doch hij, antwoordende, zeide tot den vader: Zie, ik dien u nu zo vele jaren, en heb nooit uw gebod overtreden, en gij hebt mij nooit een bokje gegeven, opdat ik met mijn vrienden mocht vrolijk zijn.
\par 30 Maar als deze uw zoon gekomen is, die uw goed met hoeren doorgebracht heeft, zo hebt gij hem het gemeste kalf geslacht.
\par 31 En hij zeide tot hem: Kind, gij zijt altijd bij mij, en al het mijne is uwe.
\par 32 Men behoorde dan vrolijk en blijde te zijn; want deze uw broeder was dood, en is weder levend geworden; en hij was verloren, en is gevonden.

\chapter{16}

\par 1 En Hij zeide ook tot Zijn discipelen: Er was een zeker rijk mens, welke een rentmeester had; en deze werd bij hem verklaagd, als die zijn goederen doorbracht.
\par 2 En hij riep hem, en zeide tot hem: Hoe hoor ik dit van u? Geef rekenschap van uw rentmeesterschap; want gij zult niet meer kunnen rentmeester zijn.
\par 3 En de rentmeester zeide bij zichzelven: Wat zal ik doen, dewijl mijn heer dit rentmeesterschap van mij neemt? Graven kan ik niet; te bedelen schaam ik mij.
\par 4 Ik weet, wat ik doen zal, opdat, wanneer ik van het rentmeesterschap afgezet zal wezen, zij mij in hun huizen ontvangen.
\par 5 En hij riep tot zich een iegelijk van de schuldenaars zijns heeren, en zeide tot den eersten: Hoeveel zijt gij mijn heer schuldig?
\par 6 En hij zeide: Honderd vaten olie. En hij zeide tot hem: Neem uw handschrift, en nederzittende, schrijf haastelijk vijftig.
\par 7 Daarna zeide hij tot een anderen: En gij, hoeveel zijt gij schuldig? En hij zeide: Honderd mudden tarwe. En hij zeide tot hem: Neem uw handschrift, en schrijf tachtig.
\par 8 En de heer prees den onrechtvaardigen rentmeester, omdat hij voorzichtiglijk gedaan had; want de kinderen dezer wereld zijn voorzichtiger, dan de kinderen des lichts, in hun geslacht.
\par 9 En Ik zeg ulieden: Maakt uzelven vrienden uit den onrechtvaardigen Mammon, opdat, wanneer u ontbreken zal, zij u mogen ontvangen in de eeuwige tabernakelen.
\par 10 Die getrouw is in het minste, die is ook in het grote getrouw; en die in het minste onrechtvaardig is, die is ook in het grote onrechtvaardig.
\par 11 Zo gij dan in den onrechtvaardigen Mammon niet getrouw zijt geweest, wie zal u het ware vertrouwen?
\par 12 En zo gij in eens anders goed niet getrouw zijt geweest, wie zal u het uwe geven?
\par 13 Geen huisknecht kan twee heren dienen; want of hij zal den enen haten, en den anderen liefhebben, of hij zal den enen aanhangen, en den anderen verachten; gij kunt God niet dienen en den Mammon.
\par 14 En al deze dingen hoorden ook de Farizeen, die geldgierig waren, en zij beschimpten Hem.
\par 15 En Hij zeide tot hen: Gij zijt het, die uzelven rechtvaardigt voor de mensen; maar God kent uw harten; want dat hoog is onder de mensen, is een gruwel voor God.
\par 16 De wet en de profeten zijn tot op Johannes; van dien tijd af wordt het Koninkrijk Gods verkondigd, en een iegelijk doet geweld op hetzelve.
\par 17 En het is lichter, dat de hemel en de aarde voorbijgaan, dan dat een tittel der wet valle.
\par 18 Een iegelijk, die zijn vrouw verlaat, en een andere trouwt, die doet overspel; en een iegelijk, die de verlatene van den man trouwt, die doet ook overspel.
\par 19 En er was een zeker rijk mens, en was gekleed met purper en zeer fijn lijnwaad, levende allen dag vrolijk en prachtig.
\par 20 En er was een zeker bedelaar, met name Lazarus, welke lag voor zijn poort vol zweren;
\par 21 En begeerde verzadigd te worden van de kruimkens, die van de tafel des rijken vielen; maar ook de honden kwamen en lekten zijn zweren.
\par 22 En het geschiedde, dat de bedelaar stierf, en van de engelen gedragen werd in den schoot van Abraham.
\par 23 En de rijke stierf ook, en werd begraven. En als hij in de hel zijn ogen ophief, zijnde in de pijn, zag hij Abraham van verre, en Lazarus in zijn schoot.
\par 24 En hij riep en zeide: Vader Abraham, ontferm u mijner, en zend Lazarus, dat hij het uiterste zijns vingers in het water dope, en verkoele mijn tong; want ik lijde smarten in deze vlam.
\par 25 Maar Abraham zeide: Kind, gedenk, dat gij uw goed ontvangen hebt in uw leven, en Lazarus desgelijks het kwade; en nu wordt hij vertroost, en gij lijdt smarten.
\par 26 En boven dit alles, tussen ons en ulieden is een grote klove gevestigd, zodat degenen, die van hier tot u willen overgaan, niet zouden kunnen, noch ook die daar zijn, van daar tot ons overkomen.
\par 27 En hij zeide: Ik bid u dan, vader, dat gij hem zendt tot mijns vaders huis;
\par 28 Want ik heb vijf broeders; dat hij hun dit betuige, opdat ook zij niet komen in deze plaats der pijniging.
\par 29 Abraham zeide tot hem: Zij hebben Mozes en de profeten, dat zij die horen.
\par 30 En hij zeide: Neen, vader Abraham, maar zo iemand van de doden tot hen heenging, zij zouden zich bekeren.
\par 31 Doch Abraham zeide tot hem: Indien zij Mozes en de profeten niet horen, zo zullen zij ook, al waren het, dat er iemand uit de doden opstond, zich niet laten gezeggen.

\chapter{17}

\par 1 En Hij zeide tot de discipelen: Het kan niet wezen, dat er geen ergernissen komen; doch wee hem, door welken zij komen;
\par 2 Het zoude hem nuttiger zijn, dat een molensteen om zijn hals gedaan ware, en hij in de zee geworpen, dan dat hij een van deze kleinen zou ergeren.
\par 3 Wacht uzelven. En indien uw broeder tegen u zondigt, zo bestraf hem; en indien het hem leed is, zo vergeef het hem.
\par 4 En indien hij zevenmaal daags tegen u zondigt, en zevenmaal daags tot u wederkeert, zeggende: Het is mij leed; zo zult gij het hem vergeven.
\par 5 En de apostelen zeiden tot den Heere: Vermeerder ons het geloof.
\par 6 En de Heere zeide: Zo gij een geloof hadt als een mostaardzaad, gij zoudt tegen dezen moerbezienboom zeggen: Word ontworteld, en in de zee geplant, en hij zou u gehoorzaam zijn.
\par 7 En wie van u heeft een dienstknecht ploegende, of de beesten hoedende, die tot hem, als hij van den akker inkomt, terstond zal zeggen: Kom bij, en zit aan?
\par 8 Maar zal hij niet tot hem zeggen: Bereid, dat ik te avond zal eten, en omgord u, en dien mij, totdat ik zal gegeten en gedronken hebben; en eet en drink gij daarna?
\par 9 Dankt hij ook denzelven dienstknecht omdat hij gedaan heeft, hetgeen hem bevolen was? Ik meen, neen.
\par 10 Alzo ook gij, wanneer gij zult gedaan hebben al hetgeen u bevolen is, zo zegt: Wij zijn onnutte dienstknechten; want wij hebben maar gedaan, hetgeen wij schuldig waren te doen.
\par 11 En het geschiedde, als Hij naar Jeruzalem reisde, dat Hij door het midden van Samaria en Galilea ging.
\par 12 En als Hij in een zeker vlek kwam, ontmoetten Hem tien melaatse mannen, welke stonden van verre;
\par 13 En zij verhieven hun stem, zeggende: Jezus, Meester! ontferm U onzer!
\par 14 En als Hij hen zag, zeide Hij tot hen: Gaat heen en vertoont uzelven den priesteren. En het geschiedde, terwijl zij heengingen, dat zij gereinigd werden.
\par 15 En een van hen, ziende, dat hij genezen was, keerde wederom, met grote stemme God verheerlijkende.
\par 16 En hij viel op het aangezicht voor Zijn voeten, Hem dankende; en dezelve was een Samaritaan;
\par 17 En Jezus, antwoordende, zeide: Zijn niet de tien gereinigd geworden, en waar zijn de negen?
\par 18 En zijn er geen gevonden, die wederkeren, om Gode eer te geven, dan deze vreemdeling?
\par 19 En Hij zeide tot hem: Sta op, en ga heen; uw geloof heeft u behouden.
\par 20 En gevraagd zijnde van de Farizeen, wanneer het Koninkrijk Gods komen zou, heeft Hij hun geantwoord en gezegd: Het Koninkrijk Gods komt niet met uiterlijk gelaat.
\par 21 En men zal niet zeggen: Ziet hier, of ziet daar, want, ziet, het Koninkrijk Gods is binnen ulieden.
\par 22 En Hij zeide tot de discipelen: Er zullen dagen komen, wanneer gij zult begeren een der dagen van den Zoon des mensen te zien, en gij zult dien niet zien.
\par 23 En zij zullen tot u zeggen: Ziet hier, of ziet daar is Hij; gaat niet heen, en volgt niet.
\par 24 Want gelijk de bliksem, die van het ene einde onder den hemel bliksemt, tot het andere onder den hemel schijnt, alzo zal ook de Zoon des mensen wezen in Zijn dag.
\par 25 Maar eerst moet Hij veel lijden, en verworpen worden van dit geslacht.
\par 26 En gelijk het geschied is in de dagen van Noach, alzo zal het ook zijn in de dagen van den Zoon des mensen.
\par 27 Zij aten, zij dronken, zij namen ten huwelijk, zij werden ten huwelijk gegeven, tot den dag, op welken Noach in de ark ging, en de zondvloed kwam, en verdierf ze allen.
\par 28 Desgelijks ook, gelijk het geschiedde in de dagen van Lot; zij aten, zij dronken, zij kochten, zij verkochten, zij plantten, zij bouwden;
\par 29 Maar op den dag, op welken Lot van Sodom uitging, regende het vuur en sulfer van den hemel, en verdierf ze allen.
\par 30 Even alzo zal het zijn in den dag, op welken de Zoon des mensen geopenbaard zal worden.
\par 31 In dienzelven dag, wie op het dak zal zijn, en zijn huisraad in huis, die kome niet af, om hetzelve weg te nemen; en wie op den akker zijn zal, die kere desgelijks niet naar hetgeen, dat achter is.
\par 32 Gedenkt aan de vrouw van Lot.
\par 33 Zo wie zijn leven zal zoeken te behouden, die zal het verliezen; en zo wie hetzelve zal verliezen, die zal het in het leven behouden.
\par 34 Ik zeg u: In dien nacht zullen twee op een bed zijn; de een zal aangenomen, en de ander zal verlaten worden.
\par 35 Twee vrouwen zullen te zamen malen; de ene zal aangenomen, en de andere zal verlaten worden.
\par 36 Twee zullen op den akker zijn; de een zal aangenomen, en de ander zal verlaten worden.
\par 37 En zij antwoordden en zeiden tot Hem: Waar, Heere? En Hij zeide tot hen: Waar het lichaam is, aldaar zullen de arenden vergaderd worden.

\chapter{18}

\par 1 En Hij zeide ook een gelijkenis tot hen, daartoe strekkende, dat men altijd bidden moet, en niet vertragen;
\par 2 Zeggende: Er was een zeker rechter in een stad, die God niet vreesde, en geen mens ontzag.
\par 3 En er was een zekere weduwe in dezelfde stad, en zij kwam tot hem, zeggende: Doe mij recht tegen mijn wederpartij.
\par 4 En hij wilde voor een langen tijd niet; maar daarna zeide hij bij zichzelven: Hoewel ik God niet vreze, en geen mens ontzie;
\par 5 Nochtans, omdat deze weduwe mij moeielijk valt, zo zal ik haar recht doen, opdat zij niet eindelijk kome, en mij het hoofd breke.
\par 6 En de Heere zeide: Hoort, wat de onrechtvaardige rechter zegt.
\par 7 Zal God dan geen recht doen Zijn uitverkorenen, die dag en nacht tot Hem roepen, hoewel Hij lankmoedig is over hen?
\par 8 Ik zeg u, dat Hij hun haastelijk recht doen zal. Doch de Zoon des mensen, als Hij komt, zal Hij ook geloof vinden op de aarde?
\par 9 En Hij zeide ook tot sommigen, die bij zichzelven vertrouwden, dat zij rechtvaardig waren, en de anderen niets achtten, deze gelijkenis:
\par 10 Twee mensen gingen op in den tempel om te bidden, de een was een Farizeer, en de ander een tollenaar.
\par 11 De Farizeer, staande, bad dit bij zichzelven: O God! ik dank U, dat ik niet ben gelijk de andere mensen, rovers, onrechtvaardigen, overspelers; of ook gelijk deze tollenaar.
\par 12 Ik vast tweemaal per week; ik geef tienden van alles, wat ik bezit.
\par 13 En de tollenaar, van verre staande, wilde ook zelfs de ogen niet opheffen naar den hemel, maar sloeg op zijn borst, zeggende: O God! wees mij zondaar genadig!
\par 14 Ik zeg ulieden: Deze ging af gerechtvaardigd in zijn huis, meer dan die; want een ieder, die zichzelven verhoogt, zal vernederd worden, en die zichzelven vernedert, zal verhoogd worden.
\par 15 En zij brachten ook de kinderkens tot Hem, opdat Hij die zou aanraken; en de discipelen, dat ziende, bestraften dezelve.
\par 16 Maar Jezus riep dezelve kinderkens tot Zich, en zeide: Laat de kinderkens tot Mij komen, en verhindert hen niet; want derzulken is het Koninkrijk Gods.
\par 17 Voorwaar, zeg Ik u: Zo wie het Koninkrijk Gods niet zal ontvangen als een kindeken, die zal geenszins in hetzelve komen.
\par 18 En een zeker overste vraagde Hem, zeggende: Goede Meester, wat doende zal ik het eeuwige leven beerven?
\par 19 En Jezus zeide tot hem: Wat noemt gij Mij goed? Niemand is goed, dan Een, namelijk God.
\par 20 Gij weet de geboden: Gij zult geen overspel doen; gij zult niet doden; gij zult niet stelen; gij zult geen valse getuigenis geven; eer uw vader en uw moeder.
\par 21 En hij zeide: Al deze dingen heb ik onderhouden van mijn jonkheid aan.
\par 22 Doch Jezus, dit horende, zeide tot hem: Nog een ding ontbreekt u; verkoop alles, wat gij hebt, en deel het onder de armen, en gij zult een schat hebben in den hemel; en kom herwaarts, volg Mij.
\par 23 Maar als hij dit hoorde, werd hij geheel droevig; want hij was zeer rijk.
\par 24 Jezus nu, ziende, dat hij geheel droevig geworden was, zeide: Hoe bezwaarlijk zullen degenen, die goed hebben, in het Koninkrijk Gods ingaan!
\par 25 Want het is lichter, dat een kemel ga door het oog van een naald, dan dat een rijke in het Koninkrijk Gods inga.
\par 26 En die dit hoorden, zeiden: Wie kan dan zalig worden?
\par 27 En Hij zeide: De dingen, die onmogelijk zijn bij de mensen, zijn mogelijk bij God.
\par 28 En Petrus zeide: Zie, wij hebben alles verlaten, en zijn U gevolgd.
\par 29 En Hij zeide tot hen: Voorwaar, Ik zeg ulieden, dat er niemand is, die verlaten heeft huis, of ouders, of broeders, of vrouw, of kinderen, om het Koninkrijk Gods;
\par 30 Die niet zal veelvoudig weder ontvangen in dezen tijd, en in de toekomende eeuw het eeuwige leven.
\par 31 En Hij nam de twaalven bij Zich, en zeide tot hen: Ziet, wij gaan op naar Jeruzalem, en het zal alles volbracht worden aan den Zoon des mensen, wat geschreven is door de profeten.
\par 32 Want Hij zal den heidenen overgeleverd worden, en Hij zal bespot worden, en smadelijk behandeld worden, en bespogen worden.
\par 33 En Hem gegeseld hebbende, zullen zij Hem doden; en ten derden dage zal Hij wederopstaan.
\par 34 En zij verstonden geen van deze dingen; en dit woord was voor hen verborgen, en zij verstonden niet, hetgeen gezegd werd.
\par 35 En het geschiedde, als Hij nabij Jericho kwam, dat een zeker blinde aan den weg zat, bedelende.
\par 36 En deze, horende de schare voorbijgaan, vraagde, wat dat ware.
\par 37 En zij boodschapten hem, dat Jezus de Nazarener voorbijging.
\par 38 En hij riep, zeggende: Jezus, Gij Zone Davids, ontferm U mijner!
\par 39 En die voorbijgingen, bestraften hem, opdat hij zwijgen zou; maar hij riep zoveel te meer: Zone Davids, ontferm U mijner!
\par 40 En Jezus, sti staande, beval, dat men denzelven tot Hem brengen zou; en als hij nabij Hem gekomen was, vraagde Hij hem,
\par 41 Zeggende: Wat wilt gij, dat Ik u doen zal? En hij zeide: Heere! dat ik ziende mag worden.
\par 42 En Jezus zeide tot hem: Word ziende; uw geloof heeft u behouden.
\par 43 En terstond werd hij ziende, en volgde Hem, God verheerlijkende. En al het volk, dat ziende, gaf Gode lof.

\chapter{19}

\par 1 En Jezus, ingekomen zijnde, ging door Jericho.
\par 2 En zie, er was een man, met name geheten Zacheus; en deze was een overste der tollenaren, en hij was rijk;
\par 3 En zocht Jezus te zien, wie Hij was; en kon niet vanwege de schare, omdat hij klein van persoon was.
\par 4 En vooruitlopende, klom hij op een wilden vijgeboom, opdat hij Hem mocht zien; want Hij zou door dien weg voorbijgaan.
\par 5 En als Jezus aan die plaats kwam, opwaarts ziende, zag Hij hem, en zeide tot hem: Zacheus! haast u, en kom af; want Ik moet heden in uw huis blijven.
\par 6 En hij haastte zich en kwam af, en ontving Hem met blijdschap.
\par 7 En allen, die het zagen, murmureerden, zeggende: Hij is tot een zondigen man ingegaan, om te herbergen.
\par 8 En Zacheus stond, en zeide tot den Heere: Zie, de helft van mijn goederen, Heere, geef ik den armen; en indien ik iemand iets door bedrog ontvreemd heb, dat geef ik vierdubbel weder.
\par 9 En Jezus zeide tot hem: Heden is dezen huize zaligheid geschied, nademaal ook deze een zoon van Abraham is.
\par 10 Want de Zoon des mensen is gekomen, om te zoeken en zalig te maken, dat verloren was.
\par 11 En als zij dat hoorden, voegde Hij daarbij, en zeide een gelijkenis; omdat Hij nabij Jeruzalem was, en omdat zij meenden, dat het Koninkrijk Gods terstond zou openbaar worden.
\par 12 Hij zeide dan: Een zeker welgeboren man reisde in een ver gelegen land, om voor zichzelven een koninkrijk te ontvangen, en dan weder te keren.
\par 13 En geroepen hebbende zijn tien dienstknechten, gaf hij hun tien ponden, en zeide tot hen: Doet handeling, totdat ik kome.
\par 14 En zijn burgers haatten hem, en zonden hem gezanten na, zeggende: Wij willen niet, dat deze over ons koning zij.
\par 15 En het geschiedde, toen hij wederkwam, als hij het koninkrijk ontvangen had, dat hij zeide, dat die dienstknechten tot hem zouden geroepen worden, wien hij het geld gegeven had; opdat hij weten mocht, wat een iegelijk met handelen gewonnen had.
\par 16 En de eerste kwam, en zeide: Heer, uw pond heeft tien ponden daartoe gewonnen.
\par 17 En hij zeide tot hem: Wel, gij goede dienstknecht, dewijl gij in het minste getrouw zijt geweest, zo heb macht over tien steden.
\par 18 En de tweede kwam, en zeide: Heer, uw pond heeft vijf ponden gewonnen.
\par 19 En hij zeide ook tot dezen: En gij, wees over vijf steden.
\par 20 En een ander kwam, zeggende: Heer, zie hier uw pond, hetwelk ik in een zweetdoek weggelegd had;
\par 21 Want ik vreesde u, omdat gij een straf mens zijt; gij neemt weg, wat gij niet gelegd hebt, en gij maait, wat gij niet gezaaid hebt.
\par 22 Maar hij zeide tot hem: Uit uw mond zal ik u oordelen, gij boze dienstknecht! Gij wist, dat ik een straf mens ben, nemende weg, wat ik niet gelegd heb, en maaiende, wat ik niet gezaaid heb.
\par 23 Waarom hebt gij dan mijn geld niet in de bank gegeven, en ik, komende, had hetzelve met woeker mogen eisen?
\par 24 En hij zeide tot degenen, die bij hem stonden: Neemt dat pond van hem weg, en geeft het dien, die de tien ponden heeft.
\par 25 En zij zeiden tot hem: Heer, hij heeft tien ponden.
\par 26 Want ik zeg u, dat een iegelijk, die heeft, zal gegeven worden; maar van degene, die niet heeft, van dien zal genomen worden ook wat hij heeft.
\par 27 Doch deze mijn vijanden, die niet hebben gewild, dat ik over hen koning zoude zijn, brengt ze hier, en slaat ze hier voor mij dood.
\par 28 En dit gezegd hebbende, reisde Hij voor hen heen, en ging op naar Jeruzalem.
\par 29 En het geschiedde, als Hij nabij Beth-fage en Bethanie gekomen was, aan den berg, genaamd den Olijfberg, dat Hij twee van Zijn discipelen uitzond,
\par 30 Zeggende: Gaat henen in dat vlek, dat tegenover is; in hetwelk inkomende, zult gij een veulen gebonden vinden, waarop geen mens ooit heeft gezeten; ontbindt hetzelve, en brengt het.
\par 31 En indien iemand u vraagt: Waarom ontbindt gij dat, zo zult gij alzo tot hem zeggen: Omdat het de Heere van node heeft.
\par 32 En die uitgezonden waren, heengegaan zijnde, vonden het, gelijk Hij hun gezegd had.
\par 33 En als zij het veulen ontbonden, zeiden de heren van hetzelve tot hen: Waarom ontbindt gij het veulen?
\par 34 En zij zeiden: De Heere heeft het van node.
\par 35 En zij brachten hetzelve tot Jezus. En hun klederen op het veulen geworpen hebbende, zetten zij Jezus daarop.
\par 36 En als Hij voort reisde, spreidden zij hun klederen onder Hem op den weg.
\par 37 En als Hij nu genaakte aan den afgang des Olijfbergs, begon al de menigte der discipelen zich te verblijden, en God te loven met grote stemme, vanwege al de krachtige daden, die zij gezien hadden;
\par 38 Zeggende: Gezegend is de Koning, Die daar komt in den Naam des Heeren! Vrede zij in den hemel, en heerlijkheid in de hoogste plaatsen!
\par 39 En sommigen der Farizeen uit de schare zeiden tot Hem: Meester, bestraf Uw discipelen.
\par 40 En Hij, antwoordende, zeide tot hen: Ik zeg ulieden, dat, zo deze zwijgen, de stenen haast roepen zullen.
\par 41 En als Hij nabij kwam, en de stad zag, weende Hij over haar,
\par 42 Zeggende: Och, of gij ook bekendet, ook nog in dezen uw dag, hetgeen tot uw vrede dient! Maar nu is het verborgen voor uw ogen.
\par 43 Want er zullen dagen over u komen, dat uw vijanden een begraving rondom u zullen opwerpen, en zullen u omsingelen, en u van alle zijden benauwen;
\par 44 En zullen u tot den grond nederwerpen, en uw kinderen in u; en zij zullen in u den enen steen op den anderen steen niet laten; daarom dat gij den tijd uwer bezoeking niet bekend hebt.
\par 45 En gegaan zijnde in den tempel, begon Hij uit te drijven degenen, die daarin verkochten en kochten,
\par 46 Zeggende tot hen: Er is geschreven: Mijn huis is een huis des gebeds; maar gij hebt dat tot een kuil der moordenaren gemaakt.
\par 47 En Hij leerde dagelijks in den tempel; en de overpriesters, en de Schriftgeleerden, en de oversten des volks zochten Hem te doden.
\par 48 En zij vonden niet, wat zij doen zouden; want al het volk hing Hem aan, en hoorde Hem.

\chapter{20}

\par 1 En het geschiedde in een van die dagen, als Hij in den tempel het volk leerde, en het Evangelie verkondigde, dat de overpriesters, en Schriftgeleerden, met de ouderlingen daarover kwamen,
\par 2 En spraken tot Hem zeggende: Zeg ons, door wat macht Gij deze dingen doet; of wie Hij is, Die U deze macht heeft gegeven?
\par 3 En Hij, antwoordende, zeide tot hen: Ik zal u ook een woord vragen, en zegt Mij:
\par 4 De doop van Johannes, was die uit den Hemel, of uit de mensen?
\par 5 En zij overleiden onder zich, zeggende: Indien wij zeggen: Uit den Hemel; zo zal Hij zeggen: Waarom hebt gij dan hem niet geloofd?
\par 6 En indien wij zeggen: Uit de mensen; zo zal ons al het volk stenigen; want zij houden voor zeker, dat Johannes een profeet was.
\par 7 En zij antwoordden, dat zij niet wisten, vanwaar die was.
\par 8 En Jezus zeide tot hen: Zo zeg Ik u ook niet, door wat macht Ik deze dingen doe.
\par 9 En Hij begon tot het volk deze gelijkenis te zeggen: Een zeker mens plantte een wijngaard, en hij verhuurde dien aan landlieden, en trok een langen tijd buiten 's lands.
\par 10 En als het de tijd was, zond hij tot de landlieden een dienstknecht, opdat zij hem van de vrucht des wijngaards geven zouden; maar de landlieden sloegen denzelven, en zonden hem ledig heen.
\par 11 En wederom zond hij nog een anderen dienstknecht; maar ook dien geslagen en smadelijk behandeld hebbende, zonden zij hem ledig heen.
\par 12 En wederom zond hij nog een derden; maar zij verwondden ook dezen, en wierpen hem uit.
\par 13 En de heer des wijngaards zeide: Wat zal ik doen? Ik zal mijn geliefden zoon zenden; mogelijk dezen ziende, zullen zij hem ontzien.
\par 14 Maar als de landlieden hem zagen, overleiden zij onder elkander, en zeiden: Deze is de erfgenaam; komt, laat ons hem doden, opdat de erfenis onze worde.
\par 15 En als zij hem buiten den wijngaard uitgeworpen hadden, doodden zij hem. Wat zal dan de heer des wijngaards hun doen?
\par 16 Hij zal komen en deze landlieden verderven, en zal den wijngaard aan anderen geven. En als zij dat hoorden, zeiden zij: Dat zij verre!
\par 17 Maar Hij zag hen aan, en zeide: Wat is dan dit, hetwelk geschreven staat: De steen, dien de bouwlieden verworpen hebben, deze is tot een hoofd des hoeks geworden?
\par 18 Een iegelijk, die op dien steen valt, zal verpletterd worden, en op wien hij valt, dien zal hij vermorzelen.
\par 19 En de overpriesteren en de Schriftgeleerden zochten te dierzelver ure de handen aan Hem te slaan; maar zij vreesden het volk; want zij verstonden, dat Hij deze gelijkenis tegen hen gesproken had.
\par 20 En zij namen Hem waar, en zonden verspieders uit, die zichzelven veinsden rechtvaardig te zijn; opdat zij Hem in Zijn rede vangen mochten, om Hem aan de heerschappij en de macht des stadhouders over te leveren.
\par 21 En zij vraagden Hem, zeggende: Meester, wij weten, dat Gij recht spreekt en leert, en den persoon niet aanneemt, maar den weg Gods leert in der waarheid.
\par 22 Is het ons geoorloofd den keizer schatting te geven, of niet?
\par 23 En Hij, hun arglistigheid bemerkende, zeide tot hen: Wat verzoekt gij Mij?
\par 24 Toont Mij een penning; wiens beeld en opschrift heeft hij? En zij, antwoordende, zeiden: Des keizers.
\par 25 En Hij zeide tot hen: Geeft dan den keizer, dat des keizers is, en Gode, dat Gods is.
\par 26 En zij konden Hem in Zijn woord niet vatten voor het volk; en zich verwonderende over Zijn antwoord, zwegen zij stil.
\par 27 En tot Hem kwamen sommigen der Sadduceen, welke tegensprekende zeggen, dat er geen opstanding is, en vraagden Hem,
\par 28 Zeggende: Meester! Mozes heeft ons geschreven: Zo iemands broeder sterft, die een vrouw heeft, en hij sterft zonder kinderen, dat zijn broeder de vrouw nemen zal, en zijn broeder zaad verwekken.
\par 29 Er waren nu zeven broeders; en de eerste nam een vrouw, en hij stierf zonder kinderen.
\par 30 En de tweede nam die vrouw, en ook deze stierf zonder kinderen.
\par 31 En de derde nam dezelve vrouw; en desgelijks ook de zeven, en hebben geen kinderen nagelaten, en zijn gestorven.
\par 32 En ten laatste na allen stierf ook de vrouw.
\par 33 In de opstanding dan, wiens vrouw van dezen zal zij zijn? Want die zeven hebben dezelve tot een vrouw gehad.
\par 34 En Jezus, antwoordende, zeide tot hen: De kinderen dezer eeuw trouwen, en worden ten huwelijk uitgegeven;
\par 35 Maar die waardig zullen geacht zijn die eeuw te verwerven en de opstanding uit de doden, zullen noch trouwen, noch ten huwelijk uitgegeven worden;
\par 36 Want zij kunnen niet meer sterven, want zij zijn den engelen gelijk; en zij zijn kinderen Gods, dewijl zij kinderen der opstanding zijn.
\par 37 En dat de doden opgewekt zullen worden, heeft ook Mozes aangewezen bij het doornenbos, als hij den Heere noemt den God Abrahams, en den God Izaks, en den God Jakobs.
\par 38 God nu is niet een God der doden, maar der levenden; want zij leven Hem allen.
\par 39 En sommigen der Schriftgeleerden, antwoordende, zeiden: Meester! Gij hebt wel gezegd.
\par 40 En zij durfden Hem niet meer iets vragen.
\par 41 En Hij zeide tot hen: Hoe zeggen zij, dat de Christus Davids Zoon is?
\par 42 En David zelf zegt in het boek der psalmen: De Heere heeft gezegd tot mijn Heere: Zit aan Mijn rechter hand,
\par 43 Totdat Ik Uw vijanden zal gezet hebben tot een voetbank Uwer voeten.
\par 44 David dan noemt Hem zijn Heere; en hoe is Hij zijn Zoon?
\par 45 En daar al het volk het hoorde, zeide Hij tot Zijn discipelen:
\par 46 Wacht u van de Schriftgeleerden, die daar willen wandelen in lange klederen, en beminnen de groetingen op de markten, en de voorgestoelten in de synagogen, en de vooraanzittingen in de maaltijden;
\par 47 Die der weduwen huizen opeten, en onder een schijn lange gebeden doen; dezen zullen zwaarder oordeel ontvangen.

\chapter{21}

\par 1 En opziende, zag Hij de rijken hun gaven in de schatkist werpen.
\par 2 En Hij zag ook een zekere arme weduwe twee kleine penningen daarin werpen.
\par 3 En Hij zeide: Waarlijk, Ik zeg u, dat deze arme weduwe meer dan allen heeft ingeworpen.
\par 4 Want die allen hebben van hun overvloed geworpen tot de gaven Gods; maar deze heeft van haar gebrek, al den leeftocht, dien zij had, daarin geworpen.
\par 5 En als sommigen zeiden van den tempel, dat hij met schone stenen en begiftigingen versierd was, zeide Hij:
\par 6 Wat deze dingen aangaat, die gij aanschouwt, er zullen dagen komen, in welke niet een steen op den anderen steen zal gelaten worden, die niet zal worden afgebroken.
\par 7 En zij vraagden Hem, zeggende: Meester, wanneer zullen dan deze dingen zijn, en welk is het teken, wanneer deze dingen zullen geschieden?
\par 8 En Hij zeide: Ziet, dat gij niet verleid wordt; want velen zullen er komen onder Mijn Naam, zeggende: Ik ben de Christus; en de tijd is nabij gekomen, gaat dan hen niet na.
\par 9 En wanneer gij zult horen van oorlogen en beroerten, zo wordt niet verschrikt; want deze dingen moeten eerst geschieden; maar nog is terstond het einde niet.
\par 10 Toen zeide Hij tot hen: Het ene volk zal tegen het andere volk opstaan, en het ene koninkrijk tegen het andere koninkrijk.
\par 11 En er zullen grote aardbevingen wezen in verscheidene plaatsen, en hongersnoden, en pestilentien; er zullen ook schrikkelijke dingen, en grote tekenen van den hemel geschieden.
\par 12 Maar voor dit alles, zullen zij hun handen aan ulieden slaan, en u vervolgen, u overleverende in de synagogen en gevangenissen; en gij zult getrokken worden voor koningen en stadhouders, om Mijns Naams wil.
\par 13 En dit zal u overkomen tot een getuigenis.
\par 14 Neemt dan in uw harten voor, van te voren niet te overdenken, hoe gij u verantwoorden zult;
\par 15 Want Ik zal u mond en wijsheid geven, welke niet zullen kunnen tegenspreken, noch wederstaan allen, die zich tegen u zetten.
\par 16 En gij zult overgeleverd worden ook van ouders, en broeders, en magen, en vrienden; en zij zullen er sommigen uit u doden.
\par 17 En gij zult van allen gehaat worden om Mijns Naams wil.
\par 18 Doch niet een haar uit uw hoofd zal verloren gaan.
\par 19 Bezit uw zielen in uw lijdzaamheid.
\par 20 Maar wanneer gij zien zult, dat Jeruzalem van heirlegers omsingeld wordt, zo weet alsdan, dat haar verwoesting nabij gekomen is.
\par 21 Alsdan die in Judea zijn, dat zij vlieden naar de bergen; en die in het midden van dezelve zijn, dat zij daaruit trekken; en die op de velden zijn, dat zij in dezelve niet komen.
\par 22 Want deze zijn dagen der wraak, opdat alles vervuld worde, dat geschreven is.
\par 23 Doch wee den bevruchten en den zogenden vrouwen in die dagen, want er zal grote nood zijn in het land, en toorn over dit volk.
\par 24 En zij zullen vallen door de scherpte des zwaards, en gevankelijk weggevoerd worden onder alle volken; en Jeruzalem zal van de heidenen vertreden worden, totdat de tijden der heidenen vervuld zullen zijn.
\par 25 En er zullen tekenen zijn in de zon, en maan, en sterren, en op de aarde benauwdheid der volken met twijfelmoedigheid, als de zee en watergolven groot geluid zullen geven;
\par 26 En den mensen het hart zal bezwijken van vrees en verwachting der dingen, die het aardrijk zullen overkomen; want de krachten der hemelen zullen bewogen worden.
\par 27 En alsdan zullen zij den Zoon des mensen zien komen in een wolk, met grote kracht en heerlijkheid.
\par 28 Als nu deze dingen beginnen te geschieden, zo ziet omhoog, en heft uw hoofden opwaarts, omdat uw verlossing nabij is.
\par 29 En Hij zeide tot hen een gelijkenis: Ziet den vijgeboom, en al de bomen.
\par 30 Wanneer zij nu uitspruiten, en gij dat ziet, zo weet gij uit uzelven, dat de zomer nu nabij is.
\par 31 Alzo ook gij, wanneer gij deze dingen zult zien geschieden, zo weet, dat het Koninkrijk Gods nabij is.
\par 32 Voorwaar Ik zeg u, dat dit geslacht geenszins zal voorbijgaan, totdat alles zal geschied zijn.
\par 33 De hemel en de aarde zullen voorbijgaan, maar Mijn woorden zullen geenszins voorbijgaan.
\par 34 En wacht uzelven, dat uw harten niet te eniger tijd bezwaard worden met brasserij en dronkenschap, en zorgvuldigheden dezes levens, en dat u die dag niet onvoorziens over kome.
\par 35 Want gelijk een strik zal hij komen over al degenen, die op den gansen aardbodem gezeten zijn.
\par 36 Waakt dan te aller tijd, biddende, dat gij moogt waardig geacht worden te ontvlieden al deze dingen, die geschieden zullen, en te staan voor den Zoon des mensen.
\par 37 Des daags nu was Hij lerende in den tempel; maar des nachts ging Hij uit, en vernachtte op den berg, genaamd den Olijf berg.
\par 38 En al het volk kwam des morgens vroeg tot Hem in den tempel, om Hem te horen.

\chapter{22}

\par 1 En het feest der ongehevelde broden, genaamd pascha, was nabij.
\par 2 En de overpriesters en de Schriftgeleerden zochten, hoe zij Hem ombrengen zouden; want zij vreesden het volk.
\par 3 En de satan voer in Judas, die toegenaamd was Iskariot, zijnde uit het getal der twaalven.
\par 4 En hij ging heen en sprak met de overpriesters en de hoofdmannen, hoe hij Hem hun zou overleveren.
\par 5 En zij waren verblijd, en zijn het eens geworden, dat zij hem geld geven zouden.
\par 6 En hij beloofde het, en zocht gelegenheid, om Hem hun over te leveren, zonder oproer.
\par 7 En de dag der ongehevelde broden kwam, op denwelken het pascha moest geslacht worden.
\par 8 En Hij zond Petrus en Johannes uit, zeggende: Gaat heen, en bereidt ons het pascha, opdat wij het eten mogen.
\par 9 En zij zeiden tot Hem: Waar wilt Gij, dat wij het bereiden?
\par 10 En Hij zeide tot hen: Ziet, als gij in de stad zult gekomen zijn, zo zal u een mens ontmoeten, dragende een kruik waters; volgt hem in het huis, daar hij ingaat.
\par 11 En gij zult zeggen tot den huisvader van dat huis: De Meester zegt u: Waar is de eetzaal, daar Ik het pascha met Mijn discipelen eten zal?
\par 12 En hij zal u een grote toegeruste opperzaal wijzen, bereidt het aldaar.
\par 13 En zij, heengaande, vonden het, gelijk Hij hun gezegd had, en bereidden het pascha.
\par 14 En als de ure gekomen was, zat Hij aan, en de twaalf apostelen met Hem.
\par 15 En Hij zeide tot hen: Ik heb grotelijks begeerd, dit pascha met u te eten, eer dat Ik lijde;
\par 16 Want Ik zeg u, dat Ik niet meer daarvan eten zal, totdat het vervuld zal zijn in het Koninkrijk Gods.
\par 17 En als Hij een drinkbeker genomen had, en gedankt had, zeide Hij: Neemt dezen, en deelt hem onder ulieden.
\par 18 Want Ik zeg u, dat Ik niet drinken zal van de vrucht des wijnstoks, totdat het Koninkrijk Gods zal gekomen zijn.
\par 19 En Hij nam brood, en als Hij gedankt had, brak Hij het, en gaf het hun, zeggende: Dat is Mijn lichaam, hetwelk voor u gegeven wordt; doet dat tot Mijn gedachtenis.
\par 20 Desgelijks ook den drinkbeker na het avondmaal, zeggende: Deze drinkbeker is het nieuwe testament in Mijn bloed, hetwelk voor u vergoten wordt.
\par 21 Doch ziet, de hand desgenen, die Mij verraadt, is met Mij aan de tafel.
\par 22 En de Zoon des mensen gaat wel heen, gelijk besloten is; doch wee dien mens, door welken Hij verraden wordt!
\par 23 En zij begonnen onder elkander te vragen, wie van hen het toch mocht zijn, die dat doen zou.
\par 24 En er werd ook twisting onder hen, wie van hen scheen de meeste te zijn.
\par 25 En Hij zeide tot hen: De koningen der volken heersen over hen; en die macht over hen hebben, worden weldadige heren genaamd.
\par 26 Doch gij niet alzo; maar de meeste onder u, die zij gelijk de minste, en die voorganger is, als een die dient.
\par 27 Want wie is meerder, die aanzit, of die dient? Is het niet die aanzit? Maar Ik ben in het midden van u, als een die dient.
\par 28 En gij zijt degenen, die met Mij steeds gebleven zijt in Mijn verzoekingen.
\par 29 En Ik verordineer u het Koninkrijk, gelijkerwijs Mijn Vader dat Mij verordineerd heeft;
\par 30 Opdat gij eet en drinkt aan Mijn tafel in Mijn Koninkrijk, en zit op tronen, oordelende de twaalf geslachten Israels.
\par 31 En de Heere zeide: Simon, Simon, ziet, de satan heeft ulieden zeer begeerd om te ziften als de tarwe;
\par 32 Maar Ik heb voor u gebeden, dat uw geloof niet ophoude; en gij, als gij eens zult bekeerd zijn, zo versterk uw broeders.
\par 33 En hij zeide tot Hem: Heere, ik ben bereid, met U ook in de gevangenis en in den dood te gaan.
\par 34 Maar Hij zeide: Ik zeg u, Petrus, de haan zal heden niet kraaien, eer gij driemaal zult verloochend hebben, dat gij Mij kent.
\par 35 En Hij zeide tot hen: Als Ik u uitzond, zonder buidel, en male, en schoenen, heeft u ook iets ontbroken? En zij zeiden: Niets.
\par 36 Hij zeide dan tot hen: Maar nu, wie een buidel heeft, die neme hem, desgelijks ook een male; en die geen heeft, die verkope zijn kleed, en kope een zwaard.
\par 37 Want Ik zeg u, dat nog dit, hetwelk geschreven is, in Mij moet volbracht worden, namelijk: En Hij is met de misdadigen gerekend. Want ook die dingen, die van Mij geschreven zijn, hebben een einde.
\par 38 En zij zeiden: Heere! zie hier twee zwaarden. En Hij zeide tot hen: Het is genoeg.
\par 39 En uitgaande, vertrok Hij, gelijk Hij gewoon was, naar den Olijfberg; en Hem volgden ook Zijn discipelen.
\par 40 En als Hij aan die plaats gekomen was, zeide Hij tot hen: Bidt, dat gij niet in verzoeking komt.
\par 41 En Hij scheidde Zich van hen af, omtrent een steenworp; en knielde neder en bad,
\par 42 Zeggende: Vader, of Gij wildet dezen drinkbeker van Mij wegnemen, doch niet Mijn wil, maar de Uwe geschiede.
\par 43 En van Hem werd gezien een engel uit den hemel, die Hem versterkte.
\par 44 En in zwaren strijd zijnde, bad Hij te ernstiger. En zijn zweet werd gelijk grote droppelen bloeds, die op de aarde afliepen.
\par 45 En als Hij van het gebed opgestaan was, kwam Hij tot Zijn discipelen, en vond hen slapende van droefheid.
\par 46 En Hij zeide tot hen: Wat slaapt gij? Staat op en bidt, opdat gij niet in verzoeking komt.
\par 47 En als Hij nog sprak, ziet daar een schare; en een van de twaalven, die genaamd was Judas, ging hun voor, en kwam bij Jezus, om Hem te kussen.
\par 48 En Jezus zeide tot hem: Judas, verraadt gij den Zoon des mensen met een kus?
\par 49 En die bij Hem waren, ziende, wat er geschieden zou, zeiden tot Hem: Heere, zullen wij met het zwaard slaan?
\par 50 En een uit hen sloeg den dienstknecht des hogepriesters, en hieuw hem zijn rechteroor af.
\par 51 En Jezus, antwoordende, zeide: Laat hen tot hiertoe geworden; en raakte zijn oor aan, en heelde hem.
\par 52 En Jezus zeide tot de overpriesters, en de hoofdmannen des tempels, en ouderlingen, die tegen Hem gekomen waren: Zijt gij uitgegaan met zwaarden en stokken als tegen een moordenaar?
\par 53 Als Ik dagelijks met u was in den tempel, zo hebt gij de handen tegen Mij niet uitgestoken; maar dit is uw ure, en de macht der duisternis.
\par 54 En zij grepen Hem en leidden Hem weg, en brachten Hem in het huis des hogepriesters. En Petrus volgde van verre.
\par 55 En als zij vuur ontstoken hadden in het midden van de zaal, en zij te zamen nederzaten, zat Petrus in het midden van hen.
\par 56 En een zekere dienstmaagd, ziende hem bij het vuur zitten, en haar ogen op hem houdende, zeide: Ook deze was met Hem.
\par 57 Maar hij verloochende Hem, zeggende: Vrouw, ik ken Hem niet.
\par 58 En kort daarna een ander, hem ziende, zeide: Ook gij zijt van die. Maar Petrus zeide: Mens, ik ben niet.
\par 59 En als het omtrent een uur geleden was, bevestigde dat een ander, zeggende: In der waarheid, ook deze was met Hem; want hij is ook een Galileer.
\par 60 Maar Petrus zeide: Mens, ik weet niet, wat gij zegt. En terstond, als hij nog sprak, kraaide de haan.
\par 61 En de Heere, Zich omkerende, zag Petrus aan; en Petrus werd indachtig het woord des Heeren, hoe Hij hem gezegd had: Eer de haan zal gekraaid hebben, zult gij Mij driemaal verloochenen.
\par 62 En Petrus, naar buiten gaande, weende bitterlijk.
\par 63 En de mannen, die Jezus hielden, bespotten Hem, en sloegen Hem.
\par 64 En als zij Hem overdekt hadden, sloegen zij Hem op het aangezicht, en vraagden Hem, zeggende: Profeteer, wie het is, die U geslagen heeft?
\par 65 En vele andere dingen zeiden zij tegen Hem, lasterende.
\par 66 En als het dag geworden was, vergaderden de ouderlingen des volks, en de overpriesters en Schriftgeleerden, en brachten Hem in hun raad,
\par 67 Zeggende: Zijt Gij de Christus, zeg het ons. En Hij zeide tot hen: Indien Ik het u zeg, gij zult het niet geloven;
\par 68 En indien Ik ook vraag, gij zult Mij niet antwoorden, of loslaten;
\par 69 Van nu aan zal de Zoon des mensen gezeten zijn aan de rechter hand der kracht Gods.
\par 70 En zij zeiden allen: Zijt Gij dan de Zoon Gods? En Hij zeide tot hen: Gij zegt, dat Ik het ben.
\par 71 En zij zeiden: Wat hebben wij nog getuigenis van node? Want wij zelven hebben het uit Zijn mond gehoord.

\chapter{23}

\par 1 En de gehele menigte van hen stond op, en leidde Hem tot Pilatus.
\par 2 En zij begonnen Hem te beschuldigen, zeggende: Wij hebben bevonden, dat Deze het volk verkeert, en verbiedt den keizer schattingen te geven, zeggende, dat Hij Zelf Christus, de Koning is.
\par 3 En Pilatus vraagde Hem, zeggende: Zijt Gij de Koning der Joden? En Hij antwoordde hem en zeide: Gij zegt het.
\par 4 En Pilatus zeide tot de overpriesters en de scharen: Ik vind geen schuld in dezen Mens.
\par 5 En zij hielden te sterker aan, zeggende: Hij beroert het volk, lerende door geheel Judea, begonnen hebbende van Galilea tot hier toe.
\par 6 Als nu Pilatus van Galilea hoorde, vraagde hij, of die Mens een Galileer was?
\par 7 En verstaande, dat Hij uit het gebied van Herodes was, zond hij Hem heen tot Herodes, die ook zelf in die dagen binnen Jeruzalem was.
\par 8 En als Herodes Jezus zag, werd hij zeer verblijd; want hij was van over lang begerig geweest Hem te zien, omdat hij veel van Hem hoorde; en hoopte enig teken te zien, dat van Hem gedaan zou worden.
\par 9 En hij vraagde Hem met vele woorden; doch Hij antwoordde hem niets.
\par 10 En de overpriesters en de Schriftgeleerden stonden, en beschuldigden Hem heftiglijk.
\par 11 En Herodes met zijn krijgslieden Hem veracht en bespot hebbende, deed Hem een blinkend kleed aan, en zond Hem weder tot Pilatus.
\par 12 En op denzelfde dag werden Pilatus en Herodes vrienden met elkander; want zij waren te voren in vijandschap tegen den anderen.
\par 13 En als Pilatus de overpriesters, en de oversten, en het volk bijeengeroepen had, zeide hij tot hen:
\par 14 Gij hebt dezen Mens tot mij gebracht, als een, die het volk afkerig maakt; en ziet, ik heb Hem in uw tegenwoordigheid ondervraagd, en heb in dezen Mens geen schuld gevonden, van hetgeen daar gij Hem mede beschuldigt;
\par 15 Ja, ook Herodes niet; want ik heb ulieden tot hem gezonden, en ziet, er is van Hem niets gedaan, dat des doods waardig is.
\par 16 Zo zal ik Hem dan kastijden en loslaten.
\par 17 En hij moest hun op het feest een loslaten.
\par 18 Doch al de menigte riep gelijkelijk, zeggende: Weg met Dezen, en laat ons Bar-abbas los.
\par 19 Dewelke was om zeker oproer, dat in de stad geschied was, en om een doodslag, in de gevangenis geworpen.
\par 20 Pilatus dan riep hun wederom toe, willende Jezus loslaten.
\par 21 Maar zij riepen daartegen, zeggende: Kruis Hem, kruis Hem!
\par 22 En hij zeide ten derden male tot hen: Wat heeft Deze dan kwaads gedaan? Ik heb geen schuld des doods in Hem gevonden. Zo zal ik Hem dan kastijden en loslaten.
\par 23 Maar zij hielden aan met groot geroep, eisende, dat Hij zou gekruist worden; en hun en der overpriesteren geroep werd geweldiger.
\par 24 En Pilatus oordeelde, dat hun eis geschieden zou.
\par 25 En hij liet hun los dengene, die om oproer en doodslag in de gevangenis geworpen was, welken zij geeist hadden; maar Jezus gaf hij over tot hun wil.
\par 26 En als zij Hem wegleidden, namen zij een Simon van Cyrene, komende van den akker, en legden hem het kruis op, dat hij het achter Jezus droeg.
\par 27 En een grote menigte van volk en van vrouwen volgde Hem, welke ook weenden en Hem beklaagden.
\par 28 En Jezus, Zich tot haar kerende, zeide: Gij dochters van Jeruzalem! weent niet over Mij, maar weent over uzelven, en over uw kinderen.
\par 29 Want ziet, er komen dagen, in welke men zeggen zal: Zalig zijn de onvruchtbaren, en de buiken, die niet gebaard hebben, en de borsten, die niet gezoogd hebben.
\par 30 Alsdan zullen zij beginnen te zeggen tot de bergen: Valt op ons; en tot de heuvelen: Bedekt ons.
\par 31 Want indien zij dit doen aan het groene hout, wat zal aan het dorre geschieden?
\par 32 En er werden ook twee anderen, zijnde kwaaddoeners, geleid, om met Hem gedood te worden.
\par 33 En toen zij kwamen op de plaats, genaamd Hoofdschedel plaats, kruisigden zij Hem aldaar, en de kwaaddoeners, den een ter rechter zijde en den ander ter linker zijde.
\par 34 En Jezus zeide: Vader, vergeef het hun; want zij weten niet, wat zij doen. En verdelende Zijn klederen, wierpen zij het lot.
\par 35 En het volk stond en zag het aan. En ook de oversten met hen beschimpten Hem, zeggende: Anderen heeft Hij verlost, dat Hij nu Zichzelven verlosse, zo Hij is de Christus, de Uitverkorene Gods.
\par 36 En ook de krijgsknechten, tot Hem komende, bespotten Hem, en brachten Hem edik;
\par 37 En zeiden: Indien gij de Koning der Joden zijt, zo verlos Uzelven.
\par 38 En er was ook een opschrift boven Hem geschreven, met Griekse, en Romeinse en Hebreeuwse letters: DEZE IS DE KONING DER JODEN.
\par 39 En een der kwaaddoeners, die gehangen waren, lasterde Hem, zeggende: Indien Gij de Christus zijt, verlos Uzelven en ons.
\par 40 Maar de andere, antwoordende, bestrafte hem, zeggende: Vreest gij ook God niet, daar gij in hetzelfde oordeel zijt?
\par 41 En wij toch rechtvaardiglijk; want wij ontvangen straf, waardig hetgeen wij gedaan hebben; maar Deze heeft niets onbehoorlijks gedaan.
\par 42 En hij zeide tot Jezus: Heere, gedenk mijner, als Gij in Uw Koninkrijk zult gekomen zijn.
\par 43 En Jezus zeide tot hem: Voorwaar, zeg Ik u: Heden zult gij met Mij in het Paradijs zijn.
\par 44 En het was omtrent de zesde ure, en er werd duisternis over de gehele aarde, tot de negende ure toe.
\par 45 En de zon werd verduisterd, en het voorhangsel des tempels scheurde midden door.
\par 46 En Jezus, roepende met grote stemme, zeide: Vader, in Uw handen beveel Ik Mijn geest. En als Hij dat gezegd had, gaf Hij den geest.
\par 47 Als nu de hoofdman over honderd zag, wat er geschied was, verheerlijkte hij God, en zeide: Waarlijk, deze Mens was rechtvaardig.
\par 48 En al de scharen, die samengekomen waren om dit te aanschouwen, ziende de dingen, die geschied waren, keerden wederom, slaande op hun borsten.
\par 49 En al Zijn bekenden stonden van verre, ook de vrouwen, die Hem te zamen gevolgd waren van Galilea, en zagen dit aan.
\par 50 En zie, een man, met name Jozef, zijnde een raadsheer, een goed en rechtvaardig man,
\par 51 (Deze had niet mede bewilligd in hun raad en handel) van Arimathea, een stad der Joden, en die ook zelf het Koninkrijk Gods verwachtte;
\par 52 Deze ging tot Pilatus, en begeerde het lichaam van Jezus.
\par 53 En als hij hetzelve afgenomen had, wond hij dat in een fijn lijnwaad, en legde het in een graf, in een rots gehouwen, waarin nog nooit iemand gelegd was.
\par 54 En het was de dag der voorbereiding, en de sabbat kwam aan.
\par 55 En ook de vrouwen, die met Hem gekomen waren uit Galilea, volgden na en aanschouwden het graf, en hoe Zijn lichaam gelegd werd.
\par 56 En wedergekeerd zijnde, bereidden zij specerijen en zalven; en op den sabbat rustten zij naar het gebod.

\chapter{24}

\par 1 En op den eersten dag der week, zeer vroeg in den morgenstond, gingen zij naar het graf, dragende de specerijen, die zij bereid hadden, en sommigen met haar.
\par 2 En zij vonden den steen afgewenteld van het graf.
\par 3 En ingegaan zijnde, vonden zij het lichaam van den Heere Jezus niet.
\par 4 En het geschiedde, als zij daarover twijfelmoedig waren, zie, twee mannen stonden bij haar in blinkende klederen.
\par 5 En als zij zeer bevreesd werden, en het aangezicht naar de aarde neigden, zeiden zij tot haar: Wat zoekt gij den Levende bij de doden?
\par 6 Hij is hier niet, maar Hij is opgestaan. Gedenkt, hoe Hij tot u gesproken heeft, als Hij nog in Galilea was,
\par 7 Zeggende: De Zoon des mensen moet overgeleverd worden in de handen der zondige mensen, en gekruisigd worden, en ten derden dage wederopstaan.
\par 8 En zij werden indachtig Zijner woorden.
\par 9 En wedergekeerd zijnde van het graf, boodschapten zij al deze dingen aan de elven, en aan al de anderen.
\par 10 En deze waren Maria Magdalena, en Johanna, en Maria, de moeder van Jakobus, en de andere met haar, die dit tot de apostelen zeiden.
\par 11 En haar woorden schenen voor hen als ijdel geklap, en zij geloofden haar niet.
\par 12 Doch Petrus opstaande, liep tot het graf, en nederbukkende, zag hij de linnen doeken, liggende alleen, en ging weg, zich verwonderende bij zichzelven van hetgeen geschied was.
\par 13 En zie, twee van hen gingen op denzelfden dag naar een vlek, dat zestig stadien van Jeruzalem was, welks naam was Emmaus;
\par 14 En zij spraken samen onder elkander van al deze dingen, die er gebeurd waren.
\par 15 En het geschiedde, terwijl zij samen spraken, en elkander ondervraagden, dat Jezus Zelf bij hen kwam, en met hen ging.
\par 16 En hun ogen werden gehouden, dat zij Hem niet kenden.
\par 17 En Hij zeide tot hen: Wat redenen zijn dit, die gij, wandelende, onder elkander verhandelt, en waarom ziet gij droevig?
\par 18 En de een, wiens naam was Kleopas, antwoordende, zeide tot Hem: Zijt Gij alleen een vreemdeling te Jeruzalem, en weet niet de dingen, die deze dagen daarin geschied zijn?
\par 19 En Hij zeide tot hen: Welke? En zij zeiden tot Hem: De dingen aangaande Jezus den Nazarener, Welke een Profeet was, krachtig in werken en woorden, voor God en al het volk.
\par 20 En hoe onze overpriesters en oversten Denzelven overgeleverd hebben tot het oordeel des doods, en Hem gekruisigd hebben.
\par 21 En wij hoopten, dat Hij was Degene, Die Israel verlossen zou. Doch ook, benevens dit alles, is het heden de derde dag, van dat deze dingen geschied zijn.
\par 22 Maar ook sommige vrouwen uit ons hebben ons ontsteld, die vroeg in den morgenstond aan het graf geweest zijn;
\par 23 En Zijn lichaam niet vindende, kwamen zij en zeiden, dat zij ook een gezicht van engelen gezien hadden, die zeggen, dat Hij leeft.
\par 24 En sommigen dergenen, die met ons zijn, gingen heen tot het graf, en bevonden het alzo, gelijk ook de vrouwen gezegd hadden; maar Hem zagen zij niet.
\par 25 En Hij zeide tot hen: O onverstandigen en tragen van hart, om te geloven al hetgeen de profeten gesproken hebben!
\par 26 Moest de Christus niet deze dingen lijden, en alzo in Zijn heerlijkheid ingaan?
\par 27 En begonnen hebbende van Mozes en van al de profeten, legde Hij hun uit, in al de Schriften, hetgeen van Hem geschreven was.
\par 28 En zij kwamen nabij het vlek, daar zij naar toegingen; en Hij hield Zich, alsof Hij verder gaan zou.
\par 29 En zij dwongen Hem, zeggende: Blijf met ons; want het is bij den avond, en de dag is gedaald. En Hij ging in, om met hen te blijven.
\par 30 En het geschiedde, als Hij met hen aanzat, nam Hij het brood, en zegende het, en als Hij het gebroken had, gaf Hij het hun.
\par 31 En hun ogen werden geopend, en zij kenden Hem; en Hij kwam weg uit hun gezicht.
\par 32 En zij zeiden tot elkander: Was ons hart niet brandende in ons, als Hij tot ons sprak op den weg, en als Hij ons de Schriften opende?
\par 33 En zij, opstaande ter zelfder ure, keerden weder naar Jeruzalem, en vonden de elven samenvergaderd, en die met hen waren;
\par 34 Welke zeiden: De Heere is waarlijk opgestaan, en is van Simon gezien.
\par 35 En zij vertelden, hetgeen op den weg geschied was, en hoe Hij hun bekend was geworden in het breken des broods.
\par 36 En als zij van deze dingen spraken, stond Jezus Zelf in het midden van hen, en zeide tot hen: Vrede zij ulieden!
\par 37 En zij verschrikt en zeer bevreesd geworden zijnde, meenden, dat zij een geest zagen.
\par 38 En Hij zeide tot hen: Wat zijt gij ontroerd, en waarom klimmen zulke overleggingen in uw harten?
\par 39 Ziet Mijn handen en Mijn voeten; want Ik ben het Zelf; tast Mij aan, en ziet; want een geest heeft geen vlees en benen, gelijk gij ziet, dat Ik heb.
\par 40 En als Hij dit zeide, toonde Hij hun de handen en de voeten.
\par 41 En toen zij het van blijdschap nog niet geloofden, en zich verwonderden, zeide Hij tot hen: Hebt gij hier iets om te eten?
\par 42 En zij gaven Hem een stuk van een gebraden vis, en van honigraten.
\par 43 En Hij nam het, en at het voor hun ogen.
\par 44 En Hij zeide tot hen: Dit zijn de woorden, die Ik tot u sprak, als Ik nog met u was, namelijk dat het alles moest vervuld worden, wat van Mij geschreven is in de Wet van Mozes, en de Profeten, en Psalmen.
\par 45 Toen opende Hij hun verstand, opdat zij de Schriften verstonden.
\par 46 En zeide tot hen: Alzo is er geschreven, en alzo moest de Christus lijden, en van de doden opstaan ten derden dage.
\par 47 En in Zijn Naam gepredikt worden bekering en vergeving der zonden, onder alle volken, beginnende van Jeruzalem.
\par 48 En gij zijt getuigen van deze dingen.
\par 49 En ziet, Ik zende de belofte Mijns Vaders op u; maar blijft gij in de stad Jeruzalem, totdat gij zult aangedaan zijn met kracht uit de hoogte.
\par 50 En Hij leidde hen buiten tot aan Bethanie, en Zijn handen opheffende, zegende Hij hen.
\par 51 En het geschiedde, als Hij hen zegende, dat Hij van hen scheidde, en werd opgenomen in den hemel.
\par 52 En zij aanbaden Hem, en keerden weder naar Jeruzalem met grote blijdschap.
\par 53 En zij waren allen tijd in den tempel, lovende en dankende God. Amen.




\end{document}