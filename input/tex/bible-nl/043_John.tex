\begin{document}

\title{John}



\chapter{1}

\par 1 In den beginne was het Woord, en het Woord was bij God, en het Woord was God.
\par 2 Dit was in den beginne bij God.
\par 3 Alle dingen zijn door Hetzelve gemaakt, en zonder Hetzelve is geen ding gemaakt, dat gemaakt is.
\par 4 In Hetzelve was het Leven, en het Leven was het Licht der mensen.
\par 5 En het Licht schijnt in de duisternis, en de duisternis heeft hetzelve niet begrepen.
\par 6 Er was een mens van God gezonden, wiens naam was Johannes.
\par 7 Deze kwam tot een getuigenis, om van het Licht te getuigen, opdat zij allen door hem geloven zouden.
\par 8 Hij was het Licht niet, maar was gezonden, opdat hij van het Licht getuigen zou.
\par 9 Dit was het waarachtige Licht, Hetwelk verlicht een iegelijk mens, komende in de wereld.
\par 10 Hij was in de wereld, en de wereld is door Hem gemaakt; en de wereld heeft Hem niet gekend.
\par 11 Hij is gekomen tot het Zijne, en de Zijnen hebben Hem niet aangenomen.
\par 12 Maar zovelen Hem aangenomen hebben, dien heeft Hij macht gegeven kinderen Gods te worden, namelijk die in Zijn Naam geloven;
\par 13 Welke niet uit den bloede, noch uit den wil des vleses, noch uit den wil des mans, maar uit God geboren zijn.
\par 14 En het Woord is vlees geworden, en heeft onder ons gewoond (en wij hebben Zijn heerlijkheid aanschouwd, een heerlijkheid als des Eniggeborenen van den Vader), vol van genade en waarheid.
\par 15 Johannes getuigt van Hem, en heeft geroepen, zeggende: Deze was het, van Welken ik zeide: Die na mij komt, is voor mij geworden, want Hij was eer dan ik.
\par 16 En uit Zijn volheid hebben wij allen ontvangen, ook genade voor genade.
\par 17 Want de wet is door Mozes gegeven, de genade en de waarheid is door Jezus Christus geworden.
\par 18 Niemand heeft ooit God gezien; de eniggeboren Zoon, Die in den schoot des Vaders is, Die heeft Hem ons verklaard.
\par 19 En dit is de getuigenis van Johannes, toen de Joden enige priesters en Levieten afzonden van Jeruzalem, opdat zij hem zouden vragen: Wie zijt gij?
\par 20 En hij beleed en loochende het niet; en beleed: Ik ben de Christus niet.
\par 21 En zij vraagden hem: Wat dan? Zijt gij Elias? En hij zeide: Ik ben die niet. Zijt gij de profeet? En hij antwoordde: Neen.
\par 22 Zij zeiden dan tot hem: Wie zijt gij? opdat wij antwoord geven mogen dengenen, die ons gezonden hebben; wat zegt gij van uzelven?
\par 23 Hij zeide: Ik ben de stem des roependen in de woestijn: Maakt den weg des Heeren recht, gelijk Jesaja, de profeet, gesproken heeft.
\par 24 En de afgezondenen waren uit de Farizeen;
\par 25 En zij vraagden hem en spraken tot hem: Waarom doopt gij dan, zo gij de Christus niet zijt, noch Elias, noch de profeet?
\par 26 Johannes antwoordde hun, zeggende: Ik doop met water, maar Hij staat midden onder ulieden, Dien gij niet kent;
\par 27 Dezelve is het, Die na mij komt, Welke voor mij geworden is, Wien ik niet waardig ben, dat ik Zijn schoenriem zou ontbinden.
\par 28 Deze dingen zijn geschied in Bethabara, over de Jordaan, waar Johannes was dopende.
\par 29 Des anderen daags zag Johannes Jezus tot zich komende, en zeide: Zie het Lam Gods, Dat de zonde der wereld wegneemt!
\par 30 Deze is het, van Welken ik gezegd heb: Na mij komt een Man, Die voor mij geworden is, want Hij was eer dan ik.
\par 31 En ik kende Hem niet; maar opdat Hij aan Israel zou geopenbaard worden, daarom ben ik gekomen, dopende met het water.
\par 32 En Johannes getuigde, zeggende: Ik heb den Geest zien nederdalen uit den hemel, gelijk een duif, en bleef op Hem.
\par 33 En ik kende Hem niet; maar Die mij gezonden heeft, om te dopen met water, Die had mij gezegd: Op Welken gij den Geest zult zien nederdalen, en op Hem blijven, Deze is het, Die met den Heiligen Geest doopt.
\par 34 En ik heb gezien, en heb getuigd, dat Deze de Zoon van God is.
\par 35 Des anderen daags wederom stond Johannes, en twee uit zijn discipelen.
\par 36 En ziende op Jezus, daar wandelende, zeide hij: Ziet, het Lam Gods!
\par 37 En die twee discipelen hoorden hem dat spreken, en zij volgden Jezus.
\par 38 En Jezus Zich omkerende, en ziende hen volgen, zeide tot hen:
\par 39 Wat zoekt gij? En zij zeiden tot Hem: Rabbi! (hetwelk is te zeggen, overgezet zijnde, Meester) waar woont Gij?
\par 40 Hij zeide tot hen: Komt en ziet! Zij kwamen en zagen, waar Hij woonde, en bleven dien dag bij Hem. En het was omtrent de tiende ure.
\par 41 Andreas, de broeder van Simon Petrus, was een van de twee, die het van Johannes gehoord hadden, en Hem gevolgd waren.
\par 42 Deze vond eerst zijn broeder Simon, en zeide tot hem: Wij hebben gevonden den Messias, hetwelk is, overgezet zijnde, de Christus.
\par 43 En hij leidde hem tot Jezus. En Jezus, hem aanziende, zeide: Gij zijt Simon, de zoon van Jonas; gij zult genaamd worden Cefas, hetwelk overgezet wordt Petrus.
\par 44 Des anderen daags wilde Jezus heengaan naar Galilea, en vond Filippus, en zeide tot hem: Volg Mij.
\par 45 Filippus nu was van Bethsaida, uit de stad van Andreas en Petrus.
\par 46 Filippus vond Nathanael en zeide tot hem: Wij hebben Dien gevonden, van Welken Mozes in de wet geschreven heeft, en de profeten, namelijk Jezus, den zoon van Jozef, van Nazareth.
\par 47 En Nathanael zeide tot hem: Kan uit Nazareth iets goeds zijn? Filippus zeide van hem: Kom en zie.
\par 48 Jezus zag Nathanael tot Zich komen, en zeide van hem: Zie, waarlijk een Israeliet, in welken geen bedrog is.
\par 49 Nathanael zeide tot Hem: Van waar kent Gij mij? Jezus antwoordde en zeide tot hem: Eer u Filippus riep, daar gij onder den vijgeboom waart, zag Ik u.
\par 50 Nathanael antwoordde en zeide tot Hem: Rabbi! Gij zijt de Zone Gods, Gij zijt de Koning Israels.
\par 51 Jezus antwoordde en zeide tot hem: Omdat Ik u gezegd heb: Ik zag u onder de vijgeboom, zo gelooft gij; gij zult grotere dingen zien dan deze.
\par 52 En Hij zeide tot hem: Voorwaar, voorwaar zeg Ik ulieden: Van nu aan zult gij den hemel zien geopend, en de engelen Gods opklimmende en nederdalende op den Zoon des mensen.

\chapter{2}

\par 1 En op den derden dag was er een bruiloft te Kana in Galilea; en de moeder van Jezus was aldaar.
\par 2 En Jezus was ook genood, en Zijn discipelen, tot de bruiloft.
\par 3 En als er wijn ontbrak, zeide de moeder van Jezus tot Hem: Zij hebben geen wijn.
\par 4 Jezus zeide tot haar: Vrouw, wat heb Ik met u te doen? Mijn ure is nog niet gekomen.
\par 5 Zijn moeder zeide tot de dienaars: Zo wat Hij ulieden zal zeggen, doet dat.
\par 6 En aldaar waren zes stenen watervaten gesteld, naar de reiniging der Joden, elk houdende twee of drie metreten.
\par 7 Jezus zeide tot hen: Vult de watervaten met water. En zij vulden ze tot boven toe.
\par 8 En Hij zeide tot hen: Schept nu, en draagt het tot den hofmeester; en zij droegen het.
\par 9 Als nu de hofmeester het water, dat wijn geworden was, geproefd had (en hij wist niet, van waar de wijn was; maar de dienaren, die het water geschept hadden, wisten het), zo riep de hofmeester den bruidegom.
\par 10 En zeide tot hem: Alle man zet eerst den goeden wijn op, en wanneer men wel gedronken heeft, alsdan den minderen; maar gij hebt den goeden wijn tot nu toe bewaard.
\par 11 Dit beginsel der tekenen heeft Jezus gedaan te Kana in Galilea, en heeft Zijn heerlijkheid geopenbaard; en Zijn discipelen geloofden in Hem.
\par 12 Daarna ging Hij af naar Kapernaum, Hij, en Zijn moeder, en Zijn broeders, en Zijn discipelen; en zij bleven aldaar niet vele dagen.
\par 13 En het pascha der Joden was nabij, en Jezus ging op naar Jeruzalem.
\par 14 En Hij vond in den tempel, die ossen, en schapen, en duiven verkochten, en de wisselaars daar zittende.
\par 15 En een gesel van touwtjes gemaakt hebbende, dreef Hij ze allen uit den tempel, ook de schapen en de ossen; en het geld der wisselaren stortte Hij uit, en keerde de tafelen om.
\par 16 En Hij zeide tot degenen, die de duiven verkochten: Neemt deze dingen van hier weg; maakt niet het huis Mijns Vaders tot een huis van koophandel.
\par 17 En Zijn discipelen werden indachtig, dat er geschreven is: De ijver van Uw huis heeft mij verslonden.
\par 18 De Joden antwoordden dan, en zeiden tot Hem: Wat teken toont Gij ons, dat Gij deze dingen doet?
\par 19 Jezus antwoordde en zeide tot hen: Breekt dezen tempel, en in drie dagen zal Ik denzelven oprichten.
\par 20 De Joden zeiden dan: Zes en veertig jaren is over dezen tempel gebouwd, en Gij, zult Gij dien in drie dagen oprichten?
\par 21 Maar Hij zeide dit van den tempel Zijns lichaams.
\par 22 Daarom, als Hij opgestaan was van de doden, werden Zijn discipelen gedachtig, dat Hij dit tot hen gezegd had, en zij geloofden de Schrift, en het woord, dat Jezus gesproken had.
\par 23 En als Hij te Jeruzalem was, op het pascha, in het feest, geloofden velen in Zijn Naam, ziende Zijn tekenen, die Hij deed.
\par 24 Maar Jezus Zelf betrouwde hun Zichzelven niet, omdat Hij hen allen kende,
\par 25 En omdat Hij niet van node had, dat iemand getuigen zou van den mens; want Hij Zelf wist, wat in den mens was.

\chapter{3}

\par 1 En er was een mens uit de Farizeen, wiens naam was Nicodemus, een overste der Joden;
\par 2 Deze kwam des nachts tot Jezus, en zeide tot Hem: Rabbi, wij weten, dat Gij zijt een Leraar van God gekomen; want niemand kan deze tekenen doen, die Gij doet, zo God met hem niet is.
\par 3 Jezus antwoordde en zeide tot hem: Voorwaar, voorwaar zeg Ik u: Tenzij dat iemand wederom geboren worde, hij kan het Koninkrijk Gods niet zien.
\par 4 Nicodemus zeide tot Hem: Hoe kan een mens geboren worden, nu oud zijnde? Kan hij ook andermaal in zijner moeders buik ingaan, en geboren worden?
\par 5 Jezus antwoordde: Voorwaar, voorwaar zeg Ik u: Zo iemand niet geboren wordt uit water en Geest, hij kan in het Koninkrijk Gods niet ingaan.
\par 6 Hetgeen uit het vlees geboren is, dat is vlees; en hetgeen uit den Geest geboren is, dat is geest.
\par 7 Verwonder u niet, dat Ik u gezegd heb: Gijlieden moet wederom geboren worden.
\par 8 De wind blaast, waarheen hij wil, en gij hoort zijn geluid; maar gij weet niet, van waar hij komt, en waar hij heen gaat; alzo is een iegelijk, die uit den Geest geboren is.
\par 9 Nicodemus antwoordde en zeide tot Hem: Hoe kunnen deze dingen geschieden?
\par 10 Jezus antwoordde en zeide tot hem: Zijt gij een leraar van Israel, en weet gij deze dingen niet?
\par 11 Voorwaar, voorwaar zeg Ik u: Wij spreken, wat Wij weten, en getuigen, wat Wij gezien hebben; en gijlieden neemt Onze getuigenis niet aan.
\par 12 Indien Ik ulieden de aardse dingen gezegd heb, en gij niet gelooft, hoe zult gij geloven, indien Ik ulieden de hemelse zou zeggen?
\par 13 En niemand is opgevaren in den hemel, dan Die uit den hemel nedergekomen is, namelijk de Zoon des mensen, Die in den hemel is.
\par 14 En gelijk Mozes de slang in de woestijn verhoogd heeft, alzo moet de Zoon des mensen verhoogd worden;
\par 15 Opdat een iegelijk, die in Hem gelooft, niet verderve, maar het eeuwige leven hebbe.
\par 16 Want alzo lief heeft God de wereld gehad, dat Hij Zijn eniggeboren Zoon gegeven heeft, opdat een iegelijk die in Hem gelooft, niet verderve, maar het eeuwige leven hebbe.
\par 17 Want God heeft Zijn Zoon niet gezonden in de wereld, opdat Hij de wereld veroordelen zou, maar opdat de wereld door Hem zou behouden worden.
\par 18 Die in Hem gelooft, wordt niet veroordeeld, maar die niet gelooft, is alrede veroordeeld, dewijl hij niet heeft geloofd in den Naam des eniggeboren Zoons van God.
\par 19 En dit is het oordeel, dat het licht in de wereld gekomen is, en de mensen hebben de duisternis liever gehad dan het licht; want hun werken waren boos.
\par 20 Want een iegelijk, die kwaad doet, haat het licht, en komt tot het licht niet, opdat zijn werken niet bestraft worden.
\par 21 Maar die de waarheid doet, komt tot het licht, opdat zijn werken openbaar worden, dat zij in God gedaan zijn.
\par 22 Na dezen kwam Jezus en Zijn discipelen in het land van Judea, en onthield Zich aldaar met hen, en doopte.
\par 23 En Johannes doopte ook in Enon bij Salim, dewijl aldaar vele wateren waren; en zij kwamen daar, en werden gedoopt.
\par 24 Want Johannes was nog niet in de gevangenis geworpen.
\par 25 Er rees dan een vraag van enigen uit de discipelen van Johannes met de Joden over de reiniging.
\par 26 En zij kwamen tot Johannes, en zeiden tot hem: Rabbi, Die met u was over de Jordaan, Welken gij getuigenis gaaft, zie, Die doopt, en zij komen allen tot Hem.
\par 27 Johannes antwoordde en zeide: Een mens kan geen ding aannemen, zo het hem uit den hemel niet gegeven zij.
\par 28 Gijzelven zijt mijn getuigen, dat ik gezegd heb: Ik ben de Christus niet; maar dat ik voor Hem heen uitgezonden ben.
\par 29 Die de bruid heeft, is de bruidegom, maar de vriend des bruidegoms, die staat en hem hoort, verblijdt zich met blijdschap om de stem des bruidegoms. Zo is dan deze mijn blijdschap vervuld geworden.
\par 30 Hij moet wassen, maar ik minder worden.
\par 31 Die van boven komt, is boven allen; die uit de aarde is voortgekomen, die is uit de aarde, en spreekt uit de aarde. Die uit den hemel komt, is boven allen.
\par 32 En hetgeen Hij gezien en gehoord heeft, dat getuigt Hij; en Zijn getuigenis neemt niemand aan.
\par 33 Die Zijn getuigenis aangenomen heeft, die heeft verzegeld, dat God waarachtig is.
\par 34 Want Dien God gezonden heeft, Die spreekt de woorden Gods; want God geeft Hem den Geest niet met mate.
\par 35 De Vader heeft den Zoon lief, en heeft alle dingen in Zijn hand gegeven.
\par 36 Die in den Zoon gelooft, die heeft het eeuwige leven; maar die den Zoon ongehoorzaam is, die zal het leven niet zien, maar de toorn Gods blijft op hem.

\chapter{4}

\par 1 Als dan de Heere verstond, dat de Farizeen gehoord hadden, dat Jezus meer discipelen maakte en doopte dan Johannes;
\par 2 (Hoewel Jezus zelf niet doopte, maar Zijn discipelen),
\par 3 Zo verliet Hij Judea, en ging wederom heen naar Galilea.
\par 4 En Hij moest door Samaria gaan.
\par 5 Hij kwam dan in een stad van Samaria, genaamd Sichar, nabij het stuk land, hetwelk Jakob zijn zoon Jozef gaf.
\par 6 En aldaar was de fontein Jakobs. Jezus dan, vermoeid zijnde van de reize, zat alzo neder nevens de fontein. Het was omtrent de zesde ure.
\par 7 Er kwam een vrouw uit Samaria om water te putten. Jezus zeide tot haar: Geef Mij te drinken.
\par 8 (Want Zijn discipelen waren heengegaan in de stad, opdat zij zouden spijze kopen.)
\par 9 Zo zeide dan de Samaritaanse vrouw tot Hem: Hoe begeert Gij, Die een Jood zijt, van mij te drinken, die een Samaritaanse vrouw ben? Want de Joden houden geen gemeenschap met de Samaritanen.
\par 10 Jezus antwoordde en zeide tot haar: Indien gij de gave Gods kendet, en Wie Hij is, Die tot u zegt: Geef Mij te drinken, zo zoudt gij van Hem hebben begeerd, en Hij zoude u levend water gegeven hebben.
\par 11 De vrouw zeide tot Hem: Heere! Gij hebt niet om mede te putten, en de put is diep; van waar hebt Gij dan het levend water?
\par 12 Zijt Gij meerder dan onze vader Jakob, die ons den put gegeven heeft, en hijzelf heeft daaruit gedronken, en zijn kinderen en zijn vee?
\par 13 Jezus antwoordde, en zeide tot haar: Een ieder, die van dit water drinkt, zal wederom dorsten;
\par 14 Maar zo wie gedronken zal hebben van het water, dat Ik hem geven zal, dien zal in eeuwigheid niet dorsten; maar het water, dat Ik hem zal geven, zal in hem worden een fontein van water, springende tot in het eeuwige leven.
\par 15 De vrouw zeide tot Hem: Heere, geef mij dat water, opdat mij niet dorste, en ik hier niet moet komen, om te putten.
\par 16 Jezus zeide tot haar: Ga heen, roep uw man, en kom hier.
\par 17 De vrouw antwoordde en zeide: Ik heb geen man. Jezus zeide tot haar: Gij hebt wel gezegd: Ik heb geen man.
\par 18 Want gij hebt vijf mannen gehad, en dien gij nu hebt, is uw man niet; dat hebt gij met waarheid gezegd.
\par 19 De vrouw zeide tot Hem: Heere, ik zie, dat Gij een profeet zijt.
\par 20 Onze vaders hebben op dezen berg aangebeden; en gijlieden zegt, dat te Jeruzalem de plaats is, waar men moet aanbidden.
\par 21 Jezus zeide tot haar: Vrouw, geloof Mij, de ure komt, wanneer gijlieden, noch op dezen berg, noch te Jeruzalem, den Vader zult aanbidden.
\par 22 Gijlieden aanbidt, wat gij niet weet; wij aanbidden, wat wij weten; want de zaligheid is uit de Joden.
\par 23 Maar de ure komt, en is nu, wanneer de ware aanbidders den Vader aanbidden zullen in geest en waarheid; want de Vader zoekt ook dezulken, die Hem alzo aanbidden.
\par 24 God is een Geest, en die Hem aanbidden, moeten Hem aanbidden in geest en waarheid.
\par 25 De vrouw zeide tot Hem: Ik weet, dat de Messias komt (Die genaamd wordt Christus); wanneer Die zal gekomen zijn, zo zal Hij ons alle dingen verkondigen.
\par 26 Jezus zeide tot haar: Ik ben het, Die met u spreek.
\par 27 En daarop kwamen Zijn discipelen en verwonderden zich, dat Hij met een vrouw sprak. Nochtans zeide niemand: Wat vraagt Gij, of: Wat spreekt Gij met haar?
\par 28 Zo verliet de vrouw dan haar watervat, en ging heen in de stad en zeide tot de lieden:
\par 29 Komt, ziet een Mens, Die mij gezegd heeft alles, wat ik gedaan heb; is Deze niet de Christus?
\par 30 Zij dan gingen uit de stad, en kwamen tot Hem.
\par 31 En ondertussen baden Hem de discipelen, zeggende: Rabbi, eet.
\par 32 Maar Hij zeide tot hen: Ik heb een spijs om te eten, die gij niet weet.
\par 33 Zo zeiden dan de discipelen tegen elkander: Heeft Hem iemand te eten gebracht?
\par 34 Jezus zeide tot hen: Mijn spijs is, dat Ik doe den wil Desgenen, Die Mij gezonden heeft, en Zijn werk volbrenge.
\par 35 Zegt gijlieden niet: Het zijn nog vier maanden, en dan komt de oogst? Ziet, Ik zeg u: Heft uw ogen op en aanschouwt de landen; want zij zijn alrede wit om te oogsten.
\par 36 En die maait, ontvangt loon, en vergadert vrucht ten eeuwigen leven; opdat zich te zamen verblijde, beide, die zaait en die maait.
\par 37 Want hierin is die spreuk waarachtig: Een ander is het, die zaait, en een ander, die maait.
\par 38 Ik heb u uitgezonden, om te maaien, hetgeen gij niet bearbeid hebt; anderen hebben het bearbeid, en gij zijt tot hun arbeid ingegaan.
\par 39 En velen der Samaritanen uit die stad geloofden in Hem, om het woord der vrouw, die getuigde: Hij heeft mij gezegd alles, wat ik gedaan heb.
\par 40 Als dan de Samaritanen tot Hem gekomen waren, baden zij Hem, dat Hij bij hen bleef; en Hij bleef aldaar twee dagen.
\par 41 En er geloofden er veel meer om Zijns woords wil;
\par 42 En zeiden tot de vrouw: Wij geloven niet meer om uws zeggens wil; want wij zelven hebben Hem gehoord, en weten, dat Deze waarlijk is de Christus, de Zaligmaker der wereld.
\par 43 En na de twee dagen ging Hij van daar en ging heen naar Galilea;
\par 44 Want Jezus heeft Zelf getuigd, dat een profeet in zijn eigen vaderland geen eer heeft.
\par 45 Als Hij dan in Galilea kwam, ontvingen Hem de Galileers, gezien hebbende al de dingen, die Hij te Jeruzalem op het feest gedaan had; want ook zij waren tot het feest gegaan.
\par 46 Zo kwam dan Jezus wederom te Kana in Galilea, waar Hij het water wijn gemaakt had. En er was een zeker koninklijk hoveling, wiens zoon krank was, te Kapernaum.
\par 47 Deze, gehoord hebbende, dat Jezus uit Judea in Galilea kwam, ging tot Hem, en bad Hem, dat Hij afkwame, en zijn zoon gezond maakte; want hij lag op zijn sterven.
\par 48 Jezus dan zeide tot hem: Tenzij dat gijlieden tekenen en wonderen ziet, zo zult gij niet geloven.
\par 49 De koninklijke hoveling zeide tot Hem: Heere, kom af, eer mijn kind sterft.
\par 50 Jezus zeide tot hem: Ga heen, uw zoon leeft. En de mens geloofde het woord, dat Jezus tot hem zeide, en ging heen.
\par 51 En als hij nu afging, kwamen hem zijn dienstknechten tegemoet, en boodschapten, zeggende: Uw kind leeft!
\par 52 Zo vraagde hij dan van hen de ure, in welke het beter met hem geworden was. En zij zeiden tot hem: Gisteren te zeven ure verliet hem de koorts.
\par 53 De vader bekende dan, dat het in dezelve ure was, in dewelke Jezus tot hem gezegd had: Uw zoon leeft. En hij geloofde zelf, en zijn gehele huis.
\par 54 Dit tweede teken heeft Jezus wederom gedaan, als Hij uit Judea in Galilea gekomen was.

\chapter{5}

\par 1 Na dezen was een feest der Joden, en Jezus ging op naar Jeruzalem.
\par 2 En er is te Jeruzalem aan de Schaaps poort, een badwater, hetwelk in het Hebreeuws toegenaamd wordt Bethesda, hebbende vijf zalen.
\par 3 In dezelve lag een grote menigte van kranken, blinden, kreupelen, verdorden, wachtende op de roering des waters.
\par 4 Want een engel daalde neder op zekeren tijd in dat badwater, en beroerde het water; die dan eerst daarin kwam, na de beroering van het water, die werd gezond, van wat ziekte hij ook bevangen was.
\par 5 En aldaar was een zeker mens, die acht en dertig jaren krank gelegen had.
\par 6 Jezus, ziende dezen liggen, en wetende, dat hij nu langen tijd gelegen had, zeide tot hem: Wilt gij gezond worden?
\par 7 De kranke antwoordde Hem: Heere, ik heb geen mens, om mij te werpen in het badwater, wanneer het water beroerd wordt; en terwijl ik kom, zo daalt een ander voor mij neder.
\par 8 Jezus zeide tot hem: Sta op, neem uw beddeken op, en wandel.
\par 9 En terstond werd de mens gezond, en nam zijn beddeken op en wandelde. En het was sabbat op denzelven dag.
\par 10 De Joden zeiden dan tot dengene, die genezen was: Het is sabbat; het is u niet geoorloofd het beddeken te dragen.
\par 11 Hij antwoordde hun: Die mij gezond gemaakt heeft, Die heeft mij gezegd: Neem uw beddeken op, en wandel.
\par 12 Zij vraagden hem dan: Wie is de Mens, Die u gezegd heeft: Neem uw beddeken op, en wandel?
\par 13 En die gezond gemaakt was, wist niet, Wie Hij was; want Jezus was ontweken, alzo er een grote schare in die plaats was.
\par 14 Daarna vond hem Jezus in den tempel, en zeide tot hem: Zie, gij zijt gezond geworden; zondig niet meer, opdat u niet wat ergers geschiede.
\par 15 De mens ging heen, en boodschapte den Joden, dat het Jezus was, Die hem gezond gemaakt had.
\par 16 En daarom vervolgden de Joden Jezus, en zochten Hem te doden, omdat Hij deze dingen op den sabbat deed.
\par 17 En Jezus antwoordde hun: Mijn Vader werkt tot nu toe, en Ik werk ook.
\par 18 Daarom zochten dan de Joden te meer Hem te doden, omdat Hij niet alleen den sabbat brak, maar ook zeide, dat God Zijn eigen Vader was, Zichzelven Gode evengelijk makende.
\par 19 Jezus dan antwoordde en zeide tot hen: Voorwaar, voorwaar zeg Ik u: De Zoon kan niets van Zichzelven doen, tenzij Hij den Vader dat ziet doen; want zo wat Die doet, hetzelve doet ook de Zoon desgelijks.
\par 20 Want de Vader heeft den Zoon lief, en toont Hem alles, wat Hij doet; en Hij zal Hem groter werken tonen dan deze, opdat gij u verwondert.
\par 21 Want gelijk de Vader de doden opwekt en levend maakt, alzo maakt ook de Zoon levend, Die Hij wil.
\par 22 Want ook de Vader oordeelt niemand, maar heeft al het oordeel den Zoon gegeven;
\par 23 Opdat zij allen den Zoon eren, gelijk zij den Vader eren. Die den Zoon niet eert, eert den Vader niet, Die Hem gezonden heeft.
\par 24 Voorwaar, voorwaar zeg Ik u: Die Mijn woord hoort, en gelooft Hem, Die Mij gezonden heeft, die heeft het eeuwige leven, en komt niet in de verdoemenis, maar is uit den dood overgegaan in het leven.
\par 25 Voorwaar, voorwaar zeg Ik u: De ure komt, en is nu, wanneer de doden zullen horen de stem des Zoons Gods, en die ze gehoord hebben, zullen leven.
\par 26 Want gelijk de Vader het leven heeft in Zichzelven, alzo heeft Hij ook den Zoon gegeven, het leven te hebben in Zichzelven;
\par 27 En heeft Hem macht gegeven, ook gericht te houden, omdat Hij des mensen Zoon is.
\par 28 Verwondert u daar niet over, want de ure komt, in dewelke allen, die in de graven zijn, Zijn stem zullen horen;
\par 29 En zullen uitgaan, die het goede gedaan hebben, tot de opstanding des levens, en die het kwade gedaan hebben, tot de opstanding der verdoemenis.
\par 30 Ik kan van Mijzelven niets doen. Gelijk Ik hoor, oordeel Ik, en Mijn oordeel is rechtvaardig; want Ik zoek niet Mijn wil, maar den wil des Vaders, Die Mij gezonden heeft.
\par 31 Indien Ik van Mijzelven getuig, Mijn getuigenis is niet waarachtig.
\par 32 Er is een ander, die van Mij getuigt, en Ik weet, dat de getuigenis, welke hij van Mij getuigt, waarachtig is.
\par 33 Gijlieden hebt tot Johannes gezonden, en hij heeft der waarheid getuigenis gegeven.
\par 34 Doch Ik neem geen getuigenis van een mens; maar dit zeg Ik, opdat gijlieden zoudt behouden worden.
\par 35 Hij was een brandende en lichtende kaars; en gij hebt ulieden voor een korten tijd in zijn licht willen verheugen.
\par 36 Maar Ik heb een getuigenis meerder, dan die van Johannes; want de werken, die Mij de Vader gegeven heeft, om die te volbrengen, dezelve werken, die Ik doe, getuigen van Mij, dat Mij de Vader gezonden heeft.
\par 37 En de Vader, Die Mij gezonden heeft, Die heeft Zelf van Mij getuigd. Gij hebt noch Zijn stem ooit gehoord, noch Zijn gedaante gezien.
\par 38 En Zijn woord hebt gij niet in u blijvende; want gij gelooft Dien niet, Dien Hij gezonden heeft.
\par 39 Onderzoekt de Schriften; want gij meent in dezelve het eeuwige leven te hebben; en die zijn het, die van Mij getuigen.
\par 40 En gij wilt tot Mij niet komen, opdat gij het leven moogt hebben.
\par 41 Ik neem geen eer van mensen;
\par 42 Maar Ik ken ulieden, dat gij de liefde Gods in uzelven niet hebt.
\par 43 Ik ben gekomen in den Naam Mijns Vaders, en gij neemt Mij niet aan; zo een ander komt in zijn eigen naam, dien zult gij aannemen.
\par 44 Hoe kunt gij geloven, gij, die eer van elkander neemt, en de eer, die van God alleen is, niet zoekt?
\par 45 Meent niet, dat Ik u verklagen zal bij den Vader; die u verklaagt, is Mozes, op welken gij gehoopt hebt.
\par 46 Want indien gij Mozes geloofdet, zo zoudt gij Mij geloven; want hij heeft van Mij geschreven.
\par 47 Maar zo gij zijn Schriften niet gelooft, hoe zult gij Mijn woorden geloven?

\chapter{6}

\par 1 Na dezen vertrok Jezus over de zee van Galilea, welke is de zee van Tiberias.
\par 2 En Hem volgde een grote schare, omdat zij Zijn tekenen zagen, die Hij deed aan de kranken.
\par 3 En Jezus ging op den berg, en zat aldaar neder met Zijn discipelen.
\par 4 En het pascha, het feest der Joden, was nabij.
\par 5 Jezus dan, de ogen opheffende, en ziende, dat een grote schare tot Hem kwam, zeide tot Filippus: Van waar zullen wij broden kopen, opdat deze eten mogen?
\par 6 (Doch dit zeide Hij, hem beproevende; want Hij wist Zelf, wat Hij doen zou.)
\par 7 Filippus antwoordde Hem: Voor tweehonderd penningen brood is voor dezen niet genoeg, opdat een iegelijk van hen een weinig neme.
\par 8 Een van Zijn discipelen, namelijk Andreas, de broeder van Simon Petrus, zeide tot Hem:
\par 9 Hier is een jongsken, dat vijf gerstebroden heeft, en twee visjes; maar wat zijn deze onder zo velen?
\par 10 En Jezus zeide: Doet de mensen nederzitten. En er was veel gras in die plaats. Zo zaten dan de mannen neder, omtrent vijf duizend in getal.
\par 11 En Jezus nam de broden, en gedankt hebbende, deelde Hij ze den discipelen, en de discipelen dengenen, die nedergezeten waren; desgelijks ook van de visjes, zoveel zij wilden.
\par 12 En als zij verzadigd waren, zeide Hij tot Zijn discipelen: Vergadert de overgeschoten brokken, opdat er niets verloren ga.
\par 13 Zij vergaderden ze dan, en vulden twaalf korven met brokken van de vijf gerstebroden, welke overgeschoten waren dengenen, die gegeten hadden.
\par 14 De mensen dan, gezien hebbende het teken, dat Jezus gedaan had, zeiden: Deze is waarlijk de Profeet, Die in de wereld komen zou.
\par 15 Jezus dan, wetende, dat zij zouden komen, en Hem met geweld nemen, opdat zij Hem Koning maakten, ontweek wederom op den berg, Hij Zelf alleen.
\par 16 En als het avond geworden was, gingen Zijn discipelen af naar de zee.
\par 17 En in het schip gegaan zijnde, kwamen zij over de zee naar Kapernaum. En het was alrede duister geworden, en Jezus was tot hen niet gekomen.
\par 18 En de zee verhief zich, overmits er een grote wind waaide.
\par 19 En als zij omtrent vijf en twintig of dertig stadien gevaren waren, zagen zij Jezus, wandelende op de zee, en komende bij het schip; en zij werden bevreesd.
\par 20 Maar Hij zeide tot hen: Ik ben het; zijt niet bevreesd.
\par 21 Zij hebben dan Hem gewilliglijk in het schip genomen; en terstond kwam het schip aan het land, daar zij naar toe voeren.
\par 22 Des anderen daags de schare, die aan de andere zijde der zee stond, ziende, dat aldaar geen ander scheepje was dan dat ene, daar Zijn discipelen ingegaan waren, en dat Jezus met Zijn discipelen in dat scheepje niet was gegaan, maar dat Zijn discipelen alleen weggevaren waren;
\par 23 (Doch er kwamen andere scheepjes van Tiberias, nabij de plaats, waar zij het brood gegeten hadden, als de Heere gedankt had.)
\par 24 Toen dan de schare zag, dat Jezus aldaar niet was, noch Zijn discipelen, zo gingen zij ook in de schepen, en kwamen te Kapernaum, zoekende Jezus.
\par 25 En als zij Hem gevonden hadden over de zee, zeiden zij tot Hem: Rabbi, wanneer zijt Gij hier gekomen?
\par 26 Jezus antwoordde hun en zeide: Voorwaar, voorwaar zeg Ik u: gij zoekt Mij, niet omdat gij tekenen gezien hebt, maar omdat gij van de broden gegeten hebt, en verzadigd zijt.
\par 27 Werkt niet om de spijs, die vergaat, maar om de spijs, die blijft tot in het eeuwige leven, welke de Zoon des mensen ulieden geven zal; want Dezen heeft God de Vader verzegeld.
\par 28 Zij zeiden dan tot Hem: Wat zullen wij doen, opdat wij de werken Gods mogen werken?
\par 29 Jezus antwoordde en zeide tot hen: Dit is het werk Gods, dat gij gelooft in Hem, Dien Hij gezonden heeft.
\par 30 Zij zeiden dan tot Hem: Wat teken doet Gij dan, opdat wij het mogen zien, en U geloven? Wat werkt Gij?
\par 31 Onze vaders hebben het Manna gegeten in de woestijn; gelijk geschreven is: Hij gaf hun het brood uit den hemel te eten.
\par 32 Jezus dan zeide tot hen: Voorwaar, voorwaar zeg Ik u: Mozes heeft u niet gegeven het brood uit den hemel; maar Mijn Vader geeft u dat ware Brood uit den hemel.
\par 33 Want het Brood Gods is Hij, Die uit den hemel nederdaalt, en Die der wereld het leven geeft.
\par 34 Zij zeiden dan tot Hem: Heere, geef ons altijd dit Brood.
\par 35 En Jezus zeide tot hen: Ik ben het Brood des levens; die tot Mij komt, zal geenszins hongeren, en die in Mij gelooft, zal nimmermeer dorsten.
\par 36 Maar Ik heb u gezegd, dat gij Mij ook gezien hebt, en gij gelooft niet.
\par 37 Al wat Mij de Vader geeft, zal tot Mij komen; en die tot Mij komt, zal Ik geenszins uitwerpen.
\par 38 Want Ik ben uit den hemel nedergedaald, niet opdat Ik Mijn wil zou doen, maar den wil Desgenen, Die Mij gezonden heeft.
\par 39 En dit is de wil des Vaders, Die Mij gezonden heeft, dat al wat Hij Mij gegeven heeft, Ik daaruit niet verlieze, maar hetzelve opwekke ten uitersten dage.
\par 40 En dit is de wil Desgenen, Die Mij gezonden heeft, dat een iegelijk, die den Zoon aanschouwt, en in Hem gelooft, het eeuwige leven hebbe; en Ik zal hem opwekken ten uitersten dage.
\par 41 De Joden dan murmureerden over Hem, omdat Hij gezegd had: Ik ben het Brood, Dat uit den hemel nedergedaald is.
\par 42 En zij zeiden: Is deze niet Jezus, de Zoon van Jozef, Wiens vader en moeder wij kennen? Hoe zegt Deze dan: Ik ben uit den hemel nedergedaald?
\par 43 Jezus antwoordde dan, en zeide tot hen: Murmureert niet onder elkander.
\par 44 Niemand kan tot Mij komen, tenzij dat de Vader, Die Mij gezonden heeft, hem trekke; en Ik zal hem opwekken ten uitersten dage.
\par 45 Er is geschreven in de profeten: En zij zullen allen van God geleerd zijn. Een iegelijk dan, die het van den Vader gehoord en geleerd heeft, die komt tot Mij.
\par 46 Niet dat iemand den Vader gezien heeft, dan Die van God is; Deze heeft den Vader gezien.
\par 47 Voorwaar, voorwaar zeg Ik u: Die in Mij gelooft, heeft het eeuwige leven.
\par 48 Ik ben het Brood des levens.
\par 49 Uw vaders hebben het Manna gegeten in de woestijn, en zij zijn gestorven.
\par 50 Dit is het Brood, dat uit den hemel nederdaalt, opdat de mens daarvan ete, en niet sterve.
\par 51 Ik ben dat levende Brood, dat uit den hemel nedergedaald is; zo iemand van dit Brood eet, die zal in der eeuwigheid leven. En het Brood, dat Ik geven zal, is Mijn vlees, hetwelk Ik geven zal voor het leven der wereld.
\par 52 De Joden dan streden onder elkander, zeggende: Hoe kan ons deze Zijn vlees te eten geven?
\par 53 Jezus dan zeide tot hen: Voorwaar, voorwaar zeg Ik ulieden: Tenzij dat gij het vlees des Zoons des mensen eet, en Zijn bloed drinkt, zo hebt gij geen leven in uzelven.
\par 54 Die Mijn vlees eet, en Mijn bloed drinkt, die heeft het eeuwige leven; en Ik zal hem opwekken ten uitersten dage.
\par 55 Want Mijn vlees is waarlijk Spijs, en Mijn bloed is waarlijk Drank.
\par 56 Die Mijn vlees eet, en Mijn bloed drinkt, die blijft in Mij, en Ik in hem.
\par 57 Gelijkerwijs Mij de levende Vader gezonden heeft, en Ik leve door den Vader; alzo die Mij eet, dezelve zal leven door Mij.
\par 58 Dit is het Brood, dat uit den hemel nedergedaald is; niet gelijk uw vaders het Manna gegeten hebben, en zijn gestorven. Die dit Brood eet, zal in der eeuwigheid leven.
\par 59 Deze dingen zeide Hij in de synagoge, lerende te Kapernaum.
\par 60 Velen dan van Zijn discipelen, dit horende, zeiden: Deze rede is hard; wie kan dezelve horen?
\par 61 Jezus nu, wetende bij Zichzelven, dat Zijn discipelen daarover murmureerden, zeide tot hen: Ergert ulieden dit?
\par 62 Wat zou het dan zijn, zo gij den Zoon des mensen zaagt opvaren, daar Hij te voren was?
\par 63 De Geest is het, Die levend maakt; het vlees is niet nut. De woorden, die Ik tot u spreek, zijn geest en zijn leven.
\par 64 Maar er zijn sommigen van ulieden, die niet geloven. Want Jezus wist van den beginne, wie zij waren, die niet geloofden, en wie hij was, die Hem verraden zou.
\par 65 En Hij zeide: Daarom heb Ik u gezegd, dat niemand tot Mij komen kan, tenzij dat het hem gegeven zij van Mijn Vader.
\par 66 Van toen af gingen velen Zijner discipelen terug, en wandelden niet meer met Hem.
\par 67 Jezus dan zeide tot de twaalven: Wilt gijlieden ook niet weggaan?
\par 68 Simon Petrus dan antwoordde Hem: Heere, tot Wien zullen wij heengaan? Gij hebt de woorden des eeuwigen levens.
\par 69 En wij hebben geloofd en bekend, dat Gij zijt de Christus, de Zoon des levenden Gods.
\par 70 Jezus antwoordde hun: Heb Ik niet u twaalf uitverkoren? En een uit u is een duivel.
\par 71 En Hij zeide dit van Judas, Simons zoon, Iskariot; want deze zou Hem verraden, zijnde een van de twaalven.

\chapter{7}

\par 1 En na dezen wandelde Jezus in Galilea; want Hij wilde in Judea niet wandelen, omdat de Joden Hem zochten te doden.
\par 2 En het feest der Joden, namelijk de loof huttenzetting, was nabij.
\par 3 Zo zeiden dan Zijn broeders tot Hem: Vertrek van hier, en ga heen in Judea, opdat ook Uw discipelen Uw werken mogen aanschouwen, die Gij doet.
\par 4 Want niemand doet iets in het verborgen, en zoekt zelf, dat men openlijk van hem spreke. Indien Gij deze dingen doet, zo openbaar Uzelven aan de wereld.
\par 5 Want ook Zijn broeders geloofden niet in Hem.
\par 6 Jezus dan zeide tot hen: Mijn tijd is nog niet hier, maar uw tijd is altijd bereid.
\par 7 De wereld kan ulieden niet haten, maar Mij haat zij, omdat Ik van dezelve getuig, dat haar werken boos zijn.
\par 8 Gaat gijlieden op tot dit feest; Ik ga nog niet op tot dit feest; want Mijn tijd is nog niet vervuld.
\par 9 En als Hij deze dingen tot hen gezegd had, bleef Hij in Galilea.
\par 10 Maar als Zijn broeders opgegaan waren, toen ging Hij ook Zelf op tot het feest, niet openlijk, maar als in het verborgen.
\par 11 De Joden dan zochten Hem in het feest, en zeiden: Waar is Hij?
\par 12 En er was veel gemurmels van Hem onder de scharen. Sommigen zeiden: Hij is goed; en anderen zeiden: Neen, maar Hij verleidt de schare.
\par 13 Nochtans sprak niemand vrijmoediglijk van Hem, om de vrees der Joden.
\par 14 Doch als het nu in het midden van het feest was, zo ging Jezus op in den tempel, en leerde.
\par 15 En de Joden verwonderden zich, zeggende: Hoe weet Deze de Schriften, daar Hij ze niet geleerd heeft?
\par 16 Jezus antwoordde hun, en zeide: Mijn leer is Mijne niet, maar Desgenen, Die Mij gezonden heeft.
\par 17 Zo iemand wil Deszelfs wil doen, die zal van deze leer bekennen, of zij uit God is, dan of Ik van Mijzelven spreek.
\par 18 Die van zichzelven spreekt, zoekt zijn eigen eer; maar Die de eer zoekt Desgenen, Die Hem gezonden heeft, Die is waarachtig, en geen ongerechtigheid is in Hem.
\par 19 Heeft Mozes u niet de wet gegeven? En niemand van u doet de wet. Wat zoekt gij Mij te doden?
\par 20 De schare antwoordde en zeide: Gij hebt den duivel; wie zoekt U te doden?
\par 21 Jezus antwoordde en zeide tot hen: Een werk heb Ik gedaan, en gij verwondert u allen.
\par 22 Daarom heeft Mozes ulieden de besnijdenis gegeven (niet dat zij uit Mozes is, maar uit de vaderen), en gij besnijdt een mens op den sabbat.
\par 23 Indien een mens de besnijdenis ontvangt op den sabbat, opdat de wet van Mozes niet gebroken worde; zijt gij toornig op Mij, dat Ik een gehelen mens gezond gemaakt heb op den sabbat?
\par 24 Oordeelt niet naar het aanzien, maar oordeelt een rechtvaardig oordeel.
\par 25 Sommigen dan uit die van Jeruzalem zeiden: Is Deze niet, Dien zij zoeken te doden?
\par 26 En ziet, Hij spreekt vrijmoediglijk, en zij zeggen Hem niets. Zouden nu wel de oversten waarlijk weten, dat Deze waarlijk is de Christus?
\par 27 Doch van Dezen weten wij, van waar Hij is; maar de Christus, wanneer Hij komen zal, zo zal niemand weten, van waar Hij is.
\par 28 Jezus dan riep in den tempel, lerende en zeggende: En gij kent Mij, en gij weet, van waar Ik ben; en Ik ben van Mijzelven niet gekomen, maar Hij is waarachtig, Die Mij gezonden heeft, Welken gijlieden niet kent.
\par 29 Maar Ik ken Hem; want Ik ben van Hem, en Hij heeft Mij gezonden.
\par 30 Zij zochten Hem dan te grijpen; maar niemand sloeg de hand aan Hem; want Zijn ure was nog niet gekomen.
\par 31 En velen uit de schare geloofden in Hem, en zeiden: Wanneer de Christus zal gekomen zijn, zal Hij ook meer tekenen doen dan die, welke Deze gedaan heeft?
\par 32 De Farizeen hoorden, dat de schare dit van Hem murmelde; en de Farizeen en de overpriesters zonden dienaren, opdat zij Hem grijpen zouden.
\par 33 Jezus dan zeide tot hen: Nog een kleinen tijd ben Ik bij u, en Ik ga heen tot Dengene, Die Mij gezonden heeft.
\par 34 Gij zult Mij zoeken, en gij zult Mij niet vinden; en waar Ik ben, kunt gij niet komen.
\par 35 De Joden dan zeiden tot elkander: Waar zal Deze heengaan, dat wij Hem niet zullen vinden? Zal Hij tot de verstrooide Grieken gaan, en de Grieken leren?
\par 36 Wat is dit voor een rede, die Hij gezegd heeft: Gij zult Mij zoeken, en zult Mij niet vinden; en waar Ik ben, kunt gij niet komen?
\par 37 En op den laatsten dag, zijnde de grote dag van het feest, stond Jezus en riep, zeggende: Zo iemand dorst, die kome tot Mij en drinke.
\par 38 Die in Mij gelooft, gelijkerwijs de Schrift zegt, stromen des levenden waters zullen uit zijn buik vloeien.
\par 39 (En dit zeide Hij van den Geest, Denwelken ontvangen zouden, die in Hem geloven; want de Heilige Geest was nog niet, overmits Jezus nog niet verheerlijkt was.)
\par 40 Velen dan uit de schare, deze rede horende, zeiden: Deze is waarlijk de Profeet.
\par 41 Anderen zeiden: Deze is de Christus. En anderen zeiden: Zal dan de Christus uit Galilea komen?
\par 42 Zegt de Schrift niet, dat de Christus komen zal uit den zade Davids, en van het vlek Bethlehem, waar David was?
\par 43 Er werd dan tweedracht onder de schare, om Zijnentwil.
\par 44 En sommigen van hen wilden Hem grijpen; maar niemand sloeg de handen aan Hem.
\par 45 De dienaars dan kwamen tot de overpriesters en Farizeen; en die zeiden tot hen: Waarom hebt gij Hem niet gebracht?
\par 46 De dienaars antwoordden: Nooit heeft een mens alzo gesproken, gelijk deze Mens.
\par 47 De Farizeen dan antwoordden hun: Zijt ook gijlieden verleid?
\par 48 Heeft iemand uit de oversten in Hem geloofd, of uit de Farizeen?
\par 49 Maar deze schare, die de wet niet weet, is vervloekt.
\par 50 Nicodemus zeide tot hen, welke des nachts tot Hem gekomen was, zijnde een uit hen:
\par 51 Oordeelt ook onze wet den mens, tenzij dat zij eerst van hem gehoord heeft, en verstaat, wat hij doet?
\par 52 Zij antwoordden en zeiden tot hem: Zijt gij ook uit Galilea? Onderzoek en zie, dat uit Galilea geen profeet opgestaan is.
\par 53 En een iegelijk ging heen naar zijn huis.

\chapter{8}

\par 1 Maar Jezus ging naar den Olijfberg.
\par 2 En des morgens vroeg kwam Hij wederom in den tempel, en al het volk kwam tot Hem; en nedergezeten zijnde, leerde Hij hen.
\par 3 En de Schriftgeleerden en de Farizeen brachten tot Hem een vrouw, in overspel gegrepen.
\par 4 En haar gesteld hebbende in het midden, zeiden zij tot Hem: Meester, deze vrouw is op de daad zelve gegrepen, overspel begaande.
\par 5 En Mozes heeft ons in de wet geboden, dat dezulken gestenigd zullen worden; Gij dan, wat zegt Gij?
\par 6 En dit zeiden zij, Hem verzoekende, opdat zij iets hadden, om Hem te beschuldigen. Maar Jezus, nederbukkende, schreef met den vinger in de aarde.
\par 7 En als zij Hem bleven vragen, richtte Hij Zich op, en zeide tot hen: Die van ulieden zonder zonde is, werpe eerst den steen op haar.
\par 8 En wederom nederbukkende, schreef Hij in de aarde.
\par 9 Maar zij, dit horende, en van hun geweten overtuigd zijnde, gingen uit, de een na den andere, beginnende van de oudsten tot de laatsten; en Jezus werd alleen gelaten; en de vrouw in het midden staande.
\par 10 En Jezus, Zich oprichtende, en niemand ziende dan de vrouw, zeide tot haar: Vrouw, waar zijn deze uw beschuldigers? Heeft u niemand veroordeeld?
\par 11 En zij zeide: Niemand, Heere! En Jezus zeide tot haar: Zo veroordeel Ik u ook niet; ga heen, en zondig niet meer.
\par 12 Jezus dan sprak wederom tot henlieden, zeggende: Ik ben het licht der wereld; die Mij volgt, zal in de duisternis niet wandelen, maar zal het licht des levens hebben.
\par 13 De Farizeen dan zeiden tot Hem: Gij getuigt van Uzelven; Uw getuigenis is niet waarachtig.
\par 14 Jezus antwoordde, en zeide tot hen: Hoewel Ik van Mijzelven getuig, zo is nochtans Mijn getuigenis waarachtig; want Ik weet, van waar Ik gekomen ben, en waar Ik heenga; maar gijlieden weet niet, van waar Ik kom, en waar Ik heenga.
\par 15 Gij oordeelt naar het vlees; Ik oordeel niemand.
\par 16 En indien Ik ook oordeel, Mijn oordeel is waarachtig; want Ik ben niet alleen, maar Ik en de Vader, Die Mij gezonden heeft.
\par 17 En er is ook in uw wet geschreven, dat de getuigenis van twee mensen waarachtig is.
\par 18 Ik ben het, Die van Mijzelven getuig, en de Vader, Die Mij gezonden heeft, getuigt van Mij.
\par 19 Zij dan zeiden tot Hem: Waar is Uw Vader? Jezus antwoordde: Gij kent noch Mij, noch Mijn Vader; indien gij Mij kendet, zo zoudt gij ook Mijn Vader kennen.
\par 20 Deze woorden sprak Jezus bij de schatkist, lerende in den tempel; en niemand greep Hem; want Zijn ure was nog niet gekomen.
\par 21 Jezus dan zeide wederom tot hen: Ik ga heen, en gij zult Mij zoeken, en in uw zonden zult gij sterven; waar Ik heenga, kunt gijlieden niet komen.
\par 22 De Joden dan zeiden: Zal Hij ook Zichzelven doden, omdat Hij zegt: Waar Ik heenga, kunt gijlieden niet komen?
\par 23 En Hij zeide tot hen: Gijlieden zijt van beneden, Ik ben van boven; gij zijt uit deze wereld, Ik ben niet uit deze wereld.
\par 24 Ik heb u dan gezegd, dat gij in uw zonden zult sterven; want indien gij niet gelooft, dat Ik Die ben, gij zult in uw zonden sterven.
\par 25 Zij zeiden dan tot Hem: Wie zijt Gij? En Jezus zeide tot hen: Wat Ik van den beginne ulieden ook zegge.
\par 26 Ik heb vele dingen van u te zeggen en te oordelen; maar Die Mij gezonden heeft, is waarachtig; en de dingen, die Ik van Hem gehoord heb, dezelve spreek Ik tot de wereld.
\par 27 Zij verstonden niet, dat Hij hun van den Vader sprak.
\par 28 Jezus dan zeide tot hen: Wanneer gij den Zoon des mensen zult verhoogd hebben, dan zult gij verstaan, dat Ik Die ben, en dat Ik van Mijzelven niets doe; maar deze dingen spreek Ik, gelijk Mijn Vader Mij geleerd heeft.
\par 29 En Die Mij gezonden heeft, is met Mij. De Vader heeft Mij niet alleen gelaten, want Ik doe altijd, wat Hem behagelijk is.
\par 30 Als Hij deze dingen sprak, geloofden velen in Hem.
\par 31 Jezus dan zeide tot de Joden, die in Hem geloofden: Indien gijlieden in Mijn woord blijft, zo zijt gij waarlijk Mijn discipelen;
\par 32 En zult de waarheid verstaan, en de waarheid zal u vrijmaken.
\par 33 Zij antwoordden Hem: Wij zijn Abrahams zaad, en hebben nooit iemand gediend; hoe zegt Gij dan: Gij zult vrij worden?
\par 34 Jezus antwoordde hun: Voorwaar, voorwaar zeg Ik u: Een iegelijk, die de zonde doet, is een dienstknecht der zonde.
\par 35 En de dienstknecht blijft niet eeuwiglijk in het huis, de zoon blijft er eeuwiglijk.
\par 36 Indien dan de Zoon u zal vrijgemaakt hebben, zo zult gij waarlijk vrij zijn.
\par 37 Ik weet, dat gij Abrahams zaad zijt; maar gij zoekt Mij te doden; want Mijn woord heeft in u geen plaats.
\par 38 Ik spreek wat Ik bij Mijn Vader gezien heb; gij doet dan ook, wat gij bij uw vader gezien hebt.
\par 39 Zij antwoordden en zeiden tot Hem: Abraham is onze vader. Jezus zeide tot hen: Indien gij Abrahams kinderen waart, zo zoudt gij de werken van Abraham doen.
\par 40 Maar nu zoekt gij Mij te doden, een Mens, Die u de waarheid gesproken heb, welke Ik van God gehoord heb. Dat deed Abraham niet.
\par 41 Gij doet de werken uws vaders. Zij zeiden dan tot Hem: Wij zijn niet geboren uit hoererij; wij hebben een Vader, namelijk God.
\par 42 Jezus dan zeide tot hen: Indien God uw Vader ware, zo zoudt gij Mij liefhebben; want Ik ben van God uitgegaan; en kom van Hem. Want Ik ben ook van Mijzelven niet gekomen, maar Hij heeft Mij gezonden.
\par 43 Waarom kent gij Mijn spraak niet? Het is, omdat gij Mijn woord niet kunt horen.
\par 44 Gij zijt uit den vader den duivel, en wilt de begeerten uws vaders doen; die was een mensenmoorder van den beginne, en is in de waarheid niet staande gebleven; want geen waarheid is in hem. Wanneer hij de leugen spreekt, zo spreekt hij uit zijn eigen; want hij is een leugenaar, en de vader derzelve leugen.
\par 45 Maar Mij, omdat Ik u de waarheid zeg, gelooft gij niet.
\par 46 Wie van u overtuigt Mij van zonde? En indien Ik de waarheid zeg, waarom gelooft gij Mij niet?
\par 47 Die uit God is, hoort de woorden Gods; daarom hoort gijlieden niet, omdat gij uit God niet zijt.
\par 48 De Joden dan antwoordden en zeiden tot Hem: Zeggen wij niet wel, dat Gij een Samaritaan zijt, en den duivel hebt?
\par 49 Jezus antwoordde: Ik heb den duivel niet; maar Ik eer Mijn Vader, en gij onteert Mij.
\par 50 Doch Ik zoek Mijn eer niet; er is Een, Die ze zoekt en oordeelt.
\par 51 Voorwaar, voorwaar zeg Ik u: Zo iemand Mijn woord zal bewaard hebben, die zal den dood niet zien in der eeuwigheid.
\par 52 De Joden dan zeiden tot Hem: Nu bekennen wij, dat Gij den duivel hebt. Abraham is gestorven, en de profeten; en zegt Gij: Zo iemand Mijn woord bewaard zal hebben, die zal den dood niet smaken in der eeuwigheid?
\par 53 Zijt Gij meerder, dan onze vader Abraham, welke gestorven is, en de profeten zijn gestorven; wien maakt Gij Uzelven?
\par 54 Jezus antwoordde: Indien Ik Mijzelven eer, zo is Mijn eer niets; Mijn Vader is het, Die Mij eert, Welken gij zegt, dat uw God is.
\par 55 En gij kent Hem niet, maar Ik ken Hem; en indien Ik zeg, dat Ik Hem niet ken, zo zal Ik ulieden gelijk zijn, dat is een leugenaar; maar Ik ken Hem, en bewaar Zijn woord.
\par 56 Abraham, uw vader, heeft met verheuging verlangd, opdat hij Mijn dag zien zou; en hij heeft hem gezien, en is verblijd geweest.
\par 57 De Joden dan zeiden tot Hem: Gij hebt nog geen vijftig jaren, en hebt Gij Abraham gezien?
\par 58 Jezus zeide tot hen: Voorwaar, voorwaar zeg Ik u: Eer Abraham was, ben Ik.
\par 59 Zij namen dan stenen op, dat zij ze op Hem wierpen. Maar Jezus verborg Zich, en ging uit den tempel, gaande door het midden van hen; en ging alzo voorbij.

\chapter{9}

\par 1 En voorbijgaande, zag Hij een mens, blind van de geboorte af.
\par 2 En Zijn discipelen vraagden Hem, zeggende: Rabbi, wie heeft er gezondigd, deze, of zijn ouders, dat hij blind zou geboren worden?
\par 3 Jezus antwoordde: Noch deze heeft gezondigd, noch zijn ouders, maar dit is geschied, opdat de werken Gods in hem zouden geopenbaard worden.
\par 4 Ik moet werken de werken Desgenen, Die Mij gezonden heeft, zolang het dag is; de nacht komt, wanneer niemand werken kan.
\par 5 Zolang Ik in de wereld ben, zo ben Ik het Licht der wereld.
\par 6 Dit gezegd hebbende, spoog Hij op de aarde, en maakte slijk uit dat speeksel, en streek dat slijk op de ogen des blinden;
\par 7 En zeide tot hem: Ga heen, was u in het badwater Siloam (hetwelk overgezet wordt: uitgezonden). Hij dan ging heen en wies zich, en kwam ziende.
\par 8 De geburen dan, en die hem te voren gezien hadden, dat hij blind was, zeiden: Is deze niet, die zat en bedelde?
\par 9 Anderen zeiden: Hij is het; en anderen: Hij is hem gelijk. Hij zeide: Ik ben het.
\par 10 Zij dan zeiden tot hem: Hoe zijn u de ogen geopend?
\par 11 Hij antwoordde en zeide: De Mens, genaamd Jezus, maakte slijk, en bestreek mijn ogen, en zeide tot mij: Ga heen naar het badwater Siloam, en was u. En ik ging heen, en wies mij, en ik werd ziende.
\par 12 Zij dan zeiden tot hem: Waar is Die? Hij zeide: Ik weet het niet.
\par 13 Zij brachten hem tot de Farizeen, hem namelijk, die te voren blind geweest was.
\par 14 En het was sabbat, als Jezus het slijk maakte, en zijn ogen opende.
\par 15 De Farizeen dan vraagden hem ook wederom, hoe hij ziende geworden was. En hij zeide tot hen: Hij legde slijk op mijn ogen, en ik wies mij, en ik zie.
\par 16 Sommigen dan uit de Farizeen zeiden: Deze Mens is van God niet, want Hij houdt den sabbat niet. Anderen zeiden: Hoe kan een mens, die een zondaar is, zulke tekenen doen? En er was tweedracht onder hen.
\par 17 Zij zeiden wederom tot den blinde: Gij, wat zegt gij van Hem; dewijl Hij uw ogen geopend heeft? En hij zeide: Hij is een Profeet.
\par 18 De Joden dan geloofden van hem niet, dat hij blind geweest was, en ziende was geworden, totdat zij geroepen hadden de ouders desgenen, die ziende geworden was.
\par 19 En zij vraagden hun, zeggende: Is deze uw zoon, welken gij zegt, dat blind geboren is? Hoe ziet hij dan nu?
\par 20 Zijn ouders antwoordden hun en zeiden: Wij weten, dat deze onze zoon is, en dat hij blind geboren is;
\par 21 Maar hoe hij nu ziet, weten wij niet; of wie zijn ogen geopend heeft, weten wij niet; hij heeft zijn ouderdom, vraagt hemzelven; hij zal van zichzelven spreken.
\par 22 Dit zeiden zijn ouders, omdat zij de Joden vreesden; want de Joden hadden alrede te zamen een besluit gemaakt, zo iemand Hem beleed Christus te zijn, dat die uit de synagoge zou geworpen worden.
\par 23 Daarom zeiden zijn ouders: Hij heeft zijn ouderdom, vraagt hemzelven.
\par 24 Zij dan riepen voor de tweede maal den mens, die blind geweest was, en zeiden tot hem: Geef God de eer; wij weten, dat deze Mens een zondaar is.
\par 25 Hij dan antwoordde en zeide: Of Hij een zondaar is, weet ik niet; een ding weet ik, dat ik blind was, en nu zie.
\par 26 En zij zeiden wederom tot hem: Wat heeft Hij u gedaan? Hoe heeft Hij uw ogen geopend?
\par 27 Hij antwoordde hun: Ik heb het u alrede gezegd, en gij hebt het niet gehoord; wat wilt gij het wederom horen? Wilt gijlieden ook Zijn discipelen worden?
\par 28 Zij gaven hem dan scheldwoorden, en zeiden: Gij zijt Zijn discipel; maar wij zijn discipelen van Mozes.
\par 29 Wij weten, dat God tot Mozes gesproken heeft; maar Dezen weten wij niet, van waar Hij is.
\par 30 De mens antwoordde, en zeide tot hen: Hierin is immers wat wonders, dat gij niet weet, van waar Hij is, en nochtans heeft Hij mijn ogen geopend.
\par 31 En wij weten, dat God de zondaars niet hoort; maar zo iemand godvruchtig is, en Zijn wil doet, dien hoort Hij.
\par 32 Van alle eeuw is het niet gehoord, dat iemand eens blindgeborenen ogen geopend heeft.
\par 33 Indien Deze van God niet ware, Hij zou niets kunnen doen.
\par 34 Zij antwoordden, en zeiden tot hem: Gij zijt geheel in zonden geboren, en leert gij ons? En zij wierpen hem uit.
\par 35 Jezus hoorde, dat zij hem uitgeworpen hadden, en hem vindende, zeide Hij tot hem: Gelooft gij in den Zoon van God?
\par 36 Hij antwoordde en zeide: Wie is Hij, Heere, opdat ik in Hem moge geloven?
\par 37 En Jezus zeide tot Hem: En gij hebt Hem gezien, en Die met u spreekt, Dezelve is het.
\par 38 En hij zeide: Ik geloof, Heere! En hij aanbad Hem.
\par 39 En Jezus zeide: Ik ben tot een oordeel in deze wereld gekomen, opdat degenen, die niet zien, zien mogen, en die zien, blind worden.
\par 40 En dit hoorden enigen uit de Farizeen, die bij Hem waren, en zeiden tot Hem: Zijn wij dan ook blind?
\par 41 Jezus zeide tot hen: Indien gij blind waart, zo zoudt gij geen zonde hebben; maar nu zegt gij: Wij zien; zo blijft dan uw zonde.

\chapter{10}

\par 1 Voorwaar, voorwaar zeg Ik ulieden: Die niet ingaat door de deur in den stal der schapen, maar van elders inklimt, die is een dief en moordenaar.
\par 2 Maar die door de deur ingaat, is een herder der schapen.
\par 3 Dezen doet de deurwachter open, en de schapen horen zijn stem; en hij roept zijn schapen bij name, en leidt ze uit.
\par 4 En wanneer hij zijn schapen uitgedreven heeft, zo gaat hij voor hen heen; en de schapen volgen hem, overmits zij zijn stem kennen.
\par 5 Maar een vreemde zullen zij geenszins volgen, maar zullen van hem vlieden; overmits zij de stem des vreemden niet kennen.
\par 6 Deze gelijkenis zeide Jezus tot hen; maar zij verstonden niet, wat het was, dat Hij tot hen sprak.
\par 7 Jezus dan zeide wederom tot hen: Voorwaar, voorwaar zeg Ik u: Ik ben de Deur der schapen.
\par 8 Allen, zovelen als er voor Mij zijn gekomen, zijn dieven en moordenaars; maar de schapen hebben hen niet gehoord.
\par 9 Ik ben de Deur; indien iemand door Mij ingaat, die zal behouden worden; en hij zal ingaan en uitgaan, en weide vinden.
\par 10 De dief komt niet, dan opdat hij stele, en slachte, en verderve; Ik ben gekomen, opdat zij het leven hebben, en overvloed hebben.
\par 11 Ik ben de goede Herder; de goede herder stelt zijn leven voor de schapen.
\par 12 Maar de huurling, en die geen herder is, wien de schapen niet eigen zijn, ziet den wolf komen, en verlaat de schapen, en vliedt; en de wolf grijpt ze, en verstrooit de schapen.
\par 13 En de huurling vliedt, overmits hij een huurling is, en heeft geen zorg voor de schapen.
\par 14 Ik ben de goede Herder; en Ik ken de Mijnen, en worde van de Mijnen gekend.
\par 15 Gelijkerwijs de Vader Mij kent, alzo ken Ik ook den Vader; en Ik stel Mijn leven voor de schapen.
\par 16 Ik heb nog andere schapen, die van dezen stal niet zijn; deze moet Ik ook toebrengen; en zij zullen Mijn stem horen; en het zal worden een kudde, en een Herder.
\par 17 Daarom heeft mij de Vader lief, overmits Ik Mijn leven afleg, opdat Ik hetzelve wederom neme.
\par 18 Niemand neemt hetzelve van Mij, maar Ik leg het van Mijzelven af; Ik heb macht hetzelve af te leggen, en heb macht hetzelve wederom te nemen. Dit gebod heb Ik van Mijn Vader ontvangen.
\par 19 Er werd dan wederom tweedracht onder de Joden, om dezer woorden wil.
\par 20 En velen van hen zeiden: Hij heeft den duivel, en is uitzinnig; wat hoort gij Hem?
\par 21 Anderen zeiden: Dit zijn geen woorden eens bezetenen; kan ook de duivel der blinden ogen openen?
\par 22 En het was het feest der vernieuwing des tempels te Jeruzalem; en het was winter.
\par 23 En Jezus wandelde in den tempel, in het voorhof van Salomo.
\par 24 De Joden dan omringden Hem, en zeiden tot Hem: Hoe lang houdt Gij onze ziel op? Indien Gij de Christus zijt, zeg het ons vrijuit.
\par 25 Jezus antwoordde hun: Ik heb het u gezegd, en gij gelooft het niet. De werken, die Ik doe in den Naam Mijns Vaders, die getuigen van Mij.
\par 26 Maar gijlieden gelooft niet; want gij zijt niet van Mijn schapen, gelijk Ik u gezegd heb.
\par 27 Mijn schapen horen Mijn stem, en Ik ken dezelve, en zij volgen Mij.
\par 28 En Ik geef hun het eeuwige leven; en zij zullen niet verloren gaan in der eeuwigheid, en niemand zal dezelve uit Mijn hand rukken.
\par 29 Mijn Vader, die ze Mij gegeven heeft, is meerder dan allen; en niemand kan ze rukken uit de hand Mijns Vaders.
\par 30 Ik en de Vader zijn een.
\par 31 De Joden dan namen wederom stenen op, om Hem te stenigen.
\par 32 Jezus antwoordde hun: Ik heb u vele treffelijke werken getoond van Mijn Vader; om welk werk van die stenigt gij Mij?
\par 33 De Joden antwoordden Hem, zeggende: Wij stenigen U niet over enig goed werk, maar over gods lastering, en omdat Gij, een Mens zijnde, Uzelven God maakt.
\par 34 Jezus antwoordde hun: Is er niet geschreven in uw wet: Ik heb gezegd, gij zijt goden?
\par 35 Indien de wet die goden genaamd heeft, tot welke het woord Gods geschied is, en de Schrift niet kan gebroken worden;
\par 36 Zegt gijlieden tot Mij, Dien de Vader geheiligd en in de wereld gezonden heeft: Gij lastert God; omdat Ik gezegd heb: Ik ben Gods Zoon?
\par 37 Indien Ik niet doe de werken Mijns Vaders, zo gelooft Mij niet;
\par 38 Maar indien Ik ze doe, en zo gij Mij niet gelooft, zo gelooft de werken; opdat gij moogt bekennen en geloven, dat de Vader in Mij is, en Ik in Hem.
\par 39 Zij zochten dan wederom Hem te grijpen, en Hij ontging uit hun hand.
\par 40 En Hij ging wederom over de Jordaan, tot de plaats, waar Johannes eerst doopte; en Hij bleef aldaar.
\par 41 En velen kwamen tot Hem, en zeiden: Johannes deed wel geen teken; maar alles, wat Johannes van Dezen zeide, was waar.
\par 42 En velen geloofden aldaar in Hem.

\chapter{11}

\par 1 En er was een zeker man krank, genaamd Lazarus, van Bethanie, uit het vlek van Maria en haar zuster Martha.
\par 2 (Maria nu was degene, die den Heere gezalfd heeft met zalf, en Zijn voeten afgedroogd heeft met haar haren; welker broeder Lazarus krank was.)
\par 3 Zijn zusters dan zonden tot Hem, zeggende: Heere, zie, dien Gij liefhebt, is krank.
\par 4 En Jezus, dat horende, zeide: Deze krankheid is niet tot den dood, maar ter heerlijkheid Gods; opdat de Zone Gods door dezelve verheerlijkt worde.
\par 5 Jezus nu had Martha, en haar zuster, en Lazarus lief.
\par 6 Als Hij dan gehoord had, dat hij krank was, toen bleef Hij nog twee dagen in de plaats, waar Hij was.
\par 7 Daarna zeide Hij verder tot de discipelen: Laat ons wederom naar Judea gaan.
\par 8 De discipelen zeiden tot Hem: Rabbi! de Joden hebben U nu onlangs gezocht te stenigen, en gaat Gij wederom derwaarts?
\par 9 Jezus antwoordde: Zijn er niet twaalf uren in den dag? Indien iemand in den dag wandelt, zo stoot hij zich niet, overmits hij het licht dezer wereld ziet;
\par 10 Maar indien iemand in den nacht wandelt, zo stoot hij zich, overmits het licht in hem niet is.
\par 11 Dit sprak Hij; en daarna zeide Hij tot hen: Lazarus, onze vriend, slaapt; maar Ik ga heen, om hem uit den slaap op te wekken.
\par 12 Zijn discipelen dan zeiden: Heere, indien hij slaapt, zo zal hij gezond worden.
\par 13 Doch Jezus had gesproken van zijn dood; maar zij meenden, dat Hij sprak van de rust des slaaps.
\par 14 Toen zeide dan Jezus tot hen vrijuit: Lazarus is gestorven.
\par 15 En Ik ben blijde om uwentwil, dat Ik daar niet geweest ben, opdat gij geloven moogt; doch laat ons tot hem gaan.
\par 16 Thomas dan, genaamd Didymus, zeide tot zijn medediscipelen: Laat ons ook gaan, opdat wij met Hem sterven.
\par 17 Jezus dan, gekomen zijnde, vond, dat hij nu vier dagen in het graf geweest was.
\par 18 (Bethanie nu was nabij Jeruzalem, omtrent vijftien stadien van daar.)
\par 19 En velen uit de Joden waren gekomen tot Martha en Maria, opdat zij haar vertroosten zouden over haar broeder.
\par 20 Martha dan, als zij hoorde, dat Jezus kwam, ging Hem tegemoet; doch Maria bleef in huis zitten.
\par 21 Zo zeide Martha dan tot Jezus: Heere, waart Gij hier geweest, zo ware mijn broeder niet gestorven;
\par 22 Maar ook nu weet ik, dat alles, wat Gij van God begeren zult, God U het geven zal.
\par 23 Jezus zeide tot haar: Uw broeder zal wederopstaan.
\par 24 Martha zeide tot Hem: Ik weet, dat hij opstaan zal in de opstanding ten laatsten dage.
\par 25 Jezus zeide tot haar: Ik ben de Opstanding en het Leven; die in Mij gelooft zal leven, al ware hij ook gestorven;
\par 26 En een iegelijk, die leeft, en in Mij gelooft, zal niet sterven in der eeuwigheid. Gelooft gij dat?
\par 27 Zij zeide tot Hem: Ja, Heere; ik heb geloofd, dat Gij zijt de Christus, de Zone Gods, Die in de wereld komen zou.
\par 28 En dit gezegd hebbende, ging zij heen, en riep Maria, haar zuster, heimelijk, zeggende: De Meester is daar, en Hij roept u.
\par 29 Deze, als zij dat hoorde, stond haastelijk op, en ging tot Hem.
\par 30 (Jezus nu was nog in het vlek niet gekomen, maar was in de plaats, waar Hem Martha tegemoet gekomen was.)
\par 31 De Joden dan, die met haar in het huis waren, en haar vertroostten, ziende Maria, dat zij haastelijk opstond en uitging, volgden haar, zeggende: Zij gaat naar het graf, opdat zij aldaar wene.
\par 32 Maria dan, als zij kwam, waar Jezus was, en Hem zag, viel aan Zijn voeten, zeggende tot Hem: Heere, indien Gij hier geweest waart, zo ware mijn broeder niet gestorven.
\par 33 Jezus dan, als Hij haar zag wenen, en de Joden, die met haar kwamen, ook wenen, werd zeer bewogen in den geest, en ontroerde Zichzelven;
\par 34 En zeide: Waar hebt gij hem gelegd? Zij zeiden tot Hem: Heere, kom en zie het.
\par 35 Jezus weende.
\par 36 De Joden dan zeiden: Ziet, hoe lief Hij hem had!
\par 37 En sommigen uit hen zeiden: Kon Hij, Die de ogen des blinden geopend heeft, niet maken, dat ook deze niet gestorven ware?
\par 38 Jezus dan wederom in Zichzelven zeer bewogen zijnde, kwam tot het graf; en het was een spelonk, en een steen was daarop gelegd.
\par 39 Jezus zeide: Neemt den steen weg. Martha, de zuster des gestorvenen, zeide tot Hem: Heere, hij riekt nu al, want hij heeft vier dagen aldaar gelegen.
\par 40 Jezus zeide tot haar: Heb Ik u niet gezegd, dat, zo gij gelooft, gij de heerlijkheid Gods zien zult?
\par 41 Zij namen dan den steen weg, waar de gestorvene lag. En Jezus hief de ogen opwaarts, en zeide: Vader, Ik dank U, dat Gij Mij gehoord hebt.
\par 42 Doch Ik wist, dat Gij Mij altijd hoort; maar om der schare wil, die rondom staat, heb Ik dit gezegd, opdat zij zouden geloven, dat Gij Mij gezonden hebt.
\par 43 En als Hij dit gezegd had, riep Hij met grote stemme: Lazarus, kom uit!
\par 44 En de gestorvene kwam uit, gebonden aan handen en voeten met grafdoeken, en zijn aangezicht was omwonden met een zweetdoek. Jezus zeide tot hen: Ontbindt hem, en laat hem heengaan.
\par 45 Velen dan uit de Joden, die tot Maria gekomen waren, en aanschouwd hadden, hetgeen Jezus gedaan had, geloofden in Hem.
\par 46 Maar sommigen van hen gingen tot de Farizeen, en zeiden tot hen, hetgeen Jezus gedaan had.
\par 47 De overpriesters dan en de Farizeen vergaderden den raad, en zeiden: Wat zullen wij doen? want deze Mens doet vele tekenen.
\par 48 Indien wij Hem alzo laten geworden, zij zullen allen in Hem geloven, en de Romeinen zullen komen, en wegnemen beide onze plaats en volk.
\par 49 En een uit hen, namelijk Kajafas, die deszelven jaars hogepriester was, zeide tot hen: Gij verstaat niets;
\par 50 En gij overlegt niet, dat het ons nut is, dat een mens sterve voor het volk, en het gehele volk niet verloren ga.
\par 51 En dit zeide hij niet uit zichzelven; maar, zijnde hogepriester deszelven jaars, profeteerde hij, dat Jezus sterven zou voor het volk;
\par 52 En niet alleen voor dat volk, maar opdat Hij ook de kinderen Gods, die verstrooid waren, tot een zou vergaderen.
\par 53 Van dien dag dan af beraadslaagden zij te zamen, dat zij Hem doden zouden.
\par 54 Jezus dan wandelde niet meer vrijelijk onder de Joden; maar ging van daar naar het land bij de woestijn, naar de stad, genaamd Efraim, en verkeerde aldaar met Zijn discipelen.
\par 55 En het pascha der Joden was nabij, en velen uit dat land gingen op naar Jeruzalem, voor het pascha, opdat zij zichzelven reinigden.
\par 56 Zij zochten dan Jezus, en zeiden onder elkander, staande in den tempel: Wat dunkt u? Dunkt u, dat Hij niet komen zal tot het feest?
\par 57 De overpriesters nu en de Farizeen hadden een gebod gegeven, dat, zo iemand wist, waar Hij was, hij het zou te kennen geven, opdat zij Hem mochten vangen.

\chapter{12}

\par 1 Jezus dan kwam zes dagen voor het pascha te Bethanie, daar Lazarus was, die gestorven was geweest, welken Hij opgewekt had uit de doden.
\par 2 Zij bereidden Hem dan aldaar een avondmaal, en Martha diende; en Lazarus was een van degenen, die met Hem aanzaten.
\par 3 Maria dan, genomen hebbende een pond zalf van onvervalsten, zeer kostelijken nardus, heeft de voeten van Jezus gezalfd, en met haar haren Zijn voeten afgedroogd; en het huis werd vervuld van den reuk der zalf.
\par 4 Zo zeide dan een van Zijn discipelen, namelijk Judas, Simons zoon, Iskariot, die Hem verraden zou:
\par 5 Waarom is deze zalf niet verkocht voor driehonderd penningen, en den armen gegeven?
\par 6 En dit zeide hij, niet omdat hij bezorgd was voor de armen, maar omdat hij een dief was, en de beurs had, en droeg hetgeen gegeven werd.
\par 7 Jezus dan zeide: Laat af van haar; zij heeft dit bewaard tegen den dag Mijner begrafenis.
\par 8 Want de armen hebt gijlieden altijd met u, maar Mij hebt gij niet altijd.
\par 9 Een grote schare dan der Joden verstond, dat Hij aldaar was; en zij kwamen, niet alleen om Jezus' wil, maar opdat zij ook Lazarus zouden zien, dien Hij uit de doden opgewekt had.
\par 10 En de overpriesters beraadslaagden, dat zij ook Lazarus doden zouden.
\par 11 Want velen van de Joden gingen heen om zijnentwil, en geloofden in Jezus.
\par 12 Des anderen daags, een grote schare, die tot het feest gekomen was, horende, dat Jezus naar Jeruzalem kwam,
\par 13 Namen de takken van palmbomen, en gingen uit Hem tegemoet, en riepen: Hosanna! Gezegend is Hij, Die komt in den Naam des Heeren, Hij, Die is de Koning Israels!
\par 14 En Jezus vond een jongen ezel, en zat daarop, gelijk geschreven is:
\par 15 Vrees niet, gij dochter Sions, zie, uw Koning komt, zittende op het veulen ener ezelin.
\par 16 Doch dit verstonden Zijn discipelen in het eerst niet; maar als Jezus verheerlijkt was, toen werden zij indachtig, dat dit van Hem geschreven was, en dat zij Hem dit gedaan hadden.
\par 17 De schare dan, die met Hem was, getuigde dat Hij Lazarus uit het graf geroepen, en hem uit de doden opgewekt had.
\par 18 Daarom ging ook de schare Hem tegemoet, overmits zij gehoord had, dat Hij dat teken gedaan had.
\par 19 De Farizeen dan zeiden onder elkander: Ziet gij wel, dat gij gans niet vordert? Ziet, de gehele wereld gaat Hem na.
\par 20 En er waren sommige Grieken uit degenen, die opgekomen waren, opdat zij op het feest zouden aanbidden;
\par 21 Dezen dan gingen tot Filippus, die van Bethsaida in Galilea was, en baden hem, zeggende: Heere, wij wilden Jezus wel zien.
\par 22 Filippus kwam en zeide het Andreas; en Andreas en Filippus wederom zeiden het Jezus.
\par 23 Maar Jezus antwoordde hun, zeggende: De ure is gekomen, dat de Zoon des mensen zal verheerlijkt worden.
\par 24 Voorwaar, voorwaar zeg Ik u: Indien het tarwegraan in de aarde niet valt, en sterft, zo blijft hetzelve alleen; maar indien het sterft, zo brengt het veel vrucht voort.
\par 25 Die zijn leven liefheeft, zal hetzelve verliezen; en die zijn leven haat in deze wereld, zal hetzelve bewaren tot het eeuwige leven.
\par 26 Zo iemand Mij dient, die volge Mij; en waar Ik ben, aldaar zal ook Mijn dienaar zijn. En zo iemand Mij dient, de Vader zal hem eren.
\par 27 Nu is Mijn ziel ontroerd; en wat zal Ik zeggen? Vader, verlos Mij uit deze ure! Maar hierom ben Ik in deze ure gekomen.
\par 28 Vader, verheerlijk Uw Naam. Er kwam dan een stem uit den hemel, zeggende: En Ik heb Hem verheerlijkt, en Ik zal Hem wederom verheerlijken.
\par 29 De schare dan, die daar stond, en dit hoorde, zeide, dat er een donderslag geschied was. Anderen zeiden: Een engel heeft tot Hem gesproken.
\par 30 Jezus antwoordde en zeide: Niet om Mijnentwil is deze stem geschied, maar om uwentwil.
\par 31 Nu is het oordeel dezer wereld; nu zal de overste dezer wereld buiten geworpen worden.
\par 32 En Ik, zo wanneer Ik van de aarde zal verhoogd zijn, zal hen allen tot Mij trekken.
\par 33 (En dit zeide Hij, betekenende, hoedanigen dood Hij sterven zou.)
\par 34 De schare antwoordde Hem: Wij hebben uit de wet gehoord, dat de Christus blijft in der eeuwigheid; en hoe zegt Gij, dat de Zoon des mensen moet verhoogd worden? Wie is deze Zoon des mensen?
\par 35 Jezus dan zeide tot hen: Nog een kleinen tijd is het Licht bij ulieden; wandelt, terwijl gij het Licht hebt, opdat de duisternis u niet bevange. En die in de duisternis wandelt, weet niet, waar hij heengaat.
\par 36 Terwijl gij het Licht hebt, gelooft in het Licht, opdat gij kinderen des Lichts moogt zijn. Deze dingen sprak Jezus; en weggaande verborg Hij Zich van hen.
\par 37 En hoewel Hij zovele tekenen voor hen gedaan had, nochtans geloofden zij in Hem niet;
\par 38 Opdat het woord van Jesaja, den profeet, vervuld werd, dat hij gesproken heeft: Heere, wie heeft onze prediking geloofd, en wien is de arm des Heeren geopenbaard?
\par 39 Daarom konden zij niet geloven, dewijl Jesaja wederom gezegd heeft:
\par 40 Hij heeft hun ogen verblind, en hun hart verhard; opdat zij met de ogen niet zien, en met het hart niet verstaan, en zij bekeerd worden, en Ik hen geneze.
\par 41 Dit zeide Jesaja, toen hij Zijn heerlijkheid zag, en van Hem sprak.
\par 42 Nochtans geloofden ook zelfs velen uit de oversten in Hem; maar om der Farizeen wil beleden zij het niet; opdat zij uit de synagoge niet zouden geworpen worden.
\par 43 Want zij hadden de eer der mensen lief, meer dan de eer van God.
\par 44 En Jezus riep, en zeide: Die in Mij gelooft, gelooft in Mij niet, maar in Dengene, Die Mij gezonden heeft.
\par 45 En die Mij ziet, die ziet Dengene, Die Mij gezonden heeft.
\par 46 Ik ben een Licht, in de wereld gekomen, opdat een iegelijk, die in Mij gelooft, in de duisternis niet blijve.
\par 47 En indien iemand Mijn woorden gehoord, en niet geloofd zal hebben, Ik oordeel hem niet; want Ik ben niet gekomen, opdat Ik de wereld oordele, maar opdat Ik de wereld zalig make.
\par 48 Die Mij verwerpt, en Mijn woorden niet ontvangt, heeft, die hem oordeelt; het woord, dat Ik gesproken heb, dat zal hem oordelen ten laatsten dage.
\par 49 Want Ik heb uit Mijzelven niet gesproken; maar de Vader, Die Mij gezonden heeft, Die heeft Mij een gebod gegeven, wat Ik zeggen zal, en wat Ik spreken zal.
\par 50 En Ik weet, dat Zijn gebod het eeuwige leven is. Hetgeen Ik dan spreek, dat spreek Ik alzo, gelijk Mij de Vader gezegd heeft.

\chapter{13}

\par 1 En voor het feest van het pascha, Jezus wetende, dat Zijn ure gekomen was, dat Hij uit deze wereld zou overgaan tot den Vader, alzo Hij de Zijnen, die in de wereld waren, liefgehad had, zo heeft Hij hen liefgehad tot het einde.
\par 2 En als het avondmaal gedaan was, toen nu de duivel in het hart van Judas, Simons zoon, Iskariot, gegeven had, dat hij Hem verraden zou),
\par 3 Jezus, wetende, dat de Vader Hem alle dingen in de handen gegeven had, en dat Hij van God uitgegaan was, en tot God heenging,
\par 4 Stond op van het avondmaal, en legde Zijn klederen af, en nemende een linnen doek, omgordde Zichzelven.
\par 5 Daarna goot Hij water in het bekken, en begon de voeten der discipelen te wassen, en af te drogen met den linnen doek, waarmede Hij omgord was.
\par 6 Hij dan kwam tot Simon Petrus; en die zeide tot Hem: Heere, zult Gij mij de voeten wassen?
\par 7 Jezus antwoordde en zeide tot hem: Wat Ik doe, weet gij nu niet, maar gij zult het na dezen verstaan.
\par 8 Petrus zeide tot Hem: Gij zult mijn voeten niet wassen in der eeuwigheid! Jezus antwoordde hem: Indien Ik u niet wasse, gij hebt geen deel met Mij.
\par 9 Simon Petrus zeide tot Hem: Heere, niet alleen mijn voeten, maar ook de handen en het hoofd.
\par 10 Jezus zeide tot hem: Die gewassen is, heeft niet van node, dan de voeten te wassen, maar is geheel rein. En gijlieden zijt rein, doch niet allen.
\par 11 Want Hij wist, wie Hem verraden zou; daarom zeide Hij: Gij zijt niet allen rein.
\par 12 Als Hij dan hun voeten gewassen, en Zijn klederen genomen had, zat Hij wederom aan, en zeide tot hen: Verstaat gij, wat Ik ulieden gedaan heb?
\par 13 Gij heet Mij Meester en Heere; en gij zegt wel, want Ik ben het.
\par 14 Indien dan Ik, de Heere en de Meester, uw voeten gewassen heb, zo zijt gij ook schuldig, elkanders voeten te wassen.
\par 15 Want Ik heb u een voorbeeld gegeven, opdat, gelijkerwijs Ik u gedaan heb, gijlieden ook doet.
\par 16 Voorwaar, voorwaar zeg Ik u: Een dienstknecht is niet meerder dan zijn heer, noch een gezant meerder, dan die hem gezonden heeft.
\par 17 Indien gij deze dingen weet, zalig zijt gij, zo gij dezelve doet.
\par 18 Ik zeg niet van u allen: Ik weet, welke Ik uitverkoren heb; maar dit geschiedt, opdat de Schrift vervuld worde: Die met Mij het brood eet, heeft tegen Mij zijn verzenen opgeheven.
\par 19 Van nu zeg Ik het ulieden, eer het geschied is, opdat, wanneer het geschied zal zijn, gij geloven moogt, dat Ik het ben.
\par 20 Voorwaar, voorwaar zeg Ik u: Zo Ik iemand zende, wie dien ontvangt, die ontvangt Mij, en wie Mij ontvangt, die ontvangt Hem, Die Mij gezonden heeft.
\par 21 Jezus, deze dingen gezegd hebbende, werd ontroerd in den geest, en betuigde, en zeide: Voorwaar, voorwaar, Ik zeg u, dat een van ulieden Mij zal verraden.
\par 22 De discipelen dan zagen op elkander, twijfelende, van wien Hij dat zeide.
\par 23 En een van Zijn discipelen was aanzittende in den schoot van Jezus, welken Jezus liefhad.
\par 24 Simon Petrus dan wenkte dezen, dat hij vragen zou, wie hij toch ware, van welken Hij dit zeide.
\par 25 En deze, vallende op de borst van Jezus, zeide tot Hem: Heere, wie is het?
\par 26 Jezus antwoordde: Deze is het, dien Ik de bete, als Ik ze ingedoopt heb, geven zal. En als Hij de bete ingedoopt had, gaf Hij ze Judas, Simons zoon, Iskariot.
\par 27 En na de bete, toen voer de satan in hem. Jezus dan zeide tot hem: Wat gij doet, doe het haastelijk.
\par 28 En dit verstond niemand dergenen, die aanzaten, waartoe Hij hem dat zeide.
\par 29 Want sommigen meenden, dewijl Judas de beurs had, dat hem Jezus zeide: Koop, hetgeen wij van node hebben tot het feest, of, dat hij den armen wat geven zou.
\par 30 Hij dan, de bete genomen hebbende, ging terstond uit. En het was nacht.
\par 31 Als hij dan uitgegaan was, zeide Jezus: Nu is de Zoon des mensen verheerlijkt, en God is in Hem verheerlijkt.
\par 32 Indien God in Hem verheerlijkt is, zo zal ook God Hem verheerlijken in Zichzelven, en Hij zal Hem terstond verheerlijken.
\par 33 Kinderkens, nog een kleinen tijd ben Ik bij u. Gij zult Mij zoeken, en gelijk Ik den Joden gezegd heb: Waar Ik heenga, kunt gij niet komen; alzo zeg Ik ulieden nu ook.
\par 34 Een nieuw gebod geef Ik u, dat gij elkander liefhebt; gelijk Ik u liefgehad heb, dat ook gij elkander liefhebt.
\par 35 Hieraan zullen zij allen bekennen, dat gij Mijn discipelen zijt, zo gij liefde hebt onder elkander.
\par 36 Simon Petrus zeide tot Hem: Heere, waar gaat Gij heen? Jezus antwoordde hem: Waar Ik heenga, kunt gij Mij nu niet volgen; maar gij zult Mij namaals volgen.
\par 37 Petrus zeide tot Hem: Heere, waarom kan ik U nu niet volgen? Ik zal mijn leven voor U zetten.
\par 38 Jezus antwoordde hem: Zult gij uw leven voor Mij zetten? Voorwaar, voorwaar zeg Ik u: De haan zal niet kraaien, totdat gij Mij driemaal verloochend zult hebben.

\chapter{14}

\par 1 Uw hart worde niet ontroerd; gijlieden gelooft in God, gelooft ook in Mij.
\par 2 In het huis Mijns Vaders zijn vele woningen; anderszins zo zou Ik het u gezegd hebben; Ik ga heen om u plaats te bereiden.
\par 3 En zo wanneer Ik heen zal gegaan zijn, en u plaats zal bereid hebben, zo kome Ik weder en zal u tot Mij nemen, opdat gij ook zijn moogt, waar Ik ben.
\par 4 En waar Ik heenga, weet gij, en den weg weet gij.
\par 5 Thomas zeide tot Hem: Heere, wij weten niet, waar Gij heengaat; en hoe kunnen wij den weg weten?
\par 6 Jezus zeide tot hem: Ik ben de Weg, en de Waarheid, en het Leven. Niemand komt tot den Vader, dan door Mij.
\par 7 Indien gijlieden Mij gekend hadt, zo zoudt gij ook Mijn Vader gekend hebben; en van nu kent gij Hem, en hebt Hem gezien.
\par 8 Filippus zeide tot Hem: Heere, toon ons den Vader, en het is ons genoeg.
\par 9 Jezus zeide tot hem: Ben Ik zo langen tijd met ulieden, en hebt gij Mij niet gekend, Filippus? Die Mij gezien heeft, die heeft den Vader gezien; en hoe zegt gij: Toon ons den Vader?
\par 10 Gelooft gij niet, dat Ik in den Vader ben, en de Vader in Mij is? De woorden, die Ik tot ulieden spreek, spreek Ik van Mijzelven niet, maar de Vader, Die in Mij blijft, Dezelve doet de werken.
\par 11 Gelooft Mij, dat Ik in den Vader ben en de Vader in Mij is; en indien niet, zo gelooft Mij om de werken zelve.
\par 12 Voorwaar, voorwaar zeg Ik ulieden: Die in Mij gelooft, de werken, die Ik doe, zal hij ook doen, en zal meerder doen, dan deze; want Ik ga heen tot Mijn Vader.
\par 13 En zo wat gij begeren zult in Mijn Naam, dat zal Ik doen; opdat de Vader in den Zoon verheerlijkt worde.
\par 14 Zo gij iets begeren zult in Mijn Naam, Ik zal het doen.
\par 15 Indien gij Mij liefhebt, zo bewaart Mijn geboden.
\par 16 En Ik zal den Vader bidden, en Hij zal u een anderen Trooster geven, opdat Hij bij u blijve in der eeuwigheid;
\par 17 Namelijk den Geest der waarheid, Welken de wereld niet kan ontvangen; want zij ziet Hem niet, en kent Hem niet; maar gij kent Hem; want Hij blijft bij ulieden, en zal in u zijn.
\par 18 Ik zal u geen wezen laten; Ik kom weder tot u.
\par 19 Nog een kleinen tijd, en de wereld zal Mij niet meer zien; maar gij zult Mij zien; want Ik leef, en gij zult leven.
\par 20 In dien dag zult gij bekennen, dat Ik in Mijn Vader ben, en gij in Mij, en Ik in u.
\par 21 Die Mijn geboden heeft, en dezelve bewaart, die is het, die Mij liefheeft; en die Mij liefheeft, zal van Mijn Vader geliefd worden; en Ik zal hem liefhebben, en Ik zal Mijzelven aan hem openbaren.
\par 22 Judas, niet de Iskariot, zeide tot Hem: Heere, wat is het, dat Gij Uzelven aan ons zult openbaren, en niet aan de wereld?
\par 23 Jezus antwoordde en zeide tot hem: Zo iemand Mij liefheeft, die zal Mijn woord bewaren; en Mijn Vader zal hem liefhebben, en Wij zullen tot hem komen, en zullen woning bij hem maken.
\par 24 Die Mij niet liefheeft, die bewaart Mijn woorden niet; en het woord dat gijlieden hoort, is het Mijne niet, maar des Vaders, Die Mij gezonden heeft.
\par 25 Deze dingen heb Ik tot u gesproken, bij u blijvende.
\par 26 Maar de Trooster, de Heilige Geest, Welken de Vader zenden zal in Mijn Naam, Die zal u alles leren, en zal u indachtig maken alles, wat Ik u gezegd heb.
\par 27 Vrede laat Ik u, Mijn vrede geef Ik u; niet gelijkerwijs de wereld hem geeft, geef Ik hem u. Uw hart worde niet ontroerd en zij niet versaagd.
\par 28 Gij hebt gehoord, dat Ik tot u gezegd heb: Ik ga heen, en kom weder tot u. Indien gij Mij liefhadt, zo zoudt gij u verblijden, omdat Ik gezegd heb: Ik ga heen tot den Vader; want Mijn Vader is meerder dan Ik.
\par 29 En nu heb Ik het u gezegd, eer het geschied is; opdat, wanneer het geschied zal zijn, gij geloven moogt.
\par 30 Ik zal niet meer veel met u spreken; want de overste dezer wereld komt, en heeft aan Mij niets.
\par 31 Maar opdat de wereld wete, dat Ik den Vader liefheb, en alzo doe, gelijkerwijs Mij de Vader geboden heeft. Staat op, laat ons van hier gaan.

\chapter{15}

\par 1 Ik ben de ware Wijnstok, en Mijn Vader is de Landman.
\par 2 Alle rank, die in Mij geen vrucht draagt, die neemt Hij weg; en al wie vrucht draagt, die reinigt Hij, opdat zij meer vrucht drage.
\par 3 Gijlieden zijt nu rein om het woord, dat Ik tot u gesproken heb.
\par 4 Blijft in Mij, en Ik in u. Gelijkerwijs de rank geen vrucht kan dragen van zichzelve, zo zij niet in den wijnstok blijft; alzo ook gij niet, zo gij in Mij niet blijft.
\par 5 Ik ben de Wijnstok, en gij de ranken; die in Mij blijft, en Ik in hem, die draagt veel vrucht; want zonder Mij kunt gij niets doen.
\par 6 Zo iemand in Mij niet blijft, die is buiten geworpen, gelijkerwijs de rank, en is verdord; en men vergadert dezelve, en men werpt ze in het vuur, en zij worden verbrand.
\par 7 Indien gij in Mij blijft, en Mijn woorden in u blijven, zo wat gij wilt, zult gij begeren, en het zal u geschieden.
\par 8 Hierin is Mijn Vader verheerlijkt, dat gij veel vrucht draagt; en gij zult Mijn discipelen zijn.
\par 9 Gelijkerwijs de Vader Mij liefgehad heeft, heb Ik ook u liefgehad; blijft in deze Mijn liefde.
\par 10 Indien gij Mijn geboden bewaart, zo zult gij in Mijn liefde blijven; gelijkerwijs Ik de geboden Mijns Vaders bewaard heb, en blijf in Zijn liefde.
\par 11 Deze dingen heb Ik tot u gesproken, opdat Mijn blijdschap in u blijve, en uw blijdschap vervuld worde.
\par 12 Dit is Mijn gebod, dat gij elkander liefhebt, gelijkerwijs Ik u liefgehad heb.
\par 13 Niemand heeft meerder liefde dan deze, dat iemand zijn leven zette voor zijn vrienden.
\par 14 Gij zijt Mijn vrienden, zo gij doet wat Ik u gebiede.
\par 15 Ik heet u niet meer dienstknechten; want de dienstknecht weet niet, wat zijn heer doet; maar Ik heb u vrienden genoemd; want al wat Ik van Mijn Vader gehoord heb, dat heb Ik u bekend gemaakt.
\par 16 Gij hebt Mij niet uitverkoren, maar Ik heb u uitverkoren, en Ik heb u gesteld, dat gij zoudt heengaan en vrucht dragen, en dat uw vrucht blijve; opdat, zo wat gij van den Vader begeren zult in Mijn Naam, Hij u dat geve.
\par 17 Dit gebied Ik u, opdat gij elkander liefhebt.
\par 18 Indien u de wereld haat, zo weet, dat zij Mij eer dan u gehaat heeft.
\par 19 Indien gij van de wereld waart, zo zou de wereld het hare liefhebben; doch omdat gij van de wereld niet zijt, maar Ik u uit de wereld heb uitverkoren, daarom haat u de wereld.
\par 20 Gedenk des woords, dat Ik u gezegd heb: Een dienstknecht is niet meerder dan zijn heer. Indien zij Mij vervolgd hebben, zij zullen ook u vervolgen; indien zij Mijn woord bewaard hebben, zij zullen ook het uwe bewaren.
\par 21 Maar al deze dingen zullen zij doen om Mijns Naams wil, omdat zij Hem niet kennen, Die Mij gezonden heeft.
\par 22 Indien Ik niet gekomen ware, en tot hen gesproken had, zij hadden geen zonde; maar nu hebben zij geen voorwendsel voor hun zonde.
\par 23 Die Mij haat, die haat ook Mijn Vader.
\par 24 Indien Ik de werken onder hen niet had gedaan, die niemand anders gedaan heeft, zij hadden geen zonde; maar nu hebben zij ze gezien, en beiden Mij en Mijn Vader gehaat.
\par 25 Maar dit geschiedt, opdat het woord vervuld worde, dat in hun wet geschreven is: Zij hebben mij zonder oorzaak gehaat.
\par 26 Maar wanneer de Trooster zal gekomen zijn, Dien Ik u zenden zal van den Vader, namelijk de Geest der waarheid, Die van den Vader uitgaat, Die zal van Mij getuigen.
\par 27 En gij zult ook getuigen, want gij zijt van den beginne met Mij geweest.

\chapter{16}

\par 1 Deze dingen heb Ik tot u gesproken, opdat gij niet geergerd wordt.
\par 2 Zij zullen u uit de synagogen werpen; ja, de ure komt, dat een iegelijk, die u zal doden, zal menen Gode een dienst te doen.
\par 3 En deze dingen zullen zij u doen, omdat zij den Vader niet gekend hebben, noch Mij.
\par 4 Maar deze dingen heb Ik tot u gesproken, opdat, wanneer de ure zal gekomen zijn, gij dezelve moogt gedenken, dat Ik ze u gezegd heb; doch deze dingen heb Ik u van het begin niet gezegd, omdat Ik bij ulieden was.
\par 5 En nu ga Ik heen tot Dengene, die Mij gezonden heeft, en niemand van u vraagt Mij: Waar gaat Gij henen?
\par 6 Maar omdat Ik deze dingen tot u gesproken heb, zo heeft de droefheid uw hart vervuld.
\par 7 Doch Ik zeg u de waarheid: Het is u nut, dat Ik wegga; want indien Ik niet wegga, zo zal de Trooster tot u niet komen; maar indien Ik heenga, zo zal Ik Hem tot u zenden.
\par 8 En Die gekomen zijnde, zal de wereld overtuigen van zonde, en van gerechtigheid, en van oordeel:
\par 9 Van zonde, omdat zij in Mij niet geloven;
\par 10 En van gerechtigheid, omdat Ik tot Mijn Vader heenga, en gij zult Mij niet meer zien;
\par 11 En van oordeel, omdat de overste dezer wereld geoordeeld is.
\par 12 Nog vele dingen heb Ik u te zeggen, doch gij kunt die nu niet dragen.
\par 13 Maar wanneer Die zal gekomen zijn, namelijk de Geest der waarheid, Hij zal u in al de waarheid leiden; want Hij zal van Zichzelven niet spreken, maar zo wat Hij zal gehoord hebben, zal Hij spreken, en de toekomende dingen zal Hij u verkondigen.
\par 14 Die zal Mij verheerlijken; want Hij zal het uit het Mijne nemen, en zal het u verkondigen.
\par 15 Al wat de Vader heeft, is Mijn; daarom heb Ik gezegd, dat Hij het uit het Mijne zal nemen, en u verkondigen.
\par 16 Een kleinen tijd, en gij zult Mij niet zien; en wederom een kleinen tijd, en gij zult Mij zien, want Ik ga heen tot den Vader.
\par 17 Sommigen dan uit Zijn discipelen zeiden tot elkander: Wat is dit, dat Hij tot ons zegt: Een kleinen tijd, en gij zult Mij niet zien; en wederom een kleinen tijd, en gij zult Mij zien; en: Want Ik ga heen tot den Vader?
\par 18 Zij zeiden dan: Wat is dit, dat Hij zegt: Een kleinen tijd? Wij weten niet, wat Hij zegt.
\par 19 Jezus dan bekende, dat zij Hem wilden vragen, en zeide tot hen: Vraagt gij daarvan onder elkander, dat Ik gezegd heb: Een kleinen tijd, en gij zult Mij niet zien, en wederom een kleinen tijd, en gij zult Mij zien?
\par 20 Voorwaar, voorwaar, Ik zeg u, dat gij zult schreien, en klagelijk wenen, maar de wereld zal zich verblijden; en gij zult bedroefd zijn, maar uw droefheid zal tot blijdschap worden.
\par 21 Een vrouw, wanneer zij baart, heeft droefheid, dewijl haar ure gekomen is; maar wanneer zij het kindeken gebaard heeft, zo gedenkt zij de benauwdheid niet meer, om de blijdschap, dat een mens ter wereld geboren is.
\par 22 En gij dan hebt nu wel droefheid; maar Ik zal u wederom zien, en uw hart zal zich verblijden, en niemand zal uw blijdschap van u wegnemen.
\par 23 En in dien dag zult gij Mij niets vragen. Voorwaar, voorwaar Ik zeg u: Al wat gij den Vader zult bidden in Mijn Naam, dat zal Hij u geven.
\par 24 Tot nog toe hebt gij niet gebeden in Mijn Naam; bidt, en gij zult ontvangen, opdat uw blijdschap vervuld zij.
\par 25 Deze dingen heb Ik door gelijkenissen tot u gesproken; maar de ure komt, dat Ik niet meer door gelijkenissen tot u spreken zal, maar u vrijuit van den Vader zal verkondigen.
\par 26 In dien dag zult gij in Mijn Naam bidden; en Ik zeg u niet, dat Ik den Vader voor u bidden zal;
\par 27 Want de Vader Zelf heeft u lief, dewijl gij Mij liefgehad hebt, en hebt geloofd, dat Ik van God ben uitgegaan.
\par 28 Ik ben van den Vader uitgegaan, en ben in de wereld gekomen; wederom verlaat Ik de wereld, en ga heen tot den Vader.
\par 29 Zijn discipelen zeiden tot Hem: Zie, nu spreekt Gij vrijuit, en zegt geen gelijkenis.
\par 30 Nu weten wij, dat Gij alle dingen weet, en Gij hebt niet van node, dat U iemand vrage. Hierom geloven wij, dat Gij van God uitgegaan zijt.
\par 31 Jezus antwoordde hun: Gelooft gij nu?
\par 32 Ziet, de ure komt, en is nu gekomen, dat gij zult verstrooid worden, een iegelijk naar het zijne, en gij Mij alleen zult laten; en nochtans ben Ik niet alleen; want de Vader is met Mij.
\par 33 Deze dingen heb Ik tot u gesproken, opdat gij in Mij vrede hebt. In de wereld zult gij verdrukking hebben, maar hebt goeden moed, Ik heb de wereld overwonnen.

\chapter{17}

\par 1 Dit heeft Jezus gesproken, en Hij hief Zijn ogen op naar den hemel, en zeide: Vader, de ure is gekomen, verheerlijk Uw Zoon, opdat ook Uw Zoon U verheerlijke.
\par 2 Gelijkerwijs Gij Hem macht gegeven hebt over alle vlees, opdat al wat Gij Hem gegeven hebt, Hij hun het eeuwige leven geve.
\par 3 En dit is het eeuwige leven, dat zij U kennen, den enigen waarachtigen God, en Jezus Christus, Dien Gij gezonden hebt.
\par 4 Ik heb U verheerlijkt op de aarde; Ik heb voleindigd het werk, dat Gij Mij gegeven hebt om te doen;
\par 5 En nu verheerlijk Mij, Gij Vader, bij Uzelven, met de heerlijkheid, die Ik bij U had, eer de wereld was.
\par 6 Ik heb Uw Naam geopenbaard den mensen, die Gij Mij uit de wereld gegeven hebt. Zij waren Uw, en Gij hebt Mij dezelve gegeven; en zij hebben Uw woord bewaard.
\par 7 Nu hebben zij bekend, dat alles, wat Gij Mij gegeven hebt, van U is.
\par 8 Want de woorden, die Gij Mij gegeven hebt, heb Ik hun gegeven, en zij hebben ze ontvangen, en zij hebben waarlijk bekend, dat Ik van U uitgegaan ben, en hebben geloofd, dat Gij Mij gezonden hebt.
\par 9 Ik bid voor hen; Ik bid niet voor de wereld, maar voor degenen, die Gij Mij gegeven hebt, want zij zijn Uw.
\par 10 En al het Mijne is Uw, en het Uwe is Mijn; en Ik ben in hen verheerlijkt.
\par 11 En Ik ben niet meer in de wereld, maar deze zijn in de wereld, en Ik kome tot U, Heilige Vader, bewaar ze in Uw Naam, die Gij Mij gegeven hebt, opdat zij een zijn, gelijk als Wij.
\par 12 Toen Ik met hen in de wereld was, bewaarde Ik ze in Uw Naam. Die Gij Mij gegeven hebt, heb Ik bewaard, en niemand uit hen is verloren gegaan, dan de zoon der verderfenis, opdat de Schrift vervuld worde.
\par 13 Maar nu kom Ik tot U, en spreek dit in de wereld, opdat zij Mijn blijdschap vervuld mogen hebben in zichzelven.
\par 14 Ik heb hun Uw woord gegeven; en de wereld heeft ze gehaat, omdat zij van de wereld niet zijn, gelijk als Ik van de wereld niet ben.
\par 15 Ik bid niet, dat Gij hen uit de wereld wegneemt, maar dat Gij hen bewaart van den boze.
\par 16 Zij zijn niet van de wereld, gelijkerwijs Ik van de wereld niet ben.
\par 17 Heilig ze in Uw waarheid; Uw woord is de waarheid.
\par 18 Gelijkerwijs Gij Mij gezonden hebt in de wereld, alzo heb Ik hen ook in de wereld gezonden.
\par 19 En Ik heilige Mijzelven voor hen, opdat ook zij geheiligd mogen zijn in waarheid.
\par 20 En Ik bid niet alleen voor dezen, maar ook voor degenen, die door hun woord in Mij geloven zullen.
\par 21 Opdat zij allen een zijn, gelijkerwijs Gij, Vader, in Mij, en Ik in U, dat ook zij in Ons een zijn; opdat de wereld gelove, dat Gij Mij gezonden hebt.
\par 22 En Ik heb hun de heerlijkheid gegeven, die Gij Mij gegeven hebt; opdat zij een zijn, gelijk als Wij Een zijn;
\par 23 Ik in hen, en Gij in Mij; opdat zij volmaakt zijn in een, en opdat de wereld bekenne, dat Gij Mij gezonden hebt, en hen liefgehad hebt, gelijk Gij Mij liefgehad hebt.
\par 24 Vader, Ik wil, dat waar Ik ben, ook die bij Mij zijn, die Gij Mij gegeven hebt; opdat zij Mijn heerlijkheid mogen aanschouwen, die Gij Mij gegeven hebt; want Gij hebt Mij liefgehad, voor de grondlegging der wereld.
\par 25 Rechtvaardige Vader, de wereld heeft U niet gekend; maar Ik heb U gekend, en dezen hebben bekend, dat Gij Mij gezonden hebt.
\par 26 En Ik heb hun Uw Naam bekend gemaakt, en zal Hem bekend maken; opdat de liefde, waarmede Gij Mij liefgehad hebt, in hen zij, en Ik in hen.

\chapter{18}

\par 1 Jezus, dit gezegd hebbende, ging uit met Zijn discipelen over de beek Kedron, waar een hof was, in welken Hij ging, en Zijn discipelen.
\par 2 En Judas, die Hem verried, wist ook die plaats, dewijl Jezus aldaar dikwijls vergaderd was geweest met Zijn discipelen.
\par 3 Judas dan, genomen hebbende de bende krijgsknechten en enige dienaars van de overpriesters en Farizeen, kwam aldaar met lantaarnen, en fakkelen, en wapenen.
\par 4 Jezus dan, wetende alles, wat over Hem komen zou, ging uit, en zeide tot hen: Wien zoekt gij?
\par 5 Zij antwoordden Hem: Jezus den Nazarener. Jezus zeide tot hen: Ik ben het. En Judas, die Hem verried, stond ook bij hen.
\par 6 Als Hij dan tot hen zeide: Ik ben het; gingen zij achterwaarts, en vielen ter aarde.
\par 7 Hij vraagde hun dan wederom: Wien zoekt gij? En zij zeiden: Jezus den Nazarener.
\par 8 Jezus antwoordde: Ik heb u gezegd, dat Ik het ben. Indien gij dan Mij zoekt, zo laat dezen heengaan.
\par 9 Opdat het woord vervuld zou worden, dat Hij gezegd had: Uit degenen, die Gij Mij gegeven hebt, heb Ik niemand verloren.
\par 10 Simon Petrus dan, hebbende een zwaard, trok hetzelve uit, en sloeg des hogepriesters dienstknecht, en hieuw zijn rechteroor af. En de naam van den dienstknecht was Malchus.
\par 11 Jezus dan zeide tot Petrus: Steek uw zwaard in de schede. Den drinkbeker, dien Mij de Vader gegeven heeft, zal Ik dien niet drinken?
\par 12 De bende dan, en de overste over duizend, en de dienaars der Joden namen Jezus gezamenlijk, en bonden Hem;
\par 13 En leidden Hem henen, eerst tot Annas; want hij was de vrouws vader van Kajafas, welke deszelven jaars hogepriester was.
\par 14 Kajafas nu was degene, die den Joden geraden had, dat het nut was, dat een Mens voor het volk stierve.
\par 15 En Simon Petrus volgde Jezus, en een ander discipel. Deze discipel nu was den hogepriester bekend, en ging met Jezus in des hogepriesters zaal.
\par 16 En Petrus stond buiten aan de deur. De andere discipel dan, die den hogepriester bekend was, ging uit, en sprak met de deurwaarster, en bracht Petrus in.
\par 17 De dienstmaagd dan, die de deurwaarster was, zeide tot Petrus: Zijt ook gij niet uit de discipelen van dezen Mens? Hij zeide: Ik ben niet.
\par 18 En de dienstknechten en de dienaars stonden, hebbende een kolenvuur gemaakt, omdat het koud was, en warmden zich. Petrus stond bij hen, en warmde zich.
\par 19 De hogepriester dan vraagde Jezus van Zijn discipelen, en van Zijn leer.
\par 20 Jezus antwoordde hem: Ik heb vrijuit gesproken tot de wereld; Ik heb allen tijd geleerd in de synagoge en in den tempel, waar de Joden van alle plaatsen samenkomen; en in het verborgen heb Ik niets gesproken.
\par 21 Wat ondervraagt gij Mij? Ondervraag degenen, die het gehoord hebben, wat Ik tot hen gesproken heb; zie, dezen weten, wat Ik gezegd heb.
\par 22 En als Hij dit zeide, gaf een van de dienaren, die daarbij stond, Jezus een kinnebakslag, zeggende: Antwoordt Gij alzo den hogepriester?
\par 23 Jezus antwoordde hem: Indien Ik kwalijk gesproken heb, betuig van het kwade; en indien wel, waarom slaat gij Mij?
\par 24 (Annas dan had Hem gebonden gezonden tot Kajafas, den hogepriester.)
\par 25 En Simon Petrus stond en warmde zich. Zij zeiden dan tot hem: Zijt gij ook niet uit Zijn discipelen? Hij loochende het, en zeide: Ik ben niet.
\par 26 Een van de dienstknechten des hogepriesters, die maagschap was van dengene, dien Petrus het oor afgehouwen had, zeide: Heb ik u niet gezien in den hof met Hem?
\par 27 Petrus dan loochende het wederom. En terstond kraaide de haan.
\par 28 Zij dan leidden Jezus van Kajafas in het rechthuis. En het was's morgens vroeg; en zij gingen niet in het rechthuis, opdat zij niet verontreinigd zouden worden, maar opdat zij het pascha eten mochten.
\par 29 Pilatus dan ging tot hen uit, en zeide: Wat beschuldiging brengt gij tegen dezen Mens?
\par 30 Zij antwoordden en zeiden tot hem: Indien Deze geen kwaaddoener ware, zo zouden wij Hem u niet overgeleverd hebben.
\par 31 Pilatus dan zeide tot hen: Neemt gij Hem, en oordeelt Hem naar uw wet. De Joden dan zeiden tot hem: Het is ons niet geoorloofd iemand te doden.
\par 32 Opdat het woord van Jezus vervuld wierd, dat Hij gezegd had, betekenende, hoedanigen dood Hij sterven zoude.
\par 33 Pilatus dan ging wederom in het rechthuis, en riep Jezus, en zeide tot Hem: Zijt Gij de Koning der Joden?
\par 34 Jezus antwoordde hem: Zegt gij dit van uzelven, of hebben het u anderen van Mij gezegd?
\par 35 Pilatus antwoordde: Ben ik een Jood? Uw volk en de overpriesters hebben U aan mij overgeleverd; wat hebt Gij gedaan?
\par 36 Jezus antwoordde: Mijn Koninkrijk is niet van deze wereld. Indien Mijn Koninkrijk van deze wereld ware, zo zouden Mijn dienaren gestreden hebben, opdat Ik den Joden niet ware overgeleverd; maar nu is Mijn Koninkrijk niet van hier.
\par 37 Pilatus dan zeide tot Hem: Zijt Gij dan een Koning? Jezus antwoordde: Gij zegt, dat Ik een Koning ben. Hiertoe ben Ik geboren en hiertoe ben Ik in de wereld gekomen, opdat Ik der waarheid getuigenis geven zou. Een iegelijk, die uit de waarheid is, hoort Mijn stem.
\par 38 Pilatus zeide tot Hem: Wat is waarheid? En als hij dat gezegd had, ging hij wederom uit tot de Joden, en zeide tot hen: Ik vind geen schuld in Hem.
\par 39 Doch gij hebt een gewoonte, dat ik u op het pascha een loslate. Wilt gij dan, dat ik u den Koning der Joden loslate?
\par 40 Zij dan riepen allen wederom, zeggende: Niet Dezen, maar Bar-abbas! En Bar-abbas was een moordenaar.

\chapter{19}

\par 1 Toen nam Pilatus dan Jezus, en geselde Hem.
\par 2 En de krijgsknechten, een kroon van doornen gevlochten hebbende, zetten die op Zijn hoofd, en wierpen Hem een purperen kleed om;
\par 3 En zeiden: Wees gegroet, Gij Koning der Joden! En zij gaven Hem kinnebakslagen.
\par 4 Pilatus dan kwam wederom uit, en zeide tot hen: Ziet, ik breng Hem tot ulieden uit, opdat gij wetet, dat ik in Hem geen schuld vinde.
\par 5 Jezus dan kwam uit, dragende de doornenkroon, en het purperen kleed. En Pilatus zeide tot hen: Ziet, de Mens!
\par 6 Als Hem dan de overpriesters en de dienaars zagen, riepen zij, zeggende: Kruis Hem, kruis Hem; Pilatus zeide tot hen: Neemt gijlieden Hem en kruist Hem; want ik vind in Hem geen schuld.
\par 7 De Joden antwoordden hem: Wij hebben een wet, en naar onze wet moet Hij sterven, want Hij heeft Zichzelven Gods Zoon gemaakt.
\par 8 Toen Pilatus dan dit woord hoorde, werd hij meer bevreesd;
\par 9 En ging wederom in het rechthuis, en zeide tot Jezus: Van waar zijt Gij? Maar Jezus gaf hem geen antwoord.
\par 10 Pilatus dan zeide tot Hem: Spreekt Gij tot mij niet? Weet Gij niet, dat ik macht heb U te kruisigen, en macht heb U los te laten?
\par 11 Jezus antwoordde: Gij zoudt geen macht hebben tegen Mij, indien het u niet van boven gegeven ware; daarom die Mij aan u heeft overgeleverd, heeft groter zonde.
\par 12 Van toen af zocht Pilatus Hem los te laten; maar de Joden riepen, zeggende: Indien gij Dezen loslaat, zo zijt gij des keizers vriend niet; een iegelijk, die zichzelven koning maakt, wederspreekt den keizer.
\par 13 Als Pilatus dan dit woord hoorde, bracht hij Jezus uit, en zat neder op den rechterstoel, in de plaats, genaamd Lithostrotos, en in het Hebreeuws Gabbatha.
\par 14 En het was de voorbereiding van het pascha, en omtrent de zesde ure; en hij zeide tot de Joden: Ziet, uw Koning!
\par 15 Maar zij riepen: Neem weg, neem weg, kruis Hem! Pilatus zeide tot hen: Zal ik uw Koning kruisigen? De overpriesters antwoordden: Wij hebben geen koning, dan den keizer.
\par 16 Toen gaf hij Hem dan hun over, opdat Hij gekruist zou worden. En zij namen Jezus, en leidden Hem weg.
\par 17 En Hij, dragende Zijn kruis, ging uit naar de plaats, genaamd Hoofdschedelplaats, welke in het Hebreeuws genaamd wordt Golgotha;
\par 18 Alwaar zij Hem kruisten, en met Hem twee anderen, aan elke zijde een, en Jezus in het midden.
\par 19 En Pilatus schreef ook een opschrift, en zette dat op het kruis; en er was geschreven: JEZUS, DE NAZARENER, DE KONING DER JODEN.
\par 20 Dit opschrift dan lazen velen van de Joden; want de plaats, waar Jezus gekruist werd, was nabij de stad; en het was geschreven in het Hebreeuws, in het Grieks, en in het Latijn.
\par 21 De overpriesters dan der Joden zeiden tot Pilatus: Schrijf niet: De Koning der Joden; maar, dat Hij gezegd heeft: Ik ben de Koning der Joden.
\par 22 Pilatus antwoordde: Wat ik geschreven heb, dat heb ik geschreven.
\par 23 De krijgsknechten dan, als zij Jezus gekruist hadden, namen Zijn klederen, (en maakten vier delen, voor elken krijgsknecht een deel) en den rok. De rok nu was zonder naad, van boven af geheel geweven.
\par 24 Zij dan zeiden tot elkander: Laat ons dien niet scheuren, maar laat ons daarover loten, wiens die zijn zal; opdat de Schrift vervuld worde, die zegt: Zij hebben Mijn klederen onder zich verdeeld, en over Mijn kleding hebben zij het lot geworpen. Dit hebben dan de krijgsknechten gedaan.
\par 25 En bij het kruis van Jezus stonden Zijn moeder en Zijner moeders zuster, Maria, de vrouw van Klopas, en Maria Magdalena.
\par 26 Jezus nu, ziende Zijn moeder, en den discipel, dien Hij liefhad, daarbij staande, zeide tot Zijn moeder: Vrouw, zie, uw zoon.
\par 27 Daarna zeide Hij tot den discipel: Zie, uw moeder. En van die ure aan nam haar de discipel in zijn huis.
\par 28 Hierna Jezus, wetende, dat nu alles volbracht was, opdat de Schrift zou vervuld worden, zeide: Mij dorst.
\par 29 Daar stond dan een vat vol ediks, en zij vulden een spons met edik, en omlegden ze met hysop, en brachten ze aan Zijn mond.
\par 30 Toen Jezus dan den edik genomen had, zeide Hij: Het is volbracht! En het hoofd buigende, gaf den geest.
\par 31 De Joden dan, opdat de lichamen niet aan het kruis zouden blijven op den sabbat, dewijl het de voorbereiding was (want die dag des sabbats was groot), baden Pilatus, dat hun benen zouden gebroken, en zij weggenomen worden.
\par 32 De krijgsknechten dan kwamen, en braken wel de benen des eersten, en des anderen, die met Hem gekruist was;
\par 33 Maar komende tot Jezus, als zij zagen, dat Hij nu gestorven was, zo braken zij Zijn benen niet.
\par 34 Maar een der krijgsknechten doorstak Zijn zijde met een speer, en terstond kwam er bloed en water uit.
\par 35 En die het gezien heeft, die heeft het getuigd, en zijn getuigenis is waarachtig; en hij weet, dat hij zegt, hetgeen waar is, opdat ook gij geloven moogt.
\par 36 Want deze dingen zijn geschied, opdat de Schrift vervuld worde: Geen been van Hem zal verbroken worden.
\par 37 En wederom zegt een andere Schrift: Zij zullen zien, in Welken zij gestoken hebben.
\par 38 En daarna Jozef van Arimathea (die een discipel van Jezus was, maar bedekt om de vreze der Joden), bad Pilatus, dat hij mocht het lichaam van Jezus wegnemen; en Pilatus liet het toe. Hij dan ging en nam het lichaam van Jezus weg.
\par 39 En Nicodemus kwam ook (die des nachts tot Jezus eerst gekomen was), brengende een mengsel van mirre en aloe; omtrent honderd ponden gewichts.
\par 40 Zij namen dan het lichaam van Jezus, en bonden dat in linnen doeken met de specerijen, gelijk de Joden de gewoonte hebben van begraven.
\par 41 En er was in de plaats, waar Hij gekruist was, een hof, en in den hof een nieuw graf, in hetwelk nog nooit iemand gelegd was geweest.
\par 42 Aldaar dan legden zij Jezus, om de voorbereiding der Joden, overmits het graf nabij was.

\chapter{20}

\par 1 En op den eersten dag der week ging Maria Magdalena vroeg, als het nog duister was, naar het graf; en zag den steen van het graf weggenomen.
\par 2 Zij liep dan, en kwam tot Simon Petrus en tot den anderen discipel, welken Jezus liefhad, en zeide tot hen: Zij hebben den Heere weggenomen uit het graf, en wij weten niet, waar zij Hem gelegd hebben.
\par 3 Petrus dan ging uit, en de andere discipel, en zij kwamen tot het graf.
\par 4 En deze twee liepen tegelijk; en de andere discipel liep vooruit, sneller dan Petrus, en kwam eerst tot het graf.
\par 5 En als hij nederbukte, zag hij de doeken liggen; nochtans ging hij er niet in.
\par 6 Simon Petrus dan kwam en volgde hem, en ging in het graf, en zag de doeken liggen.
\par 7 En den zweetdoek, die op Zijn hoofd geweest was, zag hij niet bij de doeken liggen, maar in het bijzonder in een andere plaats samengerold.
\par 8 Toen ging dan ook de andere discipel er in, die eerst tot het graf gekomen was, en zag het, en geloofde.
\par 9 Want zij wisten nog de Schrift niet, dat Hij van de doden moest opstaan.
\par 10 De discipelen dan gingen wederom naar huis.
\par 11 En Maria stond buiten bij het graf, wenende. Als zij dan weende, bukte zij in het graf;
\par 12 En zag twee engelen in witte klederen zitten, een aan het hoofd, en een aan de voeten, waar het lichaam van Jezus gelegen had.
\par 13 En die zeiden tot haar: Vrouw! wat weent gij? Zij zeide tot hen: Omdat zij mijn Heere weggenomen hebben, en ik weet niet, waar zij Hem gelegd hebben.
\par 14 En als zij dit gezegd had, keerde zij zich achterwaarts, en zag Jezus staan, en zij wist niet, dat het Jezus was.
\par 15 Jezus zeide tot haar: Vrouw, wat weent gij? Wien zoekt gij? Zij, menende, dat het de hovenier was, zeide tot Hem: Heere, zo gij Hem weg gedragen hebt, zeg mij, waar gij Hem gelegd hebt, en ik zal Hem wegnemen.
\par 16 Jezus zeide tot haar: Maria! Zij, zich omkerende, zeide tot Hem: Rabbouni, hetwelk is gezegd, Meester.
\par 17 Jezus zeide tot haar: Raak Mij niet aan, want Ik ben nog niet opgevaren tot Mijn Vader; maar ga heen tot Mijn broeders, en zeg hun: Ik vare op tot Mijn Vader en uw Vader, en tot Mijn God en uw God.
\par 18 Maria Magdalena ging en boodschapte den discipelen, dat zij den Heere gezien had, en dat Hij haar dit gezegd had.
\par 19 Als het dan avond was, op denzelven eersten dag der week, en als de deuren gesloten waren, waar de discipelen vergaderd waren om de vreze der Joden, kwam Jezus en stond in het midden, en zeide tot hen: Vrede zij ulieden!
\par 20 En dit gezegd hebbende, toonde Hij hun Zijn handen en Zijn zijde. De discipelen dan werden verblijd, als zij den Heere zagen.
\par 21 Jezus dan zeide wederom tot hen: Vrede zij ulieden, gelijkerwijs Mij de Vader gezonden heeft, zende Ik ook ulieden.
\par 22 En als Hij dit gezegd had, blies Hij op hen, en zeide tot hen: Ontvangt den Heiligen Geest.
\par 23 Zo gij iemands zonden vergeeft, dien worden zij vergeven; zo gij iemands zonden houdt, dien zijn zij gehouden.
\par 24 En Thomas, een van de twaalven, gezegd Didymus, was met hen niet, toen Jezus daar kwam.
\par 25 De andere discipelen dan zeiden tot hem: Wij hebben den Heere gezien. Doch hij zeide tot hen: Indien ik in Zijn handen niet zie het teken der nagelen, en mijn vinger steke in het teken der nagelen, en steke mijn hand in Zijn zijde, ik zal geenszins geloven.
\par 26 En na acht dagen waren Zijn discipelen wederom binnen, en Thomas met hen; en Jezus kwam, als de deuren gesloten waren, en stond in het midden, en zeide: Vrede zij ulieden!
\par 27 Daarna zeide Hij tot Thomas: Breng uw vinger hier, en zie Mijn handen, en breng uw hand, en steek ze in Mijn zijde; en zijt niet ongelovig, maar gelovig.
\par 28 En Thomas antwoordde en zeide tot Hem: Mijn Heere en mijn God!
\par 29 Jezus zeide tot hem: Omdat gij Mij gezien hebt, Thomas, zo hebt gij geloofd; zalig zijn zij, die niet zullen gezien hebben, en nochtans zullen geloofd hebben.
\par 30 Jezus dan heeft nog wel vele andere tekenen in de tegenwoordigheid Zijner discipelen gedaan, die niet zijn geschreven in dit boek;
\par 31 Maar deze zijn geschreven, opdat gij gelooft, dat Jezus is de Christus, de Zone Gods; en opdat gij, gelovende, het leven hebt in Zijn Naam.

\chapter{21}

\par 1 Na dezen openbaarde Jezus Zichzelven wederom den discipelen aan de zee van Tiberias. En Hij openbaarde Zich aldus:
\par 2 Er waren te zamen Simon Petrus, en Thomas, gezegd Didymus, en Nathanael, die van Kana in Galilea was, en de zonen van Zebedeus, en twee anderen van Zijn discipelen.
\par 3 Simon Petrus zeide tot hen: Ik ga vissen. Zij zeiden tot hem: Wij gaan ook met u. Zij gingen uit, en traden terstond in het schip; en in dien nacht vingen zij niets.
\par 4 En als het nu morgenstond geworden was, stond Jezus op den oever; doch de discipelen wisten niet, dat het Jezus was.
\par 5 Jezus dan zeide tot hen: Kinderkens, hebt gij niet enige toespijs? Zij antwoordden Hem: Neen.
\par 6 En Hij zeide tot hen: Werpt het net aan de rechterzijde van het schip, en gij zult vinden. Zij wierpen het dan, en konden hetzelve niet meer trekken vanwege de menigte der vissen.
\par 7 De discipel dan, welken Jezus liefhad, zeide tot Petrus: Het is de Heere! Simon Petrus dan, horende, dat het de Heere was, omgordde het opperkleed (want hij was naakt), en wierp zichzelven in de zee.
\par 8 En de andere discipelen kwamen met het scheepje (want zij waren niet verre van het land, maar omtrent tweehonderd ellen), slepende het net met de vissen.
\par 9 Als zij dan aan het land gegaan waren, zagen zij een kolenvuur liggen, en vis daarop liggen, en brood.
\par 10 Jezus zeide tot hen: Brengt van den vissen, die gij nu gevangen hebt.
\par 11 Simon Petrus ging op, en trok het net op het land, vol grote vissen, tot honderd drie en vijftig; en hoewel er zovele waren, zo scheurde het net niet.
\par 12 Jezus zeide tot hen: Komt herwaarts, houdt het middagmaal. En niemand van de discipelen durfde Hem vragen: Wie zijt Gij? wetende, dat het de Heere was.
\par 13 Jezus dan kwam, en nam het brood, en gaf het hun, en den vis desgelijks.
\par 14 Dit was nu de derde maal, dat Jezus Zijn discipelen geopenbaard is, nadat Hij van de doden opgewekt was.
\par 15 Toen zij dan het middagmaal gehouden hadden, zeide Jezus tot Simon Petrus: Simon, zoon van Jonas, hebt gij Mij liever dan dezen? Hij zeide tot Hem: Ja, Heere! Gij weet, dat ik U liefheb. Hij zeide tot hem: Weid Mijn lammeren.
\par 16 Hij zeide wederom tot hem ten tweeden maal: Simon, zoon van Jonas, hebt gij Mij lief? Hij zeide tot Hem: Ja, Heere, gij weet, dat ik U liefheb. Hij zeide tot hem: Hoed Mijn schapen.
\par 17 Hij zeide tot hem ten derden maal: Simon, zoon van Jonas, hebt gij Mij lief? Petrus werd bedroefd, omdat Hij ten derden maal tot hem zeide: Hebt gij Mij lief, en zeide tot Hem: Heere! Gij weet alle dingen, Gij weet, dat ik U liefheb. Jezus zeide tot hem: Weid Mijn schapen.
\par 18 Voorwaar, voorwaar, zeg Ik u: Toen gij jonger waart, gorddet gij uzelven, en wandeldet, alwaar gij wildet; maar wanneer gij zult oud geworden zijn, zo zult gij uw handen uitstrekken, en een ander zal u gorden, en brengen, waar gij niet wilt.
\par 19 En dit zeide Hij, betekenende, met hoedanigen dood hij God verheerlijken zou. En dit gesproken hebbende, zeide Hij tot hem: Volg Mij.
\par 20 En Petrus, zich omkerende, zag den discipel volgen, welken Jezus liefhad, die ook in het avondmaal op Zijn borst gevallen was, en gezegd had: Heere! wie is het, die U verraden zal?
\par 21 Als Petrus dezen zag, zeide hij tot Jezus: Heere, maar wat zal deze?
\par 22 Jezus zeide tot hem: Indien Ik wil, dat hij blijve, totdat Ik kome, wat gaat het u aan? Volg gij Mij.
\par 23 Dit woord dan ging uit onder de broederen, dat deze discipel niet zou sterven. En Jezus had tot hem niet gezegd, dat hij niet sterven zou, maar: Indien Ik wil, dat hij blijve, totdat Ik kome, wat gaat het u aan?
\par 24 Deze is de discipel, die van deze dingen getuigt, en deze dingen geschreven heeft; en wij weten, dat zijn getuigenis waarachtig is.
\par 25 En er zijn nog vele andere dingen, die Jezus gedaan heeft, welke, zo zij elk bijzonder geschreven wierden, ik acht, dat ook de wereld zelve de geschrevene boeken niet zou bevatten. Amen.




\end{document}