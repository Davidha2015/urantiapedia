\begin{document}

\title{Acts of the Apostles}



\chapter{1}

\par 1 Het eerste boek heb ik gemaakt, o Theofilus, van al hetgeen Jezus begonnen heeft beide te doen en te leren;
\par 2 Tot op den dag, in welken Hij opgenomen is, nadat Hij door den Heiligen Geest aan de apostelen, die Hij uitverkoren had, bevelen had gegeven.
\par 3 Aan welke Hij ook, nadat Hij geleden had, Zichzelven levend vertoond heeft, met vele gewisse kentekenen, veertig dagen lang, zijnde van hen gezien, en sprekende van de dingen, die het Koninkrijk Gods aangaan.
\par 4 En als Hij met hen vergaderd was, beval Hij hun, dat zij van Jeruzalem niet scheiden zouden, maar verwachten de belofte des Vaders, die gij, zeide Hij, van Mij gehoord hebt.
\par 5 Want Johannes doopte wel met water, maar gij zult met den Heiligen Geest gedoopt worden, niet lang na deze dagen.
\par 6 Zij dan, die samengekomen waren, vraagden Hem, zeggende: Heere, zult Gij in dezen tijd aan Israel het Koninkrijk wederoprichten?
\par 7 En Hij zeide tot hen: Het komt u niet toe, te weten de tijden of gelegenheden, die de Vader in Zijn eigen macht gesteld heeft;
\par 8 Maar gij zult ontvangen de kracht des Heiligen Geestes, Die over u komen zal; en gij zult Mijn getuigen zijn, zo te Jeruzalem, als in geheel Judea en Samaria, en tot aan het uiterste der aarde.
\par 9 En als Hij dit gezegd had, werd Hij opgenomen, daar zij het zagen, en een wolk nam Hem weg van hun ogen.
\par 10 En alzo zij hun ogen naar den hemel hielden, terwijl Hij heenvoer, ziet, twee mannen stonden bij hen in witte kleding;
\par 11 Welke ook zeiden: Gij Galilese mannen, wat staat gij en ziet op naar den hemel? Deze Jezus, Die van u opgenomen is in den hemel, zal alzo komen, gelijkerwijs gij Hem naar den hemel hebt zien heenvaren.
\par 12 Toen keerden zij wederom naar Jeruzalem, van den berg, die genaamd wordt de Olijf berg, welke is nabij Jeruzalem, liggende van daar een sabbatsreize.
\par 13 En als zij ingekomen waren, gingen zij op in de opperzaal, waar zij bleven, namelijk Petrus en Jakobus, en Johannes en Andreas, Filippus en Thomas, Bartholomeus en Mattheus, Jakobus, de zoon van Alfeus, en Simon Zelotes, en Judas, de broeder van Jakobus.
\par 14 Deze allen waren eendrachtelijk volhardende in het bidden en smeken, met de vrouwen, en Maria, de moeder van Jezus, en met Zijn broederen.
\par 15 En in dezelve dagen stond Petrus op in het midden der discipelen, en sprak (er was nu een schare bijeen van omtrent honderd en twintig personen):
\par 16 Mannen broeders, deze Schrift moest vervuld worden, welke de Heilige Geest door den mond Davids voorzegd heeft van Judas, die de leidsman geweest is dergenen, die Jezus vingen;
\par 17 Want hij was met ons gerekend, en had het lot dezer bediening verkregen.
\par 18 Deze dan heeft verworven een akker, door het loon der ongerechtigheid, en voorwaarts overgevallen zijnde, is midden opgeborsten, en al zijn ingewanden zijn uitgestort.
\par 19 En het is bekend geworden allen, die te Jeruzalem wonen, alzo dat die akker in hun eigen taal genoemd wordt Akeldama, dat is, een akker des bloeds.
\par 20 Want er staat geschreven in het boek der Psalmen; Zijn woonstede worde woest, en er zij niemand die in dezelve wone. En: Een ander neme zijn opzienersambt.
\par 21 Het is dan nodig, dat van de mannen, die met ons omgegaan hebben al den tijd, in welken de Heere Jezus onder ons in gegaan en uitgegaan is,
\par 22 Beginnende van den doop van Johannes, tot den dag toe, in welken Hij van ons opgenomen is, een derzelven met ons getuige worde van Zijn opstanding.
\par 23 En zij stelden er twee, Jozef, genaamd Barsabas, die toegenaamd was Justus, en Matthias.
\par 24 En zij baden en zeiden: Gij Heere! Gij Kenner der harten van allen, wijs van deze twee een aan, dien Gij uitverkoren hebt;
\par 25 Om te ontvangen het lot dezer bediening en des apostelschaps, waarvan Judas afgeweken is, dat hij heenging in zijn eigen plaats.
\par 26 En zij wierpen hun loten; en het lot viel op Matthias, en hij werd met gemene toestemming tot de elf apostelen gekozen.

\chapter{2}

\par 1 En als de dag van het Pinkster feest vervuld werd, waren zij allen eendrachtelijk bijeen.
\par 2 En er geschiedde haastelijk uit den hemel een geluid, gelijk als van een geweldigen, gedreven wind, en vervulde het gehele huis, waar zij zaten.
\par 3 En van hen werden gezien verdeelde tongen als van vuur, en het zat op een iegelijk van hen.
\par 4 En zij werden allen vervuld met den Heiligen Geest, en begonnen te spreken met andere talen, zoals de Geest hun gaf uit te spreken.
\par 5 En er waren Joden, te Jeruzalem wonende, godvruchtige mannen van allen volke dergenen, die onder den hemel zijn.
\par 6 En als deze stem geschied was, kwam de menigte samen, en werd beroerd, want een iegelijk hoorde hen in zijn eigen taal spreken.
\par 7 En zij ontzetten zich allen, en verwonderden zich, zeggende tot elkander: Ziet, zijn niet alle dezen, die daar spreken, Galileers?
\par 8 En hoe horen wij hen een iegelijk in onze eigen taal, in welke wij geboren zijn?
\par 9 Parthers, en Meders, en Elamieten, en de inwoners zijn van Mesopotamie, en Judea, en Cappadocie, Pontus en Azie.
\par 10 En Frygie, en Pamfylie, Egypte, en de delen van Libye, hetwelk bij Cyrene ligt, en uitlandse Romeinen, beiden Joden en Jodengenoten;
\par 11 Kretenzen en Arabieren, wij horen hen in onze talen de grote werken Gods spreken.
\par 12 En zij ontzetten zich allen, en werden twijfelmoedig, zeggende, de een tegen den ander: Wat wil toch dit zijn?
\par 13 En anderen, spottende, zeiden: Zij zijn vol zoeten wijns.
\par 14 Maar Petrus, staande met de elven, verhief zijn stem, en sprak tot hen: Gij Joodse mannen, en gij allen, die te Jeruzalem woont, dit zij u bekend, en laat mijn woorden tot uw oren ingaan.
\par 15 Want deze zijn niet dronken, gelijk gij vermoedt; want het is eerst de derde ure van de dag.
\par 16 Maar dit is het, wat gesproken is door den profeet Joel:
\par 17 En het zal zijn in de laatste dagen, (zegt God) Ik zal uitstorten van Mijn Geest op alle vlees; en uw zonen en uw dochters zullen profeteren, en uw jongelingen zullen gezichten zien, en uw ouden zullen dromen dromen.
\par 18 En ook op Mijn dienstknechten, en op Mijn dienstmaagden, zal Ik in die dagen van Mijn Geest uitstorten, en zij zullen profeteren.
\par 19 En Ik zal wonderen geven in den hemel boven, en tekenen op de aarde beneden, bloed en vuur, en rookdamp.
\par 20 De zon zal veranderd worden in duisternis, en de maan in bloed, eer dat de grote en doorluchtige dag des Heeren komt.
\par 21 En het zal zijn, dat een iegelijk, die den Naam des Heeren zal aanroepen, zalig zal worden.
\par 22 Gij Israelietische mannen, hoort deze woorden: Jezus den Nazarener, een Man van God, onder ulieden betoond door krachten, en wonderen, en tekenen, die God door Hem gedaan heeft, in het midden van u, gelijk ook gijzelven weet;
\par 23 Dezen, door den bepaalden raad en voorkennis Gods overgegeven zijnde, hebt gij genomen, en door de handen der onrechtvaardigen aan het kruis gehecht en gedood;
\par 24 Welken God opgewekt heeft, de smarten des doods ontbonden hebbende, alzo het niet mogelijk was, dat Hij van denzelven dood zou gehouden worden.
\par 25 Want David zegt van Hem: Ik zag den Heere allen tijd voor mij; want Hij is aan mijn rechter hand, opdat ik niet bewogen worde.
\par 26 Daarom is mijn hart verblijd; en mijn tong verheugt zich; ja, ook mijn vlees zal rusten in hope;
\par 27 Want Gij zult mijn ziel in de hel niet verlaten, noch zult Uw Heilige over geven, om verderving te zien.
\par 28 Gij hebt mij de wegen des levens bekend gemaakt; Gij zult mij vervullen met verheuging door Uw aangezicht.
\par 29 Gij mannen broeders, het is mij geoorloofd vrij uit tot u te spreken van den patriarch David, dat hij beide gestorven en begraven is, en zijn graf is onder ons tot op dezen dag.
\par 30 Alzo hij dan een profeet was, en wist, dat God hem met ede gezworen had, dat hij uit de vrucht zijner lenden, zoveel het vlees aangaat, den Christus verwekken zou, om Hem op zijn troon te zetten;
\par 31 Zo heeft hij, dit voorziende, gesproken van de opstanding van Christus, dat Zijn ziel niet is verlaten in de hel, noch Zijn vlees verderving heeft gezien.
\par 32 Dezen Jezus heeft God opgewekt; waarvan wij allen getuigen zijn.
\par 33 Hij dan, door de rechter hand Gods verhoogd zijnde, en de belofte des Heiligen Geestes, ontvangen hebbende van den Vader, heeft dit uitgestort, dat gij nu ziet en hoort.
\par 34 Want David is niet opgevaren in de hemelen; maar hij zegt: De Heere heeft gesproken tot Mijn Heere: Zit aan Mijn rechter hand.
\par 35 Totdat Ik Uw vijanden zal gezet hebben tot een voetbank Uwer voeten.
\par 36 Zo wete dan zekerlijk het ganse huis Israels, dat God Hem tot een Heere en Christus gemaakt heeft, namelijk dezen Jezus, Dien gij gekruist hebt.
\par 37 En als zij dit hoorden, werden zij verslagen in het hart, en zeiden tot Petrus en de andere apostelen: Wat zullen wij doen mannen broeders?
\par 38 En Petrus zeide tot hen: Bekeert u, en een iegelijk van u worde gedoopt in den Naam van Jezus Christus, tot vergeving der zonden; en gij zult de gave des Heiligen Geestes ontvangen.
\par 39 Want u komt de belofte toe, en uw kinderen, en allen, die daar verre zijn, zo velen als er de Heere, onze God, toe roepen zal.
\par 40 En met veel meer andere woorden betuigde hij, en vermaande hen, zeggende: Wordt behouden van dit verkeerd geslacht!
\par 41 Die dan zijn woord gaarne aannamen, werden gedoopt; en er werden op dien dag tot hen toegedaan omtrent drie duizend zielen.
\par 42 En zij waren volhardende in de leer der apostelen, en in de gemeenschap, en in de breking des broods, en in de gebeden.
\par 43 En een vreze kwam over alle ziel; en vele wonderen en tekenen geschiedden door de apostelen.
\par 44 En allen, die geloofden, waren bijeen, en hadden alle dingen gemeen;
\par 45 En zij verkochten hun goederen en have, en verdeelden dezelve aan allen, naar dat elk van node had.
\par 46 En dagelijks eendrachtelijk in den tempel volhardende, en van huis tot huis brood brekende, aten zij te zamen met verheuging en eenvoudigheid des harten;
\par 47 En prezen God, en hadden genade bij het ganse volk. En de Heere deed dagelijks tot de Gemeente, die zalig werden.

\chapter{3}

\par 1 Petrus nu en Johannes gingen te zamen op naar den tempel, omtrent de ure des gebeds, zijnde de negende ure;
\par 2 En een zeker man, die kreupel was van zijner moeders lijf, werd gedragen, welken zij dagelijks zetten aan de deur des tempels, genaamd de Schone, om een aalmoes te begeren van degenen, die in den tempel gingen;
\par 3 Welke, Petrus en Johannes ziende, als zij in den tempel zouden ingaan, bad, dat hij een aalmoes mocht ontvangen.
\par 4 En Petrus, sterk op hem ziende, met Johannes, zeide: Zie op ons.
\par 5 En hij hield de ogen op hen, verwachtende, dat hij iets van hen zou ontvangen.
\par 6 En Petrus zeide: Zilver en goud heb ik niet, maar hetgeen ik heb, dat geve ik u; in den Naam van Jezus Christus, den Nazarener, sta op en wandel!
\par 7 En hem grijpende bij de rechterhand richtte hij hem op, en terstond werden zijn voeten en enkelen vast.
\par 8 En hij, opspringende, stond en wandelde, en ging met hen in den tempel, wandelende en springende, en lovende God.
\par 9 En al het volk zag hem wandelen en God loven.
\par 10 En zij kenden hem, dat hij die was, die om een aalmoes gezeten had aan de Schone poort des tempels; en zij werden vervuld met verbaasdheid en ontzetting over hetgeen hem geschied was.
\par 11 En als de kreupele, die gezond gemaakt was, aan Petrus en Johannes vasthield, liep al het volk gezamenlijk tot hen in het voorhof, hetwelk Salomo's voorhof genaamd wordt, verbaasd zijnde.
\par 12 En Petrus, dat ziende, antwoordde tot het volk: Gij Israelietische mannen, wat verwondert gij u over dit, of wat ziet gij zo sterk op ons, alsof wij door onze eigen kracht of godzaligheid dezen hadden doen wandelen?
\par 13 De God Abrahams, en Izaks, en Jakobs, de God onzer vaderen, heeft Zijn Kind Jezus verheerlijkt, Welken gij overgeleverd hebt, en hebt Hem verloochend, voor het aangezicht van Pilatus, als hij oordeelde, dat men Hem zoude loslaten.
\par 14 Maar gij hebt den Heilige en Rechtvaardige verloochend, en hebt begeerd, dat u een man, die een doodslager was, zou geschonken worden;
\par 15 En den Vorst des levens hebt gij gedood, Welken God opgewekt heeft uit de doden; waarvan wij getuigen zijn.
\par 16 En door het geloof in Zijn Naam heeft Zijn Naam dezen gesterkt, dien gij ziet en kent; en het geloof, dat door Hem is, heeft hem deze volmaakte gezondheid gegeven, in uw aller tegenwoordigheid.
\par 17 En nu, broeders, ik weet, dat gij het door onwetendheid gedaan hebt, gelijk als ook uw oversten.
\par 18 Maar God heeft alzo vervuld, hetgeen Hij door den mond van al Zijn profeten te voren verkondigd had, dat de Christus lijden zou.
\par 19 Betert u dan, en bekeert u, opdat uw zonden mogen uitgewist worden; wanneer de tijden der verkoeling zullen gekomen zijn van het aangezicht des Heeren,
\par 20 En Hij gezonden zal hebben Jezus Christus, Die u tevoren gepredikt is;
\par 21 Welken de hemel moet ontvangen tot de tijden der wederoprichting aller dingen, die God gesproken heeft door den mond van al Zijn heilige profeten van alle eeuw.
\par 22 Want Mozes heeft tot de vaderen gezegd: De Heere, uw God, zal u een Profeet verwekken, uit uw broederen, gelijk mij; Dien zult gij horen, in alles, wat Hij tot u spreken zal.
\par 23 En het zal geschieden, dat alle ziel, die dezen Profeet niet zal gehoord hebben, uitgeroeid zal worden uit den volke.
\par 24 En ook al de profeten, van Samuel aan, en die daarna gevolgd zijn, zovelen als er hebben gesproken, die hebben ook deze dagen te voren verkondigd.
\par 25 Gijlieden zijt kinderen der profeten, en des verbonds, hetwelk God met onze vaderen opgericht heeft, zeggende tot Abraham: En in uw zade zullen alle geslachten der aarde gezegend worden.
\par 26 God, opgewekt hebbende Zijn Kind Jezus, heeft Denzelven eerst tot u gezonden, dat Hij ulieden zegenen zou, daarin dat Hij een iegelijk van u afkere van uw boosheden.

\chapter{4}

\par 1 En terwijl zij tot het volk spraken, kwamen daarover tot hen de priesters, en de hoofdman des tempels, en de Sadduceen;
\par 2 Zeer ontevreden zijnde, omdat zij het volk leerden, en verkondigden in Jezus de opstanding uit de doden.
\par 3 En zij sloegen de handen aan hen, en zetten ze in bewaring tot den anderen dag; want het was nu avond.
\par 4 En velen van degenen, die het woord gehoord hadden, geloofden; en het getal der mannen werd omtrent vijf duizend.
\par 5 En het geschiedde des anderen daags, dat hun oversten en ouderlingen en Schriftgeleerden te Jeruzalem vergaderden;
\par 6 En Annas, de hogepriester, en Kajafas, en Johannes, en Alexander, en zovele er van het hogepriesterlijk geslacht waren.
\par 7 En als zij hen in het midden gesteld hadden, vraagden zij: Door wat kracht, of door wat naam hebt gijlieden dit gedaan?
\par 8 Toen zeide Petrus, vervuld zijnde met den Heiligen Geest, tot hen: Gij oversten des volks, en gij ouderlingen van Israel!
\par 9 Alzo wij heden rechterlijk onderzocht worden over de weldaad aan een krank mens geschied, waardoor hij gezond geworden is;
\par 10 Zo zij u allen kennelijk, en het ganse volk Israel, dat door den Naam van Jezus Christus, den Nazarener, Dien gij gekruist hebt, Welken God van de doden heeft opgewekt, door Hem, zeg ik, staat deze hier voor u gezond.
\par 11 Deze is de Steen, Die van u, de bouwlieden, veracht is, Welke tot een hoofd des hoeks geworden is.
\par 12 En de zaligheid is in geen Anderen; want er is ook onder den hemel geen andere Naam, Die onder de mensen gegeven is, door Welken wij moeten zalig worden.
\par 13 Zij nu, ziende de vrijmoedigheid van Petrus en Johannes, en vernemende, dat zij ongeleerde en slechte mensen waren, verwonderden zich, en kenden hen, dat zij met Jezus geweest waren.
\par 14 En ziende den mens bij hen staan, die genezen was, hadden zij niets daartegen te zeggen.
\par 15 En hun geboden hebbende uit te gaan buiten den raad, overlegden zij met elkander,
\par 16 Zeggende: Wat zullen wij dezen mensen doen? Want dat er een bekend teken door hen geschied is, is openbaar aan allen, die te Jeruzalem wonen, en wij kunnen het niet loochenen.
\par 17 Maar opdat het niet meer en meer onder het volk verspreid worde, laat ons hen scherpelijk dreigen, dat zij niet meer tot enig mens in dezen Naam spreken.
\par 18 En als zij hen geroepen hadden, zeiden zij hun aan, dat zij ganselijk niet zouden spreken, noch leren, in den Naam van Jezus.
\par 19 Maar Petrus en Johannes, antwoordende, zeiden tot hen: Oordeelt gij, of het recht is voor God, ulieden meer te horen dan God.
\par 20 Want wij kunnen niet laten te spreken, hetgeen wij gezien en gehoord hebben.
\par 21 Maar zij dreigden hen nog meer, en lieten ze gaan, niets vindende, hoe zij hen straffen zouden, om des volks wil; want zij verheerlijkten allen God over hetgeen er geschied was.
\par 22 Want de mens was meer dan veertig jaren oud, aan welken dit teken der genezing geschied was.
\par 23 En zij, losgelaten zijnde, kwamen tot de hunnen, en verkondigden al wat de overpriesters en de ouderlingen tot hen gezegd hadden.
\par 24 En als dezen dat hoorden, hieven zij eendrachtelijk hun stem op tot God, en zeiden: Heere! Gij zijt de God, Die gemaakt hebt den hemel, en de aarde, en de zee, en alle dingen, die in dezelve zijn.
\par 25 Die door den mond van David Uw knecht, gezegd hebt: Waarom woeden de heidenen, en hebben de volken ijdele dingen bedacht?
\par 26 De koningen der aarde zijn te zamen opgestaan, en de oversten zijn bijeenvergaderd tegen den Heere, en tegen Zijn Gezalfde.
\par 27 Want in der waarheid zijn vergaderd tegen Uw heilig Kind Jezus, Welken Gij gezalfd hebt, beiden Herodes en Pontius Pilatus, met de heidenen en de volken Israels;
\par 28 Om te doen al wat Uw hand en Uw raad te voren bepaald had, dat geschieden zou.
\par 29 En nu dan, Heere, zie op hun dreigingen, en geef Uw dienstknechten met alle vrijmoedigheid Uw woord te spreken;
\par 30 Daarin, dat Gij Uw hand uitstrekt tot genezing, en dat tekenen en wonderen geschieden door den Naam van Uw heilig Kind Jezus.
\par 31 En als zij gebeden hadden, werd de plaats, in welke zij vergaderd waren, bewogen. En zij werden allen vervuld met den Heiligen Geest, en spraken het Woord Gods met vrijmoedigheid.
\par 32 En de menigte van degenen, die geloofden, was een hart en een ziel; en niemand zeide, dat iets van hetgeen hij had, zijn eigen ware, maar alle dingen waren hun gemeen.
\par 33 En de apostelen gaven met grote kracht getuigenis van de opstanding van den Heere Jezus; en er was grote genade over hen allen.
\par 34 Want er was ook niemand onder hen, die gebrek had; want zovelen als er bezitters waren van landen of huizen, die verkochten zij, en brachten den prijs der verkochte goederen, en legden dien aan de voeten der apostelen.
\par 35 En aan een iegelijk werd uitgedeeld, naar dat elk van node had.
\par 36 En Joses, van de apostelen toegenaamd Barnabas (hetwelk is, overgezet zijnde, een zoon der vertroosting), een Leviet, van geboorte uit Cyprus,
\par 37 Alzo hij een akker had, verkocht dien, en bracht het geld, en legde het aan de voeten der apostelen.

\chapter{5}

\par 1 En een zeker man, met name Ananias, met Saffira, zijn vrouw, verkocht een have;
\par 2 En onttrok van den prijs, ook met medeweten zijner vrouw; en bracht een zeker deel, en legde dat aan de voeten der apostelen.
\par 3 En Petrus zeide: Ananias, waarom heeft de satan uw hart vervuld, dat gij den Heiligen Geest liegen zoudt, en onttrekken van den prijs des lands?
\par 4 Zo het gebleven ware, bleef het niet uw, en verkocht zijnde, was het niet in uw macht? Wat is het, dat gij deze daad in uw hart hebt voorgenomen? Gij hebt den mensen niet gelogen, maar Gode.
\par 5 En Ananias, deze woorden horende, viel neder en gaf den geest. En er kwam grote vrees over allen, die dit hoorden.
\par 6 En de jongelingen, opstaande, schikten hem toe, en droegen hem uit, en begroeven hem.
\par 7 En het was omtrent drie uren daarna, dat ook zijn vrouw daar inkwam, niet wetende, wat er geschied was;
\par 8 En Petrus antwoordde haar: Zeg mij, hebt gijlieden het land voor zoveel verkocht? En zij zeide: Ja, voor zoveel.
\par 9 En Petrus zeide tot haar: Wat is het, dat gij onder u hebt overeengestemd te verzoeken den Geest des Heeren? Zie, de voeten dergenen, die uw man begraven hebben, zijn voor de deur, en zullen u uitdragen.
\par 10 En zij viel terstond neder voor zijn voeten, en gaf den geest. En de jongelingen ingekomen zijnde, vonden haar dood en droegen ze uit, en begroeven haar bij haar man.
\par 11 En er kwam grote vreze over de gehele Gemeente, en over allen, die dit hoorden.
\par 12 En door de handen der apostelen geschiedden vele tekenen en wonderen onder het volk; en zij waren allen eendrachtelijk in het voorhof van Salomo.
\par 13 En van de anderen durfde niemand zich bij hen voegen; maar het volk hield hen in grote achting.
\par 14 En er werden meer en meer toegedaan, die den Heere geloofden, menigten beide van mannen en van vrouwen;
\par 15 Alzo dat zij de kranken uitdroegen op de straten, en legden op bedden en beddekens, opdat, als Petrus kwam, ook maar de schaduw iemand van hen beschaduwen mocht.
\par 16 En ook de menigte uit de omliggende steden kwam gezamenlijk te Jeruzalem, brengende kranken, en die van onreine geesten gekweld waren; welke allen genezen werden.
\par 17 En de hogepriester stond op, en allen, die met hem waren (welke was de sekte der Sadduceen), en werden vervuld met nijdigheid;
\par 18 En sloegen hun handen aan de apostelen, en zetten hen in de gemene gevangenis.
\par 19 Maar de engel des Heeren opende des nachts de deuren der gevangenis en leidde hen uit, en zeide:
\par 20 Gaat heen, en staat, en spreekt in den tempel tot het volk al de woorden dezes levens.
\par 21 Als zij nu dit gehoord hadden, gingen zij tegen den morgenstond in den tempel, en leerden. Maar de hogepriester, en die met hem waren, gekomen zijnde, riepen den raad te zamen, en al de oudsten der kinderen Israels, en zonden naar den kerker, om hen te halen.
\par 22 Doch als de dienaars daar kwamen, vonden zij hen in de gevangenis niet, maar keerden wederom, en boodschapten dit.
\par 23 Zeggende: Wij vonden wel den kerker met alle verzekerdheid toegesloten, en de wachters buiten staande voor de deuren; maar als wij die geopend hadden, vonden wij niemand daarbinnen.
\par 24 Toen nu de hoge priester en de hoofdman des tempels, en de overpriesters deze woorden hoorden, werden zij twijfelmoedig over hen, wat toch dit worden zou.
\par 25 En er kwam een, en boodschapte hun, zeggende: Ziet, de mannen, die gij in de gevangenis gezet hebt, staan in den tempel, en leren het volk.
\par 26 Toen ging de hoofdman heen, met de dienaren, en bracht hen, doch niet met geweld (want zij vreesden het volk, opdat zij niet gestenigd wierden).
\par 27 En als zij hen gebracht hadden, stelden zij hen voor den raad; en de hogepriester vraagde hun, en zeide:
\par 28 Hebben wij u niet ernstiglijk aangezegd, dat gij in dezen Naam niet zoudt leren? En ziet, gij hebt met deze uw leer Jeruzalem vervuld, en gij wilt het bloed van dezen Mens over ons brengen.
\par 29 Maar Petrus en de apostelen antwoordden, en zeiden: Men moet Gode meer gehoorzaam zijn, dan den mensen.
\par 30 De God onzer vaderen heeft Jezus opgewekt, Welken gij omgebracht hebt, hangende Hem aan het hout.
\par 31 Deze heeft God door Zijn rechter hand verhoogd tot een Vorst en Zaligmaker, om Israel te geven bekering en vergeving der zonden.
\par 32 En wij zijn Zijn getuigen van deze woorden; en ook de Heilige Geest, Welken God gegeven heeft dengenen, die Hem gehoorzaam zijn.
\par 33 Als zij nu dit hoorden, barstte hun het hart, en zij hielden raad, om hen te doden.
\par 34 Maar een zeker Farizeer stond op in den raad, met name Gamaliel, een leraar der wet, in waarde gehouden bij al het volk, en gebood, dat men de apostelen een weinig zou doen buiten staan.
\par 35 En hij zeide tot hen: Gij Israelietische mannen, ziet voor u, wat gij doen zult aangaande deze mensen.
\par 36 Want voor deze dagen stond Theudas op, zeggende, dat hij wat was, dien een getal van omtrent vierhonderd mannen aanhing; welke is omgebracht, en allen, die hem gehoor gaven, zijn verstrooid en tot niet geworden.
\par 37 Na hem stond op Judas, de Galileer in de dagen der beschrijving, en maakte veel volks afvallig achter zich; en deze is ook vergaan, en allen, die hem gehoor gaven, zijn verstrooid geworden.
\par 38 En nu zeg ik ulieden: Houdt af van deze mensen, en laat hen gaan; want indien deze raad, of dit werk uit mensen is, zo zal het gebroken worden.
\par 39 Maar indien het uit God is, zo kunt gij dat niet breken; opdat gij niet misschien bevonden wordt ook tegen God te strijden.
\par 40 En zij gaven hem gehoor; en als zij de apostelen tot zich geroepen hadden, geselden zij dezelve, en geboden hun, dat zij niet zouden spreken in den Naam van Jezus; en lieten hen gaan.
\par 41 Zij dan gingen heen van het aangezicht des raads, verblijd zijnde, dat zij waren waardig geacht geweest, om Zijns Naams wil smaadheid te lijden.
\par 42 En zij hielden niet op, allen dag, in den tempel en bij de huizen, te leren, en Jezus Christus te verkondigen.

\chapter{6}

\par 1 En in dezelfde dagen, als de discipelen vermenigvuldigden, ontstond een murmurering der Grieksen tegen de Hebreen, omdat hun weduwen in de dagelijkse bediening verzuimd werden.
\par 2 En de twaalven riepen de menigte der discipelen tot zich, en zeiden: Het is niet behoorlijk, dat wij het Woord Gods nalaten, en de tafelen dienen.
\par 3 Ziet dan om, broeders, naar zeven mannen uit u, die goede getuigenis hebben, vol des Heiligen Geestes en der wijsheid, welke wij mogen stellen over deze nodige zaak.
\par 4 Maar wij zullen volharden in het gebed, en in de bediening des Woords.
\par 5 En dit woord behaagde aan al de menigte; en zij verkoren Stefanus, een man vol des geloofs en des Heiligen Geestes, en Filippus, en Prochorus, en Nicanor, en Timon, en Parmenas, en Nicolaus, een Jodengenoot van Antiochie;
\par 6 Welken zij voor de apostelen stelden; en dezen, als zij gebeden hadden, legden hun de handen op.
\par 7 En het woord Gods wies, en het getal der discipelen vermenigvuldigde te Jeruzalem zeer; en een grote schare der priesteren werd den gelove gehoorzaam.
\par 8 En Stefanus, vol van geloof en kracht, deed wonderen en grote tekenen onder het volk.
\par 9 En er stonden op sommigen, die waren van de synagoge, genaamd der Libertijnen, en der Cyreneers, en der Alexandrijnen, en dergenen, die van Cilicie en Azie waren, en twistten met Stefanus.
\par 10 En zij konden niet wederstaan de wijsheid en den Geest, door Welken hij sprak.
\par 11 Toen maakten zij mannen uit, die zeiden: Wij hebben hem horen spreken lasterlijke woorden tegen Mozes en God.
\par 12 En zij beroerden het volk, en de ouderlingen en de Schriftgeleerden; en hem aanvallende grepen zij hem, en leidden hem voor den raad;
\par 13 En stelden valse getuigen, die zeiden: Deze mens houdt niet op lasterlijke woorden te spreken tegen deze heilige plaats en de wet.
\par 14 Want wij hebben hem horen zeggen, dat deze Jezus, de Nazarener, deze plaats zal verbreken, en dat Hij de zeden veranderen zal, die ons Mozes overgeleverd heeft.
\par 15 En allen, die in den raad zaten, de ogen op hem houdende, zagen zijn aangezicht als het aangezicht eens engels.

\chapter{7}

\par 1 En de hogepriester zeide: Zijn dan deze dingen alzo?
\par 2 En hij zeide: Gij mannen broeders en vaders, hoort toe: de God der heerlijkheid verscheen onzen vader Abraham, nog zijnde in Mesopotamie, eer hij woonde in Charran;
\par 3 En zeide tot hem: Ga uit uw land en uit uw maagschap, en kom in een land, dat Ik u wijzen zal.
\par 4 Toen ging hij uit het land der Chaldeen, en woonde in Charran. En van daar, nadat zijn vader gestorven was, bracht Hij hem over in dit land, daar gij nu in woont.
\par 5 En Hij gaf hem geen erfdeel in hetzelve, ook niet een voetstap; en beloofde, dat Hij hem het zelve tot een bezitting geven zou, en zijn zade na hem, als hij nog geen kind had.
\par 6 En God sprak alzo, dat zijn zaad vreemdeling zijn zoude in een vreemd land, en dat zij het zouden dienstbaar maken, en kwalijk handelen, vierhonderd jaren.
\par 7 En het volk, dat zij dienen zullen, zal Ik oordelen, sprak God; en daarna zullen zij uitgaan, en zij zullen Mij dienen in deze plaats.
\par 8 En Hij gaf hem het verbond der besnijdenis; en alzo gewon hij Izak, en besneed hem op den achtsten dag; en Izak gewon Jakob, en Jakob de twaalf patriarchen.
\par 9 En de patriarchen, nijdig zijnde, verkochten Jozef, om naar Egypte gebracht te worden; en God was met hem,
\par 10 En verloste hem uit al zijn verdrukkingen, en gaf hem genade en wijsheid voor Farao, den koning van Egypteland; en hij stelde hem tot een overste over Egypte, en zijn gehele huis.
\par 11 En er kwam een hongersnood over het gehele land van Egypte en Kanaan, en grote benauwdheid; en onze vaders vonden geen spijs.
\par 12 Maar als Jakob hoorde, dat in Egypte koren was, zond hij onze vaders de eerste maal uit.
\par 13 En in de tweede reize werd Jozef zijn broederen bekend; en het geslacht van Jozef werd aan Farao openbaar.
\par 14 En Jozef zond heen, en ontbood zijn vader Jakob, en al zijn geslacht, bestaande in vijf en zeventig zielen.
\par 15 En Jakob kwam af in Egypte, en stierf, hijzelf en onze vaders.
\par 16 En zij werden overgebracht naar Sichem, en gelegd in het graf, hetwelk Abraham gekocht had voor een som gelds, van de zonen van Emmor, den vader van Sichem.
\par 17 Maar als nu de tijd der belofte, die God aan Abraham gezworen had, genaakte, wies het volk en vermenigvuldigde in Egypte;
\par 18 Totdat een ander koning opstond, die Jozef niet gekend had.
\par 19 Deze gebruikte listigheid tegen ons geslacht, en handelde kwalijk met onze vaderen, zodat zij hun jonge kinderen moesten wegwerpen, opdat zij niet zouden voorttelen.
\par 20 In welken tijd Mozes werd geboren, en was uitnemend schoon; welke drie maanden opgevoed werd in het huis zijns vaders.
\par 21 En als hij weggeworpen was, nam hem de dochter van Farao op, en voedde hem voor zichzelve op tot een zoon.
\par 22 En Mozes werd onderwezen in alle wijsheid der Egyptenaren; en was machtig in woorden en in werken.
\par 23 Als hem nu de tijd van veertig jaren vervuld was, kwam hem in zijn hart, zijn broeders, de kinderen Israels, te bezoeken.
\par 24 En ziende een, die onrecht leed, beschermde hij hem, en wreekte dengene, dien overlast geschiedde, en versloeg den Egyptenaar.
\par 25 En hij meende, dat zijn broeders zouden verstaan, dat God door zijn hand hun verlossing geven zou; maar zij hebben het niet verstaan.
\par 26 En den volgenden dag werd hij van hen gezien, daar zij vochten; en hij drong ze tot vrede, zeggende: Mannen, gij zijt broeders; waarom doet gij elkander ongelijk?
\par 27 En die zijn naaste ongelijk deed, verstiet hem, zeggende: Wie heeft u tot een overste en rechter over ons gesteld?
\par 28 Wilt gij mij ook ombrengen, gelijkerwijs gij gisteren den Egyptenaar omgebracht hebt?
\par 29 En Mozes vluchtte op dat woord en werd een vreemdeling in het land Madiam, waar hij twee zonen gewon.
\par 30 En als veertig jaren vervuld waren, verscheen hem de Engel des Heeren, in de woestijn van den berg Sinai, in een vlammig vuur van het doornenbos.
\par 31 Mozes nu, dat ziende, verwonderde zich over het gezicht; en als hij derwaarts ging, om dat te bezien, zo geschiedde een stem des Heeren tot hem,
\par 32 Zeggende: Ik ben de God uwer vaderen, de God Abrahams, en de God Izaks, en de God Jakobs. En Mozes werd zeer bevende, en durfde het niet bezien.
\par 33 En de Heere zeide tot hem: Ontbind de schoenen van uw voeten; want de plaats in welke gij staat, is heilig land.
\par 34 Ik heb merkelijk gezien de mishandeling Mijns volks, dat in Egypte is, en Ik heb hun zuchten gehoord en ben nedergekomen, om hen daaruit te verlossen; en nu, kom herwaarts, Ik zal u naar Egypte zenden.
\par 35 Dezen Mozes, welken zij verloochend hadden, zeggende: Wie heeft u tot een overste en rechter gesteld? dezen, zeg ik, heeft God tot een overste en verlosser gezonden, door de hand des Engels, Die hem verschenen was in het doornenbos.
\par 36 Deze heeft hen uitgeleid, doende wonderen en tekenen in het land van Egypte, en in de Rode zee, en in de woestijn, veertig jaren.
\par 37 Deze is de Mozes, die tot de kinderen Israels gezegd heeft: De Heere, uw God, zal u een Profeet verwekken uit uw broederen, gelijk mij; Dien zult gij horen.
\par 38 Deze is het, die in de vergadering des volks in de woestijn was met den Engel, Die tot hem sprak op den berg Sinai, en met onze vaderen; welke de levende woorden ontving, om ons die te geven.
\par 39 Denwelken onze vaders niet wilden gehoorzaam zijn, maar verwierpen hem, en keerden met hun harten weder naar Egypte;
\par 40 Zeggende tot Aaron: Maak ons goden, die voor ons heengaan; want wat dezen Mozes aangaat, die ons uit het land van Egypte geleid heeft, wij weten niet, wat hem geschied is.
\par 41 En zij maakten een kalf in die dagen, en brachten offerande tot den afgod, en verheugden zich in de werken hunner handen.
\par 42 En God keerde Zich, en gaf hen over, dat zij het heir des hemels dienden, gelijk geschreven is in het boek der profeten: Hebt gij ook slachtofferen en offeranden Mij opgeofferd, veertig jaren in de woestijn, gij huis Israels?
\par 43 Ja, gij hebt opgenomen den tabernakel van Moloch, en het gesternte van uw god Remfan, de afbeeldingen, die gij gemaakt hebt, om die te aanbidden; en Ik zal u overvoeren op gene zijde van Babylon.
\par 44 De tabernakel der getuigenis was onder onze vaderen in de woestijn, gelijk geordineerd had Hij, Die tot Mozes zeide, dat hij denzelven maken zou naar de afbeelding, die hij gezien had;
\par 45 Welken ook onze vaders ontvangen hebbende, met Jozua gebracht hebben in het land, dat de heidenen bezaten, die God verdreven heeft van het aangezicht onzer vaderen, tot de dagen van David toe;
\par 46 Dewelke voor God genade gevonden heeft, en begeerd heeft te vinden een woonstede voor den God Jakobs.
\par 47 En Salomo bouwde Hem een huis.
\par 48 Maar de Allerhoogste woont niet in tempelen met handen gemaakt; gelijk de profeet zegt:
\par 49 De hemel is Mij een troon, en de aarde een voetbank Mijner voeten. Hoedanig huis zult gij Mij bouwen, zegt de Heere, of welke is de plaats Mijner ruste?
\par 50 Heeft niet Mijn hand al deze dingen gemaakt?
\par 51 Gij hardnekkigen en onbesnedenen van hart en oren, gij wederstaat altijd den Heiligen Geest; gelijk uw vaders, alzo ook gij.
\par 52 Wien van de profeten hebben uw vaders niet vervolgd? En zij hebben gedood degenen, die te voren verkondigd hebben de komst des Rechtvaardigen, van Welken gijlieden nu verraders en moordenaars geworden zijt.
\par 53 Gij, die de wet ontvangen hebt door bestellingen der engelen, en hebt ze niet gehouden!
\par 54 Als zij nu dit hoorden, berstten hun harten, en zij knersten de tanden tegen hem.
\par 55 Maar hij, vol zijnde des Heiligen Geestes, en de ogen houdende naar den hemel, zag de heerlijkheid Gods, en Jezus, staande ter rechter hand Gods.
\par 56 En hij zeide: Ziet, ik zie de hemelen geopend, en den Zoon des mensen, staande ter rechter hand Gods.
\par 57 Maar zij, roepende met grote stemme, stopten hun oren, en vielen eendrachtelijk op hem aan;
\par 58 En wierpen hem ter stad uit, en stenigden hem; en de getuigen legden hun klederen af aan de voeten eens jongelings, genaamd Saulus.
\par 59 En zij stenigden Stefanus, aanroepende en zeggende: Heere Jezus, ontvang mijn geest.
\par 60 En vallende op de knieen, riep hij met grote stem: Heere, reken hun deze zonde niet toe! En als hij dat gezegd had, ontsliep hij.

\chapter{8}

\par 1 En Saulus had mede een welbehagen aan zijn dood. En er werd te dien dage een grote vervolging tegen de Gemeente, die te Jeruzalem was; en zij werden allen verstrooid door de landen van Judea en Samaria, behalve de apostelen.
\par 2 En enige godvruchtige mannen droegen Stefanus te zamen ten grave en maakten groten rouw over hem.
\par 3 En Saulus verwoestte de Gemeente, gaande in de huizen; en trekkende mannen en vrouwen, leverde hen over in de gevangenis.
\par 4 Zij dan nu, die verstrooid waren, gingen het land door, en verkondigden het Woord.
\par 5 En Filippus kwam af in de stad van Samaria, en predikte hun Christus.
\par 6 En de scharen hielden zich eendrachtelijk aan hetgeen van Filippus gezegd werd, dewijl zij hoorden en zagen de tekenen, die hij deed.
\par 7 Want van velen, die onreine geesten hadden, gingen dezelve uit, roepende met grote stem; en vele geraakten en kreupelen werden genezen.
\par 8 En er werd grote blijdschap in die stad.
\par 9 En een zeker man, met name Simon, was te voren in de stad plegende toverij, en verrukkende de zinnen des volks van Samaria, zeggende van zichzelven, dat hij wat groots was.
\par 10 Welken zij allen aanhingen, van den kleine tot den grote, zeggende: Deze is de grote kracht Gods.
\par 11 En zij hingen hem aan, omdat hij een langen tijd met toverijen hun zinnen verrukt had.
\par 12 Maar toen zij Filippus geloofden, die het Evangelie van het Koninkrijk Gods, en van den Naam van Jezus Christus verkondigde, werden zij gedoopt, beiden, mannen en vrouwen.
\par 13 En Simon geloofde ook zelf, en gedoopt zijnde, bleef gedurig bij Filippus; en ziende de tekenen en grote krachten, die er geschiedden, ontzette hij zich.
\par 14 Als nu de apostelen, die te Jeruzalem waren, hoorden, dat Samaria het Woord Gods aangenomen had, zonden zij tot hen Petrus en Johannes;
\par 15 Dewelken, afgekomen zijnde, baden voor hen, dat zij den Heiligen Geest ontvangen mochten.
\par 16 (Want Hij was nog op niemand van hen gevallen, maar zij waren alleenlijk gedoopt in den Naam van den Heere Jezus.)
\par 17 Toen legden zij de handen op hen, en zij ontvingen den Heiligen Geest.
\par 18 En als Simon zag, dat, door de oplegging van de handen der apostelen de Heilige Geest gegeven werd, zo bood hij hun geld aan,
\par 19 Zeggende: Geeft ook mij deze macht, opdat, zo wien ik de handen opleg, hij den Heiligen Geest ontvange.
\par 20 Maar Petrus zeide tot hem: Uw geld zij met u ten verderve, omdat gij gemeend hebt, dat de gave Gods door geld verkregen wordt!
\par 21 Gij hebt geen deel noch lot in dit woord: want uw hart is niet recht voor God.
\par 22 Bekeer u dan van deze uw boosheid, en bid God, of misschien u deze overlegging uws harten vergeven wierd.
\par 23 Want ik zie, dat gij zijt in een gans bittere gal en samenknoping der ongerechtigheid.
\par 24 Doch Simon, antwoordende, zeide: Bidt gijlieden voor mij tot den Heere, opdat niets over mij kome van hetgeen gij gezegd hebt.
\par 25 Zij dan nu, als zij het Woord des Heeren betuigd en gesproken hadden, keerden wederom naar Jeruzalem, en verkondigden het Evangelie in vele vlekken der Samaritanen.
\par 26 En een engel des Heeren sprak tot Filippus, zeggende: Sta op, en ga heen tegen het zuiden, op den weg, die van Jeruzalem afdaalt naar Gaza, welke woest is.
\par 27 En hij stond op en ging heen; en ziet, een Moorman, een kamerling, en een machtig heer van Candace, de koningin der Moren, die over al haar schat was, welke was gekomen om aan te bidden te Jeruzalem;
\par 28 En hij keerde wederom, en zat op zijn wagen, en las den profeet Jesaja.
\par 29 En de Geest zeide tot Filippus: Ga toe, en voeg u bij dezen wagen.
\par 30 En Filippus liep toe, en hoorde hem den profeet Jesaja lezen, en zeide: Verstaat gij ook, hetgeen gij leest?
\par 31 En hij zeide: Hoe zou ik toch kunnen, zo mij niet iemand onderricht? En hij bad Filippus, dat hij zou opkomen, en bij hem zitten.
\par 32 En de plaats der Schriftuur, die hij las, was deze: Hij is gelijk een schaap ter slachting geleid; en gelijk een lam stemmeloos is voor dien, die het scheert, alzo doet Hij Zijn mond niet open.
\par 33 In Zijn vernedering is Zijn oordeel weggenomen; en wie zal Zijn geslacht verhalen? Want Zijn leven wordt van de aarde weggenomen.
\par 34 En de kamerling antwoordde Filippus en zeide: Ik bid u, van Wien zegt de profeet dit, van zichzelven, of van iemand anders?
\par 35 En Filippus deed zijn mond open en beginnende van diezelfde Schrift, verkondigde hem Jezus.
\par 36 En alzo zij over weg reisden, kwamen zij aan een zeker water; en de kamerling zeide: Ziedaar water; wat verhindert mij gedoopt te worden?
\par 37 En Filippus zeide: Indien gij van ganser harte gelooft, zo is het geoorloofd. En hij, antwoordende, zeide: Ik geloof, dat Jezus Christus de Zoon van God is.
\par 38 En hij gebood den wagen stil te houden; en zij daalden beiden af in het water, zo Filippus als de kamerling, en hij doopte hem.
\par 39 En toen zij uit het water waren opgekomen, nam de Geest des Heeren Filippus weg, en de kamerling zag hem niet meer; want hij reisde zijn weg met blijdschap.
\par 40 Maar Filippus werd gevonden, te Azote; en het land doorgaande, verkondigde hij het Evangelie in alle steden, totdat hij te Cesarea kwam.

\chapter{9}

\par 1 En Saulus, blazende nog dreiging en moord tegen de discipelen des Heeren, ging tot de hogepriester,
\par 2 En begeerde brieven van hem naar Damaskus, aan de synagogen, opdat, zo hij enigen, die van dien weg waren, vond, hij dezelve, beiden mannen en vrouwen, zou gebonden brengen naar Jeruzalem.
\par 3 En als hij reisde, is het geschied, dat hij nabij Damaskus kwam, en hem omscheen snellijk een licht van den hemel;
\par 4 En ter aarde gevallen zijnde, hoorde hij een stem, die tot hem zeide: Saul, Saul! wat vervolgt gij Mij?
\par 5 En hij zeide: Wie zijt Gij, Heere? En de Heere zeide: Ik ben Jezus, Dien gij vervolgt. Het is u hard, de verzenen tegen de prikkels te slaan.
\par 6 En hij, bevende en verbaasd zijnde, zeide: Heere, wat wilt Gij, dat ik doen zal? En de Heere zeide tot hem: Sta op, en ga in de stad, en u zal aldaar gezegd worden, wat gij doen moet.
\par 7 En de mannen, die met hem over weg reisden, stonden verbaasd, horende wel de stem, maar niemand ziende.
\par 8 En Saulus stond op van de aarde; en als hij zijn ogen opendeed, zag hij niemand. En zij, hem bij de hand leidende, brachten hem te Damaskus.
\par 9 En hij was drie dagen, dat hij niet zag, en at niet, en dronk niet.
\par 10 En er was een zeker discipel te Damaskus, met name Ananias; en de Heere zeide tot hem in een gezicht: Ananias! En hij zeide: Zie, hier ben ik, Heere!
\par 11 En de Heere zeide tot hem: Sta op, en ga in de straat, genaamd de Rechte, en vraag in het huis van Judas naar een, met name Saulus, van Tarsen; want zie, hij bidt.
\par 12 En hij heeft in een gezicht gezien, dat een man, met name Ananias, inkwam, en hem de hand oplegde, opdat hij wederom ziende werd.
\par 13 En Ananias antwoordde: Heere! ik heb uit velen gehoord van dezen man, hoeveel kwaad hij Uw heiligen in Jeruzalem gedaan heeft;
\par 14 En heeft hier macht van de overpriesters, om te binden allen, die Uw Naam aanroepen.
\par 15 Maar de Heere zeide tot hem: Ga heen; want deze is Mij een uitverkoren vat, om Mijn Naam te dragen voor de heidenen, en de koningen, en de kinderen Israels.
\par 16 Want Ik zal hem tonen, hoeveel hij lijden moet om Mijn Naam.
\par 17 En Ananias ging heen en kwam in het huis; en de handen op hem leggende, zeide hij: Saul, broeder! de Heere heeft mij gezonden, namelijk Jezus, Die u verschenen is op den weg, dien gij kwaamt, opdat gij weder ziende en met den Heiligen Geest vervuld zoudt worden.
\par 18 En terstond vielen af van zijn ogen gelijk als schellen, en hij werd terstond wederom ziende; en stond op, en werd gedoopt.
\par 19 En als hij spijze genomen had, werd hij versterkt. En Saulus was sommige dagen bij de discipelen, die te Damaskus waren.
\par 20 En hij predikte terstond Christus in de synagogen, dat Hij de Zoon van God is.
\par 21 En zij ontzetten zich allen, die het hoorden, en zeiden: Is deze niet degene, die te Jeruzalem verstoorde, wie dezen Naam aanriepen, en die daarom hier gekomen is, opdat hij dezelve gebonden zou brengen tot de overpriesters?
\par 22 Doch Saulus werd meer en meer bekrachtigd, en overtuigde de Joden, die te Damaskus woonden, bewijzende, dat deze de Christus is.
\par 23 En als vele dagen verlopen waren, zo hielden de Joden te zamen raad, om hem te doden.
\par 24 Maar hun lage werd Saulus bekend; en zij bewaarden de poorten, beide des daags en des nachts, opdat zij hem doden mochten.
\par 25 Doch de discipelen namen hem des nachts, en lieten hem neder door den muur, hem aflatende in een mand.
\par 26 Saulus nu, te Jeruzalem gekomen zijnde, poogde zich bij de discipelen te voegen; maar zij vreesden hem allen, niet gelovende, dat hij een discipel was.
\par 27 Maar Barnabas, hem tot zich nemende, leidde hem tot de apostelen, en verhaalde hun, hoe hij op den weg den Heere gezien had, en dat Hij tot hem gesproken had; en hoe hij te Damaskus vrijmoediglijk gesproken had in den Naam van Jezus.
\par 28 En hij was met hen ingaande en uitgaande te Jeruzalem;
\par 29 En vrijmoediglijk sprekende in den Naam van den Heere Jezus, sprak hij ook, en handelde tegen de Griekse Joden; maar deze trachtten hem te doden.
\par 30 Doch de broeders, dit verstaande geleidden hem tot Cesarea, en zonden hem af naar Tarsen.
\par 31 De Gemeenten dan, door geheel Judea, en Galilea, en Samaria, hadden vrede, en werden gesticht; en wandelende in de vreze des Heeren, en de vertroosting des Heiligen Geestes, werden vermenigvuldigd.
\par 32 En het geschiedde, als Petrus alom doortrok, dat hij ook afkwam tot de heiligen, die te Lydda woonden.
\par 33 En aldaar vond hij een zeker mens, met name Eneas, die acht jaren te bed gelegen had, welke geraakt was.
\par 34 En Petrus zeide tot hem: Eneas! Jezus Christus maakt u gezond; sta op en spreid uzelven het bed. En hij stond terstond op.
\par 35 En zij zagen hem allen, die te Lydda en Sarona woonden, dewelke zich bekeerden tot den Heere.
\par 36 En te Joppe was een zekere discipelin, met name Tabitha, hetwelk overgezet zijnde, is gezegd Dorkas. Deze was vol van goede werken en aalmoezen, die zij deed.
\par 37 En het geschiedde in die dagen, dat zij krank werd en stierf; en als zij haar gewassen hadden, legden zij haar in de opperzaal.
\par 38 En alzo Lydda nabij Joppe was, de discipelen, horende, dat Petrus aldaar was, zonden twee mannen tot hem, biddende, dat hij niet zou vertoeven tot hen over te komen.
\par 39 En Petrus stond op, en ging met hen; welken zij, als hij daar gekomen was, in de opperzaal leidden. En al de weduwen stonden bij hem, wenende, en tonende de rokken en klederen, die Dorkas gemaakt had, als zij bij haar was.
\par 40 Maar Petrus, hebbende hen allen uitgedreven, knielde neder en bad: en zich kerende tot het lichaam, zeide hij: Tabitha, sta op! En zij deed haar ogen open, en Petrus gezien hebbende, zat zij over einde.
\par 41 En hij gaf haar de hand, en richtte haar op, en de heiligen en de weduwen geroepen hebbende, stelde hij haar levend voor hen.
\par 42 En dit werd bekend door geheel Joppe, en velen geloofden in den Heere.
\par 43 En het geschiedde, dat hij vele dagen te Joppe bleef, bij een zekeren Simon, een lederbereider.

\chapter{10}

\par 1 En er was een zeker man te Cesarea, met name Cornelius, een hoofdman over honderd, uit de bende, genaamd de Italiaanse;
\par 2 Godzalig en vrezende God, met geheel zijn huis, en doende vele aalmoezen aan het volk, en God geduriglijk biddende.
\par 3 Deze zag in een gezicht klaarlijk, omtrent de negende ure des daags, een engel Gods tot hem inkomen, en tot hem zeggende: Cornelius!
\par 4 En hij, de ogen op hem houdende, en zeer bevreesd geworden zijnde, zeide: Wat is het Heere? En hij zeide tot hem: Uw gebeden en uw aalmoezen zijn tot gedachtenis opgekomen voor God.
\par 5 En nu, zend mannen naar Joppe, en ontbied Simon, die toegenaamd wordt Petrus.
\par 6 Deze ligt te huis bij een Simon, lederbereider, die zijn huis heeft bij de zee; deze zal u zeggen, wat gij doen moet.
\par 7 En als de engel, die tot Cornelius sprak, weggegaan was, riep hij twee van zijn huisknechten, en een godzaligen krijgsknecht van degenen, die gedurig bij hem waren;
\par 8 En als hij hun alles verhaald had, zond hij hen naar Joppe.
\par 9 En des anderen daags, terwijl deze reisden, en nabij de stad kwamen, klom Petrus op het dak, om te bidden, omtrent de zesde ure.
\par 10 En hij werd hongerig, en begeerde te eten. En terwijl zij het bereidden, viel over hem een vertrekking van zinnen.
\par 11 En hij zag den hemel geopend, en een zeker vat tot hem nederdalen, gelijk een groot linnen laken, aan de vier hoeken gebonden, en nedergelaten op de aarde;
\par 12 In hetwelk waren al de viervoetige dieren der aarde, en de wilde, en de kruipende dieren, en de vogelen des hemels.
\par 13 En er geschiedde een stem tot hem: Sta op, Petrus! slacht en eet.
\par 14 Maar Petrus zeide: Geenszins, Heere! want ik heb nooit gegeten iets, dat gemeen of onrein was.
\par 15 En een stem geschiedde wederom ten tweeden male tot hem: Hetgeen God gereinigd heeft, zult gij niet gemeen maken.
\par 16 En dit geschiedde tot drie maal; en het vat werd wederom opgenomen in den hemel.
\par 17 En alzo Petrus in zichzelven twijfelde, wat toch het gezicht mocht zijn, dat hij gezien had, ziet, de mannen, die van Cornelius afgezonden waren, gevraagd hebbende naar het huis van Simon, stonden aan de poort.
\par 18 En iemand geroepen hebbende, vraagden zij, of Simon, toegenaamd Petrus, daar te huis lag.
\par 19 En als Petrus over dat gezicht dacht, zeide de Geest tot hem: Zie, drie mannen zoeken u;
\par 20 Daarom sta op, en ga af, en reis met hen, niet twijfelende; want ik heb hen gezonden.
\par 21 En Petrus ging af tot de mannen die van Cornelius tot hem gezonden waren, en zeide: Ziet, ik ben het, dien gij zoekt; wat is de oorzaak, waarom gij hier zijt?
\par 22 En zij zeiden: Cornelius, een hoofdman over honderd, een rechtvaardig man, en vrezende God, en die goede getuigenis heeft van het ganse volk der Joden, is door Goddelijke openbaring vermaand van een heiligen engel, dat hij u zou ontbieden te zijnen huize, en dat hij van u woorden der zaligheid zou horen.
\par 23 Als hij hen dan ingeroepen had, ontving hij ze in huis. Doch des anderen daags ging Petrus met hen heen, en sommigen der broederen, die van Joppe waren, gingen met hem.
\par 24 En des anderen daags kwamen zij te Cesarea. En Cornelius verwachtte hen, samengeroepen hebbende die van zijn maagschap en bijzonderste vrienden.
\par 25 En als het geschiedde, dat Petrus inkwam, ging hem Cornelius tegemoet, en vallende aan zijn voeten, aanbad hij.
\par 26 Maar Petrus richtte hem op, zeggende: Sta op, ik ben ook zelf een mens.
\par 27 En met hem sprekende, ging hij in, en vond er velen, die samengekomen waren.
\par 28 En hij zeide tot hen: Gij weet, hoe het een Joodsen man ongeoorloofd is, zich te voegen of te gaan tot een vreemde; doch God heeft mij getoond, dat ik geen mens zou gemeen of onrein heten.
\par 29 Daarom ben ik ook zonder tegenspreken gekomen, ontboden zijnde. Zo vraag ik dan, om wat reden gijlieden mij hebt ontboden.
\par 30 En Cornelius zeide: Over vier dagen was ik vastende tot deze ure toe, en ter negende ure bad ik in mijn huis.
\par 31 En ziet, een man stond voor mij, in een blinkend kleed, en zeide: Cornelius! uw gebed is verhoord, en uw aalmoezen zijn voor God gedacht geworden.
\par 32 Zend dan naar Joppe, en ontbied Simon, die toegenaamd wordt Petrus; deze ligt te huis in het huis van Simon, den lederbereider, aan de zee, welke, hier gekomen zijnde, tot u spreken zal.
\par 33 Zo heb ik dan van stonde aan tot u gezonden, en gij hebt welgedaan, dat gij hier gekomen zijt. Wij zijn dan allen nu hier tegenwoordig voor God, om te horen al hetgeen u van God bevolen is.
\par 34 En Petrus, den mond opendoende, zeide: Ik verneem in der waarheid, dat God geen aannemer des persoons is;
\par 35 Maar in allen volke, die Hem vreest en gerechtigheid werkt, is Hem aangenaam.
\par 36 Dit is het woord, dat Hij gezonden heeft den kinderen Israels, verkondigende vrede door Jezus Christus; deze is een Heere van allen.
\par 37 Gijlieden weet de zaak, die geschied is door geheel Judea, beginnende van Galilea, na den doop, welken Johannes gepredikt heeft;
\par 38 Belangende Jezus van Nazareth, hoe Hem God gezalfd heeft met den Heiligen Geest en met kracht; Welke het land doorgegaan is, goeddoende, en genezende allen, die van den duivel overweldigd waren; want God was met Hem.
\par 39 En wij zijn getuigen van al hetgeen Hij gedaan heeft, beide in het Joodse land en te Jeruzalem; Welken zij gedood hebben, Hem hangende aan het hout.
\par 40 Dezen heeft God opgewekt ten derden dage, en gegeven, dat Hij openbaar zou worden;
\par 41 Niet al den volke, maar den getuigen, die van God te voren verkoren waren, ons namelijk, die met Hem gegeten en gedronken hebben, nadat Hij uit de doden opgestaan was.
\par 42 En heeft ons geboden den volke te prediken, en te betuigen, dat Hij is Degene, Die van God verordend is tot een Rechter van levenden en doden.
\par 43 Dezen geven getuigenis al de profeten, dat een iegelijk, die in Hem gelooft, vergeving der zonden ontvangen zal door Zijn Naam.
\par 44 Als Petrus nog deze woorden sprak, viel de Heilige Geest op allen, die het Woord hoorden.
\par 45 En de gelovigen, die uit de besnijdenis waren, zovelen als met Petrus gekomen waren, ontzetten zich, dat de gave des Heiligen Geestes ook op de heidenen uitgestort werd.
\par 46 Want zij hoorden hen spreken met vreemde talen, en God groot maken. Toen antwoordde Petrus:
\par 47 Kan ook iemand het water weren, dat dezen niet gedoopt zouden worden, welke den Heiligen Geest ontvangen hebben, gelijk als ook wij?
\par 48 En hij beval, dat zij zouden gedoopt worden in den Naam des Heeren. Toen baden zij hem, dat hij enige dagen bij hen wilde blijven.

\chapter{11}

\par 1 De apostelen nu, en de broeders, die in Judea waren, hebben gehoord, dat ook de heidenen het Woord Gods aangenomen hadden.
\par 2 En toen Petrus opgegaan was naar Jeruzalem, twistten tegen hem degenen, die uit de besnijdenis waren,
\par 3 Zeggende: Gij zijt ingegaan tot mannen, die de voorhuid hebben, en hebt met hen gegeten.
\par 4 Maar Petrus, beginnende, verhaalde het hun vervolgens, zeggende:
\par 5 Ik was in de stad Joppe, biddende en zag in een vertrekking van zinnen een gezicht, namelijk een zeker vat, gelijk een groot linnen laken, nederdalende, bij de vier hoeken nedergelaten uit den hemel, en het kwam tot bij mij;
\par 6 Op welk laken als ik de ogen hield, zo merkte ik, en zag de viervoetige dieren der aarde, en de wilde, en de kruipende dieren, en de vogelen des hemels.
\par 7 En ik hoorde een stem, die tot mij zeide: Sta op, Petrus, slacht en eet.
\par 8 Maar ik zeide: Geenszins, Heere, want nooit is iets, dat gemeen of onrein was, in mijn mond ingegaan.
\par 9 Doch de stem antwoordde mij ten tweeden male uit den hemel: Hetgeen God gereinigd heeft, zult gij niet gemeen maken.
\par 10 En dit geschiedde tot driemaal; en alles werd wederom opgetrokken in den hemel.
\par 11 En ziet, ter zelfder ure stonden er drie mannen voor het huis, daar ik in was, die van Cesarea tot mij afgezonden waren.
\par 12 En de Geest zeide tot mij, dat ik met hen gaan zou, niet twijfelende. En met mij gingen ook deze zes broeders, en wij zijn in des man huis ingegaan.
\par 13 En hij heeft ons verhaald, hoe hij een engel gezien had, die in zijn huis stond, en tot hem zeide: Zend mannen naar Joppe, en ontbied Simon, die toegenaamd is Petrus;
\par 14 Die woorden tot u zal spreken, door welke gij zult zalig worden, en al uw huis.
\par 15 En als ik begon te spreken, viel de Heilige Geest op hen, gelijk ook op ons in het begin.
\par 16 En ik werd gedachtig aan het woord des Heeren, hoe Hij zeide: Johannes doopte wel met water, maar gijlieden zult gedoopt worden met den Heiligen Geest.
\par 17 Indien dan God hun evengelijke gave gegeven heeft, als ook ons, die in de Heere Jezus Christus geloofd hebben, wie was ik toch, die God konde weren?
\par 18 En als zij dit hoorden, waren zij tevreden, en verheerlijkten God, zeggende: Zo heeft dan God ook den heidenen de bekering gegeven ten leven!
\par 19 Degenen nu, die verstrooid waren door de verdrukking, die over Stefanus geschied was, gingen het land door tot Fenicie toe, en Cyprus, en Antiochie, tot niemand het Woord sprekende, dan alleen tot de Joden.
\par 20 En er waren enige Cyprische en Cyreneische mannen uit hen, welken te Antiochie gekomen zijnde, spraken tot de Grieksen, verkondigende den Heere Jezus.
\par 21 En de hand des Heeren was met hen; en een groot getal geloofde, en bekeerde zich tot den Heere.
\par 22 En het gerucht van hen kwam tot de oren der Gemeente, die te Jeruzalem was; en zij zonden Barnabas uit, dat hij het land doorging tot Antiochie toe.
\par 23 Dewelke, daar gekomen zijnde, en de genade Gods ziende, werd verblijd, en vermaande hen allen, dat zij met een voornemen des harten bij den Heere zouden blijven.
\par 24 Want hij was een goed man, en vol des Heiligen Geestes en des geloofs; en er werd een grote schare den Heere toegevoegd.
\par 25 En Barnabas ging uit naar Tarsen, om Saulus te zoeken; en als hij hem gevonden had, bracht hij hem te Antiochie.
\par 26 En het is geschied, dat zij een geheel jaar samen vergaderden in de Gemeente, en een grote schare leerden; en dat de discipelen eerst te Antiochie Christenen genaamd werden.
\par 27 En in dezelfde dagen kwamen enige profeten af van Jeruzalem te Antiochie.
\par 28 En een uit hen, met name Agabus, stond op, en gaf te kennen door den Geest, dat er een grote hongersnood zou wezen over de gehele wereld; dewelke ook gekomen is onder den keizer Claudius.
\par 29 En naardat een iegelijk der discipelen vermocht, besloot elk van hen iets te zenden ten dienste der broederen, die in Judea woonden.
\par 30 Hetwelk zij ook deden, en zonden het tot de ouderlingen, door de hand van Barnabas en Saulus.

\chapter{12}

\par 1 En omtrent denzelfden tijd sloeg de koning Herodes de handen aan sommigen van de Gemeente, om die kwalijk te handelen.
\par 2 En hij doodde Jakobus, den broeder van Johannes, met het zwaard.
\par 3 En toen hij zag, dat het den Joden behagelijk was, voer hij voort, om ook Petrus te vangen (en het waren de dagen der ongehevelde broden);
\par 4 Denwelken ook gegrepen hebbende, hij in de gevangenis zette, en gaf hem over aan vier wachten, elk van vier krijgsknechten, om hem te bewaren, willende na het paas feest hem voorbrengen voor het volk.
\par 5 Petrus dan werd in de gevangenis bewaard; maar van de Gemeente werd een gedurig gebed tot God voor hem gedaan.
\par 6 Toen hem nu Herodes zou voorbrengen, sliep Petrus dienzelfden nacht tussen twee krijgsknechten, gebonden met twee ketenen; en de wachters voor de deur bewaarden de gevangenis.
\par 7 En ziet, een engel des Heeren stond daar, en een licht scheen in de woning, en slaande de zijde van Petrus, wekte hij hem op, zeggende: Sta haastelijk op. En zijn ketenen vielen af van de handen.
\par 8 En de engel zeide tot hem: Omgord u, en bind uw schoenzolen aan. En hij deed alzo. En hij zeide tot hem: Werp uw mantel om, en volg mij.
\par 9 En uitgaande volgde hij hem, en wist niet, dat het waarachtig was, hetgeen door den engel geschiedde, maar hij meende, dat hij een gezicht zag.
\par 10 En als zij door de eerste en tweede wacht gegaan waren, kwamen zij aan de ijzeren poort, die naar de stad leidt; dewelke van zelve hun geopend werd. En uitgegaan zijnde, gingen zij een straat voort, en terstond scheidde de engel van hem.
\par 11 En Petrus, tot zichzelven gekomen zijnde, zeide: Nu weet ik waarachtiglijk dat de Heere Zijn engel uitgezonden heeft, en mij verlost heeft uit de hand van Herodes, en uit al de verwachting van het volk der Joden.
\par 12 En als hij alles overlegd had, ging hij naar het huis van Maria, de moeder van Johannes, die toegenaamd was Markus, alwaar velen samenvergaderd en biddende waren.
\par 13 En als Petrus aan de deur van de voorpoort klopte, kwam een dienstmaagd voor om te luisteren, met name Rhode.
\par 14 En zij de stem van Petrus bekennende, deed van blijdschap de voorpoort niet open, maar liep naar binnen en boodschapte, dat Petrus voor aan de voorpoort stond.
\par 15 En zij zeiden tot haar: Gij raast. Doch zij bleef er sterk bij, dat het alzo was. En zij zeiden: Het is zijn engel.
\par 16 Maar Petrus bleef kloppende: en als zij opengedaan hadden, zagen zij hem, en ontzetten zich.
\par 17 En als hij hen met de hand gewenkt had, dat zij zwijgen zouden, verhaalde hij hun, hoe hem de Heere uit de gevangenis uitgeleid had, en zeide: Boodschapt dit aan Jakobus en de broederen. En hij uitgegaan zijnde, reisde naar een andere plaats.
\par 18 En als het dag was geworden, was er geen kleine beroerte onder de krijgsknechten, wat toch aan Petrus mocht geschied zijn.
\par 19 En als Herodes hem gezocht had, en niet vond, en de wachters rechtelijk ondervraagd had, gebood hij, dat zij weggeleid zouden worden. En hij vertrok van Judea naar Cesarea, en hield zich aldaar.
\par 20 En Herodes had in den zin tegen de Tyriers en Sidoniers te krijgen; maar zij kwamen eendrachtelijk tot hem, en Blastus, die des konings kamerling was, overreed hebbende, begeerden vrede, omdat hun land gespijzigd werd van des konings land.
\par 21 En op een gezetten dag, Herodes, een koninklijk kleed aangedaan hebbende, en op den rechterstoel gezeten zijnde, deed een rede tot hen.
\par 22 En het volk riep hem toe: Een stem Gods, en niet eens mensen!
\par 23 En van stonde aan sloeg hem een engel des Heeren, daarom dat hij Gode de eer niet gaf; en hij werd van de wormen gegeten, en gaf den geest.
\par 24 En het Woord Gods wies, en vermenigvuldigde.
\par 25 Barnabas nu en Saulus keerden wederom van Jeruzalem, als zij den dienst volbracht hadden, medegenomen hebbende ook Johannes, die toegenaamd werd Markus.

\chapter{13}

\par 1 En er waren te Antiochie, in de Gemeente, die daar was, enige profeten en leraars, namelijk Barnabas, en Simeon, genaamd Niger, en Lucius van Cyrene, en Manahen, die met Herodes den viervorst opgevoed was, en Saulus.
\par 2 En als zij den Heere dienden, en vastten, zeide de Heilige Geest: Zondert Mij af beiden Barnabas en Saulus tot het werk, waartoe Ik hen geroepen heb.
\par 3 Toen vastten en baden zij, en hun de handen opgelegd hebbende, lieten zij hen gaan.
\par 4 Dezen dan, uitgezonden zijnde van den Heiligen Geest, kwamen af tot Seleucie, en van daar scheepten zij af naar Cyprus.
\par 5 En gekomen zijnde te Salamis, verkondigden zij het woord Gods in de synagogen der Joden; en zij hadden ook Johannes tot een dienaar.
\par 6 En als zij het eiland doorgegaan waren tot Pafos toe, vonden zij een zekeren tovenaar, een valse profeet, een Jood, wiens naam was Bar-jezus;
\par 7 Welke was bij den stadhouder Sergius Paulus, een verstandigen man. Deze, Barnabas en Saulus tot zich geroepen hebbende, zocht zeer het Woord Gods te horen.
\par 8 Maar Elymas, de tovenaar (want alzo wordt zijn naam overgezet), wederstond hen, zoekende den stadhouder van het geloof af te keren.
\par 9 Doch Saulus (die ook Paulus genaamd is), vervuld met den Heiligen Geest, en de ogen op hem houdende, zeide:
\par 10 O gij kind des duivels, vol van alle bedrog, en van alle arglistigheid, vijand van alle gerechtigheid, zult gij niet ophouden te verkeren de rechte wegen des Heeren?
\par 11 En nu zie, de hand des Heeren is tegen u, en gij zult blind zijn, en de zon niet zien voor een tijd. En van stonde aan viel op hem donkerheid en duisternis: en rondom gaande, zocht hij, die hem met de hand mochten leiden.
\par 12 Als de stadhouder zag, hetgeen geschied was, toen geloofde hij, verslagen zijnde over de leer des Heeren.
\par 13 En Paulus, en die met hem waren, van Pafos afgevaren zijnde, kwamen te Perge, een stad in Pamfylie. Maar Johannes, van hen scheidende, keerde weder naar Jeruzalem.
\par 14 En zij, van Perge het land doorgaande, kwamen te Antiochie, een stad in Pisidie; en gegaan zijnde in de synagoge op den dag des sabbats, zaten zij neder.
\par 15 En na het lezen der wet en der profeten, zonden de oversten der synagogen tot hen, zeggende: Mannen broeders, indien er enig woord van vertroosting tot het volk in u is, zo spreekt.
\par 16 En Paulus stond op, en wenkte met de hand, en zeide: Gij Israelietische mannen, en gij, die God vreest, hoort toe.
\par 17 De God van dit volk Israel heeft onze vaderen uitverkoren, en het volk verhoogd, als zij vreemdelingen waren in het land Egypte, en heeft hen met een hogen arm daaruit geleid.
\par 18 En heeft omtrent den tijd van veertig jaren hun zeden verdragen in de woestijn.
\par 19 En zeven volken uitgeroeid hebbende in het land Kanaan, heeft Hij hun door het lot het land derzelve uitgedeeld.
\par 20 En daarna omtrent vierhonderd en vijftig jaren, gaf Hij hun rechters, tot op Samuel, den profeet.
\par 21 En van toen aan begeerden zij een koning; en God gaf hun Saul, den zoon van Kis, een man uit den stam van Benjamin, veertig jaren.
\par 22 En dezen afgezet hebbende, verwekte Hij hun David tot een koning; denwelken Hij ook getuigenis gaf, en zeide: Ik heb gevonden David, den zoon van Jesse; een man naar Mijn hart, die al Mijn wil zal doen.
\par 23 Van het zaad dezes heeft God Israel, naar de belofte, verwekt den Zaligmaker Jezus;
\par 24 Als Johannes eerst al den volke Israels voor Zijn aankomst, gepredikt had den doop der bekering.
\par 25 Doch als Johannes den loop vervulde, zeide hij: Wien meent gijlieden, dat ik ben? Ik ben de Christus niet; maar ziet, Hij komt na mij, Wien ik niet waardig ben de schoenen Zijner voeten te ontbinden.
\par 26 Mannen broeders, kinderen van het geslacht Abrahams, en die onder u God vrezen, tot u is het woord dezer zaligheid gezonden.
\par 27 Want die te Jeruzalem wonen, en hun oversten, Dezen niet kennende, hebben ook de stemmen der profeten, die op elken sabbat dag gelezen worden, Hem veroordelende, vervuld;
\par 28 En geen oorzaak des doods vindende, hebben zij van Pilatus begeerd, dat Hij zou gedood worden.
\par 29 En als zij alles volbracht hadden, wat van Hem geschreven was, namen zij Hem af van het hout, en legden Hem in het graf.
\par 30 Maar God heeft Hem uit de doden opgewekt;
\par 31 Welke gezien is geweest, vele dagen lang, van degenen, die met Hem opgekomen waren van Galilea tot Jeruzalem, die Zijn getuigen zijn bij het volk.
\par 32 En wij verkondigen u de belofte, die tot de vaderen geschied is, dat namelijk God dezelve vervuld heeft aan ons, hun kinderen, als Hij Jezus verwekt heeft.
\par 33 Gelijk ook in den tweeden psalm geschreven staat: Gij zijt Mijn Zoon, heden heb Ik U gegenereerd.
\par 34 En dat Hij Hem uit de doden heeft opgewekt, alzo dat Hij niet meer zal tot verderving keren, heeft Hij aldus gezegd: Ik zal ulieden de weldadigheden Davids geven, die getrouw zijn;
\par 35 Waarom hij ook in een anderen psalm zegt: Gij zult Uw Heilige niet over geven, om verderving te zien.
\par 36 Want David, als hij in zijn tijd den raad Gods gediend had, is ontslapen, en is bij zijn vaderen gelegd; en heeft wel verderving gezien;
\par 37 Maar Hij, Dien God opgewekt heeft, heeft geen verderving gezien.
\par 38 Zo zij u dan bekend, mannen broeders, dat door Dezen u vergeving der zonden verkondigd wordt;
\par 39 En dat van alles, waarvan gij niet kondet gerechtvaardigd worden door de wet van Mozes, door Dezen een iegelijk, die gelooft, gerechtvaardigd wordt.
\par 40 Ziet dan toe, dat over ulieden niet kome, hetgeen gezegd is in de profeten:
\par 41 Ziet, gij verachters, en verwondert u, en verdwijnt; want Ik werk een werk in uw dagen, een werk, hetwelk gij niet zult geloven, zo het u iemand verhaalt.
\par 42 En als de Joden uitgegaan waren uit de synagoge, baden de heidenen, dat tegen den naasten sabbat hun dezelfde woorden zouden gesproken worden.
\par 43 En als de synagoge gescheiden was, volgden velen van de Joden en van de godsdienstige Jodengenoten Paulus en Barnabas; welke tot hen spraken, en hen vermaanden te blijven bij de genade Gods.
\par 44 En op den volgenden sabbat kwam bijna de gehele stad samen, om het Woord Gods te horen.
\par 45 Doch de Joden, de scharen ziende, werden met nijdigheid vervuld, en wederspraken, hetgeen van Paulus gezegd werd, wedersprekende en lasterende.
\par 46 Maar Paulus en Barnabas, vrijmoedigheid gebruikende, zeiden: Het was nodig, dat eerst tot u het Woord Gods gesproken zou worden; doch nademaal gij hetzelve verstoot, en uzelven des eeuwigen levens niet waardig oordeelt, ziet, wij keren ons tot de heidenen.
\par 47 Want alzo heeft ons de Heere geboden, zeggende: Ik heb u gesteld tot een licht der heidenen, opdat gij zoudt zijn tot zaligheid, tot aan het uiterste der aarde.
\par 48 Als nu de heidenen dit hoorden, verblijdden zij zich, en prezen het Woord des Heeren; en er geloofden zovelen, als er geordineerd waren tot het eeuwige leven.
\par 49 En het Woord des Heeren werd door het gehele land uitgebreid.
\par 50 Maar de Joden maakten op de godsdienstige en eerlijke vrouwen, en de voornaamsten van de stad, en verwekten vervolging tegen Paulus en Barnabas, en wierpen ze uit hun landpalen.
\par 51 Doch zij schudden het stof van hun voeten af tegen dezelve, en kwamen te Ikonium.
\par 52 En de discipelen werden vervuld met blijdschap en met den Heiligen Geest.

\chapter{14}

\par 1 En het geschiedde te Ikonium, dat zij te zamen gingen in de synagoge der Joden, en alzo spraken, dat een grote menigte, beiden van Joden en Grieken, geloofde.
\par 2 Maar de Joden, die ongehoorzaam waren, verwekten en verbitterden de zielen der heidenen tegen de broeders.
\par 3 Zij verkeerden dan aldaar een langen tijd, vrijmoediglijk sprekende in den Heere, Die getuigenis gaf aan het Woord Zijner genade, en gaf, dat tekenen en wonderen geschiedden door hun handen.
\par 4 En de menigte der stad werd verdeeld, en sommigen waren met de Joden, en sommigen met de apostelen.
\par 5 En als er een oploop geschiedde, beiden van heidenen en van Joden, met hun oversten, om hun smaadheid aan te doen, en hen te stenigen,
\par 6 Zijn zij, alles overlegd hebbende, gevlucht naar de steden van Lykaonie, namelijk Lystre en Derbe, en het omliggende land;
\par 7 En verkondigden aldaar het Evangelie.
\par 8 En een zeker man, te Lystre, zat onmachtig aan de voeten, kreupel zijnde van zijner moeders lijf, die nooit had gewandeld.
\par 9 Deze hoorde Paulus spreken; welke de ogen op hem houdende, en ziende, dat hij geloof had om gezond te worden,
\par 10 Zeide met grote stem: Sta recht op uw voeten! En hij sprong op en wandelde.
\par 11 En de scharen, ziende, hetgeen Paulus gedaan had, verhieven hun stemmen, en zeiden in het Lycaonisch: De goden zijn den mensen gelijk geworden, en tot ons nedergekomen.
\par 12 En zij noemden Barnabas Jupiter, en Paulus Mercurius, omdat hij het woord voerde.
\par 13 En de priester van Jupiter, die voor hun stad was, als hij ossen en kransen aan de voorpoorten gebracht had, wilde hij offeren met de scharen.
\par 14 Maar de apostelen, Barnabas en Paulus, dat horende, scheurden hun klederen, en sprongen onder de schare, roepende,
\par 15 En zeggende: Mannen, waarom doet gij deze dingen? Wij zijn ook mensen van gelijke bewegingen als gij, en verkondigen ulieden, dat gij u zoudt van deze ijdele dingen bekeren tot den levenden God, Die gemaakt heeft den hemel, en de aarde, en de zee, en al hetgeen in dezelve is;
\par 16 Welke in de verledene tijden al de heidenen heeft laten wandelen in hun wegen;
\par 17 Hoewel Hij nochtans Zichzelven niet onbetuigd gelaten heeft, goed doende van den hemel, ons regen en vruchtbare tijden gevende, vervullende onze harten met spijs en vrolijkheid.
\par 18 En dit zeggende, wederhielden zij nauwelijks de scharen, dat zij hun niet offerden.
\par 19 Maar daarover kwamen Joden van Antiochie en Ikonium, en overreedden de scharen, en stenigden Paulus, en sleepten hem buiten de stad, menende, dat hij dood was.
\par 20 Doch als hem de discipelen omringd hadden, stond hij op, en kwam in de stad; en des anderen daags ging hij met Barnabas uit naar Derbe.
\par 21 En als zij derzelve stad het Evangelie verkondigd en vele discipelen gemaakt hadden, keerden zij weder naar Lystre, en Ikonium, en Antiochie;
\par 22 Versterkende de zielen der discipelen, en vermanende, dat zij zouden blijven in het geloof, en dat wij door vele verdrukkingen moeten ingaan in het Koninkrijk Gods.
\par 23 En als zij in elke Gemeente, met opsteken der handen, ouderlingen verkoren hadden, gebeden hebbende met vasten, bevalen zij hen den Heere, in Welken zij geloofd hadden.
\par 24 En Pisidie doorgereisd hebbende, kwamen zij in Pamfylie.
\par 25 En als zij te Perge het Woord gesproken hadden, kwamen zij af naar Attalie.
\par 26 En van daar scheepten zij af naar Antiochie, van waar zij der genade Gods bevolen waren geweest tot het werk, dat zij volbracht hadden.
\par 27 En daar gekomen zijnde, en de Gemeente vergaderd hebbende, verhaalden zij, wat grote dingen God met hen gedaan had, en dat Hij den heidenen de deur des geloofs geopend had.
\par 28 En zij verkeerden aldaar geen kleinen tijd met de discipelen.

\chapter{15}

\par 1 En sommigen, die afgekomen waren van Judea, leerden de broederen, zeggende: Indien gij niet besneden wordt naar de wijze van Mozes, zo kunt gij niet zalig worden.
\par 2 Als er dan geen kleine wederstand en twisting geschiedde bij Paulus en Barnabas tegen hen, zo hebben zij geordineerd, dat Paulus en Barnabas, en enige anderen uit hen, zouden opgaan tot de apostelen en ouderlingen naar Jeruzalem, over deze vraag.
\par 3 Zij dan, van de Gemeente uitgeleid zijnde, reisden door Fenicie en Samarie, verhalende de bekering der heidenen; en deden al den broederen grote blijdschap aan.
\par 4 En te Jeruzalem gekomen zijnde, werden zij ontvangen van de Gemeente, en de apostelen, en de ouderlingen; en zij verkondigden, wat grote dingen God met hen gedaan had.
\par 5 Maar, zeiden zij, er zijn sommigen opgestaan van die van de sekte der Farizeen, die gelovig zijn geworden, zeggende, dat men hen moet besnijden, en gebieden de wet van Mozes te onderhouden.
\par 6 En de apostelen en de ouderlingen vergaderden te zamen, om op deze zaak te letten.
\par 7 En als daarover grote twisting geschiedde, stond Petrus op en zeide tot hen: Mannen broeders, gij weet, dat God van over langen tijd onder ons mij verkoren heeft, dat de heidenen door mijn mond het woord des Evangelies zouden horen, en geloven.
\par 8 En God, de Kenner der harten, heeft hun getuigenis gegeven, hun gevende den Heiligen Geest, gelijk als ook ons;
\par 9 En heeft geen onderscheid gemaakt tussen ons en hen, gereinigd hebbende hun harten door het geloof.
\par 10 Nu dan, wat verzoekt gij God, om een juk op den hals der discipelen te leggen, hetwelk noch onze vaders, noch wij hebben kunnen dragen?
\par 11 Maar wij geloven, door de genade van den Heere Jezus Christus, zalig te worden, op zulke wijze als ook zij.
\par 12 En al de menigte zweeg stil, en zij hoorden Barnabas en Paulus verhalen, wat grote tekenen en wonderen God door hen onder de heidenen gedaan had.
\par 13 En nadat deze zwegen, antwoordde Jakobus, zeggende: Mannen broeders, hoort mij.
\par 14 Simeon heeft verhaald hoe God eerst de heidenen heeft bezocht, om uit hen een volk aan te nemen door Zijn Naam.
\par 15 En hiermede stemmen overeen de woorden der profeten, gelijk geschreven is:
\par 16 Na dezen zal Ik wederkeren, en weder opbouwen de tabernakel van David, die vervallen is, en hetgeen daarvan verbroken is, weder opbouwen, en Ik zal denzelven weder oprichten.
\par 17 Opdat de overblijvende mensen den Heere zoeken, en al de heidenen, over welken Mijn Naam aangeroepen is, spreekt de Heere, Die dit alles doet.
\par 18 Gode zijn al Zijn werken van eeuwigheid bekend.
\par 19 Daarom oordeel ik, dat men degenen, die uit de heidenen zich tot God bekeren, niet beroere;
\par 20 Maar hun zal aanschrijven, dat zij zich onthouden van de dingen, die door de afgoden besmet zijn, en van hoererij, en van het verstikte, en van bloed.
\par 21 Want Mozes heeft er van oude tijden in elke stad, die hem prediken, en hij wordt op elken sabbat in de synagogen gelezen.
\par 22 Toen heeft het den apostelen en den ouderlingen, met de gehele Gemeente, goed gedacht, enige mannen uit zich te verkiezen, en met Paulus en Barnabas te zenden naar Antiochie: namelijk Judas, die toegenaamd wordt Barsabas, en Silas, mannen, die voorgangers waren onder de broeders.
\par 23 En zij schreven door hen dit navolgende: De apostelen, en de ouderlingen, en de broeders wensen den broederen uit de heidenen, die in Antiochie, en Syrie, en Cilicie zijn, zaligheid.
\par 24 Nademaal wij gehoord hebben, dat sommigen, die van ons uitgegaan zijn, u met woorden ontroerd hebben en uw zielen wankelende gemaakt, zeggende, dat gij moet besneden worden, en de wet onderhouden; welken wij dat niet bevolen hadden;
\par 25 Zo heeft het ons eendrachtelijk te zamen zijnde, goed gedacht, enige mannen te verkiezen, en tot u te zenden, met onze geliefden, Barnabas en Paulus.
\par 26 Mensen, die hun zielen overgegeven hebben voor den Naam van onzen Heere Jezus Christus.
\par 27 Wij hebben dan Judas en Silas gezonden, die ook met den mond hetzelfde zullen verkondigen.
\par 28 Want het heeft den Heiligen Geest en ons goed gedacht, ulieden geen meerderen last op te leggen dan deze noodzakelijke dingen:
\par 29 Namelijk, dat gij u onthoudt van hetgeen den afgoden geofferd is, en van bloed, en van het verstikte, en van hoererij; van welke dingen, indien gij uzelven wacht, zo zult gij weldoen. Vaart wel.
\par 30 Dezen dan, hun afscheid ontvangen hebbende, kwamen te Antiochie; en de menigte vergaderd hebbende, gaven zij den brief over.
\par 31 En zij, dien gelezen hebbende, verblijdden zich over de vertroosting.
\par 32 Judas nu en Silas, die ook zelven profeten waren, vermaanden de broeders met vele woorden, en versterkten hen.
\par 33 En als zij daar een tijd lang vertoefd hadden, lieten hen de broeders wederom gaan met vrede, tot de apostelen.
\par 34 Maar het dacht Silas goed aldaar te blijven.
\par 35 En Paulus en Barnabas onthielden zich te Antiochie, lerende en verkondigende met nog vele anderen, het Woord des Heeren.
\par 36 En na enige dagen zeide Paulus tot Barnabas: Laat ons nu wederkeren, en bezoeken onze broeders in elke stad, in welke wij het Woord des Heeren verkondigd hebben, hoe zij het hebben.
\par 37 En Barnabas ried, dat zij Johannes, die genaamd is Markus, zouden medenemen.
\par 38 Maar Paulus achtte billijk, dat men dien niet zoude medenemen, die van Pamfylie af van hen was afgeweken, en met hen niet was gegaan tot het werk.
\par 39 Er ontstond dan een verbittering, alzo dat zij van elkander gescheiden zijn, en dat Barnabas Markus medenam, en naar Cyprus afscheepte;
\par 40 Maar Paulus verkoos Silas, en reisde heen, der genade Gods van de broederen bevolen zijnde.
\par 41 En hij doorreisde Syrie en Cilicie, versterkende de Gemeenten.

\chapter{16}

\par 1 En hij kwam te Derbe en Lystre. En ziet, aldaar was een zeker discipel, met name Timotheus, zoon van een gelovige Joodse vrouw, maar van een Grieksen vader;
\par 2 Welken goeden getuigenis gegeven werd van de broederen te Lystre en Ikonium.
\par 3 Deze wilde Paulus, dat met hem zou reizen; en hij nam en besneed hem, om der Joden wil, die in die plaatsen waren; want zij kenden allen zijn vader, dat hij een Griek was.
\par 4 En alzo zij de steden doorreisden, gaven zij hun de verordeningen over, die van de apostelen en de ouderlingen te Jeruzalem goed gevonden waren, om die te onderhouden.
\par 5 De Gemeenten dan werden bevestigd in het geloof, en werden dagelijks overvloediger in getal.
\par 6 En als zij Frygie, en het land van Galatie doorgereisd hadden, werden zij van den Heiligen Geest verhinderd het Woord in Azie te spreken.
\par 7 En aan Mysie gekomen zijnde, poogden zij naar Bithynie te reizen; en de Geest liet het hun niet toe.
\par 8 En zij, Mysie voorbij gereisd zijnde, kwamen af tot Troas.
\par 9 En van Paulus werd in den nacht een gezicht gezien: er was een Macedonisch man staande, die hem bad en zeide: Kom over in Macedonie, en help ons.
\par 10 Als hij nu dit gezicht gezien had, zo zochten wij terstond naar Macedonie te reizen, besluitende daaruit, dat ons de Heere geroepen had, om denzelven het Evangelie te verkondigen.
\par 11 Van Troas dan afgevaren zijnde, liepen wij recht naar Samothrace, en den volgende dag naar Neapolis.
\par 12 En van daar naar Filippi, welke is de eerste stad van dit deel van Macedonie, een kolonie. En wij onthielden ons in die stad ettelijke dagen.
\par 13 En op den dag des sabbats gingen wij buiten de stad aan de rivier, waar het gebed placht te geschieden; en nedergezeten zijnde, spraken wij tot de vrouwen, die samengekomen waren.
\par 14 En een zekere vrouw, met name Lydia, een purperverkoopster, van de stad Thyatira, die God diende, hoorde ons; welker hart de Heere heeft geopend, dat zij acht nam op hetgeen van Paulus gesproken werd.
\par 15 En als zij gedoopt was, en haar huis, bad zij ons, zeggende: Indien gij hebt geoordeeld, dat ik den Heere getrouw ben, zo komt in mijn huis, en blijft er. En zij dwong ons.
\par 16 En het geschiedde, als wij tot het gebed heengingen, dat een zekere dienstmaagd, hebbende een waarzeggenden geest, ons ontmoette, welke haar heren groot gewin toebracht met waarzeggen.
\par 17 Dezelve volgde Paulus en ons achterna, en riep, zeggende: Deze mensen zijn dienstknechten Gods des Allerhoogsten, die ons den weg der zaligheid verkondigen.
\par 18 En dit deed zij vele dagen lang. Maar Paulus, daarover ontevreden zijnde, keerde zich om, en zeide tot den geest: Ik gebied u in den Naam van Jezus Christus, dat gij van haar uitgaat. En hij ging uit ter zelfder ure.
\par 19 Als nu de heren van dezelve zagen, dat de hoop huns gewins weg was, grepen zij Paulus en Silas, en trokken hen naar de markt voor de oversten.
\par 20 En als zij hen tot de hoofdmannen gebracht hadden, zeiden zij: Deze mensen beroeren onze stad, daar zij Joden zijn.
\par 21 En zij verkondigen zeden, die ons niet geoorloofd zijn aan te nemen noch te doen, alzo wij Romeinen zijn.
\par 22 En de schare stond gezamenlijk tegen hen op; en de hoofdmannen, hun de klederen afgescheurd hebbende, bevalen hen te geselen.
\par 23 En als zij hun vele slagen gegeven hadden, wierpen zij hen in de gevangenis, en geboden den stokbewaarder, dat hij hen zekerlijk bewaren zou.
\par 24 Dewelke, zulk een gebod ontvangen hebbende, wierp hen in den binnensten kerker, en verzekerde hun voeten in de stok.
\par 25 En omtrent den middernacht baden Paulus en Silas, en zongen Gode lofzangen en de gevangenen hoorden naar hen.
\par 26 En er geschiedde snellijk een grote aardbeving, alzo dat de fundamenten des kerkers bewogen werden; en terstond werden al de deuren geopend, en de banden van allen werden los.
\par 27 En de stokbewaarder, wakker geworden zijnde, en ziende de deuren der gevangenis geopend, trok een zwaard, en zou zichzelven omgebracht hebben, menende, dat de gevangenen ontvloden waren.
\par 28 Maar Paulus riep met grote stem, zeggende: Doe uzelven geen kwaad; want wij zijn allen hier.
\par 29 En als hij licht geeist had, sprong hij in, en werd zeer bevende, en viel voor Paulus en Silas neder aan de voeten;
\par 30 En hen buiten gebracht hebbende, zeide hij: Lieve heren, wat moet ik doen, opdat ik zalig worde?
\par 31 En zij zeiden: Geloof in den Heere Jezus Christus, en gij zult zalig worden, gij en uw huis.
\par 32 En zij spraken tot hem het woord des Heeren, en tot allen, die in zijn huis waren.
\par 33 En hij nam hen tot zich in dezelve ure des nachts, en wies hen van de striemen; en hij werd terstond gedoopt, en al de zijnen.
\par 34 En hij bracht hen in zijn huis, en zette hun de tafel voor, en verheugde zich, dat hij met al zijn huis aan God gelovig geworden was.
\par 35 En als het dag geworden was, zonden de hoofdmannen de stadsdienaars, zeggende: Laat die mensen los.
\par 36 En de stokbewaarder boodschapte deze woorden aan Paulus, zeggende: De hoofdmannen hebben gezonden, dat gij zoudt losgelaten worden; gaat dan nu uit, en reist heen in vrede.
\par 37 Maar Paulus zeide tot hen: Zij hebben ons, die Romeinen zijn, onveroordeeld in het openbaar gegeseld, en in de gevangenis geworpen, en werpen zij ons nu heimelijk daaruit? Niet alzo; maar dat zij zelven komen, en ons uitleiden.
\par 38 En de stadsdienaars boodschapten deze woorden wederom den hoofdmannen; en zij werden bevreesd, horende, dat zij Romeinen waren.
\par 39 En zij, komende, baden hen, en als zij hen uitgeleid hadden, begeerden zij, dat zij uit de stad gaan zouden.
\par 40 En uitgegaan zijnde uit de gevangenis, gingen zij in tot Lydia; en de broeders gezien hebbende, vertroostten zij dezelve, en gingen uit de stad.

\chapter{17}

\par 1 En door Amfipolis en Apollonia hun weg genomen hebbende, kwamen zij te Thessalonica, alwaar een synagoge der Joden was.
\par 2 En Paulus, gelijk hij gewoon was, ging tot hen in, en drie sabbatten lang handelde hij met hen uit de Schriften,
\par 3 Dezelve openende, en voor ogen stellende, dat de Christus moest lijden en opstaan uit de doden, en dat deze Jezus is de Christus, Dien ik, zeide hij, ulieden verkondige.
\par 4 En sommigen uit hen geloofden, en werden Paulus en Silas toegevoegd, en van de godsdienstige Grieken een grote menigte, en van de voornaamste vrouwen niet weinige.
\par 5 Maar de Joden, die ongehoorzaam waren, dit benijdende, namen tot zich enige boze mannen uit de marktboeven, en maakten, dat het volk te hoop liep, en beroerden de stad; en op het huis van Jason aanvallende, zochten zij hen tot het volk te brengen.
\par 6 En als zij hen niet vonden, trokken zij Jason en enige broeders voor de oversten der stad, roepende: Dezen, die de wereld in roer hebben gesteld, zijn ook hier gekomen;
\par 7 Welke Jason in zijn huis genomen heeft; en alle dezen doen tegen de geboden des keizers, zeggende, dat er een andere Koning is, namelijk een Jezus.
\par 8 En zij beroerden de schare, en de oversten der stad, die dit hoorden.
\par 9 Doch als zij van Jason en de anderen vergenoeging ontvangen hadden, lieten zij hen gaan.
\par 10 En de broeders zonden terstond des nachts Paulus en Silas weg naar Berea; welke, daar gekomen zijnde, gingen heen naar de synagoge der Joden;
\par 11 En dezen waren edeler, dan die te Thessalonica waren, als die het woord ontvingen met alle toegenegenheid, onderzoekende dagelijks de Schriften, of deze dingen alzo waren.
\par 12 Velen dan uit hen geloofden, en van de Griekse eerlijke vrouwen en van de mannen niet weinige.
\par 13 Maar als de Joden van Thessalonica verstonden, dat het Woord Gods ook te Berea van Paulus verkondigd werd, kwamen zij ook daar en bewogen de scharen.
\par 14 Doch de broeders zonden toen van stonde aan Paulus weg, dat hij ging als naar de zee; maar Silas en Timotheus bleven aldaar.
\par 15 En die Paulus geleidden, brachten hem tot Athene toe; en als zij bevel gekregen hadden aan Silas en Timotheus, dat zij op het spoedigste tot hem zouden komen, vertrokken zij.
\par 16 En terwijl Paulus hen te Athene verwachtte, werd zijn geest in hem ontstoken, ziende, dat de stad zo zeer afgodisch was.
\par 17 Hij handelde dan in de synagoge met de Joden, en met degenen, die godsdienstig waren, en op de markt alle dagen met degenen, die hem voorkwamen.
\par 18 En sommigen van de Epikureische en Stoische wijsgeren streden met hem; en sommigen zeiden: Wat wil toch deze klapper zeggen? Maar anderen zeiden: Hij schijnt een verkondiger te zijn van vreemde goden; omdat hij hun Jezus en de opstanding verkondigde.
\par 19 En zij namen hem, en brachten hem op de plaats, genaamd Areopagus, zeggende: Kunnen wij niet weten, welke deze nieuwe leer zij, daar gij van spreekt?
\par 20 Want gij brengt enige vreemde dingen voor onze oren; wij willen dan weten, wat toch dit zijn wil.
\par 21 (Die van Athene nu allen, en de vreemdelingen, die zich daar onthielden, besteedden hun tijd tot niets anders dan om wat nieuws te zeggen en te horen.)
\par 22 En Paulus, staande in het midden van de plaats, genaamd Areopagus, zeide: Gij mannen van Athene! ik bemerke, dat gij alleszins gelijk als godsdienstiger zijt.
\par 23 Want de stad doorgaande, en aanschouwende uw heiligdommen, heb ik ook een altaar gevonden, op hetwelk een opschrift stond: DEN ONBEKENDEN GOD. Dezen dan, Dien gij niet kennende dient, verkondig ik ulieden.
\par 24 De God, Die de wereld gemaakt heeft en alles wat daarin is; Deze, zijnde een Heere des hemels en der aarde, woont niet in tempelen met handen gemaakt;
\par 25 En wordt ook van mensenhanden niet gediend, als iets behoevende, alzo Hij Zelf allen het leven, en den adem, en alle dingen geeft;
\par 26 En heeft uit een bloede het ganse geslacht der mensen gemaakt, om op den gehelen aardbodem te wonen, bescheiden hebbende de tijden te voren geordineerd, en de bepalingen van hun woning;
\par 27 Opdat zij den Heere zouden zoeken, of zij Hem immers tasten en vinden mochten; hoewel Hij niet verre is van een iegelijk van ons.
\par 28 Want in Hem leven wij, en bewegen ons, en zijn wij; gelijk ook enigen van uw poeten gezegd hebben: Want wij zijn ook Zijn geslacht.
\par 29 Wij dan, zijnde Gods geslacht, moeten niet menen, dat de Godheid goud, of zilver, of steen gelijk zij, welke door mensenkunst en bedenking gesneden zijn.
\par 30 God dan, de tijden der onwetendheid overzien hebbende, verkondigt nu allen mensen alom, dat zij zich bekeren.
\par 31 Daarom dat Hij een dag gesteld heeft, op welken Hij den aardbodem rechtvaardiglijk zal oordelen, door een Man, Dien Hij daartoe geordineerd heeft, verzekering daarvan doende aan allen, dewijl Hij Hem uit de doden opgewekt heeft.
\par 32 Als zij nu van de opstanding der doden hoorden, spotten sommigen daarmede; en sommigen zeiden: Wij zullen u wederom hiervan horen.
\par 33 En alzo is Paulus uit het midden van hen uitgegaan.
\par 34 Doch sommige mannen hingen hem aan, en geloofden; onder welke was ook Dionysius, de Areopagiet, en een vrouw, met name Damaris, en anderen met dezelve.

\chapter{18}

\par 1 En na dezen scheidde Paulus van Athene en kwam te Korinthe;
\par 2 En vond een zekeren Jood, met name Aquila, van geboorte uit Pontus, die onlangs van Italie gekomen was, en Priscilla, zijn vrouw, (omdat Claudius bevolen had, dat al de Joden uit Rome vertrekken zouden), en hij ging tot hen;
\par 3 En omdat hij van hetzelfde handwerk was, bleef hij bij hen, en wrocht; want zij waren tentenmakers van handwerk.
\par 4 En hij handelde op elken sabbat in de synagoge, en bewoog tot het geloof Joden en Grieken.
\par 5 En als Silas en Timotheus van Macedonie afgekomen waren, werd Paulus door den Geest gedrongen, betuigende den Joden, dat Jezus is de Christus.
\par 6 Maar als zij wederstonden en lasterden, schudde hij zijn klederen af, en zeide tot hen: Uw bloed zij op uw hoofd; ik ben rein; en van nu voortaan zal ik tot de heidenen heengaan.
\par 7 En vandaar gegaan zijnde, kwam hij in het huis van een man, met name Justus, die God diende, wiens huis paalde aan de synagoge.
\par 8 En Crispus, de overste der synagoge, geloofde aan den Heere met geheel zijn huis; en velen van de Korinthiers, hem horende, geloofden, en werden gedoopt.
\par 9 En de Heere zeide tot Paulus door een gezicht in den nacht: Zijt niet bevreesd, maar spreek en zwijg niet.
\par 10 Want Ik ben met u, en niemand zal de hand aan u leggen om u kwaad te doen; want Ik heb veel volks in deze stad.
\par 11 En hij onthield zich aldaar een jaar en zes maanden, lerende onder hen het Woord Gods.
\par 12 Maar als Gallio stadhouder van Achaje was, stonden de Joden eendrachtelijk tegen Paulus op, en brachten hem voor den rechterstoel.
\par 13 Zeggende: Deze raadt den mensen aan, dat zij God zouden dienen tegen de wet.
\par 14 En als Paulus zijn mond zou opendoen, zeide Gallio tot de Joden: Zo er enig ongelijk, of kwaad stuk begaan ware, o Joden, zo zou ik met reden ulieden verdragen;
\par 15 Maar indien er geschil is over een woord, en namen, en over de wet, die onder u is, zo zult gij zelven toezien; want ik wil over deze dingen geen rechter zijn.
\par 16 En hij dreef hen weg van den rechterstoel.
\par 17 Maar al de Grieken namen Sosthenes, den overste der synagoge, en sloegen hem voor den rechterstoel; en Gallio trok zich geen van deze dingen aan.
\par 18 En als Paulus er nog vele dagen gebleven was, nam hij afscheid van de broederen, en scheepte van daar naar Syrie; en Priscilla en Aquila met hem, zijn hoofd te Kenchreen geschoren hebbende; want hij had een gelofte gedaan.
\par 19 En hij kwam te Efeze aan, en liet hen aldaar; maar hij ging in de synagoge, en handelde met de Joden.
\par 20 En als zij baden, dat hij langer bij hen blijven zoude, bewilligde hij het niet.
\par 21 Maar hij nam afscheid van hen, zeggende: Ik moet ganselijk het toekomende feest te Jeruzalem houden; doch ik zal tot u wederkeren, zo God wil. En hij voer weg van Efeze.
\par 22 En als hij te Cesarea was gekomen, ging hij op naar Jeruzalem, en de Gemeente gegroet hebbende, ging hij af naar Antiochie.
\par 23 En als hij aldaar enigen tijd geweest was, ging hij weg, en doorreisde vervolgens het land van Galatie en Frygie, versterkende al de discipelen.
\par 24 En een zeker Jood, met name Apollos, van geboorte een Alexandrier, een welsprekend man, kwam te Efeze, machtig zijnde in de Schriften.
\par 25 Deze was in den weg des Heeren onderwezen; en vurig zijnde van geest, sprak hij en leerde naarstiglijk de zaken des Heeren, wetende alleenlijk den doop van Johannes.
\par 26 En deze begon vrijmoediglijk te spreken in de synagoge. En als hem Aquila en Priscilla gehoord hadden, namen zij hem tot zich, en legden hem den weg Gods bescheidenlijker uit.
\par 27 En als hij wilde naar Achaje reizen, de broeders, hem vermaand hebbende, schreven aan de discipelen, dat zij hem ontvangen zouden; welke, daar gekomen zijnde, heeft veel toegebracht aan degenen, die geloofden door de genade.
\par 28 Want hij overtuigde de Joden met groten ernst in het openbaar, bewijzende door de Schriften, dat Jezus de Christus was.

\chapter{19}

\par 1 En het geschiedde, terwijl Apollos te Korinthe was, dat Paulus, de bovenste delen des lands doorreisd hebbende, te Efeze kwam; en enige discipelen aldaar vindende,
\par 2 Zeide hij tot hen: Hebt gij den Heiligen Geest ontvangen, als gij geloofd hebt? En zij zeiden tot hem: Wij hebben zelfs niet gehoord, of er een Heiligen Geest is.
\par 3 En hij zeide tot hen: Waarin zijt gij dan gedoopt? En zij zeiden: In den doop van Johannes.
\par 4 Maar Paulus zeide: Johannes heeft wel gedoopt den doop der bekering, zeggende tot het volk, dat zij geloven zouden in Dengene, Die na hem kwam, dat is, in Christus Jezus.
\par 5 En die hem hoorden werden gedoopt in den Naam van den Heere Jezus.
\par 6 En als Paulus hun de handen opgelegd had, kwam de Heilige Geest op hen; en zij spraken met vreemde talen, en profeteerden.
\par 7 En alle dezen waren omtrent twaalf mannen.
\par 8 En hij ging in de synagoge, en sprak vrijmoediglijk, drie maanden lang met hen handelende, en hun aanradende de zaken van het Koninkrijk Gods.
\par 9 Maar als sommigen verhard werden, en ongehoorzaam waren, kwaadsprekende van den weg des Heeren voor de menigte, week hij van hen, en scheidde de discipelen af, dagelijks handelende in de school van zekeren Tyrannus.
\par 10 En dit geschiedde twee jaren lang, alzo dat allen, die in Azie woonden, het Woord van den Heere Jezus hoorden, beiden Joden en Grieken.
\par 11 En God deed ongewone krachten door de handen van Paulus;
\par 12 Alzo dat ook van zijn lijf op de kranken gedragen werden de zweetdoeken of gordeldoeken, en dat de ziekten van hen weken, en de boze geesten van hen uitvoeren.
\par 13 En sommigen van de omzwervende Joden, zijnde duivel bezweerders, hebben zich onderwonden den Naam van den Heere Jezus te noemen over degenen, die boze geesten hadden, zeggende: Wij bezweren u bij Jezus, Dien Paulus predikt!
\par 14 Dezen nu waren zekere zeven zonen van Sceva, een Joodsen overpriester, die dit deden.
\par 15 Maar de boze geest, antwoordende, zeide: Jezus ken ik, en Paulus weet ik; maar gijlieden, wie zijt gij?
\par 16 En de mens, in welken de boze geest was, sprong op hen, en hen meester geworden zijnde, kreeg de overhand tegen hen, alzo dat zij naakt en gewond uit dat huis ontvloden.
\par 17 En dit werd allen bekend, beiden Joden en Grieken, die te Efeze woonden; en er viel een vreze over hen allen, en de Naam van den Heere Jezus werd groot gemaakt.
\par 18 En velen dergenen, die geloofden, kwamen, belijdende en verkondigende hun daden.
\par 19 Velen ook dergenen, die ijdele kunsten gepleegd hadden, brachten de boeken bijeen, en verbrandden ze in aller tegenwoordigheid; en berekenden de waarde derzelve, en bevonden vijftig duizend zilveren penningen.
\par 20 Alzo wies het Woord des Heeren met macht, en nam de overhand.
\par 21 En als deze dingen volbracht waren, nam Paulus voor in den Geest, Macedonie en Achaje doorgegaan hebbende, naar Jeruzalem te reizen, zeggende: Nadat ik aldaar zal geweest zijn, moet ik ook Rome zien.
\par 22 En als hij naar Macedonie gezonden had twee van degenen, die hem dienden, namelijk Timotheus en Erastus, bleef hij zelf een tijd lang in Azie.
\par 23 Maar op dienzelfden tijd ontstond er geen kleine beroerte, vanwege den weg des Heeren.
\par 24 Want een, met name Demetrius, een zilversmid, die kleine zilveren tempelen van Diana maakte, bracht dien van die kunst geen klein gewin toe;
\par 25 Welke hij samenvergaderd hebbende, met de handwerkers van dergelijke dingen, zeide: Mannen, gij weet, dat wij uit dit gewin onze welvaart hebben;
\par 26 En gij ziet en hoort, dat deze Paulus veel volk, niet alleen van Efeze, maar ook bijna van geheel Azie, overreed en afgekeerd heeft, zeggende, dat het geen goden zijn, die met handen gemaakt worden.
\par 27 En wij zijn niet alleen in gevaar, dat dit deel in verachting kome, maar dat ook de tempel van de grote godin Diana als niets geacht zal worden, en dat ook haar majesteit zal ten ondergaan, aan welke gans Azie en de gehele wereld godsdienst bewijst.
\par 28 Als zij nu dit hoorden, werden zij vol van toornigheid, en riepen, zeggende: Groot is de Diana de Efezeren!
\par 29 En de gehele stad werd vol verwarring; en zij liepen met een gedruis eendrachtelijk naar de schouwplaats, met zich trekkende Gajus en Aristarchus, Macedoniers, metgezellen van Paulus op de reis.
\par 30 En als Paulus tot het volk wilde ingaan, lieten het hem de discipelen niet toe.
\par 31 En sommigen ook der oversten van Azie, die hem vrienden waren, zonden tot hem, en baden, dat hij zichzelven op de schouwplaats niet zou begeven.
\par 32 Zij riepen dan de ene dit, de andere wat anders; want de vergadering was verward en het meerder deel wist niet, om wat oorzaak zij samengekomen waren.
\par 33 En zij deden Alexander uit de schare voortkomen, alzo hem de Joden voortstieten. En Alexander gewenkt hebbende met de hand, wilde bij het volk verantwoording doen.
\par 34 Maar als zij verstonden, dat hij een Jood was, werd er een stem van allen, roepende omtrent twee uren lang: Groot is de Diana der Efezeren!
\par 35 En als de stads schrijver de schare gestild had, zeide hij: Gij mannen van Efeze! wat mens is er toch, die niet weet, dat de stad der Efezeren de kerkbewaarster zij van de grote godin Diana, en van het beeld, dat uit den hemel gevallen is?
\par 36 Dewijl dan deze dingen onwedersprekelijk zijn, zo is het behoorlijk dat gij stil zijt, en niets onbedachts doet.
\par 37 Want gij hebt deze mannen hier gebracht, die noch kerkrovers zijn, noch uw godin lasteren.
\par 38 Indien dan nu Demetrius, en die met hem van de kunst zijn, tegen iemand enige zaak hebben, de rechtsdagen worden gehouden, en er zijn stadhouders; laat hen elkander verklagen.
\par 39 En indien gij iets van andere dingen verzoekt, dat zal in een wettelijke vergadering beslecht worden.
\par 40 Want wij staan in gevaar, dat wij van oproer zullen verklaagd worden om den dag van heden, alzo er geen oorzaak is, waardoor wij reden zullen kunnen geven van dezen oploop.
\par 41 En dit gezegd hebbende, liet hij de vergadering gaan.

\chapter{20}

\par 1 Nadat nu het oproer gestild was, Paulus, de discipelen tot zich geroepen en gegroet hebbende, ging uit om naar Macedonie te reizen.
\par 2 En als hij die delen doorgereisd, en hen met vele redenen vermaand had, kwam hij in Griekenland.
\par 3 En als hij aldaar drie maanden overgebracht had, en hem van de Joden lagen gelegd werden, als hij naar Syrie zoude varen, zo werd hij van zin weder te keren door Macedonie.
\par 4 En hem vergezelschapte tot in Azie Sopater van Berea; en van de Thessalonicensen Aristarchus en Sekundus; en Gajus van Derbe, en Timotheus en van die van Azie Tychikus en Trofimus.
\par 5 Dezen, vooraf heengegaan zijnde, wachtten ons te Troas.
\par 6 Wij nu scheepten af van Filippi na de dagen der ongehevelde broden, en kwamen in vijf dagen bij hen te Troas, alwaar wij ons zeven dagen onthielden.
\par 7 En op den eersten dag der week, als de discipelen bijeengekomen waren om brood te breken, handelde Paulus met hen, zullende des anderen daags verreizen; en hij strekte zijne rede uit tot den middernacht.
\par 8 En er waren vele lichten in de opperzaal waar zij vergaderd waren.
\par 9 En een zeker jongeling, met name Eutychus, zat in het venster en met een diepen slaap overvallen zijnde, alzo Paulus lang tot hen sprak, door den slaap nederstortende, viel van de derde zoldering nederwaarts, en werd dood opgenomen.
\par 10 Doch Paulus, afgekomen zijnde, viel op hem, en hem omvangende, zeide hij: Weest niet beroerd; want zijn ziel is in hem.
\par 11 En als hij weder boven gegaan was, en brood gebroken en wat gegeten had, en lang, tot den dageraad toe, met hen gesproken had, vertrok hij alzo.
\par 12 En zij brachten den knecht levende, en waren bovenmate vertroost.
\par 13 Maar wij, vooruit naar het schip gegaan zijnde, voeren af naar Assus, waar wij Paulus zouden innemen; want hij had het alzo bevolen, en hij zelf zou te voet gaan.
\par 14 En als hij zich te Assus bij ons gevoegd had, namen wij hem in, en kwamen te Mitylene.
\par 15 En van daar afgescheept zijnde, kwamen wij den volgenden dag tegen Chios over, en des anderen daags legden wij aan te Samos, en bleven te Trogyllion, en den dag daaraan kwamen wij te Milete.
\par 16 Want Paulus had voorgenomen Efeze voorbij te varen, opdat hij niet den tijd in Azie zou verslijten; want hij spoedde zich, om (zo het hem mogelijk ware) op den pinksterdag te Jeruzalem te zijn.
\par 17 Maar hij zond van Milete naar Efeze, en hij ontbood de ouderlingen der Gemeente.
\par 18 En als zij tot hem gekomen waren, zeide hij tot hen: Gijlieden weet, van den eersten dag af, dat ik in Azie ben aangekomen, hoe ik bij u den gansen tijd geweest ben;
\par 19 Dienende den Heere met alle ootmoedigheid, en vele tranen, en verzoekingen, die mij overkomen zijn door de lagen der Joden;
\par 20 Hoe ik niets achtergehouden heb van hetgeen nuttig was, dat ik u niet zou verkondigd en u geleerd hebben, in het openbaar en bij de huizen;
\par 21 Betuigende, beiden Joden en Grieken, de bekering tot God en het geloof in onzen Heere Jezus Christus.
\par 22 En nu ziet, ik, gebonden zijnde door den Geest, reis naar Jeruzalem, niet wetende, wat mij daar ontmoeten zal;
\par 23 Dan dat de Heilige Geest van stad tot stad betuigt, zeggende, dat mij banden en verdrukkingen aanstaande zijn.
\par 24 Maar ik acht op geen ding, noch houde mijn leven dierbaar voor mijzelven, opdat ik mijn loop met blijdschap mag volbrengen, en den dienst, welken ik, van den Heere Jezus ontvangen heb, om te betuigen het Evangelie der genade Gods.
\par 25 En nu ziet, ik weet, dat gij allen, waar ik doorgegaan ben, predikende het Koninkrijk Gods, mijn aangezicht niet meer zien zult.
\par 26 Daarom betuig ik ulieden op dezen huidigen dag, dat ik rein ben van het bloed van u allen.
\par 27 Want ik heb niet achtergehouden, dat ik u niet zou verkondigd hebben al den raad Gods.
\par 28 Zo hebt dan acht op uzelven, en op de gehele kudde, over dewelke u de Heilige Geest tot opzieners gesteld heeft, om de Gemeente Gods te weiden, welke Hij verkregen heeft door Zijn eigen bloed.
\par 29 Want dit weet ik, dat na mijn vertrek zware wolven tot u inkomen zullen, die de kudde niet sparen.
\par 30 En uit uzelven zullen mannen opstaan, sprekende verkeerde dingen, om de discipelen af te trekken achter zich.
\par 31 Daarom waakt, en gedenkt, dat ik drie jaren lang nacht en dag, niet opgehouden heb een iegelijk met tranen te vermanen.
\par 32 En nu, broeders, ik bevele u Gode, en den woorde Zijner genade, Die machtig is u op te bouwen, en u een erfdeel te geven onder al de geheiligden.
\par 33 Ik heb niemands zilver, of goud, of kleding begeerd.
\par 34 En gijzelve weet, dat deze handen tot mijn nooddruft, en dergenen, die met mij waren, gediend hebben.
\par 35 Ik heb u in alles getoond, dat men, alzo arbeidende, de zwakken moet opnemen, en gedenken aan de woorden van den Heere Jezus, dat Hij gezegd heeft: Het is zaliger te geven, dan te ontvangen.
\par 36 En als hij dit gezegd had, heeft hij nederknielende met hen allen gebeden.
\par 37 En er werd een groot geween van hen allen; en zij, vallende om den hals van Paulus, kusten hem;
\par 38 Zeer bedroefd zijnde, allermeest over het woord, dat hij gezegd had, dat zij zijn aangezicht niet meer zien zouden; en zij geleidden hem naar het schip.

\chapter{21}

\par 1 En als het geschiedde, dat wij van hen gescheiden en afgevaren waren, zo liepen wij rechtuit en kwamen te Kos, en den dag daaraan te Rhodus, en van daar te Patara.
\par 2 En een schip gevonden hebbende, dat naar Fenicie overvoer, gingen wij er in en voeren af.
\par 3 En als wij Cyprus in het gezicht gekregen, en dat aan de linker hand gelaten hadden, voeren wij naar Syrie, en kwamen aan te Tyrus; want het schip zoude aldaar den last ontladen.
\par 4 En de discipelen gevonden hebbende, bleven wij daar zeven dagen; dewelke tot Paulus zeiden door den Geest, dat hij niet zou opgaan naar Jeruzalem.
\par 5 Toen het nu geschiedde, dat wij deze dagen doorgebracht hadden, gingen wij uit, en reisden voort; en zij geleidden ons allen met vrouwen en kinderen tot buiten de stad; en aan den oever nederknielende, hebben wij gebeden.
\par 6 En als wij elkander gegroet hadden, gingen wij in het schip; maar zijlieden keerden wederom, elk naar het zijne.
\par 7 Wij nu, de scheepvaart volbracht hebbende van Tyrus, kwamen aan te Ptolemais, en de broeders gegroet hebbende, bleven een dag bij hen.
\par 8 En des anderen daags, Paulus en wij, die met hem waren, gingen van daar en kwamen te Cesarea; en gegaan zijnde in het huis van Filippus, den evangelist (die een was van de zeven), bleven wij bij hem.
\par 9 Deze nu had vier dochters, nog maagden, die profeteerden.
\par 10 En als wij daar vele dagen gebleven waren, kwam er een zeker profeet af van Judea, met name Agabus;
\par 11 En hij kwam tot ons, en nam den gordel van Paulus, en zichzelven handen en voeten gebonden hebbende, zeide: Dit zegt de Heilige Geest: Den man, wiens deze gordel is, zullen de Joden alzo te Jeruzalem binden, en overleveren in de handen der heidenen.
\par 12 Als wij nu dit hoorden, baden beiden wij en die van die plaats waren, dat hij niet zou opgaan naar Jeruzalem.
\par 13 Maar Paulus antwoordde: Wat doet gij, dat gij weent, en mijn hart week maakt? Want ik ben bereid niet alleen gebonden te worden, maar ook te sterven te Jeruzalem voor den Naam van den Heere Jezus.
\par 14 En als hij zich niet liet afraden, hielden wij ons tevreden, zeggende: De wil des Heeren geschiede.
\par 15 En na die dagen maakten wij ons gereed, en gingen op naar Jeruzalem.
\par 16 En met ons gingen ook sommigen der discipelen van Cesarea, leidende met zich een zekeren Mnason, van Cyprus, een ouden discipel, bij dewelken wij zouden te huis liggen.
\par 17 En als wij te Jeruzalem gekomen waren, ontvingen ons de broeders blijdelijk.
\par 18 En den volgenden dag ging Paulus met ons in tot Jakobus; en al de ouderlingen waren daar gekomen.
\par 19 En als hij hen gegroet had, verhaalde hij van stuk tot stuk, wat God onder de heidenen door zijn dienst gedaan had.
\par 20 En zij, dat gehoord hebbende, loofden den Heere, en zeiden tot hem: Gij ziet, broeder, hoevele duizenden van Joden er zijn, die geloven; en zij zijn allen ijveraars van de wet.
\par 21 En zij zijn aangaande u bericht, dat gij al de Joden, die onder de heidenen zijn, leert van Mozes afvallen, zeggende: dat zij de kinderen niet zouden besnijden, noch naar de wijze der wet wandelen.
\par 22 Wat is er dan te doen? Het is gans nodig, dat de menigte samenkome; want zij zullen horen, dat gij gekomen zijt.
\par 23 Doe dan hetgeen wij u zeggen: Wij hebben vier mannen, die een gelofte gedaan hebben.
\par 24 Neem dezen tot u, en heilig u met hen, en doe de onkosten nevens hen, opdat zij het hoofd bescheren mogen; en alle mogen weten, dat er niets is aan hetgeen, waarvan zij, aangaande u, bericht zijn; maar dat gij alzo wandelt, dat gij ook zelve de wet onderhoudt.
\par 25 Doch van de heidenen, die geloven, hebben wij geschreven en goed gevonden, dat zij niets dergelijks zouden onderhouden, dan dat zij zich wachten van hetgeen den afgoden geofferd is, en van bloed, en van het verstikte, en van hoererij.
\par 26 Toen nam Paulus de mannen met zich, en den dag daaraan met hen geheiligd zijnde, ging hij in den tempel, en verkondigde, dat de dagen der heiliging vervuld waren, blijvende daar, totdat voor een iegelijk van hen de offerande opgeofferd was.
\par 27 Als nu de zeven dagen zouden voleindigd worden, zagen hem de Joden van Azie in den tempel, en beroerden al het volk, en sloegen de handen aan hem,
\par 28 Roepende: Gij Israelietische mannen, komt te hulp! Deze is de mens, die tegen het volk, en de wet, en deze plaats allen man overal leert; en bovendien heeft hij ook Grieken in den tempel gebracht, en heeft deze heilige plaats ontheiligd.
\par 29 Want zij hadden te voren Trofimus, den Efezier, met hem in de stad gezien, welken zij meenden, dat Paulus in den tempel gebracht had.
\par 30 En de gehele stad kwam in roer en het volk liep samen; en zij grepen Paulus, en trokken hem buiten den tempel; en terstond werden de deuren gesloten.
\par 31 En als zij hem zochten te doden, kwam het gerucht tot den overste der bende, dat geheel Jeruzalem in verwarring was.
\par 32 Welke terstond krijgsknechten en hoofdmannen over honderd tot zich nam, en liep af naar hen toe. Zij nu, den oversten en de krijgsknechten ziende, hielden op van Paulus te slaan.
\par 33 Toen naderde de overste en greep hem, en beval, dat men hem met twee ketenen zou binden; en vraagde, wie hij was, en wat hij gedaan had.
\par 34 En onder de schare riep de ene dit, de andere wat anders. Doch als hij de zekerheid niet kon weten vanwege de beroerte, beval hij, dat men hem in de legerplaats zou brengen.
\par 35 En als hij aan de trappen gekomen was, gebeurde het, dat hij van de krijgsknechten gedragen werd vanwege het geweld der schare.
\par 36 Want de menigte des volks volgde, al roepende: Weg met hem!
\par 37 En als Paulus nu in de legerplaats zou geleid worden, zeide hij tot den overste: Is het mij geoorloofd tot u wat te spreken? En hij zeide: Kent gij Grieks?
\par 38 Zijt gij dan niet de Egyptenaar, die voor deze dagen oproer verwekte, en de vier duizend moordenaars naar de woestijn uitleidde?
\par 39 Maar Paulus zeide: Ik ben een Joods man van Tarsen, een burger van geen onvermaarde stad in Cilicie, en ik bid u, laat mij toe tot het volk te spreken.
\par 40 En als hij het toegelaten had, Paulus, staande op de trappen, wenkte met de hand tot het volk; en als er grote stilte geworden was, sprak hij hen aan in de Hebreeuwse taal, zeggende:

\chapter{22}

\par 1 Mannen broeders en vaders, hoort mijn verantwoording, die ik tegenwoordig tot u doen zal.
\par 2 (Als zij nu hoorden, dat hij in de Hebreeuwse taal hen aansprak, hielden zij zich te meer stil. En hij zeide:)
\par 3 Ik ben een Joods man, en te Tarsen in Cilicie geboren, opgevoed in deze stad, aan de voeten van Gamaliel onderwezen naar de bescheidenste wijze der vaderlijke wet, zijnde een ijveraar Gods, gelijkerwijs gij allen heden zijt;
\par 4 Die dezen weg vervolgd heb tot den dood, bindende en in de gevangenissen overleverende beiden mannen en vrouwen.
\par 5 Gelijk mij ook de hogepriester getuige is, en de gehele raad der ouderlingen; van dewelke ik ook brieven genomen hebbende tot de broeders, ben naar Damaskus gereisd, om ook degenen, die daar waren, gebonden te brengen naar Jeruzalem, opdat zij gestraft zouden worden.
\par 6 Maar het geschiedde mij, als ik reisde, en Damaskus genaakte, omtrent den middag, dat snellijk uit den hemel een groot licht mij rondom omscheen.
\par 7 En ik viel ter aarde, en ik hoorde een stem, tot mij zeggende: Saul, Saul, wat vervolgt gij Mij?
\par 8 En ik antwoordde: Wie zijt Gij, Heere? En Hij zeide tot mij: Ik ben Jezus, de Nazarener, Welken gij vervolgt.
\par 9 En die met mij waren, zagen wel het licht, en werden zeer bevreesd; maar de stem Desgenen, Die tot mij sprak, hoorden zij niet.
\par 10 En ik zeide: Heere! wat zal ik doen? En de Heere zeide tot mij: Sta op, en ga heen naar Damaskus; en aldaar zal met u gesproken worden, van al hetgeen u geordineerd is te doen.
\par 11 En als ik vanwege de heerlijkheid deszelven lichts niet zag, zo werd ik bij de hand geleid van degenen, die met mij waren, en kwam te Damaskus.
\par 12 En een zekere Ananias, een godvruchtig man naar de wet, goede getuigenis hebbende van al de Joden, die daar woonden,
\par 13 Kwam tot mij, en bij mij staande, zeide tot mij: Saul, broeder, word weder ziende! En ter zelfder ure werd ik ziende op hem.
\par 14 En hij zeide: De God onzer vaderen heeft u te voren verordineerd, om Zijn wil te kennen, en den Rechtvaardige te zien, en de stem uit Zijn mond te horen.
\par 15 Want gij zult Hem getuige zijn bij alle mensen, van hetgeen gij gezien en gehoord hebt.
\par 16 En nu, wat vertoeft gij? Sta op, en laat u dopen, en uw zonden afwassen, aanroepende den Naam des Heeren.
\par 17 En het gebeurde mij, als ik te Jeruzalem wedergekeerd was, en in den tempel bad, dat ik in een vertrekking van zinnen was;
\par 18 En dat ik Hem zag, en Hij tot mij zeide: Spoed u, en ga in der haast uit Jeruzalem; want zij zullen uw getuigenis van Mij niet aannemen.
\par 19 En ik zeide: Heere, zij weten, dat ik in de gevangenis wierp, en in de synagogen geselde, die in U geloofden;
\par 20 En toen het bloed van Stefanus, Uw getuige, vergoten werd, dat ik daar ook bij stond, en mede een welbehagen had in zijn dood, en de klederen bewaarde dergenen, die hem doodden.
\par 21 En Hij zeide tot mij: Ga heen; want Ik zal u ver tot de heidenen afzenden.
\par 22 Zij hoorden hem nu tot dit woord toe; en zij verhieven hun stem, zeggende: Weg van de aarde met zulk een, want het is niet behoorlijk, dat hij leve.
\par 23 En als zij riepen, en de klederen van zich smeten, en stof in de lucht wierpen;
\par 24 Zo beval de overste, dat men hem in de legerplaats zou brengen, en zeide, dat men hem met geselen onderzoeken zou, opdat hij verstaan mocht, om wat oorzaak zij alzo over hem riepen.
\par 25 En alzo zij hem met de riemen uitrekten, zeide Paulus tot den hoofdman over honderd, die daar stond: Is het ulieden geoorloofd een Romeinsen mens, en dien onveroordeeld, te geselen?
\par 26 Als nu de hoofdman over honderd dat hoorde, ging hij toe, en boodschapte het den overste, zeggende: Zie, wat gij te doen hebt; want deze mens is een Romein.
\par 27 En de overste kwam toe, en zeide tot hem: Zeg mij, zijt gij een Romein? En hij zeide: Ja.
\par 28 En de overste antwoordde: Ik heb dit burgerrecht voor een grote som gelds verkregen. En Paulus zeide: Maar ik ben ook een burger geboren.
\par 29 Terstond dan lieten zij van hem af, die hem zouden onderzocht hebben. En de overste werd ook bevreesd, toen hij verstond, dat hij een Romein was, en dat hij hem had gebonden.
\par 30 En des anderen daags, willende de zekerheid weten, waarom hij van de Joden beschuldigd werd, maakte hij hem los van de banden, en beval, dat de overpriesters en hun gehele raad zouden komen; en Paulus afgebracht hebbende, stelde hij hem voor hen.

\chapter{23}

\par 1 En Paulus, de ogen op den raad houdende, zeide: Mannen broeders! ik heb met alle goed geweten voor God gewandeld tot op dezen dag.
\par 2 Maar de hogepriester Ananias beval dengenen, die bij hem stonden, dat zij hem op den mond zouden slaan.
\par 3 Toen zeide Paulus tot hem: God zal u slaan, gij gewitte wand! Zit gij ook om mij te oordelen naar de wet, en beveelt gij, tegen de wet, dat men mij zal slaan?
\par 4 En die daarbij stonden, zeiden: Scheldt gij den hogepriester Gods?
\par 5 En Paulus zeide: Ik wist niet, broeders! dat het de hogepriester was; want er is geschreven: Den overste uws volks zult gij niet vloeken.
\par 6 En Paulus wetende dat het ene deel was van de Sadduceen, en het andere van de Farizeen, riep in den raad: Mannen broeders, ik ben een Farizeer, eens Farizeers zoon; ik word over de hoop en opstanding der doden geoordeeld.
\par 7 En als hij dit gesproken had, ontstond er tweedracht tussen de Farizeen en de Sadduceen, en de menigte werd verdeeld.
\par 8 Want de Sadduceen zeggen, dat er geen opstanding is, noch engel, noch geest, maar de Farizeen belijden het beide.
\par 9 En er geschiedde een groot geroep; en de Schriftgeleerden van de zijde der Farizeen stonden op, en streden, zeggende: Wij vinden geen kwaad in dezen mens; en indien een geest tot hem gesproken heeft, of een engel, laat ons tegen God niet strijden.
\par 10 En als er grote tweedracht ontstaan was, de overste, vrezende, dat Paulus van hen verscheurd mocht worden, gebood, dat het krijgsvolk zou afkomen, en hem uit het midden van hen wegrukken, en in de legerplaats brengen.
\par 11 En den volgenden nacht stond de Heere bij hem, en zeide: Heb goeden moed, Paulus, want gelijk gij te Jeruzalem van Mij betuigd hebt alzo moet gij ook te Rome getuigen.
\par 12 En als het dag geworden was, maakten sommigen van de Joden een samenrotting, en vervloekten zichzelven, zeggende, dat zij noch eten noch drinken zouden, totdat zij Paulus zouden gedood hebben.
\par 13 En zij waren meer dan veertig, die dezen eed te zamen gedaan hadden;
\par 14 Dewelke gingen tot de overpriesters en de ouderlingen, en zeiden: Wij hebben ons zelven met vervloeking vervloekt, niets te zullen nuttigen, totdat wij Paulus zullen gedood hebben.
\par 15 Gij dan nu, laat den overste weten met den raad, dat hij hem morgen tot u afbrenge, alsof gij nadere kennis zoudt nemen van zijn zaken; en wij zijn bereid hem om te brengen, eer hij bij u komt.
\par 16 En als de zoon van Paulus' zuster deze lage gehoord had, kwam hij daar, en ging in de legerplaats, en boodschapte het Paulus.
\par 17 En Paulus riep tot zich een van de hoofdmannen over honderd, en zeide: Leid dezen jongeling heen tot den overste; want hij heeft hem wat te boodschappen.
\par 18 Deze dan nam hem en bracht hem tot den overste, en zeide: Paulus, de gevangene, heeft mij tot zich geroepen, en begeerd, dat ik dezen jongeling tot u zou brengen, die u wat heeft te zeggen.
\par 19 De overste nu nam hem bij de hand, en bezijden gegaan zijnde, vraagde hij: Wat is het dat gij mij hebt te boodschappen?
\par 20 En hij zeide: De Joden zijn overeengekomen, om van u te begeren, dat gij Paulus morgen in den raad zoudt afbrengen, alsof zij iets van hem nader zouden onderzoeken.
\par 21 Doch geloof hen niet; want meer dan veertig mannen uit hen leggen hem lagen, welke zichzelven met een vervloeking verbonden hebben noch te eten noch te drinken, totdat zij hem zullen omgebracht hebben; en zij zijn nu gereed, verwachtende de toezegging van u.
\par 22 De overste dan liet den jongeling gaan, hem gebiedende: Zeg niemand voort, dat gij mij zulks geopenbaard hebt.
\par 23 En zekere twee van de hoofdmannen over honderd tot zich geroepen hebbende, zeide hij: Maakt tweehonderd krijgsknechten gereed, opdat zij naar Cesarea trekken, en zeventig ruiters, en tweehonderd schutters, tegen de derde ure des nachts;
\par 24 En laat ze zadel beesten bestellen, opdat zij Paulus daarop zetten, en behouden overbrengen tot den stadhouder Felix.
\par 25 En hij schreef een brief, hebbende dezen inhoud:
\par 26 Claudius Lysias aan den machtigsten stadhouder Felix groetenis.
\par 27 Alzo deze man van de Joden gegrepen was, en van hen omgebracht zou geworden zijn, ben ik daarover gekomen met het krijgsvolk, en heb hem hun ontnomen, bericht zijnde, dat hij een Romein is.
\par 28 En willende de zaak weten, waarover zij hem beschuldigden, bracht ik hem af in hun raad;
\par 29 Welken ik bevond beschuldigd te worden over vragen hunner wet; maar geen beschuldiging tegen hem te zijn, die den dood of banden waardig is.
\par 30 En als mij te kennen gegeven was, dat van de Joden een lage tegen deze man gelegd zou worden, zo heb ik hem terstond aan u gezonden; gebiedende ook den beschuldigers voor u te zeggen, hetgeen zij tegen hem hadden. Vaarwel.
\par 31 De krijgsknechten dan, gelijk hun bevolen was, namen Paulus, en brachten hem des nachts tot Antipatris.
\par 32 En des anderen daags, latende de ruiters met hem trekken, keerden zij wederom naar de legerplaats.
\par 33 Dewelken als zij te Cesarea gekomen waren, en den brief den stadhouder overgeleverd hadden, hebben zij ook Paulus voor hem gesteld.
\par 34 En de stadhouder, den brief gelezen hebbende, vraagde, uit wat provincie hij was; en verstaande, dat hij van Cilicie was,
\par 35 Zeide hij: Ik zal u horen, als ook uw beschuldigers hier zullen gekomen zijn. En hij beval, dat hij in het rechthuis van Herodes zou bewaard worden.

\chapter{24}

\par 1 En vijf dagen daarna kwam de hogepriester Ananias af met de ouderlingen, en een zekeren voorspraak, genaamd Tertullus, dewelke verschenen voor den stadhouder tegen Paulus.
\par 2 En als hij geroepen was, begon Tertullus hem te beschuldigen, zeggende:
\par 3 Dat wij grote vrede door u bekomen, en dat vele loffelijke diensten dezen volke geschieden door uw voorzichtigheid, machtigste Felix, nemen wij ganselijk en overal met alle dankbaarheid aan.
\par 4 Maar opdat ik u niet lang ophoude, ik bid u, dat gij ons, naar uw bescheidenheid, kortelijk hoort.
\par 5 Want wij hebben dezen man bevonden te zijn een pest, en een, die oproer verwekt onder al de Joden, door de ganse wereld, en een oppersten voorstander van de sekte der Nazarenen.
\par 6 Die ook gepoogd heeft den tempel te ontheiligen, welken wij ook gegrepen hebben, en naar onze wet hebben willen oordelen.
\par 7 Maar Lysias, de overste, daarover komende, heeft hem met groot geweld uit onze handen weggebracht;
\par 8 Gebiedende zijn beschuldigers tot u te komen; van dewelken gij zelf, hem onderzocht hebbende, zult kunnen verstaan al hetgeen, waarvan wij hem beschuldigen.
\par 9 En ook de Joden stemden het toe, zeggende, dat deze dingen alzo waren.
\par 10 Maar Paulus, als hem de stadhouder gewenkt had, dat hij zou spreken, antwoordde: Dewijl ik weet, dat gij nu vele jaren over dit volk rechter zijt geweest, zo verantwoord ik mijzelven met des te beteren moed.
\par 11 Alzo gij kunt weten, dat het niet meer dan twaalf dagen zijn, van dat ik ben opgekomen om te aanbidden te Jeruzalem;
\par 12 En zij hebben mij noch in den tempel gevonden tot iemand sprekende, of enige samenrotting des volks makende, noch in de synagogen, noch in de stad;
\par 13 En zij kunnen niet bewijzen, waarvan zij mij nu beschuldigen.
\par 14 Maar dit beken ik u, dat ik naar dien weg, welken zij sekte noemen, den God der vaderen alzo diene, gelovende alles, wat in de wet en in de profeten geschreven is;
\par 15 Hebbende hoop op God, welke dezen ook zelf verwachten, dat er een opstanding der doden wezen zal, beiden der rechtvaardigen en der onrechtvaardigen.
\par 16 En hierin oefen ik mijzelven, om altijd een onergerlijk geweten te hebben bij God en de mensen.
\par 17 Doch na vele jaren ben ik gekomen om aalmoezen te doen aan mijn volk, en offeranden.
\par 18 Waarover mij gevonden hebben, geheiligd zijnde, in den tempel, niet met volk, noch met beroerte, enige Joden uit Azie;
\par 19 Welke behoorden hier voor u tegenwoordig te zijn, en mij te beschuldigen, indien zij iets hadden tegen mij.
\par 20 Of dat dezen zelf zeggen of zij enig onrecht in mij gevonden hebben, als ik voor den raad stond;
\par 21 Dan van dit enig woord, hetwelk ik riep, staande onder hen: Over de opstanding der doden word ik heden van ulieden geoordeeld!
\par 22 Toen nu Felix dit gehoord had, stelde hij hen uit, zeggende: Als ik nader wetenschap van dezen weg zal hebben, wanneer Lysias, de overste, zal afgekomen zijn, zo zal ik volle kennis nemen van uw zaken.
\par 23 En hij beval den hoofdman over honderd, dat Paulus zou bewaard worden, en verlichting hebben, en dat hij niemand van de zijnen zou beletten hem te dienen, of tot hem te komen.
\par 24 En na sommige dagen, Felix, daar gekomen zijnde met Drusilla, zijn vrouw, die een Jodin was, ontbood Paulus, en hoorde hem van het geloof in Christus.
\par 25 En als hij handelde van rechtvaardigheid, en matigheid, en van het toekomende oordeel, Felix, zeer bevreesd geworden zijnde, antwoordde: Voor ditmaal ga heen; en als ik gelegenen tijd zal hebben bekomen, zo zal ik u tot mij roepen.
\par 26 En tegelijk ook hopende, dat hem van Paulus geld gegeven zou worden, opdat hij hem losliet; waarom hij hem ook dikwijls ontbood, en sprak met hem.
\par 27 Maar als twee jaren vervuld waren, kreeg Felix Porcius Festus in zijn plaats; en Felix, willende den Joden gunst bewijzen, liet Paulus gevangen.

\chapter{25}

\par 1 Festus dan, in de provincie gekomen zijnde, ging na drie dagen van Cesarea op naar Jeruzalem.
\par 2 En de hogepriester, en de voornaamsten der Joden, verschenen voor hem tegen Paulus en baden hem,
\par 3 Begerende gunst tegen hem, opdat hij hem zou doen komen te Jeruzalem; en leggende een lage, om hem op den weg om te brengen.
\par 4 Doch Festus antwoordde, dat Paulus te Cesarea bewaard werd, en dat hij zelf haast derwaarts zou verreizen.
\par 5 Die dan, zeide hij, onder u kunnen, dat zij mede afreizen, en zo er iets onbehoorlijks in dezen man is, dat zij hem beschuldigen.
\par 6 En als hij onder hen niet meer dan tien dagen doorgebracht had, kwam hij af naar Cesarea; en des anderen daags, op den rechterstoel gezeten zijnde, beval hij, dat Paulus zou voor gebracht worden.
\par 7 En als hij daar gekomen was, stonden de Joden, die van Jeruzalem afgekomen waren, rondom hem, vele en zware beschuldigingen tegen Paulus voortbrengende, die zij niet konden bewijzen;
\par 8 Dewijl hij, verantwoordende, zeide: Ik heb noch tegen de wet der Joden, noch tegen den tempel, noch tegen den keizer iets gezondigd.
\par 9 Maar Festus, willende den Joden gunst bewijzen, antwoordde Paulus, en zeide: Wilt gij naar Jeruzalem opgaan, en aldaar voor mij over deze dingen geoordeeld worden?
\par 10 En Paulus zeide: Ik sta voor den rechterstoel des keizers, waar ik geoordeeld moet worden; den Joden heb ik geen onrecht gedaan; gelijk gij ook zeer wel weet.
\par 11 Want indien ik onrecht doe, en iets des doods waardig gedaan heb, ik weiger niet te sterven; maar indien er niets is van hetgeen, waarvan dezen mij beschuldigen, zo kan niemand mij hun uit gunst overgeven. Ik beroep mij op den keizer.
\par 12 Toen antwoordde Festus, als hij met den raad gesproken had: Hebt gij u op den keizer beroepen? Gij zult tot den keizer gaan.
\par 13 En als enige dagen voorbijgegaan waren, kwamen de koning Agrippa en Bernice te Cesarea, om Festus te begroeten.
\par 14 En toen zij aldaar vele dagen doorgebracht hadden, heeft Festus de zaken van Paulus aan den koning verhaald, zeggende: Hier is een zeker man van Felix gevangen gelaten;
\par 15 Om wiens wil, als ik te Jeruzalem was, de overpriesters en de ouderlingen der Joden verschenen, begerende vonnis tegen hem;
\par 16 Aan dewelke ik antwoordde, dat de Romeinen de gewoonte niet hebben, enigen mens uit gunst ter dood over te geven, eer de beschuldigde de beschuldigers tegenwoordig heeft, en plaats van verantwoording gekregen heeft over de beschuldiging.
\par 17 Als zij dan gezamenlijk alhier gekomen waren, zo heb ik, geen uitstel nemende, des daags daaraan op den rechterstoel gezeten, en beval, dat de man zoude voor gebracht worden;
\par 18 Over welken de beschuldigers, hier staande, geen zaak hebben voorgebracht, waarvan ik vermoedde;
\par 19 Maar hadden tegen hem enige vragen van hun godsdienst, en van zekeren Jezus, Die gestorven was, Welken Paulus zeide te leven.
\par 20 En als ik over de onderzoeking van deze zaak in twijfeling was, zeide ik, of hij wilde gaan naar Jeruzalem, en aldaar over deze dingen geoordeeld worden.
\par 21 En als Paulus zich beriep, dat men hem tot de kennis des keizers bewaren zou, zo heb ik bevolen, dat hij bewaard zoude worden, ter tijd toe, dat ik hem tot den keizer zenden zou.
\par 22 En Agrippa zeide tot Festus: Ik wilde ook zelf dien mens wel horen. En hij zeide: Morgen zult gij hem horen.
\par 23 Des anderen daags dan, als Agrippa gekomen was en Bernice, met grote pracht, en als zij ingegaan waren in het rechthuis, met de oversten over duizend, en de mannen, die de voornaamsten de stad waren, werd Paulus op bevel van Festus voor gebracht.
\par 24 En Festus zeide: Koning Agrippa, en gij mannen allen, die met ons hier tegenwoordig zijt, gij ziet dezen, van welken mij de ganse menigte der Joden heeft aangesproken, beide te Jeruzalem en hier, roepende, dat hij niet meer behoort te leven.
\par 25 Maar ik bevonden hebbende, dat hij niets des doods waardig gedaan had, en dewijl hij ook zelf zich op den keizer beroepen heeft, heb besloten hem te zenden.
\par 26 Van welken ik niets zekers heb aan den heer te schrijven; daarom heb ik hem voor ulieden voorgebracht, en meest voor u, koning Agrippa, opdat ik, na gedane onderzoeking, wat heb te schrijven.
\par 27 Want het dunkt mij tegen rede, een gevangene te zenden, en niet ook de beschuldigingen, die tegen hem zijn, te kennen te geven.

\chapter{26}

\par 1 En Agrippa zeide tot Paulus: Het is u geoorloofd voor uzelven te spreken. Toen strekte Paulus de hand uit, en verantwoordde zich aldus:
\par 2 Ik acht mijzelven gelukkig, o koning Agrippa, dat ik mij heden voor u zal verantwoorden van alles, waarover ik van de Joden beschuldigd word;
\par 3 Allermeest, dewijl ik weet, dat gij kennis hebt van alle gewoonten en vragen, die onder de Joden zijn. Daarom bid ik u, dat gij mij lankmoediglijk hoort.
\par 4 Mijn leven dan van der jonkheid aan, hetwelk van den beginne onder mijn volk te Jeruzalem geweest is, weten al de Joden;
\par 5 Als die van over lang mij te voren gekend hebben (indien zij het wilden getuigen), dat ik, naar de bescheidenste sekte van onzen godsdienst, als een Farizeer geleefd heb.
\par 6 En nu sta ik, en word geoordeeld over de hoop der belofte, die van God tot de vaderen geschied is;
\par 7 Tot dewelke onze twaalf geslachten, geduriglijk nacht en dag God dienende, verhopen te komen; over welke hoop ik, o koning Agrippa, van de Joden word beschuldigd.
\par 8 Wat? wordt het bij ulieden ongelofelijk geoordeeld, dat God de doden opwekt?
\par 9 Ik meende waarlijk bij mijzelven, dat ik tegen den Naam van Jezus van Nazareth vele wederpartijdige dingen moest doen.
\par 10 Hetwelk ik ook gedaan heb te Jeruzalem, en ik heb velen van de heiligen in de gevangenissen gesloten, de macht van de overpriesters ontvangen hebbende; en als zij omgebracht werden, stemde ik het toe.
\par 11 En door al de synagogen heb ik hen dikwijls gestraft, en gedwongen te lasteren; en boven mate tegen hen woedende, heb ik hen vervolgd, ook tot in de buiten landse steden.
\par 12 Waarover ook als ik naar Damaskus reisde, met macht en last, welk ik van de overpriesters had,
\par 13 Zag ik, o koning, in het midden van den dag, op den weg een licht, boven den glans der zon, van den hemel mij en degenen, die met mij reisden, omschijnende.
\par 14 En als wij allen ter aarde nedergevallen waren, hoorde ik een stem, tot mij sprekende, en zeggende in de Hebreeuwse taal: Saul, Saul, wat vervolgt gij Mij? Het is u hard, tegen de prikkels de verzenen te slaan.
\par 15 En ik zeide: Wie zijt Gij, Heere? En Hij zeide: Ik ben Jezus, Dien gij vervolgt.
\par 16 Maar richt u op, en sta op uw voeten; want hiertoe ben Ik u verschenen, om u te stellen tot een dienaar en getuige der dingen, beide die gij gezien hebt en in welke Ik u nog zal verschijnen;
\par 17 Verlossende u van dit volk, en van de heidenen, tot dewelke Ik u nu zende;
\par 18 Om hun ogen te openen, en hen te bekeren van de duisternis tot het licht, en van de macht des satans tot God; opdat zij vergeving der zonden ontvangen, en een erfdeel onder de geheiligden, door het geloof in Mij.
\par 19 Daarom, o koning Agrippa, ben ik dat Hemels gezicht niet ongehoorzaam geweest;
\par 20 Maar heb eerst dengenen, die te Damaskus waren, en te Jeruzalem, en in het gehele land van Judea, en den heidenen verkondigd, dat zij zich zouden beteren, en tot God bekeren, werken doende der bekering waardig.
\par 21 Om dezer zaken wil hebben mij de Joden in den tempel gegrepen en gepoogd om te brengen.
\par 22 Dan, hulp van God verkregen hebbende, sta ik tot op dezen dag, betuigende beiden klein en groot; niets zeggende buiten hetgeen de profeten en Mozes gesproken hebben, dat geschieden zoude;
\par 23 Namelijk dat de Christus lijden moest, en dat Hij, de Eerste uit de opstanding der doden zijnde, een licht zou verkondigen dezen volke, en den heidenen.
\par 24 En als hij deze dingen tot verantwoording sprak, zeide Festus met grote stem: Gij raast, Paulus, de grote geleerdheid brengt u tot razernij!
\par 25 Maar hij zeide: Ik raas niet, machtigste Festus, maar ik spreek woorden van waarheid en van een gezond verstand;
\par 26 Want de koning weet van deze dingen, tot welken ik ook vrijmoedigheid gebruikende spreek; want ik geloof niet, dat hem iets van deze dingen verborgen is; want dit is in geen hoek geschied.
\par 27 Gelooft gij, o koning Agrippa, de profeten? Ik weet dat gij ze gelooft.
\par 28 En Agrippa zeide tot Paulus: Gij beweegt mij bijna een Christen te worden.
\par 29 En Paulus zeide: Ik wenste wel van God, dat, en bijna en geheellijk, niet alleen gij, maar ook allen, die mij heden horen, zodanigen wierden, gelijk als ik ben, uitgenomen deze banden.
\par 30 En als hij dit gezegd had, stond de koning op, en de stadhouder, en Bernice, en die met hen gezeten waren;
\par 31 En aan een zijde gegaan zijnde, spraken zij tot elkander, zeggende: Deze mens doet niets des doods of der banden waardig.
\par 32 En Agrippa zeide tot Festus: Deze mens kon losgelaten worden, indien hij zich op den keizer niet had beroepen.

\chapter{27}

\par 1 En als het besloten was, dat wij naar Italie zouden afvaren, leverden zij Paulus en enige andere gevangenen, over aan een hoofdman over honderd, met name Julius van de keizerlijke bende.
\par 2 En in een Adramyttenisch schip gegaan zijnde, alzo wij de plaatsen langs Azie bevaren zouden, voeren wij af; en Aristarchus, de Macedonier van Thessalonica, was met ons.
\par 3 En des anderen daags kwamen wij aan te Sidon. En Julius, vriendelijk met Paulus handelende, liet hem toe tot de vrienden te gaan, om van hen bezorgd te worden.
\par 4 En van daar afgevaren zijnde, voeren wij onder Cyprus heen, omdat de winden ons tegen waren.
\par 5 En de zee, die langs Cilicie en Pamfylie is, doorgevaren zijnde, kwamen wij aan te Myra in Lycie.
\par 6 En de hoofdman, aldaar een schip gevonden hebbende van Alexandrie, dat naar Italie voer, deed ons in hetzelve overgaan.
\par 7 En als wij vele dagen langzaam voortvoeren, en nauwelijks tegenover Knidus gekomen waren, overmits het ons de wind niet toeliet, zo voeren wij onder Kreta heen, tegenover Salmone.
\par 8 En hetzelve nauwelijks voorbij zeilende, kwamen wij in een zekere plaats genaamd Schonehavens, waar de stad Lasea nabij was.
\par 9 En als veel tijd verlopen, en de vaart nu zorgelijk was, omdat ook de vasten nu voorbij was, vermaande hen Paulus,
\par 10 En zeide tot hen: Mannen, ik zie, dat de vaart zal geschieden met hinder en grote schade, niet alleen van de lading en van het schip, maar ook van ons leven.
\par 11 Doch de hoofdman geloofde meer den stuurman en den schipper, dan hetgeen van Paulus gezegd werd.
\par 12 En alzo de haven ongelegen was om te overwinteren, vond het meerder deel geraden ook van daar te varen, of zij enigszins te Fenix konden aankomen om te overwinteren, zijnde een haven in Kreta, strekkende tegen het zuidwesten en tegen het noordwesten.
\par 13 En alzo de zuidenwind zachtelijk waaide, meenden zij hun voornemen verkregen te hebben, en afgevaren zijnde, zeilden zij dicht voorbij Kreta henen.
\par 14 Maar niet lang daarna, sloeg tegen hetzelve een stormwind, genaamd Euroklydon.
\par 15 En als het schip daarmede weggerukt werd, en niet kon tegen den wind opzeilen, gaven wij het op, en dreven heen.
\par 16 En lopende onder een zeker eilandje, genaamd Klauda, konden wij nauwelijks de boot machtig worden.
\par 17 Dewelke opgehaald hebbende, gebruikten zij alle behulpselen, het schip ondergordende; en alzo zij vreesden, dat zij op de droogte Syrtis vervallen zouden, streken zij het zeil, en dreven alzo henen.
\par 18 En alzo wij van het onweder geweldiglijk geslingerd werden, deden zij den volgende dag een uitworp;
\par 19 En den derden dag wierpen wij met onze eigen handen het scheepsgereedschap uit.
\par 20 En als noch zon noch gesternten verschenen in vele dagen, en geen klein onweder ons drukte, zo werd ons voort alle hoop van behouden te worden benomen.
\par 21 En als men langen tijd zonder eten geweest was, toen stond Paulus op in het midden van hen, en zeide: O mannen, men behoorde mij wel gehoor gegeven te hebben, en van Kreta niet afgevaren te zijn, en dezen hinder en deze schade verhoed te hebben;
\par 22 Doch alsnu vermaan ik ulieden goedsmoeds te zijn; want er zal geen verlies geschieden van iemands leven onder u, maar alleen van het schip.
\par 23 Want dezen zelfden nacht heeft bij mij gestaan een engel Gods, Wiens ik ben, Welken ook ik dien,
\par 24 Zeggende: Vrees niet, Paulus, gij moet voor den keizer gesteld worden; en zie, God heeft u geschonken allen, die met u varen.
\par 25 Daarom zijt goedsmoeds, mannen, want ik geloof Gode, dat het alzo zijn zal, gelijkerwijs het mij gezegd is.
\par 26 Doch wij moeten op een zeker eiland vervallen.
\par 27 Als nu de veertiende nacht gekomen was, alzo wij in de Adriatische zee herwaarts en derwaarts gedreven werden, omtrent het midden des nachts, vermoedden de scheepslieden, dat hun enig land naderde.
\par 28 En het dieplood uitgeworpen hebbende, vonden zij twintig vademen; en een weinig voortgevaren zijnde, wierpen zij wederom het dieplood uit, en vonden vijftien vademen;
\par 29 En vrezende, dat zij ergens op harde plaatsen vervallen mochten, wierpen zij vier ankers van het achterschip uit, en wensten, dat het dag werd.
\par 30 Maar als de scheepslieden zochten uit het schip te vlieden, en de boot nederlieten in de zee, onder den schijn, alsof zij uit het voorschip de ankers zouden uitbrengen,
\par 31 Zeide Paulus tot den hoofdman en tot de krijgsknechten: Indien dezen in het schip niet blijven, gij kunt niet behouden worden.
\par 32 Toen hieuwen de krijgsknechten de touwen af van de boot, en lieten haar vallen.
\par 33 En ondertussen dat het dag zou worden, vermaande Paulus hen allen, dat zij zouden spijze nemen, en zeide: Het is heden de veertiende dag, dat gij verwachtende blijft zonder eten, en niets hebt genomen.
\par 34 Daarom vermaan ik u spijze te nemen, want dat dient tot uw behouding; want niemand van u zal een haar van het hoofd vallen.
\par 35 En als hij dit gezegd had en brood genomen had, dankte hij God in aller tegenwoordigheid; en hetzelve gebroken hebbende, begon hij te eten.
\par 36 En zij allen, goedsmoeds geworden zijnde, namen ook zelven spijze.
\par 37 Wij waren nu in het schip in alles tweehonderd zes en zeventig zielen.
\par 38 En als zij met spijze verzadigd waren, lichtten zij het schip, en wierpen het koren uit in de zee.
\par 39 En toen het dag werd, kenden zij het land niet; maar zij merkten een zekeren inham, die een oever had, tegen denwelken zij geraden vonden, zo zij konden, het schip aan te zetten.
\par 40 En als zij de ankers opgehaald hadden, gaven zij het schip aan de zee over, meteen de roerbanden losmakende; en het razeil naar den wind opgehaald hebbende, hielden zij het naar den oever toe.
\par 41 Maar vervallende op een plaats, die de zee aan beide zijden had, zetten zij het schip daarop; en het voorschip, vastzittende, bleef onbewegelijk, maar het achterschip brak van het geweld der baren.
\par 42 De raadslag nu der krijgslieden was, dat zij de gevangenen zouden doden, opdat niemand, ontzwommen zijnde, zoude ontvlieden.
\par 43 Maar de hoofdman, willen Paulus behouden, belette hun dat voornemen, en beval, dat degenen, die zwemmen konden, zich eerst zouden afwerpen, en te land komen;
\par 44 En de anderen, sommigen op planken, en sommigen op enige stukken van het schip. En alzo is het geschied, dat zij allen behouden aan het land gekomen zijn.

\chapter{28}

\par 1 En als zij ontkomen waren, toen verstonden zij, dat het eiland Melite heette.
\par 2 En de barbaren bewezen ons geen gemene vriendelijkheid; want een groot vuur ontstoken hebbende, namen zij ons allen in, om den regen, die overkwam, en om de koude.
\par 3 En als Paulus een hoop rijzen bijeengeraapt en op het vuur gelegd had, kwam er een adder uit door de hitte, en vatte zijn hand.
\par 4 En als de barbaren het beest zagen aan zijn hand hangen, zeiden zij tot elkander: Deze mens is gewisselijk een doodslager, welken de wraak niet laat leven, daar hij uit de zee ontkomen is.
\par 5 Maar hij schudde het beest af in het vuur, en leed niets kwaads.
\par 6 En zij verwachtten, dat hij zou opzwellen, of terstond dood nedervallen. Maar als zij lang gewacht hadden, en zagen, dat geen ongemak hem overkwam, werden zij veranderd, en zeiden, dat hij een god was.
\par 7 En hier, omtrent dezelfde plaats, had de voornaamste van het eiland, met name Publius, zijn landhoeven, die ons ontving, en drie dagen vriendelijk herbergde.
\par 8 En het geschiedde, dat de vader van Publius, met koortsen en den roden loop bevangen zijnde, te bed lag; tot denwelken Paulus inging, en als hij gebeden had, legde hij de handen op hem, en maakte hem gezond.
\par 9 Als dit dan geschied was, kwamen ook tot hem de anderen, die krankheden hadden in het eiland, en werden genezen.
\par 10 Die ons ook eerden met veel eer, en als wij vertrekken zouden, bestelden zij ons hetgeen van node was.
\par 11 En na drie maanden voeren wij af in een schip van Alexandrie, dat in het eiland overwinterd had, hebbende tot een teken, Kastor en Pollux.
\par 12 En als wij te Syrakuse aangekomen waren, bleven wij aldaar drie dagen;
\par 13 Van waar wij omvoeren, en kwamen aan te Regium; en alzo, na een dag, de wind zuid werd, kwamen wij den tweeden dag te Puteoli;
\par 14 Alwaar wij broeders vonden, en werden gebeden, zeven dagen bij hen te blijven; en alzo gingen wij naar Rome.
\par 15 En vandaar kwamen de broeders, van onze zaken gehoord hebbende, ons tegemoet tot Appiusmarkt, en de drie tabernen; welke Paulus ziende, dankte hij God en greep moed.
\par 16 En toen wij te Rome gekomen waren, gaf de hoofdman de gevangenen over aan den overste des legers; maar aan Paulus werd toegelaten op zichzelven te wonen met den krijgsknecht, die hem bewaarde.
\par 17 En het geschiedde na drie dagen dat Paulus samenriep degenen, die de voornaamsten der Joden waren. En als zij samengekomen waren, zeide hij tot hen: Mannen broeders, ik, die niets gedaan heb tegen het volk of de vaderlijke gewoonten, ben gebonden uit Jeruzalem overgeleverd in de handen der Romeinen;
\par 18 Dewelken, mij onderzocht hebbende, wilden mij loslaten, omdat geen schuld des doods in mij was.
\par 19 Maar als de Joden zulks tegenspraken, werd ik genoodzaakt mij op den keizer te beroepen; doch niet, alsof ik iets had, mijn volk te beschuldigen.
\par 20 Om deze oorzaak dan heb ik u bij mij geroepen, om u te zien en aan te spreken; want vanwege de hope Israels ben ik met deze keten omvangen.
\par 21 Maar zij zeiden tot hem: Wij hebben noch brieven u aangaande van Judea ontvangen; noch iemand van de broeders, hier gekomen zijnde, heeft van u iets kwaads geboodschapt of gesproken.
\par 22 Maar wij begeren wel van u te horen, wat gij gevoelt; want wat deze sekte aangaat, ons is bekend, dat zij overal tegengesproken wordt.
\par 23 En als zij hem een dag gesteld hadden, kwamen er velen in zijn woonplaats; denwelken hij het Koninkrijk Gods uitlegde, en betuigde, en poogde hen te bewegen tot het geloof in Jezus, beide uit de wet van Mozes en de profeten, van des morgens vroeg tot den avond toe.
\par 24 En sommigen geloofden wel, hetgeen gezegd werd, maar sommigen geloofden niet.
\par 25 En tegen elkander oneens zijnde, scheidden zij; als Paulus dit ene woord gezegd had, namelijk: Wel heeft de Heilige Geest gesproken door Jesaja, den profeet, tot onze vaderen,
\par 26 Zeggende: Ga heen tot dit volk, en zeg: Met het gehoor zult gij horen, en geenszins verstaan; en ziende zult gij zien, en geenszins bemerken.
\par 27 Want het hart dezes volks is dik geworden, en met de oren hebben zij zwaarlijk gehoord, en hun ogen hebben zij toegedaan; opdat zij niet te eniger tijd met de ogen zouden zien, en met de oren horen, en met het hart verstaan, en zij zich bekeren, en Ik hen geneze.
\par 28 Het zij u dan bekend, dat de zaligheid Gods den heidenen gezonden is, en dezelve zullen horen.
\par 29 En als hij dit gezegd had, gingen de Joden weg, veel twisting hebbenden onder elkander.
\par 30 En Paulus bleef twee gehele jaren in zijn eigen gehuurde woning; en ontving allen, die tot hem kwamen;
\par 31 Predikende het Koninkrijk Gods, en lerende van den Heere Jezus Christus met alle vrijmoedigheid, onverhinderd.




\end{document}