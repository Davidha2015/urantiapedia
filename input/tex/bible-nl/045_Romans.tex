\begin{document}

\title{Romeinen}



\chapter{1}

\par 1 Paulus, een dienstknecht van Jezus Christus, een geroepen apostel, afgezonderd tot het Evangelie van God,
\par 2 (Hetwelk Hij te voren beloofd had door Zijn profeten, in de heilige Schriften).
\par 3 Van Zijn Zoon, Die geworden is uit het zaad van David, naar het vlees;
\par 4 Die krachtelijk bewezen is te zijn de Zoon van God, naar den Geest der heiligmaking, uit de opstanding der doden) namelijk Jezus Christus, onzen Heere:
\par 5 (Door Welken wij hebben ontvangen genade en het apostelschap, tot gehoorzaamheid des geloofs onder al de heidenen, voor Zijn Naam;
\par 6 Onder welken gij ook zijt, geroepenen van Jezus Christus!)
\par 7 Allen, die te Rome zijt, geliefden Gods, en geroepen heiligen, genade zij u, en vrede van God, onzen Vader, en den Heere Jezus Christus.
\par 8 Eerstelijk dank ik mijn God door Jezus Christus over u allen, dat uw geloof verkondigd wordt in de gehele wereld.
\par 9 Want God is mijn Getuige, Welken ik diene in mijn geest, in het Evangelie Zijns Zoons, hoe ik zonder nalaten uwer gedenke;
\par 10 Allen tijd in mijn gebeden biddende, of mogelijk mij nog te eniger tijd goede gelegenheid gegeven wierd, door den wil van God, om tot ulieden te komen.
\par 11 Want ik verlang om u te zien, opdat ik u enige geestelijke gave mocht mededelen, ten einde gij versterkt zoudt worden;
\par 12 Dat is, om mede vertroost te worden onder u, door het onderlinge geloof, zo het uwe als het mijne.
\par 13 Doch ik wil niet, dat u onbekend zij, broeders, dat ik menigmaal voorgenomen heb tot u te komen (en ben tot nog toe verhinderd geweest), opdat ik ook onder u enige vrucht zou hebben, gelijk als ook onder de andere heidenen.
\par 14 Beiden Grieken en Barbaren, beiden wijzen en onwijzen ben ik een schuldenaar.
\par 15 Alzo hetgeen in mij is, dat is volvaardig, om u ook, die te Rome zijt, het Evangelie te verkondigen.
\par 16 Want ik schaam mij des Evangelies van Christus niet; want het is een kracht Gods tot zaligheid een iegelijk, die gelooft, eerst den Jood, en ook den Griek.
\par 17 Want de rechtvaardigheid Gods wordt in hetzelve geopenbaard uit geloof tot geloof; gelijk geschreven is: Maar de rechtvaardige zal uit het geloof leven.
\par 18 Want de toorn Gods wordt geopenbaard van den hemel over alle goddeloosheid, en ongerechtigheid der mensen, als die de waarheid in ongerechtigheid ten onder houden.
\par 19 Overmits hetgeen van God kennelijk is, in hen openbaar is; want God heeft het hun geopenbaard.
\par 20 Want Zijn onzienlijke dingen worden, van de schepping der wereld aan, uit de schepselen verstaan en doorzien, beide Zijn eeuwige kracht en Goddelijkheid, opdat zij niet te verontschuldigen zouden zijn.
\par 21 Omdat zij, God kennende, Hem als God niet hebben verheerlijkt of gedankt; maar zijn verijdeld geworden in hun overleggingen en hun onverstandig hart is verduisterd geworden;
\par 22 Zich uitgevende voor wijzen, zijn zij dwaas geworden;
\par 23 En hebben de heerlijkheid des onverderfelijken Gods veranderd in de gelijkenis eens beelds van een verderfelijk mens, en van gevogelte, en van viervoetige en kruipende gedierten.
\par 24 Daarom heeft God hen ook overgegeven in de begeerlijkheden hunner harten tot onreinigheid, om hun lichamen onder elkander te onteren;
\par 25 Als die de waarheid Gods veranderd hebben in de leugen, en het schepsel geeerd en gediend hebben boven den Schepper, Die te prijzen is in der eeuwigheid, amen.
\par 26 Daarom heeft God hen overgegeven tot oneerlijke bewegingen; want ook hun vrouwen hebben het natuurlijk gebruik veranderd in het gebruik tegen nature;
\par 27 En insgelijks ook de mannen, nalatende het natuurlijk gebruik der vrouw, zijn verhit geworden in hun lust tegen elkander, mannen met mannen schandelijkheid bedrijvende, en de vergelding van hun dwaling, die daartoe behoorde, in zichzelven ontvangende.
\par 28 En gelijk het hun niet goed gedacht heeft God in erkentenis te houden, zo heeft God hen overgegeven in een verkeerden zin, om te doen dingen, die niet betamen;
\par 29 Vervuld zijnde met alle ongerechtigheid, hoererij, boosheid, gierigheid, kwaadheid, vol van nijdigheid, moord, twist, bedrog, kwaadaardigheid;
\par 30 Oorblazers, achterklappers, haters Gods, smaders, hovaardigen, laatdunkenden, vinders van kwade dingen, den ouderen ongehoorzaam;
\par 31 Onverstandigen, verbondbrekers, zonder natuurlijke liefde, onverzoenlijken, onbarmhartigen;
\par 32 Dewelken, daar zij het recht Gods weten, namelijk, dat degenen, die zulke dingen doen, des doods waardig zijn) niet alleen dezelve doen, maar ook mede een welgevallen hebben in degenen, die ze doen.

\chapter{2}

\par 1 Daarom zijt gij niet te verontschuldigen, o mens, wie gij zijt, die anderen oordeelt; want waarin gij een ander oordeelt, veroordeelt gij uzelven; want gij, die anderen oordeelt, doet dezelfde dingen.
\par 2 En wij weten, dat het oordeel Gods naar waarheid is, over degenen, die zulke dingen doen.
\par 3 En denkt gij dit, o mens, die oordeelt dengenen, die zulke dingen doen, en dezelve doet, dat gij het oordeel Gods zult ontvlieden?
\par 4 Of veracht gij den rijkdom Zijner goedertierenheid, en verdraagzaamheid, en lankmoedigheid, niet wetende, dat de goedertierenheid Gods u tot bekering leidt?
\par 5 Maar naar uw hardigheid, en onbekeerlijk hart, vergadert gij uzelven toorn als een schat, in den dag des toorns, en der openbaring van het rechtvaardig oordeel Gods.
\par 6 Welke een iegelijk vergelden zal naar zijn werken;
\par 7 Dengenen wel, die met volharding in goeddoen, heerlijkheid, en eer, en onverderfelijkheid zoeken, het eeuwige leven;
\par 8 Maar dengenen, die twistgierig zijn, en die der waarheid ongehoorzaam, doch der ongerechtigheid gehoorzaam zijn, zal verbolgenheid en toorn vergolden worden;
\par 9 Verdrukking en benauwdheid over alle ziel des mensen, die het kwade werkt, eerst van den Jood, en ook van den Griek;
\par 10 Maar heerlijkheid, en eer, en vrede een iegelijk, die het goede werkt, eerst den Jood, en ook den Griek.
\par 11 Want er is geen aanneming des persoons bij God.
\par 12 Want zovelen, als er zonder wet gezondigd hebben, zullen ook zonder wet verloren gaan; en zovelen, als er onder de wet gezondigd hebben, zullen door de wet geoordeeld worden;
\par 13 (Want de hoorders der wet zijn niet rechtvaardig voor God, maar de daders der wet zullen gerechtvaardigd worden;
\par 14 Want wanneer de heidenen, die de wet niet hebben, van nature de dingen doen, die der wet zijn, deze, de wet niet hebbende, zijn zichzelven een wet;
\par 15 Als die betonen het werk der wet geschreven in hun harten, hun geweten medegetuigende, en de gedachten onder elkander hen beschuldigende, of ook ontschuldigende).
\par 16 In den dag wanneer God de verborgene dingen der mensen zal oordelen door Jezus Christus, naar mijn Evangelie.
\par 17 Zie, gij wordt een Jood genaamd en rust op de wet; en roemt op God,
\par 18 En gij weet Zijn wil, en beproeft de dingen, die daarvan verschillen, zijnde onderwezen uit de wet;
\par 19 En gij betrouwt uzelven te zijn een leidsman der blinden, een licht dergenen, die in duisternis zijn;
\par 20 Een onderrichter der onwijzen, en een leermeester der onwetenden, hebbende de gedaante der kennis en der waarheid in de wet.
\par 21 Die dan een anderen leert, leert gij uzelven niet? Die predikt, dat men niet stelen zal, steelt gij?
\par 22 Die zegt, dat men geen overspel doen zal, doet gij overspel? Die van de afgoden een gruwel hebt, berooft gij het heilige?
\par 23 Die op de wet roemt, onteert gij God door de overtreding der wet?
\par 24 Want de Naam van God wordt om uwentwil gelasterd onder de heidenen, gelijk geschreven is.
\par 25 Want de besnijdenis is wel nut, indien gij de wet doet; maar indien gij een overtreder der wet zijt, zo is uw besnijdenis voorhuid geworden.
\par 26 Indien dan de voorhuid de rechten der wet bewaart, zal niet zijn voorhuid tot een besnijdenis gerekend worden?
\par 27 En zal de voorhuid, die uit de natuur is, als zij de wet volbrengt, u niet oordelen, die door de letter en besnijdenis een overtreder der wet zijt?
\par 28 Want die is niet een Jood, die het in het openbaar is; noch die is de besnijdenis, die het in het openbaar in het vlees is;
\par 29 Maar die is een Jood, die het in het verborgen is, en de besnijdenis des harten, in den geest, niet in de letter, is de besnijdenis; wiens lof niet is uit de mensen, maar uit God.

\chapter{3}

\par 1 Welk is dan het voordeel van den Jood? Of welk is de nuttigheid der besnijdenis?
\par 2 Vele in alle manier; want dit is wel het eerste, dat hun de Woorden Gods zijn toebetrouwd.
\par 3 Want wat is het, al zijn sommigen ongelovig geweest? Zal hun ongelovigheid het geloof van God te niet doen?
\par 4 Dat zij verre. Doch God zij waarachtig, maar alle mens leugenachtig; gelijk als geschreven is: Opdat Gij gerechtvaardigd wordt in Uw woorden, en overwint, wanneer Gij oordeelt.
\par 5 Indien nu onze ongerechtigheid Gods gerechtigheid bevestigt, wat zullen wij zeggen? Is God onrechtvaardig, als Hij toorn over ons brengt? (Ik spreek naar den mens.)
\par 6 Dat zij verre, anderszins hoe zal God de wereld oordelen?
\par 7 Want indien de waarheid Gods door mijn leugen overvloediger is geworden, tot Zijn heerlijkheid, wat word ik ook nog als een zondaar geoordeeld?
\par 8 En zeggen wij niet liever (gelijk wij gelasterd worden, en gelijk sommigen zeggen, dat wij zeggen): Laat ons het kwade doen, opdat het goede daaruit kome? Welker verdoemenis rechtvaardig is.
\par 9 Wat dan? Zijn wij uitnemender? Ganselijk niet; want wij hebben te voren beschuldigd beiden Joden en Grieken, dat zij allen onder de zonde zijn;
\par 10 Gelijk geschreven is: Er is niemand rechtvaardig, ook niet een;
\par 11 Er is niemand, die verstandig is, er is niemand, die God zoekt.
\par 12 Allen zijn zij afgeweken, te zamen zijn zij onnut geworden; er is niemand, die goed doet, er is ook niet tot een toe.
\par 13 Hun keel is een geopend graf; met hun tongen plegen zij bedrog; slangenvenijn is onder hun lippen.
\par 14 Welker mond vol is van vervloeking en bitterheid;
\par 15 Hun voeten zijn snel om bloed te vergieten;
\par 16 Vernieling en ellendigheid is in hun wegen;
\par 17 En den weg des vredes hebben zij niet gekend.
\par 18 Er is geen vreze Gods voor hun ogen.
\par 19 Wij weten nu, dat al wat de wet zegt, zij dat spreekt tot degenen, die onder de wet zijn; opdat alle mond gestopt worde en de gehele wereld voor God verdoemelijk zij.
\par 20 Daarom zal uit de werken der wet geen vlees gerechtvaardigd worden, voor Hem; want door de wet is de kennis der zonde.
\par 21 Maar nu is de rechtvaardigheid Gods geopenbaard geworden zonder de wet, hebbende getuigenis van de wet en de profeten:
\par 22 Namelijk de rechtvaardigheid Gods door het geloof van Jezus Christus, tot allen, en over allen, die geloven; want er is geen onderscheid.
\par 23 Want zij hebben allen gezondigd, en derven de heerlijkheid Gods;
\par 24 En worden om niet gerechtvaardigd, uit Zijn genade, door de verlossing, die in Christus Jezus is;
\par 25 Welken God voorgesteld heeft tot een verzoening, door het geloof in Zijn bloed, tot een betoning van Zijn rechtvaardigheid, door de vergeving der zonden, die te voren geschied zijn onder de verdraagzaamheid Gods;
\par 26 Tot een betoning van Zijn rechtvaardigheid in dezen tegenwoordigen tijd; opdat Hij rechtvaardig zij, en rechtvaardigende dengene, die uit het geloof van Jezus is.
\par 27 Waar is dan de roem? Hij is uitgesloten. Door wat wet? Der werken? Neen, maar door de wet des geloofs.
\par 28 Wij besluiten dan, dat de mens door het geloof gerechtvaardigd wordt, zonder de werken der wet.
\par 29 Is God een God der Joden alleen? en is Hij het niet ook der heidenen? Ja, ook der heidenen;
\par 30 Nademaal Hij een enig God is, Die de besnijdenis rechtvaardigen zal uit het geloof, en de voorhuid door het geloof.
\par 31 Doen wij dan de wet te niet door het geloof? Dat zij verre; maar wij bevestigen de wet.

\chapter{4}

\par 1 Wat zullen wij dan zeggen, dat Abraham, onze vader, verkregen heeft naar het vlees?
\par 2 Want indien Abraham uit de werken gerechtvaardigd is, zo heeft hij roem, maar niet bij God.
\par 3 Want wat zegt de Schrift? En Abraham geloofde God, en het is hem gerekend tot rechtvaardigheid.
\par 4 Nu dengene, die werkt, wordt het loon niet toegerekend naar genade, maar naar schuld.
\par 5 Doch dengene, die niet werkt, maar gelooft in Hem, Die den goddeloze rechtvaardigt, wordt zijn geloof gerekend tot rechtvaardigheid.
\par 6 Gelijk ook David den mens zalig spreekt, welken God de rechtvaardigheid toerekent zonder werken;
\par 7 Zeggende: Zalig zijn zij, welker ongerechtigheden vergeven zijn, en welker zonden bedekt zijn;
\par 8 Zalig is de man, welken de Heere de zonden niet toerekent.
\par 9 Deze zaligspreking dan, is die alleen over de besnijdenis, of ook over de voorhuid? Want wij zeggen, dat Abraham het geloof gerekend is tot rechtvaardigheid.
\par 10 Hoe is het hem dan toegerekend? Als hij in de besnijdenis was, of in de voorhuid? Niet in de besnijdenis, maar in de voorhuid.
\par 11 En hij heeft het teken der besnijdenis ontvangen tot een zegel der rechtvaardigheid des geloofs, die hem in de voorhuid was toegerekend: opdat hij zou zijn een vader van allen, die geloven in de voorhuid zijnde, ten einde ook hun de rechtvaardigheid toegerekend worde;
\par 12 En een vader der besnijdenis, dengenen namelijk, die niet alleen uit de besnijdenis zijn, maar die ook wandelen in de voetstappen des geloofs van onzen vader Abraham, hetwelk in de voorhuid was.
\par 13 Want de belofte is niet door de wet aan Abraham of zijn zaad geschied, namelijk, dat hij een erfgenaam der wereld zou zijn, maar door de rechtvaardigheid des geloofs.
\par 14 Want indien degenen, die uit de wet zijn, erfgenamen zijn, zo is het geloof ijdel geworden, en de beloftenis te niet gedaan.
\par 15 Want de wet werkt toorn; want waar geen wet is, daar is ook geen overtreding.
\par 16 Daarom is zij uit het geloof, opdat zij naar genade zij; ten einde de belofte vast zij al den zade, niet alleen dat uit de wet is, maar ook dat uit het geloof Abrahams is, welke een vader is van ons allen;
\par 17 (Gelijk geschreven staat: Ik heb u tot een vader van vele volken gesteld) voor Hem, aan Welken hij geloofd heeft, namelijk God, Die de doden levend maakt, en roept de dingen, die niet zijn, alsof zij waren;
\par 18 Welke tegen hoop op hoop geloofd heeft, dat hij zou worden een vader van vele volken; volgens hetgeen gezegd was: Alzo zal uw zaad wezen.
\par 19 En niet verzwakt zijnde in het geloof, heeft hij zijn eigen lichaam niet aangemerkt, dat alrede verstorven was, alzo hij omtrent honderd jaren oud was, noch ook dat de moeder in Sara verstorven was.
\par 20 En hij heeft aan de beloftenis Gods niet getwijfeld door ongeloof; maar is gesterkt geweest in het geloof, gevende God de eer;
\par 21 En ten volle verzekerd zijnde, dat hetgeen beloofd was, Hij ook machtig was te doen.
\par 22 Daarom is het hem ook tot rechtvaardigheid gerekend.
\par 23 Nu is het niet alleen om zijnentwil geschreven, dat het hem toegerekend is;
\par 24 Maar ook om onzentwil, welken het zal toegerekend worden, namelijk dengenen, die geloven in Hem, Die Jezus, onzen Heere, uit de doden opgewekt heeft;
\par 25 Welke overgeleverd is om onze zonden, en opgewekt om onze rechtvaardigmaking.

\chapter{5}

\par 1 Wij dan, gerechtvaardigd zijnde uit het geloof, hebben vrede bij God, door onzen Heere Jezus Christus;
\par 2 Door Welken wij ook de toeleiding hebben door het geloof tot deze genade, in welke wij staan, en roemen in de hoop der heerlijkheid Gods.
\par 3 En niet alleenlijk dit, maar wij roemen ook in de verdrukkingen, wetende, dat de verdrukking lijdzaamheid werkt;
\par 4 En de lijdzaamheid bevinding, en de bevinding hoop;
\par 5 En de hoop beschaamt niet, omdat de liefde Gods in onze harten uitgestort is door den Heiligen Geest, Die ons is gegeven.
\par 6 Want Christus, als wij nog krachteloos waren, is te Zijner tijd voor de goddelozen gestorven.
\par 7 Want nauwelijks zal iemand voor een rechtvaardige sterven; want voor den goede zal mogelijk iemand ook bestaan te sterven.
\par 8 Maar God bevestigt Zijn liefde jegens ons, dat Christus voor ons gestorven is, als wij nog zondaars waren.
\par 9 Veel meer dan, zijnde nu gerechtvaardigd door Zijn bloed, zullen wij door Hem behouden worden van den toorn.
\par 10 Want indien wij, vijanden zijnde, met God verzoend zijn door den dood Zijns Zoons, veel meer zullen wij, verzoend zijnde, behouden worden door Zijn leven.
\par 11 En niet alleenlijk dit, maar wij roemen ook in God, door onzen Heere Jezus Christus, door Welken wij nu de verzoening gekregen hebben.
\par 12 Daarom, gelijk door een mens de zonde in de wereld ingekomen is, en door de zonde de dood; en alzo de dood tot alle mensen doorgegaan is, in welken allen gezondigd hebben.
\par 13 Want tot de wet was de zonde in de wereld; maar de zonde wordt niet toegerekend, als er geen wet is.
\par 14 Maar de dood heeft geheerst van Adam tot Mozes toe, ook over degenen, die niet gezondigd hadden in de gelijkheid der overtreding van Adam, welke een voorbeeld is Desgenen, Die komen zou.
\par 15 Doch niet, gelijk de misdaad, alzo is ook de genadegift, want indien, door de misdaad van een, velen gestorven zijn, zo is veel meer de genade Gods, en de gave door de genade, die daar is van een mens Jezus Christus, overvloedig geweest over velen.
\par 16 En niet, gelijk de schuld was door den een, die gezondigd heeft, alzo is de gift; want de schuld is wel uit een misdaad tot verdoemenis, maar de genadegift is uit vele misdaden tot rechtvaardigmaking.
\par 17 Want indien door de misdaad van een de dood geheerst heeft door dien enen, veel meer zullen degenen, die den overvloed der genade en der gave der rechtvaardigheid ontvangen, in het leven heersen door dien Enen, namelijk Jezus Christus.
\par 18 Zo dan, gelijk door een misdaad de schuld gekomen is over alle mensen tot verdoemenis; alzo ook door een rechtvaardigheid komt de genade over alle mensen tot rechtvaardigmaking des levens.
\par 19 Want gelijk door de ongehoorzaamheid van dien enen mens velen tot zondaars gesteld zijn geworden, alzo zullen ook door de gehoorzaamheid van Enen velen tot rechtvaardigen gesteld worden.
\par 20 Maar de wet is bovendien ingekomen, opdat de misdaad te meerder worde; en waar de zonde meerder geworden is, daar is de genade veel meer overvloedig geweest;
\par 21 Opdat, gelijk de zonde geheerst heeft tot den dood, alzo ook de genade zou heersen door rechtvaardigheid tot het eeuwige leven, door Jezus Christus onzen Heere.

\chapter{6}

\par 1 Wat zullen wij dan zeggen? Zullen wij in de zonde blijven, opdat de genade te meerder worde?
\par 2 Dat zij verre. Wij, die der zonde gestorven zijn, hoe zullen wij nog in dezelve leven?
\par 3 Of weet gij niet, dat zovelen als wij in Christus Jezus gedoopt zijn, wij in Zijn dood gedoopt zijn?
\par 4 Wij zijn dan met Hem begraven, door den doop in den dood, opdat, gelijkerwijs Christus uit de doden opgewekt is tot de heerlijkheid des Vaders, alzo ook wij in nieuwigheid des levens wandelen zouden.
\par 5 Want indien wij met Hem een plant geworden zijn in de gelijkmaking Zijns doods, zo zullen wij het ook zijn in de gelijkmaking Zijner opstanding;
\par 6 Dit wetende, dat onze oude mens met Hem gekruisigd is, opdat het lichaam der zonde te niet gedaan worde, opdat wij niet meer de zonde dienen.
\par 7 Want die gestorven is, die is gerechtvaardigd van de zonde.
\par 8 Indien wij nu met Christus gestorven zijn, zo geloven wij, dat wij ook met Hem zullen leven;
\par 9 Wetende, dat Christus, opgewekt zijnde uit de doden, niet meer sterft; de dood heerst niet meer over Hem.
\par 10 Want dat Hij gestorven is, dat is Hij der zonde eenmaal gestorven; en dat Hij leeft, dat leeft Hij Gode.
\par 11 Alzo ook gijlieden, houdt het daarvoor dat gij wel der zonde dood zijt, maar Gode levende zijt in Christus Jezus, onzen Heere.
\par 12 Dat dan de zonde niet heerse in uw sterfelijk lichaam, om haar te gehoorzamen in de begeerlijkheden deszelven lichaams.
\par 13 En stelt uwe leden niet der zonde tot wapenen der ongerechtigheid; maar stelt uzelven Gode, als uit de doden levende geworden zijnde, en stelt uw leden Gode tot wapenen der gerechtigheid.
\par 14 Want de zonde zal over u niet heersen; want gij zijt niet onder de wet, maar onder de genade.
\par 15 Wat dan? Zullen wij zondigen, omdat wij niet zijn onder de wet, maar onder de genade? Dat zij verre.
\par 16 Weet gij niet, dat wien gij uzelven stelt tot dienstknechten ter gehoorzaamheid, gij dienstknechten zijt desgenen, dien gij gehoorzaamt, of der zonde tot den dood, of der gehoorzaamheid tot gerechtigheid?
\par 17 Maar Gode zij dank, dat gij wel dienstknechten der zonde waart, maar dat gij nu van harte gehoorzaam geworden zijt aan het voorbeeld der leer, tot hetwelk gij overgegeven zijt;
\par 18 En vrijgemaakt zijnde van de zonde, zijt gemaakt dienstknechten der gerechtigheid.
\par 19 Ik spreek op menselijke wijze, om der zwakheid uws vleses wil; want gelijk gij uw leden gesteld hebt, om dienstbaar te zijn der onreinigheid en der ongerechtigheid, tot ongerechtigheid, alzo stelt nu uw leden, om dienstbaar te zijn der gerechtigheid, tot heiligmaking.
\par 20 Want toen gij dienstknechten waart der zonde, zo waart gij vrij van de gerechtigheid.
\par 21 Wat vrucht dan hadt gij toen van die dingen, waarover gij u nu schaamt? Want het einde derzelve is de dood.
\par 22 Maar nu, van de zonde vrijgemaakt zijnde, en Gode dienstbaar gemaakt zijnde, hebt gij uw vrucht tot heiligmaking, en het einde het eeuwige leven.
\par 23 Want de bezoldiging der zonde is de dood, maar de genadegift Gods is het eeuwige leven, door Jezus Christus, onzen Heere.

\chapter{7}

\par 1 Weet gij niet, broeders! (want ik spreek tot degenen, die de wet verstaan) dat de wet heerst over den mens, zo langen tijd als hij leeft?
\par 2 Want een vrouw, die onder den man staat, is aan den levenden man verbonden door de wet; maar indien de man gestorven is, zo is zij vrijgemaakt van de wet des mans.
\par 3 Daarom dan, indien zij eens anderen mans wordt, terwijl de man leeft, zo zal zij een overspeelster genaamd worden; maar indien de man gestorven is, zo is zij vrij van de wet, alzo dat zij geen overspeelster is, als zij eens anderen mans wordt.
\par 4 Zo dan, mijn broeders, gij zijt ook der wet gedood door het lichaam van Christus, opdat gij zoudt worden eens Anderen, namelijk Desgenen, Die van de doden opgewekt is, opdat wij Gode vruchten dragen zouden.
\par 5 Want toen wij in het vlees waren, wrochten de bewegingen der zonden, die door de wet zijn, in onze leden, om den dood vruchten te dragen.
\par 6 Maar nu zijn wij vrijgemaakt van de wet, overmits wij dien gestorven zijn, onder welken wij gehouden waren; alzo dat wij dienen in nieuwigheid des geestes, en niet in de oudheid der letter.
\par 7 Wat zullen wij dan zeggen? Is de wet zonde? Dat zij verre. Ja, ik kende de zonde niet dan door de wet; want ook had ik de begeerlijkheid niet geweten zonde te zijn, indien de wet niet zeide: Gij zult niet begeren.
\par 8 Maar de zonde, oorzaak genomen hebbende door het gebod, heeft in mij alle begeerlijkheid gewrocht; want zonder de wet is de zonde dood.
\par 9 En zonder de wet, zo leefde ik eertijds; maar als het gebod gekomen is, zo is de zonde weder levend geworden, doch ik ben gestorven.
\par 10 En het gebod, dat ten leven was, hetzelve is mij ten dood bevonden.
\par 11 Want de zonde, oorzaak genomen hebbende door het gebod, heeft mij verleid, en door hetzelve gedood.
\par 12 Alzo is dan de wet heilig, en het gebod is heilig, en rechtvaardig, en goed.
\par 13 Is dan het goede mij de dood geworden? Dat zij verre. Maar de zonde is mij de dood geworden; opdat zij zou openbaar worden zonde te zijn; werkende mij door het goede den dood; opdat de zonde boven mate wierd zondigende door het gebod.
\par 14 Want wij weten, dat de wet geestelijk is, maar ik ben vleselijk, verkocht onder de zonde.
\par 15 Want hetgeen ik doe, dat ken ik niet; want hetgeen ik wil, dat doe ik niet, maar hetgeen ik haat, dat doe ik.
\par 16 En indien ik hetgene doe, dat ik niet wil, zo stem ik de wet toe, dat zij goed is.
\par 17 Ik dan doe datzelve nu niet meer, maar de zonde, die in mij woont.
\par 18 Want ik weet, dat in mij, dat is, in mijn vlees, geen goed woont; want het willen is wel bij mij, maar het goede te doen, dat vind ik niet.
\par 19 Want het goede dat ik wil, doe ik niet, maar het kwade, dat ik niet wil, dat doe ik.
\par 20 Indien ik hetgene doe, dat ik niet wil, zo doe ik nu hetzelve niet meer, maar de zonde, die in mij woont.
\par 21 Zo vind ik dan deze wet in mij; als ik het goede wil doen, dat het kwade mij bijligt.
\par 22 Want ik heb een vermaak in de wet Gods, naar den inwendigen mens;
\par 23 Maar ik zie een andere wet in mijn leden, welke strijdt tegen de wet mijns gemoeds, en mij gevangen neemt onder de wet der zonde, die in mijn leden is.
\par 24 Ik ellendig mens, wie zal mij verlossen uit het lichaam dezes doods?
\par 25 Ik dank God, door Jezus Christus, onzen Heere.
\par 26 Zo dan, ik zelf dien wel met het gemoed de wet Gods, maar met het vlees de wet der zonde.

\chapter{8}

\par 1 Zo is er dan nu geen verdoemenis voor degenen, die in Christus Jezus zijn, die niet naar het vlees wandelen, maar naar den Geest.
\par 2 Want de wet des Geestes des levens in Christus Jezus heeft mij vrijgemaakt van de wet der zonde en des doods.
\par 3 Want hetgeen der wet onmogelijk was, dewijl zij door het vlees krachteloos was, heeft God, Zijn Zoon zendende in gelijkheid des zondigen vleses, en dat voor de zonde, de zonde veroordeeld in het vlees.
\par 4 Opdat het recht der wet vervuld zou worden in ons, die niet naar het vlees wandelen, maar naar den Geest.
\par 5 Want die naar het vlees zijn, bedenken, dat des vleses is; maar die naar den Geest zijn, bedenken, dat des Geestes is.
\par 6 Want het bedenken des vleses is de dood; maar het bedenken des Geestes is het leven en vrede;
\par 7 Daarom dat het bedenken des vleses vijandschap is tegen God; want het onderwerpt zich der wet Gods niet; want het kan ook niet.
\par 8 En die in het vlees zijn, kunnen Gode niet behagen.
\par 9 Doch gijlieden zijt niet in het vlees, maar in den Geest, zo anders de Geest Gods in u woont. Maar zo iemand den Geest van Christus niet heeft, die komt Hem niet toe.
\par 10 En indien Christus in ulieden is, zo is wel het lichaam dood om der zonden wil; maar de geest is leven om der gerechtigheid wil.
\par 11 En indien de Geest Desgenen, Die Jezus uit de doden opgewekt heeft, in u woont, zo zal Hij, Die Christus uit de doden opgewekt heeft, ook uw sterfelijke lichamen levend maken, door Zijn Geest, Die in u woont.
\par 12 Zo dan, broeders, wij zijn schuldenaars niet aan het vlees, om naar het vlees te leven.
\par 13 Want indien gij naar het vlees leeft, zo zult gij sterven; maar indien gij door den Geest de werkingen des lichaams doodt, zo zult gij leven.
\par 14 Want zovelen als er door den Geest Gods geleid worden, die zijn kinderen Gods.
\par 15 Want gij hebt niet ontvangen den Geest der dienstbaarheid wederom tot vreze; maar gij hebt ontvangen den Geest der aanneming tot kinderen, door Welken wij roepen: Abba, Vader!
\par 16 Dezelve Geest getuigt met onzen geest, dat wij kinderen Gods zijn.
\par 17 En indien wij kinderen zijn, zo zijn wij ook erfgenamen, erfgenamen van God, en medeerfgenamen van Christus; zo wij anders met Hem lijden, opdat wij ook met Hem verheerlijkt worden.
\par 18 Want ik houde het daarvoor, dat het lijden dezes tegenwoordigen tijds niet is te waarderen tegen de heerlijkheid, die aan ons zal geopenbaard worden.
\par 19 Want het schepsel, als met opgestoken hoofde, verwacht de openbaring der kinderen Gods.
\par 20 Want het schepsel is der ijdelheid onderworpen, niet gewillig, maar om diens wil, die het der ijdelheid onderworpen heeft;
\par 21 Op hoop, dat ook het schepsel zelf zal vrijgemaakt worden van de dienstbaarheid der verderfenis, tot de vrijheid der heerlijkheid der kinderen Gods.
\par 22 Want wij weten, dat het ganse schepsel te zamen zucht, en te zamen als in barensnood is tot nu toe.
\par 23 En niet alleen dit, maar ook wij zelven, die de eerstelingen des Geestes hebben, wij ook zelven, zeg ik, zuchten in onszelven, verwachtende de aanneming tot kinderen, namelijk de verlossing onzes lichaams.
\par 24 Want wij zijn in hope zalig geworden. De hoop nu, die gezien wordt, is geen hoop; want hetgeen iemand ziet, waarom zal hij het ook hopen?
\par 25 Maar indien wij hopen, hetgeen wij niet zien, zo verwachten wij het met lijdzaamheid.
\par 26 En desgelijks komt ook de Geest onze zwakheden mede te hulp; want wij weten niet, wat wij bidden zullen, gelijk het behoort, maar de Geest Zelf bidt voor ons met onuitsprekelijke zuchtingen.
\par 27 En Die de harten doorzoekt, weet, welke de mening des Geestes zij, dewijl Hij naar God voor de heiligen bidt.
\par 28 En wij weten, dat dengenen, die God liefhebben, alle dingen medewerken ten goede, namelijk dengenen, die naar Zijn voornemen geroepen zijn.
\par 29 Want die Hij te voren gekend heeft, die heeft Hij ook te voren verordineerd, den beelde Zijns Zoons gelijkvormig te zijn, opdat Hij de Eerstgeborene zij onder vele broederen.
\par 30 En die Hij te voren verordineerd heeft, dezen heeft Hij ook geroepen; en die Hij geroepen heeft, dezen heeft Hij ook gerechtvaardigd; en die Hij gerechtvaardigd heeft, dezen heeft Hij ook verheerlijkt.
\par 31 Wat zullen wij dan tot deze dingen zeggen? Zo God voor ons is, wie zal tegen ons zijn?
\par 32 Die ook Zijn eigen Zoon niet gespaard heeft, maar heeft Hem voor ons allen overgegeven, hoe zal Hij ons ook met Hem niet alle dingen schenken?
\par 33 Wie zal beschuldiging inbrengen tegen de uitverkorenen Gods? God is het, Die rechtvaardig maakt.
\par 34 Wie is het, die verdoemt? Christus is het, Die gestorven is; ja, wat meer is, Die ook opgewekt is, Die ook ter rechter hand Gods is, Die ook voor ons bidt.
\par 35 Wie zal ons scheiden van de liefde van Christus? Verdrukking, of benauwdheid, of vervolging, of honger, naaktheid, of gevaar, of zwaard?
\par 36 (Gelijk geschreven is: Want om Uwentwil worden wij den gansen dag gedood; wij zijn geacht als schapen ter slachting.)
\par 37 Maar in dit alles zijn wij meer dan overwinnaars, door Hem, Die ons liefgehad heeft.
\par 38 Want ik ben verzekerd, dat noch dood, noch leven, noch engelen, noch overheden, noch machten, noch tegenwoordige, noch toekomende dingen,
\par 39 Noch hoogte, noch diepte, noch enig ander schepsel ons zal kunnen scheiden van de liefde Gods, welke is in Christus Jezus, onzen Heere.

\chapter{9}

\par 1 Ik zeg de waarheid in Christus, ik lieg niet (mijn geweten mij mede getuigenis gevende door den Heiligen Geest),
\par 2 Dat het mij een grote droefheid, en mijn hart een gedurige smart is.
\par 3 Want ik zou zelf wel wensen verbannen te zijn van Christus, voor mijn broederen, die mijn maagschap zijn naar het vlees;
\par 4 Welke Israelieten zijn, welker is de aanneming tot kinderen, en de heerlijkheid, en de verbonden, en de wetgeving, en de dienst van God, en de beloftenissen;
\par 5 Welker zijn de vaders, en uit welke Christus is, zoveel het vlees aangaat, Dewelke is God boven allen te prijzen in der eeuwigheid. Amen.
\par 6 Doch ik zeg dit niet, alsof het woord Gods ware uitgevallen; want die zijn niet allen Israel, die uit Israel zijn.
\par 7 Noch omdat zij Abrahams zaad zijn, zijn zij allen kinderen; maar: In Izaak zal u het zaad genoemd worden.
\par 8 Dat is, niet de kinderen des vleses, die zijn kinderen Gods; maar de kinderen der beloftenis worden voor het zaad gerekend.
\par 9 Want dit is het woord der beloftenis: Omtrent dezen tijd zal Ik komen, en Sara zal een zoon hebben.
\par 10 En niet alleenlijk deze, maar ook Rebekka is daarvan een bewijs, als zij uit een bevrucht was, namelijk Izaak, onzen vader.
\par 11 Want als de kinderen nog niet geboren waren, noch iets goeds of kwaads gedaan hadden, opdat het voornemen Gods, dat naar de verkiezing is, vast bleve, niet uit de werken, maar uit den Roepende;
\par 12 Zo werd tot haar gezegd: De meerdere zal den mindere dienen.
\par 13 Gelijk geschreven is: Jakob heb Ik liefgehad, en Ezau heb Ik gehaat.
\par 14 Wat zullen wij dan zeggen? Is er onrechtvaardigheid bij God? Dat zij verre.
\par 15 Want Hij zegt tot Mozes: Ik zal Mij ontfermen, diens Ik Mij ontferm, en zal barmhartig zijn, dien Ik barmhartig ben.
\par 16 Zo is het dan niet desgenen, die wil, noch desgenen, die loopt, maar des ontfermenden Gods.
\par 17 Want de Schrift zegt tot Farao: Tot ditzelve heb Ik u verwekt, opdat Ik in u Mijn kracht bewijzen zou, en opdat Mijn Naam verkondigd worde op de ganse aarde.
\par 18 Zo ontfermt Hij Zich dan, diens Hij wil, en verhardt, dien Hij wil.
\par 19 Gij zult dan tot mij zeggen: Wat klaagt Hij dan nog? Want wie heeft Zijn wil wederstaan?
\par 20 Maar toch, o mens, wie zijt gij, die tegen God antwoordt? Zal ook het maaksel tot dengenen, die het gemaakt heeft, zeggen: Waarom hebt gij mij alzo gemaakt?
\par 21 Of heeft de pottenbakker geen macht over het leem, om uit denzelfden klomp te maken, het ene vat ter ere, en het andere ter onere?
\par 22 En of God, willende Zijn toorn bewijzen, en Zijn macht bekend maken, met vele lankmoedigheid verdragen heeft de vaten des toorns, tot het verderf toebereid;
\par 23 En opdat Hij zou bekend maken den rijkdom Zijner heerlijkheid over de vaten der barmhartigheid, die Hij te voren bereid heeft tot heerlijkheid?
\par 24 Welke Hij ook geroepen heeft, namelijk ons, niet alleen uit de Joden, maar ook uit de heidenen.
\par 25 Gelijk Hij ook in Hosea zegt: Ik zal hetgeen Mijn volk niet was, Mijn volk noemen, en die niet bemind was, Mijn beminde.
\par 26 En het zal zijn, in de plaats, waar tot hen gezegd was: Gijlieden zijt Mijn volk niet, aldaar zullen zij kinderen des levenden Gods genaamd worden.
\par 27 En Jesaja roept over Israel: Al ware het getal der kinderen Israels gelijk het zand der zee, zo zal het overblijfsel behouden worden.
\par 28 Want Hij voleindt een zaak en snijdt ze af in rechtvaardigheid; want de Heere zal een afgesneden zaak doen op de aarde.
\par 29 En gelijk Jesaja te voren gezegd heeft: Indien de Heere Sebaoth ons geen zaad had overgelaten, zo waren wij als Sodom geworden, en Gomorra gelijk gemaakt geweest.
\par 30 Wat zullen wij dan zeggen? Dat de heidenen, die de rechtvaardigheid niet zochten, de rechtvaardigheid verkregen hebben, doch de rechtvaardigheid, die uit het geloof is.
\par 31 Maar Israel, die de wet der rechtvaardigheid zocht, is tot de wet der rechtvaardigheid niet gekomen.
\par 32 Waarom? Omdat zij die zochten niet uit het geloof, maar als uit de werken der wet, want zij hebben zich gestoten aan den steen des aanstoots;
\par 33 Gelijk geschreven is: Ziet, Ik leg in Sion een steen des aanstoots, en een rots der ergernis; en een iegelijk, die in Hem gelooft, zal niet beschaamd worden.

\chapter{10}

\par 1 Broeders, de toegenegenheid mijns harten, en het gebed, dat ik tot God voor Israel doe, is tot hun zaligheid.
\par 2 Want ik geef hun getuigenis, dat zij een ijver tot God hebben, maar niet met verstand.
\par 3 Want alzo zij de rechtvaardigheid Gods niet kennen, en hun eigen gerechtigheid zoeken op te richten, zo zijn zij der rechtvaardigheid Gods niet onderworpen.
\par 4 Want het einde der wet is Christus, tot rechtvaardigheid een iegelijk, die gelooft.
\par 5 Want Mozes beschrijft de rechtvaardigheid, die uit de wet is, zeggende: De mens, die deze dingen doet, zal door dezelve leven.
\par 6 Maar de rechtvaardigheid, die uit het geloof is, spreekt aldus: Zeg niet in uw hart: Wie zal in den hemel opklimmen? Hetzelve is Christus van boven afbrengen.
\par 7 Of, wie zal in den afgrond nederdalen? Hetzelve is Christus uit de doden opbrengen.
\par 8 Maar wat zegt zij? Nabij u is het Woord, in uw mond en in uw hart. Dit is het Woord des geloofs, hetwelk wij prediken.
\par 9 Namelijk, indien gij met uw mond zult belijden den Heere Jezus, en met uw hart geloven, dat God Hem uit de doden opgewekt heeft, zo zult gij zalig worden.
\par 10 Want met het hart gelooft men ter rechtvaardigheid en met den mond belijdt men ter zaligheid.
\par 11 Want de Schrift zegt: Een iegelijk, die in Hem gelooft, die zal niet beschaamd worden.
\par 12 Want er is geen onderscheid, noch van Jood noch van Griek; want eenzelfde is Heere van allen, rijk zijnde over allen, die Hem aanroepen.
\par 13 Want een iegelijk, die den Naam des Heeren zal aanroepen, zal zalig worden.
\par 14 Hoe zullen zij dan Hem aanroepen, in Welken zij niet geloofd hebben? En hoe zullen zij in Hem geloven, van Welken zij niet gehoord hebben? En hoe zullen zij horen, zonder die hun predikt?
\par 15 En hoe zullen zij prediken, indien zij niet gezonden worden? Gelijk geschreven is: Hoe liefelijk zijn de voeten dergenen, die vrede verkondigen, dergenen, die het goede verkondigen!
\par 16 Doch zij zijn niet allen het Evangelie gehoorzaam geweest; want Jesaja zegt: Heere, wie heeft onze prediking geloofd?
\par 17 Zo is dan het geloof uit het gehoor, en het gehoor door het Woord Gods,
\par 18 Maar ik zeg: Hebben zij het niet gehoord? Ja toch, hun geluid is over de gehele aarde uitgegaan, en hun woorden tot de einden der wereld.
\par 19 Maar ik zeg: Heeft Israel het niet verstaan? Mozes zegt eerst: Ik zal ulieden tot jaloersheid verwekken door degenen, die geen volk zijn; door een onverstandig volk zal ik u tot toorn verwekken.
\par 20 En Jesaja verstout zich, en zegt: Ik ben gevonden van degenen, die Mij niet zochten; Ik ben openbaar geworden dengenen, die naar Mij niet vraagden.
\par 21 Maar tegen Israel zegt Hij: Den gehelen dag heb Ik Mijn handen uitgestrekt tot een ongehoorzaam en tegensprekend volk.

\chapter{11}

\par 1 Ik zeg dan: Heeft God Zijn volk verstoten? Dat zij verre; want ik ben ook een Israeliet, uit het zaad Abrahams, van den stam Benjamin.
\par 2 God heeft Zijn volk niet verstoten, hetwelk Hij te voren gekend heeft. Of weet gij niet, wat de Schrift zegt van Elia, hoe hij God aanspreekt tegen Israel, zeggende:
\par 3 Heere! zij hebben Uw profeten gedood, en Uw altaren omgeworpen; en ik ben alleen overgebleven en zij zoeken mijn ziel.
\par 4 Maar wat zegt tot hem het Goddelijk antwoord? Ik heb Mijzelven nog zeven duizend mannen overgelaten, die de knie voor het beeld van Baal niet gebogen hebben.
\par 5 Alzo is er dan ook in dezen tegenwoordigen tijd een overblijfsel geworden, naar de verkiezing der genade.
\par 6 En indien het door genade is, zo is het niet meer uit de werken; anderszins is de genade geen genade meer; en indien het is uit de werken, zo is het geen genade meer; anderszins is het werk geen werk meer.
\par 7 Wat dan? Hetgeen Israel zoekt, dat heeft het niet verkregen; maar de uitverkorenen hebben het verkregen, en de anderen zijn verhard geworden.
\par 8 (Gelijk geschreven is: God heeft hun gegeven een geest des diepen slaaps; ogen om niet te zien, en oren om niet te horen) tot op den huidigen dag.
\par 9 En David zegt: Hun tafel worde tot een strik, en tot een val, en tot een aanstoot, en tot een vergelding voor hen.
\par 10 Dat hun ogen verduisterd worden, om niet te zien; en verkrom hun rug allen tijd.
\par 11 Zo zeg ik dan: Hebben zij gestruikeld, opdat zij vallen zouden? Dat zij verre; maar door hun val is de zaligheid den heidenen geworden, om hen tot jaloersheid te verwekken.
\par 12 En indien hun val de rijkdom is der wereld, en hun vermindering de rijkdom der heidenen, hoeveel te meer hun volheid!
\par 13 Want ik spreek tot u, heidenen, voor zoveel ik der heidenen apostel ben; ik maak mijn bediening heerlijk;
\par 14 Of ik enigszins mijn vlees tot jaloersheid verwekken, en enigen uit hen behouden mocht.
\par 15 Want indien hun verwerping de verzoening is der wereld, wat zal de aanneming wezen, anders dan het leven uit de doden?
\par 16 En indien de eerstelingen heilig zijn, zo is ook het deeg heilig, en indien de wortel heilig is, zo zijn ook de takken heilig.
\par 17 En zo enige der takken afgebroken zijn, en gij, een wilde olijfboom zijnde, in derzelver plaats zijt ingeent, en des wortels en der vettigheid des olijfbooms mede deelachtig zijt geworden,
\par 18 Zo roem niet tegen de takken; en indien gij daartegen roemt, gij draagt den wortel niet, maar de wortel u.
\par 19 Gij zult dan zeggen: De takken zijn afgebroken, opdat ik zou ingeent worden.
\par 20 Het is wel; zij zijn door ongeloof afgebroken, en gij staat door het geloof. Zijt niet hooggevoelende, maar vrees.
\par 21 Want is het, dat God de natuurlijke takken niet gespaard heeft, zie toe, dat Hij ook mogelijk u niet spare.
\par 22 Zie dan de goedertierenheid en de strengheid van God; de strengheid wel over degenen, die gevallen zijn, maar de goedertierenheid over u, indien gij in de goedertierenheid blijft; anderszins zult ook gij afgehouwen worden.
\par 23 Maar ook zij, indien zij in het ongeloof niet blijven, zullen ingeent worden; want God is machtig om dezelve weder in te enten.
\par 24 Want indien gij afgehouwen zijt uit den olijfboom, die van nature wild was, en tegen nature in den goeden olijfboom ingeent; hoeveel te meer zullen deze, die natuurlijke takken zijn, in hun eigen olijfboom geent worden?
\par 25 Want ik wil niet, broeders, dat u deze verborgenheid onbekend zij (opdat gij niet wijs zijt, bij uzelven), dat de verharding voor een deel over Israel gekomen is, totdat de volheid der heidenen zal ingegaan zijn.
\par 26 En alzo zal geheel Israel zalig worden; gelijk geschreven is: De Verlosser zal uit Sion komen en zal de goddeloosheden afwenden van Jakob.
\par 27 En dit is hun een verbond van Mij, als Ik hun zonden zal wegnemen.
\par 28 Zo zijn zij wel vijanden aangaande het Evangelie, om uwentwil, maar aangaande de verkiezing zijn zij beminden, om der vaderen wil;
\par 29 Want de genadegiften en de roeping Gods zijn onberouwelijk.
\par 30 Want gelijkerwijs ook gijlieden eertijds Gode ongehoorzaam geweest zijt, maar nu barmhartigheid verkregen hebt door dezer ongehoorzaamheid;
\par 31 Alzo zijn ook dezen nu ongehoorzaam geweest, opdat ook zij door uw barmhartigheid zouden barmhartigheid verkrijgen.
\par 32 Want God heeft hen allen onder de ongehoorzaamheid besloten, opdat Hij hun allen zou barmhartig zijn.
\par 33 O diepte des rijkdoms, beide der wijsheid en der kennis Gods, hoe ondoorzoekelijk zijn Zijn oordelen, en onnaspeurlijk Zijn wegen!
\par 34 Want wie heeft den zin des Heeren gekend? Of wie is Zijn raadsman geweest?
\par 35 Of wie heeft Hem eerst gegeven, en het zal hem wedervergolden worden?
\par 36 Want uit Hem, en door Hem, en tot Hem zijn alle dingen. Hem zij de heerlijkheid in der eeuwigheid. Amen.

\chapter{12}

\par 1 Ik bid u dan, broeders, door de ontfermingen Gods, dat gij uw lichamen stelt tot een levende, heilige en Gode welbehagelijke offerande, welke is uw redelijke godsdienst.
\par 2 En wordt dezer wereld niet gelijkvormig; maar wordt veranderd door de vernieuwing uws gemoeds, opdat gij moogt beproeven, welke de goede, en welbehagelijke en volmaakte wil van God zij.
\par 3 Want door de genade, die mij gegeven is, zeg ik een iegelijk, die onder u is, dat hij niet wijs zij boven hetgeen men behoort wijs te zijn; maar dat hij wijs zij tot matigheid, gelijk als God een iegelijk de mate des geloofs gedeeld heeft.
\par 4 Want gelijk wij in een lichaam vele leden hebben, en de leden alle niet dezelfde werking hebben;
\par 5 Alzo zijn wij velen een lichaam in Christus, maar elkeen zijn wij elkanders leden.
\par 6 Hebbende nu verscheidene gaven, naar de genade, die ons gegeven is,
\par 7 Zo laat ons die gaven besteden, hetzij profetie, naar de mate des geloofs; hetzij bediening, in het bedienen; hetzij die leert, in het leren;
\par 8 Hetzij die vermaant, in het vermanen; die uitdeelt, in eenvoudigheid; die een voorstander is, in naarstigheid; die barmhartigheid doet, in blijmoedigheid.
\par 9 De liefde zij ongeveinsd. Hebt een afkeer van het boze, en hangt het goede aan.
\par 10 Hebt elkander hartelijk lief met broederlijke liefde; met eer de een den ander voorgaande.
\par 11 Zijt niet traag in het benaarstigen. Zijt vurig van geest. Dient den Heere.
\par 12 Verblijdt u in de hoop. Zijt geduldig in de verdrukking. Volhardt in het gebed.
\par 13 Deelt mede tot de behoeften der heiligen. Tracht naar herbergzaamheid.
\par 14 Zegent hen, die u vervolgen; zegent en vervloekt niet.
\par 15 Verblijdt u met de blijden; en weent met de wenenden.
\par 16 Weest eensgezind onder elkander. Tracht niet naar de hoge dingen, maar voegt u tot de nederige. Zijt niet wijs bij uzelven.
\par 17 Vergeldt niemand kwaad voor kwaad. Bezorgt hetgeen eerlijk is voor alle mensen.
\par 18 Indien het mogelijk is, zoveel in u is, houdt vrede met alle mensen.
\par 19 Wreekt uzelven niet, beminden, maar geeft den toorn plaats; want er is geschreven: Mij komt de wraak toe; Ik zal het vergelden, zegt de Heere.
\par 20 Indien dan uw vijand hongert, zo spijzigt hem; indien hem dorst, zo geeft hem te drinken; want dat doende, zult gij kolen vuurs op zijn hoofd hopen.
\par 21 Wordt van het kwade niet overwonnen, maar overwint het kwade door het goede.

\chapter{13}

\par 1 Alle ziel zij den machten, over haar gesteld, onderworpen; want er is geen macht dan van God, en de machten, die er zijn, die zijn van God geordineerd.
\par 2 Alzo dat die zich tegen de macht stelt, de ordinantie van God wederstaat; en die ze wederstaan, zullen over zichzelven een oordeel halen.
\par 3 Want de oversten zijn niet tot een vreze den goeden werken, maar den kwaden. Wilt gij nu de macht niet vrezen, doe het goede, en gij zult lof van haar hebben;
\par 4 Want zij is Gods dienares, u ten goede. Maar indien gij kwaad doet, zo vrees; want zij draagt het zwaard niet te vergeefs; want zij is Gods dienares, een wreekster tot straf dengene, die kwaad doet.
\par 5 Daarom is het nodig onderworpen te zijn, niet alleen om der straffe, maar ook om des gewetens wil.
\par 6 Want daarom betaalt gij ook schattingen; want zij zijn dienaars van God, in ditzelve geduriglijk bezig zijnde.
\par 7 Zo geeft dan een iegelijk, wat gij schuldig zijt; schatting, dien gij de schatting, tol, dien gij den tol, vreze, dien gij de vreze, eer, dien gij de eer schuldig zijt.
\par 8 Zijt niemand iets schuldig, dan elkander lief te hebben; want die den ander liefheeft, die heeft de wet vervuld.
\par 9 Want dit: Gij zult geen overspel doen, gij zult niet doden, gij zult niet stelen, gij zult geen valse getuigenis geven, gij zult niet begeren; en zo er enig ander gebod is, wordt in dit woord als in een hoofdsom begrepen, namelijk in dit: Gij zult uw naaste liefhebben gelijk uzelven.
\par 10 De liefde doet den naaste geen kwaad. Zo is dan de liefde de vervulling der wet.
\par 11 En dit zeg ik te meer, dewijl wij de gelegenheid des tijds weten, dat het de ure is, dat wij nu uit den slaap opwaken; want de zaligheid is ons nu nader, dan toen wij eerst geloofd hebben.
\par 12 De nacht is voorbijgegaan, en de dag is nabij gekomen. Laat ons dan afleggen de werken der duisternis, en aandoen de wapenen des lichts.
\par 13 Laat ons, als in den dag, eerlijk wandelen; niet in brasserijen en dronkenschappen, niet in slaapkameren en ontuchtigheden, niet in twist en nijdigheid;
\par 14 Maar doet aan den Heere Jezus Christus, en verzorgt het vlees niet tot begeerlijkheden.

\chapter{14}

\par 1 Dengene nu, die zwak is in het geloof, neemt aan, maar niet tot twistige samensprekingen.
\par 2 De een gelooft wel, dat men alles eten mag, maar die zwak is, eet moeskruiden.
\par 3 Die daar eet, verachte hem niet, die niet eet; en die niet eet, oordele hem niet, die daar eet; want God heeft hem aangenomen.
\par 4 Wie zijt gij, die eens anderen huisknecht oordeelt? Hij staat, of hij valt zijn eigen heer; doch hij zal vastgesteld worden, want God is machtig hem vast te stellen.
\par 5 De een acht wel den enen dag boven den anderen dag; maar de ander acht al de dagen gelijk. Een iegelijk zij in zijn eigen gemoed ten volle verzekerd.
\par 6 Die den dag waarneemt, die neemt hem waar den Heere; en die den dag niet waarneemt, die neemt hem niet waar den Heere. Die daar eet, die eet zulks den Heere, want hij dankt God; en die niet eet, die eet zulks den Heere niet, en hij dankt God.
\par 7 Want niemand van ons leeft zichzelven, en niemand sterft zichzelven.
\par 8 Want hetzij dat wij leven, wij leven den Heere; hetzij dat wij sterven, wij sterven den Heere. Hetzij dan dat wij leven, hetzij dat wij sterven, wij zijn des Heeren.
\par 9 Want daartoe is Christus ook gestorven, en opgestaan, en weder levend geworden, opdat Hij beiden over doden en levenden heersen zou.
\par 10 Maar gij, wat oordeelt gij uw broeder? Of ook gij, wat veracht gij uw broeder? Want wij zullen allen voor den rechterstoel van Christus gesteld worden.
\par 11 Want er is geschreven: Ik leef, zegt de Heere; voor Mij zal alle knie zich buigen, en alle tong zal God belijden.
\par 12 Zo dan een iegelijk van ons zal voor zichzelven Gode rekenschap geven.
\par 13 Laat ons dan elkander niet meer oordelen; maar oordeelt dit liever, namelijk, dat gij den broeder geen aanstoot of ergernis geeft.
\par 14 Ik weet en ben verzekerd in den Heere Jezus, dat geen ding onrein is in zichzelven; dan die acht iets onrein te zijn, dien is het onrein.
\par 15 Maar indien uw broeder om der spijze wil bedroefd wordt, zo wandelt gij niet meer naar liefde. Verderf dien niet met uw spijze, voor welken Christus gestorven is.
\par 16 Dat dan uw goed niet gelasterd worde.
\par 17 Want het Koninkrijk Gods is niet spijs en drank, maar rechtvaardigheid, en vrede, en blijdschap, door den Heiligen Geest.
\par 18 Want die Christus in deze dingen dient, is Gode welbehagelijk, en aangenaam den mensen.
\par 19 Zo dan laat ons najagen, hetgeen tot den vrede, en hetgeen tot de stichting onder elkander dient.
\par 20 Verbreek het werk van God niet om der spijze wil. Alle dingen zijn wel rein; maar het is kwaad den mens, die met aanstoot eet.
\par 21 Het is goed geen vlees te eten, noch wijn te drinken, noch iets, waaraan uw broeder zich stoot, of geergerd wordt, of waarin hij zwak is.
\par 22 Hebt gij geloof? hebt dat bij uzelven voor God. Zalig is hij, die zichzelven niet oordeelt in hetgeen hij voor goed houdt.
\par 23 Maar die twijfelt, indien hij eet, is veroordeeld, omdat hij niet uit het geloof eet. En al wat uit het geloof niet is, dat is zonde.

\chapter{15}

\par 1 Maar wij, die sterk zijn, zijn schuldig de zwakheden der onsterken te dragen, en niet onszelven te behagen.
\par 2 Dat dan een iegelijk van ons zijn naaste behage ten goede, tot stichting.
\par 3 Want ook Christus heeft Zichzelven niet behaagd, maar gelijk geschreven is: De smadingen dergenen, die U smaden, zijn op Mij gevallen.
\par 4 Want al wat te voren geschreven is, dat is tot onze lering te voren geschreven, opdat wij, door lijdzaamheid en vertroosting der Schriften, hoop hebben zouden.
\par 5 Doch de God der lijdzaamheid en der vertroosting geve u, dat gij eensgezind zijt onder elkander naar Christus Jezus;
\par 6 Opdat gij eendrachtelijk, met een mond, moogt verheerlijken den God en Vader van onzen Heere Jezus Christus.
\par 7 Daarom neemt elkander aan, gelijk ook Christus ons aangenomen heeft, tot de heerlijkheid Gods.
\par 8 En ik zeg, dat Jezus Christus een dienaar geworden is der besnijdenis, vanwege de waarheid Gods, opdat Hij bevestigen zou de beloftenissen der vaderen;
\par 9 En de heidenen God vanwege de barmhartigheid zouden verheerlijken; gelijk geschreven is: Daarom zal ik U belijden onder de heidenen, en Uw Naam lofzingen.
\par 10 En wederom zegt Hij: Weest vrolijk, gij heidenen met Zijn volk!
\par 11 En wederom: Looft den Heere, al gij heidenen, en prijst Hem, al gij volken!
\par 12 En wederom zegt Jesaja: Er zal zijn de wortel van Jessai, en Die opstaat, om over de heidenen te gebieden; op Hem zullen de heidenen hopen.
\par 13 De God nu der hoop vervulle ulieden met alle blijdschap en vrede in het geloven, opdat gij overvloedig moogt zijn in de hoop, door de kracht des Heiligen Geestes.
\par 14 Doch, mijn broeders, ook ik zelf ben verzekerd van u, dat gij ook zelven vol zijt van goedheid, vervuld met alle kennis, machtig om ook elkander te vermanen.
\par 15 Maar ik heb u eensdeels te stoutelijker geschreven, broeders, u als wederom dit indachtig makende, om de genade, die mij van God gegeven is;
\par 16 Opdat ik een dienaar van Jezus Christus zij onder de heidenen, het Evangelie van God bedienende, opdat de offerande der heidenen aangenaam worde, geheiligd door den Heiligen Geest.
\par 17 Zo heb ik dan roem in Christus Jezus in die dingen, die God aangaan.
\par 18 Want ik zou niet durven iets zeggen, hetwelk Christus door mij niet gewrocht heeft, tot gehoorzaamheid der heidenen, met woorden en werken;
\par 19 Door kracht van tekenen en wonderheden, en door de kracht van den Geest Gods, zodat ik, van Jeruzalem af, en rondom, tot Illyrikum toe, het Evangelie van Christus vervuld heb.
\par 20 En alzo zeer begerig geweest ben om het Evangelie te verkondigen, niet waar Christus genoemd was, opdat ik niet op eens anders fondament zou bouwen;
\par 21 Maar gelijk geschreven is: Denwelken van Hem niet was geboodschapt, die zullen het zien; en dewelke het niet gehoord hebben, die zullen het verstaan.
\par 22 Waarom ik ook menigmaal verhinderd geweest ben tot u te komen.
\par 23 Maar nu geen plaats meer hebbende in deze gewesten, en van over vele jaren groot verlangen hebbende, om tot u te komen,
\par 24 Zo zal ik, wanneer ik naar Spanje reis, tot u komen; want ik hoop in het doorreizen u te zien, en van u derwaarts geleid te worden, als ik eerst van ulieder tegenwoordigheid eensdeels verzadigd zal zijn.
\par 25 Maar nu reis ik naar Jeruzalem, dienende de heiligen.
\par 26 Want het heeft dien van Macedonie en Achaje goed gedacht een gemene handreiking te doen aan de armen onder de heiligen, die te Jeruzalem zijn.
\par 27 Want het heeft hun zo goed gedacht; ook zijn zij hun schuldenaars; want indien de heidenen hunner geestelijke goederen deelachtig zijn geworden, zo zijn zij ook schuldig hen van lichamelijke goederen te dienen.
\par 28 Als ik dan dit volbracht, en hun deze vrucht verzegeld zal hebben, zo zal ik door ulieder stad naar Spanje afkomen.
\par 29 En ik weet, dat ik, tot u komende, met vollen zegen des Evangelies van Christus komen zal.
\par 30 En ik bid u, broeders, door onzen Heere Jezus Christus, en door de liefde des Geestes, dat gij met mij strijdt in de gebeden tot God voor mij;
\par 31 Opdat ik mag bevrijd worden van de ongehoorzamen in Judea, en dat deze mijn dienst, dien ik aan Jeruzalem doe, aangenaam zij den heiligen;
\par 32 Opdat ik met blijdschap, door den wil van God, tot u mag komen, en met u verkwikt worden.
\par 33 En de God des vredes zij met u allen. Amen.

\chapter{16}

\par 1 En ik beveel u Febe, onze zuster, die een dienares is der Gemeente, die te Kenchreen is;
\par 2 Opdat gij haar ontvangt in den Heere, gelijk het den heiligen betaamt, en haar bijstaat, in wat zaak zij u zou mogen van doen hebben; want zij is een voorstandster geweest van velen, ook van mijzelven.
\par 3 Groet Priscilla en Aquila, mijn medewerkers in Christus Jezus;
\par 4 Die voor mijn leven hun hals gesteld hebben; denwelken niet alleen ik danke, maar ook al de Gemeenten der heidenen.
\par 5 Groet ook de Gemeente in hun huis. Groet Epenetus, mijn beminde, die de eersteling is van Achaje in Christus.
\par 6 Groet Maria, die veel voor ons gearbeid heeft.
\par 7 Groet Andronikus en Junias, mijn magen, en mijn medegevangenen, welke vermaard zijn onder de apostelen, die ook voor mij in Christus geweest zijn.
\par 8 Groet Amplias, mijn beminde in den Heere.
\par 9 Groet Urbanus, onzen medearbeider in Christus, en Stachys, mijn beminde.
\par 10 Groet Apelles, die beproefd is in Christus. Groet hen, die van het huisgezin van Aristobulus zijn.
\par 11 Groet Herodion, die van mijn maagschap is. Groet hen, die van het huisgezin van Narcissus zijn, degenen namelijk, die in den Heere zijn.
\par 12 Groet Tryfena en Tryfosa, vrouwen die in den Heere arbeiden. Groet Persis, de beminde zuster, die veel gearbeid heeft in den Heere.
\par 13 Groet Rufus, den uitverkorene in den Heere, en zijn moeder en de mijne.
\par 14 Groet Asynkritus, Flegon, Hermas, Patrobas, Hermes, en de broeders, die met hen zijn.
\par 15 Groet Filologus en Julia, Nereus en zijn zuster, en Olympas, en al de heiligen, die met henlieden zijn.
\par 16 Groet elkander met een heiligen kus. De Gemeenten van Christus groeten ulieden.
\par 17 En ik bid u, broeders, neemt acht op degenen, die tweedracht en ergernissen aanrichten tegen de leer, die gij van ons geleerd hebt; en wijkt af van dezelve.
\par 18 Want dezulken dienen onzen Heere Jezus Christus niet, maar hun buik; en verleiden door schoonspreken en prijzen de harten der eenvoudigen.
\par 19 Want uw gehoorzaamheid is tot kennis van allen gekomen. Ik verblijde mij dan uwenthalve; en ik wil, dat gij wijs zijt in het goede, doch onnozel in het kwade.
\par 20 En de God des vredes zal den satan haast onder uw voeten verpletteren. De genade van onzen Heere Jezus Christus zij met ulieden. Amen.
\par 21 U groeten, Timotheus, mijn medearbeider, en Lucius, en Jason, en Socipater, mijn bloedverwanten.
\par 22 Ik, Tertius, die den brief geschreven heb, groet u in den Heere.
\par 23 U groet Gajus, de huiswaard van mij en van de gehele Gemeente. U groet Erastus, de rentmeester der stad, en de broeder Quartus.
\par 24 De genade van onzen Heere Jezus Christus zij met u allen. Amen.
\par 25 Hem nu, Die machtig is u te bevestigen, naar mijn Evangelie en de prediking van Jezus Christus, naar de openbaring der verborgenheid, die van de tijden der eeuwen verzwegen is geweest;
\par 26 Maar nu geopenbaard is, en door de profetische Schriften, naar het bevel des eeuwigen Gods, tot gehoorzaamheid des geloofs, onder al de heidenen bekend is gemaakt;
\par 27 Den zelven alleen wijzen God zij door Jezus Christus de heerlijkheid in der eeuwigheid. Amen.




\end{document}