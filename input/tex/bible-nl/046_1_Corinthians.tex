\begin{document}

\title{1 Corinthians}



\chapter{1}

\par 1 Paulus, een geroepen apostel van Jezus Christus, door den wil van God, en Sosthenes, de broeder,
\par 2 Aan de Gemeente Gods, die te Korinthe is, den geheiligden in Christus Jezus, den geroepenen heiligen, met allen, die den Naam van onzen Heere Jezus Christus aanroepen in alle plaats, beide hun en onzen Heere;
\par 3 Genade zij u en vrede van God onzen Vader, en den Heere Jezus Christus.
\par 4 Ik dank mijn God allen tijd over u, vanwege de genade Gods, die u gegeven is in Christus Jezus;
\par 5 Dat gij in alles rijk zijt geworden in Hem, in alle rede en alle kennis;
\par 6 Gelijk de getuigenis van Christus bevestigd is onder u;
\par 7 Alzo dat het u aan gene gave ontbreekt, verwachtende de openbaring van onzen Heere Jezus Christus.
\par 8 Welke God u ook zal bevestigen tot het einde toe, om onstraffelijk te zijn in den dag van onzen Heere Jezus Christus.
\par 9 God is getrouw, door Welken gij geroepen zijt tot de gemeenschap van Zijn Zoon Jezus Christus, onzen Heere.
\par 10 Maar ik bid u, broeders, door den Naam van onzen Heere Jezus Christus, dat gij allen hetzelfde spreekt, en dat onder u geen scheuringen zijn, maar dat gij samengevoegd zijt in eenzelfden zin, en in een zelfde gevoelen.
\par 11 Want mij is van u bekend gemaakt, mijn broeders, door die van het huisgezin van Chloe zijn, dat er twisten onder u zijn.
\par 12 En dit zeg ik, dat een iegelijk van u zegt: Ik ben van Paulus, en ik van Apollos; en ik van Cefas; en ik van Christus.
\par 13 Is Christus gedeeld? Is Paulus voor u gekruist? Of zijt gij in Paulus naam gedoopt?
\par 14 Ik dank God, dat ik niemand van ulieden gedoopt heb, dan Krispus en Gajus;
\par 15 Opdat niet iemand zegge, dat ik in mijn naam gedoopt heb.
\par 16 Doch ik heb ook het huisgezin van Stefanus gedoopt; voorts weet ik niet, of ik iemand anders gedoopt heb.
\par 17 Want Christus heeft mij niet gezonden, om te dopen, maar om het Evangelie te verkondigen; niet met wijsheid van woorden, opdat het kruis van Christus niet verijdeld worde.
\par 18 Want het woord des kruises is wel dengenen, die verloren gaan, dwaasheid; maar ons, die behouden worden, is het een kracht Gods;
\par 19 Want er is geschreven: Ik zal de wijsheid der wijzen doen vergaan, en het verstand der verstandigen zal Ik te niet maken.
\par 20 Waar is de wijze? Waar is de schriftgeleerde? Waar is de onderzoeker dezer eeuw? Heeft God de wijsheid dezer wereld niet dwaas gemaakt?
\par 21 Want nademaal, in de wijsheid Gods, de wereld God niet heeft gekend door de wijsheid, zo heeft het Gode behaagd, door de dwaasheid der prediking, zalig te maken, die geloven;
\par 22 Overmits de Joden een teken begeren, en de Grieken wijsheid zoeken;
\par 23 Doch wij prediken Christus, den Gekruisigde, den Joden wel een ergernis, en den Grieken een dwaasheid;
\par 24 Maar hun, die geroepen zijn, beiden Joden en Grieken, prediken wij Christus, de kracht Gods, en de wijsheid Gods.
\par 25 Want het dwaze Gods is wijzer dan de mensen; en het zwakke Gods is sterker dan de mensen.
\par 26 Want gij ziet uw roeping, broeders, dat gij niet vele wijzen zijt naar het vlees, niet vele machtigen, niet vele edelen.
\par 27 Maar het dwaze der wereld heeft God uitverkoren, opdat Hij de wijzen beschamen zou; en het zwakke der wereld heeft God uitverkoren, opdat Hij het sterke zou beschamen;
\par 28 En het onedele der wereld, en het verachte heeft God uitverkoren, en hetgeen niets is, opdat Hij hetgeen iets is, te niet zou maken;
\par 29 Opdat geen vlees zou roemen voor Hem.
\par 30 Maar uit Hem zijt gij in Christus Jezus, Die ons geworden is wijsheid van God, en rechtvaardigheid, en heiligmaking, en verlossing;
\par 31 Opdat het zij, gelijk geschreven is: Die roemt, roeme in den Heere.

\chapter{2}

\par 1 En ik, broeders, als ik tot u ben gekomen, ben niet gekomen met uitnemendheid van woorden, of van wijsheid, u verkondigende de getuigenis van God.
\par 2 Want ik heb niet voorgenomen iets te weten onder u, dan Jezus Christus, en Dien gekruisigd.
\par 3 En ik was bij ulieden in zwakheid, en in vreze, en in vele beving.
\par 4 En mijn rede, en mijn prediking was niet in bewegelijke woorden der menselijke wijsheid, maar in betoning des geestes en der kracht;
\par 5 Opdat uw geloof niet zou zijn in wijsheid der mensen, maar in de kracht Gods.
\par 6 En wij spreken wijsheid onder de volmaakten; doch een wijsheid, niet dezer wereld, noch der oversten dezer wereld, die te niet worden;
\par 7 Maar wij spreken de wijsheid Gods, bestaande in verborgenheid, die bedekt was, welke God te voren verordineerd heeft tot heerlijkheid van ons, eer de wereld was;
\par 8 Welke niemand van de oversten dezer wereld gekend heeft; want indien zij ze gekend hadden, zo zouden zij den Heere der heerlijkheid niet gekruist hebben.
\par 9 Maar gelijk geschreven is: Hetgeen het oog niet heeft gezien, en het oor niet heeft gehoord, en in het hart des mensen niet is opgeklommen, hetgeen God bereid heeft dien, die Hem liefhebben.
\par 10 Doch God heeft het ons geopenbaard door Zijn Geest; want de Geest onderzoekt alle dingen, ook de diepten Gods.
\par 11 Want wie van de mensen weet, hetgeen des mensen is, dan de geest des mensen, die in hem is? Alzo weet ook niemand, hetgeen Gods is, dan de Geest Gods.
\par 12 Doch wij hebben niet ontvangen den geest der wereld, maar den Geest, Die uit God is, opdat wij zouden weten de dingen, die ons van God geschonken zijn;
\par 13 Dewelke wij ook spreken, niet met woorden, die de menselijke wijsheid leert, maar met woorden, die de Heilige Geest leert, geestelijke dingen met geestelijke samenvoegende.
\par 14 Maar de natuurlijke mens begrijpt niet de dingen, die des Geestes Gods zijn; want zij zijn hem dwaasheid, en hij kan ze niet verstaan, omdat zij geestelijk onderscheiden worden.
\par 15 Doch de geestelijke mens onderscheidt wel alle dingen, maar hij zelf wordt van niemand onderscheiden.
\par 16 Want wie heeft den zin des Heeren gekend, die Hem zou onderrichten? Maar wij hebben den zin van Christus.

\chapter{3}

\par 1 En ik, broeders, kon tot u niet spreken als tot geestelijken, maar als tot vleselijken, als tot jonge kinderen in Christus.
\par 2 Ik heb u met melk gevoed, en niet met vaste spijs; want gij vermocht toen nog niet; ja, gij vermoogt ook nu nog niet.
\par 3 Want gij zijt nog vleselijk; want dewijl onder u nijd is, en twist, en tweedracht, zijt gij niet vleselijk, en wandelt gij niet naar den mens?
\par 4 Want als de een zegt: Ik ben van Paulus; en een ander: Ik ben van Apollos; zijt gij niet vleselijk?
\par 5 Wie is dan Paulus, en wie is Apollos, anders dan dienaars, door welke gij geloofd hebt, en dat, gelijk de Heere aan een iegelijk gegeven heeft?
\par 6 Ik heb geplant, Apollos heeft nat gemaakt; maar God heeft den wasdom gegeven.
\par 7 Zo is dan noch hij, die plant, iets, noch hij, die nat maakt, maar God, Die den wasdom geeft.
\par 8 En die plant, en die nat maakt, zijn een; maar een iegelijk zal zijn loon ontvangen naar zijn arbeid.
\par 9 Want wij zijn Gods medearbeiders; Gods akkerwerk, Gods gebouw zijt gij.
\par 10 Naar de genade Gods, die mij gegeven is, heb ik als een wijs bouwmeester het fondament gelegd; en een ander bouwt daarop. Maar een iegelijk zie toe, hoe hij daarop bouwe.
\par 11 Want niemand kan een ander fondament leggen, dan hetgeen gelegd is, hetwelk is Jezus Christus.
\par 12 En indien iemand op dit fondament bouwt: goud, zilver, kostelijke stenen, hout, hooi, stoppelen;
\par 13 Eens iegelijks werk zal openbaar worden; want de dag zal het verklaren, dewijl het door vuur ontdekt wordt; en hoedanig eens iegelijks werk is, zal het vuur beproeven.
\par 14 Zo iemands werk blijft, dat hij daarop gebouwd heeft, die zal loon ontvangen.
\par 15 Zo iemands werk zal verbrand worden, die zal schade lijden; maar zelf zal hij behouden worden, doch alzo als door vuur.
\par 16 Weet gij niet, dat gij Gods tempel zijt, en de Geest Gods in ulieden woont?
\par 17 Zo iemand den tempel Gods schendt, dien zal God schenden; want de tempel Gods is heilig, welke gij zijt.
\par 18 Niemand bedriege zichzelven. Zo iemand onder u dunkt, dat hij wijs is in deze wereld, die worde dwaas, opdat hij wijs moge worden.
\par 19 Want de wijsheid dezer wereld is dwaasheid bij God; want er is geschreven: Hij vat de wijzen in hun arglistigheid;
\par 20 En wederom: De Heere kent de overleggingen der wijzen, dat zij ijdel zijn.
\par 21 Niemand dan roeme op mensen; want alles is uwe.
\par 22 Hetzij Paulus, hetzij Apollos, hetzij Cefas, hetzij de wereld, hetzij leven, hetzij dood, hetzij tegenwoordige, hetzij toekomende dingen, zij zijn alle uwe.
\par 23 Doch gij zijt van Christus, en Christus is Gods.

\chapter{4}

\par 1 Alzo houde ons een ieder mens, als dienaars van Christus, en uitdelers der verborgenheden Gods.
\par 2 En voorts wordt in de uitdelers vereist, dat elk getrouw bevonden worde.
\par 3 Doch mij is voor het minste, dat ik van ulieden geoordeeld worde, of van een menselijk oordeel; ja, ik oordeel ook mijzelven niet.
\par 4 Want ik ben mijzelven van geen ding bewust; doch ik ben daardoor niet gerechtvaardigd; maar Die mij oordeelt, is de Heere.
\par 5 Zo dan oordeelt niets voor den tijd, totdat de Heere zal gekomen zijn, Welke ook in het licht zal brengen, hetgeen in de duisternis verborgen is, en openbaren de raadslagen der harten; en als dan zal een iegelijk lof hebben van God.
\par 6 En deze dingen, broeders, heb ik op mijzelven en Apollos bij gelijkenis toegepast, om uwentwil; opdat gij aan ons zoudt leren, niet te gevoelen boven hetgeen geschreven is, dat gij niet, de een om eens anders wil, opgeblazen wordt tegen den ander.
\par 7 Want wie onderscheidt u? En wat hebt gij, dat gij niet hebt ontvangen? En zo gij het ook ontvangen hebt, wat roemt gij, alsof gij het niet ontvangen hadt?
\par 8 Alrede zijt gij verzadigd, alrede zijt gij rijk geworden, zonder ons hebt gij geheerst; en och, of gij heerstet, opdat ook wij met u heersen mochten!
\par 9 Want ik acht, dat God ons, die de laatste apostelen zijn, ten toon heeft gesteld als tot den dood verwezen; want wij zijn een schouwspel geworden der wereld, en den engelen, en den mensen.
\par 10 Wij zijn dwazen om Christus' wil, maar gij zijt wijzen in Christus; wij zijn zwakken, maar gij sterken; gij zijt heerlijken, maar wij verachten.
\par 11 Tot op deze tegenwoordige ure lijden wij honger, en lijden wij dorst, en zijn naakt, en worden met vuisten geslagen, en hebben geen vaste woonplaats;
\par 12 En arbeiden, werkende met onze eigen handen; wij worden gescholden, en wij zegenen; wij worden vervolgd, en wij verdragen;
\par 13 Wij worden gelasterd, en wij bidden; wij zijn geworden als uitvaagsels der wereld en aller afschrapsel tot nu toe.
\par 14 Ik schrijf deze dingen niet om u te beschamen, maar als mijn lieve kinderen vermaan ik u.
\par 15 Want al hadt gij tien duizend leermeesters in Christus, zo hebt gij toch niet vele vaders; want in Christus Jezus heb ik u door het Evangelie geteeld.
\par 16 Zo vermaan ik u dan: zijt mijn navolgers.
\par 17 Daarom heb ik Timotheus tot u gezonden, die mijn lieve en getrouwe zoon is in den Heere, welke u zal indachtig maken mijn wegen, die in Christus zijn, gelijkerwijs ik alom in alle Gemeenten leer.
\par 18 Doch sommigen zijn opgeblazen, alsof ik tot ulieden niet komen zou.
\par 19 Maar ik zal haast tot u komen, zo de Heere wil, en ik zal dan verstaan, niet de woorden dergenen, die opgeblazen zijn, maar de kracht.
\par 20 Want het Koninkrijk Gods is niet gelegen in woorden, maar in kracht.
\par 21 Wat wilt gij? Zal ik met de roede tot u komen, of in liefde en in den geest der zachtmoedigheid?

\chapter{5}

\par 1 Men hoort ganselijk, dat er hoererij onder u is, en zodanige hoererij, die ook onder de heidenen niet genoemd wordt, alzo dat er een zijns vaders huisvrouw heeft.
\par 2 En zijt gij nog opgeblazen, en hebt niet veel meer leed gedragen, opdat hij uit het midden van u weggedaan worde, die deze daad begaan heeft?
\par 3 Doch ik, als wel met het lichaam afwezend, maar tegenwoordig zijnde met den geest, heb alrede, als of ik tegenwoordig ware, dengene, die dat alzo bedreven heeft, besloten,
\par 4 In den Naam van onzen Heere Jezus Christus, als gijlieden en mijn geest samen vergaderd zullen zijn, met de kracht van onzen Heere Jezus Christus,
\par 5 Denzulken over te geven aan den satan, tot verderf des vleses, opdat de geest behouden moge worden in den dag van den Heere Jezus.
\par 6 Uw roem is niet goed. Weet gij niet, dat een weinig zuurdesem het gehele deeg zuur maakt?
\par 7 Zuivert dan den ouden zuurdesem uit, opdat gij een nieuw deeg zijn moogt, gelijk gij ongezuurd zijt. Want ook ons Pascha is voor ons geslacht, namelijk Christus.
\par 8 Zo dan laat ons feest houden, niet in den ouden zuurdesem, noch in den zuurdesem der kwaadheid en der boosheid, maar in de ongezuurde broden der oprechtheid en der waarheid.
\par 9 Ik heb u geschreven in den brief, dat gij u niet zoudt vermengen met de hoereerders;
\par 10 Doch niet geheellijk met de hoereerders dezer wereld, of met de gierigaards, of met de rovers, of met de afgodendienaars; want anders zoudt gij moeten uit de wereld gaan.
\par 11 Maar nu heb ik u geschreven, dat gij u niet zult vermengen, namelijk indien iemand, een broeder genaamd zijnde, een hoereerder is, of een gierigaard, of een afgodendienaar, of een lasteraar, of een dronkaard, of een rover; dat gij met zodanig een ook niet zult eten.
\par 12 Want wat heb ik ook die buiten zijn te oordelen? Oordeelt gijlieden niet die binnen zijn?
\par 13 Maar die buiten zijn oordeelt God. En doet gij dezen boze uit ulieden weg.

\chapter{6}

\par 1 Durft iemand van ulieden, die een zaak heeft tegen een ander, te recht gaan voor de onrechtvaardigen, en niet voor de heiligen?
\par 2 Weet gij niet, dat de heiligen de wereld oordelen zullen? En indien door u de wereld geoordeeld wordt, zijt gij onwaardig de minste gerechtzaken?
\par 3 Weet gij niet, dat wij de engelen oordelen zullen? Hoeveel te meer de zaken, die dit leven aangaan?
\par 4 Zo gij dan gerechtzaken hebt, die dit leven aangaan, zet die daarover, die in de Gemeente minst geacht zijn.
\par 5 Ik zeg u dit tot schaamte. Is er dan alzo onder u geen, die wijs is, ook niet een, die zou kunnen oordelen tussen zijn broeders?
\par 6 Maar de ene broeder gaat met den anderen broeder te recht, en dat voor ongelovigen.
\par 7 Zo is er dan nu ganselijk gebrek onder u, dat gij met elkander rechtzaken hebt. Waarom lijdt gij niet liever ongelijk? Waarom lijdt gij niet liever schade?
\par 8 Maar gijlieden doet ongelijk, en doet schade, en dat den broederen.
\par 9 Of weet gij niet, dat de onrechtvaardigen het Koninkrijk Gods niet zullen beerven?
\par 10 Dwaalt niet; noch hoereerders, noch afgodendienaars, noch overspelers, noch ontuchtigen, noch die bij mannen liggen, noch dieven, noch gierigaards, noch dronkaards, geen lasteraars, geen rovers zullen het Koninkrijk Gods beerven.
\par 11 En dit waart gij sommigen; maar gij zijt afgewassen, maar gij zijt geheiligd, maar gij zijt gerechtvaardigd, in den Naam van den Heere Jezus, en door den Geest onzes Gods;
\par 12 Alle dingen zijn mij geoorloofd, maar alle dingen zijn niet oorbaar; alle dingen zijn mij geoorloofd, maar ik zal onder de macht van geen mij laten brengen.
\par 13 De spijzen zijn voor de buik, en de buik is voor de spijzen; maar God zal beide dezen en die te niet doen. Doch het lichaam is niet voor de hoererij, maar voor den Heere en de Heere voor het lichaam.
\par 14 En God heeft ook den Heere opgewekt, en zal ons opwekken door Zijn kracht.
\par 15 Weet gij niet, dat uw lichamen leden van Christus zijn? Zal ik dan de leden van Christus nemen, en maken ze leden ener hoer? Dat zij verre.
\par 16 Of weet gij niet, dat die de hoer aanhangt, een lichaam met haar is? Want die twee, zegt Hij, zullen tot een vlees wezen.
\par 17 Maar die den Heere aanhangt, is een geest met Hem.
\par 18 Vliedt de hoererij. Alle zonde, die de mens doet, is buiten het lichaam, maar die hoererij bedrijft, zondigt tegen zijn eigen lichaam.
\par 19 Of weet gij niet, dat ulieder lichaam een tempel is van den Heiligen Geest, Die in u is, Dien gij van God hebt, en dat gij uws zelfs niet zijt?
\par 20 Want gij zijt duur gekocht: zo verheerlijkt dan God in uw lichaam en in uw geest, welke Godes zijn.

\chapter{7}

\par 1 Aangaande nu de dingen, waarvan gij mij geschreven hebt; het is een mens goed geen vrouw aan te raken.
\par 2 Maar om der hoererijen wil zal een iegelijk man zijn eigen vrouw hebben, en een iegelijke vrouw zal haar eigen man hebben.
\par 3 De man zal aan de vrouw de schuldige goedwilligheid betalen; en desgelijks ook de vrouw aan den man.
\par 4 De vrouw heeft de macht niet over haar eigen lichaam, maar de man; en desgelijks ook de man heeft de macht niet over zijn eigen lichaam, maar de vrouw.
\par 5 Onttrekt u elkander niet, tenzij dan met beider toestemming voor een tijd, opdat gij u tot vasten en bidden moogt verledigen; en komt wederom bijeen, opdat u de satan niet verzoeke, omdat gij u niet kunt onthouden.
\par 6 Doch dit zeg ik uit toelating, niet uit bevel.
\par 7 Want ik wilde, dat alle mensen waren, gelijk als ikzelf ben; maar een iegelijk heeft zijn eigen gave van God, de een wel aldus, maar de andere alzo.
\par 8 Doch ik zeg den ongetrouwden, en den weduwen: Het is hun goed, indien zij blijven, gelijk als ik.
\par 9 Maar indien zij zich niet kunnen onthouden, dat zij trouwen; want het is beter te trouwen dan te branden.
\par 10 Doch den getrouwden gebiede niet ik, maar de Heere, dat de vrouw van den man niet scheide.
\par 11 En indien zij ook scheidt, dat zij ongetrouwd blijve, of met den man verzoene; en dat de man de vrouw niet verlate.
\par 12 Maar den anderen zeg ik, niet de Heere: Indien enig broeder een ongelovige vrouw heeft, en dezelve tevreden is bij hem te wonen, dat hij ze niet verlate;
\par 13 En een vrouw, die een ongelovigen man heeft, en hij tevreden is bij haar te wonen, dat zij hem niet verlate.
\par 14 Want de ongelovige man is geheiligd door de vrouw, en de ongelovige vrouw is geheiligd door den man; want anders waren uw kinderen onrein, maar nu zijn zij heilig.
\par 15 Maar indien de ongelovige scheidt, dat hij scheide. De broeder of de zuster wordt in zodanige gevallen niet dienstbaar gemaakt; maar God heeft ons tot vrede geroepen.
\par 16 Want wat weet gij, vrouw, of gij den man zult zalig maken? Of wat weet gij, man, of gij de vrouw zult zalig maken?
\par 17 Doch gelijk God aan een iegelijk heeft uitgedeeld, gelijk de Heere een iegelijk geroepen heeft, dat hij alzo wandele; en alzo verordene ik in al de Gemeenten.
\par 18 Is iemand, besneden zijnde, geroepen, die late zich geen voorhuid aantrekken; is iemand, in de voorhuid zijnde, geroepen, die late zich niet besnijden.
\par 19 De besnijdenis is niets, en de voorhuid is niets, maar de onderhouding der geboden Gods.
\par 20 Een iegelijk blijve in die beroeping, daar hij in geroepen is.
\par 21 Zijt gij, een dienstknecht zijnde, geroepen, laat u dat niet bekommeren; maar indien gij ook kunt vrij worden, gebruik dat liever.
\par 22 Want die in den Heere geroepen is, een dienstknecht zijnde, die is een vrijgelatene des Heeren; desgelijks ook, die vrij zijnde geroepen is, die is een dienstknecht van Christus.
\par 23 Gij zijt duur gekocht, wordt geen dienstknechten der mensen.
\par 24 Een iegelijk, waarin hij geroepen is, broeders, die blijve in hetzelve bij God.
\par 25 Aangaande de maagden nu, heb ik geen bevel des Heeren; maar ik zeg mijn gevoelen, als die barmhartigheid van den Heere gekregen heb, om getrouw te zijn.
\par 26 Ik houde dan dit goed te zijn, om den aanstaanden nood, dat het, zeg ik, den mens goed is alzo te zijn.
\par 27 Zijt gij aan een vrouw verbonden, zoek geen ontbinding; zijt gij ongebonden van een vrouw, zoek geen vrouw.
\par 28 Maar indien gij ook trouwt, gij zondigt niet; en indien een maagd trouwt, zij zondigt niet. Doch dezulken zullen verdrukking hebben in het vlees; en ik spare ulieden.
\par 29 Maar dit zeg ik, broeders, dat de tijd voorts kort is; opdat ook die vrouwen hebben, zouden zijn als niet hebbende;
\par 30 En die wenen, als niet wenende; en die blijde zijn, als niet blijde zijnde; en die kopen, als niet bezittende;
\par 31 En die deze wereld gebruiken, als niet misbruikende; want de gedaante dezer wereld gaat voorbij.
\par 32 En ik wil, dat gij zonder bekommernis zijt. De ongetrouwde bekommert zich met de dingen des Heeren, hoe hij den Heere zal behagen;
\par 33 Maar die getrouwd is, bekommert zich met de dingen der wereld, hoe hij de vrouw zal behagen.
\par 34 Een vrouw en een maagd zijn onderscheiden. De ongetrouwde bekommert zich met de dingen des Heeren, opdat zij heilig zij, beide aan lichaam en aan geest; maar die getrouwd is, bekommert zich met de dingen der wereld, hoe zij den man zal behagen.
\par 35 En dit zeg ik tot uw eigen voordeel; niet opdat ik een strik over u zou werpen, maar om u te leiden tot hetgeen wel voegt, en bekwaam is, om den Heere wel aan te hangen, zonder herwaarts en derwaarts getrokken te worden.
\par 36 Maar zo iemand acht, dat hij ongevoegelijk handelt met zijn maagd, indien zij over den jeugdigen tijd gaat, en het alzo moet geschieden; die doe wat hij wil, hij zondigt niet; dat zij trouwen.
\par 37 Doch die vast staat in zijn hart, geen noodzaak hebbende, maar macht heeft over zijn eigen wil, en dit in zijn hart besloten heeft, dat hij zijn maagd zal bewaren, die doet wel.
\par 38 Alzo dan, die haar ten huwelijk uitgeeft, die doet wel; en die ze ten huwelijk niet uitgeeft, die doet beter.
\par 39 Een vrouw is door de wet verbonden, zo langen tijd haar man leeft; maar indien haar man ontslapen is, zo is zij vrij, om te trouwen, dien zij wil, alleenlijk in den Heere.
\par 40 Maar zij is gelukkiger, indien zij alzo blijft, naar mijn gevoelen. En ik meen ook den Geest Gods te hebben.

\chapter{8}

\par 1 Aangaande nu de dingen, die den afgoden geofferd zijn, wij weten, dat wij allen te zamen kennis hebben. De kennis maakt opgeblazen, maar de liefde sticht.
\par 2 En zo iemand meent iets te weten, die heeft nog niets gekend, gelijk men behoort te kennen.
\par 3 Maar zo iemand God liefheeft, die is van Hem gekend.
\par 4 Aangaande dan het eten der dingen, die den afgoden geofferd zijn, wij weten, dat een afgod niets is in de wereld, en dat er geen ander God is dan een.
\par 5 Want hoewel er ook zijn, die goden genaamd worden, hetzij in den hemel, hetzij op de aarde (gelijk er vele goden en vele heren zijn),
\par 6 Nochtans hebben wij maar een God, den Vader, uit Welken alle dingen zijn, en wij tot Hem; en maar een Heere, Jezus Christus, door Welken alle dingen zijn, en wij door Hem.
\par 7 Doch in allen is de kennis niet; maar sommigen, met een geweten des afgods tot nog toe, eten als iets dat den afgoden geofferd is; en hun geweten, zwak zijnde, wordt bevlekt.
\par 8 De spijze nu maakt ons Gode niet aangenaam; want hetzij dat wij eten, wij hebben geen overvloed; en hetzij dat wij niet eten, wij hebben geen gebrek.
\par 9 Maar ziet toe, dat deze uw macht niet enigerwijze een aanstoot worde dengenen, die zwak zijn.
\par 10 Want zo iemand u, die de kennis hebt, ziet in der afgoden tempel aanzitten, zal het geweten deszelven, die zwak is, niet gestijfd worden, om te eten de dingen, die den afgoden geofferd zijn?
\par 11 En zal de broeder, die zwak is, door uw kennis verloren gaan, om welken Christus gestorven is?
\par 12 Doch gijlieden, alzo tegen de broeders zondigende, en hun zwak geweten kwetsende, zondigt tegen Christus.
\par 13 Daarom, indien de spijs mijn broeder ergert, zo zal ik in eeuwigheid geen vlees eten, opdat ik mijn broeder niet ergere.

\chapter{9}

\par 1 Ben ik niet een apostel? Ben ik niet vrij? Heb ik niet Jezus Christus, onzen Heere, gezien? Zijt gijlieden niet mijn werk in den Heere?
\par 2 Zo ik anderen geen apostel ben, nochtans ben ik het ulieden; want het zegel mijns apostelschaps zijt gijlieden in den Heere.
\par 3 Mijn verantwoording aan degenen, die onderzoek over mij doen, is deze.
\par 4 Hebben wij niet macht, om te eten en te drinken?
\par 5 Hebben wij niet macht, om een vrouw, een zuster zijnde, met ons om te leiden, gelijk ook de andere apostelen, en de broeders des Heeren, en Cefas?
\par 6 Of hebben alleen ik en Barnabas geen macht van niet te werken?
\par 7 Wie dient ooit in den krijg op eigen bezoldiging? Wie plant een wijngaard, en eet niet van zijn vrucht? Of wie weidt een kudde, en eet niet van de melk der kudde?
\par 8 Spreek ik dit naar den mens, of zegt ook de wet hetzelfde niet?
\par 9 Want in de wet van Mozes is geschreven: Gij zult een dorsenden os niet muilbanden. Zorgt ook God voor de ossen?
\par 10 Of zegt Hij dat ganselijk om onzentwil? Want om onzentwil is dat geschreven; overmits die ploegt, op hoop moet ploegen, en die op hoop dorst, moet zijn hoop deelachtig worden.
\par 11 Indien wij ulieden het geestelijke gezaaid hebben, is het een grote zaak, zo wij het uwe, dat lichamelijk is, maaien?
\par 12 Indien anderen deze macht over u deelachtig zijn, waarom niet veel meer wij? Doch wij hebben deze macht niet gebruikt, maar wij verdragen het al, opdat wij niet enige verhindering geven aan het Evangelie van Christus.
\par 13 Weet gij niet, dat degenen, die de heilige dingen bedienen, van het heilige eten? en die steeds bij het altaar zijn, met het altaar delen?
\par 14 Alzo heeft ook de Heere geordineerd dengenen, die het Evangelie verkondigen, dat zij van het Evangelie leven.
\par 15 Maar ik heb geen van deze dingen gebruikt. En ik heb dit niet geschreven, opdat het alzo aan mij geschieden zou; want het ware mij beter te sterven, dan dat iemand dezen mijn roem zou ijdel maken.
\par 16 Want indien ik het Evangelie verkondige, het is mij geen roem; want de nood is mij opgelegd. En wee mij, indien ik het Evangelie niet verkondig!
\par 17 Want indien ik dat gewillig doe, zo heb ik loon, maar indien onwillig, de uitdeling is mij evenwel toebetrouwd.
\par 18 Wat loon heb ik dan? Namelijk dat ik, het Evangelie verkondigende, het Evangelie van Christus kosteloos stelle, om mijn macht in het Evangelie niet te misbruiken.
\par 19 Want daar ik van allen vrij was, heb ik mijzelven allen dienstbaar gemaakt, opdat ik er meer zou winnen.
\par 20 En ik ben den Joden geworden als een Jood, opdat ik de Joden winnen zou; dengenen, die onder de wet zijn, ben ik geworden als onder de wet zijnde, opdat ik degenen, die onder de wet zijn, winnen zou.
\par 21 Degenen, die zonder de wet zijn, ben ik geworden als zonder de wet zijnde (Gode nochtans zijnde niet zonder de wet, maar voor Christus onder de wet), opdat ik degenen, die zonder de wet zijn, winnen zou.
\par 22 Ik ben den zwakken geworden als een zwakke, opdat ik de zwakken winnen zou; allen ben ik alles geworden, opdat ik immers enigen behouden zou.
\par 23 En dit doe ik om des Evangelies wil, opdat ik hetzelve mede deelachtig zou worden.
\par 24 Weet gijlieden niet, dat die in de loopbaan lopen, allen wel lopen, maar dat een den prijs ontvangt? Loopt alzo, dat gij dien moogt verkrijgen.
\par 25 En een iegelijk, die om prijs strijdt, onthoudt zich in alles. Dezen dan doen wel dit, opdat zij een verderfelijke kroon zouden ontvangen, maar wij een onverderfelijke.
\par 26 Ik loop dan alzo, niet als op het onzekere; ik kamp alzo, niet als de lucht slaande;
\par 27 Maar ik bedwing mijn lichaam, en breng het tot dienstbaarheid, opdat ik niet enigszins, daar ik anderen gepredikt heb, zelf verwerpelijk worde.

\chapter{10}

\par 1 En ik wil niet, broeders, dat gij onwetende zijt, dat onze vaders allen onder de wolk waren, en allen door de zee doorgegaan zijn;
\par 2 En allen in Mozes gedoopt zijn in de wolk en in de zee;
\par 3 En allen dezelfde geestelijke spijs gegeten hebben;
\par 4 En allen denzelfden geestelijken drank gedronken hebben; want zij dronken uit de geestelijke steenrots, die volgde; en de steenrots was Christus.
\par 5 Maar in het meerder deel van hen heeft God geen welgevallen gehad; want zij zijn in de woestijn ter nedergeslagen.
\par 6 En deze dingen zijn geschied ons tot voorbeelden, opdat wij geen lust tot het kwaad zouden hebben, gelijkerwijs als zij lust gehad hebben.
\par 7 En wordt geen afgodendienaars, gelijkerwijs als sommigen van hen, gelijk geschreven staat: Het volk zat neder om te eten, en om te drinken, en zij stonden op om te spelen.
\par 8 En laat ons niet hoereren, gelijk sommigen van hen gehoereerd hebben, en er vielen op een dag drie en twintig duizend.
\par 9 En laat ons Christus niet verzoeken, gelijk ook sommigen van hen verzocht hebben, en werden van de slagen vernield.
\par 10 En murmureert niet, gelijk ook sommigen van hen gemurmureerd hebben, en werden vernield van den verderver.
\par 11 En deze dingen alle zijn hunlieden overkomen tot voorbeelden; en zijn beschreven tot waarschuwing van ons, op dewelke de einden der eeuwen gekomen zijn.
\par 12 Zo dan, die meent te staan, zie toe, dat hij niet valle.
\par 13 Ulieden heeft geen verzoeking bevangen dan menselijke; doch God is getrouw, Die u niet zal laten verzocht worden boven hetgeen gij vermoogt; maar Hij zal met de verzoeking ook de uitkomst geven, opdat gij ze kunt verdragen.
\par 14 Daarom, mijn geliefden, vliedt van den afgodendienst.
\par 15 Als tot verstandigen spreek ik; oordeelt gij, hetgeen ik zeg.
\par 16 De drinkbeker der dankzegging, dien wij dankzeggende zegenen, is die niet een gemeenschap des bloeds van Christus? Het brood, dat wij breken, is dat niet een gemeenschap des lichaams van Christus?
\par 17 Want een brood is het, zo zijn wij velen een lichaam, dewijl wij allen eens broods deelachtig zijn.
\par 18 Ziet Israel, dat naar het vlees is; hebben niet degenen, die de offeranden eten, gemeenschap met het altaar?
\par 19 Wat zeg ik dan? Dat een afgod iets is, of dat het afgodenoffer iets is?
\par 20 Ja, ik zeg, dat hetgeen de heidenen offeren, zij den duivelen offeren, en niet Gode; en ik wil niet, dat gij met de duivelen gemeenschap hebt.
\par 21 Gij kunt den drinkbeker des Heeren niet drinken, en den drinkbeker der duivelen; gij kunt niet deelachtig zijn aan de tafel des Heeren, en aan de tafel der duivelen.
\par 22 Of tergen wij den Heere? Zijn wij sterker dan Hij?
\par 23 Alle dingen zijn mij geoorloofd, maar alle dingen zijn niet oorbaar; alle dingen zijn mij geoorloofd, maar alle dingen stichten niet.
\par 24 Niemand zoeke dat zijns zelfs is; maar een iegelijk zoeke dat des anderen is.
\par 25 Eet al wat in het vleeshuis verkocht wordt, niets ondervragende, om des gewetens wil;
\par 26 Want de aarde is des Heeren, en de volheid derzelve.
\par 27 En indien u iemand van de ongelovigen noodt, en gij daar gaan wilt, eet al wat ulieden voorgesteld wordt, niets ondervragende, om des gewetens wil.
\par 28 Maar zo iemand tot ulieden zegt: Dat is afgodenoffer; eet het niet, om desgenen wil, die u dat te kennen gegeven heeft, en om des gewetens wil. Want de aarde is des Heeren, en de volheid derzelve.
\par 29 Doch ik zeg: om het geweten, niet van uzelven, maar des anderen; want waarom wordt mijn vrijheid geoordeeld van een ander geweten?
\par 30 En indien ik door genade der spijze deelachtig ben, waarom word ik gelasterd over hetgeen, waarvoor ik dankzeg?
\par 31 Hetzij dan dat gijlieden eet, hetzij dat gij drinkt, hetzij dat gij iets anders doet, doet het al ter ere Gods.
\par 32 Weest zonder aanstoot te geven, en den Joden, en den Grieken, en der Gemeente Gods.
\par 33 Gelijkerwijs ik ook in alles allen behaag, niet zoekende mijn eigen voordeel, maar het voordeel van velen, opdat zij mochten behouden worden.

\chapter{11}

\par 1 Weest mijn navolgers, gelijkerwijs ook ik van Christus.
\par 2 En ik prijs u, broeders, dat gij in alles mijner gedachtig zijt, en de inzettingen behoudt, gelijk ik die u overgegeven heb.
\par 3 Doch ik wil, dat gij weet, dat Christus het Hoofd is eens iegelijken mans, en de man het hoofd der vrouw, en God het Hoofd van Christus.
\par 4 Een iegelijk man, die bidt of profeteert, hebbende iets op het hoofd, die onteert zijn eigen hoofd;
\par 5 Maar een iegelijke vrouw, die bidt of profeteert met ongedekten hoofde, onteert haar eigen hoofd; want het is een en hetzelfde, alsof haar het haar afgesneden ware.
\par 6 Want indien een vrouw niet gedekt is, dat zij ook geschoren worde; maar indien het lelijk is voor een vrouw geschoren te zijn, of het haar afgesneden te hebben, dat zij zich dekke.
\par 7 Want de man moet het hoofd niet dekken, overmits hij het beeld en de heerlijkheid Gods is; maar de vrouw is de heerlijkheid des mans.
\par 8 Want de man is uit de vrouw niet, maar de vrouw is uit den man.
\par 9 Want ook is de man niet geschapen om de vrouw, maar de vrouw om den man.
\par 10 Daarom moet de vrouw een macht op het hoofd hebben, om der engelen wil.
\par 11 Nochtans is noch de man zonder de vrouw, noch de vrouw zonder den man, in den Heere.
\par 12 Want gelijkerwijs de vrouw uit den man is, alzo is ook de man door de vrouw; doch alle dingen zijn uit God.
\par 13 Oordeelt gij onder uzelven: is het betamelijk, dat de vrouw ongedekt God bidde?
\par 14 Of leert u ook de natuur zelve niet, dat zo een man lang haar draagt, het hem een oneer is?
\par 15 Maar zo een vrouw lang haar draagt, dat het haar een eer is; omdat het lange haar voor een deksel haar is gegeven?
\par 16 Doch indien iemand schijnt twistgierig te zijn, wij hebben zulke gewoonten niet, noch de Gemeenten Gods.
\par 17 Dit nu, hetgeen ik u aanzegge, prijs ik niet, namelijk dat gij niet tot beter, maar tot erger samenkomt.
\par 18 Want eerstelijk, als gij samenkomt in de Gemeente, zo hoor ik, dat er scheuringen zijn onder u; en ik geloof het ten dele;
\par 19 Want er moeten ook ketterijen onder u zijn, opdat degenen, die oprecht zijn, openbaar mogen worden onder u.
\par 20 Als gij dan bijeen samenkomt, dat is niet des Heeren avondmaal eten.
\par 21 Want in het eten neemt een iegelijk te voren zijn eigen avondmaal; en deze is hongerig, en de andere is dronken.
\par 22 Hebt gij dan geen huizen, om er te eten en te drinken? Of veracht gij de Gemeente Gods, en beschaamt gij degenen, die niet hebben? Wat zal ik u zeggen? Zal ik u prijzen? In dezen prijs ik u niet.
\par 23 Want ik heb van den Heere ontvangen, hetgeen ik ook u overgegeven heb, dat de Heere Jezus in den nacht, in welken Hij verraden werd, het brood nam;
\par 24 En als Hij gedankt had, brak Hij het, en zeide: Neemt, eet, dat is Mijn lichaam, dat voor u gebroken wordt; doet dat tot Mijn gedachtenis.
\par 25 Desgelijks nam Hij ook den drinkbeker, na het eten des avondmaals, en zeide: Deze drinkbeker is het Nieuwe Testament in Mijn bloed. Doet dat, zo dikwijls als gij dien zult drinken, tot Mijn gedachtenis.
\par 26 Want zo dikwijls als gij dit brood zult eten, en dezen drinkbeker zult drinken, zo verkondigt den dood des Heeren, totdat Hij komt.
\par 27 Zo dan, wie onwaardiglijk dit brood eet, of den drinkbeker des Heeren drinkt, die zal schuldig zijn aan het lichaam en bloed des Heeren.
\par 28 Maar de mens beproeve zichzelven, en ete alzo van het brood, en drinke van den drinkbeker.
\par 29 Want die onwaardiglijk eet en drinkt, die eet en drinkt zichzelven een oordeel, niet onderscheidende het lichaam des Heeren.
\par 30 Daarom zijn onder u vele zwakken en kranken, en velen slapen.
\par 31 Want indien wij onszelven oordeelden, zo zouden wij niet geoordeeld worden.
\par 32 Maar als wij geoordeeld worden, zo worden wij van den Heere getuchtigd, opdat wij met de wereld niet zouden veroordeeld worden.
\par 33 Zo dan, mijn broeders, als gij samenkomt om te eten, verwacht elkander.
\par 34 Doch zo iemand hongert, dat hij te huis ete, opdat gij niet tot een oordeel samenkomt. De overige dingen nu zal ik verordenen, als ik zal gekomen zijn.

\chapter{12}

\par 1 En van de geestelijke gaven, broeders, wil ik niet, dat gij onwetende zijt.
\par 2 Gij weet, dat gij heidenen waart, tot de stomme afgoden heengetrokken, naar dat gij geleid werdt.
\par 3 Daarom maak ik u bekend, dat niemand, die door den Geest Gods spreekt, Jezus een vervloeking noemt; en niemand kan zeggen, Jezus den Heere te zijn, dan door den Heiligen Geest.
\par 4 En er is verscheidenheid der gaven, doch het is dezelfde Geest;
\par 5 En er is verscheidenheid der bedieningen, en het is dezelfde Heere;
\par 6 En er is verscheidenheid der werkingen, doch het is dezelfde God, Die alles in allen werkt.
\par 7 Maar aan een iegelijk wordt de openbaring des Geestes gegeven tot hetgeen oorbaar is.
\par 8 Want dezen wordt door den Geest gegeven het woord der wijsheid, en een ander het woord der kennis, door denzelfden Geest;
\par 9 En een ander het geloof, door denzelfden Geest; en een ander de gaven der gezondmakingen, door denzelfden Geest.
\par 10 En een ander de werkingen der krachten; en een ander profetie; en een ander onderscheidingen der geesten; en een ander menigerlei talen; en een ander uitlegging der talen.
\par 11 Doch deze dingen alle werkt een en dezelfde Geest, delende aan een iegelijk in het bijzonder, gelijkerwijs Hij wil.
\par 12 Want gelijk het lichaam een is, en vele leden heeft, en al de leden van dit ene lichaam, vele zijnde, maar een lichaam zijn, alzo ook Christus.
\par 13 Want ook wij allen zijn door een Geest tot een lichaam gedoopt; hetzij Joden, hetzij Grieken, hetzij dienstknechten, hetzij vrijen; en wij zijn allen tot een Geest gedrenkt.
\par 14 Want ook het lichaam is niet een lid, maar vele leden.
\par 15 Indien de voet zeide: Dewijl ik de hand niet ben, zo ben ik van het lichaam niet; is hij daarom niet van het lichaam?
\par 16 En indien het oor zeide: Dewijl ik het oog niet ben, zo ben ik van het lichaam niet; is het daarom niet van het lichaam?
\par 17 Ware het gehele lichaam het oog, waar zou het gehoor zijn? Ware het gehele lichaam gehoor, waar zou de reuk zijn?
\par 18 Maar nu heeft God de leden gezet, een iegelijk van dezelve in het lichaam, gelijk Hij gewild heeft.
\par 19 Waren zij alle maar een lid, waar zou het lichaam zijn?
\par 20 Maar nu zijn er wel vele leden, doch maar een lichaam.
\par 21 En het oog kan niet zeggen tot de hand: Ik heb u niet van node; of wederom het hoofd tot de voeten: Ik heb u niet van node.
\par 22 Ja veeleer, de leden, die ons dunken de zwakste des lichaams te zijn, die zijn nodig.
\par 23 En die ons dunken de minst eerlijke leden des lichaams te zijn, denzelven doen wij overvloediger eer aan; en onze onsierlijke leden hebben overvloediger versiering.
\par 24 Doch onze sierlijke hebben het niet van node; maar God heeft het lichaam alzo samengevoegd, gevende overvloediger eer aan hetgeen gebrek aan dezelve heeft;
\par 25 Opdat geen tweedracht in het lichaam zij, maar de leden voor elkander gelijke zorg zouden dragen.
\par 26 En hetzij dat een lid lijdt, zo lijden al de leden mede; hetzij dat een lid verheerlijkt wordt, zo verblijden zich al de leden mede.
\par 27 En gijlieden zijt het lichaam van Christus, en leden in het bijzonder.
\par 28 En God heeft er sommigen in de Gemeente gesteld, ten eerste apostelen, ten tweede profeten, ten derde leraars, daarna krachten, daarna gaven der gezondmakingen, behulpsels, regeringen, menigerlei talen.
\par 29 Zijn zij allen apostelen? Zijn zij allen profeten? Zijn zij allen leraars? Zijn zij allen krachten?
\par 30 Hebben zij allen gaven der gezondmakingen? Spreken zij allen met menigerlei talen? Zijn zij allen uitleggers?
\par 31 Doch ijvert naar de beste gaven; en ik wijs u een weg, die nog uitnemender is.

\chapter{13}

\par 1 Al ware het, dat ik de talen der mensen en der engelen sprak, en de liefde niet had, zo ware ik een klinkend metaal, of luidende schel geworden.
\par 2 En al ware het dat ik de gave der profetie had, en wist al de verborgenheden en al de wetenschap; en al ware het, dat ik al het geloof had, zodat ik bergen verzette, en de liefde niet had, zo ware ik niets.
\par 3 En al ware het, dat ik al mijn goederen tot onderhoud der armen uitdeelde, en al ware het, dat ik mijn lichaam overgaf, opdat ik verbrand zou worden, en had de liefde niet, zo zou het mij geen nuttigheid geven.
\par 4 De liefde is lankmoedig, zij is goedertieren; de liefde is niet afgunstig; de liefde handelt niet lichtvaardiglijk, zij is niet opgeblazen;
\par 5 Zij handelt niet ongeschiktelijk, zij zoekt zichzelve niet, zij wordt niet verbitterd, zij denkt geen kwaad;
\par 6 Zij verblijdt zich niet in de ongerechtigheid, maar zij verblijdt zich in de waarheid;
\par 7 Zij bedekt alle dingen, zij gelooft alle dingen, zij hoopt alle dingen, zij verdraagt alle dingen.
\par 8 De liefde vergaat nimmermeer; maar hetzij profetieen, zij zullen te niet gedaan worden; hetzij talen, zij zullen ophouden; hetzij kennis, zij zal te niet gedaan worden.
\par 9 Want wij kennen ten dele, en wij profeteren ten dele;
\par 10 Doch wanneer het volmaakte zal gekomen zijn, dan zal hetgeen ten dele is, te niet gedaan worden.
\par 11 Toen ik een kind was, sprak ik als een kind, was ik gezind als een kind, overlegde ik als een kind; maar wanneer ik een man geworden ben, zo heb ik te niet gedaan hetgeen eens kinds was.
\par 12 Want wij zien nu door een spiegel in een duistere rede, maar alsdan zullen wij zien aangezicht tot aangezicht; nu ken ik ten dele, maar alsdan zal ik kennen, gelijk ook ik gekend ben.
\par 13 En nu blijft geloof, hoop en liefde, deze drie; doch de meeste van deze is de liefde.

\chapter{14}

\par 1 Jaagt de liefde na, en ijvert om de geestelijke gaven, maar meest, dat gij moogt profeteren.
\par 2 Want die een vreemde taal spreekt, spreekt niet den mensen, maar Gode; want niemand verstaat het, doch met den geest spreekt hij verborgenheden.
\par 3 Maar die profeteert, spreekt den mensen stichting, en vermaning en vertroosting.
\par 4 Die een vreemde taal spreekt, die sticht zichzelven; maar die profeteert die sticht de Gemeente.
\par 5 En ik wil wel, dat gij allen in vreemde talen spreekt, maar meer, dat gij profeteert; want die profeteert, is meerder dan die vreemde talen spreekt, tenzij dan, dat hij het uitlegge, opdat de Gemeente stichting moge ontvangen.
\par 6 En nu, broeders, indien ik tot u kwam, en sprak vreemde talen, wat nuttigheid zou ik u doen, zo ik tot u niet sprak, of in openbaring, of in kennis, of in profetie of in lering?
\par 7 Zelfs ook de levenloze dingen, die geluid geven, hetzij fluit, hetzij citer, zo zij geen onderscheid met hun klank geven, hoe zal bekend worden, hetgeen op de fluit of op de citer gespeeld wordt?
\par 8 Want ook indien de bazuin een onzeker geluid geeft, wie zal zich tot den krijg bereiden?
\par 9 Alzo ook gijlieden, indien gij niet door de taal een duidelijke rede geeft, hoe zal verstaan worden hetgeen gesproken wordt? Want gij zult zijn als die in de lucht spreekt.
\par 10 Er zijn, naar het voorvalt, zo vele soorten van stemmen in de wereld, en geen derzelve is zonder stem.
\par 11 Indien ik dan de kracht der stem niet weet, zo zal ik hem, die spreekt, barbaars zijn; en hij, die spreekt, zal bij mij barbaars zijn.
\par 12 Alzo ook gij, dewijl gij ijverig zijt naar geestelijke gaven, zo zoekt dat gij moogt overvloedig zijn tot stichting der Gemeente.
\par 13 Daarom, die in een vreemde taal spreekt, die bidde, dat hij het moge uitleggen.
\par 14 Want indien ik in een vreemde taal bid, mijn geest bidt wel, maar mijn verstand is vruchteloos.
\par 15 Wat is het dan? Ik zal wel met den geest bidden, maar ik zal ook met het verstand bidden; ik zal wel met den geest zingen, maar ik zal ook met het verstand zingen.
\par 16 Anderszins, indien gij dankzegt met den geest, hoe zal degene, die de plaats eens ongeleerden vervult, amen zeggen op uw dankzegging, dewijl hij niet weet, wat gij zegt?
\par 17 Want gij dankzegt wel behoorlijk, maar de ander wordt niet gesticht.
\par 18 Ik dank mijn God, dat ik meer vreemde talen spreek, dan gij allen;
\par 19 Maar ik wil liever in de Gemeente vijf woorden spreken met mijn verstand, opdat ik ook anderen moge onderwijzen, dan tien duizend woorden in een vreemde taal.
\par 20 Broeders, wordt geen kinderen in het verstand, maar zijt kinderen in de boosheid, en wordt in het verstand volwassen.
\par 21 In de wet is geschreven: Ik zal door lieden van andere talen, en door andere lippen tot dit volk spreken, en ook alzo zullen zij Mij niet horen, zegt de Heere.
\par 22 Zo dan, de vreemde talen zijn tot een teken niet dengenen, die geloven, maar den ongelovigen; en de profetie niet den ongelovigen, maar dengenen, die geloven.
\par 23 Indien dan de gehele Gemeente bijeenvergaderd ware, en zij allen in vreemde talen spraken, en enige ongeleerden of ongelovigen inkwamen, zouden zij niet zeggen, dat gij uitzinnig waart?
\par 24 Maar indien zij allen profeteerden, en een ongelovige of ongeleerde inkwame, die wordt van allen overtuigd, en hij wordt van allen geoordeeld.
\par 25 En alzo worden de verborgene dingen zijns harten openbaar; en alzo, vallende op zijn aangezicht, zal hij God aanbidden, en verkondigen, dat God waarlijk onder u is.
\par 26 Wat is het dan, broeders? Wanneer gij samenkomt, een iegelijk van u, heeft hij een psalm, heeft hij een leer, heeft hij een vreemde taal, heeft hij een openbaring, heeft hij een uitlegging; laat alle dingen geschieden tot stichting;
\par 27 En zo iemand een vreemde taal spreekt, dat het door twee, of ten meeste drie geschiede, en bij beurte; en dat een het uitlegge.
\par 28 Maar indien er geen uitlegger is, dat hij zwijge in de Gemeente; doch dat hij tot zichzelven spreke, en tot God.
\par 29 En dat twee of drie profeten spreken, en dat de anderen oordelen.
\par 30 Doch indien een ander, die er zit, iets geopenbaard is, dat de eerste zwijge.
\par 31 Want gij kunt allen, de een na den ander profeteren, opdat zij allen leren, en allen getroost worden.
\par 32 En de geesten der profeten zijn den profeten onderworpen.
\par 33 Want God is geen God van verwarring, maar van vrede, gelijk in al de Gemeenten der heiligen.
\par 34 Dat uw vrouwen in de Gemeenten zwijgen; want het is haar niet toegelaten te spreken, maar bevolen onderworpen te zijn, gelijk ook de wet zegt.
\par 35 En zo zij iets willen leren, laat haar te huis haar eigen mannen vragen; want het staat lelijk voor de vrouwen, dat zij in de Gemeente spreken.
\par 36 Is het Woord Gods van u uitgegaan? Of is het tot u alleen gekomen?
\par 37 Indien iemand meent een profeet te zijn, of geestelijke, die erkenne, dat, hetgeen ik u schrijf, des Heeren geboden zijn.
\par 38 Maar zo iemand onwetend is, die zij onwetend.
\par 39 Zo dan, broeders, ijvert om te profeteren, en verhindert niet in vreemde talen te spreken.
\par 40 Laat alle dingen eerlijk en met orde geschieden.

\chapter{15}

\par 1 Voorts, broeders, ik maak u bekend het Evangelie, dat ik u verkondigd heb, hetwelk gij ook aangenomen hebt, in hetwelk gij ook staat;
\par 2 Door hetwelk gij ook zalig wordt, indien gij het behoudt op zodanige wijze, als ik het u verkondigd heb; tenzij dan dat gij tevergeefs geloofd hebt.
\par 3 Want ik heb ulieden ten eerste overgegeven, hetgeen ik ook ontvangen heb, dat Christus gestorven is voor onze zonden, naar de Schriften;
\par 4 En dat Hij is begraven, en dat Hij is opgewekt ten derden dage, naar de Schriften;
\par 5 En dat Hij is van Cefas gezien, daarna van de twaalven.
\par 6 Daarna is Hij gezien van meer dan vijfhonderd broeders op eenmaal, van welken het meren deel nog over is, en sommigen ook zijn ontslapen.
\par 7 Daarna is Hij gezien van Jakobus, daarna van al de apostelen.
\par 8 En ten laatste van allen is Hij ook van mij, als van een ontijdig geborene, gezien.
\par 9 Want ik ben de minste van de apostelen, die niet waardig ben een apostel genaamd te worden, daarom dat ik de Gemeente Gods vervolgd heb.
\par 10 Doch door de genade Gods ben ik, dat ik ben; en Zijn genade, die aan mij bewezen is, is niet ijdel geweest, maar ik heb overvloediger gearbeid dan zij allen; doch niet ik, maar de genade Gods, Die met mij is.
\par 11 Hetzij dan ik, hetzij zijlieden, alzo prediken wij, en alzo hebt gij geloofd.
\par 12 Indien nu Christus gepredikt wordt, dat Hij uit de doden opgewekt is, hoe zeggen sommigen onder u, dat er geen opstanding der doden is?
\par 13 En indien er geen opstanding der doden is, zo is Christus ook niet opgewekt.
\par 14 En indien Christus niet opgewekt is, zo is dan onze prediking ijdel, en ijdel is ook uw geloof.
\par 15 En zo worden wij ook bevonden valse getuigen Gods; want wij hebben van God getuigd, dat Hij Christus opgewekt heeft, Dien Hij niet heeft opgewekt, zo namelijk de doden niet opgewekt worden.
\par 16 Want indien de doden niet opgewekt worden, zo is ook Christus niet opgewekt.
\par 17 En indien Christus niet opgewekt is, zo is uw geloof tevergeefs, zo zijt gij nog in uw zonden.
\par 18 Zo zijn dan ook verloren, die in Christus ontslapen zijn.
\par 19 Indien wij alleenlijk in dit leven op Christus zijn hopende, zo zijn wij de ellendigste van alle mensen.
\par 20 Maar nu, Christus is opgewekt uit de doden, en is de Eersteling geworden dergenen, die ontslapen zijn.
\par 21 Want dewijl de dood door een mens is, zo is ook de opstanding der doden door een Mens.
\par 22 Want gelijk zij allen in Adam sterven, alzo zullen zij ook in Christus allen levend gemaakt worden.
\par 23 Maar een iegelijk in zijn orde: de eersteling Christus, daarna die van Christus zijn, in Zijn toekomst.
\par 24 Daarna zal het einde zijn, wanneer Hij het Koninkrijk aan God en den Vader zal overgegeven hebben; wanneer Hij zal te niet gedaan hebben alle heerschappij, en alle macht en kracht.
\par 25 Want Hij moet als Koning heersen, totdat Hij al de vijanden onder Zijn voeten zal gelegd hebben.
\par 26 De laatste vijand, die te niet gedaan wordt, is de dood.
\par 27 Want Hij heeft alle dingen Zijn voeten onderworpen. Doch wanneer Hij zegt, dat Hem alle dingen onderworpen zijn, zo is het openbaar, dat Hij uitgenomen wordt, Die Hem alle dingen onderworpen heeft.
\par 28 En wanneer Hem alle dingen zullen onderworpen zijn, dan zal ook de Zoon Zelf onderworpen worden Dien, Die Hem alle dingen onderworpen heeft, opdat God zij alles in allen.
\par 29 Anders, wat zullen zij doen, die voor de doden gedoopt worden, indien de doden ganselijk niet opgewekt worden? Waarom worden zij voor de doden ook gedoopt?
\par 30 Waarom zijn ook wij alle ure in gevaar?
\par 31 Ik sterf alle dagen, hetwelk ik betuig bij onzen roem, dien ik heb in Christus Jezus, onzen Heere.
\par 32 Zo ik, naar den mens, tegen de beesten gevochten heb te Efeze, wat nuttigheid is het mij, indien de doden niet opgewekt worden? Laat ons eten en drinken, want morgen sterven wij.
\par 33 Dwaalt niet. Kwade samensprekingen verderven goede zeden.
\par 34 Waakt op rechtvaardiglijk, en zondigt niet. Want sommigen hebben de kennis van God niet. Ik zeg het u tot schaamte.
\par 35 Maar, zal iemand zeggen: Hoe zullen de doden opgewekt worden, en met hoedanig een lichaam zullen zij komen?
\par 36 Gij dwaas, hetgeen gij zaait, wordt niet levend, tenzij dat het gestorven is;
\par 37 En hetgeen gij zaait, daarvan zaait gij het lichaam niet, dat worden zal, maar een bloot graan, naar het voorvalt, van tarwe, of van enig der andere granen.
\par 38 Maar God geeft hetzelve een lichaam, gelijk Hij wil, en aan een iegelijk zaad zijn eigen lichaam.
\par 39 Alle vlees is niet hetzelfde vlees; maar een ander is het vlees der mensen, en een ander is het vlees der beesten, en een ander der vissen, en een ander der vogelen.
\par 40 En er zijn hemelse lichamen, en er zijn aardse lichamen; maar een andere is de heerlijkheid der hemelse, en een andere der aardse.
\par 41 Een andere is de heerlijkheid der zon, en een andere is de heerlijkheid der maan, en een andere is de heerlijkheid der sterren; want de ene ster verschilt in heerlijkheid van de andere ster.
\par 42 Alzo zal ook de opstanding der doden zijn. Het lichaam wordt gezaaid in verderfelijkheid, het wordt opgewekt in onverderfelijkheid;
\par 43 Het wordt gezaaid in oneer, het wordt opgewekt in heerlijkheid; het wordt gezaaid in zwakheid, het wordt opgewekt in kracht.
\par 44 Een natuurlijk lichaam wordt er gezaaid, een geestelijk lichaam wordt er opgewekt. Er is een natuurlijk lichaam, en er is een geestelijk lichaam.
\par 45 Alzo is er ook geschreven: De eerste mens Adam is geworden tot een levende ziel; de laatste Adam tot een levendmakenden Geest.
\par 46 Doch het geestelijke is niet eerst, maar het natuurlijke, daarna het geestelijke.
\par 47 De eerste mens is uit de aarde, aards; de tweede Mens is de Heere uit den Hemel.
\par 48 Hoedanig de aardse is, zodanige zijn ook de aardsen; en hoedanig de Hemelse is, zodanige zijn ook de hemelsen.
\par 49 En gelijkerwijs wij het beeld des aardsen gedragen hebben, alzo zullen wij ook het beeld des Hemelsen dragen.
\par 50 Doch dit zeg ik, broeders, dat vlees en bloed het Koninkrijk Gods niet beerven kunnen, en de verderfelijkheid beerft de onverderfelijkheid niet.
\par 51 Ziet, ik zeg u een verborgenheid: wij zullen wel niet allen ontslapen, maar wij zullen allen veranderd worden;
\par 52 In een punt des tijds, in een ogenblik, met de laatste bazuin; want de bazuin zal slaan, en de doden zullen onverderfelijk opgewekt worden, en wij zullen veranderd worden.
\par 53 Want dit verderfelijke moet onverderfelijkheid aandoen, en dit sterfelijke moet onsterfelijkheid aandoen.
\par 54 En wanneer dit verderfelijke zal onverderfelijkheid aangedaan hebben, en dit sterfelijke zal onsterfelijkheid aangedaan hebben, alsdan zal het woord geschieden, dat geschreven is: De dood is verslonden tot overwinning.
\par 55 Dood, waar is uw prikkel? Hel, waar is uw overwinning?
\par 56 De prikkel nu des doods is de zonde; en de kracht der zonde is de wet.
\par 57 Maar Gode zij dank, Die ons de overwinning geeft door onzen Heere Jezus Christus.
\par 58 Zo dan, mijn geliefde broeders! Zijt standvastig, onbewegelijk, altijd overvloedig zijnde in het werk des Heeren, als die weet, dat uw arbeid niet ijdel is in den Heere.

\chapter{16}

\par 1 Aangaande nu de verzameling, die voor de heiligen geschiedt, gelijk als ik aan de Gemeenten in Galatie verordend heb, doet ook gij alzo.
\par 2 Op elken eersten dag der week, legge een iegelijk van u iets bij zichzelven weg, vergaderende een schat, naar dat hij welvaren verkregen heeft; opdat de verzamelingen alsdan niet eerst geschieden, wanneer ik gekomen zal zijn.
\par 3 En wanneer ik daar zal gekomen zijn, zal ik hen, die gij zult bekwaam achten door brieven, zenden, om uw gave naar Jeruzalem over te dragen.
\par 4 En indien het der moeite waardig mocht zijn, dat ik ook zelf reizen zou, zo zullen zij met mij reizen.
\par 5 Doch ik zal tot u komen, wanneer ik Macedonie zal doorgegaan zijn, (want ik zal door Macedonie gaan)
\par 6 En ik zal mogelijk bij u blijven, of ook overwinteren, opdat gij mij moogt geleiden, waar ik zal henenreizen.
\par 7 Want ik wil u nu niet zien in het voorbijgaan, maar ik hoop enigen tijd bij u te blijven, indien het de Heere zal toelaten.
\par 8 Maar ik zal te Efeze blijven tot den pinkster dag.
\par 9 Want mij is een grote en krachtige deur geopend, en er zijn vele tegenstanders.
\par 10 Zo nu Timotheus komt, ziet, dat hij buiten vreze bij u zij; want hij werkt het werk des Heeren, gelijk als ik.
\par 11 Dat hem dan niemand verachte; maar geleidt hem in vrede, opdat hij tot mij kome; want ik verwacht hem met de broederen.
\par 12 En wat aangaat Apollos, den broeder, ik heb hem zeer gebeden, dat hij met de broederen tot u komen zou; maar het was ganselijk zijn wil niet, dat hij nu zou komen; doch hij zal komen, wanneer het hem wel gelegen zal zijn.
\par 13 Waakt, staat in het geloof, houdt u mannelijk, zijt sterk.
\par 14 Dat al uw dingen in de liefde geschieden.
\par 15 En ik bid u, broeders, gij kent het huis van Stefanas, dat het is de eersteling van Achaje, en dat zij zichzelven den heiligen ten dienst hebben geschikt;
\par 16 Dat gij ook u aan de zodanigen onderwerpt, en aan een iegelijk, die medewerkt en arbeidt.
\par 17 En ik verblijde mij over de aankomst van Stefanas, en Fortunatus, en Achaikus, want dezen hebben vervuld hetgeen mij aan u ontbrak;
\par 18 Want zij hebben mijn geest verkwikt, en ook den uwen. Erkent dan de zodanigen.
\par 19 U groeten de Gemeenten van Azie. U groeten zeer in den Heere Aquila en Priscilla, met de Gemeente, die te hunnen huize is.
\par 20 U groeten al de broeders. Groet elkander met een heiligen kus.
\par 21 De groetenis met mijn hand van Paulus.
\par 22 Indien iemand den Heere Jezus Christus niet liefheeft, die zij een vervloeking; Maran-atha!
\par 23 De genade van den Heere Jezus Christus zij met u.
\par 24 Mijn liefde zij met u allen in Christus Jezus. Amen.




\end{document}