\begin{document}

\title{2 Thessalonicenzen}



\chapter{1}

\par 1 Paulus, en Silvanus, en Timotheus, aan de Gemeente der Thessalonicensen, welke is in God, onzen Vader, en den Heere Jezus Christus:
\par 2 Genade zij u, en vrede, van God, onzen Vader, en den Heere Jezus Christus.
\par 3 Wij moeten God te allen tijd danken over u, broeders, gelijk billijk is, omdat uw geloof zeer wast, en dat de liefde eens iegelijken van u allen jegens elkander overvloedig wordt;
\par 4 Alzo dat wij zelven van u roemen in de Gemeenten Gods, over uw lijdzaamheid en geloof in al uw vervolgingen en verdrukkingen, die gij verdraagt;
\par 5 Een bewijs van Gods rechtvaardig oordeel, opdat gij waardig geacht wordt het Koninkrijk Gods, voor hetwelk gij ook lijdt;
\par 6 Alzo het recht is bij God verdrukking te vergelden dengenen, die u verdrukken;
\par 7 En u, die verdrukt wordt, verkwikking met ons, in de openbaring van den Heere Jezus van den hemel met de engelen Zijner kracht;
\par 8 Met vlammend vuur wraak doende over degenen, die God niet kennen, en over degenen, die het Evangelie van onzen Heere Jezus Christus niet gehoorzaam zijn.
\par 9 Dewelken zullen tot straf lijden het eeuwig verderf, van het aangezicht des Heeren, en van de heerlijkheid Zijner sterkte,
\par 10 Wanneer Hij zal gekomen zijn, om verheerlijkt te worden in Zijn heiligen, en wonderbaar te worden in allen, die geloven (overmits onze getuigenis onder u is geloofd geworden) in dien dag.
\par 11 Waarom wij ook altijd bidden voor u, dat onze God u waardig achte der roeping, en vervulle al het welbehagen Zijner goedigheid, en het werk des geloofs met kracht.
\par 12 Opdat de Naam van onzen Heere Jezus Christus verheerlijkt worde in u, en gij in Hem, naar de genade van onzen God en den Heere Jezus Christus.

\chapter{2}

\par 1 En wij bidden u, broeders, door de toekomst van onzen Heere Jezus Christus, en onze toevergadering tot Hem,
\par 2 Dat gij niet haastelijk bewogen wordt van verstand, of verschrikt, noch door geest, noch door woord, noch door zendbrief, als van ons geschreven, alsof de dag van Christus aanstaande ware.
\par 3 Dat u niemand verleide op enigerlei wijze; want die komt niet, tenzij dat eerst de afval gekomen zij, en dat geopenbaard zij de mens der zonde, de zoon des verderfs;
\par 4 Die zich tegenstelt, en verheft boven al wat God genaamd, of als God geeerd wordt, alzo dat hij in den tempel Gods als een God zal zitten, zichzelven vertonende, dat hij God is.
\par 5 Gedenkt gij niet, dat ik, nog bij u zijnde, u deze dingen gezegd heb?
\par 6 En nu, wat hem wederhoudt, weet gij, opdat hij geopenbaard worde te zijner eigen tijd.
\par 7 Want de verborgenheid der ongerechtigheid wordt alrede gewrocht; alleenlijk, Die hem nu wederhoudt, Die zal hem wederhouden, totdat hij uit het midden zal weggedaan worden.
\par 8 En alsdan zal de ongerechtige geopenbaard worden, denwelken de Heere verdoen zal door den Geest Zijns monds, en te niet maken door de verschijning Zijner toekomst;
\par 9 Hem, zeg ik, wiens toekomst is naar de werking des satans, in alle kracht, en tekenen, en wonderen der leugen;
\par 10 En in alle verleiding der onrechtvaardigheid in degenen, die verloren gaan; daarvoor dat zij de liefde der waarheid niet aangenomen hebben, om zalig te worden.
\par 11 En daarom zal God hun zenden een kracht der dwaling, dat zij de leugen zouden geloven;
\par 12 Opdat zij allen veroordeeld worden, die de waarheid niet geloofd hebben, maar een welbehagen hebben gehad in de ongerechtigheid.
\par 13 Maar wij zijn schuldig altijd God te danken over u, broeders, die van den Heere bemind zijt, dat u God van den beginne verkoren heeft tot zaligheid, in heiligmaking des Geestes, en geloof der waarheid;
\par 14 Waartoe Hij u geroepen heeft door ons Evangelie, tot verkrijging der heerlijkheid van onzen Heere Jezus Christus.
\par 15 Zo dan, broeders, staat vast en houdt de inzettingen, die u geleerd zijn, hetzij door ons woord, hetzij door onzen zendbrief.
\par 16 En onze Heere Jezus Christus Zelf, en onze God en Vader, Die ons heeft liefgehad, en gegeven heeft een eeuwige vertroosting en goede hoop in genade,
\par 17 Vertrooste uw harten, en versterke u in alle goed woord en werk.

\chapter{3}

\par 1 Voorts, broeders, bidt voor ons, opdat het Woord des Heeren zijn loop hebbe, en verheerlijkt worde, gelijk ook bij u;
\par 2 En opdat wij mogen verlost worden van de ongeschikte en boze mensen; want het geloof is niet aller.
\par 3 Maar de Heere is getrouw, Die u zal versterken en bewaren van den boze.
\par 4 En wij vertrouwen van u in den Heere, dat gij, hetgeen wij u bevelen, ook doet, en doen zult.
\par 5 Doch de Heere richte uw harten tot de liefde van God, en tot de lijdzaamheid van Christus.
\par 6 En wij bevelen u, broeders, in den Naam van onzen Heere Jezus Christus, dat gij u onttrekt van een iegelijk broeder, die ongeregeld wandelt, en niet naar de inzetting, die hij van ons ontvangen heeft.
\par 7 Want gijzelven weet, hoe men ons behoort na te volgen; want wij hebben ons niet ongeregeld gedragen onder u;
\par 8 En wij hebben geen brood bij iemand gegeten voor niet, maar in arbeid en moeite, nacht en dag werkende, opdat wij niet iemand van u zouden lastig zijn;
\par 9 Niet, dat wij de macht niet hebben, maar opdat wij onszelven u geven zouden tot een voorbeeld, om ons na te volgen.
\par 10 Want ook toen wij bij u waren, hebben wij u dit bevolen, dat, zo iemand niet wil werken, hij ook niet ete.
\par 11 Want wij horen, dat sommigen onder u ongeregeld wandelen, niet werkende, maar ijdele dingen doende.
\par 12 Doch de zodanigen bevelen en vermanen wij door onzen Heere Jezus Christus, dat zij met stilheid werkende, hun eigen brood eten.
\par 13 En gij, broeders, vertraagt niet in goed te doen.
\par 14 Maar indien iemand ons woord, door dezen brief geschreven, niet gehoorzaam is, tekent dien; en vermengt u niet met hem, opdat hij beschaamd worde;
\par 15 En houdt hem niet als een vijand, maar vermaant hem als een broeder.
\par 16 De Heere nu des vredes Zelf geve u vrede te allen tijd, in allerlei wijze. De Heere zij met u allen.
\par 17 De groetenis met mijn hand, van Paulus; hetwelk is een teken in iederen zendbrief; alzo schrijf ik.
\par 18 De genade van onzen Heere Jezus Christus zij met u allen. Amen.



\end{document}