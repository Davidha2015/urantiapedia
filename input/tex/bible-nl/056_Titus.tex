\begin{document}

\title{Titus}



\chapter{1}

\par 1 Paulus, een dienstknecht Gods, en een apostel van Jezus Christus, naar het geloof der uitverkorenen Gods, en de kennis der waarheid, die naar de godzaligheid is;
\par 2 In de hoop des eeuwigen levens, welke God, Die niet liegen kan, beloofd heeft, voor de tijden der eeuwen, maar geopenbaard heeft te Zijner tijd;
\par 3 Namelijk Zijn Woord, door de prediking, die mij toebetrouwd is, naar het bevel van God, onze Zaligmaker; aan Titus, mijn oprechten zoon, naar het gemeen geloof:
\par 4 Genade, barmhartigheid, vrede zij u van God den Vader, en den Heere Jezus Christus, onzen Zaligmaker.
\par 5 Om die oorzaak heb ik u te Kreta gelaten, opdat gij, hetgeen nog ontbrak, voorts zoudt te recht brengen, en dat gij van stad tot stad zoudt ouderlingen stellen, gelijk ik u bevolen heb:
\par 6 Indien iemand onberispelijk is, ener vrouwe man, gelovige kinderen hebbende, die niet te beschuldigen zijn van overdadigheid, of ongehoorzaam zijn.
\par 7 Want een opziener moet onberispelijk zijn, als een huisverzorger Gods, niet eigenzinnig, niet genegen tot toornigheid, niet genegen tot den wijn, geen smijter, geen vuil-gewinzoeker;
\par 8 Maar die gaarne herbergt, die de goeden liefheeft, matig, rechtvaardig, heilig, kuis;
\par 9 Die vasthoudt aan het getrouwe woord, dat naar de leer is, opdat hij machtig zij, beide om te vermanen door de gezonde leer, en om de tegensprekers te wederleggen.
\par 10 Want er zijn ook vele ongeregelden, ijdelheidsprekers en verleiders van zinnen, inzonderheid die uit de besnijdenis zijn;
\par 11 Welken men moet den mond stoppen, die gehele huizen verkeren, lerende wat niet behoort, om vuil gewins wil.
\par 12 Een uit hen, zijnde hun eigen profeet, heeft gezegd: De Kretensen zijn altijd leugenachtig, kwade beesten, luie buiken.
\par 13 Deze getuigenis is waar. Daarom bestraf hen scherpelijk, opdat zij gezond mogen zijn in het geloof.
\par 14 En zich niet begeven tot Joodse fabelen, en geboden der mensen, die hen van de waarheid afkeren.
\par 15 Alle dingen zijn wel rein den reinen, maar den bevlekten en ongelovigen is geen ding rein, maar beide hun verstand en geweten zijn bevlekt.
\par 16 Zij belijden, dat zij God kennen, maar zij verloochenen Hem met de werken, alzo zij gruwelijk zijn en ongehoorzaam, en tot alle goed werk ongeschikt.

\chapter{2}

\par 1 Doch gij, spreek hetgeen der gezonde leer betaamt.
\par 2 Dat de oude mannen nuchter zijn, stemmig, voorzichtig, gezond in het geloof, in de liefde, in de lijdzaamheid.
\par 3 De oude vrouwen insgelijks, dat zij in haar dracht zijn, gelijk den heiligen betaamt, dat zij geen lasteressen zijn, zich niet tot veel wijns begevende, maar leraressen zijn van het goede;
\par 4 Opdat zij de jonge vrouwen leren voorzichtig te zijn, haar mannen lief te hebben, haar kinderen lief te hebben;
\par 5 Matig te zijn, kuis te zijn, het huis te bewaren, goed te zijn, haar eigen mannen onderdanig te zijn, opdat het Woord Gods niet gelasterd worde.
\par 6 Vermaan den jongen mannen insgelijks, dat zij matig zijn.
\par 7 Betoon uzelven in alles een voorbeeld van goede werken, betoon in de leer onvervalstheid, deftigheid, oprechtheid;
\par 8 Het woord gezond en onverwerpelijk, opdat degene, die daartegen is, beschaamd worde, en niets kwaads hebbe van ulieden te zeggen.
\par 9 Vermaan den dienstknechten, dat zij hun eigen heren onderdanig zijn, dat zij in alles welbehagelijk zijn, niet tegensprekende;
\par 10 Niet onttrekkende, maar alle goede trouw bewijzende; opdat zij de leer van God, onzen Zaligmaker, in alles mogen versieren.
\par 11 Want de zaligmakende genade Gods is verschenen aan alle mensen.
\par 12 En onderwijst ons, dat wij, de goddeloosheid en de wereldse begeerlijkheden verzakende, matig en rechtvaardig, en godzalig leven zouden in deze tegenwoordige wereld;
\par 13 Verwachtende de zalige hoop en verschijning der heerlijkheid van den groten God en onzen Zaligmaker Jezus Christus;
\par 14 Die Zichzelven voor ons gegeven heeft, opdat Hij ons zou verlossen van alle ongerechtigheid, en Zichzelven een eigen volk zou reinigen, ijverig in goede werken.
\par 15 Spreek dit, en vermaan, en bestraf met allen ernst. Dat niemand u verachte.

\chapter{3}

\par 1 Vermaan hen, dat zij aan de overheden en machten onderdanig zijn, dat zij hun gehoorzaam zijn, dat zij tot alle goed werk bereid zijn;
\par 2 Dat zij niemand lasteren, geen vechters zijn, maar bescheiden zijn, alle zachtmoedigheid bewijzende jegens alle mensen.
\par 3 Want ook wij waren eertijds onwijs, ongehoorzaam, dwalende, menigerlei begeerlijkheden en wellusten dienende, in boosheid en nijdigheid levende, hatelijk zijnde, en elkander hatende.
\par 4 Maar wanneer de goedertierenheid van God, onzen Zaligmaker, en Zijn liefde tot de mensen verschenen is,
\par 5 Heeft Hij ons zalig gemaakt, niet uit de werken der rechtvaardigheid, die wij gedaan hadden, maar naar Zijn barmhartigheid, door het bad der wedergeboorte en vernieuwing des Heiligen Geestes;
\par 6 Denwelken Hij over ons rijkelijk heeft uitgegoten door Jezus Christus, onzen Zaligmaker;
\par 7 Opdat wij, gerechtvaardigd zijnde door Zijn genade, erfgenamen zouden worden naar de hope des eeuwigen levens.
\par 8 Dit is een getrouw woord, en deze dingen wil ik, dat gij ernstelijk bevestigt, opdat degenen, die aan God geloven, zorg dragen, om goede werken voor te staan; deze dingen zijn het, die goed en nuttig zijn den mensen.
\par 9 Maar wedersta de dwaze vragen en geslachtsrekeningen, en twistingen, en strijdingen over de wet; want zij zijn onnut en ijdel.
\par 10 Verwerp een kettersen mens na de eerste en tweede vermaning;
\par 11 Wetende, dat de zodanige verkeerd is, en zondigt, zijnde bij zichzelf veroordeeld.
\par 12 Als ik Artemas tot u zal zenden, of Tychikus, zo benaarstig u tot mij te komen te Nikopolis; want aldaar heb ik voorgenomen te overwinteren.
\par 13 Geleid Zenas, den wetgeleerde, en Apollos zorgvuldiglijk, opdat hun niets ontbreke.
\par 14 En dat ook de onzen leren, goede werken voor te staan tot nodig gebruik, opdat zij niet onvruchtbaar zijn.
\par 15 Die met mij zijn, groeten u allen. Groet ze, die ons liefhebben in het geloof. De genade zij met u allen. Amen.




\end{document}