\begin{document}

\title{Hebrews}



\chapter{1}

\par 1 God, voortijds veelmaal en op velerlei wijze, tot de vaderen gesproken hebbende door de profeten, heeft in deze laatste dagen tot ons gesproken door den Zoon;
\par 2 Welken Hij gesteld heeft tot een Erfgenaam van alles, door Welken Hij ook de wereld gemaakt heeft;
\par 3 Dewelke, alzo Hij is het Afschijnsel Zijner heerlijkheid, en het uitgedrukte Beeld Zijner zelfstandigheid, en alle dingen draagt door het woord Zijner kracht, nadat Hij de reinigmaking onzer zonden door Zichzelven te weeg gebracht heeft, is gezeten aan de rechter hand der Majesteit in de hoogste hemelen;
\par 4 Zoveel treffelijker geworden dan de engelen, als Hij uitnemender Naam boven hen geerfd heeft.
\par 5 Want tot wien van de engelen heeft Hij ooit gezegd: Gij zijt Mijn Zoon, heden heb ik U gegenereerd? En wederom: Ik zal Hem tot een Vader zijn, en Hij zal Mij tot een Zoon zijn?
\par 6 En als Hij wederom den Eerstgeborene inbrengt in de wereld, zegt Hij: En dat alle engelen Gods Hem aanbidden.
\par 7 En tot de engelen zegt Hij wel: Die Zijn engelen maakt geesten, en Zijn dienaars een vlam des vuurs.
\par 8 Maar tot den Zoon zegt Hij: Uw troon, o God, is in alle eeuwigheid; de schepter Uws koninkrijks is een rechte schepter.
\par 9 Gij hebt rechtvaardigheid liefgehad, en ongerechtigheid gehaat; daarom heeft U, o God! Uw God gezalfd met olie der vreugde boven Uw medegenoten.
\par 10 En: Gij, Heere! hebt in den beginne de aarde gegrond, en de hemelen zijn werken Uwer handen;
\par 11 Dezelve zullen vergaan, maar Gij blijft altijd, en zij zullen alle als een kleed verouden;
\par 12 En als een dekkleed zult Gij ze ineenrollen, en zij zullen veranderd worden; maar Gij zijt Dezelfde, en Uw jaren zullen niet ophouden.
\par 13 En tot welken der engelen heeft Hij ooit gezegd: Zit aan Mijn rechter hand, totdat Ik Uw vijanden zal gezet hebben tot een voetbank Uwer voeten?
\par 14 Zijn zij niet allen gedienstige geesten, die tot dienst uitgezonden worden, om dergenen wil, die de zaligheid beerven zullen?

\chapter{2}

\par 1 Daarom moeten wij ons te meer houden aan hetgeen van ons gehoord is, opdat wij niet te eniger tijd doorvloeien.
\par 2 Want indien het woord, door de engelen gesproken, vast is geweest, en alle overtreding en ongehoorzaamheid rechtvaardige vergelding ontvangen heeft;
\par 3 Hoe zullen wij ontvlieden, indien wij op zo grote zaligheid geen acht nemen? dewelke, begonnen zijnde verkondigd te worden door den Heere, aan ons bevestigd is geworden van degenen, die Hem gehoord hebben;
\par 4 God bovendien medegetuigende door tekenen, en wonderen, en menigerlei krachten en bedelingen des Heiligen Geestes, naar Zijn wil.
\par 5 Want Hij heeft aan de engelen niet onderworpen de toekomende wereld, van welke wij spreken.
\par 6 Maar iemand heeft ergens betuigd, zeggende: Wat is de mens, dat Gij zijner gedenkt, of des mensen zoon, dat Gij hem bezoekt!
\par 7 Gij hebt hem een weinig minder gemaakt dan de engelen; met heerlijkheid en eer hebt Gij hem gekroond, en Gij hebt hem gesteld over de werken Uwer handen;
\par 8 Alle dingen hebt Gij onder zijn voeten onderworpen. Want daarin, dat Hij hem alle dingen heeft onderworpen, heeft Hij niets uitgelaten, dat hem niet onderworpen zij; doch nu zien wij nog niet, dat hem alle dingen onderworpen zijn;
\par 9 Maar wij zien Jezus met heerlijkheid en eer gekroond, Die een weinig minder dan de engelen geworden was, vanwege het lijden des doods, opdat Hij door de genade Gods voor allen den dood smaken zou.
\par 10 Want het betaamde Hem, om Welken alle dingen zijn, en door Welken alle dingen zijn, dat Hij, vele kinderen tot de heerlijkheid leidende, den oversten Leidsman hunner zaligheid door lijden zou heiligen.
\par 11 Want en Hij, Die heiligt, en zij, die geheiligd worden, zijn allen uit een; om welke oorzaak Hij Zich niet schaamt hen broeders te noemen.
\par 12 Zeggende: Ik zal Uw naam Mijn broederen verkondigen; in het midden der Gemeente zal Ik U lofzingen.
\par 13 En wederom: Ik zal Mijn betrouwen op Hem stellen. En wederom: Zie daar, Ik en de kinderen, die Mij God gegeven heeft.
\par 14 Overmits dan de kinderen des vleses en bloeds deelachtig zijn, zo is Hij ook desgelijks derzelve deelachtig geworden, opdat Hij door den dood te niet doen zou dengene, die het geweld des doods had, dat is, den duivel;
\par 15 En verlossen zou al degenen, die met vreze des doods, door al hun leven, der dienstbaarheid onderworpen waren.
\par 16 Want waarlijk, Hij neemt de engelen niet aan, maar Hij neemt het zaad Abrahams aan.
\par 17 Waarom Hij in alles den broederen moest gelijk worden, opdat Hij een barmhartig en een getrouw Hogepriester zou zijn, in de dingen, die bij God te doen waren, om de zonden des volks te verzoenen.
\par 18 Want in hetgeen Hij Zelf, verzocht zijnde, geleden heeft, kan Hij dengenen, die verzocht worden, te hulp komen.

\chapter{3}

\par 1 Hierom, heilige broeders, die der hemelse roeping deelachtig zijt, aanmerkt den Apostel en Hogepriester onzer belijdenis, Christus Jezus;
\par 2 Die getrouw is Dengene, Die Hem gesteld heeft, gelijk ook Mozes in geheel zijn huis was.
\par 3 Want Deze is zoveel meerder heerlijkheid waardig geacht dan Mozes, als degene, die het huis gebouwd heeft, meerder eer heeft, dan het huis.
\par 4 Want een ieder huis wordt van iemand gebouwd; maar Die dit alles gebouwd heeft, is God.
\par 5 En Mozes is wel getrouw geweest in geheel zijn huis, als een dienaar, tot getuiging der dingen, die daarna gesproken zouden worden;
\par 6 Maar Christus, als de Zoon over Zijn eigen huis; Wiens huis wij zijn, indien wij maar de vrijmoedigheid en den roem der hoop tot het einde toe vast behouden.
\par 7 Daarom, gelijk de Heilige Geest zegt: Heden, indien gij Zijn stem hoort,
\par 8 Zo verhardt uw harten niet, gelijk het geschied is in de verbittering, ten dage der verzoeking, in de woestijn;
\par 9 Alwaar Mij uw vaders verzocht hebben; zij hebben Mij beproefd, en hebben Mijn werken gezien, veertig jaren lang.
\par 10 Daarom was Ik vertoornd over dat geslacht, en sprak: Altijd dwalen zij met het hart, en zij hebben Mijn wegen niet gekend.
\par 11 Zo heb Ik dan gezworen in Mijn toorn; Indien zij in Mijn rust zullen ingaan!
\par 12 Ziet toe, broeders, dat niet te eniger tijd in iemand van u zij een boos, ongelovig hart, om af te wijken van den levenden God;
\par 13 Maar vermaant elkander te allen dage, zolang als het heden genaamd wordt, opdat niet iemand uit u verhard worde door de verleiding der zonde.
\par 14 Want wij zijn Christus deelachtig geworden, zo wij anders het beginsel van dezen vasten grond tot het einde toe vast behouden;
\par 15 Terwijl er gezegd wordt: Heden, indien gij Zijn stem hoort, zo verhardt uw harten niet, gelijk in de verbittering geschied is.
\par 16 Want sommigen, als zij die gehoord hadden, hebben Hem verbitterd, doch niet allen, die uit Egypte door Mozes uitgegaan zijn.
\par 17 Over welke nu is Hij vertoornd geweest veertig jaren? Was het niet over degenen, die gezondigd hadden, welker lichamen gevallen zijn in de woestijn?
\par 18 En welken heeft Hij gezworen, dat zij in Zijn rust niet zouden ingaan, anders dan dengenen, die ongehoorzaam geweest waren?
\par 19 En wij zien, dat zij niet hebben kunnen ingaan vanwege hun ongeloof.

\chapter{4}

\par 1 Laat ons dan vrezen, dat niet te eniger tijd, de belofte van in Zijn rust in te gaan nagelaten zijnde, iemand van u schijne achtergebleven te zijn.
\par 2 Want ook ons is het Evangelie verkondigd, gelijk als hun; maar het woord der prediking deed hun geen nut, dewijl het met het geloof niet gemengd was in degenen, die het gehoord hebben.
\par 3 Want wij, die geloofd hebben, gaan in de rust, gelijk Hij gezegd heeft: Zo heb Ik dan gezworen in Mijn toorn: Indien zij zullen ingaan in Mijn rust! hoewel Zijn werken van de grondlegging der wereld af al volbracht waren.
\par 4 Want Hij heeft ergens van den zevenden dag aldus gesproken: En God heeft op den zevenden dag van al Zijn werken gerust.
\par 5 En in deze plaats wederom: Indien zij in Mijn rust zullen ingaan!
\par 6 Dewijl dan blijft, dat sommigen in dezelve rust ingaan, en degenen, dien het Evangelie eerst verkondigd was, niet ingegaan zijn vanwege de ongehoorzaamheid,
\par 7 Zo bepaalt Hij wederom een zekeren dag, namelijk heden, door David zeggende, zo langen tijd daarna (gelijkerwijs gezegd is): Heden, indien gij Zijn stem hoort, zo verhardt uw harten niet.
\par 8 Want indien Jozua hen in de rust gebracht heeft, zo had Hij daarna niet gesproken van een anderen dag.
\par 9 Er blijft dan een rust over voor het volk Gods.
\par 10 Want die ingegaan is in zijn rust, heeft zelf ook van zijn werken gerust, gelijk God van de Zijne.
\par 11 Laat ons dan ons benaarstigen, om in die rust in te gaan; opdat niet iemand in hetzelfde voorbeeld der ongelovigheid valle.
\par 12 Want het Woord Gods is levend en krachtig, en scherpsnijdender dan enig tweesnijdend zwaard, en gaat door tot de verdeling der ziel, en des geestes, en der samenvoegselen, en des mergs, en is een oordeler der gedachten en der overleggingen des harten.
\par 13 En er is geen schepsel onzichtbaar voor Hem; maar alle dingen zijn naakt en geopend voor de ogen Desgenen, met Welken wij te doen hebben.
\par 14 Dewijl wij dan een groten Hogepriester hebben, Die door de hemelen doorgegaan is, namelijk Jezus, den Zoon van God, zo laat ons deze belijdenis vasthouden.
\par 15 Want wij hebben geen hogepriester, die niet kan medelijden hebben met onze zwakheden, maar Die in alle dingen, gelijk als wij, is verzocht geweest, doch zonder zonde.
\par 16 Laat ons dan met vrijmoedigheid toegaan tot den troon der genade, opdat wij barmhartigheid mogen verkrijgen, en genade vinden, om geholpen te worden ter bekwamer tijd.

\chapter{5}

\par 1 Want alle hogepriester, uit de mensen genomen, wordt gesteld voor de mensen in de zaken, die bij God te doen zijn, opdat hij offere gaven en slachtofferen voor de zonden;
\par 2 Die behoorlijk medelijden kan hebben met de onwetenden en dwalenden, overmits hij ook zelf met zwakheid omvangen is;
\par 3 En om derzelver zwakheid wil moet hij gelijk voor het volk, alzo ook voor zichzelven, offeren voor de zonden.
\par 4 En niemand neemt zichzelven die eer aan, maar die van God geroepen wordt, gelijkerwijs als Aaron.
\par 5 Alzo heeft ook Christus Zichzelven niet verheerlijkt, om Hogepriester te worden, maar Die tot Hem gesproken heeft: Gij zijt Mijn Zoon, heden heb Ik U gegenereerd.
\par 6 Gelijk Hij ook in een andere plaats zegt: Gij zijt Priester in der eeuwigheid, naar de ordening van Melchizedek.
\par 7 Die in de dagen Zijns vleses, gebeden en smekingen tot Dengene, Die Hem uit den dood kon verlossen, met sterke roeping en tranen geofferd hebbende, en verhoord zijnde uit de vreze.
\par 8 Hoewel Hij de Zoon was, nochtans gehoorzaamheid geleerd heeft, uit hetgeen Hij heeft geleden.
\par 9 En geheiligd zijnde, is Hij allen, die Hem gehoorzaam zijn, een oorzaak der eeuwige zaligheid geworden;
\par 10 En is van God genaamd een Hogepriester, naar de ordening van Melchizedek.
\par 11 Van Denwelken wij hebben vele dingen, en zwaar om te verklaren, te zeggen, dewijl gij traag om te horen geworden zijt.
\par 12 Want gij, daar gij leraars behoordet te zijn vanwege den tijd, hebt wederom van node, dat men u lere, welke de eerste beginselen zijn der woorden Gods; en gij zijt geworden, als die melk van node hebben, en niet vaste spijze.
\par 13 Want een iegelijk, die der melk deelachtig is, die is onervaren in het woord der gerechtigheid; want hij is een kind.
\par 14 Maar der volmaakten is de vaste spijze, die door de gewoonheid de zinnen geoefend hebben, tot onderscheiding beide des goeds en des kwaads.

\chapter{6}

\par 1 Daarom, nalatende het beginsel der leer van Christus, laat ons tot de volmaaktheid voortvaren; niet wederom leggende het fondament van de bekering van dode werken, en van het geloof in God,
\par 2 Van de leer der dopen, en van de oplegging der handen, en van de opstanding der doden, en van het eeuwig oordeel.
\par 3 En dit zullen wij ook doen, indien het God toelaat.
\par 4 Want het is onmogelijk, degenen, die eens verlicht geweest zijn, en de hemelse gave gesmaakt hebben, en des Heiligen Geestes deelachtig geworden zijn,
\par 5 En gesmaakt hebben het goede woord Gods, en de krachten der toekomende eeuw,
\par 6 En afvallig worden, die, zeg ik, wederom te vernieuwen tot bekering, als welke zichzelven den Zoon van God wederom kruisigen en openlijk te schande maken.
\par 7 Want de aarde, die den regen, menigmaal op haar komende, indrinkt, en bekwaam kruid voortbrengt voor degenen, door welke zij ook gebouwd wordt, die ontvangt zegen van God;
\par 8 Maar die doornen en distelen draagt, die is verwerpelijk, en nabij de vervloeking, welker einde is tot verbranding.
\par 9 Maar, geliefden! wij verzekeren ons van u betere dingen, en met de zaligheid gevoegd, hoewel wij alzo spreken.
\par 10 Want God is niet onrechtvaardig dat Hij uw werk zou vergeten, en den arbeid der liefde, die gij aan Zijn Naam bewezen hebt, als die de heiligen gediend hebt en nog dient.
\par 11 Maar wij begeren, dat een iegelijk van u dezelfde naarstigheid bewijze, tot de volle verzekerdheid der hoop, tot het einde toe;
\par 12 Opdat gij niet traag wordt, maar navolgers zijt dergenen, die door geloof en lankmoedigheid de beloftenissen beerven.
\par 13 Want als God aan Abraham de belofte deed, dewijl Hij bij niemand, die meerder was, had te zweren, zo zwoer Hij bij Zichzelven,
\par 14 Zeggende: Waarlijk, zegenende zal Ik u zegenen, en vermenigvuldigende zal Ik u vermenigvuldigen.
\par 15 En alzo, lankmoediglijk verwacht hebbende, heeft hij de belofte verkregen.
\par 16 Want de mensen zweren wel bij den meerdere dan zij zijn, en de eed tot bevestiging is denzelven een einde van alle tegenspreken;
\par 17 Waarin God, willende den erfgenamen der beloftenis overvloediger bewijzen de onveranderlijkheid van Zijn raad, met een eed daartussen is gekomen;
\par 18 Opdat wij, door twee onveranderlijke dingen, in welke het onmogelijk is dat God liege, een sterke vertroosting zouden hebben, wij namelijk, die de toevlucht genomen hebben, om de voorgestelde hoop vast te houden;
\par 19 Welke wij hebben als een anker der ziel, hetwelk zeker en vast is, en ingaat in het binnenste van het voorhangsel;
\par 20 Daar de Voorloper voor ons is ingegaan, namelijk Jezus, naar de ordening van Melchizedek, een Hogepriester geworden zijnde in der eeuwigheid.

\chapter{7}

\par 1 Want deze Melchizedek was koning van Salem, een priester des Allerhoogsten Gods, die Abraham tegemoet ging, als hij wederkeerde van het slaan der koningen, en hem zegende;
\par 2 Aan welken ook Abraham van alles de tienden deelde; die vooreerst overgezet wordt, koning der gerechtigheid, en daarna ook was een koning van Salem, hetwelk is een koning des vredes;
\par 3 Zonder vader, zonder moeder, zonder geslachtsrekening, noch beginsel der dagen, noch einde des levens hebbende; maar den Zoon van God gelijk geworden zijnde, blijft hij een priester in eeuwigheid.
\par 4 Aanmerkt nu, hoe groot deze geweest zij, aan denwelken ook Abraham, de patriarch, tienden gegeven heeft uit den buit.
\par 5 En die uit de kinderen van Levi het priesterdom ontvangen, hebben wel bevel om tienden te nemen van het volk, naar de wet, dat is, van hun broederen, hoewel die uit de lenden van Abraham voortgekomen zijn.
\par 6 Maar hij, die zijn geslachtsrekening uit hen niet heeft, die heeft van Abraham tienden genomen, en hem, die de beloftenissen had, heeft hij gezegend.
\par 7 Nu, zonder enig tegenspreken, hetgeen minder is, wordt gezegend van hetgeen meerder is.
\par 8 En hier nemen wel tienden de mensen, die sterven, maar aldaar neemt ze die, van welken getuigd wordt, dat hij leeft.
\par 9 En, om zo te spreken, ook Levi, die tienden neemt, heeft door Abraham tienden gegeven;
\par 10 Want hij was nog in de lenden des vaders, als hem Melchizedek tegemoet ging.
\par 11 Indien dan nu de volkomenheid door het Levietische priesterschap ware (want onder hetzelve heeft het volk de wet ontvangen), wat nood was het nog, dat een ander priester naar de ordening van Melchizedek zou opstaan, en die niet zou gezegd worden te zijn naar de ordening van Aaron?
\par 12 Want het priesterschap veranderd zijnde, zo geschiedt er ook noodzakelijk verandering der wet.
\par 13 Want Hij, op Wien deze dingen gezegd worden, behoort tot een anderen stam, van welken niemand zich tot het altaar begeven heeft.
\par 14 Want het is openbaar, dat onze Heere uit Juda gesproten is; op welken stam Mozes niets gesproken heeft van het priesterschap.
\par 15 En dit is nog veel meer openbaar, zo er naar de gelijkenis van Melchizedek een ander priester opstaat:
\par 16 Die dit niet naar de wet des vleselijken gebods is geworden, maar naar de kracht des onvergankelijken levens.
\par 17 Want Hij getuigt: Gij zijt Priester in der eeuwigheid naar de ordening van Melchizedek.
\par 18 Want de vernietiging van het voorgaande gebod geschiedt om deszelfs zwakheids en onprofijtelijkheids wil;
\par 19 Want de wet heeft geen ding volmaakt, maar de aanleiding van een betere hoop, door welke wij tot God genaken.
\par 20 En voor zoveel het niet zonder eedzwering is geschied, (want genen zijn wel zonder eedzwering priesters geworden;
\par 21 Maar Deze met eedzwering, door Dien, Die tot Hem gezegd heeft: De Heere heeft gezworen, en het zal Hem niet berouwen: Gij zijt Priester in der eeuwigheid naar de ordening van Melchizedek).
\par 22 Van een zoveel beter verbond is Jezus Borg geworden.
\par 23 En genen zijn wel vele priesters geworden, omdat zij door den dood verhinderd werden altijd te blijven;
\par 24 Maar Deze, omdat Hij in der eeuwigheid blijft, heeft een onvergankelijk Priesterschap.
\par 25 Waarom Hij ook volkomenlijk kan zalig maken degenen, die door Hem tot God gaan, alzo Hij altijd leeft om voor hen te bidden.
\par 26 Want zodanig een Hogepriester betaamde ons, heilig, onnozel, onbesmet, afgescheiden van de zondaren, en hoger dan de hemelen geworden;
\par 27 Dien het niet allen dag nodig was, gelijk den hogepriesters, eerst voor zijn eigen zonden slachtofferen op te offeren, daarna, voor de zonden des volks; want dat heeft Hij eenmaal gedaan, als Hij Zichzelven opgeofferd heeft.
\par 28 Want de wet stelt tot hogepriesters mensen, die zwakheid hebben; maar het woord der eedzwering, die na de wet is gevolgd, stelt den Zoon, Die in der eeuwigheid geheiligd is.

\chapter{8}

\par 1 De hoofdsom nu der dingen, waarvan wij spreken, is, dat wij hebben zodanigen Hogepriester, Die gezeten is aan de rechter hand van den troon der Majesteit in de hemelen:
\par 2 Een Bedienaar des heiligdoms, en des waren tabernakels, welken de Heere heeft opgericht, en geen mens.
\par 3 Want een iegelijk hogepriester wordt gesteld, om gaven en slachtofferen te offeren; waarom het noodzakelijk was, dat ook Deze wat had, dat Hij zou offeren.
\par 4 Want indien Hij op aarde ware, zo zou Hij zelfs geen Priester zijn, dewijl er priesters zijn, die naar de wet gaven offeren;
\par 5 Welke het voorbeeld en de schaduw der hemelse dingen dienen, gelijk Mozes door Goddelijke aanspraak vermaand was, als hij den tabernakel volmaken zou: Want zie, zegt Hij, dat gij het alles maakt naar de afbeelding, die u op den berg getoond is.
\par 6 En nu heeft Hij zoveel uitnemender bediening gekregen, als Hij ook eens beteren verbonds Middelaar is, hetwelk in betere beloftenissen bevestigd is.
\par 7 Want indien dat eerste verbond onberispelijk geweest ware, zo zou voor het tweede geen plaats gezocht zijn geweest.
\par 8 Want hen berispende, zegt Hij tot hen: Ziet, de dagen komen, spreekt de Heere, en Ik zal over het huis Israels, en over het huis van Juda een nieuw verbond oprichten;
\par 9 Niet naar het verbond, dat Ik met hun vaderen gemaakt heb, ten dage, als Ik hen bij de hand nam, om hen uit Egypteland te leiden; want zij zijn in dit Mijn verbond niet gebleven, en Ik heb op hen niet geacht, zegt de Heere.
\par 10 Want dit is het verbond, dat Ik met het huis Israels maken zal na die dagen, zegt de Heere: Ik zal Mijn wetten in hun verstand geven, en in hun harten zal Ik die inschrijven; en Ik zal hun tot een God zijn, en zij zullen Mij tot een volk zijn.
\par 11 En zij zullen niet leren, een iegelijk zijn naaste, en een iegelijk zijn broeder, zeggende: Ken den Heere; want zij zullen Mij allen kennen van den kleine onder hen tot den grote onder hen.
\par 12 Want Ik zal hun ongerechtigheden genadig zijn, en hun zonden en hun overtredingen zal Ik geenszins meer gedenken.
\par 13 Als Hij zegt: Een nieuw verbond, zo heeft Hij het eerste oud gemaakt; dat nu oud gemaakt is en verouderd, is nabij de verdwijning.

\chapter{9}

\par 1 Zo had dan wel ook het eerste verbond rechten van de gods dienst, en het wereldlijk heiligdom.
\par 2 Want de tabernakel was toebereid, namelijk de eerste, in welken was de kandelaar, en de tafel, en de toonbroden, welke genaamd wordt het heilige;
\par 3 Maar achter het tweede voorhangsel was de tabernakel, genaamd het heilige der heiligen;
\par 4 Hebbende een gouden wierookvat, en de ark des verbonds, alom met goud overdekt, in welke was de gouden kruik, daar het Manna in was, en de staf van Aaron, die gebloeid had, en de tafelen des verbonds.
\par 5 En boven over deze ark waren de cherubijnen der heerlijkheid, die het verzoendeksel beschaduwden; van welke dingen wij nu van stuk tot stuk niet zullen zeggen.
\par 6 Deze dingen nu, aldus toebereid zijnde, zo gingen wel de priesters in den eersten tabernakel, te allen tijde, om de gods diensten te volbrengen;
\par 7 Maar in den tweeden tabernakel ging alleen de hogepriester, eenmaal des jaars, niet zonder bloed, hetwelk hij offerde voor zichzelven en voor des volks misdaden.
\par 8 Waarmede de Heilige Geest dit beduidde, dat de weg des heiligdoms nog niet openbaar gemaakt was, zolang de eerste tabernakel nog stand had;
\par 9 Welke was een afbeelding voor dien tegenwoordigen tijd, in welken gaven en slachtofferen geofferd werden, die dengene, die den dienst pleegde, niet konden heiligen naar het geweten;
\par 10 Bestaande alleen in spijzen, en dranken, en verscheidene wassingen en rechtvaardigmakingen des vleses, tot op den tijd der verbetering opgelegd.
\par 11 Maar Christus, de Hogepriester der toekomende goederen, gekomen zijnde, is door den meerderen en volmaakten tabernakel, niet met handen gemaakt, dat is, niet van dit maaksel,
\par 12 Noch door het bloed der bokken en kalveren, maar door Zijn eigen bloed, eenmaal ingegaan in het heiligdom, een eeuwige verlossing teweeggebracht hebbende.
\par 13 Want indien het bloed der stieren en bokken, en de as der jonge koe, besprengende de onreinen, hen heiligt tot de reinigheid des vleses;
\par 14 Hoeveel te meer zal het bloed van Christus, Die door den eeuwigen Geest Zichzelven Gode onstraffelijk opgeofferd heeft, uw geweten reinigen van dode werken, om den levenden God te dienen?
\par 15 En daarom is Hij de Middelaar des nieuwen testaments, opdat, de dood daartussen gekomen zijnde, tot verzoening der overtredingen, die onder het eerste testament waren, degenen, die geroepen zijn, de beloftenis der eeuwige erve ontvangen zouden.
\par 16 Want waar een testament is, daar is het noodzaak, dat de dood des testamentmakers tussen kome;
\par 17 Want een testament is vast in de doden, dewijl het nog geen kracht heeft, wanneer de testamentmaker leeft.
\par 18 Waarom ook het eerste niet zonder bloed is ingewijd.
\par 19 Want als al de geboden, naar de wet van Mozes, tot al het volk uitgesproken waren, nam hij het bloed der kalveren en bokken, met water, en purperen wol, en hysop, besprengde beide het boek zelf, en al het volk,
\par 20 Zeggende: Dit is het bloed des testaments, hetwelk God aan ulieden heeft geboden.
\par 21 En hij besprengde desgelijks ook den tabernakel, en al de vaten van den dienst met het bloed.
\par 22 En alle dingen worden bijna door bloed gereinigd naar de wet, en zonder bloedstorting geschiedt geen vergeving.
\par 23 Zo was het dan noodzaak, dat wel de voorbeeldingen der dingen, die in de hemelen zijn, door deze dingen gereinigd werden, maar de hemelse dingen zelve door betere offeranden dan deze.
\par 24 Want Christus is niet ingegaan in het heiligdom, dat met handen gemaakt is, hetwelk is een tegenbeeld van het ware, maar in den hemel zelven, om nu te verschijnen voor het aangezicht van God voor ons;
\par 25 Noch ook, opdat Hij Zichzelven dikwijls zou opofferen, gelijk de hogepriester alle jaar in het heiligdom ingaat met vreemd bloed;
\par 26 (Anders had Hij dikwijls moeten lijden van de grondlegging der wereld af) maar nu is Hij eenmaal in de voleinding der eeuwen geopenbaard, om de zonde te niet te doen, door Zijnszelfs offerande.
\par 27 En gelijk het den mensen gezet is, eenmaal te sterven, en daarna het oordeel;
\par 28 Alzo ook Christus, eenmaal geofferd zijnde, om veler zonden weg te nemen, zal ten anderen male zonder zonde gezien worden van degenen, die Hem verwachten tot zaligheid.

\chapter{10}

\par 1 Want de wet, hebbende een schaduw der toekomende goederen, niet het beeld zelf der zaken, kan met dezelfde offeranden, die zij alle jaren geduriglijk opofferen, nimmermeer heiligen degenen, die daar toegaan.
\par 2 Anderszins zouden zij opgehouden hebben, geofferd te worden, omdat degenen, die den dienst pleegden, geen geweten meer zouden hebben der zonden, eenmaal gereinigd geweest zijnde;
\par 3 Maar nu geschiedt in dezelve alle jaren weder gedachtenis der zonden.
\par 4 Want het is onmogelijk, dat het bloed van stieren en bokken de zonden wegneme.
\par 5 Daarom, komende in de wereld, zegt Hij: Slachtoffer en offerande hebt Gij niet gewild, maar Gij hebt Mij het lichaam toebereid;
\par 6 Brandofferen en offer voor de zonde hebben U niet behaagd.
\par 7 Toen sprak Ik: Zie, Ik kom (in het begin des boeks is van Mij geschreven), om Uw wil te doen, o God!
\par 8 Als Hij te voren gezegd had: Slachtoffer, en offerande, en brandoffers, en offer voor de zonde hebt Gij niet gewild, noch hebben U behaagd (dewelke naar de wet geofferd worden);
\par 9 Toen sprak Hij: Zie, Ik kom, om Uw wil te doen, o God! Hij neemt het eerste weg, om het tweede te stellen.
\par 10 In welken wil wij geheiligd zijn, door de offerande des lichaams van Jezus Christus, eenmaal geschied.
\par 11 En een iegelijk priester stond wel alle dagen dienende, en dezelfde slachtofferen dikmaals offerende, die de zonden nimmermeer kunnen wegnemen;
\par 12 Maar Deze, een slachtoffer voor de zonden geofferd hebbende, is in eeuwigheid gezeten aan de rechter hand Gods;
\par 13 Voorts verwachtende, totdat Zijn vijanden gesteld worden tot een voetbank Zijner voeten.
\par 14 Want met een offerande heeft Hij in eeuwigheid volmaakt degenen, die geheiligd worden.
\par 15 En de Heilige Geest getuigt het ons ook;
\par 16 Want nadat Hij te voren gezegd had: Dit is het verbond, dat Ik met hen maken zal na die dagen, zegt de Heere: Ik zal Mijn wetten geven in hun harten, en Ik zal die inschrijven in hun verstanden;
\par 17 En hun zonden en hun ongerechtigheden zal Ik geenszins meer gedenken.
\par 18 Waar nu vergeving derzelve is, daar is geen offerande meer voor de zonde.
\par 19 Dewijl wij dan, broeders, vrijmoedigheid hebben, om in te gaan in het heiligdom door het bloed van Jezus,
\par 20 Op een versen en levenden weg, welken Hij ons ingewijd heeft door het voorhangsel, dat is, door Zijn vlees;
\par 21 En dewijl wij hebben een groten Priester over het huis Gods;
\par 22 Zo laat ons toegaan met een waarachtig hart, in volle verzekerdheid des geloofs, onze harten gereinigd zijnde van het kwaad geweten, en het lichaam gewassen zijnde met rein water.
\par 23 Laat ons de onwankelbare belijdenis der hoop vast houden; (want Die het beloofd heeft, is getrouw);
\par 24 En laat ons op elkander acht nemen, tot opscherping der liefde en der goede werken;
\par 25 En laat ons onze onderlinge bijeenkomst niet nalaten, gelijk sommigen de gewoonte hebben, maar elkander vermanen; en dat zoveel te meer, als gij ziet, dat de dag nadert.
\par 26 Want zo wij willens zondigen, nadat wij de kennis der waarheid ontvangen hebben, zo blijft er geen slachtoffer meer over voor de zonden;
\par 27 Maar een schrikkelijke verwachting des oordeels, en hitte des vuurs, dat de tegenstanders zal verslinden.
\par 28 Als iemand de wet van Mozes heeft te niet gedaan, die sterft zonder barmhartigheid, onder twee of drie getuigen;
\par 29 Hoeveel te zwaarder straf, meent gij, zal hij waardig geacht worden, die den Zoon van God vertreden heeft, en het bloed des testaments onrein geacht heeft, waardoor hij geheiligd was, en den Geest der genade smaadheid heeft aangedaan?
\par 30 Want wij kennen Hem, Die gezegd heeft: Mijn is de wraak, Ik zal het vergelden, spreekt de Heere. En wederom: De Heere zal Zijn volk oordelen.
\par 31 Vreselijk is het te vallen in de handen des levenden Gods.
\par 32 Doch gedenkt de vorige dagen, in dewelke, nadat gij verlicht zijt geweest, gij veel strijd des lijdens hebt verdragen.
\par 33 Ten dele, als gij door smaadheden en verdrukkingen een schouwspel geworden zijt; en ten dele, als gij gemeenschap gehad hebt met degenen, die alzo behandeld werden.
\par 34 Want gij hebt ook over mijn banden medelijden gehad, en de roving uwer goederen met blijdschap aangenomen, wetende, dat gij hebt in uzelven een beter en blijvend goed in de hemelen.
\par 35 Werpt dan uw vrijmoedigheid niet weg, welke een grote vergelding des loons heeft.
\par 36 Want gij hebt lijdzaamheid van node, opdat gij, den wil van God gedaan hebbende, de beloftenis moogt wegdragen;
\par 37 Want: Nog een zeer weinig tijds en Hij, Die te komen staat, zal komen, en niet vertoeven.
\par 38 Maar de rechtvaardige zal uit het geloof leven; en zo iemand zich onttrekt, Mijn ziel heeft in hem geen behagen.
\par 39 Maar wij zijn niet van degenen, die zich onttrekken ten verderve, maar van degenen, die geloven tot behouding der ziel.

\chapter{11}

\par 1 Het geloof nu is een vaste grond der dingen, die men hoopt, en een bewijs der zaken, die men niet ziet.
\par 2 Want door hetzelve hebben de ouden getuigenis bekomen.
\par 3 Door het geloof verstaan wij, dat de wereld door het woord Gods is toebereid, alzo dat de dingen, die men ziet, niet geworden zijn uit dingen, die gezien worden.
\par 4 Door het geloof heeft Abel een meerdere offerande Gode geofferd dan Kain, door hetwelk hij getuigenis bekomen heeft, dat hij rechtvaardig was, alzo God over zijn gave getuigenis gaf; en door hetzelve geloof spreekt hij nog, nadat hij gestorven is.
\par 5 Door het geloof is Enoch weggenomen geweest, opdat hij den dood niet zou zien; en hij werd niet gevonden, daarom dat hem God weggenomen had; want voor zijn wegneming heeft hij getuigenis gehad, dat hij Gode behaagde.
\par 6 Maar zonder geloof is het onmogelijk Gode te behagen. Want die tot God komt, moet geloven, dat Hij is, en een Beloner is dergenen, die Hem zoeken.
\par 7 Door het geloof heeft Noach, door Goddelijke aanspraak vermaand zijnde van de dingen, die nog niet gezien werden, en bevreesd geworden zijnde, de ark toebereid tot behoudenis van zijn huisgezin; door welke ark hij de wereld heeft veroordeeld, en is geworden een erfgenaam der rechtvaardigheid, die naar het geloof is.
\par 8 Door het geloof is Abraham, geroepen zijnde, gehoorzaam geweest, om uit te gaan naar de plaats, die hij tot een erfdeel ontvangen zou; en hij is uitgegaan, niet wetende, waar hij komen zou.
\par 9 Door het geloof is hij een inwoner geweest in het land der belofte, als in een vreemd land, en heeft in tabernakelen gewoond met Izak en Jakob, die medeerfgenamen waren derzelfde belofte.
\par 10 Want hij verwachtte de stad, die fondamenten heeft, welker Kunstenaar en Bouwmeester God is.
\par 11 Door het geloof heeft ook Sara zelve kracht ontvangen, om zaad te geven, en boven den tijd haars ouderdoms heeft zij gebaard; overmits zij Hem getrouw heeft geacht, Die het beloofd had.
\par 12 Daarom zijn ook van een, en dat een verstorvene, zovelen in menigte geboren, als de sterren des hemels, en als het zand, dat aan den oever der zee is, hetwelk ontallijk is.
\par 13 Deze allen zijn in het geloof gestorven, de beloften niet verkregen hebbende, maar hebben dezelve van verre gezien, en geloofd, en omhelsd, en hebben beleden, dat zij gasten en vreemdelingen op de aarde waren.
\par 14 Want die zulke dingen zeggen, betonen klaarlijk, dat zij een vaderland zoeken.
\par 15 En indien zij aan dat vaderland gedacht hadden, van hetwelk zij uitgegaan waren, zij zouden tijd gehad hebben, om weder te keren;
\par 16 Maar nu zijn zij begerig naar een beter, dat is, naar het hemelse. Daarom schaamt Zich God hunner niet, om hun God genaamd te worden; want Hij had hun een stad bereid.
\par 17 Door het geloof heeft Abraham, als hij verzocht werd, Izak geofferd, en hij, die de beloften ontvangen had, heeft zijn eniggeborene geofferd,
\par 18 (Tot denwelken gezegd was: In Izak zal u het zaad genoemd worden) overleggende, dat God machtig was, hem ook uit de doden te verwekken;
\par 19 Waaruit hij hem ook bij gelijkenis wedergekregen heeft.
\par 20 Door het geloof heeft Izak zijn zonen Jakob en Ezau gezegend aangaande toekomende dingen.
\par 21 Door het geloof heeft Jakob, stervende, een iegelijk der zonen van Jozef gezegend, en heeft aangebeden, leunende op het opperste van zijn staf.
\par 22 Door het geloof heeft Jozef, stervende, gemeld van den uitgang der kinderen Israels, en heeft bevel gegeven van zijn gebeente.
\par 23 Door het geloof werd Mozes, toen hij geboren was, drie maanden lang van zijn ouders verborgen, overmits zij zagen, dat het kindeken schoon was; en zij vreesden het gebod des konings niet.
\par 24 Door het geloof heeft Mozes, nu groot geworden zijnde, geweigerd een zoon van Farao's dochter genoemd te worden;
\par 25 Verkiezende liever met het volk van God kwalijk gehandeld te worden, dan voor een tijd de genieting der zonde te hebben;
\par 26 Achtende de versmaadheid van Christus meerderen rijkdom te zijn, dan de schatten in Egypte; want hij zag op de vergelding des loons.
\par 27 Door het geloof heeft hij Egypte verlaten, niet vrezende den toorn des konings; want hij hield zich vast, als ziende den Onzienlijke.
\par 28 Door het geloof heeft hij het pascha uitgericht, en de besprenging des bloeds, opdat de verderver der eerstgeborenen hen niet raken zou.
\par 29 Door het geloof zijn zij de Rode zee doorgegaan, als door het droge; hetwelk de Egyptenaars, ook verzoekende, zijn verdronken.
\par 30 Door het geloof zijn de muren van Jericho gevallen, als zij tot zeven dagen toe omringd waren geweest.
\par 31 Door het geloof is Rachab, de hoer, niet omgekomen met de ongehoorzamen, als zij de verspieders met vrede had ontvangen.
\par 32 En wat zal ik nog meer zeggen? Want de tijd zal mij ontbreken, zou ik verhalen van Gideon, en Barak, en Samson, en Jeftha, en David, en Samuel, en de profeten;
\par 33 Welken door het geloof koninkrijken hebben overwonnen, gerechtigheid geoefend, de beloftenissen verkregen, de muilen der leeuwen toegestopt;
\par 34 De kracht des vuurs hebben uitgeblust, de scherpte des zwaards zijn ontvloden, uit zwakheid krachten hebben gekregen, in den krijg sterk geworden zijn, heirlegers der vreemden op de vlucht hebben gebracht;
\par 35 De vrouwen hebben hare doden uit de opstanding weder gekregen; en anderen zijn uitgerekt geworden, de aangeboden verlossing niet aannemende, opdat zij een betere opstanding verkrijgen zouden.
\par 36 En anderen hebben bespottingen en geselen geproefd, en ook banden en gevangenis;
\par 37 Zijn gestenigd geworden, in stukken gezaagd, verzocht, door het zwaard ter dood gebracht; hebben gewandeld in schaapsvellen en in geitenvellen; verlaten, verdrukt, kwalijk gehandeld zijnde;
\par 38 (Welker de wereld niet waardig was) hebben in woestijnen gedoold, en op bergen, en in spelonken, en in holen der aarde.
\par 39 En deze allen, hebbende door het geloof getuigenis gehad, hebben de belofte niet verkregen;
\par 40 Alzo God wat beters over ons voorzien had, opdat zij zonder ons niet zouden volmaakt worden.

\chapter{12}

\par 1 Daarom dan ook, alzo wij zo groot een wolk der getuigen rondom ons hebben liggende, laat ons afleggen allen last, en de zonde, die ons lichtelijk omringt, en laat ons met lijdzaamheid lopen de loopbaan, die ons voorgesteld is;
\par 2 Ziende op den oversten Leidsman en Voleinder des geloofs, Jezus, Dewelke, voor de vreugde, die Hem voorgesteld was, het kruis heeft verdragen, en schande veracht, en is gezeten aan de rechter hand des troons van God.
\par 3 Want aanmerkt Dezen, Die zodanig een tegenspreken van de zondaren tegen Zich heeft verdragen, opdat gij niet verflauwt en bezwijkt in uw zielen.
\par 4 Gij hebt nog tot den bloede toe niet tegengestaan, strijdende tegen de zonde;
\par 5 En gij hebt vergeten de vermaning, die tot u als tot zonen spreekt: Mijn zoon, acht niet klein de kastijding des Heeren, en bezwijkt niet, als gij van Hem bestraft wordt;
\par 6 Want dien de Heere liefheeft, kastijdt Hij, en Hij geselt een iegelijken zoon, die Hij aanneemt.
\par 7 Indien gij de kastijding verdraagt, zo gedraagt Zich God jegens u als zonen; (want wat zoon is er, dien de vader niet kastijdt?)
\par 8 Maar indien gij zonder kastijding zijt, welke allen deelachtig zijn geworden, zo zijt gij dan bastaarden, en niet zonen.
\par 9 Voorts, wij hebben de vaders onzes vleses wel tot kastijders gehad, en wij ontzagen hen; zullen wij dan niet veel meer den Vader der geesten onderworpen zijn, en leven?
\par 10 Want genen hebben ons wel voor een korten tijd, naar dat het hun goed dacht, gekastijd; maar Deze kastijdt ons tot ons nut, opdat wij Zijner heiligheid zouden deelachtig worden.
\par 11 En alle kastijding als die tegenwoordig is, schijnt geen zaak van vreugde, maar van droefheid te zijn; doch daarna geeft zij van zich een vreedzame vrucht der gerechtigheid dengenen, die door dezelve geoefend zijn.
\par 12 Daarom richt weder op de trage handen, en de slappe knieen;
\par 13 En maakt rechte paden voor uw voeten, opdat hetgeen kreupel is, niet verdraaid worde, maar dat het veelmeer genezen worde.
\par 14 Jaagt den vrede na met allen, en de heiligmaking, zonder welke niemand den Heere zien zal;
\par 15 Toeziende, dat niet iemand verachtere van de genade Gods; dat niet enige wortel der bitterheid, opwaarts spruitende, beroerte make en door dezelve velen ontreinigd worden.
\par 16 Dat niet iemand zij een hoereerder, of een onheilige, gelijk Ezau, die om een spijze het recht van zijn eerstgeboorte weggaf.
\par 17 Want gij weet, dat hij ook daarna, de zegening willende beerven, verworpen werd; want hij vond geen plaats des berouws, hoewel hij dezelve met tranen zocht.
\par 18 Want gij zijt niet gekomen tot den tastelijken berg, en het brandende vuur, en donkerheid, en duisternis, en onweder,
\par 19 En tot het geklank der bazuin, en de stem der woorden; welke die ze hoorden, baden, dat het woord tot hen niet meer zou gedaan worden.
\par 20 (Want zij konden niet dragen, hetgeen er geboden werd: Indien ook een gedierte den berg aanraakt, het zal gestenigd of met een pijl doorschoten worden.
\par 21 En Mozes, zo vreselijk was het gezicht, zeide: Ik ben gans bevreesd en bevende).
\par 22 Maar gij zijt gekomen tot den berg Sion, en de stad des levenden Gods, tot het hemelse Jeruzalem, en de vele duizenden der engelen;
\par 23 Tot de algemene vergadering en de Gemeente der eerstgeborenen, die in de hemelen opgeschreven zijn, en tot God, den Rechter over allen, en de geesten der volmaakte rechtvaardigen;
\par 24 En tot den Middelaar des nieuwen testaments, Jezus, en het bloed der besprenging, dat betere dingen spreekt dan Abel.
\par 25 Ziet toe, dat gij Dien, Die spreekt, niet verwerpt; want indien dezen niet zijn ontvloden, die dengene verwierpen, welke op aarde Goddelijke antwoorden gaf, veelmeer zullen wij niet ontvlieden, zo wij ons van Dien afkeren, Die van de hemelen is;
\par 26 Wiens stem toen de aarde bewoog; maar nu heeft Hij verkondigd, zeggende: Nog eenmaal zal Ik bewegen niet alleen de aarde, maar ook den hemel.
\par 27 En dit woord: Nog eenmaal, wijst aan de verandering der bewegelijke dingen, als welke gemaakt waren, opdat blijven zouden de dingen, die niet bewegelijk zijn.
\par 28 Daarom, alzo wij een onbewegelijk Koninkrijk ontvangen, laat ons de genade vast houden, door dewelke wij welbehagelijk Gode mogen dienen, met eerbied en godvruchtigheid.
\par 29 Want onze God is een verterend vuur.

\chapter{13}

\par 1 Dat de broederlijke liefde blijve.
\par 2 Vergeet de herbergzaamheid niet; want hierdoor hebben sommigen onwetend engelen geherbergd.
\par 3 Gedenkt der gevangenen, alsof gij mede gevangen waart; en dergenen, die kwalijk gehandeld worden, alsof gij ook zelven in het lichaam kwalijk gehandeld waart.
\par 4 Het huwelijk is eerlijk onder allen, en het bed onbevlekt; maar hoereerders en overspelers zal God oordelen.
\par 5 Uw wandel zij zonder geldgierigheid; en zijt vergenoegd met het tegenwoordige; want Hij heeft gezegd: Ik zal u niet begeven, en Ik zal u niet verlaten.
\par 6 Zodat wij vrijmoediglijk durven zeggen: De Heere is mij een Helper, en ik zal niet vrezen, wat mij een mens zal doen.
\par 7 Gedenkt uwer voorgangeren, die u het Woord Gods gesproken hebben; en volgt hun geloof na, aanschouwende de uitkomst hunner wandeling.
\par 8 Jezus Christus is gisteren en heden dezelfde en in der eeuwigheid.
\par 9 Wordt niet omgevoerd met verscheidene en vreemde leringen; want het is goed, dat het hart gesterkt wordt door genade, niet door spijzen, door welke geen nuttigheid bekomen hebben, die daarin gewandeld hebben.
\par 10 Wij hebben een altaar, van hetwelk geen macht hebben te eten, die den tabernakel dienen.
\par 11 Want welker dieren bloed voor de zonde gedragen werd in het heiligdom door den hogepriester, derzelver lichamen werden verbrand buiten de legerplaats.
\par 12 Daarom heeft ook Jezus, opdat Hij door Zijn eigen bloed het volk zou heiligen, buiten de poort geleden.
\par 13 Zo laat ons dan tot Hem uitgaan buiten de legerplaats, Zijn smaadheid dragende.
\par 14 Want wij hebben hier geen blijvende stad, maar wij zoeken de toekomende.
\par 15 Laat ons dan door Hem altijd Gode opofferen een offerande des lofs, dat is, de vrucht der lippen, die Zijn Naam belijden.
\par 16 En vergeet de weldadigheid en de mededeelzaamheid niet; want aan zodanige offeranden heeft God een welbehagen.
\par 17 Zijt uw voorgangeren gehoorzaam, en zijt hun onderdanig; want zij waken voor uw zielen, als die rekenschap geven zullen; opdat zij dat doen mogen met vreugde en niet al zuchtende; want dat is u niet nuttig.
\par 18 Bidt voor ons; want wij vertrouwen, dat wij een goed geweten hebben, als die in alles willen eerlijk wandelen.
\par 19 En ik bid u te meer, dat gij dit doet, opdat ik te eerder ulieden moge wedergegeven worden.
\par 20 De God nu des vredes, Die den grote Herder der schapen, door het bloed des eeuwigen testaments, uit de doden heeft wedergebracht, namelijk onze Heere Jezus Christus,
\par 21 Die volmake u in alle goed werk, opdat gij Zijn wil moogt doen; werkende in u, hetgeen voor Hem welbehagelijk is, door Jezus Christus; Denwelken zij de heerlijkheid in alle eeuwigheid. Amen.
\par 22 Doch ik bid u, broeders, verdraagt het woord dezer vermaning; want ik heb u in het kort geschreven.
\par 23 Weet, dat de broeder Timotheus losgelaten is, met welken (zo hij haast komt) ik u zal zien.
\par 24 Groet al uw voorgangeren, en al de heiligen. U groeten die van Italie zijn.
\par 25 De genade zij met u allen. Amen.




\end{document}