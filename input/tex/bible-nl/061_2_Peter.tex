\begin{document}

\title{2 Peter}



\chapter{1}

\par 1 Simeon Petrus, een dienstknecht en apostel van Jezus Christus, aan degenen, die even dierbaar geloof met ons verkregen hebben, door de rechtvaardigheid van onzen God en Zaligmaker, Jezus Christus;
\par 2 Genade en vrede zij u vermenigvuldigd door de kennis van God, en van Jezus, onzen Heere;
\par 3 Gelijk ons Zijn Goddelijke kracht alles, wat tot het leven en de godzaligheid behoort, geschonken heeft, door de kennis Desgenen, Die ons geroepen heeft tot heerlijkheid en deugd;
\par 4 Door welke ons de grootste en dierbare beloften geschonken zijn, opdat gij door dezelve der goddelijke natuur deelachtig zoudt worden, nadat gij ontvloden zijt het verderf, dat in de wereld is door de begeerlijkheid.
\par 5 En gij, tot hetzelve ook alle naarstigheid toebrengende, voegt bij uw geloof deugd, en bij de deugd kennis,
\par 6 En bij de kennis matigheid, en bij de matigheid lijdzaamheid, en bij de lijdzaamheid godzaligheid,
\par 7 En bij de godzaligheid broederlijke liefde, en bij de broederlijke liefde, liefde jegens allen.
\par 8 Want zo deze dingen bij u zijn, en in u overvloedig zijn, zij zullen u niet ledig noch onvruchtbaar laten in de kennis van onzen Heere Jezus Christus.
\par 9 Want bij welken deze dingen niet zijn, die is blind, van verre niet ziende, hebbende vergeten de reiniging zijner vorige zonden.
\par 10 Daarom, broeders, benaarstigt u te meer, om uw roeping en verkiezing vast te maken; want dat doende zult gij nimmermeer struikelen.
\par 11 Want alzo zal u rijkelijk toegevoegd worden de ingang in het eeuwig Koninkrijk van onzen Heere en Zaligmaker, Jezus Christus.
\par 12 Daarom zal ik niet verzuimen u altijd daarvan te vermanen, hoewel gij het weet, en in de tegenwoordige waarheid versterkt zijt.
\par 13 En ik acht het recht te zijn, zolang ik in dezen tabernakel ben, dat ik u opwekke door vermaning;
\par 14 Alzo ik weet, dat de aflegging mijns tabernakels haast zijn zal, gelijkerwijs ook onze Heere Jezus Christus mij heeft geopenbaard.
\par 15 Doch ik zal ook naarstigheid doen bij alle gelegenheid, dat gij na mijn uitgang van deze dingen gedachtenis moogt hebben.
\par 16 Want wij zijn geen kunstelijk verdichte fabelen nagevolgd, als wij u bekend gemaakt hebben de kracht en toekomst van onzen Heere Jezus Christus, maar wij zijn aanschouwers geweest van Zijn majesteit.
\par 17 Want Hij heeft van God den Vader eer en heerlijkheid ontvangen, als zodanig een stem van de hoogwaardige heerlijkheid tot Hem gebracht werd: Deze is Mijn geliefde Zoon, in Denwelken Ik Mijn welbehagen heb.
\par 18 En deze stem hebben wij gehoord, als zij van de hemel gebracht is geweest, toen wij met Hem op den heiligen berg waren.
\par 19 En wij hebben het profetische woord, dat zeer vast is, en gij doet wel, dat gij daarop acht hebt, als op een licht, schijnende in een duistere plaats, totdat de dag aanlichte, en de morgenster opga in uw harten.
\par 20 Dit eerst wetende, dat geen profetie der Schrift is van eigen uitlegging;
\par 21 Want de profetie is voortijds niet voortgebracht door de wil eens mensen, maar de heilige mensen Gods, van den Heiligen Geest gedreven zijnde, hebben ze gesproken.

\chapter{2}

\par 1 En er zijn ook valse profeten onder het volk geweest, gelijk ook onder u valse leraars zijn zullen, die verderfelijke ketterijen bedektelijk invoeren zullen, ook den Heere, Die hen gekocht heeft, verloochenende, en een haastig verderf over zichzelven brengende;
\par 2 En velen zullen hun verderfenissen navolgen, door welke de weg der waarheid zal gelasterd worden.
\par 3 En zij zullen door gierigheid, met gemaakte woorden, van u een koopmanschap maken; over welke het oordeel van over lang niet ledig is, en hun verderf sluimert niet.
\par 4 Want indien God de engelen, die gezondigd hebben, niet gespaard heeft, maar, die in de hel geworpen hebbende, overgegeven heeft aan de ketenen der duisternis, om tot het oordeel bewaard te worden;
\par 5 En de oude wereld niet heeft gespaard, maar Noach, den prediker der gerechtigheid, zijn achttal bewaard heeft, als Hij den zondvloed over de wereld der goddelozen heeft gebracht;
\par 6 En de steden van Sodoma en Gomorra tot as verbrandende met omkering veroordeeld heeft, en tot een voorbeeld gezet dengenen, die goddelooslijk zouden leven;
\par 7 En den rechtvaardigen Lot, die vermoeid was van den ontuchtigen wandel der gruwelijke mensen, daaruit verlost heeft;
\par 8 (Want deze rechtvaardige man, wonende onder hen, heeft dag op dag zijn rechtvaardige ziel gekweld, door het zien en horen van hun ongerechtige werken);
\par 9 Zo weet de Heere de godzaligen uit de verzoeking te verlossen, en de onrechtvaardigen te bewaren tot den dag des oordeels, om gestraft te worden;
\par 10 Maar allermeest degenen, die naar het vlees in onreine begeerlijkheid wandelen, en de heerschappij verachten; die stout zijn, zichzelven behagen, en die de heerlijkheden niet schromen te lasteren;
\par 11 Daar de engelen in sterkte en kracht meerder zijnde, geen lasterlijk oordeel tegen hen voor den Heere voortbrengen.
\par 12 Maar deze, als onredelijke dieren, die de natuur volgen, en voortgebracht zijn om gevangen en gedood te worden, dewijl zij lasteren, hetgeen zij niet verstaan, zullen in hun verdorvenheid verdorven worden;
\par 13 En zullen verkrijgen het loon der ongerechtigheid, als die de dagelijkse weelde hun vermaak achten, zijnde vlekken en smetten, en zijn weelderig in hun bedriegerijen, als zij in de maaltijden met u zijn;
\par 14 Hebbende de ogen vol overspel, en die niet ophouden van zondigen; verlokkende de onvaste zielen, hebbende het hart geoefend in gierigheid, kinderen der vervloeking;
\par 15 Die den rechten weg verlaten hebbende, zijn verdwaald, en volgen den weg van Balaam, den zoon van Bosor, die het loon der ongerechtigheid liefgehad heeft;
\par 16 Maar hij heeft de bestraffing zijner ongerechtigheid gehad; want het jukdragende stomme dier, sprekende met mensenstem, heeft des profeten dwaasheid verhinderd.
\par 17 Deze zijn waterloze fonteinen, wolken van een draaiwind gedreven, denwelken de donkerheid der duisternis in der eeuwigheid bewaard wordt.
\par 18 Want zij, zeer opgeblazene ijdelheid sprekende, verlokken, door de begeerlijkheden des vleses en door ontuchtigheden, degenen, die waarlijk ontvloden waren van degenen, die in dwaling wandelen;
\par 19 Belovende hun vrijheid, daar zijzelven dienstknechten zijn der verdorvenheid; want van wien iemand overwonnen is, dien is hij ook tot een dienstknecht gemaakt.
\par 20 Want indien zij, nadat zij door de kennis van den Heere en Zaligmaker Jezus Christus, de besmettingen der wereld ontvloden zijn, en in dezelve wederom ingewikkeld zijnde, van dezelve overwonnen worden, zo is hun het laatste erger geworden dan het eerste.
\par 21 Want het ware hun beter, dat zij den weg der gerechtigheid niet gekend hadden, dan dat zij, dien gekend hebbende, weder afkeren van het heilige gebod, dat hun overgegeven was.
\par 22 Maar hun is overkomen, hetgeen met een waar spreekwoord gezegd wordt: De hond is wedergekeerd tot zijn eigen uitbraaksel; en de gewassen zeug tot de wenteling in het slijk.

\chapter{3}

\par 1 Dezen tweeden zendbrief, geliefden, schrijf ik nu aan u, in welke beide ik door vermaning uw oprecht gemoed opwekke;
\par 2 Opdat gij gedachtig zijt aan de woorden, die van de heilige profeten te voren gesproken zijn, en aan ons gebod, die des Heeren en Zaligmakers apostelen zijn;
\par 3 Dit eerst wetende, dat in het laatste der dagen spotters komen zullen, die naar hun eigen begeerlijkheden zullen wandelen,
\par 4 En zeggen: Waar is de belofte Zijner toekomst? want van dien dag, dat de vaders ontslapen zijn, blijven alle dingen alzo gelijk van het begin der schepping.
\par 5 Want willens is dit hun onbekend, dat door het woord Gods de hemelen van over lang geweest zijn, en de aarde uit het water en in het water bestaande;
\par 6 Door welke de wereld, die toen was, met het water van den zondvloed bedekt zijnde, vergaan is.
\par 7 Maar de hemelen, die nu zijn, en de aarde, zijn door hetzelfde woord als een schat weggelegd, en worden ten vure bewaard tegen den dag des oordeels, en der verderving der goddeloze mensen.
\par 8 Doch deze ene zaak zij u niet onbekend, geliefden, dat een dag bij den Heere is als duizend jaren, en duizend jaren als een dag.
\par 9 De Heere vertraagt de belofte niet (gelijk enigen dat traagheid achten), maar is lankmoedig over ons, niet willende, dat enigen verloren gaan, maar dat zij allen tot bekering komen.
\par 10 Maar de dag des Heeren zal komen als een dief in den nacht, in welken de hemelen met een gedruis zullen voorbijgaan, en de elementen branden zullen en vergaan, en de aarde en de werken, die daarin zijn, zullen verbranden.
\par 11 Dewijl dan deze dingen alle vergaan, hoedanigen behoort gij te zijn in heiligen wandel en godzaligheid!
\par 12 Verwachtende en haastende tot de toekomst van den dag Gods, in welken de hemelen, door vuur ontstoken zijnde, zullen vergaan, en de elementen brandende zullen versmelten.
\par 13 Maar wij verwachten, naar Zijn belofte, nieuwe hemelen en een nieuwe aarde, in dewelke gerechtigheid woont.
\par 14 Daarom, geliefden, verwachtende deze dingen, benaarstigt u, dat gij onbevlekt en onbestraffelijk van Hem bevonden moogt worden in vrede;
\par 15 En acht de lankmoedigheid onzes Heeren voor zaligheid; gelijkerwijs ook onze geliefde broeder Paulus, naar de wijsheid, die hem gegeven is, ulieden geschreven heeft;
\par 16 Gelijk ook in alle zendbrieven, daarin van deze dingen sprekende; in welke sommige dingen zwaar zijn om te verstaan, die de ongeleerde en onvaste mensen verdraaien, gelijk ook de andere Schriften, tot hun eigen verderf.
\par 17 Gij dan, geliefden, zulks te voren wetende, wacht u, dat gij niet door de verleiding der gruwelijke mensen mede afgerukt wordt, en uitvalt van uw vastigheid;
\par 18 Maar wast op in de genade en kennis van onzen Heere en Zaligmaker Jezus Christus. Hem zij de heerlijkheid, beide nu en in den dag der eeuwigheid. Amen.




\end{document}