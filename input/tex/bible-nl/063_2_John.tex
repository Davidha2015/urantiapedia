\begin{document}

\title{2 Johannes}



\chapter{1}

\par 1 De ouderling aan de uitverkoren vrouwe en aan haar kinderen, die ik in waarheid liefheb, en niet alleen ik, maar ook allen, die de waarheid gekend hebben;
\par 2 Om der waarheid wil, die in ons blijft, en met ons zal zijn in der eeuwigheid:
\par 3 Genade, barmhartigheid, vrede zij met ulieden van God den Vader, en van den Heere Jezus Christus, den Zoon des Vaders, in waarheid en liefde.
\par 4 Ik ben zeer verblijd geweest, dat ik van uw kinderen gevonden heb, die in de waarheid wandelen, gelijk wij een gebod ontvangen hebben van den Vader.
\par 5 En nu bid ik u, uitverkoren vrouwe, niet als u schrijvende een nieuw gebod, maar hetgeen wij gehad hebben van den beginne, namelijk dat wij elkander liefhebben.
\par 6 En dit is de liefde, dat wij wandelen naar Zijn geboden. Dit is het gebod, gelijk gijlieden van den beginne gehoord hebt, dat gij in hetzelve zoudt wandelen.
\par 7 Want er zijn vele verleiders in de wereld gekomen, die niet belijden, dat Jezus Christus in het vlees gekomen is. Deze is de verleider en de antichrist.
\par 8 Ziet toe voor uzelven, dat wij niet verliezen, hetgeen wij gearbeid hebben, maar een vol loon mogen ontvangen.
\par 9 Een iegelijk, die overtreedt, en niet blijft in de leer van Christus, die heeft God niet; die in de leer van Christus blijft, deze heeft beiden den Vader en den Zoon.
\par 10 Indien iemand tot ulieden komt, en deze leer niet brengt, ontvangt hem niet in huis, en zegt tot hem niet: Zijt gegroet.
\par 11 Want die tot hem zegt: Zijt gegroet, die heeft gemeenschap aan zijn boze werken.
\par 12 Ik heb veel aan ulieden te schrijven, doch ik heb niet gewild door papier en inkt; maar ik hoop tot ulieden te komen, en mond tot mond met u te spreken, opdat onze blijdschap volkomen moge zijn.
\par 13 U groeten de kinderen van uw zuster, de uitverkorene. Amen.




\end{document}