\begin{document}

\title{Rodzaju}


\chapter{1}

\par 1 Na poczatku stworzyl Bóg niebo i ziemie.
\par 2 A ziemia byla nieksztaltowna i prózna, i ciemnosc byla nad przepascia, a Duch Bozy unaszal sie nad wodami.
\par 3 I rzekl Bóg: Niech bedzie swiatlosc; i stala sie swiatlosc.
\par 4 I widzial Bóg swiatlosc, ze byla dobra; i uczynil Bóg rozdzial miedzy swiatloscia i miedzy ciemnoscia.
\par 5 I nazwal Bóg swiatlosc dniem a ciemnosc nazwal noca; i stal sie wieczór, i stal sie zaranek, dzien pierwszy.
\par 6 Potem rzekl Bóg: Niech bedzie rozpostarcie, w posrodku wód, a niech dzieli wody od wód.
\par 7 I uczynil Bóg rozpostarcie; uczynil tez rozdzial miedzy wodami, które sa pod rozpostarciem; i miedzy wodami, które sa nad rozpostarciem; i stalo sie tak.
\par 8 I nazwal Bóg rozpostarcie niebem. I stal sie wieczór, i stal sie zaranek, dzien wtóry.
\par 9 I rzekl Bóg: Niech sie zbiora wody, które sa pod niebem, na jedno miejsce, a niech sie okaze miejsce suche; i stalo sie tak.
\par 10 I nazwal Bóg suche miejsce ziemia, a zebranie wód nazwal morzem.
\par 11 I widzial Bóg, ze to bylo dobre. Potem rzekl Bóg: Niech zrodzi ziemia trawe, ziele, wydawajace nasienie, i drzewo rodzajne, czyniace owoc, wedlug rodzaju swego, którego by nasienie bylo w nim na ziemi; i stalo sie tak.
\par 12 I zrodzila ziemia trawe, ziele wydawajace nasienie, wedlug rodzaju swego, i drzewo czyniace owoc, w którym nasienie jego, wedlug rodzaju swego; i widzial Bóg, ze to bylo dobre.
\par 13 I stal sie wieczór, i stal sie zaranek, dzien trzeci.
\par 14 I rzekl Bóg: Niech beda swiatla na rozpostarciu niebieskim, ku rozdzielaniu dnia od nocy, a niech beda na znaki, i pewne czasy, i dni, i lata.
\par 15 I niech beda za swiatla na rozpostarciu nieba, aby swiecily nad ziemia; i stalo sie tak.
\par 16 I uczynil Bóg dwa swiatla wielkie: swiatlo wieksze, aby rzadzilo dzien, a swiatlo mniejsze, aby rzadzilo noc, i gwiazdy.
\par 17 I postawil je Bóg na rozpostarciu nieba, aby swiecily nad ziemia.
\par 18 I zeby rzadzily dzien i noc, i czynily rozdzial miedzy swiatloscia, i miedzy ciemnoscia; i widzial Bóg, ze to bylo dobre.
\par 19 I stal sie wieczór, i stal sie zaranek, dzien czwarty.
\par 20 I rzekl Bóg: Niech hojnie wywioda wody plaz duszy zywiacej; a ptactwo niech lata nad ziemia, pod rozpostarciem niebieskim.
\par 21 I stworzyl Bóg wieloryby wielkie, i wszelka dusze zywiaca plazajaca sie, która hojnie wywiodly wody, wedlug rodzaju ich; i wszelkie ptactwo skrzydlaste, wedlug rodzaju ich; i widzial Bóg, ze to bylo dobre.
\par 22 Blogoslawil im tedy Bóg, mówiac: Rozradzajcie sie, i rozmnazajcie sie, a napelniajcie wody morskie; i ptactwo niech sie rozmnaza na ziemi.
\par 23 I stal sie wieczór, i stal sie zaranek, dzien piaty.
\par 24 Rzekl tez Bóg: Niech wyda ziemia dusze zywiaca wedlug rodzaju swego; bydlo i plaz, i zwierz ziemski, wedlug rodzaju swego; i stalo sie tak.
\par 25 Uczynil tedy Bóg zwierz ziemski wedlug rodzaju swego; i bydlo wedlug rodzaju swego; i wszelki plaz ziemski wedlug rodzaju swego; i widzial Bóg, ze to bylo dobre.
\par 26 Zatem rzekl Bóg: Uczynmy czlowieka na wyobrazenie nasze, wedlug podobienstwa naszego; a niech panuje nad rybami morskimi, i nad ptactwem niebieskim, i nad zwierzety, i nad wszystka ziemia, i nad wszelkim plazem, plazajacym sie po ziemi.
\par 27 Stworzyl tedy Bóg czlowieka na wyobrazenie swoje; na wyobrazenie Boze stworzyl go; mezczyzne i niewiaste stworzyl je.
\par 28 I blogoslawil im Bóg, i rzekl do nich Bóg: Rozradzajcie sie, i rozmnazajcie sie, i napelniajcie ziemie; i czyncie ja sobie poddana; i panujcie nad rybami morskimi, i nad ptactwem niebieskim, i nad wszelkim zwierzem, który sie rusza na ziemi.
\par 29 I rzekl Bóg: Oto dalem wam wszelkie ziele, wydawajace z siebie nasienie, które jest na obliczu wszystkiej ziemi; i wszelkie drzewo, na którym jest owoc drzewa, wydawajace z siebie nasienie, bedzie wam ku pokarmowi.
\par 30 I wszelkiemu zwierzowi ziemskiemu, i wszystkiemu ptactwu niebieskiemu, i wszelkiej rzeczy ruszajacej sie na ziemi, w której jest dusza zywiaca; wszelka jarzyna ziela bedzie ku pokarmowi; i stalo sie tak.
\par 31 I widzial Bóg wszystko, co uczynil, a oto bylo bardzo dobre; i stal sie wieczór, i stal sie zaranek, dzien szósty.

\chapter{2}

\par 1 Dokonczone tedy sa niebiosa i ziemia, i wszystko wojsko ich.
\par 2 I dokonczyl Bóg dnia siódmego dziela swego, które uczynil; i odpoczal w dzien siódmy od wszelkiego dziela swego, które uczynil.
\par 3 I blogoslawil Bóg dniowi siódmemu, i poswiecil go; iz wen odpoczal od wszelkiego dziela swego, które byl stworzyl Bóg, aby uczynione bylo.
\par 4 Tec sa zrodzenia niebios, i ziemi, gdy byly stworzone, dnia, którego uczynil Pan Bóg ziemie i niebo.
\par 5 Wszelka rózdzke polna, przedtem niz byla na ziemi; i wszelkie ziele polne, pierwej niz weszlo; albowiem nie spuscil jeszcze byl dzdzu Pan Bóg na ziemie; i czlowieka nie bylo, któryby sprawowal ziemie.
\par 6 Ale para wychodzila z ziemi, która odwilzala wszystek wierzch ziemi.
\par 7 Stworzyl tedy Pan Bóg czlowieka z prochu ziemi, i natchnal w oblicze jego dech zywota. I stal sie czlowiek dusza zywiaca.
\par 8 Nasadzil tez byl Pan Bóg sad w Eden, na wschód slonca, i postawil tam czlowieka, którego byl stworzyl.
\par 9 I wywiódl Pan Bóg z ziemi wszelkie drzewo wdzieczne na wejrzeniu, i smaczne ku jedzeniu: i drzewo zywota w posrodku sadu; i drzewo wiadomosci dobrego i zlego.
\par 10 A rzeka wychodzila z Eden dla odwilzenia sadu; i stamtad dzielila sie na cztery glówne rzeki;
\par 11 Imie jednej Fyson; ta okraza wszystka ziemie Hewila, gdzie sie rodzi zloto.
\par 12 A zloto ziemi onej jest wyborne. Tamze jest Bdellion, i kamien Onychyn.
\par 13 A imie rzeki drugiej Gihon; ta okraza wszystke ziemie Murzynska.
\par 14 Imie zas rzeki trzeciej Chydekel, ta plynie na wschód slonca ku Asyryi. A rzeka czwarta jest Eufrates.
\par 15 Wzial tedy Pan Bóg czlowieka, i postawil go w sadzie Eden, aby go sprawowal, i aby go strzegl.
\par 16 Tedy rozkazal Pan Bóg czlowiekowi, mówiac: Z kazdego drzewa sadu jesc bedziesz.
\par 17 Ale z drzewa wiadomosci dobrego i zlego, jesc z niego nie bedziesz; albowiem dnia, którego jesc bedziesz z niego, smiercia umrzesz.
\par 18 Rzekl tez Pan Bóg: Nie dobrze byc czlowiekowi samemu; uczynie mu pomoc, która by byla przy nim.
\par 19 A gdy stworzyl Pan Bóg z ziemi wszelki zwierz polny, i wszelkie ptactwo niebieskie, tedy je przywiódl do Adama, aby obaczyl jakoby je nazwac mial; a jakoby nazwal Adam kazda dusze zywiaca, tak aby bylo imie jej.
\par 20 Tedy dal Adam imiona wszystkiemu bydlu, i ptactwu niebieskiemu, i wszelkiemu zwierzowi polnemu. Lecz Adamowi nie byla znaleziona pomoc, która by przy nim byla.
\par 21 Tedy przypuscil Pan Bóg twardy sen na Adama, i zasnal; i wyjal jedno zebro jego, i napelnil cialem miasto niego.
\par 22 I zbudowal Pan Bóg z zebra onego, które wyjal z Adama, niewiaste, i przywiódl ja do Adama.
\par 23 I rzekl Adam: Toc teraz jest kosc z kosci moich, i cialo z ciala mego; dla tegoz bedzie nazwana mezatka, bo ona z meza wzieta jest.
\par 24 Przetoz opusci czlowiek ojca swego i matke swoje, a przylaczy sie do zony swojej, i beda jednem cialem.
\par 25 A byli oboje nadzy, Adam i zona jego; a nie wstydzili sie.

\chapter{3}

\par 1 A waz byl chytrzejszy nad wszystkie zwierzeta polne, które byl uczynil Pan Bóg; ten rzekl do niewiasty: Takze to, ze wam Bóg rzekl: Nie bedziecie jedli z kazdego drzewa sadu tego?
\par 2 I rzekla niewiasta do weza: Z owocu drzewa sadu tego pozywamy;
\par 3 Ale z owocu drzewa, które jest w posród sadu, rzekl Bóg: Nie bedziecie jedli z niego, ani sie go dotykac bedziecie, byscie snac nie pomarli.
\par 4 I rzekl waz do niewiasty: Zadnym sposobem smiercia nie pomrzecie;
\par 5 Ale wie Bóg, ze któregokolwiek dnia z niego jesc bedziecie, otworza sie oczy wasze; a bedziecie jako bogowie, znajacy dobre i zle.
\par 6 Widzac tedy niewiasta, iz dobre bylo drzewo ku jedzeniu; a iz bylo wdzieczne na wejrzeniu, a pozadliwe drzewo dla nabycia umiejetnosci, wziela z owocu jego, i jadla; dala tez i mezowi swemu, który z nia byl; i on tez jadl.
\par 7 Zatem otworzyly sie oczy obojga, i poznali, ze byli nagimi; i spletli liscie figowe, a poczynili sobie zaslony.
\par 8 A wtem uslyszeli glos Pana Boga chodzacego po sadzie z wiatrem dniowym; i skryl sie Adam, i zona jego od oblicza Pana Boga miedzy drzewa sadu.
\par 9 I zawolal Pan Bóg Adama, i rzekl mu: Gdziezes?
\par 10 Który odpowiedzial: Glos twój uslyszalem w sadzie, i zleklem sie dla tego, zem nagi, i skrylem sie.
\par 11 I rzekl Bóg: Któz ci pokazal, zes jest nagim? izalis nie jadl z drzewa onego, z któregom zakazal tobie, abys nie jadl?
\par 12 Tedy rzekl Adam: Niewiasta, któras mi dal, aby byla ze mna, ona mi dala z tego drzewa, i jadlem.
\par 13 I rzekl Pan Bóg do niewiasty: Cózes to uczynila? i rzekla niewiasta: Waz mie zwiódl, i jadlam.
\par 14 Tedy rzekl Pan Bóg do weza: Izes to uczynil, przekletym bedziesz nad wszystkie zwierzeta, i nad wszystkie bestyje polne; na brzuchu twoim czolgac sie bedziesz, a proch zrec bedziesz po wszystkie dni zywota twego.
\par 15 Nieprzyjazn tez poloze miedzy toba i niewiasta, i miedzy nasieniem twoim, i miedzy nasieniem jej; to potrze tobie glowe, a ty mu potrzesz piete.
\par 16 A do niewiasty rzekl: Obficie rozmnoze bolesci twoje, i poczecia twoje; w bolesci rodzic bedziesz dzieci, a wola twa poddana bedzie mezowi twemu, a on nad toba panowac bedzie.
\par 17 Zas rzekl do Adama: Izes usluchal glosu zony twojej, a jadles z drzewa tego, o któremem ci przykazal, mówiac: Nie bedziesz jadl z niego; przekleta bedzie ziemia dla ciebie, w pracy z niej pozywac bedziesz po wszystkie dni zywota twego.
\par 18 A ona ciernie i oset rodzic bedzie tobie; i bedziesz pozywal ziela polnego.
\par 19 W pocie oblicza twego bedziesz pozywal chleba, az sie nawrócisz do ziemi, gdyzes z niej wziety; bos proch, i w proch sie obrócisz.
\par 20 I nazwal Adam imie zony swej Ewa, iz ona byla matka wszystkich zywiacych.
\par 21 I uczynil Pan Bóg Adamowi, i zonie jego odzienie skórzane, i oblókl je.
\par 22 Tedy rzekl Pan Bóg: Oto Adam stal sie jako jeden z nas, wiedzacy dobre i zle; tedy wyzenmy go, by snac nie sciagnal reki swej, i nie wzial z drzewa zywota, i nie jadl, i zylby na wieki.
\par 23 I wypuscil go Pan Bóg z sadu Eden, ku sprawowaniu ziemi, z której byl wziety.
\par 24 A tak wygnal czlowieka; i postawil na wschód slonca sadu Eden Cheruby, i miecz plomienisty i obrotny ku strzezeniu drogi do drzewa zywota.

\chapter{4}

\par 1 Potem Adam poznal Ewe, zone swoje, która poczela i porodzila Kaina, i rzekla: Otrzymalam meza od Pana.
\par 2 I porodzil zasie brata jego Abla; i byl Abel pasterzem owiec, a Kain byl rolnikiem.
\par 3 I stalo sie po wielu dni, iz przyniósl Kain z owocu ziemi ofiare Panu.
\par 4 Takze i Abel przyniósl z pierworodztw trzód swoich i z tlustosci ich; i wejrzal Pan na Abla i na ofiare jego.
\par 5 Ale na Kaina i na ofiare jego nie wejrzal; i rozgniewal sie Kain bardzo, i spadla twarz jego.
\par 6 Tedy rzekl Pan do Kaina: Przeczzes sie zapalil gniewem a czemu spadla twarz twoja?
\par 7 Azaz, jezli dobrze czynic bedziesz, nie bedziesz wywyzszon? a jezli nie bedziesz dobrze czynil, we drzwiach grzech lezy; a do ciebie chuc jego bedzie, a ty nad nim panowac bedziesz.
\par 8 I rozmawial Kain z Ablem bratem swoim. I stalo sie, gdy byli na polu, ze powstal Kain na Abla brata swego, i zabil go.
\par 9 I rzekl Pan do Kaina: Gdziez jest Abel brat twój? który odpowiedzial: Nie wiem; izalim ja strózem brata mego?
\par 10 I rzekl Bóg: Cózes uczynil? Glos krwi brata twego wola do mnie z ziemi.
\par 11 Teraz tedy przekletym bedziesz na ziemi, która otworzyla usta swe, aby przyjela krew brata twego z reki twojej.
\par 12 Gdy bedziesz sprawowal ziemie, nie wyda wiecej mocy swej tobie; tulaczem, i biegunem bedziesz na ziemi.
\par 13 Tedy rzekl Kain do Pana: Wieksza jest nieprawosc moja, nizby mi ja odpuscic miano.
\par 14 Oto mie dzis wyganiasz z oblicza tej ziemi, a przed twarza twoja skryje sie, i bede tulaczem, i biegunem na ziemi; i stanie sie, ze ktokolwiek mie znajdzie, zabije mie.
\par 15 I rzekl mu Pan: Zaiste, ktobykolwiek zabil Kaina, siedmioraka odniesie pomste. I wlozyl Pan na Kaina pietno, aby go nie zabijal, ktobykolwiek znalazl.
\par 16 Tedy odszedl Kain od oblicza Panskiego, i mieszkal w ziemi Nod, na wschód slonca od Eden.
\par 17 I poznal Kain zone swa, która poczela, i porodzila Enocha; i zbudowal miasto, i nazwal imie miasta tego imieniem syna swego, Enoch.
\par 18 I urodzil sie Enochowi Irad, a Irad splodzil Mawiaela, a Mawiael splodzil Matusaela, a Matusael splodzil Lamecha.
\par 19 I pojal sobie Lamech dwie zony; imie jednej, Ada, a imie drugiej, Sella.
\par 20 Tedy urodzila Ada Jabala, który byl ojcem mieszkajacych w namieciech, i pasterzów.
\par 21 A imie brata jego bylo Jubal, który byl ojcem wszystkich grajacych na harfie, i na muzyckiem naczyniu.
\par 22 Sella tez urodzila Tubalkaina, rzemieslnika wszelkiej roboty, od miedzi i od zelaza. A siostra Tubalkainowa byla Noema.
\par 23 Tedy rzekl Lamech zonom swym, Adzie i Selli: Sluchajcie glosu mego, zony Lamechowe, posluchajcie slów moich; zabilbym ja meza za zranienie moje, i mlodzienca za sinosc moje.
\par 24 Jezlic siedmiokroc mscic sie beda za Kaina, tedyc za Lamecha siedemdziesiat i siedem kroc.
\par 25 I poznal jeszcze Adam zone swa, która urodzila syna, i nazwala imie jego Set, mówiac: Dal mi Bóg inne potomstwo miasto Abla, którego zabil Kain.
\par 26 Setowi tez urodzil sie syn, i nazwal imie jego Enos. Na ten czas poczeto wzywac imienia Panskiego.

\chapter{5}

\par 1 Tec sa ksiegi rodzajów Adamowych. W dzien, którego stworzyl Bóg czlowieka, na podobienstwo Boze uczynil go.
\par 2 Mezczyzne i niewiaste stworzyl je; i blogoslawil im, i nazwal imie ich, czlowiek, w dzien, którego sa stworzeni.
\par 3 I zyl Adam sto i trzydziesci lat, i splodzil syna na podobienstwo swoje, i na wyobrazenie swoje, i nazwal imie jego Set.
\par 4 I bylo dni Adamowych po splodzeniu Seta osiem set lat, i splodzil syny i córki.
\par 5 A tak bylo wszystkich dni Adamowych, których zyl, dziewiec set lat i trzydziesci, lat i umarl.
\par 6 A Set zyl sto lat i piec lat, i splodzil Enosa.
\par 7 I zyl Set po splodzeniu Enosa, osiem set lat, i siedem lat, i splodzil syny i córki.
\par 8 I bylo wszystkich dni Setowych dziewiec set lat, i dwanascie lat, i umarl.
\par 9 A Enos zyl dziewiecdziesiat lat, i splodzil Kenana.
\par 10 I zyl Enos po splodzeniu Kenana, osiem set lat, i pietnascie lat, i splodzil syny i córki.
\par 11 Bylo tedy wszystkich dni Enosowych dziewiec set lat, i piec lat, i umarl.
\par 12 Kenan tez zyl siedemdziesiat lat, i splodzil Mahalaleela.
\par 13 I zyl Kenan po splodzeniu Mahalaleela osiem set lat, i czterdziesci lat, i splodzil syny i córki.
\par 14 Bylo tedy wszystkich dni Kenanowych dziewiec set i dziesiec lat, i umarl.
\par 15 A Mahalaleel zyl szescdziesiat i piec lat, i splodzil Jareda.
\par 16 A po splodzeniu Jareda, zyl Mahalaleel osiem set lat i trzydziesci lat, i splodzil syny i córki.
\par 17 I bylo wszystkich dni Mahalaleelowych osiem set dziewiecdziesiat i piec lat, i umarl.
\par 18 Zyl tez Jared sto szescdziesiat i dwa lata, i splodzil Enocha.
\par 19 I zyl Jared po splodzeniu Enocha osiem set lat, i splodzil syny i córki.
\par 20 I bylo wszystkich dni Jaredowych dziewiec set szescdziesiat i dwa lat, i umarl.
\par 21 A Enoch zyl szescdziesiat lat, i piec, i splodzil Matuzalema.
\par 22 I chodzil Enoch z Bogiem po splodzeniu Matuzalema trzy sta lat, i splodzil syny i córki.
\par 23 I bylo wszystkich dni Enochowych trzy sta szescdziesiat i piec lat.
\par 24 I chodzil Enoch z Bogiem, a nie bylo go wiecej, bo go wzial Bóg.
\par 25 I zyl Matuzalem sto osiemdziesiat i siedem lat, i splodzil Lamecha.
\par 26 I zyl Matuzalem po splodzeniu Lamecha siedem set osiemdziesiat lat, i dwa lata, i splodzil syny i córki.
\par 27 I bylo wszystkich dni Matuzalemowych dziewiec set szescdziesiat i dziewiec lat, i umarl.
\par 28 A Lamech zyl sto osiemdziesiat i dwa lat, i splodzil syna.
\par 29 I nazwal imie jego Noe, mówiac: Ten nas pocieszy z pracy naszej, i z roboty rak naszych, z strony ziemi, która Pan przeklal.
\par 30 Potem zyl Lamech po splodzeniu Noego, piec set dziewiecdziesiat lat i piec, i splodzil syny i córki.
\par 31 I bylo wszystkich dni Lamechowych siedem set siedemdziesiat i siedem lat, i umarl.
\par 32 A gdy bylo Noemu piec set lat, splodzil Noe Sema, Chama, i Jafeta.

\chapter{6}

\par 1 I stalo sie, gdy sie ludzie poczeli rozmnazac na ziemi, a córki sie im zrodzily;
\par 2 Ze, widzac synowie Bozy córki ludzkie, iz byly piekne, brali je sobie za zony, ze wszystkich, które sobie upodobali.
\par 3 I rzekl Pan: Nie bedzie sie wadzil duch mój z czlowiekiem na wieki, gdyz jest cialem; i beda dni jego sto i dwadziescia lat.
\par 4 A byli olbrzymowie na ziemi w one dni; nawet i potem, gdy weszli synowie Bozy do córek ludzkich, rodzily im syny. A cic sa mocarze, którzy od wieku byli mezowie slawni.
\par 5 A widzac Pan, ze wielka byla zlosc ludzka na ziemi, a wszystko zmyslanie mysli serca ich tylko zle bylo po wszystkie dni;
\par 6 Zalowal Pan, ze uczynil czlowieka na ziemi, i bolal w sercu swem.
\par 7 I rzekl Pan: Wygladze czlowieka, któregom stworzyl, z oblicza ziemi, od czlowieka az do bydlecia, az do gadziny, i az do ptastwa niebieskiego; bo mi zal, zem je uczynil.
\par 8 Ale Noe znalazl laske w oczach Panskich.
\par 9 Tec sa rodzaje Noego: Noe maz sprawiedliwy, doskonalym byl za wieku swego; z Bogiem chodzil Noe.
\par 10 I splodzil Noe trzech synów, Sema, Chama, i Jafeta.
\par 11 Ale ziemia popsowala sie byla przed Bogiem; i napelnila sie nieprawoscia.
\par 12 Tedy wejrzal Bóg na ziemie, a oto popsowana byla (albowiem zepsowalo bylo wszelkie cialo droge swoje na ziemi).
\par 13 I rzekl Bóg do Noego: Koniec wszelkiego ciala przyszedl przed oblicze moje, bo napelniona jest ziemia nieprawoscia od oblicza ich; przetoz je wytrace z ziemi.
\par 14 Uczyn sobie korab z drzewa Gofer; przegrody poczynisz w korabiu, i oblejesz go wewnatrz i zewnatrz smola.
\par 15 A uczynisz go na ten ksztalt: Trzy sta lokci bedzie dlugosc korabia; piecdziesiat lokci szerokosc jego, a trzydziesci lokci wysokosc jego.
\par 16 Okno uczynisz w korabiu; a na lokiec wywiedziesz je wzwyz, i drzwi korabiu w boku jego postawisz; pietra spodnie wtóre i trzecie uczynisz w nim.
\par 17 A Ja oto, Ja przywiode potop wód na ziemie, ku wytraceniu wszelkiego ciala, w którem jest duch zywota pod niebem; wszystko, cokolwiek jest na ziemi, pozdycha.
\par 18 Ale z toba postanowie przymierze moje; i wnijdziesz do korabia, ty i synowie twoi, i zona twoja, i zony synów twoich z toba.
\par 19 I ze wszech zwierzat wszelkiego ciala po dwojgu ze wszech, wprowadzisz do korabia, aby zywo zachowane byly z toba, samiec i samica beda.
\par 20 Z ptastwa wedlug rodzaju jego, i z bydla wedlug rodzaju jego, i z wszelkiej gadziny ziemskiej wedlug rodzaju jej, po dwojgu z kazdego rodzaju wnijda z toba, aby zywe zostaly.
\par 21 A ty wezmiesz z soba wszelkiego pokarmu, który sie jesc godzi, a zbierzesz do siebie, i bedzie tobie i onym na pokarm.
\par 22 I uczynil Noe wedlug wszystkiego; jako mu rozkazal Bóg, tak uczynil.

\chapter{7}

\par 1 I rzekl Pan do Noego: Wnijdz ty i wszystek dom twój do korabia; bom cie widzial sprawiedliwym przed obliczem mojem w narodzie tym.
\par 2 Z kazdego bydlecia czystego wezmiesz z soba siedmioro a siedmioro, samca i samice jego; ale z zwierzat nieczystych po dwojgu, samca i samice jego.
\par 3 Takze z ptastwa niebieskiego siedmioro a siedmioro, samca i samice, aby zywe zachowane bylo nasienie na wszystkiej ziemi.
\par 4 Albowiem jeszcze po siedmiu dniach spuszcze deszcz na ziemie, przez czterdziesci dni i czterdziesci nocy, i wygladze wszystko stworzenie, którem uczynil, z oblicza ziemi.
\par 5 Uczynil tedy Noe wedlug wszystkiego, jako mu byl Pan rozkazal.
\par 6 A Noemu bylo szesc set lat, gdy przyszedl potop wód na ziemie.
\par 7 I wszedl Noe, i synowie jego, i zona jego, i zony synów jego z nim, do korabia, dla potopu wód.
\par 8 Z zwierzat tez czystych, i z zwierzat, które nie byly czyste, i z ptastwa, i ze wszystkiego, co sie plaza po ziemi;
\par 9 Po parze weszlo do Noego do korabia, to jest samiec i samica, jako byl rozkazal Bóg Noemu.
\par 10 I stalo sie po siedmiu dniach, iz wody potopu przyszly na ziemie.
\par 11 Roku szescsetnego wieku Noego, miesiaca wtórego, siedemnastego dnia tegoz miesiaca, w tenze dzien przerwaly sie wszystkie zródla przepasci wielkiej, i okna niebieskie otworzyly sie.
\par 12 I padal deszcz na ziemie, czterdziesci dni i czterdziesci nocy.
\par 13 Onegoz dnia wszedl Noe i Sem i Cham i Jafet, synowie Noego, i zona Noego, i trzy zony synów jego z nim do korabia.
\par 14 Oni, i wszelki zwierz wedlug rodzaju swego, i wszelkie bydle wedlug rodzaju swego, i wszelka gadzina plazajaca sie po ziemi, wedlug rodzaju swego, i wszystko latajace wedlug rodzaju swego, i wszelki ptak, i wszelka rzecz skrzydlasta.
\par 15 A tak weszlo do Noego w korab po parze z kazdego ciala, w którem byl duch zywota.
\par 16 A które weszly, samiec i samica z kazdego ciala weszly, jako mu Bóg rozkazal. I zamknal Pan za nim.
\par 17 Byl tedy potop przez czterdziesci dni na ziemi, i wezbraly wody i podniosly korab, i byl podniesiony od ziemi.
\par 18 I wzmogly sie wody, a wezbraly bardzo nad ziemia, i plywal korab po wodach.
\par 19 Tedy sie wody wzmogly nader bardzo nad ziemia, i okryly sie wszystkie góry wysokie, które byly pod wszystkiem niebem.
\par 20 Pietnascie lokci wzwyz wezbraly wody, gdy byly okryte góry.
\par 21 Zaginelo tedy wszelkie cialo ruchajace sie na ziemi, i z ptaków, i z bydla, i z zwierzat, i wszelkiej gadziny plazajacej sie po ziemi, i wszyscy ludzie.
\par 22 Wszystko, którego tchnacy duch zywota byl w nozdrzach jego, ze wszystkiego, co na suszy bylo, pomarlo.
\par 23 Tak wygladzil Bóg wszystko stworzenie, które bylo na ziemi, od czlowieka az do bydlecia, az do gadziny, i az do ptastwa niebieskiego, wygladzone sa z ziemi, i zostal tylko Noe i którzy z nim byli w korabiu.
\par 24 I trwaly wody nad ziemia sto i piecdziesiat dni.

\chapter{8}

\par 1 I wspomnial Bóg na Noego i na wszystkie zwierzeta, i na wszystko bydlo, które bylo z nim w korabiu; i przywiódl Bóg wiatr na ziemie, a zastanowily sie wody.
\par 2 I zawarte sa zródla przepasci, i okna niebieskie, i zahamowany jest deszcz z nieba.
\par 3 I wrócily sie wody z wierzchu ziemi idac, i wracajac sie; i opadly wody po skonczeniu stu i piecdziesieciu dni.
\par 4 I odpoczal korab miesiaca siódmego, siedemnastego dnia tegoz miesiaca, na górach Ararad.
\par 5 A wody sciekaly i opadaly, az do dziesiatego miesiaca, dziesiatego bowiem miesiaca, pierwszego dnia, okazaly sie wierzchy gór.
\par 6 I stalo sie po skonczeniu czterdziestu dni, otworzyl Noe okno korabia, które byl uczynil.
\par 7 I wypuscil kruka, który tam i sam latajac, zasie sie wracal, az oschly wody na ziemi.
\par 8 Potem wypuscil golebice od siebie, aby obaczyl, jezli opadly wody z wierzchu ziemi.
\par 9 Ale nie znalazlszy golebica odpocznienia stopie nogi swojej, wrócila sie do niego do korabia; jeszcze bowiem wody byly po wszystkiej ziemi; i wyciagnawszy reke swoje, wzial ja, i wniósl ja do siebie do korabia.
\par 10 A poczekawszy jeszcze drugie siedem dni, po wtóre wypuscil golebice z korabia.
\par 11 I wrócila do niego golebica pod wieczór; a oto, rózdzka oliwy urwana w usciech jej; a tak poznal Noe, ze opadly wody z wierzchu ziemi.
\par 12 I czekal jeszcze drugie siedem dni, i wypuscil golebice, która sie wiecej nie wrócila do niego.
\par 13 I stalo sie szescsetnego i pierwszego roku, miesiaca pierwszego, dnia pierwszego, oschly wody z ziemi; i zdjal Noe przykrycie korabia, a ujrzal, ze osechl wierzch ziemi.
\par 14 A miesiaca wtórego, dwudziestego siódmego dnia tegoz miesiaca, oschla ziemia.
\par 15 I rzekl Bóg do Noego, mówiac:
\par 16 Wynijdz z korabia, ty, i zona twoja, i synowie twoi, i zony synów twoich z toba.
\par 17 Wszystkie zwierzeta, które sa z toba, z wszelkiego ciala, z ptastwa i z bydla, i z wszelkiej gadziny, plazajacej sie po ziemi, wywiedz z soba, a niech sie rozpladzaja na ziemi, i niech rosna, i rozmnazaja sie na ziemi.
\par 18 I wyszedl Noe, i synowie jego, i zona jego, i zony synów jego z nim.
\par 19 Wszelkie zwierze, wszelka gadzina, i wszelkie ptactwo, wszystko co sie plaza po ziemi, wedlug rodzajów swoich, wyszly z korabia.
\par 20 Zatem zbudowal Noe oltarz Panu, i wzial z kazdego bydla czystego, i z kazdego ptastwa czystego, i ofiarowal calopalenia na oltarzu onym.
\par 21 I zawonial Pan wonnosci wdziecznej, i rzekl Pan w sercu swem: Nie bede wiecej przeklinal ziemi dla czlowieka: albowiem mysl serca czlowieczego zla jest od mlodosci jego, nie zatrace wiecej wszystkiego co zyje, jakom teraz uczynil.
\par 22 A póki ziemia trwac bedzie, siew i zniwo, i zimno, i goraco, i lato, i zima, i dzien, i noc nie ustana.

\chapter{9}

\par 1 I blogoslawil Bóg Noego, i syny jego, i rzekl im: Rozradzajcie sie, i rozmnazajcie sie, i napelniajcie ziemie.
\par 2 A strach wasz i bojazn wasza bedzie nad wszelkiem zwierzeciem ziemi, i nad wszystkiem ptastwem niebieskiem, i nad wszystkiem, co sie rucha na ziemi, i nad wszystkiemi rybami morskiemi: w reke wasze podane sa.
\par 3 Wszystko co sie rucha, i co zyje, wam bedzie na pokarm, jako jarzyne zielona, dalem wam to wszystko.
\par 4 Wszakze miesa z dusza jego, która jest krew jego, jesc nie bedziecie.
\par 5 A zaiste krwi waszej, dusz waszych szukac bede, z reki kazdej bestyi szukac jej bede: takze z reki czlowieczej, z reki kazdego brata jego bede szukal duszy czlowieczej.
\par 6 Kto wyleje krew czlowiecza, przez czlowieka krew jego wylana bedzie: bo na wyobrazenie Boze uczynion jest czlowiek.
\par 7 A wy rozradzajcie sie, i rozmnazajcie sie, rozpladzajcie sie na ziemi, i mnózcie sie na niej.
\par 8 Tedy rzekl Bóg do Noego, i do synów jego z nim, mówiac:
\par 9 A Ja, oto Ja stanowie przymierze moje z wami, i z nasieniem waszem po was.
\par 10 I z kazda dusza zywiaca, która jest z wami: w ptastwie, w bydle, i w kazdem zwierzeciu ziemi, które sa z wami, ze wszystkich, co wyszly z korabia, az do kazdego zwierzecia na ziemi.
\par 11 I postanowie przymierze moje z wami; a nie bedzie zatracone wiecej wszelkie cialo wodami potopu; i nie bedzie wiecej potop na skazenie ziemi.
\par 12 Tedy rzekl Bóg: To jest znak przymierza, który Ja dawam miedzy mna i miedzy wami, i miedzy kazda dusza zywiaca, która jest z wami, w rodzaje wieczne.
\par 13 Luk mój polozylem na obloku, który bedzie na znak przymierza miedzy mna, i miedzy ziemia.
\par 14 I stanie sie, gdy wzbudze ciemny oblok nad ziemia, a ukaze sie luk na obloku:
\par 15 Ze wspomne na przymierze moje, które jest miedzy mna i miedzy wami, i miedzy kazda dusza zywiaca w kazdem ciele; i nie beda wiecej wody na potop, ku wytraceniu wszelkiego ciala.
\par 16 Bedzie tedy luk on na obloku, i wejrze nan, abym wspomnial na przymierze wieczne, miedzy Bogiem i miedzy wszelka dusza zywiaca w kazdem ciele, które jest na ziemi.
\par 17 Zatem rzekl Bóg do Noego: Tenci jest znak przymierza, którem postanowil miedzy mna, i miedzy wszelkiem cialem, które jest na ziemi.
\par 18 A byli synowie Noego, którzy wyszli z korabia, Sem, i Cham, i Jafet; a Cham jest ojcem Chanaan.
\par 19 Ci trzej synowie Noego, przez które sie napelnila ludem wszystka ziemia.
\par 20 Tedy Noe poczal uprawiac ziemie, i nasadzil winnice.
\par 21 Potem pil wino; a upiwszy sie, odkryl sie w namiocie swoim.
\par 22 A ujrzawszy Cham, ojciec Chanaanów, nagosc ojca swego, oznajmil to dwom braciom swoim na dworze.
\par 23 Tedy wziawszy Sem i Jafet szate, a wlozywszy ja oba na ramiona swe, szli wspak, i zakryli nagosc ojca swego; a oblicza ich odwrócone byly, ze nagosci ojca swego nie widzieli.
\par 24 A ocuciwszy sie Noe z wina swego, gdy sie dowiedzial, co mu uczynil syn jego mlodszy, rzekl:
\par 25 Przeklety Chanaan, sluga slug braci swojej bedzie.
\par 26 Rzekl tez: Blogoslawiony Pan Bóg Semów, a niech bedzie Chanaan sluga ich.
\par 27 Niech rozszerzy Bóg Jafeta, i niech mieszka w namieciech Semowych a niech Chanaan sluga ich.
\par 28 I zyl Noe po potopie trzy sta lat, i piecdziesiat lat.
\par 29 I bylo wszystkich dni Noego, dziewiec set lat, i piecdziesiat lat, i umarl.

\chapter{10}

\par 1 Tec sa rodzaje synów Noego, Sema, Chama i Jafeta, którym sie narodzilo synów po potopie.
\par 2 Synowie Jafetowi Gomer, i Magog, i Madai, i Jawan, i Tubal, i Mesech, i Tyras.
\par 3 A synowie Gomerowi: Aschenaz, i Ryfat, i Togorma.
\par 4 A synowie Jawanowi: Elisa, i Tarsis, Cytym, i Dodanim.
\par 5 Od tych rozdzielone sa wyspy narodów po swych ziemiach; kazdy wedlug jezyka swego, i wedlug pokolenia swego, w narodziech swoich.
\par 6 A synowie Chamowi: Chus, i Micraim, i Put, i Chanaan.
\par 7 Synowie zas Chusowi: Seba, i Hewila, i Sabta, i Regma, i Sabtacha. A synowie Regmy: Seba i Dedan.
\par 8 A Chus splodzil Nemroda, który poczal byc moznym na ziemi.
\par 9 Ten byl moznym mysliwcem przed obliczem Panskiem; przetoz sie mówi: Jako Nemrod mozny mysliwiec przed Panem.
\par 10 A poczatek królestwa jego byl Babel, i Erech, i Achad, i Chalne w ziemi Senaar.
\par 11 Z tej ziemi wyszedl Assur, i zbudowal Niniwe, i Rechobot miasto, i Chale.
\par 12 Takze Resen, miedzy Niniwe i miedzy Chale; to miasto jest wielkie.
\par 13 Micraim tez splodzil Ludyma, i Hananima, i Laubima, i Neftuhyma.
\par 14 I Patrusyma, i Chasluchyma, (z których poszli Filistynowie,)i Kaftoryma.
\par 15 Chanaan tez splodzil Sydona pierworodnego swego, i Heta.
\par 16 I Jebusa, i Amorra, i Gergesa.
\par 17 I Hewa, i Archa, i Syma.
\par 18 I Arada, i Samara, i Chamata, skad sie potem rozrodzily domy Chananejczyków.
\par 19 A granice Chananejskie byly od Sydonu idac do Gerary, az do Gazy, az wnijdziesz do Sodomy i Gomorry, i Adamy, i Seboima, az do Lazy.
\par 20 Ci sa synowie Chamowi w familijach swych, w jezykach swych, w ziemiach swych, w narodziech swych.
\par 21 A Semowi, ojcu wszystkich synów Heberowych, bratu Jafeta starszego, urodzili sie synowie.
\par 22 Synowie Semowi: Elam i Assur, i Arfachsad, i Lud, i Aram.
\par 23 Synowie zas Aramowi: Hus, i Hul, i Geter, i Mesech.
\par 24 Arfachsad zas splodzil Selecha, a Selech splodzil Hebera.
\par 25 A Heberowi urodzili sie dwa synowie: imie jednemu Faleg, iz za dni jego rozdzielona jest ziemia; a imie brata jego Jektan.
\par 26 Jektan tez splodzil Elmodada, i Salefa, i Hasarmota, i Jarecha.
\par 27 I Adorama, i Uzala, i Dekla.
\par 28 I Hebala, i Abymaela, i Sebaja.
\par 29 I Ofira, i Hewila, i Jobaba: ci wszyscy sa synowie Jektanowi.
\par 30 A bylo mieszkanie ich od Mescha idac, do góry Sefar na wschód slonca.
\par 31 Cic sa synowie Semowi w domach swych, w jezykach swych, w ziemiach swych, w narodziech swych.
\par 32 Tec sa domy synów Noego, wedlug pokolenia ich, i w narodziech ich, i od nich rozdzielone sa narody na ziemi po potopie.

\chapter{11}

\par 1 A byla wszystka ziemia jednego jezyka, i jednej mowy.
\par 2 I stalo sie, gdy wyszli od wschodu slonca, znalezli równine w ziemi Senaar, i mieszkali tam.
\par 3 I rzekl jeden do drugiego: Nuze naczynmy cegly i wypalmy ja ogniem: i mieli cegle miasto kamienia, a gline ilowata mieli miasto wapna.
\par 4 Potem rzekli: Nuzez, zbudujmy sobie miasto i wieza, której by wierzch dosiegal do nieba, a uczynmy sobie imie; bysmy sie snac nie rozproszyli po obliczu wszystkiej ziemi.
\par 5 Tedy Pan zstapil, aby ogladal miasto ono, i wieza, która budowali synowie ludzcy.
\par 6 I rzekl Pan: Oto lud jeden, i jezyk jeden tych wszystkich; a toc jest zaczecie dziela ich, a teraz nie zabroni im nikt wszystkiego, co zamyslili uczynic.
\par 7 Przetoz zstapmy, a pomieszajmy tam jezyk ich, aby jeden drugiego jezyka nie zrozumial.
\par 8 A tak rozproszyl je Pan stamtad po obliczu wszystkiej ziemi; i przestali budowac miasta onego.
\par 9 Przetoz nazwal imie jego Babel; iz tam pomieszal Pan jezyk wszystkiej ziemi; i stamtad rozproszyl je Pan po obliczu wszystkiej ziemi.
\par 10 Tec sa rodzaje Semowe: Sem gdy mial sto lat, splodzil Arfachsada we dwa lata po potopie.
\par 11 I zyl Sem po splodzeniu Arfachsada piec set lat, i splodzil syny i córki.
\par 12 Arfachsad tez zyl trzydziesci i piec lat, i splodzil Selecha.
\par 13 I zyl Arfachsad po splodzeniu Selecha cztery sta lat, i trzy lata, i splodzil syny i córki.
\par 14 Selech zas zyl trzydziesci lat, i splodzil Hebera.
\par 15 I zyl Selech po splodzeniu Hebera cztery sta lat, i trzy lata, i splodzil syny i córki.
\par 16 I zyl Heber trzydziesci lat i cztery, i splodzil Pelega.
\par 17 Zyl tez Heber po splodzeniu Pelega, cztery sta lat, i trzydziesci lat, i splodzil syny i córki.
\par 18 Zyl tez Peleg trzydziesci lat, i splodzil Rehu.
\par 19 I zyl Peleg po splodzeniu Rehu dwiescie lat, i dziewiec lat, i splodzil syny i córki.
\par 20 Takze Rehu zyl trzydziesci lat, i dwa, i splodzil Saruga.
\par 21 I zyl Rehu po splodzeniu Saruga dwiescie lat, i siedem lat, i splodzil syny i córki.
\par 22 Sarug zas zyl trzydziesci lat, i splodzil Nachora.
\par 23 I zyl Sarug po splodzeniu Nachora dwiescie lat, i splodzil syny i córki.
\par 24 Takze Nachor zyl dwadziescia i dziewiec lat, i splodzil Tarego.
\par 25 I zyl Nachor po splodzeniu Tarego sto lat i dziewietnascie lat, i splodzil syny i córki.
\par 26 I zyl Tare siedemdziesiat lat, i splodzil Abrama, Nachora i Harana.
\par 27 A tec sa rodzaje Tarego: Tare splodzil Abrama, Nachora i Harana. Haran zas splodzil Lota.
\par 28 I umarl Haran przed obliczem Tarego ojca swego, w ziemi narodzenia swego, w Ur Chaldejskiem.
\par 29 I pojeli Abram i Nachor sobie zony: imie zony Abramowej bylo Saraj, a imie zony Nachorowej Melcha, córka Harana, ojca Melchy, i ojca Jeschy.
\par 30 A byla Saraj nieplodna, i nie miala dziatek.
\par 31 Wzial tedy Tare Abrama syna swego, i Lota syna Haranowego, wnuka swego, i Saraj niewiaste swoje, zone Abrama syna swego; i wyszli spolu z Ur Chaldejskiego, aby szli do ziemi Chananejskiej; a przyszli az do Haranu, i mieszkali tam.
\par 32 I bylo dni Tarego dwiescie lat, i piec lat; i umarl Tare w Haranie.

\chapter{12}

\par 1 I rzekl Pan do Abrama: Wynijdz z ziemi twej, i od rodziny twojej, i z domu ojca twego, do ziemi, którac pokaze.
\par 2 A uczynie cie w naród wielki, i bedec blogoslawil, i uwielbie imie twoje, i bedziesz blogoslawienstwem.
\par 3 I bede blogoslawil blogoslawiacym tobie; a przeklinajace cie przeklinac bede: i beda blogoslawione w tobie wszystkie narody ziemi.
\par 4 Tedy wyszedl Abram, jako mu rozkazal Pan. Poszedl tez z nim i Lot. A bylo Abramowi siedmdziesiat lat i piec lat, gdy wyszedl z Haran.
\par 5 Wzial tez Abram Saraj zone swoje, i Lota syna brata swego, i wszystke swa majetnosc, której nabyli, i dusze, których nabyli w Haranie, i wyszli, aby szli do ziemi Chananejskiej; i przyszli do ziemi Chananejskiej.
\par 6 Tedy przeszedl Abram ziemie one az do miejsca Sychem, i az do równiny Morech; a Chananejczyk na ten czas byl w onej ziemi.
\par 7 I ukazal sie Pan Abramowi, i rzekl: Nasieniu twemu dam ziemie te; i zbudowal tam oltarz Panu, który mu sie ukazal.
\par 8 A przeszedl stamtad do góry na wschód Betela, i rozbil tam namiot swój, majac Betel od zachodu, a Haj od wschodu; i zbudowal tam oltarz Panu, i wzywal imienia Panskiego.
\par 9 Potem ruszyl sie Abram idac, i ciagnac ku poludniu.
\par 10 A byl glód w ziemi onej: przeto zstapil Abram do Egiptu, aby tam byl gosciem do czasu, ciezki bowiem byl glód w ziemi.
\par 11 I stalo sie, gdy juz blisko byl, aby wszedl do Egiptu, rzekl do Sarai, zony swej: Oto teraz wiem, zes niewiasta piekna na wejrzeniu.
\par 12 I stanie sie, ze gdy cie obacza Egipczanie, rzeka: Zona to jego; i zabija mie, a ciebie zywo zostawia.
\par 13 Mów, prosze, zes jest siostra moja, aby mi dobrze bylo dla ciebie, i zywa zostala dla ciebie dusza moja.
\par 14 I stalo sie, gdy wszedl Abram do Egiptu, ujrzeli Egipczanie niewiaste one, iz byla bardzo piekna.
\par 15 Widzieli ja tez ksiazeta Faraonowe, i chwalili ja przed nim; i wzieto one niewiaste do domu Faraonowego.
\par 16 Który Abramowi dobrze czynil dla niej; i mial Abram owce, i woly, i osly i slugi, i sluzebnice, i oslice, i wielblady.
\par 17 Ale uderzyl Pan Faraona plagami wielkiemi, i dom jego dla Sarai, zony Abramowej.
\par 18 Przetoz wezwal Farao Abrama, i rzekl: Cózes mi to uczynil? czemus mi nie oznajmil, ze to zona twoja?
\par 19 Przeczzes powiedzial, siostra to moja? i wzialem ja sobie za zone; a teraz, oto zona twoja, wezmijze ja, a idz.
\par 20 I przykazal o nim Farao mezom, i puscili go wolno i zone jego, i wszystko, co bylo jego.

\chapter{13}

\par 1 A tak wyszedl Abram z Egiptu, on i zona jego, i wszystko co mial, i Lot z nim, ku poludniowi.
\par 2 A Abram byl bardzo bogaty w bydlo, w srebro, i w zloto.
\par 3 I szedl goscincami swemi, od poludnia, i az do Betel, az do onego miejsca, gdzie przedtem byl namiot jego, miedzy Betel i miedzy Haj.
\par 4 Do miejsca onego oltarza, który tam byl przedtem uczynil; i wzywal tam Abram imienia Panskiego.
\par 5 Takze Lot, który chodzil z Abramem, mial owce, i woly i namioty.
\par 6 I nie mogla ich zniesc ona ziemia, zeby spolem mieszkali, albowiem byla majetnosc ich wielka, tak, ze nie mogli mieszkac pospolu.
\par 7 I wszczal sie poswarek miedzy pasterzami trzody Abramowej, i miedzy pasterzami trzody Lotowej. Chananejczyk i Ferezejczyk mieszkal na on czas w ziemi.
\par 8 Rzekl tedy Abram do Lota: Niech prosze nie bedzie swaru miedzy mna i miedzy toba, takze miedzy pasterzami moimi i miedzy pasterzami twoimi, poniewazesmy bracia.
\par 9 Izali nie wszystka ziemia jest przed obliczem twojem? odlacz sie prosze ode mnie; jezli w lewa pójdziesz ja pójde w prawa, a jezli ty w prawa, ja sie udam w lewa.
\par 10 Tedy podnióslszy Lot oczy swe, obaczyl wszystke równine nad Jordanem, iz wszystka wilgotna byla przedtem, niz zatracil Pan Sodome i Gomorre, jako sad Panski, i jako ziemia Egipska, idac do Zoar.
\par 11 I obral sobie Lot wszystke one równine nad Jordanem, i odszedl Lot ku wschodu slonca, i rozlaczyli sie bracia jeden od drugiego.
\par 12 Abram mieszkal w ziemi Chananejskiej, a Lot mieszkal w miejscach onej równiny, i rozbil namiot az do Sodomy.
\par 13 Ale ludzie w Sodomie byli zli i wielcy grzesznicy przed Panem.
\par 14 I rzekl Pan do Abrama, potem gdy sie odlaczyl Lot od niego: Podnies teraz oczy swe, a spojrzyj z miejsca, na któremes teraz na pólnocy, i na poludnie, i na wschód, i na zachód slonca.
\par 15 Wszystke bowiem ziemie, która ty widzisz, dam tobie, i nasieniu twemu az na wieki.
\par 16 A rozmnoze nasienie twoje jako proch ziemi; bo jezli kto bedzie mógl zliczyc proch ziemi, tedy i nasienie twoje zliczone bedzie.
\par 17 Wstanze, schodz te ziemie wzdluz i wszerz, bo ja tobie dam.
\par 18 Ruszywszy sie tedy z namiotem Abram, przyszedl i mieszkal w równinach Mamre, które sa w Hebron i zbudowal tam oltarz Panu.

\chapter{14}

\par 1 I stalo sie za dni Amrafela, króla Senaarskiego, Aryjocha, króla Ellasarskiego, Chodorlahomera, króla Elamskiego i Tydala, króla Goimskiego:
\par 2 Ze podniesli wojne przeciw Borowi królowi Sodomskiemu, i przeciw Bersie królowi Gomorskiemu, i Senaabowi królowi Adamackiemu, i Semeberowi królowi Seboimskiemu, i królowi Belamskiemu, to jest Zoarskiemu.
\par 3 Wszyscy ci zaciagneli sie w doline Syddym, ta jest teraz morzem slonem.
\par 4 Bo ci dwanascie lat sluzyli Chodorlahomerowi, a trzynastego roku odstapili od niego.
\par 5 A tak roku czternastego wyciagnal Chodorlahomer z królmi, którzy z nim byli, i porazil Rafaimy w Astarot Karnaimie, i Zuzymy w Hamie, i Emimy w Sawie Karyjataim.
\par 6 Takze Chorajczyki na górze ich Seir, az do równiny Paran, która jest przy puszczy.
\par 7 Potem sie wrócili, i przyciagneli do En Myspat, która jest Kades, i wybili wszystke kraine Amalekitów; takze tez Amorrejczyka mieszkajacego w Hasesontamar.
\par 8 Tedy wyciagnal król Sodomski, i król Gomorski, i król Adamacki, i król Zeboimski, i król Belamski, to jest Zoarski, i uszykowali sie ku bitwie przeciwko im w dolinie Syddym.
\par 9 Przeciwko Chodorlahomerowi królowi Elamskiemu, i Tydalowi królowi Goimskiemu, i Amrafelowi królowi Senaarskiemu, i Aryjochowi królowi Ellasarskiemu, czterech królów, przeciw pieciu.
\par 10 A w onej dolinie Syddym, bylo wiele studzien ilowatych; i uciekali król Sodomski i Gomorski, a polegli tam, a którzy zostali, na góre uciekli.
\par 11 A zabrawszy wszystke majetnosc Sodomska, i Gomorska, i wszystke zywnosc ich, odciagneli.
\par 12 Zabrali tez Lota synowca Abramowego, i majetnosc jego, i poszli; bo on mieszkal w Sodomie.
\par 13 I przyszedl jeden, który uszedl, i oznajmil to Abramowi Hebrejczykowi, który mieszkal w równinach Mamrego Amorrejczyka, brata Eschola, i brata Anera; ci bowiem uczynili byli przymierze z Abramem.
\par 14 A uslyszawszy Abram, iz byl pojmany brat jego, wyprawil cwiczonych slug swoich zrodzonych w domu swym, trzy sta i osiemnascie, i gonil je az do Dan.
\par 15 I rozdzieliwszy sie przypadl na nie w nocy, sam i sludzy jego, i porazil je; i gonil je az do Hoby, która lezy po lewej stronie Damaszku.
\par 16 I odebral nazad wszystke majetnosc, takze i Lota brata swego z majetnoscia jego wrócil, takze i niewiasty, i lud.
\par 17 Tedy wyszedl król Sodomski przeciw niemu, gdy sie wracal od porazki Chodorlahomera, i królów, którzy z nim byli na dolinie Sawe, która jest dolina królewska.
\par 18 A Melchisedek, król Salemski, wyniósl chleb i wino; a ten byl kaplanem Boga najwyzszego.
\par 19 I blogoslawil mu, a rzekl: Blogoslawiony Abram od Boga najwyzszego, dzierzawcy nieba i ziemi.
\par 20 I blogoslawiony Bóg najwyzszy, który podal nieprzyjacioly twe w reke twoje; i dal mu Abram dziesiecine ze wszystkiego.
\par 21 Zatem rzekl król Sodomski do Abrama: Daj mi ludzie, a majetnosc pobierz sobie.
\par 22 Tedy rzekl Abram królowi Sodomskiemu: Podnioslem reke swa ku Panu Bogu najwyzszemu, dzierzawcy nieba i ziemi;
\par 23 Ze i najmniejszej nitki ani rzemyczka obuwia nie wezme ze wszystkiego, co twego jest; zebys nie rzekl: Jam zbogacil Abrama.
\par 24 Okrom tego, co strawili sludzy, i okrom dzialu mezów, którzy chodzili ze mna, Anera, Eschola, i Mamrego; ci niech wezma dzial swój.

\chapter{15}

\par 1 Po tem wszystkiem stalo sie slowo Panskie do Abrama w widzeniu, mówiac: Nie bój sie Abramie, jam tarcza twoja, i nagroda twoja obfita wielce.
\par 2 I rzekl Abram: Panie Boze, cóz mi dasz? gdyz ja schodze bez dziatek, a sprawca domu mego jest ten Damaszczenski Eliezer.
\par 3 I mówil Abram: Otos mi nie dal potomka, ale oto sluga domu mego dziedzicem moim bedzie.
\par 4 A oto slowo Panskie stalo sie do niego mówiac: Nie bedzie ten dziedzicem twoim; lecz który wynijdzie z zywota twego, ten bedzie dziedzicem twoim.
\par 5 I wywiódl go na dwór, i rzekl: Spojrzyj teraz ku niebu, a zlicz gwiazdy, bedzieszli je mógl zliczyc; i rzekl mu: Tak bedzie nasienie twoje.
\par 6 Uwierzyl tedy Panu, i poczytano mu to ku sprawiedliwosci.
\par 7 I rzekl do niego: Ja Pan, którym cie wywiódl z Ur Chaldejskiego, abym ci dal ziemie te w osiadlosc.
\par 8 Zatem rzekl Abram: Panie Boze, po czemze poznam, iz ja odziedzicze?
\par 9 I odpowiedzial mu: Wezmij mi jalowice trzyletnia, i koze trzyletnia, i barana trzyletniego, i synogarlice, i golabiatko.
\par 10 Wzial tedy wszystko to i rozcial na poly; a jedne czesc polozyl przeciw drugiej, ale ptaków nie rozcinal.
\par 11 Tedy sie zlecialo ptactwo do onych scierwów, i odganial je Abram.
\par 12 I stalo sie, gdy slonce zachodzilo, ze przypadl twardy sen na Abrama, a oto strach i ciemnosc wielka przypadla nan.
\par 13 I rzekl Pan do Abrama: Wiedz wiedzac, iz gosciem bedzie nasienie twoje w ziemi cudzej, i podbija je w niewola, i utrapia je przez cztery sta lat.
\par 14 A wszakze naród on, któremu sluzyc beda, ja sadzic bede; a potem wynijda stamtad z majetnoscia wielka.
\par 15 Ale ty pójdziesz do ojców twoich w pokoju; i pogrzebion bedziesz w starosci dobrej.
\par 16 A w czwartem pokoleniu tu sie wróca; bo jeszcze nie wypelnila sie nieprawosc Amorrejczyka az do tego czasu.
\par 17 I stalo sie, gdy zaszlo slonce, a ciemnosc byla, a oto ukazal sie piec kurzacy sie, i pochodnia ognista, która przechodzila miedzy onemi podzialy.
\par 18 Onegoz dnia uczynil Pan z Abramem przymierze, mówiac: Nasieniu twemu dam te ziemie, od rzeki Egipskiej, az do rzeki wielkiej, rzeki Eufrates.
\par 19 Kenejczyka, i Kenezejczyka, i Kadmonejczyka.
\par 20 I Hettejczyka, i Ferezejczyka, i Rafaimczyka.
\par 21 I Amorrejczyka, i Chananejczyka, i Gergezejczyka, i Jebuzejczyka.

\chapter{16}

\par 1 Saraj tedy, zona Abramowa, nie rodzila mu; ale miala sluge Egipczanke, której imie bylo Agar.
\par 2 I rzekla Saraj do Abrama: Oto teraz zamknal mie Pan, abym nie rodzila; wnijdz, prosze, do sluzebnicy mojej, azali wzdy z niej bede miala dziatki; i usluchal Abram glosu Sarai.
\par 3 I wziela Saraj, zona Abramowa, Agare Egipczanke, sluzebnice swoje, po dziesieciu latach, jako poczal Abram mieszkac w ziemi Chananejskiej; i dala ja Abramowi mezowi swemu za zone.
\par 4 Tedy wszedl do Agary, i poczela; a widzac, ze poczela, wzgardzona byla pani jej w oczu jej.
\par 5 I rzekla Saraj do Abrama: Krzywdy mojej tys winien; jamci dala sluzebnice moje na lono twoje; ale ona, widzac ze poczela, wzgardzila mie w oczach swych; niech rozsadzi Pan miedzy mna i miedzy toba.
\par 6 I rzekl Abram do Sarai: Oto sluzebnica twoja w rekach twoich, czyn z nia coc sie zda najlepszego; i trapila ja Saraj, i uciekla od oblicza jej.
\par 7 I znalazl ja Aniol Panski u zródla wód na puszczy, nad zródlem, przy drodze Sur.
\par 8 I rzekl: Agaro, sluzebnico Sarai, skad idziesz? i dokad idziesz? a ona odpowiedziala: Od oblicza Sarai, pani swej, ja uciekam.
\par 9 Rzekl jej Aniol Panski: Wróc sie do pani swej, a ukorz sie pod rece jej.
\par 10 Rzekl jej zas Aniol Panski: Mnozac rozmnoze nasienie twoje, iz nie bedzie moglo byc zliczone przez mnóstwo.
\par 11 Potem jej rzekl Aniol Panski: Otos ty poczela, i porodzisz syna, a nazwiesz imie jego Ismael; bo uslyszal Pan utrapienie twoje.
\par 12 Ten bedzie srogim czlowiekiem: reka jego przeciwko wszystkim, a reka wszystkich przeciwko jemu; a przed obliczem wszystkiej braci swej mieszkac bedzie.
\par 13 I nazwala imie Pana, który mówil do niej: Tys Bóg widzacy mie; rzekla bowiem: Izalim tu nie widziala tylu widzacego mie?
\par 14 Przetoz nazwala studnia one studnia zywiacego, widzacego mie; a tac jest miedzy Kades, i miedzy Barad.
\par 15 I urodzila Agar Abramowi syna, i nazwal Abram imie syna swego, którego urodzila Agar, Ismael.
\par 16 A Abram mial osiemdziesiat lat, i szesc lat, gdy mu urodzila Agar Ismaela.

\chapter{17}

\par 1 A gdy juz bylo Abramowi dziewiecdziesiat lat i dziewiec lat, ukazal sie Pan Abramowi, i rzekl do niego: Jam jest Bóg Wszechmogacy; chodz przed obliczem mojem, a badz doskonaly.
\par 2 A uczynie przymierze moje, miedzy mna i miedzy toba, i rozmnoza cie bardzo obficie.
\par 3 Tedy upadl Abram na oblicze swoje, i rzekl do niego Bóg, mówiac:
\par 4 Jam jest, oto stanowie przymierze moje z toba, i bedziesz ojcem wielu narodów.
\par 5 I nie bedzie zwane dalej imie twoje Abram; ale bedzie imie twoje Abraham; albowiem ojcem wielu narodów postanowilem cie.
\par 6 A rozmnoze cie bardzo, i rozkrzewie cie w narody, i królowie z ciebie wynijda.
\par 7 I utwierdze przymierze moje miedzy mna, i miedzy toba, i miedzy nasieniem twojem po tobie, w narodziech ich umowa wieczna; zebym ci byl Bogiem i nasieniu twemu po tobie.
\par 8 Dam tez tobie, i nasieniu twemu po tobie ziemie, w której teraz jestes gosciem; wszystke ziemie Chananejska w osiadlosc wieczna, i bede Bogiem ich.
\par 9 Nad to rzekl Bóg Abrahamowi: Ty tez przymierza mego przestrzegac bedziesz, ty i nasienie twoje po tobie, w narodziech swoich.
\par 10 A toc jest przymierze moje, które zachowywac bedziecie, miedzy mna, i miedzy wami, i miedzy nasieniem twojem po tobie, aby byl obrzezany miedzy wami kazdy mezczyzna.
\par 11 Obrzezcie tedy cialo nieobrzezki waszej; a to bedzie znakiem przymierza miedzy mna, i miedzy wami.
\par 12 Syn osmiu dni, bedzie obrzezany miedzy wami kazdy mezczyzna w narodziech waszych, tak doma narodzony jako i kupiony za pieniadze, od jakiegozkolwiek cudzoziemca, któryby nie byl z nasienia twego.
\par 13 Koniecznie obrzezany bedzie, urodzony w domu twoim, i kupiony za pieniadze twoje; a bedzie przymierze moje na ciele waszem, na przymierze wieczne.
\par 14 A nie obrzezany mezczyzna, którego by nie bylo obrzezane cialo nieobrzezki jego, bedzie wytracona dusza ona z ludu swego; albowiem zgwalcil przymierze moje.
\par 15 Potem rzekl Bóg do Abrahama: Sarai, zony twojej, nie bedziesz zwal imienia jej Saraj, ale Sara bedzie imie jej.
\par 16 I bede jej blogoslawil, a dam ci z niej syna; bede jej blogoslawil, i bedzie rozmnozona w narody, a królowie narodów z niej wynijda.
\par 17 Tedy Abraham padl na oblicze swoje, i rozesmial sie, a mówil w sercu swem: Zaz czlowiekowi stuletniemu urodzi sie syn? i azaz Sara w dziewiecdziesieciu latach porodzi?
\par 18 I rzekl Abraham do Boga: O by tylko Ismael zyl przed obliczem twojem!
\par 19 I rzekl Bóg: Zaiste, Sara, zona twoja, urodzi tobie syna, i nazowiesz imie jego Izaak; i utwierdze przymierze moje z nim, umowa wieczna, i z nasieniem jego po nim.
\par 20 O Ismaela tez wysluchalem cie: oto, blogoslawilem mu, i rozrodze go, i rozmnoze go bardzo wielce. Dwanascie ksiazat splodzi, i rozkrzewie go w naród wielki.
\par 21 Ale przymierze moje utwierdze z Izaakiem, którego tobie urodzi Sara, o tym czasie w roku drugim.
\par 22 A przestawszy mówic z nim, odszedl Bóg od Abrahama.
\par 23 Tedy wzial Abraham Ismaela, syna swego, i wszystkie urodzone w domu swym, i wszystkie kupione za pieniadze, kazdego mezczyzne, z mezów domu Abrahamowego, i obrzezal cialo nieobrzeski ich, onegoz to dnia, jako mówil z nim Bóg.
\par 24 A Abrahamowi bylo dziewiecdziesiat lat i dziewiec, gdy obrzezane bylo cialo nieobrzeski jego.
\par 25 A Ismaelowi synowi jego bylo trzynascie lat, gdy obrzezane bylo cialo nieobrzeski jego.
\par 26 Tegoz dnia obrzezany jest Abraham, i Ismael, syn jego.
\par 27 I wszyscy mezowie domu jego, urodzeni w domu, i kupieni za pieniadze od cudzoziemców, obrzezani sa z nim.

\chapter{18}

\par 1 Potem ukazal mu sie Pan w równinie Mamre, a on siedzial we drzwiach namiotu swego, gdy byl najgoretszy dzien.
\par 2 A podnióslszy oczy swe, obaczyl, a oto trzej mezowie staneli przeciw niemu; i ujrzawszy je, wybiezal przeciwko nim ze drzwi namiotu, i poklonil sie do ziemi.
\par 3 I rzekl: Panie mój, jezlim teraz znalazl laske w oczach twoich, nie mijaj, prosze, slugi swego.
\par 4 Przyniosa troche wody, a umyjecie nogi wasze, i odpoczniecie pod drzewem.
\par 5 I przyniosa kes chleba, a posilicie serce wasze; potem odejdziecie, dla tegoscie bowiem przyszli do mnie slugi swego. Tedy rzekli: Tak uczyn, jakos powiedzial.
\par 6 I pospieszyl sie Abraham do namiotu do Sary, i rzekl: Spiesz sie: rozczyn trzy miarki maki swiatlej, a uczyn podplomyków.
\par 7 Abraham zas szedl do trzody, i wzial ciele mlode i wyborne, i dal je sludze, który sie pospieszyl, i nagotowal je.
\par 8 Wzial tez masla i mleka, i ciele, które byl nagotowal, i postawil przed nie, a sam stal przy nich pod drzewem; i jedli.
\par 9 I rzekli do niego: Gdzie jest Sara, zona twoja? a on odpowiedzial: Oto jest w namiocie.
\par 10 Tedy rzekl Pan: Wróce sie pewnie do ciebie o tymze czasie w rok, a oto, bedzie miala syna Sara, zona twoja; a Sara sluchala u drzwi namiotu, które byly za nim.
\par 11 A Abraham i Sara byli starzy, i zeszli w leciech, ;i przestalo bywac Sarze wedlug zwyczaju niewiast.
\par 12 I rozesmiala sie Sara sama w sobie, mówiac: Gdym sie zestarzala, rozkoszy zazywac bede; i pan mój zestarzal sie.
\par 13 Zatem rzekl Pan do Abrahama: Czemu sie rozsmiala Sara, mówiac: Zaz prawdziwie porodze, gdym sie zestarzala? Izali jest co trudnego u Pana?
\par 14 O tymze czasie wróce do ciebie roku przyszlego, a Sara bedzie miala syna.
\par 15 I zaprzala sie Sara, mówiac: Nie smialam sie; bo sie bala. A Pan rzekl: Nie mów tak; bos sie smiala.
\par 16 Potem wstali stamtad mezowie oni, i obrócili sie ku Sodomie; a Abraham szedl z nimi wyprowadzajac je.
\par 17 Tedy rzekl Pan: Izali ja zataje przed Abrahamem, co mam uczynic?
\par 18 Poniewaz Abraham pewnie rozmnozon bedzie w lud wielki i mozny, a w nim beda ublogoslawione wszystkie narody ziemi.
\par 19 Znam go bowiem; przetoz przykaze synom swoim, i domowi swemu po sobie, aby strzegli drogi Panskiej, i czynili sprawiedliwosc i sad; aby przywiódl Pan na Abrahama, co mu powiedzial.
\par 20 Rzekl tedy Pan: Krzyk Sodomy i Gomorry, iz sie rozmnozyl grzech ich, iz bardzo ociezal;
\par 21 Zstapie teraz, a obacze, jezli sie wedlug krzyku tego, który mie doszedl, do konca sprawuja; a jezliz nie, abym sie wzdy dowiedzial.
\par 22 I obrócili sie stamtad mezowie, i poszli do Sodomy; lecz Abraham jeszcze stal przed Panem.
\par 23 I przystapil Abraham, i rzekl: Izali tez zatracisz sprawiedliwego z niezboznym?
\par 24 Jezli snac bedzie piecdziesiat sprawiedliwych w tem miescie, izali je wytracisz, a nie przepuscisz miejscu temu dla piecdziesiat sprawiedliwych, którzy w niem sa?
\par 25 Niech to nie bedzie u ciebie, abys uczynic mial rzecz takowa, i zabil sprawiedliwego z niezboznym, a zeby byl sprawiedliwy, jako niezbozny. Niech to nie bedzie u ciebie. Izali Sedzia wszystkiej ziemi nie uczyni sprawiedliwosci?
\par 26 Tedy rzekl Pan: Jezli znajde w Sodomie piecdziesiat sprawiedliwych w samem miescie, odpuszcze wszystkiemu miejscu dla nich.
\par 27 A odpowiadajac Abraham rzekl: Otom teraz zaczal mówic do Pana mego, aczem ja proch i popiól.
\par 28 A jezliby nie stawalo do piecdziesieciu sprawiedliwych, pieciu, izali wytracisz dla tych pieciu wszystko miasto? I rzekl Pan: Nie wytrace, jezli tam znajde czterdziestu i pieciu.
\par 29 Na to jeszcze mówiac do niego Abraham rzekl: A jezliby sie ich tam znalazlo czterdziesci? i odpowiedzial: Nie uczynie nic dla tych czterdziestu.
\par 30 I rzekl Abraham: Prosze niech sie nie gniewa Pan mój, ze jeszcze mówic bede: A jezliby sie ich tam znalazlo trzydziesci? odpowiedzial: Nie uczynie, jezliz tam znajde trzydziestu.
\par 31 Tedy jeszcze rzekl Abraham: Otom teraz zaczal mówic do Pana mego: A jezliby sie ich tam snac znalazlo dwadziescia? odpowiedzial Pan: Nie zatrace i dla tych dwudziestu.
\par 32 Nad to rzekl Abraham: Prosze niech sie nie gniewa Pan mój, ze jeszcze raz tylko przemówie: A jezliby sie ich tam znalazlo dziesiec? Tedy rzekl Pan: Nie wytrace i dla tych dziesieciu.
\par 33 I poszedl Pan skonczywszy rozmowe z Abrahamem; a Abraham wrócil sie do miejsca swego.

\chapter{19}

\par 1 I przyszli dwaj Aniolowie do Sodomy w wieczór, a Lot siedzial w bramie Sodomskiej. Gdy je tedy ujrzal Lot, wstawszy szedl przeciwko nim, i sklonil sie twarza ku ziemi, i rzekl:
\par 2 Oto prosze panowie moi, wstapcie teraz do domu slugi swego, a badzcie tu na noc, i umyjcie nogi swe; potem rano wstawszy pójdziecie w droge wasze. Którzy odpowiedzieli: Bynajmniej; ale na ulicy bedziemy nocowali.
\par 3 Ale on przymuszal ich bardzo, iz sie sklonili do niego, i weszli w dom jego; zaczem sprawil im uczte, i napiekl chleba przasnego, i jedli.
\par 4 Lecz pierwej niz oni poszli spac, oto obywatele miasta, mezowie Sodomscy, obstapili dom, od mlodego az do starego, wszystek lud zewszad.
\par 5 I wolali na Lota, i rzekli mu: Gdzie sa mezowie, którzy przyszli do ciebie w nocy? wywiedz je do nas, abysmy je poznali.
\par 6 Tedy wyszedl do nich Lot ze drzwi, i zamknal drzwi za soba.
\par 7 I rzekl: Nie czyncie prosze, bracia moi, nic zlego.
\par 8 Oto teraz mam dwie córki, które nie poznaly meza; wywiode je teraz do was, a czyncie z niemi, co sie wam podoba, tylko mezom tym nic nie czyncie; bo dlatego weszli pod cien dachu mego.
\par 9 A oni rzekli: Pójdzze tam; i mówili: Ten sam przyszedl, aby tu gosciem byl, a mialby nas sadzic? przetoz gorzej uczynimy tobie, niz onym i czynili gwalt wielki mezowi onemu Lotowi, i przystapili, aby drzwi wylamali.
\par 10 Ale mezowie oni, wyciagnawszy reke swoje, wwiedli Lota do siebie w dom, i zamkneli drzwi.
\par 11 A meze one, którzy byli u drzwi domu, pozarazali slepota, od najmniejszego, az do najwiekszego; tak, iz sie spracowali, szukajac drzwi.
\par 12 Tedy rzekli mezowie oni do Lota: Maszli tu jeszcze kogo, ziecia, albo syny twe, albo córki twoje, i wszystko, co masz w miescie, wyprowadz z miejsca tego.
\par 13 Skazimy bowiem to miejsce, przeto, ze sie wzmógl krzyk ich przed Panem, i poslal nas Pan, abysmy je skazili.
\par 14 Tedy wyszedlszy Lot mówil do zieciów swoich, którzy mieli pojac córki jego, i rzekl: Wstancie, wynijdzcie z miejsca tego, bo skazi Pan to miasto; ale sie zdalo w oczach zieciów jego, jakoby zartowal.
\par 15 A gdy weszla zorza, przymuszali Aniolowie Lota, mówiac: Wstan, wezmij zone twoje, i dwie córki twoje, które tu sa, bys snac nie zginal w nieprawosci miasta tego.
\par 16 A gdy sie ociagal, ujeli mezowie oni reke jego, i reke zony jego, i reke dwóch córek jego, (albowiem mu Pan folgowal,)i wywiedli go, i postawili go przed miastem.
\par 17 I gdy je wywiedli precz, rzekl jeden: Jezli chcesz, zachowaj dusze twoje, a nie ogladaj sie nazad, ani stawaj na tej wszystkiej równinie; uchodz na góre, bys snac nie zginal.
\par 18 A Lot rzekl do nich: Nie tak, prosze, panowie moi;
\par 19 Oto teraz znalazl sluga twój laske w oczach twoich, i okazales obficie milosierdzie twoje, któres uczynil ze mna, zachowawszy dusze moje; alec ja nie bede mógl ujsc na te góre, by mie snac nie zachwycilo to zle, i umarlbym.
\par 20 Ale oto tu jest miasto nie daleko, do którego bym uciekl, malec jest; prosze niech tam ujde, (wszak male jest,)a bedzie zywa dusza moja.
\par 21 Tedy rzekl do niego: Oto, i wtem wysluchalem cie, abym nie wywrócil miasta tego, o któremes mówil.
\par 22 Spieszze sie a uchodz tam, bo nie bede mógl nic uczynic, az ty tam dojdziesz; przetoz nazwane jest imie miasta onego Zoar.
\par 23 Wtem slonce weszlo na ziemie, a Lot wszedl do Zoar.
\par 24 Tedy Pan spuscil jako deszcz na Sodome i na Gomorre siarke i ogien, od Pana z nieba.
\par 25 I wywrócil miasta one, i wszystke one równine, wszystkie obywatele miast onych, i urodzaje onej ziemi.
\par 26 I obejrzala sie zona jego idac za nim, a obrócila sie w slup solny.
\par 27 Wstawszy tedy Abraham rano, pospieszyl sie na ono miejsce, kedy stal przed Panem.
\par 28 I spojrzal ku Sodomie i Gomorze, i ku wszystkiej ziemi onej równiny, i obaczyl, a oto wychodzil dym z onej ziemi, jako dym z pieca.
\par 29 A gdy wywracal Bóg miasta onej równiny, wspomnial Bóg na Abrahama i wybawil Lota z posrodku wywrócenia, gdy wywracal one miasta, w których Lot mieszkal.
\par 30 Potem wyszedl Lot z Zoar, i mieszkal na górze, i dwie córki jego z nim, albowiem sie bal mieszkac w Zoar; ale mieszkal w jaskini, on i dwie córki jego.
\par 31 Tedy rzekla starsza do mlodszej: Ojciec nasz stary a nie masz meza na ziemi, który by wszedl do nas, wedlug zwyczaju wszystkiej ziemi.
\par 32 Pójdz, upójmy ojca naszego winem, a spijmy z nim, abysmy zachowaly z ojca naszego nasienie.
\par 33 Daly tedy pic ojcu swemu wina onej nocy. I wszedlszy starsza spala z ojcem swym; ale on nie czul ani kiedy sie ukladla, ani kiedy wstala.
\par 34 I stalo sie nazajutrz, ze rzekla starsza do mlodszej: Otom ja spala przeszlej nocy z ojcem swym, dajmyz mu pic wina jeszcze tej nocy, i wnijdziesz, i bedziesz spala z nim, a zachowamy z ojca naszego nasienie.
\par 35 Tedy daly pic i onej nocy ojcu swemu wina; i przyszedlszy mlodsza spala z nim; ale on nie czul, ani kiedy sie ukladla, ani kiedy wstala.
\par 36 A tak poczely obie córki Lotowe z ojca swego.
\par 37 I urodzila starsza syna, a nazwala imie jego Moab; ten jest ojcem Moabitów, az do dnia tego.
\par 38 Mlodsza tez urodzila syna, i nazwala imie jego Benammi; ten jest ojcem synów Ammonowych, az do dnia tego.

\chapter{20}

\par 1 I ruszyl sie stamtad Abraham do ziemi poludniowej, a mieszkal miedzy Kades i miedzy Sur, i byl gosciem w Gerar.
\par 2 Tam powiedzial Abraham o Sarze, zonie swej: Siostra moja jest; przetoz poslal Abimelech, król Gerary, i wzial Sare.
\par 3 Ale Bóg przyszedl do Abimelecha we snie w nocy, i rzekl mu: Oto ty umrzesz dla niewiasty, któras wzial, bo ona ma meza.
\par 4 Ale Abimelech nie przyblizyl sie byl do niej, i rzekl: Panie, izali tez lud sprawiedliwy zabijesz?
\par 5 Azaz mi on sam nie powiadal, siostra moja jest? a ona tez sama nie mówila, brat mój jest? w prostosci serca mojego, i w niewinnosci rak moich uczynilem to.
\par 6 Tedy mu rzekl Bóg we snie: Wiemci ja, zes to w prostosci serca swego uczynil; i dla tegom cie zawsciagnal, abys nie zgrzeszyl przeciwko mnie, i nie dopuscilem ci, abys sie jej dotknal.
\par 7 Teraz tedy wróc zone mezowi, bo prorokiem jest; i bedzie sie modlil za cie, a bedziesz zyl; a jezliz jej nie wrócisz, wiedz, iz smiercia umrzesz, ty, i wszystko, co twego jest.
\par 8 Tedy Abimelech wstawszy rano, zwolal wszystkich slug swoich, i opowiedzial im to wszystko; co uslyszawszy, polekali sie mezowie oni bardzo.
\par 9 Potem wezwal Abimelech Abrahama, i rzekl mu: Cos nam uczynil? a com zgrzeszyl przeciwko tobie? izes przywiódl na mie i na królestwo moje grzech wielki? uczyniles mi, czegos czynic nie mial.
\par 10 I rzekl po wtóre Abimelech do Abrahama: Cózes upatrywal, zes te rzecz uczynil?
\par 11 I odpowiedzial Abraham: Myslilem sobie: Podobno nie masz bojazni Bozej na tem miejscu, i zabija mie dla zony mojej.
\par 12 A wszakze prawdziwie siostra moja jest, córka ojca mego, choc nie córka matki mojej; pojalem ja za zone.
\par 13 I stalo sie, gdy mie wyprawil Bóg na pielgrzymowanie z domu ojca mego, zem rzekl do niej: To milosierdzie twoje bedzie, które uczynisz ze mna: Na kazdem miejscu, do którego przyjdziemy, powiesz o mnie: Brat to mój jest.
\par 14 Tedy nabrawszy Abimelech owiec, i wolów, i slug, i sluzebnic, dal Abrahamowi, i wrócil mu Sare, zone jego.
\par 15 I rzekl Abimelech: Oto ziemia moja przed obliczem twojem; gdziec sie kolwiek podoba, mieszkaj.
\par 16 A do Sary rzekl: Otom dal tysiac srebrników bratu twemu, onci jest zaslona oczu twoich u wszystkich, którzy sa z toba; a tem wszystkiem Sara wyuczona byla.
\par 17 I modlil sie Abraham Bogu, a uzdrowil Bóg Abimelecha, i zone jego, i sluzebnice jego, i rodzily.
\par 18 Zawarl bowiem byl Pan cale kazdy zywot domu Abimelechowego dla Sary, zony Abrahamowej.

\chapter{21}

\par 1 A Pan nawiedzil Sare, jako byl rzekl: i uczynil Pan Sarze, jako byl powiedzial.
\par 2 Bo poczela i porodzila Sara Abrahamowi syna w starosci jego, na tenze czas, który mu byl Bóg przepowiedzial.
\par 3 I nazwal Abraham imie syna swego, który mu sie urodzil, którego mu urodzila Sara, Izaak.
\par 4 I obrzezal Abraham Izaaka, syna swego, gdy byl w osmiu dniach, jako mu byl rozkazal Bóg.
\par 5 A bylo Abrahamowi sto lat, gdy mu sie urodzil Izaak, syn jego.
\par 6 Tedy rzekla Sara: Smiech mi uczynil Bóg; ktokolwiek uslyszy, smiac sie bedzie ze mna.
\par 7 I rzekla: Któzby to byl rzekl Abrahamowi, ze Sara bedzie karmila piersiami syny? gdyzem urodzila syna w starosci jego.
\par 8 Roslo tedy dziecie, i odstawione jest od piersi; i uczynil Abraham uczte wielka w dzien odstawienia Izaaka.
\par 9 Potem ujrzala Sara syna Hagary, Egipczanki, przeszydzajacego, którego urodzila Abrahamowi;
\par 10 I rzekla do Abrahama: Wyrzuc te sluzebnice, i syna jej; albowiem nie bedzie dziedziczyl syn tej sluzebnicy z synem mym Izaakiem.
\par 11 Ale sie to bardzo nie podobalo w oczach Abrahamowych, dla syna jego.
\par 12 Tedy rzekl Bóg do Abrahama: Niech to przykro nie bedzie w oczach twoich z strony dzieciecia, i z strony sluzebnicy twojej; cockolwiek rzecze Sara, usluchaj glosu jej; boc w Izaaku nazwane bedzie nasienie.
\par 13 Wszakze i syna sluzebnicy rozmnoze w naród, przeto iz nasieniem twojem jest.
\par 14 Wstal tedy Abraham bardzo rano, a wziawszy chleb i lagiew wody, dal Hagarze; i wlozywszy to na ramie jej, i z dziecieciem, odprawil ja; która poszedlszy blakala sie po puszczy Beerseba.
\par 15 A gdy nie stalo wody w lagwi, porzucila dziecie pod jednem drzewem;
\par 16 I odszedlszy usiadla przeciw niemu, tak daleko, jako na strzeleniu z luku; bo mówila: Nie bede patrzyla na smierc dzieciecia; a siedzac przeciw niemu, podniosla glos swój, i plakala.
\par 17 Tedy uslyszal Bóg glos dzieciecy, i zawolal Aniol Bozy na Hagare z nieba, i rzekl jej: Cózci Hagaro? nie bój sie, boc uslyszal Bóg glos dzieciecy z miejsca, na którem jest.
\par 18 Wstan, wezmij dziecie, a ujmij je reka swoja: bo w naród wielki rozmnoze je.
\par 19 Otworzyl tedy Bóg oczy jej, ze ujrzala zródlo wody; a szedlszy napelnila lagiew woda, i dala pic dziecieciu.
\par 20 I byl Bóg z onem dziecieciem, Które uroslo, i mieszkalo na puszczy, byl z niego strzelec dobry z luku.
\par 21 A mieszkal na puszczy Faran; i wziela mu matka jego zone z ziemi Egipskiej.
\par 22 I stalo sie onegoz czasu, ze rzekl Abimelech, i Fikol, hetman wojska jego, do Abrahama mówiac: Bóg z toba we wszystkiem, co ty czynisz.
\par 23 A tak teraz, przysiaz mi przez Boga, ze mie w niczem podchodzic nie bedziesz, ani syna mego, ani wnuka mego; ale wedlug milosierdzia, którem uczynil z toba, uczynisz ze mna, i z ziemia, w którejs byl przychodniem.
\par 24 Tedy odpowiedzial Abraham: Ja przysiegne.
\par 25 I przymawial Abraham Abimelechowi o studnia wody, która mu byli gwaltem odjeli sludzy Abimelechowi.
\par 26 I rzekl Abimelech: Nie wiem kto by to uczynil, nawet i tys mi nie oznajmil, i jam nie slyszal o tem dopiero dzis.
\par 27 Nabral tedy Abraham owiec i wolów, i dal Abimelechowi, i uczynili oba przymierze.
\par 28 I postawil Abraham siedmioro owiec z stada osobno.
\par 29 Tedy Abimelech rzekl do Abrahama: Na cóz to siedmioro owiec, któres postawil osobno?
\par 30 A on odpowiedzial: Iz te siedem owiec wezmiesz z rak moich, aby mi byly na swiadectwo, zem wykopal te studnie.
\par 31 Dlatego nazwano miejsce ono Beerseba; albowiem tam obaj przysiegli.
\par 32 A tak zawarli przymierze w Beerseba. Potem wstawszy Abimelech, i Fikol, hetman wojska jego, wrócili sie do ziemi Filistynskiej.
\par 33 I nasadzil Abraham drzewa w Beerseba, i wzywal tam imienia Pana Boga wiecznego.
\par 34 I mieszkal Abraham w ziemi Filistynskiej przez wiele dni.

\chapter{22}

\par 1 To gdy sie stalo, kusil Bóg Abrahama, i rzekl do niego: Abrahamie! A on odpowiedzial: Owom ja.
\par 2 I rzekl Bóg: Wezmij teraz syna twego, jedynego twego, którego milujesz, Izaaka, a idz do ziemi Moryja, i tam go ofiaruj na ofiare palona, na jednej górze, o którejc powiem.
\par 3 Tedy wstawszy Abraham bardzo rano, osiodlal osla swego, i wzial dwóch slug swoich z soba, i Izaaka syna swego, a narabawszy drew na ofiare palona, wstal i szedl na miejsce, o którem mu Bóg powiedzial.
\par 4 A dnia trzeciego, podnióslszy Abraham oczy swe, ujrzal ono miejsce z daleka.
\par 5 I rzekl Abraham do slug swoich: Zostancie wy tu z oslem, a ja z dziecieciem pójdziemy az do onad, a odprawiwszy modlitwy, wrócimy sie do was.
\par 6 Wzial tedy Abraham drwa na ofiare palona, i wlozyl je na Izaaka, syna swego, a sam wzial w reke swoje ogien i miecz, i szli obaj pospolu.
\par 7 I rzekl Izaak do Abrahama, ojca swego, mówiac: Ojcze mój! A on odpowiedzial: Owom ja, synu mój. I rzekl Izaak: Oto ogien i drwa, a gdziez baranek na ofiare palona?
\par 8 Odpowiedzial Abraham: Bóg sobie obmysli baranka na ofiare palona, synu mój; i szli obaj pospolu.
\par 9 A gdy przyszli na miejsce, o którem mu Bóg powiedzial, zbudowal tam Abraham oltarz, i ulozyl drwa, a zwiazawszy Izaaka, syna swego, wlozyl go na oltarz na drwa.
\par 10 I wyciagnal Abraham reke swoje, i wzial miecz, aby zabil syna swego.
\par 11 Lecz zawolal nan Aniol Panski z nieba, i rzekl: Abrahamie! Abrahamie! A on rzekl: Owom ja.
\par 12 I rzekl Aniol: Nie wyciagaj reki twej na dziecie, i nie czyn mu nic; bom teraz doznal, iz sie ty boisz Boga, i nie sfolgowales synowi twemu, jedynemu twemu, dla mnie.
\par 13 A podnióslszy Abraham oczy swe, ujrzal, a oto baran za nim uwiazl w cierniu za rogi swoje; a szedlszy Abraham, wzial barana i ofiarowal go na ofiare palona, miasto syna swego.
\par 14 I nazwal Abraham imie miejsca onego: Pan obmysli; stadze po dzis dzien mówia: Na górze Panskiej bedzie obmyslono.
\par 15 Tedy zawolal Aniol Panski na Abrahama po wtóre z nieba mówiac:
\par 16 Przez siebie samego przysiaglem, mówi Pan: Poniewazes to uczynil, a nie sfolgowales synowi twemu, jedynemu twemu;
\par 17 Blogoslawiac, blogoslawic ci bede, a rozmnazajac rozmnoze nasienie twoje, jako gwiazdy niebieskie, i jako piasek, który jest na brzegu morskim; a odziedziczy nasienie twoje bramy nieprzyjaciól twoich.
\par 18 I blogoslawione beda w nasieniu twojem wszystkie narody ziemi, dla tego, zes usluchal glosu mego.
\par 19 Wrócil sie tedy Abraham do slug swych, i wstawszy, przyszli pospolu do Beerseba; bo mieszkal Abraham w Beerseba.
\par 20 I stalo sie potem, iz oznajmiono Abrahamowi, mówiac: Oto narodzila i Melcha synów Nachorowi, bratu twemu.
\par 21 Husa, pierworodnego swego, i Buza, brata jego, i Chemuela, ojca Aramczyków.
\par 22 I Kaseda, i Kasana, i Feldasa, i Jedlafa, i Batuela.
\par 23 A Batuel splodzil Rebeke; osmioro tych dzieci urodzila Melcha Nachorowi, bratu Abrahamowemu.
\par 24 A zaloznica jego, której imie Reuma, urodzila tez Tabe, i Gahama, i Tahasa, i Maacha.

\chapter{23}

\par 1 A zyla Sara sto lat, i dwadziescia lat, i siedem lat; te sa lata zywota Sary.
\par 2 I umarla Sara w miescie Arba, które zowia Hebron, w ziemi Chananejskiej: i przyszedl Abraham, aby zalowal Sary, i plakal jej.
\par 3 Potem wstal Abraham od umarlego swego, i rzekl do synów Hetowych, mówiac:
\par 4 Gosciem i przychodniem jestem u was; dajciez mi osiadlosc grobu miedzy wami, abym pogrzebal umarlego mego od twarzy mojej.
\par 5 Tedy opowiedzieli synowie Hetowi Abrahamowi, mówiac mu:
\par 6 Sluchaj nas, panie mój: Ksiazeciem Bozym jestes ty w posrodku nas: w najprzedniejszych grobach naszych pogrzeb umarlego twego; zaden z nas nie bedzie bronil grobu swego tobie, abys nie mial pogrzebac umarlego twego.
\par 7 Tedy wstawszy Abraham, poklonil sie ludowi onej ziemi, to jest synom Hetowym, i rzekl do nich, mówiac:
\par 8 Jezli sie wam podoba, abym pogrzebal umarlego mego od twarzy mojej, sluchajciez mie, a przyczyncie sie za mna, do Efrona, syna Socharowego,
\par 9 Aby mi ustapil jaskini swojej Machpela, która ma na koncu pola swego, za sluszne pieniadze; niech mi ja spusci przed wami w osiadlosc grobu.
\par 10 (A Efron siedzial w posrodku synów Hetowych.) Tedy odpowiedzial Efron Hetejczyk Abrahamowi, w przytomnosci synów Hetowych, przed wszystkimi, którzy chodzili w brame miasta jego, mówiac:
\par 11 Nie tak, panie mój, ale sluchaj mie: Pole to dam tobie i jaskinia, która jest w niem, dawam ja tobie; przed oczyma synów ludu mego, dawam ja tobie, pogrzebze umarlego twego.
\par 12 Tedy sie poklonil Abraham przed ludem onej ziemi;
\par 13 I rzekl do Efrona, w przytomnosci ludu onej ziemi, mówiac: Raczej, jezlic sie zda, prosze, sluchaj mie: dam ci pieniadze za pole, wezmijze je ode mnie, a pogrzebie tam umarlego mego.
\par 14 I odpowiedzial Efron Abrahamowi mówiac mu:
\par 15 Panie mój, sluchaj mie. Ziemia ta stoi za cztery sta syklów srebra; ale cóz to jest miedzy mna i miedzy toba? pogrzeb umarlego twego.
\par 16 I usluchal Abraham Efrona; i odwazyl Abraham Efronowi srebro, jako byl rzekl w przytomnosci synów Hetowych, cztery sta syklów srebra, tak jako szly miedzy kupcami.
\par 17 I dostalo sie pole Efronowe (które jest w Machpelu przeciwko Mamre, pole i jaskinia, która jest na niem, i wszystkie drzewa, które byly na polu, które byly na wszystkich granicach jego w okolo).
\par 18 Abrahamowi w osiadlosc przed oczyma synów Hetowych, i wszystkich, którzy wchodzili w brame miasta onego.
\par 19 A tak pogrzebal Abraham Sare, zone swoje, w jaskini pola w Machpelu przeciwko Mamre, to jest Hebron, w ziemi Chananejskiej.
\par 20 I oddane jest pole i jaskinia, która byla na niem, Abrahamowi w osiadlosc grobu, od synów Hetowych.

\chapter{24}

\par 1 A Abraham byl stary i podeszly w leciech, a Pan blogoslawil mu we wszystkiem.
\par 2 Tedy rzekl Abraham do starszego slugi swego w domu swym, który wszystkiem rzadzil, co mial: Polóz, prosze, reke twoje pod biodro moje;
\par 3 A zaprzysiegne cie przez Pana, Boga nieba, i Boga ziemi, abys nie bral zony synowi memu z córek Chananejskich, miedzy któremi ja mieszkam;
\par 4 Ale pójdziesz do ziemi mojej, i do rodziny mojej, a stamtad wezmiesz zone Izaakowi, synowi memu.
\par 5 Tedy mu rzekl sluga: A jezliby snac nie chciala niewiasta ona isc ze mna do tej ziemi, mamze odprowadzic syna twego do ziemi, z którejs ty wyszedl?
\par 6 I rzekl mu Abraham: Strzez sie, abys tam zasie nie zaprowadzal syna mego.
\par 7 Pan Bóg niebieski, który mie wzial z domu ojca mego, i z ziemi rodziny mojej, i który mówil ze mna, a który mi przysiagl, mówiac: Nasieniu twemu dam ziemie te; on posle Aniola swego przed obliczem twojem, i wezmiesz stamtad zone synowi memu.
\par 8 A jezliby nie chciala ona niewiasta isc z toba, wolny bedziesz od tego poprzysiezenia mego; tylko syna mego nie zaprowadzaj tam.
\par 9 Podlozyl tedy sluga reke swoje pod biodro Abrahama, pana swego, i przysiagl mu na to.
\par 10 I wzial on sluga dziesiec wielbladów, z wielbladów pana swego, i poszedl; bo wszystkie dobra pana swego mial w rekach swych; a wstawszy puscil sie do Aram Naharaim, do miasta Nachorowego.
\par 11 I postawil wielblady przed miastem u studni wody, pod wieczór, tego czasu, którego zwykly niewiasty wychodzic czerpac wode.
\par 12 I rzekl: Panie, Boze pana mego Abrahama! Niech mie prosze spotka dzis, czego zadam, a uczyn milosierdzie z panem moim Abrahamem.
\par 13 Oto, ja stoje u studni, a córki obywateli miasta tego wyjda czerpac wode;
\par 14 Panienka tedy, do której bym rzekl: Nachyl prosze wiadra twego, ze sie napije, a ona by rzekla: Pij, owszem i wielblady twoje napoje; ta niech bedzie, któras zgotowal sludze twemu Izaakowi; a po tem poznam, zes uczynil milosierdzie z panem moim.
\par 15 I stalo sie, ze pierwej niz przestal mówic, oto, Rebeka wychodzila, która sie urodzila Batuelowi, synowi Melchy, zony Nachora, brata Abrahamowego, niosac wiadro na ramieniu swem.
\par 16 A dzieweczka ona byla bardzo piekna na wejrzeniu, panna, a której maz nie uznal; ta przyszedlszy do studni, napelnila wiadro swe, i wracala sie.
\par 17 Tedy zabiezal jej on sluga, i rzekl: Daj mi sie prosze napic troche wody z wiadra twego.
\par 18 A ona rzekla: Pij, panie mój, i predko zlozyla wiadro swe na reke swoje, i dala mu pic.
\par 19 A gdy mu sie dala napic, rzekla: I wielbladom twoim naczerpie, az sie napija.
\par 20 I wylala predko wode z wiadra swego w koryto, a biezawszy jeszcze do studni czerpac, naczerpala wszystkim wielbladom jego.
\par 21 A on maz zdumiewal sie nad nia, uwazajac z milczeniem, jezli mu Pan zdarzyl droge jego, czyli nie.
\par 22 I gdy sie napily wielblady, wyjal on maz nausznice zlota, która wazyla pól sykla, i dwie manele, i dal na rece jej, które wazyly dziesiec syklów zlota.
\par 23 I rzekl: Czyjas ty córka, powiedz mi, prosze? a jezli w domu ojca twego miejsce dla nas, gdzie bysmy przenocowali?
\par 24 A ona mu rzekla: Jestem córka Batuela, syna Melchy, którego urodzila Nachorowi.
\par 25 Nad to rzekla mu: Jest u nas dosyc plew i pastwy, i miejsce do przenocowania.
\par 26 I poklonil sie on czlowiek i dal chwale Panu,
\par 27 I rzekl: Blogoslawiony Pan, Bóg pana mego Abrahama, który nie oddalil milosierdzia swego i prawdy swojej od pana mojego, albowiem gdym byl w drodze, przyprowadzil mie Pan w dom braci pana mego.
\par 28 Biezala tedy dzieweczka, i oznajmila w domu matki swej, jako sie co stalo.
\par 29 I miala Rebeka brata imieniem Labana; i wybiezal Laban przeciwko onemu mezowi az ku studni.
\par 30 Bo ujrzawszy nausznice, i manele na reku siostry swej, i uslyszawszy slowa Rebeki, siostry swej, mówiacej: Tak mówil do mnie ten maz; przyszedl do onego meza, a oto, on stal przy wielbladach u studni.
\par 31 I rzekl do niego: Wnijdz blogoslawiony Panski; przecz bys stal na dworze, juzem ja nagotowal dom, i miejsce wielbladom?
\par 32 Tedy wszedl maz on w dom; a Laban rozsiodlal wielblady, i dal plew i pastwy wielbladom, i wody dla umycia nóg jego, i nóg mezów onych, którzy z nim byli.
\par 33 I polozyl przeden, coby jadl; ale on rzekl: Nie bede jadl, az pierwej odprawie rzecz swoje. Tedy rzekl Laban: Mówze.
\par 34 I rzekl: Jam jest sluga Abrahamów;
\par 35 A Pan ublogoslawil pana mego bardzo, i stal sie moznym; bo mu nadal owiec, i wolów, i srebra, i zlota, i slug, i sluzebnic, i wielbladów, i oslów.
\par 36 A urodzila Sara, zona pana mego syna panu memu, w starosci jego, któremu dal wszystko, co ma.
\par 37 I poprzysiagl mie pan mój, mówiac: Nie wezmiesz zony synowi memu z córek Chananejskich, w których ziemi ja mieszkam;
\par 38 Ale do domu ojca mego pójdziesz i do rodziny mojej; a wezmiesz stamtad zone synowi mojemu.
\par 39 I rzeklem do pana mego: Nie pójdzie snac ta niewiasta ze mna.
\par 40 Tedy mi odpowiedzial: Pan, przed któregom ja obliczem chodzil, posle Aniola swego z toba, i poszczesci droge twoje; a wezmiesz zone synowi memu z rodziny mojej, i z domu ojca mego.
\par 41 Tedy wolen bedziesz od poprzysiezenia mego, gdy przyjdziesz do rodziny mojej; ale jezlicby jej nie dano, wolen bedziesz od poprzysiezenia mego.
\par 42 Przyszedlem tedy dzis do studni, i rzeklem Panie, Boze pana mego Abrahama, jezliz ty teraz szczescisz droge moje, które ja ide:
\par 43 Oto, ja stoje u studni wody; niechajze panienka, która wynijdzie czerpac wode, a gdybym jej rzekl: Daj mi prosze napic sie troche wody z wiadra twego;
\par 44 A ona by rzekla do mnie: I ty pij, naczerpie tez i wielbladom twoim: ta bedzie zona, która zgotowal Pan synowi pana mego.
\par 45 Nizelim ja tedy przestal mówic w sercu swem, oto, Rebeka wychodzila, niosac wiadro swe na ramieniu swem, i przyszla do studni, a czerpala; którejm rzekl: Daj mi pic prosze.
\par 46 Ona tedy predko zlozywszy wiadro z siebie, rzekla: Pij, owszem i wielblady twoje napoje. I pilem; napoila tez i wielblady.
\par 47 I pytalem jej, mówiac: Czyjas ty córka? i odpowiedziala: Jestem córka Batuela, syna Nachorowego, którego mu urodzila Melcha, tedym wlozyl nausznice na twarz jej, i manele na rece jej.
\par 48 Zatem pokloniwszy sie, dalem chwale Panu, i blogoslawilem Panu, Bogu pana mego Abrahama, który mie prowadzil droga prawa, abym wzial córke brata pana mego, synowi jego
\par 49 Przetoz teraz, jezli chcecie uczynic milosierdzie i prawde z panem moim, oznajmijcie mi: a jezli nie, powiedzcie mi tez, zebym sie obrócil na prawo albo na lewo.
\par 50 Tedy odpowiedzial Laban i Batuel, mówiac: Od Pana ta rzecz wyszla; my tobie w niczem przeczyc nie mozemy.
\par 51 Oto Rebeka przed toba; wezmij ja, a idz; a niech bedzie zona syna pana twego, jako rzekl Pan.
\par 52 I stalo sie, gdy uslyszal sluga Abrahamów slowa ich poklonil sie az do ziemi Panu.
\par 53 Zatem wyjal sluga on naczynia srebrne, i naczynia zlote, i szaty, a oddal je Rebece; dal tez upominki drogie bratu jej, i matce jej.
\par 54 Jedli tedy i pili, on i mezowie, którzy z nim byli, i zostali tam na noc; a rano wstawszy, rzekl: Pusccie mie do pana mego.
\par 55 I rzekl brat jej, i matka jej: Niechaj pomieszka z nami dzieweczka dzien, albo dziesiec; potem pójdziesz.
\par 56 A on rzekl do nich: Nie zatrzymywajcie mie, gdyz Pan poszczescil droge moje, pusccie mie, abym jechal do pana mego.
\par 57 Zatem rzekli: Zawolajmy dzieweczki, a spytajmy, co na to rzecze.
\par 58 Tedy zawolali Rebeki, i mówili do niej: Chceszze jechac z tym czlowiekiem? A ona odpowiedziala: Pojade.
\par 59 I puscili Rebeke siostre swoje, z mamka jej, i sluge Abrahamowego, z mezami jego.
\par 60 Tedy blogoslawili Rebece, mówiac jej: Siostras nasza, rozmnóz sie w tysiac tysiecy, a niech posiadzie nasienie twoje bramy nieprzyjaciól swych.
\par 61 Tedy wstawszy Rebeka z dzieweczkami swemi, i wsiadlszy na wielblady jechaly za onym mezem; i wzial sluga on Rebeke, i odjechal.
\par 62 A Izaak wracal sie z przechadzki od studni, która zowia Zywiacego i Widzacego mie; bo mieszkal w ziemi poludniowej.
\par 63 A wyszedl byl Izaak, dla modlitwy na pole pod wieczór, i podnióslszy oczy swe, ujrzal wielblady przychodzace.
\par 64 Podniosla tez i Rebeka oczy swe, i ujrzala Izaaka, i zsiadla z wielblada;
\par 65 Bo rzekla do slugi: Cóz on za maz, który idzie przez pole przeciwko nam? I odpowiedzial sluga: Ten jest pan mój. A ona wziawszy rantuch nakryla sie.
\par 66 I powiedzial on sluga Izaakowi wszystko, co sprawil.
\par 67 I wprowadzil ja Izaak do namiotu Sary, matki swojej; i wzial Rebeke, i byla mu zona, i milowal ja. I ucieszyl sie Izaak po smierci matki swojej.

\chapter{25}

\par 1 Potem Abraham pojal druga zone, której imie bylo Ketura.
\par 2 Która mu urodzila Zamrama, i Joksana, i Madana, i Midyjana, i Jesobaka, i Suacha.
\par 3 A Joksan splodzil Sabe, i Dedana; a synowie Dedanowi byli Asurymowie i Letusymowie, i Leumymowie.
\par 4 Synowie zas Midyjanowi byli Hefa, i Hefer, i Henoch, i Abyda, i Eldaa; wszyscy ci byli synowie Ketury.
\par 5 I dal Abraham wszystko, co mial, Izaakowi.
\par 6 A synom zaloznic, które mial Abraham, dal upominki; i wyprawil je od Izaaka syna swego, jeszcze za zywota swego, ku wschodowi do krainy wschodniej.
\par 7 Tec sa dni lat zywota Abrahamowego, które przezyl, sto i siedemdziesiat, i piec lat.
\par 8 I ustawajac umarl Abraham w starosci dobrej, zeszly w leciech, i syty dni; i przylaczon jest do ludu swego.
\par 9 I pogrzebli go Izaak i Ismael, synowie jego, w jaskini Machpela, na polu Efrona, syna Socharowego, Hetejczyka, które bylo przeciwko Mamre;
\par 10 Na polu, które byl kupil Abraham u synów Hetowych; tam pogrzebiony jest Abraham, i Sara, zona jego.
\par 11 A po smierci Abrahamowej blogoslawil Bóg Izaakowi, synowi jego, a Izaak mieszkal u studni Zywiacego i Widzacego mie.
\par 12 A tec sa rodzaje Ismaela, syna Abrahamowego, którego urodzila Hagar, Egipczanka, sluzebnica Sary, Abrahamowi.
\par 13 I te sa imiona synów Ismaelowych w nazwiskach ich, wedlug rodzajów ich: pierworodny Ismaelów, Nebajot; po nim Kedar, i Abdeel, i Mabsan.
\par 14 I Masma, i Duma, i Masa.
\par 15 Hadar, i Tema, Jetur, Nafis i Kedma.
\par 16 Ci sa synowie Ismaelowi, i te imiona ich, wedlug miasteczek ich, i zamków ich, dwanascie ksiazat w familijach ich.
\par 17 A bylo lat zywota Ismaelowego, sto lat, i trzydziesci lat i siedem lat, i zszedl a umarl, i przylaczon jest do ludu swego.
\par 18 I mieszkali od Hewila az do Sur, która lezy na przeciwko Egiptowi, idac do Asyryi; przed obliczem wszystkich braci swych umarl.
\par 19 Te zas sa rodzaje Izaaka syna Abrahamowego: Abraham splodzil Izaaka.
\par 20 A Izaak mial czterdziesci lat, gdy sobie pojal Rebeke, córke Batuela Syryjczyka, z krainy Syryjskiej, siostre Labana, Syryjczyka, za zone.
\par 21 Tedy sie modlil Izaak Panu za zone swa, iz byla nieplodna; i wysluchal go Pan, i poczela Rebeka, zona jego.
\par 22 A gdy sie dziatki tracaly w zywocie jej, rzekla: Jezliz tak mialo byc, dlaczegozem poczela? Szla tedy, aby sie pytala Pana.
\par 23 I rzekl jej Pan: dwa narody sa w zywocie twoim, i dwojaki lud z zywota twego rozdzieli sie, a jeden lud nad drugi lud mozniejszy bedzie, i wiekszy bedzie sluzyl mniejszemu.
\par 24 A gdy sie wypelnily dni jej, aby porodzila, oto bliznieta byly w zywocie jej.
\par 25 I wyszedl pierwszy syn lisowaty, i wszystek jako szata kosmaty; i nazwali imie jego Ezaw.
\par 26 A potem wyszedl brat jego, reka swa trzymajac za piete, Ezawa i nazwano imie jego Jakób; a Izaakowi bylo szescdziesiat lat, gdy mu sie oni narodzili.
\par 27 A gdy urosly one dzieci, Ezaw byl mezem w myslistwie bieglym i rolnikiem, a Jakób byl maz prosty mieszkajacy w namieciech.
\par 28 I milowal Izaak Ezawa, iz jadal z lowu jego; Rebeka zas milowala Jakóba.
\par 29 I uwarzyl sobie Jakób potrawe, a na ten czas przyszedl Ezaw z pola spracowany.
\par 30 Tedy rzekl Ezaw do Jakóba: Daj mi jesc, prosze cie z tej czerwonej potrawy, bom sie spracowal: a przetoz nazwano imie jego Edom.
\par 31 Któremu rzekl Jakób: Przedajze mi dzis pierworodztwo twoje.
\par 32 I rzekl Ezaw: Otom ja bliski smierci, cóz mi po pierworodztwie?.
\par 33 I rzekl Jakób: Przysiazze mi dzis, i przysiagl mu. I sprzedal pierworodztwo swoje Jakóbowi.
\par 34 Tedy Jakób dal Ezawowi chleba, i potrawe z soczewicy, a on jadl i pil, a potem powstawszy odszedl; i pogardzil Ezaw pierworodztwem swojem.

\chapter{26}

\par 1 Potem byl glód na ziemi, po onym glodzie pierwszym, który byl za dni Abrahamowych; I poszedl Izaak do Abimelecha, do króla Filistynskiego do Gerar.
\par 2 Bo mu sie byl ukazal Pan i mówil: Nie zstepuj do Egiptu, ale mieszkaj w ziemi, o której Ja powiem tobie.
\par 3 Badzze gosciem w tej ziemi, a Ja bede z toba, i bedec blogoslawil; albowiem tobie i nasieniu twemu dam te wszystkie krainy, i utwierdze przysiege, któram przysiagl Abrahamowi, ojcu twemu.
\par 4 I rozmnoze nasienie twoje jako gwiazdy niebieskie, a dam nasieniu twemu wszystkie te krainy: A beda blogoslawione w nasieniu twojem wszystkie narody ziemi.
\par 5 Przeto, iz Abraham byl posluszny glosowi mojemu, a strzegl postanowienia mego, przykazan moich, ustaw moich i praw moich.
\par 6 Tedy Izaak mieszkal w Gerar.
\par 7 I pytali sie mezowie onego miejsca o zonie jego; a on powiedzial: Siostra to moja; bo sie bal mówic: Zona to moja; by go snac nie zabili mezowie miejsca tego dla Rebeki, iz byla piekna na wejrzeniu.
\par 8 I stalo sie, gdy tam mieszkal przez nie malo dni, ze wygladal Abimelech, król Filistynski, oknem, i ujrzal, ze Izaak zartowal z Rebeka, zona swa.
\par 9 Tedy przyzwal Abimelech do siebie Izaaka, i rzekl: Prawdziwiec to zona twoja; czemuzes powiadal, siostra to moja? I odpowiedzial mu Izaak: Izem u siebie mówil: Bym snac nie umarl dla niej.
\par 10 I rzekl Abimelech: Cózes nam to uczynil? bez mala ktokolwiek z ludu nie spal z zona twoja: i przywiódlbys byl na nas grzech.
\par 11 Rozkazal tedy Abimelech wszystkiemu ludowi mówiac: Kto by sie dotknal meza tego, albo zony jego, smiercia umrze.
\par 12 Tedy sial Izaak w onej ziemi, i zebral roku onego sto korcy, albowiem blogoslawil mu Pan.
\par 13 I zbogacil sie on maz, a im dalej, tem wiecej wzmagal sie, az urósl wielce.
\par 14 I mial stada owiec, i stada wolów, i czeladzi dosyc; przetoz mu zajrzeli Filistynczycy.
\par 15 I wszystkie studnie, które byli wykopali sludzy ojca jego, za dni Abrahama, ojca jego, zasypali Filistynczycy, i napelnili je ziemia.
\par 16 I rzekl Abimelech do Izaaka: Odejdz od nas, albowiemes daleko mozniejszy niz my.
\par 17 I odszedl stamtad Izaak, i rozbil namioty w dolinie Gerar, i mieszkal tam.
\par 18 I kopal zasie Izaak studnie wód, które byli wykopali za dni Abrahama, ojca jego, co je byli zasypali Filistyni po smierci Abrahamowej, i zwal je temiz imiony, któremi je byl nazwal ojciec jego.
\par 19 Tedy kopali sludzy Izaakowi w onej dolinie, i znalezli tam studnia wód zywych.
\par 20 Lecz poswarzyli sie pasterze Gerarscy z pasterzami Izaakowymi, mówiac: Nasza to woda; przeto nazwal imie studni onej, Hesek, iz sie swarzyli z nim o nie.
\par 21 Potem wykopali druga studnia, i swarzyli sie tez o nie; dla tegoz nazwal imie jej Sydna.
\par 22 Zatem przeniósl sie stamtad, i wykopal druga studnia o która zadnego sporu nie bylo; i nazwal imie jej Rechobot, i mówil: Oto, teraz rozszerzyl nas Pan, i uroslismy na ziemi.
\par 23 I wstapil stamtad do Beerseby.
\par 24 I ukazal mu sie Pan onejze nocy, mówiac: Jam jest Bóg Abrahama, ojca twego, nie bój sie, bom Ja jest z toba; i bedec blogoslawil, i rozmnoze nasienie twoje, dla Abrahama, slugi mego.
\par 25 Tedy tam zbudowal oltarz, i wzywal imienia Panskiego, i rozbil tam namiot swój, tamze tez wykopali sludzy Izaakowi studnie.
\par 26 Abimelech potem przyjechal do niego z Gerar, i Achuzat przyjaciel jego, i Fikol, hetman wojska jego.
\par 27 Do których rzekl Izaak: Przeczzescie przyjechali do mnie, gdyz wy mnie macie w nienawisci, i wypedziliscie mie od siebie?
\par 28 A oni odpowiedzieli: Obaczylismy to dobrze, ze Pan jest z toba, i rzeklismy: Uczynmy teraz przysiege miedzy soba, miedzy nami i miedzy toba, i postanowimy przymierze z toba;
\par 29 Abys nam nic zlego nie czynil, jakosmy sie tez ciebie nie tykali; i jakosmyc tylko dobrze czynili, a puscilismy cie w pokoju; a tys teraz blogoslawiony od Pana.
\par 30 Tedy im sprawil uczte, a jedli i pili.
\par 31 Potem wstawszy bardzo rano przysiegli jeden drugiemu; i wyprowadzil je Izaak, i odeszli od niego w pokoju.
\par 32 I stalo sie onegoz dnia, przyszli sludzy Izaakowi, i powiedzieli mu o studni, która wykopali, mówiac mu: Znalezlismy wode.
\par 33 I nazwal ja Syba; dlategoz imie miasta onego jest nazwane Beerseba az do dnia dzisiejszego.
\par 34 Potem Ezaw majac czterdziesci lat, pojal sobie za zone Judyte, córke Beery Hetejczyka, i Basemat, córke Elona, Hetejczyka.
\par 35 Które sie bardzo naprzykrzaly Izaakowi i Rebece.

\chapter{27}

\par 1 I stalo sie, gdy sie zestarzal Izaak, i zacmily sie oczy jego, tak, iz widziec nie mógl: tedy wezwal Ezawa, syna swego starszego, i rzekl mu: Synu mój! a on odpowiedzial: Owom ja.
\par 2 I rzekl Izaak: Otom sie juz zestarzal, a nie wiem dnia smierci swej.
\par 3 Przetoz teraz wezmij prosze naczynia twoje, sajdak twój, i luk twój, a wynijdz w pole, i ulów mi zwierzyne.
\par 4 I nagotuj mi potrawy smaczne, w jakich sie kocham, i przynies mi, a bede jadl, abyc blogoslawila dusza moja, pierwej, niz umre.
\par 5 Ale Rebeka slyszala, gdy to mówil Izaak do Ezawa, syna swego; tedy wyszedl Ezaw na pole, aby ulowil zwierzyne i przyniósl.
\par 6 I rzekla Rebeka do Jakóba, syna swego, mówiac: Otom slyszala, gdy ojciec twój mówil do Ezawa, brata twego, i rzekl:
\par 7 Przynies mi co z oblowu, a nagotuj mi potrawy smaczne, abym jadl, i blogoslawil ci przed obliczem Panskiem, pierwej, niz umre.
\par 8 A tak teraz synu mój, usluchaj glosu mego w tem, co ja rozkazuje tobie.
\par 9 A szedlszy do trzody, przynies mi stamtad dwoje kozlat dobrych, a nagotuje z nich potrawy smaczne ojcu twemu, jako rad jada.
\par 10 I zaniesiesz ojcu twemu, a bedzie jadl; dlatego abyc blogoslawil, pierwej niz umrze.
\par 11 Tedy rzekl Jakób do Rebeki, matki swej: Oto, Ezaw brat mój, czlowiek kosmaty, a jam czlowiek gladki;
\par 12 Jezli mie pomaca ojciec mój, a bedzie rozumial, ze z niego szydze, przywiode na sie przeklestwo, a nie blogoslawienstwo.
\par 13 I rzekla mu matka jego: Na mie niech bedzie przeklestwo twoje, synu mój; tylko usluchaj glosu mego, a szedlszy, przynies mi.
\par 14 Tedy on szedlszy wzial, i przyniósl matce swej; i nagotowala matka jego potrawy smaczne, jako rad jadal ojciec jego.
\par 15 I wziawszy Rebeka szaty Ezawa, syna swego starszego, najkosztowniejsze, które miala u siebie w domu, oblokla w nie Jakóba, syna swego mlodszego.
\par 16 A skórkami kozlecemi obwinela rece jego, i gladkosc szyi jego.
\par 17 I dala chleb i potrawy smaczne, które nagotowala, w rece Jakóba syna swego.
\par 18 A on wszedlszy do ojca swego mówil: Ojcze mój! a on rzekl: Owom ja! Ktos ty jest, synu mój?
\par 19 I rzekl Jakób do ojca swego: Jam jest Ezaw, pierworodny twój. Uczynilem, jakos mi rozkazal; wstan prosze, siadz, a jedz z oblowu mego, aby mi blogoslawila dusza twoja.
\par 20 I rzekl Izaak do syna swego: Cóz to jest? Predkos to znalazl, synu mój? a on odpowiedzial: Sposobil to Pan Bóg twój, ze mi sie nagodzilo.
\par 21 Zatem Izaak rzekl do Jakóba: Przystap sam, abym cie pomacal, synu mój, jezlis ty jest syn mój Ezaw, czyli nie.
\par 22 Tedy przystapil Jakób do Izaaka, ojca swego, który pomacawszy go, rzekl: Glos jest glos Jakóbów, ale rece Ezawowe.
\par 23 I nie poznal go; albowiem byly rece jego jako rece Ezawa, brata jego, kosmate; i blogoslawil mu.
\par 24 I rzekl: Tyzes jest syn mój Ezaw? a on odpowiedzial: Ja.
\par 25 Zatem rzekl: Podajze mi, zebym jadl z oblowu syna mego, abyc blogoslawila dusza moja. Tedy mu podal, i jadl. Przyniósl mu tez wina, i pil.
\par 26 I rzekl mu Izaak, ojciec jego: Przystapze teraz a pocaluj mie, synu mój.
\par 27 Tedy przystapiwszy pocalowal go, a skoro poczul wonnosc szat jego, blogoslawil mu, mówiac: Oto wonnosc syna mego, jako wonnosc pola, któremu blogoslawil Pan.
\par 28 Niechajzec da Bóg z rosy niebieskiej, i z tlustosci ziemskiej, i obfitosc zboza i wina.
\par 29 Niechaj ci sluza ludzie, a niech ci sie klaniaja narodowie. Badz panem braci twojej, a niech ci sie klaniaja synowie matki twojej; którzy by cie przeklinali, niech beda przekletymi, a którzy by cie blogoslawili, niech beda blogoslawionymi.
\par 30 I stalo sie, gdy przestal Izaak blogoslawic Jakóbowi, i ledwie Jakób odszedl od oblicza Izaaka, ojca swego, tedy Ezaw brat jego, przyszedl z lowu swego.
\par 31 Który nagotowawszy potrawy smaczne, przyniósl je ojcu swemu, i mówil do ojca swego: Wstanze ojcze mój, a jedz z oblowu syna twego, aby mi blogoslawila dusza twoja.
\par 32 Tedy mu rzekl Izaak, ojciec jego: Któzes ty? A on rzekl: Jam jest syn twój, pierworodny twój, Ezaw.
\par 33 I zlakl sie Izaak zleknieniem bardzo wielkiem, i rzekl: Któz to, a gdzie jest ten, co ulowil zwierzyne, i przyniósl mi? i jadlem ze wszystkiego, pierwej, nizes ty przyszedl, i blogoslawilem mu, i bedzie blogoslawionym.
\par 34 A uslyszawszy Ezaw slowa ojca swego, zawolal glosem wielkiem, i byl zaloscia wielka zjety, i rzekl ojcu swemu: Blogoslawze tez i mnie, ojcze mój.
\par 35 A on mu rzekl: Przyszedl brat twój chytrze, i wzial blogoslawienstwo twoje.
\par 36 Tedy Ezaw rzekl: Sluszniec nazwano imie jego Jakób, podszedl mie bowiem juz dwa kroc; pierworodztwo moje wzial, a teraz oto odniósl blogoslawienstwo moje. I rzekl: Tos mi nie zachowal blogoslawienstwa?
\par 37 Odpowiedzial Izaak, i rzekl do Ezawa: Otom go panem postanowil nad toba, i wszystkich braci jego dalem mu za slugi, i zbozem, i winem opatrzylem go; a tobie cóz teraz mam uczynic, synu mój?
\par 38 I rzekl Ezaw do ojca swego: Izali tylko jedno blogoslawienstwo masz, ojcze mój? Blogoslawze i mnie; i jamci syn twój, ojcze mój. I podniósl Ezaw glos swój, a plakal.
\par 39 I odpowiedzial Izaak, ojciec jego, i rzekl mu: Oto w tlustosci ziemi bedzie mieszkanie twoje, i w rosie niebieskiej z góry.
\par 40 A z miecza twego zyc bedziesz, i bratu twemu bedziesz sluzyl; ale stanie sie, ze i ty panowac bedziesz, i zrzucisz jarzmo jego z szyi twojej.
\par 41 Przetoz nienawidzil Ezaw Jakóba dla blogoslawienstwa, którem mu blogoslawil ojciec jego; i mówil Ezaw w sercu swem: Przyblizaja sie dni zaloby ojca mego, a zabije Jakóba, brata mego.
\par 42 I oznajmiono Rebece slowa Ezawa, syna jej starszego, która poslawszy, wezwala Jakóba, syna swego mlodszego, i rzekla do niego: Oto Ezaw, brat twój, cieszy sie tem, iz cie zabije.
\par 43 Przetoz teraz, synu mój, usluchaj glosu mego, a wstawszy, uciecz do Labana, brata mego, do Haranu,
\par 44 I pomieszkaj z nim przez jaki czas, az ucichnie gniew brata twego,
\par 45 Az sie odwróci zapalczywosc brata twego od ciebie, i zapomni tego, cos mu uczynil; potem ja posle, a wezme stamtad; bo czemuz mam was obydwóch postradac jednego dnia?
\par 46 I rzekla Rebeka do Izaaka: Obmierzl mi zywot mój dla córek Hetejskich; jezlize i Jakób wezmie sobie zone z córek Hetejskich, jakie sa córki ziemi tej, cóz mi po zywocie?

\chapter{28}

\par 1 Tedy wezwal Izaak Jakóba, i blogoslawil mu, a rozkazal mu, mówiac: Nie pojmuj zony z córek Chananejskich.
\par 2 Ale wstawszy idz do krainy Syryjskiej, do domu Batuela, ojca matki twojej, a wezmij sobie stamtad zone, z córek Labana, brata matki twojej.
\par 3 A Bóg Wszechmogacy niech ci blogoslawi, a niech cie rozrodzi i rozmnozy, abys byl w mnóstwo ludu;
\par 4 I niech ci da blogoslawienstwo Abrahamowe, tobie i nasieniu twemu z toba, abys odziedziczyl ziemie pielgrzymstwa twojego, która dal Bóg Abrahamowi.
\par 5 I tak wyslal Izaak Jakóba, który szedl do krainy Syryjskiej, do Labana, syna Batuelowego, Syryjczyka, brata Rebeki, matki Jakóbowej i Ezawowej.
\par 6 A widzac Ezaw, iz blogoslawil Izaak Jakóbowi, i poslal go do krainy Syryjskiej, aby sobie pojal stamtad zone, a iz blogoslawiac mu, przykazal, mówiac: Nie wezmiesz zony z córek Chananejskich;
\par 7 I Jakób posluszny byl ojcu swemu i matce swojej, i poszedl do krainy Syryjskiej;
\par 8 Widzac tez Ezaw, ze sie nie podobaja córki Chananejskie w oczach Izaaka, ojca jego:
\par 9 Tedy szedl Ezaw do Ismaela, i pojal mimo inne zony swoje, Mahalate córke Ismaela, syna Abrahamowego, siostre Nebajotowe, sobie za zone.
\par 10 A Jakób wyszedlszy z Beerseba, szedl do Haranu.
\par 11 I przyszedl na jedno miejsce, i nocowal tam (albowiem juz bylo zaszlo slonce) a wziawszy jeden z kamieni miejsca onego, podlozyl pod glowe swoje, i spal na temze miejscu.
\par 12 I snilo mu sie, a ono drabina stala na ziemi, a wierzch jej dosiegal nieba; a oto, Aniolowie Bozy wstepowali i zstepowali po niej.
\par 13 A Pan stal nad nia i rzekl; Jam jest Pan, Bóg Abrahama, ojca twego, i Bóg Izaaka, ziemie te, na której ty spisz, tobie dam i nasieniu twojemu.
\par 14 A bedzie nasienie twoje jako proch ziemi, i rozmnozysz sie na zachód, i na wschód, i na pólnocy, i na poludnie; a beda ublogoslawione w tobie wszystkie narody ziemi i w nasieniu twojem.
\par 15 A oto, Ja jestem z toba i strzec cie bede gdziekolwiek pójdziesz, i przywróce cie do tej ziemi; bo nie opuszcze cie, az uczynie com ci rzekl.
\par 16 Tedy gdy sie ocknal Jakób ze snu swego, rzekl: Zaprawde Pan jest na tem miejscu, a jam nie wiedzial.
\par 17 I zleknawszy sie, rzekl: O jako to straszne miejsce! nic tu nie jest innego jedno dom Bozy, a tu brama niebieska.
\par 18 I wstal Jakób bardzo rano, a wziawszy kamien, który byl podlozyl pod glowe swoje, postawil go na znak, i nalal oliwy na wierzch jego.
\par 19 I nazwal imie miejsca onego Betel; bo bylo przedtem imie miasta onego Luz.
\par 20 Tedy uczynil Jakób slub, mówiac: Jezliz bedzie Bóg ze mna, a strzec mie bedzie na tej drodze, która ja ide, i da mi chleb ku jedzeniu, i odzienie ku oblóczeniu,
\par 21 A wróce sie w pokoju do domu ojca mego: tedy bedzie mi Pan za Boga.
\par 22 A kamien ten, którym wystawil na znak, bedzie domem Bozym, a ze wszystkiego, co mi dasz, dziesiecine pewna oddawac ci bede.

\chapter{29}

\par 1 Tedy Jakób wstawszy, poszedl do ziemi, mieszkajacych na wschód slonca.
\par 2 I ujrzal studnia na polu, i trzy stada owiec lezacych przy niej; bo z onej studni napawano stada, a kamien wielki byl na wierzchu onej studni.
\par 3 Albowiem schodzily sie tam wszystkie stada, i odwalono kamien z wierzchu studni, a napawano stada; potem zas kladziono kamien na wierzch studni na miejsce jego.
\par 4 Tedy rzekl do nich Jakób: Bracia moi, skadescie? i odpowiedzieli: Z Haranu jestesmy.
\par 5 I rzekl do nich Jakób: A znacie Labana, syna Nachorowego? odpowiedzieli: Znamy.
\par 6 Zatem rzekl do nich: A dobrze sie ma? a oni odpowiedzieli: Dobrze; a oto, Rachel córka jego idzie z stadem.
\par 7 Tedy rzekl: Oto, jeszcze dosyc dnia, i nie czas zganiac stada; napójciez owce, a idzcie, popascie ich.
\par 8 A oni odpowiedzieli: Nie mozemy, azby sie zebraly wszystkie stada, i odwalony byl kamien z wierzchu studni, abysmy napoili stada.
\par 9 A gdy to jeszcze mówil z nimi, Rachel nadeszla z owcami ojca swego, bo je ona pasla.
\par 10 I gdy ujrzal Jakób Rachele, córke Labana, brata matki swojej, z owcami Labana, brata matki swej: tedy przystapil Jakób, i odwalil kamien z wierzchu studni, a napoil owce Labana, brata matki swojej.
\par 11 I pocalowal Jakób Rachele, i podnióslszy glos swój plakal.
\par 12 I oznajmil Jakób Racheli, ze jest bratem ojca jej, a iz jest synem Rebeki: a ona biezawszy opowiedziala to ojcu swemu.
\par 13 A gdy uslyszal Laban wiesc o Jakóbie, synu siostry swojej, wybiezal przeciwko niemu, i oblapil go, a pocalowawszy, wwiódl do domu swego. A on Labanowi powiedzial o wszystkiem.
\par 14 I rzekl mu Laban: Zaistes ty jest kosc moja, i cialo moje. I mieszkal u niego przez caly miesiac.
\par 15 Potem rzekl Laban do Jakóba: Izali, zes mi brat, sluzyc mi bedziesz darmo? powiedz mi, jaka ma byc zaplata twoja.
\par 16 A mial Laban dwie córki: imie starszej Lija, a imie mlodszej Rachel.
\par 17 Ale Lija byla chorych oczu, a Rachel zas pieknego oblicza, i wdzieczna na wejrzeniu.
\par 18 Milowal tedy Jakób Rachele, i rzekl: Bedec sluzyl siedem lat za Rachele, córke twoje mlodsza.
\par 19 Odpowiedzial Laban: Lepiej ze ja tobie dam, nizlibym ja mial dac mezowi innemu: mieszkajze ze mna.
\par 20 I sluzyl Jakób za Rachele siedem lat, i zdal mu sie ten czas jako kilka dni, przeto ze ja milowal.
\par 21 Potem rzekl Jakób do Labana: Daj mi zone moje, poniewaz sie wypelnily dni moje, abym wszedl do niej.
\par 22 Tedy wezwawszy Laban wszystkich mezów miejsca onego, sprawil uczte.
\par 23 A gdy byl wieczór, wzial Lije, córke swoje, i wwiódl ja do niego, a Jakób wszedl do niej.
\par 24 Dal tez Laban i Zelfe, dziewke swoje, Lii, córce swej, za sluzebnice.
\par 25 A gdy bylo rano, poznal Jakób, ze to Lija, i rzekl do Labana: Cózes mi to uczynil? Izalim ja nie za Rachele tobie sluzyl? czemuzes mie tedy oszukal?
\par 26 I odpowiedzial Laban: Nie jest to w zwyczaju u nas, aby miano wydawac za maz mlodsza przed starsza.
\par 27 Wytrwaj z ta tydzien, a dam ci i te za sluzbe, która mi bedziesz sluzyl jeszcze drugie siedem lat.
\par 28 I uczynil tak Jakób, i wypelnil z ta tydzien; potem dal mu Laban Rachele, córke swoje, za zone.
\par 29 Dal tez Laban Racheli, córce swej, Bale dziewke swoje; dal jej za sluzebnice.
\par 30 Tedy tez wszedl Jakób do Racheli, i milowal Rachele bardziej niz Lije, a sluzyl mu jeszcze drugie siedem lat.
\par 31 A widzac Pan, ze nienawidzil Lije, otworzyl zywot jej; a Rachel nieplodna byla.
\par 32 Tedy poczawszy Lija porodzila syna, i nazwala imie jego Ruben, bo rzekla: Zaiste wejrzal Pan na utrapienie moje; a tak teraz milowac mie bedzie maz mój.
\par 33 I zasie poczela, i porodzila syna, a rzekla: Zaiste uslyszal Pan, zem ja byla w nienawisci, przetoz dal mi tez i tego; i nazwala imie jego Symeon.
\par 34 Potem jeszcze poczela, i porodzila syna, i rzekla: I tym razem przylaczy sie maz mój do mnie, bom mu urodzila trzech synów; przetoz nazwala imie jego Lewi.
\par 35 Nad to jeszcze poczela, i porodzila syna, i rzekla: Teraz juz chwalic bede Pana; przetoz nazwala imie jego Juda, i przestala rodzic.

\chapter{30}

\par 1 A widzac Rachel, ze nie rodzila Jakóbowi, zajrzala Rachel siostrze swej, rzekla do Jakóba: Daj mi syny, a jezli nie dasz, umre.
\par 2 Zapalil sie tedy gniewem Jakób na Rachele, i rzekl: Zazem ja Bóg, który zawsciagnal plód zywota twego?
\par 3 A ona rzekla: Oto sluzebnica moja Bala; wnijdzze do niej, i porodzi na kolanach moich, a bede tez miala syny z niej.
\par 4 I dala mu Bale, sluzebnice swoje, za zone; i wszedl Jakób do niej.
\par 5 Tedy poczela Bala, i urodzila Jakóbowi syna.
\par 6 I rzekla Rachel: Skazal za mna Bóg, i uslyszal glos mój, a dal mi syna; i dlatego nazwala imie jego Dan.
\par 7 Potem zas poczawszy porodzila Bala, sluzebnica Racheli, drugiego syna Jakóbowi.
\par 8 Tedy rzekla Rachel: Mezniem sie biedzila z siostra moja, a przemoglam; i nazwala imie jego Neftali.
\par 9 A obaczywszy Lija, ze przestala rodzic, wziela tez Zelfe, sluzebnice swoje, i dala ja Jakóbowi za zone.
\par 10 I urodzila Zelfa, sluzebnica Lii, Jakóbowi syna.
\par 11 Zatem Lija rzekla: Przyszedl huf; i nazwala imie jego Gad.
\par 12 Porodzila tez Zelfa, sluzebnica Lii, drugiego syna Jakóbowi.
\par 13 I rzekla Lija: To na szczescie moje; bo mie szczesliwa beda zwaly niewiasty; i nazwala imie jego Aser.
\par 14 I wyszedl Ruben czasu zniwa pszenicznego, i znalazl pokrzyki na polu, a przyniósl je do Lii, matki swej; i rzekla Rachel do Lii: Daj mi tez prosze z pokrzyków syna twego.
\par 15 A ona jej odpowiedziala: A maloz na tem, zes mi wziela meza mego, iz tez chcesz wziac i pokrzyki syna mego? Tedy rzekla Rachel: Niechajze spi z toba tej nocy za pokrzyki syna twego.
\par 16 A gdy sie wracal Jakób z pola pod wieczór, wyszla Lija przeciwko jemu, i rzekla: Do mnie wnijdziesz, gdyzem cie pewna zaplata najela sobie pokrzykami syna mego; i spal z nia onej nocy.
\par 17 Tedy wysluchal Bóg Lije; i poczela, i porodzila Jakóbowi syna piatego.
\par 18 I rzekla Lija: Oddal mi Bóg zaplate moja, zem byla dala sluzebnice moje mezowi mojemu; i nazwala imie jego Isaszar.
\par 19 Potem poczawszy jeszcze Lija, porodzila szóstego syna Jakóbowi.
\par 20 I mówila Lija: Obdarzyl mnie Bóg zacnym upominkiem; juz teraz bedzie ze mna mieszkal maz mój, bom mu urodzila szesciu synów; i nazwala imie jego Zabulon.
\par 21 Potem porodzila córke, i nazwala imie jej Dyna.
\par 22 Wspomnial tez Bóg, na Rachele, i wysluchal ja Bóg, a otworzyl zywot jej.
\par 23 Tedy poczawszy porodzila syna, i rzekla: Odjal Bóg zelzywosc moje.
\par 24 I nazwala imie jego Józef, mówiac: Niech mi przyda Pan drugiego syna.
\par 25 I stalo sie, gdy porodzila Rachel Józefa, mówil Jakób do Labana: Pusc mie, abym sie wrócil do miejsca mego, i do ziemi mojej.
\par 26 Daj mi zony moje, i dzieci moje, za którem ci sluzyl, ze odejde; bo ty wiesz poslugi moje, jakom ci sluzyl.
\par 27 I rzekl do niego Laban: Prosze, jezlim znalazl laske w oczach twoich, zostan ze mna; bom doznal tego, ze mi Pan dla ciebie blogoslawil.
\par 28 I rzekl: Mianuj mi zaplate twoje, a dam ci ja.
\par 29 Tedy mu odpowiedzial Jakób: Ty wiesz, jakom ci sluzyl, i jaki byl dobytek twój przy mnie.
\par 30 Bo ta trocha, któras mial przede mna, rozmnozyla sie wielce; i blogoslawil ci Pan na przyjscie moje, a teraz kiedyz ja sie tez starac bede o dom swój?
\par 31 I rzekl: Cózci mam dac? I odpowiedzial Jakób: Nie dasz mi nic; ale jezli to uczynisz coc powiem, tedy sie wróce, a bede pasl i strzegl bydla twego.
\par 32 Przejde dzis przez wszystkie trzody twoje, odlaczajac stamtad kazde bydle pstre i nakrapiane, i kazde bydle plowe miedzy owcami, a nakrapiane i pstre miedzy kozami; to bedzie zaplata moja.
\par 33 I da swiadectwo o mnie sprawiedliwosc moja na potem, gdy przyjdzie do zaplaty mojej przed toba; wszystko co nie bedzie pstre i nakrapiane miedzy kozami, a plowe miedzy owcami, niech bedzie za kradziez poczytane przy mnie.
\par 34 Tedy rzekl Laban: Oby sie stalo wedlug slowa twego!
\par 35 I odlaczyl onegoz dnia kozly strokate, i nakrapiane, i wszystkie kozy pstre, i nakrapiane, i wszystkie, co jaka biala odmiane mialy, takze i plowe miedzy owcami, i oddal je do rak synów swych.
\par 36 I odlaczyl sie Laban od Jakóba, jakoby na trzy dni drogi; a Jakób pasl ostatek owiec Labanowych.
\par 37 Nabral tedy Jakób pretów zielonych topolowych, i laskowych, i kasztanowych, i oblupil miejscami skóre ich do bialego, obnazajac bialosc, która na pretach byla.
\par 38 I nakladl onych pretów, które byl oblupil, do rynien i do koryt, gdzie lano wody (gdy przychodzily owce, aby pily) nakladl ich przeciwko owcom, aby poczynaly, gdyby pic przychodzily.
\par 39 I poczynaly owce patrzac na one prety, i rodzily jagnieta strokate, pstre i nakrapiane.
\par 40 I odlaczyl Jakób jagnieta, a stawial owce twarza do jagniat strokatych, i do wszystkich plowych w stadzie Labanowem, a swoje stada stawial osobno, ani ich obracal ku stadu Labanowemu.
\par 41 A gdy wszystkich owiec co ranszych przypuszczanie bywalo, kladl Jakób prety przed oczy owiec w koryta, aby poczynaly patrzac na prety.
\par 42 Lecz gdy pózniejszych owiec przypuszczanie bylo, nie kladl ich: i byly pózniejsze Labanowe, a ransze Jakóbowe.
\par 43 I tak zbogacil sie on czlowiek bardzo, i mial owiec wiele, i sluzebnic i slug, i wielbladów, i oslów.

\chapter{31}

\par 1 Potem gdy uslyszal Jakób slowa synów Labanowych mówiacych: Pobral Jakób wszystko, co mial ojciec nasz, i z tego, co bylo ojca naszego, tej wszystkiej zacnosci dostal.
\par 2 Widzial tez Jakób twarz Labanowa, a oto, nie byl takim przeciwko niemu, jako przedtem.
\par 3 Tedy rzekl Pan do Jakóba: Wróc sie do ziemi ojców twoich, i do rodziny twojej, a bede z toba.
\par 4 Przetoz poslal Jakób, i wyzwal Rachele i Lije na pole do trzody swojej.
\par 5 I rzekl im: Widze ja twarz ojca waszego, ze nie jest takim przeciwko mnie, jako przedtem, lecz Bóg ojca mego byl ze mna.
\par 6 Wy tez same wiecie, zem ze wszystkich sil moich sluzyl ojcu waszemu;
\par 7 Ale ojciec wasz oszukal mie, i odmienil zaplate moje po dziesiec kroc; jednak nie dopuscil mu Bóg, aby mi szkodzil.
\par 8 Jezli kiedy powiedzial: Pstre beda zaplata twoja, tedy rodzily wszystkie owce jagnieta pstre; a gdy zas mówil: Strokate beda zaplata twoja, tedy rodzily wszystkie owce jagnieta strokate.
\par 9 I odjal Bóg dobytek ojca waszego, a dal go mnie.
\par 10 Stalo sie bowiem w ten czas, gdy sie owce zlaczaly, zem podniósl oczy swe, i widzialem przez sen, a oto, samcy zlaczaly sie z owcami strokatemi, pstremi, i bialo nakrapianemi.
\par 11 Tedy mi rzekl Aniol Bozy we snie: Jakóbie! A jam odpowiedzial: Owom ja.
\par 12 Potem rzekl: Podnies teraz oczy swe, a obacz wszystkie samce zlaczajace sie z owcami strokatemi, pstremi, i bialo nakrapianemi; bom widzial wszystko, coc Laban uczynil.
\par 13 Jam Bóg Betel, gdzies namazal kamien, gdzies mi poslubil slub. Teraz tedy wstan, wynijdz z ziemi tej, a wróc sie do ziemi rodziny twojej.
\par 14 Tedy odpowiedziala Rachel i Lija, i rzekly mu: Izaz jeszcze mamy czastke jaka i dziedzictwo w domu ojca naszego?
\par 15 Izazesmy za obce nie byly poczytane u niego? Iz nas przedal; i mialze by jeszcze do szczetu zjesc majetnosc nasze?
\par 16 Albowiem wszystko bogactwo, które odjal Bóg ojcu naszemu, nasze jest, i synów naszych; przetoz teraz wszystko uczyn, coc Bóg rozkazal.
\par 17 Wstal tedy Jakób, i wsadzil syny swe, i zony swe na wielblady.
\par 18 I zabral wszystke trzode swoje, i wszystke majetnosc swoje, której byl nabyl, dobytek nabycia swego, którego byl nabyl w krainie Syryjskiej, aby sie wrócil do Izaaka, ojca swego, do ziemi Chananejskiej.
\par 19 A Laban odszedl byl strzyc owce swoje: wtem ukradla Rachel balwany, które mial ojciec jej.
\par 20 I wykradl sie Jakób potajemnie od Labana Syryjczyka, tak ze mu nie oznajmil, iz uciekal.
\par 21 I uciekl sam ze wszystkiem co mial, a wstawszy przeprawil sie przez rzeke, i udal sie ku górze Galaad.
\par 22 I dano znac Labanowi dnia trzeciego, ze uciekl Jakób.
\par 23 Który wziawszy bracia swoje z soba, gonil go przez siedem dni, i doscignal go na górze Galaad.
\par 24 Lecz przyszedl Bóg do Labana Syryjczyka we snie onej nocy, i rzekl mu: Strzez sie, abys nie mówil z Jakóbem nic przykrego.
\par 25 I dogonil Laban Jakóba, a Jakób juz byl namiot swój rozbil na górze; Laban tez rozbil namiot z bracia swa na onejze górze Galaad.
\par 26 Tedy Laban rzekl do Jakóba: Cózes uczynil, zes sie wykradl potajemnie ode mnie, a uwiodles córki moje, jakoby pojmane mieczem?
\par 27 Przeczzes potajemnie uciekl, a wykradles sie ode mnie, a nie oznajmiles mi, gdyzbym cie byl puscil z radoscia, i z piesniami, i z bebnem, i z harfa?
\par 28 I nie dopusciles mi, abym pocalowal syny moje, i córki moje? Zaiste glupies sobie poczal.
\par 29 Jest to w mocy reki mojej, uczynic wam co zlego; ale Bóg ojca waszego przeszlej nocy rzekl do mnie, mówiac: Strzez sie abys z Jakóbem nie mówil nic przykrego.
\par 30 A teraz gdyc sie chcialo odejsc, zes wielce pragnal do domu ojca twego, czemuzes ukradl bogi moje?
\par 31 I odpowiadajac Jakób, rzekl do Labana: Izem sie bal, i myslalem, bys mi snac nie wydarl córek twoich.
\par 32 Lecz ten, u kogo znajdziesz bogi twoje, niech umrze; przed bracia nasza poznajze, co twego u mnie, i wezmij sobie; a nie wiedzial Jakób, ze je Rachel ukradla.
\par 33 Wszedl tedy Laban do namiotu Jakóbowego, i do namiotu Lii, i do namiotu obydwóch sluzebnic, a nie znalazl; a wyszedlszy z namiotu Lii wszedl do namiotu Racheli.
\par 34 A Rachel wziawszy one balwany wlozyla je pod sidlo wielbladowe, i usiadla na nich; i zmacal Laban wszystek namiot, a nie znalazl.
\par 35 Tedy ona rzekla do ojca swego: Niech sie nie gniewa pan mój, ze nie moge powstac przed twarza twoja, bo wedlug zwyczaju niewiast przypadlo na mie; i szukal, a nie znalazl balwanów.
\par 36 Rozgniewal sie tedy Jakób, i fukal na Labana; a odpowiadajac Jakób, rzekl do Labana: Cóz za przestepstwo moje, co za grzech mój, zes mie gonil zapaliwszy sie?
\par 37 Otos zmacal wszystek sprzet mój: cózes znalazl ze wszystkiego sprzetu domu twego? polóz tu przed bracia moja, i bracia twoja, a niech rozsadza miedzy nami dwoma.
\par 38 Juz dwadziescia lat mieszkalem z toba; owce twoje i kozy twoje nie pomiataly, a baranów stada twego nie jadalem.
\par 39 Rozszarpanego od zwierza nie przynioslem ci, jam szkode nagradzal; z reki mojej szukales tego, co bylo ukradzione we dnie, i co bylo ukradzione w nocy.
\par 40 Bywalo to, ze we dnie trapilo mie goraco, a mróz w nocy, tak, ze odchadzal sen mój od oczu moich.
\par 41 Juzem ci dwadziescia lat w domu twoim sluzyl; czternascie lat za dwie córki twoje, a szesc lat za bydlo twoje; a odmieniales zaplate moje po dziesiec kroc.
\par 42 I by byl Bóg ojca mego, Bóg Abrahama, i strach Izaaka, nie byl przy mnie, pewnie bys mie byl teraz próznego puscil; ale na utrapienie moje, i na prace rak moich wejrzal Bóg, i przestrzegal cie nocy przeszlej.
\par 43 Tedy odpowiedzial Laban, i rzekl do Jakóba: Córki te córki sa moje, i synowie ci sa synowie moi, i dobytek ten dobytek mój, i wszystko co widzisz, moje jest; a tym córkom moim, cóz dzis uczynie, albo synom ich, które zrodzily?
\par 44 Pójdzze tedy, a uczynmy przymierze, ja i ty, a bedzie swiadectwo miedzy mna, i miedzy toba.
\par 45 I wzial Jakób kamien, a postawil go na znak.
\par 46 I rzekl Jakób do braci swej: Nazbierajcie kamieni; którzy nanosili kamieni, i uczynili kupe, i jedli tam na onej kupie.
\par 47 I nazwal ja Laban Jegar Sahaduta, a Jakób ja nazwal Galed.
\par 48 Bo mówil Laban: Kupa ta niech bedzie swiadkiem miedzy mna i miedzy toba dzisiaj; przetoz Jakób nazwal imie jej Galed,
\par 49 I Myspa; albowiem rzekl Laban: Niech upatruje Pan miedzy mna i miedzy toba, gdy sie rozejdziemy jeden od drugiego.
\par 50 Jezli bedziesz trapil córki moje, i jezli pojmiesz zony nad córki moje, nie masz tu nikogo miedzy nami; bacz, ze Bóg jest swiadkiem miedzy mna i miedzy toba.
\par 51 I rzekl nad to Laban do Jakóba: Oto, ta kupa kamieni, i oto, znak ten, którym postanowil miedzy mna i miedzy toba.
\par 52 Swiadkiem ta kupa, i swiadkiem ten znak bedzie tego, iz ja do ciebie nie pójde dalej za te kupe, i ty tez nie pójdziesz do mnie za te kupe, i za ten znak, na zle.
\par 53 Bóg Abrahamów i Bóg Nachorów niechaj rozsadza miedzy nami, Bóg ojca ich. Przysiagl tedy Jakób przez strach ojca swego Izaaka.
\par 54 I nabil Jakób bydla na górze, i wezwal braci swej ku jedzeniu chleba. Tedy jedli chleb, i nocowali na onej górze.
\par 55 Potem Laban wstawszy bardzo rano, pocalowal syny swoje i córki swe, i blogoslawil im; a odszedlszy, wrócil sie Laban na miejsce swoje.

\chapter{32}

\par 1 A Jakób tez poszedl w droge swoje i potkali sie z nim Aniolowie Bozy.
\par 2 I rzekl Jakób ujrzawszy je: Obóz to Bozy; i nazwal imie miejsca onego Mahanaim.
\par 3 Potem poslal Jakób posly przed soba do Ezawa, brata swego, do ziemi Seir, do krainy Edomskiej.
\par 4 I rozkazal im mówiac: Tak rzeczecie do pana mego Ezawa: To mówi sluga twój Jakób: U Labana bylem gosciem, i mieszkalem z nim az do tego czasu.
\par 5 A mam woly i osly, owce, i slugi, i sluzebnice, a posylam odpowiedziec panu memu, zebym znalazl laske w oczach twoich.
\par 6 I wrócili sie poslowie do Jakóba, mówiac: Przyszlismy do brata twego Ezawa, który tez idzie przeciwko tobie, a cztery sta mezów z nim.
\par 7 I zlakl sie Jakób bardzo a strwozyl sie; i rozdzielil lud, który z nim byl, i owce, i woly, i wielblady, na dwa hufce;
\par 8 I rzekl: Jezliby przyszedl Ezaw do jednego hufca, a porazilby go, tedy hufiec, który pozostanie, bedzie zachowany.
\par 9 I rzekl Jakób: Boze ojca mego Abrahama, i Boze ojca mego Izaaka, Panie, którys do mnie rzekl: Wróc sie do ziemi twojej, i do rodziny twojej, a uczyniec dobrze.
\par 10 Mniejszym jest niz wszystkie zmilowania, i niz wszystka prawda, któras uczynil z sluga swym. Albowiem tylko o lasce mojej przeszedlem ten Jordan, a teraz mam dwa hufce.
\par 11 Wyrwij mie prosze z reki brata mego, z reki Ezawa; boc sie go boje, by snac przyszedlszy nie zabil mie, i matki z synami.
\par 12 Wszakes rzekl: Dobrze czyniac bedec dobrze czynil, a rozmnoze nasienie twoje jako piasek morski, który zliczon byc nie moze, dla mnóstwa.
\par 13 I przenocowal tam onej nocy, i wzial z tego, co mial przy reku, upominek dla Ezawa, brata swego.
\par 14 To jest kóz dwiescie, i kozlów dwadziescia, owiec dwiescie, i baranów dwadziescia.
\par 15 Wielbladzic odchowujacych mlode, ze zrebiety ich, trzydziesci, krów czterdziesci, i wolów dziesiec, dwadziescia oslic, i dziesiec oslat.
\par 16 I oddal je w rece slug swoich, kazde stado z osobna, i rzekl do slug swoich: Idzcie przede mna, a plac uczyncie miedzy stadem a miedzy stadem.
\par 17 I rozkazal pierwszemu, mówiac: Gdy sie spotka z toba Ezaw, brat mój, a spyta cie, mówiac: Czyjes ty? i dokad idziesz? a czyje to stado przed toba?
\par 18 Tedy powiesz: Slugi twego Jakóba jest to upominek, poslany panu memu Ezawowi, a oto, i sam idzie za nami.
\par 19 Takze tez rozkazal drugiemu, i trzeciemu, i wszystkim idacym za temi stady, mówiac: Temiz slowy mówcie do Ezawa, gdy go potkacie.
\par 20 Powiecie mu tez: Oto, sluga twój Jakób idzie za nami, mówil bowiem: Ublagam oblicze jego upominkiem, który idzie przede mna, a potem ujrze oblicze jego; owa mie snac w laske przyjmie.
\par 21 I poszedl w przód on upominek przed obliczem jego, a sam przenocowal onej nocy z hufcem swoim.
\par 22 Wstawszy tedy onej nocy, wzial obie zony swe, i dwie sluzebnice swoje, i jedenascie synów swoich, i przeszedl przez bród Jabok,
\par 23 A wziawszy je, przeprawil je przez tez rzeke, i przeprowadzil wszystko, co mial.
\par 24 A tylko sam Jakób zostal.
\par 25 A oto, biedzil sie z nim maz az do wejscia zorzy; który widzac, ze go nie mógl przemóc, uderzyl Jakóba w staw biodry jego, i wytracila sie z stawu biodra Jakóbowa, gdy sie z nim mocowal.
\par 26 I rzekl: Pusc mie, bo juz wschodzi zorza. I odpowiedzial: Nie puszcze cie, az mi bedziesz blogoslawil.
\par 27 Tedy mu rzekl: Co za imie twoje? I odpowiedzial: Jakób.
\par 28 I rzekl: Nie bedzie nazywane wiecej imie twoje Jakób, ale Izrael; bos sobie meznie poczynal z Bogiem, i z ludzmi, i przemogles.
\par 29 I spytal Jakób mówiac: Oznajmij mi prosze imie twoje; a on odpowiedzial: Czemu sie pytasz o imieniu mojem? I tamze mu blogoslawil.
\par 30 Tedy nazwal Jakób imie miejsca onego Fanuel, mówiac: Izem widzial Boga twarza w twarz, a zachowana jest dusza moja.
\par 31 I weszlo mu slonce, kiedy minal miejsce Fanuel, a on uchramowal na biodre swoje.
\par 32 Przetoz nie jadaja synowie Izraelscy zyly skurczonej, która jest przy stawie biodry, az do dnia tego, iz byl uderzyl w staw biodry Jakóbowej, i w zyle skurczona.

\chapter{33}

\par 1 A podnióslszy Jakób oczy swe ujrzal, a oto, Ezaw idzie, a z nim cztery sta mezów; i rozdzielil dzieci, z osobna Lii, i z osobna Racheli, i z osobna dwu sluzebnic.
\par 2 I postawil sluzebnice, i dzieci ich, na przodku, a Lije, i syny jej, za nimi, Rachele zas z Józefem na ostatku.
\par 3 A sam szedl przed nimi, i poklonil sie az do ziemi siedem kroc, niz przyszedl do brata swego.
\par 4 I zaszedl mu droge Ezaw, i oblapiwszy go, upadl na szyje jego, i calowal go; i plakali.
\par 5 Potem podnióslszy (Ezaw) oczy swe, ujrzal zony i dzieci, i rzekl: A ci co zacz sa, twoiz to? i odpowiedzial: Dziatki to sa, które Bóg dal z laski sludze twemu.
\par 6 I przyblizyly sie sluzebnice i synowie ich, a poklonily sie.
\par 7 Przyblizyla sie tez i Lija, i dzieci jej, i poklonili sie; a potem przyblizyl sie Józef i Rachel, i poklonili sie.
\par 8 I rzekl Ezaw: A ów wszystek hufiec na co, z którymem sie spotkal? Odpowiedzial Jakób: Abym znalazl laske przed oczyma pana mego.
\par 9 Tedy rzekl Ezaw: Mam ja dosyc, bracie mily, miej ty swoje.
\par 10 I rzekl Jakób: Nie tak bedzie prosze; jezlim teraz znalazl laske w oczach twoich, wezmij upominek mój z reki mojej, przeto, izem widzial oblicze twoje, jakobym widzial oblicze Boze, i laskawies mie przyjal;
\par 11 Przyjmijze prosze dar mój, którym ci przyniósl, gdyz mie hojnie blogoslawil Bóg, a mam wszystkiego dosyc. A tak uprosil go, ze to przyjal.
\par 12 Tedy rzekl Ezaw: Ruszmy sie, a idzmy, a ja pójde przed toba.
\par 13 I odpowiedzial mu Jakób: Wie pan mój, ze z soba mam dziatki mlode, i owce kotne, i krowy cielne, które jezlibym przegnal dnia jednego, pozdychaja wszystkie stada.
\par 14 Niech w przód prosze jedzie pan mój przed sluga swoim, a ja poprowadze sie z lekka, jako zdazy trzoda, która jest przede mna, i jako nadaza dzieci, az przyjde do pana mego do Seir.
\par 15 Tedy rzekl Ezaw: Niech wzdy zostawie przy tobie cokolwiek ludu, który jest ze mna. A on odpowiedzial: A na cóz to? bylem znalazl laske w oczach pana mego.
\par 16 I wrócil sie dnia onego Ezaw droga swa do Seir.
\par 17 A Jakób obrócil sie do Suchot, i zbudowal sobie dom, i dla stad swoich poczynil obory, a dla tego nazwal imie miejsca onego Suchot.
\par 18 I przyszedl Jakób zdrowo do miasta Sychem, które bylo w ziemi Chananejskiej, gdy sie wrócil z Padan Syryjskiego, i polozyl sie przed miastem.
\par 19 I kupil czesc pola, na którem rozbil namiot swój, od synów Hemora, ojca Sychemowego, za sto jagniat;
\par 20 A postawil tam oltarz, i nazwal go: Mocny Bóg Izraelski.

\chapter{34}

\par 1 I wyszla Dyna, córka Lii, która byla urodzila Jakóbowi, aby ogladala córki onej ziemi.
\par 2 A ujrzawszy ja Sychem, syn Hemora Hewejczyka, ksiazecia ziemi onej, porwal ja, i spal z nia, i zelzyl ja.
\par 3 I spoila sie dusza jego z Dyna, córka Jakóbowa, a rozmilowawszy sie dzieweczki, mówil do serca jej.
\par 4 Tedy Sychem rzekl do Hemora, ojca swego, mówiac: Wezmij mi te dzieweczke za zone.
\par 5 A gdy Jakób uslyszal, ze zgwalcona byla Dyna, córka jego, a synowi jego byli z bydlem jego na polu, zamilczal tego Jakób, az sie oni zwrócili.
\par 6 Tedy wyszedl Hemor, ojciec Sychemów, do Jakóba, aby z nim mówil.
\par 7 A synowie Jakóbowi gdy przyszli z pola, a uslyszeli to, bolescia zjeci byli mezowie oni, i rozgniewali sie bardzo, ze te sprosnosc uczynil w Izraelu, spiac z córka Jakóbowa, co byc nie mialo.
\par 8 I rzekl Hemor do nich mówiac: Sychem, syn mój, przylozyl serce swe ku córce waszej; dajciez mu ja prosze za zone.
\par 9 A spowinowaccie sie z nami, córki wasze dawajac nam, a córki nasze pojmujac sobie.
\par 10 I bedziecie z nami mieszkac, a ziemia bedzie przed wami; mieszkajcie, i handlujcie w niej, i osadzajcie sie w niej.
\par 11 I mówil tez Sychem do ojca jej, i braci jej: Niech znajde laske w oczach waszych, a co mi rzeczecie, to dam.
\par 12 Podwyzszcie mi znacznie wiana, i upominków zadajcie, a dam jako mi rzeczecie; tylko mi dajcie te dzieweczke za zone.
\par 13 Tedy odpowiedzieli synowie Jakóbowi Sychemowi i Hemorowi, ojcu jego, na zdradzie mówiac z nimi, dla tego iz zgwalcil Dyne, siostre ich.
\par 14 I rzekli im: Nie mozemy tej rzeczy uczynic, abysmy mieli dac siostre nasze mezowi nieobrzezanemu; bo to obrzydla rzecz u nas.
\par 15 A wszakze tym sposobem wam pozwolimy, jezlize chcecie byc nam podobni, aby byl obrzezany miedzy wami kazdy mezczyzna;
\par 16 Tedy wam damy córki nasze, a córki wasze pojmiemy sobie, i bedziemy mieszkac z wami, a bedziemy ludem jednym.
\par 17 Ale jezlibyscie nas nie usluchali, abyscie sie obrzezali, wezmiemy córke nasze, i odejdziemy.
\par 18 I podobala sie ta rzecz ich Hemorowi i Sychemowi, synowi Hemorowemu.
\par 19 Tedy nie odkladal on mlodzieniec dlugo tej rzeczy, bo sie byl rozmilowal córki Jakóbowej; a on byl ze wszech najzacniejszy w domu ojca swego.
\par 20 I przyszedl Hemor i Sychem, syn jego, do bramy miasta swego, i rzekli do mezów miasta swego mówiac:
\par 21 Mezowie ci spokojnie zyja z nami; niechze mieszkaja w tej ziemi, i niech handluja w niej, gdyz oto ziemia nasza dosyc jest przestronna dla nich; córki ich bedziemy brac sobie za zony, a córki nasze bedziemy im dawac.
\par 22 Ale tym sposobem pozwalaja mezowie ci, mieszkac z nami, abysmy byli jednym ludem: zeby byl obrzezan miedzy nami kazdy mezczyzna, tak jako oni sa obrzezani.
\par 23 Trzody ich, i majetnosci ich, i wszystkie bydla ich, azaz nie nasze beda? na to tylko im pozwólmy, a beda mieszkac z nami.
\par 24 I usluchali Hemora i Sychema, syna jego, wszyscy wychodzacy z bramy miasta jego, i obrzezal sie kazdy mezczyzna, cokolwiek ich wychodzilo z bramy miasta jego.
\par 25 I stalo sie dnia trzeciego gdy byli w najciezszym bólu, tedy wzieli dwaj synowie Jakóbowi, Symeon i Lewi, bracia Dyny, kazdy miecz swój, a weszli do miasta smiele, i pomordowali wszystkie mezczyzny.
\par 26 Hemora tez i Sychema, syna jego, zabili mieczem, a wziawszy Dyne z domu Sychemowego, odeszli.
\par 27 Drudzy tez synowie Jakóbowi przyszli do pobitych, i zlupili miasto, przeto iz zgwalcili siostre ich.
\par 28 Owce ich, i woly ich, i osly ich, i co w miescie bylo, i co na polu, pobrali.
\par 29 I wszystke majetnosc ich, i wszystkie dzieci ich, i zony ich, w niewola zabrali, i wybrali wszystko, co w domach bylo.
\par 30 Tedy rzekl Jakób do Symeona i Lewiego: Zafrasowaliscie mie, a przywiedliscie mie w ohyde u obywateli ziemi tej, u Chananejczyków i Ferezejczyków; ja niewielka liczbe ludu mam, a zebrawszy sie przeciwko mnie, poraza mie, a tak zgine ja, i dom mój
\par 31 A oni odpowiedzieli: Izali jako wszetecznicy mial uzywac siostry naszej?

\chapter{35}

\par 1 Rzekl potem Bóg do Jakóba: Wstan, wstap do Betela, a mieszkaj tam, i uczyn tam oltarz Bogu, któryc sie ukazal, gdys uciekal przed obliczem Ezawa, brata twego.
\par 2 Tedy rzekl Jakób do domowników swych, i do wszystkich, którzy z nim byli: Odrzuccie bogi cudze, którzy w posrodku was sa, a oczysccie sie, i odmiencie szaty wasze.
\par 3 A wstawszy pójdzmy do Betela, i uczynie tam oltarz Bogu, który mie wysluchal w dzien utrapienia mego, i byl ze mna w drodze, któram chodzil.
\par 4 A oddali Jakóbowi wszystkie bogi cudze, które mieli, i nausznice, które byly na uszach ich, i zakopal je Jakób pod onym debem, który byl niedaleko Sychem.
\par 5 I wyszli stamtad; a strach Bozy padl na miasta, które byly okolo nich, iz nie gonili synów Jakóbowych.
\par 6 Przyszedl tedy Jakób do Luzy, która jest w ziemi Chananejskiej, ta jest Betel, sam i wszystek lud, który z nim byl.
\par 7 I zbudowal tam oltarz, a nazwal miejsce ono El Betel; bo mu sie tam byl Bóg ukazal, gdy uciekal przed obliczem brata swego.
\par 8 Tedy umarla Debora, mamka Rebeki, i pogrzebiona jest przy Betel pod debem, i nazwal imie onego miejsca, Allon Bachut.
\par 9 I ukazal sie Bóg znowu Jakóbowi, gdy sie wracal z Padan Syryjskiego, i blogoslawil mu.
\par 10 I rzekl mu Bóg: Imie twoje jest Jakób; nie tylko bedzie zwane imie twoje na potem Jakób, ale Izrael bedzie imie twoje; i nazwal imie jego Izrael.
\par 11 I rzekl mu Bóg: Jam jest Bóg wszechmogacy, rozradzaj sie, i rozmnazaj sie; naród, i mnóstwo narodów bedzie z ciebie, a królowie z biódr twoich wynijda.
\par 12 I ziemie, któram dal Abrahamowi i Izaakowi, tobie ja dam, i nasieniu twemu po tobie dam te ziemie.
\par 13 I odszedl Bóg od niego z miejsca, na którem mówil z nim.
\par 14 Zatem postawil Jakób znak na miejscu onem, gdzie Bóg mówil z nim, a znak on byl kamienny, i pokropil go pokropieniem, i polal go oliwa.
\par 15 I nazwal Jakób imie miejsca onego, gdzie Bóg z nim mówil, Betel.
\par 16 Potem odeszli z Betel; i bylo jeszcze jakoby mila drogi do Efraty, i rodzila Rachel a ciezkie rodzenie miala.
\par 17 A gdy ciezko pracowala przy rodzeniu, rzekla baba do niej: Nie bój sie; bo i tego syna bedziesz miala.
\par 18 A stalo sie, gdy wychodzila dusza jej, (bo tamze umarla), nazwala imie jego Ben Oni; ale ojciec jego nazwal go Benjamin.
\par 19 A tak umarla Rachel, i pogrzebiona jest na drodze ku Efracie; tac jest Betlehem.
\par 20 I postawil Jakób znak nad grobem jej; toc jest znak grobu Rachelinego az po dzis dzien.
\par 21 I poszedl stamtad Izrael, i rozbil namiot swój za wieza Heder.
\par 22 Stalo sie tedy, gdy mieszkal Izrael w onej krainie, ze szedl Ruben, i spal z Bala, zaloznica ojca swego, i uslyszal to Izrael. A bylo synów Jakóbowych dwanascie.
\par 23 Synowie Lii: pierworodny Jakóbów Ruben, i Symeon, i Lewi, i Judas, i Isaszar, i Zabulon.
\par 24 Synowie Racheli: Józef i Benjamin.
\par 25 A synowie Bali, sluzebnicy Rachelinej: Dan i Neftali.
\par 26 Synowie tez Zelfy, sluzebnicy Lii: Gad i Aser. Ci sa synowie Jakóbowi, którzy mu sie urodzili w Padanie Syryjskim.
\par 27 I przyszedl Jakób do Izaaka, ojca swego, do Mamre, do miasta Arba, to jest Hebron, gdzie mieszkal Abraham i Izaak.
\par 28 A bylo dni Izaakowych sto lat, i osiemdziesiat lat.
\par 29 I dokonal Izaak, i umarl, i przylaczony jest do ludu swego, stary i pelen dni; a pogrzebli go Ezaw, i Jakób, synowie jego.

\chapter{36}

\par 1 A tec sa rodzaje Ezawowe, który jest Edom.
\par 2 Ezaw pojal zony swoje z córek Chananejskich: Ade, córke Elona, Hetejczyka; i Oolibame, córke Any, córki Sebeona, Hewejczyka;
\par 3 I Basemat, córke Ismaelowe, siostre Nebajotowe.
\par 4 I urodzila Ada Ezawowi Elifasa, a Basemat urodzila Rehuela.
\par 5 Oolibama tez urodzila Jehusa, i Jeloma, i Korego. Ci sa synowie Ezawowi, którzy mu sie urodzili w ziemi Chananejskiej.
\par 6 I wzial Ezaw zony swoje, i syny swoje, i córki swoje, i wszystkie dusze domu swego, i trzody swoje, i wszystko bydlo swoje, i wszystke majetnosc swoje, której byl nabyl w ziemi Chananejskiej, i odszedl do ziemi inszej od Jakóba, brata swego;
\par 7 Bo byla majetnosc ich wielka, ze nie mogli mieszkac pospolu, i nie mogla ich zniesc ziemia pielgrzymowania ich, dla mnóstwa stad ich.
\par 8 I mieszkal Ezaw na górze Seir, a ten Ezaw jest Edom.
\par 9 A tec sa pokolenia Ezawa, ojca Edomczyków, na górze Seir.
\par 10 I te sa imiona synów Ezawowych: Elifas, syn Ady, zony Ezawowej, Rehuel, syn Basematy, zony Ezawowej.
\par 11 Synowie zas Elifasowi byli: Teman, Omar, Sefo, i Gaatan, i Kenaz.
\par 12 A Tamna byla zaloznica Elifasa, syna Ezawowego, i urodzila Elifasowi Amaleka. Ci sa synowie Ady, zony Ezawowej.
\par 13 Ci tez sa synowie Rehuelowi: Nahat i Zara, Samma i Meza; ci byli synowie Basematy, zony Ezawowej.
\par 14 Ci zas byli synowie Oolibamy, córki Any, córki Sebeona, zony Ezawowej: i urodzila Ezawowi Jehusa, i Jeloma, i Korego.
\par 15 Tec sa ksiazeta z synów Ezawowych, synowie Elifasa pierworodnego Ezawowego: Ksiaze Teman, ksiaze Omar, ksiaze Sefo, ksiaze Kenaz.
\par 16 Ksiaze Kore, ksiaze Gaatam, ksiaze Amalek. Tec ksiazeta z Elifasa poszly, w ziemi Edomskiej, ci sa synowie z Ady.
\par 17 Ci zas sa synowie Rehuela, syna Ezawowego: Ksiaze Nahat, ksiaze Zara, ksiaze Samma, ksiaze Meza. Te ksiazeta poszly z Rehuela, w ziemi Edomskiej, ci sa synowie Basematy zony Ezawowej.
\par 18 Ci zas sa synowie Oolibamy, zony Ezawowej: Ksiaze Jehus, ksiaze Jelom, ksiaze Kore. Te ksiazeta poszly z Oolibamy, córki Any, zony Ezawowej.
\par 19 Ci sa synowie Ezawowi, i te ksiazeta ich. Onze jest Edom.
\par 20 Ci tez sa synowie Seira Chorejczyka, mieszkajacy w onej ziemi: Lotan, i Sobal, i Sebeon, i Hana.
\par 21 I Dysson, i Eser, i Disan; tec sa ksiazeta Chorejskie, synowie Seirowi w ziemi Edomskiej.
\par 22 A synowie Lotanowi byli Chory i Heman; a siostra Lotanowa Tamna.
\par 23 Synowie zas Sobalowi: Halwan, i Manahat, i Hewal, Sefo, i Onam.
\par 24 Synowie tez Sebeonowi ci sa: Aja i Ana. Tenci to Ana, który wynalazl muly na puszczy, gdy pasl osly Sebeona, ojca swego.
\par 25 Dzieci zas Anowe te sa: Dyson, i Oolibama, córka Anowa.
\par 26 A synowie Dysonowi: Hamdan, i Eseban, i Jetran, i Charan.
\par 27 A synowie Eserowi sa ci: Balaan, i Zawan, i Akan.
\par 28 A zasie synowie Dysanowi: Hus i Aran.
\par 29 Tec sa ksiazeta Chorejskie: ksiaze Lotan, ksiaze Sobal, ksiaze Sebeon, ksiaze Ana,
\par 30 Ksiaze Dyson, ksiaze Eser, ksiaze Dysan. Te byly ksiazeta Chorejskie, wedlug ksiestw ich, w ziemi Seir.
\par 31 Ci tez byli królowie, którzy królowali w ziemi Edomskiej, pierwej niz królowal król nad syny Izraelskimi.
\par 32 Królowal tedy w Edom Bela, syn Beorów, a imie miasta jego Dynhaba.
\par 33 I umarl Bela, a królowal miasto niego Jobab, syn Zerachów z Bosry.
\par 34 I umarl Jobab, a królowal miasto niego Chusam, z ziemi Temanskiej.
\par 35 I umarl Chusam, a królowal miasto niego Hadad, syn Badadów, który porazil Madyjanczyki, na polu Moabskiem, a imie miasta jego Hawid.
\par 36 I umarl Hadad, a królowal miasto niego Samla z Masreki.
\par 37 I umarl Samla, a królowal miasto niego Saul, z Rechobot u rzeki.
\par 38 I umarl Saul, a królowal miasto niego Balanan, syn Achborów.
\par 39 I umarl Balanan, syn Achborów, a królowal miasto niego Hadar, a imie miasta jego Pahu, a imie zony jego Mehetabel, córka Matredy, córki Mezaabowej.
\par 40 Tec sa imiona ksiazat Ezawowych, wedlug ich pokolenia, i wedlug miejsc ich, i imion ich: Ksiaze Tamna, ksiaze Halwa, ksiaze Jetet.
\par 41 Ksiaze Oolibama, ksiaze Ela, ksiaze Pynon.
\par 42 Ksiaze Kenaz, ksiaze Teman, ksiaze Mabsar
\par 43 Ksiaze Magdyjel, ksiaze Hyram, te sa ksiazeta Edomskie, wedlug mieszkania ich, w ziemi osiadlosci ich. Ten jest Ezaw, ojciec Edomczyków.

\chapter{37}

\par 1 I mieszkal Jakób w ziemi, gdzie przychodniem byl ojciec jego, w ziemi Chananejskiej.
\par 2 Tec sa pokolenia Jakóbowe: Józef, gdy mial siedemnascie lat, pasl z bracia swoja trzody, (bedac pacholeciem), z synami Bali, i z synami Zelfy, zon ojca swego; i odnosil Józef slawe ich zla do ojca ich.
\par 3 A Izrael milowal Józefa nad wszystkie syny swe, iz mu sie byl w starosci jego urodzil, i sprawil mu suknia rozmaitych farb.
\par 4 A widzac bracia jego, ze go milowal ojciec ich nad wszystke bracia jego, nienawidzili go, i nie mogli nic laskawie z nim mówic.
\par 5 I snil sie Józefowi sen, a gdy go powiedzial braci swej, tem go wiecej mieli w nienawisci.
\par 6 Bo rzekl do nich: Sluchajcie prosze snu tego, który mi sie snil.
\par 7 Otosmy wiazali snopy na polu, a oto, wstawszy snop mój stanal, a okolo niego stojace snopy wasze klanialy sie snopowi mojemu.
\par 8 I odpowiedzieli mu bracia jego: Izali królowac bedziesz nad nami? i panowac nam bedziesz? stadze go jeszcze mieli w wiekszej nienawisci, dla snów jego, i dla slów jego.
\par 9 Snil mu sie tez jeszcze drugi sen, i powiedzial go braci swej, mówiac: Oto mi sie znowu snil sen: A ono slonce i miesiac, i jedenascie gwiazd klanialo mi sie.
\par 10 I powiedzial ojcu swemu i braci swej, i gromil go ojciec jego i mówil mu: Cóz to za sen, coc sie snil? Izali przyjdziemy, ja i matka twoja z bracia twoja, abysmyc sie klaniali az do ziemi.
\par 11 I nienawidzili go bracia jego; ale ojciec jego pilnie uwazal te rzecz.
\par 12 I odeszli bracia jego, aby pasli trzody ojca swego w Sychem.
\par 13 Tedy rzekl Izrael do Józefa: Izali bracia twoi nie pasa w Sychem? pójdzze, a posle cie do nich; a on odpowiedzial: Otom ja.
\par 14 Rzekl mu tedy: Idzze teraz, a dowiedz sie, jako sie maja bracia twoi, i co sie dzieje z trzodami, i dasz mi znac. Wyslal go tedy z doliny Hebron, i przyszedl do Sychem.
\par 15 I nadszedl go niektóry maz, a on sie blakal po polu; i pytal go maz on mówiac:
\par 16 Czegóz szukasz? A on odpowiedzial: Braci mojej szukam; powiedz mi prosze, gdzie oni pasa.
\par 17 Tedy rzekl on czlowiek: Odeszli stad; bom slyszal, gdy mówili: Pójdzmy do Dotain. I szedl Józef za bracia swoja, a znalazl je w Dotain.
\par 18 I ujrzeli go z daleka, a pierwej niz do nich przyszedl, radzili o nim, aby go zabili.
\par 19 I mówili jeden do drugiego: Onoz mistrz on snów idzie.
\par 20 Teraz tedy pójdzcie, a zabijmy go, i wrzucmy go w jaka studnia, a rzeczemy: Zly go zwierz pozarl; a tak obaczymy, na co mu wynijda sny jego.
\par 21 Co gdy uslyszal Ruben, chcial go wybawic z rak ich, mówiac: Nie zabijajmy go.
\par 22 Nad to rzekl do nich Ruben: Nie wylewajcie krwi, ale wrzuccie go w te studnia, która jest na puszczy, a reki nie sciagajcie nan. A to mówil, aby go wybawil z rak ich, i powrócil go ojcu swemu.
\par 23 I stalo sie, gdy przyszedl Józef do braci swej, zwlekli go z sukni jego, z sukni rozmaitych farb, która mial na sobie.
\par 24 A porwawszy go, wrzucili go w studnia, która studnia byla czcza, i nie bylo w niej wody.
\par 25 A usiadlszy, aby jedli chleb, podniesli oczy swe, i ujrzeli, a ono poczet Ismaelitów, idacych z Galaad; a wielblady ich niosly korzenie, i kadzidlo, i myrre, a szly, aby to zaniosly do Egiptu.
\par 26 Tedy rzekl Judas do braci swej: Cóz za pozytek, chocbysmy zabili brata naszego, i zataili krwi jego?
\par 27 Pójdzcie, a przedajmy go Ismaelitom, a reka nasza niech nie bedzie na nim; brat bowiem nasz, i cialo nasze jest; i usluchali go bracia jego.
\par 28 A gdy mijali oni mezowie, Madyjanscy kupcy, tedy wyciagneli, i wyjeli Józefa z studni, i sprzedali Józefa Ismaelitom za dwadziescia srebrników, którzy zaprowadzili Józefa do Egiptu.
\par 29 Tedy sie wrócil Ruben do onej studni, a oto, juz nie bylo Józefa w studni; i rozdarl szaty swoje.
\par 30 A wróciwszy sie do braci swej, rzekl: Pacholecia nie masz, a ja dokad? ja dokad pójde?
\par 31 Tedy wzieli suknia Józefowe, i zabili kozla, a umaczali suknia we krwi.
\par 32 I poslali one suknia rozmaitych farb, aby ja zaniesiono do ojca jego, i rzekli: Tes my znalezli, poznajze teraz, jezli to suknia syna twego, czyli nie
\par 33 A poznawszy ja, rzekl: Suknia jest syna mego; zwierz zly pozarl go; koniecznie rozszarpany jest Józef.
\par 34 Tedy rozdarlszy Jakób szaty swe, wlozyl wór na biodra swoje, zalujac syna swego przez wiele dni.
\par 35 I zeszli sie wszyscy synowie jego, i wszystkie córki jego, aby go cieszyli, lecz nie dal sie cieszyc, ale mówil: Zaprawde zstapie za synem moim do grobu; i plakal go ojciec jego.
\par 36 A Madyjanczycy sprzedali Józefa do Egiptu Potyfarowi, dworzaninowi Faraonowemu, hetmanowi zolnierstwa.

\chapter{38}

\par 1 I stalo sie czasu onego, ze Judas odszedl od braci swej, i wstapil do niektórego meza Odolamickiego, którego imie bylo Chyra.
\par 2 I ujrzal tam Judas córke meza Chananejskiego, którego zwano Sua; a pojawszy ja, wszedl do niej.
\par 3 A ona poczawszy porodzila syna, i nazwala imie jego Her.
\par 4 Zasie poczawszy porodzila syna, i nazwala imie jego Onan.
\par 5 Nad to jeszcze urodzila syna, i nazwala imie jego Sela; a Judas byl w Chezybie, gdy mu urodzila.
\par 6 I dal Judas zone Herowi pierworodnemu swemu, której imie bylo Tamar.
\par 7 I byl Her, pierworodny Judasów, zly w oczach Panskich, i zabil go Pan.
\par 8 Tedy rzekl Judas do Onana: Wnijdz do zony brata twego, a zlacz sie z nia prawem powinowactwa, i wzbudz nasienie bratu twemu.
\par 9 Lecz wiedzac Onan, iz to potomstwo nie jemu byc mialo, gdy wchodzil do zony brata swego, tracil z siebie nasienie na ziemie, aby nie wzbudzil potomstwa bratu swemu.
\par 10 I nie podobalo sie to Panu, co Onan czynil; przeto go tez Pan zabil.
\par 11 Zatem rzekl Judas do Tamary, niewiasty swej: Mieszkaj wdowa w domu ojca twego, az urosnie Sela, syn mój, bo rzekl: By on tez snac nie umarl jako bracia jego. I odeszla Tamar, i mieszkala w domu ojca swego.
\par 12 A gdy minelo wiele dni, umarla córka Suego, zona Judasowa; i pocieszywszy sie Judas, szedl do tych, co strzygli owce jego, sam i Chyra, towarzysz jego, Odolamita, do Timnat.
\par 13 I oznajmiono to Tamarze, mówiac: Oto, swieker twój idzie do Timnat, aby strzygl owce swoje.
\par 14 Która zlozywszy z siebie szaty wdowienstwa swego, okryla sie rabkiem, i zatknela sie, i usiadla na rozstaniu drogi, która wiedzie do Timnat; bo widziala, ze byl urósl Sela, a ona nie byla mu dana za zone.
\par 15 A ujrzawszy ja Judas, mniemal, ze to nierzadnica, bo zakryla byla twarz swoje.
\par 16 Tedy ustapiwszy do niej z drogi, mówil: Prosze niech wnijde do ciebie; albowiem nie wiedzial, zeby jego synowa byla. I rzekla: Cóz mi dasz, zebys do mnie wszedl?
\par 17 I odpowiedzial: Poslec kozlatko z trzody; a ona rzekla: Daszze mi zastaw, az mi je przyslesz?
\par 18 I rzekl: Cóz ci mam dac w zastaw? A ona odpowiedziala: Pierscien twój, i chustke twoje, i laske twoje, która masz w rece swej. Tedy jej dal, i wszedl do niej; a poczela z niego.
\par 19 A wstawszy odeszla, i zlozywszy z siebie odzienie swoje, oblekla sie w szaty wdowienstwa swego.
\par 20 Potem poslal Judas kozlatko, przez reke towarzysza swego Odolamite, aby odebral zastawe z reki niewiasty onej; ale jej nie znalazl.
\par 21 I pytal mezów miejsca onego, mówiac: Gdzie jest nierzadnica ona, która byla na rozstaniu tej drogi? Którzy odpowiedzieli: Nie bylo tu nierzadnicy.
\par 22 Wrócil sie tedy do Judasa, i rzekl: Nie znalazlem jej; lecz i mezowie miejsca onego powiedzieli: Nie bylo tu zadnej nierzadnicy.
\par 23 Tedy rzekl Judas: Niechze sobie ma ten zaklad, abysmy nie byli na wzgarde; otom posylal to kozlatko, a tys jej nie znalazl.
\par 24 I stalo sie, jakoby po trzech miesiacach, powiedziano Judzie, mówiac: Dopuscila sie nierzadu Tamar, synowa twoja, a oto, juz brzemienna jest z nierzadu. Tedy rzekl Judas: Wywiedzcie ja, aby byla spalona.
\par 25 A gdy byla wywiedziona, poslala do swiekra swego, mówiac: Z meza, którego te rzeczy sa, jestem brzemienna. Przy tem powiedziala: Poznaj prosze, czyj to pierscien, i chustka, i laska?
\par 26 Tedy poznawszy to Judas rzekl: Sprawiedliwsza jest nad mie, poniewazem jej nie dal Seli, synowi memu; i wiecej jej nie uznal.
\par 27 I stalo sie, gdy przyszedl czas rodzenia jej, oto bliznieta byly w zywocie jej.
\par 28 A gdy rodzila, wytknelo reke jedno dziecie, która ujawszy baba, uwiazala u reki nic czerwona mówiac:Ten pierwej wynijdzie.
\par 29 I stalo sie, gdy zasie wciagnelo reke swoje, oto, wyszedl brat jego; i rzekla: Czemus przerwal? na tobie niech bedzie rozerwanie; i nazwala imie jego Fares.
\par 30 A potem wyszedl brat jego, na którego rece byla nic czerwona; i nazwala imie jego Zera.

\chapter{39}

\par 1 Tedy Józef byl zawiedzion do Egiptu; i kupil go Potyfar, dworzanin Faraonów, hetman zolnierstwa, maz Egipczanin, z reki Ismaelitów, którzy go tam byli zawiedli.
\par 2 I byl Pan z Józefem, który byl mezem szczesliwie postepujacym, a mieszkal w domu pana swego Egipczanina.
\par 3 I baczyl pan jego, ze Pan byl z nim, a iz wszystko co on czynil, Pan szczescil w reku jego.
\par 4 I znalazl Józef laske w oczach jego, i sluzyl mu; i przelozyl go nad domem swym, a podal wszystko co mial, w rece jego.
\par 5 I stalo sie, gdy go przelozyl nad domem swym, i nad wszystkiem, co mial, blogoslawil Pan domowi Egipczanina dla Józefa; i bylo blogoslawienstwo Panskie nad wszystkiem, cokolwiek mial w domu i na polu.
\par 6 Przetoz poruczyl wszystko, co mial, w rece Józefowe, i ni o czem u siebie nie wiedzial, tylko o chlebie, którego pozywal. A byl Józef pieknej twarzy, i wdzieczny na wejrzeniu.
\par 7 I stalo sie potem, iz obrócila zona pana jego oczy swoje na Józefa, i rzekla: Spij ze mna.
\par 8 Ale nie chcial: i rzekl do zony pana swego: Oto, pan mój nie wie, tak jako ja, co jest w domu jego; bo wszystko, co mial, podal w rece moje.
\par 9 I nie masz nikogo w domu tym nad mie przedniejszego, i nie wyjal mi nic z mocy, prócz ciebie, przeto zes ty jest zona jego; jakoz bym tedy mial uczynic te wielka zlosc, i grzeszyc przeciwko Bogu?
\par 10 I stalo sie, gdy ona namawiala Józefa na kazdy dzien, a on jej nie zezwalal, aby spal z nia, albo bywal z nia:
\par 11 Tedy dnia niektórego, gdy wszedl do domu, dla odprawowania pracy swej, a nie bylo tam z domowników nikogo w domu;
\par 12 Uchwycila go za szate jego, mówiac: Spij ze mna. Ale on zostawiwszy szate swoje w reku jej, uciekl, i wyszedl precz.
\par 13 A gdy ona obaczyla, iz zostawil szate swoje w reku jej, a uciekl precz;
\par 14 Tedy zawolala na czeladz domu swego, i rzekla do nich, mówiac: Wejcie, wprowadzil pan do nas meza Hebrejczyka, aby nas zelzyl; albowiem wszedl do mnie, aby ze mna spal, azem wolala glosem wielkim.
\par 15 A gdy uslyszal, zem wyniosla glos mój, i zawolala, zostawiwszy szate swoje u mnie, uciekl, i wyszedl precz.
\par 16 I zatrzymala jego szate u siebie, az przyszedl pan jego do domu swego;
\par 17 I rzekla do niego w te slowa, mówiac: Wszedl do mnie sluga ten Hebrejczyk, któregos przywiódl do nas, aby mie zelzyl.
\par 18 A gdym podniosla glos swój, i zawolala, tedy zostawil szate swa u mnie, i uciekl precz.
\par 19 I stalo sie, gdy uslyszal pan jego slowa zony swojej, które rzekla do niego, mówiac: Tak mi uczynil sluga twój, rozgniewal sie bardzo.
\par 20 I wzial pan Józefa, a dal go do domu wiezienia, tam, gdzie wieznie królewskie sadzano, i byl tam w domu wiezienia.
\par 21 A Pan byl z Józefem, i skloniwszy ku niemu milosierdzie, dal mu laske w oczach przelozonego nad domem wiezienia.
\par 22 Tedy przelozony nad domem wiezienia, podal w moc Józefowi wszystkie wieznie, którzy byli w domu wiezienia; a wszystko, co tam czynic mieli, to on sprawowal.
\par 23 A przelozony nad domem wiezienia nie dogladal tego, czego mu sie powierzyl, dla tego iz Pan byl z nim, a co on czynil, to Pan szczescil.

\chapter{40}

\par 1 I stalo sie potem, ze cos przewineli podczaszy króla Egipskiego, i piekarz przeciw panu swemu, królowi Egipskiemu.
\par 2 I rozgniewal sie Farao na obu dworzanów swoich, na przelozonego nad podczaszymi, i na przelozonego nad piekarzami.
\par 3 A dal je do wiezienia w dom hetmana zolnierzów, na miejsce, gdzie byl Józef wiezniem.
\par 4 I oddal im hetman zolnierzów Józefa, i sluzyl im; i byli przez niemaly czas w wiezieniu.
\par 5 Tedy sie onym obiema snil sen, kazdemu sen jego, jednejze nocy, kazdemu wedlug wykladu snu jego, podczaszemu i piekarzowi króla Egipskiego, którzy byli wiezniami w domu wiezienia.
\par 6 A przyszedlszy do nich Józef rano, ujrzal je, a oto byli strwozeni.
\par 7 I pytal dworzan Faraonowych, którzy byli z nim w wiezieniu, w domu pana jego, mówiac: Czemuzescie dzis tak smutnej twarzy?
\par 8 I odpowiedzieli mu: Snil sie nam sen, a nie masz kto by go wylozyl. Tedy rzekl do nich Józef: Izali nie Boze sa wyklady? powiedzcie mi prosze.
\par 9 A tak powiedzial przelozony nad podczaszymi sen swój Józefowi, i rzekl mu: Snilo mi sie, a oto winna macica przede mna,
\par 10 A na winnej macicy byly trzy galazki, a ona jakoby paki wypuszczala, a wychodzil kwiat jej, i dostawaly sie jagody gron winnych.
\par 11 A kubek Faraonów byl w rece mojej, wzialem tedy jagody, i wytlaczalem je w kubek Faraonów, i podawalem kubek w rece Faraonowe.
\par 12 Tedy mu powiedzial Józef: Ten jest wyklad snu tego: Trzy galazki, trzy dni sa.
\par 13 Po trzech dniach wywyzszy Farao glowe twa, a przywróci cie do pierwszego urzedu, i bedziesz podawal kubek Faraonowi do reki jego, wedlug zwyczaju pierwszego, gdys byl podczaszym jego.
\par 14 Tylko wspomnij sobie na mie, gdy sie bedziesz mial dobrze, i uczyn prosze ze mna milosierdzie, abys wzmianke uczynil o mnie przed Faraonem, i wybawil mie z domu tego;
\par 15 Bo mie kradzieza wzieto z ziemi Hebrajskiej, a do tego nicem tu nie uczynil, ze mie wrzucono do tego wiezienia.
\par 16 A widzac przelozony nad piekarzami, iz dobrze wylozyl, rzekl do Józefa: Jam tez we snie moim widzial, a oto, trzy kosze biale nad glowa moja.
\par 17 A w koszu najwyzszym byly wszelakie potrawy Faraonowe, roboty piekarskiej, a ptactwo jadlo je z kosza, który byl nad glowa moja.
\par 18 Tedy odpowiedzial Józef, i rzekl: Tenci jest wyklad jego: Trzy kosze, trzy dni sa;
\par 19 A po trzech dniach odejmie Farao glowe twoje od ciebie, i obwiesi cie na drzewie, a bedzie ptactwo jadlo cialo twoje z ciebie.
\par 20 I stalo sie dnia trzeciego, dnia narodzenia Faraonowego, ze uczynil uczte na wszystkie slugi swe, i policzyl glowe przelozonego nad podczaszymi, i glowe przelozonego na piekarzami w poczet slug swoich.
\par 21 I przywrócil przelozonego nad podczaszymi do podczastwa, aby podawal kubek do rak Faraonowych.
\par 22 A przelozonego nad piekarzami obwiesil, jako im byl sen wylozyl Józef.
\par 23 Jednak nie wspomnial przelozony nad podczaszymi na Józefa, ale go zapomnial.

\chapter{41}

\par 1 I stalo sie po wyjsciu dwóch lat, ze sie snilo Faraonowi, jakoby stal nad rzeka.
\par 2 A oto z rzeki wychodzilo siedem krów, pieknych na wejrzeniu i tlustych na ciele, które sie pasly na lace.
\par 3 Oto, tez siedem krów innych wychodzilo za niemi z rzeki, szpetnych na wejrzeniu, i chudych na ciele, które staly wedle krów pierwszych nad brzegiem rzeki.
\par 4 I pozarly one krowy szpetne na wejrzeniu i chude na ciele, siedem krów pieknych na wejrzeniu i tlustych; zatem ocknal sie Farao.
\par 5 A gdy usnal, snilo mu sie po wtóre; a ono siedem klosów wyrastalo z jednego zdzbla, pelnych i cudnych.
\par 6 Oto, tez siedem klosów cienkich i wysuszonych od wiatru wschodniego wyrastalo za nimi.
\par 7 I pozarly te klosy cienkie, siedem onych klosów pieknych i zupelnych; i ocknal sie Farao.
\par 8 A toc byl sen. A gdy bylo rano, strwozony byl duch jego; i poslawszy wezwal wszystkich wieszczków Egipskich, i wszystkich medrców jego, i opowiedzial im Farao sny swoje; a nie bylo, kto by je wylozyl Faraonowi.
\par 9 Zatem rzekl przelozony nad podczaszymi do Faraona, mówiac: Grzechy moje ja dzis przypominam sobie.
\par 10 Farao rozgniewawszy sie na slugi swe, dal mie byl pod straz do domu hetmana zolnierzów, mnie i przelozonego nad piekarzami.
\par 11 Tam sie nam snil sen jednejze nocy, mnie i jemu; kazdemu wedlug wykladu snu jego snilo sie.
\par 12 A byl tam z nami mlodzieniec Hebrejczyk, sluga hetmana zolnierzów, któremusmy powiedzieli, i wylozyl nam sny nasze, kazdemu wedlug snu jego wylozyl.
\par 13 I stalo sie, ze jako nam wylozyl, tak bylo; mie przywrócil król na miejsce moje, a onego obwiesil.
\par 14 Tedy poslawszy Farao, wezwal Józefa, i predko go wyprowadzono z wiezienia; który ostrzyglszy sie, i odmieniwszy szaty swoje przyszedl do Faraona.
\par 15 I rzekl Farao do Józefa: Snil mi sie sen, a nie mam, kto by mi go wylozyl; alem ja o tobie slyszal, gdy mówiono, ze gdy uslyszysz sen, umiesz go wylozyc.
\par 16 I odpowiedzial Józef Faraonowi, mówiac: Oprócz mnie Bóg opowie rzeczy szczesliwe Faraonowi.
\par 17 Tedy rzekl Farao do Józefa: Zdalo mi sie we snie, jakobym stal na brzegu rzeki.
\par 18 A oto z rzeki wychodzilo siedem krów tlustych na ciele, i pieknych na wejrzeniu, a pasly sie na lace.
\par 19 Oto, zas wychodzilo siedem krów innych za nimi, nedznych i szpetnych na wejrzeniu, i chudych na ciele; nie widzialem we wszystkiej ziemi Egipskiej tak szpetnych.
\par 20 I pozarly krowy chude i szpetne siedem krów pierwszych tlustych.
\par 21 A choc sie dostaly do wnetrznosci ich, przecie nie bylo znac, ze sie dostaly do wnetrznosci ich: bo na wejrzeniu byly szpetne, jako i przedtem; i ocknalem sie.
\par 22 Widzialem zas we snie, a oto, siedem klosów wyrastalo z jednego zdzbla pelnych i pieknych.
\par 23 Oto, tez siedem klosów suchych, cienkich, i wyschlych od wiatru wschodniego, wyrastalo za nimi.
\par 24 I pozarly te klosy cienkie siedem onych klosów pieknych. I powiedzialem to wieszczkom; ale nie bylo, kto by mi wylozyl.
\par 25 Tedy rzekl Józef do Faraona: Sen Faraonów jedenze jest: co Bóg uczyni, oznajmil Faraonowi.
\par 26 Siedem krów pieknych jest siedem lat, a siedem klosów cudnych, jest tez siedem lat; sen to jeden.
\par 27 Siedem zas krów chudych i szpetnych, które wychodzily za niemi, jest siedem lat, a siedem klosów czczych, i wyschlych od wiatru wschodniego, bedzie siedem lat glodnych.
\par 28 A toc jest, com powiedzial Faraonowi; co Bóg bedzie czynil, ukazal Faraonowi.
\par 29 Oto, siedem lat nadejdzie bardzo obfitych we wszystkiej ziemi Egipskiej.
\par 30 A po nich nastapi siedem lat glodu, i w zapomnienie przyjdzie wszystka ona obfitosc w ziemi Egipskiej, i wytrawi glód ziemie.
\par 31 Tak, ze nie bedzie znac w ziemi obfitosci onej dla glodu przyszlego: albowiem ciezki bedzie bardzo.
\par 32 A iz sie po dwa kroc snil sen Faraonowi, znaczy, ze to pewna rzecz od Boga, i pospiesza Bóg wykonac ja.
\par 33 Przetoz teraz niech znajdzie Farao meza rozumnego i madrego, a przelozy go nad ziemia Egipska.
\par 34 Niech tak uczyni Farao, a postanowi urzedniki nad ziemia i zbierze piata czesc urodzajów ziemi Egipskiej przez te siedem lat obfitych.
\par 35 I niech zbieraja wszelaka zywnosc lat dobrych nastepujacych, i zgromadzaja zboza pod reke Faraonowa, i zywnosc w miesciech niech chowaja.
\par 36 A bedzie ona zywnosc na wychowanie ziemi na siedem lat glodu, które beda w ziemi Egipskiej aby nie niszczala ziemia od glodu.
\par 37 I podobalo sie to Faraonowi, i wszystkim slugom jego.
\par 38 I rzekl Farao do slug swoich: Izaz znajdziemy podobnego mezowi temu, w którym by byl Duch Bozy?
\par 39 Zatem rzekl Farao do Józefa: Poniewaz ci oznajmil Bóg to wszystko, nie masz zadnego tak rozumnego i madrego jako ty.
\par 40 Ty bedziesz nad domem moim, a wedlug rozkazania ust twoich sprawowac sie bedzie wszystek lud mój: tylko stolica wiekszy nad cie bede.
\par 41 Nad to rzekl Farao do Józefa: Oto, postanowilem cie nad wszystka ziemia Egipska.
\par 42 Zdjal tedy Farao pierscien swój z reki swej, i dal go na reke Józefowa; oblekl go tez w szate bisiorowa, i wlozyl lancuch zloty na szyje jego.
\par 43 I kazal go wozic na wtórym wozie swoim, a wolano przed nim: Klaniajcie sie. I przelozyl go nad wszystka ziemia Egipska.
\par 44 Zatem rzekl Farao do Józefa: Jam jest Farao, a bez twego pozwolenia nie podniesie zaden ani reki, ani nogi swej, we wszystkiej ziemi Egipskiej.
\par 45 I nazwal Farao imie Józefowe, Safnat Paneach, a dal mu Asenate, córke Potyfara, przelozonego Onskiego, za zone. I wyjechal Józef na ziemie Egipska.
\par 46 A bylo Józefowi trzydziesci lat, gdy stanal przed Faraonem, królem Egipskim; i wyszedlszy Józef od oblicza Faraonowego, objechal wszystke ziemie Egipska.
\par 47 Zrodzila tedy ziemia w onych siedmiu latach urodzajnych obficie.
\par 48 I zgromadzil Józef wszystke zywnosc onych siedmiu lat, która byla w ziemi Egipskiej i skladal zywnosc w miesciech; urodzaj polny kazdego miasta, który byl okolo niego, skladal w niem.
\par 49 Zaczem nagromadzil Józef zboza, jako piasku morskiego bardzo wiele, az go zaniechano liczyc; bo mu nie bylo liczby.
\par 50 A Józefowi urodzili sie dwaj synowie, pierwej niz przyszedl rok glodu, które mu urodzila Asenat, córka Potyfara, przelozonego Onskiego.
\par 51 Nazwal tedy Józef imie pierworodnego Manases, mówiac: Ze mi dal Bóg zapomniec wszelkiej pracy mojej, i wszystkiego domu ojca mego.
\par 52 A imie drugiego nazwal Efraim, mówiac: Iz mie rozmnozyl Bóg w ziemi utrapienia mego.
\par 53 Tedy sie skonczylo siedem lat obfitosci, która byla w ziemi Egipskiej.
\par 54 I poczelo siedem lat glodu nastepowac, jako byl przepowiedzial Józef. I byl glód po wszystkich krainach: ale we wszystkiej ziemi Egipskiej byl chleb.
\par 55 Jednak potem scisniona byla glodem wszystka ziemia Egipska, i wolal lud do Faraona, o chleb. I rzekl Farao wszystkim Egipczanom: Idzcie do Józefa, a co wam rzecze uczyncie.
\par 56 I byl glód po wszystkiej ziemi. Tedy otworzyl Józef wszystkie gumna, w których bylo zboze, i sprzedawal Egipczanom; bo sie byl glód zmocnil w ziemi Egipskiej.
\par 57 I ze wszystkiej ziemi przyjezdzano do Egiptu, kupowac zywnosc od Józefa; bo sie byl zmocnil glód po wszystkiej ziemi.

\chapter{42}

\par 1 A widzac Jakób, ze bylo zboze w Egipcie, rzekl do synów swoich: Czemuz sie ogladacie jeden na drugiego?
\par 2 I mówil im: Otom slyszal, ze jest zboze w Egipcie. Jedzciez tam, a kupcie nam stamtad, abysmy zywi byli, a nie pomarli.
\par 3 Jechalo tedy dziesiec braci Józefowych kupowac zboze, do Egiptu;
\par 4 Ale Benjamina, brata Józefowego nie poslal Jakób z bracia jego, bo mówil: By snac nie przypadlo nan co zlego.
\par 5 I szli synowie Izraelowi pospolu z innymi tamze idacymi kupowac zboze; albowiem byl glód w ziemi Chananejskiej.
\par 6 A Józef byl przedniejszym rzadca w onej ziemi; onze sprzedawal zboza wszystkiemu ludowi ziemi. A gdy przyszli bracia Józefowi, klaniali mu sie twarza do ziemi.
\par 7 A ujrzawszy Józef bracia swa, poznal je; lecz stawil sie im jako obcy, i mówil do nich surowo, i rzekl do nich: Skadescie przyszli? I odpowiedzieli: Z ziemi Chananejskiej, abysmy nakupili zywnosci.
\par 8 Tedy poznal Józef bracia swa; ale go oni nie poznali.
\par 9 I wspomnial Józef na sny, które mu sie snily o nich, i rzekl im: Szpiegowiescie wy, a przyszliscie, abyscie przepatrzyli miejsca nieobronne tej ziemi.
\par 10 A oni mu odpowiedzieli: Nie tak, panie mój; ale sludzy twoi przyszli, aby nakupili zywnosci.
\par 11 Wszyscysmy synowie jednego meza; ludziesmy szczerzy, a nie sa sludzy twoi szpiegami.
\par 12 A on rzekl do nich: Nie tak, alescie nieobronne miejsca tej ziemi przyszli przepatrowac.
\par 13 I rzekli: Dwanascie nas braci bylo slug twoich, synów jednego meza w ziemi Chananejskiej; a oto, najmlodszy z ojcem naszym teraz jest w domu, a jednego juz nie masz.
\par 14 I rzekl im Józef: Toc jest com ja wam powiedzial, mówiac: Szpiegowiescie wy.
\par 15 Przez to was doswiadcze; zywie Farao, nie wynijdziecie stad, az mi tu przyjdzie brat wasz mlodszy.
\par 16 Poslijciez jednego z was, aby przywiódl brata waszego, a wy w wiezieniu bedziecie, azby byly doswiadczone slowa wasze, jestli prawda przy was; a jezli nie, zywie Farao, zescie wy szpiegowie.
\par 17 Tedy je dal pod straz do trzech dni.
\par 18 I mówil do nich Józef dnia trzeciego: Uczyncie tak, a zyc bedziecie; boc sie ja boje Boga.
\par 19 Jezliscie szczerzy, brat wasz jeden niech bedzie okowany w wiezieniu, gdziescie wy byli; a wy jedzcie i odniescie zboze, abyscie odjeli glodowi domy wasze.
\par 20 A brata waszego mlodszego przywiedzcie do mnie, a sprawdza sie slowa wasze, i nie pomrzecie. I uczynili tak.
\par 21 I mówili jeden do drugiego: Zaprawdesmy zgrzeszyli przeciwko bratu naszemu; bo widzac utrapienie duszy jego, gdy sie nam modlil, nie wysluchalismy go; dla tegoz przyszedl na nas ten klopot.
\par 22 Odpowiedzial im tedy Ruben, mówiac: Izalim wam nie mówil temi slowy: Nie grzeszcie przeciw pacholeciu? a nie usluchaliscie; otóz teraz krwi jego z rak naszych szukaja.
\par 23 A oni nie wiedzieli, zeby rozumial Józef; bo tlumacz byl miedzy nimi.
\par 24 Odwróciwszy sie tedy od nich Józef, plakal; a obróciwszy sie do nich, mówil z nimi, i wziawszy od nich Symeona, zwiazal go przed oczyma ich.
\par 25 I rozkazal Józef, aby napelniono wory ich zbozem, i wrócono pieniadze ich kazdemu do woru jego, i zeby im dano zywnosci na droge; i uczyniono tak.
\par 26 Tedy oni wlozywszy zboza swoje na osly swe, odjechali stamtad.
\par 27 I rozwiazawszy jeden z nich wór swój, aby dal obrok oslowi swemu w gospodzie, ujrzal pieniadze swoje, które byly na wierzchu w worze jego.
\par 28 I rzekl do braci swej: Wrócono mi pieniadze moje, a oto, sa w worze moim. Tedy im upadlo serce, i zdumieli sie, jeden do drugiego mówiac: Cóz nam to Bóg uczynil?
\par 29 Zatem przyszli do Jakóba, ojca swego, do ziemi Chananejskiej, i powiedzieli mu wszystko, co sie im przydalo, mówiac:
\par 30 Mówil z nami on maz, pan onej ziemi, surowo, i udal nas za szpiegi ziemi;
\par 31 A mysmy mu rzekli: Szczerzysmy, nie bylismy szpiegami;
\par 32 Dwanascie nas bylo braci synów ojca naszego; jednego juz nie masz, a mlodszy teraz jest z ojcem naszym w ziemi Chananejskiej.
\par 33 I mówil do nas maz on, pan onej ziemi: Po tem poznam, zescie szczerzy; brata waszego jednego zostawcie u mnie, a zboze dla odjecia glodowi domów waszych wezmijcie a idzcie;
\par 34 Potem przywiedzcie brata waszego mlodszego do mnie, abym poznal, zescie wy nie szpiegowie, ale szczerzy; tedy wam wróce brata waszego, a w tej ziemi handlowac bedziecie.
\par 35 I stalo sie, gdy wyprózniali wory swoje, a oto, kazdy znalazl wezel pieniedzy swych w worze swoim; a obaczywszy wezly z pieniedzmi swymi, oni i ojciec ich, polekali sie.
\par 36 I rzekl im Jakób, ojciec ich: Osierociliscie mie, Józefa nie masz, i Symeona nie masz, a Benjamina wezmiecie; na mie sie to wszystko zle zwalilo.
\par 37 I rzekl Ruben do ojca swego, mówiac: Dwóch synów moich zabij, jezlic go zas nie przywiode; daj go do reki mojej, a ja go tobie przywróce.
\par 38 Ale on rzekl: Nie pójdzie syn mój z wami, gdyz brat jego umarl, a on sam tylko zostal; a jezliby nan przypadlo co zlego na drodze, która pójdziecie, tedy doprowadzicie sedziwosc moje z zaloscia do grobu.

\chapter{43}

\par 1 A glód wielki byl w onej ziemi.
\par 2 I stalo sie, gdy strawili one zywnosc, która byli przyniesli z Egiptu, ze rzekl do nich ojciec ich: Idzcie znowu, a kupcie nam cokolwiek zywnosci.
\par 3 I rzekl do niego Judas, mówiac: Oswiadczajac oswiadczyl sie przeciwko nam ten maz mówiac: Nie ujrzycie oblicza mojego, jezli nie bedzie brat wasz z wami:
\par 4 Jezli tedy poslesz brata naszego z nami, pojedziemy i nakupiemyc zywnosci;
\par 5 Ale jezli nie poslesz, nie pojedziemy; bo on maz mówil do nas: Nie ujrzycie twarzy mojej, jezli nie bedzie brata waszego z wami.
\par 6 Tedy rzekl Izrael: Przeczzescie mi tak zle uczynili, powiedziawszy temu mezowi, ze jeszcze macie brata?
\par 7 I rzekli: Pilnie sie pytal on maz o nas, i o rodzinie naszej, mówiac: Zywze jeszcze ojciec wasz? macieli jeszcze którego brata? I odpowiedzielismy mu wedlug pytania jego; cózesmy wiedzieli, ze mial mówic: Przywiedzcie mi tu brata waszego?
\par 8 I rzekl Judas do Izraela, ojca swego: Poslij tego mlodzienca ze mna, a wstawszy pojedziemy, abysmy zyli a nie pomarli glodem, tak my, jako i ty, i dziateczki nasze.
\par 9 Ja przyrzekam zan, z reki mojej szukaj go; jezli go nie przywiode do ciebie, a nie stawie go przed toba, bedec winien grzechu po wszystkie dni;
\par 10 Bo gdybysmy byli nie omieszkali, juz bysmy sie byli dwa kroc wrócili.
\par 11 Tedy rzekl do nich Izrael, ojciec ich: Jezlize tak byc musi, uczynciez to; nabierzcie najlepszych pozytków ziemi w naczynia wasze, a zaniescie mezowi onemu w upominku: troche balsamu, i troche miodu, i rzeczy wonnych, i myrry, orzechów terebintowych, i migdalów.
\par 12 Pieniadze tez dwoje wezmijcie do rak waszych, a pieniadze przywrócone na wierzchu worów waszych odniescie w rece swoje; snac sie to omylka stalo.
\par 13 Ale i brata waszego wezmijcie, a wstawszy jedzcie znowu do meza onego;
\par 14 A Bóg Wszechmogacy niech wam da milosierdzie przed obliczem tego meza, aby wam wypuscil brata waszego drugiego i Benjamina; a ja jako osierocialy bez dziatek bede.
\par 15 Tedy wziawszy oni mezowie on podarek, i dwoje pieniadze wziawszy w rece swe, i Benjamina, wstali, i jechali do Egiptu, i staneli przed Józefem.
\par 16 A ujrzawszy Józef z nimi Benjamina, rzekl do tego, który byl sprawca domu jego: Wprowadz te meze w dom, a zabij bydle i nagotuj; bo ze mna jesc beda mezowie ci w poludnie.
\par 17 I uczynil on maz, jako mu rozkazal Józef, a wprowadzil on maz one ludzie w dom Józefów.
\par 18 Bali sie tedy mezowie oni, gdy byli wprowadzeni w dom Józefów, i mówili: Dla onychci to pieniedzy, które pierwej wlozono bylo do worów naszych, wprowadzeni tu jestesmy, aby potwarz na nas zwaliwszy targnal sie na nas, a pobral w niewola nas i osly nasze.
\par 19 A przystapiwszy do meza tego, który byl sprawca domu Józefowego, mówili do niego we drzwiach domu.
\par 20 I rzekli: Sluchaj mie, panie mój! przyjechalismy byli pierwej kupowac zywnosc.
\par 21 I stalo sie, gdysmy przyjechali do gospody, i rozwiazalismy wory nasze, oto, pieniadze kazdego byly na wierzchu woru jego, pieniadze nasze, wedlug wagi ich, któresmy zas przyniesli w rekach naszych.
\par 22 Pieniadze tez drugie przynieslismy w rekach naszych, abysmy nakupili zywnosci, a nie wiemy, kto wlozyl te pieniadze nasze do worów naszych.
\par 23 A on rzekl: Pokój wam, nie bójcie sie; Bóg wasz, i Bóg ojca waszego dal wam skarb do worów waszych; pieniadze wasze doszly mie. I wywiódl do nich Symeona.
\par 24 A przywiódlszy on maz one ludzie w dom Józefów, dal im wody, i umyli nogi swe; dal tez obrok oslom ich.
\par 25 Zatem nagotowali podarek, niz przyszedl Józef w poludnie; slyszeli bowiem, iz tam mieli jesc chleb.
\par 26 A gdy wszedl Józef w dom, przyniesli mu podarek, który mieli w rekach swych w domu onym, i klaniali mu sie az do ziemi.
\par 27 I pytal ich, jakoby sie mieli, i rzekl: Zdrówze jest ojciec wasz stary, o którymescie mi powiadali? Zywze jeszcze?
\par 28 A oni odpowiedzieli: Zdrówci sluga twój, ojciec nasz, jeszczec zyw. A schyliwszy sie, poklonili mu sie.
\par 29 Tedy podnióslszy oczy swe, ujrzal Benjamina, brata swego, syna matki swej, i rzekl: Tenze jest brat wasz mlodszy, o którymescie mi powiadali? I rzekl mu: Bóg niech ci bedzie milosciw, mily synu.
\par 30 I pokwapil sie Józef wynijsc, bo sie byly wzruszyly wnetrznosci jego ku bratu swemu, i szukal miejsca, gdzie by plakal, i wszedlszy do komory, plakal tam.
\par 31 Potem umywszy twarz swoje, wyszedl zasie, i wstrzymal sie, i rzekl: Kladzcie chleb.
\par 32 I polozono jemu osobno, i onym osobno, Egipczanom tez, którzy jedli z nim, osobno; bo nie moga jesc Egipczanie z Hebrejczykami chleba, gdyz to jest obrzydliwoscia Egipczanom.
\par 33 I usiedli przed obliczem jego, pierworodny wedlug pierworodztwa swego, a mlodszy wedlug mlodosci swej; i dziwowali sie mezowie oni patrzac jeden na drugiego.
\par 34 I biorac potrawy przed soba dawal im; a dostala sie piec kroc wieksza czesc Benjaminowi nad inne czesci onych wszystkich; i pili, a podpili sobie z nim.

\chapter{44}

\par 1 Rozkazal tedy Józef temu, który byl sprawca domu jego, mówiac: Napelnij wory mezów tych zbozem, jako moga zniesc, a wlóz pieniadze kazdego na wierzch woru jego.
\par 2 Kubek tez mój, kubek srebrny, wlóz na wierzch woru mlodszego z pieniedzmi za zboze jego; i uczynil wedlug slów Józefowych, jako mu rozkazal.
\par 3 A gdy bylo rano, mezowie oni puszczeni sa, sami i oslowie ich.
\par 4 Wyszedlszy tedy z miasta, gdy nie daleko byli, rzekl Józef do tego, który byl sprawca domu jego: Wstan, gon te meze, a dogoniwszy ich, mów do nich: Czemuscie oddali zle za dobre?
\par 5 Azaz nie ten jest kubek, z którego pija pan mój? i azaz on pewnie nie zgadnie przezen, jacyscie wy? zlescie uczynili, coscie uczynili.
\par 6 Dogoniwszy ich tedy, mówil do nich te slowa.
\par 7 Ale oni odpowiedzieli mu: Czemu mówi pan mój takowe slowa? nie daj tego Boze, aby to sludzy twoi uczynic mieli.
\par 8 Oto pieniadze, któresmy byli znalezli na wierzchu worów naszych, odnieslismy zas do ciebie z ziemi Chananejskiej; a jakoz bysmy krasc mieli z domu pana twego srebro albo zloto?
\par 9 U którego by to znaleziono z slug twoich, niechaj umrze; a my bedziemy pana mego niewolnikami.
\par 10 Tedy on rzekl: Niechze tak bedzie, jako mówicie; jednak przy którym sie znajdzie kubek, ten niech bedzie niewolnikiem, a wy bedziecie niewinnymi.
\par 11 Predko tedy kazdy zlozyl wór swój na ziemie; i rozwiazali kazdy wór swój.
\par 12 I szukal od starszego poczawszy; a u mlodszego przestal; i znalazl kubek w worze Benjaminowym.
\par 13 Tedy oni rozdarli szaty swoje, i wlozywszy brzemie kazdy z nich na osla swego, wrócili sie do miasta.
\par 14 Przyszedl tedy Judas, i bracia jego do domu Józefa, który tam jeszcze byl, i upadli przed obliczem jego na ziemie.
\par 15 I rzekl do nich Józef: Cózescie to uczynili? azascie nie wiedzieli, ze pewnie zgadnie maz taki, jakim ja jest?
\par 16 Tedy odpowiedzial Judas: Cóz odpowiemy panu memu, cóz rzeczemy? i jako sie my usprawiedliwic mamy? Bóg znalazl nieprawosc slug twoich; otosmy niewolnikami pana mego, i my, i ten, w którego reku znaleziony jest kubek.
\par 17 A on rzekl: Nie daj Boze, abym to uczynic mial! maz, w którego reku znaleziony jest kubek, ten bedzie niewolnikiem moim; a wy jedzcie w pokoju do ojca waszego.
\par 18 Zatem przystapil do niego Judas i rzekl: Sluchaj mie panie mój; niechaj przemówi prosze sluga twój które slowo w uszy pana mego, a niech sie nie zapala gniew twój na sluge twego, gdyzes ty jest jako sam Farao.
\par 19 Pan mój pytal slug swoich mówiac: Maciez ojca albo brata?
\par 20 Tedysmy odpowiedzieli panu swemu: Mamyc ojca starego, i chlopie w starosci jego splodzone male, a brat jego umarl, a zostal sam tylko po matce swej, i ojciec jego miluje go.
\par 21 Potem mówiles do nas slug swoich: Przywiedzcie go do mnie, abym go ogladal oczyma memi:
\par 22 I mówilismy do pana mego: Nie bedzie moglo chlopie opuscic ojca swego; bo gdyby opuscilo ojca swego, umarlby.
\par 23 Tedys rzekl do slug swoich: Jezli nie przyjdzie brat wasz mlodszy z wami, nie ujrzycie wiecej oblicza mojego.
\par 24 I stalo sie, gdysmy odeszli do slugi twego, ojca mojego, i powiedzielismy mu te slowa pana mego:
\par 25 Tedy rzekl ojciec nasz: Jedzcie znowu, a kupcie nam troche zywnosci.
\par 26 I powiedzielismy: Nie mozemy tam isc: lecz jezli brat nasz mlodszy bedzie z nami, tedy pojedziemy; bo inaczej nie bedziemy mogli ogladac oblicza meza onego, jezli brat nasz mlodszy nie bedzie z nami.
\par 27 I rzekl sluga twój, ojciec mój, do nas: Wy wiecie, ze dwóch synów urodzila mi zona moja;
\par 28 I wyszedl jeden ode mnie, i rzeklem: Zaiste od zwierza rozdarty jest, i nie widzialem go do tych miast;
\par 29 A wezmiecieli i tego od oblicza mego, a przypadnie nan smierc, tedy doprowadzicie sedziwosc moje z zaloscia do grobu.
\par 30 Przetoz teraz jeslibym przyszedl do slugi twego, ojca mojego, a dzieciecia by z nami nie bylo, (poniewaz dusza jego jest przywiazana do duszy jego),
\par 31 Stanie sie, skoro ujrzy, iz dzieciecia nie bedzie, ze umrze; a odprowadza sludzy twoi sedziwosc slugi twego, ojca naszego, z zaloscia do grobu.
\par 32 Bo sluga twój przyrzekl za to dziecie, gdy je bral od ojca swego, mówiac: Jezlic go zas nie przywiode, tedy bede winien grzechu przeciw ojcu memu po wszystkie dni.
\par 33 Teraz tedy niech zostanie prosze sluga twój miasto dzieciecia tego niewolnikiem pana mego, a dziecie niech idzie z bracia swoja.
\par 34 Bo jakoz ja mam wrócic sie do ojca mego, gdy tego dzieciecia ze mna nie bedzie? chybabym chcial patrzyc na zalosc, która by przyszla na ojca mego.

\chapter{45}

\par 1 Tedy sie Józef nie mógl dalej wstrzymac przed wszystkimi, którzy stali przed nim, i zawolal: Wyprowadzcie wszystkie ode mnie. I nie zostal nikt przy nim, gdy sie dal poznac Józef braci swej.
\par 2 I podniósl glos swój z placzem; co slyszeli Egipczanie, slyszal tez dom Faraonów.
\par 3 I rzekl Józef do braci swej: Jamci jest Józef; a zywze jeszcze ojciec mój? i nie mogli mu bracia jego odpowiedziec, bo sie zlekli oblicza jego.
\par 4 Tedy rzekl Józef do braci swej: Przystapcie, prosze, do mnie; i przystapili. Zatem rzekl: Jam jest Józef, brat wasz, któregoscie sprzedali do Egiptu.
\par 5 Jednak teraz nie frasujcie sie, ani trwozcie soba, zescie mie tu sprzedali; boc dla zachowania zywota waszego poslal mie Bóg przed wami.
\par 6 Bo juz dwa lata glodu bylo na ziemi, a jeszcze piec lat zostaje, których nie beda orac ani zac.
\par 7 Poslal mie tedy Bóg przed wami, abym was zachowal ostatek na ziemi, a zebym wam dodal zywnosci na oswobodzenie wielkie.
\par 8 Teraz tedy nie wyscie mie tu poslali, ale Bóg, który mie postanowil ojcem Faraonowym, i panem wszystkiego domu jego, a panujacym nad wszystka ziemia Egipska.
\par 9 Spieszciez sie, a idzcie do ojca mego, i mówcie do niego: Toc wskazuje syn twój Józef: Uczynil mie Bóg panem wszystkiego Egiptu, przyjedzze do mnie, a nie omieszkaj.
\par 10 I bedziesz mieszkal w ziemi Gosen; a bedziesz blisko mnie, ty i synowie twoi, i synowie synów twoich, i trzody twoje, i woly twoje, i wszystko, co masz.
\par 11 A bede cie tam zywil; bo jeszcze piec lat glodu bedzie, abys od niedostatku nie zginal, ty, i dom twój, i wszystko, co masz.
\par 12 A oto, oczy wasze widza, i oczy brata mego, Benjamina, ze usta moje mówia do was.
\par 13 Oznajmijcie tez ojcu memu wszystke zacnosc moje w Egipcie, i wszystko coscie widzieli; spieszcie sie tedy, a przyprowadzcie tu ojca mojego.
\par 14 Zatem padl na szyje Benjamina, brata swego, i plakal; Benjamin tez plakal na szyi jego.
\par 15 I pocalowawszy wszystke bracia swoje, plakal nad nimi; a potem rozmawiali z nim bracia jego.
\par 16 I rozgloszono te wiesc w domu Faraonowym, mówiac: Przyjechali bracia Józefowi; i podobalo sie to w oczach Faraonowych, i w oczach slug jego.
\par 17 Tedy rzekl Farao do Józefa: Powiedz braci swej: Uczyncie tak: nakladlszy brzemion na bydla wasze, idzcie; a wróccie sie do ziemi Chananejskiej;
\par 18 A wziawszy ojca waszego, i czeladz wasze, przyjedzcie do mnie; i dam wam dobre miejsce w ziemi Egipskiej, i bedziecie uzywac tlustosci ziemi.
\par 19 I rozkaz im mówiac: To uczyncie: wezmijcie sobie z ziemi Egipskiej wozów, dla dziatek waszych i dla zon waszych, a wziawszy ojca waszego przyjedzcie tu.
\par 20 A oko wasze niech nie zaluje sprzetu waszego, gdyz dobro wszystkiej ziemi Egipskiej wasze bedzie.
\par 21 Uczynili tedy tak synowie Izraelowi; i dal im Józef wozy wedlug rozkazania Faraonowego; dal im tez zywnosci na droge.
\par 22 Dal z onychze wszystkich kazdemu odmienne szaty; ale Benjaminowi dal trzy sta srebrników, i piecioro szat odmiennych.
\par 23 Ojcu tez swemu poslal te rzeczy: dziesiec oslów, niosacych z najlepszych rzeczy Egipskich, i dziesiec oslic, niosacych zboze, i chleb, i zywnosc ojcu jego na droge.
\par 24 Puscil tedy bracia swa, i odjechali, a mówil do nich: Nie wadzcie sie na drodze.
\par 25 Którzy wyjechawszy z Egiptu przyjechali do ziemi Chananejskiej, do Jakóba ojca swego.
\par 26 I oznajmili mu, mówiac: Jeszczec zyw Józef, a onci jest panem nad wszystka ziemia Egipska; i zemdlalo serce jego; bo im nie wierzyl.
\par 27 Lecz oni powiedzieli mu wszystkie slowa Józefowe, które mówil do nich. A ujrzawszy wozy, które poslal Józef, aby go na nich przywieziono, tedy ozyl duch Jakóba, ojca ich.
\par 28 I rzekl Izrael: Dosyc mam na tem, gdy jeszcze Józef, syn mój, zyje; pójde a ogladam go, pierwej niz umre.

\chapter{46}

\par 1 A tak jechal Izrael ze wszystkiem, co mial, a przyjechawszy do Beerseby, ofiarowal ofiary Bogu ojca swego Izaaka.
\par 2 I rzekl Bóg do Izraela w widzeniu nocnem, mówiac: Jakóbie, Jakóbie; a on odpowiedzial: Owom ja.
\par 3 I rzekl: Jam jest Bóg, Bóg ojca twojego; nie bój sie zstapic do Egiptu, bo cie tam w naród wielki rozmnoze.
\par 4 Ja zstapie z toba do Egiptu, i Ja cie stamtad takze zasie wywiode, a Józef polozy reke swoje na oczy twoje.
\par 5 I powstal Jakób z Beerseby; i wzieli synowie Izraelowi Jakóba ojca swego, i dziatki swe, i zony swe na wozy, które byl poslal Farao, aby go przywieziono.
\par 6 Pobrali tez bydla swe, i majetnosc swoje, której byli nabyli w ziemi Chananejskiej, i przyjechali do Egiptu, Jakób i wszystka rodzina jego z nim;
\par 7 Syny swe, i syny synów swych, córki swe i córki synów swych, i wszystko nasienie swoje prowadzil z soba do Egiptu.
\par 8 A tec sa imiona synów Izraelowych, którzy weszli do Egiptu: Jakób i synowie jego: pierworodny Jakóbów Ruben.
\par 9 A synowie Rubenowi: Henoch, i Fallu, i Hesron i Charmi.
\par 10 A synowie Symeonowi: Jemuel, i Jamyn, i Achod, i Jachyn, i Sochar, i Saul, syn niewiasty Chananejskiej.
\par 11 Synowie tez Lewiego: Gerson, Kaat, i Merary.
\par 12 A synowie Judasowi: Her, i Onan, i Sela, i Fares, i Zara; ale umarl Her i Onan w ziemi Chananejskiej. A byli synowie Faresowi: Hesron i Hamuel.
\par 13 A synowie Isaszarowi: Tola, i Fua, i Job, i Simron.
\par 14 Synowie zas Zabulonowi: Zared, i Elon, i Jaleel.
\par 15 A cic sa synowie Lii, które urodzila Jakóbowi w Padanie Syryjskim, i Dyna córka jego; wszystkich dusz synów jego, i córek jego, trzydziesci i trzy.
\par 16 A synowie Gadowi: Sefon, i Aggi, Suny, i Esebon, Ery, i Arody, i Areli.
\par 17 A synowie Aser: Jemna, i Jesua, i Isui, i Beryja, i Sera, siostra ich. A synowie Beryjego: Heber, i Melchyjel.
\par 18 Cic sa synowie Zelfy, która byl dal Laban Lii, córce swej, których ona urodzila Jakóbowi, szesnascie dusz.
\par 19 Synowie Racheli, zony Jakóbowej: Józef i Benjamin.
\par 20 Józefowi zas urodzili sie synowie w ziemi Egipskiej, które mu urodzila Asenat, córka Potyfara, ksiazecia Onskiego: Manases i Efraim.
\par 21 A synowie Benjaminowi: Bela, i Bechor, i Asbel, Gera, i Naamann, Echy i Ros, Mupim, i Chupim, i Ared.
\par 22 Cic sa synowie Racheli, którzy sie urodzili Jakóbowi; wszystkich dusz czternascie.
\par 23 A synowie Danowi: Chusym.
\par 24 Synowie tez Neftalimowi: Jachsyjel, i Gunny, i Jeser, i Selem.
\par 25 Ci sa synowie Bali, która byl dal Laban Racheli, córce swej, która je urodzila Jakóbowi; wszystkich dusz siedem.
\par 26 Wszystkie dusze, które przyszly z Jakóbem do Egiptu, co wyszly z biódr jego, okrom zon synów Jakóbowych, wszystkich dusz bylo szescdziesiat i szesc.
\par 27 A synów Józefowych, którzy mu sie urodzili w Egipcie, dusz dwie. A tak wszystkich dusz domu Jakóbowego, które weszly do Egiptu, bylo siedemdziesiat.
\par 28 I poslal przed soba Judasa do Józefa, aby mu oznajmil pierwej, nizby przyjechal do Gosen. I przyjechali do ziemi Gosen.
\par 29 A zaprzaglszy Józef wóz swój, wyjechal przeciw Izraelowi, ojcu swemu, do Gosen; a ujrzawszy go (Jakób) padl na szyje jego, i plakal na szyi jego chwile.
\par 30 Tedy rzekl Izrael do Józefa: Niechze juz umre, gdym ujrzal oblicze twoje, poniewazes ty jeszcze zyw.
\par 31 Zatem rzekl Józef do braci swej i do domu ojca swego: Pojade, a opowiem Faraonowi, i rzeke mu: Bracia moi i dom ojca mego, którzy byli w ziemi Chananejskiej, przyjechali do mnie;
\par 32 A ci mezowie sa pasterze trzód, bo sie bawili chowaniem bydla; przeto owce swoje, i woly swoje, i wszystko, co mieli, przywiedli.
\par 33 A tak gdy was przyzowie Farao, i spyta: Czem sie bawicie?
\par 34 Odpowiecie: Pasterze byli sludzy twoi od dziecinstwa naszego az dotad, i my i ojcowie nasi; a to dla tego abyscie mogli mieszkac w ziemi Gosen, bo obrzydloscia Egipczanom jest wszelki pasterz bydla.

\chapter{47}

\par 1 Tedy przyjechawszy Józef, oznajmil Faraonowi, i rzekl: Ojciec mój i bracia moi z owcami swemi, i z wolami swymi, i ze wszystkiem, co maja, przyjechali z ziemi Chananejskiej; a oto, sa w ziemi Gosen.
\par 2 A z liczby braci swej wzial pieciu mezów, i postawil je przed Faraonem.
\par 3 I rzekl Farao do braci jego: Czem sie bawicie? A oni odpowiedzieli Faraonowi: Pasterzami owiec sa sludzy twoi, i my i ojcowie nasi.
\par 4 Rzekli jeszcze do Faraona: Abysmy byli przychodniami w tej ziemi, przyszlismy; bo nie masz paszy dla bydla, które maja sludzy twoi, gdyz ciezki glód jest w ziemi Chananejskiej; a teraz niech mieszkaja, prosimy, sludzy twoi w ziemi Gosen.
\par 5 Tedy rzekl Farao do Józefa mówiac: Ojciec twój i bracia twoi przyjechali do ciebie;
\par 6 Ziemia Egipska przed toba jest: w najlepszem miejscu tej ziemi daj mieszkanie ojcu twemu i braci twojej, niech mieszkaja w ziemi Gosen; a zrozumieszli, ze sa miedzy nimi mezowie godni, tedy je uczynisz przelozonymi nad trzodami memi.
\par 7 I przywiódl Józef Jakóba, ojca swego, i postawil go przed Faraonem; a blogoslawil Jakób Faraonowi.
\par 8 Tedy rzekl Farao do Jakóba: Wiele jest dni lat zywota twego?
\par 9 I odpowiedzial Jakób Faraonowi: Dni lat pielgrzymstwa mego jest sto i trzydziesci lat; krótkie i zle byly dni lat zywota mego, i nie doszly dni lat zywota ojców moich, w których dniach oni pielgrzymowali.
\par 10 Zatem poblogoslawiwszy Jakób Faraonowi, wyszedl od oblicza Faraonowego.
\par 11 Tedy dal mieszkanie Józef ojcu swemu i braci swej, i dal im osiadlosc w ziemi Egipskiej, w najlepszem miejscu onej krainy, w ziemi Rameses, jako byl rozkazal Farao.
\par 12 I zywil Józef ojca swego i bracia swoje, i wszystek dom ojca swego chlebem az do najmniejszego.
\par 13 A chleba nie bylo po wszystkiej ziemi; bo ciezki bardzo byl glód, i utrapiona byla ziemia Egipska, i ziemia Chananejska od glodu.
\par 14 Tedy zebral Józef wszystkie pieniadze, które sie znajdowaly w ziemi Egipskiej i w ziemi Chananejskiej, za zywnosc, która kupowano; i wniósl one pieniadze Józef do skarbu Faraonowego.
\par 15 A gdy nie stalo pieniedzy w ziemi Egipskiej, i w ziemi Chananejskiej, tedy przyszli wszyscy Egipczanie do Józefa, mówiac: Daj nam chleba, i czemuz mamy umierac przed toba, gdyz nam juz nie staje pieniedzy?
\par 16 Na to odpowiedzial Józef: Dawajcie bydla wasze, a dam wam zywnosci za bydla wasze, poniewaz wam nie stalo pieniedzy.
\par 17 I przygnali bydla swe do Józefa; i dal im Józef chleba za konie, i za stada owiec, i za stada wolów, i za osly, i przechowal je chlebem za wszystkie bydla ich, onego roku.
\par 18 A gdy wyszedl rok on, przyszli do niego roku drugiego, mówiac mu: Nie zatajemy przed panem naszym, ze nam juz pieniedzy nie stalo, i stada bydel sa u pana naszego; nie zostawa nam przed panem naszym, tylko ciala nasze i role nasze.
\par 19 A czemuz umierac mamy przed oczyma twemi? i nas, i role nasze kupuj i ziemie nasze za chleb, a bedziemy, my i ziemia nasza, w niewoli u Faraona; tylko nam daj nasienia, abysmy zyli a nie pomarli, i ziemia nie spustoszala.
\par 20 A tak kupil Józef wszystke ziemie Egipska Faraonowi; bo sprzedali Egipczanie, kazdy rola swoje, gdyz sie byl wzmógl miedzy nimi glód; i dostala sie Faraonowi wszystka ziemia.
\par 21 I przeniósl lud do miast, od ostatnich granic Egiptu az do konca jego.
\par 22 Tylko ziemi kaplanskiej nie kupil; bo kaplani mieli obrok postanowiony od Faraona, i zywili sie obrokiem swym, który im dal Farao; dlategoz nie sprzedawali ziemi swej.
\par 23 I rzekl Józef do ludu: Otom was teraz poskupowal i ziemie wasze Faraonowi; otóz macie nasienie, posiejciez tedy role.
\par 24 A z urodzajów waszych bedziecie dawali piata czesc Faraonowi; cztery zasie czesci beda wam na zasianie roli, i na zywnosc wasze i tych, którzy sa w domach waszych, i na zywnosc dziatek waszych.
\par 25 Tedy odpowiedzieli: Zachowales zywot nasz; niechze znajdziemy laske w oczach pana swego, i bedziemy niewolnikami Faraonowymi.
\par 26 I postanowil to Józef za prawo az do dnia dzisiejszego w ziemi Egipskiej, aby dawana byla Faraonowi piata czesc; tylko ziemia samych kaplanów nie dostala sie Faraonowi.
\par 27 I mieszkal Izrael w ziemi Egipskiej, w ziemi Gosen, i osadziwszy sie w niej, rozrodzili sie, i rozmnozyli sie wielce.
\par 28 I zyl Jakób w ziemi Egipskiej siedemnascie lat; a bylo dni Jakóbowych, lat zywota jego, sto czterdziesci i siedem lat.
\par 29 I przyblizyly sie dni Izraelowe, aby umarl; i wezwal syna swego Józefa i rzekl do niego: Jezlim teraz znalazl laske w oczach twoich, polóz prosze reke twoje pod biodro moje, a uczyn ze mna milosierdzie i prawde; prosze nie chowaj mie w Egipcie;
\par 30 Ale gdy zasne z ojcy moimi, wyniesiesz mie z Egiptu, a pochowasz mie w grobie ich. A on rzekl: Uczynie wedlug slowa twego.
\par 31 A Jakób rzekl: Przysiazze mi: i przysiagl mu. Zatem naklonil sie Izrael ku glowom loza.

\chapter{48}

\par 1 To gdy sie stalo, dano znac Józefowi: Oto, ojciec twój zachorzal; który wziawszy dwóch synów swoich z soba, Manasesa i Efraima, jechal do niego.
\par 2 I powiedziano Jakóbowi, mówiac: Oto, syn twój Józef idzie do ciebie. A Izrael pokrzepiwszy sie, usiadl na lozu.
\par 3 I rzekl Jakób do Józefa: Bóg wszechmogacy ukazal mi sie w Luzie, w ziemi Chananejskiej, i blogoslawil mi.
\par 4 A mówil do mnie: Oto, ja rozrodze cie, i rozmnoze cie, i wywiode z ciebie wielki naród; a dam ziemie te nasieniu twemu po tobie w dziedzictwo wieczne.
\par 5 Przetoz teraz dwaj synowie twoi, którzyc sie urodzili w ziemi Egipskiej, pierwej nizem ja tu do ciebie przyszedl do Egiptu, moi sa, Efraim i Manases; jako Ruben i Symeon moi beda.
\par 6 Ale dzieci twoje, które po tych splodzisz, twoje beda; imieniem braci swojej beda zwani w osiadlosciach swych.
\par 7 A gdym sie wracal z Padan, umarla mi Rachel w ziemi Chananejskiej w drodze, gdym jeszcze byl jakoby na mile od Efraty, i pogrzeblem ja tam przy drodze ku Efracie; a toc jest Betlehem.
\par 8 A ujrzawszy Izrael syny Józefowe, rzekl: Czyi to sa?
\par 9 Tedy odpowiedzial Józef ojcu swemu: Synowie to moi, które mi tu dal Bóg; a on rzekl: Przywiedz je prosze do mnie, abym im blogoslawil.
\par 10 A oczy Izraelowe ociezaly byly dla starosci, i nie mógl dojrzec: i przywiódl je do niego, które Jakób pocalowal i oblapil.
\par 11 Zatem rzekl Izrael do Józefa: Ogladac wiecej oblicza twego nie spodziewalem sie, a oto, dal mi Bóg widziec i nasienie twoje.
\par 12 Tedy Józef odwiódl je od lona jego, i poklonil sie obliczem swem az do ziemi.
\par 13 A wziawszy Józef obydwu, postawil Efraima po prawej rece swojej, a po lewej Izraelowej; a Manasesa po lewej rece swojej a po prawej Izraelowej, i przywiódl je do niego.
\par 14 A wyciagnawszy Izrael prawice swoje, wlozyl ja na glowe Efraima, który byl mlodszy, lewice zas swoje na glowe Manasesa, umyslnie przelozywszy rece swoje, choc Mananses byl pierworodny.
\par 15 I blogoslawil Józefowi, mówia: Bóg, przed którego obliczem chodzili ojcowie moi, Abraham i Izaak, Bóg który mie zywil od mlodosci mojej az do dnia tego;
\par 16 Aniol, który mie wyrwal ze wszystkiego zlego, niech blogoslawi dzieciom tym, a niech beda nazywani od imienia mego, i od imienia ojców moich, Abrahama i Izaaka, a jako ryby niech sie rozmnoza na ziemi.
\par 17 A obaczywszy Józef, iz wlozyl ojciec jego reke prawa swoje na glowe Efraimowe, nie milo mu bylo; i ujal reke ojca swego, aby ja przeniósl z glowy Efraimowej na glowe Manasesowe.
\par 18 I rzekl Józef do ojca swego: Nie tak, ojcze mój; albowiem ten jest pierworodny, wlózze prawice swoje na glowe jego.
\par 19 Ale sie zbranial ojciec jego, i rzekl: Wiemci synu mily, wiem; i tenci sie stanie w lud wielki, tenci tez urosnie; a wszakze brat jego mlodszy urosnie naden, a z nasienia jego wyjdzie mnóstwo narodów.
\par 20 Blogoslawil im tedy dnia onego, mówiac: Przez cie bedzie blogoslawil Izrael, mówiac: Niech cie wystawi Bóg jako Efraima, i jako Manasesa; a tak przelozyl Efraima nad Manasesa.
\par 21 Potem rzekl Izrael do Józefa: Oto, ja umieram, a Bóg bedzie z wami, i przywróci was do ziemi ojców waszych.
\par 22 Oto, ja dawam ci czesc jedne mimo bracia twoje, którejm nabyl z reki Amorejczyków mieczem moim, i lukiem moim.

\chapter{49}

\par 1 Wezwal tedy Jakób synów swoich i rzekl: Zbierzcie sie, a oznajmie wam, co ma przyjsc na was w ostatnie dni.
\par 2 Zbierzcie sie, i sluchajcie synowie Jakóbowi, a sluchajcie Izraela, ojca waszego.
\par 3 Ruben pierworodny mój, tys moc moja, i poczatek sily mojej, zacny dostojenstwem, i zacny mestwem.
\par 4 Splyniesz jako woda: nie bedziesz zacnym, bos wstapil na loze ojca twego, i splugawiles loze moje, i zginelo dostojenstwo twoje.
\par 5 Symeon i Lewi, bracia, naczynia nieprawosci miecze ich.
\par 6 W rade ich niechaj nie wchodzi dusza moja, a z zgromadzeniem ich niech sie nie jednoczy slawa moja; bo w zapalczywosci swej zabili meza, a w swej woli wywrócili mur.
\par 7 Przekleta zapalczywosc ich, iz uporna, i gniew ich, iz zatwardzialy. Rozdziele je w Jakóbie, a rozprosze je w Izraelu.
\par 8 Juda, tys jest, ciebie chwalic beda bracia twoi; reka twoja bedzie na szyi nieprzyjaciól twoich; klaniac sie tobie beda synowie ojca twego.
\par 9 Szczenie lwie Juda, od lupu, synu mój, wróciles sie; sklonil sie i polozyl sie jako lew, i jako lwica, a któz go obudzi?
\par 10 Nie bedzie odjete sceptrum od Judy, ani Zakonodawca od nóg jego, az przyjdzie Szylo, i jemu bedzie oddane posluszenstwo narodów.
\par 11 Uwiaze u winnej macicy osle swe, a u wybornej macicy winnej oslatko oslicy swej; omyje w winie szate swoje, a we krwi jagód winnych odzienie swoje
\par 12 Czerwiensze oczy jego nad wino, a bielsze zeby jego nad mleko.
\par 13 Zabulon na brzegu morskim mieszkac bedzie, i przy porcie okretów, a granice jego az do Sydonu.
\par 14 Isaszar jako osiel koscisty, lezacy miedzy dwoma brzemiony.
\par 15 Upatrzyl pokój, ze jest dobry, i ziemie, ze piekna, nachylil ramie swe ku noszeniu, dla tegoz bedzie hold dawal.
\par 16 Dan sadzic bedzie lud swój, jako jedno z pokolen Izraelskich.
\par 17 Dan bedzie wezem na drodze, zmija na sciezce, kasajac piety konskie, ze spadnie nazad jezdziec jego.
\par 18 Zbawienia twego oczekiwam, Panie!
\par 19 Gad od wojska zwyciezony bedzie; ale i on potem zwyciezy.
\par 20 Z Asera tlusty chleb jego, a on wyda rozkoszy królewskie.
\par 21 Neftali jako lani wypuszczona, mówiac piekne slowa.
\par 22 Latorosl plodna Józef, latorosl wyrastajaca nad zródlem, a latorosli jego rozchodza sie po murze.
\par 23 Acz gorzkoscia napelnili go, i strzelali nan, a nienawidzili go strzelcy.
\par 24 Jednak zostal potezny luk jego, a zmocnily sie ramiona rak jego, w rekach mocnego Boga Jakóbowego, stad sie stal pasterzem i opoka Izraelowa.
\par 25 Od Boga ojca twego, który cie wspomógl, i od Wszechmogacego, któryc blogoslawil blogoslawienstwy niebieskiemi z wysoka, i blogoslawienstwy przepasci lezacej gleboko, i blogoslawienstwy piersi i zywota.
\par 26 Blogoslawienstwa ojca twego mocniejsze beda nad blogoslawienstwa przodków moich, az do granic pagórków wiecznych; beda nad glowa Józefowa, i nad wierzchem glowy odlaczonego miedzy bracia swa.
\par 27 Benjamin jako wilk porywajacy, rano jesc bedzie lup, a wieczór bedzie dzielil korzysc.
\par 28 Tec wszystkie sa dwanascie pokolenia Izraelskie, i to, co im powiedzial ojciec ich, i blogoslawil im; kazdemu wedlug blogoslawienstwa jego blogoslawil im.
\par 29 A rozkazal im, i rzekl do nich: Ja bede przylaczon do ludu mego; pogrzebciez mie z ojcy moimi w jaskini, która jest na polu Efrona Hetejczyka;
\par 30 W jaskini, która jest na polu Machpela, która jest na przeciwko Mamre w ziemi Chananejskiej, która kupil Abraham z rola od Efrona Hetejczyka, w osiadlosc grobu.
\par 31 Tam pogrzebiono Abrahama, i Sare, zone jego; tam pogrzebiono Izaaka, i Rebeke zone jego; tamem tez pogrzebal Lije.
\par 32 A kupiono te rola i jaskinia, która na niej, od synów Hetowych.
\par 33 Tedy przestawszy Jakób mówic do synów swoich, zlozyl nogi swe na loze i umarl, i przylaczon jest do ludu swego.

\chapter{50}

\par 1 Zatem upadl Józef na twarz ojca swego, i plakal nad nim, a calowal go.
\par 2 I rozkazal Józef slugom swym lekarzom, aby wonnemi masciami namazali ojca jego; i namazali wonnemi masciami lekarze Izraela.
\par 3 A gdy sie mazania jego wypelnilo czterdziesci dni, (bo sie tak wypelniaja dni tych, którzy wonnemi masciami mazani bywaja), tedy go plakali Egipczanie przez siedemdziesiat dni.
\par 4 A po wyjsciu dni zaloby jego rzekl Józef do slug Faraonowych, mówiac: Jezlim teraz znalazl laske w oczach waszych, powiedzcie prosze Faraonowi, mówiac:
\par 5 Ojciec mój poprzysiagl mie mówiac: Oto, ja umieram; w grobie moim, którym sobie wykopal w ziemi Chananejskiej, tam mie pogrzebiesz; a teraz niech jade, prosze, i pogrzebie ojca mego, i zas sie wróce.
\par 6 Tedy rzekl Farao: Jedz a pogrzeb ojca twego, jako cie poprzysiagl.
\par 7 Jechal tedy Józef, aby pogrzebal ojca swego; jechali tez z nim wszyscy sludzy Faraonowi, takze starsi domu jego, i wszyscy starsi ziemi Egipskiej;
\par 8 I wszystek dom Józefów, i bracia jego, i dom ojca jego; tylko dziatki swoje, i owce swoje, i woly swoje zostawili w ziemi Gosen.
\par 9 Szly tez z nimi i wozy, i jezdni; a byl poczet bardzo wielki.
\par 10 I przyjechali az na pole Atad, które jest przy brodzie Jordanskim, i plakali tam placzem wielkim i bardzo ciezkim; i obchodzil Józef po ojcu swym zalobe przez siedem dni.
\par 11 A ujrzawszy obywatele ziemi Chananejskiej zalobe one na polu Atad, mówili: Zaloba to ciezka Egipczanów; przetoz nazwano imie miejsca onego Abel Micraim, które jest przy brodzie Jordanskim.
\par 12 Uczynili tedy z nim synowie jego, jak im byl rozkazal.
\par 13 I zawiezli go synowie jego do ziemi Chananejskiej, i pogrzebli go w jaskini na polu Machpela, która Abraham byl kupil z rola na osiadlosc grobu, od Efrona Hetejczyka, przeciwko Mamre.
\par 14 Zatem sie wrócil Józef do Egiptu z bracia swa, i ze wszystkimi, którzy jezdzili z nim na pogrzeb ojca jego, odprawiwszy pogrzeb ojca swego.
\par 15 A widzac bracia Józefowi, ze umarl ojciec ich, mówili: Podobno bedzie nas mial w nienawisci Józef, i sowicie odda nam wszystko zle, któresmy mu uczynili.
\par 16 Wskazali tedy do Józefa, mówiac: Ojciec twój rozkazal, pierwej niz umarl, mówiac:
\par 17 Tak powiedzcie Józefowi: Prosze, odpusc teraz przestepstwo braci twej, i grzech ich, zec zlosc wyrzadzili; prosze odpusc teraz wystepek slugom Boga ojca twego. I plakal Józef, gdy to mówili do niego.
\par 18 I przystapili bracia jego, a upadlszy przed nim, mówili: Otosmy slugami twoimi.
\par 19 I rzekl do nich Józef: Nie bójcie sie: bo azazem ja wam za Boga?
\par 20 Wyscie zle myslili przeciwko mnie, ale Bóg obrócil to w dobre, chcac uczynic to, co sie dzis dzieje, aby zachowal tak wielki lud.
\par 21 A przetoz nie bójcie sie, ja zywic bede was i dziatki wasze; a tak cieszyl je, i mówil z nimi lagodnie.
\par 22 I mieszkal Józef w Egipcie, sam i dom ojca jego, a zyl Józef sto i dziesiec lat.
\par 23 I ogladal Józef syny Efraimowe az do trzeciego pokolenia. Synowie tez Machyra, syna Manasesowego, porodzili sie na kolanach Józefowych.
\par 24 I rzekl Józef do braci swej: Ja umre, ale Bóg zapewnie nawiedzi was, i wyprowadzi was z ziemi tej do ziemi, o która przysiagl Abrahamowi, Izaakowi i Jakóbowi.
\par 25 I poprzysiagl Józef syny Izraelowe, mówiac: Gdy was nawiedzi Pan Bóg, wyniescie tez kosci moje stad.
\par 26 I umarl Józef, majac sto i dziesiec lat; którego namazawszy wonnemi masciami, wlozono do trumny w Egipcie.


\end{document}