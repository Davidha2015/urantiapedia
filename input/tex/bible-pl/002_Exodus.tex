\begin{document}

\title{Wyjścia}


\chapter{1}

\par 1 Tec sa imiona synów Izraelskich, którzy weszli do Egiptu z Jakóbem; kazdy z domem swym weszli:
\par 2 Ruben, Symeon, Lewi, i Judas.
\par 3 Isaszar, Zabulon, i Benjamin.
\par 4 Dan, i Neftali, Gad, i Aser.
\par 5 A bylo wszystkich dusz, które wyszly z biódr Jakóbowych, siedmdziesiat dusz; a Józef byl przedtem w Egipcie.
\par 6 I umarl Józef, i wszyscy bracia jego, i wszystek on rodzaj.
\par 7 A synowie Izraelscy rozrodzili sie, i rozplodzili sie, i rozmnozyli sie, i zmocnili sie bardzo wielce, a napelniona jest ziemia nimi.
\par 8 Miedzy tem powstal król nowy nad Egiptem, który nie znal Józefa;
\par 9 I rzekl do ludu swego: Oto lud synów Izraelskich wielki, i mozniejszy nad nas.
\par 10 Przetoz madrze sobie pocznijmy z nimi, by sie snac nie rozmnozyl, a jezliby przypadla wojna, aby sie nie przylaczyl i on do nieprzyjaciól naszych, i nie walczyl przeciwko nam, i nie uszedl z ziemi.
\par 11 A tak ustawili nad nimi poborce, aby go dreczyli ciezarami swemi; i zbudowal lud Izraelski miasta skladu Faraonowi: Pytom i Rameses.
\par 12 Ale im wiecej go trapili, tem wiecej sie rozmnazal, i tem wiecej rósl, tak, iz scisnieni byli dla synów Izraelskich.
\par 13 I podbili Egipczanie syny Izraelskie w niewola ciezka.
\par 14 I przykrzyli im zywot ich robota ciezka okolo gliny, i okolo cegiel, i okolo kazdej roboty na polu, mimo wszelaka robote swa, do której ich uzywali bez litosci.
\par 15 I rozkazal król Egipski babom Hebrejskim, z których imie jednej Zefora, a imie drugiej Fua;
\par 16 A rzekl: Gdy bedziecie babic niewiastom Hebrejskim, a ujrzycie ze rodza, bylliby syn, zabijciez go, a jezli córka, niech zywa zostanie.
\par 17 Ale baby one baly sie Boga, i nie czynily, jako im rozkazal król Egipski, ale zywo zachowywaly chlopiatka.
\par 18 Zaczem wezwawszy król Egipski onych bab, mówil do nich: Czemuscie to uczynily, zescie zywo zachowaly chlopiatka?
\par 19 I odpowiedzialy baby Faraonowi: Iz nie sa jako niewiasty Egipskie, niewiasty Hebrejskie; bo sa duze, pierwej niz przyjdzie do nich baba, rodza.
\par 20 I czynil dobrze Bóg onym babom; i krzewil sie lud, i zmocnili sie bardzo.
\par 21 I stalo sie, przeto ze sie baly one baby Boga, pobudowal im domy.
\par 22 Tedy rozkazal Farao wszystkiemu ludowi swemu, mówiac: Kazdego syna, który sie urodzi, w rzeke go wrzuccie, a kazda córke zywo zachowajcie.

\chapter{2}

\par 1 I wyszedl maz niektóry z domu Lewiego; który pojal córke z pokolenia Lewiego.
\par 2 I poczela ona niewiasta, a porodzila syna; a widzac go, ze byl nadobny, kryla go przez trzy miesiace.
\par 3 Ale gdy go nie mogla dluzej zataic, wziela plecionke z sitowia i obmazala ja klejem i smola; a wlozywszy w nie ono dziecie, polozyla je miedzy rogóz na brzegu rzeki.
\par 4 A stala siostra jego z daleka, aby wiedziala, co sie z nim dziac bedzie.
\par 5 Wtem wyszla córka Faraonowa, aby sie kapala w rzece, a panny jej przechadzaly sie po brzegu rzeki; i ujrzala plecionke w rogozy, i poslala sluzebnice swa, aby ja wziela.
\par 6 A otworzywszy ujrzala dziecie, a ono chlopiatko plakalo; a uzaliwszy sie go, rzekla: Z dziatek Hebrejskich jest ten.
\par 7 I rzekla siostra jego do córki Faraonowej: Chceszze, pójde, i zawolam ci niewiasty, mamki Hebrejskiej, która byc odchowala to dziecie?
\par 8 I rzekla jej córka Faraonowa: Idz. Tedy poszla ona dzieweczka, i zawolala matki onegoz dzieciecia.
\par 9 Do której rzekla córka Faraonowa: Wezmij to dziecie, i chowaj mi je, a ja tobie dam zaplate twoje; i wziawszy niewiasta dziecie, chowala je.
\par 10 A gdy podroslo ono dziecie, przywiodla je do córki Faraonowej, i bylo jej za syna; a nazwala imie jego Mojzesz, bo mówila: Zem z wody wyciagnela go.
\par 11 I stalo sie za onych dni, gdy urósl Mojzesz, ze wyszedl do braci swej, i widzial ciezary ich; obaczyl tez meza Egipczanina, który bil jednego Hebrajczyka z braci jego.
\par 12 A obejrzawszy sie tam i sam, gdy widzial, ze nikogo nie masz, zabil Egipczanina, i zagrzebl go w piasek.
\par 13 A wyszedlszy zas dnia wtórego, ujrzal, a oto, dwaj mezowie Hebrejscy wadzili sie; i rzekl onemu, który krzywde czynil:
\par 14 Czemuz bijesz blizniego swego? Ale mu odpowiedzial: Któz cie postanowil ksiazeciem i sedzia nad nami? albo mie ty myslisz zabic, jakos zabil Egipczanina? i zlakl sie Mojzesz i rzekl:
\par 15 Pewnie sie ta rzecz wyjawila. Uslyszal tedy Farao te rzecz, i szukal zabic Mojzesza. Lecz Mojzesz uciekl od twarzy Faraonowej, i mieszkal w ziemi Madyjanskiej; a przyszedlszy tam siedzial u studni.
\par 16 A kaplan Madyjanski mial siedem córek, które wyszedlszy czerpaly wode, i nalewaly do koryt, aby napoily trzode ojca swego;
\par 17 A przyszedlszy pasterze odganiali je. Tedy wstawszy Mojzesz obronil ich, i napoil bydlo ich.
\par 18 A gdy sie wrócily do Raguela, ojca swego, rzekl: Czemuscie dzis tak predko przyszly?
\par 19 I odpowiedzialy: Maz Egipski obronil nas od reki pasterzów; nadto czerpajac naczerpal nam i wody i napoil trzody.
\par 20 Zatem rzekl do córek swych: A gdziez ten jest? Czemuscie opuscily czlowieka tego? Zawolajcie go, aby jadl chleb.
\par 21 I przyzwolil Mojzesz mieszkac z onym mezem, który dal Zefore, córke swoje, Mojzeszowi.
\par 22 I urodzila syna, a nazwal imie jego Gerson, bo mówil: Bylem przychodniem w ziemi cudzej.
\par 23 I stalo sie po niemalym czasie, ze umarl król Egipski; i wzdychali, i wolali synowie Izraelscy dla niewoli; a wstapilo wolanie ich do Boga przed niewola.
\par 24 I uslyszal Bóg wolanie ich; i wspomnial Bóg na przymierze swoje z Abrahamem, z Izaakiem, i z Jakóbem.
\par 25 I wejrzal Bóg na syny Izraelskie, i poznal Bóg.

\chapter{3}

\par 1 A Mojzesz pasl trzode Jetra, swiekra swego, kaplana Madyjanskiego; i zagnal trzode na puszcza, a przyszedl do góry Bozej, do Horeb,
\par 2 I ukazal mu sie Aniol Panski w plomieniu ognistym, z posrodku krza; i widzial, a oto, kierz gorzal ogniem, a on kierz nie zgorzal.
\par 3 Tedy rzekl Mojzesz: Pójde teraz, a ogladam to widzenie wielkie, czemu nie zgore ten kierz.
\par 4 A widzac Pan, iz szedl patrzac, zawolal nan Bóg z posrodku onego krza, mówiac: Mojzeszu, Mojzeszu! A on odpowiedzial: Otom ja.
\par 5 Tedy rzekl: Nie przystepuj sam; zzuj buty twe z nóg twoich: albowiem miejsce, na którem ty stoisz, ziemia swieta jest.
\par 6 Zatem rzekl: Jam jest Bóg ojca twego, Bóg Abrahamów, Bóg Izaaków, i Bóg Jakóbów; i zakryl Mojzesz oblicze swe, bo sie bal patrzac na Boga.
\par 7 I rzekl Pan: Widzac widzialem utrapienie ludu mojego, który jest w Egipcie, a wolanie ich slyszalem przed przystawy ich; bom doznal bolesci jego.
\par 8 Przetoz zstapilem, abym go wybawil z reki Egipskiej i wywiódl go z ziemi tej do ziemi dobrej i przestronnej, do ziemi oplywajacej mlekiem i miodem, na miejsce Chananejczyka, i Hetejczyka, i Amorejczyka, i Ferezejczyka, i Hewejczyka, i Jebuzejczyka.
\par 9 A teraz oto krzyk synów Izraelskich przyszedl przed mie: i widzialem tez ich ucisk, którym je Egipczanie uciskaja.
\par 10 Przetoz teraz, pójdz, a posle cie do Faraona, abys wywiódl lud mój, syny Izraelskie z Egiptu.
\par 11 I rzekl Mojzesz do Boga: Któzem ja, abym szedl do Faraona, a wywiódl syny Izraelskie z Egiptu?
\par 12 I rzekl Bóg: Oto, bede z toba, a to miej na znak, zem cie ja poslal: Gdy wywiedziesz ten lud z Egiptu, bedziecie sluzyc Bogu na tej górze.
\par 13 I rzekl Mojzesz do Boga: Oto, ja pójde do synów Izraelskich, i rzeke im: Bóg ojców waszych poslal mie do was: Jezli mi rzeka: Które jest imie jego? Cóz im odpowiem?
\par 14 Tedy rzekl Bóg do Mojzesza: Bede który Bede. I rzekl: Tak powiesz synom Izraelskim: Bede poslal mie do was.
\par 15 I mówil jeszcze Bóg do Mojzesza: Tak rzeczesz do synów Izraelskich: Pan, Bóg ojców waszych, Bóg Abrahamów, Bóg Izaaków, i Bóg Jakóbów poslal mie do was; toc imie moje na wieki, i to pamietne moje od narodu do narodu;
\par 16 Idzze, a zgromadz starsze Izraelskie, i mów do nich: Pan, Bóg ojców waszych, ukazal mi sie, Bóg Abrahamów, Izaaków i Jakóbów, mówiac: Wspominajac wspomnialem na was, i widzialem, co sie wam dzialo w Egipcie.
\par 17 I rzeklem: Wyprowadze was z utrapienia Egipskiego do ziemi Chananejczyka, i Hetejczyka, i Amorejczyka, i Ferezejczyka, i Hewejczyka, i Jebuzejczyka, do ziemi oplywajacej mlekiem i miodem.
\par 18 Tedy usluchaja glosu twego, a pójdziesz ty i starsi Izraelscy do króla Egipskiego, i rzeczecie do niego: Pan, Bóg Hebrajczyków, zabiezal nam; przetoz teraz niech pójdziemy prosze w droge trzech dni na puszcza, abysmy ofiarowali Panu, Bogu naszemu.
\par 19 Ale ja wiem, ze wam nie pozwoli król Egipski odejsc, jedno przez mozna reke.
\par 20 A tak wyciagne reke moje, i uderze Egipt wszystkiemi cudami mojemi, które bede czynil w posrodku jego: a potem wypusci was.
\par 21 I dam laske ludowi temu w oczach Egipczanów: i stanie sie, gdy wychodzic bedziecie, ze nie wynijdziecie prózni.
\par 22 Ale wypozyczy niewiasta u sasiadki swojej, i u gospodyni domu swego, naczynia srebrnego, i naczynia zlotego, i szat; i wlozycie je na syny wasze, i na córki wasze, i zlupicie Egipt.

\chapter{4}

\par 1 Potem odpowiadajac Mojzesz, rzekl: Ale oto, nie uwierza mi, i nie usluchaja glosu mego, bo rzeka: Nie ukazal sie tobie Pan.
\par 2 I rzekl mu Pan: Cóz jest w rece twojej? i odpowiedzial: Laska.
\par 3 I rzekl: Porzuc ja na ziemie; i porzucil ja na ziemie, a obrócila sie w weza, i uciekal Mojzesz przed nim.
\par 4 I rzekl Pan do Mojzesza: Wyciagnij reke twoje, a ujmij go za ogon; i wyciagnal reke swoje, i ujal go, i obrócil sie w laske w rece jego.
\par 5 Abyc uwierzyli, iz ci sie ukazal Pan, Bóg ojców ich, Bóg Abrahamów, Bóg Izaaków, i Bóg Jakóbów.
\par 6 I rzekl mu Pan jeszcze: Wlóz teraz reke twoje w zanadrza twoje; i wlozyl reke swoje w zanadrza swoje; i wyjal ja, a oto, reka jego byla tredowata jako snieg.
\par 7 I rzekl: Wlóz znowu reke twoje w zanadrza twe; i wlozyl znowu reke swoje w zanadrza swe; a gdy ja wyjal z zanadrza swego, a oto, stala sie znowu jako inne cialo jego.
\par 8 I stanie sie, jezlic nie uwierza i nie usluchaja glosu znaku pierwszego, tedy uwierza glosowi znaku posledniego.
\par 9 I stanie sie, jezli nie uwierza ani tym dwom znakom, i nie usluchaja glosu twego, wezmiesz wody rzecznej, i wylejesz ja na ziemie; tedy sie przemieni woda ona, która wezmiesz z rzeki, a obróci sie w krew na ziemi.
\par 10 I rzekl Mojzesz do Pana. Prosze Panie, nie jestem ja mezem wymownym ani przedtem ani odtad, jakos mówil do slugi twego: bom ciezkich ust i ciezkiego jezyka.
\par 11 A Pan mu rzekl: Któz uczynil usta czlowiekowi? albo kto uczynil niemego, albo gluchego, albo widzacego, albo slepego, izaz nie Ja Pan?
\par 12 Idzze teraz, a Ja bede z usty twojemi, i naucze cie, co bys mial mówic.
\par 13 I rzekl Mojzesz: Sluchaj Panie, poslij prosze tego, kogo poslac masz.
\par 14 I zapalil sie gniew Panski na Mojzesza, i rzekl: Azaz nie wiem, iz Aaron, brat twój Lewita wymownym jest? a oto i ten wynijdzie przeciwko tobie, i ujrzawszy cie, uraduje sie w sercu swojem.
\par 15 I bedziesz mówil do niego, i wlozysz slowa w usta jego, a Ja bede z usty twemi, i z usty jego, i naucze was, co byscie mieli czynic.
\par 16 On bedzie mówil za cie do ludu, i stanie sie, ze on bedzie tobie za usta, a ty mu bedziesz za Boga;
\par 17 Laske tez te wezmij w reke twoje, która bedziesz czynil znaki.
\par 18 Odszedl tedy Mojzesz, i wrócil sie do Jetra, swiekra swego, i mówil do niego: Pójde teraz, a wróce sie do braci mojej, którzy sa w Egipcie, a obacze, sali jeszcze zywi. A Jetro rzekl do Mojzesza: Idz w pokoju.
\par 19 I rzekl Pan do Mojzesza w ziemi Madyjanskiej: Idz, wróc sie do Egiptu; pomarli bowiem wszyscy mezowie, którzy szukali duszy twojej.
\par 20 Wziawszy tedy Mojzesz zone swa, i syny swe, wsadzil je na osla, i wrócil sie do ziemi Egipskiej; wzial tez Mojzesz laske Boza w reke swa.
\par 21 I rzekl Pan do Mojzesza: Gdy pójdziesz i wrócisz sie do Egiptu, patrzajze, abys wszystkie cuda, którem Ja podal w reke twoje, czynil przed Faraonem, a Ja zatwardze serce jego, aby nie wypuscil lud.
\par 22 I rzeczesz do Faraona: Tak mówi Pan: Syn mój, pierworodny mój jest Izrael.
\par 23 I rzeklem do ciebie: Wypusc syna mego, aby mi sluzyl; a zes go nie chcial wypuscic, oto, Ja zabije syna twojego, pierworodnego twego.
\par 24 I stalo sie w drodze, w gospodzie, ze zabiezal Pan Mojzeszowi, i chcial go zabic.
\par 25 Tedy wziawszy Zefora krzemien ostry, obrzezala nieobrzezke syna swego, i porzucila przed nogi jego, i rzekla: Zaprawde oblubiencem krwi jestes mi.
\par 26 I odszedl od niego Pan. Tedy go nazwala oblubiencem krwi, dla obrzezania.
\par 27 I rzekl Pan do Aarona: Idz przeciwko Mojzeszowi na puszcza. I szedl i zaszedl mu na górze Bozej, i pocalowal go.
\par 28 Tedy powiedzial Mojzesz Aaronowi wszystkie slowa Pana, który go poslal, i wszystkie znaki, które mu rozkazal.
\par 29 Szedlszy tedy Mojzesz z Aaronem, zgromadzili wszystkie starsze synów Izraelskich.
\par 30 I powiedzial Aaron wszystkie slowa, które mówil Pan do Mojzesza, a Mojzesz czynil znaki przed oczyma ludu.
\par 31 I uwierzyl lud, i zrozumieli, ze nawiedzil Pan syny Izraelskie, a iz wejzal na utrapienie ich; i schyliwszy sie, poklon uczynili.

\chapter{5}

\par 1 Potem tedy przyszli Mojzesz i Aaron, i mówili do Faraona: Tak mówi Pan, Bóg Izraelski: Pusc lud mój, aby mi obchodzili swieto na puszczy.
\par 2 I rzekl Farao: Któz jest Pan, zebym mial sluchac glosu Jego, i puscic Izraela? Nie znam Pana, a Izraela tez nie puszcze.
\par 3 I odpowiedzieli: Bóg Hebrajczyków zabiezal nam, pójdziemy teraz droga trzech dni na puszcza, abysmy ofiarowali Panu Bogu naszemu, by snac nie przepuscil na nas moru albo miecza.
\par 4 I rzekl do nich król Egipski: Przecz ty Mojzeszu i Aaronie odrywacie lud od robót ich? Idzciez do robót waszych.
\par 5 Rzekl tez Farao: Oto, wielki teraz jest ten lud w ziemi, a wy je odrywacie od robót ich.
\par 6 Rozkazal tedy Farao onegoz dnia przystawom nad ludem, i urzednikom jego, mówiac:
\par 7 Juz wiecej nie bedziecie dawac plew ludowi do czynienia cegly, jako przedtem; sami niech ida, i zbieraja sobie plewy.
\par 8 A tez liczbe cegiel, która czynili przedtem, wlozycie im, nie umniejszycie z niej; bo próznuja, dla tego oni wolaja, mówiac: Pójdziemy i bedziemy ofiarowac Bogu naszemu.
\par 9 Niech sie przyczyni roboty mezom tym, a niech ja wykonywaja, aby nie ufali slowom klamliwym.
\par 10 Wyszedlszy tedy przystawowie nad ludem i urzednicy jego, rzekli do ludu, mówiac: Tak mówi Farao: Ja nie bede wam dawal plew.
\par 11 Sami idzcie, zbierajcie sobie plewy, gdzie znajdziecie; bo sie najmniej nie umniejszy z roboty waszej.
\par 12 I rozbiezal sie lud po wszystkiej ziemi Egipskiej, aby zbieral sciernisko miasto plew.
\par 13 A przystawowie przynaglali, mówiac: Wykonywajcie roboty wasze, zamiar kazdodzienny, jako gdy wam dawano plewy.
\par 14 Tedy bito przystawy synów Izraelskich, które postanowili nad nimi urzednicy Faraonowi, mówiac: Przecz nie wykonywacie zamiaru swego w robieniu cegiel jako pierwej, ani wczoraj ani dzis?
\par 15 I przyszli przelozeni synów Izraelskich i wolali do Faraona, mówiac: Czemuz tak czynisz slugom twoim?
\par 16 Plew nie daja slugom twoim, a mówia: Cegle róbcie. I oto, slugi twe bija a lud twój grzeszy.
\par 17 Który rzekl: Próznujecie, próznujecie, dla tegoz mówicie: Pójdziemy, ofiarowac bedziemy Panu.
\par 18 Przetoz teraz idzcie, róbcie, a plew wam nie dadza, ale wy liczbe cegiel oddawac bedziecie.
\par 19 A widzac przelozeni synów Izraelskich, ze zle z nimi, poniewaz mówiono: Nie umniejszycie z cegiel waszych zamiaru kazdodziennego.
\par 20 Tedy oni zabiezeli Mojzeszowi i Aaronowi, którzy stali, aby sie z nimi spotkali, gdy wychodzili od Faraona.
\par 21 I rzekli do nich: Niech wejrzy Pan na was a rozsadzi, zescie nas ohydzili w oczach Faraonowych, i w oczach slug jego, i daliscie miecz w reke ich, aby nas zabili.
\par 22 I wrócil sie Mojzesz do Pana, a rzekl: Panie, czemus to zle wprowadzil na lud twój, czemus mie tu poslal?
\par 23 Bo od onego czasu, jakom wszedl do Faraona, abym mówil imieniem twojem, gorzej sie obchodzi z ludem tym; a przecies nie wybawil ludu twego.

\chapter{6}

\par 1 I rzekl Pan do Mojzesza: Teraz ujrzysz, co uczynie Faraonowi; bo w rece moznej wypusci je, i w rece silnej wypedzi je z ziemi swojej.
\par 2 Nadto mówil Bóg do Mojzesza i rzekl do niego: Jam Pan,
\par 3 Którym sie ukazal Abrahamowi, Izaakowi i Jakóbowi w tem imieniu, zem Bóg Wszechmogacy; ale w imieniu mojem, Jehowa, nie jestem poznany od nich.
\par 4 Postanowilem tez i przymierze moje z nimi, abym im dal ziemie Chananejska, ziemie pielgrzymowania ich, w której przychodniami byli.
\par 5 Jam tez uslyszal krzyk synów Izraelskich, które Egipczanie w niewola podbijaja, i wspomnialem na przymierze moje.
\par 6 A tak rzecz do synów Izraelskich: Jam Pan, a wywiode was z ciezarów Egipskich, i wyrwe was z niewoli ich, i wybawie was w ramieniu wyciagnionem, i w sadziech wielkich.
\par 7 A wezme was sobie za lud, i bede wam za Boga, i poznacie, zem Pan, Bóg wasz, który was wywodze z ciezarów Egipskich.
\par 8 A wprowadze was do ziemi, o która podnioslem reke moje, abym ja dal Abrahamowi, Izaakowi, i Jakóbowi; a dam ja wam w dziedzictwo, Ja Pan.
\par 9 I mówil tak Mojzesz do synów Izraelskich, ale nie usluchali Mojzesza dla scisnionego ducha, i dla niewoli ciezkiej.
\par 10 Rzekl tedy Pan do Mojzesza, mówiac:
\par 11 Wnijdz, mów do Faraona, króla Egipskiego, zeby wypuscil syny Izraelskie z ziemi swej.
\par 12 I rzekl Mojzesz przed Panem, mówiac: Oto, synowie Izraelscy nie usluchali mie, a jakoz mie uslucha Farao? a jam nie obrzezanych warg.
\par 13 Tedy rzekl Pan do Mojzesza i do Aarona, i dal im rozkazanie do synów Izraelskich i do Faraona, króla Egipskiego, aby wywiedli syny Izraelskie z ziemi Egipskiej.
\par 14 A cic sa przedniejsi z domów ojców ich; synowie Rubena, pierworodnego Izraelowego: Henoch i Falu, Hesron, i Charmi. Tec sa rodzaje Rubenowe.
\par 15 A synowie Symeonowi: Jamuel i Jamyn, i Ahod, i Jachyn, i Sochar, i Saul, syn niewiasty Chananejskiej. Tec sa rodzaje Symeonowe.
\par 16 Imiona zas synów Lewiego wedlug rodzajów ich: Gerson i Kaat, i Merary; a lat zywota Lewiego bylo sto trzydziesci i siedem lat.
\par 17 Synowie Gersonowi: Lobni i Semei, wedlug domów ich.
\par 18 A synowie Kaatowi: Amram, i Izaar, i Hebron, i Husyjel; a lat zywota Kaatowego bylo sto i trzydziesci i trzy lata.
\par 19 Synowie tez Merarego: Maheli, i Muzy. Tec sa domy Lewiego wedlug rodzajów ich.
\par 20 I pojal Amram Jochabede, ciotke swoje, za zone, która mu urodzila Aarona i Mojzesza; a lat zywota Amramowego bylo sto i trzydziesci i siedem lat.
\par 21 Synowie zas Izaarowi: Kore i Nefeg, i Zychry.
\par 22 A synowie Husyjelowi: Myzael, i Elisafan, i Setry.
\par 23 I pojal Aaron Elizabete, córke Aminadaba, siostre Nasonowe, sobie za zone, która mu urodzila Nadaba, i Abiu. Eleazara, i Itamara.
\par 24 A synowie Korego: Asyr, i Elkana, i Abyjazaf. Tec sa domy Korytów.
\par 25 A Eleazar, syn Aaronów, pojal jedne z córek Putyjelowych sobie za zone, która mu urodzila Fyneesa. Cic sa przedniejsi z ojców Lewitskich wedlug rodzajów ich.
\par 26 Tenci jest Aaron i Mojzesz, do których mówil Pan: Wywiedzcie syny Izraelskie z ziemi Egipskiej wedlug hufców ich.
\par 27 Cic mówili do Faraona, króla Egipskiego, aby wyprowadzili syny Izraelskie z Egiptu; toc jest ten Mojzesz i Aaron.
\par 28 I stalo sie w ten dzien, którego mówil Pan do Mojzesza w ziemi Egipskiej.
\par 29 Ze rzekl Pan do Mojzesza mówiac: Jam Pan; mów do Faraona, króla Egipskiego, wszystko, co Ja mówie do ciebie.
\par 30 I rzekl Mojzesz przed Panem: Otom ja nie obrzezanych warg, a jakoz mie uslucha Farao?

\chapter{7}

\par 1 I rzekl Pan do Mojzesza: Oto, postanowilem cie za Boga Faraonowi, a Aaron, brat twój, bedzie prorokiem twoim.
\par 2 Ty powiesz wszystko, coc rozkaze: ale Aaron, brat twój, bedzie mówil do Faraona, aby wypuscil syny Izraelskie z ziemi swej.
\par 3 A Ja zatwardze serce Faraonowe, i rozmnoze znaki moje i cuda moje w ziemi Egipskiej.
\par 4 I nie uslucha was Farao; lecz Ja wloze reke moje na Egipt, i wyprowadze wojska moje, lud mój, syny Izraelskie, z ziemi Egipskiej w sadziech wielkich.
\par 5 A poznaja Egipczanie, zem Ja Pan, gdy wyciagne reke moje na Egipt, i wywiode syny Izraelskie z posrodku ich.
\par 6 Uczynil tedy Mojzesz i Aaron, jako im przykazal Pan, tak uczynili.
\par 7 A Mojzesz mial osiemdziesiat lat, a Aaron osiemdziesiat i trzy lata, gdy mówili do Faraona.
\par 8 Rzekl tedy Pan do Mojzesza, i do Aarona, mówiac:
\par 9 Gdy wam rzecze Farao, mówiac: Uczyncie jaki cud, tedy rzeczesz do Aarona: Wezmij laske twoje, a porzuc przed Faraonem, a obróci sie w weza.
\par 10 I przyszedl Mojzesz z Aaronem do Faraona, i uczynili tak, jako rozkazal Pan; i porzucil Aaron laske swoje przed Faraonem, i przed slugami jego, która sie obrócila w weza.
\par 11 Wezwal tez Farao medrców i czarowników, i uczynili i ci czarownicy Egipscy przez czary swe takze.
\par 12 I porzucil kazdy laske swa, a obrócily sie w weze; ale pozarla laska Aaronowa laski ich.
\par 13 I zatwardzialo serce Faraonowe, i nie usluchal ich, jako powiedzial Pan.
\par 14 Zatem rzekl Pan do Mojzesza: Ociezalo serce Faraonowe; nie chce puscic ludu tego.
\par 15 Idz do Faraona rano, oto, wynijdzie nad wode, tedy staniesz przeciwko niemu nad brzegiem rzeki, a laske, która sie byla obrócila w weza, wezmiesz w reke twoje,
\par 16 I rzeczesz do niego: Pan Bóg Hebrejczyków poslal mie do ciebie, mówiac: Wypusc lud mój, aby mi sluzyli na puszczy, a oto, nie usluchales dotad.
\par 17 Przetoz tak mówi Pan: Po tem poznasz, zem ja Pan; oto, ja uderze laska, która jest w rece mojej, na wody, które sa w rzece, a obróca sie w krew.
\par 18 A ryby, które sa w rzece, pozdychaja, i zsmierdnie sie rzeka, i spracuja sie Egipczanie, szukajac dla napoju wód z rzeki.
\par 19 Tedy rzekl Pan do Mojzesza: Mów do Aarona: Wezmij laske twoje, a wyciagnij reke twa na wody Egipskie, na rzeki ich, na strugi ich, i na jeziora ich, i na wszelkie zgromadzenie wód ich; i obróca sie w krew, i bedzie krew po wszystkiej ziemi Egipskiej, tak w naczyniach drewnianych, jako w kamiennych.
\par 20 I uczynili tak Mojzesz i Aaron, jako rozkazal Pan; i podnióslszy laske uderzyl wody, które byly w rzece, przed oczyma Faraonowemi, i przed oczyma slug jego; i obrócily sie wszystkie wody, które byly w rzece, w krew.
\par 21 A ryby, które byly w rzece, pozdychaly, i zsmierdla sie rzeka, ze nie mogli Egipczanie pic wody z rzeki; i byla krew po wszystkiej ziemi Egipskiej.
\par 22 I uczynili takze czarownicy Egipscy czarami swemi; i zatwardzialo serce Faraonowe, i nie usluchal ich, jako byl powiedzial Pan.
\par 23 A odwróciwszy sie Farao, poszedl do domu swego, a nie przylozyl serca swego i do tego.
\par 24 I kopali wszyscy Egipczanie okolo rzeki, szukajac wody, aby pili; bo nie mogli pic wody z rzeki.
\par 25 I wypelnilo sie siedem dni, jako zarazil Pan rzeke.

\chapter{8}

\par 1 I rzekl Pan do Mojzesza: Wnijdz do Faraona, a mów do niego: Tak mówi Pan: Wypusc lud mój, aby mi sluzyl.
\par 2 Ale jezli ty nie bedziesz chcial puscic, oto, Ja zaraze wszystkie granice twoje zabami.
\par 3 I wyda rzeka zaby, które wyleza i wnijda do domu twego, i do komory loza twego, i na posciel twoje, i do domu slug twoich, i miedzy lud twój, i do pieców twoich, i w dzieze twoje.
\par 4 Tak na cie, jako na lud twój, i na wszystkie slugi twoje poleza zaby.
\par 5 Tedy rzekl Pan do Mojzesza: Rzecz do Aarona: Wyciagnij reke twoje z laska twa na rzeki, na strugi, i na jeziora, a wywiedz zaby na ziemie Egipska.
\par 6 Tedy wyciagnal Aaron reke swa na wody Egipskie, i wylazly zaby, które okryly ziemie Egipska.
\par 7 I uczynili takze czarownicy czarami swemi, i wywiedli zaby na ziemie Egipska.
\par 8 Zatem Farao wezwal Mojzesza i Aarona, mówiac: Módlcie sie Panu, aby oddalil zaby ode mnie, i od ludu mego; a wypuszcze lud, aby ofiarowali Panu.
\par 9 I rzekl Mojzesz do Faraona: Poczcze cie tem, a powiedz, kiedy sie modlic mam za cie, i za slugi twoje, i za lud twój, aby wytracone byly zaby od ciebie, i z domów twoich, a tylko w rzece zostaly.
\par 10 A on rzekl: Jutro. Tedy rzekl Mojzesz: Uczynie wedlug slowa twego, abys wiedzial, ze nie masz, jako Pan Bóg nasz.
\par 11 I odejda zaby od ciebie, i od domów twoich, i od slug twoich, i od ludu twego, tylko w rzece zostana.
\par 12 Wyszedl tedy Mojzesz i Aaron od Faraona. I zawolal Mojzesz do Pana, aby odjal zaby, które byl przepuscil na Faraona.
\par 13 I uczynil Pan wedlug slowa Mojzeszowego, ze wyzdychaly zaby z domów, ze wsi, i z pól.
\par 14 I zgromadzali je na kupy, i zsmierdla sie ziemia.
\par 15 A widzac Farao, ze mial wytchnienie, obciazyl serce swoje, i nie usluchal ich, jako byl powiedzial Pan.
\par 16 I rzekl Pan do Mojzesza: Mów do Aarona: Wyciagnij laske twoje, a uderz w proch ziemi, aby sie obrócil we wszy po wszystkiej ziemi Egipskiej.
\par 17 I uczynili tak; bo wyciagnal Aaron reke swoje z laska swoja, i uderzyl w proch ziemi; i byly wszy na ludziach, i na bydle; wszystek proch ziemi obrócil sie we wszy po wszystkiej ziemi Egipskiej.
\par 18 Czynili tez takze czarownicy, przez czary swoje, aby wywiedli wszy, ale nie mogli; i byly wszy na ludziach i na bydle.
\par 19 Tedy rzekli czarownicy do Faraona: Palec to Bozy jest. I zatwardzialo serce Faraonowe, i nie usluchal ich, jako powiedzial Pan.
\par 20 I rzekl Pan do Mojzesza: Wstan rano, a stan przed Faraonem, (oto, wynijdzie do wody,)i rzeczesz do niego: Tak mówi Pan: Wypusc lud mój, aby mi sluzyl;
\par 21 Bo jezli ty nie wypuscisz ludu mego, oto, Ja posylam na cie, i na slugi twe, i na lud twój, i na domy twoje rozmaite robactwo; a beda napelnione domy Egipskie rozmaitem robactwem, nadto i ziemia, na której oni sa.
\par 22 A oddziele dnia onego ziemie Gosen, w której lud mój mieszka, aby tam nie bylo rozmaitego robactwa, abys poznal, zem Ja Pan w posrodku ziemi.
\par 23 I poloze znak odkupienia miedzy ludem moim i miedzy ludem twoim; jutro bedzie znak ten.
\par 24 Tedy uczynil tak Pan. I przyszlo rozmaite robactwo ciezkie na dom Faraonów, i na dom slug jego, i na wszystke ziemie Egipska; i psowala sie ziemia od rozmaitego robactwa.
\par 25 Zatem wezwal Farao Mojzesza i Aarona, i rzekl: Idzcie, ofiarujcie Bogu waszemu w tej ziemi.
\par 26 I odpowiedzial Mojzesz: Nie godzi sie tak czynic; bo bysmy obrzydliwosc Egipska ofiarowali Panu Bogu naszemu; a gdybysmy ofiarowali obrzydliwosc Egipska przed oczyma ich, zazby nas nie ukamionowali?
\par 27 Droge trzech dni pójdziemy na puszcza, i ofiarowac bedziemy Panu Bogu naszemu, jako nam rozkaze.
\par 28 I rzekl Farao: Jac wypuszcze was, abyscie ofiarowali Panu Bogu waszemu na puszczy, wszakze daleko nie odchodzcie, i módlcie sie za mna.
\par 29 I odpowiedzial Mojzesz: Ja wychodze od ciebie, i bede sie modlil Panu, a odejdzie rozmaite robactwo od Faraona, od slug jego, i od ludu jego jutro; tylko niech wiecej Farao nie klamie, aby nie mial wypuscic ludu dla ofiarowania Panu.
\par 30 Wyszedlszy tedy Mojzesz od Faraona, modlil sie Panu.
\par 31 I uczynil Pan wedlug slowa Mojzeszowego, i oddalil ono rozmaite robactwo od Faraona, i od slug jego, i od ludu jego, a nie zostalo i jednego.
\par 32 Jednak Farao obciazyl serce swe i tym razem, a nie wypuscil ludu.

\chapter{9}

\par 1 Potem rzekl Pan do Mojzesza: Wnijdz do Faraona, a mów do niego: Tak mówi Pan, Bóg Hebrejczyków: Wypusc lud mój, aby mi sluzyl;
\par 2 Bo jezli go ty nie bedziesz chcial wypuscic, ale jeszcze zatrzymywac go bedziesz:
\par 3 Oto, reka Panska bedzie na bydle twojem, które jest na polu, na koniach, na oslach, na wielbladach, na wolach i na owcach, powietrze bardzo ciezkie.
\par 4 I uczyni Pan rozdzial miedzy trzodami Izraelskiemi, i miedzy trzodami Egipskiemi, aby nic nie zdechlo ze wszystkiego, co jest synów Izraelskich.
\par 5 I postanowil Pan czas, mówiac: Jutro uczyni Pan te rzecz na ziemi.
\par 6 I uczynil Pan te rzecz nazajutrz, ze wyzdychaly wszystkie bydla Egipskie; ale z bydla synów Izraelskich nie zdechlo ani jedno.
\par 7 I poslal Farao, a oto, nie zdechlo z bydla Izraelskiego i jedno; ale ociezalo serce Faraonowe, i nie wypuscil ludu.
\par 8 Zatem rzekl Pan do Mojzesza i do Aarona: Wezmijcie pelne garsci wasze popiolu z pieca, a niech go rozrzuci Mojzesz ku niebu przed oczyma Faraonowemi.
\par 9 I obróci sie w proch po wszystkiej ziemi Egipskiej, i bedzie na ludziach, i na bydle wrzodem czyniacym pryszczele, po wszystkiej ziemi Egipskiej.
\par 10 Wzieli tedy popiolu z pieca, i staneli przed Faraonem, i rozrzucil go Mojzesz ku niebu; i stal sie wrzodem, pryszczele czyniacym na ludziach i na bydle;
\par 11 I nie mogli czarownicy stac przed Mojzeszem dla wrzodu; bo byl wrzód na czarownikach i na wszystkich Egipczanach.
\par 12 I zatwardzil Pan serce Faraonowe, i nie usluchal ich, jako byl powiedzial Pan Mojzeszowi.
\par 13 I rzekl Pan do Mojzesza: Wstan rano, a stan przed Faraonem, i mów do niego: Tak mówi Pan, Bóg Hebrejczyków: Wypusc lud mój, aby mi sluzyl:
\par 14 Bo ta raza Ja posylam wszystkie plagi moje na serce twoje, i na slugi twoje, i na lud twój, abys wiedzial, ze nie masz mnie podobnego po wszystkiej ziemi.
\par 15 Bo teraz sciagne reke moje a uderze cie i lud twój powietrzem, i wytracony bedziesz z ziemi.
\par 16 A zaiste, dla tegom cie zachowal, abym okazal na tobie moc moje, i zeby opowiadane bylo imie moje po wszystkiej ziemi.
\par 17 Jeszczez sie ty wynosisz przeciw ludowi memu, nie chcac go wypuscic?
\par 18 Oto, Ja spuszcze o tym czasie jutro grad bardzo ciezki, jakiemu nie bylo podobnego w Egipcie ode dnia, którego jest zalozon, az do tego czasu.
\par 19 A tak poslij teraz, zgromadz bydlo twoje, i wszystko, co masz na polu; bo na kazdego czlowieka, i na bydle, które znalezione bedzie na polu, a nie bedzie zegnane w dom, spadnie na nie grad, i pozdychaja.
\par 20 Kto sie tedy ulakl slowa Panskiego z slug Faraonowych, kazal uciekac slugom swym, i z bydlem swojem do domu;
\par 21 Ale kto nie przylozyl serca swego do slowa Panskiego, ten zostawil slugi swe i bydlo swe na polu.
\par 22 I rzekl Pan do Mojzesza: Wyciagnij reke twa ku niebu, ze bedzie grad po wszystkiej ziemi Egipskiej, na ludzi, i na bydlo, i na wszelakie ziola polne w ziemi Egipskiej.
\par 23 A tak wyciagnal Mojzesz laske swa ku niebu, a Pan dal gromy i grad, i zstapil ogien na ziemie, i spuscil Pan grad na ziemie Egipska.
\par 24 I byl grad, i ogien zmieszany z gradem ciezkim bardzo, jakiemu nie byl podobny we wszystkiej ziemi Egipskiej, jako w niej mieszkac poczeto.
\par 25 I potlukl on grad po wszystkiej ziemi Egipskiej, cokolwiek bylo na polu, od czlowieka az do bydlecia; i wszystko ziele polne potlukl grad, i wszystko drzewo polne polamal;
\par 26 Tylko w ziemi Gosen, gdzie synowie Izraelscy mieszkali, nie bylo gradu.
\par 27 Poslal tedy Farao, a wezwal Mojzesza i Aarona, mówiac do nich: Zgrzeszylem i tym razem; Pan jest sprawiedliwy, ale ja i lud mój niezboznismy.
\par 28 Módlciez sie Panu, (boc dosyc jest,)aby przestaly gromy Boze i grad; a wypuszcze was, i nie bedziecie tu mieszkac dalej.
\par 29 I rzekl Mojzesz do niego: Gdy wynijde z miasta, wyciagne rece swe do Pana, a gromy ustana, i grad nie bedzie wiecej, abys wiedzial, ze Panska jest ziemia;
\par 30 Ale ty i sludzy twoi, wiem, ze sie jeszcze nie boicie oblicza Pana Boga.
\par 31 Len tedy i jeczmien potluczony jest; bo jeczmien byl niedostaly, a len podrastal.
\par 32 Pszenica jednak i zyto potluczone nie byly; bo pózne byly.
\par 33 Wyszedlszy tedy Mojzesz od Faraona z miasta, wyciagnal rece swe do Pana; i przestaly gromy i grad, a deszcz nie padal na ziemie.
\par 34 A widzac Farao, ze przestal deszcz, i grad, i gromy, przyczynil grzechu, a obciazyl serce swe, sam i sludzy jego.
\par 35 I zatwardzialo serce Faraonowe, i nie wypuscil synów Izraelskich, jako byl powiedzial Pan przez Mojzesza.

\chapter{10}

\par 1 Rzekl zatem Pan do Mojzesza: Wnijdz do Faraona; bom Ja obciazyl serce jego, i serce slug jego, abym czynil te znaki moje miedzy nimi;
\par 2 Azebys opowiadal w uszach synów twoich, i wnuków twoich, com uczynil w Egipcie, i znaki moje, którem pokazal na nich, abyscie wiedzieli, zem Ja Pan.
\par 3 Wszedl tedy Mojzesz i Aaron do Faraona, i mówili mu: Tak mówi Pan, Bóg Hebrajczyków: Dokadze nie chcesz unizyc sie przede mna? Wypusc lud mój, aby mi sluzyl.
\par 4 Bo jezli nie bedziesz chcial wypuscic ludu mego, oto ja przywiode jutro szarancze na granice twoje,
\par 5 Która okryje wierzch ziemi, ze jej widac nie bedzie, i zje ostatek pozostaly, który wam zostal po gradzie, i pogryzie kazde drzewo rodzace na polu.
\par 6 I na pelni domy twoje, i domy wszystkich slug twoich, i domy wszystkiego Egiptu, czego nie widzieli ojcowie twoi, i ojcowie ojców twoich od poczatku bytu swego na ziemi az do tego dnia. A odwróciwszy sie, wyszedl od Faraona.
\par 7 Tedy rzekli sludzy Faraonowi do niego: Dlugoz bedzie nam ten ku zgorszeniu? Wypusc te meze, aby sluzyli Panu Bogu swemu; zaz jeszcze nie wiesz, ze zniszczyl Egipt?
\par 8 I zawolano zas Mojzesza z Aaronem do Faraona, i rzekl do nich: Idzcie, sluzcie Panu Bogu waszemu; którzyz to sa, co pójda?
\par 9 I odpowiedzial Mojzesz: Z dziecmi naszemi i z starcami naszymi pójdziemy, z synami naszymi, i z córkami naszemi, z trzodami naszemi, i z bydlem naszem pójdziemy; bo swieto Panu obchodzic mamy.
\par 10 Tedy im on rzekl: Niech tak bedzie Pan z wami, i jako ja was puszcze, i dzieci wasze; patrzcie, ze cos zlego przed soba macie.
\par 11 Nie tak; ale idzcie sami mezowie, a sluzcie Panu, poniewaz wy tego szukacie. I wygnal je od siebie Farao.
\par 12 Potem Pan rzekl do Mojzesza: Wyciagnij reke twa na ziemie Egipska dla szaranczy, aby przyszla na ziemie Egipska, a pozarla wszystko ziele na ziemi, wszystko, co zostalo po gradzie.
\par 13 I wyciagnal Mojzesz laske swoje na ziemie Egipska, a Pan przywiódl wiatr wschodni na ziemie przez caly on dzien, i przez cala noc; a gdy bylo rano, wiatr wschodni przyniósl szarancze.
\par 14 I przyszla szarancza na wszystke ziemie Egipska, i przypadla na wszystkie granice Egipskie bardzo ciezka; przedtem nie bylo tej podobnej szaranczy, i po niej nie bedzie takowej.
\par 15 I okryla wierzch wszystkiej ziemi, tak, iz ziemi znac nie bylo; a pozarla wszystke trawe ziemi, i wszystek owoc drzewa, który zostal po gradzie, a nie zostalo zadnej zielonosci na drzewie i na trawie polnej we wszystkiej ziemi Egipskiej.
\par 16 Przetoz co rychlej Farao wezwal Mojzesza i Aarona, i rzekl: Zgrzeszylem przeciwko Panu Bogu waszemu, i przeciwko wam.
\par 17 A teraz odpusc prosze grzech mój aby ten raz, a módlcie sie Panu Bogu waszemu, aby oddalil ode mnie tylko te smierc.
\par 18 I wyszedlszy Mojzesz od Faraona, modlil sie Panu.
\par 19 A obróciwszy Pan wiatr zachodni mocny bardzo, porwal szarancze, i wrzucil ja do morza czerwonego, i nie zostala ani jedna szarancza we wszystkich granicach Egipskich.
\par 20 I zatwardzil Pan serce Faraonowe, i nie wypuscil synów Izraelskich.
\par 21 Tedy rzekl Pan do Mojzesza: Wyciagnij reke twa ku niebu, a beda ciemnosci po ziemi Egipskiej, i macac ich beda.
\par 22 I wyciagnal Mojzesz reke swa ku niebu, i byly ciemnosci geste po wszystkiej ziemi Egipskiej przez trzy dni.
\par 23 Nie widzial jeden drugiego, ani sie kto ruszyl z miejsca swego przez one trzy dni; lecz u wszystkich synów Izraelskich byla swiatlosc w mieszkaniach ich.
\par 24 A wezwawszy Farao Mojzesza, rzekl: Idzcie, sluzcie Panu; tylko trzody wasze, i bydla wasze niech zostana, i dzieci wasze niech ida z wami.
\par 25 I odpowiedzial Mojzesz: Owszem ty dasz do rak naszych ofiary i calopalenia, które ofiarowac bedziemy Panu, Bogu naszemu.
\par 26 Przetoz i dobytek nasz pójdzie z nami, a nie zostanie i kopyto; albowiem z tego wezmiemy do sluzby Panu, Bogu naszemu; bo my nie wiemy, czem sluzyc mamy Panu, az tam przyjdziemy.
\par 27 I zatwardzil Pan serce Faraonowe, i nie chcial ich puscic.
\par 28 I rzekl Farao do Mojzesza: Idz ode mnie, a strzez sie, abys wiecej nie widzial oblicza mego; bo dnia, którego ujrzysz oblicze moje, umrzesz.
\par 29 I odpowiedzial Mojzesz: Dobrzes powiedzial; nie ujrze wiecej oblicza twego.

\chapter{11}

\par 1 I rzekl Pan do Mojzesza: Jeszcze jedne plage przywiode na Faraona, i na Egipt, potem wypusci was stad; wypusci cale, owszem wypedzi was stad.
\par 2 Mówze teraz do uszu ludu, aby wypozyczyl kazdy od blizniego swego, i kazda u sasiadki swej naczynia srebrnego, i naczynia zlotego.
\par 3 A dal Pan laske ludowi w oczach Egipczan, i byl Mojzesz maz bardzo wielki w ziemi Egipskiej, w oczach slug Faraonowych, i w oczach ludu.
\par 4 Tedy rzekl Mojzesz: Tak mówi Pan: O pólnocy Ja pójde przez posrodek Egiptu.
\par 5 A umrze kazde pierworodne w ziemi Egipskiej, od pierworodnego Faraonowego, który mial siedziec na stolicy jego, az do pierworodnego niewolnicy, która jest przy zarnach, i kazde pierworodne z bydlat.
\par 6 A bedzie krzyk wielki po wszystkiej ziemi Egipskiej, jaki przedtem nie byl, i jaki potem nie bedzie.
\par 7 Ale u wszystkich synów Izraelskich nie ruszy jezykiem swym, ani pies, ani czlowiek, ani bydle, abyscie wiedzieli, ze Pan uczynil rozdzial miedzy Egipczany i miedzy Izraelem.
\par 8 I przyjda ci wszyscy sludzy twoi do mnie, a klaniac mi sie beda, mówiac: Wynijdz ty, i wszystek lud, który jest pod sprawa twoja; a potem wynijde. I wyszedl od Faraona z wielkim gniewem.
\par 9 I rzekl Pan do Mojzesza: Nie uslucha was Farao, abym rozmnozyl cuda moje w ziemi Egipskiej.
\par 10 Tedy Mojzesz i Aaron czynili te wszystkie cuda przed Faraonem; ale Pan zatwardzil serce Faraonowe, i nie wypuscil synów Izraelskich z ziemi swojej.

\chapter{12}

\par 1 Rzekl jeszcze Pan do Mojzesza i do Aarona w ziemi Egipskiej, mówiac:
\par 2 Miesiac ten bedzie wam poczatkiem miesiecy: pierwszy wam bedzie miedzy miesiacami w roku.
\par 3 Rzeczcie do wszystkiego zgromadzenia Izraelskiego, mówiac: Dziesiatego dnia miesiaca tego wezmie sobie kazdy baranka wedlug familii, baranka wedlug domu.
\par 4 A jezliby mniejszy byl dom nizeliby zjesc mogli baranka, tedy przybierze i sasiada swego, który jest najblizszy domu jego, wedlug liczby dusz, naliczywszy tyle osób, ileby ich zjesc moglo baranka.
\par 5 Baranka zupelnego, samca rocznego, miec bedziecie; z owiec albo z kóz wezmiecie go.
\par 6 I bedziecie go chowali az do czternastego dnia miesiaca tego; a zabije go wszystko zebranie zgromadzenia Izraelskiego miedzy dwoma wieczorami.
\par 7 I wezma ze krwi jego, i pokropia obydwa podwoje i nadproznik u domu; w którym go beda spozywac.
\par 8 I beda jesc mieso onej nocy pieczone przy ogniu, i przasniki, z zioly gorzkiemi beda go jesc.
\par 9 Nie jedzcie z niego nic surowego, ani warzonego w wodzie, ale upieczone przy ogniu. Glowe jego z nogami jego, z wnetrznosciami jego.
\par 10 A nie zostanie z niego nic do jutra; a jezliby co z niego do jutra zostalo, ogniem spalicie.
\par 11 Tak go tedy pozywac bedziecie: Biodra swe przepaszecie, obuwie wasze bedzie na nogach waszych, a laska wasza w rece waszej, a jesc go bedziecie spieszno, albowiem przejscie jest Panskie.
\par 12 Gdyz przejde przez ziemie Egipska tej nocy, i zabije wszelkie pierworodne w ziemi Egipskiej, od czlowieka az do bydlecia, i nad wszystkimi bogi Egipskimi wykonam sady, Ja Pan.
\par 13 A bedzie wam ona krew na znak na domach, w których bedziecie; bo ujrzawszy krew, mine was, ze nie bedzie u was plaga ku zatraceniu, gdy bede zabijal w ziemi Egipskiej.
\par 14 A bedzie wam ten dzien na pamiatke; i bedziecie go obchodzic za swieto Panu w narodziech waszych; ustawa wieczna obchodzic go bedziecie.
\par 15 Przez siedem dni przasniki jesc bedziecie, a pierwszego dnia zaraz wypróznicie kwas z domów waszych; bo ktobykolwiek jadl co kwaszonego od pierwszego dnia az do dnia siódmego, wytracona bedzie dusza ona z Izraela.
\par 16 W tenze dzien pierwszy bedzie zebranie swiete, takze dnia siódmego zgromadzenie swiete miec bedziecie; zadnej roboty nie bedziecie w nich czynic, oprócz tego, czego kazdy do jedzenia uzywa, to samo gotowac bedziecie.
\par 17 I bedziecie przestrzegac przasników; albowiem w ten dzien wywiodlem wojska wasze z ziemi Egipskiej; przetoz przestrzegac bedziecie dnia tego w narodziech waszych ustawa wieczna.
\par 18 Pierwszego miesiaca, czternastego dnia tegoz miesiaca, na wieczór jesc bedziecie przasniki az do dnia dwudziestego pierwszego tegoz miesiaca na wieczór.
\par 19 Przez siedem dni kwas niech sie nie znajduje w domach waszych; bo ktobykolwiek jadl co kwaszonego, wytracona bedzie dusza jego z zgromadzenia Izraelskiego, tak przychodzien, jako i zrodzony w ziemi.
\par 20 Nic kwaszonego jesc nie bedziecie; we wszystkich mieszkaniach waszych jesc bedziecie przasniki.
\par 21 Wezwal tedy Mojzesz wszystkich starszych Izraelskich, i rzekl do nich: Odlaczcie, a wezmijcie sobie baranka wedlug familii swych, a zarznijcie na swieto przejscia.
\par 22 Wezmiecie tez snopek hysopu, i omoczycie we krwi, która bedzie w miednicy, a pokropicie odrzwi, i oba podwoje ona krwia, która bedzie w miednicy; a z was nie wynijdzie zaden ze drzwi domu swego az do poranku.
\par 23 Bo przejdzie Pan zabijajac Egipt; a ujrzawszy krew na odrzwiach i na obu podwojach, przestapi Pan drzwi, i nie dopusci morderzowi, wchodzic do domów waszych zabijac was.
\par 24 I przestrzegac bedziecie tego za ustawe, tobie i synom twoim az na wieki.
\par 25 A gdy wnijdziecie do ziemi, która wam da Pan, jako obiecal, tych obrzedów przestrzegac bedziecie.
\par 26 A gdy wam rzeka synowie wasi: Co to za obrzedy wasze?
\par 27 Tedy rzeczecie: Ofiara to przejscia Panskiego, który przestepowal domy synów Izraelskich w Egipcie, gdy zabijal Egipt, a domy nasze wyzwalal. Zatem schylil sie lud, i poklonil sie.
\par 28 I poszedlszy uczynili synowie Izraelscy, jako rozkazal Pan Mojzeszowi i Aaronowi, tak uczynili.
\par 29 I stalo sie o pólnocy, ze Pan zabijal wszystkie pierworodztwa w ziemi Egipskiej, i od pierworodnego Faraonowego, siedzacego na stolicy jego, az do pierworodnego wieznia, który byl w wiezieniu, i wszelkie pierworodne z bydlat.
\par 30 Zatem wstal Farao onej nocy, i wszyscy sludzy jego, i wszystek Egipt, i wszczal sie wielki krzyk w Egipcie; bo nie bylo domu, w którym by nie byl umarly.
\par 31 A wezwawszy Farao Mojzesza i Aarona w nocy, rzekl: Wstancie, wynijdzcie z posrodku ludu mego, i wy i synowie Izraelscy, a poszedlszy sluzcie Panu, jakoscie mówili.
\par 32 Nadto trzody wasze, i bydla wasze zabierzcie, jakoscie zadali, a odchodzac, mnie tez blogoslawcie.
\par 33 I przynaglali Egipczanie ludowi, aby ich co rychlej wyprawili z ziemi; bo mówili: Pomrzemy wszyscy.
\par 34 Wzial tedy lud ciasta swe, pierwej niz zakisialy; a one ciasta swe uwinawszy w szaty swe, kladli na ramiona swoje.
\par 35 Tedy synowie Izraelscy uczynili wedlug rozkazania Mojzeszowego, i wypozyczali u Egipczan naczynia srebrnego, i naczynia zlotego, i szat.
\par 36 A Pan dal laske ludowi w oczach Egipczanów, ze im pozyczali; i zlupili Egipt.
\par 37 Ciagneli tedy synowie Izraelscy z Rameses do Suchotu, okolo szesc kroc sto tysiecy pieszych mezów tylko, okrom dzieci.
\par 38 Ale i ludu pospolitego wiele szlo z nimi, i owiec, i bydla, dobytek bardzo wielki.
\par 39 I popiekli z ciasta zadzialanego, które wyniesli z Egiptu, placki przasne; bo nie bylo zakwaszone, przeto ze wygnani byli z Egiptu, a nie mogli zmieszkac; zywnosci tez sobie byli nie przygotowali.
\par 40 A czasu mieszkania synów Izraelskich, którego mieszkali w Egipcie, bylo cztery sta lat, i trzydziesci lat.
\par 41 I stalo sie po czterech set lat i trzydziestu lat, stalo sie onegoz dnia, wyszly wszystkie wojska Panskie z ziemi Egipskiej.
\par 42 Noc ta obchodzona ma byc Panu, ze je wywiódl ze ziemi Egipskiej. Ta tedy noc Panu obchodzona ma byc od wszystkich synów Izraelskich w narodziech ich.
\par 43 I rzekl Pan do Mojzesza i do Aarona: Ta jest ustawa swieta przejscia: Zaden obcy nie bedzie jadl z niego.
\par 44 Przetoz kazdego sluge waszego, a za pieniadze kupionego, obrzezecie go, tedy bedzie jadl z niego.
\par 45 Przychodzien i najemnik nie bedzie jadl z niego.
\par 46 W domu jednym bedzie jedzony; nie wyniesiesz nic z domu z miesa jego, a kosci nie zlamiecie w nim.
\par 47 Wszystko zgromadzenie Izraelskie tak uczyni z nim.
\par 48 A jezliby kto z przychodniów byl gosciem u ciebie, i chcialby obchodzic swieto przejscia Panu, pierwej obrzezany bedzie kazdy mezczyzna jego, a zatem przystapi obchodzic je, i bedzie jako urodzony w ziemi. A ktobykolwiek nie byl obrzezany, nie bedzie jadl z niego.
\par 49 Prawo jedno bedzie w ziemi urodzonemu i przychodniowi, który jest gosciem miedzy wami.
\par 50 Uczynili tedy wszyscy synowie Izraelscy, jako rozkazal Pan Mojzeszowi i Aaronowi, tak uczynili.
\par 51 I stalo sie onegoz dnia, wywiódl Pan syny Izraelskie z ziemi Egipskiej z wojski ich.

\chapter{13}

\par 1 I rzekl Pan do Mojzesza mówiac:
\par 2 Poswiec mi wszelkie pierworodne; cokolwiek otwiera kazdy zywot miedzy syny Izraelskimi, tak z ludzi, jako z bydla; bo moje jest.
\par 3 Tedy rzekl Mojzesz do ludu: Pamietajciez na ten dzien, któregoscie wyszli z Egiptu, z domu niewoli; bo w moznej rece wywiódl was Pan stamtad; a tak nie bedziecie jedli kwaszonego.
\par 4 Dzis wy wychodzicie, w miesiacu Abib.
\par 5 A gdy sie wprowadzi Pan do ziemi Chananejczyka, i Hetejczyka, i Amorejczyka, i Hewejczyka, i Jebuzejczyka, o która przysiagl ojcom twoim, abyc ja dal, ziemie oplywajaca mlekiem i miodem, tedy bedziesz obchodzil te sluzbe w tymze miesiacu.
\par 6 Przez siedem dni jesc bedziesz przasniki, a dnia siódmego bedzie swieto Panu.
\par 7 Przasniki jesc bedziecie przez siedem dni, i nie ukaze sie u ciebie nic kwaszonego, ani widziano bedzie kwas we wszystkich granicach twoich.
\par 8 I opowiesz synowi twemu onegoz dnia, mówiac: Dla tego, co mi uczynil Pan, gdym wychodzil z Egiptu, obchodze to.
\par 9 I bedziesz to mial za znak na rece twojej, i na pamietne przed oczyma twemi, aby Zakon Panski byl w usciech twoich, poniewaz reka mozna wywiódl cie Pan z Egiptu.
\par 10 I bedziesz strzegl ustawy tej na pewny czas, od roku do roku.
\par 11 A gdy cie Pan wprowadzi do ziemi Chananejczyka, jako przysiagl tobie i ojcom twoim, i da ja tobie:
\par 12 Tedy odlaczysz wszystko, co otwiera zywot, Panu: i kazdy plód otwierajacy zywot z bydla twego, kazdy samiec bedzie Panu.
\par 13 Kazde zas pierworodne osle odkupisz barankiem; a jezlibys nie odkupil, tedy zlamiesz mu szyje; a kazde pierworodne czlowieka miedzy synami twoimi odkupisz.
\par 14 A gdyby cie spytal syn twój potem mówiac: Cóz to jest? Tedy mu odpowiesz: Mozna reka wywiódl nas Pan z Egiptu, z domu niewoli.
\par 15 Bo gdy sie byl zatwardzil Farao, nie chcac nas wypuscic, tedy zabil Pan wszelkie pierworodne w ziemi Egipskiej, od pierworodnego z ludzi az do pierworodnego z bydla. Dla tegoz ja ofiaruje Panu kazdego samca, otwierajacego zywot, ale kazde pierworodne synów moich odkupuje.
\par 16 I bedzie to za znak na rece twojej, i za naczelniki miedzy oczyma twemi, iz w moznej rece wywiódl nas Pan z Egiptu.
\par 17 I stalo sie, gdy wypuscil Farao lud, ze nie prowadzil ich Bóg droga ziemi Filistynskiej, chociaz blizsza byla; bo mówil Bóg: By snac nie zalowal lud, gdyby ujrzal przeciw sobie wojne i nie wrócil sie do Egiptu.
\par 18 Ale obwodzil Bóg lud droga pustyni nad morzem czerwonem; i uszykowani wyszli synowie Izraelscy z ziemi Egipskiej.
\par 19 Wzial tez Mojzesz kosci Józefowe z soba dla tego, ze byl Józef przysiega obowiazal syny Izraelskie, mówiac: Zapewne nawiedzi was Bóg; przetoz wyniescie kosci moje stad z soba.
\par 20 I wyciagnawszy z Suchotu polozyli sie obozem w Etam, na koncu puszczy.
\par 21 A Pan szedl przed nimi we dnie w slupie obloku, aby je prowadzil droga, a w nocy w slupie ognia, aby im swiecil, zeby szli we dnie i w nocy.
\par 22 Nie odejmowal slupa oblokowego we dnie, ani slupa ognistego w nocy od ludu.

\chapter{14}

\par 1 I rzekl Pan do Mojzesza mówiac:
\par 2 Mów do synów Izraelskich, niech sie wróca i poloza obozem przed Fihahirot miedzy Migdol, i miedzy morzem, przeciw Baalsefon, przeciw jemu polozycie obóz nad morzem.
\par 3 Bedzie bowiem Farao mówil o synach Izraelskich: Strwozeni sa w ziemi, zawarla je puszcza.
\par 4 I zatwardze serce Faraonowe, ze je gonic bedzie; i uwielbiony bede w Faraonie i we wszystkiem wojsku jego; a poznaja Egipczanie, zem Ja Pan; i uczynili tak.
\par 5 Tedy dano znac królowi Egipskiemu, ze lud ucieka; i odmienilo sie serce Faraonowe i slug jego przeciw ludowi, i rzekli: Cózesmy to uczynili, zesmy wypuscili Izraela, aby nam nie sluzyl?
\par 6 Zaprzagl tedy wóz swój, i lud swój wzial z soba.
\par 7 Wzial tez szesc set wozów wybornych, i wszystkie wozy Egipskie, i przelozone nad tem wszystkiem.
\par 8 I zatwardzil Pan serce Faraona, króla Egipskiego, i gonil syny Izraelskie; lecz synowie Izraelscy wyszli w rece moznej.
\par 9 I gonili je Egipczanie, a dogonili je w obozie nad morzem, wszystkie konie, wozy Faraonowe, i jezdne jego, i wojska jego, nie daleko Fihahirot, przeciw Baalsefon.
\par 10 A gdy Farao nastepowal, tedy podniesli synowie Izraelscy oczy swe, a oto, Egipczanie ciagna za nimi; i bali sie bardzo, i wolali synowie Izraelscy do Pana.
\par 11 I mówili do Mojzesza: Azaz nie bylo grobów w Egipcie? wywiodles nas, abysmy pomarli na puszczy; cózes nam to uczynil, zes nas wywiódl z Egiptu?
\par 12 Azaz nie to jest, cosmy do ciebie mówili w Egipcie, mówiac: Zaniechaj nas, abysmy sluzyli Egipczanom? bo lepiej bylo nam sluzyc Egipczanom, nizeli pomrzec na puszczy.
\par 13 I rzekl Mojzesz do ludu: Nie bójcie sie, stójcie, a patrzajcie na wybawienie Panskie, które wam dzis uczyni; bo Egipczanów, których teraz widzicie, wiecej nie ogladacie na wieki
\par 14 Pan bedzie walczyl za was, a wy milczec bedziecie.
\par 15 I rzekl Pan do Mojzesza: Cóz wolasz do mnie? Mów do synów Izraelskich, aby ciagneli;
\par 16 A ty podnies laske twa, i wyciagnij reke twoje na morze, i przedziel je; a niech ida synowie Izraelscy srodkiem morza po suszy.
\par 17 A oto, Ja, Ja zatwardze serce Egipczanów, ze wnijda za nimi; a bede uwielbiony w Faraonie, i we wszystkiem wojsku jego, w woziech jego, i w jezdnych jego.
\par 18 I dowiedza sie Egipczanie, zem Ja Pan, gdy uwielbiony bede w Faraonie, w woziech jego, i w jezdnych jego.
\par 19 A ruszywszy sie Aniol Bozy, który chodzil przed obozem Izraelskim, szedl pozad ich; ruszyl sie tez slup oblokowy, który szedl przed nimi, i stanal pozad im.
\par 20 A przyszedlszy miedzy obóz Egipski, i miedzy obóz Izraelski, byl on oblok Egipczanom ciemny a Izraelczykom oswiecajacy noc, tak, iz przystapic nie mogli jedni do drugich przez cala noc.
\par 21 I wyciagnal Mojzesz reke swoje na morze, a Pan rozpedzil morze wiatrem wschodnim gwaltownie wiejacym przez cala noc, i osuszyl morze; a rozstapily sie wody.
\par 22 I szli synowie Izraelscy srodkiem morza po suszy; a wody im byly jako mur. po prawej stronie ich, i po lewej stronie ich.
\par 23 A goniac Egipczanie, weszli za nimi; wszystkie konie Faraonowe, wozy jego, i jezdni jego, w posrodek morza.
\par 24 Stalo sie tedy okolo strazy zarannej, ze wejrzal Pan na obóz Egipski z slupa ognia i obloku, i pomieszal wojsko Egipskie.
\par 25 I porzucal kola wozów ich, ze je wlekli z ciezkoscia; zaczem rzekli Egipczanie: Uciekajmy przed Izraelem, bo Pan walczy za nimi przeciwko Egipczanom.
\par 26 I rzekl Pan do Mojzesza: Wyciagnij reke twoje na morze, ze sie wróca wody na Egipczany, na wozy ich, i na jezdne ich.
\par 27 I wyciagnal Mojzesz reke swoje na morze, i wrócilo sie morze zaraz z rana do mocy swojej; a Egipczanie uciekali przeciw jemu; lecz Pan wrazil Egipczany w posród morza.
\par 28 Wróciwszy sie tedy wody, okryly wozy i jezdne, ze wszystkiem wojskiem Faraonowem, które weszlo za nimi w morze, tak iz nie zostalo z nich i jednego.
\par 29 Synowie zas Izraelscy szli po suszy srodkiem morza, a wody im byly jako mur, po prawej stronie ich, i po lewej stronie ich.
\par 30 I wybawil Pan w on dzien Izraela z reki Egipczanów; i widzieli Izraelczycy Egipczany pomarle na brzegu morskim.
\par 31 Widzial tez Izrael one moc wielka, która uczynil Pan nad Egipczany; a bal sie lud Pana, i uwierzyli Panu, i Mojzeszowi, sludze jego.

\chapter{15}

\par 1 Zaspiewal tedy Mojzesz i synowie Izraelscy te piesn Panu, a rzekli mówiac: Spiewac bede Panu, iz wielmoznie wywyzszon jest; konia i jezdnego jego wrzucil w morze.
\par 2 Moc moja i chwala moja Pan, bo mi sie stal zbawieniem; ten jest Bogiem moim, przetoz przybytek wystawie mu; Bóg ojca mego, przetoz wywyzszac go bede.
\par 3 Pan, maz waleczny, Pan imie jego.
\par 4 Wozy Faraonowe i wojsko jego wrzucil w morze, a wybrani wodzowie jego potopieni sa w morzu czerwonem.
\par 5 Przepasci okryly je; poszli w glebia jako kamien.
\par 6 Prawica twoja, Panie, uwielbiona jest w mocy, prawica twoja, Panie, potarla nieprzyjaciela.
\par 7 A w wielkosci Majestatu twego podwróciles przeciwniki twoje; pusciles gniew twój, który je pozarl jako slome.
\par 8 A tchnieniem nozdrzy twoich zebraly sie wody; stanely jako kupa ciekace wody, zsiadly sie otchlani w posród morza.
\par 9 Mówil nieprzyjaciel: Bede gonil, dogonie; bede dzielil lupy; nasyci sie ich dusza moja, dobede miecza mojego, wygladzi je reka moja.
\par 10 Wionales wiatrem twym, okrylo je morze; polknieni sa jako olów w wodach gwaltownych.
\par 11 Któz podobny tobie miedzy bogami, Panie? któz jako ty wielmozny w swiatobliwosci, straszliwy w chwale, czyniacy cuda?
\par 12 Wyciagnales prawice twoje, pozarla je ziemia.
\par 13 Prowadzisz w milosierdziu twojem ten lud, którys odkupil; poprowadzisz w moznosci twej do mieszkania swiatobliwosci twojej.
\par 14 Uslysza narodowie, zadrza; bolesc zdejmie obywatele Filistynskie.
\par 15 Tedy sie polekaja ksiazeta Edomskie, mocarze Moabskie strach zdejmie; struchleja wszyscy obywatele Chananejscy.
\par 16 Padnie na nie strach i lekanie; od wielkosci ramienia twego umilkna jako kamien, az przejdzie lud twój Panie, az przejdzie lud ten, któregos sobie nabyl.
\par 17 Wprowadzisz je, i wszczepisz je na górze dziedzictwa twego, na miejscu, któres ku mieszkaniu twemu sprawil, Panie; w swiatnicy, Panie, która umocnia rece twoje.
\par 18 Pan królowac bedzie na wieki wieczne.
\par 19 Bo weszly konie Faraonowe z wozami jego, i z jezdnymi jego w morze, a obrócil Pan na nie wody morskie; ale synowie Izraelscy szli po suszy srodkiem morza.
\par 20 Tedy Maryja, prorokini, siostra Aaronowa, wziela beben w reke swoje, a wyszly wszystkie niewiasty za nia z bebnami i muzyka.
\par 21 I mówila do nich Maryja: Spiewajcie Panu, albowiem moznie wywyzszon jest; konia i jezdnego jego wrzucil do morza.
\par 22 Potem ruszyl Mojzesz Izraela od morza czerwonego, i weszli w puszcza Sur; a idac trzy dni przez puszcza, nie znalezli wody.
\par 23 A gdy przyszli do Mara, nie mogli pic wód z Mara, bo gorzkie byly; dlategoz nazwano imie onego miejsca Mara.
\par 24 Tedy szemral lud przeciw Mojzeszowi, mówiac: Cóz bedziemy pic?
\par 25 I wolal (Mojzesz) do Pana; a ukazal mu Pan drzewo, które gdy wrzucil do wód, staly sie slodkie wody. Tam mu ustawil prawa i sady, i tam go kusil;
\par 26 I rzekl: Bedzieszli pilnie sluchal glosu Pana Boga twego, a co dobrego w oczach jego czynic bedziesz, i naklonisz uszy ku przykazaniom jego, strzegac wszystkich ustaw jego, zadnej niemocy, któram dopuscil na Egipt, nie dopuszcze na cie; bom Ja Pan, który cie lecze.
\par 27 I przyszli do Elim, gdzie bylo dwanascie zródel wód, i siedmdziesiat palm; i polozyli sie tam obozem nad wodami.

\chapter{16}

\par 1 Ruszyli sie potem z Elimu, i przyszlo wszystko mnóstwo synów Izraelskich na puszcza Zyn, która lezy miedzy Elim i miedzy Synaj, pietnastego dnia miesiaca wtórego po wyjsciu ich z ziemi Egipskiej.
\par 2 I szemralo wszystko zgromadzenie synów Izraelskich przeciw Mojzeszowi i przeciw Aaronowi na puszczy.
\par 3 A mówili do nich synowie Izraelscy: Obysmy byli pomarli od reki Panskiej w ziemi Egipskiej, gdysmy siadali nad garncy miesa, gdysmy sie najadali chleba do sytosci; bo teraz wywiedliscie nas na te puszcza, abyscie pomorzyli to wszystko mnóstwo glodem.
\par 4 Tedy rzekl Pan do Mojzesza: Oto, Ja, spuszcze wam, jako deszcz chleb z nieba, i bedzie wychodzil lud, a bedzie zbieral, coby dosc bylo na kazdy dzien, abym go doswiadczyl, bedzieli chodzil w zakonie moim, czyli nie.
\par 5 Ale dnia szóstego nagotuja to, co przyniosa, a bedzie tyle dwoje niz co zbierac zwykli na kazdy dzien.
\par 6 I mówil Mojzesz i Aaron do wszystkich synów Izraelskich: W wieczór poznacie, iz Pan wywiódl was z ziemi Egipskiej;
\par 7 A rano ogladacie chwale Panska; bo uslyszal szemrania wasze przeciw Panu. A my co jestesmy, iz szemrzecie przeciwko nam?
\par 8 I rzekl Mojzesz: Da wam Pan, w wieczór mieso do jedzenia, a chleb rano do nasycenia; bo uslyszal Pan szemrania wasze, któremi szemrzecie przeciw jemu. A my co jestesmy? Nie przeciwko nam sa szemrania wasze, ale przeciwko Panu.
\par 9 I rzekl Mojzesz do Aarona: Mów do wszystkiego zgromadzenia synów Izraelskich: Przystapcie przed oblicznosc Panska; bo uslyszal szemranie wasze.
\par 10 I stalo sie, gdy mówil Aaron do wszystkiego zgromadzenia synów Izraelskich, ze spojrzeli ku puszczy, a oto, chwala Panska ukazala sie w obloku.
\par 11 Zatem rzekl Pan do Mojzesza, mówiac:
\par 12 Uslyszalem szemranie synów Izraelskich; rzeczze do nich, mówiac: Miedzy dwoma wieczorami bedziecie jesc mieso, a rano nasyceni bedziecie chlebem, i poznacie, zem Ja Pan, Bóg wasz.
\par 13 Stalo sie tedy wieczór, ze sie zlecialy przepiórki, a okryly obóz, a rano rosa lezala okolo obozu;
\par 14 A gdy przestala padac rosa, oto, ukazalo sie na puszczy cos drobnego, okraglego drobnego jako szron na ziemi.
\par 15 Co gdy ujrzeli synowie Izraelscy, mówili jeden do drugiego: Man hu? bo nie wiedzieli, co bylo. I rzekl Mojzesz do nich: Tenci jest chleb, który wam dal Pan ku jedzeniu.
\par 16 Toc jest, co rozkazal Pan: Zbierajcie z niego kazdy, ile trzeba ku jedzeniu, po mierze Gomer na osobe, wedlug liczby dusz waszych; kazdy na tych, którzy sa w namiocie jego, zbierajcie.
\par 17 I uczynili tak synowie Izraelscy, i zbierali jedni wiecej, drudzy mniej.
\par 18 I mierzyli w Gomer, i nie zbywalo temu, co wiecej nazbieral, ani nie dostawalo temu, co mniej; kazdy wedlug tego, co mógl zjesc, nazbieral.
\par 19 Mówil tez Mojzesz do nich: Zaden niech nie zostawia z niego az do zarania.
\par 20 Jednak nie usluchali Mojzesza; ale zostawili z niego niektórzy az do poranku, i obrócilo sie w robaki, i zsmierdlo sie; i rozgniewal sie na nie Mojzesz.
\par 21 A zbierali to na kazdy dzien rano, kazdy wedlug tego, co mógl zjesc; a gdy sie zagrzalo slonce, tedy ono topnialo.
\par 22 A gdy bylo dnia szóstego, zbierali chleb w dwójnasób, po dwu Gomer na kazdego. I zeszly sie wszystkie ksiazeta zgromadzenia, oznajmujac to Mojzeszowi.
\par 23 Który im rzekl: Toc jest, co mówil Pan: Odpocznienie sabbatu swietego Panu jutro bedzie; co macie piec, pieczcie, a co macie warzyc, warzcie, a cokolwiek zbedzie, zostawcie sobie, a zachowajcie do jutra.
\par 24 Zostawiali tedy ono na jutro, jako byl rozkazal Mojzesz; a nie zsmierdlo sie, i robak nie byl w niem.
\par 25 I mówil Mojzesz: Jedzciez to dzis, bo dzis sabbat Panu; dzis nie znajdziecie tego na polu.
\par 26 Przez szesc dni zbierac to bedziecie, a dnia siódmego sabbat; nie bedzie wen manny.
\par 27 I stalo sie dnia siódmego, wyszli niektórzy z ludu, aby zbierali; ale nie znalezli.
\par 28 Tedy rzekl Pan do Mojzesza: I pókiz nie bedziecie chcieli przestrzegac przykazan moich i zakonu mego?
\par 29 Patrzcie, iz wam Pan dal sabbat, dlatego w dzien szósty daje wam chleb na dwa dni; zostawajcie kazdy na miejscu swem, niech nie wychodzi zaden z miejsca swego w dzien siódmy.
\par 30 I odpoczywal lud dnia siódmego.
\par 31 I nazwal dom Izraelski imie onego pokarmu Man, który byl jako nasienie koryjandrowe, bialy, a smak jego jako placki z miodem.
\par 32 Mówil tez Mojzesz: Tak rozkazal Pan: Napelnij Gomer z niego na chowanie w narodziech waszych, aby widzieli chleb ten, którymem was karmil na puszczy, gdym was wywiódl z ziemi Egipskiej.
\par 33 Rzekl zatem Mojzesz do Aarona: Wezmij wiadro jedno, a nasyp w nie pelen Gomer manny, a postaw je przed Panem na chowanie do narodów waszych.
\par 34 Jako przykazal Pan Mojzeszowi, tak postawil je Aaron przed swiadectwem na chowanie.
\par 35 A synowie Izraelscy jedli manne przez czterdziesci lat, az przyszli do ziemi mieszkania; manne jedli, az przyszli do granic ziemi Chananejskiej.
\par 36 A Gomer jest dziesiata czesc miary Efa.

\chapter{17}

\par 1 Ruszylo sie tedy wszystko mnóstwo synów Izraelskich z puszczy Zyn stanowiskami swemi, wedlug rozkazania Panskiego, i polozyli sie obozem w Rafidym, gdzie wody nie bylo, aby pil lud.
\par 2 Przetoz swarzyl sie lud z Mojzeszem, mówiac: Dajcie nam wody, abysmy pili. Którym odpowiedzial Mojzesz: Cóz sie swarzycie ze mna? a czemu kusicie Pana?
\par 3 I pragnal tam lud wody, a szemral przeciwko Mojzeszowi, mówiac: Po cózes nas wywiódl z Egiptu, abys pomorzyl mnie, i syny moje, i bydlo moje pragnieniem?
\par 4 Zawolal tedy Mojzesz do Pana, mówiac: Cóz mam czynic ludowi temu? blisko tego, ze mie ukamionuja.
\par 5 I rzekl Pan do Mojzesza: Idz przed ludem, a wezmij z soba niektóre z starszych Izraelskich; laske tez twoje któras uderzyl w rzeke, wezmij w reke twoje, a idz.
\par 6 Oto, Ja stane przed toba tam na skale w Horeb, i uderzysz w skale, a wynijda z niej wody, które bedzie pil lud. I uczynil tak Mojzesz przed oczyma starszych Izraelskich.
\par 7 I nazwal imie onego miejsca Masa i Meryba, dla swarów synów Izraelskich, a iz kusili Pana mówiac: I jestze Pan miedzy nami czyli nie?
\par 8 Tedy przyciagnal Amalek, aby walczyl z Izraelem w Rafidym.
\par 9 I rzekl Mojzesz do Jozuego: Wybierz nam meze, a wyszedlszy, stocz bitwe z Amalekity: a jutro stane na wierzchu pagórka, majac laske Boza w rece mojej.
\par 10 I uczynil Jozue, jako mu rozkazal Mojzesz, i stoczyl bitwe z Amalekiem; a Mojzesz, Aaron i Chur wstapili na wierzch pagórka.
\par 11 A gdy podnosil Mojzesz reke swoje, przemagal Izrael; a gdy opuszczal reke swoje, przemagal Amalek.
\par 12 Ale rece Mojzeszowe ociezaly byly; wziawszy tedy kamien, podlozyli poden, i usiadl na nim; a Aaron, i Chur podpierali rece jego, jeden z jednej, drugi z drugiej strony; i nie ustaly rece jego az do zajscia slonca.
\par 13 Tedy porazil Jozue Amaleka i lud jegoz ostrzem miecza.
\par 14 Potem rzekl Pan do Mojzesza: Wpisz to dla pamieci w ksiegi, a wlóz to w uszy Jozuego, ze pewnie wygladze pamiatke Amaleka pod niebem.
\par 15 I zbudowal Mojzesz oltarz, a nazwal imie jego: Pan choragiew moja;
\par 16 Bo rzekl: Iz reka stolicy Panskiej, i wojna Panska, bedzie przeciwko Amalekowi od rodzaju do rodzaju.

\chapter{18}

\par 1 A gdy uslyszal Jetro, kaplan Madyjanski, swiekier Mojzesza, wszystko, co uczynil Bóg Mojzeszowi, i Izraelowi, ludowi swemu, ze wywiódl Pan Izraela z Egiptu;
\par 2 Tedy wzial Jetro, swiekier Mojzesza, Zefore, zone Mojzeszowe, która byl odeslal.
\par 3 I dwu synów jej, z których imie jednemu Gerson; bo byl powiedzial Mojzesz: Bylem przychodniem w ziemi cudzej.
\par 4 A imie drugiego Eliezer; iz mówil: Bóg ojca mego byl mi ku pomocy, i wyrwal mie od miecza Faraonowego.
\par 5 I przyszedl Jetro, swiekier Mojzesza, z synami jego i z zona jego do Mojzesza na puszcza, gdzie sie byl obozem polozyl przy górze Bozej.
\par 6 I wskazal do Mojzesza: Ja swiekier twój Jetro ide do ciebie, i zona twoja, i jej dwa synowie z nia.
\par 7 Zatem Mojzesz wyszedl przeciwko swiekrowi swemu, a ukloniwszy sie calowal go; i przywitawszy jeden drugiego, potem weszli do namiotu.
\par 8 I rozpowiadal Mojzesz swiekrowi swemu wszystko, co uczynil Pan Faraonowi i Egipczanom za przyczyna Izraela; i wszystke trudnosc, która je spotykala w drodze, i jako je Pan wybawil.
\par 9 I radowal sie Jetro ze wszystkiego dobrego, które uczynil Pan Izraelowi, iz go wyrwal z reki Egipczanów.
\par 10 I rzekl Jetro: Blogoslawiony Pan, który was wyrwal z reki Egipczanów i z reki Faraonowej, który wyrwal lud z niewoli Egipskiej.
\par 11 Terazem doznal, ze wiekszy jest Pan nad wszystkie bogi; albowiem czem oni hardzie powstawali przeciwko niemu, tem pogineli.
\par 12 I wzial Jetro, swiekier Mojzesza, calopalenie i ofiary Bogu. Przyszedl tez Aaron i wszyscy starsi Izraelscy, aby jedli chleb z swiekrem Mojzeszowym przed Bogiem.
\par 13 I stalo sie nazajutrz, ze usiadl Mojzesz, aby sadzil lud, i stal lud przed Mojzeszem od poranku az do wieczora.
\par 14 A widzac swiekier Mojzesza wszystko, co on czynil z ludem, rzekl: Cóz to jest, co ty czynisz z ludem? czemuz ty sam siedzisz, a lud wszystek stoi przed toba od poranku az do wieczora?
\par 15 Tedy Mojzesz odpowiedzial swiekrowi swemu: Iz przychodzi lud do mnie, aby sie radzil Boga.
\par 16 Gdy sprawe jaka maja, przychodza do mnie, a rozsadzam miedzy nimi, oznajmujac ustawy Boze i prawa jego.
\par 17 Zatem rzekl swiekier Mojzesza do niego: Nie dobra rzecz, która ty czynisz.
\par 18 Pewnie ustaniesz, i ty i lud ten, który jest z toba, bo ciezsza to rzecz nad sily twoje; nie bedziesz jej mógl ty sam podolac.
\par 19 Przetoz usluchaj teraz glosu mego, poradzec, a bedzie Bóg z toba; stój ty za lud przed Bogiem, a odnos sprawy do Boga;
\par 20 A onych tez nauczaj ustaw i praw, oznajmujac im droge, która by chodzic, i dzielo, które by czynic mieli.
\par 21 Ty tez upatrz ze wszystkiego ludu meze stateczne, bojace sie Boga, meze prawdomówne, którzy by nienawidzili lakomstwa, a postanów z nich przelozone, tysiaczniki, setniki, piecdziesiatniki i dziesiatniki.
\par 22 Którzy na kazdy czas lud sadzic beda; a gdy bedzie rzecz wielka, odniosa do ciebie, a kazda rzecz mala sadzic beda sami; tedy ulzysz sobie, gdy poniosa ciezar z toba.
\par 23 To jezli uczynisz, a rozkazec Bóg, ostoisz sie, i ten wszystek lud na miejsca swoje wracac sie bedzie w pokoju.
\par 24 I usluchal Mojzesz rady swiekra swojego, a uczynil wszystko, jako mu powiedzial.
\par 25 I wybral Mojzesz meze stateczne ze wszystkiego Izraela, i postanowil je przelozonymi nad ludem, tysiaczniki, setniki, piecdziesiatniki i dziesiatniki.
\par 26 Którzy sadzili lud kazdego czasu; trudne rzeczy odnosili do Mojzesza, a kazda rzecz mniejsza sami sadzili.
\par 27 Zatem puscil od siebie Mojzesz swiekra swego, który odszedl do ziemi swej.

\chapter{19}

\par 1 Miesiaca trzeciego po wyjsciu synów Izraelskich z ziemi Egipskiej, w tenze dzien przyszli na puszcza Synaj.
\par 2 Bo ruszywszy sie z Rafidym, i przyszedlszy az na puszcza Synaj, polozyli sie obozem na puszczy, i rozbil tam Izrael namioty przeciw górze.
\par 3 A Mojzesz wstapil do Boga, i zawolal nan Pan z góry, mówiac: Tak powiesz domowi Jakóbowemu, i oznajmisz synom Izraelskim:
\par 4 Wyscie widzieli, com uczynil Egipczanom, i jakom was nosil niby na skrzydlach orlowych, i przywiodlem was do siebie.
\par 5 Przetoz teraz jezli sluchajac posluszni bedziecie glosu memu, i strzec bedziecie przymierza mego, bedziecie mi wlasnoscia nad wszystkie narody; chociaz moja jest wszystka ziemia.
\par 6 A wy bedziecie mi królestwem kaplanskiem, i narodem swietym. Tec sa slowa, które mówic bedziesz do synów Izraelskich.
\par 7 Przyszedlszy tedy Mojzesz zwolal starszych z ludu, i przelozyl im wszystkie te slowa, które mu rozkazal Pan.
\par 8 I odpowiedzial wszystek lud, spólnie mówiac: Wszystko, co Pan rzekl, uczynimy. I odniósl Mojzesz slowa ludu do Pana.
\par 9 I rzekl Pan do Mojzesza: Oto, Ja, przyjde do ciebie w gestym obloku, aby sluchal lud, gdy bede mówil z toba, azeby tez wierzyli tobie na wieki; albowiem opowiedzial byl Mojzesz slowa ludu onego Panu.
\par 10 Mówil zas Pan do Mojzesza: Idz do ludu, a poswiec je dzis i jutro, a niech wypiora szaty swoje.
\par 11 I niech beda gotowi na dzien trzeci; albowiem trzeciego dnia zstapi Pan przed oczyma wszystkiego ludu na góre Synaj.
\par 12 I zamierzysz granice ludowi w okolo, mówiac: Strzezcie sie, abyscie nie wstepowali na góre, ani sie dotykali brzegu jej; wszelki, kto by sie dotknal góry, smiercia umrze.
\par 13 Nie tknie sie go reka, ale kamieniem ukamionuja go; albo strzelajac zastrzela go; badz bydle, badz czlowiek, nie bedzie zyl. Gdy przewlocznie trabic beda, niech wstapia na góre.
\par 14 Zstapil tedy Mojzesz z góry do ludu, i poswiecil lud; a uprali szaty swoje.
\par 15 I mówil do ludu: Badzcie gotowi na dzien trzeci, nie przystepujcie do zon.
\par 16 Stalo sie tedy dnia trzeciego rano, ze byly grzmienia, i blyskawice, i gesty oblok nad góra, i glos traby bardzo potezny; a bal sie wszystek lud, który byl w obozie.
\par 17 I wywiódl Mojzesz lud na przeciwko Bogu z obozu, a staneli pod sama góra.
\par 18 A góra Synaj kurzyla sie wszystka, przeto, iz zstapil na nia Pan w ogniu; i wystepowal dym z niej, jako dym z pieca, i trzesla sie wszystka góra bardzo.
\par 19 A gdy sie glos traby im dalej tem bardziej rozlegal, Mojzesz mówil, a Bóg mu odpowiadal glosem.
\par 20 I zstapil Pan na góre Synaj, na wierzch góry, i wezwal Pan Mojzesza na wierzch góry, i wstapil tam Mojzesz.
\par 21 Zaczem rzekl Pan do Mojzesza: Zstap, przestrzez lud, by snac nie przestapili kresu, chcac Pana widziec, aby ich nie padlo wiele:
\par 22 Nawet i kaplani, którzy przystepuja do Pana, niech sie poswieca, by ich snac nie potracil Pan.
\par 23 I rzekl Mojzesz do Pana: Nie bedzie lud mógl wnijsc na góre Synaj, poniewazes ty nas przestrzegl, mówiac: Ogranicz góre, a poswiec ja.
\par 24 Któremu Pan rzekl: Idz, zstap, a zas tu wstapisz, ty i Aaron z toba: lecz kaplani i lud niech nie przestepuja kresu, aby wstapili do Pana, by ich snac nie potracil.
\par 25 Tedy zstapil Mojzesz do ludu i powiedzial im to.

\chapter{20}

\par 1 I mówil Bóg wszystkie te slowa a rzekl:
\par 2 Jam jest Pan Bóg twój, którym cie wywiódl z ziemi Egipskiej, z domu niewoli.
\par 3 Nie bedziesz mial bogów innych przede mna.
\par 4 Nie czyn sobie obrazu rytego, ani zadnego podobienstwa rzeczy tych, które sa na niebie wzgóre, i które na ziemi nisko, i które sa w wodach pod ziemia.
\par 5 Nie bedziesz sie im klanial, ani im bedziesz sluzyl; bom Ja Pan Bóg twój, Bóg zawistny w milosci, nawiedzajacy nieprawosci ojców nad syny w trzeciem i w czwartem pokoleniu tych, którzy mie nienawidza;
\par 6 A czyniacy milosierdzie nad tysiacami tych, którzy mie miluja, i strzega przykazania mego.
\par 7 Nie bierz imienia Pana Boga twego nadaremno; bo sie Pan mscic bedzie nad tym, który imie jego nadaremno bierze.
\par 8 Pamietaj na dzien odpocznienia, abys go swiecil.
\par 9 Szesc dni robic bedziesz, i wykonasz wszystke robote twoje.
\par 10 Ale dnia siódmego odpocznienie jest Pana Boga twego; nie bedziesz czynil zadnej roboty wen, ty i syn twój, i córka twoja, sluga twój, i sluzebnica twoja, bydle twoje, i gosc twój, który jest w bramach twoich;
\par 11 Bo przez szesc dni stworzyl Pan niebo i ziemie, morze, i cokolwiek w nich jest, i odpoczal dnia siódmego; przetoz blogoslawil Pan dzien odpocznienia, i poswiecil go.
\par 12 Czcij ojca twego i matke twoje, aby przedluzone byly dni twoje na ziemi, która Pan Bóg twój da tobie.
\par 13 Nie bedziesz zabijal.
\par 14 Nie bedziesz cudzolozyl.
\par 15 Nie bedziesz kradl.
\par 16 Nie bedziesz mówil przeciw blizniemu twemu swiadectwa falszywego.
\par 17 Nie bedziesz pozadal domu blizniego twego, ani bedziesz pozadal zony blizniego twego, ani slugi jego, ani dziewki jego, ani wolu jego, ani osla jego, ani zadnej rzeczy blizniego twego.
\par 18 Tedy wszystek lud widzac gromy, i blyskawice, i glos traby, i góre kurzaca sie, to widzac lud cofneli sie, i staneli z daleka;
\par 19 I mówili do Mojzesza: Mów ty z nami, a bedziemy sluchac; a niech nie mówi do nas Bóg, bysmy snac nie pomarli.
\par 20 I odpowiedzial Mojzesz ludowi: Nie bójcie sie; bo aby was doswiadczyl, przyszedl Bóg, zeby bojazn jego byla przed obliczem waszem, byscie nie grzeszyli.
\par 21 Stal tedy lud z daleka; a Mojzesz przystapil do ciemnej mgly, w której byl Bóg.
\par 22 I rzekl Pan do Mojzesza: Tak powiesz synom Izraelskim: Wyscie widzieli, zem z nieba mówil do was:
\par 23 Nie bedziecie czynic przy mnie bogów srebrnych, ani bogów zlotych nie bedziecie sobie czynic.
\par 24 Oltarz z ziemi uczynisz mi, a ofiarowac bedziesz na nim calopalenie twoje, i spokojne ofiary twoje, owce twoje, i woly twoje; na któremkolwiek miejscu pamiatke uczynie imienia mego, przyjde do ciebie, i bedec blogoslawil.
\par 25 A jezli oltarz kamienny uczynisz mi, nie buduj go z ciosanego kamienia; bo jezlibys zelazne naczynie twoje podniósl nan, splugawisz go.
\par 26 Nie bedziesz wstepowal po stopniach do oltarza mojego, aby nie byla odkryta nagosc twoja przy nim.

\chapter{21}

\par 1 A tec sa sady, które przelozysz przed obliczem ich.
\par 2 Jezli kupisz niewolnika Hebrejczyka, szesc lat sluzyc ci bedzie, a siódmego wynijdzie wolny darmo.
\par 3 Jezliby sam tylko przyszedl, sam odejdzie; a jezliby mial zone, i zona jego z nim wynijdzie.
\par 4 Jezli mu pan jego dal zone, a zrodzila mu syny albo córki, zona i dzieci jego beda pana jego, a on sam tylko odejdzie.
\par 5 A jezliby mówiac rzekl niewolnik: Miluje pana mego, zone moje, i syny moje, nie wynijde wolnym:
\par 6 Tedy przywiedzie go pan jego do sedziów, a postawi go u drzwi albo u podwoi; i przekole mu pan jego ucho jego szydlem, i bedzie mu niewolnikiem na wieki.
\par 7 Zas jezliby kto zaprzedal córke swoje, aby byla niewolnica, nie wynijdzie jako wychodza niewolnicy.
\par 8 Jezliby sie nie spodobala w oczach pana swego, a nie poslubilby jej sobie, niech pozwoli, aby ja odkupiono; obcemu ludowi nie bedzie jej mógl sprzedac, poniewaz zgrzeszyl przeciwko niej.
\par 9 A jezliby ja synowi swemu poslubil, wedlug prawa córek uczyni jej.
\par 10 Jezliby tez inna wzial mu za zone, tedy pozywienia jej, odzienia jej, i prawa malzenskiego nie umniejszy jej.
\par 11 A jezli tych trzech rzeczy nie uczyni jej, tedy wynijdzie darmo bez okupu.
\par 12 Kto by uderzyl czlowieka, azby umarl, smiercia umrze;
\par 13 Lecz kto by nie czyhal na kogo, aleby go Bóg podal w reke jego, naznaczec miejsce, na które bedzie mial uciec.
\par 14 Ale jezliby kto umyslnie przeciw blizniemu swemu zasadziwszy sie zdrada zabil go, i od oltarza mego wezmiesz go, aby umarl.
\par 15 Kto by uderzyl ojca swego, albo matke swoje, smiercia umrze.
\par 16 Kto by ukradl czlowieka a sprzedalby go, a znaleziony by byl w reku jego, smiercia umrze.
\par 17 Kto by zlorzeczyl ojcu swemu albo matce swojej, smiercia umrze.
\par 18 A gdyby sie poswarzyli mezowie, i uderzylby kto blizniego swego kamieniem albo piescia, a on by zaraz nie umarl, aleby sie polozyl na loze;
\par 19 A wstawszy chodzilby po ulicy o lasce swej, nie bedzie winien ten, który uderzyl; tylko omieszkanie jego nagrodzi, a na wyleczenie jego nalozy.
\par 20 Jezliby zas uderzyl kto niewolnika swego, albo niewolnice swoje kijem, i umarliby w reku jego, koniecznie karanie odniesie;
\par 21 Wszakze, jezliby dzien albo dwa zyw zostal, nie bedzie karany; bo za pieniadze jego on jest kupiony.
\par 22 Jezliby sie tez powadziwszy mezowie, uderzyl który z nich niewiaste brzemienna, tak zeby z niej plód wyszedl, jednakby nie zaszla smierc koniecznie karanie odniesie, jakie wlozy nan maz onejze niewiasty, a da wedle uznania sedziów.
\par 23 Ale gdzie by smierc zaszla, tedy dasz dusze za dusze;
\par 24 Oko za oko, zab za zab, reke za reke, noge za noge,
\par 25 Sparzeline za sparzeline, rane za rane, sinosc za sinosc.
\par 26 Jezliby zas kto wybil oko niewolnikowi swemu, albo oko niewolnicy swojej, azby sie zepsowalo, wolno go pusci za oko jego.
\par 27 Jezliby tez kto zab niewolnikowi swemu, albo zab niewolnicy swojej wybil, wolno go pusci za zab jego.
\par 28 Jezliby tez czyj wól ubódl meza albo niewiaste, a umarliby, koniecznie ukamionowany bedzie on wól, a nie beda jesc miesa jego; a pan wolu onego nie bedzie winien.
\par 29 Wszakze, jezliby wól bódl przedtem, a ostrzeganoby w tym pana jego, i nie mialby go pod straza, a zabilby meza albo niewiaste, wól on ukamionowany bedzie, nadto i pan jego umrze.
\par 30 Jezliby nan wlozono, zeby sie odkupil, tedy da okup za dusze swoje, jakikolwiek nan wloza.
\par 31 Chocby syna ubódl, albo córke ubódl, podlug tegoz sadu postapia z nim.
\par 32 Jezliby niewolnika ubódl wól, albo niewolnice, srebra trzydziesci syklów da panu jego, a wól on ukamionowany bedzie.
\par 33 Jezliby kto otworzyl studnia, albo jezliby kto wykopal studnia, i nie nakrylby jej, a wpadlby w nia wól albo osiel:
\par 34 Pan onej studni odda zaplate, i nagrodzi panu ich, a co zdechlo, jego bedzie.
\par 35 Takze, gdyby wól czyj ubódl wolu sasiada jego, azby zdechl, tedy sprzedadza wolu zywego, i rozdziela sie zaplata jego, onym tez zdechlym podziela sie.
\par 36 Ale jezliby to bylo wiadomo, ze on wól bódl przedtem, a nie mial go pod straza pan jego, koniecznie odda wolu za wolu, a zdechlego sobie wezmie.

\chapter{22}

\par 1 Jezliby kto ukradl wolu albo owce, i zabilby je, albo je sprzedal, piec wolów wróci za jednego wolu, a cztery owce za jedna owce,
\par 2 Jezliby przy podkopywaniu zastany byl zlodziej, a ubity bedac umarlby, kto zabil, nie bedzie winien krwi;
\par 3 Jezliby to po wejsciu slonca uczynil, krwi winien bedzie, koniecznie wróci; a jezli nie ma, sprzedany bedzie za zlodziejstwo swoje.
\par 4 Jezli znaleziona bedzie w reku jego rzecz kradziona, badz wól, badz osiel, badz owca, jeszcze zywe, we dwójnasób wróci.
\par 5 Gdyby wypasl kto pole, albo winnice, i puscilby bydle swoje, aby sie paslo na polu cudzem: z najlepszego urodzaju pola swego, i z najlepszego urodzaju winnicy swej nagrodzi.
\par 6 Jezliby wyszedl ogien, a trafilby na ciernie, i spalilby stóg, albo stojace zboze, albo pole samo, koniecznie nagrodzi ten, co zapalil, co zgorzalo.
\par 7 Gdyby kto dal blizniemu swemu srebro, albo naczynie do schowania, a to by ukradziono bylo z domu onego czlowieka: jezliby znaleziony byl zlodziej, wróci dwojako.
\par 8 Jezliby nie byl znaleziony zlodziej, stawi sie pan domu onego przed sedziów, i przysieze, ze nie sciagnal reki swojej na rzecz blizniego swego.
\par 9 O kazda rzecz, o która by byl spór, o wolu, o osla, o owce, o szate, o kazda, rzecz zgubiona, gdyby kto rzekl, ze to jest moje, przed sedziów przyjdzie sprawa obydwu; kogo winnym znajda sedziowie, nagrodzi w dwójnasób blizniemu swemu.
\par 10 Jezliby kto dal blizniemu swemu osla, albo wolu, albo owce, albo inne bydle na chowanie, a zdechloby albo okaleczalo, albo gwaltem zajete bylo, gdzie by nikt nie widzial:
\par 11 Przysiega Panska bedzie miedzy obiema, ze nie sciagnal reki swej na rzecz blizniego swego: i przyjmie pan onej rzeczy przysiege, a on nie bedzie nagradzal.
\par 12 A jezliby mu to kradzieza wzieto, nagrodzi panu rzeczy onej.
\par 13 Jezliby od zwierza rozszarpane bylo, postawi rozszarpane za swiadka, a rozszarpanego nie nagrodzi.
\par 14 Gdyby tez kto pozyczyl bydlecia od blizniego swego, a okaleczaloby, albo zdechlo w niebytnosci pana jego, koniecznie nagrodzi.
\par 15 Jezliby pan jego byl przy nim, nie bedzie nagradzal; a jezliby najete bylo, najem tylko zaplaci.
\par 16 Jezliby kto zwiódl panne, która nie jest poslubiona, i spalby z nia, da jej koniecznie wiano, i wezmie ja sobie za zone.
\par 17 Jezliby zadna miara ojciec jej nie chcial mu jej dac, odwazy srebra wedlug zwyczaju wiana panienskiego.
\par 18 Czarownicy zyc nie dopuscisz.
\par 19 Kazdy, kto by sie zlaczal z bydleciem, smiercia umrze.
\par 20 Kto by ofiarowal bogom, oprócz samego Pana, wytracony bedzie.
\par 21 Przychodniowi nie uczynisz krzywdy, ani go ucisniesz: boscie byli przychodniami w ziemi Egipskiej.
\par 22 Zadnej wdowy ani sieroty trapic nie bedziecie.
\par 23 Jezlibys je bez litosci trapil, a one by wolaly do mnie, slyszac wyslucham wolanie ich.
\par 24 I rozgniewa sie zapalczywosc moja, a pobije was mieczem, i beda zony wasze wdowami, a synowie wasi sierotami.
\par 25 Jezlibys pieniedzy pozyczyl ludowi memu ubogiemu, który mieszka z toba, nie badziesz mu jako lichwiarz, nie obciazycie go lichwa.
\par 26 Jezli w zastawie wezmiesz szate blizniego twego, przed zachodem slonca wrócisz mu ja;
\par 27 Bo to odzienie jego tylko to jest nakrycie ciala jego, na którem sypia; bedzieli do mnie wolal, wyslucham go, bom Ja milosierny.
\par 28 Sedziom nie bedziesz zlorzeczyl, a przelozonego ludu twego nie bedziesz przeklinal.
\par 29 Z obfitosci zboza, i ciekacych rzeczy twych nie omieszkasz pierwiastek ofiarowac; pierworodnego z synów twoich oddasz mi.
\par 30 Toz uczynisz z wolów twych i z owiec twoich; siedem dni bedzie z matka swoja, a ósmego dnia oddasz mi je.
\par 31 Ludem swietym bedziecie mi, a miesa na polu rozszarpanego jesc nie bedziecie; psom je wyrzucicie.

\chapter{23}

\par 1 Nie przyjmuj powiesci klamliwej; nie miej spólki z niepoboznym, abys mial byc swiadkiem falszywym.
\par 2 Nie udawaj sie za wielkoscia do zlego, i nie mów tak za sprawa, cobys sie naklonil za wiela ich ku podwróceniu sadu.
\par 3 I nie szanuj ubogiego przy sprawie jego.
\par 4 Gdybys natrafil wolu nieprzyjaciela twego, albo osla jego bladzacego, zwrócisz, a dowiedziesz go do niego.
\par 5 Jezlibys ujrzal, ze osiel tego, który cie ma w nienawisci, lezy pod brzemieniem swojem, azali zaniechasz, abys mu pomóc nie mial? owszem poratujesz go pospolu z nim.
\par 6 Nie bedziesz podwracal sadu ubogiemu twemu w sprawie jego.
\par 7 Od rzeczy klamliwej oddalisz sie, a niewinnego i sprawiedliwego nie zabijesz; bo nie usprawiedliwie niezboznego.
\par 8 Darów tez brac nie bedziesz, poniewaz dar zaslepia madre, i wywraca slowa sprawiedliwych.
\par 9 Przychodnia tez nie uciskaj: bo sami wiecie, jaki jest zywot przychodnia, boscie byli przychodniami w ziemi Egipskiej.
\par 10 Przez szesc lat obsiewac bedziesz ziemie twoje, a bedziesz zgromadzal urodzaj jej;
\par 11 Ale siódmego roku zaniechasz jej; ze odpocznie, aby jedli ubodzy ludu twego, a co zostanie po nich, zje zwierz polny. Takze uczynisz winnicy twojej, i oliwnicy twojej.
\par 12 Przez szesc dni bedziesz odprawowal roboty twoje; ale dnia siódmego odpoczniesz, aby sobie wytchnal wól twój, osiel twój, i zeby wytchnal syn niewolnicy twojej, i przychodzien.
\par 13 A we wszystkiem, com wam powiedzial, ostroznymi badzcie. Imienia tez cudzych bogów nie wspominajcie, nie bedzie slyszane z ust twoich.
\par 14 Trzy kroc swieto obchodzic mi bedziecie na kazdy rok.
\par 15 Swieta przasników przestrzegac bedziesz; siedem dni jesc bedziesz przasniki, jakom ci rozkazal, czasu miesiaca Abiba; bos wen wyszedl z Egiptu, a nie ukazecie sie przed twarz moje próznymi.
\par 16 Takze swieto zniwa pierwiastek pracy twojej, coskolwiek sial na polu; swieto tez zbierania na schodzie roku, gdy zbierzesz prace twoje z pola.
\par 17 Trzykroc do roku ukaze sie kazdy mezczyzna twój przed obliczem Panujacego Pana.
\par 18 Nie bedziesz ofiarowal przy kwasie krwi ofiary mojej, ani zostanie przez noc tlustosc ofiary mojej az do poranku.
\par 19 Pierwiastki pierwszych urodzajów ziemi twej przyniesiesz w dom Pana Boga twego; nie bedziesz warzyl kozlecia w mleku matki jego.
\par 20 Oto ja posylam Aniola przed toba, aby cie strzegl w drodze, i wprowadzil cie na miejsce, którem ci zgotowal.
\par 21 Ostroznym badz przed oblicznoscia jego; a sluchaj glosu jego; nie draznij go, boc nie przepusci przestepstwu waszemu, gdyz imie moje w nim jest.
\par 22 Bo jezli pilnie sluchac bedziesz glosu jego, i uczynisz, cokolwiek rzeke, nieprzyjacielem bede nieprzyjaciól twych, i trapic bede tych, którzy cie trapili.
\par 23 Pójdzie bowiem Aniol mój przed toba, i wprowadzi cie do Amorejczyka, i Hetejczyka, i Ferezejczyka, i Chananejczyka, Hewejczyka, i Jebuzejczyka, i wytrace je.
\par 24 Nie klaniajze sie bogom ich, ani im sluz, ani czyn wedlug spraw ich; ale do gruntu popsujesz je, i wszczat pokruszysz obrazy ich.
\par 25 Lecz sluzyc bedziecie Panu Bogu waszemu, a on blogoslawic bedzie chlebowi twemu, i wodom twoim; i odejme niemoc z posrodku ciebie.
\par 26 Nie bedzie poroniajaca ani nieplodna w ziemi twojej; liczbe dni twoich dopelnie.
\par 27 Strach mój puszcze przed toba, i strwoze wszelki lud, przeciw któremu pójdziesz, i uczynie, ze wszyscy nieprzyjaciele twoi podadza tyl przed toba.
\par 28 Posle tez szerszenie przed toba, które wypedza Hewejczyka, Chananejczyka, i Hetejczyka przed oblicznoscia twoja.
\par 29 Nie wyrzuce go przed obliczem twojem za jeden rok, by sie snac ziemia w pustynia nie obrócila, a nie namnozylo sie przeciwko tobie zwierza dzikiego.
\par 30 Pomaluczku bede je wyrzucal od oblicza twego, az sie rozmnozysz i osiadziesz ziemie.
\par 31 A poloze granice twoje od morza czerwonego az do morza Filistynskiego, a od puszczy az do rzeki; bo podam w rece wasze obywatele ziemi, i wypedzisz je od oblicza twego.
\par 32 Nie postanowisz z nimi, ani z bogami ich przymierza.
\par 33 Niech nie mieszkaja w ziemi twej, by cie snac nie przywiedli do grzechu przeciwko mnie, gdybys sluzyl bogom ich, co by tobie bylo sidlem.

\chapter{24}

\par 1 I rzekl do Mojzesza: Wstap do Pana ty i Aaron, Nadab i Abiu, i siedmdziesiat starszych Izraelskich, i pokloncie sie z daleka.
\par 2 A sam tylko Mojzesz wstapi do Pana; ale oni nie przybliza sie ani lud wstapi z nim.
\par 3 Przyszedl tedy Mojzesz, i opowiedzial ludowi wszystkie slowa Panskie, i wszystkie sady. I odpowiedzial wszystek lud glosem jednym, mówiac: Wszystkie slowa, które rzekl Pan, uczynimy.
\par 4 I napisal Mojzesz wszystkie slowa Panskie: a wstawszy rano zbudowal oltarz pod góra, i dwanascie slupów wedlug dwanascie pokolenia Izraelskiego.
\par 5 I poslal mlodzience z synów Izraelskich, którzy ofiarowali calopalenia; i ofiarowali za ofiary spokojne Panu, cielce.
\par 6 Zatem wzial Mojzesz polowe krwi i wlal w czasze, a polowe druga wylal na oltarz.
\par 7 I wziawszy ksiegi przymierza, czytal w uszach ludu; którzy rzekli: Cokolwiek mówil Pan, uczynmy, i posluszni bedziemy.
\par 8 Wzial tez Mojzesz krew, i pokropil lud i rzekl: Oto, krew przymierza, które Pan postanowil z wami, na wszystkie te slowa.
\par 9 I wstapil Mojzesz, i Aaron, Nadab, i Abiu, i siedemdziesiat starszych Izraelskich;
\par 10 I widzieli Boga Izraelskiego; a bylo pod nogami jego jako robota z kamienia szafirowego, a jako niebo gdy jest jasne.
\par 11 A na ksiazeta synów Izraelskich nie sciagnal Pan reki swej: choc widzieli Boga, przecie jedli i pili.
\par 12 Rzekl tedy Pan do Mojzesza: Wstap do mnie na góre, i badz tam a dam ci tablice kamienne, i zakon, i przykazanie którem napisal, abys ich nauczal.
\par 13 Wstal tedy Mojzesz i Jozue sluga jego; i wstapil Mojzesz na góre Boza.
\par 14 A do starszych rzekl: Zostancie tu, az sie wrócimy do was. A oto Aaron i Chur beda z wami; kto by mial sprawe jaka, niech idzie do nich.
\par 15 Tedy wstapil Mojzesz na góre, a oblok zakryl góre.
\par 16 I mieszkala chwala Panska na górze Synaj, a okryl ja oblok przez szesc dni; potem zawolal na Mojzesza dnia siódmego z posrodku obloku.
\par 17 A pozór chwaly Panskiej byl jako ogien pozerajacy na wierzchu góry przed oczyma synów Izraelskich.
\par 18 I wszedl Mojzesz w posrodek obloku, wstapiwszy na góre; i byl Mojzesz na górze czterdziesci dni i czterdziesci nocy.

\chapter{25}

\par 1 I rzekl Pan do Mojzesza mówiac:
\par 2 Mów do synów Izraelskich, aby mi zebrali podarek; od kazdego czlowieka, którego dobrowolnym uczyni serce jego, odbierac bedzie podarek mój.
\par 3 A ten jest podarek, który bedziecie brac od nich: zloto, i srebro, i miedz,
\par 4 I hijacynt, i szarlat, i karmazyn dwa kroc farbowany; i bialy jedwab, i siersc kozia;
\par 5 I skóry baranie czerwono farbowane, i skóry borsukowe, i drzewo sytym:
\par 6 Oliwe do swiecenia, wonne rzeczy na olejek pomazywania, i na wonne kadzenie;
\par 7 Kamienne onychiny, i kamienie ku osadzaniu naramiennika i napiersnika.
\par 8 I uczynia mi swiatnice, abym mieszkal w posrodku ich.
\par 9 Wedlug wszystkiego, jako ukaze tobie podobienstwo przybytku, i podobienstwo wszystkiego naczynia jego, tak uczynicie.
\par 10 Uczynia tez skrzynia z drzewa sytym; póltrzecia lokcia bedzie dlugosc jej a póltora lokcia szerokosci jej, a póltora lokcia wysokosc jej.
\par 11 I powleczesz ja zlotem czystem; z wierzchu i wewnatrz powleczesz ja, a uczynisz nad nia korone zlota w okolo.
\par 12 Ulejesz tez do niej cztery kolce zlote, które przyprawisz do czterech weglów jej; dwa kolce do jednego jej boku, i dwa kolce do drugiego jej boku.
\par 13 I uczynisz drazki z drzewa sytym, i powleczesz je zlotem.
\par 14 I przewleczesz drazki przez kolce na bokach skrzyni, aby na nich skrzynie noszono.
\par 15 W kolcach u skrzyni beda te drazki; nie beda ich odejmowac od niej.
\par 16 A wlozysz w te skrzynie swiadectwo, którec dam.
\par 17 Uczynisz tez ublagalnie ze zlota czystego: póltrzecia lokcia bedzie dlugosc jej, a póltora lokcia szerokosc jej.
\par 18 I uczynisz dwa Cheruby zlote: z ciagnionego zlota uczynisz je na obu koncach ublagalni.
\par 19 A uczynisz Cheruba jednego na jednym koncu, a Cheruba drugiego na drugim koncu; na ublagalni uczynicie Cheruby na obu koncach jej.
\par 20 A beda miec Cherubowie skrzydla rozciagnione z wierzchu, zakrywajac skrzydlami swemi ublagalnie; a twarze ich beda obrócone jednego ku drugiemu; ku ublagalni beda twarze Cherubów.
\par 21 I wlozysz ublagalnie na wierzch skrzyni, a do skrzyni wlozysz swiadectwo, którec dam.
\par 22 Tam sie z toba schodzic bede, i z toba rozmawiac z ublagalni, z posrodku dwu Cherubów, którzy beda nad skrzynia swiadectwa, o wszystkiem, coc rozkaze synom Izraelskim.
\par 23 Uczynisz tez stól z drzewa sytym: dwa lokcie bedzie dlugosc jego, a lokiec szerokosc jego, a póltora lokcia wysokosc jego.
\par 24 I powleczesz go zlotem czystem, a uczynisz mu korone zlota w okolo.
\par 25 Uczynisz tez w okolo niego listwe w szerz na cztery palce, i korone zlota w okolo listwy.
\par 26 Takze uczynisz do niego cztery kolce zlote, i przybijesz kolce na czterech rogach, które sa u czterech nóg jego.
\par 27 Pod ta listwa beda kolce, przez które przewloka drazki do noszenia stolu.
\par 28 A uczynisz te drazki z drzewa sytym, i powleczesz je zlotem, i bedzie na nich stól noszony.
\par 29 Sprawisz tez misy jego, i przystawki jego, i czasze jego, i kubki jego do nalewania ofiar mokrych; ze zlota szczerego porobisz je.
\par 30 I klasc bedziesz na ten stól chleby pokladne przed twarz moje ustawicznie.
\par 31 Urobisz tez swiecznik ze zlota szczerego, z ciagnionego zlota bedzie swiecznik ten; slupiec jego, prety jego, czaszki jego, galki jego, i kwiaty jego, z tegoz beda.
\par 32 A szesc pretów wychodzic bedzie ze stron jego: trzy prety swiecznika ze strony jego jednej, a trzy prety swiecznika ze strony jego drugiej.
\par 33 Trzy czaszki na ksztalt orzecha migdalowego na precie jednym, takze galka i kwiat; i trzy czaszki na ksztalt orzecha migdalowego na precie drugim, takze galka i kwiat; tak bedzie na wszystkich szesciu pretach, wychodzacych ze swiecznika.
\par 34 Ale na swieczniku beda cztery czaszki na ksztalt orzecha migdalowego, galki jego, i kwiaty jego.
\par 35 I bedzie galka pod dwiema pretami z niego, takze galka pod drugiemi dwiema pretami jego, i zas galka pod innemi dwiema pretami jego: tak bedzie pod szescia pretów z swiecznika wychodzacych.
\par 36 Galki ich, i prety ich z niego beda; to wszystko calokowane z szczerego zlota bedzie.
\par 37 Uczynisz tez siedem lamp jego, i zaswiecisz lampy jego, aby swiecily po stronach jego.
\par 38 Nozyczki tez jego, i kaganki jego ze zlota szczerego.
\par 39 Z talentu zlota szczerego uczynisz go, i wszystko naczynie jego.
\par 40 Patrzajze, abys uczynil wszystko wedlug podobienstwa tego, którec ukazano na górze.

\chapter{26}

\par 1 Przybytek tez uczynisz z dziesieciu opon, które beda z bialego jedwabiu kreconego, z hijacyntu, i z szarlatu, i z karmazynu dwa kroc farbowanego, i Cherubiny robota haftarska uczynisz.
\par 2 Dlugosc opony jednej osiem a dwadziescia lokci, a szerokosc opony jednej cztery lokcie: pod jedna miara beda wszystkie opony.
\par 3 Piec opon beda spinane, jedna z druga; takze drugie piec opon beda spinane, jedna z druga.
\par 4 I naczynisz petlic hijacyntowych na kraju opony jednej, gdzie sie kraje spinac maja; takze uczynisz na krajach opony drugiej, gdzie sie kraje spinac maja.
\par 5 Piecdziesiat petlic uczynisz na oponie jednej, a piecdziesiat petlic uczynisz po kraju opony, któremi sie spinac ma z druga; petlica jedna przeciw drugiej bedzie.
\par 6 Uczynisz tez piecdziesiat haczyków zlotych, a spoisz opone jedne z druga temi haczykami; i tak bedzie przybytek jeden.
\par 7 Urobisz tez opony z siersci koziej na namiot ku zakrywaniu przybytku z wierzchu; jedenascie takich opon urobisz.
\par 8 Dlugosc opony jednej trzydziesci lokci, a szerokosc opony jednej cztery lokcie; jednaz miara bedzie tych jedenastu opon.
\par 9 I zepniesz piec opon osobno, a szesc opon osobno; we dwoje zlozysz opone szósta na przodku namiotu.
\par 10 Uczynisz tez piecdziesiat petlic po kraju jednej opony, na koncu, gdzie sie ma spinac, i piecdziesiat petlic po kraju opony ku spinaniu drugiemu.
\par 11 Uczynisz tez haczyków miedzianych piecdziesiat, i zawiedziesz haczyki w petlice, i spoisz namiot, aby byl jeden.
\par 12 A co zas zbywa opon namiotowych, to jest pól opony zbywajacej, zawieszono bedzie w tyle przybytku.
\par 13 A lokiec z jednej, i lokiec z drugiej strony, który zbywa z dlugosci opon namiotu, bedzie wisial po stronach przybytku, tam i sam, zeby go okrywal.
\par 14 Uczynisz tez przykrycie na namiot z skór baranich czerwono farbowanych, i przykrycie z skór borsukowych na wierzch.
\par 15 Naczynisz tez do przybytku desek z drzewa sytym prosto stojacych.
\par 16 Dziesiec lokci dlugosc deski, a póltora lokcia szerokosc deski jednej.
\par 17 Dwa czopy deska jedna miec bedzie, na ksztalt stopniów wschodowych sporzadzone, jeden przeciw drugiemu; tak uczynisz u wszystkich desek przybytku.
\par 18 Uczynisz tez deski do przybytku, dwadziescia desek ku stronie poludniowej, ku wiatrowi poludniowemu.
\par 19 Czterdziesci zas podstawków urobisz srebrnych pod tych dwadziescia desek; dwa podstawki pod jedne deske do dwu czopów jej, takze dwa podstawki do deski drugiej do dwu czopów jej.
\par 20 Na drugim zas boku przybytku ku stronie pólnocnej, dwadziescia desek.
\par 21 A czterdziesci podstawków ich srebrnych; dwa podstawki pod jedne deske, i dwa podstawki pod druga deske.
\par 22 Ale na stronie przybytku ku zachodowi uczynisz szesc desek.
\par 23 A dwie deski uczynisz w kaciech przybytku w obydwu stronach.
\par 24 Które beda spojone od spodku, takze spolu spojone beda z wierzchu do jednego kolca; tak bedzie przy tych obu, które we dwu kaciech beda.
\par 25 A tak bedzie osiem desek, a podstawki ich srebrne; szesnascie podstawków, dwa podstawki pod deska jedna, a dwa podstawki pod deska druga.
\par 26 Uczynisz tez dragi z drzewa sytym; piec ich bedzie do desek jednej strony przybytku.
\par 27 Piec takze dragów do desek przybytku na druga strone; piec tez dragów do desek przybytku przestawajacych do obu weglów na zachód slonca.
\par 28 Ale drag posredni w posrodku desek przewleczony bedzie od jednego konca do drugiego.
\par 29 One tez deski powleczesz zlotem, a poczynisz do nich kolce zlote, przez które maja byc przewleczone dragi; powleczesz tez i dragi zlotem.
\par 30 Wystawisz tedy przybytek na ten ksztalt, któryc ukazano na górze.
\par 31 Uczynisz tez zaslone z hijacyntu, i z szarlatu, i z karmazynu dwa kroc farbowanego, i z bialego jedwabiu kreconego; robota haftarska uczynisz ja z Cherubiny.
\par 32 I zawiesisz ja na czterech slupach z drzewa sytym powleczonych zlotem, (których tez haki zlote) na czterech podstawkach srebrnych.
\par 33 A zawiesisz zaslone na haczykach, i wniesiesz za zaslone skrzynie swiadectwa, a dzielic wam bedzie ta zaslona swiatnice od swiatnicy najswietszej.
\par 34 Polozysz tez ublagalnie na skrzyni swiadectwa w swiatnicy najswietszej.
\par 35 A postawisz stól przed zaslona, a swiecznik przeciw stolowi przy stronie przybytku na poludnie, a stól postawisz przy stronie pólnocnej.
\par 36 Uczynisz tez zaslone do drzwi przybytku z hijacyntu, i z szarlatu, i z karmazynu dwa kroc farbowanego, i z jedwabiu bialego kreconego, robota haftarska.
\par 37 A uczynisz do tej zaslony piec slupów z drzewa sytym, które powleczesz zlotem; haki ich beda zlote, a ulejesz do nich piec podstawków miedzianych.

\chapter{27}

\par 1 Uczynisz tez oltarz z drzewa sytym na piec lokci wzdluz, a na piec lokci wszerz; czworograniasty bedzie oltarz, a na trzy lokcie wzwyz.
\par 2 I poczynisz mu rogi na czterech weglach jego; z niego beda rogi jego, i obijesz je miedzia.
\par 3 Poczynisz tez do niego kotly dla zsypowania popiolu; i miotly jego, i miednice jego, i widelki jego, i lopaty jego, wszystkie naczynia jego uczynisz z miedzi.
\par 4 Uczynisz tez do niego krate, na ksztalt sieci, miedziana; a uczynisz u tej kraty cztery kolce miedziane na czterech rogach jej.
\par 5 I wlozysz ja w okrag oltarza na dól, a bedzie ta krata az do polowy oltarza.
\par 6 Porobisz tez drazki do oltarza, drazki z drzewa sytym, a obijesz je miedzia.
\par 7 Które drazki przewleczone beda przez kolce; a beda te drazki na obydwu stronach oltarza, gdy go nosic beda.
\par 8 Aby byl czczy wewnatrz, uczynisz go z desek; jakoc ukazano na górze, tak go uczynia.
\par 9 Uczynisz tez sien przybytku na poludnie ku prawej stronie; opony tej sieni beda z bialego jedwabiu kreconego; na sto lokci wzdluz bedzie strona jego.
\par 10 Slupów tez do nich dwadziescia, a podstawków do nich dwadziescia miedzianych; glówki na slupach, i okrecenia ich beda srebrne.
\par 11 Na tenze ksztalt na stronie pólnocnej wzdluz opony beda, sto lokci wzdluz; slupów tez do nich dwadziescia, a podstawków do nich dwadziescia miedzianych; glówki na slupach i okrecenia ich srebrne.
\par 12 A szerokosc sieni od strony zachodniej bedzie miala opony na piecdziesiat lokci; slupów ich dziesiec, i podstawków ich dziesiec.
\par 13 Szerokosc zas sieni na przedniej stronie, na wschód slonca, piecdziesiat lokci.
\par 14 Pietnascie tez lokci opon na jedne strone; slupów ich trzy i podstawków ich trzy.
\par 15 Na drugiej zas stronie opon pietnascie lokci; slupów ich trzy i podstawków ich trzy.
\par 16 A do bramy sieni zaslona na dwadziescia lokci z hijacyntu, i z szarlatu, i z karmazynu dwa kroc farbowanego, i z jedwabiu bialego kreconego robota haftarska; slupów jej cztery, i podstawków jej cztery.
\par 17 Wszystkie slupy sieni w okolo otoczone beda srebrem; glówki ich srebrne, a podstawki ich miedziane.
\par 18 Dlugosc sieni na sto lokci, a szerokosc na piecdziesiat, wszedzie jednostajna; a wysokosc na piec lokci, z bialego jedwabiu kreconego, a podstawki jej miedziane.
\par 19 Wszystkie naczynia przybytku do wszelakiej uslugi jego, i wszystkie kolki jego, i wszystkie kolki sieni, miedziane beda.
\par 20 A ty rozkazesz synom Izraelskim, aby przyniesli do ciebie oliwy z oliwnego drzewa czystej, wytloczonej, do swiecenia, aby lampy zawsze gorzaly.
\par 21 W przybytku zgromadzenia przed zaslona, która zakrywa skrzynie swiadectwa, stawiac je bedzie Aaron i synowie jego od wieczora az do poranku przed Panem. Ta ustawa bedzie wieczna w potomstwie ich miedzy synami Izraelskimi.

\chapter{28}

\par 1 A ty wezmij do siebie Aarona, brata twego, i syny jego z nim, z posrodku synów Izraelskich, aby mi urzad kaplanski odprawowali, Aaron, Nadab i Abiu, Eleazar, i Itamar, synowie Aaronowi.
\par 2 A sprawisz szaty swiete Aaronowi, bratu twemu, na czesc i na ozdobe.
\par 3 Ty sie tez rozmówisz z kazdym umiejetnym rzemieslnikiem, któregom napelnil Duchem madrosci, aby urobili szaty Aaronowi na poswiecenie jego, aby mi urzad kaplanski odprawowal.
\par 4 A tec sa szaty, które urobia: Napiersnik, i naramiennik, i plaszcz, i suknia haftowana, czapka i pas. I urobia te szaty swiete Aaronowi bratu twemu i synom jego, aby mi kaplanski urzad sprawowali.
\par 5 I nabiora zlota, i hijacyntu, i szarlatu, i karmazynu dwa kroc farbowanego, i jedwabiu bialego.
\par 6 I uczynia naramiennik ze zlota, i z hijacyntu, i z szarlatu, z karmazynu dwa kroc farbowanego, i z jedwabiu bialego kreconego, robota haftarska.
\par 7 Dwa zwierzchne kraje zszyte miec bedzie na dwu koncach swych, a tak spolu spiete beda.
\par 8 A przepasanie naramiennika tego, które na nim bedzie, podobne bedzie robocie jego; bedzie takze ze zlota, z hijacyntu, i z szarlatu, i z karmazynu dwa kroc farbowanego, i z jedwabiu bialego kreconego.
\par 9 I wezmiesz dwa kamienie onychiny, i wyryjesz na nich imiona synów Izraelskich;
\par 10 Szesc imion ich na jednym kamieniu, a imion szesc drugich na drugim kamieniu, wedlug narodzenia ich.
\par 11 Robota snycerzów, którzy kamienie rzeza, wyryjesz na obu kamieniach imiona synów Izraelskich, i osadzisz je we zlote osadzenia.
\par 12 I polozysz te obadwa kamienie na wierzchnich krajach naramiennika, kamienie pamiatki dla synów Izraelskich; i nosic bedzie Aaron imiona ich przed Panem na obu ramionach swych na pamiatke.
\par 13 Uczynisz tez haczyki zlote.
\par 14 Dwa tez lancuszki ze zlota szczerego jednostajne; uczynisz je robota pleciona, i zawiesisz te lancuszki plecione na haczykach.
\par 15 Uczynisz tez napiersnik sadu robota haftarska, wedlug roboty naramiennika urobisz go; ze zlota, z hijacyntu, i z szarlatu, i z karmazynu dwa kroc farbowanego, i z bialego jedwabiu kreconego uczynisz go.
\par 16 Czworograniasty bedzie i dwoisty, na piedzi dlugosc jego, i na piedzi szerokosc jego.
\par 17 I nasadzisz wen pelno kamienia, cztery rzedy kamienia, tym porzadkiem: sardyjusz, topazyjusz i szmaragd w pierwszym rzedzie;
\par 18 W drugim zasie rzedzie: karbunkul, szafir, i jaspis.
\par 19 A w trzecim rzedzie: linkuryjusz, achates, i ametyst.
\par 20 A w czwartym rzedzie: chryzolit, onychin i beryl; te beda wsadzone w zloto w rzedziech swoich.
\par 21 A tych kamieni z imionami synów Izraelskich bedzie dwanascie wedlug imion ich; tak jako rzeza pieczeci, kazdy wedlug imienia swego beda, dla dwunastu pokolen.
\par 22 Uczynisz tez do napiersnika lancuszki jednostajne robota pleciona ze zlota szczerego.
\par 23 Uczynisz tez do napiersnika dwa kolce zlote, i przyprawisz te dwa kolce do obu krajów napiersnika.
\par 24 I przewleczesz dwa lancuszki zlote przez oba kolce u krajów napiersnika.
\par 25 Drugie zasie dwa kolce dwu lancuszków zawleczesz na dwa haczyki, i przyprawisz do wierzchnich krajów naramiennika na przodku.
\par 26 Uczynisz tez dwa kolce zlote, które przyprawisz do dwu konców napiersnika na kraju jego, który jest od naramiennika ze spodku.
\par 27 Do tego uczynisz dwa drugie kolce zlote, które przyprawisz na dwie strony naramiennika ze spodku na przeciwko spojeniu jego, z wierzchu nad przepasaniem naramiennika.
\par 28 Tak zwiaza napiersnik ten kolce jego z kolcami naramiennika sznurem hijacyntowym, aby byl nad przepasaniem naramiennika, zeby nie odstawal napiersnik od naramiennika.
\par 29 I bedzie nosil Aaron imiona synów Izraelskich na napiersniku sadu, na piersiach swych, gdy bedzie wchodzil do swiatnicy, na pamiatke ustawiczna przed Panem.
\par 30 Polozysz tez na napiersniku sadu Urim i Tummim, które beda na piersiach Aaronowych, gdy wchodzic bedzie przed Pana; i poniesie Aaron sad synów Izraelskich na piersiach przed Panem ustawicznie.
\par 31 Uczynisz tez plaszcz pod naramiennik, wszystek z hijacyntu.
\par 32 A na wierzchu w posród jego bedzie rozpór, który rozpór obwiedziesz brama pleciona w pancerzowy wzór, aby sie nie rozdzieral.
\par 33 Uczynisz tez na podolku jego jablka granatowe z hijacyntu, i z szarlatu, i z karmazynu dwa kroc farbowanego na podolku jego w okolo, a dzwonki zlote miedzy niemi w okolo.
\par 34 Dzwonek zloty a jablko granatowe; i zas dzwonek zloty i jablko granatowe u podolka plaszcza w okolo.
\par 35 A bedzie to mial na sobie Aaron przy poslugiwaniu, aby slyszany byl dzwiek jego, gdy bedzie wchodzil do swiatnicy przed Pana, i gdy zas wychodzic bedzie, zeby nie umarl.
\par 36 Uczynisz tez blache ze zlota szczerego, a wyryjesz na niej robota tych, co pieczeci rzeza: Swietosc Panu.
\par 37 Te przywiazesz do sznuru hijacyntowego, i bedzie na czapce; na przodku na czapce bedzie.
\par 38 A ta bedzie nad czolem Aaronowem, aby nosil Aaron nieprawosc poswieconych rzeczy, które by poswiecali synowie Izraelscy przy wszystkich darach poswieconych rzeczy swych; a bedzie nad czolem jego ustawicznie, aby im zjednal laske u Pana.
\par 39 Sprawisz tez szate z bialego jedwabiu dziana; takze uczynisz czapke z jedwabiu bialego, pas tez uczynisz robota haftarska.
\par 40 Synom takze Aaronowym poczynisz szaty; i poczynisz im pasy, i czapki im poczynisz na czesc i na ozdobe.
\par 41 A ubierzesz w nie Aarona, brata twego, i syny jego z nim; i pomazesz je, a napelnisz rece ich, i poswiecisz je, aby mi urzad kaplanski sprawowali.
\par 42 Urobisz im tez ubiory lniane, dla zakrycia nagosci ciala; od biódr az do udów beda.
\par 43 A beda na Aaronie i na synach jego, gdy wchodzic beda do namiotu zgromadzenia, albo gdy beda przystepowac do oltarza, aby sluzyli w swiatnicy, zeby niosac nieprawosc, nie pomarli. Ustawa to wieczna bedzie jemu, i nasieniu jego po nim.

\chapter{29}

\par 1 To tez uczynisz im na poswiecenie ich, aby mi odprawowali urzad kaplanski: Wezmij cielca jednego mlodego, i dwu baranów zupelnych;
\par 2 I chleby przasne, i placki przasne z oliwa zaczynione, i kolacze przasne, namazane oliwa; z przedniej maki pszenicznej naczynisz ich.
\par 3 A wlozysz to w jeden kosz, ofiarowac je bedziesz w tymze koszu, z cielcem, i z dwiema barany.
\par 4 A Aaronowi i synom jego przystapic kazesz do drzwi namiotu zgromadzenia, i omyjesz je woda,
\par 5 A wziawszy szaty, obleczesz Aarona w suknia, i w plaszcz pod naramiennik, i w naramiennik, i napiersnik, i opaszesz go pasem naramiennika;
\par 6 I wlozysz czapke na glowe jego, a wstawisz korone swietosci na czapke.
\par 7 Na ostatek wezmiesz olejek pomazywania, i wylejesz na glowe jego, a pomazesz go.
\par 8 Potem synom jego przystapic kazesz, a obleczesz je w szaty;
\par 9 I opaszesz je pasem, Aarona i syny jego, a wlozysz na nie czapki, i beda mieli kaplanstwo ustawa wieczna; poswiecisz tez rece Aaronowe, i rece synów jego.
\par 10 Przywiedziesz tez cielca przed namiot zgromadzenia, i wlozy Aaron i synowie jego rece swoje na glowe cielca.
\par 11 I zabijesz cielca przed Panem, u drzwi namiotu zgromadzenia.
\par 12 A wziawszy krwi z cielca pomazesz na rogach oltarza palcem swym, a ostatek krwi wylejesz ku spodku oltarza.
\par 13 Wezmiesz tez wszystke tlustosc okrywajaca wnetrze, i odzieczke z watroby, i dwie nerki z tlustoscia ich, a zapalisz to na oltarzu.
\par 14 A mieso cielca, i skóre jego, i gnój jego, spalisz ogniem za obozem; bo to jest ofiara za grzech.
\par 15 Barana takze jednego wezmiesz, na którego glowe Aaron i synowie jego wloza rece swoje.
\par 16 I zabijesz barana tego, a wziawszy krwi jego, pokropisz wierzch oltarza w okolo.
\par 17 A barana zrabiesz na sztuki, i opluczesz trzewa jego i nogi jego, i wlozysz je na sztuki z niego i na glowe jego.
\par 18 I zapalisz tego calego barana na oltarzu; calopalenie to jest Panu, wonia przyjemna, ofiara ognista jest Panu.
\par 19 Zatem wezmiesz barana drugiego, a wlozy Aaron i synowie jego rece swoje na glowe barana.
\par 20 A zabiwszy onego barana wezmiesz ze krwi jego, i pomazesz koniec ucha Aaronowego, i konce ucha prawego synów jego, i wielkie palce reki ich prawej, takze wielkie palce nogi ich prawej, a wylejesz te krew na oltarz w okolo.
\par 21 Wziawszy zas ze krwi, która na oltarzu, takze z olejku pomazywania, pokropisz Aarona, i szaty jego, i szaty synów jego z nim; i bedzie poswiecony on i szaty jego, i synowie jego, i szaty synów jego z nim.
\par 22 Potem wezmiesz z barana lój, i ogon, i tlustosc, która okrywa wnetrze, i odzieczke watroby, i dwie nerki, i lój, który jest na nich, i lopatke prawa, albowiem jest baran poswiecenia;
\par 23 I bochen chleba jeden, i kolacz chleba z oliwa jeden, i placek jeden z kosza przasników, który jest przed Panem.
\par 24 A polozysz to wszystko na rece Aaronowe, i na rece synów jego, i obracac to bedziesz tam i sam za ofiare obracania przed Panem;
\par 25 A wziawszy to z reku ich, zapalisz na oltarzu, na calopalenie, na wonnosc wdzieczna przed Panem; ofiara ognista jest Panu.
\par 26 Wezmiesz tez piersi z barana poswiecenia, które naleza Aaronowi, i obracac je bedziesz tam i sam za ofiare obracania przed Panem, a to bedzie dzial twój,
\par 27 Poswiecisz tez piersi obracania i lopatke podnoszenia, która obracano, i która podnoszono, z barana poswiecenia dla Aarona, i dla synów jego.
\par 28 A to bedzie Aaronowi i synom jego ustawa wieczna od synów Izraelskich, gdyz ofiara podnoszenia jest: i ofiara podnoszenia bedzie od synów Izraelskich z ofiar ich spokojnych, ofiara podnoszenia ich bedzie Panu.
\par 29 A szaty swiete, które sa Aaronowe, zostana synom jego po nim, aby pomazywani byli w nich, a byly poswiecane w nich rece ich.
\par 30 Siedem dni bedzie w nich chodzil kaplan, który bedzie na jego miejscu z synów jego, który wchodzic bedzie do namiotu zgromadzenia, aby sluzyl w swiatnicy.
\par 31 Barana tez poswiecenia wezmiesz, i uwarzysz mieso jego na miejscu swietem.
\par 32 I jesc beda Aaron i synowie jego mieso onego barana, i chleb, który jest w koszu, u drzwi namiotu zgromadzenia.
\par 33 Beda to jesc ci, za które sie oczyszczenie stalo, ku poswieceniu rak ich, aby poswieceni byli; obcy zas nie bedzie jadl z tego, bo swieta rzecz jest.
\par 34 A zbyloliby co miesa poswiecenia, i chleba az do poranku, spalisz ostatki ogniem: nie beda tego jesc, bo swieta rzecz jest.
\par 35 Tak tedy uczynisz Aaronowi, i synom jego, wedlug wszystkiego, com ci przykazal; przez siedem dni poswiecac bedziesz rece ich.
\par 36 Cielca tez na grzech ofiarowac bedziesz na kazdy dzien na oczyszczenie, i oczyscisz oltarz, czyniac oczyszczenie na nim, i pomazesz go ku poswieceniu jego.
\par 37 Siedem dni bedziesz oczyszczal oltarz, i poswiecisz go, i bedzie ten oltarz najswietszy; cózkolwiek sie dotknie oltarza, poswiecono bedzie,
\par 38 A to jest, co ofiarowac bedziesz na oltarzu: dwa baranki roczne, dwa na kazdy dzien ustawicznie.
\par 39 Baranka jednego ofiarowac bedziesz rano, a baranka drugiego ofiarowac bedziesz miedzy dwoma wieczorami.
\par 40 Takze dziesiata czesc efy maki pszennej, zmieszanej z oliwa wytloczona, której by bylo czwarta czesc hyn, a do ofiary mokrej czwarta czesc hyn wina do jednego baranka.
\par 41 Takze baranka drugiego ofiarowac bedziesz miedzy dwoma wieczorami; wedlug obrzedu ofiary porannej i wedlug ofiary mokrej jej, tak przy niej uczynisz nad wonia przyjemna, i ofiare zapalona Panu.
\par 42 Calopalenie to ustawicznie bedzie w narodziech waszych u drzwi namiotu zgromadzenia przed Panem, gdzie sie z wami schodzic bede, abym tam z toba rozmawial.
\par 43 Tam sie tez schodzic bede z synami Izraelskimi, i bedzie miejsce to chwala moja.
\par 44 Bo poswiece namiot zgromadzenia, i oltarz, i Aarona, i syny jego poswiece, aby mi urzad kaplanski sprawowali.
\par 45 I bede mieszkal w posrodku synów Izraelskich, i bede im za Boga.
\par 46 A poznaja, zem Ja Pan Bóg ich, którym je wywiódl z ziemi Egipskiej, abym mieszkal w posrodku ich, Ja Pan Bóg ich.

\chapter{30}

\par 1 Uczynisz tez oltarz dla kadzenia; z drzewa sytym uczynisz go.
\par 2 Na lokiec wzdluz, i na lokiec wszerz, czworograniasty bedzie, a na dwa lokcie wzwyz; z niego wychodzic beda rogi jego.
\par 3 A powleczesz go szczerem zlotem, wierzch jego i sciany jego w okolo, i rogi jego. Uczynisz tez korone zlota okolo niego.
\par 4 I dwa kolce zlote uczynisz tez pod korona we dwu katach jego, po obu stronach jego, a przez nie przewleczesz drazki, aby noszony byl na nich.
\par 5 A uczynisz drazki one z drzewa sytym, i powleczesz je zlotem.
\par 6 I postawisz go przed zaslona, za która jest skrzynia swiadectwa przed ublagalnia, która jest nad swiadectwem, gdzie sie z toba schodzic bede.
\par 7 A bedzie kadzil na nim Aaron kadzeniem z wonnych rzeczy na kazdy poranek; przygotowawszy lampy, bedzie kadzil.
\par 8 Takze gdy rozpali Aaron lampy miedzy dwoma wieczorami, kadzic bedzie kadzeniem ustawicznem przed Panem w narodziech waszych.
\par 9 Nie wlozycie nan kadzidla obcego, ani calopalenia, ani ofiary suchej; ani ofiary mokrej ofiarowac bedziecie na nim.
\par 10 Tylko wykona oczyszczenie Aaron nad rogami jego raz w rok; przez krew ofiary za grzech, w dzien oczyszczenia, raz w rok oczyszczenie odprawi na nim w narodziech waszych; bo to rzecz najswietsza Panu.
\par 11 Zatem rzekl Pan do Mojzesza, mówiac:
\par 12 Gdy zbierzesz glówna sume synów Izraelskich, z tych, którzy maja isc w liczbe, da kazdy okup za dusze swa Panu, gdy je liczyc bedziesz, aby nie przyszla na nie plaga, gdy zliczeni beda.
\par 13 To dawac beda: kazdy, który idzie w liczbe, da pól sykla wedlug sykla swiatnicy dwadziescia pieniedzy sykiel wazy; pól sykla bedzie podarek Panu.
\par 14 Ktokolwiek idzie w liczbe ode dwudziestu lat i wyzej, odda podarek Panu.
\par 15 Bogaty nie da wiecej, a ubogi nie da mniej nad pól sykla, gdy beda dawac ofiare Panu, dla oczyszczenia dusz swoich.
\par 16 A wybrawszy pieniadze oczyszczenia od synów Izraelskich, dasz je na potrzeby namiotu zgromadzenia, co bedzie synom Izraelskim na pamiatke przed Panem, ku oczyszczeniu dusz waszych.
\par 17 Potem rzekl Pan do Mojzesza, mówiac:
\par 18 Uczynisz tez wanne miedziana, i stolec jej miedziany do umywania, a postawisz ja miedzy namiotem zgromadzenia, i miedzy oltarzem, i nalejesz w nia wody.
\par 19 I umywac beda Aaron i synowie jego z niej rece swoje i nogi swoje.
\par 20 Gdy wchodzic beda do namiotu zgromadzenia, umywac sie beda woda, aby nie pomarli; takze gdyby mieli przystepowac do oltarza, aby sluzyli, i zapalili ofiare ognista Panu.
\par 21 I beda umywali rece swoje i nogi swoje, aby nie pomarli; i bedzie im to ustawa wieczna, jemu i nasieniu jego, w rodzaju ich.
\par 22 Rzekl jeszcze Pan do Mojzesza, mówiac:
\par 23 Ty tez wezmij sobie wonnych rzeczy przednich: Myrry co najczystszej piecset lutów, a cynamonu wonnego polowe tego, to jest, dwiescie i piecdziesiat lutów, i tatarskiego ziela dwiescie i piecdziesiat;
\par 24 Kasyi tez piecset lutów wedlug sykla swiatnicy, i oliwy z drzew oliwnych hyn.
\par 25 I uczynisz z tego olejek pomazywania swietego, masc najwyborniejsza, robota aptekarska: olejek to pomazywania swietego bedzie.
\par 26 I pomazesz nim namiot zgromadzenia, i skrzynie swiadectwa.
\par 27 Takze stól i wszystkie naczynia jego, i swiecznik, i naczynia jego, i oltarz, na którym kadza;
\par 28 Oltarz tez do calopalenia ze wszystkiem naczyniem jego, i wanne z stolcem jej.
\par 29 A poswiecisz je, aby najswietsze byly; cokolwiek sie ich dotknie, poswiecone bedzie.
\par 30 Aarona tez, i syny jego pomazesz, i poswiecisz je, aby mi sprawowali urzad kaplanski.
\par 31 A synom Izraelskim tak powiesz, mówiac: Olejek pomazywania swietego mnie bedzie swietym w narodziech waszych;
\par 32 Cialo czlowiecze nie bedzie nim mazane, a wedlug zlozenia jego nie uczynicie temu podobnego: bo swiety jest, i swiety wam bedzie.
\par 33 Ktobykolwiek uczynil taka masc, a namazalby nia kogo obcego, wytracony bedzie z ludu swego.
\par 34 I rzekl Pan do Mojzesza: Wezmij sobie rzeczy wonnych, balsamu, i onychy, i galbanu wonnego, i kadzidla czystego, wszystkiego w równej wadze;
\par 35 A uczynisz z tego kadzenia wonne robota aptekarska; to zmieszanie czyste i swiete bedzie.
\par 36 A utluklszy to mialko, klasc bedziesz z niego przed swiadectwem w namiocie zgromadzenia, gdzie sie z toba schodzic bede; najswietsze to bedzie.
\par 37 Kadzenia tez, które bys czynil wedlug zlozenia tego, nie uczynicie sobie; toc bedzie swieta rzecza dla Pana.
\par 38 Ktobykolwiek uczynil co podobnego, aby wonial z niego, wytracony bedzie z ludu swego.

\chapter{31}

\par 1 Potem rzekl Pan do Mojzesza, mówiac:
\par 2 Otom wezwal z imienia Besaleela, syna Urowego, syna Churowego z pokolenia Judy.
\par 3 I napelnilem go Duchem Bozym, madroscia, i rozumem, i umiejetnoscia we wszelakiem rzemiosle.
\par 4 Ku dowcipnemu wymyslaniu, cokolwiek moze byc urobione ze zlota, i z srebra, i z miedzi.
\par 5 Do rzezania kamienia na osadzenie, i na wyrobienie drzewa ku wystawieniu kazdej roboty.
\par 6 A oto, Ja przydalem mu Acholijaba, syna Achysamechowego z pokolenia Dan, a w serce kazdego dowcipnego dalem madrosc, aby zrobili wszystko com ci przykazal.
\par 7 Namiot zgromadzenia, i skrzynie swiadectwa, i ublagalnia, która ma byc nad nia, i wszystkie naczynia namiotu.
\par 8 Stól takze i naczynia jego, i swiecznik czysty ze wszystkiem naczyniem jego, i oltarz do kadzenia.
\par 9 Takze oltarz do calopalenia ze wszystkiem naczyniem jego, i wanne ze stolcem jej.
\par 10 Takze szaty do sluzby, i szaty swiete Aaronowi kaplanowi, i szaty synom jego ku sprawowaniu kaplanstwa.
\par 11 I olejek pomazywania, i kadzenie wonne do swiatnicy; wedlug wszystkiego, jakom ci rozkazal, uczynia.
\par 12 Potem rzekl Pan do Mojzesza, mówiac:
\par 13 Ty tez powiedz synom Izraelskim, mówiac: Przecie sabbatów moich przestrzegac bedziecie; bo ten znak jest miedzy mna i miedzy wami w narodziech waszych, abyscie wiedzieli, zem Ja Pan, który was poswiecam.
\par 14 Przetoz przestrzegajcie sabbatu, swiety bowiem jest wam. Kto by go zgwalcil, smiercia umrze; bo kazdy, coby wen robote odprawowal, wytracona bedzie dusza jego, z posrodku ludu swego.
\par 15 Przez szesc dni odprawowana bedzie robota; ale w dzien siódmy sabbat jest, odpocznienie swiete Panu; kazdy, kto by robil robote w dzien sabbatu, smiercia umrze.
\par 16 Przetoz beda strzec synowie Izraelscy sabbatu, zachowujac sabbat w narodziech swych ustawa wieczna.
\par 17 Miedzy mna i miedzy syny Izraelskimi znakiem jest wiecznym; bo w szesciu dniach uczynil Pan niebo i ziemie, a dnia siódmego przestal i odpoczal.
\par 18 I dal Pan Mojzeszowi dokonawszy mowy z nim na górze Synaj dwie tablice swiadectwa, tablice kamienne, pisane palcem Bozym.

\chapter{32}

\par 1 A widzac lud, iz omieszkiwal Mojzesz zejsc z góry, tedy zebral sie lud przeciw Aaronowi, i mówili do niego: Wstan, uczyn nam bogi, którzy by szli przed nami; bo Mojzeszowi, mezowi temu, który nas wywiódl z ziemi Egipskiej, nie wiemy co sie stalo.
\par 2 Tedy im rzekl Aaron: Odejmijcie nausznice zlote, które sa na uszach zon waszych, synów waszych, i córek waszych, a przyniescie do mnie.
\par 3 I poodrywal wszystek lud nausznice zlote, które byly na uszach ich, a przyniesli do Aarona.
\par 4 Które gdy odebral z reku ich, wyksztaltowal je rylcem i uczynil z nich cielca odlewanego. I rzekli: Ci sa bogowie twoi, Izraelu, którzy cie wywiedli z ziemi Egipskiej.
\par 5 Co ujrzawszy Aaron, zbudowal oltarz przed nim; a wolajac Aaron mówil: Swieto Panskie jutro bedzie.
\par 6 A wstawszy bardzo rano nazajutrz, ofiarowali calopalenia, i przywiedli ofiary spokojne; i siadl lud, aby jadl i pil, i wstali grac.
\par 7 Tedy rzekl Pan do Mojzesza: Idz zstap; bo sie popsowal lud twój, którys wywiódl z ziemi Egipskiej.
\par 8 Ustapili predko z drogi, któram im przykazal; uczynili sobie cielca odlewanego, i klaniali sie mu, i ofiarowali mu mówiac: Ci sa bogowie twoi, Izraelu, którzy cie wywiedli z ziemi Egipskiej.
\par 9 Rzekl zasie Pan do Mojzesza: Widzialem lud ten, a oto, jest lud twardego karku.
\par 10 Przetoz teraz pusc mie, ze sie rozpali popedliwosc moja na nie i wygladze je; a ciebie uczynie w naród wielki.
\par 11 I modlil sie Mojzesz Panu Bogu swemu, a rzekl: Przeczze o Panie, rozpala sie popedliwosc twoja przeciwko ludowi twemu, którys wywiódl z ziemi Egipskiej moca wielka i reka mozna?
\par 12 A przeczzeby Egipczanie rzec mieli, mówiac: Na ich zle wywiódl je, aby je pobil na górach, i aby je wygladzil z wierzchu ziemi? Odwróc sie od gniewu zapalczywosci twojej, a ulituj sie nad zlem ludu twego.
\par 13 Wspomnij na Abrahama, Izaaka, i Izraela, slugi twoje, którymes przysiagl sam przez sie i mówiles do nich: Rozmnoze nasienie wasze jako gwiazdy niebieskie, i wszystke te ziemie, o którejm mówil. Dam ja nasieniu waszemu, i odziedzicza ja na wieki.
\par 14 I uzalil sie Pan nad zlem, które mówil, ze uczynic mial ludowi swemu.
\par 15 A obróciwszy sie Mojzesz zstapil z góry, dwie tablice swiadectwa majac w reku swych, tablice pisane po obu stronach; i na tej, i na owej stronie byly pisane.
\par 16 A one tablice robota Boza byly; pismo takze pismo Boze bylo, wyryte na tablicach.
\par 17 A uslyszawszy Jozue glos ludu wolajacego, rzekl do Mojzesza: Glos bitwy w obozie.
\par 18 Który odpowiedzial: Nie jest to glos zwyciezajacych, ani glos porazonych: glos spiewajacych ja slysze.
\par 19 I stalo sie, gdy sie przyblizyl do obozu, ze ujrzal cielca i tance; a rozgniewawszy sie bardzo Mojzesz, porzucil z reku swoich tablice, i stlukl je pod góra.
\par 20 Wzial tez cielca, którego byli uczynili, i spalil go w ogniu, i skruszyl go az na proch, a wysypawszy na wode, dal pic synom Izraelskim.
\par 21 I rzekl Mojzesz do Aarona: Cóz ci ten lud uczynil, zes wprowadzil nan grzech wielki?
\par 22 Odpowiedzial Aaron: Niech sie nie rozpala gniew pana mego; ty znasz ten lud, jako do zlego sklonny jest.
\par 23 Bo mi mówili: Uczyn nam bogi, którzy by szli przed nami, gdyz Mojzeszowi, mezowi temu, który nas wywiódl z ziemi Egipskiej, nie wiemy, co sie stalo.
\par 24 I odpowiedzialem im: Kto ma zloto, odrywajcie je z siebie. I dali mi, i wrzucilem je w ogien, i ulal sie ten cielec.
\par 25 Widzac tedy Mojzesz lud obnazony, (bo go byl zlupil Aaron na zelzenie przed nieprzyjaciolmi ich).
\par 26 Stanal Mojzesz w bramie obozu, i rzekl: Kto Panski, przystap do mnie. I zebrali sie do niego wszyscy synowie Lewiego.
\par 27 I rzekl do nich: Tak mówi Pan, Bóg Izraelski: Przypasz kazdy miecz swój do biodry swojej; przychodzcie a wracajcie sie od bramy do bramy w obozie, a zabijajcie kazdy brata swego, i kazdy przyjaciela swego, i kazdy blizniego swego.
\par 28 I uczynili synowie Lewiego wedlug slowa Mojzeszowego; i poleglo z ludu dnia onego okolo trzech tysiecy mezów.
\par 29 Bo byl rzekl Mojzesz: Poswieccie rece swoje dzis Panu, kazdy na synu swym, i na bratu swym, aby wam dane bylo dzis blogoslawienstwo.
\par 30 A gdy bylo nazajutrz, mówil Mojzesz do ludu: Wyscie zgrzeszyli grzechem wielkim; przetoz teraz wstapie do Pana, aza go ublagam za grzech wasz.
\par 31 Wróciwszy sie tedy Mojzesz do Pana, mówil: Prosze, zgrzeszyl ten lud grzechem wielkim; bo sobie uczynili bogi zlote.
\par 32 Teraz tedy, albo odpusc grzech ich, albo jezli nie, wymaz mie prosze z ksiag twoich, któres napisal.
\par 33 I rzekl Pan do Mojzesza: Kto mi zgrzeszyl, tego wymaze z ksiag moich.
\par 34 A teraz idz, prowadz ten lud, gdziem ci rozkazal. Oto, Aniol mój pójdzie przed toba; ale w dzien nawiedzenia mego nawiedze tez i na nich ten grzech ich.
\par 35 Skaral tedy Pan lud, przeto, ze uczynili byli cielca, którego byl uczynil Aaron.

\chapter{33}

\par 1 Potem mówil Pan do Mojzesza: Idz, rusz sie stad, ty i lud, którys wywiódl z ziemi Egipskiej, do ziemi, o któram przysiagl Abrahamowi, Izaakowi i Jakóbowi, mówiac: Nasieniu twemu dam ja.
\par 2 I posle przed toba Aniola, i wyrzuce Chananejczyka, Amorejczyka, i Hetejczyka, i Ferezejczyka, Hewejczyka, i Jebuzejczyka.
\par 3 Do ziemi oplywajacej mlekiem i miodem; lecz sam nie pójde z toba, gdyzes jest lud karku twardego, bym cie snac nie wytracil w drodze.
\par 4 A uslyszawszy lud te rzecz zla, zasmucil sie, i nie wlozyl zaden ochedóstwa swego na sie.
\par 5 Albowiem rzekl Pan do Mojzesza: Powiedz synom Izraelskim: Wyscie ludem twardego karku; przyjde kiedy z nagla w posród ciebie, i wygladze cie. Przetoz teraz zlóz ochedóstwo twoje z siebie, a bede wiedzial, coc bym uczynic mial.
\par 6 I zlozyli synowie Izraelscy ochedóstwo swoje przy górze Horeb.
\par 7 A Mojzesz wziawszy namiot, rozbil go sobie za obozem, opodal od obozu, i nazwal go namiotem zgromadzenia. Tedy kazdy, który chcial o co pytac Pana, wychodzil do namiotu zgromadzenia, który byl za obozem.
\par 8 A gdy wychodzil Mojzesz do namiotu, powstawal wszystek lud, i stal kazdy we drzwiach namiotu swego; i patrzali za Mojzeszem, az wszedl do namiotu.
\par 9 I bywalo to, ze gdy wchadzal Mojzesz do namiotu, zstepowal slup oblokowy, a stawal u drzwi namiotu, i mawial Bóg z Mojzeszem.
\par 10 A widzac wszystek lud slup oblokowy, stojacy u drzwi namiotu, powstawal wszystek lud i klanial sie kazdy u drzwi namiotu swego.
\par 11 I mawial Pan do Mojzesza twarza w twarz, jako mawia czlowiek do przyjaciela swego; potem wracal sie do obozu, a sluga jego Jozue, syn Nunów, mlodzieniec, nie odchodzil z posrodku namiotu.
\par 12 Tedy mówil Mojzesz do Pana: Wej, ty mi mówisz: Prowadz lud ten, a tys mi nie oznajmil, kogo poslesz ze mna? Nad to powiedziales: Znam cie z imienia, znalazles tez laske w oczach moich.
\par 13 Teraz tedy, jezlim znalazl laske w oczach twoich, ukaz mi prosze droge twoje, zebym cie poznal, i zebym znalazl laske w oczach twoich, a obacz, ze ludem twoim jest naród ten.
\par 14 I odpowiedzial Pan: Oblicze moje pójdzie przed toba, a dam ci odpocznienie.
\par 15 I rzekl Mojzesz do niego: Nie pójdzieli oblicze twoje z nami, nie wywódz nas stad.
\par 16 Albowiem po czemze tu znac bedzie, zem znalazl laske w oczach twoich, ja i lud twój? izali nie po tem, gdy pójdziesz z nami? bo tak oddzieleni bedziemy, ja i lud twój, od kazdego ludu, który jest na ziemi.
\par 17 I rzekl Pan do Mojzesza: I te rzecz, o któras mówil, uczynie; bos znalazl laske w oczach moich, i znam cie z imienia.
\par 18 Nad to rzekl Mojzesz: Ukaz mi prosze, chwale twoje.
\par 19 A on odpowiedzial: Ja sprawie, ze przejdzie wszystko dobre moje przed twarza twoja, i zawolam z imienia: Pan przed twarza twoja; zmiluje sie, nad kim sie zmiluje; a zlituje sie, nad kim sie zlituje.
\par 20 I rzekl: Nie bedziesz mógl widziec oblicza mego; bo nie ujrzy mie czlowiek, aby zyw zostal.
\par 21 I rzekl Pan: Oto, miejsce u mnie, a staniesz na opoce.
\par 22 A gdy przechodzic bedzie chwala moja, tedy cie postawie w rozpadlinie opoki, i zakryje cie dlonia moja, póki nie przejde.
\par 23 Potem odejme dlon moje, i ujrzysz tyl mój; ale twarz moja nie bedzie widziana.

\chapter{34}

\par 1 I rzekl Pan do Mojzesza: Wyciesz sobie dwie tablice kamienne, podobne pierwszym, a napisze na tychze tablicach slowa, które byly na tablicach pierwszych, któres stlukl.
\par 2 A badz gotów rano, ze wstapisz jutro na góre Synaj, i staniesz przede mna na wierzchu tej góry.
\par 3 Ale zaden niech nie wstepuje z toba, a nikt tez niech nie bedzie widziany po wszystkiej górze; ani owce, ani woly, niech sie nie pasa przeciwko tej górze.
\par 4 Tedy wyciosal Mojzesz dwie tablice kamienne, podobne pierwszym; i wstawszy rano, wstapil na góre Synaj, jako mu rozkazal Pan, wziawszy w rece swe dwie tablice kamienne.
\par 5 I zstapil Pan w obloku, i stanal tam z nim, i zawolal imieniem Pan.
\par 6 Bo przechodzac Pan przed twarza jego, wolal: Pan, Pan, Bóg milosierny i litosciwy, nie rychly do gniewu, a obfity w milosierdziu i w prawdzie;
\par 7 Zachowujacy milosierdzie nad tysiacami, gladzacy nieprawosc i przestepstwo i grzech, nie usprawiedliwiajacy winnego, nawiedzajac nieprawosc ojcowska w synach i w synach synów ich do trzeciego i do czwartego pokolenia.
\par 8 Pospieszywszy sie tedy Mojzesz nachylil sie ku ziemi i poklonil sie,
\par 9 I rzekl; Jezlim teraz znalazl laske w oczach twoich, Panie, niech idzie prosze Pan w posrodku nas, lud bowiem ten twardego karku jest, a odpusc nieprawosci nasze, i grzech nasz, a miej nas za dziedzictwo.
\par 10 Który odpowiedzial: Oto, Ja postanowie przymierze; przed wszystkim ludem twoim czynic bede cuda, które nie byly czynione po wszystkiej ziemi i we wszystkich narodziech; i obaczy wszystek lud, miedzy którymes ty, sprawe Panska; bo straszne bedzie to, co Ja uczynie z toba.
\par 11 Strzezze tego, co Ja rozkazuje tobie: Oto, Ja wypedze przed obliczem twojem Amorejczyka, i Chananejczyka, i Hetejczyka, i Ferezejczyka, i Hewejczyka, i Jebuzejczyka.
\par 12 Strzezze sie, abys snac nie stanowil przymierza z obywatelami ziemi onej, do której ty wnijdziesz, zebyc to nie bylo sidlem posrodku ciebie.
\par 13 Przetoz oltarze ich zburzycie, balwany ich polamiecie, i gaje ich swiecone wyrabiecie.
\par 14 Nie bedziesz sie klanial bogu innemu, przeto ze Pan jest, zawistny imie jego, Bóg zawistny jest;
\par 15 By snac, uczyniwszy przymierze z obywatelami tej ziemi, gdyby oni cudzolozyli z bogami swymi, i ofiarowali bogom swym, ciebie nie wezwali, a jadlbys z ofiar ich:
\par 16 I bralbys z córek ich zony synom swym, i cudzolozylyby córki ich z bogi swymi, a przywiodlyby syny twoje do wszeteczenstwa z bogi swymi.
\par 17 Bogów odlewanych nie czyn sobie.
\par 18 Swieto przasników zachowywac bedziesz; przez siedem dni jesc bedziesz przasniki, jakom ci rozkazal, czasu miesiaca Abib; albowiem tegoz miesiaca Abib wyszedles z Egiptu.
\par 19 Wszystko, co otwiera zywot, moje jest; i wszystko z dobytku twego cokolwiek samcem jest, pierworodne i z owiec, i z wolów;
\par 20 Ale pierworodne osle odkupisz owca; a jezlibys go nie odkupil, zlamiesz mu szyje. Kazdego pierworodnego z synów twych odkupisz, i nie ukaza sie przed twarz moje prózni.
\par 21 Szesc dni robic bedziesz, a dnia siódmego odpoczniesz; czasu orania i czasu zniwa odpoczniesz.
\par 22 Swieto Tygodni uczynisz tez sobie, w pierwiastki zniwa pszenicznego, i swieto zbierania na skonczeniu roku.
\par 23 Trzy kroc do roku ukaze sie kazdy mezczyzna twój przed obliczem Panujacego Pana, Boga Izraelskiego.
\par 24 Albowiem wypedze narody przed toba, a rozszerze granice twoje; i nie bedzie pozadal nikt ziemi twojej, gdy pójdziesz, abys sie ukazal przed obliczem Pana Boga twego trzy kroc do roku.
\par 25 Nie bedziesz ofiarowal przy kwasie krwi ofiary mojej, i nie zostanie nic do jutra z ofiary obchodu swieta przejscia.
\par 26 Pierwiastki pierwszych urodzajów ziemi twej przyniesiesz w dom Pana Boga twego. Nie bedziesz warzyl kozlecia w mleku matki jego.
\par 27 Zatem rzekl Pan do Mojzesza: Napisz sobie te slowa, bo wedlug slów tych postanowilem z toba przymierze, i z Izraelem.
\par 28 I byl tam z Panem czterdziesci dni i czterdziesci nocy; chleba nie jadl, i wody nie pil; i napisal Pan na tablicach slowa przymierza, dziesiec slów.
\par 29 I stalo sie, gdy zstepowal Mojzesz z góry Synaj, a dwie tablice swiadectwa mial w reku Mojzesz, gdy zstepowal z góry, ze nie wiedzial Mojzesz, izby sie lsnila skóra twarzy jego, gdy Pan mówil z nim.
\par 30 I ujrzeli Aaron, i wszyscy synowie Izraelscy Mojzesza, a oto lsnila sie skóra twarzy jego, i bali sie przystapic do niego.
\par 31 Ale zawolal na nich Mojzesz, i nawrócili sie ku niemu Aaron, i wszystkie ksiazeta zgromadzenia, i mówil Mojzesz do nich.
\par 32 Potem tez przyszli wszyscy synowie Izraelscy, którym przykazal wszystko, co mówil Pan z nim na górze SYnaj.
\par 33 A póki Mojzesz mówil z nimi, miewal na twarzy swojej zaslone;
\par 34 Ale gdy wchadzal Mojzesz przed twarz Panska, aby rozmawial z nim, odejmowal zaslone, póki nie wyszedl; a wyszedlszy, mówil do synów Izraelskich, co mu bylo rozkazano.
\par 35 Widzieli tedy synowie Izraelscy twarz Mojzeszowa, ze sie lsnila skóra twarzy Mojzeszowej; i kladl zas Mojzesz zaslone na twarz swoje, póki nie wszedl aby mówil z nim.

\chapter{35}

\par 1 Potem zebral Mojzesz wszystko zgromadzenie synów Izraelskich, i mówil do nich: Te sa rzeczy, które rozkazal Pan, abyscie je czynili.
\par 2 Przez szesc dni odprawowana bedzie robota; ale dzien siódmy bedzie wam swiety, sabbat odpocznienia Panskiego; kto by wen robil robote, umrze.
\par 3 Nie rozniecicie ognia we wszystkich mieszkaniach waszych w dzien sabbatu.
\par 4 Rzekl tez Mojzesz do wszystkiego zgromadzenia synów Izraelskich, mówiac: Tac jest rzecz, która przykazal Pan mówiac:
\par 5 Zlózcie od siebie podarek Panu: kazdy, kto jest ochotnego serca, przyniesie ten podarek Panu, zloto, i srebro, i miedz.
\par 6 I hijacynt, i szarlat, i karmazyn dwa kroc farbowany, i bialy jedwab, i siersc kozia;
\par 7 Skóry tez baranie czerwono farbowane i skóry borsukowe, i drzewo sytym;
\par 8 I oliwe do swiecenia, i rzeczy wonne na olejek pomazywania, i dla kadzenia wonnego;
\par 9 Kamienie tez onychiny, i kamienie do osadzania naramiennika i napiersnika.
\par 10 A wszyscy dowcipnego serca miedzy wami przyjda, i robic beda, cokolwiek rozkazal Pan:
\par 11 Przybytek, namiot jego, i przykrycie jego, haczyki jego, i deski jego, dragi jego, slupy jego, i podstawki jego;
\par 12 Skrzynie i drazki jej, ublagalnia, i opone do zaslony,
\par 13 Stól i drazki jego, ze wszystkiem naczyniem jego, i chleby pokladne;
\par 14 I swiecznik do swiecenia z naczyniem jego, i lampy jego, i oliwe do swiecenia.
\par 15 Oltarz takze do kadzenia z drazkami jego, i olejek pomazywania, i kadzenia wonne, i zaslone do drzwi przybytku;
\par 16 Oltarz do calopalenia, i krate jego miedziana, drazki jego, i wszystkie naczynia jego, wanne z stolcem jej.
\par 17 Opony do sieni, slupy jej, i podstawki jej, i zaslony do drzwi u sieni.
\par 18 Kolki do przybytku i kolki do sieni z sznurami jej.
\par 19 Szaty sluzebne do uslugiwania w swiatnicy, szaty swiete Aaronowi kaplanowi, i szaty synom jego, dla sprawowania urzedu kaplanskiego
\par 20 Wyszlo tedy wszystko zgromadzenie synów Izraelskich od oblicznosci Mojzeszowej.
\par 21 I przyszedl kazdy maz, którego pobudzilo serce jego, i kazdy, w którym dobrowolny byl duch jego, przyniesli podarek Panu do robienia namiotu zgromadzenia, i do wszelkiej potrzeby jego, i na szaty swiete.
\par 22 Przychodzili tedy mezowie z niewiastami, kazdy, kto byl ochotnego serca, przynosili zapony, i nausznice, i pierscienie, i manele, i wszelakie naczynia zlote, i ktokolwiek przynosil ofiare zlota Panu.
\par 23 Kazdy tez co mial hijacynt, i szarlat, i karmazyn dwa kroc farbowany, i bialy jedwab, i siersc kozia, i skóry baranie czerwono farbowane, i skóry borsukowe, przynosili.
\par 24 Ktokolwiek ofiarowal podarek srebra i miedzi, przynosili na ofiare Panu, kazdy tez, co mial drzewo sytym, na wszelaka potrzebe ku usludze przynosili.
\par 25 I wszystkie niewiasty dowcipnego serca rekami swemi przedly, a przynosily co naprzedly, hijacynt, i szarlat, karmazyn dwa kroc farbowany, i bialy jedwab.
\par 26 A wszystkie niewiasty których pobudzilo serce ich umiejetne, przedly siersc kozia.
\par 27 Przelozeni zasie przynosili kamienie onychiny, i kamienie do osadzania naramiennika i napiersnika.
\par 28 Takze rzeczy wonne i oliwe do swiecenia, i na olejek pomazywania i na wonne kadzenia.
\par 29 Kazdy maz i niewiasta, w których ochotne serce bylo do ofiarowania, na kazda robote, która rozkazal Pan czynic przez Mojzesza, przynosili synowie Izraelscy ofiare dobrowolna Panu.
\par 30 Zatem rzekl Mojzesz do synów Izraelskich: Oto, wezwal Pan z imienia Besaleela, syna Urowego, syna Churowego, z pokolenia Judy,
\par 31 I napelnil go Duchem Bozym, madroscia, i umiejetnoscia wszelkiego rzemiosla;
\par 32 I ku dowcipnemu wymyslaniu, cokolwiek moze byc urobione ze zlota, i z srebra, i z miedzi;
\par 33 Do rzezania kamienia ku osadzeniu, i na wyrobienie drzewa, do czynienia wszelakiej roboty zmyslnej.
\par 34 Dal nadto do serca jego, aby uczyc mógl inszych, on, i Acholijab, syn Achysamechów z pokolenia Dan.
\par 35 Napelnil je madroscia serca, aby robili wszelakie rzemioslo ciesielskie, i haftarskie, i tkackie z hijacyntu, i z szarlatu, z karmazynu dwa kroc farbowanego, i z bialego jedwabiu tkacka robota, aby robili kazda robote dowcipnie wymyslajac.

\chapter{36}

\par 1 Tedy robil Besaleel, i Acholijab, i kazdy maz dowcipny, którym dal Bóg madrosc i rozum, aby umieli urobic kazda robote ku usludze swiatnicy, wszystko, co rozkazal Pan.
\par 2 I wezwal Mojzesz Besaleela, i Acholijaba, i kazdego meza dowcipnego, któremu dal Pan madrosc w serce jego; kazdego tez, którego pobudzilo serce jego, aby przystapil do czynienia tej roboty.
\par 3 I wzieli od Mojzesza wszystkie podarki, które byli przyniesli synowie Izraelscy na robote ku usludze swiatnicy, aby ja wykonali; ale oni przynaszali do niego jeszcze dobrowolne dary na kazdy poranek.
\par 4 Tedy sie zeszli wszyscy dowcipni, którzy robili wszelaka robote swiatnicy, kazdy opusciwszy robote swoje, która czynili.
\par 5 I rzekli do Mojzesza, mówiac: Daleko wiecej lud przynosi, niz potrzeba do wyrobienia tej uslugi, która rozkazal Pan uczynic.
\par 6 Rozkazal tedy Mojzesz, aby obwolano w obozie mówiac: Ani maz, ani niewiasta niech wiecej ni przynosza ofiar na robienie swiatnicy. I zabroniono ludowi, aby nie nosili.
\par 7 Bo mieli potrzeb dostatek do wszystkiej roboty, aby ja wyrobili, i zbywalo.
\par 8 I urobili kazdy dowcipny z rzemieslników te robote: przybytek z dziesieciu opon z bialego jedwabiu kreconego, i z hijacyntu, i szarlatu i z karmazynu dwa kroc farbowanego; z Cherubiny, robota misterna robili je.
\par 9 Dlugosc opony jednej dwadziescia i osiem lokci, a szerokosc opony jednej na cztery lokcie; pod jedna miara byly wszystkie opony.
\par 10 I spoil piec opon jedne z druga, takze drugie piec opon spoil jedne z druga.
\par 11 Naczynil tez petlic hijacyntowych po kraju opony jednej, na koncu, gdzie sie spinac maja; takze uczynil po kraju opony drugiej, na koncu, gdzie sie spinac maja.
\par 12 Piecdziesiat petlic uczynil na oponie jednej, a piecdziesiat petlic uczynil po kraju opony, któremi spojona byla do drugiej; petlica jedna przeciw drugiej byla.
\par 13 Uczynil tez piecdziesiat haczyków zlotych, a spial opony jedne ku drugiej haczykami; i tak uczyniony jest przybytek jeden.
\par 14 Urobil tez opony z siersci koziej na namiot ku zakrywaniu przybytku z wierzchu, jedenascie opon urobil.
\par 15 Dlugosc opony jednej trzydziesci lokci, a cztery lokcie szerokosc opony jednej; jednaz miara byla tych jedenascie opon.
\par 16 I spoil piec opon osobno, a szesc opon osobno.
\par 17 Uczynil tez petlic piecdziesiat po kraju jednej opony na koncu, gdzie sie ma spinac; i piecdziesiat petlic uczynil po kraju opony drugiej ku spinaniu.
\par 18 Uczynil tez haczyków miedzianych piecdziesiat, do spiecia namiotu, aby byl jeden.
\par 19 Nad to uczynil przykrycie na namiot z skór baranich czerwono farbowanych, i przykrycie z skór borsukowych na wierzch.
\par 20 Naczynil tez desek do przybytku z drzewa sytym stojacych.
\par 21 Dziesiec lokci dlugosc deski, a póltora lokcia szerokosc deski jednej.
\par 22 Dwa czopy miala deska jedna, sporzadzone jeden przeciwko drugiemu; tak uczynil u wszystkich desek przybytku.
\par 23 Zgotowal tez i deski do przybytku, dwadziescia desek ku stronie poludniowej, ku wiatrowi poludniowemu.
\par 24 I czterdziesci podstawków urobil ze srebra pod dwadziescia desek: dwa podstawki pod deske jedne do dwóch czopów jej, takze dwa podstawki pod deske druga do dwu czopów jej.
\par 25 Takze na drugiej stronie przybytku ku stronie pólnocnej, uczynil dwadziescia desek.
\par 26 I czterdziesci podstawków ich srebrnych: dwa podstawki pod deske jedne, i dwa podstawki pod deske druga.
\par 27 Lecz na stronie przybytku ku zachodowi, uczynil szesc desek.
\par 28 Dwie deski uczynil na weglach po obu stronach przybytku;
\par 29 A byly spojone od spodku, takze spojone byly od wierzchu do jednegoz kolca; tak uczynil po obu stronach na dwu weglach.
\par 30 A tak bylo osiem desek, i podstawków ich srebrnych szesnascie podstawków, po dwu podstawkach pod kazda deska.
\par 31 Naczynil i dragów z drzewa sytym; piec do desek przybytku na jedne strone;
\par 32 Piec takze dragów do desek przybytku na druga strone, piec tez dragów do desek przybytku do obu weglów, na zachód.
\par 33 A uczynil tez drag posredni, aby przechodzil przez posrodek desek od konca do konca.
\par 34 A deski one powlókl zlotem, i kolce do nich porobil ze zlota, aby w nich dragi byly, i powlókl dragi zlotem.
\par 35 Uczynil zas zaslone z hijacyntu, i z szarlatu, i z karmazynu dwa kroc farbowanego, i z bialego jedwabiu kreconego; robota misterna uczynil to z Cherubiny.
\par 36 A do niej nagotowal cztery slupy z drzewa sytym, i powlókl je zlotem, haki tez ich byly zlote, i ulal do nich cztery podstawki srebrne.
\par 37 Uczynil tez zaslone do drzwi namiotu z hijacyntu, i z szarlatu, i z karmazynu dwa kroc farbowanego, i z bialego jedwabiu kreconego, robota haftarska.
\par 38 A slupów do niej piec z haczykami ich; i powlókl wierzchy ich i przepasania ich zlotem, a podstawków ich bylo piec miedzianych.

\chapter{37}

\par 1 Uczynil tez Besaleel skrzynie z drzewa sytym, a byla póltrzecia lokcia dlugosc jej, a póltora lokcia szerokosc jej, takze póltora lokcia wysokosc jej.
\par 2 I powlókl ja zlotem szczerem wewnatrz, i zewnatrz, i uczynil jej korone zlota w okolo.
\par 3 Ulal tez do niej cztery kolce zlote do czterech weglów jej: dwa kolce po jednej stronie jej, a dwa kolce po drugiej stronie jej.
\par 4 Uczynil i drazki z drzewa sytym, a powlókl je zlotem.
\par 5 I przewlekl drazki przez kolce po stronach skrzyni, aby na nich noszona byla skrzynia.
\par 6 Uczynil tez ublagalnia ze zlota szczerego: póltrzecia lokcia dlugosc jej, a póltora lokcia szerokosc jej.
\par 7 Urobil i dwa Cheruby zlote, z ciagnionego zlota urobil je na obu koncach ublagalni.
\par 8 Cheruba jednego na jednym koncu, a Cheruba drugiego na drugim koncu; na ublagalni uczynil Cheruby na obu koncach jej.
\par 9 Którzy Cherubowie mieli rozciagnione skrzydla, z wierzchu zakrywajac skrzydlami swemi ublagalnia, a twarzy ich byly jednemu ku drugiemu; ku ublagalni byly twarzy Cherubów.
\par 10 Przytem sprawil stól z drzewa sytym, dwa lokcie dlugosc jego, i lokiec szerokosc jego, a póltora lokcia wysokosc jego.
\par 11 I powlókl go zlotem szczerem, i uczynil mu korone zlota w okolo.
\par 12 Uczynil mu tez listwe na dlon w szerz w okolo; uczynil tez i korone zlota w okolo onej listwy.
\par 13 I ulal do niego cztery kolce zlote, które kolce przyprawil na czterech rogach, u czterech nóg jego.
\par 14 Na przeciwko onej listwie byly kolce, w które zawlaczano drazki do noszenia stolu.
\par 15 Porobil i drazki z drzewa sytym, i powlókl je zlotem do noszenia stolu.
\par 16 Poczynil tez naczynia do stolu nalezace, misy jego i przystawki jego, i kubki jego, i czasze do nalewania ofiar mokrych, z szczerego zlota.
\par 17 Urobil tez swiecznik ze zlota szczerego, z ciagnionego zlota uczynil swiecznik ten, slupiec jego, i prety jego, czaszki jego, galki jego, i kwiaty jego z tegoz byly.
\par 18 A szesc pretów wychodzilo po stronach jego: trzy prety z jednej strony swiecznika, a trzy prety z drugiej strony swiecznika.
\par 19 Trzy czaszki na ksztalt orzecha migdalowego na precie jednym, takze galka i kwiat; i trzy czaszki na ksztalt orzecha migdalowego na precie drugim, takze galka i kwiat; tak bylo na wszystkich szesciu pretach wychodzacych z swiecznika.
\par 20 Ale na swieczniku byly cztery czaszki na ksztalt orzecha migdalowego, galki jego i kwiaty jego.
\par 21 I byla galka pod dwiema pretami jego, takze galka pod drugiemi dwiema pretami jego, i zas galka pod innemi dwiema pretami jego; tak bylo pod szescia pretów wychodzacych z niego.
\par 22 Galki ich i prety ich z niego byly; to wszystko ze zlota calokowane bylo, ze zlota szczerego.
\par 23 Uczynil tez siedem lamp do niego, i nozyczki do nich i kaganki jego ze zlota szczerego.
\par 24 Z talentu zlota szczerego uczynil go, i wszystko naczynie jego.
\par 25 Uczynil tez oltarz do kadzenia z drzewa sytym, na lokiec wzdluz, i na lokiec wszerz, czworograniasty, a na dwa lokcie wzwyz, a z niego wychodzily rogi jego.
\par 26 I powlókl go zlotem szczerem, wierzch jego, i sciany jego w okolo, i rogi jego; uczynil mu tez korone zlota w okolo.
\par 27 Po dwu takze kolcach zlotych uczynil u niego, pod korona jego, we dwu katach jego, po obu stronach jego przez które przewlaczano drazki, aby byl noszony na nich.
\par 28 Uczynil tez drazki z drzewa sytym i powlókl je zlotem.
\par 29 Uczynil tez olejek pomazywania swietego, i kadzenie wonne, robota aptekarska.

\chapter{38}

\par 1 Uczynil tez oltarz na calopalenie z drzewa sytym, na piec lokci wzdluz, i na piec lokci wszerz, czworogranisty, a na trzy lokcie wzwyz.
\par 2 I uczynil mu rogi na czterech weglach jego; z niego wychodzily rogi jego, a obil je miedzia.
\par 3 Poczynil tez wszelakie naczynia do oltarza, kotly, i miotly, i miednice, i widly, i lopaty, wszystkie naczynia jego uczynil z miedzi.
\par 4 Uczynil tez do oltarza krate miedziana na ksztalt sieci miedzy okregiem jego, od spodku az do polowy jego.
\par 5 I ulal cztery kolce na czterech rogach kraty miedzianej, na zakladanie drazków.
\par 6 Drazki takze porobil z drzewa sytym, a obil je miedzia.
\par 7 I przewlókl drazki przez one kolce po stronach oltarza, aby noszony byl na nich; czczy z desek uczynil go.
\par 8 Uczynil tez wanne miedziana, i stolec jej miedziany ze zwierciadel niewiast gromada przychodzacych, które przychodzily do drzwi namiotu zgromadzenia.
\par 9 Uczynil i sien ku stronie poludniowej na poludnie, i opony sieni z bialego jedwabiu kreconego, na sto lokci.
\par 10 Slupów do nich dwadziescia, i podstawków do nich dwadziescia miedzianych, glówki na slupiech, i okrecenia ich srebrne.
\par 11 Takze na stronie pólnocnej opon na sto lokci; slupów do nich dwadziescia i podstawków do nich miedzianych dwadziescia; glówki na slupiech i okrecenia ich srebrne.
\par 12 A zasie od zachodniej strony byly opony na piecdziesiat lokci; slupów do nich dziesiec, i podstawków ich dziesiec; glówki na slupiech, i okrecenia ich srebrne.
\par 13 A na stronie przedniej ku wschodowi bylo opon na piecdziesiat lokci.
\par 14 Opony na pietnascie lokci byly po jednej stronie, slupów do nich trzy, i podstawków do nich trzy.
\par 15 A po drugiej stronie, stad i zowad u bramy sieni, opon pietnascie lokci, slupów do nich trzy, takze podstawków do nich trzy.
\par 16 Wszystkie opony sieni w okolo byly z jedwabiu bialego kreconego.
\par 17 A podstawki slupów miedziane, glówki na slupiech, i okrecenia ich srebrne, do tego przykrycie wierzchów ich srebrne, a byly okrecane srebrem wszystkie slupy sieni.
\par 18 Nad to zaslone bramy u sieni uczynil robota haftarska z hijacyntu, i z szarlatu i z karmazynu dwa kroc farbowanego, i z jedwabiu kreconego; na dwadziescia lokci byla dlugosc jej, wysokosc szeroka na piec lokci, jako inne opony sieni.
\par 19 A slupów do nich cztery, takze podstawków ich cztery miedzianych; glówki ich srebrne, i zakrycia wierzchów ich, takze okrecenia ich srebrne.
\par 20 Takze wszystkie kolki przybytku, i sieni w okolo, byly miedziane.
\par 21 Te sa rzeczy policzone do przybytku, do przybytku swiadectwa, które policzone byly na rozkazanie Mojzeszowe przez Itamara, syna Aarona kaplana, ku usludze Lewitom.
\par 22 A Besaleel, syn Urów, syna Churowego z pokolenia Judy, uczynil to wszystko, co byl Pan rozkazal Mojzeszowi;
\par 23 A z nim Acholijab, syn Achysamechów z pokolenia Dan, ciesla, subtelny rzemieslnik, i haftujacy na hijacyncie, i na szarlacie, i na karmazynie dwa kroc farbowanym, i na bialym jedwabiu.
\par 24 Wszystkiego zlota wynalozonego na sama robote, na wszystka robote swiatnicy, które zloto bylo podarkowe, bylo dziewiec i dwadziescia talentów, siedem set i trzydziesci syklów wedlug sykla swiatnicy.
\par 25 Srebra zasie od policzonych w poczet zgromadzenia sto talentów, i tysiac siedem set siedmdziesiat i piec syklów wedlug sykla swiatnicy.
\par 26 Od kazdej glowy pól sykla wedlug sykla swiatnicy, od wszystkich, którzy szli w liczbe, bedac we dwudziestu lat i dalej, których ludzi bylo szesc kroc sto tysiecy, i trzy tysiace, i piecset, i piecdziesiat.
\par 27 A bylo sto talentów srebra do odlewania podstawków swiatnicy i podstawków zaslony; sto podstawków ze sta talentów; talent na podstawek.
\par 28 A z tysiaca, siedmiu set, siedmdziesiat i pieciu syklów uczynil haki na slupy, i powlókl wierzchy ich, i przepasal je.
\par 29 Miedzi zas ofiarowanej bylo siedmdziesiat talentów, dwa tysiace i cztery sta syklów.
\par 30 I uczynil z niej podstawki do drzwi namiotu zgromadzenia, i oltarz miedziany, i krate miedziana do niego, takze wszystko naczynie do oltarza.
\par 31 I podstawki do sieni w okolo; takze podstawki bramy siennej, i wszystkie kolki przybytku, takze kolki sieni w okolo.

\chapter{39}

\par 1 Takze z hijacyntu i z szarlatu, z karmazynu dwa kroc farbowanego poczynili szaty do uslugi, ku uslugiwaniu w swiatnicy. Urobili tez szaty swiete Aaronowi, jako byl rozkazal Pan Mojzeszowi.
\par 2 I uczynil naramiennik ze zlota, z hijacyntu, i z szarlatu, i z karmazynu dwa kroc farbowanego, i z bialego jedwabiu kreconego.
\par 3 Naklepali tez blaszek zlotych, i nastrzygli z nich nici do przetykania hijacyntu, i do przetykania szarlatu, i do przetykania karmazynu dwa kroc farbowanego, i do przetykania bialego jedwabiu, robota haftarska.
\par 4 Naramienniki przytem porobili tak, aby sie jeden z drugim spoic mógl; na dwu krajach ich spajaly sie.
\par 5 Pas tez naramiennika, który byl na nim, z tegoz byl, i taz robota ze zlota, z hijacyntu, i z szarlatu, i z karmazynu dwa kroc farbowanego, i z bialego jedwabiu kreconego, jako byl Pan rozkazal Mojzeszowi.
\par 6 Do tego wygotowali kamienie onychiny, oprawione zlotem osadzeniem, rzezane, jako ryte bywaja pieczeci, z imiony synów Izraelskich.
\par 7 I wprawil je na wierzchne kraje naramiennika, aby byly kamienmi na pamiatke synom Izraelskim, jako byl rozkazal Pan Mojzeszowi.
\par 8 Uczynil tez napiersnik robota haftarska, wedlug roboty naramiennika, ze zlota, z hijacyntu, i z szarlatu, i z karmazynu dwa kroc farbowanego, i z bialego jedwabiu kreconego.
\par 9 Czworograniasty byl dwoisty uczynili napiersnik, na piedzi dlugosc jego, i na piedzi szerokosc jego, dwoisty byl.
\par 10 I nasadzili wen cztery rzedy kamienia tym porzadkiem: sardyjusz, topazyjusz i szmaragd w rzedzie pierwszym.
\par 11 A w drugim rzedzie: karbunkul, szafir i jaspis.
\par 12 A w trzecim rzedzie: linkuryjusz, achates i ametyst.
\par 13 A w czwartym rzedzie: chrysolit, onychin i beryl, wszystkie osadzone w zloto w rzedziech swych.
\par 14 A tych kamieni z imionami synów Izraelskich dwanascie wedlug imion ich bylo, tak, jako rzeza pieczeci; kazdy wedlug imienia swego podlug dwunastu pokolen.
\par 15 Poczynili tez do napiersnika lancuszki jednostajne robota pleciona ze zlota szczerego.
\par 16 Sprawili tez dwa haczyki zlote, i dwa kolce zlote, i przyprawili one dwa kolce do obu krajów napiersnika.
\par 17 A przewlekli one dwa lancuszki zlote przez oba kolce u krajów napiersnika.
\par 18 Drugie zas dwa kolce obu lancuszków zawlekli do onych dwu haczyków, i przyprawili do zwierzchnich krajów naramiennika na przodku.
\par 19 Uczynili takze dwa kolce zlote, które przyprawili do dwu konców napiersnika na kraju jego, który byl po stronie naramiennika ze spodku.
\par 20 Uczynili jeszcze dwa kolce zlote, które przyprawili na dwu stronach naramiennika ze spodku, na przodku przeciwko spojeniu jego, które jest nad przepasaniem naramiennika.
\par 21 I przywiazali napiersnik do kolców jego, do kolców naramiennika sznurem hijacyntowym, aby byl przepasaniem naramiennika, zeby nie odstawal napiersnik od naramiennika; jako byl rozkazal Pan Mojzeszowi.
\par 22 Urobil takze plaszcz pod naramiennik robota tkana, wszystek hijacyntowy;
\par 23 A rozpór plaszcza w posród jego, jako rozpór u pancerza, i brama okolo kraju jego, aby sie nie rozdzieral.
\par 24 Takze u podolka plaszcza onego uczynili jablka granatowe z hijacyntu, i z szarlatu, i z karmazynu dwa kroc farbowanego, i z bialego jedwabiu kreconego.
\par 25 Poczynili tez dzwonki ze zlota szczerego, i pozawieszali one dzwonki miedzy one jablka granatowe u podolka plaszcza w okolo, w posród jablek granatowych;
\par 26 Dzwonek a jablko granatowe, i zas dzwonek i jablko granatowe, u podolka plaszcza w okolo ku poslugiwaniu, jako rozkazal Pan Mojzeszowi.
\par 27 Porobili tez szaty z bialego jedwabiu robota tkacka Aaronowi, i synom jego.
\par 28 Czapeczke tez z bialego jedwabiu, i czapki ozdobne z bialego jedwabiu, i ubiory cienkie z bialego jedwabiu kreconego.
\par 29 Pas takze z bialego jedwabiu kreconego, i z hijacyntu, i z szarlatu, i z karmazynu dwa kroc farbowanego robota haftarska, jako rozkazal Pan Mojzeszowi.
\par 30 Do tego uczynili blache korony swietobliwosci ze zlota szczerego, i wyrysowali na niej robota ryta, jako pieczeci rzezana Swietosc Panu.
\par 31 A przyprawili do niej sznur hijacyntowy, aby przywiazana byla do czapki na wierzchu, jako byl rozkazal Pan Mojzeszowi.
\par 32 A tak skonczyla sie wszystka robota okolo przybytku i namiotu zgromadzenia. I uczynili synowie Izraelscy wszystko, jako byl rozkazal Pan Mojzeszowi, tak uczynili.
\par 33 I przyniesli ten przybytek do Mojzesza, namiot, i wszystkie naczynia jego, haki jego, deski jego, dragi jego, i slupy jego, i podstawki jego.
\par 34 Przykrycie tez ze skór baranich czerwono farbowanych, i przykrycie z skór borsukowych, i opone zaslony;
\par 35 Skrzynie swiadectwa, i drazki jej, i ublagalnia.
\par 36 Stól, wszystkie naczynia jego, i chleb pokladny.
\par 37 Swiecznik ochedozny, lampy jego, lampy sporzadzone, i wszystkie naczynia jego, i oliwe ku swieceniu.
\par 38 Oltarz takze zloty i olejek pomazywania, i kadzidlo wonne, i zaslone do drzwi namiotu,
\par 39 Oltarz miedziany, i krate jego miedziana, drazki jego, i wszystkie naczynia jego, wanne i stolec jej.
\par 40 Opony do sieni, i slupy ich, i zaslone do bramy siennej, i sznury jej, i kolki jej, i wszelakie naczynia ku sluzbie przybytku i namiotu zgromadzenia.
\par 41 Szaty sluzebne do uslugowania w swiatnicy, szaty swiete Aaronowi kaplanowi, i szaty synów jego do odprawowania urzedu kaplanskiego.
\par 42 Wedlug wszystkiego, jako byl rozkazal Pan Mojzeszowi, tak uczynili synowie Izraelscy wszystka te robote.
\par 43 I obejrzal Mojzesz te wszystka robote, a oto, uczynili ja, jako byl rozkazal Pan, tak uczynili; i blogoslawil im Mojzesz.

\chapter{40}

\par 1 Potem rzekl Pan do Mojzesza, mówiac:
\par 2 W dzien miesiaca pierwszego, pierwszego dnia tegoz miesiaca wystawisz przybytek, namiot zgromadzenia.
\par 3 I postawisz tam skrzynie swiadectwa, i zakryjesz ja zaslona.
\par 4 Wstawisz i stól, i porzadnie go sporzadzisz, wniesiesz takze swiecznik, i zaswiecisz lampy jego.
\par 5 Postawisz tez oltarz zloty do kadzenia przed skrzynia swiadectwa, i zawiesisz zaslone u drzwi przybytku.
\par 6 Takze postawisz oltarz calopalenia przed drzwiami przybytku, namiotu zgromadzenia.
\par 7 Postawisz tez wanne miedzy namiotem zgromadzenia a miedzy oltarzem, w która nalejesz wody.
\par 8 Wystawisz tez i sien w okolo, a zawiesisz zaslone we drzwiach u sieni.
\par 9 Zatem wezmiesz olejek pomazywania, i pomazesz przybytek, i wszystko, co w nim jest, i poswiecisz go ze wszystkiem naczyniem jego, a bedzie swietym.
\par 10 Pomazesz tez oltarz calopalenia, i wszystkie naczynia jego, i poswiecisz oltarz, a bedzie oltarzem najswietszym.
\par 11 Nad to pomazesz wanne i stolec jej, a poswiecisz ja.
\par 12 Zatem kazesz przystapic Aaronowi i synom jego do drzwi namiotu zgromadzenia, i umyjesz je woda.
\par 13 I obleczesz Aarona w szaty swiete, a pomazesz, i poswiecisz go, aby mi sprawowal urzad kaplanski.
\par 14 Synom takze jego przystapic kazesz, i obleczesz je w szaty,
\par 15 A pomazesz je, jakos pomazal ojca ich, aby mi sprawowali urzad kaplanski; i bedzie pomazanie ich onym ku wiecznemu kaplanstwu w narodziech ich.
\par 16 Tedy uczynil Mojzesz wszystko; jako mu byl rozkazal Pan, tak uczynil.
\par 17 Stalo sie tedy miesiaca pierwszego, roku wtórego, pierwszego dnia miesiaca, ze wystawiony jest przybytek.
\par 18 I wystawil Mojzesz przybytek, a podstawil podstawki jego, i postawil deski jego, i zalozyl dragi jego, i podniósl slupy jego.
\par 19 Rozbil tez i namiot nad przybytkiem, i polozyl przykrycie namiotu nad nim z wierzchu, tak jako byl Pan rozkazal Mojzeszowi.
\par 20 Potem wziawszy swiadectwo, wlozyl je do skrzyni, i przewlókl drazki u skrzyni, i wlozyl ublagalnia z wierzchu na skrzynie.
\par 21 I wniósl skrzynie do przybytku, i zawiesil opone zakrycia, i zaslonil skrzynie swiadectwa, jako byl Pan rozkazal Mojzeszowi.
\par 22 Postawil i stól w namiocie zgromadzenia ku pólnocnej stronie przed zaslona.
\par 23 I sporzadzil na nim sporzadzenie chlebów przed Panem, jako byl rozkazal Pan Mojzeszowi.
\par 24 Postawil tez swiecznik w namiocie zgromadzenia na przeciwko stolowi ku poludniowej stronie przybytku.
\par 25 Zapalil tez lampy przed Panem, jako byl Pan rozkazal Mojzeszowi.
\par 26 Postawil i oltarz zloty w namiocie zgromadzenia przed zaslona.
\par 27 I kadzil na nim kadzeniem wonnem, jako byl rozkazal Pan Mojzeszowi.
\par 28 Potem zawiesil zaslone we drzwiach przybytku.
\par 29 Nadto oltarz postawil calopalenia przede drzwiami przybytku namiotu zgromadzenia, i ofiarowal na nim calopalenie i ofiare sucha, jako byl rozkazal Pan Mojzeszowi.
\par 30 Potem postawil wanne miedzy namiotem zgromadzenia, a miedzy oltarzem, w która nalal wody dla umywania.
\par 31 I umywali sie z niej Mojzesz, i Aaron, i synowie jego, rece swe i nogi swe.
\par 32 Gdy wchodzili do namiotu zgromadzenia, i gdy mieli przystepowac do oltarza, umywali sie, jako rozkazal Pan Mojzeszowi.
\par 33 Na ostatek wystawil sien okolo przybytku i oltarza, i zawiesil zaslone w bramie sieni. A tak dokonczyl Mojzesz roboty onej.
\par 34 Tedy okryl oblok namiot zgromadzenia, a chwala Panska napelnila przybytek.
\par 35 Tak, iz nie mógl Mojzesz wnijsc do namiotu zgromadzenia; bo byl nad nim oblok, a chwala Panska napelnila byla przybytek.
\par 36 A gdy odstepowal oblok od przybytku, ruszali sie synowie Izraelscy w ciagnieniu swem.
\par 37 A jezli nie odstepowal oblok, nie ruszali sie az do dnia, którego odstepowal.
\par 38 A oblok Panski bywal nad przybytkiem we dnie, a ogien bywal w nocy nad nim przed oczyma wszystkiego domu Izraelskiego, ilekroc ciagneli.


\end{document}