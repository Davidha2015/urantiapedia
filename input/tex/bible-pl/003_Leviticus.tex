\begin{document}

\title{Kapłańska}


\chapter{1}

\par 1 I wezwal Pan Mojzesza, i rzekl do niego z namiotu zgromadzenia, mówiac:
\par 2 Mów do synów Izraelskich, a rzecz im: Gdyby kto z was ofiarowal ofiare Panu, z bydla, z wolów, i z drobnego bydla, ofiarowac bedziecie ofiare wasze.
\par 3 Jezli calopalona ofiara jego bedzie z rogatego bydla, samca zupelnego ofiarowac bedzie; u drzwi namiotu zgromadzenia ofiarowac go bedzie dobrowolnie przed obliczem Panskiem.
\par 4 I polozy reke swa na glowe ofiary calopalenia, a bedzie przyjemna zan na oczyszczenie jego.
\par 5 Zabije tedy cielca tego kaplan przed oblicznoscia Panska; a synowie Aaronowi, kaplani, ofiarowac beda krew, a pokropia ta krwia oltarz z wierzchu w okolo, który jest przede drzwiami namiotu zgromadzenia.
\par 6 A wziawszy skóre z ofiary calopalenia rozrabie ja na sztuki.
\par 7 Potem naloza synowie Aarona kaplana, ogien na oltarzu, a uloza drwa na ogien.
\par 8 Potem porzadnie wloza synowie Aaronowi, kaplani, one sztuki, glowe, i tlustosc, na drwa, które sa na ogniu, który jest na oltarzu.
\par 9 A wnetrznosci jego, i nogi jego, oplucze woda, i zapali kaplan to wszystko na oltarzu; calopalenie jest ofiary ognistej ku wdziecznej wonnosci Panu.
\par 10 A jezlizby z drobnego bydla kto chcial ofiarowac z owiec albo z kóz, na ofiare calopalenia, samca zupelnego ofiarowac bedzie;
\par 11 I zabije go po bok oltarza ku pólnocy przed oblicznoscia Panska; a pokropia synowie Aaronowi, kaplani, krwia jego po wierzchu oltarza w okolo.
\par 12 I rozrabie go na sztuki, i glowe jego, i tlustosc jego; a wlozy je kaplan porzadnie na drwa, które sa na ogniu, który jest na oltarzu.
\par 13 A wnetrznosci i nogi oplucze woda; i bedzie ofiarowal kaplan to wszystko, i zapali to na oltarzu. Calopalenie jest ofiary ognistej ku wdziecznej wonnosci Panu.
\par 14 A jezliby z ptastwa calopalenia ofiare chcial kto ofiarowac Panu, tedy niech przyniesie z synogarlic, albo z golabiat ofiare swoje.
\par 15 A bedzie ja ofiarowal kaplan na oltarzu, i paznogciem nadrze glowe jego, i zapali na oltarzu, wycisnawszy krew jego na stronie oltarza.
\par 16 Odejmie tez gardziel jego z pierzem jego, a porzuci je blisko oltarza ku wschodniej stronie, na miejsce, gdzie popiól bywa;
\par 17 I rozedrze mu skrzydla jego, a wszakze ich nie oderwie; i spali to kaplan na oltarzu, na drwach, które sa na ogniu; calopalenie jest ofiary ognistej ku wdziecznej wonnosci Panu.

\chapter{2}

\par 1 Gdyby tez kto ofiarowac chcial dar ofiary sniednej Panu, pszenna maka bedzie ofiara jego; i poleje ja oliwa, i nakladzie na nia kadzidla.
\par 2 I przyniesie ja do synów Aaronowych, kaplanów, a wezmie stad pelna garsc swoje tej pszennej maki, i tej oliwy, ze wszystkiem kadzidlem; i zapali to kaplan na pamiatke jej na oltarzu; ofiara ognista jest ku wdziecznej wonnosci Panu;
\par 3 Ale co zostanie od onej ofiary sniednej, Aaronowi i synom jego bedzie; najswietsza rzecz jest z ognistych ofiar Panu.
\par 4 A jezlibys tez ofiarowal dar ofiary sniednej w piecu pieczonej, niechze bedzie z pszennej maki placek przasny zagnieciony w oliwie, i kreple przasne, pomazane oliwa.
\par 5 Jezlize zas ofiare sniedna smazona w panwi ofiarowac bedziesz, niechze bedzie z maki pszennej zagniecionej w oliwie, oprócz zakwaszenia.
\par 6 Polamiesz ja na kesy, i polejesz ja oliwa; ofiara to sniedna jest.
\par 7 A jezli ofiare sniedna w kotle zgotowana ofiarowac bedziesz, z maki pszennej z oliwa bedzie.
\par 8 I przyniesiesz ofiare sniedna z tych rzeczy sprawiona Panu, i oddasz ja kaplanowi, a odniesie ja na oltarz.
\par 9 I wezmie kaplan z onej ofiary sniednej pamiatke jej, i zapali na oltarzu; ofiara to ognista ku wdziecznej wonnosci Panu.
\par 10 A co pozostanie od onej ofiary sniednej, Aaronowi i synom jego bedzie; najswietsza rzecz jest z ognistych ofiar Panu.
\par 11 Wszelka ofiara sniedna, która ofiarowac bedziecie Panu, bez kwasu bedzie; bo zadnego kwasu i zadnego miodu nie bedziecie zapalac na ofiare ognista Panu.
\par 12 Tylko w ofiarach pierwiastek ofiarowac to bedziecie Panu; ale na oltarz nie bedziecie ich klasc ku wdziecznej wonnosci.
\par 13 Kazdy dar ofiary twojej sniednej sola posolisz a nie odejmiesz soli przymierza Boga twojego od ofiary twojej sniednej; przy kazdej ofierze twojej ofiarowac bedziesz sól.
\par 14 A jezli ofiarowac bedziesz ofiare sniedna z pierwszych urodzajów Panu, swieze klosy uprazysz ogniem, a zboze wykruszone z klosów swiezych ofiarowac bedziesz na ofiare sniedna pierwszych urodzajów twoich;
\par 15 I nalejesz na nia oliwy, a nakladziesz na nia kadzidla; bo ofiara sniedna jest.
\par 16 Tedy zapali kaplan pamiatke jej ze zboza wykruszonego jej, i z oliwy jej, ze wszystkiem kadzidlem jej; bo ofiara ognista jest Panu.

\chapter{3}

\par 1 A jezliby ofiara spokojna byla ofiara jego, a bylaby z rogatego bydla, ofiarowac bedzie samca, albo samice; zupelnie ofiarowac je bedzie przed obliczem Panskiem.
\par 2 I polozy reke swa na glowe ofiary swojej, i zabije ja kaplan przede drzwiami namiotu zgromadzenia; i wyleja synowie Aaronowi, kaplani, krew na wierzch oltarza w okolo.
\par 3 Potem ofiarowac bedzie z ofiary spokojnej palona ofiare Panu; tlustosc okrywajaca wnetrznosci, i wszystke tlustosc, która jest na wnetrznosciach.
\par 4 Obie tez nerki z tlustoscia, która jest na nich, i na poledwicach, i odzieczke, która na watrobie, z nerkami odejmie.
\par 5 I zapala to synowie Aaronowi na oltarzu, pospolu z ofiara calopalenia, która bedzie na drwach, które sa na ogniu; ofiara to ognista ku wdziecznej wonnosci Panu.
\par 6 Ale jezliby z drobnego bydla byla ofiara jego na ofiare spokojna Panu, samca albo samice zupelne ofiarowac je bedzie.
\par 7 Jezliby baranka ofiarowal na ofiare swoje, ofiarowac go bedzie przed obliczem Panskiem.
\par 8 A wlozy reke swa na glowe ofiary swojej, i zabije ja przed namiotem zgromadzenia; i wyleja synowie Aaronowi krew jej na wierzch oltarza w okolo.
\par 9 I bedzie ofiarowal z ofiary spokojnej ofiare ognista Panu, tlustosc jej, ogon caly, który od grzbietu odejmie, takze i tlustosc, okrywajaca wnetrznosci, i wszystke tlustosc, która jest na wnetrznosciach.
\par 10 Obie tez nerki z tlustoscia, która jest na nich, i na poledwicach, i odzieczke, która na watrobie, z nerkami odejmie.
\par 11 I zapali to kaplan na oltarzu, pokarm to jest ofiary ognistej Panu.
\par 12 Jezliby zas koza byla ofiara jego, tedy ja ofiarowac bedzie przed obliczem Panskiem.
\par 13 I polozy reke swoje na glowe jej, i zabije ja przed namiotem zgromadzenia, i wyleja synowie Aaronowi krew jej na wierzch oltarza w okolo.
\par 14 I ofiarowac bedzie z niej ofiare swoje na ofiare ognista Panu, tlustosc okrywajaca wnetrznosci, i wszystke tlustosc, która jest na wnetrznosciach.
\par 15 Obie tez nerki z tlustoscia która jest na nich i na poledwicach, i odzieczke która na watrobie, z nerkami odejmie.
\par 16 I zapali to kaplan na oltarzu, pokarm to jest ofiary ognistej na wdzieczna wonnosc; bo wszystka tlustosc jest Panska.
\par 17 Prawem wiecznem w narodziech waszych, we wszystkich mieszkaniach waszych, zadnej tlustosci i zadnej krwi jesc nie bedziecie.

\chapter{4}

\par 1 I rzekl Pan do Mojzesza mówiac:
\par 2 Mów do synów Izraelskich, a rzecz: Gdyby kto zgrzeszyl z niewiadomosci, przeciw któremu ze wszystkich przykazan Panskich, czyniac, czego by czynic nie mial, a przestapilby jedno z nich,
\par 3 Jezliby kaplan pomazany zgrzeszyl jako jeden z ludu pospolitego, tedy niech ofiaruje za grzech swój, którego sie dopuscil, cielca mlodego zupelnego Panu na ofiare za grzech.
\par 4 I przywiedzie cielca onego do drzwi namiotu zgromadzenia przed oblicznosc Panska, a wlozy reke swoje na glowe onego cielca, i zabije go przed obliczem Panskiem.
\par 5 Tedy wezmie kaplan pomazany ze krwi onego cielca, i wniesie ja do namiotu zgromadzenia.
\par 6 Potem omoczy kaplan palec swój we krwi, a kropic bedzie ona krwia siedem kroc przed obliczem Panskiem przed zaslona swiatnicy.
\par 7 I pomaze kaplan krwia ona rogi oltarza kadzenia wonnego, przed obliczem Panskiem, który jest w namiocie zgromadzenia, a ostatek krwi onego cielca wyleje u spodku oltarza calopalenia, który jest u drzwi namiotu zgromadzenia.
\par 8 Wszystke zas tlustosc cielca tego za grzech ofiarowanego wyjmie z niego tlustosc okrywajaca wnetrznosci i wszystke tlustosc, która jest na wnetrznosciach.
\par 9 Obie tez nerki z tlustoscia, która jest na nich, i na poledwicach, i odzieczke, która jest na watrobie i na nerkach, odejmie.
\par 10 Jako odejmuja z wolu ofiary spokojnej, i zapali to kaplan na oltarzu calopalonych ofiar.
\par 11 Skóre zas cielca tego, i wszystko mieso jego z glowa jego i z nogami jego i z wnetrznosciami jego i z gnojem jego
\par 12 Owa, calego cielca wyniesie precz za obóz na miejsce czyste, tam gdzie sie wysypuje popiól, i spali go na drwach ogniem, gdzie wysypuja popiól, tam spalony bedzie.
\par 13 Jezliby tez wszystko zgromadzenie Izraelskie z nieobaczenia zgrzeszylo, a bylaby rzecz zakryta od oczu zgromadzenia tego, i uczyniliby przeciw któremu ze wszystkich przykazan Panskich, coby byc nie mialo, a byliby winni,
\par 14 I poznaliby grzech, którym zgrzeszyli, ofiarowac bedzie ono zgromadzenie cielca mlodego na ofiare za grzech, a przywioda go przed namiot zgromadzenia,
\par 15 I poloza starsi zgromadzenia rece swe na glowe cielca onego przed obliczem Panskiem i zabija tegoz cielca przed obliczem Panskiem.
\par 16 Tedy wniesie kaplan pomazany, ze krwi cielca onego do namiotu zgromadzenia,
\par 17 I omoczy kaplan palec swój w onej krwi, a bedzie nia kropil siedem kroc przed obliczem Panskiem, przed zaslona.
\par 18 A ona krwia pomaze rogi oltarza, który jest przed obliczem Panskiem, w namiocie zgromadzenia; a ostatek krwi wyleje u spodku oltarza calopalenia, który jest u drzwi namiotu zgromadzenia.
\par 19 Wszystke tez tlustosc jego wyjmie z niego, i zapali na oltarzu.
\par 20 I uczyni z tem cielcem, jako uczynil z cielcem, za grzech ofiarowanym, tak uczyni z nim; a tak oczysci je kaplan, i bedzie im odpuszczono.
\par 21 Potem wyniesie cielca onego precz za obóz, i spali go, jako spalil cielca pierwszego. Tac jest ofiara za grzech zgromadzenia.
\par 22 Jezliby ksiaze zgrzeszyl, i uczynil przeciw któremu ze wszystkich przykazan Pana Boga swego, coby byc nie mialo, a to z nieobaczenia, a bylby winien:
\par 23 I bylby jawny grzech jego, którym zgrzeszyl, przywiedzie na ofiare swoje kozla z kóz, samca zupelnego;
\par 24 I polozy reke swoje na glowe tegoz kozla, i zabije go na miejscu, gdzie bija ofiary na calopalenie, przed obliczem Panskiem. Ofiara to jest za grzech.
\par 25 I wezmie kaplan ze krwi ofiary za grzech na palec swój, a pomaze rogi oltarza calopalonych ofiar, a ostatek krwi jego wyleje u spodku oltarza calopalenia.
\par 26 Wszystke zas tlustosc jego zapali na oltarzu, jako i tlustosc ofiary spokojnej; a tak oczysci go kaplan od grzechu jego i bedzie mu odpuszczony.
\par 27 A jezliby kto zgrzeszyl z ludu pospolitego z nieobaczenia, a uczynilby przeciw któremu z przykazan Panskich, coby byc nie mialo, i bylby winien,
\par 28 A bylby znajomy grzech jego, którym zgrzeszyl, przywiedzie ofiare swoje, koze z kóz, zupelna samice, za grzech swój, którego sie dopuscil.
\par 29 A polozywszy reke swa na glowe tej ofiary za grzech, zabije te ofiare za grzech na miejscu ofiar calopalonych.
\par 30 Potem wziawszy kaplan ze krwi onej na palec swój, pomaze rogi oltarza calopalonych ofiar, a ostatek krwi jej wyleje u spodku onegoz oltarza.
\par 31 Wszystke takze tlustosc jej odejmie, jako sie odejmuje tlustosc od ofiary spokojnej, i spali to kaplan na oltarzu ku wdziecznej wonnosci Panu; a tak oczysci go kaplan, i bedzie mu odpuszczono.
\par 32 A jezliby owce przywiódl na ofiare swoje za grzech, samice zupelna niech przywiedzie.
\par 33 I wlozy reke swa na glowe onej ofiary za grzech, i zabije ja na ofiare za grzech na miejscu, gdzie zabijaja ofiary calopalenia.
\par 34 Potem wezmie kaplan onej krwi z ofiary za grzech na palec swój, i pomaze rogi u oltarza palonych ofiar, a ostatek krwi jej wyleje u spodku onego oltarza.
\par 35 I wszystke tlustosc jej odejmie, jako odejmuja tlustosc baranka z ofiary spokojnej, i spali ja kaplan na oltarzu na calopalona ofiare Panu. A tak oczysci go kaplan od grzechu jego, którym zgrzeszyl, i bedzie mu odpuszczony.

\chapter{5}

\par 1 Jezliby tez czlowiek zgrzeszyl, zeby slyszal glos bluznierstwa, a bylby swiadkiem tego, co albo widzial, albo slyszal, a nie oznajmilby, poniesie karanie za nieprawosc swoje.
\par 2 Albo jezliby sie kto dotknal rzeczy nieczystej, badz scierwu zwierza nieczystego, badz scierwu bydlecia nieczystego, badz scierwu gadziny nieczystej, a byloby to zakryto przed nim, przecie nieczysty bedzie, i winien jest.
\par 3 Albo jezliby sie kto dotknal nieczystosci czlowieczej, jakazbykolwiek byla nieczystosc jego, przez która sie nieczystym stawa, a byloby to skryto przed nim, i dowiedzialby sie, winien jest.
\par 4 Albo jezliby kto przysiagl wymówiwszy usty, ze zle albo dobrze uczynil, o wszystko, co wymawia czlowiek z przysiega, a byloby to skryto przed nim i dowiedzialby sie potem, ze winien jest w jednej rzeczy z tych:
\par 5 Bedac tedy winien w jednej rzeczy z tych, wyzna grzech swój;
\par 6 I przywiedzie ofiare za wine swoje Panu za grzech swój, którym zgrzeszyl, samice z drobnego bydla, owce, albo koze za grzech, a oczysci go kaplan od grzechu jego.
\par 7 A jezliby nie przemógl ofiarowac bydlatka, tedy przyniesie ofiare za wystepek swój, którym zgrzeszyl, pare synogarlic, albo pare golabiat Panu, jedno na ofiare za grzech, a drugie na ofiare calopalenia;
\par 8 I przyniesie je do kaplana, a on naprzód ofiarowac bedzie to, co ma byc na ofiare za grzech, i paznogciem nadrze glowe jego ku szyi, ale jej nie oderwie.
\par 9 I pokropi krwia z ofiary za grzech strone oltarza, a ostatek onej krwi wycisnie u spodku oltarza. Ofiara to za grzech jest.
\par 10 Z drugiego zasie uczyni ofiare calopalenia wedlug zwyczaju. A tak oczysci go kaplan od grzechu jego, którym zgrzeszyl, a bedzie mu odpuszczony.
\par 11 A jezliby nie przemógl ofiarowac pare synogarlic, albo pare golabiat, tedy przyniesie ofiare swoje za to, co zgrzeszyl, dziesiata czesc efy maki pszennej na ofiare za grzech, nie naleje na nie oliwy, ani wlozy na nie kadzidla; bo jest ofiara za grzech.
\par 12 A gdy ja przyniesie do kaplana, tedy nabrawszy kaplan z niej pelna garsc swoje na pamiatke jego, spali ja na oltarzu mimo ofiare ognista Panu; ofiara to za grzech jest.
\par 13 I oczysci go kaplan od grzechu jego, którym zgrzeszyl w którejkolwiek z tych rzeczy, a bedzie mu odpuszczony; a ostatek bedzie kaplanowi, jako przy ofierze sniednej.
\par 14 Nad to rzekl Pan do Mojzesza, mówiac:
\par 15 Gdyby czlowiek przestapil przestepstwem, a zgrzeszylby z niewiadomosci, ujmujac rzeczy poswieconych Panu, tedy przywiedzie ofiare za wystepek swój Panu, barana zupelnego z drobnego bydla wedlug oszacowania twego, za dwa sykle srebra wedlug sykla swiatnicy, na ofiare za wystepek.
\par 16 A to, coby wzial z poswieconych rzeczy, wróci, i piata czesc nadto przyda i odda kaplanowi; a kaplan go oczysci przez barana ofiary za grzech, a bedzie mu odpuszczono.
\par 17 Jezliby tez czlowiek zgrzeszyl, uczyniwszy przeciw któremukolwiek z przykazan Panskich, coby nie mialo byc, z niewiadomosci, a bylby winien, poniesie karanie za nieprawosc swoje.
\par 18 Tedy przywiedzie barana zupelnego z drobnego bydla wedlug szacunku twojego, na ofiare za wystepek do kaplana; i oczysci go kaplan od niewiadomosci jego, której sie dopuscil nie wiedzac, a bedzie mu odpuszczono.
\par 19 Ofiara to za wystepek jest, którym wystapil przeciwko Panu.

\chapter{6}

\par 1 Potem rzekl Pan do Mojzesza, mówiac:
\par 2 Gdyby czlowiek zgrzeszyl, a wystepkiem wystapil przeciwko Panu, a zaprzalby rzeczy sobie zwierzonej, i do schowania danej, alboby co wydarl, alboby gwaltem wzial od blizniego swego;
\par 3 Takze jezliby rzecz zgubiona znalazl, a zaprzalby jej, alboby tez przysiagl falszywie o którakolwiek rzecz z tych, które czyni czlowiek, grzeszac przez nie;
\par 4 Gdyby tedy zgrzeszyl, a winien byl, wróci rzecz, która wydarl, albo która gwaltem wzial, albo tez rzecz sobie powierzona albo rzecz zgubiona, która znalazl;
\par 5 Albo tez o cobykolwiek falszywie przysiagl: tedy wróci to cale, i piata czesc do tego przyda temu, czyje bylo to; wróci w dzien ofiary za grzech swój.
\par 6 A ofiare za wystepek swój przywiedzie Panu, barana zupelnego z drobnego bydla wedlug oszacowania twego na ofiare za grzech do kaplana.
\par 7 I oczysci go kaplan przed Panem, a bedzie mu odpuszczona kazda z tych rzeczy, która uczynil, i byl jej winien.
\par 8 I rzekl Pan do Mojzesza, mówiac:
\par 9 Rozkaz Aaronowi i synom jego, i rzecz: Tac bedzie ustawa ofiary calopalenia; ofiara calopalenia jest od palenia na oltarzu, przez cala noc az do poranku; bo ogien na oltarzu ustawicznie gorzec bedzie.
\par 10 I oblecze sie kaplan w odzienie swoje lniane, i ubiory lniane oblecze na cialo swoje, i zbierze popiól, gdy spali ogien ofiare calopalenia na oltarzu, a wysypie go podle oltarza.
\par 11 Potem zewlecze szaty swe, i oblecze sie w szaty inne, a wyniesie popiól on za obóz na miejsce czyste.
\par 12 A ogien na oltarzu ustawicznie gorzec bedzie, nie bedzie gaszony; a bedzie zapalal na nim kaplan drwa na kazdy poranek, i wlozy nan ofiare calopalenia, a palic bedzie na nim tlustosc ofiar spokojnych.
\par 13 Ogien ustawicznie bedzie gorzal na oltarzu, nie bedzie gaszony.
\par 14 A tac tez jest ustawa ofiary sniednej, która ofiarowac beda synowie Aaronowi przed obliczem Panskiem u oltarza.
\par 15 Wezmie garsc swoje pszennej maki z tej ofiary sniednej, i z oliwy jej ze wszystkiem kadzidlem, które bedzie na ofierze sniednej, i to spali na oltarzu ku wdziecznej wonnosci na pamiatke jej Panu.
\par 16 A co zostanie z niej, jesc beda Aaron i synowie jego; bez kwasu jedzone bedzie na miejscu swietem; w sieni namiotu zgromadzenia jesc to beda.
\par 17 Nie beda tego wazyc z kwasem bo za dzial ich dalem im to, z ofiar moich ognistych; rzecz najswietsza to jest jako i ofiara za grzech, i jako ofiara za wystepek.
\par 18 Kazdy mezczyzna z synów Aaronowych jesc to bedzie; ustawa to wieczna w narodziech waszych o palonych ofiarach Panskich; wszystko, co sie ich dotknie, poswiecone bedzie.
\par 19 Potem rzekl Pan do Mojzesza, mówiac:
\par 20 Tac jest ofiara Aarona, i synów jego, która ofiarowac beda Panu w dzien pomazania swego: dziesiata czesc efy maki pszennej za ofiare sniedna ustawiczna, polowe jej rano, a polowe jej w wieczór.
\par 21 W panwi z oliwa bedzie gotowana; smazona przyniesiesz ja, usmazona ofiare sniedna w sztukach ofiarowac bedziesz ku wdziecznej wonnosci Panu.
\par 22 A kaplan pomazany z synów jego po nim ofiarowac ja bedzie; ustawa to wieczna Panu, wszystka spalona bedzie.
\par 23 I kazda sniedna ofiara kaplanska, wszystka spalona bedzie; nie beda jej jesc.
\par 24 Zatem rzekl Pan do Mojzesza, mówiac:
\par 25 Mów do Aarona, i synów jego, a rzecz: Ta bedzie ustawa ofiary za grzech: Na miejscu, gdzie bija ofiary calopalenia, bedzie zabita ofiara za grzech przed Panem; rzecz najswietsza jest.
\par 26 Kaplan, który by ja ofiarowal za grzech, jesc ja bedzie; na miejscu swietem jedzona bedzie; w sieni namiotu zgromadzenia.
\par 27 Wszystko, co sie dotknie miesa jej, bedzie poswiecone; a jezliby krwia jej szata pokropiona byla, co sie pokropilo, omyjesz na miejscu swietem.
\par 28 Naczynie tez gliniane, w którem by ja warzono, stluczone bedzie; a jezliby w naczyniu miedzianem warzona byla, wytra je, i wymyja woda.
\par 29 Wszelki mezczyzna z kaplanów jesc ja bedzie; najswietsza to rzecz jest.
\par 30 A zadna ofiara za grzech, której krew wnaszana bywa do namiotu zgromadzenia dla oczyszczenia w swiatnicy, nie bedzie jedzona, ale ogniem spalona bedzie.

\chapter{7}

\par 1 Tac jest ustawa ofiary za wystepek, która jest najswietsza.
\par 2 Na miejscu, gdzie bija ofiary calopalenia, zabija ofiare za wystepek, a krwia jej pokropia oltarz z wierzchu w okolo.
\par 3 A wszystke tlustosc jej ofiarowac bedzie z niej, ogon i tlustosc okrywajaca wnetrznosci;
\par 4 Obiedwie tez nerki z tlustoscia, która jest na nich, i na poledwicach, i odzieczke, która jest na watrobie i na nerkach, odejmie.
\par 5 I spali to kaplan na oltarzu na ofiare ognista Panu; ofiara to jest za wystepek.
\par 6 Wszelki mezczyzna z kaplanów bedzie ja jadl, na miejscu swietem jedzona bedzie; rzecz to najswietsza.
\par 7 Jako ofiara za grzech, tak ofiara za wystepek jednaka ustawe maja; kaplanowi, który by go oczyszczal, nalezec bedzie.
\par 8 Kaplanowi, który by czyje ofiare calopalona ofiarowal, skóra tejze ofiary, która ofiarowal, nalezec bedzie.
\par 9 Takze kazda ofiara sniedna w piecu upieczona, i wszystko, co na panwi albo w kotle gotowane bedzie, kaplanowi, który to ofiaruje, nalezec bedzie.
\par 10 Przytem wszelaka ofiara sniedna, zagnieciona z oliwa albo prazona, wszystkim synom Aaronowym nalezec bedzie, tak jednemu, jako drugiemu.
\par 11 Tac tez jest ustawa ofiary spokojnej, która beda ofiarowali Panu.
\par 12 Jezliby kto ofiarowal na ofiare dziekczynienia, tedy ofiarowac bedzie na ofiare dziekczynienia placki przasne, zagniatane z oliwa, i kreple przasne, pomazane oliwa, i make pszenna, smazona z temi plackami w oliwie zagniecionemi.
\par 13 Przy tych plackach bedzie tez chleb kwaszony ofiarowal na ofiare swoje z ofiara dziekczynienia spokojnych ofiar swoich.
\par 14 I bedzie ofiarowal z niego jeden chleb z kazdej ofiary na podnoszenie Panu. Kaplanowi, który kropi krwia ofiar spokojnych, nalezec to bedzie.
\par 15 Mieso zas ofiary dziekczynienia, która jest spokojna, w dzien ofiarowania ofiary jego jedzone bedzie; nie zostawia z niego nic do jutra.
\par 16 A jezliby kto slubna albo dobrowolna przyniósl ofiare swoja, w dzien ofiarowania ofiary jego jedzone bedzie; a nazajutrz, coby zostalo z niej, zjedza.
\par 17 Ale jezliby co zostalo miesa z tej ofiary do trzeciego dnia, ogniem spalone bedzie.
\par 18 A jezliby kto przecie jadl mieso tej ofiary spokojnej dnia trzeciego, nie bedzie przyjemny ten, który ja ofiarowal; nie bedzie mu platna, owszem obrzydliwoscia bedzie; a kto by jadl z niej, nieprawosc swoje poniesie.
\par 19 Mieso tez, które by sie dotknelo czego nieczystego, nie bedzie jedzone, ale ogniem spalone bedzie; mieso zas inne, kazdy czysty jesc je bedzie.
\par 20 A ktobykolwiek jadl mieso z ofiary spokojnej, ofiarowanej Panu, a bylby nieczysty, wytracony bedzie czlowiek ten z ludu swego.
\par 21 Jezliby sie tez kto dotknal czego nieczystego, badz nieczystosci czlowieczej, badz nieczystosci bydlecej, badz jakiejkolwiek obrzydliwosci nieczystej, a jadlby mieso z ofiary spokojnej, ofiarowanej Panu, tedy wytracony bedzie czlowiek ten z ludu swego.
\par 22 Rzekl jeszcze Pan do Mojzesza, mówiac:
\par 23 Powiedz synom Izraelskim, i rzecz: Zadnej tlustosci z wolu, ani z owiec, ani z kozy, nie bedziesz jadl,
\par 24 Aczkolwiek tlustosc bydlecia zdechlego, albo tlustosc rozszarpanego moze byc do wszelakiej potrzeby; ale jesc jej zadnym sposobem nie bedziecie.
\par 25 Albowiem ktobykolwiek jadl tlustosc z bydlecia, które ofiarowac bedzie czlowiek na ofiare ognista Panu, niechaj wytracony bedzie czlowiek ten, który jadl, z ludu swego.
\par 26 Takze zadnej krwi jesc nie bedziecie we wszystkich mieszkaniach waszych, tak z ptaków, jako i z bydlat.
\par 27 Wszelki czlowiek, który by jadl jakakolwiek krew, wytracony bedzie czlowiek on z ludu swego.
\par 28 Rzekl jeszcze Pan do Mojzesza, mówiac:
\par 29 Mów do synów Izraelskich, a rzecz im: Kto by ofiarowal ofiare spokojna swoje Panu, przyniesie ofiare swoje Panu z ofiary spokojnej swojej.
\par 30 Reka jego przyniesie ofiare ognista Panu; tlustosc z mostkiem przyniesie, a mostek niech bedzie tam i sam obracany na ofiare przed Panem.
\par 31 Potem spali kaplan tlustosc na oltarzu; ale mostek zostanie Aaronowi i synom jego.
\par 32 A lopatke prawa oddacie na podnoszenie kaplanowi z ofiar spokojnych waszych.
\par 33 Kto tez z synów Aaronowych ofiarowac bedzie krew ofiar spokojnych i tlustosc, temu sie dostanie lopatka prawa dzialem.
\par 34 Albowiem mostek sam i tam obracania, i lopatke podnoszenia, wzialem od synów Izraelskich z ofiar ich spokojnych, i dalem je Aaronowi kaplanowi, i synom jego prawem wiecznem od synów Izraelskich.
\par 35 Toc jest dzial pomazanego Aarona, i pomazanych synów jego z ofiar ognistych Panskich, od dnia, któregom im przystapic rozkazal ku sprawowaniu urzedu kaplanskiego Panu.
\par 36 I rozkazal Pan, aby im to dawano bylo od dnia, którego je pomazal, od synów Izraelskich prawem wiecznem w narodziech ich.
\par 37 Tac jest ustawa ofiary calopalenia, ofiary sniednej, i ofiary za grzech, i za wystepek, i poswiecenia, i ofiary spokojnej.
\par 38 Która rozkazal Pan Mojzeszowi na górze Synaj, dnia, którego przykazywal synom Izraelskim, aby ofiarowali ofiary swe Panu na puszczy Synaj.

\chapter{8}

\par 1 Potem rzekl Pan do Mojzesza, mówiac:
\par 2 Wezmij Aarona i syny jego z nim, i szaty ich, i olejek pomazywania, i cielca na ofiare za grzech, i dwa barany, i kosz chlebów przasnych.
\par 3 A wszystek lud zbierz do drzwi namiotu zgromadzenia.
\par 4 I uczynil Mojzesz, jako mu rozkazal Pan; i zebral sie wszystek lud do drzwi namiotu zgromadzenia.
\par 5 Tedy rzekl Mojzesz do zgromadzenia: Toc jest slowo, które rozkazal Pan czynic.
\par 6 A kazawszy przystapic Mojzesz Aaronowi i synom jego, omyl je woda;
\par 7 I oblekl go w suknia, a opasal go pasem, i odzial go plaszczem, i na wierzch wlozyl nan naramiennik, i przypasal go pasem naramiennika, i opasal go nim.
\par 8 Wlozyl tez nan napiersnik, i przyprawil do niego Urim i Tummim.
\par 9 Takze wlozyl czapke na glowe jego, a wlozyl na czapke na przodek blache zlota, korone swieta, jako byl rozkazal Pan Mojzeszowi.
\par 10 Wzial tez Mojzesz olejek pomazywania, i pomazal przybytek, i wszystkie rzeczy, które w nim byly, i poswiecil je.
\par 11 Potem pokropil nim oltarz siedem kroc, i pomazal oltarz ze wszystkiem naczyniem jego, i wanne, i stolec jej, aby je poswiecil.
\par 12 Wlal takze olejku pomazywania na glowe Aaronowe, i pomazal go na poswiecenie jego.
\par 13 Zatem rozkazal Mojzesz przystapic synom Aaronowym, a oblóklszy je w szaty, opasal je pasem, i wlozyl na nie czapki, jako byl Pan rozkazal Mojzeszowi.
\par 14 Tamze przywiódl cielca ku ofierze za grzech; i wlozyli Aaron i synowie jego rece swe na glowe cielca ofiary za grzech.
\par 15 I zabil go Mojzesz, a wziawszy krwi jego, pomazal rogi oltarza w okolo palcem swym, i oczyscil oltarz. Ostatek zas krwi wylal u spodku oltarza, i poswiecil go dla oczyszczania na nim.
\par 16 Wzial potem wszystke, tlustosc, która na wnetrznosciach byla, i odzieczke z watroby, i dwie nerki z tlustoscia ich, i spalil to Mojzesz na oltarzu.
\par 17 A cielca z skóra jego i z miesem jego i z gnojem jego spalil ogniem precz za obozem, jako byl Pan rozkazal Mojzeszowi.
\par 18 Potem przywiódl barana na ofiare calopalenia; i wlozyli Aaron i synowie jego rece swe na glowe tego barana.
\par 19 I zabil go Mojzesz a pokropil krwia jego oltarz z wierzchu w okolo.
\par 20 A barana porabal na sztuki jego, i spalil Mojzesz glowe, i sztuki i tlustosc.
\par 21 A wnetrznosci i nogi oplukal woda; i tak spalil Mojzesz wszystkiego barana na oltarzu. Calopalenie to jest ku wdziecznej wonnosci, ofiara ognista jest Panu, jako byl Pan rozkazal Mojzeszowi.
\par 22 Potem kazal przywiesc barana drugiego, barana poswiecenia; i wlozyli Aaron i synowie jego rece swoje na glowe tegoz barana.
\par 23 A zabiwszy go Mojzesz wzial ze krwi jego, i pomazal nia koniec prawego ucha Aaronowego, i wielki palec prawej reki jego, takze palec wielki prawej nogi jego.
\par 24 Rozkazal tez przystapic synom Aaronowym, i pomazal Mojzesz onaz krwia koniec ucha ich prawego, i palce wielkie ich prawej reki, i palce wielkie nogi ich prawej; i wylal Mojzesz krew na wierzch oltarza w okolo.
\par 25 Potem wzial tlustosc i ogon, i wszystke tlustosc, która jest okolo wnetrznosci, i odzieczke z watroby, i dwie nerki z tlustoscia ich i lopatke prawa.
\par 26 Takze z kosza przasnych chlebów, które byly przed Panem, wzial placek przasny jeden, i bochen chleba z oliwa jeden, i krepel jeden, a polozyl je na onych tlustosciach i na lopatce prawej.
\par 27 I dal to wszystko w rece Aaronowi i w rece synów jego, i obracal to tam i sam za ofiare obracania przed Panem.
\par 28 Potem ono wzial Mojzesz z rak ich, a spalil na oltarzu na calopalenie; poswiecenie to jest na wdzieczna wonnosc, ofiara ognista jest Panu.
\par 29 Wzial tez Mojzesz mostek, i obracal go sam i tam za ofiare obracania przed Panem; a z barana poswiecenia dostal sie Mojzeszowi dzial, jako mu byl rozkazal Pan.
\par 30 Wzial jeszcze Mojzesz olejku pomazywania i krwi, która byla na oltarzu, a pokropil Aarona i szaty jego, takze syny jego, i szaty synów jego z nim. A tak poswiecil Aarona i szaty jego, i syny jego, i szaty synów jego z nim.
\par 31 I rzekl Mojzesz do Aarona, i do synów jego: Warzcie to mieso u drzwi namiotu zgromadzenia, i tam je jedzcie, i chleb, który jest w koszu poswiecenia, jakom przykazal, mówiac: Aaron i synowie jego beda je jedli.
\par 32 A coby zostalo z miesa i z chleba, ogniem spalicie.
\par 33 A ze drzwi namiotu zgromadzenia nie wychodzcie przez siedem dni, az do dnia, którego sie wypelni czas poswiecenia waszego; bo przez siedem dni poswiecane beda rece wasze.
\par 34 Jako sie stalo dzis, tak przykazal Pan czynic na oczyszczenie wasze.
\par 35 Przetoz przy drzwiach namiotu zgromadzenia trwac bedziecie we dnie i w nocy przez siedem dni, a strzec bedziecie rozrzadzenia Panskiego, abyscie nie pomarli; bo mi tak rozkazano.
\par 36 I uczynili Aaron i synowie jego to wszystko, co im rozkazal Pan przez Mojzesza.

\chapter{9}

\par 1 I stalo sie dnia ósmego, wezwal Mojzesz Aarona i synów jego, i starszych Izraelskich.
\par 2 I rzekl do Aarona: Wezmij sobie cielca mlodego na ofiare za grzech, i barana na ofiare calopalenia, oboje zupelne, i ofiaruj je przed obliczem Panskiem.
\par 3 Do synów zas Izraelskich rzeczesz, mówiac: Wezmijcie kozla z kóz na ofiare za grzech, i cielca, i barana, roczniaki zupelne, zdrowe, na ofiare calopalenia;
\par 4 Takze wolu, i barana na ofiary spokojne ku ofiarowaniu przed Panem, i ofiare sniedna nagnieciona z oliwa; albowiem sie wam dzis Pan ukaze.
\par 5 I przyniesli, co rozkazal Mojzesz, przed namiot zgromadzenia; a przystapiwszy wszystek lud, stanal przed Panem.
\par 6 Zatem rzekl Mojzesz: Tac jest rzecz, która wam Pan rozkazal; czynciez ja, a ukaze sie wam chwala Panska.
\par 7 Rzekl zas Mojzesz do Aarona: Przystap do oltarza, a uczyn ofiare za grzech swój, i ofiare palona twoje, a wykonaj oczyszczenie za sie i za lud; uczyn tez ofiare od ludu, i uczyn oczyszczenie za lud, jako rozkazal Pan.
\par 8 Tedy przystapil Aaron do oltarza, i zabil cielca na ofiare za grzech swój.
\par 9 I podali mu synowie Aaronowi krew, który omoczywszy palec swój we krwi pomazal rogi oltarza, a ostatek krwi wylal u spodku oltarza;
\par 10 Ale tlustosc z nerkami, i odzieczke z watroba z ofiary za grzech spalil na oltarzu, jako byl rozkazal Pan Mojzeszowi;
\par 11 Mieso zas i skóre spalil ogniem precz za obozem.
\par 12 Zabil tez ofiare calopalenia; i podali mu synowie Aaronowi krew, która pokropil wierzch oltarza w okolo.
\par 13 Przyniesli mu tez ofiare calopalenia, i sztuki jej i glowe jej; a spalil ja na oltarzu;
\par 14 Omyl tez wnetrznosci, i nogi, i spalil je z ofiara calopalenia na oltarzu.
\par 15 Potem sprawowal ofiare wszystkiego ludu, i wzial kozla na ofiare za grzech ludu, którego zabil, i ofiarowal go, jako i pierwszego.
\par 16 Ofiarowal tez ofiare calopalenia, i uczynil jej wedlug zwyczaju.
\par 17 Ofiarowal tez ofiare sniedna, a wziawszy z niej pelna garsc swoje, spalil na oltarzu oprócz ofiary calopalenia porannej.
\par 18 Zabil tez wolu, i barana na ofiare spokojna, która byla za lud; i podali mu synowie Aaronowi krew, która pokropil oltarz z wierzchu okolo.
\par 19 Podali mu takze tlustosc z wolu, i z barana ogon, i tlustosc okrywajaca wnetrznosci i nerki, i odzieczke z watroby.
\par 20 Wlozyli tez tlustosci na mostek, i spalili tez tlustosc na oltarzu;
\par 21 Ale mostek i lopatke prawa obracal Aaron tam i sam na ofiare obracania przed obliczem Panskiem, jako byl Pan rozkazal Mojzeszowi.
\par 22 Tedy podnióslszy Aaron rece swe do ludu blogoslawil im, a zstapil od ofiarowania ofiary za grzech, i ofiary calopalenia, i ofiary spokojnej.
\par 23 I wszedl Mojzesz i Aaron do namiotu zgromadzenia, a wszedlszy blogoslawili ludowi; i okazala sie chwala Panska wszystkiemu ludowi;
\par 24 Bo zstapiwszy ogien od oblicznosci Panskiej spalil na oltarzu ofiare calopalenia i tlustosci; co gdy widzial wszystek lud, wykrzykali a padali na twarzy swoje.

\chapter{10}

\par 1 Tedy synowie Aaronowi, Nadab i Abiju, wziawszy kazdy kadzielnice swoje, wlozyli w nia ognia, i wlozywszy nan kadzidla ofiarowali przed obliczem Panskiem ogien obcy, czego im byl nie rozkazal.
\par 2 Przetoz wyszedlszy ogien od twarzy Panskiej, porazil je; i pomarli przed Panem.
\par 3 Zatem rzekl Mojzesz do Aarona: Toc to jest, co powiedzial Pan, mówiac: W tych, którzy przystepuja do mnie, poswiecony bede, i przed oblicznoscia wszystkiego ludu uwielbiony bede; i zamilkl Aaron.
\par 4 Tedy wezwal Mojzesz Misaela i Elisafana, synów Husyjela, stryja Aaronowego, i rzekl do nich: Przystapcie, a wyniescie bracia wasze z swiatnicy precz za obóz.
\par 5 Przyszli tedy, a wyniesli je i z szatami ich precz za obóz, jako byl rozkazal Mojzesz.
\par 6 Rzekl potem Mojzesz do Aarona i do Eleazara i do Itamara, synów jego: Glów waszych nie obnazajcie, ani szat swych rozdzierajcie, byscie nie pomarli, a Bóg nie rozgniewal sie na wszystko zgromadzenie. Ale bracia wasi, wszystek dom synów Izraelskich, niech placza tego spalenia, które uczynil Pan.
\par 7 Ze drzwi tez namiotu zgromadzenia nie wychodzcie, byscie snac nie pomarli; albowiem olejek pomazania Panskiego jest na was. I uczynili wedlug rozkazania Mojzeszowego.
\par 8 Zatem rzekl Pan do Aarona, mówiac:
\par 9 Wina i napoju mocnego nie bedziesz pil, ty i synowie twoi z toba, gdy bedziecie mieli wchodzic do namiotu zgromadzenia, abyscie nie pomarli; ustawa to wieczna bedzie w narodziech waszych;
\par 10 Abyscie rozeznawac mogli miedzy rzecza swieta, i miedzy rzecza pospolita, i miedzy rzecza nieczysta, i miedzy rzecza czysta;
\par 11 Azebyscie nauczali synów Izraelskich wszystkich ustaw, które im rozkazal Pan przez Mojzesza.
\par 12 Mówil potem Mojzesz do Aarona i do Eleazara i Itamara, synów jego, którzy byli pozostali: Wezmijcie ofiare sniedna, która zostala od ognistych ofiar Panskich, a jedzcie ja z przasnikami przy oltarzu; bo rzecz najswietsza jest.
\par 13 Przetoz jesc ja bedziecie na miejscu swietem, bo to prawo twoje, i prawo synów twoich, z ognistych ofiar Panskich; bo mi tak rozkazano.
\par 14 Takze mostek obracania, i lopatke podnoszenia bedziecie jedli na miejscu czystem, ty i synowie twoi, i córki twoje z toba; albowiem to prawem tobie, i prawem synom twoim dano z ofiar spokojnych synów Izraelskich.
\par 15 Lopatke podnoszenia, i mostek obracania z ofiarami ognistemi, i tlustoscia przyniosa, aby je tam i sam obracano przed obliczem Panskiem; a to bedzie tobie i synom twoim z toba prawem wiecznem, jako rozkazal Pan.
\par 16 Potem Mojzesz szukal pilnie kozla ofiarowanego za grzech, a oto, juz spalony byl; i dla tego rozgniewawszy sie na Eleazara i Itamara, syny Aaronowe, którzy byli pozostali, mówil:
\par 17 Przeczzescie nie jedli ofiary za grzech na miejscu swietem? albowiem to jest rzecz najswietsza, poniewaz ja wam dano, abyscie nosili nieprawosc wszystkiego ludu na oczyszczenie ich przed obliczem Panskiem.
\par 18 A oto, nie jest wniesiona krew jego wewnatrz do swiatnicy; mieliscie go jesc w swiatnicy, jakom rozkazal.
\par 19 Tedy Aaron odpowiedzial Mojzeszowi: Oto, dzis ofiarowali ofiare swoja za grzech, i ofiare calopalenia swego przed obliczem Panskiem, a oto mie spotkalo; gdybym byl jadl dzis ofiare za grzech, izaliby sie to bylo podobalo Panu?
\par 20 To gdy uslyszal Mojzesz, przestal na tem.

\chapter{11}

\par 1 Potem mówil Pan do Mojzesza i do Aarona, i rzekl do nich: Powiedzcie synom Izraelskim, mówiac:
\par 2 Te sa zwierzeta, które jesc bedziecie ze wszystkich zwierzat, które sa na ziemi,
\par 3 Wszelkie bydle, które ma rozdzielone stopy, i rozdwojone kopyta, a przezuwa, to jesc bedziecie.
\par 4 Ale z tych jesc nie bedziecie, które tylko przezuwaja, i z tych, które tylko kopyta dwoja: Wielblad, który choc przezuwa, ale kopyta rozdzielonego nie ma, nieczystym wam bedzie.
\par 5 Takze królik, który choc przezuwa, ale kopyta rozdzielonego nie ma, nieczystym wam bedzie.
\par 6 Zajac tez, choc przezuwa, ale kopyta rozdzielonego nie ma, nieczystym wam bedzie.
\par 7 Swinia takze, choc ma rozdzielone stopy i rozdwojone kopyto, ale iz nie przezuwa, nieczysta wam bedzie.
\par 8 Miesa ich nie bedziecie jesc, ani scierwu ich dotykac sie bedziecie, nieczyste wam beda.
\par 9 To jesc bedziecie ze wszystkich rzeczy zyjacych w wodach, wszystko co ma skrzele i luske, w wodach, w morzu, i w rzekach, to jesc bedziecie.
\par 10 Wszystko zas, co nie ma skrzeli i luski w morzu i w rzekach, cokolwiek sie rucha w wodach i kazda rzecz zywiaca, która jest w wodach obrzydliwoscia wam bedzie.
\par 11 Obrzydliwoscia beda wam; miesa ich jesc nie bedziecie, a scierwem ich brzydzic sie bedziecie.
\par 12 Owa cokolwiek nie ma skrzeli i luski w wodach, obrzydliwoscia wam bedzie.
\par 13 Tem sie tez brzydzic bedzie z ptastwa, i jesc ich nie bedziecie, bo sa obrzydliwoscia; jako orla, i gryfa, i morskiego orla,
\par 14 I sepa, i kani, wedlug rodzaju ich;
\par 15 Kazdego kruka wedlug rodzaju jego;
\par 16 Takze strusia, i sowy i wodnej kani i jastrzebia, wedlug rodzaju ich;
\par 17 I puchacza, i norka i lelka,
\par 18 I labedzia, i baka, i bociana,
\par 19 I czapli, i sojki, wedlug rodzaju ich, i dudka, i nietoperza.
\par 20 Wszystko, co sie czolga po ziemi skrzydla majac, a na czterech nogach chodzi, obrzydliwoscia wam bedzie.
\par 21 Wszakze jesc bedziecie wszystko, co sie czolga po ziemi skrzydla majace, co na czterech nogach chodzi, co ma w nogach sciegneczka przedluzsze ku skakaniu na nich po ziemi.
\par 22 Te z nich jesc bedziecie: Szarancza wedlug rodzaju jej, i koniki wedlug rodzaju ich, i skoczki wedlug rodzaju ich, i chrzaszcze wedlug rodzaju ich.
\par 23 Wszystko zas, co sie czolga po ziemi skrzydlaste, cztery nogi majace, obrzydliwoscia wam bedzie;
\par 24 Bo sie niemi pokalacie. Kto by sie dotknal zdechliny ich, nie bedzie czystym az do wieczora;
\par 25 A ktobykolwiek nosil scierw ich, upierze szaty swoje, i bedzie nieczystym az do wieczora.
\par 26 Wszelkie bydle, które ma rozdzielona stope, a kopyta rozdwojonego nie ma, ani tez przezuwa, nieczyste wam bedzie; kto by sie go dotknal, nieczystym bedzie.
\par 27 A cokolwiek chodzi na lapach swych ze wszystkich zwierzat, które chodza na czterech nogach, nieczyste wam bedzie; kto by sie dotknal scierwu ich, nieczystym bedzie az do wieczora.
\par 28 A kto by nosil scierw ich, upierze odzienie swe, a nieczystym bedzie az do wieczora, bo nieczyste wam sa.
\par 29 Takze i te za nieczyste miec bedziecie miedzy plazami, które sie wlócza po ziemi, lasica, i mysz, i zaba wedlug rodzajów swoich;
\par 30 I jez, i jaszczurka, i tchórz, i slimak, i kret.
\par 31 Te nieczyste wam beda miedzy wszystkiemi plazami; kto by sie dotknal zdechliny ich, nieczystym bedzie az do wieczora.
\par 32 A kazda rzecz, na która by co zdechlego z tych rzeczy upadlo, nieczysta bedzie, tak drzewiane naczynie, jako szata, tak skóra, jako wór; owa kazde naczynie, w którem co sprawuja, do wody wlozone bedzie, i nieczyste zostanie az do wieczora, potem czyste bedzie.
\par 33 Wszelkie zas naczynie gliniane, w które by co z tych rzeczy wpadlo, ze wszystkiem, coby w niem bylo, nieczyste sie stanie, a samo stluczone bedzie.
\par 34 Kazda tez potrawa, która jadaja, gdyby wody nieczystej do niej wlano, nieczysta bedzie; i wszelki napój, który pijaja z kazdego takiego naczynia, nieczystym bedzie.
\par 35 Owa wszystko, na coby upadlo co z onych zdechlin, nieczyste bedzie; piec i ognisko rozwalone beda, bo nieczyste sa, i za nieczyste wam beda.
\par 36 Ale studnia i cysterna, i kazde zgromadzenie wód czyste beda; coby sie jednak dotknelo scierwu tych rzeczy, nieczyste bedzie.
\par 37 A jezliby upadlo nieco z scierwu ich na jakie nasienie, które siane bywa, czyste zostanie.
\par 38 Ale jezliby na nasienie w wodzie moczone upadlo co z scierwu ich, nieczyste wam bedzie.
\par 39 Jezliby zdechlo bydle, które jadacie: kto by sie dotknal scierwu jego, nieczystym bedzie az do wieczora.
\par 40 A kto by jadl scierw jego, upierze szaty swoje, i nieczystym bedzie az do wieczora; ten, coby precz wynosil on scierw, upierze szaty swoje, i nieczystym bedzie az do wieczora.
\par 41 Wszelki takze plaz, co sie czolga po ziemi, obrzydliwoscia jest; nie bedziecie go jesc.
\par 42 Cokolwiek sie czolga po brzuchu, i cokolwiek na czterech albo wiecej nogach sie wlóczy miedzy wszystkim plazem, który sie czolga po ziemi, nie bedziecie ich jesc, bo obrzydliwoscia sa.
\par 43 Nie plugawcie dusz waszych wszelkim plazem, który sie czolga po ziemi, i nie mazcie sie niemi, byscie nie byli splugawieni przez nie;
\par 44 Albowiem Jam jest Pan Bóg wasz: przetoz poswiecajcie sie, a badzcie swietymi, bom Ja swiety jest; a nie plugawcie dusz waszych zadnym plazem, który sie czolga po ziemi.
\par 45 Bom Ja jest Pan, którym was wywiódl z ziemi Egipskiej, abym wam byl za Boga; przetoz badzcie swietymi, bom Ja swiety jest.
\par 46 Tac jest ustawa okolo bydla, i ptastwa, i wszelkiej duszy zywej, która sie rucha w wodach, i wszelkiej duszy zywej, która sie czolga po ziemi.
\par 47 Ku rozeznaniu miedzy nieczystem i miedzy czystem, a miedzy zwierzety, które sie jesc godzi, i miedzy zwierzety, których sie jesc nie godzi.

\chapter{12}

\par 1 Zatem rzekl Pan do Mojzesza, mówiac:
\par 2 Powiedz synom Izraelskim, i rzecz: Niewiasta, która by poczela i urodzila mezczyzne, nieczysta bedzie przez siedem dni: wedlug dni, których odlaczona bywa dla choroby swej, nieczysta bedzie.
\par 3 A dnia ósmego obrzezane bedzie cialo nieobrzeski jego.
\par 4 Ale ona przez trzydziesci dni i trzy dni zostanie we krwi oczyszczenia; zadnej rzeczy swietej nie dotknie sie, i ku swiatnicy nie pójdzie, az sie wypelnia dni oczyszczenia jej.
\par 5 A jezli dzieweczke urodzi, nieczysta bedzie przez dwie niedziele wedlug oddzielenia swego, a szescdziesiat dni i szesc dni zostanie we krwi oczyszczenia swego.
\par 6 A gdy sie wypelnia dni oczyszczenia jej po synu albo po córce, przyniesie baranka rocznego na ofiare calopalenia, i golabiatko, albo synogarlice na ofiare za grzech do drzwi namiotu zgromadzenia do kaplana;
\par 7 Którego ofiarowac bedzie przed obliczem Panskiem, i oczysci ja; a tak oczyszczona bedzie od plywania krwi swojej. Tac jest ustawa tej, która porodzila mezczyzne albo dzieweczke.
\par 8 A jezli nie przemoze dac baranka, tedy wezmie pare synogarlic, albo pare golabiat, jedno na ofiare calopalenia a drugie na ofiare za grzech; i oczysci ja kaplan, a tak oczyszczona bedzie.

\chapter{13}

\par 1 Rzekl zasie Pan do Mojzesza i do Aarona, mówiac:
\par 2 Czlowiek, który by mial na skórze ciala swego sadzel, albo liszaj, albo biale blizny, tak izby sie na skórze ciala jego okazala plaga tradu, przywiedziony bedzie do Aarona kaplana, albo do którego z synów jego kaplanów.
\par 3 Tedy oglada kaplan on sadzel na skórze ciala jego; jezliby wlos na onym sadzelu pobielal, a on sadzel na spojrzeniu bylby glebszy, niz insza skóra ciala jego, plaga tradu jest; przetoz ogladawszy go kaplan, osadzi go byc nieczystym.
\par 4 A jezliby biala tylko blizna byla na skórze ciala jego, a nie bylaby glebsza na spojrzeniu, niz inna skóra, i wlosy w niej nie pobielalyby, tedy zamknie kaplan majacego taka zaraze przez siedem dni.
\par 5 Potem obejrzy go kaplan dnia siódmego; a jezliby ona blizna tak zostala w oczach jego, a nie szerzyla sie ona blizna po skórze, tedy go zamknie kaplan przez siedem dni po wtóre.
\par 6 I obejrzy go kaplan dnia siódmego po wtóre; a jezliby ta zaraza poczerniala a nie szerzylaby sie ta zaraza po skórze, tedy go za czystego osadzi kaplan, bo swierzb jest; a on upierze szaty swe, a bedzie czystym.
\par 7 Ale jezliby sie bardziej rozszerzal po skórze swierzb on po ogladaniu kaplanowem i po oczyszczeniu jego, pokaze sie znowu kaplanowi.
\par 8 Tedy obejrzy go kaplan; a jezliby sie ta zaraza rozszerzala po skórze jego, osadzi go za nieczystego kaplan; bo trad jest.
\par 9 Zaraza tradu gdy bedzie na czlowieku, przywiedzion bedzie do kaplana.
\par 10 Którego obejrzy kaplan; a bedzieli sadzel bialy na skórze, zeby sie uczynily wlosy biale, chocby i zdrowe cialo bylo na tym sadzelu,
\par 11 Trad zastarzaly jest na skórze ciala jego; i osadzi go za nieczystego kaplan, a nie bedzie go zawieral, gdyz nieczystym jest.
\par 12 A jezliby sie trad rozszerzal na skórze, i okrylby trad wszystke skóre zarazonego od glowy jego az do nóg jego, wszedy gdzie oczyma kaplan dojrzec moze;
\par 13 I obejrzy kaplan; a jezli okryl trad wszystko cialo jego, za czysta osadzi zaraze jego; bo iz wszystka pobielala, dla tego czysty jest.
\par 14 Ale którego dnia okazaloby sie na takowym mieso dziwie, nieczystym bedzie.
\par 15 A tak oglada kaplan mieso dziwie, a osadzi go byc za nieczystego; bo ono mieso dziwie nieczyste jest; trad to jest.
\par 16 Ale jezliby zas zginelo mieso dziwie, i obróciloby sie w bialosc, tedy przyjdzie do kaplana.
\par 17 A widzac go kaplan, iz sie obrócila zaraza w bialosc, za czystego osadzi kaplan zarazonego; bo czystym jest.
\par 18 Jezliby zas byl na skórze ciala jego wrzód, a zagoilby sie;
\par 19 A na miejscu wrzodu onego uczynilby sie sadzel bialy, albo blizna biala zaczerwieniala, tedy okazana bedzie kaplanowi.
\par 20 A widzac kaplan, ze na wejrzeniu glebsza jest, niz inna skóra, i wlosy by jej pobielaly, za nieczystego osadzi go kaplan; zaraza tradu jest, z wrzodu wyrosla.
\par 21 Ale jezliby ja obaczyl kaplan, ze w niej wlos nie bieleje, i nie jest glebsza nad insza skóre, ale tylko naczerniala, tedy zamknie kaplan takowego przez siedem dni.
\par 22 A jezliby sie szerzyla po skórze, za nieczystego osadzi go kaplan; zaraza to tradu.
\par 23 Wszakze jezliby blizna ona biala na swem miejscu zostawala; i nie szerzylaby sie, zapalenie wrzodu jest; przetoz za czystego osadzi go kaplan.
\par 24 Takze cialo, na którego skórze bylaby sparzelina od ognia, a po zgojeniu onej sparzeliny bylaby blizna biala, zaczerwieniala, albo biala tylko,
\par 25 Oglada to kaplan; a jezliby wlos w bliznie pobielal i lsnil sie, a na spojrzeniu bylaby glebsza ona blizna niz skóra, trad jest z sparzeliny wyrosly; przetoz za nieczystego osadzi go kaplan, bo zaraza tradu jest.
\par 26 A jezli kaplan obaczy, ze na onej bliznie bialej wlos nie pobielal, a iz nie jest glebsza nad insza skóre, ale iz nieco naczerniala, zamknie kaplan takowego przez siedem dni.
\par 27 I obejrzy go kaplan dnia siódmego; jezliby sie bardziej szerzyla po skórze, osadzi go za nieczystego kaplan: zaraza to tradu.
\par 28 A jezliz ta blizna biala zostawa na swem miejscu, a nie szerzy sie po skórze, ale sie zaczerniwa, przyszczela z sparzenia jest; i osadzi go za czystego kaplan, bo blizna sparzeliny jest.
\par 29 Gdyby maz, albo niewiasta mieli jaka plame na glowie albo na brodzie:
\par 30 Tedy obejrzy kaplan one plame; a bedzieli na spojrzeniu glebsza niz insza skóra, i bylby na niej wlos pozólkly i subtelny, tedy takowego za nieczystego osadzi kaplan, zmaza jest; trad na glowie albo na brodzie jest.
\par 31 Ale jezliby obaczyl kaplan zaraze onej plamy, a oto na wejrzeniu jest glebsza nad insza skóre, a nie bylby na niej wlos czarny, zamknie kaplan zaraze plamy majacego przez siedem dni.
\par 32 Potem obejrzy kaplan te zaraze dnia siódmego; a jezli sie nie szerzy zmaza, i nie masz na niej pozólklego wlosa, i na spojrzeniu ta zmaza nie bylaby glebsza nad insza skóre:
\par 33 Tedy ogolony bedzie ten czlowiek, ale zmazy onej golic nie bedzie; i zamknie kaplan majacego zmaze przez siedem dni po wtóre.
\par 34 I oglada kaplan one zmaze dnia siódmego; a jezli sie nie rozszerzyla zmaza po skórze, a na spojrzeniu nie jest glebsza nad insza skóre, osadzi go za czystego kaplan; a on uprawszy odzienie swoje, czystym bedzie.
\par 35 A jezliby sie poczela szerzyc ona zmaza na skórze po oczyszczeniu jego:
\par 36 Tedy obejrzy go kaplan; a jezliz sie szerzy ona zmaza po skórze, nie bedzie wiecej upatrowal kaplan wlosu zóltego; nieczystym jest.
\par 37 Wszakze jezli przed oczyma jego tak zostawa ona zmaza, i wlos czarny wyróslby na niej, zgoila sie ona zmaza, czysty jest i za czystego osadzi go kaplan.
\par 38 A gdyby tez na skórze ciala mezczyzny albo niewiasty byly blizny biale,
\par 39 Tedy obejrzy je kaplan; a jezliby sie blizny one biale na skórze ciala jego zaczerniwaly, blizna biala jest, wyrosla na skórze; czystym jest.
\par 40 Maz takze, któremu by opadly wlosy z glowy jego, lysy jest, i czysty jest.
\par 41 A jezlizby przeciwko jednej stronie twarzy opadly wlosy glowy jego, przelysialy jest, czysty jest.
\par 42 Wszakze jezliby na lysinie albo na tem przelysieniu, okazala sie blizna biala a sczerwienialaby, trad wyrósl z lysiny jego albo z przelysienia jego.
\par 43 I obejrzy go kaplan; a jezlizby sadzel zarazy jego byl bialy, albo sczerwienialy na lysinie jego, albo na oblysieniu jego, na ksztalt tradu na skórze ciala:
\par 44 Takowy czlowiek tredowaty jest, nieczysty jest, i osadzi go bezpiecznie kaplan za nieczystego; bo na glowie jego jest trad jego.
\par 45 Tredowaty zas, który by mial na sobie te zaraze, szaty jego beda rozdarte, i glowa jego bedzie odkryta, i usta sobie zakryje; a wolac bedzie: Nieczysty, nieczysty jestem.
\par 46 Po wszystkie dni, póki jest zaraza na nim, nieczystym bedzie, bo nieczystym jest, sam bedzie mieszkal; precz za obozem bedzie mieszkanie jego.
\par 47 Jezliby tez na szacie byla zaraza tradu, na szacie suknianej albo na szacie lnianej,
\par 48 Albo na osnowie, albo na watku ze lnu albo z welny, albo na skórze, albo na jakiejkolwiek rzeczy skórzanej;
\par 49 A bylaby ta zaraza zielona, albo czerwona na szacie, albo na skórze albo na osnowie, albo na watku, albo na jakiemkolwiek naczyniu skórzanem, zaraza tradu jest; bedzie ukazana kaplanowi.
\par 50 A ogladawszy kaplan zaraze one, zamknie one rzecz zarazona przez siedem dni.
\par 51 Potem obejrzy zaraze one dnia siódmego: jezliby sie szerzyla zmaza ona na szacie, albo na osnowie, albo na watku, albo na skórze, i na kazdej rzeczy z skóry urobionej, trad jest jadowity, zaraza nieczysta jest.
\par 52 Tedy spali te szate, albo osnowe, albo watek z welny, albo ze lnu, albo jakiekolwiek naczynie skórzane, na którem by byla zaraza: albowiem jest trad jadowity, ogniem spalono bedzie.
\par 53 Ale gdyby obaczyl kaplan, iz ona zmaza nie szerzy sie na szacie, albo na osnowie, albo na watku, albo na jakimkolwiek naczyniu skórzanem
\par 54 Rozkaze kaplan, aby uprano to, na czem jest zaraza, i zamknie to przez siedem dni po wtóre.
\par 55 I obejrzy kaplan po upraniu one zaraze; a jezli nie odmienila ona zaraza barwy swojej, chocby sie ona zaraza nie rozszerzyla, rzecz nieczysta jest, ogniem ja spalisz; zarazliwa rzecz jest, badz na zwierzchniej badz na spodniej stronie jej.
\par 56 Wszakze, jezliby kaplan obaczyl, iz przyczerniejsza bedzie zaraza po wypraniu swem, odedrze ja od szaty, albo od skóry, albo od osnowy, albo od watku.
\par 57 A jezliby sie jeszcze ukazala na szacie, albo na osnowie, albo na watku, albo na jakiem naczyniu skórzanem, trad jest szerzacy sie: ogniem to spalisz, na czem by byla takowa zaraza.
\par 58 Szate zas, albo osnowe, albo watek, albo kazde naczynie skórzane, które bys upral, a odeszlaby od niego zaraza, upierzesz je po wtóre, a czyste bedzie.
\par 59 Tac jest ustawa o zarazie tradu, na szacie suknianej, albo lnianej albo na osnowie, albo na watku, albo na jakiemkolwiek naczyniu skórzanem, jako to ma byc rozeznano, iz jest czyste albo nieczyste.

\chapter{14}

\par 1 Potem rzekl Pan do Mojzesza, mówiac:
\par 2 Tac jest ustawa okolo tredowatego w dzien oczyszczenia jego: przywiedziony bedzie do kaplana.
\par 3 A wynijdzie kaplan precz za obóz: a obaczyli kaplan, ze oto uleczona jest zaraza tradu, tradem zarazonego,
\par 4 Tedy rozkaze kaplan temu, który sie oczyszcza, aby wzial dwa wróble zywe i zdrowe, i drzewo cedrowe, i jedwabiu karmazynowego, i hizopu.
\par 5 I rozkaze kaplan zabic jednego wróbla nad naczyniem glinianem, nad woda zywa.
\par 6 Wróbla tedy zywego wezmie, i drzewo cedrowe, i jedwab karmazynowy i hizop, a omoczy to wszystko z wróblem zywym we krwi wróbla zabitego nad woda zywa.
\par 7 I pokropi tego, który sie oczyszcza od tradu, siedem kroc, i oglosi go byc czystym, a pusci wróbla zywego w pole.
\par 8 A ten, który sie oczyszcza, upierze szaty swoje, i ogoli wszystkie wlosy swoje, a umyje sie woda, i czystym bedzie. Potem wnijdzie do obozu, a bedzie mieszkal przed namiotem swoim przez siedem dni.
\par 9 Potem dnia siódmego ogoli wszystkie wlosy swe, glowe swa, i brode swa, i brwi nad oczyma swemi, i wszystkie inne wlosy swe ogoli; przytem upierze szaty swe, i cialo swe omyje woda, a tak oczyszczon bedzie.
\par 10 A dnia ósmego wezmie dwu baranków zupelnych, i owce jedne roczna, zupelna, i trzy dziesiate czesci efy maki pszennej na ofiare sniedna, zmieszana z oliwa, i miarke oliwy,
\par 11 Tedy kaplan, który oczyszcza czlowieka, który ma byc oczyszczony, postawi z temi rzeczami przed Panem, u drzwi namiotu zgromadzenia.
\par 12 Potem wezmie kaplan barana jednego, i bedzie go ofiarowal na ofiare za wystepek, z ona miarka oliwy, i bedzie to tam i sam obracal na ofiare obracania przed obliczem Panskiem.
\par 13 Zabije tez baranka onego na miejscu, gdzie bija ofiary za grzech i ofiare calopalenia, na miejscu swietem; bo jako ofiara za grzech tak ofiara za wystepek nalezy kaplanowi; rzecz najswietsza jest.
\par 14 I wezmie kaplan krwi z ofiary za wystepek, i pomaze kaplan koniec ucha prawego onemu, który sie oczyszcza; takze palec wielki u prawej reki jego i palec wielki u prawej nogi jego.
\par 15 Wezmie tez kaplan z onej miarki oliwy, a naleje na dlon swoje lewa;
\par 16 A omoczy palec swój prawy w oliwie, która jest na lewej dloni jego, i pokropi oliwa z palca swego siedem kroc przed obliczem Panskiem.
\par 17 A z ostatku oliwy, która jest na dloni jego, pomaze kaplan koniec ucha prawego onemu, który sie oczyszcza, i wielki palec prawej reki jego, takze wielki palec prawej nogi jego z onejze krwi, która jest ofiara za wystepek.
\par 18 A coby zostalo oliwy, która jest na dloni kaplanowej, pomaze tem glowe onego, który sie oczyszcza; i tak go oczysci kaplan przed obliczem Panskiem.
\par 19 Uczyni takze kaplan ofiare za grzech, i oczysci tego, który sie oczyszcza, od nieczystosci jego, a potem zabije ofiare calopalenia.
\par 20 I ofiarowac bedzie kaplan ofiare calopalenia, i ofiare sniedna na oltarzu; tak oczysci go kaplan, i czystym bedzie.
\par 21 A jezliby kto byl tak ubogi, izby tego nie przemógl, tedy wezmie baranka jednego na ofiare za wystepek na podnoszenie dla oczyszczenia swego, i jedne dziesiata czesc efy maki pszennej zagniecionej z oliwa na ofiare sniedna i miarke oliwy.
\par 22 Nad to dwie synogarlice, albo dwoje golabiat, czego dostac moze, z których jedno bedzie na ofiare za grzech, a drugie na ofiare calopalenia;
\par 23 I przyniesie je w ósmy dzien oczyszczenia swego do kaplana, do drzwi namiotu zgromadzenia, przed oblicznosc Panska.
\par 24 Wezmie tedy kaplan baranka ofiary za wystepek, i miarke oliwy; i bedzie to obracal tam i sam kaplan na ofiare obracania przed Panem.
\par 25 A zabije baranka na ofiare za wystepek: a wziawszy kaplan ze krwi ofiary za wystepek, pomaze koniec ucha prawego temu, który sie oczyszcza; i palec wielki prawej reki jego, i palec wielki prawej nogi jego.
\par 26 Oliwy takze naleje kaplan na lewa dlon swoje.
\par 27 I kropic bedzie kaplan palcem swoim prawym z oliwy, która jest na lewej rece jego siedem kroc przed obliczem Panskiem.
\par 28 Pomaze tez kaplan ona oliwa, która jest na dloni jego, koniec ucha prawego temu, który sie oczyszcza; takze wielki palec prawej reki jego, i wielki palec prawej nogi jego na miejscu krwi z ofiary za wystepek;
\par 29 A ostatkiem oliwy, która jest na dloni kaplana, pomaze glowe onego, który sie oczyszcza, aby go oczyscil przed Panem.
\par 30 Takze uczyni z jedna synogarlica, albo z jednem golebieciem, czegokolwiek z tych dostac moze.
\par 31 Czego dostac mógl, jedno z tych bedzie ofiara za grzech, a drugie na ofiare calopalenia z ofiara sniedna; a tak oczysci kaplan tego, który sie oczyszcza przed obliczem Panskiem.
\par 32 A tac jest ustawa o tym, na którym by byla zaraza tradu, który wszystkiego miec nie moze ku oczyszczeniu swemu.
\par 33 Rzekl potem Pan do Mojzesza i do Aarona, mówiac:
\par 34 Gdy wnijdziecie do ziemi Chananejskiej, która Ja wam dawam w osiadlosc, a dopuscilbym zaraze tradu na który dom osiadlosci waszej:
\par 35 Tedy on, którego dom jest, przyjdzie i opowie to kaplanowi, mówiac: Jakoby zaraza tradu zda mi sie byc w domu moim,
\par 36 Rozkaze tedy kaplan wyprzatnac dom pierwej niz sam wnijdzie, aby ogladal zaraze one, izby sie nic nie splugawilo, coby bylo w domu, a potem kaplan wnijdzie, aby ogladal on dom.
\par 37 A ogladajac one zaraze, ujrzeli zaraze na scianie domu, jakoby dolki czarne, przyzielenszym, albo przyczerwienszym, a na spojrzeniu byloby glebsze niz sciana.
\par 38 Tedy wynijdzie kaplan z domu onego przede drzwi, i zamknie on dom przez siedem dni.
\par 39 Wróci sie potem kaplan dnia siódmego i obejrzy; a jezli sie rozszerzyla zaraza na scianach domu onego,
\par 40 Rozkaze kaplan wylamac ono kamienie, na którem by byla zaraza, i wyrzucic je precz za miasto na miejsce nieczyste;
\par 41 A dom rozkaze wewnatrz oskrobac wszedy w okolo; i wyrzuca on proch który oskrobali, precz za miasto na miejsce nieczyste;
\par 42 I wezma kamienie insze i wprawia na miejsce innych kamieni; i wapna tez inszego wezma a potynkuja dom.
\par 43 A jezliby sie odnowila ona zaraza, i rozszerzyla sie po domu po wyrzuceniu kamienia, i po wyskrobaniu domu i po tynkowaniu jego:
\par 44 Tedy wnijdzie kaplan; a ujrzeli, ze sie rozszerzyla ona zaraza po domu, trad jest jadowity w domu onym, nieczysty jest.
\par 45 Zatem rozwala on dom, kamienie jego, i drzewo jego i wszystko wapno domu onego, a wyniosa precz za miasto na miejsce nieczyste.
\par 46 A ten kto by wszedl do domu onego, po wszystkie dni, póki byl zawarty, nieczystym bedzie az do wieczora.
\par 47 A kto by spal w onym domu, upierze szaty swoje; takze kto by jadl w tymze domu, upierze szaty swoje.
\par 48 Lecz jezliby wyszedlszy kaplan obaczyl, iz sie nie szerzy zaraza po domu po tynkowaniu jego, tedy osadzi kaplan, ze dom on jest czysty; bo uleczona jest zaraza ona.
\par 49 A wezmie na oczyszczenie onego domu dwu wróblów, i drzewo cedrowe, i jedwabiu, karmazynu, i hizopu;
\par 50 I zabije wróbla jednego nad naczyniem glinianem, nad woda zywa;
\par 51 A wziawszy drzewo cedrowe i hizop, i jedwab karmazynowy, i wróbla zywego, omoczy to wszystko we krwi wróbla zabitego i w wodzie zywej, a pokropi ten dom siedem kroc.
\par 52 I tak oczysci on dom krwia wróbla onego, i woda zywa i wróblem zywym i drzewem cedrowem, i hizopem, i jedwabiem czerwonym.
\par 53 Potem pusci wróbla zywego precz za miasto w pole; tak oczysci on dom, i czystym bedzie.
\par 54 Tac jest ustawa o kazdej zarazie tradu, i plamy czarnej:
\par 55 I o tradzie na szacie i na domu;
\par 56 I o sadzelu, i o swierzbie, i o bialej plamie,
\par 57 Zeby poznac, gdy kto jest nieczystym, i gdy kto czystym. Tac jest ustawa okolo tradu.

\chapter{15}

\par 1 Rzekl potem Pan do Mojzesza i do Aarona, mówiac:
\par 2 Powiedzcie synom Izraelskim, a mówcie do nich: Maz, który by cierpial plynienie nasienia z ciala swego, nieczysty jest.
\par 3 A tac bedzie nieczystosc plynienia jego: Jezli wypusci cialo jego plynienie swe, albo zeby sie to plynienie zastanowilo w ciele jego, nieczystosc jego jest.
\par 4 Kazda posciel, na której by lezal plynienie cierpiacy, nieczysta bedzie, i wszystko, na czem by usiadl, nieczyste bedzie.
\par 5 Kto by sie dotknal poscieli jego, upierze szaty swoje, i umyje sie woda, a bedzie nieczystym az do wieczora.
\par 6 Kto by tez siadl na tem, na czem ten siedzial, co z niego nasienie plynie, upierze szaty swe, i umyje sie woda, a bedzie nieczystym az do wieczora.
\par 7 Jezliby sie tez kto dotknal ciala meza cierpiacego plynienie, upierze szaty swe, i umyje sie woda, a bedzie nieczystym az do wieczora.
\par 8 A jezliby plunal plynienie cierpiacy na czystego, oplwany upierze szaty swe, i umyje sie woda, a bedzie nieczysty az do wieczora.
\par 9 Kazde tez siodlo, na którem by siedzial plynienie cierpiacy, nieczyste bedzie.
\par 10 Kto by sie tez jakiejkolwiek rzeczy dotknal, która byla po nim, nieczysty bedzie az do wieczora; a kto by co z tego nosil, upierze szaty swe, i umyje sie woda, a bedzie nieczystym az do wieczora.
\par 11 Takze kazdy, którego by sie dotknal cierpiacy plynienie, nie umywszy przedtem rak swoich w wodzie, upierze szaty swoje, i umyje sie woda, i bedzie nieczystym az do wieczora.
\par 12 Naczynie tez gliniane, którego by sie dotknal, co plynienie cierpi, stluczone bedzie, a kazde naczynie drzewiane woda umyte bedzie.
\par 13 A gdyby oczyszczony byl ten, który cierpi plynienie, od plynienia swego, naliczy sobie siedem dni podlug swego oczyszczenia, i upierze szaty swe, i omyje cialo swoje woda zywa, i bedzie czystym.
\par 14 Potem dnia ósmego wezmie sobie dwie synogarlice, albo dwoje golabiat, a przyszedlszy przed Pana do drzwi namiotu zgromadzenia, odda je kaplanowi.
\par 15 Tedy ofiarowac bedzie kaplan jedno z nich za grzech, a drugie na ofiare calopalenia. Tak oczysci go kaplan przed obliczem Panskiem od plynienia jego.
\par 16 A maz, z którego by wyszlo nasienie zlaczenia, o myje woda wszystko cialo swe, i bedzie nieczystym az do wieczora.
\par 17 Kazda tez szata, i kazda skóra, na której by bylo nasienie zlaczenia, wyprana bedzie woda, a bedzie nieczysta az do wieczora.
\par 18 Niewiasta takze, z która by obcowal maz cierpiacy plynienie nasienia, oboje umyja sie woda, a nieczystymi beda az do wieczora.
\par 19 Takze niewiasta, która by cierpiala chorobe swoje, a plynelaby krew z ciala jej, przez siedem dni bedzie w odlaczeniu swem; kazdy, coby sie jej dotknal, nieczysty bedzie az do wieczora.
\par 20 Na czembykolwiek lezala w odlaczeniu swem, nieczyste bedzie; takze na czem by siedziala, nieczyste bedzie.
\par 21 Kto by sie tez dotknal poscieli jej, upierze szaty swe, i umyje sie woda, a bedzie nieczysty az do wieczora.
\par 22 Takze kto by sie dotknal tego, na czem by siedziala, upierze szaty swe, i umyje sie woda, a bedzie nieczysty az do wieczora.
\par 23 Jezliby tez co bylo na lozu jej, albo na czem by ona siedziala, a dotknalby sie kto tego, nieczystym bedzie az do wieczora.
\par 24 A jezliby maz spal z nia, a zostalaby nieczystosc jej na nim, nieczysty bedzie przez siedem dni, i kazde loze, na którem by lezal, nieczyste bedzie.
\par 25 Jezliby tez niewiasta plynienie krwi cierpiala przez wiele dni, mimo czasu miesiaców jej, albo zeby krwi plynienie cierpiala w zwyczajnej chorobie, tedy po wszystkie dni plynienia nieczystosci swojej, jako i czasu choroby swojej, nieczysta bedzie.
\par 26 Kazda posciel jej, na której by lezala po wszystkie dni plynienia swego, jako posciel w przyrodzonej chorobie bedzie, i kazda rzecz, na której by siedziala, nieczysta bedzie wedlug nieczystosci przyrodzonej choroby jej.
\par 27 Kto by sie kolwiek dotknal tych rzeczy, nieczystym bedzie, i upierze szaty swe, i umyje sie woda, a bedzie nieczystym az do wieczora.
\par 28 A gdy bedzie oczyszczona od plynienia swego, naliczy sobie siedem dni, a potem oczyszczac sie bedzie.
\par 29 A dnia ósmego wezmie sobie dwie synogarlice, albo dwoje golabiat, i przyniesie je kaplanowi do drzwi namiotu zgromadzenia;
\par 30 Z których ofiarowac bedzie kaplan, jedno na ofiare za grzech, a drugie na ofiare calopalenia: tak ja oczysci kaplan przed Panem od plynienia nieczystosci jej.
\par 31 Tak odlaczac bedziecie syny Izraelskie od nieczystosci ich, aby nie pomarli w nieczystosci swej, gdyby splugawili przybytek mój, który jest w posrodku ich.
\par 32 Tac jest ustawa okolo tego, który plynienie cierpi, i z którego wychodzi nasienie zlaczenia, dla czego splugawiony bywa.
\par 33 Takze i okolo niewiasty, chorujacej w odlaczeniu swem, i kazdego cierpiacego plynienie swe, tak mezczyzny, jako i niewiasty, i meza, który lezal z nieczysta.

\chapter{16}

\par 1 Potem mówil Pan do Mojzesza po smierci dwu synów Aaronowych, którzy ofiarujac przed Panem, pomarli;
\par 2 I rzekl Pan do Mojzesza: Mów do Aarona, brata twego, niech nie wchodzi kazdego czasu do swiatnicy wewnatrz za zaslone przed ublagalnia, która jest na skrzyni, aby nie umarl, bo w obloku okazowac sie bede nad ublagalnia.
\par 3 Ale tak wchodzic bedzie Aaron do swiatnicy z cielcem na ofiare za grzech, a z baranem na ofiare calopalenia.
\par 4 W szate lniana poswiecona oblecze sie, a ubiory lniane beda na ciele jego; i pasem lnianym opasze sie, i czapke lniana wlozy na glowe; szaty swiete sa; i umyje woda cialo swe, a oblecze sie w nie.
\par 5 A od zgromadzenia synów Izraelskich wezmie dwu kozlów na ofiare za grzech, i jednego baranka na calopalenie.
\par 6 I bedzie ofiarowal Aaron cielca swego na ofiare za grzech, i uczyni oczyszczenie za sie, i za dom swój.
\par 7 Wezmie tez dwu kozlów, a postawi je przed Panem u drzwi namiotu zgromadzenia.
\par 8 I rzuci Aaron na oba kozly losy, los jeden Panu, a los drugi Azazelowi.
\par 9 I bedzie ofiarowal Aaron onego kozla, na którego padl los Panu, i ofiarowac go bedzie za grzech.
\par 10 Ale kozla, na którego padl los Azazela, postawi zywego przed Panem, aby oczyszczenie uczynil przezen, a wypuscil go do Azazela na puszcza.
\par 11 I bedzie ofiarowal Aaron cielca, na ofiare za grzech twój, a oczyszczenie uczyni za sie, i za dom swój, i zabije cielca na ofiare za grzech swój.
\par 12 Tedy wezmie pelna kadzielnice wegla rozpalonego z oltarza przed oblicznoscia Panska, i pelne garsci swe kadzenia wonnego utluczonego i wniesie za zaslone.
\par 13 A wlozy ono kadzenie na ogien przed Panem, aby okryl dym kadzenia ublagalnia, która jest nad swiadectwem, a nie umrze.
\par 14 Potem wziawszy ze krwi cielca onego, kropic bedzie palcem swym na ublagalni ku wschodowi slonca; takze przed ublagalnia kropic bedzie siedem kroc ta krwia palcem swym.
\par 15 Zabije tez kozla na ofiare za grzech ludu, a wniesie wewnatrz krew jego za zaslone; i uczyni ze krwia jego, jako uczynil ze krwia cielca, i kropic bedzie nia nad ublagalnia i przed ublagalnia.
\par 16 Tak oczysci swiatnice od nieczystot synów Izraelskich, i od przestepstw ich, i od wszystkich grzechów ich; toz tez uczyni namiotowi zgromadzenia, który jest miedzy nimi, w posrodku nieczystot ich.
\par 17 A zaden czlowiek niech nie bedzie w namiocie zgromadzenia, gdy on wchodzic bedzie ku oczyszczaniu do swiatnicy, az wynijdzie i wykona oczyszczenie sam za sie i za dom swój, i za wszystko zgromadzenie Izraelskie.
\par 18 I wynijdzie do oltarza, który jest przed Panem, a oczysci go; i wziawszy krwi cielcowej i krwi kozlowej, pomaze rogi oltarza w okolo;
\par 19 A pokropi go z wierzchu taz krwia palcem swym siedem kroc, a oczysci go, i poswieci go od nieczystot synów Izraelskich.
\par 20 Potem gdy odprawi oczyszczenie swiatnicy i namiotu zgromadzenia i oltarza, ofiarowac bedzie kozla zywego.
\par 21 A wlozywszy Aaron obie rece swe na glowe kozla zywego, wyznawac bedzie nad nim wszystkie nieprawosci synów Izraelskich, i wszystkie przestepstwa ich ze wszystkiemi grzechami ich, a wlozy je na glowe kozla onego, i wypusci go przez czlowieka na to obranego na puszcza.
\par 22 A tak poniesie on koziel na sobie wszystkie nieprawosci ich do ziemi pustej; i wypusci kozla onego na puszcza.
\par 23 Potem wróciwszy sie Aaron do namiotu zgromadzenia, zlozy z siebie szaty lniane, w które sie byl oblekl, wchodzac do swiatnicy, i zostawi je tam.
\par 24 Omyje tez cialo swoje woda, na miejscu swietem, i oblecze sie w szaty swe; a wyszedlszy sprawowac bedzie ofiare calopalenia swego, i ofiare calopalenia ludu, i uczyni oczyszczenie za sie, i za lud.
\par 25 A tlustosc ofiary za grzech spali na oltarzu.
\par 26 A ten, który zawiódl kozla do Azazela, upierze szaty swe; a omywszy cialo swoje woda, potem wnijdzie do obozu.
\par 27 Cielca zas ofiarowanego za grzech, i kozla za grzech, których krew wniesiona byla ku sprawowaniu oczyszczenia do swiatnicy, wyniosa precz za obóz, i spala ogniem skóry ich, i mieso ich, i gnój ich.
\par 28 A ten, co je palic bedzie, upierze szaty swoje, a omywszy cialo swoje woda, potem wnijdzie do obozu.
\par 29 To tez bedzie wam za ustawe wieczna: Miesiaca siódmego, dziesiatego dnia tegoz miesiaca, trapic bedziecie dusze wasze, i zadnej roboty nie bedziecie robic, tak w domu zrodzony, jako przychodzien, który gosciem jest miedzy wami;
\par 30 Bo w ten dzien oczysci was kaplan, abyscie oczyszczeni byli od wszystkich grzechów waszych przed Panem oczyszczeni bedziecie.
\par 31 Sabatem odpocznienia bedzie wam to, w który trapic bedziecie dusze wasze ustawa wieczna.
\par 32 A oczyszczac bedzie kaplan, który jest pomazany, a którego poswiecone sa rece ku sprawowaniu urzedu miasto ojca jego, a oblecze sie w szaty lniane, w szaty swiete;
\par 33 I oczysci swiatnice swietobliwosci, i namiot zgromadzenia; i oltarz oczysci, i kaplany, i wszystek lud zgromadzony oczysci.
\par 34 I bedzie to wam za ustawe wieczna ku oczyszczaniu synów Izraelskich od wszystkich grzechów ich raz w rok.
\par 35 I uczynil Mojzesz, jako mu byl rozkazal Pan.

\chapter{17}

\par 1 Tedy rzekl Pan do Mojzesza, mówiac:
\par 2 Mów do Aarona i do synów jego, i do wszystkich synów Izraelskich, a powiedz im: Tac jest rzecz, która przykazal Pan mówiac:
\par 3 Ktobykolwiek z domu Izraelskiego zabil wolu, albo owce, albo koze w obozie, albo kto by zabil za obozem,
\par 4 A do drzwi namiotu zgromadzenia nie przywiódlby tego, aby ofiarowal ofiare Panu, przed przybytkiem Panskim, krwi winien bedzie on maz, krew przelal; przetoz wytracony bedzie on maz z posrodku ludu swego.
\par 5 Synowie tedy Izraelscy przywioda ofiary swoje, które zabijali na polu; przywioda je Panu do drzwi namiotu zgromadzenia, do kaplana; a tak niechaj sprawuja ofiary spokojne Panu.
\par 6 I wyleje kaplan krew na oltarz Panski u drzwi namiotu zgromadzenia, a spali tlustosc ku wdziecznej wonnosci Panu.
\par 7 I nie beda ofiarowac wiecej ofiar swych dyjablom, z którymi cudzolozyli; ta ustawa wieczna bedzie im w narodziech ich.
\par 8 Nadto im jeszcze powiedz: Jezliby kto z domu Izraelskiego, albo z przychodniów miedzy wami mieszkajacych chcial ofiarowac ofiare calopalenia, albo insza ofiare,
\par 9 A do drzwi namiotu zgromadzenia nie przywiódlby jej, aby ja ofiarowal Panu, wytracony bedzie czlowiek ten z ludu swego.
\par 10 A ktobykolwiek z domu Izraelskiego, albo z przychodniów którzy by goscmi byli miedzy nimi, jadl krew jaka, postawie rozgniewana twarz swa przeciwko czlowiekowi krew jedzacemu, i wygladze go z posrodku ludu jego.
\par 11 Albowiem dusza wszelkiego ciala we krwi jego jest; a Ja dalem ja wam na oltarz ku oczyszczeniu dusz waszych; bo krew jest, która dusze oczyszcza.
\par 12 Dla tegoz powiedzialem synom Izraelskim: Zaden miedzy wami nie bedzie jadal krwi; ani przychodzien, który gosciem jest miedzy wami, nie bedzie jadal krwi.
\par 13 I ktobykolwiek z synów Izraelskich, albo z przychodniów, którzy sa goscmi miedzy wami, goniac ulowil jakie zwierze albo ptaka, co sie godzi jesc, tedy krew z niego wypusci, i zasypie ja piaskiem.
\par 14 Bo dusza kazdego ciala jest krew jego, która jest miasto duszy jego; przetozem powiedzial synom Izraelskim: Krwi wszelkiego ciala jesc nie bedziecie; bo dusza wszelkiego ciala jest krew jego; kto by ja kolwiek jadl, wytracony bedzie.
\par 15 Jezliby tez kto jadl co zdechlego, albo od zwierza rozszarpanego, tak w domu zrodzony, jako przychodzien, tedy upierze szaty swoje i omyje sie woda, a nieczystym bedzie az do wieczora; potem czysty bedzie.
\par 16 Ale jezliby nie upral szat swoich, a ciala swego nie omyl, poniesie nieprawosc swoje.

\chapter{18}

\par 1 Rzekl jeszcze Pan do Mojzesza, mówiac:
\par 2 Mów do synów Izraelskich, i rzecz im: Jam jest Pan, Bóg wasz.
\par 3 Wedlug obyczajów ziemi Egipskiej, w którejscie mieszkali, nie czyncie, ani wedlug obyczajów ziemi Chananejskiej, do której Ja was prowadze, nie czyncie, a w ustawach ich nie chodzcie.
\par 4 Sady moje czyncie, a ustaw moich strzezcie, abyscie chodzili w nich; Jam Pan, Bóg wasz.
\par 5 Przestrzegajciez tedy ustaw moich i sadów moich: które zachowywajac czlowiek, bedzie w nich zyl; Jam Pan.
\par 6 Zaden czlowiek do bliskiej pokrewnej swojej nie przystepuj, aby odkryl sromote jej; Jam Pan.
\par 7 Sromoty ojca twego, takze sromoty matki twojej nie odkryjesz; matka twoja jest, nie odkryjesz sromoty jej.
\par 8 Sromoty zony ojca twego nie odkryjesz; sromota ojca twego jest.
\par 9 Sromoty siostry twej, córki ojca twego, takze córki matki twojej, tak rodzonej, jako i przyrodniej, nie odkryjesz sromoty ich.
\par 10 Sromoty córki syna twego, takze sromoty córki córki twojej, nie odkryjesz; bo to sromota twoja.
\par 11 Sromoty córki zony ojca twego, która sie narodzila z ojca twego, siostra twoja jest, nie odkryjesz sromoty jej.
\par 12 Sromoty siostry ojca twego nie odkryjesz; bo jest pokrewna ojca twego.
\par 13 Sromoty siostry matki twojej nie odkryjesz; bo pokrewna matki twojej jest.
\par 14 Sromoty brata ojca twego nie odkryjesz, do zony jego nie wnijdziesz; zona stryja twego jest.
\par 15 Sromoty synowej twojej nie odkryjesz; zona jest syna twego, nie odkryjesz sromoty jej;
\par 16 Sromoty zony brata twego nie odkryjesz: sromota brata twego jest.
\par 17 Sromoty zony, i córki jej, nie odkryjesz; córki syna jej, i córki córki jej nie pojmiesz, abys odkryl sromote jej; bo pokrewne sa, i sprosna to rzecz jest.
\par 18 Siostry zony twej nie pojmuj, abys jej nie trapil, odkrywajac sromote jej, póki ona zywa.
\par 19 Do niewiasty, gdy jest w odlaczeniu nieczystosci, nie przystepuj, abys odkryl sromote jej.
\par 20 Z zona blizniego twego obcowac nie bedziesz, bo bys sie splugawil z nia,
\par 21 Z nasienia twego nie dopuszczaj ofiarowac Molochowi, abys nie splugawil imienia Boga twego; Jam Pan
\par 22 Z mezczyna nie bedziesz obcowal, jako z niewiasta; obrzydliwoscia to jest.
\par 23 Takze z bydleciem zadnem obcowac nie bedziesz, abys sie z niem mial splugawiac. Niewiasta tez niech nie podlega bydleciu dla obcowania z nim; sprosna rzecz jest.
\par 24 Nie plugawciez sie temi wszystkiemi rzeczami; bo tem wszystkiem splugawili sie poganie, które Ja wyrzucam przed obliczem waszem.
\par 25 Bo splugawila sie ziemia; przetoz nawiedze nieprawosc jej na niej, i wyrzuci ziemia obywatele swoje.
\par 26 A tak wy przestrzegajcie ustaw moich i sadów moich, a nie czyncie zadnych obrzydliwosci tych, w domu zrodzony, i przychodzien, który jest gosciem w posrodku was.
\par 27 Albowiem wszystkie te obrzydliwosci czynili ludzie tej ziemi, którzy byli przed wami, czem splugawiona jest ziemia.
\par 28 Aby was nie wyrzucila ziemia, gdybyscie ja splugawili, jako wyrzucila naród, który byl przed wami.
\par 29 Albowiem ktobykolwiek co uczynil z tych wszystkich obrzydliwosci, zaiste wytracone beda dusze to czyniace z posrodku ludu swego.
\par 30 Strzezciez tedy ustaw moich, nie czyniac ustaw obrzydliwych, które czyniono przed wami, ani sie plugawcie niemi; Jam Pan, Bóg wasz.

\chapter{19}

\par 1 Potem rzekl Pan do Mojzesza, mówiac:
\par 2 Mów do wszystkiego zgromadzenia synów Izraelskich, a powiedz im: Swietymi badzcie, bom Ja jest swiety, Pan, Bóg wasz.
\par 3 Kazdy matki swojej i ojca swego bójcie sie, a sabatów moich przestrzegajcie; Jam Pan, Bóg wasz.
\par 4 Nie udawajcie sie za balwany, a bogów litych nie czyncie sobie; Jam Pan, Bóg wasz.
\par 5 A gdy ofiarowac bedziecie ofiare spokojna Panu, tedy z dobrej woli swej ofiarowac ja bedziecie.
\par 6 W dzien, którego ofiarowac bedziecie, jedzcie ja i nazajutrz; ale coby zostalo az do trzeciego dnia, ogniem spalone bedzie.
\par 7 A jezlibyscie to jedli dnia trzeciego, obrzydle bedzie, i nie przyjemne.
\par 8 Ktobykolwiek to jadl, karanie za nieprawosc swoje poniesie, bo swietosc Panska splugawil; przetoz wytracona bedzie dusza ona z ludu swego.
\par 9 Gdy bedziecie zac zboza ziemi waszej, nie bedziesz do konca pola twego wyrzynal, ani pozostalych klosów zniwa twego zbierac bedziesz.
\par 10 Takze winnicy twojej gron do szczetu obierac nie bedziesz, ani jagód opadajacych z winnicy twej nie pozbierasz; ubogiemu i przychodniowi zostawisz je; Jam Pan, Bóg wasz.
\par 11 Nie kradnijcie, ani zapierajcie, i nie oszukiwajcie zaden blizniego swego.
\par 12 Nie przysiegajcie falszywie przez imie moje, i nie lzyj imienia Boga twego; Jam Pan.
\par 13 Nie uciskaj gwaltem blizniego twego, ani go odzieraj; nie zostanie zaplata najemnika u ciebie do jutra.
\par 14 Nie zlorzecz gluchemu, a przed slepym nie kladz zawady, ale sie bój Boga twego; Jam Pan.
\par 15 Nie czyn nieprawosci w sadzie. Nie ogladaj sie na osobe ubogiego, ani szanuj osoby bogatego; sprawiedliwie sadz blizniego twego.
\par 16 Nie bedziesz chodzil jako obmówca miedzy ludem twoim; nie bedziesz stal o krew blizniego twego; Jam Pan.
\par 17 Nie bedziesz nienawidzil brata twego w sercu twojem; jawnie strofowac bedziesz blizniego twego, a nie scierpisz przy nim grzechu.
\par 18 Nie mscij sie, i nie chowaj gniewu przeciw synom ludu twego; ale miluj blizniego twego, jako siebie samego; Jam Pan.
\par 19 Ustaw moich przestrzegajcie; bydlecia twego nie spuszczaj z bydlety rodzaju inszego; pola twego nie osiewaj z mieszanem nasieniem; takze szaty z róznych rzeczy utkanej, jako z welny i ze lnu, nie oblócz na sie.
\par 20 Jezliby maz spal z niewiasta, i obcowal z nia, a ona bedac niewolnica, bylaby mezowi poslubiona, a nie bylaby okupiona, ani wolnoscia darowana, oboje beda karani; ale nie na gardle, poniewaz nie byla wolno puszczona.
\par 21 I przywiedzie ofiare za wystepek swój Panu do drzwi namiotu zgromadzenia, barana za wystepek.
\par 22 Tedy oczysci go kaplan przez onego barana za wystepek przed Panem od grzechu jego, którym zgrzeszyl; a bedzie mu odpuszczony grzech jego, który popelnil.
\par 23 Gdy tez wnijdziecie do ziemi, a naszczepicie wszelakiego drzewa rodzacego owoc, tedy oberzniecie nieobrzezke jego, owoce jego; przez trzy lata miejcie je za nieobrzezanie, i jesc ich nie bedziecie.
\par 24 Ale roku czwartego wszystek owoc ich poswiecony bedzie na ofiare chwaly Panu.
\par 25 A piatego roku jesc bedziecie owoc jego, aby sie wam rozmnozyl urodzaj jego; Jam Pan, Bóg wasz.
\par 26 Nie jedzcie nic ze krwia. Nie bawcie sie wieszczbami, ani czarami.
\par 27 Nie strzyzcie w kolo wlosów glowy waszej, ani brody swojej oszpecajcie.
\par 28 Dla umarlego nie rzezcie ciala waszego, ani zadnego piatna na sobie nie czyncie; Jam Pan.
\par 29 Nie podasz na splugawienie córki twej, dopuszczajac jej wszeteczenstwa, aby sie ziemia nie splugawila, i nie byla napelniona ziemia sprosnoscia.
\par 30 Sabaty moje zachowywajcie, a swiatnice moje w uczciwosci miejcie; Jam Pan.
\par 31 Nie udawajcie sie do czarowników, ani do wieszczków, ani od nich rady szukajcie, abyscie sie od nich nie splugawili; Jam Pan, Bóg wasz.
\par 32 Przed czlowiekiem sedziwym powstan, a czcij osobe starego, i bój sie Boga swego; Jam Pan.
\par 33 Bedzieli mieszkal z toba przychodzien w ziemi waszej, nie czyncie mu krzywdy;
\par 34 Jako jeden z waszych w domu zrodzonych bedzie u was przychodzien, który jest u was gosciem, i milowac go bedziesz jako sam siebie; boscie i wy przychodniami byli w ziemi Egipskiej; Jam Pan, Bóg wasz.
\par 35 Nie czyncie nieprawosci w sadzie; w rozmierzaniu, w wadze, i w mierze.
\par 36 Szale sprawiedliwe, gwichty sprawiedliwe, korzec sprawiedliwy i kwarte sprawiedliwa miec bedziecie; Jam jest Pan, Bóg wasz, którym was wywiódl z ziemi Egipskiej.
\par 37 Przetoz strzezcie wszystkich ustaw moich, i wszystkich sadów moich, a czyncie je; Jam Pan.

\chapter{20}

\par 1 Potem rzekl Pan do Mojzesza, mówiac:
\par 2 Powiedz synom Izraelskim: Ktobykolwiek z synów Izraelskich, albo z przychodniów mieszkajacych w Izraelu ofiarowal potomstwo swoje Molochowi, smiercia niech umrze; lud ziemi niechaj go ukamionuje;
\par 3 Bo Ja postawie twarz moje rozgniewana przeciwko temu mezowi, i wytrace go z posrodku ludu jego, przeto, iz potomstwo swoje ofiarowal Molochowi, i splugawil swiatnice moje, a zmazal imie swiatobliwosci mojej.
\par 4 A jezliby lud ziemi nie dbajac przegladal mezowi takiemu, który by ofiarowal Molochowi potomstwo swe, i nie zabilby go:
\par 5 Tedy Ja postawie twarz moje zagniewana przeciw temu mezowi i przeciw domowi jego, i wytrace go i wszystkie, którzy cudzolozac, szliby za nim, aby cudzolozyli, nasladujac Molocha, z posrodku ludu jego.
\par 6 Czlowiek, który by sie udal do czarowników, i do wieszczków, aby cudzolozyl idac za nimi, postawie twarz swoje rozgniewana przeciwko niemu, i wytrace go z posrodku ludu jego.
\par 7 Przetoz poswiecajcie sie, a badzcie swietymi; bom Ja Pan, Bóg wasz.
\par 8 A strzezcie ustaw moich, i czyncie je; Jam Pan poswiecajacy was.
\par 9 Ktobykolwiek zlorzeczyl ojcu swemu, albo matce swej, smiercia umrze; ojcu swemu, i matce swej zlorzeczyl, krew jego bedzie na nim.
\par 10 Kto by sie kolwiek cudzolóstwa dopuscil z czyja zona, poniewaz cudzolozyl z zona blizniego swego, smiercia umrze cudzoloznik on i cudzoloznica.
\par 11 Ktobykolwiek tez spal z zona ojca swego, sromote ojca swego odkryl, smiercia umra oboje; krew ich bedzie na nich.
\par 12 Jezliby tez kto spal z synowa swoja, smiercia umra oboje; obrzydliwosci sie dopuscili, krew ich bedzie na nich.
\par 13 Maz takze, który by z mezczyzna obcowal sposobem niewiescim, obrzydliwosc uczynili oba; smiercia umra, krew ich bedzie na nich.
\par 14 Kto by tez pojal córke z matka jej, sprosna rzecz jest; ogniem spala onego i one, aby nie byla ta sprosnosc miedzy wami.
\par 15 Takze kto by sie zlaczyl z bydleciem, smiercia umrze, bydle tez zabijecie.
\par 16 Niewiasta, która by przystapila do jakiego bydlecia, aby z niem obcowala, zabijesz niewiaste i bydle; smiercia umra, krew ich bedzie na nich.
\par 17 Kto by tez pojal siostre swoje, córke ojca swego, albo córke matki swej, i widzialby sromote jej, i ona by widziala sromote jego, rzecz haniebna jest; przetoz wytraceni beda przed oczyma synów ludu swego; sromote siostry swej odkryl, nieprawosc swoje poniesie.
\par 18 Kto by spal z niewiasta czasu przyrodzonej choroby jej, i odkrylby sromote jej, i obnazylby plynienie jej, i ona by tez odkrywala plynienie krwi swojej, wytraceni beda oboje z posrodku ludu swego.
\par 19 Sromoty siostry matki twej i siostry ojca twego nie odkryjesz; bo kto by pokrewna swoje obnazyl, nieprawosc swoje poniesie.
\par 20 Kto by tez spal z zona stryja swego, sromote stryja swego odkryl, grzech swój poniosa, bez dzieci pomra.
\par 21 Takze kto by pojal zone brata swego, sprosnosc jest; sromote brata swego odkryl, bez dzieci beda.
\par 22 Strzezciez tedy wszystkich ustaw moich, i wszystkich sadów moich, a czyncie je, aby was nie wyrzucila ziemia, do której Ja was wprowadze, abyscie w niej mieszkali.
\par 23 A nie chodzcie w ustawach tego narodu, który Ja wypedzam od oblicza waszego; bo to wszystko czynili, i obrzydzilem je sobie.
\par 24 Wam zas powiedzialem: Wy posiadziecie ziemie ich, a Ja wam ja dam w dziedzictwo, ziemie oplywajaca mlekiem i miodem. Jam Pan, Bóg wasz, którym was wylaczyl od innych narodów.
\par 25 A tak wy rozeznawajcie miedzy bydleciem czystem i nieczystem, i miedzy ptakiem nieczystym i czystym, a nie plugawcie dusz waszych bydlem i ptastwem i wszystkiem, co sie czolga po ziemi, którem wam odlaczyl za nieczyste.
\par 26 I bedziecie mi swietymi, bom Ja swiety Pan, i odlaczylem was od innych narodów, abyscie byli moimi.
\par 27 Maz albo niewiasta, w których by byl duch czarnoksieski albo wieszczy, smiercia umra: kamieniem ukamionuja ich, krew ich bedzie na nich.

\chapter{21}

\par 1 Rzekl tez Pan do Mojzesza: Mów do kaplanów, synów Aaronowych, a powiedz im: Niech sie nad umarlym nie plugawi zaden kaplan w ludu swym;
\par 2 Tylko przy pokrewnym swoim, powinowatym swoim, przy matce swej, i przy ojcu swym, i przy synu swym, i przy córce swej, i przy bracie swym;
\par 3 Takze przy siostrze swej, pannie sobie najblizszej, która nie miala meza; przy tych splugawic sie moze.
\par 4 Nie splugawi sie przy przelozonym ludu swego, tak zeby sie zmazal.
\par 5 Nie beda sobie czynic lysiny na glowie swej, i brody swej nie maja golic, ani na ciele swem czynic beda rzezania.
\par 6 Swietymi beda Bogu swemu, i nie splugawia imienia Boga swego; albowiem ofiary ogniste Panskie, chleb Boga swego, ofiaruja; przetoz beda swietymi.
\par 7 Niewiasty wszetecznej, i w panienstwie naruszonej, pojmowac nie beda; takze niewiasty odrzuconej od meza jej, pojmowac nie beda; bo swiety jest kazdy z nich Bogu swemu.
\par 8 A tak bedziesz go mial za swietego, bo chleb Boga twego ofiaruje; przetoz swietym bedzie tobie, bom Ja swiety Pan, który poswiecam was.
\par 9 Jezliby sie córka kaplanska nierzadu dopuscila, ojca swego zelzyla, ogniem spalona bedzie.
\par 10 Najwyzszy tez kaplan miedzy bracia swa, na którego glowe wylany jest olejek pomazania, i który poswiecil rece swe, aby obloczyl szaty swiete, glowy swej nie obnazy i szat swoich nie rozedrze;
\par 11 I do zadnego z umarlych nie przystapi, a nawet i przy ojcu swym, i przy matce swej plugawic sie nie bedzie.
\par 12 Z swiatnicy tez nie wynijdzie, aby nie splugawil swiatnicy Boga swego, gdyz korona olejku pomazania Boga jego jest na nim: Jam Pan.
\par 13 Tenze panne w panienstwie jej pojmie.
\par 14 Wdowy, i odrzuconej i splugawionej nierzadnicy, zadnej z tych nie pojmie; ale panne z ludu swego za zone.
\par 15 A nie bedzie plugawil nasienia swego w ludu swym; bom Ja Pan, który go poswiecam.
\par 16 Przytem rzekl Pan do Mojzesza, mówiac:
\par 17 Powiedz Aaronowi, i rzecz: Ktobykolwiek z potomstwa swego w narodziech swych, mial na sobie wade, niechaj nie przystepuje, aby ofiarowal chleb Boga swego;
\par 18 Bo zaden maz, który by mial na sobie wade, przystepowac nie ma; maz slepy, albo chromy, albo niezupelnych albo zbytnich czlonków;
\par 19 Takze maz, który by mial zlamana noge, albo zlamana reke;
\par 20 Takze garbaty, i plynacych oczu, albo który ma bielmo na oku swem, albo krostawy, albo parszywy, albo wypukly:
\par 21 Wszelki maz, który by mial jaka wade, z potomstwa Aarona kaplana nie przystapi, aby ofiarowal ofiary ogniste Panu; wada na nim jest, nie przystapi, aby ofiarowal chleb Boga swego.
\par 22 Chleba jednak Boga swego z rzeczy najswietszych i poswieconych pozywac bedzie.
\par 23 Wszakze za zaslone nie wnijdzie, i do oltarza nie przystapi, bo wada na nim jest, aby nie splugawil swiatnicy mojej; bom Ja Pan, który ja poswiecam.
\par 24 To mówil Mojzesz do Aarona, i do synów jego, i do wszystkich synów Izraelskich.

\chapter{22}

\par 1 Zatem rzekl Pan do Mojzesza, mówiac:
\par 2 Powiedz Aaronowi i synom jego, aby sie wstrzymywali od rzeczy, które sa poswiecone od synów Izraelskich, a nie plugawili swietego imienia mojego w tem, co mi oni poswiecaja; Jam Pan.
\par 3 A tak rzecz do nich: W narodziech waszych, ktobykolwiek przystapil ze wszystkiego potomstwa waszego do poswieconych rzeczy, które by poswiecili synowie Izraelscy Panu, gdy nieczystosc jego na nim jest, wytracony bedzie ten od oblicznosci mojej; Jam Pan.
\par 4 Ktobykolwiek z nasienia Aaronowego byl tredowatym albo plynienie nasienia cierpiacym, rzeczy poswieconych jesc nie bedzie, póki by sie nie oczyscil; takze kto by sie dotknal jakiej nieczystosci ciala zmarlego, albo tego, z którego by plynelo nasienie zlaczenia.
\par 5 Takze kto by sie dotknal czego, co sie czolga po ziemi, przez coby sie nieczystym stal, albo czlowieka, przez którego by sie splugawil wedlug wszelakiej nieczystosci jego;
\par 6 Ten, kto by sie czego z tych rzeczy dotknal, nieczystym bedzie az do wieczora, i nie bedzie jadl rzeczy poswieconych, azby umyl cialo swoje woda.
\par 7 I az po zachodzie slonca czystym bedzie; a potem bedzie jesc z rzeczy poswieconych, bo to jest pokarm jego.
\par 8 Scierwu tez i rozszarpanego od zwierza jesc nie bedzie, aby sie tem nie splugawil; Jam Pan.
\par 9 A tak przestrzegac beda rozkazania mego, aby nie podlegli grzechowi, i nie pomarli w nim, gdyby sie splugawili; Jam Pan, który je poswiecam.
\par 10 Zaden obcy nie bedzie jadl z rzeczy poswieconych; komornik kaplanski, ani najemnik nie bedzie jadl rzeczy poswieconych.
\par 11 A jezliby kaplan czlowieka kupil za pieniadze swoje, ten jesc bedzie z rzeczy tych; takze zrodzony w domu jego, ci beda jadac z pokarmów jego.
\par 12 Lecz córka kaplanska, która by szla za meza obcego, ta z ofiar podnoszenia rzeczy swietych jesc nie bedzie.
\par 13 Gdyby zas córka kaplanska wdowa zostala, albo odrzucona byla od meza, i dziatek nie miala, a wrócilaby sie w dom ojca swego, tak jako w dziecinstwie swem, chleb ojca swego jesc bedzie; ale zaden obcy jesc z niego nie bedzie.
\par 14 A jezliby kto jadl z niewiadomosci rzeczy poswiecone, nadda piata czesc do tego, i odda kaplanowi rzecz poswiecona.
\par 15 Aby nie plugawili rzeczy poswieconych, które synowie Izraelscy ofiaruja Panu,
\par 16 I nie przywodzili na sie karania za wystepek, gdyby jedli poswiecone rzeczy ich; bom Ja Pan, który je poswiecam.
\par 17 Potem rzekl Pan do Mojzesza, mówiac:
\par 18 Powiedz Aaronowi i synom jego, i wszystkim synom Izraelskim, a mów do nich: Ktobykolwiek z domu Izraelskiego, albo z przychodniów w Izraelu ofiarowal ofiare swoje wedlug wszystkich slubów swoich, i wedlug wszystkich darów dobrowolnych swoich, które by ofiarowali Panu na ofiare calopalenia;
\par 19 Z dobrej woli swej ofiarowac bedzie zupelnego samca z bydla rogatego, z owiec, i z kóz.
\par 20 Coby mialo na sobie wade, ofiarowac nie bedzie; bo nie bedzie przyjemne od was.
\par 21 Jezliby kto ofiarowal ofiare spokojna Panu, pelniac slub, albo dobrowolny dar oddajac z rogatego bydla, albo z drobnego bydla, bez wady bedzie, aby przyjemne bylo; zadnej wady nie bedzie na nim.
\par 22 Slepego, albo ulomnego, albo na czem ochromionego, albo guzowatego, albo krostawego, albo parszywego, nie ofiarujcie Panu, ani na ofiare ognista dawajcie ich na oltarz Panu.
\par 23 Wolu tez albo owce zbytnich albo niezupelnych czlonków za dobrowolny dar ofiarowac je mozesz: ale slub z nich przyjemny nie bedzie.
\par 24 Zgniecionego i stluczonego i przerwanego, i rzezanego nie bedziecie ofiarowac Panu; w ziemi waszej nie uczynicie tego.
\par 25 Ani z reki cudzoziemca nie bedziecie ofiarowac chleba Bogu waszemu z tych wszystkich rzeczy, bo ulomek jest w nich; wade maja, nie beda przyjemne od was.
\par 26 Nad to rzekl Pan do Mojzesza, mówiac:
\par 27 Wól, albo owca, albo koza, gdy sie urodzi, niech bedzie siedem dni przy matce swojej, a dnia ósmego, i potem bedzie przyjemne ku palonej ofierze Panu.
\par 28 Krowy tez, ani owcy z plodem ich, nie zabijecie dnia jednego.
\par 29 A gdybyscie ofiarowali ofiare dziekczynienia Panu, z dobrej woli swej ofiarowac bedziecie.
\par 30 Onegoz dnia jedzona bedzie; nie zostawicie z niej nic az do jutra; Jam Pan.
\par 31 Przetoz strzezcie przykazan moich, a czyncie je; Jam Pan.
\par 32 I nie plugawcie imienia mego swietego, abym byl poswiecony w posrodku synów Izraelskich. Ja Pan, który was poswiecam;
\par 33 Którym was wywiódl z ziemi Egipskiej, abym wam byl za Boga; Ja Pan.

\chapter{23}

\par 1 Rzekl jeszcze Pan do Mojzesza, mówiac:
\par 2 Powiedz synom Izraelskim, a mów im: Swieta uroczyste Panskie, które nazywac bedziecie zgromadzenia swiete, te sa swieta uroczyste moje.
\par 3 Przez szesc dni robic bedziecie; ale w dzien siódmy sabat odpocznienia, zgromadzenie swiete, zadnej roboty czynic nie bedziecie; sabat Panski jest we wszystkich mieszkaniach waszych.
\par 4 A tec sa uroczyste swieta Panskie, zgromadzenia swiete, które obchodzic bedziecie pewnego ich czasu.
\par 5 Miesiaca pierwszego, dnia czternastego tegoz miesiaca, miedzy dwoma wieczorami swieto przejscia Panskiego.
\par 6 Potem dnia pietnastego tegoz miesiaca, swieto przasników bedzie Panu; przez siedem dni chleby przasne jesc bedziecie.
\par 7 A dnia pierwszego zgromadzenie swiete miec bedziecie; zadnej roboty sluzebniczej czynic nie bedziecie.
\par 8 Ale bedziecie ofiarowali ofiare ognista Panu przez siedem dni. Dnia takze siódmego zgromadzenie swiete bedzie; zadnej roboty sluzebniczej czynic nie bedziecie.
\par 9 I rzekl Pan do Mojzesza, mówiac:
\par 10 Powiedz synom Izraelskim, i rzecz im: Gdy wnijdziecie do ziemi która Ja wam dawam, a bedziecie zac zboze wasze, tedy przyniesiecie snop pierwiastek zniwa waszego do kaplana.
\par 11 I bedzie tam i sam obracal on snop przed obliczem Panskiem, aby byl przyjemny za was; nazajutrz po sabacie podnosic go bedzie kaplan.
\par 12 Zabijecie tez dnia, którego obracac bedziecie on snop, baranka zupelnego, rocznego na ofiare calopalenia Panu;
\par 13 Przy tem ofiare jego sniedna ze dwu dziesiatych czesci efy maki pszennej, zadzialanej z oliwa na palona ofiare Panu dla wdziecznej wonnosci; takze ofiare jego mokra, wina czwarta czesc hynu.
\par 14 A chleba i prazma, i nowego zboza jesc nie bedziecie az do dnia, którego przyniesiecie ofiare Bogu waszemu; ustawa to wieczna bedzie w narodziech waszych, we wszystkich mieszkaniach waszych.
\par 15 Naliczycie takze sobie od dnia pierwszego po sabacie, od dnia, któregoscie ofiarowali snop podnoszenia, siedem tygodni zupelnych niech bedzie.
\par 16 Az do pierwszego dnia po siódmym tygodniu naliczycie piecdziesiat dni; tedy ofiarowac bedziecie ofiare sniedna nowa Panu.
\par 17 Z domów waszych przyniesiecie chleby na obracanie tam i sam; dwa chleby, ze dwu dziesiatych czesci pszennej maki z kwasem upieczone beda; pierwiastki to Panu.
\par 18 A ofiarowac z tym chlebem bedziecie siedem baranków rocznych zupelnych, i cielca jednego, i dwu baranów; na ofiare calopalenia beda Panu z ofiara sniedna ich i z mokremi ofiarami ich; ofiara to ognista na wdzieczna wonnosc Panu.
\par 19 Zabijecie tez kozla jednego za grzech, i dwa baranki roczne na ofiare spokojna.
\par 20 I bedzie je obracal tam i sam kaplan z chlebem pierwiastek na ofiare sam i tam obracania przed obliczem Panskiem, i ze dwiema barankami; i beda swiete rzeczy Panu dla kaplana.
\par 21 I oglosicie w ten dzien swieto; zgromadzenie swiete miec bedziecie; zadnej roboty sluzebniczej czynic nie bedziecie; ustawa to bedzie wieczna we wszystkich mieszkaniach waszych, w narodziech waszych.
\par 22 A gdy zac bedziecie zboze ziemi waszej, nie bedziesz do konca pola twego dozynal, i klosów pozostalych zniwa twego zbierac nie bedziesz: ubogiemu, i przychodniowi zostawisz je; Jam Pan Bóg wasz.
\par 23 Zatem rzekl Pan do Mojzesza, mówiac:
\par 24 Powiedz synom Izraelskim, mówiac: Miesiaca siódmego, pierwszego dnia tegoz miesiaca, bedziecie mieli sabat, pamiatke trabienia, zgromadzenie swiete.
\par 25 Zadnej roboty sluzebniczej nie bedziecie czynili, lecz ofiarowac bedziecie ofiare ognista Panu.
\par 26 Rzekl jeszcze Pan do Mojzesza, mówiac:
\par 27 Lecz dziesiatego dnia tegoz miesiaca siódmego, dzien oczyszczania jest; zgromadzenie swiete miec bedziecie, a bedziecie trapic dusze wasze, ofiarujac ognista ofiare Panu.
\par 28 Zadnej roboty nie bedziecie czynili w ten dzien; bo dzien oczyszczania jest na oczyszczenie was przed obliczem Pana, Boga waszego.
\par 29 A wszelka dusza, która by sie nie trapila tego dnia, wytracona bedzie z ludu swego.
\par 30 Takze, ktobykolwiek czynil robote jaka w tenze dzien, wytrace czlowieka tego z posrodku ludu jego.
\par 31 Zadnej roboty nie czyncie; ustawa to bedzie wieczna w narodziech waszych.
\par 32 Sabat odpocznienia miec bedziecie, gdy trapic bedziecie dusze swe; dziewiatego dnia tegoz miesiaca, wieczór, od wieczora az do wieczora, obchodzic bedziecie sabat wasz.
\par 33 Rzekl zas Pan do Mojzesza, mówiac:
\par 34 Powiedz synom Izraelskim, i rzecz: Pietnastego dnia tegoz siódmego miesiaca bedzie swieto kuczek przez siedem dni Panu.
\par 35 Dnia pierwszego zgromadzenie swiete bedzie; zadnej roboty sluzebniczej czynic nie bedziecie.
\par 36 Przez siedem dni ofiarowac bedziecie ofiare ognista Panu; dnia ósmego zgromadzenie swiete miec bedziecie, a bedziecie ofiarowali ofiare ognista Panu; swieto jest, zadnej roboty sluzebniczej nie bedziecie czynili.
\par 37 Tec sa swieta uroczyste Panskie, które obchodzic bedziecie, zgromadzenia swiete, abyscie ofiarowali ofiare ognista Panu, calopalenie, i ofiare sniedna, i ofiare spokojna i ofiary mokre, kazda w dzien swój.
\par 38 Oprócz sabatów Panskich, i oprócz darów waszych, i oprócz wszystkich slubów waszych, i oprócz wszystkich dobrowolnych podarków waszych, które oddawac bedziecie Panu.
\par 39 Wszakze pietnastego dnia miesiaca siódmego, gdy zbierzecie urodzaj ziemi, bedziecie obchodzili swieto Panu przez siedem dni; dnia pierwszego odpocznienie, takze dnia ósmego odpocznienie bedzie.
\par 40 Tedy wezmiecie sobie pierwszego dnia owocu z drzewa co najpiekniejszego, i galazek palmowych, i galazek drzewa gestego, i wierzbiny od potoku, i weselic sie bedziecie przed Panem Bogiem waszym przez siedem dni.
\par 41 A obchodzic bedziecie to swieto Panu przez siedem dni na kazdy rok. Ustawa to wieczna w narodziech waszych; kazdego miesiaca siódmego obchodzic je bedziecie.
\par 42 W kuczkach mieszkac bedziecie przez siedem dni; kazdy zrodzony w Izraelu mieszkac bedzie w kuczkach,
\par 43 Aby wiedzieli potomkowie wasi, izem w namiotach kazal mieszkac synom Izraelskim, gdym je wywiódl z ziemi Egipskiej; Ja Pan, Bóg wasz.
\par 44 I opowiedzial Mojzesz swieta uroczyste Panskie synom Izraelskim.

\chapter{24}

\par 1 I rzekl Pan do Mojzesza, mówiac:
\par 2 Rozkaz synom Izraelskim, abyc przyniesli oliwy z drzewa oliwnego czystej, wytloczonej ku swieceniu, aby lampy gorzaly ustawicznie.
\par 3 Przed zaslona swiadectwa w namiocie zgromadzenia sporzadzi je Aaron, aby sie palily od wieczora az do poranku przed obliczem Panskiem ustawicznie; ustawa to wieczna w narodziech waszych.
\par 4 Na swieczniku czystym stawiac bedzie lampy przed obliczem Panskiem zawzdy.
\par 5 Wezmiesz tez maki pszennej, a upieczesz z niej dwanascie placków; ze dwu dziesiatych czesci efy bedzie placek jeden.
\par 6 Potem polozysz je dwiema rzedami, szesc w rzedzie jednym na stole czystym przed obliczem Panskiem.
\par 7 Wlozysz tez na kazdy rzad kadzidla czystego, aby bylo miasto chleba spalone, na pamiatke ku ofierze ognistej Panu.
\par 8 Na kazdy dzien sabatu klasc je bedzie kaplan porzadnie przed Panem zawzdy, biorac je od synów Izraelskich przymierzem wiecznem.
\par 9 I beda nalezaly Aaronowi i synom jego, którzy je jesc beda na miejscu swietem, albowiem rzecza im to najswietsza jest z ognistych ofiar Panskich, ustawa wieczna.
\par 10 Tedy wyszedl syn niewiasty Izraelskiej, którego miala z mezem Egipskim, miedzy syny Izraelskimi; i poswarzyli sie w obozie syn onej niewiasty Izraelskiej z mezem Izraelskim.
\par 11 I zlorzeczyl syn niewiasty Izraelskiej a imie Boze bluznil, dla czego przywiedziony byl do Mojzesza. A imie matki jego bylo Salomit, córka Dybrego z pokolenia Dan.
\par 12 I podali go do wiezienia, azby im oznajmiono, co z nim rozkaze Pan czynic.
\par 13 Tedy rzekl Pan do Mojzesza, mówiac:
\par 14 Wywiedz tego bluznierce precz za obóz, a niech wloza wszyscy, którzy to slyszeli, rece swe na glowe jego, i niech go ukamionuje wszystko zgromadzenie.
\par 15 A synom Izraelskim opowiedz, mówiac: Ktobykolwiek zlorzeczyl Bogu swemu, odniesie karanie za grzech swój.
\par 16 Takze kto by zbluznil imie Panskie, smiercia umrze, kamionujac ukamionuje go wszystko zgromadzenie; tak przychodzien jako w domu zrodzony, gdyby zbluznil imie Panskie, umrze.
\par 17 Takze jezliby kto zabil jakiegokolwiek czlowieka, smiercia umrze.
\par 18 A jezliby kto zabil bydle wróci inne bydle za bydle.
\par 19 Kto by tez oszkaradzil blizniego swego, wedlug tego, jako uczynil, niech mu sie stanie.
\par 20 Zlamanie za zlamanie, oko za oko, zab za zab; wedlug tego, jako oszkaradzil czlowieka, tak mu sie tez niech stanie.
\par 21 Kto by zabil bydle, wróci insze; ale kto by zabil czlowieka, umrze.
\par 22 Prawo jednakie miec bedziecie; tak przychodzien, jako i w domu zrodzony bedzie u was; bom ja Pan, Bóg wasz.
\par 23 To gdy opowiedzial Mojzesz synom Izraelskim, wywiedli onego bluznierce za obóz, i ukamionowali go.
\par 24 I uczynili synowie Izraelscy wedlug tego, jako przykazal Pan Mojzeszowi.

\chapter{25}

\par 1 Rzekl nad to Pan do Mojzesza na górze Synaj, mówiac:
\par 2 Powiedz synom Izraelskim, a mów do nich: Gdy wnijdziecie do ziemi, która Ja wam daje, tedy swiecic bedzie ziemia sabat Panu.
\par 3 Przez szesc lat osiewac bedziesz pole twoje, i przez szesc lat winnice twoje obrzynac bedziesz, zbierajac urodzaje z niej;
\par 4 Ale roku siódmego sabat odpocznienia miec bedzie ziemia, sabat Panski; pola twego nie bedziesz osiewal, ani winnicy twojej obrzynal.
\par 5 Co sie samo przez sie zrodzi zboza twego, nie bedziesz_zal, i jagód zaniechanej winnicy twojej nie bedziesz zbieral; rok odpocznienia bedzie miala ziemia.
\par 6 I bedzie, co sie urodzi w onem odpocznieniu ziemi, tobie na pokarm, i sludze twemu, i sluzebnicy twej, i najemnikowi twemu i przychodniowi twemu, który mieszka z toba.
\par 7 Takze bydlu twemu, i zwierzowi, który jest w ziemi twojej, bedzie wszystek urodzaj jej na pokarm.
\par 8 Naliczysz tez sobie siedem tygodni lat, to jest siedem kroc siedem lat; i uczyniac dni siedmiu tygodni lat czterdziesci i dziewiec lat.
\par 9 Tedy kazesz zatrabic w trabe huczna miesiaca siódmego, dnia dziesiatego tegoz miesiaca; w dzien oczyszczenia kazecie zatrabic po wszystkiej ziemi waszej.
\par 10 I swiecic bedziecie rok piecdziesiaty, a obwolacie wolnosc w ziemi wszystkim obywatelom jej. Lato milosciwe miec bedziecie; i wróci sie kazdy do osiadlosci swojej, i kazdy do rodziny swojej wróci sie.
\par 11 To milosciwe lato piecdziesiatego roku miewac bedziecie; nie bedziecie siac, i nie bedziecie zac tego, co sie samo przez sie zrodzi, ani zbierac bedziecie gron z winnic zaniechanych.
\par 12 Bo milosciwy rok jest, swiety wam bedzie; co sie na polu przedtem zrodzilo, to jesc bedziecie.
\par 13 W ten milosciwy rok wróci sie kazdy do osiadlosci swojej.
\par 14 Jezli co sprzedasz blizniemu twemu, albo co kupisz od blizniego twego, niech nie oszukiwa jeden drugiego.
\par 15 Wedlug liczby lat po milosciwym roku kupisz od blizniego twego; i wedlug liczby lat dochody sprzeda tobie.
\par 16 Jezli wiecej bedzie lat, tem drozej oszacujesz kupno ono; a jezli mniej bedzie lat, tedy tez taniej oszacujesz kupno ono, poniewaz tylko liczba dochodów sprzedawa sie tobie.
\par 17 A tak nie oszukiwajcie zaden blizniego swego, ale sie bój kazdy Boga swego; bom Ja Pan, Bóg wasz.
\par 18 Przestrzegajcie ustaw moich, i sady moje zachowywajcie, i czyncie je, abyscie mieszkac mogli w ziemi onej bezpiecznie.
\par 19 Tedy wyda ziemia owoc swój, a bedziecie jesc az do sytosci, i bedziecie mieszkac bezpiecznie w niej.
\par 20 A jezlibyscie rzekli: Cóz bedziemy jesc roku siódmego, jezli nie bedziem siac ani zbierac urodzajów naszych?
\par 21 Tedy rozkaze blogoslawienstwu memu przyjsc na was roku szóstego, i przyniesie urodzaj na trzy lata.
\par 22 I bedziecie siac roku ósmego, a bedziecie jesc urodzaj stary az do roku dziewiatego; póki nie nadejda pozytki jego, stare jesc bedziecie.
\par 23 Ziemia tedy nie bedzie sprzedawana na wiecznosc; bo moja jest ziemia, a wyscie goscmi i przychodniami u mnie.
\par 24 A po wszystkiej ziemi osiadlosci waszej pozwolicie wykupywac ziemie.
\par 25 Gdyby zubozal brat twój, a sprzedalby nieco z majetnosci swojej, i przyszedlby majacy prawo odkupienia, powinny jego niech wykupi, co sprzedal brat jego.
\par 26 A jezliby kto nie mial tego coby odkupic mógl, a sam by przemógl, i znalazl dostatek na to wykupno:
\par 27 Tedy obrachowawszy lata od sprzedania swego, wróci co zbywa temu, któremu sprzedal: a wróci sie do majetnosci swojej.
\par 28 A jezliby nie mial dostatku, aby wrócil, tedy zostanie majetnosc sprzedana w reku tego, który ja kupil, az do roku milosciwego, i ustapi mu jej w rok milosciwy, a on wróci sie do majetnosci swojej.
\par 29 Jezliby tez sprzedal dom mieszkania w miescie murowanem, bedzie mial wolnosc wykupic go, póki nie wynijdzie rok sprzedania jego, caly rok bedzie mial prawo do wykupienia jego.
\par 30 A jezli go nie wykupi, póki nie wynijdzie rok caly, tedy zostanie on dom w miescie murowanem temu, który go kupil, dziedzicznie, i potomkom jego
\par 31 I nie ustapi w milosciwe lato. Ale domy we wsiach, które nie sa murem obtoczone, te prawem jako pole ziemi szacowane beda; beda mogly byc odkupowane, i w milosciwe lato z rak obcych wynijda.
\par 32 Ale miasta Lewitów, i domy w dziedzicznych miesciech ich kazdego czasu wykupowane byc moga przez Lewity.
\par 33 Lecz temu co kupuje od Lewitów, wynijdzie kupno domu, i miejskiej osiadlosci jego, w rok milosciwy gdyz domy miast Lewickich sa dziedziczne ich, w posrodku synów Izraelskich.
\par 34 Ale pole na przedmiesciu ich nie bedzie sprzedawane; bo dziedzictwem ich jest wiecznem.
\par 35 Gdyby tez zubozal brat twój, a oslabialaby reka jego przy tobie, podeprzesz go; a jako i przychodzien niech sie zywi przy tobie.
\par 36 Nie bierz od niego lichwy, ani platu, ale sie bój Boga swego, aby sie zywil brat twój przy tobie.
\par 37 Pieniedzy twoich nie dawaj mu na lichwe, ani mu z zysku pozyczaj zywnosci twojej.
\par 38 Jam Pan, Bóg wasz, którym was wywiódl z ziemi Egipskiej, abym wam dal ziemie Chananejska, a byl wam za Boga.
\par 39 Jezliby tez zubozal brat twój przy tobie, tak zeby sie tobie zaprzedal, nie bedziesz go dreczyl sluzba niewolnicza;
\par 40 Jako najemnik, jako przychodzien bedzie u ciebie, az do roku milosciwego sluzyc ci bedzie.
\par 41 Potem wynijdzie od ciebie on, i dzieci jego z nim, a wróci sie do rodziny swojej, i do dziedzictwa przodków swych wróci sie.
\par 42 Sludzy bowiem moi sa, którem Ja wywiódl z ziemi Egipskiej, niechze nie beda sprzedawani jako niewolnicy.
\par 43 Nie bedziesz panowal nad nimi surowie, ale sie bedziesz bal Pana Boga twego.
\par 44 Niewolnik tez twój, i niewolnica twoja, które miec bedziesz, beda z narodów tych, które sa okolo was, z nich kupowac bedziecie niewolnika i niewolnice.
\par 45 Takze tez syny przychodniów mieszkajacych miedzy wami kupowac bedziecie, i z potomstwa tych, którzy sa z wami, które splodzili w ziemi waszej, a ci beda wam za dziedzictwo.
\par 46 Prawem dziedzicznem trzymac je bedziecie, i synowie wasi po was, abyscie je dziedzicznie odzierzeli, na wieki sluzby ich uzywac bedziecie; lecz nad bracia swa, syny Izraelskimi, zaden nad bratem swoim nie bedzie panowal surowie.
\par 47 Jezliby sie tez gosc albo przychodzien zbogacil, który mieszka z toba, a zubozalby brat twój przy nim, tak zeby sie zaprzedal gosciowi, albo przychodniowi, który jest z toba, albo potomstwu z domu cudzoziemców,
\par 48 Gdyby sie zaprzedal, moze byc wykupiony; ktokolwiek z braci jego odkupi go;
\par 49 Albo stryj jego, albo syn stryja jego odkupi go, albo z bliskich pokrewnych jego z rodziny jego, odkupi go, albo jezliby przemógl, wykupi sie sam.
\par 50 I porachuje sie z onym, co go kupil, od roku, którego mu sie sprzedal, az do milosciwego lata, aby pieniadze, za które sie sprzedal, odlozone byly wedlug liczby lat, jako z najemnikiem, z nim sobie postapi.
\par 51 Jezliby jeszcze nie malo lat zostawalo, wedle nich wróci okup swój z pieniedzy, za które kupiony jest.
\par 52 A jezliby nie wiele lat zostawalo do milosciwego lata, tedy porachuje sie z nim, a wedlug onych lat wróci okup swój.
\par 53 Jako najemnik doroczny niech bedzie u niego; nie bedzie nad nim surowie panowal przed oczyma twemi.
\par 54 A jezliby sie tym obyczajem nie wykupil, tedy wynijdzie w milosciwe lato, on i dzieci jego z nim;
\par 55 Albowiem synowie Izraelscy sa slugami moimi; slugami moimi sa, którem wywiódl z ziemi Egipskiej, Ja Pan, Bóg wasz.

\chapter{26}

\par 1 Nie czyncie sobie balwanów, ani obrazu rytego; ani slupów stawiajcie sobie, ani kamienia w obraz wyrytego stawiajcie w ziemi waszej, abyscie mu sie klaniali; bom Ja Pan, Bóg wasz.
\par 2 Sabaty moje zachowywajcie, a swiatnice moje w uczciwosci miejcie; Jam Pan.
\par 3 Jezli w ustawach moich chodzic bedziecie, i przykazania moje chowac i czynic bedziecie:
\par 4 Spuszcze wam deszcz czasu swego, i wyda ziemia urodzaj swój, i drzewa polne wydadza owoc swój;
\par 5 I trwac bedzie mlocba do zbierania wina, a zbieranie wina trwac bedzie do siewu; bedziecie jesc chleb swój do sytosci, i mieszkac bedziecie bezpiecznie w ziemi swej.
\par 6 Bo dam pokój w ziemi, i bedziecie spali, a nie bedzie, kto by was przestraszyl; wyplenie tez zlego zwierza z ziemi, a miecz nie przejdzie ziemi waszej.
\par 7 Owszem bedziecie gonic nieprzyjacioly wasze, i upadna przed wami od miecza.
\par 8 Piec waszych beda gonic sto, a sto waszych dziesiec tysiecy gonic beda, i polegna nieprzyjaciele wasi przed wami od miecza.
\par 9 Bo obróce sie do was, a rozkrzewie was, i rozmnoze was, i utwierdze przymierze moje z wami.
\par 10 I bedziecie jedli z dawna zachowale zboze, i stare, gdy nowe nastana, wyprzatniecie.
\par 11 I wystawie przybytek mój miedzy wami, a nie uprzykrzy was sobie dusza moja.
\par 12 I bede chodzil miedzy wami, a bede wam za Boga, a wy mnie bedziecie za lud.
\par 13 Jam Pan, Bóg wasz, którym was wywiódl z ziemi Egipskiej, abyscie im nie sluzyli; i polamalem lancuchy jarzma waszego, abyscie chodzili prosto.
\par 14 A jezlibyscie mie nie sluchali, i nie czynili wszystkich tych przykazan;
\par 15 I jezli ustawy moje wzgardzicie, a sadami moimi bedzie sie brzydzila dusza wasza, zebyscie nie czynili wszystkich przykazan moich, i wzruszylibyscie przymierze moje:
\par 16 Ja tez wam to uczynie: nawiedze was strachem, suchotami i goraczka, które wam oczy popsuja a bolescia napelnia dusze wasze, a siac bedziecie prózno nasienie wasze, bo je zjedza nieprzyjaciele wasi;
\par 17 I postawie twarz moje przeciwko wam, i porazeni bedziecie od nieprzyjaciól waszych, i panowac beda nad wami, którzy was maja w nienawisci; i bedziecie uciekali, choc was nikt gonic nie bedzie.
\par 18 A jezliz ani tak nie usluchacie mie, przydam siedem kroc wiecej karania dla grzechów waszych;
\par 19 I zetre pyche mocy waszej, i uczynie niebo nad wami jako zelazo, a ziemie wasze jako miedz;
\par 20 I wniwecz sie obróci praca wasza; bo nie wyda ziemia wasza uzytku swego, i drzewa ziemi nie wydadza owocu swego.
\par 21 A jezli chodzic bedziecie, mnie sie sprzeciwiajac, a nie zechcecie mie sluchac, przydam kazni waszych siedmiorako dla grzechów waszych.
\par 22 Bo puszcze na was zwierz polny, i osieroci was, i wyniszczy bydlo wasze, i upleni was, i spustoszeja drogi wasze.
\par 23 A jezliz tem sie nie nakarzecie, ale chodzic bedziecie, mnie sie sprzeciwiajac:
\par 24 Ja tez pójde wam sie sprzeciwiajac, i bic was bede siedmiorako dla grzechów waszych;
\par 25 I przywiode na was miecz, który sie sowicie zemsci zgwalcenia przymierza; a gdy sie zbiezycie do miast waszych, tedy puszcze powietrze morowe miedzy was, a bedziecie podani w rece nieprzyjacielskie.
\par 26 A gdy zlamie podpore chleba waszego, beda piekly dziesiec niewiast chleb wasz w piecu jednym, i beda wam oddawac chleb wasz pod waga, i bedziecie jesc, a nie najecie sie.
\par 27 A jezli i przeto nie usluchacie mie, ale chodzic bedziecie, mnie sie sprzeciwiajac:
\par 28 Ja tez pójde w gniewie przeciwko wam; i Ja tez karac was bede siedmiorako wiecej dla grzechów waszych.
\par 29 I bedziecie jesc cialo synów waszych, i cialo córek waszych jesc bedziecie.
\par 30 I wygubie po górach kaplice wasze, a porozwalam sloneczne balwany wasze; i sklade trupy wasze na kloce obrzydlych balwanów waszych, a bedzie sie wami brzydzila dusza moja.
\par 31 I podam miasta wasze na spustoszenie, a poburze swiatnice wasze, i nie przyjme wiecej wdziecznej wonnosci waszej.
\par 32 I spustosze ziemie, ze sie nad nia zdumieja nieprzyjaciele wasi, mieszkajac w niej.
\par 33 A was samych rozprosze miedzy narody, i dobede za wami miecza; a bedzie ziemia wasza pusta, i miasta wasze zburzone.
\par 34 Tedy rada bedzie ziemia odpocznieniu swemu po wszystkie dni spustoszenia swego; a wy bedziecie w ziemi nieprzyjaciól waszych, tedy odpocznie ziemia, i rada bedzie odpocznieniu swemu.
\par 35 Przez wszystkie dni spustoszenia swego odpoczywac bedzie; bo nie miala odpocznienia w sabaty wasze, gdyscie wy mieszkali w niej.
\par 36 A którzy z was pozostana, tedy przywiode strach na serca ich, w ziemiach nieprzyjaciól ich, ze je gonic bedzie chrzest liscia padajacego; i beda uciekali jako przed mieczem, i padac beda, chociaz ich nikt gonic nie bedzie.
\par 37 I padnie jeden na drugiego jako od miecza, choc ich nikt gonic nie bedzie; ani sie ostoicie przed nieprzyjacioly waszymi.
\par 38 I poginiecie miedzy narody, i pozre was ziemia nieprzyjaciól waszych.
\par 39 A którzy z was zostana, wywiedna dla nieprawosci swojej w ziemi nieprzyjaciól swoich; takze dla nieprawosci ojców swych z nimi wywiedna.
\par 40 Ale jezli wyznaja nieprawosc swoje, i nieprawosc ojców swych wedlug przestepstwa swego, którem wystapili przeciwko mnie, i wedlug którego chodzili, sprzeciwiajac mi sie;
\par 41 Zem tez i Ja chodzil sprzeciwiajac sie im, a izem je wprowadzil do ziemi nieprzyjaciól ich; jezli, mówie, na ten czas ponizy sie serce ich nieobrzezane, i cierpliwie znosic beda kazn za nieprawosci swoje:
\par 42 Tedy ja tez wspomne na przymierze moje z Jakóbem, i na przymierze moje z Izaakiem, i na przymierze moje z Abrahamem wspomne, i na te ziemie wspomne.
\par 43 A ziemia bedac od nich uwolniona, rada bedzie odpocznieniu swemu, gdy pusta bedzie dla nich; a oni beda cierpliwie nosic karanie za nieprawosc swa, przeto ze sady moje wzgardzili, i ustawami mojemi brzydzila sie dusza ich.
\par 44 Wszakze dla tego i na ten czas, gdy beda w ziemi nieprzyjaciól swoich, nie odrzuce ich, ani ich tak sobie obrzydze, zebym je wyniszczyc mial, i wzruszyc przymierze moje z nimi;
\par 45 Bom Ja Pan, Bóg ich. Ale wspomne na nie dla przymierza uczynionego z przodkami ich; którem wywiódl z ziemi Egipskiej, przed oczyma poganów, abym im byl za Boga, Ja Pan.
\par 46 Tec sa ustawy i sady i prawa, które postanowil Pan miedzy soba, i miedzy syny Izraelskimi na górze Synaj przez Mojzesza.

\chapter{27}

\par 1 Potem rzekl Pan do Mojzesza, mówiac:
\par 2 Mów do synów Izraelskich, a powiedz im: Gdyby czlowiek slubem poslubil dusze Panu, wedlug szacunku twego da okup.
\par 3 A bedzie tak szacunek twój: Za mezczyzne od dwudziestu lat az do szescdziesiat lat, bedzie szacunek twój piecdziesiat syklów srebra wedlug wagi swiatnicy.
\par 4 A jezli jest biala glowa szacunek twój bedzie trzydziesci syklów,
\par 5 A jezli od piatego roku az do dwudziestego roku, tedy bedzie szacunek twój za mezczyzne dwadziescia syklów a za biala glowe dziesiec syklów.
\par 6 A jezli za dziecie od jednego miesiaca az do pieciu lat, tedy bedzie szacunek twój za mezczyzne piec syklów srebra, a za dzieweczke szacunek twój trzy sykle srebra.
\par 7 A jezli od szescdziesiat lat i wyzej bedzieli mezczyzna tedy bedzie szacunek twój pietnascie syklów a za biala glowe dziesiec syklów.
\par 8 Lecz jezliby byl tak ubogi, zeby nie mógl oddac szacunku twego, tedy go stawia przed kaplana, i oszacuje go kaplan, wedlug przemozenia tego który slubowal, oszacuje go kaplan.
\par 9 Jezliby tez bydle z tych, które sie ofiaruja na ofiare Panu, poslubil, kazde, które odda Panu bedzie swiete.
\par 10 Nie odmieni go, ani da innego za nie, lepszego za gorsze, albo gorszego za lepsze; jezliby tez jakokolwiek odmienil bydle za bydle, tedy i ono, i to, które za nie dano bedzie swiete.
\par 11 A jezliby które nie czyste bydle poslubil z tych, co nie bywaja ofiarowane Panu, tedy stawi to bydle przed kaplana,
\par 12 I oszacuje kaplan badz dobre, badz zle, a jako je oszacuje kaplan, tak bedzie.
\par 13 A jezliby je kto odkupic chcial, przyda piata czesc nad szacunek twój.
\par 14 Jezliby tez kto poswiecil dom swój, zeby byl swiety Panu, tedy go oszacuje kaplan badz dobry, badz zly; jako go oszacuje kaplan, tak zostanie.
\par 15 A gdyby ten, który poswiecil, chcial odkupic dom swój, przyda piata czesc pieniedzy na szacunek twój, i bedzie jego.
\par 16 Jezli tez kto czesc roli z dziedzictwa swego poswiecil Panu tedy bedzie szacunek twój wedlug zasiewku jej; gdzie sie wysieje chomer jeczmienia, za piecdziesiat syklów srebra szacowane bedzie.
\par 17 Jezli do milosciwego lata poswiecil role swoje, wedlug szacunku twego zostanie.
\par 18 Ale jezlizby po milosciwem lecie poswiecil role twoje tedy kaplan obrachuje mu pieniadze wedlug lat zostawajacych do milosciwego lata i umniejszy mu sie z szacunku twego.
\par 19 A chcialliby odkupic rola, ten, który ja poswiecil, przyda piata czesc pieniedzy do szacunku twego i zostanie przy niej.
\par 20 Ale gdzie by nie odkupil roli onej, a sprzedana by byla rola komu inszemu, nie moze byc odkupiona.
\par 21 I bedzie ona rola, gdy wynijdzie milosciwe lato swieta Panu, jako rola poswiecona a przyjdzie w osiadlosc kaplanowi.
\par 22 A jezliby kto rola kupiona, która nie byla z ról dziedzictwa jego poslubil Panu.
\par 23 Tedy porachuje mu kaplan sume szacunku twego az do roku milosciwego, i da szacunek ten dnia onego za rzecz poswiecona Panu.
\par 24 A w milosciwe lato wróci sie rola od tego, od kogo ja kupiono, do tego, który dziedzicznie trzymal rola one.
\par 25 A kazdy szacunek twój bedzie wedle sykla swiatnicy, a dwadziescia pieniedzy sykiel wazy.
\par 26 Wszakze pierworodnego, a które prawem pierworodztwa bywa ofiarowane Panu z bydla, nikt go nie poswieci, badz wól badz owca, poniewaz Panskie sa.
\par 27 A jezliby z bydlat nieczystych bylo, odkupi je wedlug szacunku twego, i przyda piata czesc nad to; a jezliby go nie odkupiono, niechze sprzedane bedzie wedlug szacunku twego.
\par 28 Kazda jednak rzecz poslubiona, która by kto poslubil Panu ze wszystkiego, co ma z ludzi, i z bydla, i z ról osiadlosci swojej, nie bedzie sprzedawana, ani odkupowana; bo wszelka rzecz poslubiona najswietsza jest Panu.
\par 29 Wszelkie bydle poslubione, które sie pod slubem oddawa, od czlowieka nie bedzie odkupione, ale smiercia umrze.
\par 30 Wszystkie takze dziesieciny ziemi z nasienia ziemi, z owocu drzewa, Panskie sa; bo poswiecone sa Panu.
\par 31 Ale kto by chcial odkupic co z dziesiecin swoich, piata czesc ceny przyda do nich.
\par 32 Takze wszystkie dziesieciny z rogatego bydla, i z drobnego bydla, wszystkiego, co przechodzi pod laska pasterska, kazde dziesiate bedzie poswiecone Panu.
\par 33 Nie bedzie przebieral miedzy dobrem albo zlem, ani go odmieniac bedzie; a jezliby je jakokolwiek odmienil, bedzie to i ono odmienione poswiecone, nie ma byc odkupione.
\par 34 Tec sa przykazania, które rozkazal Pan Mojzeszowi do synów Izraelskich na górze Synaj.


\end{document}