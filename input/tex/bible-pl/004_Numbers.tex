\begin{document}

\title{Liczb}


\chapter{1}

\par 1 I mówil Pan do Mojzesza na puszczy Synaj, w namiocie zgromadzenia pierwszego dnia miesiaca wtórego, roku wtórego, po wyjsciu ich z ziemi Egipskiej, temi slowy:
\par 2 Obliczcie sume wszystkiego zgromadzenia synów Izraelskich wedlug narodów ich, i wedlug domów ojców ich wedlug imion ich, kazdego mezczyzne wedlug glów ich.
\par 3 Ode dwudziestu lat, i wyzej, wszystkich wychodzacych na wojne z Izraela; policzycie je wedlug hufców ich, ty i Aaron.
\par 4 I bedzie z wami z kazdego pokolenia jeden maz, któryby przedniejszy byl w domu ojców swoich.
\par 5 A tec sa imiona mezów, którzy z wami beda; z pokolenia Rubenowego Elizur, syn Sedeurów.
\par 6 Z pokolenia Symeonowego Salamijel, syn Surysaddajów.
\par 7 Z pokolenia Judowego Naason, syn Aminadabów.
\par 8 Z pokolenia Isascharowego Natanael syn Suharów.
\par 9 Z pokolenia Zabulonowego Elijab, syn Helonów.
\par 10 Z synów Józefowych, z pokolenia Efraimowego Elisama, syn Ammiudów; z pokolenia Manasesowego Gamalijel, syn Pedasurów.
\par 11 Z pokolenia Benjaminowego Abidan, syn Gedeonów.
\par 12 Z pokolenia Danowego Achyjezer, syn Ammisadajów.
\par 13 Z pokolenia Aserowego Pagijel, syn Ochranów.
\par 14 Z pokolenia Gadowego Elijazaf, syn Duelów.
\par 15 Z pokolenia Neftalimowego Achyra, syn Enanów.
\par 16 Ci zwolywani beda najzacniejsi z ludu ksiazeta w pokoleniach ojców swych; wodzami wojsk Izraelskich beda.
\par 17 Przyzwali tedy do siebie Mojzesz i Aaron mezów tych, którzy z imienia mianowani sa.
\par 18 I zebrali wszystko zgromadzenie dnia pierwszego miesiaca wtórego, i przyznawali sie do rodzajów swych wedlug familii swych, wedlug domów ojców swych i wedlug liczby imion, od dwudziestu lat i wyzej, wedlug osób swych.
\par 19 Jako rozkazal Pan Mojzeszowi, tak je policzyl na puszczy Synaj.
\par 20 I bylo synów Rubena, pierworodnego Izraelowego, rodzajów ich wedlug familii ich, wedlug domów ojców ich, wedlug liczby imion, wedlug osób ich, wszystkiego poglowia meskiego od dwudziestu lat i wyzej, wszystkich wychodzacych na wojne;
\par 21 Naliczono ich z pokolenia Rubenowego czterdziesci i szesc tysiecy i piec set.
\par 22 Z synów Symeonowych rodzajów ich, wedlug familii ich, wedlug domów ojców ich, naliczonych jego, wedlug liczby imion, wedlug osób ich, wszystkiego poglowia meskiego od dwudziestu lat i wyzej, wszystkich wychodzacych na wojne;
\par 23 Naliczono ich z pokolenia Symeonowego piecdziesiat i dziewiec tysiecy i trzy sta.
\par 24 Z synów Gadowych rodzajów ich wedlug familii ich, wedlug domów ojców ich, wedlug liczby imion, od dwudziestu lat i wyzej, wszystkich wychodzacych na wojne.
\par 25 Naliczono ich z pokolenia Gadowego czterdziesci i piec tysiecy i szesc set i piecdziesiat.
\par 26 Z synów Judowych rodzajów ich wedlug familii ich, wedlug domów ojców ich, wedlug liczby imion, od dwudziestu lat i wyzej, wszystkich wychodzacych na wojne;
\par 27 Naliczono ich z pokolenia Judowego siedemdziesiat i cztery tysiace i szesc set.
\par 28 Z pokolenia Isascharowego rodzajów ich wedlug familii ich, wedlug domów ojców ich, wedlug liczby imion, od dwudziestu lat i wyzej, wszystkich wychodzacych na wojne;
\par 29 Naliczono ich z pokolenia Isascharowego piecdziesiat i cztery tysiace i cztery sta.
\par 30 Z synów Zabulonowych rodzajów ich wedlug familii ich, wedlug domów ojców ich, wedlug liczby imion, od dwudziestu lat i wyzej, wszystkich wychodzacych na wojne;
\par 31 Naliczono ich z pokolenia Zabulonowego piecdziesiat i siedem tysiecy i cztery sta.
\par 32 Z synów Józefowych, a naprzód z synów Efraimowych, rodzajów ich wedlug familii ich, wedlug domów ojców ich, wedlug liczby imion, od dwudziestu lat i wyzej, wszystkich wychodzacych na wojne;
\par 33 Naliczono ich z pokolenia Efraimowego czterdziesci tysiecy i piec set.
\par 34 Z synów zas Manasesowych rodzajów ich wedlug familii ich, wedlug domów ojców ich, wedlug liczby imion, od dwudziestu lat i wyzej, wszystkich wychodzacych na wojne;
\par 35 Naliczono ich z pokolenia Manasesowego trzydziesci i dwa tysiace i dwiescie.
\par 36 Z synów Benjaminowych rodzajów ich, wedlug familii ich, wedlug domów ojców ich, wedlug liczby imion, od dwudziestu lat i wyzej, wszystkich wychodzacych na wojne;
\par 37 Naliczono ich z pokolenia Benjaminowego trzydziesci i piec tysiecy i cztery sta.
\par 38 Z synów Danowych rodzajów ich wedlug familii ich, wedlug domów ojców ich, wedlug liczby imion, od dwudziestu lat i wyzej, wszystkich wychodzacych na wojne;
\par 39 Naliczono ich z pokolenia Danowego szescdziesiat i dwa tysiace i siedemset.
\par 40 Z synów Aserowych rodzajów ich wedlug familii ich, wedlug domów ojców ich, wedlug liczby imion, od dwudziestu lat i wyzej, wszystkich wychodzacych na wojne;
\par 41 Naliczono ich z pokolenia Aserowego czterdziesci tysiecy i jeden i piec set.
\par 42 Z synów Neftalimowych rodzajów ich wedlug familii ich, wedlug domów ojców ich, wedlug liczby imion, od dwudziestu lat i wyzej, wszystkich wychodzacych na wojne;
\par 43 Naliczono ich z pokolenia Neftalimowego piecdziesiat i trzy tysiace i cztery sta.
\par 44 Cic sa policzeni, które policzyl Mojzesz i Aaron, i ksiazeta Izraelskie, dwanascie mezów, którzy byli wybrani po jednemu z domów ojców swych.
\par 45 I bylo wszystkich policzonych synów Izraelskich wedlug domów ojców ich, od dwudziestu lat i wyzej, wszystkich wychodzacych na wojne z Izraela;
\par 46 Bylo wszystkich policzonych szesc kroc sto tysiecy i trzy tysiace, i piec set i piecdziesiat.
\par 47 Ale Lewitowie wedlug pokolenia ojców swych nie byli policzeni miedzy nimi.
\par 48 Bo rozkazal byl Pan Mojzeszowi, mówiac:
\par 49 Tylko pokolenia Lewi nie bedziesz liczyl, a sumy ich nie policzysz miedzy syny Izraelskie;
\par 50 Ale postanowisz Lewity nad przybytkiem swiadectwa, i nad wszystkiem naczyniem jego, i nad wszystkiem co nalezy do niego. Oni nosic beda przybytek i wszystkie naczynia jego; oni tez sluzyc beda w nim, a okolo przybytku obozem sie klasc beda.
\par 51 A gdy sie bedzie ruszal przybytek, skladac go beda Lewitowie; takze gdy stanowic sie bedzie przybytek, stawiac go beda Lewitowie; a kto by obcy do niego przystapil, umrze.
\par 52 I beda stawac obozem synowie Izraelscy, kazdy wedlug pulków swoich, i kazdy pod choragwia swoja, w wojsku swem.
\par 53 Ale Lewitowie klasc sie beda obozem okolo przybytku swiadectwa, aby nie przyszedl gniew mój na zgromadzenie synów Izraelskich: i beda Lewitowie trzymac straz u przybytku swiadectwa.
\par 54 Uczynili tedy synowie Izraelscy wedlug wszystkiego, co byl rozkazal Pan Mojzeszowi, tak uczynili.

\chapter{2}

\par 1 Zatem rzekl Pan do Mojzesza i do Aarona, mówiac:
\par 2 Kazdy z synów Izraelskich klasc sie beda obozem pod choragwia swoja wedlug znaków domów ojców swych; naprzeciwko okolo namiotu zgromadzenia klasc sie beda.
\par 3 A ci sie obozem poloza na wschód slonca: Choragiew wojska Judowego wedlug hufców swych, a hetmanem nad syny Judowymi Naason, syn Aminadabów;
\par 4 A w wojsku jego policzonych siedemdziesiat i cztery tysiace i szesc set.
\par 5 Podle niego polozy sie obozem pokolenie Isascharowe, a hetmanem nad syny Isascharowymi Natanael, syn Suharów;
\par 6 A w wojsku jego policzonych piecdziesiat i cztery tysiace i cztery sta.
\par 7 Podle nich pokolenie Zabulonowe, a hetmanem nad syny Zabulonowymi Elijab, syn Helonów.
\par 8 A w wojsku jego policzonych piecdziesiat i siedem tysiecy i cztery sta.
\par 9 Wszystkich policzonych w obozie Judowym sto tysiecy, osiemdziesiat tysiecy, i szesc tysiecy i cztery sta wedlug hufców ich; ci naprzód pociagna.
\par 10 Choragiew obozu Rubenowego polozy sie na poludnie wedlug hufców swych, a hetmanem nad syny Rubenowymi Elisur, syn Sedeurów;
\par 11 A w wojsku jego policzonych czterdziesci i szesc tysiecy i piec set.
\par 12 Podle niego polozy sie obozem pokolenie Symeonowe, a hetmanem nad syny Symeonowymi Selumijel, syn Surysaddajów;
\par 13 A w wojsku jego policzonych piecdziesiat i dziewiec tysiecy i trzy sta.
\par 14 Potem pokolenie Gadowe, a hetmanem nad syny Gadowymi Elijazaf, syn Rehuelów;
\par 15 A w wojsku jego policzonych czterdziesci i piec tysiecy, i szesc set i piecdziesiat.
\par 16 Wszystkich policzonych w obozie Rubenowym sto tysiecy, piecdziesiat tysiecy i jeden, cztery sta i piecdziesiat wedlug hufców ich; a ci w rzedzie wtórym pociagna.
\par 17 Potem pójdzie namiot zgromadzenia z wojskiem Lewitów, w posrodku wojska; jakim porzadkiem stawac beda obozem, takim pociagna kazdy w szyku swym pod choragwia swoja.
\par 18 Choragiew obozu Efraimowego wedlug hufców swych ku zachodowi, a hetmanem nad syny Efraimowymi Elisama, syn Ammiudów;
\par 19 A w wojsku jego policzonych czterdziesci tysiecy i piec set.
\par 20 A podle niego pokolenie Manasesowe, a hetmanem nad syny Manasesowymi Gamalijel, syn Pedasurów.
\par 21 A w wojsku jego policzonych trzydziesci i dwa tysiace i dwiescie.
\par 22 Podle nich pokolenie Benjaminowe, a hetmanem nad syny Benjaminowymi Abidan, syn Giedeonów;
\par 23 A w wojsku jego policzonych trzydziesci i piec tysiecy i cztery sta.
\par 24 Wszystkich policzonych w obozie Efraimowym sto tysiecy i osiem tysiecy i sto wedlug hufców swoich; a ci w trzecim rzedzie pociagna.
\par 25 Choragiew obozu Danowego polozy sie ku pólnocy wedlug hufców swych, a hetmanem nad syny Danowymi Achiezer, syn Ammisadajów;
\par 26 A w wojsku jego policzonych szescdziesiat i dwa tysiace i siedem set.
\par 27 A podle niego polozy sie obozem pokolenie Aserowe a hetmanem nad syny Aserowymi Pagijel, syn Ochranów;
\par 28 A w wojsku jego policzonych czterdziesci tysiecy i jeden i piec set.
\par 29 Potem pokolenie Neftalimowe, a hetmanem nad syny Neftalimowymi Ahira, syn Enanów;
\par 30 A w jego wojsku policzonych piecdziesiat i trzy tysiace i cztery sta.
\par 31 A tak wszystkich policzonych obozu Danowego sto tysiecy, piecdziesiat i siedem tysiecy i szesc set; a ci na ostatku pociagna pod choragwia swoja.
\par 32 Cic sa policzeni synów Izraelskich wedlug domów ojców ich, wszystkich policzonych w obozie wedlug hufców ich szesc kroc sto tysiecy, i trzy tysiace i piec set i piecdziesiat.
\par 33 Ale Lewitów nie liczono miedzy syny Izraelskie, jako byl Pan rozkazal Mojzeszowi.
\par 34 I uczynili synowie Izraelscy wedlug wszystkiego; jako rozkazal Pan Mojzeszowi, tak sie stanowili obozem przy choragwiach swych, i ciagneli kazdy wedlug familii swych, i wedlug domów ojców swych.

\chapter{3}

\par 1 A tec sa rodzaje Aarona i Mojzesza w dzien, którego mówil Pan z Mojzeszem na górze Synaj.
\par 2 A te sa imiona synów Aaronowych: Pierworodny Nadab, potem Abiu, Eleazar, i Itamar.
\par 3 Te sa imiona synów Aaronowych, kaplanów pomazanych, których rece poswiecone byly ku sprawowaniu urzedu kaplanskiego.
\par 4 Ale pomarli Nadab i Abiu przed Panem, gdy ofiarowali ogien obcy przed Panem na puszczy Synaj, a zeszli bez potomstwa; dla tego Eleazar i Itamar odprawowali urzad kaplanski przed obliczem Aarona, ojca swego.
\par 5 Tedy rzekl Pan do Mojzesza, mówiac:
\par 6 Kaz przystapic pokoleniu Lewiego, a postaw je przed Aaronem kaplanem, aby mu sluzyli,
\par 7 A trzymali straz jego, i straz wszystkiego zgromadzenia, przed namiotem zgromadzenia, wykonywajac usluge przybytku.
\par 8 Takze aby strzegli wszystkiego naczynia namiotu zgromadzenia, i trzymali straz synów Izraelskich, a odprawowali usluge przybytku.
\par 9 Oddasz tedy Lewity Aaronowi, i synom jego; bo wlasnie oddani sa jemu z synów Izraelskich.
\par 10 Aarona zas i syny jego przelozysz, aby przestrzegali kaplanstwa swego: bo przystapilliby kto obcy, umrze.
\par 11 Zatem rzekl Pan do Mojzesza, mówiac:
\par 12 A oto, Ja wzialem Lewity z posrodku synów Izraelskich miasto wszelkiego pierworodnego, otwierajacego zywot, miedzy synami Izraelskimi, i beda moi Lewitowie.
\par 13 Albowiem mnie przynalezy wszelkie pierworodne; ode dnia, któregom pobil wszelkie pierworodne w ziemi Egipskiej, poswiecilem sobie kazde pierworodne w Izraelu; od czlowieka az do bydlecia moi beda; Jam Pan.
\par 14 Rzekl tez Pan do Mojzesza na puszczy Synaj, mówiac:
\par 15 Policz syny Lewiego, wedlug domów ojców ich, wedlug familii ich; kazdego mezczyzne urodzonego od miesiaca i wyzej, policzysz je.
\par 16 I policzyl je Mojzesz wedlug mowy Panskiej, jako mu bylo rozkazane.
\par 17 I byly synów Lewiego te imiona: Gerson, i Kaat, i Merary.
\par 18 Te zas imiona synów Gersonowych wedlug domów ich: Lobni i Semei.
\par 19 A synowie Kaatowi wedlug domów swych: Amram, i Izaar, Hebron, i Husyjel.
\par 20 Synowie zas Merarego wedlug domów swych: Naheli i Muzy; te sa familije Lewiego wedlug domów ojców ich.
\par 21 Z Gersona familija Lobnicka, i familija Semeicka; tec sa familije Gersonowe.
\par 22 Policzonych ich wedlug liczby kazdego mezczyzny urodzonego od miesiaca i wyzej, bylo policzonych siedem tysiecy i piec set.
\par 23 Te familije Gersonowe za przybytkiem klac sie beda ku zachodowi.
\par 24 A ksiazeciem domu ojca Gersonitów: Elijazaf, i syn Laelów.
\par 25 A pod straza synów Gersonowych bedzie przy namiocie przybytek zgromadzenia, przybytek i namiot, przykrycie jego, i zaslona u drzwi namiotu zgromadzenia;
\par 26 I opony sieni, i zaslona we drzwiach u sieni, która jest przed przybytkiem i przy oltarzu w okolo, i sznury jego, do wszelkiej potrzeby jego.
\par 27 Z Kaata zas poszla familija Amramitów, i familija Izaaritów, i familija Husyjelitów. Tec byly domy Kaatytów.
\par 28 W liczbie wszystkich mezczyzn urodzonych od miesiaca i wyzej, osiem tysiecy i szesc set, trzymajacych straz przy swiatnicy.
\par 29 Te familije synów Kaatowych klasc sie beda obozem po bok przybytku ku poludniowi;
\par 30 A ksiazeciem domu ojca familii Kaatytów Elisafan, syn Husyjelów.
\par 31 A bedzie pod straza ich skrzynia, i stól, i swiecznik, i oltarze, i naczynia swiatnicy, któremi uslugowac beda, i zaslona, i ze wszystkiemi potrzebami jej.
\par 32 A ksiazeciem nad ksiazety Lewitów bedzie Eleazar, syn Aarona kaplana, postanowiony nad tymi, którzy trzymaja straz przy swiatnicy.
\par 33 Od Merarego zas poszla familija Mahelitów, i familija Muzytów; a tec sa domy Merarytów.
\par 34 A policzonych ich, wedlug liczby kazdego mezczyzny urodzonego od miesiaca i wyzej, szesc tysiecy i dwiescie.
\par 35 Ksiazeciem zas domu ojca familii Merarego Suryjel, syn Abihailów; a ci klasc sie beda po bok przybytku, ku pólnocy.
\par 36 A nalezec beda do strazy synów Merarego deski przybytku, i dragi jego, slupy jego, i podstawki jego, i wszystkie naczynia jego, i wszystkie potrzeby jego;
\par 37 Takze slupy sienne w okolo, i podstawki ich, i kotly i sznury ich.
\par 38 A klasc sie beda obozem przed przybytkiem, po przedniej stronie namiotu zgromadzenia, na wschód Mojzesz, i Aaron, i synowie jego, trzymajacy straz przy swiatnicy; straz za syny Izraelskie; a obcy gdyby przystapil, umrze.
\par 39 A tak wszystkich policzonych Lewitów od Mojzesza i Aarona, na rozkazanie Panskie, wedlug domów ich, wszystkich mezczyzn urodzonych od miesiaca i wyzej, bylo dwadziescia tysiecy i dwa.
\par 40 Tedy rzekl Pan do Mojzesza: Policz wszystkie pierworodne mezczyzny miedzy syny Izraelskimi od miesiaca i wyzej, a uczyn summe imion ich.
\par 41 A wezmiesz mi Lewity (Ja Pan) miasto wszystkich pierworodnych w syniech Izraelskich, takze bydla Lewitów ze wszystkie pierworodne bydla synów Izraelskich.
\par 42 Policzyl tedy Mojzesz, jako mu Pan rozkazal, wszystkie pierworodne w syniech Izraelskich.
\par 43 A bylo wszystkich pierworodnych mezczyzn wedlug liczby imion, urodzonych od miesiaca i wyzej, policzonych ich dwadziescia i dwa tysiace, dwiescie, siedemdziesiat i trzy.
\par 44 I rzekl Pan do Mojzesza, mówiac:
\par 45 Wezmij Lewity miasto wszystkich pierworodnych z synów Izraelskich, bydla takze Lewitów miasto bydla ich, i beda moimi Lewitowie; Jam Pan.
\par 46 A za okup onych dwu set, siedmiudziesiat i trzech, którzy zbywaja nad liczbe Lewitów, z pierworodnych synów Izraelskich,
\par 47 Wezmiesz po piec syklów na kazda glowe; wedlug sykla swiatnicy brac bedziesz; dwadziescia pieniedzy wazy sykiel.
\par 48 I oddasz te pieniadze Aaronowi i synom jego za okup onych, którzy zbywaja nad liczbe ich.
\par 49 Wzial tedy Mojzesz pieniadze okupu od tych, którzy zbywali nad te, które okupili soba Lewitowie.
\par 50 Od pierworodnych synów Izraelskich wzial pieniedzy onych tysiac, trzysta, szescdziesiat i piec syklów wedlug sykla swiatnicy;
\par 51 I oddal te pieniadze okupu Mojzesz Aaronowi i synom jego wedlug slowa Panskiego, jako Pan rozkazal Mojzeszowi.

\chapter{4}

\par 1 Nad to rzekl Pan do Mojzesza i Aarona, mówiac:
\par 2 Zbierz summe synów Kaatowych z posród synów Lewiego wedlug familii ich, i wedlug domów ojców ich.
\par 3 Od tego, który ma trzydziesci lat i wyzej, i az do tego, co ma piecdziesiat lat, którzy bedac sposobnymi do tej pracy, mogliby odprawowac posluge w namiocie zgromadzenia.
\par 4 Tac bedzie powinnosc synów Kaatowych przy namiocie zgromadzenia, przy miejscu najswietszem;
\par 5 I przyjdzie Aaron z synami swymi, gdy sie bedzie mial ruszyc obóz, a zdejma opone zaslony, i okryja nia skrzynie swiadectwa;
\par 6 A wloza na nie przykrycie z borsukowych skór, i przykryja z wierzchu wszystko opona hijacyntowa, i zaloza drazki jej.
\par 7 Takze stól chlebów pokladnych przykryja opona hijacyntowa, a poloza na nim misy, i przystawki, i kubki, i czasze do nalewania; a chleb ustawicznie na nim bedzie.
\par 8 I rozciagna na tem opone szarlatowa, a przykryja to przykryciem skór borsukowych, i zaloza drazki do niego.
\par 9 Wezma tez opone hijacyntowa, która nakryja swiecznik do swiecenia z lampami jego, i nozyczki jego, i kaganki jego, i wszystkie naczynia do oliwy jego, których uzywaja przy nim:
\par 10 I uwina go ze wszystkiem naczyniem jego w przykrycie z skór borsukowych, i wloza na drazki.
\par 11 Na oltarz takze zloty rozpostrza opone hijacyntowa, a wloza nan przykrycie z skór borsukowych, i zaloza drazki jego.
\par 12 Pobiora tez wszystkie naczynia uslugi, któremi sluza w swiatnicy, a obwinawszy opona hijacyntowa, przykryja je przykryciem z skór borsukowych, i wloza na drazki.
\par 13 Do tego zmiota popiól z oltarza, a na nim rozpostrza opone szarlatowa;
\par 14 I wloza nan wszystkie naczynia jego, któremi usluguja przy nim, to jest lopaty, widly, i miotly, i kocielki, i wszystkie naczynia oltarzowe, i rozpostrza na nim przykrycie z skór borsukowych, i zaloza drazki jego.
\par 15 A gdy to wykona Aaron z synami swymi, ze przykryje swiatnice ze wszystkiem naczyniem swiatnicy, a bedzie sie mial ruszyc obóz, tedy potem przyjda synowie Kaatowi, aby one rzeczy niesli; ale sie nie beda dotykali swiatnicy, aby nie pomarli. Tac jest posluga synów Kaatowych, przy namiocie zgromadzenia.
\par 16 Staranie zasie Eleazara, syna Aarona kaplana, bedzie o oliwie do swiecenia, o kadzeniu wonnem, o ofierze sniednej ustawicznej, i o olejku pomazywania, dogladanie przybytku, i wszystkiego, co w nim jest, i swiatnicy z naczyniami jej.
\par 17 Potem rzekl Pan do Mojzesza i Aarona, mówiac:
\par 18 Nie zatracajcie pokolenia domów Kaatowych z posrodku Lewitów.
\par 19 Ale to im uczynicie, aby zyli a nie pomarli, gdy przystepowac beda do miejsca najswietszego: Aaron i synowie jego przyjda, i postanowia kazdego z nich nad praca jego i nad brzemieniem jego.
\par 20 Ale niech nie wchodza patrzyc, gdy beda uwijane rzeczy swiete, aby nie pomarli.
\par 21 Zatem rzekl Pan do Mojzesza, mówiac:
\par 22 Zbierz summe synów Gersonowych wedlug domów ojców ich, i wedlug familii ich;
\par 23 Od tego, który ma trzydziesci lat i wyzej, az do tego, który ma piecdziesiat lat, policzysz je, którzy sposobni beda do tej pracy, aby mogli uslugowac przy namiocie zgromadzenia.
\par 24 A tac bedzie powinnosc domów synów Gersonowych ku posludze i ku noszeniu.
\par 25 Nosic beda opony przybytku, i namiot zgromadzenia z przykryciem jego; takze przykrycie borsukowe, które z wierzchu na nim jest, i zaslone od drzwi namiotu zgromadzenia;
\par 26 I opony do sieni, i zaslone drzwi bramy u sieni, która jest u przybytku, i przy oltarzu w okolo, i sznury jej, i wszystkie naczynia uslugi ich, i wszystko, czego uzywaja okolo uslugi ich; i to czynic beda.
\par 27 Wedlug rozkazania Aarona i synów jego bedzie wszelka usluga synów Gersonowych przy kazdem brzemieniu ich, i przy kazdej usludze ich; a poruczycie im pod straz wszystkie brzemiona ich.
\par 28 Tac bedzie powinnosc domów synów Gersonowych w namiocie zgromadzenia, a bedzie ich dogladal Itamar, syn Aarona kaplana.
\par 29 Syny takze Merarego wedlug familii ich, i wedlug domów ojców ich policzysz:
\par 30 Od tego, który ma trzydziesci lat i wyzej, i az do tego, który ma piecdziesiat lat, policzysz je; którzy bedac sposobni do tej pracy mogliby uslugowac przy namiocie zgromadzenia.
\par 31 A ta bedzie powinnosc pracy ich we wszystkiej usludze ich w namiocie zgromadzenia: deski przybytku, i dragi jego, i slupy jego, i podstawki jego nosic;
\par 32 Przytem slupy sieni w okolo, i podstawki ich z kolkami ich, i sznury ich ze wszystkiem naczyniem ich, do wszelkiej sluzby ich; a mianowicie policzycie naczynia, które im poruczycie pod straz ich.
\par 33 Tac powinnosc bedzie familii synów Merarego, wedlug wszelkiej sluzby ich, przy namiocie zgromadzenia pod dozorem Itamara, syna Aarona kaplana.
\par 34 Obliczyli tedy Mojzesz i Aaron, i ksiazeta zgromadzenia syny Kaatowe wedlug familii ich, i wedlug domów ojców ich.
\par 35 Od tych, którym bylo trzydziesci lat i wyzej, i az do tych, którym bylo piecdziesiat lat, którzy sposobni bedac ku tej pracy mogliby uslugowac przy namiocie zgromadzenia.
\par 36 A bylo ich policzonych wedlug familii ich dwa tysiace, siedem set i piecdziesiat.
\par 37 Cic byli policzeni z familii Kaatytów wszyscy sluzacy przy namiocie zgromadzenia, które zliczyl Mojzesz i Aaron wedlug rozkazania Panskiego przez Mojzesza.
\par 38 Takze policzeni sa synowie Gersonowi wedle familii swych, i wedlug domów ojców swych,
\par 39 Od tego, który mial trzydziesci lat i wyzej, i az do tego, który mial piecdziesiat lat, którzy sposobni bedac ku pracy mogli uslugowac przy namiocie zgromadzenia.
\par 40 A bylo ich policzonych wedlug familii ich, i domów ojców ich dwa tysiace, szesc set i trzydziesci.
\par 41 Cic byli policzeni z familii synów Gersonowych, wszyscy sluzacy w namiocie zgromadzenia, które zliczyl Mojzesz i Aaron wedlug slowa Panskiego.
\par 42 Takze policzeni z familii synów Merarego wedlug familii swych i domów ojców swych,
\par 43 Od tego, który mial trzydziesci lat i wyzej, i az do tego, który mial piecdziesiat lat; którzy sposobni bedac ku pracy mogli uslugiwac przy namiocie zgromadzenia;
\par 44 A bylo ich policzonych wedlug familii ich trzy tysiace i dwiescie.
\par 45 A tac byla summa policzonych z familii synów Merarego, które zliczyl Mojzesz i Aaron wedlug rozkazania Panskiego przez Mojzesza.
\par 46 Wszystkich policzonych, które policzyl Mojzesz i Aaron, i ksiazeta Izraelskie z Lewitów wedlug familii ich, i domów ojców ich,
\par 47 Od tego, który mial trzydziesci lat i wyzej, i az do tego, który mial piecdziesiat lat, kazdego przychodzacego do odprawowania powinnosci uslugi, i powinnosci noszenia brzemion w namiocie zgromadzenia.
\par 48 A bylo ich policzonych osiem tysiecy, i piec set i osiemdziesiat.
\par 49 Wedlug rozkazania Panskiego policzeni sa przez Mojzesza, kazdy z osobna wedlug uslugi jego, i wedlug brzemienia jego; a policzeni byli ci, które Pan rozkazal liczyc Mojzeszowi.

\chapter{5}

\par 1 I rzekl Pan do Mojzesza, mówiac:
\par 2 Rozkaz synom Izraelskim, aby wyrzucili z obozu kazdego tredowatego, i kazdego, który cierpi plynienie nasienia, i kazdego, który sie splugawil nad umarlym;
\par 3 Tak mezczyzne jako i niewiaste wyrzucicie; precz za obóz wyrzucicie je, aby nie splugawili obozu tych, miedzy którymi Ja mieszkam.
\par 4 I uczynili tak synowie Izraelscy, a wygnali je precz za obóz; jako rozkazal Pan Mojzeszowi, tak uczynili synowie Izraelscy.
\par 5 Nad to rzekl Pan do Mojzesza, mówiac:
\par 6 Powiedz synom Izraelskim: Maz albo niewiasta, gdyby popelnili jakikolwiek grzech ludzki, dopusciwszy sie wystepku przeciwko Panu, a bylaby winna ona dusza:
\par 7 Tedy wyznaja grzech swój, którego sie dopuscili, i wróca to, w czem by winni byli cale; a przydawszy jeszcze nad to piata czesc, oddadza onemu, przeciw któremu zgrzeszyli.
\par 8 A jezliby nie bylo tego, komu by szkode trzeba nagrodzic, ona szkoda oddana bedzie Panu, i zostanie kaplanowi oprócz barana oczyszczenia, przez którego ma byc oczyszczony.
\par 9 Kazda tez ofiara podnoszenia ze wszech rzeczy poswieconych od synów Izraelskich, która przyniosa do kaplana, jemu sie dostanie.
\par 10 Owe rzeczy poswiecone od kogozkolwiek, jego beda; i kto by co oddal kaplanowi, jemu zostanie.
\par 11 Zatem rzekl Pan do Mojzesza, mówiac:
\par 12 Mów do synów Izraelskich, a powiedz im: Kazdy maz, którego by zona wystapila, i dopuscilaby sie grzechu przeciwko niemu;
\par 13 A zlaczylby sie inszy z nia zlaczeniem nasienia, a byloby to skryte przed oczyma meza jej, i tailaby sie, bedac splugawiona, a swiadka by nie bylo przeciwko niej, aniby jej zastano;
\par 14 Jednak przypadlby nan duch zapalczywosci, i mialby w podejrzeniu zone swa, która by splugawiona byla; albo zeby przypadl nan duch zapalczywosci, i mialby w podejrzeniu zone swa, która by splugawiona nie byla:
\par 15 Tedy przywiedzie on maz zone swoje do kaplana, i przyniesie z nia ofiare jej, dziesiata czesc efy maki jeczmiennej nie lejac na nia oliwy, ani kladac na nia kadzidla; albowiem jest ofiara podejrzenia, ofiara sniedna, pamietna, przywodzaca na pamiec grzech.
\par 16 A tak bedzie ja ofiarowal kaplan, i stawi ja przed oblicznoscia Panska.
\par 17 I wezmie kaplan wody swietej w naczynie gliniane, i prochu, który bedzie na tle przybytku, wezmie kaplan, a wsypie do wody.
\par 18 Potem postawi kaplan niewiaste przed Panem, i odkryje glowe niewiasty, a da w rece jej ofiare sniedna pamietna; ofiara to sniedna podejrzenia; a kaplan bedzie mial w rece wode gorzka przeklestwa.
\par 19 I poprzysieze ja kaplan, i rzecze do niewiasty: Jezli nie spal kto inszy z toba, a jezlis sie nie uniosla w grzech nieczysty przy mezu swym, badz nienaruszona od tej wody gorzkiej przeklestwa;
\par 20 Ale jezlizes ustapila od meza twego, i jestes splugawiona, a kto inny spal z toba oprócz meza twego:
\par 21 Tedy poprzysieze kaplan niewiaste one przysiega przeklestwa, i rzecze do niej: Niechaj cie poda Pan na zlorzeczenie, i na przeklinanie miedzy ludem twoim, przepusciwszy, aby lono twoje wypadlo, i zywot twój opuchl:
\par 22 Niechze przenikna te wody przeklete wnetrznosci twoje, aby opuchl zywot twój, i wypadlo lono twoje; i odpowie niewiasta: Amen. Amen.
\par 23 Tedy napisze te przeklestwa kaplan na ksiegach, a omyje je ona woda gorzka;
\par 24 I da sie napic niewiescie wody gorzkiej przeklestwa, i przenikna ja wody przeklestwa, i obróca sie w gorzkosc.
\par 25 Potem wezmie kaplan z rak niewiasty onej ofiare sniedna podejrzenia, a bedzie ja podnosil przed Panem, ofiarujac ja na oltarzu;
\par 26 Wezmie tez kaplan na garsc pamietnego z ofiary sniednej, i spali to na oltarzu, potem da wypic wode niewiescie.
\par 27 A gdy sie jej da napic onej wody, stanie sie, jezliby splugawiona byla, i wystapila grzechem przeciwko mezowi swemu, ze ja przenikna wody przeklestwa, i obróca sie w gorzkosc, i opuchnie zywot jej, i wypadnie lono jej, i stanie sie niewiasta ona przeklestwem miedzy ludem swoim.
\par 28 A jezliby nie byla splugawiona niewiasta, aleby czysta byla, niewinna zostanie, i dziatki rodzic bedzie.
\par 29 Tac jest ustawa podejrzenia, gdyby ustapila zona od meza swego, i bylaby splugawiona;
\par 30 Albo zeby na meza przypadl duch zapalczywy, a mialby w podejrzeniu zone swoje, i postawilby ja przed Panem, a uczynilby z nia kaplan wszystko wedlug tej ustawy;
\par 31 Tedy nie bedzie maz on winien grzechu; ale niewiasta ona poniesie nieprawosc swoje.

\chapter{6}

\par 1 I rzekl Pan do Mojzesza, mówiac:
\par 2 Powiedz synom Izraelskim, a mów do nich: Maz albo niewiasta, gdy sie odlaczy, czyniac slub Nazarejstwa, aby byli odlaczeni Panu,
\par 3 Od wina i mocnego napoju wstrzymywac sie bedzie; octu z wina, i octu z mocnego napoju pic nie bedzie, i wszystkiego, co sie z jagód wytlacza, nie bedzie pil; takze jagód winnych, swiezych ani suchych, jesc nie bedzie.
\par 4 Po wszystkie dni Nazarejstwa swego ze wszystkiego, co wyrasta z macicy winnej, od ziarnka az do lupiny, jesc nie bedzie.
\par 5 Po wszystkie dni slubu Nazarejstwa swego brzytwa nie postoi na glowie jego, az wynijdzie czas, do którego sie poswiecil Panu; bedzie swietym, a zapusci wlos na glowie swojej.
\par 6 Po wszystkie dni, których sie odlaczy Panu, do umarlego nie wnijdzie.
\par 7 Nad ojcem swym, i nad matka swa, nad bratem swym, i nad siostra swa, nie splugawi sie, gdyby zmarli; albowiem poswiecenie Boga swego ma na glowie swojej.
\par 8 Po wszystkie dni Nazarejstwa swego swietym bedzie Panu.
\par 9 I gdyby kto umarl przy nim z predka a nagle, i splugawilby glowe poswiecenia jego, ogoli glowe swoje w dzien oczyszczenia swego; dnia siódmego ogoli ja.
\par 10 A dnia ósmego przyniesie dwie synogarlice, albo dwoje golabiat do kaplana ku drzwiom namiotu zgromadzenia;
\par 11 I bedzie kaplan ofiarowal jedno za grzech, a drugie na ofiare calopalenia, i oczysci go od tego, czem zgrzeszyl nad umarlym, a poswieci glowe jego dnia onego.
\par 12 Potem odlaczy Panu dni Nazarejstwa swego, ofiarujac baranka rocznego za wystepek; a dni one pierwsze daremne beda, gdyz splugawione bylo Nazarejstwo jego.
\par 13 A toc jest prawo Nazarejczyka: Gdy sie wypelnia dni Nazarejstwa jego, przyjdzie do drzwi namiotu zgromadzenia,
\par 14 I ofiarowac bedzie ofiare swa Panu, baranka rocznego, zupelnego jednego na ofiare calopalenia, i owce jedne roczna i zdrowa na ofiare za grzech, i barana jednego zupelnego na ofiare spokojna;
\par 15 Przytem kosz chlebów przasnych, z maki pszennej, placki zagniatane z oliwa, i kreple przasne oliwa namazane, z ofiara ich sniedna, i z ofiara ich mokra.
\par 16 I bedzie ofiarowal kaplan przed Panem, i uczyni ofiare za grzech jego, i calopalenie jego.
\par 17 Barana takze ofiarowac bedzie na spokojna ofiare Panu z koszem chlebów przasnych; takze ofiarowac bedzie kaplan ofiare jego sniedna i ofiare jego mokra.
\par 18 I ogoli Nazarejczyk przede drzwiami namiotu zgromadzenia glowe Nazarejstwa swego, a wziawszy wlosy z glowy Nazarejstwa swego, wlozy je na ogien, który jest pod ofiara spokojna.
\par 19 Przytem wezmie kaplan lopatke warzona barania, i jeden placek przasny z kosza, i jeden krepel niekwaszony, a da w rece Nazarejczykowe po ogoleniu Nazarejstwa jego;
\par 20 I bedzie to tam i sam obracal kaplan na ofiare obracania przed Panem; a rzecz ta poswiecona dostanie sie kaplanowi, tak piersi obracania, jako i lopatka podnoszenia; a potem bedzie mógl Nazarejczyk pic wino.
\par 21 Toc jest prawo Nazarejczyka, któryby slub uczynil, i ta ofiara jego Panu za Nazarejstwo jego, okrom tego, coby wiecej uczynic mógl; wedlug slubu swego, który uczynil, tak uczyni wedlug prawa Nazarejstwa swego.
\par 22 Potem rzekl Pan do Mojzesza, mówiac:
\par 23 Mów do Aarona i do synów jego, a rzecz: Tak blogoslawic bedziecie synom Izraelskim, mówiac do nich:
\par 24 Niech ci blogoslawi Pan, a niechaj cie strzeze;
\par 25 Niech rozjasni Pan oblicze swoje nad toba, a niech ci milosciw bedzie;
\par 26 Niech obróci Pan twarz swoje ku tobie, a niechaj ci da pokój.
\par 27 I beda wzywac imienia mego nad synami Izraelskimi, a Ja im blogoslawic bede.

\chapter{7}

\par 1 I stalo sie w dzien, którego dokonczyl Mojzesz, a wystawil przybytek, a pomazal go, i poswiecil go ze wszystkim sprzetem jego, i oltarz ze wszystkiem naczyniem jego, pomazal je, i poswiecil je,
\par 2 Ze ofiarowaly ksiazeta Izraelskie, przedniejsze z domów ojców swych, (co byli hetmany z kazdego pokolenia, i przelozonymi nad policzonymi.)
\par 3 A przyniesli ofiary swe przed Pana: szesc wozów przykrytych, i dwanascie wolów, jeden wóz od dwojga ksiazat, a od kazdego wól jeden, i postawili to przed przybytkiem.
\par 4 Tedy rzekl Pan do Mojzesza, mówiac:
\par 5 Wezmij od nich, aby to bylo na potrzebe przy sluzbie w namiocie zgromadzenia, i oddaj to Lewitom, kazdemu wedlug potrzeby urzedu jego.
\par 6 Wzial tedy Mojzesz one wozy i woly i oddal je Lewitom.
\par 7 Dwa wozy, i cztery woly dal synom Gersonowym wedlug potrzeby urzedów ich.
\par 8 Cztery zas wozy i osiem wolów dal synom Merarego wedlug potrzeby urzedów ich, pod wladze Itamara, syna Aarona kaplana.
\par 9 Ale synom Kaatowym nic nie dal: bo usluga swiatnicy byla przy nich, na ramieniu ja nosic musieli.
\par 10 Ofiarowaly tedy ksiazeta ku poswieceniu oltarza onegoz dnia, gdy byl pomazany; i ofiarowaly ksiazeta dary swe przed oltarzem.
\par 11 I rzekl Pan do Mojzesza: Jeden ksiaze jednego dnia, drugi ksiaze drugiego dnia oddawac bedzie dary swoje ku poswieceniu oltarza.
\par 12 I ofiarowal pierwszego dnia dar swój Naason, syn Aminadabów z pokolenia Judy.
\par 13 A dar jego byl: misa srebrna jedna, sto i trzydziesci syklów wagi jej, czasza srebrna jedna, siedemdziesiat syklów wagi jej wedlug sykla swiatnicy, obie pelne pszennej maki zagniecionej z oliwa na ofiare sniedna;
\par 14 Kadzielnica jedna z dziesieciu syklów zlota, pelna kadzidla dla kadzenia;
\par 15 Cielec jeden mlody, baran jeden, i baranek jeden roczny na ofiare palona;
\par 16 Koziel jeden z kóz za grzech;
\par 17 A na ofiare spokojna dwa woly, baranów piec, kozlów piec, i baranków rocznych piec. Tac byla ofiara Naasona, syna Aminadabowego.
\par 18 Wtórego dnia ofiarowal Natanael, syn Suharów, ksiaze z pokolenia Isascharowego.
\par 19 I ofiarowal dar swój, mise srebrna jedne, sto i trzydziesci syklów wagi jej, czasze srebrna jedne, siedemdziesiat syklów wagi jej wedlug sykla swiatnicy, obie pelne pszennej maki zagniecionej z oliwa na ofiare sniedna;
\par 20 Kadzielnice jedne z dziesieciu syklów zlota, pelna kadzidla;
\par 21 Cielca jednego mlodego, barana jednego, i baranka jednego rocznego na palona ofiare;
\par 22 Kozla tez jednego z kóz za grzech;
\par 23 A na ofiare spokojna dwa woly, baranów piec, kozlów piec, i baranków rocznych piec. Tac byla ofiara Natanaela, syna Suharowego.
\par 24 Trzeciego dnia ksiaze synów Zabulon Eliab, syn Helonów.
\par 25 Ofiara jego byla misa srebrna jedna, sto i trzydziesci syklów wagi jej, czasza srebrna jedna, siedemdziesiat syklów wagi jej wedlug sykla swiatnicy, obie pelne maki pszennej z oliwa zagniecionej na ofiare sniedna;
\par 26 Kadzielnica jedna z dziesieciu syklów zlota, pelna kadzidla;
\par 27 Cielec jeden mlody, baran jeden, i baranek roczny jeden na calopalenie.
\par 28 Koziel jeden z kóz, za grzech
\par 29 A na ofiare spokojna dwa woly, baranów piec, kozlów piec, baranków rocznych piec. Ta byla ofiara Eliaba, syna Helonowego.
\par 30 Dnia czwartego ksiaze z synów Rubenowych Elisur, syn Sedeurów.
\par 31 Ofiara jego byla misa srebrna jedna, sto i trzydziesci syklów wagi jej, czasza srebrna jedna siedemdziesiat syklów wagi jej wedlug syklów swiatnicy, obie pelne maki pszennej z oliwa zagniecionej na ofiare sniedna;
\par 32 Kadzielnica jedna z dziesieciu syklów zlota, pelna kadzidla;
\par 33 Cielec jeden mlody, baran jeden, baranek jeden roczny na palona ofiare;
\par 34 Koziel jeden z kóz, za grzech
\par 35 A na ofiare spokojna dwa woly, baranów piec, kozlów piec, baranków rocznych piec. Ta byla ofiara Elisura, syna Sedeurowego.
\par 36 Dnia piatego ksiaze synów Symeonowych Selumijel, syn Surysaddajów.
\par 37 Ofiara jego byla misa srebrna jedna, sto i trzydziesci syklów wagi jej, czasza srebrna jedna, siedemdziesiat syklów wagi jej wedlug sykla swiatnicy, obie pelne maki pszennej z oliwa zagniecionej na ofiare sniedna;
\par 38 Kadzielnica jedna z dziesieciu syklów zlota, pelna kadzidla.
\par 39 Cielec jeden mlody, baran jeden, baranek jeden roczny na palona ofiare;
\par 40 Koziel jeden z kóz, za grzech.
\par 41 A na ofiare spokojna dwa woly, baranów piec, kozlów piec, baranków rocznych piec. Ta byla ofiara Selumijela, syna Surysaddajowego.
\par 42 Dnia szóstego ksiaze synów Gadowych Elijazaf, syn Duelów.
\par 43 Ofiara jego byla misa srebrna jedna, sto i trzydziesci syklów wagi jej, czasza srebrna jedna, siedemdziesiat syklów wagi jej wedlug sykla swiatnicy, obie pelne maki pszennej z oliwa zagniecionej na ofiare sniedna;
\par 44 Kadzielnica jedna z dziesieciu syklów zlota, pelna kadzidla;
\par 45 Cielec jeden mlody, baran jeden, baranek roczny jeden na palona ofiare.
\par 46 Koziel jeden z kóz, za grzech.
\par 47 A na ofiare spokojna dwa woly, baranów piec, kozlów piec, baranków rocznych piec. Ta byla ofiara Elijazafa, syna Duelowego.
\par 48 Dnia siódmego ksiaze synów Efraimowych, Elisama, syn Ammiudów.
\par 49 Ofiara jego byla misa srebrna jedna, sto i trzydziesci syklów wagi jej, czasza srebrna jedna, siedemdziesiat syklów wagi jej wedlug sykla swiatnicy, obie pelne maki pszennej z oliwa zagniecionej, na ofiare sniedna;
\par 50 Kadzielnica jedna z dziesieciu syklów zlota, pelna kadzidla;
\par 51 Cielec jeden mlody, baran jeden, baranek jeden roczny na palona ofiare;
\par 52 Koziel jeden z kóz, za grzech.
\par 53 A na ofiare spokojna dwa woly, baranów piec, kozlów piec, baranków rocznych piec. Tac byla ofiara Elisamy, syna Ammiudowego.
\par 54 Dnia ósmego ksiaze synów Manasesowych Gamalijel, syn Pedasurów.
\par 55 Ofiara jego byla misa srebrna jedna, sto i trzydziesci syklów wagi jej, czasza srebrna jedna, siedemdziesiat syklów wagi jej wedlug sykla swiatnicy, obie pelne maki pszennej, z oliwa zagniecionej, na ofiare sniedna;
\par 56 Kadzielnica jedna z dziesieciu syklów zlota, pelna kadzidla;
\par 57 Cielec jeden mlody, baran jeden, baranek jeden roczny na palona ofiare;
\par 58 Koziel jeden z kóz, za grzech;
\par 59 A na ofiare spokojna dwa woly, baranów piec, kozlów piec, baranków rocznych piec. Ta byla ofiara Gamalijela, syna Pedasurowego.
\par 60 Dnia dziewiatego ksiaze synów Benjaminowych Abidan, syn Gedeonów.
\par 61 Ofiara jego byla misa srebrna jedna, sto i trzydziesci syklów wagi jej, czasza srebrna jedna, siedemdziesiat syklów wagi jej wedlug sykla swiatnicy, obie pelne maki pszennej, zagniecionej z oliwa na ofiare sniedna;
\par 62 Kadzielnica jedna z dziesieciu syklów zlota pelna kadzidla;
\par 63 Cielec jeden mlody, baran jeden, baranek roczny jeden na palona ofiare;
\par 64 Koziel jeden z kóz za grzech;
\par 65 A na ofiare spokojna dwa woly, baranów piec, kozlów piec, baranków rocznych piec. Ta byla ofiara Abidana, syna Gedeonowego.
\par 66 Dnia dziesiatego ksiaze synów Danowych Achyjezer, syn Ammisadajów.
\par 67 Ofiara jego byla misa srebrna jedna, sto i trzydziesci syklów wagi jej, czasza jedna srebrna, siedemdziesiat syklów wagi jej wedlug sykla swiatnicy obie pelne maki pszennej, zagniecionej z oliwa, na ofiare sniedna;
\par 68 Kadzielnica jedna z dziesieciu syklów zlota, pelna kadzidla;
\par 69 Cielec jeden mlody, baran jeden, baranek roczny jeden na palona ofiare;
\par 70 Koziel jeden z kóz, za grzech;
\par 71 A na spokojna ofiare dwa woly, baranów piec, kozlów piec, baranków rocznych piec. Ta byla ofiara Achyjezera, syna Ammisadajowego.
\par 72 Dnia jedenastego ksiaze synów Aserowych Pagijel, syn Ochranów.
\par 73 Ofiara jego byla misa srebrna jedna, sto i trzydziesci syklów wagi jej, i czasza jedna srebrna, siedemdziesiat syklów wagi jej wedlug sykla swiatnicy, obie pelne pszennej maki, zagniecionej z oliwa na ofiare sniedna;
\par 74 Kadzielnica jedna, z dziesieciu syklów zlota, pelna kadzidla;
\par 75 Cielec jeden mlody, baran jeden, baranek jeden roczny na palona ofiare;
\par 76 Koziel jeden z kóz, za grzech.
\par 77 A na ofiare spokojna dwa woly, baranów piec, kozlów piec, baranków rocznych piec. Ta byla ofiara Pagijela, syna Ochranowego.
\par 78 Dnia dwunastego ksiaze synów Neftalimowych Ahira, syn Enanów.
\par 79 Ofiara jego byla misa srebrna jedna, sto i trzydziesci syklów wagi jej, czasza srebrna jedna, siedemdziesiat syklów wagi jej wedlug sykla swiatnicy, obie pelne pszennej maki, zagniecionej z oliwa na ofiare sniedna;
\par 80 Kadzielnica jedna z dziesieciu syklów zlota, pelna kadzidla;
\par 81 Cielec jeden mlody, baran jeden, baranek jeden roczny na ofiare palona;
\par 82 Koziel jeden z kóz, za grzech;
\par 83 A na spokojna ofiare dwa woly, baranów piec, kozlów piec, baranków rocznych piec. Tac byla ofiara Ahira, syna Enanowego.
\par 84 Toc bylo poswiecenie oltarza, onegoz dnia, gdy pomazan jest od ksiazat Izraelskich: Mis srebrnych dwanascie, czasz srebrnych dwanascie, kadzielnic zlotych dwanascie;
\par 85 Sto i trzydziesci syklów jedna misa srebrna wazyla, siedemdziesiat syklów czasza jedna; wszystkiego srebra w onem naczyniu bylo dwa tysiace i cztery sta syklów wedlug sykla swiatnicy;
\par 86 Kadzielnic zlotych dwanascie pelnych kadzidla; dziesiec syklów wazyla kazda wedlug sykla swiatnicy; wszystkiego zlota w onych kadzielnicach bylo sto i dwadziescia syklów.
\par 87 A wszystkiego bydla ku ofierze palonej dwanascie cielców, baranów dwanascie, z baranków rocznych dwanascie, z ofiara ich sniedna, i kozlów z kóz za grzech dwanascie.
\par 88 Wszystkiego zasie bydla na ofiare spokojna bylo wolów dwadziescia i cztery, baranów szescdziesiat, kozlów szescdziesiat; baranków rocznych szescdziesiat. Toc bylo poswiecenie oltarza po pomazaniu jego.
\par 89 A gdy Mojzesz wchodzil do namiotu zgromadzenia, by sie rozmawial z Bogiem, tedy slyszal glos mówiacego do siebie z ublagalni, która byla nad skrzynia swiadectwa, miedzy dwiema Cheruby, a stamtad mawial do niego.

\chapter{8}

\par 1 Potem Pan rzekl do Mojzesza, mówiac:
\par 2 Powiedz Aaronowi, a rzecz mu: Gdy zapalisz lampy, siedem lamp przeciwko swiecznikowi swiecic beda.
\par 3 I uczynil tak Aaron, a przeciwko swiecznikowi zapalil lampy jego, jako byl rozkazal Pan Mojzeszowi.
\par 4 A byla robota swiecznika z ciagnionego zlota, i slupiec jego, i kwiaty jego ciagnione byly; na ten ksztalt, jaki byl Pan ukazal Mojzeszowi, tak urobil swiecznik.
\par 5 Potem rzekl Pan do Mojzesza, mówiac:
\par 6 Wezmij Lewity z posród synów Izraelskich, a oczysc je.
\par 7 A to uczynisz oczyszczajac je: Pokropisz je woda oczyszczenia; ciz ogola brzytwa wszystko cialo swoje, a uprawszy szaty swe, czystymi beda.
\par 8 Potem wezma cielca mlodego, z ofiara jego sniedna, maki pszennej, zagniecionej z oliwa, a cielca mlodego drugiego wezmiesz na ofiare za grzech.
\par 9 I przywiedziesz Lewity przed namiot zgromadzenia, a przyzowiesz wszystkiego zgromadzenia synów Izraelskich;
\par 10 I postawisz Lewity przed Panem, i wloza synowie Izraelscy rece swe na Lewity;
\par 11 I ofiarowac bedzie Aaron Lewity na ofiare przed panem od synów Izraelskich, aby sprawowali poslugi Panskie.
\par 12 Lewitowie zas beda klasc rece swe na glowy onych cielców, a ofiarowac bedziesz jednego za grzech, a drugiego na ofiare calopalenia Panu ku oczyszczeniu Lewitów.
\par 13 Potem postawisz Lewity przed Aaronem, i przed syny jego, a ofiarowac je bedziesz na ofiare Panu.
\par 14 I odlaczysz Lewity z posród synów Izraelskich, i beda moimi Lewitowie.
\par 15 A potem przyjda Lewitowie, aby sluzyli w namiocie zgromadzenia, gdy oczyscisz i poswiecisz je na ofiare.
\par 16 Albowiem wlasnie oddani sa mnie z posród synów Izraelskich; za kazde otwierajace zywot, za kazde pierworodne z synów Izraelskich obralem je sobie,
\par 17 Gdyz wszyscy pierworodni z synów Izraelskich moi sa z ludzi i z bydla: ode dnia, któregom pobil wszystkie pierworodne w ziemi Egipskiej, poswiecilem je sobie.
\par 18 A przyjalem Lewity miasto wszelkiego pierworodnego z synów Izraelskich.
\par 19 I dalem Lewity darem Aaronowi i synom jego z posród synów Izraelskich, aby odprawowali sluzby miasto synów Izraelskich w namiocie zgromadzenia, i oczyszczali syny Izraelskie, aby nie przyszlo na syny Izraelskie karanie, gdyby przystepowali synowie Izraelscy do swiatnicy.
\par 20 Uczynili tedy Mojzesz i Aaron i wszystko zgromadzenie synów Izraelskich z Lewitami wszystko, co rozkazal Pan Mojzeszowi o Lewitach, tak uczynili z nimi synowie Izraelscy.
\par 21 I oczyscili sie Lewitowie, a uprali szaty swoje, i ofiarowal je Aaron na ofiare przed Panem, i oczyscil je Aaron, aby byli czystymi.
\par 22 Dopiero potem przystapili Lewitowie ku sprawowaniu urzedu swego w namiocie zgromadzenia przed Aaronem i przed syny jego; jako rozkazal Pan Mojzeszowi o Lewitach, tak im uczynili.
\par 23 Rzekl nadto Pan do Mojzesza, mówiac:
\par 24 To tez Lewitom nalezy: Od dwudziestego i piatego roku i wyzej kazdy przystapi, aby sprawowal urzad przy posludze namiotu zgromadzenia.
\par 25 A w piecdziesiat lat przestanie pracowac w urzedzie, i wiecej sluzyc nie bedzie.
\par 26 Ale nadslugowac bedzie braci swej w namiocie zgromadzenia straz trzymajacym, lecz sluzby samej odprawowac nie bedzie. Tak sobie postapisz z Lewitami w urzedziech ich.

\chapter{9}

\par 1 I rzekl Pan do Mojzesza na puszczy Synaj, roku wtórego po wyjsciu ich z ziemi Egipskiej, miesiaca pierwszego, mówiac:
\par 2 Niech obchodza synowie Izraelscy swieto przejscia czasu naznaczonego.
\par 3 Czternastego dnia miesiaca tego, miedzy dwoma wieczorami, obchodzic je bedziecie czasu naznaczonego; wedlug wszystkich obrzedów jego, i wedlug wszystkich ceremonii jego, obchodzic je bedziecie.
\par 4 Mówil tedy Mojzesz do synów Izraelskich, aby obchodzili swieto przejscia.
\par 5 I obchodzili swieto przejscia, pierwszego miesiaca, czternastego dnia, miedzy dwoma wieczorami, na puszczy Synaj; wedlug wszystkiego, jak rozkazal Pan Mojzeszowi, tak uczynili synowie Izraelscy.
\par 6 I byli niektórzy ludzie, którzy sie byli splugawili nad umarlym czlowiekiem, i nie mogli obchodzic swieta przejscia dnia onego; tedy przystapili do Mojzesza i do Aarona w tenze dzien;
\par 7 I rzekli oni ludzie do niego: Zmazalismy sie nad umarlym: i nie bedziez nam wolno oddac ofiary Panu czasu naznaczonego wespól z synami Izraelskimi?
\par 8 Którym odpowiedzial Mojzesz: Postójcie, az uslysze, co rozkaze Pan o was.
\par 9 Tedy rzekl Pan do Mojzesza, mówiac:
\par 10 Powiedz synom Izraelskim i rzecz: Jezliby sie kto zmazal nad umarlym, alboby na drodze dalekiej byl, tak z was, jako i z potomstwa waszego, przecie bedzie odprawowal swieto przejscia Panu.
\par 11 Miesiaca wtórego, czternastego dnia, miedzy dwoma wieczorami, odprawowac je beda; z przasnemi chleby, i z gorzkiemi zioly jesc je beda:
\par 12 Nie zostawia nic z niego do jutra, i kosci nie zlamia w nim; wedlug wszystkiego postanowienia swieta przejscia odprawowac je beda:
\par 13 Ale czlowiek, któryby byl czysty, a nie bylby w drodze, i nie obchodzilby swieta przejscia, tedy dusza ona wykorzeniona bedzie z ludu swego, bo ofiary Panskiej nie odprawowal czasu naznaczonego; grzech swój poniesie on czlowiek.
\par 14 A jezliby przychodzien mieszkajacy miedzy wami obchodzil swieto przejscia Panu, wedlug ustawy swieta przejscia i wedlug obrzedów jego obchodzic je bedzie; ustawa jedna bedzie, wam i przychodniowi, i zrodzonemu w ziemi.
\par 15 Dnia tedy onego, którego wystawiony byl przybytek, oblok okryl przybytek nad namiotem swiadectwa, a wieczór bywalo nad przybytkiem jako widzenie ognia az do poranku.
\par 16 Tak bywalo ustawicznie; we dnie okrywal go oblok, a jako widzenie ognia w nocy.
\par 17 A gdy sie podnaszal oblok od namiotu, tedy sie ruszali synowie Izraelscy; a gdziekolwiek stawal oblok, tamze stanowili obóz synowie Izraelscy.
\par 18 Na rozkazanie Panskie ciagneli synowie Izraelscy, i na rozkazanie Panskie stanowili obóz; po wszystkie dni, których zostawal oblok nad przybytkiem, i oni lezeli obozem.
\par 19 A gdy trwal oblok nad przybytkiem przez wiele dni, tedy odprawowali synowie Izraelscy straz Panu, a nie ruszali sie.
\par 20 Ale gdy nie dlugo trwal oblok nad przybytkiem, na rozkazanie Panskie stanowili obóz, i na rozkazanie Panskie ciagneli.
\par 21 A gdy bywal oblok od wieczora az do poranku, a podnosil sie zas poranku, tedy ciagneli; tak we dnie jako i w nocy, gdy sie podniósl oblok, ciagneli.
\par 22 A jezli przez dwa dni, albo przez miesiac, albo tez przez rok trwal oblok nad przybytkiem, zostawajac nad nim, obozem lezeli synowie Izraelscy, i nie ruszali sie; ale gdy sie on podnosil, i oni sie ruszali.
\par 23 Na rozkazanie Panskie stanowili obóz, i na rozkazanie Panskie ciagneli, straz Panska trzymajac, jako im Pan rozkazal przez Mojzesza.

\chapter{10}

\par 1 Potem Pan rzekl do Mojzesza, mówiac:
\par 2 Spraw sobie dwie traby srebrne; robota ciagniona uczynisz je, których uzywac bedziesz do zwolywania ludu, i gdyby sie wojsko ruszac mialo.
\par 3 A gdy zatrabia w nie, tedy sie do ciebie zbiezy wszystek lud ku drzwiom namiotu zgromadzenia.
\par 4 A jezliby w jedne tylko zatrabiono, tedy sie zejda do ciebie ksiazeta, i hetmani wojsk Izraelskich.
\par 5 Gdyby zas zatrabiono glos przerywajac, tedy sie ruszy obóz lezacych na wschód slonca.
\par 6 A gdy drugi raz zatrabia, glos przerywajac, tedy sie ruszy obóz lezacych na poludnie; z przerywaniem trabic beda, gdy sie ruszyc beda mieli.
\par 7 Ale gdy zwolywac lud bedziecie, trabic bedziecie, a nie bedziecie przerywac.
\par 8 A synowie Aaronowi, kaplani, trabic beda w traby: i bedzie wam to za ustawe wieczna w potomstwie waszem.
\par 9 A gdy wyciagniecie na wojne w ziemi waszej przeciwko nieprzyjacielowi, któryby was trapil, z przerywaniem w traby trabic bedziecie; a przyjdziecie na pamiec przed Panem, Bogiem waszym, i zachowani bedziecie od nieprzyjciól waszych.
\par 10 W dzien takze wesela waszego, i w swieta uroczyste wasze, i na nowiu miesiecy waszych, bedziecie trabic w te traby przy ofiarach waszych calopalnych, i przy ofiarach waszych spokojnych, i przywioda was na pamiec przed Bogiem waszym; Ja Pan, Bóg wasz.
\par 11 I stalo sie roku wtórego, miesiaca wtórego, dnia dwudziestego tegoz miesiaca, ze sie podniósl oblok przybytku swiadectwa.
\par 12 I ruszyli sie synowie Izraelscy z hufcami swymi z puszczy Synaj, a stanal oblok na puszczy Faran.
\par 13 I ruszyli sie najpierwej tak, jako byl Pan rozkazal przez Mojzesza.
\par 14 Albowiem ruszyla sie choragiew obozu synów Judowych naprzód z hufcami swemi, a nad wojskiem jego byl hetman Naason, syn Aminadabów.
\par 15 A nad wojskiem pokolenia synów Isascharowych byl hetmanem Natanael, syn Suharów.
\par 16 A nad wojskiem pokolenia synów Zabulonowych byl hetmanem Elijab, syn Helonów.
\par 17 Zatem zlozono przybytek, i ciagneli synowie Gersonowi, i synowie Merarego, niosac przybytek.
\par 18 Ruszyla sie zas choragiew obozu Rubenowego z hufcami swemi, a nad wojskiem jego byl hetmanem Elisur, syn Sedeurów.
\par 19 A nad wojskiem pokolenia synów Symeonowych byl hetmanem Selumijel, syn Surysaddajów.
\par 20 A nad wojskiem tez pokolenia synów Gadowych byl hetmanem Elijazaf, syn Duelów.
\par 21 Zatem ruszyli sie Kaatytowie, niosac swiatnice, i stanowili przybytek, az ci nadciagneli.
\par 22 Potem ruszyla sie choragiew obozu synów Efraimowych z hufcami swemi, a nad wojskiem jego byl hetmanem Elisama, syn Ammiudów.
\par 23 Nad wojskiem zas pokolenia synów Manasesowych byl hetmanem Gamalijel, syn Pedasurów.
\par 24 Nad wojskiem zas pokolenia synów Benjaminowych byl hetmanem Abidan, syn Gedeonów.
\par 25 Potem ruszyla sie choragiew obozu synów Danowych zawierajac wszystkie obozy z wojski ich, a nad wojskiem jego byl hetmanem Achyjezer, syn Ammisaddajów.
\par 26 A nad wojskiem pokolenia synów Eserowych byl hetmanem Pagijel, syn Ochranów.
\par 27 A nad wojskiem pokolenia synów Neftalimowych byl hetmanem Ahira, syn Enanów.
\par 28 Takiec bylo ciagnienie synów Izraelskich z hufcami ich; i tak ciagneli.
\par 29 Potem rzekl Mojzesz do Hobaba, syna Raguelowego Madyjanczyka, swiekra swego: My ciagniemy do miejsca, o którem rzekl Pan: Dam je wam. Pójdz z nami, a uczynimyc dobrze, poniewaz Pan obiecal wiele dobrego Izraelowi.
\par 30 Któremu on odpowiedzial: Nie pójde: ale sie wróce do ziemi mojej i do rodziny mojej.
\par 31 I rzekl Mojzesz: Prosze, nie opuszczaj nas; bo ty wiesz, gdzie bysmy na puszczy obóz stanowic mieli, i bedziesz przewodnikiem naszym.
\par 32 A jezliz pójdziesz z nami, i spotka nas to dobre, którem nam Pan uczyni dobrze, i my dobrze uczynimy tobie.
\par 33 I odciagneli od góry Panskiej droga trzech dni, a skrzynia przymierza Panskiego szla przed nimi droga trzech dni, aby im upatrzyla miejsce odpocznienia.
\par 34 A oblok Panski szedl nad nimi we dnie, gdy sie ruszali z stanowiska.
\par 35 A gdy sie ruszyc miala skrzynia, tedy mawial Mojzesz: Powstan Panie, a niech beda rozproszeni nieprzyjaciele twoi, a niech uciekaja, którzy cie nienawidza, przed obliczem twojem.
\par 36 A gdy zas stanela, tedy mówil: Nawróc sie Panie do niezliczonego mnóstwa wojska Izraelskiego.

\chapter{11}

\par 1 I stalo sie, ze sie lud uskarzal nieslusznie, co sie nie podobalo Panu. Przetoz uslyszawszy to Pan bardzo sie rozgniewal, i zapalil sie przeciwko nim ogien Panski, i popalil ostatnia czesc obozu.
\par 2 Tedy wolal lud na Mojzesza; i modlil sie Mojzesz Panu, i zgasl ogien.
\par 3 I nazwal imie miejsca onego Tabera: bo sie zapalil przeciwko nim ogien Panski.
\par 4 A lud pospolity, który byl miedzy nimi, chciwoscia wielka zjety odwracal sie; i plakali tez synowie Izraelscy, mówiac: Któz nas nakarmi miesem?
\par 5 Wspominamy sobie na ryby, któresmy jadali w Egipcie darmo, na ogórki, i na melony, i na luczek, i na cebule, i na czosnek.
\par 6 A teraz dusza nasza wywiedla nic inszego nie majac, oprócz tej manny, przed oczyma swemi.
\par 7 A manna byla jako nasienie koryjandrowe, a barwa tej jako barwa Bdelijowa.
\par 8 I wychodzil lud, a zbierali ja, i melli w zarnach, albo tlukli w mozdzierzach a warzyli w kotlach i czynili z niej podplomyki; a byl smak jej jako smak swiezej oliwy.
\par 9 Gdy bowiem padala rosa na obóz w nocy, padala tez manna nan.
\par 10 Tedy uslyszal Mojzesz, ze lud plakal po domach swych, kazdy u drzwi namiotu swego; dla czego zapalila sie popedliwosc Panska wielce, i nie podobalo sie tez to Mojzeszowi.
\par 11 I rzekl Mojzesz do Pana: Czemuzes tak zle uczynil sludze twemu? czemuzem nie znalazl laski w oczach twoich, zes wlozyl ciezar tego wszystkiego ludu na mie?
\par 12 Izalim ja poczal ten wszystek lud? izalim go ja zrodzil, iz mi mówisz: Nies je na lonie twojem, jako piastun nosi niemowlatko, do ziemi, o któras przysiagl ojcom ich?
\par 13 Gdziez mam mieso, abym dal wszystkiemu temu ludowi? bo placza na mie, mówiac: Daj nam miesa, abysmy jedli.
\par 14 Nie moge ja sam zniesc wszystkiego ludu tego; bo to nad moznosc moje.
\par 15 A jezli sie tak ze mna obchodzic chcesz, prosze raczej mie zabij, jezlim znalazl laske w oczach twoich, abym nie patrzyl na swoje zle.
\par 16 I rzekl Pan do Mojzesza: Zbierz mi siedemdziesiat mezów z starszych Izraelskich, które znasz, ze sa starszymi ludu, i ksiazeta jego, a przywiedz je przed namiot zgromadzenia, i stac tam bede z toba;
\par 17 A Ja zstapie, i bede tam mówil z toba, i wziawszy z Ducha, który jest w tobie, udziele im; i poniosa z toba brzemie ludu, a nie poniesiesz go ty sam.
\par 18 A do ludu rzeczesz: Poswieccie sie na jutro, a bedziecie jesc mieso; boscie plakali w uszach Panskich, mówiac: Któz nas nakarmi miesem? bo nam lepiej bylo w Egipcie; i da wam Pan miesa, i bedziecie jedli.
\par 19 Nie przez jeden dzien jesc bedziecie, ani przez dwa dni, ani przez piec, ani przez dziesiec dni, ani przez dwadziescia dni:
\par 20 Ale przez caly miesiac, az polezie przez nozdrza wasze, a zbrzydzi sie wam, przeto zescie pogardzili Panem, który jest miedzy wami, a plakaliscie przed nim mówiac: Na cózesmy wyszli z Egiptu?
\par 21 I rzekl Mojzesz: Szesc kroc sto tysiecy pieszych jest ludu, miedzy którym ja mieszkam, a tys powiedzial: Dam im miesa, ze beda jesc caly miesiac.
\par 22 Izali im owiec i wolów nabija, aby sie im dostalo? Izali wszystkie ryby morskie zbiora im, aby dostatek mieli?
\par 23 I rzekl Pan do Mojzesza: Azaz reka Panska jest skurczona? teraz ujrzysz, jesli sie wypelni slowo moje, czyli nie.
\par 24 Tedy wyszedl Mojzesz, i opowiedzial ludowi slowa Panskie; a zebrawszy siedemdziesiat mezów starszych z ludu, postawil je okolo namiotu.
\par 25 I zstapil Pan w obloku, i mówil do niego, a wziawszy z Ducha, który byl w nim, podzielil go miedzy siedemdziesiat mezów starszych; i stalo sie, gdy odpoczal nad nimi on Duch, tedy prorokowali, a potem nigdy.
\par 26 Ale zostali byli dwa mezowie w obozie, imie jednego Eldad, a imie drugiego Medad; na których tez odpoczal on Duch, bo oni byli napisani, choc byli nie przyszli do namiotu; a tak ci prorokowali w obozie.
\par 27 Tedy przybiezalo pachole, i oznajmilo Mojzeszowi, mówiac: Eldad i Medad prorokuja w obozie.
\par 28 Ale odpowiedziawszy Jozue, syn Nunów, sluga Mojzeszów, jeden z mlodzienców jego, rzekl: Panie mój Mojzeszu, zabron im.
\par 29 Któremu odpowiedzial Mojzesz: Czemuz im ty zazdroscisz dla mnie? Boze daj, aby wszystek lud Panski prorokowal, a izby dal Pan Ducha swego na nie!
\par 30 Wrócil sie tedy Mojzesz do obozu, on i starsi Izraelscy.
\par 31 Zatem wyszedl wiatr od Pana, i porwawszy przepiórki od morza, spuscil je na obóz, z jednej strony jako na jeden dzien chodu, z drugiej strony takze jako na jeden dzien chodu, okolo obozu, a jakoby na dwa lokcie byly nad ziemia.
\par 32 Tedy wstawszy lud przez caly on dzien i przez cala noc, i nazajutrz przez wszystek dzien zbierali one przepiórki, a kto najmniej nazbieral, mial dziesiec homerów: i nawieszali ich wszedy sobie okolo obozu.
\par 33 Mieso jeszcze bylo miedzy zebami ich nie zezwane, gdy gniew Panski zapalil sie na lud, i pobil Pan lud on plaga bardzo wielka.
\par 34 I nazwane jest imie miejsca onego Kibrot Hataawa; albowiem tam pogrzebli lud, który pozadal miesa.
\par 35 A z Kibrot Hataawy ruszyl sie lud do Haserotu, i mieszkali w Haserocie.

\chapter{12}

\par 1 Tedy mówila Maryja i Aaron przeciw Mojzeszowi dla zony Murzynskiej, która pojal; bo zone murzynke byl pojal;
\par 2 I mówili: Izali tylko przez Mojzesza mówil Pan? azaz tez nie mówil przez nas? a to uslyszal Pan.
\par 3 A Mojzesz byl maz najpokorniejszy ze wszystkich ludzi, którzy mieszkali na ziemi.
\par 4 A natychmiast rzekl Pan do Mojzesza, i do Aarona, i do Maryi: Wynijdzcie was troje przed namiot zgromadzenia; i wyszli samo troje.
\par 5 Zatem zstapil Pan w slupie oblokowym, i stanal we drzwiach namiotu, i zawolal Aarona i Maryi, i przyszli oboje.
\par 6 I rzekl do nich: Sluchajcie teraz slów moich: Jezli miedzy wami bedzie prorok, Ja Pan w widzeniu ukaze mu sie we snie bede mówil z nim;
\par 7 Ale nie taki jest sluga mój Mojzesz, który we wszystkim domu moim najwierniejszy jest.
\par 8 Usty do ust mawiam z nim: nie w widzeniu, ani w zagadaniu, ani w podobienstwach Pana widywa; czemuzescie sie nie bali, mówic przeciw sludze memu Mojzeszowi?
\par 9 A tak zapalil sie gniewem przeciwko nim Pan, i odszedl.
\par 10 Oblok takze odstapil od namiotu, a oto, Maryja otredowaciala, zbielawszy jako snieg; a wejrzawszy Aaron na Maryje, ujrzal tredowata.
\par 11 Potem rzekl Aaron do Mojzesza: Prosze panie mój, nie kladz teraz na nas grzechu tego, zesmy glupio uczynili, a zesmy zgrzeszyli.
\par 12 Niech prosze nie bedzie jako martwy plód, który gdy wychodzi z zywota matki swej, polowa ciala jego zepsowana bywa.
\par 13 Tedy zawolal Mojzesz do Pana mówiac: Boze, prosze uzdrów ja teraz.
\par 14 I odpowiedzial Pan Mojzeszowi: Gdyby ojciec jej plunal na twarz jej, azazby nie miala sie wstydac przez siedem dni? niech bedzie wylaczona przez siedem dni z obozu, a potem przyjeta bedzie.
\par 15 I wylaczona byla Maryja z obozu przez siedem dni; a lud sie nie ruszyl, az byla Maryja przyjeta.

\chapter{13}

\par 1 Potem ruszyl sie lud z Haserotu, a polozyli sie obozem na puszczy Faran.
\par 2 I rzekl Pan do Mojzesza, mówiac:
\par 3 Poslij meze, którzy by przeszpiegowali ziemie Chananejska, która Ja dawam synom Izraelskim; po jednym mezu z kazdego pokolenia poslecie, którzy by byli przedniejsi miedzy nimi.
\par 4 Wyslal je tedy Mojzesz z puszczy Faran wedlug rozkazania Panskiego; a oni mezowie wszyscy byli co przedniejsi z synów Izraelskich.
\par 5 A tec sa imiona ich: Z pokolenia Ruben Samua, syn Zachurów.
\par 6 Z pokolenie Symeonowego Safat, syn Hurów.
\par 7 Z pokolenia Judy Kaleb, syn Jefunów.
\par 8 Z pokolenia Isaschar Igal, syn Józefów.
\par 9 Z pokolenia Efraimowego Ozeasz, syn Nunów.
\par 10 Z pokolenia Benjaminowego Falty, syn Rafuów.
\par 11 Z pokolenia Zabulon Gedyjel, syn Sodego.
\par 12 Z pokolenia Józefowego, to jest z potomstwa Manasesowego, Gady, syn Susego.
\par 13 Z pokolenia Dan Ammijel, syn Gemmalego.
\par 14 Z pokolenia Aser Setur, syn Michaelów.
\par 15 Z pokolenia Neftali, Nabi, syn Wafsego.
\par 16 Z pokolenia Gad Guel, syn Machego.
\par 17 Tec sa imiona mezów, które poslal Mojzesz na przeszpiegowanie ziemi: i nazwal Mojzesz Ozeasza, syna Nunowego, Jozue.
\par 18 A posylajac je Mojzesz na przeszpiegowanie ziemi Chananejskiej, mówil do nich: Idzcie w te strone ku poludniowi, a wnijdzcie na góre:
\par 19 I ogladajcie ziemie, jaka jest, i lud, który mieszka w niej, jezli jest mocny, czyli mdly? jezli ich malo, czyli wiele?
\par 20 Jaka tez jest ziemia, w której mieszkaja, jezli dobra, czyli zla? i co sa za miasta, w których mieszkaja, jezli w namieciech, czyli w obronnych miejscach?
\par 21 Takze co za ziemia, jezli urodzajna, czyli nieplodna? jezli w niej sa drzewa, czyli nie? a badzcie meznego serca i przyniescie nam owocu tamtej ziemi; a bylo to w ten czas, gdy sie wina dostawaly.
\par 22 I szli, a przeszpiegowali ziemie od puszczy Syn az do Rochob, któredy chodza do Emat.
\par 23 A idac ku poludniowi przyszli az do Hebron, gdzie byli Ahiman, Sesai i Talmai, synowie Enakowi; a Hebron siedem lat bylo zbudowane przed Soan Egipskiem.
\par 24 Przyszli potem az do rzeki Eschol, i urzneli tam Gala? z gronem jednem jagód winnych, i niesli ja na drazku, dwa takze granatowe jablka i figi.
\par 25 I nazwano miejsce ono Nachal Eschol, od grona winnego, które tam byli urzneli synowie Izraelscy.
\par 26 Zatem wrócili sie od onego szpiegowania ziemi po wyjsciu czterdziestu dni.
\par 27 A wróciwszy sie, przyszli do Mojzesza i do Aarona, i do wszystkiego zgromadzenia synów Izraelskich na puszcza Faran, która jest w Kades; i dali im i wszystkiemu pospólstwu sprawe, ukazawszy im owoc onej ziemi.
\par 28 A powiadali im, mówiac: Przyszlismy do ziemi, do którejs nas byl poslal, która w prawdzie oplywa mlekiem i miodem, a oto, ten jest owoc jej;
\par 29 Tylko ze mocny jest lud, który mieszka w onej ziemi; miasta takze obronne sa, i bardzo wielkie; do tego i syny Enakowe tamesmy widzieli.
\par 30 Amalek mieszka w ziemi na poludnie, a Hetejczyk, i Jebuzejczyk, i Amorejczyk mieszka na górach; Chananejczyk zas mieszka nad morzem, i nad brzegiem Jordanu.
\par 31 I hamowal Kaleb lud szemrzacy przeciw Mojzeszowi, i mówil: Pójdzmy a posiadzmy ziemie, bo ja pewnie sobie podbijemy.
\par 32 Ale mezowie oni, którzy z nim chodzili, rzekli: Nie bedziem mogli ciagnac przeciw tamtemu ludowi; bo mocniejszy jest nad nas.
\par 33 I zganili one ziemie, która byli przeszpiegowali, synom Izraelskim, mówiac: Ziemia ta, którasmy przeszli, i przeszpiegowali ja, jest ziemia pozerajaca obywatele swoje, a lud wszystek, którysmy w niej widzieli, lud jest wysokiego wzrostu.
\par 34 Tamesmy tez widzieli olbrzymy, syny Enakowe z rodu olbrzymów; i zdalismy sie sobie przy nich jako szarancza, takimiz zdalismy sie i onym.

\chapter{14}

\par 1 Tedy wzruszywszy sie wszystko mnóstwo krzyczeli i plakal lud przez one noc.
\par 2 I szemrali przeciwko Mojzeszowi, i przeciwko Aaronowi wszyscy synowie Izraelscy; i mówilo do nich wszystko mnóstwo: Obysmy byli pomarli w ziemi Egipskiej, albo na tej puszczy!
\par 3 Obysmy byli pomarli! Czemuz wzdy Pan prowadzi nas do tej ziemi, abysmy padli od miecza? zony nasze, i dziatki nasze aby byly na lup? Izali nam nie lepiej wrócic sie do Egiptu?
\par 4 I mówili miedzy soba: Postanówmy sobie wodza, a wrócmy sie do Egiptu.
\par 5 Tedy upadl Mojzesz i Aaron na oblicze swoje przed wszystkiem zgromadzeniem synów Izraelskich.
\par 6 A Jozue, syn Nunów, i Kaleb, syn Jefunów, którzy szpiegowali ziemie, rozdarli szaty swoje;
\par 7 I rzekli do wszystkiego zgromadzenia synów Izraelskich, mówiac: Ziemia, którasmy przeszli, i przeszpiegowali ja, ziemia jest bardzo dobra.
\par 8 Bedzieli nam Pan milosciw, tedy nas wprowadzi do tej ziemi, a da ja nam, ziemie te, która oplywa mlekiem i miodem.
\par 9 Jedno Panu nie badzcie odpornymi, ani sie wy bójcie ludu onej ziemi, bo jako chleb pojesc je mozemy; odstapila obrona ich od nich, ale Pan jest z nami; nie bójciez sie ich.
\par 10 I mówilo wszystko zgromadzenie, aby je ukamionowano; ale chwala Panska okazala sie nad namiotem zgromadzenia wszystkim synom Izraelskim.
\par 11 I rzekl Pan do Mojzesza: Dokadze mie draznic bedzie ten lud? I dokadze mi wierzyc nie beda dla tych wszystkich znaków, którem czynil miedzy nimi?
\par 12 Poraze je morem, i rozprosze je; a ciebie uczynie w naród wielki i mozniejszy, niz ten jest.
\par 13 Ale rzekl Mojzesz do Pana: Oto, uslysza Egipczanie, z których posrodku wywiodles moca swoja ten lud;
\par 14 I mówic beda z obywatelami ziemi tej, bo slyszeli, zes ty Panie byl w posrodku ludu tego; zes okiem w oko widziany byl, o Panie, a oblok twój stawal nad nimi, a iz w slupie oblokowym chodziles przed nimi we dnie, a w slupie ognistym w nocy.
\par 15 Gdybys tedy pobil lud ten wszystek az do jednego, rzekliby poganie, którzy o twej slawie slychali, mówiac:
\par 16 Iz nie mógl wprowadzic Pan ludu tego do ziemi, o która im przysiagl: przeto je pobil na puszczy.
\par 17 A tak teraz niech prosze uwielbiona bedzie moc Panska, jakos rzekl, mówiac:
\par 18 Pan nie rychly ku gniewowi a wielki w milosierdziu, znoszac nieprawosc i przestepstwo, który winnego nie czyni niewinnym, karzac nieprawosc ojców w synach do trzeciego i do czwartego pokolenia;
\par 19 Odpusc prosze nieprawosc ludu tego wedlug wielkosci milosierdzia twego, tak jakos odpuszczal ludowi temu z Egiptu az dotad.
\par 20 Tedy rzekl Pan: Odpuscilem wedlug slowa twego.
\par 21 A wszakze, jako Ja zyje, i napelniona jest chwala Panska wszystka ziemia:
\par 22 Tak wszyscy, którzy widzieli chwale moje, i znaki moje, którem czynil w Egipcie, i na puszczy, a kusili mie juz po dziesiec kroc, ani byli posluszni glosowi memu,
\par 23 Nie ogladaja ziemi tej, o któram przysiagl ojcom ich, a zaden z tych, którzy mie draznili, nie ogladaja jej.
\par 24 Ale sluge mego Kaleba, gdyz byl w nim duch inakszy, i trwal statecznie przy mnie, wprowadze do ziemi, do której chodzil, a nasienie jego odziedziczy ja.
\par 25 Ale poniewaz Amalekita i Chananejczyk mieszkaja w dolinie, przetoz jutro obróccie sie, a idzcie na puszcza, droga ku morzu czerwonemu.
\par 26 Nad to rzekl Pan do Mojzesza i do Aarona, mówiac:
\par 27 I dokadze znosic mam ten zly lud, który szemrze przeciwko mnie? dlugoz szemrania synów Izraelskich, którzy szemrza przeciwko mnie, sluchac bede?
\par 28 Mów do nich: Zyje Ja, mówi Pan, ze jakoscie mówili w uszy moje, tak uczynie wam.
\par 29 Na tej puszczy polega ciala wasze, i wszyscy policzeni wasi wedlug wszystkiej liczby waszej od dwudziestego roku i wyzej, którzyscie szemrali przeciwko mnie.
\par 30 A wy nie wnijdziecie do ziemi tej, o któram podniósl reke moje, abym ja wam dal na mieszkanie, okrom Kaleba, syna Jefunowego, i Jozuego, syna Nunowego;
\par 31 A dziatki wasze, o którychescie mówili, ze beda na lup, te wprowadze, i ogladaja te ziemie, którascie wy wzgardzili.
\par 32 Ale trupy wasze, wasze trupy mówie, polega na tej puszczy;
\par 33 A synowie wasi beda sie tulali po tej puszczy przez czterdziesci lat, i poniosa karanie za cudzolóstwa wasze, az wygina trupy wasze na puszczy.
\par 34 Wedlug liczby dni, w którychescie szpiegowali ziemie, to jest czterdziesci dni, dzien kazdy za rok liczac, poniesiecie nieprawosci wasze czterdziesci lat, i poznacie pomste swego odstapienia ode mnie.
\par 35 Ja Pan mówilem, ze to uczynie temu wszystkiemu zgromadzeniu zlemu, które sie spiknelo przeciwko mnie; na tej puszczy pogina, i tu pomra.
\par 36 Oni tedy mezowie, których slal Mojzesz na przeszpiegowanie ziemi, którzy wróciwszy sie pobudzili do szemrania przeciwko niemu wszystek lud, puszczajac zla slawe o ziemi onej;
\par 37 Pomarli mezowie oni, którzy puszczali slawe zla o ziemi, sroga plaga przed Panem.
\par 38 Ale Jozue, syn Nunów, i Kaleb, syn Jefunów, zostali zywi z mezów onych, którzy chodzili ku przeszpiegowaniu ziemi.
\par 39 I opowiedzial Mojzesz te slowa wszystkim synom Izraelskim, i plakal lud bardzo.
\par 40 Tedy rano wstawszy wstapili na wierzch góry, mówiac: Oto my pójdziemy na to miejsce, o którem nam Pan powiedzial; bosmy zgrzeszyli.
\par 41 Ale im powiedzial Mojzesz: Przeczze wy przestepujecie slowo Panskie? to sie wam nie nada.
\par 42 Nie chodzcie; bo nie masz Pana miedzy wami, abyscie nie byli pobici od nieprzyjaciól waszych.
\par 43 Bo Amalekita i Chananejczyk tuz przed wami sa, i polezecie od miecza; bo dla tego, zescie sie odwrócili od Pana, nie bedzie Pan z wami.
\par 44 A oni przecie kusili sie wnijsc na wierzch góry; lecz skrzynia przymierza Panskiego i Mojzesz nie odchodzili od obozu.
\par 45 Tedy zstapil Amalekita i Chananejczyk, mieszkajacy na onej górze, a porazili je, i gonili je az do Hormy.

\chapter{15}

\par 1 I rzekl Pan do Mojzesza, mówiac:
\par 2 Powiedz synom Izraelskim, a mów do nich: Gdy przyjdziecie do ziemi mieszkania waszego, która Ja wam dam,
\par 3 A bedziecie chcieli czynic ofiare ognista Panu na calopalenie, albo ofiare, badz poslubiona badz dobrowolna, albo tez na uroczyste swieta wasze, czyniac wdzieczna wonnosc Panu z wolów albo z owiec:
\par 4 Tedy, ktobykolwiek ofiarowal ofiare swoje Panu, niechze ofiaruje ofiare sniedna, pszennej maki dziesiata czesc, zagniecionej z oliwa, której bedzie czwarta czesc hynu.
\par 5 Przytem wina na ofiare mokra czwarta czesc hynu ofiarowac bedziesz przy calopaleniu, albo przy ofierze innej do kazdego baranka.
\par 6 Przy baranie tez ofiarowac bedziesz ofiare sniedna, maki pszennej dwie dziesiate czesci, zaczynionej z oliwa z trzecia czescia hynu.
\par 7 Wina takze na ofiare mokra trzecia czesc hynu ofiarowac bedziesz na wdzieczna wonnosc Panu.
\par 8 Jezli zas cielca ofiarowac bedziesz na ofiare calopalenia, albo na ofiare wypelnienia slubu, albo na ofiare spokojna Panu,
\par 9 Tedy bedziesz ofiarowal spolem z cielcem ofiare sniedna, pszennej maki trzy dziesiate czesci, zagniecionej z oliwa z polowa hynu.
\par 10 Wina takze bedziesz ofiarowal na ofiare mokra polowe hynu, na ofiare ognista ku wdziecznej wonnosci Panu.
\par 11 Takze uczynisz przy kazdym wole, i przy kazdym baranie i baranku, badz z owiec badz z kóz.
\par 12 Wedlug liczby, która ofiarowac bedziecie, tak uczynicie przy kazdym z nich wedlug liczby ich.
\par 13 Kazdy w domu zrodzony tak tez bedzie czynil, gdy bedzie oddawal ofiare ognista na wdzieczna wonnosc Panu.
\par 14 A gdyby, gosciem bedac miedzy wami przychodzien, albo mieszkajacy miedzy wami w narodziech waszych, ofiarowal ognista ofiare ku wdziecznej wonnosci Panu, jako wy czynicie, tak i on uczyni.
\par 15 O ludu mój! Ustawa jedna niechaj bedzie tak wam, jako i przychodniowi, mieszkajacemu z wami; ustawa to jest wieczna w narodziech waszych; jako wy, tak przychodzien bedzie przed Panem.
\par 16 Prawo jedno, i jeden sad bedzie wam i przychodniowi mieszkajacemu z wami.
\par 17 I rzekl Pan do Mojzesza, mówiac:
\par 18 Powiedz synom Izraelskim, a rzecz do nich: Gdy wnijdziecie do ziemi, do której Ja was wprowadze:
\par 19 A jesc bedziecie chleb onej ziemi, ofiarowac bedziecie ofiare podnoszenia Panu.
\par 20 Z pierwszych ciast waszych placek ofiarowac bedziecie na ofiare podnoszenia; jako ofiare z bojewiska, tak go ofiarowac bedziecie.
\par 21 Z pierwszych ciast waszych ofiarowac bedziecie Panu ofiare podnoszenia w narodziech waszych.
\par 22 A gdybyscie pobladzili, i nie uczynilibyscie wszystkich przykazan tych, które rozkazal Pan przez Mojzesza;
\par 23 Wszystkiego, co wam Pan rozkazal przez Mojzesza, od onego dnia, którego wydal Pan przykazanie, i potem w narodziech waszych:
\par 24 Tedy jezliby z niewiadomosci zgromadzenia stalo sie to pobladzenie, wszystko zgromadzenie ofiarowac bedzie cielca mlodego jednego na calopalenie, na wdzieczna wonnosc Panu, takze sniedna ofiare jego, i mokra ofiare jego wedlug zwyczaju, i kozla jednego z stada za grzech.
\par 25 Tak oczysci kaplan wszystko zgromadzenie synów Izraelskich, i bedzie im odpuszczona, gdyz sie z niewiadomosci stalo. A oni ofiarowac beda ofiare swoje na ofiare ognista Panu, i na ofiare za grzech swój przed Panem za obladzenie swoje.
\par 26 I bedzie odpuszczono wszystkiemu zgromadzeniu synów Izraelskich, i przychodniowi, który mieszka miedzy nimi, poniewaz wszystkiego ludu pobladzenie jest.
\par 27 A jezliby kto sam tylko zgrzeszyl z niewiadomosci, tedy przyniesie Panu koze roczna na ofiare za grzech;
\par 28 I oczysci kaplan czlowieka obladzonego, któryby zgrzeszyl z niewiadomosci; przed Panem oczysci go, i bedzie mu odpuszczono.
\par 29 Tak zrodzonemu miedzy synami Izraelskimi, jako przychodniowi, który mieszka miedzy nimi, zakon jeden bedzie wam, gdyby kto zgrzeszyl z niewiadomosci.
\par 30 Ale czlowiek, któryby z hardosci swawolnie zgrzeszyl, tak urodzony w domu, jako i przychodzien, takowy Pana zelzyl; przetoz wykorzeniony bedzie on czlowiek z posrodku ludu swego.
\par 31 Albowiem slowem Panskiem pogardzil, i przykazanie jego zgwalcil; koniecznie wytracony bedzie takowy czlowiek; nieprawosc jego na nim zostanie.
\par 32 I stalo sie, gdy byli synowie Izraelscy na puszczy, ze znalezli czlowieka zbierajacego drwa w dzien sabatu.
\par 33 I przywiedli go, którzy go znalezli zbierajacego drwa, przed Mojzesza, i przed Aarona, i przed wszystko zgromadzenie.
\par 34 I dali go do wiezienia; bo jeszcze im nie bylo oznajmiono, coby miano czynic z takowym.
\par 35 Tedy rzekl Pan do Mojzesza: smiercia niech umrze czlowiek ten; bez litosci niechaj go ukamionuje wszystko zgromadzenie za obozem.
\par 36 I wywiedli go wszystko zgromadzenie za obóz, i ciskali nan kamieniem, az umarl, jako rozkazal Pan Mojzeszowi.
\par 37 Zatem rzekl Pan do Mojzesza, mówiac:
\par 38 Mów do synów Izraelskich, a powiedz im, aby sobie poczynili bramy na krajach szat swoich w narodziech swych, a niech przyprawia do bram sznurek hijacyntowy.
\par 39 I bedziecie mieli te bramy, zebyscie pogladajac na nie, wspominali sobie na wszystkie przykazania Panskie, abyscie je czynili, i abyscie sie nie unosili za sercem waszem, i za oczyma waszemi, za któremi wy idac cudzolozylibyscie.
\par 40 Ale zebyscie pamietali i czynili wszystkie przykazania moje, a byli swietymi Bogu waszemu.
\par 41 Ja Pan, Bóg wasz, którym was wywiódl z ziemi Egipskiej, abym wam byl za Boga; Jam Pan, Bóg wasz.

\chapter{16}

\par 1 Tedy sie zbuntowal Kore, syn Izaara, syna Kaatowego, syna Lewiego, takze Datan i Abiron, synowie Elijabowi, i Hon, syn Faletów z synów Rubenowych.
\par 2 I powstali przeciw Mojzeszowi, a z nimi mezów z synów Izraelskich dwiescie i piecdziesiat, ksiazeta miedzy ludem, których do rady przyzywano, ludzie zacni.
\par 3 Ci zebrawszy sie przeciw Mojzeszowi, i przeciw Aaronowi, rzekli im: Wiele to na was, poniewaz wszystek ten lud, wszyscy sa swieci, a w posrodku nich jest Pan; przeczze sie wy wynosicie nad zgromadzeniem Panskiem?
\par 4 Co gdy uslyszal Mojzesz, upadl na oblicze swoje,
\par 5 I rzekl do Korego i do wszystkiej roty jego, mówiac: Rano pokaze Pan, kto jest jego, i kto jest swiety, i kto ma przystepowac przeden; bo kogo sobie wybral, temu rozkaze przystapic do siebie.
\par 6 To tedy uczynicie: Wezmiecie sobie kadzielnice, ty Kore, i wszystka rota twoja.
\par 7 I nakladlszy w nie ognia, wlózcie nan kadzidla przed Panem jutro; i stanie sie, ze kogokolwiek obierze Pan, ten bedzie swietym: wiele to na was synowie Lewiego.
\par 8 Nad to rzekl Mojzesz do Korego: Sluchajcie prosze synowie Lewiego;
\par 9 Izali wam to malo, ze was oddzielil Bóg Izraelski od zgromadzenia Izraelskiego, abyscie przystepowali do niego ku odprawowaniu uslugi w przybytku Panskim, a izbyscie stali przed zgromadzeniem, i sluzyli mu?
\par 10 I przyjal ciebie, i wszystke bracia twoje, syny Lewiego z toba, ze jeszcze szukacie kaplanstwa?
\par 11 Dla tegoz, ty i wszystka rota twoja, zbuntowaliscie sie przeciw Panu; bo Aaron cóz jest, zescie szemrali przeciw niemu?
\par 12 Tedy poslal Mojzesz, aby zawolano Datana, i Abirona, synów Elijabowych, którzy odpowiedzieli: Nie pójdziemy.
\par 13 Izali malo na tem, zes nas wywiódl z ziemi oplywajacej mlekiem i miodem, abys nas pomorzyl na tej puszczy, ze jeszcze chcesz miec nad nami zwierzchnosc i nam rozkazowac?
\par 14 Ku temu do ziemi oplywajacej mlekiem i miodem nie wprowadziles nas, anis nam dal w dziedzictwo pól i winnic: izali oczy tym mezom wylupic chcesz? Nie pójdziemy
\par 15 Tedy sie rozgniewal Mojzesz bardzo, i rzekl do Pana: Nie patrz na ofiare ich; ni jednego osla nie wzialem od nich, anim co zlego komu z nich uczynil.
\par 16 Potem rzekl Mojzesz do Korego: Ty, i wszystka rota twoja, stawcie sie przed Pana jutro, ty, i oni, i Aaron:
\par 17 A wziawszy kazdy kadzielnice swoje, wlózcie w nie kadzidla, i stawcie sie przed Pana, kazdy z kadzielnica swoja, dwiescie i piecdziesiat kadzielnic, i ty, i Aaron, kazdy z kadzielnica swoja.
\par 18 Wzial tedy kazdy kadzielnice swoje, a wlozywszy w nia ognia nakladli w nia kadzidla; i staneli u drzwi namiotu zgromadzenia Mojzesz i Aaron.
\par 19 Ale Kore juz byl zebral przeciwko nim wszystke rote do drzwi namiotu zgromadzenia; tedy chwala Panska ukazala sie wszystkiemu ludowi.
\par 20 I rzekl Pan do Mojzesza i do Aarona, mówiac:
\par 21 Odlaczcie sie z posrodku zebrania tego, abym je nagle zatracil.
\par 22 Lecz oni upadli na oblicza swe i rzekli: O Boze, Boze Duchów, i wszelkiego ciala! czlowiek jeden zgrzeszyl, a na wszystekze lud gniewac sie bedziesz?
\par 23 Tedy rzekl Pan do Mojzesza, mówiac:
\par 24 Rzecz do zgromadzenia, a mów: Odstapcie od namiotu Korego, Datana i Abirona.
\par 25 A wstawszy Mojzesz, szedl do Datana i Abirona; i szli za nim starsi Izraelscy.
\par 26 I rzekl do zgromadzenia, mówiac: Odstapcie, prosze, od namiotów mezów tych niepoboznych, ani sie dotykajcie wszystkiego, co ich jest, byscie snac nie zgineli we wszystkich grzechach ich.
\par 27 I odstapili od namiotu Korego, Datana i Abirona zewszad. Ale Datan i Abiron wyszedlszy stali u drzwi namiotów swoich, i zony ich, i synowie ich, i maluczcy ich.
\par 28 Tedy rzekl Mojzesz: Po tem poznacie, ze mie Pan poslal, abym czynil te wszystkie sprawy, a ze nic z domyslu serca swego nie czynie;
\par 29 Jezlize tak, jako inni ludzie umieraja, pomra ci, a zwyklem innych ludzi karaniem, karani beda, nie poslal mie Pan;
\par 30 Ale jezliz co nowego uczyni Pan, ze otworzy ziemia usta swe, i pozre je i wszystkiego co maja, i zstapia zywo do piekla, tedy poznacie, ze rozdraznili ci mezowie Pana.
\par 31 I stalo sie, gdy przestal mówic wszystkich tych slów, ze sie rozstapila ziemia pod nimi;
\par 32 A otworzywszy ziemia paszczeke swoje, pozarla je, i domy ich, ze wszystkimi ludzmi, którzy byli przy Korem, i wszystkie majetnosci ich.
\par 33 I zstapili oni ze wszystkiem co mieli, zywo do piekla, i okryla je ziemia, i pogineli z posrodku zgromadzenia.
\par 34 Wszyscy zas Izraelitowie, którzy byli okolo nich, uciekali na krzyk ich, bo mówili: By snac nie pozarla i nas ziemia.
\par 35 Wyszedl takze ogien od Pana, i spalil onych dwiescie i piecdziesiat mezów, którzy ofiarowali kadzenie.
\par 36 Potem rzekl Pan do Mojzesza mówiac:
\par 37 Rzecz do Eleazara, syna Aaronowego, kaplana, aby pozbieral kadzielnice z onego pogorzeliska, a ogien i tam i sam niech rozrzuci; bo sa poswiecone.
\par 38 A z kadzielnic tych, którzy zgrzeszyli przeciwko duszom swym, rozbiwszy je na blachy, niech obije oltarz, bo iz w nich ofiarowali przed Panem, poswiecone sa, przetoz beda na znak synom Izraelskim.
\par 39 Tedy pozbieral Eleazar kaplan one miedziane kadzielnice, w których ofiarowali oni popaleni; i rozbito je na blachy, na obicie oltarza.
\par 40 Na pamiatke synom Izraelskim, aby nie przystepowal zaden obcy, któryby nie byl z nasienia Aaronowego, do odprawowania kadzenia przed Panem, aby mu sie nie stalo jako Koremu, i jako rocie jego, jako mu byl powiedzial Pan przez Mojzesza.
\par 41 I szemralo wszystko zgromadzenie synów Izraelskich nazajutrz przeciwko Mojzeszowi, i przeciwko Aaronowi, mówiac: Wyscie przyczyna smierci ludu Panskiego,
\par 42 I stalo sie, gdy sie zbieral lud przeciw Mojzeszowi, i przeciw Aaronowi, ze spojrzeli ku namiotowi zgromadzenia, a oto, okryl go oblok, i okazala sie chwala Panska.
\par 43 I przyszedl Mojzesz z Aaronem przed namiot zgromadzenia.
\par 44 I rzekl Pan do Mojzesza, mówiac:
\par 45 Wynijdzcie z posrodku zgromadzenia tego, a wytrace je w okamgnieniu; i upadli na oblicza swoje.
\par 46 Potem rzekl Mojzesz do Aarona: Wezmij kadzielnice, a wlóz w nia ognia z oltarza, wlóz tez kadzidlo, a natychmiast idz do zgromadzenia, i oczysc je; boc juz wyszla popedliwosc od twarzy Panskiej, a juz sie zaczelo karanie.
\par 47 Wzial tedy Aaron kadzielnice, jako mu rozkazal Mojzesz, i przybiezal w posrodek zgromadzenia, a oto juz sie byla zaczela plaga w ludzie; i uczyniwszy kadzenie oczyscil lud.
\par 48 I stanal Aaron miedzy umarlymi i miedzy zywymi, a zahamowana jest plaga.
\par 49 A bylo umarlych od onej plagi czternascie tysiecy i siedem set, oprócz onych, którzy pomarli z przyczyny Korego.
\par 50 Zatem sie wrócil Aaron do Mojzesza ku drzwiom namiotu zgromadzenia, gdy plaga zahamowana byla.

\chapter{17}

\par 1 Potem rzekl Pan do Mojzesza, mówiac:
\par 2 Mów do synów Izraelskich, a wezmij od nich po lasce wedlug domów ojców ich, to jest, od wszystkich ksiazat ich wedlug domów ojców ich, dwanascie lasek, a kazdego imie napiszesz na lasce jego;
\par 3 Ale imie Aaronowe napiszesz na lasce Lewiego; bo kazda laska bedzie od kazdego ksiazecia z domu ojców ich.
\par 4 I zostawisz je w namiocie zgromadzenia przed swiadectwem, gdzie sie z wami schodze.
\par 5 I stanie sie, kogo obiore, tego laska zakwitnie; i usmierze przed soba szemrania synów Izraelskich, któremi szemrza przeciwko wam.
\par 6 To gdy Mojzesz mówil do synów Izraelskich, oddali mu wszyscy ksiazeta ich laski swoje, kazdy ksiaze laske z domu ojca swego, dwanascie lasek; a laska Aaronowa byla miedzy laskami ich.
\par 7 I postawil mojzesz laski one przed Panem w namiocie swiadectwa.
\par 8 A gdy nazajutrz przyszedl Mojzesz do namiotu swiadectwa, oto sie zazielenila laska Aaronowa z domu Lewiego, i wypuscila listki i wydala kwiat, i zrodzila dojrzale migdaly.
\par 9 I wyniósl Mojzesz one wszystkie laski od oblicznosci Panskiej do wszystkich synów Izraelskich; które gdy ujrzeli, wzial kazdy laske swa.
\par 10 I rzekl Pan do Mojzesza: Odnies laske Aaronowa przed swiadectwo, aby byla zachowana na znak synom odpornym, a zahamujesz szemranie ich przeciwko mnie, aby nie pomarli.
\par 11 I uczynil Mojzesz; jako mu Pan rozkazal, tak uczynil.
\par 12 I rzekli synowie Izraelscy do Mojzesza, mówiac: Oto umieramy, niszczejemy, wszyscy giniemy;
\par 13 Kazdy, ktokolwiek przystepuje do przybytku Panskiego, umiera; izali do szczetu wytraceni byc mamy?

\chapter{18}

\par 1 Potem rzekl Pan do Aarona: Ty i synowie twoi, i dom ojca twego z toba, poniesiecie nieprawosc swiatnicy. I ty i synowie twoi z toba poniesiecie nieprawosc kaplanstwa waszego.
\par 2 Bracia takze twoje, pokolenie Lewiego, ród ojca twego, przypuscisz do siebie, aby byli przy tobie, i poslugowali tobie; a ty i synowie twoi z toba sluzyc bedziecie przed namiotem swiadectwa.
\par 3 Oni beda przestrzegali rozkazania twego, i pilnowali wszystkiego namiotu; wszakze do naczynia swatnicy, i do oltarza, przystepowac nie beda, aby nie pomarli, i oni i wy.
\par 4 I przylacza sie do ciebie, pilnie strzegac namiotu zgromadzenia w kazdej usludze namiotu; a nikt obcy niechaj sie nie miesza miedzy was.
\par 5 Wy tedy pilnie strzezcie swiatnicy, i uslugi oltarzowej, by sie napotem nie wzruszyl gniew przeciwko synom Izraelskim.
\par 6 Bom oto ja obral bracia wasze Lewity, z posród synów Izraelskich, wam za dar, oddane Panu, aby odprawowali usluge w namiocie zgromadzenia.
\par 7 Ale ty i synowie twoi z toba, przestrzegac bedziecie kaplanstwa waszego przy kazdej usludze oltarzowej, i za zaslona sluzyc bedziecie; urzad kaplanstwa waszego dalem wam darmo; gdyby kto obcy przystapil, umrze.
\par 8 Nadto mówil Pan do Aarona: Otom ja dal tobie pod straz ofiary moje podnoszone; wszystkie rzeczy poswiecone od synów Izraelskich tobiem je dal dla pomazania, i synom twoim prawem wiecznem.
\par 9 To bedzie twoje z rzeczy poswieconych, które nie bywaja palone. Kazda ofiara ich, badz ofiara sniedna ich, badz ofiara za grzech ich, albo ofiara za wystepek ich, cokolwiek mi oddawac beda to bedzie rzecza poswiecona tobie i synom twoim,
\par 10 Na miejscu najswietszem jadac to bedziesz; wszelki mezczyzna bedzie jadl z tego; rzecza poswiecona to bedzie tobie.
\par 11 To tez twoje bedzie, ofiara podnoszenia darów ich ze wszystkiemi ofiarami tam i sam obracania synów Izraelskich; tobiem je dal, i synom twoim, i córkom twoim z toba, prawem wiecznem; kazdy czysty w domu twoim bedzie je jadl.
\par 12 Kazda co najprzedniejsza oliwe, i kazde co najlepsze wino, i zboze, pierwiastki ich, które ofiaruja Panu, tobiem je dal.
\par 13 Pierwociny ze wszystkich rzeczy, które beda w ziemi ich, które przyniosa Panu, twoje beda; kazdy czysty w domu twoim jadac je bedzie.
\par 14 Wszystko, cokolwiek jest poslubione w Izraelu, twoje bedzie.
\par 15 Cokolwiek otwiera zywot wszelkiego ciala, a bywa ofiarowane Panu, tak z ludzi jako z bydla, twoje bedzie; ale pierworodne z ludzi okupisz, takze pierworodne nieczystego bydla okupisz.
\par 16 A okup jego, gdy mu miesiac minie, dasz wedlug szacunku twego piec syklów srebra wedlug sykla swiatnicy; dwadziescia pieniedzy wazy sykiel.
\par 17 Ale pierworodnego od krowy, albo pierworodnego od owcy, albo pierworodnego od kozy, nie dasz na okup; bo poswiecone sa; krew ich wylejesz na oltarz, a tlustosc ich zapalisz na ofiare ognista dla wdziecznej wonnosci Panu.
\par 18 Ale mieso ich twoje bedzie; jako mostek podnoszenia, i jako lopatka prawa, twoje beda.
\par 19 Wszystkie ofiary podnoszenia z rzeczy poswieconych, które przynosic beda synowie Izraelscy Panu, dalem tobie, i synom twym, i córkom twoim z toba, prawem wiecznem; ustawa to nienaruszona, i wieczna przed Panem, tobie i nasieniu twemu z toba.
\par 20 Potem mówil Pan do Aarona: W ziemi ich dziedzictwa miec nie bedziesz, i dzialu nie bedziesz mial miedzy nimi; Jam dzial twój, i dziedzictwo twoje w posród synów Izraelskich.
\par 21 Synom zas Lewiego otom dal wszystke dziesiecine w Izraelu dziedzicznie za sluzbe ich, która wykonywaja sluzac okolo namiotu zgromadzenia.
\par 22 A niechaj nie przystepuja wiecej synowie Izraelscy do namiotu zgromadzenia, aby nie poniesli grzechu i nie pomarli;
\par 23 Ale sami Lewitowie odprawowac beda usluge okolo namiotu zgromadzenia, i sami poniosa nieprawosc swoje. Ustawa to wieczna bedzie w narodziech waszych, aby w posród synów Izraelskich dziedzictwa nie mieli.
\par 24 Albowiem dziesieciny synów Izraelskich które przynosic beda na ofiare podnoszenia Panu, dalem Lewitom za dziedzictwo; przetoz rzeklem do nich: W posród synów Izraelskich nie beda mieli dziedzictwa.
\par 25 Potem rzekl Pan do Mojzesza mówiac:
\par 26 Mów tez do Lewitów, a powiedz im: Gdy wezmiecie od synów Izraelskich dziesieciny, którem ja wam dal od nich za dziedzictwo wasze, tedy ofiarowac bedziecie ofiare podnoszenia Panu dziesiata czesc dziesiecin.
\par 27 A poczytana wam bedzie ta ofiara wasza jako zboze z bojewiska, i jako obfitosc wina z prasy.
\par 28 Tak wy tez ofiarowac bedziecie ofiare podnoszenia Panu ze wszystkich dziesiecin waszych, które wezmiecie od synów Izraelskich, a oddacie z nich ofiare podnoszenia Panu, Aaronowi kaplanowi:
\par 29 Ze wszystkich dochodów waszych ofiarowac bedziecie wszelka ofiare podnoszenia Panu; ze wszystkiego co najlepsze jest, ofiarujecie czastke poswiecona.
\par 30 Powiesz im tez: Gdy oddawac bedziecie z tego, co najlepsze jest, tedy poczytano bedzie Lewitom jako urodzaje z bojewiska; i jako urodzaj z prasy.
\par 31 I bedziecie to jesc na kazdem miejscu, wy i czeladz wasza; albowiem to jest zaplata wasza za sluzbe wasze przy namiocie zgromadzenia;
\par 32 I nie poniesiecie za to grzechu, gdy ofiarowac bedziecie co najlepszego jest z tego, i nie splugawicie rzeczy poswieconych od synów Izraelskich, i nie pomrzecie.

\chapter{19}

\par 1 I rzekl Pan do Mojzesza i do Aarona, mówiac:
\par 2 Tac jest ustawa zakonu, która rozkazal Pan, mówiac: Powiedz synom Izraelskim, aby przywiedli do ciebie jalowice plowa, zdrowa, na której by nie bylo zmazy, i na której by nie postalo jarzmo;
\par 3 Te oddacie Eleazarowi kaplanowi, który ja wywiedzie za obóz, i zabita bedzie przed nim.
\par 4 A wziawszy Eleazar kaplan ze krwi jej na palec swój, kropic bedzie przeciw namiotowi zgromadzenia ona krwia siedem kroc.
\par 5 Potem kaze te jalowice spalic przed oczyma swemi; skóre jej, i mieso jej, i krew jej, z gnojem jej spali.
\par 6 I wezmie kaplan drzewo cedrowe, i hizop, i karmazyn dwa kroc farbowany, a wrzuci do ognia, gdzie sie jalowica pali;
\par 7 I upierze szaty swe kaplan, a omyje cialo swoje woda; a potem wnijdzie do obozu, i bedzie nieczystym kaplan az do wieczora.
\par 8 Ten tez, który ja palic bedzie, upierze szaty swe w wodzie, i omyje cialo swe woda, a bedzie nieczystym az do wieczora.
\par 9 I zbierze czlowiek czysty popiól onej jalowicy, i wysypie go precz za obóz, na miejscu czystem, a bedzie dla zgromadzenia synów Izraelskich chowany do wody oczyszczenia, gdyz jest ofiara za grzech.
\par 10 I upierze ten, co bedzie zbieral popiól onej jalowicy, szaty swe, i bedzie nieczystym az do wieczora. A bedzie to synom Izraelskim, i przychodniowi mieszkajacemu miedzy nimi, ustawa wieczna.
\par 11 Kto by sie dotknal jakiegokolwiek trupa czlowieczego, nieczystym bedzie przez siedem dni;
\par 12 Taki oczyszczac sie bedzie ta woda dnia trzeciego i dnia siódmego, a czystym bedzie; a jesliby sie nie oczyscil dnia trzeciego i dnia siódmego, nieczystym bedzie.
\par 13 Kto by sie kolwiek dotknal martwego ciala czlowieka, który umarl, a nie oczyscil sie, przybytek Panski splugawil; przetoz takowy wytracony bedzie z Izraela, bo woda oczyszczenia nie byl pokropiony; nieczystym bedzie, nieczystosc jego zostanie na nim.
\par 14 Ta zas jest ustawa: gdyby czlowiek umarl w namiocie, ktobykolwiek wszedl do tego namiotu, i cokolwiek bylo w onym namiocie, nieczystym bedzie przez siedem dni.
\par 15 Takze wszelkie naczynie odkryte, które by nie mialo nakrycia na sobie, nieczyste bedzie.
\par 16 Takze kto by sie kolwiek dotknal na polu, badz mieczem zabitego, badz umarlego, badz kosci czlowieczej, albo grobu, nieczystym bedzie przez siedem dni.
\par 17 Wezma tedy dla onego nieczystego popiolu jalowicy spalonej za grzech, i naleja nan wody zywej do naczynia.
\par 18 Wezmie tez hizopu, i omoczy go w onej wodzie czlowiek czysty, i pokropi namiot, i wszystko naczynie, i wszystkich ludzi, którzy by tam byli, takze onego, który sie dotknal kosci, albo zabitego, albo zmarlego, albo grobu;
\par 19 Pokropi tedy czysty nieczystego dnia trzeciego i dnia siódmego; a gdy go oczysci dnia siódmego, tedy upierze szaty swe, i omyje sie woda, a bedzie czystym w wieczór.
\par 20 A maz, któryby nieczystym bedac nie oczyscil sie, wykorzeniona bedzie ta dusza z posrodku zgromadzenia, bo swiatnice Panska splugawil; woda oczyszczenia nie bedac pokropionym, nieczystym jest.
\par 21 I bedzie im to za ustawe wieczna; a kto bedzie pokrapial woda oczyszczenia, upierze szaty swoje; a kto by sie dotknal tej wody oczyszczenia, nieczystym bedzie az do wieczora.
\par 22 Czegokolwiek sie dotknie nieczysty, nieczyste bedzie; czlowiek takze, któryby sie tego dotknal nieczysty bedzie az do wieczora.

\chapter{20}

\par 1 I przyszlo wszystko mnóstwo synów Izraelskich na puszcza Syn, miesiaca pierwszego; i mieszkal lud w Kades; gdzie umarla Maryja, i tamze jest pogrzebiona.
\par 2 A gdy lud nie mial wody, zebrali sie przeciw Mojzeszowi, i przeciw Aaronowi.
\par 3 I swarzyl sie lud z Mojzeszem, i rzekli mówiac: Obysmy byli pomarli, gdy pomarli bracia nasi przed Panem.
\par 4 I przeczzescie zawiedli to zgromadzenie Panskie na te puszcze, abysmy tu pomarli, my i dobytki nasze?
\par 5 A po cózescie nas wywiedli z Egiptu, abyscie nas wprowadzili na to zle miejsce, na którem sie nie rodzi ani zboze, ani figi, ani grona winne, ani jablka granatowe; nawet wody nie masz dla napoju?
\par 6 Tedy odszedl Mojzesz i Aaron od ludu do drzwi namiotu zgromadzenia, i upadli na oblicza swoje; i ukazala sie chwala Panska nad nimi.
\par 7 I rzekl Pan do Mojzesza mówiac:
\par 8 Wezmij laske, a zgromadziwszy wszystek lud, ty i Aaron, brat twój, mówcie do tej skaly przed oczyma ich, a wyda wode swa; i wywiedziesz im wode z skaly, a dasz napój temu mnóstwu, i bydlu ich.
\par 9 Tedy wzial Mojzesz laske przed obliczem Panskiem, jako mu rozkazal.
\par 10 I zgromadzil Mojzesz z Aaronem wszystek lud przed skale, i mówil do nich: Sluchajciez teraz ludzie odporni, izali z tej skaly mozemy wam wywiesc wode?
\par 11 Zatem podniósl Mojzesz reke swoje, i uderzyl w skale laska swa dwa kroc, a wyszly wody obfite i pilo ono mnóstwo, i bydlo ich.
\par 12 I rzekl Pan do Mojzesza i do Aarona: Dlatego, zescie mi nie uwierzyli, abyscie mie poswiecili przed oczyma synów Izraelskich, przetoz nie wprowadzicie zgromadzenia tego do ziemi, któram im dal.
\par 13 Tec sa wody poswarku, o które sie swarzyli synowie Izraelscy z Panem, i poswiecony jest w nich.
\par 14 Potem poslal Mojzesz posly z Kades do króla Edomskiego, mówiac: Tak ci kazal powiedziec brat twój Izrael: Ty wiesz o wszystkich trudnosciach, które przyszly na nas;
\par 15 Jako zstapili byli ojcowie nasi do Egiptu, i mieszkalismy w Egipcie przez wiele lat; i jako nas trapili Egipczanie, i ojce nasze;
\par 16 I wolalismy do Pana, a wysluchal glos nasz, i poslawszy Aniola, wywiódl nas z Egiptu; a otosmy juz w Kades, miescie przy granicy twojej
\par 17 Niech, prosze, przejdziemy przez ziemie twoje; nie pójdziemy przez pola, ani przez winnice, ani bedziemy pic wód z twoich studzien; goscincem pójdziemy, nie uchylimy sie na prawo ani na lewo, az przejdziemy granice twoje.
\par 18 Na to odpowiedzial mu Edom: Nie pójdziesz przez moje ziemie, bym snac z mieczem nie wyszedl przeciw tobie.
\par 19 I rzekli mu synowie Izraelscy: Bitym goscincem pójdziemy, a jeslibysmy wody twoje pili, my i bydla nasze, zaplacimyc je; nic innego nie zadamy, tylko abysmy pieszo przeszli.
\par 20 I powiedzial: Nie przejdziesz. I ruszyl sie Edom przeciwko nim, z wojskiem wielkiem, i mozna reka.
\par 21 A gdy nie chcial Edom pozwolic Izraelowi przejscia przez granice swoje, tedy Izrael uczynil odwrót od niego.
\par 22 A ruszywszy sie z Kades; przyszli synowie Izraelscy i wszystko zgromadzenie do góry Hor.
\par 23 I rzekl Pan do Mojzesza i do Aarona na górze Hor, nad granica ziemi Edomskiej, mówiac:
\par 24 Bedzie Aaron przylaczon do ludu swego; albowiem nie wnijdzie do ziemi, któram dal synom Izraelskim, przeto zescie odporni byli slowu mojemu przy wodach poswarku.
\par 25 Wezmijze Aarona i Eleazara syna jego, a kaz im wstapic na góre Hor;
\par 26 I zewlecz Aarona z szat jego, a oblecz w nie Eleazara, syna jego; bo Aaron przylaczon bedzie do ludu swego, i tam umrze.
\par 27 I uczynil Mojzesz, jako rozkazal Pan; i wstapili na góre Hor przed oczyma wszystkiego zgromadzenia.
\par 28 I zewlókl Mojzesz Aarona z szat jego, a oblekl w nie Eleazara syna jego; i umarl tam Aaron na wierzchu góry, a Mojzesz z Eleazarem zstapili z góry.
\par 29 Widzac tedy wszystko zgromadzenie, iz Aaron umarl, plakali Aarona przez trzydziesci dni, wszystek dom Izraelski.

\chapter{21}

\par 1 A gdy uslyszal Chananejczyk, król Harat, który mieszkal na poludnie, ze Izraelczycy ciagneli ona droga, która byli szpiegowie przeszli, tedy zwiódl bitwe z Izraelem, i pojmal ich wiele.
\par 2 Tam uczynil Izrael slub Panu, mówiac: Jezlize podaz lud ten w rece moje, do gruntu wywróce miasta ich.
\par 3 I wysluchal Pan glos Izraela, a podal mu Chananejczyki: i wytracil je z gruntu, i miasta ich, a nazwal imie onego miejsca Chorma.
\par 4 Potem ruszyli sie od góry Hor droga ku morzu czerwonemu, aby obeszli ziemie Edomska; i utrudzil sie lud bardzo w onej drodze.
\par 5 Przetoz mówil lud przeciw Bogu, i przeciw Mojzeszowi: Przeczzescie nas wywiedli z Egiptu, aby my pomarli na tej puszczy? bo nie masz chleba, ani wody, a dusza nasza obrzydzila sobie ten chleb nikczemny.
\par 6 Przetoz przypuscil Pan na lud weze ogniste, którzy kasali lud; i pomarlo wiele ludu z Izraela.
\par 7 I przyszedlszy lud do Mojzesza, rzekli: Zgrzeszylismy, zesmy mówili przeciw Panu, i przeciw tobie. Módl sie Panu, aby oddalil od nas te weze; i modlil sie Mojzesz za ludem.
\par 8 I rzekl Pan do Mojzesza: uczyn sobie weza miedzianego, a wystaw go na drzewcu; i stanie sie, ktokolwiek ukaszony bedac wejrzy nan, ze zyw zostanie.
\par 9 Sprawil tedy Mojzesz weza miedzianego, i wystawil go na drzewcu; i bylo to, gdy kogo waz ukasil, a spojrzal na weza miedzianego, ze zyw zostal.
\par 10 Zatem ruszyli sie synowie Izraelscy, a staneli obozem w Obot.
\par 11 A z Obot ruszywszy sie polozyli sie obozem na pagórkach gór Habarym na puszczy, która jest przeciw Moabczykom od wschodu slonca.
\par 12 A odszedlszy stamtad polozyli sie obozem nad potokiem Zered.
\par 13 Stamtad odciagnawszy polozyli sie obozem u brodu Arnon, który jest na puszczy, a wychodzi z granicy Amorejskiej: albowiem Arnon jest granica Moabska miedzy Moabczykiem i Amorejczykiem.
\par 14 Przetoz mówi sie w ksiegach wojen Panskich: Przeciwko Wahebowi w wichrze walczyl, i przy potokach Arnon.
\par 15 Bo sciekanie tych potoków, które sie nachylilo (toczy) ku polozeniu Har, to sie sciaga ku granicy Moabskiej.
\par 16 Stamtad potem przyszli do Beer; a tac jest ona studnia, o której mówil Pan do Mojzesza: Zgromadz lud a dam im wody.
\par 17 Tedy spiewal Izrael te piosnke: Wystap studnio; spiewajciez o niej;
\par 18 Studnia, która wykopali ksiazeta, wykopali ja hetmani ludu z ustawca zakonu, laskami swojemi. A z tej puszczy ruszyli sie do Matana;
\par 19 A z Matana do Nahalijelu, a z Nahalijelu do Bamotu;
\par 20 A z Bamotu ku Hagaj, które jest w polach Moabskich, na wierzchu pagórka, który lezy ku puszczy.
\par 21 I poslal Izrael posly do Sehona, króla Amorejskiego, mówiac:
\par 22 Niech przejdziemy przez ziemie twoje; nie pójdziemy ani przez pola ani przez winnice; nie bedziemy pic wód z studzien twoich; goscincem pójdziemy, az przejdziemy granice twoje.
\par 23 Ale nie pozwolil Sehon Izraelowi isc przez granice swoje; i zebrawszy Sehon wszystek lud swój, wyciagnal przeciw Izraelowi na puszcze, a gdy przyszedl do Jahazy, zwiódl bitwe z Izraelem.
\par 24 I porazil go Izrael ostrzem miecza, i odziedziczyl ziemie jego od Arnonu az do Jaboku, i az do ziemi synów Ammonowych; albowiem opatrzone byly granice Ammonitów.
\par 25 Tedy pobral Izrael wszystkie miasta one, i mieszkal we wszystkich miastach Amorejskich, w Hesebon, i we wszystkich wsiach jego.
\par 26 Bo Hesebon bylo miasto Sehona, króla Amorejskiego, który, gdy pierwej walczyl z królem Moabskim, wzial mu byl wszystke ziemie jego z rak jego az po Arnon.
\par 27 Dla tegoz mówia w przypowiesci: Pójdzcie do Hesebon, a niech zbuduja i naprawia miasto Sechonowe.
\par 28 Albowiem wyszedl ogien z Hesebon, a plomien z miasta Sechonowego, i popalil Ar Moabskie, i obywatele wysokich miejsc Arnon.
\par 29 Biada tobie Moab, zginales o ludu Chamos! podal syny swoje na uciekanie, i córki swoje do wiezienia królowi Amorejskiemu Sehonowi.
\par 30 A zaginelo panowanie ich od Hesebona az do Dybona; a poburzylismy je az do Nofe, które idzie az do Medady.
\par 31 I mieszkal Izrael w ziemi Amorejskiej.
\par 32 Tedy poslal Mojzesz na szpiegi do Jazer, którego wsi pobrali, wypedziwszy Amorejczyki, którzy tam byli.
\par 33 Potem obróciwszy sie szli ku Basan; gdzie wyciagnal Og, król Basanski, przeciwko nim, sam i wszystek lud jego, aby z nimi stoczyl bitwe w Edrej.
\par 34 Tedy rzekl Pan do Mojzesza: Nie bój sie go; bo w rece twoje podalem go, i wszystek lud jego, i ziemie jego, i uczynisz mu, jakos uczynil Sehonowi, królowi Amorejskiemu, który mieszkal w Hesebon.
\par 35 I porazili go, i syny jego, ze wszystkim ludem jego, tak iz nikogo z niego nie zostawili, i posiedli dziedzicznie ziemie jego.

\chapter{22}

\par 1 Stamtad ruszyli sie synowie Izraelscy, i polozyli sie obozem na polach Moabskich, z tej strony Jordanu przeciw Jerychowi.
\par 2 A widzac Balak, syn Seforów, wszystko, co uczynil Izrael Amorejczykowi,
\par 3 Ulakl sie Moab tego ludu wielce, bo go bylo wiele; i zatrwozyl soba Moab dla synów Izraelskich.
\par 4 Przetoz rzekl Moab do starszych Madyjanskich: Teraz pozre to mnóstwo wszystko, co jest okolo nas, jako wól pozera trawe polna. A Balak, syn Seforów, byl królem Moabskim na on czas.
\par 5 I poslal posly do Balaama, syna Beorowego, do Pethoru miasta, które jest nad rzeka ziemi synów ludu jego, aby go wezwano, mówiac: Oto lud wyszedl z Egiptu, oto okryl wierzch ziemi, i osadza sie przeciwko mnie.
\par 6 Przetoz teraz przyjdz prosze, a przeklinaj mnie kwoli lud ten, bo mozniejszy jest nad mie; owa snac go bede mógl porazic, i wygnac go z ziemi; bo ja wiem, ze komu blogoslawisz, blogoslawiony bedzie; a kogo przeklinasz, przeklety bedzie.
\par 7 Poszli tedy starsi Moabscy, i starsi Madyjanscy, majac zaplate za wieszczbe w rekach swych.
\par 8 A przyszedlszy do Balaama, powiedzieli mu slowa Balakowe. I rzekl do nich: Zostancie tu przez noc, a dam wam odpowiedz, jako mi oznajmi Pan.
\par 9 I zostaly ksiazeta Moabskie z Balaamem. Tedy przyszedl Bóg do Balaama, i rzekl: Cóz to za mezowie u ciebie?
\par 10 I odpowiedzial Balaam Bogu: Balak, syn Seforów, król Moabski, poslal do mnie, mówiac:
\par 11 Oto lud, który wyszedl z Egiptu, i okryl wierzch ziemi; terazze pójdz, przeklinaj mi go; snac bede mógl walczyc z nim, i wypedze go.
\par 12 Tedy rzekl Bóg do Balaama: Nie chodz z nimi, ani przeklinaj ludu tego; bo jest blogoslawiony.
\par 13 A wstawszy rano Balaam rzekl do ksiazat Balakowych: Wróccie sie do ziemi waszej; bo mi nie pozwala Pan puscic sie w droge z wami.
\par 14 Wstawszy tedy ksiazeta Moabskie, wrócili sie do Balaka, i powiedzieli: Nie chcial Balaam isc z nami.
\par 15 Tedy po wtóre poslal Balak wiecej ksiazat, i zacniejszych nad pierwsze.
\par 16 Którzy przyszedlszy do Balaama, mówili mu: Tak mówi Balak, syn Seforów: Nie ociagaj sie prosze przyjsc do mnie;
\par 17 Albowiem ci wielka uczciwosc wyrzadze, i wszystko, cobys mi rzekl, uczynie; tylko przyjdz prosze a przeklinaj mi ten lud.
\par 18 Tedy odpowiedzial Balaam, i rzekl do slug Balakowych: Chocby mi dal Balak pelen dom swój srebra i zlota, nie móglbym przestapic slów Pana, Boga mego, i uczynic przeciwko niemu, badz malo badz wiele;
\par 19 Ale prosze zostancie tu i wy przez noc, a dowiem sie, co jeszcze Pan bedzie mówil ze mna.
\par 20 Tedy przyszedl Bóg do Balaama w nocy, i rzekl do niego: Jezliz, aby cie wezwali, przyszli mezowie ci, wstanze, idz z nimi; a wszakze, coc rozkaze, to uczynisz.
\par 21 Tedy wstawszy Balaam rano, osiodlal oslice swoja, i jechal z ksiazety Moabskimi.
\par 22 I rozpalil sie gniew Bozy, ze on jechal; i stanal Aniol Panski na drodze, aby mu zastapil; a on jechal na oslicy swojej, i dwoje pacholat jego z nim.
\par 23 A gdy ujrzala oslica Aniola Panskiego, stojacego na drodze, a miecz jego dobyty w rece jego, tedy ustapila oslica z drogi a szla na role, lecz bil Balaam oslice, aby ja nawiódl na droge.
\par 24 Tedy stanal Aniol Panski na sciezce u winnicy miedzy dwoma ploty.
\par 25 A widzac oslica Aniola Panskiego, przyciskala sie do plota, tak iz przyparla noge Balaamowa do sciany; a on znowu ja bil.
\par 26 Potem Aniol Panski szedl dalej, i stanal na miejscu ciasnem, gdzie nie bylo drogi do ustapienia na prawo ani na lewo;
\par 27 A widzac oslica Aniola Panskiego, padla pod Balaamem; i rozgniewal sie Balaam wielce, a bil oslice kijem.
\par 28 Zatem otworzyl Pan usta onej oslicy, i rzekla do Balaama: Cózem ci uczynila, zes mie bil juz po trzy kroc?
\par 29 I rzekl Balaam do oslicy: Iz ze mnie szydzisz; bym byl mial miecz w reku swych, bylbym cie teraz zabil.
\par 30 Tedy oslica rzekla do Balaama: Azazem ja nie oslica twoja, na którejs jezdzal, jakos mie dostal, az do tego czasu? A on rzekl: Nigdy.
\par 31 Wtem otworzyl Pan oczy Balaamowe, ze obaczyl Aniola Panskiego, stojacego na drodze, i miecz jego dobyty w rece jego; tedy skloniwszy sie, poklonil sie twarza swoja.
\par 32 I rzekl do niego Aniol Panski: Przeczzes bil oslice swoje juz po trzy kroc? Otom Ja wyszedl, abym sie tobie sprzeciwil; bo przewrotna jest droga twoja przede mna;
\par 33 A widzac mie oslica ustapila przede mna po trzy kroc, a gdyby byla nie ustapila przede mna, juz bym cie byl teraz zabil a one bym byl zywa zostawil.
\par 34 Zatem rzekl Balaam do Aniola Panskiego: Zgrzeszylem, albowiem nie widzialem, zes ty stanal przeciwko mnie na drodze; przetoz teraz, jezlic sie to nie podoba, wróce sie.
\par 35 Lecz Aniol Panski rzekl do Balaama: Jedz z ludzmi tymi; wszakze tylko, co Ja tobie powiem, mówic bedziesz. I jechal Balaam z ksiazety Balakowymi.
\par 36 A gdy uslyszal Balak, iz przyjezdza Balaam, wyjechal przeciwko niemu, do niektórego miasta Moabskiego, które jest na granicy Arnonu, które jest przy koncu granicy.
\par 37 Tedy rzekl Balak do Balaama: Azazem z pilnoscia nie posylal do ciebie wzywajac cie? Czemuzes nie przyjechal do mnie? azaz cie zacnie uczcic nie moge?
\par 38 I rzekl Balaam do Balaka: Otom przyjechal do ciebie; izali teraz, chocbym chcial, bede mógl co mówic? slowo, które wlozy Bóg w usta moje, mówic bede.
\par 39 Tedy jechal Balaam z Balakiem a przyjechali do miasta Husot.
\par 40 A tak Balak dal nabic wolów i owiec, i poslal do Balaama, i do ksiazat, którzy z nim byli.
\par 41 I stalo sie nazajutrz, ze wzial Balak Balaama, i wprowadzil go na wyzyny Baalowe, skad widzial i najdalsza czesc ludu.

\chapter{23}

\par 1 I rzekl Balaam do Balaka: Zbuduj mi tu siedem oltarzów, a nagotuj mi tu siedem cielców, i siedem baranów;
\par 2 Uczynil tedy Balak, jako mówil Balaam, i ofiarowal Balak z Balaamem cielca, i barana na kazdym oltarzu.
\par 3 Potem rzekl Balaam do Balaka: Stan przy calopaleniu twojem, a ja odejde; owa sie snac spotka Pan ze mna, a cokolwiek mi objawi, powiem ci; i odszedl sam.
\par 4 I spotkal sie Pan z Balaamem; i rzekl mu Balaam: Postawilem siedem oltarzów, i ofiarowalem cielca i barana na kazdym oltarzu.
\par 5 Tedy Pan wlozyl slowa w usta Balaamowe, i rzekl: Wróc sie do Balaka, a mów tak.
\par 6 I wrócil sie do niego, a oto stal u ofiary swojej palonej, on i wszystkie ksiazeta Moabskie.
\par 7 A tak zaczal przypowiesc swoje, i rzekl: Z Aram przywiódl mie Balak, król Moabski, z gór wschodnich, mówiac: Przyjdz, przeklinaj mi Jakóba, a przyjdz, zlorzecz Izraelowi.
\par 8 Jakoz ja przeklinac mam, kogo Bóg nie przeklina? albo jako zlorzeczyc mam, komu Pan nie zlorzeczy?
\par 9 Bo z wierzchu skal ogladam go, a z pagórków bede nan patrzal; oto, lud ten sam mieszkac bedzie, a miedzy narody mieszac sie nie bedzie.
\par 10 Któz policzy proch Jakóbów, i liczbe czwartej czesci Izraela? Niech umrze dusza moja smiercia sprawiedliwych, i niech bedzie dokonanie moje, jako ich.
\par 11 Tedy rzekl Balak do Balaama: Cózes mi uczynil? Na przeklinanie nieprzyjaciól moich przyzwalem cie, a oto, blogoslawiac blogoslawiles im.
\par 12 A on odpowiedzial i rzekl: Azaz nie mam tego pilnowac i mówic, co Pan wlozyl w usta moje?
\par 13 I rzekl do niego Balak: Pójdz prosze ze mna na miejsce inne, zebys go stamtad widzial; (ale tylko czesc jego ujrzysz, a wszystkiego widziec nie bedziesz;) przeklinajze mi go stamtad.
\par 14 I zawiódl go na pole Sofim, na wierzch góry Fazga, i zbudowal siedem oltarzów, i ofiarowal cielca i barana na kazdym oltarzu.
\par 15 Rzekl tedy Balaam do Balaka: Zostan tu przy calopaleniu twojem, a ja zabieze tam Panu.
\par 16 I zaszedl Pan Balaamowi, który wlozyl slowa w usta jego, i rzekl:
\par 17 Wróc sie do Balaka, a tak mów. Przyszedl tedy do niego, a oto on stal przy calopaleniu swojem, i ksiazeta Moabskie z nim; i rzekl mu Balak: Cóz ci powiedzial Pan?
\par 18 I zaczal rzecz swa temi slowy: Wstan Balaku, a sluchaj: przyjmij w uszy swe slowa moje, synu Seforów.
\par 19 Nie jestci Bóg jako czlowiek, aby klamal, ani jako syn czlowieczy, azeby zalowal; azaz on rzecze, a nie uczyni? wymówi, a nie wypelni?
\par 20 Otom wzial rozkazanie, abym blogoslawil; on blogoslawienstwo dal a ja go nie odwróce.
\par 21 Nie baczy nieprawosci w Jakóbie; ani widzi przestepstwa w Izraelu; Pan, Bóg jego, jest z nim, a trabienie zwyciestwa królewskiego przy nim.
\par 22 Bóg wywiódl je z Egiptu, moca jednorozcowa byl mu.
\par 23 Albowiem nie masz wieszczby przeciw Jakóbowi, ani wrózki przeciw Izraelowi; od tego czasu mówiono bedzie o Jakóbie i o Izraelu, co z nimi uczynil Bóg.
\par 24 Oto lud ten jako lew silny powstanie, jako lwie mlode podniesie sie, az pozre lupy, i krew pobitych wypije.
\par 25 Tedy rzekl Balak do Balaama: Ani ich przeklinaj wiecej, ani im tez blogoslaw wiecej.
\par 26 I odpowiedzial Balaam, a rzekl do Balaka: Azazem ci nie powiadal, mówiac, ze cokolwiek mówic bedzie Pan, to uczynie?
\par 27 I rzekl Balak do Balaama: Pójdz, prosze, zawiode cie na insze miejsce, jezli snac podoba sie Bogu, zebys je stamtad przeklinal.
\par 28 Tedy wiódl Balak Balaama na wierzch góry Fegor, która patrzy ku puszczy.
\par 29 I rzekl Balaam do Balaka: Zbuduj mi siedem oltarzów, a nagotuj mi tu siedem cielców i siedem baranów.
\par 30 I uczynil Balak, jako mu rozkazal Balaam, i ofiarowal cielca i barana na kazdym oltarzu.

\chapter{24}

\par 1 A gdy obaczyl Balaam, ze sie podobalo Panu, aby blogoslawil Izraelowi, juz nie chodzil, jako przedtem, raz i drugi dla wieszczby; ale obrócil ku puszczy twarz swoje.
\par 2 A podnióslszy Balaam oczy swe, obaczyl Izraela mieszkajacego wedlug pokolen swoich, i byl nad nim Duch Bozy.
\par 3 I zaczal przypowiesc swoje, a mówil:
\par 4 Rzekl Balaam, syn Beorów, rzekl maz, którego oczy sa otworzone, rzekl slyszacy wymowy Boze, a który widzenie Wszechmocnego widzial, który, kiedy padnie, otworzone ma oczy:
\par 5 Jako piekne sa namioty twoje, o Jakubie! przybytki twoje, o Izraelu!
\par 6 Jako potoki rozciagnely sie, jako ogrody przy rzece, jako drzewa wonne, które Pan nasadzil, jako cedry nad wodami.
\par 7 Poplynie woda z wiadra jego, a nasienie jego bedzie nad wodami obfitemi, a bedzie wywyzszon nad Agaga król jego, a wyniesie sie królestwo jego.
\par 8 Bóg wywiódl go z Egiptu, moca jednorozcowa byl mu; pozre narody przeciwne sobie, a kosci ich pokruszy, i strzalami swemi przerazi je.
\par 9 Polozyl sie, lezy jako lwie, i jako lew silny; któz go obudzi? kto byc blogoslawil, blogoslawiony, a kto by cie przeklinal, przeklety bedzie.
\par 10 Tedy sie zapalil gniew Balaka na Balaama, a klasnawszy rekami swemi, rzekl Balak do Balaama: Dla zlorzeczenia nieprzyjaciolom moim przyzwalem cie, a oto im blogoslawil juz po trzy kroc.
\par 11 Przetoz teraz uchodz na miejsce swoje; rzeklem ci byl: Zacnie cie uczcze; ale oto pozbawil cie Pan tej czci.
\par 12 I rzekl Balaam do Balaka: Izazem i poslom twoim, któres slal do mnie, nie powiedzial mówiac:
\par 13 Chocby mi dal Balak pelen dom swój srebra i zlota, nie bede mógl przestapic slowa Panskiego, abym czynil co dobrego albo zlego sam z siebie; co mi opowie Pan, to bede mówil.
\par 14 A teraz oto ja odchodze do ludu mego, jednak oznajmiec, co uczyni lud ten ludowi twemu na potem.
\par 15 I zaczal przypowiesc swoje i rzekl: Mówil Balaam, syn Beorów, mówil maz, którego oczy sa otworzone;
\par 16 Mówil ten, który slyszal wyroki Boze, a który ma umiejetnosc Najwyzszego; który widzial widzenie Wszechmocnego; który, kiedy padnie, otworzone ma oczy:
\par 17 Ujrze go, ale nie teraz; ogladam go, ale nie z bliska; wynijdzie gwiazda z Jakuba i powstanie laska z Izraela, i pobije ksiazeta Moabskie, i wytraci wszystkie syny Setowe.
\par 18 I przyjdzie Edom w opanowanie, a Seir bedzie pod wladza nieprzyjaciól swoich, a Izrael bedzie sobie poczynal meznie.
\par 19 I bedzie panowal, który wynijdzie z Jakuba, a wytraci ostatki z miast.
\par 20 A gdy spojrzal na Amaleka, zaczal przypowiesc swoje, i rzekl: Poczatek narodów jest Amalek, a ostatek jego do szczetu zaginie.
\par 21 Potem wejrzawszy na Kenejczyka zaczal przypowiesc swoje i rzekl: mocnec jest mieszkanie twoje, a zalozyles na skale gniazdo twoje
\par 22 Wszakze spustoszony bedzie Kenejczyk, az cie Assur zaprowadzi do wiezienia.
\par 23 Znowu zaczal przypowiesc swoje, i rzekl: Ach! któz bedzie zyw, gdy to uczyni Bóg?
\par 24 Bo okrety przyplyna od brzegów Chyttymskich, i utrapia Assyryjany, utrapia Hebrejczyki; ale tez same do szczetu zagina.
\par 25 Wstal tedy Balaam i odszedl, a wrócil sie na miejsce swoje; takze i Balak poszedl w droge swa.

\chapter{25}

\par 1 Potem gdy mieszkal Izrael w Syttim, poczal lud cudzolozyc z córkami Moabskiemi,
\par 2 Które wzywaly ludu ku ofiarom bogów swoich; a jedzac lud klanial sie bogom ich.
\par 3 I przylaczyl sie Izrael do sluzby Baal Fegora; skad sie rozgniewal Pan bardzo na Izraela.
\par 4 Rzekl tedy Pan do Mojzesza: Zbierz wszystkie ksiazeta z ludu, a kaz im, te przestepce powieszac Panu przed sloncem, aby sie odwrócil gniew popedliwosci Panskiej od Izraela.
\par 5 Przetoz rzekl Mojzesz do sedziów Izraelskich: Zabijcie kazdy z was meze swe, którzy sie spospolitowali z Baal Fegorem.
\par 6 A oto, niektóry z synów Izraelskich przyszedl i przywiódl do braci swej Madyjanitke przed oczyma Mojzeszowemi, i przed oczyma wszystkiego zgromadzenia synów Izraelskich; a oni plakali przed drzwiami namiotu zgromadzenia.
\par 7 Co gdy ujrzal Finees, syn Eleazara, syna Aarona kaplana, wstawszy z posrodku zgromadzenia, wzial oszczep w rece swoje.
\par 8 A wszedlszy za onym mezem Izraelskim do namiotu, przebil oboje, meza Izraelskiego, i niewiaste przez zywot jej, i odwrócona byla plaga od synów Izraelskich.
\par 9 A bylo tych, co pomarli ona plaga, dwadziescia i cztery tysiace.
\par 10 Potem rzekl Pan do Mojzesza, mówiac:
\par 11 Finees syn Eleazara, syna Aarona kaplana odwrócil gniew mój od synów Izraelskich, bedac wzruszony zapalczywa miloscia ku mnie w posrodku ich, tak izem nie wytracil synów Izraelskich w zapalczywosci mojej.
\par 12 Przetoz powiedz mu: Oto, Ja stanowie z nim przymierze moje, przymierze pokoju;
\par 13 I przyjdzie nan, i na nasienie jego po nim, przymierze kaplanstwa wiecznego, ze sie wzruszyl zapalczywoscia za Boga swego, i oczyscil syny Izraelskie
\par 14 A imie onego meza Izraelskiego zabitego, który zabity byl z Madyjanitka, bylo Zamry, syn Salów, ksiaze domu ojca swego, z pokolenia Symeonowego.
\par 15 Imie tez niewiasty zabitej Madyjanitki bylo Kozba, córka Sury, ksiazecia w narodzie swym, w domu ojczystym miedzy Madyjanity.
\par 16 I rzekl Pan do Mojzesza, mówiac:
\par 17 Staw sie nieprzyjacielem Madyjanitom, a pobijcie je,
\par 18 Poniewaz i oni stawili sie wam nieprzyjaciolmi zdradami swemi, a podeszli was przez Baal Fegora, i przez Kozbe, córke ksiazecia Madyjanskiego, siostre swa, która zabita jest w dzien kazni dla balwana Fegor.

\chapter{26}

\par 1 I stalo sie po onej pladze, ze mówil Pan do Mojzesza i do Eleazara, syna Aarona kaplana, mówiac:
\par 2 Policzcie poczet wszystkiego zgromadzenia synów Izraelskich, od tych, którzy maja dwadziescia lat i wyzej, wedlug domów ojców ich, kazdego któryby mógl wynijsc na wojne z Izraela.
\par 3 Tedy rzekl Mojzesz, i Eleazar kaplan do nich na polach Moabskich, nad Jordanem przeciw Jerychu, mówiac:
\par 4 Liczcie lud, od tych, którzy maja dwadziescia lat i wyzej, jako byl rozkazal Pan Mojzeszowi, i synom Izraelskim, gdy wyszli z ziemi Egipskiej.
\par 5 Ruben pierworodny Izraela: synowie Rubenowi Henoch, od którego poszedl dom Henochytów; Fallu, od którego dom Faalluitów;
\par 6 I Hesron, od którego dom Hesronitów; Charmi, od którego dom Charmitów.
\par 7 Tec sa domy Rubenitów; a bylo ich policzonych czterdziesci i trzy tysiace, siedem set i trzydziesci.
\par 8 A syn Fallów Elijab.
\par 9 Synowie zasie Elijabowi byli: Namuel, i Datan, i Abiron. A ci, Datan i Abiron, zacniejsi byli miedzy zgromadzeniem, którzy sie swarzyli z Mojzeszem i z Aaronem w spiknieniu Korego, gdy sie zbuntowali przeciwko Panu.
\par 10 I otworzyla ziemia usta swoje, a pozarla onych, i Korego, gdy zginela ona rota, a pozarl ich ogien dwie cie i piecdziesiat mezów, którzy sie stali na przyklad innym;
\par 11 Ale synowie Korego nie pomarli.
\par 12 Synowie Symeonowi wedle domów swych, ci sa: Namuel, od którego poszedl dom Namuelitów; Jamin, od którego dom Jaminitów; Jachin, od którego dom Jachinitów; Zare, od którego dom Zareitów;
\par 13 Saul, od którego dom Saulitów.
\par 14 Tec byly domy Symeonitów, których bylo dwadziescia i dwa tysiace i dwiescie.
\par 15 Synowie Gadowi wedlug domów swych: Sefon, od którego poszedl dom Sefonitów; Aggi, od którego dom Aggitów; Suni, od którego dom Sunitów.
\par 16 Ozni, od którego dom Oznitów; Hery, od którego dom Herytów;
\par 17 Arod, od którego dom Arodytów; Aryjel, od którego dom Aryjelitów.
\par 18 Tec sa domy synów Gadowych, wedlug pocztów ich czterdziesci tysiecy i piec set.
\par 19 Synowie Judowi: Her, i Onan; ale pomarli Her i Onan w ziemi Chananejskiej.
\par 20 I byli synowie Judowi wedle domów swych: Sela, od którego poszedl dom Selaitów; Fares, od którego dom Faresytów; Zare, od którego dom Zarejczyków;
\par 21 Byli tez synowie Faresowi: Hesron, od którego dom Hesronitów; Hamuel, od którego dom Hamuelitów.
\par 22 Tec sa domy Judy, wedlug pocztów ich siedemdziesiat tysiecy, i szesc, i piec set.
\par 23 Synowie Isascharowi wedlug domów swych: Tola, od którego dom Tolaitów; Fua, od którego dom Fuaitów;
\par 24 Jasub, od którego dom Jasubitów; Semram, od którego dom Semramitów.
\par 25 Tec sa domy Isascharowe, wedle pocztów ich szescdziesiat tysiecy i cztery, i trzy sta.
\par 26 Synowie Zabulonowi wedlug domów swych: Zared, od którego dom Zaredczyków; Elon, od którego dom Elonitów; Jaleel, od którego dom Jaleelitów.
\par 27 A tec sa domy Zabulonitów, wedlug pocztów ich szescdziesiat tysiecy i piec set.
\par 28 Synowie Józefowi wedlug domów swych: Manases i Efraim.
\par 29 Synowie Manasesowi: Machir, od którego dom Machirytów; a Machir splodzil Galaada, od Galaada dom Galaadytów.
\par 30 Ci sa synowie Galaadowi: Jezer, od którego dom Jezerytów; Chelek, od którego dom Chelekitów;
\par 31 I Asryjel, od którego dom Asryjelitów; i Sechem, od którego dom Sechemitów;
\par 32 I Semida, od którego dom Semidaitów; i Chefer, od którego dom Cheferytów.
\par 33 A Salfaad, syn Cheferów, nie mial synów, tylko córki, a imiona córek Salfaadowych: Machla, i Noa, Hegla, Melcha, i Tersa.
\par 34 Tec sa domy Manasesowe, a poczet ich piecdziesiat i dwa tysiace i siedem set.
\par 35 Synowie zas Efraimowi wedlug domów swych: Sutala, od którego dom Sutalitów; Becher, od którego dom Becherytów; Techen, od którego dom Techenitów.
\par 36 A ci sa synowie Sutalego: Heran, od którego dom Heranitów.
\par 37 Tec sa domy synów Efraimowych, wedlug pocztów ich trzydziesci tysiecy i dwa, i piec set. Ci sa synowie Józefowi wedlug domów swych.
\par 38 A synowie Benjaminowi wedlug domów swych, ci sa: Bela, od którego dom Belitów; Asbel, od którego dom Asbelitów; Achiram, od którego dom Achiramitów;
\par 39 Sufam, od którego dom Sufamitów; Hufam, od którego dom Hufamitów.
\par 40 Byli tez synowie Beli: Hereda i Noemana; z Hereda dom Heredytów, a z Noemana dom Noemanitów.
\par 41 Cic sa synowie Benjaminowi, wedlug domów ich, a poczet ich czterdziesci i piec tysiecy i szesc set.
\par 42 Synowie zas Danowi wedlug domów swych: Sucham, od którego dom Suchamitów. Tec byly domy Danowe wedlug familii ich.
\par 43 Wszystkie domy Suchamitów wedlug pocztów ich szescdziesiat i cztery tysiace i cztery sta.
\par 44 Synowie Aserowi wedlug domów swych byli: Jemna, od którego dom Jemnitów; Iswi, od którego dom Iswitów; Beryja, od którego dom Berytów.
\par 45 Synowie Beryjego: Heber, od którego dom Heberytów; Melchyjel, od którego dom Melchyjelitów.
\par 46 A imie córki Aserowej bylo Sara.
\par 47 Te sa domy synów Aserowych, wedlug pocztów ich piecdziesiat i trzy tysiace i cztery sta.
\par 48 Synowie Neftalimowi wedlug domów swych: Jachsel, od którego dom Jachselitów; Guni, od którego dom Gunitów; Jesser, od którego dom Jesserytów.
\par 49 Selem, od którego dom Selemitów.
\par 50 Tec sa domy Neftalimowe, wedlug familii ich, a poczet ich czterdziesci i piec tysiecy i cztery sta.
\par 51 Tac jest liczba synów Izraelskich, szesc kroc sto tysiecy i tysiac, siedem set i trzydziesci.
\par 52 Zatem rzekl Pan do Mojzesza, mówiac:
\par 53 Miedzy te podzielicie te ziemie w dziedzictwo wedlug liczby imion.
\par 54 Wiekszej liczbie wiecej dziedzictwa dasz, a mniejszej mniejsze dziedzictwo dasz; kazdemu wedlug pocztów policzonych jego bedzie dane dziedzictwo jego.
\par 55 Wszakze losem niech bedzie rozdzielona ziemia; wedlug imion pokolen ojców swych dziedzictwo brac beda.
\par 56 Losem rozdzielone bedzie dziedzictwo jej badz ich wiele badz malo bedzie.
\par 57 Ci zasie sa policzeni z Lewitów wedlug domów swych: Gerson, od którego dom Gersonitów; Kaat, od którego dom Kaatytów; Merary, od którego dom Merarytów.
\par 58 Tec sa domy Lewi: dom Libnitów, dom Hebronitów, dom Moholitów, dom Musytów, dom Korytów; a Kaat splodzil Amrama.
\par 59 A imie zony Amramowej bylo Jochabod, córka Lewiego, która mu sie urodzila w Egipcie; ta Amramowi urodzila Aarona, i Mojzesza, i Maryja, siostre ich.
\par 60 Aaronowi tez urodzili sie Nadabi i Abiju, Eleazar i Itamar.
\par 61 Ale pomarli Nadab i Abiju, gdy ofiarowali ogien obcy przed Panem.
\par 62 A byla liczba ich dwadziescia i trzy tysiace, wszystkich mezczyzn urodzonych od miesiaca i wyzej; jednak nie byli policzeni miedzy syny Izraelskie, bo im nie dano dziedzictwa miedzy syny Izraelskimi.
\par 63 Ci policzeni byli od Mojzesza i Eleazara kaplana, którzy policzyli syny Izraelskie na polach Moabskich, nad Jordanem przeciw Jerychu.
\par 64 A miedzy tymi nie byl zaden z onych policzonych od Mojzesza i Aarona kaplana, gdy liczyli syny Izraelskie na puszczy Synaj;
\par 65 Bo rzekl byl Pan o nich: Smiercia pomra na puszczy; a nie zostal zaden z nich, oprócz Kaleba, syna Jefunowego, i Jozuego, syna Nunowego.

\chapter{27}

\par 1 Tedy przyszly córki Salfaada, syna Heferowego, syna Galaadowego, syna Machyrowego, syna Manasesowego, z pokolenia Manasesa syna Józefowego; a te sa imiona córek jego: Machla, Noa i Hegla, i Melcha i Tersa;
\par 2 I stanely przed Mojzeszem i przed Eleazarem kaplanem i przed ksiazety i wszystkiem zgromadzeniem u drzwi namiotu zgromadzenia, i rzekly:
\par 3 Ojciec nasz umarl na puszczy, a on nie byl w poczcie tych, którzy sie byli przeciw Panu zbuntowali w spiknieniu Korego; ale dla grzechu swego umarl, nie majac synów.
\par 4 Czemuz by zginac mialo imie ojca naszego z domu jego, przeto, ze nie mial syna? dajcie nam dziedzictwo miedzy bracia ojca naszego.
\par 5 Tedy odniósl Mojzesz sprawe ich do Pana.
\par 6 I rzekl Pan do Mojzesza:
\par 7 Dobrze mówia córki Salfaadowe: Daj im koniecznie osiadlosc dziedzictwa miedzy bracia ojca ich, a przenies dziedzictwo ojca ich na nie.
\par 8 Synom takze Izraelskim powiedz, mówiac: Gdyby kto umarl, nie majac syna, tedy przeniesiecie dziedzictwo jego na córke jego.
\par 9 A jesliby nie mial i córki, tedy dacie dziedzictwo jego braci jego.
\par 10 A jesliby i braci nie mial, tedy dacie dziedzictwo jego braci ojca jego.
\par 11 A jesliby nie bylo braci ojca jego, tedy dacie dziedzictwo jego pokrewnemu jego, najblizszemu jego z domu jego, aby je odziedziczyl. A bedzie to synom Izraelskim za ustawe prawna, jako rozkazal Pan Mojzeszowi.
\par 12 Potem rzekl Pan do Mojzesza: Wstap na te góre Abarym, a ogladaj ziemie, któram dal synom Izraelskim.
\par 13 A gdy ja ogladasz, przylaczon bedziesz do ludu twego i ty, jako jest przylaczony Aaron, brat twój.
\par 14 Przeto zescie byli odpornymi slowu mojemu na puszczy Syn, przy poswarku zgromadzenia, i nie poswieciliscie mie przy wodach przed oczyma ich. Onec to sa wody poswarku w Kades, na puszczy Syn.
\par 15 Tedy rzekl Mojzesz do Pana, mówiac:
\par 16 Niech opatrzy Pan, Bóg duchów wszelkiego ciala, mezem godnym to zgromadzenie;
\par 17 Któryby wychodzil przed nimi, i któryby wchodzil przed nimi, i któryby je przywodzil, aby nie byl lud Panski jako owce, nie majace pasterza.
\par 18 Tedy rzekl Pan do Mojzesza: Wezmij do siebie Jozuego, syna Nunowego, meza, w którym jest Duch mój, a wlóz nan reke swoje;
\par 19 I postaw go przed Eleazarem kaplanem, i przed wszystkiem zgromadzeniem, a dasz mu nauke przed oczyma ich;
\par 20 A udzielisz mu zacnosci swej, aby go sluchalo wszystko zgromadzenie synów Izraelskich;
\par 21 Który przed twarza Eleazara kaplana stawac bedzie, aby sie zan radzil sadu Urim przed Panem. Na rozkazanie jego wychodzic beda, on, i wszyscy synowie Izraelscy z nim, i wszystko zgromadzenie.
\par 22 Uczynil tedy Mojzesz, jako mu byl rozkazal Pan; a wziawszy Jozuego postawil go przed Eleazarem kaplanem, i przed wszystkiem zgromadzeniem.
\par 23 I wlozywszy nan rece swe, dal mu nauke, jako mówil Pan przez Mojzesza.

\chapter{28}

\par 1 I rzekl Pan do Mojzesza mówiac:
\par 2 Rozkaz synom Izraelskim, a powiedz im: Ofiary mojej chleba mego, w ofiarach moich ognistych, na wdziecznosc wonnosci mojej, przestrzegac bedziecie, abyscie mi je ofiarowali czasu swego.
\par 3 I rzeczesz do nich: Tac jest ofiara ognista, która ofiarowac bedziecie Panu: Baranki roczne zupelne dwa na kazdy dzien, na calopalenie ustawiczne;
\par 4 Baranka jednego ofiarowac bedziesz poranku, a baranka drugiego ofiarowac bedziesz miedzy dwoma wieczorami.
\par 5 Do tego dziesiata czesc efy maki pszennej na ofiare sniedna, nagniatanej z oliwa czysta z czwarta czescia hynu.
\par 6 Toc jest calopalenie ustawiczne, jakie bylo sprawowane na górze Synaj na wdzieczna wonnosc; ognista to ofiara Panu.
\par 7 A ofiara jej mokra bedzie czwarta czesc hynu do kazdego baranka; w swiatnicy sprawowac bedziesz ofiare mokra z mocnego napoju Panu.
\par 8 A drugiego baranka ofiarowac bedziesz miedzy dwoma wieczorami; jako ofiare sniedna poranna, i jako ofiare mokra jej ofiarowac bedziesz na ofiare ognista ku wdziecznej wonnosci Panu.
\par 9 Ale w dzien sabatu ofiarowac bedziesz dwa baranki roczne zupelne, i dwie dziesiate czesci efy maki pszennej z oliwa nagniecionej na ofiare sniedna i z mokra jej ofiara.
\par 10 Toc jest calopalenie sobotnie w kazdy sabat, oprócz calopalenia ustawicznego i mokrej ofiary jego.
\par 11 A na nowiu miesiaców waszych ofiarowac bedziecie calopalenie Panu, cielców mlodych dwa, i barana jednego, baranków rocznych siedem;
\par 12 I trzy dziesiate efy maki pszennej nagniecionej z oliwa na ofiare sniedna do kazdego cielca, i dwie dziesiate czesci pszennej maki zagniecionej z oliwa na ofiare sniedna do kazdego barana;
\par 13 A jedna dziesiata czesc maki pszennej zagniecionej z oliwa na ofiare sniedna do kazdego baranka, na calopalenie ku wdziecznosci wonnosci na ofiare ognista Panu.
\par 14 Takze mokre ich ofiary z wina pól hynu bedzie dla kazdego cielca, a trzecia czesc hynu do barana, czwarta zas czesc hynu do kazdego baranka; toc jest calopalenie na nowiu miesiaca, kazdego miesiaca przez rok.
\par 15 Kozla tez jednego z stada za grzech ofiarowac bedziecie Panu oprócz ustawicznego calopalenia, i mokrej ofiary jego.
\par 16 Ale miesiaca pierwszego w dzien czternasty tegoz miesiaca, swieto przejscia bedzie Panu.
\par 17 A w pietnasty dzien tegoz miesiaca uroczyste swieto bedzie; przez siedem dni chleby przasne jesc bedziecie.
\par 18 Pierwszego dnia zgromadzenie swiete; zadnej roboty sluzebniczej nie bedziecie czynic wen.
\par 19 Ale ofiarowac bedziecie ofiare ognista na calopalenie Panu: dwóch cielców mlodych, i barana jednego, i siedem baranków rocznych; zupelni niech wam beda.
\par 20 A na ofiare ich sniedna pszennej maki nagniecionej z oliwa trzy dziesiate czesci efy do kazdego cielca, a dwie dziesiate czesci do kazdego barana ofiarowac bedziecie.
\par 21 Jedna dziesiata czesc ofiarowac bedziesz przy kazdym baranku z onych siedmiu baranków;
\par 22 Kozla tez jednego na ofiare za grzech ku oczyszczeniu was.
\par 23 Nad calopalenie poranne, które ma byc calopalenie ustawiczne, ofiarowac to bedziecie,
\par 24 Tak ofiarowac bedziecie kazdego dnia przez onych siedem dni pokarm ofiary ognistej na wdzieczna wonnosc Panu, oprócz calopalenia ustawicznego, i mokrej ofiary jego.
\par 25 A dnia siódmego swiete zgromadzenie miec bedziecie; zadnej roboty sluzebniczej nie bedziecie czynic.
\par 26 W dzien zas pierwocin; gdy bedziecie ofiarowali nowa sniedna ofiare Panu, gdy sie wypelnia tygodnie wasze, zgromadzenie swiete miec bedziecie; zadnej roboty sluzebniczej czynic nie bedziecie.
\par 27 I ofiarowac bedziecie calopalenie ku wdziecznosci wonnosci Panu: dwóch cielców mlodych, barana jednego, siedem baranków rocznych.
\par 28 A na ofiare sniedna ich maki pszennej nagniecionej z oliwa trzy dziesiate czesci efy do kazdego cielca, dwie dziesiate czesci do kazdego barana.
\par 29 Jedna dziesiata czesc do kazdego baranka z onych siedmiu baranków.
\par 30 Kozla jednego z kóz na oczyszczenie was.
\par 31 Oprócz calopalenia ustawicznego i ofiary sniednej jego ofiarowac to bedziecie; te rzeczy zupelne niech wam beda, i z mokremi ofiarami ich.

\chapter{29}

\par 1 Miesiaca zas siódmego w pierwszy dzien jego, zgromadzenie swiete miec bedziecie; zadnej roboty sluzebniczej nie bedziecie czynic; dzien jest wesolego trabienia waszego.
\par 2 A bedziecie ofiarowali calopalenie ku wdziecznej wonnosci Panu, cielca mlodego jednego, barana jednego, baranków rocznych siedem zupelnych;
\par 3 A na ofiare sniedna ich z maki pszennej nagniecionej z oliwa trzy dziesiate czesci efy do cielca, a dwie dziesiate czesci do barana.
\par 4 A dziesiate czesc jedne do kazdego baranka z onych siedmiu baranków;
\par 5 Takze kozla jednego z kóz ku ofierze za grzech na oczyszczenie was.
\par 6 Oprócz calopalenia nowego miesiaca, i ofiary sniednej jego, i oprócz calopalenia ustawicznego, i ofiary sniednej jego, i ofiar ich mokrych wedlug obrzedów ich ku wdziecznej wonnosci; ofiara to ognista Panu.
\par 7 Potem dziesiatego dnia tegoz miesiaca siódmego, zgromadzenie swiete miec bedziecie, a bedziecie trapic dusze wasze; zadnej roboty nie bedziecie robic.
\par 8 A bedziecie ofiarowali calopalenie Panu ku wdziecznej wonnosci: cielca mlodego jednego, barana jednego, baranków rocznych siedem; zupelni niech wam beda;
\par 9 A na ofiare sniedna ich z pszennej maki nagniecionej z oliwa: trzy dziesiate czesci do kazdego cielca, dwie zas dziesiate czesci do kazdego barana;
\par 10 A dziesiata czesc jedne do kazdego baranka z onych siedmiu baranków;
\par 11 Kozla z kóz jednego na ofiare za grzech, oprócz ofiary za grzech na oczyszczenie, i oprócz calopalenia ustawicznego, i ofiary sniednej jego, i mokrych ofiar ich.
\par 12 W pietnasty zas dzien tegoz siódmego miesiaca zgromadzenie swiete miec bedziecie; zadnej roboty sluzebniczej nie bedziecie czynic wen; ale obchodzic bedziecie swieto uroczyste Panu przez siedem dni.
\par 13 I ofiarowac bedziecie calopalenie na ofiare ognista ku wdziecznej wonnosci Panu, cielców mlodych trzynascie, baranów dwa, baranków rocznych czternascie; i zupelni beda.
\par 14 A na ofiare ich sniedna z pszennej maki zagniecionej z oliwa trzy dziesiate czesci efy do kazdego cielca z onych trzynascie cielców, dwie dziesiate czesci do kazdego barana z onych dwóch baranów;
\par 15 A jedna dziesiata czesc do kazdego baranka z onych czternascie baranków.
\par 16 Takze kozla jednego z kóz na ofiare za grzech, oprócz calopalenia ustawicznego, ofiary sniednej jego, i ofiary mokrej jego.
\par 17 Wtórego zas dnia ofiarowac bedziecie cielców mlodych dwanascie, baranów dwa, baranków rocznych czternascie zupelnych;
\par 18 I ofiare sniedna ich, i ofiary ich mokre do kazdego cielca, do kazdego barana, i do kazdego baranka wedlug liczby ich, i wedlug zwyczaju ich.
\par 19 Nadto kozla jednego z kóz na ofiare za grzech, oprócz calopalenia ustawicznego, i ofiary sniednej jego, i ofiar ich mokrych.
\par 20 Dnia zas trzeciego ofiarowac bedziecie jedenascie cielców, baranów dwa, baranków rocznych czternascie zupelnych.
\par 21 I ofiare sniedna ich, i ofiary mokre ich do kazdego cielca, do kazdego barana, i do kazdego baranka wedlug liczby ich, i wedlug zwyczaju ich;
\par 22 Do tego, kozla jednego na ofiare za grzech, okrom calopalenia ustawicznego, i ofiary sniednej jego, i mokrej ofiary jego.
\par 23 A dnia czwartego ofiarowac bedziecie cielców dziesiec, baranów dwa, baranków rocznych czternascie zupelnych;
\par 24 Ofiare sniedna ich, i ofiary mokre ich do kazdego cielca, do kazdego barana, i do kazdego baranka wedlug liczby ich, i wedlug zwyczaju ich;
\par 25 Kozla tez jednego z kóz na ofiare za grzech, oprócz calopalenia ustawicznego, ofiary sniednej jego, i mokrej ofiary jego.
\par 26 A dnia piatego ofiarowac bedziecie cielców dziewiec, baranów dwa, baranków rocznych czternascie zupelnych;
\par 27 I ofiare sniedna ich, i ofiary mokre ich do kazdego cielca, do kazdego barana, i do kazdego baranka wedlug liczby ich, i wedlug zwyczaju ich;
\par 28 Takze kozla jednego na ofiare za grzech, oprócz calopalenia ustawicznego, i ofiary sniednej jego, i ofiary mokrej jego.
\par 29 A dnia szóstego ofiarowac bedziecie cielców osiem, baranów dwa, baranków rocznych czternascie zupelnych;
\par 30 I ofiare sniedna ich, i ofiary mokre ich do kazdego cielca, i do kazdego barana, i do kazdego baranka wedlug liczby ich, wedlug zwyczaju ich.
\par 31 Nadto kozla za ofiare za grzech jednego, okrom calopalenia ustawicznego, ofiary sniednej jego, i ofiar mokrych jego;
\par 32 Takze dnia siódmego ofiarowac bedziecie cielców siedem, baranów dwa, baranków rocznych czternascie zupelnych.
\par 33 I ofiare sniedna ich, i ofiary mokre ich do kazdego cielca, do kazdego barana, do kazdego baranka wedlug liczby ich, i wedlug zwyczaju ich;
\par 34 Przytem kozla na ofiare za grzech jednego, oprócz calopalenia ustawicznego i ofiary sniednej jego, i ofiary mokrej jego.
\par 35 A dnia ósmego zacne swieto miec bedziecie; zadnej roboty sluzebniczej nie bedziecie czynic.
\par 36 A ofiarowac bedziecie calopalenie, i ofiare ognista ku wdziecznej wonnosci Panu, cielca jednego, barana jednego, baranków rocznych siedem zupelnych;
\par 37 Ofiare sniedna ich, i ofiary mokre ich do cielca, do barana, do kazdego baranka wedlug liczby ich, i wedlug zwyczaju ich.
\par 38 Nadto kozla na ofiare za grzech jednego, okrom calopalenia ustawicznego, ofiary sniednej jego, i ofiary mokrej jego.
\par 39 To ofiarowac bedziecie Panu w swieta uroczyste wasze, oprócz slubów waszych i dobrowolnych ofiar waszych w calopaleniach waszych, i w sniednych ofiarach waszych, i w mokrych ofiarach waszych, i w spokojnych ofiarach waszych.

\chapter{30}

\par 1 I powiedzial Mojzesz synom Izraelskim to wszystko, co rozkazal Pan Mojzeszowi.
\par 2 Potem mówil Mojzesz do ksiazat w pokoleniach miedzy synami Izraelskimi, i rzekl: Toc jest, co rozkazal Pan.
\par 3 Jezliby maz poslubil slub Panu, albo tez przysiege uczynil, obowiazkiem obowiazawszy dusze swoje, nie zlamie slowa swego: wedlug wszystkiego coby wyszlo z ust jego, uczyni.
\par 4 Ale jezliby niewiasta poslubila slub Panu, i obowiazalaby sie obowiazkiem w domu ojca swego w mlodosci swojej;
\par 5 A slyszalby ojciec jej on slub jej, i obowiazek jej, którym obowiazala dusze swoje, a milczalby na to ojciec jej, tedy platne beda wszystkie sluby jej, i kazdy obowiazek, którym by obowiazala dusze swa, platny bedzie.
\par 6 Ale jezliby byl onegoz dnia przeciw temu ojciec jej, którego by slyszal wszystkie sluby jej, i obowiazki jej, któremi obowiazala dusze swoje nie beda platne; Pan odpusci jej, bo byl przeciw temu ojciec jej.
\par 7 Ale gdyby majaca meza slub jaki uczynila, albo wyrzekla co usty swemi, czem by obowiazala dusze swoje;
\par 8 A slyszac to maz jej, milczalby na to onegoz dnia, którego slyszal, platne beda sluby jej, i obowiazki jej, któremi obowiazala dusze swoje, platne beda.
\par 9 Ale jezliby onego dnia, którego slyszal maz jej, sprzeciwil sie temu, i wzruszylby slub jej, który na sobie miala, i co wymówila usty swemi, czem obowiazala dusze swoje, takze Pan odpusci jej.
\par 10 Ale slub kazdy wdowy, i odrzuconej którym by obowiazala dusze swoje, platny bedzie.
\par 11 Lecz jezliby, póki byla w domu meza swego, slub uczynila, i obowiazala obowiazkiem dusze swoje z przysiega;
\par 12 A slyszac maz jej milczalby na to, i nie sprzeciwilby sie temu, tedy platne beda wszystkie sluby jej i kazdy obowiazek, którym obowiazala dusze swoje, platny bedzie.
\par 13 Ale jezli cale sprzeciwil sie temu maz jej dnia, którego to slyszal, wszelki slub, który wyszedl z ust jej, i obowiazek duszy jej, nie bedzie platny; maz jej wzruszyl to, a Pan odpusci jej.
\par 14 Wszelkiego slubu i wszelkiej przysiegi obowiazku na utrapienie duszy, maz jej potwierdzi go, i maz jej wzruszy go.
\par 15 A jezliby cale milczal maz jej ode dnia do dnia, tedy tem stwierdzi wszystkie sluby jej, i wszystkie obowiazki jej, które ma na sobie; stwierdzi je, przeto, ze milczal na to w dzien, którego slyszal;
\par 16 A jezliby to koniecznie wzruszyc chcial, nie zaraz gdy slyszal, ale potem, poniesie nieprawosc jej.
\par 17 Tec sa ustawy, które przykazal Pan Mojzeszowi, miedzy mezem a zona jego, miedzy ojcem a córka jego w mlodosci jej, póki jest w domu ojca swego.

\chapter{31}

\par 1 Potem rzekl Pan do Mojzesza, mówiac:
\par 2 Pomscij sie krzywdy synów Izraelskich nad Madyjanitami, i potem przylaczon bedziesz do ludu twego.
\par 3 Tedy rzekl Mojzesz do ludu, mówiac: Wyprawcie z posrodku siebie meze ku bitwie, aby szli przeciw Madyjanitom, i wykonali pomste Panska nad nimi.
\par 4 Po tysiacu z kazdego pokolenia, ze wszystkich pokolen Izraelskich wyslecie na wojne.
\par 5 I wyprawili z tysiaców Izraelskich, po tysiacu z kazdego pokolenia, dwanascie tysiecy ludzi gotowych do bitwy.
\par 6 I wyslal je Mojzesz po tysiacu z kazdego pokolenia na wojne; poslal tez z nimi Fineesa, syna Eleazara kaplana, na wojne, a naczynia swiete, i traby do trabienia byly w reku jego.
\par 7 Tedy zwiedli bitwe z Madyjanitami, jako rozkazal Pan Mojzeszowi, i pobili wszystkie mezczyzny.
\par 8 Króle tez Madyjanskie pobili miedzy inszymi pobitymi ich, Ewiego, i Rechema, i Sura, i Hura, i Rebaha, pieciu królów Madyjanskich, i Balaama, syna Beorowego, zabili mieczem.
\par 9 I pobrali w niewola synowie Izraelscy zony Madyjanczyków, i dziatki ich, i wszystko bydlo ich, i wszystkie trzody ich, i wszystkie majetnosci ich pobrali;
\par 10 A wszystkie miasta ich, w których mieszkali, i wszystkie zamki ich popalili ogniem;
\par 11 I pobrali wszystkie lupy, i wszystkie plony z ludzi, i z bydla,
\par 12 I przywiedli do Mojzesza i do Eleazara kaplana, i do zgromadzenia synów Izraelskich wieznie, i lupy, i korzysci do obozu na pola Moabskie, które sa nad Jordanem przeciw Jerychu.
\par 13 I wyszli Mojzesz i Eleazar kaplan, i wszystkie ksiazeta zgromadzenia przeciwko nim przed obóz.
\par 14 I rozgniewal sie bardzo Mojzesz na hetmany wojska onego, na pulkowniki, i na rotmistrze, którzy sie byli wrócili z onej bitwy.
\par 15 I mówil do nich Mojzesz: Przeczzescie zywo zachowali wszystkie niewiasty?
\par 16 Gdyz te sa, które synom Izraelskim za rada Balaamowa daly przyczyne do przestepstwa przeciw Panu przy balwanie Fegor, skad byla przyszla plaga na zgromadzenie Panskie.
\par 17 Przetoz teraz pozabijajcie wszystkie mezczyzny z dzieci, i kazda niewiaste, która poznala meza, obcujac z nim, zabijcie;
\par 18 Ale wszystkie dzieweczki z bialych glów, które nie poznaly loza meskiego, zywo zachowajcie sobie.
\par 19 A wy sami zostancie w namiecich za obozem przez siedem dni; kazdy, który kogo zabil, i który sie dotykal zabitego, oczyscicie sie dnia trzeciego a dnia siódmego, siebie i wieznie wasze;
\par 20 I wszelka szate, i wszelkie naczynie skórzane, i wszystko, co urobiono z koziej siersci, i wszelkie naczynie drzewiane oczyscicie.
\par 21 I rzekl Eleazar kaplan do zolnierstwa, które chodzilo na wojne: Tac jest ustawa zakonna, która byl rozkazal Pan Mojzeszowi:
\par 22 Zloto jednak i srebro, miedz, zelazo, cyne i olów;
\par 23 I kazda rzecz, która zniesc moze ogien, wyprawicie przez ogien, a bedzie oczyszczona, wszakze pierwej woda oczyszczenia bedzie oczyszczona; ale wszystko, co nie moze zniesc ognia, woda oczyscicie.
\par 24 Upierzecie tez szaty wasze dnia siódmego, i czystymi bedziecie, a potem wnijdziecie do obozu.
\par 25 Zatem rzekl Pan do Mojzesza, mówiac:
\par 26 Zbierz summe korzysci pobranych z ludzi i z bydla, ty i Eleazar kaplan, i przedniejsi z ojców w ludu;
\par 27 I rozdzielisz te lupy na dwie czesci, miedzy zolnierze, którzy na wojne wychodzili, i miedzy wszystko zgromadzenie.
\par 28 Odbierzesz tez dzial na Pana od mezów rycerskich, którzy byli wyszli na wojne, po jednemu od pieciu set, z ludzi, i z wolów, i z oslów, i z owiec;
\par 29 A z polowy ich wezmiecie, i oddacie Eleazarowi kaplanowi na ofiare podnoszenia Panu.
\par 30 A z polowy synów Izraelskich wezmiesz jedna czesc z pieciudziesiat, z ludzi, z wolów, z oslów, i z owiec, i z wszelkiego bydla, i oddasz to Lewitom trzymajacym straz w przybytku Panskim.
\par 31 I uczynil Mojzesz i Eleazar kaplan, jako rozkazal Pan Mojzeszowi.
\par 32 A bylo onej korzysci z pozostalych lupów, które rozchwycil lud wojenny: Owiec szesc kroc sto tysiecy, i siedemdziesiat tysiecy i piec tysiecy;
\par 33 Wolów zas, siedemdziesiat i dwa tysiace;
\par 34 A oslów szescdziesiat tysiecy i jeden.
\par 35 A ludzi z bialych glów, które nie poznaly obcowania z mezem, wszystkich bylo trzydziesci i dwa tysiace.
\par 36 I dostala sie polowa dzialu tym, co wychodzili na wojne, liczba owiec trzy kroc sto tysiecy, i trzydziesci tysiecy, i siedem tysiecy i piec set.
\par 37 Dostalo sie tez dzialu na Pana owiec szesc set, siedemdziesiat i piec;
\par 38 A z wolów trzydziesci i szesc tysiecy, a dzialu z nich Panu siedemdziesiat i dwa;
\par 39 Oslów tez trzydziesci tysiecy i piec set, a dzialu z nich Panu szescdziesiat i jeden.
\par 40 Przytem ludu szesnascie tysiecy, a dzialu z nich Panu trzydziesci i dwoje ludzi.
\par 41 I oddal Mojzesz dzial na ofiare Panu, Eleazarowi kaplanowi, jako byl rozkazal Pan Mojzeszowi.
\par 42 A z drugiej polowy synów Izraelskich, która wzial Mojzesz od mezów, którzy byli wyszli na wojne.
\par 43 (A polowa nalezaca zgromadzeniu, byla: Owiec trzy kroc sto tysiecy, i trzydziesci tysiecy, siedem tysiecy i piec set;
\par 44 A wolów trzydziesci i szesc tysiecy;
\par 45 A oslów trzydziesci tysiecy i piec set;
\par 46 A ludu szesnascie tysiecy.)
\par 47 Wzial Mojzesz z tej polowy nalezacej synom Izraelskim, jedne czesc z pieciudziesiat, z ludzi, i z bydla, i dal to Lewitom trzymajacym straz przybytku Panskiego, jako byl rozkazal Pan Mojzeszowi.
\par 48 Tedy przyszli do Mojzesza hetmani wojska, pulkownicy, i rotmistrze.
\par 49 I mówili do niego: My sludzy twoi przynieslismyc poczet mezów wojennych, którzy byli pod sprawa nasza, a nie zginal z nas i jeden.
\par 50 A tak przynieslismy tu na ofiare Panu, kazdy czego nabyl, naczynie zlote, zapony, i manele, pierscienie, i nausznice, i lancuszki, dla oczyszczenia dusz naszych przed Panem.
\par 51 Odebral tedy Mojzesz i Eleazar kaplan ono zloto od nich z wszelakiem naczyniem z niego urobionym.
\par 52 A bylo onego wszystkiego zlota ofiarowanego, które ofiarowali Panu, szesnascie tysiecy, siedem set i piecdziesiat syklów od pulkowników i od rotmistrzów.
\par 53 (Bo zolnierze, co lupem dostali, sobie otrzymali.)
\par 54 A wziawszy Mojzesz i Eleazar kaplan ono zloto od pulkowników i rotmistrzów, wniesli je do namiotu zgromadzenia, na pamiatke synów Izraelskich przed Panem.

\chapter{32}

\par 1 I mieli synowie Rubenowi, i synowie Gadowi bydla bardzo wiele; a obaczywszy ziemie Jazer i ziemie Galaad, ze miejsce ono bylo sposobne dla bydla,
\par 2 Przyszli ciz synowie Gadowi, i synowie Rubenowi, i mówili do Mojzesza i do Eleazara kaplana, i do ksiazat zgromadzenia, i rzekli:
\par 3 Ziemia Ataret i Dybon, i Jazer, i Nemra, i Hesebon, i Eleale, i Seban, i Nebo, i Beon:
\par 4 Ziemia, która zwojowal Pan przed zgromadzeniem Izraelskiem, jest ziemia sposobna dla bydla, a my sludzy twoi mamy bydla wiele. Przetoz rzekli:
\par 5 Jezlismy znalezli laske przed oczyma twemi, niechze bedzie dana ta ziemia slugom twym na osiadlosc, a niech nie chodzimy za Jordan.
\par 6 Tedy odpowiedzial Mojzesz synom Gadowym, i synom Rubenowym: Wiec bracia wasi pójda na wojne, a wy tu siedziec bedziecie?
\par 7 Czemuz psujecie serce synom Izraelskim, zeby nie szli do ziemi, która im dal Pan?
\par 8 Takci uczynili ojcowie wasi, gdym je byl poslal z Kades Barne ku przeszpiegowaniu tej ziemi;
\par 9 Bo gdy przyszli az do doliny Eschol, obejrzawszy one ziemie popsowali serce synom Izraelskim, aby nie szli do ziemi, która im dal Pan;
\par 10 Skad zapaliwszy sie gniewem Pan, dnia onego przysiagl, mówiac:
\par 11 Zaiste nie ogladaja ludzie ci, którzy wyszli z Egiptu, od dwudziestu lat i wyzej, tej ziemi, o któram przysiagl Abrahamowi, Izaakowi, i Jakóbowi, przeto iz mie cale nie nasladowali;
\par 12 Oprócz Kaleba, syna Jefunowego, Kenezejczyka, i Jozuego, syna Nunowego, poniewaz ci cale nasladowali Pana.
\par 13 I zapalil sie gniewem Pan na Izraela, i sprawil, ze sie tulali po puszczy przez czterdziesci lat, az poginal wszystek on naród, który czynil zle przed oczyma Panskimi.
\par 14 A oto, wy powstaliscie miasto ojców waszych, plemie ludzi grzesznych, abyscie jeszcze przyczynili gniewu zapalczywosci Panskiej przeciwko Izraelowi.
\par 15 Bo jezli sie odwrócicie od nasladowania jego, tedy on tez zaniecha go jeszcze na tej puszczy; a tak wy zgubicie ten wszystek lud.
\par 16 Tedy przystapiwszy do niego rzekli: Obory bydlu i dobytkowi naszemu, i miasta dziatkom naszym tu pobudujemy;
\par 17 Ale sami zbrojno ochotnie pójdziemy przed syny Izraelskimi, az je zaprowadzimy na miejsce ich, a dziatki nasze beda mieszkaly w miesciech obronnych dla obywateli tej ziemi.
\par 18 Nie wrócimy sie do domów naszych, az posiada synowie Izraelscy kazdy dziedzictwo swoje;
\par 19 Ani wezmiemy dziedzictwa z nimi za Jordanem i dalej, poniewaz przychodzi dziedzictwo nasze na nas z tej strony Jordanu na wschód slonca.
\par 20 I rzekl im Mojzesz: Jezliz uczynicie, coscie rzekli, a pójdziecie zbrojno przed obliczem Panskiem na wojne;
\par 21 I pójdzie kazdy z was zbrojno za Jordan przed oblicznoscia Panska, azby wypedzil nieprzyjacioly swoje od oblicza swego;
\par 22 I az bedzie poddana ziemia ona Panu, a potem sie wrócicie, i bedziecie bez winy przed Panem i przed Izraelem: tedy wam bedzie ta ziemia za osiadlosc przed obliczem Panskiem.
\par 23 Ale jezli tego nie uczynicie, oto zgrzeszycie Panu, a wiedzcie, ze grzech wasz znajdzie was.
\par 24 Budujciez tedy miasta dziatkom waszym, i obory bydlu waszemu, a co wyszlo z ust waszych, uczyncie.
\par 25 Tedy rzekli synowie Gadowi, i synowie Rubenowi do Mojzesza, mówiac: Sludzy twoi uczynia, jako pan nasz rozkazuje.
\par 26 Dziatki nasze, i zony nasze, stada nasze, i wszystkie bydla nasze, zostana tu w miesciech Galaadzkich;
\par 27 Ale sludzy twoi pójda wszyscy zbrojno przed Panem na wojne, jako pan nasz mówi.
\par 28 I przykazal o nich Mojzesz Eleazarowi kaplanowi, i Jozuemu, synowi Nunowemu, i ksiazetom ojców pokolen synów Izraelskich,
\par 29 I rzekl im: Jezli przejda synowie Gadowi, i synowe Rubenowi z wami za Jordan, wszyscy zbrojno na wojne przed Panem, a bedzie poholdowana ziemia przed wami, tedy im dacie ziemie Galaad w dziedzictwo;
\par 30 Ale jezli nie zbrojno z wami przejda, tedy niech maja dziedzictwo miedzy wami w ziemi Chananejskiej.
\par 31 I odpowiedzieli synowie Gadowi, i synowie Rubenowi, mówiac: Co wyrzekl Pan do slug swoich, to uczynimy;
\par 32 Pójdziemy zbrojno przed Panem do ziemi Chananejskiej, a zostanie przy nas osiadlosc dziedzictwa naszego z tej strony Jordanu.
\par 33 Dal tedy Mojzesz synom Gadowym, i synom Rubenowym, i polowie pokolenia Manasesa, syna Józefowego, królestwo Sehona, króla Amorejskiego, i królestwo Oga, króla Basanskiego, ziemie z miasty jej, z granicami, i miasta ziemi onej w okolo.
\par 34 I pobudowali synowie Gadowi Dybon, i Atarot, i Aroer;
\par 35 I Atrot, i Sofan, i Jazer, i Jegba,
\par 36 I Betnimera, i Betaran, miasta obronne, i obory dla bydla.
\par 37 Synowie tez Rubenowi pobudowali Hesebon, i Eleale, i Karyjataim.
\par 38 I Nebo, i Baalmeon, odmieniwszy im imiona, takze Sabana; i dali imiona insze onym miastom, które pobudowali.
\par 39 Wtargneli tez synowie Machyra, syna Manasesowego, do Galaad, a wziawszy je, wygnali Amorejczyka, który tam mieszkal.
\par 40 I dal Mojzesz Galaad Machyrowi, synowi Manasesowemu, i mieszkal w nim.
\par 41 Potem Jair, syn Manasesów, wtargnal, i pobral wsi ich, które przezwal Chawot Jair.
\par 42 Takze Nobe wtargnal, i wzial Kanat z jego wsiami, i nazwal je Nobe od imienia swego.

\chapter{33}

\par 1 Tec sa ciagnienia synów Izraelskich, którzy wyszli z ziemi Egipskiej wedlug hufów swych pod sprawa Mojzesza i Aarona.
\par 2 I spisal Mojzesz wychodzenia ich, i stanowiska ich wedlug rozkazania Panskiego. A tec sa ciagnienia ich, i stanowiska ich:
\par 3 Naprzód wyciagnawszy z Ramesses, miesiaca pierwszego, pietnastego dnia tegoz pierwszego miesiaca, nazajutrz po swiecie przejscia, wyszli synowie Izraelscy reka wyniosla przed oczyma wszystkich Egipczanów;
\par 4 Gdy Egipczanie grzebli one, które byl Pan miedzy nimi pomordowal, to jest, wszystko pierworodztwo, i gdy nad bogami ich wykonal Pan sady.
\par 5 Ruszywszy sie tedy synowie Izraelscy z Ramesses, polozyli sie obozem w Suchot.
\par 6 Ruszywszy sie z Suchot, polozyli sie obozem w Etam, które jest przy koncu puszczy.
\par 7 A ruszywszy sie z Etam, nawrócili sie do Fihahyrot, które jest przeciw Baalsefon, i polozyli sie obozem przed Migdolem.
\par 8 A ruszywszy sie z Fihahyrot, przeszli przez posrodek morza na puszcza, i uszedlszy trzy dni drogi po puszczy Etam, polozyli sie obozem w Mara.
\par 9 A ruszywszy sie z Mara, przyszli do Elim; a w Elim bylo dwanascie zródel wód, i siedemdziesiat drzew palmowych, i polozyli sie tam obozem.
\par 10 A ruszywszy sie z Elim, polozyli sie obozem nad morzem czerwonem.
\par 11 A ruszywszy sie od morza czerwonego, polozyli sie obozem na puszczy Syn.
\par 12 A ruszywszy sie z puszczy Syn, polozyli sie obozem w Dafka.
\par 13 A ruszywszy sie z Dafka, polozyli sie obozem w Alus.
\par 14 A ruszywszy sie z Alus, polozyli sie obozem w Rafidym, gdzie nie mial lud wód dla napoju.
\par 15 A ruszywszy sie z Rafidym, polozyli sie obozem na puszczy Synaj.
\par 16 A ruszywszy sie z puszczy Synaj, polozyli sie obozem w Kibrot hataawa.
\par 17 A ruszywszy sie z Kibrot hataawa, polozyli sie obozem w Hezerot.
\par 18 A ruszywszy sie z Hezerot, polozyli sie obozem w Retma.
\par 19 A ruszywszy sie z Retma, polozyli sie obozem w Remmon Fares.
\par 20 A ruszywszy sie z Remmon Fares, polozyli sie obozem w Lebna.
\par 21 A ruszywszy sie z Lebna, polozyli sie obozem w Ressa.
\par 22 A ruszywszy sie z Ressa, polozyli sie obozem w Kieelata.
\par 23 A ruszywszy sie z Kieelata, polozyli sie obozem na górze Sefer.
\par 24 A ruszywszy sie z góry Sefer, polozyli sie obozem w Charada.
\par 25 A ruszywszy sie z Charada, polozyli sie obozem w Makelot.
\par 26 A ruszywszy sie z Makelot, polozyli sie obozem w Tahat.
\par 27 A ruszywszy sie z Tahatu, polozyli sie obozem w Tare.
\par 28 A ruszywszy sie z Tare, polozyli sie obozem w Metka.
\par 29 A ruszywszy sie z Metka, polozyli sie obozem w Hesman.
\par 30 A ruszywszy sie z Hesman, polozyli sie obozem w Moserot.
\par 31 A ruszywszy sie z Moserot, polozyli sie obozem w Benejaakan.
\par 32 A ruszywszy sie z Benejaakan, polozyli sie obozem w Horgidgad.
\par 33 A ruszywszy sie z Horgidgad, polozyli sie obozem u Jotbata.
\par 34 A ruszywszy sie z Jotbata, polozyli sie obozem w Habrona.
\par 35 A ruszywszy sie z Habrona, polozyli sie obozem w Asyjongaber.
\par 36 A ruszywszy sie z Asyjongaber, polozyli sie obozem na puszczy Syn, która jest Kades.
\par 37 A ruszywszy sie z Kades, polozyli sie obozem na górze Hor, na granicach ziemi Edomskiej.
\par 38 Tedy wstapil Aaron kaplan na góre Hor wedlug rozkazania Panskiego, i tam umarl roku czterdziestego po wyjsciu synów Izraelskich z ziemi Egipskiej, miesiaca piatego, pierwszego dnia onego miesiaca.
\par 39 A mial Aaron sto dwadziescia i trzy lat, gdy umarl na górze Hor.
\par 40 Tam uslyszal Chananejczyk, król Arad, który mieszkal na poludnie, w ziemi Chananejskiej, ze ciagneli synowie Izraelscy.
\par 41 A ruszywszy sie z góry Hor, polozyli sie obozem w Salmona.
\par 42 A ruszywszy sie z Salmona, polozyli sie obozem w Funon.
\par 43 A ruszywszy sie z Funon, polozyli sie obozem w Obot.
\par 44 A ruszywszy sie z Obot, polozyli sie obozem przy pagórkach Abarym, na granicy Moabskiej.
\par 45 A ruszywszy sie z Abarym, polozyli sie obozem w Dybon Gat.
\par 46 A ruszywszy sie z Dybon Gat, polozyli sie obozem w Helmon Dyblataim.
\par 47 A ruszywszy sie z Helmon Dyblataim, polozyli sie obozem na górach Abarym, przeciwko Nebo.
\par 48 A ruszywszy sie z gór Abarym, polozyli sie obozem na polach Moabskich, nad Jordanem, przeciw Jerychu.
\par 49 I tam sie polozyli nad Jordanem, od Betiesymot az do Abelsytym, na polach Moabskich.
\par 50 I rzekl Pan do Mojzesza na polach Moabskich, nad Jordanem, przeciw Jerychu, mówiac:
\par 51 Mów do synów Izraelskich, a powiedz im: Gdy przejdziecie za Jordan do ziemi Chananejskiej,
\par 52 Tedy wypedzcie wszystkie obywatele onej ziemi od oblicza waszego, i wytraccie wszystkie malowania ich, i wszystkie obrazy balwanów ich wygubcie, takze wszystkie wyzyny ich spustoszcie.
\par 53 A wypedziwszy obywatele ziemi, mieszkac bedziecie w niej; bom wam dal te ziemie w dziedzictwo.
\par 54 I wezmiecie w dziedzictwo te ziemie losem, wedlug domów waszych; których wiecej, tym wieksze dziedzictwo dacie, a których mniej, tym mniejsze dziedzictwo dacie, a które miejsce losem na kogo przypadnie, to miec bedzie; wedlug pokolenia ojców waszych dziedzictwo brac bedziecie.
\par 55 Ale jezliz nie wypedzicie obywateli tej ziemi od oblicza waszego, tedy oni, które pozostawicie z nich, beda wam jako zadla w oczach waszych, i jako ciernie na boki wasze, i beda was trapic w tej ziemi, w której wy mieszkac bedziecie.
\par 56 I stanie sie, ze com umyslil onym uczynic, wam uczynie.

\chapter{34}

\par 1 Potem rzekl Pan do Mojzesza, mówiac:
\par 2 Rozkaz synom Izraelskim, a powiedz im: Gdy wnijdziecie do ziemi Chananejskiej, (tac jest ziemia, która sie wam dostanie za dziedzictwo, ziemia Chananejska z granicami swemi.)
\par 3 Tedy bedzie granica wasza ku poludniowi, od puszczy Syn az do granic Edomskich, która granica poludniowa pójdzie od brzegu morza slonego na wschód slonca.
\par 4 I okrazy ta granica od poludnia do Maaleakrabim, i pójdzie az ku puszczy Syn, i pójdzie od poludnia do Kades Barne; a stamtad wynijdzie do wsi Addar, i pójdzie az do Asmon.
\par 5 A zakrazy ta granica od Asmon az do rzeki Egipskiej, a skonczy sie na zachód.
\par 6 Zachodnia zas granice bedziecie mieli morze wielkie; to wam bedzie granica od zachodu.
\par 7 To zas wasza bedzie granica pólnocna; od morza wielkiego wymierzycie sobie do góry Hor.
\par 8 Potem od góry Hor wymierzycie granice, az gdzie wchodza do Hemat; a beda sie konczyc granice az do Sedada.
\par 9 I pójdzie ta granica az do Zefronu, a skonczy sie u wsi Enan; te bedziecie miec granice pólnocna.
\par 10 Granice tez od wschodu wymierzycie od wsi Enan az do Sefama.
\par 11 A pójdzie ta granica od Sefama az do Reblat, od wschodu miasta Ain; i uda sie ta granica i dojdzie do brzegu morza Cyneret na wschód slonca.
\par 12 A przyjdzie ta granica az ku Jordanu, a skonczy sie u morza slonego. Tac bedzie ziemia wasza w granicach swoich w okolo.
\par 13 Tedy oznajmil Mojzesz synom Izraelskim, mówiac: Tac jest ziemia, która dziedzicznie otrzymacie losem, jako rozkazal Pan, abym ja dal dziewieciorgu pokoleniu, i polowie pokolenia Manasesowego.
\par 14 Bo wzielo pokolenie synów Rubenowych wedlug domów ojców swych, i pokolenie synów Gadowych wedlug domów ojców swych, i polowa pokolenia Manasesowego wzieli dziedzictwo swoje.
\par 15 Dwa pokolenia, i pól pokolenia, wziely dziedzictwo swoje z tej strony Jordanu przeciw Jerychu, ku stronie na wschód slonca.
\par 16 I rzekl Pan do Mojzesza, mówiac:
\par 17 Tec sa imiona mezów, którzy wam podziela w dziedzictwo ziemie: Eleazar kaplan, i Jozue, syn Nunów.
\par 18 Ksiecia takze jednego z kazdego pokolenia wezmiecie dla podzielenia w dziedzictwo ziemi.
\par 19 A tec sa imiona tych mezów: z pokolenia Juda Kaleb, syn Jefunów;
\par 20 A z pokolenia synów Symeonowych Samuel, syn Ammiudów.
\par 21 Z pokolenia Benjamin Eliad, syn Chaselenów.
\par 22 A z pokolenia synów Danowych ksiaze Buki, syn Jogolów.
\par 23 Z synów Józefowych z pokolenia synów Manasesowych ksiaze Haniel, syn Efodów.
\par 24 A z pokolenia synów Efraimowych ksiaze Chemuel, syn Seftanów.
\par 25 Z pokolenia zas Zabulonowego ksiaze Elisafan, syn Farnatów.
\par 26 A z pokolenia synów Isascharowych ksiaze Faltijel, syn Ozanów.
\par 27 Z pokolenia synów Aserowych ksiaze Ahiud, syn Salomiego.
\par 28 A z pokolenia synów Neftalimowych ksiaze Fedael, syn Ammiudów.
\par 29 Cic sa, którym rozkazal Pan, aby dali dziedzictwo synom Izraelskim w ziemi Chananejskiej.

\chapter{35}

\par 1 I rzekl Pan do Mojzesza na polach Moabskich, nad Jordanem przeciw Jerychowi, mówiac:
\par 2 Rozkaz synom Izraelskim, aby dali Lewitom z dziedzictwa osiadlosci swojej miasta do mieszkania, i przedmiescia okolo miast ich oddacie Lewitom.
\par 3 I beda mieli miasta sobie do mieszkania, a przedmiescia ich beda im dla bydla ich, i dla majetnosci ich, i dla wszystkiego dobytku ich.
\par 4 A przedmiescia miast, które dacie Lewitom, od muru miejskiego pójda na tysiac lokci wszedy w okolo.
\par 5 Przetoz wymierzycie za kazdem miastem dwa tysiace lokci na wschód slonca, na poludnie tez dwa tysiace lokci, takze na zachód dwa tysiace lokci, i na pólnocy dwa tysiace lokci, a miasto w posrodku bedzie; takowec beda przedmiescia miast ich.
\par 6 A miedzy temi miasty, które dacie Lewitom, szesc miast beda dla ucieczki, które dacie, aby tam uciekal mezobójca; a nad te dacie im czterdziesci miast i dwa.
\par 7 Tak iz wszystkich miast, które Lewitom dacie, bedzie czterdziesci i osiem miast i z przedmiesciami ich.
\par 8 A miast, które dacie z dzierzaw synów Izraelskich, od tych, którzy wiecej maja, wiecej dacie, a od tych, którzy mniej maja, dacie mniej; kazdy wedlug miary dziedzictwa swego, które posiedzie, udzieli z miast swoich Lewitom.
\par 9 Zatem rzekl Pan do Mojzesza, mówiac:
\par 10 Mów do synów Izraelskich, i rzecz im: Gdy przejdziecie przez Jordan do ziemi Chananejskiej.
\par 11 Postanowciez sobie miasta; miasta dla ucieczki miec bedziecie, aby tam uciekal mezobójca, któryby zabil kogo z nieobaczenia.
\par 12 A beda wam te miasta dla ucieczki przed powinowatym zabitego, aby nie dal gardla ten co zabil, póki by nie stanal przed zgromadzeniem na sad.
\par 13 A miast, które odlaczycie, szesc miast dla ucieczki miec bedziecie.
\par 14 Trzy miasta dacie z tej strony Jordanu, a trzy miasta dacie w ziemi Chananejskiej; te miasta dla ucieczki beda.
\par 15 Synom Izraelskim, i przychodniowi, i mieszkajacemu miedzy nimi, beda te szesc miast do ucieczki, aby tam uciekl kazdy, kto by zabil czlowieka z nieobaczenia.
\par 16 Wszakze, jezliby go zelazna bronia uderzyl, tak zeby umarl, mezobójca jest; smiercia umrze on mezobójca.
\par 17 Albo jezliby majac kamien w reku, którym by mógl zabic, uderzyl go, tak zeby umarl, mezobójca jest; smiercia umrze on mezobójca.
\par 18 Takze jezliby majac w reku drewno, którem by mógl zabic, uderzyl go, i umarlby, mezobójca jest; smiercia umrze on mezobójca.
\par 19 Powinowaty zabitego zabije tego mezobójce; gdziekolwiek sie z nim spotka, on zabije go.
\par 20 A jezliby kogo z nienawisci popchnal, albo nan czem cisnal z zasadzki, a umarlby;
\par 21 Albo jezliby go z wasni uderzyl reka swoja, a umarlby, smiercia umrze ten, który uderzyl, mezobójca jest; powinny zabitego zabije mezobójce, gdziekolwiek go trafi.
\par 22 Ale jezliby z przygody bez wasni kogo popchnal, alboby nan cisnal czemkolwiek nie umyslnie;
\par 23 Albo jezliby jakim kamieniem, od którego by mógl umrzec, rzucil nan z nieobaczenia, a umarlby, nie bedac mu nieprzyjacielem, ani szukajac jego zlego:
\par 24 Tedy rozsadek uczyni zgromadzenie miedzy tym, który zabil, a miedzy powinnym zabitego wedlug tego prawa.
\par 25 I wybawi zgromadzenie mezobójce tego z rak powinnego onego zabitego, i kaze mu sie wrócic zgromadzenie do miasta ucieczki jego, gdzie byl uciekl; i tamze bedzie mieszkal az do smierci kaplana najwyzszego, który jest pomazany olejkiem swietym.
\par 26 A jezliby wyszedl mezobójca za granice miasta ucieczki swojej, do którego uciekl;
\par 27 I trafilby go powinny zabitego za granica miasta ucieczki jego, chociazby zabil powinny zabitego mezobójce onego, nie bedzie winien krwi.
\par 28 Albowiem w miescie ucieczki swojej mieszkac ma az do smierci kaplana najwyzszego, a po smierci kaplana najwyzszego wróci sie on mezobójca do ziemi osiadlosci swojej.
\par 29 A bedziecie to mieli za ustawe prawna w narodziech waszych, we wszystkich mieszkaniach waszych.
\par 30 Ktobykolwiek chcial zabic czlowieka, za swiadectwem swiadków zabije mezobójce; ale swiadek jeden nie bedzie mógl swiadczyc na skazanie kogo na smierc.
\par 31 Nie wezmiecie tez okupu za zywot mezobójcy, który zasluzyl smierc; niech smiercia umrze.
\par 32 Nie wezmiecie tez zaplaty od onego, który uciekl do miasta ucieczki swojej, aby sie nawrócil na mieszkanie do ziemi swojej, pierwej nizby kaplan umarl:
\par 33 Byscie nie splugawili ziemie, w której bedziecie; bo krew takowa splugawilaby ziemie; a ziemia nie moze byc oczyszczona od krwi, która jest wylana na niej, jedno krwia tego, który ja przelal.
\par 34 Przetoz nie plugawcie ziemi, w której mieszkacie, w której Ja tez mieszkam; bom Ja Pan, który mieszkam miedzy synami Izraelskimi.

\chapter{36}

\par 1 Tedy przystapili mezowie przedniejsi z ojców pokolenia synów Galaada, syna Machyrowego, syna Manasesowego, z domów Józefowych, i mówili przed Mojzeszem, i przed ksiazety przedniejszymi ojców synów Izraelskich, i rzekli:
\par 2 Tobie, panu memu, rozkazal Pan, abys podzielil ziemie w dziedzictwo losem synom Izraelskim; nadto panu memu rozkazano od Pana, abys dal dziedzictwo Salfaada, brata naszego, córkom jego.
\par 3 Które jezliby kto z inszego pokolenia synów Izraelskich wzial za zony, odjete bedzie ich dziedzictwo od dziedzictwa ojców naszych, a przylaczy sie do dziedzictwa onego pokolenia, do którego by je wzieto za zony, a tak z losu dziedzictwa naszego ubedzie.
\par 4 A gdy przyjdzie milosciwe lato synom Izraelskim, tedy przylaczone bedzie dziedzictwo ich do dziedzictwa onego pokolenia, do którego by poszly za maz; a tak od dziedzictwa pokolenia ojców naszych odjete bedzie dziedzictwo ich.
\par 5 Tedy powiedzial Mojzesz synom Izraelskim wedlug slowa Panskiego, mówiac: Dobrze mówi pokolenie synów Józefowych.
\par 6 Toc to jest, co rozkazal Pan o córkach Salfaadowych, mówiac: Jako sie im upodoba, niech ida za maz; tylko w domu pokolenia ojców swoich niech ida za maz.
\par 7 Aby nie bylo przenoszone dziedzictwo synów Izraelskich z pokolenia na pokolenie; bo kazdy z synów Izraelskich zostawac ma przy dziedzictwie pokolenia ojców swych.
\par 8 I kazda córka, która by miala dziedzictwo z pokolen synów Izraelskich, za kogokolwiek z domu pokolenia ojca swego pójdzie, zeby otrzymali dziedziczenie synowie Izraelscy, kazdy dziedzictwo ojców swych.
\par 9 Bo nie ma byc przenoszone dziedzictwo, z pokolenia na pokolenie insze; ale kazdy przy dziedzictwie swojem zostac ma z pokolenia synów Izraelskich.
\par 10 Jako tedy rozkazal Pan Mojzeszowi, tak uczynily córki Salfaadowe.
\par 11 Bo Mahala, Tersa, i Hegla, i Melcha, i Noa, córki Salfaadowe, szly za maz, za syny stryjów swoich.
\par 12 W domy synów Manasesa, syna Józefowego poszly za maz; i tak zostalo dziedzictwo ich przy pokoleniu domu ojca ich.
\par 13 Tec sa przykazania i prawa, które rozkazal Pan przez Mojzesza synom Izraelskim na polach Moabskich, nad Jordanem przeciw Jerychowi.


\end{document}