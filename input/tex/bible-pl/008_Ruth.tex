\begin{document}

\title{Rut}


\chapter{1}

\par 1 I stalo sie za onych czasów, kiedy sedziowie sadzili, byl glód w ziemi; i poszedl niektóry maz z Betlehem Juda na mieszkanie do ziemi Moabskiej z zona swoja i z dwoma synami swymi.
\par 2 A imie meza onego bylo Elimelech, imie tez zony jego Noemi, takze imiona dwóch synów jego Mahalon i Chelijon, a ci byli Efratejczykami z Betlehem Juda, którzy zaszedlszy do krainy Moabskiej, mieszkali tam.
\par 3 Potem umarl Elimelech, maz Noemi, a ona pozostala z dwoma synami swoimi.
\par 4 I pojeli sobie zony Moabskie; imie jednej Orfa, a drugiej imie Rut; i mieszkali tam okolo dziesieciu lat.
\par 5 Umarli potem i oni oba, Mahalon i Chelijon; i tak ona niewiasta osierociala po obu synach swoich i po mezu swoim.
\par 6 A wezbrawszy sie z synowemi swemi wrócila sie z ziemi Moabskiej; bo slyszala w krainie Moabskiej, ze byl nawiedzil Pan lud swój, i dal im chleb.
\par 7 Tedy wyszla z miejsca, na którem byla z onemi dwiema synowemi swemi, a udaly sie w droge, aby sie wrócily do ziemi Juda.
\par 8 Zatem rzekla Noemi do dwóch synowych swych: Idzciez, wróccie sie kazda do domu matki twojej; niechaj uczyni Pan z wami milosierdzie, jakoscie uczynily z umarlymi synami moimi i ze mna.
\par 9 Niech wam da Pan znalezc odpocznienie, kazdej w domu meza swego; i pocalowala je, a one podnióslszy glos swój, plakaly,
\par 10 I mówily do niej: Raczej sie z toba wrócimy do ludu twojego.
\par 11 A Noemi rzekla: Wrócciez sie córki moje; przeczzebyscie ze mna isc mialy? Azaz ja jeszcze moge miec syny, którzyby byli mezami waszymi?
\par 12 Wrócciez sie córki moje, a idzcie, bom sie juz zstarzala, a nie moge isc za maz. Chocbym tez rzekla, jeszcze mam nadzieje, albo chocbym tez dobrze tej nocy byla za mezem chocbym tez nawet i porodzila syny;
\par 13 Izali wy ich czekac bedziecie, azby dorosli? zaz sie dla tego zatrzymacie, abyscie nie szly za maz? Nie tak córki moje; bo zalosc moja wieksza jest nizli wasza, gdyz sie obrócila przeciwko mnie reka Panska.
\par 14 Ale one podnióslszy glos swój, znowu plakaly. I pocalowala Orfa swiekre swoje; a Ruta zostala przy niej.
\par 15 Do której Noemi rzekla: Oto sie wrócila powinna twoja do ludu swego, i do bogów swoich, wrócze sie i ty za powinna swoja.
\par 16 Na co jej odpowiedziala Ruta: Nie wiedz mie do tego, abym cie opuscic i od ciebie odejsc miala; owszem gdziekolwiek pójdziesz, z toba pójde, a gdziekolwiek mieszkac bedziesz, z toba mieszkac bede; lud twój lud mój, a Bóg twój Bóg mój.
\par 17 Gdzie umrzesz, tam i ja umre, i tam pogrzebiona bede. To mi niech uczyni Pan, i to niech przepusci na mie, ze tylko smierc rozlaczy mie z toba.
\par 18 A tak ona widzac, ze sie na to uparla, aby z nia szla, przestala jej odradzac.
\par 19 I szly obie pospolu, az przyszly do Betlehem. I stalo sie, gdy przyszly do Betlehem, wzruszylo sie wszystko miasto dla nich, mówiac: Izaz nie ta jest Noemi?
\par 20 Ale ona mówila do nich: Nie nazywajcie mie Noemi, ale mie zowcie Mara; albowiem mie gorzkoscie wielka Wszechmogacy napelnil.
\par 21 Wyszlam stad obfita a prózna mie przywrócil Pan. Przeczze mie tedy zowiecie Noemi, gdyz mie Pan utrapil, a Wszechmogacy zle na mie dopuscil?
\par 22 A tak wrócila sie Noemi i Rut Moabitka, synowa jej, z nia; wrócila sie z krainy Moabskiej, i przyszly do Betlehem na poczatku zniwa jeczmiennego.

\chapter{2}

\par 1 A Noemi miala powinowatego po mezu swym, czlowieka moznego z domu Elimelechowego, którego zwano Booz.
\par 2 I rzekla Ruta Moabitka do Noemi: Pójde prosze na pole, a niech zbieram klosy za tym, przed którego oczyma laske znajde; a ona rzekla: Idz, córko moja.
\par 3 Szla tedy, a przyszedlszy zbierala na polu za zencami; i trafilo sie, ze przyszla na dzial pola Boozowego, który byl z domu Elimelechowego.
\par 4 A wtem przyszedl Booz z Betlehem, i rzekl do zenców: Pan z wami. A oni mu odpowiedzieli: Niechzec Pan blogoslawi.
\par 5 Rzekl tedy Booz do slugi swego, który byl przystawem nad zencami: Czyjaz to dzieweczka?
\par 6 I odpowiedzial mu sluga on, który byl przystawem nad zencami, i rzekl: Ta dzieweczka jest Moabitka, która przyszla z Noemi z ziemi Moabskiej.
\par 7 I rzekla mi: Niech prosze zbieram i zgromadzam klosy miedzy snopami za zencami; a przyszedlszy bawi sie tu od samego poranku az dotad, a bardzo malo w domu siedzi.
\par 8 Tedy rzekl Booz do Rut: Sluchaj mie córko moja; nie chodz zbierac klosów na insze pole i nie odchodz stad, ale sie tu trzymaj dziewek moich.
\par 9 Pilnuj tego pola, na którem zac beda, a chodz za nimi; bom rozkazal slugom moim, zeby sie ciebie zaden nie tykal; a jezli upragniesz, idz do naczynia, a napij sie z tego, co czerpia sludzy moi.
\par 10 Tedy ona upadlszy na oblicze swoje, ukloniwszy sie az do ziemi rzekla do niego: Skadzem znalazla laske w oczach twoich, iz mie znasz, gdyzem jest cudzoziemka?
\par 11 I odpowiedzial Booz, a rzekl jej: Powiedziano mi zapewne wszystko, cos uczynila swiekrze twojej po smierci meza twego, a jakos opusciwszy ojca twego i matke twoje, i ziemie, w którejs sie urodzila, przyszla do ludu, któregos nie znala przedtem.
\par 12 Niechzec odda Pan uczynek twój, i niech bedzie zaplata twoja doskonala od Pana, Boga Izraelskiego, gdyzes przyszla, abys nadzieje miala pod skrzydlami jego.
\par 13 A ona rzekla: Znalazlam laske w oczach twoich, panie mój, gdyzes mie pocieszyl, a mówiles do serca sluzebnicy twojej, chociam ja nie jest, jako jedna z sluzebnic twoich.
\par 14 I rzekl jej Booz: Gdy bedzie czas jedzenia, przychodz tu, a jedz chleb, omoczywszy sztuczke twoje w occie. I usiadla przy zencach, i podal jej prazma, które jadla az do sytosci, i jeszcze jej zbylo.
\par 15 Potem wstala, aby zbierala; a Booz rozkazal slugom swoim mówiac: Niech i miedzy snopami zbiera, a nie broncie jej tego.
\par 16 Owszem umyslnie upuszczajcie jej z snopów, a zostawiajcie, aby zbierala, a nie fukajcie na nia.
\par 17 A tak zbierala na onem polu az do wieczora; i wymlócila to, co zebrala, i miala jakoby z efe jeczmienia.
\par 18 A wziawszy to, szla do miasta, i ogladala swiekra jej to, co nazbierala; a wyjawszy dala jej i to, co jej zostalo, gdy sie najadla.
\par 19 I rzekla do niej swiekra jej: Kedyzes dzis zbierala, a gdzies robila? niechajze ten, który na cie mial baczenie, blogoslawionym bedzie. I oznajmila swiekrze swej, u kogo robila, mówiac: Imie meza, u któregom dzis robila, Booz.
\par 20 Potem rzekla Noemi do synowy swojej: Niech bedzie blogoslawionym od Pana, który nie zawsciagnal milosierdzia swego od zywych i od umarlych. Nadto jeszcze rzekla Noemi: Ten maz jest powinowatym naszym, i z pokrewnych naszych.
\par 21 Rzekla jej tez Rut Moabitka: Nadto mi jeszcze mówil on maz: Trzymaj sie czeladzi mojej, póki nie pozna wszystkiego zboza mego.
\par 22 Tedy rzekla Noemi do Ruty, synowy swej: Dobrze, córko moja, iz bedziesz chodzila z dziewkami jego, zebyc kto przeciwnym nie byl na polu innem.
\par 23 Przetoz sie trzymala sluzebnic Boozowych, i zbierala klosy, póki sie nie skonczylo zniwo jeczmienne, i zniwo pszeniczne. Potem mieszkala u swiekry swojej.

\chapter{3}

\par 1 Potem rzekla do niej Noemi, swiekra jej: Córko moja, azazemci nie powinna szukac odpocznienia, zebys sie dobrze miala?
\par 2 A teraz azaz Booz nie jest powinowatym naszym, z któregos ty sluzebnicami byla? Oto on bedzie wial jeczmien na bojewisku tej nocy.
\par 3 Przetoz umywszy sie, namaz sie olejkami; wezmij tez szaty twoje na sie, a idz na bojewisko, a nie daj sie widziec mezowi onemu, azby sie najadl i napil.
\par 4 A gdy on spac pójdzie, upatrzze miejsce, na którem sie ukladzie, a przyszedlszy odkryjesz plaszcz z nóg jego, a tam sie ukladziesz, a on tobie oznajmi, co bedziesz miala czynic.
\par 5 I rzekla do niej Rut: Cokolwiek mi kazesz, uczynie.
\par 6 A tak szla na one bojewisko, i uczynila to, co jej rozkazala swiekra jej.
\par 7 A gdy sie najadl Booz i napil, i rozweselilo sie serce jego, poszedl a ukladl sie przy stogu; przyszla tez i ona po cichu, a odkrywszy plaszcz z nóg jego, ukladla sie.
\par 8 A gdy bylo o pólnocy, ulakl sie on maz, a obróciwszy sie, ujrzal, a oto, niewiasta lezy u nóg jego.
\par 9 I rzekl: Któzes ty? I odpowiedziala: Jam jest Rut, sluzebnica twoja; rozciagnijze plaszcz twój na sluzebnice twoje, bos mi pokrewny.
\par 10 A on rzekl: Blogoslawionas ty od Pana, córko moja; wiekszas poboznosc po sobie pokazala teraz niz pierwej, zes nie poszla za mlodziencami tak ubogimi jako i bogatymi;
\par 11 Przetoz teraz, córko moja, nie bój sie; bo wszystko, cokolwiek rzeczesz, uczynie, gdyz wie cale miasto ludu mego, zes ty niewiasta cnotliwa.
\par 12 A teraz prawdac to, zem ja jest pokrewny twój, wszakze jeszcze jest pokrewny blizszy nad mie.
\par 13 Zostanze tu tej nocy. A gdy bedzie rano, jezli cie bedzie chcial pojac prawem bliskosci, dobrze, niech pojmie; jezli cie nie bedzie chcial pojac, ja cie pojme prawem bliskosci; zywie Pan! Spijze tu az do poranku.
\par 14 A tak spala u nóg jego az do poranku, a wstala przedtem niz mógl rozeznac jeden drugiego; bo rzekl Booz: Niech nikt nie wie, ze przyszla ta niewiasta na bojewisko.
\par 15 Nadto rzekl: Daj plachte, która masz na sobie, a trzymaj ja; a gdy ja trzymala, namierzyl jej szesc miarek jeczmienia, i zalozyl na nie, i weszla do miasta.
\par 16 A przyszla do swiekry swej, która jej spytala: Któzes ty córko moja? A Rut jej powiedziala wszystko, co jej uczynil on maz,
\par 17 I rzekla: Oto szesc miarek tego jeczmienia dal mi; bo rzekl do mnie: Nie wrócisz sie prózno do swiekry twojej.
\par 18 I rzekla Noemi: Potrwajze, córko moja, az sie dowiesz, jako padnie ta rzecz; boc nie zaniecha ten maz, az te rzecz dzis skonczy.

\chapter{4}

\par 1 Potem Booz szedl do bramy, i usiadl tam; a oto, pokrewny on szedl mimo, o którym powiedzial byl Booz; i rzekl mu: Pójdz sam a siadz tu, ty a ty; a on przyszedlszy siadl.
\par 2 Wziawszy tedy dziesiec mezów starszych miasta onego, mówil do nich: Siadzciez tez tu; i usiedli.
\par 3 Zatem rzekl onemu powinowatemu: Dzial roli, który byl brata naszego Elimelecha, sprzedala Noemi, która sie wrócila z ziemi Moabskiej.
\par 4 I zdalo mi sie to odniesc do uszu twoich, mówiac: Otrzymaj te role przed tymi, którzy tu siedza, i przed starszymi ludu mego; a chceszli ja odkupic, odkup; a jezliz nie odkupisz, powiedz mi; bo wiem, ze nad cie niemasz blizszego do wykupienia, a jam po tobie. Tedy on rzekl: Ja odkupie.
\par 5 Nadto rzekl Booz: Dnia, którego otrzymasz role z rak Noemi, tedy tez i Rute Moabitke, zone zmarlego, pojmiesz, abys wzbudzil imie zmarlego w dziedzictwie jego.
\par 6 Odpowiedzial powinowaty: Nie moge odkupic, bym snac nie stracil dziedzictwa mego. Odkupze ty sobie bliskosc moje, gdyz ja nie moge odkupic jej.
\par 7 (A byl to starodawny zwyczaj w Izraelu przy wykupie, i przy zamianie, aby warowniejsza byla kazda sprawa, tedy zzuwal jeden z nich trzewik swój, i dawal go blizniemu swojemu; a toc bylo na swiadectwo ustepowania dóbr w Izraelu.)
\par 8 Tedy rzekl on powinowaty do Booza: Otrzymajze ty; i zzul trzewik swój.
\par 9 Zatem rzekl Booz do onych starszych, i do wszystkiego ludu: Swiadkami dzis jestescie wy, zem to wszystko otrzymal, co bylo Elimelechowe, i to wszystko, co bylo Chelijonowe, i Mahalonowe, z rak Noemi.
\par 10 Do tego Rute Moabitke, zone Machalonowe, wzialem sobie za zone, abym wzbudzil imie zmarlego w dziedzictwie jego, izby nie zginelo imie onego zmarlego miedzy bracia jego, i z bramy miejsca jego; tego wyscie dzisiaj swiadkami.
\par 11 I rzekl wszystek lud, który byl w bramie miasta, i starsi: Jestesmy swiadkami; niech ci da Pan, aby niewiasta, która wchodzi w dom twój, byla jako Rachel i jako Lija, które obie zbudowaly dom Izraelski. Poczynajze sobie meznie w Efracie, zjednaj sobie imie w Betlehemie.
\par 12 Niechajze dom twój bedzie jako dom Faresa, (którego porodzila Tamar Judzie,)z nasienia tego, którec da Pan z tej to bialej glowy.
\par 13 A tak pojal sobie Booz Rute, i byla mu za zone; a gdy wszedl do niej, tedy jej dal Pan, ze poczela, i porodzila syna.
\par 14 I rzekly niewiasty do Noemi: Blogoslawiony Pan, który cie dzis nie chcial miec bez powinowatego, aby zostalo imie jego w Izraelu.
\par 15 Tenci ucieszy dusze twoje, i bedzie cie zywil w starosci twojej; albowiem synowa twoja, która cie miluje, porodzila go, którac daleko jest lepsza, nizeli siedm synów.
\par 16 A tak wziawszy Noemi dzieciatko, polozyla je na lonie swojem, a byla mu za piastunke.
\par 17 I daly mu sasiady imie, mówiac: Narodzil sie syn Noemi, i nazwaly imie jego Obed; tenci jest ojciec Isajego ojca Dawidowego.
\par 18 A tec sa rodzaje Faresowe: Fares splodzil Hesrona;
\par 19 A Hesron splodzil Rama, a Ram splodzil Aminadaba;
\par 20 A Aminadab splodzil Nahasona, a Nahason splodzil Salmona;
\par 21 A Salmon splodzil Booza, a Booz splodzil Obeda;
\par 22 A Obed splodzil Isajego, a Isaj splodzil Dawida.


\end{document}