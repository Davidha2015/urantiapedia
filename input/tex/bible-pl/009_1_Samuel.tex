\begin{document}

\title{1 Samuela}


\chapter{1}

\par 1 Byl niektóry maz z Ramataim Sofim, z góry Efraim, któremu bylo imie Elkana, syn Jerohama, syna Elihu, syna Tuhu, syna Suf, Efratejczyka.
\par 2 A ten mial dwie zony, imie jednaj Anna, a imie drugiej Fenenna; i miala Fenenna dziatki, ale Anna nie miala dziatek.
\par 3 I chadzal on maz z miasta swego na kazdy rok, aby chwale dawal i ofiarowal Panu zastepów w Sylo, gdzie byli dwaj synowie Heli, Ofni i Finees, kaplani Panscy.
\par 4 A gdy przyszedl dzien, którego sprawowal ofiary Elkana, dal Fenennie, zonie swej, i wszystkim synom jej, i córkom jej, czastki;
\par 5 Ale Annie dal jedne czesc wyborna; bo Anne milowal, chociaz byl Pan zywot jej zamknal.
\par 6 I draznila ja bardzo przeciwnica jej, aby ja tylko rozgniewala, dla tego, iz zamknal byl Pan zywot jej.
\par 7 To gdy czynil Elkana na kazdy rok, a Anna tez chodzila do domu Panskiego, tak ja draznila przeciwnica, ze plakiwala i nie jadala.
\par 8 Rzekl jej tedy Elkana, maz jej: Anno, czemu placzesz i czemu nie jesz? a przecz sie tak trapi serce twe? izalim ja tobie nie jest lepszy niz dziesiec synów?
\par 9 Wstala tedy Anna, gdy sie najedli i napili w Sylo; a Heli kaplan siedzial na stolku u podwoja kosciola Panskiego.
\par 10 A ona bedac w gorzkosci serca, modlila sie Panu, i wielce plakala.
\par 11 I uczynila slub, mówiac: Panie zastepów, jezliz wejrzawszy wejrzysz na utrapienie sluzebnicy twojej, i wspomnisz na mie, a nie zabaczysz sluzebnicy twojej, i dasz sluzebnicy twojej potomstwo meskiej plci, tedy je dam Panu po wszystkie dni zywota jego, a brzytwa nie postoi na glowie jego.
\par 12 I stalo sie, gdy przedluzala modlitwy przed Panem, ze Heli przypatrowal sie ustom jej.
\par 13 Ale Anna mówila w sercu swem, tylko wargi jej ruchaly sie, ale glosu jej slychac nie bylo; i mial ja Heli za pijana.
\par 14 Przetoz rzekl do niej Heli: Dlugoz bedziesz pijana? wytrzezwij sie z wina twego.
\par 15 Ale odpowiedziala Anna i rzekla: Nie tak, panie mój, niewiasta utrapionego ducha jestem, anim wina ani napoju mocnego nie pila, alem wylala dusze moje przed obliczem Panskiem.
\par 16 Nie rozumiejze o sluzebnicy twojej, jako o niewiescie niepoboznej, gdyz z wielkiego myslenia i frasunku mego mówilam az dotad.
\par 17 Tedy odpowiedzial Heli, i rzekl: Idzze w pokoju, a Bóg Izraelski niech ci da prosbe twoje, którejs zadala od niego.
\par 18 I rzekla: Niech znajdzie sluzebnica twoja laske przed oczyma twemi; i odeszla niewiasta w droge swa, i jadla, a twarz jej nie byla wiecej smetna.
\par 19 I wstali bardzo rano, a pokloniwszy sie przed Panem, wrócili sie, i przyszli do domu swego do Ramata. Tedy poznal Elkana Anne, zone swa, a wspomnial na nie Pan.
\par 20 I stalo sie po wypelnieniu dni, jako poczela Anna, ze porodzila syna, i nazwala imie jego Samuel; bo rzekla: U Panam go uprosila.
\par 21 Potem szedl on maz Elkana, i wszystek dom jego, aby oddal Panu ofiare uroczysta, i slub swój.
\par 22 Ale Anna nie szla; bo mówila mezowi swemu: Nie pójde, az zostawie dzieciatko, potem odwiode je, ze sie ukaze przed Panem, i zostanie tam zawsze,
\par 23 I rzekl jej Elkana, maz jej: Uczyn co jest dobrego w oczach twoich, zostan, az go zostawisz; tylko niech utwierdzi Pan slowo swoje. Zostala tedy niewiasta, i karmila piersiami syna swego, az go zostawila.
\par 24 A gdy go zostawila, przywiodla go z soba ze trzema cielcami, i z jednem efa maki, i z lagwia wina, i przywiodla go do domu Panskiego w Sylo; a dziecie bylo male.
\par 25 I zabiwszy cielca, przywiedli dziecie do Heli.
\par 26 A ona rzekla: Sluchaj, panie mój! zywie dusza twoja, panie mój: Jam jest ona niewiasta, któram tu stala przy tobie, modlac sie Panu.
\par 27 Prosilam o to dzieciatko, i dal mi Pan prosbe moje, którejm zadala od niego.
\par 28 Przetoz je tez ja oddawam Panu; na wszystkie dni, których bedzie zylo, jest oddane Panu. I poklonili sie tam Panu.

\chapter{2}

\par 1 Tedy sie modlila Anna, i rzekla: Rozweselilo sie serce moje w Panu, wywyzszon jest róg mój w Panu, rozszerzyly sie usta moje przeciw nieprzyjaciolom moim; albowiemem sie rozradowala w zbawieniu twojem.
\par 2 Niemaszci swietego jako Pan; bo niemasz innego oprócz ciebie, i niemasz tak mocnego, jak Bóg nasz.
\par 3 Niemówciez napotem slów pysznych, a niech nie wychodza slowa harde z ust waszych; albowiem Bóg jest umiejetnosci Panem, a nadawaja sie sprawy jego.
\par 4 Luk i mocarze pokruszeni sa, a mdli przepasani sa moca.
\par 5 Którzy byli nasyceni, najmuja sie za chleb, a glodni przestali laknac; tak iz nieplodna siedmioro porodzila, a która rodzila wiele dziatek, zemdlala.
\par 6 Pan zabija i ozywia, wwodzi do grobu i wywodzi.
\par 7 Pan ubogiego czyni i zbogaca, uniza i wywyzsza.
\par 8 Wzbudza z prochu ubogiego, a z gnoju podnosi zebraka, aby je posadzil z ksiazety, a dal im stolice chwalebna osiadac; albowiem Panskie sa grunty ziemi, a na nich zalozyl swiat.
\par 9 Nóg swietych swoich ochrania, a niepobozni w ciemnosciach zamilkna; bo nie w sile swojej bedzie sie maz zmacnial.
\par 10 Pan pokruszy przeciwniki swoje, a zagrzmi na nie z nieba; Pan bedzie sadzil granice ziemi, a da moc królowi swemu, i wywyzszy róg pomazanca swego.
\par 11 A tak odszedl Elkana do Ramaty do domu swego, a dziecie sluzylo Panu przed Heli kaplanem.
\par 12 Ale synowie Heli byli synowie bezbozni, a nie znali Pana.
\par 13 Albowiem obyczaj kaplanów ten byl okolo ludu: ktokolwiek sprawowal ofiary, przychodzil sluga kaplanski, gdy warzono mieso, majac widelki o trzech zebach w rece swojej.
\par 14 I wrazal je w statek, albo w kociel, albo w panew, albo w garniec, a coklowiek wyjal widelkami, to sobie bral kaplan. Tak czynili wszystkim Izraelczykom, którzy tam do Sylo przychodzili.
\par 15 Takze pierwej niz zapalono tlustosc, tedy przychodzil sluga kaplanski, a mówil do czlowieka ofiarujacego: Oddaj mieso, abym je upiekl kaplanowi; albowiem nie wezmie od ciebie miesa warzonego, jedno surowe.
\par 16 A jezliz mu odpowiedzial on czlowiek: Niech sie pierwej spali tlustosc, potem sobie wezmiesz, czego bedzie zadala dusza twoje, tedy on mówil: Nic z tego; teraz daj! a nie daszli, wezme gwaltem.
\par 17 I byl to grzech onych slug bardzo wielki przed Panem; bo sie odtracali ludzie od ofiar Panskich.
\par 18 Ale Samuel sluzyl przed Panem, ubrane chlopiatko w efod lniany.
\par 19 A matka jego uczyniwszy mu sukienke mala, przynaszala mu co rok, gdy chadzala z mezem swym sprawowac ofiare uroczysta.
\par 20 I blogoslawil Heli Elkanie, i zonie jego, mówiac: Niech ci da Pan potomstwo z tej niewiasty za oddanego, którego wyprosila u Pana. I poszli na miejsca swoje.
\par 21 Tedy nawiedzil Pan Anne, która poczela i porodzila trzech synów, i dwie córek; a pachole Samuel urósl przed Panem.
\par 22 Ale Heli zstarzal sie byl bardzo, i slyszal wszystko, co czynili synowie jego calemu Izraelowi, i jako sypiali z niewiastami, które sie schadzaly przede drzwi namiotu zgromadzenia.
\par 23 I rzekl do nich: Przeczze takie rzeczy czynicie? Cóz ja slysze o waszych zlych sprawach od wszystkiego ludu?
\par 24 Nie tak synowie moi; bo nie dobra slawa, która ja slysze, ze przywodzicie ku przestepstwu lud Panski.
\par 25 Gdy kto zgrzeszy przeciw czlowiekowi, sadzic go bedzie sedzia, ale jezli przeciw Panu kto zgrzeszy, któz sie za nim ujmie? Lecz nie usluchali glosu ojca swego; bo je chcial Pan pobic.
\par 26 Ale pachole Samuel postepowal a rósl, i podobal sie tak Panu jako i ludziom.
\par 27 Potem przyszedl maz Bozy do Heli, i rzekl mu: Tak mówi Pan: Azalim sie nie jawnie objawil domowi ojca twego, gdy byli w Egipcie w domu Faraonowym?
\par 28 I obralem go sobie ze wszystkich pokolen Izraelskich za kaplana, aby ofiarowal na oltarzu moim, a kadzil rzeczami wonnemi, i nosil efod przedemna, i dalem domowi ojca twego wszystkie ofiary palone od synów Izraelskich.
\par 29 Przeczzescie podeptali ofiare moje, i sniedna ofiare moje, któram rozkazal sprawowac w przybytku? i wiecejs uczcil syny swoje nad mie, abyscie sie utuczyli z pierwocin wszystkich ofiar sniednych Izraela, ludu mego?
\par 30 Prztoz mówi Pan, Bóg Izraelski: Rzeklem wprawdzie: Dom twój i dom ojca twego bedzie sluzyl przedemna az na wieki; ale teraz mówi Pan: Nie bedziec to, gdyz ja te, którzy nie czcza, czcic bede, a którzy mna gardza, beda wzgardzeni.
\par 31 Oto, dni przychodza, a odetne ramie twe, i ramie domu ojca twego, aby nie bylo starca w domu twoim;
\par 32 I ogladasz wielki ucisk przybytku Panskiego, miasto szczescia, które Pan dawal Izraelowi, i nie bedzie starca w domu twoim po wszystkie dni.
\par 33 Wszakze meza nie wytrace z ciebie do konca od oltarza mego, abym utrapil oczy twe, a bolescia scisnal dusze twoje; a wszystko mnóstwo domu twego pomrze, doroslszy lat meskich.
\par 34 A toc bedzie na znak, co przyjdzie na dwóch synów twoich, Ofni i Fineesa; dnia jednego pomra ci oba.
\par 35 I wzbudze sobie kaplana wiernego, który wedlug serca mego, i wedlug mysli mojej czynic bedzie, i zbuduje mu dom trwaly, a bedzie sluzyl przed pomazancem moim po wszystkie dni.
\par 36 I stanie sie, ktokolwiek pozostanie z domu twego, przyjdzie, aby mu sie uklonil za pieniadz srebrny i za sztuke chleba, mówiac: Przypusc mie prosze do jednej czastki kaplanskiej, aby jadl sztuczke chleba.

\chapter{3}

\par 1 A pachole Samuel sluzyl Panu przed Heli, a slowo Panskie bylo drogie w one dni, bo nie bywalo widzenia jawnego.
\par 2 I stalo sie dnia onego, gdy Heli lezal na miejscu swojem, (a oczy jego juz sie byly poczely zaciemniac, i nie mógl dojrzec.)
\par 3 A lampa Boza jeszcze nie byla zagaszona, Samuel tez spal w kosciele Panskim, gdzie byla skrzynia Boza,
\par 4 Ze zawolal Pan na Samuela, a on sie ozwal: Owom ja.
\par 5 I przybiezal do Heliego i rzekl: Owom ja, gdyzes mie wolal. A on rzekl: Nie wolalem, wróc sie, spij; i poszedl a spal.
\par 6 Powtóre Pan jeszcze zawolal Samuela; i wstal Samuel, a poszedl do Heliego, i rzekl: Owom ja, gdyzes mie wolal; któremu on rzekl: nie wolalem, synu mój, wróc sie a spij.
\par 7 A Samuel jeszcze nie znal Pana, i jeszcze mu nie bylo objawione slowo Panskie.
\par 8 Nadto jeszcze Pan zawolal Samuela po trzecie; a on wstawszy szedl do Heliego i rzekl: Owom ja, gdyzes mie wolal. Tedy zrozumial Heli, ze Pan wolal pacholecia.
\par 9 I rzekl Heli do Samuela: Idz, spij, a jezli cie kto zawola, rzeczesz: Mów Panie, bo slyszy sluga twój. A tak Samuel szedl i spal na miejscu swojem.
\par 10 Potem przyszedl Pan, i stanal a zawolal jako i pierwszy i drugi raz: Samuelu, Samuelu! I rzekl Samuel: Mów Panie, bo sluga twój slucha.
\par 11 Tedy rzekl Pan do Samuela: Oto, Ja uczynie rzecz w Izraelu, która ktokolwiek uslyszy, zabrzmi mu w obu uszach jego.
\par 12 Dnia onego wzbudze przeciw Heliemu wszystko, com mówil przeciwko domowi jego; poczne i dokonam.
\par 13 I okaze mu, iz Ja sadze dom jego az na wieki dla nieprawosci, o której wiedzial; bo wiedzac, ze na sie przeklenstwo przywodzili synowie jego, wszakze nie bronil im tego.
\par 14 A przetoz przysiaglem domowi Heli, ze nie bedzie oczyszczona nieprawosc domu Heliego zadna ofiara, ani ofiara sniedna, az na wieki.
\par 15 I spal Samuel az do poranku, i otworzyl drzwi domu Panskiego. A Samuel bal sie oznajmic widzenia tego Heliemu.
\par 16 Tedy zawolal Heli Samuela, i rzekl: Samuelu, synu mój; który odpowiedzial: Otom ja.
\par 17 I rzekl: Cóz to za slowa, którec Pan powiedzial? prosze nie taj przedemna; to a toc Bóg niechaj uczyni, jezlize co zataisz przedemna ze wszystkich slów, które mówil do ciebie.
\par 18 I oznajmil mu Samuel wszystkie slowa, a nie zatail nic przed nim. A on rzekl: Pan jest; co dobrego w oczach jego, niech czyni.
\par 19 I rósl Samuel, a Pan byl z nim, i nie dopuscil upasc zadnemu ze wszystkich slug jego na ziemie.
\par 20 Poznal tedy wszystek Izrael od Dan az do Beerseba, iz Samuel byl wiernym prorokiem Panu.
\par 21 Bo i napotem ukazywal sie Pan Samuelowi w Sylo, tak jako mu sie przedtem objawil Pan w Sylo przez slowo swoje.

\chapter{4}

\par 1 I stalo sie wedlug mowy Samuelowej wszystkiemu Izraelowi. Bo gdy wyciagnal Izrael przeciw Filistynom na wojne, a polozyl sie obozem u Ebenezer, Filistynowie zas polozyli sie obozem w Afeku;
\par 2 I gdy sie uszykowali Filistynowie przeciwko Izraelowi, a stoczyla sie bitwa: tedy porazony jest Izrael od Filistynów, a pobito ich w onej bitwie na polu okolo czterech tysiecy mezów.
\par 3 I wrócil sie lud do obozu. I rzekli starsi Izraelscy: Przeczze nas dzis porazil Pan przed Filistynami? wezmijmyz do siebie z Sylo skrzynie przymierza Panskiego, a niech przyjdzie miedzy nas, a wybawi nas z rak nieprzyjaciól naszych.
\par 4 Przetoz poslal lud do Sylo, i wzieli stamtad skrzynie przymierza Pana zastepów, siedzacego na Cherubinach; byli tez tam dwaj synowie Heli z skrzynia przymierza Panskiego, Ofni i Finees.
\par 5 A gdy przyszla skrzynia przymierza Panskiego do obozu, zakrzyknal wszystek Izrael glosem wielkim, tak iz ziemia zabrzmiala.
\par 6 A uslyszawszy Filistynowie glos onego krzyku, rzekli: Cóz to za glos tak wielkiego wykrzykania w obozie Hebrejskim? I poznali, ze skrzynia Panska przyszla do obozu.
\par 7 Przetoz zlekli sie Filistynowie, gdyz mówiono: Przyszedl Bóg do obozu ich, i rzekli: Biada nam! bo nie bylo nic takowego przedtem.
\par 8 Biadaz nam! któz nas wybawi z rak tych Bogów mocnych? cic to bogowie, którzy porazili Egipt wszelka plaga na puszczy.
\par 9 Zmacniajciez sie, a badzcie mezami, o Filistynowie! byscie snac nie sluzyli Hebrejczykom, jako oni wam sluzyli. Badzciez tedy mezami, a potykajcie sie.
\par 10 Zwiedli tedy bitwe Filistynowie, i porazony jest Izrael, a uciekal kazdy do namiotu swego; i stala sie porazka bardzo wielka, tak iz poleglo z Izraela trzydziesci tysiecy piechoty.
\par 11 Tamze skrzynia Boza wzieta jest, i dwaj synowie Heli polegli, Ofni i Finees.
\par 12 I biezal niektóry z synów Benjaminowych z bitwy, a przyszedl do Sylo tegoz dnia, majac szaty rozdarte, a proch na glowie swojej.
\par 13 A gdy przyszedl, oto, Heli siedzial na stolku przy drodze wygladajac, bo sie serce jego lekalo o skrzynie Boza; a przyszedlszy on maz, opowiedzal miastu, i krzyczalo wszysytko miasto.
\par 14 A uslyszawszy Heli glos krzyku onego, rzekl: Cóz to za glos rozruchu tego? lecz on maz spieszac sie, przyszedl, aby to oznajmil Heliemu.
\par 15 A Heli juz mial dziewiedziesiat i osm lat, a oczy jego juz sie byly zacmily, ze nie mógl dojrzec.
\par 16 Tedy rzekl on maz do Heliego: Ja ide z bitwy, jam zaiste z bitwy dzis uciekl. Do którego on rzekl: Cóz sie tam stalo, synu mój?
\par 17 I odpowiedzal on posel, i rzekl: Uciekl Izrael przed Filistynami i stala sie wielka porazka ludu; tamze i dwaj synowie twoi legli, Ofni i Finees, i skrzynia Boza wzieta jest.
\par 18 A gdy wspomnial skrzynie Boza, spadl Heli z stolka na wznak u bramy, a zlamawszy sobie szyje umarl; albowiem byl czlowiek stary i ociezaly. A on sadzil Izraela przez czterdziesci lat.
\par 19 Synowa tez jego, zona Fineesowa, bedac brzemienna i bliska porodzenia, gdy uslyszala ona nowine, iz wzieta jest skrzynia Boza, i ze umarl swiekier jej, i maz jej, tedy sie nachylila, i porodzila; bo przypadly na nie bole jej.
\par 20 A gdy umierala, rzekly niewiasty, które byly przy niej: Nie bój sie, albowiemes syna porodzila; ale ona nic nie odpowiedziala, ani tego przypuscila do serca swego.
\par 21 I nazwala dzieciatko Ichabod, mówiac: Przeprowadzila sie slawa od Izraela, iz wzieto skrzynie Boza, a iz umarl swiekier jej, i maz jej.
\par 22 Prztoz rzekla: Przeprowadzila sie slawa od Izraela; bo wzieto skrzynie Boza.

\chapter{5}

\par 1 Tedy Filistynowie wzieli skrzynie Boza, i zaniesli ja z Ebenezer do Azotu.
\par 2 Wziawszy tedy Filistynowie one skrzynie Boza, wprowadzili ja do domu Dagonowego, i postawili ja podle Dagona.
\par 3 A gdy rano wstali Azotczanie nazajutrz, oto, Dagon lezal twarza swoja na ziemi przed skrzynia Panska; i poniesli Dagona, i postawili go na miejscu jego.
\par 4 A gdy zas wstali rano nazajutrz, oto, Dagon lezal twarza swoja na ziemi przed skrzynia Panska; a leb Dagonowy i obie dlonie rak jego ulamane byly na progu, tylko sam pien Dagonowy zostal podle niej.
\par 5 Przetoz nie wstepuja kaplani Dagonowi, i wszyscy, którzy wchodza do domu Dagonowego, na próg Dagonowy w Azocie, az do dnia tego.
\par 6 Tedy byla ciezka reka Panska nad Azotczany, a gubila je; bo je zarazala wrzodami na zadnicach, w Azocie i w granicach jego.
\par 7 A widzac mezowie z Azotu, co sie dzialo, rzekli: Niechaj nie zostawa skrzynia Boga Izraelskiego z nami; albowiem sroga jest reka jego przeciwko nam, i przeciwko Dagonowi, bogu naszemu.
\par 8 A tak obeslali i zebrali wszystkie ksiazeta Filistynskie do siebie, i mówili: Cóz uczynimy z skrzynia Boga Izraelskiego? I odpowiedzieli: Do Gad niech bedzie doprowadzona skrzynia Boga Izraelskiego; i odprowadzono tam skrzynie Boga Izraelskiego.
\par 9 A gdy ja odprowadzili, powstala reka Panska przeciw miastu trpieniem bardzo wielkiem, i zarazala meze miasta od malego az do wielkiego, i naczynilo sie im wrzodów na skrytych miejscach.
\par 10 Odeslali tedy skrzynie Boza do Akkaronu; a gdy przyszla skrzynia Boza do Akkaronu, krzyczeli Akkaronczycy, mówiac: Przyprowadzono do nas skrzynie Boga Izraelskiego, aby nas wymordowano z ludem naszym.
\par 11 Przetoz poslawszy zgromadzili wszystkie ksiazeta Filistynskie, i rzekli: Odeslijcie skrzynie Boga Izraelskiego, a niech sie wróci na miejsce swoje, i niech nas nie zabija i ludu naszego; bo byl strach smierci po wszystkiem miescie, a byla tam bardzo ciezka reka Boza.
\par 12 A mezowie którzy nie pomarli, zarazeni byli wrzodami na zadnicy, tak, iz wstepowal krzyk miasta do nieba.

\chapter{6}

\par 1 I byla skrzynia Panska w ziemi Filistynskiej przez siedm miesiecy.
\par 2 Tedy przyzwawszy Filistynowie kaplanów i wieszczków, rzekli: Cóz uczynimy z skrzynia Panska? powiedzcie nam, jako ja odeslac mamy na miejsce jej?
\par 3 Którzy odpowiedzieli: Jezli odeslecie skrzynie Boga Izraelskiego, nie odsylajciez jej próznej, ale przy niej koniecznie oddajcie ofiare za przewinienie; tedyc bedziecie uzdrowieni, i dowiecie sie, czemu nie odstapila reka jego od was.
\par 4 I rzekli: Jakaz bedzie ofiara za przewinienie, która jej oddac mamy? Odpowiedzieli: Wedlug liczby ksiazat Filistynskich piec zlotych zadnic i piec zlotych myszy; albowiem jednaka jest plaga na was wszystkich, i na ksiazeta wasze.
\par 5 A poczynicie podobienstwa zadnic waszych, i podobienstwa myszy waszych, które psowaly ziemie, i oddacie Bogu Izraelskiemu chwale; owa snac ulzy reki swej nad wami, i nad Bogami waszymi, i nad ziemia wasza.
\par 6 A czemuz obciazacie serce wasze, jako obciazali Egipczanie i Farao serce swoje? izaz nie dopiero, gdy dziwne rzeczy nad nimi czynil wypuscili je i wyszli?
\par 7 Przetoz teraz sprawcie wóz nowy jeden, a wezmijcie dwie krowy od cielat, na których nie postalo jarzmo, i zaprzezcie te krowy w wóz, a cieleta ich od nich odwiedzcie do domu.
\par 8 Wezmijcie tez skrzynie Panska, i wstawcie ja na wóz; a sztuki zlote, którescie ofiarowali za przewinienie, wlózcie w skrzynke po bok jej, a pusccie ja, ze pójdzie.
\par 9 A patrzajcie, jezli droga granic swych pójdzie do Betsemes, tedyc on na nas dopuscil to wielkie zle; a jezliz nie, tedy poznamy, ze nie reka jego dotknela sie nas, ale to z trafunku przyszlo na nas.
\par 10 I uczynili tak oni mezowie, a wziawszy dwie krowy od cielat, zaprzegli je w wóz, a cieleta ich zamkneli w domu.
\par 11 Potem wstawili skrzynie Panska na wóz, i skrzynke, i myszy zlote, i podobienstwa zadnic swoich.
\par 12 I udaly sie one krowy droga, prosto ku Betsemes, a goscincem jednym idac szly, a ryczaly; i nie zstepowaly ani w prawo ani w lewo a ksiazeta Filistynskie szly za nimi az do granic Betsemes.
\par 13 A na ten czas Betsemczycy zeli pszenice w dolinie, a podnióslszy oczów swych ujrzeli skrzynie, i uradowali sie ujrzawszy ja.
\par 14 A gdy wóz przyszedl na pole Jozuego Betsemity, tamze stanal. Tam tez byl kamien wielki; tedy porabawszy drwa od onego wozu, ofiarowali one krowy na calopalenie Panu.
\par 15 Ale Lewitowie zstawili skrzynie Panska, i skrzynke, która byla z nia, w której byly sztuki zlote, i postawili na onym kamieniu wielkim; a mezowie z Betsemes sprawowali calopalenia, i ofiarowali ofiary Panu onego dnia.
\par 16 Co widzac piecioro ksiazat Filistynskich, wrócili sie do Akkaronu onegoz dnia.
\par 17 A tec byly zadnice zlote, które oddali Filistynowie za przewinienie Panu: Od Azotu jedne, od Gazy jedne, od Aszkalonu jedne, od Gat jedne, i od Akkaronu jedne.
\par 18 Myszy takze zlote wedlug liczby wszytskich miast Filistynskich, od pieciu ksiestw, poczawszy od miasta murowanego az do wsi bez muru, i az do kamienia onego wielkiego, na którym postawili skrzynie Panska, który jest az do dnia tego na polu Jozuego Betsemity.
\par 19 Ale pobil Pan niektóre z mezów Betsemitskich, przeto iz zagladali w skrzynie Panska, i pobil z ludu piecdziesiat tysiecy i siedmdziesiat mezów; i plakal lud, przeto ze Pan lud wielka porazka porazil.
\par 20 I rzekli mezowie z Betsemes: Któz sie bedzie mógl ostac przed obliczem Pana, Boga tego swietego? i do kogoz pójdzie od nas?
\par 21 A tak wyprawili posly do obywateli Karyjatyjarym mówiac: Przywrócili Filistynowie skrzynie Panska; pójdzcie, przeprowadzcie ja do siebie.

\chapter{7}

\par 1 Przyszli tedy mezowie z Karyjatyjarym, i odwiezli skrzynie Panska, a wniesli ja do domu Abinadabowego w Gabaa; a Eleazara syna jego poswiecili, aby strzegl skrzyni Panskiej.
\par 2 I stalo sie, gdy od onego dnia, jako zostala skrzynia w Karyjatyjarym, wyszedl niemaly czas, to jest dwadziescia lat, ze plakal wszystek dom Izraelski za Panem.
\par 3 I rzekl Samuel do wszystkiego domu Izraelskiego mówiac: Jezlize ze wszystkiego serca waszego nawracacie sie do Pana, wyrzucciez bogi cudze z posrodku siebie, i Astarota, a zgotujcie serce wasze Panu, i sluzcie jemu samemu, tedyc was wybawi z reki Filistynów.
\par 4 Przetoz wyrzucili synowie Izraelscy Baala i Astarota, a sluzyli Panu samemu.
\par 5 Tedy rzekl Samuel: Zgromadzcie wszystkiego Izraela do Masfa, abym sie modlil za wami Panu.
\par 6 A tak zgromadzili sie do Masfa, a czerpajac wode, wylewali przed Panem, i poscili tam dnia onego, mówiac: Zgrzeszylismy Panu. I sadzil Samuel syny Izraelskie w Masfa.
\par 7 A gdy uslyszeli Filistynowie, ze sie zgromadzili synowie Izraelscy do Masfa, ruszyly sie ksiazeta Filisynskie przeciw Izraelowi. Co gdy uslyszeli synowie Izraelscy, zlekli sie przed Filistynami.
\par 8 I rzekli synowie Izraelscy do Samuela: Nie przestawaj za nami wolac do Pana, Boga naszego, aby nas wybawil z reki Filistynów.
\par 9 Przetoz wzial Samuela baranka ssacego jednego, i ofiarowal go calego na calopalenie Panu; i wolal Samuel do Pana za Izraelem, a wysluchal go Pan.
\par 10 I stalo sie, gdy Samuel sprawowal calopalenie, ze Filistynowie przyciagneli blisko, aby walczyli przeciw Izraelowi; ale zagrzmial Pan grzmotem wielkim dnia onego nad Filistynami, a potarl je, i porazeni sa przed obliczem Izraela.
\par 11 A mezowie Izraelscy wypadlszy z Masfa, gonili Filistyny, i bili je az pod Betchar.
\par 12 Tedy wzial Samuel kamien jeden, i postawil go miedzy Masfa a miedzy Sen, i nazwal imie jego Ebenezer, mówiac: Az póty pomagal nam Pan.
\par 13 A tak ponizeni sa Filistynowie, a potem wiecej nie przychodzili na granice Izraelska; albowiem byla reka Panska przeciwko Filistynom po wszystkie dni Samuelowe.
\par 14 I przywrócone sa miasta Izraelowi, które byli wzieli Filistynowie Izraelowi, od Akkaronu az do Get, i granice ich oswobodzil Izrael z reki Filistynów; i byl pokój miedzy Izraelem, i miedzy Amorejczykiem.
\par 15 I sadzil Samuel Izraela po wszystkie dni zywota swego.
\par 16 A chodzac na kazdy rok, obchodzil Betel, i Gilgal, i Masfa, sadzac Izraela po onych wszystkich miejscach.
\par 17 Potem sie wracal do Ramaty; bo tam byl dom jego, i tam sadzil Izraela; tamze tez zbudowal oltarz Panu.

\chapter{8}

\par 1 I stalo sie, gdy sie zstarzal Samuel, ze postanowil syny swe sedziami nad Izraelem.
\par 2 A bylo imie syna jego pierworodnego Joel, a imie drugiego syna jego Abija; ci byli sedziami w Beerseba.
\par 3 Ale nie chodzili synowie jego drogami jego; lecz udali sie za lakomstwem, i brali dary, a wywracali sad.
\par 4 Przetoz zebrali sie wszyscy starsi Izraelscy, i przyszli do Samuela do Ramaty,
\par 5 I rzekli mu: Otos sie ty zstarzal, a synowie twoi nie chodza drogami twojemi; przetoz postanów nam króla, aby nas sadzil, jako je wszystkie narody maja.
\par 6 Ale sie nie podobala ta rzecz Samuelowi, ze mówili: Daj nam króla, aby nas sadzil; przetoz modlil sie Samuel Panu.
\par 7 Tedy rzekl Pan do Samuela: Usluchaj glosu ludu tego we wszystkiem, coc powiedza; albowiem nie toba wzgardzili, ale mna wzgardzili, izbym nie królowal nad nimi.
\par 8 A wedlug wszystkich spraw, które czynili od onego dnia, któregom je wywiódl z Egiptu, az do dnia tego, gdy mie opuscili i sluzyli bogom obcym, tak tez czynia i tobie.
\par 9 Przetoz teraz usluchaj glosu ich, a wszakze oswiadcz sie jako najpilniej przed nimi, i oznajmij im prawo króla, który nad nimi ma królowac.
\par 10 A tak Samuel odniósl wszystkie slowa Panskie do ludu, który go prosil o króla,
\par 11 I mówil: Toc bedzie prawo króla, który królowac ma nad wami: Syny wasze brac bedzie a osadzi nimi wozy swoje, i poczyni je jezdnymi, a beda biegac przed wozem jego;
\par 12 Poczyni tez sobie z nich pulkowniki nad tysiacami, i rotmistrze nad piecdziesiecioma; poczyni z nich oracze ról swoich, i zence zniwa swego, i te, którzyby robili rynsztunki wojenne, i potrzeby do wozów jego.
\par 13 Córki tez wasze pobierze, by gotowaly rzeczy wonne, i byly kucharkami i piekarkami.
\par 14 Pola tez wasze, i winnice wasze, i oliwnice wasze co najlepsze pobierze, a rozda slugom swoim.
\par 15 Przytem z zasiewków waszych, i z winnic waszych bedzie bral dziesieciny, i rozda je komornikom swoim, i slugom swoim.
\par 16 Takze slugi wasze, i dziewki wasze, i mlodzience wasze co najgrzeczniejsze bedzie bral, i osly wasze pobierze, i obróci do roboty swojej.
\par 17 Z bydla waszego dziesiecine bedzie bral, a wy bedziecie niewolnikami jego.
\par 18 I bedziecie wolac dnia onego dla króla waszego, którego sobie obierzecie, a nie wyslucha was Pan dnia onego.
\par 19 Ale nie chcial lud usluchac glosu Samuelowego; owszem mówili: Nic z tego; ale król niech bedzie nad nami,
\par 20 Abysmy byli i my, jako i wszystkie narody; bedzie nas sadzil król nasz, a wychodzac przed nami, bedzie odprawowal wojny nasze.
\par 21 A wysluchwaszy Samuel wszystkich slów ludu, odniósl je do uszu Panskich.
\par 22 I rzekl Pan do Samuela: Usluchaj glosu ich, a postanów nad nimi króla. Przetoz rzekl Samuel do mezów Izraelskich: Idzcie kazdy do miasta swego.

\chapter{9}

\par 1 I byl maz z pokolenia Benjamin, którego imie bylo Cys, syn Abijelów, syna Seror, syna Bechorat, syna Afija, syna meza Jemini, duzy w sile.
\par 2 Ten mial syna imieniem Saula, mlodzienca urodziwego, a nie bylo nikogo z synów Izraelskich urodziwszego naden; glowa byl wyzszy nad wszystek inny lud.
\par 3 A zginely byly oslice Cysowi, ojcu Saulowemu. I rzekl Cys do Saula, syna swego: Wezmij teraz z soba jednego z slug, a wstawszy idz, i szukaj oslic.
\par 4 Tedy on szedl przez góre Efraim, i przeszedl ziemie Salisa, lecz nie znalezli. Przeszli takze ziemie Salim, a nie znalezli. Nadto przeszli i ziemie Jemini, a nie znalezli.
\par 5 A przyszedlszy do ziemi Suf, rzekl Saul do slugi swego, który byl z nim: Pójdz, a wrócmy sie, by snac zaniechwaszy ojciec mój oslic, nie frasowal sie o nas,
\par 6 Który mu odpowiedzial: Oto teraz jest maz Bozy w tem miescie, a maz to zacny; cokolwiek powie, wszystko sie stawa; przetoz pójdzmy tam, snac nam powie o drodze naszej, która isc mamy.
\par 7 Tedy odpowiedzial Saul sludze swemu: Wiec pójdziemy; ale cóz przyniesiemy onemu mezowi? Bo chleba nie stalo w sumkach naszych, a podarku niemasz, którybysmy przyniesli mezowi Bozemu; cóz mamy?
\par 8 Tedy sluga znowu odpowiedzial Saulowi, i rzekl: Otom znalazl u siebie czwarta czesc sykla srebrnego, która damy mezowi Bozemu, aby nam oznajmil droge nasza.
\par 9 Przedtem w Izraelu tak mawial kazdy, gdy sie szedl radzic Boga: Chodzcie, a pójdziemy az do widzacego; bo którego dzis zowia prorokiem, tego przedtem nazywano widzacym.
\par 10 Tedy rzekl Saul do slugi swego: Dobre jest slowo twoje; chodz, pójdzmy, i szli do miasta, w którem byl maz Bozy.
\par 11 A gdy wstepowali na góre miasta, a potkali dzieweczki, wychodzace czerpac wode, rzekli im: A jestze tu widzacy?
\par 12 Które odpowiadajac im, rzekly: Jest, oto przed toba; spiesz sie tedy, dzis bowiem przyszedl do miasta, gdyz dzis ofiary sprawuje lud na górze.
\par 13 Skoro wnijdziecie do miasta, znajdziecie go, pierwej niz pójdzie na góre, aby jadl; albowiem lud nie bedzie jadl, az on przyjdzie; bo on bedzie blogoslawil ofierze, potem beda jesc wezwani. A przetoz idzcie, bo go o tej godzinie znajdziecie.
\par 14 Weszli tedy do miasta; a gdy przyszli w posrodek miasta, oto, Samuel wychodzil przeciwko nim, aby szedl na góre.
\par 15 A Pan objawil byl Samuelowi dzien przedtem, nizli Saul przyszedl, mówiac:
\par 16 O tym czasie jutro posle do ciebie meza z ziemi Benjamin, którego pomazesz za wodza nad ludem moim Izraelskim; a on wybawi lud mój z rak Filistynskich. Bom wejrzal na lud mój, gdyz przyszlo wolanie jego do mnie.
\par 17 A gdy Samuel wejrzal na Saula, rzekl mu Pan. Otóz maz, o którymemci powiedzial; tenci bedzie panowal nad ludem moim.
\par 18 A tak przystapil Saul do Samuela w posrodku bramy, i rzekl: Prosze powiedz mi, gdzie tu jest dom widzacego?
\par 19 I odpowiedzial Samuel, Saulowi, mówiac: Jam jest widzacy. Wstap przedemna na góre, a bedziecie dzis jedli ze mna; potem cie odprawie rano, a cokolwiek jest w sercu twem, oznajmie tobie.
\par 20 A o oslice, którec zginely dzis trzeci dzien, nie frasuj sie, boc sie znalazly.I czyjez wszystko co najlepszego w Izraelu? izali nie twoje, i nie wszystkiego domu ojca twego?
\par 21 A odpowiadajac Saul, rzekl: Izalim ja nie syn Jemini z najmniejszego pokolenia Izraelskiego? a dom mój azaz nie najpodlejszy miedzy wszystkiemi domy pokolenia Benjaminowego? Przeczzes tedy mówil do mnie takowe slowa?
\par 22 A tak wziawszy Samuel Saula i sluge jego, wywiódl je na sale, i dal im miejsce przedniejsze miedzy wezwanymi, których bylo okolo trzydziestu mezów.
\par 23 I rzekl Samuel kucharzowi: Daj sam te czastke, któram ci dal, i o którejm ci rzekl: Schowaj ja u siebie,
\par 24 A gdy przyniósl kucharz lopatke, i to, co bylo na niej, polozyl Samuel przed Saula, i rzekl: Oto, co zostalo, wezmij przed sie, a jedz; bo na ten czas schowano to dla ciebie, gdym rzekl: Wezwalem ludu. I jadl Saul z Samuelem dnia onego.
\par 25 A gdy zstapili z góry do miasta, rozmawial z Saulem na dachu.
\par 26 Potem wstali bardzo rano. I stalo sie, gdy sie poczelo rozedniewac, zawolal Samuel Saula na dach, mówiac: Wstan, a odprawie cie; wstawszy tedy Saul, wyszli obaj z domu, on i Samuel.
\par 27 A gdy schodzili ku koncowi miasta, rzekl Samuel do Saula: Rzecz sludze, aby szedl przed nami, i szedl; a ty pozostan troche, zec opowiem slowo Boze.

\chapter{10}

\par 1 Tedy Samuel wzial banke oliwy, i wylal na glowe jego, a pocalowawszy go, rzekl: Izali cie nie pomazal Pan nad dziedzictwem swojem za wodza?
\par 2 Gdy dzis odejdziesz ode mnie, znajdziesz dwóch mezów u grobu Racheli, na granicach Benjamin w Selsa, którzyc powiedza: Nalazly sie oslice, któryches chodzil szukac, a oto zaniechawszy ojciec twój starania o oslicach, frasuje sie o was, mówiac; Cóz mam czynic z strony syna mego?
\par 3 Potem odszedlszy stamtad dalej przyjdziesz az na pole Tabor; i spotkaja cie tam trzej mezowie idacy do Boga, do domu Bozego, jeden niesie troje kozlat, a drugi niesie trzy bochny chleba, a trzeci niesie lagiew wina;
\par 4 I pozdrowia cie w pokoju, i dadzac dwa chleby, które wezmiesz z rek ich.
\par 5 Potem przyjdziesz na pagórek Bozy, kedy jest straz Filistynska; a gdy tam wnijdziesz do miasta, spotkasz sie z gromada proroków zstepujacych z góry, a przed nimi bedzie harfa, i beben, i piszczalka, i lutnia, a oni beda prorokowali.
\par 6 I zstapi na cie Duch Panski, i bedziesz z nimi prorokowal, a odmienisz sie w inszego meza.
\par 7 A gdy przyjda te znaki na cie, czyn cokolwiek znajdzie reka twoja: bo Bóg jest z toba.
\par 8 Potem pójdziesz przedemna do Galgal, a oto, ja przyjde do ciebie dla sprawowania ofiar calopalnych, i dla ofiarowania ofiar spokojnych; przez siedm dni bedziesz czekal, az przyjde do ciebie i ukazec, co bedziesz mial czynic.
\par 9 I stalo sie, gdy sie obrócil, aby odszedl od Samuela, odmienil Bóg serce jego w insze; i spelnily sie wszystkie one znaki dnia onego.
\par 10 I przyszli tam na pagórek, a oto, gromada proroków spotkala sie z nim, i odpoczal na nim Duch Bozy, i prorokowal w posrodku nich.
\par 11 Stalo sie tedy, ze wszyscy, którzy go przedtem znali, ujrzeli, a oto, z prorokami prorokowal; i mówili wszyscy jeden do drugiego: Cóz sie stalo synowi Cysowemu? Izali tez Saul miedzy prorokami?
\par 12 I odpowiedzial maz niektóry stamtad, i rzekl: I któz jest ojcem ich? przetoz weszlo to w przypowiesc: Izali i Saul miedzy prorokami?
\par 13 I przestal prorokowac, a przyszedl na góre.
\par 14 Potem rzekl stryj Saula do niego, i do slugi jego: Gdziezescie chodzili? I odpowiedzial: Szukac oslic; a widzac, zesmy ich nie mogli znalesc, poszlismy do Samuela.
\par 15 I rzekl stryj Saula: Powiedz mi prosze, co wam powiedzial Samuel.
\par 16 I odpowiedzial Saul stryjowi swemu: Oznajmil nam za pewne, iz znaleziono oslice; ale o sprawie królestwa, o którem mu Samuel powiedzial, nie oznajmil mu.
\par 17 Potem zwolal Samuel ludu do Pana do Masfa,
\par 18 I rzekl do synów Izraelskich: Tak mówi Pan, Bóg Izraelski: Jam wywiódl Izraela z Egiptu, i wybawilem was z rak Egipczanów, i z rak wszystkich królestw, które was trapily;
\par 19 Alescie wy dzis odrzucili Boga waszego, który was sam wybawia od wszystkiego zlego waszego, i od ucisków waszych, i rzekliscie mu: Postanów króla nad nami. Przetoz teraz stancie przed Panem wedlug pokolen waszych, i wedle tysiaców waszych.
\par 20 A gdy kazal przystapic Samuel wszystkim pokoleniom Izraelskim, padl los na pokolenie Benjaminowe.
\par 21 Potem kazal przystapic pokoleniu Benjaminowemu wedlug domów jego, i padl los na dom Metry, a trafil na Saula, syna Cysowego; i szukano go, ale go nie znaleziono.
\par 22 Przetoz pytali sie znowu Pana: Przyjdzieli jeszcze sam ten maz? I odpowiedzial Pan: Oto sie skryl miedzy sprzetem.
\par 23 Tedy poszedlszy wzieli go stamtad. I stanal w posród ludu, i byl glowa wyzszy nad wszystek lud.
\par 24 I rzekl Samuel do wszystkiego ludu: Widziciez, kogo to Pan obral, ze mu niemasz równego miedzy wszystkim ludem? przetoz zakrzyknal wszystek lud, mówiac: Niech zyje król!
\par 25 Tedy powiedzial Samuel ludowi prawo królewskie, i spisal je na ksiegach, które polozyl przed Panem. Potem rozpuscil Samuel wszystek lud, kazdego do domu swego.
\par 26 Takze i Saul szedl do domu swego do Gabaa, i szly za nim wojska, których Bóg serca dotknal.
\par 27 Lecz ludzie niepobozni rzekli: Cóz, tenze nas wybawi? I wzgardzili nim, ani mu przyniesli darów; ale on czynil, jakoby nie slyszal.

\chapter{11}

\par 1 Tedy przyciagnal Nahas, Ammonczyk, i polozyl sie obozem przeciw Jabes Galaadskiemu. I rzekli wszyscy mezowie Jabes do Nahasa: Uczyn z nami przymierze a bedziemyc sluzyli.
\par 2 I rzkl do nich Nahas, Ammonczyk: W ten sposób uczynie z wami przymierze, jezli wylupie z was kazdemu oko prawe, a wloze to obelzenie na wszystkiego Izraela.
\par 3 I rzekli do niego starsi z Jabes: Pozwól nam siedm dni, ze rozeslemy posly po wszystkich granicach Izraelskich; a jezli nie bedzie, ktoby nas ratowal, tedy wynijdziemy do ciebie.
\par 4 I przyszli poslowie do Gabaa Saulowego, a powiedzieli te slowa, gdzie slyszal lud; i podniósl wszystek lud glos swój, a plakal.
\par 5 A oto, Saul szedl za wolami z pola, i rzekl Saul: Cóz sie stalo ludowi, iz placze? I powiedzieli mu wszystkie slowa mezów z Jabes.
\par 6 Tedy zstapil Duch Bozy na Saula, gdy uslyszal slowa te, i zapalil sie gniew jego bardzo.
\par 7 A wziawszy pare wolów, rozrabal je na sztuki, i rozeslal po wszystkich granicach Izraelskich przez tez posly, mówiac: Ktokolwiek nie wynijdzie za Saulem i za Samuelem, tak sie stanie wolom jego. I padl strach Panski na lud, i wyszli jako maz jeden.
\par 8 I obliczyl je w Bezeku; a bylo synów Izraelskich trzy kroc sto tysiecy, a mezów Juda trzydziesci tysiecy.
\par 9 I rzekli poslom, którzy byli przyszli: Tak powiedzcie mezom w Jabes Galaad: Jutro bedziecie wybawieni, gdy ogrzeje slonce. I wrócili sie poslowie, i oznajmili to mezom w Jabes, którzy sie uweselili.
\par 10 Tedy rzekli mezowie Jabese Ammonitom: Jutro wynijdziemy do was, a uczynicie z nami wszystko, co dobrego bedzie w oczach waszych.
\par 11 Nazajutrz tedy rozszykowal Saul lud na trzy hufy; i wtargnal w posrodek obozu przed switaniem; i bil Ammonity, az sie dzien ogrzal; a którzy pozostali, rozpierzchneli sie, tak, iz nie zostalo z nich i dwóch pospolu.
\par 12 I rzekl lud do Samuela: Któz jest ten, co mówil: Saulze bedzie królowal nad nami? Wydajcie meze te, abysmy je pobili.
\par 13 I rzekl Saul: Nie bedzie nikt zabity dnia tego; bo dzis Pan uczynil wybawienie w Izraelu.
\par 14 Zatem rzekl Samuel do ludu: Pójdzcie, a idzmy do Galgal, a tam odnowimy królestwo.
\par 15 Szedl tedy wszystek lud do Galgal, i postanowili tam Saula królem przed Panem w Galgal, tamze sprawowali ofiary spokojne przed Panem. I weselil sie tam Saul, i wszyscy mezowie Izraelscy bardzo.

\chapter{12}

\par 1 I rzekl Samuel do wszystkiego Izraela: Otom usluchal glosu waszego we wszystkiem, o coscie ze mna mówili, i postanowilem nad wami króla.
\par 2 A oto, teraz król chodzi przed wami, a jam sie zstarzal i osiwial; oto, i synowie moi sa z wami, a jam tez chodzil przed wami od mlodosci mojej az do dnia tego.
\par 3 Otom ja tu. Swiadczciez przeciwko mnie przed Panem, i przed pomazancem jego, jezlim wzial któremu z was wolu, albo jezlim wzial któremu z was osla, i jezlim kogo ucisnal, albo gwalt komu uczynil, i jezlim z reki czyjej wzial dar, zebym mial kryc o czy swoje dla niego; a nagrodze wam.
\par 4 I odpowiedzieli: Nie ucisnales nas, anis nam gwaltu uczynil, anis wzial z reki czyjej zadnej rzeczy.
\par 5 Nadto rzekl do nich: Swiadkiem Pan przeciwko wam, i swiadkiem pomazaniec jego dnia tego, izescie nic nie znalezli w rece mojej. A oni rzekli: Swiadkiem.
\par 6 I rzekl Samuel do lud: Pan swiadkiem, który uczynil Mojzesza i Aarona, i który wywiódl ojce wasze z ziemi Egipskiej.
\par 7 Przetoz teraz stancie, abym sie rozpieral z wami przed Panem, o wszystkie dobrodziejstwa Panskie, które wam czynil i ojcom waszym.
\par 8 Gdy zaszedl Jakób do Egiptu, wolali ojcowie wasi do Pana, i poslal Pan Mojzesza i Aarona, którzy wywiedli ojce wasze z Egiptu, a posadzili je na tem miejscu;
\par 9 A gdy zapomnieli Pana Boga swego, podal je w reke Sysarze, hetmanowi wojska Hasor, i w reke Filistynów, takze w reke króla Moabskiego, którzy walczyli przeciwko nim.
\par 10 Ale gdy wolali do Pana, i mówili: Zgrzeszylismy, zesmy opuscili Pana, a sluzylismy Baalom i Astarotowi, przetoz teraz wybaw nas z rak nieprzyjaciól naszych, a bedziemyc sluzyli:
\par 11 Tedy poslal Pan Jerubaala, i Bedona, i Jeftego, i Samuela, a wyrwal was z reki nieprzyjaciól waszych okolicznych, i mieszkaliscie bezpiecznie.
\par 12 Potem widzac, iz Nahas, król synów Ammonowych, przyciagnal przeciwko wam, rzekliscie do mnie: Zadnym sposobem; ale król bedzie królowal nad nami: choc Pan Bóg wasz byl królem waszym,
\par 13 Teraz tedy oto król, któregoscie obrali, któregoscie zadali; oto, przelozyl Pan króla nad wami.
\par 14 Jezli sie bedziecie bali Pana, a jemu sluzyli, i sluchali glosu jego a nie rozdraznicie ust Panskich, tedy i wy, i król, który króluje nad wami, bedziecie szczesliwie chodzic za Panem, Bogiem waszym.
\par 15 Ale jezliz nie bedziecie sluchac glosu Panskiego, a rozdraznicie usta Panskie, bedzie reka Panska przeciwko wam, jako i przeciwko ojcom waszym.
\par 16 Jeszcze teraz stójcie, a obaczcie te rzecz wielka, która Pan uczyni przed oczyma waszemi.
\par 17 Izali dzis nie pszeniczne zniwa? Bede wzywal Pana, a pusci gromy i dzdze, a dowiecie sie, i obaczycie, jaka jest wielka zlosc wasza, którejscie sie dopuscili przed oczyma Panskiemi, zadajac sobie króla.
\par 18 Przetoz wolal Samuel do Pana, i puscil Pan gromy i deszcz dnia onego, i bal sie wszystek lud bardzo Pana i Samuela.
\par 19 I rzekl wszystek lud do Samuela: Módl sie za slugami twymi Panu Bogu twemu, zebysmy nie pomarli: bosmy przydali do wszystkich grzechów naszych te zlosc, zesmy sobie prosili o króla.
\par 20 Tedy rzekl Samuel do ludu: Nie bójcie sie, aczescie wy to wszystko zle uczynili; wszakze przeto nie odstepujcie od Pana, ale sluzcie Panu ze wszystkiego serca waszego;
\par 21 A nie udawajcie sie za próznosciami, które wam nic nie pomoga, ani was wybawia, gdyz próznosciami sa.
\par 22 Albowiemci nie opusci Pan ludu swego, dla imienia swego wielkiego, gdyz sie upodobalo Panu, uczynic was sobie ludem.
\par 23 A mnie nie daj Boze, abym mial grzeszyc przeciw Panu, przestawajac modlic sie za wami; owszem was bede nauczal drogi dobrej i prostej.
\par 24 Jedno sie bójcie Pana, a sluzcie mu w prawdzie ze wszystkiego serca waszego, a to upatrujcie, jako wielmoznie poczynal z wami.
\par 25 Ale jezli przecie w zlosci trwac bedziecie, tedy i wy, i król wasz poginiecie.

\chapter{13}

\par 1 Saul tedy pierwszego roku królowania swego (bo tylko dwa lata królowal nad Izraelem,)
\par 2 Wybral sobie trzy tysiace z Izraela; i byli przy Saulu dwa tysiace w Machmas, i na górze Betel, a tysiac byl z Jonatanem w Gabaa Benjamin, a ostatek ludu rozpuscil, kazdego do przybytku swego.
\par 3 Tedy Jonatan pobil straz Filistynska, która byla w Gabaa, i uslyszeli Filistynowie. Zatem Saul zatrabil w trabe po wszystkiej ziemi, mówiac: Niech uslysza Hebrejczycy.
\par 4 A tak uslyszal wszystek Izrael, ze powiadano: Pobil Saul straz Filistynska, dla czego tez obrzydlym byl Izrael miedzy Filistyny. I zwolano lud za Saulem do Galgal.
\par 5 Filistynowie tez zebrali sie, aby walczyli z Izraelem, majac trzydziesci tysiecy wozów, i szesc tysiecy jezdnych, a ludu bardzo wiele jako piasku, który jest na brzegu morskim, i ciagneli a polozyli sie obozem w Machmas, na wschód slonca od Betawen.
\par 6 Ale mezowie Izraelscy widzac, iz byli scisnieni, (bo byl ucisniony lud,)pokryli sie w jaskini, i w obronne miejsca, i w skaly, i w wieze, i w jamy.
\par 7 Niektórzy tez Hebrejczykowie przeprawili sie za Jordan, do ziemi Gad i Galaad; ale Saul jeszcze pozostal byl w Galgal, a wszystek lud potrwozony szedl za nim.
\par 8 I czekal przez siedm dni wedlug czasu zamierzonego od Samuela, a gdy nie przyszedl Samuel do Galgal, rozbiezal sie lud od niego.
\par 9 Tedy rzekl Saul: Przyniescie do mnie ofiare calopalenia, i ofiary spokojne; tamze ofiarowal calopalenie.
\par 10 A gdy dokonczyl ofiary calopalenia, oto Samuel przyszedl, i wyszedl Saul przeciwko niemu, zeby go przywital.
\par 11 I rzekl Samuel: Cózes uczynil? Odpowiedzial Saul: Izem widzial, ze sie rozchodzi lud odemnie, a tys nie przyszedl na czas naznaczony, Filistynowie sie tez zebrali do Machmas,
\par 12 Tedym rzekl: Oto przypadna Filistynowie na mie w Galgal, a jam jeszcze nie ublagal twarzy Panskiej, i tak powazylem sie, i ofiarowalem calopalenie.
\par 13 I rzekl Samuel do Saula; Glupies uczynil, nie zachowales przykazania Pana Boga twego, którec rozkazal; albowiem terazby byl utwierdzil Pan królestwo twoje nad Izraelem az na wieki.
\par 14 Ale teraz królestwo twoje nie ostoi sie; Pan sobie znalazl meza wedlug serca swego, któremu rozkazal Pan, aby byl wodzem nad ludem jego, gdyzes nie zachowal, coc przykazal Pan.
\par 15 Wstawszy tedy Samuel, poszedl z Galgal do Gabaa w Benjamin, i policzyl Saul lud, którego sie znalazlo przy nim okolo szesciu set mezów.
\par 16 Przetoz Saul, i Jonatan, syn jego, i lud, który sie znalazl przy nim, zostali w Gabaa w Benjamin, a Filistynowie lezeli obozem w Machmas.
\par 17 I wyszly dla zdobyczy z obozu Filistynskiego trzy hufce: hufiec jeden obrócil sie droga ku Ofra do ziemi Saul;
\par 18 A drugi hufiec obrócil sie droga ku Betoron; trzeci zas hufiec udal sie droga ku granicy przyleglej dolinie Soboim ku puszczy.
\par 19 Ale kowal nie znajdowal sie we wszystkiej ziemi Izraelskiej; bo byli zabiezeli temu Filistynowie, zeby snac Hebrejczycy nie robili mieczów ani oszczepów.
\par 20 Przetoz chadzal wszystek Izrael do Filistynów, ostrzyc sobie kazdy lemiesz swój, i motyke swoje, i siekiere swoje, i rydel swój.
\par 21 Bo stepialy byly ostrza lemieszów, i motyk, i widel, i siekier az do oscienia, które bylo ostrzyc potrzeba.
\par 22 I bylo pod czas wojny, ze sie nie znajdowal miecz, ani oszczep w reku wszystkiego ludu, który byl z Saulem, i z Jonatanem; tylko sie znajdowal u Saula i Jonatana, syna jego.
\par 23 A straz Filistynska wyszla na droge ku Machmas.

\chapter{14}

\par 1 I stalo sie dnia niektórego, ze rzekl Jonatan, syn Saula, do slugi, który nosil bron jego: Pójdz, przejdziemy do strazy Filistynskiej, która jest na onej stronie; a ojcu swemu o tem nie oznajmil.
\par 2 Ale Saul zostal byl przy pagórku pod jablonia granatowa, która byla w Migron, i lud, który byl z nim, okolo szesciu set mezów.
\par 3 A Achijas, syn Achitoba, brata Ichaboda, syna Fineesowego, syna Heli, kaplana Panskiego w Sylo, nosil Efod: a lud nie wiedzial, iz odszedl Jonatan.
\par 4 Ale miedzy przechodami, kedy szukal Jonatan przejscia ku strazy Filistynskiej, byla skala ostra po jednej stronie, takze skala ostra po drugiej stronie; jednej imie Boses a drugiej Sene.
\par 5 Skala jedna byla na pólnocy przeciwko Machmas, a druga na poludnie przeciwko Gabaa.
\par 6 I rzekl Jonatan do wyrostka, który nosil bron jego: Pójdzmy, a przejdziemy do strazy tych nieobrzezanców, snac uczyni Pan przez nas wybawienie; boc nie trudno Panu wybawic w wielu albo w trosze.
\par 7 I rzekl mu sluga, noszacy bron jego: Czyn, co sie podoba sercu twemu; idz, gdzie chcesz, oto ja bede z toba wedlug woli twojej.
\par 8 Tedy rzekl Jonatan: oto my idziemy do tych mezów, a ukazemy sie im.
\par 9 Jezli nam tak rzeka: Czekajcie, az przyjdziemy do was, stójmyz na miejscu swem, a nie chodzmy do nich;
\par 10 Ale jezliz tak rzeka: Pójdzcie do nas, pójdzmyz; boc je dal Pan w rece nasze, a to bedziemy mieli za znak.
\par 11 Ukazali sie tedy obaj strazy Filistynskiej. I rzekli Filistynowie: Onoz Hebrejczycy wychodza z jaskini, w której sie byli pokryli.
\par 12 I mówili mezowie, co na strazy byli, do Jonatana, i do wyrostka, co za nim bron nosil, i rzekli: Pójdzcie ku nam, a oznajmiemy wam cos. I rzekl Jonatan do slugi swego: Pójdz za mna; boc je Pan podal w rece Izraelczykom.
\par 13 Lazl tedy Jonatan na rekach swych, i na nogach swych, a wyrostek jego za nim; i padli przed Jonatanem, i przed wyrostkiem jego, który tez zabijal, idac za nim.
\par 14 A tac byla porazka pierwsza, w której pobil Jonatan i wyrostek jego, co bron za nim nosil, okolo dwudziestu mezów, jakoby na pól staja roli.
\par 15 I przyszedl strach na obóz na polu, i na wszystek lud; straz tez, i ci którzy byli wyjechali na zdobycz, lekali sie, az sie ziemia trzesla; bo byla w strachu Bozym.
\par 16 I obaczyla straz Saulowa w Gabaa Benjaminowym, ze sie ono mnóstwo rozsypalo, i pierzchlo, i ze sie go urywalo.
\par 17 Tedy rzekl Saul do ludu, który przy nim byl: Wywiedzcie sie zaraz, a obaczcie, kto odszedl z naszych; a gdy sie wywiadowali, oto nie bylo Jonatana, i wyrostka, co za nim bron nosil.
\par 18 I rzekl Saul do Achijasa: Przystaw skrzynie Boza; (bo byla skrzynia Boza dnia onego z syny Izraelskimi.)
\par 19 I stalo sie, gdy jeszcze Saul mówil do kaplana, ze zamieszanie, które bylo w obozie Filistynskim, wzmagalo sie i rozmnazalo; przetoz rzekl Saul do kaplana: Zawsciagnij reki twojej.
\par 20 A tak zebrawszy sie Saul, i wszystek lud, który byl z nim, przyszli, gdzie byla bitwa, a oto, kazdego miecz byl obrócony na towarzysza jego, i byla porazka bardzo wielka.
\par 21 A Hebrejczycy, którzy przedtem przestawali z Filistynami, którzy z nimi ciagneli w obozie tam i sam, ci sie tez obróciwszy staneli przy Izraelu, który byl z Saulem i z Jonatanem.
\par 22 Nadto wszyscy mezowie Izraelscy, którzy sie byli pokryli na górze Efraim, gdy uslyszeli, iz uciekaja Filistynowie, szli za nimi w pogon w onej bitwie.
\par 23 I wybawil Pan dnia onego Izraela, a bitwa ona zaszla az do Betawen.
\par 24 A mezowie Izraelscy strudzeni byli onego dnia. I poprzysiagl Saul lud, mówiac: Przeklety maz, któryby jadl chleb przed wieczorem, az sie pomszcze nad nieprzyjacioly mymi. I nie skosztowal wszystek lud chleba.
\par 25 Tedy wszystek lud onej ziemi przyszedl do lasu, gdzie bylo wiele miodu na ziemi.
\par 26 Wszedlszy tedy lud do lasu, ujrzal plynacy miód; wszakze nie doniósl zaden z miodem reki swojej do ust swoich, bo sie bal lud onej przysiegi.
\par 27 Ale Jonatan nie slyszal, gdy poprzysiegal lud ojciec jego; i sciagnal koniec laski, która mial w rece swej, a omoczyl go w plastrze miodu, i obrócil reke swoje do ust swoich, i oswiecily sie oczy jego.
\par 28 A odpowiadajac jeden z ludu, rzekl: Przysiega zawiazal ojciec twój lud, mówiac: Przeklety maz, któryby jadl chleb dzisiaj; stadze ustal lud.
\par 29 Tedy rzekl Jonatan: Strwozyl ojciec mój lud ziemi. Patrzcie prosze jako sa oswiecone oczy moje, izem skosztowal troche miodu tego;
\par 30 Jako daleko wiecej, gdyby sie byl najadl dzis lud z lupu nieprzyjaciól swoich, których nabyl; izaliby nie byla wieksza porazka miedzy Filistynami?
\par 31 Porazili tedy dnia onego Filistyny od Machmas az do Ajalon, i spracowal sie lud bardzo.
\par 32 Tedy sie lud udal na lup, a nabrawszy owiec, i wolów, i cielat, rzezali je na ziemi, a jadl lud ze krwia.
\par 33 I powiedziano Saulowi mówiac: Oto lud grzeszy przeciw Panu, jedzac ze krwia; który rzekl: Zgrzeszyliscie; przytoczciez sam do mnie teraz kamien wielki.
\par 34 Zatem rzekl Saul: Rozejdzcie sie miedzy lud, a rzeczcie do nich: Przywiedzcie do mnie kazdy wolu swego, i kazdy owce swa, bijciez je tu, a jedzcie, a nie zgrzeszycie przeciw Panu, jedzac ze krwia. I przywiedli wszystek lud, kazdy wolu swego w rece swej w nocy, i bito je tam.
\par 35 I zbudowal Saul oltarz Panu; toc najpierwszy oltarz, który zbudowal Panu.
\par 36 I rzekl Saul: Puscmy sie za Filistynami noca, a bijmy je az do switania, a nie zostawujmy z nich i jednego. Którzy mu odpowiedzieli: Cokolwiek dobrego jest w oczach twoich uczyn, ale kaplan rzekl: Przystapmy sam do Boga.
\par 37 Tedy sie radzil Saul Boga: Mamli sie puscic za Filistynami? podaszli je w rece Izraela? I nie odpowiedzial mu dnia tego.
\par 38 Przetoz rzekl Saul: Przystapcie sam wszyscy celniejsi z ludu, i wywiedzcie sie, a patrzcie, przy kimby grzech dzis byl.
\par 39 Bo jako zywy Pan, który wybawia Izraela, chocby byl i przy Jonatanie, synu moim, ze smiercia umrze. I nie odpowiedzial mu nikt ze wszystkiego ludu.
\par 40 Nadto rzekl do wszystkiego Izraela: Wy bedziecie na jednej stronie, a ja i Jonatan, syn mój, bedziemy na drugiej stronie. I odpowiedzial lud Saulowi: Co dobrego jest w oczach twoich, uczyn.
\par 41 Zatem rzekl Saul do Pana, Boga Izraelskiego: Panie, pokaz sprawiedliwa; i nalezion jest Jonatan i Saul, a lud wyszedl z tego.
\par 42 Potem rzekl Saul: Rzuccie los miedzy mna i miedzy Jonatanem, synem moim; i znalezony jest Jonatan.
\par 43 Zatem rzekl Saul do Jonatana: Powiedz mi, cos uczynil? I powiedzal mu Jonatan, i rzekl: Skosztowalem tylko koncem laski, któram mial w rece mojej, troche miodu, i dla tegoz ja mam umrzec?
\par 44 I odpowiedzal Saul: To a to mi niech Bóg uczyni, ze smiercia umrzesz Jonatanie.
\par 45 Ale lud rzekl do Saula: Izali Jonatan umrze, który uczynil to wybawienie wielkie w Izraelu? Boze uchowaj! jako zywy Pan, nie spadnie i wlos z glowy jego na ziemie; albowiem za pomoca Boza uczynil to dzisiaj. A tak wybawil lud Jonatana, ze nie umarl.
\par 46 Tedy sie wrócil Saul od Filistynów, a Filistynowie odeszli na miejsce swoje.
\par 47 A Saul otrzymawszy królestwo nad Izraelem, walczyl przeciwko okolicznym wszystkim nieprzyjaciolom swoim, przeciw Moabitom, i przeciw synom Ammonowym, i przeciw Edomczykom, i przeciw królom Soba, i przeciw Filistynom; a gdzie sie kolwiek obrócil, meznie sie sprawowal.
\par 48 Zebrawszy tez wojsko, porazil Amalekity, i wyrwal Izraela z reki tego, który go pustoszyl.
\par 49 A mial Saul syny Jonatana, i Jesujego, i Melchisua, a imiona dwóch córek jego: imie pierworodnej Merob, a mlodszej Michol;
\par 50 A imie zony Saulowej Achinoam, która byla córka Achimaasowa; a imie Hetmana wojska jego Abner, syn Nera, stryja Saulowego.
\par 51 Bo Cys byl ojciec Saula, a Ner ojciec Abnera, syn Abijela.
\par 52 I byla wojna wielka z Filistynami po wszystkie dni Saulowe. Przetoz, gdziekolwiek widzial Saul jakiego silnego i dzielnego meza, przyjmowal go do siebie.

\chapter{15}

\par 1 I rzekl Samuel do Saula: Poslal mie Pan, abym cie pomazal za króla nad ludem jego Izraelskim, przetoz teraz posluchaj glosu slów Panskich.
\par 2 Tak mówi Pan zastepów: Wspomnialem na to, co uczynil Amalek Izraelowi, jako sie nan zasadzil na drodze gdy wychodzil z Egiptu.
\par 3 Przetoz idz, a pobij Amaleka, i wytrac jako przeklete wszystko, co ma; nie folguj mu, ale wybij od meza az do niewiasty, od malego az do ssacego, od wolu az do owcy, od wielblada az do osla.
\par 4 A tak Saul zebrawszy lud, policzyl go w Telaim, dwa kroc sto tysiecy pieszych, a dziesiec tysiecy mezów z Juda.
\par 5 A gdy przyciagnal Saul az do miasta Amalek, aby zwiódl bitwe nad potokiem,
\par 6 Rzekl Saul do Cynejczyka: Idzcie, odstapcie, a wynijdzcie z posrodku Amalekitów, abym was nie wytracil z nimi; bos ty uczynil milosierdzie ze wszystkimi syny Izraelskimi, gdy szli z Egiptu. A tak odstapil Cynejczyk z posrodku Amalekitów.
\par 7 I porazil Saul Amaleka, od Hewila, któredy chodza do Sur, które jest przeciw Egiptowi.
\par 8 I pojmal Agaga, króla Amalekitów, zywego, a wszystek lud pobil ostrzem miecza.
\par 9 A przepuscil Saul i lud jego Agagowi, co najlepszym owcom, i wolom, i bydlu tlustemu, i baranom, i wszystkiemu, co bylo najlepszego, a nie chcieli go wygubic; tylko cokolwiek bylo nikczemnego i podlego, to wygubili.
\par 10 Przetoz stalo sie slowo Panskie do Samuela, mówiac:
\par 11 Zal mi, zem postanowil Saula za króla; albowiem odwrócil sie odemnie, a slowa mego nie wypelnil, i rozgniewal sie bardzo Samuel, i wolal do Pana przez cala noc.
\par 12 Wstawszy tedy Samuel, szedl przeciwko Saulowi rano; bo dano znac Samuelowi, mówiac: Przyszedl Saul do Karmelu, tamze wystawil sobie pamiatke zwyciestwa; a obróciwszy sie poszedl, i przyszedl do Galgal.
\par 13 A gdy przyszedl Samuel do Saula, rzekl mu Saul: Blogoslawionys ty od Pana, wypelnilem slowo Panskie.
\par 14 Ale Samuel rzekl: A to co za wrzask trzód w uszach moich, i co za ryk wolów, który ja slysze?
\par 15 I odpowiedzial Saul; Od Amalekitów przygnano je; albowiem lud przepuscil co najlepszym owcom, i wolom aby je ofiarowal Panu, Bogu twemu, a ostatekiesmy wytracili jako przeklete.
\par 16 Tedy rzekl Samuel do Saula: Dopusc, a powiem ci, co mówil Pan do mnie w nocy; a on mu rzekl: Powiedz.
\par 17 I rzekl mu Samuel: Izali, gdys byl maly w oczach twoich, nie stales sie glowa pokolen Izraelskich, i nie pomazal cie Pan za króla nad Izraelem?
\par 18 I poslal cie Pan w te droge, i rzekl: Idz, a wytrac Amalekity jako przklete grzeszniki, i walcz przeciwko nim, azbys je do szczetu wytracil.
\par 19 Przeczzes tedy nie usluchal glosu Panskiego, ales sie udal za korzyscia, i uczyniles zle przed oczyma Panskiemi?
\par 20 Tedy odpowiedzial Saul Samuelowi: I owszem usluchalem glosu Panskiego, a szedlem droga, która mie poslal Pan, i przywiodlem Agaga, króla Amalekitskiego, a Amalekity wytracilem jako przeklete.
\par 21 Ale lud pobral z korzysci owce i woly co przedniejsze z przeklestwa, aby je ofiarowal Panu, Bogu twemu, w Galgal.
\par 22 I rzekl Samuel: Izali sie tak kocha Pan w calopaleniach i w ofiarach, jako gdy kto slucha glosu Panskiego? oto, posluszenstwo lepsze jest nizeli ofiara, a sluchac lepiej jest, niz ofiarowac tlustosc baranów.
\par 23 Bo przeciwic sie jest jako grzech czarowania, a przestapic przykazanie jest jako balwochwalstwo i obrazy; przetoz izes odrzucil slowo Panskie, tedy cie tez odrzucil Pan, abys nie byl królem.
\par 24 Tedy rzekl Saul do Samuela: Zgrzeszylem, zem przestapil rozkazanie Panskie i slowa twoje, gdyzem sie bal ludu, i usluchalem glosu ich.
\par 25 A teraz znies prosze grzech mój, a wróc sie ze mna, abym sie poklonil Panu.
\par 26 I rzekl Samuel do Saula: Nie wróce sie z toba; gdyzes odrzucil slowo Panskie, ciebie tez odrzucil Pan, abys nie byl królem nad Izraelem.
\par 27 A gdy sie odwrócil Samuel, zeby odszedl, uchwycil skrzydlo plaszcza jego, i oderwalo sie.
\par 28 Tedy mu rzekl Samuel: Oderwal Pan królestwo Izraelskie dzisiaj od ciebie, i dal je blizniemu twemu, lepszemu nizes ty.
\par 29 A zaistec Mocarz Izraelski nie sklamie, ani bedzie zalowal; bo nie jest czlowiekiem, aby mial zalowac.
\par 30 A on rzekl: Zgrzeszylem; wszakze uczcij mie prosze przed starszymi ludu mego, i przed Izraelem, a wróc sie zemna, abym sie poklonil Panu, Bogu twemu.
\par 31 Wróciwszy sie tedy Samuel szedl za Saulem, i poklonil sie Saul Panu.
\par 32 I rzekl Samuel: Przywiedzcie do mnie Agaga, króla Amalekitskiego; i szedl do niego Agag powaznie, i rzekl Agag: Zaiste uszedlem gorzkosci smierci.
\par 33 Ale rzekl Samuel: Jako osierocil niewiasty miecz twój, tak osierocona bedzie nad inne niewiasty matka twoja. I rozsiekal w kesy Samuel Agaga przed obliczem Panskiem w Galgal.
\par 34 Potem odszedl Samuel do Ramaty, a Saul szedl do domu swego, do Gabaa Saulowego.
\par 35 A juz potem wiecej Samuel nie widzial Saula, az do dnia smierci swojej: wszakze zalowal Samuel Saula, a Pan tez zalowal, ze uczynil królem Saula nad Izraelem.

\chapter{16}

\par 1 Tedy rzekl Pan do Samuela: I pókiz ty bedziesz zalowal Saula, gdyzem go Ja odrzucil, aby nie królowal nad Izraelem? Napelnij róg twój oliwa, a pójdz, posle cie do Isajego Betlehemczyka; bom tam sobie upatrzyl miedzy syny jego króla.
\par 2 I rzekl Samuel: Jakoz mam isc? Bo uslyszy Saul, i zabije mie. I odpowiedzial Pan: Wezmij z soba jalowice z stada, i rzeczesz: Przyszedlem, abym ofiarowal Panu.
\par 3 I wezwiesz Isajego na ofiare, a Ja tobie oznajmie, co masz czynic; i pomazesz mi tego, o którym ci powiem.
\par 4 I uczynil Samuel, jako mu powiedzial Pan, a przyszedl do Betlehem; a uleklszy sie starsi miasta, zabiezeli mu, i mówili: Spokojneli jest przyjscie twoje?
\par 5 I rzekl: Spokojne; przyszedlem, abym ofiarowal Panu. Poswiecciez sie, a pójdzcie ze mna na ofiare. I poswiecil Isajego, i syny jego, a wezwal ich na ofiare.
\par 6 A gdy przyszli, ujrzal Elijaba, i rzekl: Zaiste ten jest przed Panem pomazaniec jego.
\par 7 Lecz rzekl Pan do Samuela: Nie patrz na urode jego, ani na wysokosc wzrostu jego, gdyzem go odrzucil. Albowiem Ja nie patrze na to, na co patrzy czlowiek; bo czlowiek patrzy na to, co jest przed oczyma, ale Pan patrzy na serce.
\par 8 Zawolal tedy Isaj Abinadaba, i kazal mu isc przed Samuela; który rzekl: I tego nie obral Pan.
\par 9 Potem kazal przyjsc Isaj Sammie, a on rzekl: I tego nie obral Pan.
\par 10 Tedy kazal przyjsc Isaj siedmiu synom swoim przed Samuela. I rzekl Samuel do Isajego: Nie obral Pan i tych.
\par 11 Potem rzekl Samuel do Isajego: Wszyscyz to juz synowie? Odpowiedzial: Jeszcze zostal najmlodszy, który pasie owce. Tedy rzekl Samuel do Isajego: Poslijze, a przywiedz go; boc nie usiadziemy, az on tu przyjdzie.
\par 12 A tak poslal i przywiódl go; a on byl lisowaty, i wdziecznych oczu, a piekny na wejrzeniu. Tedy rzekl Pan: Wstan, a pomaz go, boc ten jest.
\par 13 Wziawszy tedy Samuel róg z oliwa, pomazal go w posród braci jego. I zostal Duch Panski nad Dawidem od onegoz dnia, i na potem. Zatem Samuel wstal, i poszedl do Ramaty.
\par 14 A Duch Panski odstapil od Saula, i trwozyl go duch zly od Pana.
\par 15 I rzekli sludzy Saulowi do niego: Oto teraz Duch Bozy zly trwozy cie:
\par 16 Niech rozkaze Pan nasz, a sludzy twoi, którzy sa przed toba, poszukaja meza, coby umial grac na harfie, ze gdy cie napadnie Duch Bozy zly, zagra reka swa, a ulzyc sie.
\par 17 Rzekl tedy Saul do slug swoich: Upatrzcie mi prosze meza, coby dobrze grac umial, a przywiedzcie do mnie.
\par 18 I odpowiedzial jeden z slug, i rzekl: Otom widzial syna Isajego Betlehemczyka, który umie grac, a jest czlowiek mezny, i rycerski, i sprawny, i gladki, a Pan jest z nim.
\par 19 Poslal tedy Saul posly do Isajego, mówiac: Poslij do mnie Dawida, syna twego, który jest przy stadzie.
\par 20 Tedy wziawszy Isaj osla, chleb, i flaszke wina, wziawszy i koziolka jednego z stada, poslal przez Dawida, syna swego, Saulowi.
\par 21 A gdy przyszedl Dawid do Saula, stanal przed nim; a rozmilowal sie go bardzo, i byl u niego za wyrostka bron noszacego.
\par 22 I poslal Saul do Isajego, mówiac: Niech stoi prosze Dawid przedemna; bo znalazl laske w oczach moich.
\par 23 I bywalo, gdy przychodzil Duch Bozy na Saula, ze wziawszy Dawid harfe, gral reka swa; tedy Saul mial ulzenie, i lepiej sie mial, bo odchodzil od niego on duch zly.

\chapter{17}

\par 1 Tedy zebrali Filistynowie wojska swe, aby walczyli, a zeszli sie u Sochot, które jest w Judzie, i polozyli sie obozem miedzy Sochotem i miedzy Asekiem na granicach Domim.
\par 2 A Saul i mezowie Izraelscy zebrali sie, i polozyli sie obozem w dolinie Ela, i uszykowali wojsko przeciw Filistynom.
\par 3 A Filistynowie stali na górze z jednej strony, ale Izraelczycy stali na górze z drugiej strony; a dolina byla miedzy nimi.
\par 4 I wyszedl maz miedzy nie z obozu Filistynskiego imieniem Golijat z Get, wzwyz na szesciu lokci i na piedzi.
\par 5 A przylbica miedziana byla na glowie jego, a w karacene luszczasta ubieral sie, a waga karaceny piec tysiecy syklów miedzi wazyla.
\par 6 Nadto nakolanki miedziane mial na nogach swoich, i tarcz miedziana miedzy ramionami swemi.
\par 7 A oszczepisko oszczepu jego jako nawój tkacki, a grot oszczepu jego mial szesc set syklów zelaza, a niosacy tarcz jego szedl przed nim.
\par 8 I stanawszy wolal do hufów Izraelskich, i mówil im: Nacoscie wyciagneli z wojskiem ku potykaniu? izazem ja nie jest Filistynczyk, a wy sludzy Saulowi? Obiezciez miedzy soba meza, a niech mi sie stawi.
\par 9 Bedzieli sie mógl potkac ze mna, a zabije mie, bedziemy waszymi niewolnikami; lecz jezli go ja przemoge, i zabije go, wy bedziecie naszymi niewolnikami, i sluzyc nam bedziecie. Nadto rzekl Filistynczyk:
\par 10 Jam dzis uragal hufom Izraelskim; dajcie mi meza, a niech czyni ze mna pojedynkiem.
\par 11 A uslyszawszy Saul i wszystek Izrael te slowa Filistynczyka, ulekli sie, i strwozyli sie bardzo.
\par 12 (A Dawid byl synem meza Efratejczyka z Betlehem Juda, którego imie Isaj, który mial osm synów, a ten byl za dni Saulowych stary i podeszly w leciech.
\par 13 I poszli trzej synowie Isajego starsi za Saulem na wojne; a imiona trzech synów jego, którzy poszli na wojne, sa te: Elijab pierworodny, a wtóry po nim Abinadab, a trzeci Samma;
\par 14 Lecz Dawid byl najmlodszy; a tak trzej najstarsi poszli byli za Saulem.)
\par 15 Dawid tedy odchodzil i wracal sie od Saula, aby pasl trzody ojca swego w Betlehem.
\par 16 Ale Filistynczyk wychadzal wstawajac rano i wieczór, i stawal przez czterdziesci dni.
\par 17 I rzekl Isaj do Dawida, syna swego: Wezmij teraz braciom swym to efa prazma, i dziesiecioro chleba tego, a biez do obozu do braci swych.
\par 18 A tych dziesiec mlodych serów doniesiesz rotmistrzowi; a bracia swa nawiedziwszy dowiesz sie, jako sie maja, i zastawe ich wydzwigniesz.
\par 19 A Saul, i oni, i wszyscy synowie Izraelscy lezeli w dolinie Ela, walczac przeciwko Filistynom.
\par 20 Wstawszy tedy Dawid rano na switaniu, a poruczywszy trzode strózowi, wzial to na sie, i szedl, jako mu byl rozkazal Isaj, i przyszedl do obozu; a wojsko wyszlo bylo do szyku, i okrzyk uczynilo ku potykaniu.
\par 21 A juz byli uszykowali Izraelczycy i Filistynowie wojsko przeciwko wojsku.
\par 22 Przetoz zostawiwszy Dawid to, co przyniósl, i zlozywszy to z siebie pod reke stróza sprzetu zolnierskiego, biezal do wojska, a przyszedlszy przywital sie z bracia swoja w pokoju.
\par 23 A gdy rozmawial z nimi, oto maz imieniem Golijat, Filistynczyk z Get, wystepowal miedzy nie z wojska Filistynskiego, i mówil onez slowa, co slyszal i Dawid.
\par 24 A wszyscy synowie Izraelscy ujrzawszy onego meza, uciekali od oblicza jego, i bali sie bardzo.
\par 25 I mówili mezowie Izraelscy: A widzielizescie tego meza, który wyszedl? Bo wyszedl, aby uragal Izraelowi, ale ktoby go zabil, ubogaci go król bogactwy wielkiemi, i córke mu swoje da, a dom ojca jego uczyni wolnym w Izraelu.
\par 26 Tedy Dawid rzekl do mezów, którzy z nim stali, mówiac: Co dadza mezowi, któryby zabil tego Filistynczyka, a odjal pohanbienie od Izraela? Bo cóz to za Filistynczyk nieobrzezany, ze uraga wojskom Boga zyjacego?
\par 27 I powiedzial mu lud onez slowa, mówiac: To dadza mezowi, który go zabije.
\par 28 A gdy uslyszal Elijab, brat jego starszy, co mówil z onymi mezami, zapalil sie gniewem Elijab na Dawida, i rzekl: Po cos tu przyszedl, a komus poruczyl one troche owiec na puszczy? znam ci ja pyche twoje, i zlosc serca twego, zes przyszedl, abys sie przypatrywal bitwie.
\par 29 Tedy rzekl Dawid: Cózem teraz uczynil? Wszakiem tu na rozkazanie przyszedl.
\par 30 I odwrócil sie od niego ku drugiemu, i pytal sie jako i przedtem; a odpowiedzial mu lud tak jako i pierwej.
\par 31 I uslyszano slowa, które mówil Dawid, i opowiedziano je Saulowi, którego Saul wzial do siebie.
\par 32 I mówil Dawid do Saula: Niech niczyje serce nie upada dla tego; sluga twój pójdzie, a bedzie sie bil z tym Filistynem.
\par 33 Ale Saul rzekl do Dawida: Nie mozesz ty isc przeciwko temu Filistynowi, abys sie z nim potykal, bos jest dziecina, a on jest mezem walecznym od mlodosci swojej.
\par 34 I odpowiedzial Dawid Saulowi: Pasal sluga twój trzode ojca swego, a gdy przychodzil lew, i niedzwiedz, a porywal barana z stada,
\par 35 Tedym go gonil, i bilem go, i wydzieralem z paszczeki jego; a gdy sie rzucal na mie, ulapiwszy go za gardlo jego, tluklem go, i zabijalem go.
\par 36 I lwa i niedzwiedzia zabil sluga twój; tedyc tez bedzie i ten Filistynczyk nieobrzezany, jako jeden z tych, gdyz uragal wojskom Boga zyjacego.
\par 37 Nadto rzekl Dawid: Pan, który mie wyrwal z mocy lwa, i z mocy niedzwiedzia, tenze mie wyrwie z rak Filistyna tego. Tedy rzekl Saul do Dawida: Idz, a Pan niech bedzie z toba.
\par 38 I ubral Saul Dawida w szaty swe, i wlozyl przylbice miedziana na glowe jego, a oblukl go w pancerz.
\par 39 Przypasal tez Dawid miecz jego na szaty swoje, i kosztowal, jezliby mógl chodzic (bo przedtem tego nie doswiadczal). Tedy rzekl Dawid do Saula: Nie moge w tem chodzic, bom temu nie przywykl. I zlozyl to Dawid z siebie.
\par 40 Ale wzial kij swój w reke swoje, i obral sobie piec gladkich kamieni z potoku, i wlozyl je do naczynia pasterskiego, które mial, to jest do torby, a proce swoje niósl w rekach swoich, i przyblizyl sie ku Filistynowi.
\par 41 Szedl tez Filistynczyk, postepujac i przyblizajac sie ku Dawidowi, i wyrostek, który niósl tarcz, przed nim.
\par 42 A gdy spojrzal Filistynczyk i obaczyl Dawida, lekce go sobie powazyl, przeto, ze byl dziecina, a lisowatym i pieknym na wejrzeniu.
\par 43 Rzekl tedy Filistynczyk do Dawida: Izalim ja pies, iz ty idziesz na mie z kijem? I przeklinal Filistynczyk Dawida przez bogi swoje.
\par 44 Nadto rzekl Filistynczyk do Dawida: Pójdz do mnie, a dam cialo twoje ptastwu powietrznemu i bestyjom polnym.
\par 45 Tedy rzekl Dawid do Filistyna: Ty idziesz do mnie z mieczem i z oszczepem, i z tarcza, a ja ide do ciebie w imieniu Pana zastepów, Boga wojsk Izraelskich, któremus uragal.
\par 46 Dzis cie poda Pan w rece moje, a zabije cie, i odejme glowe twoje od ciebie, a dam trupy wojska Filistynskiego ptastwu powietrznemu, i bestyjom ziemskim; a pozna wszystka ziemia, ze jest Bóg w Izraelu;
\par 47 I dozna wszystko to zgromadzenie, ze nie mieczem, ani oszczepem wybawia Pan, gdyz Panska jest walka, a poda was w rece nasze.
\par 48 I stalo sie, gdy powstal Filistynczyk, i szedl, a przyblizal sie przeciwko Dawidowi, ze pospieszyl i Dawid, a biezal na spotkanie przeciwko Filistynowi:
\par 49 A sciagnawszy Dawid reke swa do torby, wyjal z niej kamien, i cisnal z procy, a ugodzil Filistynczyka w czolo jego, tak iz utknal kamien w czole jego, i padl twarza swa na ziemie.
\par 50 A tak przemógl Dawid Filistynczyka proca i kamieniem, a uderzywszy Filistynczyka, zabil go, choc miecza nie mial Dawid w reku.
\par 51 A przybiezawszy Dawid, stanal nad Filistynczykiem, i wzial miecz jego, i dobyl go z pochwy jego, i zabil go, i ucial nim glowe jego. A gdy ujrzeli Filistynowie, iz umarl mocarz ich, uciekli.
\par 52 Powstawszy tedy mezowie Izraelscy i Judzcy, okrzyk uczynili i gonili Filistyny, az kedy chodza do doliny, i az do bram Akkaronu, i padali ranni Filistynowie po drodze Saraim az do Get i az do Akkaronu.
\par 53 A wróciwszy sie synowie Izraelscy z pogoni Filistynów, rozchwycili obóz ich.
\par 54 Potem wziawszy Dawid glowe onego Filistynczyka, przyniósl ja do Jeruzalemu, a zbroje jego wlozyl do namiotu swego.
\par 55 A gdy widzial Saul Dawida, wychodzacego przeciw Filistynowi, mówil do Abnera, hetmana wojska swego: Czyim jest synem ten mlodzieniec? Abnerze! I odpowiedzial Abner: Jako zywa dusza twoja, królu, zec niewiem.
\par 56 Tedy rzekl król: Pytaj, czyim jest synem ten mlodzieniec.
\par 57 A gdy sie wracal Dawid, zabiwszy Filistynczyka, tedy go wzial Abner, i przywiódl go przed Saula, a Dawid mial glowe Filistynowe w rekach swych.
\par 58 I rzekl do niego Saul: Czyjes ty syn, mlodziencze? I odpowiedzial Dawid: Jestem syn slugi twego Isajego Betlehemczyka.

\chapter{18}

\par 1 I stalo sie, gdy przestal mówic do Saula, ze dusza Jonatanowa spoila sie z dusza Dwidowa, i umilowal go Jonatan, jako dusze swoje.
\par 2 I wzial go Saul dnia onego, ani mu dopuscil, zeby sie wracal do domu ojca swego.
\par 3 A tak uczynil Jonatan z Dawidem przymierze, bo go milowal jak dusze swoje.
\par 4 A zdjawszy z siebie Jonatan plaszcz, który mial na sobie, dal go Dawidowi, i szaty swe, az do miecza swego, i az do pasa swego rycerskiego.
\par 5 I wychadzal Dawid do wszystkiego, do czego go kolwiek posylal Saul, a roztropnie sie sprawowal; i przelozyl go Saul nad rycerstwem, i byl wdziecznym w oczach wszystkiego ludu, takze i w oczach slug Saulowych.
\par 6 I stalo sie, gdy sie wracali, a Dawid sie tez wracal od porazki Filistynów, ze wyszly niewiasty ze wszystkich miast Izraelskich, spiewajac i grajac przeciwko Saulowi królowi z bebnami, z weselem, i z geslami.
\par 7 A spiewajac na przemiany one niewiasty, graly i mówily: Porazil Saul swój tysiac, ale Dawid swoich dziesiec tysiecy.
\par 8 I rozgniewal sie Saul bardzo, bo sie nie podobaly w oczach jego te slowa; i rzekl: Przywlaszczyli Dawidowi dziesiec tysiecy, a mnie przywlaszczyli tysiac: a czegoz mu niedostaje, jedno królestwa?
\par 9 Przetoz Saul krzywo patrzyl na Dawida od onegoz dnia i na potem.
\par 10 I stalo sie drugiego dnia, ze przypadl Duch Bozy zly na Saula, i prorokowal w posrodku domu, a Dawid gral reka swoja dnia onego, jako i przedtem, a Saul mial wlócznia w rece swej.
\par 11 I cisnal Saul wlócznia, mówiac: Przebije Dawida az ku scianie; ale sie uchylil Dawid przed nim po dwa kroc.
\par 12 I bal sie Saul Dawida, przeto ze Pan byl z nim, a od Saula odstapil.
\par 13 I odprawil go Saul od siebie, a uczynil go hetmanem nad tysiacem, i wychadzal a wchadzal przed ludem.
\par 14 Owa Dawid we wszystkich drogach swych roztropnie sie sprawowal; bo Pan byl z nim.
\par 15 Co gdy widzial Saul, iz tak bardzo roztropnie sobie poczynal, bal sie go.
\par 16 Ale wszystek Izrael i Juda milowal Dawida; bo on wychadzal i wchadzal przed nimi.
\par 17 I rzekl Saul do Dawida: Oto, córke moje starsza Merob dam ci za zone, jedno badz mezem mocnym, i odprawuj wojny Panskie; bo tak Saul sobie mówil: Niech nie bedzie reka moja na nim, ale niech bedzie na nim reka Filistynów.
\par 18 Tedy rzekl Dawid do Saula: Któzem ja? albo co za stan mój, albo co za dom ojca mego w Izraelu, zebym byl zieciem królewskim?
\par 19 I stalo sie, gdy przyszedl czas, którego miala byc dana Merob, córka Saulowa, Dawidowi, ze ona dana jest Adryjelowi Meholatyckiemu za zone.
\par 20 Ale sie rozmilowala Michol, córka Saulowa, Dawida; co gdy powiedziano Saulowi, milo mu to bylo.
\par 21 I rzekl Saul: Dam mu ja, zeby mu byla sidlem, a zeby byla na nim reka Filistynów. Przetoz rzekl Saul do Dawida: Po drugie bedziesz zieciem moim dzisiaj.
\par 22 Tedy rozkazal Saul slugom swoim: Rzeczcie do Dawida potajemnie, mówiac: Oto, upodobal cie sobie król, i wszyscy sludzy jego miluja cie, a tak teraz badz zieciem królewskim.
\par 23 A gdy mówili sludzy Saulowi w uszy Dawidowe te slowa, odpowiedzial Dawid: Czy sie wam mala rzecz widzi, byc zieciem królewskim, gdyzem ja jest mezem ubogim i podlym?
\par 24 Tedy sludzy Saulowi oznajmili mu, mówiac: Tak powiedzial Dawid.
\par 25 I rzekl Saul: Tak powiedzcie Dawidowi: Nie dbac król o wiano, tylko chce miec sto nieobrzezek Filistynskich, aby sie stala pomsta nad nieprzyjaciolmi królewskimi; bo Saul myslil, jakoby Dawida podac w rece Filistynom.
\par 26 Tedy sludzy jego powiedzieli te slowa Dawidowi; i spodobalo sie to Dawidowi, aby zostal zieciem królewskim; a jeszcze sie nie byly wypelnily dni one.
\par 27 Wstal tedy Dawid, i poszedl, on i mezowie jego, i zabil z Filistynów dwiescie mezów, i przyniósl Dawid nieobrzezki ich, i oddano je spelna królowi, aby byl zieciem królewskim. A tak dal mu Saul Michol, córke swa za zone.
\par 28 A widzac Saul, i baczac, ze Pan byl z Dawidem, a iz Michol, córka jego, milowala go,
\par 29 Tem wiecej Saul obawial sie Dawida, i stal sie Saul nieprzyjacielem Dawidowi po wszystkie dni.
\par 30 I wpadaly ksiazeta Filistynskie do ziemi. A kiedykolwiek wpadaly, roztropniej sobie poczynal Dawid nad wszystkie slugi Saulowe; przetoz slawne bylo imie jego bardzo.

\chapter{19}

\par 1 Tedy mówil Saul do Jonatana, syna swego, i do wszystkich slug swoich, aby zabili Dawida; ale Jonatan syn Saula, kochal sie w Dawidzie bardzo.
\par 2 I oznajmil to Jonatan Dawidowi, mówiac: Saul, ojciec mój, mysli cie zabic; przetoz teraz strzez sie prosze az do zaranku, a zataisz sie, i skryjesz sie.
\par 3 A ja wynijde, i stane podle ojca mego na polu, gdzie ty bedziesz, i bede mówil o tobie z ojcem moim, a cokolwiek pobacze, toc oznajmie.
\par 4 A tak mówil Jonatan o Dawidzie dobrze do Saula, ojca swego, i rzekl do niego: Niech nie grzeszy król przeciwko sludze swemu Dawidowi: boc nic nie winien, owszem sprawy jego bylyc bardzo pozyteczne;
\par 5 Gdyz polozyl dusze swa w rece swej, i zabil Filistynczyka, i uczynil Pan wybawienie wielkie wszystkiemu Izraelowi. Cos widzial, i uradowales sie. Przeczzebys tedy mial grzeszyc przeciw krwi niewinnej, chcac zabic Dawida bez przyczyny?
\par 6 I usluchal Saul slów Jonatanowych, i przysiagl Saul: Jako zywy Pan, ze nie umrze.
\par 7 A tak przyzwal Jonatan Dawida, i opowiedzial mu Jonatan wszystkie one slowa; i przywiódl Jonatan Dawida do Saula, i byl przed nim, jako i przedtem.
\par 8 I wszczela sie znowu wojna, a ciagnal Dawid, i walczyl przeciwko Filistynom, i porazil je porazka wielka, i uciekli przed obliczem jego.
\par 9 Wtem Duch Panski zly przypadl na Saula, który w domu swym siedzial, majac wlócznia swoje w rece swej, a Dawid gral reka swa.
\par 10 I myslil Saul przebic Dawida wlócznia az ku scianie: ale sie uchylil przed Saulem, i uderzyla wlócznia w sciane, a Dawid uciekl, i uszedl onej nocy.
\par 11 Potem poslal Saul posly do domu Dawidowego, aby nan strzegli, i zabili go rano. I oznajmila to Dawidowi Michol, zona jego, mówiac: Jezlize nie ochronisz duszy twojej tej nocy, jutro zabity bedziesz.
\par 12 Przetoz spuscila Michol Dawida oknem, który uszedlszy uciekl, i zachowany jest.
\par 13 A wziawszy Michol obraz, polozyla na lozu, a wezglówko z koziej skóry polozyla pod glowe jego, i przykryla szata.
\par 14 Tedy poslal Saul posly, aby porwali Dawida; ale rzekla: Choruje.
\par 15 Znowu poslal Saul posly, aby ogladali Dawida, mówiac: Przyniescie go na lozu do mnie, abym go zabil.
\par 16 A gdy przyszli poslowie, oto obraz na lozu, a wezglówko z koziej skóry pod glowami jego.
\par 17 I rzekl Saul do Michol: Czemus mie tak oszukala, a wypuscilas nieprzyjaciela mego, aby uszedl? Tedy rzekla Michol Saulowi: Bo mi mówil, pusc mie, inaczej zabije cie.
\par 18 A tak Dawid ucieklszy uszedl, a przyszedl do Samuela, do Ramaty, i oznajmil mu wszystko, co mu czynil Saul. Poszedl tedy on i Samuel, a mieszkali w Najot.
\par 19 I oznajmiono Saulowi, mówiac: Oto Dawid jest w Najot w Ramacie.
\par 20 Tedy poslal Saul posly, aby pojmali Dawida; którzy gdy ujrzeli gromade proroków prorokujacych, i Samuela stojacego, a przelozonego nad nimi, przyszedl i na posly Saulowe Duch Bozy, a prorokowali i oni.
\par 21 Co gdy oznajmiono Saulowi, poslal inne posly, a prorokowali oni. Znowu tedy Saul poslal i trzecie posly, lecz prorokowali i ci.
\par 22 Potem szedl i sam do Ramaty, i przyszedl az do studni wielkiej, która jest w Sokot, i pytal sie a mówil: Gdzie jest Samuel i Dawid? i powiedziano mu: Oto sa w Najot w Ramacie.
\par 23 I szedl tam do Najot w Ramacie, a przyszedl tez nan Duch Bozy; a tak idac dalej prorokowal, az przyszedl do Najot w Ramacie.
\par 24 I zewlekl tez sam szaty swoje, a prorokowal i on przed Samuelem, a padlszy lezal nagim przez on caly dzien i przez cala noc; stadze weszlo w przypowiesc: Azaz i Saul miedzy prorokami?

\chapter{20}

\par 1 Ale Dawid ucieklszy z Najotu, który jest w Ramacie, przyszedl, i mówil przed Jonatanem, cózem uczynil? co za nieprawosc moja? i co za grzech mój przeciw ojcu twemu, ze szuka duszy mojej?
\par 2 Który mu odpowiedzial: Boze uchowaj! nie umrzesz; oto nie czyni ojciec mój nic wielkiego albo malego, az mi pierwej oznajmi; azazby taic mial ojciec mój przedemna i tego? Nie uczyni tego.
\par 3 A nadto przysiagl Dawid, rzeklszy: Wie zaiste ojciec twój, zem znalazl laske w oczach twoich, i mysli: Niech o tem niewie Jonatan, by sie snac nie frasowal; i owszem jako zywy Pan, zywa i dusza twoja, ze tylko krok jeden jest miedzy mna, i miedzy smiercia.
\par 4 I odpowiedzial Jonatan Dawidowi: Co mi kolwiek rzecze dusza twoja, uczyniec.
\par 5 Tedy rzekl Dawid do Jonatana: Oto, nów miesiaca jutro, a jam zwykl siadac z królem przy stole; pusc mie tedy, ze sie skryje na polu az do wieczora trzeciego dnia.
\par 6 A jezliby sie pilnie pytal o mnie ojciec twój, rzeczesz: Prosil mie bardzo Dawid, aby szedl do Betlehem, miasta swego; bo tam ofiare uroczysta ma sprawowac wszystka rodzina jego.
\par 7 Jezli tak rzecze: Dobrze, pokój bedzie sludze twemu; ale jezli sie rozgniewa, wiedz, iz sie dopelnila zlosc jego.
\par 8 Przetoz uczyn milosierdzie nad sluga twoim, gdyzes w przymierze Panskie przywiódl z soba sluge twego; a jezli we mnie jest nieprawosc, ty mie zabij; a do ojca twego przeczbys mie mial wodzic?
\par 9 I rzekl Jonatan: Boze cie tego uchowaj; bo jezli sie pewnie dowiem, ze sie dopelnila zlosc ojca mego, aby przyszla przeciw tobie, izalibym ci tego nie oznajmil?
\par 10 I rzekl Dawid do Jonatana: Któz mi oznajmi, jezlizec co odpowie ojciec twój przykrego?
\par 11 Odpowiedzial Jonatan Dawidowi: Pójdz, a wynijdzmy na pole. I wyszli obaj na pole.
\par 12 Tedy rzekl Jonatan do Dawida; Pan, Bóg Izraelski, (skoro sie wywiem o woli ojca mego o tym czasie jutro, albo dnia trzeciego, a bedzie co dobrego o Dawidzie, a jezli zarazem nie posle do ciebie, i nie oznajmiec,)
\par 13 To niech uczyni Pan, Bóg Izraelski, mówie, Jonatanowi, i to niech przyczyni. A jezlize bedzie chcial ojciec mój przywiesc zle na cie, i toc objawie, i puszcze cie, abys szedl w pokoju, a niech bedzie Pan z toba, jako byl z ojcem moim.
\par 14 Takze i ty, bedeli zyw, i ty mówie uczynisz ze mna milosierdzie Panskie, a chocbym i umarl,
\par 15 Przecie nie oddalisz milosierdzia twego od domu mego az na wieki, ani gdy wykorzeni Pan nieprzyjacioly Dawidowe wszystkie z ziemi.
\par 16 I uczynil Jonatan przymierze z domem Dawidowym, mówiac: Niech tego szuka Pan z reki nieprzyjaciól Dawidowych.
\par 17 Nadto jeszcze Jonatan przysiagl Dawidowi przez milosc, która go milowal; bo jako milowal dusze swoje, tak go tez milowal.
\par 18 I rzekl do niego Jonatan: Jutro nów miesiaca, a beda sie pytac o tobie, poniewaz prózne bedzie miejsce twoje.
\par 19 Przetoz przez trzy dni bedziesz sie ukrywal, i zstapisz predko, a przyjdziesz na miejsce, gdzies sie byl ukryl, gdy byla sprawa o tobie, a bedziesz siedzial u kamienia Ezel.
\par 20 A ja wystrzele trzy strzaly po bok jego, zmierzajac sobie do celu.
\par 21 A potem posle chlopca, mówiac mu: Idz, najdzij strzaly. A jezli rzeke chlopcu: Owo strzaly za toba sam blizej, przynies je, tedy przyjdz; bo masz pokój, i nie staniec sie nic zlego, jako zywy Pan.
\par 22 Ale jezliz rzeke chlopcu: Oto strzaly przed toba tam dalej; idz, bo cie wypuscil Pan.
\par 23 A tego, o czemesmy mówili ja i ty, tego Pan swiadkiem bedzie miedzy mna a miedzy toba az na wieki.
\par 24 A tak skryl sie Dawid w polu. A gdy przyszedl nów miesiaca, siadl król do stolu, aby jadl.
\par 25 A gdy usiadl król na stolicy swojej wedlug zwyczaju, na stolicy przy scianie, powstal Jonatan; i siadl Abner podle Saula, a zostalo prózne miejsce Dawidowe.
\par 26 Lecz nie rzekl Saul nic onego dnia, bo myslal: Przydalo mu sie podobno cos, lub jest czystym lub nieczystym.
\par 27 A gdy bylo nazajutrz dnia wtórego po nowiu miesiaca, bylo zas prózne miejsce Dawidowe. I rzekl Saul do Jonatana, syna swego: Czemuz nie przyszedl syn Isajego, ani wczoraj, ani dzis do stolu?
\par 28 Odpowiedzial Jonatan Saulowi: Usilnie mie prosil Dawid, aby szedl do Betlehem;
\par 29 I mówil: Pusc mie prosze, bo sprawuje ofiare rodzina nasza w miescie; tamze mie wezwal brat mój. A tak teraz jezlim znalazl laske w oczach twoich, pójde prosze, i ogladam bracia moje; dla tegoc nie przyszedl do stolu królewskiego.
\par 30 I zapalil sie gniewem Saul na Jonatana, i rzekl mu: Synu zlosliwy, a upornej matki, azaz nie wiem, izes sobie obral syna Isajego, ku zelzywosci twojej, i ku pohanbieniu i sromocie matki twojej?
\par 31 Bo po wszystkie dni, których syn Isajego bedzie zyl na ziemi, nie bedziesz umocniony, ty i królestwo twoje; a tak teraz poslij, a przywiedz go do mnie, bo jest godzien smierci.
\par 32 Tedy odpowiedzial Jonatan Saulowi, ojcu swemu, i rzekl do niego: Przecz ma umrzec? cóz uczynil?
\par 33 I cisnal Saul wlócznia na niego, aby go przebil. Tedy poznal Jonatan, ze koniecznie ojciec jego umyslil zabic Dawida.
\par 34 I wstal Jonatan od stolu z wielkim gniewem, i nie jadl dnia wtórego po nowiu miesiaca chleba; bo sie zafrasowal o Dawida, a iz go zelzyl ojciec jego.
\par 35 A rano wyszedl Jonatan na pole wedlug czasu postanowionego z Dawidem, i chlopiec maly z nim.
\par 36 I rzekl do chlopca swego: Biez, szukaj predko strzal, które ja wystrzele. Tedy chlopiec biezal; a on wystrzelil strzaly dalej przeden.
\par 37 A gdy przyszedl chlopiec az na miejsce strzaly, która byl wystrzelil Jonatan, zawolal Jonatan za chlopcem, i rzekl: Azaz strzala nie jest za toba tam dalej?
\par 38 I wolal Jonatan za chlopcem: Spiesz sie co najrychlej, nie stój. Tedy zebrawszy chlopiec Jonatana strzaly, przyszedl do pana swego.
\par 39 (Ale chlopiec nic nie wiedzial, tylko Jonatan i Dawid wiedzieli, co sie dzialo.)
\par 40 I dal Jonatan orez swój chlopcu, który z nim byl, i rzekl mu: Idz, odnies do miasta.
\par 41 A gdy odszedl chlopiec, Dawid wstal od strony poludniowej, i upadlszy twarza swoja na ziemie, uklonil sie po trzy kroc, i pocalowawszy jeden drugiego, plakali pospolu; ale Dawid obficiej.
\par 42 I rzekl Jonatan do Dawida: Idz w pokoju; a to, cosmy sobie obaj przysiegli przez imie Panskie, mówiac: Pan niech bedzie miedzy mna i miedzy toba, i miedzy nasieniem mojem, i miedzy nasieniem twojem swiadkiem az na wieki, trzymac bedziemy.
\par 43 A tak wstawszy Dawid odszedl, a Jonatan wszedl do miasta.

\chapter{21}

\par 1 Potem, przyszedl Dawid do Noby do Achimelecha kaplana, a zleklszy sie Achimelech wyszedl przeciwko Dawidowi, i rzekl mu: Przeczzes ty sam, a niemasz nikogo z toba?
\par 2 I odpowiedzal Dawid Achimelechowi kaplanowi: Rozkazal mi król nieco, i rzekl do mnie: Niech nikt nie wie tego, po co cie posylam, i com ci zlecil, przetozem slugi zostawil na pewnem miejscu.
\par 3 A tak teraz maszli co przy rekach twoich, aby z piecioro chleba, daj do reki mojej, albo cokolwiek znajdziesz.
\par 4 I odpowiedzial kaplan Dawidowi, i rzekl: Nie mam chleba pospolitego przy rece mojej, tylko chleb poswiecony; jezli sie tylko wstrzymali sludzy od niewiast.
\par 5 Tedy odpowiedzial Dawid kaplanowi, i rzekl mu: Zaiste niewiasty oddalone byly od nas od wczorajszego i dzis trzeciego dnia, gdym wyszedl; przetoz byly naczynia slug swiete. Ale jezli ta droga zmazana jest, wszakze i ta dzisiaj poswiecona bedzie w naczyniach.
\par 6 A tak dal mu kaplan chleby poswiecone; albowiem nie bylo tam chleba, tylko chleby pokladne, które byly odjete od oblicznosci Panskiej, aby polozono chleby cieple onegoz dnia, którego one odjete byly.
\par 7 A byl tam maz z slug Saulowych onego dnia, zabawiony przed Panem, którego imie Doeg, Edomczyk, najmozniejszy z pasterzy, które mial Saul.
\par 8 I rzekl Dawid do Achimelecha: A niemaszze tu przy rece swej wlóczni, albo miecza? bom ani miecza mego, ani zadnej broni mojej nie wzial w reke moje, gdyz slowo królewskie przynaglalo.
\par 9 Tedy rzekl kaplan: Miecz Golijata Filistynczyka, któregos zabil w dolinie Ela, oto jest uwiniony w sukno za efodem; jezli ten chcesz sobie wziac, wezmij; bo tu inszego niemasz oprócz tego. I rzekl Dawid: Niemasz podobnego temu, daj mi go.
\par 10 A tak wstal Dawid, i uciekl dnia onego przed Saulem, i przyszedl do Achisa, króla Getskiego.
\par 11 Tedy rzekli sludzy Achisowi do niego: Izali nie ten jest Dawid, król ziemi? Izali nie temu spiewano w hufcach, mówiac: Porazil Saul swój tysiac, a Dawid swoich dziesiec tysiecy?
\par 12 I zlozyl Dawid slowa te do serca swego, a bal sie bardzo Achisa, króla Getskiego.
\par 13 Przetoz zmienil obyczaje swoje przed oczyma ich, a czynil sie szalonym w rekach ich, i kreslil na drzwiach bramy, i puszczal sliny na brode swoje.
\par 14 Tedy rzekl Achis do slug swoich: Otoscie widzieli czlowieka szalonego, czemuzescie go przywiedli do mnie?
\par 15 Nie dostawa mi szalonych, zescie przywiedli tego, aby szalal przedemna? tenze ma wnijsc do domu mego?

\chapter{22}

\par 1 Potem wyszedl Dawid stamtad, i uszedl do jaskini Adullam. Co gdy uslyszeli bracia jego i wszystek dom jego, przyszli tam do niego.
\par 2 I zebrali sie do niego wszyscy, którzy byli utrapieni, i wszyscy, którzy byli dluzni, i wszyscy, którzykolwiek byli w gorzkosci serca, i byl nad nimi ksiazeciem, a bylo z nim okolo czterech set mezów.
\par 3 I poszedl Dawid stamtad do Masfa Moabskiego, i rzekl do króla Moabskiego: Niech sie przeprowadzi prosze ojciec mój, i matka moja, aby mieszkali z wami, az sie dowiem co uczyni Bóg ze mna.
\par 4 A tak przywiódl je przed króla Moabskiego; i mieszkali z nim po wszystkie dni, których byl Dawid na onym zamku.
\par 5 Rzekl potem Gad prorok do Dawida: Nie mieszkaj wiecej na tym zamku; idz, a wróc sie do ziemi Judzkiej. Tedy poszedl Dawid, a przyszedl do lasu Haret.
\par 6 A uslyszawszy Saul, ze sie pojawil Dawid, i mezowie, którzy byli z nim, (bo Saul mieszkal w Gabaa pod gajem w Ramacie, majac wlócznia swoje w rekach swych, a wszyscy sludzy jego stali przed nim).
\par 7 Rzekl tedy Saul do slug swych, którzy stali przed nim: Sluchajcie prosze synowie Jemini: Izaz wam wszystkim da syn Isajego role i winnice, a wszystkich was poczyni pólkownikami i rotmistrzami.
\par 8 Zescie sie sprzysiegli wy wszyscy przeciwko mnie, a niemasz ktoby mi objawil? gdyz sie zbuntowal i syn mój z synem Isajego, a niemasz ktoby sie mnie uzalil miedzy wami, a oznajmil mi, iz podburzyl syn mój sluge mego przeciwko mnie, aby czyhal na mie, jako sie to dzis okazuje.
\par 9 Tedy odpowiedzial Doeg Edomczyk, który tez stal z slugami Saulowymi, i rzekl: Widzialem syna Isajego, gdy przyszedl do Noby, do Achimelecha, syna Achitobowego.
\par 10 Który sie on radzil Pana, i dal mu zywnosci, dal mu tez i miecz Golijata Filistynczyka.
\par 11 A tak poslal król, aby przyzwano Achimelecha, syna Achitobowego, kaplana, i wszystkiego domu ojca jego kaplanów, którzy byli w Nobe. I przyszli oni wszyscy do króla.
\par 12 Tedy rzekl Saul: Sluchaj teraz synu Achitoba; a on rzekl: Owom ja panie mój.
\par 13 I rzekl do niego Saul: Czemuscie sie sprzysiegli przeciwko mnie, ty i syn Isajego, gdys mu dal chleb i miecz, a radziles sie on Boga, aby powstal przeciwko mnie, czyhajac na mie, jako sie to dzis okazuje?
\par 14 I odpowiedzial Achimelech królowi, a rzekl: I któz jest tak wierny miedzy wszystkimi slugami twoimi, jako Dawid, który jest i zieciem królewskim i idzie za rozkazaniem twojem, a jest zacnym w domu twoim?
\par 15 Azaz dzis poczalem sie on radzic Boga? Uchowaj mie Boze! Niech nie wklada król na sluge twego nic takiego, ani na wszystek dom ojca mego; bo nie wiedzial sluga twój o tem wszystkiem najmniejszej rzeczy.
\par 16 I rzekl król: Smiercia umrzesz Achimelechu, ty i wszystek dom ojca twego.
\par 17 Przytem rzekl król slugom, którzy stali przed nim: Obróccie sie, a pobijcie kaplany Panskie; bo tez reka ich jest z Dawidem, gdyz wiedzac, ze on uciekal, nie oznajmili mi. Ale sludzy królewscy nie chcieli podniesc reki swej, ani sie rzucic na kaplany Panskie.
\par 18 Przetoz rzekl król do Doega: Obróc sie ty, a rzuc sie na kaplany. A tak obróciwszy sie Doeg Edomczyk, rzucil sie na kaplany, i zabil onegoz dnia osmdziesiat i piec mezów, którzy nosili efod lniane.
\par 19 Nobe tez miasto kaplanskie wysiekl ostrzem miecza, od meza az do niewiasty, od malego az do ssacego, i woly, i osly, i owce wysiekl ostrzem miecza.
\par 20 Uszedl tylko syn jeden Achimelecha, syna Achitobowego; a imie jego Abijatar; i uciekl do Dawida.
\par 21 Tedy oznajmil Abijatar Dawidowi, ze pobil Saul kaplany Panskie.
\par 22 I rzekl Dawid do Abijatara: Wiedzialem onegoz dnia, gdyz tam byl Doeg Edomczyk, ze pewnie oznajmic mial Saulowi: Jam jest przyczyna smierci wszystkich dusz domu ojca twego.
\par 23 Zostanze przy mnie, nie bój sie; bo ktoby szukal duszy mojej, bedzie szukal duszy twojej; ale ty bedziesz schroniony przy mnie.

\chapter{23}

\par 1 Tedy powiedziano Dawidowi, mówiac: Oto, Filistynowie dobywaja Ceili, i plondruja gumna.
\par 2 I radzil sie Dawid Pana, mówiac: Mamli isc, a uderzyc na te Filistyny? I odpowiedzial Pan Dawidowi: Idz, a porazisz Filistyny, i wybawisz Ceile.
\par 3 Tedy rzekli mezowie Dawidowi do niego: Oto my tu w Judzkiej ziemi boimy sie, jakoz daleko wiecej, jezli pójdziemy do Ceili przeciw wojskom Filistynskim.
\par 4 I pytal sie jeszcze powtóre Dawid Pana. A odpowiedzial mu Pan, mówiac: Wstawszy idz do Ceili; bo ja dam Filistyny w rece twoje.
\par 5 Poszedl tedy Dawid i mezowie jego do Ceili, i walczyl z Filistynami, i zabral bydla ich, i porazil ich porazka wielka, i wybawil Dawid obywatele Ceili.
\par 6 I stalo sie, gdy uciekal Abijatar, syn Achimelecha, do Dawida do Ceili, ze sie dostal efod w reke jego.
\par 7 Potem powiedziano Saulowi, iz Dawid przyszedl do Ceili. Tedy rzekl Saul: Dal go Bóg w rece moje; bo sie zawarl, wszedlszy do miasta, w którem sa bramy i zamki.
\par 8 A tak zebral Saul wszystek lud, aby szedl na wojne do Ceili, i oblegl Dawida, i meze jego.
\par 9 Co gdy wzwiedzial Dawid, iz Saul potajemnie przeciw niemu myslal wszystko zle, tedy rzekl do Abijatara kaplana: Wlóz na sie efod.
\par 10 I rzekl Dawid: Panie, Boze Izraelski, za pewne slyszal sluga twój, ze Saul chce przyjsc do Ceili, aby miasto zburzyl dla mnie;
\par 11 Wydadzali mnie starsi miasta Ceili w rece jego? przyjdzieli tez Saul, jako slyszal sluga twój? Panie, Boze Izraelski, oznajmij prosze sludze twemu. I odpowiedzial Pan: Przyjdzie.
\par 12 Nadto rzekl Dawid: Wydadzali starsi z Ceili mnie i meze moje w rece Saulowe? I odpowiedzial Pan: Wydadza.
\par 13 Wstal tedy Dawid i mezowie jego okolo szesciu set mezów, i wyszli z Ceili, a uchodzili kedy mogli. A gdy oznajmiono Saulowi, ze uszedl Dawid z Ceili, tedy zaniechal wyciagnienia.
\par 14 I mieszkal Dawid na puszczy w miejscach obronnych, a zostal na górze w puszczy Zyf. I szukal go Saul po wszystkie dni; lecz nie podal go Bóg w rece jego.
\par 15 A widzac Dawid, ze wyszedl Saul, aby szukal dusze jego, zostal Dawid na puszczy Zyf w lesie.
\par 16 Wtedy wstal Jonatan, syn Saula, i szedl do Dawida do lasu, i posilil reke jego w Bogu,
\par 17 Mówiac do niego: Nie bój sie, bo cie nie znajdzie reka Saula, ojca mego; a ty bedziesz królowal nad Izraelem, ja zas bede wtórym po tobie; wszak i Saul, ojciec mój, wie o tem.
\par 18 I uczynili obaj z soba przymierze przed Panem; i zostal Dawid w lesie, ale Jonatan wrócil sie do domu ojca swego.
\par 19 Tedy przyszli Zyfejczycy do Saula do Gabaa, powiadajac: Azaz Dawid nie kryje sie u nas po miejscach obronnych w lesie na pagórku Hachila, który jest po prawej stronie Jesymona?
\par 20 Przetoz teraz wedlug wszystkiej zadnosci duszy twojej, królu, zejdz co najrychlej, a my sie postaramy, ze go wydamy w rece królewskie.
\par 21 Tedy rzekl Saul: Blogoslawieniscie wy od Pana, zescie sie mnie uzalili.
\par 22 Idzciez prosze, a starajcie sie tem pilniej; wywiedzcie sie, a wyszpiegujcie to miejsce jego, gdzie sie obraca. Kto go tam widzial? bo mi powiadano, ze sobie bardzo chytrze postepuje.
\par 23 Wypatrzciez tedy, a obaczcie wszystkie te miejsca skryte, w których sie ukrywa; potem wrócicie sie do mnie z czem pewnem, i pójde z wami; a bedzieli w ziemi, tedy go bede szukal po wszystkich tysiacach Judzkich.
\par 24 Wstali tedy, i poszli do Zyf przed Saulem; ale Dawid i mezowie jego byli na puszczy Maon, w polach po prawej stronie Jesymon.
\par 25 Bo gdy wyszedl Saul, i mezowie jego, szukac go, oznajmiono Dawidowi, który zstapil z skaly, i mieszkal na puszczy Maon. Co uslyszawszy Saul, gonil Dawida az na puszcze Maon.
\par 26 I szedl Saul po jednej stronie góry, a Dawid i mezowie jego po drugiej stronie góry. I spieszyl sie Dawid, aby mógl ujsc przed Saulem; bo Saul i lud jego obtaczali Dawida i meze jego, aby je pojmali.
\par 27 Wtem posel przybiezal do Saula, mówiac: Pospiesz sie, a pójdz; albowiem Filistynowie wtargneli w ziemie.
\par 28 Przetoz wrócil sie Saul od pogoni za Dawidem, a ciagnal przeciw Filistynom; dla tego nazwali miejsce ono Sela Hammalekot.

\chapter{24}

\par 1 A tak wyciagnal stamtad Dawid, i mieszkal na miejscach obronnych Engaddy.
\par 2 I stalo sie, gdy sie wrócil Saul z pogoni za Filistynami, powiedziano mu, mówiac: Oto, Dawid jest na puszczy Engaddy.
\par 3 Wziawszy tedy Saul trzy tysiace mezów przebranych z wszystkiego Izraela, poszedl szukac Dawida i mezów jego, po wierzchu skal kóz dzikich.
\par 4 I przyszedl ku oborom owczym, które byly podle drogi, kedy byla jaskinia; do której wszedl Saul na potrzebe przyrodzona, a Dawid i mezowie jego siedzieli po stronach jaskini.
\par 5 I rzekli mezowie Dawidowi do niego: Oto dzien, o którym ci powiedzial Pan: Oto Ja dawam nieprzyjaciela twego w rece twoje, a uczynisz mu, jako sie bedzie podobalo w oczach twoich. Wstal tedy Dawid, i urznal po cichu kraj plaszcza Saulowego.
\par 6 I stalo sie, ze uderzylo Dawida serce jego, przeto ze urznal kraj plaszcza Saulowego.
\par 7 I rzekl do mezów swoich: Uchowaj mie tego Panie, zebym to uczynic mial panu memu, pomazancowi Panskiemu, zebym mial sciagnac nan reke moje, poniewaz jest pomazancem Panskim.
\par 8 I przelomil Dawid meze swe slowy, a nie dopuscil im powstac przeciwko Saulowi; zatem Saul wstawszy z jaskini, poszedl w droge.
\par 9 Potem tez Dawid wstal, i wyszedl z jaskini, a zawolal za Saulem, mówiac: Królu, Panie mój! Tedy sie obejrzal Saul nazad, a Dawid schyliwszy sie twarza ku ziemi, poklonil sie.
\par 10 I rzekl Dawid do Saula: Czemuz sluchasz powiesci ludzi mówiacych: Otóz Dawid szuka twego zlego:
\par 11 Oto, dnia tego widza oczy twoje, ze cie byl podal Pan w rece moje w jaskini, i mówiono mi, abym cie zabil; alem ci sfolgowal, i rzeklem: Nie sciagne reki mojej na pana mego; bo jest pomazancem Panskim.
\par 12 Oto, ojcze mój, obacz a ogladaj kraj plaszcza twego w rece mojej, ze gdym urzynal kraj plaszcza twego, nie zabilem cie. Poznaj a obacz, ze niemasz w rece mojej zlosci i nieprawosci, anim zgrzeszyl przeciwko tobie: a ty godzisz na dusze moje, abys mi ja odjal.
\par 13 Niech rozsadzi Pan miedzy mna i miedzy toba, a niech sie zemsci Pan krzywdy mojej nad toba; lecz reka moja nie bedzie na tobie.
\par 14 Jako mówi przypowiesc starodawna: Od niezboznych wynijdzie niezboznosc; przetoz reka moja nie bedzie na tobie.
\par 15 Za kimze wzdy wyszedl król Izraelski? kogóz gonisz? psa zdechlego? pchle jedne?
\par 16 Niechze bedzie Pan sedzia, a niech rozsadzi miedzy mna i miedzy toba, a niech obaczy i rozejmie prza moje, a niech mie wyswobodzi z reki twojej.
\par 17 A gdy przestal Dawid mówic slów tych do Saula, rzekl Saul: A twójze to glos, synu mój Dawidzie? I podnióslszy Saul glos swój, plakal.
\par 18 I rzekl do Dawida: Sprawiedliwszys ty nizli ja: bo tys mnie oddal dobrem, a jam tobie oddal zlem.
\par 19 Tys zaiste okazal dzisiaj, zes mi uczynil dobre; bo choc mie podal Pan w reke twoje, przecies mie nie zabil.
\par 20 Izaz znalazlszy kto nieprzyjaciela swego, wypusci go na droge dobra? niechajzec Pan dobrem odda za to, cos mi dzis uczynil.
\par 21 A teraz oto wiem, ze zapewne bedziesz królowal, a ostoi sie w rece twojej królestwo Izraelskie.
\par 22 Przetoz prosze, przysiaz mi przez Pana, ze nie wygubisz nasienia mego po mnie, i nie wytracisz imienia mego z domu ojca mego.
\par 23 A tak przysiagl Dawid Saulowi. I odszedl Saul do domu swego, a Dawid i mezowie jego poszli na miejsca obronne.

\chapter{25}

\par 1 W tem umarl Samuel. A zebrawszy sie wszyscy Izraelczycy, plakali go, i pogrzebli go w domu jego w Ramacie. Tedy wstawszy Dawid, poszedl na puszcze Faran.
\par 2 A byl niektóry maz w Maon, który mial majetnosc na Karmelu; a on maz byl mozny bardzo, majac owiec trzy tysiace, a tysiac kóz; i trafilo sie, ze strzygl owce swoje na Karmelu.
\par 3 A bylo imie meza onego Nabal, a imie zony jego Abigail, która niewiasta byla madra, i piekna; ale maz jej byl nieuzyty i zlych postepków, a byl narodu Kalebowego.
\par 4 A uslyszawszy Dawid na puszczy, iz Nabal strzygl owce swoje,
\par 5 Poslal dziesieciu slug, i rzekl im: Idzcie do Karmelu, a przyszedlszy do Nabala, pozdrówcie go imieniem mojem spokojnie,
\par 6 A mówcie tak: Zyj, a niech bedzie tobie pokój, i domowi twemu pokój, i wszystkiemu, co masz, pokój!
\par 7 A teraz slyszalem, ze masz te, coc owce strzyga; a pasterze twoi bywali z nami, niebylismy im przykrymi, i nic im nie zginelo po wszystkie dni, których byli w Karmelu;
\par 8 Spytaj slug twoich, a powiedzac. Przetoz niech znajda sludzy laske w oczach twoich, gdyzesmy w dobry dzien przyszli; daj prosze cokolwiek znajdzie reka twoja, slugom twoim, i synowi twemu Dawidowi.
\par 9 A tak przyszli sludzy Dawidowi, i mówili do Nabala wszystkie one slowa imieniem Dawidowem, i przestali.
\par 10 A odpowiadajac Nabal slugom Dawidowym, rzekl: Cóz jest Dawid? a co zacz syn Isajego? dzis sie namnozylo slug, którzy uciekaja od panów swoich.
\par 11 I wezmez ja chleb mój, i wode moje, i miesa bydla mego, którem pobil dla tych, którzy strzyga owce moje, a dam mezom, których nie znam, skad sa?
\par 12 A obróciwszy sie sludzy Dawidowi w droge swoje, wrócili sie, i przyszli a powiedzieli mu wszystkie te slowa.
\par 13 Tedy rzekl Dawid mezom swym: Przypaszcie kazdy miecz swój; przypasal tez i Dawid miecz swój, i szlo za Dawidem okolo czterech set mezów, a dwiescie zostalo przy rzeczach.
\par 14 Ale Abigaili, zonie Nabalowej, oznajmil to jeden czeladnik z slug Nabalowych, mówiac: Oto przyslal Dawid posly z puszczy, aby blogoslawili Panu naszemu, lecz on je sfukal.
\par 15 A mezowie ci dobrzy nam byli bardzo, i nie przykrzyli sie nam; nic nam nie zginelo po wszystkie dni, pókismy z nimi chodzili, bedac na polu;
\par 16 Miasto muru byli nam, tak w nocy jako we dnie, po wszystkie dni, pókismy przy nich trzody pasli.
\par 17 Przetoz teraz obacz, a rozmysl sie, co masz czynic; boc juz gotowe nieszczescie na pana naszego, i na wszystek dom jego; lecz on jest czlowiekiem niezboznym, ze z nim trudno mówic.
\par 18 Pospieszyla sie tedy Abigail, i wziela dwiescie chleba, i dwie lagwi wina, i piec owiec oprawnych, i piec miar prazma, i sto wiazanek rodzynków, i dwiescie funtów fig, a wlozyla to na osly;
\par 19 I rzekla slugom swoim: Idzcie przedemna, a ja pojade za wami; ale mezowi swemu Nabalowi nie oznajmila.
\par 20 I stalo sie, ze jadac na osle, i zjezdzajac glebia góry, oto tez Dawid i mezowie jego zjezdzali przeciwko niej, i spotkala sie z nimi.
\par 21 (A Dawid byl rzekl: Zaprawde darmom strzegl wszystkiego, co ten mial na puszczy, ze nic nie zginelo ze wszystkiego, co ma; bo mi oddal zlem za dobre.
\par 22 To niech uczyni Bóg nieprzyjaciolom Dawidowym, i to niech przyczyni, jezli co do zarania zostawie ze wszystkiego co ma, az do najmniejszego szczeniecia).
\par 23 Tedy ujrzawszy Abigail Dawida, pospieszyla sie, i zsiadla z osla, i upadla przed Dawidem na oblicze swoje, i uklonila sie az do ziemi;
\par 24 A upadlszy do nóg jego, mówila: Niech bedzie na mnie, panie mój, ta nieprawosc, a niech mówi prosze sluzebnica twoja do uszu twoich, i posluchaj slów sluzebnicy twojej.
\par 25 Niech sie prosze nie obraza pan mój na meza tego bezboznego Nabala, gdyz wedlug imienia swego takim jest. Nabal jest imie jego, i glupstwo jest przy nim; alemci ja sluzebnica twoja nie widziala slug pana mego, któres byl poslal.
\par 26 Przetoz teraz panie mój, jako zywy Pan, jako zywa i dusza twoja, ze cie zawsciagnal Pan, abys nie szedl na rozlanie krwi, i zeby sie nie mscila reka twoja; a teraz niech beda jako Nabal nieprzyjaciele twoi, i którzy szukaja zlego panu memu.
\par 27 Teraz tedy to blogoslawienstwo, które przyniosla sluzebnica twoja panu swemu, niech bedzie dane slugom, którzy chodza za panem moim.
\par 28 Przepusc prosze wystepek sluzebnicy twojej, gdyz zapewne uczyni Pan panu memu dom trwaly, poniewaz walki Panskie pan mój odprawuje, a zlosc nie jest znalezina w tobie az dotad.
\par 29 A chocby powstal czlowiek, coby cie przesladowal, i szukal duszy twojej, tedy bedzie dusza pana mego zachowana w wiazance zywiacych u Pana, Boga twego; lecz dusze nieprzyjaciól twoich Bóg jako z procy wyrzuci.
\par 30 A gdy uczyni Pan panu memu wszystko, co mówil dobrego o tobie, a zlecic, abys byl wodzem nad Izraelem:
\par 31 Tedyc nie bedzie to ku zachwianiu, ani ku urazie serca pana mego, jako gdyby rozlal krew niewinna, i gdyby sie sam pomscil pan mój. Gdy tedy dobrze uczyni Pan panu memu, wspomnisz na sluzebnice twoje.
\par 32 I rzekl Dawid do Abigaili: Blogoslawiony Pan, Bóg Izraelski, który cie dzis poslal przeciwko mnie.
\par 33 Blogoslawiona wymowa twoja, i blogoslawionas ty, któras mie zawsciagnela dzisiaj, zem nie szedl na rozlanie krwi, a zem sie sam nie mscil krzywdy swojej.
\par 34 A zaprawde, jako zywy Pan, Bóg Izraelski, który mie zawsciagnal, abym ci nic zlego nie uczynil: bo gdybys sie byla nie pospieszyla, a nie zajechala mi drogi, nie zostaloby bylo Nabalowi az do switania, i najmniejszego szczeniecia.
\par 35 A tak przyjal Dawid z reki jej, co mu byla przyniosla, i rzekl do niej: Idz w pokoju do domu twego: otom usluchal glosu twego, i przyjalem cie laskawie.
\par 36 Tedy sie wrócila Abigail do Nabala; a on mial uczte w domu swoim, jako uczte królewska, a serce Nabalowe bylo wesole w nim, a byl pijany bardzo; i nie oznajmila mu najmniejszej rzeczy az do poranku.
\par 37 Ale nazajutrz, gdy wytrzezwial Nabal z wina, oznajmila mu zona jego te rzeczy: i zmartwialo w nim serce jego, i stal sie jako kamien
\par 38 A gdy wyszlo jakoby dziesiec dni, uderzyl Pan Nabala, i umarl.
\par 39 A uslyszawszy Dawid, iz umarl Nabal, rzekl: Blogoslawiony Pan, który sie pomscil pohanbienia mego nad Nabalem, a sluge swego zatrzymal od zlego, a zlosc Nabalowe obrócil Pan na glowe jego. Tedy poslal Dawid wskazujac do Abigaili, ze ja sobie chce wziac za zone.
\par 40 I przyszli sludzy Dawidowi do Abigaili do Karmelu, i rzekli do niej, mówiac: Dawid poslal nas do ciebie, aby cie sobie wzial za zone.
\par 41 Która wstawszy, poklonila sie obliczem do ziemi, i rzekla: Oto, sluzebnica twoja niech bedzie sluga, aby umywala nogi slug pana mego.
\par 42 Przetoz pospieszywszy sie wstala Abigail, i wsiadla na osla z piecioma panienkami swemi, które z nia chodzily; i tak jechala za poslami Dawidowymi, a byla mu za zone.
\par 43 Ale i Achinoame wzial Dawid z Jezreel, i byly mu te dwie za zony.
\par 44 Albowiem Saul dal byl Michol, córke swoje, zone Dawidowe, Faltemu, synowi Laisowemu, który byl z Gallim.

\chapter{26}

\par 1 I przyszli Zyfejczycy do Saula do Gabaa, a mówili: Izali sie nie kryje Dawid na pagórku Hachila, przeciw Jesymon?
\par 2 Ruszyl sie tedy Saul, i ciagnal na puszcza Zyf, a z nim trzy tysiace mezów przebranych z Izraela, aby szukal Dawida na puszczy Zyf.
\par 3 I polozyl sie Saul obozem na pagórku Hachila, które jest przeciw Jesymon podle drogi; a Dawid mieszkal na puszczy, i dowiedzial sie, ze przyciagnal Saul za nim na puszcza,
\par 4 Bo poslawszy Dawid szpiegi dowiedzial sie, ze przyciagnal Saul zapewne.
\par 5 Przetoz wstal Dawid, i przyszedl az ku miejscu, gdzie sie polozyl obozem Saul; i upatrzyl Dawid miejsce, gdzie spal Saul, i Abner, syn Nera, hetman wojska jego; bo Saul spal w obozie, a lud lezal okolo niego.
\par 6 I odpowiedzial Dawid, a rzekl do Achimelecha Hetejczyka, i do Abisajego, syna Sarwii, brata Joabowego, mówiac: Któz pójdzie ze mna do Saula do obozu? Odpowiedzial Abisaj: Ja z toba pójde.
\par 7 A tak przyszedl Dawid i Abisaj do ludu w nocy, a oto, Saul lezac spal w obozie, a wlócznia jego byla utkniona w ziemi u glowy jego; Abner tez z ludem lezeli okolo niego.
\par 8 Tedy rzekl Abisaj do Dawida: Zamknal dzis Bóg nieprzyjaciela twego w rece twoje, a teraz niech go przebije prosze wlócznia ku ziemi raz, a wiecej nie powtórze.
\par 9 Ale rzekl Dawid do Abisajego: Nie zabijaj go; bo któz sciagnawszy reke swa na pomazanca Panskiego, niewinnym bedzie?
\par 10 Nadto rzekl Dawid: Jako zyje Pan, ze jezli go Pan nie zabije, albo dzien jego nie przyjdzie, aby umarl, albo na wojne wyjechawszy, nie zginie,
\par 11 Tedy uchowaj mie Panie, abym mial sciagnac reke moje na pomazanca Panskiego; ale wezmij prosze wlócznia, która jest u glów jego, i kubek od wody, a odejdzmy.
\par 12 Tedy wzial Dawid wlócznia, i kubek od wody, który byl u glów Saulowych, i odeszli, a nie byl, ktoby widzial, ani ktoby wiedzial, ani ktoby sie ocucil, ale wszyscy spali; bo sen twardy od Pana przypadl byl na nie.
\par 13 I przyszedl Dawid na druga strone, i stanal na wierzchu góry z daleka, a byl plac wielki miedzy nimi.
\par 14 I zawolal Dawid na lud, i na Abnera, syna Nerowego, mówiac: Nie ozwieszze sie Abnerze? I odpowiadajac Abner, rzekl: Któzes ty, co wolasz na króla?
\par 15 I rzekl Dawid do Abnera: Azaz ty nie maz? A któz jakos ty w Izraelu? przeczzes tedy nie strzegl króla, pana twego? bo przyszedl jeden z ludu, chcac zabic króla, pana twego.
\par 16 Nie dobra to, cos uczynil. Jako zywy Pan, zescie winni smierci, którzyscie nie strzegli pana waszego, pomazanca Panskiego. A teraz, patrz, kedy jest wlócznia królewska, i kubek od wody, co byl w glowach jego?
\par 17 Poznal tedy Saul glos Dawida, i rzekl: Twójze to glos, synu mój Dawidzie? Odpowiedzial Dawid: Glos to mój, królu, panie mój.
\par 18 Nadto rzekl: Czemuz pan mój przesladuje sluge swego? bo cózem uczynil? a co jest zlego w rece mojej?
\par 19 Przetoz teraz niech poslucha prosze król, pan mój, slów slugi swego; jezli cie Pan pobudzil przeciwko mnie, niech powonia ofiary; ale jezli synowie Judzcy, przekleci sa przed Panem, którzy mie dzis wygnali, abym nie mieszkal w dziedzictwie Panskiem, jakoby rzekli: Idz, sluz bogom cudzym.
\par 20 A teraz niech nie bedzie wylana krew moja na ziemie przed obliczem Panskiem; bo wyszedl król Izraelski szukac pchly jednej, jakoby tez kto gonil kuropatwe po górach.
\par 21 Tedy rzekl Saul: Zgrzeszylem. Wrócze sie, synu mój Dawidzie, boc juz nic zlego nie uczynie wiecej, poniewaz droga byla dusza moja w oczach twoich dnia tego; otom glupio uczynil, i zbladzilem nader bardzo.
\par 22 A odpowiadajac Dawid rzekl: Oto wlócznia królewska; niech sam przyjdzie kto z slug, a wezmie ja.
\par 23 A Pan niech odda kazdemu sprawiedliwosc jego, i wiare jego. Albowiem podal cie byl Pan dzis w rece moje; alem nie chcial sciagnac reki mojej na pomazanca Panskiego.
\par 24 Przetoz jako dzis powazona byla dusza twoja w oczach moich, tak niech bedzie powazona dusza moja w oczach Panskich, a niech mie wyrwie Pan ze wszego ucisku.
\par 25 I rzekl Saul do Dawida: Blogoslawionys ty, synu mój Dawidzie; tak czyniac dokazesz, a tak sie wzmacniajac, mocnym bedziesz. Odszedl potem Dawid w droge swa, a Saul sie wrócil na miejsce swoje,

\chapter{27}

\par 1 Tedy rzekl Dawid w sercu swojem: Zgine ja kiedyzkolwiek od reki Saulowej. Azaz mnie nie lepiej, abym co predzej uszedl do ziemi Filistynskiej, aby zwatpil o mnie Saul, i nie szukal mie wiecej po wszystkich granicach Izraelskich, i tak abym uszedl rak jego?
\par 2 Wstawszy tedy Dawid, poszedl sam i onych szesc set mezów, którzy byli z nim, do Achisa, syna Maocha, króla Get.
\par 3 I mnieszkal Dawid przy Achisie w Get, sam i mezowie jego, kazdy z czeladzia swoja, Dawid i dwie zony jego, Achinoam Jezreelitka, i Abigail, zona przedtem Nabalowa z Karmelu.
\par 4 A gdy powiedziano Saulowi, ze uciekl Dawid do Get, przestal go wiecej szukac.
\par 5 I rzekl Dawid do Achisa: Jezlim prosze znalazl laske w oczach twoich, niech mi dadza miejsce w jednem z miast tego kraju, abym tam mieszkal; bo czemuzby mial mieszkac sluga twój w miescie królewskiem z toba?
\par 6 I dal mu Achis dnia onego Syceleg; dla tego Syceleg bylo królów Judzkich az do dnia tego.
\par 7 A byla liczba dni, których mieszkal Dawid w krainie Filistynskiej, rok i cztery miesiace.
\par 8 I wypadal Dawid i mezowie jego, a wtargiwali do Giessurytów, i do Gierzytów, i do Amalekitów; bo ci mieszkali w onej ziemi zdawna; któredy chodza przez Sur az do ziemi Egipskiej.
\par 9 I pustoszyl Dawid ziemie one, a nie zostawial zywego meza i niewiasty; a zabrawszy owce, i woly, i osly i wielblady, i szaty, wracal sie zasie, i przychadzal do Achisa.
\par 10 A gdy sie pytal Achis: Gdziezescie byli dzis wpadli? odpowiadal Dawid: Ku poludniu Judy, i ku poludniu w Jerameel, i ku poludniu Ceni.
\par 11 Ale Dawid nie zywil meza, ani niewiasty, ani ich przywodzil do Get, mówiac: By snac nie skarzyli na nas, mówiac: Tak uczynil Dawid. I byl to jego zwyczaj po wszystkie dni, póki mieszkal w ziemi Filistynskiej.
\par 12 Wierzyl tedy Achis Dawidowi i mówil: Prawie sie juz stal obrzydlym ludowi swemu Izraelskiemu, a tak bedzie mi sluga wiecznym.

\chapter{28}

\par 1 I stalo sie w one dni, ze zebrali Filistynowie wojska swe na wojne, aby walczyli z Izraelem. Tedy Achis rzekl do Dawida: Wiedz wiedzac, iz ze mna pociagniesz na wojne, ty i mezowie twoi.
\par 2 I odpowiedzial Dawid Achisowi: Dopiero sie ty dowiesz, co uczyni sluga twój. I rzekl Achis do Dawida: Zaiste strózem glowy mojej postanowie cie po wszystkie dni.
\par 3 A Samuel juz byl umarl, i plakal go wszystek Izrael, i pogrzebli go w Ramacie, miescie jego; a Saul wygnal byl wieszczki i czarowniki z ziemi.
\par 4 Zebrawszy sie tedy Filistynowie, przyciagneli, a polozyli sie obozem u Sunam; zebral tez Saul wszystkiego Izraela, a polozyl sie obozem w Gielboe.
\par 5 A widzac Saul obóz Filistynski, bal sie, a uleklo sie serce jego bardzo.
\par 6 I radzil sie Saul Pana; ale mu nie odpowiedzial Pan ani przez sny, ani przez urym, ani przez proroki;
\par 7 Przetoz rzekl Saul do slug swoich: Szukajcie mi niewiasty, któraby miala ducha wieszczego, a pójde do niej, i wywiem sie przez nie. I rzekli sludzy jego do niego: Oto, niewiasta w Endor, majaca ducha wieszczego.
\par 8 Tedy odmienil odzienie swoje Saul, a oblóklszy sie w insze szaty, szedl sam i dwaj mezowie z nim, a przyszli do niewiasty w nocy, i rzekl: Wróz mi, prosze, przez ducha wieszczego, a wywiedz tego, kogoc powiem.
\par 9 I rzekla do niego niewiasta: Oto ty wiesz, co uczynil Saul, iz wygladzil wieszczki i czarowniki z ziemi; przeczze ty sidlo kladziesz na dusze moje, abys mie na smierc podal?
\par 10 I przysiagl jej Saul przez Pana, mówiac: Jako zywy Pan, ze nie przyjdzie na cie karanie dla tego.
\par 11 Tedy rzekla niewiasta: Kogoz ci mam wywiesc? A on rzekl: Wywiedz mi Samuela.
\par 12 A widzac niewiasta Samuela, zawolala glosem wielkim, i rzekla niewiasta do Saula, mówiac: Przeczzes mie zdradzil, gdyzes ty jest Saul?
\par 13 I rzekl jej król: Nie bój sie; cózes widziala? I rzekla niewiasta do Saula: Widzialam bogi wystepujace z ziemi.
\par 14 Tedy rzekl do niej: Co za osoba jego? I rzekla: Maz stary wyszedl, a ten odziany plaszczem. I poznal Saul, ze to byl Samuel, i schyliwszy sie twarza ku ziemi, poklonil mu sie.
\par 15 Zatem rzekl Samuel do Saula: Przecz mi nie dasz pokoju, wzbudzajac mie? Odpowiedzial mu Saul: Jestem ucisniony bardzo, gdyz Filistynowie walcza przeciwko mnie, a Bóg odstapil odemnie, i nie odpowiada mi wiecej, ani przez proroki, ani przez sny; przetoz przyzwalem cie, abys mi oznajmil, co mam czynic.
\par 16 I rzekl Samuel: Czemuz mie tedy pytasz, gdyz Pan odstapil od ciebie, a przestawa z nieprzyjacielem twoim?
\par 17 I uczynil mu Pan, jakoc powiedzial przez mie, i wyrwal Pan królestwo z rak twoich, a dal je blizniemu twemu Dawidowi.
\par 18 Bo zes ty nie byl poslusznym glosowi Panskiemu, anis wykonal gniewu zapalczywosci jego nad Amalekiem, przetozci to uczynil Pan dzisiaj.
\par 19 Nadto poda Pan i Izraela z toba w reke Filistynów, a jutro ty i synowie twoi ze mna bedziecie; obóz tez Izraelski poda Pan w rece Filistynów.
\par 20 A natychmiast Saul upadl jako dlugi na ziemie, bo sie zlakl bardzo slów Samuelowych, i sily nie bylo w nim, przeto ze nic nie jadl przez caly dzien i przez cala noc.
\par 21 Potem weszla niewiasta do Saula, a widzac, iz sie bardzo przelakl, rzekla mu: Oto, usluchala sluzebnica twoja glosu twego, i odwazylam zdrowie swoje, i usluchalam slów twoich, któres mówil do mnie.
\par 22 Przetoz teraz usluchaj prosze i ty glosu sluzebnicy twojej; a poloze przed cie sztuczke chleba, abys jadl, i posilil sie, abys mógl isc w droge.
\par 23 Ale nie chcial, i mówil: Nie bede jadl. I przymusili go sludzy jego, takze i niewiasta; i usluchal glosu ich, a wstawszy z ziemi, usiadl na lózku.
\par 24 A ona niewiasta miala karmne ciele w domu, a pospieszywszy sie, zabila je; potem wziawszy maki zaczynila, i napiekla z niej przasników.
\par 25 I przyniosla przed Saula, i przed slugi jego, którzy najadlszy sie, wstali, i poszli onej nocy.

\chapter{29}

\par 1 Tedy zebrali Filistynowie wszystkie wojska swe do Afeku; a Izraelczycy polozyli sie obozem nad zródlem, które bylo w Jezreelu.
\par 2 A ksiazeta Filistynskie ciagneli stami i tysiacami, a Dawid i mezowie jego ciagneli pozad z Achisem.
\par 3 I rzekly ksiazeta Filistynskie: Cóz tu czynia ci Hebrejczycy? I rzekl Achis do ksiazat Filistynskich: Azaz nie to jest Dawid, sluga Saula, króla Izraelskiego, który byl przy mnie przez te dni, owszem przez te lata? I nie doswiadczylem go w niczem od onego dnia, jako zbiegl do mnie, az do dnia tego?
\par 4 I rozgniewaly sie nan ksiazeta Filistynskie, i rzekly mu ksiazeta Filistynskie: Odpraw tego meza, a niech sie wróci do miejsca swego, na któremes go postawil; niech nie chodzi z nami na wojne, aby sie nam nie stawil nieprzyjacielem w bitwie. Bo jakoz inaczej moze przyjsc do laski pana swego, jedno przez glowy tych mezów?
\par 5 Azaz nie ten jest Dawid, któremu spiewano hufcami, mówiac: Porazil Saul swój tysiac, ale Dawid swoich dziesiec tysiecy?
\par 6 A tak wezwal Achis Dawida, i rzekl mu: Jako zywy Pan, zes ty szczery i dobry jest w oczach moich, a podoba mi sie wyjscie twoje, i wejscie twoje ze mna do obozu, bom nie znalazl w tobie nic zlego ode dnia, któregos przyszedl do mnie, az do dnia te go; tylko w oczach ksiazat nie masz laski.
\par 7 Przetoz teraz wróc sie, a idz w pokoju i nie czyn nic, coby bylo przeciwnego w oczach ksiazat Filistynskich.
\par 8 I rzekl Dawid do Achisa: Cózem wzdy uczynil? a cos znalazl w sludze twym ode dnia, któregom byl przy tobie, az do dnia tego, abym nie szedl i nie walczyl przeciwko nieprzyjaciolom króla, pana mego?
\par 9 A odpowiadajac Achis, rzekl Dawidowi: Wiem, izes ty dobry w oczach moich, jako Aniol Bozy; ale ksiazeta Filistynskie rzekly: Niech nie chodzi z nami na wojne.
\par 10 A przetoz wstan bardzo rano, i sludzy pana twego, którzy z toba przyszli, a wstawszy rano, skoro pocznie switac, odejdzcie.
\par 11 Wstal tedy Dawid, sam i mezowie jego, aby odszedl tem raniej, i nawrócil sie do ziemi Filistynskiej, a Filistynowie ciagneli do Jezreel.

\chapter{30}

\par 1 A gdy sie wrócil Dawid i mezowie jego do Sycelegu, byl dzien trzeci, jako Amalekitowie wtargneli byli na poludnie, i do Sycelegu, a zburzyli Syceleg, i spalili go ogniem.
\par 2 I pobrali w niewole niewiasty, które byly w nim; od najmniejszego az do wielkiego, nie zabili nikogo, ale tylko pojmali, i odeszli droga swa.
\par 3 A gdy przyszedl Dawid i mezowie jego do miasta, oto, spalone bylo ogniem, a zony ich, i syny ich, i córki ich w niewole zabrano.
\par 4 Tedy podniósl Dawid, i lud, który byl z nim, glos swój, i plakali, az im sily do placzu nie stalo.
\par 5 Obiedwie tez zony Dawidowe byly wziete w niewola: Achinoam Jezreelitka, i Abigail, przedtem zona Nabalowa z Karmelu.
\par 6 I byl utrapiony Dawid bardzo; bo sie zmawial lud ukamionowac go, gdyz gorzkosci pelna byla dusza wszystkiego ludu, kazdego dla synów swych, i dla córek swych; wszakze Dawid zmocnil sie w Panu, Bogu swoim.
\par 7 Tedy rzekl Dawid do Abijatara kaplana, syna Achimelechowego: Wezmij prosze dla mnie efod; i wzial Abijatar efod dla Dawida.
\par 8 A tak sie Dawid radzil Pana, mówiac; Mamli gonic to wojsko, i dogonieli go? A Pan mu rzekl: Gon; bo zapewne ich dogonisz, i zapewne odbijesz plon.
\par 9 Szedl tedy Dawid, sam i one szesc set mezów, które mial z soba, a przyszli az do potoku Besor; a niektórzy pozostali.
\par 10 I gonil je Dawid, sam i cztery sta mezów; bo pozostalo bylo dwiescie mezów spracowanych, a nie poszli za potok Besor.
\par 11 I znalezli meza Egipczanina na polu, a przywiedli go do Dawida, i dali mu chleba, i jadl; dali mu tez wody, i pil;
\par 12 Dali mu takze i wiazanke fig i dwie gronie rodzynków. Jadl tedy, i wrócil sie duch jego wen; bo nie jadl chleba, ani pil wody przez trzy dni i przez trzy nocy.
\par 13 I rzekl do niego Dawid: Czyjes ty? a skades? Który odpowiedzial: jestem rodem z Egiptu, sluga meza Amalekity, i zostawil mie pan mój, zem sie rozniemógl dzis trzeci dzien.
\par 14 Wtargnelismy byli na poludnie do Cerety i do Juda, i na poludnie do Kaleb, i spalilismy Syceleg ogniem.
\par 15 I rzekl mu Dawid: Móglzebys mie dowiesc do tego wojska? Który rzekl: Przysiez mi przez Boga, iz mie nie zabijesz, ani mie wydasz w reke pana mego; tedy cie nawiode na to wojsko.
\par 16 Nawiódl go tedy; a oto, oni lezeli po wszystkiej onej ziemi, jedzac i pijac, i weselac sie ze wszystkich korzysci wielkich, które byli zabrali z ziemi Filistynskiej, i z ziemi Judzkiej.
\par 17 Przetoz bil je Dawid od wieczora az do wieczora dnia drugiego, tak iz z nich zaden nie uszedl, oprócz czterech set mlodzienców, którzy wsiadlszy na wielblady, uciekli.
\par 18 A tak odjal Dawid wszystko, co byli pobrali Amalekitowie, i dwie zony swoje odjal tez Dawid;
\par 19 Tak iz im nic nie zginelo od mala az do wiela, i az do synów, i córek, i do korzysci, i az do wszystkiego, cokolwiek im zabrali, wszystko zasie przywiódl Dawid.
\par 20 Przytem zabral Dawid wszystkie trzody i stada, które gnano przed bydlem jego, i mówiono: Toc jest korzysc Dawidowa.
\par 21 I przyszedl Dawid do onych dwóch set mezów, którzy byli spracowani, ze nie mogli isc za Dawidem, którym byl kazal zostac u potoku Besor, którzy wyszli przeciw Dawidowi i przeciw ludowi, który z nim byl; a przystapiwszy Dawid do ludu, pozdrowil je spokojnie.
\par 22 A odpowiadajac wszyscy mezowie zli i niepobozni, którzy chodzili z Dawidem, rzekli: Poniewaz ci nie chodzili z nami, nie damy im z lupów, któresmy odjeli, tylko kazdemu zone jego, i syny jego; te wziawszy, niech odejda.
\par 23 Tedy rzekl Dawid: Nie uczynicie tak, bracia moi, z tem, co nam dal Pan, który nas strzegl, a podal wojsko, które bylo wyszlo przeciwko nam, w rece nasze.
\par 24 I któz was w tem uslucha? Bo jaki dzial tego, który wyszedl na wojne, taki dzial i tego, który zostal przy tlomokach; równo sie podziela.
\par 25 I stalo sie od onego dnia i napotem, ze uchwalono to prawo i ten zwyczaj w Izraelu, az do dnia tego.
\par 26 A tak przyszedl Dawid do Sycelega, i poslal z onego lupu starszym w Juda, przyjaciolom swym, mówiac: Oto macie blogoslawienstwo z korzysci nieprzyjaciól Panskich:
\par 27 Tym, co byli w Betel, i co w Ramacie na poludnie, i co byli w Gieter,
\par 28 I co byli w Aroer, i co byli w Sefamot, i co byli w Estamo;
\par 29 I co byli w Racha, i co byli w miastach Jerameel, i co byli w miastach Ceni,
\par 30 I co byli w Horma, i co byli w Chorasan, i co byli w Atach,
\par 31 I co byli w Hebronie, i co byli na wszystkich miejscach, kedy przemieszkiwal Dawid, sam i mezowie jego.

\chapter{31}

\par 1 A Filistynowie zwiedli bitwe z Izraelem; i uciekli mezowie Izraelscy przed Filistynami, a polegli zranieni na górze Gielboe.
\par 2 I gonili Filistynowie Saula i syny jego, i zabili Filistynowie Jonatana, i Abinadaba, i Melchisuego, syny Saulowe.
\par 3 A gdy sie wzmagala bitwa przeciwko Saulowi, trafili nan strzelcy, mezowie strzelajacy z luku, i zraniony jest bardzo od strzelców.
\par 4 I rzekl Saul do wyrostka swego, który nosil bron jego: Dobadz miecza twego, a przebij mie nim, by snac nie przyszli ci nieobrzezancy, i nie przebili mie, a nie czynili igrzyska ze mnie. Ale nie chcial wyrostek jego, bo sie bardzo bal. Przetoz Saul porwal miecz i upadl nan.
\par 5 A widzac wyrostek jego, iz umarl Saul, padl i on na miecz swój, i umarl z nim.
\par 6 Umarl tedy Saul, i trzej synowie jego, i wyrostek jego, co za nim bron nosil, i wszyscy mezowie jego dnia onego wespól.
\par 7 Co gdy ujrzeli mezowie Izraelscy, którzy za dolina, i za Jordanem mieszkali, iz uciekali mezowie Izraelscy, a iz umarl Saul, i synowie jego, odbiezawszy miast, pouciekali tez, a przyszedlszy Filistynowie mieszkali w nich.
\par 8 A nazajutrz przyszli Filistynowie, aby odzierali pobite; i znalezli Saula, i trzech synów jego lezacych na górze Gielboe.
\par 9 A uciawszy glowe jego, zdarli z niego zbroje jego, i poslali po ziemi Filistynskiej wszedzie, aby to opowiadano w kosciele balwanów ich, i miedzy ludem.
\par 10 I polozyli zbroje jego w kosciele Astarot; ale cialo jego przybili na murze Betsan.
\par 11 Tedy uslyszawszy o tem obywatele Jabes Galaad, co uczynili Filistynowie Saulowi;
\par 12 Wstali wszyscy mezowie mocni, i szli przez one cala noc, i wzieli cialo Saulowe, i ciala synów jego z muru Betsan, a przyszedlszy do Jabes spalili je tam.
\par 13 A wziawszy kosci ich, pogrzebli je pod drzewem w Jabes, i poscili siedm dni.


\end{document}