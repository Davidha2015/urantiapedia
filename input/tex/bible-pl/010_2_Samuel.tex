\begin{document}

\title{2 Samuela}


\chapter{1}

\par 1 I stalo sie po smierci Saulowej, gdy sie Dawid wrócil od porazki Amalekitów, ze zamieszkal Dawid w Sycelegu przez dwa dni.
\par 2 Tedy dnia trzeciego przybiezal maz z obozu Saulowego, a szaty jego rozdarte byly, i proch na glowie jego; który przyszedlszy do Dawida, upadl na ziemie, i poklonil sie.
\par 3 I rzekl do niego Dawid: Skad idziesz? I odpowiedzial mu: Z obozum Izraelskiego uszedl.
\par 4 Rzekl mu znowu Dawid: Cóz sie stalo? prosze powiedz mi. I odpowiedzial: To, ze uciekl lud z bitwy, a do tego wiele poleglo z ludu i pomarlo, takze i Saul, i Jonatan, syn jego, polegli.
\par 5 Zatem rzekl Dawid do mlodzienca, który mu to powiedzial: Jakoz wiesz, iz umarl Saul i Jonatan, syn jego?
\par 6 Odpowiedzial mu mlodzieniec, który mu to oznajmil: Przyszedlem z trafunku na góre Gilboe, a oto, Saul tkwial na wlóczni swojej, a wozy i jezdni doganiali go.
\par 7 Tedy obejrzawszy sie, obaczyl mie, i zawolal na mie, i rzeklem: Owom ja.
\par 8 I rzekl mi: Cos ty zacz? A jam mu odpowiedzial: Jestem Amalekita.
\par 9 I rzekl mi: Stan, prosze, nademna, a zabij mie: bo mie zdjely ciezkosci, gdyz jeszcze wszystka dusza moja we mnie jest.
\par 10 Przetoz stanawszy nad nim, zabilem go: bom wiedziel, ze nie bedzie zyw po upadku swoim, i wzialem korone, która byla na glowie jego, i zawieszenie, które bylo na ramieniu jego, a przynioslem je tu do pana mego.
\par 11 Tedy Dawid pochwyciwszy szaty swoje, rozdarl je, takze i wszyscy mezowie, którzy z nim byli.
\par 12 A zalujac plakali, i poscili az do wieczora dla Saula, i dla Jonatana, syna jego, i dla ludu Panskiego, i dla domu Izraelskiego, ze polegli od miecza.
\par 13 I rzekl Dawid do mlodzienca, który mu to oznajmil: Skades ty? I odpowiedzial: Jestem synem meza przychodnia Amalekity.
\par 14 Zatem rzekl do niego Dawid: Jakozes sie nie bal sciagnac reki twej, abys zabil pomazanca Panskiego?
\par 15 Zawolal tedy Dawid jednego z slug, i rzekl: Przystap, a zabij go, a on go uderzyl, ze umarl.
\par 16 I rzekl do niego Dawid: Krew twoja na glowe tweje: bo usta twoje swiadczyly przeciw tobie, mówiac: Jam zabil pomazanca Panskiego.
\par 17 Lamentowal tedy Dawid lamentem takowym nad Saulem, i nad Jonatanem, synem jego;
\par 18 (Rozkazawszy jednak, aby uczono synów Judzkich z luku strzelac, jako napisano w ksiegach Jasar.)
\par 19 O ozdobo Izraelska! na górach twoich zranionys jest; jakoz polegli mocarze twoi!
\par 20 Nie powiadajciez w Get, a nie rozglaszajcie po ulicach w Aszkalonie, aby sie snac nie weselily córki Filistynskie, by sie snac nie radowaly córki nieobrzezanców.
\par 21 O góry Gielboe! ani rosa, ani deszcz niech nie upada na was, i niech tam nie beda pola urodzajne; albowiem tam porzucona jest tarcz mocarzów, tarcz Saulowa, jakoby nie byl pomazany olejem.
\par 22 Od krwi zabitych, i od sadla mocarzów strzala luku Jonatanowego nie wracala sie na wstecz, a miecz Saulowy nie wracal sie prózno.
\par 23 Saul i Jonatan milosni i przyjemni za zywota swego, i w smierci swojej nie sa rozlaczeni, nad orly lekciejsi, nad lwy mocniejsi byli.
\par 24 Córki Izraelskie placzcie nad Saulem, który was przyodziewal szarlatem rozkosznym, a który was ubieral w klejnoty zlote na szaty wasze.
\par 25 Jakoz polegli mocarze posród bitwy! Jonatan na górach twoich zabity jest.
\par 26 Bardzo mi cie zal, bracie mój Jonatanie, byles mi bardzo wdziecznym; wieksza u mnie byla milosc twoja, niz milosc niewiescia.
\par 27 Jakoz polegli mocarze, a poginela bron wojenna!

\chapter{2}

\par 1 I stalo sie potem, ze pytal Dawid Pana, mówiac: Mamze isc do któregokolwiek miasta Judzkiego? Któremu Pan odpowiedzial: Idz. I rzekl Dawid: Dokadze pójde? I odpowiedzial: Do Hebronu.
\par 2 Tedy tam jechal Dawid, takze i dwie zony jego, Achinoam Jezreelitka, i Abigail zona przedtem Nabalowa z Karmelu.
\par 3 Takze meze swe, którzy z nim byli, zabral Dawid, kazdego z domem jego, i mieszkali w miastach Hebronskich.
\par 4 Przyszli potem mezowie Juda, i pomazali tam Dawida za króla nad domem Juda; tedy opowiadano Dawidowi, mówiac: Mezowie z Jabes Galaad ci pogrzebli Saula.
\par 5 Tedy wyprawil Dawid posly do mezów z Jabes Galaad, i rzekl do nich: Blogoslawieniscie wy od Pana, którzyscie ucznili to milosierdzie nad Panem waszym Saulem, zescie go pogrzebli.
\par 6 Przetoz teraz niech uczyni Pan z wami milosierdzie, i prawde, a ja tez oddam wam to dobrodziejstwo, zescie uczynili te rzecz.
\par 7 Teraz tedy niech sie zmacniaja rece wasze, a badzcie meznymi; bo choc umarl Saul, pan wasz, wszakze mnie pomazal dom Juda za króla nad soba.
\par 8 A Abner, syn Nera, hetman nad wojskiem Saulowem, wzial Izboseta syna Saulowego, i przyprowadzil go do Machanaim,
\par 9 I uczynil go królem nad Gaaladem, i nad Assury, i nad Jezreelem, i nad Efraimem, i nad Benjaminem, i nad wszystkim Izraelem.
\par 10 Czterdziesci lat mial Izboset, syn Saula, gdy poczal królowac nad Izraelem, a dwa lata królowal; tylko dom Juda stal przy Dawidzie.
\par 11 I byla liczba dni, których byl Dawid królem w Hebronie nad domem Judzkim, siedm lat i szesc miesiecy.
\par 12 Potem wyszedl Abner, syn Nera, i sludzy Izboseta, syna Saulowego, z Machanaim do Gabaonu.
\par 13 Joab takze, syn Sarwii, z slugami Dawidowymi wyszli, i spotkali sie z soba prawie u stawu Gabaonskiego, i zostali jedni na jednej stronie stawu, a drudzy na drugiej stronie stawu.
\par 14 Tedy rzekl Abner do Joaba: Niech teraz wstana mlodziency, a poigraja przed nami. I rzekl Joab: Niech wstana.
\par 15 Wstali tedy; i wyszlo w liczbie dwanascie z Benjaminczyków ze strony Izboseta, syna Saulowego, a dwanascie z slug Dawidowych.
\par 16 Którzy uchwyciwszy sie, kazdy za glowe przeciwnika swego, utopil miecz swój w boku jeden drugiego, i polegli pospolu. Przetoz nazwano miejsce ono Helkatassurym, które jest w Gabaonie.
\par 17 I byla bitwa bardzo sroga dnia onego, a porazon jest Abner i mezowie Izraelscy od slug Dawidowych.
\par 18 I byli tez tam trzej synowie Sarwii: Joab, Abisaj, i Asael; ale Asael byl predkich nóg, jako dzika koza.
\par 19 I gonil Asael Abnera, a nie ustapil idac ani na prawo ani na lewo, scigajac Abnera.
\par 20 A obejrzawszy sie Abner nazad, rzekl: Tyzes jest Asael? A on mu odpowiedzial: Ja.
\par 21 Tedy mu rzekl Abner: Uchyl sie na prawa strone twoje, albo na lewa strone twoje, a pojmij sobie jednego z mlodzienców, i wezmij sobie lupy z niego; ale Asael nie chcial od niego ustapic.
\par 22 Tedy po wtóre rzekl Abner do Asaela: Idz precz ode mnie, bym cie snac nie przebil ku ziemi; bo jakoze bym smial podniesc twarz moje na Joaba, brata twego?
\par 23 A gdy nie chcial ustapic, uderzyl go Abner koncem wlóczni pod piate zebro, tak iz wyszla wlócznia na druga strone. Tamze padl, i umarl na onemze miejscu, a wszyscy, którzy przychodzili do onego miejsca, gdzie polegl Asael i umarl, zastanawiali sie.
\par 24 Wszakze gonili Joab i Abisaj Abnera; i zachodzilo slonce, gdy przyszli do pagórka Amma, który jest przeciw Gia na drodze pustyni Gabaonskiej.
\par 25 Tedy sie zebrali synowie Benjaminowi do Abnera, skupiwszy sie w jeden huf, i staneli na wierzchu jednego pagórka.
\par 26 I zawolal Abner na Joaba i rzekl: Izali sie na wieki bedzie srozyl ten miecz? azaz nie wiesz, ze na ostatku bywa gorzkosc? i dokadze nie rzeczesz ludowi, aby sie wrócil od pogoni braci swych?
\par 27 Tedy rzekl Joab: Jako zywy Bóg, bys ty byl nie wyzywal, zarazby sie byl z poranku lud wrócil, kazdy od pogoni braci swych.
\par 28 A tak zatrabil Joab w trabe i zastanowil sie wszystek lud, a nie gonili dalej Izraela, ani sie wiecej potykali.
\par 29 Ale Abner i mezowie jego szli polem cala one noc, a przeprawiwszy sie przez Jordan, przeszli przez wszystko Betoron, az przyszli do Mahanaim.
\par 30 A Joab wróciwszy sie z pogoni za Abnerem, zebral wszystek lud, i nie dostawalo mu slug Dawidowych dziewietnastu mezów, i Asaela.
\par 31 Ale sludzy Dawidowi pobili z Benjaminczyków, i z mezów Abnerowych trzy sta i szescdziesiat mezów, którzy tam pomarli.
\par 32 A wziawszy Asaela, pogrzebli go w grobie ojca jego, który byl w Betlehem; potem szli cala noc Joab i mezowie jego, a na switaniu przyszli do Hebronu.

\chapter{3}

\par 1 I byla dluga wojna miedzy domem Saulowym i miedzy domem Dawidowym. Wszakze Dawid postepowal, i zmacnial sie; ale dom Saulów schodzil i niszczal.
\par 2 I narodzilo sie Dawidowi w Hebronie synów. A byl pierworodny jego Amnon z Achinoamy Jezreelitki;
\par 3 Wtóry po nim byl Helijab z Abigaili, zony przedtem Nabalowej z Karmelu, a trzeci Absalom, syn Moachy, córki Tolmaja króla Giessur;
\par 4 A czwarty Adonijasz, syn Hagity, a piaty Sefatyjasz, syn Abitali;
\par 5 A szósty Jetraam z Egli, zony Dawidowej. Cic sie urodzili Dawidowi w Hebronie.
\par 6 I stalo sie, gdy byla wojna miedzy domem Saulowym i miedzy domem Dawidowym, a Abner sie meznie zastawial o dom Saulowy.
\par 7 (A Saul mial zaloznice, której imie bylo Resfa, córka Aje,)ze rzekl Izboset do Abnera: Czemus wszedl do zaloznicy ojca mojego?
\par 8 I rozgniewal sie Abner bardzo o one slowa Izbosetowe, i rzekl: Izalim ja psia glowa, którym przeciw Judzie dzis uczynil milosierdzie nad domem Saula, ojca twego, i nad bracia jego, i nad przyjaciólmi jego, i nie wydalem cie w reke Dawidowe, a ty d zis szukasz na mnie nieprawosci tej niewiasty?
\par 9 To niech uczyni Bóg Abnerowi, i to niech mu przyczyni, jezlize, jako przysiagl Pan Dawidowi, nie pomoge do tego.
\par 10 Aby przeniesione bylo królestwo od domu Saulowego, a wystawiona stolica Dawidowa nad Izraelem, i nad Juda od Dan az do Beerseba.
\par 11 I nie mógl nic wiecej odpowiedziec Abnerowi, przeto ze sie go bal.
\par 12 A tak wyprawil Abner posly do Dawida od siebie, mówiac: Czyjaz jest ziemia? i zeby mówili: Uczyn przymierze twoje ze mna, a oto reka moja bedzie z toba, aby obrócon byl do ciebie wszystek Izrael.
\par 13 Któremu odpowiedzial: Dobrze, uczynie z toba przymierze. A wszakze o jedno cie prosze, mianowicie, abys nie przychodzil przed oblicze moje, az mi pierwej przywiedziesz Michol, córke Saulowe, gdy bedziesz chcial przyjsc, abys widzial twarz moje.
\par 14 I wyprawil Dawid posly do Izboseta syn Saulowego, mówiac: Wydaj mi zone moje Michol, któram sobie poslubil stem nieobrzezek Filistynskich.
\par 15 Przetoz poslal Izboset, i wzial ja od meza, od Faltejela, syna Laisowego.
\par 16 Tedy szedl z nia maz jej, a idac za nia, plakal jej az do Bachurym; i rzekl do niego Abner: Idz, a wróc sie; i wrócil sie.
\par 17 Uczynil potem Abner rzecz do starszych Izraelskich mówiac: Przeszlych czasów szukaliscie Dawida, aby byl królem nad wami.
\par 18 Przetoz teraz uczyncie tak; bo Pan rzekl o Dawidzie, mówiac: Przez reke Dawida, slugi mego, wybawie lud mój Izraelski z reki Filistynskiej, i z reki wszystkich nieprzyjaciól jego.
\par 19 To tez mówil Abner i do Benjaminczyków. Potem odszedl Abner, aby mówil z Dawidem w Hebronie wszystko, co dobrego bylo w oczach Izraela, i w oczach wszystkiego domu Benjaminowego.
\par 20 Gdy tedy przeszedl Abner do Dawida do Hebronu, a z nim dwadziescia mezów, sprawil Dawid na Abnera, i na meze, którzy z nim byli, uczte.
\par 21 I rzekl Abner do Dawida: Wstane, a pójde, abym zebral do króla, pana mego, wszystkiego Izraela, którzy z toba uczynia przymierze; a bedziesz królowal nad wszystkimi, jako zada dusza twoja. A tak odprawil Dawid Abnera, który odszedl w pokoju.
\par 22 A oto, sludzy Dawidowi i Joab wracali sie z wojny, korzysci wielkie z soba prowadzac, ale Abnera juz nie bylo u Dawida w Hebronie: bo go byl odprawil, i odszedl byl w pokoju.
\par 23 Joab tedy i wszystko wojsko, które z nim bylo, przyszli tam; i dano znac Joabowi, mówiac: Byl tu Abner, syn Nera, u króla; ale go odprawil, i odszedl w pokoju.
\par 24 Przetoz wszedlszy Joab do króla, rzekl: Cózes uczynil? Oto, przyszedl byl Abner do ciebie; przeczzes go puscil, aby zas odszedl.
\par 25 Znasz Abnera, syna Nerowego, gdyz przyszedl, aby cie zdradzil, i zeby wiedzial wyjscie twoje, i wejscie twoje, aby sie wywiedzial o wszystkiem, co ty czynisz.
\par 26 Tedy odszedlszy Joab od Dawida, wyprawil posly za Abnerem, którzy go wrócili od studni Syra, o czem Dawid nie wiedzial.
\par 27 A gdy sie wrócil Abner do Hebronu, odwiódl go Joab w posród bramy, aby z nim po cichu (osobno) mówil, i przebil go tam pod piate zebro, ze umarl dla krwi Asaela, brata swego.
\par 28 Co gdy potem uslyszal Dawid, rzekl: Nie jestem winien, ani królestwo moje, przed Panem az na wieki krwi Abnera, syna Nerowego.
\par 29 Niechaj przyjdzie na glowe Joabowe, i na wszystek dom ojca jego, i niech nie ustaje z domu Joabowego plynienie nasienia cierpiacy, i tredowaty, i o kiju chodzacy, i od miecza upadajacy, i nie majacy chleba.
\par 30 A tak Joab i Abisaj, brat jego, zabili Abnera, przeto iz on tez byl zabil Asaela, brata ich, w bitwie u Gabaona.
\par 31 Rzekl potem Dawid do Joaba i do wszystkiego ludu, który byl z nim: Porozdzierajcie odzienia wasze a opaszcie sie w wory, i placzcie nad Abnerem. A król Dawid szedl za marami.
\par 32 A gdy pogrzebli Abnera w Hebronie, podniósl król glos swój, i plakal nad grobem Abnerowym; plakal tez wszystek lud.
\par 33 A tak lamentujac król nad Abnerem, rzekl: Izali tak mial umrzec Abner, jako umiera nikczemnik?
\par 34 Rece twoje nie byly zwiazane, a nogi twoje nie byly petami obciazone; polegles jako ten, który pada przed synami niezboznymi. Tedy tem wiecej wszystek lud plakal nad nim.
\par 35 Potem przyszedl wszystek lud prosic Dawida, aby jadl chleb, gdy jeszcze byl jasny dzien: ale przysiagl Dawid, mówiac: To mi niech uczyni Bóg, i to niech przyczyni, jezli przed zajsciem slonca skosztuje chleba, albo czego innego.
\par 36 Co gdy wszystek lud obaczyl, podobalo sie im to; a wszystko cokolwiek czynil król, podobalo sie w oczach wszystkiego ludu.
\par 37 I poznal wszystek Izrael dnia onego, ze nie z naprawy królewskiej zabity byl Abner, syn Nera.
\par 38 I rzekl król do slug swoich: Azaz nie wiecie, ze hetman, a bardzo wielki, polegl dzis w Izraelu?
\par 39 A ja dzis jako nowy, i dopiero pomazany król; ci zasie mezowie, synowie Sarwii, srozsi sa nizli ja; niechze odda Pan czyniacemu zle wedlug zlosci jego.

\chapter{4}

\par 1 A uslyszawszy Izboset, syn Saula, ze polegl Abner w Hebronie, zemdlaly rece jego, i wszystek Izrael byl przestraszony.
\par 2 Mial tez syn Saula dwóch mezów hetmanów nad hufcami, imie jednego Baana, a imie drugiego Rechab, synowie Remmona Berotczyka z synów Benjaminowych; bo tez Berot policzon byl w Benjaminie.
\par 3 Uciekli tedy Berotczykowie do Gietaim, i byli tam przychodniami az do onego dnia.
\par 4 A Jonatan, syn Saula, mial jednego syna chromego na nogi, (bo gdy mial piec lat, a przyszla wiesc o smierci Saulowej, i Jonatanowej z Jezreel, a wziawszy go mamka jego uciekla, a gdy predko uciekala, upadl i ochromial,)a imie jego Mefiboset.
\par 5 Poszli tedy synowie Remmona Berotczyka, Rechab i Baana, i weszli, gdy byl najgoretszy dzien, do domu Izboseta, który spal na lózku w poludnie.
\par 6 Ci tedy weszli w dom jego, jakoby kupowac zboza; tamze go przebili pod piate zebro Rechab i Baana, brat jego, i uciekli.
\par 7 Bo gdy byli weszli w dom, a on spal na lózku swem w pokoju, kedy legal, tedy go przebili, i zabili go, a uciawszy glowe jego, wzieli ja, i poszli droga puszczy przez cala one noc.
\par 8 I przyniesli glowe Izbosetowe do Dawida do Hebronu, i rzekli do króla; Oto, glowa Izboseta, syna Saulowego, nieprzyjaciela twego, który szukal duszy twojej; a dal Pan królowi, panu memu, pomste dzisiaj nad Saulem i nad nasieniem jego.
\par 9 Tedy odpowiedzial Dawid Rechabowi i Baanie, bratu jego, synom Remmona Berotczyka, i rzekl do nich: Jako zyw Pan, który wybawil dusze moje od wszelkiego ucisku:
\par 10 Jezlim onego, który mi oznajmil, mówiac: Oto umarl Saul, (choc mu sie zdalo, ze wesola nowine przyniósl,)pojmawszy zabil w Syclegu, który rozumial, zem mu mial dac zaplate za poselstwo jego:
\par 11 Jako daleko wiecej ludzie niepobozne, gdyz zabili meza sprawiedliwego w domu jego, na lozu jego? A teraz, izali nie mam szukac krwi jego z reki waszej, i wygladzic was z ziemi?
\par 12 A tak rozkazal Dawid slugom, i zabili je, a obciazywszy rece ich, i nogi ich, zawiesili je nad stawem w Hebronie; ale glowe Izbosetowa wziawszy pogrzebali w grobie Abnerowym w Hebronie.

\chapter{5}

\par 1 Zeszly sie tedy wszystkie pokolenia Izraelskie do Dawida do Hebronu, i rzekly, mówiac: Oto, my jestesmy kosc twoja i cialo twoje.
\par 2 A przeszlych czasów, gdy Saul byl królem nad nami, tys wywodzil i przywodzil Izraela. Nad to rzekl Pan do ciebie: Ty bedziesz pasl lud mój Izraelski, a ty bedziesz wodzem nad Izraelem.
\par 3 A tak wszyscy starsi Izraelscy przyszli do króla do Hebronu; i uczynil z nimi król Dawid przymierze w Hebronie przed Panem; i pomazali Dawida za króla nad Izraelem.
\par 4 Trzydziesci lat bylo Dawidowi, gdy poczal królowac, a królowal przez czterdziesci lat.
\par 5 W Hebronie królowal nad Juda przez siedm lat i przez szesc miesiecy, a w Jeruzalemie królowal przez trzydziesci i trzy lata nad wszystkim Izraelem i nad Juda.
\par 6 A tak poszedl król i mezowie jego do Jeruzalemu przeciw Jebuzejczykowi mieszkajacmu w onej ziemi, który rzekl do Dawida, mówiac; Nie wnijdziesz sam, az zniesiesz slepe i chrome, jakoby mówili: Nie wnijdzie tu Dawid.
\par 7 Wszakze wzial Dawid zamek Syonski, a toc jest miasto Dawidowe.
\par 8 Bo rzekl byl Dawid onego dnia: Ktobykolwiek zabil Jebuzejczyka, a wszedlby na rynny, a pobil te slepe i chrome, które ma w nienawisci dusza Dawidowa, postanowie go hetmanem. Dla tegoz mawiano: Slepy i chromy nie wnijdzie do tego domu.
\par 9 I mieszkal Dawid na onym zamku a przezwal go miastem Dawidowem, a pobudowal je Dawid wszedy w kolo od Mello, i wewnatrz.
\par 10 A Dawid idac postepowal i rósl; bo Pan Bóg zastepów byl z nim.
\par 11 Tedy poslal Hyram, król Tyrski, posly do Dawida, i drzewa cedrowe, i ciesle, i kamienniki, i murarze, którzy zbudowali dom Dawidowi.
\par 12 I poznal Dawid, iz go utwierdzil Pan za króla nad Izraelem, a iz wywyzszyl królestwo jego dla ludu swego Izraelskiego.
\par 13 I napojmowal sobie Dawid jeszcze wiecej zaloznic i zon z Jeruzalemu przyszedlszy z Hebronu, a narodzilo sie wiecej Dawidowi synów i córek.
\par 14 A tec sa imiona tych, którzy mu sie urodzili w Jeruzalemie: Samma, i Sobab, i Natan, i Salomon.
\par 15 I Ibchar, i Elisua, i Nefeg, i Jafija.
\par 16 I Elisama, i Elijada, i Elifelet.
\par 17 A uslyszawszy Filistynowie, ze pomazano Dawida za króla nad Izraelem, ruszyli sie wszyscy Filistynowie, zeby szukali Dawida; co gdy uslyszal Dawid, ustapil na zamek.
\par 18 Tedy Filistynowie przyciagnawszy, rozpostarli sie w dolinie Refaim.
\par 19 I radzil sie Dawid Pana, mówiac: Mamli isc przeciwko Filistynom? podaszli je w rece moje? Odpowiedzial Pan Dawidowi: Idz, bo pewnie podam Filistyny w rece twoje.
\par 20 A tak przyciagnal Dawid do Baal Perazym, i porazil je tam Dawid i rzekl: Rozerwal Pan nieprzyjacioly moje przedemna, jako sie rozrywaja wody. Przetoz nazwal imie miejsca onego Baal Perazym.
\par 21 I zostawili tam ryte swe obrazy, które popalil Dawid i mezowie jego.
\par 22 Znowu jeszcze przyciagneli Filistynowie, i rozpostarli sie w dolinie Refaim.
\par 23 I pytal sie Dawid Pana, który odpowiedzial: Nie pójdziesz przeciwko nim; ale je obtoczywszy z tylu natrzesz na nie przeciwko morwom.
\par 24 A gdy uslyszysz, iz zaszumia wierzchy morwów, tedy sie ruszysz, gdyz na ten czas pójdzie Pan przed toba, aby porazil wojska Filistynskie.
\par 25 I uczynil Dawid tak jako mu rozkazal Pan, a porazil Filistyny od Gabaa, az kedy chodza do Gazer.

\chapter{6}

\par 1 Nadto zebral jeszcze Dawid wszystkich przebranych z Izraela trzydziesci tysiecy.
\par 2 A ruszywszy sie, szedl Dawid i wszystek lud, który byl przy nim, z Baala Judowego, aby przeniesli stamtad skrzynie Boza, przy której wzywano imienia Pana zastepów, siedzacego na Cherubinach, którzy sa na niej.
\par 3 I wstawili skrzynie Boza na wóz nowy, wziawszy ja z domu Abinadabowego, który jest w Gabaa; lecz Oza i Achyjo, synowie Abinadabowi, prowadzili on wóz nowy.
\par 4 I wzieli ja z domu Abinadaba, który byl w Gabaa, a szli z skrzynia Boza; lecz Achyjo szedl przed skrzynia.
\par 5 Dawid zasie i wszystek Izrael grali przed Panem na wszelakich instrumentach z cedrowego drzewa, na harfach i na skrzypcach, i na bebnach, i na piszczalkach, i na cymbalach.
\par 6 A gdy przyszli do gumna Nachonowego, sciagnal Oza reke swoje ku skrzyni Bozej, i zadzierzal ja: bo woly byly wystapily z drogi:
\par 7 Przetoz rozgniewal sie bardzo Pan na Oze, i zabil go tam Bóg dla smialosci, i tamze umarl przy skrzyni Bozej.
\par 8 I zafrasowal sie Dawid, ze Pan srodze zarazil Oze, i nazwal miejsce ono Perezoza, az do dnia tego.
\par 9 A tak ulakl sie Dawid Pana dnia onego, i mówil: Jakoz wnijdzie do mnie skrzynia Panska?
\par 10 Przetoz nie chcial Dawid wprowadzic do siebie skrzyni Panskiej do miasta swego, ale ja kazal wprowadzic do domu Obededoma Gietejczyka.
\par 11 I zostala skrzynia Panska w domu Obededoma Gietejczyka przez trzy miesiace, i blogoslawil Pan Obededomowi, i wszystkiemu domowi jego.
\par 12 I oznajmiono królowi Dawidowi, mówiac: Blogoslawi Pan domowi Obededoma, i wszystkiemu, co ma, dla skrzyni Bozej. Tedy szedlszy Dawid, wzial skrzynie Boza z domu Obededoma do miasta Dawidowego z weselem.
\par 13 A gdy ci, którzy niesli skrzynie Panska, postapili na szesc kroków, ofiarowal wolu i barana tlustego.
\par 14 I skakal Dawid ze wszystkiej mocy przed Panem, a byl Dawid obleczony w efod lniany.
\par 15 A tak Dawid, i wszystek dom Izraelski prowadzili skrzynie Panska z weselem, i z trabieniem.
\par 16 I stalo sie, gdy skrzynia Panska wchodzila do miasta Dawidowego, ze Michol, córka Saulowa, wygladajac oknem, a widzac króla Dawida ze wszystkiej mocy skaczacego przed Panem, wzgardzila go w sercu swojem.
\par 17 A gdy przyniesli skrzynie Panska, postawili ja na miejscu swem posrodku namiotu, który byl dla niej rozbil Dawid, i ofiarowal Dawid przed Panem calopalenia i spokojne ofiary.
\par 18 A gdy dokonczyl Dawid ofiarowac calopalenia, i ofiar spokojnych, blogoslawil ludowi w imie Pana zastepów.
\par 19 I dal miedzy wszystek lud, i miedzy wszystko zgromadzenie Izraelskie, od meza az do niewiasty, kazdemu bochenek chleba jeden, i jedne sztuke miesa, i lagiew jedne wina. I odszedl wszystek lud, kazdy do domu swego.
\par 20 Potem wrócil sie Dawid, aby blogoslawil domowi swemu. I wyszla Michol, córka Saulowa, przeciwko Dawidowi, a rzekla: O jakoz chwalebny byl dzis król Izraelski, który sie odkrywal dzis przed oczyma sluzebnic slug swoich, jako sie zwykl odkrywac jeden z szalonych!
\par 21 Tedy rzekl Dawid do Michol: Przed Panem (który mnie raczej obral niz ojca twego, i nizeli wszystek dom jego, rozkazujac mi, abym byl ksiazeciem nad ludem Panskim, nad Izraelem.) gralem, i bede gral przed Panem.
\par 22 A im bede podlejszym, nizelim sie stal i unizenszym w oczach moich, tem u tych sluzebnic, o których mi powiadasz, bede chwalebniejszy.
\par 23 Przetoz Michol, córka Saulowa, niemiala dziatek az do dnia smierci swojej.

\chapter{7}

\par 1 I stalo sie, gdy siedzial król w domu swym, a Pan mu dal odpocznienie zewszad od wszystkich nieprzyjaciól jego.
\par 2 Ze rzekl król do Natana proroka: Obacz prosze, ja mieszkam w domu cedrowym, a skrzynia Boza mieszka miedzy kortynami.
\par 3 Tedy rzekl Natan do króla: Cokolwiek jest w sercu twojem, idz, uczyn; bo Pan jest z toba.
\par 4 Potem onejze nocy stalo sie slowo Panskie do Natana, mówiac:
\par 5 Idz a mów do slugi mego Dawida: Tak mówi Pan: Izali mi ty zbudujesz dom ku mieszkaniu?
\par 6 Poniewazem nie mieszkal w domu ode dnia, któregom wywiódl syny Izraelskie z Egiptu, az do dnia tego, alem chodzil w namiocie i w przybytku;
\par 7 I wszedzie, gdziem chodzil ze wszystkimi syny Izraelskimi, izalim i slowo rzekl któremu z sedziów Izraelskich, któremum rozkazal pasc lud mój Izraelski, mówiac: Czemuzescie mi nie zbudowali domu cedrowego?
\par 8 Przetoz teraz to powiedz sludze memu Dawidowi: Tak mówi Pan zastepów: Jam ciebie wzial z owczarni od owiec, abys byl wodzem nad ludem moim, nad Izraelem;
\par 9 I bylem z toba wszedy, gdzieskolwiek chodzil, i wygladzilem wszystkie nieprzyjacioly twoje przed toba, i uczynilem ci imie wielkie, jako imie wielkich ludzi, którzy sa na ziemi.
\par 10 I postanowie miejsce ludowi memu Izraelskiemu, i wszczepie go, iz bedzie mieszkal na miejscu swem, i nie bedzie wiecej poruszony, ani go wiecej synowie nieprawosci trapic beda, jako przedtem.
\par 11 Ode dnia, któregom postanowil sedzie nad ludem moim Izraelskim; i dam ci odpocznienie ode wszystkich nieprzyjaciól twoich. Przetoz opowiadac Pan, ze on sam tobie dom zbuduje.
\par 12 Gdy sie wypelnia dnia twoje, i zasniesz z ojcy twoimi, wzbudze nasienie twoje po tobie, które wynijdzie z zywota twego, a umocnie królestwo jego;
\par 13 On zbuduje dom imieniowi memu, a Ja utwierdze stolice królestwa jego az na wieki.
\par 14 Ja mu bede za ojca, a on mi bedzie za syna, który gdy wystapi, skarze go rózga ludzka, i plagami synów czlowieczych.
\par 15 Lecz milosierdzie moje nie bedzie odjete od niego, jakom je odjal od Saula, któregom odrzucil przed twarza twoja.
\par 16 I bedzie utwierdzony dom twój, i królestwo twoje az na wieki przed toba, a stolica twoja bedzie trwala az na wieki.
\par 17 Wedlug wszystkich slów tych, i wedlug wszystkiego widzenia tego, tak mówi Natan do Dawida.
\par 18 Tedy wszedlszy król Dawid, usiadl przed obliczem Panskiem, i rzekl: Cozem ja jest Panie Boze, i co za dom mój, zes mie przywiódl az dotad?
\par 19 Lecz i to malo bylo przed oblicznoscia twoja, Panie Boze; ales tez obietnice uczynil o domie slugi twego na czas daleki, a to prawie obyczajem ludzkim, Panie Boze!
\par 20 I cóz wiecej ma mówic Dawid przed toba? albowiem ty znasz sluge twego, o Panie Boze.
\par 21 Dla slowa twego, a wedlug serca twego uczyniles te wszystkie wielkie rzeczy, oznajmujac je sludze twemu.
\par 22 Przetoz wielmoznym jestes, Panie Boze; bo nie masz podobnego tobie, i nie masz Boga oprócz ciebie, wedlug wszystkiego, cosmy slyszeli w uszy nasze.
\par 23 I gdziez jest taki lud na ziemi, jako Izrael? dla któregoby Bóg szedl aby go sobie odkupil za lud, i uczynil sobie imie, i sprawil wam wielkie i straszne rzeczy w ziemi twojej, przed obliczem ludu twego, którys sobie wykupil z Egiptu, z poganstw a i z bogów ich;
\par 24 I zmocniles sobie lud twój Izraelski, abyc byl ludem az na wieki; a ty Panie stales sie im za Boga.
\par 25 Przetoz teraz, o Panie Boze, slowo, któres powiedzial o sludze twoim i o domu jego, utwierdz az na wieki, a uczyn tak jakos mówil,
\par 26 Aby uwielbione bylo imie twoje az na wieki, zeby mówiono: Pan zastepów Bogiem nad Izraelem, a dom slugi twego Dawida bedzie umocniony przed twarza twoja.
\par 27 Albowiem ty Panie zastepów, Boze Izraelski, objawiles sludze twemu, mówiac: Dom zbuduje tobie. Przetoz za sluszna rzecz znalazl sluga twój w sercu swojem, aby sie modlil tobie ta modlitwa.
\par 28 Teraz tedy, Panie Boze, tys sam Bóg, a slowa twe sa prawda, i rzekles do slugi twojego te dobre rzeczy.
\par 29 Raczze juz teraz blogoslawic domowi slugi twego, aby trwal na wieki przed toba; bos ty Panie Boze rzekl: Ze blogoslawienstwem twojem bedzie ublogoslawion dom slugi twego na wieki.

\chapter{8}

\par 1 I stalo sie potem, ze porazil Dawid Filistyny, i ponizyl je; a wzial Dawid Meteg Amma z rak Filistynskich.
\par 2 Porazil tez i Moabity, które pomierzyl sznurem, zrównawszy je z ziemia, i wymierzyl ich dwa sznury na zabicie, a caly sznur na zachowanie przy zywocie; i byli Moabitowie slugami Dawidowymi, przynoszac mu podatki.
\par 3 Porazil tez Dawid Hadadezera, syna Rochobowego, króla Soby, gdy wyjechal, aby rozprzestrzenil granice swe nad rzeka Eufrates.
\par 4 I pojmal z nich Dawid tysiac i siedm set jezdnych, a dwadziescie tysiecy ludu pieszego. I poderznal Dawid zyly wszystkim woznikom, zachowawszy koni do sta wozów.
\par 5 Przyciagnal tedy Syryjczyk z Damaszku na pomoc Hadadezerowi, królowi Soby, i porazil Dawid Syryjczyków dwadziescia i dwa tysiace mezów.
\par 6 I osadzil Dawid zolnierzem Syryja Damaska. A tak Syryjczycy byli slugami Dawidowymi, przynoszac mu podatki; i bronil Pan Dawida wszedzie gdziekolwiek sie obrócil.
\par 7 Pobral tez Dawid tarcze zlote, które mieli sludzy Hadadezerowi, i wniósl je do Jeruzalemu.
\par 8 Przytem z Betachu i z Berotu, miast Hadadezerowych, przyniósl król Dawid bardzo wiele miedzi.
\par 9 To uslyszawszy Tohy, król Emat, iz porazil Dawid wszystko wojsko Hadadezerowe,
\par 10 Poslal Tohy Jorama, syna swego, do króla Dawida, aby go pozdrowil w pokoju, i winszowal mu, przeto ze zwalczyl Hadadezera, i porazil go, (albowiem walczyl z Tohym Hadadezer,)i przyniósl z soba naczynia srebrne, i naczynia zlote, i naczynia miedziane.
\par 11 Które tez rzeczy poswiecil król Dawid Panu z innem srebrem i zlotem, które byl poswiecil, pobrawszy od wszystkich narodów, które sobie podbil.
\par 12 Jako od Syryjczyków, i od Moabczyków, i od synów Ammonowych, i od Filistynów, i od Amalekitów, i z lupów Hadadezera, syna Rochobowego, króla Soby.
\par 13 A tak uczynil sobie Dawid imie, gdy sie wrócil poraziwszy Syryjczyki w dolinie solnej, gdzie pobil osmnascie tysiecy ludu.
\par 14 Postanowil tez straz w Edom, wszystke ziemie Edomska osadziwszy zolnierzami; i byli wszyscy Edomczycy slugami Dawidowymi, a bronil Pan Dawida wszedzie, gdzie sie obrócil.
\par 15 I królowal Dawid nad wszystkim Izraelem, i czynil sad i sprawiedliwosc wszystkiemu ludowi swemu.
\par 16 A Joab, syn Sarwii, byl nad wojskiem, a Jozafat syn Ahiluda, kanclerzem.
\par 17 A Sadok, syn Achitoba, i Achimelech, syn Abijatara, byli kaplanami, a Saraja pisarzem.
\par 18 Banajas tez, syn Jojada, nad Cheretczykami i Feletczykami, a synowie Dawidowi byli ksiazetami.

\chapter{9}

\par 1 Tedy rzekl Dawid: Jestze jeszcze kto, coby pozostal z domu Saulowego, abym uczynil nad nim milosierdzie dla Jonatana?
\par 2 I byl z domu Saulowego sluga, którego zwano Syba: tego zawolano do Dawida. I rzekl król do niego: Tyzes jest Syba? A on odpowiedzial: Jam jest, sluga twój.
\par 3 Potem rzekl król: Jestze jeszcze kto z domu Saulowego, abym nad nim uczynil milosierdzie Boze? Odpowiedzial Syba królowi: Jest jeszcze syn Jonatana, chromy na nogi.
\par 4 I rzekl do niego król: Gdzieli jest? A Syba odpowiedzial królowi: Oto, jest w domu Machira, syna Ammijelowego, w Lodebarze.
\par 5 Przetoz poslal król Dawid, i wzial go z domu Machira, syna Ammijelowego z Lodebaru.
\par 6 A gdy przyszedl Mefiboset, syn Jonatana, syna Saulowego, do Dawida, upadl na oblicze swe, i poklonil sie. I rzekl Dawid: Mefibosecie! Który odpowiedzial: Oto, sluga twój.
\par 7 I rzekl do nigo Dawid: Nie bój sie: bo zapewne uczynie z toba milosierdzie dla Jonatana, ojca twego, i przywrócec wszystke rola Saula, dziada twego. a ty bedziesz jadl chleb u stolu mego zawzdy.
\par 8 Tedy ukloniwszy sie, rzekl: Coz jest sluga twój, zes sie obejrzal na psa zdechlego, jakom ja jest?
\par 9 Zatem wezwal król Syby, slugi Saulowego i rzekl mu: Cokolwiek mial Saul, i wszystek dom jego, dalem synowi pana twego.
\par 10 Bedziesz tedy sprawowal rola jego, ty, synowie twoi, i sludzy twoi, a bedziesz dodawal, aby mial chleb syn pana twego, któryby jadl; ale Mefiboset, syn pana twego, bedzie zawzdy jadal chleb u stolu mego. A Syba mial pietnascie synów i dwadziescia slug.
\par 11 I odpowiedzial Syba królowi: Wszystko, co rozkazal król, pan mój, sludze swemu, tak uczyni sluga twój, aczkolwiek Mefiboset móglby jadac u stolu mego, jako jeden z synów królewskich.
\par 12 Mial tez Mefiboset syna malego, imieniem Micha; a wszyscy, którzy mieszkali w domu Sybowym, byli slugami Mefibosetowymi.
\par 13 A tak Mefiboset mieszkal w Jeruzalemie, bo on u stolu królewskiego zawzdy jadal; a byl chromy na obie nogi.

\chapter{10}

\par 1 I stalo sie potem, ze umarl król synów Ammonowych, a królowal Hanon, syn jego, po nim.
\par 2 Tedy rzekl Dawid: Uczynie milosierdzie z Hanonem, synem Nahasowym, jako uczynil ojcec jego milosierdzie ze mna.I poslal Dawid cieszac go przez slugi swe po ojcu jego, a tak przyszli sludzy Dawidowi do ziemi synów Ammonowych.
\par 3 Ale ksiazeta synów Ammonowych rzekly do Hanona, pana swego: I mniemasz, zeby to uczciwosc czynil Dawid ojcu twemu, iz przyslal do ciebie tych, którzyby cie cieszyli? Azaz raczej nie dla tego poslal Dawid slugi swe do ciebie aby przepatrzyli miasto, i wyszpiegowali je, aby je potem zburzyl?
\par 4 A tak wziawszy Hanon slugi Dawidowe, ogolil im po polowie brody ich, i poobrzynal szaty ich az do polowy, az do zadków ich, i puscil je.
\par 5 A gdy to opwiedziano Dawidowi, poslal przeciwko nim, (poniewaz oni mezowie byli bardzo obelzeni,)i rzekl im król: Zostncie w Jerycho, az odrosna brody wasze, a potem sie wrócicie.
\par 6 Widzac tedy synowie Ammonowi, ze sie zbrzydzili Dawidowi poslali ciz synowie Ammonowi, i najeli za pieniadze Syryjczyka z domu Rechob, i Syryjczyka w Soba, dwadziescia tysiecy pieszych, a od króla Maacha tysiac mezów, a od Istoba dwanascie tysiecy mezów.
\par 7 Co gdy uslyszal Dawid, poslal Joaba ze wszystkiem wojskiem ludzi rycerskich.
\par 8 Tedy synowie Ammonowi wyciagneli, a uszykowali sie do bitwy przed samem wejsciem w brame; Syryjczyk zasie z Soby, i Rechob, i Istob, i Maacha byli osobno w polu.
\par 9 Przetoz widzac Joab uszykowane wojska przeciwko sobie z przodku i z tylu, wybral niektóre ze wszystkich przebranych z Izraela, i uszykowal wojsko przeciwko Syryjczykom.
\par 10 A ostatek ludu dal pod reke Abisaja, brata swego, i uszykowal je przeciwko synom Ammonowym.
\par 11 I rzekl: Jezli mi Syryjczycy beda silnymi, bedziesz mi na pomoc, a jezli tobie synowie Ammonowi beda silnymi przyjdec na pomoc.
\par 12 Zmacniaj sie, a badzmy meznymi za lud nasz, i za miasto Boga naszego, a Pan niech uczyni, co dobrego jest w oczach jego.
\par 13 Nastapil tedy Joab, i lud, który byl z nim, aby zwiódl bitwe z Syryjczykami; a oni uciekali przed nim.
\par 14 Tedy synowie Ammonowi ujrzawszy, ze uciekli Syryjczycy; uciekli i oni przed Abisajem, i weszli do miasta. A Joab wrócil sie od synów Ammonowych, i przyszedl do Jeruzalemu.
\par 15 A gdy obaczyli Syryjczycy, iz sa porazeni od Izraela, zebrali sie wespól.
\par 16 I poslal Hadadezer, a wywiódl Syryjczyki, którzy byli za rzeka, i przyciagneli do Helam, a Sobach, hetman wojska Hadadezerowego prowadzil je.
\par 17 I oznajmiono to Dawidowi; który zebrawszy wszystkiego Izraela, przeprawil sie przez Jordan, i przyszedl do Helam, gdzie uszykowawszy wojsko Syryjczycy przeciw Dawidowi, zwiedli z nim bitwe.
\par 18 Tedy uciekli Syryjczycy przed Izraelem, i porazil Dawid Syryjczyków siedm set wozów, i czterdziesci tysiecy jezdnych; do tego Sobacha, hetmana wojska ich, ranil, który tamze umarl.
\par 19 A gdy ujrzeli wszyscy królowie, holdownicy Hadadezerowi, iz porazeni byli od Izraela, uczynili pokój z Izraelem, i sluzyli im; i bali sie Syryjczycy dawac pomocy na potem synom Ammonowym.

\chapter{11}

\par 1 I stalo sie po roku tego czasu, gdy zwykli królowie wyjezdzac na wojne, poslal Dawid Joaba, i slugi swoje z nim, i wszystkiego Izraela, aby pustoszyli syny Ammonowe. I oblegli Rabbe, a Dawid zostal w Jeruzalemie.
\par 2 I stalo sie przed wieczorem, gdy wstal Dawid z loza swego, a przechadzal sie po dachu domu królewskiego, ze ujrzal z dachu niewiaste, myjaca sie; a ta niewiasta byla bardzo piekna na wejrzeniu.
\par 3 Tedy poslal Dawid, pytajac sie o onej niewiescie, i rzekl: Azaz to nie Betsabee, córka Elijamowa, zona Uryjasza Hetejczyka?
\par 4 Poslal tedy Dawid posly, i wzial ja. Która gdy weszla do niego, spal z nia; a ona sie byla oczyscila od nieczystoty swojej: potem wrócila sie do domu swego.
\par 5 I poczela ona niewiasta, a poslawszy oznajmila Dawidowi, i rzekla: Jam brzemienna.
\par 6 I poslal Dawid do Joaba mówiac: Poslij do mnie Uryjasza Hetejczyka. I poslal Joab Uryjasza do Dawida.
\par 7 A gdy przyszedl Uryjasz do niego, pytal go Dawid jakoby sie powodzilo Joabowi, i jakoby sie powodzilo ludowi, i jakoby sie powodzilo wojsku.
\par 8 Nadto rzekl Dawid do Uryjasza: Idz do domu twego, a umyj nogi twoje. I wyszedl Uryjasz z domu królewskiego, a niesiono za nim potrawy królewskie.
\par 9 Ale Uryjasz spal przede drzwiami domu królewskiego ze wszystkimi slugami pana swego, i nie szedl do domu swojego.
\par 10 I opowiedziano Dawidowi, mówiac: Nie szedlci Uryjasz do domu swego. I rzekl Dawid do Uryjasza: Azazes ty nie z drogi przyszedl? przeczzes wzdy nie szedl do domu twego?
\par 11 I rzekl Uryjasz do Dawida: Skrzynia Boza, i Izrael, i Juda zostawaja w namiotach, a pan mój Joab, i sludzy pana mego w polu obozem leza, a jabym mial wnijsc do domu mego, abym jadl, i pil, i spal z zona swa? Jakos ty zyw, i jako zywa dusza twoja, zec tego nie uczynie.
\par 12 Tedy rzekl Dawid do Uryjasza: Zostanze tu jeszcze dzis, a jutro cie odprawie. I zostal Uryjasz w Jeruzalemie przez on dzien, i nazajutrz.
\par 13 Potem go wezwal Dawid, aby jadl i pil przed nim, i upoil go: wszakze wyszedlszy w wieczór, spal na lozu swojem z slugami pana swego, a do domu swego nie wszedl.
\par 14 A gdy bylo rano, napisal Dawid list do Joaba, i poslal go przez rece Uryjasza.
\par 15 A w liscie napisal te slowa: Postawcie Uryjasza na czele bitwy najtezszej; miedzy tem odstapcie nazad od niego, aby bedac raniony umarl.
\par 16 I stalo sie, gdy oblegl Joab miasto, postawil Uryjasza na miejscu, kedy wiedzial, ze byli mezowie najmocniejsi.
\par 17 A wypadlszy mezowie z miasta, stoczyli bitwe z Joabem, i poleglo z ludu kilka slug Dawidowych, polegl tez Uryjasz Hetejczyk,
\par 18 Tedy poslal Joab, i oznajmil Dawidowi wszystko, co sie stalo w bitwie.
\par 19 A rozkazal poslowi, mówiac: Gdy wypowiesz królowi, co sie stalo w bitwie,
\par 20 Tedy jezliby sie król rozgniewal, a rzeklciby: Przeczzescie tak blisko przystapili do miasta ku bitwie? azazescie nie wiedzieli, iz ciskaja z muru?
\par 21 Któz zabil Abimelecha, syna Jerubbesetowego? izali nie niewiasta zrzuciwszy nan sztuke kamienia mlynskiego z muru, tak ze umarl w Tebes? przeczzescie przystepowali do muru? Tedy rzeczesz: Sluga tez twój Uryjasz Hetejczyk polegl.
\par 22 A tak poszedl posel, i przyszedlszy oznajmil Dawidowi wszystko, z czem go byl poslal Joab.
\par 23 I rzekl on posel do Dawida: Zmocnili sie przeciwko nam mezowie, i wyszli przeciwko nam w pole, a gonilismy je az do samej bramy.
\par 24 Wtem strzelili strzelcy na slugi twoje z muru, i zabito kilka slug królewskich, tamze i sluga twój Uryjasz Hetejczyk polegl.
\par 25 Tedy rzekl Dawid do posla: Tak powiesz Joabowi: Niech ci to serca nie psuje, boc tak miecz to tego, to owego pozera; nastepuj poteznie na miasto, i burz je, a dodawaj serca rycerstwu.
\par 26 A uslyszawszy zona Uryjaszowa, iz umarl Uryjasz, maz jej, plakala meza swego.
\par 27 A gdy wyszla zaloba, poslal Dawid, i wzial ja w dom swój, i byla mu za zone, i porodzila mu syna. Ale to byla zla rzecz, która uczynil Dawid przed oczyma Panskiemi.

\chapter{12}

\par 1 Przetoz poslal Pan Natana do Dawida; który przyszedlszy do niego, rzekl mu: Dwaj mezowie byli w jednem miescie, jeden bogaty a drugi ubogi.
\par 2 Bogaty mial owiec i wolów bardzo wiele;
\par 3 A ubogi nie mial jedno owieczke jedne mala, która byl kupil, i chowal ja, az urosla przy nim, takze i przy dziatkach jego; z bochna jego jadala, i z kubka jego pijala, i na lonie jego sypiala, a byla mu jako córka.
\par 4 A gdy przyszedl gosc do onego meza bogatego, zalowal wziac z owiec swoich albo z wolów swoich, aby nagotowal uczte gosciowi, który byl do niego przyszedl: ale wzial owieczke meza onego ubogiego, i nagotowal ja mezowi, który byl do niego przyszedl.
\par 5 Tedy zapaliwszy sie gniewem Dawid na onegoz meza bardzo, rzekl do Natana: Jako zywy Pan, ze godzien smierci jest maz, który to uczynil;
\par 6 Owce te nagrodzi czworako, przeto iz to uczynil, a nie zalowal go.
\par 7 I rzekl Natan do Dawida: Tys jest tym mezem. Tak ci mówi Pan, Bóg Izraelski: Jam cie pomazal, abys byl królem nad Izraelem, i Jam cie wyrwal z rak Saulowych;
\par 8 A podalem ci dom pana twego i zony pana twego na lono twoje; nadto oddalem ci dom Izraelski i Judzki, a byloliby to malo, przydalbym ci byl daleko wiecej.
\par 9 Czemuzes zniewazyl slowo Panskie, czyniac to zle przed oczyma jego? Uryjasza Hetejczyka zabiles mieczem, a zone jego wziales sobie za zone, a samegos zabil mieczem synów Ammonowych.
\par 10 Przetoz teraz nie odejdzie miecz z domu twego az na wieki, dlatego, izes mie zniewazyl, a wziales zone Uryjasza Hetejczyka, abyc byla za zone.
\par 11 Tak mówi Pan: Oto, Ja wzbudze przeciwko tobie zle z domu twego, a pobrawszy zony twe przed oczyma twemi, dam je blizniemu twemu, a badzie jawnie spal z zonami twojemi.
\par 12 A chociazes to ty uczynil potajemnie, Ja jednak uczynie to przed wszystkim Izraelem, i przed sloncem.
\par 13 Tedy rzekl Dawid do Natana: Zgrzeszylem Panu. Zas rzekl Natan do Dawida: Pan tez przeniósl grzech twój, nie umrzesz.
\par 14 Wszakze izes dal przyczyne, aby uragali nieprzyjaciele Panscy dla tej sprawy, przetoz syn, któryc sie urodzil, pewnie umrze.
\par 15 Potem odszedl Natan do domu swego. Wtem zarazil Pan dziecie, które byla urodzila zona Uryjaszowa Dawidowi, i zwatpiono o niem.
\par 16 Tedy sie modlil Dawid Bogu za dziecieciem i poscil, a wszedlszy do pokoju, lezal przez noc na ziemi.
\par 17 I przyszli starsi domu jego do niego aby go podniesli z ziemi; ale niechcial, i nie jadl z nimi chleba.
\par 18 I stalo sie dnia siódmego, ze umarlo dziecie. A obawiali sie sludzy Dawidowi, oznajmic mu, iz umarlo dziecie, bo mówili: Oto, póki jeszcze dziecie bylo zywe, mówilismy z nim, a nie sluchal glosu naszego; cóz gdy mu powiemy: Umarlo dziecie, dopier oz sie bedzie trapil.
\par 19 A widzac Dawid, ze sludzy jego szeptali z soba, porozumial Dawid, iz umarlo dziecie. I rzekl Dawid do slug swoich: Albo umarlo dziecie? A oni odpowiedzieli: Umarlo.
\par 20 Tedy wstawszy Dawid z ziemi, umyl sie, i namazal sie, i odmienil szaty swoje, a wszedlszy do domu Panskiego, modlil sie; potem wróciwszy sie do domu swego, kazal sobie dac jesc, i polozono przeden chleb, i jadl.
\par 21 I rzekli sludzy jego do niego: Cóz to jest, cos uczynil? Dla dziciecia, póki jeszcze zylo, posciles i plakales, a gdy umarlo dziecie, wstales i jadles chleb?
\par 22 A on rzekl: Póki jeszcze dziecie zylo, poscilem i plakalem; bom mówil: Któz wie, nie zmilujeli sie Pan nademna, ze bedzie zywe dziecie.
\par 23 Ale teraz, gdy juz umarlo, przeczzebym mial poscic? Izali je moge jeszcze nazad wrócic? Ja pójde do niego, ale sie ono nie wróci do mnie.
\par 24 I cieszyl Dawid Betsabee, zone swa, a wszedlszy do niej, spal z nia. I porodzila syna, i nazwal imie jego Salomon, a Pan go milowal.
\par 25 Przetoz poslal Natana proroka, i nazwal imie jego Jedydyja, dla Pana.
\par 26 Potem walczyl Joab przeciw Rabbie synów Ammonowych, i wzial miasto królewskie
\par 27 A poslawszy Joab posly do Dawida, rzekl: Walczylem przeciw Rabbie i wzialam miasto wód.
\par 28 Przetoz teraz zbierz ostatek ludu, a polóz sie obozem przeciwko miastu, i wezmij je, bym ja snac nie wzial miasta tego, a przypisanoby zwyciestwo imieniowi memu.
\par 29 A tak zebral Dawid wszystek lud, i ciagnal przeciw Rabbie, i dobywal go, a wzial je.
\par 30 Wzial tez korone króla ich z glowy jego, która wazyla talent zlota, a kamien drogi byl na niej; i wlozono ja na glowe Dawidowe, a lupów z miasta wyniósl bardzo wiele.
\par 31 Lud tez, który byl w miescie, wywiódlszy, podal pod pily, i pod brony zelazne, i pod siekiery zelazne, i wegnal je w piec cegielny. A tak uczynil wszystkim miastom synów Ammonowych; i wrócil sie Dawid, i wszystek lud jego do Jeruzalem.

\chapter{13}

\par 1 I stalo sie potem, ze Absalom syn Dawida, mial siostre piekna, imieniem Tamar, której sie rozmilowal Amnon, syn Dawida.
\par 2 I trapil sie Amnon tak, ze zachorowal dla Tamary, siostry swojej; bo panna byla, i trudno sie zdalo Amnonowi, aby jej co mial uczynic.
\par 3 Lecz Amnon mial przyjaciela, którego zwano Jonadab, syn Semmy, brata Dawidowego; a ten Jonadab byl mezem bardzo madrym.
\par 4 Który mu rzekl: Czumuz tak schniesz, synu królewski, ode dnia do dnia? czemuz mi nie oznajmisz? Tedy mu rzekl Amnon: Romilowalem sie Tamary siostry Absaloma, brata mego.
\par 5 I rzekl mu Jonadab: Ukladz sie na lózku twojem, a uczyn sie chorym; a gdy przyjdzie ojciec twój, aby cie nawiedzil, rzeczesz mu: Niech przyjdzie prosze Tamar, siostra moja, i da mi jesc, a nagotuje przed oczyma memi potrawe, abym widzial a jadl z reki jej.
\par 6 Tedy sie ukladl Amnon, zmyslajac sobie chorobe. A gdy przyszedl król nawiedzac go, rzekl Amnon do króla: Niech przyjdzie prosze Tamar, siostra moja, aby zgotowala przed oczyma memi dwa placki, abym jadl z reki jej.
\par 7 Przetoz poslal Dawid do Tamary w dom, mówiac: Idz zaraz do domu Amnona, brata twego, a nagotuj mu potrawe.
\par 8 Przyszla tedy Tamar do domu Amnona, brata swego, a on lezal; a wziawszy maki rozmacila, i uczynila placki przed oczyma jego, i upiekla je.
\par 9 Potem wziawszy panewke, wylozyla przeden; ale nie chcial jesc. I rzekl Amnon: Kazcie wyjsc precz wszystkim odemnie; a tak wyszli wszyscy od niego.
\par 10 Tedy rzekl Amnon do Tamary: Przynies sama te potrawe do pokoju, abym jadl z reki twej. A tak wziawszy Tamara placki, które nagotowala, przyniosla je przed Amnona, brata swego, do pokoju.
\par 11 A gdy mu podawala, aby jadl, uchwyciwszy ja, rzekl do niej: Pójdz, lez ze mna, siostro moja.
\par 12 Która mu rzekla: Zaniechaj, bracie mój, a nie czyn mi gwaltu, bo sie niema dziac nic takiego w Izraelu; nie czynze tego szalenstwa.
\par 13 Bo gdziezebym sie obrócila z zelzywoscia moja? a ty bedziesz jako jeden z szalonych w Izraelu. Ale raczej mów prosze z królem; bo mie nie odmówi tobie.
\par 14 Lecz on nie chcial usluchac glosu jej, ale zmóglszy ja, uczynil jej gwalt, i lezal z nia.
\par 15 Potem nienawidzial jej Amnon nienawiscia bardzo wielka, tak iz wieksza byla nienawisc, która ja nienawidzial, niz milosc, która ja pierwej milowal. I rzekl jej Amnon: Wstan, idz precz.
\par 16 Która mu odpowiedziala: Dlatego to wieksza zlosc, niz owa, któras zemna popelnil, ze mie wyganiasz. Ale jej on nie chcial usluchac.
\par 17 Owszem zawolawszy chlopca swego, który mu poslugiwal, rzekl: Wywiedzcie te zaraz precz odemnie, a zamknij drzwi za nia.
\par 18 (Ale ona miala na sobie pstra suknia; albowiem w takowych sukniach chadzaly córki królewskie, panny,)i wywiódl ja precz sluga jego, i zawarl drzwi za nia.
\par 19 Tedy posypala Tamar popiolem glowe swa, a pstra szate, która byla na niej, rozdarla, i wlozywszy reke swa na glowe swoje, poszla, a idac krzyczala.
\par 20 I rzekl do niej Absalom, brat jej: Albo Amnon, brat twój, byl z toba? Milczze, siostro moja; brat twój jest, nie przypuszczaj tego do serca swego. A tak mieszkala Tamar bedac opuszczona, w domu Absaloma, brata swego.
\par 21 A król Dawid uslyszawszy o tem wszystkiem, rozgniewal sie bardzo.
\par 22 I nie mówil Absalom z Amnonem ani zle ani dobrze; bo nienawidzial Absalom Amnona, przeto ze zgwalcil Tamare, siostre jego.
\par 23 I stalo sie po wyjsciu dwóch lat, gdy strzyzono owce Absalomowe w Baalchasor, które jest w Efraim, ze wyzwal Absalom wszystkich synów królewskich.
\par 24 Bo przyszedl Absalom do króla i rzekl: Oto teraz strzyge owce sludze twemu; niech idzie prosze król i sludzy jego z sluga twoim.
\par 25 I rzekl król do Absaloma: Nie, synu mój; niech teraz nie chodzimy wszyscy, abysmy cie nie obciazyli. A choc mu przynaglal, nie chcial isc, ale mu blogoslawil.
\par 26 Rzekl potem Absalom: Poniewaz ty nie chcesz, niechze idzie prosze z nami Amnon, brat mój. I rzekl mu król: A pocózby mial isc z toba?
\par 27 A gdy nan nalegal Absalom, poslal z nim Amnona i wszystkie syny królewskie.
\par 28 Tedy przykazal Absalom slugom swoim, mówiac: Pilnujcie prosze, gdy podweseli serce swoje Amnon winem, a rzeke do was: Bijcie Amnona, zabijciez go, nie bójcie sie, bom ja wam rozkazal; zmocnijciez sie, a meznie sobie pocznijcie.
\par 29 I uczynili sludzy Absalomowi Amnonowi, jako im byl rozkazal Absalom. Przetoz wstawszy wszyscy synowie królewscy, wsiedli kazdy na mula swego, i uciekali.
\par 30 Wtem gdy jeszcze byli w drodze, wiesc przyszla do Dawida w te slowa: Pozabijal Absalom wszystkie syny królewskie, i nie zostal z nich ani jeden.
\par 31 Tedy wstal król i rozdarl szaty swoje, i padl na ziemie, a wszyscy sludzy jego stali okolo niego, rozdarlszy szaty swoje.
\par 32 A ozwawszy sie Jonadab syn Semmy, brata Dawidowego, rzekl: Niech nie mówi pan mój, ze wszystkie mlodzience, syny królewskie, pobito; boc tylko sam Amnon zabity, gdyz to w umysle Absalomowym ulozono bylo od onego dnia, którego zgwalcil Tamare, siostre jego.
\par 33 Przetoz teraz niech nie przypuszcza tego król, pan mój, do serca swego, mówiac: Wszyscy synowie królewscy polegli, gdyz tylko sam Amnon polegl.
\par 34 Tedy uciekl Absalom; a podnióslszy sluga, który byl na strazy, oczy swe, ujrzal, a oto, lud wielki przychodzil droga, która chadzano do niego z boku góry.
\par 35 I rzekl Jonadab do króla: Oto synowie królewscy jada; wedle slowa slugi twego tak sie stalo.
\par 36 A gdy przestal mówic, oto synowie królewscy przyszli, a podnióslszy glosy swe plakali; takze i król, i wszyscy sludzy jego plakali placzem bardzo wielkim.
\par 37 Ale Absalom ucieklszy uszedl do Tolmaja, syna Ammihudowego, króla Giessur; i zalowal Dawid syna swego po one wszystkie dni.
\par 38 A Absalom uciekl, i przyszedl do Giessur, a byl tam przez trzy lata.
\par 39 Potem pragnal król Dawid widziec Absaloma; bo juz byl odzalowal smierci Amnonowej.

\chapter{14}

\par 1 A porozumiawszy Joab, syn Sarwii, ze sie serce królewskie obrócilo ku Absalomowi,
\par 2 Poslal Joab do Tekuj, i wzial stamtad niewiaste madra, i rzekl do niej: Prosze zmysl, jakobys w zalobie byla, a oblecz sie prosze w szaty zalobne, i nie namazuj sie olejkiem, ale badz jako niewiasta, która przez wiele dni w zalobie chodzila po uma rlym.
\par 3 I wnijdziesz do króla, a bedziesz mówila do niego w ten sposób; i nauczyl jej Joab, jako miala mówic.
\par 4 Przetoz mówila ona niewiasta Tekuitska do króla, upadlszy obliczem swem na ziemie, a pokloniwszy sie rzekla: Ratuj królu!
\par 5 I rzekl jej król: Cóz ci? A ona odpowiedziala: Zaistem ja niewiasta wdowa; bo mi maz mój umarl:
\par 6 A sluzebnica twoja miala dwóch synów, którzy sie powadzili z soba na polu; a gdy nie byl, ktoby je rozwadzil, a ranil jeden drugiego, i zabil go.
\par 7 A oto, powstawszy wszystka rodzina przeciw sluzebnicy twojej, mówia: Wydaj tego, który zabil brata swego, ze go zabijemy za dusze brata jego, którego zamordowal, owszem zgladzimy i dziedzica; a tak zagasza iskierke moje, która pozostala, aby nie z ostawili mezowi memu imienia i szczatku na ziemi.
\par 8 Tedy rzekl król do niewiasty: Idz do domu twego, a ja skaze za toba.
\par 9 I odpowiedziala niewiasta Tekuicka królowi: Królu, panie mój, niech bedzie na mnie ta nieprawosc, i na dom ojca mego; ale król i stolica jego niech bedzie niewinna.
\par 10 I rzekl król: Bedzieli kto mówil przeciwko tobie, przywiedz go do mnie, a potem nie tknie sie ciebie wiecej.
\par 11 Tedy ona rzekla: Wspomnij prosze, królu, na Pana Boga twego, aby sie nie mnozyli msciciele krwi na zgube, a nie zgladzili syna mego. I odpowiedzial: Jako zywy Pan, ze nie spadnie i najmniejszy wlos syna twego na ziemie.
\par 12 Zatem rzekla niewiasta: Niech przemówi prosze sluzebnica twoja do króla, pana mego, slowo. A on rzekl: Mów.
\par 13 Rzekla tedy niewiasta: I czemuzes umyslil podobna rzecz przeciw ludowi Bozemu? albowiem król mówi to slowo, jakoby byl winny, poniewaz nie chcesz przywrócic, królu, wygnanca swego.
\par 14 Wszyscy umieramy, a jestesmy jako wody rozlane po ziemi, które nie moga byc zebrane; lecz jemu Bóg nie odjal zywota, ale pewnie umyslil, aby nie wyganial od siebie wygnanca.
\par 15 A teraz, zem przyszla mówic do króla, pana mego, te slowa, przyczyna jest, ze mie postraszyl lud; przetoz rzekla sluzebnica twoja: Bede teraz mówila do króla, snac co uczyni król na prosbe sluzebnicy swojej.
\par 16 Albowiem uslyszy to król, i wybawi sluzebnice swoje z rak meza, który wygladzic chce mnie, i syna mego spolem, z dziedzictwa Bozego.
\par 17 Rzekla tez sluzebnica twoja: Wzdyc mi bedzie slowo króla, pana mego, ku pociesze; albowiem jako Aniol Bozy, tak jest król, pan mój, sluchajac dobrego i zlego, a Pan Bóg twój niech bedzie z toba.
\par 18 A odpowiadajac król rzekl do niewiasty: Prosze nie taj przedemna tego, o co sie spytam. I rzekla niewiasta: Mów prosze, królu, panie mój.
\par 19 Tedy rzekl król: Izali ty tego wszystkiego nie czynisz z naprawy Joabowej? I odpowiedziala niewiasta, i rzekla: Jako zyje dusza twoja, królu, panie mój, ze nie mozna uchylic sie ani na prawo, ani na lewo od wszystkiego, co mówil król, pan mój; albowiem sluga twój Joab, on mi to rozkazal, i on nauczyl sluzebnicy twojej tych wszystkich slów.
\par 20 Zem odmienila sposób tej mowy, sprawil to sluga twój Joab; lecz pan mój madry jest, jako jest madry Aniol Bozy, wiedzac wszystko, co sie dzieje na ziemi.
\par 21 Przetoz rzekl król do Joaba: Otom teraz to uczynil. Idzze a przywróc dziecie me Absaloma.
\par 22 I upadl Joab obliczem swem na ziemie, a pokloniwszy sie blogoslawil królowi, i rzekl Joab: Dzis poznal sluga twój, zem znalazl laske w oczach twoich, królu, panie mój, poniewaz uczynil król zadosc prosbie slugi swego.
\par 23 Wstal tedy Joab, a szedl do Giessur, i przywiódl Absaloma do Jeruzalemu.
\par 24 I rzekl król: Niech sie wróci do domu swego, ale oblicza mego niech nie widzi. A tak wrócil sie Absalom do domu swego, ale oblicza królewskiego nie widzial.
\par 25 A nie bylo meza tak krasnego, jako Absalom we wszystkim Izraelu, coby mial tak wielka chwale; od stopy nogi jego az do wierzchu glowy jego nie bylo na nim zmazy.
\par 26 A gdy strzygl glowe swoje, (a zwykl ja na kazdy rok strzydz; bo mu ciazyla, przetoz ja strzygl,)wazyly wlosy glowy jego dwiescie syklów wagi królewskiej.
\par 27 I urodzili sie Absalomowi trzej synowie, i córka jedna, której imie bylo Tamar, która niewiasta byla piekna na wejrzeniu.
\par 28 I mieszkal Absalom w Jeruzalemie dwa lata, a twarzy królewskiej nie widzial.
\par 29 Przetoz poslal Absalom do Joaba, chcac go poslac do króla, ale on nie chcial przyjsc do niego; poslal potem powtóre, i nie chcial przyjsc.
\par 30 Tedy rzekl do slug swoich: Przepatrzcie role Joabowa podle roli mojej, gdzie ma jeczmien; idzciez, a spalcie go ogniem. I zapalili sludzy Absalomowi role one ogniem.
\par 31 Zatem wstawszy Joab, przyszedl do Absaloma w dom, i rzekl do niego: Czemóz sludzy twoi spalili role moje ogniem?
\par 32 I odpowiedzial Absalom Joabowi: Otom poslal do ciebie, mówiac: Przyjdz sam, a posle cie do króla, abys mówil: Nacózem przyszedl z Giessur? Lepiej mi bylo tam jeszcze zostac; przetoz teraz niech ogladam oblicze królewskie; wszak jezli jest przy mnie nieprawosc, niech mie rozkaze zabic.
\par 33 Tedy przyszedl Joab do króla, i oznajmil mu. I przyzwal Absaloma, który przyszedl do króla, i uklonil sie twarza swa ku ziemi przed królem; i pocalowal król Absaloma.

\chapter{15}

\par 1 I stalo sie potem, ze sobie nasprawial Absalom wozów, i koni, i piecdziesiat mezów, którzy chodzili przed nim.
\par 2 I wstawajac rano Absalom stawal podle drogi u bramy, a kazdego meza, majacego sprawe a idacego do króla na sad, przyzywal Absalom do siebie, i mówil: Z któregozes ty miasta? A gdy mu odpowiedzial: Z jednego pokolenia Izraelskiego jest sluga twój.
\par 3 Mówil mu Absalom: Oto, sprawa twoja dobra jest, i sprawiedliwa; ale niemasz, ktoby cie wysluchal u króla.
\par 4 Nadto mówil Absalom: O ktoby mie postanowil sedzia w tej ziemi! aby do mnie chodzil kazdy, któryby mial sprawe u sadu, dopomóglbym mu do sprawiedliwosci.
\par 5 A gdy kto przystapil, i uklonil mu sie, sciagnal reke swa, a ujawszy go, calowal go.
\par 6 A toc czynil Absalom wszystkiemu Izraelowi, który przychodzil na sady do króla, i ukradal Absalom serca mezów Izraelskich.
\par 7 I stalo sie po czterdziestu latach, ze rzekl Absalom do króla: Niech ide prosze, a oddam slub mój w Hebronie, którym poslubil Panu.
\par 8 Albowiem slub poslubil sluga twój, kiedym mieszkal w Giessur Syryjskim, mówiac: Jezlize mie zasie kiedy przywróci Pan do Jeruzalemu, tedy sluzyc bede Panu.
\par 9 I rzekl mu król: Idz w pokoju. A on wstawszy poszedl do Hebronu.
\par 10 Tedy rozeslal Absalom szpiegi miedzy wszystkie pokolenia Izraelskie, aby rzekli: Skoro uslszycie glos traby, mówciez: Króluje Absalom w Hebronie.
\par 11 A z Absalomem poszlo bylo dwiescie mezów z Jeruzalemu zaproszonych, którzy szli w prostosci swojej, niewiedzac o niczem.
\par 12 Poslal tez Absalom po Achitofela Giloniczyka, radce Dawidowego, aby przyszedl z miasta swego Gilo, gdy mial sprawowac ofiary. I stalo sie sprzysiezenie wielkie, a lud sie schodzil, i przybywalo go Absalomowi.
\par 13 Potem przyszedl posel do Dawida, mówiac: Obrócilo sie serce mezów Izraelskich za Absalomem.
\par 14 Tedy rzekl Dawid do wszystkich slug swoich, którzy z nim byli w Jeruzalemie: Wstancie, a uciekajmy; inaczej nieuszlibysmy przed twarza Absalomowa. Spieszciez sie, azaz ujdziem, by sie snac nie pospieszyl, a nie zajechal nas, i nie obalil na nas z lego, i nie wysiekl miasta ostrzem miecza.
\par 15 I rzekli sludzy królewscy do króla: Wszystko, cokolwiek sobie upodoba król, pan nasz, oto sludzy twoi.
\par 16 A tak wyszedl król, i wszystek dom jego pieszo; tylko zostawil król dziesiec niewiast zaloznic, aby strzegly domu.
\par 17 A gdy wyszedl król i wszystek lud pieszo, staneli na jednem miejscu z daleka.
\par 18 Wszyscy tez sludzy jego szli przy nim, i wszyscy Cheretczycy, i wszyscy Feletczycy, i wszyscy Gietejczycy, szesc set mezów, którzy byli przyszli pieszo z Giet, szli przed twarza królewska.
\par 19 Tedy rzekl król do Itaja Gietejczyka: Czemuz i ty z nami idziesz? Wróc sie, a zostan przy królu; bos ty cudzoziemiec, a nie dlugo wrócisz sie do miejsca twego.
\par 20 Niedawnos przyszedl, a dzisbym cie ruszyc mial, abys z nami szedl? Gdyz ja ide, sam nie wiem dokad; wrócze sie, a odprowadz bracia swoje: niech bedzie z toba milosierdzie i prawda.
\par 21 Ale odpowiedzial Itaj królowi, mówiac: Jako zywy Pan, jako zywy tez król pan mój, ze na któremkolwiek miejscu bedzie król, pan mój, choc w smierci, choc w zywocie, tam tez badzie sluga twój.
\par 22 I rzekl Dawid do Itaja: Pójdzze, a przejdz. I przeszedl Itaj Gietejczyk, i wszyscy mezowie jego, i wszystkie dziatki, które byly z nim.
\par 23 Tedy wszystka ziemia plakala glosem wielkim, i wszystek lud, który przechodzil. A tak król przeszedl przez potok Cedron, a wszystek lud przeszedl przeciw drodze ku puszczy.
\par 24 A oto i Sadok i wszyscy Lewitowie byli z nim, niosac skrzynie przymierza Bozego, i postawili skrzynie Boza; szedl tez Abijater, az wszystek on lud przeszedl z miasta.
\par 25 I rzekl król do Sadoka: Odnies zasie skrzynie Boza do miasta. Jezlic znajde laske w oczach Panskich, przywróci mie zasie, a ukaze mi ja, i przbytek swój.
\par 26 Ale jezliby tak rzekl: Nie podobasz mi sie; otom ja, niech mi uczyni, co dobrego jest w oczach jego.
\par 27 Nadto rzekl król do Sadoka kaplana: Izalis nie jest widzacym? Wrócze sie do miasta w pokoju, i Achimaas, syn twój, i Jonatan, syn Abijatara, dwaj synowie wasi, z wami.
\par 28 Oto, ja pomieszkam w równinach na puszczy, póki nie przyjdzie od was poselstwo dawajace mi znac.
\par 29 A tak odprowadzili zasie Sadok i Abijatar skrzynie Boza do Jeruzalemu, i zostali tam.
\par 30 Ale Dawid szedl na góre oliwna wstepujac i placzac, majac glowe przykryta, i idac boso; wszystek tez lud, który z nim byl, zakryli kazdy glowe swoje, a szli wstepujac i placzac.
\par 31 Tedy dano znac Dawidowi, mówiac: Achitofel jest z tymi, którzy sie zbuntowali z Absalomem. I rzekl Dawid: O Panie, prosze, obróc w glupstwo rade Achitofelowe.
\par 32 I stalo sie, gdy Dawid przyszedl az na wierzch góry, aby sie tam pomodlil Bogu, oto, spotkal sie z nim Chusaj Arachita, miawszy rozdarte szaty swe, a proch na glowie swojej.
\par 33 I rzekl mu Dawid: Jezli pójdziesz ze mna, bedziesz mi ciezarem;
\par 34 Ale jezli sie do miasta wrócisz, a rzeczesz do Absaloma: Królu, sluga twoim bede, bom byl sluga ojca twego zdawna, ale teraz jam sluga twoim: tedy mi obrócisz wniwecz rade Achitofelowa.
\par 35 Azaz tam nie bedzie z toba Sadoka i Abijatara, kaplanów? Przetoz cokolwiek uslyszysz z domu królewskiego, oznajmisz Sadokowi i Abijatarowi, kaplanom.
\par 36 Sa tez tam z nimi dwaj synowie ich, Achimaas, syn Sadoka, i Jonatan, syn Abijatara, przez które dacie mi znac o wszystkiem, co jedno usluszycie.
\par 37 Szedl tedy Chusaj przyjaciel Dawida do miasta, a Absalom tez wjechal do Jeruzalemu.

\chapter{16}

\par 1 A gdy Dawid zszedl troche z wierzchu góry, oto Syba, sluga Mefiboseta, zaszedl mu droge z para oslów osiodlanych, na których bylo dwiescie chlebów, i sto wiazanek rodzynków, i sto wiazanek fig, i lagiew wina.
\par 2 Tedy rzekl król do Syby: Na cóz to? I odpowiedzial Syba: Osly te dla czeladzi królewskiej, aby na nich jezdzila, a chleby i figi, aby jedli sludzy, a wino, aby pil, ktoby ustal na puszczy.
\par 3 I rzekl mu król: A gdziez jest syn pana twego? I odpowiedzial Syba królowi: Oto zostal w Jeruzalemie; albowiem mówil: Dzis mi przywróci dom Izraelski królestwo ojca mego.
\par 4 Zatem rzekl król do Syby: Oto twoje jest wszystko, cokolwiek mial Mefiboset. I rzekl Syba, poklon uczyniwszy: Niech znajde laske przed oczyma twemi, królu, panie mój.
\par 5 I przyszedl król Dawid az do Bahurym, a oto, stamtad maz wyszedl z rodu domu Saulowego, a imie jego bylo Semej, syn Giery; który wyszedlszy, idac zlorzeczyl.
\par 6 A ciskal kamienmi na Dawida, i na wszystkie slugi króla Dawida, choc wszystek lud, i wszystko rycerstwo szlo po prawej stronie jego, i po lewej stronie jego.
\par 7 I tak mówil Semej, zlorzeczac mu: Wynijdz, wynijdz mezu krwi, i mezu niezbozny.
\par 8 Obrócil na cie Pan wszystke krew domu Saulowego, na któregos miejscu królowal, a podal Pan królestwo w rece Absaloma, syna twego; a otos ty we zlem twojem, bos jest mezem krwi.
\par 9 I rzekl Abisaj, syn Sarwii, do króla: Czemuz zlorzeczy ten zdechly pies królowi, panu memu? Niech ide prosze, a utne glowe jego.
\par 10 Ale król rzekl: Cóz wam do tego, synowie Sarwii, ze zlorzeczy? Poniewaz mu Pan rzekl: Zlorzecz Dawidowi, i któzby smial rzec: Czemu tak czynisz?
\par 11 Nadto rzekl Dawid do Abisajego i do wszystkich slug swoich: Oto syn mój, który wyszedl z zywota mego, szuka duszy mojej, jakoz daleko wiecej teraz syn Jemini? Zaniechajcie go, niech zlorzeczy; boc mu Pan rozkazal.
\par 12 Snac wejrzy Pan na utrapienie moje, a odda mi Pan dobrem za zlorzeczenia jego dzisiejsze.
\par 13 A tak szedl Dawid, i mezowie jego droga, a Semej szedl strona góry przeciwko niemu, a idac zlorzeczyl, i ciskal kamienmi przeciw niemu, i miotal prochem.
\par 14 I przyszedl król ze wszystkim ludem, który byl przy nim spracowany, i tamze odpoczal.
\par 15 Lecz Absalom i wszystek lud mezów Izraelskich, przyszli do Jeruzalemu, takze i Achitofel z nim.
\par 16 A gdy szedl Chusaj Arachita, przyjaciel Dawida, do Absaloma, rzekl Chusaj do Absaloma: Niech zyje król, niech zyje król!
\par 17 Tedy rzekl Absalom do Chusaja: A takaz to milosc twoja ku przyjacielowi twemu? przeczzes nie szedl z przyjacielem twoim?
\par 18 Odpowiedzial Chusaj Absalomowi: Nie; ale którego obral Pan, i lud ten, i wszyscy mezowie Izraelscy, tego bede, i z nim zostane.
\par 19 Do tego, komuz ja bede sluzyl? izali nie synowi jego? Jakom sluzyl ojcu twemu, tak bede i tobie.
\par 20 Rzekl potem Absalom do Achitofela: Radzciez, co mam czynic?
\par 21 Odpowiedzial Achitofel Absalomowi: Wnijdz do zaloznic ojca twego, które zostaly, aby strzegly domu; a uslyszawszy wszystek Izrael, zes sie omierzyl ojcu twemu, zmocnia sie rece wszystkich, którzy sa z toba.
\par 22 Przetoz rozbili Absalomowi namiot na dachu. I szedl Absalom do zaloznic ojca swego przed oczyma wszystkiego Izraela.
\par 23 A rada Achitofelowa, która dawal, byla na on czas w takiej wadze, jakoby sie kto radzil Boga. Takowac byla wszelka rada Achitofelowa, jako u Dawida, tak u Absaloma.

\chapter{17}

\par 1 Nadto rzekl Achitofel do Absaloma: Niech prosze wybiore dwanascie tysiecy mezów, a wstawszy bede gonil Dawida tej nocy;
\par 2 I przypadne nan, pokad jest spracowany i zemdlonych rak, a strwoze go, i uciecze wszystek lud, który jest z nim, a zabije króla samego.
\par 3 A tak przywróce wszystek lud do ciebie; bo jakoby sie wszyscy ku tobie nawrócili, gdy zabije tego meza, którego ty szukasz, a wszystek sie lud uspokoi.
\par 4 I spodobalo sie to Absalomowi, i wszystkim starszym Izraelskim.
\par 5 Jednak rzekl Absalom: Zawolaj rychlo i Chusaja Arachity, abysmy uslyszeli, co on tez powie.
\par 6 A gdy przyszedl Chusaj do Absaloma, rzekl Absalom do niego, mówiac: Tak powiedzial Achitofel: Mamyli uczynic wedlug rady jego, czyli nie? i ty powiedz.
\par 7 Tedy odpowiedzial Chusaj Absalomowi: Niedobra jest rada, która teraz dal Achitofel.
\par 8 Nadto rzekl Chusaj: Swiadomys ojca twego i mezów jego, iz sa mezni, i serca zajuszonego, jako niedzwiedzica osierociala w polu; do tego ojciec twój jest maz waleczny, i nie bedzie nocowal z ludem.
\par 9 A podobno i teraz sie kryje w jakiej jaskini, albo na któremkolwiek miejscu. I staloby sie, jezlizeby kto z twoich polegl na tym poczatku, zeby kazdy, ktoby o tem uslyszal, rzekl: Stala sie porazka w ludzie, który szedl za Absalomem.
\par 10 Tedy i najmezniejszy, którego serce jako serce lwie, bardzo oslabieje; bo wie wszystek Izrael, ze meznym jest ojciec twój, i mezni wszyscy, którzy sa z nim.
\par 11 Alec radze, aby sie do ciebie cale zebral wszystek Izrael od Dan az do Beerseba, jako piasek, który jest przy morzu w mnóstwie, a ty zebys osoba swoja szedl na wojne.
\par 12 A tak pociagniemy przeciwko niemu, na któremkolwiek miejscu znaleziony bedzie, i przypadniemy nan, jako pada rosa na ziemie, i nie zostanie z niego, to jest, z tych wszystkich mezów, którzy sa z nim, ani jeden.
\par 13 A jezlizby do którego miasta uszedl, tedy zniesie wszystek Izrael do onego miasta powrozy, a pociagniemy je az do potoku, tak iz tam nie bedzie znalezion ani kamyk.
\par 14 Tedy rzekl Absalom i wszyscy mezowie Izraelscy: Lepsza jest rada Chusajego Arachity, niz rada Achitofelowa. Albowiem Pan byl postanowil, aby rozerwana byla rada Achitofelowa, która byla dobra, a tak aby przywiódl Pan zle na Absaloma.
\par 15 I oznajmil Chusaj Sadokowi i Abijatarowi, kaplanom: Tak a tak radzil Achitofel Absalomowi, i starszym Izraelskim; alem ja tak a tak radzil.
\par 16 Teraz tedy poslijcie co rychlej, a oznajmijcie Dawidowi, mówiac: Nie zostawaj tej nocy w równinach puszczy; ale bez odwloki przejdz, by snac nie byl pozarty król, i wszystek lud, który jest z nim.
\par 17 A Jonatan i Achimmas stali u studni Rogiel: i poszla dziewka, a oznajmila im, aby poszli, i doniesli to królowi Dawidowi; bo sie nie smieli ukazac, ani wnijsc do miasta.
\par 18 Wszakze obaczyl je niektóry sluga i powiedzial Absalomowi. Przetoz poszedlszy obadwaj spieszno, weszli w dom niektórego meza w Bahurym, który mial studnie na dworze swym, i spuscili sie do niej.
\par 19 A wziawszy niewiasta plachte, rozciagnela ja na wierzchu studni, i nasypala na niej krup; a tak sie tego nie dowiedziano.
\par 20 Bo gdy przyszli sludzy Absalomowi do onej niewiasty w dom, rzekli: Gdzie jest Achimaas i Jonatan? odpowiedziala im niewiasta: Przeszli przez rzeke; a poszukawszy ich, i nie znalazlszy, wrócili do Jeruzalemu.
\par 21 A gdy oni odeszli, tedy owi wystapiwszy z studni poszli, i oznajmili królowi Dawidowi, mówiac do niego: Wstancie, przeprawcie sie co rychlej przez wode; albowiem tak radzil przeciwko wam Achitofel.
\par 22 Przetoz wstawszy Dawid, i wszystek lud, który byl z nim, przeprawili sie przez Jordan, pierwej niz sie rozednialo, a nie zostal i jeden, któryby sie nie przeprawil przez Jordan.
\par 23 Tedy Achitofel widzac, iz sie nie stalo podlug rady jego, osiodlal osla, a wstawszy jechal do domu swego, do miasta swego, a rozprawiwszy dom swój, powiesil sie, i umarl, a pogrzebion jest w grobie ojca swego.
\par 24 A Dawid juz byl przyszedl do Mahanaim, gdy Absalom przeprawil sie przez Jordan, on i wszyscy mezowie Izraelscy z nim.
\par 25 I przelozyl Absalom Amaze, miasto Joaba, nad wojskiem. A ten Amaza byl synem meza, którego imie bylo Itra, Izraelczyk, który byl wszedl do Abigajli, córki Nahasowej, siostry Sarwii, matki Joabowej.
\par 26 I polozyl sie obozem Izrael z Absalomem na ziemi Galaad.
\par 27 I stalo sie, gdy przyszedl Dawid do Mahanaim, ze Soby syn Nahasowy z Rabby, synów Ammonowych, i Machir, syn Ammijelowy z Lodebaru, i Barsylaj Galaadczyk z Rogielim,
\par 28 Posciel, i miednice, i naczynia zdunskie, i pszenice, i jeczmien, i maki, i krupy, i boby, i soczewice, i prazma,
\par 29 I miodu, i masla, i owiec, i serów krowich przyniesli Dawidowi, i ludowi, który byl z nim, aby jedli: bo mówili: Lud ten glodny jest, i spracowany, i pragnieniem zmorzony na puszczy.

\chapter{18}

\par 1 Tedy obliczyl Dawid lud, który mial z soba, a postanowil nad nimi hetmany, i rotmistrze.
\par 2 I poruczyl Dawid ludu trzecia czesc pod reke Joabowe, a trzecia czesc pod reke Abisaja, syna Sarwii, brata Joabowego, a trzecia czesc pod reke Itaja Gietejczyka; i rzekl król do ludu: Wynijde i ja takze z wami.
\par 3 Ale lud rzekl: Nie wynijdziesz; bo jezlibysmy my tyl podali, oni malo dbac o nas beda, choc by tez nas polegla polowa, malo dbac o nas beda; albowiemes ty sam jako nas dziesiec tysiecy. Przetoz teraz lepiej, abys nam byl w miescie na pomocy.
\par 4 I rzekl do nich król: Co sie wam zda dobrego, to uczynie. Tedy stal król przy bramie, a wszystek lud wychodzil po stu i po tysiacu.
\par 5 I rozkazal król Joabowi, i Abisajowi, i Itajowi, mówiac: Laskawie mi sie obchodzcie z synem moim Absalomem. A wszystek lud slyszl, gdy przykazywal król wszystkim hetmanom o Absalomie.
\par 6 A tak wyciagnal lud w pole przeciw Izraelowi, i zwiedli bitwe w lesie Efraim.
\par 7 Tamze porazon jest lud Izraelski od slug Dawidowych; i stala sie tam porazka wielka dnia onego, a poleglo ich dwadziescia tysiecy.
\par 8 Bo gdy byla bitwa rozproszona po wszystkiej ziemi, wiecej las pogubil ludu, niz ich miecz pozarl dnia onego.
\par 9 I napadl Absalom na slugi Dawidowe; a Absalom jechal na mule, i wbiezal z nim mul pod gesty a wielki dab, i uwiezla glowa jego na debie, i zawisl miedzy niebem i miedzy ziemia; ale mul, który byl pod nim, wybiegl.
\par 10 Co ujrzawszy maz niektóry, oznajmil Joabowi, mówiac: Otom widzal Absaloma wiszacego na debie.
\par 11 Tedy rzekl Joab mezowi, który mu to oznajmil: Jezlis widzial, a czmuzes go tam nie zabil i nie zrzucil na ziemie? A ja bym ci byl powinien dac dziesiec srebników i jeden pas rycerski.
\par 12 I odpowiedzial on maz Joabowi: A ja chocbym mial odwazonych na rekach mych tysiac srebników, nie podnióslbym reki mojej na syna królewskiego; bosmy slyszeli, gdy przykazal król tobie i Abisajowi i Itajowi, mówiac: Ochraniajcie wszyscy syna mego Absaloma.
\par 13 Chyba, zebym chcial wdac dusze moje w niebezpieczenstwo; bo nie bywa nic zatajono przed królem; i ty sam bylbys przeciwko mnie.
\par 14 Tedy rzekl Joab: Nie bedec sie ja tu bawil z toba; przetoz wziawszy trzy drzewca w reke swoje, wrazil je w serce Absalomowe, gdy jeszcze zyw byl na debie.
\par 15 A obskoczywszy Absaloma dziesiec slug, którzy nosili bron Joabowe, bili, i zbili go.
\par 16 Wtem zatrabil Joab w trabe, i wrócil sie lud z pogoni za Izraelem; bo Joab zatrzymal lud.
\par 17 A wziawszy Absaloma wrzucili go w tymze lesie w dól wielki, i nanosili nan bardzo wielka kupe kamienia. Ale wszystek Izrael uciekl, kazdy do namiotów swoich.
\par 18 A Absalom wzial byl, i wystawil sobie za zywota swego slup, który jest w dolinie królewskiej; bo mówil: Niemam syna; jednak zostawie pamiatke imienia mego. Przetoz nazwal on slup imieniem swojem, który zowia miejsce Absalomowe az do dzisiejszego dnia.
\par 19 Tedy Achimaas, syn Sadoka, rzekl: Prosze niech ide a oznajmie królowi nowine, iz go wybawil Pan z reki nieprzyjaciól jego.
\par 20 Ale mu rzekl Jaob: Nie bylbys wdziecznym poslem dzisiaj; lecz to opowiesz dnia drugiego, a dzis nie dawaj o tem znac, przeto iz syn królewski zginal.
\par 21 Potem Joab rzekl do Chusego: Idz oznajmij królowi, cos widzial. A tak ukloniwszy sie Chusy Joabowi, biezal.
\par 22 I mówil powtóre Achimaas, syn Sadoka, i rzekl do Joaba: Badz co badz, prosze niech i ja bieze za Chusym. I rzekl Joab: Przeczbys ty mial biezec synu mój, gdyz niemasz, cobys dobrego zwiastowal?
\par 23 I rzekl: Badz co badz, pobieze. I rzekl mu Joab: Biezze. A tak biezal Achimaas prostsza droga, i uprzedzil Chusego.
\par 24 A Dawid siedzial miedzy dwiema bramami. I wyszedl stróz na dach bramy na mur, a podnióslszy oczy swe, ujrzal meza jednego biezacego.
\par 25 Tedy zawolawszy stróz, opowiedzial to królowi. I rzekl król: Jezlizec sam jest, dobre poselstwo w ustach jego. A gdy ten spiesznie szedl, i przyblizal sie,
\par 26 Ujrzal stróz i drugiego meza biezacego, i zawolal stróz na wrotnego, mówiac: Oto i drugi maz biezy sam. I rzekl król: I ten dobre poselstwo niesie.
\par 27 Nadto rzekl stróz: Zda mi sie bieg pierwszego, jako bieg Achimaasa, syna Sadokowego. I rzekl król: Maz to dobry, i z dobrem poselstwem idzie.
\par 28 Tedy zawolal Achimaas, i rzekl do króla: Pokój; i uklonil sie królowi twarza swoja ku ziemi i rzekl: Blogoslawiony Pan Bóg twój, któryc podal te meze, co podniesli rece swe przeciw królowi, panu memu.
\par 29 I rzekl król: Jakoli sie ma syn mój Absalom? Tedy Achimaas odpowiedzial: Widzialem zamieszanie wielkie, gdy posylal sluge królewskiego Joab, i mnie, sluge twego; ale nie wiem co bylo.
\par 30 Potem rzekl król: Odstap, a stan tam; a on odstapiwszy stanal.
\par 31 A wtem Chusy przyszedl i rzekl: Opowiada sie królowi, panu memu, ze cie wybawil Pan dzisiaj z reki wszystkich, którzy powstali przeciwko tobie.
\par 32 I rzekl król do Chusego: A jakoli sie ma syn mój Absalom? Odpowiedzial Chusy: Bodaj tak byli nieprzyjaciele króla, pana mego, i wszyscy, którzy powstawaja przeciw tobie na zle, jako syn twój!
\par 33 Tedy sie zasmucil król, i wstapil na sale onej bramy, a plakal, i tak mówil idac: Synu mój Absalomie, synu mój! Synu mój Absalomie! obym ja byl umarl miasto ciebie! Absalomie, synu mój, synu mój!

\chapter{19}

\par 1 I oznajmino Joabowi: Oto król placze i zaluje Absaloma.
\par 2 Przetoz sie ono zwyciestwo dnia onego obrócilo w placz wszystkiemu ludowi; albowiem uslyszawszy lud dnia onego, ze mówiono: Zalosny jest król dla syna swego.
\par 3 Zaczem wkradl sie lud onego dnia wchodzac do miasta, jako sie wiec wkrada lud, który sie wstydzi, uciekajac z bitwy.
\par 4 A król nakrywszy oblicze swoje, wolal glosem wielkim: Synu mój Absalomie, Absalomie, synu mój, synu mój!
\par 5 Tedy wszedl Joab do króla w dom, i rzekl: Shanbiles dzis oblicze wszystkich slug twoich, którzy wybawili dusze twoje dzisiaj, i dusze synów twoich, i córek twoich, i dusze zon twoich, i dusze naloznic twoich.
\par 6 Milujac te, którzy cie maja w nienawisci, a nienawidzac tych, którzy cie miluja. Albowiem pokazales dzis, ze sobie nie powazasz hetmanów, i slug twoich; bom doznal tego dzis, ze gdyby Absalom byl zyw, chocbysmy my wszyscy dzis byli pobici, tedyc by sie to bardzo podobalo.
\par 7 Przetoz teraz wstan, wynijdz, a mów lagodnie do slug twoich. Boc przez Pana przysiegam, jezli ty nie wynijdziesz, ze nie zostanie zaden z toba tej nocy, a bedziec to gorzej, nizli wszystko zle, którekolwiek na cie przychodzilo od mlodosci twojej az dotad.
\par 8 Wstal tedy król, i siadl w bramie; i opowiedziano to wszystkiemu ludowi, mówiac: Oto król siedzi w bramie. I przyszedl wszystek lud przed oblicze królewskie; ale Izraelczycy uciekli byli, kazdy do namiotu swego.
\par 9 I stalo sie, ze sie wszystek lud sprzeczal z soba we wszystkich pokoleniach Izraelskich, mówiac: Król wyrwal nas z rak nieprzyjaciól naszych, tenze nas tez wyrwal z rak Filistynskich, a teraz uciekl z ziemi przed Absalomem.
\par 10 Lecz Absalom, któregosmy byli pomazali nad soba, zginal w bitwie; a teraz przeczze wy zaniedbywacie przyprowadzic zasie króla?
\par 11 Przetoz król Dawid poslal do Sadoka i do Abijatara, kaplanów, z temi slowy: Powiedzcie starszym Judzkim, mówiac: Przeczze macie byc posledniejszymi w przyprowadzeniu zasie króla do domu jego? Albowiem slowa wszystkiego Izraela dochodzily króla do domu jego.
\par 12 Braciascie moi, kosc moja, a cialoscie moje; przeczze tedy macie byc posledniejszymi w przywróceniu króla?
\par 13 Amazie takze powiedzcie: Izalis ty nie jest kosc moja, i cialo moje? To niech mi uczyni Bóg, i to niech przyczyni, jezli hetmanem wojska nie bedziesz przedemna po wszystkie dni, miasto Joaba.
\par 14 A tak naklonil serce wszystkich mezów Judzkich jako meza jednego, ze poslali do króla mówiac: Nawróc sie ty, i wszyscy sludzy twoi.
\par 15 Wrócil sie tedy król, i przyszedl az do Jordanu; a lud Judzki wyszedl byl do Galgal, aby zaszedl w drodze królowi, a przeprowadzil króla przez Jordan.
\par 16 Pospieszyl sie takze Semej, syn Giery, syna Jemini, który byl z Bachurym, i wyszedl z mezami Judzkimi przeciwko królowi Dawidowi.
\par 17 A bylo tysiac mezów z nim z Benjamitów; Syba takze, sluga domu Saulowego, i pietnascie synów jego, i dwadziescia slug jego z nim, i szczesliwie sie przeprawili za Jordan do króla.
\par 18 Przeprawili tez prom, aby przewieziono czeladz królewska, a izby uczyniono, coby mu sie najlepiej podobalo; a Semej, syn Giery, upadl przed królem, gdy sie przeprawil przez Jordan,
\par 19 I rzekl do króla: Nie przyczytaj mi, panie mój, nieprawosci, ani wspominaj, co lekkomyslnie uczynil sluga twój onegoz dnia, gdy wyszedl król, pan mój, z Jeruzalemu, aby to mial przypuszczac król do serca swego;
\par 20 Albowiem zna sluga twój, zem zgrzeszyl; a otom dzis przyszedl pierwej niz kto ze wszystkiego domu Józefowego, abym zajechal droge królowi, panu memu.
\par 21 Tedy odpowiedzial Abisaj, syn Sarwii, i rzekl: Izaz dla tego nie ma byc zabity Semej, ze zlorzeczyl pomazancowi Panskiemu?
\par 22 Ale mu rzekl Dawid: Cóz wam do tego, synowie Sarwii, zescie mi dzis przeciwnymi? Izali dzis ma byc zabity kto w Izraelu? Bo azaz nie wiem, zem ja dzis zostal królem nad Izraelem?
\par 23 I rzekl król do Semeja: Nie umrzesz; i przysiagl mu król.
\par 24 Mefiboset takze, wnuk Saula, wyjechal przeciw królowi; który ani obmyl nóg swoich, ani czesal brody swojej, ani pral szat swoich, ode dnia, którego byl wyszedl król az do dnia, którego sie wrócil w pokoju.
\par 25 I stalo sie, gdy zabiezal w Jeruzalemie królowi, rzekl mu król: przeczzes nie szedl ze mna, Mefibosecie?
\par 26 A on mu odpowiedzial: Królu, panie mój, zdradzil mie sluga mój; bo rzekl byl sluga twój, osiodlam sobie osla, zebym siadlszy nan, jechal z królem, gdyz jest chromy sluga twój.
\par 27 I oskarzyl sluge twego przed królem, panem moim; ale król, pan mój, jest jako Aniol Bozy; przetoz czyn, co dobrego jest w oczach twoich.
\par 28 Albowiem wszyscy z domu ojca mego bylismy godni smierci przed królem, panem moim, a przecies ty posadzil sluge twego miedzy tymi, którzy jadaja u stolu twego: I cóz jeszcze za sprawiedliwosc moja, abym sie mial wiecej uskarzac na króla?
\par 29 Rzekl mu tedy król: Cóz masz wiecej mówic w sprawie twojej? Juzem rzekl, ty i Syba, podzielcie sie majetnoscia.
\par 30 A Mefiboset rzekl do króla: I wszystko niech wezmie, gdy sie tylko wrócil król, pan mój, w pokoju do domu swego.
\par 31 Barsylaj tez Galaatczyk wyszedlszy z Rogielim, przeprawil sie z królem przez Jordan, aby go prowadzil za Jordan.
\par 32 A Barsylaj byl bardzo stary, majac osiemdziesiat lat, który podejmowal króla, póki mieszkal w Mahanaim; bo byl czlowiekiem bogatym bardzo.
\par 33 I rzekl król do Barsylajego: Pójdz ze mna, a bede cie chowal przy sobie w Jeruzalemie.
\par 34 Ale Barsylaj odpowiedzial królowi: Wielez jest dni lat zywota mego, zebym mial isc z królem do Jeruzalemu?
\par 35 Osiemdziesiat lat mi dzisiaj; izali moge rozeznac miedzy dobrem a zlem? izali poczuje smak sluga twój w tem, cobym jadl, albo cobym pil? izali sluchac moge wiecej glosu spiewaków i spiewaczek? a przeczzeby mial byc sluga twój jeszcze ciezarem królowi, panu memu?
\par 36 Jeszcze troche pójdzie sluga twój za Jordan z królem: bo czemuzby mi mial dawac król takowa nagrode?
\par 37 Niech sie wróci prosze sluga twój, abym umarl w miescie mojem, przy grobie ojca mego i matki mojej; ale oto sluga twój Chymham pójdzie z królem, panem moim: uczynze mu, co dobrego jest w oczach twoich.
\par 38 I rzekl król: Niechze ze mna idzie Chymham, a ja mu uczynie, co dobrego bedzie w oczach twoich; nadto, cokolwiek zadac bedziesz ode mnie, toc uczynie.
\par 39 A gdy sie przeprawil wszystek lud przez Jordan, król sie tez przeprawil. Tedy pocalowal król Barsylajego, i blogoslawil mu; który sie wrócil do miejsca swego.
\par 40 Potem przyszedl król do Galgal, przyszedl tez z nim Chymham. Wszystek tez lud Judzki prowadzil króla, takze i polowa ludu Izraelskiego.
\par 41 A oto, wszyscy mezowie Izraelscy, zszedlszy sie do króla, mówili do niego: Czemuz cie wykradli bracia nasi, mezowie Judzcy, i przeprowadzili króla i dom jego przez Jordan, i wszystkie meze Dawidowe z nim?
\par 42 I odpowiedzieli wszyscy mezowie Judzcy mezom Izraelskim: Przeto, iz nam powinny jest król. A przeczze sie gniewac macie o to? izali nam za to jesc król dawa, albo nam jakie dary rozdal?
\par 43 Tedy odpowiedzieli mezowie Izraelscy mezom Judzkim, i rzekli: Dziesiec kroc wiecej mamy do króla; przetoz i Dawid wiecej do nas nalezy, niz do was. Przeczzescie nas lekce powazali? Azazesmy my o to pierwej nie mówili, abysmy przywrócili króla swe go? Ale srozsza byla mowa mezów Judzkich, niz mowa mezów Izraelskich.

\chapter{20}

\par 1 Tedy sie tam pojawil maz niepobozny, którego zwano Seba, syn Bichry, maz Jemini. Ten zatrabil w trabe, i rzekl: Nie mamy my dzialu w Dawidzie, ani mamy dziedzictwa w synu Isajego; wróc sie kazdy do namiotów swoich, o Izraelu!
\par 2 A tak odstapili wszyscy mezowie Izraelscy od Dawida za Seba, synem Bichry; ale mezowie Judzcy trzymali sie króla swego, od Jordanu az do Jeruzalemu.
\par 3 I przyszedl Dawid do domu swego w Jeruzalemie; a wziawszy król dziesiec niewiast zaloznic, które byl zostawil, aby strzegly domu, oddal je pod straz, i zywil je, ale do nich nie wchodzil; i byly pod straza az do dnia smierci swojej, we wdowim stanie.
\par 4 Potem rzekl król do Amazy: Zbierz mi meze Judzkie za trzy dni; ty sie tez tu staw.
\par 5 A tak poszedl Amaza, aby zebral lud Judzki; lecz sie zabawil nad czas naznaczony, który mu byl naznaczyl.
\par 6 I rzekl Dawid do Abisajego: Teraz gorzej nam uczyni Seba, syn Bichry, niz Absalom; przetoz ty wezmij slugi pana twego, a gon go, by snac nie znalazl sobie miast obronnych, i nie uszedl z oczu naszych.
\par 7 Tedy wyszli z nim mezowie Joabowi, i Chertczycy i Feletczycy, i wszystko rycerstwo, a wyszli z Jeruzalemu w pogon za Seba, synem Bichry.
\par 8 A gdy byli u wielkiego kamienia, który jest w Gabaon, tedy im Amaza zabiezal. A Joab mial przepasana szate swa, w której chodzil, a na niej pas z mieczem przypasany do biódr swoich w pochwach swych, którego snadnie mógl dobyc, i zas schowac.
\par 9 I rzekl Joab do Amazy: Jakoz sie masz, bracie mój? I ujal reka prawa Joab Amaze za brode, jakoby go calowac mial.
\par 10 Ale Amaza nie postrzegl miecza, który byl w rece Joabowej: i przebil go nim pod piate zebro, i wylal trzewa jego na ziemie, a tak za jadna rana umarl. A Joab i Abisaj, brat jego, szli w pogon za Seba, synem Bichry.
\par 11 Tedy stanal jeden nad nim z slug Joabowych, i rzekl: Ktokolwiek jest zyczliwy Joabowi, a ktokolwiek trzyma z Dawidem, niech idzie za Joabem.
\par 12 Lecz Amaza walal sie w krwi w posród drogi. A widzac on maz, iz sie zastanawial wszystek lud nad nim, zwlekl Amaze z drogi na pole, i przyrzucil go szata, gdyz widzial, ze ktokolwiek szedl mimo niego, zastanawial sie.
\par 13 A gdy byl zwleczony z drogi, biezal kazdy maz za Joabem, goniac Sebe, syna Bichry.
\par 14 Który juz byl przeszedl przez wszystkie pokolenia Izraelskie, az do Abel i Betmaacha, ze wszystkimi Berymczykami, którzy sie tez byli zebrali, a szli za nim.
\par 15 A gdy sie tam sciagneli, oblegli go w Abeli Betmaacha, i usypali szance przeciw miastu, tak iz stali przed murem, a wszystek lud, który byl z Joabem, usilowal obalic mury.
\par 16 Wtem zawolala z miasta niektóra niewiasta madra: Sluchajcie, sluchajcie! rzeczcie prosze do Joaba: Przystap sam, a rozmówie sie z toba.
\par 17 Który gdy do niej przystapil, rzekla mu ona niewiasta: Tyzes jest Joab? I odpowiedzial: Jestem. Tedy mu rzekla: Sluchaj slów sluzebnicy twojej; i odpowiedzial: Slucham.
\par 18 Przetoz rzekla, mówiac: Powiadano przedtem, mówiac: Koniecznie pytac sie beda w Abelu, a tak sie wszystko sprawi.
\par 19 Jam jest jedno miasto z spokojnych i wiernych w Izraelu, a ty szukasz, abys zatracil miasto i matke w Izraelu; przeczze chcesz zburzyc dziedzictwo Panskie?
\par 20 I odpowiedzial jej Joab, mówiac: Niedaj, niedaj mi tego Boze, abym mial podwrócic i zburzyc je.
\par 21 Nie takci sie rzecz ma. Ale maz z góry Efraim, imieniem Seba, syn Bichry, podniósl reke swa przeciw królowi Dawidowi; wydajciez go samego, a odciagne od miasta. Zatem rzekla niewiasta do Jaoba: Oto glowe jego zrzuca do ciebie z muru.
\par 22 A tak sprawila to ona niewiasta u wszystkigo ludu madroscia swoja, ze sciawszy glowe Sebie, synowi Bichry, zrzucili ja do Jaoaba; który zatrabil w trabe, i rozeszli sie wszyscy od miasta, kazdy do namiotów swoich; Joab sie tez wrócil do króla do Jeruzalemu.
\par 23 I byl Joab hetmanem nad wszystkiem wojskiem Izraelskiem, a Banajas, syn Jojady, nad Chretczykami i nad Feletczykami.
\par 24 Adoram byl poborca, a Jozafat, syn Ahiluda, kanclerzem.
\par 25 Seja pisarzem, a Sadok i Abijatar byli kaplanami.
\par 26 Hira takze Jairtczyk byl ksiazeciem u Dawida.

\chapter{21}

\par 1 I byl glód za dni Dawidowych przez trzy lata, jednego roku po drugim. Tedy szukal Dawid oblicza Panskiego, któremu rzekl Pan: Dla Saula, i dla domu jego krwawego, przeto iz pomordowal Gabaonity.
\par 2 Przyzwal tedy król Gabaonitów, i rzekl do nich: (A ci Gabaonitowie nie byli z synów Izraelskich, ale z ostatków Amorejczyków, którym acz byli synowie Izraelscy przysiegli, wszakze je usilowal Saul wyplenic z gorliwosci swej dla synów Izraelskich i Judzkich.)
\par 3 I rzekl Dawid do Gabaonitów: Cóz wam mam uczynic? a czem was ublagac, abyscie blogoslawili dziedzictwu Panskiemu?
\par 4 I odpowiedzieli mu Gabaonitowie: Nie idzie nam o srebro ani o zloto z Saulem, i z domem jego, ani o to, zebysmy zabili kogo w Izraelu. A on rzekl: Cokolwiek rzeczecie, uczynie wam.
\par 5 Którzy rzekli do króla: Meza, który nas wygubil, i na tem byl, aby nas do szczetu wytracil, zeby nas nic nie zostalo we wszystkich granicach Izraelskich.
\par 6 Wydajcie nam siedmiu mezów z synów jego, a powiesimy je Panu w Gabaa Saula, niekiedy wybranego Panskiego. Tedy rzekl: Wydam.
\par 7 Lecz sfolgowal król Mefibosetowi, synowi Jonatana, syna Saulowego, dla przysiegi Panskiej, która byla miedzy nimi, miedzy Dawidem i miedzy Jonatanem, synem Saulowym.
\par 8 Ale wzial król dwóch synów Resfy, córki Ai, które porodzila Saulowi, Armoniego i Mefiboseta, i pieciu synów Micholi, córki Saulowej, które porodzila Adryjelowi, synowi Barsyla Meholatyckiego,
\par 9 I wydal je w rece Gabaonitów, i powiesili je na górze przed Panem. I umarli oni siedmiu pospolu, a pobici sa w pierwsze dni zniwa, na poczatku zniwa jeczmiennego.
\par 10 A wziawszy Resfa, córka Ai, wór, ropostarla go na skale, na poczatku zniwa, azby na nie kropil deszcz z nieba, i nie dopuszczala ptastwu powietrznemu, padac na nie we dnie, ani zwierzowi polnemu w nocy.
\par 11 Tedy oznajmiono Dawidowi, co uczynila Resfa, córka Ai, zaloznica Saulowa.
\par 12 Przetoz szedlszy Dawid wzial kosci Saulowe, i kosci Jonatana, syna jego od starszych Jabez Galaadskiego, którzy je byli ukradli z ulicy Betsanskiej, kedy je byli zawiesili Filistynowie onegoz dnia, gdy porazili Filistynowie Saula w Gielboe.
\par 13 A tak wzial stamtad kosci Saulowe i kosci Jonatana, syna jego; zebrano tez kosci powieszonych,
\par 14 I pogrzebli kosci Saulowe i Jonatana, syna jego, w ziemi Benjamina w Sela, grobie Cysa, ojca jego, a uczynili wszystko, co byl rozkazal król; a tak potem ublagany byl Bóg ziemi.
\par 15 I byla zasie wojna miedzy Filistynami i Izraelem; i ciagnal Dawid i sludzy jego z nim, a walczyl przeciwko Filistynom, tak, ze ustal Dawid.
\par 16 Tedy Jesbibenob, który byl z synów jednego olbrzyma, (a grot drzewca jego wazyl trzy sta syklów miedzi, a mial przepasany miecz nowy) umyslil byl zabic Dawida.
\par 17 Ale go ratowal Abisaj, syn Sarwii, a raniwszy Filistyna zabil go. Przetoz przysiegli mezowie Dawidowi, mówiac mu: Nie pójdziesz wiecej z nami na wojne, abys nie zgasil pochodni Izraelskiej.
\par 18 I stalo sie potem, ze byla znowu wojna w Gob z Filistynami, i zabil Sobochaj Husatycki Safa, który byl z synów tegoz olbrzyma.
\par 19 Byla tez jeszcze inna wojna w Gob z Filistynami, kedy zabil Elhana, syna Jaara Oregim, Betlehemczyk, brata Golijatowego z Giet, którego drzewce u wlóczni bylo jako nawój tkacki.
\par 20 Nadto jeszcze byla wojna w Giet, kedy byl maz wielkiego wzrostu, majac po szesc palców u rak swoich, i po szesc palców u nóg swoich, wszystkich dwadziescia i cztery; a ten tez byl synem tegoz olbrzyma.
\par 21 Ten gdy uragal Izraelowi, zabil go Jonatan, syn Samaa, brata Dawidowego.
\par 22 Ci czterej byli synowie jednego olbrzyma z Giet, a ci polegli od reki Dawidowej, i od reki slug jego.

\chapter{22}

\par 1 I mówil Dawid Panu slowa tej piesni w on dzien, gdy go wybawil Pan z rak wszystkich nieprzyjaciól jego i z reki Saulowej.
\par 2 I rzekl: Pan opoka moja i twierdza moja, i wybawiciel mój ze mna.
\par 3 Bóg, skala moja, w nim bede ufal, tarcz moja, róg zbawienia mego, podwyzszenie moje, i ucieczka moja, zbawiciel mój, który mie od gwaltu wybawia.
\par 4 Wzywalem Pana chwaly godnego, a od nieprzyjaciól moich bylem wybawiony.
\par 5 Albowiem ogarnely mie byly bolesci smierci, potoki niezboznych przestraszyly mie.
\par 6 Bolesci grobu ogarnely mie, zachwycily mie sidla smierci.
\par 7 W utrapieniu mojem wzywalem Pana, a do Boga mego wolalem, i wysluchal z kosciola swego glos mój, a wolanie moje przyszlo do uszów jego.
\par 8 Tedy sie wzruszyla, a zadrzala ziemia, a fundamenty nieba zatrzasnely, i wzruszyly sie dla gniewu jego.
\par 9 Wystapil dym z nózdrz jego, a ogien z ust jego pozerajacy; wegle rozpalily sie od niego.
\par 10 Naklonil niebios i zstapil, a ciemnosc byla pod nogami jego.
\par 11 I jezdzil na Cherubinach, i latal, i widzian jest na skrzydlach wiatrowych.
\par 12 Polozyl ciemnosc okolo siebie miasto przybytku, zgromadzenie wód z obloki niebieskimi.
\par 13 Od jasnosci oblicza jego rozpalily sie wegle ogniste.
\par 14 Zagrzmial Pan z nieba, a najwyzszy wydal glos swój.
\par 15 Wypuscil i strzaly, a rozproszyl je, i blyskawica potarl je.
\par 16 I okazaly sie glebokosci morskie, a odkryly sie grunty swiata na fukanie Panskie, na tchnienie Ducha z nózdrz jego.
\par 17 Poslawszy z wysokosci, przyjal mie, wyrwal mie z wód wielkich.
\par 18 Wybawil mie od nieprzyjaciela mego poteznego, od tych, którzy mie mieli w nienawisci, choc byli mocniejszymi nad mie.
\par 19 Uprzedzili mie w dzien utrapienia mego; ale Pan byl podpora moja.
\par 20 I wywiódl mie na przestrzenstwo; wybawil mie; bo mie sobie upodobal.
\par 21 Oddal mi Pan wedlug sprawiedliwosci mojej, wedlug czystosci rak moich oddal mi,
\par 22 Gdyzem strzegl dróg Panskich, anim niezboznie nie odstawal od Boga mego.
\par 23 Albowiem wszystkie sady jego sa przed obliczem mojem i ustawy jego, nie odstapilem od nich.
\par 24 A bedac doskonaly przed nim, wystrzegalem sie nieprawosci mojej.
\par 25 Przetoz oddal mi Pan wedlug sprawiedliwosci mojej, wedlug czystosci mojej przed oblicznoscia oczu swych.
\par 26 Z milosiernym milosiernie postepujesz, z mezem doskonalym doskonalym jestes.
\par 27 Z czystym czysty jestes, a z przewrotnym surowie sie obchodzisz.
\par 28 Ale wybawiasz lud ubogi, a oczy twoje przed wynioslymi opuszczasz.
\par 29 Tys zaiste pochodnia moja, o Panie, a Pan oswieci ciemnosci moje.
\par 30 Bo w tobie przebieglem wojsko, w Bogu moim przekroczylem mur.
\par 31 Droga Boza jest doskonala, wyrok Panski nader czysty, tarcza jest wszystkim, którzy w nim ufaja.
\par 32 Albowiem któz jest Bogiem oprócz Pana? a kto opoka oprócz Boga naszego?
\par 33 Bóg jest moca moja w wojsku, on czyni doskonala droge moje.
\par 34 Równa nogi moje z jeleniemi, na wysokich miejscach moich stawia mie.
\par 35 Cwiczy rece me do boju, tak ze krusze luk miedziany ramiony swemi.
\par 36 Albowiem dales mi tarcz zbawienia mego, a w cichosci twojej rozmnozyles mie.
\par 37 Rozszerzyles kroki moje podemna, tak iz sie nie zachwialy kostki moje.
\par 38 Gonilem nieprzyjacioly moje, i wytracilem je, a nie wrócilem sie, azem je wyplenil.
\par 39 I wyniszczylem je, i poprzebijalem je, tak iz nie powstana: upadli pod nogami mojemi.
\par 40 Tys mie przepasal moca ku bitwie, a powaliles pod mie powstajace przeciwko mnie.
\par 41 Nadto podales mi szyje nieprzyjaciól moich, którzy mie mieli w nienawisci, i wykorzenilem je.
\par 42 Pogladali, ale nie byl wybawiciel; wolali na Pana, ale ich nie wysluchal.
\par 43 I potarlem je jako proch ziemi, jako bloto na ulicach podeptawszy je, rozmiotalem je.
\par 44 Tys mie od sporu ludu mego wyrwal; zachowales mie, abym byl glowa narodów; lud, któregom nie znal, sluzy mi.
\par 45 Synowie obcy klamali mna, a skoro uslyszeli, byli mi posluszni.
\par 46 Synowie obcy opadali, a drzeli i w zamknieniu swem.
\par 47 Zyje Pan, i blogoslawiona skala moja; niechze bedzie wywyzszony Bóg, opoka zbawienia mego.
\par 48 Bóg jest, który mi dawa pomsty, a podbija narody pod mie.
\par 49 Który mie wywodzi od nieprzyjaciól moich, a nad tymi, którzy powstaja przeciwko mnie, wywyzszasz mie, od czlowieka niepoboznego wybawiasz mie.
\par 50 Przetoz bede cie wyznawal Panie miedzy narodami; a imieniowi twemu spiewac bede.
\par 51 On jest wieza zbawienia króla swego, a czyniacy milosierdzie nad pomazancem swoim Dawidem, i nad nasieniem jego az na wieki.

\chapter{23}

\par 1 A tec sa ostateczne slowa Dawidowe. Rzekl Dawid, syn Isajego, rzekl mówie maz, który byl zacnie wywyzszony, pomazaniec Boga Jakóbowego, i wdzieczny w piesniach Izraelskich;
\par 2 Duch Panski mówil przez mie, a slowa jego przechodzily przez jezyk mój.
\par 3 Mówil Bóg Izraelski do mnie, mówila skala Izraelska: Ten, który panowac bedzie nad ludem, bedzie sprawiedliwy, panowac bedzie w bojazni Bozej.
\par 4 Bedzie jako bywa swiatlosc poranna, gdy slonce rano bez obloków wschodzi, a jako od jasnosci po deszczu wyrasta ziele z ziemi.
\par 5 A choc nie taki jest dom mój przed Bogiem, jednak przymierze wieczne postanowil ze mna, utwierdzone we wszystkiem i obwarowane. A w temci jest wszystko zbawienie moje, i wszystka uciecha moja, aczkolwiek temu jeszcze wzrostu nie dawa.
\par 6 Ale nipobozni wszyscy beda jako ciern wyrwani, którego rekoma nie biora.
\par 7 Lecz kto sie go jedno chce dotknac, obwaruje sie zelazem i drzewem wlóczni, albo ogniem wypala go do szczetu na miejscu jego.
\par 8 Tec sa imiona mocarzów, które mial Dawid: Jozeb Basebet Tachmojczyk, najprzedniejszy miedzy trzema; który sie z uciecha rzucil na osm set ludu z wlócznia, aby je zabil w jednej potrzebie.
\par 9 A po nim byl Eleazar, syn Dodona, syna Ahohowego, miedzy trzema mocarzami, którzy byli z Dawidem; a sromotnie lzyli Filistyny, którzy sie byli zebrali ku bitwie, gdy byli odciagneli mezowie Izraelscy.
\par 10 Ten powstawszy bil Filistyny, tak iz ustala reka jego, i zdretwiala reka jego przy mieczu. Tedy sprawil Pan wielkie wybawienie dnia onego, tak, ze sie lud wrócil za nim, tylko aby korzysci zebral.
\par 11 A po nim byl Semma, syn Agi, Hararczyk; albowiem gdy sie byli Filistynowie zebrali do kupy, kedy byla czesc pola pelnego soczewicy, a lud inny byl uciekl przed Filistynami:
\par 12 Tedy stanawszy w posród onej czesci pola, bronil go, i pobil Filistyny. A tak sprawil Pan wielkie wybawienie.
\par 13 Wyszli tez oni trzej z trzydziestu przedniejszych, a przyszli we zniwa do Dawida, do jaskini Odollam, gdy sie wojsko Filistynskie bylo obozem polozylo w dolinie Refaim.
\par 14 A Dawid na ten czas byl na miejscu obronnem: straz tez Filistynska na ten czas byla w Betlehem.
\par 15 Tedy pragnal Dawid, i rzekl: O by mi sie kto dal napic wody z studni Betlehemskiej, która jest u bramy!
\par 16 Przetoz wpadli ci trzej mocarze do obozu Filistynskiego, i naczerpali wody z studni Betlehemskiej, która byla u bramy; która niesli, i przyniesli do Dawida. Ale jej on nie chcial pic, lecz ja wylal przed Panem.
\par 17 I rzekl: Nie daj mi tego Panie, abym to mial uczynic. Izali to nie krew mezów, którzy szli z niebezpieczenstwem dusz swoich? I nie chcial jej pic. Toc uczynili oni trzej mocarze.
\par 18 Takze Abisaj brat Joaba, syn Sarwii byl przedniejszym miedzy trzema. Ten podniósl wlócznia swa przeciwko trzema stom, i zabil je, i byl slawnym miedzy trzema.
\par 19 Z tych trzech bedac najslawniejszym, byl ich hetmanem; wszakze onych trzech pierwszych nie doszedl.
\par 20 Banajas tez syn Jojady, syn meza rycerskiego, zacny w swych sprawach, z Kabseel; ten zabil dwóch mocarzów Moabskich, tenze szedlszy zabil lwa w posrodku studni, we dni sniezne.
\par 21 Tenze zabil meza Egipczanina, meza na podziw wielkiego, który Egipczanin mial w reku wlócznia; a on szedl ku niemu z kijem, a wydarlszy wlócznia z reki Egipczanina, zabil go wlócznia jego.
\par 22 Toc uczynil Banajas, syn Jojady, który tez byl slawny miedzy onymi trzema mocarzami.
\par 23 Z tych trzydziestu byl najslawniejszym; wszakze onych trzech nie doszedl; i postawil go Dawid nad drabantami swoimi.
\par 24 Byl tez Asael, brat Joaba, miedzy trzydziestoma. A ci sa: Elkanan, syn Dodana, Betlehemczyk;
\par 25 Samma Harodczyk; Elika Harodczyk.
\par 26 Heles Faltyczyk; Hyra, syn Ikkiesa, Tekuitczyk;
\par 27 Abijezer Anatotczyk; Mobonaj Husatczyk;
\par 28 Selmon Ahohytczyk; Maharaj Netofatczyk;
\par 29 Heleb, syn Baany, Netofatczyk; Itaj, syn Rybajego, z Gabaad synów Benjaminowych;
\par 30 Banajas Faratonczyk; Haddaj od potoku Gaas;
\par 31 Abijalbon Arbatczyk; Asmewet Barchomczyk;
\par 32 Elijachba Salabonczyk; z synów Jassonowych Jonatana;
\par 33 Semma Hororczyk; Ahijam, syn Sarara, Ararytczyk;
\par 34 Elifelet, syn Achasbaja, syna Machatego; Elijam, syn Achitofela Gilonczyka.
\par 35 Hezraj Karmelczyk; Faraj Arbitczyk.
\par 36 Igal, syn Natana z Soby; Bani Gadczyk.
\par 37 Selek Ammonitczyk; Nacharaj Berotczyk, który nosil bron Joaba, syna Sarwii;
\par 38 Hira Jetrytczyk; Gareb Jetrytczyk.
\par 39 Uryjasz Hetejczyk. Owa wszystkich trzydziesci i siedm.

\chapter{24}

\par 1 Tedy sie znowu popedliwosc Panska zapalila na Izraela, gdy pobudzil szatan Dawida przeciwko nim mówiac: Idz, policz Izraela i Jude.
\par 2 I rzekl król do Joaba, hetmana wojska swego: Przbiez zaraz wszystkie pokolenia Izraelskie od Dan az do Beerseba, a policzcie lud, abym wiedzial poczet ludu.
\par 3 Lecz Joab rzekl do króla: Niech przymnozy Pan, Bóg twój, ludu, jako teraz jest tyle stokroc, aby na to oczy króla, pana mego patrzaly; ale król, pan mój, przeczze sie tego napiera?
\par 4 Wszakze przemoglo slowo królewskie Joaba i hetmany wojska. Przetoz wyszedl Joab, i hetmani wojska od oblicza królewskiego, aby policzyli lud Izraelski.
\par 5 A przeprawiwszy sie przez Jordan, polozyli sie obozem przy Aroer, po prawej stronie miasta, które jest w posród potoku Gad i przy Jazer.
\par 6 Potem przyszli do Galaad, i do ziemi dolnej Hadsy, a stamtad przyszli do Dan Jaan i w okól Sydonu.
\par 7 Potem przyszli ku twierdzy Tyrskiej, i do wszystkich miast Hewejskich i Chananejskich, skad wyszli na poludnie Judy do Beerseba.
\par 8 A obszedlszy wszystke ziemie, przyszli po dziewieciu miesiacach, i po dwudziestu dniach do Jeruzalemu.
\par 9 I oddal Joab poczet obliczonego ludu królowi. A bylo w Izraelu osm kroc sto tysiecy mezów rycerskich, godnych ku bojowi, a mezów Juda piec kroc sto tysiecy mezów.
\par 10 Potem uderzylo Dawida serce jego, gdy obliczyl lud, i rzekl Dawid do Pana: Zgrzeszylem bardzo, zem to uczynil: ale teraz o Panie! przenies prosze nieprawosc slugi twego, bomci bardzo glupio uczynil.
\par 11 A gdy wstal Dawid rano, oto slowo Panskie stalo sie do Gada proroka, Widzacego Dawidowego, mówiac:
\par 12 Idz, a powiedz Dawidowi: Tak mówi Pan: Trzyc rzeczy podawam, obierz sobie jedne z tych, abym ci uczynil.
\par 13 A tak przyszedl Gad do Dawida, i oznajmil mu, a rzekl mu: Albo przyjdzie na cie glód przez siedm lat w ziemi twojej, albo przez trzy miesiace bedziesz uciekal przed nieprzyjacioly twymi, a oni cie gonic beda, albo wiec przez trzy dni bedzie morowe powietrze w ziemi twojej; rozmysl ze sie predko, a obacz, co mam odpowiedziec temu, który mie poslal.
\par 14 I rzekl Dawid do Gada: Jestem bardzo scisniony. Niech prosze raczej wpadniemy w reke Panska, gdyz wielkie sa zlitowania jego; ale w reke ludzka niech nie wpadam.
\par 15 Tedy przepuscil Pan powietrze morowe na Izraela od poranku az do czasu naznaczonego, i umarlo z ludu od Dan az do Beerseba siedmdziesiat tysiecy mezów.
\par 16 A gdy wyciagnal Aniol reke swa na Jeruzalem, aby je wytracil, tedy sie uzalil Pan onego zlego, i rzekl do Aniola, który tracil lud: Dosyc teraz; zawsciagnij reke twa. A Aniol Panski byl podle bojewiska Arawny Jebuzejczyka.
\par 17 I rzekl Dawid do Pana, gdy ujrzal Aniola bijacego lud, mówiac: Otom ja zgrzeszyl, jam zle uczynil; ale te owce cóz uczynily? niech sie prosze obróci reka twoja na mie i na dom ojca mego.
\par 18 Tedy przyszedl Gad do Dawida onegoz dnia, i rzekl mu: Idz, a zbuduj oltarz Panu na bojewisku Arawy Jebuzejczyka.
\par 19 I szedl Dawid podlug slowa Gadowego, jako byl rozkazal Pan.
\par 20 Tedy spojrzawszy Arawna, ujrzal króla, i slugi jego, przychodzace do siebie: i wyszedl Arawna, a poklonil sie królowi twarza swa ku ziemi.
\par 21 I rzekl Arawna: Przeczze przyszedl król, pan mój, do slugi swego? I odpowiedzial Dawid: Abym kupil u ciebie to bojewisko, i zbudowal na niem oltarz Panu, zeby zahamowana byla ta plaga miedzy ludem.
\par 22 Tedy rzekl Arawna do Dawida: Niech wezmie, a ofiaruje król, pan mój, co mu sie dobrego widzi: oto woly na calopalenie, i wozy, i jarzma wolów na drwa.
\par 23 Wszystko to dawal król Arawna królowi Dawidowi. I mówil Arawna do króla: Pan, Bóg twój, niech cie sobie upodoba.
\par 24 Lecz król rzekl do Arawny: Nie tak, ale raczej kupie u ciebie i zaplace; ani bede ofiarowal Panu, Bogu memu, calopalenia darmo danego. A tak kupil Dawid ono bojewisko i woly za piecdziesiat syklów srebra.
\par 25 Tamze zbudowal Dawid oltarz Panu, i sprawowal calopalenia, i spokojne ofiary. I ublagany byl Pan ziemi, a zahamowana jest ona plaga od Izraela.


\end{document}