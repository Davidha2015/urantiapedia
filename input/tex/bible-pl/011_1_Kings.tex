\begin{document}

\title{1 Królewska}


\chapter{1}

\par 1 A gdy sie król Dawid zstarzal, i zaszedl w lata, chodz go odziewano szatami, przecie sie zagrzac nie mógl.
\par 2 I rzekli mu sludzy jego: Niech poszukaja królowi, panu naszemu mlodej panienki, któraby stawala przed królem, i opatrowala go, a sypiajac na lonie jego, zeby zagrzewala króla, pana naszego.
\par 3 Szukali tedy panienki pieknej po wszystkich granicach Izraelskich, i znalezli Abisag Sunamitke, a przywiedli ja do króla.
\par 4 A ta panienka byla bardzo piekna, i opatrowala króla, i sluzyla mu; ale jej król nie uznal.
\par 5 Lecz Adonijasz, syn Haggity, wynosil sie, mówiac: Ja bede królowal. I nasprawial sobie wozów i jezdnych, i piecdziesiat mezów, którzy biegali przed nim.
\par 6 A nie gromil go nigdy ojciec jego, mówiac: Przeczzes to uczynil? A byl i ten bardzo pieknej urody, którego byla porodzila Haggita po Absalomie.
\par 7 A mial zmowe z Joabem, synem Sarwii, i z Abijatarem kaplanem, którzy pomagali za Adonijaszem.
\par 8 Ale Sadok kaplan, i Banajas, syn Jojady, i Natan prorok, i Semej, i Rehy i rycerstwo Dawidowe, nie przestawali z Adonijaszem.
\par 9 Tedy nabil Adonijasz owiec, i wolów, i bydla tlustego u kamienia Zohelet, który byl nad zródlem Rogiel, i wezwal wszystkich braci swych, synów królewskich, i wszystkich mezów z Juda, slug królewskich.
\par 10 Ale Natana proroka, i Banajasa, i innego rycerstwa, ani Salomona, brata swego, nie wezwal.
\par 11 Tedy rzekl Natan do Betsaby, matki Salomonowej, mówiac: A nie slyszalas, iz króluje Adonijasz, syn Haggity, a Dawid, pan nasz, nie wie o tem?
\par 12 Przetoz teraz pójdz prosze, dam ci rade, a zachowasz zdrowie twoje i zdrowie syna twego Salomona.
\par 13 Idz, a wnijdz do króla Dawida, i mów do niego: Izalis ty królu, panie mój, nieprzysiagl sluzebnicy twojej, mówiac: Salomon, syn twój, bedzie królowal po mnie, a on bedzie siedzial na stolicy mojej? Przeczze tedy króluje Adonijasz?
\par 14 A gdy ty jeszcze tam bedziesz mówila z królem, ja przyjde za toba, i dopelnie slów twoich.
\par 15 A tak weszla Betsaba do króla na pokój; a król sie juz byl bardzo zstarzal, a Abisag Sunamitka poslugowala królowi.
\par 16 Tedy nachyliwszy sie Betsaba, poklonila sie królowi, której rzekl król: Czego chcesz?
\par 17 A ona mu odpowiedziala: Panie mój, tys przysiagl przez Pana, Boga twego, sluzebnicy twojej, ze Salomon, syn twój, bedzie królowal po mnie, a on bedzie siedzial na stolicy mojej.
\par 18 A oto, juz Adonijasz króluje, a ty teraz, królu, panie mój, o tem nie wiesz.
\par 19 Albowiem nabil wolów i bydla tlustego, i owiec bardzo wiele, i wezwal wszystkich synów królewskich, i Abijatara kaplana, i Joaba, hetmana wojska; ale Salomona, slugi twego, nie wezwal.
\par 20 Lecz ty królu, panie mój, wiesz, iz sie oczy wszystkiego Izraela ogladaja na cie, abys im oznajmil, kto bedzie siedzial na stolicy króla, pana mego, po tobie.
\par 21 Inaczej stanie sie, gdy zasnie król, pan mój, z ojcy swymi, ze bedziemy ja i Salomon, syn mój, jako grzesznicy.
\par 22 A oto, gdy ona jeszcze mówila z królem, przyszedl Natan prorok.
\par 23 I opowiedziano to królowi, mówiac: Oto Natan, prorok; który wszedlszy do króla, poklonil sie królowi twarza swa ku ziemi.
\par 24 Rzekl zatem Natan: Królu, panie mój, zazes ty rzekl: Adonijasz bedzie królowal po mnie, a on usiedzie na stolicy mojej?
\par 25 Albowiem dzis szedlszy nabil wolów i bydla tlustego, i owiec bardzo wiele, i wezwal wszystkich synów królewskich, i hetmanów wojsk, i Abijatara kaplana, a oto oni jedza z nim i pija, i mówia: Niech zyje król Adonijasz!
\par 26 Ale mnie, slugi twego, i Sadoka kaplana, i Banajasa, syna Jojadowego, i Salomona, slugi twego, nie wezwal.
\par 27 Izali od króla, pana mego, stala sie ta rzecz? a nie oznajmiles sludze twemu, kto ma siedziec na stolicy króla, pana mego, po nim?
\par 28 I odpowiedzial król Dawid, mówiac: Zawolajcie do mnie Betsaby; która przyszedlszy przed oblicznosc królewska, stanela przed królem.
\par 29 Tedy przysiagl król, mówiac: Jako zywy Pan, który wybawil dusze moje z kazdego ucisku:
\par 30 Iz jakom ci przysiagl przez Pana, Boga Izraelskiego, mówiac: Ze Salomon, syn twój, królowac bedzie po mnie, a on usiedzie na stolicy mojej miasto mnie, tak dzis uczynie.
\par 31 I nachyliwszy sie Betsaba twarza ku ziemi, uklonila sie królowi, i rzekla: Niech zyje Dawid król, pan mój, na wieki!
\par 32 Zatem rzekl król Dawid: Zawolajcie do mnie Sadoka kaplana, i Natana proroka, i Banajasa, syna Jojadowego. I weszli do króla.
\par 33 I rzekl im król: Wezmijcie z soba slugi pana waszego, a wsadzcie Salomona, syna mego, na mulice moje, i prowadzcie go do Gihonu.
\par 34 A tam go pomaze Sadok kaplan, i Natan prorok za króla nad Izraelem; potem zatrabicie w trabe, a rzeczecie: Niech zyje król Salomon!
\par 35 Potem pójdziecie za nim; a on przyszedlszy, siadzie na stolicy mojej, i bedzie królowal miasto mnie; bom mu ja rozkazal, aby byl wodzem nad Izraelem i nad Juda.
\par 36 I odpowiedzial Banajas, syn Jojada, królowi, mówiac: Amen. Niech to stwierdzi Pan, Bóg króla, pana mego.
\par 37 Jako byl Pan z królem, panem moim, tak niech bedzie z Salomonem, a niechaj wywyzszy stolice jego nad stolice Dawida króla, pana mego.
\par 38 A tak szedl Sadok kaplan, i Natan prorok, i Banajas, syn Jojada, przytem Cheretczycy, i Feletczycy, i wsadzili Salomona na mulice króla Dawida, a prowadzili go do Gihonu.
\par 39 Tedy wzial Sadok kaplan róg olejku z namiotu, i pomazal Salomona. Potem trabili w trabe, i zakrzyknal wszystek lud: Niech zyje król Salomon!
\par 40 I szedl wszystek lud za nim. Tenze lud gral na piszczalkach, weselac sie weselem wielkiem, tak iz drzala ziemia od glosu ich.
\par 41 Co gdy uslyszal Adonijasz, i wszyscy wezwani, którzy byli z nim, (a juz sie tez byla dokonczyla uczta,)slyszac tez i Joab glos traby, rzekl: Cóz to za krzyk miasta huczacego?
\par 42 A gdy on tego domawial, oto Jonatan, syn Abijatara kaplana, przyszedl. Któremu rzekl Adonijasz: Wnijdz; bos ty maz stateczny, a powiesz nam co dobrego.
\par 43 Tedy odpowiedzial Jonatan, i rzekl Adonijaszowi: Dawid król, pan nasz, postanowil zapewne Salomona królem.
\par 44 Albowiem poslal z nim król Sadoka kaplana, i Natana proroka, i Banajasa, syna Jojadowego, do tego Cheretczyki i Feletczyki, którzy go wsadzili na mulice królewska;
\par 45 I pomazali go Sadok kaplan, i Natan prorok za króla w Gihonie, i szli stamtad weselac sie, tak, ze zadrzalo miasto; tenci jest krzyk, któryscie slyszeli;
\par 46 I juz usiadl Salomon na stolicy królestwa.
\par 47 Nadto i sludzy królewscy przyszli, aby blogoslawili Dawidowi królowi, panu naszemu, mówiac: Niech slawniejsze uczyni Bóg imie Salomonowe nad imie twoje, a niech wywyzszy stolice jego nad stolice twoje. I poklonil sie król na lozu swem.
\par 48 Przytem tak rzekl król: Blogoslawiony Pan, Bóg Izraelski, który dal dzis siedzacego na stolicy mojej, na co patrza oczy moje.
\par 49 Zlekli sie tedy, i wstali wszyscy wezwani, którzy byli z Adonijaszem, i poszli kazdy w droge swa.
\par 50 Adonijasz takze, bojac sie Salomona, wstal, i poszedl, a uchwycil sie rogów oltarza.
\par 51 I oznajmiono Salomonowi, mówiac: Oto Adonijasz boi sie króla Salomona, a oto uchwycil sie rogów oltarza, mówiac: Niech mi dzis przysieze król Salomon, ze nie zabije slugi swego mieczem.
\par 52 Tedy rzekl Salomon: Jezli bedzie mezem statecznym, nie spadnie i wlos z niego na ziemie; ale jezli sie w nim znajdzie co zlego, pewnie umrze.
\par 53 A tak poslal król Salomon, i przywiedziono go od oltarza; a gdy przyszedl, uklonil sie królowi Salomonowi, i rzekl mu Salomon: Idz do domu twego.

\chapter{2}

\par 1 A gdy sie przyblizal czas smierci Dawidowej, rozkazal Salomonowi, synowi swemu, mówiac:
\par 2 Ja ide w droge wszystkiej ziemi, a ty zmacniaj sie, i badz mezem.
\par 3 Zachowywaj ustawy Pana, Boga twego, abys chodzil drogami jego, i przestrzegal wyroków jego, i przykazan jego i sadów jego, i swiadectw jego, jako napisano w zakonie Mojzeszowym, abyc sie szczescilo wszystko, co sprawowac bedziesz, i we wszystkiem, do czego sie obrócisz:
\par 4 Zeby utwierdzil Pan slowo swoje, które rzekl do mnie, mówiac: Jezli beda strzedz synowie twoi drogi swej, chodzac przedemna w prawdzie, ze wszystkiego serca swego, i ze wszystkiej duszy swojej, tedy nie bedzie wytracony po tobie maz z stolicy Izrael skiej.
\par 5 A ty tez wiesz, co mi uczynil Joab syn Sarwii, co uczynil dwom hetmanom wojsk Izraelskich, Abnerowi, synowi Nerowemu, i Amazie, synowi Jeterowemu, ze je pozabijal, a wylal krew jako na wojnie czasu pokoju, i zmazal krwia jako na wojnie pas swój rycerski, który mial na biodrach swoich, i bóty swoje, które mial na nogach swoich.
\par 6 Uczynisz tedy wedlug madrosci twojej, a nie dopuscisz zejsc sedziwosci jego w pokoju do grobu.
\par 7 Ale nad synami Barsylai Galaadczyka uzyjesz milosierdzia, a niech jadaja wespól z inszymi u stolu twego; albowiem oni takimze sposobem przyszli do mnie, gdym uciekal przed Absalomem, bratem twoim.
\par 8 Oto tez jest u ciebie Semej, syn Giery, syna Jemini, z Bahurym, który mi tez zlorzeczyl zlorzeczeniem wielkiem w on dzien, gdym szedl do Mahanaim; a wszakze zaszedl mi droge u Jordanu, i przysiaglem mu przez Pana, mówiac: Nie zamorduje cie mieczem.
\par 9 Teraz jednak nie przepuszczaj mu tego, a izes jest mezem madrym, bedziesz wiedzial, co mu masz uczynic, abys wprowadzil sedziwosc jego ze krwia do grobu.
\par 10 Zatem zasnal Dawid z ojcy swoimi, a pogrzebiony jest w miescie Dawidowem.
\par 11 A dni, których królowal Dawid nad Izraelem, bylo czterdziesci lat. W Hebronie królowal siedm lat, a w Jeruzalemie królowal trzydziesci i trzy lata.
\par 12 A tak Salomon usiadl na stolicy Dawida ojca swego, i zmocnilo sie bardzo królestwo jego.
\par 13 Tedy przyszedl Adonijasz, syn Haggity, do Betsaby, matki Salomonowej, któremu ona rzekla: A spokojnez jest przyjscie twoje? A on odpowiedzial: Spokojne.
\par 14 Nadto rzekl: Mam nieco mówic z toba. A ona rzekla: Mów.
\par 15 Tedy rzekl: Ty wiesz, iz moje bylo królestwo, a na mie obrócili byli wszyscy Izraelczycy twarz swoje, abym królowal; ale przeniesione jest królestwo, i dostalo sie bratu memu; bo mu od Pana naznaczone bylo.
\par 16 Przetoz cie teraz prosze o jedne rzecz, a nie odmawiaj mi tego. A ona mu rzekla: Mów.
\par 17 Zatem on rzekl: Mów prosze do Salomona króla, (bo wiem, zec nie odmówi,)aby mi dal Abisag Sunamitke za zone.
\par 18 I odpowiedziala Betsaba: Dobrze; bede mówila o cie z królem.
\par 19 A tak szla Betsaba do króla Salomona, aby z nim mówila za Adonijaszem; i wstal król przeciwko niej, a pokloniwszy sie jej usiadl na stolicy swej; kazal tez postawic stolice matce swej, która siadla po prawicy jego.
\par 20 I rzekla: Prosze cie o jedne mala rzecz, nie odmawiaj mi. I odpowiedzial jej król: Pros matko moja; albowiem ci nie odmówie.
\par 21 Tedy rzekla: Niech bedzie dana Abisag Sunamitka Adonijaszowi, bratu twemu, za zone.
\par 22 Lecz odpowiedzial król Salomon, i rzekl matce swojej: Przeczze prosisz o Abisag Sunamitke Adonijaszowi? upros mu i królestwo, albowiem on jest bratem moim starszym nad mie, a ma po sobie Abijatara kaplana, i Joaba, syna Sarwii.
\par 23 I przysiagl król Salomon przez Pana, mówiac: To mi niech uczyni Bóg, i to niech przyczyni, ze przeciwko duszy swej mówil Adonijasz te slowa.
\par 24 A teraz jako zywy Pan, który mie utwierdzil, i posadzil na stolicy Dawida ojca mojego, i który mi zbudowal dom, jako obiecal, iz dzis zabity bedzie Adonijasz.
\par 25 A tak poslal król Salomon Banajasa, syna Jojadowego, który sie nan targnal, i zabil go.
\par 26 A do Abijatara kaplana rzekl król: Idz do Anatot, do osiadlosci twojej, albowiemes mezem smierci; wszakze cie dzis nie zabije, gdyzes nosil skrzynie Panska przed Dawidem, ojcem moim, a izes to wszystko cierpial, czem byl trapiony ojciec mój.
\par 27 I wyrzucil Salomon Abijatara, aby nie byl kaplanem Panskim, zeby sie wypelnilo slowo Panskie, które byl wyrzekl nad domem Heli w Sylo.
\par 28 Ta wiesc gdy przyszla do Joaba, (albowiem Joab przestawal z Adonijaszem, chociaz z Absalomem nie przestawal,)tedy uciekl Joab do namiotu Panskiego, a uchwycil sie rogów oltarza.
\par 29 I oznajmiono królowi Salomonowi, ze uciekl Joab do namiotu Panskiego, a ze jest u oltarza: tedy poslal Salomon Banajasa, syna Jojadowego, mówiac: Idz, zabij go.
\par 30 A przyszedlszy Banajas do namiotu Panskiego, rzekl do niego: Tak mówi król: Wynijdz. Który odpowiedzial: Nie wyjde, ale tu umre. I odniósl to Banajas królowi mówiac: Tak mówil Joab, i tak mi odpowiedzial.
\par 31 I rzekl mu król: Uczynze, jako mówil, a zabij go, i pogrzeb go, a odejmiesz krew niewinna, która wylal Joab, odemnie i od domu ojca mego.
\par 32 A obróci Pan krew jego na glowe jego: albowiem targnal sie na dwóch mezów sprawiedliwszych i lepszych nizli sam, i zabil je mieczem, a ojciec mój Dawid nie wiedzial o tem: Abnera, syna Nerowego, hetmana wojska Izraelskiego, i Amaze, syna Jeterowego, hetmana wojska Judzkiego.
\par 33 A tak wróci sie krew ich na glowe Joabowe, i na glowe nasienia jego na wieki; lecz Dawidowi i nasieniu jego, i domowi jego, i stolicy jego niech bedzie pokój az na wieki od Pana.
\par 34 Szedl tedy Banajas, syn Jojada, a rzuciwszy sie nan zabil go, i pogrzebiony jest w domu swym na puszczy.
\par 35 I postanowil król Banajasa syna Jojadowego, miasto niego nad wojskiem, a Sadoka kaplana postanowil król miasto Abijatara.
\par 36 Potem poslal król, i przyzwal Semejego, i rzekl mu: Zbuduj sobie dom w Jeruzalemie, i mieszkaj tam, a nie wychodz stamtad nigdzie;
\par 37 Bo któregobys dnia wyszedl a przyszedl za potok Cedron, wiedz wiedzac, ze pewnie umrzesz; krew twoja bedzie na glowe twoje.
\par 38 Tedy rzekl Semej do króla: Dobre jest to slowo; jako mówil król, pan mój, tak uczyni sluga twój, I mieszkal Semej w Jeruzalemie przez wiele dni.
\par 39 I stalo sie po trzech lat, ze uciekli dwaj sludzy Semejemu do Achisa syna Maachy, króla Gietskiego, i opowiedziano Semejemu, mówiac: Oto sludzy twoi sa w Giet.
\par 40 Przetoz wstawszy Semej, i osiodlawszy osla swego, jechal do Giet, do Achisa, aby szukal slug swoich; i wrócil sie Semej i przywiódl slugi swe z Giet.
\par 41 I oznajmiono Salomonowi, ze byl wyjechal Semej z Jeruzalemu do Giet, i zasie sie wrócil.
\par 42 Tedy poslal król, i wezwal Semejego, i rzekl mu: Izalim cie nie poprzysiagl przez Pana, a nie oswiadczylem sie przed toba, mówiac: Któregobyskolwiek dnia gdzie wyszedl, wiedz wiedzac, ze zapewne umrzesz? I mówiles do mnie: Dobre to slowo, którem slyszal.
\par 43 Przeczzes tedy nie strzegl przysiegi Panskiej i przykazania, którem ci byl przykazal?
\par 44 Nadto król rzekl do Semejego: Ty wiesz wszystko zle, którego swiadome jest serce twoje, cos uczynil Dawidowi, ojcu memu, i oddal Pan zlosc twoje na glowe twoje.
\par 45 Ale król Salomon blogoslawiony, stolica Dawidowa bedzie utwierdzona przed Panem az na wieki.
\par 46 A tak rozkazal król Banajasowi, synowi Jojadowemu, który wyszedlszy targnal sie nan, i zabil go. A tak utwierdzone jest królestwo w rece Salomonowej.

\chapter{3}

\par 1 I spowinowacil sie Salomon z Faraonem, królem Egipskim; bo pojal córke Faraonowa, i przyprowadzil ja do miasta Dawidowego, azby dobudowal domu swego, i domu Panskiego, i muru Jeruzalemskiego w okolo.
\par 2 Wszakze lud ofiarowal po górach, przeto, ze nie byl jeszcze zbudowany dom imieniowi Panskiemu az do onych dni.
\par 3 I milowal Salomon Pana, chodzac w przykazaniach Dawida, ojca swego; tylko ze na górach ofiarowal i kadzil.
\par 4 Szedl tedy król do Gabaon, aby tam ofiarowal; bo tam byla góra najwieksza; tysiac ofiar calopalonych ofiarowal Salomon na onym oltarzu.
\par 5 I ukazal sie Pan w Gabaon Salomonowi przez sen w nocy, i rzekl Bóg: Pros czego chcesz, a dam ci.
\par 6 Tedy rzekl Salomon: Tys uczynil z sluga twoim Dawidem, ojcem moim, milosierdzie wielkie, gdyz chodzil przed toba w prawdzie i w sprawiedliwosci, i w prostosci serca stal przy tobie; i zachowales mu to milosierdzie wielkie, izes mu dal syna, który by siedzial na stolicy jego, jako sie to dzis okazuje.
\par 7 A teraz, o Panie Boze mój, tys postanowil królem sluge twego miasto Dawida, ojca mego, a jam jest dziecie male, i nie umiem wychodzic ani wchodzic.
\par 8 A sluga twój jest w posrodku ludu twego, którys obral, ludu wielkiego, który nie moze zliczony ani porachowany byc przez mnóstwo.
\par 9 Przetoz daj sludze twemu serce rozumne, aby sadzil lud twój, i aby rozeznawal miedzy dobrem i zlem; albowiem któz moze sadzic ten lud twój tak wielki?
\par 10 I podobalo sie to Panu, ze zadal Salomon tej rzeczy.
\par 11 Tedy rzekl Bóg do niego: Dla tego, zes o to prosil, a niezadales sobie dlugich dni, anis zadal sobie bogactw, anis prosil o wytracenie nieprzyjaciól twoich, ales sobie prosil o rozum dla rozeznania sadu:
\par 12 Otozem uczynil wedlug slów twoich; otom ci dal serce madre i rozumne, tak, iz zaden równy tobie nie byl przed toba, ani po tobie powstanie równy tobie.
\par 13 Do tego i to, czegos nie zadal, dalem ci, to jest bogactwa i slawe, tak aby nie bylo zadnego tobie równego miedzy królmi po wszystkie dni twoje.
\par 14 A bedzieszli chodzil drogami mojemi, strzegac wyroków moich, i rozkazania mego, jako chodzil Dawid, ojciec twój, tedy przedluze dni twoje.
\par 15 A gdy sie ocucil Salomon, zrozumial, ze to byl sen. I przyszedl do Jeruzalemu, a stanawszy przed skrzynia przymierza Panskiego, sprawowal calopalenia, i ofiarowal ofiary spokojne, sprawil tez uczte na wszystkie slugi swoje.
\par 16 Tedy przyszly dwie niewiasty wszetecznice do króla, i stanely przed nim.
\par 17 I rzekla jedna z onych niewiast: Prosze panie mój, ja i ta niewiasta mieszkamy w jednym domu, i porodzilam u niej w tymze domu.
\par 18 I stalo sie dnia trzeciego po porodzeniu mojem, ze porodzila i ta niewiasta; i bylysmy pospolu, a nie bylo nikogo obcego z nami w domu, oprócz nas dwóch w tymze domu.
\par 19 I umarl syn tej niewiasty w nocy, przeto, iz go byla przylegla.
\par 20 A wstawszy o pólnocy, wziela syna mego odemnie, gdy sluzebnica twoja spala, i polozyla go na lonie swojem, a syna swego umarlego polozyla na lonie mojem.
\par 21 A gdym wstala rano, chcac dac ssac synowi memu, otom znalazla umarlego; któremu gdym sie rano przypatrzyla, a oto nie byl syn mój, któregom porodzila.
\par 22 I rzekla ona druga niewiasta: Nie tak; ale syn mój jest ten zywy, a syn twój ten umarly. Ale ona rzekla: Nie; ale syn twój jest ten umarly, a syn mój ten zywy. I tak sie spieraly przed królem.
\par 23 I rzekl król: Ta mówi: Ten zywy jest syn mój, a syn twój ten umarly; a ta zasie mówi: Nie tak; ale syn twój ten umarly, a syn mój ten zywy.
\par 24 Przetoz rzekl król: Przyniescie mi miecz. I przyniesiono miecz przed króla.
\par 25 Tedy rzekl król: Rozetnijcie to zywe dziecie na dwoje, a dajcie polowe jednej, a polowe drugiej.
\par 26 Ale niewiasta, której byl ten syn zywy, mówila do króla, (bo sie byly poruszyly wnetrznosci jej nad synem jej,)i rzekla: Prosze, panie mój, dajcie jej to dziecie zywe, a zadnym sposobem nie zabijajcie go. Ale druga rzekla: Niech nie bedzie ani mnie, ani tobie, rozetnijcie je.
\par 27 Tedy odpowiedzial król, i rzekl: Dajciez tej dziecie zywe, a zadna miara nie zabijajcie go; tac jest matka jego.
\par 28 A uslyszawszy wszystek lud Izraelski ten sad, który osadzil król, bali sie króla: albowiem wiedzieli, ze madrosc Boza byla w sercu jego ku czynieniu sadu.

\chapter{4}

\par 1 A tak król Salomon byl królem nad wszystkim Izraelem.
\par 2 A tec sa ksiazeta, które mial: Azaryjasz, syn Sadoka kaplana.
\par 3 Elichoref i Achija, synowie Sysy, byli pisarzami; Jozafat, syn Ahiluda, kanclerzem;
\par 4 A Banajas, syn Jojady, byl hetmanem; Sadok zas i Abijatar kaplanami.
\par 5 A Azaryjasz, syn Natana, nad urzednikami, a Zabud, syn Natana, byl ksiazeciem, przyjacielem królewskim.
\par 6 Ahisar zas byl przelozony nad domem, a Adoniram, syn Abdy, nad wybranym ludem.
\par 7 Mial tez Salomon dwanascie przelozonych nad wszystkim Izraelem, którzy dodawali zywnosci królowi i domowi jego. Kazdy z nich przez jeden miesiac w roku królowi zywnosci dodawal.
\par 8 A tec sa imiona ich: Syn Hura na górze Efraim;
\par 9 Syn Dekara w Makas, i w Salbim, i w Betsames, i w Elon i Bethanan;
\par 10 Syn Heseda w Arubot, który trzymal Socho i wszystke ziemie Chefer;
\par 11 Syn Abinadaba, który trzymal wszystkie granice Dor, a Tafet, córke Salomonowa, mial za zone.
\par 12 Baana, syn Ahiluda, który trzymal Tanach i Magieddo, i wszystko Betsan, które jest podle Sartany pod Jezreelem, od Betsan az do Abelmehola, az za Jekmaam.
\par 13 Syn Gaber w Ramot Galaadskiem, który trzymal wsi Jaira, syna Manasesowego, które leza w Galaad. Jemu tez nalezala kraina Argob, która jest w Basan, szescdziesiat miast wielkich murowanych z zaworami miedzianemi.
\par 14 Achinadab, syn Iddona, w Mahanaim.
\par 15 Achimaas w Neftalim, który tez pojal Basemate, córke Salomonowa, za zone.
\par 16 Baana, syn Husai, w Aser i w Alot.
\par 17 Jozafat, syn Paruacha, w Isaschar.
\par 18 Semej, syn Eli, w Benjamin.
\par 19 Gaber, syn Ury, w ziemi Galaad i w ziemi Sehona króla Amorejskiego, i Oga króla Basanskiego; a ten sam byl rzadca onej ziemi.
\par 20 Tedy Juda i Izrael bedac niezliczeni jako piasek, który jest nad morzem w mnóstwie, jedli, i pili, i weselili sie.
\par 21 A Salomon panowal nad wszystkiemi królestwy od rzeki az do ziemi Filistynskiej, i az do granicy Egipskiej. I przynosili dary, a sluzyli Salomonowi po wszystkie dni zywota jego.
\par 22 A tenci byl rozchód Salomona na kazdy dzien, trzydziesci korcy maki bialej, i szescdziesiat korcy innej maki.
\par 23 Dziesiec wolów karmnych, i dwadziescia wolów pastewnych, i sto owiec, oprócz jeleni i sarn, i bawolów, i ptastwa karmnego.
\par 24 Albowiem on panowal wszedy z tej strony rzeki od Tasfa az do Gazy nad wszystkimi królmi, którzy byli przed rzeka, a mial pokój ze wszystkich stron w okolo.
\par 25 I mieszkal Juda i Izrael bezpiecznie, kazdy pod winna macica swoja, i pod figa swoja, od Dan az do Beerseba, po wszystkie dni Salomonowe.
\par 26 Mial tez Salomon czterdziesci tysiecy koni na staniu do wozów swoich, a dwanascie tysiecy jezdnych.
\par 27 A tak podejmowali oni przelozeni króla Salomona, i wszystkie, którzy przychodzili do stolu króla Salomona, kazdy miesiaca swego, nie dopuszczajac, aby na czem schodzic mialo.
\par 28 Jeczmien takze, i plewy dla koni i mulów, zwozili na to miejsce, gdzie byl król, kazdy wedlug tego, jako mu postanowiono.
\par 29 Nadto dal Bóg Salomonowi madrosc i roztropnosc bardzo wielka, a przestronnosc serca, jako piasek, który jest na brzegu morskim.
\par 30 Albowiem wieksza byla madrosc Salomonowa, nizli madrosc wszystkich narodów wschodnich, i nizli wszelka madrosc Egipczanów.
\par 31 Owszem medrszym byl nad wszystkie ludzie, az i nad Etana Ezrahyte, i nad Hemana, i Chalkola, i Darda, syny Maholowe; a byl slawny u wszystkich narodów okolicznych.
\par 32 Nadto zlozyl trzy tysiace przypowiesci, a piesni jego bylo tysiac i piec.
\par 33 Rozprawial tez o drzewach, poczawszy od cedru, który jest na Libanie, az do hizopu, który wyrasta z sciany. Mówil tez o zwierzetach i o ptakach, i o gadzinach, i o rybach.
\par 34 Przetoz przychodzili ze wszystkich narodów sluchac madrosci Salomonowej, i od wszystkich królów ziemi, którzy slyszeli o madrosci jego.

\chapter{5}

\par 1 I poslal Hiram, król Tyrski, slugi swe do Salomona, bo uslyszal, ze go pomazano za króla miasto ojca jego; albowiem milowal Hiram Dawida po wszystkie dni.
\par 2 Salomon tez zas poslal do Hirama, mówiac:
\par 3 Ty wiesz, ze Dawid, ojciec mój, nie mógl budowac domu imieniowi Pana, Boga swego, dla wojen, które go byly ogarnely, az nieprzyjacioly podal Pan pod stopy nóg jego;
\par 4 Ale teraz Pan, Bóg mój, dal mi odpoczynienie zewszad, i niemam zadnego przeciwnika, ani zabiegu zlego.
\par 5 A otom umyslil budowac dom imieniowi Pana, Boga mego, jako powiedzial Pan do Dawida, ojca mego, mówiac: Syn twój, któremu dam miasto ciebie osiasc stolice twoje, ten zbuduje dom ten imieniowi memu.
\par 6 Przetoz teraz rozkaz, aby mi narabano ceder na Libanie, a sludzy moi beda z slugami twoimi, a zaplate slug twoich dam tobie tak, jako rzeczesz; albowiem ty wiesz, ze niemasz miedzy nami meza, któryby umial tak rabac drzewo, jako Sydonczycy.
\par 7 A gdy uslyszal Hiram slowa Salomonowe, uradowal sie bardzo i rzekl: Blogoslawiony Pan dzisiaj, który dal Dawidowi syna madrego nad tym ludem wielkim.
\par 8 I poslal Hiram do Salomona, mówiac: Slyszalem z czemes poslal do mnie. Uczynie wszystke wola twoje okolo drzewa cedrowego, i okolo drzewa jodlowego.
\par 9 Sludzy moi zwioza je z Libanu do morza, a ja je kaze zlozyc w tratwy, i spuscic morzem az do miejsca, o którem mi dasz znac, i tam je rozwiaze, a ty je pobierzesz. Ty tez uczynisz wola moje, a dasz obrok czeladzi mojej.
\par 10 A tak Hiram dodawal Salomonowi drzewa cedrowego, i drzewa jodlowego, jako wiele chcial.
\par 11 Salomon takze dawal Hiramowi dwadziescia tysiecy miar pszenicy na zywnosc czeladzi jego, i dwadziescia tysiecy miar oliwy wybijanej; to dawal Salomon Hiramowi na kazdy rok.
\par 12 A Pan dal madrosc Salomonowi, jako mu byl obiecal, i byl pokój miedzy Hiramem i miedzy Salomonem, a uczynili przymierze miedzy soba.
\par 13 Tedy kazal wybierac król Salomon robotniki ze wszystkiego Izraela, a bylo wybranych trzydziesci tysiecy mezów.
\par 14 Które slal do Libanu po dziesiec tysiecy na kazdy miesiac, na przemiany; po miesiacu mieszkali na Libanie, a po dwa miesiace w domu swym; a Adoniram byl przelozony nad tymi wybranymi.
\par 15 Mial tez Salomon siedmdziesiat tysiecy tych, którzy nosili ciezary, a osmdziesiat tysiecy tych, którzy rabali na górze;
\par 16 Oprócz przedniejszych urzedników Salomonowych, których bylo nad robota trzy tysiace i trzy sta, którzy byli przelozeni nad ludem odprawujacym robote.
\par 17 Rozkazal tez król, aby wozono kamienie wielkie, kamienie drogie, i kamienie ciosane, na zalozenie gruntów domu.
\par 18 Ciosali tedy rzemieslnicy Salomonowi, i rzemieslnicy Hiramowi, i Gimblimczycy. A tak gotowali drzewo i kamienie na budowanie domu.

\chapter{6}

\par 1 I stalo sie czterechsetnego i osmdziesiatego roku po wyjsciu synów Izraelskich z ziemi Egipskiej, miesiaca Kwietnia (ten jest miesiac wtóry,)roku czwartego królowania Salomonowego nad Izraelem, ze poczal budowac dom Panu.
\par 2 A ten dom, który budowal król Salomon Panu, byl wdluz na szescdziesiat lokci, a wszerz na dwadziescia, a wzwyz na trzydziesci lokci.
\par 3 Przysionek zasie przed kosciolem byl na dwadziescia lokci wdluz, jako byl szeroki dom, a wszerz byl na dziesiec lokci przed domem.
\par 4 I poczynil w domu okna wewnatrz przestronne, a z dworu waskie.
\par 5 I zbudowal przy murze koscielnym ganki wszedy w okolo, przy murze domu okolo kosciola i swiatnicy; uczynil tez gmachy w okolo.
\par 6 Ganek spodni byl na piec lokci wszerz, a sredni byl na szesc lokci wszerz, a trzeci byl na siedem lokci wszerz; bo byl ustepy uczynil okolo domu z nadworza, aby balki nie przechodzily do murów koscielnych.
\par 7 A gdy ten dom budowano, z kamienia wyrobionego, jakie przywozono, budowano go; a mlota, ani siekiery, ani zadnego naczynia zelaznego nie slychac bylo w domu, gdy go budowano.
\par 8 Drzwi do gmachu sredniego byly na prawej stronie domu, któremi po okraglych schodach wchodzono do sredniego, a z sredniego do trzeciego.
\par 9 A tak zbudowal on dom, i dokonczyl go, i nakryl go balkami na ksztalt zasklepienia, i deskami cedrowemi.
\par 10 Przybudowal tez ganki okolo calego domu na piec lokci wzwyz, a przypojone byly do domu balkami cedrowymi.
\par 11 I stalo sie slowo Panskie do Salomona, mówiac:
\par 12 Toc jest ten dom, który ty budujesz. Jezli bedziesz chodzil w ustach moich, i sady moje bedziesz czynil, i zachowywal wszystkie rozkazania moje, chodzac w nich, tedy utwierdze slowo moje z toba, którem wyrzekl do Dawida, ojca twego.
\par 13 I bede mieszkal w posrodku synów Izraelskich, a nie opuszcze ludu mego Izraelskiego.
\par 14 A tak zbudowal Salomon dom on, i dokonczyl go.
\par 15 I oblozyl mury domu wewnatrz deskami cedrowemi; od tla domu az do stropu okryl drzewem wewnatrz, a tlo domu polozyl tarcicami jodlowemi.
\par 16 Zbudowal tez przegrodzenie na dwadziescia lokci wdluz od strony do strony domu, z tarcic cedrowych od tla az do stropu. A tak zbudowal Bogu wewnatrz przybytek, aby byl swiatnica najswietsza.
\par 17 A na czterdziesci lokci byl sam dom, to jest, kosciól przed swiatnica.
\par 18 A na deskach cedrowych wewnatrz w domu bylo rzezanie naksztalt jablek lesnych, i kwiecia rozkwitlego, wszystko z cedru, tak, ze ani kamienia nie bylo widziec.
\par 19 A swiatnice najswietsza w domu wewnatrz nagotowal, aby tam postawiona byla skrzynia przymierza Panskiego.
\par 20 Która swiatnica najswietsza wewnatrz byla dwadziescia lokci wdluz, a dwadziescia lokci wszerz, i dwadziescia lokci wzwyz; a obil ja zlotem szczerem, oltarz takze cedrowy obil zlotem.
\par 21 A tak oblozyl Salomon dom on wewnatrz szczerem zlotem, i zaciagnal lancuchami zlotemi przegrodzenie przed swiatnica swietych, które tez oblozyl zlotem.
\par 22 Takze wszystek dom obil zlotem, nie opuszczajac zadnej strony, i caly oltarz, który byl przed swiatnica najswietsza, powlókl zlotem.
\par 23 Uczynil tez w swiatnicy najswietszej dwa Cherubiny z drzewa oliwnego; dziesiec lokci wzwyz byl kazdy z nich.
\par 24 A bylo na piec lokci skrzydlo Cherubinowe jedno, a na piec lokci skrzydlo Cherubinowe drugie: dziesiec lokci bylo od konca skrzydla jednego az do konca skrzydla drugiego.
\par 25 Takze na dziesiec lokci byl i Cherub drugi: miara jednaka, i rzezanie jednakie bylo obu Cherubinów.
\par 26 Wysokosc Cherubina jednego byla na dziesiec lokci, takze i drugiego Cherubina.
\par 27 I postawil one Cherubiny w posrodku domu wnetrznego, i rozciagneli skrzydla Cherubinowie, tak iz sie dotykalo skrzydlo jednego jednej sciany, a skrzydlo Cheruba drugiego dotykalo sie drugiej sciany, a skrzydla ich w posród domu dotykaly sie siebie wespolek.
\par 28 I powlókl one Cherubiny zlotem.
\par 29 Nadto i wszystkie sciany okolo domu przyozdobil wyryciem Cherubinów i palm, i kwiatów rozkwitlych wewnatrz i zewnatrz.
\par 30 I tlo domu polozyl zlotem wewnatrz i zewnatrz.
\par 31 Uczynil tez w wejsciu do swiatnicy najswietszej drzwi z oliwnego drzewa, a podwoje i odrzwi byly na piec grani.
\par 32 A te obie drzwi byly z drzewa oliwnego, i przyozdobil je wyryciem Cherubinów, i palm, i rozkwitlych kwiatów, i powlókl je zlotem; oblozyl tez Cherubiny i palmy zlotem.
\par 33 Takze tez uczynil i w wejsciu koscielnem podwoje z drzewa oliwnego na cztery granie.
\par 34 A obie drzwi byly z drzewa jodlowego; na dwie sie strony jedne drzwi otwieraly, takze na dwie strony drzwi drugie otwieraly sie.
\par 35 I wyryl na nich Cherubiny i palmy, i rozkwitle kwiaty, a powlókl zlotem ciagnionem to, co bylo wyryto.
\par 36 Przytem zbudowal sien wnetrzna we trzy rzedy z kamienia ciosanego, a jednym rzedem z heblowanego drzewa cedrowego.
\par 37 Roku czwartego, miesiaca Kwietnia, zalozony jest dom Panski;
\par 38 A roku jedenastego, miesiaca Pazdziernika, (ten jest miesiac ósmy,)dokonany jest dom ze wszystkiem naczyniem jego i ze wszystkiem, co do niego nalezalo. A budowal go przez siedm lat.

\chapter{7}

\par 1 Potem dom swój budowal Salomon przez trzynascie lat, i dokonal wszystkiego domu swego.
\par 2 Zbudowal tez dom lasu Libanowego na sto lokci wdluz, a na piecdziesiat lokci wszerz, a na trzydziesci lokci wzwyz, na czterech rzedach slupów cedrowych, a balki cedrowe lezaly na onych slupach.
\par 3 A byl nakryty cedrem z wierzchu na onych balkach, które byly na czterdziestu i pieciu slupach, których bylo w kazdym rzedzie pietnascie.
\par 4 Okna tez byly we trzy rzedy, a okno przeciwko oknu trzema rzedami.
\par 5 A wszystkie drzwi i podwoje byly na cztery granie, i okna: a sporzadzone byly okna przeciw oknom trzema rzedami.
\par 6 Uczynil tez przysionek na slupach na piecdziesiat lokci wdluz, a wszerz na trzydziesci lokci. A byl on przysionek na przodku, takze i slupy i balki na przodku domu tego.
\par 7 Nadto uczynil przysionek dla stolicy, gdzie sadzil, przysionek sadowy, który nakryty byl cedrem od tla az do stropu.
\par 8 A w domu swym, w którym mieszkal, uczynil sale druga za przysionkiem takaz robota; zbudowal tez dom córce Faraonowej, która byl pojal Salomon, podobny temuz przysionkowi.
\par 9 To wszystko bylo z kosztownego kamienia pod miara wyciosanego, i pila rzezanego, wewnatrz i zewnatrz, od tla az do stropu, a z dworu az do wielkiej sieni.
\par 10 A fundament byl z kamienia kosztownego, i z kamienia wielkiego, z kamienia na dziesiec lokci, i z kamienia na osm lokci.
\par 11 A nad tem kamienie kosztowne pod miara wyciosane, z deskami cedrowemi.
\par 12 Sien takze wielka miala w okolo trzy rzedy kamienia ciosanego, a jednym rzedem drzewo cedrowe, tak jako sien wnetrzna domu Panskiego i przysionek domu tego.
\par 13 Poslal tez król Salomon, i wezwal Hirama z Tyru.
\par 14 A ten byl synem niewiasty wdowy z pokolenia Neftalim, a ojciec jego byl obywatel Tyrski, który robil miedzia, a byl pelen madrosci i roztropnosci, i umiejetnosci na robienie wszelkiej roboty z miedzi; ten przyszedlszy do króla Salomona, zrobil wszelka robote jego.
\par 15 Naprzód ulal dwa slupy miedziane; osimnascie lokci bylo wzwyz slupa jednego, a w okrag dwanascie lokci; takiz byl i drugi slup.
\par 16 Potem uczynil dwie galki, które miano postawic na wierzchu slupów, ulane z miedzi; piec lokci wzwyz bylo galki jednej, a piec lokci wzwyz galki drugiej.
\par 17 Siatki tez robota dziana i sznury naksztalt lancuchów posprawial do tych galek, które byly na wierzchu slupów, siedm na galke jedne a siedm na druga galke.
\par 18 A uczyniwszy slupy sprawil dwa rzedy jablek granatowych w okolo na siatce jednej, aby okrywaly galki, które byly na wierzchu; takze tez uczynil i na drugiej galce.
\par 19 A na onych galkach, które byly na wierzchu slupów w przysionku, byla robota lilii, na cztery lokcie.
\par 20 I mialy one galki na onych dwóch slupach, tak z wierzchu jako i przeciwko srodkowi pod siatka, jablka granatowe, których bylo dwiescie, dwoma rzedami w okolo, na jednej i na drugiej galce.
\par 21 I postawil one slupy w przysionku koscielnym; a postawiwszy slup prawy, nazwal imie jego Jachin; postawiwszy zas slup lewy, nazwal imie jego Boaz.
\par 22 A na wierzchu onych slupów byly wyrobione lilije. A tak dokonana jest robota onych slupów.
\par 23 Przytem uczynil morze odlewane na dziesiec lokci od jednego brzegu do drugiego brzegu, okragle w okolo; a na piec lokci byla wysokosc jego, a okrag jego na trzydziesci lokci w okolo.
\par 24 A pod brzegiem jego byly pukle naksztalt jablek lesnych, wszedy w okolo, w kazdym lokciu po dziesiec, które okrazyly morze w okolo; dwa rzedy jablek lanych z nim ulano.
\par 25 To morze stalo na dwunastu wolach; trzy patrzaly na pólnocy, a trzy patrzaly ku zachodowi, a trzy patrzaly ku poludniowi, a trzy patrzaly ku wschodowi; a morze stalo na nich z wierzchu, a wszystkie zady ich byly pod morzem.
\par 26 A bylo miazsze na dlon, a brzeg jego byl jako kraje u kubka, naksztalt kwiatu lilijowego, a dwa tysiace wiader bralo w sie.
\par 27 Uczynil tez dziesiec podstawków miedzianych, na cztery lokcie wdluz podstawek jeden, a na cztery lokcie wszerz, a na trzy lokcie wzwyz.
\par 28 A taka byla robota kazdego podstawka: listwowania mialy w okolo, które listwowania byly miedzy krancami.
\par 29 A na onem listwowaniu, które bylo miedzy krancami, lwy, woly, i Cherubinowie byly; a na krancach byl podstawek z wierzchu, a pod onemi lwami i wolmi bylo przydane obwiedzienie robota ciagniona.
\par 30 A cztery kola miedziane byly pod kazdym podstawkiem, i deski miedziane; a na czterech rogach jego byly podpory jako ramiona, a pod wanna byly te ramiona ulane przy kazdej stronie obwiedzienia.
\par 31 Glebokosc wanny od wierzchu do dna nad slupcem byla na lokiec, takze wierzch jej byl okragly, jako i slupiec, który byl na póltora lokcia: a na wierzchu jej byly rzezania, i listwowania czworograniste, nie okragle.
\par 32 A tak bylo po cztery kola pod onem listwowaniem, a osi kól wychodzily z podstawka, a kazde kolo bylo wzwyz na póltora lokcia.
\par 33 A robota tych kól byla jako robota kól wozowych; osi ich, i szpice ich, i dzwona ich, i piasty ich, wszystko bylo odlewane.
\par 34 Byly tez cztery ramiona na czterech rogach kazdego podstawka, z którego wychodzily one ramiona.
\par 35 A na wierzchu podstawka byl slupek wzwyz na pól lokcia zewszad okragly, i na wierzchu tegoz podstawka byly krance jego i listwowania, które wychodzily z niego.
\par 36 I wyrzezal na deszczkach po krancach jego, i po listwowaniach jego Cherubiny, lwy, i palmy, jedno podle drugiego, po kazdem przydaniu w okolo.
\par 37 Na tenze ksztalt uczynil dziesiec podstawków odlewanych; jednakiem odlewaniem, jednakiej miary, i jednakiego rzezania wszystkie byly.
\par 38 Przytem uczynil dziesiec wiader miedzianych; czterdziesci wanien brala w sie jedna wanna, a kazda wanna byla na cztery lokcie; jedna wanna stala na jednym podstawku, a tak staly na dziesieciu podstawkach.
\par 39 I postawil piec podstawków po prawej stronie domu, a piec po lewej stronie domu; postawil tez morze po prawej stronie domu na wschód slonca ku poludniu.
\par 40 Naczynil tedy Hiram wanien i lopat i miednic. A tak dokonal Hiram pracy wszystkiej roboty, która czynil królowi Salomonowi do domu Panskiego.
\par 41 To jest, dwa slupy, i dwie galki okragle, które byly na wierzchu dwóch slupów, i dwie siatki, aby okrywaly te dwie galki okragle, które byly na wierzchu slupów.
\par 42 I jablek granatowych cztery sta na onych dwóch siatkach; dwa rzedy jablek granatowych byly na kazdej siatce, aby okrywaly te dwie galki okragle, które byly na wierzchu slupów.
\par 43 Takze dziesiec podstawków, i dziesiec wanien na podstawkach.
\par 44 I morze jedno, a wolów dwanascie pod morzem.
\par 45 I panwie, i lopaty, i miednice, i wszystko naczynie, które uczynil Hiram królowi Salomonowi do domu Panskiego, bylo z miedzi polerowanej.
\par 46 To odlewal król na równinie u Jordanu w ilowatej ziemi, miedzy Sochotem i miedzy Sartanem.
\par 47 Ale Salomon zaniechal wazyc tego wszystkiego naczynia dla mnóstwa bardzo wielkiego; nie upatrowano wagi miedzi.
\par 48 Uczynil tez Salomon wszystko inne naczynie do domu Panskiego: oltarz zloty, i stól zloty, na którym lezaly chleby pokladne;
\par 49 I piec lichtarzy po prawej stronie, a piec po lewej stronie przed swiatnica z szczerego zlota, i kwiaty, i lampy, i nozyczki ze zlota;
\par 50 I kubki, i harfy, i miednice, i misy, i kadzielnice z szczerego zlota, i zawiasy zlote do drzwi domu wnetrznego, to jest swiatnicy swietych, takze do drzwi domu kosciola zawiasy zlote.
\par 51 A tak dokonana jest wszystka robota, która sprawil król Salomon do domu Panskiego. I wniósl tam Salomon rzeczy, które byl poswiecil Dawid, ojciec jego, srebro i zloto, i naczynia, i wlozyl do skarbu domu Panskiego.

\chapter{8}

\par 1 Tedy zebral Salomon starsze Izraelskie, i wszystkie celniejsze z kazdego pokolenia, i przedniejsze z ojców synów Izraelskich, do siebie do Jeruzalemu, aby przeniesiona byla skrzynia przymierza Panskiego z miasta Dawidowego, które jest Syon.
\par 2 I zeszli sie do króla Salomona wszyscy mezowie Izraelscy miesiaca Wrzesnia w uroczyste swieto; a ten miesiac jest siódmy.
\par 3 A gdy sie zeszli wszyscy starsi Izraelscy, wzieli kaplani skrzynie.
\par 4 I przeniesli skrzynie Panska, i namiot zgromadzenia, i wszystkie naczynia swiete, które byly w namiocie, a przeniesli je kaplani i Lewitowie.
\par 5 Lecz król Salomon, i wszystko mnóstwo Izraelskie, które sie zeszlo do niego, z nim przed skrzynia ofiarowali owce i woly, których nie liczono ani rachowano dla mnóstwa.
\par 6 Wniesli tedy kaplani skrzynie przymierza Panskiego na miejsce jej do wnetrznego domu, do swiatnicy swietych, pod skrzydla Cherubinów.
\par 7 Albowiem Cherubinowie mieli rozciagnione skrzydla nad miejscem skrzyni, a okrywali Cherubinowie skrzynie i drazki jej z wierzchu.
\par 8 I powyciagali one drazki, tak, ze widac bylo konce ich w swiatnicy na przodku swiatnicy swietych; ale nie widac ich bylo zewnatrz; i tamze byly az do dnia tego.
\par 9 Nic nie bylo w skrzyni tylko dwie tablice kamienne, które tam byl schowal Mojzesz na Horebie, gdy stanowil przymierze Pan z synami Izraelskimi, gdy szli z ziemi Egipskiej.
\par 10 I stalo sie, gdy wychodzili kaplani z swiatnicy, ze oblok napelnil dom Panski.
\par 11 Tak iz sie nie mogli kaplani ostac i sluzyc dla onego obloku; albowiem napelnila byla chwala Panska dom Panski.
\par 12 Tedy rzekl Salomon: Pan powiedzial, iz mial mieszkac we mgle.
\par 13 Juzem zbudowal dom na mieszkanie tobie, miejsce, abys tam przebywal na wieki.
\par 14 I obrócil król oblicze swoje, i blogoslawil wszystkiemu zgromadzeniu Izraelskiemu; a wszystko zgromadzenie Izraelskie stalo.
\par 15 I rzekl: Blogoslawiony Pan, Bóg Izraelski, który mówil usty swemi do Dawida, ojca mego, i skutecznie to wypelnil, mówiac:
\par 16 Ode dnia, któregom wywiódl lud mój Izraelski z Egiptu, nie obralem miasta ze wszystkich pokolen Izraelskich ku zbudowaniu domu, gdzieby przebywalo imie moje, alem obral Dawida, aby byl nad ludem moim Izraelskim.
\par 17 Postanowilci byl wprawdzie w sercu swem Dawid, ojciec mój, zbudowac dom imieniowi Pana, Boga Izraelskiego;
\par 18 Ale rzekl Pan do Dawida, ojca mego: Aczkolwiekes postanowil w sercu twem, zbudowac dom imieniowi memu, i dobrzes uczynil, zes to umyslil w sercu twojem.
\par 19 Wszakze ty nie bedziesz budowal tego domu; ale syn twój, który wynijdzie z biódr twoich, ten zbuduje dom imieniowi memu.
\par 20 A tak utwierdzil Pan slowo swoje, które byl powiedzial. Bom ja powstal miasto Dawida, ojca mego, i usiadlem na stolicy Izraelskiej, jako byl powiedzial Pan, i zbudowalem dom imieniowi Pana, Boga Izraelskiego
\par 21 I naznaczylem tam miejsce skrzyni, w której jest przymierze Panskie, które uczynil z ojcy naszymi, gdy je wywiódl z ziemi Egipskiej.
\par 22 Tedy stanal Salomon przed oltarzem Panskim, przed wszystkiem zgromadzeniem Izraelskiem, i wyciagnal rece swoje ku niebu,
\par 23 I rzekl: Panie, Boze Izraelski, niemasz tobie podobnego Boga na niebie wzgóre, ani na ziemi nisko; który chowasz umowe i milosierdzie slugom twym, którzy chodza przed toba calem sercem swojem;
\par 24 Którys spelnil sludze twemu Dawidowi, ojcu memu, cos mu powiedzial; cos mówil usty swemi, tos skutecznie wypelnil, jako sie dnia tego pokazuje.
\par 25 Przetoz teraz o Panie, Boze Izraelski, zisc sludze twemu Dawidowi, ojcu memu, cos mu powiedzial, mówiac: Nie bedzie odjety przed twarza moja z narodu twego maz, któryby siedzial na stolicy Izraelskiej, jezli tylko beda przestrzegali synowie twoi drogi swej, chodzac przede mna, jakos ty chodzil przed oblicznoscia moja.
\par 26 Przetoz teraz, o Boze Izraelski, niech bedzie utwierdzone prosze slowo twoje, któres mówil do slugi twego Dawida, ojca mego.
\par 27 (Aczci wprawdzie, izali Bóg bedzie mieszkal na ziemi? Oto niebiosa, i nieba niebios, nie moga cie ogarnac; jakoz daleko mniej ten dom, którym zbudowal.)
\par 28 A wejrzyj na modlitwe slugi twego, i na prosbe jego, o Panie Boze mój, wysluchaj wolanie i modlitwe, która sie dzis sluga twój modli przed toba.
\par 29 A niech beda otworzone oczy twoje nad tym domem w nocy i we dnie, nad tem miejscem, o któremes powiedzial: Tu bedzie imie moje; abys wysluchiwal modlitwe, która sie bedzie modlil sluga twój na miejscu tem.
\par 30 Wysluchajze prosby slugi twego, i ludu twego Izraelskiego, który sie modlic bedzie na tem miejscu. Ty wysluchaj z miejsca mieszkania twego, z nieba, a wysluchawszy badz milosciw.
\par 31 Gdyby czlowiek zgrzeszyl przeciwko blizniemu swemu, a przywiódlby go do przysiegi, tak, zeby przysiegac musial, a przyszlaby ta przysiega przed oltarz twój w tym domu:
\par 32 Ty wysluchaj z nieba, a rozeznaj i rozsadz slugi twoje, potepiajac niezboznego, i obracajac sprawy jego na glowe jego, a usprawiedliwiajac sprawiedliwego, oddawajac mu wedlug sprawiedliwosci jego.
\par 33 Gdyby byl porazony lud twój Izraelski od nieprzyjaciela, przeto iz zgrzeszyli przeciw tobie, a nawróciliby sie do ciebie, wyznawajac imie twoje, a modlac sie, przepraszaliby cie w tym domu:
\par 34 Ty wysluchaj z nieba, i odpusc grzech ludowi twemu Izraelskiemu, a przywróc je zasie do ziemi, któras dal ojcom ich.
\par 35 Gdyby zawarte bylo niebo, a nie byloby dzdzu, przeto ze zgrzeszyli przeciwko tobie, a modliliby sie na tem miejscu, wyznawajac imie twoje, a od grzechów swoich odwróciliby sie, gdybys je utrapil:
\par 36 Ty wysluchaj z nieba, a odpusc grzech slug twoich, i ludu twego Izraelskiego, nauczywszy ich drogi prawej, po której chodzic maja, a daj deszcz na ziemie twoje, któras dal w dziedzictwo ludowi twemu.
\par 37 Bylliby glód na ziemi, bylliby mor, susza, rdza, szarancza, jezliby byly chrzaszcze, jezliby go scisnal nieprzyjaciel jego w ziemi mieszkania jego, albo jakakolwiek plaga, albo jakakolwiek niemoc:
\par 38 Wszelka modlitwe, i wszelka prosbe, któraby czynil którykolwiek czlowiek, albo wszystek lud twój Izraelski, ktoby jedno uznal rane serca swego, i wyciagnalby rece swe w domu tym:
\par 39 Ty wysluchaj z nieba, z miejsca mieszkania twego, a odpusc, i uczyn, i oddaj kazdemu wedlug wszelkich dróg jego, które znasz w sercu jego; bo ty, ty sam znasz serca wszystkich synów ludzkich;
\par 40 Aby sie ciebie bali po wszystkie dni, które zyc beda na ziemi, któras dal ojcom naszym.
\par 41 Nadto i cudzoziemiec, który nie jest z ludu twego Izraelskiego, przyjdzieli z ziemi dalekiej dla imienia twego;
\par 42 (Bo uslysza o imieniu twojem wielkiem, i o rece twojej moznej, i o ramieniu twojem wyciagnionem,)przyjdzieli tedy, a bedzie sie modlil w tym domu:
\par 43 Ty wysluchaj z nieba, z miejsca mieszkania twego, a uczyn wszystko, o co zawola do ciebie on cudzoziemiec, aby poznali wszyscy narodowie ziemscy imie twoje, i bali sie ciebie, jako lud twój Izraelski, a zeby wiedzieli, ze imie twoje wzywane jest nad tym domem, którym zbudowal.
\par 44 Gdyby wyszedl lud twój na wojne przeciwko nieprzyjacielowi swemu droga, która je poslesz, a modliliby sie Panu, obróciwszy sie ku miastu, któres obral, i ku domowi, którym zbudowal imieniowi twemu:
\par 45 Wysluchajze z nieba modlitwe ich, i prosbe ich, a wykonaj sad ich.
\par 46 Gdyby tez zgrzeszyli przeciwko tobie, (bo niemasz czlowieka, któryby nie grzeszyl,)a rozgniewawszy sie na nie, podalbys je pod moc nieprzyjacielowi, któryby je pojmawszy, zawiódl je w niewole, do ziemi nieprzyjacielskiej, dalekiej albo bliskiej:
\par 47 A upamietaliby sie w sercu swojem w onej ziemi, do której sa zaprowadzeni w niewole, i nawróciliby sie, a przepraszaliby cie w ziemi tych, którzy je pojmali, mówiac: Zgrzeszylismy, i zlesmy uczynili, niepobozniesmy sie sprawowali;
\par 48 A tak nawróciliby sie do ciebie z calego serca swego, i z calej duszy swej, w ziemi nieprzyjaciól swoich, którzy je pojmali, a modliliby sie tobie, obróciwszy sie ku ziemi swej, któras dal ojcom ich, ku miastu, któres obral, i ku domowi, którym zbudowal imieniowi twemu:
\par 49 Wysluchajze tedy z nieba, z miejsca mieszkania twego, modlitwe ich i prosbe ich, a wykonaj sad ich,
\par 50 A badz milosciw ludowi twemu, który przeciw tobie zgrzeszyl, i wszystkim nieprawosciom ich, któremi wystapili przeciw tobie, a naklon ku nim milosierdzia tych, którzy je pojmali, aby sie zmilowali nad nimi;
\par 51 Poniewaz sa ludem twoim, i dziedzictwem twojem, któres wywiódl z Egiptu, z posrodku pieca zelaznego.
\par 52 Niech beda oczy twoje otwarte na prosbe slugi twego, i na prosbe ludu twego Izraelskiego, abys je wysluchal we wszystkiem, o co cie wzywac beda.
\par 53 Albowiemes je ty sobie odlaczyl za dziedzictwo ze wszystkich narodów ziemi, jakos powiedzial przez Mojzesza, sluge twego, gdys wywiódl ojce nasze z Egiptu, o Panie Boze!
\par 54 I stalo sie, gdy Salomon modlac sie Panu dokonal wszystkiej onej modlitwy i prosby, ze wstal od oltarza Panskiego, a przestal kleczec i podnosic rak swoich ku niebu;
\par 55 A stojac blogoslawil wszystkiemu zgromadzeniu Izraelskiemu wielkim glosem, mówiac:
\par 56 Blogoslawiony Pan, który dal odpocznienie ludowi swemu Izraelskiemu, wedlug wszystkiego, co powiedzial; nie chybilo zadne slowo ze wszystkich slów jego dobrych, które mówil przez Mojzesza, sluge swego.
\par 57 Niechze bedzie Pan, Bóg nasz, z nami, jako byl z ojcy naszymi; niech nas nie opuszcza, ani nas odrzuca;
\par 58 Ale niech nakloni serce nasze ku sobie, zebysmy chodzili po wszystkich drogach jego, strzegac rozkazania jego, i wyroków jego, i sadów jego, które przykazal ojcom naszym.
\par 59 A niech beda te slowa moje, któremim sie modlil przed Panem, bliskie Pana Boga naszego we dnie i w nocy, aby wykonywal sad slugi swego, i sad ludu swego Izraelskiego, sadzac kazda sprawe dnia swojego;
\par 60 Zeby poznali wszyscy narodowie ziemscy, iz Pan sam jest Bogiem, a nikt inszy.
\par 61 Niechze tedy bedzie serce wasze doskonale ku Panu, Bogu naszemu, abyscie chodzili w wyrokach jego, a strzegli przykazan jego, jako i dnia dzisiejszego.
\par 62 Tedy król, i wszystek Izrael z nim, sprawowali ofiary przed Panem.
\par 63 I ofiarowal Salomon ofiare spokojna, która sprawowal Panu, wolów dwadziescia i dwa tysiace, i owiec sto i dwadziescia tysiecy. A tak poswiecali dom Panski król i wszyscy synowie Izraelscy.
\par 64 Onegoz dnia poswiecil król posrodek sieni, która byla przed domem Panskim; bo tam ofiarowal calopalenie, i ofiare sniedna, i tlustosci ofiar spokojnych, przeto ze oltarz miedziany, który byl przed Panem, byl maly, i nie mogly sie na nim zmiescic calopalenia, i ofiary sniedne, i tlustosci ofiar spokojnych.
\par 65 A tak obchodzil Salomon na on czas swieto zacne, i wszystek Izrael z nim, zgromadzenie wielkie od wejscia do Emat az do rzeki Egipskiej, przed Panem, Bogiem naszym, przez siedm dni i przez siedm dni, to jest, przez czternascie dni.
\par 66 A dnia ósmego rozpuscil lud; którzy blogoslawiac królowi, rozeszli sie do przybytków swoich, weselac sie, i cieszac sie w sercu swem ze wszystkiego dobrego, które uczynil Pan Dawidowi, sludze swemu, i Izraelowi, ludowi swemu.

\chapter{9}

\par 1 I stalo sie, gdy dokonczyl Salomon budowania domu Panskiego, i domu królewskiego, i wszystkiego, co zadal Salomon i chcial uczynic,
\par 2 Ze sie Pan ukazal Salomonowi powtóre, jako mu sie ukazal w Gabaon.
\par 3 I rzekl Pan do niego: Wysluchalem modlitwe twoje, i prosbe twoje, któras sie modlil przedemna, a poswiecilem ten dom, którys zbudowal, aby tam przebywalo imie moje az na wieki; i beda tam oczy moje i serce moje po wszystkie dni.
\par 4 A jezli ty bedziesz chodzil przedemna, jako chodzil Dawid, ojciec twój, w doskonalosci serca i w prostosci, a bedziesz sie sprawowal wedlug wszystkiego, comci przykazal, strzegac wyroków moich i sadów moich:
\par 5 Tedy utwierdze stolice królestwa twego nad Izraelem na wieki, jakom powiedzial Dawidowi, ojcu twemu, mówiac: Nie bedzie odjety z narodu twego maz z stolicy Izraelskiej.
\par 6 Ale jezli sie nazad odwrócicie wy i synowie wasi ode mnie, a nie bedziecie strzegli przykazan moich, i wyroków moich, którem wam podal, ale odszedlszy bedziecie sluzyli bogom cudzym, i bedziecie sie im klaniali:
\par 7 Tedy wytrace Izraela z ziemi, któram im dal, a dom, którym poswiecil imieniowi memu, odrzuce od oblicza mego, a bedzie Izrael przypowiescia i basnia miedzy wszystkimi narodami.
\par 8 A tak i ten dom, który byl slawny, kazdemu mimo idacemu bedzie na podziw i na poswistanie, i rzecze: Przeczze tak uczynil Pan tej ziemi i temu domowi?
\par 9 Tedy odpowiedza: Przeto, iz opuscili Pana, Boga swego, który wywiódl ojce ich z ziemi Egipskiej, a chwycili sie bogów cudzych, i klaniali sie im, a sluzyli im: dla tegoz przywiódl Pan na nie to wszystko zle.
\par 10 I stalo sie po wyjsciu dwudziestu lat, w których zbudowal Salomon owe oba domy, dom Panski i dom królewski.
\par 11 Do czego Hiram, król Tyrski, nadal byl Salomonowi drzewa cedrowego, i drzewa jodlowego, i zlota, ile jedno chcial: tedy tez król Salomon dal Hiramowi dwadziescia miast w ziemi Galilejskiej.
\par 12 I wyjechal Hiram z Tyru, aby ogladal miasta, które mu dal Salomon: ale mu sie niepodobaly.
\par 13 I rzekl: Cóz to za miasta, któres mi dal, bracie mój? I nazwal je ziemia Chabul, az do dnia tego.
\par 14 Albowiem poslal byl Hiram królowi sto i dwadziescia talentów zlota.
\par 15 A przyczyna poboru, który byl rozkazal wybierac król Salomon, byla, aby zbudowal dom Panski, i dom swój, i Mello, i mury Jeruzalemskie, i Hasor, i Magieddo, i Gazer.
\par 16 Farao bowiem, król Egipski, wyciagnal byl, i wzial Gazer, i popalil je ogniem, a Chananejczyka, który mieszkal w tem miescie, wymordowal, a dal je za posag córce swej, zonie Salomonowej.
\par 17 A tak zbudowal Salomon Gazer i Betoron nizsze;
\par 18 Przytem Baalat i Tadmor na puszczy w tejze ziemi,
\par 19 I wszystkie miasta, w których mial sklady Salomon, i miasta wozów, i miasta jezdnych, i wszystko wedlug zadosci Salomonowej, cokolwiek chcial budowac w Jeruzalemie i na Libanie, i we wszystkiej ziemi panstwa swojego.
\par 20 Wszystek takze lud, który byl pozostal z Amorejczyków, Hetejczyków, Ferezejczyków, Hewejczyków, i Jebuzejczyków, którzy nie byli z synów Izraelskich;
\par 21 To jest, syny ich, którzy byli pozostali po nich w ziemi, których nie mogli synowie Izraelscy wytracic, uczynil Salomon holdownikami i niewolnikami az do dnia dzisiejszego.
\par 22 Ale z synów Izraelskich nie uczynil Salomon zadnego niewolnikiem; jedno byli ludzmi rycerskimi, i slugami jego, i ksiazety, i hetmany jego, i przelozonymi nad wozami jego, i nad jezdnymi jego.
\par 23 Bylo tych przedniejszych z przelozonych, którzy byli nad robota Salomonowa, piec set i piecdziesiat, co byli nad ludzmi wykonywajacymi robote.
\par 24 Lecz córka Faraonowa przeprowadzila sie z miasta Dawidowego do domu swego, który jej zbudowal Salomon. Tedy zbudowal i Mello.
\par 25 I ofiarowal Salomon trzy kroc na kazdy rok calopalenia a spokojne ofiary na oltarzu, który byl zbudowal Panu; ale kadzil na onym oltarzu, który byl przed Panem, gdy dokonal domu.
\par 26 Okretów tez nabudowal król Salomon w Asyjongaber, które jest podle Elotu, nad brzegiem morza czerwonego, w ziemi Edomskiej.
\par 27 I poslal Hiram na tychze okretach slugi swe, zeglarze swiadome morza, z slugami Salomonowymi;
\par 28 Którzy przyplynawszy do Ofir, wzieli stamtad zlota cztery sta i dwadziescia talentów, i przywiezli je do króla Salomona.

\chapter{10}

\par 1 A królowa z Saby uslyszawszy slawe o Salomonie i o imieniu Panskiem przyjechala, aby go doswiadczyla w zagadkach.
\par 2 I wjechala do Jeruzalemu z wielkim bardzo pocztem, z wielbladami niosacemi rzeczy wonne, i zlota bardzo wiele, i kamienia drogiego, a przyszedlszy do Salomona mówila do niego o wszystkiem, co miala w sercu swojem.
\par 3 Ale jej odpowiedzial Salomon na jej wszystkie slowa; nie bylo nic skrytego przed królem, na coby jej nie odpowiedzial.
\par 4 Przetoz widzac królowa z Saby wszystke madrosc Salomonowa, i dom, który byl zbudowal,
\par 5 Takze potrawy stolu jego, i siadania slug jego, i stawania sluzacych mu, i szaty ich, i podczasze jego, i wschody, po których wstepowal do domu Panskiego, zdumiala sie bardzo;
\par 6 I rzekla do króla: Prawdziwac to mowa, któram slyszala w ziemi mojej o sprawach twoich, i o madrosci twojej;
\par 7 Alem nie wierzyla powiesciom onym, azem sama przyjechawszy ogladala to oczyma swemi. Ale mi tego nie powiedziano i polowy. Wieksza jest madrosc i dobroc twoja nizeli slawa, któram slyszala.
\par 8 Blogoslawieni mezowie twoi, blogoslawieni sludzy twoi, którzy zawsze przed toba stoja, i sluchaja madrosci twojej.
\par 9 Niechze bedzie Pan, Bóg twój, blogoslawiony, który cie sobie upodobal, aby cie posadzil na stolicy Izraelskiej, przeto iz Pan umilowal Izraela na wieki, i postanowil cie królem, abys czynil sad i sprawiedliwosc.
\par 10 I dala królowi sto i dwadziescia talentów zlota, i rzeczy wonnych bardzo wiele, i kamienia drogiego. Nie przyszlo nigdy potem tak wiele wonnych rzeczy, jako dala królowa z Saby królowi Salomonowi.
\par 11 Nadto okrety Hirama, które przynosily zloto z Ofir, przyniosly z Ofir drzewa almugimowego bardzo wiele i kamienia drogiego.
\par 12 I poczynil król z drzewa almugimowego wschody do domu Panskiego, i do domu królewskiego, i harfy, i lutnie spiewakom; a nigdy nie przywozono takiego drzewa almugimowego, ani widziano az do dnia tego.
\par 13 Król takze Salomon dal królowej z Saby wszystko, czego chciala i czego zadala, oprócz tego, co jej dal z dobrej woli reka królewska. Potem odjechawszy, wrócila sie do ziemi swojej, ona i sludzy jej.
\par 14 A byla waga onego zlota, które przychodzilo Salomonowi na kazdy rok, szesc set szescdziesiat i szesc talentów zlota,
\par 15 Oprócz tego, co przychodzilo od kupców i z handlu tych, którzy rzeczami wonnemi kupczyli, i od wszystkich królów Arabskich, i ksiazat ziemi.
\par 16 Przetoz uczynil król Salomon dwiescie tarczy ze zlota ciagnionego, szesc set syklów zlota wychodzilo na kazda tarcza;
\par 17 Przytem trzy sta puklerzy ze zlota ciagnionego, trzy grzywny zlota odwazyl na kazdy puklerz. I schowal je król w domu lasu Libanowego.
\par 18 Uczynil takze król stolice wielka z kosci sloniowej, i powlókl ja szczerem zlotem.
\par 19 Szesc stopni bylo u onej stolicy, a wierzch okragly byl na stolicy z tylu; i porecze byly z obudwu stron siedzenia, a dwa lwy staly u poreczy;
\par 20 A dwanascie lwów stalo na onych szesciu stopniach z obu stron. Nie bylo nic takiego urobiono w zadnych królestwach.
\par 21 Nadto wszystkie naczynia, z których pijal król Salomon, byly zlote, takze i wszystkie naczynia domu lasu Libanowego byly z szczerego zlota; nic nie bylo ze srebra, ani go miano w jakiej cenie za dni Salomonowych.
\par 22 Albowiem okrety królewskie byly na morzu z okretami Hiramowemi: raz we trzy lata wracaly sie okrety z morza, przynoszac zloto i srebro, kosci sloniowe, i koczkodany, i pawie.
\par 23 A tak uwielmozniony jest król Salomon nad wszystkie króle ziemskie bogactwy i madroscia.
\par 24 Przetoz wszyscy obywatele ziemi pragneli widziec Salomona, aby sluchali madrosci jego, która byl dal Bóg w serce jego.
\par 25 I przynosil mu kazdy dary swe, naczynia srebrne i naczynia zlote, i szaty, i zbroje, i rzeczy wonne, konie, i muly, a to na kazdy rok.
\par 26 Tak iz nazgromadzal Salomon wozów, i jezdnych, a mial tysiac i czterysta wozów, i dwanascie tysiecy jezdnych, które rozsadzil po miastach wozów, i przy sobie w Jeruzalemie.
\par 27 I zlozyl król srebra w Jeruzalemie tak wiele, jako kamienia, a ceder jako sykomorów, których na polu rosnie bardzo wiele.
\par 28 Przywodzono tez konie Salomonowi z Egiptu, i towary rozliczne; bo kupcy królewscy brali towary rozliczne za pewne pieniadze.
\par 29 A wychodzil i przychodzil cug wozników z Egiptu za szesc set srebrników, a kon za sto i piecdziesiat. A tak wszyscy królowie Hetejscy, i królowie Syryjscy z rak ich koni dostawali.

\chapter{11}

\par 1 Tedy król Salomon rozmilowal sie niewiast obcych wiele: nie tylko córki Faraonowej, ale i Moabitczanek, Ammonitczanek, Edomczanek, Sydonczanek, Hetejczanek.
\par 2 Z tych narodów, o których powiedzial Pan synom Izraelskim: Nie wchodzcie do nich, i one niech nie wchodza do was; albowiem naklonilyby serce wasze za bogi swymi. Do tych przylgnal Salomon miloscia.
\par 3 Tak iz mial zon królowych siedm set, a zaloznic trzy sta; i odwrócily zony jego serce jego.
\par 4 I stalo sie, gdy juz byl Salomon stary, ze zony jego naklonily serce jego za bogi cudzymi, tak iz nie bylo serce jego zupelne z Panem, Bogiem jego, jako serce Dawida, ojca jego.
\par 5 Ale udal sie Salomon za Astarota, boginia Sydonska, i za Molochem, obrzydliwoscia Ammonitów.
\par 6 I uczynil Salomon, co sie nie podobalo Panu, ani chodzil doskonale za Panem, jako Dawid, ojciec jego.
\par 7 Tedy zbudowal Salomon kaplice Chamosowi, obrzydliwosci Moabskiej, na górze przeciw Jeruzalemowi i Molochowi, obrzydliwosci synów Ammonowych.
\par 8 I tak uczynil wszystkim zonom swym cudzoziemkom, które kadzily, i ofiarowaly bogom swoim.
\par 9 I rozgniewal sie Pan na Salomona, ze sie odwrócilo serce jego od Pana, Boga Izraelskiego, który sie mu byl ukazal po dwa kroc.
\par 10 I zakazal mu tego, aby nie chodzil za bogi cudzymi; wszakze nie strzegl tego, co Pan przykazal.
\par 11 Przetoz rzekl Pan do Salomona: Poniewazes sie tego dopuscil, nie strzegac przymierza mego, ani wyroków moich, którem ci przykazal, pewnie oderwe królestwo od ciebie, a dam je sludze twojemu,
\par 12 Wszakze za dni twoich nie uczynie tego dla Dawida, ojca twego; ale z reki syna twego oderwe je.
\par 13 Lecz wszystkiego królestwa nie oderwe; pokolenie jedno dam synowi twemu dla Dawida, slugi mego, i dla Jeruzalemu, którem obral.
\par 14 Przetoz wzbudzil Pan przeciwnika Salomonowi, Adada Edomczyka z nasienia królewskiego, który byl w Edom.
\par 15 Albowiem stalo sie, gdy Dawid byl w Edom, a Joab, hetman wojska, wyjechal, aby pochowal pobite, i pobil wszystkie mezczyzny w Edom;
\par 16 (Bo tam szesc miesiecy mieszkal Joab ze wszystkimi Izraelczykami, az wytracil wszystkie mezczyzny w Edom.)
\par 17 Tedy uciekl Adad sam, i niektórzy mezowie Edomscy z slug ojca jego z nim, aby szli do Egiptu; a Adad byl chlopcem nie wielkim.
\par 18 Którzy wyszedlszy z Madyjan przyszli do Faran, a wziawszy z soba niektóre meze z Faran weszli do Egiptu, do Faraona króla Egipskiego, który mu dal dom, i zywnosc mu naznaczyl, dal mu tez i ziemie.
\par 19 I znalazl Adad wielka laske w oczach Faraonowych, tak, iz mu dal za zone siostre zony swej, siostre królowej Tafnes.
\par 20 I porodzila mu siostra Tafnes Gienubata, syna jego, którego odchowala Tafnes w domu Faraonowym. I byl Gienubat w domu Faraonowym miedzy syny Faraonowymi.
\par 21 A gdy uslyszal Adad w Egipcie, iz zasnal Dawid z ojcy swymi, a iz umarl Joab, hetman wojska, tedy rzekl Adad do Faraona: Pusc mie, ze pójde do ziemi mojej.
\par 22 Któremu odpowiedzial Farao: Czegoz ci niedostawa u mnie, ze chcesz isc do ziemi twojej? A on rzekl: Niczego; a wszakze pusc mie.
\par 23 Wzbudzil takze Bóg nan przeciwnika, Rezona, syna Elijadowego, który byl uciekl od Adarezera, króla Soby, Pana swego.
\par 24 A zebrawszy do siebie meze, byl ksiazeciem roty, gdy je Dawid mordowal; przetoz odszedlszy do Damaszku, mieszkali w nim, a królowali nad Damaszkiem.
\par 25 I byl przeciwnikiem Izraelowi po wszystkie dni Salomonowe, a to oprócz szkód, które mu czynil Adad; bo sie brzydzil Izraelem, gdy królowal w Syryi.
\par 26 Jeroboam tez, syn Nabata Efratejczyka z Saredy, (a imie matki jego Serwa, która byla wdowa,)sluga Salomonowy, podniósl przeciwko królowi reke.
\par 27 A tac byla przyczyna, dla której podniósl reke swa przeciwko królowi, ze Salomon zbudowawszy Mello, zaprawil dziure w miescie Dawida, ojca swego.
\par 28 A Jeroboam byl maz mocny i mozny. Przetoz widzac Salomon mlodzienca, ze byl sprawny, postanowil go nad podatkami wszystkiemi domu Józefowego.
\par 29 I stalo sie tegoz czasu, gdy Jeroboam wyszedl z Jeruzalemu, ze go znalazl na drodze Achyjasz Sylonitczyk, prorok, bedac odziany plaszczem nowym; a tylko sami dwaj byli na polu.
\par 30 Tedy wziawszy Achyjasz plaszcz nowy, który mial na sobie, rozdarl go na dwanascie sztuk.
\par 31 I rzekl do Jeroboama: Wezmij sobie dziesiec sztuk; bo tak mówi Pan, Bóg Izraelski: Oto Ja oderwe królestwo z rak Salomonowych, a dam tobie dziesiec pokolen.
\par 32 Jedno tylko pokolenie zostanie mu dla slugi mego Dawida, i dla miasta Jeruzalem, którem obral ze wszystkich pokolen Izraelskich;
\par 33 Przeto, ze mie opuscili, a klaniali sie Astarocie, bogini Sydonskiej, i Chamosowi, bogu Moabskiemu, i Molochowi, bogu synów Ammonowych, a nie chodzili drogami mojemi, aby czynili to, co sie mnie podoba, ani tez strzegli wyroków moich, i sadów mo ich, jako Dawid, ojciec jego.
\par 34 Wszakze nie odejme wszystkiego królestwa z reki jego, owszem zostawie go ksiazeciem po wszystkie dni zywota jego dla Dawida, slugi mego, któregom obral, który strzegl rozkazania mego i wyroków moich.
\par 35 Ale wziawszy królestwo z reki syna jego, dam tobie z niego dziesiec pokolen:
\par 36 A synowi jego dam pokolenie jedno, aby zostala pochodnia Dawidowi, sludze memu, po wszystkie dni przedemna w miescie Jeruzalemie, którem sobie obral, aby tam przebywalo imie moje;
\par 37 A ciebie wezme, abys królowal nad wszystkiem, czego zada dusza twoja, a bedziesz królem nad Izraelem.
\par 38 Przetoz jezli bedziesz sluchal wszystkiego, coc przykaze, a bedziesz chodzil drogami mojemi, czyniac to, co dobrego jest w oczach moich, strzegac wyroków moich, i przykazan moich, jako czynil Dawid, sluga mój: tedy bede z toba, a zbudujec dom mocny, jakom zbudowal Dawidowi, i podam ci Izraela;
\par 39 I trapic bede nasienie Dawidowe dla tego; a wszakze nie po wszystkie dni.
\par 40 Przetoz Salomon chcial zabic Jaroboama; ale wstawszy Jeroboam uciekl do Egiptu, do Sesaka, króla Egipskiego, i byl w Egipcie az do smierci Salomonowej.
\par 41 A inne sprawy Salomonowe, którekolwiek czynil, i madrosc jego, izali nie sa wypisane w ksiegach spraw Salomonowych?
\par 42 A dni, których królowal Salomon w Jeruzalemie nad wszystkim Izraelem, bylo czterdziesci lat.
\par 43 I zasnal Salomon z ojcy swymi, a pogrzebiony jest w miescie Dawida, ojca swego; i królowal Roboam, syn jego, miasto niego.

\chapter{12}

\par 1 Tedy jechal Roboam do Sychem; bo w Sychem zebral sie byl wszystek Izrael, aby go postanowili królem.
\par 2 I stalo sie, gdy uslyszal Jeroboam, syn Nabata, który byl jeszcze w Egipcie; (bo byl uciekl przed królem Salomonem, i mieszkal Jeroboam w Egipcie.)
\par 3 Tedy poslali i wezwali go. Przetoz przyszedlszy Jeroboam, i wszystko zgromadzenie Izraelskie, rzekli do Roboama, mówiac:
\par 4 Ojciec twój wlozyl na nas jarzmo ciezkie; ale ty teraz ulzyj nam niewoli srogiej ojca twego, i jarzma jego ciezkiego, które wlozyl na nas, a bedziemyc sluzyli.
\par 5 Który im rzekl: Odejdzcie, a po trzech dniach wróccie sie do mnie. I odszedl lud.
\par 6 Tedy wszedl w rade król Roboam z starszymi, którzy stawali przed Salomonem, ojcem jego, za zywota jego, mówiac: Co wy radzicie, jakabym mial dac odpowiedz ludowi temu?
\par 7 Którzy mu odpowiedzieli, mówiac: Jezli dzis powolny bedziesz ludowi temu, a posluchasz ich, i dasz im odpowiedz, a bedziesz mówil do nich slowa lagodne, beda slugami twymi po wszystkie dni.
\par 8 Ale on opusciwszy rade starszych, która mu podali, wszedl w rade z mlodziencami, którzy z nim wzrosli, a którzy stawali przed nim;
\par 9 I rzekl do nich: A wy co radzicie, abysmy odpowiedzieli ludowi temu, który rzekl do mnie, mówiac: Ulzyj jarzma, które wlozyl ojciec twój na nas?
\par 10 Tedy mu odpowiedzieli oni mlodziency, którzy z nim wzrosli, mówiac:Tak odpowiesz temu ludowi, którzy mówili do ciebie, a rzekli: Ojciec twój wlozyl na nas jarzmo ciezkie, ale nam go ty ulzyj; tak rzeczesz do nich: Najmniejszy palec mój miezszy jest niz biodra ojca mego.
\par 11 Przetoz teraz ojciec mój kladl na was jarzmo ciezkie, ale ja przydam do jarzma waszego; ojciec mój karal was biczykami, ale ja was bede karal korbaczami.
\par 12 Przyszedl tedy Jeroboam, i wszystek lud do Roboama dnia trzeciego, jako byl rozkazal król, mówiac: Wróccie sie do mnie dnia trzeciego.
\par 13 I dal sroga odpowiedz król ludowi, opusciwszy rade starszych, która mu byli dali,
\par 14 A rzekl do nich wedlug rady mlodzienców, mówiac: Ojciec mój obciazal was jarzmem ciezkiem, ale ja przydam do jarzma waszego; ojciec mój karal was biczykami, ale ja was bede karal korbaczami.
\par 15 I nie usluchal król ludu; bo byla przyczyna od Pana, aby dosyc uczynil slowu swemu, które byl powiedzial Pan przez Achyjasza Sylonitczyka do Jeroboama, syna Nabatowego.
\par 16 A gdy widzial wszystek Izrael, ze ich nie usluchal król, odpowiedzial lud królowi, tak mówiac: Cóz my mamy za dzial w Dawidzie? a co za dziedzictwo w synu Isajowym? Idz do namiotów swych, o Izraelu, a ty Dawidzie opatrz teraz dom twój. I rozeszli sie Izraelczycy do namiotów swoich.
\par 17 A tak tylko nad synami Izraelskimi, którzy mieszkali w miesciech Judzkich, królowal Roboam.
\par 18 I poslal król Roboam Adorama, który byl poborca, i ukamionowal go wszystek Izrael, az umarl; przetoz król Roboam, wsiadlszy co rychlej na wóz, uciekl do Jeruzalemu.
\par 19 A tak odstapili Izraelczycy od domu Dawidowego, az do dnia tego.
\par 20 I stalo sie, gdy uslyszal wszystek Izrael, ze sie wrócil Jeroboam, poslawszy przyzwali go do zgromadzenia, i postanowili go królem nad wszystkim Izraelem. Nie zostalo przy domu Dawidowym jedno samo pokolenie Judowe.
\par 21 A przyjechawszy Roboam do Jeruzalemu, zebral wszystek dom Judowy, i pokolenie Benjaminowe, sto i osmdziesiat tysiecy mezów przebranych ku bojowi, aby walczyli z domem Izraelskim, azeby przywrócone bylo królestwo Roboamowi, synowi Salomonowemu.
\par 22 I stalo sie slowo Boze do Semejasza, meza Bozego, mówiac:
\par 23 Powiedz Roboamowi, synowi Salomonowemu, królowi Judzkiemu i wszystkiemu domowi Judowemu i Benjaminowemu, i innemu ludowi, mówiac:
\par 24 Tak mówi Pan: Nie wychodzcie, ani walczcie z bracia swoja, synmi Izraelskimi; wróccie sie kazdy do domu swego: albowiem odemnie sie ta rzecz stala. I usluchali rozkazania Panskiego, a wrócili sie, aby odeszli wedlug slowa Panskiego.
\par 25 Potem zbudowal Jeroboam Sychem na górze Efraim, i mieszkal w nim, a stamtad wyszedlszy pobudowal Fanuel.
\par 26 I rzekl Jeroboam w sercu swem: Wnetby sie wrócilo królestwo do domu Dawidowego.
\par 27 Gdyby chadzal ten lud sprawowac ofiary do domu Panskiego do Jeruzalemu, i obróciloby sie serce ludu tego do pana swego, do Roboama, króla Judzkiego, a zabiwszy mie, wróciliby sie do Roboama, króla Judzkiego.
\par 28 Przetoz naradziwszy sie król, uczynil dwóch cielców zlotych, i mówil do ludu: Dosyciescie sie nachodzili do Jeruzalemu; oto bogowie twoi, o Izraelu, którzy cie wywiedli z ziemi Egipskiej.
\par 29 I postawil jednego w Betel, a drugiego postawil w Dan.
\par 30 I bylo to pobudka do grzechu, bo chadzal lud do jednego z tych bogów az do Dan,
\par 31 Uczynil tez dom na wyzynach, i postanowil kaplany niektóre z pospólstwa, którzy nie byli z synów Lewiego.
\par 32 Nadto ustanowil Jeroboam swieto uroczyste miesiaca ósmego, pietnastego dnia tegoz miesiaca, naksztalt swieta, które obchodzono w Judzie, i ofiarowal na oltarzu. Toz uczynil w Betel, ofiarujac cielcom, które byl uczynil; postanowil tez kaplany w Betel na wyzynach, które byl poczynil.
\par 33 I sprawowal tez ofiary na oltarzu, który byl uczynil w Betel, pietnastego dnia miesiaca ósmego, onegoz miesiaca, który byl wymyslil w sercu swojem; i uczynil swieto uroczyste synom Izraelskim, a przystapil do oltarza, aby kadzil.

\chapter{13}

\par 1 A oto maz Bozy przyszedl z Judztwa z slowem Panskiem do Betel, gdy Jeroboam stal u oltarza, aby kadzil.
\par 2 I zawolal przeciw oltarzowi slowem Panskiem, i rzekl: Oltarzu, oltarzu, tak mówi Pan: Oto syn narodzi sie domowi Dawidowemu imieniem Jozyjasz, który bedzie ofiarowal na tobie kaplany wyzyn, kadzace na tobie, i kosci ludzkie popala na tobie.
\par 3 I dal mu znak dnia onegoz, mówiac: Tenci jest znak, ze to mówil Pan: Oto sie oltarz rozpadnie, a wysypie sie popiól, który jest na nim.
\par 4 A gdy uslyszal król Jeroboam slowo meza Bozego, które obwolywal przeciw oltarzowi w Betel, sciagnal reke swa od oltarza, mówiac: Pojmajcie go. I uschla reka jego, która byl wyciagnal przeciw niemu, a niemógl jej przyciagnac do siebie.
\par 5 Oltarz sie tez rozpadl, a wysypal sie popiól z oltarza wedlug znaku, który byl dal maz Bozy slowem Panskiem.
\par 6 Przetoz odpowiadajac król, rzekl do meza Bozego: Prosze cie, pros oblicza Pana Boga twego, a módl sie za mna, aby sie wrócila reka moja do mnie. I modlil sie maz Bozy obliczu Panskiemu, i wrócila sie reka królewska do niego, i byla jako pierwej.
\par 7 Tedy rzekl król do meza Bozego: Pójdz ze mna do domu, abys sie posilil, a dam ci upominek.
\par 8 Ale rzekl maz Bozy do króla: Bysmi dal polowe domu twego, nie pojade z toba, ani bede jadl chleba, ani bede pil wody na tem miejscu.
\par 9 Bo mi tak Pan rozkazal slowem swojem, mówiac: Nie bedziesz jadl chleba, ani bedziesz pil wody, ani sie wrócisz ta droga, któras przyszedl.
\par 10 Odszedl tedy insza droga, a nie wrócil sie ta droga, która byl przyszedl do Betel.
\par 11 A prorok niejaki stary mieszkal w Betel, którego syn przyszedlszy, opowiedzial mu wszystke sprawe, która byl uczynil onegoz dnia maz Bozy w Betel, i slowa, które mówil do króla, opowiedzieli ojcu swemu.
\par 12 I rzekl im ojciec ich: Któraz droga poszedl? I pokazali synowie jego droge, która poszedl on maz Bozy, który byl przyszedl z Judztwa.
\par 13 Zatem rzekl synom swym: Osiodlajcie mi osla; i osiodlali mu osla, i wsiadl nan,
\par 14 I jechal za mezem Bozym, a znalazl go siedzacego pod debem, i rzekl mu: Tyzes jest on maz Bozy, którys przyszedl z Judztwa? A on rzekl: Jestem.
\par 15 I rzekl do niego: Pójdz zemna do domu, zebys jadl chleb.
\par 16 Ale mu on rzekl: Nie moge sie wrócic z toba, ani isc z toba, ani bede jadl chleba, ani bede pil wody z toba na tem miejscu;
\par 17 Bo sie stala do mnie mowa slowem Panskiem: Nie bedziesz tam jadl chleba, ani pil wody, ani sie wrócisz idac ta droga, któras szedl.
\par 18 Któremu on odpowiedzial: I jam prorok jako i ty; Aniol tez rzekl do mnie slowem Panskiem, mówiac: Wróc go z soba do domu twego, aby jadl chleb, i pil wode. I tak sklamal przed nim.
\par 19 I wrócil sie z nim, a jadl chleb w domu jego, i pil wode.
\par 20 A gdy siedzieli u stolu, stalo sie slowo Panskie do proroka, który go byl wrócil;
\par 21 I zawolal na meza Bozego, który byl przyszedl z Judztwa, mówiac: Tak mówi Pan: Przeto zes byl odpornym ustom Panskim, a nie strzegles rozkazania, którec przykazal Pan, Bóg twój:
\par 22 Ales sie wrócil, i jadles chleb, a piles wode na miejscu, o któremem ci byl rzekl: Nie bedziesz tam jadl chleba, ani pil wody: nie bedzie pochowany trup twój w grobie ojców twoich.
\par 23 A tak gdy sie najadl chleba i napil sie, osiodlal osla prorokowi onemu, którego byl wrócil.
\par 24 A gdy odjechal, spotkal go lew w drodze, i zabil go. A trup jego byl porzucony w drodze, a osiel stal wedle niego, lew takze stal podle trupa.
\par 25 A oto mezowie mimo idacy ujrzeli trupa porzuconego na drodze i lwa stojacego podle niego: którzy przyszedlszy powiedzieli to w miescie, w którem on stary prorok mieszkal.
\par 26 Co gdy uslyszal on prorok, który go byl wrócil z drogi, rzekl: Maz Bozy jest, który byl odpornym ustom Panskim; przetoz podal go Pan lwowi, który go potarl, i zabil go wedlug slowa Panskiego, które mu byl powiedzial.
\par 27 Nadto rzekl do synów swoich, mówiac: Osiodlajcie mi osla. I osiodlali.
\par 28 A wyjechawszy znalazl trupa jego porzuconego na drodze, a osla i lwa stojace przy trupie, ale nie jadl lew onego trupa, ani obrazil osla.
\par 29 Tedy wzial prorok trupa meza Bozego, a wlozywszy go na osla, przywiózl go; i przyszedl do miasta swego, aby plakal i pogrzebl go.
\par 30 A polozyl trupa jego w grobie swoim, i plakali go, mówiac: Ach bracie mój!
\par 31 A pochowawszy go, rzekl do synów swoich: Gdy ja umre. pochowajcie mie w tym grobie, w którym jest maz Bozy pochowany; podle kosci jego polózcie kosci moje,
\par 32 Bo zapewne sie to stanie, co obwolal slowem Panskiem, przeciw oltarzowi, który jest w Betel, i przeciwko wszystkim domom wyzyn, które sa w miesciech Samaryjskich.
\par 33 To gdy sie stalo, przeciez sie nie odwrócil Jeroboam od drogi swej zlej, ale znowu naczynil z pospolitego ludu kaplanów wyzyn; kto jedno chcial, poswiecal reke jego, aby byl kaplanem wyzyn.
\par 34 I byla ta rzecz domowi Jeroboamowemu przyczyna do grzechu, aby byl wykorzeniony i wygladzony z ziemi.

\chapter{14}

\par 1 Tegoz czasu rozniemógl sie Abijas, syn Jeroboama.
\par 2 I rzekl Jeroboam do zony swej: Wstan teraz, a odmien sie, aby nie poznano, zes ty zona Jeroboamowa, a idz do Sylo; oto tam jest Achyjasz prorok, który mi powiedzial, zem mial zostac królem nad tym ludem.
\par 3 A wziawszy z soba dziesiecioro chleba, i placków, i faske miodu, idz do niego; on ci oznajmi, co sie stanie dziecieciu.
\par 4 I uczynila tak zona Jeroboamowa, a wstawszy poszla do Sylo, i weszla do domu Achyjaszowego; ale Achyjasz nie mógl juz widziec, bo mu byly zaszly oczy dla starosci jego.
\par 5 A Pan rzekl do Achyjasza: Oto zona Jeroboamowa wchodzi, aby sie od ciebie czego wywiedziala o synu swym, przeto ze choruje; ale tak a tak rzeczesz jej, a stanie sie, gdy bedzie wchodzila, zmysli sie byc insza.
\par 6 Przetoz gdy uslyszal Achyjasz tupanie nóg jej, wchodzacej we drzwi, rzekl: Wnijdz zono Jeroboamowa; przecz sie zmyslasz byc insza? Jam bowiem srogim poslem do ciebie.
\par 7 Idz, powiedz Jeroboamowi: Tak mówi Pan, Bóg Izraelski: Poniewazem cie wywyzszyl z posrodku ludu, a postanowilem cie ksiazeciem nad ludem moim Izraelskim;
\par 8 I oderwalem królestwo od domu Dawidowego, a dalem je tobie, tys jednak nie byl jako sluga mój Dawid, który strzegl rozkazania mego, i który chodzil za mna calem sercem swojem, to tylko czyniac, co jest dobrego w oczach moich;
\par 9 Ales czynil zle nad wszystkie, którzy byli przed toba; albowiem odszedlszy, poczyniles sobie bogi cudze, i lane, abys mie pobudzil do gniewu, a mnies zarzucil w tyl sobie:
\par 10 Przetoz oto Ja przywiode zle na dom Jeroboamowy, i wytrace z Jeroboama az do najmniejszego szczeniecia, wieznia, i opuszczonego w Izraelu, i wymiote ostatki domu Jeroboamowego, jako wymiataja gnój, az do czysta.
\par 11 Tego, który z domu Jeroboamowego umrze w miescie, zjedza psy, a który umrze na polu, zjedza ptaki powietrzne, poniewaz Pan wyrzekl.
\par 12 A ty wstawszy idz do domu twego a gdy wchodzic bedziesz do miasta, tedy umrze dziecie.
\par 13 I bedzie go plakal wszystek Izrael, i pochowaja go; bo ten sam z domu Jeroboamowego wnijdzie do grobu, przeto iz sie znalazlo o nim samym slowo dobre od Pana, Boga Izraelskiego w domu Jeroboamowym.
\par 14 Wszakze postanowi sobie Pan króla nad Izraelem, który wykorzeni dom Jeroboamowy dnia tego; a co mówie, wzbudzi? I owszem juz wzbudzil.
\par 15 I uderzy Pan Izraela, i zachwieje nim, jako sie chwieje trzcina na wodach, a wykorzeni Izraela z ziemi tej dobrej, która dal ojcom ich, i rozproszy je za rzeke, przeto iz sobie poczynili gaje, wzruszajac Pana ku gniewu.
\par 16 A tak wyda Izraela dla grzechu Jeroboamowego, który grzeszyl, i który do grzechu przywiódl Izraela.
\par 17 Tedy wstala zona Jeroboamowa, i poszla, a przyszla do Tersa; a gdy wstepowala na próg domu, umarlo dziecie.
\par 18 I pochowali je, a plakal go wszystek Izrael wedlug slowa Panskiego, które opowiedzial przez sluge swego Achyjasza proroka.
\par 19 A inne sprawy Jeroboamowe, jako walczyl, i jako królowal, oto spisane sa w kronikach o królach Izraelskich.
\par 20 A dni, których królowal Jeroboam, bylo dwadziescia i dwa lat, i zasnal z ojcy swymi, a Nadab, syn jego, królowal miasto niego.
\par 21 Roboam tez, syn Salomona, królowal w Judzie. A bylo Roboamowi czterdziesci lat i jeden, gdy poczal królowac, a siedmnascie lat królowal w miescie Jeruzalemie, które Pan obral ze wszystkich pokolen Izraelskich, aby tam przebywalo imie jego. A imie matki jego bylo Naama, Ammonitka.
\par 22 I czynil Juda zle przed Panem, a wzruszyli go ku gniewu grzechami swemi, któremi grzeszyli nad wszystko, co czynili ojcowie ich.
\par 23 Albowiem i oni pobudowali sobie wyzyny, i slupy, i gaje na kazdym pagórku wysokim, i pod kazdem drzewem zielonem.
\par 24 Byli tez i Sodomczycy w onej ziemi, sprawujacy sie wedlug wszystkich obrzydliwosci poganów, które wyrzucil Pan od oblicznosci synów Izraelskich.
\par 25 I stalo sie roku piatego królowania Roboama, ze wyciagnal Sesak, król Egipski, przeciw Jeruzalemowi.
\par 26 I pobral skarby domu Panskiego, i skarby domu królewskiego, wszystko to pobral; wzial tez wszystkie tarcze zlote, które byl sprawil Salomon;
\par 27 Miasto których król Roboam sprawil tarcze miedziane, i poruczyl je przelozonym nad piechota, którzy strzegli drzwi domu królewskiego.
\par 28 A gdy wchodzil król do domu Panskiego, brala je piechota, i zasie odnosila do komor swoich.
\par 29 A inne sprawy Roboamowe, i wszystko co czynil, azaz nie sa napisane w kronikach o królach Judzkich?
\par 30 I byla wojna miedzy Roboamem i miedzy Jeroboamem po wszystkie dni.
\par 31 I zasnal Roboam z ojcy swymi, a pochowany jest z nimi w miescie Dawidowem; a imie matki jego bylo Naama, Ammonitka. I królowal Abijam, syn jego, miasto niego.

\chapter{15}

\par 1 Roku tedy osmnastego królowania Jeroboama, syna Nabatowego, królowal Abijam nad Juda.
\par 2 Trzy lata królowal w Jeruzalemie; a imie matki jego bylo Maacha, córka Abisalomowa.
\par 3 Ten chodzil we wszystkich grzechach ojca swego, które czynil przed nim; a nie bylo serce jego doskonale przy Panu, Bogu swoim, jako serce Dawida, ojca jego.
\par 4 Wszakze dla Dawida dal mu Pan, Bóg jego, pochodnia w Jeruzalemie, wzbudziwszy syna jego po nim, a utwierdziwszy Jeruzalem,
\par 5 Przeto, ze czynil Dawid, co bylo dobrego w oczach Panskich, a nie uchylal sie od wszystkiego, co mu rozkazal, po wszystkie dni zywota swego, oprócz sprawy z Uryjaszem Hetejczykiem.
\par 6 I byla wojna miedzy Roboamem, i miedzy Jeroboamem po wszystkie dni zywota jego.
\par 7 A insze sprawy Abijamowe, i wszystko, co czynil, azaz nie jest napisane w kronikach o królach Judzkich, jako i wojna miedzy Abijamem i miedzy Jeroboamem?
\par 8 A gdy zasnal Abijam z ojcy swymi, pochowano go w miescie Dawidowem. I królowal Aza, syn jego, miasto niego.
\par 9 A tak roku dwudziestego Jeroboama, króla Izraelskiego, królowal Aza nad Juda.
\par 10 Czterdziesci lat i jeden królowal w Jeruzalemie, a imie matki jego bylo Maacha, córka Abisalomowa.
\par 11 I czynil Aza, co dobrego bylo w oczach Panskich, jako Dawid ojciec jego.
\par 12 Albowiem wytracil Sodomczyki z ziemi, i wyrzucil wszystkie balwany, których byli naczynili ojcowie jego.
\par 13 Nadto i Maache, matke swoje, zrzucil z panowania, bo byla sprawila strasznego balwana w gaju; przetoz porabal Aza tego strasznego balwana jej, i spalil u potoku Cedron.
\par 14 A chociaz wyzyny nie byly skazone, jednak serce Azy bylo doskonale przy Panu po wszystkie dni jego.
\par 15 I wniósl rzeczy poswiecone ojca swego, i rzeczy, które sam poswiecil, do domu Panskiego, srebro, i zloto, i naczynia.
\par 16 I byla wojna miedzy Aza i miedzy Baaza, królem Izraelskim, po wszystkie dni ich.
\par 17 Albowiem Baaza, król Izraelski, wyciagnal przeciw Judzie, a zbudowal Rame, aby nie dopuscil wychodzic i wchodzic nikomu do Azy, króla Judzkiego.
\par 18 Ale wziawszy Aza wszystko srebro i zloto, które bylo pozostalo w skarbiech domu Panskiego, i w skarbiech domu królewskiego, dal je w rece slugom swoim; i poslal je król Aza do Benadada, syna Tabremonowego, syna Hezyjonowego, króla Syryjskiego, który mieszkal w Damaszku, mówiac:
\par 19 Przymierze jest miedzy mna i miedzy toba, miedzy ojcem moim i miedzy ojcem twoim; otoc posylam dary, srebro i zloto; idzze, wzrusz przymierze twoje z Baaza, królem Izraelskim, aby odciagnal odemnie.
\par 20 I usluchal Benadad króla Azy; a poslawszy hetmany z wojski, które mial przeciw miastom Izraelskim, zburzyl Hijon i Dan i Abelbetmaache, i wszystko Cynnerot, i wszystke ziemie Neftalim.
\par 21 Co gdy uslyszal Baaza, przestal budowac Ramy, i mieszkal w Tersie.
\par 22 Tedy król Aza zebral wszystek lud Judzki, nikogo nie wyjmujac; a pobrali kamienie z Ramy i drzewo jego, z którego budowal Baaza: a zbudowal z niego król Aza Gabaa Benjaminowe, i Masfa.
\par 23 A inne wszystkie sprawy Azy, i wszystka moc jego, i cokolwiek czynil, i miasta, które zbudowal, azaz to nie jest napisane w kronikach o królach Judzkich? Ale czasu starosci swej chorowal na nogi swoje.
\par 24 I zasnal Aza z ojcy swymi, a pochowany jest z nimi w miescie Dawida, ojca swego. A Jozafat, syn jego, królowal miasto niego.
\par 25 Ale Nadab, syn Jeroboama, nastapil na królestwo Izraelskie roku wtórego Azy, króla Judzkiego, i królowal nad Izraelem dwa lata;
\par 26 I czynil zle przed oczyma Panskiemi, chodzac drogami ojca swego, i w grzechach jego, któremi do grzechu przywodzil Izraela.
\par 27 I zbuntowal sie przeciw niemu Baaza, syn Ahyjasza, z domu Isaschar; a porazil go Baaza u Giebbeton, które bylo Filistynskie; bo Nadab ze wszystkim Izraelem oblegl byl Giebbeton.
\par 28 I zabil go Baaza roku trzeciego Azy, króla Judzkiego, a sam królowal miasto niego.
\par 29 I stalo sie, gdy poczal królowac, ze wymordowal wszystek dom Jeroboamowy; a nie zostawil zadnej duszy z narodu Jeroboamowego, az je wytracil wedlug slowa Panskiego, które byl opowiedzial przez sluge swego Achyjasza Sylonitczyka;
\par 30 A to dla grzechów Jeroboamowych, który grzeszyl, i który do grzechu przywiódl Izraelczyki, i dla przestepstwa, którem wzruszyl ku gniewu Pana, Boga Izraelskiego.
\par 31 A inne sprawy Nadabowe, i wszystko co czynil, azaz to nie jest napisane w kronikach królów Izraelskich?
\par 32 I byla wojna miedzy Aza, i miedzy Baaza, królem Izraelskim, po wszystkie dni ich.
\par 33 Roku trzeciego Azy, króla Judzkiego, królowal Baaza, syn Achyjasza, nad wszystkim Izraelem w Tersie przez dwadziescia i cztery lat.
\par 34 I czynil zle przed Panem, chodzac drogami Jeroboamowemi, i w grzechu jego, którym do grzechu przywiódl Izraelity.

\chapter{16}

\par 1 I stalo sie slowo Panskie do Jehu, syna Hananijego, przeciw Baazie, mówiace:
\par 2 Dla tego zem cie wydzwignal z prochu, a postanowilem cie wodzem nad ludem moim Izraelskim, a tys chodzil drogami Jeroboamowemi, i przywiodles do grzechu lud mój Izraelski, pobudzajac mie ku gniewu grzechami ich:
\par 3 Otoz ja wygladze potomki Baazy, i potomki domu jego, a uczynie dom twój, jako dom Jeroboama, syna Nabatowego.
\par 4 Tego, który z rodu Baazy umrze w miescie, zjedza psy, a tego, który umrze na polu, zjedza ptaki powietrzne.
\par 5 A inne sprawy Baazy, i co czynil, i moc jego, azaz to nie jest napisano w kronikach królów Izraelskich?
\par 6 A gdy zasnal Baaza z ojcy swymi, pochowany jest w Tersie, i królowal Ela, syn jego, miasto niego.
\par 7 A tak przez proroka Jehu, syna Hananijego, stalo sie slowo Panskie przeciw Baazie i przeciw domowi jego, i przeciw wszystkiemu zlemu, które czynil przed obliczem Panskiem, wzruszajac go ku gniewu sprawami rak swoich, ze ma byc podobnym domowi Jeroboamowemu, i dla tego, ze go zabil.
\par 8 Roku dwudziestego i szóstego Azy, króla Judzkiego, królowal Ela, syn Baazy, nad Izraelem w Tersie dwa lata.
\par 9 I sprzysiagl sie przeciw niemu sluga jego Zymry, hetman nad polowa wozów, gdy Ela w Tersie pijac pijany byl w domu Arsy, który byl sprawca domu królewskiego w Tersie.
\par 10 Wtem przypadl Zymry i ranil go, i zabil go roku dwudziestego i siódmego Azy, króla Judzkiego, a królowal miasto niego.
\par 11 A gdy juz królowal i siedzial na stolicy jego, wymordowal wszystek dom Baazy, i powinne jego, i przyjacioly jego; nie zostawil z niego i szczeniatka.
\par 12 A tak wygladzil Zymry wszystek dom Baazy wedlug slowa Panskiego, które powiedzial o Baazie przez proroka Jehu.
\par 13 Dla wszystkich grzechów Baazy, i grzechów Eli, syna jego, którzy grzeszyli, i którzy przywiedli do grzechu Izraela, wzruszajac ku gniewu Pana, Boga Izraelskiego, próznosciami swemi.
\par 14 Ale inne sprawy Eli, i wszystko co czynil, izali nie napisane w kronikach o królach Izraelskich?
\par 15 Roku dwudziestego i siódmego Azy, króla Judzkiego, królowal Zymry siedm dni w Tersie, gdy lud obozem lezal u Giebbeton, które jest Filistynskie.
\par 16 A gdy uslyszal lud lezacy w obozie takowa rzecz, iz Zymry sprzysiaglszy sie zabil króla: tedy wszystek Izrael postanowili królem Amrego, który byl hetmanem na wojskiem Izraelskiem onegoz dnia w obozie.
\par 17 Przetoz odciagnal Amry i wszystek Izrael z nim od Giebbeton, a oblegli Terse.
\par 18 A gdy obaczyl Zymry, iz wzieto miasto, wszedl na palac domu królewskiego, i spalil sie ogniem z domem królewskim, i umarl.
\par 19 A to sie stalo dla grzechów jego, których sie dopuscil, czyniac zle przed obliczem Panskiem, a chodzac droga Jeroboama, i w grzechach jego, których sie dopuszczal, do grzechu przywodzac Izraela.
\par 20 A inne sprawy Zymry i sprzysiezenie jego, które uczynil, azaz nie jest napisane w kronikach o królach Izraelskich?
\par 21 Tedy sie rozerwal lud Izraelski na dwie czesci; polowa ludu szla za Tebni, synem Ginetowym, aby go uczynili królem, a polowa szla za Amrym.
\par 22 Ale przemógl lud, który przestawal z Amrym, on lud, który zostawal przy Tebni, synem Ginetowym; i umarl Tebni, a królowal Amry.
\par 23 Roku trzydziestego i pierwszego Azy, króla Judzkiego, królowal Amry nad Izraelem dwanascie lat; w Tersie królowal szesc lat.
\par 24 I kupil góre Samaryi od Semera za dwa talenty srebra, i pobudowal na onej górze, a nazwal imie miasta, które zbudowal imieniem Semera, pana góry onej, Samaryi.
\par 25 Ale czynil Amry zle przed oczyma Panskiemi, a dopuszczal sie rzeczy gorszych, nizeli wszyscy, którzy przed nim byli.
\par 26 Albowiem chodzil wszystkiemi drogami Jeroboama, syna Nabatowego, i w grzechu jego, którym przywiódl w grzechy Izraela, wzruszajac ku gniewu Pana Boga Izraelskiego, próznosciami swemi.
\par 27 A inne sprawy Amrego, i wszystko, co czynil, i moc jego, która pokazywal, azaz to nie jest napisane w kronikach o królach Izraelskich?
\par 28 I zasnal Amry z ojcy swymi, a pochowany jest w Samaryi; i królowal Achab, syn jego, miasto niego.
\par 29 Achab tedy, syn Amrego, królowac poczal nad Izraelem roku trzydziestego i ósmego Azy, króla Judzkiego, a królowal Achab, syn Amrego, nad Izraelem w Samaryi dwadziescia i dwa lat.
\par 30 I czynil Achab, syn Amrego, zle przed oczyma Panskiemi nad wszystkie, którzy byli przed nim.
\par 31 I stalo sie, nie majac na tem dosyc, iz chodzil w grzechach Jeroboama, syna Nabatowego, ze sobie wzial za zone Jezabele, córke Etbaala, króla Sydonskiego, a szedlszy sluzyl Baalowi, i klanial mu sie.
\par 32 I wystawil oltarz Baalowi w domu Baalowym, który byl zbudowal w Samaryi.
\par 33 Do tego nasadzil Achab gaj, a tem wiecej wzruszyl ku gniewu Pana, Boga Izraelskiego, nad wszystkie króle Izraelskie, którzy byli przed nim.
\par 34 Za dni jego zbudowal Hijel, Betelczyk, miasto Jerycho. Na Abiramie, pierworodnym swoim, zalozyl je, a na Segubie najmlodszym synu swym, wystawil bramy jego, wedlug slowa Panskiego, które powiedzial przez Jozuego, syna Nunowego.

\chapter{17}

\par 1 Tedy rzekl Elijasz Tesbita, jeden z obywateli Galaadu, do Achaba: Jako zywy Pan, Bóg Izraelski, przed którego oblicznoscia stoje, ze nie bedzie tych lat rosy, ani deszczu, jedno wedlug slów ust moich.
\par 2 I stalo sie slowo Panskie do niego, mówiac:
\par 3 Odejdz stad, a obróc sie na wschód slonca, i skryj sie u potoku Charyt, który jest przeciwko Jordanowi.
\par 4 I bedziesz pil z potoku: a rozkazalem krukom, aby cie tam zywili.
\par 5 I poszedl, a uczynil wedlug slowa Panskiego, i przyszedlszy usiadl u potoku Charyt, który byl przeciwko Jordanowi.
\par 6 A kruki przynosily mu chleb i mieso rano, takze chleb i mieso w wieczór; a pil z potoku.
\par 7 Lecz po wyjsciu niektórych dni wysechl on potok; bo nie padal deszcz na ziemie.
\par 8 I stalo sie slowo Panskie do niego, mówiac:
\par 9 Wstan, idz do Sarepty Sydonskiej, i mieszkaj tam; otom tam rozkazal niewiescie wdowie, aby cie zywila.
\par 10 Tedy on wstawszy szedl do Sarepty, i przyszedl do bramy miasta, a oto tam niewiasta wdowa zbierala drwa; który zawolawszy jej, rzekl: Przynies mi prosze troche wody w naczyniu, abym sie napil.
\par 11 A gdy ona szla, aby przyniosla, tedy na nie zawolal, i rzekl: Przynies mi tez prosze sztuczke chleba w rece twojej.
\par 12 I odpowiedziala: Jako zywy Pan, Bóg twój, zec niemam pieczonego chleba, oprócz z garsc pelna maki w garncu, a troche oliwy w bance; a oto zbieram troche drewek, abym szla, i zgotowala to sobie i synowi swemu, a zjadlszy to, abysmy pomarli.
\par 13 Tedy rzekl do niej Elijasz: Nie bój sie. Idz, uczyn jakos rzekla: wszakze uczyn mi z tego pierwej podplomyk maly, i przynies mi; potem tez sobie i synowi swemu uczynisz.
\par 14 Albowiem tak powiedzial Pan, Bóg Izraelski: Maka z garnca tego nie bedzie strawiona, ani oliwy z tej banki ubedzie, az do dnia, gdy Pan spusci deszcz na ziemie.
\par 15 I poszla, a uczynila podlug slowa Elijaszowego, i jadla ona i on, i wszystka czeladz jej, az sie wypelnily te dni.
\par 16 Nie byla strawiona maka z onego garnca, ani oliwy z banki ubylo, wedlug slowa Panskiego, które powiedzial przez Elijasza.
\par 17 I stalo sie potem, ze sie rozniemógl syn onej niewiasty, pani domu onego, a byla niemoc jego bardzo ciezka, tak, ze w nim tchu nie zostalo.
\par 18 Przetoz rzekla do Elijasza: Cóz mnie i tobie, mezu Bozy? przyszedles do mnie, abys przywiódlszy na pamiec nieprawosc moje, umorzyl syna mego?
\par 19 I rzekl do niej: Daj mi syna twego; i wziawszy go z lona jej, wniósl go na sale, na której mieszkal, i polozyl go na lozu swojem.
\par 20 I wolal do Pana, a rzekl: Panie, Boze mój, izali tez utrapisz wdowe, u której mieszkam, izes zabil syna jej?
\par 21 A rozciagnawszy sie nad dziecieciem po trzy kroc, wolal do Pana, mówiac: Panie, Boze mój, niechaj sie prosze wróci dusza dzieciatka tego w cialo jego.
\par 22 I wysluchal Pan glos Elijaszowy: i wrócila sie dusza dzieciecia w cialo jego, i ozylo.
\par 23 Tedy wzial Elijasz dziecie, i zniósl je z sali do domu, a oddal go matce jego, i rzekl Elijasz: Wej, syn twój zyje.
\par 24 I rzekla niewiasta do Elijasza: Terazem poznala, izes jest maz Bozy, a slowo Panskie w usciech twoich jest prawda.

\chapter{18}

\par 1 Potem po wielu dniach, mianowicie po onym roku trzecim, stalo sie slowo Panskie do Elijasza, mówiac: Idz, ukaz sie Achabowi; bo spuszcze deszcz na ziemie.
\par 2 Szedl tedy Elijasz, aby sie ukazal Achabowi; a byl glód gwaltowny w Samaryi.
\par 3 I zawolal Achab Abdyjasza, który byl sprawca domu jego. (A Abdyjasz sie bardzo Pana bal;
\par 4 Bo gdy mordowala Jezabel proroki Panskie, tedy wzial Abdyjasz sto proroków, i skryl ich po piecdziesiat do jaskini, i zywil je chlebem i woda.)
\par 5 I rzekl Achab do Abdyjasza: Idz przez ziemie do wszystkich zródel wód, i do wszystkich potoków, aza gdzie znajdziemy trawe, zebysmy zywo zachowali konie i muly, i zebysmy nie zgubili bydla.
\par 6 I rozdzielili sobie ziemie, która przejsc mieli. Achab sam szedl jedna droga, Abdyjasz tez szedl druga droga osobno.
\par 7 A gdy Abdyjasz byl w drodze, oto sie z nim Elijasz spotkal, który gdy go poznal, upadl na oblicze swoje, i rzekl: A tyzes jest pan mój Elijasz?
\par 8 I odpowiedzial mu: Jam jest. Idz, powiedz panu twemu: Oto Elijasz tu jest.
\par 9 Do którego on rzekl: Cózem zgrzeszyl, iz wydawasz sluge twego w rece Achabowe, aby mie zabil?
\par 10 Jako zywy Pan, Bóg twój, ze niemasz narodu, i królestwa, gdzieby nie poslal Pan mój, aby cie szukano; a gdy powiedziano, iz cie niemasz, tedy obowiazal przysiega królestwa i narody, jako cie znalesc nie moga.
\par 11 A ty mi teraz mówisz: Idz, a powiedz panu twemu: Oto Elijasz.
\par 12 I staloby sie, gdybym ja odszedl od ciebie, zeby cie Duch Panski zaniósl, gdziebym nie wiedzial; a ja szedlszy opowiedzialbym Achabowi, a gdyby cie nie znalazl, zabilby mie; a sluga twój boi sie Pana od dziecinstwa swego.
\par 13 Azaz nie powiedziano panu memu, com uczynil, gdy mordowala Jezabel proroki Panskie? zem skryl z proroków Panskich sto mezów, po piecdziesiat mezów w jaskini, i zywilem je chlebem i woda?
\par 14 A ty teraz mówisz: Idz, powiedz panu twemu: Oto Elijasz; i zabije mie.
\par 15 I odpowiedzial Elijasz: Jako zywy Pan zastepów, przed którego oblicznoscia stoje, ze mu sie dzis ukaze.
\par 16 A tak szedl Abdyjasz przeciw Achabowi, i oznajmil mu to. Przetoz szedl Achab przeciw Elijaszowi.
\par 17 A ujrzawszy Achab Elijasza, rzekl Achab do niego: Azaz nie ty jestes, który czynisz zamieszanie w Izraelu?
\par 18 Na co mu odpowiedzial: Nie jac czynie zamieszanie w Izraelu, ale ty i dom ojca twego, gdyz opusciwszy rozkazania Panskie nasladujecie Baalów.
\par 19 Przetoz teraz poslij, a zbierz do mnie wszystkiego Izraela na góre Karmel, i proroków Baalowych cztery sta i piecdziesiat, przytem proroków gajowych cztery sta, którzy jadaja z stolu Jezabeli.
\par 20 Poslal tedy Achab do wszystkich synów Izraelskich, i zebral te proroki na góre Karmel.
\par 21 A przystapiwszy Elijasz do wszystkiego ludu, rzekl: I dlugoz bedziecie chramac na obie strony? Jezli Pan jest Bogiem, idzciez za nim; a jezli Baal, idzciez za nim. I nie odpowiedzial mu lud i slowa.
\par 22 Tedy rzekl Elijasz do ludu: Jam tylko sam zostal prorok Panski; a proroków Baalowych cztery sta i piecdziesiat mezów.
\par 23 Niech nam dadza dwóch cielców, a niech sobie obiora cielca jednego, a porabia go na sztuki, i wloza na drwa; ale ognia niech nie podkladaja: ja tez przygotuje drugiego cielca, którego wloze na drwa, a ognia nie podloze.
\par 24 Potem wzywajcie imienia bogów waszych, a ja bede wzywal imienia Panskiego, a Bóg, który sie ozwie przez ogien, ten niech bedzie Bogiem. Na co odpowiadajac wszystek lud rzekl: Dobrzes powiedzial.
\par 25 I rzekl Elijasz do proroków Baalowych: Obierzcie sobie cielca jednego, a zgotujcie go pierwej, bo was jest wiecej; wzywajciez imienia bogów waszych, ale ognia nie podkladajcie.
\par 26 A tak wzieli cielca, którego im dal, a zgotowawszy wzywali imienia Baalowego od poranku az do poludnia, mówiac: O Baalu, wysluchaj nas! Ale nie bylo glosu, ani ktoby odpowiedzial. I skakali kolo oltarza, który byli uczynili.
\par 27 A gdy bylo poludnie, nasmiewal sie z nich Elijasz, mówiac: Wolajcie wiekszym glosem, poniewaz jest bóg; tylko ze sie albo zamyslil, albo jest zabawny, albo tez jest w drodze; albo tez spi, aza ocuci.
\par 28 A tak wolali glosem wielkim, i rzezali sie wedlug zwyczaju swego nozami i wlóczenkami, az sie krwia oblewali.
\par 29 I stalo sie, gdy minelo poludnie, ze prorokowali az do czasu ofiarowania ofiary sniednej; ale nie bylo glosu, ani ktoby odpowiedzial, ani ktoby wysluchal.
\par 30 Zatem rzekl Elijasz do wszystkiego ludu: Przystapcie do mnie. I przystapil wszystek lud do niego. Tedy naprawil oltarz Panski, który byl rozwalony.
\par 31 Albowiem wzial Elijasz dwanascie kamieni; (wedlug liczby pokolenia synów Jakóbowych, do którego sie stalo slowo Panskie, mówiac: Izrael bedzie imie twoje.)
\par 32 I zbudowal z tego kamienia oltarz w imie Panskie, a uczynil okolo oltarza szeroki rów, coby mógl dwie miary zboza wysiac.
\par 33 Potem ulozyl drwa, i na sztuki porabal cielca, i kladl go na drwa.
\par 34 I rzekl: Napelnijcie cztery wiadra woda, a wylijcie na calopalenie i na drwa. Rzekl nadto: Powtórzcie, i powtórzyli; rzekl jeszcze: Uczyncie po trzecie, i uczynili po trzecie,
\par 35 Tak ze plynely wody okolo oltarza, az i rów byl napelniony woda.
\par 36 I stalo sie, gdy byl czas sprawowania ofiary sniednej, przystapil Elijasz prorok, i rzekl: Panie, Boze Abrahama, Izaaka, i Izraela! dzis niech poznaja, zes ty jest Bogiem w Izraelu, a jam sluga twój, a zem wedlug slowa twego uczynil to wszystko.
\par 37 Wysluchaj mie Panie, wysluchaj mie, aby poznal ten lud, zes ty Panie jest Bogiem, gdybys zas nawrócil serca ich.
\par 38 Tedy spadl ogien Panski, i pozarl calopalenie, i drwa, i kamienie, i proch; a wode, która byla w rowie, wysuszyl.
\par 39 Co gdy ujrzal wszystek lud, upadli na oblicze swe, i rzekli: Pan jest Bogiem, Panci jest Bogiem.
\par 40 Tedy rzekl Elijasz do nich: Pojmajcie proroki Baalowe, a zaden niech z nich nie uchodzi. I pojmano je. A tak odwiódl je Elijasz do potoku Cyson, i tamze je pobil.
\par 41 Potem rzekl Elijasz do Achaba: Idz, jedz, a pij; albowiem oto szum dzdzu wielkiego.
\par 42 Tedy szedl Achab, aby jadl i pil; a Elijasz wstapil na wierzch Karmelu, i polozyl sie na ziemie, a wlozyl twarz swoje miedzy kolana swoje.
\par 43 Potem rzekl do slugi swego: Idz teraz, a spojrzyj ku morzu. Który poszedl, a spojrzawszy rzekl: Niemasz nic. Zasie rzekl: Idz, a wracaj sie po siedm kroc.
\par 44 A za siódmym razem rzekl: Oto oblok maly jako dlon czlowiecza wystepuje z morza. Tedy on rzekl: Idz, a powiedz Achabowi: Zaprzegaj, a ujezdzaj, aby cie deszcz nie zastal.
\par 45 I stalo sie miedzy tem, ze sie niebiosa oblokami i wiatrem zacmily, skad byl deszcz wielki. A tak wsiadlszy Achab, jechal do Jezreela.
\par 46 A reka Panska byla nad Elijaszem; i przepasal biodra swe, i biezal przed Achabem, az przyszedl do Jezreela.

\chapter{19}

\par 1 Tedy oznajmil Achab Jezabeli wszystko, co uczynil Elijasz, a iz prawie wszystkie proroki pomordowal mieczem.
\par 2 Przetoz poslala Jezabela posla do Elijasza, mówiac: To mi niech uczynia bogowie, i to mi niech przyczynia, jezli o tym czasie jutro nie poloze duszy twojej, jako duszy którego z onych.
\par 3 Co gdy wyrozumial Elijasz, wstal i odszedl, aby dusze swa zachowal, a przyszedl do Beerseby, która byla w Judztwie, i zostawil tam sluge swego.
\par 4 A sam poszedl w puszcze na jeden dzien drogi: a gdy przyszedl, i usiadl pod jednym jalowcem, zyczyl sobie umrzec, i rzekl: Dosyc juz, o Panie; wezmijze dusze moje, bom nie jest lepszym nad ojców moich.
\par 5 I polozyl sie, a zasnal pod onym jalowcem, a oto w tenze czas tknal go Aniol i rzekl mu: Wstan, a jedz.
\par 6 A gdy sie obejrzal, oto w glowach jego byl chleb na weglu upieczony i czasza wody. A tak jadl i pil, i polozyl sie znowu.
\par 7 Potem wrócil sie Aniol Panski powtóre, i tknal go, a rzekl: Wstan, jedz, albowiem daleka masz droge przed soba.
\par 8 A tak wstawszy jadl i pil, a szedl w mocy pokarmu onego czterdziesci dni i czterdziesci nocy, az do góry Bozej Horeb.
\par 9 I wszedl tam do jaskini, a przenocowal tam. A oto slowo Panskie do niego, mówiac: Cóz tu czynisz Elijaszu?
\par 10 Który odpowiedzial; Gorliwiem sie zastawial o Pana, Boga zastepów; albowiem synowie Izraelscy opuscili przymierze twoje, oltarze twoje zburzyli, i proroki twoje mieczem pomordowali, a zostalem ja sam, i szukaja duszy mojej, aby mi ja odjeli.
\par 11 Tedy onze glos rzekl: Wynijdz, a stan na górze przed Panem. A oto Pan przechodzil, i wiatr gwaltowny i mocny podwracajacy góry, i lamiacy skaly przed Panem; ale Pan nie byl w onym wietrze. Za wiatrem bylo trzesiemie ziemi; ale nie byl Pan i w onem trzesieniu.
\par 12 Za trzesieniem byl ogien; ale Pan nie byl w ogniu; za ogniem byl glos cichy i wolny.
\par 13 To gdy uslyszal Elijasz, zakryl oblicze swoje plaszczem swoim, a wyszedlszy stanal we drzwiach jaskini. A oto do niego glos mówiacy: Co tu czynisz Elijaszu?
\par 14 A on odpowiedzial: Gorliwiem sie zastawial o Pana, Boga zastepów: albowiem opuscili przymierze twoje synowie Izraelscy, oltarze twoje poburzyli, a proroki twoje mieczem pomordowali, i zostalem ja sam, a szukaja duszy mojej, aby mi ja odjeli.
\par 15 Ale Pan rzekl do niego: idz, wróc sie droga twa na puszcze Damaska, a gdy tam przyjdziesz, pomazesz Hazaela za króla nad Syryja;
\par 16 A Jehu, syna Namsy, pomazesz za króla nad Izraelem, a Elizeusza, syna Safatowego, z Abelmechola, pomazesz za proroka miasto siebie.
\par 17 I stanie sie, ze ktokolwiek ujdzie miecza Hazaelowego, zabije go Jehu, a ktokolwiek ujdzie miecza Jehu, zabije go Elizeusz.
\par 18 Jednakiem sobie zachowal w Izraelu siedm tysiecy, których wszystkich kolana nie klanialy sie Baalowi, i których wszystkich usta nie calowaly go.
\par 19 A tak on odszedlszy stamtad, znalazl Elizeusza, syna Safatowego, a on orze, a dwanascie jarzm wolów przed nim, a sam byl przy dwunastem jarzmie, a idac mimo niego Elijasz, wrzucil nan plaszcz swój.
\par 20 Który opusciwszy woly bierzal za Elijaszem, i rzekl: Niech pocaluje prosze ojca mego, i matke moje, a pójde za toba; któremu rzekl: Idz, a wróc sie zasie, poniewaz widzisz, com ci uczynil.
\par 21 A tak wróciwszy sie do niego wzial pare wolów, i zabil je, a przy drwach z pluga nawarzyl miesa z nich, i dal ludowi, i jedli. A wstawszy szedl za Elijaszem, i sluzyl mu.

\chapter{20}

\par 1 Tedy Benadad, król Syryjski, zebral wszystko wojsko swoje, majac z soba trzydziesci i dwóch królów, przytem jezdne i wozy; a przyciagnawszy oblegl Samaryje i dobywal jej.
\par 2 I wyprawil posly do Achaba, króla Izraelskiego, do onego miasta;
\par 3 I rzekl mu: Tak mówi Benadad: Srebro twoje i zloto twoje mojec jest; takze zony twoje i synowie twoi najcudniejsi moi sa.
\par 4 I odpowiedzial król Izraelski a rzekl: Wedlug slowa twego królu, panie mój, twojem ja, i wszystko, co mam.
\par 5 A wróciwszy sie poslowie do niego, rzekli: Tak powiedzial Benadad, mówiac: Poslalem do ciebie, abyc rzeczono: Srebro twoje, i zloto twoje, i zony twoje, i syny twoje dasz mi.
\par 6 Ale wiedz, ze jutro o tym czasie posle slugi moje do ciebie, którzy wyszperaja dom twój, i domy slug twoich, i wszystko, w czem sie kochasz, w rece swe wezma, i rozbiora.
\par 7 A tak wezwal król Izraelski wszystkich starszych ziemi onej, i rzekl im: Uwazciez prosze, a obaczcie, zec ten zlego szuka; albowiem poslal do mnie po zony moje, i po syny moje, i po srebro moje, i po zloto moje, a nie odmówilem mu.
\par 8 I rzekli do niego oni wszyscy starsi, i wszystek lud: Nie sluchaj ani przyzwalaj.
\par 9 Przetoz odpowiedzial poslom Benadadowym: Powiedzcie królowi, panu memu: wszystko, o cos poslal do slugi twego przedtem, uczynie: ale tej rzeczy uczynic nie moge. A tak poslowie odeszli, i odniesli mu odpowiedz.
\par 10 Znowu poslal do niego Benadad i rzekl: Niech mi to uczynia bogowie, i to niech przyczynia, jezli sie dostanie prochu Samaryi po garsci wszystkiemu ludowi, który za mna idzie.
\par 11 I odpowiedzial król Izraelski, a rzekl: Powiedzcie mu: Niech sie nie chlubi zbrojny, jako ten, który zlozyl zbroje.
\par 12 A gdy Benadad uslyszal to slowo, (a on w tenczas z królmi pil w namiotach,)rzekl do slug swych: Ruszcie sie. I ruszyli sie ku miastu.
\par 13 A oto, niektóry prorok przyszedl do Achaba, króla Izraelskiego, i rzekl: Tak powiada Pan: Izazes nie wiedzial tego wszystkiego wielkiego mnóstwa? Oto Ja je dam w reke twoje dzisiaj, abys wiedzial, zem Ja Pan.
\par 14 Tedy rzekl Achab: Przez kogoz? A on odpowiedzial: Tak mówi Pan: przez slugi ksiazat powiatowych. I rzekl: Którz pocznie bitwe? Tedy mu on odpowiedzial: Ty.
\par 15 Obliczyl tedy slugi ksiazat powiatowych, których bylo dwiescie trzydziesci i dwa; a po nich policzyl wszystek lud, wszystkich synów Izraelskich siedm tysiecy.
\par 16 I wyszli o poludniu. A Benadad pil, i upil sie w namiotach, sam i trzydziesci i dwóch królów, pomocników jego.
\par 17 A tak wyszli sludzy ksiazat powiatowych naprzód. Tedy poslal Benadad, (gdy mu powiedziano, mówiac: Mezowie wyszli z Samaryi,)
\par 18 I rzekl: Chociazby o pokój szli prosic, pojmajcie je zywo; chociazby tez ku bitwie wyszli, zywo je pojmajcie.
\par 19 A gdy oni sludzy ksiazat powiatowych wyszli z miasta, i inne wojsko za nimi,
\par 20 Porazil kazdy meza swego, tak, ze uciekli Syryjczycy, i gonil je Izrael; uciekl tez Benadad, król Syryjski, na koniu i z jezdnymi.
\par 21 Potem wyciagnal król Izraelski, i pobil konie i wozy, a porazil Syryjczyka porazka wielka.
\par 22 Znowu przyszedl prorok do króla Izraelskiego, i rzekl mu: Idz, zmacniaj sie, a wiedz i bacz, co masz czynic; albowiem po roku król Syryjski wyciagnie przeciwko tobie.
\par 23 Tedy sludzy króla Syryjskiego rzekli do niego: Bogowie ich sa bogowie górni, przetoz nas przemogli; a wszakze zwiedzmy z nimi bitwe w polu a ujrzysz, jezli ich nie przemozemy.
\par 24 Przetoz tak uczyn: Odpraw królów, kazdego z miejsca swego, a postanów hetmanów miasto nich.
\par 25 A ty nalicz sobie wojska z swoich, jako bylo wojsko onych, którzy polegli, a koni jako one konie, a wozów, jako one wozy, i stoczymy bitwe z nimi w polu, a ujrzysz, jezli ich nie przemozemy. I usluchal glosu ich, a uczynil tak.
\par 26 A gdy wyszedl rok, obliczyl Benadad Syryjczyki, a ciagnal ku Afeku na wojne przeciw Izraelitom.
\par 27 Synowie Izraelscy takze sa obliczeni, a nabrawszy z soba zywnosci, ciagneli przeciwko nim. I polozyli sie obozem synowie Izraelscy przeciwko nim, jakoby dwa male stadka kóz; a Syryjczycy napelnili ziemie.
\par 28 Tedy przyszedl maz Bozy, i mówil do króla Izraelskiego, a rzekl: Tak mówi Pan: Przeto iz mówili Syryjczycy: Bogiem gór jest Pan, a nie jest Bogiem równim, podam to wszystko mnóstwo wielkie w rece twoje, abyscie wiedzeli, zem Ja Pan.
\par 29 A tak oni lezeli obozem przeciwko nim przez siedm dni. I stalo sie dnia siódmego, ze stoczyli bitwe, i porazili synowie Izraelscy Syryjczyków sto tysiecy pieszych jednegoz dnia.
\par 30 A ostatek uciekli do Afeku miasta, i upadl mur na dwadziescia i siedm tysiecy mezów, co byli pozostali. A Benadad ucieklszy przyszedl do miasta, i skryl sie do najskrytszej komory.
\par 31 Ale mu rzekli sludzy jego: Slychalismy za pewne, ze królowie domu Izraelskiego sa królowie milosierni. Niech wlozymy prosze wory na biodra nasze, i powrozy na glowy nasze, a wynijdziemy do króla Izraelskiego, snac zywo zostwi dusze twoje.
\par 32 Tedy opasali wormi biodra swe, a wlozyli powrozy na glowy swoje i przyszli do króla Izraelskiego, i rzekli: Benadad, sluga twój, mówi: Niech zyje prosze dusza moja! A on rzekl: a zywze jeszcze? Brat to mój.
\par 33 A oni mezowie wziawszy to za dobry znak, i predko uchwyciwszy to slowo od niego, rzekli: Bratci twój Benadad. I rzekl: Idzciez, przywiedzcie go. Przetoz wyszedl do niego Benadad, i kazal mu wsiasc na wóz.
\par 34 I rzekl do niego Benadad: miasta, które wzial ojciec mój ojcu twemu, powróce, a ty ulice sobie poczynisz w Damaszku, jako poczynil ojciec mój w Samaryi. I odpowiedzal: Ja wedlug przymierza puszcze cie wolno. A tak z nim uczynil przmierze i puscil go wolno.
\par 35 Tedy niektóry maz z synów prorockich rzekl do blizniego swego z rozkazania Panskiego: Uderz mie prosze; ale go nie chcial on maz uderzyc.
\par 36 I rzekl mu: Przeto izes nie usluchal glosu Panskiego, oto skoro odejdziesz odemnie, zabije cie lew. A gdy odszedl od niego, znalazl go lew, i zabil go.
\par 37 Potem znalazl drugiego meza, i rzekl mu: Uderz mie prosze; który maz tak go uderzyl, ze go ranil.
\par 38 Szedl tedy on prorok, a zabiezal królowi na drodze, i odmienil sie, zasloniwszy oczy swoje.
\par 39 A gdy król mijal, zawolal na króla, i rzekl; sluga twój wyszedl w posrodek bitwy, a oto maz przyszedlszy przywiódl do mnie meza, i rzekl: Strzez tego meza; bo jezlibys go opuscil, dusza twoja bedzie za dusze jego, albo talent srebra odwazysz.
\par 40 Wtem gdy sie sluga twój zabawil tem i owem, oto on zniknal. Tedy rzekl do niego król Izraelski: Taki jest sad twój, sames sie osadzil.
\par 41 A on zaraz odjal zaslone od oczu swych, i poznal go król Izraelski, ze byl prorokiem.
\par 42 Zatem rzekl do niego: Tak mówi Pan: Poniewazes wypuscil z reki swej meza godnego smierci, dusza twoja bedzie za dusze jego, i lud twój za lud jego.
\par 43 Przetoz odszedl król Izraelski do domu swego smutny i zagniewany, i przyszedl do Samaryi.

\chapter{21}

\par 1 I stalo sie potem: Mial Nabot Jezreelita winnice, która byla w Jezreelu podle palacu Achaba, króla w Samaryi.
\par 2 I rzekl Achab do Nabota, mówiac: Daj mi winnice twoje, abym mial z niej ogród dla jarzyn, albowiem bliska jest domu mego; a dam ci za nie winnice lepsza, nizli ta jest; albo jezlic sie zda, dam ci pieniedzy, ile stoi.
\par 3 I odpowiedzal Nabot Achabowi: Nie daj tego Panie, abym ci mial dac dziedzictwo ojców moich.
\par 4 Tedy przyszedl Achab do domu swego smutny i zagniewany dla slowa, które mu rzekl Nabot Jezreelita, mówiac: Nie dam ci dziedzictwa ojców moich; i ukladl sie na lozu swem, a odwrócil twarz swoje, i nie jadl chleba.
\par 5 Wtem przyszedlszy do niego Jezabela, zona jego, rzekla mu: Przedze duch twój tak smutny, ze nie jesz chleba?
\par 6 I odpowiedzial jej: Przeto zem mówil z Nabotem Jezreelita, i rzeklem mu: Daj mi winnice twoje za pieniadze, albo jezli chcesz, dam ci winnice za nie; ale on odpowiedzial: Nie dam ci winnicy mojej.
\par 7 Tedy rzekla do niego Jezabela, zona jego: I takze ty sprawujesz królestwo Izraelskie? Wstan, jedz chleb, a badz dobrej mysli; ja tobie dam winnice Nabota Jezeelity.
\par 8 A tak napisala list imieniem Achabowem, który zapieczetowala pieczecia jego, i poslala on list do starszych i do przedniejszych, którzy byli w miescie jego, i mieszkali z Nabotem.
\par 9 A napisala on list w ten sposób: Zapowiedzcie post, a posadzcie Nabota miedzy przedniejszymi z ludu;
\par 10 I postawcie dwóch mezów przewrotnych przeciw niemu, którzyby przeciwko niemu swiadczyli, mówiac: Zlozeczyles Bogu i królowi; potem wywiedzcie go, a ukamionujcie go, aby umarl.
\par 11 I uczynili mezowie onego miasta starsi i przedniejsi, którzy mieszkali w onem miescie jego, jako byla wskazala do nich Jezabela, wedlug tego, jako napisano bylo w liscie, który poslala do nich.
\par 12 Zapowiedzieli post, i posadzili Nabota miedzy przedniejszymi z ludu.
\par 13 Potem przyszli dwaj mezowie przewrotni, i usiedli przeciw niemu, a swiadczyli przeciwko niemu oni mezowie przewrotni, to jest przeciw Nabotowi przed ludem, mówiac: Zlozeczyl Nabot Bogu i królowi. I wywiedli go za miasto, i ukamionowali go, i umarl.
\par 14 I poslali do Jezabeli, mówiac: Ukamionowan jest Nabot, i umarl.
\par 15 I stalo sie, gdy uslyszala Jezabela, ze ukamionowany byl Nabot, a iz umarl, rzekla Jezabela do Achaba: Wstan, posiadz winnice Nabota Jezreelity, któryc jej nie chcial dac za pieniadze; albowiem nie zyje Nabot, ale umarl.
\par 16 A tak uslyszawszy Achab, ze umarl Nabot, wstal, a szedl do winnicy Nabota Jezreelity, aby ja posiadl.
\par 17 Tedy sie stalo slowo Panskie do Elijasza Tasbity, mówiac:
\par 18 Wstan, idz przeciw Achabowi, królowi Izraelskiemu, który jest w Samaryi; oto jest na winnicy Nabotowej, do której szedl, aby ja posiadl.
\par 19 I rzeczesz do niego, mówiac: Tak mówi Pan: Azas nie zabil i nie posiadl? Powiedzze mu, mówiac: Tak mówi Pan: Tak jako psy lizali krew Nabotowe, tak tez pewnie psy beda lizac krew twoje.
\par 20 I rzekl Achab do Elijasza: A juzes mie znalazl nieprzyjacielu mój? A on odpowiedzial: Znalazlem; albowiemes sie zaprzedal, abys czynil zlosc przed oczyma Panskiemi.
\par 21 Oto Ja przywiode na cie zle, a odejme potomki twe, i wytrace z domu Achabowego, az do najmniejszego szczeniecia, i wieznia, i opuszczonego w Izraelu.
\par 22 A uczynie z domem twoim, jako z domem Jeroboama, syna Nabatowego, i jako z domem Baazy, syna Ahyjaszowego, dla rozdraznienia, któremes mie do gniewu pobudzil, i przywiodles do grzechu Izraela.
\par 23 Takze i o Jezabeli rzekl Pan, mówiac: Psy zjedza Jezabele miedzy murami Jezreelskimi.
\par 24 Tego, który umrze z domu Achabowego w miescie, psy zjedza, a tego, który umrze na polu, zjedza ptaki powietrzne.
\par 25 Albowiem nie byl nikt jako Achab, któryby siebie samego zaprzedal, aby czynil zle przed oczyma Panskiemi; bo go poduszczala Jezabela, zona jego.
\par 26 Albowiem sie dopuscil rzeczy bardzo obrzydlych, chodzac za balwanami wedlug wszystkiego, jako czynili Amorejczycy, których wygnal Pan przed obliczem synów Izraelskich.
\par 27 A gdy uslyszal Achab te slowa, rozdarl odzienie swoje, a wlozywszy wór na cialo swoje, poscil i lezal w worze, a chodzil po maluczku.
\par 28 I stalo sie slowo Panskie do Elijasza Tesbity, mówiac:
\par 29 Widzialzes, jako sie upokorzyl Achab przed twarza moja? Poniewaz sie tedy upokorzyl przed twarza moja, nie przywiode tego zlego za dni jego; ale za dni syna jego przywiode to zle na dom jego.

\chapter{22}

\par 1 A nie bylo przez trzy lata wojny miedzy Syryjczykami i miedzy Izraelczykami.
\par 2 I stalo sie roku trzeciego, ze przyjechal Jozafat król Judzki, do króla Izraelskiego.
\par 3 Tedy rzekl król Izraelski do slug swoich: Nie wieciez, iz nasze jest Ramot Galaad? A my zaniedbywamy odebrac go z reki króla Syryjskiego.
\par 4 Przetoz rzekl do Jozafata: Pociagnieszze ze mna na wojne przeciwko Ramot Galaad? I rzekl Jozafat do króla Izraelskiego: Jakom ja, tak i ty; jako lud mój, tak lud twój; jako konie moje, tak konie twoje.
\par 5 Nadto rzekl Jozafat do króla Izraelskiego: Spytaj sie prosze dzis slowa Panskiego.
\par 6 A tak zebral król Izraelski proroków okolo czterech set mezów, i rzekl do nich: Mamze ciagnac na wojne przeciwko Ramot Galaad? czy zaniechac? I odpowiedzieli mu: Ciagnij; bo je Pan da w rece królewskie.
\par 7 Ale Jozafat rzekl: Nie maszze tu którego proroka Panskiego, zebysmy sie go pytali?
\par 8 I rzekl król Izraelski do Jozafata: Jest jeszcze maz jeden, przez któregobysmy sie mogli radzic Pana; ale go ja nienawidze, bo mi nic dobrego nie prorokuje, jedno zle, Micheasz, syn Jemla. I rzekl Jozafat: Niech tak nie mówi król.
\par 9 A tak zawolal król Izraelski komornika niektórego, i rzekl: Przywiedz tu rychlo Micheasza, syna Jemlowego.
\par 10 Miedzy tem król Izraelski, i Jozafat, król Judzki, siedzieli na stolicach swoich, ubrani w szaty królewskie, na placu u wrót bramy Samaryjskiej, a wszyscy prorocy prorokowali przed nimi.
\par 11 A Sedechyjasz, syn Chenaana, sprawil sobie rogi zelazne, i rzekl: Tak mówi Pan: Temi bedziesz bódl Syryjczyki, az je wyniszczysz.
\par 12 Takze wszyscy prorocy prorokowali, mówiac: Ciagnij do Ramot Galaad, a bedziec sie szczescilo; albowiem je poda Pan w rece królewskie.
\par 13 Tedy posel, który chodzil, aby przyzwal Micheasza, rzekl do niego, mówiac: Oto teraz slowa proroków jednemi usty dobrze tusza królowi; niechze bedzie prosze slowo twoje, jako slowo jednego z nich, a mów dobre rzeczy.
\par 14 I rzekl Micheasz: Jako zywy Pan, ze co mi kolwiek rzecze Pan, to mówic bede.
\par 15 A gdy przyszedl do króla, rzekl król do niego: Micheaszu, mamyz ciagnac na wojne przeciw Ramot Galaad, czyli zaniechac? A on mu rzekl: Ciagnij, a bedziec sie szczescilo; albowiem je poda Pan w rece królewskie.
\par 16 I rzekl do niego król: A wielez cie razy mam przysiega obowiazac, abys mi nie mówil jedno prawde w imieniu Panskiem?
\par 17 Przetoz rzekl: Widzialem wszystek lud Izraelski rozproszony po górach jako owce, które nie maja pasterza; bo rzekl Pan: Nie maja ci pana; niech sie wróci kazdy do domu swego w pokoju.
\par 18 I rzekl król Izraelski do Jozafata: Izazem ci nie powiadal, ze mi nie mial prorokowac dobrego, ale zle?
\par 19 A Micheasz rzekl: Sluchajze tedy slowa Panskiego: Widzialem Pana siedzacego na stolicy swojej, i wszystko wojsko niebieskie stojace po prawicy jego, i po lewicy jego.
\par 20 I rzekl Pan: Kto zwiedzie Achaba, aby szedl a upadl w Ramot Galaad? A gdy mówil jeden tak, a drugi inaczej;
\par 21 Tedy wystapil duch, i stanal przed Panem, mówiac: Ja go zwiode. A Pan mu rzekl: Przez cóz?
\par 22 Odpowiedzial: Wynijde, a bede duchem klamliwym w ustach wszystkich proroków jego. I rzekl mu Pan: Zwiedziesz, i pewnie przemozesz. Idzze, a czyn tak.
\par 23 Przetoz teraz oto dal Pan ducha klamliwego w usta tych wszystkich proroków twoich, gdyz Pan wyrzekl przeciwko tobie zle.
\par 24 Tedy przystapiwszy Sedechyjasz, syn Chenaana, uderzyl Micheasza w policzek, mówiac: Kiedyz odszedl Duch Panski odemnie, aby z toba mówil?
\par 25 I odpowiedzial Micheasz: Oto ty ujrzysz dnia onego, kiedy wnijdziesz do najskrytszej komory, abys sie skryl.
\par 26 I rzekl król Izraelski: Wezmij Micheasza, a wiedz go do Amona, starosty miejskiego, i do Joasa, syna królewskiego.
\par 27 I rzeczesz: Tak mówi król: Wsadzcie tego meza do wiezienia; a dawajcie mu jesc chleb utrapienia i wode ucisku, az sie wróce w pokoju.
\par 28 Ale odpowiedzial Micheasz: Jezlize sie wrócisz w pokoju, tedyc nie mówil Pan przez mie. Nadto rzekl: Sluchajciez wszyscy ludzie.
\par 29 A tak ciagnal król Izraelski i Jozafat, król Judzki, do Ramot Galaad.
\par 30 I rzekl król Izraelski do Jozafata: Odmienie sie, gdy pójde do bitwy; ale ty ubierz sie w szaty twoje. I odmienil sie król Izraelski, a szedl ku bitwie.
\par 31 A król Syryjski rozkazal byl hetmanom, których bylo trzydziesci i dwa nad wozami jego, mówiac: Nie potykajcie sie ani z malym, ani z wielkim, tylko z samym królem Izraelskim.
\par 32 I stalo sie, gdy ujrzeli Jozafata hetmani, co byli nad wozami, rzekli: Zaprawde to król Izraelski; i obrócili sie przeciwko niemu, chcac sie z nim potykac; ale Jozafat zawolal.
\par 33 Wtem obaczywszy hetmani, co byli nad wozami, ze nie ten byl król Izraelski, odwrócili sie od niego.
\par 34 Lecz maz niektóry strzelil z luku na niepewne, i postrzelil króla Izraelskiego miedzy nity i miedzy pancerz; który rzekl woznicy swemu: Nawróc, a wywiez mie z wojska; bom jest raniony.
\par 35 I wzmogla sie bitwa dnia onego, a król stal na wozie przeciw Syryjczykom: potem umarl w wieczór, a krew ciekla z rany jego na wóz.
\par 36 Tedy wolal wozny w wojsku, gdy juz slonce zachodzilo, mówiac: Wróc sie kazdy do miasta swego i kazdy do ziemi swojej.
\par 37 A tak umarl król, a odwiezion jest do Samaryi, i pochowano go w Samaryi.
\par 38 A gdy umywano wóz w sadzawce Samaryjskiej, lizali psy krew jego, takze gdy umywano zbroje jego: wedlug slowa Panskiego, które byl powiedzial.
\par 39 A inne sprawy Achabowe i wszystko, co czynil, i dom z kosci sloniowych, który zbudowal, wszystkie tez miasta, które pobudowal, azaz to nie jest spisane w kronikach o królach Izraelskich?
\par 40 I zasnal Achab z ojcami swymi, a królowal Ochozyjasz, syn jego, miasto niego.
\par 41 A Jozafat, syn Azy, poczal królowac nad Juda czwartego roku za panowania Achaba, króla Izraelskiego.
\par 42 A Jozafat mial trzydziesci i piec lat, gdy królowac poczal, a dwadziescia i piec lat królowal w Jeruzalemie; a imie matki jego bylo Azuba, córka Salajowa.
\par 43 I chodzil po wszystkiej drodze Azy, ojca swego, a nie odchylal sie od niej, czyniac to, co bylo dobrego przed oczyma Panskiemi.
\par 44 Wszakze iz wyzyn nie poburzyli, jeszcze lud ofiarowal i kadzil po wyzynach.
\par 45 Uczynil tez pokój Jozafat z królem Izraelskim.
\par 46 A inne sprawy Jozafatowe, i moc jego, której dokazywal, i jako walczyl, azaz to nie jest napisane w kronikach królów Judzkich?
\par 47 Ten wyplenil z ziemi ostatek Sodomczyków, którzy byli pozostali za dni Azy, ojca jego.
\par 48 Na ten czas nie bylo króla w Edomskiej ziemi; tylko starosta byl miasto króla.
\par 49 I nasprawial Jozafat okretów na morze, aby chodzily do Ofir po zloto. Ale nie doszly; bo sie rozbily one okrety w Asyjon Gaber.
\par 50 Rzekl takze byl Ochozyjasz, syn Achaba, do Jozafata: Niech jada sludzy moi z slugami twymi w okretach. Ale niechcial Jozafat.
\par 51 Zasnal tedy Jozafat z ojcami swymi, i pochowany jest z ojcami swymi w miescie Dawida, ojca swego; a królowal Joram, syn jego, miasto niego.
\par 52 Ochozyjasz, syn Achaba, poczal królowac nad Izraelem w Samaryi roku siedmnastego Jozafata, króla Judzkiego, i królowal nad Izraelem dwa lata.
\par 53 I czynil zle przed oczyma Panskiemi, chodzac droga ojca swego, i droga matki swej, i droga Jeroboama, syna Nabatowego, który przywiódl do grzechów Izraela.
\par 54 Sluzyl takze Baalowi, a klanial mu sie, i pobudzal do gniewu Pana, Boga Izraelskiego, wedlug wszystkiego, co czynil ojciec jego.


\end{document}