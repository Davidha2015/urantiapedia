\begin{document}

\title{2 Kronik}


\chapter{1}

\par 1 Zmocnil sie tedy Salomon, syn Dawidowy, w królestwie swem, a Pan, Bóg jego byl z nim, i uwielbil go wysoce.
\par 2 I przykazal Salomon wszystkiemu Izraelowi, pólkownikom, rotmistrzom, i sedziom, takze wszystkim przelozonym nad wszystkim Izraelem, i przedniejszym domów ojcowskich.
\par 3 I szedl Salomon i wszystko zgromadzenie z nim na wyzyne, która byla w Gabaonie; albowiem tam byl namiot zgromadzenia Bozego, który sprawil Mojzesz sluga Panski, na puszczy.
\par 4 (Ale skrzynie Boza przeniósl byl Dawid z Karyjatyjarym, nagotowawszy jej miejsce; bo jej byl namiot rozbil w Jeruzalemie.)
\par 5 Oltarz tez miedziany, który byl urobil Besaleel, syn Urowy, syna Hurowego, byl tam przed przybytkiem Panskim, gdzie Pana szukal Salomon, i wszystko zgromadzenie.
\par 6 I przystapil tam Salomon do oltarza miedzianego, który byl przed namiotem zgromadzenia i ofiarowal na nim ofiar palonych tysiac.
\par 7 Onejze nocy ukazal sie Bóg Salomonowi, i rzekl do niego: Pros czego chcesz, a dam ci.
\par 8 Tedy rzekl Salomon do Boga: Tys uczynil z Dawidem, ojcem moim, milosierdzie wielkie, i postanowiles mie królem miasto niego.
\par 9 A teraz, o Panie Boze! niech bedzie stale slowo twoje, któres mówil do Dawida, ojca mego; bos ty mie uczynil królem nad ludem wielkim, który jest jako proch ziemi.
\par 10 Przetoz daj mi madrosc i umiejetnosc, abym wychodzil i wchodzil przed tym ludem: albowiem któz jest, coby mógl sadzic ten lud twój tak wielki.
\par 11 Tedy rzekl Bóg do Salomona: Dlatego, izes to mial w sercu swem, a nie prosiles o bogactwa, o majetnosci, i o slawe, anis prosil o wytracenie tych, co cie nienawidza, anis tez prosil o dlugie zycie, ales sobie prosil o madrosc i umiejetnosc, abys sadzil lud mój, nad którymem cie postanowil królem:
\par 12 Madrosc i umiejetnosc dana jest tobie; do tego dam ci bogactwa, i majetnosc, i slawe, tak, ze zaden nie byl tobie równy z królów, którzy byli przed toba, i po tobie nie bedzie tobie równy.
\par 13 I wrócil sie Salomon od onej wyzyny, która byla w Gabaonie, do Jeruzalemu, od namiotu zgromadzenia, i królowal nad Izraelem.
\par 14 A nazbieral Salomon wozów i jezdnych, i mial tysiac i cztery sta wozów, i dwanascie tysiecy jezdnych, których rozsadzil po miastach wozów, i przy sobie w Jeruzalemie.
\par 15 I zlozyl król w Jeruzalemie zlota i srebra, jako kamienia, a cedrów zlozyl jako sykomorów, których na polu rosnie bardzo wiele.
\par 16 I przywodzono konie Salomonowi z Egiptu, i rozliczne towary; bo kupcy królewscy brali towary rozliczne za pewne pieniadze.
\par 17 A wychodzili i przywodzili z Egiptu cug wozników za szesc set srebrników, a konia za sto i za piecdziesiat. A tak wszyscy królowie Hetejscy, i królowie Syryjscy z rak ich koni dostawali.

\chapter{2}

\par 1 Umyslil tedy Salomon budowac dom imieniowi Panskiemu, i palac królewski dla siebie.
\par 2 I naliczyl Salomon siedmdziesiat tysiecy mezów, co nosili ciezary, a osmdziesiat tysiecy mezów, którzy rabali drzewo na górze a do nich przystawów trzy tysiace i szesc set.
\par 3 Wyprawil tez Salomon do Hirama, króla Tyrskiego, mówiac: Jakos sie obchodzil z Dawidem, ojcem moim, posylajac mu drzewo cedrowe, aby zbudowal sobie dom do mieszkania, tak sie obejdz ze mna.
\par 4 Oto ja chce budowac dom imieniowi Pana, Boga mego, abym mu go poswiecil, a izbym kadzil przed nim rzeczami wonnemi, i dla ustawicznego pokladania chleba, i dla calopalenia porannego, i wieczornego w sabaty, i na nowiu miesiaców, i w swieta uroczys te Pana, Boga naszego, co ma byc na wieki w Izraelu.
\par 5 A dom, który budowac mam, wielki bedzie; albowiem wiekszy jest Bóg nasz nad wszystkich bogów.
\par 6 Acz któz tak wiele przemoze, aby mu mógl dom zbudowac? poniewaz go niebiosa, i nieba niebios ogarnac nie moga; a ja cózem jest, zebym mu dom budowac mial? chyba tylko dla kadzenia przed nim.
\par 7 Przetoz teraz poslij mi meza umiejetnego, coby umial robic zlotem, i srebrem, i miedzia, i zelazem, i szarlatem, i karmazynem, i hijacyntem, a coby umial rysowac i rzezac z innymi umiejetnymi, którzy przy mnie sa w Judzie i w Jeruzalemie, których sporzadzil Dawid, ojciec mój.
\par 8 Poslij mi tez drzewa cedrowego, jodlowego, i almugimowego z Libanu; bo ja wiem, iz sludzy twoi umieja wyrabywac drzewo na Libanie; a oto sludzy moi beda z slugami twoimi,
\par 9 Aby mi wygotowali drzewa co najwiecej; bo dom, który ja budowac chce, wielki ma byc na podziw.
\par 10 A oto robotnikom slugom twoim, którzy maja wycinac drzewo, dam pszenicy meltej korcy dwadziescia tysiecy, i jeczmienia korcy dwadziescia tysiecy, i wina wiader dwadziescia tysiecy, i oliwy barel dwadziescia tysiecy.
\par 11 Tedy odpowiedzial Hiram, król Tyrski, przez pisanie, które poslal do Salomona: Iz umilowal Pan lud swój, postanowil cie nad nim królem.
\par 12 I przydal Hiram mówiac: Blogoslawiony Pan, Bóg Izraelski, który uczynil niebo i ziemie, który dal Dawidowi królowi syna madrego, i umiejetnego, rozumnego, i roztropnego, aby budowal dom Panu, i palac królewski dla siebie.
\par 13 Poslalem ci tedy meza madrego, i umiejetnego, i roztropnego, Chirama Abijego.
\par 14 Syna niewiasty z córek Danowych, którego ojciec byl obywatel Tyrski; który umial robic zlotem, srebrem, miedzia, zelazem, kamieniem, i drzewem, i szarlatem, hijacyntem ze lnu subtelnego, i jedwabiu karmazynowego; który umie rzezac wszelkie rzezanie, i wymyslic rozmaita misterna robote, która mu zadadza z madrymi twymi, i z madrymi pana mego Dawida, ojca twego.
\par 15 Pszenice tylko, i jeczmien, oliwe, i wino, co obiecal pan mój, niech przysle slugom swym.
\par 16 A my nawycinamy drzewa z Libanu, ilec go bedzie potrzeba, i spuscimyc na tratwach po morzu do Joppy, a ty je kazesz zwiesc do Jeruzalemu.
\par 17 Obliczyl tedy Salomon wszystkich cudzoziemców, którzy byli w ziemi Izraelskiej po onem policzeniu, którem ich policzyl Dawid, ojciec jego, i znalazlo sie ich sto i piecdziesiat tysiecy, i trzy tysiace i szesc set.
\par 18 A postanowil z nich siedmdziesiat tysiecy, co ciezary nosili, a osmdziesiat tysiecy tych, co wyrabywali na górze, a trzy tysiace i szesc set przystawów nad robotami ludu.

\chapter{3}

\par 1 Poczal tedy Salomon budowac dom Panski w Jeruzalemie na górze Moryja, która byla ukazana Dawidowi, ojcu jego, na miejscu, które zgotowal Dawid na bojewisku Ornana Jebuzejczyka.
\par 2 A poczal go budowac miesiaca wtórego, dnia wtórego, roku czwartego królestwa swego.
\par 3 Tenci jest pomiar Salomonowy, wedlug którego budowal dom Bozy, wdluz lokci szescdziesiat, lokci podlug miary pierwszej, wszerz lokci dwadziescia.
\par 4 Przysionek zasie, który byl przed ona dlugoscia i przed szerokoscia domu, byl na dwadziescia lokci, a na wzwyz sto i dwadziescia lokci; który oblozyl wewnatrz szczerem zlotem.
\par 5 A dom wielki okryl drzewem jodlowem, który tez obil szczerem zlotem, i dal po wierzchu naczynic palm i lancuszków.
\par 6 Nakryl tez dom kamieniem drogim ozdobnie, a zloto bylo zloto parwaimskie.
\par 7 Nadto powlekl dom, balki, podwoje, i sciany jego, i drzwi jego zlotem, a wyryl Cherubiny na scianach.
\par 8 Sprawil tez dom swiatnicy najswietszej, którego dlugosc byla wedlug szerokosci domu na dwadziescia lokci, a szerokosc jego na dwadziescia lokci, i powlekl go zlota szczerego szescia set talentów;
\par 9 Gwozdzie tez wazyly piecdziesiat syklów zlota, takze i sale powlekl zlotem.
\par 10 Sprawil tez w domu swiatnicy najswietszej dwa Cherubiny robota misterna, i oprawil je zlotem.
\par 11 A skrzydla Cherubinów byly wdluz na dwadziescia lokci; skrzydlo jedno na piec lokci, a dosiegalo sciany domu; drugie takze skrzydlo na piec lokci dosiegalo skrzydla Cherubina drugiego.
\par 12 Takze skrzydlo Cherubina drugiego na piec lokci dosiegalo sciany domu, a skrzydlo drugie na piec lokci skrzydla Cherubina drugiego.
\par 13 A tak skrzydla onych Cherubinów rozszerzone byly na dwadziescia lokci; a oni stali na nogach swych, a twarze ich byly w dom obrócone.
\par 14 Sprawil tez zaslone z hijacyntu, i z szarlatu, i z jedwabiu, i z subtelnego lnu, i dal wyhaftowac na niej Cherubiny.
\par 15 Uczynil tez przed domem dwa slupy na trzydziesci i na piec lokci wzwyz, a galka, która byla na wierzchu ich, kazda byla na piec lokci.
\par 16 Sprawil tez lancuszki, jako w swiatnicy, a przyprawil je na wierzch onych slupów; sprawil tez sto jablek granatowych, które wprawil miedzy one lancuszki.
\par 17 I postawil one slupy przed kosciolem, jeden po prawej a drugi po lewej stronie; i nazwal imie tego, co byl na prawej stronie, Jachyn, a imie tego, co byl na lewej stronie, Boaz.

\chapter{4}

\par 1 Uczynil tez oltarz miedziany na dwadziescia lokci wdluz, i na dwadziescia lokci wszerz, a na dziesiec lokci wzwyz.
\par 2 Dal tez urobic i morze odlewane na dziesiec lokci od jednego brzegu az do drugiego brzegu, okragle w okolo, a na piec lokci wysokosc jego, a okrag jego byl na trzydziesci lokci w okolo.
\par 3 A pod niem zewszad w okolo byly podobienstwa wolów, których bylo dziesiec w lokciu, a otaczaly morze w okolo; a byly dwa rzedy tych wolów odlanych pospolu z morzem.
\par 4 A stalo na dwunastu wolach, z których trzy patrzaly na pólnocy, a trzy patrzaly na zachód slonca, a trzy patrzaly na poludnie, a trzy patrzaly na wschód slonca; a morze stalo na nich na wierzchu, a wszystkie zadki ich byly pod morzem.
\par 5 A bylo miazsze na dloni; a brzeg jego byl jako kraje u kubka naksztalt kwiatu lilijowego, a bralo w sie trzy tysiace wiader.
\par 6 Przy tem uczynil wanien dziesiec, i postawil ich piec po prawej a piec po lewej stronie, do umywania z nich; wszystko, co nalezalo na calopalenie, obmywano z nich; ale morze bylo, izby sie kaplani z niego umywali.
\par 7 Uczynil tez swieczników zlotych dziesiec na ten ksztalt, jako byc mialy, i postawil je w kosciele, piec po prawej a piec po lewej stronie.
\par 8 Uczynil tez stolów dziesiec, które postawil w kosciele, piec po prawej a piec po lewej stronie; uczynil tez czasz zlotych sto.
\par 9 Zbudowal tez sien kaplanska, i sien wielka, i drzwi u onej sieni, a drzwi ich obil miedzia.
\par 10 A morze postawil po prawej stronie na wschód slonca ku poludniowej stronie.
\par 11 Poczynil tez Chiram kotly, i miotly, i miednice.
\par 12 A tak dokonczyl Chiram roboty, która uczynil Salomonowi królowi do domu Bozego, to jest, slupy dwa z okregami i z galkami na wierzchu onych dwóch slupów, i dwie siatki, które okrywaly one dwie galki okragle, co byly na wierzchu slupów.
\par 13 Nadto jablek granatowych cztery sta do onych dwóch siatek, które dwa rzedy jablek granatowych byly na kazdej siatce, aby okrywaly one dwie galki okragle, które byly na wierzchu slupów.
\par 14 Porobil takze podstawki, a wanny postawil na podstawkach;
\par 15 Morze jedno, a wolów dwanascie pod niem.
\par 16 Do tego kotly, i miotly, i wszystkie naczynia ich porobil Chiram Abi królowi Salomonowi do domu Panskiego z miedzi czystej.
\par 17 Na równinach Jordanskich zlewal je król w ilowatej ziemi, miedzy Sochotem i miedzy Saredata.
\par 18 A tak nasprawial Salomon naczynia tego wszystkiego bardzo wiele, tak iz wagi miedzi nie dochodzono.
\par 19 Sprawil takze Salomon wszystko naczynie, które nalezalo do domu Bozego, jako oltarz zloty, i stoly, na których bywaly chleby pokladne.
\par 20 Takze swieczniki, i lampy ich z szczerego zlota, aby je rozswiecano wedlug obyczaju przed swiatnica;
\par 21 I kwiaty, i lampy, i nozyczki zlote, które zloto bylo wyborne;
\par 22 Naczynia tez muzyczne, i miednice, i lyzki, i kadzielnice ze zlota szczerego, i brame do domu, drzwi wnetrzne do swiatnicy najswietszej, i drzwi do domu, to jest do kosciola ze zlota.

\chapter{5}

\par 1 A tak dokonczona jest wszystka robota, która sprawil Salomon do domu Panskiego, i wniósl tam Salomon rzeczy, które byl poswiecil Dawid ojciec jego: srebro, i zloto, i wszystkie naczynia wlozyl do skarbów domu Bozego.
\par 2 Tedy zebral Salomon starszych z Izraela, i wszystkich przedniejszych z kazdego pokolenia, i przedniejszych z ojców synów Izraelskich, do Jeruzalemu, aby przeniesli skrzynie przymierza Panskiego z miasta Dawidowego, które jest Syon.
\par 3 I zebrali sie do króla wszyscy mezowie Izraelscy w swieto uroczyste, które bywa miesiaca siódmego.
\par 4 A gdy sie zeszli wszyscy starsi z Izraela, wzieli Lewitowie skrzynie;
\par 5 I niesli skrzynie, i namiot zgromadzenia, i wszystkie naczynia swiete, które byly w namiocie, przeniesli je kaplani i Lewitowie.
\par 6 Zatem król Salomon, i wszystko zgromadzenie Izraelskie, co sie byli zeszli do niego przed skrzynie, ofiarowali owce i woly, których nie mozna obliczyc, ani obrachowac przez mnóstwo.
\par 7 Wniesli tedy kaplani skrzynie przymierza Panskiego na miejsce jej, do wnetrznego domu, to jest do swiatnicy najswietszej pod skrzydla Cherubinów;
\par 8 Albowiem Cherubinowie mieli rozciagnione skrzydla nad miejscem skrzyni, i okrywali Cherubinowie skrzynie, i drazki jej z wierzchu.
\par 9 I powyciagali one drazki, tak ze konce ich bylo widac z skrzyni na przodku swiatnicy; ale ich nie widac bylo zewnatrz, i tamze zostaly az do dnia tego.
\par 10 A nic nie bylo w skrzyni, tylko dwie tablice, które tam byl wlozyl Mojzesz na Horebie, gdy stanowil przymierze Pan z synami Izraelskimi po wyjsciu ich z Egiptu.
\par 11 A gdy wychodzili kaplani z swiatnicy, (bo wszyscy kaplani, ile ich bylo, poswiecili sie byli, a nie przestrzegali porzadków.)
\par 12 Stali Lewitowie spiewacy, i wszyscy, którzy byli przy Asafie, Hemmanie, i Jedytunie, i synowie ich, i bracia ich, obleczeni bedac w szaty bisiorowe, z cymbalami i z harfami i z cytrami, stali mówie na wschodniej stronie oltarza, a przy nich kaplanów sto i dwadziescia trabiacych w traby.
\par 13 I stalo sie, gdy jednostajnie trabili, i spiewali, i wydawali jednaki glos, chwalac i slawiac Pana: i gdy podnosili glos na trabach, i na cymbalach, i na innych instrumentach muzycznych, i chwalili Pana, ze dobry, ze na wieki milosierdzie jego, tedy dom on napelniony jest oblokiem, to jest dom Panski.
\par 14 Tak iz sie nie mogli kaplani ostac, i sluzyc dla onego obloku; albowiem napelnila byla chwala Panska dom Bozy.

\chapter{6}

\par 1 Tedy rzekl Salomon: Pan powiedzial, iz mieszkac mial we mgle.
\par 2 A jam zbudowal dom na mieszkanie tobie, o Panie! i miejsce, gdziebys mial mieszkac na wieki.
\par 3 A obróciwszy król oblicze swe blogoslawil wszystkiemu zgromadzeniu Izraelskiemu, (a wszystko zgromadzenie Izraelskie stalo.)
\par 4 I rzekl: Blogoslawiony Pan, Bóg Izraelski, który mówil usty swemi do Dawida, ojca mego, i wypelnil to skutecznie, mówiac:
\par 5 Ode dnia, któregom wywiódl lud mój z ziemi Egipskiej, nie obralem miasta ze wszystkich pokolen Izraelskich ku zbudowaniu domu, gdzieby przebywalo imie moje, anim obral meza, któryby byl wodzem nad ludem moim Izraelskim.
\par 6 Alem obral Jeruzalem, aby tam przebywalo imie moje; obralem tez i Dawida, aby byl nad ludem moim Izraelskim.
\par 7 Postanowil byl Dawid ojciec mój, w sercu swem, zbudowac dom imieniowi Pana, Boga Izraelskiego.
\par 8 Ale rzekl Pan do Dawida, ojca mego: Aczkolwiekes byl postanowil w sercu swem, zbudowac dom imieniowi memu, i dobrzes uczynil, zes to umyslil w sercu swem:
\par 9 Wszakze ty nie bedziesz budowal tego domu; ale syn twój, który wynijdzie z biódr twych, ten zbuduje dom imieniowi memu.
\par 10 A tak utwierdzil Pan slowo swoje, które byl powiedzial; bom ja powstal miasto Dawida, ojca mego, a usiadlem na stolicy Izraelskiej, jako byl powiedzial Pan, zbudowalem ten dom imieniowi Pana, Boga Izraelskiego.
\par 11 Tamzem tez postawil skrzynie ona, w której jest przymierze Panskie, które uczynil z synami Izraelskimi.
\par 12 Tedy stanal Salomon przed oltarzem Panskim, przed wszystkiem zgromadzeniem Izraelskiem, i wyciagnal rece swe.
\par 13 Albowiem byl uczynil Salomon stolec miedziany, który postawil w posród sieni, na piec lokci wdluz, a na piec lokci wszerz, a na trzy lokcie wzwyz; i wstapil nan a pokleknawszy na kolana swoje przed wszystkiem zgromadzeniem Izraelskiem, wyciagnal rece swe ku niebu,
\par 14 I rzekl: Panie, Boze Izraelski! nie masz tobie podobnego Boga na niebie i na ziemi, który chowasz umowe i milosierdzie nad slugami twymi, którzy chodza przed toba calem sercem swem;
\par 15 Którys spelnil sludze twemu Dawidowi, ojcu memu, cos powiedzial; i cos mówil usty twemi, tos skutecznie wypelnil, jako sie to dzis pokazuje.
\par 16 Przetoz teraz, o Panie, Boze Izraelski! speln sludze twemu Dawidowi, ojcu memu, cos mu byl powiedzial, mówiac: Nie bedzie odjety maz z narodu twego przed twarza moja, aby nie mial siedziec na stolicy Izraelskiej, jezli tylko przestrzegac beda syno wie twoi drogi swej chodzac w zakonie moim, jakos ty chodzil przed oblicznoscia moja.
\par 17 Przetoz teraz, o Panie, Boze Izraelski! niech bedzie utwierdzone slowo twoje, któres mówil do slugi twego Dawida.
\par 18 (Aczci wprawdzie, izali Bóg bedzie mieszkal z czlowiekiem na ziemi? Oto niebiosa, i nieba niebios nie moga cie ogarnac, jakoz daleko mniej ten dom, którym zbudowal?)
\par 19 A wejrzyj na modlitwe slugi twego, i na prosbe jego, o Panie, Boze mój! wysluchaj wolanie i modlitwe, która sie modli sluga twój przed toba;
\par 20 Aby byly oczy twoje otworzone nad tym domem we dnie i w nocy; nad tem miejscem, o któremes powiedzial, ze tu przebywac bedzie imie twoje: abys wysluchiwal modlitwe, która sie bedzie modlil sluga twój na tem miejscu.
\par 21 Wysluchajze tedy prosbe slugi twego, i ludu twego Izraelskiego, którac sie modlic beda na tem miejscu; ty wysluchaj z miejsca mieszkania twego, z nieba, a wysluchawszy badz milosciw.
\par 22 Gdyby czlowiek zgrzeszyl przeciwko blizniemu swemu, a przywiódlby go do przysiegi, tak zeby przysiegac musial, a przyszlaby przysiega ona przed oltarz twój w tym domu:
\par 23 Ty wysluchaj z nieba, a rozeznaj i rozsadz slug twoich, potepiajac niezboznego, a obracajac droge jego na glowe jego, i usprawiedliwiajac sprawiedliwego, a oddawajac mu wedlug sprawiedliwosci jego.
\par 24 Gdyby tez byl porazony lud twój Izraelski od nieprzyjaciela, przeto, iz zgrzeszyli przeciwko tobie, a nawróciliby sie, wyznawajac imie twoje, i modlac sie przepraszaliby cie w tym domu:
\par 25 Ty wysluchaj z nieba, a odpusc grzech ludowi twemu Izraelskiemu, a przywróc ich zasie do ziemi, któras im dal i ojcom ich.
\par 26 Takze gdyby zawarte bylo niebo, i nie byloby deszczu, przeto, ze zgrzeszyli tobie, a modliliby sie na tem miejscu, wyznawajac imie twoje, i od grzechu swego odwróciliby sie, gdybys ich utrapil:
\par 27 Ty wysluchaj z nieba, a odpusc grzech slug twoich, i ludu twego Izraelskiego, nauczywszy ich drogi prawej, po której chodzic maja, a daj deszcz na ziemie twoje, któras dal ludowi twemu w dziedzictwo.
\par 28 Bylliby glód na ziemi, bylliby mór, susza, i rdza, szarancza i chrzaszcze; jezliby go scisnal nieprzyjaciel jego w ziemi mieszkania jego, albo jakakolwiek plaga, albo jakakolwiek niemoc:
\par 29 Wszelka modlitwe, i wszelka prosbe, któraby czynil którykolwiek czlowiek, albo wszystek lud twój Izraelski, poznawszy kazdy z nich karanie swoje, i bolesc swoje, i podnióslby rece swe w tym domu:
\par 30 Ty wysluchaj z nieba, z miejsca mieszkania twego, a odpusc i oddaj kazdemu wedlug wszystkich dróg jego, które znasz w sercu jego, (bo ty, ty sam znasz serca synów ludzkich.)
\par 31 Aby sie ciebie bali, i chodzili drogami twemi po wszyskie dni, których zyc beda na ziemi, któras dal ojcom naszym.
\par 32 Nadto i cudzoziemiec, który nie jest z ludu twego Iraelskiego, i ten przyjdzieli z ziemi dalekiej dla imienia twego wielkiego, i reki twej moznej, i dla ramienia twego wyciagnionego; przyjdali a modlic sie beda w tym domu:
\par 33 Ty wysluchaj z nieba, z miejsca mieszkania twego, i uczyn to wszystko, o co zawola do ciebie on cudzoziemiec, aby poznali wszyscy narodowie ziemscy imie twoje, i bali sie ciebie jako lud twój Izraelski, i wiedzieli, ze imie twoje wzywane jest nad domem tym, którym zbudowal.
\par 34 Gdyby wyszedl lud twój na wojne przeciwko nieprzyjaciolom swoim, droga, którabys ich poslal, i modlilicby sie, obróciwszy sie ku temu miastu, któres obral, i ku domowi, którym zbudowal imieniowi twemu:
\par 35 Wysluchajze z nieba modlitwy ich, i prosbe ich, a wykonaj sad ich.
\par 36 Gdyby zgrzeszyli przeciwko tobie, (bo nie masz czlowieka coby nie grzeszyl) a rozgniewawszy sie na nich, podalbys ich pod moc nieprzyjacielowi, któryby ich pojmawszy zawiódl ich w niewole do ziemi dalekiej, albo bliskiej;
\par 37 A upamietaliby sie w sercu swojem w onej ziemi, do której sa zaprowadzeni w niewole, a nawróciwszy sie modliliby sie w ziemi niewoli swojej, mówiac: Zgrzeszylismy zlesmy uczynili, i niepobozniesmy sie sprawowali;
\par 38 A nawróciliby sie do ciebie z calego serca swego, i z calej duszy swej w ziemi niewoli swojej, do której bedac pojmani zaprowadzeni byli, a modliliby sie, obróciwszy sie ku ziemi swej, któras dal ojcom ich, i ku miastu, któres obral, i ku domowi, którym zbudowal imieniowi twemu:
\par 39 Wysluchajze z nieba, z miejsca mieszkania twego, modlitwe ich i prosbe ich, a wykonaj sad ich, i odpusc ludowi twemu, który tobie zgrzeszyl.
\par 40 A tak teraz, o Boze! prosze, niech beda oczy twoje otworzone, i uszy twoje naklonione ku modlitwie, uczynionej na tem miejscu.
\par 41 Teraz tedy powstan, o Panie Boze! ku odpocznieniu twemu, ty i skrzynia mocy twojej; kaplani twoi, o Panie Boze! niech beda obleczeni zbawieniem, a swieci twoi niechaj sie w dobrach raduja.
\par 42 O Panie Boze! nie odwracaj oblicza od pomazanca twego; pamietaj na milosierdzie obiecane Dawidowi, sludze twemu.

\chapter{7}

\par 1 A gdy dokonczyl Salomon modlitwy, tedy ogien zstapil z nieba, i pozarl calopalenie i inne ofiary, a chwala Panska napelnila on dom.
\par 2 I nie mogli kaplani wnijsc do domu Panskiego, przeto, ze chwala Panska napelnila dom Panski.
\par 3 I wszyscy synowie Izraelscy, widzac, gdy zstepowal ogien, i chwala Panska na dom, upadli twarza swa na ziemie, na tlo, a klaniajac sie chwalili Pana, ze dobry, ze na wieki milosierdzie jego.
\par 4 A król i wszystek lud sprawowali ofiary przed Panem.
\par 5 Tedy nabil król Salomon na ofiary wolów dwadziescia i dwa tysiace, a owiec sto i dwadziescia tysiecy, gdy poswiecali dom Bozy król i wszystek lud.
\par 6 Ale kaplani stali w rzedach swych: Lewitowie takze z instrumentami muzyki Panskiej, które byl sprawil Dawid król ku chwaleniu Pana, (ze na wieki milosierdzie jego) piesnia Dawidowa, która im podal. Inni tez kaplani trabili przeciwko nim, a wszyscy Izraelczycy stali.
\par 7 Nadto poswiecil Salomon posrodek sieni, która byla przed domem Panskim; bo tam ofiarowal calopalenia, i tlustosci spokojnych ofiar, przeto, ze na oltarzu miedzianym, który byl sprawil Salomon, nie mogly sie zmiescic calopalenia, i ofiary sniedne i tlustosci.
\par 8 I obchodzil Salomon onego czasu swieto uroczyste przez siedm dni, i wszystek Izrael z nim, zgromadzenie bardzo wielkie, od wejscia do Emat az do rzeki Egipskiej.
\par 9 (Potem uczynili dnia ósmego swieto; albowiem poswiecenie oltarza sprawowali przez siedm dni, i swieto uroczyste obchodzili przez siedm dni.
\par 10 A dnia dwudziestego i trzeciego, miesiaca siódmego, rozpuscil lud do przybytków swoich, weselacy sie i cieszacy sie w sercu swem z dobrodziejstwa, które uczynil Pan Dawidowi i Salomonowi i Izraelowi, ludowi swemu.
\par 11 A tak dokonczyl Salomon domu Panskiego, i domu królewskiego, a wszystko, co byl umyslil w sercu swem, uczynic w domu Panskim i w domu swym, wykonal szczesliwie.
\par 12 Potem ukazal sie Pan Salomonowi w nocy, i rzekl do niego: Wysluchalem modlitwe twoje, i obralem to miejsce sobie za dom do ofiary.
\par 13 Jezli zamkne niebo, zeby nie bylo deszczu, i jezli rozkaze szaranczy, aby pozarla ziemie; jezli tez posle powietrze na lud mój;
\par 14 A jezliby sie upokorzyl lud mój, nad którym wzywano imienia mego, a modlilby sie, i szukalby twarzy mojej, odwróciwszy sie od dróg swoich zlych: tedy Ja wyslucham z nieba, i odpuszcze grzech ich, a uzdrowie ziemie ich.
\par 15 Oczy tez moje otworzone beda, a uszy moje naklonione ku modlitwie uczynionej na tem miejscu.
\par 16 Bom teraz obral i poswiecil ten dom, aby tu przebywalo imie moje az na wieki; i beda tu oczy moje, i serce moje po wszystkie dni.
\par 17 A ty bedzieszli chodzil przede mna, jako chodzil Dawid, ojciec twój, a bedziesz sie sprawowal wedlug wszystkiego, com ci przykazal, strzegac ustaw moich i sadów moich:
\par 18 Tedy utwierdze stolice królestwa twego, jakom uczynil umowe z Dawidem, ojcem twoim, mówiac: Nie bedzie odjety z narodu twego maz panujacy nad Izraelem.
\par 19 A jezli sie wy odwrócicie, a opuscicie ustawy moje, i przykazania moje, którem wam podal, a odszedlszy bedziecie sluzyli bogom cudzym, i bedziecie sie im klaniali:
\par 20 Tedy ich wykorzenie z ziemi mojej, któram im dal; a ten dom, którym poswiecil imieniowi memu, odrzuce od oblicza mego, i podam go na przypowiesc, i na basn miedzy wszystkie narody.
\par 21 Nadto i ten dom, który byl slawny, kazdemu mimo idacemu bedzie na podziw, i rzecze: Przeczze tak uczynil Pan tej ziemi i temu domowi?
\par 22 Tedy odpowiedza: Przeto, iz opuscili Pana, Boga ojców swoich, który ich wywiódl z ziemi Egipskiej, a chwycili sie bogów cudzych, i klaniali sie im, i sluzyli im, dlategoz przywiódl na nich to wszystko zle.

\chapter{8}

\par 1 A po wyjsciu dwudziestu lat, w których budowal Salomon dom Panski i dom swój,
\par 2 Pobudowal tez miasta, które byl wrócil Hiram Salomonowi, a dal tam mieszkanie synom Izraelskim.
\par 3 Potem ciagnal Salomon do Emat Soby, i wzial ja.
\par 4 Pobudowal tez Tadmor na puszczy, i wszystkie miasta, w których mial sklady, pobudowal w Emat.
\par 5 Nadto zbudowal Betoron wyzsze i Betoron nizsze, miasta obronne w murach, z bramami i z zaworami;
\par 6 Takze Baalat, i wszystkie miasta, w których mial sklady Salomon; i wszystkie miasta dla wozów, i miasta dla jezdnych: owa wszystko wedlug upodobania swego, cokolwiek zamyslil budowac w Jeruzalemie i na Libanie, i po wszystkiej ziemi panowania sweg o.
\par 7 Wszystek tez lud, który byl pozostal z Hetejczyków, i Amorejczyków, i Ferezejczyków, i Hewejczyków, i Jebuzejczyków, którzy nie byli z Izraela:
\par 8 Idacy z synów ich, którzy byli zostali po nich w onej ziemi, których byli nie wygubili synowie Izraelscy, poczynil Salomon holdownikami az do dnia tego.
\par 9 Ale z synów Izraelskich, których nie poczynil Salomon niewolnikami do robót swoich, (bo oni byli mezowie waleczni, i przedniejsi hetmani jego, i przelozeni nad wozami jego i nad jezdnymi jego.)
\par 10 Z tych bylo przedniejszych przelozonych, których mial król Salomon, dwiescie i piecdziesiat panujacych nad ludem.
\par 11 Lecz córke Faraonowa przeniósl Salomon z miasta Dawidowego do domu, który jej byl zbudowal; albowiem mówil: Nie bedzie mieszkala zona moja w domu Dawida, króla Izraelskiego, bo swiety jest: przeto iz weszla do niego skrzynia Panska.
\par 12 Tedy Salomon ofiarowal calopalenia Panu na oltarzu Panskim, który byl zbudowal przed przysionkiem.
\par 13 Cokolwiek zwyczajnie na kazdy dzien ofiarowac miano wedlug rozkazania Mojzeszowego w sabaty, na nowiu miesiaca, i w swieta uroczyste, trzy kroc do roku, w swieto przasników, i w swieto tygodni, i w swieto kuczek.
\par 14 I postanowil wedlug rozrzadzenia Dawida, ojca swego, rozdzialy kaplanów w poslugiwaniu ich, i Lewitów w powinnosciach ich, aby chwalili Boga, i sluzyli przy kaplanach wedlug zwyczaju kazdego dnia; odzwiernych tez w rzedach ich przy kazdej bramie; albowiem tak bylo rozkazanie Dawida, meza Bozego.
\par 15 I nie ustapili od rozkazania królewskiego o kaplanach i o Lewitach, okolo wszystkich rzeczy i okolo skarbów.
\par 16 A tak dogotowano wszystkiego dziela Salomonowego, od onego dnia, którego zalozony byl dom Panski, az do wystawienia jego; i tak dokonczony byl dom Panski.
\par 17 Tedy jechal Salomon do Asyjongaber i do Elot, które jest nad brzegiem morskim w ziemi Edomskiej.
\par 18 I poslal mu Hiram przez reke slug swoich okrety i zeglarzy swiadomych morza, którzy jechali z slugami Salomonowymi do Ofir, a wziawszy stamtad czterysta i piecdziesiat talentów zlota, przyniesli je do króla Salomona.

\chapter{9}

\par 1 Tedy królowa z Saby slyszac slawe Salomonowa, przyjechala do Jeruzalemu, aby doswiadczala Salomona w zagadkach, z wielkim bardzo pocztem, i z wielbladami niosacemi rzeczy wonne, i zlota bardzo wiele, i kamienia drogiego, a przyszedlszy do Salomo na, mówila z nim o wszystkiem, co miala w sercu swojem.
\par 2 Ale jej odpowiedzial Salomon na wszystkie jej slowa, a nie bylo nic skrytego przed Salomonem, na coby jej nie odpowiedzial.
\par 3 Przetoz widzac królowa z Saby madrosc Salomonowa, i dom, który zbudowal;
\par 4 Takze potrawy stolu jego, i siadania slug jego, i stawania sluzacych mu, i szaty ich, i podczaszy jego, i szaty ich, i schody, po których wstepowal do domu Panskiego, zdumiewala sie bardzo.
\par 5 I rzekla do króla: Prawdziwac to mowa, któram slyszala w ziemi mojej o sprawach twoich, i o madrosci twojej.
\par 6 Alem nie wierzyla slowom ich, azem przyjechawszy ogladala oczyma swemi; ale oto nie powiedziano mi i polowy o wielkiej madrosci twojej; przeszedles slawe, o którejm slyszala.
\par 7 Blogoslawieni mezowie twoi i blogoslawieni ci sludzy twoi, którzy stoja przed toba zawsze, a sluchaja madrosci twojej.
\par 8 Niechze bedzie Pan, Bóg twój, blogoslawiony, który cie sobie upodobal, aby cie posadzil na stolicy swojej za króla przed Panem, Bogiem twoim. Dla tego, iz umilowal Bóg twój Izraela, aby go umocnil na wieki, przetoz postanowil cie nad nimi za króla, abys czynil sad i sprawiedliwosc.
\par 9 I dala królowi sto i dwadziescia talentów zlota, i rzeczy wonnych bardzo wiele, i kamienia drogiego; a nieprzyszlo nigdy wiecej takich rzeczy wonnych, jakie dala królowa z Saby królowi Salomonowi.
\par 10 Nadto i sludzy Hiramowi, i sludzy Salomonowi, którzy byli przywiezli zlota z Ofir, przywiezli i drzewa almugimowego, i kamienia drogiego.
\par 11 I poczynil król z onego drzewa almugimowego schody do domu Panskiego, i do domu królewskiego, i harfy, i lutnie spiewakom; a nie widziano przedtem nigdy takich rzeczy w ziemi Judzkiej.
\par 12 Król takze Salomon dal królowej z Saby wszystko, czego chciala, i czego zadala, oprócz nagrody za to, co byla przyniosla do króla. Potem sie wrócila, i odjechala do ziemi swej, ona i sludzy jej.
\par 13 A byla waga tego zlota, które przychodzilo Salomonowi na kazdy rok, szesc set i szescdziesiat i szesc talentów zlota.
\par 14 Oprócz tego, co kupcy, i którzy handluja przynosili; takze wszyscy królowie Arabscy, i ksiazeta onej ziemi przywozili zloto i srebro Salomonowi.
\par 15 Przetoz uczynil król Salomon dwiescie tarczy ze zlota ciagnionego; szesc set syklów zlota ciagnionego wychodzilo na kazda tarcza.
\par 16 Przytem trzysta puklerzy ze zlota ciagnionego, trzysta syklów zlotych wychodzilo na kazdy puklerz, które schowal król w dom lasu Libanowego.
\par 17 Uczynil takze król stolice wielka z kosci sloniowej, i powlókl ja szczerem zlotem.
\par 18 A szesc stopni bylo u onej stolicy, a podnózek byl ze zlota, trzymajacy sie stolicy; porecze tez byly z obudwu stron, kedy siadano, a dwa lwy staly u poreczy.
\par 19 Dwanascie tez lwów stalo na szesciu stopniach z obu stron; nie bylo nic takowego urobiono w zadnem królestwie.
\par 20 Nadto wszystkie naczynia, z których pijal król Salomon, byly zlote, i wszystek sprzet w domu lasu Libanowego ze zlota szczerego; nic nie bylo ze srebra, bo go nie miano w zadnej cenie za dni Salomonowych.
\par 21 Albowiem okrety królewskie chodzily na morze z slugami Hiramowymi; raz we trzy lata wracaly sie tez okrety z morza, przynoszac zloto, i srebro, kosci sloniowe, i koczkodany, i pawie.
\par 22 A tak uwielbiony jest król Salomon nad wszystkich królów ziemskich, bogactwy i madroscia.
\par 23 Przetoz wszyscy królowie ziemscy pragneli widziec Salomona, aby sluchali madrosci jego, która byl dal Bóg w serce jego.
\par 24 I przynosili mu kazdy upominek swój, naczynia srebrne, i naczynia zlote, szaty, zbroje, i rzeczy wonne, konie i muly, a to na kazdy rok.
\par 25 I mial Salomon cztery tysiace stajen koni i wozów, a dwanascie tysiecy jezdnych, których rozsadzil po miastach wozów, i przy sobie w Jeruzalemie.
\par 26 I panowal nad wszystkimi królmi od rzeki az do ziemi Filistynskiej, i az do granicy Egipskiej.
\par 27 A zlozyl król srebra w Jeruzalemie jako kamienia, a ceder zlozyl jako plonnych fig, których rosnie na polu bardzo wiele.
\par 28 Przywodzono tez konie Salomonowi z Egiptu i ze wszystkich innych ziem.
\par 29 A ostatek spraw Salomonowych pierwszych i ostatnich zapisano w ksiedze Natana proroka, i w proroctwie Achyjasza Sylonitczyka, i w widzeniach Jaddy widzacego, który prorokowal przeciw Jeroboamowi, synowi Nabatowemu.
\par 30 I królowal Salomon w Jeruzalemie nad wszystkim Izraelem czterdziesci lat.
\par 31 Zasnal potem Salomon z ojcami swymi, a pochowano go w miescie Dawida, ojca jego, a Roboam syn jego, królowal miasto niego.

\chapter{10}

\par 1 Tedy jechal Roboam do Sychem; bo w Sychem zebral sie byl wszystek Izrael, aby go postanowil królem.
\par 2 A gdy to uslyszal Jeroboam, syn Nabatowy, który byl w Egipcie, gdzie byl uciekl przed królem Salomonem, wrócil sie Jeroboam z Egiptu;
\par 3 Bo poslali i wezwali go. Tedy przyszedl Jeroboam i wszystek Izrael, i rzekli do Roboama, mówiac:
\par 4 Ojciec twój wlozyl na nas jarzmo ciezkie; ale ty teraz ulzyj nam niewoli srogiej ojca twego, i jarzma ciezkiego, które wlozyl na nas, a bedziemyc sluzyli.
\par 5 Który im rzekl: Po trzech dniach wróccie sie do mnie. I odszedl lud.
\par 6 Miedzytem wszedl król Roboam w rade z starszymi, którzy stawali przed Salomonem, ojcem jego, za zywota jego, mówiac: Co wy radzicie? Jakabym mial dac odpowiedz ludowi temu?
\par 7 Którzy mu odpowiedzieli, mówiac: Jezli dzis powolny bedziesz ludowi temu, a uczynisz im kwoli, i bedziesz mówil do nich slowa lagodne, beda slugami twymi po wszystkie dni.
\par 8 Ale on opusciwszy rade starszych, która mu podali, wszedl w rade z mlodziencami, którzy z nim wzrosli i stawali przed nim.
\par 9 I rzekl do nich: Cóz wy radzicie, abysmy odpowiedzieli ludowi temu, którzy rzekli do mnie, mówiac: Ulzyj tego jarzma, które wlozyl ojciec twój na nas?
\par 10 Tedy mu odpowiedzieli oni mlodziency, którzy z nim wzrosli, mówiac: Tak odpowiesz temu ludowi, którzy do ciebie rzekli, mówiac: Ojciec twój wlozyl na nas ciezkie jarzmo, ale ty ulzyj go nam. Tak rzeczesz do nich: Najmniejszy palec mój miezszy niz biodra ojca mego.
\par 11 Przetoz teraz, ojciec mój kladl na was jarzmo ciezkie, ale ja przydam do jarzma waszego; ojciec mój karal was biczykami, a ja was bede karal korbaczami.
\par 12 Przyszedl tedy Jeroboam i wszystek lud do Roboama dnia trzeciego, jako byl im rozkazal król, mówiac: Wróccie sie do mnie dnia trzeciego.
\par 13 I odpowiedzial im król surowie, bo opuscil król Roboam rade starców.
\par 14 A rzekl do nich wedlug rady mlodzienców, mówiac: Ojciec mój obciazyl was jarzmem ciezkiem, ale ja przydam do niego; ojciec mój karal was biczykami, ale ja was bede karal korbaczami.
\par 15 I nie usluchal król ludu: (bo byla przyczyna od Boga, aby dosyc uczynil Pan slowu swemu, które byl powiedzial przez Achyjasza Sylonitczyka do Jeroboama, syna Nabatowego.)
\par 16 Przetoz widzac wszystek Izrael, ze ich król nie usluchal, odpowiedzial lud królowi, mówiac: Cóz mamy za dzial w Dawidzie? a co za dziedzictwo w synu Isajowym? Kazdy idz do namiotów swych, o Izraelu! a ty Dawidzie! opatrz teraz dom swój. I rozeszli sie wszyscy Izraelczycy do namiotów swoich.
\par 17 A tak tylko nad synami Izraelskimi, którzy mieszkali w miastach Judzkich, królowal Roboam.
\par 18 I poslal król Roboam Adorama, który byl poborca, i ukamionowali go synowie Izraelscy, ze umarl; przetoz król Roboam wsiadlszy co rychlej na wóz, uciekl do Jeruzalemu.
\par 19 A tak odstapili Izraelczycy od domu Dawidowego, az do dnia tego.

\chapter{11}

\par 1 Przyjechawszy tedy Roboam do Jeruzalemu, zebral dom Judowy i pokolenie Benjaminowe, sto i osmdziesiat tysiecy mezów przebranych do boju, aby walczyli z Izraelem, a zeby przywrócone bylo królestwo do Roboama.
\par 2 I stalo sie slowo Panskie do Semejasza, meza Bozego, mówiac:
\par 3 Powiedz Roboamowi, synowi Salomonowemu, królowi Judzkiemu, i wszystkiemu Izraelowi w Judzie i w pokoleniu Benjaminowem, mówiac:
\par 4 Tak mówi Pan: Nie wychodzcie, ani walczcie z bracmi waszymi, wróccie sie kazdy do domu swego; albowiem odemnie stala sie ta rzecz. I usluchali slowa Panskiego, i wrócili sie, a nie ciagneli przeciw Jeroboamowi.
\par 5 I mieszkal Roboam w Jeruzalemie, a pobudowal miasta obronne w Judzie.
\par 6 I zbudowal Betlehem, i Etam i Tekue;
\par 7 I Betsur, i Soko, i Adullam:
\par 8 I Get, i Maresa, i Zyf:
\par 9 I Adoraim, i Lachis, i Aseka;
\par 10 I Sora, i Ajalon, i Hebron, które byly w Judzie i w pokoleniu Benjaminowem miasta obronne.
\par 11 A gdy zmocnil one twierdze, postanowil w nich starostów, i wystawil szpichlerze dla zboza, i dla oliwy, i dla wina.
\par 12 A w kazdem miescie zlozyl tarcze i wlócznie, a opatrzyl je bardzo mocno: a tak panowal nad Juda i Benjaminem.
\par 13 Kaplani tez i Lewitowie, którzy byli we wszystkim Izraelu, zebrali sie do niego ze wszystkich granic swoich.
\par 14 Bo opusciwszy Lewitowie przedmiescia swoje, i osiadlosci swoje, szli do Judy i do Jeruzalemu; (gdyz byl ich wyrzucil Jeroboam i synowie jego, aby nie odprawowali urzedu kaplanskiego Panu.
\par 15 I postanowil sobie kaplanów po wyzynach, i dyjablom, i cielcom, których byl naczynil.)
\par 16 A za nimi ze wszystkich pokolen Izraelskich, którzy obrócili serca swe ku szukaniu Pana, Boga Izraelskiego, przyszli do Jeruzalemu, aby ofiarowali Panu, Bogu ojców swoich.
\par 17 A tak umocnili królestwo Judzkie, i utwierdzili Roboama, syna Salomonowego, przez trzy lata; albowiem chodzili droga Dawida i Salomona przez one trzy lata.
\par 18 Potem pojal sobie Roboam za zone Mahalate córke Jerymota, syna Dawidowego, i Abihaile, córke Elijaba, syna Isajego.
\par 19 Która mu urodzila synów: Jehusa, i Semaryjasza, i Zaama.
\par 20 A po niej pojal Maache, córke Absalomowa, która mu urodzila Abijasza, i Etaja, i Syse, i Salomite.
\par 21 I milowal Roboam Maache, córke Absalomowa, nad wszystkie zony swoje, i nad zaloznice swoje. Albowiem pojal byl zon osmnascie, a zaloznic szescdziesiat, i splodzil dwadziescia i osm synów i szescdziesiat córek.
\par 22 I postanowil Roboam Abijasza, syna Maachy, za ksiecia, za hetmana miedzy bracmi jego; albowiem zamyslal go uczynic królem.
\par 23 A roztropnie sobie postepujac, rozsadzil wszystkich innych synów swych po wszystkich krainach Judzkich i Benjaminowych, po wszystkich miastach obronnych, i opatrzyl ich dostatkiem zywnosci, i nadal im wiele zon.

\chapter{12}

\par 1 A gdy utwierdzil królestwo swoje Roboam i zmocnil je, opuscil zakon Panski, i wszystek Izrael z nim.
\par 2 I stalo sie roku piatego panowania Roboamowego, ze wyciagnal Sesak, król Egipski, przeciw Jeruzalemowi (albowiem byli zgrzeszyli przeciw Panu.)
\par 3 Z tysiacem i dwoma stami wozów, i z szescdziesiat tysiecy jezdnych, a nie bylo liczby ludu, który przyciagnal z nim z Egiptu, Lubimczyków, Suchymczyków, i Chusymczyków.
\par 4 I pobral miasta obronne, które byly w Judzie, i przyciagnal az ku Jeruzalemowi.
\par 5 Tedy Semejasz prorok przyszedl do Roboama i do ksiazat Judzkich, którzy sie byli zebrali do Jeruzalemu, uciekajac przed Sesakiem, i rzekl do nich: Tak mówi Pan: Wyscie mie opuscili, dla tegom i Ja was opuscil i podal w rece Sesakowe.
\par 6 I upokorzyli sie ksiazeta Izraelscy, i król, i mówili: Sprawiedliwy jest Pan.
\par 7 A gdy ujrzal Pan iz sie upokorzyli, stalo sie slowo Panskie do Semejasza, mówiac: Upokorzyli sie, nie wytrace ich; ale im dam wkrótce wybawienie, ani sie wyleje zapalczywosc moja przeciw Jeruzalemowi przez rece Sesaka.
\par 8 Wszakze beda mu za slugi, aby wiedzieli, co to jest, sluzyc mnie, albo sluzyc królestwom ziemskim.
\par 9 A tak ciagnal Sesak, król Egipski, przeciw Jeruzalemowi, i pobral skarby domu Panskiego, i skarby domu królewskiego, wszystko to pobral; wzial tez tarcze zlote, które byl sprawil Salomon.
\par 10 I sprawil król Roboam miasto nich tarcze miedziane, i poruczyl je przelozonym nad piechota, którzy strzegli drzwi domu królewskiego.
\par 11 A gdy wchadzal król do domu Panskiego, tedy przychodzila piechota, i brali je; potem zasie odnosili je do swoich komor.
\par 12 A tak iz sie upokorzyl, odwrócil sie od niego gniew Panski, i nie wytracil go do konca; albowiem jeszcze i w Judzie bylo nieco dobrego.
\par 13 Zmocnil sie tedy król Roboam w Jeruzalemie i królowal. A bylo czterdziesci lat i jeden rok Roboamowi, gdy królowac poczal, a siedmnascie lat królowal w Jeruzalemie, w miescie, które obral Pan ze wszystkich pokolen Izraelskich, aby tam przebywalo imie jego. A imie matki jego bylo Naama Ammonitka.
\par 14 Ten czynil zle; bo nie przygotowal serca swego, aby szukal Pana.
\par 15 Ale sprawy Roboamowe pierwsze i poslednie zapisane sa w ksiedze Semejasza proroka, i Jaddy widzacego; gdzie sie opisuje porzadek rodzajów, takze wojny miedzy Roboamem i Jeroboamem po wszystkie dni.
\par 16 I zasnal Roboam z ojcami swymi, i pochowan jest w miescie Dawidowem, a królowal Abijasz, syn jego, miasto niego.

\chapter{13}

\par 1 Roku osmnastego króla Jeroboama, królowal Abijasz nad Juda.
\par 2 Trzy lata królowal w Jeruzalemie, a imie matki jego bylo Michaja, córka Uryjelowa z Gabaa. I byla wojna miedzy Abijaszem i miedzy Jeroboamem.
\par 3 Przetoz Abijasz uszykowal wojsko ludzi bardzo walecznych cztery kroc sto tysiecy mezów przebranych; Jeroboam takze uszykowal sie przeciwko niemu, majac osm kroc sto tysiecy mezów przebranych bardzo walecznych.
\par 4 Tedy stanal Abijasz na wierzchu góry Semeron, która byla miedzy górami Efraimskiemi, i rzekl: Sluchajcie mie, Jeroboamie i wszystek Izraelu!
\par 5 Izali wam nie nalezy wiedziec, ze Pan, Bóg Izraelski, dal królestwo Dawidowi nad Izraelem na wieki, onemu i synom jego przymierzem trwalem?
\par 6 Lecz powstal Jeroboam, syn Nabatowy, sluga Salomona, syna Dawidowego, i stal sie odpornym panu swemu.
\par 7 I zebrali sie do niego synowie lekkomyslni, a ludzie niepobozni, i zmocnili sie przeciw Roboamowi, synowi Salomonowemu; a Roboam bedac dziecieciem, i serca lekliwego, nie mógl sie im meznie oprzec:
\par 8 Zaczem wy sie teraz myslicie zmocnic przeciw królestwu Panskiemu, które jest w rekach synów Dawidowych, a jest was kupa wielka, i macie z soba cielce zlote, których wam naczynil Jeroboam za bogów.
\par 9 Izazescie nie zarzucili kaplanów Panskich, synów Aaronowych i Lewitów, a poczyniliscie sobie kaplanów, jako inni narodowie ziemscy? Ktokolwiek przychodzi, aby poswiecone byly rece jego, z cielcem mlodym i siedmia baranów, staje sie kaplanem tych, którzy nie sa bogowie.
\par 10 Ale my jestesmy Pana, Boga naszego, i nie opuscilismy go; a kaplani, którzy sluza Panu, sa synowie Aaronowi, i Lewitowie, którzy pilnuja urzedu swego.
\par 11 I ofiaruja Panu calopalenia na kazdy zaranek, i na kazdy wieczór, i kadza rzeczami wonnemi, i pokladaja chleby na stole czystym; takze swiecznik zloty, i lampy jego sporzadzaja, aby gorzaly na kazdy wieczór. A tak my strzezemy rozkazania Pana, Boga naszego, a wyscie go opuscili.
\par 12 Przetoz oto, z nami jest na czele Bóg i kaplani jego, i traby glosne, aby brzmialy przeciwko wam. Synowie Izraelscy! nie walczciez z Panem, Bogiem ojców waszych; bo sie wam nie powiedzie.
\par 13 Miedzytem obwiódl zasadzke Jeroboam, aby na nich przypadli z tylu; i byli jedni w oczy Judzie, a drudzy na zasadzce z tylu ich.
\par 14 Tedy ujrzawszy synowie Judzcy, ze na nich nastepowala bitwa z przodku i z tylu, wolali do Pana, a kaplani trabili w traby.
\par 15 Uczynili tez okrzyk mezowie Judy. I stalo sie w onym okrzyku mezów Judzkich, ze Bóg porazil Jeroboama, i wszystkiego Izraela przed Abijaszem i Juda.
\par 16 I uciekali synowie Izraelscy przed Juda; ale ich podal Bóg w rece ich.
\par 17 I porazili ich Abijasz i lud jego porazka wielka, tak iz leglo pobitych z Izraela piec kroc sto tysiecy mezów na wybór.
\par 18 Przetoz ponizeni sa synowie Izraelscy naonczas: a zmocnili sie synowie Judzcy, poniewaz spolegali na Panu, Bogu ojców swych.
\par 19 I gonil Abijasz Jeroboama, a odjal mu miasta Betel i wsi jego, i Jesana i wsi jego, i Efron i wsi jego.
\par 20 A nie mógl sie wiecej pokrzepic Jeroboam za dni Abijaszowych, i uderzyl go Pan, ze umarl.
\par 21 Ale Abijasz zmocnil sie, i pojal sobie zon czternascie, i splodzil dwadziescia i dwóch synów, i szesnascie córek.
\par 22 Ale inne sprawy Abijaszowe, i postepki jego, i slowa jego, zapisane sa w ksiedze proroka Jaddy.

\chapter{14}

\par 1 A gdy zasnal Abijasz z ojcami swymi, a pochowano go w miescie Dawidowem, tedy królowal Aza, syn jego, miasto niego. Za dni jego byla w pokoju ziemia, przez dziesiec lat.
\par 2 I czynil Aza co bylo dobrego, i przyjemnego w oczach Pana, Boga swego.
\par 3 Bo poburzyl oltarze obce, i wyzyny, i podruzgotal balwany ich, i wyrabal gaje ich;
\par 4 A przykazal Judzie, aby szukali Pana, Boga ojców swych, i przestrzegali zakonu i przykazan jego.
\par 5 Zniósl tez ze wszystkich miast Judzkich wyzyny, i sloneczne balwany, a bylo w pokoju królestwo za czasu jego.
\par 6 Nadto pobudowal miasta obronne w Judzie, przeto, iz byla w pokoju ziemia i nie powstala wojna przeciwko niemu za onych lat; bo mu dal Pan odpocznienie.
\par 7 I rzekl do Judy: Pobudujmy te miasta, i otoczmy je murem i wiezami, bramami, i zaworami, póki ziemia jest w mocy naszej; bosmy szukali Pana, Boga naszego; szukalismy go, i sprawil nam odpocznienie zewszad. Przetoz budowali, a szczescilo sie im.
\par 8 I mial Aza wojsko noszacych tarcz i drzewce: z Judy trzykroc sto tysiecy, a z Benjamina noszacych puklerz i ciagnacych luk dwa kroc sto tysiecy i osmdziesiat tysiecy. Ci wszyscy byli bardzo mezni.
\par 9 Tedy wyciagnal przeciw nim Zara Etyjopczyk, majac wojska dziesiec kroc sto tysiecy, a wozów trzy sta, i przyciagnal az do Maresy.
\par 10 Wyciagnal tez i Aza przeciw niemu, i uszykowali wojska w dolinie Sefata u Maresy.
\par 11 Tedy zawolal Aza do Pana, Boga swego, i rzekl: O Panie! ty nie potrzebujesz wielu, abys ratowal tego, który nie ma potegi. Ratujze nas, o Panie, Boze nasz! gdyz na tobie spolegamy, a w imie twoje idziemy przeciwko temu mnóstwu. Tys Pan, Bóg nasz; niech nie ma góry nad toba czlowiek smiertelny.
\par 12 A tak porazil Pan Etyjopczyków przed Aza i przed Juda, tak iz uciekli Etyjopczycy.
\par 13 A gonil ich Aza i lud, który byl z nim, az do Gierary. I polegli Etyjopczycy, tak, ze nie mogli wskórac: bo starci sa przed obliczem Panskiem i przed wojskiem jego. I nabrali lupów bardzo wiele.
\par 14 Nadto poburzyli wszystkie miasta okolo Gierary; albowiem przypadl strach Panski na nich, i splundrowali one wszystkie miasta; bo w nich byla wielka korzysc.
\par 15 Takze i obory trzód poburzyli; a zajawszy owiec i wielbladów bardzo wiele, wrócili sie do Jeruzalemu.

\chapter{15}

\par 1 Tedy na Azaryjasza, syna Obedowego, przypadl Duch Bozy.
\par 2 Który wyszedlszy przeciw Azie rzekl mu: Sluchajcie mie, Aza i wszystko pokolenie Judowe i Benjaminowe! Pan byl z wami, pókiscie byli z nim, a jezli go szukac bedziecie, znajdziecie go: ale jezli go opuscicie, opusci was.
\par 3 Przez wiele dni byl Izrael bez Boga prawdziwego, i bez kaplana, nauczyciela, i bez zakonu:
\par 4 Wszakze gdyby sie byli nawrócili w utrapieniu swem do Pana, Boga Izraelskiego, a szukali go, dalby sie im byl znalesc.
\par 5 Ale terazniejszych czasów niebezpieczno wychodzic i wchodzic; bo zamieszanie wielkie miedzy wszystkimi obywatelami ziemi.
\par 6 I depcze naród po narodzie, a miasto po miescie, przeto, ze ich Bóg strwozyl wszelakiem ucisnieniem.
\par 7 Przetoz wy zmacniajcie sie, a niech nie slabieja rece wasze; bo czeka zaplata pracy waszej.
\par 8 A gdy uslyszal Aza te slowa, i proroctwo Obeda proroka, zmocnil sie, i zniósl obrzydliwosci ze wszystkiej ziemi Judzkiej i Benjaminskiej, i z miast, które byl pobral na górze Efraim, i odnowil oltarz Panski, który byl przed przysionkiem Panskim.
\par 9 Potem zebral wszystek lud z Judy i z Benjamina, i przychodniów, którzy u nich byli z Efraima, i z Manasesa i z Symeona; bo ich bylo bardzo wiele zbieglo z Izraela do niego, widzac, iz Pan, Bóg jego, z nim byl.
\par 10 I zgromadzili sie do Jeruzalemu miesiaca trzeciego, roku pietnastego królestwa Azy.
\par 11 I sprawowali ofiary Panu dnia onego z lupów, które byli przygnali, wolów siedm set, a owiec siedm tysiecy.
\par 12 I uczynili umowe, aby szukali Pana, Boga ojców swoich, ze wszystkiego serca swego, i ze wszystkiej duszy swojej.
\par 13 A ktobykolwiek nie szukal Pana, Boga Izraelskiego, aby byl zabity, od najmniejszego az do najwiekszego, od meza az do niewiast.
\par 14 I przysiegli Panu glosem wielkim, i z krzykiem, i z trabami, i z kornetami.
\par 15 A weselil sie wszystek lud Judzki z onej przysiegi: albowiem ze wszystkiego serca swego przysiegali, i ze wszystkiej ochoty szukali go, i dal sie im znalesc, i dal im Pan odpocznienie zewszad.
\par 16 Nadto i Maache, matke swa, król Aza z panstwa zlozyl, przeto, ze byla uczynila w gaju balwana brzydkiego; i podcial Aza balwana jej, i pokruszyl go, a spalil u potoku Cedron.
\par 17 A choc wyzyny nie byly zniesione z Izraela, przeciez serce Azy bylo doskonale po wszystkie dni jego.
\par 18 Wniósl tez, co byl poswiecil ojciec jego, i co sam poswiecil, do domu Bozego, srebro i zloto i naczynia.
\par 19 A nie bylo wojny az do roku trzydziestego i piatego królowania Azy.

\chapter{16}

\par 1 Roku trzydziestego i szóstego królowania Azy, wyciagnal Baaza, król Izraelski, przeciwko Judzie, i zbudowal Rame, aby nie dopuscil wychodzic i wchodzic nikomu do Azy, króla Judzkiego.
\par 2 Tedy wziawszy Aza srebro i zloto ze skarbów domu Panskiego i domu królewskiego, poslal je do Benadada, króla Syryjskiego, który mieszkal w Damaszku, mówiac:
\par 3 Przymierze jest miedzy mna i miedzy toba, i miedzy ojcem moim i miedzy ojcem twoim: otoc posylam srebro i zloto. Idzze, a wzrusz przymierze twoje z Baaza, królem Izraelskim, aby odciagnal odemnie.
\par 4 I usluchal Benadad króla Azy, a poslawszy hetmanów z wojskami, które mial, przeciwko miastom Izraelskim, zburzyl Hijon i Dan i Abelmaim, i wszystkie miasta obronne Neftalimskie, w których byly skarby.
\par 5 To gdy uslyszal Baaza, przestal budowac Ramy, i zaniechal roboty swej.
\par 6 Tedy król Aza wziawszy z soba wszystek lud Judzki, pobrali z Ramy kamienie, i drzewo jego, z którego budowal Baaza, a zbudowal z niego Gabaa i Masfa.
\par 7 A onegoz czasu przyszedl Hanani, widzacy, do Azy, króla Judzkiego, i mówil do niego: Izes spolegl na królu Syryjskim, a nie spolegles na Panu, Bogu twoim, dlatego uszlo wojsko króla Syryjskiego z reki twojej.
\par 8 Azaz Etyjopczycy, i Lubimczycy nie mieli wojsk bardzo wielkich z wozami i z jezdnymi w mnóstwie bardzo wielkiem? a wzdy gdys spolegl na Panu, podal je w reke twoje.
\par 9 Albowiem oczy Panskie przepatruja wszystke ziemie, aby dokazywal mocy swej przy tych, którzy przy nim stoja sercem doskonalem. Glupios to uczynil: przetoz od tego czasu powstana przeciwko tobie wojny.
\par 10 Tedy Aza rozgniewawszy sie na widzacego, podal go do wiezienia; bo sie byl nan o to rozgniewal; i utrapil Aza niektórych z ludu onego czasu.
\par 11 Ale inne sprawy Azy pierwsze i posledniejsze, zapisane sa w ksiegach o królach Judzkich i Izraelskich.
\par 12 Potem rozniemógl sie Aza roku trzydziestego i dziewiatego królowania swego, na nogi swoje, choroba bardzo ciezka; a wszakze w onej chorobie swej nie szukal Pana, ale lekarzy.
\par 13 A tak zasnal Aza z ojcami swymi, a umarl roku czterdziestego i pierwszego królowania swego.
\par 14 I pochowano go w grobie jego, który sobie byl wykopal w miescie Dawidowem; i polozono go na lozu, które byl napelnil rzeczami wonnemi, i rozmaitemi masciami aptekarska robota przygotowanemi. I palili mu zapal wonny bardzo wielki.

\chapter{17}

\par 1 Tedy królowal Jozafat, syn jego miasto niego, a zmocnil sie przeciw Izraelowi.
\par 2 I osadzil zolnierzem wszystkie miasta Judzkie obronne; osadzil tez i ziemie Judzka, takze miasta Efraimskie, które byl pobral Aza, ojciec jego.
\par 3 A byl Pan z Jozafatem, przeto, iz chodzil drogami pierwszemi Dawida, ojca swego, a nie szukal Baalów;
\par 4 Ale Boga ojca swego szukal, i w przykazaniach jego chodzil, a nie wedlug spraw ludu Izraelskiego.
\par 5 I utwierdzil Pan królestwo w rece jego; a dawal wszystek lud Judzki dary Jozafatowi, tak iz mial bogactwa, i slawe bardzo wielka.
\par 6 A nabywszy wielkiego serca na drogach Panskich, tem wiecej znosil wyzyny i gaje balwochwalcze z ziemi Judzkiej.
\par 7 Potem roku trzeciego królowania swego poslal ksiazat swoich, Benchaila, i Obadyjasza, i Zacharyjasza, i Natanaela, i Micheasza, aby uczyli w miastach Judzkich.
\par 8 A przy nich Lewitów, Semejasza, i Natanijasza, i Zabadyjasza, i Asaela i Semiramota, i Jonatana, i Adonijasza, i Tobijasza, i Tobadonijasza, Lewitów; a z nimi Elisama, i Jorama, kaplanów;
\par 9 Którzy uczyli w Judzie, majac z soba ksiegi zakonu Panskiego, i obchodzili wszystkie miasta Judzkie, i nauczali lud.
\par 10 Przetoz przyszedl strach Panski na wszystkie królestwa ziemi, które byly okolo Judy, i nie smieli walczyc przeciw Jozafatowi.
\par 11 Nawet i Filistynowie przynosili Jozafatowi dary i dan pieniezna. Arabczycy tez przygnali mu drobnego bydla, baranów siedm tysiecy i siedm set, kozlów takze siedm tysiecy i siedm set.
\par 12 Wzmagal sie tedy Jozafat, i rosl nader bardzo, i pobudowal w Judzie zamki, i miasta dla skladów.
\par 13 A pracy wiele podjal okolo miast Judzkich; mezów tez walecznych i poteznych mial w Jeruzalemie.
\par 14 A tac jest liczba ich wedlug domów ojców ich: z Judy ksiazeta nad tysiacami: ksiaze Adna, a z nim bardzo duzych mezów trzy kroc sto tysiecy.
\par 15 A po nim ksiaze Johanan, a z nim dwa kroc sto tysiecy, i osmdziesiat tysiecy.
\par 16 A po nim Amazyjasz, syn Zychry, który sie byl dobrowolnie oddal Panu, a z nim dwa kroc sto tysiecy mezów duzych.
\par 17 Przytem z synów Benjaminowych, maz duzy Elijada, a z nim ludu zbrojnego z lukami i z tarczami dwa kroc sto tysiecy.
\par 18 A po nim Jozabad, a z nim sto i osmdziesiet tysiecy gotowych do boju.
\par 19 Ci sluzyli królowi, oprócz tych, którymi byl król osadzil miasta obronne po wszystkiej ziemi Judzkiej.

\chapter{18}

\par 1 I mial Jozafat bogactw i slawy bardzo wiele, a spowinowacil sie z Achabem.
\par 2 I przyjechal po kilku latach do Achaba do Samaryi; i nabil Achab owiec i wolów wiele dla niego, i dla ludu, który byl z nim, i namawial go, aby ciagnal do Ramot Galaad.
\par 3 I rzekl Achab, król Izraelski, do Jozafata, króla Judzkiego: Pociagnijze ze mna do Ramot Galaad? A on mu odpowiedzial: Jako ja, tak i ty, a jako lud twój, tak lud mój, i bedziemy z toba na wojnie.
\par 4 Nadto rzekl Jozafat do króla Izraelskiego: Prosze, pytaj sie dzis slowa Panskiego.
\par 5 A tak zebral król Izraelski proroków cztery sta mezów, i rzekl do nich: Mamze ciagnac do Ramot Galaad na wojne, czyli zaniechac? A oni odpowiedzieli: Ciagnij; bo je da Bóg w rece królewskie.
\par 6 Ale Jozafat rzekl: Niemaszze tu jeszcze którego proroka Panskiego, zebysmy sie go pytali?
\par 7 I rzekl król Izraelski do Jozafata: Jest jeszcze maz jeden, przez któregobysmy sie mogli radzic Pana, ale go ja nienawidze; bo mi nie prorokuje nic dobrego, ale zawzdy zle; a tenci jest Micheasz, syn Jemlowy. I rzekl Jozafat: Niech tak nie mówi król.
\par 8 Tedy zawolal król Izraelski niektórego komornika, i rzekl: Przywiedz tu rychlo Micheasza, syna Jemlowego.
\par 9 Miedzytem król Izraelski, i Jozafat, król Judzki, siedzieli kazdy z nich na stolicy swojej, ubrani w szaty królewskie, a siedzieli na placu u wrót bramy Samaryjskiej, a wszyscy prorocy prorokowali przed nimi.
\par 10 A Sedechyjasz, syn Chanaanowy, sprawil sobie rogi zelazne, i rzekl: Tak mówi Pan: Temi bedziesz bódl Syryjczyków, az ich wyniszczysz.
\par 11 Toz wszyscy prorocy prorokowali, mówiac: Ciagnij do Ramot Galaad, a bedziec sie szczescilo; albowiem je poda Pan, w rece królewskie.
\par 12 Tedy posel, który chodzil, aby przyzwal Micheasza, rzekl do niego, mówiac: Oto slowa proroków jednemi usty dobrze tusza królowi; niechze bedzie, prosze, slowo twoje jako jednego z nich, a mów dobre rzeczy.
\par 13 I rzekl Micheasz: Jako zyje Pan, ze co mi kolwiek rozkaze Bóg mój, to mówic bede.
\par 14 A gdy przyszedl do króla, rzekl król do niego: Micheaszu! mamyz ciagnac na wojne przeciw Ramot Galaad, czyli zaniechac? A on odpowiedzial: Ciagnijcie, a poszczesci sie wam, i beda podani w rece wasze.
\par 15 I rzekl król do niego: A wielez cie razy mam przysiega zobowiazywac, abys mi nie mówil jedno prawde w imieniu Panskiem?
\par 16 Przetoz rzekl: Widzialem wszystek lud Izraelski rozproszony po górach jako owce, które nie maja pasterza; bo Pan rzekl: Nie majaci Pana; niech sie wróci kazdy do domu swego w pokoju.
\par 17 I rzekl król Izraelski do Jozafata: Izazem ci nie powiadal, ze mi nic dobrego prorokowac nie mial, ale zle?
\par 18 Ale on rzekl: Sluchajciez tedy slowa Panskiego: Widzialem Pana siedzacego na stolicy jego, i wszystko wojsko niebieskie stojace po prawicy jego i po lewicy jego.
\par 19 I rzekl Pan: Kto zwiedzie Achaba, króla Izraelskiego, aby szedl, a polegl w Ramot Galaad? A gdy mówil jeden tak, a drugi mówil inaczej,
\par 20 Wystapil duch, i stanal przed Panem, i rzekl: Ja go zwiode. A Pan mu rzekl: Przez cóz?
\par 21 I odpowiedzial: Wynijde, a bede klamliwym duchem w ustach wszystkich proroków jego. I rzekl Pan: Zwiedziesz, i pewnie przemozesz: Idzze, a uczyn tak.
\par 22 Przetoz teraz, oto dal Pan ducha klamliwego w usta tych proroków twoich, gdyz Pan wyrzekl przeciwko tobie zle.
\par 23 Tedy przystapiwszy Sedechijasz, syn Chanaanowy, uderzyl Micheasza w policzek, mówiac: A któraz droga odszedl duch Panski odemnie, aby mówil z toba?
\par 24 I odpowiedzial Micheasz: Oto ty ujrzysz dnia onego, kiedy wnijdziesz do najskrytszej komory, abys sie skryl.
\par 25 I rzekl król Izraelski: Wezmijcie Micheasza, a odwiedzcie go do Amona, starosty miejskiego, i do Joaza, syna królewskiego.
\par 26 I rzeczecie im: Tak mówi król: Wsadzcie tego do wiezienia, a dawajcie mu jesc chleb utrapienia, i wode ucisku, az sie wróce w pokoju.
\par 27 Ale odpowiedzial Micheasz: Jezlize sie wrócisz w pokoju, tedy nie mówil Pan przez mie. Nadto rzekl: Sluchajciez wszyscy ludzie.
\par 28 A tak ciagnal król Izraelski, i Jozafat, król Judzki, do Ramot Galaad.
\par 29 I rzekl król Izraelski do Jozafata: Odmienie sie, a pójde do bitwy: ale ty ubierzesz sie w szaty swe. I odmienil sie król Izraelski, a szli do bitwy.
\par 30 A król Syryjski rozkazal byl hetmanom, którzy byli nad wozami jego, mówiac: Nie potykajcie sie ani z malym ani z wielkim, tylko z samym królem Izraelskim.
\par 31 A gdy ujrzeli Jozafata hetmani, co byli nad wozami, rzekli: Król Izraelski jest. I obrócili sie przeciw niemu, aby sie z nim potykali; ale zawolal Jozafat, a Pan go ratowal; i odwrócil ich Bóg od niego.
\par 32 Bo obaczywszy hetmani, co byli nad wozami, ze nie ten byl król Izraelski, odwrócili sie od niego.
\par 33 Lecz niektóry maz strzelil na niepewne z luku, i postrzelil króla Izraelskiego, miedzy nity i miedzy pancerz; który rzekl woznicy swemu: Nawróc, a wywiez mie z wojska; bom jest raniony.
\par 34 I wzmogla sie bitwa dnia onego, a król Izraelski stal na wozie przeciw Syryjczykom az do wieczora: i umarl, gdy zachodzilo slonce.

\chapter{19}

\par 1 A gdy sie wracal Jozafat, król Judzki, do domu swego w pokoju, do Jeruzalemu,
\par 2 Wyszedl przeciwko niemu Jehu, syn Hananiego, widzacy, i rzekl do króla Jozafata: Izalis niezboznemu mial pomagac, a tych, którzy nienawidza Pana, milowac? Przetoz nad toba jest gniew Panski.
\par 3 Wszakze znalazly sie sprawy dobre w tobie, przeto, zes powycinal gaje swiecone z ziemi, a zgotowales serce swe, abys szukal Boga.
\par 4 A pomieszkawszy Jozafat w Jeruzalemie wyjechal zas, i objechal lud od Beersaby az do góry Efraimskiej, i nawrócil ich do Pana, Boga ojców swoich.
\par 5 I postanowil sedziów w ziemi po wszystkich miastach Judzkich obronnych, w kazdem miescie.
\par 6 Tedy rzekl do sedziów: Baczciez, co czynicie; bo nie ludzki sad odprawujecie, ale Panski, który jest z wami przy sprawie sadowej.
\par 7 A przetoz niechaj bedzie bojazn Panska z wami; przestrzegajciez tego, a tak sie sprawujcie; bo nie masz u Pana, Boga naszego, nieprawosci, i niema wzgledu na osoby, ani przyjmuje darów.
\par 8 Takze i w Jeruzalemie postanowil Jozafat niektórych z Lewitów, i z kaplanów, i z przedniejszych domów ojcowskich w Izraelu, dla sadu Panskiego, i dla sporów tych, którzy sie udawali do Jeruzalemu.
\par 9 I przykazal im, mówiac: Tak czyncie w bojazni Panskiej, wiernie, i sercem doskonalem.
\par 10 A przy wszystkich sporach, któreby przyszly przed was od braci waszych, którzy mieszkaja w miastach swych, miedzy krwia a krwia, miedzy zakonem a przykazaniem, ustawami i sadami, napominajcie ich, aby nie grzeszyli przeciwko Panu, aby nie przyszedl gniew na was, i na braci waszych. Tak czyncie, a nie zgrzeszycie.
\par 11 A oto Amaryjasz, kaplan najwyzszy, bedzie miedzy wami we wszystkich sprawach Panskich; a Zabadyjasz, syn Ismaelowy, ksiaze w domu Judzkim, we wszystkich sprawach królewskich: takze Lewitowie beda rzadzcami miedzy wami. Zmacniajciez sie, a tak czyncie, a Pan bedzie z dobrym.

\chapter{20}

\par 1 I stalo sie potem, ze przyciagneli synowie Moabowi, i synowie Ammonowi, a z nimi niektórzy mieszkajacy z Ammonitami, przeciwko Jozafatowi na wojne.
\par 2 I przyszlo, a opowiedziano Jozafatowi, mówiac: Przyciagnelo przeciwko tobie wojsko wielkie z zamorza, z Syryi, a oto sa w Chasesontamar, które jest Engaddy.
\par 3 I ulakl sie, a obrócil Jozafat twarz swoje, aby szukal Pana, i zapowiedzial post wszystkiemu ludowi Judzkiemu.
\par 4 Tedy sie zgromadzil lud Judzki, aby szukali Pana; takze ze wszystkich miast Judzkich zeszli sie szukac Pana.
\par 5 A tak stanal Jozafat w zgromadzeniu Judzkiem i Jeruzalemskiem w domu Panskiem przed sienia nowa.
\par 6 I rzekl: Panie, Boze ojców naszych: Izalis nie ty sam Bogiem na niebie? a nie ty panujesz na wszystkiemi królestwami narodów? azaz nie w rekach twoich jest moc i sila? a nie masz, ktoby sie mógl ostac przed toba.
\par 7 Izalis nie ty, Boze nasz! wypedzil obywateli tej ziemi przed twarza ludu twego Izraelskiego, i podales ja nasieniu Abrahama przyjaciela twego na wieki?
\par 8 I miszkali w niej, i zbudowali tobie w niej swiatnice dla imienia twego, mówiac:
\par 9 Jezliby na nas przyszlo zle, miecz pomsty, albo powietrze, albo glód, a staniemy przed tym domem, i przed obliczem twojem, gdyz imie twoje jest w domu tym, a zawolamy do ciebie w uciskach naszych, tedy wysluchasz i wybawisz.
\par 10 Teraz tedy, oto synowie Ammonowi, i Moabowi, i góra Seir, przez któryches ty nie dopuscil przejsc Izraelowi, gdy szli z ziemi Egipskiej, ale ich mineli, a nie wytracili ich.
\par 11 Oto teraz nam oni oddawaja, gdyz przyszli, aby nas wyrzucili z dziedzictwa twego, któres nam dal prawem dziedzicznem.
\par 12 O Boze nasz! izali ich nie bedziesz sadzil? W nasci zaiste nie masz zadnej mocy przeciw mnóstwu tak wielkiemu, które przyszlo na nas, i nie wiemy, co czynic mamy, tylko ku tobie obracamy oczy nasze.
\par 13 Wszystek tez lud Judzki stal przed Panem, i dziatki ich, zony ich, i synowie ich.
\par 14 Ale Jehazyjel, syn Zacharyjasza, syna Benajaszowego, syna Jehyjelowego, syna Matanijaszowego, Lewita z synów Asafowych, na którego przyszedl Duch Panski w posród onego zgromadzenia,
\par 15 Rzekl: Sluchajcie wszyscy z Judy, i obywatele Jeruzalemscy, i ty królu Jozafacie! Tak wam powiada Pan: Nie bójcie sie wy, ani sie lekajcie tego mnóstwa tak wielkiego; nie wasza jest walka, ale Boza.
\par 16 Jutro sie ruszcie przeciwko nim; oto oni pójda strona góry Sys, i znajdziecie ich na koncu potoku przeciw puszczy Jeruel.
\par 17 Nie wy sie potykac bedziecie w tej bitwie; stawcie sie, i stójcie, a ogladajcie wybawienie Panskie nad wami, o Judo, i Jeruzalemie! Nie bójciez sie, ani sie lekajcie; jutro wynijdziecie przeciwko nim, a Pan bedzie z wami.
\par 18 I poklonil sie Jozafat twarza ku ziemi, a wszystek lud Judzki, i obywatele Jeruzalemscy padli przed obliczem Panskiem, klaniajac sie Panu.
\par 19 Wstali tez Lewitowie z synów Kaatowych, i z synów Korego, i chwalili Pana, Boga Izraelskiego, glosem wielkim i wynioslym.
\par 20 Potem wstawszy rano ciagneli na puszcze Tekula; a gdy wychodzili, stanal Jozafat, i rzekl: Sluchajcie mie Judo, i obywatele Jeruzalemscy! Wierzcie Panu, Bogu waszemu, a bedziecie bezpieczni; wierzciez prorokom jego, a bedzie sie wam szczescilo.
\par 21 A wszedlszy w rade z ludem, postanowil spiewaków Panu, którzyby go chwalili w ozdobie swietobliwosci, idac przed uszykowanymi do bitwy, i mówiac: Wyslawiajcie Pana, albowiem na wieki milosierdzie jego.
\par 22 A wtenczas, gdy oni zaczeli spiewanie i chwaly, obrócil Pan zasadzke na synów Ammonowych i Moabowych, i na obywateli góry Seir, która byla przyszla przeciw Judzie, i bili sie sami.
\par 23 Bo powstali synowie Ammonowi i Moabowi przeciwko obywatelom góry Seir, aby ich pobili i wygladzili. A gdy juz do konca porazili onych, co mieszkali w Seir, oburzyl sie jeden przeciw drugiemu, az sie spólnie wybili.
\par 24 A wtem lud Judzki przyszedl do Masfa, blisko puszczy; i spojrzawszy na ono mnóstwo widzieli, a oto trupy lezaly po ziemi, tak iz nikt nie uszedl.
\par 25 Przetoz przyszedl Jozafat i lud jego, aby rozchwycili lupy ich; i znalezli przy nich bardzo wiele bogactw i na trupach klejnotów kosztownych, których rozchwycili miedzy sie tak wiele, ze ich zaniesc nie mogli: przez trzy dni brali one lupy, albowiem ich bylo bardzo wiele.
\par 26 Potem dnia czwartego zebrali sie w dolinie Beracha; bo iz tam blogoslawili Pana, przetoz nazwali imie miejsca onego dolina Beracha, az do dzisiejszego dnia.
\par 27 Zatem obrócili sie wszyscy mezowie Judzcy i Jeruzalemscy, i Jozafat przed nimi, aby sie wrócili do Jeruzalemu z radoscia; albowiem ucieszyl ich byl Pan nad nieprzyjaciolami ich.
\par 28 I wjechali do Jeruzalemu z harfami, i z cytrami, i z trabami, do domu Panskiego.
\par 29 Tedy przypadl strach Bozy na wszystkie królestwa ziemi, gdy uslyszaly, iz Pan walczyl przeciw nieprzyjaciolom ludu Izraelskiego.
\par 30 A tak uspokoilo sie królestwo Jozafatowe, bo mu dal odpocznienie Bóg jego zewszad.
\par 31 I królowal Jozafat nad Juda, a mial trzydziesci i piec lat, gdy królowac poczal, a dwadziescia i piec lat królowal w Jeruzalemie; a imie matki jego bylo Azuba, córka Salajowa.
\par 32 A chodzil droga ojca swego Azy, i nie uchylal sie od niej, czyniac co bylo dobrego w oczach Panskich.
\par 33 Wszakze wyzyny nie byly zniesione; bo jeszcze lud nie zgotowal byl serca swego ku Bogu ojców swoich.
\par 34 A ostatek spraw Jozafatowych pierwszych i poslednich jest zapisany w ksiedze Jehu, syna Hananiego, któremu bylo rozkazane, aby to wlozyl w ksiegi królów Izraelskich.
\par 35 Potem stowarzyszyl sie Jozafat, król Judzki, z Ochozyjaszem, królem Izraelskim, którego sprawy byly bardzo niepobozne.
\par 36 A stowarzyszyl sie z nim na to, aby nabudowali okretów, któreby chodzily do Tarsys; i budowali one okrety w Asjongaber.
\par 37 Przetoz prorokowal Eliezer, syn Dodawahowy z Maresy, przeciw Jozafatowi mówiac: Izes sie stowarzyszyl z Ochozyjaszem, rozerwal Pan sprawy twoje, i porozbijaly sie okrety, i nie mogly isc do Tarsys.

\chapter{21}

\par 1 Potem zasnal Jozafat z ojcami swymi, i pochowany jest z ojcami swymi w miescie Dawidowem, a królowal Joram, syn jego, miasto niego.
\par 2 Który mial braci, synów Jozafatowych: Azaryjasza i Jehijela, i Zacharyjasza, i Azaryjasza, i Michaela, i Sefatyjasza; wszyscy ci byli synowie Jozafata, króla Izraelskiego.
\par 3 Którym byl dal ojciec ich upominków wiele, srebra i zlota, i rzeczy kosztownych z miastami obronnemi w Judzie; ale królestwo oddal Joramowi, poniewaz on byl pierworodny.
\par 4 Nastapil tedy Joram na królestwo ojca swego, a zmocniwszy sie pozabijal wszystkich braci swoich mieczem, takze i niektórych z przedniejszych w Izraelu.
\par 5 Trzydziesci i dwa lata mial Joram, gdy poczal królowac, a osm lat królowal w Jeruzalemie.
\par 6 I chodzil drogami królów Izraelskich, jako czynil dom Achabowy, bo córka Achabowa byla zona jego; i czynil zle przed oczyma Panskiemi.
\par 7 Ale nie chcial Pan wytracic domu Dawidowego, dla przymierza, które byl uczynil z Dawidem, a iz byl przyrzekl dac mu pochodnie i synom jego po wszystkie dni.
\par 8 Za jego dni odstapili Edomczycy, aby nie byli poddanymi Judzie, a postanowili nad soba króla.
\par 9 Przetoz sie ruszyl Joram z hetmanami swymi, i ze wszystkiemi wozami swemi, a wstawszy w nocy, porazil Edomczyków, którzy go byli otoczyli i hetmanów wozów jego.
\par 10 Ale przeciez odstapili Edomczycy, aby nie byli pod moca Judy, az do dnia tego; odstapila tez i Lebna tegoz czasu, aby nie byla pod moca jego, przeto, iz byl Joram opuscil Pana, Boga ojców swoich.
\par 11 Nadto nabudowal wyzyn po górach Judzkich, i przywiódl w cudzolóstwo obywateli Jeruzalemskich, i przymuszal jako i Jude.
\par 12 Tedy przyszlo do niego pisanie od Elijasza proroka w ten sposób: Tak mówi Pan, Bóg Dawida, ojca twego: Przeto, izes nie chodzil drogami Jozafata, ojca twego, i drogami Azy, króla Judzkiego;
\par 13 Ales chodzil droga królów Izraelskich, a przywiodles w cudzolóstwo Jude, i obywateli Jeruzalemskich, tak jako cudzolozyl dom Achabowy; nadto i braci twoich, dom ojca twego, lepszych nad cie, pomordowales:
\par 14 Otoz Pan uderzy plaga wielka lud twój, i synów twoich, i zony twoje, i wszystke majetnosc twoje;
\par 15 Na cie tez przyjda niemocy wielkie, bolesc wnetrza twego, az wyplyna trzewa twoje dla bolesci dzien ode dnia ciezszej.
\par 16 A tak pobudzil Pan przeciwko Joramowi ducha Filistynczyków, i Arabczyków, którzy byli pograniczni Etyjopczykom;
\par 17 Którzy wtargnawszy do ziemi Judzkiej, splundrowali ja i pobrali wszystke majetnosc, która sie znalazla w domu królewskim; do tego i synów jego, i zony jego, tak, iz mu nie zostal syn, jedno Joachaz, najmlodszy z synów jego.
\par 18 A nad to wszystko zarazil go Pan na wnetrzu jego niemoca nieuleczona.
\par 19 A gdy dzien po dniu nastepowal, a czas dwóch lat wychodzil, wyplynely wnetrznosci jego z bolescia, i umarl w niemocach ciezkich; a nie uczynil mu lud jego przy pogrzebie zapalu, jako czynili zapal ojcom jego.
\par 20 Trzydziesci i dwa lata mial, gdy królowac poczal, a osm lat królowal w Jeruzalemie, a zszedl tak, iz go nikt nie zalowal; i pogrzebiony jest w miescie Dawidowem, wszakze nie w grobach królewskich.

\chapter{22}

\par 1 Tedy postanowili królem obywatele Jeruzalemscy Ochozyjasza, syna jego najmlodszego, miasto niego; bo wszystkich starszych pobila byla kupa swawolna, która byla przyszla z Arabczykami do obozu, a tak królowal Ochozyjasz, syn Jorama, króla Judzkiego.
\par 2 Czterdziesci i dwa lata bylo Ochozyjaszowi, gdy poczal królowac, a jeden rok królowal w Jeruzalemie; a imie matki jego bylo Atalija, córka Amrego.
\par 3 Ten tez chodzil drogami domu Achabowego; bo matka jego radzila mu, aby sie niepoboznie sprawowal.
\par 4 Przetoz czynil zle przed oczyma Panskiemi, jako dom Achabowy; tych bowiem mial za radców swoich po smierci ojca swego, na zginienie swoje.
\par 5 Bo wedlug rady ich chodzil, i jechal z Joramem, synem Achabowym, królem Izraelskim, na wojne przeciw Hazaelowi, królowi Syryjskiemu, do Ramot Galaadzkiego, kedy zranili Syryjczycy Jorama.
\par 6 A gdy sie wrócil, aby sie leczyl w Jezreelu, (albowiem mial rany, które mu zadano w Ramie, gdy sie potykal z Hazaelem, królem Syryjskim) tedy Azaryjasz, syn Jorama, króla Judzkiego, przyjechal do Jezreela, nawiedzac niemocnego Jorama, syna Achabowego, bo byl chory.
\par 7 A bylo to od Boga na upadek Ochozyjaszowi, ze przyjechal do Jorama. Albowiem przyjechawszy wyjechal z Joramem przeciw Jehu, synowi Namsy, którego byl pomazal Pan, aby wytracil dom Achabowy.
\par 8 A gdy sad wykonywal Jehu nad domem Achabowym, znalazl niektórych ksiazat Judzkich, i synów braci Ochozyjaszowych, którzy sluzyli Ochozyjaszowi, i pobil ich.
\par 9 Potem szukal Ochozyjasza, i pojmali go, gdy sie kryl w Samaryi, a przywiódlszy go do Jehu, zabili go, i pogrzebli go, bo mówili: Synci to jest Jozafata, który szukal Pana calem sercem swojem. A tak nie bylo nikogo w domu Ochozyjaszowem, któryby mógl otrzymac królestwo.
\par 10 Przetoz Atalija, matka Ochozyjaszowa, widzac, ze umarl syn jej, wstawszy wytracila wszystko nasienie królewskie z domu Judy.
\par 11 Ale Josabet, córka królewska, wziela Joaza, syna Ochozyjaszowego, i wykradla go z posród synów królewskich, których zabijano, i schowala go i mamke jego do gmachu, gdzie byly loza. I skryla go Josabet, córka króla Jorama, zona Jojady kaplana, (bo ona byla siostra Ochozyjaszowa) przed Atalija, aby go nie zabila.
\par 12 I byl z nimi w domu Bozym, bedac skrytym przez szesc lat, których Atalija królowala nad ta ziemia.

\chapter{23}

\par 1 A siódmego roku zmocniwszy sie Jojada, zaciagnal rotmistrzów, Azaryjasza, syna Jerohamowego, i Ismaela, syna Johananowego, i Azaryjasza, syna Obedowego, i Maasajasza, syna Adajaszowego, i Elizafata, syna Zychry, z soba w przymierze.
\par 2 Którzy obchodzac Judzka ziemie zebrali Lewitów ze wszystkich miast Judzkich, i przedniejszych z domów ojcowskich w Izraelu, a przyszli do Jeruzalemu.
\par 3 I uczynilo wszystko zgromadzenie przymierze w domu Bozym z królem; bo im byl rzekl Jojada: Oto syn królewski bedzie królowal, jako powiedzial Pan o synach Dawidowych.
\par 4 Toc jest co uczynicie: Trzecia czesc z was, którzy przychodzicie w sabat z kaplanów i z Lewitów, bedzie odzwiernymi w bramach.
\par 5 A trzecia czesc bedzie w domu królewskim, a trzecia czesc w bramie fundamentu; ale wszystek lud zostanie w sieniach domu Panskiego.
\par 6 A niechaj nikt nie wchodzi w dom Panski, tylko kaplani, a uslugujacy Lewitowie; ci niechaj wchodza, albowiem sa poswieceni; ale wszystek inny lud niech trzyma straz Panska.
\par 7 I obstapia Lewitowie króla zewszad, majac kazdy bron swa w rece swej; a ktobykolwiek wszedl w dom, niech bedzie zabity; a badzcie przy królu, gdy bedzie wchodzil, i gdy bedzie wychodzil.
\par 8 I uczynili Lewitowie, i wszystek lud Judzki, wedlug wszystkiego, co byl rozkazal Jojada kaplan; i wzial kazdy mezów swych, którzy przychodzili w sabat i którzy odchodzili w sabat, bo byl nie rozpuscil Jojada kaplan pocztów ich.
\par 9 I rozdal Jojada kaplan rotmistrzom wlócznie, i tarcze, i puklerze, które byly króla Dawida, które byly w domu Bozym.
\par 10 Postawil tez wszystek lud; a kazdy mial bron w rece swej, od prawej strony domu, az do lewej strony domu przeciwko oltarzowi, i domowi, okolo króla zewszad.
\par 11 Zatem wywiedli syna królewskiego, i wlozyli nan korone i swiadectwo, a postanowili go królem; i pomazali go Jojada i synowie jego, i mówili: Niech zyje król!
\par 12 Wtem uslyszawszy Atalija krzyk zbiegajacego sie ludu, i chwalacego króla, weszla do ludu do domu Panskiego.
\par 13 A gdy ujrzala, ze król stal na majestacie swoich w wejsciu, i ksiazeta i traby okolo króla, i wszystek lud onej ziemi weselacy sie, i trabiacy w traby, i spiewaki z instrumentami muzycznemi, i tych, którzy zaczynali spiewanie, tedy rozdarla Atalij a szaty swoje, mówiac: Sprzysiezenie! sprzysiezenie!
\par 14 Przetoz rozkazal wynijsc Jojada kaplan rotmistrzom i hetmanom wojska, i rzekl do nich: Wywiedzcie ja z zagrodzenia kosciola, a ktoby za nia szedl, niech bedzie zabity mieczem; bo byl kaplan rzekl: Nie zabijajcie jej w domu Panskim.
\par 15 I uczynili jej plac. A gdy przyszla ku wejsciu bramy, która wodzono konie do domu królewskiego, tamze ja zabili.
\par 16 Tedy uczynil Jojada przymierze miedzy Panem, i miedzy wszystkim ludem, i miedzy królem, aby byli ludem Panskim.
\par 17 Potem wszedl wszystek lud do domu Baalowego, i zburzyli go, i oltarze jego, i balwany jego polamali, Matana takze kaplana Baalowego zabili przed oltarzami.
\par 18 I postanowil znowu Jojada przelozonych nad domem Panskim pod rzadem kaplanów i Lewitów, których byl rozrzadzil Dawid w domu Panskim, aby ofiarowali calopalenia Panu, jako napisano w zakonie Mojzeszowym, z weselem, i z piesniami, wedlug rozrzadzenia Dawidowego.
\par 19 Postawil tez odzwiernych u bram domu Panskiego, aby tam nie wchodzil nieczysty dla jakiejkolwiek rzeczy.
\par 20 Potem wziawszy rotmistrzów i przedniejszych, i tych, którzy panowali nad ludem, i wszystek lud onej ziemi, prowadzili króla z domu Panskiego; i przyszli brama wyzsza do domu królewskiego, a posadzili króla na stolicy królestwa.
\par 21 I weselil sie wszystek lud onej ziemi, i uspokoilo sie miasto, gdy Atalije zabili mieczem.

\chapter{24}

\par 1 W siedmiu latach byl, Joaz, gdy królowac poczal, a czterdziesci lat królowal w Jeruzalemie. Imie matki jego bylo Sebija z Beersaby.
\par 2 I czynil Joaz, co bylo dobrego przed oczyma Panskiemi, po wszystkie dni Jojady kaplana.
\par 3 A Jojada dal mu dwie zony; i plodzil synów i córki.
\par 4 I stalo sie potem, ze umyslil w sercu swojem Joaz odnowic dom Panski.
\par 5 Przetoz zebrawszy kaplanów i Lewitów rzekl do nich: Wynijdzcie do miast Judzkich, i wybierajcie od wszystkiego Izraela pieniadze na poprawe domu Boga waszego na kazdy rok, a wy sie z tem pospieszcie; ale sie nie spieszyli Lewitowie.
\par 6 Tedy wezwal król Jojady, przedniejszego kaplana, i rzekl mu: Przeczze sie nie upominasz u Lewitów, aby znosili z Judy i z Jeruzalemu podarki postanowione przez Mojzesza, sluge Panskiego, zgromadzeniu Izraelskiemu, na namiot zgromadzenia?
\par 7 Bo Atalija niezbozna i synowie jej wylupili dom Bozy, a wszystkie rzeczy poswiecone z domu Panskiego obrócili na balwany.
\par 8 Przetoz rozkazal król, aby uczyniono skrzynie jedne, a postawiono ja przed brama domu Panskiego.
\par 9 I obwolano w Judzie i w Jeruzalemie, aby znoszono Panu podarek postanowiony przez Mojzesza, sluge Bozego, na Izraela na puszczy.
\par 10 I weselili sie wszyscy ksiazeta, i wszystek lud, a przynoszac, rzucali do onej skrzyni, az ja napelnili.
\par 11 A gdy przynosili skrzynie na rozkaz królewski przez rece Lewitów, (gdy widzieli, ze bylo wiele pieniedzy) przychodzil pisarz królewski, i przystaw kaplana najwyzszego, i wyprózniali skrzynie; potem ja odnosili, i stawiali ja na miejscu swem. Tak czynili na kazdy dzien, i zebrali pieniedzy bardzo wiele.
\par 12 Które oddawal król i Jojada przelozonym nad robota okolo domu Panskiego; a ci najmowali kamienników i ciesli do poprawy domu Panskiego, takze i kowali robiacych zelazem i miedzia, ku zmocnieniu domu Panskiego.
\par 13 A tak robili robotnicy, i bralo naprawe ono dzielo przez rece ich; i przywiedli dom Bozy do calosci swojej, i zmocnili go.
\par 14 A gdy dokonczyli, przyniesli przed króla i przed Jojade ostatek pieniedzy, a narobiono z nich naczynia do domu Panskiego, naczynia ku poslugiwaniu, i mozdzierzy i czasz, i innego naczynia zlotego i srebrnego, a ofiarowali calopalenia w domu Pansk im ustawicznie po wszystkie dni Jojady.
\par 15 Potem zstarzal sie Jojada, a bedac pelen dni, umarl; sto i trzydziesci lat mial, gdy umarl.
\par 16 I pochowano go w miescie Dawidowem z królmi, przeto, ze czynil dobrze w Izraelu, i Bogu, i domowi jego.
\par 17 A gdy umarl Jojada, przyszli ksiazeta Judzcy, i poklonili sie królowi; tedy ich usluchal król.
\par 18 Skad opusciwszy dom Pana, Boga ojców swych, sluzyli gajom i balwanom; przetoz przyszedl gniew na Jude i na Jeruzalem dla tego wystepku ich.
\par 19 I posylal do nich proroków, zeby ich nawrócili do Pana; którzy choc sie oswiadczali przeciw nim, ale ich przeciez nie sluchali.
\par 20 Owszem, gdy Duch Bozy wzbudzil Zacharyjasza, syna Jojady kaplana, (który stanawszy przed ludem, rzekl im: Tak mówi Bóg: Przeczze przestepujecie przykazania Panskie? Nie poszczesci sie wam; albowiem izescie wy opuscili Pana, on was tez opusci.)
\par 21 Tedy sie sprzysiegli przeciwko niemu, i ukamionowali go za rozkazaniem królewskim w sieni domu Panskiego.
\par 22 I nie pamietal król Joaz na milosierdzie, które byl uczynil z nim Jojada, ojciec jego, ale zabil syna jego; który gdy umieral, mówil: Niech to obaczy Pan, a zemsci sie.
\par 23 I stalo sie po roku, przyciagnelo przeciwko niemu wojsko Syryjskie, a przyszlo do Judy i do Jeruzalemu, i wygladzili z ludu wszystkich ksiazat ich, a wszystkie lupy ich poslali królowi w Damaszku.
\par 24 Bo w malym poczcie ludu przyciagnelo bylo wojsko Syryjskie; a wzdy Pan podal w rece ich bardzo wielkie wojsko, przeto, iz opuscili Pana, Boga ojców swoich. A tak nad Joazem wykonali sady.
\par 25 A gdy odciagneli od niego, zostawiwszy go w wielkich niemocach, sprzysiegli sie przeciwko niemu sludzy jego dla krwi synów Jojady kaplana, i zabili go na lozu jego. I tak umarl, a pochowano go w miescie Dawidowem; ale go nie pochowano w grobach królewskich.
\par 26 A cic sa, którzy sie byli sprzysiegli przeciw niemu: Zabat, syn Semaaty Ammonitki, i Jozabat, syn Semaryty Moabitki.
\par 27 Lecz o synach jego, i o wielkim podatku nan wlozonym, i o naprawie domu Bozego, to wszystko napisane w ksiegach królewskich; i królowal Amazyjasz, syn jego, miasto niego.

\chapter{25}

\par 1 Dwadziescia i piec lat mial Amazyjasz, gdy królowac poczal, a dwadziescia i dziewiec lat królowal w Jeruzalemie. Imie matki jego Joadana, z Jeruzalemu.
\par 2 I czynil co bylo dobrego przed oczyma Panskiemi, wszakze nie doskonalem sercem.
\par 3 I stalo sie, gdy bylo utwierdzone królestwo jego, ze pomordowal slugi swe, którzy zabili króla, ojca jego.
\par 4 Wszakze synów ich niepobil: ale uczynil, jako napisano w zakonie, w ksiegach Mojzeszowych, gdzie przykazal Pan, mówiac: Nie umra ojcowie za synów, ani synowie umra za ojców, ale kazdy za grzech swój umrze.
\par 5 Tedy zgromadzil Amazyjasz lud Judzki, i postanowil ich wedlug domów ojcowskich za pólkowników i za rotmistrzów po wszystkiem pokoleniu Judowem, i Benjaminowem, a policzywszy ich od dwudziestu lat i wyzej, znalazl ich trzy kroc sto tysiecy na wybór, gotowych do boju, noszacych drzewce i tarcze.
\par 6 Najal takze za pieniadze z Izraela sto tysiecy mezów duzych za sto talentów srebra.
\par 7 Lecz maz Bozy przyszedl do niego mówiac: Królu! niech nie wychodzi z toba wojsko Izraelskie; bo nie masz Pana z Izraelem i ze wszystkimi synami Efraimowymi.
\par 8 Ale jezli mi nie wierzysz, idz, i zmocnij sie ku bitwie, a porazi cie Bóg przed nieprzyjacielem; bo w mocy Bozej jest ratowac, i do upadku przywiesc.
\par 9 Tedy rzekl Amazyjasz mezowi Bozemu: A cóz mam czynic z tem stem talentów, którem dal wojsku Izraelskiemu? Odpowiedzial maz Bozy: Ma Pan skad ci moze dac daleko wiecej nadto.
\par 10 Przetoz oddzielil Amazyjasz to wojsko, które bylo przyszlo do niego z Efraima, aby szlo na miejsce swe; i rozgniewali sie bardzo na Jude, i wrócili sie do miejsca swego z wielkim gniewem.
\par 11 Lecz Amazyjasz zmocniwszy sie, wywiódl lud swój, i ciagnal na doline Soli, i porazil synów Seir dziesiec tysiecy.
\par 12 Dziesiec takze tysiecy zywo pojmali synowie Judzcy, i przywiedli ich na wierzch skaly, i zrzucili ich z wierzchu skaly, az sie wszyscy porozpukali.
\par 13 Ono zasie wojsko, które rozpuscil Amazyjasz, aby nie szlo z nim na wojne, wtargnelo do miast Judzkich, od Samaryi az do Betoronu, a poraziwszy w nich trzy tysiace ludu, zebrali korzysc wielka.
\par 14 A gdy sie Amazyjasz wrócil od porazki Idumejczyków, przyniósl z soba bogów synów Seir, i wystawil ich sobie za bogów, a klanial sie przed nimi, i kadzil im.
\par 15 Przetoz rozgniewal sie Pan bardzo na Amazyjasza, i poslal do niego proroka, który mu rzekl: Przeczze szukasz bogów ludu tego, którzy nie wyrwali ludu swego z reki twej?
\par 16 A gdy on do niego mówil, rzekl mu król: Izali cie za radce królewskiego obrano? Przestan tego, aby cie nie zabito. A tak przestal prorok; wszakze rzekl: Wiem, ze cie umyslil Bóg zatracic, gdyzes to uczynil, a nie usluchales rady mojej.
\par 17 Tedy naradziwszy sie Amazyjasz, król Judzki, poslal do Joaza, syna Joachaza, syna Jehu, króla Izraelskiego, mówiac: Pójdz, a wejrzymy sobie w oczy.
\par 18 I poslal Joaz, król Izraelski, do Amazyjasza, króla Judzkiego, mówiac: Oset, który jest na Libanie, poslal do cedru Libanskiego, mówiac: Daj córke twoje synowi memu za zone. Wtem idac tedy zwierz polny, który byl na Libanie, podeptal on oset.
\par 19 Mysliles: Otom porazil Edomczyków; przetoz wynioslo cie serce twoje, abys sie tem chlubil. Siedzze tedy w domu twym; przecz sie masz wdawac w to zle, abys upadl, ty i Juda z toba?
\par 20 Ale nie usluchal Amazyjasz; bo to bylo od Boga, aby ich podal w rece nieprzyjacielskie, przeto, ze szukali bogów Idumejskich:
\par 21 Wyciagnal tedy Joaz, król Izraelski, i wejrzeli sobie w oczy, on i Amazyjasz, król Judzki, w Betsemes, które jest w Judzie.
\par 22 I porazony jest Juda przed Izraelem, a pouciekali kazdy do namiotów swoich.
\par 23 Lecz Amazyjasza, króla Judzkiego, syna Joazowego, syna Joachaza, pojmal król Izraelski w Betsemes, i przywiódl go do Jeruzalemu, a obalil mury Jeruzalemskie od bramy Efraimskiej az do bramy naroznej, na czterysta lokci.
\par 24 I zabral wszystko zloto i srebro, i wszystkie naczynia, które sie znalazly w domu Bozym u Obededoma i w skarbach domu królewskiego, i ludzi, zastawnych, a wrócil sie do Samaryi.
\par 25 I zyl Amazyjasz, syn Joazowy, król Judzki, po smierci Joaza, syna Joachaza, króla Izraelskiego, pietnascie lat.
\par 26 A inne sprawy Amazyjaszowe, pierwsze i poslednie, izali nie sa zapisane w ksiegi królów Judzkich i Izraelskich?
\par 27 A od onego czasu, jako odpadl Amazyjasz od Pana, uczyniono przeciwko niemu sprzysiezenie w Jeruzalemie. Lecz on uciekl do Lachis; ale poslano za nim do Lachis, i zabito go tam.
\par 28 A przynióslszy go na koniach, pochowali go z ojcami jego w miescie Judzkiem.

\chapter{26}

\par 1 Tedy wszystek lud Judzki wzieli Uzyjasza, który mial szesnascie lat, i postanowili go królem miasto ojca jego Amazyjasza.
\par 2 Ten pobudowal Elat, a przywrócil je do Judy, gdy zasnal król z ojcami swymi.
\par 3 Szesnascie lat bylo Uzyjaszowi, gdy królowac poczal, a piecdziesiat i dwa lata królowal w Jeruzalemie; a imie matki jego Jechalija z Jeruzalemu.
\par 4 Ten czynil, co bylo dobrego w oczach Panskich wedlug wszystkiego, jako czynil Amazyjasz, ojciec jego.
\par 5 I szukal Boga za dnia Zacharyjasza, rozumiejacego widzenia Boze; a po wszystkie one dni, których szukal Pana, szczescil go Bóg.
\par 6 Bo ruszywszy sie, walczyl z Filistynami, i poburzyl mury w Get, i mury w Jabnie, i mury w Azocie, a pobudowal miasta w Azocie i w ziemi Filistynskiej.
\par 7 Albowiem pomagal mu Bóg przeciw Filistynom, i przeciw Arabczykom, którzy mieszkali w Gurbaalu, i przeciw Mahunitom.
\par 8 I dawali Ammonitowie Uzyjaszowi dary, a roznioslo sie imie jego az do samego Egiptu; bo sie byl zmocnil nader bardzo.
\par 9 I budowal Uzyjasz wieze w Jeruzalemie nad brama narozna, i nad brama doliny, i nad Mikzoa, i umocnil je.
\par 10 Pobudowal tez wieze na puszczy, i pokopal wiele studzien; bo mial bardzo wiele stad, tak w dolinach, jako i w równinach, i rolników, i winiarzy po górach, i na Karmelu: albowiem sie kochal w uprawianiu ról.
\par 11 Mial tez Uzyjasz wojsko gotowe do bitwy, które wychodzilo na wojne w pocztach swych wedlug liczby, jako byli obliczeni przez Jechijela kanclerza, i Maasajasza kaplana, pod sprawa Hananijasza, ksiazecia królewskiego.
\par 12 Wszystka liczba przedniejszych z domów ojcowskich, ludzi rycerskich, dwa tysiace i szesc set.
\par 13 A pod sprawa ich ludu walecznego trzy kroc i sto tysiecy, i siedm tysiecy i piec set, ludu sposobnego do wojny, na pomoc królowi przeciw nieprzyjacielowi.
\par 14 I zgotowal Uzyjasz wszystkiemu onemu wojsku tarcze i drzewce, i przylbice, i pancerze, i luki, i kamienie do proc.
\par 15 Naczynil tez w Jeruzalemie sztuk wojennych bardzo misternych, aby byly na wiezach, i na weglach ku wypuszczaniu strzal, i kamienia wielkiego; i roznioslo sie imie jego daleko, przeto, ze mial dziwna pomoc, az sie zmocnil.
\par 16 Ale gdy sie zmocnil, podnioslo sie serce jego ku zginieniu jego, i wystapil przeciw Panu, Bogu swemu, i wszedl do kosciola Panskiego, aby kadzil na oltarzu kadzenia.
\par 17 I wszedl za nim Azaryjasz kaplan, a z nim kaplanów Panskich osmdziesiat, mezów duzych.
\par 18 I staneli przeciw Uzyjaszowi królowi, a mówili mu: Nie twoja to rzecz Uzyjaszu! kadzic Panu, ale kaplanów, synów Aaronowych, którzy sa poswieceni, aby kadzili. Wynijdzze z swiatnicy; albowiemes wystapil, a nie bedziec to ku slawie przed Panem Bogiem.
\par 19 Przetoz sie rozgniewal Uzyjasz, majac w rekach swych kadzielnice, aby kadzil. A gdy sie srozyl przeciwko kaplanom, trad wystapil na czolo jego przed kaplanami w domu Panskim u oltarza kadzenia.
\par 20 A wejrzawszy nan Azaryjasz, kaplan najwyzszy, i wszyscy kaplani, a oto byl tredowatym na czole swojem; przetoz predko wygnali go stamtad; owszem i sam pospieszal wynijsc, przeto, iz go zarazil Pan.
\par 21 A tak byl król Uzyjasz tredowatym az do dnia smierci swej, i mieszkal w domu osobnym, bedac tredowatym; albowiem byl wylaczon z domu Panskiego; miedzytem Joatam, syn jego, byl nad domem królewskim, sadzac lud ziemi.
\par 22 A inne sprawy Uzyjaszowe, pierwsze i poslednie, opisal prorok Izajasz, syn Amosowy.
\par 23 I zasnal Uzyjasz z ojcami swymi, a pochowano go z ojcami jego na polu grobów królewskich; bo mówili: Tredowaty jest. A królowal Joatam, syn jego, miasto niego.

\chapter{27}

\par 1 Dwadziescia i piec lat mial Joatam, gdy królowac poczal, a szesnascie lat królowal w Jeruzalemie. Imie matki jego Jerusa, córka Sadokowa.
\par 2 I czynil, co dobrego bylo przed oczyma Panskiemi wedlug wszystkiego, jako czynil Uzyjasz, ojciec jego; wszakze nie wchadzal do kosciola Panskiego, a lud jeszcze byl zepsowany.
\par 3 On zbudowal brame domu Panskiego wysoka, i na murach Ofel wiele pobudowal.
\par 4 Nadto pobudowal miasta na górach Judzkich, a w lasach pobudowal palace i wieze.
\par 5 Ten tez walczyl z królem synów Ammonowych, i zwyciezyl ich. I dali mu synowie Ammonowi tegoz roku sto talentów srebra, i dziesiec tysiecy korcy pszenicy, i jeczmienia dziesiec tysiecy korcy; tylez mu dali synowie Ammonowi i drugiego, i trzeciego roku.
\par 6 A tak zmocnil sie Joatam; bo zgotowal drogi swoje przed Panem, Bogiem swoim.
\par 7 A inne sprawy Joatamowe, i wszystkie wojny jego, i drogi jego, sa napisane w ksiegach o królach Izraelskich i Judzkich.
\par 8 Dwadziescia i piec lat mial, gdy poczal królowac, a szesnascie lat królowal w Jeruzalemie.
\par 9 Potem zasnal Joatam z ojcami swymi, i pochowano go w miescie Dawidowem; a królowal Achaz, syn jego, miasto niego.

\chapter{28}

\par 1 Dwadziescia lat mial Achaz, gdy królowac poczal, a szesnascie lat królowal w Jeruzalemie, i nie czynil, co bylo dobrego przed oczyma Panskiemi, jako Dawid, ojciec jego;
\par 2 Ale chodzil drogami królów Izraelskich; nadto ulal i slupy balwochwalskie.
\par 3 Takze i sam kadzil w dolinie Benhennon, i palil synów swych ogniem wedlug obrzydliwosci pogan, których wygnal Pan przed synami Izraelskimi.
\par 4 Ofiarowal tez i kadzil na wyzynach, i na pagórkach, i pod kazdem drzewem galezistem.
\par 5 Przetoz dal go Pan, Bóg jego, w reke króla Syryjskiego, którzy poraziwszy go, pojmali z ludu jego wiezniów wiele, a przywiedli ich do Damaszku. Nadto i w reke króla Izraelskiego podany jest, który go porazil porazka wielka.
\par 6 Albowiem Facejasz, syn Romelijaszowy, pobil w Judzie sto i dwadziescia tysiecy dnia jednego, wszystko mezów walecznych, przeto, iz opuscili Pana, Boga ojców swoich.
\par 7 Zychry takze, mocarz Efraimski, zabil Maasajasza, syna królewskiego, i Asrykama, przelozonego domu jego, i Elkana, wtórego po królu.
\par 8 Nadto pojmali synowie Izraelscy z braci swych dwa kroc sto tysiecy niewiast, synów, i córek, i bardzo wiele lupów pobrali od nich, i zaprowadzili korzysc do Samaryi.
\par 9 I byl tam prorok Panski, imieniem Obed, który zaszedlszy onemu wojsku idacemu do Samaryi, rzekl im: Oto, rozgniewawszy sie Pan, Bóg ojców waszych, na Jude, podal ich w reke wasze, a wyscie ich pomordowali w popedliwosci, która az do nieba przyszla.
\par 10 A jeszcze lud z Judy i z Jeruzalemu chcecie sobie podbic za niewolników i za niewolnice; azaz i przy was samych nie ma wystepku przeciw Panu, Bogu waszemu?
\par 11 Przetoz teraz mie sluchajcie, a odwiedzcie wiezniów, którychescie pojmali z braci waszych; bo pewnie gniew popedliwosci Panskiej wisi nad wami.
\par 12 Tedy powstali mezowie z ksiazat synów Efraimowych: Azaryjasz, syn Johananowy, Barachyjasz, syn Mesyllemotowy, i Ezechyjasz, syn Sallumowy, i Amasa, syn Hadlajowy przeciwko tym, którzy sie wracali z wojny;
\par 13 I rzekli do nich: Nie wodzcie tu tych wiezniów; bo grzech przeciwko Panu na nas myslicie przywiesc, przyczyniajac do grzechów naszych, i do wystepków naszych: bo wielki jest grzech nasz, i gniew popedliwosci nad Izraelem.
\par 14 Przetoz ono wojsko zostawilo i wiezniów i lupy swe przed ksiazetami, i przed wszystkiem zgromadzeniem.
\par 15 A powstawszy mezowie, którzy sa z imienia mianowani, wzieli onych wiezniów, a wszystkich obnazonych miedzy nimi przyodziali z onych korzysci, a przyodziawszy ich i dawszy im obuwie, nakarmili ich, i napoili ich, i pomazali ich, i odprowadzili na oslach kazdego slabego, a przyprowadzili ich do Jerycha, miasta palm, do braci ich: a potem sie wrócili do Samaryi.
\par 16 Naonczas poslal król Achaz do królów Assyryjskich, aby mu pomoc dali:
\par 17 Bo jeszcze byli przyciagneli i Edomczycy, i porazili Jude, a nabrali wiezniów.
\par 18 Nadto i Filistynowie wtargneli do miast w równinach, i na poludnie do Judy, a wzieli Betsemes, i Ajalon, i Gaderot, i Sokot i wsi jego, i Tamne i wsi jego, i Gimzo i wsi jego, a mieszkali w nich.
\par 19 Pan bowiem ponizal Jude dla Achaza, króla Izraelskiego, przeto, iz odwrócil Jude, aby sie przewrotnie obchodzil z Panem.
\par 20 I przyciagnal do niego Tyglat Filneser, król Assyryjski, który go bardziej ucisnal, anizeli mu pomógl.
\par 21 A chociaz pobral Achaz skarby z domu Panskiego, i z domu królewskiego, i od ksiazat, a dal królowi Assyryjskiemu, przeciez go nie ratowal.
\par 22 A czasu najwiekszego swego ucisku przyczynial grzechów przeciwko Panu. Takic byl król Achaz.
\par 23 Albowiem ofiarowal bogom z Damaszku, od których byl porazony, i mówil: Poniewaz bogowie królów Syryjskich pomagaja im, bede im ofiarowal, aby mie ratowali; ale mu oni byli na upadek, i wszystkiemu Izraelowi.
\par 24 Przetoz pobral Achaz naczynia domu Bozego, i pokruszyl one naczynia domu Bozego, i zamknal drzwi domu Panskiego, i pobudowal sobie oltarze po wszystkich katach w Jeruzalemie.
\par 25 Takze i w kazdem miescie Judzkiem poczynil wyzyny, aby kadzil bogom cudzym, i wzruszyl ku gniewu Pana, Boga ojców swoich,
\par 26 Ale inne sprawy jego, i wszystkie postepki jego, pierwsze i poslednie, zapisane sa w ksiegach o królach Judzkich i Izraelskich.
\par 27 I zasnal Achaz z ojcami swymi, i pochowali go w miescie w Jeruzalemie; bo go nie wprowadzili do grobów królów Izraelskich; a Ezechyjasz, syn jego, królowal miasto niego.

\chapter{29}

\par 1 A Ezechyjasz gdy poczal królowac, mial dwadziescia i piec lat; a dwadziescia i dziewiec lat królowal w Jeruzalemie. Imie matki jego Abi, córka Zacharyjaszowa.
\par 2 A czynil co bylo dobrego przed oczami Panskiemi, wedlug wszystkiego, jako czynil Dawid ojciec jego.
\par 3 Ten roku pierwszego królowania swego, miesiaca pierwszego, otworzyl drzwi domu Panskiego, i poprawil je.
\par 4 I przywiódl kaplanów i Lewitów, a zgromadzil ich na ulice wschodnia.
\par 5 I rzekl do nich: Sluchajcie mie Lewitowie: Teraz sie poswieccie: poswieccie tez i dom Pana, Boga ojców waszych, i wyrzuccie plugastwa z swiatnicy:
\par 6 Albowiem zgrzeszyli ojcowie nasi, i czynili zle przed oczyma Pana, Boga naszego, opuszczajac go, i odwracajac oblicza swoje od przybytku Panskiego, a obracajac sie tylem do niego.
\par 7 Zamkneli tez drzwi u przysionka, i pogasili lampy, a kadzidlem nie kadzili, ani calopalenia nie ofiarowali w swiatnicy Bogu Izraelskiemu.
\par 8 Przetoz byl gniew Panski nad Juda, i nad Jeruzalemem, a podal ich na rozproszenie, na spustoszenie, i na posmiech, jako sami widzicie oczyma waszemi.
\par 9 Bo oto polegli ojcowie nasi od miecza, a synowie nasi, i córki nasze, i zony nasze zawiedzione sa w niewole dla tego.
\par 10 Teraz tedy umyslilem uczynic przymierze z Panem, Bogiem Izraelskim, aby odwrócil od nas gniew popedliwosci swojej.
\par 11 Synowie moi! nie badzciez juz niedbalymi; bo was Pan obral, abyscie stojac przed nim sluzyli mu, a byli slugami jego, i kadzili mu.
\par 12 Tedy powstali Lewitowie: Machat, syn Amasajowy, i Joel, syn Azaryjaszowy, z synów Kaatowych, a z synów Merarego: Cys, syn Abdy, i Azaryjasz, syn Jehaleelowy; a z Giersonczyków: Joach, syn Zamy, i Eden, syn Joachowy:
\par 13 A z synów Elisafanowych: Symry i Jehijel: a z synów Asafowych: Zacharyjasz i Matanijasz;
\par 14 A z synów Hemanowych: Jehijel i Symchy; a z synów Jedytunowych: Semejasz i Uzyjel.
\par 15 I zgromadzili braci swych, którzy poswieciwszy sie przyszli wedlug rozkazania królewskiego, i rozkazania Panskiego, aby wyczyscili dom Panski.
\par 16 A wszedlszy kaplani do domu Panskiego, aby go oczyscili, wyniesli wszystkie plugastwa, które znalezli w kosciele Panskim, do sieni domu Panskiego; a Lewitowie zabrawszy to wyniesli precz do potoku Cedron.
\par 17 I poczeli pierwszego dnia miesiaca pierwszego poswiecac, a dnia ósmego tego miesiaca weszli do przysionku Panskiego i poswiecali dom Panski przez osm dni, a dnia szesnastego, miesiaca pierwszego, dokonczyli.
\par 18 Potem weszli do króla Ezechyjasza, i rzekli: Oczyscilismy wszystek dom Panski, i oltarz calopalenia, i wszystkie naczynia jego, i stól pokladny i wszystkie naczynia jego;
\par 19 Takze wszystko naczynie, które byl odrzucil król Achaz za królowania swego, gdy grzeszyl, zgotowalismy i poswiecili; a oto sa przed oltarzem Panskim.
\par 20 A tak wstawszy rano król Ezechyjasz zgromadzil przedniejszych miasta, i szedl do domu Panskiego.
\par 21 I przywiedziono mu cielców siedm, i baranów siedm, i baranków siedm, i kozlów siedm, na ofiare za grzech, za królestwo, i za swiatnice, i za Jude, a rozkazal synom Aaronowym, kaplanom, aby ofiarowali na oltarzu Panskim.
\par 22 A tak pobili one woly, a kaplani wziawszy krew ich kropili po oltarzu; pobili tez i barany, a kropili krwia ich po oltarzu; pobili tez i baranki, a kropili krwia ich po oltarzu.
\par 23 Przywiedli kozly tez na ofiare za grzech przed króla i przed zgromadzenie, którzy wlozyli rece swoje na nie.
\par 24 I pobili je kaplani, i oczyscili krwia ich oltarz na oczyszczenie wszystkiego Izraela; albowiem za wszystkiego Izraela rozkazal król ofiarowac calopalenie i ofiare za grzech.
\par 25 Postanowil tez i Lewitów w domu Panskim z cymbalami, i z cytrami, i z harfami, wedlug rozkazania Dawidowego, i Gada, widzacego królewskiego, i Natana proroka; bo to bylo rozkazanie Panskie przez proroków jego.
\par 26 A tak stali Lewitowie z instrumentami Dawidowemi, i kaplani z trabami.
\par 27 I rozkazal Ezechyjasz, aby ofiarowali calopalenia na oltarzu; a gdy sie zaczelo calopalenie, poczelo sie spiewanie Panu, i trabienie, i granie na instrumentach Dawida, króla Izraelskiego.
\par 28 Tedy wszystko zgromadzenie klanialo sie, a spiewacy spiewali, i trebacze trabili; co wszystko trwalo, póki sie nie dokonczylo calopalenie.
\par 29 A gdy sie skonczylo calopalenie, poklekneli król i wszyscy, którzy z nim byli, i modlili sie.
\par 30 Tedy rozkazal król Ezechyjasz i ksiazeta Lewitom, aby chwalili Pana slowy Dawidowemi, i Asafa, widzacego; chwalili z weselem wielkiem, a klaniajac sie modlili sie.
\par 31 Zatem rzekl Ezechyjasz, mówiac: Terazescie poswiecili rece wasze Panu; przystapciez, a przywiedzcie ofiary spokojne, i ofiary chwaly do domu Panskiego. Przetoz przywiodlo ono zgromadzenie ofiary spokojne i ofiary chwaly, i kazdy ochotnego serca przywiódl ofiary na calopalenie.
\par 32 I byla liczba ofiar na calopalenie, które przywiodlo zgromadzenie, wolów siedmdziesiat, baranów sto, baranków dwiescie, wszystko to na calopalenie Panu.
\par 33 Innych tez rzeczy poswieconych bylo: wolów szesc set, i owiec trzy tysiace,
\par 34 Lecz kaplanów malo bylo, tak, iz nie mogli nadazyc odzierac ze skóry wszystkich ofiar na calopalenie: przetoz im pomagali Lewitowie, bracia ich, póki nie dokonczyli onej pracy, i póki sie nie poswiecili drudzy kaplani; albowiem Lewitowie byli och otniejsi, aby sie poswiecili, niz kaplani.
\par 35 Nadto i calopalenia bylo bardzo wiele z tlustosciami spokojnych ofiar, i z mokremi ofiarami na calopalenie. A tak wygotowana byla sluzba domu Panskiego.
\par 36 I weselil sie Ezechyjasz i wszystek lud z tego, co Bóg ludowi przygotowal; bo sie ta rzecz byla z predka stala.

\chapter{30}

\par 1 Potem rozeslal Ezechyjasz do wszystkiego Izraela i do Judy; takze tez listy napisal do Efraima i do Manasesa, aby przyszli do domu Panskiego do Jeruzalemu i obchodzili swieto przejscia Panu, Bogu Izraelskiemu.
\par 2 Bo uradzil król i ksiazeta jego i wszystko zgromadzenie w Jeruzalemie, aby obchodzili swieto przejscia miesiaca wtórego;
\par 3 Gdyz nie mogli obchodzic czasu swego, przeto, iz kaplanów poswieconych nie bylo, ile ich bylo potrzeba, i lud nie byl zgromadzony do Jeruzalemu.
\par 4 A podobala sie ta rzecz królowi i wszystkiemu zgromadzeniu.
\par 5 I postanowili, aby obwolano po wszystkim Izraelu, od Beersaby az do Dan, zeby sie zeszli na obchód swieta przejscia Panu, Bogu Izraelskiemu, do Jeruzalemu; albowiem juz go dawno nie obchodzili, jako bylo napisane.
\par 6 Przetoz poslowie szli z listami od króla i od ksiazat jego po wszystkim Izraelu i Judzie z rozkazem królewskiem, mówiac: Synowie Izraelscy! nawróccie sie do Pana, Boga Abrahamowego, Izaakowego, i Izraelowego, a on sie nawróci do ostatków, które z was uszly z rak królów Assyryjskich.
\par 7 I nie badzcie jako ojcowie wasi, i jako bracia wasi, którzy wystapili przeciwko Panu, Bogu ojców swoich; i podal ich w spustoszenie, jako sami widzicie.
\par 8 Teraz tedy nie zatwardzajcie karku waszego, jako ojcowie wasi: dajcie reke Panu, a pójdzcie do swiatnicy jego, która poswiecil na wieki, i sluzcie Panu, Bogu waszemu, a odwróci sie od was gniew popedliwosci jego.
\par 9 Albowiem jezli sie nawrócicie do Pana, bracia wasi, i synowie wasi milosierdzie otrzymaja u tych, którzy ich zawiedli w niewole, tak, iz sie nawróca do tej ziemi; milosierny bowiem, i dobrotliwy jest Pan, Bóg wasz, a nie odwróci od was oblicza swe go, jezli sie nawrócicie do niego.
\par 10 A gdy oni poslowie chodzili od miasta do miasta przez ziemie Efraimowa o Manasesowa az do Zabulon, nasmiewali sie z nich, i szydzili z nich.
\par 11 Wszakze niektórzy mezowie z Aser, i z Manase, i z Zabulon, ukorzywszy sie przyszli do Jeruzalemu.
\par 12 W Judzie tez juz byla reka Boza, gdy im dal serce jedno, aby czynili rozkazanie królewskie i ksiazat, wedlug slowa Panskiego.
\par 13 I zebralo sie do Jeruzalemu wiele ludu, aby obchodzili swieto uroczyste przasników miesiaca wtórego; a bylo zgromadzenie bardzo wielkie.
\par 14 Tedy powstawszy zniesli oltarze, które byly w Jeruzalemie, wszystkie tez oltarze, na których kadzono, porozwalali, a wrzucili do potoku Cedron.
\par 15 Potem ofiarowali baranka wielkanocnego, dnia czternastego, miesiaca wtórego; a kaplani i Lewitowie zawstydziwszy sie, poswiecali sie, a przywodzili calopalenie do domu Panskiego.
\par 16 I stali w porzadku swym wedlug zwyczaju swego, i wedlug zakonu Mojzesza, meza Bozego; a kaplani kropili krwia, wziawszy ja z reki Lewitów.
\par 17 A iz takich bylo wiele w zgromadzeniu, którzy sie byli nie poswiecili, przetoz Lewitowie ofiarowali ofiary swieta przejscia za kazdego nieczystego, aby byl poswiecony Panu.
\par 18 Bo wielka liczba ludu tego, to jest, wiele ich z Efraima, i z Manasesa, i z Isaschara, i z Zabulonu nie byli oczyszczeni, a przeciez jedli baranka wielkanocnego, inaczej niz napisano; ale sie Ezechyjasz modlil za nich, mówiac: Dobrotliwy Pan niech oczysci kazdego,
\par 19 Którykolwiek zgotowal wszystko serce swe, aby szukal Boga, Pana Boga ojców swoich, chocby oczyszczony nie byl wedlug oczyszczenia swiatnicy.
\par 20 I wysluchal Pan Ezechyjasza, i zachowal lud.
\par 21 A tak obchodzili synowie Izraelscy, którzy byli w Jeruzalemie, uroczyste swieto przasników przez siedm dni z weselem wielkiem: i chwalili Pana. Lewitowie na kazdy dzien, a kaplani na instrumentach slawili moc Panska.
\par 22 I mówil Ezechyjasz laskawie do wszystkich Lewitów, którzy mieli dobre rozumienie o Panu. I jedli przez siedm dni onego swieta, sprawujac ofiary spokojne, a wyslawiajac Pana, Boga ojców swoich.
\par 23 Tedy uradzilo wszystko zgromadzenie, aby to jeszcze czynili przez drugie siedm dni; a tak obchodzili znowu siedm dni z weselem.
\par 24 Albowiem Ezechyjasz, król Judzki, dal byl zgromadzeniu tysiac cielców, i siedm tysiecy owiec: ksiazeta tez dali zgromadzeniu tysiac cielców, i owiec dziesiec tysiecy. I poswiecilo sie kaplanów bardzo wiele.
\par 25 A tak weselilo sie wszystko zgromadzenie Judzkie, i kaplani, i Lewitowie, i wszystko zgromadzenie, które bylo przyszlo z Izraela, i przychodniowie, którzy przyszli z ziemi Izraelskiej, i mieszkajacy w Judzie.
\par 26 I bylo wielkie wesele w Jeruzalemie; bo ode dni Salomona, syna Dawidowego, króla Izraelskiego, nic takiego nie bylo w Jeruzalemie.
\par 27 Potem powstali kaplani i Lewitowie, i blogoslawili ludowi; a wysluchany jest glos ich, i przyszla modlitwa ich do mieszkania swietobliwosci Panskiej, do nieba.

\chapter{31}

\par 1 A gdy sie to wszystko odprawilo, wyszedl wszystek lud Izraelski, który sie znajdowal w miastach Judzkich, i polamali slupy, a wyrabali gaje, i poburzyli wyzyny i oltarze po wszystkim Judzie i Benjaminie, i w Efraimie, i w Manasesie, az do szcze tu; a potem sie wrócili wszyscy synowie Izraelscy, kazdy do osiadlosci swojej, i do miasta swego.
\par 2 I postanowil Ezechyjasz porzadki kaplanów i Lewitów wedlug rozdzialów ich, kazdego wedlug powinnosci urzedu jego, kaplanów i Lewitów do calopalenia i ofiar spokojnych, aby sluzyli i wyslawiali i chwalili Pana w bramach obozu jego.
\par 3 Takze dzial z królewskiej majetnosci ku sprawowaniu calopalenia rano i w wieczór, takze calopalenia w sabaty, i na nowiu miesiaca, i w uroczyste swieta, jako napisane w zakonie Panskim.
\par 4 Rozkazal tez ludowi mieszkajacemu w Jeruzalemie, aby oddawali dzial kaplanom i Lewitom, aby byli tem pilniejszymi w zakonie Panskim.
\par 5 A gdy sie ta rzecz rozglosila, zniesli synowie Izraelscy wiele pierwocin zboza, moszczu i oliwy, i owocu palmowego, i wszystkich urodzajów polnych, i dziesieciny ze wszystkiego bardzo wiele przynosili.
\par 6 Nadto synowie Izraelscy i Judzcy, którzy mieszkali w miastach Judzkich, i oni dziesiecine z wolów i z owiec, i dziesiecine z rzeczy swietych, poswieconych Panu, Bogu ich, znióslszy skladli na gromady.
\par 7 Trzeciego miesiaca poczeli zakladac te gromady, a miesiaca siódmego dokonali.
\par 8 Tedy przyszedlszy Ezechyjasz z ksiazetami, obaczyl one gromady, i blogoslawili Panu i ludowi jego Izraelskiemu.
\par 9 Zatem wywiadywal sie Ezechyjasz od kaplanów i Lewitów o onych gromadach.
\par 10 Któremu odpowiedzial Azaryjasz kaplan najwyzszy z domu Sadokowego, mówiac: Jako poczeto te ofiary znaszac do domu Panskiego, jedlismy, i bylismy nasyceni, a jeszcze zostalo bardzo wiele: bo Pan blogoslawil ludowi swemu, i tej wielkosci, która jeszcze zostala.
\par 11 I rozkazal Ezechyjasz, aby sprawiono szpichlerze przy domu Panskim. I sprawiono je.
\par 12 A zniesiono tam wiernie ofiary podnoszenia, i dziesieciny, i rzeczy poswiecone; a nad nimi byl przelozonym Kienanijasz Lewita, i Symchy, brat jego wtóry.
\par 13 Jehijel takze, i Azaryjasz, i Nahat, i Asael, i Jerymot, i Josabad, i Elijel, i Ismachyjasz, i Machat, i Benajasz, byli szafarzami przy rece Kienanijasza, i Symchy, brata jego, z rozkazania Ezechyjasza króla, i Azaryjasza, przedniejszego w domu Bozym.
\par 14 Kore tez, syn Jemny, Lewita, odzwierny bramy na wschód slonca, byl nad rzeczami dobrowolnie ofiarowanemi Bogu, aby rozdzielal ofiary Panu i rzeczy najswietsze.
\par 15 A byli mu na pomoc Eden, i Minjamin, i Jesua, i Semejasz, Amaryjasz, i Sechanijasz po miastach kaplanskich, mezowie wierni, aby rozdawali braciom swym dzialy, tak wielkiemu jako i malemu:
\par 16 Tak z narodu ich mezczyznie we trzech latach i wyzej, jako kazdemu wchodzacemu do domu Panskiego, do powinnosci kazdodziennej, wedlug urzedów ich, i wedlug uslug ich, i wedlug podzialów ich;
\par 17 I tym, którzy byli policzeni w narodzie kaplanskim wedlug domów ojców ich, i Lewitom od tego, który mial dwadziescia lat i wyzej, wedlug poslug i podzialów ich;
\par 18 I narodowi ich, wszystkim dziatkom ich, i zonom ich, i synom ich, i córkom ich, owa wszystkiemu zgromadzeniu; bo sie oni wiernie poswiecili na urzad swietobliwoscia.
\par 19 Synom takze Aaronowym, kaplanom, na polach przedmiejskich miast ich, po wszystkich miastach, ci mezowie, którzy z imienia mianowani sa, oddawali dzialy, kazdemu mezczyznie z kaplanów, i kazdemu urodzonemu z Lewitów.
\par 20 I uczynil tak Ezechyjasz po wszystkiem Judztwie; i czynil, co bylo dobrego i prawego i prawdziwego przed obliczem Pana Boga swego.
\par 21 I w kazdej sprawie, która zaczal okolo uslugi domu Bozego, i w zakonie i w przykazaniu, szukajac Boga swego, wszystko czynil z calego serca swego, i szczescilo mu sie.

\chapter{32}

\par 1 Po tych sprawach i pewnem ich postanowieniu przyciagnal Sennacheryb, król Assyryjski, a wtargnawszy do Judzkiej ziemi, polozyl sie obozem przeciw miastom obronnym, a umyslil je sobie dobyc.
\par 2 A widzac Ezechyjasz, ze przyciagnal Sennacheryb, a iz twarz swoje obrócil, aby walczyl przeciw Jeruzalemowi:
\par 3 Tedy wszedl w rade z ksiazetami swymi i z rycerstwem swem, aby zatkali zródla wód, które byly za miastem; i pomogli mu.
\par 4 Bo zebrawszy sie lud wielki zatkali wszystkie zródla, i potok, który plynal przez posrodek ziemi, mówiac: Czemuzby przyszedlszy królowie Assyryjscy mieli znalesc tak wiele wód?
\par 5 A pokrzepiwszy sie, pobudowal wszystkie mury obalone, i nabudowal wiez, przytem zewnatrz drugi mur; i zmocnil Mello w miescie Dawidowem, i poczynil broni bardzo wiele, i tarczy.
\par 6 Postanowil tez hetmanów wojennych nad ludem, których zgromadzil do siebie na ulice bramy miejskiej, i mówil laskawie do nich, a rzekl:
\par 7 Zmacniajcie sie, i meznie sobie poczynajcie; nie bójcie sie, ani sie lekajcie twarzy króla Assyryjskiego, ani twarzy wszystkiego mnóstwa, które jest z nim; bo wiekszy jest z nami nizeli z nim.
\par 8 Z nimi jest ramie cielesne; ale z nami jest Pan, Bóg nasz, aby nas ratowal i odprawial wojny nasze. Tedy spolegl lud na slowach Ezechyjasza, króla Judzkiego.
\par 9 Potem poslal Sennacheryb, król Assyryjski, slugi swe do Jeruzalemu, (a sam dobywal Lachis, a wszystka moc jego byla z nim,)do Ezechyjasza, króla Judzkiego, i do wszystkich z Judy, którzy byli w Jeruzalemie, mówiac:
\par 10 Tak mówi Sennacheryb, król Assyryjski: W czemze wzdy ufacie, ze siedzicie w murach Jeruzalemskich?
\par 11 Izali Ezechyjasz nie zwodzi was, aby was pomorzyl glodem i pragnieniem, mówiac: Pan, Bóg nasz, wyrwie nas z reki króla Assyryjskiego?
\par 12 Izali nie ten Ezechyjasz zniósl wyzyny jego, i oltarze jego, i rozkazal Judzie i Jeruzalemczykom, mówiac: Przy jednym tylko oltarzu klaniac sie bedziecie, i na nim kadzic?
\par 13 Izali nie wiecie, com uczynil ja i ojcowie moi wszystkim narodom ziemskim? Azaz jakim sposobem mogli bogowie narodów ziemskich wyrwac ziemie swoje z reki mojej?
\par 14 Któz byl ze wszystkich bogów onych narodów, które wytracili ojcowie moi, coby mógl wybawic lud swój z reki mojej, aby tez mógl Bóg wasz wyrwac was z reki mojej?
\par 15 Przetoz teraz niech was nie zwodzi Ezechyjasz, a niech was na to nie namawia, ani mu wierzcie. Jezlic nie mógl zaden bóg wszystkich narodów i królestw wyrwac ludu swego z reki mojej, i z reki ojców moich, pogotowiu Bóg wasz nie wyrwie was z reki mojej.
\par 16 Nadto jeszcze mówili sludzy jego przeciw Panu Bogu, i przeciwko Ezechyjaszowi sludze jego.
\par 17 Listy tez pisal, uragajac Panu, Bogu Izraelskiemu, a mówiac przeciwko niemu temi slowy: Jako bogowie narodów ziemskich nie wyrwali ludu swego z reki mojej, tak nie wyrwie Bóg Ezechyjaszowy ludu swego z reki mojej.
\par 18 I wolali glosem wielkim po zydowsku przeciwko ludowi Jeruzalemskiemu, który byl na murach, straszac go i trwozac go, aby tak miasto wzieli.
\par 19 A mówili przeciw Bogu Jeruzalemskiemu, jako przeciw bogom narodów ziemskich, którzy sa robota rak ludzkich.
\par 20 Tedy sie modlil Ezechyjasz król, i Izajasz prorok, syn Amosa, dla tego, i krzyczeli ku niebu.
\par 21 I poslal Pan Aniola, który wytracil kazdego mocarza w wojsku, i wodza, i hetmana w obozie króla Assyryjskiego. I wrócil sie z pohanbieniem twarzy do ziemi swojej. A gdy wszedl do domu boga swego, ci, którzy wyszli z zywota jego, tam go zabili mie czem.
\par 22 A tak wybawil Pan Ezechyjasza i obywateli Jeruzalemskich z rak Sennacheryba, króla Assyryjskiego, i z rak wszystkich, a sprawil im pokój zewszad.
\par 23 Tedy wiele ich przynosilo Panu dary do Jeruzalemu i upominki kosztowne Ezechyjaszowi, królowi Judzkiemu; i wywyzszony jest w oczach wszystkich narodów potem.
\par 24 W one dni zachorowal Ezechyjasz az na smierc, i modlil sie Panu; który mówil do niego, a dal mu znak.
\par 25 Ale nie wedlug dobrodziejstw sobie uczynionych sprawowal sie Ezechyjasz, bo sie wynioslo serce jego; przetoz powstal przeciw niemu gniew, i przeciw Judzie, i przeciw Jeruzalemowi.
\par 26 Ale gdy sie upokorzyl Ezechyjasz (bo sie bylo wynioslo serce jego) on i obywatele Jeruzalemscy, nie przyszedl na nich gniew Panski za dni Ezechyjaszowych.
\par 27 A mial Ezechyjasz bogactwa i slawe bardzo wielka; bo sobie zebral skarby srebra i zlota, i kamieni drogich, i rzeczy wonnych, i rynsztunku, i wszelakiego naczynia kosztownego.
\par 28 Mial tez szpichlerze dla urodzajów zboza i wina, i oliwy, i obory dla bydel, i zwierzyniec dla rozmaitych zwierzat.
\par 29 Miasta tez sobie pobudowal, i mial stada owiec, i wolów mnóstwo; albowiem mu dal Bóg majetnosc bardzo wielka.
\par 30 Ten tez Ezechyjasz zatkal zródlo wód w Gichonie wyzsze, a przywiódl je dolem na zachód slonca ku miastu Dawidowemu; i szczescilo sie Ezechyjaszowi we wszystkich sprawach jego.
\par 31 A wszakze dla poslów ksiazat Babilonskich, którzy byli poslani do niego, aby go pytali o znak, który sie byl stal w ziemi, opuscil go Bóg, aby go kusil, a izby wiedziano wszystko, co bylo w sercu jego.
\par 32 Ale inne sprawy Ezechyjaszowe, i dobroczynnosci jego, napisane sa w widzeniu Izajasza proroka, syna Amosowego, i w ksiegach królów Judzkich i Izraelskich.
\par 33 I zasnal Ezechyjasz z ojcami swymi, i pogrzebiony jest nad grobami synów Dawidowych. I wyrzadzili mu wszystek Juda i obywatele Jeruzalemscy uczciwosc przy smierci jego. A królowal Manases, syn jego, miasto niego.

\chapter{33}

\par 1 We dwunastym roku byl Manases, gdy królowac poczal; a piecdziesiat i piec lat królowal w Jeruzalemie.
\par 2 Ten czynil zle przed oczyma Panskiemi wedlug obrzydlosci onych narodów, które wygnal Pan przed obliczem synów Izraelskich.
\par 3 Albowiem znowu pobudowal wyzyny, które byl poburzyl Ezechyjasz, ojciec jego; wystawil tez i oltarze Baalom, a nasadzil gajów, i klanial sie wszystkiemu wojsku niebieskiemu, a sluzyl mu,
\par 4 Pobudowal tez oltarze w domu Panskim, o którym powiedzial byl Pan: W Jeruzalemie bedzie imie moje na wieki.
\par 5 Nadto nabudowal oltarze wszystkiemu wojsku niebieskiemu we dwóch sieniach domu Panskiego.
\par 6 I przewodzil synów swych przez ogien w dolinie synów Hennomowych; nadto czasów przestrzegal, i bawil sie wieszczba i czarnoksiestwem, a ustawil czarnoksiezników i guslarzy, i bardzo wiele zlego czynil przed oczyma Panskiemi, drazniac go.
\par 7 Postawil takze balwana rytego, którego byl uczynil w domu Bozym, o którym byl rzekl Bóg do Dawida i do Salomona, syna jego: W domu tym i w Jeruzalemie, którem obral ze wszystkich pokolen Izraelskich, poloze imie moje na wieki;
\par 8 A nie dopuszcze sie wiecej ruszyc nodze Izraela z ziemi, któram naznaczyl ojcom waszym, by jedno strzegli i sprawowali sie wedlug wszystkiego, com i rozkazal, wedlug wszystkiego zakonu, i ustaw, i sadów wydanych przez Mojzesza.
\par 9 Ale Manases zwiódl Jude i obywateli Jeruzalemskich, tak iz sie gorzej sprawowali niz narody, które Pan wygladzil przed obliczem synów Izraelskich.
\par 10 Bo choc mówil Pan do Manasesa, i do ludu jego, przeciez oni nie sluchali.
\par 11 Przetoz Pan nawiódl na nich hetmanów wojska króla Assyryjskiego, którzy pojmawszy Manasesa w cierniu, i zwiazawszy go dwoma lancuchami, zawiedli go do Babilonu.
\par 12 Który bedac ucisniony, modlil sie Panu, Bogu swemu, i upokorzyl sie bardzo przed obliczem Boga ojców swoich,
\par 13 I prosil go; a dal mu sie uprosic, i wysluchal modlitwe jego, a przywrócil go do Jeruzalemu na królestwo jego. Tedy poznal Manases, iz sam Pan jest Bogiem.
\par 14 Potem budowal mur okolo miasta Dawidowego ku zachodniej stronie Gichonu potoku az do wejscia do bramy rybnej, i otoczyl murem Ofel, i wywiódl go bardzo wysoko; postanowil tez hetmanów po wszystkich miastach obronnych w Judzie.
\par 15 Zniósl tez bogów cudzych, i balwana z domu Panskiego, i wszystkie oltarze, które byl pobudowal na górze domu Panskiego, i w Jeruzalemie, i wyrzucil za miasto.
\par 16 Zatem naprawil oltarz Panski, i sprawowal na nim ofiary spokojne, i dziekczynienia, a przykazal Judzie, aby sluzyli Panu, Bogu Izraelskiemu.
\par 17 Wszakze jeszcze lud ofiarowal na wyzynach, lecz tylko Panu, Bogu swemu.
\par 18 Ale inne sprawy Manasesowe, i modlitwa jego do Boga jego, i slowa widzacych, którzy mawiali do niego w imie Pana, Boga Izraelskiego, sa w ksiegach spraw królów Izraelskich.
\par 19 Modlitwa zas jego, i jako jest wysluchany, i kazdy grzech jego, i przestepstwo jego, i miejsca, na których byl pobudowal wyzyny, i wystawil gaje swiecone, i balwany, przedtem niz sie byl upokorzyl, zapisane w ksiegach Chozaja.
\par 20 Potem zasnal Manases z ojcami swymi, i pochowali go w domu jego; a Amon, syn jego, królowal miasto niego.
\par 21 We dwudziestu i dwóch latach byl Amon, gdy królowac poczal, a dwa lata królowal w Jeruzalemie.
\par 22 I czynil zle przed oczyma Panskiemi, jako czynil Manases, ojciec jego; albowiem wszystkim balwanom, których byl naczynil Manases, ojciec jego, ofiarowal Amon, i sluzyl im.
\par 23 A nie upokorzyl sie przed obliczem Panskiem, jako sie upokorzyl Manases, ojciec jego; owszem ten Amon daleko wiecej grzeszyl.
\par 24 I sprzysiegli sie przeciw niemu sludzy jego, i zabili go w domu jego.
\par 25 Ale lud onej ziemi pobil wszystkich, co sie byli sprzysiegli przeciw królowi Amonowi; a postanowil lud ziemi królem Jozjasza, syna jego, miasto niego.

\chapter{34}

\par 1 Osm lat bylo Jozyjaszowi, gdy królowac poczal, a trzydziesci i jeden lat królowal w Jeruzalemie.
\par 2 Ten czynil, co bylo dobrego przed oczyma Panskiemi, chodzac drogami Dawida, ojca swego, i nie uchylal sie ani na prawo ani na lewo.
\par 3 Bo ósmego roku królowania swego, bedac jeszcze dziecieciem, poczal szukac Boga Dawida, ojca swego, a dwunastego roku poczal oczyszczac Jude i Jeruzalem od wyzyn i od gajów swieconych, i od balwanów, i od rytych obrazów.
\par 4 Albowiem przed oczyma jego pokazono oltarze Baalów, i balwany sloneczne, które byly w górze na nich, podcial: takze i gaje swiecone, i obrazy ryte, i obrazy odlane pokruszyl i potarl, a rozmiotal je po grobach tych, którzy im ofiarowali.
\par 5 Kosci tez kaplanów popalil na oltarzach ich, i oczyscil Jude i Jeruzalem;
\par 6 Takze i miasta Manasesowe i Efraimowe, i Symeonowe, az do Neftalima, i pustynie ich okoliczne.
\par 7 A tak poburzyl oltarze, i gaje swiecone, i balwany pokruszyl w sztuki, i wszystkie obrazy powycinal we wszystkiej ziemi Izraelskiej; potem sie wrócil do Jeruzalemu.
\par 8 A roku osmnastego królowania swego, gdy oczyscil ziemie i dom Panski, poslal Safana, syna Azalijaszowego, i Maasajasza, przelozonego miasta, i Joacha, syna Joachazowego, kanclerza, aby naprawiono dom Pana, Boga jego.
\par 9 Którzy przyszedlszy do Helkijasza, kaplana najwyzszego, oddali pieniadze zniesione do domu Bozego, które byli zebrali Lewitowie, strózowie progu, od synów Manasesowych i Efraimowych, i od wszystkich ostatków Izraelskich, i od wszystkiego Judy i Benjamina, a wrócili sie do Jeruzalemu.
\par 10 I oddali je w rece rzemieslników, przelozonych nad robota domu Panskiego, a oni je wydawali na robotników, którzy robili w domu Panskim, naprawiajac i utwierdzajac dom.
\par 11 A dawali je cieslom i murarzom za skupowanie kamienia ciosanego, i drzewa na spajanie i na pietra domów, które byli popsuli królowie Judzcy.
\par 12 A mezowie oni byli wiernymi w tej pracy: a nad nimi byli przelozonymi Jachat, i Abdyjasz, Lewitowie, z synów Merarego, i Zacharyjasz i Mesullam z synów Kaatowych, którzy przynaglali robocie; a kazdy z Lewitów umial grac na instrumentach muzycznych.
\par 13 Nad tymi tez, którzy nosili brzemiona, i przynaglali robotnikom przy kazdej robocie, byli z Lewitów pisarze, i przystawowie, i odzwierni.
\par 14 A gdy wynaszali pieniadze zniesione do domu Panskiego, znalazl Helkijasz kaplan ksiegi zakonu Panskiego, podanego przez Mojzesza.
\par 15 Tedy odpowiedzial Helkijasz i rzekl do Safana pisarza: Znalazlem ksiegi zakonu w domu Panskim. I oddal Helkijasz ksiege Safanowi.
\par 16 A Safan przyniósl one ksiege do króla; a przytem oznajmil to królowi, mówiac: Wszystko, cos poruczyl w reke slug twoich, wykonywaja:
\par 17 Bo zebrawszy pieniadze, które sie znalazly w domu Panskim, oddali je w rece przystawów i w rece robotników.
\par 18 Nadto oznajmil Safan, pisarz, królowi mówiac: Ksiege mi tez dal Helkijasz kaplan; i czytal ja Safan przed królem.
\par 19 A gdy slyszal król slowa zakonu, rozdarl szaty swoje.
\par 20 I rozkazal król Helkijaszowi i Achykamowi, synowi Safanowemu, i Abdonowi, synowi Michasowemu, i Safanowi, pisarzowi, i Asajaszowi, sludze królewskiemu, mówiac:
\par 21 Idzcie, radzcie sie Pana o mie, i o ostatek ludu w Izraelu i w Judzie okolo slów tych ksiag, które sa znalezione; bo wielka jest popedliwosc Panska, która jest wylana na nas, przeto, ze nie strzegli ojcowie nasi slowa Panskiego, aby czynili wedlug wszystkiego, co jest napisane w tych ksiegach.
\par 22 A tak poszedl Helkijasz, i którzy byli przy królu, do Huldy prorokini, zony Selluma, syna Tekui, syna Hasrowego, stróza szat; a ona mieszkala w Jeruzalemie na drugiej stronie miasta; i mówili z nia o tem.
\par 23 Która rzekla do nich: Tak mówi Pan, Bóg Izraelski: Powiedzcie mezowi, który was poslal do mnie:
\par 24 Tak mówi Pan: Oto, Ja przywiode zle na to miejsce, i na obywateli jego, wszystkie przeklestwa napisane w tych ksiegach, które czytano przed królem Judzkim.
\par 25 Przeto, ze mie opuscili, i kadzili bogom cudzym, aby mie draznili wszystkiemi sprawami rak swoich; dlaczego wyleje sie popedliwosc moja na to miejsce, i nie bedzie ugaszona.
\par 26 A królowi Judzkiemu, który was poslal o rade do Pana, tak powiedzcie: Tak mówi Pan, Bóg Izraelski, o slowach, któres slyszal:
\par 27 Gdyz serce twoje zmiekczone jest, i upokorzyles sie przed obliczem Bozem, slyszac slowa jego przeciwko temu miejscu, i przeciwko obywatelom jego, a upokorzywszy sie przedemna rozdarles szaty swe, i plakales przedemna, przetozem cie wysluchal, mówi Pan;
\par 28 Oto ja cie zbiore do ojców twoich, a bedziesz wlozony do grobu twego w pokoju, aby nie ogladaly oczy twoje wszystkiego zlego, które Ja przywiode na to miejsce, i na obywateli jego. I odniesli te rzecz królowi.
\par 29 Tedy poslawszy król zgromadzil wszystkich starszych z Judy i z Jeruzalemu.
\par 30 I wstapil król do domu Panskiego, i wszyscy mezowie Judzcy, i obywatele Jeruzalemscy, i kaplani i Lewitowie, i wszystek lud, od wielkiego az do malego, i czytal gdy wszyscy slyszeli, wszystkie slowa ksiag przymierza, które bylo znalezione w domu Panskim.
\par 31 Potem stojac król na miejscu swem, uczynil przymierze przed Panem, ze chce chodzic za Panem, i strzedz przykazan jego, i swiadectw jego, i ustaw jego, ze wszystkiego serca swego i ze wszystkiej duszy swojej, i pelnic slowa przymierza tego, które bylo w onych ksiegach napisane.
\par 32 I rozkazal stac w tem przymierzu wszystkim, którzy znalezieni byli w Jeruzalemie i w Benjaminie; i czynili obywatele Jeruzalemscy wedlug przymierza Boga, Boga ojców swoich.
\par 33 Tedy uprzatnal Jozyjasz wszystkie obrzydlosci ze wszystkich krain synów Izraelskich, a przywiódl do tego wszystkich, którzy sie znajdowali w Izraelu, aby sluzyli Panu, Bogu swemu. Po wszystkie dni jego nie odstapili od nasladowania Pana, Boga ojców swoich.

\chapter{35}

\par 1 Potem obchodzil Jozyjasz w Jeruzalemie swieto przejscia Panu; i zabili baranka wielkanocnego czternastego dnia, miesiaca pierwszego.
\par 2 I postanowil kaplanów na urzedach ich, a potwierdzil ich ku poslugiwaniu w domu Panskim.
\par 3 I rzekl Lewitom, którzy uczyli wszystkiego Izraela, i byli poswieceni Panu: Postawcie skrzynie swieta w domu, który zbudowal Salomon, syn Dawidowy, król Izraelski; nie bedzie wiecej brzemieniem na ramionach waszych. Terazze sluzcie Panu, Bogu waszemu, i ludowi jego Izraelskiemu.
\par 4 A nagotujcie sie wedlug domów ojców waszych, i wedlug podzialów waszych, jako je opisal Dawid, król Izraelski, i jako je opisal Salomon, syn jego;
\par 5 A stójcie w swiatnicy wedlug podzialów domów ojcowskich, braci waszych, którzy sa z ludu, i wedlug podzialu domu ojcowskiego Lewitów.
\par 6 A tak zabijcie baranka wielkanocnego, a poswieccie sie, i przygotujcie braci waszych, sprawujac sie wedlug slowa Panskiego, podanego przez Mojzesza.
\par 7 Tedy dal Jozyjasz pospólstwu baranków z trzód, i kozielków, to wszystko na ofiary swieta przejscia wedlug tego, ile sie znalazlo w liczbie, trzydziesci tysiecy, a wolów trzy tysiace; to wszystko z majetnosci królewskiej.
\par 8 Ksiazeta tez jego dobrowolnie ludowi, kaplanom, i Lewitom dawali na ofiare; Helkijasz, i Zacharyjasz, i Jehijel, ksiazeta domu Bozego, oddali kaplanom na ofiary swieta przejscia dwa tysiace i szesc set drobnego bydla, i wolów trzysta.
\par 9 Nadto Kienanijasz, i Semejasz, i Natanael, bracia jego, i Chasabijasz, i Jehijel, i Josabad, przedniejsi z Lewitów, oddali innym Lewitom na ofiary swieta przejscia piec tysiecy drobnego bydla, i wolów piec set.
\par 10 A tak wszystko zgotowawszy ku sluzbie, staneli kaplani na miejscach swych, i Lewitowie w porzadkach swych wedlug rozkazania królewskiego.
\par 11 I bili baranki wielkanocne, a kaplani kropili krwia ich, a Lewitowie odzierali ze skór.
\par 12 I podzielili z nich na calopalenie, aby to dali pospólstwu wedlug podzialów domów ojcowskich na ofiare Panu, jako napisane w ksiegach Mojzeszowych; takze tez uczynili z strony wolów.
\par 13 I piekli baranki wielkanocne ogniem wedlug zwyczaju; a inne rzeczy poswiecone warzyli w garncach, i w kotlach, i w panwiach, i rozdawali spieszno wszystkiemu pospólstwu.
\par 14 Potem tez nagotowali sobie i kaplanom. Bo kaplani, synowie Aaronowi, okolo calopalenia i tlustosci zabawieni byli az do nocy; przetoz Lewitowie gotowali sobie i kaplanom, synom Aaronowym.
\par 15 Takze i spiewacy, synowie Asafowi, stali w porzadku swym wedlug rozkazania Dawida i Asafa, i Hemmana, i Jedytuna, widzacego królewskiego; odzwierni tez stali u kazdej bramy, bo sie im nie godzilo odchodzic od poslug ich; przetoz bracia ich Lewito wie gotowali dla nich.
\par 16 A tak zgotowana jest wszystka sluzba Panska dnia onego dla obchodzenia swieta przejscia, i dla ofiarowania calopalenia na oltarzu Panskim wedlug rozkazania króla Jozyjasza.
\par 17 I obchodzili synowie Izraelscy, ile sie ich znalazlo, swieto przejscia onegoz czasu, i swieto uroczyste przasników przez siedm dni.
\par 18 A nie bylo obchodzone swieto przejscia temu podobne w Izraelu ode dni Samuela proroka; ani zaden z królów Izraelskich obchodzil takiego swieta przejscia, jakie obchodzil Jozyjasz i kaplani, i Lewitowie, i wszystek lud Judzki, i Izraelski, ile sie go znalazlo, i obywatele Jeruzalemscy.
\par 19 Osmnastego roku królowania Jozyjaszowego to swieto przejscia obchodzono.
\par 20 Po tem wszystkiem, gdy naprawil Jozyjasz dom Bozy, wyciagnal Necho, król Egipski, aby walczyl przeciw Karchemis nad rzeka Eufrates; a Jozyjasz tez wyjechal przeciwko niemu.
\par 21 Ale on poslal do niego poslów swych, mówiac: Cóz ja mam z toba, królu Judzki? Nie przeciwkoc tobie dzis ciagne, ale przeciwko domowi, który ze mna walczy; i rozkazal mi Bóg, abym sie pospieszyl. Przestan walczyc z Bogiem, który jest ze mna, aby cie nie zabil.
\par 22 Ale nie odwrócil Jozyjasz twarzy swej od niego; owszem odmienil szaty swe, aby z nim walczyl; a nie przestal na slowach Necha, które wyszly z ust Bozych. A tak przyciagnal, aby sie z nim potykal na polu Magieddo.
\par 23 I postrzelili strzelcy króla Jozyjasza. Tedy rzekl król do slug swoich: Wyprowadzcie mie z bitwy, bom jest bardzo zraniony.
\par 24 I przeniesli go sludzy jego z onego wozu, a wlozyli go na drugi wóz, który mial, i odwiezli go do Jeruzalemu. Tamze umarl, i pochowano go w grobach ojców jego; a wszystek lud Judzki i Jeruzalemski plakal nad Jozyjaszem.
\par 25 Uczynil tez i Jeremijasz narzekanie nad Jozyjaszem, które przypominaja wszyscy spiewacy, i spiewaczki w lamentach swych o Jozyjaszu az po dzis dzien, i wprowadzili to w zwyczaj w Izraelu; a zapisano te rzeczy w lamentach Jeremijaszowych.
\par 26 A inne sprawy Jozyjaszowe i dobroczynnosci jego wedlug tego, jako napisane w zakonie Panskim,
\par 27 I uczynki jego pierwsze i poslednie zapisane sa w ksiegach królów Izraelskich i Judzkich.

\chapter{36}

\par 1 Tedy wzial lud ziemi Joachaza, syna Jozyjaszowego, a postanowil go za króla na miejscu ojca jego w Jeruzalemie.
\par 2 Dwadziescia i trzy lata bylo Joachazowi, gdy poczal królowac, a trzy miesiace królowal w Jeruzalemie.
\par 3 Bo go zlozyl król Egipski w Jeruzalemie, i nalozyl wine na one ziemie sto talentów srebra, i talent zlota.
\par 4 I postanowil królem król Egipski Elijakima, brata jego, nad Juda i nad Jeruzalemem, i odmienil imie jego, a nazwal go Joakim; a Joachaza, brata jego, wziawszy Necho, zawiódl go do Egiptu.
\par 5 Dwadziescia i piec lat mial Joakim, gdy królowac poczal, a jedenascie lat królowal w Jeruzalemie; i czynil zle przed oczyma Pana Boga swego.
\par 6 Przeciw któremu wyciagnal Nabuchodonozor, król Babilonski, i zwiazal go dwoma lancuchami miedzianemi, aby go zawiódl do Babilonu.
\par 7 Naczynia tez domu Panskiego zniósl Nabuchodonozor do Babilonu, i oddal je do kosciola swego w Babilonie.
\par 8 A ostatek spraw Joakimowych, i obrzydlosci jego, które czynil, i cokolwiek sie znajdowalo przy nim, to zapisane w ksiegach królów Izraelskich i Judzkich. A królowal Joachyn, syn jego, miasto niego.
\par 9 Osm lat mial Joachyn, gdy królowac poczal, a trzy miesiace i dziesiec dni królowal w Jeruzalemie; i czynil zle przed oczyma Panskimi;
\par 10 Potem po roku poslal król Nabuchodonozor, i kazal go przywiesc do Babilonu, i z naczyniem kosztownem domu Panskiego, a postanowil królem Sedekijasza, brata jego, nad Juda i nad Jeruzalemem.
\par 11 Dwadziescia i jeden lat mial Sedekijasz, gdy królowac poczal, a jedenascie lat królowal w Jeruzalemie.
\par 12 I czynil zle przed oczyma Pana, Boga swego, a nie upokorzyl sie przed Jeremijaszem prorokiem, który mówil z ust Panskich.
\par 13 Owszem i przeciwko królowi Nabuchodonozorowi powstal, który go byl przysiega zawiazal przez Boga; a zatwardziwszy kark swój uparl sie w sercu swojem, aby sie nie nawrócil do Pana, Boga Izraelskiego.
\par 14 Wszyscy tez przedniejsi kaplani, i lud, wielce rozmnozyli nieprawosci wedlug wszystkich obrzydliwosci poganskich; i splugawili dom Panski, który byl poswiecil w Jeruzalemie.
\par 15 A Pan, Bóg ojców ich, posylal do nich poslów swoich, a posylal rano wstawajac; bo folgowal ludowi swemu, i mieszkaniu swemu.
\par 16 Ale oni szydzili z poslów Bozych, i lekce sobie powazyli slów jego, a nasmiewali sie z proroków jego, az przyszla popedliwosc Panska na lud jego, tak, iz zadnego uleczenia nie bylo.
\par 17 Bo przywiódl na nich króla Chaldejskiego, który pomordowal mlodzienców ich mieczem w domu swiatnicy ich, a nie przepuscil ani mlodziencowi ani pannie, starcowi i zgrzybialemu; wszystkich podal w rece jego.
\par 18 Nadto wszystkie naczynia domu Bozego, wielkie i male, i skarby domu Panskiego, i skarby królewskie i ksiazat jego, wszystko przeniósl do Babilonu.
\par 19 I spalili dom Bozy, a zburzyli mury Jeruzalemskie, i wszystkie palace jego popalili ogniem, i wszystkie jego naczynia kosztowne popsuli.
\par 20 A tych, którzy uszli miecza, przeniósl do Babilonu; i byli niewolnikami jego i synów jego az do królowania króla Perskiego;
\par 21 Aby sie wypelnilo slowo Panskie powiedziane przez usta Jeremijaszowe, azby odprawila ziemia sabaty swoje; bo po wszystkie dni spustoszenia swego odpoczywala, az sie wypelnilo siedmdziesiat lat.
\par 22 Potem roku pierwszego Cyrusa, króla Perskiego, aby sie wypelnilo slowo Panskie powiedziane przez usta Jeremijaszowe, wzbudzil Pan ducha Cyrusa, króla Perskiego, ze kazal obwolac i rozpisac po wszystkiem królestwie swojem, mówiac:
\par 23 Tak mówi Cyrus, król Perski: Wszystkie królestwa ziemi dal mi Pan, Bóg niebieski; i ten mi rozkazal, abym mu zbudowal dom w Jeruzalemie, które jest w Judztwie. Kto tedy jest miedzy wami ze wszystkiego ludu jego, który budowac chce, z tym niech bedzie Pan, Bóg jego, a ten niechaj idzie.


\end{document}