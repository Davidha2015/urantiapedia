\begin{document}

\title{Ezdrasza}


\chapter{1}

\par 1 Roku pierwszego Cyrusa, króla Perskiego, aby sie wypelnilo slowo Panskie powiedziane przez usta Jeremijaszowe, wzbudzil Pan ducha Cyrusa, króla Perskiego, ze kazal obwolac i rozpisac po wszystkiem królestwie swojem, mówiac:
\par 2 Tak mówi Cyrus, król Perski: Wszystkie królestwa ziemi dal mi Pan, Bóg niebieski; i ten mi rozkazal, abym mu zbudowal dom w Jeruzalemie, które jest w Judztwie.
\par 3 Kto tedy jest miedzy wami ze wszystkiego ludu jego, z tym niech bedzie Bóg jego, a ten niech idzie do Jeruzalemu, które jest w Judztwie, a niech buduje dom Pana, Boga Izraelskiego; onci jest Bóg, który jest w Jeruzalemie.
\par 4 A ktoby zostal w któremkolwiek miejscu, gdzie jest przychodniem, niech go podpomoga mezowie miejsca onego srebrem i zlotem, i majetnoscia, i bydlem, oprócz dobrowolnej ofiary na dom Bozy, który jest w Jeruzalemie.
\par 5 Tedy powstali przedniejsi z domów ojcowskich, z Judy i z Benjamina, i kaplani, i Lewitowie; wszelki, którego ducha pobudzil Bóg, aby szli, a budowali dom Panski, który jest w Jeruzalemie;
\par 6 Których wszyscy mieszkajacy okolo nich wspomagali naczyniem srebrnem i zlotem, majetnoscia, i bydlem, i rzeczami kosztownemi, oprócz wszystkiego, co dobrowolnie ofiarowano.
\par 7 Król tez Cyrus wyniósl naczynia domu Panskiego (które byl zabral Nabuchodonozor z Jeruzalemu, a oddal je byl do domu Boga swego).
\par 8 A wyniósl je Cyrus, król Perski, przez rece Mitrydatesa podskarbiego, który je pod liczba oddal Sesbasarowi, ksiazeciu Judzkiemu.
\par 9 A tak jest liczba ich: Miednic zlotych trzydziesci, miednic srebrnych tysiac, nozów dwadziescia i dziwiec.
\par 10 Kubków zlotych trzydziesci, kubków srebrnych podlejszych cztery sta i dziesiec, a naczynia innego tysiacami.
\par 11 Wszystkiego naczynia zlotego i srebrnego piec tysiecy i cztery sta; wszystko to wyniósl Sesbasar, gdy sie prowadzil lud z niewoli, z Babilonu do Jeruzalemu.

\chapter{2}

\par 1 A cic sa ludzie onej krainy, którzy wyszli z pojmania i z niewoli, w która ich byl zaprowadzil Nabuchodonozor, król Babilonski, do Babilonu, a wrócili sie do Jeruzalemu i do Judy, kazdy do miasta swego.
\par 2 Którzy przyszli z Zorobabelem, z Jesua, Nehemijaszem, Sarajaszem, Rehelijaszem, Mardocheuszem, Bilsanem, Misparem Bigwajem, Rechumem, i Baana. A poczet ludu Izraelskiego ten byl:
\par 3 Synów Farosowych dwa tysiace sto siedmdziesiat i dwa;
\par 4 Synów Sefatyjaszowych trzy sta siedmdziesiat i dwa;
\par 5 Synów Arachowych siedm set siedmdziesiat i piec;
\par 6 Synów Pachat Moabowych, synów Jesui Joabowych dwa tysiace osm set i dwanascie;
\par 7 Synów Elamowych tysiac dwiescie piecdziesiat i cztery;
\par 8 Synów Zatuowych dziewiec set i czterdziesci i piec;
\par 9 Synów Zachajowych siedm set i szescdziesiat;
\par 10 Synów Bani szesc set czterdziesci i dwa;
\par 11 Synów Bebajowych szesc set dwadziescia i trzy.
\par 12 Synów Azgadowych tysiac dwiescie dwadziescia i dwa.
\par 13 Synów Adonikamowych szesc set szescdziesiat i szesc;
\par 14 Synów Bigwajowych dwa tysiace piecdziesiat i szesc;
\par 15 Synów Adynowych cztery sta piecdziesiat i cztery.
\par 16 Synów Aterowych, co poszli z Ezechyjasza, dziewiecdziesiat i osm;
\par 17 Synów Besajowych trzy sta dwadziescia i trzy.
\par 18 Synów Jory sto i dwanascie;
\par 19 Synów Hasumowych dwiescie dwadziescia i trzy.
\par 20 Synów Gibbarowych dziewiecdziesiat i piec;
\par 21 Synów z Betlehemu sto dwadziescia i trzy;
\par 22 Mezów z Netofatu piecdziesiat i szesc;
\par 23 Mezów z Anatotu sto dwadziescia i osm;
\par 24 Synów z Azmawetu czterdziesci i dwa;
\par 25 Synów z Karyjatyjarymu, z Kafiry i z Beerotu siedm set i czterdziesci i trzy;
\par 26 Synów z Ramy i z Gabaa szesc set dwadziescia i jeden;
\par 27 Mezów z Machmas sto dwadziescia i dwa;
\par 28 Mezów z Betela i z Haj dwiescie dwadziescia i trzy;
\par 29 Synów z Nebo piecdziesiat i dwa;
\par 30 Synów Magbisowych sto piecdziesiat i szesc;
\par 31 Synów Elama drugiego tysiac dwiescie piecdziesiat i cztery;
\par 32 Synów Harymowych trzy sta i dwadziescia;
\par 33 Synów Lodowych, Hadydowych, i Onowych siedm set dwadziescia i piec;
\par 34 Synów Jerechowych trzy sta czterdziesci i piec;
\par 35 Synów Senaa trzy tysiace i szesc set i trzydziesci.
\par 36 Kaplanów: Synów Jedajaszowych z domu Jesui, dziewiec set siedmdziesiat i trzy;
\par 37 Synów Immerowych tysiac piecdziesiat i dwa;
\par 38 Synów Pashurowych tysiac dwiescie czterdziesci i siedm;
\par 39 Synów Harymowych tysiac i siedmnascie.
\par 40 Lewitów: synów Jesui i Kadmiela, synów Hodawyjaszowych siedmdziesiat i cztery.
\par 41 Spiewaków: synów Asafowych sto dwadziescia i osm.
\par 42 Synów odzwiernych: synów Sallumowych, synów Aterowych, synów Talmonowych, synów Akkubowych, synów Hatytowych, synów Sobajowych, wszystkich sto trzydziesci i dziewiec.
\par 43 Z Netynejczyków: synów Sycha, synów Chasufa, synów Tabbaota,
\par 44 Synów Kierosa, synów Syaa, synów Fadona,
\par 45 Synów Lebana, synów Hagaba,
\par 46 Synów Akkuba, synów Hagaba, synów Salmaja, synów Hanana,
\par 47 Synów Gieddela, synów Gachera, synów Reajasza,
\par 48 Synów Rezyna, synów Nekoda, synów Gazama,
\par 49 Synów Uzy, synów Fasejacha, synów Besaja,
\par 50 Synów Asena, synów Mehunima, synów Nefusyma;
\par 51 Synów Bakbuka, synów Chakufa, synów Charchura,
\par 52 Synów Basluta, synów Mechyda, synów Charsa,
\par 53 Synów Barkosa, synów Sysera, synów Tamacha,
\par 54 Synów Nezyjacha, synów Chatyfa,
\par 55 Synów slug Salomonowych, synów Sotaja, synów Sofereta, synów Peruda,
\par 56 Synów Jahala, synów Darkona, synów Giddela,
\par 57 Synów Sefatyjasza, synów Chatyla, synów Pocheret Hasebaim, synów Ami;
\par 58 Wszystkich Netynejczyków, i synów slug Salomonowych trzy sta dziewiecdziesiat i dwa.
\par 59 Ci tez zasie wyszli z Telmelachu: Telcharsa, Cherub, Addam i Immer; ale nie mogli okazac domu ojców swoich, i nasienia swego, jezli z Izraela byli.
\par 60 Synów Delajaszowych, synów Tobijaszowych, synów Nekodowych, szesc set piecdziesiat i dwa.
\par 61 A synów kaplanskich: synowie Habajowi, synowie Kozowi, synowie Barsylajego, który byl pojal zone z córek Barsylaja Galaadczyka; i nazwany byl od imienia ich.
\par 62 Ci szukali opisania rodu swego, ale nie znalezli; przetoz zrzuceni sa z kaplanstwa.
\par 63 I zakazal im Tyrsata, aby nie jadali z rzeczy najswietszych, azby powstal kaplan z Urym i z Tummim.
\par 64 Wszystkiego zgromadzenia bylo w jednym poczcie cztredziesci tysiecy dwa tysiace trzy sta i szescdziesiat;
\par 65 Oprócz slug ich, i sluzebnic ich, których bylo siedm tysiecy trzy sta trzydziesci i siedm, a miedzy nimi bylo spiewaków i spiewaczek dwiescie.
\par 66 Koni ich siedm set trzydziesci i szesc; mulów ich dwiescie czterdzisci i piec.
\par 67 Wielbladów ich cztery sta trzydziesci i piec; oslów szesc tysiecy siedm set i dwadziescia.
\par 68 A niektórzy z ksiazat domów ojcowskich przyszli do domu Panskiego, który byl w Jeruzalemie, ofiarowawszy sie dobrowolnie, aby budowali dom Bozy na miejscu jego.
\par 69 Wedlug przemozenia swego dali naklad na budowanie: zlota lótów szescdziesiat tysiecy i jeden, a srebra grzywien piec tysiecy, i szat kaplanskich sto.
\par 70 A tak osadzili sie kaplani i Lewitowie, i niektórzy z ludu, i spiewacy, i odzwierni, i Netynejczycy w miastach swych, i wszystek Izrael w miastach swych.

\chapter{3}

\par 1 A gdy nastal miesiac siódmy, a synowie Izraelscy byli w miastach, zgromadzil sie lud jednomyslnie do Jeruzalemu.
\par 2 Tedy wstawszy Jesua, syn Jozedeka, i bracia jego kaplani, i Zorobabel, syn Salatyjela, i bracia jego, zbudowali oltarz Boga Izraelskiego, aby na nim sprawowali calopalenia, jako napisane w zakonie Mojzesza, meza Bozego.
\par 3 A gdy postawili on oltarz na fundamencie swym, choc sie bali narodów postronnych, jednak ofiarowali na nim calopalenia Panu, calopalenia rano i w wieczór.
\par 4 Obchodzili tez swieto uroczyste kuczek, jako napisane, sprawujac calopalenia na kazdy dzien wedlug liczby i wedlug zwyczaju kazda rzecz dnia swego;
\par 5 Potem calopalenie ustawiczne, i na nowiu miesiaca, i na kazde uroczyste swieto Panu poswiecone, i od kazdego dobrowolnie ofiarujacego dobrowolna ofiare Panu.
\par 6 Ode dnia pierwszego miesiaca siódmego poczeli sprawowac calopalenia Panu, choc jeszcze kosciól Panski nie byl zalozony.
\par 7 I oddali pieniadze kamiennikom, i rzemieslnikom, takze strawe i napój, i oliwe Sydonczykom, i Tyryjczykom, aby przywozili drzewo cedrowe z Libanu do morza Joppy, jako im pozwolil Cyrus, król Perski.
\par 8 Potem roku wtórego po ich nawróceniu do domu Bozego w Jeruzalemie, miesiaca wtórego, zaczeli Zorobabel, syn Salatyjela, i Jesua, syn Jozedeka, i inni bracia ich kaplani, i Lewitowie, i wszyscy, którzy byli przyszli z onej niewoli do Jeruzalemu, a postanowili Lewitów od dwudziestu lat i wyzej, aby byli dozorcami nad robota domu Panskiego.
\par 9 I stanal Jesua, synowie jego, i bracia jego; Kadmiel tez i synowie jego, synowie Judy spolu, aby przynaglali tym, którzy robili okolo domu Bozego; synowie Chenadadowi, synowie ich, i bracia ich Lewitowie.
\par 10 A gdy zakladali budownicy grunty kosciola Panskiego, postawili kaplanów ubranych z trabami, i Lewitów, synrw Asafowych z cymbalami, aby chwalili Pana wedlug postanowienia Dawida, króla Izraelskiego.
\par 11 I spiewali jedni po drugich chwalac a wyslawiajac Pana, ze dobry, ze na wieki milosierdzie jego nad Izraelem. Wszystek takze lud krzyczal krzykiem wielkim, chwalac Pana, przeto, iz byl zalozony dom Panski.
\par 12 A wiele starców, z kaplanów, i z Lewitów, i z przedniejszych domów ojcowskich, którzy widzieli dom pierwszy, gdy zakladano ten dom przed oczyma ich, plakali glosem wielkim, a zasie wiele ich krzyczalo, z radoscia wynoszac glosy;
\par 13 Tak, iz lud nie mógl rozeznac glosu krzyku wesolego od glosu placzacego ludu; albowiem lud on krzyczal glosem wielkim, tak ze glos bylo daleka slyszec.

\chapter{4}

\par 1 A gdy uslyszeli nieprzyjaciele Judy i Benjamina, iz lud, który przyszedl z pojmania, budowali kosciól Panu, Bogu Izraelskiemu;
\par 2 Tedy przyszli do Zorobabela i do przedniejszych z domów ojcowskich, mówiac im: Bedziemy budowac z wami, a jako i wy bedziemy szukac Boga waszego, gdyzesmy mu ofiary czynili ode dni Asarhaddona, króla Assyryjskiego, który nas tu przywiódl.
\par 3 Ale im rzekl Zorobabel, i Jesua, i inni przedniejsi domów ojcowskich z Izraela: Nie wam, ale nam nalezy budowac dom Bogu naszemu; przetoz my sami budowac bedziemy Panu, Bogu Izraelskiemu, jako nam rozkazal Cyrus, król Perski.
\par 4 A tak lud onej krainy watlil rece ludu Judzkiego, i przeszkadzal im, aby nie budowali.
\par 5 Nadto przenajmowali przeciwko nim radców, aby rozrywali rade ich po wszystkie dni Cyrusa, króla Perskiego, az do królowania Daryjusza, króla Perskiego.
\par 6 Bo gdy królowal Aswerus, tedy na poczatku królestwa jego, napisali skarge przeciwko obywatelom Judzkim i Jeruzalemskim,
\par 7 Tak jako za dni Artakserksesa pisal Bislan, Mitrydates, Tabeel, i inni towarzysze jego do Artakserksesa króla Perskiego; a pismo listu tego napisane bylo po syryjsku, i wylozone tez bylo po syryjsku.
\par 8 Rechum kanclerz, i Symsaj pisarz napisali list jeden przeciwko Jeruzalemowi do Artakserksesa króla w ten sposób:
\par 9 To uczynili natenczas Rechum kanclerz, i Symsaj pisarz, i inni towarzysze ich, Dynajczycy, i Afarsadchajczycy, Tarpelajczycy, Afarsajczycy, Arkiewajczycy, Babilonczycy, Susanchajczycy, Dehawejczycy i Elmajczycy;
\par 10 I inne narody, które byl przyprowadzil Asnapar wielki i slawny, a osadzil nimi miasta Samaryjskie; i inni, którzy byli za rzeka, i Cheenetczycy.
\par 11 A tenci jest przepis listu, który poslali do Artakserksesa króla:
\par 12 Sludzy twoi, ludzie mieszkajacy za rzeka, i Cheenetczycy. Niech bedzie wiadomo królowi, ze Zydowie, którzy sie wrócili od ciebie, przyszedlszy do nas do Jeruzalemu, miasto odporne i zle buduja, i mury zakladaja, a z gruntu je wywodza.
\par 13 Przetoz niech bedzie wiadomo królowi, Ze bedzieli to miasto pobudowane, i mury jego z gruntu wywiedzione, tedy cla, czynszów, i dani dorocznej nie beda dawac, a tak dochodom królewskim ujma sie stanie.
\par 14 Teraz tedy, poniewaz uzywamy dobrodziejstw palacu twego, na szkode królewska nie godzi sie nam patrzyc; dla tegosmy poslali, oznajmujac to królowi,
\par 15 Abys dal szukac w ksiegach historyj ojców swoich, a znajdziesz w ksiegach historyj, i dowiesz sie, iz to miasto jest miasto odporne i szkodliwe królom i krainom, a iz sie w niem wszczynaly bunty od dawnych dni, przez co to miasto bylo zburzone.
\par 16 Nadto wiadomo czynimy królowi, ze jezli sie to miasto pobuduje, i mury jego z gruntu wywiedzione beda, tedy juz ta czesc za rzeka nie bedzie twoja.
\par 17 Tedy dal odpowiedz król Rechumowi kanclerzowi, i Symsajemu pisarzowi, i innym towarzyszom ich, którzy mieszkali w Samaryi, takze i innym za rzeka w Selam i w Cheet:
\par 18 List, któryscie poslali do nas, jawnie przedemna czytano.
\par 19 Przetoz rozkazalem, aby szukano; i znaleziono, ze to miasto z dawna powstawalo przeciwko królom, a bunty i rozruchy bywaly w niem;
\par 20 Nadto królowie mozni bywali w Jeruzalemie, którzy panowali nad wszystkiem, co jest za rzeka, którym cla, czynsze, i dani doroczne dawano.
\par 21 Przetoz wydajcie wyrok, aby zabroniono onym mezom, aby to miasto nie bylo budowane, pókiby odemnie inszy rozkaz nie wyszedl.
\par 22 Patrzciez, abyscie sie w tem nie omylili. Przeczzeby urosc mialo co zlego na szkode królom?
\par 23 A tak, gdy przepis listu Artakserksesa byl czytany przed Rechumem, i Symsajem pisarzem, i przed towarzyszami ich, jechali predko do Jeruzalemu do Zydów, a zabronili im gwaltem i moca budowac.
\par 24 A tak ustala robota okolo domu Bozego, który byl w Jeruzalemie, i zaniechano jej az do wtórego roku królestwa Daryjusza, króla Perskiego.

\chapter{5}

\par 1 Tego czasu prorokowal Haggieusz prorok, i Zacharyjasz, syn Iddy, prorokujac zydom, którzy byli w Judzie i w Jeruzalemie, w imie Boga Izraelsiego, mówiac do nich.
\par 2 Tedy powstawszy Zorobabel, syn Salatyjela, i Jesua, syn Jozedeka, poczeli budowac dom Bozy, który jest w Jeruzalemie; a byli z nimi prorocy Bozy, pomagajac im.
\par 3 Pod tenze czas przyszedl do nich Tattenaj, starosta za rzeka, i Setarbozenaj, i towarzysze ich, i tak mówili do nich: Któz wam rozkazal ten dom budowac, i mury jego z gruntu wywodzic?
\par 4 Na cosmy im odpowiedzieli, i mianowalismy tych mezów, którzy okolo tego budowania robili.
\par 5 Lecz oko Boga ich bylo nad starszymi Zydowskimi, i nie mogli im przeszkadzac, póki ta rzecz do Daryjusza nie przyszla, a pókiby przez list nie dano znac o tem.
\par 6 Tenci jest przepis listu, który poslal do króla Daryjusza Tattenaj, starosta za rzeka, i Setarbozenaj, i towarzysze jego Afarsechajczycy, którzy byli za rzeka, do króla Daryjusza.
\par 7 List mu poslali, w którem to bylo napisane: Daryjuszowi królowi pokój na wszystkiem!
\par 8 Niechaj bedzie wiadomo królowi, zesmy przyszli do Judzkiej krainy, do domu Boga wielkiego, który buduja z kamienia wielkiego, a drzewo klada w sciany; ta robota sporo idzie, i szczesci sie w rekach ich.
\par 9 Pytalismy tedy starszych onych mewiac do nich: Któz wam rozkazal ten dom budowac, i te muru z gruntu wywodzic?
\par 10 Nawet i o imiona ich pytalismy sie, abysmyc oznajmili, i opisali imiona mezów, którzy sa przedniejsi miedzy nimi.
\par 11 Ale nam tak odpowiedzieli, mówiac: Mysmy sludzy Boga nieba i ziemi, a budujemy dom, który byl zbudowany przedtem przed wieloma laty, który byl wielki król Izraelski zbudowal i wystawil.
\par 12 Lecz gdy wzruszyli ku gniewu ojcowie nasi Boga niebieskiego, podal ich w rece Nabuchodonozora, króla Babilonskiego, Chaldejczyka, który ten dom zburzyl, a lud jego zawiódl w niewole do Babilonu.
\par 13 Wszakze roku pierwszego Cyrusa, króla Babilonskiego, król Cyrus wydal dekret, aby ten dom Bozy budowano.
\par 14 Nadto i naczynia domu Bozego zlote i srebrne, które byl zabral Nabuchodonozor z kosciola, który byl w Jeruzalemie, i wniósl je do kosciola Babilonskiego, te wyniósl król Cyrus z kosciola Babilonskiego, i dane sa nijakiemu Sesbasarowi, którego byl ksiazeciem uczynil.
\par 15 I rzekl mu: Te naczynia wziawszy, idz, a zlóz je w kosciele, który jest w Jeruzalemie, a dom Bozy niech bedzie budowany na miejscu swojem.
\par 16 Tedy ten Sesbasar przyszedlszy zalozyl grunty domu Bozego, który jest w Jeruzalemie, i od onego czasu az dotad buduja go, a nie jest dokonczony.
\par 17 A tak, królu! jezlic sie zda byc rzecza dobra, niechby poszukano w domu skarbów królewskich, który jest w Babilonie, jezliz tak jest, ze król Cyrus rozkazal, aby budowano ten dom Bozy, który jest w Jeruzalemie, a wola królewska o tem niech bedzie do nas poslana.

\chapter{6}

\par 1 Tedy król Daryjusz rozkazal, aby szukano w biblijotece miedzy skarbami tamze zlozonemi w Babilonie.
\par 2 I znaleziono w Achmecie na zamku, który jest w ziemi Medskiej, ksiege jedne, a taka byla zapisana w niej pamiec:
\par 3 Roku pierwszego Cyrusa króla, król Cyrus wydal wyrok o domu Bozym, który byl w Jeruzalemie, aby byl dom zbudowany dla miejsca, gdzieby ofiary sprawowano; aby tez grunty jego byly wybudowane, wysokosc jego na szescdziesiat lokci, a szerokosc jego na szescdziesiat lokci.
\par 4 Trzy rzedy z kamienia wielkiego, a jeden rzad z drzewa nowego, a naklad z domu królewskiego dawany bedzie.
\par 5 Nadto i naczynia domu Bozego, zlote i srebrne, które byl zabral Nabuchodonozor z kosciola, który jest w Jeruzalemie, a przeniósl do Babilonu, niech wróca, aby sie dostaly do kosciola, który jest w Jeruzalemie, na miejsce swe, i zlozone byly w domu Bozym.
\par 6 Przetoz teraz Tattenaju, starosto za rzeka! z Setarbozenaimem, i z towarzyszami twymi, i Afarsechajczycy, którzyscie za rzeka, ustapcie stamtad.
\par 7 Dopusccie, zeby byl budowany ten dom Bozy od ksiazecia Zydowskiego, i od starszych Zydowskich, aby ten dom Bozy zbudowali na miejscu swem.
\par 8 Odemnie tez wyszedl wyrok o tem, cobyscie mieli czynic z starszymi tych Zydów przy budowaniu tego domu Bozego; to jest, aby z majetnosci królewskich, z dochodów, które sa za rzeka, dawano bez omieszkania naklad mezom tym, aby nie przestawali.
\par 9 A ile potrzeba i wolów, i baranów, i baranków na calopalenia Bogu niebieskiemu, zboza, soli, wina, i oliwy, na rozkazanie kaplanów, którzy sa w Jeruzalemie, aby im dawano na kazdy dzien, a to bez omieszkania;
\par 10 Aby mieli skad ofiarowac wonne kadzenia Bogu niebieskiemu, i aby sie modlili za zdrowie królewskie, i synów jego.
\par 11 Nadto uczyniony jest odemnie dekret: Ktobykolwiek wzruszyl to przykazanie, aby wyjeto drzewo z domu jego, i aby je podniesiono, a na niem go powieszono, a dom jego aby byl gnojowiskiem dla tego.
\par 12 A Bóg, który tam uczynil mieszkanie imieniowi swemu, niech zniszczy kazdego króla i naród, któryby sciagnal reke swa na odmiane i skaze tego domu Bozego, który jest w Jeruzalemie. Ja Daryjusz uczynilem ten dekret; bez omieszkania niech bedzie wyk onany.
\par 13 Tedy Tattenaj, starosta za rzeka, i Setarbozenaj, i towarzysze ich wedlug tego, jako rozkazal król Daryjusz, tak uczynili bez omieszkania.
\par 14 A starsi Zydowscy budowali, i szczescilo sie im wedlug proroctwa Haggieusza proroka, i Zachariasza, syna Iddy; i budowali i dokonali za rozkazaniem Boga Izraelskiego, i za rozkazaniem Cyrusa, i Daryjusza, i Artakserksesa, królów Perskich.
\par 15 I dokonczony jest on dom trzeciego dnia miesiaca Adar, a ten byl rok szósty panowania Daryjusza króla.
\par 16 Tedy synowie Izraelscy, kaplani i Lewitowie, i inni z ludu, którzy przyszli z wiezienia, poswiecali on dom Bozy z radoscia.
\par 17 A ofiarowali przy poswiecaniu onego domu Bozego, cielców sto, baranów dwiescie, baranków cztery sta, i kozlów z kóz na ofiare za grzech za wszystkiego Izraela, dwanascie, wedlug liczby pokolenia Izraelskiego.
\par 18 I postawili kaplanów w rzedach swych, i Lewitów w przemianach swoich, nad sluzba Boza w Jeruzalemie, jako napisane w ksiegach Mojzeszowych.
\par 19 Obchodzili tez ci, co przyszli z niewoli, swieto przejscia czternastego dnia miesiaca pierwszego.
\par 20 Bo sie oczyscili kaplani i Lewitowie jednostajnie, wszyscy byli oczyszczeni; przetoz ofiarowali baranka swieta przejscia za wszystkich, którzy przyszli z niewoli, i za braci swoich kaplanów, i za siebie samych.
\par 21 A tak jedli synowie Izraelscy, którzy sie wrócili z niewoli, i kazdy, który sie odlaczyl od sprosnosci narodów onej ziemi do nich, aby szukal Pana, Boga Izraelskiego.
\par 22 I obchodzili swieto uroczyste przasników przez siedm dni z radoscia, przeto, ze ich Pan byl rozweselil, a obrócil serce króla Assyryjskiego do nich, aby zmocnil rece ich w robocie okolo domu Bozego, Boga Izraelskiego.

\chapter{7}

\par 1 A po tych sprawach za królowania Artakserksesa, króla Perskiego, Ezdrasz, syn Sarajasza, syna Azaryjaszowego, syna Helkijaszowego,
\par 2 Syna Sallumowego, syna Sadokowego, syna Achitobowego,
\par 3 Syna Amaryjaszowego, syna Azaryjaszowego, syna Merajotowego,
\par 4 Syna Zerahyjaszowego, syna Uzego, syna Bukkiego,
\par 5 Syna Abisujego, syna Fineesowego, syna Eleazarowego, syna Aarona kaplana najwyzszego:
\par 6 Ten Ezdrasz wyszedl z Babilonu, a byl czlowiekiem bieglym w zakonie Mojzeszowym, który byl dal Pan, Bóg Izraelski; a pozwolil mu byl król wedlug reki Pana Boga jego nad nim, na wszystke prosbe jego.
\par 7 (Wyszli tez niektórzy z synów Izraelskich, i z kaplanów, i z Lewitów, i spiewaków, i z odzwiernych, i z Netynejczyków, do Jeruzalemu roku siódmego Artakserksesa króla)
\par 8 I przyszedl do Jeruzalemu miesiaca piatego; tenci byl rok siódmy króla Daryjusza.
\par 9 Albowiem w pierwszy dzien miesiaca pierwszego wyszedl z Babilonu, a dnia pierwszego miesiaca piatego przyszedl do Jeruzalemu wedlug laskawego wspomozenia Boga swego.
\par 10 Bo Ezdrasz przygotowal byl serce swe, aby szukal zakonu Panskiego, i aby czynil, i nauczal w Izraelu ustaw i sadów.
\par 11 A tenci jest przepis listu, który dal król Artakserkses Ezdraszowi, kaplanowi nauczonemu w zakonie, i bieglemu w tych rzeczach, które przykazal Pan, i w ustawach jego w Izraelu.
\par 12 Artakserkses, król nad królmi, Ezdraszowi, kaplanowi nauczonemu w zakonie Boga niebieskiego, mezowi doskonalemu, i Cheenetczykom.
\par 13 Wydany odemnie jest dekret, iz ktobykolwiek dobrowolnie w królestwie mojem z ludu Izraelskiego, i z kaplanów jego i z Lewitów chcial isc z toba do Jeruzalemu, aby szedl.
\par 14 Poniewaz od króla i od siedmiu radnych panów jego jestes poslany, abys dojrzal Judy i Jeruzalemu wedlug zakonu Boga twego, który jest w rekach twoich.
\par 15 A izbys odniósl srebro i zloto, które król i radni panowie jego dobrowolnie ofiarowali Bogu Izraelskiemu, którego przybytek jest w Jeruzalemie.
\par 16 Do tego wszystko srebro i zloto, któregobys nabyl we wszystkiej krainie Babilonskiej, z dobrowolnemi darami od ludu i od kaplanów, którzyby co dobrowolnie ofiarowali na dom Boga swego, który jest w Jeruzalemie;
\par 17 Abys predko nakupil za to srebro cielców, baranów, baranków z sniednemi ofiarami ich, i z mokremi ofiarami ich, a ofiarowal je na oltarzu domu Boga waszego, który jest w Jeruzalemie;
\par 18 A cokolwiek sie tobie i braciom twoim bedzie dobrego zdalo, z ostatkiem srebra i zlota uczynic, wedlug woli Boga waszego uczyncie.
\par 19 Naczynia tez, którec sa oddane do uslugi domu Boga twego, oddaj przed Bogiem w Jeruzalemie.
\par 20 Takze i inne rzeczy, nalezace do domu Boga twego, i coby potrzeba dac, dasz z domu skarbów królewskich.
\par 21 A ja, ja król Artakserkses, rozkazalem wszystkim podskarbim, którzyscie za rzeka, aby wszystko, czegobykolwiek zadal od was Ezdrasz kaplan, nauczyciel zakonu Boga niebieskiego, predko sie stalo,
\par 22 Az do sta talentów srebra i az do sta korcy pszenicy, i az do sta wiader wina, i az do sta barel oliwy, a soli bez miary.
\par 23 Cobykolwiek bylo z rozkazania Boga niebieskiego, niech bedzie predko dodane do domu Boga niebieskiego; bo przecz ma byc wzruszony gniew jego przeciwko królestwu, królowi i synom jego?
\par 24 Takze oznajmujemy wam, aby na zadnego z kaplanów, i z Lewitów, i z spiewaków, i z odzwiernych, Netynejczyków, i innych slug domu Boga tego, cla, czynszów, i dani dorocznej zaden starosta nie wkladal.
\par 25 A ty Ezdraszu! wedlug madrosci Boga twego, która jest w tobie, postanowisz sedziów, i w prawie bieglych, aby sadzili wszystek lud, który jest za rzeka, ze wszystkich, którzy sa powiadomi zakonu Boga twego; a ktoby nie umial, uczyc go bedziecie.
\par 26 A ktobykolwiek nie czynil dosyc zakonowi Boga twego, i prawu królewskiemu, aby predki dekret byl o nim wyadany albo na smierc, albo na wygnanie jego, albo na skaranie na majetnosci, albo na wiezienie. I rzekl Ezdrasz:
\par 27 Blogoslawiony Pan, Bóg ojców naszych, który to dal w serce królewskie, aby uwielbil dom Panski, który jest w Jeruzalemie;
\par 28 A ku mnie sklonil milosierdzie przed królem i rada jego, i wszystkimi ksiazetami królewskimi moznymi. Przetoz ja, bedac umocniony reka Pana, Boga mojego, która jest nademna, zgromadzilem z Izraela przedniejszych, którzy wyszli zemna.

\chapter{8}

\par 1 A cic sa przedniejsi z domów swych ojcowskich, i ród tych, którzy wyszli zemna z Babilonu za królowania króla Artakserksesa:
\par 2 Z synów Fineaszowych Gierson; z synów Itamarowych Danijel; z synów Dawidowych Hattus;
\par 3 Z synów Sechanijaszowych, który byl z synów Faresowych, Zacharyjasz, a z nim poczet mezów sto i piecdziesiat;
\par 4 Z synów Pachatmoabowych Elijeoenaj, syn Zerachyjaszowy, a z nim dwiescie mezów;
\par 5 Z synów Sechanijaszowych syn Jahazyjelowy, a z nim trzy sta mezów;
\par 6 A z synów Adynowych Ebed, syn Jonatana, a z nim piecdziesiat mezów;
\par 7 A z synów Elamowych Isajasz, syn Atalijasza, a z nim siedmdziesiat mezów;
\par 8 A z synów Sefatyjaszowych Zabadyjasz, syn Michaelowy, a z nim osmdziesiat mezów;
\par 9 Z synów Joabowych Obadyjasz, syn Jechyjelowy, a z nim dwiescie i osmnascie mezów;
\par 10 A z synów Selomitowych, syn Josyfijaszowy, a z nim sto i szescdziesiat mezów;
\par 11 A z synów Bebajowych Zacharyjasz, syn Bebajowy, a z nim dwadziescia i osm mezów;
\par 12 A z synów Azgadowych Johanan, syn Hakatanowy, a z nim sto i dziesiec mezów;
\par 13 A z synów Adonikamowych najostateczniejszych, których te sa imiona: Elifelet, Jehijel, i Semejasz, a z nimi szesdziesiat mezów,
\par 14 A z synów Bigwajowych Utaj i Zabud, a z nimi siedmdziesiat mezów;
\par 15 A tak zgromadzilem ich do rzeki, która wpada do Achawy, i lezelismy tam obozem przez trzy dni: potem przegladalem lud i kaplanów, a z synów Lewiego nie znalazlem tam zadnego.
\par 16 Przetoz poslalem Elijezera, Aryjela, Semejasza, i Elnatana, i Jaryba, i Elnatana i Natana, i Zacharyjasza, i Mesullama, przedniejszych, i Jojaryba, i Elnatana, mezów uczonych;
\par 17 I rozkazalem im do Iddona, przelozonego nad miejscem Kasyfii, i wlozylem w usta ich slowa, które mieli mówic do Iddona, Achywa i Netynejczyków na miejscu Kasyfii, aby nam przywiedli slug do domu Boga naszego.
\par 18 I przywiedli nam wedlug reki Boga naszego laskawej nad nami, meza nauczonego z synów Maheli, syna Lewiego, syna Izraelowego, i Serebijasza, i synów jego, i braci jego osmnascie;
\par 19 I Hasabijasza, a z nim Jesajasza z synów Merarego, braci jego, i synów ich dwadziescia;
\par 20 Nadto z Netynejczyków, których byl postanowil Dawid i przedniejsi ku posludze Lewitów, Netynejczyków dwiescie i dwadziescia; ci wszyscy z imienia mianowani byli.
\par 21 Tedym tam zapowiedzial post u rzeki Achawy, abysmy sie dreczyli przed Bogiem naszym, a szukali od niego drogi prostej sobie, i dziatkom naszym, i wszystkiej majetnosci naszej.
\par 22 Bom sie wstydzil prosic u króla o jaki poczet jezdnych, aby nam byli na pomocy przeciwko nieprzyjaciolom w drodze; bosmy byli powiedzieli królowi, mówiac: Reka Boga naszego jest nad wszystkimi, którzy go szukaja uprzejmie, ale moc jego i popedliw osc jego przeciwko wszystkim, którzy go opuszczaja.
\par 23 A gdysmy poscili, i prosilismy o to Boga naszego, wysluchal nas.
\par 24 Tedym odlaczyl z przedniejszych kaplanów dwanascie: Serebijasza, Hasabijasza, a z nimi braci ich dziesiec;
\par 25 I odwazylem im srebro, i zloto, i naczynia na ofiare podnoszenia do domu Boga naszego, które ofiarowali król i rada jego, i ksiazeta jego, i wszystek lud Izraelski, ile sie go znalazlo.
\par 26 Odwazylem, mówie do rak ich srebra talentów szescset i piecdziesiat, a naczynia srebrnego sto talentów, przytem zlota sto talentów.
\par 27 Czasz tez zlotych dwadziescia, wazacych po tysiac lótów, a dwa naczynia z mosiadzu wybornego, tak piekne jako zloto.
\par 28 Potemem rzekl do nich: Wyscie poswieceni Panu, i naczynia takze poswiecone, a to srebro i to zloto dobrowolnie ofiarowane jest Panu, Bogu ojców waszych.
\par 29 Pilnujciez a strzezcie, az to odwazycie przed kaplanami przedniejszymi, i Lewitami, i ksiazetami z domów ojcowskich w Izraelu w Jeruzalemie, w gmachach domu Panskiego.
\par 30 A tak wzieli kaplani i Lewitowie wage onego srebra, i zlota, i naczynia, aby je odniesli do Jeruzalemu, do domu Boga naszego.
\par 31 Zatem ruszylismy sie od rzeki Achawy dwunastego dnia, miesiaca pierwszego, abysmy szli do Jeruzalemu; a reka Boga naszego byla nad nami, i wyrwala nas z reki nieprzyjaciela i czyhajacego na nas w drodze.
\par 32 I przyszlismy do Jeruzalemu, i zamieszkalismy tam przez trzy dni.
\par 33 A dnia czwartego odwazono srebro, i zloto, i naczynie ono w domu Boga naszego do reki Meremota, syna Uryjasza kaplana, z którym byl Eleazar, syn Fineesowy; z nimi tez byli Josabad, syn Jesuego, i Noadyjasz, syn Binnujego, Lewitowie;
\par 34 Pod liczba i waga wszystko, i zapisano wage tego wszystkiego onegoz czasu.
\par 35 Wróciwszy sie tedy z niewoli ci, którzy byli w pojmaniu, ofiarowali Bogu Izraelskiemu cielców dwanascie za wszystkiego Izraela, baranów dziewiecdziesiat i szesc, baranków siedmdziesiat i siedm, kozlów za grzech dwanascie, to wszystko na calopalenie Panu.
\par 36 I oddali wyroki królewskie starostom królewskim, i ksiazetom za rzeka, a ci byli pomoca ludowi i domowi Bozemu.

\chapter{9}

\par 1 A gdy sie to odprawilo, przystapili do mnie ksiazeta, mówiac: Nie odlaczyl sie lud Izraelski, i kaplani, i Lewitowie od narodów tych ziem; ale czynia wedlug obrzydliwosci Chananejczyków, Hetejczyków, Ferezejczyków, Jebuzejczyków, Ammonitczyków, Moabczyków, Egipczyków, i Amorejczyków.
\par 2 Albowiem pojeli córki ich sobie i synom swym, a pomieszalo sie nasienie swiete z narodami tych ziem, a reka ksiazat i zwierzchnosci pierwsza byla w tem przestepstwie.
\par 3 Co gdym uslyszal, rozdarlem suknie moje i plaszcz mój, a rwalem wlosy na glowie mojej, i na brodzie mojej, i siedzialem, zdumiawszy sie.
\par 4 I zgromadzili sie do mnie wszyscy, którzy drza przed slowem Boga Izraelskiego dla przestepstwa tych, którzy przyszli z niewoli, a jam siedzial, zdumiawszy sie, az do ofiary wieczornej.
\par 5 Ale pod czas ofiary wieczornej wstalem z utrapienia mego, majac rozdarta suknie moje i plaszcz mój, a pokleknawszy na kolana swe, wyciagnalem rece swe ku Panu, Bogu memu,
\par 6 I rzeklem: Boze mój! wstydci mie, i sromam sie podniesc, Boze mój! oblicza mego do ciebie; albowiem nieprawosci nasze rozmnozyly sie nad glowa, a grzechy nasze urosly az ku niebu.
\par 7 Ode dni ojców naszych bylismy w wielkim grzechu az do dnia tego, a przez nieprawosci nasze wydanismy, królowie nasi i kaplani nasi, w rece królów ziemskich pod miecz, w niewole, i na lup, i na zawstydzenie twarzy naszej, jako sie to dzis dzieje.
\par 8 Ale teraz, jakoby w predkiem okamgnieniu, stala sie nam laska od Pana, Boga naszego, ze nam zostawil ostatki, i dal nam mieszkanie na miejscu swietem swojem, aby oswiecil oczy nasze Bóg nasz, a dal nam troche wytchnienia z niewoli naszej.
\par 9 Bo chociasmy byli niewolnikami, przeciez w niewoli naszej nie opuscil nas Bóg nasz, ale sklonil ku nam milosierdzie przed królmi Perskimi, dawszy nam wytchnienie, abysmy wystawili dom Boga naszego, i naprawili spustoszenia jego; nawet dal nam ogrodzenie w Judztwie i w Jeruzalemie.
\par 10 Przetoz cóz teraz rzeczemy, o Boze nasz! po tem? poniewazesmy opuscili rozkazania twoje,
\par 11 Któres ty przykazal przez slug twoich proroków, mówiac: Ziemia, do której wnijdziecie, abyscie ja posiedli, jest ziemia nieczysta przez nieczystote ludu tych ziem, dla obrzydlosci ich, któremi ja napelnili od konca do konca nieczystoscia swoja.
\par 12 A przetoz nie dawajcie córek waszych synom ich, ani bierzcie synom waszym córek ich, i nie szukajcie pokoju ich, i dobrego ich az na wieki, abyscie byli umocnieni, a pozywali dóbr tej ziemi, i podali ja w dziedzictwo synom waszym az na wieki.
\par 13 A po tem wszystkiem, co przyszlo na nas dla spraw naszych zlych i dla grzechu naszego wielkiego, poniewazes ty, Boze nasz! zawsciagnal karania, abysmy nie byli potlumieni dla nieprawosci naszej, ales nam dal wybawienie takowe;
\par 14 Izali sie obrócimy ku zgwalceniu przykazan twoich, powinowacac sie z tymi narodami obrzydlymi? izalibys sie surowie nie gniewal na nas, abys nas wyniszczyl, aby nikt nie zostal i nie uszedl?
\par 15 O Panie, Boze Izraelski! sprawiedliwys ty; bosmy pozostale ostatki, jako sie to dzis pokazuje. Otosmy my przed obliczem twojem w przewinieniu naszeem, choc sie nie godzi stawiac przed oblicze twoje dla tego.

\chapter{10}

\par 1 A gdy sie modlil Ezdrasz, i wyznawal grzechy z placzem, lezac przed domem Bozym, zebralo sie do niego z Izraela zgromadzenie bardzo wielkie mezów i niewiast i dziatek; a plakal lud wielkim placzem.
\par 2 Tedy odpowiadajac Sechanijasz, syn Jechyjelowy z synów Elamowych, rzekl do Ezdrasza: Mysmyc zgrzeszyli przeciwko Panu, Bogu naszemu, zesmy pojeli zony obce z narodu tej ziemi; ale wzdy ma jeszcze nadzieje Izrael przytem.
\par 3 Tylko teraz uczynmy przymierze z Bogiem naszym, ze porzucimy wszystkie zony i narodzone z nich, wedlug rady Panskiej, i tych, którzy drza przed przykazaniem Boga naszego, a niech to bedzie podlug zakonu.
\par 4 Wstanze, bo ta rzecz tobie nalezy, a my bedziemy z toba; zmocnij sie, a uczyn tak.
\par 5 Tedy wstal Ezdrasz, i poprzysiagl ksiazat kaplanskich, i Lewitów, i wszystkiego Izraela, aby uczynili wedlug tego slowa. I przysiegli.
\par 6 A tak wstawszy Ezdrasz od domu Bozego szedl do komory Jochanana, syna Elijasybowego, a wszedlszy tam, nie jadl chleba, i wody nie pil; albowiem byl zalosny dla przestepstwa tych, co sie wrócili z niewoli.
\par 7 Zatem kazali obwolac w Judztwie i w Jeruzalemie miedzy wszystkimi, którzy przyszli z niewoli, aby sie zgromadzili do Jeruzalemu.
\par 8 A ktobykolwiek nie przyszedl we trzech dniach wedlug uradzenia ksiazat i starszych, aby przepadla wszystka majetnosc jego, a sam aby byl wylaczony od zgromadzenia tych, co przyszli z niewoli.
\par 9 Przetoz zgromadzili sie wszyscy mezowie z Judy i z Benjamina do Jeruzalemu we trzech dniach, dwudziestego dnia miesiaca dziewiatego, i siedzial wszystek lud na placu przed domem Bozym, drzac dla onej rzeczy i dla deszczu.
\par 10 Tedy powstawszy Ezdrasz kaplan rzekl do nich: Wyscie zgrzeszyli, izescie pojeli zony obce, przydawajac do grzechów Izraelskich.
\par 11 Przetoz uczyncie teraz wyznanie przed Panem, Bogiem ojców waszych, a wykonajcie wole jego, i odlaczcie sie od narodów tej ziemi, i od zon obcych.
\par 12 I odpowiedzialo wszystko ono zgromadzenie, i rzeklo glosem wielkim: Jakos nam powiedzial, tak uczynimy.
\par 13 Ale wielki jest lud, i czas dzdzysty, i nie mozemy stac na dworze; dotego ta sprawa nie jest dnia jednego, ani dwóch; bo nas wiele, którzysmy sie tego przestepstwa dopuscili.
\par 14 Prosimy tedy, niechze beda postanowieni ksiazeta nasi nad wszystkiem zgromadzeniem; a ktobykolwiek byl w miastach naszych, co pojal zony obce, niechaj przyjdzie na czas zamierzony, a z nimi starsi z kazdego miasta, sedziowie ich, abysmy tak odwrócili gniew popedliwosci Boga naszego od nas dla tej sprawy.
\par 15 A tak Jonatan, syn Asahijelowy, i Jachsyjasz, syn Tekujego, byli na to wysadzeni; ale Mesullam i Sebetaj, Lewitowie, pomagali im.
\par 16 Tedy uczynili tak ci, co przyszli z niewoli. I odlaczeni sa Ezdrasz kaplan, i mezowie przedniejsi z domów ojcowskich wedlug domów ojców swoich; a ci wszyscy z imienia mianowani byli, i zasiedli dnia pierwszego, miesiaca dziesiatego, aby sie o tem wywiadywali.
\par 17 A odprawowali to przy wszystkich mezach, którzy byli pojeli zony obce, az do pierwszego dnia, miesiaca pierwszego.
\par 18 I znalezli sie z synów kaplanskich, którzy byli pojeli zony obce: z synów Jesui, syna Jozedekowego, i z braci jego Maasejasz i Elijezer, i Jaryb, i Giedalijasz.
\par 19 I dali rece swe, ze mieli porzucic zony swe; a ci, którzy zgrzeszyli, ofiarowali kazdy barana z stada za wystepek swój.
\par 20 A z synów Immerowych: Hanani i Zabadyjasz;
\par 21 A z synów Harymowych: Maasyjasz i Elijasz, i Semejasz, i Jechyjel, i Uzyjasz;
\par 22 A z synów Passurowych: Elijenaj, Maasejasz, Izmael, Natanael, Jozabad, i Elasa.
\par 23 A z Lewitów: Jozabad, i Symei, i Kielajasz, (ten jest Kielita) Petachyjasz, Judas, i Elijezer.
\par 24 A z spiewaków: Elijasyb; a z odzwiernych: Sallum i Telem, i Ury.
\par 25 A z Izraela, z synów Farosowych: Ramijasz, i Jezyjasz, i Malchyjasz, i Miamin, i Elazar, i Malchyjasz, i Benajasz;
\par 26 A z synów Elamowych: Matanijasz, Zacharyjasz, i Jechyjel, i Abdy, i Jerymot, i Elijasz;
\par 27 A z synów Zattuowych: Elijenaj, Elijasyb, Matanijasz, i Jerymot, i Zabad, i Asysa;
\par 28 A z synów Bebajowych: Johanan, Hananijasz, Zabbaj, Atlaj;
\par 29 A z synów Bani: Mesullam, Malluch, i Adajasz, Jasub, i Seal, Jeramot;
\par 30 A z synów Pachatmoabowych Adna, i Chelal, Benajasz, Maasejasz, Matanijasz, Besaleel, i Binnui, i Manase;
\par 31 A z synów Harymowych: Elijezer, Isyjasz, Malchyjasz, Semaajasz, Symeon,
\par 32 Benjamin, Maluch, Samaryjasz;
\par 33 Z synów Hasumowych: Matenajasz, Matata, Zabad, Elifelet, Jeremijasz, Manase, Symhy;
\par 34 Z synów Bani: Maadaj, Amram, i Uel.
\par 35 Banajasz, Bedyjasz, i Cheluhu,
\par 36 Wanijasz, Meremot, Elijasyb,
\par 37 Mattanijasz, Matenajasz, i Jahasaw.
\par 38 I Bani, i Binnui, Symhy,
\par 39 I Selemijasz, i Natan, i Adajasz,
\par 40 Machnadbaj, Sasaj, Saraj,
\par 41 Asarel, I Selemijasz, Semaryjasz,
\par 42 Sallum, Amaryjasz, i Józef.
\par 43 Z synów Nebowych: Jehijel, Matytyjasz, Zabad, Zebina, Jaddaj, i Joel, i Benajasz.
\par 44 Ci wszyscy pojeli byli zony obce; a byly miedzy niemi niewiasty, które im narodzily synów.


\end{document}