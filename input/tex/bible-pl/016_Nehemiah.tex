\begin{document}

\title{Nehemiasza}


\chapter{1}

\par 1 Slowa Nehemijasza, syna Hachalijaszowego. I stalo sie miesiaca Chyslew, roku dwudziestego (Aswerusa króla) gdym byl na zamku w Susan,
\par 2 Ze przyszedl Chanani, jeden z braci moich, a z nim niektórzy mezowie z Judy, którychem sie pytal o Zydów, którzy pozostali i wyszli z wiezienia, i o Jeruzalem.
\par 3 I odpowiedzieli mi: Te ostatki, które pozostaly z wiezienia tam w onej krainie, sa w wielkiem utrapieniu, i w zelzywosci: nadto mur Jeruzalemski rozwalony jest, i bramy jego spalone sa ogniem.
\par 4 A gdym uslyszal te slowa, siadlszy plakalem i narzekalem przez kilka dni, poszczac i modlac sie przed obliczem Boga niebieskiego.
\par 5 I rzeklem; Prosze Panie, Boze niebieski, mocny, wielki i straszny! który strzezesz umowy, i milosierdzia z tymi, którzy cie miluja, i strzega przykazania twego.
\par 6 Niech bedzie prosze ucho twoje naklonione, a oczy twoje otworzone, abys uslyszal modlitwe slugi twego, która sie ja dzis modle przed toba we dnie i w nocy za synami Izraelskimi, slugami twymi, i wyznaje grzechy synów Izraelskich, któremismy zgrzes zyli przeciw tobie; i ja i dom ojca mego zgrzeszylismy.
\par 7 Srodzesmy wystapili przeciwko tobie, i nie przestrzegalismy przykazan, i ustaw, i sadów, któres przykazal Mojzeszowi, sludze twemu.
\par 8 Wspomnij prosze na slowo, któres przykazal Mojzeszowi, sludze twemu, mówiac: Jezli wy wystapicie, tedy Ja was rozprosze miedzy narody;
\par 9 Ale jezli sie nawrócicie do mnie, a strzedz bedziecie przykazan moich, i czynic je, chocbyscie byli wygnani na kraj swiata, tedy i stamtad was zgromadze, i przyprowadze was na miejsce, którem obral, aby tam przebywalo imie moje.
\par 10 Wszak oni sa sludzy twoi, i lud twój, którys odkupil moca twoja wielka, i reka twa silna.
\par 11 Prosze o Panie! niech teraz bedzie ucho twoje naklonione ku modlitwie slugi twego i ku modlitwie slug twoich, którzy maja wole bac sie imienia twego; a zdarz dzis, prosze, sludze twemu, i spraw mu milosierdzie przed tym mezem. A jam byl podczaszym królewskim.

\chapter{2}

\par 1 I stalo sie miesiaca Nisan roku dwudziestego Artakserksesa króla, gdy bylo wino przed nim, ze wziawszy wino, podalem je królowi, a nie bywalem przedtem tak smutny przed nim.
\par 2 I rzekl mi król: Czemuz twarz twoja tak smutna, gdyz nie chorujesz? Nic to innego, jedno smutek serca. I zleklem sie nader bardzo.
\par 3 I rzeklem do króla: Niech król na wiki zyje. Jakoz nie ma byc smutna twarz moja, gdyz miasto, dom grobów ojców moich, zburzono, a bramy jego ogniem popalono?
\par 4 Znowu rzekl do mnie król: Czegoz ty zadasz? A jam sie modlil Bogu niebieskiemu.
\par 5 I rzeklem do króla: Zdali sie to za rzecz dobra królowi, i jezli ma laske sluga twój przed obliczem twojem, prosze, abys mie poslal do ziemi Judzkiej, do miasta grobów ojców moich, abym je pobudowal.
\par 6 Nadto rzekl mi król (a królowa siedziala podle niego): Dlugoz bedziesz na tej drodze, i kiedy sie wrócisz? I podobalo sie to królowi, i poslal mie, gdym mu zamierzyl pewny czas.
\par 7 Zatemem rzekl do króla: Zdali sie to za rzecz dobra królowi, niech mi dadza listy do starostów za rzeka, aby mie przeprowadzili, azbym przyszedl do ziemi Judzkiej;
\par 8 I list do Asafa, dozorcy lasów królewskich, aby mi dal drzewa na przykrycie bram palacu przy domu Bozym, i na mur miejski, i na dom, do którego wnijde. I dal mi król listy wedlug reki Boga mego laskawej nademna.
\par 9 A gdym przyszedl do starostów za rzeka, oddalem im listy królewskie. Poslal tez byl ze mna król rotmistrzów i jezdnych:
\par 10 Co gdy uslyszal Sanballat Horonitczyk, i Tobijasz, sluga Ammonitczyk, bardzo ich to mierzialo, ze przyszedl czlowiek, który sie staral o dobro synów Izraelskich.
\par 11 Zatem przyszedlszy do Jeruzalemu, mieszkalem tam przez trzy dni.
\par 12 A wstawszy w nocy, ja i mezów trocha ze mna, nie oznajmilem nikomu, co Bóg mój podal do serca mego, abym uczynil w Jeruzalemie; bydlecia tez nie mialem z soba, oprócz bydlecia, na któremem jechal.
\par 13 I wyjechalem brama nad dolina w nocy, ku zródlu smoczemu, i ku bramie gnojowej, i ogladalem mury Jeruzalemskie, które byly rozwalone, i bramy jego, które byly popalone ogniem.
\par 14 Potem jechalem ku bramie zródla, i ku sadzawce królewskiej, gdzie nie bylo miejsca bydleciu, na któremem jechal, aby przejsc moglo.
\par 15 Przetoz jechalem nad potokiem w nocy, a ogladalem mury; skad wracajac sie, wyjechalem brama nad dolina, i takiem powrócil.
\par 16 Ale ksiazeta nie wiedzieli, gdziem jezdzil, i com czynil; bom Zydom, ani kaplanom, ani ksiazetom, ani urzednikom, ani zadnemu rzemieslnikowi tego az dotad nie oznajmil.
\par 17 Przetozem rzekl do nich: Wy widzicie, w jakiemesmy ucisnieniu, a jako Jeruzalem spustoszone, i bramy jego popalone sa ogniem. Pójdzciez, a budujmy mury Jeruzalemskie, abysmy nie byli wiecej na hanbe.
\par 18 A gdym im oznajmil, ze reka Boga mego byla laskawa nademna, takze i slowa królewskie, które do mnie mówil, rzekli: Wstanmyz a budujmy. I zmocnili rece swe ku dobremu.
\par 19 Co slyszac Sanballat Horonitczyk, i Tobijasz, sluga Ammonitczyk i Giesem Arabczyk, szydzili z nas, i lekce nas sobie powazyli mówiac: Cóz to za rzecz, która czynicie? albo sie przeciw królowi buntujecie?
\par 20 I odpowiedzialem im, a rzeklem do nich: Bóg niebieski, ten nam poszczesci, a my sludzy jego, wstanmy a budujmy; ale wy nie macie dzialu, ani prawa, ani pamiatki w Jeruzalemie.

\chapter{3}

\par 1 Potem powstal Elijasyb, kaplan najwyzszy, i bracia jego kaplani, i budowali brame owcza. I zbudowali ja, i przyprawili wrota do niej; az do wiezy Mea zbudowali ja, i az do wiezy Chananeel.
\par 2 A podle niego budowali mezowie z Jerycha, a podle nich budowal Zachur, syn Immrego.
\par 3 A brame rybna budowali synowie Senaa, którzy ja tez przykryli i przyprawili wrota do niej, i zamki jej, i zawory jej.
\par 4 A podle nich poprawial Meremot, syn Uryjasza, syna Kosowego; a podle nich poprawial Mesullam, syn Barachyjaszowy, syna Mesezabelowego; a podle nich poprawial Sadok, syn Baany.
\par 5 Podle nich zasie poprawiali Tekuitczycy; ale ci, co byli zacniejsi z nich, nie podlozyli szyi swej pod robote pana swego.
\par 6 A brame stara poprawiali Jojada, syn Faseachowy, i Mesullam, syn Besodyjaszowy; ci ja przykryli, i przyprawili wrota do niej, i zamki jej, i zawory jej.
\par 7 A podle nich poprawial Melatyjasz Gabaonitczyk, Jadon Meronitczyk, mezowie z Gabaon i z Masfa, az do stolicy ksiazecej, z tej strony rzeki.
\par 8 Podle nich poprawial Husyjel, syn Charchajaszowy, z zlotnikami; a podle niego poprawial Chananijasz, syn aptekarski; a Jeruzalemu zaniechali az do muru szerokiego.
\par 9 A podle nich poprawial Rafajasz, syn Churowy, przelozony nad polowa powiatu Jeruzalemskiego.
\par 10 A podle nich poprawial Jedajasz, syn Harumafowy, i przeciw swemu domowi; a podle niego poprawial Hattus, syn Hasbonijaszowy.
\par 11 Czesci zas drugiej poprawial Malchyjasz, syn Harymowy, i Hasub, syn Pachatmoabowy, takze i wieze Tannurym.
\par 12 A podle niego poprawial Sallum, syn Hallochesowy, przelozony nad polowa powiatu Jeruzalemskiego, sam i córki jego.
\par 13 Bramy nad dolina poprawial Chanun, i obywatele Zanoe; cic ja budowali, i przyprawili wrota do niej, zamki jej, i zawory jej; i na tysiac lokci muru az do bramy gnojowej.
\par 14 Bramy zas gnojowej poprawial Melchyjasz, syn Rechaby, przelozony nad powiatem Betcherem; tenci ja zbudowal, i przyprawil wrota do niej, zamki jej, i zawory jej,
\par 15 Dotego bramy zródla poprawial Sallon, syn Cholhozowy, przelozony nad powiatem Masfa; a ten ja zbudowal, i przykryl ja, i przyprawil wrota do niej, i zamki jej, i zawory jej, i mur nad stawem Selach ku ogrodowi królewskiemu az do schodu, po którym schodza z miasta Dawidowego.
\par 16 Zatem poprawial Nehemijasz, syn Hasbuka, przelozony nad polowa powiatu Betsur, az przeciwko grobom Dawidowym i az do stawu urobionego, i az do domu mocarzów.
\par 17 Za nim poprawiali Lewitowie Rehum, syn Bani; podle niego poprawial Hasabijasz, przelozony nad polowa powiatu Ceile z powiatem swoim.
\par 18 Za nim zasie poprawiali bracia ich, Bawaj, syn Chenadadowy, przelozony nad polowa powiatu Ceile.
\par 19 A podle niego poprawial Eser, syn Jesui, przelozony nad Masfa, czesci drugiej przeciw miejscu, kedy chodza do zbrojowni nazwanej Mikzoa.
\par 20 Po nim wzruszony gorliwoscia poprawial Baruch, syn Zabbajowy, czesci drugiej od Mikzoa az do drzwi domu Elijasyba, najwyzszego kaplana.
\par 21 Za nim poprawial Meremot, syn Uryjasza, syn Kosowego, czesci drugiej, ode drzwi domu Elijasybowego az do konca domu jego.
\par 22 A za nim poprawiali kaplani, którzy mieszkali w równinie.
\par 23 Za nimi poprawial Benjamin i Hasub, przeciw domom swoim; za nimi poprawial Azaryjasz, syn Maasejasza, syna Ananijaszowego, podle domu swego.
\par 24 Za nim poprawial Bennui, syn Chenadadowy, czesci drugiej od domu Azaryjaszowego az do Mikzoa, i az do rogu.
\par 25 Palal, syn Uzajego przeciw Mikzoa, i wiezy wysokiej, wywiedzionej z domu królewskiego, która byla w sieni wiezienia; po nim poprawial Fadajasz, syn Farosowy.
\par 26 A Netynejczycy, co mieszkali w Ofel, poprawiali az na przeciwko bramie wodnej na wschód slonca, i wiezy wysokiej.
\par 27 Za nimi poprawiali Tekuitczykowie druga czesc przeciw wiezy wielkiej i wysokiej az do muru Ofel.
\par 28 Od bramy konskiej poprawiali kaplani, kazdy przeciw domowi swemu.
\par 29 Za nimi poprawial Sadok, syn Immerowy, przeciw domowi swemu, a za nim poprawial Semejasz, syn Sechenijaszowy, stróz barmy wschodniej.
\par 30 Za nim poprawial Chananijasz, syn Selemijaszowy, i Chanun, syn Salafowy szósty, czesci drugiej; za nim poprawial Mesullam, syn Berechyjaszowy, przeciw gmachowi swemu.
\par 31 Za nim poprawial Malchijasz, syn zlotniczy, az do domu Netynejczyków, i kupców, przeciw bramie sadowej, i az do sali naroznej.
\par 32 A miedzy sala narozna az do bramy owczej poprawiali zlotnicy i kupcy.

\chapter{4}

\par 1 A gdy uslyszal Sanballat, iz budujemy mury, rozgniewal sie, a rozgniewawszy sie bardzo, szydzil z Zydów.
\par 2 I mówil przed bracmi swymi i przed rycerstwem Samaryjskiem, i rzekl: Cóz wzdy ci Zydowie niedolezni czynia? Takze ich zaniechamy? I bedaz ofiarowac? Izali tego za dzien dokoncza? Izali wskrzesza kamienie z gromad gruzu, które spalono?
\par 3 Ale Tobijasz Ammonitczyk bedac przy nim, rzekl: Niech buduja; jednak kiedy przyjdzie liszka, przebije mur ich kamienny.
\par 4 Wysluchajze, o Boze nasz! bosmy wzgardzeni, a obróc pohanbienie ich na glowe ich, a daj ich na lup w ziemi niewoli.
\par 5 Nie pokrywajze nieprawosci ich, a grzech ich od twarzy twej niech nie bedzie zgladzony; bo cie do gniewu pobudzili dla tych, co buduja.
\par 6 Leczesmy my budowali ten mur, i spojony jest wszystek mur az do polowy swej, a lud mial serce do roboty.
\par 7 A gdy uslyszeli Sanballat i Tobijasz, i Arabczycy, i Ammonitowie, i Azodczycy, ze przybywalo wzdluz murów Jeruzalemskich, a iz sie poczeli rozerwania murów zawierac, bardzo sie rozgniewali.
\par 8 Przetoz zbuntowali sie wszyscy wespól, aby szli walczyc przeciw Jeruzalemowi, i uczynic wstret robocie.
\par 9 Mysmy sie jednak modlili Bogu naszemu, i postawilismy straz przeciwko nim we dnie i w nocy, bojac sie ich.
\par 10 Bo rzekli Zydowie: Zwatlala sila noszacego, a gruzu jeszcze wiele; a my nie bedziemy mogli budowac muru.
\par 11 Nadto rzekli nieprzyjaciele nasi: Niech nie wzwiedza ani obacza, az przyjdziemy miedzy nich, i pomordujemy ich, a tak zastanowimy te robote.
\par 12 A gdy przyszli Zydowie, którzy mieszkali okolo nich, i powiedzieli nam na dziesiec kroc: Pilnujcie ze wszystkich miejsc, z którychby przyjsc mogli do nas;
\par 13 Tedym postawil na dolnych miejscach za murem i na miejscach wysokich, postawilem mówie lud wedlug domów z mieczami ich, z wlóczniami, i z lukami ich.
\par 14 A gdym to ogladal, wstawszy rzeklem do starszych, i do przelozonych, i do innego ludu: Nie bójcie sie ich; na Pana wielkiego i straszliwego pamietajcie, a walczcie za braci waszych, za synów waszych, i za córki wasze, za zony wasze, i za domy was ze.
\par 15 A gdy uslyszeli nieprzyjaciele nasi, iz nam to oznajmiono, tedy rozproszyl Bóg rade ich, a mysmy sie wszyscy wrócili do murów, kazdy do roboty swojej.
\par 16 A wszakze od onego czasu polowa slug moich robila, a polowa ich trzymala wlócznie, i tarcze, i luki, i pancerze, a przedniejsi stali za wszystkim domem Judzkim.
\par 17 Ci tez, którzy budowali mury, i którzy nosili brzemiona, i co nakladali, jedna reka swoja robili, a druga trzymali bron.
\par 18 A z onych, którzy budowali, mial kazdy miecz swój przypasany do biódr swych, i tak budowali; a ten co w trabe trabil, byl przy mnie.
\par 19 Bom rzekl do starszych i przelozonych, i do innego ludu: Robota wielka i szeroka; a mysmy sie rozstrzelali po murze daleko jeden od drugiego.
\par 20 A przetoz na któremkolwiek byscie miejscu uslyszeli glos traby, tam sie zbierajcie do nas; Bóg nasz bedzie walczyl za nas.
\par 21 Pilnowalismy tedy roboty, a polowa ich trzymala wlócznie, od wejscia zorzy, az gwiazdy wschodzily.
\par 22 Na tenze czas rzeklem do ludu: Kazdy z sluga swym niech nocuje w Jeruzalemie, aby nam byli w nocy dla strazy, a we dnie dla roboty.
\par 23 Przetoz i ja, i bracia moi, i sludzy moi, i straz, która jest ze mna, nie zewleczemy szat naszych, a kazdy niech ma bron swa i wode.

\chapter{5}

\par 1 I wszczelo sie wielkie wolanie ludu i zon ich przeciw Zydom, braciom swym.
\par 2 Albowiem niektórzy mówili: Wiele nas, co synów naszych i córki nasze zastawiamy, abysmy nabywszy zboza, jesc i zyc mogli.
\par 3 Inni zas mówili: Role nasze, i winnice nasze, i domy nasze zastawiac musimy, abysmy nabyli zboza w tym glodzie.
\par 4 Inni zas mówili: Napozyczalismy pieniedzy, zebysmy dali podatek królowi, zastawiwszy role nasze i winnice nasze.
\par 5 Choc oto cialo nasze jest jako cialo braci naszych, a synowie nasi sa jako synowie ich: wszakze oto my musimy dawac synów naszych i córki nasze w niewole, i niektóre z córek naszych sa juz w niewole podane, a nie mamy przemozenia w rekach naszych, abysmy je wykupili, gdyz role nasze i winnice nasze inni trzymaja.
\par 6 Przetoz rozgniewalem sie bardzo, gdym uslyszal wolanie ich, i slowa takowe.
\par 7 I umyslilem w sercu swem, abym sfukal przedniejszych i przelozonych, mówiac do nich: Wy jestescie, którzy obciazacie kazdy brata swego; i zebralem przeciwko nim zgromadzenie wielkie;
\par 8 I rzeklem do nich: Mysmy odkupili braci naszych, Zydów, którzy byli zaprzedani poganom, podlug przemozenia naszego; a jeszczez wy sprzedawac bedziecie braci waszych, a tak jakoby nam ich sprzedawac bedziecie? I umilkneli, i nie znalezli, coby odpo wiedziec.
\par 9 Nadtom rzekl: Nie dobra to rzecz, która wy czynicie; azaz nie w bojazni Boga naszego chodzic macie raczej niz w hanbie poganów, nieprzyjaciól naszych?
\par 10 I jac tez z bracmi moimi, i z slugami moimi, pozyczylismy im pieniedzy, i zboza; odpuscmyz im prosze ten ciezar.
\par 11 Wrócciez im dzis prosze role ich, winnice ich, oliwnice ich, i domy ich, i setna czesc pieniedzy i zboza, wina, i oliwy, która wy od nich wyciagacie.
\par 12 Tedy odpowiedzieli: Wrócimy, a nie bedziemy sie od nich tego upominac; tak uczynimy, jakos ty powiedzial. Wezwalem tez i kaplanów, a poprzysiaglem ich, aby takze uczynili.
\par 13 Potemem wytrzasnal zanadrza moje, i rzeklem: Niech tak wytrzasnie Bóg kazdego meza z domu jego i z pracy jego, któryby nie uczynil dosyc temu slowu; a niech tak bedzie wytrzasniony i wyprózniony. I rzeklo wszystko zgromadzenie: Amen. I chwalili Pana, a lud uczynil jako bylo rzeczono.
\par 14 Owszem ode dnia, którego mi przykazal król, abym byl ksiazeciem ich w ziemi Judzkiej, od roku dwudziestego az do roku trzydziestego i wtórego Artakserksesa króla, przez dwanascie lat, ja i bracia moi obrokusmy ksiazecego nie jedli.
\par 15 Choc ksiazeta pierwsi, którzy byli przedemna, obciazali lud, biorac od nich chleb i wino, mimo srebra syklów czterdziesci; takze i sludzy ich uzywali okrucienstwa nad ludem; alem ja tak nie czynil dla bojazni Bozej.
\par 16 Owszem i okolo poprawy tego muru pracowalem, a przeciesmy roli nie kupili; wiec i wszyscy sludzy moi byli tam zgromadzeni dla roboty.
\par 17 Nadto z Zydów i przelozonych sto i piecdziesiat mezów, i którzy do nas przychodzili z pogan okolicznych, jadali u stolu mego.
\par 18 Przetoz gotowano na kazdy dzien wolu jednego, owiec szesc wybornych, i ptaki gotowano dla mnie, a kazdego dziesiatego dnia rozmaitego wina hojnie dawano; wszakzem sie obroku ksiazecego nie upominal; albowiem ciezka byla niewola na ten lud.
\par 19 Wspomnijze na mie, Boze mój! ku dobremu wedlug wszystkiego, com czynil ludowi twemu.

\chapter{6}

\par 1 A gdy uslyszal Sanballat, i Tobijasz, i Giesem Arabczyk, i inni nieprzyjaciele nasi, zem zbudowal mur, a ze w nim nie zostawalo zadnej rozwaliny, chociazem jeszcze wtenczas nie byl przyprawil wrót do bram:
\par 2 Tedy poslal Sanballat, i Giesem do mnie mówiac: Przyjdz, a zejdzmy sie spolem we wsiach, które sa na polu Ono. Ale oni myslili uczynic mi co zlego.
\par 3 Przetoz poslalem do nich poslów, wskazujac: Zaczalem robote wielka, przetoz nie moge zjechac; bo przeczzeby miala ustac ta robota, gdybym jej zaniechawszy jechal do was?
\par 4 Tedy poslali do mnie w tejze sprawie po cztery kroc. A jam im odpowiedzial temiz slowy.
\par 5 Potem Sanballat poslal do mnie w tejze sprawie piaty raz sluge swego i list otwarty, w rece jego,
\par 6 W którym to bylo napisane: Jest posluch miedzy narodami, jako Gasmus powiada, ze ty i Zydowie myslicie sie z mocy wybic, a ze ty dlatego budujesz mur, abys byl nad nimi królem ich, jako sie to pokazuje.
\par 7 Do tego, zes i proroków postanowil, aby powiadali o tobie w Jeruzalemie, mówiac: On jest królem w Judzie. Teraz tedy dojdzie to króla; przetoz przyjdz, a naradzimy sie spólecznie.
\par 8 Alem poslal do niego, mówiac: Nie jest to, co powiadasz; ale sam sobie to wymyslasz.
\par 9 Albowiem oni wszyscy straszyli nas, mówiac: Oslabieja rece ich przy robocie, i nie dokonaja; prztoz teraz, o Boze! zmocnij rece moje.
\par 10 A gdym wszedl w dom Semejasza, syna Delajaszowego, syna Mehetabelowego, który byl w zawarciu, rzekl mi: Zejdzmy sie do domu Bozego, w posród kosciola, i zamknijmy drzwi koscielne; bo przyjda, chcac cie zabic, a w nocy przyjda, aby cie zabili.
\par 11 Któremum rzekl: Takowyzby maz, jakim ja jest, mial uciekac? Któz takowy, jakom ja, coby wszedlszy do kosciola, zyw zostal? Nie wnijde.
\par 12 I poznalem, ze go Bóg nie poslal ale proroctwo mówil przeciwko mnie, bo go Tobijasz i Sanballat byli przenajeli.
\par 13 Przeto bowiem przenajety byl, abym sie ulakl, i tak uczynil, i zgrzeszyl, azeby mi to u nich bylo na zle imie, czemby mi uragali.
\par 14 Pomnijze, o Boze mój! na Tobijasza i Sanballata, wedlug takowych uczynków ich: takze na Noadyje prorokinie, i na innych proroków, którzy mie straszyli.
\par 15 A tak dokonany jest on mur dwudziestego i piatego dnia miesiaca Elul, piecdziesiatego i drugiego dnia.
\par 16 A gdy to uslyszeli wszyscy nieprzyjaciele nasi, i widzieli to wszyscy narodowie, którzy byli okolo nas, upadlo im bardzo serce;
\par 17 Bo poznali, ze sie ta sprawa od Boga naszego stala. W onez dni wiele przedniejszych z Judy listy swe czesto posylali do Tobijasza, takze od Tobijasza przychodzily do nich.
\par 18 Bo wiele ich bylo w Judzie, co sie z nim sprzysiegli, gdyz on byl zieciem Sechanijasza, syna Arachowego; a Jochanan, syn jego, pojal byl córke Mesullama, syna Barachyjaszowego.
\par 19 Nadto i dobroczynnosc jego opowiadali przedemna, i slowa moje odnosili mu; a listy posylal Tobijasz, aby mie straszyl.

\chapter{7}

\par 1 A gdy byl dobudowany mur, i wystawilem wrota, i postanowieni byli odzwierni, i spiewacy, i Lewitowie:
\par 2 Rozkazalem Chananijemu, bratu memu, i Chananijaszowi, staroscie zamku Jeruzalemskiego: (bo ten byl maz wierny, i bojacy sie Boga nad wielu innych;)
\par 3 I rzeklem do nich: Niech nie otwieraja bram Jeruzalemskich, az ogrzeje slonce; a gdy ci, co tu stawaja, zamkna brame, opatrzciez ja. A tak postanowilem straz z obywateli Jeruzalemskich, kazdego na strazy jego, i kazdego na przeciwko domowi jego.
\par 4 A miasto bylo szerokie i wielkie, ale ludu malo w murach jego, a jeszcze nie byly domy pobudowane.
\par 5 Przetoz podal Bóg mój do serca mego, abym zebral przedniejszych, i przelozonych, i lud, aby byli obliczeni podlug rodzaju. I znalazlem ksiegi rodu tych, którzy sie tu najpierwej przyprowadzili, i znalazlem w nich to opisanie.
\par 6 Cic sa ludzie onej krainy, którzy wyszli z pojmania i z niewoli, w która ich byl zaprowadzil Nabuchodonozor, król Babilonski; a wrócili sie do Jeruzalemu i do Judy, kazdy do miasta swego.
\par 7 Którzy przyszli z Zorobabelem, z Jesua, Nehemijaszem, Azaryjaszem, Rahamijaszem, Nechamanem, Mardocheuszem, Bilsanem, Misperetem, Bigwajem, Nechumem, Baana.
\par 8 A poczet mezów ludu Izraelskiego taki jest: Synów Farosowych dwa tysiace sto i siedmdziesiat i dwa;
\par 9 Synów Sefatyjaszowych trzy sta siedmdziesiat i dwa;
\par 10 Synów Arachowych szesc set piecdziesiat i dwa;
\par 11 Synów Pachatmoabowych, synów Jesui, i Joabowych dwa tysiace osm set i osmnascie;
\par 12 Synów Elamowych tysiac dwiescie piecdziesiat i cztery;
\par 13 Synów Zattuowych osm set czterdziesci i piec;
\par 14 Synów Zachajowych siedm set i szescdziesiat;
\par 15 Synów Binnujowych szesc set czterdziesci i osm;
\par 16 Synów Bebajowych szesc set dwadziescia i osm;
\par 17 Synów Azgadowych dwa tysiace trzy sta dwadziescia i dwa;
\par 18 Synów Adonikamowych szesc set szescdziesiat i siedm;
\par 19 Synów Bigwajowych dwa tysiace szescdziesiat i siedm;
\par 20 Synów Adynowych szesc set piecdziesiat i piec;
\par 21 Synów Aterowych, co poszli z Ezechyjasza, dziewiecdziesiat i osm;
\par 22 Synów Hasumowych trzy sta dwadziescia i osm;
\par 23 Synów Besajowych trzy sta dwadziescia i cztery;
\par 24 Synów Charyfowych sto i dwanascie;
\par 25 Synów z Gabaonu dziewiecdziesiat i piec.
\par 26 Mezów z Betlehemu i Netofatu sto osmdziesiat i osm;
\par 27 Mezów z Anatotu sto dwadziescia i osm;
\par 28 Mezów z Bet Azmawetu czterdziesci i dwa;
\par 29 Mezów z Karyjatyjarymu, z Kafiry i z Beerotu siedm set czterdziesci i trzy;
\par 30 Mezów z Ramy i z Gabaa szesc set dwadziescia i jeden;
\par 31 Mezów z Machmas sto i dwadziescia i dwa;
\par 32 Mezów z Betela i z Haj sto dwadziescia i trzy;
\par 33 Mezów z Nebo drugiego piecdziesiat i dwa.
\par 34 Synów Elama drugiego tysiac dwiescie piecdziesiat i cztery;
\par 35 Synów Harymowych trzy sta i dwadziescia;
\par 36 Synów Jerechowych trzy sta czterdziesci i piec;
\par 37 Synów Lodowych, Hadydowych, i Onowych siedm set i dwadziescia i jeden.
\par 38 Synów Senaa trzy tysiace dziewiec set i trzydziesci.
\par 39 Kaplanów: Synów Jedajaszowych, z domu Jesui, dziewiec set siedmdziesiat i trzy;
\par 40 Synów Immerowych tysiac piecdziesiat i dwa;
\par 41 Synów Passurowych tysiac dwiescie czterdziesci i siedm;
\par 42 Synów Harymowych tysiac i siedmnascie;
\par 43 Lewitów: Synów Jesuego, i Kadmielowych, synów Hodowijaszowych siedmdziesiat i cztery.
\par 44 Spiewaków: Synów Asafowych sto czterdziesci i osm.
\par 45 Odzwiernych: Synów Sallumowych, synów Aterowych, synów Talmonowych, synów Akkubowych, synów Hatytowych, synów Sobajowych sto trzydziesci i osm.
\par 46 Z Netynejczyków: Synów Sycha, synów Chasufa, synów Tabbaota,
\par 47 Synów Kierosa, synów Syjaa, synów Fadona,
\par 48 Synów Lebana, synów Hagaba, synów Salmaja,
\par 49 Synów Hanana, synów Giddela, synów Gachara,
\par 50 Synów Raajasza, synów Rezyna, synów Nekoda,
\par 51 Synów Gazama, synów Uzy, synów Faseacha.
\par 52 Synów Besaja, synów Mechynima, synów Nefusesyma,
\par 53 Synów Bakbuka, synów Chakufa, synów Charchura,
\par 54 Synów Basluta, synów Mechyda, synów Charsa,
\par 55 Synów Barkosa, synów Sysera, synów Tamacha,
\par 56 Synów Nezyjacha, synów Chatyfa,
\par 57 Synów slug Salomonowych, synów Sotaja, synów Soferata, synów Peruda.
\par 58 Synów Jahala, synów Darkona, synów Giddela,
\par 59 Synów Sefatyjasza, synów Chatyla, synów Pochereta z Hasebaim, synów Amona:
\par 60 Wszystkich Netynejczyków i synów slug Salomonowych trzy sta dziewiecdziesiat i dwa.
\par 61 A cic sa, którzy wyszli z Telmelachu i z Telcharsa: Cherub, Addan, i Immer: ale nie mogli okazac domu ojców swoich i nasienia swego, jezli z Izraela byli.
\par 62 Synów Delajaszowych, synów Tobijaszowych, synów Nekodowych szesc set czterdziesci i dwa.
\par 63 A z kaplanów synowie Hobajowi, synowie Kozowi, synowie Barsylajego; który byl pojal z córek Barsylaja Galaadczyka zone, i nazwany byl od imienia ich.
\par 64 Ci szukali opisania swego, wywodzac ród swój, ale nie znalezli; przetoz zrzuceni sa z kaplanstwa.
\par 65 I zakazal im Tyrsata, aby nie jedli z rzeczy najswietszych, azby powstal kaplan z Urym i z Tummim.
\par 66 Wszystkiego zgromadzenia w jednym poczcie bylo czterdziesci tysiecy dwa tysiace trzy sta i szescdziesiat;
\par 67 Oprócz slug ich i sluzebnic ich, których bylo siedm tysiecy trzy sta trzydziesci i siedm; a miedzy nimi bylo spiewaków i spiewaczek dwiescie i czterdziesci i piec.
\par 68 Koni ich siedm set trzydziesci i szesc; mulów ich dwiescie czterdziesci i piec.
\par 69 Wielbladów cztery sta trzydziesci i piec; oslów szesc tysiecy siedm set i dwadziescia.
\par 70 A niektórzy przedniejsi z domów ojcowskich dawali na robote. Tyrsata dal do skarbu zlota tysiac lótów, czasz piecdziesiat, szat kaplanskich piec set i trzydziesci.
\par 71 Niektórzy tez z przedniejszych domów ojcowskich dali do skarbu na robote zlota dwadziescia tysiecy lótów, a srebra grzywien dwa tysiace i dwiescie.
\par 72 A co dal inszy lud, bylo zlota dwadziescia tysiecy lótów, a srebra dwa tysiace grzywien, a szat kaplanskich szescdziesiat i siedm.
\par 73 A tak osiedli kaplani i Lewitowie, i odzwierni, i spiewacy, i lud pospolity, i Netynejczycy, i wszystek Izrael miasta swoje. A gdy nastal miesiac siódmy, byli synowie Izraelscy w miastach swoich.

\chapter{8}

\par 1 Zebral sie tedy wszystek lud jednostajnie na ulice, która jest przed brama wodna, i rzekli do Ezdrasza, nauczonego w Pismie, aby przyniósl ksiegi zakonu Mojzeszowego, który byl przykazal Pan Izraelowi.
\par 2 Tedy przyniósl Ezdrasz kaplan zakon przed ono zgromadzenie mezów i niewiast, i wszystkich, którzyby rozumnie sluchac mogli; a dzialo sie to dnia pierwszego, miesiaca siódmego.
\par 3 I czytal w nim na onej ulicy, która jest przed brama wodna, od poranku az do poludnia przed mezami i niewiastami, i którzy zrozumiec mogli, a uszy wszystkiego ludu obrócone byly do ksiag zakonu.
\par 4 I stanal Ezdrasz nauczony w Pismie na kazalnicy, która byli zgotowali na to, a podle niego stal Matytyjasz, i Sema, i Ananijasz, i Uryjasz, i Helkijasz, i Maasyjasz, po prawej rece jego, a po lewej rece jego Fedajasz, i Misael, i Malchyjasz, i Chasum, i Chasbadana, Zacharyjasz i Mesullam.
\par 5 Otworzyl tedy Ezdrasz ksiegi przed oczyma wszystkiego ludu, bo stal wyzej niz wszystek lud; a gdy je otworzyl, wszystek lud powstal.
\par 6 I blogoslawil Ezdrasz Panu, Bogu wielkiemu, a wszystek lud odpowiadal: Amen! Amen! podnoszac rece swoje; a nachyliwszy glowy, klaniali sie Panu twrza ku ziemi.
\par 7 Takze i Jesua, i Bani, i Serebijasz, Jamin, Chakub, Sabbetaj, Hodyjasz, Maasyjasz, Kielita, Azaryjasz, Jozabad, Chanan, Felajasz, i Lewitowie nauczali ludu zakonu, a lud stal na miejscu swem.
\par 8 Bo czytali w ksiegach zakonu Bozego wyraznie, a wykladajac zmysl objasniali to, co czytali.
\par 9 Zatem Nehemijasz (ten jest Tyrsata) i Ezdrasz kaplan, nauczony w Pismie, i Lewitowie, którzy uczyli lud, rzekli do wszystkiego ludu: Ten dzien poswiecony jest Panu, Bogu waszemu, nie smucciez sie, ani placzcie. (Bo plakal wszystek lud, slyszac slowa zakonu.)
\par 10 I rzekl im: Idzciez, jedzcie rzeczy tluste a pijcie napój slodki, a posylajcie czastki tym, którzy sobie nic nie nagotowali; albowiem swiety jest dzien Panu naszemu. Przetoz sie nie frasujcie; albowiem wesele Panskie jest sila wasza,
\par 11 A gdy Lewitowie uczynili milczenie miedzy wszystkim ludem, mówiac: Milczciez, bo dzien swiety jest, a nie smuccie sie:
\par 12 Tedy odszedl wszystek lud, aby jedli i pili, i aby innym czastki posylali. I weselili sie bardzo, przeto, ze zrozumieli slowa, których i nauczano.
\par 13 Potem zebrali sie dnia drugiego przedniejsi domów ojcowskich ze wszystkiego ludu, kaplani i Lewitowie, do Ezdrasza nauczonego w Pismie, aby wyrozumieli slowa zakonu.
\par 14 I znalezli napisane w zakonie, ze rozkazal Pan przez Mojzesza, aby mieszkali synowie Izraelscy w kuczkach w swieto uroczyste miesiaca siódmego;
\par 15 A izby to opowiedziano i obwolano we wszystkich miastach ich, i w Jeruzalemie, mówiac: Wynijdzcie na góre, a nanoscie galezia oliwnego, i galezia sosnowego, i galezia myrtowego, i galezia palmowego, i galezia drzewa gestego, abyscie poczynili kuczki, jako jest napisane.
\par 16 Przetoz wyszedl lud, a nanosili i poczynili sobie kuczki, kazdy na dachu swym, i w sieniach swych, i w sieniach domu Bozego, i na ulicy bramy wodnej, i na ulicy bramy Efraimowej.
\par 17 A tak naczynilo kuczek wszystko zgromadzenie, które sie wrócilo z niewoli, i mieszkali w kuczkach, (choc tego nie czynili synowie Izraelscy ode dni Jozuego, syna Nunowego, az do dnia onego;) i bylo wesele bardzo wielkie.
\par 18 A Ezdrasz czytal w ksiegach zakonu Bozego na kazdy dzien; od pierwszego dnia az do dnia ostatniego; i obchodzili swieto uroczyste przez siedm dni, a dnia ósmego bylo zgromadzenie wedlug zwyczaju.

\chapter{9}

\par 1 Potem dnia dwudziestego i czwartego tegoz miesiaca zgromadzili sie synowie Izraelscy, i poscili w worzech, i posypali sie prochem.
\par 2 A odlaczylo sie nasienie Izraelskie od wszystkich cudzoziemców, a stanawszy wyznawali grzechy swe i nieprawosci ojców swych.
\par 3 I powstali na miejscach swych, a czytali ksiegi zakonu Pana, Boga swego, cztery kroc przez dzien, i cztery kroc wyznawali a klaniali sie Panu Bogu swemu.
\par 4 Zatem staneli na stopniach Lewitów Jesua i Bani, Kadmiel, Sebanijasz, Bunni, Serebijasz, Bani, Chenani, a wolali glosem wielkim do Pana Boga swego.
\par 5 I mówili Lewitowie Jesua, i Kadmiel, Bani, Hasabnejasz, Serebijasz, Odyjasz, Sebanijasz, Petachyjasz: Wstancie, blogoslawcie Panu, Bogu waszemu, od wieku az na wieki, a niech blogoslawia imieniowi twojemu chwalebnemu i wywyzszonemu nad wszelkie blogoslawienstwo i chwale.
\par 6 Ty, Panie! sam, ty sam, tys uczynil niebo, nieba niebios, i wszystko wojsko ich, ziemie, i wszystko co jest na niej, morza, i wszystko, co w nich jest, a ty ozywiasz to wszystko; a wojska niebieskie tobie sie klaniaja.
\par 7 Tys jest, Panie Boze! którys wybral Abrama, a wywiodles go z Ur Chaldejskiego, i dales mu imie Abraham.
\par 8 I znalazles serce jego wierne przed obliczem twojem, i uczyniles z nim przymierze, ze dasz ziemie Chananejczyka, Hetejczyka, Amorejczyka, i Ferezejczyka, i Jebuzejczyka, i Giergiezejczyka, ze ja dasz nasieniu jego, i zisciles slowa twoje; bos ty sprawiedliwy.
\par 9 Wejzales zaiste na utrapienie ojców naszych w Egipcie, a wolanie ich wysluchales nad morzem Czerwonem.
\par 10 A pokazywales znaki i cuda na Faraonie, i na wszystkich slugach jego, i na wszystkim ludu ziemi jego; bos poznal, ze sobie hardzie postepowali przeciwko nim, i uczyniles sobie imie, jako sie to dzis pokazuje.
\par 11 I rozdzieliles morze przed nimi, a przeszli przez posrodek morza po suszy; a tych, którzy ich gonili, wrzuciles w glebokosci, jako kamien w wody gwaltowne.
\par 12 A w slupie oblokowym prowadziles ich we dnie, a w slupie ognistym w nocy, abys im oswiecal droge, która isc mieli.
\par 13 Potemes na góre Synaj zstapil, i mówiles do nich z nieba, a dales im sady prawe, i zakony prawdziwe, ustawy i rozkazania dobre.
\par 14 I sabat twój swiety oznajmiles im, a przykazania i ustawy, i zakon wydales im przez Mojzesza, sluge twego.
\par 15 Dales im tez chleb w glodzie ich z nieba, i wodes im z skaly wywiódl w pragnieniu ich, a rozkazales im, aby szli, i posiedli ziemie, o któras podniósl reke swa, ze im ja dasz.
\par 16 Ale oni i ojcowie nasi hardzie sobie poczynali, i zatwardzili kark swój, i nie sluchali rozkazania twego.
\par 17 Owszem nie chcieli sluchac, ani wspomnieli na cuda twoje, któres czynil, przy nich; ale zatwardziwszy kark swój postanowili sobie wodza, chcac sie wrócic w niewole swoje w uporze swoim. Lecz ty, o Boze milosciwy, laskawy, i milosierny, nieskwapli wy, i wielkiego milosierdzia! nie opusciles ich.
\par 18 Nawet gdy sobie uczynili cielca ulanego, a mówili: Ten jest Bóg twój, który cie wywiódl z ziemi Egipskiej, i dopuscili sie bluznierstw wielkich,
\par 19 Ty jednak dla litosci twoich wielkich nie opusciles ich na puszczy; slup oblokowy nie odstapil od nich we dnie, prowadzac ich w drodze, ani slup ognisty w nocy, oswiecajac ich, i droge, która isc mieli.
\par 20 Nadto ducha twojego dobrego dales im, aby ich uczyl, i manny twojej nie odjales od ust ich, i wode dales im w pragnieniu ich.
\par 21 A tak przez czterdziesci lat zywiles ich na puszczy; na niczem im nie schodzilo, szaty ich nie zwiotszaly, i nogi ich nie napuchly.
\par 22 I podales im królestwa i narody, któres rozegnal po katach, tak, ze posiedli ziemie Sehonowa, i ziemie króla Hesebonskiego, i ziemie Oga, króla Basanskiego.
\par 23 A synów ich rozmnozyles jako gwiazdy niebieskie, i wwiodles ich do ziemi, o którejs mawial ojcom ich, ze wnijda i posieda ja.
\par 24 Bo przyszedlszy synowie ich posiedli te ziemie, gdys ponizyl przed nimi obywateli onej ziemi, Chananejczyków, i podales ich w rece ich, i królów ich, i narody onej ziemi, aby sie z nimi obchodzili wedlug woli swojej.
\par 25 Pobrali tedy miasta obronne, i ziemie tlusta, i posiedli domy pelne wszelkich dóbr, studnie wykopane, winnice, oliwnice, i drzew rodzajnych bardzo wiele; a jedli, i byli nasyceni, i otyli, i oplywali w rozkoszy z dobroci twojej wielkiej.
\par 26 Ale gdy cie rozdraznili, i stalic sie odpornymi, zarzuciwszy zakon twój w tyl swój, a proroków twoich pobili, którzy sie oswiadczali przed nimi, aby ich nawrócili do ciebie, i dopuszczali sie bluznierstwa wielkiego:
\par 27 Podales ich w rece nieprzyjaciolom ich, którzy ich trapili. A gdy czasu utrapienia swego wolali do ciebie, tys ich z nieba wysluchal, a wedlug litosci twoich wielkich dawales im wybawicieli, którzy ich wybawiali z rak nieprzyjaciól ich.
\par 28 Wtem, gdy troche odpoczeli, znowu czynili zlosc przed twarza twoja; przetoz opusciles ich w rece nieprzyjaciól ich, aby panowali nad nimi. Lecz gdy sie znowu nawrócili, a wolali do ciebie, tys ich z nieba wysluchal, i wybawiles ich wedlug litosci twoich przez wiele czasów.
\par 29 I oswiadczales sie przed nimi, abys ich nawrócil do zakonu swego; ale oni sobie hardzie poczynali, a niesluchali przykazan twoich, owszem przeciw sadom twoim grzeszyli, które gdyby czlowiek czynil, zylby przez nie; ale uchylajac ramion swych, kark swój zatwardzili, i nie sluchali.
\par 30 Wszakzes ty im folgowal przez wiele lat, oswiadczajac sie przed nimi Duchem twym przez proroków twoich; a gdy nie sluchali, podales ich w rece narodów onych ziem.
\par 31 Ale dla litosci twoich wielkich nie dales ich na wytracenie, anis ich opuscil; bos ty Bóg laskawy i milosierny.
\par 32 Teraz tedy o Boze nasz, Boze wielki, mozny, i straszny! który strzezesz przymierza i milosierdzia, niech nie bedzie male przed toba kazde utrapienie, które przyszlo na nas, na królów naszych, na kaiazat naszych, i na kaplanów naszych, i na proroków naszych, i na ojców naszych, i na wszystek lud twój, ode dni królów Assyryjskich az do dnia tego;
\par 33 Aczes ty jest sprawiedliwy we wszystkiem tem, co przyszlo na nas; bos sprawiedliwie uczynil, a mysmy niezboznie czynili.
\par 34 I królowie nasi, ksiazeta nasi, kaplani nasi, i ojcowie nasi nie pelnili zakonu twego, i nie przestrzegali przykazan twoich, i swiadectw twych, któremis sie oswiadczal przed nimi.
\par 35 Bo oni w królestwie swem i w dobroci twojej wielkiej, któras im pokazal, i w ziemi przestronnej i tlustej, któras im byl dal, nie sluzyli tobie, ani sie odwrócili od spraw zlych swoich.
\par 36 Oto mysmy dzis niewolnikami, i ziemia, któras dal ojcom naszym, aby jedli owoc jej, i dobra jej, otosmy niewolnikami w niej.
\par 37 Juz urodzaje swoje obfite wydaje królom, któres postanowil nad nami dla grzechów naszych; panuja nad cialy naszemi, i nad bydlem naszem wedlug woli swej, tak, zesmy w wielkiem ucisnieniu.
\par 38 Wszakze w tem wszystkiem czynimy mocne przymierze, i zapisujemy je, które pieczetuja ksiazeta nasi, Lewitowie nasi, i kaplani nasi.

\chapter{10}

\par 1 A którzy pieczetowali, ci byli: Nehemijasz, Tyrsata, syn Hachalijaszowy, i Sedekijasz,
\par 2 Sarajasz, Azaryjasz, Jeremijasz,
\par 3 Passur, Amaryjasz, Malchyjasz,
\par 4 Hattus, Sebanijasz, Malluch,
\par 5 Harym, Meremot, Obadyjasz,
\par 6 Danijel, Ginneton, Baruch,
\par 7 Mesullam, Abijasz, Mijamin,
\par 8 Maazyjasz, Bilgaj, Semajasz. Ci byli kaplani.
\par 9 A Lewitowie byli: Jesua, syn Azanijaszowy, Binnui z synów Chenadadowych, Kadmiel;
\par 10 Bracia tez ich: Sebanijasz, Odyjasz, Kielita, Felejasz, Chanan,
\par 11 Micha, Rechob, Hasabijasz,
\par 12 Zachur, Serebijasz, Sebanijasz,
\par 13 Odyjasz, Bani, Beninu.
\par 14 Przedniejsi z ludu: Faros, Pachatmoab, Elam, Zattu, Bani.
\par 15 Bunni, Azgad, Bebaj,
\par 16 Adonijasz, Bygwaj, Adyn,
\par 17 Ater, Ezechyjasz, Azur,
\par 18 Chodyjasz, Hasum, Besaj,
\par 19 Haryf, Anatot, Nebaj,
\par 20 Magpijasz, Mesullam, Chesyr,
\par 21 Mesezabel, Sadok, Jaddua,
\par 22 Pelatyjasz, Chanan, Anajasz,
\par 23 Ozeasz, Hananijasz, Hasub,
\par 24 Halloches, Pilcha, Sobek,
\par 25 Rehum, Hasabna, Maasejasz,
\par 26 I Achyjasz, Chanan, Anan,
\par 27 Malluch, Harym, Baana, .
\par 28 Takze i inni z ludu: Kaplani, Lewitowie, odzwierni, spiewacy, Netynejczycy, i wszyscy, którzy sie odlaczyli od narodów onych ziem do zakonu Bozego, zony ich, synowie ich, i córki ich; wszelki umiejetny i rozumny.
\par 29 Chwyciwszy sie tego z bracmi swymi, i z przedniejszymi ich przychodzili, obowiazujac sie przeklestwem i przysiega, ze chca chodzic w zakonie Bozym, który jest podany przez Mojzesza, sluge Bozego, i chca strzedz a czynic wszystkie przykazania Pana, Boga naszego, i sady jego, i ustawy jego;
\par 30 A ze nie damy córek naszych narodom onej ziemi, i córek ich brac nie bedziemy synom naszym.
\par 31 Ani od narodów onej ziemi, którzyby nam przynosili jakie towary, albo jakie zboze w dzien sabatu na sprzedaj, brac od nich bedziemy w sabat, ani w dzien swiety; a ze zaniechamy roku siódmego siania roli i wyciagania wszelakiego dlugu.
\par 32 Postanowilismy tez miedzy soba prawo, abysmy dawali po trzeciej czesci sykla na kazdy rok na potrzebe domu Boga naszego;
\par 33 Na chleby pokladne, i na ofiare ustawiczna, i na calopalenia ustawiczne w sabaty, w pierwsze dni miesiaca, w swieta uroczyste, i na rzeczy swiete, i na ofiary za grzech ku oczyszczeniu Izraela, i na wszelka potrzebe domu Boga naszego.
\par 34 Rzucilismy tez losy okolo noszenia drew miedzy kaplanów, Lewitów, i miedzy lud, aby ich dodawali do domu Boga naszego wedlug domów ojców naszych, na czasy naznaczone, od roku do roku, aby gorzalo na oltarzu Pana, Boga naszego, jako jest napisane w zakonie.
\par 35 Takze aby przynosili pierworodztwa ziemi naszej, i pierworodztwa wszelkiego owocu kazdego drzewa, od roku do roku, do domu Panskiego.
\par 36 Do tego pierworodztwa synów naszych, i bydel naszych, jako napisano w zakonie, i pierworodztwa wolów naszych, i owiec naszych, zeby przynosili do domu Boga naszego kaplanom sluzacym w domu Boga naszego.
\par 37 Nadto pierwociny ciast naszych, i podnoszonych ofiar naszych, i owoce wszelkiego drzewa, moszczu, i oliwy swiezej, aby przynosili kaplanom do gmachów domu Boga naszego, i dziesiecine ziemi naszej Lewitom; a cic Lewitowie wybierac beda te dziesiecine we wszystkich miastach robót naszych.
\par 38 A bedzie kaplan, syn Aarona, przy Lewitach, gdy Lewitowie te dziesiecine odbierac beda; a Lewitowie wniosa dziesiecine z dziesieciny do domu Boga naszego, do komór w domu skarbnicy.
\par 39 Bo do tych komór odnosic beda synowie Izraelscy, i synowie Lewiego, ofiare zboza, moszczu, i oliwy swiezej, gdzie sa naczynia swiatnicy, i kaplani sluzacy, i odzwierni, i spiewacy, abysmy nie opuszczali domu Boga naszego.

\chapter{11}

\par 1 Mieszkali tedy przedniejsi z ludu w Jeruzalemie, a inny lud miotali losy, aby wzieli dziesiatego czlowieka na mieszkanie w Jeruzalemie, miescie swietem, a dziewiec czesci w innych miastach.
\par 2 I blogoslawil lud wszystkim mezom, którzy sie dobrowolnie ofiarowali, aby mieszkali w Jeruzalemie.
\par 3 A cic sa przedniejsi onej krainy, którzy mieszkali w Jeruzalemie. (A w inszych miastach Judzkich mieszkal kazdy w osiadlosci swojej, i w miastach swych, Izraelczycy, kaplani, i Lewitowie, i Netynejczycy, i synowie slug Salomonowych.)
\par 4 A tak w Jeruzalemie mieszkali niektórzy z synów Judowych i z synów Benjaminowych. Z synów Judowych: Atajasz, syn Uzyjasza, syna Zacharyjaszowego, syna Amaryjaszowego, syna Sefatyjaszowego, syna Mahaleelowego z synów Faresowych:
\par 5 Takze Maasejasz, syn Barucha, syna Cholhozowego, syna Hasajaszowego, syna Hadajaszowego, syna Jojarybowego, syna Zacharyjaszowego, syna Sylonczykowego.
\par 6 Wszystkich synów Faresowych, mieszkajacych w Jeruzalemie, cztery sta szescdziesiat i osm mezów duzych.
\par 7 A synowie Benjaminowi ci sa: Sallu, syn Mesullama, syna Joedowego, syna Fadajaszowego, syna Kolajaszowego, syna Maasajaszowego, syna Ityjelowego, syna Izajaszowego;
\par 8 A po nim Gabaj, Sallaj, wszystkich dziewiec set dwadziescia i osm.
\par 9 I Joel, syn Zychry, byl przelozonym nad nimi, a Juda, syn Senua, nad miastem wtóry.
\par 10 Z kaplanów mieszkali Jedajasz, syn Jojaryba, i Jachyn;
\par 11 Serajasz, syn Helkijasza, syna Mesullamowego, syna Sadokowego, syna Merajotowego, syna Achytobowego, przelozony w domu Bozym.
\par 12 A braci ich, którzy odprawiali roboty domowe, osm set dwadziescia i dwa; i Adajasz, syn Jerohama, syna Pelalijaszowego, syna Amsego, syna Zacharyjaszowego, syna Passurowego, syna Malchyjaszowego.
\par 13 A braci jego przedniejszych z domów ojcowskich dwiescie czterdziesci i dwa; i Amasesaj, syn Asareli, syna Achzajowego, syna Mesullemitowego, syna Immerowego.
\par 14 A braci ich, mezów duzych, sta dwadziescia i osm, i przelozony nad nimi Zabdyjel, syn Giedolima.
\par 15 A z Lewitów Semejasz, syn Hassuby, syna Asrykamowego, syna Hasabijaszowego, syna Bunni.
\par 16 A Sabbataj i Jozabad byli nad robota, która byla z dworu przy domu Bozym, a cic byli z przedniejszych Lewitów.
\par 17 A Matanijasz, syn Michasa, syna Zabadyjaszowego, syna Asafowego, byl przedniejszy w zaczynaniu chwaly przy modlitwie; a Bakbukijasz wtóry z braci jego, i Abda, syn Sammuj, syna Galilowego, syna Jedytunowego.
\par 18 Wszystkich Lewitów bylo w miescie swietem dwiescie osmdziesiat i czterech.
\par 19 A z odzwiernych: Akkub, Talmon, i braci ich, i strózów w bramach, sto siedemdziesiat i dwa.
\par 20 A drudzy z Izraela, z kaplanów, z Lewitów, mieszkali we wszystkich miastach Judzkich, kazdy w dziedzictwie swojem.
\par 21 Ale Netynejczycy mieszkali w Ofelu; a Sycha i Gipsa byli nad Netynejczykami.
\par 22 A przelozony nad Lewitami w Jeruzalemie byl Uzy, syn Bani, syna Chasabajaszowego, syna Matanijaszowego, syna Michasowego. Ci byli z synów Asafowych spiewacy przy sluzbie domu Bozego.
\par 23 Albowiem rozkazanie królewskie bylo o nich, i pewne opatrzenie dla spiewaków na kazdy dzien.
\par 24 A Petachyjasz, syn Mesezabelowy, z synów Zachara, syna Judowego, byl na miejscu królewskiem w kazdej sprawie do ludu.
\par 25 A we wsiach i polach ich z synów Judowych mieszkali w Karyjat Arbie i we wsiach jego, i w Dybon i we wsiach jego, i w Jekabseel i we wsiach jego;
\par 26 I w Jesue, i w Molada, i w Betfelet;
\par 27 I w Hasersual, i w Beersabe i we wsiach jego;
\par 28 I w Sycelegu, i w Mechona i we wsiach jego;
\par 29 I w Enrymmon, i w Saraa, i w Jerymut;
\par 30 W Zanoe, w Adullam i we wsiach ich; w Lachys i na polach jego; w Aseku i we wsiach jego. A tak mieszkali od Beerseby az do Giehennom.
\par 31 A synowie Benjaminowi z Gabaa mieszkali w Machmas, i w Haju, i w Betel i we wsiach jego;
\par 32 W Anatot, w Nobie, w Ananija;
\par 33 W Chasor, w Rama, w Gietaim;
\par 34 W Hadyd, w Seboim, w Neballat;
\par 35 W Lod, i w Ono, i w dolinie rzemieslników.
\par 36 A z Lewitów mieszkali niektórzy w dzialach Judzkich i w Benjamickich.

\chapter{12}

\par 1 A cic sa kaplani i Lewitowie, którzy przyszli z Zorobabelem, synem Sealtyjelowym, i z Jesua: Serejasz, Jeremijasz, Ezdrasz.
\par 2 Amaryjasz, Malluch, Hattus,
\par 3 Sechanijasz, Rehum, Meremot,
\par 4 Iddo, Ginnetoj, Abijasz;
\par 5 Mijamin, Maadyjasz, Bilgal,
\par 6 Semejasz, i Jojaryb, Jedajasz,
\par 7 Sallu, Amok, Helkijasz, Jedajasz. Cic byli przedniejsi z kaplanów i z braci swych, za dni Jesuego.
\par 8 A Lewitowie: Jesua, Binnui, Kadmiel, Serebijasz, Juda; Matanijasz nad piesniami, sam i bracia jego.
\par 9 A Bakbukijasz i Hunni, bracia ich, byli przeciwko nim w porzadku swoim.
\par 10 A Jesua splodzil Joakima, a Joakim splodzil Elijasyba, a Elijasyb splodzil Jojade:
\par 11 A Jojada splodzil Jonatana, a Jonatan splodzil Jaddue.
\par 12 A za dni Joakima byli kaplani przedniejsi z domów ojcowskich: z Serajaszowego Merajasz, z Jeremijaszowego Chananijasz;
\par 13 Z Ezdraszowego Mesullam, z Amaryjaszowego Jochanan;
\par 14 Z Melchowego Jonatan, z Sechanijaszowego Józef;
\par 15 Z Harymowego Adna, z Merajotowego Helkaj;
\par 16 Z Iddowego Zacharyjasz, z Ginnetowego Mesullam;
\par 17 Z Abijaszowego Zychry, z Miniaminowego i z Maadyjaszowego Piltaj;
\par 18 Z Bilgowego Sammua, z Semejaszowego Jonatan;
\par 19 A z Jojarybowego Mattenaj, z Jedajaszowego Uzy;
\par 20 Z Sallajowego Kalaj, z Amokowego Heber;
\par 21 Z Helkijaszowego Hasabijasz, z Jedajaszowego Natanael.
\par 22 A Lewitowie za dni Elijasyba, Jojady, i Jochanana, i Jadduego popisani sa, którzy byli przedniejszymi z domów ojcowskich; takze i kaplani az do królestwa Daryjusza, króla Perskiego.
\par 23 Synowie mówie Lewiego, przedniejsi z domów ojcowskich, zapisani sa w ksiegach kroniki az do dni Jochanana, syna Elijasybowego.
\par 24 Przedniejsi mówie z Lewitów byli Hasabijasz, Serebijasz, i Jesua, syn Kadmielowy, i bracia ich przeciwko nim, ku chwaleniu i wyslawianiu Boga wedlug rozkazania Dawida, meza Bozego, straz przeciwko strazy.
\par 25 Matanijasz, i Bakbukijasz, Obadyjasz, Mesullam, Talmon, Akkub, byli strózami odzwiernymi przy domu skarbu u bram.
\par 26 Cic byli za dni Joakima, syna Jesuego, syna Jozedekowego, i za dni Nehemijasza wodza, i Ezdrasza kaplana, nauczonego w Pismie.
\par 27 A przy poswiecaniu muru Jeruzalemskiego szukano Lewitów ze wszystkich miejsc ich, aby ich przywiedziono do Jeruzalemu, zeby wykonali poswiecania i wesela, a to z wyslawianiem i z spiewaniem, z cymbalami, z lutniami, i z cytrami.
\par 28 Przetoz zgromadzeni sa synowie spiewaków, i z równin okolo Jeruzalemu, i ze wsi Netofatyckich.
\par 29 Takze z domu Gilgal, i z pól Gieba, i z Azmawet; bo sobie spiewacy budowali wsi okolo Jeruzalemu.
\par 30 A oczysciwszy sie kaplani i Lewitowie, oczyscili tez lud, i bramy, i mur.
\par 31 Zatemem rozkazal wstapic ksiazetom Judzkim na mur, i postawilem dwa hufy wielkie chwalacych, z których jedni szli na prawo od wyzszej strony muru ku bramie gnojowej.
\par 32 A za nimi szedl Hozajasz, i polowa ksiazat Judzkich;
\par 33 Takze Azaryjasz, Ezdrasz, i Mesullam,
\par 34 Juda, i Benjamin, i Semejasz, i Jeremijasz.
\par 35 Potem niektórzy z synów kaplanskich z trabami, mianowicie Zacharyjasz, syn Jonatana, syna Semejaszowego, syna Matanijaszowego, syna Michajaszowego, syna Zachurowego, syna Asafowego;
\par 36 A bracia jego Semejasz, i Asarel, Milalaj, Gilalaj, Maaj, Netaneel, i Juda, Chanani, z instrumentami muzycznemi Dawida, meza Bozego, a Ezdrasz, nauczony w Pismie, przed nimi.
\par 37 Potem ku bramie zródla, która przeciwko nim byla, wstepowali po schodach miasta Dawidowego, któredy chodza na mur, a od muru przy domu Dawidowym az do bramy wodnej na wschód slonca.
\par 38 A drugi huf chwalacych szedl przeciwko nim, a ja za nim, a polowa ludu po murze od wiezy Tannurym az do muru szerokiego;
\par 39 A od bramy Efraim ku bramie starej, i ku bramie rybnej, i wiezy Chananeel, i wiezy Mea, az do bramy owczej. I stanal u bramy strazy.
\par 40 Potem stanely one dwa hufy chwalacych w domu Bozym, i ja i polowa przelozonych ze mna.
\par 41 Takze kaplani: Elijakim, Maasejasz, Minijamin, Michajasz, Elijenaj, Zacharyjasz, Chananijasz, z trabami;
\par 42 I Maazyjasz, i Semejasz, i Eleazar, i Uzy, i Jochanan, i Malchyjasz, i Elam, i Ezer; a spiewacy glosno spiewali, i Izrachyjasz, przelozony ich.
\par 43 Sprawowali takze onegoz dnia ofiary wielkie, i weselili sie; albowiem Bóg rozweselil ich byl weselem wielkiem, tak, iz sie i niewiasty i dziatki weselily; i bylo slyszec wesele Jeruzalemskie daleko.
\par 44 Obrani tez sa dnia onego mezowie nad komorami skarbów, i ofiar, pierwocin, i dziesiecin, aby zgromadzali do nich z pól miejskich dzialy, zakonem warowane kaplanom i Lewitom; bo sie weselil Juda z kaplanów i z Lewitów tam stojacych;
\par 45 Którzy strzegli strazy Boga swego, i strazy oczyszczania, i spiewaków, i odzwiernych, wedlug rozkazania Dawida i Salomona, syna jego.
\par 46 Bo za dni Dawida i Asafa byli postanowieni z starodawna przelozeni nad spiewakami dla spiewania, wychwalania i dziekczynienia Bogu.
\par 47 Przetoz wszystek Izrael za dni Zorobabela, i za dni Nehemijasza, dawali dzialy dla spiewaków i odzwiernych, kazdodzienny wymiar, a oddawali to, co poswiecili, Lewitom; Lewitowie zas oddawali synom Aaronowym.

\chapter{13}

\par 1 Onegoz dnia czytano w ksiegach Mojzeszowych, tak, iz lud slyszal. I znaleziono w nich napisane, ze nie mial wchodzic Ammonitczyk i Moabczyk do zgromadzenia Bozego, az na wieki;
\par 2 Przeto, iz nie zaszli synom Izraelskim z chlebem i z woda; owszem najeli przeciwko nim Balaama, aby ich przeklinal; ale obrócil Bóg nasz ono przeklestwo w blogoslawienstwo.
\par 3 A gdy uslyszeli zakon, odlaczyli wszystek lud pospolity od Izraela.
\par 4 Ale sie przedtem Elijasyb kaplan, przelozony nad skarbnica domu Boga naszego, spowinowacil z Tobijaszem;
\par 5 I zbudowal mu gmach wielki, kedy przedtem odkladano dary, kadzidlo, i naczynia, i dziesieciny zboza, moszczu, i oliwy swiezej, opatrzenie Lewitom, i spiewakom, i odzwiernym, takze ofiary podnoszone kaplanskie.
\par 6 Ale przy tem wszystkiem nie bylem w Jeruzalemie; albowiem roku trzydziestego i wtórego Artakserksesa, króla Babilonskiego, przyszedlem do króla, a po wyjsciu kilku lat uprosilem sie u króla.
\par 7 A gdym przyszedl do Jeruzalemu, wyrozumialem to zle, które uczynil Elijasyb kwoli Tobijaszowi, iz mu zbudowal gmach w sieniach domu Bozego.
\par 8 Co mi sie bardzo nie podobalo: przetoz wyrzucilem wszystkie naczynia domu Tobijaszowego precz z onegoz gmachu;
\par 9 I rozkazalem oczyscic one gmachy, i wnioslem tam zas nacznia domu Bozego, dary i kadzidlo.
\par 10 Nadtom sie dowiedzial, ze dzialy Lewitom nie byly oddawane, a iz sie Lewitowie i spiewacy, którzy pilnowali pracy, rozbiegli, kazdy do roli swojej.
\par 11 Przetoz zgromilem przelozonych, mówiac: Przeczze opuszczamy dom Bozy? A tak zebrawszy ich, postawilem ich na miejscach ich.
\par 12 A wszystek Juda przynosili dziesieciny zboza, i moszczu, i swiezej oliwy do skarbu.
\par 13 I postanowilem podskarbich nad skarbmi: Selemijasza kaplana, i Sadoka nauczonego w Pismie, i Fadajasza z Lewitów, a przy nich Hanana, syna Zachurowego, syna Matanijaszowego, bo ich za wiernych miano; a ich powinnosc byla, dzialy rozdawac braciom swym.
\par 14 Wspomnij na mie o Boze mój! dla tego, a nie wymazuj dobroczynnosci moich, którem czynil przy domu Boga mego, i przy obrzedach jego.
\par 15 W onez dni widzialem w Judzie tloczacych prasy w sabat, i noszacych snopy, które kladli na osly, takze i winne grona, i figi, i wszelki ciezar, a przynoszacych do Jeruzalemu w dzien sabatu, i zgromilem ich onegoz dnia, którego sprzedawali zywnosc.
\par 16 Takze Tyryjczycy, którzy mieszkali w niem, przynosili ryby, i rozmaite towary, a sprzedawali w sabat synom Judy, i w Jeruzalemie.
\par 17 Przetozem zgromil przelozonych w Judzie, mówiac do nich: Cóz to jest za nieprawosc, której sie dopuszczacie, gwalcac dzien sabatu?
\par 18 Izali nie toz czynili ojcowie wasi, przez co Bóg nasz przywiódl na nas to wszystko zle, i na to miasto? A wy przyczyniacie gniewu na Izraela, gwalcac sabat.
\par 19 A gdy okryl cien bramy Jeruzalemskie przed sabatem, rozkazalem zamknac wrota, i nie kazalem ich otwierac, az po sabacie; a niektórych z slug moich postawilem w bramach, zeby nie wnoszono zadnych brzemion w dzien sabatu.
\par 20 Przetoz zostali przez noc kupcy, i sprzedawajacy towary rozmaite przed miastem Jeruzalemskiem, raz i drugi.
\par 21 Przeciwko którym oswiadczylem sie, i rzeklem do nich: Przecz wy zostawacie przez noc za murem? Uczynicieli to wiecej, sciagne reke na was. A tak od onego czasu nie przychodzili w sabat.
\par 22 I rozkazalem Lewitom, aby sie oczyscili, a przyszedlszy strzegli bram, i swiecili dzien sabatu. I dlatego wspomnij na mie, o Boze mój! a badz mi milosciw wedlug obfitosci milosierdzia twego.
\par 23 W tychze dniach ujrzalem tez Zydów, którzy sobie pojeli zony Azotyckie, Ammonickie, Moabskie.
\par 24 A synowie ich na poly mówili po azotycku, nie umiejac mówic po zydowsku, ale wedlug jezyka swego narodu.
\par 25 Przetozem ich zgromil, i przeklinalem ich, i bilem niektórych z nich, a rwalem ich za wlosy, i poprzysiaglem ich przez Boga, aby nie dawali córek swoich synom ich, ani brali córek ich synom swym i sobie.
\par 26 Izaliz nie przez to zgrzeszyl Salomon, król Izraelski? choc miedzy wieloma narodami nie bylo króla jemu podobnego, który byl mily Bogu swemu, tak, ze go Bóg postanowil królem, nad wszystkim Izraelem, przeciez i tego przywiodly niewiasty cudzoziemskie do grzechu.
\par 27 A wam izali pozwolimy, zebyscie sie dopuszczali tej wielkiej zlosci, a wystepowali przeciwko Bogu naszemu pojmujac zony cudzoziemskie?
\par 28 Lecz jeden z synów Jojady, syna Elijasyba, kaplana najwyzszego, byl zieciem Sanballata Horonczyka, i wygnalem go od siebie.
\par 29 Wspomnijze na to, o Boze mój! przeciw tym, którzy plugawia kaplanstwo, i umowe kaplanska i Lewitska.
\par 30 Przetozem ich oczyscil od wszelkiego cudzoziemca, i postanowilem porzadki kaplanom i Lewitom, kazdemu w pracy jego,
\par 31 I ku noszeniu drew do ofiar, czasów postanowionych, takze i pierwocin. Wspomnijze na mie, Boze mój! ku dobremu.


\end{document}