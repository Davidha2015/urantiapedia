\begin{document}

\title{Job}


\chapter{1}

\par 1 Byl maz w ziemi Uz, imieniem Ijob; a ten maz byl doskonaly, i szczery, i bojacy sie Boga, a odstepujacy od zlego.
\par 2 I urodzilo mu sie siedm synów, a trzy córki.
\par 3 A mial dobytku siedm tysiecy owiec, i trzy tysiace wielbladów, i piec set jarzm wolów, i piec set oslic, i czeladzi bardzo wiele, tak, iz on maz byl najmozniejszym nad wszystkich ludzi na wschód slonca.
\par 4 I schadzali sie synowie jego, a sprawowali uczty, kazdy w domu swym, dnia swojego; i posylali, a wzywali trzy siostry swoje, aby jadaly i pijaly z nimi.
\par 5 A gdy wkolo obeszly dni uczty, posylal Ijob, a poswiecal ich, a wstawajac rano sprawowal calopalenia wedlug liczby ich wszystkich; bo mówil Ijob: Podobno zgrzeszyli synowie moi, a zlorzeczyli Bogu w sercach swych. Tak czynil Ijob po one wszystkie dni.
\par 6 I stalo sie niektórego dnia, gdy przyszli synowie Bozy, aby staneli przed Panem, ze tez przyszedl i szatan miedzy nich.
\par 7 Tedy rzekl Pan do szatana: Skad idziesz? I odpowiedzial szatan Panu, i rzekl: Okrazalem ziemie, i przechadzalem sie po niej.
\par 8 I rzekl Pan do szatana: Przypatrzylzes sie sludze memu Ijobowi, ze mu niemasz równego na ziemi? Maz to doskonaly i szczery, bojacy sie Boga, i odstepujacy od zlego.
\par 9 I odpowiedzial szatan Panu i rzekl: Izaz sie Boga Ijob darmo boi?
\par 10 Azazes go ty nie ogrodzil, i domu jego, i wszystkiego co ma, w okolo zewszad? Blogoslawiles sprawom rak jego, i dobytek jego rozmnozyl sie na ziemi;
\par 11 Ale sciagnij tylko reke twoje a dotknij wszystkiego, co ma: obaczysz, jezlic w oczy zlorzeczyc nie bedzie.
\par 12 Tedy rzekl Pan do szatana: Oto wszystko co ma, jest w rece twojej: tylko nan nie sciagaj reki twej. I odszedl szatan od oblicza Panskiego.
\par 13 Stalo sie tedy niektórego dnia, gdy synowie jego i córki jego jedli, i pili wino w domu brata swego pierworodnego:
\par 14 Przybiezal posel do Ijoba i rzekl: Woly oraly, a oslice sie pasly podle nich:
\par 15 I przypadli Sabejczycy, i zabrali je, a slugi pozabijali ostrzem miecza; a uszedlem tylko ja, ja sam, abym ci oznajmil.
\par 16 A gdy ten jeszcze mówil, przyszedl drugi, i rzekl: Ogien Bozy spadl z nieba i spalil owce i slugi, i pozarl ich; a uszedlem tylko ja, ja sam, abym ci oznajmil.
\par 17 A gdy ten jeszcze mówil, przyszedl i inny, i rzekl: Chaldejczycy, rozsadziwszy sie na trzy hufce, wypadli na wielblady, i zabrali je, i slugi pozabijali ostrzem miecza; a uszedlem tylko ja, ja sam, abym ci oznajmil.
\par 18 A gdy ten jeszcze mówil, przybiezal i inny, a rzekl: Synowie twoi, i córki twoje jedli i pili wino w domu brata swego pierworodnego;
\par 19 A oto wiatr gwaltowny przypadl od onej strony pustyni, i uderzyl na cztery wegly domu, tak, ze upadl na dzieci, i pomarly; a uszedlem tylko ja, ja sam, abym ci oznajmil.
\par 20 Tedy wstal Ijob, i rozdarl plaszcz swój, i ogolil glowe swa, a upadlszy na ziemie, uczynil Panu poklon,
\par 21 I rzekl: Nagim wyszedl z zywota matki mojej, i nagim sie zas tamze wróce; Pan dal, Pan tez wzial, niech bedzie imie Panskie blogoslawione.
\par 22 W tem wszystkiem nie zgrzeszyl Ijob, a nie przypisal Bogu nic nieprzystojnego.

\chapter{2}

\par 1 I stalo sie niektórego dnia, gdy przyszli synowie Bozy, aby staneli przed Panem, przyszedl tez szatan miedzy nich, aby stanal przed Panem.
\par 2 Tedy rzekl Pan do szatana: Gdzies byl, skad idziesz? I odpowiedzial szatan Panu, a rzekl: Okrazalem ziemie, i przechodzilem sie po niej.
\par 3 Zatem rzekl Pan do szatana: Przypatrzylzes sie sludze memu Ijobowi, ze mu nie masz równego na ziemi? Maz to doskonaly i szczery, bojacy sie Boga, a odstepujacy od zlego, i który jeszcze trwa w uprzejmosci swojej; a tys mie pobudzil przeciw niemu, abym go niszczyl bez przyczyny.
\par 4 I odpowiedzial szatan Panu, i rzekl: Skóre za skóre, i wszystko, co ma czlowiek, da za dusze swoje;
\par 5 Ale sciagnij tylko reke twoje, a dotknij kosci jego, i ciala jego, ujrzysz, jezlizec w oczy zlorzeczyc nie bedzie.
\par 6 Tedy rzekl Pan do szatana: Oto w rece twojej jest; wszakze zywot jego zachowaj.
\par 7 Wszedlszy tedy szatan od oblicza Panskiego, zarazil Ijoba wrzodem zlym od stopy nogi jego az do wierzchu glowy jego;
\par 8 Tak, ze wzial skorupe, aby sie nia skrobal; i siedzial w popiele.
\par 9 I rzekla mu zona jego: A jeszczez trwasz w uprzejmosci twojej? Zlorzecz Bogu, a umrzyj.
\par 10 I rzekl do niej: Tak wlasnie mówisz, jako szalone niewiasty mawiaja. Izali tylko dobre przyjmowac bedziemy od Boga, a zlego przyjmowac nie bedziemy? W tem wszystkiem nie zgrzeszyl Ijob usty swemi.
\par 11 A gdy uslyszeli trzej przyjaciele Ijobowi to wszystko utrapienie, które nan przypadlo, przyszli kazdy z miejsca swego: Elifas Temanczyk, i Bildad Suhytczyk, i Sofar Naamatczyk; bo sie byli namówili pospolu, aby przyszedlszy pozalowali go, i cieszyli go.
\par 12 A podnióslszy oczy swoje z daleka, nie poznali go, i wynióslszy glos swój plakali, a rozdarlszy kazdy plaszcz swój miotali proch nad glowy swe ku niebu;
\par 13 I siedzieli z nim na ziemi siedm dni i siedm nocy, a zaden do niego slowa nie przemówil; bo widzieli, ze sie gwaltownie wzmagala bolesc jego.

\chapter{3}

\par 1 Potem otworzyl Ijob usta swoje, i zlorzeczyl dniowi swemu.
\par 2 I zawolal Ijob, mówiac:
\par 3 Bodaj byl zginal dzien, któregom sie urodzil! i noc, w która rzeczono: Poczal sie mezczyzna!
\par 4 Bodaj sie byl on dzien obrócil w ciemnosc! By sie byl o nim nie pytal Bóg z wysokosci, i nie byl oswiecony swiatloscia!
\par 5 Bodaj go byla zacmila ciemnosc i cien smierci! by go byl ogarnal oblok, i ustraszyla go goracosc dzienna!
\par 6 Bodaj byla noc one osiadla ciemnosc, aby nie szla w liczbe dni rocznych, i w liczbe miesiecy nie przyszla!
\par 7 Bodaj noc ona byla samotna, a spiewania aby nie bylo w niej!
\par 8 Bodaj ja byli przekleli, którzy przeklinaja dzien, którzy sa gotowi, wzruszac placz swój!
\par 9 Bodaj sie byly zacmily gwiazdy przy zmierzkaniu jej! a czekajac swiatla, aby sie go byla nie doczekala, ani nie ogladala zorzy porannej!
\par 10 Iz nie zawarla drzwi zywota mego, a nie skryla bolesci od oczu moich.
\par 11 Przeczzem w zywocie nie umarl, albo, gdym z zywota wyszedl, czemum nie zginal?
\par 12 Przeczze mie piastowano na kolanach? a przeczzem ssal piersi?
\par 13 Albowiembym teraz lezal i odpoczywal; spalbym i mialbym pokój,
\par 14 Z królmi i z radcami ziemi, którzy sobie budowali na miejscach pustych;
\par 15 Albo z ksiazetami, którzy mieli zloto, a napelniali domy swe srebrem,
\par 16 Albo czemum sie nie stal jako martwy plód skryty? albo jako niemowlatka, które nie ogladaly swiatlosci?
\par 17 Tam niepobozni przestawaja straszyc, i tam odpoczywaja zwatleni w sile.
\par 18 Tamze wiezniowie sobie wydychaja, a nie slysza glosu trapiacego ich,
\par 19 Maly i wielki tam sobie sa równi a niewolnik wolny od pana swego.
\par 20 Przecz nedznemu dana jest swiatlosc, a zywot tym, którzy sa utrapionego ducha?
\par 21 Którzy czekaja smierci, a nie przychodzi, choc jej pilniej szukaja niz skarbów skrytych;
\par 22 Którzyby sie z radoscia weselili, plasajac, gdyby znalezli grób.
\par 23 Przecz dana jest swiatlosc mezowi, którego droga skryta jest, a którego Bóg ciezkosciami ogarnal?
\par 24 Albowiem kiedy mam jesc, wzdychanie moje przychodzi, a rozchodzi sie jako woda ryczenie moje;
\par 25 Bo strach, któregom sie lekal, przyszedl na mie, a czegom sie obawial, przydalo mi sie.
\par 26 Nie bylem bezpieczny, anim sie uspokoil, anim odpoczywal, a przeciez na mie przyszla trwoga.

\chapter{4}

\par 1 Tedy odpowiedzial Elifas Temanczyk, i rzekl:
\par 2 Jezli bedziemy mówili z toba, nie bedzie ci to przykro? Ale któz sie moze od mówienia zatrzymac?
\par 3 Otos ich wiele uczyl, i reces mdle potwierdzal.
\par 4 Upadajacego wspieraly mowy twoje, a kolana zemdlone posilales.
\par 5 A teraz, gdy to na cie przyszlo, niecierpliwie znosisz, a iz cie dotknelo, trwozysz soba.
\par 6 Azaz poboznosc twoja nie byla ufnoscia twoja, a uprzejmosc spraw twoich oczekiwaniem twojem?
\par 7 Wspomnij prosze, kto kiedy niewinny zginal? albo gdzieby ludzie szczerzy zniszczeli?
\par 8 Jakom widal, ze ci, którzy orali zlosc, i rozsiewali przewrotnosc, toz tez zasie zeli.
\par 9 Bo tchnieniem Bozem gina, a od ducha gniewu jego niszczeja.
\par 10 Ryk lwi, i glos lwicy, i zeby lwiat wytracaja.
\par 11 Lew ginie, iz nie ma lupu, i szczenieta lwie rozproszone bywaja.
\par 12 Nadto doszlo mie slowo potajemnie, i pojelo ucho moje cokolwiek z niego.
\par 13 W rozmyslaniu widzenia nocnego, gdy przypada twardy sen na ludzi,
\par 14 Zdjal mie strach i lekanie, które wszystkie kosci moje przestraszylo.
\par 15 A duch szedl przed twarza moja, tak, iz wlosy wstaly na ciele mojem.
\par 16 Stanal, a nie znalem twarzy jego, ksztalt tylko jakis byl przed oczyma memi; uciszylem sie, i slyszalem glos mówiacy:
\par 17 Izali czlowiek moze byc sprawiedliwszy nizeli Bóg; albo maz czystszy niz Stworzyciel jego?
\par 18 Oto w slugach jego niemasz doskonalosci, a w Aniolach swoich znalazl niedostatek;
\par 19 Daleko wiecej w tych, co mieszkaja w domach glinianych, których grunt jest na prochu, i starci bywaja snadniej nizeli mól.
\par 20 Od poranku az do wieczora bywaja starci; a iz tego nie uwazaja, na wieki zgina.
\par 21 Azaz zacnosc ich nie pomija z nimi? umieraja, ale nie w madrosci.

\chapter{5}

\par 1 Zawolajze tedy, jezli kto jest, cocby odpowiedzial? a do któregoz sie z swietych obrócisz?
\par 2 Zaiste glupiego zabija gniew, a prostaka umarza zawisc.
\par 3 Jam widzial glupiego, iz sie rozkorzenil; alem wnet zle tuszyl mieszkaniu jego.
\par 4 Oddaleni beda synowie jego od zbawienia, i starci beda w bramie, a nie bedzie, ktoby ich wyrwal.
\par 5 Zniwo jego glodny pozre, i z samego ciernia wybierze je; a polknie chciwy bogactwa takowych.
\par 6 Albowiem nie z prochu wychodzi utrapienie, ani z ziemi wyrasta klopot.
\par 7 Ale czlowiek na klopot sie rodzi jako iskry z wegla lataja w góre.
\par 8 Zaiste jabym szukal Boga, Bogubym przelozyl sprawe swoje;
\par 9 Który czyni rzeczy wielkie i niewybadane, dziwne, którym liczby niemasz;
\par 10 Który daje deszcz na ziemie, i spuszcza wody na pola;
\par 11 Który sadza pokornych wysoko, a smutnych wywyzsza ku zbawieniu;
\par 12 Który w niwecz obraca mysli chytrych, tak, iz rece ich nie sprawia nic skutecznego;
\par 13 Który chwyta madrych w chytrosci ich, a rade przewrotnych predko niszczy.
\par 14 We dnie taczaja sie jako w ciemnosciach, a jako w nocy macaja w poludnie.
\par 15 Który zachowuje ubogiego od miecza, od ust ich, i od reki gwaltownika.
\par 16 Mac ucisniony nadzieje; ale nieprawosc stuli usta swe.
\par 17 Oto blogoslawiony czlowiek, którego Bóg karze; przetoz karaniem Wszechmocnego nie pogardzaj,
\par 18 Bo on zrania i zawiazuje; uderza, a rece jego uzdrawiaja.
\par 19 Z szesciu ucisków wyrwie cie, a w siódmym nie tknie sie ciebie zle.
\par 20 W glodzie wybawi cie od smierci, a na wojnie z rak miecza.
\par 21 Przed biczem jezyka ukryty bedziesz, a nie ulekniesz sie w spustoszeniu, gdy przyjdzie.
\par 22 W spustoszeniu i w glodzie smiac sie bedziesz, a zwierzat ziemskich bac sie nie bedziesz.
\par 23 Bo z kamieniem polnym bedzie przymierze twoje, a okrutny zwierz polny spokojnym ci sie stawi.
\par 24 I poznasz, ze jest spokojny przybytek twój, i nawiedzisz mieszkanie twoje, a nie zgrzeszysz.
\par 25 Doznasz tez, iz rozmnozone bedzie nasienie twoje, a potomstwo twoje bedzie jako ziele ziemi.
\par 26 Wnijdziesz w sedziwosci do grobu, jako znoszone bywa zboze w stóg czasu swego.
\par 27 Otosmy tego doszli, ze tak jest: sluchajze tego, a uwazaj to sam u siebie.

\chapter{6}

\par 1 I odpowiedzial Ijob, a rzekl:
\par 2 O gdyby pilnie zwazono narzekanie moje, a biede moje pospolu na wage wlozono,
\par 3 Tedyby byla ciezsza nad piasek morski; przetoz mi slów niestaje.
\par 4 Albowiem strzaly Wszechmocnego tkwia we mnie, których jad wysuszyl ducha mego, a strachy Boze walcza przeciwko mnie.
\par 5 Izali osiel dziki ryczy nad trawa? albo wól izali ryczy nad pasza swoja?
\par 6 Izali moze byc jedzona rzecz niesmaczna bez soli? albo jestli jaki smak w bialku jajowym?
\par 7 Czego sie przedtem nie chciala dotknac dusza moja, to teraz jest bolescia ciala mego.
\par 8 Bodajze sie spelnila prosba moja! Niechze mi Bóg da, czego oczekuje!
\par 9 Oby sie Bogu podobalo, zeby mie zniszczyl, a zeby mie wycial, rozpusciwszy reke swoje!
\par 10 Bo mam jeszcze pocieche swoje, (chociaz palam w bolesci, a Bóg mi nie folguje) zem nie tail slów Swietego.
\par 11 Cóz jest za moc moja, abym potrwal? albo co za koniec mój, abym przedluzyl zywota mego?
\par 12 Izali moc kamienna moc moja? albo cialo moje miedziane?
\par 13 Azaz obrony mojej niemasz przy mnie? azaz rozsadek oddalony odemnie?
\par 14 Przeciwko temu, którego litosc slabieje ku blizniemu swemu, i który bojazn Wszechmogacego opuscil?
\par 15 Bracia moi omylili mie jako potok; pomineli jako gwaltowne potoki,
\par 16 Które bywaja metne od lodu, w których sie snieg ukrywa;
\par 17 Czasu którego topnieja, zagina; a czasu goracosci niszczeja z miejsca swego.
\par 18 Udawaja sie tam i sam z dróg swoich; rozciekaja sie po miejscach bezwodnych, i gina.
\par 19 Podrózni ludzie z krainy Teman obaczyli je; a którzy szli do Seba, mieli w nich nadzieje.
\par 20 Ale sie zawstydzili, iz w nich ufali; a gdy tam przyszli, oszukali sie.
\par 21 Tak zaiste i wy, bywszy nie jestescie; widzac utrapienie moje, lekacie sie.
\par 22 Izalim mówil: Przyniescie mi co, a z majetnosci waszej dajcie mi dary?
\par 23 I wybawcie mie z rak nieprzyjaciela, a z rak okrutników odkupcie mie?
\par 24 Nauczciez mie, a ja umilkne; a w czemem zbladzil pokazcie mi.
\par 25 O jakoz sa mocne slowa prawdziwe! Ale cóz sprawi obwinienie wasze?
\par 26 Izali slowa moje obwinic myslicie, a przewiewac mowy utrapionego?
\par 27 I na sierote targacie sie, i kopiecie doly pod przyjacielem swoim.
\par 28 Przetoz przypatrzcie mi sie teraz, a obaczycie, jezli klamie przed obliczem waszem.
\par 29 Obaczcie sie, prosze, a niech nie bedzie w was nieprawosc; obaczcie sie, a poznacie, ze jest sprawiedliwosc moja przy mnie.
\par 30 A iz nie masz w jezyku mym nieprawosci: i nie mamze znac utrapienia mego?

\chapter{7}

\par 1 Izali czas nie jest zamierzony czlowiekowi na ziemi? a jako dni najemnicze nie sa dni jego?
\par 2 Jako sluga pragnie cienia, a jako najemnik czeka konca pracy swojej:
\par 3 Takiem ja prawem dziedzicznem wzial miesiace prózne, a nocy bolesne sa mi naznaczone.
\par 4 Ukladeli sie, tedy mówie: Kiedyz wstane? a rychlo pominie noc? i pelen bywam myslenia az do switania.
\par 5 Obleczone jest cialo moje w robaki i w plugastwo z prochu; skóra moja popadala sie, i rozsiadla sie.
\par 6 Dni moje predsze sa, niz czólnek tkacki, i strawione sa bez nadziei.
\par 7 Wspomnij, o Panie! iz wiatrem jest zywot mój, nie wróci sie oko moje, aby widzialo dobre rzeczy.
\par 8 Ani mie oglada oko, które mie widywalo; oczy twoje obrócone beda na mie, a mnie nie bedzie.
\par 9 Jako niszczeje oblok i przemija, tak zstepujacy do grobu nie wynijdzie;
\par 10 Nie wróci sie wiecej do domu swego, ani go wiecej pozna miejsce jego.
\par 11 Przetoz ja nie moge zawsciagnac ust moich; mówic bede w utrapieniu ducha mego, bede rozmawial w gorzkosci duszy mojej.
\par 12 Izazem ja jest morze, albo wieloryb, zasie mie osadzil straza?
\par 13 Gdym rzekl: Pocieszy mie loze moje, i ulzy mi narzekania mego posciel moja:
\par 14 Tedy mie straszysz przez sny, i przez widzenia trwozysz mna.
\par 15 A przetoz obrala sobie powieszenie dusz moja, a smierc raczej, niz zostac w kosciach.
\par 16 Sprzykrzylem sobie zywot, nie wiecznie bede zyw. Zaniechajze mie, bo marnoscia sa dni moje.
\par 17 Cóz jest czlowiek, ze go tak wielce wazysz? a ze przykladasz ku niemu serce twoje?
\par 18 A ze go nawiedzasz na kazdy zaranek? i na kazda chwile doswiadczasz go?
\par 19 Pokadze sie nie odwrócisz odemnie? a nie zaniechasz mie, azbym przelknal sline moje?
\par 20 Zgrzeszylem, cóz mam czynic? o strózu ludzki! czemus mie sobie za cel polozyl, abym byl sam sobie ciezarem?
\par 21 Przecz nie odejmiesz przestepstwa mego, i nie przepuscisz nieprawosci mojej? Bo sie teraz w prochu poloze, a chocbys mie szukal rano, nie bedzie mie.

\chapter{8}

\par 1 I odpowiedzial Bildad Suhytczyk, a rzekl:
\par 2 Pokadze rzeczy takowe mówic bedziesz? a pokad beda slowa ust twoich jako wiatr gwaltowny?
\par 3 Izazby mial Bóg sad podwrócic? a Wszechmocny mialby sprawiedliwosc wynicowac?
\par 4 Ze synowie twoi zgrzeszyli przeciw niemu, przetoz ich puscil w reke nieprawosci ich.
\par 5 Jezli sie ty wczas nawrócisz do Boga, a bedziesz sie modlil Wszechmocnemu;
\par 6 Jezli bedziesz czystym i szczerym; tedyc pewnie ocuci dla ciebie, i spokojne uczyni mieszkanie sprawiedliwosci twojej.
\par 7 A choc poczatek twój maly bedzie, jednak ostatek twój bardzo sie rozmnozy.
\par 8 Bo spytaj sie prosze wieku starego, a nagotuj sie ku wyszpiegowaniu ojców ich.
\par 9 (Gdyz wczorajszymi jestesmy, a nic nie wiemy, poniewaz jako cien sa dni nasze na ziemi.)
\par 10 Oni cie naucza i powiedzac, i z serca swego wypuszcza slowa.
\par 11 Azaz urosnie sitowie bez wilgotnosci? Izali urosnie rogoza bez wody?
\par 12 Owszem jeszcze w zielonosci swojej, niz bywa podcieta, predzej niz inna trawa usycha.
\par 13 Takiec sa drogi wszystkich, którzy zapominaja Boga; i tak nadzieja obludnika zginie.
\par 14 Podcieta bywa nadzieja jego, a jako dom pajaka ufanie jego.
\par 15 Spolezeli na domu swoim, nie ostoi sie; wesprzeli sie na nim, nie zadzierzy sie.
\par 16 Zieleni sie na sloncu, i w ogrodzie jego swieza latorosl jego wyrasta.
\par 17 Nad ródlem splataja sie korzenie jego, i na miejscu kamienistem rozklada sie.
\par 18 Ale gdy go wytna z miejsca jego, tedy sie go miejsce zaprze, mówiac: Niewidzialem cie.
\par 19 Toc to jest wesele drogi jego, a inny z ziemi wyrosnie.
\par 20 Oto Bóg nie odrzuci czlowieka szczerego, ale zlosnikom nie poda reki.
\par 21 Az sie napelnia smiechem usta twe, a wargi twoje wykrzykaniem.
\par 22 Gdyz, którzy cie maja w nienawisci, obleczeni beda wstydem, a przybytku niepoboznych nie bedzie.

\chapter{9}

\par 1 I odpowiedzial Ijob, a rzekl:
\par 2 Prawdziwiec wiem, ze tak jest; bo jakozby mial byc usprawiedliwiony czlowiek przed Bogiem?
\par 3 Jezliby sie z nim chcial spierac, nie odpowie mu z tysiaca na jedne rzecz.
\par 4 Madry jest sercem, i mocny sila; któz uzyl pokoju, stawiwszy sie mu upornie?
\par 5 On przenosi góry, a nie wiedza ludzie, kto je podwraca w gniewie swym.
\par 6 On wzrusza ziemie z miejsca swego, a slupy jej trzesa sie.
\par 7 Gdy on zakaze sloncu, nie wschodzi; i gwiazdy pieczetuje.
\par 8 On sam rozposciera niebiosa, i depcze po walach morskich.
\par 9 On sprawil wóz niebieski z gwiazd, Oryjona i Hyjady, i inne gwiazdy skryte na poludnie.
\par 10 On czyni rzeczy wielkie, a niewybadane i dziwne, którym niemasz liczby.
\par 11 Oto, idzieli mimo mie, nie widze go; a przychodzili, nie bacze go.
\par 12 Oto gdy co porwie, któz go przymusi, aby przywrócil? Albo któz mu rzecze: Cóz czynisz?
\par 13 Gdyby Bóg nie odwrócil gniewu swego, upadliby przed nim pomocnicy hardzi.
\par 14 Jakoz mu ja tedy odpowiem? Jakie slowa obiore przeciwko niemu?
\par 15 Któremu, chociazbym byl sprawiedliwym, nie odpowiem; owszem sie sedziemu memu upokorze.
\par 16 Chocbym go wzywal, a onby mi sie ozwal, przecie nie wierze, aby przypuscil do uszów glos mój:
\par 17 Bo mie starl w wichrze, i rozmnozyl rany moje bez przyczyny;
\par 18 Nie dopuszcza mi odetchnac, owszem mie nasyca gorzkosciami.
\par 19 Jezli sie udam do mocy, oto on najmocniejszy; a jezli do sadu, któz mie z nim sprowadzi?
\par 20 Jezlibym sie usprawiedliwial, usta moje potepia mie; jezlibym sie doskonalym czynil, tedy mie przewrotnym byc pokaze.
\par 21 Chociazbym byl doskonaly, przeciez ja tego do siebie znac nie bede; ale dam nagane zywotowi memu.
\par 22 Jedno jest, dla czegom to mówil: ze tak doskonalego, jako i niezboznego on niszczy;
\par 23 Jezli biczem nagle zabija, z pokuszenia niewinnych nasmiewa sie;
\par 24 Ziemia podana bywa w rece niezboznika, oblicze sedziów jej zakrywa. A jezliz nie on, któz tedy inny jest, co to czyni?
\par 25 Ale dni moje predsze byly niz posel; uciekly, a nie widzialy nic dobrego.
\par 26 Przeminely jako predkie lodzie, jako orzel lecacy do zeru.
\par 27 Jezli rzeke: Zapomne narzekania mego, zaniecham gniewu swego, a posile sie:
\par 28 Tedy sie lekam wszystkich bolesci moich, widzac, ze mie z nich nie wypuscisz.
\par 29 Jezlim ja niezbozny, przeczze prózno pracuje?
\par 30 A chocbym sie umywal wodami snieznemi, i oczyscilbym mydlem rece moje:
\par 31 Wszakze w dole zanurzysz mie, i brzydzic sie mna beda szaty moje.
\par 32 Albowiem on nie jest czlowiekiem jako ja, abym mu smial odpowiedziec, albo zebym z nim mial isc w prawo.
\par 33 Bo nie masz miedzy nami rozjemcy, któryby mógl rozwiesc sprawe nasze.
\par 34 Niech tylko zdejmie zemnie rózge swoje, a strach jego niech mie nie straszy;
\par 35 Tedy bede mówil, a nie bede sie go bal; bom ja nie jest taki sam u siebie.

\chapter{10}

\par 1 Teskni sobie dusza moja w zywocie moim; rozpuszcze przeciw sobie narzekanie moje, a bede mówil w gorzkosci duszy mojej.
\par 2 Rzeke Bogu: Nie potepiajze mie; raczej mi oznajmij, czemu spór ze mna wiedziesz?
\par 3 Cóz masz za pozytek, ze mie uciskasz? a iz odrzucasz sprawe rak twoich? a rade niepoboznych oswiecasz?
\par 4 Azaz ty masz oczy cielesne? Albo jako czlowiek widzi, ty widzisz?
\par 5 Dni twoje, zaz sa jako dni czlowiecze? a lata twoje jako lata ludzkie?
\par 6 Iz sie wywiadujesz nieprawosci mojej, a o grzechu moim badasz sie?
\par 7 Ty wiesz, zem niepoboznie nie poczynal; wszakze nie jest, ktoby mie mial wyrwac z rak twoich.
\par 8 Rece twoje wyksztaltowaly mie, i uczynily mie; a przecie mie zewszad gubisz.
\par 9 Pomnij prosze, zes mie jako gline ulepil, a w proch mie zas obrócisz.
\par 10 Izali jako mleko nie zlales mie, a jako ser nie utworzyles mie?
\par 11 Skóra i cialem przyoblokles mie, a kosciami i zylami pospinales mie.
\par 12 Zywotem i milosierdziem darowales mie, a opatrznosc twoja strzegla ducha mego.
\par 13 A chociazes to skryl w sercu twojem, wiem jednak, ze to jest z woli twojej.
\par 14 Jezli zgrzesze, wnet tego postrzezesz, a dla nieprawosci mojej nie przepuscisz mi.
\par 15 Jezlim bezbozny, biada mi! a chocbym tez byl sprawiedliwym, nie podniose glowy mojej, bedac nasycony pohanbieniem, i widzac utrapienie moje,
\par 16 Którego przybywa; bo jako lew srogi gonisz mie, a coraz dziwniej sie przeciwko mnie stawiasz.
\par 17 Odnawiasz swiadków twoich przeciwko mnie, a rozmnazasz rozgniewanie twoje na mie; wojska jedne po drugich sa przeciwko mnie.
\par 18 Przeczzes mie z zywota wywiódl? Ach, bym byl umarl, zeby mie bylo oko nie widzialo!
\par 19 Obym byl, jakoby mie nie bylo! oby mie bylo zaraz z zywota do grobu zaniesiono!
\par 20 Izaz nie trocha dni moich? Przetoz przestan, a zaniechaj mie, abym sie troszeczke posilil,
\par 21 Pierwej niz odejde tam, skad sie nie wróce, do ziemi ciemnosci, i do cienia smierci;
\par 22 Do ziemi ciemnej, jako chmura, i do cienia smierci, gdzie niemasz przemiany, jedno sama gesta ciemnosc.

\chapter{11}

\par 1 I odpowiedzial Sofar Naamatczyk, i rzekl:
\par 2 Izaz sie nie godzi na wiele slów odpowiedziec? Albo izali maz wielomowny bedzie usprawiedliwiony?
\par 3 Bedaz na twoje plotki ludzie milczec? A gdy ty sobie przeszydzasz, ciebie nikt nie zawstydzi?
\par 4 Albowiemes powiedzial: Czysta jest nauka moja, a jestem czystym przed oczyma twemi.
\par 5 Ale gdyby Bóg chcial mówic, i otworzyc usta swoje przeciwko tobie:
\par 6 Tedycby objawil tajemnice madrosci, zes dwa kroc wieksze karanie nadto zasluzyl; przetoz uznaj, ze cie Bóg przebaczyl dla nieprawosci twojej.
\par 7 Izali tajemnice Boze wybadasz? albo doskonalosci Wszechmocnego doscigniesz?
\par 8 Wyzsze sa niz niebiosa, cóz uczynisz? Glebsze niz pieklo, jakoz poznasz?
\par 9 Dluzsza miara ich, niz ziemia, a szersza, niz morze.
\par 10 Jezli wypelni, albo jezli zawrze, albo jezli w jedno scisnie, któz go zawsciagnie?
\par 11 Albowiem on zna marnosc ludzka, i widzi nieprawosc; a nie mialby tego baczyc?
\par 12 Czlowiek nierozumny nabywa rozumu, choc sie jako zrebie lesnego osla rodzi czlowiek.
\par 13 Jezli ty przygotujesz serce twoje, a wyciagniesz do niego rece twoje;
\par 14 Jezliz nieprawosc jest w rece twej oddal ja, a mieszkac nie dopuszczaj nieprawosci w przybytkach twoich;
\par 15 Tedy podniesiesz oblicze twoje bez zmazy, a bedziesz staly, i nie bedziesz sie bal.
\par 16 Albowiem zapomnisz klopotu, a jako wody, które pominely, wspominac go bedziesz.
\par 17 I nad poludnie jasniejszy nastanie czas twój; zacmiszli sie, bedziesz jako zaranek.
\par 18 I bedziesz ufal, majac nadzieje, a jako w okopach bezpiecznie spac bedziesz.
\par 19 Bedziesz lezal, a nikt cie nie przestraszy; i unizac sie beda przed twarza twoja wiele ich.
\par 20 Ale oczy niepoboznych ustana i ucieczka ich zginie, a nadzieja ich bedzie jako wyjscie duszy z czlowieka.

\chapter{12}

\par 1 Zatem odpowiedzial Ijob i, rzekl:
\par 2 Wierascie wy sami ludzmi? i z wamiz umrze madrosc?
\par 3 Tezci ja mam serce jako i wy, anim jest podlejszym nizeli wy; a któz i tego nie wie, co i wy?
\par 4 Posmiewiskiem jestem przyjacielowi memu, który gdy wola do Boga, ozywa mu sie; nasmiewiskiem jest sprawiedliwy i doskonaly.
\par 5 Ten, co jest upadku bliski, jest pochodnia wzgardzona czlowiekowi, wedlug mysli pokoju zazywajacemu.
\par 6 Spokojne i bezpieczne sa namioty zbójców tych, którzy draznia Boga, którym Bóg daje w rece dobre rzeczy.
\par 7 A nawet pytaj sie prosze bydlat, a one cie naucza; i ptastwa niebieskiego, a oznajmi tobie.
\par 8 Albo sie rozmów z ziemia, a ona cie nauczy, i rozpowiedzac ryby morskie.
\par 9 Któz nie wie z tych wszystkich rzeczy, ze to reka Panska sprawila?
\par 10 W którego reku jest dusza wszelkiej rzeczy zywej, i duch wszelkiego ciala ludzkiego.
\par 11 Azaz nie ucho mowy doswiadcza, jako usta pokarmu smakuja?
\par 12 W ludziach starych jest madrosc, a w dlugich dniach roztropnosc.
\par 13 Dopieroz u Pana jest madrosc, i sila, i rada, i umiejetnosc.
\par 14 Oto on burzy, a nikt nie zbuduje; zamknie czlowieka, a nikt mu nie otworzy.
\par 15 On gdy zatrzyma wody, wyschna; a gdy je wypusci, podwracaja ziemie.
\par 16 U niego jest moc i madrosc. Jego jest bladzacy, i w blad zawodzacy.
\par 17 On obiera radców z madrosci, a sedziów przywodzi do glupstwa.
\par 18 On pas królów rozwiazuje, i znowu przepasuje pasem biodra ich.
\par 19 Podaje ksiazeta na lup, a mocarze podwraca.
\par 20 Odejmuje usta krasomówcom, a rozsadek starym odbiera.
\par 21 Wylewa wzgarde na ksiazeta, a mdli sily mocarzów.
\par 22 On odkrywa glebokie rzeczy z ciemnosci, a wywodzi na jasnie cien smierci.
\par 23 Rozmnaza narody, i wytraca je; rozszerza lud, i umniejsza go.
\par 24 On odejmuje serca przelozonym ludu ziemi, a czyni, ze bladza po pustyni bezdroznej;
\par 25 Ze macaja w ciemnosciach, gdzie nie masz swiatlosci, a sprawuje, ze bladza jako pijani.

\chapter{13}

\par 1 Oto te wszystkie rzeczy widzialo oko moje, slyszalo ucho moje, i zrozumialo.
\par 2 Jako wy to wiecie, tak ja tez wiem, i nie jestem podlejszym nizli wy.
\par 3 Wszakze radbym z Wszechmocnym mówil, i radbym sie z Bogiem rozpieral.
\par 4 Boscie wy sprawcy klamstwa: wszyscyscie wy lekarze nikczemni.
\par 5 Byscie wy raczej milczeli, a poczytanoby wam to za madrosc.
\par 6 Sluchajciez teraz odporu mego, a dowody ust moich obaczcie.
\par 7 Izali broniac Boga mówic bedziecie nieprawosc? albo za nim mówic bedziecie falsz?
\par 8 Czy sie na osobe jego ogladac bedziecie? Czy sie o Boga bedziecie spierac?
\par 9 Zaz to dobrze bedzie, gdy on was bedzie próbowal? Zaz, jako czlowiek oszukany bywa, tak wy go oszukacie?
\par 10 Zaiste karac was bedzie, jezlibyscie skrycie twarz jego przyjmowali.
\par 11 Izali zacnosc jego nie ustraszy was? a strach jego nie przypadnie na was?
\par 12 Pamiatki wasze podobne sa popiolowi, a wynioslosc wasza kupie blota.
\par 13 Milczciez, zaniechajcie mie, a ja mówic bede; a niech przyjdzie na mie, co chce.
\par 14 Czemuz mam szarpac cialo moje zebami mojemi, i dusze moje klasc w rece swe?
\par 15 Oto, chocby mie i zabil, przecie w nim bede ufal; wszakze dróg moich przed obliczem jego bede bronil.
\par 16 Onci sam bedzie zbawieniem mojem, ale przed oblicze jego obludnik nie przyjdzie;
\par 17 Sluchajciez z pilnoscia mowy mojej, a powiesc moja niech przyjdzie w uszy wasze.
\par 18 Oto sie teraz gotuje do prawa, i wiem, ze usprawiedliwiony bede.
\par 19 Któz sie bedzie spieral ze mna, tak abym umilknal i umarl?
\par 20 Tylko dwóch rzeczy, o Boze! nie czyn ze mna, przed oblicznoscia twoja nie skryje sie.
\par 21 Reke twoje odemnie oddal, a strach twój niech mna nie trwozy.
\par 22 Potem zawolaj mie, a ja tobie odpowiem; albo ja niech mówie, a ty mnie odpowiedz.
\par 23 Wielez jest nieprawosci i grzechów moich? przestepstwo moje, grzech mój pokaz mi.
\par 24 Przeczze oblicze twoje zakrywasz, a poczytasz mie sobie za nieprzyjaciela?
\par 25 Izali skruszysz lisc chwiejacy sie? a zdzblo suche gonic bedziesz?
\par 26 Albowiem piszesz przeciwko mnie gorzkosci, a przywlaszczasz mi nieprawosc mlodosci mojej;
\par 27 I wlozyles w peta nogi moje, a podstrzegasz wszystkich sciezek moich, i na slad nóg moich nastepujesz.
\par 28 Choc jako spróchniale drzewo niszczeje; a jako szata, która mól psuje.

\chapter{14}

\par 1 Czlowiek, narodzony z niewiasty, dni krótkich jest, i pelen klopotów;
\par 2 Wyrasta jako kwiat, i bywa podciety, a ucieka jako cien, i nie ostoi sie.
\par 3 Wszakze i na takiego otwierasz oczy twoje, a przywodzisz mie do sadu z soba.
\par 4 Któz pokaze czystego z nieczystego? Ani jeden;
\par 5 Gdyz zamierzone sa dni jego, liczba miesiecy jego u ciebie; zamierzyles mu kres, którego nie moze przestapic.
\par 6 Odstapze od niego, az odpocznie, az przejdzie jako najemniczy dzien jego.
\par 7 Albowiem i o drzewie jest nadzieja, choc je wytna, ze sie jeszcze odmlodzi, a latorosl jego nie ustanie.
\par 8 Choc sie zstarzeje w ziemi korzen jego, i w prochu obumrze pien jego:
\par 9 Wszakze gdy uczuje wilgotnosc, pusci sie, i rozpusci galezie, jako szczep mlody.
\par 10 Ale czlowiek umiera, zemdlony bedac, a umarlszy czlowiek gdziez jest?
\par 11 Jako uchodza wody z morza, a rzeka opada i wysycha.
\par 12 Tak czlowiek, gdy sie ukladzie, nie wstanie wiecej, a pokad stoja nieba, nie ocuci sie, ani bedzie obudzony ze snu swego.
\par 13 Obyzes mie w grobie ukryl i utail, azby sie uciszyl gniew twój, a izbys mi zamierzyl kres, kedy chcesz wspomniec na mie!
\par 14 Gdy umrze czlowiek, izali zyc bedzie? Po wszystkie dni wymierzonego czasu mego bede oczekiwal przyszlej odmiany mojej.
\par 15 Zawolasz, a ja tobie odpowiem; a spraw rak twoich pozadasz.
\par 16 Aczkolwiekes teraz kroki moje obliczyl, ani odwlóczysz karania za grzech mój.
\par 17 Zapieczetowane jest w wiazance przestepstwo moje, a zgromadzasz nieprawosci moje.
\par 18 Prawdziwie jako góra padlszy rozsypuje sie, a skala przenosi sie z miejsca swego.
\par 19 Jako woda wzdraza kamienie, a powodzia zalane bywa, co samo od siebie rosnie z prochu ziemi: tak nadzieje ludzka w niwecz obracasz.
\par 20 Przemagasz go ustawicznie, a on schodzi; odmieniasz postac jego, i wypuszczasz go.
\par 21 Bedali zacni synowie jego, tego on nie wie; jezli tez wzgardzeni, on nie baczy.
\par 22 Tylko cialo jego, póki zyw, boleje, a dusza jego w nim kwili.

\chapter{15}

\par 1 A odpowiadajac Elifas Temanczyk rzekl:
\par 2 Izali madry ma na wiatr mówic? albo napelniac wschodnim wiatrem mysl swoje?
\par 3 Przytaczajac slowa niepozyteczne, i mowy, z których nie masz pozytku?
\par 4 Zaiste ty psujesz bojazn Boza i znosisz modlitwy do Boga.
\par 5 Albowiem pokazuja nieprawosc twa usta twoje, chociazes sobie obral jezyk chytrych,
\par 6 Potepiaja cie usta twoje, a nie ja; a wargi twoje swiadcza przeciwko tobie.
\par 7 Czys sie najpierwszym czlowiekiem urodzil? czys przed pagórkami utworzony?
\par 8 Izazes tajemnic Bozych sluchal, a nie masz madrosci jedno w tobie?
\par 9 Cóz ty umiesz, czego my nie wiemy? albo cóz ty rozumiesz, czegobysmy my nie rozumieli?
\par 10 I sedziwyc i starzec miedzy nami jest starszy w latach niz ojciec twój.
\par 11 I lekcez sobie wazysz pociechy Boskie? i maszze jeszcze co tak skrytego w sobie?
\par 12 Czemuz cie tak unioslo serce twoje? Czemu mrugaja oczy twoje?
\par 13 Ze tak odpowiada Bogu duch twój, a wypuszczasz z ust twoich takowe mowy?
\par 14 Cóz jest czlowiek, aby mial byc czystym, albo zeby mial byc sprawiedliwym, urodzony z niewiasty?
\par 15 Oto i w swietych jego niemasz doskonalosci, i niebiosa nie sa czyste w oczach jego.
\par 16 Daleko wiecej obrzydly jest, i nieuzyteczny czlowiek, który pije nieprawosc jako wode.
\par 17 Okazec, tylko mie sluchaj; a com widzial, oznajmiec,
\par 18 Co medrzy powiedzieli, a nie zataili, co mieli od przodków swoich;
\par 19 Którym samym dana byla ziemia, a zaden obcy nie przeszedl przez nie.
\par 20 Po wszystkie dni swoje sam siebie niepobozny bolesnie trapi, a nie wiele lat zamierzono okrutnikowi.
\par 21 Glos straszliwy brzmi w uszach jego, ze czasu pokoju pustoszacy przypadnie nan.
\par 22 Nie wierzy, zeby sie mial nawrócic z ciemnosci, obawiajac sie zewszad miecza.
\par 23 Tula sie za chlebem, szukajac gdzieby byl; wie, ze zgotowany jest dla niego dzien ciemnosci.
\par 24 Strasza go utrapienie i ucisk, i zmocnia sie przeciwko niemu jako król gotowy do boju.
\par 25 Bo wyciagna przeciw Bogu reke swa, a przeciwko Wszechmocnemu zmocnil sie.
\par 26 Natrze nan na szyje jego z gestemi i wynioslemi tarczami swemi.
\par 27 Bo okryl twarz swa tlustoscia swoja, a faldów mu sie naczynilo na slabiznie.
\par 28 I mieszka w miastach popustoszonych, i w domach, w których nie mieszkano, które sie mialy obrócic w kupe rumu.
\par 29 Nie zbogaci sie, i nie ostoi sie majetnosc jego, ani sie rozszerzy na ziemi doskonalosc takowych.
\par 30 Nie wynijdzie z ciemnosci; swieza jego latorosl ususzy plomien, a zginie od ducha ust jego.
\par 31 Nie wierzy, ze w próznosci jest, który bladzi; a ze próznosc bedzie nagroda jego.
\par 32 Przed wypelnieniem dni swoich wyciety bedzie, a rózdzka jego nie zakwitnie.
\par 33 Jako winna macica utraci niedojrzale grona swoje, a jako oliwa kwiat swój zrzuci.
\par 34 Albowiem zgromadzenie obludnych spustoszone bedzie, a ogien pozre przybytki pobudowane za dary.
\par 35 Poczeli klopot, a porodzili nieprawosc; a zywot ich gotuje zdrade.

\chapter{16}

\par 1 Ale odpowiedzial Ijob, i rzekl:
\par 2 Slyszalem takowych rzeczy wiele; przykrymi cieszycielami wy wszyscy jestescie.
\par 3 I kiedyz bedzie koniec tym próznym slowom? albo co cie przymusza, ze tak odpowiadasz?
\par 4 Azazbym ja tak mówil, jako wy, gdybyscie wy byli na miejscu mojem? azazbym zbieral przeciwko wam slowa, i kiwallibym nad wami glowa swoja?
\par 5 Owszembym was posilal ustami memi, a ruchanie warg moich ulzyloby bolesci waszych.
\par 6 Ale jezli bede mówil, przeciez sie nie ukoi bolesc moja; a jezli tez przestane, izaz odejdzie odemnie?
\par 7 A teraz zemdlil mie; spustoszyles, o Boze! wszystko zgromadzenie moje.
\par 8 Pomarszczyles mie na swiadectwo, a znaczne na mnie schudzenie moje na twarzy mojej, jawnie swiadczy przeciwko mnie.
\par 9 Popedliwosc jego porwala mie, i wzial nienawisc przeciwko mnie; a zgrzytajac na mie zebami swemi, jako nieprzyjaciel mój, bystremi oczyma swemi spojrzal na mie.
\par 10 Rozdzieraja na mie usta swe, i sromotnie mie policzkowali, zebrawszy sie spolu przeciwko mnie.
\par 11 Podal mie Bóg przewrotnemu, a w rece niepoboznych wydal mie.
\par 12 Bylem w pokoju, ale mie potarl; a uchwyciwszy mie za szyje moje, roztrzaskal mie, i wystawil mie sobie za cel.
\par 13 Ogarneli mie strzelcy jego; rozcial nerki moje, a nie przepuscil, i rozlal na ziemie zólc moje.
\par 14 Zranil mie, rana na rane; rzucil sie na mie, jako olbrzym.
\par 15 Uszylem wór na zsiniala skóre moje, a oszpecilem prochem glowe moje.
\par 16 Twarz moja placzem oszpecona, a na powiekach moich jest cien smierci.
\par 17 Chociaz zadnego lupiestwa niemasz w rekach moich, a modlitwa moja jest czysta. (a jezli nie tak,)
\par 18 O ziemio! nie zakrywajze krwi mojej, a niech nie ma miejsca wolanie moje!
\par 19 Otoc i teraz w niebie jest swiadek mój, jest swiadek mój na wysokosci.
\par 20 O krasomówcy moi, przyjaciele moi! wylewa lzy do Boga oko moje.
\par 21 Oby sie godzilo wiesc spór czlowiekowi z Bogiem, i jako synowi czlowieczemu z bliznim swym!
\par 22 Bo lata zamierzone nadchodza, a scieszka, która sie nie wróce, juz ida.

\chapter{17}

\par 1 Dech mój skazony jest; dni moje gina; groby mie czekaja.
\par 2 Zaiste nasmiewcy sa przy mnie, a w ich draznieniu mieszka oko moje.
\par 3 Staw mi, prosze, rekojmie za sie. Któz jest ten? Niech mi na to da reke.
\par 4 Bos serce ich ukryl przed wyrozumieniem; przetoz ich nie wywyzszysz.
\par 5 Kto pochlebia przyjaciolom, oczy synów jego ustana.
\par 6 Wystawil mie zaiste na przypowiesc ludziom, i jako smiechowisko przed nimi.
\par 7 Zacmione jest dla zalosci oko moje, a wszystkie mysli moje sa jako cien.
\par 8 Zdumieja sie szczerzy nad tem; a niewinny przeciwko obludnikowi powstanie.
\par 9 Bedzie sie trzymal sprawiedliwy drogi swojej; a kto ma czyste rece, przyczyni mocy.
\par 10 Wy tedy wszyscy nawróccie sie, a pójdzcie, prosze; bo nie znajduje miedzy wami madrego.
\par 11 Dni moje przeminely; mysli moje rozerwane sa, to jest, zamysly serca mego.
\par 12 Noc mi sie w dzien obraca; a swiatlosc skraca sie dla ciemosci.
\par 13 Jezlibym czego oczekiwal, grób bedzie domem moim, a w ciemnosciach usciele loze moje.
\par 14 Do dolu rzeke: Ojcem moim jestes; a do robaków: Wy jestescie matka moja, i siostra moja.
\par 15 Bo gdziez teraz jest nadzieja moja? a oczekiwanie moje któz oglada?
\par 16 W glebie grobu zstapie, poniewaz w prochu spólny odpoczynek wszystkich.

\chapter{18}

\par 1 A odpowiadajac Bildad Suhytczyk rzekl:
\par 2 Dokadze nie uczynicie konca mowom? pomyslcie pierwej, a potem mówic bedziemy.
\par 3 Czemuz nas poczytaja jako bydlo? zdajemy sie mu przemierzlymi, jako sami widzicie.
\par 4 Ty, który dusze twoje tracisz w zapalczywosci twojej, azaz dla ciebie bedzie opuszczona ziemia, a beda przeniesione skaly z miejsca swego?
\par 5 Owszem, swiatlosc niepoboznych zgasnie, i nie bedzie swiecila iskra ognia ich.
\par 6 Swiatlo sie zacmi w przybytku jego, i pochodnia jego nad nim zgasnie.
\par 7 Scisnione beda kroki sily jego, a porazi go rada jego.
\par 8 Bo zawioda w sieci nogi jego, i w uwiklaniu chodzic bedzie.
\par 9 Uchwyci go sidlo za piete jego, i przemoze go lupiezca.
\par 10 Skryty jest w ziemi powróz jego, a samolówka jego na scieszce.
\par 11 Zewszad go straszyc beda strachy, a nacierac beda na nogi jego.
\par 12 Wymorzy sie glodem sila jego, a zginienie pogotowiu jest przy boku jego.
\par 13 Pozre zyly skóry jego, pozre czlonki jego pierworodny smierci.
\par 14 Ufanie jego bedzie wykorzenione z przybytku jego, a przywiedzie go do króla strachów.
\par 15 Bedzie mieszkal strach w przybytku jego, chociaz nie byl jego, a siarka bedzie potrzasnione mieszkanie jego.
\par 16 Ze spodku korzen jego uschnie, a z wierzchu bedzie obcieta Gala? jego.
\par 17 Pamiatka jego zginie z ziemi, a imienia jego nie wspomna po ulicach.
\par 18 Wypedza go z swiatlosci do ciemnosci, a z okregu swiata wyrzuca go.
\par 19 Nie bedzie syn ani wnuk miedzy ludem jego, i nikt nie pozostanie w mieszkaniach jego.
\par 20 Nade dniem jego zdumiewaja sie potomkowie, a przodków ogarnie strach.
\par 21 Takowec sa mieszkania niezboznego, i do tego przychodzi temu, który nie zna Boga.

\chapter{19}

\par 1 A odpowiadajac Ijob rzekl:
\par 2 Dokadze trapic bedziecie dusze moje, a nacierac na mie mowami swemi?
\par 3 Juz dziesieckroc zawstydziliscie mie, i nie wstydze was, ze sie tak zatwardzacie przeciwko mnie?
\par 4 A niech tak bedzie, zem zbladzil; przy mnie zostanie blad mój.
\par 5 A jezli sie przeciw mnie wynosicie, a obwiniacie mie pohanbieniem mojem,
\par 6 Wiedzciez, zec mie Bóg odwrócil, i siecia swoja obtoczyl mie.
\par 7 Oto, wolamli o krzywde, nie bywam wysluchany; krzyczeli, niemasz sadu.
\par 8 Droge moje zagrodzil, zebym przejsc nie mógl, a na scieszce mojej ciemnosci polozyl.
\par 9 Z slawy mojej zlupil mie, i zdjal korone z glowy mojej.
\par 10 Popsul mie zewszad, abym zaginal, a wyrwal jako drzewo nadzieje moje.
\par 11 Nadto zapalil sie na mie gniew jego, a policzyl mie w poczet nieprzyjaciól swoich.
\par 12 Przyszly razem hufy jego, i utorowaly przeciwko mnie droge swoje, i oblegly w okolo namiot mój.
\par 13 Braci moich odemnie oddalil, a znajomi moi stronia odemnie.
\par 14 Opuscili mie bliscy moi, a znajomi moi zapomnieli mie.
\par 15 Komornicy domu mego, i sluzebnice moje maja mie za obcego, cudzoziemcem stalem sie w oczach ich.
\par 16 Wolamli na sluge mego, nie ozywa mi sie, chociaz go prosze ustami memi.
\par 17 Tchem moim brzydzi sie zona moja, choc prosze przez synów zywota mego.
\par 18 I najlichsi pogardzaja mna, a gdy powstaje, uragaja mi.
\par 19 Brzydza sie mna wszyscy najwierniejsi moi, a którychem umilowal, stali mi sie przeciwnymi.
\par 20 Do skóry mojej, jako do ciala mego przyschla kosc moja; skóra tylko zostala okolo zebów moich.
\par 21 Zmilujcie sie nademna, zmilujcie sie nademna, wy przyjaciele moi! bo reka Boza dotknela mie.
\par 22 Czemuz mie przesladujecie, jako Bóg, a ciala mego nie mozecie sie nasycic?
\par 23 Oby teraz napisane byly slowa moje! oby je na ksiegach wyrysowano!
\par 24 Oby rylcem zelaznym i olowiem na wieczna pamiatke na kamieniu wydrazone byly!
\par 25 Aczci ja wiem, iz Odkupiciel mój zyje, a iz w ostateczny dzien nad prochem stanie.
\par 26 A choc ta skóra moja roztoczona bedzie, przeciez w ciele mojem ogladam Boga;
\par 27 Którego ja sam ogladam, i oczy moje ujrza go, a nie inny; choc zniszczaly nerki moje we wnetrznosciach moich.
\par 28 Przeczze nie mówicie: Czemuz go przesladujemy? gdyz sie przy mnie znajduje grunt dobrej sprawy.
\par 29 Uleknijcie sie sami miecza, bo pomsta nieprawosci jest miecz; a wiedzcie, ze bedzie sad.

\chapter{20}

\par 1 A odpowiadajac Sofar Naamatczyk rzekl:
\par 2 Do tego mie mysli moje przywodza, abym odpowiedzial; przetozem sie pospieszyl.
\par 3 Slyszalem mnie hanbiaca nagane; ale duch wyrozumienia mego odpowie za mie.
\par 4 Izaz nie wiesz, ze to jest od wieku, od tego czasu, jako postawil Bóg czlowieka na ziemi?
\par 5 Iz chwala niepoboznych krótka jest, a wesele obludnika na mgnienie oka?
\par 6 By tez wstapila az do nieba hardosc jego, a obloku sie dotknela glowa jego:
\par 7 A wszakze na wieki zginie jako gnój jego, a ci, którzy go widzieli, rzeka: Gdziez sie podzial?
\par 8 Uleci jako sen, a nie znajda go; bo uciecze, jako widzenie nocne.
\par 9 Oko, które go widzialo, nie oglada go wiecej, i nie ujrzy go wiecej miejsce jego.
\par 10 Synowie jego beda sie korzyc ubogim; bo rece jego musza wracac, co wydarl.
\par 11 Kosci jego napelnione sa grzechami mlodosci jego, a w prochu z nim lezec beda.
\par 12 A choc zlosc slodnieje w ustach jego, i tai ja pod jezykiem swoim;
\par 13 Kocha sie w niej, a nie opuszcza jej, zatrzymywajac ja w posrodku podniebienia swego:
\par 14 Wszakze pokarm jego we wnetrznosciach jego odmieni sie; zólcia padalcowa stanie sie w trzewach jego.
\par 15 Bogactwa, które pozarl, zwróci, a z brzucha jego wyzenie je Bóg.
\par 16 Glowe padalcowa ssac bedzie; zabije go jezyk jaszczurczy.
\par 17 Nie oglada zródel rzek, strumieni mówie miodu i masla.
\par 18 Wróci prace cudza, a nie zazyje jej; i choc znowu nabedzie wielkich majetnosci, nie ucieszy sie niemi.
\par 19 Bo ubogich dreczyl i opuszczal, zlupil dom, którego nie budowal; przetoz nic spokojnego nie poczuje w zywocie swoim,
\par 20 Ani rzeczy swych wdziecznych nie bedzie mógl zatrzymac.
\par 21 Nic nie zostanie z pokarmów jego, ani sie rozmnozy dobro jego.
\par 22 Chocby i nazbyt mial wszystkiego, scisniony bedzie; wszelka reka trapiacych oburzy sie nan.
\par 23 Choc bedzie mial czem napelnic brzuch swój, przeciez nan Bóg pusci popedliwosc gniewu swego, która jako deszcz spusci nan, i na pokarmy jego.
\par 24 Gdy uciekac bedzie przed bronia zelazna, przebije go luk hartowny.
\par 25 Wyjeta bedzie strzala z sajdaku wypuszczona, a grot przeniknie zólc jego; a gdy uchodzic bedzie, ogarna go strachy.
\par 26 Wszystkie nieszczescia zasadzily sie nan w tajemnych miejscach jego, a pozre go ogien nierozdymany: pozostaly w przybytku jego utrapiony bedzie.
\par 27 Odkryja niebiosa zlosc jego, a ziemia powstanie przeciwko niemu.
\par 28 Przeniesie sie urodzaj domu jego; dobra jego rozplyna sie w dzien gniewu jego.
\par 29 Tenci jest dzial czlowieka niepoboznego od Boga, to dziedzictwo naznaczone mu od Boga.

\chapter{21}

\par 1 A odpowiadajac Ijob rzekl:
\par 2 Sluchajciez z pilnoscia slów moich, a bedzie mi to od was pociecha.
\par 3 Znoscie mie, a ja bede mówil; a gdy domówie, nasmiewajcie sie.
\par 4 Izaz do czlowieka obracam narzekanie moje? a poniewaz mam o co, jakoz sie niema trapic duch mój?
\par 5 Wejrzyjciez na mie, a zdumiewajcie sie, a polózcie reke na usta wasze.
\par 6 Bo co sobie wspomne, tedy sie lekam, a strach zdejmuje cialo moje.
\par 7 Przeczze niepobozni zyja, starzeja sie, i wzmagaja sie w bogactwa?
\par 8 Nasienie ich trwale jest przed obliczem ich z nimi, a rodzina ich przed oczyma ich.
\par 9 Domy ich bezpieczne od strachu, a niemasz rózgi Bozej nad nimi.
\par 10 Byk ich przypuszczon bywa, a nie traci nasienia; krowa ich rodzi, a nie pomiata.
\par 11 Wypuszczaja maluczkie dziatki swoje jako trzode, a synowie ich wyskakuja.
\par 12 Wykrzykaja przy bebnie i przy harfie, a wesela sie przy glosie muzyki.
\par 13 Trawia w dobrem dni swoje, a we mgnieniu oka do grobu zstepuja.
\par 14 Którzy mawiaja Bogu: Odejdz od nas; bo dróg twoich znac nie chcemy.
\par 15 Któz jest Wszechmocny, abysmy mu sluzyli? a cóz nam to pomoze, chocbysmy mu sie modlili?
\par 16 Ale oto, dobra ich nie sa w rekach ich; przetoz rada niepoboznych daleka jest odemnie.
\par 17 Czestoz pochodnia niepoboznych gasnie? a zginienie ich przychodzi na nich? Oddziela im Bóg bolesci w gniewie swoim.
\par 18 Stawaja sie jako plewa przed wiatrem, i jako perz, który wicher porywa.
\par 19 Bo Bóg chowa synom jego pomste jego; nadgradza mu, aby to poczul.
\par 20 Ogladaja oczy jego nieszczescie swoje, a z popedliwosci Wszechmocnego pic bedzie.
\par 21 Co za staranie jego o domu jego po nim, gdyz liczba miesiecy jego umniejszona jest?
\par 22 Izali Boga kto nauczy umiejetnosci, gdyz on wysokich sadzi?
\par 23 Ten umiera w doskonalej sile swojej, gdy zewszad bezpieczny i spokojny jest;
\par 24 Gdy piersi jego pelne sa mleka, a szpik kosci jego odwilza sie,
\par 25 Inny zas umiera w gorzkosci ducha, który nie jadal z uciecha.
\par 26 Spólnie w prochu lezec beda, a robaki ich okryja.
\par 27 Oto ja znam mysli wasze i zamysly, które przeciwko mnie zlosliwie zmyslacie.
\par 28 Bo mówicie: Gdziez jest dom ksiazecy? gdzie namiot przybytków niepoboznych?
\par 29 Izaliscie nie pytali podróznych? a znaków ich izali znac nie chcecie?
\par 30 Ze w dzien zatracenia zly zachowany bywa, w dzien, którego gniew przywiedziony bywa.
\par 31 Któz mu oznajmi w oczy droge jego? a to, co czynil, kto mu odplaci?
\par 32 Wszakze i on do grobów zaprowadzony bedzie, a w kupie umarlych zawzdy zostanie.
\par 33 Slodnieja mu bryly grobowe, i ciagnie za soba wszystkich ludzi; a tych, którzy go poprzedzili, niemasz liczby.
\par 34 Jakoz mie tedy prózno cieszycie, gdyz w odpowiedziach waszych zostaje klamstwo?

\chapter{22}

\par 1 A odpowiadajac Elifas Temanczyk rzekl:
\par 2 Izali Bogu czlowiek moze byc pozytecznym? raczej pozyteczny jest sam sobie, madrze sie sprawujac.
\par 3 Izali sie kocha Wszechmogacy w tem, ze sie ty usprawiedliwiasz? albo co za zysk ma, gdy doskonale pokazujesz drogi twoje?
\par 4 Aza cie bedzie karal bojac sie ciebie? albo z toba pójdzie do sadu?
\par 5 Azaz zlosc twoja nie jest wielka, i niemasz konca nieprawosciom twoim?
\par 6 Albowiemes pobieral zastaw od braci twoich bez przyczyny, a z szat odzierales nagich.
\par 7 Wodys spracowanemu nie podal, a glodnemu odmówiles chleba.
\par 8 Ale czlowiekowi moznemu dales ziemie, a ten, który byl w powadze, mieszkal w niej.
\par 9 Wdowy puszczales prózne, a sierót ramiona potarles.
\par 10 A przetoz ogarnely cie sidla, a trwozy cie strach nagly.
\par 11 Albo cie ogarnely ciemnosci, iz nie widzisz? a wielkosci wód okryly cie.
\par 12 Mówisz: Izali Bóg nie jest na wysokosci niebios? Spojrzyj prosze na wierzch gwiazd, jako sa wysokie.
\par 13 Przetoz mówisz: A cóz wie Bóg? izaz przez chmury sadzic bedzie?
\par 14 Obloki sa skrytoscia jego, iz nie widzi, a po okregu niebieskim przechadza sie.
\par 15 Izaz scieszki wieku przeszlego nie baczysz, która deptali ludzie zlosliwi?
\par 16 Którzy sa wykorzenieni przed czasem, a powodzia zalaly sie grunty ich.
\par 17 Którzy mawiali Bogu: Odejdz od nas; cózby im uczynil Wszechmogacy?
\par 18 Gdyz on byl napelnil dobrem domy ich; (ale rada niepoboznych daleka jest odemnie.)
\par 19 Co widzac sprawiedliwi, weselili sie, a niewinny nasmiewal sie z nich.
\par 20 Zwlaszcza, iz nie byla wycieta majetnosc nasza, lecz ostatki ich ogien pozarl.
\par 21 Przyuczaj sie, prosze, z nim przestawac, a uczyn sobie z nim pokój: boc sie tak bedzie szczescilo.
\par 22 Przyjmij, prosze, z ust jego zakon, a zlóz wyroki jego w sercu twojem.
\par 23 Jezli sie nawrócisz do Wszechmocnego, zbudowany bedziesz, a oddalisz nieprawosc od przybytku twego:
\par 24 Tedy nakladziesz po ziemi wybornego zlota; a zlota z Ofir, jako kamienia z potoku.
\par 25 I bedzie Wszechmocny wybornem zlotem twojem, i srebrem, i sila twoja.
\par 26 Tedy sie w Wszechmocnym rozkochasz, a podniesiesz ku Bogu oblicze twoje.
\par 27 Bedziesz mu sie modlil, a wyslucha cie, i sluby twoje oddasz mu.
\par 28 Bo cokolwiek postanowisz, bedziec sie darzylo, a na drogach twoich rozjasni sie swiatlosc.
\par 29 Gdy inni znizeni beda, ty rzeczesz: Jam jest wywyzszon; bo tego, co jest unizonych oczów, Bóg zbawia.
\par 30 Wybawi i tego, który nie jest niewinny, i wybawion bedzie w czystosci rak twoich.

\chapter{23}

\par 1 A odpowiadajac Ijob rzekl:
\par 2 Czemuz jeszcze uporem zowiecie narzekanie moje, choc bieda moja ciezsza jest niz wzdychanie moje?
\par 3 Obym wiedzial, gdziebym go mógl znalesc, szedlbym az do stolicy jego.
\par 4 Przelozylbym przed nim sprawe moje, a usta moje napelnilbym dowodami.
\par 5 Dowiedzialbym sie, jakoby mi odpowiedzial, a zrozumialbym, coby mi rzekl.
\par 6 Izaz sie w wielkosci sily swojej bedzie spieral ze mna? Nie; i owszem sam mi doda sily.
\par 7 Tamby sie czlowiek szczery rozprawil z nim, i bylbym wolnym wiecznie od sedziego mego.
\par 8 Ale oto, pójdeli wprost, niemasz go; a jezli nazad, nie dojde go.
\par 9 Pójdeli w lewo, chocby zatrudniony byl, nie ogladam go; ukrylliby sie w prawo, nie ujrze go,
\par 10 Gdyz on zna droge moje; a bedzieli mie doswiadczal, jako zloto wynijde.
\par 11 Sladu jego trzymala sie noga moja; drogim jego przestrzegal, a nie zstepowalem z niej.
\par 12 Od przykazania ust jego nie odchylalem sie; owszem, postanowilem u siebie zachowac slowa ust jego.
\par 13 Jezli on przy swem stanie, któz go odwróci? bo co dusza jego zada, to uczyni:
\par 14 Bo on wykona, co postanowil o mnie, a takowych przykladów dosyc jest u niego.
\par 15 Przetoz od oblicza jego strwozylem sie, a uwazajac to, lekam sie go.
\par 16 Bóg zemdlil serce moje, a Wszechmocny zatrwozyl mna.
\par 17 Tak, zem malo nie zginal od ciemnosci; bo przed oblicznoscia moja nie zakryl zamroczenia.

\chapter{24}

\par 1 Czemuz od Wszechmocnego nie sa zakryte czasy? a którzy go znaja, nie widza dni jego?
\par 2 Niezbozni granice przenosza, trzody zabieraja i pasa.
\par 3 Osla sierotek zajmuja, a wolu od wdowy w zastawie biora.
\par 4 Spychaja ubogich z drogi; spólnie sie musza nedzni kryc na ziemi.
\par 5 Oto jako lesne osly w puszczach wychodza na robote swoje, wstawajac rano na lupiestwo; pustynia jest chlebem ich, i dzieci ich.
\par 6 Na polu ubogiego pozynaja zboze, a niepobozni z winnic zbieraja.
\par 7 Nagich nocowac przymuszaja bez odzienia, którzy sie nie maja czem nakryc na zimnie.
\par 8 Powodzia gór zmaczani bywaja, nie majac mieszkania przytulaja sie do skaly.
\par 9 Porywaja sierotke od piersi, a od ubogiego biora zastaw.
\par 10 Nagiemu dopuszczaja chodzic bez odzienia, a o glodzie chowaja tych, którzy ich snopy nosza.
\par 11 A ci, którzy miedzy murami ich wyciskaja oliwe i prasy tlocza, pragna.
\par 12 Ludzie w miescie wzdychaja, a dusze zabitych wolaja, a Bóg temu wstretu nie czyni.
\par 13 Cic to sa, którzy sie sprzeciwiaja swiatlosci, a nie znaja dróg jej, ani staneli na scieszkach jej.
\par 14 Raniuczko wstaje mezobójca, zabija ubogiego i niedostatecznego, a w nocy jest jako zlodziej.
\par 15 Oko cudzoloznika pilnuje zmierzku, mówiac: Nie ujrzy mie nikt; i zakrywa oblicze swe.
\par 16 Podkopywaja w ciemnosci domy, które sobie naznaczyli, i nienawidza swiatla.
\par 17 Ale zaranek jest im jako cien smierci; jezli ich kto pozna, przypada na nich strach cienia smierci.
\par 18 Lekkimi sa na wodach; przeklety dzial ich na ziemi; nie patrza na droge wolna.
\par 19 Jako susza i goracosc trawia wody sniezne, tak grób grzeszników.
\par 20 Zapomina go zywot matki jego, a robak slodkosc z niego czuje; niemasz wiecej pamiatki jego, a nieprawosc polamana jest jako drzewo.
\par 21 Roztraca nieplodna, która nierodzila, a wdowie nie czyni dobrze.
\par 22 Pociaga tez mocarzy moznoscia swoja: a gdy na nich powstal, zwatpili o zywocie swoim.
\par 23 Daje mu Bóg, na czemby bezpiecznie spolegac mógl: wszakze oczy jego patrza na drogi ich.
\par 24 Na chwile wywyzszeni sa, alic ich niemasz; znizeni i scisnieni beda jako inni wszyscy, a jako wierzch klosa scieci beda.
\par 25 A jezli nie tak jest, gdziez jest ten, coby mi zadal klamstwo, a coby obrócil wniwecz slowa moje?

\chapter{25}

\par 1 A odpowiadajac Bildad Suhytczyk rzekl:
\par 2 Panowanie i strach jest przy nim; on czyni pokój na wysokosciach swoich.
\par 3 Izali jest liczba wojskom jego? a nad kim nie wschodzi swiatlosc jego?
\par 4 Jakoz tedy nedzny czlowiek usprawiedliwiony byc moze przed Bogiem? albo jako moze byc czysty urodzony z niewiasty?
\par 5 Oto i miesiacby nie swiecil i gwiazdyby nie byly czyste w oczach jego:
\par 6 Jakoz daleko mniej czlowiek, który jest robakiem, a syn czlowieczy, który jest czerwiem.

\chapter{26}

\par 1 A Ijob odpowiadajac rzekl:
\par 2 Jakozes ratowal tego, który nie ma mocy? a jakos wybawil ramie, które nie ma sily?
\par 3 Jakazes dal rade temu, co nie ma madrosci? Azas go samej rzeczy gruntownie nie wyuczyl?
\par 4 Komuzes powiedzial te slowa? Czyjze duch wyszedl od ciebie?
\par 5 I martwe rzeczy rodza sie pod wodami, i obywatele ich.
\par 6 Odkryte sa przepasci przed nim, a nie ma przykrycia zatracenie.
\par 7 Rozciagnal pólnocy nad miejscem próznem, a ziemie zawiesil na niczem.
\par 8 Zawiazuje wody na oblokach swoich, a nie rwie sie oblok pod nimi.
\par 9 Zatrzymuje stolice swoje, rozpostarlszy nad nia oblok swój.
\par 10 Polozyl granice wodom, az wezmie koniec swiatlosc i ciemnosc.
\par 11 Slupy niebieskie trzesa sie, i chwieja sie na gromienie jego.
\par 12 Moca swa dzieli morze, a roztropnoscia swa usmierza nawalnosci jego.
\par 13 Duchem swym niebiosa przyozdobil, a reka jego stworzyla weza skretnego.
\par 14 Oto tec sa tylko czesci dróg jego, lecz i ta trocha niewybadana, cosmy slyszeli o nim, a grzmot wielkiej moznosci jego któz zrozumie?

\chapter{27}

\par 1 Potem dalej Ijob prowadzil rzecz swoje, i rzekl:
\par 2 Jako zyje Bóg, który odrzucil sad mój, a Wszechmocny, który gorzkosci nabawil duszy mojej:
\par 3 Ze póki staje tchu we mnie, i ducha Bozego w nozdrzach moich,
\par 4 Nie beda mówily wargi moje nieprawosci, a jezyk mój nie bedzie powiadal zdrady.
\par 5 Nie daj Boze, zebym was mial usprawiedliwiac; póki dech we mnie, nie odstapie od niewinnosci mojej.
\par 6 Sprawiedliwosci mojej trzymac sie bede, a nie puszcze sie jej; nie zawstydzi mie serce moje, pókim zyw.
\par 7 Nieprzyjaciel mój bedzie jako niezboznik, a który powstaje przeciwko mnie, jako zlosnik.
\par 8 Co bowiem za nadzieja jest obludnika, który sie w lakomstwie kocha, gdy Bóg wydrze dusze jego.
\par 9 Izali Bóg uslyszy wolanie jego, gdy nan ucisk przyjdzie?
\par 10 Izaz sie w Wszechmocnym rozkocha? a bedzie wzywal Boga na kazdy czas?
\par 11 Ucze was, bedac w rece Bozej, a jako ide z Wszechmocnym, nie taje.
\par 12 Oto wy to wszyscy widzicie; przeczze wzdy próznosc mówicie?
\par 13 Tenci jest dzial czlowieka bezboznego u Boga, a toc dziedzictwo okrutnicy od Wszechmocnego wezma.
\par 14 Jezli sie rozmnoza synowie jego, pójda pod miecz: a potomstwo jego nie nasyci sie chleba.
\par 15 Którzy po nim zostana w smierci pogrzebieni beda, a wdowy jego nie beda go plakaly;
\par 16 Chocby srebra nazgromadzal jako prochu, a nasprawial szat jako blota:
\par 17 Tedy nasprawiac ich on, ale sprawiedliwy oblekac je bedzie, a srebro ono niewinny dzielic bedzie.
\par 18 Zbuduje dom swój jako mól, a jako stróz bude wystawi.
\par 19 Bogaty zasnie, a nie bedzie pogrzebiony; spojrzyli kto, alic go niemasz.
\par 20 Zachwyca go strachy jako wody, w nocy go porwie wicher.
\par 21 Pochwyci go wiatr wschodni, a odejdzie; bo wicher ruszyl go z miejsca swego.
\par 22 Toc Bóg nan dopusci, a nie przepusci mu, choc przed reka jego predko uciekac bedzie.
\par 23 Klasnie kazdy nad nim rekoma swemi, i wysyka go z miejsca swego.

\chapter{28}

\par 1 Mac w prawdzie srebro poczatki zyl swoich, a zloto miejsce, kedy bywa plawione.
\par 2 Zelazo z ziemi biora, a z kamienia zlewaja miedz.
\par 3 Celu ciemnosciom ulozonego i konca wszystkich rzeczy on dochodzi, i kamieni, które w ciemnosci i cieniu smierci leza.
\par 4 Wyleje rzeka z miejsca swojego, tak, iz jej nikt przebyc nie moze, bywa jednak zahamowana przemyslem nedznego czlowieka, i odchodzi.
\par 5 Z ziemi wychodzi chleb, chociaz pod nia cos róznego, podobnego ogniowi.
\par 6 W niektórych miejscach jest kamien Safir, i piasek zloty;
\par 7 A tej scieszki ani ptak nie wie, ani jej widzalo oko sepie.
\par 8 Nie depcza po niej zwierzeta srogie, ani lew przeszedl przez nie.
\par 9 Na krzemien sciagnal reke swoje, wywrócil góry z korzenia;
\par 10 Z skal wywodzi strumienie, a kazda rzecz kosztowna widzi oko jego.
\par 11 Wylewac sie rzekom nie dopuszcza, a rzeczy skryte wywodzi na jasnie.
\par 12 Ale madrosc gdziez moze byc znaleziona? a kedy jest miejsce roztropnosci?
\par 13 Nie wie czlowiek smiertelny ceny jej, ani bywa znaleziona w ziemi zyjacych.
\par 14 Przepasc mówi: Niemasz jej we mnie; i morze tez powiada: Niemasz jej u mnie.
\par 15 Nie dawaja szczerego zlota za nie; ani odwazaja srebra, za odmiane jej.
\par 16 Nie moze byc oszacowana za zloto Ofir, ani za Onychyn drogi, ani za Safir.
\par 17 Nie porówna z nia zloto, ani krysztal, ani odmiana jej moze byc za klejnot zlota szczerego.
\par 18 Koralów i perel nie wspomina, bo nabycie madrosci kosztowniejsze jest nad perly.
\par 19 Nie zrówna z nia i szmaragd z ziemi etyjopskiej; ani za zloto najczystsze szacowana byc moze.
\par 20 Skadze tedy madrosc pochodzi? albo gdzie jest miejsce rozumu?
\par 21 Gdyz zakryta jest od oczu wszystkich zyjacych, i przed ptastwem niebieskim zatajona jest.
\par 22 Zginienie i smierc rzekly: Uszyma swemi slyszalysmy slawe jej.
\par 23 Bóg sam rozumie droge jej, a on wie miejsce jej.
\par 24 Bo on na konczyny ziemi patrzy, a wszystko, co jest pod niebem, widzi.
\par 25 Wiatrom uczynil wage, a wody odwazyl pod miara.
\par 26 On tez prawo dzdzom postanowil, a droge blyskawicom gromów.
\par 27 W ten czas ja widzial, i glosil ja: zgotowal ja, i doszedl jej.
\par 28 Ale czlowiekowi rzekl: Oto bojazn Panska jest madroscia, a warowac sie zlego, jest rozumem.

\chapter{29}

\par 1 Jeszcze dalej Ijob prowadzil rzecz swoje, i rzekl:
\par 2 Któz mi to da, abym byl jako za miesiecy dawnych, za dni onych, których mie Bóg strzegl;
\par 3 Gdy pochodnia jego swiecila nad glowa moja, a przy swietle jego przechodzilem ciemnosci;
\par 4 Jakom byl za dni mlodosci mojej, gdy byla przytomnosc Boza nad przybytkiem moim;
\par 5 Gdy jeszcze Wszechmocny byl ze mna, a okolo mnie dziatki moje;
\par 6 Gdy scieszki moje oplywaly maslem, a opoka wylewala mi zródla oliwy;
\par 7 Gdym wychodzil do bramy przez miasto, a na ulicy kazalem sobie gotowac stolice moje.
\par 8 Widzac mie mlodzi ukrywali sie, a starcy powstawszy stali.
\par 9 Przelozeni przestawali mówic, a reka zatykali usta swoje.
\par 10 Glos ksiazat ucichal, a jezyk ich do podniebienia ich przylegal.
\par 11 Bo ucho sluchajace blogoslawilo mie, a oko widzace dawalo o mnie swiadectwo,
\par 12 Zem wybawial ubogiego wolajacego, i sierotke, i tego, który nie mial pomocnika.
\par 13 Blogoslawienstwo ginacego przychodzilo na mie, a serce wdowy rozweselalem.
\par 14 W sprawiedliwosc obloczylem sie, a ona zdobila mie; sad mój byl jako plaszcz i korona.
\par 15 Bylem okiem slepemu, a noga chromemu.
\par 16 Bylem ojcem ubogich, a sprawy, którejm nie wiedzial, wywiadywalem sie.
\par 17 I kruszylem szczeki zlosnika, a z zebów jego wydzieralem lup.
\par 18 Przetozem rzekl: W gniazdzie swojem umre, a jako piasek rozmnoze dni moje.
\par 19 Korzen mój rozlozy sie przy wodach, a rosa trwac bedzie przez noc na galazkach moich.
\par 20 Chwala moja odmlodzi sie przy mnie, a luk mój w rece mojej odnowi sie.
\par 21 Sluchano mie, i oczekiwano na mie, a milczano na rade moje.
\par 22 Po slowie mojem nie powtarzano, tak na nich kropila mowa moja.
\par 23 Bo mie oczekiwali jako deszczu, a usta swe otwierali jako na deszcz pózny.
\par 24 Jezlim zartowal z nimi, nie wierzyli, a powagi twarzy mojej nie odrzucali.
\par 25 Jezlim kiedy do nich przyszedl, siadalem na przedniejszem miejscu, i mieszkalem jako król w wojsku, a jako ten, który smutnych cieszy.

\chapter{30}

\par 1 Ale teraz smieja sie ze mnie mlodsi nad mie w latach, których ojcówbym ja byl nie chcial polozyc ze psami trzody mojej.
\par 2 Acz na cózby mi sie byla sila rak ich przydala? bo przy nich starosc ich zginela.
\par 3 Albowiem dla niedostatku i glodu samotni byli, i uciekali na nieplodne, ciemne, osobne, i puste miejsce;
\par 4 Którzy sobie rwali chwasty po chróstach, a korzonki jalowcowe byly pokarmem ich.
\par 5 Z posrodku ludzi wyganiano ich; wolano za nimi jako za zlodziejem,
\par 6 Tak, iz w lozyskach potoków mieszkac musieli, w jamach podziemnych i w skalach.
\par 7 Miedzy chróstami ryczeli, pod pokrzywy zgromadzali sie.
\par 8 Synowie ludzi wzgardzonych, i synowie ludzi bezecnych, podlejsi byli nad proch ziemi.
\par 9 Alem teraz piesnia ich, i stalem sie im przypowiescia.
\par 10 Brzydza sie mna, a oddalaja sie odemnie, i na twarz moje plwac sie nie wstydza.
\par 11 Bo Bóg powage moje odjal i utrapil mie; dlatego oni wedzidlo przed twarza moja odrzucili.
\par 12 Po prawicy mojej mlodzikowie powstawaja, nogi moje potracaja, i toruja na przeciwko mnie drogi zginienia swego.
\par 13 Popsuli scieszke moje, i nedzy do nedzy mojej przyczynili, a nie potrzebuja do tego pomocnika.
\par 14 Jako przerwa szeroka napadaja na mie, i na spustoszenie moje wala sie.
\par 15 Obrócily sie przeciwko mnie strachy, jako wiatr sciagaja dusze moje; bo jako oblok przemija zdrowie moje.
\par 16 A teraz we mnie rozlala sie dusza moja; ogarnely mie dni utrapienia;
\par 17 Które w nocy wierca kosci moje we mnie, skad zyly moje nie maja odpoczynku.
\par 18 Dla wielkiej bolesci zmienila sie szata moja, a jako kolnierz sukni mojej sciska mie.
\par 19 Wrzucil mie w bloto, a jestem podobien prochowi i popiolowi.
\par 20 Wolam do ciebie, a nie wysluchujesz mie; stoje przed toba, a nie patrzysz na mie.
\par 21 Odmieniles mi sie w okrutnego, a moca reki twej sprzeciwiasz mi sie.
\par 22 Podnosisz mie na wiatr, i wsadzasz mie nan, a zdrowemu rozsadkowi rozplynac sie dopuszczasz.
\par 23 Wiemci, ze mie na smierc podasz, i do domu wszystkim zyjacym naznaczonego.
\par 24 Wszakze na grób nie sciagnie reki swej, a gdy ich niszczyc bedzie, wolac nie beda.
\par 25 Izalim nie plakal nad dniem utrapionego? izali sie nie smucila dusza moja nad ubogim?
\par 26 Gdym dobrego oczekiwal, oto przyszlo zle; a gdym sie spodziewal swiatlosci, przyszla ciemnosc.
\par 27 Wnetrznosci moje wezwrzaly, a nie uspokoily sie, i ubiezaly mie dni utrapienia.
\par 28 Chodze szczerniawszy, ale nie od slonca; powstaje i wolam w zgromadzeniu.
\par 29 Stalem sie bratem smoków, a towarzyszem strusiów mlodych.
\par 30 Skóra moja poczerniala na mnie, i kosci moje wypiekly sie od upalenia.
\par 31 Obrócila sie w lament harfa moja, a instrument mój w glos placzacych.

\chapter{31}

\par 1 Uczynilem przymierze z oczyma swemi, abym nie pomyslal o pannie.
\par 2 Bo cóz za dzial od Boga z góry? a co za dziedzictwo Wszechmocnego z wysokosci?
\par 3 Azaz nie nagotowane zginienie zlosnikom, a sroga pomsta czyniacym nieprawosc?
\par 4 Azaz on nie widzi dróg moich, a wszystkich kroków moich nie liczy?
\par 5 Jezlim chodzil w klamstwie, a spieszyla sie na zdrade noga moja:
\par 6 Niech mie zwazy na wadze sprawiedliwej, a niech Bóg pozna szczerosc moje.
\par 7 Jezliz ustapila noga moja z drogi, a za oczyma memi szloli serce moje, i do rak moich jezliz przylgnela jaka zmaza:
\par 8 Tedy niechze ja sieje, a inszy niech pozywa, a moje latorosle niech beda wykorzenione.
\par 9 Jezli zwiedzione jest serce moje do niewiasty, i jezlim czyhal u drzwi przyjaciela mego:
\par 10 Niechajze mele innemu zona moja, a niechaj sie nad nia inni schylaja.
\par 11 Boc to jest sprosny wystepek, a nieprawosc osadzenia godna,
\par 12 Gdyz ten ogien az do zatracenia pozera, a dochody moje wszystkie wykorzenic moze.
\par 13 Jezlim stronil od sadu z sluga moim, albo z sluzebnica moja, gdy ze mna sprzeczke mieli,
\par 14 (Bo cózbym czynil, gdyby powstal Bóg? albo gdyby pytal, cobym mu odpowiedzial?
\par 15 Izaz nie ten, który mie w zywocie uczynil, nie uczynil tez i onego? a nie onze nas sam w zywocie wyksztaltowal?)
\par 16 Jezlizem odmówil ubogim, czego chcieli, a oczy wdowy jezlizem zasmucil;
\par 17 Jezlizem jadl sztuczke swoje sam, a nie jadala i sierota z niej;
\par 18 (Albowiem sierota z mlodosci mojej rosla ze mna, jako u ojca; a jakom wyszedl z zywota matki mojej, bylem wdowie za wodza.)
\par 19 Jezlizem widzial kogo ginacego dla tego, ze szaty nie mial, a nie dalem zebrakowi odzienia;
\par 20 Jezlize mi nie blogoslawily biodra jego, ze sie welna owiec moich zagrzal;
\par 21 Jezlizem podniósl przeciwko sierocie reke swoje, gdym widzial w bramie pomoc moje:
\par 22 Tedy niech odpadnie lopatka moja od plec swych, a ramie moje z stawu swego niech wytracone bedzie.
\par 23 Albowiem lekalem sie skruszenia od Boga, a przed jego zacnoscia nie móglbym sie ostac.
\par 24 Jezlim pokladal w zlocie nadzieje moje, a do bryly zlota mawialem: Tys ufanie moje;
\par 25 Jezlim sie weselil z wielu bogactw moich, a iz wiele nabyla reka moja;
\par 26 Jezlim patrzal na swiatlosc slonca, gdy swiecilo, a na miesiac, gdy wspanialo chodzil;
\par 27 I dalo sie uwiesc potajemnie serce moje, a calowaly reke moje usta moje:
\par 28 I tocby byla nieprawosc osadzenia godna; bobym sie tem zaprzal Boga z wysokosci.
\par 29 Jezlizem sie weselil z upadku nienawidzacego mie, a jezlim sie cieszyl, gdy mu sie zle powodzilo.
\par 30 (I owszem nie dalem zgrzeszyc ustom moim, abym mial zadac przeklestwa duszy jego.)
\par 31 Azaz nie mawiali domownicy moi: Oby nam kto dal miesa tego, nie mozemy sie i najesc?
\par 32 Bo gosc nie nocowal na dworze, a drzwi moje otwieralem podróznemu.
\par 33 Jezlim zakrywal, jako ludzie zwykli, przestepstwa moje, i chowalem w skrytosci mojej nieprawosc moje;
\par 34 I chocbym byl mógl potlumic zgraje wielka, jednak i najpodlejszy z domu ustraszyl mie; przetozem milczal, i nie wychodzilem ze drzwi.
\par 35 Obym mial kogo, coby mie wysluchal; ale oto ten jest znak mój, ze Wszechmogacy sam odpowie za mie, i ksiega, która napisal przeciwnik mój.
\par 36 Czylibym jej na ramieniu swojem nie nosil? a nie przywiazalbym jej sobie miasto korony?
\par 37 Liczbe kroków moich oznajmilbym mu; jako do ksiazecia przystapilbym do niego.
\par 38 Jezliz przeciw mnie ziemia moja wolala, a jezlize z nia spolem zagony jej plakaly;
\par 39 Jezlizem pozytków jej uzywal bez pieniedzy, i jezlim do wzdychania przywodzil dzierzawców jej:
\par 40 Miasto pszenicy niech wznijdzie oset, a miasto jeczmienia kakol. Tu sie skonczyly slowa Ijobowe.

\chapter{32}

\par 1 A gdy przestali oni trzej mezowie odpowiadac Ijobowi, przeto, ze sie sobie zdal byc sprawiedliwym:
\par 2 Tedy sie rozpalil gniewem Elihu, syn Barachela Buzytczyka z rodu Syryjskiego, przeciw Ijobowi sie rozpalil gniewem, iz usprawiedliwial dusze swoje, wiecej niz Boga.
\par 3 Takze przeciwko trzem przyjaciolom jego rozpalil sie gniew jego, ze nie znalazlszy odpowiedzi, przecie potepiali Ijoba.
\par 4 Bo Elihu oczekiwal, jako oni Ijobowi odpowiedza, gdyz oni starsi byli w latach niz on.
\par 5 Ale widzac Elihu, ze nie bylo odpowiedzi w ustach onych trzech mezów, rozpalil sie w gniewie swoim.
\par 6 I odpowiedzial Elihu, syn Barachela Buzytczyka, i rzekl: Jam najmlodszy w latach, a wyscie starcy; przetoz wstydzilem sie, i nie smialem wam oznajmic zdania swego.
\par 7 Myslalem: Dlugi wiek mówic bedzie, a mnóstwo lat nauczy madrosci.
\par 8 Alec duch, który jest w ludziach, i natchnienie Wszechmogacego daje rozum.
\par 9 Zacni nie zawsze madrzy, a starcy nie zawzdy rozumieja sadu.
\par 10 Przetoz mówie: sluchaj mie; ja tez oznajmie zdanie swoje.
\par 11 Otom oczekiwal slów waszych, a przysluchiwalem sie dowodom waszym, czekajac, azbyscie doszli rzeczy.
\par 12 I przypatrywalem sie wam, a oto zaden z was Ijoba przekonac nie mógl; i nie masz miedzy wami, ktoby odpowiedzial slowom jego.
\par 13 Ale snac rzeczecie: Znalezlismy madrosc; sam go Bóg przekonywa, nie czlowiek.
\par 14 Aczci sie Ijob nie zemna wdal w rzecz, a ja mu tez nie waszemi slowy odpowiem.
\par 15 Polekali sie, nie odpowiadaja dalej; niedostaje im slów.
\par 16 Czekalemci, ale nie mówia; umilkneli, a nic wiecej nie odpowiadaja.
\par 17 Odpowiem ja tez z mej strony; oznajmie ja tez zdanie swoje.
\par 18 Bom pelen slów; ciasno we mnie duchowi zywota mego.
\par 19 Oto zywot mój jest jako moszcz bez oddechu, a jako beczka nowa rozpeklby sie.
\par 20 Bede tedy mówil, a wytchne sobie; otworze wargi swe, i odpowiem.
\par 21 Nie bede teraz mial wzgledu na zadna osobe, a z czlowiekiem bez tytulów mówic bede.
\par 22 Bo nie umiem tytulowac, by mie w rychle nie porwal stworzyciel mój.

\chapter{33}

\par 1 A przetoz, Ijobie! sluchaj prosze mów moich, a wszystkie slowa moje przyjmij w uszy.
\par 2 Oto teraz otworze usta moje, a jezyk mój bedzie mówil w podniebieniu mojem.
\par 3 Szczeroscia serca mego beda slowa moje, a czyste zdania wargi moje mówic beda.
\par 4 Duch Bozy uczynil mie, a tchnienie Wszechmocnego ozywilo mie.
\par 5 Mozeszli, odpowiedz mi; sporzadz sie, a stan przeciwko mnie.
\par 6 Oto ja wedlug slów twoich odpowiem ci za Boga, chociazem ja tez z blota utworzony.
\par 7 Oto strach mój nie zatrwozy cie, a reka moja nie obciazy cie.
\par 8 A wszakzes rzekl w uszy moje, i slyszalem glos slów moich.
\par 9 Czystym ja bez przestepstwa; niewinnym ja, i nie masz we mnie nieprawosci.
\par 10 Oto znajduje Bóg przyczyny przeciwko mnie, a poczytuje mie za nieprzyjaciela swego.
\par 11 Podaje w okowy nogi moje, a podstrzega wszystkich sciezek moich.
\par 12 Otosci na to tak odpowiadam: W tem nie jestes sprawiedliwy; bo wiekszy jest Bóg, niz czlowiek.
\par 13 Przeczze sie z nim spierasz, zec wszystkich spraw swoich nie objawia?
\par 14 Wszak Bóg mówi i raz i drugi, a czlowiek tego nie uwaza.
\par 15 We snie w widzeniu nocnem, gdy twardy sen przypada na ludzie gdy spia na lozu:
\par 16 Tedy otwiera ucho ludzkie, a to, czem ich cwiczy, pieczetuje,
\par 17 Aby czlowieka odwiódl od zlej sprawy jego, i pyche od meza aby odjal;
\par 18 Aby zahamowal dusze jego od dolu, a zywot jego aby na miecz nie trafil.
\par 19 Kaze go tez bolescia na lozu jego, a we wszystkich kosciach jego ciezka niemoca.
\par 20 Tak, ze sobie zywot jego chleb obrzydzi, a dusza jego pokarm wdzieczny.
\par 21 Zniszczeje znacznie cialo jego, i wysadza sie kosci jego, których nie widac bylo;
\par 22 I przybliza sie do grobu dusza jego a zywot jego do rzeczy smierc przynoszacych.
\par 23 Jezli bedzie u niego jaki Aniol wymowny, jeden z tysiaca, aby opowiedzial czlowiekowi powinnosc jego:
\par 24 Tedy sie nad nim Bóg zmiluje, a rzecze: Wybaw go, aby nie zstepowal do grobu, bom znalazl ublaganie.
\par 25 I odmlodnieje cialo jego jako dzieciece, a nawróci sie do dni mlodosci swojej.
\par 26 Bedzie sie modlil Bogu, i przyjmie go laskawie, i oglada z weselem oblicze jego, i przywróci czlowiekowi sprawiedliwosc jego;
\par 27 Który pogladajac na ludzi, rzecze: Zgrzeszylem byl, i co bylo prawego, podwrócilem; ale mi to nie bylo pozyteczno.
\par 28 Lecz Bóg wybawil dusze moje, aby nie zstapila do dolu, a zywot mój aby ogladal swiatlosc.
\par 29 Oto wszystko to czyni Bóg po dwakroc i po trzykroc z czlowiekiem,
\par 30 Aby odwrócil dusze jego od dolu, a zeby oswiecon byl swiatloscia zyjacych.
\par 31 Uwazaj to, Ijobie, sluchaj mie; milcz, a ja bede mówil.
\par 32 Wszakze maszli co mówic, a odpowiedzze mi; mów, bobym cie rad usprawiedliwil.
\par 33 A jezli niemasz, sluchajze mie, a naucze cie madrosci.

\chapter{34}

\par 1 Nadto mówil Elihu, i rzekl:
\par 2 Sluchajciez, madrzy! mów moich, a nauczeni posluchajcie mie.
\par 3 Bo ucho slów doswiadcza, jako podniebienie smakuje pokarmu.
\par 4 Obierzmy sobie sad, a rozeznajmy miedzy soba, co jest dobrego.
\par 5 Poniewaz Ijob rzekl: Jestem sprawiedliwym, a Bóg odrzucil sprawe moje:
\par 6 I mamze klamac, majac sprawiedliwa? Bolesny jest postrzal mój bez przewinienia.
\par 7 Któryz jest maz taki, jako Ijob, coby pil posmiewisko jako wode?
\par 8 A coby chodzil w towarzystwie czyniacych nieprawosc; i przestawalby z ludzmi niepoboznymi?
\par 9 Bo powiedzial: Nie pomoze czlowiekowi, chocby sie podobal Bogu.
\par 10 Przetoz mie sluchajcie, mezowie rozumni! Niech bedzie daleka niepoboznosc od Boga, i nieprawosc od Wszechmocnego.
\par 11 Bo on wedlug uczynku placi czlowiekowi, a wedlug drogi jego kazdemu nagradza.
\par 12 A zgola Bóg przewrotnie nie czyni, a Wszechmocny nie podwraca sadu.
\par 13 Któz go przelozyl nad ziemia? a kto wystawil caly okrag swiata?
\par 14 Jezliby obrócil przeciwko niemu serce swoje, a ducha jego, i dech jego do siebie wzial:
\par 15 Zgineloby wszelkie cialo spolu, a czlowiekby sie do prochu nawrócil.
\par 16 Maszli tedy rozum, sluchaj tego, a przyjmuj w uszy swe glos mowy mojej.
\par 17 Azaz ten, który ma w nienawisci sad, panowac moze? azaz tego, który jest wielce sprawiedliwy, niepoboznym uczynisz?
\par 18 Zaz potepisz tego, który moze rzec królowi: O bezecny! a ksiazetom: O niepobozny!
\par 19 Który nie ma wzgledu na osoby ksiazat, i nie wazy sobie wiecej bogacza nad ubogiego; bo oni wszyscy sa czynem rak jego.
\par 20 Nagle umieraja; a o pólnocy wzruszony bywa naród, i przemija, a mocarz zniesiony bywa bez reki ludzkiej.
\par 21 Oczy bowiem jego nad drogami czlowieczemi, a on widzi wszystkie kroki jego.
\par 22 Niemasz ciemnosci, ani cienia smierci, kedyby sie skryli ci, którzy czynia nieprawosc.
\par 23 Bo na nikogo nie wklada wiecej, tak, zeby mial wchodzic w sad z Bogiem.
\par 24 Pociera bardzo wiele mocarzów, a inszych miasto nich wystawia.
\par 25 Przeto, iz zna sprawy ich, obraca im dzien w noc, aby byli potarci.
\par 26 Poraza ich jako niepoboznych na miejscu jawnem.
\par 27 Przeto, iz odstapili od niego, a zadnych dróg jego zrozumiec nie chcieli:
\par 28 Aby przywiódl na nich wolanie znedznialych, a pokazal, ze wysluchuje wolanie ubogich.
\par 29 Gdy on sprawi pokój, któz go wzruszy? takze, gdy skryje oblicze, któz go ujrzy? A to czyni tak calemu narodowi, jako kazdemu czlowiekowi,
\par 30 Aby dalej nie panowal czlowiek obludny na upadek ludzki.
\par 31 Zaprawde mialbys mówic do Boga: Przepusc; poniose, a nie bede sie wzbranial.
\par 32 Nadto jezlibym czego nie baczyl, ty mie naucz; jezlim nieprawosc popelnil, nie uczynie tego wiecej.
\par 33 Izali wedlug zdania twego bedziesz placil, zec sie to nie podoba, a zes ty owo obral, a nie on? Ale wieszli co lepszego, powiedz.
\par 34 Mezowie rozumni toz rzeka ze mna, a czlowiek madry przypadnie na to,
\par 35 Ze Ijob nie mówi madrze, a slowa jego nie sa roztropne.
\par 36 Boze, Ojcze mój! niech bedzie Ijob doskonale doswiadczony, przeto, iz nam odpowiada, jako ludziom zlym.
\par 37 Bo przestepstwa przyczynia do grzechu swego; chlubi sie miedzy nami, i mówi bardzo wiele przeciwko Bogu.

\chapter{35}

\par 1 Nadto mówil Elihu, i rzekl:
\par 2 I mniemasz, zes to z rozsadkiem rzekl: Sprawiedliwosc moja przechodzi Boska?
\par 3 Bos powiedzial: Cóz mi pomoze? a co wezme za pozytek, chocbym nie grzeszyl?
\par 4 Ale ja tobie dowodnie odpowiem, i towarzyszom twoim z toba.
\par 5 Spojrzyj w niebo, a obacz; przypatrz sie oblokom, jako sa wyzsze nad cie.
\par 6 Jezli zgrzeszysz, cóz uczynisz przeciwko niemu? a jezliby byly rozmnozone nieprawosci twoje, cóz mu uczynisz?
\par 7 Jezlibys byl sprawiedliwym, cóz mu dasz? albo cóz wezmie z reki twojej?
\par 8 Czlowiekowi podobnemu tobie niezboznosc twoja zaszkodzi, a synowi czlowieczemu pomoze sprawiedliwosc twoja.
\par 9 Z mnóstwa ucisnionych, którzy do tego przywiedzieni sa; aby narzekali i wolali dla ramienia mocarzów,
\par 10 Zaden nie mówi: Gdziez jest Bóg, stworzyciel mój, choc on daje spiewanie i w nocy?
\par 11 Choc nas wyucza nad bydleta ziemskie, a nad ptastwo niebieskie czyni nas medrszymi.
\par 12 Tedy wolajali dla hardosci zlych, on ich nie wysluchuje.
\par 13 Bo obludy nie wyslucha Bóg, a Wszechmocny nie patrzy na nich.
\par 14 Dopieroz nie wyslucha ciebie, poniewaz mówisz: Nie widzisz tego; osadzze sie przed nim, a oczekuj go,
\par 15 Gdyz cie jedno troche nawiedzil gniew jego, jakoby nie wiedzial wielkosci grzechów twoich.
\par 16 Przetoz Ijob prózno otwiera usta swe, a bez umiejetnosci rozmnaza slowa swoje.

\chapter{36}

\par 1 Do tego przydal Elihu, i rzekl:
\par 2 Poczekaj mie maluczko, a ukazec; bo jeszcze mam, cobym za Bogiem mówil.
\par 3 Zaczne umiejetnosc moje z daleka, a Stworzycielowi memu przywlaszcze sprawiedliwosc.
\par 4 Boc zaprawde bez klamstwa beda mowy moje, a maz doskonaly w umiejetnosci jest przed toba.
\par 5 Oto Bóg mocny jest, a nie odrzuca nikogo; on jest mocny w sile serca.
\par 6 Nie zywi niepoboznego, a u sadu ubogim dopomaga.
\par 7 Nie odwraca od sprawiedliwego oczów swoich; ale z królmi na stolicy sadza ich na wieki, i bywaja wywyzszeni.
\par 8 A jezliby byli okowani w peta, albo uwiklani powrozami utrapienia:
\par 9 Tedy przez to im oznajmuje sprawy ich, i przestepstwa ich, ze sie zmocnily;
\par 10 I otwiera im ucho, aby przyjeli karanie, a mówi, aby sie nawrócili od nieprawosci.
\par 11 Jezli beda posluszni, a beda mu sluzyc, dokoncza dni swoich w dobrem, a lat swych w rozkoszach.
\par 12 Ale jezli nie usluchaja, od miecza zejda, a pomra bez umiejetnosci.
\par 13 Bo ludzie obludnego serca obalaja na sie gniew, a nie wolaja, kiedy ich wiaze.
\par 14 Umrze w mlodosci dusza ich, a zywot ich miedzy nierzadnikami.
\par 15 Wyrwie utrapionego z utrapienia jego, a otworzy w ucisnieniu ucho jego.
\par 16 Takby i ciebie wyrwal z miejsca ciasnego na przestronne, gdzie niemasz ucisku, a spokojny stól twój bylby pelen tlustosci.
\par 17 Ales ty sad niepoboznego zasluzyl, przetoz prawo i sad beda cie trzymac.
\par 18 Zaistec gniew Bozy jest nad toba; patrzze, aby cie nie porazil plaga wielka, tak, zeby cie nie wybawil zaden okup.
\par 19 Izali sobie bedzie wazyl bogactwa twoje? Zaiste ani zlota, ani jakiejkolwiek sily, albo potegi twojej.
\par 20 Nie kwapze sie tedy ku nocy, w która zstepuja narody na miejsca swoje.
\par 21 Strzez, abys sie nie ogladal na nieprawosc, obierajac ja sobie nad utrapienia.
\par 22 Oto Bóg jest najwyzszy w mocy swojej, któz tak nauczyc moze jako on?
\par 23 Któz mu wymierzyl droge jego? albo kto mu rzecze: Uczyniles nieprawosc?
\par 24 Pamietajze, abys wyslawial sprawe jego, której sie przypatruja ludzie.
\par 25 Wszyscy ludzie widza ja, a czlowiek przypatruje sie jej z daleka.
\par 26 Oto Bóg jest wielki, a poznac go nie mozemy, ani liczba lat jego doscigniona byc moze.
\par 27 Bo on wyciaga krople wód, które wylewaja z obloków jego deszcz,
\par 28 Który spuszczaja obloki, a spuszczaja na wiele ludzi.
\par 29 (Nadto, któz zrozumie rozciagnienie obloków, i grzmot namiotu jego.
\par 30 Jako rozciaga nad nim swiatlosc swoje, a glebokosci morskie okrywa?
\par 31 Bo przez te rzeczy sadzi narody, i daje pokarm w hojnosci.
\par 32 Oblokami nakrywa swiatlosc, i rozkazuje jej ukrywac sie za oblok nastepujacy.)
\par 33 Daje o nim znac szum jego, takze i bydlo i para w góre wstepujaca.
\par 34 A nad tem zdumiewa sie serce moje, i porusza sie z miejsca swego.

\chapter{37}

\par 1 Sluchajcie z pilnoscia grzmienia glosu jego, i dzwieku który wychodzi z ust jego.
\par 2 Pod wszystkiem niebem prosto go wypuszcza, a swiatlosc jego po wszystkich konczynach ziemi.
\par 3 Za nia wnet huczy dzwiekiem, grzmi glosem zacnosci swojej, i nie odklada innych rzeczy, gdy bywa slyszany glos jego.
\par 4 Dziwnie Bóg grzmi glosem swoim; sprawuje rzeczy tak wielkie, ze ich rozumiec nie mozemy.
\par 5 Bo mówi do sniegu: Padaj na ziemie; takze i do deszczu wolnego, i do deszczu gwaltownego.
\par 6 Reke wszystkich ludzi zawiera, aby nikt z ludzi nie dogladal roboty swojej.
\par 7 Tedy zwierz wchodzi do jaskini, a w jamach swoich zostaje.
\par 8 Wicher z skrytych miejsc wychodzi, a zima z wiatrów pólnocnych.
\par 9 Tchnieniem swojem Bóg czyni lód, tak iz sie szerokosc wód sciska.
\par 10 Takze dla pokropienia ziemi obciaza oblok, i rozpedza chmure swiatlem swojem.
\par 11 A ten sie obraca w kolo wedlug rady jego, aby czynil wszystko, co Bóg rozkaze, na oblicze okregu ziemskiego.
\par 12 A czyni to Bóg, ze sie stawia badz na skaranie, badz dla pozytku ziemi swojej, badz dla jakiej dobroczynnosci.
\par 13 Sluchajze tego pilnie, Ijobie! zastanów sie, a uwazaj dziwne sprawy Boze.
\par 14 Izali wiesz, kiedy co Bóg stanowi o tych rzeczach? albo gdy ma rozjasnic swiatlo obloku swego?
\par 15 Izali wiesz, co za waga obloków? Izali wiesz cuda Doskonalego we wszelakiej umiejetnosci?
\par 16 Wieszze, jako cie szaty twoje ogrzewaja, gdy ucisza ziemie od poludnia?
\par 17 Izazes z nim rozposcieral niebiosa, które sa trwale, a zwierciadlu odlewanemu podobne?
\par 18 Ukazze nam, co mu mamy powiedziec; bo nie mozemy sporzadzic slów dla ciemnosci.
\par 19 Izali mu kto odniesie to, cobym mówil? I owszem, gdyby to kto przedlozyl, bylby pewnie pozarty.
\par 20 Wszak teraz nie moga ludzie patrzyc i na swiatlo, gdy jest jasne na oblokach, gdy wiatr przechodzi, i przeczyszcza je.
\par 21 Od pólnocy jako zloto przychodzi, ale w Bogu straszniejsza jest chwala.
\par 22 Wszechmogacy jest, doscignac go nie mozemy; wielki w mocy, wszakze sadem i ostra sprawiedliwoscia ludzi nie trapi.
\par 23 Przetoz boja sie go ludzie; nie ma wzgledu na zadnego, by tez byl i najmedrszy.

\chapter{38}

\par 1 Tedy odpowiedzial Pan Ijobowi z wichru, i rzekl:
\par 2 Któz to jest, co zaciemnia rade Boza mowami nieroztropnemi?
\par 3 Przepasz teraz jako maz biodra swoje, a bede cie pytal, a ty mi daj sprawe.
\par 4 Gdziezes byl, kiedym Ja zakladal grunty ziemi? Powiedz, jezlize nasz rozum.
\par 5 Któz uczynil rozmierzenie jej? powiedz, jezli wiesz; albo kto sznur nad nia rozciagnal?
\par 6 Na czem sa podstawki jej ugruntowane? albo kto zalozyl kamien jej wegielny?
\par 7 Gdy wespól spiewaly gwiazdy zaranne, a weselili sie wszyscy synowie Bozy.
\par 8 Któz zamknal drzwiami morze, gdy sie wyrywalo, jakoby z zywota wychodzac?
\par 9 Gdym polozyl oblok za szate jego, a ciemnosc za pieluchy jego;
\par 10 Gdym postanowil o niem dekret mój, a przyprawilem zawore i drzwi do niego,
\par 11 I rzeklem: Az dotad wychodzic bedziesz, a dalej nie postapisz, a tu polozysz nadete waly twoje.
\par 12 Izazes za dni twoich rozkazywal switaniu, i ukazales zorzy miejsce jej?
\par 13 Aby ogarnela konczyny ziemi, a izby byli z niej wyrzuceni niepobozni.
\par 14 Aby sie odmieniala jako glina, do której pieczec przykladaja, a oni aby sie stali jako szata nakryci.
\par 15 I aby byla zawsciagniona od niepoboznych swiatlosc ich, a ramie wysokie bylo pokruszone.
\par 16 Izazes przyszedl az do zródel morskich, a po dnie przepasci przechodziles sie?
\par 17 Azaz odkryte sa tobie bramy smierci? bramy cienia smierci widzialzes?
\par 18 Izalis rozumem twym doszedl szerokosci ziemi? Powiedz mi, jezli to wszystko wiesz?
\par 19 Gdziez jest ta droga do miejsca swiatlosci? a ciemnosci gdzie maja miejsce swoje?
\par 20 Abys ja ujawszy odprowadzil do granicy jej, poniewaz zrozumiewasz scieszki do domu jej.
\par 21 Wiedzialzes na on czas, zes sie mial urodzic? i liczba dni twoich jak wielka byc miala?
\par 22 Izalis przyszedl do skarbów sniegów? aby skarby gradu widzalesli?
\par 23 Które zatrzymywam na czas ucisku, na dzien bitwy i wojny.
\par 24 Któraz sie droga dzieli swiatlosc, i gdzie sie rozchodzi wiatr wschodni po ziemi?
\par 25 Któz rozdzielil stok powodziom? a droge blyskawicy gromów?
\par 26 Aby szedl deszcz na ziemie, w której nikt nie mieszka, i na pustynie, gdzie niemasz czlowieka;
\par 27 Aby nasycil miejsce puste i nieplodne, a wywiódl z niego zielona trawe.
\par 28 Izali ma deszcz ojca? a krople rosy kto plodzi?
\par 29 Z czyjegoz zywota wychodzi mróz? a szron niebieski któz plodzi?
\par 30 Jakoz sie kamieniem wody nakrywaja, gdy wierzch przepasci zamarza.
\par 31 Mozeszze zwiazac jasne gwiazdy Bab? albo zwiazek Oryjona rozerwac?
\par 32 Izali wywiedziesz gwiazdy poludniowe czasu swego, albo Wóz niebieski z gwiazdami jego powiedziesz?
\par 33 I znaszze porzadek nieba? a mozeszze rozrzadzic panowanie jego na ziemi?
\par 34 Izali podniesiesz ku oblokowi glos twój, aby cie wielkosc wód okryla?
\par 35 Izali mozesz wypuscic blyskawice, aby przyszly, i rzeklyc: Otosmy?
\par 36 Któz zlozyl we wnetrznosciach ludzkich madrosc, a kto dal rozumowi bystrosc?
\par 37 Któz obrachowal niebiosa madroscia swoja? a co sie leje z nieba, któz uspokoi?
\par 38 Aby polany proch stezal, a bryly aby sie spolu zelgnely?

\chapter{39}

\par 1 Izali lwowi lup lowisz, a lwiat zywot napelniasz?
\par 2 Gdy sie tula w jaskiniach swoich, i czyhaja w cieniu jam swoich?
\par 3 Któz gotuje krukowi pokarm jego, gdy dzieci jego do Boga wolaja a tulaja sie, nie majac pokarmu?
\par 4 Izali wiesz czas rodzenia kóz skalnych, a kiedy rodza lanie, postrzeglzes?
\par 5 Mozeszze zliczyc miesiace, jako dlugo plód nosza? a czas rodzenia ich wieszze?
\par 6 Jako sie kurcza, plód swój wyciskaja, a rozstepujac sie z bolescia go pozbywaja;
\par 7 Jako moc biora dzieci ich, i odchowywuja sie po zbozach, a odszedlszy nie wracaja sie do nich.
\par 8 Któz wypuscil osla dzikiego na wolnosc? a peta osla dzikiego któz rozwiazal?
\par 9 Któremu dal pustynie miasto domu jego, a miasto mieszkania jego miejsca slone.
\par 10 On sie nasmiewa ze zgrai miejskiej, a na glos tego, co go goni, nic niedba.
\par 11 Patrzy po górach pastwy, a wszelkiej zielonej trawy szuka.
\par 12 Izalic bedzie chcial jednorozec sluzyc, albo bedzie nocowal u jasli twoich?
\par 13 Izali mozesz zaprzadz w powróz swój jednorozca do orania? izali powleka bedzie brózdy za toba?
\par 14 Izali sie spuscisz nan, przeto, ze wielka moc jego? albo poruczyszli mu robote twoje?
\par 15 Powierzyszze mu sie, zeby zwiózl nasienie twoje, a do gumna twojego zgromadzil?
\par 16 Izalis dal pawiowi piekne skrzydla, a pierze bocianowi i strusiowi?
\par 17 Który niesie na ziemi jajka swoje, a w prochu ogrzewa je.
\par 18 A nie pomni na to, ze je noga zetrzec, a zwierze polne zdeptac moze.
\par 19 Zatwardza sie przeciwko dzieciom swoim, jakoby nie byly jego, a zeby nie byla prózna praca jego, nie obawia sie.
\par 20 Bo mu nie dal Bóg madrosci, i nie udzielil mu wyrozumienia.
\par 21 Wedlug czasu podnosi sie ku górze, a nasmiewa sie z konia i z jezdzca jego.
\par 22 Izali mozesz dac koniowi moc? izali rzaniem ozdobisz szyje jego?
\par 23 Izali go ustraszysz jako szarancze? i owszem chrapanie nozdrzy jego jest straszne.
\par 24 Kopie dól, a weseli sie w mocy swej, i biezy przeciwko zbrojnym.
\par 25 Smieje sie z postrachu, a ani sie leka, ani nazad ustepuje przed ostrzem miecza.
\par 26 Choc na nim chrzesci sajdak, i blyszczy sie oszczep, i drzewce.
\par 27 Z grzmotem i z gniewem kopie ziemie, a nie stoi spokojnie na glos traby.
\par 28 Miedzy trabami poryza, a z daleka czuje bitwe, krzyk ksiazat, i wolanie.
\par 29 Izali wedlug twego rozumu lata jastrzab, i rozciaga skrzydla swe ku poludniowi?
\par 30 Izali na twoje rozkazanie wzbija sie orzel w góre, i sklada na wysokich miejscach gniazdo swoje?
\par 31 Na opoce mieszka, i bawi sie na ostrej skale, jako na zamku.
\par 32 Stamtad upatruje sobie pokarm, a daleko oczy jego widza.
\par 33 Dzieci tez jego pija krew, a gdzie sa pobici, tam on jest.
\par 34 A tak odpowiedzial Pan Ijobowi, i rzekl:
\par 35 Izali ten, co wiedzie spór z Wszechmogacym, uczyc go bedzie? a kto chce strofowac Boga, niech na to odpowie.
\par 36 Zatem odpowiedzial Ijob Panu, i rzekl:
\par 37 Otom ja lichy, cóz ci mam odpowiedziec? Reke moje wloze na usta moje.
\par 38 Mówilem raz i drugi, ale wiecej nie odpowiem, i nic wiecej nie przydam.

\chapter{40}

\par 1 Nadto odpowiedzial Pan Ijobowi z wichru, i rzekl:
\par 2 Przepasz teraz jako maz biodra swe: bede cie pytal, a ty mi daj sprawe;
\par 3 Izali wniwecz obrócisz sad mój? a obwinisz mie, abys sie sam usprawiedliwil?
\par 4 Izali masz ramie jako Bóg? a glosem zagrzmisz jako on?
\par 5 Ozdóbze sie teraz zacnoscia i dostojnoscia, a w chwale i w ochedóstwo oblecz sie.
\par 6 Rozpostrzyj popedliwosc gniewu twego, a patrz na kazdego pysznego, i poniz go.
\par 7 Spojrzyjze na kazdego hardego a skróc go, a zetrzyj niepoboznych na miejscu ich.
\par 8 Zakryj ich pospolu w prochu, a oblicza ich zawiaz w skrytosci.
\par 9 Tedyc i Ja przyznam, ze cie moze zachowac prawica twoja.
\par 10 Oto teraz slon, któregom uczynil jako i ciebie, trawe je jako wól.
\par 11 Oto teraz moc jego jest w biodrach jego, a sila jego w pepku brzucha jego.
\par 12 Rusza ogonem swoim, jako chce, choc jest jako drzewo cedrowe; zyly lona jego sa powiklane jako latorosli.
\par 13 Kosci jego jako traby miedziane; gnaty jego jako drag zelazny.
\par 14 On jest przedniejszym z uczynków Bozych; który go uczynil, sam nan natrzec moze mieczem swoim.
\par 15 Jemuc pastwe góry przynosza, a wszystek zwierz polny tam igra.
\par 16 Pod cienistem drzewem lega w skrytosciach trzciny i blota.
\par 17 Okrywaja go drzewa cieniste cieniem swoim, a ogarniaja go wierzby nad potokami.
\par 18 Oto zatrzymuje strumien, ze sie nie spieszy; tuszy sobie, iz Jordan wypije geba swoja.
\par 19 Azali go kto przed oczyma jego ulapi? albo powrozy przeciagnie przez nozdrze jego?
\par 20 Wyciagnieszze weda wieloryba? albo sznurem utopionym w jezyku jego?
\par 21 Izali zawleczesz kolce przez nozdrza jego? albo hakiem przekoleszli czelusc jego?
\par 22 Izalic sie bedzie wiele modlil, albo z toba lagodnie mówic bedzie?
\par 23 Izali uczyni przymierze z toba, a przyjmiesz go za sluge wiecznego?
\par 24 Izali z nim bedziesz igral jako z ptaszkiem, a uwiazesz go dziatkom twoim?
\par 25 Sprawize sobie nad nim towarzystwo uczte, a podziela go miedzy kupców?
\par 26 Izali zawadzisz hakami za skóre jego, a widelcami rybackiemi za glowe jego?
\par 27 Polóz tylko nan reke twa, slubujec, ze nie wspomnisz wiecej na bitwe.
\par 28 Oto nadzieja ulowienia jego omylna jest; izali i wejrzawszy nan czlowiek nie upada?

\chapter{41}

\par 1 Niemasz tak smialego, coby go obudzil; owszem któz sie stawi przed twarza moja?
\par 2 Któz mi co dal, abym mu oddal? cokolwiek jest pod wszystkiem niebem, moje jest.
\par 3 Nie zamilcze czlonków jego, ani silnej mocy jego, a grzecznego ksztaltu jego.
\par 4 Któz odkryje wierzch odzienia jego? z dwoistemi wedzidlami swemi któz przystapi do niego?
\par 5 Wrota geby jego któz otworzy? bo strach okolo zebów jego.
\par 6 Luski jego mocne jako tarcze, bardzo scisle spojone.
\par 7 Jedna z druga tak spojona, ze wiatr nie wchodzi miedzy nie.
\par 8 Jedna do drugiej przylgnela, ujely sie, a nie dziela sie.
\par 9 Kichanie jego czyni blask, a oczy jego sa jako powieki zorzy.
\par 10 Z ust jego lampy wychodza, a iskry ogniste wyrywaja sie.
\par 11 Z nozdrzy jego wychodzi dym, jako z garnca wrzacego, albo kotla.
\par 12 Dech jego wegle rozpala, a plomien z ust jego wychodzi.
\par 13 W szyi jego przemieszkuje moc, a bolesc przed nim ucieka.
\par 14 Sztuki ciala jego spoily sie, calowite sa w nim, ze sie nie porusza.
\par 15 Serce jego twarde jako kamien, tak twarde, jako sztuka spodniego kamienia mlynskiego.
\par 16 Gdy sie podnosi, drza mocarze, a od strachu oczyszczaja sie.
\par 17 Miecz, który go siega, nie ostoi sie, ani drzewce, ani strzala, ani pancerz.
\par 18 Zelazo poczyta sobie za plewe, a miedz za drzewo zbótwiale.
\par 19 Nie uploszy go strzala, a jako zdzblo sa u niego kamienie z procy.
\par 20 Strzelbe sobie poczyta jako slome, a posmiewa sie z szermowania wlócznia.
\par 21 Pod nim sa ostre skorupy; sciele sobie na rzeczach ostrych jako na blocie.
\par 22 Czyni, ze wre glebokosc jako garniec, a ze sie maci morze jako w mozdzierzu.
\par 23 Za soba jasna scieszke czyni, tak, ze sie zdaje, iz przepasc ma siwizne.
\par 24 Niemasz na ziemi równego mu, który tak stworzony jest, ze sie niczego nie boi.
\par 25 Wszelka rzecz wysoka lekce wazy; on jest królem nad wszystkiemi srogiemi zwierzetami.

\chapter{42}

\par 1 Tedy odpowiedzial Ijob Panu, i rzekl:
\par 2 Wiem, ze wszystko mozesz, i nie moze byc zahamowany zamysl twój.
\par 3 Któz jest ten, pytasz, który zaciemnia rade Boza nieumiejetnie? Dlatego przyznaje, zem nie zrozumial; dziwniejsze sa te rzeczy, nizbym je mógl pojac i zrozumiec.
\par 4 Wysluchajze, prosze, gdybym mówil; a gdy sie bede pytal, oznajmijze mi.
\par 5 Przedtem tylko ucho slyszalo o tobie; ale teraz oko moje widzi cie.
\par 6 Przetoz zaluje i pokutuje w prochu i w popiele.
\par 7 A gdy odmówil Pan te slowa do Ijoba, rzekl Pan do Elifasa Tamanczyka: Rozpalil sie gniew mój przeciw tobie, i przeciw dwom przyjaciolom twoim, zescie o mnie nie mówili tak przystojnie, jako Ijob, sluga mój.
\par 8 Przetoz teraz, wezmijcie sobie siedm cielców, i siedm baranów, a idzcie do slugi mego Ijoba, i ofiarujcie calopalenie za sie; a Ijob, sluga mój, niech sie modli za wami; bo oblicze jego przyjme, abym nie uczynil z wami wedlug glupstwa waszego; boscie nie mówili tak przystojnie o mnie, jako Ijob, sluga mój.
\par 9 A tak odeszli Elifas Temanczyk, i Bildad Suhitczyk, i Sofar Naanatczyk, i uczynili, jako im rozkazal Pan; i przyjal Pan oblicze Ijobowe.
\par 10 Zatem Pan przywrócil to, co bylo pobrane Ijobowi, gdy sie modlil za przyjaciól swoich; i rozmnozyl Pan wszystko, cokolwiek mial Ijob, w dwójnasób.
\par 11 Zeszli sie tedy do niego wszyscy bracia jego, i wszystki siostry jego, i inni wszyscy, którzy go przedtem znali, i jedli z nim chleb w domu jego, a zalujac go cieszyli go z strony wszystkiego zlego, które byl Pan nan przywiódl; i dal mu kazdy z nich upominek jeden, i kazdy nausznice zlota jedne.
\par 12 A tak Pan blogoslawil ostatnim czasom Ijobowym, wiecej niz poczatkom jego. Bo mial czternascie tysiecy owiec, i szesc tysiecy wielbladów, i tysiac jarzm wolów, i tysiac oslic.
\par 13 Mial tez siedm synów, i trzy córki.
\par 14 I dal imie pierwszej Jemina, a imie drugiej Kietzyja, a imie trzeciej Kierenhappuch.
\par 15 A nie znajdowaly sie niewiasty tak piekne, jako córki Ijobowe, we wszystkiej onej ziemi; i dal im ojciec ich dziedzictwo miedzy bracmi ich.
\par 16 Potem Ijob zyl sto i czterdziesci lat, i ogladal synów swych, i synów synów swoich, az do czwartego pokolenia.
\par 17 A umarl Ijob, bedac starym i dni sytym.


\end{document}