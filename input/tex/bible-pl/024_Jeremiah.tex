\begin{document}

\title{Jeremiasza}


\chapter{1}

\par 1 Slowa Jeremijasza, syna Helkijaszowego, z kaplanów, którzy byli w Anatot, w ziemi Benjamin
\par 2 Do którego stalo sie slowo Panskie za dni Jozyjasza, syna Amonowego, króla Judzkiego trzynastego roku królowania jego.
\par 3 A stalo sie za dni Joakima, syna Jozyjaszowego, króla Judzkiego, az do skonczenia jedenastego roku Sedekijasza, syna Jozyjasza, króla Judzkiego, az do przeprowadzenia Jeruzalemczyków, miesiaca piatego,
\par 4 Stalo sie mówie, slowo Panskie do mnie, mówiac
\par 5 Pierwej nizelim cie utworzyl w zywocie, znalem cie, a pierwej nizelis wyszedl z zywota, poswiecilem cie, za proroka narodom dalem cie.
\par 6 I rzeklem: Ach, ach panujacy Panie! Oto nie umiem mówic, bom jest dziecieciem.
\par 7 Ale Pan rzekl do mnie: Nie mów: Jestem dziecieciem, owszem, na wszystko, na co cie posle, idz, a wszystko, coc rozkaze, mów
\par 8 Nie bój sie oblicza ich, bom Ja jest z toba, abym cie wybawil, mówi Pan.
\par 9 A wyciagnawszy Pan reke swoje, dotknal sie ust moich, i rzekl mi Pan: Otom dal slowa moje do ust twoich.
\par 10 Oto cie dzis postanawiam nad narodami i nad królestwami, abys wykorzenial, i psul, i wytracal, i obalal, i abys budowal i szczepil.
\par 11 Potem stalo sie slowo Panskie do mnie, mówiac:Co widzisz Jeremijaszu? I rzeklem: Widze rózge migdalowa.
\par 12 I rzekl Pan do mnie: Dobrze widzisz: albowiem sie Ja pospieszam z slowem swem, abym je wykonal.
\par 13 I stalo sie slowo Panskie do mnie powtóre, mówiac: Co widzisz? I rzeklem: Widze garniec wrzacy, a przednia strona jego ku stronie pólnocnej.
\par 14 I rzekl Pan do mnie: Od pólnocy przypadnie zle na wszystkich mieszkajacych na tej ziemi.
\par 15 Bo oto Ja zawolam wszystkich rodzajów z królestw pólnocnych, mówi Pan, aby przyciagnawszy kazdy z nich, postanowil stolice swoje w wejsciu bram Jeruzalemskich i przy wszystkich murach jego, i przy wszystkich miastach Judzkich.
\par 16 A tak opowiem sady moje przeciwko nim dla wszelakiej zlosci tych, którzy mie opuscili, a kadzili bogom obcym, i klaniali sie robocie rak swoich.
\par 17 Przetoz ty przepasz biodra swoje, a wstawszy mów do nich wszystko, co Ja tobie rozkazuje, nie bój sie ich bym cie snac nie starl przed obliczem ich,
\par 18 Bo oto Ja postanawiam cie dzis miastem obronnem, i slupem zelaznym, i murem miedzianym przeciwko tej wszystkiej ziemi, przeciwko królom Judzkim, przeciwko ksiazetom ich, przeciwko kaplanom ich, i przeciwko ludowi tej ziemi;
\par 19 Którzy walczyc beda przeciwko tobie, ale cie nie przemoga: bom Ja z toba mówi Pan, abym cie wybawil.

\chapter{2}

\par 1 I stalo sie slowo Panskie do mnie, mówiac:
\par 2 Idz, a wolaj w uszy Jeruzalemskie, mówiac: Tak mówi Pan: Wspomnialem na cie dla milosierdzia mlodosci twojej pokazanego, i dla milosci slubin twoich; gdys chodzila za mna na puszczy, w ziemi, w której nie osiewaja.
\par 3 Kiedy Izrael byl swiatobliwoscia Panu, i pierwocinami urodzajów jego; wszyscy, którzy go pozerali, winni byli, zle rzeczy przyszly na nich, mówi Pan.
\par 4 Sluchajcie slowa Panskiego, domie Jakóbowy, i wszystkie rodzaje domu Izraelowego!
\par 5 Tak mówi Pan: Jakaz nieprawosc znalezli ojcowie wasi przy mnie, iz sie oddalili odemnie, a chodzac za marnoscia marnymi sie stali?
\par 6 Tak iz ani rzekli: Gdziez jest Pan, który was wywiódl z ziemi Egipskiej? który nas wodzil po puszczy, po ziemi pustej i strasznej, po ziemi suchej i cieniu smierci, po ziemi, po której nikt nie chodzil, a gdzie zaden czlowiek nie mieszkal?
\par 7 Owszem, gdym was wprowadzil do ziemi obfitej, abyscie pozywali owoców jej, i dóbr jej, wszedlszy tam splugawiliscie ziemie moje, a dziedzictwo moje uczyniliscie obrzydliwoscia.
\par 8 Kaplani nie rzekli: Gdziez jest Pan? ani ci, którzy sie obieraja uczeni w zakonie, poznali mie, i pasterze odstapili odemnie, i prorocy prorokowali przez Baala, i za rzeczami nie uzytecznemi chodzili.
\par 9 Przeczze sie wzdy wadze z wami, mówi Pan, a z synami synów waszych spierac sie musze?
\par 10 Przejdzcie przynajmniej wyspy Cytym, a obaczcie; i do Kedar poslijcie a uwazajcie pilnie, i przypatrzcie sie, jezli sie stalo co takowego;
\par 11 Jezli odmienil który naród bogów swoich, chociaz oni nie sa bogami; ale lud mój odmienil slawe swoje w rzecz niepozyteczna.
\par 12 Zdumiejcie sie niebiosa nad tem, a uleknijcie sie, a zatrwozcie sie bardzo, mówi Pan;
\par 13 Bo dwojaka zlosc popelnil lud mój: mnie opuscili, zródlo wód zywych, a wykopali sobie cysterny, cysterny dziurawe, które wody zatrzymac nie moga.
\par 14 Izali Izrael jest niewolnikiem albo wychowancem w domu splodzonym? Czemuz jest podany na lup?
\par 15 Rycza nan lwieta, i wydawaja glos swój, a obracaja ziemie jego w pustynie; miasta jego spalone sa, tak, ze niemasz i jednego obywatela.
\par 16 Synowie tez Nof i Tachpanes wierzch glowy twojej zetra.
\par 17 Zaz tego sam sobie nie sprawujesz? opuszczajac Pana, Boga swego, wtenczas, kiedy cie prowadzi droga swa.
\par 18 A teraz co za sprawe masz na drogach Egipskich, iz pijesz wode z Nilu? albo co masz za sprawe na drogach Assyryjczyków, iz pijesz wode z rzeki ich?
\par 19 Skarze cie zlosc twoja a odwrócenie twoje sfuka cie. Wiedzze tedy i obacz, iz jest rzecz zla i gorzka, izes opuscil Pana, Boga twego, a niemasz bojazni mojej w tobie, mówi Pan, Pan zastepów.
\par 20 Gdym dawno polamal jarzmo twoje, i rozerwalem zwiazki twoje mówilas: Nie bede sluzyla balwanom; a przecie na kazdym pagórku wysokim, i pod kazdem drzewem galezistem tulasz sie, o nierzadnico!
\par 21 A Jam cie byl nasadzil winna macica wyborna, którejby wszystko nasienie bylo prawdziwe; jakozes mi sie tedy odmienila w plonne galezie obcej macicy?
\par 22 Bo chocbys sie umywala i saletra, i mydlem sie jako najbardziej tarla, przeciez znaczna zostanie nieprawosc twoja przedemna, mówi panujacy Pan.
\par 23 Jakoz mówisz: Nie jestem splugawiona, za Baalami nie chodzilam? Spojrzyj na droge twoje w tej dolinie, obacz, cos czynila, o wielbladzico predka, która wiklesz drogi swoje?
\par 24 Oslicas dzika, przywyklas na puszczy, która wedlug zadzy duszy swej wiatr lapie, gdy sie jej przyczyna da; któz ja odwróci? Wszyscy którzy jej szukaja, nie strudza sie, i w miesiacu jej znajda ja.
\par 25 Rzekelic: Zawsciagnij nogi twojej, aby bosa nie byla, i gardlo twe od pragnienia, tedy mówisz: Juz to prózno, nie uczynie; bom sie rozmilowala w cudzych, i za nimi pójde.
\par 26 Jako wstyd zlodzieja, kiedy go zastana, tak sie zawstydzi dom Izraelski, sami królowie ich, ksiazeta ich, i kaplani ich, i prorocy ich,
\par 27 Którzy mówia drewnu: Tys jest ojciec mój, a kamieniowi: Tys mie splodzil. Bo sie do mnie obrócili tylem, a nie twarza; ale czasu utrapienia swego mawiaja: Wstan a wybaw nas.
\par 28 I gdziez sa bogowie twoi, któryches sobie naczynil? Niech wstana, jezli cie moga wybawic czasu utrapienia twego, poniewaz ile masz miast swoich, tyle masz bogów swoich, o Judo!
\par 29 Czemuz sie zemna spierac chcecie? Wyscie wszyscy odstapili odemnie, mówi Pan.
\par 30 Próznom bil synów waszych, karania nie przyjeli; miecz wasz pozarl przecie proroków waszych, jako lew tracacy.
\par 31 O narodzie! wy rozsadzcie slowa Panskie. Izalim byl pustynia Izraelowi? Izali ziemia ciemna? Przeczze mówi lud mój: Panujemy, nie pójdziemy wiecej do ciebie?
\par 32 Izali zapomina panna ubioru swego, i oblubienica klejnotów swoich? Ale lud mój zapomnial mie przez dni niezliczone.
\par 33 Przecz dobra byc twierdzisz droge twoje, szukajac tego, w czem sie kochasz? Przecz i innych nierzadnic uczysz zlosliwych dróg twoich?
\par 34 Nadto i na podolkach twoich znajduje sie krew dusz ubogich i niewinnych; nie z praca znalazlem to, bo to widziec na wszystkich podolkach twoich.
\par 35 A przeciez mówisz: Poniewazem niewinna, pewnie odwrócona jest zapalczywosc jego odemnie. Oto Ja w sad wnijde z toba, przeto, ze mówisz: Nie zgrzeszylam.
\par 36 Przeczze tak biegasz, odmieniajac drogi swe? Tak bedziesz pohanbiona od Egipczanów, jakos pohanbiona byla od Assyryjczyków.
\par 37 I stamtad wyjdziesz, majac rece swe nad glowa swa: bo Pan odrzuca ufnosci twoje, a nie poszczescic sie w nich.

\chapter{3}

\par 1 Pan mówi: Opuscilliby maz zone swoje, a ona odszedlszy od niego szlaby za innego meza, izali sie wiecej do niej wróci? Izaliby nie byla wielce splugawiona ona ziemia? Ale ty, chociazes nierzad plodzila z wiela zalotników, wszakze nawróc sie d o mnie, mówi Pan.
\par 2 Podnies oczu swych na miejsca wysokie, a obacz, jezlis gdzie nierzadu nie plodzila. Na drogach siadalas kwoli nim, jako Arabczyk na puszczy, a splugawilas ziemie wszeteczenstwem twem, i zloscia twoja.
\par 3 A chociaz zawsciagnione sa dzdze jesienne, a deszczu na wiosne nie bywalo, przeciezes czolo niewiasty wszetecznej majac, nie chcialas sie wstydzic.
\par 4 Azaz od tego czasu wolac bedziesz na mnie: Ojcze mój! Tys wodzem mlodosci mojej?
\par 5 Izali Bóg zatrzyma gniew na wiecznosc? Izali go zachowa na wieki? Oto mówisz i czynisz zle, ile mozesz.
\par 6 Tedy Pan rzekl do mnie za dni Jozyjasza króla: Widzialzes, co uczynila odporna córka Izraelska? jako chodzila na kazda góre wysoka, i pod kazde drzewo zielone, i tam nierzad plodzila.
\par 7 A chociazem rzekl, gdy to wszystko uczynila: Nawróc sie do mnie! przecie sie nie nawrócila; a na to patrzyla przestepnica siostra jej, córka Judzka.
\par 8 A tak zdalo mi sie dla tych wszystkich przyczyn, poniewaz nierzad plodzila uporna córka Izraelska, opuscic ja, i dac jej list rozwodny; a przeciez sie nie ulekla przestepnica siostra jej, córka Judzka, ale szedlszy i sama nierzad plodzila.
\par 9 I stalo sie, ze haniebnym nierzadem swoim splugawila ziemie; bo cudzolozyla z kamieniem i z drewnem.
\par 10 A wszakze w tem wszystkiem nie nawrócila sie do mnie przestepnica siostra jej, córka Judzka, z wszystkiego serca swego; ale obludnie, mówi Pan.
\par 11 Przetoz rzekl Pan do mnie: Usprawiedliwila dusze swa odporna córka Izraelska wiecej, nizeli przestepnica Judzka.
\par 12 Idzze, a wolaj temi slowy ku pólnocy a mów: Nawróc sie, odporna córko Izraelska! mówi Pan, a nie obórzy sie twarz moja surowa na was, bom Ja dobrotliwy, mówi Pan, a nie chowam gniewu na wieki.
\par 13 Tylko uznaj nieprawosc twoje, zes od Pana, Boga swego, odstapila, a tam i sam biegala drogami swemi do obcych bogów pod kazde drzewo zielone, a glosu mojego nie sluchaliscie, mówi Pan.
\par 14 Nawrócciez sie, synowie uporni! mówi Pan; bom Ja jest malzonkiem waszym, a przyjme was jednego z miasta, a dwóch z rodzaju, abym was wprowadzil do Syonu,
\par 15 Gdzie wam dam pasterzy wedlug serca mego, i beda was pasc umiejetnie i rozumnie.
\par 16 I stanie sie, gdy sie rozmnozycie a rozrodzicie sie w tej ziemi w onez dni, mówi Pan, nie beda wiecej mówic: "Skrzynia przymierza Panskiego", ani wstapi na serce, ani wspomna na nia, ani jej nawiedzac, ani jej wiecej powazac beda.
\par 17 Czasu onego nazwane bedzie Jeruzalem stolica Panska, a zgromadza sie do niego wszystkie narody, do imienia Panskiego, do Jeruzalemu, i nie beda wiecej chodzic za uporem serca swego zlosliwego.
\par 18 W one dni pójda dom Judzki z domem Izraelskim, i przyjda pospolu z ziemi pólnocnej do ziemi, któram dal w dziedzictwo ojcom waszym.
\par 19 Chociazem Ja rzekl: Jakozbym cie polozyl miedzy synami, a dal ci ziemie pozadana, dziedzictwo zacne zastepów poganskich? chyba zebys mie wzywal, mówiac: Ojcze mój! a od nasladowania mnie nie odwrócil sie.
\par 20 Poniewaz jako zona przeniewierza sie mezowi swemu, takescie mi sie przeniewierzyli, o domie Izraelski! mówi Pan.
\par 21 Glos na wysokich miejscach niech bedzie slyszany, placz modlitw synów Izraelskich; bo przewrotne uczyniwszy drogi swe zapamietali na Pana, Boga swego,
\par 22 Mówiacego: Nawróccie sie, synowie odporni! a ulecze odwrócenia wasze; mówcie: Oto my idziemy do ciebie, bos ty jest Pan, Bóg nasz.
\par 23 Zaiste prózna jest nadzieja w pagórkach i w mnóstwie gór; zaiste w Panu, Bogu naszym, jest zbawienie Izraelskie.
\par 24 Bo ta hanba pozarla prace ojców naszych od mlodosci naszej, trzody ich, i stada ich, syny ich, i córki ich.
\par 25 Lezymy w pohanbieniu swem, a przykrywa nas zelzywosc nasza; albowiemsmy przeciwko Panu, Bogu naszemu, zgrzeszyli, my i ojcowie nasi, od mlodosci naszej az do dnia tego, a nie usluchalismu glosu Pana, Boga naszego.

\chapter{4}

\par 1 Jezlibys sie chcial nawrócic, Izraelu! mówi Pan, do mnie sie nawróc. Bo jezli odejmiesz obrzydlosci twoje od oblicza mego, a nie bedziesz sie tulal,
\par 2 I przysiezeszli w prawdzie, w sadzie i w sprawiedliwosci, mówiac: Jako zyje Pan; tedy blogoslawic sobie w nim beda narody, i w nim sie przechwalac.
\par 3 Albowiem tak mówi Pan mezom Judzkim i Jeruzalemskim: Poorzcie sobie nowine, a nie siejcie na cierniu;
\par 4 Obrzezcie sie Panu, a odejmijcie nieobrzezki serca waszego, mezowie Judzcy, a obywatele Jeruzalemscy! by snac nie wyszla jako ogien popedliwosc moja, a nie zapalila sie, a nie bylby, ktoby ugasil dla zlosci przedsiewziecia waszego.
\par 5 Opowiadajcie w Judzie, a w Jeruzalemie oglaszajcie, i mówcie: Zatrabcie w trabe w ziemi, zwolajcie, zgromadzcie lud, a mówcie: Zbierzcie sie, a wejdzmy do miast obronnych.
\par 6 Podniescie choragiew na Syonie, badzcie serca dobrego, nie zastanawiajcie sie; bo Ja zle przywiode od pólnocy, i porazke wielka.
\par 7 Wychodzi lew z jaskini swojej, a ten, który niszczy narody, wyszedlszy z miejsca swego, ciagnie, aby obrócil ziemie twoje w pustynie, a miasta twoje aby zburzone byly, aby nie bylo zadnego obywatela.
\par 8 Przetoz przepascie sie worami, narzekajcie a kwilcie, bo nie jest odwrócony gniew zapalczywosci Panskiej od nas.
\par 9 I stanie sie dnia onego, mówi Pan, ze zginie serce królewskie, i serce ksiazat, a zdumieja sie kaplani, i prorocy dziwowac sie beda.
\par 10 I rzeklem: Ach panujacy Panie! zaprawdes bardzo ten lud i Jeruzalem omylil, mówiac: Pokój miec bedziecie! a wzdy miecz przeniknal az do duszy.
\par 11 Czasu onego rzeka temu ludowi i Jeruzalemowi: Wiatr gwaltowny z miejsc wysokich na puszczy idzie prosto na lud mój, nie zeby przewiewal, ani wyczyszczal.
\par 12 Wiatr gwaltowniejszy niz oni przyjdzie mi; teraz Ja tez opowiem im sady.
\par 13 Oto wystepuje jako obloki, a wozy jego jako wicher, predsze sa niz orly konie jego. Biada nam! bosmy spustoszeni.
\par 14 Omyj od zlosci serce twoje, Jeruzalem! abys wybawione bylo. Dokadze trwac beda w posrodku ciebie mysli nieprawosci twojej.
\par 15 Bo glos opowiadajacego idzie od Danu, a tego, który oglasza nieprawosc, z góry Efraim.
\par 16 Przypominajciez tym narodom: Oto oglaszajcie Jeruzalemczykom, ze strózowie przychodza z ziemi dalekiej, a wydawaja przeciwko miastom Judzkim glos swój.
\par 17 Jako strózowie pól poloza sie przeciwko niemu w okolo; bo mie do gniewu wzruszylo, mówi Pan.
\par 18 Droga twoja i postepki twoje to uczynily tobie; toc zlosc twoja przyniosla, ze to jest gorzkie, a ze przenika az do serca twego.
\par 19 O wnetrznosci moje, wnetrznosci moje! bolesc cierpie. O osierdzia moje! trwozy sie we mnie serce moje, nie zamilcze; bo glos traby slyszysz, duszo moja! i okrzyk wojenny.
\par 20 Porazka za porazka oglasza sie, spustoszona bedzie zaiste wszystka ziemia; nagle spustoszone beda namioty moje, i opony moje w okamgnieniu.
\par 21 Dokadze widziec bede choragiew, i slyszec glos traby?
\par 22 Bo glupi lud mój nie zna mie, synowie niemadrzy i nierozumni sa; madrzy sa do czynienia zlego, ale dobrze czynic nie umieja.
\par 23 Spojrzeli na ziemie, a oto jest niepozorna i prózna; jezeli na niebo, nie masz na niem swiatlosci.
\par 24 Spojrzeli na góry, a oto sie trzesa, i wszystkie pagórki chwieja sie.
\par 25 Spojrzeli, a oto niemasz czlowieka, i wszelkie ptastwo niebieskie odlecialo.
\par 26 Spojrzeli, a oto pole urodzajne jest pustynia, a wszystkie miasta jego zburzone sa od oblicza Panskiego, i od oblicza gniewu zapalczywosci jego.
\par 27 Bo tak mówi Pan: Spustoszona bedzie wszystka ziemia, wszakze konca jeszcze nie uczynie.
\par 28 Nad tem ziemia kwilic bedzie, a niebiosa w górze zacmia sie, przeto, zem mówil, com umyslil, a nie zaluje, ani sie odwróce od tego.
\par 29 Przed grzmotem jezdzców i strzelajacych z luku uciecze wszystko miasto; wejda do gestych obloków, i na skaly wstapia. Wszystkie miasta opuszczone beda, a nie bedzie, ktoby w nich mieszkal.
\par 30 A ty zburzona bedac cóz uczynisz? Chocbys sie ubrala w szarlat, chocbys sie ozdobila ozdoba zlota, chocbys tez oblicze twe przyprawila barwiczka, prózno sie stroisz; wzgardza toba zalotnicy twoi, a duszy twojej szukac beda.
\par 31 Bom slyszal glos jako rodzacej, uciski jako pierworodzacej, glos córki Syonskiej narzekajacej, a zalamujacej rece swe, mówiac: Biada mnie teraz! bo ustala dusza moja dla morderców.

\chapter{5}

\par 1 Obchodzcie ulice Jeruzalemskie, a upatrujcie teraz, i obaczcie, a szukajcie po ulicach jego, jezli znajdziecie meza, jezli kto jest coby czynil sad i szukal prawdy, a przepuszcze mu.
\par 2 Ale choc mówia: Jako zyje Pan, tedy przecie krzywo przysiegaja.
\par 3 O Panie! izali oczy twoje nie patrza na prawde? Bijesz ich, ale ich nie boli; wniwecz ich obracasz, ale nie chca przyjac karania; zatwardzili oblicza swe nad opoke, nie chca sie nawrócic.
\par 4 Tedym Ja rzekl: Podobno ci nedzni sa, glupio sobie poczynaja; bo nie sa powiadomi drogi Panskiej, i sadu Boga swego.
\par 5 Pójde do celniejszych, i bede mówil do nich; bo oni sa powiadomi drogi Panskiej, i sadu Boga swego; ale i ci wespól polamali jarzmo, potargali zwiazki.
\par 6 Przetoz ich pobije lew z lasu, wilk wieczorny wygubi ich, lampart czyhac bedzie u miast ich. Ktokolwiek wyjdzie z nich, rozszarpany bedzie; bo sie rozmnozyly przestepstwa ich, i zmogly sie odwrócenia ich.
\par 7 Jestze co, dlaczegobym ci mial przepuscic? Synowie twoi opuscili mie, a przysiegaja przez onych, którzy nie sa bogami. Jakom ich jedno nakarmil, zaraz cudzoloza, a do domu wszetecznicy hurmem sie wala.
\par 8 Rano wstawajac sa jako konie wytuczone, kazdy z nich rzy do zony blizniego swego.
\par 9 Izali dlatego nawiedzic ich nie mam? mówi Pan; izali sie nad takim narodem nie ma mscic dusza moja?
\par 10 Wstapcie na mury jego, a rozwalcie je, wszakze ich do gruntu nie znoscie; zniescie filarzyki murów jego, gdyz nie sa Panskie.
\par 11 Wielce zaiste wystapil przeciwko mnie dom Izraelski i dom Judzki, mówi Pan.
\par 12 Zadali klamstwo Panu, i rzekli: Nie tak, nie przyjdziec na nas nic zlego, a miecza i glodu nie doznamy.
\par 13 A ci prorocy pomina z wiatrem, a zadnego slowa Bozego niemasz u nich; i owszem tak sie im samym stanie.
\par 14 Przetoz tak mówi Pan, Bóg zastepów: Poniewazescie to mówili, oto Ja klade slowa moje w usta twoje za ogien, a lud ten za drwa, i pozre ich.
\par 15 Oto Ja przywiode na was naród z daleka, o domie Izraelski! mówi Pan, naród mocny, naród starodawny, naród, którego jezyka umiec nie bedziesz, ani zrozumiesz, co mówi.
\par 16 Którego sajdak jako grób otwarty, wszyscy sa mezni.
\par 17 I zjedza urodzaj twój, i chleb twój: pozra synów twoich i córki twoje; poje trzody twoje i woly twoje; poje winna macice twoje, i figi twoje, a miasta twoje obronne, w których ty ufasz, mieczem znedzi.
\par 18 A wszakze i w one dni, mówi Pan, konca z wami nie uczynie.
\par 19 Albowiem gdy rzeczecie: Przeczze nam Pan, Bóg nasz, to wszystko czyni? Tedy im odpowiesz: Jakoscie mie opuscili, a sluzyli bogom cudzym w ziemi waszej, tak sluzyc bedziecie cudzoziemcom w ziemi nie waszej.
\par 20 Oznajmijcie to domowi Jakóbowemu, a rozgloscie w Judzie, mówiac:
\par 21 Sluchajciez teraz tego, ludu glupi! który niemasz serca, który oczy majac, a nie widzisz, który uszy majac, a nie slyszysz.
\par 22 I nie bedzieciez sie mnie bali? mówi Pan; a przed obliczem mojem nie bedzieciez sie lekali? którym polozyl morzu piasek za granice ustawa wieczna, a nie przestapi jej. Choc sie wzrusza, wszakze nie przemoga; choc sie wzburza waly jego, wszakze nie przeskocza go.
\par 23 Ale ten lud ma serce ociazale i odporne; odstapili odemnie i odeszli;
\par 24 Ani rzekli w sercu swem: Bójmy sie juz Pana, Boga naszego, który daje deszcz i w jesieni i na wiosne czasu swego, który tygodni pewnych i zniwa naszego przestrzega.
\par 25 Ale nieprawosci wasze odwrócily to, a grzechy wasze zahamowaly to dobro od was.
\par 26 Bo sie znajduja w ludu moim niezboznicy, którzy czyhaja jako lowcy, rozciagaja sieci, zastawiaja sidla, a lapia ludzi.
\par 27 Jako klatka pelna ptaków, tak domy ich pelne sa zdrady; przetoz sie wzmogli i zbogacili.
\par 28 Roztyli, lsnia sie, i innych w zlosciach przewyzszaja; sprawy nie sadza, ani sprawy sierotki; wszakze sie im szczesci, chociaz sprawy ubogiego nie rozsadzili.
\par 29 Izali dlatego nie nawiedze ich? mówi Pan; izali sie nad narodem takowym nie bedzie mscic dusza moja?
\par 30 Rzecz dziwna i sroga dzieje sie w tej ziemi:
\par 31 Prorocy klamliwie prorokuja, a kaplani panuja przez rece ich, a lud mój kocha sie w tem; czegozbyscie na ostatek nie uczynili?

\chapter{6}

\par 1 Zgromadzcie sie, synowie Benjaminowi! z posrodku Jeruzalemu, a w Tekue trabcie w trabe, i nad Betcherem podniescie choragiew! bo zle ukazalo sie z pólnocy, i zburzenie wielkie.
\par 2 Pieknej, rozkosznej pannie przypodobalem byl córke Syonska;
\par 3 Ale do niej przyciagna pasterze i trzody ich; rozbija przeciwko niej namioty w okolo, spasie kazdy miejsce swoje, i rzeka:
\par 4 Podniescie przeciwko niej wojne, wstancie, a wtargniemy w poludnie; biada nam, ze sie nachylil dzien, ze sie rozciagnely cienie wieczorne!
\par 5 Wstancie, a wtargniemy w nocy, i rozwalmy palace jej.
\par 6 Bo tak mówi Pan zastepów: Narabcie drzewa, a usypcie przeciw Jeruzalemowi szance; toc to miasto jest, które ma byc nawiedzione; jakozkolwiek wielkie, niemasz jedno ucisk w posrodku jego.
\par 7 Jako zródlo wylewa wody swe, tak ono wylewa zlosc swoje; ucisk i spustoszenie slychac w niem przed obliczem mojem ustawicznie, bolesc i bicie.
\par 8 Çwicz sie Jeruzalemie! by snac nie odstapila dusza moja od ciebie, bym cie snac nie obrócil w pustynie ziemi do mieszkania niesposobna,
\par 9 Tak mówi Pan zastepów: Ostatek Izraela az do grona wyzbieraja, jako winnice, i rzeka: Siegaj reka twoja, jako ten, co zbiera wino do kosza.
\par 10 Do kogoz mówic bede, i kim oswiadcze, aby slyszeli? Oto nieobrzezane sa uszy ich, tak, ze sluchac nie moga; oto slowo Panskie maja za hanbe i nie kochaja sie w niem.
\par 11 Przetoz pelenem zapalczywosci Panskiej, upracowalem sie, zawsciagajac ja w sobie. Wylana bedzie tak na maluczkiego na ulicy, jako i na zebranie mlodzienców; owszem, i maz z zona, a starzec ze zgrzybialym pojmany bedzie.
\par 12 I przypadna domy ich na inszych, takze pola i zony ich, gdyz wyciagne reke moje na obywateli tej ziemi, mówi Pan.
\par 13 Zaiste, od najmniejszego z nich az do najwiekszego z nich, wszyscy sie udali za lakomstwem; od proroka az do kaplana, wszyscy zgola bawia sie klamstwem.
\par 14 I lecza skruszenie córki ludu mego tylko po wierzchu, mówiac: Pokój, pokój! choc niemasz pokoju.
\par 15 Izali sie zawstydzili, przeto ze obrzydlosc czynili? Zaiste ani sie lud wstydzil, ani ich prorocy do wstydu przywiesc mogli; przetoz upadna miedzy padajacymi; czasu, którego ich nawiedze, upadna, mówi Pan.
\par 16 Gdy tak Pan mawial: Zastanówcie sie na drogach, a spojrzyjcie i pytajcie sie o scieszkach starych, któraby byla droga dobra, a chodzcie nia, a znajdziecie odpocznienie duszy waszej: tedy odpowiadali: Nie bedziemy chodzili.
\par 17 A gdym postanawial nad wami strózów, mówiac: Sluchajcie glosu traby! tedy mawiali: Nie bedziemy sluchac.
\par 18 Przetoz sluchajcie, o narody! a poznaj, o zgromadzenie! co sie dzieje miedzy nimi.
\par 19 Sluchaj, o ziemio! Oto Ja przywiode zle na ten lud, owoce mysli ich, przeto, ze nie sluchaja slów moich, ani zakonu mego, ale go odrzucaja.
\par 20 Na cóz mi kadzidlo z Saby przychodzi, a cynamon wonny wyborny z ziemi dalekiej? Calopalenia wasze nie sa mi przyjemne, i ofiary wasze nie podobaja mi sie.
\par 21 Przetoz tak mówi Pan: Oto ja naklade ludowi temu zawad, o które sie otracac beda ojcowie, takze i synowie, sasiad i blizni jego, i pogina.
\par 22 Tak mówi Pan: Oto lud przyciagnie z ziemi pólnocnej, a naród wielki powstanie od konczyn ziemi;
\par 23 Luk i wlócznie pochwyci, okrutny bedzie, a nie zlituje sie. Glos ich jako morze zahuczy, a na koniach jezdzic beda, naród uszykowany jako maz do boju przeciwko tobie, o córko Syonska!
\par 24 Skoro uslyszymy wiesc o nim, oslabieja rece nasze, ucisk nas ogarnie, i bolesc jako rodzaca.
\par 25 Nie wychodzcie na pole, i w droge nie chodzcie; bo miecz nieprzyjacielski a strach w okolo.
\par 26 O córko ludu mojego! przepasz sie worem a walaj sie w popiele; uczyn sobie zal jako po jedynaku, zal gorzki; bo na nas nagle burzyciel przypadnie.
\par 27 Dalem cie za baszte i za wieze w ludu moim, abys upatrywal i doswiadczal drogi ich.
\par 28 Wszyscy sa miedzy krnabrnymi najkrnabrniejsi, chodza jako obmowca, sa jako miedz i zelaza; wszyscy zgola sa skazonymi.
\par 29 Murzszeja miechy, olów od ognia niszczeje, prózno ustawicznie zlotnik plawi; bo zle rzeczy nie moga byc oddalone.
\par 30 Srebrem falszywem beda nazwani; bo ich Pan odrzucil.

\chapter{7}

\par 1 Slowo, które sie stalo do Jeremijasza od Pana, mówiac:
\par 2 Staw sie w bramie domu Panskiego, a opowiadaj tam to slowo, i mów: Sluchajcie slowa Panskiego wszystek Judo, którzy wchodzicie do bram ich, abyscie sie klaniali Panu.
\par 3 Tak mówi Pan zastepów, Bóg Izraelski: Polepszajcie dróg swoich, i przedsiewziecia swego, a sprawie to, abyscie mieszkali na tem miejscu.
\par 4 Nie pokladajcie nadziei swej w slowach klamliwych, mówiac: Kosciól Panski, kosciól Panski, kosciól Panski jest!
\par 5 Ale jezlize polepszajac polepszycie dróg swoich, i przedsiewziecia swego; jezlize sprawiedliwy sad czynic bedziecie miedzy mezem a miedzy bliznim jego;
\par 6 Przychodnia, sierotki i wdowy nie ucisniecie, i krwi niewinnej nie rozlejecie na tem miejscu, a za bogami cudzymi nie pójdziecie na swe zle:
\par 7 Tedy sprawie, abyscie mieszkali na tem miejscu w ziemi, któram dal ojcom waszym, od wieku az na wieki.
\par 8 Oto wy pokladacie nadzieje swoje w slowach klamliwych, które nie pomoga.
\par 9 Izali kradnac, zabijajac, i cudzolozac, i krzywoprzysiegajac, i kadzac Baalowi, a chodzac za bogami obcymi, których nie znacie,
\par 10 Przeciez chodzic a stawac bedziecie przed obliczem mojem w tym domu, który nazwany jest od imienia mego, i mówic: Wybawienismy, abysmy czynili te wszystkie obrzydliwosci?
\par 11 Azaz jaskinia lotrowska jest dom ten przed oczyma waszemi, który nazwany jest od imienia mego? Oto widzec Ja to, mówi Pan.
\par 12 Ale idzcie przynajmniej na miejsce moje, które bylo w Sylo, gdziem byl sprawil przybytek imieniowi memu z poczatku, a obaczcie, com mu uczynil dla zlosci ludu mego Izraelskiego.
\par 13 Przetoz teraz, poniewaz czynicie te wszystkie sprawy, mówi Pan, a gdy mówie do was rano wstawajac a to ustawicznie, a nie sluchacie, gdy wolam na was, a nie ozywacie sie:
\par 14 Przetoz uczynie domowi temu, który nazwany jest od imienia mego, w którym wy ufacie, i miejscu temu, którem wam dal i ojcom waszym, jakom uczynil Sylo;
\par 15 I odrzuce was od oblicza mego, jakom odrzucil braci waszych, wszystko nasienie Efraimowe.
\par 16 Ty tedy nie módl sie za tym ludem, ani podnos za nim glosu modlitwy, i nie przyczyniaj sie do mnie; bo cie nie wyslucham.
\par 17 Azaz sam nie widzisz, co oni broja w miastach Judzkich i po ulicach Jeruzalemskich?
\par 18 Synowie zbieraja drwa, a ojcowie rozniecaja ogien, a zony ich rozczyniaja ciasto, aby czynily placki królowej niebieskiej, i sprawowaly mokre ofiary bogom cudzym, aby mie do gniewu pobudzali.
\par 19 Izali to przeciwko mnie jest, ze mie do gniewu wzruszaja? mówi Pan; izali to nie raczej przeciwko nim, ku pohanbieniu twarzy ich?
\par 20 Przetoz tak mówi panujacy Pan: Oto gniew mój i popedliwosc moja bedzie wylana na to miejsce, na ludzi i na bydleta, i na drzewa polne, i na owoce ziemi, i zapali sie, a nie ugasnie.
\par 21 Tak mówi Pan zastepów, Bóg Izraelski: Calopalenie wasze przydajcie do ofiar waszych, a jedzcie mieso.
\par 22 Bom nie mówil z ojcami waszymi, anim im przykazal onego dnia, któregom ich wywiódl z ziemi Egipskiej, o calopaleniu i ofiarach;
\par 23 Ale tom im przykazal, mówiac: Sluchajcie glosu mojego, i bede Bogiem waszym, a wy bedziecie ludem moim; a chodzcie kazda droga, któram wam przykazal, aby wam dobrze bylo.
\par 24 Lecz nie posluchali, ani naklonili ucha swego, aby chodzili za radami i za uporem serca swego zlego; i obrócili sie grzbietem, a nie twarza.
\par 25 Ode dnia, którego wyszli ojcowie wasi z ziemi egipskiej, az do dnia tego, posylalem do was wszystkich slug moich proroków, co dzien rano wstawajac i posylajac;
\par 26 A wszakze nie sluchali mie, i nie naklonili ucha swego, ale zatwardziwszy kark swój, gorzej czynili nizeli ojcowie ich.
\par 27 Gdy im bedziesz mówil te wszystkie slowa, i ciebie nie usluchaja; a gdy na nich wolac bedziesz, nie ozwac sie.
\par 28 Przetoz mów do nich: Ten jest naród, który nie slucha glosu Pana, Boga swego, ani przyjmuje nauki; zginela prawda, i odjeta jest od ust ich.
\par 29 Ogól wlosy swe i odrzuc, a narzekaj glosno na miejscach wysokich; bo odrzucil Pan i opuscil rodzaj, na który sie bardzo gniewa.
\par 30 Zaiste synowie Judzcy czynili zlosc przed oczyma mojemi, mówi Pan; nastawiali obrzydliwosci swych w tym domu, który nazwany jest od imienia mojego, aby go splugawili.
\par 31 Nadto pobudowali wyzyny Tofet, które jest w dolinie syna Hennomowego, aby palili synów swych i córki swe ogniem, czegom nie rozkazal, ani wstapilo na serce moje.
\par 32 Dlatego oto dni ida, mówi Pan, gdy to wiecej nie bedzie zwano Tofet, ani dolina syna Hennomowego, ale dolina morderstwa; i beda pogrzebywac w Tofet; bo indziej miejsca nie bedzie.
\par 33 I beda trupy ludu tego pokarmem ptastwu niebieskiemu, i zwierzowi ziemskiemu, a nie bedzie, ktoby odegnal.
\par 34 I uczynie, ze ustanie w miastach Judzkich, i w ulicach Jeruzalemskich glos radosci, i glos wesela, glos oblubienca, i glos oblubienicy; bo ziemia bedzie spustoszona.

\chapter{8}

\par 1 Czasu onego, mówi Pan, wybiora kosci królów Judzkich, i kosci ksiazat ich, i kosci kaplanów, i kosci proroków, i kosci obywateli Jeruzalemskich z grobów ich;
\par 2 I rozrzuca je przed slonce, i przed miesiac, i przed wszystko wojsko niebieskie, które miluja, i którym sluza, i za którymi chodza, i których szukaja, i którym sie klaniaja; nie pozbieraja ich, ani pogrzebia, ale beda miasto gnoju na wierzchu ziem i.
\par 3 I obiora raczej smierc nizeli zywot wszystkie ostatki, które zostana z tego rodzaju zlosliwego po wszystkich miejscach, gdziebykolwiek zostali, tam, kedy ich zapedze, mówi Pan zastepów.
\par 4 Przetoz rzeczesz do nich: Tak mówi Pan: Takze upadli, aby nie mogli powstac? takze sie odwrócil, aby sie zas nie mógl nawrócic?
\par 5 Przeczze sie odwrócil ten lud Jeruzalemski odwróceniem wiecznem? chwytaja sie klamstwa a nie chca sie nawracac.
\par 6 Pilnowalem i sluchalem: nic nie mówiac, co jest prawego; niemasz ktoby zalowal zlosci swej, mówiac: Cózem uczynil? Kazdy sie obrócil za biegiem swoim, jako kon, który pedem biezy ku potkaniu.
\par 7 I bocian na powietrzu zna ustawione czasy swoje, i synogarlica, i zuraw, i jaskólka przestrzegaja czasu przylecenia swego; ale lud mój nie zna sadu Panskiego.
\par 8 Jakoz mówicie: Mysmy madrzy, a zakon Panski jest przy nas? zaprawde, oto daremnie pióro pisarz czyni; daremnie sa w zakonie bieglymi.
\par 9 Kogoz zawstydzili ci medrcy? Którzyz sa przestraszeni i pojmani? Oto slowo Panskie odrzucaja; cóz to tedy za madrosc ich?
\par 10 Dlatego dam zony ich innym, pola ich tym, którzy ich opanuja; bo od najmniejszego az do najwiekszego, wszyscy zgola udali sie za lakomstwem; od proroka az do kaplana wszyscy przewodza klamstwo.
\par 11 Bo lecza skruszenie córki ludu mego tylko po wierzchu, mówiac: Pokój, pokój! choc niemasz pokoju.
\par 12 Izali sie zawstydzili, przeto, ze obrzydliwosc czynili? Zaiste, ani sie zapalac ani wstydzic umieli; przetoz upadna miedzy padajacymi, czasu nawiedzenia swego upadna, mówi Pan.
\par 13 Do szczetu ich wykorzenie, mówi Pan; nie bedzie zadnego grona na winnej macicy, ani zadnych fig na drzewie figowem; nawet i lisc opadnie, a com im dal, odjete bedzie.
\par 14 Przecz my tu siedzimy? Zejdzcie sie, a wynijdzmy do miast obronnych, a tam odpoczniemy; ale Pan, Bóg nasz kaze nam odpocznac, gdy nas napoi woda zólci, izesmy zgrzeszyli przeciwko Panu.
\par 15 Czekaj pokoju, alic nic dobrego; czasu uzdrowienia, alic oto strach.
\par 16 Od Dan slyszec chrapanie koni jego, od glosu wykrzykania mocarzy jego wszystka ziemia zadrzala, którzy ciagna, aby pozarli ziemie, i wszystko, co jest na niej, miasto i tych, którzy mieszkaja w niem.
\par 17 Bo oto Ja posle na was weze najjadowitsze, przeciwko którym niemasz zaklinania; i pokasaja was, mówi Pan.
\par 18 Serce moje we mnie, któreby mie mialo posilac w smutku, mdle jest.
\par 19 Oto glos krzyku córki ludu mego z ziemi bardzo dalekiej mówiacej: Izali Pana niemasz na Syonie? Izali króla jego niemasz na nim? Przeczze mie wzruszyli do gniewu balwanami swemi, próznosciami cudzoziemców?mówi Pan.
\par 20 Pominelo zniwo, skonczylo sie lato, a mysmy nie wybawieni.
\par 21 Dla skruszenia córki ludu mojego skruszonym jest, zalobe ponosze, zdumienie zdjelo mie.
\par 22 Izali niemasz balsamu w Galaad? Izali tam niemasz lekarza? Czemuz tedy nie jest uleczona córka ludu mojego?

\chapter{9}

\par 1 Kto mi to da, aby glowa moja woda byla, a oczy moje zródlem lez, abym we dnie i w nocy plakal pomordowanych córki ludu mego!
\par 2 Któz mi da na puszczy gospode podróznych, abym opuscil lud mój, i odszedl od nich? bo wszyscy sa cudzoloznicy, zgraja przestepników;
\par 3 I naciagaja jezyka swego do klamstwa jako luku swego, zmocnili sie na ziemi, ale nie ku prawdzie; bo ze zlego w zle postepuja, a mnie nie znaja, mówi Pan.
\par 4 Kazdy niech sie strzeze blizniego swego, a nie kazdemu bratu dowierza; bo kazdy brat jest na tem jakoby podszedl, a kazdy blizni zdradliwie postepuje.
\par 5 Kazdy tez blizniego swego oszukuje, a prawdy nie mówi; naucza jezyka swego mówic klamstwo, zle czyniac ustawaja
\par 6 Mieszkanie twoje, o proroku! jest w posrodku ludu zdradliwego; dla zdrad nie chca mie poznac, mówi Pan.
\par 7 A przetoz tak mówi Pan zastepów: Oto Ja plawiac ich próbowalem ich; jakoz sie tedy juz mam obchodzic z córka ludu mego?
\par 8 Strzala smiertelna jest jezyk ich, zdrade mówi; usty swemi o pokoju z przyjacielem swym mówi, ale w sercu swem zaklada nan sidla swoje.
\par 9 Izali dlatego nienawidze ich? mówi Pan; izali nad narodem takowym nie pomsci sie dusza moja?
\par 10 Dla tych gór udam sie na placz i na narzekanie, i dla pastwisk, które sa na puszczy, na kwilenie; bo spalone beda, tak, ze nie bedzie, ktoby je przechodzil, ani tam glosu bydlecia slychac bedzie; ptastwo niebieskie i bydleta rozbieza sie i odejda.
\par 11 I obróce Jeruzalem w gromady rumu, w mieszkanie smoków; a miasta Judzkie obróce w pustynie, tak, iz nie bedzie obywatela.
\par 12 Któz jest tak madry, coby to wyrozumial? a do kogo mówily usta Panskie, coby to oznajmil, dlaczego zginac ma ta ziemia, i wypalona byc ma jako pustynia, tak aby nie bylo, ktoby ja przeszedl?
\par 13 Bo Pan mówi: Iz opuscili zakon mój, którym im przedlozyl, a nie sluchali glosu mojego, ani chodzili za nim;
\par 14 Ale chodzili za uporem serca swego i za Baalem, czego ich nauczyli ojcowie ich.
\par 15 Dlatego tak mówi Pan zastepów, Bóg Izraelski: Oto Ja nakarmie ich, to jest lud ten, piolunem, a napoje ich woda zólci.
\par 16 Albowiem rozprosze ich miedzy narody, których nie znali oni i ojcowie ich, i posle za nimi miecz, az ich do konca wygladze.
\par 17 Tak mówi Pan zastepów: Uwazcie to, a przyzówcie narzekajacych niewiast, niech przyjda, a do tych, które sa w tem wycwiczone, poslijcie, aby przyszly;
\par 18 Niech sie pospiesza, a niech uczynia nad nami narzekanie, aby oczy nasze lzy wylewaly, a powieki nasze oplywaly woda.
\par 19 Glos zaiste narzekania slyszec z Syonu: O jakosmy spustoszeni! bardzosmy zelzeni; bosmy stracili ziemie, bo rozrzucone sa przybytki nasze.
\par 20 Owszem, sluchajcie niewiasty! slowa Panskiego, a niech przyjmie ucho wasze wyrok ust jego, abyscie uczyly córek swoich lamentu, a kazda z was towarzyszke swoje narzekania;
\par 21 Bo wlazla smierc oknami naszemi, weszla na palace nasze, aby wytracila dzieci z rynku, a mlodzience z ulic.
\par 22 (Mów i to: Tak mówi Pan:) I padly trupy ludzkie jako gnój po polu, a jako snopy za zencami, a niemasz ktoby pochowal.
\par 23 Tak mówi Pan: Niech sie nie chlubi madry z madrosci swojej, i niech sie nie chlubi mocarz z mocy swojej, i niech sie nie chlubi bogaty z bogactw swoich;
\par 24 Ale w tem niechaj sie chlubi, kto sie chlubi, ze rozumie a zna mie, zem Ja jest Pan, który czynie milosierdzie, sad i sprawiedliwosc na ziemi; bo mi sie to podoba, mówi Pan.
\par 25 Oto dni ida, mówi Pan, w których nawiedze kazdego obrzezanca i nieobrzezanca:
\par 26 Egipczanów, i Jude, i Edomczyków, i Amonitczyków, i Moabczyków, i wszystkich, którzy w ostatnim kacie mieszkaja na puszczy; bo te wszystkie narody nieobrzezane sa, a wszystek dom Izraelski jest nieobrzezany sercem.

\chapter{10}

\par 1 Sluchajcie slowa tego, które Pan mówi do was, o domie Izraelski!
\par 2 Tak mówi Pan: Drogi poganskiej nie uczcie sie, a zanamion niebieskich nie bójcie sie; bo sie ich poganie boja.
\par 3 Ustawy zaiste tych narodów sa wierutna marnosc; bo uciawszy drzewo siekiera w lesie, dzielo rak rzemieslnika,
\par 4 Srebrem i zlotem ozdabia je, gwozdziami i mlotami utwierdza je, aby sie nie ruchalo.
\par 5 Stoja prosto jako palma, a nie mówia; noszone byc musza, bo chodzic nie moga. Nie bójcie sie ich; bo zle czynic nie moga, i dobrze czynic nie moga.
\par 6 Zaden z tych nie jest tobie podobny, Panie! wielkis ty, i wielkie jest imie twoje w mocy.
\par 7 Któzby sie ciebie nie bal? Królu narodów! Tobie to zaiste nalezy, poniewaz miedzy wszystkimi medrcami narodów, i we wszystkich królestwach ich nigdy nie byl podobny tobie.
\par 8 A wszakze spolem zglupieli i poszaleli; z drewna brac nauke, jest wierutna marnosc.
\par 9 Srebro ciagnione z zamorza przywozone bywa, a zloto z Ufas, dzielo rzemieslnicze, i reki zlotnika; hijacynt i szarlat odzienie ich, wszystko to jest dzielo umiejetnych.
\par 10 Ale Pan jest Bóg prawy, jest Bóg zywy, i król wieczny; przed jego zapalczywoscia ziemia drzy, a narody nie moga zniesc rozgniewania jego.
\par 11 Tak im tedy powiecie: Bogowie ci, którzy nieba i ziemi nie stworzyli, niech zgina z ziemi, a niech ich nie bedzie pod niebem.
\par 12 Ale on uczynil ziemie moca swa; on utwierdzil okrag swiata madroscia swoja, i roztropnoscia swoja rozciagnal niebiosa.
\par 13 Gdy on wydaje glos, szum wód bywa na niebie, i to sposabia, aby wystepowaly pary z krajów ziemi; blyskawice z deszczem przywodzi, a wywodzi wiatry z skarbów swoich.
\par 14 Tak zglupial kazdy czlowiek, ze tego nie zna, iz pohanbiony bywa kazdy rzemieslnik dla balwana; bo falszem jest to, co ulal, i niemasz ducha w nich.
\par 15 Marnoscia sa, a dzielem bledów; czasu nawiedzenia swego pogina.
\par 16 Nie jest tym podobien dzial Jakóbowy, bo on jest stworzyciel wszystkiego; Izrael takze jest pretem dziedzictwa jego, Pan zastepów jest imie jego.
\par 17 Zbierz z ziemi towary twoje, ty, która mieszkasz na miejscu obronnem.
\par 18 Bo tak mówi Pan: Oto Ja jako z procy ugodze obywateli ziemi jednym razem, i udrecze, aby tego doznali i rzekli:
\par 19 Biada mnie nad zniszczeniem mojem: bolesna jest rana moja, chociazem byl rzekl: Zaiste te niemoc bede mógl zniesc.
\par 20 Namiot mój zburzony jest, i wszystkie powrozy moje porwane sa; synowie moi poszli odemnie, i niemasz ich; niemasz, ktoby wiecej rozbijal namiot mój, a rozciagal opony moje.
\par 21 Bo pasterze zglupieli, a Pana sie nie dokladali; dlatego nie powodzi sie im szczesliwie, a wszystka trzoda pastwiska ich rozproszona jest.
\par 22 Oto wiesc pewna przychodzi, a wzruszenie wielkie z ziemi pólnocnej, aby obrócone byly miasta Judzkie w pustynie, i w mieszkanie smoków.
\par 23 Wiem, Panie! ze nie jest w mocy czlowieka droga jego, ani jest w mocy meza tego, który chodzi, aby sprawowal postepki swe.
\par 24 Karz mie, Panie! ale laskawie, nie w gniewie swym, bys mie snac wniwecz nie obrócil.
\par 25 Wylej popedliwosc twoje na te narody, które cie nie znaja, i na rodzaje, które imienia twego nie wzywaja; bo jedza Jakóba, i pozeraja go, aby go wszystkiego strawili, i mieszkanie jego w pustki obrócili.

\chapter{11}

\par 1 Slowo, które sie stalo do Jeremijasza od Pana mówiac:
\par 2 Sluchajcie slów przymierza tego, którebyscie mówili do mezów Judzkich i do obywateli Jeruzalemskich;
\par 3 A rzeczesz do nich: Tak mówi Pan, Bóg Izraelski: Przeklety ten czlowiek, któryby nie usluchal slów przymierza tego,
\par 4 Którem przykazal ojcom waszym dnia, któregom ich wywiódl z ziemi Egipskiej, z pieca zelaznego, mówiac: Sluchajcie glosu mojego, a czyncie to wszystko, co wam rozkazuje, i bedziecie ludem moim, a Ja bede Bogiem waszym;
\par 5 Abym spelnil przysiege, któram przysiagl ojcom waszym, ze im dam ziemie oplywajaca mlekiem i miodem: jako sie to dzis okazuje. Któremu odpowiedziawszy rzeklem: Amen, Panie!
\par 6 I rzekl Pan do mnie:Obwolywaj wszystkie te slowa w miastach Judzkich i po ulicach Jeruzalemskich, mówiac: Sluchajcie slów przymierza tego, a czyncie je.
\par 7 Bo oswiadczajac oswiadczalem sie przed ojcami waszymi ode dnia, któregom ich wywiódl z ziemi Egipskiej, az do dnia tego; rano wstawajac i oswiadczajac sie, mawialem: Sluchajcie glosu mego.
\par 8 Ale nie usluchali, ani naklonili ucha swego; owszem kazdy szedl za uporem serca swego zlego. Przetozem przywiódl na nich wszystkie slowa przymierza tego, którem rozkazal, aby czynili; ale oni nie czynili.
\par 9 I rzekl Pan do mnie: Znalazlo sie sprzysiezenie miedzy mezami Judzkimi, i miedzy obywatelami Jeruzalemskimi;
\par 10 Obrócili sie do nieprawosci ojców swoich pierwszych, którzy nie chcieli sluchac slów moich; takze i ci chodza za bogami cudzymi, sluzac im; zgwalcili dom Izraelski i dom Judzki przymierze moje, którem byl postanowil z ojcami ich.
\par 11 Dlatego tak mówi Pan: Oto Ja przywiode na nich zle, z którego nie beda mogli wyjsc; chocby wolali do mnie, nie wyslucham ich.
\par 12 I pójda miasta Judzkie i obywatele Jeruzalemscy, a beda wolali do bogów, którym kadza; ale ich zadnym sposobem nie wybawia czasu utrapienia ich.
\par 13 Aczkolwiek ile jest miast twoich, tyle bogów twoich, o Judo! a ile ulic Jeruzalemskich, tylescie nastawiali oltarzów obrzydliwosci, oltarzów do kadzenia Baalowi.
\par 14 Przetoz sie ty nie módl za tym ludem, ani podnos za nimi glosu i modlitwy; bo Ja ich nie wyslucham natenczas, gdy do mnie zawolaja w utrapieniu swojem.
\par 15 Cóz milemu memu do domu mego? poniewaz bez wstydu pacha zlosci z wieloma, a ofiary swiete odeszly od ciebie; i ze sie w zlosci swojej radujesz.
\par 16 Oliwa zielona, piekna, dla owocu slicznego nazwal byl Pan imie twoje; ale z szumem burzy wielkiej zapali ja ogniem z góry, gdy polamie galezie jej.
\par 17 Bo Pan zastepów, który cie byl wszczepil, wyrzekl zle przeciwko tobie dla zlosci domu Izraelskiego i domu Judzkiego, które czynili miedzy soba, aby mie draznili, kadzac Baalowi.
\par 18 Pan zaiste oznajmil mi, i dowiedzialem sie; tedys mi ukazal przedsiewziecia ich,
\par 19 Gdym byl jako baranek i wól, którego wioda na rzez; bom nie wiedzial, aby przeciwko mnie rady zmyslali mówiac: Popsujemy drzewo z owocem jego, a wykorzenmy go z ziemi zyjacych, aby imie jego nie bylo wiecej wspomniane.
\par 20 Ale, o Panie zastepów! który sprawiedliwie sadzisz, a doswiadczasz nerek i serca, niech widze pomste twoje nad nimi; bom ci objawil sprawe moje.
\par 21 Dlatego tak mówi Pan o mezach z Anatot, którzy szukaja duszy twojej, a mówia: Nie prorokuj w imieniu Panskiem, abys nie umarl od rak naszych:
\par 22 A przetoz tak mówi Pan zastepów: Oto ja nawiedze ich; mlodziency ich pomra od miecza, synowie ich i córki ich pomra glodem.
\par 23 I nic nie zostanie z nich; bo przywiode zle na mezów z Anatot roku nawiedzenia ich.

\chapter{12}

\par 1 Sprawiedliwym zostaniesz, Panie! jezli sie z toba rozpierac bede; a wszakze o sadach twoich z toba mówic bede. Czemuz sie droga niezboznych szczesci? Czemuz spokojnie zyja wszyscy, którzy bardzo wystapili przeciwko tobie?
\par 2 Wszczepiles ich, i rozkorzenili sie; rosna i owoc wydawaja ci, któryches ust bliskim, ,ale dalekim od nerek ich.
\par 3 Ale ty, Panie, znasz mie, wypatrujesz mie, a doswiadczyles serca mego, ze z toba jest; ale onych ciagniesz jako owce na rzez i gotujesz ich na dzien zabicia, i mówisz:
\par 4 Dokadzeby ziemia plakac, a trawa na wszystkich polach schnac miala? Dla zlosci mieszkajacych w niej gina wszystkie zwierzeta i ptastwo; bo mówia: Nie widzic Pan skonczenia naszego.
\par 5 Poniewaz cie z pieszymi biezacego do ustania przywodza, jakozbys mial zdazyc przy koniach? a poniewaz w ziemi pokoju, w którejs ufal, ustawasz, a cóz sprawisz przy tej nadetosci Jordanu?
\par 6 Bo i bracia twoi i dom ojca twego przeniewierzyli sie tobie, i ci takze wolaja za toba pelnemi usty; ale nie wierzz im, chocby mówili z toba po przyjacielsku.
\par 7 Opuscilem dom swój, odrzucilem dziedzictwo moje; dalem to, co milowala dusza moja, w rece nieprzyjaciól jego.
\par 8 Stalo mi sie dziedzictwo moje jako lew w lesie; wydaje przeciwko mnie glos swój, przetoz go nienawidze.
\par 9 Izali ptakiem drapieznym jest mi dziedzictwo moje? Izali ptastwo bedzie w okolo przeciwko niemu? Idzciez, zbierzcie sie wszystkie zwierzeta polne, zejdzci sie do zeru.
\par 10 Wiele pasterzy popsuje winnice moje, podepcza dzial mój; dzial mój bardzo mily obróca w pustynie sroga.
\par 11 Obróca go w pustynie; plakac bedzie, spustoszony bedac odemnie; ta wszystka ziemia spustoszeje, bo niemasz, ktoby to skladal do serca.
\par 12 Na wszystkie miejsca wysokie w pustyniach przyjda burzyciele, bo miecz Panski pozre wszystko od konca ziemi az do konca ziemi; nie bedzie mialo pokoju zadne cialo.
\par 13 Nasieja pszenicy, ale ciernie zac beda; frasowac sie beda, ale nic nie sprawia, i wstydzic sie beda za urodzaje swoje dla gniewu popedliwosci Panskiej.
\par 14 Tak mówi Pan o wszystkich zlych sasiadach moich, którzy sie dotykaja dziedzictwa, którem dal w dziedzictwo ludowi memu Izraelskiemu: Oto Ja wykorzenie ich z ziemi ich, kiedy dom Judzki wyplenie z posrodku ich.
\par 15 Wszakze gdy ich wyplenie, nawróce sie i zmiluje sie nad nimi, a przywiode zasie kazdego z nich do dziedzictwa jego, i kazdego z nich do ziemi jego.
\par 16 I stanie sie, jezli sie uczac naucza dróg ludu mojego, a przysiegac beda w imieniu mojem, mówiac: Jako zyje Pan, jako oni nauczali lud mój przysiegac przez Baala, tedy pobudowani beda w posrodku ludu mego.
\par 17 Ale jezliby nie usluchali, tedy wykorzenie ten naród, wyplenie i wytrace go, mówi Pan.

\chapter{13}

\par 1 Tak rzekl Pan do mnie: Idz, a kup sobie pas lniany, a opasz nim biodra swoje; ale do wody nie kladz go.
\par 2 Kupilem tedy pas wedlug rozkazania Panskiego, i opasalem biodra moje.
\par 3 Potem stalo sie slowo Panskie do mnie powtóre, mówiac:
\par 4 Wezmij ten pas, którys kupil, który jest na biodrach twoich, a wstawszy idz do Eufratesa, a skryj go tam w dziure skalna.
\par 5 I szedlem a skrylem go u Eufratesa, jako mi byl Pan rozkazal.
\par 6 A po wyjsciu wielu dni rzekl Pan do mnie: Wstan, idz do Eufratesa, a wezmij stamtad on pas, którym ci tam rozkazal skryc.
\par 7 Szedlem tedy do Eufratesa, a wykopawszy wzialem on pas z miejsca onego, gdziem go byl skryl, a oto skazony byl on pas, tak, iz sie niczemu nie godzil.
\par 8 I stalo sie slowo Panskie do mnie, mówiac:
\par 9 Tak mówi Pan: Tak skaze pyche Judzka i wielka pyche Jeruzalemska,
\par 10 Ludu tego bardzo zlego, który sie zbrania sluchac slów moich, który chodzi w uporze serca swego, i chodzi za bogami obcymi, sluzac im i klaniajac sie im; i bedzie podobien temu pasowi, który sie niczemu nie godzi.
\par 11 Bo jako pas przylega do biódr meza, takiem Ja byl przypoil do siebie wszystek dom Izraelski, i wszystek dom Judzki, mówi Pan, aby byli ludem moim, a to ku slawie i ku chwale, i ku ozdobie; ale nie byli posluszni.
\par 12 Przetoz rzecz im to slowo. Tak mówi Pan, Bóg Izraelski: Wszelkie naczynie winne bywa napelnione winem; a gdy rzeka: Wiemyc to dobrze, ze wszelkie naczynie winne bywa napelnione winem,
\par 13 Tedy im rzeczesz: Tak mówi Pan: Oto Ja napelnie wszystkich obywateli tej ziemi, i królów, którzy siedza miasto Dawida na stolicy jego, i kaplanów i proroków, takze i wszystkich obywateli Jeruzalemskich pijanstwem;
\par 14 I rozraze jednego o drugiego, jako ojców tak i synów, mówi Pan; nie przepuszcze, nie sfolguje, ani sie zmiluje, abym ich skazic nie mial.
\par 15 Sluchajciez, a pojmujcie uszyma, nie podnoscie sie; boc Pan mówi.
\par 16 Dajcie Panu, Bogu swemu, chwale, pierwej nizby ciemnosci przywiódl, a pierwej nizby sie obrazily nogi wasze o góry ciemne; i czekalibyscie swiatlosci, ale Bóg obrócilby je w cien smierci i przemienilby je w zacmienie.
\par 17 A jezliz tego sluchac nie bedziecie, w skrytosciach plakac bedzie dusza moja dla pychy waszej, a placzac plakac bedzie, i wyleje oko moje lzy, bo pojmana bedzie trzoda Panska.
\par 18 Mów królowi i królowej: Upokorzcie sie, usiadzcie na ziemi; bo spadla z glowy waszej korona chwaly waszej.
\par 19 Miasta na poludnie zawarte beda, tak, ze nie bedzie, ktoby je otworzyl; przeniesiony bedzie wszystek Juda, przeniesiony bedzie do szczetu.
\par 20 Podniescie oczy wasze, a obaczcie tych, którzy ida z pólnocy. Gdzie jest ta trzoda, którejc sie zwierzono? gdzie jest stado chwaly twojej?
\par 21 Cóz rzeczesz, gdy cie (nieprzyjaciel) nawiedzi? Bos ich ty nauczyla, aby byli nad toba ksiazetami przednimi; izali cie bolesci nie ogarna, jako niewiaste rodzaca?
\par 22 Mówiszli w sercu swojem: Przeczzeby to przypasc mialo na mie? Dla mnóstwa nieprawosci twojej odkryte beda podolki twoje, gwaltem obnazone beda piety twoje.
\par 23 Azaz moze murzyn odmienic skóre swoje, albo lampart pstrociny swoje? takze i wy, azaz bedziecie mogli dobrze czynic, nauczywszy sie zle czynic?
\par 24 Przetoz rozprosze ich jako zdzblo, które sie rozlatuje od wiatru z pustyni.
\par 25 Tenci bedzie los twój, i dzial odmierzony tobie odemnie, mówi Pan, przeto, zes mie zapomniala, a ufalas w klamstwie.
\par 26 A tak i Ja odkryje podolek twój az na twarz twoje, aby sie okazala sromota twoja.
\par 27 Widzialem cudzolóstwa twoje i poryzanie twoje, sprosnosc wszeteczenstwa twego na pagórkach, i na polu; widzialem, mówie, obrzydliwosci twoje. Biada tobie, Jeruzalemie! i pókiz sie nie oczyscisz? kiedyz to wzdy bedzie?

\chapter{14}

\par 1 Slowo Panskie, które sie stalo do Jeremijasza o suszy.
\par 2 Ziemia Judzka plakac bedzie, a bramy jej zemdleja, zalobe nosic beda na ziemi, a narzekanie Jeruzalemskie wstapi w góre;
\par 3 I zacniejsi z nich rozsylac beda najpodlejszych swoich po wode; a przyszedlszy do cystern, i nie znalazlszy wody, nawróca sie z naczyniem swojem próznem, zaplonawszy i zawstydziwszy sie; przetoz nakryja glowe swoje.
\par 4 Dla ziemi upragnionej, przeto, ze deszczu nie bedzie na ziemi, i oracze wstydzac sie nakryja glowy swoje.
\par 5 Owszem i lani, co na polu porodzila, opusci; bo na polu trawy nie bedzie.
\par 6 A osly dzikie, stajac na wysokich miejscach, chwytac beda wiatr jako smoki; ustana oczy ich, bo nie bedzie trawy.
\par 7 O Panie! poniewaz nieprawosci nasze swiadcza przeciwko nam, zmiluj sie dla imienia twego; boc wielkie sa odwrócenia nasze, tobiesmy zgrzeszyli.
\par 8 O nadzejo Izraelowa, wybawicielu jego czasu utrapienia! czemuz masz byc jako przychodzien w tej ziemi, a jako podrózny wstepujacy na nocleg?
\par 9 Czemuz sie pokazujesz jako maz strudzony, albo jako mocarz, który nie moze wybawic? Wszakes ty jest w posrodku nas, Panie! a imie twoje wzywane jest nad nami; nie opuszczajze nas.
\par 10 Tak mówi Pan o tym ludu: Iz tak miluja tulanie, a nóg swych nie powsciagaja, przetoz sie Panu nie podobaja, i teraz wspomina nieprawosci ich, a nawiedza grzechy ich.
\par 11 Potem rzekl Pan do mnie: Nie módl sie za tym ludem.
\par 12 Gdy poscic beda, Ja nie wyslucham wolania ich; a gdy ofiarowac beda calopalenie, i ofiare sniedna, Ja tego nie przyjme; ale mieczem, i glodem, i morem wytrace ich.
\par 13 I rzeklem: Ach, panujacy Panie! oto im ci prorocy mówia: Nie ogladacie miecza, a glód nie przyjdzie na was, ale pokój pewny dam wam na tem miejscu.
\par 14 I rzekl Pan do mnie: Falsz prorokuja ci prorocy w imieniu mojem; nie poslalem ich, anim im rozkazal, owszem, anim mówil do nich; widzenie klamliwe, i wieszczbe, i marnosc, i klamstwo serca swego oni wam prorokuja.
\par 15 Przetoz tak mówi Pan o prorokach, którzy prorokuja w imieniu mojem, chociazem Ja ich nie poslal, i którzy mówia: Miecza ani glodu nie bedzie w tej ziemi; ci sami prorocy mieczem i glodem zgina.
\par 16 A lud ten, któremu oni prorokuja, rozrzucony bedzie po ulicach Jeruzalemskich od glodu i od miecza, a nie bedzie, ktoby ich pogrzebal, onych samych, zony ich, i synów ich, i córki ich; tak wyleje na nich zlosc ich.
\par 17 Przetoz rzeczesz do nich to slowo: Oczy moje wylewaja lzy w nocy i we dnie bez przestanku; bo skruszeniem wielkiem skruszona bedzie panna, córka ludu mojego, i rana bardzo bolesna.
\par 18 Wyjdeli na pole, oto tam pomordowani mieczem; wyjdeli do miasta, oto i tam zmorzeni glodem; bo jako prorok tak i kaplan obchodzac kupcza ziemia, a ludzie tego nie bacza.
\par 19 Izali do konca odrzucasz Jude? Izali Syon obrzydzila sobie dusza twoja? Przecz nas bijesz, tak abysmy juz nie byli uzdrowieni? Oczekujemyli na pokój, alic oto nastepuje nic dobrego; a jezli na czas uleczenia, a oto zatrwozenie.
\par 20 Uznajemy, Panie! niezboznosc swoje, i nieprawosc ojców naszych, izesmy zgrzeszyli przeciw tobie.
\par 21 Nie odrzucajze nas dla imienia twego, nie podawajze w lekkosc stolicy chwaly twojej; wspomnijze, nie targaj przymierza twego z nami.
\par 22 Izali sa miedzy marnosciami poganskiemi, coby spuszczali deszcz? albo niebiosa mogali same przez sie dawac deszcze? Izalis nie ty sam Pan, Bóg nasz? Przetoz oczekujemy na cie; bo to wszystko ty czynisz.

\chapter{15}

\par 1 Tedy rzekl Pan do mnie: Chocby stanal Mojzesz i Samuel przed obliczem mojem, nie mialbym serca do ludu tego; pusc ich od oblicza mego, a niech precz ida.
\par 2 A jezliby rzekli: Dokadze pójdziemy? Tedy im rzeczesz: Tak mówi Pan: Kto oddany na smierc, na smierc pójdzie; a kto pod miecz, pod miecz; a kto na glód, na glód; a kto w niewole, w niewole.
\par 3 Bo ich ta czworaka rzecza nawiedze, mówi Pan: Mieczem na zamordowanie, i psami na rozszarpanie, i ptastwem niebieskiem, i zwierzetami ziemskiemi na pozarcie i na wygubienie.
\par 4 I podam ich na potlukanie sie po wszystkich królestwach ziemi dla Manasesa, syna Ezechijasza, króla Judzkiego, za to, co uczynil w Jeruzalemie.
\par 5 Bo któzby sie zmilowal nad toba? Jeruzalemie! albo ktoby sie uzalil nad toba? albo ktoby przyszedl, aby sie pytal, jakoc sie powodzi?
\par 6 Tys mie opuscilo, mówi Pan, poszlos nazad. Przetoz wyciagne reke moje na cie, abym cie wytracil; ustalem od zalu.
\par 7 Przetoz ich rozwieje wiejaczka po bramach tej ziemi, osieroce i wygubie ich; bo sie od dróg swoich nie nawracali.
\par 8 Wiecej sie namnozy wdów jego, niz piasku morskiego; przywiode na nich, na matki, na mlodzienców burzyciela i w poludnie; sprawie, ze przypadna nagle na to miasto; i beda przestraszeni.
\par 9 Zemdleje i ta, która rodzila po siedmiorgu, wypusci dusze swoje, zajdzie jej slonce jeszcze za dnia, zaplonie i wstydzic sie bedzie; a ostatek ich dam pod miecz przed obliczem nieprzyjaciól ich, mówi Pan.
\par 10 Biada mnie, matko moja! zes mie urodzila meza swaru, i meza sporu po wszystkiej ziemi; nie dawalem im na lichwe, ani mnie oni na lichwe dawali, a wzdy mi kazdy zlorzeczy.
\par 11 I rzekl Pan: Izali tobie, który pozostaniesz, nie bedzie dobrze? Izali sie nie zastawie o cie nieprzyjacielowi czasu utrapienia i czasu ucisku?
\par 12 Izali zelazo proste pokruszy zelazo pólnocne i stal?
\par 13 Majetnosc twoje, o Judo! i skarby twoje dam w rozszarpanie darmo po wszystkich granicach twoich, a to dla wszystkich grzechów twoich;
\par 14 A sprawie to, ze pójdziesz z nieprzyjaciólmi twymi do ziemi, którejs nie znal; albowiem ogien rozniecony w zapalczywosci mojej na was palac bedzie.
\par 15 Ty mie znasz, Panie! wspomnijze na mie, a nawiedz mie, i pomscij sie za mie nad tymi, co docieraja na mie; odwlaczajac zapalczywosci twojej przeciwko nim, nie porywaj mie; wiedz, ze podejmuje dla ciebie pohanbienie.
\par 16 Gdy sie znalazly mowy twoje, zjadlem ich, a bylo mi slowo twoje weselem i radoscia serca mego, poniewaz sie nazywam od imienia twego, Panie, Boze zastepów!
\par 17 Nie siadam w radzie nasmiewców, ani sie z nimi raduje; ale dla surowosci reki twojej samotny siadam; bo zapalczywoscia napelniles mie.
\par 18 Przeczze ma byc zal mój wieczny? a rana moja smiertelna, która sie uleczyc nie da? Przeczze mi tak masz byc jako omylny, jako wody niepewne?
\par 19 Przetoz tak mówi Pan: Jezli sie nawrócisz, tedy cie nawróce, abys stal przed obliczem mojem; a jezli odlaczysz rzecz kosztowna od nikczemnej, bedziesz jako usta moje; oni niech sie obróca do ciebie, ale sie ty nie obracaj do nich.
\par 20 Bom cie postawil przeciw ludowi temu jako mur miedziany i obronny; i beda walczyc przeciwko tobie, ale cie nie przemoga; bom Ja z toba, abym cie wybawial i wyrywal, mówi Pan.
\par 21 Wyrwe cie zaiste z rak ludzi zlych, i odkupie cie z rak okrutników.

\chapter{16}

\par 1 I stalo sie slowo Panskie do mnie, mówiac:
\par 2 Nie pojmuj sobie zony, ani miej synów ani córek na tem miejscu.
\par 3 Albowiem tak mówi Pan o synach i o córkach splodzonych na tem miejscu, i o matkach ich, które ich zrodzily, i o ojcach ich, którzy ich splodzili w tej ziemi:
\par 4 Smierciami ciezkiemi pomra; nie beda ich plakac, ani ich pochowaja, ale miasto gnoju na wierzchu ziemi beda; a mieczem i glodem wytraceni beda; i beda trupy ich pokarmem ptastwu niebieskiemu i i zwierzowi ziemskiemu.
\par 5 Bo tak mówi Pan: Nie wchodz do domu zaloby, ani chodz na placz, ani ich zaluj; bom odjal pokój mój od ludu tego, milosierdzie i litosc, mówi Pan.
\par 6 Gdy pomra wielcy i mali w tej ziemi, nie beda pogrzebieni, ani ich plakac beda; i nie beda sie rzezac, ani sobie lysiny czynic dla nich;
\par 7 Ani im dadza jesc, aby ich w smutku cieszyli nad umarlym; ani im dadza pic z kubka pocieszenia po ojcu ich i po matce ich;
\par 8 Takze do domu uczty nie wchodz, abys zasiadal z nimi, i jadl, i pil.
\par 9 Albowiem tak mówi Pan zastepów, Bóg Izraelski: Oto Ja sprawie, iz ustanie na tem miejscu przed oczyma waszemi, i za dni waszych glos wesela, i glos radosci, glos oblubienca, i glos oblubienicy.
\par 10 A gdy opowiesz ludowi temu wszystkie te slowa, a rzekliby do ciebie: Przecz Pan wyrzekl przeciwko nam to wszystko wielkie zle? I cóz jest za nieprawosc nasza, i co za grzech nasz, którymesmy zgrzeszyli przeciwko Panu, Bogu naszemu?
\par 11 Tedy rzeczesz do nich: Przeto, iz mie opuscili ojcowie wasi, (mówi Pan) a chodzili za bogami cudzymi, i sluzyli im, i klaniali sie im, lecz mnie opuscili, i zakonu mego nie przestrzegali.
\par 12 A wy dalekoscie gorzej czynili, niz ojcowie wasi; albowiem oto kazdy chodzi za uporem zlego serca swego, nie sluchajac mie;
\par 13 Dlatego wyrzuce was z tej ziemi do ziemi, którejscie nie znali, wy i ojcowie wasi, a tam sluzyc bedziecie bogom cudzym we dnie i w nocy, dokad wam nie okaze milosierdzia.
\par 14 Przetoz oto dni ida, mówi Pan, ze nie rzeka wiecej: Jako zyje Pan, który wywiódl synów Izraelskich z ziemi Egipskiej.
\par 15 Ale: Jako zyje Pan, który wywiódl synów Izraelskich z ziemi pólnocnej, i ze wszystkich ziem, do których ich byl wygnal, gdy ich zasie przywiode do ziemi ich, któram dal ojcom ich.
\par 16 Oto Ja posle do wielu rybitwów, (mówi Pan) aby ich lowili; potem posle do wielu lowców, aby ich lapali na wszelkiej górze i na wszelkim pagórku, i w dziurach skalnych.
\par 17 Oczy moje patrza na wszystkie drogi ich; nie sa utajone przed obliczem mojem, ani jest zakryta nieprawosc ich przed oczyma mojemi.
\par 18 I oddam im pierwej w dwójnasób za nieprawosci ich, i za grzechy ich, przeto, ze ziemie moje splugawili trupami obrzydliwosci swojej, i sprosnosciami swemi napelnili dziedzictwo moje.
\par 19 Panie, mocy moja i silo moja, i ucieczko moja w dzien utrapienia! do ciebie przyjda narody od konczyn ziemi, i rzekna: Zaiste sie falszu trzymali ojcowie nasi, i marnosci, w których zadnego pozytku nie bylo.
\par 20 Izali sobie czlowiek uczynic moze bogów? poniewaz sami nie sa bogami.
\par 21 Dlatego oto Ja sprawie, aby poznali tym razem; sprawie, mówie, aby poznali reke moje i moc moje, i dowiedzieli sie, ze imie moje jest Pan.

\chapter{17}

\par 1 Grzech Judzki napisany jest piórem zelaznem, a ostrym dyjamentem wyryty jest na tablicy serca ich, i na rogach oltarzów waszych;
\par 2 Gdy wspominaja synowie ich na oltarze ich, i na gaje ich pod drzewem zielonem, na pagórkach wysokich.
\par 3 O góro, i pole moje! majetnosc twoje i wszystkie skarby twoje podam na rozszarpanie dla grzechu wyzyn twoich, które masz po wszystkich granicach twoich.
\par 4 A ty musisz zaniechac za przewinieniem twojem dziedzictwa twego, którem ci dal. I dam cie w niewole nieprzyjaciolom twoim, i ziemi, której nie znasz; boscie ogien rozniecili w popedliwosci mojej, który az na wieki gorzec bedzie.
\par 5 Tak mówi Pan:Przeklety maz, który ufa w czlowieku, i który poklada cialo ramieniem swojem, a od Pana odstepuje serce jego.
\par 6 Albowiem stanie sie jako wrzos na puszczy, który nie czuje, gdy co dobrego przychodzi, ale bywa na suchych miejscach na puszczy w ziemi slonej, i w której nikt nie mieszka.
\par 7 Blogoslawiony maz, który ufa w Panu, a Pan jest nadzieja jego.
\par 8 Bo bedzie jako drzewo wszczepione nad wodami, a nad strumieniem zapuszczajace korzenie swoje, które nie czuje, gdy przychodzi goracosc, ale lisc jego zostaje zielony, a roku suchego nie frasuje sie, i nie przestaje przynosic owocu.
\par 9 Najzdradliwsze jest serce nadewszystko i najprzewrotniejsze, któz je pozna?
\par 10 Ja Pan, który sie badam serca, i doswiadczam nerek, tak abym kazdemu oddal wedlug drogi jego, i wedlug owocu spraw jego.
\par 11 Jako kuropatwa zgromadza jajka, ale ich nie wylega: tak, kto zbiera bogactwa, a niesprawiedliwie, w polowie dni swoich opusci je, a na ostatek bedzie glupim;
\par 12 Ale miejsce swiatnicy naszej, to jest stolica chwaly Najwyzszego, wiecznie trwa.
\par 13 O nadziejo Izraelska, Panie! wszyscy, którzy cie opuszczaja, niech beda zawstydzeni; którzy odstepuja odemnie, niech na ziemi zapisani beda; albowiem opuscili zródlo wód zywych, Pana.
\par 14 Uzdrów mie, Panie! a bede uzdrowiony: zbaw mie, a bede zbawiony; albowiemes ty chwala moja.
\par 15 Oto oni do mnie mówia: Gdziez jest to slowo Panskie? Niechze juz przyjdzie;
\par 16 Chociazem Ja tego nie zabiegal, abym byl pasterzem twoim, anim dnia bolesci pragnal, ty wiesz, cokolwiek wyszlo z ust moich, przed obliczem twojem jest.
\par 17 Nie badzze mi na postrach; tys nadzieja moja w dzien utrapienia.
\par 18 Niech beda pochanbieni, którzy mie przesladuja, a ja niech nie bede zawstydzony; niech sie oni lekaja, a ja niech sie nie lekam; przywiedz na nich dzien utrapienia, a dwojakiem skruszeniem skrusz ich.
\par 19 Tak Pan rzekl do mnie: Idz, a stan w bramie synów ludu tego, która wchodza królowie Judzcy, i która wychodza, i we wszystkich bramach Jeruzalemskich,
\par 20 I rzecz do nich: Sluchajcie slowa Panskiego, królowie Judzcy, i wszystek Judo, i wszyscy obywatele Jeruzalemscy, którzy chadzacie temi bramami!
\par 21 Tak mówi Pan: Strzezcie pilnie dusz waszych, a nie noscie brzemion zadnych w dzien sabatu, ani ich wnoscie bramami Jeruzalemskiemi;
\par 22 Ani wynaszajcie brzemion z domów waszych w dzien sabatu, ani zadnej roboty odprawujcie, ale swieccie dzien sabatu, jakom rozkazal ojcom waszym.
\par 23 Wszakze nie usluchali, ani naklonili ucha swego, owszem, zatwardzili kark swój, nie sluchajac ani przyjmujac nauki.
\par 24 A jezli mie pilnie sluchac bedziecie, mówi Pan, tak, zebyscie nie wnosili brzemion bramami miasta tego w dzien sabatu, ale swiecili dzien sabatu, nie odprawujac wen zadnej roboty;
\par 25 Tedy wchodzic beda bramami miasta tego królowie i ksiazeta siedzacy na stolicy Dawidowej, jezdzac na wozach i na koniach, oni i ksiazeta ich, mezowie Judzcy i obywatele Jeruzalemscy, i stac bedzie to miasto az na wieki.
\par 26 I zbieza sie z miast Judzkich i z okolicznych miejsc Jeruzalemskich, i z ziemi Benjaminowej, i z równin, i z tej góry, i od poludnia, przynoszac calopalenie, i ofiare, i dar, i kadzidlo, takze i dziekczynienie niosac do domu Panskiego.
\par 27 Ale jezli mie nie usluchacie, abyscie swiecili dzien sabatu, a nie nosili brzemion, wchodzac bramami Jeruzalemskiemi w dzien sabatu, tedy rozniece ogien w bramach jego, który pozre palace Jeruzalemskie, a nie bedzie ugaszony.

\chapter{18}

\par 1 Slowo, które sie stalo do Jeremijasza od Pana mówiac:
\par 2 Wstan, a wstap do domu garncarzowego, a tam sprawie, ze uslyszysz slowa moje.
\par 3 I wstapilem do domu garncarzowego, a oto on robil robote na kregu.
\par 4 A gdy sie zepsulo naczynie w rece garncarzowej, które on czynil z gliny, tedy zas uczynil z niej naczynie insze, jako sie mu najlepiej zdalo uczynic.
\par 5 I stalo sie slowo Panskie do mnie, mówiac:
\par 6 Izalibym tak nie mógl z wami postapic, jako ten garncarz, o domie Izraelski? mówi Pan. Oto jako glina w rece garncarzowej, takescie wy w rece mojej, o domie Izraelski!
\par 7 Jezlibym rzekl nagle przeciwko narodowi, i przeciwko królestwu, ze je wykorzenie, i zepsuje, i wygubie;
\par 8 Wszakze jezliby sie odwrócil on naród od zlosci swojej, przeciw któremum mówil; i Jabym zalowal tego zlego, którem mu umyslil uczynic.
\par 9 Zasie, jezlibym rzekl nagle o narodzie i o królestwie, ze je pobuduje i wszczepie;
\par 10 Wszakze jezliby czynil, co zlego jest przed oczyma memi, nie sluchajac glosu mego: i Jabym zalowal tego dobrodziejstwa, którem mu obiecal uczynic.
\par 11 A przetoz rzecz teraz do mezów Judzkich, i do obywateli Jeruzalemskich, mówiac: Tak mówi Pan: Oto Ja gotuje na was zla rzecz, i mysle cos przeciwko wam. Nawrócciez sie juz kazdy od zlej drogi swojej, a poprawcie kazdy dróg waszych, i spraw waszych.
\par 12 Ale oni rzekli: Nic z tego; bo za myslami naszemi pójdziemy, a kazdy upór serca swojego zlego czynic bedziemy.
\par 13 Przetoz tak mówi Pan: Pytajcie teraz miedzy poganami, któz slyszal co takowego? Sprosnosc wielka popelnila panna Izraelska.
\par 14 Izali kto opusci pola moje dla skal i dla sniegu na Libanie? Izali kto opusci wody ciekace dla wody bardzo zimnej?
\par 15 Ale lud mój zapomniawszy na mie, marnosci kadza i potykaja sie na drogach swych, na scieszkach starodawnych, chodzac scieszkami drogi nieutorowanej;
\par 16 Tak, abym podal ziemie ich na spustoszenie, na swistanie wieczne, aby kazdy, ktoby szedl przez nia, zdumial sie, i kiwal glowa swoja.
\par 17 Wiatrem wschodnim rozprosze ich przed nieprzyjacielem; tyl a nie twarz ukaze im w dzien zatracenia ich.
\par 18 I rzekli: Pójdzcie a wymyslmy co przeciwko Jeremijaszowi; bo nie zginie zakon od kaplana, ani rada od madrego, ani slowo od proroka; pójdzciez, a ubijmy go jezykiem, a nie dbajmy na zadne slowa jego.
\par 19 Pilnuj mie Panie! a sluchaj glosu tych, którzy sie spieraja ze mna.
\par 20 Izali sie ma oddawac zlem za dobre, ze ukopali dól duszy mojej? Wspomnij, zem stawal przed obliczem twojem, abym za nimi mówil ku ich dobremu, i odwrócil zapalczywosc twoje od nich.
\par 21 Dlatego dopusc glód na synów ich, a spraw, ze okrutnie beda pobici od miecza, ze beda zony ich osierociale i owdowiale, a mezowie ich ze beda haniebnie zamordowani, a dzieci ich pobite mieczem na wojnie.
\par 22 Niech bedzie slyszany krzyk z domów ich, gdy na nich nagle wojsko przywiedziesz; bo ukopali dól, aby mie ulapili a sidla ukryli na nogi moje.
\par 23 Ales ty, Panie! powiadomy wszystkiej rady ich przeciwko mnie na smierc: nie badz milosciw nieprawosciom ich, a grzechu ich przed obliczem twojem nie zagladzaj; ale niech sie potkna przed oblicznoscia twoja, czasu zapalczywosli twojej surowo sie o bchodz z nimi.

\chapter{19}

\par 1 Tak mówi Pan: Idz, a kup dzban gliniany od garncarza, a wziawszy niektórych z starszych ludu i z starszych kaplanów;
\par 2 Wnijdz do doliny syna Hennomowego, która jest u wrót bramy wschodniej, a tam opowiadaj slowa, które do ciebie mówic bede.
\par 3 A rzecz: Sluchajcie slowa Panskiego, królowie Judzcy i obywatele Jeruzalemscy! Tak mówi pan zastepów, Bóg Izraelski: Oto Ja przywiode zle na to miejsce, o którem ktokolwiek uslyszy, zabrzmi mu w uszach jego.
\par 4 Przeto, ze mie opuscili, a splugawili to miejsce, kadzac na niem bogom cudzym, których nie znali oni i ojcowie ich, i królowie Judzcy, i napelnili to miejsce krwia niewinnych;
\par 5 I pobudowali wyzyny Baalowi, aby palili synów swych ogniem na calopalenie Baalowi, czegom nie rozkazal, anim oto mówil, ani to wstapilo na serce moje:
\par 6 Dlatego oto dni ida, mówi Pan, w których nie bedzie nazywane wiecej to miejsce Tofet, ani dolina syna Hennomowego, ale dolina mordu.
\par 7 Bo wniwecz obróce rade Judzka i Jeruzalemska na tem miejscu, a sprawie, ze oni upadna od miecza przed twarza nieprzyjaciól swoich, i od reki szukajacych duszy ich; i dam trupy ich na pokarm ptastwu niebieskiemu, i zwierzowi ziemskiemu;
\par 8 Podam takze to miasto na spustoszenie i na swistanie; kazdy idacy mimo nie zdumieje sie, a swistac bedzie nad wszystkiemi plagami jego.
\par 9 I sprawie to, ze beda jesc ciala synów swoich, i ciala córek swoich, a kazdy z nich cialo blizniego swego jesc bedzie w oblezeniu i w ucisnieniu, którem ich ucisna nieprzyjaciele ich, i ci, którzy szukaja duszy ich.
\par 10 Potem stlucz ten dzban przed oczyma mezów, którzy pójda z toba,
\par 11 A rzecz do nich: Tak mówi Pan zastepów: Tak stluke ten lud, i to miasto, jako gdy kto tlucze naczynie garncarskie, które wiecej naprawione byc nie moze: a w Tofet pogrzebywac beda, iz miejsca inszego nie bedzie ku pogrzebowi.
\par 12 Tak uczynie temu miejscu, mówi Pan, i obywatelom jego, i postapie sobie z tem miastem, tak jako z Tofet.
\par 13 Bo beda domy Jeruzalemskie i domy królów Judzkich, jako to miejsce Tofet, nieczyste ze wszystkiemi domami temi, na których dachach kadzili wszystkiemu wojsku niebieskiemu, i sprawowali ofiary mokre bogom cudzym.
\par 14 Tedy wróciwszy sie Jeremijasz z Tofet, gdzie go byl Pan poslal, aby tam prorokowal, stanal w sieni domu Panskiego, i rzekl do wszystkiego ludu:
\par 15 Tak mówi Pan zastepów, Bóg Izraelski: Oto Ja przywiode na to miasto i na wszystkie miasto jego wszystko to zle, którem wyrzekl przeciw niemu; bo zatwardzili kark swój, aby nie sluchali slów moich.

\chapter{20}

\par 1 Tedy uslyszawszy Fassur, syn Immerowy, kaplan, który byl postanowiony przedniejszym w domu Panskim, Jeremijasza prorokujacego o tem;
\par 2 Ubil Fassur Jeremijasza proroka, i dal go do wiezienia, które bylo najwyzsze w bramie Benjaminowej, a ta byla przy domu Panskim.
\par 3 A nazajutrz, gdy wywiódl Fassur Jeremijasza z wiezienia, rzekl do niego Jeremijasz: Nie nazwal cie Pan Fassurem, ale Magor Missabib.
\par 4 Bo tak mówi Pan: Oto Ja puszcze na cie strach, na cie i na wszystkich przyjaciól twoich, którzy upadna od miecza nieprzyjaciól swych, na co oczy twoje patrzyc beda; a wszystkiego Jude podam w rece króla Babilonskiego, który ich zaprowadzi do Babil onu, i pozabija ich mieczem.
\par 5 Dam tez wszystke majetnosc miasta tego, i wszstke prace jego, i wszystkie kosztowne rzeczy jego, i wszstkie skarby królów Judzkich dam w rece nieprzyjaciól ich; i rozchwyca je, i zabiora je, i zaprowadza je do Babilonu.
\par 6 Ale ty, Fassurze! i wszyscy, którzy mieszkaja w domu twym, pójdziecie w pojmanie, i do Babilonu przyjdziesz, i tam umrzesz, i tam pogrzebiony bedziesz; ty i wszyscy milujacy cie, którymes klamliwie prorokowal.
\par 7 Namówiles mie, Panie! a dalem sie namówic; mocniejszys byl niz ja, i przemogles; jestem na posmiech kazdy dzien, kazdy sie ze mnie nasmiewa.
\par 8 Bo jakom poczal mówic, wolam, dla gwaltu i spustoszenia krzycze; bo mi slowo Panskie jest ku pohanbieniu i na posmiech kazdy dzien.
\par 9 I rzeklem: Nie bede go wspominal, ani bede wiecej mówil w imieniu jego; ale slowo Boze jest w sercu mojem, jako ogien palajacy, zamkniony w kosciach moich, którym usilowal zatrzymac, alem nie mógl.
\par 10 Chociaz slysze uraganie od wielu i od Magor Missabiba, mówiacych: Powiedzcie co nan, a oznajmiemy to królowi. Wszyscy przyjaciele moi czyhaja na upadek mój, mówiac: Aza snac zwiedziony bedzie, i przemozemy go, a pomscimy sie nad nim.
\par 11 Alec Pan jest ze mna, jako mocarz straszny; przetoz ci, którzy mie przesladuja, upadna, a nie przemoga; bardzo beda pohanbieni, ze sobie niemadrze poczeli, hanba ich wieczna nie bedzie zapamietana.
\par 12 Przetoz, o Panie zastepów! który doswiadczasz sprawiedliwego, który wypatrujesz nerki i serce, niech widze pomste twoje nad nimi; tobiem zaiste odkryl sprawe moje.
\par 13 Spiewajciez Panu, chwalcie Pana, ze wybawil dusze ubogiego z reki zlosników.
\par 14 Przeklety dzien, w którym sie urodzil; dzien, którego mie porodzila matka moja, niech nie bedzie blogoslawiony.
\par 15 Przeklety maz, który oznajmil ojcu memu, mówiac: Urodziloc sie dziecie plci meskiej, aby go bardzo uweselil.
\par 16 Niechze bedzie on maz jako miasta, które Pan podwrócil, a nie zalowal tego; niech slyszy krzyk z poranku, i narzekanie czasu poludnia.
\par 17 O ze mie nie zabil zaraz z zywota! Oby mi byla matka moja grobem moim, a zywot jej wiecznie brzemiennym!
\par 18 Przeczzem wyszedl z zywota, abym doznal pracy i smutku, a zeby dni moje w hanbie strawione byly?

\chapter{21}

\par 1 Slowo, które sie stalo do Jeremijasza od Pana, gdy do niego król Sedekijasz poslal Fassura, syna Malchyjaszowego, i Sofonijasza, syna Maasejaszowego, kaplana, aby rzekli:
\par 2 Poradz sie, prosze, o nas Pana; bo Nabuchodonozor, król Babilonski, walczy przeciwko nam: owa snac uczyni Pan z nami wedlug wszystkich dziwnych spraw swoich, zeby odciagnal od nas.
\par 3 I rzekl Jeremijasz do nich: Tak powiedzcie Sedekijaszowi:
\par 4 Tak mówi Pan, Bóg Izraelski: Oto Ja odwróce naczynia wojenne, które sa w rekach waszych, i któremi wy walczycie przeciw królowi Babilonskiemu, i Chaldejczykom, którzy was oblegli okolo muru, i zgromadze ich w posrodek miasta tego.
\par 5 A Ja sam walczyc bede przeciwko wam reka wyciagniona i ramieniem moznem, a to w gniewie, i w popedliwosci, i w zapalczywosci wielkiej;
\par 6 I uderze obywateli tego miasta, tak, ze i ludzie i bydleta morem wielkim pomra.
\par 7 A potem, tak mówi Pan, podam Sedekijasza, króla Judzkiego, i slugi jego, i lud, to jest tych, którzy pozostana w tem miescie po morze, i po mieczu, i po glodzie, w reke Nabuchodonozora, króla Babilonskiego i w reke nieprzyjaciól ich, a tak w reke szukajacych duszy ich, który ich pobije ostrzem miecza: nie przepusci im, ani im sfolguje, ani sie zmiluje.
\par 8 Przetoz rzecz do ludu tego: Tak mówi Pan: Oto ja klade przed wami droge zywota i droge smierci.
\par 9 Ktokolwiek zostanie w tem miescie, zginie od miecza, albo od glodu, albo od moru: ale kto wyjdzie i poda sie Chaldejczykom, którzy was oblegli, pewnie zyw zostanie, i bedzie mu dusza jego w korzysci.
\par 10 Bom obrócil oblicze moje przeciwko temu miastu ku zlemu, a nie ku dobremu, mówi Pan. W reke króla Babilonskiego podane bedzie, i spali je ogniem.
\par 11 Ale domowi króla Judzkiego rzecz: Sluchajcie slowa Panskiego.
\par 12 O domie Dawidowy! Tak mówi Pan: Odprawujcie sad kazdego poranku, a wyrywajcie ucisnionego z reki gwaltownika, by snac nie wyszla jako ogien zapalczywosc moja, a nie gorzala, tak, zeby nie byl, ktoby ugasil dla zlosci spraw waszych.
\par 13 Otom Ja przeciwko tobie, która mieszkasz w tej dolinie, jako skala w tej równinie, mówi Pan, którzy mówicie: Któz przyciagnie na nas? a kto wnijdzie do przybytków naszych?
\par 14 Bo was nawiedze wedlug owocu spraw waszych, mówi Pan; a rozniece ogien w lesie twoim, który pozre wszystko okolo niego.

\chapter{22}

\par 1 Tak mówi Pan: Zstap do domu króla Judzkiego, a mów tam to slowo,
\par 2 I rzecz: Sluchaj slowa Panskiego, królu Judzki! który siedzisz na stolicy Dawidowej, ty i sludzy twoi, i lud twój, którzy chodzicie bramami temi.
\par 3 Tak mówi Pan: Czyncie sad i sprawiedliwosc, a wyzwalajcie ucisnionego z reki gwaltownika, a przychdniowi, sierotce i wdowie nie czyncie krzywdy, ani ich uciskajcie; ani krwi niewinnej nie wylewajcie na tem miejscu.
\par 4 Bo jezli czyniac uczynicie to slowo, tedy pewnie wnijda bramami domu tego królowie, siedzacy miasto Dawida na stolicy jego, jezdzacy na wozach i na koniach, sam król i sludzy jego, i lud jego.
\par 5 Lecz jezli nie posluchacie tych slów, sam na sie przysiegam, mówi Pan, ze ten dom pustynia bedzie.
\par 6 Bo tak mówi Pan o domie króla Judzkiego: Tys mi byl jako Galaad i wierzch Libanski; ale cie pewnie obróce w pustenie, i miasta, w których nie mieszkaja;
\par 7 I zgotuje na cie burzycieli, kazdego z orezem jego, którzy wyrabia wyborne cedry twoje, i wrzuca je na ogien.
\par 8 A gdy pójdzie wiele narodów mimo to miasto, i rzecze jeden do drugiego: Dlaczegoz tak uczynil Pan temu miastu wielkimu?
\par 9 Tedy odpowiedza: Przeto, iz opuscili przymierze Pana, Boga swego, a klaniali sie bogom cudzym, i sluzyli im.
\par 10 Nie placzciez umarlego, ani go zalujcie, ale ustawicznie placzcie nad tym, który odchodzi; bo sie wiecej nie wróci, aby ogladal ziemie, w której sie narodzil.
\par 11 Bo tak mówi Pan o Sellumie, synu Jozyjasza, króla Judzkiego, który króluje miasto Jozyjasza, ojca swego: Gdy wyjdzie z miejsca tego, nie wróci sie wiecej,
\par 12 Ale tam na onem miejscu, gdzie go przeniosa, umrze, a tak tej ziemi wiecej nie oglada.
\par 13 Biada temu, kto buduje dom swój z niesprawiedliwoscia, a palace swoje z krzywda, który blizniego swego darmo zniewala, a zaplaty mu jego nie daje!
\par 14 Który mówi: Zbuduje sobie dom wielki, i palace przestworne; i wycina sobie okna, a obija drzewem cedrowem, i maluje cynobrem.
\par 15 Izali bedziesz królowal, ze mieszkasz miedzy cedrami? Ojciec twój izali nie jadal i nie pijal? kiedy czynil sad i sprawiedliwosc, tedy sie mial dobrze;
\par 16 Gdy sadzil sprawe ubogiego, i nedznego, tedy sie mial dobrze; izali to nie jest poznac mie? mówi Pan.
\par 17 Ale oczy twoje i serce twoje nie szuka jedno lakomstwa swego, i abys krew niewinna wylewal, a gwalt i krzywde czynil.
\par 18 Przetoz tak mówi Pan o Joakimie, synu Jozyjasza, króla Judzkiego: Nie bada go plakac ani mówic: Ach bracie mój! albo: Ach siostro! Nie beda go plakac ani mówic: Ach panie! albo: Ach! gdziez dostojnosc jego?
\par 19 Pogrzebem oslim pogrzebiony badzie; wywleczony i wyrzucony bedzie za bramy Jeruzalemskie.
\par 20 Wstap na Liban, a wolaj, i na górze Basan wydaj glos twój; wolaj i u brodów, gdyz starci beda wszyscy milosnicy twoi.
\par 21 Mawialem z toba w najwiekszem szczesciu twojem; ales ty rzekla: Nie poslucham. Tac jest droga twoja od dziecinstwa twego, nie usluchales zaiste glosu mego.
\par 22 Wszystkich pasterzy twoich wiatr spasie, a milosnicy twoi w niewole pójda; tedy sie zaiste zapalac i wstydzic bedziesz dla wszelakiej zlosci twojej.
\par 23 O ty, która mieszkasz na Libanie, która sobie gniazdo czynisz na cedrach! jako wdzieczna bedziesz, gdy cie ogarna bolesci, a ucisk jako rodzaca.
\par 24 Jako zyje Ja, mówi Pan, iz chocby byl Chonijasz, syn Joakima, króla Judzkiego, sygnetem na prawej rece mojej, wszakze cie i stamtad zerwe;
\par 25 I podam cie w reke tych, którzy szukaj duszy twojej, i w reke tych, których sie ty twarzy lekasz, to jest, w reke Nabuchodonozora, króla Babilonskiego, i w reke Chaldejczyków;
\par 26 A wyrzuce cie i matke twoje, która cie urodzila, do ziemi cudzej, w którejscie sie nie rodzili, i tam pomrzecie.
\par 27 Ale do ziemi, do której tesknic bedziecie, abyscie sie tam wrócili, nie wrócicie sie.
\par 28 Izali ten maz Chonijasz bedzie balwanem nikczemnym, który podruzgotany bywa? Albo naczyniem, w którem niemasz zednej wdziecznosci? Przeczzeby odrzuceni byli on i nasienie jego, a wyrzuceni do ziemi, której nie znaja?
\par 29 O ziemio, ziemio, ziemio! sluchaj slowa Panskiego.
\par 30 Tak mówi Pan: Zapiszcie to, ze ten maz bedzie bez dzieci, a ze mu sie nie poszczesci za dni jego; owszem, nie poszczesci sie i mezowi, któryby z nasienia jego siedzial na stolicy Dawidowej, a panowal jeszcze w Judzie.

\chapter{23}

\par 1 Biada pasterzom gubiacym i rozpraszajacym trzode pastwiska mego! mówi Pan.
\par 2 Przetoz tak mówi Pan, Bóg Izraelski, do pasterzy, którzy pasa lud mój: Wy rozpraszacie owce moje, owszem, rozganiacie je, a nie nawiedzacie ich; oto Ja nawiedze was dla zlosci spraw waszych, mówi Pan.
\par 3 A ostatek owiec moich Ja zgromadze ze wszystkich ziem, do którychem je rozegnal, i przywróce je do obór ich, gdzie sie rozplodza i rozmnoza.
\par 4 Nadto posanowie nad niemi pasterzy, którzyby je pasli, aby sie wiecej nie lekaly, ani strachaly, i zeby ich nie ubywalo, mówi Pan.
\par 5 Oto ida dni, mówi Pan, których wzbudze Dawidowi latorosl sprawiedliwa, i bedzie królowal król, a poszczesci mu sie; sad zaiste i sprawiedliwosc bedzie czynil na ziemi.
\par 6 Za dni jego Juda zbawiony bedzie, a Izrael bezpiecznie mieszkac bedzie; a toc jest imie jego, którem go zwac beda: Pan sprawiedliwosc nasza.
\par 7 Przetoz oto przychodza dni, mówi Pan, których nie rzeka wiecej: Jako zyje Pan, który wywiódl synów Izraelskich z ziemi Egipskiej.
\par 8 Ale: Jako zyje Pan, który wywiódl, i który sprowadzil nasienie domu Izraelskiego z ziemi pólnocnej i ze wszystkich ziem, do którychem ich byl rozegnal; gdyz beda mieszkac w ziemi swoiej.
\par 9 Dla proroków skruszylo sie serce moje we mnie, poruszyly sie wszystkie kosci moje, stalem sie jako czlowiek pijany, a jako maz, po którym sie rozeszlo wino, dla Pana i dla slów swietobliwosci jego.
\par 10 Bo ta ziemia pelna jest cudzolozników, a dla krzywoprzysiestwa placze ta ziemia, pastwiska na puszczy poschly; zaiste zly jest bieg ich, a moc ich nieprawa.
\par 11 Bo i prorok i kaplan sa obludnikami, a domu moim znajduje sie zlosc ich, mówi Pan.
\par 12 Przetoz im bedzie droga ich jako slizgawica w ciemnosci, na która wepchnieni beda i upadna, gdy przywiode na nich biede czasu nawiedzenia ich, mówi Pan.
\par 13 Takze przy prorokach Samaryjskich widzialem glupstwo, prorokowali przez Baale, i zwodzili lud mój Izraelski.
\par 14 Ale przy prorokach Jeruzalemskich widze rzecz brzydka, ze cudzolozac i klamliwie sie obchodzac utwierdzaja tez rece zlosników, aby sie nie nawracali zaden od zlosci swojej; stali sie wszyscy przedemna jako Sodoma, a obywatele jego jako Gomora.
\par 15 Dlategoz tak mówi Pan zastepów o tych prorokach: Oto Ja nakarmie ich piolunem, a napoje ich woda gorzka; bo od proroków Jeruzalemskich wyszlo splugawienie na te wszystke ziemie.
\par 16 Tak mówi Pan zastepów: Nie sluchajcie slów tych proroków, którzy wam prorokuja, i zwodza was, widzenie serca swego opowiadaja, a nie z ust Panskich;
\par 17 Ustawicznie mówia tym, którzy mna gardza: Pan mówil, pokój miec bedziecie; i kazdemu chodzacemu wedlug uporu serca swego, mówia: Nie przyjdzie na was nic zlego.
\par 18 Bo któz stanal w radzie Panskiej, a widzial i slyszal slowo jego? kto pilnowal slowa jego, i sluchal go?
\par 19 Oto wicher Panski z zapalczywoscia wyjdzie, a wicher trwaly nad glowa niepoboznych zostanie;
\par 20 Nie odwróci sie gniew Panski, az uczyni i wykona mysli serca swego; w ostateczne dni to doskonale zrozumiecie.
\par 21 Nie posylalem tych proroków, a wszakze biezeli; nie mówilem do nich, a wszakze oni prorokowali,
\par 22 Bo gdyby byli stali w radzie mojej, tedyby byli oglaszali slowa moje ludowi memu, a byliby ich odwracali od drogi ich zlej, i od zlosci spraw ich.
\par 23 Izalim Ja tylko Bogiem z bliska? mówi Pan, a nie jestem Bogiem i z daleka?
\par 24 Izali sie kto skryje w skrytosci, abym go Ja nie widzial? mówi Pan. Izali Ja nieba i ziemi nie napelniam? mówi Pan.
\par 25 Slyszec Ja, co mówia prorocy, którzy prorokuja, klamstwo w imieniu mojem, mówiac: Snilo mi sie, snilo mi sie!
\par 26 Dlugoz tego bedzie? Izali w sercu tych proroków, którzy prorokuja, niemasz klamstwa? owszem, sa prorokami zdrady serca swego;
\par 27 Którzy mysla, jakoby z pamieci wywiesc ludowi mojemu imie moje snami swemi, które powiadaja kazdy blizniemu swemu, jako zapomnieli ojcowie ich na imie moje dla Baala.
\par 28 Prorok, który ma sen, niech powiada sen: ale który ma slowo moje, niech mówi slowo moje w prawdzie. Cóz plewie do pszenicy? mówi Pan.
\par 29 Izali slowo moje nie jest jako ogien? mówi Pan, i jako mlot kruszacy skale?
\par 30 Przetoz oto Ja powstaje przeciwko tym prorokom, mówi Pan, którzy kradna slowa moje, kazdy przed bliznim swoim.
\par 31 Oto Ja powstaje przeciwko tym prorokom, mówi Pan, którzy uzywaja jezyka swego, aby mówili: Mówi Pan.
\par 32 Oto Ja powstaje, mówi Pan, przeciwko tym, którzy prorokuja sny klamliwe, a opowiadajac je zwodza lud mój klamstwami swemi i plotkami swemi, chociazem Ja ich nie poslal, anim im rozkazal; skad zgola nic nie pomagaja ludowi twemu, mówi Pan.
\par 33 Przetoz gdyby sie ciebie pytal lud ten, albo prorok, albo kaplan, mówiac: Cóz jest za brzemie Panskie? Tedy rzeczesz do nich które brzemie. To: Opuszcze was, mówi Pan.
\par 34 Bo proroka i kaplana tego, i lud ten, któryby rzekl: Toc jest brzemie Panskie, pewnie nawiedze meza tego i dom jego.
\par 35 Ale tak mówcie kazdy do blizniego swego, i kazdy do brata swego: Cóz odpowiedzial Pan? albo: Cóz mówil Pan?
\par 36 A brzmienia Panskiego nie wspominajcie wiecej; bo brzmieniem bedzie kazdemu slowo jego, gdyzescie wy wywrócili slowa Boga zywego, Pana zastepów, Boga naszego.
\par 37 Tak tedy rzeczesz do proroka: Cóz ci odpowiedzial Pan? albo: Co mówil Pan?
\par 38 Ale poniewaz mówicie: Brzemie Panskie, tedy tak mówi Pan: Poniewaz mówicie to slowo: Brzemie Panskie, chociazem do was posylal, mówiac: Nie mówcie: Brzemie Panskie;
\par 39 Przetoz oto i Ja was zapomne do konca, i odrzuce was i to miasto, którem wam dal i ojcom waszym, od oblicza mego;
\par 40 I podam was na uraganie wieczne, i na hanbe wieczna, która nigdy nie przyjdzie w zapamietanie.

\chapter{24}

\par 1 Ukazal mi Pan, a oto dwa kosze fig postawione byly przed kosciolem Panskim, gdy byl w niewole zabral Nabuchodonozor, król Babilonski, Jechonijasza, syna Joakima, króla Judzkiego, i ksiazat Judzkich, i ciesli, i kowali z Jeruzalemu, a zawiódl ich do Babilonu.
\par 2 Kosz jeden mial figi bardzo dobre, jako bywaja figi dojrzale; a kosz drugi mial figi bardzo zle, których jesc nie mozna, przeto, iz byly zle.
\par 3 I rzekl Pan do mnie: Cóz widzisz? Jeremijaszu! I rzeklem: Figi. Figi dobre sa bardzo dobre, a zle sa bardzo zle, których jesc nie moga, przeto, iz sa zle.
\par 4 I stalo sie slowo Panskie do mnie, mówiac:
\par 5 Tak mówi Pan, Bóg Izraelski: Jako te figi sa dobre, tak mi przyjemni beda w niewole zaprowadzeni z Judy, którychem wyslal z miejsca tego do ziemi Chaldejskiej ku dobremu.
\par 6 I obróce oko moje na nich ku dobremu, i przywiode ich do tej ziemi, gdzie ich pobuduje, a nie skaze, i wszczepie ich, a nie wykorzenie.
\par 7 Albowiem dam im serce, aby mie poznali, zem Ja Pan; i beda mi ludem moim, a Ja bede Bogiem ich, gdy sie nawróca do mnie calem sercem swojem.
\par 8 A jako figi zle, których nie jadaja, przeto, ze sa zle, tak zarzuce (toc zaiste mówi Pan,)Sedekijasza, króla Judzkiego, i ksiazat jego, i ostatki z Jeruzalemu, które pozostaly w tej ziemi, i tych, którzy mieszkaja w ziemi Egipskiej.
\par 9 Podam ich, mówie, na utrapienie, i na ucisk po wszystkich królestwach ziemi, na pohanbienie i na przypowiesc, na przyslowie i na przeklestwo po wszystkich miejscach, do których ich zapedze;
\par 10 I bede posylal na nich miecz, glód i mór, az do konca wytraceni beda z ziemi, któram byl dal im, i ojcom ich.

\chapter{25}

\par 1 Slowo, które sie stalo do Jeremijasza przeciwko wszystkiemu ludowi Judzkiemu roku czwartego Joakima, syna Jozyjaszowego, króla Judzkiego, (który jest rok pierwszy Nabuchodonozora, króla Babilonskiego;)
\par 2 Które mówil Jeremijasz prorok do wszystkiego ludu Judzkiego, i do wszystkich obywateli Jeruzalemskich mówiac;
\par 3 Od trzynastego roku Jozyjasza, syna Amonowego, króla Judzkiego, az do dnia tego, juz to przez dwadziescia i trzy lata bywalo slowo Panskie do mnie, którem do was mawial, rano wstawajac i opowiadajac; alescie nie sluchali.
\par 4 Posylal tez Pan do was wszystkich slug swoich proroków, rano wstawajac, i posylajac, (którychescie nie usluchali, aniscie naklonili ucha swego, abyscie slyszeli;)
\par 5 Którzy mówili: Nawróccie sie teraz kazdy od zlej drogi swojej, i od zlosci spraw waszych; a tak bedziecie mieszkali w tej ziemi, która wam dal Pan, i ojcom waszym od wieku az na wieki.
\par 6 A nie chodzcie za bogami cudzymi, abyscie im sluzyli, i klaniali sie im, ani mie gniewajcie sprawa rak waszych, a Ja wam zle nie uczynie.
\par 7 Alescie mie nie usluchali, mówi Pan, abyscie mie pobudzali do gniewu sprawa rak swioch na swe zle.
\par 8 Przetoz tak mówi Pan zastepów: Dlatego, izescie nie usluchali slów moich,
\par 9 Oto Ja posle i pobiore wszystkie narody pólnocne, mówi Pan, i do Nabuchodonozora króla Babilonskiego, slugi mego, i przywiode je na te ziemie, i na obywateli jej, i na te wszystkie narody okoliczne, które do szczetu wygladze, i poloze je na podziw, i na poswistanie, i na spustoszenie wieczne.
\par 10 I sprawie to, aby im zginal glos wesela, i glos radosci, glos oblubienca, i glos oblubienicy, glos zarn, i swiatlosc pochodni,
\par 11 I bedzie ta wszystka ziemia spustoszeniem, i zdumieniem, a sluzyc beda te narody królowi Babilonskiemu siedmdziesiat lat.
\par 12 Ale potem, gdy sie wypelni siedmdzisisat lat, nawiedze na królu Babilonskim i na tym narodzie, mówi Pan, nieprawosc ich, i na ziemi Chaldejskiej, tak, ze ja obróce w pustynie wieczne.
\par 13 A przywiode na te ziemie wszystkie slowa moje, którem mówil o niej, mianowicie to wszystko, co napisano w tych ksiegach, cokolwiek prorokowal Jaremijasz o wszystkich narodach.
\par 14 Gdyz w niewole podbici beda od narodów, takze jako i oni moznych, i od królów wielkich, tedy im oddam wedlug spraw ich i wedlug uczynków rak ich.
\par 15 Bo tak rzekl Pan, Bóg Izraelski, do mnie: Wezmij kubek wina tej popedliwosci z reki mojej, a napawaj nim wszystkie narody, do których Ja ciebie posle;
\par 16 Aby pili i potaczli sie, owszem, aby szaleli od ostrza miecza, który Ja posle miedzy nich.
\par 17 Wzialem tedy kubek z reki Panskiej, i napoilem wszystkie one narody, do których mie Pan poslal:
\par 18 Jeruzalem, i miasta ziemi Judzkiej, i królów jej, i ksiazat jej, abym ich podal na spustoszenie, na zdumienie, na poswistanie, i na przeklestwo, jako sie to dzis okazuje.
\par 19 Faraona tez, króla Egipskiego, i slug jego, i ksiazat jego, i wszystek lud jego;
\par 20 I to wszystko pospólstwo, takze wszystkich króli ziemi Uz, i wszystkich króli ziemi Filistynskiej, i Aszkalon, i Gaze, i Akkaron, i ostatek Azotu;
\par 21 Edomczyków, i Moabczyków, i synów Ammonowych;
\par 22 I wszystkich królów Tyrskich, i wszystkich królów Sydonskich, i królów tej krainy, która jest przy morzu;
\par 23 Dedana i Teme, i Buze, i wszystkich, którzy mieszkaja w ostatnich katach:
\par 24 I wszystkich królów Arabskich, i wszystkich królów tego pospólstwa, które mieszka na puszczy;
\par 25 Takze wszystkich królów Zymry i wszystkich królów Elam, i wszystkich królów Medskich;
\par 26 Owszem, wszystkich królów pólnocnych, bliskich i dalekich, jednego jako drugiego; wszystkie tez królestwa ziemi, którekolwiek sa na obliczu ziemi; a król Sesak bedzie pil po nich.
\par 27 I rzecz do nich: Tak mówi Pan zastepów, Bóg Izraelski: Pijcie, a popijcie sie, owszem zwracajcie, i padajcie tak, abyscie nie powstali dla miecza, który Ja posle miedzy was.
\par 28 A jezliby nie chcieli wziac kubka z reki twojej, aby pili, tedy rzeczesz do nich: Tak mówi Pan zastepów: Koniecznie pic musicie.
\par 29 Bo poniewaz na to miasto, które nazwane jest od imienia mego, Ja zaczynam przywodzic zle rzecz, a wybyscie bez karania byc mieli? Nie bedziecie bez karania; bom Ja miecz przyzwal na wszystkich obywateli tej ziemi, mówi Pan zastepów.
\par 30 Przetoz ty prorokuj przeciwko nim te wszyskie slowa, a mów do nich: Pan z wysoka zaryczy, a mieszkania swietobliwosci swojej wyda glos swój, ryczac zaryczy z mieszkania swego; krzyk pobudzajacych sie jako tloczacych prase, rozlegac sie bedzie prze ciwko wszystkim obywatelom tej ziemi.
\par 31 I przejdzie huk az do konczyn ziemi; bo sie Pan rozpiera z tymi narodami, w sad sam wchodzi ze wszelkiem cialem, niezboznych poda pod miecz, mówi Pan.
\par 32 Tak mówi Pan zastepów: Oto udreczenie pójdzie z narodu do narodu, a wicher wielki powstanie od konczyn ziemi;
\par 33 I beda pobici od Pana czasu onego od konca ziemi az do konca ziemi; nie beda ich plakac, ani zbierac, ani chowac; beda jako gnój na polu.
\par 34 Narzekajcie pasterze i wolajcie, a walajcie sie w popiele, wy najzacniejsi tej trzody! bo sie wypelnily dni wasze, zabicia i rozproszenia waszego, i upadniecie jako naczynie drogie.
\par 35 I zginie ucieczka pasterzom, a ujscie najzacniejszym tej trzody.
\par 36 Glos wolania pasterzy, i narzekanie najzacniejszych tej trzody slychac bedzie; bo Pan spustoszy pastwiska ich.
\par 37 I zagubione beda spokojne pastwiska dla zapalczywosci gniewu Panskiego,
\par 38 Który opusci jako lew jaskinie swoje; bo ziemia ich przyjdzie na spustoszenie dla zapalczywosci pustoszyciela, i dla popedliwosci gniewu jego.

\chapter{26}

\par 1 Na poczatku królowania Joakima, syna Jozyjasza, króla Judzkiego, stalo sie to slowo od Pana, mówiac:
\par 2 Tak mówi Pan: Stan w sieni domu Panskiego, a mów do wszystkich miast Judzkich, do przychodzacych klaniac sie w domu Panskim, wszystkie slowa, którec rozkazuje mówic do nich; nie ujmuj i slowa.
\par 3 Owa snac usluchaja, a nawróca sie kazdy od zlej drogi swej, abym pozalowal zlego, którem im umyslil uczynic dla zlosci spraw ich.
\par 4 I rzecz do nich: Tak mówi Pan: Jezli mie nie usluchacie, zebyscie chodzili w zakonie moim, którym wam przedlozyl.
\par 5 Sluchajac slów slug moich proroków, których Ja posylam do was, jakoscie, gdym ich rano wstawajac posylal, nie usluchali:
\par 6 Tedy uczynie temu domowi jako Sylo, a to miasto dam na przeklestwo wszystkim narodom ziemi.
\par 7 A kaplani i prorocy i wszystek lud slyszeli Jeremijasza mówiacego te slowa w domu Panskim.
\par 8 A skoro przestal Jeremijasz mówic wszystko, co mu byl rozkazal Pan mówic wszystkiemu ludowi, pojmali go oni kaplani i prorocy, i wszystek on lud mówiac: Smiercia umrzesz.
\par 9 Czemus prorokowal w imie Panskie, mówiac: Stanie sie temu domowi jako Sylo, a to miasto tak spustoszeje, ze w niem nie bedzie obywatela? I zgromadzal sie wszystek lud przeciwko Jeremijaszowi do domu Panskiego.
\par 10 Tedy uslyszawszy te rzeczy ksiazeta Judzcy, przyszli z domu królewskiego do domu Panskiego, i usiedli w wejsciu nowej bramy Panskiej.
\par 11 I rzekli kaplani i prorocy do onych ksiazat, i do wszystkiego ludu, mówiac: Ten maz godzien jest smierci; bo prorokowal przeciwko miastu temu, jakoscie to slyszeli w uszy swoje.
\par 12 Tedy rzekl Jeremijasz do wszystkich ksiazat, i do wszystkiego ludu, mówiac: Pan mie poslal, abym prorokowal o tym domu i o tem miescie wszystko to, coscie slyszeli.
\par 13 Przetoz teraz poprawcie drogi swoje i sprawy swe, a usluchajcie glosu Pana, Boga swego, a pozaluje Pan tego zlego, które wyrzekl przeciwko wam.
\par 14 A ja otom jest w rekach waszych; czyncie zemna, co dobrego i sprawiedliwego jest w oczach waszych.
\par 15 A wszakze zapewne wiedzcie, jezliz mie zabijecie, ze krew niewinna zwalicie na sie i na to miasto, i na obywateli jego; bo mie zaprawde Pan do was poslal, abym mówil te wszystkie slowa w uszy wasze.
\par 16 I rzekli ksiazeta i wszystek lud do kaplanów i do onych proroków: Niema byc zadnym sposobem ten maz osadzony na smierc, poniewaz w imieniu Pana, Boga naszego, mówil do nas.
\par 17 Tedy powstali niektórzy z starszych onej ziemi, i uczynili rzecz do wszystkiego zgromadzenia ludu, mówiac:
\par 18 Micheasz Morastytczyk prorokowal za dni Ezechijasza, króla Judzkiego, i rzekl do wszystkiego ludu Judzkiego, mówiac: Tak mówi Pan zastepów: Syon jako pole poorany bedzie, a Jeruzalem jako gromady rumu bedzie, a góra domu tego jako wysokie lasy.
\par 19 Izali go zaraz dlatego zabili Ezechijasz, król Judzki, i wszystek Juda? izali sie nie ulekl Pana, a nie modlil sie Panu? I zalowal Pan onego zlego, które byl wyrzekl przeciwko nim: przetoz my zla rzecz czynimy przeciwko duszom naszym.
\par 20 Takze tez byl maz prorokujacy w imieniu Panskiem, Uryjasz, syn Semejaszowy, z Karyjatyjarym, który prorokowal o tem miescie i o tej ziemi wedlug wszystkich slów Jeremijaszowych.
\par 21 A gdy uslyszal król Joakim, i wszystko rycerstwo jego, i wszyscy ksiazeta slowa jego, zaraz go chcial król zabic, co uslyszawszy Uryjasz ulakl sie, a ucieklszy przyszedl do Egiptu.
\par 22 Ale król Joakim poslal niektórych do Egiptu, Elnatana, syna Achborowego, i innych z nim do Egiptu;
\par 23 Którzy wywiódlszy Uryjasza z Egiptu, przywiedli go do króla Joakima; i zabil go mieczem, i wrzucil trupa jego do grobów ludu pospolitego.
\par 24 Wszakze reka Achikama, syna Safanowego, byla przy Jeremijaszu, aby nie byl wydan w rece ludu, i nie byl zabity.

\chapter{27}

\par 1 Na poczatku królowania Joakima, syna Jozyjasza, króla Judzkiego, stalo sie to slowo do Jeremijasza od Pana, mówiac:
\par 2 Tak mówi Pan do mnie: Uczyn sobie okowy i jarzma, a wlóz je na szyje swoje:
\par 3 Potem je poslij do króla Edomskiego, i do króla Moabskiego, i do króla synów Ammonowych, i do króla Tyrskiego, i do króla Sydonskiego, przez reke poslów, którzy przyjda do Jeruzalemu do Sedekijasza, króla Judzkiego;
\par 4 A rozkaz im, aby panom swym powiedzieli: Tak mówi Pan zastepów, Bóg Izraelski, tak powiecie panom waszym:
\par 5 Jam uczynil ziemie, takze czlowieka i bydleta, którekolwiek sa na obliczu ziemi, moca swoja wielka i ramieniem swojem wyciagnionem: przeto ja daje temu, który sie podoba oczom moim.
\par 6 A teraz dalem te wszystkie ziemie w reke Nabuchodonozora, króla Babilonskiego, slugi mego; nadto i zwierzeta polne dalem mu, aby mu sluzyly.
\par 7 Przetoz beda mu sluzyly te wszystkie narody, i synowi jego i synowi syna jego, dokadby nie przyszedl czas ziemi jego i jego samego, gdy go tez zas sobie w niewole podbija narody zacne, i królowie wielcy.
\par 8 A ten naród i to królestwo, któreby mu nie sluzylo, to jest Nabuchodonozorowi, królowi Babilonskiemu, i któryby nie poddal szyi swojej pod jarzmo króla Babilonskiego, mieczem, i glodem, i morem nawiedze ten naród, mówi Pan, dokadbym ich do konca nie wytracil reka jego.
\par 9 Przetoz wy nie sluchajcie proroków swoich, i wieszczków swoich, ani snów swoich, ani wrózków swoich, i czarowników swoich, którzy wam powiadaja, mówiac: Nie bedziecie sluzyli królowi Babilonskiemu;
\par 10 Bo wam oni klamstwo prorokuja, abym was oddalil od ziemi waszej, a wygnal was, abyscie pogineli.
\par 11 A naród, który poddal szyje swa pod jarzmo króla Babilonskiego, a któryby mu sluzyl, ten zaiste zostawie w ziemi jego, (mówi Pan,)aby ja sprawowal, i mieszkal w niej.
\par 12 A do Sedekijasza, króla Judzkiego, rzeklem wedlug tych wszystkich slów, mówiac: Poddajcie szyje wasze pod jarzmo króla Babilonskiego, a sluzcie mu i ludowi jego, a zyc bedziecie.
\par 13 Przeczze macie zginac, ty i lud twój, od miecza, od glodu i od powietrza, jako mówil Pan o narodzie, któryby nie sluzyl królowi Babilonskiemu?
\par 14 Nie sluchajciez tedy slów tych proroków, którzy mówiac do was powiadaja: Nie bedziecie sluzyli królowi Babilonskiemu; albowiem wam oni klamstwo prorokuja.
\par 15 Nie poslalem ich zaiste, mówi Pan; wszakze oni prorokuja w imie moje klamliwie, abym was zapedzil, gdziebyscie zgineli, wy i ci prorocy, którzy wam prorokuja.
\par 16 Do kaplanów tez, i do tego wszystkiego ludu rzeklem, mówiac: Tak mówi Pan: Nie sluchajcie slów proroków swoich, którzy wam prorokuja, mówiac: Oto naczynia domu Panskiego teraz w rychle przywrócone beda z Babilonu; bo wam oni klamstwo prorokuja.
\par 17 Nie sluchajciez ich; sluzcie królowi Babilonskiemu, a zyc bedziecie; przeczzeby to miasto mialo byc pustynia?
\par 18 A jezliz oni sa prorokami, i jezli jest slowo Panskie w nich, prosze, niech zabieza Panu zastepów, aby te naczynia, które pozostaly w domu Panskim; i w domu króla Judzkiego, i w Jeruzalemie, nie dostaly sie do Babilonu.
\par 19 Bo tak mówi Pan zastepów o tych slupach, i o tem morzu, i o tych podstawkach, i o ostatku naczynia, które pozostalo w tem miescie;
\par 20 Którego nie zabral Nabuchodonozor, król Babilonski, gdy w niewole prowadzil Jechonijasza, syna Joakimowego, króla Judzkiego, z Jeruzalemu do Babilonu, i wszystkich przedniejszych, z Judy i z Jeruzalemu:
\par 21 Tak zaiste mówi Pan zastepów, Bóg Izraelski, o tych naczyniach, które pozostaly w domu Panskim, i w domu króla Judzkiego, i w Jeruzalemie:
\par 22 Do Babilonu zawiezione beda, a beda tam az do dnia, którego je nawiedze, mówi Pan, i kaze je przywiezc i przywrócic na to miejsce.

\chapter{28}

\par 1 I stalo sie roku onego, na poczatku królowania Sedekijasza, króla Judzkiego, roku czwartego, miesiaca piatego: Hananijasz, syn Asurowy, prorok, który byl z Gabaonu, rzekl do mnie w domu Panskim przed kaplanami i przed wszystkim ludem, mówiac:
\par 2 Tak mówi Pan zastepów, Bóg Izraelski: Polamalem jarzmo króla Babilonskiego;
\par 3 Po dwóch latach przywróce zasie na to miejsce wszystkie naczynia domu Panskiego, które pobral Nabuchodonozor, król Babilonski, z tego miejsca, a zawiózl je do Babilonu.
\par 4 Takze Jechonijasza, syna Joakimowego, króla Judzkiego, i wszystkich, którzy sa w niewole zaprowadzeni z Judy, którzy sie dostali do Babilonu, Ja zasie przywiode na to miejsce, mówi Pan; bo skrusze jarzmo króla Babilonskiego.
\par 5 Tedy rzekl Jeremijasz prorok do Hananijasza proroka przed oczyma kaplanów, i przed oczyma wszystkiego ludu, którzy stali w domu Panskim;
\par 6 Rzekl, mówie, Jeremijasz prorok: Amen, niech tak uczyni Pan; niech utwierdzi Pan slowa twoje, któres prorokowal o przywróceniu z Babilonu na to miejsce naczynia domu Panskiego, i wszystkich, którzy sa zaprowadzeni w niewole.
\par 7 Wszakze posluchaj teraz slowa tego, które ja mówie w uszy twoje, i w uszy tego wszystkiego ludu.
\par 8 Prorocy, którzy byli przedemna i przed toba z dawna, ci prorokowali przeciwko ziemiom zacnym, i przeciwko królestwom wielkim o wojnie, i o ucisnieniu, i o morze.
\par 9 Ten prorok, który prorokuje o pokoju, ten prorok, mówie, wtenczas poznany bywa, ze go Pan prawdziwie poslal, gdy sie isci slowo jego.
\par 10 Tedy zdjal Hananijasz prorok jarzmo z szyi Jeremijasza proroka, i polamal je.
\par 11 I rzekl Hananijasz przed oczyma wszystkiego ludu, mówiac: Tak mówi Pan: Tak polamie jarzmo Nabuchodonozora, króla Babilonskiego, po dwóch latach z szyi wszystkich narodów. I poszedl Jeremijasz prorok w droge swoje.
\par 12 Ale stalo sie slowo Panskie do Jeremijasza, gdy polamal Hananijasz prorok ono jarzmo z szyi Jeremijasza proroka, mówiac:
\par 13 Idz; a rzecz do Hananijasza, mówiac: Tak mówi Pan: Polamales jarzma drewniane; przetoz uczyn na to miejsce jarzmo zelazne.
\par 14 Bo tak mówi Pan zastepów, Bóg Izraelski: Jarzmo zelazne wloze na szyje tych wszystkich narodów, aby sluzyly Nabuchodonozorowi, królowi Babilonskiemu, i beda mu sluzyly; takze i zwierzeta polne podalem mu.
\par 15 Zatem rzekl Jeremijasz prorok do Hananijasza proroka: Sluchaj teraz Hananijaszu! Nie poslal cie Pan, a tys kazal nadzieje miec temu ludowi w klamstwie.
\par 16 Przetoz tak mówi Pan: Oto Ja ciebie uprzatne z tej ziemi, tego roku umrzesz; bos radzil, aby odstapil lud od Pana.
\par 17 I umarl Hananijasz prorok onegoz roku, miesiaca siódmego.

\chapter{29}

\par 1 A tec sa slowa listu, który poslal Jeremijasz prorok z Jeruzalemu do ostatków starszych, którzy byli w pojmaniu, i do kaplanów, i do proroków, i do wszystkiego ludu, których byl przeniósl Nabuchodonozor z Jeruzalemu do Babilonu,
\par 2 Gdy wyszedl Jechonijasz król i królowa, i komornicy, ksiazeta Judzcy, i Jeruzalemscy, takze ciesle i kowale z Jeruzalemu;
\par 3 Przez Elhasa, syna Safanowego, i Giemaryjasza, syna Helkijaszowego, (których byl poslal Sedekijasz, król Judzki, do Nabuchodonozora, króla Babilonskiego, do Babilonu) mówiac:
\par 4 Tak mówi Pan zastepów, Bóg Izraelski, do wszystkich pojmanych, którychem przeniósl z Jeruzalemu do Babilonu:
\par 5 Budujcie domy, i osadzajcie sie; szczepcie tez sady, a jedzcie owoc ich;
\par 6 Pojmujcie zony, a plodzcie synów i córki, i dawajcie synom waszym zony, a córki wasze wydawajcie za maz, aby rodzily synów i córki; rozmnazajcie sie tam, i niech was nie ubywa.
\par 7 Szukajcie tez pokoju miastu temu, do któregom was przeniósl, a módlcie sie za nie Panu; bo w pokoju jego bedziecie mieli pokój.
\par 8 Tak zaiste mówi Pan zastepów, Bóg Izraelski: Niech was nie zwodza prorocy wasi, którzy sa miedzy wami, i wieszczkowie wasi, a nie sprawujcie sie snami waszemi, które sie wam snia.
\par 9 Bo wam oni klamliwie prorokuja w imieniu mojem; nie poslalem ich, mówi Pan.
\par 10 Tak zaiste mówi Pan: Jako sie jedno wypelnia siedmdziesiat lat Babilonowi, nawiedze was, a potwierdze wam slowa swego wybornego o przywróceniu was na to miejsce.
\par 11 Bo Ja najlepiej wiem mysli, które mysle o was, mówi Pan: mysli o pokoju, a nie o utrapieniu, abym uczynil koniec pozadany biedom waszym.
\par 12 Gdy mie wzywac bedziecie, a pójdziecie, i modlic mi sie bedziecie, tedy was wyslucham;
\par 13 A szukajac mie, znajdziecie; gdy mie szukac bedziecie ze wszystkiego serca swego,
\par 14 Dam sie wam zaiste znalesc, mówi Pan: a przywróce wiezniów waszych, i zgromadze was ze wszystkich narodów, i ze wszystkich miejsc, gdziemkolwiek was zagnal, mówi Pan; i przyprowadze was na to miejsce, skadem was wyprowadzil.
\par 15 Gdy rzeczecie: Wzbudzal nam Pan proroków prorokujacych o zaprowadzeniu do Babilonu.
\par 16 Bo tak mówi Pan o królu, który siedzi na stolicy Dawidowej, i o wszystkim ludu, który mieszka w tem miescie, braciach waszych, którzy nie wyszli z wami w te niewole;
\par 17 Tak mówi Pan zastepów: Oto Ja posle na nich miecz, glód, i mór, a uczynie ich jako zle figi, których nie jadaja, przeto, ze sa zle.
\par 18 Albowiem przesladowac ich bede mieczem, glodem i morem, i dam ich na potlukanie po wszystkich królestwach ziemi, na przeklestwo, i na zdumienie, owszem, na poswistanie, i na uraganie miedzy wszystkimi narodami, tam, gdzie ich zapedze,
\par 19 Przeto, iz nie sluchaja slów moich, mówi Pan, gdy posylam do nich slug swoich, proroków, rano wstawajac i posylajac; a nie sluchaliscie, mówi Pan.
\par 20 Przetoz sluchajcie slowa Panskiego wy wszyscy pojmani, którychem wyslal z Jeruzalemu do Babilonu.
\par 21 Tak mówi Pan zastepów, Bóg Izraelski, o Achabie, synu Kolajaszowym, i o Sedekijaszu, synu Maazejaszowym, którzy wam prorokuja w imieniu mojem klamstwo: Oto Ja podam ich w reke Nabuchodonozora, króla Babilonskiego, aby ich pobil przed oczyma wasze mi.
\par 22 I bedzie wziete z nich przeklinanie miedzy wszystkich pojmanych z Judy, którzy sa w Babilonie, aby mówili: Niech cie uczyni Pan jako Sedekijasza, i jako Achaba, których upiekl król Babilonski na ogniu,
\par 23 Przeto, ze popelniali zlosc w Izraelu, cudzolozac z zonami bliznich swoich, a klamliwie mówiac slowo w imieniu mojem, czegom im nie przykazal; a Ja o tem wiem, i jestem tego swiadkiem, mówi Pan.
\par 24 A do Semejasza Nechalamity rzecz, mówiac:
\par 25 Tak powiada Pan zastepów, Bóg Izraelski, mówiac: Przeto zes ty poslal imieniem swojem listy do wszystkiego ludu, który jest w Jeruzalemie, i do Sofonijasza, syna Maazejaszowego, kaplana, i do wszystkich kaplanów, mówiac:
\par 26 Pan cie dal za kaplana miasto Jojady kaplana, abyscie mieli dozór w domu Panskim na kazdego meza w rozum zaszlego a prorokujacego, abys dal takowego do wiezienia i do klody.
\par 27 Przeczzes tedy teraz nie zgromil Jeremijasza Anatotczyka, który wam prorokuje?
\par 28 Bo poslal do nas do Babilonu, mówiac: Jeszczec dlugo czekac; budujcie domy, a osadzajcie sie, szczepcie sady, a jedzcie owoce ich.
\par 29 Bo Sofonijasz kaplan czytal ten list przed Jeremijaszem prorokiem.
\par 30 I stalo sie slowo Pansie do Jeremijasza mówiac:
\par 31 Przetoz tak mówi Pan: Oto Ja nawiedze Semejasza Nechalamitczyka i nasienie jego; nie bedzie mial nikogo, ktoby mieszkal w posrodku ludu tego, ani oglada tego dobra, które Ja uczynie ludowi swemu, mówi Pan; bo radzil, aby odstapil lud od Pana.

\chapter{30}

\par 1 Slowo, które sie stalo do Jeremijasza od Pana, mówiac:
\par 2 Tak powiada Pan, Bóg Izraelski, mówiac: Napisz sobie wszystkie slowa, którem mówil do ciebie, na ksiegach.
\par 3 Albowiem oto dni ida, mówi Pan, a przywróce wiezniów ludu swego Izraelskiego i Judzkiego, mówi Pan, i przyprowadze ich do ziemi, któram byl dal ojcom ich, i posieda ja.
\par 4 A tec sa slowa, które mówil Pan o Izraelu i o Judzie;
\par 5 Tak zaiste mówi Pan: Slyszelismy glos strachu i lekania, i ze niemasz pokoju.
\par 6 Pytajciez sie teraz, a obaczcie, jezli rodzi mezczyzna; przeczze tedy widze, ze kazdy maz rekami swemi trzyma sie za biodra swe jako rodzaca, i ze sie obrócily wszystkich oblicza w bladosc?
\par 7 Biada! bo wielki jest ten dzien, tak, ze mu nie bylo podobnego; ale jakizkolwiek jest czas utrapienia Jakóbowego, przecie z niego wybawiony bedzie.
\par 8 Stanie sie bowiem dnia onego, mówi Pan zastepów, iz skrusze jarzmo jego z szyi twojej, a zwiazki twoje potargam, i nie beda go wiecej cudzoziemcy w niewole podbijac;
\par 9 Ale sluzyc beda Panu, Bogu swemu, i Dawidowi, królowi swemu, którego im wzbudze.
\par 10 Przetoz nie bój sie ty, slugo mój Jakóbie! mówi Pan, ani sie strachaj o Izraelu! bo oto Ja cie wybawie z daleka, i nasienie twoje z ziemi pojmania ich. I wróci sie Jakób, aby odpoczywal i pokój mial, a nie bedzie, ktoby go ustraszyl;
\par 11 Bom Ja z toba, mówi Pan, abym cie wybawil. Poniewaz uczynie koniec wszystkim narodom, miedzy które cie rozprosze, wszakze tobie nie uczynie konca, ale cie miernie karac bede, a cale cie bez karania nie zaniecham.
\par 12 Tak zaiste mówi Pan: Ciezkie bardzo bedzie skruszenie twoje, nader bolesna rana twoja.
\par 13 Nie bedzie, ktoby sadzil sprawe twoje ku uleczeniu; lekarstwa ku uleczeniu miec nie bedziesz.
\par 14 Wszyscy milosnicy twoi zapomna cie, ani cie nawiedza, gdy cie zranie rana nieprzyjacielska, i okrutnem karaniem, dla wielkosci nieprawosci twojej i niezliczonych grzechów twoich.
\par 15 Przeczze wolasz nad skruszeniem swem i ciezka bolescia swoja? Dla wielkosci nieprawosci twojej, i dla niezliczonych grzechów twoich uczynilem ci to.
\par 16 A wszakze wszyscy, którzy cie pozeraja, pozarci beda; a wszyscy, którzy cie ciemieza, wszyscy, mówie, w niewole pójda; a którzy cie plundruja, splundrowani beda; a wszystkich, którzy cie lupia, podam na lup.
\par 17 Tedyc zdrowie przywróce, i od ran twoich ulecze cie, mówi Pan, przeto, ze wygnana nazwali cie (mówiac:) Tac jest Syon, niemasz nikogo, ktoby ja nawiedzil.
\par 18 Tak mówi Pan: Oto Ja przywróce pojmanych z namiotów Jakóbowych, a nad przybytkami jego zmiluje sie; i bedzie zas zbudowane miasto na pagórku swoim, a palac wedlug porzadku swego wystawiony bedzie.
\par 19 I wynijdzie od nich dziekczynienie, i glos weselacych sie; bo ich rozmnoze, a nie ubedzie ich, i uwielbie ich, a nie beda ponizeni.
\par 20 I beda synowie jego, jako i przedtem, a zgromadzenie jego przedemna utwierdzone bedzie; ale nawiedze wszystkich, którzy ich trapia.
\par 21 I powstanie z niego najzacniejszy jego, a panujacy nad nim z posrodku jego wynijdzie, któremu sie rozkaze przyblizyc, aby przystapil przed mie; bo któz jest ten, coby reczyl za sie, ze przystapi przed mie? mówi Pan.
\par 22 I bedziecie ludem moim, a Ja bede Bogiem waszym.
\par 23 Oto wicher Panski z popedliwoscia wynijdzie, wicher trwajacy nad glowa niezbozników zostanie.
\par 24 Nie odwróci sie zapalczywosc gniewu Panskiego, póki nie uczyni tego, a póki nie wykona umyslu serca swego; w ostatnie dni zrozumiecie to.

\chapter{31}

\par 1 Czasu onego, mówi Pan, bede Bogiem wszystkim rodzajom Izraelskim, a oni beda ludem moim.
\par 2 Tak mówi Pan: Znalazl laske na puszczy lud, który uszedl miecza, gdym chodzil przed nim, abym odpocznienie uczynil Izraelowi.
\par 3 Rzeczeszli: Zdawna mi sie Pan ukazywal. I owszem, miloscia wieczna umilowalem cie, dlategoc ustawicznie milosierdzie pokazuje;
\par 4 Jeszcze cie zbuduje, a zbudowana bedziesz, panno Izraelska! jeszcze sie weselic bedziesz przy bebnach swoich, a wychodzic z hufem plasajacych;
\par 5 Jeszcze bedziesz sadzila winnice na górach Samaryjskich; a szczepiacy szczepic beda i jesc beda.
\par 6 Bo nastanie dzien, którego strózowie wolac beda na górze Efraimowej: Wstancie a wstapmy na Syon do Pana, Boga swego.
\par 7 Tak zaiste mówi Pan: Spiewajcie Jakóbowi o rzeczach wesolych, a wykrzykajcie jawnie przed tymi narodami; wydawajcie glos, chwale oddawajcie a mówcie: Panie! wybaw ostatek ludu twego Izraelskiego.
\par 8 Oto Ja przywiode ich z ziemi pólnocnej, a zbiore ich ze wszystkich stron ziemi; z nimi pospolu slepego i chromego, brzemienna i rodzaca: tu sie gromada wielka nawróca.
\par 9 Przywiode ich zasie z placzem i z modlitwami idacych, a powiode ich podle potoków wód droga prosta, na którejby sie nie potkneli; bom sie stal Izraelowi ojcem, a Efraim jest pierworodnym moim.
\par 10 Sluchajcie slowa Panskiego, o narody! a opowiadajcie je na wyspach dalekich, i mówcie: Ten, który rozproszyl Izraela, zgromadzi go, a strzedz go bedzie, jako pasterz trzody swojej.
\par 11 Bo wykupil Pan Jakóba; przeto wybawi go z reki tego, który jest mocniejszy naden.
\par 12 I przyjda a spiewac beda na wysokosci Syonu, i zbieza sie do dobrotliwosci Panskiej ze zbozem i z winem, i z oliwa i z jagnietami, i z cieletami, a dusza ich podobna bedzie ogrodowi wilgotnemu, a nie bedzie sie wiecej smucila.
\par 13 Tedy sie weselic bedzie panna z plasaniem, takze mlodziency i starcy spolem; albowiem kwilenie ich obróce w radosc, a pociesze ich, i rozwesele ich po smutku ich;
\par 14 I opoje dusze kaplanów tlustoscia, a lud mój dobrocia moja nasyci sie, mówi Pan.
\par 15 Tak mówi Pan: Glos w Rama slyszany jest, narzekanie i placz bardzo gorzki; Rachel placzaca synów swoich nie dala sie pocieszyc po synach swoich, przeto, ze ich niemasz.
\par 16 Tak mówi Pan: Zawsciagnij glos swój od placzu, a oczy swe od lez; bo bedziesz miala zaplate za prace swoje, mówi Pan, ze sie nawróca z ziemi nieprzyjacielskiej.
\par 17 Jest mówie nadzieja, ze sie potem nawróca, mówi Pan, synowie twoi do krainy swojej.
\par 18 Wprawdzie slysze Efraima, ze sobie utyskuje, mówiac: Pokarales mie, abym byl pokarany jako cielec nieokrócony. Nawróc mie abym byl nawrócony; tys zaiste Panie! Bóg mój.
\par 19 Bo po nawróceniu mojem pokutowac beda; a gdy samego siebie poznam, uderze sie w biodro; wstydze sie, owszem i zapalam sie, ze odnosze hanbe dziecinstwa swego.
\par 20 Izali synem moim milym nie jest Efraim? Izali nie jest dziecieciem rozkosznem? Bo od onego czasu, jakom mówil przeciwko niemu, przeciez nan ustawicznie wspominam; dlatego poruszaja sie wnetrznosci moje nad nim, zaiste zlituje sie nad nim, mówi Pan.
\par 21 Wystaw sobie pamiatki; nakladz sobie gromad kamieni; pamietaj na ten gosciniec, i na droge, któras chodzila: nawróc sie, panno Izraelska! nawróc sie do tych miast swoich.
\par 22 Dokadze sie tulac bedziesz, córko odporna? Bo uczyni Pan rzecz nowa na ziemi: niewiasta ogarnie meza.
\par 23 Tak mówi Pan zastepów, Bóg Izraelski: Jeszcze mówic bede slowo to w ziemi Judzkiej, i w miastach jej, gdy przywiode wiezniów ich: Niech cie blogoslawi Pan, o mieszkanie sprawiedliwosci! o góro swietobliwosci!
\par 24 Albowiem osadzac sie beda w ziemi Judzkiej we wszystkich miastach jego spolem oracze, i ci, którzy chodza za stadem.
\par 25 Napoje zaiste dusze spracowana, a wszelka dusze smutna nasyce.
\par 26 Wtemem ocucil i spojrzalem, a sen mój byl mi wdzieczny.
\par 27 Oto dni ida, mówi Pan, w których posieje dom Izraelski i dom Judzki nasieniem czlowieczem i nasieniem bydlecem;
\par 28 A jakom sie staral, abym ich wykorzenil, i burzyl, i kazil, i gubil, i trapil: tak sie starac bede, abym ich pobudowal i rozsadzil, mówi Pan.
\par 29 Za onych dni nie beda wiecej mówic: Ojcowie jedli grona cierpkie, a synów zeby scierpnely;
\par 30 Owszem, raczej rzeka: kazdy dla nieprawosci swojej umrze; kazdego czlowieka, któryby jadl grona cierpkie, scierpna zeby jego.
\par 31 Oto dni ida, mówi Pan, których uczynie z domem Izraelskim i z domem Judzkim przymierze nowe;
\par 32 Nie takie przymierze, jakiem uczynil z ojcami ich w on dzien, któregom ich ujal za reke ich, abym ich przywiódl z ziemi Egipskiej; albowiem oni przymierze moje wzruszyli, chociazem Ja byl malzonkiem ich, mówi Pan.
\par 33 Ale to jest przymierze, które postanowie z domem Izraelskim po tych dniach, mówi Pan: Dam zakon mój do wnetrznosci ich, a na sercu ich napisze go, i bede Bogiem ich, a oni beda ludem moim.
\par 34 I nie bedzie wiecej uczyl zaden blizniego swego, i zaden brata swego, mówiac: Poznajcie Pana; bo mie oni wszyscy poznaja, od najmniejszego z nich az do najwiekszego z nich, mówi Pan; bo milosciw bede nieprawosciom ich, a grzechów ich nie wspomne wiecej.
\par 35 Tak mówi Pan, który daje slonce na swiatlosc we dnie, postanowienie miesiaca i gwiazd na swiatlosc w nocy; który rozdziela morze, a hucza nawalnosci jego; Pan zastepów imie jego.
\par 36 Jezli odstapia te ustawy od oblicza mego, mówi Pan, tedyc i nasienie Izraelskie przestanie byc narodem przed obliczem mojem po wszystkie dni.
\par 37 Tak mówi Pan: Jezli moga byc rozmierzone niebiosa z góry, a doscignione grunty ziemi na dole, tedyc i Ja cale odrzuce nasienie Izraelskie dla tego wszystkiego, co uczynili, mówi Pan.
\par 38 Oto ida dni (mówi Pan,)których bedzie zbudowane to miasto Panu od wiezy Chananeel az do bramy naroznej;
\par 39 A pójdzie jeszcze sznur pomiaru na przeciwko niej ku pagórkowi Gareb, a uda sie ku Goa.
\par 40 I wszystka dolina trupów i popiolu, i to wszystko pole az do potoku Cedron, az do wegla bramy konskiej wschodniej, poswiecone beda Panu; nie bedzie wykorzenione ani zepsute wiecej na wieki.

\chapter{32}

\par 1 Slowo, które sie stalo do Jeremijasza od Pana roku dziesiatego Sedekijasza, króla Judzkiego, który jest rok osmnasty Nabuchodonozora.
\par 2 (A wtenczas wojsko króla Babilonskiego obleglo bylo Jeruzalem, a Jeremijasz prorok zamkniety byl w sieni ciemnicy, która byla w domu króla Judzkiego.
\par 3 Bo go byl dal wsadzic Sedekijasz, król Judzki, mówiac: Czemu ty prorokujesz, mówiac: Tak mówi Pan: Oto Ja to miasto podam w reke króla Babiolonskiego, i wezmie je?
\par 4 Sedekijasz takze król Judzki nie ujdzie reki Chaldejczyków; ale zapewne wydany bedzie w reke króla Babilonskiego, i beda mówily usta jego z usty jego, a oczy jego oczy jego ogladaja.
\par 5 I zawiedzie Sedekijasza do Babilonu, aby tam byl, az go nawiedze, mówi Pan; poniewaz walczycie z Chaldejczykami, nie poszczesci sie wam.)
\par 6 Tedy rzekl Jeremijasz: Stalo sie slowo Panskie do mnie, mówiac:
\par 7 Oto Chanameel, syn Salluma, stryja twego, przyjdzie do ciebie, mówiac: Kup sobie role moje, która jest w Anatot; bo tobie nalezy prawem bliskosci, abys ja kupil.
\par 8 A gdy przyszedl do mnie Chanameel, syn stryja mego, wedlug slowa Panskiego do sieni ciemnicy, i rzekl do mnie: Prosze, kup role moje, która jest w Anatot, które jest w ziemi Benjaminowej, bo tobie nalezy prawem dziedzicznem, i bliskosciac przypada, kupze ja sobie: tedy zrozumiawszy, ze to bylo slowo Panskie,
\par 9 Kupilem od Chanameela, syna stryja swego one role, która jest w Anatot, i odwazylem mu pieniedzy, siedmnascie syklów srebra;
\par 10 A uczyniwszy zapis zapieczetowalem i oswiadczylem swiadkami, odwazywszy pieniadze na wadze.
\par 11 Potemem wzial wedlug przykazania i prawa zapis kupna zapieczetowany i otwarty;
\par 12 I oddalem on zapis kupna Baruchowi, synowi Neryjasza, syna Maasejaszowego, przed oczyma Chanameela, stryja swego, i przed oczyma swiadków, którzy sie podpisali w onym zapisie kupna, przed oczyma wszystkich Zydów, którzy byli usiedli w sieni ciemnicy;
\par 13 I rozkazalem Baruchowi przed oczyma ich, mówiac:
\par 14 Tak mówi Pan zastepów, Bóg Izraelski: Wezmij te zapisy, ten zapis tego kupna, jako zapieczetowany, tak i ten zapis otworzony, a wlóz je w naczynie gliniane, aby trwaly przez wiele lat;
\par 15 Bo tak mówi Pan zastepów, Bóg Izraelski: Jeszcze beda kupowane domy, i role, i winnice w tej ziemi.
\par 16 Potem modlilem sie Panu, kiedym oddal on zapis kupna Baruchowi, synowi Neryjaszowemu, mówiac:
\par 17 Ach panujacy Panie!otos ty uczynil niebo i ziemie moca swoja wielka i ramianiem twoim wyciagnionem, nie jestci skryta przed toba zadna rzecz;
\par 18 Czynisz milosierdzie nad tysiacami, i oddajesz nieprawosc ojcowska do lona synów ich po nich; Bóg wielki mocny, Pan zastepów imie twoje;
\par 19 Wielki w radzie i mozny w sprawie, poniewaz oczy twoje otworzone sa na wszystkie drogi synów ludzkich, abys oddal kazdemu wedlug dróg jego, i wedlug owoców spraw jego;
\par 20 Którys uczynil znaki i cuda na ziemi Egipskiej az do dnia tego, i w Izraelu, i miedzy innymi ludzmi, i uczyniles sobie imie, jako sie to dzis okazuje.
\par 21 Bos wywiódl lud twój Izraeski z ziemi Egipskiej w znakach i w cudach, i w rece mocnej, i w ramieniu wyciagnionem i w strachu wielkim;
\par 22 A podales im te ziemie, o któras przysiagl ojcom ich, zes im mial dac ziemie oplywajaca mlekiem i miodem.
\par 23 Ale ze wszedlszy do niej, a posiadlszy ja, nie usluchali glosu twojego, i w zakonie twoim nie chodzili, wszystkiego, cos im rozkazal czynic, nie czynili; przetoz sprawiles to, aby nan przyszlo to wszystko zle.
\par 24 Oto strzelbe zatoczono przeciwko miastu, aby je wzieto, a miasto podane jest w rece Chaldejczyków walczacych przeciw niemu przez miecz, i glód, o mór; a tak coskolwiek rzekl, dzieje sie, to sam widzisz.
\par 25 A ty przecie mówisz do mnie, panujacy Panie: Kup sobie te role za pieniadze, a oswiadcz to swiadkami, choc juz to miasto podane jest w rece Chaldejczyków.
\par 26 I stalo sie slowo Panskie do Jeremijasza, mówiac:
\par 27 Otom Ja Pan, Bóg wszelkiego ciala; izaliz przedemna moze byc skryta która rzecz?
\par 28 Przetoz tak mówi Pan: Oto Ja daje to miasto w reke Chaldejczyków, i w reke Nabuchodonozora, króla Babilonskiego, i wezmie je.
\par 29 A wszedlszy Chaldejczycy, którzy walcza przeciwko temu miastu, zapala to miasto ogniem, i spala je i te domy, na których dachach kadzili Baalowi, a sprawowali ofiary mokre bogom cudzym, aby mie wzruszali do gniewu.
\par 30 Bo synowie Izraelscy i synowie Judzcy od dziecinstwa swego to tylko czynia, co jest zlego przed oczyma mojemi; synowie, mówie, Izraelscy tylko mie draznili sprawa rak swoich, mówi Pan.
\par 31 Zaiste na zapalczywosc moje, i na gniew mój robi sobie to miasto ode dnia, którego je zbudowali, az do dnia tego, tak, ze mi przyjdzie oddalic od oblicza mego;
\par 32 A to dla wszelkiej zlosci synów Izraelskich i synów Judzkich, która popelniali, pobudzajac mie do gniewu, sami, królowie ich, ksiazeta ich, kaplani ich i prorocy ich, jako mezowie Judzcy, tak obywatele Jeruzalemscy.
\par 33 Obracajac sie do mnie tylem a nie twarza; a gdy ich nauczam rano wstawajac i nauczajac, wszakze nie sluchaja, aby przyjeli nauke.
\par 34 Nadto nastawiali obrzydliwosci swych w tym domu, który nazwany jest od imienia mego, aby go splugawili.
\par 35 Nabudowali, mówie, wyzyn Baalowi, które sa w dolinie Ben Hennon, aby przenaszali przez ognien synów swoich i córki swoje Molochowi, chociazem im tego nie rozkazal, ani to wstapilo na serce moje, aby kiedy czynic mieli te obrzydliwosc, a do grzech u Juda przywodzic.
\par 36 A teraz dlatego tak mówi Pan, Bóg Izraelski, o tem miescie, o którem wy powiadacie: Podane jest w reke króla Babilonskiego przez miecz, i glód, i mór:
\par 37 Oto Ja zgromadze ich ze wszystkich ziem, do którychem ich wygnal w popedliwosci mojej i w gniewie moim, i w zapalczywosci wielkiej, i przywiode ich zas na to miejsce, i uczynie, aby bezpiecznie mieszkali;
\par 38 I beda ludem moim, a Ja bede Bogiem ich.
\par 39 I dam im serce jedno, i droge jedne, aby sie mnie bali po wszystkie dni, tak, aby sie im dobrze dzialo, i synom ich po nich;
\par 40 I uczynie z nimi przymierze wieczne, ze sie nie odwróce od nich, abym im nie mial dobrze czynic; nadto bojazn moje dam do serca ich, aby nie odstepowali odemnie.
\par 41 I bede sie weselil z nich, dobrze im czyniac, gdyz ich wszczepie w tej ziemi warownie, ze wszystkiego serca mego i ze wszystkiej duszy mojej.
\par 42 Bo tak mówi Pan: Jakom przywiódl na ten lud to wszystko wielkie zle, tak przywiode na nich to wszystko dobre, o któremem z nimi mówil.
\par 43 Tedy beda kupowac role w tej ziemi, o której wy powiadacie: Spustoszona jest tak, ze w niej niemasz ani czlowieka ani bydlecia, podana jest w reke Chaldejczyków.
\par 44 Role za pieniadze kupowac beda, i zapisem warowac, i pieczetowac, i swiadkami oswiadczac w ziemi Benjaminowej, i okolo Jeruzalemu, i w miastach Judzkich, jako w miastach na górach tak w miastach na równinach, i w miastach na poludnie, gdy zas przywróce pojmanych ich, mówi Pan.

\chapter{33}

\par 1 I stalo sie slowo Panskie do Jeremijasza po wtóre, gdy jeszcze byl zamkniety w sieni ciemnicy, mówiac:
\par 2 Tak mówi Pan, który to uczyni: Pan, który to utworzy, potwierdzi to, Pan jest imie jego.
\par 3 Wolaj do mnie, a ozwec sie i oznajmiec rzeczy wielkie i skryte, o których nie wiesz.
\par 4 Albowiem tak mówi Pan, Bóg Izraelski, o domach miasta tego, i o domach królów Judzkich, które pokazone byc maja taranami wojennemi i mieczem:
\par 5 Pociagna, zeby walczyli z Chaldejczykami, a zeby napelnili te domy trupami ludzi, które pobije w zapalczywosci mojej i w gniewie moim, zakrywajac twarz moje od tego miasta dla wszelakich zlosci ich.
\par 6 Wszakze Ja przywiode ich do zdrowia, i ulecze a uzdrowie ich, i objawie im obfitosc pokoju, a pokoju pewnego.
\par 7 Bo przywróce pojmanych z Judy, i pojmanych z Izraela, a pobuduje ich jako przedtem;
\par 8 I oczyszcze ich od wszelkiej nieprawosci ich, która zgrzeszyli przeciwko mnie, i przepuszcze wszystkim zlosciom ich, któremi zgrzeszyli przeciwko mnie, i któremi wystapili przeciwko mnie.
\par 9 A bedzie mi to slawa, weselem, chwala, i ozdoba przed wszystkiemi narodami ziemi, które uslysza o wszystkiem dobrem, które Ja im czynie, i ulekna sie, a zatrwoza sie nad wszystkiem dobrem i nad wszystkim pokojem, który Ja im sposobie.
\par 10 Tak mówi Pan: Jeszcze slyszany bedzie na tem miejscu, (o którem wy powiadacie: Jest spustoszone, tak, ze niemasz ani czlowieka, ani bydlecia, w miastach Judzkich i na ulicach Jeruzalemskich spustoszonych, tak, ze niemasz ani czlowieka, ani obywa tela, ani bydlecia.)
\par 11 Slyszany, mówie, bedzie glos radosci, i glos wesela, glos oblubienca, i glos oblubienicy, glos mówiacych: Wyslawiajcie Pana zastepów; albowiem dobry jest Pan, albowiem na wieki milosierdzie jego; i glos przynoszacy ofiare chwaly do domu Panskiego, gdyz przywróce pojmanych z tej ziemi, jako na poczatku, mówi Pan.
\par 12 Tak mówi Pan zastepów: Jeszcze bedzie na tem miejscu pustem, na którem niemasz ani czlowieka, ani bydlecia, i we wszystkich miastach jego mieszkanie pasterzy, gdzieby chowali trzody;
\par 13 W miastach na górach, w miastach na równinach, i w miastach na poludnie, i w ziemi Benjaminowej, i okolo Jeruzalemu, i w miastach Judzkich jeszcze przychodzic beda trzody pod reka liczacego, mówi Pan.
\par 14 Oto dni ida, mówi Pan, w których utwierdze to slowo dobre, którem byl wyrzekl o domu Izraelskim i o domu Judzkim.
\par 15 W one dni w onym czasie uczynie to, iz wyrosnie Dawidowi latorosl sprawiedliwa, która czynic bedzie sad i sprawiedliwosc na ziemi.
\par 16 Onych dni bedzie zbawiony Juda, a Jeruzalem bezpiecznie mieszkac bedzie. A toc jest imie, którem ja nazowia: Pan sprawiedliwosc nasza.
\par 17 Bo tak mówi Pan: Nie bedzie wykorzeniony maz z rodu Dawidowego, aby nie mial siedziec na stolicy domu Izraelskiego.
\par 18 Z kaplanów tez i z Lewitów nie bedzie wykorzeniony maz od oblicza mego, aby nie mial ofiarowac calopalenia, i zapalac sniednej ofiary, i sprawowac ofiar po wszystkie dni.
\par 19 Potem stalo sie slowo Panskie do Jeremijasza, mówiac:
\par 20 Tak mówi Pan: Jezli bedziecie mogli zlamac przymierze moje ze dniem, i przymierze moje z noca, aby nie bywalo dnia ani nocy czasu swego,
\par 21 Tedy tez przymierze moje zlamane bedzie z Dawidem, sluga moim, aby nie mial syna, któryby królowal na stolicy jego, i z Lewitami kaplanami, aby nie byli slugami moimi.
\par 22 A jako nie moze policzone byc wojsko niebieskie, ani zmierzony piasek morski, tak rozmnoze nasienie Dawida, slugi mojego, i Lewitów, którzy mi sluza.
\par 23 Znowu sie stalo slowo Panskie do Jeremijasza, mówiac:
\par 24 Aza nie widzisz, co ten lud powiada, mówiac: Ze dwa domy, które byl Pan obral, te juz odrzucil, a ze ludem moim pogardzaja, jakoby nie byl wiecej narodem przed obliczem ich?
\par 25 Tak mówi Pan: Nie bedzieli przymierze moje ze dniem i z noca stale, a jezlim poczatku niebios i ziemi nie postanowil:
\par 26 Tedyc i nasienie Jakóbowe i Dawida slugi mego odrzuce, abym nie bral z nasienia jego tych, którzyby panowac mieli nad nasieniem Abrahamowem, Izaakowem, i Jakóbowem, gdyz przywróce wiezniów ich, a zlituje sie nad nimi.

\chapter{34}

\par 1 Slowo, które sie stalo do Jeremijasza od Pana, (gdy Nabuchodonozor, król Babilonski, i wszystko wojsko jego, i wszystkie królestwa ziemi, które byly pod wladza reki jego, i wszystkie narody walczyly przeciwko Jeruzalemowi i przeciwko wszystkim miastom jego,)mówiac:
\par 2 Tak mówi Pan, Bóg Izraelski: Idz, a mów Sedekijaszowi, królowi Judzkiemu, i powiedz mu: Tak mówi Pan: Oto Ja to miasto podam w reke króla Babilonskiego, aby je ogniem spalil;
\par 3 I ty nie ujdziesz reki jego, ale zapewne bedziesz pojmany, i w rece jego podany, a oczy twoje ogladaja oczy króla Babilonskiego, i usta jego z usty twemi mówic beda, a do Babilonu wnijdziesz.
\par 4 A wszakze sluchaj slowa Panskiego, Sedekijaszu, królu Judzki! Tak mówi Pan o tobie: Nie umrzesz od miecza:
\par 5 W pokoju umrzesz; a jako wonne rzeczy palono ojcom twoim, królom przeszlym, którzy byli przed toba, tak palic beda i tobie, a plakac cie beda, mówiac: Ach, panie! Bom Ja to slowo rzekl, mówi Pan.
\par 6 Tedy mówil Jeremijasz prorok do Sedekijasza, króla Judzkiego, wszystkie te slowa w Jeruzalemie.
\par 7 Gdy wojsko króla Babilonskiego walczylo przeciwko Jeruzalemowi, i przeciwko wszystkim miastom Judzkim pozostalym, przeciwko Lachys, i przeciw Asekowi; albowiem te byly pozostaly z miast Judzkich miasta obronne.
\par 8 Slowo, które sie stalo do Jeremijasza od Pana, gdy uczynil król Sedekijasz przymierze ze wszystkim ludem, co byl w Jeruzalemie, wolnosc im oglaszajac;
\par 9 To jest, aby kazdy wolno puscil sluge swego, i kazdy sluzebnice swoje, Zyda i Zydówke, aby sobie nikt nie zniewalal Zyda, brata swego.
\par 10 A tak usluchaly wszyscy ksiazeta, i wszystek lud, którzy byli weszli w przymierze, zeby kazdy wolno puscil sluge swego, i kazdy sluzebnice swoje, aby ich wiecej nie zniewalali; usluchali, mówie, i puscili ich wolno.
\par 11 Lecz potem rozmysliwszy sie, pobrali zas slugi i sluzebnice, które byli puscili wolno, a zniewolili ich sobie za slugi i za sluzebnice.
\par 12 I stalo sie slowo Panskie do Jeremijasza od Pana, mówiac:
\par 13 Tak mówi Pan, Bóg Izraelski: Jam postanowil przymierze z ojcami waszymi w dzien, któregom ich wywiódl z ziemi Egipskiej, z domu niewoli, mówiac:
\par 14 Gdy sie skoncza siedm lat, niech wolno pusci kazdy brata swego Zyda, którycby byl sprzedany, a sluzylciby przez szesc lat; wolno, mówie, pusci go od siebie. Ale mie nie sluchali ojcowie wasi, ani naklonili ucha swego.
\par 15 Wysciec sie zaiste dzis nawrócili, i uczyniliscie to, co jest dobrego przed oczyma mojemi, zescie oglosili wolnosc kazdy blizniemu swemu, i uczyniliscie przymierze przed twarza moja w domu tym, który jest nazwany od imienia mego.
\par 16 Alescie sie zas cofneli, i splugawiliscie imie moje, zescie zas wzieli kazdy sluge swego, i kazdy sluzebnice swoje, którescie byli wolno puscili wedlug zadnosci ich, i zniewoliliscie ich, aby byli slugami i sluzebnicami waszymi.
\par 17 Dlategoz tak mówi Pan: Wyscie mie nie sluchali, abyscie oglosili wolnosc kazdy bratu swemu, i kazdy blizniemu swemu; otoz Ja przeciwko wam oglaszam wolnosc, mówi Pan, miecza, moru, i glodu, a podam was na potlukanie po wszystkich królestwach ziemi.
\par 18 Podam zaiste tych ludzi, którzy przestapili przymierze moje, którzy nie dotrzymali slów przymierza tego, które uczynili przed twarza moja, gdy cielca na dwoje rozcieli, i przeszli miedzy czesciami jego;
\par 19 To jest ksiazat Judzkich, i ksiazat Jeruzalemskich, komorników i kaplanów, i wszystek lud tej ziemi, którzy przeszli miedzy czesciami tego cielca,
\par 20 Podam ich mówie, w reke nieprzyjaciól ich, i w reke szukajacych duszy ich, i beda trupy ich zerem ptastwu niebieskiemu i bestyjom ziemskim.
\par 21 Sedekijasza tez, króla Judzkiego, i ksiazat jego podam w reke nieprzyjaciól ich, i w reke szukajacych duszy ich, w reke, mówie, wojska króla Babilonskiego, które odstapilo od was.
\par 22 Oto Ja rozkaze, mówi Pan, i przywiode ich zas na to miasto, aby walczyli przeciwko niemu, a wziawszy je spalili je ogniem; miasta tez Judzkie obróce w pustynie, tak iz beda bez obywatela.

\chapter{35}

\par 1 Slowo, które sie stalo do Jeremijasza od Pana za dni Joakima, syna Jozyjaszowego, króla Judzkiego, mówiac:
\par 2 Idz do domu Rechabitów, a mów z nimi, i wprowadz ich do domu Panskiego, do jednej komory, a daj im pic wino.
\par 3 Wzialem tedy z soba Jasanijasza, syna Jeremijaszowego, syna Chabazymijaszowego, i braci jego, i wszystkich synów jego, i wszystek dom Rechabitów,
\par 4 I wprowadzilem ich do domu Panskiego, do komory synów Chanana, syna Jegdalijaszowego, meza Bozego, która byla podle komory ksiazecej, która byla nad komora Maasejasza, syna Sallumowego, strzegacego progu.
\par 5 Potem postawilem przed synami domu Rechabitów czasze pelna wina i kubki, i mówilem do nich: Pijcie wino.
\par 6 Którzy odpowiedzieli: Nie pijamy wina; bo Jonadab, syn Rechabowy, ojciec nasz, zakazal nam, mówiac: Nie pijajcie wina, wy i synowie wasi az na wieki;
\par 7 A domu nie budujcie, i nasienia nie rozsiewajcie, i winnicy nie sadzcie, ani miewajcie; ale w namiotach mieszkajcie po wszystkie dni wasze, abyscie zyli przez wiele dni na obliczu ziemi, w której jestescie przechodniami.
\par 8 Przetoz usluchalismy glosu Jonadaba, syna Rechabowego, ojca naszego, we wszystkiem, co nam rozkazal, zebysmy nie pili wina po wszystkie dni nasze, my, zony nasze, synowie nasi, i córki nasze;
\par 9 I zebysmy nie budowali domów ku mieszkaniu naszemu, a winnicy, i roli, i zadnego siewu nie mieli,
\par 10 Ale abysmy mieszkali w namiotach; i usluchalismy, i uczynimy wedlug wszystkiego, co nam rozkazal Jonadab, ojciec nasz.
\par 11 A gdy przyciagnal, Nabuchodonozor król Babilonski, do ziemi naszej, rzeklismy: Pójdzcie, a ustapmy do Jeruzalemu przed wojskiem Chaldejskiem, i przed wojskiem Syryjskiem; a takesmy zostali w Jeruzalemie.
\par 12 I stalo sie slowo Panskie do Jeremijasza, mówiac:
\par 13 Tak mówi Pan zastepów, Bóg Izraelski: Idz, a rzecz mezom Judzkim i obywatelom Jeruzalemskim: I nie przyjmieciez cwiczenia, abyscie posluszni byli slowom moim? mówi Pan.
\par 14 Wazne sa slowa Jonadaba, syna Rechabowego, które przykazal synom swoim, zeby nie pili wina; i nie pija go az do dnia tego, bo posluszni sa przykazaniu ojca swego; ale Ja mówie do was, rano wstawajac i mówiac, a przecie jestescie nieposluszni.
\par 15 Posylam tez do was wszystkich slug swych, proroków w rano wstawajac i posylajac, aby mówili: Nawróccie sie juz kazdy od zlej drogi swej, a polepszajcie spraw swoich, a nie nasladujcie bogów cudzych, ani im sluzcie; a tak mieszkajcie w tej ziemi, któram dal wam, i ojcom waszym; alescie nie naklonili ucha swego, aniscie mie usluchali;
\par 16 Choc synowie Jonadaba, syna Rechabowego, dosyc uczynili rozkazaniu ojca swego, które im przykazal, ale ten lud nie jest mi posluszny.
\par 17 Przetoz tak mówi Pan, Bóg zastepów, Bóg Izraelski: Oto Ja przywiode na Jude i na wszystkich obywateli Jeruzalemskich wszystko zle, którem wyrzekl przeciwko im, przeto, zem mówil do nich, a nie sluchali, i wolalem ich, a nie ozwali mi sie.
\par 18 A domowi Rechabitów rzekl Jeremijasz: Tak mówi Pan zastepów, Bóg Izraelski: Dlatego, zescie posluszni rozkazaniu Jonadaba, ojca waszego, i strzezecie wszystkich przykazan jego, a czynicie wedlug wszystkiego, co wam rozkazal:
\par 19 Przetoz tak mówi Pan zastepów, Bóg Izraelski: Nie bedzie wygladzony maz z rodu Jonadaba, syna Rechabowego, któryby stal przed obliczem mojem po wszystkie dni.

\chapter{36}

\par 1 I stalo sie roku czwartego Joakima, syna Jozyjaszowego, króla Judzkiego; stalo sie, mówie, to slowo do Jeremijasza od Pana, mówiac:
\par 2 Wezmij sobie ksiegi, a napisz na nich wszystkie slowa, którem mówil do ciebie przeciw Izraelowi, i przeciw Judzie, i przeciw wszystkim narodom, ode dnia, króregom mawial z toba, ode dni Jozyjaszowych az do dnia tego;
\par 3 Aza snac, gdy uslyszy dom Judzki o tem wszystkiem zlem, które Ja im mysle uczynic, nawróci sie kazdy od zlej drogi swej, abym byl milosciw nieprawosciom ich i grzechom ich.
\par 4 Przetoz wezwal Jeremijasz Barucha, syna Neryjaszowego; i napisal Baruch w ksiegi z ust Jeremijaszowych wszystkie slowa Panskie, które mówil do niego.
\par 5 Potem przykazal Jeremijasz Baruchowi, mówiac: Ja bedac zatrzymany nie moge wnijsc do domu Panskiego;
\par 6 Przetoz ty idz, a czytaj na tych ksiegach, cos napisal z ust moich, slowa Panskie, przed uszyma ludu w domu Panskim w dzien postu; takze tez przed uszyma wszystkich z Judy którzyby przyszli z miast swoich, czytaj je.
\par 7 Owa snac przyjdzie modlitwa ich przed oblicze Panskie, a nawróci sie kazdy od zlej drogi swojej; bo wielka jest zapalczywosc i gniew, w którym mówil Pan przeciwko temu ludowi.
\par 8 Tedy uczynil Baruch, syn Neryjaszowy, wedlug wszystkiego, co mu rozkazal Jeremijasz prorok, czytajac z ksiag slowa Panskie w domu Panskim.
\par 9 I stalo sie roku piatego za Joakima. syna Jozyjaszowego, króla Judzkiego, w miesiacu dziewiatym, ze zapowiedziano post przed twarza Panska wszystkiemu ludowi w Jeruzalemie, i wszystkiemu ludowi, który sie byl zszedl z miast Judzkich do Jeruzalemu.
\par 10 I czytal Baruch z ksiag slowa Jeremijaszowe w domu Panskim, w pokoju Giemaryjasza, syna Safanowego, pisarza, w sieni wyzszej, w wejsciu bramy nowej domu Panskiego przed uszyma wszystkiego ludu.
\par 11 A gdy uslyszal Micheasz, syn Giemaryjasza, syna Safanowego, wszystkie slowa Panskie z ksiag,
\par 12 Zstapil do domu królewskiego, do komory pisarzowej, a oto tam wszyscy ksiazeta siedzieli: Elisama pisarz, i Delajasz, syn Semajaszowy, i Elnatan, syn Achborowy, i Giemaryjasz, syn Safanowy, i Sedekijasz, syn Chananijaszowy, i wszyscy ksiazeta.
\par 13 I powiedzial im Micheasz wszystkie slowa, które slyszal, gdy czytal Baruch z ksiag przed uszyma ludu.
\par 14 Przetoz poslali wszyscy ksiazeta do Barucha Jehude, syna Natanijaszowego, syna Selemijaszowego, syna Chusowego, aby rzekl: Ksiegi, na któryches czytal przed uszyma ludu, wezmij w reke twa a pójdz. Wzial tedy Baruch, syn Neryjaszowy, ksiegi w reke swa i przyszedl do nich.
\par 15 I rzekli do niego: Siadz prosze, a czytaj to przed uszyma naszemi. I czytal Baruch przed uszyma ich.
\par 16 A gdy uslyszeli wszystkie one slowa, uleklszy sie wejrzal jeden na drugiego, i rzekli do Barucha: Zaiste oznajmiemy królowi te wszystkie slowa.
\par 17 I pytali Barucha, mówiac: Powiedz nam teraz, jakos pisal wszystkie te slowa z ust jego?
\par 18 I rzekl im Baruch: Z ust swych mówil do mnie wszystkie te slowa, a jam pisal na ksiegach inkaustem.
\par 19 Tedy rzekli ksiazeta do Barucha: Idz, a skryj sie, ty i Jeremijasz, a niech nikt nie wie, gdziescie.
\par 20 Potem weszli do króla do sieni, a ksiegi dali schowac do komory Elisama, pisarza, i oznajmili przed królem te wszystkie slowa.
\par 21 Poslal tedy król Juhude, aby wzial one ksiegi; który je wzial z komory Elisama, pisarza. I czytal je Jehuda przed uszyma królewskiemi, i przed uszyma wszystkich ksiazat, którzy stali przed królem.
\par 22 A król siedzial w domu, w którym w zimie bywal, miesiaca dziewiatego, a na ognisku przed nim palil sie ogien.
\par 23 A gdy przeczytal Jehuda trzy albo cztery karty, porzezal je król nozykiem pisarskim, i wrzucil je w ogien, który byl na ognisku, az zgorzaly wszystkie ksiegi w ogniu, który byl na ognisku;
\par 24 Ale sie nie ulekli, ani rozdarli szat swoich, król i wszyscy sludzy jego, którzy slyszeli wszystkie te slowa.
\par 25 Owszem, jeszcze gdy Elnatan, i Delajasz, i Giemaryjasz przyczyniali sie do króla, aby nie palil onych ksiag, tedy ich nie usluchal;
\par 26 Ale rozkazal król Jerameelowi, synowi królewskiemu, i Sarajaszowi, synowi Abdeelowemu, aby pojmali Barucha pisarza, i Jeremijasza proroka; ale ich Pan skryl.
\par 27 I stalo sie slowo Panskie do Jeremijasza, gdy król spalil one ksiegi i slowa, które byl spisal Baruch z ust Jeremijaszowych, mówiac:
\par 28 Wezmij sobie zasie ksiegi inne, a napisz na nich wszystkie slowa pierwsze, które byly w onych ksiegach pierwszych, które spalil Joakim, król Judzki.
\par 29 A o Joakimie, królu Judzkim, rzeczesz: Tak mówi Pan: Tys spalil te ksiegi, mówiac: Czemus pisal na nich, rzeklszy: Zapewne przyciagnie król Babilonski, i spustoszy te ziemie, i wygladzi z niej czlowieka i bydle.
\par 30 Przetoz tak mówi Pan o Joakimie, królu Judzkim: Nie bedzie mial, ktoby siedzial na stolicy Dawidowej, a trup jego wyrzucony bedzie na goracosc we dnie, a na mróz w nocy.
\par 31 Bo nawiedze na nim, i na nasieniu jego, i na slugach jego nieprawosc ich, i przywiode na nich i na obywateli Jeruzalemskich, i na mezów Judzkich to wszystko zle, o któremem mawial do nich; ale nie sluchali.
\par 32 Tedy Jeremijasz wzial ksiegi inne, i dal je Baruchowi, synowi Neryjaszowemu, pisarzowi, który na nich spisal z ust Jeremijaszowych wszystkie slowa onych ksiag, które byl spalil ogniem Joakim, król Judzki; a nadto przydano do onych slów wiele rzec zy tym podobnych.

\chapter{37}

\par 1 Potem królowal król Sedekijasz, syn Jozyjaszowy, miasto Chonijasza, syna Joakimowego, którego Nabuchodonozor, król Babilonski, w ziemi Judzkiej królem postanowil.
\par 2 Lecz nie byl posluszny on, i sludzy jego, i lud onej ziemi slowom Panskim, które mówil przez Jeremijasza proroka.
\par 3 Ale jednak król Sedekijasz poslal Jechuchala, syna Selemijaszowego, i Sofonijasza, syna Maazejaszowego, kaplana, do Jeremijasza proroka, aby mówili: Módl sie prosze za nami Panu, Bogu naszemu.
\par 4 Bo Jeremijasz jeszcze wolno chodzil miedzy ludem, i jeszcze go bylo nie wsadzono do wiezienia.
\par 5 A wojsko Faraonowe wyciagnelo bylo z Egiptu; (a uslyszawszy Chaldejczycy, którzy byli oblegli Jeruzalem, wiesc o tem, odciagneli od Jeruzalemu.)
\par 6 I stalo sie slowo Panskie do Jeremijasza proroka, mówiac:
\par 7 To mówi Pan, Bóg Izraelski: Tak powiedzcie królowi Judzkiemu, który was poslal do mnie, abyscie sie mnie radzili: Oto wojsko Faraonowe, które wam przyciagnelo na pomoc, wróci sie do ziemi swojej do Egiptu.
\par 8 I wróca sie zasie Chaldejczycy, i beda walczyli przeciwko temu miastu, i wezma je, i spala je ogniem.
\par 9 Tak mówi Pan: Nie zwodzcie dusz waszych, mówiac: Zapewne odciagna od nas Chaldejczycy; boc nie odciagna.
\par 10 Owszem, chocbyscie porazili wszystko wojsko Chaldejczyków, którzy walcza z wami, a zostaliby z nich tylko zranieni, ci z namiotów swoich powstana, a to miasto ogniem spala.
\par 11 Gdy tedy odciagnelo wojsko Chaldejczyków od Jeruzalemu przed wojskiem Faraonowem,
\par 12 Wychodzil Jeremijasz z Jeruzalemu, aby szedl do ziemi Benjaminowej, aby tak uszedl z posrodku ludu.
\par 13 A gdy juz byl w bramie Benjaminowej, byl tam przelozony nad straza, imieniem Jeryjasz, syn Selemijasza, syna Chananijaszowego, który pojmal Jeremijasza proroka, mówiac: Do Chaldejczyków ty uciekasz.
\par 14 A Jeremijasz odpowiedzial: Nieprawda, nie uciekam do Chaldejczyków; ale go nie chcial sluchac, owszem, pojmal Jeryjasz Jeremijasza, i przywiódl go do ksiazat.
\par 15 Tedy rozgniewawszy sie ksiazeta na Jeremijasza, ubili go, i podali go do wiezienia, do domu Jonatana pisarza; bo z niego byli uczynili dom wiezienia.
\par 16 A gdy wszedl Jeremijasz do onego domu a do tarasu ich, i siedzial tam Jeremijasz przez wiele dni;
\par 17 Tedy poslawszy król Sedekijasz, wzial go, i pytal go król w domu swoim potajemnie, mówiac: Jestze jakie slowo od Pana? I odpowiedzial Jeremijasz: Jest; przytem rzekl: W reke króla Babilonskiego podany bedziesz.
\par 18 Nadto rzekl Jeremijasz do króla Sedekijasza: Cózem zgrzeszyl przeciwko tobie, i slugom twoim, i ludowi twemu, zescie mie podali do tego domu wiezienia?
\par 19 I gdziez sa prorocy wasi, którzy wam prorokuja, mówiac: Nie przyjdzie król Babilonski na was, ani na te ziemie?
\par 20 Sluchajze teraz, prosze, królu, panie mój! niech bedzie, prosze, wazna prosba moja przed toba: Nie odsylajze mie do domu Jonatana pisarza, abym tam nie umarl.
\par 21 A tak rozkazal król Sedekijasz, aby wsadzony byl Jeremijasz do sieni strazy, a izby mu dawano bochenek chleba na dzien z ulicy piekarskiej, pókiby nie byl strawiony wszystek chleb w miescie. A tak siedzial Jeremijasz w sieni strazy.

\chapter{38}

\par 1 I uslyszal Sefatyjasz, syn Matanowy, i Godolijasz, syn Fassurowy, i Juchal, syn Selemijaszowy, i Fassur, syn Melchijaszowy, slowa, które Jeremijasz mówil do wszystkiego ludu, mówiac:
\par 2 Tak mówi Pan: Ktoby zostal w tem miescie, zginie od miecza, od glodu i od moru; ale ktoby wyszedl do Chaldejczyków, zyc bedzie, a bedzie mu dusza jego za korzysc, i zyw zostanie.
\par 3 Tak mówi Pan: Pewnie podane bedzie to miasto w rece wojska króla Babilonskiego, i wezmie je.
\par 4 Przetoz rzekli oni ksiazeta do króla: Niech umrze ten maz, poniewaz on oslabia rece mezów walecznych, pozostalych w tem miescie, i rece wszystkiego ludu, mówiac do nich takie slowa; albowiem ten maz nie szuka pokoju ludowi temu, ale zlego.
\par 5 Tedy rzekl król Sedekijasz: Oto jest w rece waszej; bo król nic zgola nie moze przeciwko wam.
\par 6 A tak wzieli Jeremijasza, który byl w sieni strazy, i wrzucili go do dolu Malchyjasza, syna królewskiego, i puscili Jeremijasza po powrozach; a w tym dole nie bylo nic wody, tylko bloto, a tak tonal Jeremijasz w onem blocie.
\par 7 Ale gdy uslyszal Ebedmelech Murzyn, dworzanin, który byl w domu królewskim, ze Jeremijasza podano do dolu, (a król siedzial w bramie Benjaminowej,)
\par 8 Wnet wyszedl Ebedmelech z domu królewskiego, i rzekl do króla, mówiac:
\par 9 Królu, panie mój! zle uczynili ci mezowie wszystko, co uczynili Jeremijaszowi prorokowi, ze go wrzucili do tego dolu; bocby byl umarl na pierwszem miejscu od glodu, poniewaz juz niemasz zadnego chleba w miescie.
\par 10 Przetoz rozkazal król Ebedmelechowi Murzynowi, mówiac: Wezmij z soba stad trzydziesci mezów, a wyciagnij Jeremijasza proroka z tego dolu, nizby umarl.
\par 11 Tedy wzial Ebedmelech onych mezów z soba, i wszedl do domu królewskiego pod skarbnice, i nabral stamtad starych szmacisk podartych, szmacisk, mówie, zbótwialych, które spuscil do Jeremijasza do onego dolu po powrozach.
\par 12 I rzekl Ebidmelech Murzyn do Jeremijasza: Nuze podlóz te stare podarte i zbótwiale szmaciska pod pachy rak twoich pod powrozy; i uczynil tak Jeremijasz.
\par 13 I wyciagneli tedy Jeremijasza powrozami, i dobyli go z onego dolu; i siedzial Jeremijasz w sieni strazy.
\par 14 Tedy poslal król Sedekijasz i wzial Jeremijasza proroka do siebie do trzecich drzwi, które byly przy domu Panskim. I rzekl król Jeremijaszowi: Spytam cie o jedne rzecz, nie taj nic przedemna.
\par 15 I rzekl Jeremijasz do króla Sedekijasza: Jezlic co powiem, pewnie mie zabijesz? A jezlic co poradze, nie usluchasz mie.
\par 16 Tedy przysiagl król Sedekijasz Jeremijaszowi potajemnie, mówiac: Jeko zyje Pan, który nam te dusze stworzyl, ze cie nie zabije, ani cie wydam w reke mezów tych, którzy szukaja duszy twojej.
\par 17 I rzekl Jeremijasz do Sedekijasza: Tak mówi Pan, Bóg zastepów, Bóg Izraelski: Jezli dobrowolnie wynijdziesz do ksiazat króla Babilonskiego, tedy zyc bedzie dusza twoja, a to miasto nie bedzie spalone ogniem; a tak zyw zostaniesz ty i dom twój;
\par 18 Ale jezli nie wynijdziesz do ksiazat króla Babilonskiego, pewnie bedzie podane to miasto w reke Chaldejczyków, i spala je ogniem, a i ty nie ujdziesz reki ich.
\par 19 Tedy rzekl król Sedekijasz do Jeremijasza: Bardzo sie boje Zydów, którzy pouciekali do Chaldejczyków, bym snac nie byl wydany w reke ich, a szydziliby ze mnie.
\par 20 I rzekl Jeremijasz: Nie wydadza, sluchaj prosze glosu Panskiego, któryc ja opowiadam, a bedziec dobrze, i zyc bedzie dusza twoja;
\par 21 A jezli sie bedziesz zbranial wynijsc, tedy to jest slowo, które mi Pan pokazal:
\par 22 Ze oto wszystkie niewiasty, które zostaly w domu króla Judzkiego, beda wywiedzione do ksiazat króla Babilonskiego, a te same rzeka: Namówili cie, i otrzymali to na tobie przyjaciele twoi, ze ulgnely w blocie nogi twoje, i cofnely sie nazad.
\par 23 Wszystkie takze zony twoje i synów twoich wywioda do Chaldejczyków, i ty sam nie ujdziesz reki ich, owszem reka króla Babilonskiego bedziesz pojmany, i to miasto spala ogniem.
\par 24 Tedy rzekl Sedekijasz do Jeremijasza: Niechaj nikt nie wie o tem, abys nie umarl;
\par 25 A jezliby ksiazeta uslyszawszy, zem mówil z toba, przyszli do ciebie, i rzeklic: Powiedz nam, prosze, cos mówil z królem, nie taj przed nami, a nie zabijemy cie: co z toba król mówil?
\par 26 Tedy im rzeczesz: Przelozylem prosbe moje przed królem, aby mie zas nie kazal odwisc do domu Jonatanowego, zebym tam nie umarl.
\par 27 A zeszli sie wszyscy ksiazeta do Jeremijasza, i pytali go; który im powiedzial wedlug tego wszsytkiego, jako mu byl król rozkazal. A tak milczac odeszli od niego, gdyz sie to bylo nie oglosilo.
\par 28 A Jeremijasz siedzial w sieni strazy az do onego dnia, którego wzieto Jeruzalem, gdzie byl, gdy dobywano Jeruzalemu.

\chapter{39}

\par 1 Roku dziewiatego Sedekijasza, króla Judzkiego, miesiaca dziesiatego, przyciagnal Nabuchodonozor, król Babilonski, ze wszystkiem wojskiem swojem do Jeruzalemu, i oblegli je.
\par 2 A jedenastego roku Sedekijasza, miesiaca czwartego, dziewiatego dnia tegoz miesiaca, dobyto miasta.
\par 3 I wpadli do niego wszyscy ksiazeta króla Babilonskiego, i usiedli w bramie sredniej: Nergalscharezer, Samgarnebu, Sarsechym, Rabsarys, Nergalscharezer, Rabmag, i wszyscy inni ksiazeta króla Babilonskiego.
\par 4 A gdy ich ujrzal Sedekijasz, król Judzki, i ze wszyscy mezowie waleczni uciekli, i wyszli w nocy z miasta droga ogrodu królewskiego, brama miedzy dwoma murami, uszedl tez i król droga ku puszczy.
\par 5 A wojsko Chaldejczyków gonilo ich, i doscigneli Sedekijasza na równinach Jerycha, i wzieli go, i przywiedli go do Nabuchodonozora, króla Babilonskiegp, do Reble, do ziemi Emat, gdzie wydal przeciwko niemu dekret.
\par 6 Bo pomordowal król Babilonski synów Sedekijaszowych w Rebli przed oczyma jego, i wszystkich najprzedniejszych z Judy pomordowal król Babilonski.
\par 7 Ale oczy Sedekijaszowi wylupil, a zwiazawszy go lancuchami miedzianemi prowadzil go do Babilonu.
\par 8 Dom takze królewski i dom onego ludu spalili Chaldejczycy ogniem, i mury Jeruzalemskie rozwalili.
\par 9 Ale ostatek ludu, który byl zostal w miescie, i zbiegi, którzy byli pouciekali do niego, i inny lud pozostaly zawiódl Nabuzardan, hetman zolnierski, do Babilonu.
\par 10 Tylko najpodlejszych z ludu, którzy nic nie mieli, zastawil Nabuzardan, hetman zolnierski, w ziemi Judzkiej, którym rozdal winnice i role dnia onego.
\par 11 A o Jeremijaszu przykazal Nabuchodonozor, król Babilonski, Nabuzardanowi, hetmanowi zolnierskiemu, mówiac:
\par 12 Wezmij go, a pilnie go dogladaj, a nie czyn mu nic zlego, ale jakoc rzeke, tak z nim postap.
\par 13 Przetoz poslawszy Nabuzardan, hetman zolnierski, i Nebusasban, Rabsarys i Nergalscharezer, Rabmag, i wszyscy hetmani króla Babilonskiego;
\par 14 Poslawszy, mówie, wzieli Jeremijasza z sieni strazy, i poruczyli go Godolijaszowi, synowi Ahikama, syna Safanowego, aby go dowiódl do domu. A tak mieszkal w posród ludu.
\par 15 I stalo sie do Jeremijasza slowo Panskie, gdy jeszcze byl zamkniety w sieni strazy, mówiac:
\par 16 Idz, a powiedz Ebedmelechowi Murzynowi, mówiac: Tak mówi Pan zastepów, Bóg Izraelski: Oto Ja przywiode slowa moje na to miasto ku zlemu a nie ku dobremu, i wypelnia sie przed obliczem twojem dnia onego;
\par 17 Ale ciebie wybawie onegoz dnia, mówi Pan, i nie bedziesz podany w reke mezów, których sie ty oblicza boisz.
\par 18 Albowiem cie pewnie wyrwe, abys od miecza nie upadl; ale bedziesz mial dusze twoje za korzysc, przeto, zes polozyl nadzieje we mnie, mówi Pan.

\chapter{40}

\par 1 Slowo, które sie stalo do Jeremijasza od Pana, gdy go wypuscil Nabuzardan, hetman zolnierski, z Ramy, wziawszy go, gdy byl zwiazany lancuchami w posród wszystkich wiezniów Jeruzalemskich i Judzkich, których wiedziono do Babilonu.
\par 2 A tak wzial hetman zolnierski Jeremijasza, i rzekl do niego: Pan, Bóg twój, opowiedzial byl to zle przeciwko miejscu temu;
\par 3 Przetoz je przywiódl, i uczynil Pan, jako mówil; boscie zgrzeszyli Panu, a nie sluchaliscie glosu jego, i dlatego sie wam to stalo.
\par 4 Teraz tedy, oto cie rozwiazuje dzis z tych lancuchów, które sa na rekach twoich. Jezlic sie zda rzecza dobra, isc zemna do Babilonu, pójdz, ja o tobie bede zawiadywal; a jezlic sie nie podoba isc zemna do Babilonu, tedy zaniechaj. Oto ta wszystka ziemia jest przed obliczem twojem; gdzie wolisz, a gdziec sie podoba isc, tam idz.
\par 5 A poniewaz sie tu on wiecej nie wróci, udaj sie do Godolijasza, syna Ahikamowego, syna Safanowego, którego przelozyl król Babilonski nad miastami Judzkiemi, a mieszkaj z nim w posród ludu, albo gdziec sie kolwiek podoba isc, idz. I dal mu hetman zolnierski na droge i upominek, i odprawil go.
\par 6 Przyszedl tedy Jeremijasz do Godolijasza, syna Ahikamowego, do Masfy, i mieszkal z nim w posród ludu, który byl pozostal w ziemi.
\par 7 A gdy uslyszeli wszyscy hetmani wojsk, którzy byli w polach, oni i wszystek lud ich, ze król Babilonski postanowil Godolijasza, syna Ahikamowego, nad ona ziemia, a iz mu zlecil mezów, i niewiasty, i dziatki, a to najpodlejszych onej ziemi, tych, którzy nie byli zaprowadzeni do Babilonu:
\par 8 Tedy przyszli do Godolijasza do Masfy, to jest, Izmael, syn Natanijaszowy, takze Johanan i Jonatan, synowie Kareaszowi, i Serajasz, syn Tanchumetowy, i synowie Efaj Netofatczyka, i Jasanijasz, syn Machatowy, oni i lud ich.
\par 9 Tedy im przysiagl Godolijasz, syn Ahikama, syna Safanowego, i ludowi ich, mówiac: Nie bójcie sie sluzyc Chaldejczykom, zostancie w ziemi, i sluzcie królowi Babilonskiemu, a dobrze wam bedzie.
\par 10 Bo oto i ja mieszkam w Masfie, abym sluzyl Chaldejczykom, którzy przychodza do nas; a wy zbierajcie wino i letni owoc i oliwe, a skladajcie do naczynia waszego, i mieszkajcie w miastach waszych, które trzymacie.
\par 11 Takze i wszyscy Zydzi, którzy byli u Moabczyków, i miedzy synami Ammonowymi, i miedzy Edomczykami, i którzy byli we wszystkich ziemiach, uslyszawszy, ze król Babilonski zostawil ostatek ludu z Judy, a iz przelozyl nad nimi Godolijasza, syna Ahika ma, syna Safanowego,
\par 12 Wrócili sie wiec wszyscy Zydzi ze wszystkich miejsc, do których byli zagnani, i przyszli do ziemi Judzkiej do Godolijasza do Masfy, i nazbierali wina, i letniego owocu bardzo wiele.
\par 13 Ale Johanan, syn Kareaszowy, i wszyscy ksiazeta wojsk, którzy byli w polu, przyszli do Godolijasza do Masfy,
\par 14 I rzekli do niego: Wieszze o tem, ze Baalis, król synów Ammonowych, poslal Izmaela, syna Natanijaszowego, aby cie zabil? Ale im nie uwierzyl Godolijasz, syn Ahikama.
\par 15 Nadto Johanan, syn Kareaszowy, rzekl do Godolijasza potajemnie w Masfie, mówiac: Niech ide, prosze, a zabije Izmaela, syna Natanijaszowego, wszak sie o tem nikt nie dowie.Przeczby cie mial zabic, a rozproszeniby mieli byc wszyscy Zydowie, którzy sie zebrali do ciebie, i zginac ostatek z Judy?
\par 16 Ale Godolijasz, syn Ahikamowy, rzekl do Johanana, syna Kareaszowego: Nie czyn tego; bo ty nieprawde mówisz o Izmaelu.

\chapter{41}

\par 1 I stalo sie miesiaca siódmego, ze przyszedl Izmael, syn Natanijasza, syna Elisamowego, z nasienia królewskiego, i hetmani królewscy, to jest, dziesiec mezów z nim, do Godolijasza, syna Ahikamowego, do Masfy, i jedli tam spolu chleb w Masfie.
\par 2 Potem wstawszy Izmael, syn Natanijaszowy, i dziesiec mezów, którzy z nim byli, zabili Godolijasza, syna Ahikama, syna Safanowego, mieczem; zabili mówie tego, którego byl przelozyl król Babilonski nad ona ziemia.
\par 3 Wszystkich takze Zydów, którzy byli z nim, z Godolijaszem w Masfie, i onych Chaldejczyków, których tam znalazl, mezów walecznych, pobil Izmael.
\par 4 A dnia wtórego, gdy zabil Godolijasza, (o czem nikt nie zwiedzial.)
\par 5 Przyszli niektórzy z Sychem, z Sylo, i z Samaryi, mezów osmdziesiat, ogoliwszy brody, i rozdarlszy szaty, i podrapawszy sie, którzy ofiare sniedna, i kadzidlo w rekach swych mieli, aby je odniesli do domu Panskiego.
\par 6 Tedy Izmael, syn Natanijaszowy, wyszedl przeciwko nim z Masfy, a idac szedl i plakal; a gdy sie spotkal z nimi, rzekl do nich: Pójdzcie do Godolijasza, syna Ahikamowego.
\par 7 Ale gdy przyszli w posród miasta, pobil ich Izmael, syn Natanijaszowy, i wrzucil ich w dól, sam i mezowie, którzy z nim byli.
\par 8 Lecz sie dziesiec mezów znalazlo miedzy nimi, którzy rzekli do Izmaela: Nie zabijaj nas, bo mamy skarby skryte w polu, pszenicy i jeczmienia, i oliwy, i miodu. I pohamowal sie; a nie zabil ich miedzy bracmi ich.
\par 9 A dól, do którego Izmael wrzucil do Godolijasza wszystkie trupy onych mezów, których pobil, ten jest, który uczynil król Aza, bojac sie Baazy, króla Izraelskiego, który napelnil Izmael syn Natanijaszowy pobitymi.
\par 10 I pobral w niewole Izmael wszystkie ostatki ludu, które byly w Masfie, córki królewskie, i wszystek lud, który byl zostal w Masfie, które byl poruczyl Nabuzardan, hetman zolnierski, Godolijaszowi, synowi Ahikamowemu, i wzial je w pojmanie Izmael, syn Natanijaszowy, i poszedl uchodzac do synów Ammonowych.
\par 11 Wtem uslyszal Johanan, syn Kareaszowy, i wszyscy hetmani owych wojsk, którzy byli z nim, o tem wszystkiem zlem, które uczynil Izmael, syn Natanijaszowy;
\par 12 I wzieli wszystek swój lud, i ciagneli, aby zwiedli bitwe z Izmaelem, synem Natanijaszowym, którego znalezli u wód wielkich, które sa w Gabaon.
\par 13 A gdy ujrzal wszystek lud, który byl z Izmaelem, Johanana, syna Kareaszowego, i wszystkich ksiazat wojsk, którzy z nim byli, uradowali sie;
\par 14 A obróciwszy sie wszystek on lud, który byl wzial w niewole Izmael z Masfy, wrócil sie zasie, a przyszedl do Johanana, syna Kareaszowego.
\par 15 Ale Izmael, syn Natanijaszowy, uszedl z osmioma mezami przed Johananem, i przyszedl do synów Ammonowych.
\par 16 Przetoz wzial Johanan, syn Kareaszowy, i wszyscy ksiazeta wojsk, którzy z nim byli, wszystek ostatek ludu, który zas przywiódl od Izmaela, syna Natanijaszowego, z Masfy, gdy zabil Godolijasza, syna Ahikamowego, mezów walecznych, i niewiasty, i dziatki, i komorników, których zas przywiódl z Gabaonu.
\par 17 A odszedlszy pomieszkali w gospodzie Chimchamowej, która jest u Betlehemu, aby idac uszli do Egiptu przed Chaldejczykami;
\par 18 Bo sie ich bali, przeto, ze byl zabil Izmael, syn Natanijaszowy, Godolijasza, syna Ahikamowego, którego byl przelozyl król Babilonski nad ona ziemia.

\chapter{42}

\par 1 Potem przystapili wszyscu ksiazeta wojsk, i Johanan, syn Karyjaszowy, i Jasanijasz, syn Hosajaszowy, i wszystek lud od malego az do wielkiego;
\par 2 I rzekli do Jeremijasza proroka:Niech przyjdzie prosze prosba nasza przed oblicze twoje, a módl sie za nami Panu, Bogu twemu, za wszystek ten ostatek; bo nas malo zostalo z wielu, jako to widzisz oczyma twemi;
\par 3 A niech nam oznajmi Pan, Bóg twój, droge, którabysmy chodzic, i cobysmy czynic mieli.
\par 4 I rzekl do nich Jeremijasz prorok: Slysze; oto ja modlic sie bede Panu, Bogu waszemu, wedlug slów waszych, a cokolwiek wam Pan odpowie, oznajmie wam, nie zataje nic przed wami.
\par 5 Oni zas rzekli do Jeremiasza: Niech bedzie Pan miedzy nami swiadkiem prawdziwym i wiernym, jezli nie uczynimy wedlug kazdego slowa, z którem cie posle Pan, Bóg twój, do nas.
\par 6 Badz dobrze badz zle, glosu Pana, Boga naszego, dla którego cie posylamy do niego, usluchamy, aby sie nam dobrze dzialo, gdy bedziemy sluchac glosu Pana, Boga naszego.
\par 7 A po wyjsciu dziesieciu dni, gdy sie stalo slowo Panskie do Jeremijasza,
\par 8 Zawolal Johanana, syna Karejaszowego, i wszystkich ksiazat wojsk, którzy z nim byli, i wszystkiego ludu, od malego az do wielkiego,
\par 9 I rzekl do nich: Tak mówi Pan, Bóg Izraelski, do któregoscie mie poslali, abym przedlozyl prosbe wasza przed obliczem jego:
\par 10 Jezli sie nawrócicie, i zostaniecie w tej ziemi, zaiste pobuduje was, a nie zepsuje, i wszczepie was, a nie wykorzenie; bo mi zal tego zlego, którem wam uczynil. Nie bójciez sie oblicza króla Babilonskiego, którego sie wy boicie;
\par 11 Nie bójcie sie go, mówi Pan; bom jest z wami, abym was wybawil i wyrwal was z reki jego;
\par 12 Nadto zjednam wam laske, aby sie zmilowal nad wami, i dal sie wam wrócic do ziemi waszej.
\par 13 Ale rzeczecieli: Nie zostaniemy w tej ziemi, nie sluchajac glosu Pana, Boga waszego,
\par 14 A mówiac: Zadna miara; ale do ziemi Egipskiej pójdziemy, gdzie nie ujrzymy wojny, ani glosu traby nie uslyszymy, a chleba laknac nie bedziemy, i tam mieszkac bedziemy;
\par 15 Przetoz teraz sluchajcie slowa Panskiego, ostatki Judzkie! Tak mówi Pan zastepów, Bóg Izraelski: Jezli wy upornie przy tem zostaniecie, abyscie szli do ziemi Egipskiej, a pójdziecieli, abyscie tam mieszkali:
\par 16 Tedy was pewnie miecz, którego sie boicie, tam w ziemi Egipskiej doscignie; i glód, którego sie obawiacie, tam przyjdzie na was w Egipcie, i tam pomrzecie.
\par 17 Takci sie stanie tym wszystkim mezom, którzy sie na to koniecznie udali isc do Egiptu, aby tam pielgrzymowali, ze pomra od miecza, od glodu i od moru, a zaden z nich nie zostanie, ani kto ujdzie przed tem zlem, które ja przywiode na nich.
\par 18 Bo tak mówi Pan zastepów, Bóg Izraelski: Jako sie wylala popedliwosc moja i gniew mój na obywateli Jeruzalemskich, tak sie wyleje zapalczywosc moja na was, gdy wnijdziecie do Egiptu; i bedziecie na przeklinanie, i na zdumienie i na zlozeczenie i na hanbe, a nie ogladacie wiecej miejsca tego.
\par 19 Do wasci mówi Pan, o ostatki Judzkie! Nie wchodzcie do Egiptu; wiedzcie wiedzac, (bo sie dzis oswiadczam przeciwko wam,)
\par 20 Poniewazescie zwiedli dusze wasze, poslawszy mie do Pana, Boga waszego, mówiac: Módl sie za nami Panu, Bogu naszemu, a wedlug wszystkiego, cokolwiek powie Pan, Bóg nasz, tak nam oznajmij, a my uczynimy.
\par 21 A gdy wam to dzis oznajmuje, wszakze nie sluchacie glosu Pana, Boga waszego, we wszystkiem, o co mie do was poslal.
\par 22 Przetoz mówie: Wiedzcie wiedzac, ze mieczem, glodem i morem pomrzecie na tem miejscu, do którego pragniecie wnijsc, abyscie tam pielgrzymowali.

\chapter{43}

\par 1 A gdy przestal Jeremijasz mówic do wszystkiego ludu wszystkich slów Pana, Boga ich, z któremi go byl poslal Pan, Bóg ich, do nich, wszystkich mówie tych slów,
\par 2 Rzekl Azaryjasz syn Hosajaszowy i Johanan, syn Karejaszowy, i wszyscy mezowie pyszni, mówiac do Jeremijasza: Klamstwo ty powiadasz; nie poslal cie Pan, Bóg nasz, mówiac: Nie chodzcie do Egiptu, abyscie tam mieszkali;
\par 3 Ale Baruch, syn Neryjaszowy, podszczuwa cie przeciwko nam, aby nas wydal w rece Chaldejczyków, zeby nas pobili, albo nas zabrali do Babilonu.
\par 4 I nie usluchal Johanan, syn Karejaszowy, i wszyscy ksiazeta wojsk, takze i wszystek lud glosu Panskiego, zeby zostali w ziemi Judzkiej.
\par 5 Ale Johanan, syn Karejaszowy, i wszyscy ksiazeta wojsk wzieli wszystek ostatek z Judy, którzy sie byli wrócili ze wszystkich narodów, do których byli wygnani, aby, mieszkali w ziemi Judzkiej:
\par 6 Mezów, i niewiasty, i dzieci, i córki królewskie, i kazda dusze, która Nabuzardan, hetman zolnierski, z Godolijaszem, synem Ahikamowym, syna Safanowego, zostawil, i z Jeremijaszem prorokiem, i z Baruchem, synem Neryjaszowym;
\par 7 I weszli do ziemi Egipskiej, bo nie byli posluszni glosowi Panskiemu, i przyszli do Tachpanches.
\par 8 I stalo sie slowo Panskie do Jeremijasza w Tachpanchos, mówiac:
\par 9 Nabierz w rece swe kamieni wielkich, a skryj je w gline w cegielnicy, która jest przed brama domu Faraonowego w Tachpanches, przed oczyma mezów Judzkich;
\par 10 A rzecz im: Tak mówi Pan zastepów, Bóg Izraelski: Oto Ja posle i przywiode Nabuchodonozora, króla Babilonskiego, sluge mego i postawie stolice jego na tych kamieniach, którem skryl: i rozbije majestat swój na nich.
\par 11 Bo przyciagnawszy wytraci ziemie Egipska; którzy na smierc oddani, na smierc pójda, a którzy do wiezienia, do wiezienia, a którzy pod miecz, pod miecz.
\par 12 I zapale ogien w domach bogów Egipskich, i popali je, a one pobierze do wiezienia: i odzieje sie ziemia Egipska jako sie odziewa pasterz szata swoja, i wynijdzie stamtad w pokoju,
\par 13 Gdy pokruszy slupy w Betsemes, które jest w ziemi Egipskiej, i domy bogów Egipskich popali ogniem.

\chapter{44}

\par 1 Slowo, które sie stalo do Jeremijasza o wszystkich Zydach, którzy mieszkali w ziemi Egipskiej, którzy mieszkali w Migdolu, i w Tachpanches, i w Nof, i w ziemi Patros, mówiac:
\par 2 Tak mówi Pan zastepów, Bóg Izraelski: Wyscie widzieli wszystko ono zle, którem przywiódl na Jeruzalem, i na wszystkie miasta Judzkie, ze oto puste sa po dzis dzien, niemasz w nich obywatela.
\par 3 Dla zlosci ich, która czynili, aby mie do gniewu pobudzali, chodzac kadzic i sluzyc Bogom cudzym, których nie znali sami, wy i ojcowie wasi;
\par 4 Chociazem posylal do was wszystkich slug moich, proroków, rano wstawajac i posylajac a mówiac: Nie czyncie prosze tej obrzydliwosci, której nienawidze.
\par 5 Ale nie usluchali ani naklonili ucha swego, aby sie odwrócili od zlosci swojej, a nie kadzili bogom cudzym.
\par 6 Przetoz wylany jest gniew mój, a zapalczywosc moja zapalila sie w miastach Judzkich, i w ulicach Jeruzalemskich, i obrócily sie w pustynie, i wzburzenie, jako sie to dzis pokazuje.
\par 7 Teraz tedy tak mówi Pan, Bóg zastepów, Bóg Izraelski: Czemu wy czynicie te zlosc wielka przeciwko duszom waszym, aby z was byl wykorzeniony maz i niewiasta, dziecie i ssacy z posrodku Judy, tak, zeby z was nic nie zostalo,
\par 8 Drazniac mie sprawami rak waszych, kadzac bogom cudzym w ziemi Egipskiej, do którejscie weszli, abyscie tam pielgrzymowali, izbyscie byli wykorzenieni, a byli na przeklestwo i na hanbe u wszystkich narodów na ziemi?
\par 9 Azascie zapamietali na zlosc ojców waszych, i na zlosc królów Judzkich, i na zlosc zon ich, i na zlosci wasze, i na zlosci zon waszych, których sie dopuscily w ziemi Judzkiej i po ulicach Jeruzalemskich?
\par 10 Nie upokorzyli sie az do dnia tego, i nie bali sie, ani chodzili w zakonie moim i w ustawach moich, które podaje wam i ojcom waszym.
\par 11 Przetoz tak mówi Pan zastepów, Bóg Izraelski: Oto Ja obróce oblicze moje przeciwko wam na zle, aby wykorzenil wszystkiego Jude.
\par 12 Wygubie zaiste ostatki Judzkie, którzy upornie weszli do ziemi Egipskiej, aby tam pielgrzymowali, tak, ze zniszczeja wszyscy w ziemi Egipskiej, polegna od miecza, od glodu zniszczeja od najmniejszego az do najwiekszego, od miecza i od glodu pomra; nadto beda na przeklinanie, i na zdumienie, i na zloszczenie, i na hanbe.
\par 13 Bo nawiedze tych, którzy mieszkaja w ziemi Egipskiej, jakom nawiedzil Jeruzalem mieczem, glodem i morem.
\par 14 I nie bedzie ktoby uszedl i zostal z ostatków Judzkich, którzy weszli do ziemi Egipskiej, aby tam pielgrzymowali, aby sie zas wrócic mieli do ziemi Judzkiej, do której sie oni pragna wrócic, i mieszkac tam; ale sie nie wróca, tylko ci, którzy ujda.
\par 15 Tedy odpowiedzieli Jeremijaszowi wszyscy mezowie, którzy wiedzieli, iz kadzaly zony ich bogom cudzym, one wszystkie niewiasty, których stalo mnóstwo wielkie, i wszystek lud, który mieszkal w ziemi Egipskiej w Patros, mówiac:
\par 16 W slowie, któres mówil do nas imieniem Panskiem, nie usluchaly cie;
\par 17 Ale dosyc czynimy kazdemu slowu, które wynijdzie z ust naszych, kadzac królowej niebieskiej, i sprawujac jej ofiary mokre, jakosmy czynili, my i ojcowie nasi, królowie nasi, i ksiazeta nasi w miastach Judzkich i po ulicach Jeruzalemskich, a najad alismy sie chleba, i dobrze nam bylo, a nic zlegosmy nie widzieli.
\par 18 Ale odtad jakosmy przestali kadzic królowej niebieskiej, i sprawowac jej ofiary mokre, na wszystkiem nam schodzi, a od miecza i od glodu niszczejemy.
\par 19 A gdy kadzimy królowej niebieskiej, i sprawujemy jej ofiary mokre, izali jej bez mezów naszych placki czynimy, ksztaltujac ja, i sprawujac jej ofiary mokre?
\par 20 Tedy rzekl Jeremijasz do wszystkiego ludu, do mezów i do niewiast, i do wszystkiego pospólstwa, którzy mu tak odpowiedzieli, mówiac:
\par 21 Izali na kadzenie, któremescie kadzili w miastach Judzkich i w ulicach Jeruzalemskich, wy i ojcowie wasi, królowie wasi, i ksiazeta wasi, i lud ziemi, nie wspomnial Pan, i nie wstapilo to na serce jego?
\par 22 Tak, ze nie mógl Pan dalej znosic zlosci spraw waszych, i obrzydliwosci, którescie broili; dlatego sie stala ziemia wasza spustoszeniem i zdumieniem, i przeklestwem, tak, ze niemasz w niej obywatela, jako sie to dzis pokazuje,
\par 23 Dlatego, zescie kadziili balwanom, i zescie grzeszyli przeciw Panu, a nie sluchaliscie glosu Panskiego, a tak w zakonie jego, i w ustawach jego, ani w swiadectwach jego nie chodziliscie, dlatego przyszlo na was to zle, jako sie to dzis pokazuje.
\par 24 Nadto rzekl Jeremijasz do wszystkiego ludu, i do wszystkich niewiast: Sluchajcie slowa Panskiego wszyscy ludzie Judzcy, którzyscie z ziemi Egipskiej.
\par 25 Tak rzekl Pan zastepów, Bóg Izraelski, mówiac: Wy i zony wasze mówiliscie usty swemi, i wypelniliscie rekami swemi, mówiac: Uczynimy dosyc slubom naszym, któresmy poslubili, abysmy kadzili królowej niebieskiej, i sprawowali jej ofiary mokre, a tak wszystka sila wypelniacie sluby wasze, i samym skutkiem wykonywacie sluby wasze.
\par 26 Przetoz sluchajcie slowa Panskiego wszyscy ludzie Judzcy, którzy mieszkacie w ziemi Egipskiej: Oto Ja przysiegam przez imie moje wielkie, mówi Pan, ze nie bedzie wiecej imie moje wzywane usty zadnego meza Judzkiego po wszystkiej ziemi Egipskiej, któryby rzekl: Jako zyje panujacy Pan!
\par 27 Oto Ja bede czul nad nimi na zle, a nie na dobre; i niszczec beda wszyscy mezowie Judzcy, którzy sa w ziemi Egipskiej, mieczem i glodem, az do szczetu wygina;
\par 28 A którzy ujda miecza, wróca sie z ziemi Egipskiej do ziemi Judzkiej, ludu maly poczet; i pozna wszystek ostatek Judzki, którzy weszli do ziemi Egipskiej, aby tam pielgrzymowali, czyje sie slowo ostoi, mojeli, czyli ich?
\par 29 A to miejcie za znak, mówi Pan, ze Ja was nawiedze na tem miejscu, abyscie wiedzieli, iz sie prawdziwie spelnia slowa moje nad wami ku zlemu.
\par 30 Tak mówi Pan: Oto Ja podam Faraona Chofra, króla Egipskiego, w reke nieprzyjaciól jego, i w reke szukajacych duszy jego, jakom podal Sedekijasza, króla Judzkiego, w reke Nabuchodonozora, króla Babilonskiego, nieprzyjaciela jego, który szukal duszy jego.

\chapter{45}

\par 1 Slowo, które mówil Jeremijasz prorok do Barucha, syna Neryjaszowego, gdy pisal te slowa w ksiegi z ust Jeremijaszowych roku czwartego za Joakima, syna Jozyjasza, króla Judzkiego, mówiac:
\par 2 Tak mówi Pan, Bóg Izraelski, o tobie, Baruchu!
\par 3 Rzekles: Biada mnie teraz! bo Pan przyczynia zalosci do bolasci mojej; upracowalem sie w wzdychaniu mojem, a odpoczynku nie znajduje.
\par 4 Tak rzeczesz do niego: Tak mówi Pan: Oto com zbudowal, Ja rozwalam, a com wszczepil, Ja wyrywam, i te wszystke ziemie.
\par 5 A ty sobie szukasz rzeczy wielkich? Nie szukaj. Bo oto Ja przywiode zle na wszelkie cialo, mówi Pan: ale tobie dam dusze twoje w korzysci na wszelkich miejscach, dokadkolwiek pójdziesz.

\chapter{46}

\par 1 Slowo Panskie, które sie stalo do Jeremijasza proroka przeciwko tym narodom.
\par 2 Przeciwko Egiptowi. Przeciwko wojsku Faraona Necha, króla Egipskiego, (które bylo nad rzeka Eufrates u Karchemis, które porazil Nabuchodonozor, król Babilonski) roku czwartego Joakima, syna Jozyjaszowego, króla Judzkiego;
\par 3 Gotujcie tarcz i paweze, a wychodzcie na wojne;
\par 4 Zaprzegajcie konie, a wsiadajcie jezdni, stancie w helmach, wycierajcie oszczepy, obleczcie sie w pancerze.
\par 5 Czemuz tych widze zatrwozonych, tyl podawajacych, a mocarzy ich startych i predko uciekajacych, tak, ze sie ani ogladaja? Strach jest zewszad, mówi Pan,
\par 6 Aby nie uciekl predki, a nie uszedl mocarz; aby sie na pólnocy o brzeg rzeki Eufrates otracili i upadli.
\par 7 Któz to jest, który jako rzeka wzbiera? Którego sie wody wzruszaja jako rzeki?
\par 8 Egipt jako rzeka wzbiera, a jego wody wzruszaja sie jako rzeki, i mówi: Pociagne, okryje ziemie, wygubie miasto, i tych, co w niem mieszkaja.
\par 9 Poskoczcie konie, a zagrzmijcie wozy, a niech sie rusza i mocarze, Murzynowie, i Putejczycy noszacy tarcz, i Ludymczycy, którzy nosza i ciagna luk,
\par 10 Bo ten dzien panujacego Pana zastepów bedzie dzien pomsty, aby sie pomscil nad nieprzyjaciolmi swymi, których miecz pozre, a nasyci sie, i opije sie krwia ich; bo ofiara panujacego Pana zastepów bedzie w ziemi pólnocnej u rzeki Eufrates.
\par 11 Wstap do Galaad, a nabierz soku balsamowego, panno, córko Egipska! Alec prózno uzywasz wiele lekarstw; bo nie bedziesz uleczona.
\par 12 Narody uslysza o sromocie twojej, a narzekanie twoje napelnilo ziemie; bo mocarz na mocarza natarl, tak, ze spolem oba upadaja.
\par 13 Slowo, które mówil Pan do Jeremijasza, proroka, o tem, ze ma przyjsc Nabuchodonozor, król Babilonski, a porazic ziemie Egipska.
\par 14 Oznajmijcie w Egipcie, a rozgloscie w Migdolu; opowiadajcie takze w Nof, i w Tachpanches; rzeczcie: Postuj a nagotuj sie; wszakze miecz pozre to, co jest okolo ciebie.
\par 15 Przecz porazony jest kazdy z mocarzów twoich? Nie moze sie ostac, przeto, ze Pan natarl nan.
\par 16 Wielec bedzie tych, którzy poszwankuja a padna jeden na drugiego, i rzeka: Wstan, a wrócmy sie do ludu naszego, i do ziemi urodzenia naszego przed ostrzem miecza pustoszacego.
\par 17 Tam beda wolac: Farao, król Egipski, jest tylko prózny trzask, juz mu pominal czas postanowiony,
\par 18 Jakom zywy Ja, mówi król, Pan zastepów imie jego; ze jako Tabor miedzy górami, i jako Karmel przy morzu, tak to przyjdzie.
\par 19 Spraw sobie naczynie przeprowadzenia, obywatelko, córko Egipska! bo Nof pustynia bedzie i spustoszeje, i bedzie bez obywatela.
\par 20 Egipt jest jako piekna jalowica; ale zabicie jej od pólnocy idzie, idzie.
\par 21 Wiec i najemnicy jego w posrodku niego sa jako ciele utuczone, ale i oni takze obróciwszy sie uciekna spolem, nie ostoja sie; bo dzien porazki ich przyszedl na nich, czas nawiedzenia ich.
\par 22 Glos jego wynijdzie jako wezowy; bo z wojskiem ida, a z siekierami przyjda nan, jako ci, co wyrabuja drzewo.
\par 23 Wyrabia las jego, mówi Pan, choc policzony byc nie moze; bo sie nad szarancze rozmnozyli, i niemasz im liczby.
\par 24 Zawstydzi sie córka Egipska; podana bedzie w reke ludu pólnocnego.
\par 25 Pan zastepów, Bóg Izraelski, mówi: Oto Ja nawiedze ludne miasto No, takze Faraona i Egipt, i bogów jego, i królów jego, Faraona mówie, i tych, którzy w nim ufaja:
\par 26 I podam ich w reke tych, którzy szukaja duszy ich, to jest w reke Nabuchodonozora, króla Babilonskiego, i reka slug jego; lecz potem mieszkac w nim beda, jako za dawnych dni, mówi Pan.
\par 27 Ale sie ty nie bój, slugo mój Jakóbie! a nie lekaj sie, o Izraelu! Bo oto Ja ciebie wybawie z daleka, i nasienie twoje z ziemi pojmania ich; i wróci sie Jakób, aby odpoczywal i aby mial pokój, a nie bedzie, ktoby go postraszyl;
\par 28 Ty, mówie, Jakóbie, slugo mój! nie bój sie, mówi Pan; bom Ja z toba. Uczynie zaiste koniec wszystkim narodom, do których cie wypedze; lecz tobie nie uczynie konca, ale cie miernie karac bede, a cale cie bez karania nie zostawie.

\chapter{47}

\par 1 Slowo Panskie, które sie stalo do Jeremijasza proroka przeciwko Filistynczykom, przedtem, niz Farao dobil Gazy.
\par 2 Tak mówi Pan: Oto wody wystepuja od pólnocy, i beda jako powódz gwaltowna, a zatopia ziemie i co jest na niej, miasto i mieszkajacych w niem, dlaczego wolac beda ludzie, i zawyja wszyscy obywatele ziemi.
\par 3 Dla glosu tetnienia kopyt wasniwych koni jego, dla grzmotu wozów jego, i trzasku kól jego nie obejrza sie ojcowie na synów, majac opuszczone rece;
\par 4 Dla dnia, który przyjsc ma na zburzenie wszystkich Filistynczyków, i na wykorzenienie Tyru i Sydonu ze wszystka pozostala pomoca; bo zburzy Pan Filistynczyków, ostatek wyspy Kaftor.
\par 5 Przyjdzie oblysienie na Gaze, i wykorzeniony bedzie Aszkalon i ostatki doliny ich; dokadze sie rzezac bedziesz?
\par 6 O mieczu Panski, dokad sie nie uspokoisz? Wróc sie do pochew twoich, usmierz sie, a ucichnij.
\par 7 Ale jakozbys sie uspokoil? Wszak mu Pan przykazal; przeciwko Aszkalonowi i przeciwko brzegowi morskiemu, tam go postawil.

\chapter{48}

\par 1 Przeciwko Moabowi. Tak mówi Pan zastepów, Bóg Izraelski: Biada miastu Nebo, bo spustoszone bedzie; pohanbione i wziete bedzie Karyjataim; zawstydzone bedzie miasto na miejscu wysokiem, i bac sie bedzie.
\par 2 Nie bedzie sie wiecej chlubil Moab z Hesebonu; mysla zle przeciwko niemu, mówiac: Pójdzcie, a wytracmy ich z narodu. I ty, Madmenie! wykorzeniony bedziesz, miecz pójdzie za toba.
\par 3 Glos krzyku z Choronaim: O spustoszenie i zburzenie wielkie!
\par 4 Starty bedzie Moab, slyszany bedzie krzyk maluczkich jego,
\par 5 Przeto, ze na drodze Luchytskiej bedzie ustawiczny placz, a któredy zstepuja do Choronaim, nieprzyjaciele krzyk zburzenia slyszec beda,
\par 6 Mówiacych: Uciekajcie, wybawcie dusze swoje, a stancie sie jako wrzos na puszczy.
\par 7 Bo dlatego, ze masz nadzieje w dostatku twoim, i w skarbach twoich, bedziesz tez wziete, i Chamos pójdzie w pojmanie, kaplani jego, takze i ksiazeta jego.
\par 8 Bo burzyciel przyjdzie na kazde miasto, a zadne miasto nie ujdzie; zginie i dolina, i równiny spustoszone beda, jako mówi Pan.
\par 9 Dajcie skrzydla Moabowi, niech predko uleci; bo miasta jego przyjda w spustoszenie, tak, ze nie bedzie w nich obywatela.
\par 10 Przeklety, kto zdradliwie czyni sprawe Panska; przeklety takze, kto hamuje miecz swój ode krwi.
\par 11 Mialci Moab pokój od dziecinstwa swego, i usadzil sie na drozdzach swoich, ani byl przelewany z naczynia w naczynie, to jest, w pojmanie nie chodzil, dla czego zostal w nim smak jego, a won jego nie zmienila sie.
\par 12 Przetoz oto dni ida, mówi Pan, ze posle nan tych, którzy wtargnienia czynia, a pojmaja go, i naczynia jego wypróznia, a lagwie jego potluka.
\par 13 I zawstydzony bedzie Moab od Chamosa, jako zawstydzony jest dom Izraelski od Betel, nadziei swojej.
\par 14 Jakoz mówicie: Mocnismy, a mezowie duzy do boju?
\par 15 Zburzony bedzie Moab, i z miast swoich wynijdzie, a wyborni mlodziency jego pójda na zabicie, mówi król, Pan zastepów imie jego.
\par 16 Blisko jest zginienie Moabowe, i przyjdzie; a nieszczescie jego predko sie pospieszy.
\par 17 Uzalcie sie go wszyscy, którzy mieszkacie okolo niego, i wszyscy, którzy znacie imie jego, mówcie: Jakoz sie zlamala laska mocy, i kij ozdobny?
\par 18 Zstap z slawy, a siadz w pragnieniu, obywatelko, córko Dybonska! bo zburzyciel Moabu przyciagnie przeciwko tobie, rozrzuci twierdze twoje.
\par 19 Stan na drodze a przypatrz sie pilnie, obywatelko Aroer! pytaj tego, który ucieka, i tej, która uchodzi, mówiac: Co sie dzieje?
\par 20 Zawstydzony jest Moab, bo jest potarty; narzekajcie a wolajcie, opowiadajcie w Arnon, ze pustosza Moaba.
\par 21 Bo sad przyszedl na ziemie tej równiny, na Holon, i na Jassa, i na Mefaat,
\par 22 I na Dybon, i na Nebo, i na Bet Dyblataim,
\par 23 I na Kiryjataim, i na Betgamul, i Betmehon.
\par 24 I na Karyjot, i na Bocre, i na wszystkie miasta ziemi Moabskiej, dalekie i bliskie.
\par 25 Odciety bedzie róg Moabski, i ramie jego bedzie starte, mówi Pan.
\par 26 Opójcie go, poniewaz sie przeciw Panu podniósl; niech sie wala Moab w blwocinach swoich, niech bedzie i on na posmiech.
\par 27 Bo azaz w posmiewisku nie byl u ciebie Izrael? Izali go miedzy zlodziejami zastano? ze, ilekroc mówisz o nim, wyskakujesz.
\par 28 Opuszczajcie miasta a mieszkajcie na skale, obywatele Moabscy! a badzcie jako golebica, która sciele gniazdo swoje na kraju dziury.
\par 29 Slyszelismy o pysze Moabowej, ze jest bardzo pyszny; o wysokomyslnosci jego, i o hardosci jego, i o nadetosci jego, i o wynioslosci serca jego.
\par 30 Znamci Ja, mówi Pan, zagniewanie jego; lecz nie ma sily; klamstwa jego nie dowioda tego.
\par 31 Dlatego nad Moabczykami narzekam, a nade wszystkim Moabem wolam, a dla obywateli Kircheres wzdycha serce moje.
\par 32 Bardziej niz plakano Jazerczyków, placze nad toba, o winna macico Sabama! Latorosli twoje dostana sie za morze, az do morza Jazer dosiegna; na letnie owoce twoje, i na zbieranie wina twego burzyciel przypadnie.
\par 33 I ustanie wesele i radosc nad polem urodzajnem w ziemi Moabskiej, a winu z prasy wstret uczynie; nie beda go tloczyc z wykrzykaniem, a wykrzykanie nie bedzie wykrzykaniem.
\par 34 Bardziej krzyczec beda niz Hesebonczycy; az do Eleale, az do Jazy wydadza glos swój, od Zoar az do Choronaim, jako jalowica trzecioletnia; bo tez i wody Nimrym zniszczeja.
\par 35 I uczynie, mówi Pan, ze ustanie w Moabie ofiarujacy na wyzynach, i kadzacy bogom swoim.
\par 36 Przetoz serce moje nad Moabem jako piszczalki piszczec bedzie; takze nad obywatelami w Kircheres serce moje jako piszczalki piszczec bedzie, i dlatego, ze i zboze zgromadzone wniwecz sie obróci.
\par 37 Bo na kazdej glowie bedzie lysina, i kazda broda ogolona bedzie; na wszystkich rekach beda szramy, a na biodrach wór.
\par 38 Po wszystkich dachach Moabskich i po ulicach jego, wszedy nic nie bedzie, tylko narzekanie; bom skruszyl Moaba jako naczynie nieuzyteczne, mówi Pan.
\par 39 Narzekac beda, mówiac: Jakoc jest starty! Jako tyl podal Moab z hanba! i jest Moab posmiewiskiem, i postrachem wszystkim, którzy sa okolo niego.
\par 40 Bo tak mówi Pan: Oto nieprzyjaciel jako orzel przyleci, a rozciagnie skrzydla swe na Moaba.
\par 41 Karyjot wziete bedzie, i zamki wziete beda, a serce mocarzów Moabskich dnia onego bedzie jako serce niewiasty bolejacej.
\par 42 I wygladzony bedzie Moab z ludu; bo sie przeciwko Panu podnosil.
\par 43 Strach i dól, i sidlo nad toba, o obywatelu Moabski! mówi Pan.
\par 44 Kto uciecze przed strachem, wpadnie w dól; a kto wynijdzie z dolu, sidlem ulapiony bedzie; bo przywiode nan, to jest na Moaba, rok nawiedzienia jego, mówi Pan.
\par 45 W cieniu Hesebon stawali ci, którzy uciekali przed gwaltem; ale ogien wynijdzie z Hesebonu, i plomien z posrodku Sehonu, i pozre kat Moabski, i wierzch glowy tych, którzy go burza.
\par 46 Biada tobie, Moabie!zaginiec lud Chamosowy; bo synowie twoi zabrani beda w niewole, i córki twoje do wiezienia.
\par 47 Wszakze zasie przywróce wiezniów Moabskich w ostateczne dni, mówi Pan. Az dotad sad o Moabie.

\chapter{49}

\par 1 Przeciwko synom Amonowym. Tak mówi Pan: Izali Izrael nie ma synów? Izali zadnego dziedzica nie ma? czemuz król ich dziedzicznie opanowal Gad, a lud jego czemu mieszka w miastach jego?
\par 2 Przetoz oto dni ida, mówi Pan, sprawie, ze uslysza trabienie wojenne przeciwko Rabbie synów Amonowych, i bedzie obrócona w kupe rumu, a inne miasta jego ogniem spalone beda, i posiedzie Izrael dzierzawców swoich, mówi Pan.
\par 3 Rozrzewnij sie, Hesebonie! bo Haj spustoszone bedzie; krzyczcie, o córki Rabby! przepaszcie sie worami, narzekajcie, a tulajcie sie okolo plotów; bo król wasz do wiezienia pójdzie, takze kaplani jego, i ksiazeta jego spolem.
\par 4 Cóz sie przechwalasz dolinami? Splynela dolina twoja, o córko uporna! która ufasz w skarbach twoich, mówiac: Któz przyciagnie przeciwko mnie?
\par 5 Oto Ja przywiode na cie strach, mówi Pan, Pan zastepów od wszystkich, którzy sa okolo ciebie, przez który rozegnani bedziecie jeden od drugiego, a nie bedzie, ktoby zebral tulajacych sie.
\par 6 Wszakze potem przywiode zas wiezniów synów Amonowych, mówi Pan.
\par 7 Przeciwko Edomczykom. Tak mówi Pan zastepów: Izali niemasz wiecej madrosci w Teman? Izali zginela rada od roztropnych, a wniwecz sie obrócila madrosc ich?
\par 8 Uciekajcie, obróccie sie, zstapcie w glebokosc na mieszkanie, obywatele Dedan! bo przywiode zatracenie na Ezawa czasu nawiedzenia jego.
\par 9 Gdyby ci, którzy zbieraja wino, przyszli na cie, izaliby nie zostawili jakich gron? gdyby sie wkradli zlodzieje w nocy, azazby szkodzili wiecej nad potrzebe swoje?
\par 10 Lecz Ja obnaze Ezawa, odkryje skrytosci jego, tak, iz sie ukryc nie bedzie mógl: zniszczone bedzie nasienie jego, i bracia jego, i sasiedzi jego, tak, ze zgola nie bedzie, ktoby rzekl:
\par 11 Zaniechaj sierotek twoich, ja je zywic bede, a wdowy twoje we mnie ufac beda.
\par 12 Tak zaiste mówi Pan: Oto ci, którym nie przysadzono pic z kubka tego, przecie pija z niego, a tybys mial zgola tego ujsc?
\par 13 Nie ujdziesz, ale koniecznie pic bedziesz; bo przez sie przysiegam, mówi Pan, iz na spustoszenie, na pohanbienie, na zburzenie, i na przeklestwo Bocra przyjdzie, i wszystkie miasta jego beda pustemi na wieki.
\par 14 Slyszalem wiesc od Pana, ze do narodów poslany jest posel mówiacy: Zgromadzcie sie, a ciagnijcie, przeciwko niemu, wstanciez do bitwy.
\par 15 Bo oto sprawie, abys byl najlichszym miedzy narodami, i wzgardzonym miedzy ludzmi.
\par 16 Hardosc twoja zdradzi cie, i pycha serca twego, ty, który mieszkasz w rozsiadlinach skalnych, który sie trzymasz wysokich pogórków; bys tez wywyzszyl jako orzel gniazdo swoje, i stamtad cie stargne, mówi Pan.
\par 17 I bedzie ziemia Edomska pustynia, i ktokolwiek pójdzie przez nie, zdumieje sie, i swistac bedzie nad wszystkiemi plagami jej.
\par 18 Jako podwrócona jest Sodoma i Gomora z przyleglosciami swojemi, mówi Pan, tak sie tam nikt nie osadzi, ani w niej syn czlowieczy mieszkac bedzie.
\par 19 Oto aczkolwiek jako lew wystepuje, i bardziej niz nadecie Jordanu sie podnosi przeciwko przybytkowi Mocnego: wszakze go nagle wypedze z tej ziemi, a tego, który jest obrany, przeloze nad nia: bo któz mnie jest podobny? i kto mi da rok; a kto jest tym pasterzem, któryby sie postawil przeciwko mnie?
\par 20 Przetoz sluchajcie rady Panskiej, która uradzil przeciwko Edomczykom; i zamyslów jego, które umyslil przeciwko obywatelom Temanskim; zaiste zec ich wywleka najmniejsi z tej trzody, zaiste pobudza ich, i przybytki ich.
\par 21 Od grzmotu upadku ich wzruszy sie ziemia; glos i krzyk ich slyszec beda na morzu Czerwonem.
\par 22 Oto jako orzel przypadnie i przyleci, a rozciagnie skrzydla swe nad Bocra, i stanie sie serce mocarzów z Edom dnia onego, jako serce niewiasty bolejacej.
\par 23 Przeciwko Damaszkowi. Zawstydzi sie Emat i Arfad; bo wiesc zla uslysza, i zatrwoza sie, tak ze sie i morze wzruszy, a nie bedzie sie moglo uspokoic.
\par 24 Oslabieje Damaszek, uda sie do uciekania, a strach go ogarnie; uciski i bolesci ogarna go, jako rodzaca.
\par 25 Ale rzeka: Jakozby sie nie mialo ostac miasto slawne, miasto wesela mego?
\par 26 Przetoz upadna mlodziency jego na ulicach jego, a wszyscy mezowie waleczni dnia onego wytraceni beda, mówi Pan zastepów.
\par 27 I rozniece ogien w murze Damaszku, który strawi palace Benadadowe.
\par 28 Przeciwko Kiedar, i przeciwko królestwom Hasor, które wytracic ma Nabuchodonozor, król Babilonski; tak mówi Pan: Wstancie, ciagnijcie przeciwko Kiedar, a zburzcie narody wschodnie.
\par 29 Namioty ich i trzody ich zabiora; opony ich ze wszystkiem naczyniem ich i wielblady ich wezma ze soba, i zawolaja na nich: Strach zewszad.
\par 30 Uciekajcie, rozpierzchnijcie sie predko, zstapcie w glebokosci na mieszkanie, obywatele Hasor! mówi Pan; bo zawarl rade przeciwko wam Nabuchodonozor, król Babilonski, i umyslil przeciwko wam zdrade.
\par 31 Wstancie, ciagnijcie przeciwko narodowi spokojnemu, mieszkajacemu bezpiecznie, mówi Pan: nie ma ani wrót, ani zawór, samotni mieszkaja.
\par 32 Beda zaiste wielblady ich podane na lup, a mnóstwo dobytku ich na korzysc; i rozprosze na wszystkie wiatry tych, którzy i w najostateczniejszych katach mieszkaja, i ze wszystkich stron zle na nich przywiode, mówi Pan.
\par 33 I stanie sie Hasor mieszkaniem smoków, pustynia az na wieki; nie osadzi sie tam nikt, ani mieszkac bedzie w nim syn czlowieczy.
\par 34 Slowo Panskie, które sie stalo do Jeremijasza proroka przeciwko Elamczykom na poczatku królowania Sedekijasza, króla Judzkiego, mówiac:
\par 35 Tak mówi Pan zastepów: Oto ja zlamie luk Elamczyków, najwieksza sile ich;
\par 36 A przywiode przeciwko Elamczykom cztery wiatry ze czterech stron swiata, i rozprosze ich na wszystkie one wiatry, tak, iz nie bedzie narodu, do któregoby sie nie dostali wygnancy z Elam;
\par 37 I zatrwoze Elamczyków przed obliczem nieprzyjaciól ich, i przed obliczem tych, którzy szukaja duszy ich; przywiode, mówie, na nich zle, gniew popedliwosci mojej, mówi Pan, a posylac za nimi bede miecz, dokad ich nie wyniszcze;
\par 38 I postawie stolice moje miedzy Elamczykami, a wytrace stamtad króla i ksiazat, mówi Pan.
\par 39 Wszakze stanie sie, ze w ostateczne dni przywróce zas wiezniów Elam, mówi Pan.

\chapter{50}

\par 1 Slowo, które mówl Pan przeciwko Babilonowi, i przeciwko ziemi Chaldejskiej przez Jeremijasza proroka;
\par 2 Opowiadajcie miedzy narodami, a rozglaszajcie, podniescie choragiew, rozglaszajcie, nie tajcie, mówcie: Wziety bedzie Babilon, pohanbiony bedzie Bel, potarty bedzie Merodach, pohanbione beda balwany jego, pokruszeni beda plugawi bogowie jego.
\par 3 Bo przyciagnie przeciwko niemu naród z pólnocy, który ziemie jego obróci w pustynie, tak, ze nie bedzie, coby mieszkal w niej; tak ludzie jako i bydleta rusza sie i odejda.
\par 4 W onych dniach, i w onych czasach, mówi Pan, przyjda synowie Izraelscy, oni i synowie Judzcy; placzac spolem ochotnie pójda, a Pana, Boga swego, szukac beda.
\par 5 O drodze do Syonu pytac sie beda, a obuciwszy sie tam twarza, rzekna: Pójdzcie, a przylaczcie sie do Pana przymierzem wiecznem, które nie przyjdzie w zapamietanie.
\par 6 Lud mój jest jako trzoda owiec straconych, pasterze ich zawiedli ich w blad, po górach rozegnali ich: z góry na pagórek chodzily, zapomniawszy legowiska swego.
\par 7 Wszyscy, którzy ich znajduja, pozeraja ich, a nieprzyjaciele ich mówia: Nie bedziemy nic winni, przeto, ze zgrzeszyli Panu w przybytku sprawiedliwosci, Panu, który jest nadzieja ojców ich.
\par 8 Uchodzcie z posrodku Babilonu, a z ziemi Chaldejskiej wychodzcie, i bedzcie jako kozly przed trzoda.
\par 9 Bo oto Ja wzbudze i przywiode na Babilon zgromadzenie narodów wielkich z ziemi pólnocnej; którzy sie uszykuja przeciwko niemu, a stamtad go wezma; których strzaly sa jako mocarza osieracajacego, z których zadna sie nie wróci prózno.
\par 10 I bedzie Chaldejska ziemia na lup, a wszyscy, którzy ja zlupia, nasyceni beda, mówi Pan.
\par 11 Przeto, ze sie weselicie, przeto, ze sie radujecie, rozchwytajac dziedzictwo moje, przeto, zescie nabrali ciala jako jalowica utuczona, a wyskakujecie jako mocarze.
\par 12 Zawstydzona bedzie matka wasza bardzo; zaplonie sie rodzicielka wasza; oto najposledniejsza bedzie z narodów, pusta ziemia, sucha i bezdrozna.
\par 13 Dla gniewu Panskiego nie bada w niej mieszkac, ale ona wszystka obróci sie w pustynie; ktokolwiek pójdzie mimo Babilonu, zdumieje sie, i zaswisnie nad wszystkiemi plagami jego.
\par 14 Szykujcie wojska przeciw Babilonowi zewszad wszyscy, co ciagniecie luk, strzelajcie do niego, nie zalujcie strzal; bo przeciwko Panu zgrzeszyl.
\par 15 Wolajcie przeciwko niemu zewszad; poddal sie, upadly grunty jego, skazone sa mury jego; bo pomsta Panska jest. Pomscijciez sie nad nim; jako on czynil innym, tak mu tez uczyncie.
\par 16 Wytraccie siejacego z Babilonu, i tego, który trzyma sierp czasu zniwa; przed ostrzem miecza burzacego kazdy niech sie do ludu swego obróci, i kazdy do ziemi swojej niech uciecze.
\par 17 Izrael jest jako bydladko zagnane, które lwy zaploszyly. Król Assyryjski najpierwszy byl, który go zrec poczal, a ten ostateczny, Nabuchodonozor, król Babilonski, kosci jego pokruszyl.
\par 18 Przetoz tak mówi Pan zastepów, Bóg Izraelski: Oto Ja nawiedze króla Babilonskiego i ziemie jego, jakom nawiedzil króla Assyryskiego;
\par 19 I przywróce zas Izraela do mieszkania jego, a pasc sie bedzie na Karmelu, i na Basanie, i na górze Efraimowej, a w Galaadzie nasycona bedzie dusza jego.
\par 20 W onych dniach i w onych czasach, mówi Pan, beda szukac nieprawosci Izraelowej, ale zadnej nie bedzie; i grzechów Judzkich, ale nie beda znalezione; bo odpuszcze tym, których pozostawie.
\par 21 Wyciagnij przeciwko tej ziemi odpornych, przeciwko niej, mówie, a obywateli jej nawiedz; spustosz a wygladz, goniac ich, mówi Pan; uczynze wedlug wszystkiego, jakoc rozkazuje.
\par 22 Niech bedzie glos wojenny w tej ziemi, i spustoszenie wielkie.
\par 23 Jakozby posiekany i polamany byc mial mlot wszystkiej ziemi? Jakozby sie Babilon stal zdumienie miedzy narodami?
\par 24 Sidlom zastawil na cie, i bedziesz pojmany, o Babilonie! nim sie obaczysz; znaleziony nawet i pojmany bedziesz, poniewazes z Panem zwade zaczal.
\par 25 Otworzyl Pan skarb swój, a wyniósl naczynia gniewu swego; bo to jest sprawa Pana, Pana zastepów w ziemi Chaldejskiej.
\par 26 Pójdzciez przeciwko niej od konczyn ziemi, otwórzciez szpichlerze jej podepczcie ja jako stogi, a wygladzcie ja tak, aby jaj nic nie zostalo;
\par 27 Pozabijajcie wszystkie cielce jej, niech zstepuja na zabicie. Biada im! bo przyszedl dzien ich, czas nawiedzenia ich.
\par 28 Glos uciekajacych, i tych, co uchodza, z ziemi Babilonskiej, aby oznajmili na Syonie pomste Pana, Boga naszego, pomste kosciola jego.
\par 29 Zgromadzcie przeciwko Babilonowi wszystkich strzelców, którzy luk ciagna; polózcie sie obozem przeciw niemu w okolo, aby nikt nie uszedl; oddajcie mu wedlug spraw jego, wedlug wszystkiego, co innym czynil, uczyncie mu; bo sie przeciwko Panu wynosil, przeciwko Swietemu Izraelskiemu.
\par 30 Przetoz polegna mlodziency jego na ulicach jego, i wszyscy mezowie waleczni jego wygladzeni beda dnia onego, mówi Pan.
\par 31 Otom ja przeciwko tobie, o hardy! mówi Pan, Pan zastepów; bo juz przyszedl dzien twój i czas, abym cie nawiedzil.
\par 32 Potknie sie zaiste hardy, i upadnie, a nie bedzie, ktoby go podniósl; i rozniece ogien w miastach jego, który pozre wszystko okolo niego.
\par 33 Tak mówi Pan zastepów: Gwalt sie dzieje synom Izraelskim i synom Judzkim spolem, a wszyscy, którzy ich pojmali, trzymaja ich, nie chca ich puscic.
\par 34 Ale odkupiciel ich mozny, Pan zastepów imie jego, pewnie ze sie ujmie o krzywde ich, aby sprawil pokój tej ziemi, i poruszyl obywateli Babilonskich.
\par 35 Miecz przyjdzie na Chaldejczyków, mówi Pan, i na obywateli Babilonskich, i na ksiazat jego, i na medrców jego;
\par 36 Miecz na klamców, aby zglupieli, miecz na mocarzy jego, aby skruszeni byli;
\par 37 Miecz na konie jego, i na wozy jego, i na wszystko pospólstwo, które jest w posrodku jego, aby byli jako niewasty; miecz na skarby jego, aby byly rozchwycone;
\par 38 Susza na wody jego, aby wyschly; bo ziemia jest pelna obrazów rytych, a przy balwanach swoich szaleja.
\par 39 Przeto tam mieszkac beda bestyje i straszne zwierzeta, mieszkac w nim beda mlode sowy; a nie beda w nim mieszkac wiecej na wieki, i nie beda w nim mieszkac od narodu do narodu.
\par 40 Jako Pan podwrócil Sodome i Gomore, i przyleglosci ich, mówi Pan, tak sie tam nikt nie osadzi, ani bedzie mieszkal w niej syn czlowieczy.
\par 41 Oto lud przyciagnie od pólnocy, i naród wielki, i królowie wielcy wzbudzeni beda ze stron ziemi.
\par 42 Luk i wlóczni pochwyca, okrutnymi beda, a nie zmiluja sie; glos ich jako morze zaszumi, a na koniach pojada, uszykowani jako maz do bitwy przeciwko tobie, o córko Babilonska!
\par 43 Jako uslyszy król Babilonski wiesc o nich, oslabieja rece jego, a ucisk ogarnie go, i bolesc jako rodzaca.
\par 44 Oto aczkolwiek jako lew wystepuje, i bardzej niz nadetosc Jordanu sie podnosi przeciwko przybytkowi mocnego, wszakze go w okamgnieniu wypedze z niej, a tego, który jest wybrany, przeloze nad nia; bo któz jest mnie podobnym; i kto mi da rok? a kto jest tym pasterzem, któryby sie postawil przeciwko mnie?
\par 45 Przetoz sluchajcie rady Panskiej, która uradzil przeciwko Babilonowi; i zamyslów jego, które umyslil przeciwko ziemi Chaldejskiej; zaiste zec ich wywloka najmniejsi z tej trzody, zaiste poburza ich i przybytek ich.
\par 46 Od huku przy dobywaniu Babilonu poruszy sie ta ziemia, a krzyk miedzy narodami slyszany bedzie.

\chapter{51}

\par 1 Tak mówi Pan: Oto Ja wzbudze przeciwko Babilonowi, i przeciwko tym, którzy mieszkaja w posród powstawajacych przeciwko mnie, wiatr zarazliwy;
\par 2 I posle na Babiilon przewiewaczy, którzy go przewiewac beda, i wypróznia ziemie jego, gdyz beda przeciwko niemu zewszad w dzien ucisku.
\par 3 Do tego, który mocno ciagnie luk swój, a postepuje w pancerzu swoim, rzeke: Nie folgujcie mlodziencom jego, wygladzcie wszystko wojsko jego.
\par 4 Niech polegna pobici w ziemi Chaldejskiej, a poprzebijani po ulicach jego.
\par 5 Bo nie jest opuszczony Izrael i Juda od Boga swego, od Pana zastepów, choc ziemia ich pelna jest przestepstwa przeciwko Swietemu Izraelskiemu.
\par 6 Uciekajcie z posrodku Babilonu, a niech zachowa kazdy dusze swoje, abyscie nie byli zatraceni w nieprawosci jego; bo czas bedzie pomsty Panskiej, sam mu zaplate odda.
\par 7 Bylci Babilon kubkiem zlotym w rece Panskiej, upajajacym wszystke ziemie; wino jego pily narody, dlatego poszalaly narody;
\par 8 Ale nagle upadnie Babilon, i starty bedzie; rozkwilcie sie nad nim, nabierzcie olejku balsamowego dla bolesci jego, owa sie wyleczy.
\par 9 Leczylismy Babilon, ale nie jest uleczony. Opuscmyz go, a pójdzmy kazdy do ziemi swej; bo sad jego az do nieba siega, i wyniósl sie az pod obloki.
\par 10 Wywiódl Pan sprawiedliwosci nasze. Pójdzcie, a opowiadajmy na Syonie dzielo Pana, Boga naszego.
\par 11 Wyostrzcie strzaly, sporzadzcie tarcze. Wzbudzil Pan ducha królów Medskich; bo przeciwko Babilonowi zamysl jego, aby go wytracil, gdyz pomsta Panska jest pomsta kosciola jego.
\par 12 Podniescie choragiew na murach Babilonskich, przyczyncie strazy, postawcie strózów, nagotujcie zasadzki; bo i umyslil Pan i wykona, co wyrzekl przeciwko obywatelom Babilonskim.
\par 13 O ty, który mieszkasz nad wodami wielkiemi! o bogaty w skarby! przyszedl koniec twój, kres lakomstwa twego.
\par 14 Przysiagl Pan zastepów na dusze swoje, ze cie napelni ludzmi, jako chrzaszczami, którzy uczynia nad toba okrzyk wojenny.
\par 15 Onci to jest, który uczynil ziemie moca swoja, który utwierdzil okrag swiata madroscia swoja, i roztropnoscia swoja rozpostarl niebiosa;
\par 16 Który gdy glos wypuszcza, wody na niebie szumia, a który sprawuje, aby wystepowaly pary od konczyn ziemi, i blyskawice ze dzdzem przywodzi, a wywodzi wiatr z skarbów swoich.
\par 17 Tak zglupial kazdy czlowiek, ze tego nie zna, ze pohanbiony bywa zlotnik od obrazu rytego; bo klamstwem jest ulanie jego, a niemasz w nich ducha.
\par 18 Marnoscia sa a dzielo bledów; zgina czasu nawiedzenia swego.
\par 19 Nie takowyc jest dzial Jakóbowy; bo on jest który wszystko stworzyl, a Izrael jest pretem dziedzictwa jego; Pan zastepów imie jego.
\par 20 Mlotemes ty moim kruszacym, orezem wojennym, abym pokruszyl przez cie narody, i poburzyl przez cie królestwa;
\par 21 Abym pokruszyl przez cie konia i jezdnego, abym pokruszyl przez cie wóz i tego, co na nim jezdzi;
\par 22 Abym pokruszyl przez cie meza i niewiaste, abym pokruszyl przez cie starca i dziecie, abym pokruszyl przez cie mlodzienca i panne;
\par 23 Abym pokruszyl przez cie pasterza i trzode jego, abym pokruszyl przez cie oracza, i sprzezaj jego, abym pokruszyl przez cie ksiazat i hetmanów.
\par 24 Ale juz oddam Babilonowi, i wszystkim obywatelom Chaldejskim za wszystkie zlosci ich, które czynili Syonowi przed oczyma waszemi, mówi Pan.
\par 25 Otom Ja jest przeciwko tobie, o góro kazaca! mówi Pan, która kazisz wszystke ziemie, i wyciagne reke moje przeciwko tobie, a zwale cie z skal, i uczynie cie góra spalenia;
\par 26 A nie wezma z ciebie kamienia do wegla, ani kamienia do gruntów; bo pustynia wieczna bedziesz, mówi Pan.
\par 27 Podniescie choragiew w ziemi, trabcie w trabe miedzy narodami, zgotujcie przeciwko niemu narody, zwolajcie przeciwko niemu królestwa Ararad, Minny, i Aschenas; postanówcie przeciwko niemu hetmana, przywiedzcie konie jako chrzaszcze najezone;
\par 28 Zgotujcie przeciwko niemu narody, królów Medskich, ksiazat ich, i wszystkich hetmanów ich, ze wszystka ziemia wladzy ich;
\par 29 Tedy zadrzy ziemia, i rozboleje sie, gdy wykonane beda przeciwko Babilonowi mysli Panskie, aby obrócil ziemie Babilonska w pustynie, aby zostala bez obywatela.
\par 30 Przestana mocarze Babilonscy walczyc, usiada w zamkach, ustanie mestwo ich, beda jako niewiasty; zapali mieszkania ich, pokruszone beda zawory ich.
\par 31 Goniec spotka drugiego gonca, a posel zabiezy poslowi, aby opowiedzieli królowi Babilonskiemu, iz wzieto miasto jego z jednej strony,
\par 32 A iz brody ubiezono, i jeziora wypalano ogniem, a mezowie waleczni ustraszeni sa.
\par 33 Bo tak mówi Pan zastepów, Bóg Izraelski: Córka Babilonska jest jako bojewisko, czas deptania jej przyszedl; jeszcze maluczko, a przyjdzie czas zniwa jej.
\par 34 Pozarl mie, potarl mie Nabuchodonozor, król Babilonski, uczynil mie naczyniem próznem, polknal mie jako smok, napelnil brzuch swój rozkoszami mojemi, i wygnal mie.
\par 35 Gwalt mnie i cialu memu uczyniony niech przyjdzie na Babilon, mówi obywatelka Syonska, a krew moja na obywateli Chaldejskich, mówi Jeruzalem.
\par 36 Przetoz tak mówi Pan: Oto sie Ja zastawie o krzywde twoje, a pomszcze sie za cie; bo wysusze morze jego, wysusze i zródla jego.
\par 37 I bedzie Babilon obrócony w mogily, w mieszkanie smoków, w zdumienie, i w poswistanie, i bedzie bez obywatela.
\par 38 Pospolu jako lwy ryczec beda, a skomlec jako szczenieta lwie.
\par 39 Gdy sie zapala, uczynie im uczte, i tak ich upoje, ze krzyczec i snem wiecznym zasnac musza, tak, aby nie ocucili, mówi Pan.
\par 40 Powiode ich jako baranki ku zabiciu, jako barany i kozly.
\par 41 Jakozby dobyty mógl byc Sesach? Jakozby wzieta byc mogla chwala wszystkiej ziemi? Jakozby mógl przyjsc na spustoszenie Babilon miedzy narodami?
\par 42 Wystapi przeciwko Babilonowi morze, mnóstwem walów jego okryte bedzie.
\par 43 Miasta jego beda spustoszeniem, ziemia sucha i pusta, ziemia, w której miastach nikt nie bedzie mieszkal, ani bedzie chodzil przez nia syn czlowieczy.
\par 44 Nawiedze tez Bela w Babilonie, i wydre, co byl polknal, z geby jego; i nie beda sie wiecej do niego zbiegac narody, i mury takze Babilonskie upadna.
\par 45 Wyjdzcie z posrodku jego, ludu mój! a wybaw kazdy dusze swoje przed gniewem zapalczywosci Panskiej.
\par 46 A nie badzcie miekkiego serca, ani sie lekajcie wiesci, która bedzie slychac w tej ziemi; gdy przyjdzie jednego roku nowina, potem drugiego roku wiesc i gwalt w ziemi, a pan na pana.
\par 47 Przetoz oto dni przyjda, w które Ja nawiedze balwany ryte Babilonskie, a wszystka ziemia jego pohanbiona bedzie, i wszyscy pobici jego polegna w posrodku niego.
\par 48 I beda nad Babilonem spiewac niebiosa i ziemia, i wszystko, co na nich jest, gdy z pólnocy przyjda nan pustoszyciele, mówi Pan.
\par 49 Jako Babilon porazil onych pobitych Izraelskich, tak z Babilonu polegna pobici po wszystkiej ziemi.
\par 50 O którzyscie uszli miecza, idzcie, nie stójcie! wspominajcie z daleka na Pana, a Jeruzalem niech wstepuje na serce wasze.
\par 51 Rzeczcie: Wstydzimy sie, ze slyszymy uraganie; hanba okryla twarzy nasze, bo cudzoziemcy wchodza do swiatnic domu Panskiego.
\par 52 Przetoz oto dni przychodza, mówi Pan, ze nawiedze ryte balwany jego, a po wszystkiej ziemi jego zraniony stekac bedzie.
\par 53 Chociazby Babilon wstapil na niebo, i obwarowal na wysokosci moc swoje, przecie odemnie przyjda pustoszyciele jego, mówi Pan.
\par 54 Glos wolania z Babilonu, a starcie wielkie z ziemi Chaldejskiej;
\par 55 Bo Pan Babilon zburzy i wytraci z niego glos wielki, chocby huczaly waly ich jako wody wielkie, i wydany byl szum glosu ich.
\par 56 Gdy nan, to jest na Babilon, pustoszyciel przyciagnie pojmani beda mocarze jego, pokruszone beda luki ich; bo Bóg nagrody, Pan nagrodzi im sowicie;
\par 57 Opoi ksiazat jego i medrców jego, wodzów jego, i urzedników jego, i mocarzy jego, aby zasneli snem wiecznym, a nie ocucili, mówi król, Pan zastepów imie jego.
\par 58 Tak mówi Pan zastepów: Mur Babilonski szeroki do gruntu zburzony bedzie, i bramy jego wysokie ogniem spalone beda, a ludzie darmo pracowac beda, a narody przy ogniu pomdleja.
\par 59 Toc jest slowo, które rozkazal Jeremijasz prorok Sarajaszowi, synowi Neryjasza, syna Maasejaszowego, gdy odszedl od Sedekijasza, króla Judzkiego, do Babilonu, roku czwartego królowania jego; (a Sarajasz byl ksiazeciem w Menucha.)
\par 60 Gdy zapisal Jeremijasz wszystko zle, które przyjsc mialo na Babilon, w ksiegi jedne, wszystkie te slowa, które sa napisane przeciwko Babilonowi.
\par 61 I rzekl Jeremijasz do Sarajasza: Gdy przyjdziesz do Babilonu, i ogladasz go, tedy przeczytasz te wszystkie slowa,
\par 62 A rzeczesz: O Panie! tys mówil przeciwko miejscu temu, ze je wytracisz, aby w niem nie mieszkal nikt, ani z ludzi ani z bydlat, ale zeby bylo pustkami wiecznemi.
\par 63 A gdy do konca przeczytasz te ksiegi, przywiazesz do nich kamien, i wrzucisz je w posród Eufratesa,
\par 64 A rzeczesz: Tak zatopiony bedzie Babilon, a nie powstanie wiecej z tego zlego, które Ja nan przywiode, choc ustawac beda. Az dotad slowa Jeremijaszowe.

\chapter{52}

\par 1 Dwadziescia i jeden lat mial Sedekijasz, gdy królowac poczal, a jedenascie lat królowal w Jeruzalemie; a imie matki jego bylo Chamutal, córka Jeremijaszowa z Lebny;
\par 2 I czynil zlosc przed oczyma Panskiemi wedlug wszystkiego, co czynil Joakim.
\par 3 Albowiem sie to stalo dla rozgniewania Panskiego przeciwko Jeruzalemowi i Judzie, az ich odrzucil od twarzy swej. Wtem zasie odstapil Sedekijasz od króla Babilonskiego.
\par 4 I stalo sie roku dziewiatego królestwa jego, miesiaca dziesiatego, dnia dziesiatego tegoz miesiaca, ze przyciagnal Nabuchodonozor, król Babilonski, on i wszystko wojsko jego przeciwko Jeruzalemowi, i polozyl sie obozem u niego, i porobil przeciwko niemu szance w okolo.
\par 5 A tak bylo miasto oblezone az do jedenastego roku króla Sedekijasza.
\par 6 Tedy miesiaca czwartego, dziewiatego dnia tegoz miesiaca, byl wielki glód w miescie, i nie mial chleba lud onej ziemi.
\par 7 I przelamano mur miejski, a wszyscy ludzie rycerscy pouciekali, i wyszli z miasta w nocy droga do bramy, która jest miedzy dwoma murami podle ogrodu królewskiego; (ale Chaldejczycy lezeli okolo miasta,)i poszli droga ku pustyni.
\par 8 I gonilo wojsko Chaldejskie króla, a doscigneli Sedekijasza na polach u Jerycha, a wszystko wojsko jego rozpierzchnelo sie od niego.
\par 9 A tak pojmawszy króla przywiedli go do króla Babilonskiego do Ryblaty w ziemi Eamat, kedy o nim uczynil sad.
\par 10 I pozabijal król Babilonski synów Sedekijaszowych przed oczyma jego, takze tez wszystkich ksiazat Judzkich pozabijal w Ryblacie.
\par 11 A Sedekijasza oslepiwszy i zwiazawszy go lancuchami miedzianemi, zawiódl go król Babilonski do Babilonu, i podal go do domu wiezienia az do smierci jego.
\par 12 Potem miesiaca piatego, dnia dziesiatego tegoz miesiaca, ten jest rok dziewietnasty królowania Nabuchodonozora, króla Babilonskiego, przyciagnal Nabuzardan, hetman zolnierski, który stawal przed królem Babilonskim, do Jeruzalemu.
\par 13 I spalil dom Panski, i dom królewski, i wszystkie domy Jeruzalemskie; owa wszystko budowanie kosztowne popalil ogniem.
\par 14 I wszystkie mury Jeruzalemskie w okolo rozwalilo wszystko wojsko Chaldejskie, które bylo z onym hetmanem zolnierskim.
\par 15 A z ubogich ludzi i z ostatku pospólstwa, które bylo pozostalo w miescie i zbiegów, którzy byli zbiegli do króla babilonskiego, i inne pospólstwo przeniósl Nabuzardan, hetman zolnierski.
\par 16 Tylko z ubogich onej ziemi zostawil Nabuzardan, hetman zolnierski, aby byli winiarzami i oraczami.
\par 17 Nadto slupy miedziane, które byly w domu Panskim, i podstawki, i morze miedziane, które bylo w domu Panskim, potlukli Chaldejczycy, i przeniesli wszystke miedz ich do Babilonu;
\par 18 Kotly tez i lopaty, i naczynia muzyczne, i miednice, i czasze, i wszystko naczynie miedziane, którem uslugiwano, pobrali;
\par 19 Nadto wiadra, i kadzielnice, i miednice, i garnce, i swieczniki, i czaszki, i kufle, co bylo zlotego w zlocie, a co bylo srebrnego w srebrze, pobral hetman zolnierski;
\par 20 Slupy dwa, morze jedno, i wolów miedzianych dwanascie, które byly pod podstawkami, które byl sprawil król Salomon w domu Panskim; nie bylo wagi miedzi onego wszystkiego naczynia.
\par 21 A z tych slupów osmnascie lokci wzwyz byl slup jeden, a w miesz w okolo dwanascie lokci, a w miazszosc jego cztery palce, a wewnatrz byl dety;
\par 22 A galka na nim miedziana, a wysokosc galki jednej byla na piec lokci, siatka tez i jablka granatowe na galce w okolo wszystko miedziane; taki tez byl i drugi slup z jablkami granatowemi;
\par 23 A bylo jablek granatowych dziewiecdziesiat i szesc po kazdej stronie; wszystkich jablek granatowych bylo po sto na siatce w okolo.
\par 24 Wzial tez hetman zolnierski Sarajego, kaplana przedniego, i Sofonijasza, kaplana wtórego po nim, i trzech strózów progu.
\par 25 Wzial tez z miasta dworzanina niektórego, który byl przelozonym nad ludem rycerskim, i siedmiu mezów z tych, którzy stawali przed królem, którzy sie znalezli w miescie, i pisarza przedniego wojskowego, który spisywal wojsko z ludu ziemi, i szescd ziesiat mezów z ludu ziemi, którzy sie znalezli w miescie.
\par 26 Wziawszy ich tedy Nabuzardan, hetman zolnierski, zawiódl ich do króla Babilonskiego do Ryblaty;
\par 27 I pobil ich król Babilonski, a pomordowal ich w Ryblacie w ziemi Emat. A tak przeniesiony jest Juda z ziemi swojej.
\par 28 Tenci jest lud, który zaprowadzil Nabuchodonozor roku siódmego: Zydów trzy tysiace, i dwadziescia i trzy.
\par 29 Roku osmnastego Nabuchodonozora, zaprowadzil z Jeruzalemu dusz osm set, trzydziesci i dwie.
\par 30 Roku dwudziestego i trzeciego Nabuchodonozora; zaprowadzil Nabuzardan, hetman zolnierski, z Zydów dusz siedm set, czterdziesci i piec; wszystkich dusz cztery tysiace i szesc set.
\par 31 A trzydziestego i siódmego roku, po pojmaniu Joachyna, króla Judzkiego, dwunastego miesiaca, dwudziestego i piatego dnia tegoz miesiaca, wywyzszyl Ewilmerodach, król Babilonski, tego roku, gdy poczal królowac, glowe Joachyna, króla Judzkiego, uwol niwszy go z domu wiezienia;
\par 32 I rozmawial z nim laskawie, i wystawil stolice jego nad stolice królów, którzy byli z nim w Babilonie.
\par 33 Odmienil tez i odzienie, w którem byl w wiezieniu, i jadal chleb zawsze przed obliczem jego po wszystkie dni zywota swego.
\par 34 Obrok tez jemu naznaczony, obrok ustawiczny dawano mu od króla Babilonskiego na kazdy dzien az do smierci jego, po wszystkie dni zywota jego.


\end{document}