\begin{document}

\title{Daniela}


\chapter{1}

\par 1 Roku trzeciego królowania Joakima, króla Judzkiego, przyciagnal Nabuchodonozor, król Babilonski, do Jeruzalemu, i oblegl je.
\par 2 I podal Pan w reke jego Joakima, króla Judzkiego, i czesc naczynia domu Bozego, który je zawiózl do ziemi Senaar, do domu boga swego, i wniósl ono naczynie do domu skarbu boga swego.
\par 3 I rozkazal król Aspenasowi przelozonemu nad komornikami swymi, aby przyprowadzil z synów Izraelskich, i z nasienia królewskiego i z ksiazat,
\par 4 Mlodzienców, na którychby nie bylo zadnej zmazy, a pieknych na wejrzeniu i dowcipnych do wszelakiej madrosci, i sposobnych do umiejetnosci, i dostapienia jej, i którzyby byli sposobni, aby stali w palacu królewskim, a uczyli sie pisma i jezyka chaldejskiego.
\par 5 I postanowil im król obrok na kazdy dzien z potraw swoich królewskich i z wina, które on sam pijal, a zeby ich tak chowal przez trzy lata, a po wyjsciu ich zeby stawali przed obliczem królewskiem.
\par 6 A byli miedzy nimi z synów Judzkich: Danijel, Ananijasz, Misael, i Azaryjasz.
\par 7 I dal im przelozony nad komornikami imiona, a Danijela nazwal Baltazarem, a Ananijasza Sadrachem, a Misaela Mesachem, a Azaryjasza Abednegiem.
\par 8 Ale Danijel postanowil w sercu swojem, zeby sie nie mazal pokarmem potraw królewskich, ani winem, które król pijal; przetoz tego szukal u przelozonego nad komornikami, zeby sie nie zmazal.
\par 9 I zjednal Bóg Danijelowi laske i milosc przed obliczem przelozonego nad komornikami.
\par 10 I rzekl przelozony nad komornikami do Danijela: Ja sie boje króla, pana mego, który wam postanowil pokarm wasz i napój wasz: który jezliby obaczyl, ze twarzy wasze chudsze sa niz innych mlodzienców, którzy jednako z wami maja byc wychowani, tedy mie przyprawicie o gardlo u króla.
\par 11 I rzekl Danijel do slugi, którego byl postanowil przelozony nad komornikami nad Danijelem, Ananijaszem, Misaelem i Azaryjaszem:
\par 12 Doswiadcz prosze slug twoich przez dziesiec dni, a niech nam dadza jarzyn, którebysmy jedli, i wody, którabysmy pili.
\par 13 Potem przypatrzysz sie twarzom naszym, i twarzom innych mlodzienców, którzy jadaja pokarm z potraw królewskich, a jako obaczysz, tak uczynisz z slugami twymi.
\par 14 I usluchal ich w tem, a doswiadczyl ich przez dziesiec dni.
\par 15 A po wyjsciu dziesieciu dni okazalo sie, ze twarze ich byly piekniejsze, i byli tlustsi na ciele, niz wszyscy mlodziency, którzy jadali pokarm z potraw królewskich.
\par 16 Przetoz on sluga bral on obrok potraw ich, i wino napoju ich, a dawal im jarzyny.
\par 17 A onym czterem mlodziencom dal Bóg umiejetnosc i rozum we wszelakiem pismie i madrosci; nadto Danijelowi dal wyrozumienie wszelakiego widzenia i snów.
\par 18 A gdy wyszly dni, po których ich król przyprowadzic rozkazal, przywiódl ich przelozony nad komornikami przed twarz Nabuchodonozora.
\par 19 I mówil z nimi król; ale nie byl znaleziony miedzy onymi wszystkimi, jako Danijel, Ananijasz, Misael i Azaryjasz; i stawali przed obliczem królewskiem.
\par 20 A w kazdem slowie madrosci i rozumu, o które sie ich król pytal, znalazl ich dziesiec kroc bieglejszych nad wszystkich medrców i praktykarzy, którzy byli we wszystkiem królewstwie jego.
\par 21 I byl tam Danijel az do roku pierwszego króla Cyrusa.

\chapter{2}

\par 1 Roku wtórego królowania Nabuchodonozora mial Nabuchodonozor sen, i strwozyl sie duch jego, i przerwal mu sie sen jego.
\par 2 Tedy król rozkazal zwolac medrców, i praktykarzy, i czarnoksiezników, i Chaldejczyków, aby oznajmili królowi sen jego; którzy przyszli i staneli przed obliczem królewskiem.
\par 3 I rzekl król do nich: Mialem sen, i strwozyl sie duch mój, tak, ze nie wiem, co mi sie snilo.
\par 4 Tedy odpowiedzieli Chaldejczycy królowi po syryjsku: Królu, zyj na wieki! Powiedz sen slugom twoim, a oznajmiemyc wyklad jego.
\par 5 Odpowiedzial król i rzekl do Chaldejczyków: Ta rzecz juz mi z pamieci wypadla; jezli mi nie oznajmicie snu i wykladu jego, na sztuki rozsiekani bedziecie, a domy wasze w gnojowisko obrócone beda;
\par 6 Ale jezli mi sen i wyklad jego oznajmicie, dary i upominki i uczciwosc wielka odniesiecie odemnie; przetoz sen i wyklad jego oznajmijcie mi.
\par 7 Odpowiedzieli powtóre, i rzekli: Król niech powie sen slugom swoim, a wyklad jego oznajmiemy.
\par 8 Odpowiedzial król, i rzekl: Zaiste wiem, ze umyslnie odwlaczacie, baczac, ze mi ten sen z pamieci wyszedl.
\par 9 Jezlize mi snu nie oznajmicie, pewny jest o was dekret, boscie rzecz klamliwa i przewrotna umyslili mówic przedemna, azby czas przeminal; przetoz mi sen powiedzcie, a dowiem sie, bedziecieli mogli wyklad jego oznajmic.
\par 10 Odpowiedzieli Chaldejczycy królowi, i rzekli: Niemasz czlowieka na ziemi, któryby te rzecz królowi oznajmic mógl; dotego zaden król, ksiaze albo pan o taka sie rzecz nie pytal zadnego medrca, i praktykarza, i Chaldejczyka.
\par 11 Bo rzecz, o która sie król pyta, trudna jest, a niemasz nikogo, coby ja mógl królowi oznajmic, oprócz bogów, którzy nie mieszkaja z ludzmi.
\par 12 Z tej przyczyny zasrozyl sie król, i rozgniewal sie bardzo, a rozkazal wytracic wszystkich medrców Babilonskich.
\par 13 A gdy wyszedl dekret, aby mordowano medrców, szukano i Danijela i towarzyszów jego, aby ich zamordowano.
\par 14 tedy Danijel odpowiedzial madrze i roztropnie Aryjochowi, hetmanowi nad zolnierzami królewskimi, który wyszedl, aby zabijal medrców Babilonskich;
\par 15 A odpowiadajac rzekl do Aryjocha, hetmana królewskiego: Przecz ten dekret tak predko wyszedl od króla? I oznajmil te rzecz Aryjoch Danijelowi.
\par 16 Skad Danijel wszedl, i prosil króla, aby mu dal czas na oznajmienie wykladu królowi.
\par 17 Odszedlszy tedy Danijel do domu swego, oznajmil te rzecz Ananijaszowi, Misaelowi i Azaryjaszowi, towarzyszom swoim,
\par 18 Aby o milosierdzie prosili Boga niebieskiego dla tej tajemnicy, zeby nie zgineli Danijel i towarzysze jego z pozostalymi medrcami Babilonskimi.
\par 19 Tedy objawiona jest Danijelowi w widzeniu nocnem ta tajemnica, za co Danijel blogoslawil Bogu niebieskiemu.
\par 20 A mówiac Danijel rzekl: Niech bedzie imie Boze blogoslawione od wieku az na wieki; albowiem madrosc i moc jego jest;
\par 21 On sam odmienia czasy i chwile; zrzuca królów i stanowi królów; daje madrosc madrym, a umiejetnym rozum;
\par 22 On odkrywa rzeczy glebokie i skryte, zna, co jest w ciemnosciach, a swiatlosc z nim mieszka.
\par 23 Ciebie ja, o Boze ojców moich! wyslawiam i chwale, zes mi dal madrosc i moc, owszem, zes mi teraz oznajmil to, o cosmy cie prosili; bos nam sen królewski oznajmil.
\par 24 Dla tego Danijel wszedl do Aryjocha, którego byl postanowil król, aby wytracil medrców Babilonskich; a przyszedlszy tak rzekl do niego: Nie trac medrców Babilonskich, wprowadz mie do króla, a ja ten wyklad królowi oznajmie.
\par 25 Tedy Aryjoch z kwapieniem wprowadzil Danijela do króla i tak mu rzekl: Znalazlem meza z wiezniów synów Judzkich, który ten wyklad królowi oznajmi.
\par 26 Odpowiedzial król, i rzekl Danijelowi, któremu imie bylo Baltazar: Izali mnie ty mozesz oznajmic sen, którym widzial, i wyklad jego?
\par 27 Odpowiedzial Danijel królowi, i rzekl: Tajemnicy, o której sie król pyta, medrcy, praktykarze, czarnoksieznicy i wieszczkowie królowi oznajmic nie moga;
\par 28 A wszakze jest Bóg na niebie, który objawia tajemnice, a on okazal królowi Nabuchodonozorowi, co ma byc potomnych dni. Sen twój i widzenia, któres widzial na lozu twojem, te sa:
\par 29 Tobie o królu! przychodzilo na mysl na lozu twojem, coby mialo byc na potem, a ten, który odkrywa tajemnice, oznajmil ci to, co ma byc.
\par 30 Mnie tez nie przez madrosc, któraby przy mnie byla nad wszystkich ludzi, tajemnica ta objawiona jest, ale przez modlitwe, aby ten wyklad królowi oznajmiony byl, a izbys sie mysli serca twego dowiedzial.
\par 31 Tys, królu! widzial, a oto obraz jeden wielki(obraz to byl wielki, a blask jego znaczny)stal przeciwko tobie, który na wejrzeniu byl straszny.
\par 32 Tego obrazu glowa byla ze zlota szczerego, piersi jego i ramiona jego ze srebra, brzuch jego i biodra jego z miedzi;
\par 33 Golenie jego z zelaza, nogi jego czescia z zelaza, a czescia z gliny.
\par 34 Patrzales na to, az odciety byl kamien, który nie bywal w reku, a uderzyl ten obraz w nogi jego zelazne i gliniane, i skruszyl je.
\par 35 Tedy sie skruszylo spolem zelazo, glina, miedz, srebro i zloto, a bylo to wszystko jako plewy na bojewisku w lecie, i rozniósl to wiatr, tak, ze ich na zadnem miejscu nie znaleziono; a kamien on, który uderzyl o obraz, stal sie góra wielka i napelnil wszystke ziemie.
\par 36 Tenci jest sen. Wyklad tez jego powiemy przed królem:
\par 37 Tys, królu! królem królów; bo tobie Bóg niebieski królestwo, moc, potege i slawe dal;
\par 38 I wszystko, gdzie jedno mieszkaja synowie ludzcy, zwierz polny i ptastwo niebieskie, dal w reke twoje i postanowil cie panem nad tem wszystkiem, a tys jest ta glowa zlota.
\par 39 Ale po tobie powstanie królestwo insze, podlejsze nizeli twoje, a inne królestwo trzecie miedziane, które panowac bedzie po wszystkiej ziemi.
\par 40 A królestwo czwarte bedzie mocne jako zelazo; bo jako zelazo lamie i kruszy wszystko, jako zelazo, mówie, kruszy wszystko, tak i ono polamie i pokruszy wszystko.
\par 41 A izes widzial nogi, i palce czescia z gliny garncarskiej a czescia z zelaza, królestwo rozdzielone znaczy, w którem bedzie nieco mocy zelaznej, tak jakos widzial zelazo zmieszane z skorupa gliniana;
\par 42 Ale palce nóg czescia z zelaza a czescia z gliny znacza królestwo czescia mocne a czescia do skruszenia snadne.
\par 43 A izes widzial zelazo zmieszane z skorupa gliniana, znaczy, ze sie spokrewnia z soba ludzie; a wszakze nie bedzie sie trzymal jeden drugiego, tak jako zelazo nie moze sie zmieszac z glina.
\par 44 Ale za dni tych królów wzbudzi Bóg niebieski królestwo, które na wieki zepsute nie bedzie, a królestwo to na inszy naród nie spadnie, ale ono polamie, i koniec uczyni tym wszystkim królestwom, a samo stac bedzie na wieki.
\par 45 Tak jakos widzial, iz z góry odciety byl kamien, który nie bywal w reku, a skruszyl zelazo, miedz, gline, srebro i zloto, przez to Bóg wielki królowi oznajmil, co ma byc na potem; i prawdziwy jest ten sen, i wierny wyklad jego.
\par 46 Tedy król Nabuchodonozor padl na oblicze swoje, i uklonil sie Danijelowi, a rozkazal, aby mu ofiare i kadzenia ofiarowali.
\par 47 Tedy odpowiadajac król Danijelowi rzekl: Zaprawde Bóg wasz jest Bogiem bogów, a Panem królów, który odkrywa tajemnice, poniewazes mógl objawic te tajemnice.
\par 48 Zatem król zacnie wywyzszyl Danijela, i darów wielkich i wiele dal mu, i uczynil go panem nad wszystka kraina Babilonska, i ksiazeciem nad przelozonymi i nad wszystkimi medrcami Babilonskimi.
\par 49 Ale Danijel prosil króla, aby przelozyl nad sprawami krainy Babilonskiej Sadracha, Mesacha i Abednego; a Danijel bywal w bramie królewskiej.

\chapter{3}

\par 1 Nabuchodonozor król uczynil obraz zloty, którego wysokosc byla na szescdziesiat lokci, a szerokosc jego na szesc lokci, i postawil go na polu Dura w krainie Babilonskiej.
\par 2 Tedy król Nabuchodonozor poslal, aby zebrano ksiazat, starostów i hetmanów, starszych, poborców, w prawach bieglych, urzedników, i wszystkich przelozonych nad krainami, aby przyszli na poswiecenie obrazu, który byl wystawil król Nabuchodonozor.
\par 3 Tedy sie zgromadzili ksiazeta, starostowie i hetmani, starsi, poborcy, w prawach biegli, urzednicy, i wszyscy przelozeni nad krainami ku poswiecaniu obrazu, który byl wystawil Nabuchodonozor król, i staneli przed obrazem, który byl wystawil Nabuch odonozor.
\par 4 A wozny wolal wielkim glosem: Wam sie opowiada, ludziom, narodom, i jezykom;
\par 5 Skoro uslyszycie glos traby, piszczalki, lutni, skrzypiec, harfy, symfonalu, i wszelakiego instrumentu muzyki, upadnijcie a klaniajcie sie obrazowi zlotemu, który wystawil król Nabuchodonozor;
\par 6 A ktoby nie upadl i nie poklonil sie, tejze godziny wrzucony bedzie w posród pieca ogniem palajacego.
\par 7 Zaraz tedy, skoro uslyszeli wszyscy ludzie glos traby, piszczalki, lutni, skrzypiec, harfy i wszelakiego instrumentu muzyki, upadli wszyscy ludzie, narody i jezyki, klaniajac sie obrazowi zlotemu, który wystawil król Nabuchodonozor.
\par 8 Przetoz tego czasu przystapiwszy mezowie Chaldejscy skarge uczynili przeciwko Zydom;
\par 9 A mówiac rzekli do króla Nabuchodonozora: Królu, zyj na wieki!
\par 10 Ty, królu! uczyniles dekret, zeby kazdy czlowiek, któryby uslyszal glos traby, piszczalki, lutni, skrzypiec, harfy, i symfonalu i wszelakiego instrumentu muzyki, upadl i poklonil sie obrazowi zlotemu;
\par 11 A ktobykolwiek nie upadl i nie poklonil sie, aby byl wrzucony w posrodek pieca ogniem palajacego.
\par 12 Wszakze sie znalezli niektórzy Zydowie, któryches przelozyl nad sprawami krainy Babilonskiej, Sadrach, Mesach i Abednego; ci mezowie lekce powazyli, o królu! dekret twój, bogów twoich nie chwala, i obrazowi sie zlotemu, którys wystawil, nie klaniaja.
\par 13 Tedy Nabuchodonozor w popedliwosci i w gniewie rozkazal przyprowadzic Sadracha, Mesacha i Abednego, których wnet przywiedziono przed króla.
\par 14 I mówil Nabuchodonozor a rzekl im: Umyslniez wy, Sadrachu, Mesachu, i Abednego, bogów moich nie czcicie, i obrazowi sie zlotemu, którym wystawil, nie klaniacie?
\par 15 Teraz tedy wy badzcie gotowi, abyscie zaraz, skoro uslyszycie glos traby, piszczalki, lutni, skrzypiec, harfy i symfonalu i wszelakiego instrumentu muzyki, upadli, i poklonili sie temu obrazowi, którym uczynil; a oto jezli sie nie poklonicie, tej ze godziny bedziecie wrzuceni w posród pieca ogniem palajacego; a któryz jest ten Bóg, coby was wyrwal z reki mojej?
\par 16 Odpowiedzieli Sadrach, Mesach i Abednego, i rzekli do króla: O Nabuchodonozorze! my sie nie frasujemy o to, cobysmy mieli odpowiedziec;
\par 17 Bo oto lubo Bóg nasz, którego my chwalimy, (który mocen jest wyrwac nas z pieca ogniem palajacego, i z reki twojej, o królu!) wyrwie nas,
\par 18 Lubo nie wyrwie, niech ci bedzie wiadomo, o królu! ze bogów twoich chwalic, i obrazowi zlotemu, którys wystawil, klaniac sie nie bedziemy.
\par 19 Tedy Nabuchodonozor pelen bedac popedliwosci, tak, ze sie ksztalt twarzy jego odmienil przeciw Sadrachowi, Mesachowi i Abednegowi, odpowiadajac rozkazal piec rozpalic siedm kroc bardziej, nizeli byl zwyczaj rozpalac go,
\par 20 A mezom co mocniejszym, którzy byli w wojsku jego, rozkazal, aby zwiazawszy Sadracha, Mesacha i Abednega, wrzucili do pieca ogniem palajacego.
\par 21 Tedy onych mezów zwiazano w plaszczach ich, w ubraniach ich i w czapkach ich i w szatach ich, a wrzucono ich w posrodek pieca ogniem palajacego.
\par 22 A iz rozkazanie królewskie przynaglalo, a piec bardzo byl rozpalony, dlatego onych mezów, którzy wrzucili Sadracha, Mesacha i Abednega, zadusil plomien ogniowy.
\par 23 Ale ci trzej mezowie, Sadrach, Mesach i Abednego, wpadli w posród pieca ogniem palajacego zwiazani.
\par 24 Tedy król Nabuchodonozor zdumial sie i powstal predko, a mówiac rzekl hetmanom swoim: Izalismy nie trzech mezów zwiazanych wrzucili w posród ognia? którzy odpowiadajac rzekli królowi: Prawda, królu!
\par 25 A on odpowiadajac rzekl: Oto Ja widze czterech mezów rozwiazanych przechodzacych sie w posrodku ognia, a niemasz zadnego naruszenia przy nich, a osoba czwartego podobna jest Synowi Bozemu.
\par 26 Tedy przystapiwszy Nabuchodonozor do czelusci pieca ogniem palajacego, rzekl mówiac: Sadrachu, Mesachu i Abednegu, sludzy Boga najwyzszego! wynijdzcie, a przyjdzcie sam; i wyszli Sadrach, Mesach i Abednego z posrodku ognia.
\par 27 A zgromadziwszy sie ksiazeta, starostowie urzednicy i hetmani królewscy ogladali onych mezów, ze nie panowal ogien nad cialami ich, i wlos glowy ich nie opalil sie, i plaszcze ich nie naruszyly sie, ani zapach ognia nie przeszedl przez nich.
\par 28 Tedy rzekl Nabuchodonozor, mówiac: Blogoslawiony Bóg ich, to jest, Sadracha, Mesacha i Abednega, który poslal Aniola swego, a wyrwal slug swoich, którzy ufali w nim, którzy slowa królewskiego nie usluchali, ale ciala swe wydali, aby nie sluzyli, a nie klaniali sie zadnemu bogu, oprócz Boga swego.
\par 29 Przetoz ja daje taki dekret, aby kazdy ze wszelkiego ludu, narodu, i jezyka, ktobykolwiek bluznierstwo wyrzekl przeciwko Bogu Sadrachowemu, Mesachowemu i Abednegowemu, byl na sztuki rozsiekany, a dom jego w gnojowisko obrócony, gdyz niemasz Boga innego, któryby mógl wyrwac, jako ten.
\par 30 Tedy król zacnie wywyzszyl Sadracha, Mesacha i Abednega w krainie Babilonskiej.

\chapter{4}

\par 1 Nabuchodonozor król, wszystkim ludziom, narodom, i jezykom, którzy mieszkaja po wszystkiej ziemi: Pokój sie wam niech rozmnozy!
\par 2 Znaki i dziwy, które uczynil ze mna Bóg najwyzszy, zdalo mi sie za rzecz przystojna opowiedziec.
\par 3 O jakoz sa wielkie znaki jego! a dziwy jego jako mocne! królestwo jego królestwo wieczne, i wladza jego od narodu do narodu.
\par 4 Ja Nabuchodonozor zyjac w pokoju w domu moim, i kwitnac na palacu moim,
\par 5 Mialem sen, który mie przestraszyl, i mysli, którem mial na lozu mojem, a widzenia, którem widzial, zatrwozyly mie.
\par 6 A przetoz wydany jest ode mnie dekret, aby przywiedziono przed mie wszystkich medrców Babilonskich, którzyby mi wyklad snu tego oznajmili.
\par 7 Tedy przyszli medrcy i praktykarze Chaldejscy, i wieszczkowie; i powiedzialem im sen, a wszakze wykladu jego nie mogli mi oznajmic;
\par 8 Az na ostatek przyszedl przed mie Danijel, którego imie Baltazar wedlug imienia boga mego, a w którym jest duch bogów swietych, a sen powiedzialem przed nim,
\par 9 Mówiac: Baltazarze, przedniejszy z medrców! Ja wiem, iz duch bogów swietych jest w tobie, a zadna tajemnica nie jest ci trudna; widzenia snu mego, którym mial, posluchaj, a wyklad jego powiedz mi.
\par 10 Te sa widzenia, którem widzial na lozu mojem: Widzialem, a oto drzewo w posrodku ziemi, którego wysokosc zbytnia byla.
\par 11 Wielkie bylo ono drzewo i mocne, a wysokosc jego dosiegala nieba, a okazale bylo az do granic wszystkiej ziemi;
\par 12 Galezie jego piekne, a owoc jego obfity, i pokarm dla wszystkich byl na niem; pod soba dawalo cien zwierzowi polnemu, a na galeziach jego mieszkalo ptastwo niebieskie, a z niego mialo pozywienie wszelkie cialo.
\par 13 Widzialem nadto w widzeniach moich na lozu mojem, a oto stróz i Swiety z nieba zstapiwszy,
\par 14 Wolal ze wszystkiej mocy, i tak rzekl: Podrabcie to drzewo, i obetnijcie galezie jego, a otluczcie liscie jego, i rozrzuccie owoc jego; niech sie rozbiezy zwierz, który jest pod niem, i ptastwo z galezi jego;
\par 15 Wszakze pien korzenia jego w ziemi zostawcie, a niech bedzie zwiazany lancuchem zelaznym i miedzianym na trawie polnej, aby rosa niebieska byl skrapiany, a z zwierzetami niech sie pasie w trawie ziemskiej;
\par 16 Serce jego od czlowieczego niech sie odmieni, a serce zwierzece niech mu dane bedzie, a siedm lat niech pomina nad nim.
\par 17 Ta rzecz wedlug wyroku strózów, a to zadanie wedlug mowy swietych stanie sie, az do tego przyjdzie, ze poznaja ludzie, iz Najwyzszy panuje nad królestwem ludzkiem, a daje je, komu chce, a najpodlejszego z ludzi stanowi nad niem.
\par 18 Ten sen widzialem ja król Nabuchodonozor; a ty, Baltazarze! powiedz wyklad jego, gdyz wszyscy medrcy królestwa mego nie mogli mi tego wykladu oznajmic; ale ty mozesz, bo duch bogów swietych jest w tobie.
\par 19 Tedy Danijel, którego imie Baltazar, zdumiewal sie przez jedne godzine, a mysli jego trwozyly go. A odpowiadajac król rzekl: Baltazarze! sen i wyklad jego niech cie nie trwozy. Odpowiedzial Baltazar, i rzekl: Panie mój! ten sen niech przyjdzie na tych, którzy cie nienawidza, a wyklad jego na nieprzyjaciól twoich.
\par 20 Drzewo, któres widzial rosle i mocne, którego wysokosc dosiegala nieba, a które okazale bylo wszystkiej ziemi,
\par 21 Którego Gala? piekna, a owoc jego obfity, a pokarm dla wszystkich na niem, pod którem mieszkal zwierz polny, a na galeziach jego przebywalo ptastwo niebieskie,
\par 22 Tys jest tym, o królu! którys sie rozwielmozyl i zmocnil, a wielkosc twoja urosla, i podniosla sie az do nieba, a wladza twoja az do konczyn ziemi.
\par 23 A iz król widzial stróza i Swietego zstepujacego z nieba a mówiacego: Podrabcie to drzewo, a zepsujcie je, wszakze pien i z korzeniem jego w ziemi zostawcie, aby rosa niebieska byl skrapiany a z zwierzetami polnemi niech sie pasie, azby sie wypel nilo siedm lat nad nim;
\par 24 Tenci jest wyklad, o królu! i ten dekret Najwyzszego, który wyszedl na króla, pana mego;
\par 25 Bo cie wyrzuca od ludzi, a z zwierzem polnym bedzie mieszkanie twoje, a trawa jako wól pasc sie bedziesz, a rosa niebieska skrapiany bedziesz, az sie wypelni siedm lat nad toba, dokadbys nie poznal, ze Najwyzszy panuje nad królestwem ludzkiem, a ze je daje, komu chce.
\par 26 A iz rozkazano zostawic pien i z korzeniem onego drzewa, znaczy, ze królestwo twoje tobie zostanie, gdy poznasz, ze niebiosa panuja.
\par 27 Przetoz o królu! rada moja niech ci sie podoba, a grzechy twoje przerwij sprawiedliwoscia, a nieprawosci twoje milosierdziem nad utrapionym, owa snac stanie sie przedluzenie pokoju twego.
\par 28 Wszystko to przyszlo na króla Nabuchodonozora;
\par 29 Bo po wyjsciu dwunastu miesiecy, przechodzac sie w Babilonie na palacu królewskim,
\par 30 Mówil król i rzekl: Izali nie to jest on Babilon wielki, którym ja w sile mocy mojej zbudowal, aby byl stolica królestwa i ku ozdobie slawy mojej?
\par 31 A gdy jeszcze ta mowa byla w ustach królewskich, oto glos z nieba przyszedl mówiac: Tobie sie mówi, królu Nabuchodonozorze! ze królestwo twoje odeszlo od ciebie;
\par 32 I od ludzi wyrzuca cie, a z zwierzem polnym bedzie mieszkanie twoje; trawa jako wól pasc sie bedziesz, azby sie wypelnilo siedm lat nad toba, dokadbys nie poznal, ze Najwyzszy panuje nad królestwem ludzkiem, a ze je daje, komu chce.
\par 33 Tejze godziny wypelnilo sie ono slowo nad Nabuchodonozorem; bo go wyrzucono od ludzi, a trawe jadal jako wól, a rosa niebieska cialo jego skrapiane bylo, az na nim wlosy urosly jako pierze orle, a paznogcie jego jako pazury u ptaków.
\par 34 A po skonczeniu onych dni podnioslem ja Nabuchodonozor w niebo oczy moje, a rozum mój do mnie sie zas wrócil, i blogoslawilem Najwyzszego, a Zyjacego na wieki chwalilem i wyslawialem; bo wladza jego wladza wieczna, a królestwo jego od narodu do narodu.
\par 35 A wszyscy obywatele ziemi jako za nic poczytani sa. Wedlug woli swojej postepuje, z wojskiem niebieskiem i z obywatelami ziemi, a niemasz, ktoby wstret uczynil rece jego i rzekl mu: Cóz to czynisz?
\par 36 Tegoz czasu rozum mój wrócil sie do mnie, a do slawy królestwa mego ozdoba moja, i dostojnosc moja wrócila sie do mnie; nadto hetmani moi i ksiazeta moi szukali mie, a na królestwie mojem zmocnilem sie, i wielmoznosc wieksza mi jest przydana.
\par 37 A tak teraz ja Nabuchodonozor chwale, i wywyzszam i wyslawiam króla niebieskiego, którego wszystkie sprawy sa prawda, a scieszki jego sadem, a który chodzacych w hardosci ponizyc moze.

\chapter{5}

\par 1 Balsazar król uczynil uczte wielka na tysiac ksiazat swoich, i przed onym tysiacem pil wino.
\par 2 A gdy pil wino Balsazar, rozkazal przyniesc naczynie zlote i srebrne, które byl zabral Nabuchodonozor, ojciec jego, z kosciola Jeruzalemskiego, aby pili z niego król i ksiazeta jego, zony jego, i zaloznice jego.
\par 3 Tedy przyniesiono naczynia zlote, które byli zabrali z kosciola domu Bozego, który byl w Jeruzalemie, i pili z nich król i ksiazeta jego, zony jego, i zaloznice jego;
\par 4 A pijac wino chwalili bogi zlote i srebrne, miedziane, zelazne, drewniane, i kamienne.
\par 5 Tejze godziny wyszly palce reki czlowieczej, które pisaly przeciwko swiecznikowi na scianie palacu królewskiego, a król widzial czesc reki, która pisala.
\par 6 Tedy sie jasnosc królewska zmienila, a mysli jego zatrwozyly nim, i zwiaski biódr jego rozwiazaly sie, a kolana jego jedno o drugie sie tlukly.
\par 7 I zawolal król ze wszystkiej sily, aby przywiedziono praktykarzy, Chaldejczyków i wieszczków. A mówiac król rzekl do medrców Babilonskich: Ktokolwiek to pismo przeczyta, a wyklad jego mnie oznajmi, obleczony bedzie w szarlat, a lancuch zloty dadza na szyje jego, i trzecim w królestwie po mnie bedzie.
\par 8 Tedy weszli wszyscy medrcy królewscy; ale nie mogli ani pisma przeczytac, ani wykladu jego królowi oznajmic.
\par 9 Skad król Balsazar byl bardzo zatrwozony, a jasnosc jego zmienila sie na nim, i ksiazeta jego potrwozyli sie.
\par 10 Tedy królowa weszla do domu uczty dla tego, co sie przydalo królowi i ksiazetom jego; a przemówiwszy królowa rzekla: Królu, zyj na wieki! Niech cie nie trwoza mysli twoje, a jasnosc twoja niech sie nie mieni.
\par 11 Jest maz w królestwie twojem, w którym jest duch bogów swietych, w którym sie znalazlo za dni ojca twego oswiecenie, i rozum, i madrosc, jako madrosc bogów, którego król Nabuchodonozor, ojciec twój, przedniejszym miedzy medrcami, i praktykarzami Chaldejczykami, i wieszczkami, postanowil, ojciec twój mówie, o królu!
\par 12 Dlatego, iz duch obfity, i umiejetnosc, i zrozumienie, wykladanie snów, i objawienie zagadek, i rozwiazanie rzeczy trudnych znalazly sie przy Danijelu, któremu król dal imie Baltazar, teraz tedy niech przyzowia Danijela, a oznajmic ten wyklad.
\par 13 Tedy przywiedziony jest Danijel do króla; a król mówiac rzekl Danijelowi: Tyzes jest on Danijel, którys jest z synów wiezniów Judzkich, którego przywiódl król, ojciec mój, z ziemi Judzkiej?
\par 14 Slyszalem zaiste o tobie, iz duch bogów jest w tobie, a oswiecenie i rozum i madrosc obfita znalazla sie w tobie.
\par 15 A teraz przywiedziono przed mie medrców i praktykarzy, aby mi to pismo przeczytali, i wyklad jego oznajmili: wszakze nie mogli wykladu tej rzeczy oznajmic.
\par 16 A jam slyszal o tobie, ze mozesz to, co jest niepojetego, wykladac, a co jest trudnego, rozwiazywac; przetoz teraz, mozeszli to pismo przeczytac a wyklad jego mnie oznajmic, w szarlat obleczony bedziesz, i lancuch zloty na szyje twoje wlozony bedzie, a trzecim w królestwie po mnie bedziesz.
\par 17 Tedy odpowiedzial Danijel przed królem i rzekl: Upominki twoje niech tobie zostana, a dary twoje daj innemu; wszakze pismo przeczytam królowi, i wyklad mu oznajmie.
\par 18 Ty, królu! sluchaj. Bóg najwyzszy królestwo i wielmoznosc i slawe, i zacnosc dal Nabuchodonozorowi, ojcu twemu;
\par 19 A dla wielmoznosci, która mu byl dal, wszyscy ludzie, narody i jezyki drzeli i bali sie przed obliczem jego; bo kogo chcial, zabijal, a kogo chcial, zywil, a kogo chcial, wywyzszal, a kogo chcial, ponizal.
\par 20 Ale gdy sie wynioslo serce jego, a duch jego zmocnil sie w pysze, zlozony jest z stolicy królestwa swego, a slawa odjeta byla od niego;
\par 21 I byl wyrzucony od synów ludzkich, a serce jego zwierzecemu podobne bylo, i z dzikiemi oslami bylo mieszkanie jego; trawa sie pasl jako wól, i rosa niebieska cialo jego skrapiane bylo, dokad nie poznal, ze Bóg najwyzszy ma wladze nad królestwem ludzkiem, a tego, kogo chce, stanowi nad niem.
\par 22 Ty, tez, Balsazarze, synu jego! nie upokorzyles serca swego, chociazes to wszystko wiedzial.
\par 23 Owszem, przeciwko Panu nieba podniosles sie, i naczynie domu jego przyniesiono przed cie; a ty i ksiazeta twoi, zony twoje, i zaloznice twoje, piliscie wino z niego; nadto bogi srebrne i zlote, miedziane, zelazne, drewniane, i kamienne, którzy nie widza, ani slysza, i nic nie wiedza, chwaliles, a Boga, w którego reku jest tchnienie twoje, i u którego sa wszystkie drogi twoje, nie uczciles.
\par 24 Przetoz teraz od niego poslana jest ta czesc reki, i pismo to wyrazone jest.
\par 25 A toc jest pismo, które wyrazone jest: Mene, Mene, Thekel, upharsin.
\par 26 A tenci jest wyklad tych slów: Mene, zliczyl Bóg królestwo twoje i do konca je przywiódl.
\par 27 Thekel, zwazonys na wadze, a znalezionys lekki.
\par 28 Peres, rozdzielone jest królestwo twoje, a dane jest Medom i Persom.
\par 29 Tedy rozkazal Balsazar; i obleczono Danijela w szarlat, a lancuch zloty wlozono na szyje jego, i obwolano o nim, ze ma byc trzecim panem w królestwie.
\par 30 Tejze nocy zabity jest Balsazar, król Chaldejski.
\par 31 A Daryjusz, Medczyk, ujal królestwo, majac lat okolo szescdziesiat i dwa.

\chapter{6}

\par 1 I podobalo sie Daryjuszowi, aby postanowil nad królestwem sto i dwadziescia starostów, którzyby byli we wszystkiem królestwie.
\par 2 A nad nimi troje ksiazat, z których byl Danijel przedniejszym, którymby oni starostowie liczbe czynili, aby król szkody nie mial.
\par 3 A sam Danijel przewyzszal onych ksiazat i starszych, przeto, ze duch znamienitszy byl w nim, skad go król myslal postanowic nad wszystkiem królestwem.
\par 4 Tedy ksiazeta i starostowie szukali, aby znalezli przyczyne przeciwko Danijelowi z strony królestwa; wszakze zadnej przyczyny ani wady znalesc nie mogli, poniewaz on byl wiernym, ani zadna wina ani wada nie znajdowala sie w nim.
\par 5 Przetoz rzekli oni mezowie: Nie znajdziemy przeciwko temu Danijelowi zadnej przyczyny, chyba zebysmy co znalezli przeciwko niemu w zakonie Boga jego.
\par 6 Tedy oni ksiazeta i starostowie zgromadzili sie do króla, i tak mu rzekli: Daryjuszu królu, zyj na wieki!
\par 7 Uradzili wszyscy ksiazeta królestwa, przelozeni i starostowie, urzednicy i hetmani, aby postanowiony byl dekret królewski, i stwierdzony wyrok, aby kazdy, któryby do trzydziestu dni o cokolwiek prosil którego boga albo czlowieka oprócz ciebie, królu! byl wrzucony do dolu lwiego.
\par 8 A tak teraz, o królu! potwierdz ten wyrok, a podaj go na pismie, zeby sie nie odmienil wedlug prawa Medskiego i Perskiego, które sie nie odmienia.
\par 9 Skad król Daryjusz podal na pismie ten wyrok.
\par 10 Czego gdy sie Danijel dowiedzial, ze byl podany na pismie, wszedl do domu swego, gdzie otworzone byly okna w pokoju jego przeciw Jeruzalemowi, a trzy kroc przez dzien klekal na kolana swoje, i modlil sie, a chwale dawal Bogu swemu, jako to byl z wykl przedtem czynic.
\par 11 Tedy oni mezowie zgromadziwszy sie, a znalazlszy Danijela modlacego sie i prosby wylewajacego do Boga swego,
\par 12 Przystapili i mówili królowi o wyroku królewskim: Izalis wyroku nie wydal, aby kazdy czlowiek, któryby do trzydziestu dni o cokolwiek prosil którego boga albo czlowieka oprócz ciebie, królu! byl wrzucony do dolu lwiego? Odpowiedzial król, i rzekl: Prawdziwa to mowa wedlug prawa Medskiego i Perskiego, które sie nie odmienia.
\par 13 Tedy odpowiadajac rzekli do króla: Ten Danijel, który jest z wiezniów synów Judzkich, nie ma wzgledu na cie, o królu! ani na twój wyrok, którys wydal; bo trzy kroc przez dzien odprawuje modlitwy swoje.
\par 14 Te slowa gdy król uslyszal, bardzo sie zasmucil nad tem; i sklonil król do Danijela serce swoje, aby go wyswobodzil; az do zachodu slonca staral sie, aby go wyrwal.
\par 15 Ale mezowie oni zgromadzili sie do króla, i rzekli królowi: Wiedz, królu! iz to jest prawo u Medów i u Persów, aby zaden wyrok i dekret, któryby król postanowil, nie byl odmieniony.
\par 16 Tedy król rozkazal, aby przywiedziono Danijela, i wrzucono go do dolu lwiego; a król mówiac rzekl do Danijela: Bóg twój, któremu ty ustawicznie sluzysz, ten cie wybawi.
\par 17 Tedy przyniesiono kamien jeden, i polozono go na dziurze onego dolu, i zapieczetowal go król sygnetem swoim, i sygnetami ksiazat swoich, aby nie byl odmieniony dekret wydany przeciwko Danijelowi.
\par 18 Potem odszedl król na palac swój, i przenocowal nic nie jadlszy, i nic nie przypuscil przed sie, coby go uweselic moglo, tak, ze i sen jego odstapil od niego.
\par 19 Tedy król wstawszy bardzo rano na switaniu z kwapieniem poszedl do dolu lwiego;
\par 20 A gdy sie przyblizyl do dolu, zawolal na Danijela glosem zalosnym, a mówiac król rzekl do Danijela: Danijelu, slugo Boga zywego! Bóg twój, któremu ty ustawicznie sluzysz, móglze cie wybawic ode lwów?
\par 21 Tedy Danijel do króla rzekl: Królu, zyj na wieki!
\par 22 Bóg mój poslal Aniola swego, który zamknal paszczeke lwom, aby mi nie zaszkodzily, dlatego, ze sie przed nim znalazla niewinnosc we mnie; owszem, ani przed toba, królu! nicem zlego nie uczynil.
\par 23 Tedy sie król wielce ucieszyl z tego, i rozkazal Danijela wyciagnac z dolu; i wyciagniono Danijela z dolu, a zadnego obrazenia nie znaleziono na nim; bo wierzyl w Boga swego.
\par 24 I rozkazal król, aby przywiedziono onych mezów, którzy byli oskarzyli Danijela, i wrzucono ich do dolu lwiego, onych samych, i synów ich, i zony ich; a pierwej niz dopadli do dna onego dolu, pochwycily ich lwy, i wszystkie kosci ich pokruszyly.
\par 25 Tedy król Daryjusz napisal do wszystkich ludzi, narodów, i jezyków, którzy mieszkali po wszystkiej ziemi: Pokój sie wam niech rozmnozy!
\par 26 Wydany jest odemnie ten wyrok, aby po wszystkiem panstwie królestwa mego wszyscy drzeli a bali sie oblicza Boga Danijelowego; bo on jest Bóg zyjacy i trwajacy na wieki, a królestwo jego ani wladza jego nie bedzie skazona az do konca;
\par 27 On wyrywa i wybawia, a czyni znaki i cuda na niebie i na ziemi, który wyrwal Danijela z mocy lwów.
\par 28 A Danijelowi sie szczesliwie powodzilo w królestwie Daryjusza, i w królestwie Cyrusa, Persy.

\chapter{7}

\par 1 Roku pierwszego Balsazara, króla Babilonskiego, mial Danijel sen i widzenia swoje na lozu swem; tedy spisal on sen, i sume rzeczy powiedzial.
\par 2 A mówiac Danijel rzekl: Widzialem w widzeniu mojem w nocy, a oto cztery wiatry niebieskie potykaly sie na morzu wielkiem;
\par 3 A cztery bestyje wielkie wystepowaly z morza, rózne jedna od drugiej.
\par 4 Pierwsza podobna lwowi, majac skrzydla orle; i przypatrywalem sie, az wyrwane byly skrzydla jej, któremi sie podnosila od ziemi, tak, ze na nogach jako czlowiek stanela, a serce czlowiecze jej dane jest.
\par 5 Potem oto bestyja druga podobna niedzwiedziowi; i stanela przy jednej stronie, a trzy zebra byly w paszczece jej miedzy zebami jej, i tak mówiono do niej: Wstan, nazryj sie dostatkiem miesa.
\par 6 Potemem widzial, a oto inna bestyja podobna lampartowi, która miala cztery skrzydla ptasze na grzbiecie swym, cztery tez glowy miala ta bestyja, i dano jej wladze wielka.
\par 7 Potemem widzial w widzeniach nocnych, a oto bestyja czwarta straszna i sroga i bardzo mocna, majaca zeby zelazne wielkie, która pozerala i kruszyla, a ostatek nogami swemi deptala; a ta byla rózna od wszystkich bestyj, które byly przed nia, i mial a dziesiec rogów.
\par 8 Pilniem sie przypatrywal rogom, a oto róg posledni maly wyrastal miedzy niemi, i trzy z tych rogów pierwszych wylamane sa przed nim; a oto w onym rogu byly oczy podobne oczom czlowieczym, i usta mówiace rzeczy wielkie.
\par 9 I przypatrywalem sie, az one stolice postawione byly, a Starodawny usiadl, którego szata byla jako snieg biala, a wlosy glowy jego jako welna czysta, stolica jego jako plomienie ogniste, a kola jej jako ogien gorejacy.
\par 10 Rzeka ognista plynac wychodzila od oblicza jego. Tysiac tysiecy sluzylo mu, a dziesiec kroc tysiac tysiecy stalo przed nim; sad zasiadl, a ksiegi otworzone byly.
\par 11 Tedym sie przypatrywal, skoro sie glos poczal tych slów wielkich, które on róg mówil; przypatrywalem sie, az byla ta bestyja zabita, i zginelo cialo jej, a podane bylo na spalenie ogniem.
\par 12 Takze i pozostalym bestyjom odjeta jest wladza ich: bo dlugosc zywota dana im byla az do czasu, a to do zamierzonego czasu.
\par 13 Widzialem tez w widzeniu nocnem, a oto przychodzil w oblokach niebieskich podobny synowi czlowieczemu, a przyszedl az do Starodawnego, i przywiedziono go przed oblicznosc jego.
\par 14 I dal mu wladze i czesc i królestwo, aby mu wszyscy ludzie, narody i jezyki sluzyli; wladza jego wladza wieczna, która nie bedzie odjeta, a królestwo jego, które nie bedzie skazone.
\par 15 I zatrwozyl sie we mnie Danijelu duch mój w posród ciala mego, a widzenia, którem widzial, przestraszyly mie.
\par 16 Tedym przystapil do jednego z tych, którzy tam stali, a pewnoscim sie dowiadywal od niego o tem wszystkiem, i powiedzial mi, i wyklad mów oznajmil mi.
\par 17 Te bestyje wielkie, których sa cztery, sa cztery królowie, którzy powstana z ziemi.
\par 18 Ci ujma królestwo swietych najwyzszych miejsc, którzy posiasc maja królestwo az na wieki, i az na wieki wieczne.
\par 19 Tedym pragnal wziac sprawe o bestyi czwartej, która byla rózna od wszystkich innych, bardzo straszna, której zeby byly zelazne, a paznogcie jej miedziane; która pozerala i kruszyla, a ostatek nogami swemi deptala.
\par 20 Takze o onych rogach dziesieciu, które byly na glowie jej, i o poslednim, który byl wyrósl, przed którym wypadly trzy; o tym rogu mówie, który mial oczy i usta mówiace wielkie rzeczy, a na wejrzeniu byl wiekszy nad inne rogi.
\par 21 I przypatrywalem sie, a oto róg ten walczyl z swietymi, i przemagal ich;
\par 22 Az przyszedl Starodawny a podany jest sad swietym najwyzszych miejsc, a czas przyszedl, aby to królestwo swieci otrzymali.
\par 23 I rzekl tak: Bestyja czwarta, czwarte królestwo znaczy na ziemi, które bedzie rózne od wszystkich królestw, a pozre wszystke ziemie, a podepcze a pokruszy ja;
\par 24 A dziesiec rogów to znaczy, ze z królestwa onego dziesiec królów powstanie; a po nich powstanie posledni, który bedzie rózny od pierwszych, i trzech królów ponizy;
\par 25 A slowa przeciw Najwyzszemu mówic bedzie, i swiete najwyzszych miejsc zetrze; nadto bedzie zamyslal, aby odmienil czasy i prawa, gdyz wydane beda w rece jego az do czasu i czasów, i pól czasu.
\par 26 Potem zasiadzie sad, a tam wladze jego odejma, aby byl zniszczony i wytracony az do konca.
\par 27 A królestwo i wladza, i dostojenstwo królewskie pod wszystkiem niebem dane bedzie ludowi swietych najwyzszych miejsc, którego królestwo bedzie królestwo wieczne, a wszystkie zwierzchnosci jemu sluzyc i onego sluchac beda.
\par 28 Az dotad koniec tych slów. A mnie Danijela mysli moje wielce zatrwozyly, a jasnosc moja zmienila sie przy mnie; wszakzem to slowo w sercu mojem zachowal.

\chapter{8}

\par 1 Roku trzeciego królowania Balsazara, króla, okazalo mi sie widzenie, mnie Danijelowi, po onem, które mi sie okazalo na poczatku.
\par 2 I widzialem w widzeniu, a (gdym to widzial, bylem w Susan, miescie glównem, które bylo w krainie Elam) widzialem, mówie, w widzeniu, gdym byl u potoku Ulaj.
\par 3 I podnioslem oczy moje, i ujrzalem, a oto u onego potoku stal baran jeden majacy dwa rogi, a te dwa rogi byly wysokie, lecz jeden byl wyzszy niz drugi; ale ten wyzszy rósl posledzej.
\par 4 Widzialem onego barana trykajacego na zachód, i na pólnoc, i na poludnie, a zadna mu bestyja zdolac nie mogla, i nie byl, ktoby co wyrwal z reki jego; skad czynil wedlug woli swojej, i stal sie wielkim.
\par 5 Co gdym ja uwazal, oto koziel z kóz przychodzil od zachodu na oblicze wszystkiej ziemi, a nikt sie go nie dotykal na ziemi; a ten koziel mial róg znaczny miedzy oczyma swemi.
\par 6 I przyszedl az do onego barana, który mial dwa rogi, któregom widzial stojacego u potoku; a przybiezal do niego w popedliwosci sily swojej.
\par 7 Widzialem takze, iz natarl na onego barana, a rozjadlszy sie nan uderzyl barana, tak, ze zlamal one oba rogi jego, i nie bylo mocy w baranie, zeby mu mógl odpór; a rzuciwszy go o ziemie zdeptal go, a nie byl, ktoby wyrwal barana z mocy jego.
\par 8 Tedy on koziel z kóz stal sie bardzo wielkim; ale gdy sie zmocnil, zlamal sie on róg wielki, a wyrosly cztery rogi znaczne miasto niego na cztery strony swiata.
\par 9 A z jednego z nich wyszedl róg jeden maly, a ten wielce urósl ku poludniowi, i ku wschodowi i ku ziemi ozdobnej;
\par 10 I wyrósl az do wojska niebieskiego, i zrzucil niektórych na ziemie z onego wojska i z gwiazd, i podeptal ich;
\par 11 Nawet az do ksiazecia onego wojska wyrósl; bo przezen odjeta byla ustawiczna ofiara, i zarzucone miejsce swiatnicy Bozej,
\par 12 Takze wojsko one podane w przestepstwo przeciwko ustawicznej ofierze, i porzucilo prawde na ziemie, a cokolwiek czynilo, szczescilo mu sie.
\par 13 Tedym uslyszal jednego z swietych mówiacego: i rzekl ten swiety do onego, który majac policzone tajemnice, mówi: Dokadze to widzenie o ofierze ustawicznej i przestepstwo pustoszace trwac bedzie, i swiete uslugi, i wojsko na podeptanie podane bedzie?
\par 14 I rzekl do mnie: Az do dwóch tysiecy i trzech set wieczorów i poranków; tedy przyjda do odnowienia swego uslugi swiete.
\par 15 A gdym ja Danijel patrzyl na to widzenie, i pytalem sie o wyrozumieniu jego, tedy oto stanal ktos podle mnie, na wejrzeniu jako maz.
\par 16 Slyszalem tez glos ludzki miedzy Ulajem, który zawolawszy rzekl: Gabryjelu! wylóz mu to widzenie.
\par 17 I przyszedl do mnie, gdziem stal; a gdy przyszedl, zleklem sie i padlem na oblicze swoje. I rzekl do mnie: Wyrozumij, synu czlowieczy! bo czasu pewnego to widzenie sie wypelni.
\par 18 A gdy on mówil ze mna, usnalem twardo, lezac twarza swoja na ziemi, i dotknal sie mnie, i postawil mie tam, gdziem pierwej stal,
\par 19 I rzekl: Oto ja tobie oznajmie, co sie dziac bedzie az do wykonania tego gniewu; bo czasu naznaczonego koniec bedzie.
\par 20 Ten baran, któregos widzial majacego dwa rogi, sa królowie, Medski i Perski.
\par 21 A ten koziel kosmaty jest król Grecki, a ten róg wielki, który jest miedzy oczyma jego, jest król pierwszy.
\par 22 A iz zlamany jest, a powstaly cztery miasto niego, czworo królestw z jego narodu powstana, ale nie z taka moca.
\par 23 A przy skonczeniu królestwa ich, gdy przestepnicy zlosci dopelnia, powstanie król niewstydliwej twarzy i chytry;
\par 24 I zmocni sie sila jego, aczkolwiek nie jego sila, tak, ze na podziw bedzie wytracal, a szczesliwie mu sie powiedzie, i wszystko wykona; bo wytracac bedzie mocarzów i lud swiety;
\par 25 A przemyslem jego poszczesci mu sie zdrada w reku jego, a uwielbi sam siebie w sercu swojem, i czasu pokoju wiele ich pogubi; nadto i przeciw ksiazeciu ksiazat powstanie, a wszakze bez reki pokruszony bedzie.
\par 26 A to widzenie wieczorne i poranne, o którem powiedziano, jest sama prawda; przetoz ty zapieczetuj to widzenie, bo jest wielu dni.
\par 27 Tedym ja Danijel zemdlal, i chorowalem przez kilka dni; potem wstawszy odprawowalem sprawy królewskie, a zdumiewalem sie nad onem widzeniem, czego jednak nikt nie obaczyl.

\chapter{9}

\par 1 Roku pierwszego Daryjusza, syna Aswerusowego, z nasienia Medów, który byl postanowiony królem na królestwem Chaldejskiem;
\par 2 Roku pierwszego królowania jego, ja Danijel zrozumialem z ksiag liczbe lat, o których bylo slowo Panskie do Jeremijasza proroka, ze sie wypelnic mialo spustoszenie Jeruzalemskie w siedmdziesiet lat.
\par 3 I obrócilem oblicze moje do Pana Boga, szukajac go modlitwa i prosbami w poscie i w worze i w popiele.
\par 4 Modlilem sie tedy Panu Bogu memu, a wzywajac rzeklem: Prosze Panie! Boze wielki i straszny, strzegacy przymierza i milosierdzia tym, którzy cie miluja, i strzega przykazan twoich;
\par 5 Zgrzeszylismy i przewrotniesmy czynili, i niezbozniesmy sie sprawowali, i sprzeciwilismy sie, a odstapilismy od przykazan twoich i od sadów twoich;
\par 6 I nie sluchalismy slug twoich, proroków, którzy mawiali w imieniu twojem do królów naszych, do ksiazat naszych, i do ojców naszych, i do wszystkiego ludu ziemi.
\par 7 Tobie, Panie! sprawiedliwosc, a nam zawstydzenie twarzy nalezy, jako sie to dzieje dnia tego mezom Judzkim i obywatelom Jeruzalemskim i wszystkiemu Izraelowi, bliskim i dalekim we wszystkich ziemiach, do któryches ich wygnal dla przestepstwa ich, którem wystapili przeciwko tobie.
\par 8 Panie! namci nalezy zawstydzenie twarzy, królom naszym, ksiazetom naszym i ojcom naszym, bosmy zgrzeszyli przeciwko tobie:
\par 9 Ale Panu, Bogu naszemu, milosierdzie i litosc; poniewazesmy mu odporni byli,
\par 10 A nie bylismy posluszni glosowi Pana, Boga naszego, zebysmy chodzili w ustawach jego, które on dal przed oblicze nasze przez proroków, slug swoich;
\par 11 Owszem, wszyscy Izraelczycy przestapili zakon twój i odchylili sie, zeby nie sluchali glosu twego: przetoz sie wylalo na nas to zlorzeczenstwo i przeklestwo, które jest napisane w zakonie Mojzesza, slugi Bozego; bosmy zgrzeszyli przeciwko niemu.
\par 12 Skad spelnil slowa swoje, które mówil przeciwko nam, i przeciwko sedziom naszym, którzy nas sadzili, a przywiódl na nas to wielkie zle, które sie nie stalo pod wszystkiem niebem, jakie sie stalo w Jeruzalemie.
\par 13 Tak jako napisano w zakonie Mojzeszowym, wszystko to zle przyszlo na nas; a wzdysmy nie prosili oblicza Pana, Boga naszego, abysmy sie odwrócili od nieprawosci naszych, a mieli wzglad na prawde jego.
\par 14 Przetoz nie omieszkal Pan z tem zlem, ale je przywiódl na nas; bo sprawiedliwy jest Pan, Bóg nasz, we wszystkich sprawach swoich, które czyni, któregosmy glosu nie sluchali.
\par 15 Wszakze teraz, o Panie, Boze nasz! którys wywiódl lud swój z ziemi Egipskiej reka mozna, i uczyniles sobie imie. jako sie to dzis pokazuje, zgrzeszylismy, niepobozniesmy czynili.
\par 16 O Panie! wedlug wszystkich sprawiedliwosci twoich niech sie prosze odwróci popedliwosc twoja i gniew twój od miasta twego Jeruzalemu, góry swietobliwosci twojej: bo dla grzechów naszych i dla nieprawosci ojców naszych Jeruzalem i lud twój nosi po hanbienie u wszystkich, którzy sa okolo nas.
\par 17 Teraz tedy wysluchaj, o Boze nasz! modlitwe slugi twego i prosby jego, a oswiec oblicze twoje nad spustoszona swiatnica twoja, dla Pana.
\par 18 Naklon, Boze mój! ucha twego a uslysz; otwórz oczy twoje a obacz spustoszenia nasze i miasto, które jest nazwane od imienia twego; bo my przekladamy modlitwy nasze przed obliczem twojem, nie dla jakiej naszej sprawiedliwosci, ale dla obfitego milosierdzia twego.
\par 19 O Panie! wysluchaj, Panie! odpusc, Panie! obacz a uczyn; nie odwlaczaj sam dla siebie, Boze mój! bo od imienia twego nazwane jest to miasto i lud twój.
\par 20 A gdym ja jeszcze mówil, i modlilem sie, i wyznawalem grzech mój i grzech ludu mego Izraelskiego, i przekladalem modlitwe moje przed twarza Pana, Boga mego, za góre swietobliwosci Boga mego;
\par 21 Prawie gdym ja jeszcze mówil i modlilem sie, oto maz on Gabryjel, któregom widzial w widzeniu na poczatku, predko lecac dotknal sie mnie czasu ofiary wieczornej,
\par 22 A uslugujac mi do zrozumienia mówil ze mna i rzekl: Danijelu! terazem wyszedl, abym cie nauczyl wyrozumienia tajemnicy.
\par 23 Na poczatku modlitw twoich wyszlo slowo, a jam przyszedl, abym ci je oznajmil, bos ty wielce przyjemny; a tak miej wzglad na to slowo, a zrozumiej to widzenie.
\par 24 Siedmdziesiat tego dni zamierzono ludowi twemu i miastu twemu swietemu na zniesienie przestepstwa, i na zagladzenie grzechów i na oczyszczenie nieprawosci, i na przywiedzienie sprawiedliwosci wiecznej, i na zapieczetowanie widzenia i proroctwa, a na pomazanie Swietego swietych.
\par 25 Przetoz wiedz a zrozumiej, ze od wyjscia slowa o przywróceniu i zbudowaniu Jeruzalemu az do Mesyjasza wodza bedzie tygodni siedm, potem tygodni szescdziesiat i dwa, gdy znowu zbudowana bedzie ulica i przekopanie, a te czasy beda bardzo trudne.
\par 26 A po onych szescdzesieciu i dwóch tygodniach zabity bedzie Mesyjasz, wszakze mu to nic nie zaszkodzi; owszem, to miasto i te swiatnice skazi lud wodza przyszlego, tak, ze koniec jego bedzie z powodzia, i az do skonczenia wojny bedzie ustawiczne pustoszenie.
\par 27 Wszakze zmocni przymierze wielom ich w tygodniu ostatnim; a w polowie onego tygodnia uczyni koniec ofierze palonej i ofierze sniednej, a przez wojsko obrzydliwe pustoszyciel przyjdzie, i az do skonczenia naznaczonego wyleje sie spustoszenie na te go, który ma byc spustoszony.

\chapter{10}

\par 1 Roku trzeciego Cyrusa, króla Perskiego objawione bylo slowo Danijelowi, którego imie nazwano Baltazar; a to slowo bylo prawdziwe, i czas zamierzony dlugi; i zrozumial ono slowo, bo wzial zrozumienie w widzeniu.
\par 2 W one dni ja Danijel bylem smutny przez trzy tygodnie dni;
\par 3 Chlebam smacznego nie jadl, a mieso i wino nie wchodzilo w usta moje, anim sie mazal olejkiem, az sie wypelnily dni trzech tygodni.
\par 4 A dnia dwudziestego i czwartego miesiaca pierwszego bylem nad brzegiem rzeki wielkiej, to jest Chydekel;
\par 5 A podnióslszy oczy moje ujrzalem, a oto maz niejaki ubrany w szate lniana, a biodra jego przepasane byly zlotem szczerem z Ufas;
\par 6 A cialo jego bylo jako Tarsys, a oblicze jego na wejrzeniu jako blyskawica, a oczy jego byly jako lampy gorejace, a ramiona jego i nogi jego na wejrzeniu jako miedz wypolerowana, a glos slów jego jako glos mnóstwa.
\par 7 A widzialem ja Danijel sam to widzenie; lecz mezowie, którzy byli ze mna nie widzieli tego widzenia; ale strach wielki przypadl na nich, i pouciekali a pokryli sie.
\par 8 A jam sam zostal, i widzialem to wielkie widzenie; ale sila nie zostala we mnie, i krasa moja odmienila sie we mnie, i skazila sie, i nie mialem zadnej sily.
\par 9 Tedym slyszal glos slów jego; a uslyszawszy glos slów jego usnalem twardo na twarzy mojej, na twarzy mojej, mówie, na ziemi.
\par 10 Wtem oto reka dotknela sie mnie, i podniosla mie na kolana moje, i na dlonie rak moich.
\par 11 I rzekl do mnie: Danijelu, mezu wielce przyjemny! miej wzglad na slowa moje, które ja bede mówil do ciebie, a stój na miejscu swem, bom teraz poslany do ciebie. A gdy przemówil do mnie to slowo, stalem drzac.
\par 12 I rzekl do mnie: Nie bój sie Danijelu! bo od pierwszego dnia, jakos przylozyl serce twoje ku wyrozumieniu i trapiles sie Bogiem twoim, wysluchane sa slowa twoje, a jam przyszedl dla slów twoich.
\par 13 Lecz ksiaze królestwa Perskiego sprzeciwial mi sie przez dwadziescia dni i jeden, az oto Michal, jeden z przedniejszych ksiazat, przyszedl mi na pomoc; przetozem ja tam zostal przy królach Perskich.
\par 14 Alem przyszedl, abym ci oznajmil, co ma przyjsc na lud twój w ostateczne dni; bo jeszcze widzenie bedzie o tych dniach.
\par 15 A gdy mówil do mnie temi slowy, spuscilem twarz moje ku ziemi, i zamilknalem.
\par 16 A oto jako podobienstwo synów ludzkich dotknelo sie warg moich; a otworzywszy usta swe mówilem i rzeklem do stojacego przeciwko mnie: Panie mój! dla tego widzenia obrócily sie na mie bolesci moje, i nie mialem zadnej sily.
\par 17 A jakoz bedzie mógl taki sluga Pana mego rozmówic sie z takim Panem moim? Gdyz od tegoz czasu nie zostala we mnie sila, ani tchnienie zostalo we mnie.
\par 18 Tedy sie mnie znowu dotknal na wejrzeniu jako czlowiek, i posilil mie,
\par 19 I rzekl: Nie bój sie, mezu wielce przyjemny, pokój tobie! posil sie, posil sie, mówie. A gdy mówil ze mna, wzialem sile i rzeklem: Niech mówi Pan mój; albowiemes mie posilil.
\par 20 I rzekl: Wieszze, dlaczegom przyszedl do ciebie? Potem sie wróce, abym walczyl z ksiazeciem Perskim, a gdy odejde, oto ksiaze Grecki przyciagnie.
\par 21 Wszakze oznajmiec to, co jest wyrazono w pismie prawdy; ale i jednego niemasz, któryby meznie stal przy mnie w tych rzeczach, oprócz Michala, ksiazecia waszego.

\chapter{11}

\par 1 Ja tedy roku pierwszego za Daryjusza Medskiego stanalem, abym go posilil i zmocnil.
\par 2 A teraz ci prawde oznajmie: Oto jeszcze trzej królowie królowac beda w Perskiej ziemi; potem czwarty zbogaci sie bogactwy wielkimi nade wszystkich, a gdy sie zmocni w bogactwach swoich, pobudzi wszystkich przeciw królestwu Greckiemu.
\par 3 I powstanie król mocny, a bedzie panowal moca wielka, a bedzie czynil wedlug woli swojej.
\par 4 A gdy sie on zmocni, bedzie skruszone królestwo jego, i bedzie rozdzielone na cztery strony swiata, wszakze nie miedzy potomków jego, ani bedzie panstwo jego takie, jakie bylo; bo wykorzenione bedzie królestwo jego, a innym mimo onych dostanie sie.
\par 5 Tedy sie zmocni król z poludnia i jeden z ksiazat jego; ten mocniejszy bedzie naden, i panowac bedzie, a panstwo jego bedzie panstwo szerokie.
\par 6 Lecz po wyjsciu kilku lat zlacza sie; bo córka króla od poludnia pójdzie za króla pólnocnego, aby uczynila przymierze; wszakze nie otrzyma sily ramienia, ani sie ostoi z ramieniem swojem, ale wydana bedzie ona, i ci, którzy ja przyprowadza, i syn jej, i ten, co ja zmacnial za onych czasów.
\par 7 Potem powstanie z latorosli korzenia jej na miejsce jego, który przyciagnie z wojskiem swem, a uderzy na miejsce obronne króla pólnocnego, i przewiedzie nad nimi i zmocni sie.
\par 8 Nadto i bogów ich z ksiazetami ich, z naczyniem ich drogiem, srebrnem i zlotem w niewole zawiedzie do Egiptu; a ten bedzie bezpieczen przez wiele lat od króla pólnocnego.
\par 9 A tak wtargnie w królestwo król od poludnia, i wróci sie do ziemi swojej.
\par 10 Ale synowie jego walczyc beda, i zbiora mnóstwo wojsk wielkich; a z nagla nastepujac jako powódz przechodzic bedzie, potem wracajac sie, wojskiem nacierac bedzie az na twierdze jego.
\par 11 Skad rozdrazniony bedac król z poludnia wyciagnie, i bedzie walczyl z nim, to jest, z królem pólnocnym; a uszykuje mnóstwo wielkie, ale ono mnóstwo bedzie podane w reke jego.
\par 12 A gdy zniesione bedzie ono mnóstwo, podniesie sie serce jego; a choc porazi wiele tysiecy, przecie sie nie zmocni.
\par 13 Bo sie wróci król pólnocny, i uszykuje wieksze mnóstwo niz pierwsze; lecz po wyjsciu czasu kilku lat z nagla przyjdzie z wielkiem wojskiem i z wielkim dostatkiem.
\par 14 Onychze czasów wiele ich powstanie przeciwko królowi z poludnia; ale synowie przestepników z ludu twego beda zniesieni dla utwierdzenia tego widzenia, i upadna.
\par 15 Bo przyciagnie król z pólnocy, i usypie waly, i wezmie miasto obronne, a ramiona poludniowe nie opra sie, ani lud jego wybrany, i nie stanie im sily, aby dali odpór.
\par 16 I uczyni on, który przyciagnie przeciwko niemu, wedlug woli swojej, i nie bedzie nikogo, coby sie stawil przeciwko niemu; stawi sie tez w ziemi ozdobnej, która zniszczeje przez reke jego.
\par 17 Potem obróci twarz swoje, aby przyszedl z moca wszystkiego królestwa swego, i okazal sie, jakoby zgody szukal, i uczyni cos; bo mu da córke piekna, aby go zgubil przez nia; ale ona w tem nie bedzie stateczna, i nie bedzie z nim przestawala.
\par 18 Zatem obróci twarz swoje do wysep, i wiele ich pobierze; ale wódz wstret uczyni hanbieniu jego, owszem, ono hanbienie jego nan obróci.
\par 19 Dlaczego obróci twarz swoje ku twierdzom ziemi swej; lecz sie potknie i upadnie, i nie bedzie wiecej znaleziony.
\par 20 I powstanie na miejsce jego taki, który rozesle poborców w slawie królewskiej; ale ten po niewielu dniach starty bedzie, a to nie w gniewie ani przez wojne.
\par 21 Potem powstanie wzgardzony na jego miejsce, acz nie wloza nan ozdoby królewskiej; wszakze przyszedlszy w pokoju, otrzyma królestwo pochlebstwem.
\par 22 A ramionami jako powodzia wiele ich zachwyceni beda przed obliczem jego, i skruszeni beda, takze tez i sam wódz, który z nimi przymierze uczynil.
\par 23 Bo wszedlszy z nimi w przyjazn, uczyni zdrade, a przyciagnawszy zmocni sie w malym poczcie ludu.
\par 24 Bezpiecznie i do najobfitszych miejsc onej krainy wpadnie, a uczyni to, czego nie czynili ojcowie jego, ani ojcowie ojców jego; lup i korzysc i majetnosci rozdzieli im, nawet i o miejscach obronnych chytrze przemysliwac bedzie, a to az do czasu.
\par 25 Potem wzbudzi moc swoje, i serce swoje przeciw królowi z poludnia z wojskiem wielkiem, z którem król z poludnia walecznie sie potykac bedzie z wojskiem wielkiem i bardzo mocnem; ale sie nie oprze, przeto, ze wymysli przeciwko niemu zdrade.
\par 26 Bo ci, którzy jedza chleb jego, zniszcza go, gdy wojsko onego jako powódz przypadnie, a pobitych wiele poleze.
\par 27 Natenczas obaj królowie w sercu swem myslic beda, jakoby jeden drugiemu szkodzic mógl, a przy jednymze stole klamstwo mówic beda; ale sie im nie nada, gdyz jeszcze koniec na inszy czas odlozony jest.
\par 28 Przetoz sie wróci do ziemi swojej z bogactwy wielkiemi, a serce jego obróci sie przeciwko przymierzu swietemu; co uczyniwszy wróci sie do ziemi swojej.
\par 29 A czasu zamierzonego wróci sie i pociagnie na poludnie; ale mu sie nie tak powiedzie, jako za pierwszym i za ostatnim razem.
\par 30 Bo przyjda przeciwko niemu okrety z Cytym, skad on nad tem bolejac znowu sie rozgniewa przeciwko przymierzu swietemu; co uczyniwszy wróci sie, a bedzie mial porozumienie z onymi, którzy opuscili przymierze swiete;
\par 31 A wojska wielkie przy nim stac beda, które splugawia swiatnice, i twierdze zniosa; odejma tez ustawiczna ofiare, a postawia obrzydliwosc spustoszenia.
\par 32 Tak aby tych, którzy niezboznie przeciwko przymierzu postepowac beda, w obludzie pochlebstwem utwierdzil, a zeby lud znajacy Boga swego imali, co tez uczynia.
\par 33 Zaczem ci, którzy nauczaja lud, którzy nauczaja wielu, padac beda od miecza i od ognia, od pojmania i od lupu przez wiele dni.
\par 34 A gdy padac beda, mala pomoc miec beda; bo sie do nich wiele pochlebców przylaczy.
\par 35 A z tych, którzy innych nauczaja, padac beda, aby doswiadczeni i oczyszczeni, i wybieleni byli az do czasu zamierzonego; bo to jeszcze potrwa az do czasu zamierzonego.
\par 36 Tak uczyni król wedlug woli swojej, i podniesie sie i wielmoznym sie uczyni nad kazdego boga, i przeciwko Bogu nad bogami dziwne rzeczy mówic bedzie, i poszczesci mu sie, az sie dokona gniew, azby sie to, co jest postanowiono, wykonalo.
\par 37 Ani na bogów ojców swych nie bedzie dbal, ani o milosc niewiast, ani o zadnego boga dbac bedzie, przeto, ze sie nade wszystko wyniesie.
\par 38 A na miejscu Boga najmocniejszego czcic bedzie boga, którego nie znali ojcowie jego; czcic bedzie zlotem i srebrem i kamieniem drogiem i rzeczami kosztownemi.
\par 39 A tak dowiedzie tego, ze twierdze Najmocniejszego beda boga obcego; a których mu sie bedzie zdalo, tych rozmnozy slawe, i uczyni, aby panowali nad wiela, a rozdzieli im ziemie miasto zaplaty.
\par 40 A przy skonczeniu tego czasu bedzie sie z nim potykal król z poludnia; ale król pólnocny jako burza nan przyjdzie z wozami i z jezdnymi i z wiela okretów, a wtargnie w ziemie, i jako powódz przejdzie.
\par 41 Potem przyciagnie do ziemi ozdobnej, i wiele krain upadnie; wszakze ci ujda rak jego, Edomczycy i Moabczycy, i pierwociny synów Ammonowych.
\par 42 A gdy reke swa sciagnie na krainy, ani ziemia Egipska tego ujsc nie bedzie mogla.
\par 43 Bo opanuje skarby zlota i srebra, i wszystkie rzeczy drogie Egipskie, a Libijczycy i Murzynowie za nim pójda.
\par 44 W tem wiesci od wschodu slonca i od pólnocy przestrasza go; przetoz wyciagnie z popedliwoscia wielka, aby wygubil i zamordowal wielu.
\par 45 I rozbije namioty palacu swego miedzy morzami na górze ozdobnej swietobliwosci; a gdy przyjdzie do konca swego, nie bedzie mial nikogo na pomocy.

\chapter{12}

\par 1 Tego czasu powstanie Michal, ksiaze wielki, który sie zastawia za synami ludu twego; a bedzie czas ucisnienia, jakiego nie bylo, jako narody poczely byc, az do tego czasu; tego, mówie, czasu wyswobodzony bedzie lud twój, ktokolwiek znaleziony bedzie napisany w ksiegach.
\par 2 A wiele z tych, którzy spia w prochu ziemi, ocuca sie, jedni ku zywotowi wiecznemu, a drudzy na pohanbienie i na wzgarde wieczna.
\par 3 Ale ci, którzy innych nauczaja, swiecic sie beda jako swiatlosc na niebie, a którzy wielu ku sprawiedliwosci przywodza, jako gwiazdy na wieki wieczne.
\par 4 Ale ty, Danijelu! zamknij te slowa, i zapieczetuj te ksiege az do czau naznaczonego; bo to wiele ich przebiezy, a rozmnozy sie umiejetnosc.
\par 5 I widzialem ja Danijel, a oto drudzy dwaj stali, jeden stad nad brzegiem rzeki, a drugi z onad nad drugim brzegiem rzeki;
\par 6 I rzekl do meza obleczonego w szate lniana, który stal nad woda onej rzeki: Kiedyz przyjdzie koniec tym dziwnym rzeczom?
\par 7 I uslyszalem tego meza obleczonego w szate lniana, który stal nad woda onej rzeki, ze podnióslszy prawice swoje i lewice swoje ku niebu przysiagl przez Zyjacego na wieki, iz po zamierzonym czasie i po zamierzonych czasach i po polowie czasu, i gdy do szczetu rozproszy sila reki ludu swietego, tedy sie to wszystko wypelni.
\par 8 A gdym ja to slyszal a nie zrozumialem, rzeklem: Panie mój! cóz za koniec bedzie tych rzeczy?
\par 9 Tedy rzekl: Idz, Danijelu! bo zawarte i zapieczetowane sa te slowa az do czasu zamierzonego.
\par 10 Oczyszczonych i wybielonych i doswiadczonych wiele bedzie, a niezbozni niezboznie czynic beda; nadto wszyscy niezbozni nie zrozumieja, ale madrzy zrozumieja.
\par 11 A od tego czasu, którego odjeta bedzie ofiara ustawiczna, a postawiona bedzie obrzydliwosc spustoszenia, bedzie dni tysiac, dwiescie i dziewiecdziesiat.
\par 12 Blogoslawiony, kto doczeka a dojdzie do tysiaca trzech set trzydziestu i pieciu dni.
\par 13 Ale ty idz do miejsca twego, a odpoczniesz, i zostaniesz w losie twoim az do skonczenia dni.


\end{document}