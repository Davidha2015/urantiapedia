\begin{document}

\title{Marka}


\chapter{1}

\par 1 Ten jest poczatek Ewangielii Jezusa Chrystusa, Syna Bozego.
\par 2 Jako napisano w prorokach: Oto Ja posylam Aniola mego przed obliczem twojem, który zgotuje droge twoje przed toba.
\par 3 Glos wolajacego na puszczy: Gotujcie droge Panska, proste czyncie sciezki jego.
\par 4 Jan chrzcil na puszczy, i kazal chrzest pokuty na odpuszczenie grzechów.
\par 5 I wychodzila do niego wszystka kraina Judzka, i Jeruzalemczycy, a wszyscy byli od niego chrzczeni w rzece Jordanie, wyznawajac grzechy swoje.
\par 6 Ale Jan przyodziany byl sierscia wielbladowa, a pas skórzany byl okolo biódr jego, a jadal szarancze i miód lesny.
\par 7 I kazal, mówiac: Idzie za mna mozniejszy nizeli ja, któremum nie jest godzien, schyliwszy sie, rozwiazac rzemyka u obuwia jego.
\par 8 Jamci was chrzcil woda; ale on was bedzie chrzcil Duchem Swietym.
\par 9 I stalo sie w one dni, ze przyszedl Jezus z Nazaretu Galilejskiego, a ochrzczony jest od Jana w Jordanie.
\par 10 A zarazem wystapiwszy z wody, ujrzal rozstepujace sie niebiosa, i Ducha jako golebice na niego zstepujacego.
\par 11 I stal sie glos z nieba: Tys jest on Syn mój mily, w którym mi sie upodobalo.
\par 12 A zaraz Duch wypedzil go na puszcza.
\par 13 I byl tam na puszczy przez czterdziesci dni, bedac kuszony od szatana, a byl z zwierzety, a Aniolowie sluzyli mu.
\par 14 Lecz potem, gdy Jan byl podany do wiezienia, przyszedl Jezus do Galilei, kazac Ewangielije królestwa Bozego,
\par 15 A mówiac: Wypelnil sie czas i przyblizylo sie królestwo Boze: Pokutujcie, a wierzcie Ewangielii.
\par 16 A przechodzac sie nad morzem Galilejskiem, ujrzal Szymona i Andrzeja, brata jego, zapuszczajacych siec w morze; bo byli rybitwi.
\par 17 I rzekl im Jezus: Pójdzcie za mna, a uczynie was rybitwami ludzi.
\par 18 A oni zarazem opusciwszy sieci swoje, poszli za nim.
\par 19 A stamtad troszeczke odszedlszy, ujrzal Jakóba, syna Zebedeuszowego, i Jana, brata jego, a oni w lodzi poprawiali sieci.
\par 20 I zaraz ich powolal; a oni zostawiwszy ojca swego Zebedeusza w lodzi z czeladzia, poszli za nim.
\par 21 Potem weszli do Kapernaum; a zaraz w sabat wszedlszy Jezus do bóznicy, nauczal.
\par 22 I zdumiewali sie nad nauka jego; albowiem on ich uczyl jako majacy moc, a nie jako nauczeni w Pismie.
\par 23 A byl w bóznicy ich czlowiek majacy ducha nieczystego, który zawolal,
\par 24 Mówiac: Ach! cóz my z toba mamy, Jezusie Nazarenski? Przyszedles, abys nas wytracil. Znam cie, ktos jest, zes on swiety Bozy.
\par 25 I zgromil go Jezus, mówiac: Umilknij, a wynijdz z niego.
\par 26 Tedy rozdarlszy go duch nieczysty i zawolawszy glosem wielkim, wyszedl z niego.
\par 27 I wylekli sie wszyscy, tak iz sie pytali miedzy soba, mówiac: Cóz to jest? cóz to za nowa nauka, iz moca i duchom nieczystym rozkazuje, i sa mu posluszni?
\par 28 I rozeszla sie powiesc o nim predko po wszystkiej krainie, lezacej okolo Galilei.
\par 29 A wyszedlszy zaraz z bóznicy, przyszli do domu Szymonowego i Andrzejowego z Jakóbem i z Janem.
\par 30 A swiekra Szymonowa lezala, majac goraczke, o której mu wnet powiedzieli.
\par 31 Tedy przystapiwszy podniósl ja, ujawszy ja za reke jej; a zaraz ja goraczka opuscila, i poslugowala im.
\par 32 A gdy byl wieczór i slonce zachodzilo, przynosili do niego wszystkie, którzy sie zle mieli, i opetane;
\par 33 A wszystko miasto zgromadzilo sie do drzwi.
\par 34 I uzdrowil wiele tych, co sie zle mieli na rozliczne choroby, i wygnal wiele dyjablów, a nie dopuscil mówic dyjablom; bo go znali.
\par 35 A bardzo rano przede dniem wstawszy, wyszedl i odszedl na puste miejsce, a tam sie modlil.
\par 36 I poszli za nim Szymon i ci, którzy z nim byli;
\par 37 A znalazlszy go, rzekli mu: Wszyscy cie szukaja.
\par 38 Tedy im on rzekl: Idzmy do przyleglych miasteczek, abym i tam kazal; bom na to przyszedl.
\par 39 I kazal w bóznicach ich po wszystkiej Galilei, i wyganial dyjably.
\par 40 Tedy przyszedl do niego tredowaty, proszac go i upadajac przed nim na kolana, i mówiac mu: Jezli chcesz, mozesz mie oczyscic.
\par 41 A tak Jezus uzaliwszy sie, wyciagnal reke, a dotknal sie go i rzekl mu: Chce, badz oczyszczony!
\par 42 A gdy to Pan rzekl, zarazem odszedl trad od niego, i byl oczyszczony.
\par 43 A srodze mu przygroziwszy Jezus, zaraz go odprawil;
\par 44 I rzekl mu: Patrz, abys nikomu nic nie powiadal, ale idz, a ukaz sie kaplanowi, i ofiaruj za oczyszczenie twoje to, co rozkazal Mojzesz, na swiadectwo przeciwko nim.
\par 45 Ale on odszedlszy, poczal wiele opowiadac i rozslawiac te rzecz, tak iz juz nie mógl Jezus jawnie wnijsc do miasta, ale byl na ustroniu na miejscach pustych. I schodzili sie do niego zewszad.

\chapter{2}

\par 1 A zasie przyszedl po kilku dniach do Kapernaum, i uslyszano, ze jest w domu.
\par 2 A wnet zeszlo sie ich tak wiele, ze sie zmiescic nie mogli ani przede drzwiami; i opowiadal im slowo Boze.
\par 3 Tedy przyszli do niego niosacy powietrzem ruszonego, którego czterej niesli.
\par 4 A gdy do niego przystapic nie mogli dla cizby, odarli dach, gdzie byl Jezus, a przelamawszy go, spuscili po powrozach na dól loze, na którem lezal powietrzem ruszony.
\par 5 A widzac Jezus wiare ich, rzekl powietrzem ruszonemu: Synu! odpuszczone sa tobie grzechy twoje.
\par 6 A byli tam niektórzy z nauczonych w Pismie, siedzac i myslac w sercach swoich:
\par 7 Czemuz ten takie mówi bluznierstwa? któz moze grzechy odpuszczac, tylko sam Bóg?
\par 8 A zaraz poznawszy Jezus duchem swym, iz tak w sobie mysleli, rzekl im: Czemuz tak myslicie w sercach waszych?
\par 9 Cóz latwiejszego jest, rzec powietrzem ruszonemu: Odpuszczone sa tobie grzechy, czyli rzec: Wstan i wezmij loze twoje, a chodz?
\par 10 Ale zebyscie wiedzieli, iz Syn czlowieczy ma moc na ziemi grzechy odpuszczac, rzekl powietrzem ruszonemu:
\par 11 Tobie mówie: Wstan, a wezmij loze twoje, a idz do domu twego.
\par 12 A on zarazem wstal, i wziawszy loze swoje, wyszedl przed wszystkimi, tak iz sie wszyscy zdumiewali i chwalili Boga, mówiac: Nigdysmy nic takiego nie widzieli.
\par 13 I wyszedl zasie nad morze, a wszystek lud przychodzil do niego, i nauczal je.
\par 14 A idac mimo cla, ujrzal Lewiego, syna Alfeuszowego, siedzacego na cle, i rzekl mu: Pójdz za mna! a on wstawszy szedl za nim.
\par 15 I stalo sie, gdy Jezus siedzial za stolem w domu jego, ze wiele celników i grzeszników wespól siedzialo z Jezusem i z uczniami jego; bo ich wiele bylo, i chodzili za nim.
\par 16 A nauczeni w Pismie i Faryzeuszowie widzac go, ze jadl z celnikami i z grzesznikami, mówili do uczniów jego: Cóz jest, iz z celnikami i z grzesznikami je i pije?
\par 17 A uslyszawszy to Jezus, rzekl im: Nie potrzebuja zdrowi lekarza, ale ci, co sie zle maja; nie przyszedlem, wzywac sprawiedliwych, ale grzesznych do pokuty.
\par 18 A uczniowie Janowi i Faryzejscy poscili, a przyszedlszy mówili do niego: Czemuz uczniowie Janowi i Faryzejscy poszcza, a twoi uczniowie nie poszcza?
\par 19 I rzekl im Jezus: Izali moga synowie loznicy malzenskiej poscic, póki z nimi jest oblubieniec? Póki z soba oblubienca maja, nie moga poscic.
\par 20 Ale przyjda dni, gdy od nich odjety bedzie oblubieniec, a tedy beda poscic w one dni.
\par 21 A zaden nie wprawuje laty sukna nowego w szate wiotcha, inaczej ona jego lata nowa ujmuje nieco od wiotchej szaty, i stawa sie gorsze rozdarcie.
\par 22 I zaden nie leje wina mlodego w stare statki; bo inaczej wino mlode rozsadza statki, i wycieka wino, a statki sie psuja; ale wino nowe ma byc wlewane w statki nowe.
\par 23 I stalo sie, ze szedl Jezus w sabat przez zboza, i poczeli uczniowie jego idac rwac klosy.
\par 24 Ale Faryzeuszowie mówili do niego: Oto czemu ci czynia w sabat, czego sie nie godzi czynic?
\par 25 A on im rzekl: Nigdyscie nie czytali, co uczynil Dawid, gdy niedostatek cierpial, a laknal, sam i ci, którzy z nim byli?
\par 26 Jako wszedl do domu Bozego za Abijatara, kaplana najwyzszego, i jadl chleby pokladne, (których sie nie godzilo jesc, tylko kaplanom), a dal i tym, którzy z nim byli.
\par 27 Dotego rzekl im: Sabat dla czlowieka uczyniony, a nie czlowiek dla sabatu.
\par 28 Dlatego Syn czlowieczy jest Panem i sabatu.

\chapter{3}

\par 1 Wszedl zasie do bóznicy, a byl tam czlowiek, który mial reke uschla.
\par 2 I podstrzegali go, jezliby go w sabat uzdrowil, aby go oskarzyli.
\par 3 I rzekl onemu czlowiekowi, który mial reke uschla: Wystap w posrodek.
\par 4 I rzekl do nich: Godzili sie w sabat dobrze czynic, czyli zle czynic? czlowieka zachowac, czyli zabic? a oni milczeli.
\par 5 Tedy spojrzawszy po nich z gniewem i zasmuciwszy sie nad zatwardzeniem serca ich, rzekl onemu czlowiekowi: Wyciagnij reke twoje! i wyciagnal; i przywrócona jest reka jego do zdrowia jako i druga.
\par 6 Tedy wyszedlszy Faryzeuszowie, uczynili wnet rade z Herodyjany przeciwko niemu, jakoby go stracili.
\par 7 Ale Jezus z uczniami swymi ustapil nad morze, a wielkie mnóstwo szlo za nim z Galilei i z ziemi Judzkiej,
\par 8 I z Jeruzalemu, i z Idumei, i zza Jordanu, i z tych, którzy mieszkali okolo Tyru i Sydonu, wielkie mnóstwo, slyszac, jak wielkie rzeczy czynil, przyszli do niego.
\par 9 I rozkazal uczniom swoim, aby lódke mieli zawsze w pogotowiu, dla ludu, aby go nie cisneli.
\par 10 Albowiem ich wiele uzdrawial, tak iz nan padali, aby sie go dotykali, którzykolwiek choroby mieli.
\par 11 A duchowie nieczysci, gdy go ujrzeli, upadali przed nim i wolali, mówiac: Ty jestes Syn Bozy!
\par 12 Ale ich on srodze gromil, zeby go nie objawiali.
\par 13 I wstapil na góre, a wezwal do siebie tych, których sam chcial, i przyszli do niego.
\par 14 I postanowil ich dwanascie, aby z nim byli, a izby je wyslal kazac Ewangielije:
\par 15 I zeby mieli moc uzdrawiac choroby i wyganiac dyjably:
\par 16 Szymona, któremu imie dal Piotr;
\par 17 I Jakóba, syna Zebedeuszowego, i Jana, brata Jakóbowego, (którym dal imie Boanerges, to jest: synowie gromu);
\par 18 I Andrzeja, i Filipa, i Bartlomieja, i Mateusza, i Tomasza, i Jakóba, syna Alfeuszowego, i Tadeusza, i Szymona Kananejczyka;
\par 19 I Judasza Iszkaryjota, który go tez wydal.
\par 20 I przyszli do domu. I zgromadzil sie znowu lud, tak iz nie mogli ani chleba jesc.
\par 21 A gdy o tem uslyszeli jego powinni, przyszli, aby go pojmali; bo mówili, ze odszedl od rozumu.
\par 22 A nauczeni w Pismie, którzy byli przyszli z Jeruzalemu, mówili: Iz ma Beelzebuba, a iz przez ksiazecia dyjabelskiego wygania dyjably.
\par 23 I wezwawszy ich, mówil do nich w podobienstwach: Jakoz moze szatan szatana wyganiac?
\par 24 A jezlize królestwo samo w sobie bedzie rozdzielone, nie moze sie ostac ono królestwo.
\par 25 I dom, jezliby sam przeciwko sobie byl rozdzielony, nie bedzie sie mógl ostac on dom.
\par 26 Takci, jezli szatan powstal sam przeciwko sobie i jest rozdzielony, nie moze sie ostac, ale koniec bierze.
\par 27 Nikt nie moze sprzetu mocarzowego, wszedlszy do domu jego, rozchwycic, jezliby pierwej mocarza onego nie zwiazal; a potem dom jego spladruje.
\par 28 Zaprawde powiadam wam, ze wszystkie grzechy synom ludzkim beda odpuszczone, i bluznierstwa, któremibykolwiek bluznili;
\par 29 Ale kto bluzni przeciwko Duchowi Swietemu, nie ma odpuszczenia na wieki, ale winien jest sadu wiecznego.
\par 30 Bo mówili: Ma ducha nieczystego.
\par 31 Przyszli tedy bracia i matka jego, a stojac przed domem, poslali do niego, i kazali go zawolac.
\par 32 A lud siedzial okolo niego, i rzekli mu: Oto matka twoja i bracia twoi szukaja cie przed domem.
\par 33 Ale im on odpowiedzial, mówiac: Któz jest matka moja, i bracia moi?
\par 34 A spojrzawszy wokolo po tych, którzy kolo niego siedzieli, rzekl: Oto matka moja, i bracia moi!
\par 35 Albowiem ktobykolwiek czynil wole Boza, ten jest brat mój, i siostra moja, i matka moja.

\chapter{4}

\par 1 I poczal zasie uczyc przy morzu; i zgromadzil sie do niego lud wielki, tak iz wstapiwszy w lódz, siedzial na morzu, a wszystek lud byl przy morzu na ziemi.
\par 2 I nauczal ich wiele rzeczy w podobienstwach, a mówil do nich w nauce swojej: Sluchajcie!
\par 3 Oto wyszedl rozsiewca, aby rozsiewal.
\par 4 I stalo sie, gdy rozsiewal, ze jedno padlo podle drogi, a ptaki niebieskie przylecialy i podziobaly je.
\par 5 Drugie zasie padlo na miejsce opoczyste, gdzie nie mialo wiele ziemi; i predko weszlo, przeto iz nie mialo glebokosci ziemi;
\par 6 A gdy slonce weszlo, wygorzalo, a iz korzenia nie mialo, uschlo.
\par 7 A drugie padlo miedzy ciernie; i wzrosly ciernie i zadusily je, i nie wydalo pozytku.
\par 8 Drugie zasie padlo na ziemie dobra, i wydalo pozytek bujno wschodzacy i rosnacy; i przynioslo jedno trzydziesiatny, a drugie szescdziesiatny, a drugie setny.
\par 9 I mówil im: Kto ma uszy ku sluchaniu, niechaj slucha.
\par 10 A gdy sam tylko byl, pytali go ci, co przy nim byli ze dwunastoma, o to podobienstwo.
\par 11 A on im odpowiedzial: Wam dano wiedziec tajemnice królestwa Bozego; ale tym, którzy sa obcymi, wszystko sie podawa w podobienstwach;
\par 12 Aby patrzac patrzeli, ale nie widzieli, i slyszac slyszeli, ale nie zrozumieli, by sie snac nie nawrócili, a bylyby im grzechy odpuszczone.
\par 13 Zatem rzekl do nich: Nie rozumiecie tego podobienstwa? A jakoz zrozumiecie wszystkie inne podobienstwa?
\par 14 Rozsiewca on rozsiewa slowo.
\par 15 A którzy podle drogi, ci sa, którym sie rozsiewa slowo; ale gdy uslyszeli, zaraz przychodzi szatan, a wybiera slowo wsiane w serca ich.
\par 16 Takze i ci, którzy na opoczystych miejscach posiani sa, ci sa, którzy, gdy uslyszeli slowo, zaraz je z radoscia przyjmuja;
\par 17 Wszakze nie maja korzenia w sobie, ale sa doczesnymi; potem, gdy przychodzi ucisk albo przesladowanie dla slowa, wnet sie gorsza;
\par 18 A którzy miedzy cierniem sa posiani, ci sa, którzy sluchaja slowa;
\par 19 Ale pieczolowanie swiata tego i omamienie bogactw, i pozadliwosci innych rzeczy, wszedlszy zaduszaja slowo, i staje sie bez pozytku.
\par 20 A którzy na dobra ziemie przyjeli nasienie, ci sa, co sluchaja slowa, i przyjmuja je, przynosza pozytek, jedno trzydziesiatny, a drugie szescdziesiatny, a drugie setny.
\par 21 Nadto mówil im: Izali przynosza swiece, aby wstawiona byla pod korzec albo pod loze? izali nie dlatego, aby ja na swiecznik wstawiono?
\par 22 Bo nic nie masz tajemnego, co by nie mialo byc objawiono; ani sie stalo co skrytego, aby na jaw nie wyszlo.
\par 23 Jezli kto ma uszy ku sluchaniu, niechaj slucha!
\par 24 I rzekl do nich: Patrzciez, czego sluchacie; jaka miara mierzycie, taka wam bedzie odmierzono, a bedzie wam przydano, którzy sluchacie.
\par 25 Albowiem kto ma, bedzie mu dano; a kto nie ma, i to, co ma, bedzie od niego odjeto.
\par 26 I mówil: Takie jest królestwo Boze, jako gdyby czlowiek wrzucil nasienie w ziemie;
\par 27 A spalby i wstawalby we dnie i w nocy, a nasienie by weszlo i uroslo, gdy on nie wie.
\par 28 Boc ziemia sama z siebie pozytek wydawa, naprzód trawe, potem klos, a potem zupelne zboze w klosie.
\par 29 A skoro sie okaze urodzaj, wnet gospodarz zapuszcza sierp; bo zniwo przyszlo.
\par 30 Nad to rzekl: Do czego przypodobamy królestwo Boze, albo którem je podobienstwem wyrazimy?
\par 31 Jest jako ziarno gorczyczne; które, gdy wsiane bywa w ziemie, najmniejsze jest ze wszystkich nasion, które sa na ziemi.
\par 32 Ale gdy bywa wsiane, wzrasta, i bywa najwieksze nad wszystkie jarzyny, i rozpuszcza galezie wielkie, tak iz pod cieniem jego moga sobie czynic gniazda ptaki niebieskie.
\par 33 I przez wiele takich podobienstw mówil do nich slowo, tak jako sluchac mogli.
\par 34 A bez podobienstwa nie mówil do nich; wszakze uczniom swym wszystko z osobna wykladal.
\par 35 I rzekl do nich w tenze dzien, gdy juz byl wieczór: Przeprawmy sie na druga strone.
\par 36 A rozpusciwszy lud, wzieli go z soba, tak jako byl w lodzi; ale i inne lódki byly przy nim.
\par 37 Tedy powstala wielka nawalnosc wiatru, a waly bily na lódz, tak ze sie juz napelniala.
\par 38 A on na zadzie lodzi spal na wezglówku; i obudzili go i mówili mu: Nauczycielu! nie dbasz, ze giniemy?
\par 39 A tak ocknawszy sie, zgromil wiatr, i rzekl morzu: Umilknij, a usmierz sie. I przestal wiatr, a stalo sie wielkie uciszenie.
\par 40 Zatem rzekl im: Przecz jestescie tak bojazliwi? Jakoz nie macie wiary?
\par 41 I bali sie bojaznia wielka, i mówili jedni do drugich: Któz wzdy ten jest, ze mu i wiatr i morze sa posluszne?

\chapter{5}

\par 1 Tedy sie przeprawili za morze do krainy Gadarenczyków.
\par 2 A gdy on wyszedl z lodzi, zaraz mu zabiezal z grobów czlowiek majacy ducha nieczystego;
\par 3 Który mial mieszkanie w grobach, a nie mógl go nikt i lancuchami zwiazac,
\par 4 Przeto ze on czesto bedac petami i lancuchami zwiazany, lancuchy porwal, i peta pokruszyl; a nie mógl go nikt ukrócic.
\par 5 A zawsze we dnie i w nocy na górach w grobach byl, wolajac i kamieniem sie tlukac.
\par 6 Ujrzawszy tedy Jezusa z daleka, biezal i poklonil mu sie;
\par 7 A wolajac glosem wielkim, rzekl: Cóz mam z toba Jezusie, Synu Boga najwyzszego? Poprzysiegam cie przez Boga, abys mie nie trapil.
\par 8 (Albowiem mu mówil: Wynijdz, duchu nieczysty! z tego czlowieka.)
\par 9 Tedy go pytal: Co masz za imie? A on odpowiadajac, rzekl: Imie moje jest wojsko: albowiem nas jest wiele.
\par 10 I prosil go bardzo, aby ich nie wyganial z onej krainy.
\par 11 A byla tam przy górach wielka trzoda swini, która sie pasla.
\par 12 I prosili go oni wszyscy dyjabli, mówiac: Pusc nas w te swinie, abysmy w nie weszli.
\par 13 I pozwolil im zaraz Jezus. A wyszedlszy oni duchowie nieczysci, weszli w one swinie; i porwala sie ona trzoda z przykra w morze (a bylo ich okolo dwóch tysiecy,)i potonely w morzu.
\par 14 A oni, którzy swinie pasli, uciekli, i oznajmili to w miescie i we wsiach; i wyszli, aby ogladali to, co sie stalo.
\par 15 I przyszli do Jezusa, i ujrzeli onego, który byl opetany, i siedzial obleczony, bedac przy dobrem baczeniu; onego, mówie, w którym bylo wojsko dyjablów; i bali sie.
\par 16 A ci, którzy to widzieli, opowiedzieli im, co sie dzialo z onym opetanym, i o swiniach.
\par 17 Tedy go poczeli prosic, aby odszedl z granic ich.
\par 18 A gdy wstapil w lódz, prosil go on, co byl opetanym, aby byl przy nim.
\par 19 Lecz mu Jezus nie dopuscil, ale mu rzekl: Idz do domu swego, do swoich, a oznajmij im, jakoc wielkie rzeczy Pan uczynil, a jako sie nad toba zmilowal.
\par 20 Tedy odszedl, i poczal opowiadac w dziesieciu miastach, jako mu wielkie rzeczy uczynil Jezus; i dziwowali sie wszyscy.
\par 21 A gdy sie zasie Jezus przeprawil w lodzi na druga strone, zebral sie do niego wielki lud; a on byl nad morzem.
\par 22 A oto przyszedl jeden z przelozonych bóznicy, imieniem Jairus, a ujrzawszy go, przypadl do nóg jego.
\par 23 I prosil go wielce, mówiac: Poniewaz córeczka moja kona, pójdzze, wlóz na nie rece, aby byla uzdrowiona, i bedzie zywa. I poszedl z nim.
\par 24 I szedl za nim lud wielki, i cisneli go.
\par 25 Tedy niektóra niewiasta, która cierpiala plynienie krwi ode dwunastu lat.
\par 26 I wiele ucierpiala od wielu lekarzy, i wynalozyla wszystko, co miala; a nic jej nie pomoglo, owszem sie jej tem wiecej pogorszalo:
\par 27 Uslyszawszy o Jezusie, przyszla z tylu miedzy ludem, i dotknela sie szaty jego;
\par 28 Bo mówila: Jezli sie tylko dotkne szaty jego, bede uzdrowiona.
\par 29 A zarazem wyschlo zródlo krwi jej, i poczula na ciele, ze uzdrowiona byla od choroby swojej.
\par 30 A wnet poznawszy Jezus sam w sobie, ze z niego moc wyszla, obrócil sie do ludu i rzekl: Kto sie dotknal szat moich?
\par 31 I rzekli mu uczniowie jego: Widzisz, ze cie ten lud cisnie, a mówisz: Kto sie mnie dotknal?
\par 32 I spojrzal w kolo, aby ujrzal te, która to uczynila:
\par 33 Ale niewiasta ona z bojaznia i ze drzeniem, wiedzac, co sie przy niej stalo, przystapila i upadla przed nim, a powiedziala mu wszystke prawde.
\par 34 Zatem jej on rzekl: Córko! wiara twoja ciebie uzdrowila, idzze w pokoju, a badz zdrowa od choroby twojej.
\par 35 A gdy on jeszcze mówil, przyszli sludzy od przelozonego bóznicy, mówiac: Córka twoja umarla, czemuz jeszcze trudzisz nauczyciela?
\par 36 Ale Jezus skoro uslyszal to, co oni mówili, rzekl do przelozonego bóznicy: Nie bój sie, tylko wierz!
\par 37 I nie dopuscil nikomu isc za soba, tylko Piotrowi, i Jakóbowi, i Janowi, bratu Jakóbowemu.
\par 38 A przyszedl do domu przelozonego bóznicy, i ujrzal tam zgielk, i placzace i bardzo narzekajace.
\par 39 Wszedlszy tedy, rzekl im: Przecz zgielk czynicie i placzecie? nie umarlac dzieweczka, ale spi.
\par 40 I nasmiewali sie z niego. Ale on wygnawszy wszystkie, wzial z soba ojca i matke dzieweczki, i te, którzy przy nim byli, i wszedl tam, gdzie dzieweczka lezala.
\par 41 A ujawszy za reke one dzieweczke, rzekl do niej: Talita kumi! co sie wyklada: Dzieweczko (tobie mówie) wstan!
\par 42 A zaraz dzieweczka wstala, i chodzila; albowiem byla w dwunastym roku. I zdumieli sie zdumieniem wielkiem.
\par 43 Tedy im przykazal wielce, aby tego nikt nie wiedzial, i rozkazal, aby jej dano jesc.

\chapter{6}

\par 1 A wyszedlszy stamtad przyszedl do ojczyzny swojej, i szli za nim uczniowie jego.
\par 2 A gdy przyszedl sabat, poczal uczyc w bóznicy, a wiele ich sluchajac, zdumiewali sie i mówili: Skadze temu to wszystko? a co to za madrosc, która mu jest dana, ze sie i takie mocy dzieja przez rece jego?
\par 3 Izali ten nie jest ciesla, syn Maryi, i brat Jakóba, i Jozesa, i Judasa, i Szymona? Azaz tu nie masz i sióstr jego u nas? I gorszyli sie z niego.
\par 4 Ale Jezus rzekl do nich: Nie jestci prorok beze czci, chyba w ojczyznie swojej, a miedzy pokrewnymi, i w domu swoim.
\par 5 I nie mógl tam uczynic zadnego cudu, oprócz iz niektóre chore, wkladajac na nie rece, uzdrowil.
\par 6 A dziwowal sie niedowiarstwu ich, i obchodzil okoliczne miasteczka, nauczajac.
\par 7 Tedy zwolawszy do siebie onych dwunastu, poczal je po dwóch rozsylac, i dal im moc nad duchami nieczystymi.
\par 8 I rozkazal im, aby nic nie brali na droge, jedno tylko laske: ani taistry, ani chleba, ani w trzos pieniedzy;
\par 9 Ale zeby sie obuli w trzewiki, a nie obloczyli dwóch sukien.
\par 10 Zatem mówil do nich: Gdziekolwiek wnijdziecie w dom, tam zostancie, póki byscie stamtad nie wyszli.
\par 11 A którzykolwiek by was nie przyjeli, ani was sluchali, wyszedlszy stamtad, otrzasnijcie proch z nóg waszych na swiadectwo im; zaprawde powiadam wam: Lzej bedzie Sodomie i Gomorze w dzien sadny, niz miastu onemu.
\par 12 Tedy wyszedlszy kazali, aby ludzie pokutowali.
\par 13 I wyganiali wiele dyjablów, i wiele chorych olejkiem mazali i uzdrawiali je.
\par 14 A uslyszal o tem król Herod, (bo sie imie jego stalo rozslawione,)i rzekl: Jan Chrzciciel zmartwychwstal, dlatego sie cuda dzieja przez niego.
\par 15 A drudzy mówili: Elijasz to jest; drudzy zas mówili: Prorok to jest, albo jako jeden z onych proroków.
\par 16 Co uslyszawszy Herod, rzekl: Ten jest Jan, któregom ja scial, on zmartwychwstal.
\par 17 Albowiem tenze Herod poslawszy pojmal Jana, i wsadzil go do wiezienia dla Herodyjady, zony Filipa, brata swego, iz ja byl pojal za zone.
\par 18 Bo Jan mówil Herodowi: Nie godzi sie miec zony brata twego.
\par 19 A Herodyjas czyhala nan, i chciala go zabic, ale nie mogla;
\par 20 Albowiem Herod obawial sie Jana, wiedzac, iz byl mezem sprawiedliwym i swietym; i ogladal sie nan, i sluchajac go, wiele czynil i rad go sluchal.
\par 21 A gdy przyszedl dzien sposobny, którego Herod, obchodzac pamiatke narodzenia swego, wieczerza sprawil na ksiazeta swoje i na hetmany i na przedniejsze z Galilei;
\par 22 A gdy weszla córka onej Herodyjady i tancowala, i podobala sie Herodowi i spólsiedzacym, rzekl król do dzieweczki: Pros mie o co chcesz, a dam ci.
\par 23 I przysiagl jej: O cokolwiek bys mie prosila, dam ci, az do polowy królestwa mego.
\par 24 Ona tedy wyszedlszy, rzekla matce swojej: O co mam prosic? A ona rzekla: O glowe Jana Chrzciciela.
\par 25 A tak ona zaraz wszedlszy predko do króla, prosila mówiac: Chce, abys mi teraz dal na misie glowe Jana Chrzciciela.
\par 26 I zasmucil sie król bardzo, wszakze dla przysiegi i dla spólsiedzacych nie chcial jej odmówic.
\par 27 A zarazem poslawszy król kata, rozkazal przyniesc glowe jego.
\par 28 A on poszedlszy scial go w wiezieniu, i przyniósl glowe jego na misie, a dal ja dzieweczce, a dzieweczka dala ja matce swojej.
\par 29 Co gdy uslyszeli uczniowie jego, przyszli i wzieli cialo jego, i polozyli je w grobie.
\par 30 A Apostolowie zszedlszy sie do Jezusa, opowiedzieli mu wszystko, i co czynili, i czego uczyli.
\par 31 I rzekl im: Pójdzcie wy sami osobno na miejsce puste, a odpocznijcie troche; bo ich wiele bylo, co przychodzili, i odchodzili, tak iz nie mieli wolnego czasu, zeby jedli.
\par 32 I odjechali w lodzi na miejsce puste osobno.
\par 33 A widzac je lud, ze odjezdzali, poznalo go wiele ich, i zbiezeli sie tam pieszo ze wszystkich miast, i poprzedzili je, i zgromadzili sie do niego.
\par 34 A wyszedlszy Jezus ujrzal wielki lud, i uzalil sie ich, iz byli jako owce nie majace pasterza, i poczal ich nauczac wiele rzeczy.
\par 35 A gdy juz czas mijal, przystapiwszy do niego uczniowie jego, rzekli: To miejsce jest puste, a juz czas mija.
\par 36 Rozpusc je, aby poszedlszy do okolicznych wsi i miasteczek, nakupili sobie chleba; bo nie maja, coby jedli.
\par 37 A on odpowiadajac, rzekl im: Dajcie wy im jesc. I rzekli mu: Szedlszy kupimy za dwiescie groszy chleba, a damy im jesc?
\par 38 A on im rzekl: Wielez chleba macie? Idzcie, a dowiedzcie sie. A oni dowiedziawszy sie, powiedzieli: Piecioro, i dwie ryby.
\par 39 Tedy im kazal wszystkie gromadami posadzic na zielonej trawie.
\par 40 I usiedli rzad podle rzadu, tu po stu, tu zas po piecdziesiat.
\par 41 A wziawszy one piec chlebów, i one dwie ryby, wejrzawszy w niebo, blogoslawil. I polamal one chleby i dawal uczniom swoim, aby kladli przed nie; i one dwie ryby rozdzielil miedzy wszystkie.
\par 42 I jedli wszyscy, i nasyceni byli.
\par 43 I zebrali ulomków, dwanascie koszów pelnych, i z onych ryb.
\par 44 A bylo tych, którzy jedli chleby, okolo pieciu tysiecy mezów.
\par 45 I wnet przymusil ucznie swoje, aby wstapili w lódz, i uprzedzili go na druga strone ku Betsaidzie, azby on rozpuscil lud.
\par 46 A odprawiwszy je, odszedl na góre, aby sie modlil.
\par 47 A gdy byl wieczór, byla lódz w posród morza, a on sam byl na ziemi.
\par 48 I widzial, ze sie spracowali, wioslami robiac; (bo wiatr mieli przeciwny,)a tak okolo czwartej strazy nocnej przyszedl do nich, chodzac po morzu, i chcial je wyminac.
\par 49 Ale oni ujrzawszy go chodzacego po morzu, mniemali, zeby byla obluda, i krzykneli:
\par 50 (Bo go wszyscy widzieli, i wylekli sie.) Ale zaraz przemówil do nich, i rzekl im: Ufajcie, jam jest; nie bójcie sie!
\par 51 I wstapil do nich w lódz, i uciszyl sie wiatr; a oni sie sami w sobie nader zdumiewali i dziwowali.
\par 52 Bo nie zrozumieli z strony chlebów, gdyz serce ich bylo zdretwialo.
\par 53 A przeprawiwszy sie przyszli do ziemi Gienezaret i przybili sie do brzegu.
\par 54 A gdy wyszli z lodzi, zaraz ci, co go poznali,
\par 55 Obiezawszy wszystke one okoliczna kraine, poczeli nosic na lozach tych, którzy sie zle mieli, gdziekolwiek uslyszeli o nim, ze tam jest.
\par 56 A gdziekolwiek on wszedl do miasteczek, albo do miast, albo do wsi, kladli niemocne po ulicach, i prosili go, aby sie tylko dotykali podolka szaty jego; a ile sie go ich dotkneli, byli uzdrowieni.

\chapter{7}

\par 1 Tedy sie zgromadzili do niego Faryzeuszowie, i niektórzy z nauczonych w Pismie, którzy byli przyszli z Jeruzalemu;
\par 2 A ujrzawszy niektóre z uczniów jego, ze pospolitemi rekoma, (to jest, nieumytemi) jedli chleb, ganili to.
\par 3 Albowiem Faryzeuszowie i wszyscy Zydzi nie jedza, jezliby pilnie rak nie umyli, trzymajac ustawe starszych.
\par 4 I z rynku przyszedlszy, jezliby sie nie umyli, nie jedza; i innych rzeczy wiele jest, które przyjeli ku trzymaniu, jako umywanie kubków, konewek, i miednic, i stolów.
\par 5 Potem go pytali Faryzeuszowie i nauczeni w Pismie: Przecz uczniowie twoi nie chodza wedlug podania starszych, ale nieumytemi rekoma chleb jedza?
\par 6 Tedy on odpowiadajac, rzekl im: Dobrze Izajasz o was obludnikach prorokowal, jako napisano: Lud ten czci mie wargami, ale serce ich daleko jest ode mnie.
\par 7 Lecz prózno mie czcza nauczajac nauk i ustaw ludzkich.
\par 8 Albowiem wy opusciwszy przykazania Boze, trzymacie ustawy ludzkie, umywanie konewek i kubków, i wiele innych takich tym podobnych rzeczy czynicie.
\par 9 Mówil im tez: Wy czysto znosicie przykazania Boze, abyscie ustawy wasze zachowali.
\par 10 Bo Mojzesz rzekl: Czcij ojca twego i matke twoje; a kto zlorzeczy ojcu albo matce, niech smiercia umrze.
\par 11 Ale wy mówicie: Jezliby czlowiek rzekl ojcu albo matce: Korban, (co jest dar), którykolwiek bedzie ode mnie, tobie pozyteczny bedzie, bez winy bedzie;
\par 12 I nie dopuscicie mu nic wiecej czynic ojcu swemu albo matce swojej,
\par 13 Wniwecz obracajac slowo Boze ustawa wasza, którascie ustawili; i wiele innych rzeczy tym podobnych czynicie.
\par 14 A zwolawszy wszystkiego ludu, mówil im: Sluchajcie mie wszyscy, a zrozumiejcie!
\par 15 Nie masz nic z rzeczy zewnetrznych, które wchodza w czlowieka, co by go moglo pokalac; ale to, co pochodzi z niego, to jest, co pokala czlowieka.
\par 16 Jezli kto ma uszy ku sluchaniu, niechaj slucha!
\par 17 A gdy wszedl w dom od onego ludu, pytali go uczniowie jego o to podobienstwo.
\par 18 Tedy im rzekl: Takze i wy bezrozumni jestescie? Azaz nie rozumiecie, iz wszystko, co zewnatrz wchodzi w czlowieka, nie moze go pokalac?
\par 19 Albowiem nie wchodzi w serce jego, ale w brzuch, i do wychodu wychodzi, czyszczac wszystkie pokarmy.
\par 20 I powiedzial, ze co pochodzi z czlowieka, to pokala czlowieka.
\par 21 Bo z wnetrznosci serca ludzkiego pochodza zle mysli, cudzolóstwa, wszeteczenstwa, mezobójstwa,
\par 22 Kradzieze, lakomstwa, zlosci, zdrada, niewstyd, oko zle, bluznierstwo, pycha, glupstwo.
\par 23 Wszystkie te zle rzeczy pochodza z wnetrznosci, i pokalaja czlowieka.
\par 24 A stamtad wstawszy, poszedl na granice Tyru i Sydonu, a wszedlszy w dom, nie chcial, aby kto wiedzial; lecz sie utaic nie mógl.
\par 25 Albowiem uslyszawszy o nim niewiasta, której córeczka miala ducha nieczystego, przyszla i przypadla do nóg jego,
\par 26 (A ta niewiasta byla Grecka, rodem z Syrofenicyi) i prosila go, aby dyjabla wygnal z córki jej.
\par 27 Ale jej Jezus rzekl: Niech sie pierwej dzieci nasyca; boc nie jest dobra, brac chleb dzieciom i miotac szczenietom.
\par 28 A ona odpowiedziala i rzekla mu: Tak jest, Panie! Wszakze i szczenieta jadaja pod stolem z odrobin dziecinnych.
\par 29 I rzekl do niej: Dla tej mowy idz, wyszedl dyjabel z córki twojej.
\par 30 A gdy ona odeszla do domu swego, znalazla iz dyjabel wyszedl, a córka lezala na lozu.
\par 31 A wyszedlszy zas z granic Tyrskich i Sydonskich, przyszedl nad morze Galilejskie, posrodkiem granic dziesieciu miast.
\par 32 I przywiedli mu gluchego i z ciezkoscia mówiacego, a prosili go, aby na niego reke wlozyl.
\par 33 A wziawszy go Pan od ludu osobno, wlozyl palce swoje w uszy jego, a plunawszy dotknal sie jezyka jego;
\par 34 A wejrzawszy w niebo, westchnal i rzekl do niego: Efata! to jest, otwórz sie.
\par 35 I wnet sie otworzyly uszy jego, i rozwiazala sie zwiazka jezyka jego, i wymawial dobrze.
\par 36 Tedy im zakazal, aby tego nikomu nie powiadali.
\par 37 Ale czem on im bardziej zakazywal, tem oni to bardziej rozglaszali, i nader sie bardzo zdumiewali, mówiac: Dobrze wszystko uczynil; bo czyni, iz glusi slysza i niemi mówia.

\chapter{8}

\par 1 A w onez dni, gdy nader wielki lud byl, a nie mieli, co by jedli, zwolawszy Jezus uczniów swoich, rzekl im:
\par 2 Zal mi tego ludu; bo juz trzy dni trwaja przy mnie, a nie maja, co by jedli;
\par 3 A jezli je rozpuszcze glodne do domów ich, pomdleja na drodze; albowiem niektórzy z nich z daleka przyszli.
\par 4 Tedy mu odpowiedzieli uczniowie jego: Skadze te kto bedzie mógl nasycic chlebem tu na puszczy?
\par 5 I spytal ich: Wielez macie chlebów? A oni rzekli: Siedm.
\par 6 I rozkazal ludowi, zeby usiadl na ziemi; a wziawszy one siedm chlebów, podziekowawszy lamal, i dawal uczniom swoim, aby przed lud kladli; i kladli przed lud.
\par 7 Mieli tez troche rybek, które poblogoslawiwszy, kazal i one przed lud klasc.
\par 8 Jedli tedy i nasyceni sa, i zebrali, co zbylo ulomków, siedm koszów.
\par 9 A bylo tych, co jedli, okolo czterech tysiecy; i rozpuscil je.
\par 10 A wnet wstapiwszy w lódz z uczniami swoimi, przyszedl w strony Dalmanutskie.
\par 11 I wyszli Faryzeuszowie, a poczeli z nim spór wiesc, szukajac od niego znamienia z nieba, a kuszac go.
\par 12 Tedy westchnawszy serdecznie w duchu swym, rzekl: Przeczze ten rodzaj znamienia szuka? Zaprawde powiadam wam, ze nie bedzie dane znamie temu rodzajowi.
\par 13 I opusciwszy ich, wstapil zasie w lódz, i przeprawil sie na druga strone.
\par 14 A zapomnieli byli uczniowie wziac z soba chleba, i nie mieli z soba nic wiecej, tylko jeden chleb w lodzi.
\par 15 Tedy im przykazal, mówiac: Baczciez, a strzezcie sie kwasu Faryzeuszów i kwasu Herodowego.
\par 16 I rozmawiali miedzy soba i rzekli: O tem snac mówi, ze nie mamy chleba.
\par 17 Co poznawszy Jezus, rzekl im: O czemze rozmawiacie, iz nie macie chleba? Jeszczez nie baczycie i nie zrozumiewacie? Jeszczez macie serce swoje zdretwiale?
\par 18 Oczy majac nie widzicie, i uszy majac nie slyszycie, i nie pamietacie?
\par 19 Gdym onych piec chlebów lamal miedzy piec tysiecy ludzi, wiele zescie pelnych koszów ulomków zebrali? Rzekli mu: Dwanascie;
\par 20 A gdym onych siedm chlebów lamal miedzy cztery tysiace ludzi, wielescie koszów pelnych ulomków zebrali? A oni rzekli: Siedm.
\par 21 A on im rzekl: Jakoz tedy nie rozumiecie?
\par 22 Potem przyszedl do Betsaidy; i przywiedli do niego slepego, proszac go, aby sie go dotknal.
\par 23 A ujawszy onego slepego za reke, wywiódl go precz za miasteczko, i plunawszy na oczy jego, wlozyl na niego rece, i pytal go, jezliby co widzial.
\par 24 A on spojrzawszy w góre, rzekl: Widze ludzi; bo widze, ze chodza jako drzewa.
\par 25 Potem zasie wlozyl rece na oczy jego, i rozkazal mu w góre spojrzec; i uzdrowiony jest na wzroku, tak ze i z daleka wszystkich jasno widzial.
\par 26 I odeslal go do domu jego, mówiac: I do tego miasteczka nie wchodz, i nikomu z miasteczka nie powiadaj.
\par 27 Tedy wyszedl Jezus i uczniowie jego do miasteczek nalezacych do Cezaryi Filipowej, a w drodze pytal uczniów swoich, mówiac im: Kimze mie powiadaja byc ludzie?
\par 28 A oni mu odpowiedzieli: Jedni Janem Chrzcicielem, a drudzy Elijaszem, a drudzy jednym z proroków.
\par 29 Ale on im rzekl: A wy kim mie byc powiadacie? A odpowiadajac Piotr, rzekl mu: Tys jest on Chrystus.
\par 30 I przygrozil im, aby o nim nikomu nie powiadali.
\par 31 I poczal je nauczac, ze Syn czlowieczy musi wiele ucierpiec, i odrzuconym byc od starszych ludu, i od przedniejszych kaplanów i nauczonych w Pismie, i byc zabity, a po trzech dniach zmartwychwstac.
\par 32 A to mówil jawnie. Tedy go Piotr wziawszy na strone, poczal go strofowac.
\par 33 Ale on obróciwszy sie, a wejrzawszy na ucznie swoje, zgromil Piotra, mówiac: Idz ode mnie, szatanie; albowiem nie pojmujesz tego, co jest Bozego, ale co jest ludzkiego.
\par 34 A zwolawszy ludu z uczniami swoimi, rzekl im: Ktokolwiek chce za mna isc, niech samego siebie zaprze, a wezmie krzyz swój, i nasladuje mie.
\par 35 Albowiem kto by chcial zachowac dusze swa, straci ja; a kto by stracil dusze swa dla mnie i dla Ewangielii, ten ja zachowa.
\par 36 Bo cóz pomoze czlowiekowi, chocby wszystek swiat pozyskal, a szkodowalby na duszy swojej?
\par 37 Albo co za zamiane da czlowiek za dusze swoje?
\par 38 Albowiem kto by sie wstydzil za mie i za slowa moje miedzy tym rodzajem cudzoloznym i grzesznym, i Syn czlowieczy wstydzic sie za niego bedzie, gdy przyjdzie w chwale Ojca swego z Anioly swietymi.

\chapter{9}

\par 1 I mówil im: Zaprawde powiadam wam, iz sa niektórzy z tych, co tu stoja, którzy nie ukusza smierci, azby ujrzeli, ze królestwo Boze przyszlo w mocy.
\par 2 A po szesciu dniach wzial z soba Jezus Piotra, Jakóba i Jana, i wiódl je na góre wysoka same osobno, i przemienil sie przed nimi.
\par 3 A szaty jego staly sie lsniace, i bardzo biale jako snieg, jak ich blecharz na ziemi nie moze wybielic.
\par 4 I ujrzeli Elijasza z Mojzeszem, którzy rozmawiali z Jezusem.
\par 5 A odpowiadajac Piotr, rzekl Jezusowi: Mistrzu! dobrze nam tu byc; przetoz uczynimy tu trzy namioty, tobie jeden, Mojzeszowi jeden, i Elijaszowi jeden.
\par 6 Albowiem nie wiedzial, co by mówil; bo przestraszeni byli.
\par 7 I stal sie oblok, który je zacienil, a przyszedl glos z obloku mówiacy: Ten jest Syn mój mily, tegoz sluchajcie.
\par 8 A wnet obejrzawszy sie, nikogo wiecej nie widzieli, tylko Jezusa samego z soba.
\par 9 A gdy oni zstepowali z góry, przykazal im, aby tego nikomu nie powiadali, co widzieli, az kiedy by Syn czlowieczy zmartwychwstal.
\par 10 A tak oni zatrzymali te rzecz u siebie, pytajac sie miedzy soba, co by to bylo zmartwychwstac.
\par 11 I pytali go, mówiac: Cóz tedy nauczeni w Pismie powiadaja, ze Elijasz pierwej przyjsc ma?
\par 12 A on odpowiadajac, rzekl im: Elijaszci przyszedlszy pierwej, naprawi wszystko, a jako napisano o Synu czlowieczym, ze musi wiele ucierpiec, a za nic poczytanym byc.
\par 13 Alec wam powiadam, ze i Elijasz przyszedl, i uczynili mu cokolwiek chcieli, jako o nim napisano.
\par 14 A przyszedlszy do uczniów, ujrzal lud wielki okolo nich, i nauczone w Pismie spór majace z nimi.
\par 15 A wnetze lud wszystek ujrzawszy go, polekali sie, i zbiezawszy sie, przywitali go.
\par 16 I pytal nauczonych w Pismie: O cóz spór macie miedzy soba?
\par 17 A odpowiadajac jeden z onego ludu, rzekl: Nauczycielu! przywiodlem do ciebie syna mego, który ma ducha niemego.
\par 18 Ten gdziekolwiek go popadnie, rozdziera go, a on sie slini, i zgrzyta zebami swemi i schnie; i mówilem uczniom twoim, aby go wygnali; ale nie mogli.
\par 19 Lecz on odpowiadajac mu, rzekl: O rodzaju niewierny! dokadze z wami bede? i dokadze was cierpiec bede? przywiedzcie go do mnie.
\par 20 I przywiedli go do niego; a skoro go ujrzal, zaraz go duch rozdarl, a on upadlszy na ziemie, przewracal sie, sliniac sie.
\par 21 Zatem spytal Jezus ojca jego: Jakoz mu sie to dawno przydalo? A on powiedzial: Z dziecinstwa.
\par 22 I czesto go miotal w ogien i w wode, zeby go stracil, ale mozeszli co, ratuj nas, uzaliwszy sie nad nami.
\par 23 Ale mu Jezus rzekl: Jezli mozesz temu wierzyc? Wszystko jest mozno wierzacemu.
\par 24 A zarazem zawolawszy ojciec onego mlodzienca, ze lzami rzekl: Wierze, Panie! ty ratuj niedowiarstwa mego.
\par 25 A widzac Jezus, iz sie lud zbiegal, zgromil onego ducha nieczystego, mówiac mu: Duchu niemy i gluchy! ja tobie rozkazuje, wynijdz z niego, a nie wchodz wiecej w niego.
\par 26 Zawolawszy tedy bardzo, rozdarlszy go, wyszedl; i stal sie on czlowiek jako umarly, tak ze ich wiele mówilo, iz umarl.
\par 27 Ale Jezus ujawszy go za reke, podniósl go; i wstal.
\par 28 A gdy wszedl w dom, pytali go osobno uczniowie jego: Czemuzesmy go wygnac nie mogli?
\par 29 A on im rzekl: Ten rodzaj dyjablów inaczej wynijsc nie moze, tylko przez modlitwe i przez post.
\par 30 A stamtad wyszedlszy, szli z soba przez Galileje; ale nie chcial, aby kto o tem wiedzial.
\par 31 Albowiem uczyl ucznie swoje, i mówil im: Syn czlowieczy bedzie wydany w rece ludzkie, i zabija go; ale gdy bedzie zabity, dnia trzeciego zmartwychwstanie.
\par 32 Lecz oni tej rzeczy nie rozumieli; wszakze bali sie go spytac.
\par 33 Zatem przyszedl do Kapernaum, a bedac w domu, pytal ich: O czemzescie w drodze miedzy soba rozmawiali?
\par 34 Lecz oni milczeli; albowiem rozmawiali miedzy soba w drodze, kto by z nich byl wiekszy.
\par 35 A usiadlszy, zawolal dwunastu i mówil im: Jezli kto chce byc pierwszym, niech bedzie ze wszystkich ostatnim, i sluga wszystkich.
\par 36 A wziawszy dzieciatko, postawil je w posrodku nich, a wziawszy je na rece, rzekl im:
\par 37 Kto by jedno z takich dziateczek przyjal w imieniu mojem, mnie przyjmuje; a kto by mnie przyjal, nie mnie przyjmuje, ale onego, który mie poslal.
\par 38 Tedy mu odpowiedzial Jan, mówiac: Nauczycielu! widzielismy niektórego w imieniu twojem dyjably wyganiajacego, który nie chodzi za nami, i zabranialismy mu, przeto ze nie chodzi za nami.
\par 39 Ale Jezus rzekl: Nie zabraniajcie mu; albowiem nikt nie jest, co by czynil cuda w imieniu mojem, aby mógl snadnie mówic zle o mnie.
\par 40 Bo kto nie jest przeciwko nam, za nami jest.
\par 41 Albowiem kto by was napoil kubkiem wody w imieniu mojem, dlatego iz jestescie Chrystusowi, zaprawde powiadam wam, nie straci zaplaty swojej.
\par 42 A kto by zgorszyl jednego z tych malych, którzy w mie wierza, daleko by mu lepiej bylo, aby byl zawieszony kamien mlynski u szyi jego, i w morze byl wrzucony.
\par 43 A jezliby cie gorszyla reka twoja, odetnij ja; bo lepiej jest tobie ulomnym wnijsc do zywota, nizeli dwie rece majac, isc do piekla w on ogien nieugaszony,
\par 44 Gdzie robak ich nie umiera, a ogien nie gasnie.
\par 45 A jezliby cie noga twoja gorszyla, odetnij ja; bo lepiej tobie chromym wnijsc do zywota, nizeli dwie nogi majac, byc wrzuconym do piekla, w ogien nieugaszony,
\par 46 Gdzie robak ich nie umiera, a ogien nie gasnie.
\par 47 A jezliby cie oko twoje gorszylo, wylup je; bo lepiej tobie jednookim wnijsc do królestwa Bozego, nizeli dwoje oczu majac, wrzuconym byc do ognia piekielnego.
\par 48 Gdzie robak ich nie umiera, a ogien nie gasnie.
\par 49 Albowiem kazdy czlowiek ogniem osolony bedzie, i kazda ofiara sola osolona bedzie.
\par 50 Dobrac jest sól; ale jezli sie sól nieslona stanie, czemze ja osolicie? Miejciez sól sami w sobie, a miejcie pokój miedzy soba.

\chapter{10}

\par 1 A wstawszy stamtad, przyszedl do granic Judzkich przez kraine za Jordanem lezaca; i zszedl sie zas do niego lud, i uczyl je zas jako mial zwyczaj.
\par 2 Tedy przystapiwszy Faryzeuszowie, pytali go: Godzili sie mezowi zone opuscic? a to czynili, kuszac go.
\par 3 Ale on odpowiadajac, rzekl im: Cóz wam przykazal Mojzesz?
\par 4 A oni rzekli: Mojzesz pozwolil napisac list rozwodny i opuscic ja.
\par 5 A odpowiadajac Jezus, rzekl im: Dla zatwardzenia serca waszego napisal wam to przykazanie.
\par 6 Alec od poczatku stworzenia mezczyzne i niewiaste uczynil je Bóg.
\par 7 Dlatego opusci czlowiek ojca swego i matke, a przylaczy sie do zony swojej,
\par 8 I beda dwoje jednem cialem; a tak juz nie sa dwoje, ale jedno cialo.
\par 9 Co tedy Bóg zlaczyl, czlowiek niechaj nie rozlacza.
\par 10 A w domu zas uczniowie jego o toz go pytali.
\par 11 I rzekl im: Ktobykolwiek opuscil zone swa, a pojalby inna, cudzolozy przeciwko niej;
\par 12 A jezliby niewiasta opuscila meza swego, a szlaby za drugiego, cudzolozy.
\par 13 Tedy przynoszono do niego dziatki, aby sie ich dotykal; ale uczniowie gromili tych, którzy je przynosili.
\par 14 Co ujrzawszy Jezus, rozgniewal sie i rzekl im: Dopusccie dziatkom przychodzic do mnie, a nie zabraniajcie im; albowiem takich jest królestwo Boze.
\par 15 Zaprawde powiadam wam: Ktobykolwiek nie przyjal królestwa Bozego jako dzieciatko, nie wnijdzie do niego.
\par 16 I biorac je na rece swoje, i kladac na nie rece, blogoslawil im.
\par 17 A gdy on wychodzil w droge, przybiezal jeden, i upadlszy przed nim na kolana, pytal go: Nauczycielu dobry! cóz czynic mam, abym odziedziczyl zywot wieczny?
\par 18 Ale mu Jezus rzekl: Przecz mie zowiesz dobrym? Nikt nie jest dobry, tylko jeden, to jest Bóg.
\par 19 Przykazania umiesz: nie bedziesz cudzolozyl, nie bedziesz zabijal, nie bedziesz kradl, nie bedziesz mówil swiadectwa falszywego, nie bedziesz oszukiwal nikogo, czcij ojca twego i matke twoje.
\par 20 A on odpowiadajac, rzekl mu: Nauczycielu! tegom wszystkiego przestrzegal od mlodosci mojej.
\par 21 A Jezus spojrzawszy nan, rozmilowal sie go, i rzekl mu: Jednego ci nie dostaje; idz, sprzedaj co masz, a rozdaj ubogim, a bedziesz mial skarb w niebie, a przyjdz, nasladuj mie, wziawszy krzyz.
\par 22 A on zafrasowawszy sie dla tego slowa, odszedl smutny; albowiem mial wiele majetnosci.
\par 23 A spojrzawszy Jezus w okolo, rzekl do uczniów swoich: Jakoz trudno ci, którzy maja bogactwa, wnijda do królestwa Bozego!
\par 24 Tedy uczniowie zdumieli sie nad temi slowami jego. Lecz Jezus zas odpowiadajac, rzekl im: Dziatki! jakoz jest trudno tym, co ufaja w bogactwach, wnijsc do królestwa Bozego.
\par 25 Snadniej jest wielbladowi przejsc przez ucho igielne, niz bogaczowi wnijsc do królestwa Bozego.
\par 26 A oni sie tem wiecej zdumiewali, mówiac miedzy soba: I któz moze byc zbawiony?
\par 27 A Jezus spojrzawszy na nie, rzekl: U ludzic to niemozno, ale nie u Boga; albowiem u Boga wszystko jest mozno.
\par 28 I poczal Piotr mówic do niego: Otosmy my opuscili wszystko, a poszlismy za toba.
\par 29 A Jezus odpowiadajac, rzekl: Zaprawde powiadam wam: Nikt nie jest, kto by opuscil dom, albo braci, albo siostry, albo ojca, albo matke, albo zone, albo dzieci, albo role dla mnie i dla Ewangielii,
\par 30 Zeby nie mial wziac stokrotnie teraz w tym czasie domów, i braci, i sióstr, i matek, i dzieci, i ról z przesladowaniem, a w przyszlym wieku zywota wiecznego.
\par 31 Alec wiele pierwszych beda ostatnimi, a ostatnich pierwszymi.
\par 32 I byli w drodze, wstepujac do Jeruzalemu; a Jezus szedl przed nimi, i zdumiewali sie, a idac za nim, bali sie. A on wziawszy zasie z soba onych dwanascie, poczal im powiadac, co nan przyjsc mialo,
\par 33 Mówiac: Oto wstepujemy do Jeruzalemu, a Syn czlowieczy bedzie wydany przedniejszym kaplanom i nauczonym w Pismie, i osadza go na smierc, i wydadza go poganom.
\par 34 A oni sie z niego nasmiewac beda, i ubiczuja go, i beda nan plwac, i zabija go; ale dnia trzeciego zmartwychwstanie.
\par 35 Tedy przystapili do niego Jakób i Jan, synowie Zebedeuszowi, mówiac: Nauczycielu! chcemy, abys nam uczynil, o co cie prosic bedziemy.
\par 36 A on im rzekl: Cóz chcecie, abym wam uczynil?
\par 37 A oni mu rzekli: Daj nam, abysmy jeden na prawicy twojej a drugi na lewicy twojej siedzieli w chwale twojej.
\par 38 Lecz im Jezus rzekl: Nie wiecie, o co prosicie. Mozeciez pic kielich, który ja pije, i chrztem, którym sie ja chrzcze, byc ochrzczeni?
\par 39 A oni mu rzekli: Mozemy. A Jezus im rzekl: Kielichci, który ja pije, pic bedziecie i chrztem, którym ja sie chrzcze, ochrzczeni bedziecie.
\par 40 Ale siedziec po prawicy mojej albo po lewicy mojej, nie moja rzecz jest dac; ale bedzie dano tym, którym zgotowano.
\par 41 A uslyszawszy to oni dziesieciu, poczeli sie gniewac na Jakóba i na Jana.
\par 42 Ale Jezus zwolawszy ich, rzekl im: Wiecie, iz ci, którym sie zda, ze wladze maja nad narody, panuja nad nimi, a którzy z nich wielcy sa, moc przewodza nad nimi.
\par 43 Lecz nie tak bedzie miedzy wami; ale ktobykolwiek chcial byc wielkim miedzy wami, bedzie sluga waszym;
\par 44 A ktobykolwiek z was chcial byc pierwszym, bedzie sluga wszystkich.
\par 45 Bo i Syn czlowieczy nie przyszedl, aby mu sluzono, ale aby sluzyl, i aby dal dusze swa na okup za wielu.
\par 46 Tedy przyszli do Jerycha; a gdy on wychodzil z Jerycha, i uczniowie jego i lud wielki, syn Tymeusza, Bartymeusz slepy, siedzial podle drogi zebrzac.
\par 47 A uslyszawszy, iz jest Jezus on Nazarenski, poczal wolac, mówiac: Jezusie, Synu Dawida! zmiluj sie nade mna.
\par 48 I gromilo go wiele ich, aby milczal, ale on tem wiecej wolal: Synu Dawida! zmiluj sie nade mna.
\par 49 Tedy zastanowiwszy sie Jezus, kazal go zawolac. I zawolano slepego, mówiac mu: Ufaj, wstan, wola cie.
\par 50 A on porzuciwszy plaszcz swój, wstal, i przyszedl do Jezusa.
\par 51 I odpowiadajac Jezus, rzekl mu: Cóz chcesz, abym ci uczynil? A slepy mu rzekl: Mistrzu! abym przejrzal.
\par 52 A Jezus mu rzekl: Idz, wiara twoja ciebie uzdrowila. A zarazem przejrzal, i szedl droga za Jezusem.

\chapter{11}

\par 1 A gdy sie przyblizyli do Jeruzalemu i do Betfagie i do Betanii ku górze oliwnej, poslal dwóch z uczniów swoich,
\par 2 I rzekl im: Idzcie do miasteczka, które jest przeciwko wam, a wszedlszy do niego, zaraz znajdziecie osle uwiazane, na którem nikt z ludzi nie siedzial; odwiazciez je, a przywiedzcie.
\par 3 A jezliby wam kto rzekl: Cóz to czynicie? Powiedzcie, iz go Pan potrzebuje; a wnet je tu posle.
\par 4 Szli tedy i znalezli osle uwiazane u drzwi na dworze na rozstaniu dróg, i odwiazali je.
\par 5 Tedy niektórzy z onych, co tam stali, mówili: Cóz czynicie, ze odwiazujecie osle?
\par 6 A oni im rzekli, jako im byl rozkazal Jezus. I puscili je.
\par 7 Przywiedli tedy osle do Jezusa, i wlozyli na nie szaty swoje; i wsiadl na nie.
\par 8 A wiele ich slali szaty swoje na drodze; drudzy zasie obcinali galazki z drzew, i slali na drodze.
\par 9 A którzy wprzód szli, i którzy pozad szli, wolali, mówiac: Hosanna, blogoslawiony, który idzie w imieniu Panskiem!
\par 10 Blogoslawione królestwo ojca naszego Dawida, które przyszlo w imieniu Panskiem! Hosanna na wysokosciach!
\par 11 I wjechal Jezus do Jeruzalemu i przyszedl do kosciola, a obejrzawszy wszystko, gdy juz byla wieczorna godzina, wyszedl do Betanii z dwunastoma.
\par 12 A drugiego dnia, gdy wychodzili z Betanii, laknal.
\par 13 I ujrzawszy z daleka figowe drzewo, majace liscie, przyszedl, jezliby snac co na niem znalazl; a gdy do niego przyszedl, nic nie znalazl, tylko liscie; bo nie byl czas figom.
\par 14 A odpowiadajac Jezus, rzekl mu: Niechajze wiecej na wieki nikt z ciebie owocu nie je. A slyszeli to uczniowie jego.
\par 15 I przyszli do Jeruzalemu; a wszedlszy Jezus do kosciola, poczal wyganiac sprzedawajace i kupujace w kosciele, i poprzewracal stoly tych, co pieniedzmi handlowali, i stolki tych, co sprzedawali golebie;
\par 16 A nie dopuscil, zeby kto mial niesc naczynie przez kosciól.
\par 17 I nauczal, mówiac im: Azaz nie napisano: Ze dom mój, dom modlitwy bedzie nazwany od wszystkich narodów? a wyscie go uczynili jaskinia zbójców.
\par 18 A slyszeli to nauczeni w Pismie i przedniejsi kaplani i szukali, jakoby go stracili; albowiem sie go bali, przeto iz wszystek lud zdumiewal sie nad nauka jego.
\par 19 A gdy przyszedl wieczór, wyszedl z miasta.
\par 20 A rano idac mimo figowe drzewo, ujrzeli, iz z korzenia uschlo.
\par 21 Tedy wspomniawszy Piotr, rzekl mu: Mistrzu! oto figowe drzewo, któres przeklal, uschlo.
\par 22 A Jezus odpowiadajac, rzekl im: Miejcie wiare w Boga.
\par 23 Bo zaprawde powiadam wam, iz ktobykolwiek rzekl tej górze: Podnies sie, a rzuc sie w morze, a nie watpilby w sercu swojem, leczby wierzyl, ze sie stanie, co mówi, stanie sie mu, cokolwiek rzecze.
\par 24 Przetoz powiadam wam: O cokolwiekbyscie, modlac sie, prosili, wierzcie, ze wezmiecie, a stanie sie wam.
\par 25 A gdy stoicie modlac sie, odpuscciez, jezli co przeciw komu macie, aby i Ojciec wasz, który jest w niebiesiech, odpuscil wam upadki wasze.
\par 26 Bo jezli wy nie odpuscicie, i Ojciec wasz, który jest w niebiesiech, nie odpusci wam upadków waszych.
\par 27 I przyszli znowu do Jeruzalemu. A gdy sie on przechodzil po kosciele, przystapili do niego przedniejsi kaplani i nauczeni w Pismie i starsi;
\par 28 I mówili do niego: Któraz to moca czynisz? a kto ci dal te moc, abys to czynil?
\par 29 Tedy Jezus odpowiadajac, rzekl im: Spytam was i ja o jedne rzecz; odpowiedzciez mi, a powiem, która moca to czynie.
\par 30 Chrzest Jana z niebaz byl, czyli z ludzi? Odpowiedzcie mi.
\par 31 I rozbierali to sami miedzy soba, mówiac: Jezli powiemy, z nieba, rzecze: Przeczzescie mu tedy nie wierzyli?
\par 32 A jezli powiemy, z ludzi, bojemy sie ludu; albowiem wszyscy Jana mieli za prawdziwego proroka.
\par 33 Tedy odpowiadajac rzekli Jezusowi: Nie wiemy. Jezus tez odpowiadajac rzekl im: I ja wam nie powiem, która moca to czynie.

\chapter{12}

\par 1 Tedy poczal do nich mówic w podobienstwach: Czlowiek jeden nasadzil winnice, i ogrodzil ja plotem, i wykopal prase, i zbudowal wieze, i najal ja winiarzom, i odjechal precz.
\par 2 I poslal, gdy tego byl czas, sluge do winiarzy, aby od winiarzy odebral pozytki onej winnicy.
\par 3 Lecz oni pojmawszy go, ubili, i odeslali próznego.
\par 4 I zasie poslal do nich sluge drugiego, którego tez oni ukamionowawszy, glowe mu zranili, i odeslali obelzonego.
\par 5 I zasie poslal inszego sluge; ale i tego zabili, i wiele innych, z których jedne ubili, a drugie pozabijali.
\par 6 A majac jeszcze jednegoz swego milego syna, poslal na ostatek do nich i tego, mówiac: Wzdyc sie beda wstydzili syna mego.
\par 7 Ale oni winiarze rzekli miedzy soba: Tenci jest dziedzic; pójdzcie, zabijmy go, a bedzie nasze dziedzictwo.
\par 8 I wziawszy go zabili, a wyrzucili precz z winnicy.
\par 9 Cóz tedy uczyni pan onej winnicy? Przyjdzie, a potraci one winiarze, i da winnice innym.
\par 10 Izaliscie nie czytali tego pisma: Kamien, który odrzucili budujacy, ten sie stal glowa wegielna?
\par 11 Od Panac sie to stalo, i jest dziwne w oczach naszych.
\par 12 Starali sie tedy, jakoby go pojmac, ale sie ludu bali; bo poznali, iz przeciwko nim ono podobienstwo powiedzial. I zaniechawszy go, odeszli.
\par 13 Potem poslali do niego niektóre z Faryzeuszów i z Herodyjanów, aby go usidlili w mowie.
\par 14 A tak oni przyszedlszy rzekli mu: Nauczycielu! wiemy, zes jest prawdziwy, a nie dbasz na nikogo; albowiem nie patrzysz na osobe ludzka, ale w prawdzie drogi Bozej uczysz; godziz sie dac czynsz cesarzowi, czyli nie? Mamyz go dac, czyli nie dac?
\par 15 A on poznawszy oblude ich, rzekl im: Czemuz mie kusicie? Przyniescie mi grosz, abym go ogladal.
\par 16 Tedy mu oni przyniesli; a on im rzekl: Czyjze to jest obraz i napis? A oni mu powiedzieli: Cesarski.
\par 17 I odpowiadajac Jezus, rzekl im: Oddawajciez tedy, co jest cesarskiego, cesarzowi, a co jest Bozego, Bogu. I dziwowali mu sie.
\par 18 I przyszli do niego Saduceuszowie, którzy mówia, iz nie masz zmartwychwstania, i pytali go mówiac:
\par 19 Nauczycielu! Mojzesz nam napisal, iz jezliby czyj brat umarl, i zostawil zone, a dziatek by nie zostawil, aby brat jego pojal zone jego, a wzbudzil nasienie bratu swemu.
\par 20 Bylo tedy siedm braci; a pierwszy pojawszy zone umarl, i nie zostawil nasienia;
\par 21 A drugi pojawszy ja, umarl, lecz i ten nie zostawil nasienia; takze i trzeci.
\par 22 A tak ja pojelo onych siedm braci, a nie zostawili nasienia. Na ostatek po wszystkich umarla i ona niewiasta.
\par 23 Przetoz przy zmartwychwstaniu gdy powstana, któregoz z nich bedzie zona? bo siedm ich mieli ja za zone.
\par 24 Na to Jezus odpowiadajac rzekl im: Zaz nie dlatego bladzicie, izescie nie powiadomi Pisma ani mocy Bozej?
\par 25 Albowiem gdy zmartwychwstana, ani sie zenia, ani za maz wydawaja; ale sa jako Aniolowie w niebiesiech.
\par 26 A o umarlych, iz beda wzbudzeni, nie czytalisciez w ksiegach Mojzeszowych, jako Bóg do niego ze krza mówil, i rzekl: Jam jest Bóg Abrahama, Bóg Izaaka, i Bóg Jakóba?
\par 27 Bóg nie jestci Bogiem umarlych, ale Bogiem zywych: przetoz wy bardzo bladzicie.
\par 28 A przystapiwszy jeden z nauczonych w Pismie, slyszac, ze z soba gadali, a widzac, ze im dobrze odpowiedzial, spytal go: Które jest najpierwsze ze wszystkich przykazanie?
\par 29 A Jezus mu odpowiedzial: Najpierwsze ze wszystkich przykazanie jest: Sluchaj, Izraelu! Pan, Bóg nasz, Pan jeden jest.
\par 30 Przetoz bedziesz milowal Pana, Boga twego, ze wszystkiego serca twego, i ze wszystkiej duszy twojej, i ze wszystkiej mysli twojej, i ze wszystkiej sily twojej; toc jest pierwsze przykazanie.
\par 31 A wtóre temu podobne to jest: Bedziesz milowal blizniego twego, jako samego siebie. Wiekszego przykazania innego nad to nie masz.
\par 32 Tedy mu rzekl on nauczony w Pismie: Nauczycielu! zaprawde dobrzes powiedzial, iz jeden jest Bóg, a nie masz inszego oprócz niego.
\par 33 I milowac go ze wszystkiego serca i ze wszystkiej mysli i ze wszystkiej duszy i ze wszystkiej sily, a milowac blizniego jako samego siebie, wiecej jest nad wszystkie calopalenia i ofiary.
\par 34 A widzac Jezus, ze on madrze odpowiedzial, rzekl mu: Niedalekos jest od królestwa Bozego. I nie smial go nikt dalej pytac.
\par 35 Tedy Jezus odpowiadajac rzekl, gdy uczyl w kosciele: Jakoz mówia nauczeni w Pismie, iz Chrystus jest Syn Dawida?
\par 36 Bo sam Dawid przez Ducha Swietego powiedzial: Rzekl Pan Panu memu, siadz po prawicy mojej, az poloze nieprzyjacioly twoje podnózkiem nóg twoich.
\par 37 Poniewaz go sam Dawid nazywa Panem, jakoz tedy jest synem jego? a wielki lud rad go sluchal.
\par 38 I mówil do nich w nauce swojej: Strzezcie sie nauczonych w Pismie, którzy chca w dlugich szatach chodzic, a byc pozdrawiani na rynkach;
\par 39 I na pierwszych stolkach siadac w bóznicach, i pierwsze miejsca miec na wieczerzach;
\par 40 Którzy pozeraja domy wdów, a to pod pokrywka dlugich modlitw; cic odniosa ciezszy sad.
\par 41 A Jezus siedzac przeciwko skarbnicy, przypatrywal sie, jako lud rzucal pieniadze do skarbnicy, i jako wiele bogaczów wiele rzucalo.
\par 42 I przyszedlszy jedna wdowa uboga, wrzucila dwa drobne pieniazki, co czyni kwartnik.
\par 43 Tedy zwolawszy uczniów swoich rzekl im: Zaprawde wam powiadam, ze ta uboga wdowa wiecej wrzucila, nizeli ci wszyscy, którzy rzucali do skarbnicy.
\par 44 Albowiem ci wszyscy z tego, co im zbywalo, rzucali; ale ta z ubóstwa swego wszystko, co miala, wrzucila, wszystke zywnosc swoje.

\chapter{13}

\par 1 A gdy on wychodzil z kosciola, rzekl mu jeden z uczniów jego: Nauczycielu! patrz, jakie to kamienie, i jakie budowania?
\par 2 A Jezus odpowiadajac rzekl mu: Widzisz te wielkie budowania? Nie bedzie zostawiony kamien na kamieniu, który by nie byl rozwalony.
\par 3 A gdy siedzial na górze oliwnej przeciwko kosciolowi, pytali go osobno Piotr, i Jakób, i Jan, i Andrzej:
\par 4 Powiedz nam, kiedy sie to stanie, i co za znak, kiedy sie to wszystko pelnic bedzie?
\par 5 A Jezus odpowiadajac im, poczal mówic: Patrzcie, aby was kto nie zwiódl.
\par 6 Albowiem wiele ich przyjdzie pod imieniem mojem, mówiac: Jam jest Chrystus, a wiele ich zwioda.
\par 7 Gdy tedy uslyszycie wojny, i wiesci o wojnach, nie trwozciez soba; boc to musi byc; ale jeszcze nie tu koniec.
\par 8 Albowiem powstanie naród przeciwko narodowi, i królestwo przeciwko królestwu, i beda miejscami trzesienia ziemi, i beda glody i zamieszania.
\par 9 A toc poczatki bolesci. Lecz wy strzezcie samych siebie; boc was podawac beda przed rady i do zgromadzenia, beda was bic, a przed starostami i królmi dla mnie stawac bedziecie na swiadectwo przeciwko nim.
\par 10 Ale u wszystkich narodów musi byc przedtem kazana Ewangielija.
\par 11 A gdy was powioda wydawajac, nie troszczciez sie przed czasem, co byscie mówic mieli, ani o tem myslcie, ale co wam bedzie dano onejze godziny, to mówcie; albowiem nie wy jestescie, którzy mówicie, ale Duch Swiety.
\par 12 I wyda brat brata na smierc, a ojciec syna; i powstana dzieci przeciwko rodzicom, i beda je zabijac.
\par 13 A bedziecie w nienawisci u wszystkich dla imienia mego; ale kto wytrwa az do konca, ten bedzie zbawion.
\par 14 Gdy tedy ujrzycie one obrzydliwosc spustoszenia, opowiedziana od Danijela proroka, stojaca, gdzie stac nie miala, (kto czyta, niechaj uwaza,)tedy ci, którzy beda w Judzkiej ziemi, niech uciekaja na góry.
\par 15 A kto bedzie na dachu, niech nie zstepuje do domu, ani wchodzi, aby co wzial z domu swego;
\par 16 A kto bedzie na roli, niech sie nie wraca nazad, aby wzial szate swoje.
\par 17 Lecz biada brzemiennym i piersiami karmiacym w one dni!
\par 18 Przetoz módlcie sie, aby uciekanie wasze nie bylo w zimie.
\par 19 Albowiem beda te dni takiem ucisnieniem, jakiego nie bylo od poczatku stworzenia, które stworzyl Bóg, az dotad, ani bedzie.
\par 20 A jezliby Pan nie skrócil dni onych, nie byloby zadne cialo zbawione; lecz dla wybranych, które wybral, skrócil dni onych.
\par 21 A tedy jezliby wam kto rzekl: Oto tu jest Chrystus, albo oto tam, nie wierzcie.
\par 22 Boc powstana falszywi Chrystusowie, i falszywi prorocy, i beda czynic znamiona i cuda ku zwiedzeniu, by mozna, i wybranych.
\par 23 Wy tedy strzezcie sie; otom wam wszystko przepowiedzial.
\par 24 Ale w one dni po ucisnieniu onem, zacmi sie slonce, i ksiezyc nie wyda swiatlosci swojej;
\par 25 I gwiazdy niebieskie beda padaly, a mocy, które sa na niebie, porusza sie.
\par 26 A tedy ujrza Syna czlowieczego, przychodzacego w oblokach z moca i z chwala wielka.
\par 27 A tedy posle Anioly swoje i zgromadzi wszystkie wybrane swoje od czterech wiatrów, od konczyn ziemi az do konczyn nieba.
\par 28 A od figowego drzewa nauczcie sie tego podobienstwa: Gdy sie juz Gala? jego odmladza i puszcza liscie, poznajecie, iz blisko jest lato.
\par 29 Takze i wy, gdy ujrzycie, iz sie to dziac bedzie, poznawajcie, ze blisko jest i we drzwiach.
\par 30 Zaprawde powiadam wam, zec nie przeminie ten rodzaj, azby sie to wszystko stalo.
\par 31 Niebo i ziemia przemina; ale slowa moje nie przemina.
\par 32 Lecz o onym dniu i godzinie nikt nie wie, ani Aniolowie, którzy sa w niebie, ani Syn, tylko Ojciec.
\par 33 Patrzciez, czujcie, a módlcie sie; bo nie wiecie, kiedy ten czas bedzie.
\par 34 Jako czlowiek, który precz odjezdzajac, zostawil dom swój, i rozdal urzedy slugom swoim, i kazdemu robote jego, i wrotnemu przykazal, aby czul.
\par 35 Czujciez tedy; (bo nie wiecie, kiedy Pan domu onego przyjdzie, z wieczorali, czyli o pólnocy, czyli gdy kury pieja, czyli rano.)
\par 36 By snac niespodzianie przyszedlszy, nie znalazl was spiacymi.
\par 37 A co wam mówie, wszystkimci mówie: Czujcie.

\chapter{14}

\par 1 A po dwóch dniach byla wielkanoc, swieto przasników; i szukali przedniejsi kaplani i nauczeni w Pismie, jakoby go zdrada pojmawszy, zabili.
\par 2 Lecz mówili: Nie w swieto, aby snac nie byl rozruch miedzy ludem.
\par 3 A gdy on byl w Betanii, w domu Szymona tredowatego, gdy siedzial u stolu, przyszla niewiasta, majac sloik alabastrowy masci szpikanardowej plynacej, bardzo kosztownej, a stluklszy sloik alabastrowy, wylala ja na glowe jego.
\par 4 I gniewali sie niektórzy sami w sobie, a mówili: Na cóz sie stala utrata tej masci?
\par 5 Albowiem sie to moglo sprzedac drozej niz za trzysta groszy, i rozdac ubogim; i szemrali przeciwko niej.
\par 6 Ale Jezus rzekl: Zaniechajcie jej, przeczze sie jej przykrzycie? Dobryc uczynek uczynila przeciwko mnie.
\par 7 Zawsze bowiem ubogie macie z soba, i kiedykolwiek chcecie, mozecie im dobrze czynic; ale mnie nie zawsze miec bedziecie.
\par 8 Ona, co mogla, to uczynila; poprzedzila, aby cialo moje pomazala ku pogrzebowi.
\par 9 Zaprawde powiadam wam: Gdziekolwiek kazana bedzie ta Ewangielija po wszystkim swiecie, i to co ona uczynila, powiadano bedzie na pamiatke jej.
\par 10 Tedy Judasz Iszkaryjot, jeden ze dwunastu, odszedl do przedniejszych kaplanów, aby im go wydal.
\par 11 Co oni uslyszawszy, uradowali sie, i obiecali mu dac pieniadze. I szukal sposobnego czasu, jakoby go wydal.
\par 12 Pierwszego tedy dnia przasników, gdy baranka wielkanocnego zabijano, rzekli mu uczniowie jego: Gdzie chcesz, abysmy szedlszy nagotowali, zebys jadl baranka?
\par 13 I poslal dwóch z uczniów swych, i rzekl im: Idzcie do miasta, a spotka sie z wami czlowiek, dzban wody niosacy; idzciez za nim.
\par 14 A dokadkolwiek wnijdzie, rzeczcie gospodarzowi: Nauczyciel mówi: Gdziez jest gospoda, kedy bym jadl baranka z uczniami moimi?
\par 15 A on wam ukaze sale wielka uslana i gotowa, tamze nam nagotujecie.
\par 16 I odeszli uczniowie jego, i przyszli do miasta, i znalezli tak, jako im byl powiedzial, i nagotowali baranka.
\par 17 A gdy byl wieczór, przyszedl ze dwunastoma.
\par 18 A gdy za stolem siedzieli i jedli, rzekl Jezus: Zaprawde wam powiadam, iz jeden z was wyda mie, który je ze mna.
\par 19 Tedy oni poczeli sie smucic, i do niego mówic, kazdy z osobna: Azazem ja jest? A drugi: Azaz ja?
\par 20 Lecz on odpowiadajac rzekl im: Jeden ze dwunastu, który ze mna macza w misie.
\par 21 Synci czlowieczy idzie, jako o nim napisano: ale biada czlowiekowi temu, przez którego Syn czlowieczy bedzie wydany! dobrze by mu bylo, by sie byl ten czlowiek nie narodzil.
\par 22 A gdy oni jedli, wzial Jezus chleb, a poblogoslawiwszy, lamal i dal im, mówiac: Bierzcie, jedzcie, to jest cialo moje.
\par 23 A wziawszy kielich, i podziekowawszy, dal im; i pili z niego wszyscy.
\par 24 I rzekl im: To jest krew moja nowego testamentu, która sie za wielu wylewa.
\par 25 Zaprawde powiadam wam: Iz nie bede wiecej pil z rodzaju winnej macicy, az do dnia onego, gdy go pic bede nowy w królestwie Bozem.
\par 26 A zaspiewawszy piesn, wyszli na góre oliwna.
\par 27 Potem im rzekl Jezus: Wszyscy wy zgorszycie sie ze mnie tej nocy; bo napisano: Uderze pasterza, i beda rozproszone owce.
\par 28 Lecz gdy zmartwychwstane, poprzedze was do Galileii.
\par 29 Ale mu Piotr powiedzial: Chocby sie wszyscy zgorszyli, ale ja nie.
\par 30 I rzekl mu Jezus: Zaprawde powiadam tobie, iz dzis tej nocy, pierwej niz dwakroc kur zapieje, trzykroc sie mnie zaprzesz.
\par 31 Ale on tem wiecej mówil: Bym z toba mial i umrzec, nie zapre sie ciebie. Toc tez i wszyscy mówili:
\par 32 I przyszli na miejsce, które zwano Gietsemane; tedy rzekl do uczniów swoich: Siedzcie tu, az sie pomodle.
\par 33 I wziawszy z soba Piotra, i Jakóba, i Jana, poczal sie lekac, i bardzo tesknic;
\par 34 I rzekl im: Bardzo jest smutna dusza moja az do smierci; zostancie tu, a czujcie.
\par 35 A postapiwszy troche, padl na ziemie i modlil sie, aby, jezli mozna, odeszla od niego ta godzina;
\par 36 I rzekl: Abba Ojcze! wszystko tobie jest mozno, przenies ode mnie ten kielich; wszakze nie co ja chce, ale co Ty.
\par 37 Tedy przyszedl, i znalazl je spiace, i rzekl Piotrowi: Szymonie, spisz? nie mogles czuc jednej godziny?
\par 38 Czujcie, a módlcie sie, abyscie nie weszli w pokuszenie; duchci jest ochotny, ale cialo mdle.
\par 39 I odszedlszy znowu, modlil sie, tez slowa mówiac.
\par 40 A wróciwszy sie znalazl je zasie spiace, (bo oczy ich byly obciazone,)a nie wiedzieli, co mu odpowiedziec mieli.
\par 41 I przyszedl po trzecie, a rzekl im: Spijciez juz i odpoczywajcie! Dosycci! przyszlac ta godzina, oto wydany bywa Syn czlowieczy w rece grzeszników.
\par 42 Wstancie, pójdzmy! oto który mie wydawa, blisko jest.
\par 43 A wnetze, gdy on jeszcze mówil, przyszedl Judasz, który byl jeden ze dwunastu, a z nim wielka zgraja z mieczami i z kijami od przedniejszych kaplanów, i od nauczonych w Pismie i od starszych.
\par 44 A ten, który go wydawal, dal im byl znak, mówiac: Któregokolwiek pocaluje, tenci jest, imajciez go, a wiedzcie ostroznie.
\par 45 A przyszedlszy, zarazem przystapil do niego, i rzekl: Mistrzu, Mistrzu! i pocalowal go.
\par 46 Tedy sie oni na niego rekoma rzucili, i pojmali go.
\par 47 A jeden z tych, co tam stali, dobywszy miecza, uderzyl sluge najwyzszego kaplana, i ucial mu ucho.
\par 48 A Jezus odpowiadajac rzekl im: Jako na zbójce wyszliscie z mieczami i z kijami, abyscie mie pojmali.
\par 49 Na kazdy dzien bywalem u was w kosciele, uczac, a nie pojmaliscie mie: ale trzeba, aby sie wypelnily Pisma.
\par 50 A tak opusciwszy go, wszyscy uciekli.
\par 51 A jeden jakis mlodzieniec szedl za nim, przyodziany przescieradlem na nagie cialo; i uchwycili go mlodziency.
\par 52 Ale on opusciwszy przescieradlo, nago uciekl od nich.
\par 53 Tedy przywiedli Jezusa do najwyzszego kaplana: a zeszli sie do niego wszyscy przedniejsi kaplani, i starsi, i nauczeni w Pismie.
\par 54 A Piotr szedl z nim z daleka az do dworu najwyzszego kaplana, i siedzial z slugami, grzejac sie u ognia.
\par 55 Ale przedniejsi kaplani, i wszystka rada szukali przeciwko Jezusowi swiadectwa, aby go na smierc wydali; wszakze nie znalezli.
\par 56 Albowiem ich wiele falszywie swiadczyli przeciwko niemu; ale swiadectwa ich nie byly zgodne.
\par 57 Tedy niektórzy powstawszy, falszywie swiadczyli przeciwko niemu, mówiac:
\par 58 Mysmy to slyszeli, ze mówil: Ja rozwale ten kosciól reka uczyniony, a we trzech dniach inny nie reka uczyniony zbuduje.
\par 59 Lecz i tak nie bylo zgodne swiadectwo ich.
\par 60 Tedy stanawszy w posrodku najwyzszy kaplan, pytal Jezusa, mówiac: Nie odpowiadasz nic? Cóz to jest, co ci przeciwko tobie swiadcza?
\par 61 Ale on milczal, a nic nie odpowiedzial. Znowu go pytal najwyzszy kaplan, i rzekl mu: Tyzes jest on Chrystus, Syn onego Blogoslawionego?
\par 62 A Jezus rzekl: Jam jest; i ujrzycie Syna czlowieczego, siedzacego na prawicy mocy Bozej, i przychodzacego z oblokami niebieskiemi.
\par 63 Tedy najwyzszy kaplan rozdarlszy szaty swoje, rzekl: Cóz jeszcze potrzebujemy swiadków?
\par 64 Slyszeliscie bluznierstwo. Cóz sie wam zda? A oni wszyscy osadzili go winnym byc smierci.
\par 65 I poczeli niektórzy nan plwac, i zakrywac oblicze jego, i bic wen piesciami i mówic mu: Prorokuj! A sludzy policzkowali go.
\par 66 A gdy Piotr byl we dworze na dole, przyszla jedna z dziewek najwyzszego kaplana;
\par 67 A ujrzawszy Piotra grzejacego sie, wejrzala nan, i rzekla: I tys byl z Jezusem Nazarenskim.
\par 68 Ale on sie zaprzal, mówiac: Nie znam go, a nie wiem, co ty mówisz. I wyszedl na dwór do przysionka, a kur zapial.
\par 69 Tedy dziewka ujrzawszy go zasie, poczela mówic tym, którzy tam stali: Ten jest jeden z nich.
\par 70 A on zasie zaprzal sie. A znowu po malej chwilce ci, co tam stali, rzekli Piotrowi: Prawdziwie z nich jestes; bo jestes i Galilejczyk, i mowa twoja podobna jest.
\par 71 A on sie poczal przeklinac i przysiegac, mówiac: Nie znam czlowieka tego, o którym mówicie.
\par 72 Tedy po wtóre kur zapial. I wspomnial Piotr na slowa, które mu byl powiedzial Jezus: ze pierwej niz kur dwakroc zapieje, trzykroc sie mnie zaprzesz. A wyszedlszy, plakal.

\chapter{15}

\par 1 A zaraz rano naradziwszy sie przedniejsi kaplani z starszymi i z nauczonymi w Pismie i ze wszystka rada, zwiazali Jezusa, i wiedli go, i podali Pilatowi.
\par 2 I pytal go Pilat: Tyzes jest król Zydowski? A on mu odpowiadajac rzekl: Ty powiadasz.
\par 3 I skarzyli nan przedniejsi kaplani o wiele rzeczy: (ale on nic nie odpowiedzial.)
\par 4 Tedy go zasie pytal Pilat, mówiac: Nic nie odpowiadasz? Oto jako wiele rzeczy swiadcza przeciwko tobie.
\par 5 Ale Jezus przecie nic nie odpowiedzial, tak iz sie Pilat dziwowal.
\par 6 A na swieto zwykl im byl wypuszczac wieznia jednego, o którego by prosili.
\par 7 I byl jeden, którego zwano Barabbasz, w wiezieniu z tymi, co rozruch czynia, którzy byli w rozruchu mezobójstwo popelnili.
\par 8 Tedy lud wystapiwszy i glosem zawolawszy, poczal prosic, zeby uczynil tak, jako im zawsze czynil,
\par 9 Ale Pilat im odpowiedzial, mówiac: Chceciez, wypuszcze wam króla Zydowskiego?
\par 10 (Wiedzial bowiem, iz go z nienawisci wydali przedniejsi kaplani.)
\par 11 Ale przedniejsi kaplani podburzali lud, izby im raczej Barabbasza wypuscil.
\par 12 A odpowiadajac Pilat, rzekl im zasie: Cóz tedy chcecie, abym uczynil temu, którego nazywacie królem zydowskim?
\par 13 A oni znowu zawolali: Ukrzyzuj go!
\par 14 A Pilat rzekl do nich: I cóz wzdy zlego uczynil? Ale oni tem bardziej wolali: Ukrzyzuj go!
\par 15 A tak Pilat, chcac ludowi dosyc uczynic, wypuscil im Barabbasza, a Jezusa ubiczowawszy, podal im, aby byl ukrzyzowany.
\par 16 Lecz zolnierze wprowadzili go do dworu, to jest do ratusza, i zwolali wszystkiej roty.
\par 17 A obleklszy go w szarlat, i uplótlszy korone z ciernia, wlozyli nan;
\par 18 I poczeli go pozdrawiac, mówiac: Badz pozdrowiony, królu zydowski!
\par 19 I bili glowe jego trzcina i plwali nan, a upadajac na kolana, klaniali mu sie.
\par 20 A gdy sie z niego nasmiali, zewlekli go z szarlatu, i oblekli go w szaty jego wlasne, i wiedli go, aby go ukrzyzowali.
\par 21 Tedy przymusili mimo idacego niektórego Szymona Cyrenejczyka, (który szedl z pola,)ojca Aleksandrowego i Rufowego, aby niósl krzyz jego.
\par 22 I przywiedli go na miejsce Golgota, co sie wyklada: Miejsce trupich glów.
\par 23 I dawali mu pic wino z myrra; ale go on nie przyjal.
\par 24 A gdy go ukrzyzowali, rozdzielili szaty jego, miecac o nie los, co by kto wziac mial.
\par 25 A byla trzecia godzina, gdy go ukrzyzowali.
\par 26 Byl tez napis winy jego napisany: Król zydowski.
\par 27 Ukrzyzowali tez z nim dwóch zbójców; jednego po prawicy, a drugiego po lewicy jego.
\par 28 I wypelnilo sie Pismo, które mówi: Z zloczyncami jest policzony.
\par 29 A ci, którzy mimo chodzili, bluznili go, chwiejac glowami swemi a mówiac: Ehej! który rozwalasz kosciól, a we trzech dniach budujesz go!
\par 30 Ratuj samego siebie, a zstap z krzyza!
\par 31 Takze tez i przedniejsi kaplani nasmiewajac sie, jedni do drugich z nauczonymi w Pismie mówili: Innych ratowal, a siebie samego ratowac nie moze;
\par 32 Niechze teraz Chrystus on król Izraelski zstapi z krzyza, abysmy ujrzeli i uwierzyli. Ci tez, co z nim byli ukrzyzowani, uragali mu.
\par 33 A gdy bylo o godzinie szóstej, stala sie ciemnosc po wszystkiej ziemi, az do godziny dziewiatej.
\par 34 A o godzinie dziewiatej zawolal Jezus glosem wielkim, mówiac: Eloi! Eloi! Lamma sabachtani, co sie wyklada: Boze mój! Boze mój! czemus mie opuscil?
\par 35 A niektórzy z tych, co tam stali, uslyszawszy mówili: Oto Elijasza wola.
\par 36 Zatem biezawszy jeden, napelnil gabke octem, a wlozywszy ja na trzcine, dawal mu pic, mówiac: Zaniechajcie, patrzmy, jezli przyjdzie Elijasz, zdejmowac go.
\par 37 A Jezus zawolawszy glosem wielkim, oddal ducha.
\par 38 I rozerwala sie zaslona koscielna na dwoje, od wierzchu az do dolu.
\par 39 Tedy widzac setnik, który stal przeciwko niemu, iz tak wolajac oddal ducha, rzekl: Prawdziwie czlowiek ten byl Synem Bozym.
\par 40 Byly tez i niewiasty z daleka sie przypatrujac, miedzy któremi byla Maryja Magdalena, i Maryja, Jakóba malego i Jozesa matka, i Salome;
\par 41 Które gdy jeszcze byly w Galilei, chodzily za nim, a poslugowaly mu; i wiele innych, które z nim byly wstapily do Jeruzalemu.
\par 42 A gdy juz byl wieczór, (poniewaz byl dzien przygotowania,)który jest przed sabatem,
\par 43 Przyszedlszy Józef z Arymatyi, poczesny radny pan, który tez sam oczekiwal królestwa Bozego, smiele wszedl do Pilata, i prosil o cialo Jezusowe.
\par 44 A Pilat sie dziwowal, jezliby juz umarl; i zawolawszy setnika, pytal go, dawnoli umarl?
\par 45 A dowiedziawszy sie od setnika, darowal cialo Józefowi.
\par 46 A on kupiwszy przescieradlo, zdjawszy go, obwinal w przescieradlo, i polozyl go w grobie, który byl wykowany z opoki, i przywalil kamien do drzwi grobowych.
\par 47 Ale Maryja Magdalena, i Maryja, matka Jozesowa, patrzaly, kedy go polozono.

\chapter{16}

\par 1 A gdy minal sabat, Maryja Magdalena, i Maryja, matka Jakóbowa, i Salome, nakupily wonnych rzeczy, aby przyszedlszy namazaly go.
\par 2 A bardzo rano pierwszego dnia po sabacie przyszly do grobu, gdy weszlo slonce.
\par 3 I mówili do siebie: Któz nam odwali kamien ode drzwi grobowych?
\par 4 (A spojrzawszy ujrzaly, iz byl kamien odwalony;) bo byl bardzo wielki.
\par 5 I wszedlszy w grób, ujrzaly mlodzienca, siedzacego na prawicy, odzianego szata biala; i ulekly sie,
\par 6 Ale im on rzekl: Nie lekajcie sie; Jezusa szukacie onego Nazarenskiego, który byl ukrzyzowany; wstal z martwych, nie masz go tu; oto miejsce, gdzie go bylo polozono.
\par 7 Ale idzcie, a powiedzcie uczniom jego, i Piotrowi, ze was uprzedza do Galilei; tam go ogladacie, jako wam powiedzial.
\par 8 A wyszedlszy predko, uciekly od grobu: albowiem zdjelo je drzenie i zdumienie, a nikomu nic nie powiadaly; bo sie baly.
\par 9 A Jezus, gdy zmartwychwstal raniuczko pierwszego dnia po sabacie, ukazal sie naprzód Maryi Magdalenie, z której byl wygnal siedm dyjablów.
\par 10 A ona szedlszy, opowiedziala tym, co z nim bywali, którzy sie smucili i plakali.
\par 11 A oni uslyszawszy, iz zyje, a iz jest widziany od niej, nie wierzyli.
\par 12 Potem sie tez dwom z nich idacym ukazal w innym ksztalcie, gdy szli przez pole.
\par 13 A ci szedlszy opowiedzieli drugim; i tym nie uwierzyli.
\par 14 Na ostatek sie tez onym jedenastu wespól siedzacym ukazal, i wyrzucal im na oczy niedowiarstwo ich, i zatwardzenie serca, iz tym, którzy go widzieli wzbudzonego, nie wierzyli.
\par 15 I rzekl im: Idac na wszystek swiat, kazcie Ewangielije wszystkiemu stworzeniu.
\par 16 Kto uwierzy, a ochrzci sie, zbawion bedzie; ale kto nie uwierzy, bedzie potepion.
\par 17 A znamiona tych, co uwierza, te nasladowac beda: W imieniu mojem dyjably wyganiac beda, nowemi jezykami mówic beda;
\par 18 Weze brac beda, a chocby co smiertelnego pili, nie zaszkodzi im; na niemocne rece klasc beda, a dobrze sie miec beda.
\par 19 A tak Pan przestawszy z nimi mówic, wziety jest do nieba, i usiadl na prawicy Bozej.
\par 20 A oni wyszedlszy kazali wszedy, a Pan im pomagal, i slowa ich potwierdzal przez cuda, które czynili.


\end{document}