\begin{document}

\title{Jana}


\chapter{1}

\par 1 Na poczatku bylo Slowo, a ono Slowo bylo u Boga, a Bogiem bylo ono Slowo.
\par 2 To bylo na poczatku u Boga.
\par 3 Wszystkie rzeczy przez nie sie staly, a bez niego nic sie nie stalo, co sie stalo.
\par 4 W niem byl zywot, a zywot byl ona swiatloscia ludzka.
\par 5 A ta swiatlosc w ciemnosciach swieci, ale ciemnosci jej nie ogarnely.
\par 6 Byl czlowiek poslany od Boga, któremu imie bylo Jan.
\par 7 Ten przyszedl na swiadectwo, aby swiadczyl o tej swiatlosci, aby przezen wszyscy uwierzyli.
\par 8 Nie bylci on ta swiatloscia, ale przyszedl, aby swiadczyl o tej swiatlosci.
\par 9 Tenci byl ta prawdziwa swiatloscia, która oswieca kazdego czlowieka, przychodzacego na swiat.
\par 10 Na swiecie byl, a swiat przezen uczyniony jest; ale go swiat nie poznal.
\par 11 Do swej wlasnosci przyszedl, ale go wlasni jego nie przyjeli.
\par 12 Lecz którzykolwiek go przyjeli, dal im te moc, aby sie stali synami Bozymi, to jest tym, którzy wierza w imie jego.
\par 13 Którzy nie z krwi, ani z woli ciala, ani z woli meza, ale z Boga narodzeni sa.
\par 14 A to Slowo cialem sie stalo, i mieszkalo miedzy nami, i widzielismy chwale jego, chwale jako jednorodzonego od Ojca, pelne laski i prawdy.
\par 15 Jan swiadczyl o nim, i wolal, mówiac: Tenci byl, o którymem powiadal: Który po mnie przyszedlszy, uprzedzil mie; bo pierwej byl niz ja.
\par 16 A z pelnosci jego mysmy wszyscy wzieli i laske za laske.
\par 17 Albowiem zakon przez Mojzesza jest dany, a laska i prawda przez Jezusa Chrystusa stala sie.
\par 18 Boga nikt nigdy nie widzial: on jednorodzony syn, który jest w lonie ojcowskiem, ten nam opowiedzial.
\par 19 A toc jest swiadectwo Janowe, gdy poslali Zydzi z Jeruzalemu kaplany i Lewity, aby go pytali: Ty ktos jest?
\par 20 I wyznal, a nie zaprzal, a wyznal, zem ja nie jest Chrystus.
\par 21 I pytali go: Cózes tedy? Elijaszes ty? A on rzekl: Nie jestem. A oni: Prorokiemes ty? i odpowiedzial: Nie jestem.
\par 22 Rzekli mu tedy: Któzes jest, zebysmy odpowiedz dali tym, którzy nas poslali? Cóz wzdy powiadasz o sobie?
\par 23 Rzekl: Jam jest glos wolajacego na puszczy: Prostujcie droge Panska, jako powiedzial Izajasz prorok.
\par 24 A ci, którzy byli poslani, byli z Faryzeuszów.
\par 25 I pytali go i rzekli mu: Czemuz tedy chrzcisz, jezlizes ty nie jest Chrystus, ani Elijasz, ani prorok?
\par 26 Odpowiedzial im Jan, mówiac: Jac chrzcze woda; ale w posrodku was stoi, którego wy nie znacie.
\par 27 Tenci jest, który po mnie przyszedlszy, uprzedzil mie, któremum ja nie jest godzien, zebym rozwiazal rzemyk obuwia jego.
\par 28 To sie stalo w Betabarze za Jordanem, gdzie Jan chrzcil.
\par 29 A nazajutrz ujrzal Jan Jezusa idacego do siebie, i rzekl: Oto Baranek Bozy, który gladzi grzech swiata.
\par 30 Tenci jest, o którymem powiadal, ze idzie za mna maz, który mie uprzedzil; bo pierwej byl niz ja.
\par 31 A jam go nie znal; ale aby byl objawiony Izraelowi, dlategom ja przyszedl, chrzczac woda.
\par 32 I swiadczyl Jan, mówiac: Widzialem Ducha zstepujacego jako golebice z nieba, i zostal na nim.
\par 33 A jam go nie znal; ale który mie poslal chrzcic woda, ten mi rzekl: Na kogo bys ujrzal Ducha zstepujacego i zostajacego na nim, tenci jest, który chrzci Duchem Swietym.
\par 34 A jam widzial i swiadczyl, ze ten jest Syn Bozy.
\par 35 Nazajutrz zasie stal Jan i dwaj z uczniów jego.
\par 36 A ujrzawszy Jezusa chodzacego, rzekl: Oto Baranek Bozy.
\par 37 I slyszeli go oni dwaj uczniowie mówiacego, i szli za Jezusem.
\par 38 A obróciwszy sie Jezus i ujrzawszy je za soba idace, rzekl do nich: Czego szukacie? A oni mu rzekli: Rabbi! (co sie wyklada: Mistrzu), gdzie mieszkasz?
\par 39 Rzekl im: Pójdzcie, a ogladajcie. I szli i widzieli, gdzie mieszkal, a zostali przy nim onego dnia; bo bylo okolo dziesiatej godziny.
\par 40 A byl Andrzej, brat Szymona Piotra, jeden z onych dwóch, którzy to slyszeli od Jana, i szli byli za nim.
\par 41 Ten najpierw znalazl Szymona, brata swego wlasnego, i rzekl mu: Znalezlismy Mesyjasza, co sie wyklada Chrystus.
\par 42 I przywiódl go do Jezusa. A wejrzawszy nan Jezus, rzekl: Tys jest Szymon, syn Jonasza; ty bedziesz nazwany Kiefas, co sie wyklada Piotr.
\par 43 A nazajutrz chcial Jezus wynijsc do Galilei, i znalazl Filipa i rzekl mu: Pójdz za mna.
\par 44 A Filip byl z Betsaidy, z miasta Andrzejowego i Piotrowego.
\par 45 Filip znalazl Natanaela i rzekl mu: Znalezlismy onego, o którym pisal Mojzesz w zakonie i prorocy, Jezusa, syna Józefowego, z Nazaretu.
\par 46 I rzekl mu Natanael: Mozesz z Nazaretu byc co dobrego? Rzekl mu Filip: Pójdz, a ogladaj!
\par 47 Ujrzawszy tedy Jezus Natanaela idacego do siebie, rzekl o nim: Oto prawdziwie Izraelczyk, w którym nie masz zdrady.
\par 48 Rzekl mu Natanael: Skadze mie znasz? Odpowiedzial Jezus i rzekl mu: Pierwej niz cie Filip zawolal, gdys byl pod figowem drzewem, widzialem cie.
\par 49 Odpowiedzial Natanael i rzekl mu: Mistrzu! tys jest on Syn Bozy, tys jest on król Izraelski.
\par 50 Odpowiedzial Jezus i rzekl mu: Izem ci powiedzial: Widzialem cie pod figowem drzewem, wierzysz; wieksze rzeczy nad te ujrzysz.
\par 51 I rzekl mu: Zaprawde, zaprawde powiadam wam: Od tego czasu ujrzycie niebo otworzone i Anioly Boze wstepujace i zstepujace na Syna czlowieczego.

\chapter{2}

\par 1 A dnia trzeciego bylo wesele w Kanie Galilejskiej, i byla tam matka Jezusowa.
\par 2 Wezwany tez byl i Jezus i uczniowie jego na ono wesele.
\par 3 A gdy nie stalo wina, rzekla matka Jezusowa do niego: Wina nie maja.
\par 4 Rzekl jej Jezus: Co ja mam z toba, niewiasto? jeszczec nie przyszla godzina moja.
\par 5 Rzekla matka jego slugom: Cokolwiek wam rzecze, uczyncie.
\par 6 I bylo tam szesc stagwi kamiennych, postawionych wedlug oczyszczenia zydowskiego, bioracych w sie kazda dwie albo trzy wiadra.
\par 7 Rzekl im Jezus: Napelnijcie te stagwie woda; i napelnili je az do wierzchu.
\par 8 Tedy im rzekl: Czerpajciez teraz, a doniescie przelozonemu wesela. I doniesli.
\par 9 A gdy skosztowal przelozony wesela onej wody, która sie stala winem, (a nie wiedzial, skad by bylo; lecz sludzy wiedzieli, którzy wode czerpali), zawolal on przelozony oblubienca;
\par 10 I rzekl mu: Kazdy czlowiek pierwej daje wino dobre, a gdy sobie podpija, tedy podlejsze; a tys dobre wino zachowal az do tego czasu.
\par 11 Tenci poczatek cudów uczynil Jezus w Kanie Galilejskiej, a objawil chwale swoje; i uwierzyli wen uczniowie jego.
\par 12 Potem zstapil do Kapernaum, on i matka jego i bracia jego i uczniowie jego, i zamieszkali tam niewiele dni;
\par 13 Albowiem byla blisko wielkanoc zydowska; i wstapil Jezus do Jeruzalemu.
\par 14 I znalazl w kosciele siedzace te, co sprzedawali woly i owce i golebie, i te, co pieniedzmi handlowali.
\par 15 A uczyniwszy bicz z powrozków, wszystkie wygnal z kosciola, i owce i woly: a tych, co pieniedzmi handlowali, pieniadze rozsypal i stoly poprzewracal;
\par 16 A tym, co golebie sprzedawali, rzekl: Wyniescie to stad, a nie czyncie domu Ojca mego domem kupieckim.
\par 17 I wspomnieli sobie uczniowie jego, iz napisano: Gorliwosc domu twego zzarla mie.
\par 18 Tedy odpowiedzieli Zydowie i rzekli mu: Cóz nam za znak pokazesz, iz to czynisz?
\par 19 Odpowiedzial Jezus i rzekl im: Rozwalcie ten kosciól, a we trzech dniach wystawie go.
\par 20 Rzekli tedy Zydowie: Czterdziesci i szesc lat budowano ten kosciól, a ty go we trzech dniach wystawisz?
\par 21 Ale on mówil o kosciele ciala swego.
\par 22 Przetoz, gdy zmartwychwstal, wspomnieli uczniowie jego, iz im to byl powiedzial; i uwierzyli Pismu i slowu, które wyrzekl Jezus.
\par 23 A gdy byl w Jeruzalemie na wielkanoc w swieto, wiele ich uwierzylo w imie jego, widzac cuda jego, które czynil.
\par 24 Ale Jezus nie zwierzal im samego siebie, przeto iz on znal wszystkie,
\par 25 A iz nie potrzebowal, aby mu kto swiadectwo wydawal o czlowieku; albowiem on wiedzial, co bylo w czlowieku.

\chapter{3}

\par 1 A byl niektóry czlowiek z Faryzeuszów, imieniem Nikodem, ksiaze zydowski.
\par 2 Ten przyszedl do Jezusa w nocy i rzekl mu: Mistrzu! wiemy, zes przyszedl od Boga nauczycielem; bo nikt tych cudów czynic nie moze, które ty czynisz, jezliby Bóg z nim nie byl.
\par 3 Odpowiedzial Jezus i rzekl mu: Zaprawde, zaprawde powiadam ci: Jezli sie kto nie narodzi znowu, nie moze widziec królestwa Bozego.
\par 4 Rzekl do niego Nikodem: Jakoz sie moze czlowiek narodzic, bedac stary? izali powtóre moze wnijsc w zywot matki swojej i narodzic sie?
\par 5 Odpowiedzial Jezus: Zaprawde, zaprawde powiadam ci: Jezliby sie kto nie narodzil z wody i z Ducha, nie moze wnijsc do królestwa Bozego.
\par 6 Co sie narodzilo z ciala, cialo jest, a co sie narodzilo z Ducha, duch jest.
\par 7 Nie dziwuj sie, zem ci powiedzial: Musicie sie znowu narodzic.
\par 8 Wiatr, gdzie chce, wieje i glos jego slyszysz, ale nie wiesz, skad przychodzi i dokad idzie; takzec jest kazdy, który sie narodzil z Ducha.
\par 9 Odpowiedzial Nikodem i rzekl mu: Jakoz to byc moze?
\par 10 Odpowiedzial Jezus i rzekl mu: Tys jest nauczycielem w Izraelu, a tego nie wiesz?
\par 11 Zaprawde, zaprawde powiadam ci, iz co wiemy, mówimy, a cosmy widzieli, swiadczymy: ale swiadectwa naszego nie przyjmujecie.
\par 12 Jezliz gdym wam ziemskie rzeczy powiadal, a nie wierzycie, jakoz, bedeli wam powiadal niebieskie, uwierzycie?
\par 13 A nikt nie wstapil do nieba, tylko ten, który zstapil z nieba, Syn czlowieczy, który jest w niebie.
\par 14 A jako Mojzesz weza na puszczy wywyzszyl, tak musi byc wywyzszony Syn czlowieczy.
\par 15 Aby kazdy, kto wen wierzy, nie zginal, ale mial zywot wieczny.
\par 16 Albowiem tak Bóg umilowal swiat, ze Syna swego jednorodzonego dal, aby kazdy, kto wen wierzy, nie zginal, ale mial zywot wieczny.
\par 17 Boc nie poslal Bóg Syna swego na swiat, aby sadzil swiat, ale aby swiat byl zbawiony przezen.
\par 18 Kto wierzy wen, nie bedzie osadzony; ale kto nie wierzy, juz jest osadzony, iz nie uwierzyl w imie jednorodzonego Syna Bozego.
\par 19 A tenci jest sad, ze swiatlosc przyszla na swiat, lecz ludzie bardziej umilowali ciemnosc niz swiatlosc; bo byly zle uczynki ich.
\par 20 Kazdy bowiem, kto zle czyni, nienawidzi swiatlosci i nie idzie na swiatlosc, aby nie byly zganione uczynki jego.
\par 21 Lecz kto czyni prawde, przychodzi do swiatlosci, aby byly jawne uczynki jego, iz w Bogu sa uczynione.
\par 22 Potem przyszedl Jezus i uczniowie jego do Judzkiej ziemi, i tam przemieszkiwal z nimi i chrzcil.
\par 23 Chrzcil tez i Jan w Enon, blisko Salim; bo tam bylo wiele wód, a ludzie przychodzili i chrzcili sie.
\par 24 Bo jeszcze Jan nie byl podany do wiezienia.
\par 25 Wszczela sie tedy gadka miedzy uczniami Janowymi i miedzy Zydami o oczyszczaniu.
\par 26 I przyszli do Jana i rzekli mu: Mistrzu! ten, który byl z toba za Jordanem, któremus ty dal swiadectwo, ten oto chrzci, a wszyscy ida do niego.
\par 27 Odpowiedzial Jan i rzekl: Nie moze nic wziac czlowiek, jezliby mu nie bylo dane z nieba.
\par 28 Wy sami jestescie mi swiadkami, zem powiedzial: Nie jestem ja Chrystus, ale zem poslany przed nim.
\par 29 Kto ma oblubienice, ten jest oblubieniec, a przyjaciel oblubienca, który stoi, a slucha go, weseli sie weselem dla glosu oblubiencowego; przetoz to wesele moje wypelnione jest.
\par 30 On musi rosc, a mnie musi ubywac.
\par 31 Kto z góry przyszedl, nade wszystkie jest; kto z ziemi jest, ziemski jest i ziemskie rzeczy mówi; ten, który z nieba przyszedl, nade wszystkie jest.
\par 32 A co widzial i slyszal, to swiadczy, ale swiadectwa jego zaden nie przyjmuje.
\par 33 Kto przyjmuje swiadectwo jego, ten zapieczetowal, ze Bóg jest prawdziwy.
\par 34 Albowiem ten, którego Bóg poslal, slowo Boze mówi; boc mu nie pod miara daje Bóg Ducha.
\par 35 Ojciec miluje Syna, i wszystko dal w rece jego.
\par 36 Kto wierzy w Syna, ma zywot wieczny; ale kto nie wierzy Synowi, nie oglada zywota, lecz gniew Bozy zostaje nad nim.

\chapter{4}

\par 1 A gdy poznal Pan, iz uslyszeli Faryzeuszowie, ze Jezus wiecej uczniów czynil i chrzcil nizeli Jan,
\par 2 (Chociaz sam Jezus nie chrzcil, ale uczniowie jego),
\par 3 Opuscil Judzka ziemie i odszedl zasie do Galilei.
\par 4 A musial isc przez Samaryje.
\par 5 I przyszedl do miasta Samaryi, które zowia Sychar, blisko folwarku, który byl dal Jakób Józefowi, synowi swemu.
\par 6 I byla tam studnia Jakóbowa; przetoz bedac Jezus na drodze spracowany, siedzial tak na studni; a bylo okolo szóstej godziny.
\par 7 I przyszla niewiasta z Samaryi czerpac wode, której rzekl Jezus: Daj mi pic!
\par 8 (Bo uczniowie jego odeszli byli do miasta, aby nakupili zywnosci.)
\par 9 Rzekla mu tedy ona niewiasta Samarytanska: Jakoz ty bedac Zydem, zadasz ode mnie napoju, od niewiasty Samarytanki? (gdyz Zydowie nie obcuja z Samarytany.)
\par 10 Odpowiedzial Jezus i rzekl jej: Gdybys wiedziala ten dar Bozy, i kto jest ten, co ci mówi: Daj mi pic, ty bys go prosila, a dalby ci wode zywa.
\par 11 I rzekla mu niewiasta: Panie! nie masz i czem naczerpac, a studnia jest gleboka, skadze tedy masz te wode zywa?
\par 12 Izazes ty jest wiekszy nizeli ojciec nasz Jakób, który nam dal te studnie, i sam z niej pil, i synowie jego, i dobytek jego?
\par 13 Odpowiedzial Jezus i rzekl jej: Kazdy, kto pije te wode, zasie bedzie pragnal;
\par 14 Lecz kto by pil one wode, która ja mu dam, nie bedzie pragnal na wieki; ale ta woda, która ja mu dam, stanie sie w nim studnia wody wyskakujacej ku zywotowi wiecznemu.
\par 15 Rzekla do niego niewiasta: Panie! daj mi tej wody, abym nie pragnela, ani tu czerpac chodzila.
\par 16 Rzekl jej Jezus: Idz, zawolaj meza swego, a przyjdz tu.
\par 17 Odpowiedziala niewiasta i rzekla: Nie mam meza. Rzekl jej Jezus: Dobrzes rzekla: Nie mam meza.
\par 18 Albowiemes pieciu mezów miala, a teraz ten, którego masz, nie jest mezem twoim; tos prawde powiedziala.
\par 19 Rzekla mu niewiasta: Panie! widze, zes ty jest prorok.
\par 20 Ojcowie nasi na tej górze chwalili Boga, a wy powiadacie, ze w Jeruzalemie jest miejsce, kedy przyzwoita chwalic.
\par 21 Rzekl jej Jezus: Niewiasto! wierz mi, iz idzie godzina, gdy ani na tej górze, ani w Jeruzalemie nie bedziecie chwalili Ojca.
\par 22 Wy chwalicie, co nie wiecie; a my chwalimy, co wiemy; albowiem zbawienie jest z Zydów.
\par 23 Alec idzie godzina, i teraz jest, gdy prawdziwi chwalcy beda chwalic Ojca w duchu i w prawdzie.
\par 24 Bo i Ojciec takowych szuka, którzy by go chwalili. Bóg jest duch, a ci, którzy go chwala, powinni go chwalic w duchu i w prawdzie.
\par 25 Rzekla mu niewiasta: Wiem, ze przyjdzie Mesyjasz, którego zowia Chrystusem, ten, gdy przyjdzie, oznajmi nam wszystko.
\par 26 Rzekl jej Jezus: Jam jest ten, który z toba mówie.
\par 27 A wtem przyszli uczniowie jego, i dziwowali sie, iz z niewiasta mówil; wszakze zaden nie rzekl: O co sie pytasz, albo co z nia rozmawiasz?
\par 28 I zostawila ona niewiasta wiadro swoje, a szla do miasta i rzekla onym ludziom:
\par 29 Pójdzcie, ogladajcie czlowieka, który mi powiedzial wszystko, comkolwiek czynila, nie tenci jest Chrystus?
\par 30 A przetoz wyszli z miasta i przyszli do niego.
\par 31 A tymczasem prosili go uczniowie, mówiac: Mistrzu! jedz.
\par 32 A on im rzekl: Mamci ja pokarm ku jedzeniu, o którym wy nie wiecie.
\par 33 Mówili tedy uczniowie miedzy soba: Alboc mu kto przyniósl jesc?
\par 34 Rzekl im Jezus: Mójci jest pokarm, abym czynil wole tego, który mie poslal, a dokonal sprawy jego.
\par 35 Izaz wy nie mówicie, ze jeszcze sa cztery miesiace, a zniwo przyjdzie? Otoz powiadam wam: Podniescie oczy wasze, a przypatrzcie sie krainom, zec juz biale sa ku zniwu.
\par 36 A kto znie, bierze zaplate, i zbiera owoc do zywota wiecznego, aby i ten, który sieje, radowal sie wespól, i ten, który znie.
\par 37 Albowiem w tem prawdziwe jest ono przyslowie: Ze inszy jest, który sieje, a inszy, który znie.
\par 38 Jam was poslal, zac to, okolo czegoscie wy nie pracowali; insic pracowali, a wyscie weszli w prace ich.
\par 39 Tedy z miasta onego wiele Samarytanów uwierzylo wen dla powiesci onej niewiasty, która swiadczyla: Ze mi wszystko powiedzial, comkolwiek czynila.
\par 40 Gdy tedy przyszli do niego Samarytanie, prosili go, aby u nich zostal; i zostal tam przez dwa dni.
\par 41 I daleko wiecej ich uwierzylo dla slowa jego.
\par 42 A onej niewiescie mówili: Iz juz nie dla twojej powiesci wierzymy; albowiemesmy sami slyszeli i wiemy, ze ten jest prawdziwie zbawiciel swiata, Chrystus.
\par 43 A po dwóch dniach wyszedl stamtad i szedl do Galilei.
\par 44 Albowiem sam Jezus swiadectwo wydal, iz prorok w ojczyznie swojej nie jest we czci.
\par 45 A gdy przyszedl do Galilei, przyjeli go Galilejczycy, widzac wszystko, co czynil w Jeruzalemie w swieto; bo i oni byli przyszli na swieto.
\par 46 Tedy zasie przyszedl Jezus do Kany Galilejskiej, gdzie byl uczynil z wody wino. A byl niektóry dworzanin królewski w Kapernaum, którego syn chorowal.
\par 47 Ten uslyszawszy, iz Jezus przyszedl z Judzkiej ziemi do Galilei, szedl do niego i prosil go, aby zstapil, a uzdrowil syna jego; bo poczynal umierac.
\par 48 I rzekl do niego Jezus: Jezli nie ujrzycie znamion i cudów, nie uwierzycie.
\par 49 Rzekl mu on królewski dworzanin: Panie! zstap pierwej niz umrze dziecie moje.
\par 50 Rzekl mu Jezus: Idz, syn twój zyje. I uwierzyl on czlowiek mowie, która mu powiedzial Jezus, i poszedl.
\par 51 A gdy juz szedl, zabiezeli mu sludzy jego i oznajmili, mówiac: Dziecie twoje zyje.
\par 52 Tedy ich pytal o godzine, w która by sie lepiej mialo; i rzekli mu, ze wczoraj o siódmej godzinie opuscila go goraczka.
\par 53 Poznal tedy ojciec, iz to ona godzina byla, której mu byl rzekl Jezus: Iz syn twój zyje. I uwierzyl sam i wszystek dom jego.
\par 54 Tenci zasie wtóry cud uczynil Jezus, przyszedlszy z Judzkiej ziemi do Galilei.

\chapter{5}

\par 1 Bylo potem swieto zydowskie, i wstapil Jezus do Jeruzalemu.
\par 2 A byla w Jeruzalemie przy owczej bramie sadzawka, która zowia po zydowsku Betesda, majaca piec ganków.
\par 3 W tych lezalo mnóstwo wielkie niedoleznych, slepych, chromych, wyschlych, którzy czekali poruszenia wody.
\par 4 Albowiem Aniol czasu pewnego zstepowal w sadzawke i poruszal wode; a tak, kto pierwszy wstapil po wzruszeniu wody, stawal sie zdrowym, jakabykolwiek choroba zdjety byl.
\par 5 A byl tam niektóry czlowiek trzydziesci i osm lat choroba zlozony.
\par 6 Tego gdy Jezus ujrzal lezacego, a poznawszy, ze juz przez dlugi czas chorowal, rzekl mu: Chcesz byc zdrów?
\par 7 Odpowiedzial mu on chory: Panie! nie ma czlowieka, który by mie, gdy bywa poruszona woda, wrzucil do sadzawki; ale gdy ja ide, inszy przede mna wstepuje.
\par 8 Rzekl mu Jezus: Wstan, wezmij loze twoje, a chodz.
\par 9 A zarazem stal sie zdrowym on czlowiek, i wzial loze swoje, i chodzil. A byl sabat onego dnia.
\par 10 Tedy rzekli Zydowie onemu uzdrowionemu: Sabat jest, nie godzi ci sie loza nosic.
\par 11 Odpowiedzial im: Ten, który mie uzdrowil, tenze mi rzekl: Wezmij loze twoje, a chodz.
\par 12 I pytali go: Któryz jest ten czlowiek, co ci powiedzial: Wezmij loze twoje, a chodz?
\par 13 A on uzdrowiony nie wiedzial, kto by byl; albowiem byl Jezus ustapil, poniewaz wiele ludu bylo na onem miejscu.
\par 14 Potem go Jezus znalazl w kosciele i rzekl mu: Otos sie stal zdrowym, nie grzesz wiecej, aby co gorszego na cie nie przyszlo.
\par 15 A odszedlszy on czlowiek, powiedzial Zydom, iz to byl Jezus, który go uzdrowil.
\par 16 A przetoz Zydowie przesladowali Jezusa i szukali, jakoby go zabili, ze to uczynil w sabat.
\par 17 A Jezus im odpowiedzial: Ojciec mój az dotad pracuje, i ja pracuje;
\par 18 Dlatego tedy tem wiecej szukali Zydowie, jakoby go zabili, nie tylko, iz gwalcil sabat, ale ze i Ojca swego powiadal byc Bogiem, czyniac sie równym Bogu.
\par 19 Odpowiedzial tedy Jezus i rzekl im: Zaprawde, zaprawde powiadam wam, nie moze Syn sam od siebie nic czynic, jedno co widzi, ze Ojciec czyni; albowiem cokolwiek on czyni, to takze i Syn czyni.
\par 20 Boc Ojciec miluje Syna i ukazuje mu wszystko, co sam czyni, i wieksze mu nad te sprawy pokaze, abyscie sie wy dziwowali.
\par 21 Albowiem jako Ojciec wzbudza umarle i ozywia, tak i Syn, które chce, ozywia.
\par 22 Bo Ojciec nikogo nie sadzi, lecz wszystek sad dal Synowi,
\par 23 Aby wszyscy czcili Syna, tak jako czcza Ojca; kto nie czci Syna, nie czci i Ojca, który go poslal.
\par 24 Zaprawde, zaprawde powiadam wam: Kto slowa mego slucha i wierzy onemu, który mie poslal, ma zywot wieczny i nie przyjdzie na sad, ale przeszedl z smierci do zywota.
\par 25 Zaprawde, zaprawde powiadam wam: Ze idzie godzina i teraz jest, gdy umarli uslysza glos Syna Bozego, a którzy uslysza, zyc beda.
\par 26 Albowiem jako Ojciec ma zywot sam w sobie, tak dal i Synowi, aby mial zywot w samym sobie.
\par 27 I dal mu moc i sad czynic; bo jest Synem czlowieczym.
\par 28 Nie dziwujciez sie temu; boc przyjdzie godzina, w która wszyscy, co sa w grobach, uslysza glos jego;
\par 29 I pójda ci, którzy dobrze czynili, na powstanie zywota; ale ci, którzy zle czynili, na powstanie sadu.
\par 30 Nie mogec ja sam od siebie nic czynic; jako slysze, tak sadze, a sad mój jest sprawiedliwy; bo nie szukam woli mojej, ale woli tego, który mie poslal, Ojca.
\par 31 Jezlizec ja sam o sobie swiadcze, swiadectwo moje nie jest prawdziwe.
\par 32 Inszy jest, co o mnie swiadczy, i wiem, ze prawdziwe jest swiadectwo, które wydaje o mnie.
\par 33 Wyscie slali do Jana, a on dal swiadectwo prawdzie.
\par 34 Ale ja nie od czlowieka swiadectwo biore, ale to mówie, abyscie wy byli zbawieni.
\par 35 Onci byl swieca gorejaca i swiecaca, a wyscie sie chcieli do czasu poradowac w swiatlosci jego.
\par 36 Ale ja mam swiadectwo wieksze niz Janowe; albowiem sprawy, które mi dal Ojciec, abym je wykonal, te same sprawy, które ja czynie, swiadcza o mnie, iz mie Ojciec poslal.
\par 37 A Ojciec, który mie poslal, onze swiadczyl o mnie, któregoscie wy glosu nigdy nie slyszeli, aniscie osoby jego widzieli;
\par 38 I slowa jego nie macie w sobie mieszkajacego; albowiem, którego on poslal, temu nie wierzycie.
\par 39 Badajciez sie Pism; boc sie wam zda, ze w nich zywot wieczny macie, a one sa, które swiadectwo wydawaja o mnie.
\par 40 A wzdy do mnie przyjsc nie chcecie, abyscie zywot mieli.
\par 41 Chwaly od ludzi nie przyjmuje.
\par 42 Alem was poznal, ze milosci Bozej nie macie w sobie.
\par 43 Jam przyszedl w imieniu Ojca mego, a nie przyjmujecie mnie: jezlizby przyszedl inny w imieniu swojem, onego przyjmiecie.
\par 44 Jakoz wy mozecie wierzyc, chwale jedni od drugich przyjmujac, poniewaz chwaly, która jest od samego Boga, nie szukacie?
\par 45 Nie mniemajcie, abym ja was mial oskarzac przed Ojcem; jestci, który skarzy na was, Mojzesz, w którym wy nadzieje macie.
\par 46 Bo gdybyscie wierzyli Mojzeszowi, wierzylibyscie i mnie; gdyz on o mnie pisal.
\par 47 Ale poniewaz pismom jego nie wierzycie, i jakoz slowom moim uwierzycie?

\chapter{6}

\par 1 Potem odszedl Jezus za morze Galilejskie, które jest Tyberyjadzkie;
\par 2 I szedl za nim lud wielki, iz widzieli cuda jego, które czynil nad chorymi.
\par 3 I wszedl Jezus na góre, i siedzial tam z uczniami swoimi;
\par 4 A byla blisko wielkanoc, swieto zydowskie.
\par 5 Tedy podniósl Jezus oczy i ujrzawszy, iz wielki lud idzie do niego, rzekl do Filipa: Skad kupimy chleba, aby ci jedli?
\par 6 (Ale to mówil, kuszac go; bo on wiedzial, co mial czynic.)
\par 7 Odpowiedzial mu Filip: Za dwiescie groszy chleba nie dosyc im bedzie, chocby kazdy z nich malo co wzial.
\par 8 Rzekl mu jeden z uczniów jego, Andrzej, brat Szymona Piotra:
\par 9 Jest tu jedno pachole, co ma piecioro chleba jeczmiennego i dwie rybki; ale cóz to jest na tak wielu?
\par 10 Tedy rzekl Jezus: Kazcie ludowi usiasc. A bylo trawy dosc na onemze miejscu, i usiadlo mezów w liczbie okolo pieciu tysiecy.
\par 11 Wzial tedy Jezus one chleby, a podziekowawszy rozdal uczniom, a uczniowie siedzacym; takze i z onych rybek, ile jedno chcieli.
\par 12 A gdy byli nasyceni, rzekl uczniom swoim: Zbierzcie te ulomki, które zbywaja, zeby nic nie zginelo.
\par 13 I zebrali i napelnili dwanascie koszów ulomków z onego pieciorga chleba jeczmiennego, które zbywaly tym, co jedli.
\par 14 A oni ludzie, ujrzawszy cud, który uczynil Jezus, mówili: Tenci jest zaprawde on prorok, który mial przyjsc na swiat.
\par 15 Tedy Jezus poznawszy, iz mieli przyjsc i porwac go, aby go uczynili królem, uszedl zasie sam tylko na góre.
\par 16 A gdy byl wieczór, zstapili uczniowie jego do morza.
\par 17 A wstapiwszy w lódz, jechali za morze do Kapernaum, a juz bylo ciemno, a Jezus nie przyszedl byl do nich.
\par 18 A morze, gdy powstal wielki wiatr, burzyc sie poczynalo.
\par 19 Gdy tedy odplyneli jakoby na dwadziescia i piec lub trzydziesci stajan, ujrzeli Jezusa chodzacego po morzu, przyblizajacego sie ku lodzi, i ulekli sie.
\par 20 A on im rzekl: Jamci jest, nie bójcie sie.
\par 21 I wzieli go ochotnie do lodzi, a zarazem lódz przyplynela do ziemi, do której jechali.
\par 22 Nazajutrz lud, który byl za morzem, widzac, ze tam nie bylo drugiej lodzi, tylko ona jedna, w która byli wstapili uczniowie jego, a iz Jezus nie wszedl byl w lódz z uczniami swoimi, ale sami uczniowie jego ujechali;
\par 23 (Przyszly tez byly drugie lodzie z Tyberyjady, blisko do onego miejsca, gdzie jedli chleb, gdy byl Pan dzieki uczynil.)
\par 24 To gdy obaczyl lud, iz tam nie bylo Jezusa, ani uczniów jego, wstapili i oni w lodzie i przeprawili sie do Kapernaum, szukajac Jezusa;
\par 25 A znalazlszy go za morzem, rzekli mu: Mistrzu! kiedys tu przybyl?
\par 26 Odpowiedzial im Jezus i rzekl: Zaprawde, zaprawde powiadam wam: Szukacie mie nie przeto, izescie widzieli cuda, ale izescie jedli chleb, i byliscie nasyceni.
\par 27 Sprawujciez nie pokarm, który ginie, ale pokarm, który trwa ku zywotowi wiecznemu, który wam da Syn czlowieczy; albowiem tego zapieczetowal Bóg Ojciec.
\par 28 Rzekli tedy do niego: Cóz bedziemy czynili, abysmy sprawowali sprawy Boze?
\par 29 Odpowiedzial Jezus i rzekl im: Toc jest sprawa Boza, abyscie wierzyli w tego, którego on poslal.
\par 30 Rzekli mu tedy: Cóz wzdy ty za znak czynisz, abysmy widzieli i wierzyli tobie? Cóz czynisz?
\par 31 Ojcowie nasi jedli manne na puszczy, jako jest napisano: Chleb z nieba dal im ku jedzeniu.
\par 32 Rzekl im tedy Jezus: Zaprawde, zaprawde powiadam wam: Nie Mojzesz wam dal chleb z nieba, ale Ojciec mój daje wam chleb on prawdziwy z nieba.
\par 33 Albowiem chleb Bozy ten jest, który zstepuje z nieba i zywot daje swiatu.
\par 34 Tedy mu rzekli: Panie! daj nam zawsze tego chleba.
\par 35 I rzekl im Jezus: Jamci jest on chleb zywota; kto do mnie przychodzi, laknac nie bedzie, a kto wierzy w mie, nigdy pragnac nie bedzie.
\par 36 Alem wam powiedzial: Owszem, widzieliscie mie, a nie wierzycie.
\par 37 Wszystko, co mi daje Ojciec, do mnie przyjdzie, a tego, co do mnie przyjdzie, nie wyrzuce precz.
\par 38 Bom zstapil z nieba, nie izbym czynil wole moje, ale wole onego, który mie poslal.
\par 39 A tac jest wola onego, który mie poslal, Ojca, abym z tego wszystkiego, co mi dal, nic nie stracil, ale abym to wzbudzil w on ostateczny dzien.
\par 40 A tac jest wola onego, który mie poslal, aby kazdy, kto widzi Syna, a wierzy wen, mial zywot wieczny; a ja go wzbudze w on ostateczny dzien.
\par 41 I szemrali Zydowie o nim, iz rzekl: Jam jest on chleb, który z nieba zstapil.
\par 42 I mówili: Izaz ten nie jest Jezus, syn Józefa, którego my ojca i matke znamy; jakoz teraz tedy ten powiada: Zem z nieba zstapil?
\par 43 Tedy odpowiedzial Jezus i rzekl im: Nie szemrzyjcie miedzy soba.
\par 44 Zaden do mnie przyjsc nie moze, jezli go Ojciec mój, który mie poslal, nie pociagnie; a ja go wzbudze w ostateczny dzien.
\par 45 Napisano w prorokach: I beda wszyscy wyuczeni od Boga; przetoz kazdy, kto slyszal od Ojca, a nauczyl sie, przychodzi do mnie.
\par 46 Nie izby kto widzial Ojca, oprócz tego, który jest od Boga; ten widzial Ojca.
\par 47 Zaprawde, zaprawde powiadam wam: Kto w mie wierzy, ma zywot wieczny.
\par 48 Jam jest on chleb zywota.
\par 49 Ojcowie wasi jedli manne na puszczy, a pomarli.
\par 50 Ten jest on chleb, który z nieba zstepuje; jezliby go kto jadl, nie umrze.
\par 51 Jamci jest chleb on zywy, którym z nieba zstapil: jezliby kto jadl z tego chleba, zyc bedzie na wieki; a chleb, który ja dam, jest cialo moje, które ja dam za zywot swiata.
\par 52 Wadzili sie tedy Zydowie miedzy soba, mówiac: Jakoz ten moze nam dac cialo swoje ku jedzeniu?
\par 53 I rzekl im Jezus: Zaprawde, zaprawde powiadam wam: Jezli nie bedziecie jedli ciala Syna czlowieczego, i pili krwi jego, nie macie zywota w sobie.
\par 54 Kto je cialo moje, a pije krew moje, ma zywot wieczny, a ja go wzbudze w on ostateczny dzien.
\par 55 Albowiem cialo moje prawdziwie jest pokarm, a krew moja prawdziwie jest napój.
\par 56 Kto je cialo moje i pije krew moje, we mnie mieszka, a ja w nim.
\par 57 Jako mie poslal zyjacy Ojciec, i ja zyje przez Ojca; tak kto mnie pozywa, i on zyc bedzie przez mie.
\par 58 Tenci jest chleb on, który z nieba zstapil, nie jako ojcowie wasi jedli manne, a pomarli; kto je ten chleb, zyc bedzie na wieki.
\par 59 To mówil w bóznicy, uczac w Kapernaum.
\par 60 Wiele ich tedy z uczniów jego slyszac to, mówili: Twardac to jest mowa, któz jej sluchac moze?
\par 61 Ale wiedzac Jezus sam w sobie, iz o tem szemrali uczniowie jego, rzekl im: Toz was obraza?
\par 62 Cóz, gdybyscie ujrzeli Syna czlowieczego wstepujacego, gdzie byl pierwej?
\par 63 Duchci jest, który ozywia, cialo nic nie pomaga; slowa, które ja wam mówie, duch sa i zywot sa.
\par 64 Ale sa niektórzy z was, co nie wierza; albowiem wiedzial od poczatku Jezus, którzy byli, co nie wierzyli, i kto jest, co go mial wydac;
\par 65 I mówil: Dlategomci wam powiedzial: Iz zaden nie moze przyjsc do mnie, jezliby mu nie bylo dane od Ojca mojego.
\par 66 Od tego czasu wiele uczniów jego odeszlo nazad, a wiecej z nim nie chodzili.
\par 67 Tedy rzekl Jezus do onych dwunastu: Izali i wy chcecie odejsc?
\par 68 I odpowiedzial mu Szymon Piotr: Panie! do kogóz pójdziemy? Ty masz slowa zywota wiecznego;
\par 69 A mysmy uwierzyli i poznali, zes ty jest Chrystus, on Syn Boga zywego.
\par 70 Odpowiedzial im Jezus: Izalim ja nie dwunastu was obral? a jeden z was jest dyjabel.
\par 71 A to mówil o Judaszu, synu Szymona, Iszkaryjocie; bo go ten wydac mial, bedac jednym z onych dwunastu.

\chapter{7}

\par 1 A potem chodzil Jezus po Galilei; bo sie nie chcial bawic w ziemi Judzkiej, przeto ze Zydowie szukali, aby go zabili.
\par 2 I bylo blisko swieto zydowskie kuczek.
\par 3 Tedy rzekli do niego bracia jego: Odejdz stad, a idz do Judzkiej ziemi, zeby uczniowie twoi widzieli sprawy twoje, które czynisz.
\par 4 Albowiem zaden nic w skrytosci nie czyni, kto chce byc widziany; przetoz ty, jezli takie rzeczy czynisz, objaw sie swiatu.
\par 5 Bo i bracia jego nie wierzyli wen.
\par 6 I rzekl im Jezus: Czas mój jeszcze nie przyszedl; ale czas wasz zawsze jest w pogotowiu.
\par 7 Nie mozec was swiat nienawidziec, ale mnie nienawidzi; bo ja swiadcze o nim, iz sprawy jego zle sa.
\par 8 Idzciez wy na to swieto, jac jeszcze nie pójde na to swieto; bo mój czas jeszcze sie nie wypelnil.
\par 9 A to im powiedziawszy, zostal w Galilei.
\par 10 A gdy poszli bracia jego, tedy i on szedl na swieto, nie jawnie, ale jakoby potajemnie.
\par 11 A Zydowie szukali go w swieto i mówili: Gdziez on jest?
\par 12 I bylo o nim wielkie szemranie miedzy ludem; bo jedni mówili: Ze jest dobry; a drudzy mówili: Nie, ale zwodzi lud.
\par 13 Wszakze o nim zaden jawnie nie mówil, dla bojazni zydowskiej.
\par 14 A gdy juz bylo w pól swieta, wstapil Jezus do kosciola i uczyl.
\par 15 I dziwowali sie Zydowie, mówiac: Jakoz ten umie Pismo, gdyz sie nie uczyl?
\par 16 Odpowiedzial im Jezus i rzekl: Nauka moja nie jestci moja, ale tego, który mie poslal.
\par 17 Jezliby kto chcial czynic wole jego, ten bedzie umial rozeznac, jezli ta nauka jest z Boga, czyli ja sam od siebie mówie.
\par 18 Ktoc z samego siebie mówi, chwaly wlasnej szuka; ale kto szuka chwaly tego, który go poslal, ten jest prawdziwy, a nie masz w nim niesprawiedliwosci.
\par 19 Izali wam Mojzesz nie dal zakonu? a zaden z was nie przestrzega zakonu. Przeczze szukacie, abyscie mie zabili?
\par 20 Odpowiedzial lud i rzekl: Dyjabelstwo masz; któz cie szuka zabic?
\par 21 Odpowiedzial Jezus i rzekl im: Jedenem uczynek uczynil, a wszyscy sie temu dziwujecie!
\par 22 Wszak Mojzesz wydal wam obrzezke, (nie izby byla z Mojzesza, ale z ojców), a w sabat obrzezujecie czlowieka.
\par 23 Poniewaz czlowiek przyjmuje obrzezke w sabat, aby nie byl zgwalcony zakon Mojzeszowy, przecz sie na mie gniewacie, zem calego czlowieka uzdrowil w sabat?
\par 24 Nie sadzcie wedlug widzenia, ale sprawiedliwy sad sadzcie.
\par 25 Mówili tedy niektórzy z Jeruzalemczyków: Izali to nie jest ten, którego szukaja zabic?
\par 26 A oto jawnie mówi, a nic mu nie mówia. Izali prawdziwie poznali ksiazeta, iz ten jest prawdziwie Chrystus?
\par 27 Ale o tym wiemy, skad jest: ale gdy Chrystus przyjdzie, nikt nie bedzie wiedzial, skad by byl.
\par 28 Wolal tedy Jezus w kosciele uczac a mówiac: I mnie znacie, i skadem jest, wiecie; a nie przyszedlem sam od siebie, ale jest prawdziwy, który mie poslal, którego wy nie znacie.
\par 29 Lecz go ja znam; bom od niego jest, a on mie poslal.
\par 30 I szukali, jakoby go pojmac; ale zaden nie sciagnal nan reki; bo jeszcze nie przyszla godzina jego.
\par 31 A wiele ich z ludu uwierzyli wen i mówili: Chrystus gdy przyjdzie, izaz wiecej cudów uczyni nad te, które ten uczynil?
\par 32 A slyszeli Faryzeuszowie, iz to lud o nim szemral; i poslali Faryzeuszowie i przedniejsi kaplani slugi, aby go pojmali.
\par 33 Rzekl im tedy Jezus: Jeszcze na maly czas jestem z wami; potem odejde do tego, który mie poslal.
\par 34 Szukac mie bedziecie, ale nie znajdziecie; a gdzie ja bede, wy przyjsc nie mozecie.
\par 35 Mówili tedy Zydowie miedzy soba: Dokadze ten pójdzie, ze my go nie znajdziemy? czyli do rozproszonych poganów pójdzie i bedzie uczyl pogany?
\par 36 Cóz to za mowa, która wyrzekl: Szukac mie bedziecie, ale nie znajdziecie, i gdzie ja bede, wy przyjsc nie mozecie?
\par 37 A w on ostateczny dzien wielki swieta onego stanal Jezus i wolal mówiac: Jezli kto pragnie, niech do mnie przyjdzie, a pije.
\par 38 Kto wierzy w mie, jako mówi Pismo, rzeki wody zywej poplyna z zywota jego.
\par 39 (A to mówil o Duchu, którego wziac mieli wierzacy wen; albowiem jeszcze nie byl dany Duch Swiety, przeto ze jeszcze Jezus nie byl uwielbiony.)
\par 40 Wiele ich tedy z owego ludu slyszac te slowa, mówili: Tenci jest prawdziwie on prorok.
\par 41 A drudzy mówili: Ten jest Chrystus; ale niektórzy mówili: Azaz z Galilei przyjdzie Chrystus?
\par 42 Azaz nie mówi Pismo, iz z nasienia Dawidowego i z Betlehemu miasteczka, gdzie byl Dawid, przyjdzie Chrystus?
\par 43 A tak stalo sie rozerwanie dla niego miedzy ludem.
\par 44 I chcieli go niektórzy z nich pojmac; ale zaden nie sciagnal nan rak swoich.
\par 45 Przyszli tedy sludzy do przedniejszych kaplanów i do Faryzeuszów; którzy im rzekli: Przeczzescie go nie przywiedli?
\par 46 Odpowiedzieli oni sludzy: Nigdy tak nie mówil czlowiek jako ten czlowiek.
\par 47 I odpowiedzieli im Faryzeuszowie: Alboscie i wy zwiedzeni?
\par 48 Izali kto uwierzyl wen z ksiazat albo z Faryzeuszów?
\par 49 Tylko ten gmin, który nie zna zakonu; przekleci sa.
\par 50 I rzekl do nich Nikodem, który byl przyszedl w nocy do niego, bedac jeden z nich:
\par 51 Izali zakon nasz sadzi czlowieka, jezliby pierwej nie slyszal od niego i nie poznalby, co czyni?
\par 52 A oni mu odpowiedzieli i rzekli: Izalis i ty Galilejczyk? Badajze sie, a obacz, zec prorok z Galilei nie powstal.
\par 53 I poszedl kazdy do domu swego.

\chapter{8}

\par 1 A Jezus poszedl na góre Oliwna.
\par 2 Potem zasie raniuczko przyszedl do kosciola, a lud wszystek zszedl sie do niego; i siadlszy uczyl je.
\par 3 I przywiedli do niego nauczeni w Pismie i Faryzeusze niewiaste na cudzolóstwie zastana, a postawiwszy ja w posrodku,
\par 4 Rzekli mu: Nauczycielu! te niewiaste zastano na samem uczynku cudzolóstwa;
\par 5 A w zakonie nam Mojzesz przykazal takie kamionowac; a ty co mówisz?
\par 6 A to mówili kuszac go, aby go mogli oskarzyc. A Jezus schyliwszy sie na dól, pisal palcem na ziemi.
\par 7 A gdy sie go nie przestawali pytac, podniósl sie, rzekl do nich: Kto z was jest bez grzechu, niech na nia pierwszy kamieniem rzuci.
\par 8 A zasie schyliwszy na dól, pisal na ziemi.
\par 9 A gdy oni uslyszeli, bedac od sumienia przekonani, jeden za drugim wychodzili, poczawszy od starszych az do ostatecznych, iz tylko sam Jezus zostal, a ona niewiasta w posrodku stojaca.
\par 10 A podnióslszy sie Jezus i zadnego nie widzac, tylko one niewiaste, rzekl jej: Niewiasto! gdziez sa oni, co na cie skarzyli? Zaden cie nie potepil?
\par 11 A ona niewiasta rzekla: Zaden, Panie! A Jezus jej rzekl: Ani ja ciebie potepiam; idzze, a juz wiecej nie grzesz.
\par 12 Zasie im rzekl Jezus, mówiac: Jam jest swiatlosc swiata; kto mie nasladuje, nie bedzie chodzil w ciemnosci, ale bedzie mial swiatlosc zywota.
\par 13 I rzekli mu tedy Faryzeuszowie: Ty sam o sobie swiadczysz, a swiadectwo twoje nie jest prawdziwe.
\par 14 Odpowiedzial Jezus i rzekl im: Chociaz ja swiadcze sam o sobie, jednak prawdziwe jest swiadectwo moje; bo wiem, skadem przyszedl i dokad ide; lecz wy nie wiecie, skadem przyszedl i dokad ide.
\par 15 Wy wedlug ciala sadzicie; ale ja nikogo nie sadze.
\par 16 A chocbym i ja sadzil, sad mój jest prawdziwy; bom nie jest sam, ale ja i który mie poslal, Ojciec.
\par 17 W zakonie waszym napisane jest: Iz dwojga ludzi swiadectwo prawdziwe jest.
\par 18 Jam jest, który sam o sobie swiadcze; swiadczy o mnie i ten, który mie poslal, Ojciec.
\par 19 Tedy mu rzekli: Gdziez jest ten twój Ojciec? Odpowiedzial Jezus: Ani mnie znacie, ani Ojca mego; byscie mnie znali, i Ojca byscie mego znali.
\par 20 Te slowa mówil Jezus w skarbnicy, uczac w kosciele, a zaden go nie pojmal; bo jeszcze byla nie przyszla godzina jego.
\par 21 Rzekl im tedy zasie Jezus: Ja ide, i bedziecie mie szukac, a w grzechu waszym pomrzecie; gdzie ja ide, wy przyjsc nie mozecie.
\par 22 Mówili tedy Zydowie: Alboz sie sam zabije, ze mówi: Gdzie ja ide, wy przyjsc nie mozecie?
\par 23 I rzekl do nich: Wyscie z niskosci, a jam z wysokosci; wyscie z tego swiata, a jam zasie nie jest z tego swiata.
\par 24 Przetomci wam powiedzial, iz pomrzecie w grzechach waszych; bo jezli nie wierzycie, zem ja jest, pomrzecie w grzechach waszych.
\par 25 Tedy mu rzekli: Któzes ty jest? I rzekl im Jezus: To, co wam z poczatku powiadam.
\par 26 Wielec mam o was mówic i sadzic; ale ten, który mie poslal, jest prawdziwy, a ja, com od niego slyszal, to mówie na swiecie.
\par 27 Ale nie zrozumieli, ze im o Ojcu mówil.
\par 28 Przetoz im rzekl Jezus: Gdy wywyzszycie Syna czlowieczego, tedy poznacie, zem ja jest, a sam od siebie nic nie uczynie, ale jako mie nauczyl Ojciec mój, tak mówie.
\par 29 A ten, który mie poslal, ze mna jest; nie zostawil mie samego Ojciec; bo co mu sie podoba, to ja zawsze czynie.
\par 30 To gdy on mówil, wiele ich wen uwierzylo.
\par 31 Tedy mówil Jezus do tych Zydów, co mu uwierzyli: Jezli wy zostaniecie w slowie mojem, prawdziwie uczniami moimi bedziecie;
\par 32 Poznacie prawde, a prawda was wyswobodzi.
\par 33 I odpowiedzieli mu: Mysmy nasienie Abrahamowe, a nigdysmy nie sluzyli nikomu; jakoz ty mówisz: Iz wolnymi bedziecie.
\par 34 Odpowiedzial im Jezus: Zaprawde, zaprawde, powiadam wam, iz wszelki, kto czyni grzech, sluga jest grzechu.
\par 35 A slugac nie mieszka w domu na wieki, ale Syn mieszka na wieki.
\par 36 A przetoz jezli was Syn wyswobodzi, prawdziwie wolnymi bedziecie.
\par 37 Wiem, zescie nasienie Abrahamowe; lecz szukacie, abyscie mie zabili, iz mowa moja nie ma u was miejsca.
\par 38 Ja com widzial u Ojca mego, powiadam, a wy tez to, coscie widzieli u ojca waszego, czynicie.
\par 39 Odpowiedzieli mu i rzekli: Ojciec nasz jest Abraham. Rzekl im Jezus: Byscie byli synami Abrahamowymi, czynilibyscie uczynki Abrahamowe.
\par 40 Ale teraz szukacie, byscie mie zabili, czlowieka tego, którym wam prawde mówil, któram slyszal od Boga; tego Abraham nie czynil.
\par 41 Wy czynicie uczynki ojca waszego. Rzekli mu tedy: My z nierzadu nie jestesmy splodzeni, jednegoz Ojca mamy, Boga.
\par 42 Tedy im rzekl Jezus: Byc byl Bóg Ojcem waszym, tedy byscie mie milowali, gdyzem ja od Boga wyszedl i przyszedlem, anim sam od siebie przyszedl, ale mie on poslal.
\par 43 Przeczze tej powiesci mojej nie pojmujecie? przeto, iz nie mozecie sluchac mowy mojej.
\par 44 Wyscie z ojca dyjabla i pozadliwosci ojca waszego czynic chcecie; onci byl mezobójca od poczatku i w prawdzie nie zostal, bo w nim prawdy nie masz: gdy mówi klamstwo, z swego wlasnego mówi, iz jest klamca i ojcem klamstwa.
\par 45 A ja, ze prawde mówie, nie wierzycie mi.
\par 46 Któz mie z was obwini z grzechu? Jezliz prawde mówie, przeczze wy mi nie wierzycie?
\par 47 Ktoc z Boga jest, slów Bozych slucha; dlatego wy nie sluchacie, ze z Boga nie jestescie.
\par 48 Odpowiedzieli tedy Zydowie i rzekli mu: Izali my nie dobrze mówimy, zes ty jest Samarytanin i dyjabelstwo masz?
\par 49 Odpowiedzial Jezus: Jac dyjabelstwa nie mam, ale czcze Ojca mego; a wyscie mie nie uczcili.
\par 50 Jac nie szukam chwaly mojej; jest ten, który szuka i sadzi.
\par 51 Zaprawde, zaprawde powiadam wam: Jezli kto slowa moje zachowywac bedzie, smierci nie oglada na wieki.
\par 52 Tedy mu rzekli Zydowie: Terazesmy poznali, ze dyjabelstwo masz, Abraham umarl i prorocy, a ty powiadasz: Jezli kto slowa moje zachowywac bedzie, smierci nie skosztuje na wieki;
\par 53 Izas ty nie wiekszy nad ojca naszego Abrahama, który umarl? i prorocy pomarli; kimze sie ty wzdy czynisz?
\par 54 Odpowiedzial Jezus: Jezli sie ja sam chwale, chwala moja nic nie jest. Jestci Ojciec mój, który mie chwali, o którym wy powiadacie, ze jest Bogiem waszym.
\par 55 Lecz go nie znacie, a ja go znam; i jezlibym rzekl, ze go nie znam, bylbym podobnym wam, klamca; ale go znam i slowa jego zachowuje.
\par 56 Abraham, ojciec wasz, z radoscia zadal, aby ogladal dzien mój, i ogladal i radowal sie.
\par 57 Tedy rzekli Zydowie do niego: Piecdziesiat lat jeszcze nie masz, a Abrahamas widzial?
\par 58 Rzekl im Jezus: Zaprawde, zaprawde powiadam wam: Pierwej niz Abraham byl, jam jest.
\par 59 Porwali tedy kamienie, aby nan ciskali; lecz Jezus schronil sie, i wyszedl z kosciola, przechodzac przez posrodek ich, i tak uszedl.

\chapter{9}

\par 1 A mimo idac, ujrzal czlowieka slepego od narodzenia.
\par 2 I pytali go uczniowie jego, mówiac: Mistrzu! któz zgrzeszyl, ten czyli rodzice jego, iz sie slepym narodzil?
\par 3 Odpowiedzial Jezus: Ani ten zgrzeszyl, ani rodzice jego; ale zeby sie okazaly sprawy Boze na nim.
\par 4 Jac musze sprawowac sprawy onego, który mie poslal, pokad dzien jest; przychodzi noc, gdy zaden nie bedzie mógl nic sprawowac.
\par 5 Pókim jest na swiecie, jestem swiatloscia swiata.
\par 6 To rzeklszy plunal na ziemie, a uczynil bloto z sliny i pomazal onem blotem oczy slepego,
\par 7 I rzekl mu: Idz, umyj sie w sadzawce Syloe, co sie wyklada poslany. Poszedl tedy i umyl sie, i przyszedl widzac.
\par 8 A tak sasiedzi i którzy go przedtem widywali slepego, mówili: Izali nie ten jest, który siadal i zebral?
\par 9 Drudzy mówili: Iz ten jest; a drudzy, iz jest jemu podobny. Lecz on mówil, zem ja jest.
\par 10 Tedy mu rzekli: Jakoz sa otworzone oczy twoje?
\par 11 A on odpowiedzial i rzekl: Czlowiek, którego zowia Jezusem, uczynil bloto i pomazal oczy moje, a rzekl mi: Idz do sadzawki Syloe, a umyj sie; a tak odszedlszy i umywszy sie, przejrzalem.
\par 12 Tedy mu rzekli: Gdziez on jest? Rzekl: Nie wiem.
\par 13 Tedy przywiedli onego, który przedtem byl slepy, do Faryzeuszów.
\par 14 A byl sabat, gdy Jezus uczynil bloto i otworzyl oczy jego.
\par 15 Tedy go znowu pytali i Faryzeuszowie, jako przejrzal? A on im rzekl: Wlozyl mi blota na oczy, i umylem sie i widze.
\par 16 Tedy niektóry z Faryzeuszów rzekl: Czlowiek ten nie jest z Boga; bo nie strzeze sabatu. Drudzy zasie mówili: Jakoz moze czlowiek grzeszny takowe cuda czynic? I bylo rozerwanie miedzy nimi.
\par 17 Rzekli tedy slepemu po wtóre: Ty co mówisz o nim, poniewaz otworzyl oczy twoje? A on rzekl: Prorok jest.
\par 18 A nie wierzyli Zydowie o nim, zeby byl slepym, a ze przejrzal, az zawolali rodziców onego, który przejrzal.
\par 19 I pytali ich, mówiac: Tenze jest syn wasz, o którym wy powiadacie, iz sie slepo narodzil? jakoz wzdy teraz widzi?
\par 20 Odpowiedzieli im rodzice jego i rzekli: Wiemy, zec to jest syn nasz, i ze sie slepo narodzil;
\par 21 Lecz jako teraz widzi, nie wiemy, albo kto otworzyl oczy jego, my nie wiemy; mac lata, pytajciez go, on sam o sobie powie.
\par 22 Tak mówili rodzice jego, ze sie bali Zydów; albowiem juz byli Zydowie postanowili, aby ktokolwiek by go Chrystusem wyznal, byl z bóznicy wylaczony.
\par 23 Przetoz rzekli rodzice jego: Mac lata, pytajciez go.
\par 24 Tedy zawolali powtóre czlowieka onego, który byl slepy, i rzekli mu: Daj chwale Bogu; myc wiemy, iz ten czlowiek jest grzeszny.
\par 25 A on odpowiedzial i rzekl: Jezli grzeszny jest, nie wiem; to tylko wiem, iz bedac slepym, teraz widze.
\par 26 I rzekli mu znowu: Cóz ci uczynil? Jakoz otworzyl oczy twoje?
\par 27 Odpowiedzial im: Juzemci wam powiedzial, a nie slyszeliscie; przeczze jeszcze slyszec chcecie? Izali i wy chcecie byc uczniami jego?
\par 28 Tedy mu zlorzeczyli i rzekli: Ty badz uczniem jego; alesmy my uczniami Mojzeszowymi.
\par 29 My wiemy, ze Bóg do Mojzesza mówil; lecz ten, skad by byl, nie wiemy.
\par 30 Odpowiedzial on czlowiek i rzekl im: Toc zaprawde rzecz dziwna, ze wy nie wiecie, skad jest, a otworzyl oczy moje.
\par 31 A wiemy, iz Bóg grzeszników nie wysluchiwa; ale jezliby kto chwalca Bozym byl i wole jego czynil, tego wysluchiwa.
\par 32 Od wieku nie slyszano, aby kto otworzyl oczy slepo narodzonego.
\par 33 Byc ten nie byl od Boga, nie móglciby nic uczynic.
\par 34 Odpowiedzieli i rzekli mu: Tys sie wszystek w grzechach narodzil, a ty nas uczysz? I wygnali go precz.
\par 35 A uslyszawszy Jezus, iz go precz wygnali i znalazlszy go, rzekl mu: Wierzyszze ty w Syna Bozego?
\par 36 A on odpowiedzial i rzekl: A któz jest, Panie! abym wen wierzyl?
\par 37 I rzekl mu Jezus: I widziales go, i który mówi z toba, onci jest.
\par 38 A on rzekl: Wierze Panie! i poklonil mu sie.
\par 39 I rzekl mu Jezus: Na sademci ja przyszedl na ten swiat, aby ci, którzy nie widza, widzieli, a ci, którzy widza, aby slepymi byli.
\par 40 I uslyszeli to niektórzy z Faryzeuszów, którzy byli z nim, i rzekli mu: Izali i my slepymi jestesmy?
\par 41 Rzekl im Jezus: Byscie byli slepymi, nie mielibyscie grzechu; lecz teraz mówicie, iz widzimy, przetoz grzech wasz zostaje.

\chapter{10}

\par 1 Zaprawde, zaprawde powiadam wam: Kto nie wchodzi drzwiami do owczarni, ale wchodzi inedy, ten jest zlodziej i zbójca;
\par 2 Lecz kto wchodzi drzwiami, pasterzem jest owiec.
\par 3 Temu odzwierny otwiera i owce sluchaja glosu jego, a on swoich wlasnych owiec z imienia wola i wywodzi je.
\par 4 A gdy wypusci owce swoje, idzie przed niemi, a owce ida za nim; bo znaja glos jego.
\par 5 Ale za cudzym nie ida, lecz uciekaja od niego; bo nie znaja glosu obcych.
\par 6 Te im przypowiesc Jezus powiedzial; lecz oni nie zrozumieli tego, co im mówil.
\par 7 Rzekl im tedy zasie Jezus: Zaprawde, zaprawde powiadam wam, izem ja jest drzwiami owiec.
\par 8 Wszyscy, ile ich przede mna przyszlo, zlodzieje sa i zbójcy; ale ich nie sluchaly owce.
\par 9 Jamci jest drzwiami; jezli kto przez mie wnijdzie, zbawiony bedzie, a wnijdzie i wynijdzie, a pastwisko znajdzie.
\par 10 Zlodziej nie przychodzi, jedno zeby kradl, a zabijal i tracil; jam przyszedl, aby zywot mialy, i obficie mialy.
\par 11 Jam jest on dobry pasterz; dobry pasterz dusze swoje kladzie za owce.
\par 12 Lecz najemnik i ten, który nie jest pasterzem, którego nie sa owce wlasne, widzac wilka przychodzacego, opuszcza owce i ucieka, a wilk porywa i rozprasza owce.
\par 13 A najemnik ucieka, iz jest najemnik i nie ma pieczy o owcach.
\par 14 Jam jest on pasterz dobry i znam moje, a moje mie tez znaja.
\par 15 Jako mie zna Ojciec i ja znam Ojca, i dusze moje klade za owce.
\par 16 A mam i drugie owce, które nie sa z tej owczarni, i tec musze przywiesc; i glosu mego sluchac beda, a bedzie jedna owczarnia i jeden pasterz.
\par 17 Dlatego mie miluje Ojciec, iz ja klade dusze moje, abym ja zasie wzial.
\par 18 Zaden jej nie bierze ode mnie, ale ja klade ja sam od siebie; mam moc polozyc ja i mam moc zasie wziac ja. Toc rozkazanie wzialem od Ojca mego.
\par 19 Tedy sie stalo znowu rozerwanie miedzy Zydami dla tych slów.
\par 20 I mówilo ich wiele z nich: Dyjabelstwo ma i szaleje; czemuz go sluchacie?
\par 21 Drudzy mówili: Te slowa nie sa dyjabelstwo majacego; izali dyjabel moze slepych oczy otwierac?
\par 22 A bylo w Jeruzalemie poswiecanie kosciola, a zima byla.
\par 23 I przechadzal sie Jezus w kosciele, w przysionku Salomonowym.
\par 24 Tedy go obstapili Zydowie i rzekli mu: Dokadze dusze nasze na rzeczy trzymasz? Jezlizes ty jest Chrystus, powiedz nam jawnie.
\par 25 Odpowiedzial im Jezus: Powiedzialem wam, a nie wierzycie; sprawy, które ja czynie w imieniu Ojca mego, te o mnie swiadcza.
\par 26 Ale wy nie wierzycie; bo nie jestescie z owiec moich, jakom wam powiedzial.
\par 27 Owce moje glosu mego sluchaja, a ja je znam i ida za mna;
\par 28 A ja zywot wieczny daje im i nie zgina na wieki, ani ich zaden wydrze z reki mojej.
\par 29 Ojciec mój, który mi je dal, wiekszy jest nad wszystkie, a zaden nie moze ich wydrzec z reki Ojca mego.
\par 30 Ja i Ojciec jedno jestesmy.
\par 31 Porwali tedy znowu kamienie Zydowie, aby go ukamionowali.
\par 32 Odpowiedzial im Jezus: Wiele dobrych uczynków ukazalem wam od Ojca mego, dla któregoz z tych uczynków kamionujecie mie?
\par 33 Odpowiedzieli mu Zydowie, mówiac: Dla dobrego uczynku nie kamionujemy cie, ale dla bluznierstwa, to jest, ze ty bedac czlowiekiem, czynisz sie sam Bogiem.
\par 34 Odpowiedzial im Jezus: Izali nie jest napisano w zakonie waszym: Jam rzekl: Bogowie jestescie?
\par 35 Jezlizec one nazwal bogami, do których sie stalo slowo Boze, a nie moze byc Pismo skazone;
\par 36 A mnie, którego Ojciec poswiecil i poslal na swiat, wy mówicie: Bluznisz, zem rzekl: Jestem Synem Bozym?
\par 37 Jezliz nie czynie spraw Ojca mego, nie wierzciez mi.
\par 38 A jezliz czynie, chociazbyscie mnie nie wierzyli, wierzciez uczynkom, abyscie poznali i wierzyli, zec Ojciec jest we mnie, a ja w nim.
\par 39 Tedy zasie szukali, jakoby go pojmac; ale uszedl z rak ich.
\par 40 I odszedl zasie za Jordan na ono miejsce, gdzie przedtem Jan chrzcil, i tamze mieszkal.
\par 41 A wiele ich do niego przychodzilo i mówili: Janci wprawdzie zadnego cudu nie uczynil; wszakze wszystko, cokolwiek Jan o tym powiedzial, prawdziwe bylo.
\par 42 I wiele ich tam uwierzylo wen.

\chapter{11}

\par 1 A byl niektóry chory Lazarz z Betanii, z miasteczka Maryi i Marty, siostry jej.
\par 2 (A to byla ona Maryja, która pomazala Pana mascia, i ucierala nogi jego wlosami swojemi, której brat Lazarz chorowal.)
\par 3 Poslaly tedy siostry do niego, mówiac: Panie! oto ten, którego milujesz, choruje.
\par 4 A uslyszawszy to Jezus, rzekl: Ta choroba nie jest na smierc, ale dla chwaly Bozej, aby byl uwielbiony Syn Bozy przez nia.
\par 5 A Jezus umilowal Marte i siostre jej, i Lazarza.
\par 6 A gdy uslyszal, iz choruje, tedy zostal przez dwa dni na onemze miejscu, gdzie byl.
\par 7 Lecz potem rzekl do uczniów swoich: Idzmy zasie do Judzkiej ziemi.
\par 8 Rzekli mu uczniowie: Mistrzu! teraz szukali Zydowie, jakoby cie ukamionowali, a zasie tam idziesz?
\par 9 Odpowiedzial Jezus: Azaz nie dwanascie jest godzin dnia? Jezli kto chodzi we dnie, nie obrazi sie; bo widzi swiatlosc tego swiata.
\par 10 A jezli kto chodzi w nocy, obrazi sie; bo w nim swiatla nie masz.
\par 11 To powiedziawszy, potem rzekl do nich: Lazarz, przyjaciel nasz, spi; ale ide, abym go ze snu obudzil.
\par 12 Tedy rzekli uczniowie jego: Panie! jezlize spi, bedzie zdrów.
\par 13 Ale Jezus mówil o smierci jego; lecz oni mniemali, iz o zasnieciu snem mówil.
\par 14 Tedy im rzekl Jezus jawnie: Lazarz umarl.
\par 15 I raduje sie dla was, (abyscie wierzyli), zem tam nie byl; ale pójdziemy do niego.
\par 16 Rzekl zatem Tomasz, którego zwano Dydymus, spóluczniom: Pójdzmy i my, abysmy z nim pomarli.
\par 17 Przyszedlszy tedy Jezus, znalazl go juz cztery dni w grobie lezacego.
\par 18 (A byla Betania blisko Jeruzalemu, jakoby na pietnascie stajan.)
\par 19 A przyszlo bylo wiele Zydów do Marty i Maryi, aby je cieszyli po bracie ich.
\par 20 Marta tedy, gdy uslyszala, ze Jezus idzie, biezala przeciwko niemu; ale Maryja w domu siedziala.
\par 21 I rzekla Marta do Jezusa: Panie! bys tu byl, nie umarlby byl brat mój.
\par 22 Ale i teraz wiem, ze o cokolwiek bys prosil Boga, da ci to Bóg.
\par 23 Rzekl jej Jezus: Wstaniec brat twój.
\par 24 Rzekla mu Marta: Wiem, iz wstanie przy zmartwychwstaniu w on ostateczny dzien.
\par 25 I rzekl jej Jezus: Jam jest zmartwychwstanie i zywot; kto w mie wierzy, chocby tez umarl, zyc bedzie.
\par 26 A wszelki, który zyje, a wierzy w mie, nie umrze na wieki. Wierzyszze temu?
\par 27 Rzekla mu: I owszem Panie! Jam uwierzyla, zes ty jest Chrystus, Syn Bozy, który mial przyjsc na swiat.
\par 28 A to rzeklszy szla i potajemnie zawolala Maryje, siostre swoje, mówiac: jest tu nauczyciel, i wola cie.
\par 29 Ona skoro uslyszala, wnet wstala i szla do niego.
\par 30 (A Jezus jeszcze byl nie przyszedl do miasteczka, ale byl na temze miejscu, gdzie Marta byla wyszla przeciwko niemu.)
\par 31 Zydowie tedy, którzy z nia byli w domu, a cieszyli ja, ujrzawszy Maryje, iz predko wstala i wyszla, szli za nia, mówiac: Idzie do grobu, aby tam plakala.
\par 32 Ale Maryja, gdy tam przyszla, gdzie byl Jezus, ujrzawszy go, przypadla do nóg jego i rzekla: Panie! bys tu byl, nie umarlby byl brat mój.
\par 33 Jezus tedy, gdy ja ujrzal placzaca, i Zydy, którzy byli z nia przyszli, placzace, rozrzewnil sie w duchu i zafrasowal sie,
\par 34 I rzekl: Gdziescie go polozyli? Rzekli mu: Panie! pójdz, a ogladaj.
\par 35 I zaplakal Jezus.
\par 36 Tedy rzekli Zydowie: Wej! jakoc go milowal.
\par 37 A niektórzy z nich mówili: Nie móglze ten, który otworzyl oczy slepego, uczynic, zeby ten byl nie umarl?
\par 38 Ale Jezus zasie rozrzewniwszy sie sam w sobie, przyszedl do grobu; a byla jaskinia, a kamien byl polozony na niej.
\par 39 I rzekl Jezus: Odejmijcie ten kamien. Rzekla mu Marta, siostra onego umarlego: Panie! juzci cuchnie; bo juz cztery dni w grobie.
\par 40 Powiedzial jej Jezus: Azazem ci nie rzekl, iz jezli uwierzysz, ogladasz chwale Boza?
\par 41 Odjeli tedy kamien, gdzie byl umarly polozony. A Jezus podnióslszy oczy swe w góre, rzekl: Ojcze! dziekuje tobie, zes mie wysluchal.
\par 42 A jamci wiedzial, ze mie zawsze wysluchiwasz; alem to rzekl dla ludu wokolo stojacego, aby wierzyli, zes ty mie poslal.
\par 43 A to rzeklszy, zawolal glosem wielkim: Lazarzu! wynijdz sam!
\par 44 I wyszedl ten, który byl umarl, majac zwiazane rece i nogi chustkami, a twarz jego byla chustka obwiazana. Rzekl im Jezus: Rozwiazcie go, a niechaj odejdzie.
\par 45 Wiele tedy z Zydów, którzy byli przyszli do Maryi, a widzieli to, co uczynil Jezus, uwierzylo wen.
\par 46 Niektórzy tez z nich odeszli do Faryzeuszów i powiedzieli im, co uczynil Jezus.
\par 47 Tedy sie zebrali przedniejsi kaplani i Faryzeuszowie w rade, i mówili: Cóz uczynimy? Albowiem ten czlowiek wiele cudów czyni.
\par 48 A jezli go tak zaniechamy, wszyscy wen uwierza, i przyjda Rzymianie, a wezma nam to miejsce nasze i lud.
\par 49 A jeden z nich, Kaifasz, bedac najwyzszym kaplanem onego roku, rzekl im: Wy nic nie wiecie,
\par 50 Ani myslicie, iz nam jest pozyteczno, zeby jeden czlowiek umarl za lud, a zeby wszystek ten naród nie zginal.
\par 51 A tegoc nie mówil sam od siebie, ale bedac najwyzszym kaplanem roku onego, prorokowal, iz Jezus mial umrzec za on naród;
\par 52 A nie tylko za on naród, ale zeby tez syny Boze rozproszone w jedno zgromadzil.
\par 53 Od onego tedy dnia radzili sie spolem, aby go zabili.
\par 54 A Jezus juz nie chodzil jawnie miedzy Zydami, ale stamtad odszedl do krainy, która jest blisko puszczy, do miasta, które zowia Efraim, i tamze mieszkal z uczniami swoimi.
\par 55 A byla blisko wielkanoc zydowska, a wiele ich szlo do Jeruzalemu z onej krainy przed wielkanoca, aby sie oczyscili.
\par 56 I szukali Jezusa, i mówili jedni do drugich, w kosciele stojac: Co sie wam zda, ze nie przyszedl na swieto?
\par 57 A przedniejsi kaplani i Faryzeuszowie wydali byli rozkazanie: Jezliby sie kto dowiedzial, gdzie by byl, zeby oznajmil, aby go pojmali.

\chapter{12}

\par 1 Tedy Jezus szóstego dnia przed wielkanoca przyszedl do Betanii, kedy byl Lazarz, który byl umarl, którego wzbudzil od umarlych.
\par 2 Tamze mu sprawili wieczerze, a Marta poslugiwala, a Lazarz byl jednym z onych, którzy z nim spolem u stolu siedzieli.
\par 3 A Maryja wziawszy funt masci szpikanardowej bardzo drogiej, namascila nogi Jezusowe, i utarla wlosami swojemi nogi jego, i napelniony byl on dom wonnoscia onej masci.
\par 4 Tedy rzekl jeden z uczniów jego, Judasz, syn Szymona, Iszkaryjot, który go mial wydac:
\par 5 Przeczze tej masci nie sprzedano za trzysta groszy, a nie dano ubogim?
\par 6 A to mówil, nie izby mial piecza o ubogich, ale iz byl zlodziejem, i mieszek mial, a cokolwiek wlozono, nosil.
\par 7 Tedy rzekl Jezus: Zaniechaj jej; na dzien pogrzebu mego to chowala.
\par 8 Albowiem ubogie zawsze z soba macie, ale mnie nie zawsze miec bedziecie.
\par 9 Dowiedzial sie tedy lud wielki z Zydów, iz tam byl, i przyszli nie tylko dla Jezusa, ale tez aby Lazarza widzieli, którego byl wzbudzil od umarlych.
\par 10 I radzili sie przedniejsi kaplani, zeby i Lazarza zabili.
\par 11 Bo wiele z Zydów dla niego odstepowali i wierzyli w Jezusa.
\par 12 Nazajutrz wielki lud, który byl przyszedl na swieto, uslyszawszy, iz Jezus idzie do Jeruzalemu,
\par 13 Nabrali galazek palmowych i wyszli naprzeciwko niemu i wolali: Hosanna! blogoslawiony, który idzie w imieniu Panskiem, król Izraelski!
\par 14 A dostawszy Jezus oslecia, wsiadl na nie, jako jest napisane:
\par 15 Nie bój sie, córko Syonska! oto król twój idzie, siedzac na osleciu.
\par 16 Ale tego z przodku nie zrozumieli uczniowie jego, ale gdy byl Jezus uwielbiony, tedy wspomnieli, iz to bylo o nim napisane, a ze mu to uczynili.
\par 17 Swiadczyl tedy lud, który z nim byl, iz Lazarza zawolal z grobu i wzbudzil go od umarlych.
\par 18 Dlatego tez wyszedl przeciwko niemu lud, ze slyszal, iz on ten cud uczynil.
\par 19 Tedy mówili Faryzeuszowie miedzy soba: Widzicie, ze nic nie sprawicie; oto swiat za nim poszedl.
\par 20 A byli niektórzy Grekowie z tych, którzy przychodzili do Jeruzalemu, zeby sie modlili w swieto.
\par 21 Ci tedy przyszli do Filipa, który byl z Betsaidy Galilejskiej, i prosili go, mówiac: Panie, chcemy Jezusa widziec.
\par 22 Przyszedl Filip i powiedzial Andrzejowi, a Andrzej zasie i Filip powiedzieli Jezusowi.
\par 23 A Jezus odpowiedzial im, mówiac: Przyszla godzina, aby byl uwielbiony Syn czlowieczy.
\par 24 Zaprawde, zaprawde powiadam wam: Jezliby ziarno pszeniczne wpadlszy do ziemi, nie obumarlo, ono samo zostaje; lecz jezliby obumarlo, wielki pozytek przynosi.
\par 25 Kto miluje dusze swoje, utraci ja, a kto nienawidzi duszy swojej na tym swiecie, ku wiecznemu zywotowi strzeze jej.
\par 26 Jezli mnie kto sluzy, niechze mie nasladuje, a gdziem ja jest, tam i sluga mój bedzie; a jezli mnie kto sluzyc bedzie, uczci go Ojciec mój.
\par 27 Terazci dusza moja zatrwozona jest; i cóz rzeke? Ojcze! zachowaj mie od tej godziny; alemci dlatego przyszedl na te godzine.
\par 28 Ojcze! uwielbij imie twoje. Przyszedl tedy glos z nieba: Uwielbilem i jeszcze uwielbie.
\par 29 A lud ten, który stal i slyszal, mówil: Zagrzmialo; a drudzy mówili: Aniol do niego mówil.
\par 30 Odpowiedzial Jezus i rzekl: Nie dla mnie sie ten glos stal, ale dla was.
\par 31 Teraz jest sad swiata tego, teraz ksiaze swiata tego precz wyrzucony bedzie.
\par 32 A ja jezli bede podwyzszony od ziemi, pociagne wszystkich do siebie.
\par 33 (A mówil to, oznajmujac, jaka smiercia mial umrzec.)
\par 34 Odpowiedzial mu on lud: Mysmy slyszeli z zakonu, iz Chrystus trwa na wieki; a jakoz ty mówisz, ze musi byc podwyzszony Syn czlowieczy? i któryz to jest Syn czlowieczy?
\par 35 Tedy im rzekl Jezus: Jeszcze do malego czasu jest z wami swiatlosc; chodzciez tedy, póki swiatlosc macie, zeby was ciemnosc nie ogarnela; bo kto w ciemnosci chodzi, nie wie, kedy idzie.
\par 36 Póki swiatlosc macie, wierzcie w swiatlosc, abyscie byli synami swiatlosci. To powiedzial Jezus, a odszedlszy schronil sie od nich.
\par 37 A choc tak wiele cudów uczynil przed nimi, przecie nie uwierzyli wen,
\par 38 Aby sie wypelnilo slowo Izajasza proroka, które powiedzial: Panie! i któz uwierzyl kazaniu naszemu, a ramie Panskie komuz jest objawione?
\par 39 Dlatego uwierzyc nie mogli, iz jeszcze powiedzial Izajasz:
\par 40 Zaslepil oczy ich, i zatwardzil serce ich, aby oczyma nie widzieli i sercem nie zrozumieli, i nie nawrócili sie, abym je uzdrowil.
\par 41 To powiedzial Izajasz, gdy widzial chwale jego, i mówil o nim.
\par 42 Wszakze jednak i z ksiazat wiele ich wen uwierzylo; ale dla Faryzeuszów nie wyznali, aby z bóznicy nie byli wylaczeni.
\par 43 Bo umilowali chwale ludzka wiecej, niz chwale Boza.
\par 44 I wolal Jezus, a mówil: Kto wierzy w mie, nie w mie wierzy, ale w onego, który mie poslal.
\par 45 I kto mie widzi, widzi onego, który mie poslal.
\par 46 Ja swiatlosc przyszedlem na swiat, aby zaden, kto wierzy w mie, w ciemnosciach nie zostal.
\par 47 A jezliby kto sluchal slów moich, a nie uwierzylby, jac go nie sadze; bom nie przyszedl, azebym sadzil swiat, ale azebym zbawil swiat.
\par 48 Kto mna gardzi, a nie przyjmuje slów moich, ma kto by go sadzil; slowa, którem ja mówil, one go osadza w ostateczny dzien.
\par 49 Bom ja z siebie samego nie mówil, ale ten, który mie poslal, Ojciec, on mi rozkazanie dal, co bym mówil i co bym powiadac mial;
\par 50 I wiem, ze rozkazanie jego jest zywot wieczny; przetoz to, co ja wam mówie, jako mi powiedzial Ojciec, tak mówie.

\chapter{13}

\par 1 A przed swietem wielkanocnem wiedzac Jezus, iz przyszla godzina jego, aby przeszedl z tego swiata do Ojca, umilowawszy swoje, którzy byli na swiecie, az do konca umilowal je.
\par 2 A gdy byla wieczerza, a dyjabel juz byl wrzucil w serce Judasza, syna Szymonowego Iszkaryjoty, aby go wydal;
\par 3 Widzac Jezus, iz wszystko Ojciec podal do rak jego, a iz od Boga wyszedl i do Boga idzie,
\par 4 Wstal od wieczerzy i zlozyl szaty, a wziawszy przescieradlo, przepasal sie.
\par 5 Potem nalal wody do miednicy, i poczal nogi umywac uczniom i ucierac przescieradlem, którem byl przepasany.
\par 6 Tedy przyszedl do Szymona Piotra; a on mu rzekl: Panie! i tyz mnie masz nogi umywac?
\par 7 Odpowiedzial Jezus, i rzekl mu: Co ja czynie, ty nie wiesz teraz, ale sie potem dowiesz.
\par 8 Rzekl mu Piotr: Nie bedziesz ty nóg moich umywal na wieki. Odpowiedzial mu Jezus: Jezli cie nie umyje, nie bedziesz mial czastki ze mna.
\par 9 Tedy mu rzekl Szymon Piotr: Panie! nie tylko nogi moje, ale i rece, i glowe.
\par 10 Rzekl mu Jezus: Ktoc jest umyty, nie potrzebuje, jedno aby nogi umyl, bo czysty jest wszystek; i wy jestescie czystymi, ale nie wszyscy;
\par 11 Albowiem wiedzial, który go wydac mial; dlategoz rzekl: Nie wszyscy jestescie czystymi.
\par 12 Gdy tedy umyl nogi ich i wzial szaty swoje, usiadlszy zasie za stól, rzekl im: Wieciez, com wam uczynil?
\par 13 Wy mie nazywacie nauczycielem i Panem, a dobrze mówicie; bomci jest nim.
\par 14 Poniewazem ja tedy umyl nogi wasze, Pan i nauczyciel, i wyscie powinni jedni drugim nogi umywac.
\par 15 Albowiem dalem wam przyklad, abyscie jakom ja wam uczynil, i wy czynili.
\par 16 Zaprawde, zaprawde powiadam wam: Nie jest sluga wiekszy nad pana swego, ani posel jest wiekszy nad onego, który go poslal.
\par 17 Jezlic to wiecie, blogoslawieni jestescie, jezli to uczynicie.
\par 18 Nie o wszystkichci was mówie, jac wiem, którem obral; ale zeby sie wypelnilo Pismo: Który je ze mna chleb, podniósl przeciwko mnie piete swoje.
\par 19 Teraz wam powiadam, przedtem niz sie to stanie, abyscie gdy sie to stanie, uwierzyli, zem ja jest.
\par 20 Zaprawde, zaprawde powiadam wam: Kto przyjmuje tego, którego bym poslal, mie przyjmuje; a kto mie przyjmuje, onego przyjmuje, który mie poslal.
\par 21 To rzeklszy Jezus, zasmucil sie w duchu, i oswiadczyl, a rzekl: Zaprawde, zaprawde powiadam wam, ze jeden z was wyda mie.
\par 22 Tedy uczniowie spogladali po sobie, watpiac, o kim by to mówil.
\par 23 A byl jeden z uczniów jego, który sie byl polozyl na lonie Jezusowem, ten, którego milowal Jezus.
\par 24 Przetoz na tego skinal Szymon Piotr, aby sie wypytal, który by to byl, o którym mówil.
\par 25 A on polozywszy sie na piersiach Jezusowych, rzekl mu: Panie! któryz to jest?
\par 26 Odpowiedzial Jezus: Ten jest, któremu ja omoczywszy sztuczke chleba, podam; a omoczywszy sztuczke chleba, dal Judaszowi, synowi Szymona, Iszkaryjotowi.
\par 27 A zaraz po onej sztuczce chleba wstapil wen szatan. Tedy mu rzekl Jezus: Co czynisz, czyn rychlo.
\par 28 A tego zaden nie zrozumial z spólsiedzacych, na co mu to rzekl.
\par 29 Albowiem niektórzy mniemali, gdyz Judasz mial mieszek, iz mu rzekl Jezus: Nakup, czego nam potrzeba na swieto, albo izby co dal ubogim.
\par 30 Tedy on wziawszy one sztuczke chleba, zarazem wyszedl; a noc byla.
\par 31 A gdy wyszedl, rzekl Jezus: Teraz jest uwielbiony Syn czlowieczy, a Bóg uwielbiony jest w nim.
\par 32 A poniewaz Bóg uwielbiony jest w nim, tedy go tez Bóg uwielbi sam w sobie, i wnetze uwielbi go.
\par 33 Synaczkowie! jeszcze maluczko jestem z wami; bedziecie mie szukac, ale ja jakom rzekl Zydom: Gdzie ja ide, wy przyjsc nie mozecie; tak i wam teraz powiadam.
\par 34 Przykazanie nowe daje wam, abyscie sie spolecznie milowali; jakom i ja was umilowal, abyscie sie i wy spolecznie milowali.
\par 35 Stadci poznaja wszyscy, zescie uczniami moimi, jezli milosc miec bedziecie jedni przeciwko drugim.
\par 36 Rzekl mu Szymon Piotr: Panie! dokadze idziesz? Odpowiedzial mu Jezus: Dokad ja ide, ty teraz za mna isc nie mozesz, ale potem pójdziesz za mna.
\par 37 Tedy mu rzekl Piotr: Panie! czemuz teraz za toba isc nie moge? Dusze moje za cie poloze.
\par 38 Odpowiedzial mu Jezus: Dusze twoje za mie polozysz? Zaprawde, zaprawde powiadam ci: Nie zapieje kur, az sie mnie po trzykroc zaprzesz.

\chapter{14}

\par 1 Niechaj sie nie trwozy serce wasze; wierzycie w Boga i w mie wierzycie.
\par 2 W domu Ojca mego wiele jest mieszkania; a jezli nie, wzdybymci wam powiedzial.
\par 3 Ide, abym wam zgotowal miejsce; a gdy odejde i zgotuje wam miejsce, przyjde zasie i wezme was do siebie, zebyscie, gdziem ja jest, i wy byli.
\par 4 A dokad ja ide, wiecie, i droge wiecie.
\par 5 Rzekl mu Tomasz: Panie! nie wiemy, dokad idziesz, a jakoz mozemy droge wiedziec?
\par 6 Rzekl mu Jezus: Jamci jest ta droga, i prawda, i zywot; zaden nie przychodzi do Ojca, tylko przez mie.
\par 7 Gdybyscie mie znali, i Ojca byscie tez mego znali; i juz go teraz znacie i widzieliscie go.
\par 8 Rzekl mu Filip: Panie! ukaz nam Ojca, a dosyc nam na tem.
\par 9 Rzekl mu Jezus: Przez tak dlugi czas jestem z wami, a nie poznales mie? Filipie! kto mnie widzi, widzi i Ojca mego; jakoz ty mówisz: Ukaz nam Ojca?
\par 10 Nie wierzysz, izem ja w Ojcu, a Ojciec we mnie? Slowa, które ja do was mówie, nie od samego siebie mówie, lecz Ojciec, który we mnie mieszka, on czyni sprawy.
\par 11 Wierzcie mi, zem ja w Ojcu, a Ojciec we mnie; wzdy przynajmniej dla samych spraw wierzcie mi.
\par 12 Zaprawde, zaprawde powiadam wam: Kto wierzy w mie, sprawy które Ja czynie, i on czynic bedzie, i wieksze nad te czynic bedzie; bo ja odchodze do Ojca mego.
\par 13 A o cokolwiek prosic bedziecie w imieniu mojem, to uczynie, aby byl uwielbiony Ojciec w Synu.
\par 14 Jezli o co bedziecie prosic w imieniu mojem, ja uczynie.
\par 15 Jezli mie milujecie, przykazania moje zachowajcie.
\par 16 A ja prosic bede Ojca, a innego pocieszyciela da wam, aby z wami mieszkal na wieki,
\par 17 Onego Ducha prawdy, którego swiat przyjac nie moze; bo go nie widzi, ani go zna; lecz wy go znacie, gdyz u was mieszka i w was bedzie.
\par 18 Nie zostawie was sierotami, przyjde do was.
\par 19 Jeszcze maluczko, a swiat mie juz wiecej nie oglada; lecz wy mie ogladacie; bo ja zyje, i wy zyc bedziecie.
\par 20 W on dzien wy poznacie, zem ja jest w Ojcu moim, a wy we mnie, i ja w was.
\par 21 Kto ma przykazania moje i zachowuje je, ten jest, który mie miluje; a kto mie miluje, bedzie go tez milowal Ojciec mój; i ja go milowac bede, i objawie mu siebie samego.
\par 22 Powiedzial mu Judasz, nie on Iszkaryjot: Panie! cóz jest, ze sie nam objawic masz, a nie swiatu?
\par 23 Odpowiedzial Jezus, i rzekl mu: Jezli mie kto miluje, slowo moje zachowywac bedzie; i Ojciec mój umiluje go, i do niego przyjdziemy, a mieszkanie u niego uczynimy.
\par 24 Kto mie nie miluje, slów moich nie zachowywuje; a slowo, które slyszycie, nie jest moje, ale onego, który mie poslal, Ojca.
\par 25 Tomci wam powiedzial, u was mieszkajac.
\par 26 Lecz pocieszyciel on, Duch Swiety, którego posle Ojciec w imieniu mojem, onci was nauczy wszystkiego, i przypomni wam wszystko, comkolwiek wam powiedzial.
\par 27 Pokój zostawuje wam, pokój on mój daje wam; nie jako daje swiat, ja wam daje; niechze sie nie trwozy serce wasze, ani sie leka.
\par 28 Slyszeliscie, zem ja wam powiedzial: Odchodze, i zas przyjde do was. Gdybyscie mie milowali, wzdybyscie sie radowali, zem rzekl: Ide do Ojca; bo Ojciec mój wiekszy jest niz ja.
\par 29 I terazem wam powiedzial, przedtem niz sie to stanie, zebyscie gdy sie to stanie, uwierzyli.
\par 30 Juz dalej z wami wiele mówic nie bede; albowiem idzie ksiaze swiata tego, a we mnie nic nie ma;
\par 31 Ale izby poznal swiat, ze miluje Ojca, a jako mi rozkazal Ojciec, tak czynie. Wstanciez, pójdzmy stad.

\chapter{15}

\par 1 Jam jest ona winna macica prawdziwa, a Ojciec mój jestci winiarzem.
\par 2 Kazda latorosl, która we mnie owocu nie przynosi, odcina, a kazda, która przynosi owoc, oczyszcza, aby obfitszy owoc przyniosla.
\par 3 Juz wy jestescie czystymi, dla slów, którem do was mówil.
\par 4 Mieszkajciez we mnie, a ja w was; jako latorosl nie moze przynosic owocu sama z siebie, jezli nie bedzie trwala w winnej macicy, takze ani wy, jezli we mnie mieszkac nie bedziecie.
\par 5 Jam jest winna macica, a wyscie latorosle; kto mieszka we mnie, a ja w nim, ten przynosi wiele owocu; bo beze mnie nic uczynic nie mozecie.
\par 6 Jezliby kto nie mieszkal we mnie, precz wyrzucony bedzie jako latorosl, i uschnie; i zbiora je i na ogien wrzuca, i zgoreje.
\par 7 Jezli we mnie mieszkac bedziecie i slowa moje w was mieszkac beda, czegobysciekolwiek chcieli, proscie, a stanie sie wam.
\par 8 W tem bedzie uwielbiony Ojciec mój, kiedy obfity owoc przyniesiecie, a bedziecie moimi uczniami.
\par 9 Jako mie umilowal Ojciec, tak i ja umilowalem was; trwajciez w milosci mojej.
\par 10 Jezli przykazania moje zachowacie, trwac bedziecie w milosci mojej, jakom i ja zachowal przykazania Ojca mego i trwam w milosci jego.
\par 11 Tomci wam powiedzial, aby wesele moje w was trwalo, a wesele wasze bylo zupelne.
\par 12 Toc jest przykazanie moje, abyscie sie spolecznie milowali, jakom i ja was umilowal.
\par 13 Wiekszej milosci nad te zaden nie ma, jedno gdyby kto dusze swoje polozyl za przyjacioly swoje.
\par 14 Wy jestescie przyjaciele moi, jezli czynic bedziecie, cokolwiek ja wam przykazuje.
\par 15 Juzci was dalej nie bede zwal slugami; bo sluga nie wie, co czyni pan jego; leczem was nazwal przyjaciólmi, bo wszystko, comkolwiek slyszal od Ojca mego, oznajmilem wam.
\par 16 Nie wyscie mnie obrali, alem ja was obral; i postanowilem, abyscie wyszli i przyniesli owoc, a owoc wasz aby trwal, i o cokolwiek byscie prosili Ojca w imieniu mojem, zeby wam dal.
\par 17 Toc wam przykazuje, abyscie sie spolecznie milowali.
\par 18 Jezli was swiat nienawidzi, wiedzcie, zec mie pierwej, nizeli was, mial w nienawisci.
\par 19 Byscie byli z swiata, swiat, co jest jego, milowalby; lecz iz nie jestescie z swiata, alem ja was wybral z swiata, przetoz was swiat nienawidzi.
\par 20 Wspomnijcie na slowo, którem ja wam powiedzial: Nie jest sluga wiekszy nad pana swego. Jezlic mie przesladowali, i was przesladowac beda; jezli slowa moje zachowywali, i wasze zachowywac beda.
\par 21 Alec wam to wszystko czynic beda dla imienia mego, iz nie znaja onego, który mie poslal.
\par 22 Bym byl nie przyszedl, a nie mówil im, nie mieliby grzechu; lecz teraz nie maja wymówki z grzechu swego.
\par 23 Kto mnie nienawidzi, i Ojca mego nienawidzi.
\par 24 Bym byl tych uczynków nie czynil miedzy nimi, których zaden inszy nie czynil, grzechu by nie mieli; lecz teraz i widzieli i nienawidzili i mnie, i Ojca mego.
\par 25 Ale izby sie wypelnilo slowo, które jest w zakonie ich napisane: Ze mie darmo mieli w nienawisci.
\par 26 A gdy przyjdzie on pocieszyciel, którego ja wam posle od Ojca, Duch prawdy, który od Ojca przychodzi, on o mnie swiadczyc bedzie.
\par 27 Ale i wy swiadczyc bedziecie; bo ze mna od poczatku jestescie.

\chapter{16}

\par 1 Tomci wam powiedzial, abyscie sie nie gorszyli.
\par 2 Wylaczac was beda z bóznic; owszem przyjdzie godzina, ze wszelki, który was zabije, bedzie mniemal, ze Bogu posluge czyni.
\par 3 A toc wam uczynia, iz nie poznali Ojca ani mnie.
\par 4 Alemci wam to powiedzial, abyscie gdy przyjdzie ta godzina, wspomnieli na to, zem ja wam opowiedzial; a tegom wam z poczatku nie powiadal, bom byl z wami.
\par 5 Lecz teraz ide do onego, który mie poslal, a zaden z was nie pyta mie: Dokad idziesz?
\par 6 Ale zem wam to powiedzial, smutek napelnil serce wasze.
\par 7 Lecz ja wam prawde mówie, wamci to pozyteczno, abym ja odszedl; bo jezli ja nie odejde, pocieszyciel on nie przyjdzie do was, a jezli odejde, posle go do was.
\par 8 A on przyszedlszy, bedzie karal swiat z grzechu i z sprawiedliwosci, i z sadu;
\par 9 Z grzechu mówie, iz nie uwierzyli we mnie;
\par 10 Z sprawiedliwosci zasie, iz do Ojca mego ide, a juz mnie wiecej nie ujrzycie;
\par 11 Z sadu, iz ksiaze tego swiata juz jest osadzony.
\par 12 Mamci wam jeszcze wiele mówic, ale teraz zniesc nie mozecie.
\par 13 Lecz gdy przyjdzie on Duch prawdy, wprowadzi was we wszelka prawde; bo nie sam od siebie mówic bedzie, ale cokolwiek uslyszy, mówic bedzie, i przyszle rzeczy wam opowie.
\par 14 On mie uwielbi; bo z mego wezmie, a opowie wam.
\par 15 Wszystko, co ma Ojciec, moje jest; dlategom rzekl: Ze z mego wezmie, a wam opowie.
\par 16 Maluczko, a nie ujrzycie mie, i zasie maluczko, a ujrzycie mie; bo ja ide do Ojca.
\par 17 Mówili tedy niektórzy z uczniów jego miedzy soba: Cóz to jest, co nam mówi: Maluczko, a nie ujrzycie mie, i zasie maluczko, a ujrzycie mie, a iz ja ide do Ojca?
\par 18 Przetoz mówili: Cóz to jest, co mówi: Maluczko? Nie wiemy, co mówi.
\par 19 Tedy Jezus poznal, ze go pytac chcieli, i rzekl im: O tem sie pytacie miedzy soba, zem rzekl: Maluczko, a nie ujrzycie mie, i zasie maluczko, a ujrzycie mie.
\par 20 Zaprawde, zaprawde powiadam wam: Ze wy bedziecie plakac i narzekac, a swiat sie bedzie weselil; wy smutni bedziecie, ale smutek wasz obróci sie wam w wesele.
\par 21 Niewiasta gdy rodzi, smutek ma, bo przyszla godzina jej; lecz gdy porodzi dzieciatko, juz nie pamieta ucisnienia, dla radosci, iz sie czlowiek na swiat narodzil.
\par 22 I wy teraz smutek macie; ale zasie ujrze was, a bedzie sie radowalo serce wasze, a radosci waszej nikt nie odejmie od was.
\par 23 A dnia onego nie bedziecie mnie o nic pytac. Zaprawde, zaprawde powiadam wam: O cokolwiek byscie prosili Ojca w imieniu mojem, da wam.
\par 24 Dotad o nicescie nie prosili w imieniu mojem; prosciez, a wezmiecie, aby radosc wasza byla doskonala.
\par 25 Tomci wam przez przypowiesc mówil; ale idzie godzina, gdy juz dalej nie przez przypowiesci mówic wam bede, ale jawnie o Ojcu moim oznajmie wam.
\par 26 W on dzien w imieniu mojem prosic bedziecie; a nie mówie wam: Iz ja bede Ojca prosil za wami;
\par 27 Albowiem sam Ojciec miluje was, zescie wy mie umilowali i uwierzyliscie, zem ja od Boga przyszedl.
\par 28 Wyszedlem od Ojca, a przyszedlem na swiat; i zasie opuszczam swiat, a ide do Ojca.
\par 29 Rzekli mu uczniowie jego: Oto teraz jawnie mówisz, a zadnej przypowiesci nie powiadasz;
\par 30 Teraz wiemy, ze wszystko wiesz, a nie potrzebujesz, aby cie kto pytal; przez to wierzymy, zes od Boga wyszedl.
\par 31 Odpowiedzial im Jezus: Teraz wierzycie.
\par 32 Oto przyjdzie godzina; owszem juz przyszla, ze sie rozproszycie kazdy do swego, a mie samego zostawicie; lecz nie jestem sam, bo Ojciec jest ze mna.
\par 33 Tomci wam powiedzial, abyscie we mnie pokój mieli. Na swiecie ucisk miec bedziecie; ale ufajcie, jam zwyciezyl swiat.

\chapter{17}

\par 1 To powiedziawszy Jezus, podniósl oczy swoje w niebo i rzekl: Ojcze! przyszla godzina, uwielbij Syna twego, aby tez i Syn twój uwielbil ciebie.
\par 2 Jakos mu dal moc nad wszelkiem cialem, aby tym wszystkim, któres mu dal, dal zywot wieczny.
\par 3 A toc jest zywot wieczny, aby cie poznali samego prawdziwego Boga, i któregos poslal, Jezusa Chrystusa.
\par 4 Jam cie uwielbil na ziemi, i dokonczylem sprawe, któras mi dal, abym ja czynil.
\par 5 A teraz uwielbij mie ty, Ojcze! u siebie samego ta chwala, któram mial u ciebie, pierwej, nizeli swiat byl.
\par 6 Objawilem imie twoje ludziom, któres mi dal z swiata; toc byli i dales mi je, i zachowali slowa twoje.
\par 7 A teraz poznali, iz wszystko, cos mi dal, od ciebie jest.
\par 8 Albowiem slowa, któres mi dal, dalem im; a oni je przyjeli, i poznali prawdziwie, izem od ciebie wyszedl, a uwierzyli, zes ty mie poslal.
\par 9 Jac za nimi prosze, a nie za swiatem prosze, ale za tymi, któres mi dal; bo twoi sa.
\par 10 I wszystko moje jest twoje, a twoje moje, i uwielbionym jest w nich.
\par 11 A nie jestem wiecej na swiecie, ale oni sa na swiecie, a ja do ciebie ide. Ojcze swiety, zachowaj je w imieniu twojem, któres mi dal, aby byli jedno, jako i my.
\par 12 Gdym z nimi byl na swiecie, jam je zachowal w imieniu twojem, któres mi dal; strzeglem ich i zaden z nich nie zginal, tylko on syn zatracenia, zeby sie Pismo wypelnilo.
\par 13 Ale teraz do ciebie ide i mówie to na swiecie, aby mieli radosc moje doskonala w sobie.
\par 14 Jam im dal slowo twoje, a swiat je mial w nienawisci; bo nie sa z swiata, jako i ja nie jestem z swiata.
\par 15 Nie prosze, abys je wzial z swiata, ale abys je zachowal ode zlego.
\par 16 Nie sac z swiata, jako i ja nie jestem z swiata.
\par 17 Poswiec je w prawdzie twojej; slowo twoje jest prawda.
\par 18 Jakos ty mie poslal na swiat, tak i ja posylam je na swiat.
\par 19 A ja poswiecam samego siebie za nich, aby i oni poswieceni byli w prawdzie.
\par 20 Nie tylko za tymi prosze, lecz i za onymi, którzy przez slowo ich uwierza w mie;
\par 21 Aby wszyscy byli jedno, jako ty, Ojcze! we mnie, a ja w tobie; aby i oni w nas jedno byli, aby swiat uwierzyl, zes ty mie poslal.
\par 22 A ja te chwale, któras mi dal, dalem im, aby byli jedno, jako my jedno jestesmy;
\par 23 Ja w nich, a ty we mnie, aby byli doskonalymi w jedno, a izby poznal swiat, zes ty mie poslal, a izes je umilowal, jakos i mie umilowal.
\par 24 Ojcze! któres mi dal, chce, abym gdziem ja jest, i oni byli ze mna, aby ogladali chwale moje, któras mi dal; albowiemes mie umilowal przed zalozeniem swiata.
\par 25 Ojcze sprawiedliwy! i swiat cie nie poznal; alem ja cie poznal, a i ci poznali, zes ty mie poslal.
\par 26 I uczynilem im znajome imie twoje i znajome uczynie, aby milosc, któras mie umilowal, w nich byla, a ja w nich.

\chapter{18}

\par 1 To powiedziawszy Jezus wyszedl z uczniami swoimi przez potok Cedron, gdzie byl ogród, do którego on wszedl i uczniowie jego.
\par 2 A wiedzial i Judasz, który go wydawal, ono miejsce; bo sie tam czesto schadzal Jezus z uczniami swoimi.
\par 3 Przetoz Judasz wziawszy rote i slugi od przedniejszych kaplanów i Faryzeuszów, przyszedl tam z latarniami i z pochodniami, i z broniami.
\par 4 Tedy Jezus wiedzac wszystko, co nan przyjsc mialo, wyszedlszy rzekl im: Kogo szukacie?
\par 5 Odpowiedzieli mu: Jezusa Nazarenskiego. Rzekl im Jezus: Jam jest. A stal z nimi i Judasz, który go wydawal.
\par 6 A skoro im rzekl: Jam jest, postapili nazad i padli na ziemie.
\par 7 Tedy ich zasie spytal: Kogo szukacie? A oni rzekli: Jezusa Nazarenskiego.
\par 8 Odpowiedzial Jezus: Powiedzialem wam, zem ja jest; jezli tedy mie szukacie, dopuscciez tym odejsc;
\par 9 Aby sie wypelnily slowa, które byl powiedzial: Nie stracilem zadnego z tych, któres mi dal.
\par 10 Tedy Szymon Piotr majac miecz, dobyl go, i uderzyl sluge kaplana najwyzszego, i ucial mu ucho jego prawe; a temu sludze imie bylo Malchus.
\par 11 I rzekl Jezus Piotrowi: Wlóz miecz twój w pochwe; izali nie mam pic kielicha tego, który mi dal Ojciec?
\par 12 Rota tedy i rotmistrz, i sludzy zydowscy pojmali Jezusa i zwiazali go.
\par 13 I wiedli go naprzód do Annasza; bo byl swiekier Kaifaszowy, który byl najwyzszym kaplanem roku onego.
\par 14 A Kaifasz ten byl, który Zydom radzil, ze pozyteczno jest, aby jeden czlowiek umarl za lud.
\par 15 I szedl za Jezusem Szymon Piotr i drugi uczen. A ten uczen byl znajomy najwyzszemu kaplanowi, i wszedl z Jezusem do dworu najwyzszego kaplana.
\par 16 Ale Piotr stal u drzwi na dworze. Wyszedl tedy on drugi uczen, który byl znajomy najwyzszemu kaplanowi, i mówil z odzwierna, i wprowadzil tam Piotra.
\par 17 Tedy rzekla Piotrowi dziewka odzwierna: Izalis i ty nie jest z uczniów tego czlowieka? On odpowiedzial: Nie jestem.
\par 18 Stali tedy sludzy i czeladz, uczyniwszy ogien, bo zimno bylo; i grzali sie; byl tez z nimi Piotr, stojac i grzejac sie.
\par 19 A tak najwyzszy kaplan pytal Jezusa o jego uczniów i o nauke jego.
\par 20 Odpowiedzial mu Jezus: Jam jawnie mówil swiatu; Jam zawsze uczyl w bóznicy i w kosciele, gdzie sie zewszad Zydowie schadzaja, a potajemnie nicem nie mówil.
\par 21 Cóz mie pytasz? Pytaj tych, którzy sluchali, com im mówil; cic to wiedza, com ja mówil.
\par 22 A gdy on to mówil, jeden z slug, który tam stal, wycial policzek Jezusowi, mówiac: I takze (to) odpowiadasz najwyzszemu kaplanowi?
\par 23 Odpowiedzial mu Jezus: Izalim zle rzekl, daj swiadectwo o zlem, a jezlim dobrze, przeczze mie bijesz?
\par 24 I odeslal go Annasz zwiazanego do Kaifasza, najwyzszego kaplana.
\par 25 A Szymon Piotr stal i grzal sie. I rzekli do niego: Azazes i ty nie jest z uczniów jego? A on sie zaprzal, mówiac: Nie jestem.
\par 26 Rzekl mu niektóry z slug kaplana najwyzszego, powinowaty onego, któremu byl Piotr ucial ucho: Izazem ja ciebie nie widzial w ogrodzie z nimi?
\par 27 Zaprzal sie zasie Piotr, a zarazem kur zapial.
\par 28 Prowadzili tedy Jezusa od Kaifasza na ratusz, a bylo rano. I nie weszli sami na ratusz, aby sie nie zmazali, ale izby pozywali baranka wielkanocnego.
\par 29 Tedy wyszedl do nich Pilat, i rzekl: Jakaz skarge przynosicie przeciwko czlowiekowi temu?
\par 30 Odpowiedzieli mu i rzekli: Byc ten nie byl zloczynca, tedybysmy ci go nie podali.
\par 31 I rzekl Pilat: Wezmijciez go wy, a wedlug zakonu waszego osadzcie go. Rzekli mu Zydowie: Nam sie nie godzi zabijac nikogo;
\par 32 Aby sie wypelnily slowa Jezusowe, które rzekl oznajmujac, jaka mial smiercia umrzec.
\par 33 Tedy zasie wszedl Pilat na ratusz i zawolal Jezusa i rzekl mu: Tyzes jest król zydowski?
\par 34 Odpowiedzial mu Jezus: A samze to od siebie mówisz, czylic insi powiedzieli o mnie?
\par 35 Odpowiedzial Pilat: Azazem ja Zyd? Naród twój i przedniejsi kaplani podali mi cie; cózes wzdy uczynil?
\par 36 Odpowiedzial Jezus: Królestwo moje nie jest z tego swiata; gdyby królestwo moje z tego swiata bylo, wzdyc by mie sludzy moi bronili, abym nie byl wydany Zydom; lecz teraz królestwo moje nie jest stad.
\par 37 Tedy mu rzekl Pilat: Tos ty przecie jest królem? Odpowiedzial mu Jezus: Ty powiadasz, zem jest królem. Jam sie na to narodzil i na tom przyszedl na swiat, abym swiadectwo wydal prawdzie; wszelki, który jest z prawdy, slucha glosu mego.
\par 38 Rzekl mu Pilat: Cóz jest prawda? A to rzeklszy, wyszedl zasie do Zydów i rzekl im: Ja w nim zadnej winy nie znajduje.
\par 39 A tez u was jest ten zwyczaj, abym wam jednego wypuscil na wielkanoc; chceciez tedy, abym wam wypuscil tego króla Zydowskiego?
\par 40 Tedy zasie wszyscy zawolali, mówiac: Nie tego, ale Barabbasza! A ten Barabbasz byl zbójca.

\chapter{19}

\par 1 Tedy Pilat wzial Jezusa i ubiczowal go.
\par 2 A zolnierze uplótlszy korone z cierni, wlozyli na glowe jego i plaszczem szarlatowym przyodziali go,
\par 3 A mówili: Badz pozdrowiony, królu zydowski! i dawali mu policzki.
\par 4 I zasie wyszedl Pilat na dwór, i rzekl im: Oto go wam wywiode na dwór, abyscie wiedzieli, iz w nim zadnej winy nie znajduje.
\par 5 Tedy Jezus wyszedl na dwór, niosac one cierniowa korone i on plaszcz szarlatowy; i rzekl im Pilat: Oto czlowiek!
\par 6 A gdy go ujrzeli przedniejsi kaplani i sludzy ich, zawolali mówiac: Ukrzyzuj, ukrzyzuj go! Rzekl im Pilat: Wezmijcie go wy, a ukrzyzujcie, boc ja w nim zadnej winy nie znajduje.
\par 7 Odpowiedzieli mu Zydowie: Myc zakon mamy i wedlug zakonu naszego ma umrzec; bo sie czynil Synem Bozym.
\par 8 A gdy Pilat uslyszal te slowa, bardziej sie ulakl.
\par 9 I wszedl zasie do ratusza i rzekl do Jezusa: Skadzes ty jest? Lecz mu Jezus nie dal odpowiedzi.
\par 10 Tedy mu rzekl Pilat: Nie mówisz ze mna? Nie wiesz, iz mam moc ukrzyzowac cie i mam moc wypuscic cie?
\par 11 Odpowiedzial Jezus: Nie mialbys zadnej mocy nade mna, jezliby ci nie byla dana z góry; przetoz, kto mie tobie wydal, wiekszy grzech ma.
\par 12 Odtad Pilat staral sie o to, jakoby go wypuscil; lecz Zydowie wolali mówiac: Jezli go wypuscisz, nie jestes przyjacielem cesarskim; kazdy bowiem, co sie królem czyni, sprzeciwia sie cesarzowi.
\par 13 A przetoz Pilat uslyszawszy te slowa, wywiódl Jezusa na dwór i siadl na stolicy, na miejscu, które zowia Litostrotos, a po zydowsku Gabbata.
\par 14 A bylo to w dzien przygotowania przed wielkanoca, okolo szóstej godziny, i rzekl Pilat Zydom: Oto król wasz!
\par 15 A oni zawolali: Strac, strac! Ukrzyzuj go! Rzekl im Pilat: Królaz waszego ukrzyzuje? Odpowiedzieli przedniejsi kaplani: Nie mamy króla, tylko cesarza.
\par 16 Tedy im go wydal, zeby byl ukrzyzowany. I wzieli Jezusa i wywiedli.
\par 17 A on niosac krzyz swój, wyszedl na ono miejsce, które zwano trupich glów, a po zydowsku zowia je Golgota;
\par 18 Gdzie go ukrzyzowali, a z nim drugich dwóch z obu stron, a w posrodku Jezusa.
\par 19 Napisal tez Pilat i napis, i postawil nad krzyzem; a bylo napisane: Jezus Nazarenski, król zydowski.
\par 20 A ten napis czytalo wiele Zydów; bo blisko miasta bylo ono miejsce, gdzie byl ukrzyzowany Jezus; a bylo napisane po zydowsku, po grecku i po lacinie.
\par 21 Tedy rzekli Pilatowi przedniejsi kaplani zydowscy: Nie pisz król zydowski; ale iz on powiadal: Jestem królem zydowskim.
\par 22 Odpowiedzial Pilat: Com napisal, tom napisal.
\par 23 A gdy zolnierze Jezusa ukrzyzowali, wzieli szaty jego i uczynili cztery czesci, kazdemu zolnierzowi czesc, i suknia; a byla ta suknia nie szyta, ale od wierzchu wszystka dziana.
\par 24 Tedy rzekli jedni do drugich: Nie krajmy jej, ale o nie rzucmy losy, czyja ma byc; aby sie Pismo wypelnilo, które mówi: Podzielili miedzy sie szaty moje, a o odzienie moje los miotali. To tedy uczynili zolnierze.
\par 25 A staly podle krzyza Jezusowego matka jego i siostra matki jego, Maryja, zona Kleofaszowa, i Maryja Magdalena.
\par 26 Tedy Jezus ujrzawszy matke i ucznia, którego milowal, tuz stojacego, rzekl matce swojej: Niewiasto, oto syn twój!
\par 27 Potem rzekl uczniowi: Oto matka twoja! a od onej godziny wzial ja on uczen do siebie.
\par 28 Potem widzac Jezus, iz sie juz wszystko wykonalo, aby sie wypelnilo Pismo, rzekl: Pragne.
\par 29 A bylo tam naczynie postawione octu pelne; tedy oni napelniwszy gabke octem, a oblozywszy (ja) hizopem podali do ust jego.
\par 30 A gdy Jezus skosztowal octu, rzekl: Wykonalo sie; a nachyliwszy glowe, oddal ducha.
\par 31 Tedy Zydowie, aby ciala na krzyzu na sabat nie zostaly, poniewaz byl dzien przygotowania, (albowiem byl wielki on dzien sabatu,)prosili Pilata, aby im golenie polamano, i zdjeto je.
\par 32 Przyszli tedy zolnierze, a pierwszemu wprawdzie zlamali golenie i drugiemu, który z nim byl ukrzyzowany.
\par 33 Ale do Jezusa przyszedlszy, gdy ujrzeli, ze juz umarl, nie lamali goleni jego.
\par 34 Lecz jeden z zolnierzy wlócznia otworzyl bok jego, a zarazem wyszla krew i woda.
\par 35 A ten, co to widzial, swiadczyl o tem i prawdziwe jest swiadectwo jego; a on wie, iz prawde powiada, abyscie wy wierzyli.
\par 36 Albowiem sie to stalo, aby sie wypelnilo Pismo: Kosc jego nie bedzie zlamana.
\par 37 I zasie drugie Pismo mówi: Ujrza, kogo przebodli.
\par 38 A potem prosil Pilata Józef z Arymatyi, (który byl uczniem Jezusowym, ale tajemnym dla bojazni zydowskiej), aby zdjal cialo Jezusowe. I pozwolil Pilat. Szedl tedy i zdjal cialo Jezusowe.
\par 39 Przyszedl tez i Nikodem, (który byl przedtem przyszedl w nocy do Jezusa), niosac zmieszanej myrry i aloes, okolo sta funtów.
\par 40 Wzieli tedy cialo Jezusowe i uwineli je w przescieradla z onemi rzeczami wonnemi, jako jest zwyczaj Zydom umarle chowac.
\par 41 A byl na onem miejscu, gdzie byl ukrzyzowany, ogród, a w ogrodzie grób nowy, w którym jeszcze nikt nie byl polozony.
\par 42 Przetoz tam dla dnia przygotowania zydowskiego, iz on grób byl blisko, polozyli Jezusa.

\chapter{20}

\par 1 A pierwszego dnia po sabacie Maryja Magdalena przyszla rano do grobu, gdy jeszcze bylo ciemno, i ujrzala kamien odwalony od grobu.
\par 2 I biezala, a przyszla do Szymona Piotra i do onego drugiego ucznia, którego milowal Jezus, i rzekla im: Wzieli Pana z grobu, a nie wiemy, gdzie go polozyli.
\par 3 Tedy wyszedl Piotr i on drugi uczen, a szli do grobu.
\par 4 I biezeli obaj spolem; ale on drugi uczen wyscignal Piotra i pierwej przyszedl do grobu.
\par 5 A nachyliwszy sie, ujrzal lezace przescieradla; wszakze tam nie wszedl.
\par 6 Przyszedl tez i Szymon Piotr, idac za nim, i wszedl w grób, i ujrzal przescieradla lezace,
\par 7 I chustke, która byla na glowie jego, nie z przescieradlami polozona, ale z osobna na jednem miejscu zwinieta.
\par 8 Potem wszedl i on drugi uczen, który byl pierwej przyszedl do grobu, i ujrzal, a uwierzyl.
\par 9 Albowiem jeszcze nie rozumieli Pisma, iz mial zmartwychwstac.
\par 10 I odeszli zas oni uczniowie do domu.
\par 11 Ale Maryja stala u grobu, na dworze placzac; a gdy plakala, nachylila sie w grób.
\par 12 I ujrzala dwóch Aniolów w bieli siedzacych, jednego u glowy, a drugiego u nóg, tam gdzie bylo polozone cialo Jezusowe.
\par 13 Którzy jej rzekli: Niewiasto! czemu placzesz? Rzekla im: Iz wzieli Pana mego, a nie wiem, gdzie go polozyli.
\par 14 A to rzeklszy, obrócila sie nazad i ujrzala Jezusa stojacego; lecz nie wiedziala, iz Jezus byl.
\par 15 Rzekl jej Jezus: Niewiasto! czemu placzesz? kogo szukasz? A ona mniemajac, ze byl ogrodnik, rzekla mu: Panie! jezlis go ty wzial, powiedz mi, gdzies go polozyl, a ja go wezme.
\par 16 Rzekl jej Jezus: Maryjo! Która obróciwszy sie, rzekla mu: Rabbuni! co sie wyklada: Nauczycielu!
\par 17 Rzekl jej Jezus: Nie dotykaj sie mnie, bom jeszcze nie wstapil do Ojca mego; ale idz do braci moich, a powiedz im: Wstepuje do Ojca mego i Ojca waszego, i do Boga mego i Boga waszego.
\par 18 Tedy przyszla Maryja Magdalena, oznajmujac uczniom, ze widziala Pana, a ze jej to powiedzial.
\par 19 A gdy byl wieczór dnia onego pierwszego po sabacie, a drzwi byly zamkniete, gdzie byli uczniowie zgromadzeni dla bojazni zydowskiej, przyszedl Jezus i stanal w posrodku nich, i rzekl im: Pokój wam!
\par 20 A to rzeklszy pokazal im rece i bok swój; a uradowali sie uczniowie, ujrzawszy Pana.
\par 21 Rzekl im zasie Jezus: Pokój wam; jako mie poslal Ojciec, tak i ja was posylam.
\par 22 A to rzeklszy tchnal na nie i rzekl im: Wezmijcie Ducha Swietego.
\par 23 Którymkolwiek grzechy odpuscicie, sa im odpuszczone, a którymkolwiek zatrzymacie, sa zatrzymane.
\par 24 A Tomasz, jeden ze dwunastu, którego zowia Dydymus, nie byl z nimi, gdy byl przyszedl Jezus.
\par 25 I rzekli mu drudzy uczniowie: Widzielismy Pana. Ale im on rzekl: Jezli nie ujrze w reku jego znaków gwozdzi, a nie wloze palca mego w znaki gwozdzi, a nie wloze reki mojej w bok jego, nie uwierze.
\par 26 A po osmiu dniach byli zasie uczniowie jego w domu, i Tomasz z nimi. I przyszedl Jezus, gdy byly drzwi zamkniete, a stanal w posrodku nich, i rzekl: Pokój wam!
\par 27 Potem rzekl Tomaszowi: Wlóz sam palec twój, a ogladaj rece moje i sciagnij reke twoje, i wlóz ja w bok mój, a nie badz niewiernym, ale wiernym.
\par 28 Tedy odpowiedzial Tomasz i rzekl mu: Panie mój, i Boze mój!
\par 29 Rzekl mu Jezus: Zes mie ujrzal, Tomaszu, uwierzyles; blogoslawieni którzy nie widzieli, a uwierzyli.
\par 30 Wielec i innych cudów uczynil Jezus przed oczyma uczniów swoich, które nie sa napisane w tych ksiegach.
\par 31 Ale te sa napisane, abyscie wy wierzyli, ze Jezus jest Chrystus, Syn Bozy, a zebyscie wierzac zywot mieli w imieniu jego.

\chapter{21}

\par 1 Potem sie zas ukazal Jezus uczniom u morza Tyberyjadzkiego, a ukazal sie tak.
\par 2 Byli pospolu Szymon Piotr i Tomasz, którego zowia Dydymus, i Natanael, który byl z Kany Galilejskiej, i synowie Zebedeuszowi, i drudzy dwaj z uczniów jego.
\par 3 Rzekl im Szymon Piotr: Pójde ryby lowic. Mówia mu: Pójdziemy i my z toba. I szli, i wnet wstapili w lódz, a onej nocy nic nie pojmali.
\par 4 A gdy juz bylo rano, stanal Jezus na brzegu; wszakze nie wiedzieli uczniowie, zeby byl Jezus.
\par 5 Rzekl im tedy Jezus: Dzieci! a maciez co jesc? Odpowiedzieli mu: Nie mamy.
\par 6 A on im rzekl: Zapusccie siec po prawej stronie lodzi, a znajdziecie. I zapuscili, a juz dalej nie mogli jej ciagnac przed mnóstwem ryb.
\par 7 I rzekl on uczen, którego milowal Jezus Piotrowi: Pan jest. Szymon tedy Piotr, uslyszawszy iz Pan jest, przepasal sie koszula, (albowiem byl nagi) i rzucil sie w morze.
\par 8 A drudzy zasie uczniowie przybyli w lodzi; (bo niedaleko bylo od brzegu, ale jakoby na dwiescie lokci) ciagnac siec z rybami.
\par 9 A gdy wstapili na brzeg, ujrzeli wegle nalozone, i rybe na nich lezaca i chleb.
\par 10 Rzekl im Jezus: Przyniescie z tych ryb, którescie teraz pojmali.
\par 11 Wstapil tedy Szymon Piotr i wyciagnal siec na ziemie, pelna wielkich ryb, których bylo sto piecdziesiat i trzy; a choc ich tak wiele bylo, nie zdarla sie siec.
\par 12 Rzekl im Jezus: Pójdzcie, obiadujcie. I zaden z uczniów nie smial go pytac: Ty ktos jest? wiedzac, ze jest Pan.
\par 13 Tedy przyszedl Jezus i wzial on chleb, i dal im, takze i rybe.
\par 14 A toc juz trzeci raz ukazal sie Jezus uczniom swoim po zmartwychwstaniu.
\par 15 A gdy obiad odprawili, rzekl Jezus Szymonowi Piotrowi: Szymonie Jonaszowy, milujesz mie wiecej nizeli ci? Rzekl mu: Tak jest, Panie! ty wiesz, ze cie miluje. Rzekl mu: Pasze baranki moje.
\par 16 Rzekl mu zasie po wtóre: Szymonie Jonaszowy! milujesz mie? Rzekl mu: Tak jest, Panie! ty wiesz, ze cie miluje. Rzekl mu: Pasze owce moje.
\par 17 Rzekl mu po trzecie: Szymonie Jonaszowy! milujesz mie? I zasmucil sie Piotr, ze mu po trzecie rzekl: Milujesz mie? I odpowiedzial mu: Panie! ty wszystko wiesz, ty znasz, ze cie miluje. Rzekl mu Jezus: Pasze owce moje.
\par 18 Zaprawde, zaprawde powiadam tobie: Gdys byl mlodszym, opasywales sie i chodziles, kedys chcial; lecz gdy sie zestarzejesz, wyciagniesz rece twoje, a inny cie opasze i poprowadzi, gdzie bys nie chcial.
\par 19 A to powiedzial, dajac znac, jaka smiercia mial uwielbic Boga. A to powiedziawszy, rzekl mu: Pójdz za mna.
\par 20 A Piotr obróciwszy sie, ujrzal onego ucznia, którego milowal Jezus, pozad idacego, który sie tez byl polozyl przy wieczerzy na piersiach jego, i rzekl byl: Panie! któryz jest ten, co cie wyda?
\par 21 Tego ujrzawszy Piotr, rzekl Jezusowi: Panie! a ten co?
\par 22 Rzekl mu Jezus: Jezlibym chcial, zeby on zostal, az przyjde, co tobie do tego? Ty pójdz za mna.
\par 23 I wyszla ta powiesc miedzy braci, zeby on uczen umrzec nie mial. Lecz mu nie rzekl Jezus, iz nie mial umrzec; ale: Jezli chce, aby zostal az przyjde, cóz tobie do tego?
\par 24 Tenci jest on uczen, który swiadczy o tem, i to napisal; a wiemy, ze prawdziwe jest swiadectwo jego.
\par 25 Jest tez jeszcze i innych wiele rzeczy, które czynil Jezus; które gdyby mialy byc wszystkie z osobna spisane, tusze, iz i sam swiat nie móglby ogarnac ksiag, które by napisane byly. Amen.


\end{document}