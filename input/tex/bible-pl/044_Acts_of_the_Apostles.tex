\begin{document}

\title{Dzieje Apostolskie}


\chapter{1}

\par 1 Pierwsze wprawdzie ksiegi napisalem, o Teofilu! o wszystkiem, co poczal Jezus i czynic, i uczyc.
\par 2 Az do dnia onego, którego dawszy rozkazanie Apostolom, które byl przez Ducha Swietego obral, wziety jest w góre.
\par 3 Którym tez samego siebie po mece swojej zywym stawil w wielu niewatpliwych dowodach, przez czterdziesci dni ukazujac sie im i mówiac o królestwie Bozem.
\par 4 A zgromadziwszy je przykazal im, aby nie odchodzili z Jeruzalemu, ale izby czekali obietnicy ojcowskiej, o którejscie mówic slyszeli ode mnie.
\par 5 Albowiemci Jan chrzcil woda; ale wy bedziecie ochrzczeni Duchem Swietym po niewielu tych dniach.
\par 6 A tak oni zszedlszy sie, pytali go, mówiac: Panie! izali w tym czasie naprawisz królestwo Izraelskie?
\par 7 Lecz on rzekl do nich: Nie wasza rzecz jest, znac czasy i chwile, które Ojciec w swojej mocy polozyl.
\par 8 Ale przyjmiecie moc Ducha Swietego, który przyjdzie na was; i bedziecie mi swiadkami i w Jeruzalemie, i we wszystkiej Judzkiej ziemi, i w Samaryi, az do ostatniego kraju ziemi.
\par 9 A to rzeklszy, gdy oni patrzali, w góre podniesiony jest, a oblok wzial go od oczów ich.
\par 10 A gdy za nim do nieba idacym pilnie patrzali, oto dwaj mezowie staneli przy nich w bialem odzieniu,
\par 11 I rzekli: Mezowie Galilejscy! przecz stoicie, patrzac w niebo? Ten Jezus, który w góre wziety jest od was do nieba, tak przyjdzie, jakoscie go widzieli idacego do nieba.
\par 12 Tedy sie wrócili do Jeruzalemu od góry, która zowia oliwna, która jest blisko Jeruzalemu, majac drogi przez jeden sabat.
\par 13 A gdy weszli, wstapili na sale, gdzie mieszkali Piotr, i Jakób, i Jan, i Andrzej, i Filip, i Tomasz, Bartlomiej, i Mateusz, Jakób Alfeuszowy, i Szymon Zelotes, i Judas Jakóbowy.
\par 14 Ci wszyscy trwali jednomyslnie na modlitwie i prosbach, z zonami i z Maryja, matka Jezusowa, i z bracmi jego.
\par 15 A w onez dni, powstawszy Piotr w posrodku uczniów, rzekl: (A byl poczet osób wespól zgromadzonych okolo sta i dwudziestu):
\par 16 Mezowie bracia! musialo sie wypelnic ono pismo, które opowiedzial Duch Swiety przez usta Dawidowe o Judaszu, który byl wodzem tych, co pojmali Jezusa;
\par 17 Bo byl policzony z nami i dostal byl czastki tego uslugiwania.
\par 18 Tenci wprawdzie otrzymal role z zaplaty niesprawiedliwosci, a powiesiwszy sie, rozpukl sie na poly i wyplynely wszystkie wnetrznosci jego.
\par 19 I bylo to jawne wszystkim mieszkajacym w Jeruzalemie, tak iz nazwano one role wlasnym ich jezykiem Akieldama, to jest rola krwi.
\par 20 Albowiem napisano w ksiegach Psalmów: Niechaj bedzie mieszkanie jego puste, a niech nie bedzie, kto by w niem mieszkal, a biskupstwo jego niech wezmie inny.
\par 21 Potrzeba tedy, aby jeden z tych mezów, którzy z nami bywali po wszystek czas, który Pan Jezus przebywal miedzy nami,
\par 22 Poczawszy od chrztu Janowego, az do tego dnia, którego jest wziety w góre od nas, byl z nami swiadkiem zmartwychwstania jego.
\par 23 I postawili dwóch: Józefa, którego zwano Barsabaszem, którego tez nazywano Justem, i Macieja.
\par 24 A modlac sie mówili: Ty Panie! który znasz serca wszystkich, okaz z tych dwóch jednego, któregos obral;
\par 25 Aby przyjal czastke uslugiwania tego i apostolstwa, z którego wypadl Judasz, aby odszedl na miejsce swoje.
\par 26 I rzucili losy ich. I padl los na Macieja; a przylaczony jest spólnem zdaniem do jedenastu Apostolów.

\chapter{2}

\par 1 A gdy przyszedl dzien piecdziesiaty, byli wszyscy jednomyslnie pospolu.
\par 2 Tedy sie stal z predka z nieba szum, jakoby przypadajacego wiatru gwaltownego i napelnil wszystek dom, kedy siedzieli.
\par 3 I ukazaly sie im rozdzielone jezyki na ksztalt ognia, który usiadl na kazdym z nich.
\par 4 I napelnieni sa wszyscy Duchem Swietym, a poczeli mówic innemi jezykami, jako im Duch on dawal wymawiac.
\par 5 A byli w Jeruzalemie mieszkajacy Zydowie, mezowie nabozni, z kazdego narodu tych, którzy sa pod niebem.
\par 6 A gdy sie stal ten glos, zeszlo sie mnóstwo ludzi i strwozyli sie, ze je slyszal kazdy z nich mówiace wlasnym jezykiem swoim.
\par 7 I zdumiewali sie wszyscy, i dziwowali sie, mówiac jedni do drugich: Izali oto ci wszyscy, którzy mówia, nie sa Galilejczycy?
\par 8 A jakoz my od nich slyszymy kazdy z nas swój wlasny jezyk, w którymesmy sie urodzili?
\par 9 Partowie i Medowie, i Elamitowie, i którzy mieszkamy w Mezopotamii, w Judzkiej ziemi, i w Kapadocyi, w Poncie, i w Azyi;
\par 10 W Frygii, i w Pamfilii, w Egipcie, i w stronach Libii, która jest podle Cyreny, i przychodniowie Rzymscy; Zydowie, i nowonawróceni;
\par 11 Kretenczycy, i Arabczycy; slyszymy ich, mówiacych jezykami naszemi wielkie sprawy Boze.
\par 12 I zdumiewali sie wszyscy, i dziwowali sie, mówiac jeden do drugiego: Cóz to wzdy ma byc?
\par 13 Lecz drudzy nasmiewajac sie, mówili: Ci sie mlodem winem popili.
\par 14 A stanawszy Piotr z jedenastoma, podniósl glos swój i przemówil do nich: Mezowie Judzcy i wszyscy, którzy mieszkacie w Jeruzalemie! niech wam to jawno bedzie, a przyjmijcie w uszy slowa moje.
\par 15 Albowiem nie sa ci, jako wy mniemacie, pijani, gdyz dopiero jest trzecia na dzien godzina.
\par 16 Alec to jest ono, co przepowiedziano przez proroka Joela:
\par 17 I bedzie w ostateczne dni, (mówi Bóg): Wyleje z Ducha mego na wszelkie cialo, a prorokowac beda synowie wasi i córki wasze, a mlodziency wasi widzenia widziec beda, a starcom waszym sny sie snic beda.
\par 18 Nawet w onez dni na slugi moje i na sluzebnice moje wyleje z Ducha mego, i beda prorokowac;
\par 19 I ukaze cuda na niebie w górze i znamiona na ziemi nisko, krew, i ogien, i pare dymu.
\par 20 Slonce sie obróci w ciemnosc, a ksiezyc w krew, przedtem niz przyjdzie on dzien Panski wielki i znaczny.
\par 21 I stanie sie, ze ktobykolwiek wzywal imienia Panskiego, zbawion bedzie.
\par 22 Mezowie Izraelscy! sluchajcie slów tych Jezusa, onego Nazarenskiego, meza od Boga wslawionego u was mocami i cudami, i znamionami, które czynil Bóg przez niego w posrodku was, jako i wy sami wiecie;
\par 23 Tego za ulozona rada i przejrzeniem Bozem wydanego wziawszy, a przez rece niezbozników ukrzyzowawszy, zabiliscie.
\par 24 Którego Bóg wzbudzil, rozwiazawszy bolesci smierci, jakoz bylo to niepodobne, aby od niej mial byc zatrzymany.
\par 25 Albowiem o nim mówi Dawid: Upatrywalem zawsze Pana przed obliczem mojem; bo mi jest po prawicy, abym nie byl wzruszony.
\par 26 Przetoz rozweselilo sie serce moje i rozradowal sie jezyk mój, nadto i cialo moje odpocznie w nadziei;
\par 27 Albowiem nie zostawisz duszy mojej w piekle, a nie dasz swietemu twojemu ogladac skazenia.
\par 28 Oznajmiles mi drogi zywota, a napelnisz mie radoscia przed obliczem twojem.
\par 29 Mezowie bracia! moge bezpiecznie mówic do was o patryjarsze Dawidzie, zec umarl i pogrzebiony jest, a grób jego jest u nas az do dnia dzisiejszego.
\par 30 Bedac tedy prorokiem i wiedzac, ze mu sie Bóg obowiazal przysiega, iz z owocu biódr jego wedlug ciala mial wzbudzic Chrystusa, a posadzic na stolicy jego.
\par 31 To przegladajac, powiedzial o zmartwychwstaniu Chrystusowem, iz nie zostala dusza jego w piekle, ani cialo jego widzialo skazenia.
\par 32 Tegoc Jezusa wzbudzil Bóg, czego my wszyscy jestesmy swiadkami.
\par 33 Prawica tedy Boza bedac wywyzszony, a obietnice Ducha Swietego wziawszy od Ojca, wylal to, co wy teraz widzicie i slyszycie.
\par 34 Albowiemci Dawid nie wstapil do nieba, lecz sam powiada: Rzekl Pan Panu memu, siadz po prawicy mojej,
\par 35 Az poloze nieprzyjacioly twoje podnózkiem nóg twoich.
\par 36 Niechajze tedy wie zapewne wszystek dom Izraelski, ze go Bóg i Panem, i Chrystusem uczynil, tego Jezusa, któregoscie wy ukrzyzowali.
\par 37 A to slyszac, przerazeni sa na sercu i rzekli do Piotra i do innych Apostolów: Cóz mamy czynic, mezowie bracia?
\par 38 Tedy Piotr rzekl do nich: Pokutujcie, a ochrzcij sie kazdy z was w imieniu Jezusa Chrystusa na odpuszczenie grzechów, a wezmiecie dar Ducha Swietego.
\par 39 Albowiemci wam ta obietnica nalezy i dziatkom waszym, i wszystkim, którzy daleko sa, którekolwiek by powolal Pan, Bóg nasz.
\par 40 I wiela inszych slów oswiadczal sie i napominal je, mówiac: Wyzwólcie sie od tego rodzaju przewrotnego.
\par 41 Którzy tedy wdziecznie przyjeli slowa jego, ochrzczeni sa i przystalo dnia onego dusz okolo trzech tysiecy.
\par 42 I trwali w nauce apostolskiej i w spolecznosci, i w lamaniu chleba, i w modlitwach.
\par 43 I przyszedl strach na kazda dusze, a wiele sie znamion i cudów przez Apostolów dzialo.
\par 44 A wszyscy, którzy uwierzyli byli pospolu, i wszystkie rzeczy mieli spólne.
\par 45 A osiadlosci i majetnosci sprzedawali, i udzielali ich wszystkim, jako komu bylo potrzeba.
\par 46 A na kazdy dzien trwajac zgodnie w kosciele i chleb lamiac po domach, przyjmowali pokarm z radoscia i w prostocie serdecznej.
\par 47 Chwalac Boga i majac laske u wszystkiego ludu. A Pan przydawal zborowi na kazdy dzien tych, którzy mieli byc zbawieni.

\chapter{3}

\par 1 A Piotr i Jan spolem wstepowali do kosciola w godzine modlitwy, dziewiata.
\par 2 A maz niektóry bedac chromy, zaraz z zywota matki swojej byl noszony, którego na kazdy dzien sadzano u drzwi koscielnych, które zwano piekne, aby prosil jalmuzny od tych, którzy wchodzili do kosciola.
\par 3 Ten ujrzawszy Piotra i Jana, ze mieli wnijsc do kosciola, prosil ich o jalmuzne.
\par 4 A Piotr z Janem pilnie na niego patrzac, rzekli: Wejrzyj na nas!
\par 5 Tedy on z pilnoscia patrzal na nie, spodziewajac sie co wziac od nich.
\par 6 I rzekl Piotr: Srebra i zlota nie mam; lecz co mam, to ci daje: W imieniu Jezusa Chrystusa Nazarenskiego wstan, a chodz.
\par 7 A ujawszy go za prawa reke jego, podniósl go, a zarazem utwierdzone byly nogi jego i kostki.
\par 8 I wyskoczywszy, stanal i chodzil, a wszedl z nimi do kosciola, chodzac i skaczac, a chwalac Boga.
\par 9 A widzial go wszystek lud chodzacego i chwalacego Boga.
\par 10 I poznali go, iz to on byl, który dla jalmuzny siadal u drzwi pieknych koscielnych; i napelnieni sa strachu i zdumienia nad tem, co mu sie stalo.
\par 11 A gdy sie trzymal on chromy, który byl uzdrowiony, Piotra i Jana, zbiezal sie do nich wszystek lud do przysionka, który zwano Salomonowym, zdumiawszy sie.
\par 12 Co widziac Piotr, przemówil do ludu: Mezowie Izraelscy! cóz sie temu dziwujecie, albo czemu sie nam tak pilnie przypatrujecie, jakobysmy to wlasna moca albo poboznoscia uczynili, aby ten chodzil?
\par 13 Bóg Abrahama i Izaaka i Jakóba, Bóg ojców naszych, uwielbil Jezusa, Syna swego, któregoscie wy wydali i zaparliscie sie go przed twarza Pilatowa, który go sadzil byc godnym wypuszczenia.
\par 14 A wyscie sie onego swietego i sprawiedliwego zaparli, a prosiliscie o mezobójce, aby wam byl darowany.
\par 15 I zabiliscie dawce zywota, którego Bóg wzbudzil od umarlych, czego my swiadkami jestesmy.
\par 16 A przez wiare w imie jego, tego, którego wy widzicie i znacie, utwierdzilo imie jego; wiara, mówie, która przez niego jest, dala temu to zupelne zdrowie przed obliczem was wszystkich.
\par 17 Ale teraz, bracia! wiem, zescie to z niewiadomosci uczynili, jako i ksiazeta wasi.
\par 18 Lecz Bóg, co przez usta wszystkich proroków swoich przepowiedzial, iz Chrystus jego cierpiec mial, to tak ziscil.
\par 19 Przetoz pokutujcie, a nawróccie sie, aby byly zgladzone grzechy wasze.
\par 20 Gdyby przyszly czasy ochlody od oblicznosci Panskiej, a poslalby onego, który wam opowiedziany jest, Jezusa Chrystusa.
\par 21 Który zaiste niebiosa ma objac az do czasu naprawienia wszystkich rzeczy, co byl przepowiedzial Bóg przez usta wszystkich swietych swoich proroków od wieków.
\par 22 Albowiem Mojzesz do ojców rzekl: Proroka wam wzbudzi Pan, Bóg wasz, z braci waszych, jako mie; onego sluchac bedziecie we wszystkiem, cokolwiek do was mówic bedzie.
\par 23 I stanie sie, ze kazda dusza, która by nie sluchala tego proroka, bedzie wygladzona z ludu.
\par 24 Alec i wszyscy prorocy od Samuela i od innych po nim, ilekolwiek ich mówilo, przepowiadali tez te dni.
\par 25 Wy jestescie synami prorockimi i przymierza, które postanowil Bóg z ojcami naszymi, mówiac do Abrahama: A w nasieniu twojem blogoslawione beda wszystkie narody ziemi.
\par 26 Wamci naprzód Bóg wzbudziwszy Syna swego Jezusa, poslal go, aby wam blogoslawil; zeby sie kazdy z was odwrócil od zlosci swoich.

\chapter{4}

\par 1 A gdy to oni mówili do ludu, nadeszli ich kaplani i hetmani koscielni, i Saduceuszowie.
\par 2 Obrazajac sie, iz uczyli lud, a opowiadali w Jezusie powstanie od umarlych.
\par 3 I wrzucili na nie rece, a podali je do wiezienia az do jutra; bo juz byl wieczór.
\par 4 A wiele z tych, którzy one slowa slyszeli, uwierzyli; i byla liczba mezów okolo pieciu tysiecy.
\par 5 I stalo sie nazajutrz, ze sie zebrali przelozeni ich i starsi, i nauczeni w Pismie w Jeruzalemie,
\par 6 I Annasz, najwyzszy kaplan, i Kaifasz, i Jan, i Aleksander, i ile ich bylo z rodu najwyzszych kaplanów;
\par 7 A postawiwszy je w posrodku, pytali ich: Która moca a któremescie to imieniem uczynili?
\par 8 Tedy Piotr, bedac pelen Ducha Swietego, rzekl do nich: Przelozeni ludu i starsi Izraelscy!
\par 9 Poniewaz my dzis mamy byc sadzeni dla dobrodziejstwa czlowiekowi niemocnemu uczynionego, jakoby on byl uzdrowiony;
\par 10 Niech wam wszystkim wiadomo bedzie i wszystkiemu ludowi Izraelskiemu, ze w imieniu Jezusa Chrystusa Nazarenskiego, któregoscie wy ukrzyzowali, którego Bóg wzbudzil od umarlych, przez tego ten stoi przed wami zdrowym.
\par 11 Tenci jest kamien on wzgardzony od was budujacych, który sie stal glowa wegielna.
\par 12 I nie masz w zadnym innym zbawienia; albowiem nie masz zadnego imienia pod niebem, danego ludziom, przez które bysmy mogli byc zbawieni.
\par 13 Widzac tedy bezpiecznosc Piotrowa i Janowa, i zrozumiawszy, iz ludzmi byli nieuczonymi i prostakami, dziwowali sie i poznali je, iz byli z Jezusem.
\par 14 Widzac tez onego czlowieka z nimi stojacego, który byl uzdrowiony, nie mieli co przeciwko temu mówic.
\par 15 A rozkazawszy im precz ustapic z rady, radzili sie miedzy soba,
\par 16 Mówiac: Cóz uczynimy tym ludziom? Bo, ze jawny cud przez nie jest uczyniony, to wszystkim mieszkajacym w Jeruzalemie wiadomo jest, a nie mozemy tego zaprzec.
\par 17 Ale aby sie to wiecej nie rozslawialo miedzy ludzmi, zagrozmy im srodze, aby wiecej w tem imieniu zadnemu czlowiekowi nie mówili.
\par 18 A zawolawszy ich, zakazali im, aby koniecznie nie mówili, ani uczyli w imieniu Jezusowem.
\par 19 Lecz Piotr i Jan odpowiedziawszy rzekli do nich: Jezliz to sprawiedliwa przed obliczem Bozem, was raczej sluchac niz Boga, rozsadzcie.
\par 20 Albowiem my nie mozemy tego, cosmy widzieli i slyszeli, nie mówic.
\par 21 A oni zagroziwszy im, wypuscili je, nic nie znalazlszy, jakoby je skarac dla ludu i wszyscy chwalili Boga za to, co sie bylo stalo.
\par 22 Bo onemu czlowiekowi bylo wiecej niz czterdziesci lat, nad którym sie stal ten cud uzdrowienia.
\par 23 A gdy je wypuszczono, przyszli do swoich i oznajmili im, cokolwiek do nich przedniejsi kaplani i starsi mówili.
\par 24 Którzy uslyszawszy to, jednomyslnie podniesli glos swój ku Bogu i rzekli: Panie! tys jest Bóg, którys uczynil niebo i ziemie, i morze i wszystko, co w nich jest:
\par 25 Którys Duchem Swietym przez usta Dawida, slugi swego, powiedzial: Przeczze sie zburzyli narodowie, a ludzie prózne rzeczy przemyslali?
\par 26 Staneli królowie ziemi i ksiazeta zebrali sie wespól przeciwko Panu i przeciwko pomazancowi jego.
\par 27 Albowiem sie zebrali prawdziwie przeciwko swietemu Synowi twemu Jezusowi, któregos pomazal, Herod i Poncki Pilat z pogany i z ludem Izraelskim.
\par 28 Aby uczynili, cokolwiek reka twoja i rada twoja przedtem postanowila, aby sie stalo.
\par 29 Przetoz teraz, Panie! wejrzyj na pogrózki ich, a daj slugom twoim ze wszystkiem bezpieczenstwem mówic slowo twoje,
\par 30 Sciagajac reke twoje ku uzdrawianiu i ku czynieniu znamion i cudów, przez imie swietego Syna twego Jezusa.
\par 31 A gdy sie oni modlili, zatrzasnelo sie ono miejsce, na którem byli zgromadzeni, i napelnieni sa wszyscy Duchem Swietym i mówili slowo Boze z bezpieczenstwem.
\par 32 A onego mnóstwa wierzacych bylo serce jedno i dusza jedna, a zaden z majetnosci swoich nie zwal nic swojem wlasnem, ale mieli wszystkie rzeczy spólne.
\par 33 A wielka moca Apostolowie dawali swiadectwo o zmartwychwstaniu Pana Jezusowem i byla wielka laska nad nimi wszystkimi.
\par 34 Bo zadnego nie bylo miedzy nimi niedostatecznego; gdyz którzykolwiek mieli role albo domy, sprzedawajac przynosili pieniadze za to, co posprzedawali,
\par 35 I kladli przed nogi apostolskie, i rozdawano to kazdemu, ile komu bylo potrzeba.
\par 36 Tedy Jozes, który nazwany byl od Apostolów Barnabaszem, (co sie wyklada: syn pociechy), Lewita, z Cypru rodem,
\par 37 Majac role, sprzedawszy ja, przyniósl pieniadze i polozyl je u nóg apostolskich.

\chapter{5}

\par 1 A maz niektóry imieniem Ananijasz, z Safira, zona swoja, sprzedal majetnosc,
\par 2 I ujal nieco z onych pieniedzy z wiadomoscia zony swojej, a przynióslszy czesc niejaka, polozyl u nóg apostolskich.
\par 3 I rzekl Piotr: Ananijaszu! przeczze szatan napelnil serce twoje, abys klamal Duchowi Swietemu i ujal z pieniedzy za role?
\par 4 Izali to, cos mial, nie twoje bylo? a cos sprzedal, nie w twojej mocy zostawalo? Przeczzes te rzecz przypuscil do serca twego? Nie sklamales ludziom, ale Bogu.
\par 5 Tedy uslyszawszy Ananijasz te slowa, padl niezywy. I przyszedl strach wielki na wszystkich, którzy to slyszeli.
\par 6 A wstawszy mlodziency, porwali go, a wynióslszy pogrzebli.
\par 7 I stalo sie po chwili, jakoby po trzech godzinach, ze i zona jego nie wiedzac, co sie stalo, weszla.
\par 8 I rzekl jej Piotr: Powiedz mi, jezliscie za tyle te role sprzedali? A ona rzekla: Tak jest, za tyle.
\par 9 A Piotr rzekl do niej: Przeczzescie sie z soba zmówili, abyscie kusili Ducha Panskiego? Oto nogi tych, którzy pogrzebli meza twego, u drzwi sa i ciebiec wyniosa.
\par 10 I padla zaraz przed nogami jego niezywa. A wszedlszy mlodziency, znalezli ja umarla, a wynióslszy pogrzebli ja podle meza jej.
\par 11 I przyszedl strach wielki na wszystek zbór i na wszystkich, którzy to slyszeli.
\par 12 Lecz przez rece apostolskie dzialo sie wiele znamion i cudów miedzy ludem, (a byli wszyscy jednomyslnie w przysionku Salomonowym.
\par 13 A z innych zaden nie smial sie do nich przylaczyc; ale lud wiele o nich trzymal.
\par 14 I owszem przybywalo mnóstwo wierzacych Panu, mezów i niewiast).
\par 15 Tak ze i na ulice wynosili chorych i kladli je na poscielach i lózkach, aby przynajmniej cien Piotra przychodzacego zacienil niektórych z nich.
\par 16 Schodzilo sie tez i mnóstwo z okolicznych miast do Jeruzalemu, przynoszac chorych i nagabanych od duchów nieczystych; a ci wszyscy byli uzdrowieni.
\par 17 Tedy powstawszy najwyzszy kaplan i wszyscy, którzy z nim byli, którzy byli z sekty Saduceuszów, napelnieni sa zazdroscia;
\par 18 I targneli sie rekoma na Apostoly i podali je do wiezienia pospolitego.
\par 19 Ale Aniol Panski w nocy otworzyl drzwi u wiezienia, a wywiódlszy je rzekl:
\par 20 Idzciez, a stawiwszy sie, mówcie do ludu w kosciele wszystkie slowa tego zywota.
\par 21 Tedy oni uslyszawszy to, weszli na switaniu do kosciola i uczyli. A przyszedlszy najwyzszy kaplan i którzy z nim byli, zwolali rade i wszystkie starsze synów Izraelskich, i poslali do wiezienia, aby byli przywiedzieni.
\par 22 A gdy sludzy przyszli, nie znalezli ich w wiezieniu, co wróciwszy sie, oznajmili, mówiac:
\par 23 Wiezieniec wprawdzie znalezlismy zamknione ze wszelka pilnoscia i stróze na dworze przede drzwiami stojace, lecz otworzywszy, zadnegosmy w niem nie znalezli.
\par 24 A gdy te slowa uslyszeli i najwyzszy kaplan, i hetman koscielny, i przedniejsi kaplani watpili o nich, co by to bylo.
\par 25 A przyszedlszy ktos, oznajmil im, mówiac: Oto mezowie, którescie podali do wiezienia, stoja w kosciele, a ucza lud.
\par 26 Tedy poszedl hetman z slugami i przywiódl je bez gwaltu; (bo sie ludu bali, aby nie byli ukamionowani.)
\par 27 A przywiódlszy je, stawili je przed rada; i pytal ich najwyzszy kaplan, mówiac:
\par 28 Izalismy wam surowo nie zakazali, abyscie w tem imieniu nie uczyli? A oto napelniliscie Jeruzalem nauka wasza i chcecie na nas wprowadzic krew czlowieka tego.
\par 29 Tedy odpowiadajac Piotr i Apostolowie, rzekli: Wiecej trzeba sluchac Boga, niz ludzi.
\par 30 Bóg on ojców naszych wzbudzil Jezusa, któregoscie wy zabili, zawiesiwszy na drzewie.
\par 31 Tego Bóg za ksiazecia i zbawiciela wywyzszyl prawica swoja, aby dana byla ludowi Izraelskiemu pokuta i odpuszczenie grzechów.
\par 32 A my jestesmy swiadkami jego w tem, co mówimy, takze i Duch Swiety, którego dal Bóg tym, którzy mu sa posluszni.
\par 33 A oni to slyszac, pukali sie i radzili o tem, jakoby je zgladzic.
\par 34 Tedy powstawszy w radzie niektóry Faryzeusz, imieniem Gamalijel, nauczyciel zakonny, zacny u wszystkiego ludu, rozkazal, aby na mala chwile precz wywiedziono Apostoly;
\par 35 I rzekl do nich: Mezowie Izraelscy! miejcie sie na baczeniu z strony tych ludzi, co byscie mieli czynic.
\par 36 Albowiem przed temi dniami powstal byl Teudas, udawajac sie za cos, do którego sie przywiazalo mezów w liczbie okolo czterechset; którego zabito, a wszyscy, którzy z nim przestawali, rozproszeni sa i wniwecz sie obrócili.
\par 37 Po nim powstal Judas Galilejczyk za dni popisu i uwiódl wiele ludu za soba; ale i on zginal, i wszyscy, którzy z nim przestawali, rozproszeni sa.
\par 38 Przetoz teraz powiadam wam: Dajcie pokój tym ludziom i zaniechajcie ich; albowiem jezlizec jest z ludzi ta rada albo ta sprawa, wniwecz sie obróci;
\par 39 Ale jezlic jest z Boga, nie bedziecie mogli tego rozerwac, byscie snac i z Bogiem walczacymi nie byli znalezieni.
\par 40 I usluchali go. A zawolawszy Apostolów i ubiwszy je, zakazali, aby nie mówili w imieniu Jezusowem; i wypuscili je.
\par 41 A tak oni szli od oblicznosci onej rady, radujac sie, iz sie stali godnymi odnosic zelzywosc dla imienia Jezusowego.
\par 42 I nie przestawali na kazdy dzien w kosciele i po domach nauczac i opowiadac Jezusa Chrystusa.

\chapter{6}

\par 1 A w onez dni, gdy sie przymnazalo uczniów, wszczelo sie szemranie Greków przeciwko Zydom, iz byly zaniedbywane w poslugiwaniu powszedniem wdowy ich.
\par 2 A tak oni dwunastu zwolawszy mnóstwo uczniów, rzekli: Nie jest sluszna, zebysmy my opusciwszy slowo Boze, sluzyli stolom.
\par 3 Upatrzciez tedy, bracia! miedzy soba siedm mezów, dobre swiadectwo majacych, pelnych Ducha Swietego i madrosci, których bysmy postanowili nad ta sprawa.
\par 4 A my modlitwy i uslugi slowa pilnowac bedziemy.
\par 5 I podobala sie ta rzecz onemu wszystkiemu mnóstwu. I obrali Szczepana, meza pelnego wiary i Ducha Swietego, i Filipa, i Prochora, i Nikanora, i Tymona, i Parmena, i Mikolaja, nowonawróconego Antyjochenczyka.
\par 6 Tych stawili przed Apostolów, którzy pomodliwszy sie, kladli na nich rece.
\par 7 I roslo slowo Boze, i pomnazal sie bardzo poczet uczniów w Jeruzalemie: wielkie tez mnóstwo kaplanów bylo posluszne wierze.
\par 8 A Szczepan bedac pelen wiary i mocy, czynil cuda i znamiona wielkie miedzy ludem.
\par 9 I powstali niektórzy z tych, którzy byli z bóznicy, która zowia Libertynów i Cyrenejczyków, i Aleksandryjanów, i tych, którzy byli z Cylicyi i z Azyi, gadajac z Szczepanem.
\par 10 Lecz nie mogli odporu dac madrosci i duchowi, który mówil.
\par 11 Tedy naprawili mezów, którzy powiedzieli: Mysmy go slyszeli mówiacego slowa bluzniercze przeciwko Mojzeszowi i przeciwko Bogu.
\par 12 A tak wzruszyli lud i starszych, i nauczonych w Pismie; a powstawszy, porwali go i przywiedli do rady.
\par 13 I stawili falszywych swiadków, którzy rzekli: Ten czlowiek nie przestaje mówic slów bluznierczych przeciwko temu swietemu miejscu i zakonowi.
\par 14 Albowiemesmy go slyszeli mówiacego: Iz ten Jezus Nazarenski zburzy to miejsce i odmieni ustawy, które nam podal Mojzesz.
\par 15 A patrzac na niego pilnie oni wszyscy, którzy siedzieli w radzie, widzieli oblicze jego jako oblicze anielskie.

\chapter{7}

\par 1 Tedy rzekl najwyzszy kaplan: A takze sie ma ta rzecz?
\par 2 A on rzekl: Mezowie bracia i ojcowie, sluchajcie! Bóg chwaly ukazal sie ojcu naszemu Abrahamowi, gdy byl w Mezopotamii, przedtem niz mieszkal w Haranie.
\par 3 I rzekl do niego: Wynijdz z ziemi twojej i od twojej rodziny, a idz do ziemi, która ci ukaze.
\par 4 Tedy wyszedlszy z ziemi Chaldejskiej, mieszkal w Haranie, a stamtad, gdy umarl ojciec jego, przeniósl go Bóg do ziemi tej, w której wy teraz mieszkacie.
\par 5 I nie dal mu w niej dziedzictwa i na stope nogi, choc mu ja byl obiecal dac w dzierzawe, i nasieniu jego po nim, gdy jeszcze nie mial potomka.
\par 6 I mówil mu tak Bóg: Nasienie twoje bedzie przychodniem w cudzej ziemi i zniewola je, i trapic je beda przez czterysta lat.
\par 7 Ale ten naród, któremu sluzyc beda, ja bede sadzil, rzekl Bóg; a potem wynijda i sluzyc mi beda na tem miejscu.
\par 8 I dal mu przymierze obrzezki; i tak Abraham splodzil Izaaka i obrzezal go dnia ósmego, a Izaak Jakóba, a Jakób dwunastu patryjarchów.
\par 9 A patryjarchowie nienawidzac Józefa, sprzedali go do Egiptu; ale Bóg byl z nim.
\par 10 I wyrwal go ze wszystkich jego ucisków, a dal mu laske i madrosc przed Faraonem, królem Egipskim, który go postanowil ksiazeciem nad Egiptem i nad wszystkim domem swoim.
\par 11 Potem przyszedl glód na wszystke ziemie Egipska i Chananejska, i ucisk wielki, i nie znajdowali zywnosci ojcowie nasi.
\par 12 A gdy uslyszal Jakób, iz zboza byly w Egipcie, poslal ojców naszych pierwszy raz.
\par 13 A za wtórym razem poznany jest Józef od braci swych i objawiony jest Faraonowi naród Józefowy.
\par 14 Tedy Józef poslawszy, przyzwal ojca swego Jakóba i wszystke swoje rodzine w siedmdziesiat i pieciu duszach.
\par 15 I zstapil Jakób do Egiptu, i tam umarl on i ojcowie nasi.
\par 16 I przeniesieni sa do Sychem, i polozeni w grobie, który byl kupil Abraham za pieniadze u synów Hemora, ojca Sychemowego.
\par 17 A gdy sie przyblizyl czas obietnicy, o która byl przysiagl Bóg Abrahamowi, rozrodzil sie lud i rozmnozyl sie w Egipcie.
\par 18 Az nastal inny król, który nie znal Józefa.
\par 19 Ten podchodzac zdradliwie naród nasz, trapil ojców naszych, tak iz musieli wymiatac niemowlatka swoje, zeby sie nie rozkrzewialy.
\par 20 Pod ten czas narodzil sie Mojzesz, a byl krasnym z daru Bozego, którego chowano przez trzy miesiace w domu ojca jego.
\par 21 A gdy byl wyrzucony, wziela go córka Faraonowa i wychowala go sobie za syna.
\par 22 I wycwiczony jest Mojzesz we wszelkiej madrosci Egipskiej, a byl mozny w mowach i w uczynkach.
\par 23 A gdy mu bylo czterdziesci lat, przyszlo mu na mysl, aby nawiedzil braci swych, syny Izraelskie.
\par 24 A ujrzawszy jednego ukrzywdzonego, ujal sie on i pomscil sie krzywdy tego, który bezprawie cierpial, zabiwszy Egipczanina.
\par 25 Albowiem mniemal, ze bracia jego rozumieja, ze Bóg przez reke jego daje im wybawienie; lecz oni tego nie rozumieli.
\par 26 A nazajutrz pokazal sie im, gdy sie z soba bili i prowadzil je do pokoju, mówiac: Mezowie! bracia jestescie sobie; przeczze sie spolem krzywdzicie?
\par 27 Lecz ten, co krzywdzil blizniego, odegnal go, mówiac: Któz cie postanowil ksiazeciem i sedzia nad nami?
\par 28 Izali mie ty chcesz zabic, jakos wczoraj zabil Egipczanina?
\par 29 I uciekl Mojzesz za temi slowy i byl przychodniem w ziemi Madyjanskiej, gdzie splodzil dwóch synów.
\par 30 A gdy sie wypelnilo czterdziesci lat, ukazal mu sie na puszczy góry Synaj Aniol Panski w plomieniu ognistym w krzaku.
\par 31 A Mojzesz ujrzawszy, zadziwil sie onemu widzeniu; a gdy przystapil, aby sie temu przypatrzyl, stal sie do niego glos Panski:
\par 32 Jam jest Bóg ojców twoich, Bóg Abrahama, Bóg Izaaka, i Bóg Jakóba. A zadrzawszy Mojzesz nie smial sie przypatrywac.
\par 33 I rzekl mu Pan: Zzuj obuwie z nóg twoich; bo miejsce, na którem stoisz, jest ziemia swieta.
\par 34 Widzac widzialem utrapienie ludu mego, który jest w Egipcie, i slyszalem wzdychanie ich, a zstapilem, zebym je wybawil; przetoz teraz chodz, posle cie do Egiptu.
\par 35 Tego Mojzesza, którego sie byli zaprzali, mówiac: Któz cie postanowil ksiazeciem i sedzia? Tego Bóg ksiazeciem i wybawicielem poslal przez reke Aniola, który mu sie ukazal w krzaku.
\par 36 I ten je wywiódl, czyniac cuda i znamiona w ziemi Egipskiej i na morzu Czerwonem, i na puszczy, przez czterdziesci lat.
\par 37 Tenci jest Mojzesz, który rzekl synom Izraelskim: Proroka wam wzbudzi Pan, Bóg wasz, z braci waszych, jako mie, onego sluchac bedziecie;
\par 38 Ten jest, który byl w zgromadzeniu na puszczy z Aniolem, który mówil do niego na górze Synaj, i z ojcami naszymi, który przyjal slowa Boze zywe, aby je nam podal.
\par 39 Któremu nie chcieli posluszni byc ojcowie nasi; ale go odrzucili i obrócili sie sercy swemi do Egiptu.
\par 40 Mówiac do Aarona: Uczyn nam bogi, którzy by szli przed nami; albowiem Mojzeszowi onemu, który nas wywiódl z ziemi Egipskiej, nie wiemy co sie stalo.
\par 41 I uczynili w one dni cielca i sprawowali ofiare onemu balwanowi, i weselili sie w sprawach rak swoich;
\par 42 I odwrócil sie Bóg i podal je, aby sluzyli wojsku niebieskiemu, jako napisano jest w ksiegach prorockich: Zazescie mu bite i inne ofiary ofiarowali na puszczy przez czterdziesci lat, domu Izraelski?
\par 43 Owszem nosiliscie namiot Molocha i gwiazde boga waszego Remfana, te obrazy, którescie sobie uczynili, abyscie sie im klaniali; przetoz was zaprowadze za Babilon.
\par 44 Namiot swiadectwa mieli ojcowie nasi na puszczy, jako byl rozrzadzil ten, który powiedzial Mojzeszowi, aby go uczynil wedlug ksztaltu, który widzial.
\par 45 Który wziawszy ojcowie nasi, wniesli z Jozuem tam, gdzie byla osiadlosc poganów, których Bóg wygnal od oblicznosci ojców naszych, az do dni Dawidowych;
\par 46 Który znalazl laske przed obliczem Bozem i prosil, aby znalazl namiot Bogu Jakóbowemu.
\par 47 A Salomon zbudowal mu dom.
\par 48 Ale on Najwyzszy nie mieszka w kosciolach reka uczynionych, jako prorok mówi:
\par 49 Niebo jest stolica moja, a ziemia podnózek nóg moich. Cóz mi za dom zbudujecie, mówi Pan, albo które jest miejsce odpocznienia mego?
\par 50 Izali reka moja tego wszystkiego nie uczynila?
\par 51 Ludzie twardego karku i nieobrzezanego serca, i uszów! wy sie zawzdy sprzeciwiacie Duchowi Swietemu jako ojcowie wasi, tak i wy.
\par 52 Któregoz z proroków nie przesladowali ojcowie wasi, i nie pozabijali tych, którzy przedtem opowiadali o przyjsciu tego Sprawiedliwego, któregoscie wy sie teraz stali zdrajcami i mordercami?
\par 53 Którzyscie wzieli zakon przez rozrzadzenie anielskie, a nie strzegliscie go.
\par 54 Tedy sluchajac tego, pukali sie w sercach swych i zgrzytali na niego zebami.
\par 55 A on bedac pelen Ducha Swietego, patrzac pilnie w niebo, ujrzal chwale Boza i Jezusa stojacego po prawicy Bozej.
\par 56 I rzekl: Oto widze niebiosa otworzone i Syna czlowieczego stojacego po prawicy Bozej.
\par 57 A oni krzyknawszy glosem wielkim, zatulili uszy swoje i rzucili sie na niego jednomyslnie.
\par 58 A wypchnawszy go z miasta, kamionowali; a swiadkowie zlozyli szaty swoje u stóp mlodzienca, którego zwano Saul.
\par 59 I kamionowali Szczepana modlacego sie i mówiacego: Panie Jezu! przyjmij ducha mojego!
\par 60 A kleknawszy na kolana, zawolal glosem wielkim: Panie! nie poczytaj im tego za grzech! A to rzeklszy, zasnal.

\chapter{8}

\par 1 A Saul zezwolil na zabicie jego. I wszczelo sie onegoz czasu wielkie przesladowanie przeciwko zborowi, który byl w Jeruzalemie, i rozproszyli sie wszyscy po krainach ziemi Judzkiej i Samaryi, oprócz Apostolów.
\par 2 I pogrzebli Szczepana mezowie bogobojni i uczynili nad nim placz wielki.
\par 3 A Saul niszczyl zbór, wchodzac w domy, a wywlóczajac meze i niewiasty, podawal je do wiezienia.
\par 4 A ci, którzy byli rozproszeni, chodzili opowiadajac slowo Boze.
\par 5 Lecz Filip zaszedlszy do miasta Samaryjskiego, opowiadal im Chrystusa.
\par 6 A lud mial wzglad jednomyslnie na to, co Filip mówil, sluchajac i widzac cuda, które czynil.
\par 7 Albowiem duchy nieczyste od wielu tych, którzy je mieli, wolajac glosem wielkim wychodzily, a wiele powietrzem ruszonych i chromych uzdrowieni sa.
\par 8 I stala sie radosc wielka w onem miescie.
\par 9 A niektóry maz, imieniem Szymon, byl przedtem w onem miescie bawiacy sie nauka czarnoksieska i lud Samaryjski mamil, powiadajac sie byc czyms wielkim.
\par 10 Na którego sie ogladali wszyscy od najmniejszego az do najwiekszego, mówiac: Tenci jest ona moc Boza wielka.
\par 11 A ogladali sie nan przeto, iz je od niemalego czasu mamil czarnoksiestwy swemi.
\par 12 A gdy uwierzyli Filipowi, opowiadajacemu królestwo Boze i imie Jezusa Chrystusa, chrzcili sie mezowie i niewiasty.
\par 13 Tedy i sam Szymon uwierzyl, a ochrzciwszy sie, trzymal sie Filipa, a widzac cuda i mocy wielkie, które sie dzialy, zdumiewal sie.
\par 14 A uslyszawszy Apostolowie, którzy byli w Jeruzalemie, iz Samaryja przyjela slowo Boze, poslali do nich Piotra i Jana.
\par 15 Którzy tam przyszedlszy, modlili sie za nimi, aby przyjeli Ducha Swietego.
\par 16 (Albowiem jeszcze byl na zadnego z nich nie zstapil; ale tylko pochrzczeni byli w imie Pana Jezusowe.)
\par 17 Tedy na nie wkladali rece, a oni przyjmowali Ducha Swietego.
\par 18 A ujrzawszy Szymon, ze przez wkladanie rak apostolskich byl dawany Duch Swiety, przyniósl im pieniadze.
\par 19 Mówiac: Dajcie i mnie te moc, aby ten, na któregokolwiek bym rece wlozyl, wzial Ducha Swietego.
\par 20 I rzekl mu Piotr: Pieniadze twoje niech z toba beda na zginienie, zes mniemal, zeby dar Bozy mial byc za pieniadze nabywany.
\par 21 Nie masz w tej rzeczy czastki, ani losu, gdy serce twoje nie jest proste przed obliczem Bozem.
\par 22 Przetoz pokutuj z tej twojej zlosci, a pros Boga; owac snac bedzie odpuszczony ten zamysl serca twego.
\par 23 Albowiem cie widze byc w gorzkosci zólci i w zwiazce nieprawosci.
\par 24 Odpowiedziawszy tedy Szymon, rzekl: Módlcie sie wy za mna Panu, aby na mie nic nie przyszlo z tych rzeczy, którescie powiedzieli.
\par 25 A tak oni oswiadczywszy i opowiedziawszy slowo Panskie, wrócili sie do Jeruzalemu i w wielu miasteczkach Samarytanskich Ewangielije opowiadali.
\par 26 Lecz Aniol Panski rzekl do Filipa, mówiac: Wstan, a idz ku poludniowi na droge, która od Jeruzalemu idzie ku Gazie, która jest pusta.
\par 27 A on wstawszy, szedl. A oto maz Murzyn rzezaniec, komornik królowej murzynskiej Kandaces, który byl nad wszystkiemi skarbami jej, a przyjechal byl do Jeruzalemu, aby sie modlil;
\par 28 I wracal sie, siedzac na wozie swoim, a czytal Izajasza proroka.
\par 29 I rzekl Duch Filipowi: Przystap, a przylacz sie do tego wozu.
\par 30 A przybiezawszy Filip, uslyszal go czytajacego Izajasza proroka i rzekl: Rozumieszze, co czytasz?
\par 31 A on rzekl: Jakoz moge rozumiec, jezliby mi kto nie wylozyl? I prosil Filipa, a wstapil i siedzial z nim.
\par 32 A miejsce onego Pisma, które czytal, to bylo: Jako owca ku zabiciu wiedziony jest, a jako baranek niemy przed tym, który go strzyze, tak nie otworzyl ust swoich;
\par 33 W unizeniu jego sad jego zniesiony jest, a rodzaj jego któz wypowie? albowiem zniesiony byl z ziemi zywot jego.
\par 34 A odpowiadajac rzezaniec Filipowi, rzekl: Prosze cie, o kim to prorok mówi? Sam o sobie, czyli o kim innym?
\par 35 Tedy otworzywszy Filip usta swe, a poczawszy od tego Pisma, opowiadal mu Jezusa.
\par 36 A gdy jechali droga, przyjechali nad jedne wode. Tedy rzekl rzezaniec: Otóz woda! Cóz na przeszkodzie, abym nie mial byc ochrzczony?
\par 37 I rzekl Filip: Jezliz wierzysz z calego serca, wolnoc. A on odpowiedziawszy, rzekl: Wierze, iz Jezus Chrystus jest Syn Bozy.
\par 38 I kazal stanac wozowi; i zstapili obadwaj w wode, Filip i rzezaniec, i ochrzcil go.
\par 39 A gdy wystapili z wody, porwal Filipa Duch Panski i nie widzial go wiecej rzezaniec, ale jechal droga swoja, radujac sie.
\par 40 A Filip az w Azocie jest znaleziony, a chodzac kazal Ewangielije po wszystkich miastach, az przyszedl do Cezaryi.

\chapter{9}

\par 1 A Saul jeszcze dychajac grozba i morderstwem przeciwko uczniom Panskim, przyszedl do najwyzszego kaplana.
\par 2 I prosil go o listy do Damaszku do bóznic, iz jezliby tam znalazl tej drogi (ta droga idacych) których mezów albo niewiasty, aby ich zwiazane przywiódl do Jeruzalemu.
\par 3 A gdy jechal, stalo sie, gdy sie przyblizal do Damaszku, ze z predka oswiecila go swiatlosc z nieba.
\par 4 A padlszy na ziemie, uslyszal glos do siebie mówiacy: Saulu! Saulu! przeczze mie przesladujesz?
\par 5 Tedy rzekl: Ktos jest, Panie? A Pan rzekl: Jam jest Jezus, którego ty przesladujesz; trudno tobie przeciw oscieniowi wierzgac.
\par 6 A Saul drzac i bojac sie, rzekl: Panie! co chcesz, abym ja uczynil? A Pan do niego: Wstan, a wnijdz do miasta, a tam ci powiedza, co bys ty mial czynic.
\par 7 A mezowie, którzy z nim byli w drodze, staneli, zdumiawszy sie; glos tylko slyszac, ale nikogo nie widzac.
\par 8 I wstal Saul z ziemi, a otworzywszy oczy swoje, nikogo nie widzial. Tedy ujawszy go za reke; prowadzili go do Damaszku,
\par 9 Kedy byl trzy dni nie widzac, i nie jadl ani pil.
\par 10 A byl niektóry uczen w Damaszku, imieniem Ananijasz; i rzekl Pan do niego w widzeniu: Ananijaszu! A on rzekl: Otom ja, Panie!
\par 11 A Pan rzekl do niego: Wstan, a idz na ulice, która zowia prosta, a szukaj w domu Judowym Saula imieniem Tarsenczyka; albowiem oto sie modli.
\par 12 I widzial w widzeniu meza, imieniem Ananijasz, wchodzacego i reke na sie wkladajacego, aby przejrzal.
\par 13 I odpowiedzial Ananijasz: Panie! slyszalem od wielu o tym mezu, jako wiele zlego czynil swietym twoim w Jeruzalemie.
\par 14 I tu ma moc od najwyzszych kaplanów, aby wiazal wszystkie wzywajace imienia twego.
\par 15 I rzekl do niego Pan: Idzze, albowiem mi ten jest naczyniem wybranem, aby nosil imie moje przed pogany i królmi, i przed syny Izraelskimi.
\par 16 Albowiem ja mu ukaze, jako wiele musi cierpiec dla imienia mego.
\par 17 I poszedl Ananijasz, i wszedl do onego domu, a wlozywszy na niego rece, rzekl: Saulu, bracie! Pan mie poslal, Jezus on, któryc sie ukazal w drodze, któras jechal, abys przejrzal, a byl napelniony Duchem Swietym.
\par 18 I zarazem spadly z oczów jego jako luski i wnet przejrzal, a wstawszy ochrzczony jest.
\par 19 A wziawszy pokarm, posilil sie. I byl Saul z uczniami, którzy byli w Damaszku, kilka dni.
\par 20 I zaraz kazal w bóznicach Chrystusa, ze on jest Synem Bozym.
\par 21 I zdumiewali sie wszyscy, którzy go sluchali, i mówili: Izali to nie jest ten, który burzyl w Jeruzalemie tych, którzy wzywali imienia tego? i tuc na to przyszedl, aby ich zwiazawszy, wiódl do najwyzszych kaplanów?
\par 22 A Saul tem wiecej zmacnial sie i zawstydzal Zydy, którzy mieszkali w Damaszku, dowodzac, iz ten jest Chrystus.
\par 23 A gdy przeszlo niemalo dni, uradzili Zydowie miedzy soba, aby go zabili.
\par 24 Ale sie dowiedzial Saul o zasadzce ich. Strzegli tez bram we dnie i w nocy, aby go zabili.
\par 25 Lecz uczniowie wziawszy go w nocy, spuscili go po powrozie przez mur w koszu.
\par 26 A gdy przyszedl Saul do Jeruzalemu, kusil sie przylaczyc do uczniów; ale sie go wszyscy bali, nie wierzac, aby byl uczniem.
\par 27 Lecz Barnabasz wziawszy go, przywiódl go do Apostolów i powiadal im, jako w drodze widzial Pana, a iz mówil do niego, i jako w Damaszku bezpiecznie mówil w imieniu Jezusowem.
\par 28 I mieszkal z nimi w Jeruzalemie.
\par 29 A bezpiecznie sobie poczynajac w imieniu Pana Jezusowem, mówil i gadal z Grekami; a oni sie starali, jako by go zabic.
\par 30 O czem dowiedziawszy sie bracia, odprowadzili go do Cezaryi i odeslali go do Tarsu.
\par 31 A tak zbory po wszystkiej Judzkiej ziemi i Galilei, i Samaryi mialy pokój, budujac sie i chodzac w bojazni Panskiej, a przez pocieche Ducha Swietego rozmnazaly sie.
\par 32 I stalo sie, gdy Piotr obchodzil wszystkie, przyszedl tez do swietych, którzy mieszkali w Liddzie.
\par 33 Tamze znalazl czlowieka niektórego, imieniem Eneasz, od osmiu lat na lozu lezacego, który byl powietrzem ruszony.
\par 34 I rzekl mu Piotr: Eneaszu! uzdrawia cie Jezus Chrystus; wstanze, a posciel sobie. I zarazem wstal.
\par 35 A widzieli go wszyscy, którzy mieszkali w Liddzie i w Saronie, którzy sie nawrócili do Pana.
\par 36 A byla w Joppie niektóra uczennica, imieniem Tabita, która wylozywszy, zowie sie Dorka; ta byla pelna dobrych uczynków i jalmuzny, które czynila.
\par 37 I stalo sie w one dni, ze rozniemóglszy sie, umarla; która omywszy, polozyli na sali.
\par 38 A iz Lidda byla blisko Joppy, uczniowie uslyszawszy, ze tam jest Piotr, poslali do niego dwóch mezów, proszac go, aby sie nie lenil przyjsc do nich.
\par 39 Tedy wstawszy Piotr, szedl z nimi; a gdy przyszedl, wprowadzili go na sale i obstapily go wszystkie wdowy, placzac i ukazujac suknie i plaszcze, które im Dorka robila, póki byla z nimi.
\par 40 A Piotr wygnawszy precz wszystkie, kleknal na kolana i modlil sie, a obróciwszy sie do onego ciala, rzekl: Tabito, wstan! a ona otworzyla oczy swoje i ujrzawszy Piotra, usiadla.
\par 41 A on podawszy jej reke, podniósl ja, a zawolawszy swietych i wdów, stawil ja zywa.
\par 42 I rozslawilo sie to po wszystkiej Joppie, i wiele ich uwierzylo w Pana.
\par 43 I stalo sie, ze przez wiele dni zostal Piotr w Joppie u niejakiego Szymona, garbarza.

\chapter{10}

\par 1 A w Cezaryi byl maz niektóry, imieniem Kornelijusz, setnik, z roty, która zwano Wloska;
\par 2 Pobozny i bojacy sie Boga ze wszystkim domem swoim, i czyniacy jalmuzny wielkie ludowi.
\par 3 A ten sie zawsze Bogu modlac, widzial jawnie w widzeniu, jakoby o dziewiatej godzinie na dzien, Aniola Bozego, ze wszedl do niego i rzekl mu: Kornelijuszu!
\par 4 A on pilnie nan patrzac, a przestraszony bedac, rzekl: Cóz jest, Panie? I rzekl mu: Modlitwy twoje i jalmuzny twoje wstapily na pamiec przed oblicznosc Boza.
\par 5 Przetoz teraz poslij meze do Joppy, a przyzwij Szymona, którego zowia Piotrem.
\par 6 Ten ma gospode u niektórego Szymona, garbarza, który ma dom nad morzem; ten ci powie, co bys mial czynic.
\par 7 A gdy odszedl Aniol, który mówil z Kornelijuszem, zawolawszy dwóch slug swoich i zolnierza poboznego z tych, którzy przy nim ustawicznie byli;
\par 8 A rozpowiedziawszy im wszystko, poslal je do Joppy.
\par 9 A nazajutrz, gdy byli w drodze, a przyblizali sie do miasta, wstapil Piotr na dach, aby sie modlil okolo godziny szóstej.
\par 10 A bedac laknacym chcial jesc; a gdy mu oni jesc gotowali, przypadlo na niego zachwycenie.
\par 11 I ujrzal niebo otworzone i zstepujace na sie naczynie niejakie, jakoby przescieradlo wielkie, za cztery rogi uwiazane i spuszczone na ziemie;
\par 12 W którem byly wszelkie ziemskie czworonogie zwierzeta i bestyje, i gadziny i ptastwo niebieskie.
\par 13 I stal sie glos do niego: Wstan Piotrze! rzez, a jedz.
\par 14 A Piotr rzekl: Zadna miara, Panie! gdyzem nigdy nie jadl nic pospolitego albo nieczystego.
\par 15 Tedy zasie powtóre stal sie glos do niego: Co Bóg oczyscil, ty nie miej tego za nieczyste.
\par 16 A to sie stalo po trzykroc. I wziete jest zasie ono naczynie do nieba.
\par 17 A gdy Piotr sam w sobie watpil, co by to bylo za widzenie, które widzial, tedy oto ci mezowie, którzy byli poslani do Kornelijusza, pytajacy sie o dom Szymonowy, stali przede drzwiami;
\par 18 A zawolawszy, wywiadywali sie, jezliby tam Szymon, którego zowia Piotrem, gospode mial.
\par 19 A gdy Piotr myslil o onem widzeniu, rzekl mu Duch: Oto cie trzej mezowie szukaja.
\par 20 Przetoz wstawszy, zstap, a idz z nimi, nic nie watpiac, bomci ja je poslal.
\par 21 Tedy Piotr zstapiwszy do onych mezów, którzy od Kornelijusza do niego poslani byli, rzekl: Otom ja jest, którego szukacie. Cóz za przyczyna, dla którejscie przyszli?
\par 22 A oni rzekli: Kornelijusz setnik, maz sprawiedliwy i bojacy sie Boga i majacy dobre swiadectwo od wszystkiego narodu zydowskiego, w widzeniu jest od Aniola swietego napomniony, aby cie wezwal w dom swój i sluchal slów od ciebie.
\par 23 Tedy zawolawszy ich do domu, przyjal je do gospody. A drugiego dnia Piotr szedl z nimi i niektórzy z braci z Joppy szli z nimi.
\par 24 A nazajutrz weszli do Cezaryi. A Kornelijusz czekal ich, wezwawszy powinowatych swoich i bliskich przyjaciól.
\par 25 I stalo sie, gdy wchodzil Piotr, zabiezawszy mu Kornelijusz, przypadl do nóg jego i poklonil sie.
\par 26 Ale go Piotr podniósl, mówiac: Wstan! i jamci tez jest czlowiek.
\par 27 A rozmawiajac z nim, wszedl, a znalazl wiele tych, którzy sie byli zeszli.
\par 28 I rzekl do nich: Wy wiecie, ze sie nie godzi mezowi Zydowinowi przylaczac albo schadzac z cudzoziemcem; lecz mnie Bóg ukazal, zebym zadnego czlowieka nie nazywal pospolitym albo nieczystym.
\par 29 Przetozem tez nie zbraniajac sie przyszedl, wezwany bedac; pytam tedy, dlaczegoscie mie wezwali?
\par 30 A Kornelijusz rzekl: Od czwartego dnia az do tej godziny poscilem, a o dziewiatej godzinie modlilem sie w domu moim, a oto maz niektóry stanal przede mna w odzieniu jasnem,
\par 31 I rzekl: Kornelijuszu! wysluchana jest modlitwa twoja, a jalmuzny twoje przyszly na pamiec przed oblicznosc Boza.
\par 32 Przetoz poslij do Joppy, a przyzwij Szymona, którego nazywaja Piotrem; ten ma gospode w domu Szymona, garbarza, nad morzem, który przyszedlszy, mówic z toba bedzie.
\par 33 Zaraz tedy poslalem do ciebie, a tys dobrze uczynil, zes przyszedl. Teraz tedy jestesmy wszyscy przed obliczem Bozem przytomni, abysmy sluchali wszystkiego, coc rozkazano od Boga.
\par 34 Tedy Piotr otworzywszy usta, rzekl: Prawdziwie dochodze tego, iz Bóg nie ma wzgledu na osoby;
\par 35 Ale w kazdym narodzie, kto sie go boi, a czyni sprawiedliwosc, jest mu przyjemnym.
\par 36 A co sie tknie slowa, które poslal synom Izraelskim, opowiadajac pokój przez Jezusa Chrystusa, który jest Panem wszystkiego,
\par 37 Wy wiecie, co sie dzialo po wszystkiem Zydostwie, poczawszy od Galilei, po chrzcie, który Jan opowiadal;
\par 38 Jako Jezusa z Nazaretu pomazal Bóg Duchem Swietym i moca, który chodzil, czyniac dobrze i uzdrawiajac wszystkie opanowane od dyjabla; albowiem Bóg byl z nim.
\par 39 A mysmy swiadkami wszystkiego tego, co czynil w krainie Judzkiej i w Jeruzalemie, którego zabili, zawiesiwszy na drzewie.
\par 40 Tego Bóg wzbudzil dnia trzeciego i sprawil, zeby byl objawiony;
\par 41 Nie wszystkiemu ludowi, ale swiadkom przedtem sporzadzonym od Boga, nam, którzysmy z nim jedli i pili po jego zmartwychwstaniu.
\par 42 I rozkazal nam, abysmy kazali ludowi i swiadczyli, ze on jest onym postanowionym od Boga sedzia zywych i umarlych.
\par 43 Temu wszyscy prorocy swiadectwo wydaja, iz przez imie jego odpuszczenie grzechów wezmie kazdy, co w niego wierzy.
\par 44 A gdy jeszcze Piotr mówil te slowa, przypadl Duch Swiety na wszystkie sluchajace tych slów.
\par 45 I zdumieli sie oni, którzy byli z obrzezania wierzacy, którzy byli z Piotrem przyszli, ze i na pogany dar Ducha Swietego jest wylany.
\par 46 Albowiem slyszeli je mówiace jezykami rozlicznemi i wielbiace Boga. Tedy odpowiedzial Piotr:
\par 47 Izali kto moze zabronic wody, zeby ci nie byli pochrzczeni, którzy wzieli Ducha Swietego jako i my?
\par 48 I rozkazal je pochrzcic w imieniu Panskiem. I prosili go, aby u nich zostal na kilka dni.

\chapter{11}

\par 1 I uslyszeli Apostolowie i bracia, którzy byli w Judzkiej ziemi, ze i poganie przyjeli slowo Boze.
\par 2 A gdy Piotr przyszedl do Jeruzalemu, spierali sie z nim ci, którzy byli z obrzezania,
\par 3 Mówiac: Wszedles do mezów nieobrzezanych, a jadles z nimi.
\par 4 Tedy poczawszy Piotr, powiadal im porzadnie, mówiac:
\par 5 Bylem w miescie Joppie, modlac sie; i widzialem w zachwyceniu widzenie, naczynie niejakie zstepujace jako przescieradlo wielkie, za cztery rogi uwiazane, i spuszczone z nieba, i przyszlo az do mnie.
\par 6 W które pilnie wejrzawszy, obaczylem i widzialem czworonogie ziemskie zwierzeta i bestyje, i gadziny, i ptastwo niebieskie;
\par 7 I uslyszalem glos mówiacy do mnie: Wstan, Piotrze; rzez, a jedz.
\par 8 I rzeklem: Zadna miara, Panie! albowiem nigdy nic pospolitego albo nieczystego nie wchodzilo w usta moje.
\par 9 Tedy mi odpowiedzial po wtóre glos z nieba: Co Bóg oczyscil, ty nie miej tego za nieczyste.
\par 10 A to sie stalo po trzykroc i zasie to wszystko wciagniono do nieba.
\par 11 A oto zarazem trzej mezowie staneli przed domem, w którymem byl, poslani bedac do mnie z Cezaryi.
\par 12 I rzekl mi Duch, abym z nimi szedl, nic nie watpiac. Szli tez ze mna i ci szesc bracia, i weszlismy do domu onego meza;
\par 13 Który nam oznajmil, jako widzial Aniola w domu swym stojacego i mówiacego do siebie: Poslij meze do Joppy, a przyzwij Szymona, którego zowia Piotrem.
\par 14 On ci powie slowa, przez które zbawiony bedziesz ty i wszystek dom twój.
\par 15 A gdym ja poczal mówic, przypadl Duch Swiety na nie, jako i na nas na poczatku.
\par 16 I wspomnialem na slowo Panskie, jako byl powiedzial: Janci chrzcil woda, ale wy bedziecie ochrzczeni Duchem Swietym.
\par 17 Poniewaz im tedy Bóg dal równy dar i jako i nam, wierzacym w Pana Jezusa Chrystusa, i któzem ja byl, abym mógl zabronic Bogu?
\par 18 A to uslyszawszy, uspokoili sie i chwalili Boga mówiac; Toc tedy i poganom dal Bóg pokute ku zywotowi.
\par 19 Lecz oni, którzy byli rozproszeni przed utrapieniem, które sie stalo dla Szczepana, przeszli az do Fenicyi i Cypru, i do Antyjochii, nikomu nie opowiadajac slowa Bozego, tylko samym Zydom.
\par 20 A byli niektórzy z nich mezowie z Cypru i z Cyreny, którzy przyszedlszy do Antyjochii, mówili Grekom, opowiadajac Pana Jezusa.
\par 21 A byla z nimi reka Panska, a wielki poczet uwierzywszy, nawrócil sie do Pana.
\par 22 I przyszla o nich wiesc do uszów zboru, który byl w Jeruzalemie, i poslali Barnabasza, aby szedl az do Antyjochii.
\par 23 Który tam przyszedlszy a ujrzawszy laske Boza, uradowal sie i napominal wszystkich, aby w przedsiewzieciu serca trwali przy Panu.
\par 24 Albowiem byl maz dobry i pelen Ducha Swietego i wiary. I przybylo wielkie mnóstwo Panu.
\par 25 Potem odszedl Barnabasz do Tarsu, aby szukal Saula, a znalazlszy go, przyprowadzil go do Antyjochii.
\par 26 I bawili sie przez caly rok przy onym zborze, i uczyli mnóstwo wielkie; a najpierwej w Antyjochii uczniowie nazwani sa Chrzescijanami.
\par 27 A w one dni przyszli prorocy z Jeruzalemu do Antyjochii.
\par 28 A powstawszy jeden z nich imieniem Agabus, oznajmil przez Ducha, iz mial byc glód wielki po wszystkim okregu ziemskiem, który tez byl za Klaudyjusza cesarza.
\par 29 Tedy uczniowie, kazdy z nich wedlug przemozenia swego, postanowili poslac na wspomozenie braci, którzy mieszkali w Judzkiej ziemi.
\par 30 Co tez uczynili, poslawszy do starszych przez reke Barnabaszowa i Saulowa.

\chapter{12}

\par 1 A pod onze czas, udal sie na to Herod król, aby trapil niektóre ze zboru.
\par 2 I zabil Jakóba, brata Janowego, mieczem.
\par 3 A widzac, ze sie to podobalo Zydom, umyslil pojmac i Piotra: (a byly dni przasników).
\par 4 Którego pojmawszy, podal do wiezienia, poruczywszy go szesnastu zolnierzom, aby go strzegli, chcac go po wielkanocy wywiesc ludowi.
\par 5 Tedy strzezono Piotra w wiezieniu, a modlitwa ustawiczna dziala sie od zboru do Boga za nim.
\par 6 A gdy go juz mial wywiesc Herod, onejze nocy spal Piotr miedzy dwoma zolnierzami, zwiazany dwoma lancuchami, a stróze przed drzwiami strzegli wiezienia.
\par 7 A oto Aniol Panski przystapil, a swiatlosc sie rozswiecila w gmachu; a traciwszy w bok Piotra, obudzil go, mówiac: Wstan rychlo! I opadly lancuchy z rak jego.
\par 8 I rzekl Aniol do niego: Opasz sie, a powiaz obuwie twoje. I uczynil tak. I rzekl mu: Odziej sie w plaszcz twój, a pójdz za mna.
\par 9 Tedy wyszedlszy Piotr, szedl za nim, a nie wiedzial, ze sie to dzialo po prawdzie, co sie dzialo przez Aniola; lecz mniemal, ze widzenie widzial.
\par 10 A gdy mineli pierwsza i wtóra straz, przyszli do bramy zelaznej, która wiedzie do miasta; a ta sie im sama przez sie otworzyla. A wyszedlszy, przeszli jedne ulice, a zarazem odstapil Aniol od niego.
\par 11 Tedy Piotr przyszedlszy do siebie rzekl: Teraz znam prawdziwie, iz poslal Pan Aniola swego i wyrwal mie z reki Herodowej i ze wszystkiego oczekiwania ludu zydowskiego.
\par 12 A obaczywszy sie, przyszedl do domu Maryi, matki Janowej, którego nazywano Markiem, gdzie sie ich bylo wiele zgromadzilo i modlili sie.
\par 13 A gdy Piotr kolatal we drzwi u przysionka, wyszla dzieweczka, imieniem Rode, aby posluchala:
\par 14 A poznawszy glos Piotrowy, od radosci nie otworzyla drzwi, ale wbiezawszy oznajmila, iz Piotr stoi u drzwi.
\par 15 A oni rzekli do niej: Szalejesz! Wszakze ona twierdzila, iz sie tak rzecz ma. A oni rzekli: Aniol jego jest.
\par 16 Ale Piotr nie przestal kolatac; a gdy otworzyli, ujrzeli go i zdumieli sie.
\par 17 A skinawszy na nie reka, aby umilkneli, rozpowiedzial im, jako go Pan wywiódl z wiezienia i rzekl: Oznajmijcie to Jakóbowi i braciom. A wyszedlszy, szedl na inne miejsce.
\par 18 A gdy byl dzien, stal sie rozruch niemaly miedzy zolnierzami o to, co by sie z Piotrem stalo.
\par 19 Lecz Herod, gdy sie o nim wywiadywal, a nie znalazl go, uczyniwszy sad o strózach, kazal je na stracenie wywiesc; a wyjechawszy z Judzkiej ziemi do Cezaryi, mieszkal tam.
\par 20 A natenczas Herod myslil o wojnie przeciwko Tyryjczykom i Sydonczykom; ale oni jednomyslnie przyszli do niego, a namówiwszy Blasta, podkomorzego królewskiego, prosili o pokój, dlatego iz kraina ich miala zywnosc z dzierzawy królewskiej.
\par 21 A dnia pewnego Herod obleklszy sie w szate królewska i siadlszy na stolicy, uczynil rzecz do nich.
\par 22 A lud wolal: Glos Bozy a nie czlowieczy.
\par 23 A zarazem uderzyl go Aniol Panski, przeto, ze nie dal chwaly Bogu, a bedac roztoczony od robactwa, zdechl.
\par 24 A slowo Panskie rozrastalo sie i rozmnazalo.
\par 25 A Barnabasz i Saul wrócili sie z Jeruzalemu, wykonawszy posluge, wziawszy z soba Jana, którego nazywano Markiem.

\chapter{13}

\par 1 A byli w Antyjochii we zborze, który tam byl, niektórzy prorocy i nauczyciele, jako Barnabasz i Symeon, którego zwano Niger, i Lucyjus Cyrenejczyk, i Manahen, który byl wychowany z Herodem Tetrarcha, i Saul.
\par 2 A gdy oni sluzbe Panska jawnie odprawiali i poscili, rzekl im Duch Swiety: Odlaczcie mi Barnabasza i Saula do tej sprawy, do którejm ich powolal.
\par 3 Tedy poszczac i modlac sie, i wkladajac na nie rece, odprawili je.
\par 4 Oni tedy wyslani bedac od Ducha Swietego, przyszli do Seleucyi, a stamtad plyneli do Cypru.
\par 5 A gdy byli w Salaminie, opowiadali slowo Boze w bóznicach zydowskich, a mieli z soba i Jana do uslugi.
\par 6 A przeszedlszy one wyspe az do Pafu, znalezli tam jakiegos czarnoksieznika, falszywego proroka, Zyda, któremu imie bylo Barjezus.
\par 7 Który byl przy zacnym staroscie, Sergijuszu Pawle, mezu roztropnym. Ten przyzwawszy Barnabasza i Saula, pragnal sluchac slowa Bozego.
\par 8 Lecz sie im sprzeciwil Elimas, on czarnoksieznik, (albowiem sie tak wyklada imie jego), starajac sie, jakoby staroste od wiary odwrócil:
\par 9 Tedy Saul, (którego zowia i Pawlem) napelniony bedac Ducha Swietego, a pilnie na niego patrzac,
\par 10 Rzekl: O pelny wszelkiej zdrady i wszelkiej przewrotnosci, synu dyjabelski, nieprzyjacielu wszelkiej sprawiedliwosci! nie przestanieszze podwracac prostych dróg Panskich?
\par 11 A oto teraz reka Panska nad toba: i bedziesz slepym, nie widzac slonca az do czasu. A zarazem przypadla na niego chmura i ciemnosc, a blakajac sie szukal, kto by go wiódl za reke.
\par 12 Tedy widzac starosta, co sie stalo, uwierzyl, zdumiewajac sie nad nauka Panska.
\par 13 A pusciwszy sie z Pafu Pawel i ci, którzy z nim byli, przyszli do Pergi Pamfiliejskiej. A Jan odszedlszy od nich, wrócil sie do Jeruzalemu.
\par 14 A oni odszedlszy z Pergi, przyszli do Antyjochii Pisydejskiej, a wszedlszy do bóznicy w dzien sobotni, usiedli.
\par 15 A po przeczytaniu zakonu i proroków, poslali do nich przelozeni bóznicy, mówiac: Mezowie bracia! macieli wole jakie napominanie uczynic do ludu, mówcie.
\par 16 Tedy powstawszy Pawel, a reka skinawszy rzekl: Mezowie Izraelscy i którzy sie boicie Boga! sluchajcie:
\par 17 Bóg ludu tego Izraelskiego wybral ojców naszych i wywyzszyl lud, gdy byli przychodniami w ziemi egipskiej, i w ramieniu wyciagnionem wywiódl je z niej.
\par 18 I przez czas czterdziestu lat znosil obyczaje ich na puszczy.
\par 19 A wygladziwszy siedm narodów w ziemi Chananejskiej, losem rozdzielil miedzy nie one ziemie ich.
\par 20 A potem okolo czterysta i piecdziesiat lat dawal im sedziów, az do Samuela proroka.
\par 21 A od onego czasu prosili o króla. I dal im Bóg Saula, syna Cysowego, meza z pokolenia Benjaminowego, przez lat czterdziesci.
\par 22 A gdy go odrzucil, wzbudzil im Dawida za króla, któremu tez swiadectwo wydawajac powiedzial: Znalazlem Dawida, syna Jessego, meza wedlug serca mego, który bedzie czynil wszystke wole moje.
\par 23 Z jegoz nasienia Bóg wedlug obietnicy wzbudzil Izraelowi zbawiciela Jezusa.
\par 24 Przed którego przyjsciem kazal Jan chrzest pokuty wszystkiemu ludowi Izraelskiemu.
\par 25 A gdy Jan dokonal biegu swego, rzekl: Kim mie byc mniemacie? Nie jestem ja, ale oto idzie za mna, u którego nóg obuwia nie jestem godzien rozwiazac.
\par 26 Mezowie bracia, synowie narodu Abrahamowego i którzy sie miedzy wami Boga boja! wamci slowo zbawienia tego poslane jest.
\par 27 Albowiem ci, co mieszkaja w Jeruzalemie i przelozeni ich, nie znajac tego Jezusa i glosów prorockich, które przez kazdy sabat bywaja czytane, wypelnili je, osadziwszy go.
\par 28 A zadnej przyczyny smierci w nim nie znalazlszy, prosili Pilata, aby byl zabity.
\par 29 A gdy wykonali wszystko, co o nim bylo napisane, zdjawszy go z drzewa, wlozyli go do grobu.
\par 30 Ale go Bóg wzbudzil od umarlych.
\par 31 Który widziany jest przez wiele dni od tych, którzy z nim pospolu przyszli z Galilei do Jeruzalemu, którzy sa swiadkami jego przed ludem.
\par 32 I my wam opowiadamy te obietnice, która sie ojcom stala, iz ja Bóg wypelnil nam, dziatkom ich, wzbudziwszy Jezusa.
\par 33 Jako tez w Psalmie wtórym napisane jest: Syn mój jestes ty, jam ciebie dzis splodzil.
\par 34 A iz go wzbudzil od umarlych, aby sie wiecej nie wrócil do skazenia, tak powiedzial: Dam wam swiete dobrodziejstwa Dawidowe wierne.
\par 35 Przeto i indziej powiada: Nie dasz Swietemu twemu widziec skazenia.
\par 36 Albowiemci Dawid za wieku swego usluzywszy woli Bozej, zasnal i przylaczony jest do ojców swoich, a widzial skazenie.
\par 37 Lecz ten, którego Bóg wzbudzil, nie widzial skazenia.
\par 38 Niechze wam tedy bedzie wiadomo, mezowie bracia, iz sie wam przez tego opowiada odpuszczenie grzechów:
\par 39 I od wszystkiego od czegoscie nie mogli byc przez zakon Mojzeszowy usprawiedliwieni, przez tego kazdy wierzacy usprawiedliwiony bywa.
\par 40 A przetoz patrzcie, aby na was nie przyszlo to, co powiedziano w prorokach:
\par 41 Obaczcie wy wzgardziciele i dziwujcie sie, a wniwecz sie obróccie; bo ja sprawuje sprawe za dni waszych, sprawe, której nie wierzycie, chocby wam kto o niej powiadal.
\par 42 A gdy oni wychodzili z bóznicy zydowskiej, prosili ich poganie, aby i w drugi sabat mówili do nich tez slowa.
\par 43 A po rozpuszczeniu zgromadzenia, poszlo wiele Zydów i naboznych nowowierników za Pawlem i Barnabaszem, którzy mówiac do nich, radzili im, aby trwali w lasce Bozej.
\par 44 A w drugi sabat niemal wszystko miasto sie zgromadzilo na sluchanie slowa Bozego.
\par 45 Tedy Zydowie widzac lud, napelnieni sa zazdroscia i sprzeciwiali sie temu, co Pawel powiadal, mówiac przeciwko temu i bluzniac.
\par 46 A Pawel i Barnabasz, bezpiecznie mówiac, rzekli: Wamci najpierwej mialo byc opowiadane slowo Boze; ale poniewaz je odrzucacie, a sadzicie sie byc niegodnymi zywota wiecznego, oto sie obracamy do pogan.
\par 47 Albowiem nam tak rozkazal Pan, mówiac: Polozylem cie swiatloscia poganom, abys byl zbawieniem az do krajów ziemi.
\par 48 A slyszac to poganie, radowali sie i wielbili slowo Panskie, i uwierzyli, ilekolwiek ich bylo sporzadzonych do zywota wiecznego.
\par 49 I roznosilo sie slowo Panskie po wszystkiej onej krainie.
\par 50 A Zydowie poduszczali niewiasty nabozne i uczciwe, i przedniejsze w miescie; a wzbudzili przesladowanie przeciwko Pawlowi i przeciwko Barnabaszowi, i wygnali je z granic swoich.
\par 51 A oni otrzasnawszy proch z nóg swoich na nie, przyszli do Ikonii.
\par 52 A uczniowie byli napelnieni radosci i Ducha Swietego.

\chapter{14}

\par 1 I stalo sie w Ikonii, ze takze weszli do bóznicy zydowskiej, a tak mówili, ze uwierzylo i Zydów, i Greków wielkie mnóstwo.
\par 2 Lecz Zydowie, którzy nie uwierzyli, podburzyli i zajatrzyli serca pogan przeciwko braciom.
\par 3 I byli tam przez dlugi czas, bezpiecznie mówiac w Panu, który dawal swiadectwo slowu laski swojej i czynil to, ze sie dzialy znamiona i cuda przez rece ich.
\par 4 I rozerwalo sie mnóstwo miejskie, a byli jedni z Zydami a drudzy z Apostolami.
\par 5 A gdy sie wzburzyli i poganie, i Zydzi z ksiazety swoimi, aby je zelzyli i ukamionowali:
\par 6 Zrozumiawszy to, uciekli do miast Likaonskich, do Listry i do Derby, i do okolicznej krainy,
\par 7 A tam kazali Ewangielije.
\par 8 A maz niektóry w Listrze chory na nogi siedzial, bedac chromy z zywota matki swojej, który nigdy nie chodzil.
\par 9 Ten sluchal Pawla mówiacego; który nan pilnie patrzac i widzac, iz mial wiare, zeby mógl byc uzdrowiony,
\par 10 Rzekl wielkim glosem: Stan prosto na nogi twoje; i wyskoczyl i chodzil.
\par 11 A lud widzac, co Pawel uczynil, podniesli glos swój, mówiac po likaonsku: Bogowie stawszy sie podobni ludziom, zstapili do nas.
\par 12 I nazwali Barnabasza Jowiszem, a Pawla Merkuryjuszem, poniewaz on prowadzil rzecz.
\par 13 Tedy kaplan Jowisza, który byl przed miastem ich, woly z wiencami do wrót przywiódlszy, chcial ofiary z ludem sprawowac.
\par 14 Co gdy uslyszeli Apostolowie Barnabasz i Pawel, rozdarlszy szaty swoje, wpadli miedzy lud, wolajac,
\par 15 I mówiac: Mezowie! cóz to czynicie? I mysmyc ludzie, tymze biedom jako i wy poddani, którzy wam opowiadamy, abyscie sie od tych marnosci nawrócili do Boga zywego, który uczynil niebo i ziemie i morze, i wszystko co w nich jest.
\par 16 Który za przeszlych wieków dopuszczal wszystkim poganom, aby chodzili za drogami swemi.
\par 17 Aczkolwiek nie zaniechal samego siebie prócz swiadectwa, czyniac dobrze, dawajac nam z nieba dzdze i czasy urodzajne, napelniajac pokarmem i weselem serca nasze.
\par 18 A to mówiac, zaledwie uspokoili lud, ze im nie ofiarowal.
\par 19 A nadeszli z Antyjochyi i z Ikonii Zydowie, którzy namówiwszy lud i ukamionowawszy Pawla, wywlekli za miasto, mniemajac zeby umarl.
\par 20 Lecz gdy go uczniowie obstapili, wstawszy wszedl do miasta, a nazajutrz odszedl z Barnabaszem do Derby.
\par 21 A opowiedziawszy Ewangielije onemu miastu i wiele uczniów pozyskawszy, wrócili sie do Listry, do Ikonii, i do Antyjochyi;
\par 22 Utwierdzajac dusze uczniów i napominajac, aby trwali w wierze, i mówiac: Ze przez wiele ucisków musimy wnijsc do królestwa Bozego.
\par 23 A gdy im przez glosy postanowili starsze w kazdym zborze i modlili sie z postami, poruczyli je Panu, w którego uwierzyli.
\par 24 A przeszedlszy Pisydyje, przyszli do Pamfilii.
\par 25 I opowiedziawszy slowo Boze w Pergi, poszli do Atalii.
\par 26 A stamtad plyneli do Antyjochyi, skad byli oddani lasce Bozej ku tej sprawie, która wykonali.
\par 27 A gdy tam przyszli i zgromadzili zbór, oznajmili, co Bóg przez nie uczynil, a iz poganom drzwi wiary otworzyl.
\par 28 I mieszkali tam czas niemaly z uczniami.

\chapter{15}

\par 1 A niektórzy przyszedlszy z Judzkiej ziemi, nauczali braci: Iz jezli sie nie obrzezecie wedlug zwyczaju Mojzeszowego, nie mozecie byc zbawieni.
\par 2 A gdy róznice i spór niemaly Pawel i Barnabasz mieli z nimi, postanowili, aby Pawel i Barnabasz i niektórzy inni z nich szli do Apostolów i do starszych do Jeruzalemu, z strony tego sporu.
\par 3 Oni tedy bedac odprowadzeni od zboru, szli przez Fenicyje i Samaryje, powiadajac o nawróceniu poganów i uczynili wielka radosc wszystkim braciom.
\par 4 A gdy przyszli do Jeruzalemu, przyjeci byli od zboru i od Apostolów, i starszych, i opowiedzieli, cokolwiek Bóg przez nie czynil.
\par 5 Ale powstali niektórzy z sekty Faryzeuszów, którzy byli uwierzyli, mówiac: Ze ich trzeba obrzezac i rozkazac im, zeby zachowali zakon Mojzeszowy.
\par 6 Zgromadzili sie tedy Apostolowie i starsi, aby wejrzeli w te sprawe.
\par 7 A gdy byl wielki spór o tem, powstawszy Piotr, rzekl do nich: Mezowie bracia! wy wiecie, ze od dawnych dni Bóg mie obral miedzy wami, aby przez usta moje poganie sluchali slowa Ewangielii i uwierzyli.
\par 8 A Bóg, który zna serca, wydal im swiadectwo, dawszy im Ducha Swietego, jako i nam.
\par 9 I nie uczynil zadnej róznicy miedzy nami a nimi, wiara oczysciwszy serca ich.
\par 10 Przetoz teraz, przecz kusicie Boga, kladac jarzmo na szyje uczniów, którego ani ojcowie nasi, ani mysmy znosic nie mogli?
\par 11 Ale przez laske Pana Jezusa Chrystusa wierzymy, iz bedziemy zbawieni tym sposobem, jako i oni.
\par 12 I milczalo wszystko ono mnóstwo, a sluchali Barnabasza i Pawla, którzy opowiadali, jako wielkie znamiona i cuda czynil Bóg przez nie miedzy pogany.
\par 13 A gdy oni umilkneli, odpowiedzial Jakób, mówiac: Mezowie bracia! sluchajcie mie.
\par 14 Szymon powiedzial, jako Bóg najpierwej wejrzal na pogany, aby z nich wzial lud imieniowi swemu.
\par 15 A z tem sie zgadzaja mowy prorockie, jako jest napisano:
\par 16 Potem sie wróce, a pobuduje zasie przybytek Dawidowy upadly, a obaliny jego zasie pobuduje i znowu go wystawie,
\par 17 Aby ci, co pozostali z ludzi, szukali Pana i wszyscy narodowie, nad którymi wzywano imienia mojego, mówi Pan, który to wszystko czyni.
\par 18 Znajomec sa Bogu od wieku wszystkie sprawy jego.
\par 19 Przetoz moje zdanie jest, zeby nie trwozyc tych, którzy sie z poganów do Boga nawracaja.
\par 20 Ale raczej pisac do nich, aby sie wstrzymywali od splugawienia balwanów i od wszeteczenstwa, i od rzeczy dlawionych, i ode krwi.
\par 21 Albowiem Mojzesz od dawnych wieków ma w kazdym miescie te, którzy go opowiadaja, gdyz go w bóznicach na kazdy sabat czytaja.
\par 22 Tedy sie zdalo Apostolom i starszym ze wszystkim zborem, aby wybrane sposród siebie meze poslali do Antyjochyi z Pawlem i z Barnabaszem, to jest Judasa, którego zwano Barsabaszem, i Syle, meze przedniejsze miedzy bracmi.
\par 23 Napisawszy to przez reke ich: Apostolowie i starsi, i bracia tym, którzy sa w Antyjochyi i w Syryi, i w Cylicyi, braciom którzy sa z pogan, zdrowia zyczymy;
\par 24 Poniewazesmy slyszeli, ze niektórzy wyszedlszy od nas, zatrwozyli was slowy, watlac dusze wasze, a mówiac, ze sie musicie obrzezac i zakon zachowywac, którymesmy tego nie poruczyli,
\par 25 Zdalo sie nam jednomyslnie zgromadzonym, poslac do was meze wybrane z milymi naszymi, Barnabaszem i z Pawlem,
\par 26 Z ludzmi, którzy wydali dusze swe dla imienia Pana naszego, Jezusa Chrystusa.
\par 27 Przetoz poslalismy Judasa i Syle, którzy wam i ustnie toz powiedza.
\par 28 Albowiem zdalo sie Duchowi Swietemu i nam, abysmy wiecej nie kladli na was zadnego ciezaru, oprócz tych rzeczy potrzebnych;
\par 29 Abyscie sie wstrzymywali od rzeczy balwanom ofiarowanych, i od krwi, i od rzeczy dlawionych, i od wszeteczenstwa, których rzeczy jezli sie strzec bedziecie, dobrze uczynicie. Miejcie sie dobrze.
\par 30 A tak oni bedac odprawieni, przyszli do Antyjochyi, a zgromadziwszy mnóstwo, oddali list.
\par 31 A przeczytawszy, radowali sie z onej pociechy.
\par 32 A Judas i Sylas, bedac i oni prorokami, dlugiemi slowy napominali braci i utwierdzali je.
\par 33 A zamieszkawszy tam do czasu, odprawieni sa z pokojem od braci do Apostolów.
\par 34 Lecz Syli zdalo sie tam zostac.
\par 35 Takze Pawel i Barnabasz zamieszkali w Antyjochyi, nauczajac i opowiadajac z wieloma innymi slowo Panskie.
\par 36 A po kilku dniach rzekl Pawel do Barnabasza: Wróciwszy sie, nawiedzmy braci naszych po wszystkich miastach, w którychesmy opowiadali slowo Panskie, jakoli sie maja.
\par 37 Tedy Barnabasz radzil, aby z soba wzieli i Jana, którego zwano Markiem.
\par 38 Ale sie to Pawlowi nie zdalo brac tego z soba, który byl odszedl od nich z Pamfilii, a nie chodzil z nimi na one prace.
\par 39 I wszczal sie miedzy nimi wielki gniew, tak iz odszedl jeden od drugiego, a Barnabasz wziawszy z soba Marka, plynal do Cypru.
\par 40 Ale Pawel obrawszy sobie Syle, wyszedl, bedac poruczony lasce Bozej od braci:
\par 41 I przechodzil Syryje, i Cilicyje, utwierdzajac zbory.

\chapter{16}

\par 1 I przyszedl do Derby i do Listry; a oto tam byl uczen niektóry, imieniem Tymoteusz, syn niektórej niewiasty Zydówki wiernej a ojca Greka.
\par 2 Temu swiadectwo dawali bracia, którzy byli w Listrze i w Ikonii.
\par 3 Chcial tedy Pawel, aby ten z nim szedl, którego wziawszy, obrzezal dla Zydów, którzy byli na onych miejscach; bo wszyscy wiedzieli, ze ojciec jego byl Grekiem.
\par 4 A gdy chodzili po miastach, podawali im ku chowaniu ustawy, które byly postanowione od Apostolów i starszych, którzy byli w Jeruzalemie.
\par 5 A tak sie zbory utwierdzaly w wierze i przybywalo ich w liczbie na kazdy dzien.
\par 6 Tedy przeszedlszy Frygije i Galatska kraine, zawsciagnieni bedac od Ducha Swietego, aby nie opowiadali slowa Bozego w Azyi,
\par 7 Przyszedlszy do Mizyi, kusili sie isc do Bitynii, ale im Duch Jezusowy nie dopuscil.
\par 8 Tedy minawszy Mizyje, zstapili do Troady.
\par 9 I pokazalo sie Pawlowi w nocy widzenie: Maz niejaki Macedonczyk stal, proszac go i mówiac: Przepraw sie do Macedonii, a ratuj nas.
\par 10 A ujrzawszy to widzenie, zarazesmy sie starali o to, jakobysmy sie puscili do Macedonii, bedac tego pewni, iz nas Pan powolal, abysmy im kazali Ewangielije.
\par 11 Pusciwszy sie tedy z Troady, prostosmy biezeli do Samotracyi, a nazajutrz do Neapolu.
\par 12 A stamtad do Filipowa, które jest pierwsze miasto tej czesci Macedonii nowo osadzone; i zostalismy w onem miescie przez kilka dni.
\par 13 A w dzien sabatu wyszlismy przed miasto nad rzeke, gdzie zwykly bywac modlitwy, a usiadlszy mówilismy do niewiast, które sie tam byly zeszly.
\par 14 A niektóra niewiasta, imieniem Lidyja, która szarlat sprzedawala w miescie Tyjatyrskiem, Boga sie bojaca, sluchala; której Pan otworzyl serce, aby pilnie sluchala tego, co Pawel mówil.
\par 15 A gdy sie ochrzcila i dom jej, prosila, mówiac: Poniewazescie mie osadzili wierna byc Panu, wszedlszy do domu mego, mieszkajcie; i przymusila nas.
\par 16 I stalo sie, gdysmy szli na modlitwe, iz niektóra dzieweczka, co miala ducha wieszczego, zabiezala nam, a ta wielki zysk panom swoim przynosila, wrózac.
\par 17 Ta chodzac za Pawlem i za nami, wolala mówiac: Ci ludzie slugami sa Boga najwyzszego, którzy nam opowiadaja droge zbawienia.
\par 18 A to czynila przez wiele dni; ale Pawel bolejac nad tem i obróciwszy sie, rzekl onemu duchowi: Rozkazuje ci w imieniu Jezusa Chrystusa, abys wyszedl od niej. I wyszedl onejze godziny.
\par 19 A widzac panowie jej, iz zginela nadzieja zysku ich, pojmawszy Pawla i Syle, ciagneli je na rynek przed urzad,
\par 20 A stawiwszy je przed hetmany, rzekli: Ci ludzie czynia zamieszanie w miescie naszem, bedac Zydami:
\par 21 I opowiadaja zwyczaje, których sie nam nie godzi przyjmowac ani zachowywac, poniewaz jestesmy Rzymianie.
\par 22 I powstalo pospólstwo przeciwko nim, a hetmani rozdarlszy szaty ich, kazali je siec rózgami.
\par 23 A gdy im wiele plag zadali, wrzucili je do wiezienia przykazawszy strózowi wiezienia, aby ich dobrze opatrzyl.
\par 24 Który wziawszy takie rozkazanie, wsadzil je do najglebszego wiezienia, a nogi ich zamknal w klode.
\par 25 A o pólnocy Pawel i Sylas modlac sie, chwalili Boga piesniami, tak ze je slyszeli wiezniowie.
\par 26 I powstalo z predka wielkie trzesienie ziemi, ze sie poruszyly grunty wiezienia, i zarazem sie otworzyly wszystkie drzwi, i wszystkich sie zwiazki rozwiazaly.
\par 27 A ocuciwszy sie stróz wiezienia i ujrzawszy otworzone drzwi u wiezienia, dobyl miecza, chcac sie sam zabic, mniemajac, iz wiezniowie pouciekali.
\par 28 Lecz Pawel zawolal glosem wielkim, mówiac: Nie czyn sobie nic zlego: bo jestesmy sami wszyscy.
\par 29 A kazawszy zaswiecic, wpadl tam, a drzac przypadl do nóg Pawlowi i Syli:
\par 30 A wywiódlszy je z wiezienia, rzekl: Panowie! co mam czynic, abym byl zbawiony?
\par 31 A oni rzekli: Wierz w Pana Jezusa Chrystusa, a bedziesz zbawiony, ty i dom twój.
\par 32 I opowiadali mu slowo Panskie i wszystkim, którzy byli w domu jego.
\par 33 A wziawszy je onejze godziny w nocy, omyl rany ich i ochrzcil sie zaraz, on i wszyscy domownicy jego.
\par 34 A wprowadziwszy je do domu swego, nagotowal im stól i weselil sie ze wszystkim domem swoim, uwierzywszy Bogu.
\par 35 A gdy byl dzien, poslali hetmani slugi miejskie, mówiac: Wypusc one ludzie.
\par 36 I oznajmil stróz wiezienia te slowa Pawlowi, iz hetmani poslali, abyscie byli wypuszczeni: teraz tedy wyszedlszy, idzcie w pokoju.
\par 37 Ale im Pawel rzekl: Usieklszy nas jawnie rózgami nie przekonanych, gdyzesmy sa ludzie Rzymianie, wrzucili do wiezienia; a teraz nas potajemnie wyganiaja? Nic z tego; ale sami niech przyjda i wyprowadza nas.
\par 38 Tedy powiedzieli hetmanom sludzy miejscy te slowa. I zlekli sie, uslyszawszy, ze byli Rzymianie,
\par 39 A przyszedlszy, przeprosili ich, a wywiódlszy ich, prosili ich, aby wyszli z miasta.
\par 40 Wyszedlszy tedy z wiezienia, weszli do Lidyi, a ujrzawszy braci pocieszyli je i odeszli.

\chapter{17}

\par 1 A przeszedlszy Amfipolim i Apolonije przyszli do Tesaloniki, gdzie byla bóznica zydowska.
\par 2 Tedy Pawel wedlug zwyczaju swego wszedl do nich, a przez trzy sabaty kazal im z Pisma.
\par 3 Wywodzac i pokazujac to, ze Chrystus mial cierpiec i powstac od umarlych, a iz ten Jezus jest Chrystusem, którego ja wam opowiadam.
\par 4 I uwierzyli niektórzy z nich, a przylaczyli sie do Pawla i do Syli, i wielkie mnóstwo naboznych Greków, i niewiast przedniejszych niemalo.
\par 5 Ale Zydowie, którzy nie uwierzyli, zdjeci zazdroscia, przywziawszy do siebie niektórych lekkomyslnych i zlych mezów, a zebrawszy kupe uczynili rozruch w miescie, a naszedlszy na dom Jazona, szukali ich, aby ich wywiedli przed lud.
\par 6 A nie znalazlszy ich, ciagneli Jazona i niektórych braci do przelozonych miasta, wolajac: Oto ci, którzy wszystek swiat wzruszyli i tu tez przyszli;
\par 7 Które przyjal Jazon; a ci wszyscy czynia przeciwko dekretom cesarskim, powiadajac, iz jest inszy król, Jezus.
\par 8 A tak wzburzyli pospólstwo i przelozonych miasta, którzy to slyszeli.
\par 9 Ale oni wziawszy sluszna sprawe od Jazona i od innych, puscili je.
\par 10 A bracia wnet w nocy wyslali i Pawla, i Syle do Berei; którzy tam przyszedlszy weszli do bóznicy zydowskiej.
\par 11 A cic byli zacniejsi nad one, co byli w Tesalonice, którzy przyjeli slowo Boze ze wszystka ochota, na kazdy dzien rozsadzajac Pisma, jezliby sie tak mialo.
\par 12 Przetoz wiele ich z nich uwierzylo, i Greckich niewiast uczciwych, i mezów niemalo.
\par 13 A gdy sie dowiedzieli oni, co byli z Tesaloniki Zydowie, ze i w Berei opowiadane bylo slowo Boze od Pawla, przyszli i tam, podburzajac pospólstwo.
\par 14 Ale bracia wnet wyslali Pawla, aby szedl jakoby do morza; a Sylas i Tymoteusz tam zostali.
\par 15 A ci, którzy prowadzili Pawla, doprowadzili go az do Aten, a wziawszy rozkazanie do Syli i do Tymoteusza, zeby co najrychlej przyszli do niego, odeszli.
\par 16 A gdy ich Pawel w Atenach czekal, poruszal sie w nim duch jego, widzac ono miasto poddane balwochwalstwu.
\par 17 A przetoz miewal rozmowe z Zydami i z ludzmi naboznymi, w bóznicy i na rynku na kazdy dzien, z kim sie mu trafilo.
\par 18 Tedy niektórzy z Epikurejczyków i Stoików filozofowie spierali sie z nim, a niektórzy mówili: Cóz wzdy ten plotkarz mówic chce? A drudzy: Zdaje sie byc opowiadaczem obcych bogów; bo im Jezusa i zmartwychwstanie opowiadal.
\par 19 A porwawszy go, wiedli do Areopagu, mówiac: Mozemyli wiedziec, co to jest za nowa nauka, która ty opowiadasz?
\par 20 Bo jakies obce rzeczy przynosisz do uszów naszych; chcemy tedy wiedziec, co wzdy z tego ma byc?
\par 21 (A wszyscy Atenczycy i cudzoziemscy goscie niczem inszem sie nie bawili, tylko powiadaniem albo sluchaniem nowin.)
\par 22 Tedy Pawel stanawszy w posrodku Areopagu, rzekl: Mezowie Atenscy! z kazdej miary was widze nader naboznych.
\par 23 Albowiem przechadzajac sie i przypatrujac waszym nabozenstwom, znalazlem tez oltarz, na którym napisano: Nieznajomemu Bogu. Którego tedy nie znajac chwalicie, tego ja wam opowiadam.
\par 24 Bo Bóg, który uczynil swiat i wszystko, co na nim, ten bedac Panem nieba i ziemi, nie mieszka w kosciolach reka uczynionych.
\par 25 Ani rekoma ludzkiemi chwalony bywa, jakoby czego potrzebowal, poniewaz on daje wszystkim zywot i oddech, i wszystko.
\par 26 I uczynil z jednej krwi wszystek naród ludzki, aby mieszkal po wszystkiem obliczu ziemi, zamierzywszy przedtem rozrzadzone czasy i zamierzone granice mieszkania ich;
\par 27 Aby szukali Pana, owaby go snac namacali i znalezli, aczkolwiek od kazdego z nas nie jest daleko.
\par 28 Albowiem w nim zyjemy i ruszamy sie, i jestesmy, jako i niektórzy z waszych poetów powiedzieli: Zesmy i my rodzina jego.
\par 29 Bedac tedy rodzina Boza, nie mamy rozumiec, zeby zlotu albo srebru, albo kamieniowi misternie rytemu, albo wymyslowi czlowieczemu, Bóg mial byc podobny.
\par 30 Aczkolwiek tedy przegladal Bóg czasom tej niewiadomosci, ale teraz oznajmuje ludziom wszystkim wszedy, aby pokutowali;
\par 31 Przeto iz postanowil dzien, w który bedzie sadzil wszystek swiat w sprawiedliwosci przez meza, którego na to naznaczyl, upewniajac o tem wszystkich, wzbudziwszy go od umarlych.
\par 32 A uslyszawszy o zmartwychwstaniu jedni sie nasmiewali, a drudzy mówili: Bedziemy cie znowu o tem sluchac.
\par 33 I tak Pawel wyszedl z posrodku ich.
\par 34 A mezowie niektórzy przylaczywszy sie do niego, uwierzyli, miedzy którymi tez byl Dyjonizyjusz Areopagitczyk i niewiasta imieniem Damarys, i inni z nimi.

\chapter{18}

\par 1 Potem Pawel odszedlszy z Aten, przyszedl do Koryntu;
\par 2 A znalazlszy niektórego Zyda, imieniem Akwilas, rodem z Pontu, który byl swiezo z Wloch przyszedl z Pryscylla, zona swa, (dlatego, iz byl Klaudyjusz postanowil, aby wszyscy Zydowie z Rzymu wyszli), przyszedl do nich;
\par 3 A iz byl tegoz rzemiosla, mieszkal u nich i robil; albowiem rzemioslo ich bylo robic namioty.
\par 4 Tedy miewal rozmowe w bóznicy na kazdy sabat i pozyskiwal i Zydy, i Greki.
\par 5 A gdy przyszli z Macedonii Sylas i Tymoteusz, scisniony byl w duchu Pawel, oswiadczajac Zydom, ze Jezus jest Chrystusem.
\par 6 Lecz gdy sie oni sprzeciwiali i bluznili, otrzasnawszy proch z szat, rzekl do nich: Krew wasza na glowe wasze; jam jest czysty, od tego czasu pójde do pogan.
\par 7 A odszedlszy stamtad wszedl do domu niejakiego czlowieka, imieniem Justus, sluzacego Bogu, którego dom byl podle samej bóznicy.
\par 8 Lecz Kryspus, przelozony bóznicy, uwierzyl Panu ze wszystkim domem swoim, i wiele z Koryntczyków sluchajac, uwierzyli i ochrzczeni sa.
\par 9 Zatem Pan rzekl Pawlowi w nocy w widzeniu: Nie bój sie, ale mów, a nie milcz.
\par 10 Bom ja jest z toba, a zaden sie na cie nie targnie, abyc mial co zlego uczynic; albowiem ja wielki lud mam w tem miescie.
\par 11 I mieszkal tam rok i szesc miesiecy, nauczajac u nich slowa Bozego.
\par 12 A gdy Galijo byl starosta w Achai, powstali jednomyslnie Zydowie przeciwko Pawlowi i przywiedli go do sadu, mówiac:
\par 13 Ten namawia ludzi, aby przeciwko zakonowi Boga chwalili.
\par 14 A gdy Pawel mial usta otworzyc, rzekl Galijo do Zydów: O Zydowie! gdyby sie wam bylo jakie bezprawie stalo, albo jaka krzywda, slusznie bym was znosil;
\par 15 Lecz jezli jest jaka gadka o slowach i o imionach i o zakonie waszym, sami tego patrzcie; albowiem ja tego sedzia byc nie chce.
\par 16 I odegnal je od sadowej stolicy.
\par 17 Tedy porwawszy wszyscy Grekowie Sostena, przelozonego bóznicy, bili go przed sadowa stolica, a Galijo na to nic nie dbal.
\par 18 A Pawel pomieszkawszy tam jeszcze przez niemalo dni, pozegnawszy sie z bracmi, plynal do Syryi, a z nim Pryscylla i Akwilas, ogoliwszy glowe w Kienchreach: bo byl uczynil slub.
\par 19 Zatem przyszedl do Efezu i tam je zostawil, a sam wszedlszy do bóznicy, mial rozmowe z Zydami.
\par 20 A gdy go oni prosili, aby u nich przez dluzszy czas zamieszkal, nie zezwolil;
\par 21 Ale sie z nimi pozegnawszy, rzekl: Koniecznie ja musze swieto nadchodzace w Jeruzalemie obchodzic; lecz sie zasie do was wróce, bedzieli wola Boza. I puscil sie z Efezu.
\par 22 A gdy przyszedl do Cezaryi, wstapiwszy do Jeruzalemu a pozdrowiwszy zbór, szedl do Antyjochyi.
\par 23 I zamieszkawszy tam przez niektóry czas, wyszedl obchodzac kraine Galatska i Frygije, utwierdzajac wszystkich uczniów.
\par 24 A Zyd niektóry imieniem Apollos, rodem z Aleksandryi, maz wymowny, przyszedl do Efezu, bedac moznym w Pismach.
\par 25 Ten byl wprawiony w droge Panska, a palajac w duchu, mówil i nauczal pilnie o Panu, wiedzac tylko o chrzcie Janowym.
\par 26 Ten poczal bezpiecznie mówic w bóznicy. Którego uslyszawszy Akwilas i Pryscylla, przyjeli go do siebie i dostateczniej mu wylozyli droge Boza.
\par 27 A gdy chcial isc do Achai, napomniawszy go bracia, pisali do uczniów, aby go przyjeli; który gdy tam przyszedl, wiele pomagal tym, którzy uwierzyli z laski Bozej.
\par 28 Albowiem poteznie Zydy przekonywal, jawnie tego dowodzac z Pisma, iz Jezus jest Chrystusem.

\chapter{19}

\par 1 I stalo sie, gdy Apollos byl w Koryncie, iz Pawel obszedlszy górne krainy, przyszedl do Efezu; a znalazlszy tam niektórych uczniów,
\par 2 Rzekl do nich: Izaliscie wzieli Ducha Swietego, uwierzywszy? A oni mu rzekli: Owszemesmy ani slyszeli, jezli jest Duch Swiety.
\par 3 Tedy rzekl do nich: W cózescie tedy ochrzczeni? A oni rzekli: W chrzest Janowy.
\par 4 Zatem rzekl Pawel: Janci chrzcil chrztem pokuty, mówiac ludowi, aby w onego, który mial przyjsc po nim, uwierzyli, to jest w Jezusa Chrystusa.
\par 5 A uslyszawszy to, ochrzczeni sa w imie Pana Jezusowe.
\par 6 A gdy na nie wlozyl Pawel rece, zstapil na nie Duch Swiety i mówili jezykami i prorokowali.
\par 7 A bylo wszystkich mezów okolo dwunastu.
\par 8 A wszedlszy do bóznicy, mówil bezpiecznie przez trzy miesiace, nauczajac i namawiajac ich do królestwa Bozego.
\par 9 A gdy sie niektórzy zatwardzili, a wierzyc nie chcieli, zle mówiac o tej drodze Bozej przed mnóstwem, odstapiwszy od nich, odlaczyl ucznie, na kazdy dzien uczac w szkole niektórego Tyranna.
\par 10 A to sie dzialo przez dwa lata, tak iz wszyscy, którzy mieszkali w Azyi, sluchali slowa Pana Jezusowego, tak Zydowie, jako i Grekowie.
\par 11 A nie lada cuda czynil Bóg przez rece Pawlowe;
\par 12 Tak iz na chore przynoszono chustki albo przepaski od ciala jego, i odchodzily od nich choroby, i duchowie zli wychodzili z nich.
\par 13 Tedy niektórzy z biegunów zydowskich, którzy sie bawili zaklinaniem, wazyli sie wzywac imienia Pana Jezusowego nad tymi, którzy mieli duchy zle, mówiac: Poprzysiegamy was przez Jezusa, którego Pawel opowiada.
\par 14 A bylo ich siedm synów jednego Zyda, imieniem Scewas, najwyzszego kaplana, którzy to czynili.
\par 15 Tedy odpowiedziawszy duch zly, rzekl: Znam Jezusa i wiem co Pawel; ale wy coscie zacz?
\par 16 A rzuciwszy sie na nie czlowiek on, w którym byl duch zly, a opanowawszy je, zmocnil sie przeciwko nim, tak iz nadzy i zranieni wybiegli z onego domu.
\par 17 I bylo to wiadomo wszystkim, i Zydom i Grekom, którzy mieszkali w Efezie; i przypadl strach na nie wszystkie, i bylo uwielbione imie Pana Jezusowe.
\par 18 A wiele tych, którzy uwierzyli, przychodzilo, wyznawajac i oznajmujac sprawy swoje.
\par 19 I wiele z tych, którzy sie naukami niepotrzebnemi parali, znióslszy ksiegi, spalili je przed wszystkimi, a obrachowawszy cene ich, znalezli tego piecdziesiat tysiecy srebrników.
\par 20 Tak poteznie roslo slowo Panskie i zmacnialo sie.
\par 21 A gdy sie to dokonalo, postanowil Pawel w duchu, aby przeszedlszy Macedonije i Achaje, szedl do Jeruzalemu, mówiac: Iz potem, gdy tam bede, musze i Rzym widziec.
\par 22 A poslawszy do Macedonii dwóch z tych, którzy mu sluzyli, Tymoteusza i Erasta, sam do czasu zostal w Azyi.
\par 23 A pod on czas stal sie rozruch niemaly okolo drogi Bozej.
\par 24 Albowiem niektóry zlotnik, imieniem Demetryjusz, który robil koscioly srebrne Dyjany, niemaly zysk przywodzil rzemieslnikom;
\par 25 Które zgromadziwszy i inne, którzy takiez rzemioslo robili, rzekl: Mezowie! wiecie, iz z tego rzemiosla mamy dostatki nasze.
\par 26 A widzicie i slyszycie, ze nie tylko w Efezie, ale malo nie po wszystkiej Azyi ten Pawel namówil i odwrócil wielki lud, mówiac: Ze to nie sa bogowie, którzy sa rekami uczynieni.
\par 27 Przetoz nam sie obawiac potrzeba, aby nie tylko rzemioslo nasze w lekkie powazenie nie przyszlo, ale aby i kosciól wielkiej bogini Dyjany za nic nie byl poczytany, a zeby nie przyszlo do skazy dostojenstwo jej, która wszystka Azyja i wszystek swiat chwali.
\par 28 A sluchajac tego i bedac pelni gniewu, krzykneli, mówiac: Wielka jest Dyjana Efeska!
\par 29 I bylo pelno po wszystkiem miescie zamieszania, i wpadli jednomyslnie na plac, porwawszy Gaja i Arystarcha, Macedonczyki, podrózne towarzysze Pawlowe.
\par 30 A gdy Pawel chcial wnijsc do pospólstwa, nie dopuscili mu uczniowie.
\par 31 A niektórzy tez z przedniejszych mezów Azyjackich, bedac mu przyjaciolmi, poslawszy do niego, prosili go, aby nie wychodzil na plac.
\par 32 Tedy jedni tak, a drudzy inaczej wolali; albowiem ona gromada byla zamieszana, a wiecej ich nie wiedzialo, dlaczego sie zbiezeli.
\par 33 A z onej zgrai wywlekli Aleksandra, którego popychali Zydowie; a Aleksander skinawszy reka, chcial dac sprawe ludowi.
\par 34 Ale gdy poznali, iz byl Zydem, wszczal sie jednostajny glos od wszystkich, jakoby przez dwie godziny wolajacych: Wielka jest Dyjana Efeska!
\par 35 Tedy pisarz usmierzywszy one zgraje, rzekl: Mezowie Efescy! i któryz jest czlowiek, co by nie wiedzial, iz miasto Efeskie opiekuje sie kosciolem wielkiej boginii Dyjany i obrazem, który spadl od Jowisza?
\par 36 A poniewaz sie temu nikt sprzeciwic nie moze, sluszna, abyscie sie uspokoili, a nic skwapliwie nie czynili.
\par 37 Albowiemescie przywiedli tych mezów, którzy nie sa ani swietokradcami, ani bluzniercami boginii waszej.
\par 38 A jezliz Demetryjusz i ci, którzy z nim sa rzemieslnicy, maja co przeciw komu, wszak bywa prawo, sa tez starostowie, niechze jedni drugich pozywaja.
\par 39 Jezli sie tez o czem inszem pytacie, to sie moze w porzadnem zgromadzeniu odprawic.
\par 40 Albowiem trzeba sie obawiac, abysmy oskarzeni nie byli o rozruch dzisiejszy, gdyz nie masz zadnej przyczyny, z której bysmy mogli dac sprawe, zesmy sie tu zbiegli. A to powiedziawszy, rozpuscil ono zgromadzenie.

\chapter{20}

\par 1 A gdy sie on rozruch uciszyl, zwolawszy Pawel uczniów i z nimi sie pozegnawszy, wyszedl stamtad, aby szedl do Macedonii.
\par 2 A przeszedlszy one strony i napomniawszy je szerokiemi slowy, przyszedl do Grecyi.
\par 3 A tam zamieszkawszy przez trzy miesiace, gdzie nan Zydowie zasadzke uczynili, gdy mial plynac do Syryi, umyslil sie powrócic przez Macedonije.
\par 4 I puscil sie z nim az do Azyi Sopater, Bereenczyk, a z Tesalonczyków Arystarchus i Sekundus, i Gajus Derbejczyk, i Tymoteusz;
\par 5 A z Azyjatczyków Tychykus i Trofimus, którzy wprzód poszedlszy, czekali nas w Troadzie.
\par 6 A my po dniach przasników odplynelismy z Filipowa i przyszlismy do nich do Troady za piec dni, gdziesmy zamieszkali siedm dni.
\par 7 Tedy pierwszy dzien po sabacie, gdy sie uczniowie zgromadzili na lamanie chleba, Pawel rozmawial z nimi, majac isc precz nazajutrz, i przedluzyl mowe az do pólnocy.
\par 8 A bylo wiele lamp na onej sali, gdzie byli zgromadzeni.
\par 9 Tam siedzac niektóry mlodzieniec, imieniem Eutychus, w oknie, bedac ciezkim snem zdjety, gdy tak Pawel dlugo mówil, snem zmorzony padl na dól z trzeciego pietra i podniesiony jest umarly.
\par 10 A Pawel zstapiwszy na dól, przypadl nan, a ujrzawszy go, rzekl: Nie trwozcie sie; boc w nim jest dusza jego.
\par 11 A wstapiwszy zasie, lamal chleb i jadl, i kazal im dlugo az do switania; potem odszedl precz.
\par 12 I przywiedli onego mlodzienca zywego, i byli nader ucieszeni.
\par 13 A my przyszedlszy wprzód do okretu, puscilismy sie do Assonu, abysmy stamtad wzieli Pawla; albowiem tak byl postanowil, majac sam pieszo isc.
\par 14 A gdy sie z nami zszedl w Assonie, wziawszy go, przyjechalismy do Mityleny.
\par 15 A stamtad odplynawszy, drugiego dnia przyszlismy przeciw Chyju, a trzeciego dnia przyplynelismy do Samu, a pomieszkawszy w Trogillu, nazajutrz przyszlismy do Miletu.
\par 16 Albowiem Pawel umyslil byl minac Efez, aby mu nie przyszlo czasu trawic w Azyi, bo sie kwapil, jezliby mu mozna, aby na dzien swiateczny byl w Jeruzalemie.
\par 17 Tedy z Miletu poslawszy do Efezu, przyzwal do siebie starszych zborowych.
\par 18 Którzy gdy do niego przyszli, rzekl im: Wy wiecie od pierwszego dnia, któregom przyszedl do Azyi, jakom z wami po wszystek czas byl,
\par 19 Sluzac Panu ze wszelka unizonoscia i z wiela lez i pokus, które na mie przychadzaly z zasadzek zydowskich.
\par 20 Jakom sie nie schranial niczego, co by bylo pozyteczne, abym wam nie oznajmil i nie uczyl was jawnie i po domach.
\par 21 Swiadectwo wydawajac i Zydom, i Grekom o pokucie ku Bogu i o wierze w Pana naszego Jezusa Chrystusa.
\par 22 A oto teraz ja bedac zwiazany duchem, ide do Jeruzalemu, nie wiedzac co tam na mie przyjsc ma.
\par 23 Tylko ze Duch Swiety po miastach swiadczy, powiadajac, ze mie wiezienie i uciski czekaja.
\par 24 Wszakze ja na nic nie dbam i nie jest mi tak droga dusza moja, bym tylko bieg mój z radoscia wykonal i posluge, któram wzial od Pana Jezusa na oswiadczenie Ewangielii laski Bozej.
\par 25 A teraz oto ja wiem, ze juz wiecej nie ogladacie oblicza mojego wy wszyscy, miedzy którymim chodzil, kazac królestwo Boze.
\par 26 Przetoz oswiadczam sie wam dnia dzisiejszego, zem ja jest czysty od krwi wszystkich.
\par 27 Albowiem nie chronilem sie, zebym wam nie mial oznajmic wszelkiej rady Bozej.
\par 28 Pilnujciez tedy samych siebie i wszystkiej trzody, w której was Duch Swiety postanowil biskupami, abyscie pasli zbór Bozy, którego nabyl przez wlasna krew.
\par 29 Boc ja to wiem, ze po odejsciu mojem wnijda miedzy was wilcy okrutni, którzy trzodzie folgowac nie beda.
\par 30 A z was samych powstana mezowie, mówiacy rzeczy przewrotne, aby za soba pociagneli uczniów.
\par 31 Przetoz czujcie, pomnac, zem przez trzy lata w nocy i we dnie nie przestawal napominac ze lzami kazdego z was.
\par 32 A teraz, bracia! poruczam was Bogu i slowu laski jego, który moze pobudowac i dac wam dziedzictwo miedzy wszystkimi poswieconymi.
\par 33 Srebra albo zlota, albo szaty nie pozadalem od nikogo.
\par 34 Owszem sami wiecie, ze moim potrzebom i tych, którzy sa ze mna, sluzyly te rece,
\par 35 Wszystkomci wam okazal, iz tak pracujac, musimy podejmowac slabe, a pamietac na slowa Pana Jezusowe, ze on rzekl: Szczesliwsza jest rzecz dawac, nizeli brac.
\par 36 A to powiedziawszy, kleknal na kolana swoje i modlil sie z nimi wszystkimi.
\par 37 I stal sie wielki placz wszystkich, a upadajac na szyje Pawlowa, calowali go;
\par 38 Smucac sie bardzo, najwiecej tych slów, które im rzekl, ze juz wiecej nie mieli ogladac oblicza jego. I prowadzili go do okretu.

\chapter{21}

\par 1 A gdysmy odjechali, rozstawszy sie z nimi, prosto jadac, przyjechalismy do Kou, a nazajutrz do Rodu, a stamtad do Patary.
\par 2 A tam znalazlszy okret, który mial plynac do Fenicyi, wsiadlszy wen, jechalismy.
\par 3 A gdy sie nam ukazal Cypr, tedy zostawiwszy go po lewej stronie, plynelismy do Syryi i przyplynelismy do Tyru; albowiem tam z okretu towary skladac miano.
\par 4 A znalazlszy uczniów, zamieszkalismy tam siedm dni; którzy mówili Pawlowi przez ducha, aby nie chodzil do Jeruzalemu.
\par 5 A gdysmy przemieszkali one dni, wyszedlszy, poszlismy, a wszyscy nas prowadzili z zonami i z dziatkami az za miasto, a kleknawszy na kolana na brzegu, modlilismy sie.
\par 6 A pozegnawszy sie jedni z drugimi, wstapilismy w okret, a oni sie wrócili do domu.
\par 7 A my odprawiwszy plynienie z Tyru, przyplynelismy do Ptolemaidy, a pozdrowiwszy braci, zamieszkalismy u nich przez jeden dzien.
\par 8 A nazajutrz wyszedlszy Pawel i my, którzysmy z nim byli, przyszlismy do Cezaryi, a wszedlszy w dom Filipa Ewangielisty, który byl jeden z onych siedmiu, zostalismy u niego.
\par 9 A ten mial cztery córki panny, które prorokowaly.
\par 10 A gdysmy tam przez niemalo dni zamieszkali, przyszedl z Judzkiej ziemi prorok niektóry, imieniem Agabus.
\par 11 Ten przyszedlszy do nas i wziawszy pas Pawla, a zwiazawszy sobie rece i nogi, rzekl: To mówi Duch Swiety: Meza, którego jest ten pas, tak zwiaza w Jeruzalemie Zydowie i podadza go w rece poganom.
\par 12 A gdysmy to uslyszeli, prosilismy i my i ci, którzy na onem miejscu byli, aby on nie chodzil do Jeruzalemu.
\par 13 Tedy odpowiedzial Pawel: Cóz czynicie placzac i serce mi psujac? Albowiem ja nie tylko byc zwiazanym, ale i umrzec jestem gotowy w Jeruzalemie dla imienia Pana Jezusowego.
\par 14 A gdy sie on nie dal namówic, dalismy pokój, mówiac: Niech sie stanie wola Panska.
\par 15 A po onych dniach, wziawszy rzeczy swoje, szlismy do Jeruzalemu.
\par 16 A szli z nami i niektórzy uczniowie z Cezaryi, wiodac z soba tego, u któregosmy gospoda stac mieli, niejakiego Mnazona Cypryjczyka, starego ucznia.
\par 17 A gdysmy przyszli do Jeruzalemu, wdziecznie nas bracia przyjeli.
\par 18 A nazajutrz wszedl z nami Pawel do Jakóba, gdzie sie byli wszyscy starsi zeszli.
\par 19 Które pozdrowiwszy, rozpowiedzial im wszystko porzadnie, co Bóg uczynil miedzy pogany przez usluge jego.
\par 20 Co oni uslyszawszy, chwalili Pana i rzekli mu: Widzisz, bracie! jako jest wiele tysiecy Zydów, którzy uwierzyli; a ci wszyscy gorliwi sa milosnicy zakonu.
\par 21 Ale o tobie wzieli sprawe, ze odwodzisz od Mojzesza wszystkich tych Zydów, którzy sa miedzy pogany, mówiac, ze nie maja obrzezywac dziatek, ani maja chodzic wedlug ustaw zakonnych.
\par 22 Cóz tedy jest? Konieczniec sie musi zejsc lud; bo uslysza, zes przyszedl.
\par 23 A przetoz czyn to, coc mówimy; Mamy tu czterech mezów, którzy na sobie slub maja;
\par 24 Tych wziawszy do siebie, oczysc sie z nimi i uczyn naklad na nie, aby ogolili glowy; a poznaja wszyscy, ze to, co o tobie slyszeli, nic nie jest, ale ze i ty sam chodzisz przestrzegajac zakonu.
\par 25 A o tych, którzy uwierzyli z pogan, mysmy pisali, stanowiac, aby nic takowego nie zachowywali, tylko aby sie wystrzegali tego, co jest ofiarowane balwanom i od krwi, i od rzeczy dlawionych, i od wszeteczenstwa.
\par 26 Tedy Pawel wziawszy z soba one meze, nazajutrz oczyszczony bedac z nimi, wszedl do kosciola, opowiadajac wypelnienie dni oczyszczenia, az za kazdego z nich oddana byla ofiara.
\par 27 A gdy sie mialo wypelnic siedm dni, niektórzy Zydowie z Azyi, ujrzawszy go w kosciele, wzbudzili wszystek lud i wrzucili na niego rece,
\par 28 Wolajac: Mezowie Izraelscy, ratujcie! Tenci to jest czlowiek, który przeciwko ludowi i zakonowi, i miejscu temu wszystkich wszedy uczy, nadto i Greki wprowadzil do kosciola, i splugawil to miejsce swiete.
\par 29 Albowiem przedtem widzieli z nim w miescie Trofima Efeskiego, o którym mniemali, zeby go Pawel wprowadzil do kosciola.
\par 30 I wzruszylo sie miasto wszystko, i zbiegl sie lud; a pojmawszy Pawla, wywlekli go precz z kosciola, a zatem zaraz drzwi zamkniono.
\par 31 A gdy sie starali, jakoby go zabili, dano znac hetmanowi wojska, iz sie wzruszylo wszystko Jeruzalem.
\par 32 Który zarazem wziawszy z soba zolnierze i setniki, przybiezal do nich. A oni ujrzawszy hetmana i zolnierze, przestali Pawla bic.
\par 33 Tedy hetman przyblizywszy sie, pojmal go i kazal go dwoma lancuchami zwiazac, i wywiadywal sie, kto by byl i co by uczynil?
\par 34 A jedni tak, drudzy inaczej miedzy ludem wolali; a gdy sie nic pewnego dla zgielku dowiedziec nie mógl, rozkazal go wiesc do obozu.
\par 35 A gdy byl u wschodu, przydalo sie, ze go prawie zolnierze niesli dla gwaltu onego ludu.
\par 36 Albowiem wielki lud szedl za nim, wolajac: Zgladz go.
\par 37 A gdy mial byc Pawel prowadzony do obozu, rzekl hetmanowi: A godzi mi sie co mówic do ciebie? A on rzekl: Umiesz po grecku?
\par 38 I nie tyzes jest on Egipczanin, którys przed temi dniami uczynil rozruch i wywiodles na puszcze cztery tysiace mezów zbójców?
\par 39 A Pawel rzekl: Jamci jest czlowiek Zyd Tarsenczyk, mieszczanin nie z podlego miasta w Cylicyi: przetoz prosze cie, dopusc mi mówic do ludu.
\par 40 A gdy on dopuscil, Pawel stojac na wschodzie, skinal reka na lud. A gdy bylo wielkie milczenie, uczynil rzecz do nich zydowskim jezykiem, mówiac:

\chapter{22}

\par 1 Mezowie bracia i ojcowie! sluchajcie mojej, która teraz do was czynie, obrony.
\par 2 A gdy uslyszeli, iz do nich rzecz czynil zydowskim jezykiem, tem sie bardziej uciszyli. I rzekl:
\par 3 Jamci jest maz Zyd, urodzony w Tarsie Cylicyjskim, lecz wychowany w miescie tem u nóg Gamalijelowych, wycwiczony dostatecznie w zakonie ojczystym, gorliwym bedac milosnikiem Bozym, jako wy wszyscy dzis jestescie.
\par 4 Którym przesladowal te droge az na smierc, wiazac i podawajac do wiezienia i meze, i niewiasty,
\par 5 Jako mi tego i najwyzszy kaplan jest swiadkiem, i wszyscy starsi, od których tez list wziawszy do braci, jechalem do Damaszku, abym i te, którzy tam byli, zwiazane przywiódl do Jeruzalemu, aby byli karani.
\par 6 I stalo sie, gdym jechal i gdym sie przyblizal do Damaszku o poludniu, ze z nagla ogarnela mie swiatlosc wielka z nieba.
\par 7 I upadlem na ziemie, a uslyszalem glos mówiacy do mnie: Saulu! Saulu! czemu mie przesladujesz?
\par 8 A jam odpowiedzial: Ktos jest, Panie? I rzekl do mnie: Jam jest Jezus Nazarenski, którego ty przesladujesz.
\par 9 A ci, którzy byli ze mna, acz widzieli swiatlosc i polekli sie, ale glosu nie slyszeli onego, który ze mna mówil.
\par 10 I rzeklem: Cóz uczynie, Panie? A Pan rzekl do mnie: Wstan; idz do Damaszku, a tam ci powiedza o wszystkiem, co postanowiono, abys ty uczynil.
\par 11 A gdym nie widzial przed jasnoscia swiatlosci onej, bedac prowadzony za reke od tych, co ze mna byli, przyszedlem do Damaszku.
\par 12 Tam niejaki Ananijasz, maz pobozny wedlug zakonu, majac swiadectwo od wszystkich Zydów tam mieszkajacych,
\par 13 Przyszedlszy do mnie i przystapiwszy, rzekl mi: Saulu bracie, przejrzyj! A jam tejze godziny wejrzal na niego.
\par 14 A on rzekl: Bóg ojców naszych obral cie, abys poznal wole jego, a izbys ogladal onego sprawiedliwego i sluchal glosu z ust jego.
\par 15 Albowiem mu bedziesz swiadkiem u wszystkich ludzi tego, cos widzial i slyszal.
\par 16 Przetoz teraz cóz odwlaczasz? Wstan, a ochrzcij sie, a omyj grzechy twoje, wzywajac imienia Panskiego.
\par 17 I stalo sie potem, gdym sie wrócil do Jeruzalemu, a modlilem sie w kosciele, zem byl w zachwyceniu.
\par 18 I widzialem go mówiacego do siebie: Spiesz sie; a wynijdz rychlo z Jeruzalemu, poniewaz swiadectwa twego nie przyjma o mnie.
\par 19 A jam rzekl: Panie! onic wiedza, zemci ja podawal do wiezienia i bijal w bóznicach te, którzy wierzyli w cie.
\par 20 I gdy wylewano krew Szczepana, swiadka twojego, jam tez przy tem stal i zezwalalem na zabicie jego, i strzeglem szat tych, którzy go zabijali.
\par 21 I rzekl do mnie: Idzze, boc ja cie do pogan daleko posle.
\par 22 A sluchali go az do tego slowa; i podniesli glos swój, mówiac: Zgladz z ziemi takiego; bo nie sluszna, aby mial zyc.
\par 23 A gdy oni wolali i miotali szaty, i ciskali proch na powietrze,
\par 24 Rozkazal go hetman wiesc do obozu i kazal go biczami spróbowac, zeby sie dowiedzial, dla której by przyczyny nan tak wolano.
\par 25 A gdy go rozciagniono, aby go biczami bito, rzekl Pawel do setnika, który tuz stal: Izali sie wam godzi czlowieka Rzymianina nieosadzonego biczami bic?
\par 26 Co uslyszawszy setnik, przystapiwszy do hetmana, powiedzial mu, mówiac: Patrz, co czynisz; boc ten czlowiek jest Rzymianinem.
\par 27 A przystapiwszy hetman, rzekl mu: Powiedz mi, jezlis ty jest Rzymianinem? A on rzekl: Tak jest.
\par 28 I odpowiedzial hetman: Jam za wielka summe tego miejskiego prawa dostal. A Pawel rzekl: A jam sie Rzymianinem i urodzil.
\par 29 A wnetze odstapili od niego ci, którzy go mieli wziac na próby. Do tego i hetman sie bal, dowiedziawszy sie, ze byl Rzymianinem, a iz go byl kazal zwiazac.
\par 30 A tak nazajutrz chcac sie pewnie dowiedziec tego, o co by byl oskarzony od Zydów, uwolnil go od onych zwiazek i rozkazal sie zejsc przedniejszym kaplanom i wszystkiej radzie ich, a wywiódlszy Pawla, stawil go przed nimi.

\chapter{23}

\par 1 A Pawel pilnie patrzac na one rade rzekl: Mezowie bracia! ja ze wszystkiego sumienia dobrego chodzilem przed Bogiem az do dnia tego.
\par 2 Tedy Ananijasz, najwyzszy kaplan, rozkazal go tym, którzy przy nim stali, bic w gebe.
\par 3 Tedy rzekl Pawel do niego: Uderzy cie Bóg, sciano pobielana! i ty siedzisz, sadzac mie wedlug zakonu, a rozkazujesz mie bic przeciwko zakonowi?
\par 4 Zatem ci, którzy tam stali, rzekli: Najwyzszemu kaplanowi Bozemu zlorzeczysz?
\par 5 A Pawel rzekl: Nie wiedzialem, bracia! zeby byl najwyzszym kaplanem; bo napisano: Ksiazeciu ludu twego zlorzeczyc nie bedziesz.
\par 6 A poznawszy Pawel, ze ich jedna czesc byla Saduceuszów a druga Faryzeuszów, zawolal w onej radzie: Mezowie bracia! jam jest Faryzeusz, syn Faryzeusza: o nadzieje i o powstanie umarlych mie tu dzis sadza.
\par 7 A gdy on to mówil, wszczal sie rozruch miedzy Faryzeuszami i Saduceuszami, i rozerwalo sie ono mnóstwo.
\par 8 Albowiem Saduceuszowie mówia, iz nie masz zmartwychwstania, ani Aniola, ani ducha; ale Faryzeuszowie to oboje wyznawaja.
\par 9 I wszczelo sie wolanie wielkie. A powstawszy nauczeni w Pismie z strony Faryzeuszów, spierali sie mówiac: Nicesmy zlego nie znalezli w tym czlowieku; i jezli mu co powiedzial duch albo Aniol, nie walczmyz z Bogiem.
\par 10 A gdy sie wszczal wielki rozruch, obawiajac sie hetman, aby Pawla miedzy soba nie rozszarpali, rozkazal isc zolnierzom na dól, a wydrzec go z posrodku ich i odwiesc do obozu.
\par 11 A drugiej nocy stanawszy przy nim Pan, rzekl: Badz dobrego serca, Pawle! albowiem jakos o mnie swiadczyl w Jeruzalemie, tak musisz swiadczyc i w Rzymie.
\par 12 A gdy byl dzien, zszedlszy sie niektórzy z Zydów, zawiazali sie klatwa, mówiac: Ze nie mieli jesc ani pic, azby Pawla zabili.
\par 13 A bylo ich wiecej niz czterdziesci, którzy to przysiezenie uczynili.
\par 14 Którzy przyszedlszy do przedniejszych kaplanów i do starszych, rzekli: Klatwasmy sie zawiazali, ze nic nie ukusimy, azbysmy Pawla zabili.
\par 15 Przetoz wy teraz dajcie znac hetmanowi z pozwoleniem wszystkiej rady, aby go jutro do was wywiódl, jakobyscie sie chcieli dostateczniej wywiedziec o sprawach jego, a my, pierwej niz tu przyjdzie, jestesmy gotowi go zabic.
\par 16 A gdy uslyszal siostrzeniec Pawla o tej zasadzce, przyszedl, a wszedlszy do obozu, oznajmil to Pawlowi.
\par 17 Tedy Pawel zawolawszy jednego z setników, rzekl: Zaprowadz tego mlodzienca do hetmana, bo mu cos ma powiedziec.
\par 18 A tak on wziawszy go, wiódl go do hetmana i rzekl: Pawel wiezien, zawolawszy mie, prosil, abym tego mlodzienca przywiódl do ciebie, któryc ma cos powiedziec.
\par 19 Tedy hetman wziawszy go za reke i ustapiwszy na strone, wywiadywal sie: Cóz to jest, co mi masz powiedziec?
\par 20 A on rzekl: Postanowili Zydowie prosic cie, abys jutro wywiódl Pawla przed rade, jakoby sie chcieli co dostateczniejszego wywiedziec o nim.
\par 21 Ale ty nie pozwalaj im tego; bo sie nan nasadzilo z nich wiecej niz czterdziesci mezów, którzy sie klatwa zawiazali, iz nie maja ani jesc ani pic, azby go zabili; i sa juz w pogotowiu, czekajac od ciebie odpowiedzi.
\par 22 Tedy hetman odprawil onego mlodzienca, przykazawszy mu, aby tego przed nikim nie powiadal, iz mu to oznajmil.
\par 23 A zawolawszy dwóch niektórych z setników, rzekl: Nagotujcie dwiescie zolnierzy, aby szli az do Cezaryi; do tego siedmdziesiat jezdnych i dwiescie drabantów na trzecia godzine w nocy;
\par 24 Nagotowac tez bydleta, aby wsadziwszy Pawla na nie, zdrowo go zaprowadzono do Feliksa starosty;
\par 25 Napisawszy list w ten sposób:
\par 26 Klaudyjusz Lizyjasz najmozniejszemu staroscie Feliksowi zdrowia zyczy.
\par 27 Tego meza pojmanego od Zydów, gdy juz od nich mial byc zabity, przypadlszy z rota, odjalem go, dowiedziawszy sie, iz jest Rzymianinem.
\par 28 A chcac wiedziec przyczyne, dla której by nan skarzyli, wywiodlem go przed ich rade;
\par 29 I znalazlem, ze nan skarza o jakies gadki z strony zakonu ich, a ze nie ma zadnej winy, dla której by byl godzien smierci albo wiezienia.
\par 30 A gdy mi powiedziano o zasadzce, która mieli uczynic Zydzi na tego meza, zarazem go poslal do ciebie, opowiedziawszy tez tym, co nan skarzyli, aby przed toba mówili to, co by przeciwko niemu mieli. Miej sie dobrze.
\par 31 Zolnierze tedy tak, jako im bylo rozkazano, wziawszy Pawla, prowadzili go noca do Antypatrydy.
\par 32 A nazajutrz, zostawiwszy jezdne, aby z nim jechali, wrócili sie do obozu.
\par 33 Którzy przyjechawszy do Cezaryi, a oddawszy list staroscie, stawili przed nim i Pawla.
\par 34 A starosta list przeczytawszy, spytal go, z której by byl krainy, a zrozumiawszy, ze byl z Cylicyi,
\par 35 Rzekl: Bede cie sluchal, gdy tez przybeda ci, którzy na cie skarza. I rozkazal go strzec na ratuszu Herodowym.

\chapter{24}

\par 1 A po pieciu dniach jechal najwyzszy kaplan Ananijasz z starszymi i z Tertullem niejakim prokuratorem; którzy staneli przed starosta przeciwko Pawlowi.
\par 2 A gdy byl pozwany, poczal nan skarzyc Tertullus, mówiac:
\par 3 Poniewazesmy wielkiego pokoju dostapili i wiele sie dobrego temu narodowi stalo przez twoje opatrznosc, i zawsze i wszedy to ze wszelkiem dziekowaniem przyznajemy, wielmozny Feliksie!
\par 4 Ale zebym cie dlugo nie bawil, prosze, abys nas maluczko posluchal wedlug zwyklej twojej ludzkosci.
\par 5 Albowiemesmy znalezli tego meza zarazliwego i wszczynajacego rozruch miedzy wszystkimi Zydami po wszystkim swiecie, i herszta tej sekty Nazarejczyków.
\par 6 Który sie tez wazyl splugawic kosciól; któregosmy tez pojmawszy, wedlug zakonu naszego chcieli sadzic.
\par 7 Lecz przyszedlszy hetman Lizyjasz z wielka moca, wzial go z rak naszych.
\par 8 Rozkazawszy tym, którzy nan skarza, isc do ciebie, od którego sie ty sam bedziesz mógl, wywiadujac sie, dowiedziec tego wszystkiego, o co my nan skarzymy.
\par 9 Na co sie zgodzili i Zydowie, mówiac: Ze sie tak rzecz ma.
\par 10 Tedy Pawel odpowiedzial, gdy nan starosta skinal, aby mówil: Od wielu lat wiedzac cie byc sedzia tego narodu, tem ochotniej dam sprawe o tem, co sie mnie dotycze.
\par 11 Gdyz ty wiedziec mozesz, iz nie masz wiecej dni tylko dwanascie, jakom ja przyszedl do Jeruzalemu, abym sie modlil.
\par 12 Do tego ani mie znalezli w kosciele z kim gadajacego albo buntujacego lud, ani w bóznicach, ani w miescie;
\par 13 Ani tego moga dowiesc, co tu teraz na mie skarza.
\par 14 To jednak przed toba wyznaje, ze wedlug onej drogi, która oni powiadaja byc heretyctwem, tak sluze ojczystemu Bogu, wierzac wszystkiemu, cokolwiek napisano w zakonie i w prorokach,
\par 15 Majac nadzieje w Bogu, ze bedzie, którego i oni czekaja, zmartwychwstanie i sprawiedliwych, i niesprawiedliwych.
\par 16 A sam sie o to pilnie staram, abym zawsze mial sumienie bez obrazenia przed Bogiem i przed ludzmi.
\par 17 A po wielu latach przyszedlem, abym przyniósl jalmuzny narodowi memu i ofiary.
\par 18 Na tem znalezli mie w kosciele oczyszczonego (nie z ludem ani z rozruchem) niektórzy Zydowie z Azyi.
\par 19 Którzy tez tu mieli stanac przed toba i skarzyc, jezliby co mieli przeciwko mnie.
\par 20 Albo niechaj ci sami powiedza, jezli we mnie znalezli jaka nieprawosc, gdym stal przed rada;
\par 21 Oprócz tego jednego glosu, zem miedzy nimi stojac, zawolal: Dla zmartwychwstania umarlych ja dzis sadzony bywam od was.
\par 22 A uslyszawszy to Feliks, odlozyl sprawe ich, mówiac: Gdy sie o tej drodze dostateczniej wywiem, kiedy tu hetman Lizyjasz przyjedzie, rozeznam sprawy wasze.
\par 23 I rozkazal setnikowi, aby strzegl Pawla i pofolgowal mu, i aby nie bronil zadnemu z przyjaciól jego poslugiwac mu albo go nawiedzac.
\par 24 A po kilku dniach przyjechawszy Feliks, z Drusylla, zona swoja, która byla Zydówka, kazal zawolac Pawla i sluchal go o wierze w Chrystusa.
\par 25 A gdy on rzecz czynil o sprawiedliwosci i o powsciagliwosci, i o przyszlym sadzie, ulakl sie Feliks i odpowiedzial: Juz teraz odejdz, a gdy czas upatrze, kaze cie zawolac.
\par 26 A przy tem spodziewal sie, ze mu Pawel mial dac pieniadze, zeby go wypuscil; dlatego tez tem czesciej go wzywajac do siebie, rozmawial z nim.
\par 27 A po wyjsciu dwóch lat mial po sobie Feliks namiestnika, Porcyjusa Festa; a chcac sobie Feliks laske zjednac u Zydów, zostawil Pawla w wiezieniu.

\chapter{25}

\par 1 Tedy Festus wjechawszy na panstwo, po trzech dniach przyjechal do Jeruzalemu z Cezaryi.
\par 2 I stawili sie przed nim najwyzszy kaplan i przedniejsi z Zydów przeciwko Pawlowi, i prosili go,
\par 3 Zadajac laski przeciwko niemu, aby go kazal przywiesc do Jeruzalemu, uczyniwszy zasadzke, aby go zabili na drodze.
\par 4 Ale Festus powiedzial: Iz Pawel jest pod straza w Cezaryi, a iz sam tam w rychle pojedzie.
\par 5 Którzy tedy, mówi, z was moga, niechze z nami jada; a jezli jest jaka wina w tym mezu, niechze nan skarza.
\par 6 A zamieszkawszy u nich nie wiecej tylko dziesiec dni, jechal do Cezaryi, a nazajutrz usiadlszy na sadzie, kazal Pawla przywiesc.
\par 7 Który gdy przyszedl, obstapili go ci, którzy byli przyszli z Jeruzalemu Zydowie, przynoszac wiele i ciezkich skarg przeciwko Pawlowi, których dowiesc nie mogli;
\par 8 Gdyz on sprawe dawal o sobie: Zem ani przeciwko zakonowi zydowskiemu, ani przeciwko kosciolowi, ani przeciwko cesarzowi nic nie zgrzeszyl.
\par 9 Ale Festus chcac sobie zjednac laske u Zydów, odpowiedziawszy Pawlowi, rzekl: Chceszze isc do Jeruzalemu, a tam o te rzeczy sadzony byc przede mna?
\par 10 Ale Pawel rzekl: Przed sadem cesarskim stoje, gdzie mie sadzic potrzeba: Zydówem w niczem nie krzywdzil, jako i ty lepiej wiesz.
\par 11 Bo jezlim w czem nieprawy i co godnego smierci uczynil, nie zbraniam sie umrzec; ale jezli nie masz nic takiego z tych rzeczy, o które na mie skarza, nikt mie im wydac nie moze; apeluje do cesarza.
\par 12 Tedy Festus rozmówiwszy sie z rada, odpowiedzial: Do cesarzas apelowal? do cesarza pójdziesz.
\par 13 A gdy wyszlo kilka dni, król Agrypa i Bernice przyjechali do Cezaryi, witac Festa.
\par 14 A gdy tam niemalo dni zamieszkali, Festus przelozyl królowi sprawe Pawlowa, mówiac: Maz niektóry zostawiony jest od Feliksa w wiezieniu.
\par 15 Dla którego, gdym byl w Jeruzalemie, stawili sie przede mna przedniejsi kaplani i starsi zydowscy, proszac o dekret przeciwko niemu.
\par 16 Którymem odpowiedzial, ze tego nie maja w zwyczaju Rzymianie, aby którego czlowieka mieli wydac na stracenie, azby pierwej oskarzony mial przed soba te, co nan skarza, i dano by mu plac do odpowiedzi na to, w czem go obwiniaja.
\par 17 Gdy sie tedy tu zeszli, bez wszelkiej odwloki nazajutrz zasiadlszy na sadzie, kazalem przywiesc tego meza.
\par 18 Przeciw któremu stanawszy ci, co nan skarzyli, zadnej winy nie przyniesli z tych, którychem sie ja spodziewal.
\par 19 Lecz jakies spory o swoich zabobonach mieli przeciwko niemu i o niejakim Jezusie umarlym, o którym Pawel twierdzil, ze zyw jest.
\par 20 Ja tedy watpiac o tem, o czem ten spór byl, rzeklem: Jezliby chcial isc do Jeruzalemu, a tam o tem byc sadzony?
\par 21 Lecz iz Pawel apelowal, aby zachowany byl do Augustowego rozeznania, rozkazalem go chowac, azbym go poslal do cesarza.
\par 22 Zatem Agrypa rzekl do Festa: Chcialbym ja tego czlowieka slyszec. A on rzekl: Jutro go uslyszysz.
\par 23 Nazajutrz tedy, gdy przyszedl Agrypa i Bernice z wielka okazaloscia, i weszli w dom sadowy z hetmanami i mezami przedniejszymi miasta onego, na rozkazanie Festowe przywiedziono Pawla.
\par 24 I rzekl Festus: Królu Agrypo i wszyscy mezowie, którzyscie tu z nami! widzicie tego, o którego mie wszystek lud zydowski prosil, i w Jeruzalemie i tu wolajac, ze nie sluszna, aby ten dluzej zyc mial.
\par 25 A ja zrozumiawszy, ze nie uczynil nic smierci godnego, a iz i on sam apelowal do Augusta, uczynilem dekret, aby byl poslany.
\par 26 O którym, co bym panu pewnego pisac mial, nie mam. Przetoz kazalem go przed was przywiesc, a najwiecej przed cie, królu Agrypo! abym, po rozsadzeniu sprawy jego, mial co pisac.
\par 27 Bo mi sie niesluszna widzi, poslac wieznia, a tego, o co go obwiniaja, nie oznajmic.

\chapter{26}

\par 1 Zatem Agrypa rzekl do Pawla: Pozwala ci sie, abys mówil sam od siebie. Tedy Pawel wyciagnawszy reke, taka sprawe dal:
\par 2 Na to wszystko, z czego mie obwiniaja Zydowie, królu Agrypo! poczytam sie byc za szczesliwego, iz dzis mam odpowiadac przed toba.
\par 3 A zwlaszcza, zes ty powiadom tych wszystkich, które sa miedzy Zydami, zwyczajów i sporów; przetoz cie prosze, zebys mie cierpliwie posluchal.
\par 4 Co sie tedy tknie zywota mego od mlodosci, jaki byl od poczatku miedzy narodem moim w Jeruzalemie, wiedza wszyscy Zydowie,
\par 5 Bedac mi swiadkami z dawna, (gdyby swiadectwo wydac chcieli), iz wedlug najdoskonalszej sekty nabozenstwa naszego zylem, bedac Faryzeuszem.
\par 6 A teraz o nadzieje onej obietnicy, ojcom od Boga uczynionej, stoje przed sadem;
\par 7 Której dwanascie naszych pokolen ustawicznie dniem i noca sluzac Bogu, maja nadzieje dostapic; o te nadzieje skarza na mie Zydowie, o królu Agrypo!
\par 8 Cóz za rzecz do wiary niepodobna u siebie sadzicie, ze Bóg umarle wzbudza?
\par 9 Mniec sie wprawdzie samemu zdalo, zem byl powinien przeciwko imieniowi Jezusa Nazarenskiego wiele przeciwnych rzeczy czynic.
\par 10 Com tez czynil w Jeruzalemie i wielem ja swietych sadzal do wiezienia, wziawszy moc od przedniejszych kaplanów; a gdy mieli byc zabijani, wotowalem przeciwko nim.
\par 11 I po wszystkich bóznicach czestokroc je trapiac, przymuszalem bluznic, a nader wsciekle przeciwko nim postepujac, przesladowalem je az i do obcych miast.
\par 12 W czem, gdym tez do Damaszku jechal, majac wladze i zlecenie od przedniejszych kaplanów,
\par 13 W poludnie, w drodze bedac, widzialem; o królu! swiatlosc z nieba, jasniejsza nad jasnosc sloneczna, która oswiecila mnie i tych, którzy jechali ze mna.
\par 14 A gdysmy wszyscy upadli na ziemie, uslyszalem glos mówiacy do siebie, a mówiacy zydowskim jezykiem: Saulu! Saulu! przeczze mie przesladujesz? trudno tobie przeciwko oscieniowi wierzgac.
\par 15 A jam rzekl: Ktos jest, Panie? A on rzekl: Jam jest Jezus, którego ty przesladujesz.
\par 16 Ale wstan, a stan na nogach twoich; gdyzem ci sie dlatego pokazal, abym cie uczynil sluga i swiadkiem tak tych rzeczy, któres widzial, jako i innych, w których ci sie pokaze.
\par 17 Wyrywajac cie od tego ludu i od pogan, do których cie teraz posylam,
\par 18 Ku otworzeniu oczu ich, aby sie nawrócili z ciemnosci do swiatlosci, a z mocy szatanskiej do Boga, aby tak wzieli odpuszczenie grzechów i dzial miedzy poswieconymi przez wiare, która jest w mie.
\par 19 Przetoz, o królu Agrypo! nie bylem nieposlusznym temu niebieskiemu widzeniu.
\par 20 Ale najprzód tym, którzy sa w Damaszku i w Jeruzalemie, i we wszystkiej krainie Judzkiej, i poganom opowiadalem, aby pokutowali i nawrócili sie do Boga, czyniac uczynki godne pokuty.
\par 21 Dla tych rzeczy Zydowie w kosciele mie pojmawszy, chcieli mie zabic.
\par 22 Ale za pomoca Boza jeszcze az do dnia tego stoje, swiadczac i malemu, i wielkiemu, nic nie mówiac oprócz tego, co opowiedzieli prorocy i Mojzesz, ze sie stac mialo;
\par 23 To jest, iz Chrystus mial cierpiec, a bedac pierwszym z zmartwychwstania opowiadac mial swiatlosc ludowi temu i poganom.
\par 24 To gdy on ku obronie swojej powiedzial, rzekl Festus glosem wielkim: Szalejesz Pawle! wielka nauka przywodzi cie do szalenstwa.
\par 25 Ale on rzekl: Nie szaleje, najmozniejszy Fescie! alec prawdziwe i zdrowe slowa powiadam.
\par 26 Wie bowiem i król o tych rzeczach, przed którym bezpiecznie mówie, gdyz nie tusze, aby co z tych rzeczy u niego bylo tajno, poniewaz sie to nie w kacie dzialo.
\par 27 Wierzysz, królu Agrypo! prorokom? Wiem, iz wierzysz.
\par 28 Zatem Agrypa rzekl do Pawla: Malo bys mnie nie namówil, zebym zostal chrzescijaninem.
\par 29 Ale Pawel rzekl: Zyczylbym od Boga, aby i w malu, i w wielu, nie tylko ty, ale i wszyscy, którzy mie dzis sluchaja, stali sie takimi, jakim i ja jest, oprócz tych zwiazek.
\par 30 A gdy on to rzekl, wstal król i starosta, i Bernice, i ci, którzy siedzieli z nim.
\par 31 A ustapiwszy na strone, rzekli jedni do drugich, mówiac: Nic godnego smierci albo wiezienia nie czyni ten czlowiek.
\par 32 Lecz Agrypa rzekl do Festa: Mógl ten czlowiek byc uwolniony, by byl do cesarza nie apelowal.

\chapter{27}

\par 1 A gdy skazano, zebysmy plyneli do Wloch, oddano i Pawla, i niektóre inne wieznie setnikowi, imieniem Julijuszowi, roty Augustowej.
\par 2 Tedy wsiadlszy w okret Adramitenski, majac plynac podle krain Azyi, puscili sie od brzegu, a byl z nami Arystarchus, Macedonczyk z Tesaloniki.
\par 3 A drugiego dnia przyplynelismy do Sydonu, kedy Julijusz ludzko sie Pawlowi stawiwszy, pozwolil mu isc do przyjaciól, aby wczasu zazyl.
\par 4 A stamtad sie pusciwszy, przyplynelismy pod Cypr, dlatego ze byly wiatry przeciwne.
\par 5 A przeplynawszy ono morze, które jest podle Cylicyi i Pamfilii, przybylismy do Miry, miasta Licyjskiego.
\par 6 A tam setnik znalazlszy okret Aleksandryjski, który plynal do Wloch, wsadzil nas wen.
\par 7 A gdysmy przez wiele dni z wolna plyneli, a zaledwie przeciwko Knidowi przyjechali, przeto ze nam wiatr nie dopuszczal, poplynelismy pod Krete podle Salmonu.
\par 8 A ledwie ja przeminawszy, przyszlismy na miejsce niektóre, które zowia piekne porty, od którego blisko bylo miasto Lasea.
\par 9 A gdy czas niemaly wyszedl, i juz bylo niebezpieczne zeglowanie, przeto iz juz byl i post przeminal, napominal je Pawel,
\par 10 Mówiac do nich: Mezowie! widze ja, iz z ukrzywdzeniem i z wielka szkoda nie tylko towarów i okretów, ale tez i dusz naszych bedzie to zeglowanie.
\par 11 Jednak setnik wiecej ufal sprawcy okretu i sternikowi, niz temu, co Pawel powiadal.
\par 12 A gdy nie bylo portu sposobnego ku zimowaniu, wiele ich rade dawalo puscic sie stamtad, owaby jakozkolwiek mogli przeprawiwszy sie do Fenicyi, przezimowac u portu Kretenskiego, który lezy miedzy wiatrem poludniowym i zachodnim.
\par 13 A gdy powional wiatr z poludnia, mniemajac, ze swego przedsiewziecia dopieli, pusciwszy sie od brzegu, plyneli blisko Krety.
\par 14 Lecz niedlugo potem uderzyl na nie wiatr gwaltowny, który zowia Euroklidon.
\par 15 A gdy byl okret porwany, a nie mógl sie oprzec wiatrowi, pusciwszy sie plynelismy.
\par 16 A gdysmy pod niektóra mala wysepke przyplyneli, która zowia Klauda, ledwiesmy mogli bacik zatrzymac.
\par 17 Który wciagnawszy, ratunku uzywali, podpasawszy okret, a bojac sie, zeby nie wpadl na hak, spusciwszy zagle, tak plyneli.
\par 18 A iz nami nawalnosci bardzo miotaly, nazajutrz towary wyrzucili.
\par 19 A trzeciego dnia rekami naszemi okretowe naczynia wyrzucilismy.
\par 20 Lecz gdy sie ani slonce, ani gwiazdy przez wiele dni nie ukazaly, a nawalnosc niemala nalegala, na ostatek odjeta byla wszystka nadzieja, zebysmy byli mogli byc zachowani.
\par 21 A gdysmy dlugo nie jedli, tedy Pawel stojac w posrodku ich rzekl: Mieliscie zaprawde, o mezowie! uslyszawszy mie, nie puszczac sie od Krety, a tak ujsc tej straty i zguby.
\par 22 Lecz i teraz napominam was, abyscie byli dobrej mysli; boc nie zginie z was zadna dusza, oprócz okretu.
\par 23 Albowiem stanal przy mnie tej nocy Aniol Boga tego, któregom ja jest i któremu sluze;
\par 24 Mówiac: Nie bój sie, Pawle! musisz stawiony byc przed cesarzem, a oto darowal ci Bóg wszystkich, którzy plyna z toba.
\par 25 Przetoz badzcie dobrej mysli, mezowie! albowiem wierze Bogu, ze tak bedzie, jako mi powiedziano.
\par 26 A musimy opasc na niektórej wyspie.
\par 27 A gdy przyszla noc czternasta, a mysmy sie blakali po morzu Adryjatyckiem, okolo pólnocy zdalo sie zeglarzom, iz sie im okazywala niektóra kraina.
\par 28 Tedy spusciwszy sznur z olowiem, znalezli glebiej dwadziescia sazni; a maluczko odplynawszy, zasie spuscili olów i znalezli pietnascie sazni.
\par 29 A bojac sie, aby snac na miejsca ostre nie wpadli, zrzuciwszy cztery kotwice z steru, pragneli, aby dzien byl.
\par 30 A gdy zeglarze myslili z okretu uciec i spuscili bacik na morze, chcac rzekomo od przodku okretu zarzucac kotwice,
\par 31 Rzekl Pawel setnikowi i zolnierzom: Jezli ci nie zostana w okrecie, wy zachowani byc nie mozecie.
\par 32 Tedy zolnierze obcieli powrozy u bacika i dopuscili mu odpasc.
\par 33 A miedzy tem niz sie rozednialo, napominal Pawel wszystkie, aby pokarm przyjeli, mówiac: Dzis temu czternasty dzien, jako czekajac trwacie bez pokarmu, nic nie jedzac.
\par 34 Dlatego prosze was, abyscie pokarm przyjeli; bo to sluzy ku zachowaniu waszemu, gdyz zadnego z was wlos z glowy nie spadnie.
\par 35 A to rzeklszy i chleb wziawszy, podziekowal Bogu przed wszystkimi i zlamawszy poczal jesc.
\par 36 Zatem wszyscy bedac lepszej mysli i sami pokarm przyjmowali.
\par 37 A bylo nas wszystkich dusz w okrecie dwiescie siedmdziesiat i szesc.
\par 38 Bedac tem pokarmem nasyceni, ulzenie czynili okretowi, wyrzucajac zboze w morze.
\par 39 A gdy byl dzien, nie poznali ziemi; wszakze obaczyli niejaka odnoge majaca brzeg, do którego uradzili jezliby moglo byc, przybic okret.
\par 40 A wyciagnawszy kotwice, puscili sie na morze; a rozpusciwszy zawiasy sterowe i podnióslszy zagiel po wietrze, mieli sie do brzegu;
\par 41 Ale napadlszy na miejsce, które mialo z obu stron morze, otracili okret; a przodek okretu uwieznawszy, zostal nie ruszajac sie, lecz zad rozbijal sie od gwaltownych walów.
\par 42 Tedy zolnierze radzili, aby wieznie pozabijali, izby który wyplynawszy nie uciekl.
\par 43 Ale setnik chcac zachowac Pawla, pohamowal je od tego przedsiewziecia i rozkazal tym, którzy mogli plywac, aby sie wprzód w morze puscili i na brzeg wyszli;
\par 44 Inni zasie, niektórzy na deskach, a niektórzy na sztukach okretu. I tak sie stalo, ze wszyscy zdrowo wyszli na ziemie.

\chapter{28}

\par 1 A gdy zdrowo uszli, dopiero poznali, iz one wyspe Melita nazywano.
\par 2 Ale on gruby lud pokazal nam nie lada ludzkosc; albowiem zapaliwszy stos drew, przyjeli nas wszystkich dla deszczu padajacego i dla zimna.
\par 3 A gdy Pawel nagarnal gromade chrustu i kladl na ogien, wyrwawszy sie zmija z goraca, przypiela sie do reki jego.
\par 4 A gdy on lud gruby ujrzal one gadzine wiszaca u reki jego, mówili jedni do drugich: Pewnie ten czlowiek jest mezobójca; bo choc z morza uszedl, przecie mu pomsta zywym byc nie dopuscila.
\par 5 Lecz on otrzasnawszy one gadzine w ogien, nic zlego nie ucierpial.
\par 6 A oni czekali, zeby opuchl albo nagle upadlszy umarl; a gdy tego dlugo czekali, a widzieli, iz mu sie nic zlego nie stalo, odmieniwszy sie, mówili, ze jest Bogiem.
\par 7 A przy onych miejscach mial folwarki przedniejszy onej wyspy, imieniem Publijusz, który przyjawszy nas, przez trzy dni przyjacielsko podejmowal.
\par 8 I stalo sie, ze ojciec onego Publijusza, majac goraczke i biegunke, lezal; do którego Pawel wszedlszy modlil sie, a wlozywszy nan rece uzdrowil go.
\par 9 To gdy sie stalo, tedy drudzy, którzy byli zlozeni chorobami na onej wyspie, przychodzili i byli uzdrowieni;
\par 10 Którzy nam tez wielka uczciwosc wyrzadzali, a gdysmy precz plynac mieli, nakladli nam, czego bylo potrzeba.
\par 11 A po trzech miesiacach puscilismy sie w okrecie Aleksandryjskim, który zimowal na onej wyspie, majacym za herb Kastora i Polluksa.
\par 12 A przyplynawszy do Syrakus, zamieszkalismy tam trzy dni.
\par 13 A stamtad plynac kolem, przybylismy do Regijum, a po jednym dniu, gdy powstal wiatr poludniowy, wtórego dnia plynelismy do Puteolów.
\par 14 Gdzie znalazlszy braci, uproszenismy byli od nich, zebysmy zamieszkali u nich przez siedm dni; a takiesmy szli do Rzymu.
\par 15 Stad, gdy uslyszeli bracia o nas, wyszli przeciwko nam az do rynku Appijuszowego i do Trzech Karczem; których gdy Pawel ujrzal, podziekowawszy Bogu, wzial smialosc.
\par 16 A gdysmy przyszli do Rzymu, setnik oddal wieznie hetmanowi wojska; ale Pawlowi dopuszczono, mieszkac osobno z zolnierzem, który go strzegl.
\par 17 I stalo sie po trzech dniach, ze zwolal Pawel przedniejszych z Zydów; a gdy sie zeszli, rzekl do nich: Mezowie bracia! ja nic nie uczyniwszy przeciwko ludowi i zwyczajom ojczystym, bedac zwiazany w Jeruzalemie, podanym jest w rece Rzymian;
\par 18 Którzy wysluchawszy mie, chcieli mie wypuscic dlatego, ze we mnie zadnej winy godnej smierci nie bylo.
\par 19 Lecz gdy sie temu sprzeciwiali Zydowie, musialem apelowac do cesarza; nie zebym mial naród mój w czem oskarzac.
\par 20 Dla tej tedy przyczyny zwolalem was, abym sie z wami ujrzal i rozmówil; albowiem dla nadziei ludu Izraelskiego tym lancuchem jestem opasany.
\par 21 Lecz oni rzekli do niego: My anismy listów dostali o tobie z Judzkiej ziemi, ani kto z braci przyszedlszy oznajmil, albo mówil o tobie co zlego.
\par 22 Wszakze bysmy radzi od ciebie slyszeli, co rozumiesz; albowiem o tej sekcie wiemy, iz wszedzie przeciwko niej mówia.
\par 23 A postanowiwszy mu dzien, przyszlo ich do niego do gospody niemalo, którym z doswiadczeniem wykladal królestwo Boze, namawiajac ich do tych rzeczy, które sa o Jezusie, z zakonu Mojzeszowego i z proroków, od poranku az do wieczora.
\par 24 Tedy niektórzy uwierzyli temu, co mówil, a niektórzy nie uwierzyli.
\par 25 A bedac niezgodnymi miedzy soba, rozeszli sie, gdy Pawel rzekl to jedno slowo: Iz dobrze Duch Swiety powiedzial przez Izajasza proroka, do ojców naszych,
\par 26 Mówiac: Idz do tego ludu, a mów: Sluchem sluchac bedziecie, ale nie zrozumiecie, a widzac widziec bedziecie, ale nic nie ujrzycie;
\par 27 Albowiem zgrubialo serce ludu tego, a ciezko uszyma slyszeli i zamruzyli oczy swe, aby snac oczyma nie widzieli, a uszyma nie slyszeli, i sercem nie zrozumieli, i nie nawrócili sie, a uzdrowilbym je.
\par 28 Niechze wam tedy wiadomo bedzie, iz poganom poslane jest to zbawienie Boze, a oni sluchac beda.
\par 29 A gdy to on rzekl, odeszli Zydowie, majac miedzy soba wielki spór.
\par 30 I mieszkal Pawel przez cale dwa lata w najemnej gospodzie swojej, i przyjmowal wszystkich, którzy przychodzili do niego;
\par 31 Kazac o królestwie Bozem i uczac tych rzeczy, które sa o Panu Jezusie Chrystusie, ze wszystkiem bezpieczenstwem bez przeszkody.


\end{document}