\begin{document}

\title{1 List do Koryntian}


\chapter{1}

\par 1 Pawel, powolany Apostol Jezusa Chrystusa przez wole Boza, i Sostenes brat.
\par 2 Zborowi Bozemu, który jest w Koryncie, poswieconym w Chrystusie Jezusie, powolanym swietym, ze wszystkimi, którzy wzywaja imienia Pana naszego Jezusa Chrystusa na wszelkiem miejscu, i ich, i naszem.
\par 3 Laska wam i pokój niech bedzie od Boga, Ojca naszego, i od Pana Jezusa Chrystusa.
\par 4 Dziekuje Bogu mojemu zawsze za was dla laski Bozej, która wam jest dana w Chrystusie Jezusie,
\par 5 Izescie we wszystkiem ubogaceni w nim we wszelkiej mowie i we wszelkiej znajomosci;
\par 6 Jako swiadectwo Chrystusowe utwierdzone jest w was,
\par 7 Tak iz wam na zadnym darze nie schodzi, którzy oczekujecie objawienia Pana naszego Jezusa Chrystusa.
\par 8 Który was tez utwierdzi az do konca, abyscie byli bez nagany w dzien Pana naszego, Jezusa Chrystusa.
\par 9 Wiernyc jest Bóg, przez którego jestescie powolani ku spolecznosci Syna jego, Jezusa Chrystusa, Pana naszego.
\par 10 A prosze was, bracia! przez imie Pana naszego, Jezusa Chrystusa, abyscie toz mówili wszyscy, a izby nie byly miedzy wami rozerwania, ale abyscie byli spojeni jednakim umyslem i jednakiem zdaniem.
\par 11 Albowiem oznajmiono mi o was, bracia moi! od domowników Chloi, iz poswarki sa miedzy wami.
\par 12 A to powiadam, iz kazdy z was mówi: Jamci jest Pawlowy, a jam Apollosowy, a jam Kiefasowy, alem ja Chrystusowy.
\par 13 Rozdzielonyz jest Chrystus? Azaz Pawel za was ukrzyzowany? Alboscie w imie Pawlowe ochrzczeni?
\par 14 Dziekuje Bogu, zem zadnego z was nie chrzcil, oprócz Kryspa i Gajusa;
\par 15 Aby kto nie rzekl, zem chrzcil w imie moje.
\par 16 Ochrzcilem tez i dom Stefanowy; nadto nie wiem, jezlim kogo drugiego ochrzcil.
\par 17 Boc mnie nie poslal Chrystus chrzcic, ale Ewangielije kazac, wszakze nie w madrosci mowy, aby nie byl wyniszczony krzyz Chrystusowy.
\par 18 Albowiem mowa o krzyzu tym, którzy gina, jest glupstwem; ale nam, którzy bywamy zbawieni, jest moca Boza.
\par 19 Bo napisano: Wniwecz obróce madrosc madrych, a rozum rozumnych odrzuce.
\par 20 Gdziez jest madry? Gdziez jest uczony w Pismie? Gdziez badacz wieku tego? Izali w glupstwo nie obrócil Bóg madrosci swiata tego?
\par 21 Albowiem poniewaz w madrosci Bozej swiat nie poznal Boga przez madrosc, upodobalo sie Bogu przez glupie kazanie zbawic wierzacych,
\par 22 Gdyz i Zydowie sie cudów domagaja, a Grekowie madrosci szukaja.
\par 23 Ale my kazemy Chrystusa ukrzyzowanego, Zydom wprawdzie zgorszenie, a Grekom glupstwo;
\par 24 Lecz samym powolanym i Zydom, i Grekom kazemy Chrystusa, który jest moca Boza i madroscia Boza.
\par 25 Albowiem glupstwo Boze jest medrsze niz ludzie; a mdlosc Boza jest mocniejsza niz ludzie.
\par 26 Widzicie zaiste powolanie wasze, bracia! iz niewiele madrych wedlug ciala, niewiele moznych, niewiele zacnego rodu;
\par 27 Ale co glupiego jest u swiata tego, to wybral Bóg, aby zawstydzil madrych, a co mdlego u swiata, wybral Bóg, aby zawstydzil mocnych.
\par 28 A podlego rodu u swiata i wzgardzone wybral Bóg, owszem te rzeczy, których nie masz, aby te, które sa, zniszczyl.
\par 29 Aby sie nie chlubilo zadne cialo przed obliczem jego.
\par 30 Lecz z niego wy jestescie w Chrystusie Jezusie, który sie nam stal madroscia od Boga i sprawiedliwoscia, i poswieceniem, i odkupieniem,
\par 31 Aby, jako napisano: Kto sie chlubi, w Panu sie chlubil.

\chapter{2}

\par 1 A ja gdym przyszedl do was, bracia! nie przyszedlem z wyniosloscia mowy albo madrosci, opowiadajac wam swiadectwo Boze.
\par 2 Albowiem nie osadzilem za rzecz potrzebna, co inszego umiec miedzy wami, tylko Jezusa Chrystusa, i to onego ukrzyzowanego.
\par 3 I bylem ja u was w slabosci i w bojazni i w strachu wielkim,
\par 4 A mowa moja i kazanie moje nie bylo w powabnych madrosci ludzkiej slowach, ale w okazaniu ducha i mocy,
\par 5 Aby sie wiara wasza nie gruntowala na madrosci ludzkiej, ale na mocy Bozej.
\par 6 A madrosc mówimy miedzy doskonalymi; ale madrosc nie tego swiata, ani ksiazat tego swiata, którzy gina;
\par 7 Ale mówimy madrosc Boza w tajemnicy, która jest zakryta, która Bóg przeznaczyl przed wieki ku chwale naszej,
\par 8 Której zaden z ksiazat tego swiata nie poznal; bo gdyby byli poznali, nigdy by Pana chwaly nie ukrzyzowali;
\par 9 Ale opowiadamy, jako napisano: Czego oko nie widzialo i ucho nie slyszalo i na serce ludzkie nie wstapilo, co nagotowal Bóg tym, którzy go miluja.
\par 10 Ale nam to Bóg objawil przez Ducha swojego; albowiem duch wszystkiego sie bada, i glebokosci Bozych.
\par 11 Bo któz z ludzi wie, co jest w czlowieku, tylko duch czlowieczy, który w nim jest? Takze tez i tego, co jest w Bogu, nikt nie wie, tylko Duch Bozy.
\par 12 Alesmy my nie przyjeli ducha swiata, lecz Ducha, który jest z Boga, abysmy wiedzieli, które rzeczy nam sa od Boga darowane;
\par 13 O których tez mówimy, nie temi slowy, których ludzka madrosc naucza, ale których Duch Swiety naucza, do duchownych rzeczy duchowne stosujac.
\par 14 Ale cielesny czlowiek nie pojmuje tych rzeczy, które sa Ducha Bozego; albowiem mu sa glupstwem i nie moze ich poznac, przeto iz duchownie bywaja rozsadzone.
\par 15 Alec duchowny rozsadza wszystko; lecz sam od nikogo nie bywa rozsadzony.
\par 16 Albowiem któz poznal zmysl Panski? Któz go bedzie uczyl? Ale my zmysl Chrystusowy mamy.

\chapter{3}

\par 1 I ja, bracia! nie moglem wam mówic jako duchownym, ale jako cielesnym i jako niemowlatkom w Chrystusie.
\par 2 Napawalem was mlekiem, a nie karmilem was pokarmem; boscie jeszcze nie mogli zniesc, owszem i teraz jeszcze nie mozecie,
\par 3 Gdyz jeszcze cielesnymi jestescie. Bo poniewaz miedzy wami jest zazdrosc i swary, i rozterki, azazescie nie cielesni i wedlug czlowieka nie chodzicie?
\par 4 Albowiem gdy kto mówi: Jam jest Pawlowy, a drugi: Jam Apollosowy, azaz cielesnymi nie jestescie?
\par 5 Bo któz jest, Pawel? Kto Apollos? jedno sludzy, przez którychescie uwierzyli, a to jako kazdemu Pan dal.
\par 6 Jam szczepil, Apollos polewal, ale Bóg wzrost dal.
\par 7 A tak, ani ten, co szczepi, jest czem, ani ten, co polewa, ale Bóg, który wzrost daje.
\par 8 Lecz ten, który szczepi, i ten, który polewa, jedno sa, a kazdy swoje zaplate wezmie wedlug pracy swojej.
\par 9 Albowiem jestesmy pomocnikami Bozymi, wy Boza rola, Bozym budynkiem jestescie.
\par 10 Wedlug laski Bozej, która mi jest dana, jako madry budownik zalozylem grunt, a drugi na nim buduje: wszakze kazdy niechaj baczy, jako na nim buduje.
\par 11 Albowiem gruntu innego nikt nie moze zalozyc, oprócz tego, który jest zalozony, który jest Jezus Chrystus.
\par 12 A jezli kto na tym gruncie buduje zloto, srebro, kamienie drogie, drwa, siano, slome,
\par 13 Kazdego robota jawna bedzie; bo to dzien pokaze, gdyz przez ogien objawiona bedzie, a kazdego roboty, jaka jest, ogien doswiadczy.
\par 14 Jezli czyja robota zostanie, która na nim budowal, zaplate wezmie.
\par 15 Jezli czyja robota zgore, ten szkode podejmie; lecz on sam bedzie zachowany, wszakze tak jako przez ogien.
\par 16 Azaz nie wiecie, iz kosciolem Bozym jestescie, a Duch Bozy mieszka w was?
\par 17 A jezli kto gwalci kosciól Bozy, tego Bóg skazi, albowiem kosciól Bozy swiety jest, którym wy jestescie.
\par 18 Niechajze nikt samego siebie nie zwodzi; jezli sie kto sobie zda byc madrym miedzy wami na tym swiecie, niech sie stanie glupim, aby sie stal madrym.
\par 19 Albowiem madrosc tego swiata glupstwem jest u Boga; bo napisano: Który chwyta madrych w chytrosci ich;
\par 20 I zasie: Pan zna mysli madrych, iz sa marnoscia.
\par 21 A tak niech sie nikt nie chlubi ludzmi; albowiem wszystkie rzeczy sa wasze.
\par 22 Badz Pawel, badz Apollos, badz Kiefas, badz swiat, badz zywot, badz smierc, badz przytomne, badz przyszle rzeczy, wszystkie sa wasze;
\par 23 Alescie wy Chrystusowi, a Chrystus Bozy.

\chapter{4}

\par 1 Tak niechaj o nas czlowiek rozumie, jako o slugach Chrystusowych i o szafarzach tajemnic Bozych.
\par 2 A tegoc wiec szukaja przy szafarzach, aby kazdy znaleziony byl wiernym.
\par 3 Alec u mnie to jest najmniejsza, zebym byl od was sadzony, albo od sadu ludzkiego; lecz i sam siebie nie sadze.
\par 4 Albowiem choc nic na sie nie wiem, wszakze nie przeto jestem usprawiedliwiony; ale ten, który mnie sadzi, Pan jest.
\par 5 A tak nie sadzcie przed czasem, azby Pan przyszedl, który tez oswieci, co skrytego jest w ciemnosci i objawi rady serc; a tedy kazdy bedzie mial chwale od Boga.
\par 6 A te rzeczy, bracia! w podobienstwie obrócilem na sie i na Apollosa dla was, abyscie sie nauczyli z nas nad to, co napisane, nie rozumiec, izbyscie sie jeden dla drugiego nie nadymali przeciwko drugiemu.
\par 7 Albowiem któz cie róznym czyni? I cóz masz, czego bys nie wzial? A jezlizes wzial, przeczze sie chlubisz, jakobys nie wzial?
\par 8 Juzescie nasyceni, juzescie ubogaceni, bez nas królujecie; a bodajescie królowali, abysmy i my z wami pospolu królowali!
\par 9 Bo mam za to, iz Bóg nas ostatnich Apostolów wystawil jakoby na smierc skazanych; albowiem stalismy sie dziwowiskiem swiatu, Aniolom i ludziom.
\par 10 Mysmy glupi dla Chrystusa, alescie wy roztropni w Chrystusie; mysmy slabi, alescie wy mocni; wyscie zacni, alesmy my bezecni.
\par 11 Jeszcze az do tej godziny i lakniemy, i pragniemy, i nadzy jestesmy, i bywamy policzkowani, i tulamy sie,
\par 12 I pracujemy, robiac wlasnemi rekami; gdy nas hanbia, dobrorzeczemy, gdy nas przesladuja, znosimy;
\par 13 Gdy nam zlorzecza, modlimy sie za nich: stalismy sie jako smieci tego swiata i jako omieciny u wszystkich, az dotad.
\par 14 To pisze, nie przeto, abym was zawstydzil; ale jako dziatki moje mile napominam.
\par 15 Bo chocbyscie mieli dziesiec tysiecy pedagogów w Chrystusie, wszakze niewiele ojców macie; bom ja was w Jezusie Chrystusie przez Ewangielije splodzil.
\par 16 Prosze was tedy, badzcie nasladowcami moimi.
\par 17 Dlategom poslal do was Tymoteusza, który jest syn mój mily i wierny w Panu; ten wam przypomni drogi moje w Chrystusie, jako wszedy w kazdym zborze nauczam.
\par 18 Ale tak sie niektórzy nadeli, jakobym nie mial przyjsc do was.
\par 19 Lecz przyjde rychlo do was, jezli Pan bedzie chcial, i poznam nie mowe tych nadetych, ale moc.
\par 20 Albowiem nie w mowie zalezy królestwo Boze, ale w mocy.
\par 21 Cóz chcecie? z rózgali mam przyjsc do was, czyli z miloscia i duchem cichosci?

\chapter{5}

\par 1 Zapewne slychac, ze jest miedzy wami wszeteczenstwo, a takie wszeteczenstwo, jakie i miedzy pogany nie bywa mianowane, aby kto mial miec zone ojca swego.
\par 2 A wyscie sie nadeli, a nie raczejscie sie smucili, aby byl uprzatniony z posrodku was ten, który ten uczynek popelnil.
\par 3 Przetoz ja, aczem odlegly cialem, lecz przytomny duchem, juzem jakobym byl przytomny, osadzil tego, który to tak popelnil,
\par 4 Gdy sie w imieniu Pana naszego Jezusa Chrystusa zgromadzicie, i z duchem moim, i z moca Pana naszego, Jezusa Chrystusa,
\par 5 Oddac szatanowi na zatracenie ciala, zeby duch byl zachowany w on dzien Pana Jezusa.
\par 6 Nie dobrac to chluba wasza. Azaz nie wiecie, iz troche kwasu wszystko zaczynienie zakwasza?
\par 7 Wyczysciez tedy stary kwas, abyscie byli nowem zaczynieniem, jako przasnymi jestescie; albowiem Baranek nasz wielkanocny za nas ofiarowany jest, Chrystus.
\par 8 A tak obchodzmy swieto nie w starym kwasie, ani w kwasie zlosci i rozpusty, ale w przasnikach szczerosci i prawdy.
\par 9 Pisalem wam w liscie, abyscie sie nie mieszali z wszetecznikami;
\par 10 Ale nie zgola z wszetecznikami tego swiata albo z lakomcami, albo z drapiezcami, albo z balwochwalcami; bo inaczej musielibyscie z tego swiata wynijsc.
\par 11 Lecz teraz pisalem wam, abyscie sie nie mieszali; jezliby kto, mieniac sie byc bratem, byl wszetecznikiem, albo lakomca, albo balwochwalca, albo obmówca, albo pijanica, albo zdzierca, zebyscie z takowym i nie jadali.
\par 12 Albowiem cóz ja mam sadzic i obcych? Azaz wy tych, co sa domowi, nie sadzicie?
\par 13 Ale tych, którzy sa obcymi, Bóg sadzi. Przetoz uprzatnijcie tego zlosnika z posrodku samych siebie.

\chapter{6}

\par 1 Smiez kto z was, majac sprawe z drugim, sadzic sie przed niesprawiedliwymi, a nie przed swietymi?
\par 2 Azaz nie wiecie, iz swieci beda sadzili swiat? A jezli swiat od was bedzie sadzony, czyliscie niegodni, abyscie sady mniejsze odprawiali?
\par 3 Azaz nie wiecie, iz Anioly sadzic bedziemy? A cóz tych doczesnych rzeczy?
\par 4 Przeto jezlibyscie mieli sady o rzeczy doczesne, tych, którzy sa najpodlejsi we zborze, na sad wysadzajcie.
\par 5 Ku zawstydzeniu waszemu to mówie. Nie maszze miedzy wami madrego i jednego, który by mógl rozsadzic miedzy bracmi swoimi?
\par 6 Ale sie brat z bratem prawuje, i to przed niewiernymi?
\par 7 Juz tedy zapewne jest miedzy wami niedostatek, ze sie z soba prawujecie. Czemuz raczej krzywdy nie cierpicie? Czemuz raczej szkody nie podejmujecie?
\par 8 Owszem wy krzywdzicie i do szkody przywodzicie, a to braci.
\par 9 Azaz nie wiecie, iz niesprawiedliwi królestwa Bozego nie odziedzicza? Nie mylcie sie: ani wszetecznicy, ani balwochwalcy, ani cudzoloznicy, ani pieszczotliwi, ani samcoloznicy,
\par 10 Ani zlodzieje, ani lakomcy, ani pijanicy, ani zlorzeczacy, ani zdziercy królestwa Bozego nie odziedzicza.
\par 11 A takimiscie niektórzy byli; alescie omyci, alescie poswieceni, alescie usprawiedliwieni w imieniu Pana Jezusa i przez Ducha Boga naszego.
\par 12 Wszystko mi wolno, ale nie wszystko pozyteczno; wszystko mi wolno, ale ja sie nie dam zniewolic zadnej rzeczy.
\par 13 Pokarmy brzuchowi naleza, a brzuch pokarmom; ale Bóg i brzuch i pokarmy skazi; lecz cialo nie nalezy wszeteczenstwu ale Panu, a Pan cialu.
\par 14 Bo Bóg i Pana wzbudzil, i nas wzbudzi moca swoja.
\par 15 Azaz nie wiecie, iz ciala wasze sa czlonkami Chrystusowymi? Wziawszy tedy czlonki Chrystusowe, czyli je uczynie czlonkami wszetecznicy? Nie daj tego Boze!
\par 16 Azaz nie wiecie, iz ten, co sie zlacza z wszetecznica, jednem cialem z nia jest? albowiem mówi: Beda dwoje jednem cialem.
\par 17 A kto sie zlacza z Panem, jednym duchem jest z nim.
\par 18 Uciekajcie przed wszeteczenstwem. Wszelki grzech, który by czlowiek popelnil, oprócz ciala jest; lecz kto wszeteczenstwo plodzi, przeciwko swemu wlasnemu cialu grzeszy.
\par 19 Azaz nie wiecie, iz cialo wasze jest kosciolem Ducha Swietego, który w was jest, którego macie od Boga? a nie jestescie sami swoi;
\par 20 Albowiemescie drogo kupieni. Wyslawiajciez tedy Boga w ciele waszym i w duchu waszym, które sa Boze.

\chapter{7}

\par 1 Lecz o tem, coscie mi pisali: Dobrzecby czlowiekowi, nie tykac sie niewiasty;
\par 2 Ale dla uwarowania sie wszeteczenstwa niech kazdy ma swoje wlasna zone, a kazda niech ma swego wlasnego meza.
\par 3 Maz niech zonie powinna chec oddaje, takze tez i zona mezowi.
\par 4 Zona wlasnego ciala swego w mocy nie ma, ale maz; takze tez i maz wlasnego ciala swego w mocy nie ma, ale zona.
\par 5 Nie oszukiwajcie jeden drugiego; chybaby z spólnego zezwolenia do czasu, abyscie sie uwolnili do postu i do modlitwy; a zasie wespól sie schodzcie, aby was szatan nie kusil dla waszej niepowsciagliwosci.
\par 6 Ale to mówie jako pozwalajac, a nie jako rozkazujac.
\par 7 Albowiem chcialbym, aby wszyscy ludzie tak byli jako i ja; alec kazdy ma swój wlasny dar od Boga, jeden tak a drugi owak.
\par 8 A mówie niezonatym i wdowom: Dobrze im jest, jezliby tak zostali, jako i ja.
\par 9 Ale jezli sie wstrzymac nie moga, niechze w stan malzenski wstapia; boc lepiej w stan malzenski wstapic, niz upalenie cierpiec.
\par 10 Tym zasie, którzy sa w stanie malzenskim, rozkazuje nie ja, ale Pan, mówiac: Zeby sie zona od meza nie odlaczala.
\par 11 Ale jezliby sie tez odlaczyla, niechajze zostaje bez meza, albo niech sie z mezem pojedna, a maz zony niechaj nie opuszcza.
\par 12 A inszym zasie ja mówie, a nie Pan: Jezli który brat ma zone niewierna, a ta z nim przyzwala mieszkac, niechze jej nie opuszcza.
\par 13 A jezli która zona meza niewiernego ma, a on przyzwala z nia mieszkac, niechze go nie opuszcza.
\par 14 Albowiem poswiecony jest maz niewierny przez zone i zona niewierna poswiecona jest przez meza; bo inaczej dziatki wasze bylyby nieczystemi, lecz teraz swietemi sa.
\par 15 A jezli ten, co jest niewierny, chce sie odlaczyc, niechze sie odlaczy; albowiem nie jest niewolnikiem brat albo siostra w takowych rzeczach; alec ku pokojowi nas Bóg powolal.
\par 16 Albowiem co ty wiesz, zono! jezli pozyskasz meza? Albo co ty wiesz, mezu! pozyskaszli zone?
\par 17 Jednak jako kazdemu udzielil Bóg, jako kazdego powolal Pan, tak niech postepuje; a takci we wszystkich zborach stanowie.
\par 18 Obrzezanym kto powolany jest, niechaj nie wprowadza na sie nieobrzezki; a w nieobrzezce kto jest powolany, niech sie nie obrzezuje.
\par 19 Obrzezka nic nie jest, takze nieobrzezka nic nie jest; ale zachowywanie przykazan Bozych.
\par 20 Kazdy w tem powolaniu, w którem powolany jest, niech zostaje.
\par 21 Jestes powolany niewolnikiem, nie dbajze na to; ale jezli tez mozesz byc wolny, raczej wolnosci uzywaj.
\par 22 Albowiem kto w Panu powolany jest niewolnikiem, wolny jest w Panu; takze tez, kto jest powolany wolnym, niewolnikiem jest Chrystusowym.
\par 23 Drogoscie kupieni; nie badzcie niewolnikami ludzkimi,
\par 24 Kazdy tedy, jakim jest powolany bracia! takim niechaj zostaje przed Bogiem.
\par 25 A o pannach rozkazania Panskiego nie mam; wszakze rade daje, jako ten, któremu Pan z milosierdzia swego dal, aby byl wiernym.
\par 26 Mniemam tedy, ze to jest rzecz dobra dla nastepujacej potrzeby, ze jest rzecz dobra czlowiekowi tak byc.
\par 27 Przywiazales sie do zony, nie szukajze rozwiazania; rozwiazanys od zony, nie szukajze zony.
\par 28 A jezlibys sie ozenil, nie zgrzeszyles; jezliby tez panna szla za maz, nie zgrzeszyla; wszakze utrapienie w ciele takowi miec beda; lecz ja was szanuje.
\par 29 A toc mówie, bracia! poniewaz czas potomny ukrócony jest, aby i ci, którzy zony maja, byli, jakoby ich nie mieli;
\par 30 A którzy placza, jakoby nie plakali; a którzy sie raduja, jakoby sie nie radowali; a którzy kupuja, jakoby nie trzymali;
\par 31 A którzy uzywaja tego swiata, jakoby zle nie uzywali; bo przemija ksztalt tego swiata.
\par 32 A chce, abyscie wy byli bez klopotu, bo kto nie ma zony, stara sie o rzeczy Panskie, jakoby sie podobal Panu;
\par 33 Ale kto sie ozenil, stara sie o rzeczy tego swiata, jakoby sie podobal zonie.
\par 34 Jest róznosc miedzy mezatka i panna; która nie szla za maz, stara sie o rzeczy Panskie, aby byla swieta i cialem i duchem; ale która szla za maz, stara sie o rzeczy tego swiata, jakoby sie podobala mezowi.
\par 35 A toc mówie ku dobru waszemu; nie abym sidlo na was wrzucil, ale abyscie slusznie i przystojnie stali przy Panu bez rozerwania.
\par 36 A jezli kto mniema, ze nieprzystojnie sobie poczyna z panna swoja, gdyby z lat swoich wyszla, i do tego by jej przyszlo, co chce, niechaj czyni, nie grzeszy; niechze idzie za maz.
\par 37 Ale kto statecznie postanowil w sercu swem, potrzeby tego nie majac, lecz ma w mocy wlasna swoje wole i to usadzil w sercu swem, aby zachowal panne swoje, dobrze czyni.
\par 38 A tak ten, kto daje za maz, dobrze czyni; ale który nie daje za maz, lepiej czyni.
\par 39 Zona zwiazana jest zakonem, póki zyje maz jej; a jezliby umarl maz jej, wolna jest, aby szla, za kogo chce, tylko w Panu.
\par 40 Ale szczesliwsza jest, jezliby tak zostala wedlug rady mojej; a mniemam, ze i ja mam Ducha Bozego.

\chapter{8}

\par 1 A o rzeczach, które balwanom ofiarowane bywaja, wiemy, iz wszyscy umiejetnosc mamy. Umiejetnosc nadyma, ale milosc buduje.
\par 2 A jezli kto mniema, zeby co umial, jeszcze nic nie umie, tak jakoby mial umiec;
\par 3 Lecz jezli kto miluje Boga, ten jest wyuczony od niego.
\par 4 A przetoz o pokarmach, które bywaja balwanom ofiarowane, wiemy, iz balwan na swiecie nic nie jest, a iz nie masz zadnego inszego Boga, tylko jeden.
\par 5 Bo choc sa, którzy bogami nazywani bywaja i na niebie, i na ziemi: (jakoz jest wiele bogów i wiele panów.)
\par 6 Ale my mamy jednego Boga Ojca, z którego wszystko, a my w nim; i jednego Pana Jezusa Chrystusa, przez którego wszystko, a my przezen.
\par 7 Ale nie we wszystkich jest ta umiejetnosc; albowiem niektórzy sumienie majac dla balwana az dotad, jedza jako rzecz balwanom ofiarowana, a sumienie ich bedac mdle, pokalane bywa.
\par 8 Alec nas pokarm nie zaleca Bogu; bo chocbysmy jedli, nic nam nie przybywa; a chocbysmy i nie jedli, nic nam nie ubywa.
\par 9 Jednak baczcie, aby snac ta wolnosc wasza nie byla mdlym ku zgorszeniu.
\par 10 Albowiem jezliby kto ujrzal cie, który masz umiejetnosc, w balwochwalni siedzacego, azaz sumienie onego, który jest mdly, nie bedzie pobudzone ku jedzeniu rzeczy balwanom ofiarowanych?
\par 11 I zginie dla onej twojej umiejetnosci brat mdly, za którego Chrystus umarl.
\par 12 A grzeszac tak przeciwko braciom i mdle ich sumienie obrazajac, grzeszycie przeciwko Chrystusowi.
\par 13 Przeto, jezli pokarm gorszy brata mego, nie bede jadl miesa na wieki, abym brata mego nie zgorszyl.

\chapter{9}

\par 1 Izalim nie jest Apostolem? Izalim nie jest wolnym? Izalim Jezusa Chrystusa, Pana naszego nie widzial? Izali wy nie jestescie praca moja w Panu?
\par 2 Chocbym innym nie byl Apostolem, alem wam jest; albowiem pieczecia apostolstwa mego wy jestescie w Panu.
\par 3 Tac jest obrona moja przeciwko tym, którzy mie sadza.
\par 4 Izali nie mamy wolnosci jesc i pic?
\par 5 Izali nie mamy wolnosci wodzic z soba siostry zony, jako i drudzy Apostolowie i bracia Panscy, i Kiefas?
\par 6 Izali ja tylko i Barnabasz nie mamy wolnosci, abysmy nie pracowali?
\par 7 Któz kiedy sluzy zolnierke swoim kosztem? Któz sadzi winnice, a owocu jej nie pozywa? Albo któz trzode pasie, a mleka trzody nie pozywa?
\par 8 Izali to obyczajem ludzkim mówie? Izali i zakon tegoz nie mówi?
\par 9 Albowiem w zakonie Mojzeszowym napisano: Nie zawiazesz geby wolowi mlócacemu; izali sie Bóg o woly stara?
\par 10 Czyli zgola dla nas to mówi? Albowiem dla nas to napisano; gdyz w nadziei ma orac ten, co orze, a kto mlóci w nadziei, nadziei swojej ma byc uczestnikiem.
\par 11 Poniewazesmy my wam duchowne dobra siali, wielkaz to, gdybysmy wasze cielesne zeli?
\par 12 Jezliz insi tej wolnosci nad wami uzywaja, czemuz nie raczej my? A wszakzesmy tej wolnosci nie uzywali, ale wszystko znaszamy, abysmy jakiego wstretu Ewangielii Chrystusowej nie uczynili.
\par 13 Azaz nie wiecie, iz ci, którzy okolo rzeczy swietych pracuja, z swietych rzeczy jadaja? a którzy oltarza pilnuja, spólna czastke z oltarzem maja?
\par 14 Tak tez Pan postanowil tym, którzy Ewangielije opowiadaja, aby z Ewangielii zyli.
\par 15 Alem ja nic z tych rzeczy nie uzywal. I nie pisalem tego, aby sie tak przy mnie dzialo; bo mnie daleko lepiej umrzec, nizby kto przechwalanie moje mial próznem uczynic.
\par 16 Bo jezli Ewangielije opowiadam, nie mam sie czem chlubic, gdyz ta powinnosc na mnie lezy; a biada mnie, jezlibym Ewangielii nie opowiadal.
\par 17 Albowiem jezli to dobrowolnie czynie, mam zaplate; jezli poniewolnie, szafarstwa mi powierzono.
\par 18 Jakaz tedy mam zaplate? Abym Ewangielije opowiadajac, bez nakladu wystawil Ewangielije Chrystusowa, na to, zebym zle nie uzywal wolnosci mojej przy Ewangielii.
\par 19 Albowiem bedac wolnym od wszystkich, samegom siebie uczynil niewolnikiem wszystkim, abym ich wiecej pozyskal.
\par 20 I stalem sie Zydom jako Zyd, abym Zydów pozyskal; a tym, którzy sa pod zakonem, jakobym byl pod zakonem, abym tych, którzy sa pod zakonem, pozyskal;
\par 21 Tym, którzy sa bez zakonu, jakobym bez zakonu, (nie bedac bez zakonu Bogu, ale bedac pod zakonem Chrystusowi), abym pozyskal tych, którzy sa bez zakonu.
\par 22 Stalem sie mdlym jako mdly, abym mdlych pozyskal. Wszystkim stalem sie wszystko, abym przecie niektórych zbawil.
\par 23 A to czynie dla Ewangielii, abym sie jej stal uczestnikiem.
\par 24 Azaz nie wiecie, iz ci, którzy w zawód bieza, wszyscyc wiec bieza, lecz jeden zaklad bierze? Takze biezcie, abyscie otrzymali.
\par 25 A kazdy, który sie potyka, we wszystkiem sie powsciaga, onic wprawdzie, aby wzieli korone skazitelna, ale my nieskazitelna.
\par 26 Ja tedy tak bieze, nie jako na niepewne; tak szermuje, nie jako wiatr bijac.
\par 27 Ale karze cialo moje i w niewole podbijam, abym snac inszym kazac, sam nie byl odrzucony.

\chapter{10}

\par 1 A nie chce, abyscie nie mieli wiedziec bracia! iz ojcowie nasi wszyscy pod oblokiem byli i wszyscy przez morze przeszli;
\par 2 I wszyscy w Mojzesza ochrzczeni sa w obloku i w morzu;
\par 3 I wszyscy tenze pokarm duchowny jedli;
\par 4 I wszyscy tenze napój duchowny pili; albowiem pili z opoki duchownej, która za nimi szla; a ta opoka byl Chrystus.
\par 5 Lecz wiekszej czesci z nich nie upodobal sobie Bóg; albowiem polegli na puszczy.
\par 6 A te rzeczy staly sie nam za wzór na to, abysmy zlych rzeczy nie pozadali, jako i oni pozadali
\par 7 Nie badzciez tedy balwochwalcami jako niektórzy z nich, tak jako napisano: Siadl lud, aby jadl i pil, i wstali grac.
\par 8 Ani sie dopuszczajmy wszeteczenstwa, jako sie niektórzy z nich wszeteczenstwa dopuszczali i padlo ich jednego dnia dwadziescia i trzy tysiace.
\par 9 Ani kusmy Chrystusa, jako niektórzy z nich kusili i od wezów pogineli.
\par 10 Ani szemrzyjcie, jako niektórzy z nich szemrali, i pogineli od tego, który wytraca.
\par 11 A te wszystkie rzeczy przydaly sie im za wzór, a napisane sa dla napomnienia naszego, na których koniec swiata przyszedl.
\par 12 A tak kto mniema, ze stoi, niechze patrzy, aby nie upadl.
\par 13 Pokuszenie sie was nie jelo, tylko ludzkie; ale wiernyc jest Bóg, który nie dopusci, abyscie byli kuszeni nad moznosc wasze, ale uczyni z pokuszeniem i wyjscie, abyscie znosic mogli.
\par 14 Przetoz, najmilsi moi! uciekajcie przed balwochwalstwem.
\par 15 Jako madrym mówie; rozsadzcie wy, co mówie.
\par 16 Kielich blogoslawienia, który blogoslawimy, izali nie jest spolecznoscia krwi Chrystusowej? Chleb, który lamiemy, izali nie jest spolecznoscia ciala Chrystusowego?
\par 17 Albowiem jednym chlebem, jednym cialem wiele nas jest; bo wszyscy chleba jednego jestesmy uczestnikami.
\par 18 Spojrzyjcie na Izraela, który jest wedlug ciala; izaz ci, którzy jedza ofiary, nie sa uczestnikami oltarza?
\par 19 Cóz tedy mówie? Zeby balwan mial byc czem, a izby to, co bywa balwanom ofiarowane, mialo czem byc?
\par 20 Owszem powiadam, iz to, co poganie ofiaruja, dyjablom ofiaruja a nie Bogu; a nie chcialbym, abyscie byli uczestnikami dyjablów.
\par 21 Nie mozecie pic kielicha Panskiego i kielicha dyjabelskiego; nie mozecie byc uczestnikami stolu Panskiego i stolu dyjabelskiego.
\par 22 I mamyz draznic Pana? Izalismy mocniejsi nizeli on?
\par 23 Wszystko mi wolno, ale nie wszystko pozyteczne; wszystko mi wolno, ale nie wszystko buduje.
\par 24 Nikt niechaj nie szuka tego, co jest jego, ale kazdy, co jest blizniego.
\par 25 Cokolwiek w jatkach sprzedawaja, jedzcie, nic nie pytajac dla sumienia.
\par 26 Albowiem Panska jest ziemia i napelnienie jej.
\par 27 A jezliby was kto zaprosil z niewiernych, a chcecie isc, wszystko, co przed was poloza, jedzcie, nic nie pytajac dla sumienia.
\par 28 Ale jezliby wam kto rzekl: To jest balwanom ofiarowane, nie jedzcie dla onego, który to oznajmil i dla sumienia; albowiem Panska jest ziemia i napelnienie jej.
\par 29 A powiadam dla sumienia, nie twego, ale onego drugiego; bo przeczze wolnosc moja ma byc osadzona od cudzego sumienia?
\par 30 A poniewaz ja z dziekowaniem pozywam, czemuz o to mam byc bluzniony, za co ja dziekuje?
\par 31 Przetoz lub jecie lub pijecie, lub cokolwiek czynicie, wszystko ku chwale Bozej czyncie.
\par 32 Nie badzcie obrazeniem i Zydom, i Grekom, i zborowi Bozemu;
\par 33 Jako i ja we wszystkiem sie wszystkim podobam, nie szukajac w tem swego pozytku, ale wielu ich, aby byli zbawieni.

\chapter{11}

\par 1 Badzcie nasladowcami moimi, jakom i ja Chrystusowy;
\par 2 A chwale was, bracia! iz pamietacie wszystkie moje nauki, a jakom wam podal, podania trzymacie.
\par 3 A chce, abyscie wiedzieli, iz kazdego meza glowa jest Chrystus, a glowa niewiasty maz, a glowa Chrystusowa Bóg.
\par 4 Kazdy maz, gdy sie modli albo prorokuje z przykryta glowa, szpeci glowe swoje.
\par 5 I kazda niewiasta, gdy sie modli albo prorokuje, nie nakrywszy glowy swojej, szpeci glowe swoje; boc to jedno, a toz samo jest, jakoby ogolona byla.
\par 6 Albowiem jezli sie nie nakrywa niewiasta, niechze sie tez strzyze; a jezli szpetna rzecz jest niewiescie, strzyc sie albo golic, niechze sie nakrywa.
\par 7 Albowiem maz nie ma nakrywac glowy, gdyz jest wyobrazeniem i chwala Boza; ale niewiasta jest chwala mezowa.
\par 8 Bo maz nie jest z niewiasty, ale niewiasta z meza.
\par 9 Albowiem maz nie jest stworzony dla niewiasty, ale niewiasta dla meza.
\par 10 A przetoz niewiasta powinna miec wladze na glowie dla Aniolów.
\par 11 A wszakze maz nie jest bez niewiasty, ani niewiasta nie jest bez meza w Panu.
\par 12 Albowiem jako niewiasta z meza jest, tak tez maz przez niewiaste; jednak wszystkie rzeczy sa z Boga.
\par 13 Sami u siebie rozsadzcie, przystoili niewiescie bez nakrycia modlic sie Bogu?
\par 14 Azaz was i samo przyrodzenie nie uczy, iz mezowi, gdyby wlosy zapuszczal, jest mu ku zelzywosci?
\par 15 Ale niewiasta, jezli zapuszcza wlosy, jest jej ku poczciwosci, przeto iz jej wlosy dane sa za przykrycie.
\par 16 A jezliby sie kto zdal byc swarliwym, my takiego obyczaju nie mamy, ani zbory Boze.
\par 17 A to opowiadajac nie chwale, ze sie nie ku lepszemu, ale ku gorszemu schodzicie.
\par 18 Albowiem najprzód, gdy sie wy schodzicie we zborze, slysze, iz rozerwania bywaja miedzy wami, i poniekad wierze.
\par 19 Bo musza byc kacerstwa miedzy wami, aby ci, którzy sa doswiadczeni, byli jawnymi miedzy wami.
\par 20 Gdy sie wy tedy wespól schodzicie, nie jest to uzywac wieczerzy Panskiej.
\par 21 Albowiem kazdy wieczerze swoje pierwej zjada i jeden laknie, a drugi jest pijany.
\par 22 Azaz domów nie macie do jedzenia i do picia? Albo zborem Bozym gardzicie i zawstydzacie tych, którzy nie maja? Cóz wam rzeke? Pochwalez was? W tem nie chwale.
\par 23 Albowiem jam wzial od Pana, com tez wam podal, iz Pan Jezus tej nocy, której byl wydan, wzial chleb,
\par 24 A podziekowawszy, zlamal i rzekl: Bierzcie, jedzcie; to jest cialo moje, które za was bywa lamane; to czyncie na pamiatke moje.
\par 25 Takze i kielich, gdy bylo po wieczerzy mówiac: Ten kielich jest nowy testament we krwi mojej; to czyncie, ilekroc pic bedziecie, na pamiatke moje
\par 26 Albowiem ilekroc byscie jedli ten chleb i ten kielich byscie pili, smierc Panska opowiadajcie, azby przyszedl.
\par 27 A tak, kto by jadl ten chleb, albo pil ten kielich Panski niegodnie, bedzie winien ciala i krwi Panskiej.
\par 28 Niechze tedy czlowiek samego siebie doswiadczy, a tak niech je z chleba tego i z kielicha tego niechaj pije.
\par 29 Albowiem kto je i pije niegodnie, sad sobie samemu je i pije, nie rozsadzajac ciala Panskiego.
\par 30 Dlatego miedzy wami wiele jest slabych i chorych, i niemalo ich zasnelo.
\par 31 Bo gdybysmy sie sami rozsadzali, nie bylibysmy sadzeni.
\par 32 Lecz gdy sadzeni bywamy, od Pana bywamy cwiczeni, abysmy z swiatem nie byli potepieni.
\par 33 Ale tak, bracia moi! gdy sie schodzicie ku jedzeniu, oczekiwajcie jedni drugich.
\par 34 A jezli kto laknie, niechze je w domu, abyscie sie na sad nie schodzili. A inne rzeczy, gdy przyjde, postanowie.

\chapter{12}

\par 1 A o duchownych darach, bracia! nie chce, abyscie wiedziec nie mieli.
\par 2 Wiecie, iz gdyscie poganami byli, do balwanów niemych, jako was wodzono, daliscie sie prowadzic.
\par 3 Przetoz oznajmuje wam, iz nikt przez Ducha Bozego mówiac, nie rzecze Jezusa byc przeklestwem; i nikt nie moze nazwac Jezusa Panem, tylko przez Ducha Swietego.
\par 4 A róznec sa dary, ale tenze Duch.
\par 5 I rózne sa poslugi, ale tenze Pan.
\par 6 I rózne sa sprawy, ale tenze Bóg, który sprawuje wszystko we wszystkich.
\par 7 A kazdemu bywa dane objawienie Ducha ku pozytkowi.
\par 8 Albowiem jednemu przez Ducha bywa dana mowa madrosci, a drugiemu mowa umiejetnosci przez tegoz Ducha;
\par 9 A drugiemu wiara w tymze Duchu, a drugiemu dar uzdrawiania w tymze Duchu, a drugiemu czynienie cudów, a drugiemu proroctwo, a drugiemu rozeznanie duchów.
\par 10 A drugiemu rozmaite jezyki, a drugiemu wykladanie jezyków.
\par 11 A to wszystko sprawuje jeden a tenze Duch, udzielajac z osobna kazdemu, jako chce.
\par 12 Albowiem jako cialo jedno jest, a czlonków ma wiele, ale wszystkie czlonki ciala jednego, choc ich wiele jest, sa jednym cialem: tak i Chrystus.
\par 13 Albowiem przez jednego Ducha my wszyscy w jedno cialo jestesmy ochrzczeni, badz Zydowie, badz Grekowie, badz niewolnicy, badz wolni, a wszyscy napojeni jestesmy w jednego Ducha.
\par 14 Albowiem cialo nie jest jednym czlonkiem, ale wieloma.
\par 15 Jezliby rzekla noga: Poniewazem nie jest reka, nie jestem z ciala; izali dlatego nie jest z ciala?
\par 16 A jezliby rzeklo ucho: Poniewazem nie jest okiem, nie jestem z ciala; izali dlatego nie jest z ciala?
\par 17 Jezliz wszystko cialo jest okiem, gdziez sluch? a jezliz wszystko sluchem, gdziez powonienie?
\par 18 Ale teraz Bóg ulozyl czlonki, kazdy z nich z osobna w ciele, jako chcial.
\par 19 A jezliby wszystkie byly jednym czlonkiem, gdziezby bylo cialo?
\par 20 Ale teraz, acz jest wiele czlonków, lecz jedno jest cialo.
\par 21 Nie moze tedy rzec oko rece: Nie potrzebuje ciebie, albo zas glowa nogom: Nie potrzebuje was.
\par 22 I owszem daleko wiecej czlonki, które sie zdadza byc najmdlejsze w ciele, potrzebne sa.
\par 23 A które mamy za najniepoczciwsze w ciele, tym wieksza poczciwosc wyrzadzamy, a niepoczciwe czlonki nasze obfitsza poczciwosc maja.
\par 24 Bo poczciwe czlonki nasze tego nie potrzebuja; lecz Bóg tak umiarkowal cialo, dawszy czlonkowi, któremu czci nie dostaje, obfitsza poczciwosc.
\par 25 Aby nie bylo rozerwania w ciele, ale izby jedne czlonki o drugich jednakiez staranie mialy.
\par 26 A przetoz jezlize jeden czlonek cierpi, cierpia z nim wszystkie czlonki; a jezli bywa uczczony jeden czlonek, raduja sie z nim wszystkie czlonki.
\par 27 Lecz wy jestescie cialem Chrystusowym i czlonkami, kazdy z osobna.
\par 28 A niektórych Bóg postanowil we zborze, najprzód Apostolów, potem proroków, po trzecie nauczycieli, potem cudotwórców, potem dary uzdrawiania, pomocników, rzadców, rozmaitosc jezyków.
\par 29 Izali wszyscy sa Apostolami? Izali wszyscy prorokami? Izali wszyscy nauczycielami? Izali wszyscy cudotwórcami?
\par 30 Izali wszyscy maja dary uzdrawiania? Izali wszyscy jezykami mówia? Izali wszyscy tlumacza?
\par 31 Starajcie sie usilnie o lepsze dary; a ja wam jeszcze zacniejsza droge ukaze.

\chapter{13}

\par 1 Chocbym mówil jezykami ludzkimi i anielskimi, a milosci bym nie mial, stalem sie jako miedz brzakajaca, albo cymbal brzmiacy.
\par 2 I chocbym mial proroctwo i wiedzialbym wszystkie tajemnice, i wszelka umiejetnosc, i chocbym mial wszystke wiare, tak zebym góry przenosil, a milosci bym nie mial, nicem nie jest.
\par 3 I chocbym wynalozyl na zywnosc ubogich wszystke majetnosc moje, i chocbym wydal cialo moje, abym byl spalony, a milosci bym nie mial, nic mi to nie pomoze.
\par 4 Milosc jest dlugo cierpliwa, dobrotliwa jest; milosc nie zajrzy, milosc nie jest rozpustna, nie nadyma sie;
\par 5 Nie czyni nic nieprzystojnego, nie szuka swoich rzeczy, nie jest porywcza do gniewu, nie mysli zlego;
\par 6 Nie raduje sie z niesprawiedliwosci, ale sie raduje z prawdy;
\par 7 Wszystko okrywa, wszystkiemu wierzy, wszystkiego sie spodziewa, wszystko cierpi.
\par 8 Milosc nigdy nie ustaje; bo choc sa proroctwa, te zniszczeja; choc jezyki, te ustana; choc umiejetnosc, wniwecz sie obróci.
\par 9 Albowiem po czesci znamy i po czesci prorokujemy.
\par 10 Ale gdy przyjdzie to, co jest doskonalego, tedy to, co jest po czesci, zniszczeje.
\par 11 Pókim byl dziecieciem, mówilem jako dziecie, rozumialem jako dziecie, rozmyslalem jako dziecie; lecz gdym sie stal mezem, zaniechalem rzeczy dziecinnych.
\par 12 Albowiem teraz widzimy przez zwierciadlo i niby w zagadce; ale na on czas twarza w twarz; teraz poznaje po czesci, ale na on czas poznam, jakom i poznany jest.
\par 13 A teraz zostaje wiara, nadzieja, milosc, te trzy rzeczy; lecz z nich najwieksza jest milosc.

\chapter{14}

\par 1 Nasladujcie milosci, starajcie sie usilnie o dary duchowne; lecz najwiecej, abyscie prorokowali.
\par 2 Albowiem kto mówi jezykiem obcym, nie ludziom mówi, ale Bogu; bo zaden nie slucha, lecz on duchem mówi tajemnice.
\par 3 Ale kto prorokuje, mówi ludziom zbudowanie i napominanie, i pocieche.
\par 4 Kto jezykiem obcym mówi, samego siebie buduje; ale kto prorokuje, ten zbór buduje.
\par 5 A chcialbym, abyscie wy wszyscy jezykami mówili, ale abyscie raczej prorokowali; albowiem wiekszy jest ten, co prorokuje, niz ten, co jezykami obcymi mówi, chyba zeby tlumaczyl, aby zbór bral zbudowanie.
\par 6 Teraz tedy, bracia! gdybym przyszedl do was, jezykami obcymi mówiac, cóz wam pomoge, jezlibym wam nie mówil albo przez objawienie, albo przez umiejetnosc, albo przez proroctwo, albo przez nauke?
\par 7 Wszak i rzeczy niezywe, które dzwiek wydawaja jako piszczalka albo cytra, jezliby róznego dzwieku nie wydawaly, jakoz poznane bedzie, co na piszczalce, albo co na cytrze graja?
\par 8 Albowiem jezliby traba niepewny glos dala, któz sie do boju gotowac bedzie?
\par 9 Takze i wy, jezlibyscie jezykiem nie wydali mowy dobrze zrozumialej, jakoz bedzie zrozumiale, co sie mówi? albowiem bedziecie tylko na wiatr mówic.
\par 10 Tak wiele, jako slyszymy, jest róznych glosów na swiecie, a nic nie jest bez glosu.
\par 11 Jezlibym tedy nie znal mocy glosu, bede temu, który mówi, cudzoziemcem; a ten, co mówi, bedzie mi takze cudzoziemcem.
\par 12 Takze i wy, poniewaz sie usilnie staracie o dary duchowne, szukajciez tego, abyscie obfitowali ku zbudowaniu zboru.
\par 13 Dlatego kto mówi obcym jezykiem, niech sie modli, aby mógl tlumaczyc.
\par 14 Bo jezlibym sie modlil obcym jezykiem, modlic sie bedzie mój duch; ale rozum mój jest bez pozytku.
\par 15 Cóz tedy jest? Bede sie modlil duchem, bede sie tez modlil i wyrozumieniem; bede spiewal duchem, bede tez spiewal i wyrozumieniem.
\par 16 Bo jezlibys blogoslawil duchem, jakoz ten, który jest z pocztu prostaków, na twoje dziekowanie rzecze Amen, poniewaz nie wie, co mówisz?
\par 17 Bo choc ty wprawdzie dobrze dziekujesz, ale sie drugi nie buduje.
\par 18 Dziekuje Bogu mojemu, iz wiecej, niz wy wszyscy, jezykami mówie.
\par 19 A wszakze we zborze wole piec slów zrozumiale przemówic, abym i drugich nauczyl, nizeli dziesiec tysiecy slów jezykiem obcym.
\par 20 Bracia! nie badzcie dziecmi wyrozumieniem, ale badzcie dziecmi zloscia, a wyrozumieniem doroslymi badzcie.
\par 21 W zakonie napisano: Iz obcemi jezykami i obcemi wargami mówic bede ludowi temu; a przecie mnie i tak nie usluchaja, mówi Pan.
\par 22 Przetoz jezyki sa za cud, nie tym, którzy wierza, ale niewiernym; a proroctwo nie niewiernym, ale wierzacym.
\par 23 Jezliby sie tedy wszystek zbór na jedno miejsce zeszedl, a wszyscy by jezykami obcemi mówili, a weszliby tam prostacy albo niewierni, izali nie rzeka, ze szalejecie?
\par 24 Ale jezliby wszyscy prorokowali, a wszedlby który niewierny albo prostak, od wszystkich przekonany i od wszystkich sadzony bywa.
\par 25 A tak skrytosci serca tego bywaja objawione, a on upadlszy na oblicze, poklonili sie Bogu, wyznawajac, ze Bóg jest prawdziwie w was.
\par 26 Cóz tedy jest, bracia? Gdy sie schodzicie, kazdy z was ma psalm, ma nauke, ma jezyk, ma objawienie, ma tlumaczenie; wszystko to niech sie dzieje ku zbudowaniu.
\par 27 Jezli kto jezykiem mówi, niech to bedzie po dwóch albo najwiecej po trzech, i to na przemiany, a jeden niech tlumaczy.
\par 28 A jezliby tlumacza nie bylo niechze we zborze milczy ten, który obcym jezykiem mówi, a niech mówi sobie i Bogu.
\par 29 Ale prorocy niech mówia dwaj albo trzej, a drudzy niech rozsadza.
\par 30 Jezliby tez inszemu siedzacemu co bylo objawione, on pierwszy niechaj milczy.
\par 31 Bo mozecie wszyscy jeden po drugim prorokowac, aby sie wszyscy uczyli i wszyscy pocieszeni byli.
\par 32 I duchy proroków sa poddane prorokom.
\par 33 Albowiem Bóg nie jest powodem nieporzadku, ale pokoju, jako i we wszystkich zborach swietych.
\par 34 Niewiasty wasze niech milcza we zborach; albowiem nie pozwolono im, aby mówily, ale aby poddanemi byly, jako i zakon mówi.
\par 35 A jezli sie czego nauczyc chca, niechze w domu mezów swoich pytaja, poniewaz sromota niewiastom we zborze mówic.
\par 36 Izali od was slowo Boze wyszlo? Izali tylko do was samych przyszlo?
\par 37 Izali kto zda sie byc prorokiem albo duchownym, niech uzna, iz te rzeczy, które wam pisze, sa Panskiem rozkazaniem.
\par 38 A jezli kto nie wie, niechajze nie wie.
\par 39 A tak, bracia! starajcie sie usilnie o to, abyscie prorokowali, a jezykami obcemi mówic nie zabraniajcie.
\par 40 Wszystko sie niech dzieje przystojnie i porzadnie.

\chapter{15}

\par 1 A oznajmuje wam, bracia! Ewangielije, któram wam opowiedzial, którascie tez przyjeli i w której stoicie.
\par 2 Przez która tez zbawienia dostepujecie, jezli pamietacie, jakim sposobem opowiedzialem wam, chyba jezliscie prózno uwierzyli.
\par 3 Albowiem naprzód podalem wam, com tez wzial, iz Chrystus umarl za grzechy nasze wedlug Pism;
\par 4 A iz byl pogrzebiony, a iz zmartwychwstal dnia trzeciego wedlug Pism.
\par 5 A iz widziany jest od Kiefasa, potem od onych dwunastu.
\par 6 Potem widziany jest wiecej niz od pieciuset braci na raz, z których wiele ich zostaje az dotad, a niektórzy tez zasneli.
\par 7 Potem jest widziany od Jakóba, potem od wszystkich Apostolów.
\par 8 A na ostatek po wszystkich ukazal sie i mnie, jako poronionemu plodowi.
\par 9 Bom ja jest najmniejszy z Apostolów, którym nie jest godzien, abym byl zwany Apostolem, przeto zem przesladowal zbór Bozy.
\par 10 Lecz laska Boza jestem tem, czemem jest, a laska jego przeciwko mnie daremna nie byla; alem obficiej niz oni wszyscy pracowal, wszakze nie ja, ale laska Boza, która jest ze mna.
\par 11 Przetoz i ja, i oni tak kazemy, i takescie uwierzyli.
\par 12 A poniewaz sie o Chrystusie kaze, iz z martwych wzbudzony jest, jakoz mówia niektórzy miedzy wami, iz zmartwychwstania nie masz?
\par 13 Bo jezlic zmartwychwstania nie masz, tedyc i Chrystus nie jest wzbudzony.
\par 14 A jezlic Chrystus nie jest wzbudzony, tedyc daremne kazanie nasze, daremna tez wiara wasza.
\par 15 I bylibysmy tez znalezieni falszywymi swiadkami Bozymi, izesmy swiadczyli o Bogu, ze Chrystusa wzbudzil, którego nie wzbudzil, jezlize umarli nie bywaja wzbudzeni.
\par 16 Albowiem jezliz umarli nie bywaja wzbudzeni, i Chrystus nie jest wzbudzony.
\par 17 A jezli Chrystus nie jest wzbudzony, daremna jest wiara wasza i jeszczescie w grzechach waszych;
\par 18 Zatem i ci pogineli, którzy zasneli w Chrystusie.
\par 19 Bo jezli tylko w tym zywocie w Chrystusie nadzieje mamy, nad wszystkich ludzi jestesmy najnedzniejszymi.
\par 20 Lecz teraz Chrystus z martwych wzbudzony jest i stal sie pierwiastkiem tych, którzy zasneli.
\par 21 Bo poniewaz przez czlowieka smierc, przez czlowieka tez powstanie umarlych.
\par 22 Albowiem jako w Adamie wszyscy umieraja, tak i w Chrystusie wszyscy ozywieni beda.
\par 23 Ale kazdy w swoim rzedzie, Chrystus jako pierwiastek, a potem ci, co sa Chrystusowi w przyjscie jego.
\par 24 A potem bedzie koniec, gdy odda królestwo Bogu i Ojcu, gdy zniszczy wszelkie przelozenstwo i wszelka zwierzchnosc, i moc.
\par 25 Bo on musi królowac, póki by nie polozyl wszystkich nieprzyjaciól pod nogi jego.
\par 26 A ostatni nieprzyjaciel, który bedzie zniszczony, jest smierc.
\par 27 Bo wszystkie rzeczy poddal pod nogi jego. A gdy mówi, ze mu wszystkie rzeczy poddane sa, jawna jest, iz oprócz tego, który mu poddal wszystkie rzeczy.
\par 28 A gdy mu wszystkie rzeczy poddane beda, tedyc tez i sam Syn bedzie poddany temu, który mu poddal wszystkie rzeczy, aby byl Bóg wszystkim we wszystkim.
\par 29 Bo inaczej cóz uczynia ci, którzy sie chrzcza nad umarlymi, jezliz zgola umarli nie bywaja wzbudzeni? przeczze sie chrzcza nad umarlymi?
\par 30 Przecz i my niebezpieczenstwa podejmujemy kazdej godziny?
\par 31 Na kazdy dzien umieram przez chwale nasze, która mam w Chrystusie Jezusie, Panu naszym.
\par 32 Jezlizem sie obyczajem ludzkim z bestyjami w Efezie potykal, cóz mam za pozytek, jezli umarli nie bywaja wzbudzeni? Jedzmy i pijmy; boc jutro pomrzemy.
\par 33 Nie bladzciez; zle rozmowy psuja dobre obyczaje.
\par 34 Ocucciez sie ku sprawiedliwosci, a nie grzeszcie; albowiem niektórzy nie maja znajomosci Bozej; ku zawstydzeniu waszemu mówie.
\par 35 Ale rzecze kto: Jakoz wzbudzeni bywaja umarli i w jakim ciele wychodza?
\par 36 O glupi! To, co ty siejesz, nie bywac ozywione, jezliby nie umarlo.
\par 37 I co siejesz, nie siejesz ciala, które ma potem wyrosc, ale gole ziarno, jako sie trafi, albo pszeniczne, albo jakiekolwiek inne.
\par 38 Ale Bóg daje mu cialo jako chce, a kazdemu nasieniu jego wlasne cialo.
\par 39 Nie kazde cialo jest jednakiem cialem; ale inszec jest cialo ludzkie, a insze cialo bydlece, insze rybie, a insze ptasze.
\par 40 I sa ciala niebieskie i ciala ziemskie; lecz insza jest chwala cial niebieskich, a insza ludzkich;
\par 41 Insza chwala slonca, a insza chwala ksiezyca, i insza chwala gwiazd; albowiem gwiazda od gwiazdy rózna jest w jasnosci.
\par 42 Takci bedzie i powstanie umarlych. Bywa wsiane ziarno w skazitelnosci, a bedzie wzbudzone w nieskazitelnosci.
\par 43 Bywa wsiane w nieslawie, a bedzie wzbudzone w slawie; bywa wsiane w slabosci, a bedzie wzbudzone w mocy; bywa wsiane cialo cielesne, a bedzie wzbudzone cialo duchowne.
\par 44 Jest cialo cielesne, jest tez cialo duchowne.
\par 45 Takci tez napisane: Stal sie pierwszy czlowiek Adam w dusze zywa, ale posledni Adam w ducha ozywiajacego.
\par 46 Wszakze nie jest pierwsze duchowne, ale cielesne, potem duchowne.
\par 47 Pierwszy czlowiek z ziemi ziemski; wtóry czlowiek sam Pan z nieba.
\par 48 Jaki jest ten ziemski, tacy tez i ziemscy; a jaki jest niebieski, tacy tez beda niebiescy.
\par 49 A jakosmy nosili wyobrazenie ziemskiego, tak tez bedziemy nosili wyobrazenie niebieskiego.
\par 50 To jednak powiadam, bracia! iz cialo i krew królestwa Bozego odziedziczyc nie moga; ani skazitelnosc nie odziedziczy nieskazitelnosci.
\par 51 Oto ja tajemnice wam powiadam; nie wszyscyc zasniemy, ale wszyscy przemienieni bedziemy, bardzo predko w okamgnieniu, na trabe ostateczna.
\par 52 Albowiem zatrabi, a umarli wzbudzeni beda nieskazitelni, a my bedziemy przemienieni.
\par 53 Boc musi to, co jest skazitelnego, przyoblec nieskazitelnosc, i co jest smiertelnego, przyoblec niesmiertelnosc.
\par 54 A gdy to, co jest skazitelnego, przyoblecze nieskazitelnosc, i to, co jest smiertelnego, przyoblecze niesmiertelnosc, tedy sie wypelni ono slowo, które napisane: Polkniona jest smierc w zwyciestwie.
\par 55 Gdziez jest, o smierci! bodziec twój? Gdziez jest, pieklo! zwyciestwo twoje?
\par 56 Lecz bodziec smierci jest grzech, a moc grzechu jest zakon.
\par 57 Ale niech beda Bogu dzieki, który nam dal zwyciestwo przez Pana naszego Jezusa Chrystusa.
\par 58 A tak, bracia moi mili! badzcie mocni, nieporuszeni, obfitujacy w uczynku Panskim zawsze, wiedzac, iz praca wasza nie jest nadaremna w Panu.

\chapter{16}

\par 1 A okolo skladania na swietych, jakom postanowil we zborach Galickich, tak tez i wy czyncie.
\par 2 Kazdego pierwszego dnia w tygodniu kazdy z was niech odklada u siebie, zbierajac wedlug tego, jako mu sie powodzi, aby nie dopiero, gdy przyjde, skladania czynione byly.
\par 3 A gdy przyjde, którychkolwiek uchwalicie przez listy, tych posle, aby odniesli dobrodziejstwo wasze do Jeruzalemu.
\par 4 A jezliby sluszna rzecz byla, zebym i ja szedl, wespól ze mna pójda.
\par 5 A przyjde do was, gdy przejde Macedonije; (bo pójde przez Macedonije).
\par 6 A podobno zamieszkam u was albo i przezimuje, abyscie wy mie odprowadzili, kedykolwiek pójde.
\par 7 Albowiem nie chce was teraz widziec, mijajac; ale spodziewam sie, iz pomieszkam z wami czas niejaki, bedzieli Pan chcial.
\par 8 A zostane w Efezie az do Swiatek.
\par 9 Albowiem drzwi mi sa otworzone wielkie i mocne, i mam wiele przeciwników.
\par 10 Jezliby tedy przyszedl Tymoteusz, patrzcie, aby bez bojazni byl miedzy wami, bo dzielo Panskie sprawuje, jako i ja.
\par 11 Przetoz niechaj go nikt nie lekcewazy; ale odprowadzcie go w pokoju, aby przyszedl do mnie; bo go czekam z bracmi.
\par 12 A o bracie Apollosie wiedzcie, zem go bardzo prosil, aby szedl do was z bracmi; ale zgola nie mial woli, aby teraz szedl; przyjdzie jednak, gdy czas po temu miec bedzie.
\par 13 Czujciez, stójcie w wierze, meznie sobie poczynajcie, zmacniajcie sie.
\par 14 Wszystkie rzeczy wasze niech sie dzieja w milosci.
\par 15 A prosze was, bracia! wiedzcie, iz dom Stefanowy jest pierwiastkiem Achai, a iz sami siebie oddali na poslugiwanie swietym.
\par 16 Abyscie i wy poddani byli takowym, i kazdemu pomagajacemu i pracujacemu.
\par 17 A ciesze sie z przyjscia Stefana i Fortunata, i Achaika; bo ci niedostatek wasz napelnili.
\par 18 Ochlodzili bowiem ducha mego i waszego; znajciez tedy takowych.
\par 19 Pozdrawiaja was zbory, które sa w Azyi. Pozdrawiaja was wielce w Panu Akwilas i Pryscylla, ze zborem, który jest w domu ich.
\par 20 Pozdrawiaja was bracia wszyscy. Pozdrówcie jedni drugich w swietem pocalowaniu.
\par 21 Pozdrowienie reka moja Pawlowa.
\par 22 Jezlize kto nie miluje Pana Jezusa Chrystusa, niech bedzie przeklestwem, które zowia Maran ata.
\par 23 Laska Pana Jezusa Chrystusa niech bedzie z wami.
\par 24 Milosc moja niech bedzie z wami wszystkimi w Chrystusie Jezusie. Amen.


\end{document}