\begin{document}

\title{2 List do Koryntian}


\chapter{1}

\par 1 Pawel, Apostol Jezusa Chrystusa przez wole Boza, i Tymoteusz brat,zborowi Bozemu, który jest w Koryncie, ze wszystkimi swietymi, którzy sa we wszystkiej Achai.
\par 2 Laska niech bedzie wam i pokój od Boga, Ojca naszego, i Pana Jezusa Chrystusa.
\par 3 Blogoslawiony niech bedzie Bóg i Ojciec Pana naszego, Jezusa Chrystusa, Ojciec milosierdzia a Bóg wszelkiej pociechy,
\par 4 Który nas cieszy w kazdym ucisku naszym, abysmy i my cieszyc mogli i tych, którzy sa w jakimkolwiek ucisku, taz pociecha, która my sami pocieszeni bywamy od Boga.
\par 5 Gdyz jako w nas obfituja utrapienia Chrystusowe, tak przez Chrystusa obfituje i pociecha nasza.
\par 6 Bo choc bywamy ucisnieni, dla waszej to pociechy i zbawienia, które sie sprawuje przez znoszenie tegoz utrapienia, które i my cierpimy; choc tez pocieszeni bywamy, i to dla waszej pociechy i zbawienia;
\par 7 A nadzieja nasza mocna jest o was, poniewaz wiemy, iz jakoscie uczestnikami utrapienia, tak i pociechy.
\par 8 Albowiem nie chcemy, abyscie nie mieli wiedziec, bracia! o ucisku naszym, który nas spotkal w Azyi, izesmy nazbyt byli obciazeni i nad moznosc, tak izesmy byli poczeli watpic i o zywocie.
\par 9 Owszem i sami w sobie mielismy wyrok smierci, abysmy nie ufali sami w sobie, ale w Bogu, który wzbudza umarlych;
\par 10 Który z tak wielkiej smierci wyrwal nas i jeszcze wyrywa, w którym nadzieje mamy, iz i napotem wyrwie;
\par 11 Zwlaszcza gdy sie tez i wy pomozecie modlic za nami, aby za ten dar, który przez wiele osób nam jest pokazany, byly tez od wielu osób dzieki czynione za nas.
\par 12 Albowiem toc jest chluba nasza, swiadectwo sumienia naszego, zesmy w prostocie i w szczerosci Bozej, nie w cielesnej madrosci, ale w lasce Bozej na swiecie obcowali, a najwiecej miedzy wami.
\par 13 Albowiem nic inszego wam nie piszemy, tylko to, co czytacie, albo tez poznawacie, a spodziewam sie, iz tez az do konca poznacie,
\par 14 Jakoscie tez nas poznali po czesci, zesmy chluba wasza, jako i wy nasza w dzien Pana Jezusa.
\par 15 I z tac ufnoscia chcialem byl isc do was najpierwej, abyscie wtóre dobrodziejstwo odebrali;
\par 16 I przez was isc do Macedonii, i zasie z Macedonii przyjsc do was, i od was byc odprowadzony do Judzkiej ziemi.
\par 17 O tem tedy myslac, izalim co lekkomyslnie uczynil? albo to, o czem mysle, izali wedlug ciala mysle, aby bylo u mnie: Tak tak i Nie nie?
\par 18 Alec wierny jest Bóg, iz mowa nasza do was nie byla: Tak i Nie.
\par 19 Albowiem Syn Bozy, Jezus Chrystus, który miedzy wami przez nas jest opowiadany, to jest, przez mie i przez Sylwana, i przez Tymoteusza, nie byl: Tak i Nie; ale Tak w nim bylo.
\par 20 Bo ile jest obietnic Bozych, w nim sa Tak i w nim sa Amen, ku chwale Bozej przez nas.
\par 21 Ale ten, który nas utwierdza z wami w Chrystusie i który nas pomazal, Bóg jest;
\par 22 Który tez zapieczetowal nas i dal zadatek Ducha w serca nasze.
\par 23 Alec ja Boga przyzywam na swiadectwo na dusze moje, iz szanujac was, dotadem nie przyszedl do Koryntu;
\par 24 Nie izbysmy panowali nad wiara wasza, ale iz jestesmy pomocnikami wesela waszego; bo wiara stoicie.

\chapter{2}

\par 1 A postanowilem to u siebie, abym znowu nie przyszedl z zasmuceniem do was.
\par 2 Bo jezlibym ja was zasmucil, i któz jest, co by mie rozweselil, tylko ten, który jest przez mie zasmucony?
\par 3 A tomci wam napisal, abym przyszedlszy, nie mial smutku z tych, z których mialbym sie weselic; pewien bedac o was wszystkich, ze radosc moje wszyscy za swoje macie.
\par 4 Albowiem z wielkiego ucisku i utrapienia serca, i z wiela lez pisalem wam, nie zebyscie mieli byc zasmuceni, ale zebyscie milosc poznali, która nader obficie mam przeciwko wam.
\par 5 Jezli tedy kto zasmucil, nie mniec zasmucil, ale poniekad (abym go nie obciazyl), was wszystkich.
\par 6 Dosycci ma taki na zgromieniu tem, które sie stalo od wielu,
\par 7 Tak iz przeciwnym obyczajem, inaczej byscie mu odpuscic mieli i onego pocieszyc, by snac zbytni smutek takiego nie pozarl.
\par 8 Przetoz prosze was, abyscie milosc przeciwko niemu utwierdzili,
\par 9 Albowiem i dlategom byl napisal, abym doswiadczenia waszego doznal, jezlize we wszystkiem posluszni jestescie.
\par 10 A komu wy co odpuszczacie, temu i ja; gdyz i ja, jezlim co odpuscil temu, komum odpuscil, uczynilem to dla was przed oblicznoscia Chrystusowa, aby was szatan nie podszedl.
\par 11 Albowiem zamysly jego nie sa nam tajne.
\par 12 A gdym przyszedl do Troady dla opowiadania Ewangielii Chrystusowej, a drzwi mi byly otworzone w Panu.
\par 13 Nie mialem ulzenia w duchu moim dlatego, zem nie znalazl Tytusa, brata mego; ale rozstawszy sie z nimi, poszedlem do Macedonii.
\par 14 Lecz chwala Bogu, który nam zawsze zwyciestwo daje w Chrystusie i wonnosc znajomosci swojej objawia przez nas na kazdem miejscu.
\par 15 Bosmy dobra wonnoscia Chrystusowa Bogu w tych, którzy zbawieni bywaja i w tych, którzy gina;
\par 16 Tymci wonnoscia smierci na smierc, ale onym wonnoscia zywota ku zywotowi; lecz do tego, któz jest sposobny?
\par 17 Bo nie jestesmy jako wiele ich, którzy falszuja slowo Boze; ale jako z szczerosci, ale jako z Boga, przed oblicznoscia Boza o Chrystusie mówimy.

\chapter{3}

\par 1 Poczynamyz zasie zalecac samych siebie? czyli potrzebujemy, jako niektórzy, listów zalecajacych do was albo tez listów zalecajacych od was?
\par 2 Listem naszym wy jestescie, napisanym na sercach naszych, który znaja i czytaja wszyscy ludzie.
\par 3 Gdyz to jawna jest, zescie listem Chrystusowym przez usluge nasze zgotowanym, napisanym nie inkaustem, ale Duchem Boga zywego, nie na tablicach kamiennych, ale na tablicach serc miesnych.
\par 4 A takiec ufanie mamy przez Chrystusa ku Bogu,
\par 5 Nie izbysmy byli sposobni, co myslec sami z siebie, jako sami z siebie, ale sposobnosc nasza z Boga jest;
\par 6 Który nas tez uczynil sposobnymi slugami nowego testamentu, nie litery, ale Ducha; albowiem litera zabija, ale Duch ozywia.
\par 7 Bo jezlic poslugiwanie smierci literami wyrazone na tablicach kamiennych bylo chwalebne, tak iz synowie Izraelscy nie mogli smiele patrzec na oblicze Mojzeszowe dla chwaly oblicza jego, która miala byc skazona:
\par 8 Jakoz daleko wiecej uslugiwanie Ducha nie ma byc chwalebne?
\par 9 Bo jezlic uslugiwanie potepienia bylo chwalebne, daleko wiecej uslugiwanie sprawiedliwosci obfituje w chwale.
\par 10 Albowiem i to, co chwale mialo, nie mialo chwaly w tej czesci, co sie tknie onej przewyzszajacej chwaly.
\par 11 Bo jezlic to, co niszczeje, bylo chwalebne, daleko wiecej to, co zostaje, jest chwalebne.
\par 12 Przetoz majac taka nadzieje, wielkiej bezpiecznosci w mowie uzywamy.
\par 13 A nie jestesmy jako Mojzesz, który kladl zaslone na oblicze swoje, aby synowie Izraelscy smiele nie patrzyli na koniec onego, co zniszczec mialo.
\par 14 Ale zatwardzone sa zmysly ich; albowiem az do dzisiejszego dnia taz zaslona w czytaniu starego testamentu zostaje nie odkryta, która przez Chrystusa skazenie bierze.
\par 15 Przetoz az do dnia dzisiejszego, gdy Mojzesz czytany bywa, zaslona jest na sercu ich polozona.
\par 16 Lecz gdyby sie do Pana obrócili, odjeta bedzie ona zaslona,
\par 17 Alec Pan jest tym Duchem; a gdzie jest ten Duch Panski, tam i wolnosc.
\par 18 Lecz my wszyscy, którzy odkrytem obliczem na chwale Panska, jako w zwierciadle patrzymy, w toz wyobrazenie przemienieni bywamy z chwaly w chwale, jako od Ducha Panskiego.

\chapter{4}

\par 1 Dlatego majac to uslugiwanie, tak jakosmy milosierdzie otrzymali, nie slabiejemy.
\par 2 Alesmy sie odrzekli skrytej sromoty, nie obchodzac sie chytrze, ani falszujac slowa Bozego; ale objawieniem prawdy zalecajac samych siebie u kazdego sumienia ludzkiego przed obliczem Bozem.
\par 3 Jezli tedy zakryta jest Ewangielija nasza, zakryta jest przed tymi, którzy gina.
\par 4 W których bóg swiata tego oslepil zmysly, to jest w niewiernych, aby im nie swiecila swiatlosc Ewangielii chwaly Chrystusowej, który jest wyobrazeniem Bozem.
\par 5 Albowiem nie samych siebie opowiadamy, ale Chrystusa Jezusa, ze jest Panem, a samych siebie slugami waszymi dla Jezusa.
\par 6 Poniewaz Bóg, który rzekl, aby sie z ciemnosci swiatlosc rozswiecila, ten sie rozswiecil w sercach naszych ku rozswieceniu (w nas) znajomosci chwaly Bozej w obliczu Jezusa Chrystusa.
\par 7 A mamy ten skarb w naczyniu glinianem, aby dostojnosc tej mocy byla z Boga, a nie z nas.
\par 8 Gdy zewszad ucisnieni bywamy, ale nie bywamy potloczeni; powatpiewamy, ale nie zwatpimy.
\par 9 Przesladowanie cierpimy, ale nie bywamy opuszczeni; bywamy porzuceni, ale nie giniemy.
\par 10 Zawsze umartwienie Pana Jezusowe na ciele nosimy, aby i zywot Jezusowy na ciele naszem byl objawiony.
\par 11 Zawsze bowiem my, którzy zyjemy, bywamy wydawani na smierc dla Jezusa, aby tez zywot Jezusowy byl objawiony w smiertelnem ciele naszem.
\par 12 Dlatego smierc mocy swojej w nas dokazuje, ale w was zywot.
\par 13 Majac tedy tegoz ducha wiary, tak jako napisane: Uwierzylem, przetom tez mówil; i my wierzymy, przeto tez mówimy,
\par 14 Wiedzac, iz ten, który wzbudzil Pana Jezusa, i nas wzbudzi przez Jezusa, i postawi z wami.
\par 15 Boc to wszystko dzieje sie dla was, aby laska ona obfitujaca przez dziekowanie wielu ich rozmnozyla sie ku chwale Bozej.
\par 16 Dlatego nie slabiejmy, ale choc sie tez nasz zewnetrzny czlowiek kazi, wszakze on wewnetrzny sie odnawia ode dnia do dnia.
\par 17 Albowiem ten króciuchny i lekki ucisk nasz nader zacnej chwaly wieczna wage nam sprawuje;
\par 18 Gdy nie patrzymy na rzeczy widzialne, ale na niewidzialne; albowiem rzeczy widzialne sa doczesne, ale niewidzialne sa wieczne.

\chapter{5}

\par 1 Wiemy bowiem, ze jezli tego naszego ziemskiego mieszkania namiot zburzony bedzie, budowanie mamy od Boga, dom nie rekoma urobiony, wieczny w niebiesiech.
\par 2 Albowiem w tym namiocie wzdychamy, domem naszym, który jest z nieba, zadajac byc przyobleczeni.
\par 3 Jezliz tylko przyobleczonymi a nie nagimi znalezieni bedziemy.
\par 4 Bo którzysmy w tym namiocie, wzdychamy, bedac obciazeni, poniewaz nie zadamy byc zewleczeni, ale przyobleczeni, aby pozarta byla smiertelnosc od zywota.
\par 5 A ten, który nas ku temuz wlasnie przygotowal, jestci Bóg, który nam tez dal zadatek Ducha.
\par 6 Przetoz majac zawsze ufnosc i wiedzac, ze póki mieszkamy w tem ciele, pielgrzymujemy od Pana:
\par 7 (Bo przez wiare chodzimy, a nie przez widzenie).
\par 8 Lecz ufamy i wolimy raczej wynijsc z ciala, a isc na mieszkanie do Pana.
\par 9 Przetoz tez usilujemy, badz w ciele mieszkamy, badz z ciala wychodzimy, onemu sie podobac.
\par 10 Albowiem musimy sie wszyscy pokazac przed sadowa stolica Chrystusowa, aby kazdy odniósl, co czynil w ciele, wedlug tego, co czynil, lub dobre, lub zle.
\par 11 Przetoz wiedzac o tym strachu Panskim, ludzi do wiary namawiamy, a Bogu jawnymi jestesmy; i mam nadzieje, iz w sumieniach waszych jawnymi jestesmy.
\par 12 Albowiem nie samych siebie wam znowu zalecamy, ale wam dajemy przyczyne, abyscie sie chlubili nami i zebyscie mieli co mówic przeciwko tym, którzy sie chlubia z powierzchownych rzeczy, a nie z serca.
\par 13 Bo choc od rozumu odchodzimy, Bogu odchodzimy, choc przy rozumie jestesmy, wam jestesmy.
\par 14 Albowiem milosc Chrystusowa przyciska nas, jako tych, którzysmy to osadzili, iz poniewaz jeden za wszystkich umarl, tedy wszyscy byli umarlymi;
\par 15 A ze za wszystkich umarl, aby ci, którzy zyja, juz wiecej sobie nie zyli, ale temu, który za nich umarl i jest wzbudzony.
\par 16 Dlatego my od tego czasu nikogo wedlug ciala nie znamy, a chociasmy tez znali Chrystusa wedlug ciala, lecz juz teraz wiecej nie znamy.
\par 17 A tak jezli kto jest w Chrystusie, nowem jest stworzeniem; stare rzeczy przeminely, oto sie wszystkie nowemi staly.
\par 18 A to wszystko z Boga jest, który nas z samym soba pojednal przez Jezusa Chrystusa i dal nam uslugiwanie tego pojednania,
\par 19 Poniewaz Bóg byl w Chrystusie, swiat z samym soba jednajac, nie przyczytujac im upadków ich, i polozyl w nas to slowo pojednania:
\par 20 Przetoz na miejscu Chrystusowem poselstwo sprawujemy, jakoby was Bóg upominal przez nas, prosimy na miejscu Chrystusowem, jednajcie sie z Bogiem;
\par 21 Albowiem on tego, który nie znal grzechu, za nas grzechem uczynil, abysmy sie my stali sprawiedliwoscia Boza w nim.

\chapter{6}

\par 1 Przetoz pomagajac mu, napominamy was, abyscie nadaremno laski Bozej nie przyjmowali.
\par 2 (Bo mówi Bóg: Czasu przyjemnego wysluchalem cie, a w dzien zbawienia poratowalem cie; oto teraz dzien przyjemny, oto teraz dzien zbawienia.)
\par 3 Zadnego w niczem nie dawajac zgorszenia, aby nie bylo zganione uslugiwanie nasze.
\par 4 Ale we wszystkiem zalecajac samych siebie, jako sludzy Bozy, w wielkiej cierpliwosci, w uciskach, w niedostatkach, w utrapieniach,
\par 5 W razach, w wiezieniach, w potlukaniu, w pracach, w niedosypianiu, w postach,
\par 6 W czystosci, w umiejetnosci, w nieskwapliwosci, w dobrotliwosci, w Duchu Swietym, w milosci nieobludnej;
\par 7 W mowie prawdy, w mocy Bozej, przez oreze sprawiedliwosci na prawo i na lewo;
\par 8 Przez chwale i zelzywosc, przez nieslawe i dobra slawe, jakoby zwodziciele, wszakze prawdziwi;
\par 9 Jako nieznajomi, wszakze znajomi; jako umierajacy, a oto zyjemy; jako pokarani, ale nie zabici;
\par 10 Jako smutni, wszakze zawsze weseli; jako ubodzy, wszakze wielu ubogacajacy; jako nic nie majacy, wszakze wszystko trzymajacy.
\par 11 Usta nasze otworzone sa ku wam, o Koryntowie! serce nasze rozszerzone jest.
\par 12 Nie jestescie scisnieni w nas, lecz scisnieni jestescie we wnetrznosciach waszych.
\par 13 O wzajemna tedy nagrode jako dziatkom mówie: Rozszerzciez sie i wy.
\par 14 Nie ciagnijciez nierównego jarzma z niewiernymi; bo cóz za spolecznosc sprawiedliwosci z nieprawoscia? albo co za spolecznosc swiatlosci z ciemnoscia?
\par 15 A co za zgoda Chrystusa z Belijalem? albo co za dzial wiernemu z niewiernym?
\par 16 A co za zgoda kosciola Bozego z balwanami? Albowiem wy jestescie kosciolem Boga zywego, tak jako mówi Bóg: Bede mieszkal w nich i bede sie przechadzal w nich, i bede Bogiem ich, a oni beda ludem moim.
\par 17 Przetoz wynijdzcie z posrodku ich i odlaczcie sie, mówi Pan, a nieczystego sie nie dotykajcie, a Ja was przyjme.
\par 18 I bede wam za Ojca, a wy mi bedziecie za synów i za córki, mówi Pan wszechmogacy.

\chapter{7}

\par 1 Te tedy obietnice majac, najmilsi! oczyszczajmy samych siebie od wszelakiej zmazy ciala i ducha, wykonywajac poswiecenie w bojazni Bozej.
\par 2 Przyjmijciez nas; nikogosmy nie ukrzywdzili, nikogosmy nie uszkodzili, nikogosmy przez lakomstwo nie podeszli.
\par 3 Nie mówiec tego, abym was potepiac mial; bom przedtem powiedzial, iz wy w sercach naszych tak jestescie, zebysmy radzi z wami spolecznie umierali i spolecznie zyli.
\par 4 Mam wielkie bezpieczenstwo do mówienia u was, mam wielka chlube z was, napelnionym jest pociecha, nader obfituje weselem w kazdym ucisku naszym.
\par 5 Albowiem gdysmy przyszli do Macedonii, cialo nasze zadnego odpoczynku nie mialo, ale we wszystkiem bylismy ucisnieni, zewnatrz walki, a wewnatrz postrachy.
\par 6 Ale Bóg, który cieszy unizonych, pocieszyl nas przez przyjscie Tytusowe.
\par 7 A nie tylko przez przyjscie jego, ale tez przez pocieche, która on ucieszony jest z was, oznajmiwszy nam zadnosc wasze, narzekanie wasze, gorliwosc wasze za mna, tak zem sie tez wiecej uweselil.
\par 8 Bo chociazem was zasmucil przez list, nie zal mi tego, chociaz mi zal bylo; bo widze, iz ten list, chociaz na chwile, zasmucil was byl.
\par 9 Jednak teraz wesele sie, nie dlatego, zescie zasmuceni byli, ale zescie zasmuceni byli w pokucie; albowiem byliscie zasmuceni wedlug Boga, zebyscie w niczem nie szkodowali przez nas.
\par 10 Albowiem smutek, który jest wedlug Boga, pokute sprawuje ku zbawieniu, której nikt nie zaluje; ale smutek wedlug swiata sprawuje smierc.
\par 11 Bo oto to samo, zescie wedlug Boga byli zasmuceni, jako wielka w was pilnosc sprawilo, owszem obrone, owszem zapalczywosc, owszem bojazn, owszem zadnosc, owszem gorliwosc, owszem pomste, tak iz we wszystkiem okazaliscie sie byc czystymi w tej sprawie.
\par 12 A tak chociazem pisal do was, nie pisalem dla tego, który krzywde uczynil, ani dla owego, któremu sie krzywda stala, ale dla tego, izby wam wiadoma byla ona pilnosc nasza o was przed oblicznoscia Boza.
\par 13 Dlategosmy sie ucieszyli z pociechy waszej; alesmy sie wiecej ucieszyli z wesela Tytusowego, i ochlodzony jest duch jego od was wszystkich.
\par 14 A iz jezlim sie w czem przed nim z was chlubil, nie zawstydzilem sie; ale jakosmy wam prawdziwie wszystko mówili, tak sie tez chluba nasza przed Tytusem prawdziwa pokazala.
\par 15 A wnetrznosci jego tem wiecej sklonione sa ku wam, gdy wspomina posluszenstwo wszystkich was, i jakoscie go bojaznia i ze drzeniem przyjeli.
\par 16 Raduje sie tedy, iz wam we wszystkiem moge zaufac.

\chapter{8}

\par 1 A oznajmujemy wam, bracia! o lasce Bozej, która jest dana zborom Macedonskim;
\par 2 Iz w rozlicznem doswiadczeniu utrapienia obfita ich radosc i bardzo wielkie ubóstwo ich obfitowalo w bogactwo szczerosci ich.
\par 3 Bo daje im swiadectwo, ze wedlug moznosci i nad moznosc ochotnymi sie pokazali.
\par 4 Z wielka prosba nas zadajac, abysmy to dobrodziejstwo i spólne udzielenie, którem sie usluguje swietym, przyjeli.
\par 5 A nie tylko tak sobie postapili, jakosmy sie spodziewali; ale najprzód samych siebie oddali Panu, potem i nam za wola Boza.
\par 6 Tak, zesmy musieli napomniec Tytusa, aby jako przedtem poczal, tak aby tez dokonal u was tegoz dobrodziejstwa.
\par 7 Przetoz jako we wszystkiem obfitujecie w wierze i w mowie, i w umiejetnosci, i we wszelakiej pilnosci, i w milosci waszej przeciwko nam, tak i w tem dobrodziejstwie obfitujcie.
\par 8 Nie mówie jako rozkazujac, ale przez pilnosc innych, jako jest szczera milosc wasza, na jawia wystawiajac.
\par 9 Albowiem znacie laske Pana naszego, Jezusa Chrystusa, ze dla was stal sie ubogim, bedac bogatym, abyscie wy ubóstwem jego ubogaceni byli.
\par 10 A w tem podaje wam zdanie swoje; albowiem to wam jest pozyteczno, którzy nie tylko czynic, ale i chciec przedtemescie poczeli roku przeszlego.
\par 11 A teraz to, coscie czynic poczeli, wykonajcie, aby jako byla ochotna mysl ku chceniu, tak tez aby bylo i dokonczenie z tego, co macie.
\par 12 Albowiem jezli przedtem byla ochotna mysl, taz przyjemna jest wedlug tego, co kto ma, a nie wedlug tego, czego nie ma.
\par 13 Bo nie chce, aby insi mieli ulzenie a wy ucisnienie, ale zeby za równo natenczas wasza obfitosc ich niedostatkowi usluzyla;
\par 14 Aby tez ich obfitosc waszemu niedostatkowi usluzyla, zeby sie stalo porównanie,
\par 15 Jako napisane: Kto wiele nazbieral, nie mial nazbyt; a kto malo nazbieral, nie mial malo.
\par 16 Ale chwala Bogu, który dal takiez staranie o was do serca Tytusowego,
\par 17 Iz ono napomnienie przyjal, a stawszy sie pilniejszym, dobrowolnie poszedl do was.
\par 18 A poslalismy wespól z nim brata, który ma chwale w Ewangielii po wszystkich zborach;
\par 19 A nie tylko to, ale obrany jest przez glosy od zborów, za towarzysza drogi naszej z tem dobrodziejstwem, którem sie dzieje usluga od nas ku chwale samego Pana i ku doswiadczeniu ochotnego umyslu waszego,
\par 20 Uchodzac tego, aby nam kto nie przyganil dla tej obfitosci, która sie przez nas usluguje,
\par 21 Pilnie sie starajac o uczciwe rzeczy, nie tylko przed Panem, ale tez i przed ludzmi.
\par 22 A poslalismy z nimi brata naszego, któregosmy czesto doswiadczyli, w wielu rzeczach byc pilnym, a teraz daleko pilniejszym dla wielkiej dowiernosci, która ma przeciwko wam.
\par 23 A jezli idzie o Tytusa, ten jest moim towarzyszem i u was pomocnikiem; a jezli tez o braci naszych, poslami sa zborów i chwala Chrystusowa.
\par 24 Przetoz oswiadczenie milosci waszej i chluby naszej z was pokazcie przeciwko nim przed oblicznoscia zborów.

\chapter{9}

\par 1 Lecz o usludze, która sie dzieje swietym, niepotrzebna mi jest wam pisac.
\par 2 Bo znam ochote umyslu waszego, która sie ja chlubie z was u Macedonczyków, iz Achaja gotowa byla od przeszlego roku; a ta wasza gorliwosc wiele ich pobudzila.
\par 3 Poslalem tedy tych braci, zeby chluba nasza, która mamy z was, nie byla daremna z tej miary, ale abyscie (jakom powiedzial), gotowymi byli;
\par 4 Abysmy snac, jezliby ze mna przyszli Macedonczycy, a znalezli was niegotowymi, nie zawstydzili sie my, (ze nie rzeke wy), za tak bezpieczna chlube.
\par 5 Zdalo mi sie tedy za rzecz potrzebna, napomniec braci, aby do was wprzód poszli i pierwej zgotowali przedtem opowiedziana wasze szczodrobliwosc, aby byla gotowa tak jako szczodrobliwosc, a nie jako rzecz przymuszona.
\par 6 Ale tak mówie: Kto skapo sieje, skapo tez zac bedzie; a kto obficie sieje, obficie tez zac bedzie.
\par 7 Kazdy jako umyslil w sercu swem, tak niech uczyni, nie z zamarszczeniem ani z przymuszenia; albowiem ochotnego dawce Bóg miluje.
\par 8 A mocen jest Bóg uczynic, aby obfitowala na was wszelka laska, abyscie majac we wszystkiem zawsze wszelaki dostatek, obfitowali ku wszelakiemu uczynkowi dobremu,
\par 9 Jako napisane: Rozproszyl, dal ubogim, sprawiedliwosc jego zostaje na wieki.
\par 10 A ten, który daje nasienie siejacemu, niechze i wam da chleb ku jedzeniu i rozmnozy nasienie wasze, i przysporzy urodzajów sprawiedliwosci waszej,
\par 11 Abyscie z kazdej miary byli ubogaceni ku wszelkiej prostosci, która sprawuje przez nas, aby dzieki Bogu czynione byly.
\par 12 Albowiem uslugiwanie tej ofiary nie tylko dopelnia niedostatki swietych, ale tez oplywa przez wielkie dziekczynienia na Boga przez pochwale tej poslugi;
\par 13 Gdy Boga chwala za wasze poddanstwo Ewangielii Chrystusowej, za szczerosc w udzielaniu przeciwko sobie i przeciwko wszystkim innym;
\par 14 I modla sie za wami, zadajac was dla laski Bozej obfitujacej w was.
\par 15 Lecz Bogu niech bedzie chwala za niewypowiedziany dar jego.

\chapter{10}

\par 1 Ja tez Pawel sam was prosze przez cichosc i dobrotliwosc Chrystusowa, który gdym jest wam przytomny, jestem pokorny miedzy wami; lecz gdym nie jest przytomny, jestem smialy przeciwko wam.
\par 2 A prosze, abym bedac przytomnym, nie musial byc smialy ta smialoscia, o której mysle, abym smialy byl przeciwko niektórym, którzy nas szacuja, jakobysmy wedlug ciala chodzili.
\par 3 Albowiem w ciele chodzac, nie wedlug ciala walczymy,
\par 4 (Albowiem bron zolnierstwa naszego nie jest cielesna, ale z Boga jest, mocna ku zburzeniu miejsc obronnych.)
\par 5 Burzac rady i wszelaka wysokosc, wynoszaca sie przeciwko znajomosci Bozej, i podbijajac wszelaka mysl pod posluszenstwo Chrystusowe;
\par 6 I w pogotowiu majac pomste na wszelakie nieposluszenstwo, gdy sie wypelni posluszenstwo wasze.
\par 7 Na toz tylko, co przed oczyma jest, patrzycie? Mali kto te nadzieje o sobie, iz jest Chrystusowy, niechze tez to sam z siebie uwaza, iz jako on jest Chrystusowy, tak tez i my Chrystusowymi jestesmy.
\par 8 Albowiem chocbym sie ja tez co wiecej chelpil z mocy naszej, która nam dal Pan ku zbudowaniu, a nie ku zepsowaniu waszemu, nie zawstydze sie;
\par 9 Abym sie nie zdal, jakobym was straszyl przez listy.
\par 10 Albowiem mówia: Listy wazne sa i potezne, ale ciala obecnosc niepotezna jest i mowa nieplatna.
\par 11 To niechaj mysli taki, iz jakimismy w mowie przez listy, nie bedac obecnymi, takimiz bedziemy i w uczynku, bedac obecnymi.
\par 12 Albowiem nie smiemy samych siebie w poczet drugich klasc, albo porównywac z niektórymi, którzy sami siebie zalecaja; ale i ci nie zrozumiewaja, iz sie sami soba miarkuja i sami sie do siebie przyrównywaja.
\par 13 Ale my nie bedziemy sie chlubili nad miare, ale wedlug sznuru miary, która miare wymierzyl nam Bóg, tak zesmy dosiegli az do was,
\par 14 Bo sie nie rozciagamy nad miare, jakobysmy nie dosiegli az do was; bosmy az i do was przyszli w Ewangielii Chrystusowej.
\par 15 A nie chlubimy sie nad miare z cudzych prac; ale majac nadzieje, iz gdy sie pomnozy wiara wasza w was, pomnozymy sie i my miedzy wami wedlug sznuru naszego z obfitoscia,
\par 16 Ku opowiadaniu Ewangielii w onych krainach, które leza za wami, nie chlubiac sie z rzeczy gotowych cudzego pomiaru.
\par 17 Kto sie tedy chlubi, niech sie Panu chlubi.
\par 18 Albowiem nie ten, co sie sam zaleca, doswiadczony jest, ale ten, którego Pan zaleca.

\chapter{11}

\par 1 Obyscie chcieli na chwile znosic glupstwo moje! ale i znaszajcie mie.
\par 2 Albowiem gorliwym jestem ku wam gorliwoscia Boza; bom was przygotowal, abym was stawil czysta panna jednemu mezowi Chrystusowi.
\par 3 Lecz boje sie, by snac jako waz oszukal Ewe chytroscia swoja, tak tez skazone mysli wasze nie odpadly od prostoty onej, która jest w Chrystusie.
\par 4 Bo gdyby kto przyszedl, co by inszego Jezusa opowiadal, któregosmy nie opowiadali; albo gdybyscie innego ducha wzieli, któregoscie nie wzieli, albo insza Ewangielije, którejscie nie przyjeli, dobrze byscie go znosili.
\par 5 Boc mam za to, zem nie byl w niczem podlejszy, nizeli oni bardzo wielcy Apostolowie.
\par 6 Bo chociazem tez i prostakiem w mowie, wszakze nie w umiejetnosci; ale zgola jawnymismy sie stali we wszystkich rzeczach u was.
\par 7 Izalim sie grzechu dopuscil, zem samego siebie unizyl, abyscie wy byli wywyzszeni, a zem wam darmo Ewangielije Boza opowiadal?
\par 8 Zlupilem inne zbory, biorac od nich zold, abym wam sluzyl; a bedac u was i cierpiac niedostatek, nie obciazylem próznujac nikogo.
\par 9 Albowiem niedostatek mój dopelnili bracia, którzy przyszli z Macedonii, i we wszystkim strzeglem sie, abym wam ciezkim nie byl, i na potem strzec sie bede.
\par 10 Jestci prawda Chrystusowa we mnie, iz ta chluba nie bedzie zatlumiona we mnie w krainach Achajskich.
\par 11 Dlategoz? czy ze was nie miluje? Bógci wie,
\par 12 Ale co czynie, czynic jeszcze bede dlatego, abym odcial przyczyne tym, którzy przyczyny szukaja, aby w tem, z czego sie chlubia, byli znalezieni tacy, jako i my.
\par 13 Albowiem takowi falszywi Apostolowie sa robotnicy zdradliwi, którzy sie przemieniaja w Apostoly Chrystusowe.
\par 14 A nie dziw: bo i szatan sam przemienia sie w Aniola swiatlosci.
\par 15 Nie wielka tedy, jezli tez sludzy jego przemieniaja sie, jakoby byli slugami sprawiedliwosci, których koniec bedzie podlug uczynków ich.
\par 16 Znowu powiadam, zeby mie kto nie mial za glupiego; jezliz inaczej, wiec jako glupiego przyjmijcie mie, abym sie ja tez nieco maluczko przechwalal.
\par 17 Co mówie, nie mówiec jako od Pana, ale jako w glupstwie z strony tej bezpiecznej chluby.
\par 18 Poniewaz sie ich wiele chlubi wedlug ciala, i ja sie chlubic bede.
\par 19 Bo radzi znosicie glupich, bedac sami madrymi.
\par 20 Bo znosicie, choc was kto zniewala, choc kto pozera, choc kto bierze, choc sie kto wynosi, choc was kto policzkuje.
\par 21 Mówiac wedlug zelzywosci, jakobysmy byli slabymi; lecz w czem kto smialym jest, (w glupstwie mówie) i jam smialy.
\par 22 Zydowie sa, jestem i ja. Izraelczycy sa, jestem i ja. Nasieniem Abrahamowem sa, jestem i ja.
\par 23 Slugami Chrystusowymi sa, (glupio mówie), wiecej ja; w pracach obficiej, w razach nad miare, w wiezieniach obficiej, w smierciach czestokroc.
\par 24 Od Zydów wzialem pieciokroc po czterdziesci plag bez jednej.
\par 25 Trzykrociem byl bity rózgami; razem byl kamionowany; trzykroc sie ze mna okret rozbil, dzien i noc bylem w glebokosci morskiej;
\par 26 W drogach czestokroc, w niebezpieczenstwach na rzekach, w niebezpieczenstwach od zbójców, w niebezpieczenstwach od swego narodu, w niebezpieczenstwach od pogan, w niebezpieczenstwach w miescie, w niebezpieczenstwach na puszczy, w niebezpieczenstwach na morzu, w niebezpieczenstwach miedzy falszywymi bracmi;
\par 27 W pracy i w utrapieniu, w niedosypianiu czesto, w glodzie, i w pragnieniu, w postach czesto, i w zimnie, i w nagosci;
\par 28 Oprócz tego, co skadinad przyda, ono naleganie na mie na kazdy dzien i ono staranie o wszystkie zbory.
\par 29 Któz choruje, a ja nie choruje? Któz sie gorszy, a ja nie pale?
\par 30 Jezli sie mam chlubic, z krewkosci moich chlubic sie bede.
\par 31 Bóg i Ojciec Pana naszego, Jezusa Chrystusa, który jest blogoslawiony na wieki, wie, iz nie klamie.
\par 32 W Damaszku hetman ludu króla Arety osadzil byl straza miasto Damaszek, chcac mie pojmac; alem oknem po powrozie w koszyku przez mur byl spuszczony i uszedlem rak jego.

\chapter{12}

\par 1 Wprawdziec mi sie chlubic nie jest pozyteczno: jednak przyjde do widzenia i objawienia Panskiego.
\par 2 Znam czlowieka w Chrystusie przed czternastoma laty, (jezli sie to dzialo w ciele, nie wiem, jezli oprócz ciala, nie wiem, Bóg wie), który zachwycony byl az do trzeciego nieba.
\par 3 A znam takiego czlowieka, (Jezli sie to dzialo w ciele, jezli oprócz ciala, nie wiem, Bóg wie).
\par 4 Iz byl zachwycony do raju i slyszal niewypowiedziane slowa, których czlowiekowi nie godzi sie mówic.
\par 5 Z takiego czlowieka chlubic sie bede; ale z siebie samego chlubic sie nie bede, tylko z krewkosci moich.
\par 6 Albowiem jezlibym sie chcial chlubic, nie bede glupi, bo prawde powiem, ale wstrzymam sie, aby kto o mnie nie rozumial nad to, czem mie byc widzi, albo co slyszy ode mnie.
\par 7 A izbym sie zacnoscia objawienia nader nie wynosil, dany mi jest bodziec cialu, aniol szatan, aby mie policzkowal, zebym sie nad miare nie wynosil.
\par 8 Dlategom trzykroc Pana prosil, aby odstapil ode mnie.
\par 9 Ale mi rzekl: Dosyc masz na lasce mojej; albowiem moc moja wykonywa sie w slabosci. Raczej sie tedy wiecej chlubic bede z krewkosci moich, aby we mnie mieszkala moc Chrystusowa.
\par 10 Dlatego sie kocham w krewkosciach, w potwarzach, w niedostatkach, w przesladowaniach, i w uciskach dla Chrystusa; bo gdym jest slaby, tedym jest mocny.
\par 11 Stalem sie glupim, chlubiac sie; wyscie mie do tego przymusili. Bom ja od was mial byc chwalony, poniewazem nie byl podlejszym, nizeli oni bardzo wielcy Apostolowie, chociazem nic nie jest.
\par 12 Jednak znaki Apostola pokazane sa u was we wszelkiej cierpliwosci, w znamionach i w cudach, i w mocach.
\par 13 Bo cóz jest, w czem byscie podlejsi byli nad insze zbory, tylko zem ja sam próznujac, nie obciazyl was? Odpuscciez mi te krzywde.
\par 14 Oto trzeci raz jestem gotów przyjsc do was, a nie obciaze was próznujac; albowiem nie szukam tego, co jest waszego, ale w was samych; boc nie dziatki maja rodzicom skarbic, ale rodzice dziatkom.
\par 15 Lecz ja bardzo rad naklad uczynie i samego siebie wynaloze za dusze wasze, aczkolwiek bardzo was milujac, mniej bywam od was milowany.
\par 16 Ale niech tak bedzie; jam was nie obciazyl, tylko chytrym bedac, zdradam was pojmal.
\par 17 Izalim was przez którego z tych, którychem do was poslal, oszukal?
\par 18 Uprosilem Tytusa i poslalem z nim brata tego. Izali was Tytus oszukal? Izalismy nie jednym duchem postepowali? Izali nie jednemi stopami?
\par 19 Znowuz mniemacie, ze sie przed wami obawiamy? Przed oblicznoscia Boza w Chrystusie mówimy, a to wszystko, najmilsi! dla waszego zbudowania.
\par 20 Bo sie boje, bym snac przyszedlszy, nie znalazl was takimi, jakimi bym nie chcial, a ja tez zebym nie byl znaleziony od was, jakiego byscie wy nie chcieli; by snac miedzy wami nie bylo swarów, zazdrosci, gniewów, zwad, obmowisk, mruczenia, nadymania i rozterków;
\par 21 By mie zasie Bóg mój, gdy przyjde, nie ponizyl u was, i zalowalbym wielu tych, którzy przedtem grzeszyli, a nie pokutowali z nieczystosci i z wszeteczenstwa, i z rozpusty, która popelnili.

\chapter{13}

\par 1 Trzeci to juz raz ide do was. W ustach dwóch lub trzech swiadków stanie kazde slowo.
\par 2 Powiedzialem przedtem i znowu powiadam jako powtóre obecny, a teraz nie bedac obecny pisze tym, którzy przedtem grzeszyli i wszystkim inszym, ze jezli znowu przyjde, nie przepuszcze im.
\par 3 Poniewaz chcecie doswiadczyc tego, który we mnie mówi, Chrystusa, który ku wam nie jest slaby, ale mocny jest w was.
\par 4 Bo aczkolwiek ukrzyzowany jest jako slaby, ale zyje z mocy Bozej, a tak i my jestesmy z nim slabymi, ale zyc bedziemy z nim z mocy Bozej przeciwko wam.
\par 5 Doswiadczajcie samych siebie, jezli jestescie w wierze, samych siebie doznawajcie. Czyli samych siebie znacie, ze Jezus Chrystus w was jest? chyba zebyscie byli odrzuceni.
\par 6 Mam jednak nadzieje, ze poznacie, iz my nie jestesmy odrzuconymi.
\par 7 I modle sie Bogu, abyscie nic zlego nie czynili; nie izbysmy sie my doswiadczonymi okazali, ale abyscie wy to, co jest dobrego, czynili, a my zebysmy byli jako odrzuceni.
\par 8 Boc nie mozemy nic przeciwko prawdzie, ale za prawda.
\par 9 Albowiem sie radujemy, ze chociasmy my slabymi, ale wy jestescie mocnymi; a tegoc i zyczymy, abyscie wy byli doskonalymi.
\par 10 Przetoz to pisze, nie bedac obecny, abym bedac obecnym, nie musial uzywac srogosci wedlug mocy, która mi dal Pan na zbudowanie, a nie na zepsowanie.
\par 11 Na ostatek, bracia! miejcie sie dobrze; doskonalymi badzcie, cieszcie sie, jednomyslnymi badzcie, w pokoju zyjcie, a Bóg milosci i pokoju bedzie z wami. Pozdrówcie jedni drugich swietem pocalowaniem.
\par 12 Pozdrawiaja was wszyscy swieci.
\par 13 Laska Pana Jezusa Chrystusa i milosc Boza, i spolecznosc Ducha Swietego niech bedzie z wami wszystkimi. Amen.


\end{document}