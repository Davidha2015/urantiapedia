\begin{document}

\title{List do Galatów}


\chapter{1}

\par 1 Pawel, Apostol (nie od ludzi, ani przez czlowieka, ale przez Jezusa Chrystusa i Boga Ojca, który go wzbudzil od umarlych;)
\par 2 I wszyscy bracia, którzy sa ze mna, zborom Galackim.
\par 3 Laska wam i pokój niech bedzie od Boga Ojca i Pana naszego, Jezusa Chrystusa.
\par 4 Który wydal samego siebie za grzechy nasze, aby nas wyrwal z terazniejszego wieku zlego wedlug woli Boga i Ojca naszego;
\par 5 Któremu niech bedzie chwala na wieki wieków. Amen.
\par 6 Dziwuje sie, iz tak predko dacie sie przenosic od tego, który was powolal ku lasce Chrystusowej, do inszej Ewangielii;
\par 7 Która nie jest insza; tylko niektórzy sa, co was turbuja i chca wywrócic Ewangielije Chrystusowa.
\par 8 Ale chocbysmy i my, albo Aniol z nieba opowiadal wam Ewangielije mimo te, którasmy wam opowiadali, niech bedzie przeklety.
\par 9 Jakosmy przedtem powiedzieli i teraz znowu mówie: Jezliby wam kto inna Ewangielije opowiadal mimo te, którascie przyjeli, niech bedzie przeklety.
\par 10 Albowiem terazze do ludzi was namawiam, czyli do Boga? Albo szukamli, abym sie podobal ludziom? Zaiste, jezlibym sie jeszcze ludziom chcial podobac, nie bylbym sluga Chrystusowym.
\par 11 A oznajmuje wam bracia! iz Ewangielija, która jest opowiadana ode mnie, nie jest wedlug czlowieka.
\par 12 Albowiem ja anim jej wzial, anim sie jej nauczyl od czlowieka, ale przez objawienie Jezusa Chrystusa.
\par 13 Boscie slyszeli o mojem obcowaniu niekiedy w Zydostwie, zem nader przesladowal zbór Bozy i burzylem go;
\par 14 I postepowalem w Zydostwie nad wiele rówiesników moich w narodzie moim, bedac nader gorliwym milosnikiem ustaw moich ojczystych.
\par 15 Ale gdy sie upodobalo Bogu, który mie odlaczyl z zywota matki mojej, i powolal z laski swojej,
\par 16 Aby objawil Syna swego we mnie, abym go opowiadal miedzy poganami, wnetze nie radzilem sie ciala i krwi;
\par 17 Anim sie wrócil do Jeruzalemu, do tych, którzy przede mna byli Apostolami, alem szedl do Arabii i wrócilem sie zasie do Damaszku.
\par 18 Potem po trzech latach wstapilem do Jeruzalemu, abym sie ujrzal z Piotrem; i mieszkalem u niego pietnascie dni.
\par 19 A inszegom z Apostolów nie widzial, oprócz Jakóba, brata Panskiego.
\par 20 A co wam pisze, oto sie przed Bogiem oswiadczam, zec nie klamie.
\par 21 Zatemem przyszedl do krain Syryi i Cylicyi;
\par 22 A bylem nieznajomym z twarzy zborom zydowskim, które sa w Chrystusie;
\par 23 Lecz tylko byli uslyszeli, iz ten, który przesladowal nas niekiedy, teraz opowiada wiare, która przedtem burzyl. I chwalili Boga ze mnie.

\chapter{2}

\par 1 Potem po czternastu latach wstapilem zasie do Jeruzalemu z Barnabaszem, wziawszy z soba i Tytusa.
\par 2 A wstapilem wedlug objawienia i przelozylem im Ewangielije, która kaze miedzy poganami, a zwlaszcza zacniejszym, bym snac nadaremno nie biezal, albo przedtem nie biegal.
\par 3 Ale ani Tytus, który ze mna byl, bedac Grekiem, nie byl przymuszony obrzezac sie,
\par 4 A to dla wprowadzonych falszywych braci, którzy sie wkradli, aby wyszpiegowali wolnosc nasze, która mamy w Chrystusie Jezusie, aby nas w niewole podbili.
\par 5 Którymesmy i na chwilke nie ustapili, i nie poddali sie, aby u was prawda Ewangielii zostala.
\par 6 A od tych, którzy sie zdadza byc czems, (acz jakimi niekiedy byli, nic mi na tem; bo osoby czlowieczej Bóg nie przyjmuje), ci mówie, którzy sie zdali byc czems, nic mi nie przydali.
\par 7 Owszem, przeciwnym obyczajem, widzac, iz mi jest zwierzona Ewangielija miedzy nieobrzezanymi, jako Piotrowi miedzy obrzezanymi,
\par 8 (Albowiem ten, który byl skuteczny przez Piotra w apostolstwie miedzy obrzezanymi, skuteczny byl i we mnie miedzy poganami.)
\par 9 I poznawszy laske mnie dana, Jakób i Kiefas, i Jan, którzy sie zdadza byc filarami, podali prawice mnie i Barnabaszowi na towarzystwo, abysmy my miedzy poganami, a oni miedzy obrzezanymi kazali.
\par 10 Tylko upomnieli, abysmy na ubogich pamietali, o com sie tez pilnie staral, abym to uczynil.
\par 11 A gdy przyszedl Piotr do Antyjochii, sprzeciwilem sie mu w twarz; i byl godzien nagany.
\par 12 Albowiem przedtem, niz przyszli niektórzy od Jakóba, wespól z poganami jadal; a gdy ci przyszli, schranial sie i odlaczal, bojac sie tych, którzy byli z obrzezania.
\par 13 A wespól z nim obludnie sie obchodzili i drudzy Zydzi, tak ze i Barnabasz uwiedziony byl ta ich obluda.
\par 14 Ale gdym obaczyl, iz nie prosto chodza w prawdzie Ewangielii, rzeklem Piotrowi przed wszystkimi: Poniewaz ty, bedac Zydem, po pogansku zyjesz a nie po zydowsku, czemuz pogan przymuszasz po zydowsku zyc?
\par 15 My, którzysmy z przyrodzenia Zydowie a nie z pogan grzesznicy,
\par 16 Wiedzac, iz nie bywa usprawiedliwiony czlowiek z uczynków zakonu, ale przez wiare w Jezusa Chrystusa, i mysmy w Jezusa Chrystusa uwierzyli, abysmy byli usprawiedliwieni z wiary Chrystusowej, a nie z uczynków zakonu, przeto ze nie bedzie usprawiedliwione z uczynków zakonu zadne cialo.
\par 17 A jezli my szukajac, abysmy byli usprawiedliwieni w Chrystusie, znajdujemy sie tez grzesznikami, tedyc Chrystus jest sluga grzechu? Nie daj tego Boze!
\par 18 Albowiem jezli to, com zburzyl, znowu zasie buduje, przestepca samego siebie czynie.
\par 19 Bom ja przez zakon zakonowi umarl, abym zyl Bogu.
\par 20 Z Chrystusem jestem ukrzyzowany, a zyje juz nie ja, lecz zyje we mnie Chrystus; a to ze teraz w ciele zyje, w wierze Syna Bozego zyje, który mie umilowal i wydal samego siebie za mie.
\par 21 Nie odrzucam tej laski Bozej; bo jezli przez zakon jest sprawiedliwosc, tedyc Chrystus prózno umarl.

\chapter{3}

\par 1 O glupi Galatowie! Któz was omamil, abyscie prawdzie nie byli posluszni, którym przed oczyma Jezus Chrystus przedtem byl wymalowany, i miedzy wami ukrzyzowany?
\par 2 Tego tylko rad bym sie nauczyl od was: Z uczynkówli zakonu wzieliscie Ducha, czyli z sluchania wiary?
\par 3 Takescie glupi? poczawszy duchem, teraz cialem dokonywacie?
\par 4 Takescie wiele cierpieli darmo, jezli tylko i darmo?
\par 5 Ten tedy, który wam dodaje ducha i czyni cuda miedzy wami, z uczynkówze zakonu to czyni, czyli z sluchania wiary?
\par 6 Tak jako "Abraham uwierzyl Bogu i przyczytano mu to ku sprawiedliwosci".
\par 7 Widzicie tedy, ze ci, którzy sa z wiary, ci sa synami Abrahamowymi.
\par 8 A upatrzywszy to Pismo, iz z wiary Bóg usprawiedliwia pogan, przedtem opowiedzialo Abrahamowi, iz w tobie beda blogoslawione wszystkie narody.
\par 9 A tak ci, którzy sa z wiary, dostepuja blogoslawienstwa z wiernym Abrahamem.
\par 10 Albowiem ile ich jest z uczynków zakonu, pod przeklestwem sa; bo napisane: Przeklety kazdy, który by nie zostal we wszystkiem, co napisane w ksiegach zakonu, aby to czynil.
\par 11 A iz przez zakon nikt nie bywa usprawiedliwiony przed Bogiem, jawna jest stad, bo "sprawiedliwy z wiary zyc bedzie".
\par 12 Ale zakon nie jestci z wiary; lecz "czlowiek, który by je czynil, zyc bedzie przez nie".
\par 13 Ale Chrystus odkupil nas z przeklestwa zakonu, stawszy sie za nas przeklestwem, (albowiem napisane: Przeklety kazdy, który wisi na drzewie).
\par 14 Aby na pogan blogoslawienstwo Abrahamowe przyszlo w Chrystusie Jezusie, i abysmy obietnice Ducha wzieli przez wiare.
\par 15 Bracia! po ludzku mówie: a wszak i czlowieczego testamentu utwierdzonego nikt nie lamie, ani do niego co przydaje.
\par 16 Lecz Abrahamowi uczynione sa obietnice i nasieniu jego; nie mówi: I nasieniom jego, jako o wielu, ale jako o jednem: I nasieniu twemu, które jest Chrystus.
\par 17 To tedy mówie, iz przymierza przedtem od Boga utwierdzonego wzgledem Chrystusa, zakon, który po czterechset i po trzydziestu lat nastal, nie znosi, aby mial zepsuc obietnice Boza.
\par 18 Albowiem jezliz z zakonu jest dziedzictwo, juzci nie z obietnicy; lecz Abrahamowi przez obietnice darowal je Bóg.
\par 19 Cóz tedy zakon? Dla przestepstwa przydany jest, azby przyszlo ono nasienie, któremu sie stala obietnica, sporzadzony przez Aniolów i przez reke posrednika.
\par 20 Lecz posrednik nie jest jednego, ale Bóg jeden jest.
\par 21 Zakon tedy jestze przeciwko obietnicom Bozym? Nie daj tego Boze! albowiem gdyby byl dany zakon, który by mógl ozywiac, prawdziwiec by z zakonu byla sprawiedliwosc.
\par 22 Ale Pismo zamknelo wszystko pod grzech, aby obietnica z wiary Jezusa Chrystusa byla dana wierzacym.
\par 23 A przedtem, niz przyszla wiara, bylismy pod zakonem strzezeni, wespól zamknieni bedac w te wiare, która potem miala byc objawiona.
\par 24 A przetoz zakon pedagogiem naszym byl do Chrystusa, abysmy z wiary byli usprawiedliwieni.
\par 25 Ale gdy przyszla wiara, juz nie jestesmy pod pedagogiem.
\par 26 Albowiem wszyscy synami Bozymi jestescie przez wiare w Chrystusie Jezusie.
\par 27 Bo którzykolwiek jestescie w Chrystusa ochrzczeni, w Chrystusascie sie oblekli.
\par 28 Nie masz Zyda, ani Greka; nie masz niewolnika ani wolnego; nie masz mezczyzny i niewiasty; albowiem wszyscy wy jednym jestescie w Chrystusie Jezusie.
\par 29 A jezliscie wy Chrystusowi, tedyscie nasieniem Abrahamowem, a wedlug obietnicy dziedzicami.

\chapter{4}

\par 1 Mówie tedy: (bracia!) Pokad dziedzic jest dziecieciem, nic nie jest rózny od slugi, panem bedac wszystkiego;
\par 2 Ale jest pod opiekunami i dozorcami az do czasu zamierzenia ojcowskiego.
\par 3 Takze i my, gdysmy byli dziecmi, pod zywioly swiata bylismy zniewoleni.
\par 4 Lecz gdy przyszlo wypelnienie czasu, poslal Bóg onego Syna swego, który sie urodzil z niewiasty, który sie stal pod zakonem,
\par 5 Aby tych, którzy pod zakonem byli, wykupil, zebysmy prawa przysposobienia za synów dostapili.
\par 6 A izescie synowie, przetoz poslal Bóg Ducha Syna swego w serca wasze, wolajacego Abba, to jest Ojcze.
\par 7 A tak juz wiecej nie jestes niewolnikiem, ale synem; a poniewaz synem, tedy i dziedzicem Bozym przez Chrystusa.
\par 8 Alec naonczas nie znajac Boga, sluzyliscie tym, którzy z przyrodzenia nie sa bogowie.
\par 9 A teraz poznawszy Boga, owszem i poznani bedac od Boga, jakoz sie zas nazad wracacie ku zywiolom mdlym i mizernym, którym zasie znowu sluzyc chcecie?
\par 10 Przestrzegacie dni i miesiace, i czasy, i lata.
\par 11 Boje sie o was, bym snac darmo nie pracowal okolo was.
\par 12 Badzcie jako ja, gdyzem i ja jest jako wy, bracia! prosze was. W niczemescie mnie nie ukrzywdzili.
\par 13 Bo wiecie, zem w slabosci ciala wam z przodku Ewangielije opowiadal.
\par 14 A pokuszenia mego, które bylo w ciele mojem, sobie nie lekcewazyliscie, aniscie niem gardzili, alescie mie jako Aniola Bozego przyjeli i jako Chrystusa Jezusa.
\par 15 Jakiez tedy bylo blogoslawienstwo wasze? albowiem wam daje swiadectwo, iz, by byla rzecz mozna, dalibyscie mi byli wylupiwszy oczy wasze.
\par 16 Takzem sie stal nieprzyjacielem waszym, prawde wam mówiac?
\par 17 Palaja ku wam miloscia nie dobrze, owszem chca was odstrychnac, abyscie ich milowali.
\par 18 A dobrac rzecz, palac miloscia w dobrem zawsze, a nie tylko, gdym jest obecnym u was.
\par 19 Dziatki moje! (które znowu z bolescia rodze, azby Chrystus byl wyksztaltowany w was),
\par 20 Chcialbym teraz byc u was i odmienic glos mój, poniewaz watpie o was.
\par 21 Powiedzcie mi, którzy pod zakonem chcecie byc, nie sluchaciez zakonu?
\par 22 Albowiem napisane, iz Abraham mial dwóch synów, jednego z niewolnicy, a drugiego z wolnej.
\par 23 Lecz ten, który byl z niewolnicy, wedlug ciala sie urodzil, a który z wolnej, wedlug obietnicy.
\par 24 Przez co znacza sie insze rzeczy; albowiem te sa one dwa testamenty; jeden z góry Synajskiej, który rodzi w niewole; a ten jest jako Agar.
\par 25 Albowiem Agar jest góra Synaj w Arabii, a stosuje sie do niej terazniejsze Jeruzalem; bo jest w niewoli z dziatkami swojemi.
\par 26 Lecz ono górne Jeruzalem wolne jest, które jest matka wszystkich nas.
\par 27 Albowiem napisano: Rozwesel sie nieplodna, która nie rodzisz; porwij sie, a zawolaj, która nie pracujesz w porodzeniu; bo ta opuszczona wiele ma dziatek, wiecej niz ta, która ma meza.
\par 28 My tedy, bracia! tak jako Izaak, jestesmy dziatkami obietnicy.
\par 29 Ale jako na on czas ten, który sie byl urodzil wedlug ciala, przesladowal tego, który sie byl urodzil wedlug ducha, tak sie dzieje i teraz.
\par 30 Ale co mówi Pismo? Wyrzuc niewolnice i syna jej; albowiem nie bedzie dziedziczyl syn niewolnicy z synem wolnej,
\par 31 A tak, bracia! nie jestesmy dziecmi niewolnicy, ale wolnej.

\chapter{5}

\par 1 Stójcie tedy w tej wolnosci, która nas Chrystus wolnymi uczynil, a nie poddawajcie sie znowu pod jarzmo niewoli.
\par 2 Oto ja Pawel mówie wam, iz jezli sie obrzezywac bedziecie, Chrystus wam nic nie pomoze.
\par 3 A oswiadczam sie zasie przed kazdym czlowiekiem, który sie obrzezuje, iz powinien wszystek zakon pelnic.
\par 4 Pozbawiliscie sie Chrystusa, którzykolwiek sie przez zakon usprawiedliwiacie; wypadliscie z laski.
\par 5 Albowiem my duchem z wiary nadziei sprawiedliwosci oczekujemy.
\par 6 Bo w Chrystusie Jezusie ani obrzezka nic nie wazy, ani nieobrzezka, ale wiara przez milosc skuteczna;
\par 7 Biezeliscie dobrze; któz wam przeszkodzil, abyscie nie byli poslusznymi prawdzie?
\par 8 Ta namowa nie jestci z tego, który was powoluje.
\par 9 Troche kwasu wszystko zaczynienie zakwasza.
\par 10 Ja mam nadzieje o was w Panu, iz nic inszego rozumiec nie bedziecie; a ten, który was turbuje, odniesie sad, ktokolwiek jest.
\par 11 A ja, bracia! jezli jeszcze obrzezke kaze, czemuz jeszcze przesladowanie cierpie? Toc tedy zniszczone jest zgorszenie krzyzowe.
\par 12 Bodajze i odjeci byli, którzy wam niepokój czynia.
\par 13 Albowiem wy ku wolnosci powolani jestescie, bracia! tylko pod zaslona tej wolnosci cialu nie pozwalajcie, ale z milosci sluzcie jedni drugim.
\par 14 Bo wszystek zakon w jednem sie slowie zamyka, to jest w tem: Bedziesz milowal blizniego twego jako samego siebie.
\par 15 Ale jezli jedni drugich kasacie i pozeracie, patrzajciez, abyscie jedni od drugich nie byli strawieni.
\par 16 A to mówie: Duchem postepujcie, a pozadliwosci ciala nie wykonywajcie.
\par 17 Albowiem cialo pozada przeciwko duchowi, a duch przeciwko cialu; a te rzeczy sa sobie przeciwne, abyscie nie to, co chcecie, czynili.
\par 18 Lecz jezli duchem bywacie prowadzeni, nie jestescie pod zakonem.
\par 19 A jawnec sa uczynki ciala, które te sa: Cudzolóstwo, wszeteczenstwo, nieczystosc, rozpusta,
\par 20 Balwochwalstwo, czary, nieprzyjazni, swary, nienawisci, gniewy, spory, niesnaski, kacerstwa,
\par 21 Zazdrosci, mezobójstwa, pijanstwa, biesiady, i tym podobne rzeczy, o których przepowiadam wam, jakom i przedtem powiedzial, iz którzy takowe rzeczy czynia, królestwa Bozego nie odziedzicza.
\par 22 Ale owoc Ducha jest milosc, wesele, pokój, nieskwapliwosc, dobrotliwosc, dobroc, wiara, cichosc, wstrzemiezliwosc.
\par 23 Przeciwko takowym nie masz zakonu.
\par 24 Albowiem którzy sa Chrystusowi, cialo swoje ukrzyzowali z namietnosciami i z pozadliwosciami.
\par 25 Jezli duchem zyjemy, duchem tez postepujmy.
\par 26 Nie badzmy chciwi próznej chwaly, jedni drugich wyzywajac, jedni drugim zajrzac.

\chapter{6}

\par 1 Bracia! jezliby tez czlowiek zachwycony byl w jakim upadku, wy duchowni: naprawiajcie takiego w duchu cichosci, upatrujac kazdy samego siebie, abys i ty nie byl kuszony.
\par 2 Jedni drugich brzemiona noscie, a tak wypelniajcie zakon Chrystusowy.
\par 3 Albowiem jezli kto mniema, zeby czem byl, nie bedac niczem, takiego zawodzi wlasny umysl jego.
\par 4 Ale kazdy niechaj wlasnego swego uczynku doswiadcza, a tedy sam w sobie chwale miec bedzie, a nie w drugim.
\par 5 Albowiem kazdy swoje wlasne brzemie poniesie.
\par 6 A niech udziela ten, który bywa nauczany w slowie, temu, który go naucza, ze wszystkich dóbr.
\par 7 Nie bladzcie; Bóg sie nie da z siebie nasmiewac; albowiem cobykolwiek sial czlowiek, to tez zac bedzie.
\par 8 Bo kto sieje cialu swemu, z ciala zac bedzie skazenie; ale kto sieje duchowi, z ducha zac bedzie zywot wieczny.
\par 9 A dobrze czyniac nie slabiejmy; albowiem czasu swojego zac bedziemy nie ustawajac.
\par 10 Przeto tedy, póki czas mamy, dobrze czynmy wszystkim, a najwiecej domownikom wiary.
\par 11 Widzicie, jakim dlugi list wam napisal reka moja.
\par 12 Którzykolwiek chca byc pozorni wedlug ciala, ci was przymuszaja, abyscie sie obrzezali, tylko aby dla krzyza Chrystusowego przesladowania nie cierpieli.
\par 13 Albowiem i ci, którzy sie obrzezuja, nie zachowywuja zakonu sami, ale chca, abyscie sie wy obrzezali, zeby sie z ciala waszego chlubili.
\par 14 Ale ja, nie daj Boze, abym sie mial chlubic, tylko w krzyzu Pana naszego Jezusa Chrystusa, przez którego mi jest swiat ukrzyzowany, a ja swiatu.
\par 15 Albowiem w Chrystusie Jezusie ani obrzezka nic nie wazy, ani nieobrzezka, ale nowe stworzenie.
\par 16 A którzykolwiek wedlug tego sznuru postepowac beda, pokój na nich przyjdzie i milosierdzie, i na lud Bozy Izraelski.
\par 17 Na ostatek niechaj mi nikt trudnosci nie zadaje; albowiem ja pietna Pana Jezusowe nosze na ciele mojem.
\par 18 Laska Pana naszego Jezusa Chrystusa niech bedzie z duchem waszym, bracia! Amen.


\end{document}