\begin{document}

\title{1 List do Tymoteusza}


\chapter{1}

\par 1 Pawel, Apostol Jezusa Chrystusa podlug rozrzadzenia Boga, zbawiciela naszego, i Pana Jezusa Chrystusa, który jest nadzieja nasza,
\par 2 Tymoteuszowi, wlasnemu synowi w wierze, niech bedzie laska, milosierdzie, pokój od Boga, Ojca naszego, i Chrystusa Jezusa, Pana naszego.
\par 3 Jakom cie prosil, abys zostal w Efezie, gdym szedl do Macedonii, patrzze, abys rozkazal niektórym, zeby inaczej nie uczyli.
\par 4 I nie bawili sie basniami i wywodami nieskonczonemi rodzaju, które wiecej sporów przynosza, niz zbudowania Bozego, które w wierze zalezy.
\par 5 Lecz koniec przykazania jest milosc z czystego serca i z sumienia dobrego, i z wiary nieobludnej.
\par 6 Czego niektórzy jako celu uchybiwszy, obrócili sie ku próznomównosci.
\par 7 Chcac byc nauczycielami zakonu, nie rozumieja ani tego, co mówia, ani co za pewne twierdza.
\par 8 A wiemy, ze dobry jest zakon, jezliby go kto przystojnie uzywal,
\par 9 Wiedzac to, ze sprawiedliwemu nie jest zakon postanowiony, ale niesprawiedliwym i niepoddanym, niepoboznym i grzesznikom, zlosliwym i nieczystym, ojcomordercom i matkomordercom, mezobójcom,
\par 10 Wszetecznikom, samcoloznikom, ludokradcom, klamcom, krzywoprzysiezcom, i jezli co innego jest przeciwnego zdrowej nauce.
\par 11 Wedlug chwalebnej Ewangielii blogoslawionego Boga, która mi jest zwierzona.
\par 12 Przetoz dziekuje temu, który mie umocnil, Chrystusowi Jezusowi, Panu naszemu, iz mie za wiernego osadzil, na uslugiwanie postanowiwszy mie.
\par 13 Którym pierwej byl bluznierca i przesladowca, i gwaltownikiem; alem milosierdzia dostapil, bom to z niewiadomosci czynil, bedac w niewierze.
\par 14 Lecz nader obfitowala laska Pana naszego z wiara i z miloscia, która jest w Chrystusie Jezusie.
\par 15 Wierna jest ta mowa i wszelkiego przyjecia godna, iz Chrystus Jezus przyszedl na swiat, aby grzeszników zbawil, z których jam jest pierwszy.
\par 16 Alem dlatego milosierdzia dostapil, aby na mnie pierwszym okazal Jezus Chrystus wszelka cierpliwosc na przyklad tym, którzy wen uwierzyc maja ku zywotowi wiecznemu.
\par 17 Przetoz królowi wieków niesmiertelnemu, niewidzialnemu samemu madremu Bogu, niech bedzie czesc i chwala na wieki wieków. Amen.
\par 18 Toc rozkazanie zalecam, synu Tymoteuszu! abys wedlug proroctw, które uprzedzily o tobie, bojowal w nich on dobry bój,
\par 19 Majac wiare i dobre sumienie, które niektórzy odrzuciwszy, szkode podjeli w wierze;
\par 20 Z których jest Hymeneusz i Aleksander, którychem oddal szatanowi, aby pokarani bedac, nauczyli sie nie bluznic.

\chapter{2}

\par 1 Napominam tedy, aby przed wszystkiemi rzeczami czynione byly prosby, modlitwy, przyczyny i dziekowania za wszystkich ludzi;
\par 2 Za królów i za wszystkich w przelozenstwie bedacych, abysmy cichy i spokojny zywot wiedli we wszelkiej poboznosci i uczciwosci.
\par 3 Albowiem to jest rzecz dobra i przyjemna przed Bogiem, zbawicielem naszym,
\par 4 Który chce, aby wszyscy ludzie byli zbawieni i ku znajomosci prawdy przyszli.
\par 5 Boc jeden jest Bóg, jeden takze posrednik miedzy Bogiem i ludzmi, czlowiek Chrystus Jezus.
\par 6 Który dal samego siebie na okup za wszystkich, co jest swiadectwem czasów jego.
\par 7 Na com ja jest postanowiony za kaznodzieje i Apostola, (prawde mówie w Chrystusie, nie klamie), za nauczyciela pogan w wierze i w prawdzie.
\par 8 Chce tedy, aby sie mezowie modlili na kazdem miejscu, podnoszac czyste rece bez gniewu i bez poswarku.
\par 9 Takze i niewiasty, aby sie ubiorem przystojnym ze wstydem i skromnoscia zdobily, nie z trefionemi wlosami, albo zlotem, albo perlami, albo szatami kosztownemi,
\par 10 Ale (jako przystoi niewiastom, które sie ozywaja do poboznosci), dobremi uczynkami.
\par 11 Niewiasta niech sie uczy w milczeniu ze wszelkiem poddanstwem.
\par 12 Bo niewiescie nie pozwalam uczyc, ani wladzy miec nad mezem, ale aby byla w milczeniu.
\par 13 Bo Adam pierwszy stworzony jest, potem Ewa.
\par 14 I Adam nie byl zwiedziony, ale niewiasta zwiedziona bedac, przestepstwa przyczyna byla.
\par 15 Lecz zbawiona bedzie przez rodzenie dziatek, jezliby zostaly w wierze i w milosci, i w swietobliwosci z miernoscia.

\chapter{3}

\par 1 Wierna jest ta mowa: Jezli kto biskupstwa zada, dobrej pracy zada.
\par 2 Ale biskup ma byc nienaganiony, maz jednej zony, czuly, trzezwy, powazny, goscinny, ku nauczaniu sposobny;
\par 3 Nie pijanica wina, nie bitny, nie sprosnego zysku chciwy, ale slusznosc milujacy, nieswarliwy, nielakomy;
\par 4 Który by dom swój dobrze rzadzil, który by dziatki mial w posluszenstwie ze wszelaka uczciwoscia;
\par 5 (Bo jezliby kto nie umial swego wlasnego domu rzadzic, jakoz piecze bedzie mial o kosciele Bozym?)
\par 6 Nie nowotny, aby bedac nadety, nie wpadl w sad potwarcy.
\par 7 Musi tez miec swiadectwo dobre od obcych, aby nie wpadl w hanbe i w sidlo potwarcy.
\par 8 Dyjakonowie takze maja byc powazni, nie dwoistego slowa, nie pijanicy wielu wina, nie chciwi sprosnego zysku,
\par 9 Majacy tajemnice wiary w czystem sumieniu.
\par 10 A ci tez niech beda pierwej doswiadczeni, zatem niech sluza, jezli sa bez nagany;
\par 11 Zony takze niech maja powazne, nie potwarliwe, trzezwe, wierne we wszystkiem.
\par 12 Dyjakonowie niech beda mezami jednej zony, którzy by dziatki dobrze rzadzili i wlasne domy.
\par 13 Albowiem którzy by dobrze sluzyli, stopien sobie dobry zjednaja i wielkie bezpieczenstwo w wierze, która jest w Chrystusie Jezusie.
\par 14 Toc tobie pisze, majac nadzieje, ze w rychle przyjde do ciebie;
\par 15 A jezlibym omieszkal, abys wiedzial, jako sie masz w domu Bozym sprawowac, który jest kosciolem Boga zywego, filarem i utwierdzeniem prawdy.
\par 16 A zaprawde wielka jest tajemnica poboznosci, ze Bóg objawiony jest w ciele, usprawiedliwiony jest w duchu, widziany jest od Aniolów, kazany jest poganom, uwierzono mu na swiecie, wziety jest w góre do chwaly.

\chapter{4}

\par 1 A Duch jawnie mówi, iz w ostateczne czasy odstana niektórzy od wiary, sluchajac duchów zwodzacych i nauk dyjabelskich,
\par 2 W obludzie klamstwo mówiacych i pietnowane majacych sumienie swoje,
\par 3 Zabraniajacych wstepowac w malzenstwo, rozkazujacych wstrzymywac sie od pokarmów, które Bóg stworzyl ku przyjmowaniu z dziekowaniem wiernym i tym, którzy poznali prawde.
\par 4 Bo wszelkie stworzenie Boze dobre jest, a nic nie ma byc odrzuconem, co z dziekowaniem bywa przyjmowane;
\par 5 Albowiem poswiecone bywa przez slowo Boze i przez modlitwe.
\par 6 To przekladajac braciom, bedziesz dobrym sluga Chrystusa Jezusa, wychowanym w slowach wiary i nauki dobrej, którejs nasladowal.
\par 7 A swieckich i babich basni chron sie; ale sie cwicz w poboznosci.
\par 8 Albowiem cielesne cwiczenie malo jest pozyteczne; lecz poboznosc do wszystkiego jest pozyteczna, majac obietnice zywota terazniejszego i przyszlego.
\par 9 Wierna to jest mowa i wszelkiego przyjecia godna.
\par 10 Albowiem przeto tez pracujemy i lzeni bywamy, ze nadzieje mamy w Bogu zywym, który jest zbawicielem wszystkich ludzi, a najwiecej wiernych.
\par 11 To przykazuj i tego nauczaj.
\par 12 Zaden mlodoscia twoja niech nie gardzi; ale badz przykladem wiernych w mowie, w obcowaniu, w milosci, w duchu, w wierze, w czystosci.
\par 13 Póki nie przyjde, pilnuj czytania, napominania i nauki.
\par 14 Nie zaniedbywaj daru Bozego, który w tobie jest, któryc dany jest przez prorokowanie z wlozeniem rak starszych.
\par 15 O tem rozmyslaj, tem sie zabawiaj, aby postepek twój jawny byl wszystkim.
\par 16 Pilnuj samego siebie i nauczania, trwaj w tych rzeczach; bo to czyniac, i samego siebie zbawisz, i tych, którzy cie sluchaja.

\chapter{5}

\par 1 Starszemu nie laj, ale jako ojca napominaj, mlodszych jako braci,
\par 2 Starsze niewiasty jako matki, mlodsze jako siostry, ze wszelaka czystoscia.
\par 3 Wdowy miej w uczciwosci, które prawdziwie sa wdowami.
\par 4 A jezli która wdowa dzieci albo wnuczeta ma, niech sie ucza pierwej przeciwko domowi wlasnemu byc poboznemi i wzajem oddawac rodzicom; albowiem to jest rzecz chwalebna i przyjemna przed obliczem Bozem.
\par 5 A która jest prawdziwie wdowa i osierociala, ma nadzieje w Bogu i trwa w prosbach i w modlitwach w nocy i we dnie.
\par 6 Ale która w rozkoszach zyje, ta zyjac umarla jest.
\par 7 To tedy rozkazuj, zeby byly nienaganione.
\par 8 A jezli kto o swoich, a najwiecej o domowych starania nie ma, wiary sie zaprzal i gorszy jest niz niewierny.
\par 9 Wdowa niech bedzie obrana, która by nie miala mniej niz szescdziesiat lat, która byla zona jednego meza,
\par 10 Majaca swiadectwo w dobrych uczynkach, jezli dzieci wychowala, jezli gosci przyjmowala, jezli swietych nogi umywala, jezli utrapionych wspomagala, jezli kazdego uczynku dobrego nasladowala.
\par 11 Wdów zasie mlodszych chron sie; bo gdyby sie zbestwily przeciw Chrystusowi, chca za maz isc,
\par 12 Majac osadzenie, iz pierwsza wiare odrzucily;
\par 13 Owszem tez próznujac ucza sie chodzic od domu do domu; a nie tylko sa próznujace, ale tez swiegotliwe, niepotrzebnemi rzeczami sie bawiace, mówiac, co nie przystoi.
\par 14 Chce tedy, aby mlodsze szly za maz, dzieci rodzily, gospodyniami byly; przeciwnikowi zadnej przyczyny nie dawaly ku obmowisku;
\par 15 Albowiem sie juz niektóre obrócily za szatanem.
\par 16 Przetoz, jezli który wierny albo która wierna ma wdowy, niechze je opatruje, a niech zbór nie bedzie obciazony, aby tym, które sa prawdziwie wdowami, starczylo.
\par 17 Starsi, którzy sie w przelozenstwie dobrze sprawuja, niech beda mieni za godnych dwojakiej czci, a zwlaszcza ci, którzy pracuja w slowie i w nauce.
\par 18 Albowiem Pismo mówi: Wolowi mlócacemu nie zawiazesz geby; i: Godzien jest robotnik zaplaty swojej.
\par 19 Przeciwko starszemu nie przyjmuj skargi, chyba za dwoma albo trzema swiadkami.
\par 20 A tych, którzy grzesza, strofuj przed wszystkimi, aby i drudzy bojazn mieli.
\par 21 Oswiadczam sie przed Bogiem i Panem Jezusem Chrystusem, i przed Anioly wybranymi, abys tych rzeczy przestrzegal, w osobach nie brakujac, nic nie czyniac z przychylnosci.
\par 22 Rak z predka na nikogo nie wkladaj, ani badz uczestnikiem cudzych grzechów: samego siebie czystym zachowaj.
\par 23 Samej wody wiecej nie pijaj, ale uzywaj po trosze wina dla zoladka twego i czestych chorób twoich.
\par 24 Grzechy niektórych ludzi przedtem sa jawne i uprzedzaja na sad, a za niektórymi ida pozad.
\par 25 Takze tez dobre uczynki przedtem sa jawne; ale które sa insze, utaic sie nie moga.

\chapter{6}

\par 1 Którzykolwiek sludzy sa pod jarzmem, niech rozumieja panów swych godnych byc wszelakiej czci, aby imie Boze i nauka nie byla bluzniona.
\par 2 A którzy maja panów wiernych, niech nimi nie gardza, dlatego iz sa bracmi, ale tem raczej niech sluza, iz sa wierni i mili, dobrodziejstwa Bozego uczestnicy. Tego nauczaj i do tego upominaj.
\par 3 Jezli kto inaczej uczy, a nie przystepuje do zdrowych mów Pana naszego, Jezusa Chrystusa, i do tej nauki, która jest wedlug poboznosci,
\par 4 Taki nadety jest i nic nie umie, ale szaleje okolo gadek i sporów o slowa, z których pochodzi zazdrosc, swar, zlorzeczenia, zle podejrzenia,
\par 5 Przewrotne cwiczenia ludzi umyslu skazonego i którzy pozbawieni sa prawdy, którzy rozumieja, ze poboznosc jest zyskiem cielesnym; odstapze od takich.
\par 6 A jestci wielki zysk poboznosc z przestawaniem na swem;
\par 7 Albowiem nicesmy nie przyniesli na ten swiat, bez pochyby ze tez wyniesc nic nie mozemy;
\par 8 Ale majac zywnosc i odzienie, na tem przestawac mamy.
\par 9 Bo którzy chca bogatymi byc, wpadaja w pokuszenie i w sidlo, i w wiele glupich i szkodliwych pozadliwosci, które pograzaja ludzi na zatracenie i zginienie.
\par 10 Albowiem korzen wszystkiego zlego jest milosc pieniedzy, których niektórzy pragnac, pobladzili od wiary i poprzebijali sie wieloma bolesciami.
\par 11 Ale ty, czlowiecze Bozy! chron sie takich rzeczy, a nasladuj sprawiedliwosci, poboznosci, wiary, milosci, cierpliwosci, cichosci.
\par 12 Bojuj on dobry bój wiary, chwyc sie zywota wiecznego, do któregos tez powolany, i wyznales dobre wyznanie przed wieloma swiadkami.
\par 13 Rozkazuje ci przed Bogiem, który wszystko ozywia i przed Chrystusem Jezusem, który oswiadczyl przed Ponckim Pilatem dobre wyznanie,
\par 14 Abys zachowal to przykazanie, bedac bez zmazy, bez nagany, az do objawienia Pana naszego, Jezusa Chrystusa,
\par 15 Które czasów swoich okaze on blogoslawiony i sam mozny król królujacych i Pan panujacych;
\par 16 Który sam ma niesmiertelnosc i mieszka w swiatlosci nieprzystepnej, którego nie widzial zaden z ludzi, ani widziec moze; któremu niech bedzie czesc i moc wieczna. Amen.
\par 17 Bogaczom w tym terazniejszym wieku rozkaz, aby nie byli wysokomyslnymi, ani nadziei pokladali w bogactwie niepewnem, ale w Bogu zywym, który nam wszystkiego obficie ku uzywaniu dodaje:
\par 18 Aby innym dobrze czynili, w uczynki dobre bogatymi byli, radzi dawali, a radzi udzielali,
\par 19 Skarbiac sami sobie grunt dobry na przyszly czas, aby otrzymali zywot wieczny.
\par 20 O Tymoteuszu! strzez tego, czegoc sie powierzono, a brzydz sie swiecka próznomównoscia i sprzeczaniem okolo falszywie nazwanej umiejetnosci,
\par 21 Która sie niektórzy szczycac z strony wiary, celu uchybili. Laska niech bedzie z toba. Amen.


\end{document}