\begin{document}

\title{List do Tytusa}


\chapter{1}

\par 1 Pawel, sluga Bozy i Apostol Jezusa Chrystusa wedlug wiary wybranych Bozych i znajomosci prawdy, która jest wedlug poboznosci,
\par 2 Ku nadziei zywota wiecznego, który obiecal przed czasy wiekuistemi ten, który nie klamie, Bóg, a objawil czasów swoich,
\par 3 To slowo swoje przez kazanie, które mi jest zwierzone wedlug rozrzadzenia zbawiciela naszego Boga:
\par 4 Tytusowi, wlasnemu synowi wedlug spólnej wiary, laska, milosierdzie i pokój niech bedzie od Boga Ojca, i Pana Jezusa Chrystusa, zbawiciela naszego.
\par 5 Dlategom cie zostawil w Krecie, abys to, co tam jeszcze zostaje, w rzad dobry wprawil i postanowil po miastach starszych, jakom ci ja byl rozkazal;
\par 6 Jezli kto jest bez nagany, maz jednej zony, dzieci wierne majacy, które by nie mogly byc obwinione w zbytku, albo niepoddane rzadowi.
\par 7 Albowiem biskup ma byc bez nagany, jako szafarz Bozy, nie sam sobie sie podobajacy, nie gniewliwy, nie pijanica wina, nie bitny, nie szukajacy zysku sprosnego;
\par 8 Ale goscinny, dobre milujacy, roztropny, sprawiedliwy, swietobliwy, powsciagliwy,
\par 9 Trzymajacy sie onej wiernej mowy, która jest wedlug nauki, izby tez mógl napominac nauka zdrowa i tych, którzy sie sprzeciwiaja, przekonywac;
\par 10 Albowiem jest wiele rzadowi niepoddanych, próznomównych i zwodzicieli mysli, a najwiecej tych, którzy sa z obrzezki,
\par 11 Którym trzeba usta zatkac; którzy cale domy podwracaja, uczac rzeczy nieslusznych dla zysku sprosnego.
\par 12 Powiedzial niektóry z nich wlasny ich prorok: Kretenczycy zawsze sa klamcami, zlemi bestyjami, brzuchami leniwemi.
\par 13 To swiadectwo jest prawdziwe; dla której przyczyny ostro ich karz, aby zdrowi byli w wierze.
\par 14 Nie pilnujac zydowskich basni i przykazan ludzi tych, którzy sie odwracaja od prawdy.
\par 15 Wszystkoc czyste czystym, lecz pokalanym i niewiernym nie masz nic czystego, ale pokalany jest i umysl, i sumienie ich.
\par 16 Udawaja, ze Boga znaja; ale uczynkami swemi tego sie zapieraja, bedac obrzydlymi i nieposlusznymi, a do wszelkiego dobrego uczynku niesposobnymi.

\chapter{2}

\par 1 A ty mów co nalezy na zdrowa nauke.
\par 2 Starcy, aby byli trzezwi, powazni, roztropni, zdrowi w wierze, w milosci, w cierpliwosci.
\par 3 Takze i stare niewiasty niech chodza w ubiorze przystojnym, jako przystoi swietym; niech nie beda potwarliwe, nie kochajace sie w wielu wina, poczciwych rzeczy nauczajace;
\par 4 Aby mlodych pan rozumu uczyly, jakoby mezów swoich i dziatki milowac mialy,
\par 5 A byly roztropne, czyste, domu pilne, dobrotliwe, mezom swym poddane, aby slowo Boze nie bylo bluznione.
\par 6 Mlodzienców takze napominaj, aby byli trzezwi;
\par 7 We wszystkiem samego siebie wystawiajac za wzór dobrych uczynków, majac w nauce calosc, powage,
\par 8 Slowo zdrowe, nienaganione, aby ten, kto by sie sprzeciwil, zawstydzic sie musial, nie majac nic, co by o was mial zlego mówic.
\par 9 Slug nauczaj, aby byli poddani panom swoim, we wszystkiem sie im podobajac, nie odmawiajac,
\par 10 W niczem nie oszukujac, ale we wszystkiem wiernosc uprzejma pokazujac, aby nauke zbawiciela naszego, Boga, we wszystkiem zdobili.
\par 11 Albowiem okazala sie laska Boza, zbawienna wszystkim ludziom,
\par 12 Cwiczaca nas, abysmy odrzeklszy sie niepoboznosci i swieckich pozadliwosci, trzezwie i sprawiedliwie, i poboznie zyli na tym swiecie,
\par 13 Oczekujac onej blogoslawionej nadziei i objawienia chwaly wielkiego Boga i zbawiciela naszego, Jezusa Chrystusa;
\par 14 Który dal samego siebie za nas, aby nas wykupil od wszelkiej nieprawosci i oczyscil sobie samemu lud wlasny, gorliwie nasladujacy dobrych uczynków.
\par 15 To mów i napominaj, i strofuj ze wszelka powaga; zaden toba niechaj nie gardzi.

\chapter{3}

\par 1 Napominaj ich, aby zwierzchnosciom i przelozenstwom poddanymi i poslusznymi byli, i aby do kazdego dobrego uczynku gotowymi byli;
\par 2 Nikogo nie lzyli, byli niezwadliwymi, ale ukladnymi, okazujac wszelka skromnosc przeciwko wszystkim ludziom.
\par 3 Albowiem i mysmy byli niekiedy glupimi, upornymi, bladzacymi, sluzac pozadliwosciom i rozkoszom rozmaitym, w zlosci i w zazdrosci mieszkajac, przemierzlymi, jedni drugich nienawidzacymi,
\par 4 Ale gdy sie okazala dobrotliwosc i milosc ku ludziom zbawiciela naszego, Boga,
\par 5 Nie z uczynków sprawiedliwosci, które bysmy my czynili, ale podlug milosierdzia swego zbawil nas przez omycie odrodzenia i odnowienia Ducha Swietego,
\par 6 Którego wylal na nas obficie przez Jezusa Chrystusa, zbawiciela naszego,
\par 7 Abysmy usprawiedliwieni bedac laska jego, stali sie dziedzicami wedlug nadziei zywota wiecznego.
\par 8 Wiernac to mowa; a chce, abys ty to twierdzil, aby sie starali, jakoby w dobrych uczynkach przodkowali, którzy uwierzyli Bogu.
\par 9 A te rzeczy sa dobre i ludziom pozyteczne; a glupich gadek i wyliczania rodzajów, i sporów, i swarów zakonnych pohamuj; albowiem sa niepozyteczne i prózne.
\par 10 Czlowieka heretyka po pierwszem i wtórem napominaniu strzez sie,
\par 11 Wiedzac, iz takowy jest wywrócony i grzeszy, bedac sam wlasnym sadem swoim osadzony.
\par 12 Gdy posle do ciebie Artemana albo Tychyka, staraj sie, abys do mnie przyszedl do Nikopolim; bom tam postanowil zimowac.
\par 13 Zenasa nauczonego w zakonie i Apollona pilnie odprowadz, aby im na niczem nie schodzilo.
\par 14 A niech sie ucza i nasi w dobrych uczynkach przodkowac, gdzie tego potrzeba, zeby nie byli nieuzytecznymi.
\par 15 Pozdrawiaja cie, którzy sa ze mna wszyscy. Pozdrów tych, którzy nas miluja w wierze. Laska Boza niech bedzie ze wszystkimi wami. Amen.


\end{document}