\begin{document}

\title{List do Hebrajczyków}


\chapter{1}

\par 1 Czestokroc i wieloma sposobami mawial niekiedy Bóg ojcom przez proroków;
\par 2 W te dni ostateczne mówil nam przez Syna swego, którego postanowil dziedzicem wszystkich rzeczy, przez którego i wieki uczynil.
\par 3 Który bedac jasnoscia chwaly i wyrazeniem istnosci jego, i zatrzymujac wszystkie rzeczy slowem mocy swojej, oczyszczenie grzechów naszych przez samego siebie uczyniwszy, usiadl na prawicy majestatu na wysokosciach,
\par 4 Tem sie zacniejszym stawszy nad Anioly, czem zacniejsze nad nie odziedziczyl imie.
\par 5 Albowiem któremuz kiedy z Aniolów rzekl: Tys jest syn mój, jam cie dzis splodzil? I zasie: Ja mu bede ojcem, a on mnie bedzie synem?
\par 6 A zasie, gdy wprowadza pierworodnego na okrag swiata, mówi: A niech sie mu klaniaja wszyscy Aniolowie Bozy.
\par 7 A zasie o Aniolach mówi: Który Anioly swoje czyni duchami, a slugi swoje plomieniem ognistym.
\par 8 Ale do Syna mówi: Stolica twoja, o Boze! na wieki wieków; laska sprawiedliwosci jest laska królestwa twego.
\par 9 Umilowales sprawiedliwosc, a nienawidziles nieprawosci; przetoz pomazal cie, o Boze! Bóg twój olejkiem wesela nad uczestników twoich.
\par 10 I tys, Panie! na poczatku ugruntowal ziemie, a niebiosa sa dzielem rak twoich.
\par 11 Onec pomina, ale ty zostajesz; a wszystkie jako szata zwiotszeja.
\par 12 A jako odzienie zwiniesz je i beda odmienione; ale ty tenzes jest, a lata twoje nie ustana.
\par 13 A do któregoz kiedy z Aniolów rzekl: Siadz po prawicy mojej, dokad nie poloze nieprzyjaciól twoich podnózkiem nóg twoich?
\par 14 Izali wszyscy nie sa duchami uslugujacymi, którzy na posluge bywaja poslani dla tych, którzy zbawienie odziedziczyc maja?

\chapter{2}

\par 1 Przetoz musimy tem pilniej przestrzegac tego, cosmy slyszeli, bysmy snac nie przeciekli.
\par 2 Bo poniewaz przez Anioly mówione slowo bylo pewne, a kazde przestepstwo i nieposluszenstwo wzielo sprawiedliwa zaplate pomsty:
\par 3 Jakoz my ucieczemy, jezli zaniedbamy tak wielkiego zbawienia, które wziawszy poczatek opowiadania przez samego Pana od tych, którzy go slyszeli, nam jest potwierdzone?
\par 4 Którym i Bóg swiadectwo wydawal przez znamiona i cuda, i rozliczne mocy, i przez udzielanie Ducha Swietego wedlug woli swojej.
\par 5 Albowiem nie Aniolom poddal swiat przyszly, o którym mówimy.
\par 6 A swiadczyl ktos na niektórem miejscu, mówiac: Cóz jest czlowiek, iz nan pamietasz, albo syn czlowieczy, iz go nawiedzasz?
\par 7 Na mala chwile mniejszym uczyniles go od Aniolów, chwala i czcia ukoronowales go i postanowiles go nad uczynkami rak twoich,
\par 8 Wszystkos poddal pod nogi jego. A poddawszy mu wszystko, nic nie zostawil, co by mu poddanego nie bylo; lecz teraz jeszcze nie widzimy, aby mu wszystko poddane bylo.
\par 9 Ale tego, który na mala chwile mniejszym stal sie od Aniolów, Jezusa, widzimy przez ucierpienie smierci chwala i czcia ukoronowanego, aby z laski Bozej za wszystkich smierci skosztowal.
\par 10 Albowiem nalezalo na tego, dla którego jest wszystko i przez którego jest wszystko, aby wiele synów do chwaly przywodzac wodza zbawienia ich przez ucierpienie doskonalym uczynil.
\par 11 Bo ten, który poswieca i ci, którzy bywaja poswieceni, z jednego sa wszyscy, dla której przyczyny nie wstydzi sie ich bracmi nazywac,
\par 12 Mówiac: Opowiem imie twoje braciom moim, w posrodku zgromadzenia spiewac ci bede.
\par 13 I zasie: Ja w nim ufac bede; a zasie: Oto ja i dzieci, które mi dal Bóg.
\par 14 Poniewaz tedy dzieci spolecznosc maja ciala i krwi, i on takze stal sie ich uczestnikiem, aby przez smierc zniszczyl tego, który mial wladze smierci, to jest dyjabla,
\par 15 A izby wyswobodzil tych, którzy dla bojazni smierci po wszystek czas zywota podlegli byli niewoli.
\par 16 Bo zaiste nigdzie nie przyjal Aniolów, ale nasienie Abrahamowe przyjal.
\par 17 Skad mial byc we wszystkiem podobny braciom, aby byl milosiernym i wiernym najwyzszym kaplanem w tem, co sie u Boga na ublaganie za grzechy ludzkie dziac mialo.
\par 18 Albowiem ze sam cierpial bedac kuszony, moze tych, którzy sa w pokusach, ratowac.

\chapter{3}

\par 1 Przetoz, bracia swieci, powolania niebieskiego uczestnicy! obaczcie Apostola i najwyzszego kaplana wyznania naszego, Chrystusa Jezusa,
\par 2 Wiernego temu, który go postanowil, jako i Mojzesz byl we wszystkim domu jego.
\par 3 Albowiem tem wiekszej chwaly ten nad Mojzesza godzien, im wieksza czesc ma budownik domu, nizeli sam dom.
\par 4 Bo kazdy dom bywa budowany od kogo; ale który wszystkie rzeczy zbudowal, Bóg jest.
\par 5 A bylci Mojzesz wiernym we wszystkim domu jego, jako sluga, na swiadectwo tego, co potem mialo byc mówione.
\par 6 Ale Chrystus jako syn nad domem swoim panuje, którego domem my jestesmy, jezli tylko te pewna ufnosc i te chwale nadziei az do konca stateczna zachowamy.
\par 7 Przetoz jako mówi Duch Swiety: Dzis, jezlibyscie glos jego uslyszeli,
\par 8 Nie zatwardzajciez serc waszych, jako w rozdraznieniu, w dzien onego pokuszenia na puszczy.
\par 9 Gdzie mie kusili ojcowie wasi i doswiadczali mie, i widzieli sprawy moje przez czterdziesci lat.
\par 10 Dlategom sie rozgniewal na ten naród i rzeklem: Ci zawsze bladza sercem, a oni nie poznawaja dróg moich.
\par 11 Jakom przysiagl w gniewie moim, ze nie wnijda do odpocznienia mojego.
\par 12 Patrzciez, bracia! by snac nie bylo w którym z was serce zle i niewierne, które by odstepowalo od Boga zywego;
\par 13 Ale napominajcie jedni drugich na kazdy dzien, póki sie Dzis nazywa, aby kto z was nie byl zatwardzony oszukaniem grzechu.
\par 14 Albowiem stalismy sie uczestnikami Chrystusa, jezlize tylko poczatek tego gruntu az do konca stateczny zachowamy.
\par 15 Przetoz póki bywa rzeczone: Dzis, jezlibyscie glos jego uslyszeli, nie zatwardzajciez serc waszych, jako w onem rozdraznieniu.
\par 16 Albowiem niektórzy uslyszawszy, rozdraznili Pana, ale nie wszyscy, którzy byli wyszli z Egiptu przez Mojzesza.
\par 17 A na którychze sie gniewal przez czterdziesci lat? Izali nie na tych, którzy grzeszyli, których ciala polegly na puszczy?
\par 18 A którymze przysiagl, ze nie mieli wnijsc do odpocznienia jego? Azaz nie tym, którzy byli nieposlusznymi?
\par 19 I widzimy, iz tam nie mogli wnijsc dla niedowiarstwa.

\chapter{4}

\par 1 Bójmyz sie tedy, aby snac zaniedbawszy obietnicy o wejsciu do odpocznienia jego, nie zdal sie kto z was byc uposledzony.
\par 2 Albowiem i nam zwiastowana jest Ewangielija jako i onym; ale im nie pomoglo slowo, które slyszeli, przeto iz nie bylo zlaczone z wiara tych, którzy slyszeli.
\par 3 Albowiem wnijdziemy do odpocznienia, którzysmy uwierzyli, jako powiedzial: Przetozem przysiagl w gniewie moim, ze nie wnijda do odpocznienia mojego; choc dokonane sa dziela Boze od zalozenia swiata.
\par 4 Albowiem tak powiedzial na jednem miejscu o siódmym dniu: I odpoczal Bóg dnia siódmego od wszystkich spraw swoich.
\par 5 A tu zasie: Ze nie wnijda do odpocznienia mego.
\par 6 Poniewaz tedy to zostaje, ze niektórzy wchodza do niego, a ci, którym pierwej zwiastowano, nie weszli dla niedowiarstwa,
\par 7 Zasie naznacza dzien niektóry: Dzis, mówiac przez Dawida po tak dlugim czasie (jako powiedziano): Dzis, jezlibyscie glos jego uslyszeli, nie zatwardzajciez serc waszych.
\par 8 Albowiem jezliby im Jozue odpocznienie sprawil, nie mówilby byl potem o inszym dniu.
\par 9 A tak zostaje jeszcze odpocznienie ludowi Bozemu.
\par 10 Albowiem ktobykolwiek wszedl do odpocznienia jego i on takze odpoczal od spraw swoich, jako i Bóg od swoich.
\par 11 Starajmyz sie tedy, abysmy weszli do onego odpocznienia, zeby kto nie wpadl w tenze przyklad niedowiarstwa.
\par 12 Boc zywe jest slowo Boze i skuteczne, i przerazliwsze nad wszelki miecz po obu stronach ostry, i przenikajace az do rozdzielenia i duszy, i ducha, i stawów, i szpików, i rozeznawajace mysli i zdania serdeczne.
\par 13 A nie masz zadnego stworzenia, które by nie bylo jawne przed obliczem jego; owszem wszystkie rzeczy obnazone sa i odkryte oczom tego, o którym mówimy.
\par 14 Przetoz majac najwyzszego kaplana wielkiego, który przeniknal niebiosa, Jezusa, Syna Bozego, trzymajmyz sie tego wyznania.
\par 15 Albowiem nie mamy najwyzszego kaplana, który by nie mógl z nami cierpiec krewkosci naszych, lecz skuszonego we wszystkiem na podobienstwo nas, oprócz grzechu.
\par 16 Przystapmyz tedy z ufnoscia do tronu laski, abysmy dostapili milosierdzia i laske znalezli ku pomocy czasu przygodnego.

\chapter{5}

\par 1 Albowiem kazdy najwyzszy kaplan z ludzi wziety, za ludzi bywa postanowiony w tych rzeczach, które do Boga naleza, to jest, aby ofiarowal dary i ofiary za grzechy.
\par 2 Który by mógl, jako przystoi, uzalic sie nieumiejetnych i bladzacych, bedacy sam oblozony krewkoscia.
\par 3 A dla tej jest powinien, jako za lud, tak i sam za sie ofiarowac za grzechy.
\par 4 A nikt sobie tej czci nie bierze, tylko ten, który bywa powolany od Boga jako i Aaron.
\par 5 Tak i Chrystus nie sam sobie tej czci przywlaszczyl, aby sie stal najwyzszym kaplanem; ale ten, który mu rzekl: Syn mój jestes ty, jam cie dzis splodzil.
\par 6 Jako i na inszem miejscu mówi: Tys jest kaplanem na wieki wedlug porzadku Melchisedekowego.
\par 7 Który za dni ciala swego modlitwy i unizone prosby do tego, który go mógl zachowac od smierci, z wolaniem wielkiem i ze lzami ofiarowal, i wysluchany jest dla uczciwosci.
\par 8 A choc byl Synem Bozym, wszakze z tego, co cierpial, nauczyl sie posluszenstwa.
\par 9 A tak doskonalym bedac, stal sie wszystkim sobie poslusznym przyczyna zbawienia wiecznego,
\par 10 Nazwany bedac od Boga kaplanem najwyzszym wedlug porzadku Melchisedekowego.
\par 11 O którym wiele by sie mialo mówic i trudnych rzeczy do wylozenia; alescie sie wy stali leniwi ku sluchaniu.
\par 12 Albowiem majac byc nauczycielami wzgledem czasu, zasie potrzebujecie, aby was uczono, które sa pierwsze poczatki mów Bozych, i staliscie sie jako mleka potrzebujacy, a nie twardego pokarmu.
\par 13 Bo kazdy, co sie tylko mlekiem karmi, ten nie jest powiadomy mowy sprawiedliwosci: (gdyz jest niemowlatkiem),
\par 14 Alec doskonalym nalezy twardy pokarm, to jest tym, którzy przez przyzwyczajenie maja zmysly wycwiczone ku rozeznaniu dobrego i zlego.

\chapter{6}

\par 1 Przetoz zaniechawszy poczatkowych nauk o Chrystusie, miejmy sie ku doskonalosci, nie znowu zakladajac grunty pokuty od uczynków martwych i wiary w Boga.
\par 2 Nauki o chrzcie i o wkladaniu rak, i o powstaniu umarlych, i o sadzie wiecznym;
\par 3 A to uczynimy, jezli tylko Bóg dopusci.
\par 4 Albowiem niemozebne jest, aby ci, którzy sa raz oswieceni i skosztowali daru niebieskiego, i uczestnikami sie stali Ducha Swietego,
\par 5 Skosztowali tez dobrego slowa Bozego i mocy przyszlego wieku,
\par 6 Gdyby odpadli, aby sie zas odnowili ku pokucie, jako ci, którzy sobie znowu krzyzuja Syna Bozego i jawnie go sromoca.
\par 7 Albowiem ziemia, która czesto na sie przychodzacy deszcz pije i rodzi ziele przygodne tym, którzy ja sprawuja, bierze blogoslawienstwo od Boga;
\par 8 Lecz która przynosi ciernie i osty, odrzucona jest i bliska przeklestwa, która na koniec bywa spalona.
\par 9 A wszakze, najmilsi! pewnismy o was cos lepszego i zbawienia blizszego, chociaz tak mówimy.
\par 10 Albowiem nie jest Bóg niesprawiedliwy, aby zapamietal pracy waszej i pracowitej milosci, którascie okazali ku imieniu jego, gdyscie sluzyli swietym i jeszcze sluzycie.
\par 11 A zadamy, aby kazdy z was toz staranie pokazywal ku nabyciu zupelnej nadziei az do konca.
\par 12 Abyscie nie byli gnusnymi, ale nasladowcami tych, którzy przez wiare i cierpliwosc odziedziczyli obietnice.
\par 13 Albowiem Bóg obietnice czyniac Abrahamowi, gdy nie mial przez kogo wiekszego przysiac, przysiagl przez siebie samego,
\par 14 Mówiac: Zaiste blogoslawiac blogoslawic ci bede i rozmnazajac rozmnoze cie.
\par 15 A tak dlugo czekajac, dostapil obietnicy.
\par 16 Ludziec wprawdzie przez wiekszego przysiegaja, a przysiega, która sie dzieje ku potwierdzeniu, jest miedzy nimi koncem wszystkich sporów.
\par 17 Dlatego tez Bóg chcac dostatecznie okazac dziedzicom obietnicy nieodmiennosc rady swojej, uczynil na to przysiege,
\par 18 Abysmy przez dwie rzeczy nieodmienne (w których niemozebne, aby Bóg klamal), warowna pocieche mieli, my, którzysmy sie uciekli ku otrzymaniu wystawionej nadziei,
\par 19 Która mamy jako kotwice duszy, i bezpieczna, i pewna, i wchodzaca az wewnatrz za zaslone,
\par 20 Gdzie przewodnik dla nas wszedl, Jezus, stawszy sie wedlug porzadku Melchisedekowego najwyzszym kaplanem na wieki.

\chapter{7}

\par 1 Albowiem ten Melchisedek byl król Salem, kaplan Boga najwyzszego, który zaszedl droge Abrahamowi, gdy sie wracal od porazki królów i blogoslawil mu.
\par 2 Któremu i dziesieciny ze wszystkiego udzielil Abraham; który najprzód wyklada sie król sprawiedliwosci, potem tez król Salem, co jest król pokoju.
\par 3 Bez ojca, bez matki, bez rodu, ani poczatku dni, ani konca zywota nie majac, ale przypodobany bedac Synowi Bozemu, zostaje kaplanem na wieki.
\par 4 Obaczciez tedy, jako wielki ten byl, któremu tez dziesiecine z lupów dal Abraham patryjarcha.
\par 5 A ci, którzy sa z synów Lewiego, urzad kaplanski przyjmujacy, rozkazanie maja, aby brali dziesiecine od ludu wedlug zakonu, to jest od braci swoich, choc wyszli z biódr Abrahamowych.
\par 6 Ale ten, którego ród nie jest poczytany miedzy nimi, dziesiecine wzial od Abrahama i temu, który mial obietnice, blogoslawil.
\par 7 A bez wszelkiego sporu mniejszy od wiekszego blogoslawienstwo bierze.
\par 8 A tuc dziesieciny biora ludzie, którzy umieraja; tam zasie on, o którym swiadczono, iz zyje.
\par 9 A iz tak rzeke i sam Lewi, który dziesieciny bierze, dal w Abrahamie dziesiecine.
\par 10 Albowiem jeszcze byl w biodrach ojcowskich, gdy wyszedl przeciwko niemu Melchisedek.
\par 11 A przetoz bylali doskonalosc przez kaplanstwo lewickie, (gdyz za niego wydany jest zakon ludowi), jakaz tego jeszcze byla potrzeba, aby inszy kaplan wedlug porzadku Melchisedekowego powstal, a nie byl wedlug porzadku Aaronowego mianowany?
\par 12 A poniewaz kaplanstwo jest przeniesione, musi tez i zakon przeniesiony byc.
\par 13 Bo ten, o którym sie to mówi, inszego jest pokolenia, z którego zaden nie sluzyl oltarzowi.
\par 14 Albowiem jawna jest, iz z pokolenia Judowego poszedl Pan nasz, o którem pokoleniu nic z strony kaplanstwa nie mówi Mojzesz.
\par 15 Owszem obficie to jeszcze i z tego jawna jest, iz powstal inszy kaplan wedlug porzadku Melchisedekowego,
\par 16 Który sie stal nie wedlug zakonu przykazania cielesnego, ale wedlug mocy zywota nieskazitelnego.
\par 17 Albowiem tak swiadczy: Tys jest kaplanem na wieki wedlug porzadku Melchisedekowego.
\par 18 Bo sie stalo zniesienie onego przyszlego przykazania dla slabosci jego i niepozytku.
\par 19 Bo niczego do doskonalosci nie przywiódl zakon; ale na miejsce jego wprowadzona jest lepsza nadzieja, przez która sie przyblizamy do Boga.
\par 20 A to i wzgledem tego, ze nie bez przysiegi jest wprowadzona.
\par 21 Boc sie oni bez przysiegi kaplanami stawali, a ten z przysiega przez tego, który rzekl do niego: Przysiagl Pan, a nie bedzie tego zalowal: Tys jest kaplanem na wieki wedlug porzadku Melchisedekowego,
\par 22 Tak dalece lepszego przymierza rekojmia stal sie Jezus.
\par 23 Wiec tez onych wiele bywalo kaplanów dlatego, iz im smierc nie dopuscila zawsze trwac.
\par 24 Ale ten, iz na wieki zostaje, wieczne ma kaplanstwo,
\par 25 Przetoz i doskonale zbawic moze tych, którzy przez niego przystepuja do Boga, zawsze zyjac, aby oredowal za nimi.
\par 26 Takiegoc zaiste przystalo nam miec najwyzszego kaplana, swietego, niewinnego, niepokalanego, odlaczonego od grzeszników i który by sie stal wyzszy nad niebiosa:
\par 27 Który by nie potrzebowal na kazdy dzien, jako oni najwyzsi kaplani, pierwej za swoje grzechy wlasne ofiar sprawowac, a potem za ludzkie; bo to uczynil raz samego siebie ofiarowawszy.
\par 28 Albowiem zakon ludzi podleglych krewkosci stanowil za najwyzszych kaplanów: ale slowo przysiegi, które sie stalo po zakonie, postanowilo Syna Bozego doskonalego na wieki.

\chapter{8}

\par 1 Ale suma tego, co sie mówi, ta jest: Iz takiego mamy najwyzszego kaplana, który usiadl na prawicy stolicy wielmoznosci na niebiesiech;
\par 2 Sluga bedac swiatnicy, a prawdziwego onego przybytku, który Pan zbudowal, a nie czlowiek.
\par 3 Albowiem kazdy najwyzszy kaplan ku ofiarowaniu darów i ofiar bywa postanowiony, skad potrzeba bylo, aby i ten mial, co by ofiarowal.
\par 4 Bo gdyby byl na ziemi, nie bylby kaplanem, póki by zostawali oni kaplani, którzy wedlug zakonu dary ofiaruja,
\par 5 Którzy sluza ksztaltowi i cieniowi rzeczy niebieskich, jako Mojzeszowi od Boga powiedziane bylo, gdy mial dokonczyc przybytku: Patrzajze, (mówi), abys uczynil wszystko wedlug ksztaltu, który ci jest okazany na tej górze.
\par 6 Ale teraz nasz kaplan tem zacniejszego urzedu dostapil, im jest posrednikiem lepszego przymierza, które lepszemi obietnicami jest utwierdzone.
\par 7 Bo gdyby ono pierwsze bylo bez przygany, tedycby wtóremu nie szukano miejsca.
\par 8 Albowiem ganiac Zydów, mówi: Oto dni ida, mówi Pan, gdy uczynie z domem Izraelskim i z domem Judzkim przymierze nowe.
\par 9 Nie wedlug przymierza, którem uczynil z ojcami ich w dzien, któregom ich ujal za reke ich, abym ich wywiódl z ziemi Egipskiej; albowiem oni nie zostali w tem przymierzu mojem, a Jam ich zaniedbal, mówi Pan.
\par 10 Przetoz toc jest przymierze, które postanowie z domem Izraelskim po tych dniach, mówi Pan: Dam prawa moje w mysl ich i na sercach ich napisze je, i bede Bogiem ich, a oni beda ludem moim.
\par 11 I nie bedzie uczyl zaden blizniego swego, i zaden brata swego, mówiac: Poznaj Pana; albowiem wszyscy mie poznaja, od najmniejszego z nich az do najwiekszego z nich.
\par 12 Bo milosciw bede nieprawosciom ich, a grzechów ich i nieprawosci ich nie wspomne wiecej.
\par 13 A gdy mówi: Nowe, pierwsze czyni wiotchem; a to, co wiotszeje i zestarzeje sie, bliskie jest zniszczenia.

\chapter{9}

\par 1 A mialoc i pierwsze ono przymierze ustawy okolo sluzby Bozej i swiatnice swiecka.
\par 2 Albowiem sprawiony byl przybytek pierwszy, w którym byl swiecznik, i stól, i pokladne chleby, który zwano swiatnica.
\par 3 A za druga zaslona byl przybytek, który zwano swiatnica najswietsza,
\par 4 Majac zlota kadzielnice i skrzynie przymierza zewszad zlotem powleczona, w której bylo wiadro zlote, majace w sobie manne, i laska Aaronowa, która byla zakwitnela, i tablice przymierza.
\par 5 A nad skrzynia byli Cherubinowie chwaly, którzy zacieniali ublagalnie, o których rzeczach teraz nie potrzeba mówic o kazdej z osobna.
\par 6 A to gdy tak przygotowano, do pierwszego przybytku zawsze wchodza kaplani, sluzby Boze odprawujac;
\par 7 Ale do drugiego raz w rok sam najwyzszy kaplan, nie bez krwi, która ofiaruje sam za sie i za ludzkie niewiadomosci.
\par 8 Przez co daje znac Duch Swiety, iz jeszcze nie byla objawiona droga do swiatnicy, póki jeszcze trwal pierwszy przybytek,
\par 9 Który byl wzorem na ten terazniejszy czas, w którym dary i ofiary bywaja ofiarowane, które nie moga w sumieniu doskonalym uczynic tego, co sluzbe Boza odprawuje;
\par 10 Tylko w pokarmach i w napojach, i w róznych omywaniach, i w ustawach cielesnych az do czasu naprawienia wlozone byly.
\par 11 Ale Chrystus przyszedlszy, najwyzszy kaplan przyszlych dóbr, przez wiekszy i doskonalszy przybytek, nie reka zbudowany, to jest nie tego budynku;
\par 12 Ani przez krew kozlów i cielców, ale przez wlasna krew swoje wszedl raz do swiatnicy, znalazlszy wieczne odkupienie.
\par 13 Albowiem jezli krew wolów i kozlów, i popiól jalowicy pokrapiajacy splugawione poswieca ku oczyszczeniu ciala:
\par 14 Jakoz daleko wiecej krew Chrystusowa, który przez Ducha wiecznego samego siebie ofiarowal nienaganionym Bogu, oczysci sumienie wasze od uczynków martwych ku sluzeniu Bogu zywemu?
\par 15 I dlatego jest nowego testamentu posrednikiem, aby gdyby smierc nastapila na odkupienie onych wystepków, które byly pod pierwszym testamentem, ci którzy sa powolani, wzieli obietnice wiecznego dziedzictwa.
\par 16 Albowiem gdzie jest testament, potrzeba, aby smierc nastapila tego, który czyni testament.
\par 17 Bo testament tych, którzy zmarli, mocny jest, gdyz jeszcze nie jest wazny, póki zyje ten, co testament uczynil.
\par 18 Skad ani on pierwszy testament bez krwi nie byl poswiecony.
\par 19 Albowiem gdy Mojzesz wszystko przykazanie wedlug zakonu opowiedzial wszystkiemu ludowi, wziawszy krew cielców i kozlów z woda i z welna szarlatowa, i z hizopem, i same ksiegi, i lud wszystek pokropil,
\par 20 Mówiac: Tac jest krew przymierza, które wam Bóg przykazal.
\par 21 Do tego i przybytek, i wszystko naczynie do sluzby Bozej nalezace krwia takze pokropil.
\par 22 A niemal wszystko wedlug zakonu krwia oczyszczone bywa, a bez rozlania krwi nie bywa odpuszczenie grzechów.
\par 23 A tak potrzeba bylo, aby ksztalty onych rzeczy, które sa na niebie, temi rzeczami byly oczyszczone, a same rzeczy niebieskie lepszemi ofiarami, nizeli te.
\par 24 Albowiem Chrystus nie wszedl do swiatnicy reka uczynionej, która by byla wizerunkiem prawdziwej, ale do samego nieba, aby sie teraz okazywal przed oblicznoscia Boza za nami,
\par 25 A nie izby czesto ofiarowal samego siebie, jako najwyzszy kaplan wchodzi do swiatnicy co rok ze krwia cudza;
\par 26 (Bo inaczej musialby byl czestokroc cierpiec od poczatku swiata), lecz teraz przy skonczeniu wieków raz objawiony jest ku zgladzeniu grzechu przez ofiarowanie samego siebie.
\par 27 A jako postanowiono ludziom raz umrzec, a potem bedzie sad:
\par 28 Tak i Chrystus, raz bedac ofiarowany na zgladzenie wielu grzechów, drugi raz sie bez grzechu okaze tym, którzy go oczekuja ku zbawieniu.

\chapter{10}

\par 1 Albowiem zakon majac cien przyszlych dóbr, a nie sam obraz rzeczy, jednakiemiz ofiarami, które na kazdy rok ustawicznie ofiaruja, nigdy nie moze tych, którzy do nich przystepuja, doskonalymi uczynic.
\par 2 Bo inaczej przestano by ich bylo ofiarowac, przeto zeby juz nie mieli zadnego sumienia o grzechy ci, którzy ofiaruja, bedac raz oczyszczeni.
\par 3 Ale przy tych ofiarach dzieje sie przypomnienie grzechów na kazdy rok.
\par 4 Albowiem nie mozna rzec, aby krew wolów i kozlów miala gladzic grzechy.
\par 5 Przetoz wchodzac na swiat, mówi: Ofiary i obiaty nie chciales, ales mi cialo sposobil.
\par 6 Calopalenia i ofiary za grzech nie upodobalyc sie.
\par 7 Tedym rzekl: Oto ide (na poczatku ksiegi napisano o mnie), abym czynil, o Boze! wole twoje;
\par 8 Powiedziawszy wyzej: Zes ofiary i obiaty, i calopalenia za grzech nie chcial, ani sobie upodobal, (które wedlug zakonu bywaja ofiarowane).
\par 9 Tedy rzekl: Oto ide, abym czynil, o Boze wole twoje; znosi pierwsze, aby wtóre postanowil.
\par 10 Przez która wole jestesmy poswieceni przez ofiare ciala Jezusa Chrystusa raz uczyniona.
\par 11 A wszelkic kaplan stoi na kazdy dzien, sluzbe Boza odprawujac, a jednakiez ofiary czestokroc ofiarujac, które nigdy grzechów zgladzic nie moga.
\par 12 Lecz ten jedne ofiare ofiarowawszy za grzechy, na wieki siedzi na prawicy Bozej,
\par 13 Na koniec oczekujac, azby polozeni byli nieprzyjaciele jego podnózkiem nóg jego.
\par 14 Albowiem jedna ofiara doskonalymi uczynil na wieki tych, którzy bywaja poswieceni.
\par 15 A swiadczy nam to i sam Duch Swiety; albowiem powiedziawszy pierwej:
\par 16 Toc jest przymierze, które postanowie z nimi po onych dniach, mówi Pan: Dam prawa moje do serca ich, a na myslach ich napisze je.
\par 17 A grzechów ich i nieprawosci ich nie wspomne wiecej:
\par 18 A gdziec jest odpuszczenie ich, juzci wiecej ofiary nie potrzeba za grzech.
\par 19 Majac tedy, bracia! wolnosc, wnijsc do swiatnicy przez krew Jezusowa,
\par 20 (Droga nowa i zywa, która nam poswiecil przez zaslone, to jest przez cialo swoje.)
\par 21 I kaplana wielkiego nad domem Bozym;
\par 22 Przystapmyz z prawdziwem sercem w zupelnosci wiary, majac oczyszczone serca od sumienia zlego,
\par 23 I omyte cialo woda czysta, trzymajmy wyznanie nadziei niechwiejace sie; (boc wierny jest ten, który obiecal;)
\par 24 I przypatrujmy sie jedni drugim ku pobudzaniu sie do milosci i do dobrych uczynków,
\par 25 Nie opuszczajac spolecznego zgromadzenia naszego, jako niektórzy obyczaj maja, ale napominajac jedni drugich, a to tem wiecej, czem wiecej widzicie, iz sie on dzien przybliza.
\par 26 Albowiem jezlibysmy dobrowolnie grzeszyli po wzieciu znajomosci prawdy, nie zostawalaby juz ofiara za grzechy;
\par 27 Ale straszliwe niejakie oczekiwanie sadu i zapalenie ognia, który pozrec ma przeciwników.
\par 28 Kto by odrzucil zakon Mojzeszowy, bez milosierdzia za swiadectwem dwóch albo trzech umiera.
\par 29 Co sie wam zda? Jakoz srozszego karania godzien jest ten, kto by Syna Bozego podeptal i krew przymierza, przez która byl poswiecony, za pospolita mial, i Ducha laski zelzyl?
\par 30 Albowiem znamy tego, który powiedzial: Mnie pomsta, Ja oddam, mówi Pan; i zasie: Pan sadzic bedzie lud swój.
\par 31 Strasznac rzecz jest wpasc w rece Boga zywego.
\par 32 Wspomnijcie na dni pierwsze, w których bedac oswieceni, znosiliscie wielki bój utrapienia,
\par 33 Lubo to, gdyscie byli i uraganiem, i utrapieniem na podziw wystawieni lub tez gdyscie sie stali uczestnikami tych, z którymi sie tak obchodzono.
\par 34 Albowiemescie i z wiezienia mego ze mna utrapieni byli i rozchwycenie majetnosci waszych przyjeliscie z radoscia, wiedzac, ze macie w sobie lepsza majetnosc w niebie, i trwajaca.
\par 35 Przetoz nie odrzucajcie ufnosci waszej, która ma wielka zaplate.
\par 36 Albowiem cierpliwosci wam potrzeba, abyscie wole Boza czyniac, odniesli obietnice.
\par 37 Boc jeszcze bardzo, bardzo maluczko, a oto ten, który ma przyjsc, przyjdzie, a nie omieszka.
\par 38 A sprawiedliwy z wiary zyc bedzie; a jezliby sie kto schranial, nie kocha sie w nim dusza moja.
\par 39 Lecz my nie jestesmy z tych, którzy sie schraniaja ku zginieniu, ale z tych, którzy wierza ku pozyskaniu duszy.

\chapter{11}

\par 1 A wiara jest gruntem tych rzeczy, których sie spodziewamy i dowodem rzeczy niewidzialnych.
\par 2 Albowiem przez nia swiadectwa doszli przodkowie.
\par 3 Wiara rozumiemy, iz swiat jest sprawiony slowem Bozem, tak iz rzeczy, które widzimy, nie staly sie z rzeczy widzialnych, ale z niczego.
\par 4 Wiara lepsza ofiare ofiarowal Abel Bogu, nizeli Kain, przez która swiadectwo otrzymal, ze jest sprawiedliwy, jakoz sam Bóg swiadectwo dal o darach jego, a przez te umarlszy jeszcze mówi.
\par 5 Wiara Enoch jest przeniesiony, aby nie ogladal smierci i nie jest znaleziony, przeto ze go Bóg przeniósl; albowiem pierwej niz jest przeniesiony, mial swiadectwo, ze sie podobal Bogu.
\par 6 A bez wiary nie mozna podobac sie Bogu; albowiem ten, co przystepuje do Boga, wierzyc musi, ze jest Bóg, a ze nagrode daje tym, którzy go szukaja.
\par 7 Wiara obwieszczony bedac od Boga Noe o tem, czego jeszcze nie bylo widziec, uczciwosci wyswiadczajac, przygotowal korab ku zachowaniu domu swego, przez który potepil swiat i stal sie dziedzicem sprawiedliwosci tej, która jest z wiary.
\par 8 Wiara powolany bedac Abraham, usluchal Boga, aby poszedl na ono miejsce, które mial wziac za dziedzictwo i wyszedl, nie wiedzac, dokad idzie.
\par 9 Wiara mieszkal w ziemi obiecanej jako w cudzej, mieszkajac w namiotach z Izaakiem i z Jakóbem, spólnymi dziedzicami tejze obietnicy.
\par 10 Albowiem oczekiwal miasta majacego grunty, którego sprawca i budownikiem jest Bóg.
\par 11 Wiara takze Sara wziela moc ku przyjeciu nasienia i mimo czasu wieku porodzila, gdyz miala za wiernego tego, który obiecal.
\par 12 A przetoz z jednego, i to obumarlego, rozplodzilo sie potomstwo jako mnóstwo gwiazd niebieskich i jako piasek niezliczony, który jest na brzegu morskim.
\par 13 Wedlug wiary umarli ci wszyscy, nie wziawszy obietnic, ale z daleka je upatrujac i cieszyli sie niemi, i witali je i wyznawali, iz sa goscmi i przychodniami na ziemi.
\par 14 Bo ci, którzy tak mówia, jawnie okazuja, iz ojczyzny szukaja.
\par 15 A wprawdzie, gdyby byli na one pamietali, z której byli wyszli, mieli dosyc czasu wrócic sie zas.
\par 16 Ale oni lepszej zadaja, to jest niebieskiej; przetoz i sam Bóg nie wstydzi sie nazywac Bogiem ich, bo im miasto zgotowal.
\par 17 Wiara ofiarowal Abraham Izaaka, bedac kuszony, a ofiarowal jednorodzonego ten, który byl wzial obietnice.
\par 18 Do którego rzeczono: W Izaaku tobie bedzie nazwane nasienie;
\par 19 Uwazajac to, iz Bóg moze i od umarlych wzbudzic; skad go tez w podobienstwie zmartwychwstania przyjal.
\par 20 Wiara okolo przyszlych rzeczy blogoslawil Izaak Jakóba i Ezawa.
\par 21 Wiara Jakób umierajac, kazdemu z synów Józefowych blogoslawil i poklonil sie podparlszy sie na wierzch laski swojej.
\par 22 Wiara Józef umierajac, o wyjsciu synów Izraelskich wzmianke uczynil i z strony kosci swoich rozkazal.
\par 23 Wiara narodziwszy sie Mojzesz, byl ukryty przez trzy miesiace od rodziców swoich, przeto ze widzieli nadobne dzieciatko i nie bali sie wyroku królewskiego.
\par 24 Wiara Mojzesz, bedac juz doroslym, zbranial sie byc zwany synem córki Faraonowej,
\par 25 Raczej sobie obrawszy zle rzeczy cierpiec z ludem Bozym, nizeli doczesna miec z grzechu rozkosz,
\par 26 Za wieksze pokladajac bogactwo nad skarby Egipskie uraganie Chrystusowe; bo sie ogladal na odplate.
\par 27 Wiara opuscil Egipt, nie bojac sie gniewu królewskiego; bo jakoby widzial niewidzialnego, meznie sobie poczynal.
\par 28 Wiara obchodzil wielkanoc i wylanie krwi, aby ten, który tracil pierworodnych, nie dotknal sie ich.
\par 29 Wiara przeszli przez morze Czerwone, jako po suszy, o co kusiwszy sie Egipczanie, potoneli.
\par 30 Wiara mury Jerycha upadly, gdy je obchodzono przez siedm dni.
\par 31 Wiara Rachab wszetecznica nie zginela wespól z nieposlusznymi, przyjawszy z pokojem szpiegów do gospody.
\par 32 A cóz wiecej mam mówic? Bo by mi czasu nie stalo, gdybym mial powiadac o Giedeonie i o Baraku, i o Samsonie, i o Jefcie, i o Dawidzie, i o Samuelu, i o prorokach.
\par 33 Którzy przez wiare zwalczyli królestwa, czynili sprawiedliwosc, dostepowali obietnic, lwom paszczeki zawierali;
\par 34 Zagaszali moc ognia, uchodzili ostrza mieczów, mocnymi sie stawali z niemocnych, meznymi bywali na wojnie, wojska cudzoziemców do uciekania przywodzili.
\par 35 Niewiasty odbieraly umarlych swoich wzbudzonych; a drudzy sa na próbach rozciagnieni, nie przyjawszy wybawienia, aby lepszego dostapili zmartwychwstania.
\par 36 Drudzy zasie posmiewisk i biczowania doswiadczyli, nadto i zwiazek i wiezienia.
\par 37 Byli kamionowani, pila przecierani, kuszeni, mieczem zabijani, chodzili w owczych i kozich skórach; byli w niedostatku, w ucisku, w niewczasach;
\par 38 (Których nie byl swiat godzien;) tulali sie po pustyniach i po górach, i jaskiniach, i jamach ziemi.
\par 39 A ci wszyscy swiadectwo otrzymawszy przez wiare, nie dostapili obietnicy.
\par 40 Przeto, ze Bóg o nas cos lepszego przejrzal, aby oni bez nas nie stali sie doskonalymi.

\chapter{12}

\par 1 Przetoz i my, majac tak wielki okolo siebie lezacy oblok swiadków, zlozywszy wszelaki ciezar i grzech, który nas snadnie obstepuje, przez cierpliwosc biezmy w zawodzie, który nam jest wystawiony;
\par 2 Patrzac na Jezusa, wodza i dokonczyciela wiary, który dla wystawionej sobie radosci, podjal krzyz, wzgardziwszy sromote, i usiadl na prawicy stolicy Bozej.
\par 3 Przetoz uwazajcie, jaki jest ten, który podejmowal takowe od grzeszników przeciwko sobie sprzeciwianie, abyscie oslabiwszy w umyslach waszych, nie ustawali.
\par 4 Jeszczescie sie az do krwi nie sprzeciwili, walczac przeciwko grzechowi.
\par 5 Czyliscie zapamietali napominania, które wam jako synom mówi: Synu mój, nie lekcewaz sobie kazni Panskiej, a nie trac serca, gdy od niego bywasz karany;
\par 6 Albowiem kogo Pan miluje, tego karze, a smaga kazdego, którego za syna przyjmuje.
\par 7 Jezli znosicie karanie, Bóg sie wam ofiaruje jako synom; albowiem któryz jest syn, którego by ojciec nie karal?
\par 8 A jezli jestescie bez karania, którego wszyscy sa uczestnikami, tedy jestescie bekartami, a nie synami.
\par 9 A nadto cielesnych ojców naszych mielismy, którzy nas karali, a balismy sie ich; zaz daleko wiecej nie mamy byc poddani Ojcu duchów, abysmy zyli?
\par 10 Albowiem oni na malo dni, jako sie im zdalo, nas karali; ale ten ku pozytkowi naszemu na to, abysmy byli uczestnikami swietobliwosci jego.
\par 11 A wszelkie karanie, gdy przytomne jest, nie zda sie byc wesole, ale smutne; lecz potem owoc sprawiedliwosci spokojny przynosi tym, którzy sa przez nie wycwiczeni.
\par 12 Przeto opuszczone rece i zemdlone kolana wyprostujcie,
\par 13 A czyncie koleje proste nogami waszemi, izby to, co jest chromego, z drogi nie ustapilo, ale raczej uzdrowione bylo.
\par 14 Pokoju nasladujcie ze wszystkimi i swietobliwosci, bez której zaden nie oglada Pana;
\par 15 Upatrujac, zeby kto nie odpadl od laski Bozej, a zeby który korzen gorzkosci nie podrósl, a nie przekazil i przez niego, aby sie ich wiele nie pokalalo;
\par 16 Aby kto nie byl wszetecznym albo sprosnym jako Ezaw, który za potrawe jedna sprzedal pierworodztwo swoje.
\par 17 Albowiem wiecie, iz i potem, gdy chcial odziedziczyc blogoslawienstwo, byl odrzucony; bo nie znalazl miejsca pokuty, choc jej z placzem szukal.
\par 18 Boscie nie przystapili do góry, która sie da dotknac, i do ognia gorejacego, i do wichru, i do ciemnosci i do burzy,
\par 19 I do dzwieku traby, i do glosu slów, który ci, co slyszeli, prosili, aby wiecej do nich nie mówiono;
\par 20 (Albowiem nie mogli zniesc tego, co im rozkazywano: Gdyby sie i bydle góry dotknelo, bedzie ukamionowane, albo pociskiem przebite.
\par 21 A tak straszne to bylo, co widzieli, ze tez Mojzesz rzekl: Ulaklem sie i drze.)
\par 22 Alescie przystapili do góry Syon i do miasta Boga zywego, do Jeruzalemu niebieskiego, i do niezliczonych tysiecy Aniolów;
\par 23 Do walnego zgromadzenia, i do zebrania pierworodnych, którzy sa spisani w niebie, i do Boga, sedziego wszystkich, i do duchów sprawiedliwych i doskonalych;
\par 24 I do posrednika nowego testamentu, Jezusa, i do krwi pokropienia, lepsze rzeczy mówiacej niz Ablowa.
\par 25 Patrzajciez, abyscie nie gardzili tym, który mówi; albowiem jezliz oni nie uszli, którzy gardzili tym, który na ziemi na miejscu Bozem mówil, daleko wiecej my, jezlize sie od tego, który z nieba jest, odwrócimy;
\par 26 Którego glos na on czas poruszyl byl ziemia, a teraz obiecal, mówiac: Jeszcze ja raz porusze nie tylko ziemia, ale i niebem.
\par 27 A to tez mówi: Jeszcze raz, pokazuje zniesienie rzeczy chwiejacych sie, jako tych, które sa uczynione, aby zostawaly te, które sie nie chwieja.
\par 28 Przetoz przyjmujac królestwo nie chwiejace sie, miejmy laske, przez która sluzymy przyjemnie Bogu ze wstydem i z uczciwoscia.
\par 29 Albowiem Bóg nasz jest ogniem trawiacym.

\chapter{13}

\par 1 Milosc braterska niech zostaje.
\par 2 Nie zapominajcie ochoty ku gosciom; albowiem przez te niektórzy nie wiedzac, Anioly za goscie przyjmowali.
\par 3 Pamietajcie na wiezniów, jakobyscie spólwiezniami byli; na utrapionych, jako ci, którzy tez w ciele jestescie.
\par 4 Uczciwe jest malzenstwo miedzy wszystkimi i loze niepokalane; ale wszeteczników i cudzolozników Bóg bedzie sadzil.
\par 5 Obcowanie wasze niech bedzie bez lakomstwa, przestwajac na tem, co macie; boc sam powiedzial: Nie zaniecham cie, ani cie opuszcze:
\par 6 Tak abysmy smiele mówic mogli: Pan mi jest pomocnikiem, nie bede sie bal, aby mi co mial uczynic czlowiek.
\par 7 Pamietajcie na wodzów waszych, którzy wam mówili slowo Boze, których obcowania koniec upatrujac nasladujcie wiary ich.
\par 8 Jezus Chrystus wczoraj i dzis, tenze i na wieki.
\par 9 Za naukami rozmaitemi i obcemi nie unoscie sie; albowiem dobra rzecz jest, aby laska bylo utwierdzone serce a nie pokarmami, które nie pomogly tym, co sie nimi bawili.
\par 10 Mamy oltarz, z którego nie maja wolnosci jesc ci, którzy przybytkowi sluza.
\par 11 Albowiem bydlat, których krew bywa wnoszona za grzech do swiatnicy przez najwyzszego kaplana, tych ciala palone bywaja za obozem.
\par 12 Dlatego i Jezus, aby poswiecil lud wlasna krwia swoja, za brama ucierpial.
\par 13 Wynijdzmyz tedy do niego za obóz, noszac uraganie jego.
\par 14 Albowiem nie mamy tu miasta trwalego, ale onego przyszlego szukamy.
\par 15 Przetoz przez niego ofiarujmy Bogu ofiare chwaly ustawicznie, to jest owoce warg wyznawajacych imieniowi jego.
\par 16 A dobroczynnosci i udzielania nie przepominajcie; albowiem sie Bóg w takowych ofiarach kocha.
\par 17 Badzcie posluszni wodzom waszym i badzcie im oddani; albowiem oni czuja nad duszami waszemi, jako ci, którzy liczbe oddac maja; aby to z radoscia czynili, a nie z wzdychaniem; boc wam to nie jest pozyteczne.
\par 18 Módlcie sie za nami; albowiem ufamy, iz mamy dobre sumienie, jako ci, którzy sie chcemy we wszystkiem dobrze zachowac.
\par 19 A tem wiecej prosze was, abyscie to czynili, abym wam tem rychlej byl przywrócony.
\par 20 A Bóg pokoju, który wywiódl od umarlych we krwi przymierza wiecznego, onego wielkiego pasterza owiec, Pana naszego Jezusa,
\par 21 Niech was doskonalymi uczyni w kazdym uczynku dobrym ku czynieniu woli swojej, sprawujac w was to, co przyjemnego przed obliczem jego, przez Jezusa Chrystusa, któremu niech bedzie chwala na wieki wieków. Amen.
\par 22 A prosze was, bracia! znoscie cierpliwie slowo napominania tego; bomci do was krótko pisal.
\par 23 Wiedzcie o bracie Tymoteuszu, ze jest wypuszczony, z którym (jezlibym szybko przyszedl), ogladam was.
\par 24 Pozdrówcie wszystkich wodzów waszych i wszystkich swietych. Pozdrawiaja was bracia z Wloch.
\par 25 Laska niech bedzie z wami wszystkimi. Amen.


\end{document}