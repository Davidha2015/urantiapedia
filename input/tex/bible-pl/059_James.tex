\begin{document}

\title{List św. Jakuba}


\chapter{1}

\par 1 Jakób, sluga Bozy i Pana Jezusa Chrystusa, dwunastu pokoleniom, które sa w rozproszeniu, zdrowia zyczy.
\par 2 Za najwieksza radosc miejcie, bracia moi! gdy w rozmaite pokusy wpadacie,
\par 3 Wiedzac, iz doswiadczenie wiary waszej sprawuje cierpliwosc;
\par 4 A cierpliwosc niech ma doskonaly uczynek, zebyscie byli doskonali i zupelni, którym by na niczem nie schodzilo.
\par 5 A jezli komu z was schodzi na madrosci, niech prosi u Boga, który ja szczerze wszystkim daje, a nie wymawia; i bedzie mu dana.
\par 6 Ale niech prosi z wiara, nic nie watpiac; albowiem kto watpi, jest podobny walowi morskiemu, który od wiatru pedzony i miotany bywa.
\par 7 Bo niechaj nie mniema ten czlowiek, aby co mial wziac od Pana.
\par 8 Maz umyslu dwoistego jest niestateczny we wszystkich drogach swoich.
\par 9 A niech sie chlubi brat niskiego stanu w wywyzszeniu swojem,
\par 10 A bogaty w ponizeniu swojem; bo jako kwiat trawy przeminie.
\par 11 Albowiem jako slonce, kiedy weszlo z goracoscia, ususzylo trawe, a kwiat jej opadl i zginela ona slicznosc ksztaltu jego, tak i bogaty w drogach swoich uwiednie.
\par 12 Blogoslawiony maz, który znosi pokuszenie; bo gdy bedzie doswiadczony, wezmie korone zywota, która obiecal Pan tym, którzy go miluja.
\par 13 Zaden, gdy bywa kuszony, niech nie mówi: Od Boga kuszony bywam; bo Bóg nie moze kuszony byc we zlem, a sam nikogo nie kusi.
\par 14 Ale kazdy bywa kuszony, gdy od swoich wlasnych pozadliwosci bywa pociagniony i przynecony.
\par 15 Zatem pozadliwosc poczawszy, rodzi grzech, a grzech bedac wykonany, rodzi smierc.
\par 16 Nie bladzciez, bracia moi mili!
\par 17 Wszelki datek dobry i wszelki dar doskonaly z góry jest, zstepujacy od Ojca swiatlosci, u którego nie masz odmiany, ani zacmienia na wstecz sie wracajacego.
\par 18 Który, przeto ze chcial, porodzil nas slowem prawdy ku temu, zebysmy byli niejakiemi pierwiastkami stworzenia jego.
\par 19 A tak, bracia moi mili! niech bedzie kazdy czlowiek predki ku sluchaniu, ale nierychly ku mówieniu i nierychly ku gniewowi.
\par 20 Bo gniew meza nie sprawuje sprawiedliwosci Bozej.
\par 21 A tak odrzuciwszy wszelakie plugastwo i zbytek zlosci, z cichoscia przyjmijcie slowo wszczepione, które moze zbawic dusze wasze.
\par 22 A badzcie czynicielami slowa, a nie sluchaczami tylko, oszukiwajacymi samych siebie.
\par 23 Albowiem jezli kto jest sluchaczem slowa a nie czynicielem, ten podobny jest mezowi przypatrujacemu sie obliczu narodzenia swego we zwierciadle;
\par 24 Bo samego siebie obejrzal i odszedl, a wnet zapomnial, jakim byl.
\par 25 Ale kto by wejrzal w on doskonaly zakon wolnosci i zostawalby w nim, ten nie bedac sluchaczem zapamietliwym, ale czynicielem uczynku, ten blogoslawionym bedzie w uczynku swoim.
\par 26 Jezli kto miedzy wami zda sie byc naboznym, nie kielznajac jezyka swego, ale zwodzac serce swe, tego nabozenstwo prózne jest.
\par 27 Nabozenstwo czyste i niepokalane u Boga i Ojca to jest: Nawiedzac sieroty i wdowy w ucisku ich i zachowac samego siebie niepokalanym od swiata.

\chapter{2}

\par 1 Bracia moi! nie miejcie z brakowaniem osób wiary Pana naszego Jezusa Chrystusa, który chwalebny jest.
\par 2 Albowiem gdyby wszedl do zgromadzenia waszego maz, majac pierscien zloty w szacie swietnej, a wszedlby tez i ubogi w podlym odzieniu:
\par 3 I wejrzelibyscie na tego, co ma swietna szate, a rzeklibyscie mu: Ty! siadz sam poczciwie! a ubogiemu byscie rzekli: Ty! tam stój, albo siadz tu pod podnózkiem moim!
\par 4 Azazescie juz nie uczynili róznosci miedzy soba i nie staliscie sie sedziami mysli zlych?
\par 5 Sluchajcie, bracia moi mili! azaz Bóg nie obral ubogich na tym swiecie, aby byli bogatymi w wierze i dziedzicami królestwa, które obiecal tym, którzy go miluja?
\par 6 Alescie wy zniewazyli ubogiego. Azaz bogacze gwaltem was nie uciskaja i do sadów was nie pociagaja?
\par 7 Azaz oni nie bluznia onego zacnego imienia, które jest wzywane nad wami?
\par 8 A jezlize pelnicie zakon królewski wedlug Pisma: Bedziesz milowal blizniego twego jako samego siebie, dobrze czynicie.
\par 9 Lecz jezli osobami brakujecie, grzech popelniacie i bywacie przekonani od zakonu jako przestepcy.
\par 10 Albowiem ktobykolwiek zachowal wszystek zakon, a w jednym by upadl, stal sie winien wszystkich przykazan.
\par 11 Bo który rzekl: Nie bedziesz cudzolozyl, ten tez rzekl: Nie bedziesz zabijal, a jezlibys nie cudzolozyl, ale bys zabil, stales sie przestepca zakonu.
\par 12 Tak mówcie i tak czyncie jako ci, którzy wedlug zakonu wolnosci macie byc sadzeni.
\par 13 Albowiem sad bez milosierdzia bedzie temu, co nie czynil milosierdzia; ale milosierdzie chlubi sie przeciwko sadowi.
\par 14 Cóz pomoze, bracia moi! jezliby kto rzekl, iz ma wiare, a uczynków by nie mial? izali go ona wiara moze zbawic?
\par 15 A gdyby brat albo siostra byli nieodziani i schodziloby im na powszedniej zywnosci,
\par 16 I rzeklby im kto z was: Idzcie w pokoju, ugrzejcie sie i najedzcie sie, a nie dalibyscie im potrzeb cialu nalezacych, cóz to pomoze?
\par 17 Takze i wiara, nie mali uczynków, martwa jest sama w sobie.
\par 18 Ale rzecze kto: Ty masz wiare, a ja mam uczynki; ukaz mi wiare twoje bez uczynków twoich, a ja tobie ukaze wiare moje z uczynków moich.
\par 19 Ty wierzysz, iz jeden jest Bóg, dobrze czynisz; i dyjabli temu wierza, wszakze drza,
\par 20 Ale chceszli wiedziec, o czlowiecze marny! iz wiara bez uczynków martwa jest?
\par 21 Abraham, ojciec nasz, izali nie z uczynków usprawiedliwiony jest, gdy ofiarowal Izaaka, syna swego, na oltarzu?
\par 22 Widzisz, iz wiara spólnie robila z uczynkami jego, a z uczynków wiara doskonala sie stala.
\par 23 A tak wypelnilo sie Pismo, które mówi: I uwierzyl Abraham Bogu, i przyczytano mu to ku sprawiedliwosci, i przyjacielem Bozym nazwany jest.
\par 24 A widziciez, iz z uczynków usprawiedliwiony bywa czlowiek, a nie z wiary tylko.
\par 25 Takze tez i Rachab, wszetecznica, izali nie z uczynków jest usprawiedliwiona, gdy przyjela onych poslów i insza droga wypuscila?
\par 26 Albowiem jako cialo bez duszy jest martwe, tak i wiara bez uczynków martwa jest.

\chapter{3}

\par 1 Niechaj was niewiele bedzie nauczycielami, bracia moi! wiedzac, ze ciezszy sad odniesiemy.
\par 2 Albowiem w wielu upadamy wszyscy; jezli kto nie upada w slowie, ten jest doskonalym mezem, który tez moze na wodzy trzymac i wszystko cialo.
\par 3 Oto koniom wedzidla w geby wprawujemy, aby nam powolne byly i wszystkiem cialem ich kierujemy.
\par 4 Oto i okrety, choc tak wielkie sa i tegiemi wiatrami pedzone bywaja, wszak i najmniejszym sterem bywaja kierowane, gdziekolwiek jest wola sternikowa;
\par 5 Tak i jezyk maly jest czlonek, wszakze bardzo sie wynosi. Oto maluczki ogien, jako wielki las zapala!
\par 6 I jezyk jest ogien i swiat niesprawiedliwosci; takci jest postanowiony jezyk miedzy czlonkami naszemi, który szpeci wszystko cialo i zapala kolo urodzenia naszego, i bywa zapalony od ognia piekielnego.
\par 7 Albowiem wszelkie przyrodzenie i dzikich zwierzat, i ptaków, i gadzin, i morskich potworów bywa okrócone i jest okrócone od ludzi;
\par 8 Ale jezyka zaden z ludzi okrócic nie moze, który jest nieokrócone zle i pelne jadu smiertelnego.
\par 9 Przez niego blogoslawimy Boga i Ojca, i przez niego przeklinamy ludzi, którzy na podobienstwo Boze stworzeni sa;
\par 10 Z jednychze ust wychodzi blogoslawienstwo i przeklestwo. Nie tak ma byc, bracia moi!
\par 11 Izali zdrój z jednego zródla wypuszcza i slodka, i gorzka wode?
\par 12 Izali moze, bracia moi! figowe drzewo przynosic oliwki, albo winna macica figi? Tak zaden zdrój slonej i slodkiej wody oraz nie wydaje.
\par 13 Jezli kto jest madry i umiejetny miedzy wami? niech pokaze dobrem obcowaniem uczynki swoje w madrej cichosci.
\par 14 Ale jezli macie gorzka zawisc i zajatrzenie w sercu waszem, nie chlubciez sie, ani klamcie przeciwko prawdzie.
\par 15 Nie jestci ta madrosc z góry zstepujaca, ale ziemska, bydleca, dyjabelska.
\par 16 Bo gdzie jest zawisc i zajatrzenie, tam i rozterki, i wszelka zla sprawa.
\par 17 Ale madrosc, która jest z góry, najprzódci jest czysta, potem spokojna, mierna, powolna, pelna milosierdzia i owoców dobrych, nieposadzajaca, i nieobludna.
\par 18 Ale owoc sprawiedliwosci w pokoju bywa siany tym, którzy pokój czynia.

\chapter{4}

\par 1 Skadze sa walki i zwady miedzy wami? Izali nie stad, to jest z lubosci waszych, które walcza w czlonkach waszych?
\par 2 Pozadacie, a nie macie, zajrzycie i zawidzicie, a nie mozecie dostac; wadzicie sie i walczycie, wszakze nie otrzymujecie, przeto iz nie prosicie.
\par 3 Prosicie, a nie bierzecie, przeto iz zle prosicie, abyscie to na rozkosze wasze obracali.
\par 4 Cudzoloznicy i cudzoloznice! nie wieciez, iz przyjazn swiata jest nieprzyjaznia Boza? Przetoz, ktobykolwiek chcial byc przyjacielem tego swiata, staje sie nieprzyjacielem Bozym.
\par 5 Albo mniemacie, iz prózno Pismo mówi: Izali ku zazdrosci pozada duch, który w nas mieszka?
\par 6 Owszem, hojniejsza daje laske; bo mówi: Bóg sie pysznym sprzeciwia, ale pokornym laske daje.
\par 7 Poddajciez sie tedy Bogu, a dajcie odpór dyjablu, a uciecze od was.
\par 8 Przyblizcie sie ku Bogu, a przyblizy sie ku wam. Ochedózcie rece grzesznicy i oczyscie serca, którzyscie umyslu dwoistego,
\par 9 Badzcie utrapieni i zalujcie, i placzcie; smiech wasz niech sie obróci w zalosc, a radosc w smutek.
\par 10 Unizajcie sie przed obliczem Panskiem, a wywyzszy was.
\par 11 Nie obmawiajcie jedni drugich, bracia! Kto obmawia brata i potepia brata swego, obmawia zakon i potepia zakon; a jezli potepisz zakon, nie jestes czynicielem zakonu, ale sedzia.
\par 12 Jeden jest zakonodawca, który moze zbawic i zatracic. Ale ty ktos jest, co potepiasz drugiego?
\par 13 Nuz teraz wy, co mówicie: Dzis albo jutro pójdziemy do tego miasta i zamieszkamy tam przez jeden rok, a bedziemy kupczyc i zysk sobie przywiedziemy;
\par 14 (Którzy nie wiecie, co jutro bedzie; bo cóz jest zywot wasz? Para zaiste jest, która sie na maly czas pokazuje, a potem niszczeje.)
\par 15 Zamiast tego, co byscie mieli mówic: Bedzieli Pan chcial, a bedziemyli zywi, uczynimy to albo owo.
\par 16 Ale teraz chlubicie sie w pysze waszej; wszelka chluba takowa zla jest.
\par 17 Przetoz, kto umie dobrze czynic, a nie czyni, grzech ma.

\chapter{5}

\par 1 Nuz teraz, bogacze! placzcie, narzekajac nad nedzami waszemi, które przyjda.
\par 2 Bogactwo wasze zgnilo, a szaty wasze mole zgryzly.
\par 3 Zloto wasze i srebro wasze pordzewialo, a rdza ich bedzie na swiadectwo przeciwko wam i pozre ciala wasze jako ogien; zgromadziliscie skarb na ostatnie dni.
\par 4 Oto, zaplata robotników, którzy zeli krainy wasze, od was zatrzymana wola, a wolania zenców weszly do uszów Pana zastepów.
\par 5 Zyliscie w rozkoszach na ziemi i bujaliscie; wytuczyliscie serca wasze jako na dzien zabijania ofiar.
\par 6 Potepiliscie, zamordowaliscie sprawiedliwego, a nie sprzeciwia sie wam.
\par 7 Przetoz, bracia! badzcie cierpliwymi az do przyjscia Panskiego. Oto, oracz oczekuje drogiego pozytku ziemi, cierpliwie go oczekujac, azby otrzymal deszcz ranny i wieczorny.
\par 8 Badzciez i wy cierpliwymi, a utwierdzajcie serca wasze; albowiem sie przybliza przyjscie Panskie.
\par 9 Nie wzdychajcie jedni przeciwko drugim, bracia! abyscie nie byli osadzeni. Oto, sedzia juz przede drzwiami stoi.
\par 10 Bierzcie za przyklad, bracia moi! utrapienia i cierpliwosci proroków, którzy mówili w imieniu Panskiem.
\par 11 Oto za blogoslawionych mamy tych, którzy cierpieli. O cierpliwosci Ijobowej slyszeliscie i koniec Panski widzieliscie, iz wielce milosierny jest Pan i litosciwy.
\par 12 A przed wszystkiemi rzeczami, bracia moi! nie przysiegajcie ani przez niebo, ani przez ziemie, ani zadna insza przysiega; ale niech bedzie mowa wasza: Tak, tak; i Nie, nie; abyscie w oblude nie wpadli.
\par 13 Jest kto utrapiony miedzy wami, niechze sie modli; jest kto dobrej mysli, niechajze spiewa.
\par 14 Choruje kto miedzy wami, niechze zawola starszych zborowych, a niech sie modla za nim, pomazujac go olejkiem w imieniu Panskiem;
\par 15 A modlitwa wiary uzdrowi chorego i podniesie go Pan; a jezliby sie grzechu dopuscil, bedzie mu odpuszczone.
\par 16 Wyznawajcie jedni przed drugimi upadki, a módlcie sie za drugimi; abyscie byli uzdrowieni. Wiele moze uprzejma modlitwa sprawiedliwego.
\par 17 Elijasz byl czlowiek tymze biedom poddany jako i my, a modlitwa modlil sie, zeby deszcz nie padal; i nie padal deszcz na ziemie trzy lata i szesc miesiecy,
\par 18 I zas sie modlil, a wydalo niebo deszcz i ziemia zrodzila owoce swoje.
\par 19 Bracia! jezliby sie kto z was obladzil od prawdy, a nawrócilby go kto,
\par 20 Niechze wie, ze kto by odwrócil grzesznika od blednej drogi jego, zachowa dusze od smierci i zakryje mnóstwo grzechów.


\end{document}