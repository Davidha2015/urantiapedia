\begin{document}

\title{2 List św. Piotra}


\chapter{1}

\par 1 Szymon Piotr, sluga i Apostol Jezusa Chrystusa, tym, którzy równie z nami kosztownej wiary dostali przez sprawiedliwosc Boga naszego i zbawiciela naszego, Jezusa Chrystusa.
\par 2 Laska i pokój niech sie wam rozmnozy przez poznanie Boga i Jezusa, Pana naszego.
\par 3 Jako nam jego Boska moc wszystko, co do zywota i do poboznosci nalezy, darowala przez poznanie tego, który nas powolal przez slawe i przez cnote;
\par 4 Przez co bardzo wielkie i kosztowne obietnice nam sa darowane, abyscie sie przez nie stali uczestnikami Boskiego przyrodzenia, uszedlszy skazenia tego, które jest na swiecie w pozadliwosciach.
\par 5 Ku temu tedy samemu wszelkiej pilnosci przykladajac, przydajcie do wiary waszej cnote, a do cnoty umiejetnosc;
\par 6 A do umiejetnosci powsciagliwosc, a do powsciagliwosci cierpliwosc, a do cierpliwosci poboznosc;
\par 7 A do poboznosci braterska milosc, a do milosci braterskiej laske.
\par 8 Albowiem gdy to bedzie przy was, a obficie bedzie, nie próznymi, ani niepozytecznymi wystawi was w znajomosci Pana naszego, Jezusa Chrystusa.
\par 9 Bo przy kim tych rzeczy nie masz, slepy jest; a tego, co jest daleko, nie widzi, zapomniawszy na oczyszczenie od dawnych grzechów swoich.
\par 10 Przetoz, bracia! raczej sie starajcie, abyscie powolanie i wybranie wasze mocne czynili; albowiem to czyniac, nigdy sie nie potkniecie.
\par 11 Tak bowiem hojnie wam dane bedzie wejscie do wiecznego królestwa Pana naszego i Zbawiciela, Jezusa Chrystusa.
\par 12 Przetoz nie zaniedbam was zawsze upominac o tych rzeczach, chociazescie umiejetni i utwierdzeni w terazniejszej prawdzie.
\par 13 Boc to mam za sluszna rzecz, pókim jest w tym przybytku, abym was pobudzal przez napominanie,
\par 14 Wiedzac, iz predkie jest zlozenie przybytku mojego, jako mi i Pan nasz, Jezus Chrystus objawil.
\par 15 A starac sie bede o to ze wszelakiej miary, abyscie wy i po zejsciu mojem te rzeczy sobie przypominali.
\par 16 Albowiem nie basni jakich misternie wymyslonych nasladujac, uczynilismy wam znajoma Pana naszego, Jezusa Chrystusa moc i przyjscie, ale jako ci, którzysmy oczami naszemi widzieli wielmoznosc jego.
\par 17 Wzial bowiem od Boga Ojca czesc i chwale, gdy mu byl przyniesiony glos taki od wielmoznej chwaly: ten jest on Syn mój mily, w którym mi sie upodobalo.
\par 18 A glos ten mysmy slyszeli z nieba przyniesiony, bedac z nim na onej górze swietej.
\par 19 I mamy mocniejsza mowe prorocka, której pilnujac jako swiecy w ciemnem miejscu swiecacej, dobrze czynicie, azby dzien oswitnal, i jutrzenka weszla w sercach waszych.
\par 20 To najpierw wiedzac, iz zadne proroctwo Pisma nie jest wlasnego wykladu.
\par 21 Albowiem nie z woli ludzkiej przyniesione jest niekiedy proroctwo, ale od Ducha Swietego pedzeni bedac mówili swieci Bozy ludzie.

\chapter{2}

\par 1 Byli tez i falszywi prorocy miedzy ludem, jako i miedzy wami beda falszywi nauczyciele, którzy z soba wprowadza kacerstwa zatracenia i Pana, który ich kupil, zaprza sie, sami na sie przywodzac predkie zginienie.
\par 2 A wiele ich nasladowac beda zginienia ich, przez których droga prawdy bedzie bluzniona.
\par 3 I przez lakomstwo zmyslonemi slowami kupczyc beda, którym sad z dawna nie omieszkuje i zatracenie ich nie drzemie.
\par 4 Albowiem jezli Bóg Aniolom, którzy byli zgrzeszyli, nie przepuscil, ale straciwszy ich do piekla, podal lancuchom ciemnosci, aby byli zachowani na sad:
\par 5 Takze i pierwszemu swiatu nie przepuscil, ale Noego samoósmego, kaznodzieje sprawiedliwosci, zachowal, przywiódlszy potop na swiat niepoboznych;
\par 6 I miasta Sodomczyków i Gomorry w popiól obróciwszy podwróceniem potepil, wystawiwszy je na przyklad tym, którzy by niepoboznie zyli;
\par 7 A sprawiedliwego Lota, onych niezbozników rozpustnem obcowaniem strapionego, wyrwal.
\par 8 Albowiem widzeniem i slyszeniem on sprawiedliwy mieszkajac miedzy nimi, dzien po dniu dusze sprawiedliwa uczynkami ich niezboznymi trapil.
\par 9 Umie Pan poboznych z pokuszenia wyrywac, a niesprawiedliwych na dzien sadu ku karaniu chowac;
\par 10 A najwiecej tych, którzy za cialem w pozadliwosci plugastwa chodza, a zwierzchnoscia pogardzaja, smieli i sobie sie podobajacy, nie wzdrygaja sie bluznic przelozenstw.
\par 11 Chociaz Aniolowie bedac wiekszymi sila i moca, nie przynosza przeciwko nim przed Pana bluznierczego sadu.
\par 12 Ale ci, jako bydlo bezrozumne, które za przyrodzeniem idzie, sprawione na ulowienie i skaze, bluzniac to, czego nie wiedza, w tej skazie swojej zagina.
\par 13 I odniosa zaplate niesprawiedliwosci jako ci, którzy maja za rozkosz kazdodzienne lubosci, bedac plugastwem i zmaza, rozkosz maja w zdradach swoich z wami bankietujac:
\par 14 Oczy maja pelne cudzolóstwa i bez przestania grzeszace, przyludzajac dusze niestateczne, majac serce wycwiczone w lakomstwie, synowie przeklestwa,
\par 15 Którzy opusciwszy prosta droge, zbladzili, nasladujac drogi Balaama, syna Bosorowego, który zaplate niesprawiedliwosci umilowal;
\par 16 Ale mial karanie za swój wystepek, poniewaz jarzmu niema oslica poddana, czlowieczym glosem przemówiwszy, zahamowala szalenstwo proroka.
\par 17 Ci sa studniami bez wody, obloki od wichru pedzone, którym chmura ciemnosci na wieki jest zachowana.
\par 18 Albowiem nadeta próznosc mówiac, przyludzaja przez pozadliwosc ciala i rozpusty tych, którzy byli prawdziwie uciekli od obcujacych w bledzie,
\par 19 Wolnosc im obiecujac, a sami bedac niewolnikami skazy. Albowiem kto jest od kogo przezwyciezony, temu tez jest zniewolony.
\par 20 Bo poniewaz oni uszli plugastw swiata przez poznanie Pana i zbawiciela, Jezusa Chrystusa, a znowu sie zas niemi uwiklawszy, zwyciezeni bywaja; staly sie ich ostateczne rzeczy gorsze niz pierwsze.
\par 21 Bo by im bylo lepiej, nie uznac drogi sprawiedliwosci, nizeli poznawszy ja, odwrócic sie od podanego im rozkazania swietego.
\par 22 Alec sie im przydalo wedlug onej prawdziwej przypowiesci: Pies wrócil sie do zwracania swego, a swinia umyta do walania sie w blocie.

\chapter{3}

\par 1 Najmilsi! juz ten drugi list do was pisze, którym wzbudzam przez napominanie uprzejma mysl wasze,
\par 2 Abyscie pamietali na slowa przepowiedziane od swietych proroków, i na przykazanie nasze, którzysmy Apostolami Pana i Zbawiciela:
\par 3 To najpierwej wiedzac, ze przyjda w ostateczne dni nasmiewcy, wedlug wlasnych swoich pozadliwosci chodzacy,
\par 4 I mówiacy: Gdziez jest obietnica przyjscia jego? Bo jako ojcowie zasneli, wszystko tak trwa od poczatku stworzenia.
\par 5 Tego zaiste umyslnie wiedziec nie chca, ze sie niebiosa dawno staly i ziemia z wody i w wodzie stanela przez slowo Boze,
\par 6 Dlaczego on pierwszy swiat woda bedac zatopiony, zginal.
\par 7 Lecz te niebiosa, które teraz sa i ziemia temze slowem odlozone sa i zachowane ogniowi na dzien sadu i zatracenia niepoboznych ludzi.
\par 8 Ale ta jedna rzecz niech wam nie bedzie tajna, najmilsi! iz jeden dzien u Pana jest jako tysiac lat, a tysiac lat, jako jeden dzien.
\par 9 Nie omieszkiwac Pan z obietnica, (jako to niektórzy maja za omieszkanie), ale uzywa cierpliwosci przeciwko nam, nie chcac, aby którzy zgineli, ale zeby sie wszyscy do pokuty udali.
\par 10 A on dzien Panski przyjdzie jako zlodziej w nocy, w który niebiosa z wielkim trzaskiem przemina, a zywioly rozpalone ogniem stopnieja, a ziemia i rzeczy, które sa na niej, spalone beda.
\par 11 Poniewaz sie tedy to wszystko ma rozplynac, jakimiz wy macie byc w swietych obcowaniach i poboznosciach?
\par 12 Którzy oczekujecie i spieszycie sie na przyjscie dnia Bozego, w który niebiosa gorejace rozpuszcza sie i zywioly palajace stopnieja.
\par 13 Lecz nowych niebios i nowej ziemi wedlug obietnicy jego oczekujemy, w których sprawiedliwosc mieszka.
\par 14 Przetoz najmilsi! tego oczekujac, starajcie sie, abyscie bez zmazy i bez nagany od niego znalezieni byli w pokoju;
\par 15 A nieskwapliwosc Pana naszego miejcie za zbawienie wasze, jako wam i mily brat nasz Pawel wedlug danej sobie madrosci pisal,
\par 16 Jako i we wszystkich listach swoich mówiac o tych rzeczach, miedzy któremi sa niektóre rzeczy trudne ku wyrozumieniu, które nieumiejetni i niestateczni wykrecaja jako i inne pisma, ku swemu wlasnemu zatraceniu.
\par 17 Wy tedy, najmilsi! wiedzac to przedtem; strzezcie sie, abyscie bledem tych niezbozników nie byli zwiedzeni i nie wypadli z waszej statecznosci;
\par 18 Ale rosccie w lasce i w znajomosci Pana naszego i Zbawiciela, Jezusa Chrystusa, któremu niech bedzie chwala i teraz, i na czasy wieczne. Amen.


\end{document}