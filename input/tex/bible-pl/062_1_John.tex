\begin{document}

\title{1 List św. Jana}


\chapter{1}

\par 1 Co bylo od poczatku, cosmy slyszeli, cosmy oczyma naszemi widzieli i na cosmy patrzyli, i czego sie rece nasze dotykaly, o Slowie zywota;
\par 2 (Bo zywot objawiony jest i widzielismy, i swiadczymy i zwiastujemy wam on zywot wieczny, który byl u Ojca, i objawiony nam jest.)
\par 3 Cosmy, mówie, widzieli i slyszeli, to wam zwiastujemy, abyscie i wy z nami spolecznosc mieli, a spolecznosc nasza, aby byla z Ojcem i z Synem jego, Jezusem Chrystusem.
\par 4 A toc wam piszemy, aby radosc wasza zupelna byla.
\par 5 A toc jest poselstwo, któresmy slyszeli od niego i zwiastujemy wam: Iz Bóg jest swiatlosc, a zadnej ciemnosci w nim nie masz.
\par 6 Jezlibysmy rzekli, iz spolecznosc mamy z nim, a w ciemnosci chodzimy, klamiemy, a nie czynimy prawdy.
\par 7 A jezli w swiatlosci chodzimy, jako on jest w swiatlosci, spolecznosc mamy miedzy soba, a krew Jezusa Chrystusa, Syna jego, oczyszcza nas od wszelkiego grzechu.
\par 8 Jezlibysmy rzekli, iz grzechu nie mamy, sami siebie zwodzimy, a prawdy w nas nie masz.
\par 9 Jezlibysmy wyznali grzechy nasze, wiernyc jest Bóg i sprawiedliwy, aby nam odpuscil grzechy i oczyscil nas od wszelkiej nieprawosci.
\par 10 Jezlismy rzekli, zesmy nie zgrzeszyli, klamca go czynimy, a slowa jego nie masz w nas.

\chapter{2}

\par 1 Dziatki moje! to wam pisze, abyscie nie grzeszyli; i jezliby kto zgrzeszyl, mamy oredownika u Ojca, Jezusa Chrystusa sprawiedliwego;
\par 2 A on jest ublaganiem za grzechy nasze; a nie tylko za nasze, ale tez za grzechy wszystkiego swiata.
\par 3 A przez to wiemy, zesmy go poznali, jezli przykazania jego zachowujemy.
\par 4 Kto mówi: Znam go, a przykazania jego nie zachowuje, klamca jest, a prawdy w nim nie masz.
\par 5 Lecz kto by zachowal slowa jego, prawdziwie sie w tym milosc Boza wykonala; przez to znamy, iz w nim jestesmy.
\par 6 Kto mówi, ze w nim mieszka, powinien, jako on chodzil, i sam takze chodzic.
\par 7 Bracia! nie nowe przykazanie wam pisze, ale przykazanie stare, którescie mieli od poczatku; a to stare przykazanie jest ono slowo, którescie slyszeli od poczatku.
\par 8 Zasie przykazanie nowe pisze wam, które jest prawdziwe w nim i w was; iz ciemnosc przemija, a prawdziwa ona swiatlosc juz swieci.
\par 9 Kto mówi, iz jest w swiatlosci, a brata swego nienawidzi, w ciemnosci jest az dotad.
\par 10 Kto miluje brata swego, w swiatlosci mieszka i zgorszenia w nim nie masz.
\par 11 Lecz kto nienawidzi brata swego, w ciemnosci jest i w ciemnosci chodzi, a nie wie, gdzie idzie, iz ciemnosc zaslepila oczy jego.
\par 12 Pisze wam, dziatki! iz wam sa odpuszczone grzechy dla imienia jego.
\par 13 Pisze wam, ojcowie! zescie poznali tego, który jest od poczatku. Pisze wam, mlodziency! zescie zwyciezyli onego zlosnika.
\par 14 Pisze wam, dziateczki! zescie poznali Ojca. Pisalem wam, ojcowie! zescie poznali onego, który jest od poczatku. Pisalem wam, mlodziency! ze jestescie mocni, a slowo Boze mieszka w was, a zescie zwyciezyli onego zlosnika.
\par 15 Nie milujcie swiata, ani tych rzeczy, które sa na swiecie; jezli kto miluje swiat, nie masz w nim milosci ojcowskiej.
\par 16 Albowiem wszystko, co jest na swiecie, jako pozadliwosc ciala i pozadliwosc oczu, i pycha zywota, toc nie jest z Ojca, ale jest z swiata.
\par 17 Swiatci przemija i pozadliwosc jego; ale kto czyni wole Boza, trwa na wieki.
\par 18 Dziateczki! ostateczna godzina jest; a jakoscie slyszeli, ze antychryst przyjsc ma, i teraz wiele antychrystów powstalo; stad wiemy, iz jest ostateczna godzina.
\par 19 Z nas wyszli, ale nie byli z nas; albowiem gdyby byli z nas, zostaliby byli z nami; ale wyszli z nas, aby objawieni byli, iz wszyscy nie byli z nas.
\par 20 Ale wy macie pomazanie od onego Swietego i wiecie wszystko.
\par 21 Nie pisalem wam, przeto zescie prawdy nie znali, ale ze ja znacie, a iz wszelkie klamstwo nie jest z prawdy.
\par 22 Kto jest klamca? Azaz nie ten, który zapiera, iz Jezus nie jest Chrystusem? Ten jest antychryst, który sie zapiera Ojca i Syna.
\par 23 Kazdy, co sie zapiera Syna, i Ojca nie ma; a kto wyznaje Syna, ma i Ojca.
\par 24 Wy tedy, coscie slyszeli od poczatku, to niechaj w was zostaje; jezliby w was zostawalo, coscie slyszeli od poczatku, i wy w Synu i w Ojcu zostaniecie.
\par 25 A tac jest obietnica, która on nam obiecal, to jest zywot on wieczny.
\par 26 Tom wam napisal o tych, którzy was zwodza.
\par 27 Ale to pomazanie, którescie wy wzieli od niego, zostaje w was, a nie potrzebujecie, aby was kto uczyl: ale jako to pomazanie uczy was o wszystkiem, a jest prawdziwe, i nie jest klamstwem, a jako was nauczylo, tak w niem zostaniecie.
\par 28 I teraz, dziateczki! zostancie w niem, abysmy, gdy sie ukaze, ufanie mieli, a nie byli zawstydzeni od niego w przyjsciu jego.
\par 29 Poniewaz wiecie, ze on sprawiedliwy jest, wiedzciez tez, iz kazdy, który czyni sprawiedliwosc, z niego narodzony jest.

\chapter{3}

\par 1 Patrzcie, jaka milosc dal nam Ojciec, abysmy dziatkami Bozemi nazwani byli. Dlategoc swiat nie zna nas, iz onego nie zna.
\par 2 Najmilsi! teraz dziatkami Bozemi jestesmy, ale sie jeszcze nie objawilo, czem bedziemy; lecz wiemy, iz gdy sie on objawi, podobni mu bedziemy; albowiem ujrzymy go tak, jako jest.
\par 3 A ktokolwiek ma te nadzieje w nim, oczyszcza sie, jako i on czysty jest.
\par 4 Kazdy, co czyni grzech, ten i zakon przestepuje; albowiem grzech jest przestepstwo zakonu.
\par 5 A wiecie, iz sie on objawil, aby grzechy nasze zgladzil, a grzechu w nim nie masz.
\par 6 Wszelki tedy, kto w nim mieszka, nie grzeszy; ale ktokolwiek grzeszy, nie widzial go, ani go poznal.
\par 7 Dziateczki! niechaj was nikt nie zwodzi; kto czyni sprawiedliwosc, sprawiedliwy jest, jako i on sprawiedliwy jest;
\par 8 Kto czyni grzech, z dyjabla jest; gdyz od poczatku dyjabel grzeszy. Na to sie objawil Syn Bozy, aby zepsowal uczynki dyjabelskie.
\par 9 Wszelki, co sie narodzil z Boga, grzechu nie czyni, iz nasienie jego w nim zostaje, i nie moze grzeszyc, iz z Boga narodzony jest.
\par 10 Po tem poznac dziatki Boze i dzieci dyjabelskie. Wszelki, który nie czyni sprawiedliwosci, nie jest z Boga, i który nie miluje brata swego.
\par 11 Albowiem to jest poselstwo, którescie slyszeli od poczatku, abysmy jedni drugich milowali.
\par 12 Nie jako Kain, który byl z tego zlosnika i zabil brata swego. A dlaczegoz go zabil? Iz uczynki jego zle byly, a brata jego sprawiedliwe.
\par 13 Nie dziwujcie sie, bracia moi! jezli was swiat nienawidzi.
\par 14 My wiemy, zesmy przeniesieni z smierci do zywota, iz milujemy braci; kto nie miluje brata, zostaje w smierci.
\par 15 Kazdy, co nienawidzi brata swego, mezobójca jest; a wiecie, iz zaden mezobójca nie ma zywota wiecznego w sobie zostawajacego.
\par 16 Przez tosmy poznali milosc Boza, iz on dusze swoje za nas polozyl; i mysmy powinni klasc dusze za braci.
\par 17 A kto by mial majetnosc swiata tego i widzialby brata swego potrzebujacego, a zawarlby wnetrznosci swoje przed nim, jakoz w nim zostaje milosc Boza?
\par 18 Dziateczki moje! nie milujmy slowem ani jezykiem, ale uczynkiem i prawda.
\par 19 A przez to poznajemy, iz z prawdy jestesmy i przed nim uspokojemy serca nasze.
\par 20 Bo jezliby nas potepialo serce nasze, daleko wiekszy jest Bóg niz serce nasze i wie wszystko.
\par 21 Najmilsi! jezliby serce nasze nas nie potepialo, ufanie mamy ku Bogu;
\par 22 I o cokolwiek bysmy prosili, bierzemy od niego; bo przykazania jego chowamy i to, co sie podoba przed obliczem jego, czynimy.
\par 23 A toc jest przykazanie jego, abysmy wierzyli imieniowi Syna jego, Jezusa Chrystusa, i milowali jedni drugich, jako nam przykazal.
\par 24 Bo kto chowa przykazania jego, w nim mieszka, a on tez w nim; a przez to znamy, iz mieszka w nas, to jest z Ducha, którego nam dal.

\chapter{4}

\par 1 Najmilsi! nie kazdemu duchowi wierzcie; ale doswiadczajcie duchów, jezli z Boga sa. Bo wiele falszywych proroków wyszlo na swiat.
\par 2 Przez to poznawajcie Ducha Bozego: Wszelki duch, który wyznaje, iz Jezus Chrystus w ciele przyszedl, z Boga jest.
\par 3 Ale wszelki duch, który nie wyznaje, ze Jezus Chrystus w ciele przyszedl, nie jest z Boga; ale ten jest on duch antychrystowy, o którymescie slyszeli, iz idzie i teraz juz jest na swiecie.
\par 4 Wy z Boga jestescie, dziateczki! i zwyciezyliscie ich; iz wiekszy jest ten, który w was jest, niz ten, który jest na swiecie.
\par 5 Onic sa z swiata; przetoz o swiecie mówia, a swiat ich slucha.
\par 6 My z Boga jestesmy. Kto zna Boga, slucha nas; kto nie jest z Boga, nie slucha nas. Przez to poznajemy ducha prawdy i ducha bledu.
\par 7 Najmilsi! milujmyz jedni drugich, gdyz milosc jest z Boga; i kazdy, co miluje, z Boga jest narodzony i zna Boga.
\par 8 Kto nie miluje, nie zna Boga; gdyz Bóg jest milosc.
\par 9 Przez to objawiona jest milosc Boza ku nam, iz Syna swego jednorodzonego poslal Bóg na swiat, abysmy zyli przez niego.
\par 10 W tem jest milosc, nie izbysmy my umilowali Boga, ale iz on umilowal nas i poslal Syna swego, aby byl ublaganiem za grzechy nasze.
\par 11 Najmilsi! poniewaz nas tak Bóg umilowal, i mysmy powinni jedni drugich milowac.
\par 12 Boga zaden nigdy nie widzial; ale jezli milujemy jedni drugich, Bóg w nas mieszka, a milosc jego doskonala jest w nas.
\par 13 Przez to poznajemy, iz w nim mieszkamy, a on w nas, iz z Ducha swego nam dal.
\par 14 A mysmy widzieli i swiadczymy, iz Ojciec poslal Syna, aby byl Zbawicielem swiata.
\par 15 Ktobykolwiek wyznal, iz Jezus jest Synem Bozym, Bóg w nim mieszka, a on w Bogu.
\par 16 I mysmy poznali i uwierzyli o milosci, która Bóg ma ku nam. Bóg jest milosc; a kto mieszka w milosci, w Bogu mieszka, a Bóg w nim.
\par 17 W tem doskonala jest milosc Boza z nami, abysmy ufanie mieli w dzien sadny, iz jaki on jest, tacy i my jestesmy na tym swiecie.
\par 18 Nie maszci bojazni w milosci, ale milosc doskonala precz wyrzuca bojazn: bo bojazn ma udreczenie, a kto sie boi, nie jest doskonaly w milosci.
\par 19 My go milujemy, iz on nas pierwej umilowal.
\par 20 Jezliby kto rzekl: Miluje Boga, a brata by swego nienawidzil, klamca jest; albowiem kto nie miluje brata swego, którego widzial, Boga, którego nie widzial, jakoz moze milowac?
\par 21 A toc rozkazanie mamy od niego, aby ten, co miluje Boga, milowal i brata swego.

\chapter{5}

\par 1 Wszelki, co wierzy, iz Jezus jest Chrystusem, z Boga sie narodzil; a wszelki, co miluje tego, który urodzil, miluje i tego, który z niego jest narodzony.
\par 2 Przez to znamy, iz milujemy dziatki Boze, gdy Boga milujemy i przykazania jego chowamy.
\par 3 Albowiem ta jest milosc Boza, abysmy przykazania jego chowali; a przykazania jego nie sa ciezkie.
\par 4 Bo wszystko, co sie narodzilo z Boga, zwycieza swiat; a to jest zwyciestwo, które zwyciezylo swiat, wiara nasza.
\par 5 Któz jest, co zwycieza swiat, tylko kto wierzy, iz Jezus jest Synem Bozym?
\par 6 Tenci jest, który przyszedl przez wode i krew, Jezus Chrystus, nie w wodzie tylko, ale w wodzie i we krwi; a Duch jest, który swiadczy, iz Duch jest prawda.
\par 7 Albowiem trzej sa, którzy swiadcza na niebie: Ojciec, Slowo i Duch Swiety, a ci trzej jedno sa.
\par 8 A trzej sa, którzy swiadcza na ziemi: Duch i woda, i krew, a ci trzej ku jednemu sa.
\par 9 Poniewaz swiadectwo ludzkie przyjmujemy, swiadectwo Boze wieksze jest; albowiem to jest swiadectwo Boze, które swiadczyl o Synu swoim.
\par 10 Kto wierzy w Syna Bozego, ma swiadectwo sam w sobie. Kto nie wierzy Bogu, klamca go uczynil, iz nie uwierzyl temu swiadectwu, które Bóg swiadczyl o Synu swoim.
\par 11 A toc jest swiadectwo, iz nam Bóg dal zywot wieczny; a ten zywot jest w Synu jego.
\par 12 Kto ma Syna, ma zywot; kto nie ma Syna Bozego, nie ma zywota.
\par 13 Te rzeczy napisalem wam, którzy wierzycie w imie Syna Bozego, zebyscie wiedzieli, iz macie zywot wieczny, i abyscie wierzyli w imie Syna Bozego.
\par 14 A toc jest ufanie, które mamy do niego, iz jezlibysmy o co prosili wedlug woli jego, slyszy nas.
\par 15 A jezli wiemy, iz nas slyszy, o cokolwiek bysmy prosili, tedy wiemy, iz mamy te rzeczy, o któresmy go prosili.
\par 16 Jezliby kto widzial brata swego grzeszacego grzechem nie na smierc, niechze sie modli za nim, a da mu Bóg zywot, to jest grzeszacym nie na smierc. Jestci grzech na smierc; nie za tym, mówie, aby sie kto modlil.
\par 17 Wszelka niesprawiedliwosc jest grzech; ale jest grzech nie na smierc.
\par 18 Wiemy, iz wszelki, który sie z Boga narodzil, nie grzeszy; ale który sie narodzil z Boga, zachowuje samego siebie, a on zlosnik nie dotyka sie go.
\par 19 Wiemy, iz z Boga jestesmy; ale swiat wszystek w zlem polozony jest.
\par 20 A wiemy, iz Syn Bozy przyszedl i dal nam zmysl, abysmy poznali onego prawdziwego Boga, i jestesmy w onym prawdziwym, to jest w Synu jego, Jezusie Chrystusie; tenci jest prawdziwy Bóg i zywot wieczny.
\par 21 Dziateczki! strzezcie sie balwanów. Amen.


\end{document}