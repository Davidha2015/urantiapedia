\begin{document}

\title{Apokalipsa św. Jana}


\chapter{1}

\par 1 Objawienie Jezusa Chrystusa, które mu dal Bóg, aby okazal slugom swoim rzeczy, które sie w rychle dziac maja; a on je oznajmil i poslal przez Aniola swojego sludze swemu Janowi,
\par 2 Który swiadectwo wydal slowu Bozemu i swiadectwu Jezusa Chrystusa, i cokolwiek widzial.
\par 3 Blogoslawiony, który czyta i ci, którzy sluchaja slów proroctwa tego, i zachowuja to, co w niem jest napisane; albowiem czas blisko jest.
\par 4 Jan siedmiu zborom, które sa w Azyi. Laska wam i pokój niech bedzie od tego, który jest i który byl, i który przyjsc ma; i od siedmiu duchów, którzy sa przed oblicznoscia stolicy jego;
\par 5 I od Jezusa Chrystusa, który jest onym swiadkiem wiernym, pierworodnym z umarlych i ksiazeciem królów ziemi; który nas umilowal i omyl nas z grzechów naszych krwia swoja;
\par 6 I uczynil nas królami i kaplanami Bogu, Ojcu swemu; jemuz niech bedzie chwala i moc na wieki wieków. Amen.
\par 7 Oto idzie z oblokami i ujrzy go wszelkie oko, i ci, którzy go przebili; i narzekac beda przed nim wszystkie pokolenia ziemi. Tak, Amen.
\par 8 Jam jest Alfa i Omega, poczatek i koniec, mówi Pan, który jest i który byl, i który przyjsc ma, on Wszechmogacy.
\par 9 Ja Jan, którym tez jest bratem waszym i uczestnikiem w ucisku i w królestwie, i w cierpliwosci Jezusa Chrystusa, bylem na wyspie, która zowia Patmos, dla slowa Bozego i dla swiadectwa Jezusa Chrystusa:
\par 10 Bylem w zachwyceniu ducha w dzien Panski i slyszalem za soba glos wielki jako traby, mówiacy:
\par 11 Jam jest Alfa i Omega, on pierwszy i ostatni; a co widzisz, napisz w ksiegi i poslij siedmiu zborom, które sa w Azyi, do Efezu i do Smyrny, i do Pergamu, i do Tyjatyru, i do Sardów, i do Filadelfii, i do Laodycei.
\par 12 I obrócilem sie, abym widzial on glos, który mówil ze mna; a obróciwszy sie, ujrzalem siedm swieczników zlotych,
\par 13 A w posrodku onych siedmiu swieczników podobnego Synowi czlowieczemu, obleczonego w dluga szate, i przepasanego na piersiach pasem zlotym;
\par 14 A glowa jego i wlosy byly biale jako welna biala, jako snieg, a oczy jego jako plomien ognia;
\par 15 A nogi jego podobne mosiadzowi, jakoby w piecu rozpalone, a glos jego jako glos wielu wód.
\par 16 I mial w prawej rece swojej siedm gwiazd, a z ust jego wychodzil miecz z obu stron ostry, a oblicze jego jako slonce, kiedy jasno swieci.
\par 17 A gdym go ujrzal, upadlem do nóg jego jako martwy. I wlozyl prawa reke swoje na mie, mówiac mi: Nie bój sie! Jam jest on pierwszy i ostatni,
\par 18 I zyjacy; a bylem umarly, a otom jest zywy na wieki wieków. I mam klucze piekla i smierci.
\par 19 Napisz te rzeczy, któres widzial i które sa, i które sie dziac maja napotem.
\par 20 Tajemnice onych siedmiu gwiazd, któres widzial w prawej rece mojej i siedmiu swieczników zlotych. Siedm onych gwiazd sa Aniolowie siedmiu zborów, a siedm swieczników, któres widzial, jest siedm zborów.

\chapter{2}

\par 1 Aniolowi zboru Efeskiego napisz: To mówi ten, który trzyma siedm gwiazd w prawej rece swojej, który sie przechadza w posród onych siedmiu swieczników zlotych:
\par 2 Znam uczynki twoje i prace twoje, i cierpliwosc twoje, a iz nie mozesz cierpiec zlych i doswiadczyles tych, którzy sie mienia byc Apostolami, a nie sa, i znalazles ich, ze sa klamcami;
\par 3 I masz cierpliwosc, i znaszales i pracowales dla imienia mego, a nie ustales.
\par 4 Ale mam nieco przeciwko tobie, zes milosc twoje pierwsza opuscil.
\par 5 Pamietajze tedy, skades wypadl, a pokutuj i czyn uczynki pierwsze; a jezli nie chcesz, przyjde przeciwko tobie rychlo, a porusze swiecznik twój z miejsca swego, jezlibys nie pokutowal.
\par 6 Ale wzdy to masz, iz nienawidzisz uczynków Nikolaitów, których i ja nienawidze.
\par 7 Kto ma uszy, niechaj slucha, co Duch mówi zborom: Temu, co zwyciezy, dam jesc z drzewa zywota, które jest w posrodku raju Bozego.
\par 8 A Aniolowi zboru Smyrnenskiego napisz: To mówi pierwszy i ostatni, który byl umarl i ozyl:
\par 9 Znam uczynki twoje i ucisk, i ubóstwo, (ales bogaty), i bluznierstwo tych, którzy sie powiadaja byc Zydami, a nie sa, ale sa bóznica szatanska.
\par 10 Nic sie nie bój tego, co masz cierpiec. Oto wrzuci dyjabel niektórych z was do wiezienia, abyscie byli doswiadczeni; i bedziecie mieli ucisk przez dziesiec dni. Badz wierny az do smierci, a dam ci korone zywota.
\par 11 Kto ma uszy, niechaj slucha, co Duch mówi zborom: Kto zwyciezy, nie bedzie obrazony od wtórej smierci.
\par 12 A Aniolowi zboru Pergamenskiego napisz: To mówi ten, który ma miecz on z obydwóch stron ostry;
\par 13 Znam uczynki twoje i gdzie mieszkasz, to jest, gdzie jest stolica szatanska, a iz trzymasz imie moje i nie zaprzales sie wiary mojej, i w one dni, w które Antypas, swiadek mój wierny, zabity jest u was, gdzie szatan mieszka.
\par 14 Ale mam nieco przeciwko tobie, iz tam masz tych, którzy trzymaja nauke Balaamowa, który uczyl Balaka, aby wrzucil zgorszenie przed syny Izraelskie, zeby jedli rzeczy balwanom ofiarowane i wszeteczenstwo plodzili.
\par 15 Takze masz i tych, którzy trzymaja nauke Nikolaitów, co ja mam w nienawisci.
\par 16 Pokutujze: a jezli nie bedziesz, przyjde przeciwko tobie w rychle i bede walczyl z nimi mieczem ust moich.
\par 17 Kto ma uszy, niechaj slucha, co Duch mówi zborom: Temu, co zwyciezy, dam jesc z onej manny skrytej i dam mu kamyk bialy, a na onym kamyku imie nowe napisane, którego nikt nie zna, tylko ten, który je przyjmuje.
\par 18 A Aniolowi zboru Tyjatyrskiego napisz: To mówi syn Bozy, który ma oczy swoje jako plomien ognia, a nogi jego podobne sa mosiadzowi:
\par 19 Znam uczynki twoje i milosc, i poslugi, i wiare, i cierpliwosc twoje, i uczynki twoje, a ze ostatnich rzeczy wiecej jest niz pierwszych.
\par 20 Ale mam nieco przeciwko tobie, iz niewiescie Jezabeli, która sie mieni byc prorokinia, dopuszczasz uczyc i zwodzic slugi moje, zeby wszeteczenstwo plodzili i rzeczy balwanom ofiarowane jedli.
\par 21 I dalem jej czas, aby pokutowala z wszeteczenstwa swego; ale nie pokutowala.
\par 22 Oto ja porzuce ja na loze i tych, którzy z nia cudzoloza, w ucisk wielki, jezliby nie pokutowali z uczynków swoich:
\par 23 A dzieci jej pobije na smierc; i poznaja wszystkie zbory, zem ja jest ten, który sie badam nerek i serc; i dam kazdemu z was wedlug uczynków waszych.
\par 24 A wam mówie i drugim, którzyscie w Tyjatyrzech, którzykolwiek nie maja tej nauki i którzy nie poznali glebokosci szatanskich, jako mówia: Nie wloze na was innego brzemienia.
\par 25 Wszakze to, co macie, trzymajcie, az przyjde.
\par 26 A kto zwyciezy i zachowa az do konca uczynki moje, dam mu zwierzchnosc nad poganami.
\par 27 I bedzie ich rzadzil laska zelazna; jako statki garncarskie skruszeni beda, jakom i ja wzial od Ojca mego.
\par 28 I dam mu gwiazde poranna.
\par 29 Kto ma uszy, niechaj slucha, co Duch mówi zborom.

\chapter{3}

\par 1 A aniolowi zboru, który jest w Sardziech, napisz: To mówi ten, który ma siedm duchów Bozych i siedm gwiazd: Znam uczynki twoje, i masz imie, ze zyjesz; ales jest umarly.
\par 2 Badz czujny, a utwierdzaj innych, którzy umrzec maja; albowiem nie znalazlem uczynków twoich zupelnych przed Bogiem.
\par 3 Pamietaj tedy, jakos wzial i slyszal, a chowaj i pokutuj. Jezli tedy czuc nie bedziesz, przyjde na cie jako zlodziej, a nie zrozumiesz, której godziny przyjde na cie.
\par 4 Ale masz niektóre osoby w Sardziech, które nie pokalaly szat swoich; przetoz chodzic beda ze mna w szatach bialych, iz godni sa.
\par 5 Kto zwyciezy, ten bedzie obleczony w szaty biale i nie wymaze imienia jego z ksiag zywota, ale wyznam imie jego przed obliczem Ojca mojego i przed Aniolami jego.
\par 6 Kto ma uszy, niechaj slucha, co Duch mówi zborom.
\par 7 A Aniolowi zboru Filadelfskiego napisz: To mówi on Swiety i Prawdziwy, który ma klucz Dawidowy, który otwiera, a nikt nie zawiera i zawiera, a nikt nie otwiera.
\par 8 Znam uczynki twoje; otom wystawil przed toba drzwi otworzone, a zaden nie moze ich zamknac; bo acz masz niewielka moc, przecies zachowal slowo moje i nie zaprzales sie imienia mego.
\par 9 Otoc dam niektórych z bóznicy szatanskiej, którzy sie powiadaja byc Zydami, a nie sa, ale klamia. Oto sprawie, ze przyjda i beda sie klaniali przed nogami twemi, i poznaja, zem ja cie umilowal.
\par 10 Zes zachowal slowo cierpliwosci mojej, ja tez cie zachowam od godziny pokuszenia, która przyjdzie na wszystek swiat, aby doswiadczyla mieszkajacych na ziemi;
\par 11 Oto ide rychlo; trzymaj, co masz, aby nikt nie wzial korony twojej.
\par 12 Kto zwyciezy, uczynie go filarem w kosciele Boga mojego, a wiecej z niego juz nie wynijdzie; i napisze na nim imie Boga mego i imie miasta Boga mego, nowego Jeruzalemu, które zstepuje z nieba od Boga mego i imie moje nowe.
\par 13 Kto ma uszy, niechaj slucha, co Duch mówi zborom.
\par 14 A Aniolowi zboru Laodycenskiego napisz: To mówi Amen, swiadek on wierny i prawdziwy, poczatek stworzenia Bozego:
\par 15 Znam uczynki twoje, zes nie jest ani zimny ani goracy, bodajzes byl zimny albo goracy!
\par 16 A tak, poniewaz jestes letni, a ani zimny ani goracy, wyrzuce cie z ust moich.
\par 17 Albowiem mówisz: Jestem bogaty i zbogacilem sie, a niczego nie potrzebuje; a nie wiesz, zes ty biedny i mizerny, i ubogi, i slepy, i nagi.
\par 18 Radze ci, abys kupil u mnie zlota w ogniu doswiadczonego, abys byl bogaty, i szaty biale, abys byl obleczony, a zeby sie nie okazywala sromota nagosci twojej; a oczy twoje namaz mascia wzrok naprawiajaca, abys widzial.
\par 19 Ja którychkolwiek miluje, strofuje i karze. Badz tedy gorliwym, a pokutuj.
\par 20 Oto stoje u drzwi i kolacze; jezliby kto uslyszal glos mój i otworzyl drzwi, wnijde do niego i bede z nim wieczerzal, a on ze mna.
\par 21 Kto zwyciezy, dam mu siedziec z soba na stolicy mojej, jakom i ja zwyciezyl i usiadlem z Ojcem moim na stolicy jego.
\par 22 Kto ma uszy, niechaj slucha, co Duch mówi zborom.

\chapter{4}

\par 1 Potemem widzial, a oto drzwi byly otworzone w niebie, a glos pierwszy, którym slyszal, jako traby mówiacej ze mna, i rzekl: Wstap sam, a pokaze ci, co sie ma dziac napotem.
\par 2 A zarazem bylem w zachwyceniu ducha, a oto stolica postawiona byla na niebie, a na stolicy siedziala osoba.
\par 3 A ten, który siedzial, podobny byl na wejrzeniu kamieniowi jaspisowi i sardynowi; a okolo onej stolicy byla tecza, na wejrzeniu podobna szmaragdowi.
\par 4 A okolo onej stolicy bylo stolic dwadziescia i cztery; a na onych stolicach widzialem dwudziestu i czterech starców siedzacych, obleczonych w szaty biale, a na glowach swoich mieli korony zlote.
\par 5 A z onej stolicy wychodzily blyskawice i gromy, i glosy, i siedm lamp ognistych gorejacych przed stolica, które sa siedm duchów Bozych.
\par 6 A przed ona stolica bylo morze szklane, podobne krysztalowi, a w posrodku stolicy i okolo stolicy czworo zwierzat pelnych oczu z przodku i z tylu.
\par 7 A pierwsze zwierze podobne bylo lwowi, a wtóre zwierze podobne cielcowi, a trzecie zwierze mialo twarz jako czlowiek, a czwarte zwierze podobne bylo orlowi latajacemu.
\par 8 A oto kazde z osobna z onych czterech zwierzat mialo szesc skrzydel wokolo, a wewnatrz byly pelne oczów, a odpoczynku nie maja we dnie i w nocy, mówiac: Swiety, swiety, swiety Pan, Bóg wszechmogacy, który byl i jest, i przyjsc ma.
\par 9 A gdy one zwierzeta dawaly chwale i czesc, i dziekowanie siedzacemu na stolicy, zyjacemu na wieki wieków;
\par 10 Upadli dwadziescia cztery starcy przed obliczem siedzacego na stolicy i klaniali sie zyjacemu na wieki wieków, i rzucali korony swoje przed stolica, mówiac:
\par 11 Godzienes jest, Panie! wziac chwale i czesc, i moc; bos ty stworzyl wszystkie rzeczy i za wola twoja trwaja, i stworzone sa.

\chapter{5}

\par 1 I widzialem po prawej rece siedzacego na stolicy ksiegi napisane, wewnatrz i zewnatrz zapieczetowane siedmioma pieczeciami.
\par 2 I widzialem Aniola mocnego, wolajacego glosem wielkim: Kto jest godzien otworzyc te ksiegi i odpieczetowac pieczeci ich?
\par 3 A nikt nie mógl ani w niebie, ani na ziemi, ani pod ziemia otworzyc onych ksiag, ani wejrzec w nie.
\par 4 I plakalem bardzo, iz nikt nie byl znaleziony godny, aby otworzyl i czytal ksiegi, i wejrzal w nie.
\par 5 Tedy mi jeden z onych starców rzekl: Nie placz! Oto zwyciezyl lew, który jest z pokolenia Judowego, korzen Dawidowy, aby otworzyl ksiegi i odpieczetowal siedm pieczeci ich.
\par 6 I spojrzalem, a oto miedzy stolica i czterema onemi zwierzetami, i miedzy onymi starcami Baranek stal jako zabity, majac siedm rogów i siedm oczy, które sa siedm duchów Bozych, poslanych na wszystke ziemie.
\par 7 Ten przyszedl i wzial one ksiegi z prawej reki siedzacego na stolicy.
\par 8 A gdy wzial one ksiegi, zaraz ono czworo zwierzat i oni dwadziescia i cztery starcy upadli przed Barankiem, majac kazdy z nich cytry i czasze zlote, pelne wonnych rzeczy, które sa modlitwy swietych.
\par 9 I spiewali nowa piesn, mówiac: Godzienes jest wziac ksiegi i otworzyc pieczeci ich, zes byl zabity i odkupiles nas Bogu przez krew swoje ze wszelkiego pokolenia i jezyka, i ludu, i narodu:
\par 10 I uczyniles nas Bogu naszemu królami i kaplanami, i królowac bedziemy na ziemi.
\par 11 I widzialem, i slyszalem glos wielu Aniolów okolo onej stolicy, i onych zwierzat i onych starców; a byla liczba ich tysiackroc sto tysiecy i dziesieckroc sto tysiecy,
\par 12 Mówiacych glosem wielkim: Godzien jest ten Baranek zabity, wziac moc i bogactwo, i madrosc, i sile, i czesc, i chwale, i blogoslawienstwo.
\par 13 A wszelkie stworzenie, które jest na niebie i na ziemi, i pod ziemia i w morzu, i wszystko, co w nich jest, slyszalem mówiace: Siedzacemu na stolicy i Barankowi blogoslawienstwo i czesc, i chwala, i sila na wieki wieków.
\par 14 A czworo onych zwierzat rzeklo: Amen. A oni dwadziescia i cztery starcy upadli i klaniali sie zyjacemu na wieki wieków.

\chapter{6}

\par 1 I widzialem, gdy otworzyl Baranek jedna z onych pieczeci, i slyszalem jedno ze czterech zwierzat mówiace, jako glos gromu: Chodz, a patrzaj!
\par 2 I widzialem, a oto kon bialy, a ten, który na nim siedzial, mial luk, i dano mu korone, i wyszedl jako zwyciezca, azeby zwyciezal.
\par 3 A gdy otworzyl wtóra pieczec, slyszalem wtóre zwierze mówiace: Chodz, a patrzaj!
\par 4 I wyszedl drugi kon rydzy; a temu, który na nim siedzial, dano, aby odjal pokój z ziemi, a izby jedni drugich zabijali, i dano mu miecz wielki.
\par 5 A gdy otworzyl trzecia pieczec, slyszalem trzecie zwierze mówiace: Chodz, a patrzaj! I widzialem, a oto kon wrony, a ten, co na nim siedzial, mial szale w rece swojej.
\par 6 I slyszalem glos z posrodku onych czworga zwierzat mówiacy: Miarka pszenicy za grosz, a trzy miarki jeczmienia za grosz; a nie szkodz oliwie i winu.
\par 7 A gdy otworzyl czwarta pieczec, slyszalem glos czwartego zwierzecia mówiacy: Chodz, a patrzaj!
\par 8 I widzialem, a oto kon plowy, a tego, który siedzial na nim, imie bylo smierc, a pieklo szlo za nim; i dana im jest moc nad czwarta czescia ziemi, aby zabijali mieczem i glodem, i morem, i przez zwierzeta ziemskie.
\par 9 A gdy otworzyl piata pieczec, widzialem pod oltarzem dusze pobitych dla slowa Bozego i dla swiadectwa, które wydawali;
\par 10 I wolali glosem wielkim, mówiac: Dokadze, Panie swiety i prawdziwy! nie sadzisz i nie mscisz sie krwi naszej nad tymi, którzy mieszkaja na ziemi?
\par 11 I dane sa kazdemu z nich szaty biale, i powiedziano im, aby odpoczywali jeszcze na maly czas, azby sie dopelnil poczet spólslug ich i braci ich, którzy maja byc pobici, jako i oni.
\par 12 I widzialem, gdy otworzyl szósta pieczec, a oto stalo sie wielkie trzesienie ziemi, a slonce sczernialo jako wór wlosiany i ksiezyc wszystek stal sie jako krew;
\par 13 A gwiazdy niebieskie padaly na ziemie, tak jako drzewo figowe zrzuca z siebie figi swoje niedostale, gdy od wiatru wielkiego bywa zachwiane.
\par 14 A niebo ustapilo jako ksiegi zwinione, a wszelka góra i wyspy z miejsca sie swego poruszyly;
\par 15 A królowie ziemi i ksiazeta, i bogacze, i hetmani, i mocarze, i kazdy niewolnik, i kazdy wolny pokryli sie w jaskinie i w skaly gór,
\par 16 I rzekli górom i skalom: Upadnijcie na nas i zakryjcie nas przed obliczem tego, który siedzi na stolicy i przed gniewem tego Baranka;
\par 17 Albowiem przyszedl dzien on wielki gniewu jego, i któz sie ostac moze?

\chapter{7}

\par 1 Potemem widzial czterech Aniolów, stojacych na czterech weglach ziemi, trzymajacych cztery wiatry ziemi, aby nie wial wiatr na ziemie, ani na morze, ani na zadne drzewo.
\par 2 I widzialem inszego Aniola wystepujacego od wschodu slonca, majacego pieczec Boga zywego, i zawolal glosem wielkim na onych czterech Aniolów, którym dano, aby szkodzili ziemi i morzu;
\par 3 Mówiac: Nie szkodzcie ziemi ani morzu, ani drzewom, az popieczetujemy slugi Boga naszego na czolach ich.
\par 4 I uslyszalem liczbe popieczetowanych: sto czterdziesci i cztery tysiace jest popieczetowanych ze wszystkich pokolen synów Izraelskich:
\par 5 Z pokolenia Judowego dwanascie tysiecy popieczetowanych; z pokolenia Rubenowego dwanascie tysiecy popieczetowanych; z pokolenia Gadowego dwanascie tysiecy popieczetowanych;
\par 6 Z pokolenia Aserowego dwanascie tysiecy popieczetowanych; z pokolenia Neftalimowego dwanascie tysiecy popieczetowanych; z pokolenia Manasesowego dwanascie tysiecy popieczetowanych;
\par 7 Z pokolenia Symeonowego dwanascie tysiecy popieczetowanych; z pokolenia Lewiego dwanascie tysiecy popieczetowanych; z pokolenia Isacharowego dwanascie tysiecy popieczetowanych;
\par 8 Z pokolenia Zabulonowego dwanascie tysiecy popieczetowanych; z pokolenia Józefowego dwanascie tysiecy popieczetowanych; z pokolenia Benjaminowego dwanascie tysiecy popieczetowanych.
\par 9 Potemem widzial, a oto lud wielki, którego nie mógl nikt zliczyc, z kazdego narodu i pokolenia, i ludzi, i jezyków, którzy stali przed stolica i przed oblicznoscia Baranka, obleczeni w szaty biale, a palmy byly w rekach ich.
\par 10 I wolali glosem wielkim, mówiac: Zbawienie nalezy Bogu naszemu, siedzacemu na stolicy i Barankowi.
\par 11 A wszyscy Aniolowie stali okolo stolicy i starców i czworga zwierzat, i upadli przed stolica na oblicze swoje, i klaniali sie Bogu,
\par 12 Mówiac: Amen! Blogoslawienstwo i chwala, i madrosc, i dziekowanie, i czesc, i moc, i sila Bogu naszemu na wieki wieków. Amen.
\par 13 I odpowiedzial jeden z onych starców i rzekl mi: Ci, którzy sa obleczeni w szaty biale, co zacz sa? i skad przyszli?
\par 14 I rzeklem mu: Panie! ty wiesz. I rzekl mi: Cic sa, którzy przyszli z ucisku wielkiego i omyli szaty swoje, i wybielili je we krwi Barankowej.
\par 15 Dlatego sa przed stolica Boza i sluza mu we dnie i w nocy w kosciele jego, a ten, który siedzi na stolicy, jako namiotem zasloni ich.
\par 16 Nie beda wiecej laknac i nie beda wiecej pragnac, i nie uderzy na nich slonce, ani zadne goraco
\par 17 Albowiem Baranek, który jest w posrodku stolicy, bedzie ich pasl i poprowadzi ich do zywych zródel wód, i otrze Bóg wszelka lze z oczów ich.

\chapter{8}

\par 1 A gdy otworzyl siódma pieczec, stalo sie milczenie na niebie, jakoby przez pól godziny.
\par 2 I widzialem siedm onych Aniolów, którzy stoja przed obliczem Bozem, a dano im siedm trab.
\par 3 A inszy Aniol przyszedl i stanal przed oltarzem, majac kadzielnice zlota; i dano mu wiele kadzenia, aby je ofiarowal z modlitwami wszystkich swietych na oltarzu zlotym, który jest przed stolica.
\par 4 I wstapil dym kadzenia z modlitwami swietych z reki Aniola przed oblicznosc Boza.
\par 5 I wzial Aniol kadzielnice i napelnil ja ogniem z oltarza, i zrzucil ja na ziemie, i staly sie glosy i gromy, i blyskawice, i trzesienie ziemi.
\par 6 A onych siedm Aniolów, którzy mieli siedm trab, nagotowalo sie, aby trabili.
\par 7 I zatrabil pierwszy Aniol i stal sie grad i ogien zmieszany ze krwia; i zrzucono to na ziemie, a trzecia czesc drzew zgorzala i wszelka trawa zielona spalona jest.
\par 8 Potem zatrabil wtóry Aniol, a jakoby góra wielka ogniem palajaca wrzucona jest w morze i obrócona jest w krew trzecia czesc morza.
\par 9 I pozdychala w morzu trzecia czesc rzeczy stworzonych, które mialy dusze i trzecia czesc okretów zginela.
\par 10 I zatrabil trzeci Aniol i spadla z nieba gwiazda wielka, gorejaca jako pochodnia, i upadla na trzecia czesc rzek i na zródla wód.
\par 11 A imie onej gwiazdy zowia piolunem; i obrócila sie trzecia czesc wód w piolun, a wiele ludzi pomarlo od onych wód, bo sie staly gorzkie.
\par 12 Potem zatrabil czwarty Aniol, a uderzona jest trzecia czesc slonca i trzecia czesc ksiezyca, i trzecia czesc gwiazd, tak iz sie trzecia czesc ich zacmila, i trzecia czesc dnia nie swiecila, takze i nocy.
\par 13 I widzialem, i slyszalem jednego Aniola lecacego przez posrodek nieba, mówiacego glosem wielkim: Biada, biada, biada mieszkajacym na ziemi dla innych glosów traby trzech Aniolów, którzy zatrabic maja!

\chapter{9}

\par 1 I zatrabil piaty Aniol i widzialem, ze gwiazda spadla z nieba na ziemie, a dano jej klucz studni przepasci.
\par 2 I otworzyla studnie przepasci; i wystapil dym z onej studni, jakoby dym pieca wielkiego, i zacmilo sie slonce i powietrze od dymu onej studni.
\par 3 A z onego dymu wyszly szarancze na ziemie i dano im moc, jako maja moc niedzwiadki ziemskie;
\par 4 A rzeczono im, zeby nie szkodzily trawie ziemi, ani zadnej rzeczy zielonej, ani zadnemu drzewu, ale tylko samym ludziom, którzy nie maja pieczeci Bozej na czolach swoich.
\par 5 A dano im, nie zeby ich zabijaly, ale aby ich dreczyly przez piec miesiecy, a udreczenie ich, aby bylo jako udreczenie od niedzwiadka, gdy czlowieka ukasi.
\par 6 Przetoz w one dni szukac beda ludzie smierci, ale jej nie znajda; i beda chcieli umrzec, ale smierc od nich uciecze.
\par 7 A ksztalt onych szaranczy podobny byl koniom zgotowanym do bitwy, a na glowach ich byly jakoby korony podobne zlotu, a twarze ich jako twarze ludzkie;
\par 8 I mialy wlosy jako wlosy niewiescie, a zeby ich byly jako lwie;
\par 9 A mialy pancerze jako pancerze zelazne, a szum skrzydel ich, jako grzmot wozów, gdy wiele koni biezy do bitwy.
\par 10 A ogony mialy podobne niedzwiadkom, a zadla byly w ogonach ich, a moc ich byla szkodzic ludziom przez piec miesiecy;
\par 11 A mialy nad soba króla, aniola przepasci, któremu imie po zydowsku Abaddon, a po grecku ma imie Apolijon.
\par 12 Biada jedno przeszlo, a oto jeszcze ida dwa biada potem.
\par 13 Tedy zatrabil Aniol szósty, a slyszalem glos jeden ze czterech rogów oltarza zlotego, który jest przed oblicznoscia Boza,
\par 14 Mówiacy szóstemu Aniolowi, który mial trabe: Rozwiaz onych czterech Aniolów zwiazanych u wielkiej rzeki Eufrates.
\par 15 I rozwiazani sa oni czterej Aniolowie, zgotowani na godzine i na dzien, i na miesiac, i na rok, aby pobili trzecia czesc ludzi.
\par 16 A liczba wojska jezdnego byla dwiesciekroc tysiac tysiecy; bom slyszal liczbe ich.
\par 17 Widzialem takze konie w widzeniu, a ci, którzy siedzieli na nich, mieli pancerze ogniste hijacyntowe i siarczane; a glowy onych koni byly jako glowy lwie, a z geby ich wychodzil ogien i dym i siarka.
\par 18 A od tego trojga pobita jest trzecia czesc ludzi od ognia i od dymu, i od siarki, które wychodzily z gab ich.
\par 19 Albowiem moc ich jest w gebach ich i w ogonach ich; bo ogony ich wezom sa podobne, majac glowy, któremi szkodza.
\par 20 A inni ludzie, którzy nie sa pobici temi plagami, ani pokutowali od uczynków rak swoich, aby sie nie klaniali dyjablom i balwanom zlotym i srebrnym, i miedzianym, i kamiennym i drewnianym, którzy ani widziec nie moga, ani slyszec, ani chodzic;
\par 21 Ani pokutowali z mezobójstw swoich, ani z czarów swoich, ani z wszeteczenstw swoich, ani z zlodziejstw swoich.

\chapter{10}

\par 1 I widzialem drugiego Aniola mocnego, zstepujacego z nieba, oblokiem odzianego, a na glowie jego byla tecza, a oblicze jego jako slonce, a nogi jego jako slupy ognia.
\par 2 A mial w rece swojej ksiazeczki otworzone i postawil noge swoje prawa na morzu, a lewa na ziemi.
\par 3 I zawolal glosem wielkim, jako lew ryczy; a gdy przestal wolac, mówilo siedm gromów glosy swoje.
\par 4 A gdy odmówilo siedm gromów glosy swoje, mialem pisac; alem uslyszal glos z nieba, mówiacy do mnie: Zapieczetuj to, co mówilo siedm gromów, a nie pisz tego.
\par 5 Tedy Aniol, któregom widzial stojacego na morzu i na ziemi, podniósl reke swoje ku niebu,
\par 6 I przysiagl przez Zyjacego na wieki wieków, który stworzyl niebo i to, co w niem jest, i ziemie, i to, co na niej jest, i morze, i to, co w niem jest, ze czasu juz nie bedzie.
\par 7 Ale we dni glosu Aniola siódmego, gdy bedzie trabil, dokona sie tajemnica Boza, jako opowiedzial slugom swoim prorokom.
\par 8 A glos, którym slyszal z nieba, zasie mówil ze mna i rzekl: Idz, a wezmij te ksiazeczki otworzone z reki Aniola stojacego na morzu i na ziemi.
\par 9 I szedlem do Aniola, i rzeklem mu: Daj mi te ksiazeczki. I rzekl mi: Wezmij, a zjedz je, a uczynia gorzkosc w brzuchu twoim; ale w ustach twoich slodkie beda jako miód.
\par 10 I wzialem ksiazeczki z reki Aniola i zjadlem je, a byly w ustach moich slodkie jako miód; ale gdym je zjadl, gorzko bylo w brzuchu moim.
\par 11 I rzekl mi: Musisz zasie prorokowac przed wiela ludzi i narodów, i jezyków, i królów.

\chapter{11}

\par 1 I dano mi trzcine podobna lasce; a Aniol stanal, mówiac: Wstan, a zmierz kosciól Bozy i oltarz, i tych, którzy sie modla w nim.
\par 2 Ale sien, która jest przed kosciolem, wyrzuc precz, a nie mierz jej; albowiem dana jest poganom, a miasto swiete deptac beda czterdziesci i dwa miesiace.
\par 3 I dam je dwom swiadkom moim, którzy prorokowac beda tysiac dwiescie i szescdziesiat dni, obleczeni bedac w wory.
\par 4 Ci sa dwie oliwy i dwa swieczniki, stojace przed obliczem Pana wszystkiej ziemi.
\par 5 A jezliby im kto chcial szkodzic, ogien wynijdzie z ust ich i pozre nieprzyjacioly ich; a jezliby im kto chcial szkodzic, ten tez tak musi byc zabity.
\par 6 Ci moc maja zamykac niebo, aby deszcz nie padal za dni proroctwa ich; i maja moc nad wodami, aby je obrócili w krew, i uderzyc ziemie wszelka plaga, ilekroc by chcieli.
\par 7 A gdy dokoncza swiadectwa swojego, bestyja, która wystepuje z przepasci, stoczy z nimi bitwe, a zwyciezy ich i pobije ich.
\par 8 A trupy ich lezec beda na ulicy miasta wielkiego, które nazywaja duchownie Sodoma i Egiptem, gdzie tez Pan nasz ukrzyzowany jest.
\par 9 I widziec beda wiele ich z ludzi, z pokolenia i z jezyków, i z narodów trupy ich przez pólczwarta dnia; ale trupów ich nie dopuszcza wlozyc w groby:
\par 10 Owszem mieszkajacy na ziemi radowac sie nad nimi beda i weselic; i posla dary jedni drugim, iz ci dwaj prorocy dreczyli mieszkajacych na ziemi.
\par 11 A po pólczwarta dnia duch zywota od Boga wstapil w nich i staneli na nogach swoich, a bojazn wielka przypadla na tych, którzy ich widzieli.
\par 12 Potem uslyszeli glos wielki z nieba, mówiacy im: Wstapcie sam! I wstapili na niebo w obloku, i patrzyli na nich nieprzyjaciele ich.
\par 13 A w onez godzine stalo sie wielkie trzesienie ziemi. I upadla dziesiata czesc miasta, i pobito w onem trzesieniu ziemi osób ludzi siedm tysiecy, a drudzy przestraszeni sa, i dali chwale Bogu niebieskiemu.
\par 14 Biada wtóra przeszla, a oto biada trzecia przyjdzie rychlo.
\par 15 I zatrabil Aniol siódmy i staly sie glosy wielkie na niebie mówiace: Królestwa swiata staly sie królestwami Pana naszego i Chrystusa jego, i królowac bedzie na wieki wieków.
\par 16 Tedy oni dwadziescia i cztery starcy, którzy przed oblicznoscia Boza siedza na stolicach swoich, upadli na oblicza swe i poklonili sie Bogu, mówiac:
\par 17 Dziekujemy tobie, Panie Boze wszechmogacy, którys jest i którys byl, i który masz przyjsc! zes wzial moc swoje wielka i ujales królestwo;
\par 18 I rozgniewaly sie narody, i przyszedl gniew twój i czas umarlych, aby byli sadzeni i abys oddal zaplate slugom twoim, prorokom i swietym, i bojacym sie imienia twego, malym i wielkim, i abys wytracil tych, co psuja ziemie.
\par 19 Tedy otworzony jest kosciól Bozy na niebie i widziana jest skrzynia przymierza jego w kosciele jego; i staly sie blyskawice i glosy, i grzmienia, i trzesienia ziemi i grad wielki.

\chapter{12}

\par 1 I ukazal sie cud wielki na niebie: Niewiasta obleczona w slonce, a ksiezyc pod nogami jej, a na glowie jej byla korona z dwunastu gwiazd;
\par 2 A bedac brzemienna, wolala pracujac ku porodzeniu i meczyla sie, aby porodzila.
\par 3 I ukazal sie drugi cud na niebie, a oto smok wielki rydzy, majac siedm glów i rogów dziesiec, a na glowach jego siedm koron;
\par 4 A ogon jego ciagnal trzecia czesc gwiazd niebieskich i zrzucil je na ziemie; a smok on stanal przed niewiasta, która miala porodzic, aby skoro by porodzila, pozarl dziecie jej.
\par 5 I urodzila syna, mezczyzne, który ma rzadzic wszystkie narody laska zelazna; i porwane jest dziecie jej do Boga i do stolicy jego,
\par 6 A niewiasta uciekla na pustynie, gdzie ma miejsce od Boga zgotowane, aby ja tam zywiono przez dni tysiac dwiescie i szescdziesiat.
\par 7 I stala sie bitwa na niebie. Michal i Aniolowie jego potykali sie z smokiem, smok sie tez potykal i aniolowie jego.
\par 8 Ale nie przemogli, ani miejsce ich dalej znalezione jest na niebie.
\par 9 I zrzucony jest smok wielki, waz on starodawny, którego zowia dyjablem i szatanem, który zwodzi wszystek okrag swiata; zrzucony jest na ziemie i aniolowie jego z nim sa zrzuceni.
\par 10 I slyszalem glos wielki mówiacy na niebie: Terazci sie stalo zbawienie i moc, i królestwo Boga naszego, i zwierzchnosc Chrystusa jego, iz zrzucony jest oskarzyciel braci naszych, który na nich skarzyl przed oblicznoscia Boga naszego we dnie i w nocy.
\par 11 Ale go oni zwyciezyli przez krew Baranka i przez slowa swiadectwa swego, a nie umilowali duszy swojej az do smierci.
\par 12 Przetoz rozweselcie sie nieba! i wy, którzy mieszkacie na nich. Biada mieszkajacym na ziemi i na morzu! iz zstapil dyjabel do was, majac wielki gniew, wiedzac, iz krótki czas ma.
\par 13 A gdy wiedzial smok, iz byl zrzucony na ziemie, przesladowal niewiaste, która byla porodzila mezczyzne.
\par 14 I dano niewiescie dwa skrzydla orla wielkiego, aby leciala od oblicznosci wezowej na pustynie, na miejsce swoje, gdzie by ja zywiono przez czas i czasy, i polowe czasu.
\par 15 I wypuscil waz z geby swojej za niewiasta wode jako rzeke, chcac sprawic, aby ja rzeka porwala.
\par 16 Ale ziemia ratowala niewiaste; i otworzyla ziemia usta swoje, i wypila rzeke, która byl wypuscil smok z geby swojej.
\par 17 I rozgniewal sie smok na niewiaste, i poszedl, aby walczyl z drugimi z nasienia jej, którzy zachowuja przykazania Boze i maja swiadectwo Jezusa Chrystusa.
\par 18 I stanalem na piasku morskim.

\chapter{13}

\par 1 I widzialem bestyje wystepujaca z morza, majaca siedm glów i rogów dziesiec; a na rogach jej bylo dziesiec koron, a na glowach jej imie bluznierstwa.
\par 2 A ta bestyja, któram widzial, podobna byla rysiowi, a nogi jej jako niedzwiedzie, a geba jej jako geba lwia; i dal jej smok moc swoje i stolice swoje, i moc wielka.
\par 3 A widzialem jedne z glów jej, jakoby na smierc zabita; ale rana jej smiertelna uleczona jest. Tedy sie dziwowala wszystka ziemia i szla za ona bestyja.
\par 4 I klaniali sie onemu smokowi, który dal moc bestyi; klaniali sie tez bestyi, mówiac: Któz podobny bestyi? Któz z nia walczyc moze?
\par 5 I dane jej sa usta, mówiace wielkie rzeczy i bluznierstwa; dana jej tez jest moc, aby wladze miala przez czterdziesci i dwa miesiace.
\par 6 I otworzyla usta swoje ku bluznierstwu przeciwko Bogu, aby bluznila imie jego i przybytek jego, i tych, którzy mieszkaja na niebie.
\par 7 Dano jej tez walczyc z swietymi i zwyciezac ich. I dana jej moc nad wszelkiem pokoleniem i jezykiem, i narodem.
\par 8 I beda sie jej klaniac wszyscy mieszkajacy na ziemi, których imiona nie sa napisane w ksiegach zywota Baranka zabitego od zalozenia swiata.
\par 9 Jezli kto ma uszy, niechaj slucha!
\par 10 Jezli kto w pojmanie wiedzie, w pojmanie pójdzie; jezli kto mieczem zabije, musi i on byc mieczem zabity. Tuc jest cierpliwosc i wiara swietych.
\par 11 Zatem widzialem druga bestyje wystepujaca z ziemi, a miala dwa rogi podobne Barankowym; ale mówila jako smok,
\par 12 A wszystkiej mocy pierwszej onej bestyi dokazuje przed twarza jej i czyni, ze ziemia i mieszkajacy na niej klaniaja sie bestyi pierwszej, której smiertelna rana byla uzdrowiona;
\par 13 A czyni cuda wielkie, tak iz i ogien z nieba zstepuje przed oczyma ludzi na ziemie;
\par 14 I zwodzi mieszkajacych na ziemi przez one cuda, które jej dano czynic przed bestyja, mówiac obywatelom ziemi, aby uczynili obraz onej bestyi, która miala rane od miecza, ale zasie ozyla.
\par 15 I dano jej, aby mogla dac ducha onemu obrazowi bestyi, zeby tez mówil obraz tej bestyi i to sprawil, aby ci, którzy by sie nie klaniali obrazowi onej bestyi, byli pobici.
\par 16 A czyni, aby wszyscy, mali i wielcy, bogaci i ubodzy, i wolni, i niewolnicy, wzieli pietna na prawa reke swoje albo na czola swe,
\par 17 A zeby zaden nie mógl kupowac ani sprzedawac, tylko ten, który ma pietno albo imie bestyi albo liczbe imienia jej.
\par 18 Tu jest madrosc. Kto ma rozum, niech zrachuje liczbe onej bestyi; albowiem jest liczba czlowieka. A ta jest liczba jej, szescset szescdziesiat i szesc.

\chapter{14}

\par 1 I widzialem, a oto Baranek stal na górze Syonskiej, a z nim sto czterdziesci i cztery tysiace, majacych imie Ojca jego napisane na czolach swoich.
\par 2 I slyszalem glos z nieba, jako glos wielu wód, i jako glos gromu wielkiego; i slyszalem glos cytrystów grajacych na cytrach swoich.
\par 3 A spiewali, jakoby nowa piesn, przed stolica i przed onem czworgiem zwierzat, i przed starcami, a zaden sie nie mógl onej piesni nauczyc, oprócz onych stu czterdziestu i czterech tysiecy, którzy sa z ziemi kupieni.
\par 4 Cic sa, którzy sie z niewiastami nie pokalali; bo pannami sa. Ci sa, którzy nasladuja Baranka, gdziekolwiek idzie. Ci kupieni sa z ludzi, aby byli pierwiastkami Bogu i Barankowi.
\par 5 A w ustach ich nie znalazla sie zdrada; albowiem sa bez zmazy przed stolica Boza.
\par 6 I widzialem drugiego Aniola, lecacego przez posrodek nieba, majacego Ewangielije wieczna, aby ja zwiastowal mieszkajacym na ziemi i wszelkiemu narodowi, i pokoleniu, i jezykowi, i ludowi,
\par 7 Mówiacego glosem wielkim: Bójcie sie Boga i chwale mu dajcie, gdyz przyszla godzina sadu jego, a klaniajcie sie temu, który uczynil niebo i ziemie, i morze, i zródla wód.
\par 8 A za nim szedl drugi Aniol, mówiac: Upadl Babilon, ono miasto wielkie! bo winem gniewu wszeteczenstwa swego napoil wszystkie narody.
\par 9 A trzeci Aniol szedl za nimi, mówiac glosem wielkim: Jezli sie kto pokloni bestyi i obrazowi jej, i jezli wezmie pietno na czolo swoje albo na reke swoje,
\par 10 I ten pic bedzie z wina gniewu Bozego, z wina szczerego i nalanego w kielich gniewu jego i bedzie meczony w ogniu i siarce przed oblicznoscia Aniolów swietych i przed oblicznoscia Baranka.
\par 11 A dym meki ich wystepuje na wieki wieków, i nie maja odpoczynku we dnie i w nocy, którzy sie klaniaja bestyi i obrazowi jej, i jezli kto bierze pietno imienia jej.
\par 12 Tuc jest cierpliwosc swietych, tuc sa ci, którzy chowaja przykazania Boze i wiare Jezusowa.
\par 13 I uslyszalem glos z nieba, mówiacy do mnie: Napisz: Blogoslawieni sa odtad umarli, którzy w Panu umieraja. Zaprawde mówi Duch im, aby odpoczywali od prac swoich, a uczynki ich ida za nimi.
\par 14 I widzialem, a oto oblok bialy; a na onym obloku siedzial podobny Synowi czlowieczemu, który mial na glowie swojej korone zlota, a w rece swojej sierp ostry.
\par 15 A drugi Aniol wyszedl z kosciola, wolajac glosem wielkim na tego, który siedzial na obloku: Zapusc sierp twój, a znij, gdyz tobie przyszla godzina, abys_zal, poniewaz sie dostalo zniwo ziemi.
\par 16 I zapuscil ten, który siedzial na obloku, sierp swój na ziemie i pozeta jest ziemia.
\par 17 A drugi Aniol wyszedl z kosciola onego, który jest w niebie, majac i ten sierp ostry.
\par 18 Potem wyszedl drugi Aniol z oltarza, który mial moc nad ogniem i zawolal glosem wielkim na tego, który mial sierp ostry, mówiac: Zapusc ten sierp twój ostry, a zbieraj grona winnicy ziemi; bo dojrzale sa jagody jej.
\par 19 Zapuscil tedy Aniol sierp swój ostry na ziemie i zebral grona winnicy ziemi, i wrzucil je w prase wielka gniewu Bozego.
\par 20 I tloczona jest prasa przed miastem, i wyszla krew z prasy az do wedzidel konskich przez tysiac i szescset stajan.

\chapter{15}

\par 1 Potemem widzial drugi cud na niebie wielki i dziwny, to jest siedm Aniolów majacych siedm plag ostatecznych, iz przez nie skonczony jest gniew Bozy.
\par 2 I widzialem, jakoby morze szklane zmieszane z ogniem; a tych, co zwyciestwo otrzymali nad ona bestyja i nad obrazem jej, i nad pietnem jej, i nad liczba imienia jej, stojacych nad morzem szklanem, majacych cytry Boze.
\par 3 A spiewali piesn Mojzesza, slugi Bozego, i piesn Barankowa, mówiac: Wielkie i dziwne sa sprawy twoje, Panie Boze wszechmogacy! sprawiedliwe i prawdziwe sa drogi twoje, o królu swietych!
\par 4 Któz by sie ciebie nie bal, Panie! i nie wielbil imienia twego? gdyzes sam swiety, gdyz wszystkie narody przyjda i klaniac sie beda przed obliczem twojem, ze sie okazaly sprawiedliwe sady twoje.
\par 5 A potemem widzial, a oto otworzony byl kosciól przybytku swiadectwa na niebie.
\par 6 I wyszlo z kosciola siedm onych Aniolów, majacych siedm plag, obleczonych plótnem czystem i swietnem, i przepasanych na piersiach zlotemi pasami.
\par 7 A jedno ze czworga zwierzat dalo siedmiu Aniolom siedm czasz zlotych, pelnych gniewu Boga zyjacego na wieki wieków.
\par 8 I napelniony jest kosciól dymem od chwaly Bozej i od mocy jego, a nie mógl nikt wnijsc do kosciola, az sie skonczylo siedm plag onych siedmiu Aniolów.

\chapter{16}

\par 1 I slyszalem glos wielki z kosciola, mówiacy siedmiu Aniolom: Idzcie, a wylejcie siedm czasz zapalczywosci Bozej na ziemie.
\par 2 I wyszedl pierwszy Aniol, a wylal czasze swoje na ziemie; i wyrzucil sie zly i szkodliwy wrzód na ludzi, którzy mieli pietno bestyi i na tych, którzy sie klaniali obrazowi jej.
\par 3 I wylal wtóry Aniol czasze swoje na morze, i stalo sie jakoby krew umarlego, a kazda dusza zywa zdechla w morzu.
\par 4 I wylal trzeci Aniol czasze swoje na rzeki i zródla wód, i obrócily sie w krew.
\par 5 I slyszalem Aniola wód mówiacego: Sprawiedliwys jest, Panie! którys jest i którys byl, i swiety, zes to rozsadzil.
\par 6 Poniewaz krew swietych i proroków wylewali, dales im tez krew pic; bo tego sa godni.
\par 7 I slyszalem drugiego od oltarza mówiacego: Zaiste, Panie, Boze wszechmogacy! prawdziwe i sprawiedliwe sa sady twoje.
\par 8 Potem czwarty Aniol wylal czasze swoje na slonce, i dano mu moc trapic ludzi goracoscia ognia.
\par 9 I byli upaleni ludzie goracoscia wielka, i bluznili imie Boga, który ma moc nad temi plagami; wszakze nie pokutowali, aby mu chwale dali.
\par 10 Tedy wylal piaty Aniol czasze swoje na stolice bestyi; i stalo sie królestwo jej zacmione, i zwali jezyki swoje od bolesci.
\par 11 I bluznili Boga niebieskiego dla bolesci swoich i dla wrzodów swoich; wszakze nie pokutowali z uczynków swoich.
\par 12 I wylal szósty Aniol czasze swoje na one wielka rzeke Eufrates i wyschla woda jej, aby zgotowana byla droga królom od wschodu slonca.
\par 13 I widzialem z ust smokowych i z ust bestyi, i z ust falszywego proroka trzy nieczyste duchy wychodzace, podobne zabom.
\par 14 Albowiem sa duchy dyjabelskie, czyniace cuda, które wychodza do królów ziemi i na wszystek okrag swiata, aby ich zgromadzili na wojne onego wielkiego dnia Boga wszechmogacego.
\par 15 Oto ide jako zlodziej: Blogoslawiony, który czuje i strzeze szat swoich, aby nie chodzil nago i nie widziano sromoty jego.
\par 16 I zgromadzil ich na miejsce, które zowia po zydowsku Armagieddon.
\par 17 Tedy wylal siódmy Aniol czasze swoje na powietrze; i wyszedl glos wielki z kosciola niebieskiego od stolicy, mówiacy: Stalo sie.
\par 18 I staly sie glosy i gromy, i blyskawice; i stalo sie wielkie trzesienie ziemi, jakiego nigdy nie bylo, jako sa ludzie na ziemi, trzesienia ziemi tak wielkiego.
\par 19 I stalo sie ono miasto wielkie na trzy czesci rozerwane, i miasta narodów upadly; i Babilon on wielki przyszedl na pamiec przed obliczem Bozem, aby mu dal kielich wina zapalczywosci gniewu swojego.
\par 20 I wszystkie wyspy uciekly, i góry nie sa znalezione.
\par 21 I wielki grad jako cetnarowy spadl z nieba na ludzi, i bluznili ludzie Boga dla plagi gradu, iz plaga jego byla bardzo wielka.

\chapter{17}

\par 1 I przyszedl jeden z siedmiu Aniolów, którzy mieli siedm czasz, i rzekl do mnie, mówiac mi: Chodz, okaze ci osadzenie onej wielkiej wszetecznicy, która siedzi nad wodami wielkiemi,
\par 2 Z która wszeteczenstwo plodzili królowie ziemi i upili sie winem wszeteczenstwa jej obywatele ziemi.
\par 3 I odniósl mie na puszcze w duchu. I widzialem niewiaste siedzaca na szarlatnoczerwonej bestyi, pelnej imion bluznierstwa, która miala siedm glów i dziesiec rogów.
\par 4 A ona niewiasta przyobleczona byla w purpure i w szarlat, i uzlocona zlotem i drogim kamieniem, i perlami, majac kubek zloty w rece swej, pelen obrzydliwosci i nieczystosci wszeteczenstwa swego.
\par 5 A na czole jej bylo imie napisane: Tajemnica, Babilon wielki, matka wszeteczenstw i obrzydliwosci ziemi.
\par 6 I widzialem niewiaste one pijana krwia swietych i krwia meczenników Jezusowych; a widzac ja, dziwowalem sie wielkim podziwieniem.
\par 7 I rzekl mi Aniol: Czemuz sie dziwujesz? Ja tobie powiem tajemnice tej niewiasty i bestyi, która ja nosi, która ma siedm glów i dziesiec rogów.
\par 8 Bestyja, któras widzial, byla, a nie jest, a ma wystapic z przepasci, a isc na zginienie; i zadziwia sie mieszkajacy na ziemi, (których imiona nie sa napisane w ksiegach zywota od zalozenia swiata), widzac bestyje, która byla, a nie jest, a przecie jest.
\par 9 Tuc jest rozum majacy madrosc: Te siedm glów sa siedm gór, na których ta niewiasta siedzi.
\par 10 A królów jest siedm, piec ich upadlo, a jeden jest, inszy jeszcze nie przyszedl, a gdy przyjdzie, na maly czas musi trwac.
\par 11 A bestyja, która byla a nie jest, toc jest ten ósmy, a jest z onych siedmiu, a idzie na zginienie.
\par 12 A dziesiec rogów, któres widzial, jest dziesiec królów, którzy królestwa jeszcze nie wzieli; ale wezma moc jako królowie, na jedne godzine z bestyja.
\par 13 Ci jedna rade maja i moc, i zwierzchnosc swoje bestyi podadza.
\par 14 Ci z Barankiem walczyc beda, i Baranek ich zwyciezy, bo jest Panem panów i królem królów, i którzy sa z nim powolani i wybrani, i wierni.
\par 15 I rzekl mi: Wody, któres widzial, gdzie wszetecznica siedzi, sa ludzie i zastepy, i narody, i jezyki.
\par 16 A dziesiec rogów, któres widzial na bestyi, cic w nienawisci miec beda wszetecznice i uczynia ja spustoszona i naga, i cialo jej beda jesc, a sama ogniem spala.
\par 17 Albowiem Bóg podal do serc ich, aby czynili wole jego, a czynili jednomyslnie, i dali królestwo swoje bestyi, azby sie wypelnily slowa Boze.
\par 18 A niewiasta, któras widzial, jest miasto ono wielkie, które ma królestwo nad królami ziemi.

\chapter{18}

\par 1 A potemem widzial drugiego Aniola zstepujacego z nieba, majacego moc wielka i oswiecila sie ziemia od chwaly jego.
\par 2 I zawolal poteznie glosem wielkim, mówiac: Upadl, upadl Babilon on wielki i stal sie przybytkiem czartów i mieszkaniem wszelkiego ducha nieczystego, i mieszkaniem wszelkiego ptastwa nieczystego i przemierzlego.
\par 3 Iz z win zapalczywosci wszeteczenstwa jego pily wszystkie narody, a królowie ziemi wszeteczenstwo z nim plodzili, i kupcy ziemscy z zbytecznej rozkoszy jego zbogacieli.
\par 4 I slyszalem inszy glos z nieba mówiacy: Wynijdzcie z niego, ludu mój! abyscie nie byli uczestnikami grzechów jego, a izbyscie nie wzieli z plag jego.
\par 5 Albowiem dosiegly grzechy jego az do nieba i wspomnial Bóg na nieprawosci jego.
\par 6 Oddajciez mu, jako i on oddawal wam, a w dwójnasób oddajcie mu wedlug uczynków jego; w kubku, który wam nalewal, nalejcie mu w dwójnasób.
\par 7 Jako sie wiele chlubil i rozkoszowal, tak mu wiele dajcie mak i smutku; bo mówi w sercu swojem: Siedze jako królowa, a nie jestem wdowa, i smutku nie ujrze.
\par 8 Przetoz w jeden dzien przyjda plagi jego, smierc i smutek, i glód, i ogniem bedzie spalony; bo mocny jest Pan Bóg, który go osadzi.
\par 9 I beda go plakac, i narzekac nad nim beda królowie ziemi, którzy z nim wszeteczenstwo plodzili i rozkoszowali, gdy ujrza dym zapalenia jego.
\par 10 Z daleka stojac dla bojazni meki jego i mówiac: Biada, biada, miasto ono wielkie Babilon, miasto ono mocne, iz w jedne godzine przyszedl sad twój!
\par 11 Do tego i kupcy ziemscy plakac beda i narzekac nad niem, przeto iz towaru ich zaden wiecej kupowac nie bedzie,
\par 12 Towaru zlota i srebra, i kamienia drogiego, i perel, i lnu cienkiego, i purpury, i jedwabiu, i szarlatu, i wszelkiego drzewa tyinowego, i wszelkiego naczynia sloniowego, i wszelkiego naczynia z drzewa najkosztowniejszego, i z miedzi, i z zelaza, i z marmuru,
\par 13 I cynamonu, i kadzenia, i masci, i kadzidla, i wina, i oliwy, i maki czystej, i pszenicy, i bydla, i owiec, i koni, i wozów, i niewolników, i dusz ludzkich.
\par 14 I owoce pozadliwosci duszy twojej odeszly od ciebie, i wszystkie rzeczy tluste i swietne odeszly od ciebie, a tych rzeczy juz wiecej nie znajdziesz.
\par 15 Kupcy tych rzeczy, zbogaciwszy sie tem, z daleka stac beda dla bojazni meki jego, placzac i narzekajac,
\par 16 A mówiac: Biada, biada, miasto ono wielkie, które bylo obleczone w bisior, i w purpure, i w szarlat, i uzlocone zlotem, i kamieniem drogim, i perlami; gdyz w jednej godzinie zginelo tak wielkie bogactwo.
\par 17 I wszelki sternik, i wszystko mnóstwo ludu, które jest na okrecie, i zeglarze, i którzykolwiek na morzu pozytku szukaja, z daleka staneli.
\par 18 I zawolali, widzac dym zapalenia jego, mówiac: Którez miasto bylo podobne temu miastu wielkiemu?
\par 19 A sypali proch na glowy swoje i wolali, placzac i smucac sie, i mówiac: Biada, biada, miasto ono wielkie, w którem zbogatnieli wszyscy, którzy mieli okrety na morzu z dostatków jego, iz jednej godziny spustoszalo!
\par 20 Rozraduj sie nad niem niebo i swieci Apostolowie i prorocy; bo sie pomscil krzywdy waszej Bóg nad niem.
\par 21 I podniósl jeden Aniol mocny kamien jakoby mlynski wielki, i wrzucil go w morze, mówiac: Takim pedem wrzucony bedzie Babilon, miasto ono wielkie, i juz wiecej nie bedzie znaleziony.
\par 22 I glos cytrystów, i spiewaków, i piszczków, i trebaczy wiecej w tobie slyszany nie bedzie, i zaden rzemieslnik wszelkiego rzemiosla nie znajdzie sie wiecej w tobie, i grzmot mlyna nie bedzie wiecej slyszany w tobie;
\par 23 I swiatlosc swiecy nie bedzie sie wiecej swiecila w tobie, i glos oblubienca i oblubienicy nie bedzie wiecej slyszany w tobie; iz kupcy twoi byli wielcy panowie ziemscy, iz czarami twojemi byli zwiedzeni wszystkie narody.
\par 24 I w niem znalazla sie krew proroków i swietych, i wszystkich, którzy sa pobici na ziemi.

\chapter{19}

\par 1 Potemem slyszal wielki glos wielkiego ludu na niebie, mówiacego: Halleluja! Zbawienie i chwala, i czesc, i moc Panu, Bogu naszemu.
\par 2 Bo prawdziwe i sprawiedliwe sa sady jego, iz osadzil wszetecznice one wielka, która kazila ziemie wszeteczenstwem swojem i pomscil sie krwi slug swoich z reki jej.
\par 3 I rzekli po wtóre: Halleluja! A dym jej wstepuje na wieki wieków.
\par 4 I upadli dwadziescia i czterej starcy, i czworo zwierzat, a poklonili sie Bogu siedzacemu na stolicy, mówiac: Amen, Halleluja!
\par 5 Tedy wyszedl glos z stolicy, mówiacy: Chwalcie Boga naszego wszyscy sludzy jego i którzy sie go boicie, i mali, i wielcy.
\par 6 I slyszalem glos jako ludu wielkiego i jako glos wielu wód, i jako glos mocnych gromów, mówiacych: Halleluja! iz ujal królestwo Pan Bóg wszechmogacy.
\par 7 Weselmy sie i radujmy sie, a dajmy mu chwale; bo przyszlo wesele Barankowe, a malzonka jego nagotowala sie.
\par 8 I dano jej, aby sie oblekla w bisior czysty i swietny, ;albowiem bisior sa usprawiedliwienia swietych.
\par 9 I rzekl mi: Napisz: Blogoslawieni, którzy sa wezwani na wieczerze wesela Barankowego. I rzekl mi: Te slowa Boze sa prawdziwe.
\par 10 I upadlem do nóg jego, abym sie mu poklonil; ale mi rzekl: Patrz, abys tego nie czynil; bom jest spólsluga twój i braci twoich, którzy maja swiadectwo Jezusowe. Bogu sie klaniaj; albowiem swiadectwo Jezusowe jest duch proroctwa.
\par 11 I widzialem niebo otworzone, a oto kon bialy, a tego, który siedzial na nim, zwano Wiernym i Prawdziwym, a sadzi w sprawiedliwosci i walczy.
\par 12 A oczy jego byly jako plomien ognia, a na glowie jego wiele koron; i mial imie napisane, którego nikt nie zna, tylko on sam.
\par 13 A przyodziany byl szata omoczona we krwi, a imie jego zowia Slowo Boze.
\par 14 A wojska, które sa na niebie, szly za nim na koniach bialych, obleczone lnem cienkim, bialym i czystym.
\par 15 A z ust jego wychodzil miecz ostry, aby nim bil narody; albowiem on je rzadzic bedzie laska zelazna, a on tloczy prase wina zapalczywosci i gniewu Boga wszechmogacego.
\par 16 A ma na szacie i na biodrach swoich imie napisane: Król królów i Pan panów.
\par 17 I widzialem jednego Aniola stojacego w sloncu i wolajacego glosem wielkim, mówiac wszystkim ptakom latajacym po posrodku nieba: Chodzcie i zgromadzcie sie na wieczerze wielkiego Boga,
\par 18 Abyscie jedli ciala królów i ciala hetmanów, i ciala mocarzy, i ciala koni, i siedzacych na nich, i ciala wszystkich wolnych i niewolników, i malych, i wielkich.
\par 19 I widzialem bestyje i królów ziemskich, i wojska ich zebrane, aby stoczyli bitwe z siedzacym na koniu i z wojskiem jego.
\par 20 Ale pojmana jest bestyja, a z nia falszywy on prorok, który czynil cuda przed nia, któremi zwodzil tych, którzy przyjeli pietno bestyi i którzy sie klaniali obrazowi jej, i obaj wrzuceni sa zywo do jeziora ognistego, gorejacego siarka.
\par 21 A drudzy pobici sa mieczem tego, który siedzial na koniu, wychodzacym z ust jego, a wszystkie ptaki nasycone sa cialami ich.

\chapter{20}

\par 1 I widzialem Aniola zstepujacego z nieba, majacego klucz od przepasci i lancuch wielki w rece swojej.
\par 2 I uchwycil smoka, weza onego starego, który jest dyjabel i szatan, i zwiazal go na tysiac lat;
\par 3 I wrzucil go w przepasc i zamknal go, i zapieczetowal z wierzchu nad nim, aby nie zwodzil wiecej narodów, azeby sie wypelnilo tysiac lat; a potem musi byc rozwiazany na maly czas.
\par 4 I widzialem stolice, a usiedli na nich i dany im jest sad i dusze poscinanych dla swiadectwa Jezusowego i dla slowa Bozego, i którzy sie nie klaniali bestyi ani obrazowi jej, i nie przyjeli pietna jej na czolo swoje i na reke swoje; i ozyli, i królowali z Chrystusem tysiac lat.
\par 5 A insi z umarlych nie ozyli, azby sie skonczylo tysiac lat. Toc jest pierwsze zmartwychwstanie.
\par 6 Blogoslawiony i swiety, który ma czesc w pierwszem zmartwychwstaniu; albowiem nad tymi wtóra smierc mocy nie ma; ale beda kaplanami Bozymi i Chrystusowymi, i beda z nim królowac tysiac lat.
\par 7 A gdy sie skonczy tysiac lat, bedzie rozwiazany szatan z ciemnicy swojej,
\par 8 I wynijdzie, aby zwodzil narody, które sa na czterech weglach ziemi, Goga, i Magoga, aby je zgromadzil do bitwy; których liczba jest jako piasek morski.
\par 9 I wstapili na szerokosc ziemi i otoczyli obóz swietych i miasto umilowane. Ale zstapil ogien od Boga z nieba i pozarl je.
\par 10 A dyjabel, który je zwodzil, wrzucony jest w jezioro ognia i siarki, gdzie jest ona bestyja i falszywy prorok; i beda meczeni we dnie i w nocy, na wieki wieków.
\par 11 I widzialem stolice wielka biala, i siedzacego na niej, przed którego obliczem uciekla ziemia i niebo, a miejsce im nie jest znalezione.
\par 12 I widzialem umarlych, wielkich i malych, stojacych przed oblicznoscia Boza, a ksiegi otworzone sa; i druga ksiega takze otworzona jest, to jest ksiega zywota; i sadzeni sa umarli wedlug tego, jako napisano bylo w onych ksiegach, to jest wedlug uczynków ich.
\par 13 I wydalo morze umarlych, którzy w niem byli, takze smierc i pieklo wydaly umarlych, którzy w nich byli; i byli sadzeni kazdy wedlug uczynków swoich.
\par 14 A smierc i pieklo wrzucone sa w jezioro ogniste. Tac jest wtóra smierc.
\par 15 A jesli sie kto nie znalazl napisany w ksiegach zywota, wrzucony jest w jezioro ogniste.

\chapter{21}

\par 1 Potemem widzial niebo nowe i ziemie nowa; albowiem pierwsze niebo i pierwsza ziemia przeminela, a morza juz wiecej nie bylo.
\par 2 A ja Jan widzialem ono swiete miasto, Jeruzalem nowe, zstepujace z nieba od Boga zgotowane, jako oblubienice ubrana mezowi swemu.
\par 3 I slyszalem glos wielki z nieba mówiacy: Oto przybytek Bozy z ludzmi, i bedzie mieszkal z nimi; a oni beda ludem jego, a sam Bóg bedzie z nimi, bedac Bogiem ich.
\par 4 I otrze Bóg wszelka lze z oczów ich; a smierci wiecej nie bedzie, ani smutku, ani krzyku, ani bolesci nie bedzie; albowiem pierwsze rzeczy pominely.
\par 5 I rzekl ten, który siedzial na stolicy: Oto wszystko nowe czynie. I rzekl mi: Napisz: bo te slowa sa wierne i prawdziwe.
\par 6 I rzekl mi: Stalo sie. Jam jest Alfa i Omega, poczatek i koniec. Ja pragnacemu dam darmo ze zródla wody zywej.
\par 7 Kto zwyciezy, odziedziczy wszystko i bede mu Bogiem, a on mi bedzie synem.
\par 8 Lecz bojazliwym i niewiernym, i obmierzlym, i mezobójcom, i wszetecznikom, i czarownikom, i balwochwalcom, i wszystkim klamcom czesc ich dana bedzie w jeziorze gorejacem ogniem i siarka: Tac jest smierc wtóra.
\par 9 Tedy przyszedl do mnie jeden z onych siedmiu Aniolów, którzy mieli siedm czasz napelnionych siedmioma plagami ostatecznemi, i mówil ze mna, i rzekl: Chodz sam, okaze ci oblubienice, malzonke Barankowa.
\par 10 I zaniósl mie w duchu na góre wielka i wysoka, i okazal mi miasto wielkie, ono swiete Jeruzalem, zstepujace z nieba od Boga,
\par 11 Majace chwale Boza, którego swiatlosc podobna byla kamieniowi najkosztowniejszemu, jako kamieniowi jaspisowi, na ksztalt krysztalu przezroczystemu;
\par 12 I majace mur wielki i wysoki, majace bram dwanascie, a na onych bramach dwanascie Aniolów i imiona napisane, które sa dwanascie pokolen synów Izraelskich.
\par 13 Od wschodu bramy trzy, od pólnocy bramy trzy, od poludnia bramy trzy, od zachodu bramy trzy.
\par 14 A mur miasta mial gruntów dwanascie, a na nich dwanascie imion dwunastu Apostolów Barankowych.
\par 15 A ten, co mówil ze mna, mial trzcine zlota, a zmierzyl miasto i bramy jego, i mur jego.
\par 16 A polozenie miasta onego jest czworograniaste, a dlugosc jego taka jest, jako i szerokosc. I pomierzyl miasto ono trzcina na dwanascie tysiecy stajan; a dlugosc i szerokosc i wysokosc jego równe sa.
\par 17 I zmierzyl mur jego na sto czterdziesci cztery lokcie miary czlowieczej, która jest miara Aniolowa.
\par 18 A bylo budowanie muru jego z jaspisu; a samo miasto bylo zloto czyste, podobne szklu czystemu.
\par 19 A grunty muru miasta ozdobione byly wszelkim kamieniem drogim. Pierwszy grunt byl jaspis, wtóry szafir, trzeci chalcedon, czwarty szmaragd.
\par 20 Piaty sardoniks, szósty sardyjusz, siódmy chrysolit, ósmy beryllus, dziewiaty topazyjusz, dziesiaty chrysopras, jedenasty hijacynt, dwunasty ametyst.
\par 21 A dwanascie bram jest dwanascie perel: a kazda brama byla z jednej perly, a rynek miasta zloto czyste jako szklo przezroczyste.
\par 22 Alem kosciola nie widzial w niem; albowiem Pan, Bóg wszechmogacy, jest kosciolem jego, i Baranek.
\par 23 A nie potrzebuje to miasto slonca ani ksiezyca, aby swiecily w niem; albowiem chwala Boza oswiecila je, a swieca jego jest Baranek.
\par 24 A narody, które beda zbawione, beda chodzily w swietle jego, a królowie ziemscy chwale i czesc swoje do niego przyniosa.
\par 25 A bramy jego nie beda zamkniete we dnie; albowiem tam nocy nie bedzie.
\par 26 I wniosa do niego chwale i czesc narodów.
\par 27 I nie wnijdzie do niego nic nieczystego i czyniacego obrzydliwosc i klamstwo, tylko ci, którzy sa napisani w ksiegach zywota Barankowych.

\chapter{22}

\par 1 I ukazal mi rzeke czysta wody zywota, jasna jako krysztal, wychodzaca z stolicy Bozej i Barankowej.
\par 2 A w posród rynku jego z obu stron rzeki bylo drzewo zywota, przynoszace owoc dwanascioraki, na kazdy miesiac wydawajace owoc swój, a liscie drzewa sluzyly ku uzdrowieniu pogan.
\par 3 I nie bedzie wiecej zadnego przeklestwa, ale stolica Boza i Barankowa w niem bedzie, a sludzy jego sluzyc mu beda,
\par 4 I patrzyc beda na oblicze jego, a imie jego na czolach ich bedzie.
\par 5 I nocy tam nie bedzie i nie beda potrzebowac swiecy i swiatlosci slonecznej; bo je Pan Bóg oswieca, i królowac beda na wieki wieków.
\par 6 I rzekl mi: Te slowa wierne sa i prawdziwe, a Pan, Bóg swietych proroków, poslal Aniola swego, aby ukazal slugom swoim, co sie ma stac w rychle.
\par 7 Oto przychodze rychlo: Blogoslawiony, który zachowuje slowa proroctwa ksiegi tej.
\par 8 A ja Jan widzialem i slyszalem te rzeczy. A gdym slyszal i widzial, upadlem, abym sie poklonil przed nogami Aniola onego, który mi to pokazywal.
\par 9 Ale mi on rzekl: Patrz, abys tego nie czynil; bom jest spólsluga twój i braci twoich proroków, i tych, co chowaja slowa ksiegi tej; Bogu sie klaniaj.
\par 10 Potem mi rzekl: Nie pieczetuj slów proroctwa ksiegi tej; albowiem czas blisko jest.
\par 11 Kto szkodzi, niech jeszcze szkodzi; a kto jest plugawy, niech jeszcze bedzie plugawszy; a kto jest sprawiedliwy, niech sie jeszcze usprawiedliwi; a kto swiety, niech jeszcze bedzie poswiecony.
\par 12 A oto przychodze rychlo, a zaplata moja jest ze mna, abym oddal kazdemu wedlug uczynków jego.
\par 13 Jam jest Alfa i Omega, poczatek i koniec, pierwszy i ostateczny.
\par 14 Blogoslawieni, którzy czynia przykazania jego, aby mieli prawo do drzewa zywota, i aby weszli bramami do miasta.
\par 15 A na dworze beda psy i czarownicy, i wszetecznicy, i mezobójcy, i balwochwalcy, i kazdy, który miluje i czyni klamstwo.
\par 16 Ja Jezus poslalem Aniola mojego, aby wam swiadczyl o tych rzeczach we zborach. Jam jest korzen i rodzaj on Dawidowy, gwiazda jasna i poranna.
\par 17 A Duch i oblubienica mówia: Przyjdz! A kto slyszy, niech rzecze: Przyjdz! A kto pragnie, niech przyjdzie; a kto chce, niech bierze wode zywota darmo.
\par 18 A oswiadczam sie kazdemu sluchajacemu slów proroctwa ksiegi tej: Jezliby kto przydal do tego, przyda mu tez Bóg plag opisanych w tej ksiedze;
\par 19 A jezliby kto ujal ze slów ksiegi proroctwa tego, odejmie tez Bóg czesc jego z ksiegi zywota i z miasta swietego, i z tych rzeczy, które sa napisane w tej ksiedze.
\par 20 Tak mówi ten, który swiadectwo daje o tych rzeczach: Zaiste, przyjde rychlo. Amen. I owszem przyjdz, Panie Jezusie!
\par 21 Laska Pana naszego, Jezusa Chrystusa, niech bedzie z wami wszystkimi. Amen.


\end{document}