\begin{document}

\title{Génesis}


\chapter{1}

\par 1 No princípio, criou Deus os céus e a terra.
\par 2 A terra, porém, estava sem forma e vazia; havia trevas sobre a face do abismo, e o Espírito de Deus pairava por sobre as águas.
\par 3 Disse Deus: Haja luz; e houve luz.
\par 4 E viu Deus que a luz era boa; e fez separação entre a luz e as trevas.
\par 5 Chamou Deus à luz Dia e às trevas, Noite. Houve tarde e manhã, o primeiro dia.
\par 6 E disse Deus: Haja firmamento no meio das águas e separação entre águas e águas.
\par 7 Fez, pois, Deus o firmamento e separação entre as águas debaixo do firmamento e as águas sobre o firmamento. E assim se fez.
\par 8 E chamou Deus ao firmamento Céus. Houve tarde e manhã, o segundo dia.
\par 9 Disse também Deus: Ajuntem-se as águas debaixo dos céus num só lugar, e apareça a porção seca. E assim se fez.
\par 10 À porção seca chamou Deus Terra e ao ajuntamento das águas, Mares. E viu Deus que isso era bom.
\par 11 E disse: Produza a terra relva, ervas que dêem semente e árvores frutíferas que dêem fruto segundo a sua espécie, cuja semente esteja nele, sobre a terra. E assim se fez.
\par 12 A terra, pois, produziu relva, ervas que davam semente segundo a sua espécie e árvores que davam fruto, cuja semente estava nele, conforme a sua espécie. E viu Deus que isso era bom.
\par 13 Houve tarde e manhã, o terceiro dia.
\par 14 Disse também Deus: Haja luzeiros no firmamento dos céus, para fazerem separação entre o dia e a noite; e sejam eles para sinais, para estações, para dias e anos.
\par 15 E sejam para luzeiros no firmamento dos céus, para alumiar a terra. E assim se fez.
\par 16 Fez Deus os dois grandes luzeiros: o maior para governar o dia, e o menor para governar a noite; e fez também as estrelas.
\par 17 E os colocou no firmamento dos céus para alumiarem a terra,
\par 18 para governarem o dia e a noite e fazerem separação entre a luz e as trevas. E viu Deus que isso era bom.
\par 19 Houve tarde e manhã, o quarto dia.
\par 20 Disse também Deus: Povoem-se as águas de enxames de seres viventes; e voem as aves sobre a terra, sob o firmamento dos céus.
\par 21 Criou, pois, Deus os grandes animais marinhos e todos os seres viventes que rastejam, os quais povoavam as águas, segundo as suas espécies; e todas as aves, segundo as suas espécies. E viu Deus que isso era bom.
\par 22 E Deus os abençoou, dizendo: Sede fecundos, multiplicai-vos e enchei as águas dos mares; e, na terra, se multipliquem as aves.
\par 23 Houve tarde e manhã, o quinto dia.
\par 24 Disse também Deus: Produza a terra seres viventes, conforme a sua espécie: animais domésticos, répteis e animais selváticos, segundo a sua espécie. E assim se fez.
\par 25 E fez Deus os animais selváticos, segundo a sua espécie, e os animais domésticos, conforme a sua espécie, e todos os répteis da terra, conforme a sua espécie. E viu Deus que isso era bom.
\par 26 Também disse Deus: Façamos o homem à nossa imagem, conforme a nossa semelhança; tenha ele domínio sobre os peixes do mar, sobre as aves dos céus, sobre os animais domésticos, sobre toda a terra e sobre todos os répteis que rastejam pela terra.
\par 27 Criou Deus, pois, o homem à sua imagem, à imagem de Deus o criou; homem e mulher os criou.
\par 28 E Deus os abençoou e lhes disse: Sede fecundos, multiplicai-vos, enchei a terra e sujeitai-a; dominai sobre os peixes do mar, sobre as aves dos céus e sobre todo animal que rasteja pela terra.
\par 29 E disse Deus ainda: Eis que vos tenho dado todas as ervas que dão semente e se acham na superfície de toda a terra e todas as árvores em que há fruto que dê semente; isso vos será para mantimento.
\par 30 E a todos os animais da terra, e a todas as aves dos céus, e a todos os répteis da terra, em que há fôlego de vida, toda erva verde lhes será para mantimento. E assim se fez.
\par 31 Viu Deus tudo quanto fizera, e eis que era muito bom. Houve tarde e manhã, o sexto dia.

\chapter{2}

\par 1 Assim, pois, foram acabados os céus e a terra e todo o seu exército.
\par 2 E, havendo Deus terminado no dia sétimo a sua obra, que fizera, descansou nesse dia de toda a sua obra que tinha feito.
\par 3 E abençoou Deus o dia sétimo e o santificou; porque nele descansou de toda a obra que, como Criador, fizera.
\par 4 Esta é a gênese dos céus e da terra quando foram criados, quando o SENHOR Deus os criou.
\par 5 Não havia ainda nenhuma planta do campo na terra, pois ainda nenhuma erva do campo havia brotado; porque o SENHOR Deus não fizera chover sobre a terra, e também não havia homem para lavrar o solo.
\par 6 Mas uma neblina subia da terra e regava toda a superfície do solo.
\par 7 Então, formou o SENHOR Deus ao homem do pó da terra e lhe soprou nas narinas o fôlego de vida, e o homem passou a ser alma vivente.
\par 8 E plantou o SENHOR Deus um jardim no Éden, na direção do Oriente, e pôs nele o homem que havia formado.
\par 9 Do solo fez o SENHOR Deus brotar toda sorte de árvores agradáveis à vista e boas para alimento; e também a árvore da vida no meio do jardim e a árvore do conhecimento do bem e do mal.
\par 10 E saía um rio do Éden para regar o jardim e dali se dividia, repartindo-se em quatro braços.
\par 11 O primeiro chama-se Pisom; é o que rodeia a terra de Havilá, onde há ouro.
\par 12 O ouro dessa terra é bom; também se encontram lá o bdélio e a pedra de ônix.
\par 13 O segundo rio chama-se Giom; é o que circunda a terra de Cuxe.
\par 14 O nome do terceiro rio é Tigre; é o que corre pelo oriente da Assíria. E o quarto é o Eufrates.
\par 15 Tomou, pois, o SENHOR Deus ao homem e o colocou no jardim do Éden para o cultivar e o guardar.
\par 16 E o SENHOR Deus lhe deu esta ordem: De toda árvore do jardim comerás livremente,
\par 17 mas da árvore do conhecimento do bem e do mal não comerás; porque, no dia em que dela comeres, certamente morrerás.
\par 18 Disse mais o SENHOR Deus: Não é bom que o homem esteja só; far-lhe-ei uma auxiliadora que lhe seja idônea.
\par 19 Havendo, pois, o SENHOR Deus formado da terra todos os animais do campo e todas as aves dos céus, trouxe-os ao homem, para ver como este lhes chamaria; e o nome que o homem desse a todos os seres viventes, esse seria o nome deles.
\par 20 Deu nome o homem a todos os animais domésticos, às aves dos céus e a todos os animais selváticos; para o homem, todavia, não se achava uma auxiliadora que lhe fosse idônea.
\par 21 Então, o SENHOR Deus fez cair pesado sono sobre o homem, e este adormeceu; tomou uma das suas costelas e fechou o lugar com carne.
\par 22 E a costela que o SENHOR Deus tomara ao homem, transformou-a numa mulher e lha trouxe.
\par 23 E disse o homem: Esta, afinal, é osso dos meus ossos e carne da minha carne; chamar-se-á varoa, porquanto do varão foi tomada.
\par 24 Por isso, deixa o homem pai e mãe e se une à sua mulher, tornando-se os dois uma só carne.
\par 25 Ora, um e outro, o homem e sua mulher, estavam nus e não se envergonhavam.

\chapter{3}

\par 1 Mas a serpente, mais sagaz que todos os animais selváticos que o SENHOR Deus tinha feito, disse à mulher: É assim que Deus disse: Não comereis de toda árvore do jardim?
\par 2 Respondeu-lhe a mulher: Do fruto das árvores do jardim podemos comer,
\par 3 mas do fruto da árvore que está no meio do jardim, disse Deus: Dele não comereis, nem tocareis nele, para que não morrais.
\par 4 Então, a serpente disse à mulher: É certo que não morrereis.
\par 5 Porque Deus sabe que no dia em que dele comerdes se vos abrirão os olhos e, como Deus, sereis conhecedores do bem e do mal.
\par 6 Vendo a mulher que a árvore era boa para se comer, agradável aos olhos e árvore desejável para dar entendimento, tomou-lhe do fruto e comeu e deu também ao marido, e ele comeu.
\par 7 Abriram-se, então, os olhos de ambos; e, percebendo que estavam nus, coseram folhas de figueira e fizeram cintas para si.
\par 8 Quando ouviram a voz do SENHOR Deus, que andava no jardim pela viração do dia, esconderam-se da presença do SENHOR Deus, o homem e sua mulher, por entre as árvores do jardim.
\par 9 E chamou o SENHOR Deus ao homem e lhe perguntou: Onde estás?
\par 10 Ele respondeu: Ouvi a tua voz no jardim, e, porque estava nu, tive medo, e me escondi.
\par 11 Perguntou-lhe Deus: Quem te fez saber que estavas nu? Comeste da árvore de que te ordenei que não comesses?
\par 12 Então, disse o homem: A mulher que me deste por esposa, ela me deu da árvore, e eu comi.
\par 13 Disse o SENHOR Deus à mulher: Que é isso que fizeste? Respondeu a mulher: A serpente me enganou, e eu comi.
\par 14 Então, o SENHOR Deus disse à serpente: Visto que isso fizeste, maldita és entre todos os animais domésticos e o és entre todos os animais selváticos; rastejarás sobre o teu ventre e comerás pó todos os dias da tua vida.
\par 15 Porei inimizade entre ti e a mulher, entre a tua descendência e o seu descendente. Este te ferirá a cabeça, e tu lhe ferirás o calcanhar.
\par 16 E à mulher disse: Multiplicarei sobremodo os sofrimentos da tua gravidez; em meio de dores darás à luz filhos; o teu desejo será para o teu marido, e ele te governará.
\par 17 E a Adão disse: Visto que atendeste a voz de tua mulher e comeste da árvore que eu te ordenara não comesses, maldita é a terra por tua causa; em fadigas obterás dela o sustento durante os dias de tua vida.
\par 18 Ela produzirá também cardos e abrolhos, e tu comerás a erva do campo.
\par 19 No suor do rosto comerás o teu pão, até que tornes à terra, pois dela foste formado; porque tu és pó e ao pó tornarás.
\par 20 E deu o homem o nome de Eva a sua mulher, por ser a mãe de todos os seres humanos.
\par 21 Fez o SENHOR Deus vestimenta de peles para Adão e sua mulher e os vestiu.
\par 22 Então, disse o SENHOR Deus: Eis que o homem se tornou como um de nós, conhecedor do bem e do mal; assim, que não estenda a mão, e tome também da árvore da vida, e coma, e viva eternamente.
\par 23 O SENHOR Deus, por isso, o lançou fora do jardim do Éden, a fim de lavrar a terra de que fora tomado.
\par 24 E, expulso o homem, colocou querubins ao oriente do jardim do Éden e o refulgir de uma espada que se revolvia, para guardar o caminho da árvore da vida.

\chapter{4}

\par 1 Coabitou o homem com Eva, sua mulher. Esta concebeu e deu à luz a Caim; então, disse: Adquiri um varão com o auxílio do SENHOR.
\par 2 Depois, deu à luz a Abel, seu irmão. Abel foi pastor de ovelhas, e Caim, lavrador.
\par 3 Aconteceu que no fim de uns tempos trouxe Caim do fruto da terra uma oferta ao SENHOR.
\par 4 Abel, por sua vez, trouxe das primícias do seu rebanho e da gordura deste. Agradou-se o SENHOR de Abel e de sua oferta;
\par 5 ao passo que de Caim e de sua oferta não se agradou. Irou-se, pois, sobremaneira, Caim, e descaiu-lhe o semblante.
\par 6 Então, lhe disse o SENHOR: Por que andas irado, e por que descaiu o teu semblante?
\par 7 Se procederes bem, não é certo que serás aceito? Se, todavia, procederes mal, eis que o pecado jaz à porta; o seu desejo será contra ti, mas a ti cumpre dominá-lo.
\par 8 Disse Caim a Abel, seu irmão: Vamos ao campo. Estando eles no campo, sucedeu que se levantou Caim contra Abel, seu irmão, e o matou.
\par 9 Disse o SENHOR a Caim: Onde está Abel, teu irmão? Ele respondeu: Não sei; acaso, sou eu tutor de meu irmão?
\par 10 E disse Deus: Que fizeste? A voz do sangue de teu irmão clama da terra a mim.
\par 11 És agora, pois, maldito por sobre a terra, cuja boca se abriu para receber de tuas mãos o sangue de teu irmão.
\par 12 Quando lavrares o solo, não te dará ele a sua força; serás fugitivo e errante pela terra.
\par 13 Então, disse Caim ao SENHOR: É tamanho o meu castigo, que já não posso suportá-lo.
\par 14 Eis que hoje me lanças da face da terra, e da tua presença hei de esconder-me; serei fugitivo e errante pela terra; quem comigo se encontrar me matará.
\par 15 O SENHOR, porém, lhe disse: Assim, qualquer que matar a Caim será vingado sete vezes. E pôs o SENHOR um sinal em Caim para que o não ferisse de morte quem quer que o encontrasse.
\par 16 Retirou-se Caim da presença do SENHOR e habitou na terra de Node, ao oriente do Éden.
\par 17 E coabitou Caim com sua mulher; ela concebeu e deu à luz a Enoque. Caim edificou uma cidade e lhe chamou Enoque, o nome de seu filho.
\par 18 A Enoque nasceu-lhe Irade; Irade gerou a Meujael, Meujael, a Metusael, e Metusael, a Lameque.
\par 19 Lameque tomou para si duas esposas: o nome de uma era Ada, a outra se chamava Zilá.
\par 20 Ada deu à luz a Jabal; este foi o pai dos que habitam em tendas e possuem gado.
\par 21 O nome de seu irmão era Jubal; este foi o pai de todos os que tocam harpa e flauta.
\par 22 Zilá, por sua vez, deu à luz a Tubalcaim, artífice de todo instrumento cortante, de bronze e de ferro; a irmã de Tubalcaim foi Naamá.
\par 23 E disse Lameque às suas esposas: Ada e Zilá, ouvi-me; vós, mulheres de Lameque, escutai o que passo a dizer-vos: Matei um homem porque ele me feriu; e um rapaz porque me pisou.
\par 24 Sete vezes se tomará vingança de Caim, de Lameque, porém, setenta vezes sete.
\par 25 Tornou Adão a coabitar com sua mulher; e ela deu à luz um filho, a quem pôs o nome de Sete; porque, disse ela, Deus me concedeu outro descendente em lugar de Abel, que Caim matou.
\par 26 A Sete nasceu-lhe também um filho, ao qual pôs o nome de Enos; daí se começou a invocar o nome do SENHOR.

\chapter{5}

\par 1 Este é o livro da genealogia de Adão. No dia em que Deus criou o homem, à semelhança de Deus o fez;
\par 2 homem e mulher os criou, e os abençoou, e lhes chamou pelo nome de Adão, no dia em que foram criados.
\par 3 Viveu Adão cento e trinta anos, e gerou um filho à sua semelhança, conforme a sua imagem, e lhe chamou Sete.
\par 4 Depois que gerou a Sete, viveu Adão oitocentos anos; e teve filhos e filhas.
\par 5 Os dias todos da vida de Adão foram novecentos e trinta anos; e morreu.
\par 6 Sete viveu cento e cinco anos e gerou a Enos.
\par 7 Depois que gerou a Enos, viveu Sete oitocentos e sete anos; e teve filhos e filhas.
\par 8 Todos os dias de Sete foram novecentos e doze anos; e morreu.
\par 9 Enos viveu noventa anos e gerou a Cainã.
\par 10 Depois que gerou a Cainã, viveu Enos oitocentos e quinze anos; e teve filhos e filhas.
\par 11 Todos os dias de Enos foram novecentos e cinco anos; e morreu.
\par 12 Cainã viveu setenta anos e gerou a Maalalel.
\par 13 Depois que gerou a Maalalel, viveu Cainã oitocentos e quarenta anos; e teve filhos e filhas.
\par 14 Todos os dias de Cainã foram novecentos e dez anos; e morreu.
\par 15 Maalalel viveu sessenta e cinco anos e gerou a Jarede.
\par 16 Depois que gerou a Jarede, viveu Maalalel oitocentos e trinta anos; e teve filhos e filhas.
\par 17 Todos os dias de Maalalel foram oitocentos e noventa e cinco anos; e morreu.
\par 18 Jarede viveu cento e sessenta e dois anos e gerou a Enoque.
\par 19 Depois que gerou a Enoque, viveu Jarede oitocentos anos; e teve filhos e filhas.
\par 20 Todos os dias de Jarede foram novecentos e sessenta e dois anos; e morreu.
\par 21 Enoque viveu sessenta e cinco anos e gerou a Metusalém.
\par 22 Andou Enoque com Deus; e, depois que gerou a Metusalém, viveu trezentos anos; e teve filhos e filhas.
\par 23 Todos os dias de Enoque foram trezentos e sessenta e cinco anos.
\par 24 Andou Enoque com Deus e já não era, porque Deus o tomou para si.
\par 25 Metusalém viveu cento e oitenta e sete anos e gerou a Lameque.
\par 26 Depois que gerou a Lameque, viveu Metusalém setecentos e oitenta e dois anos; e teve filhos e filhas.
\par 27 Todos os dias de Metusalém foram novecentos e sessenta e nove anos; e morreu.
\par 28 Lameque viveu cento e oitenta e dois anos e gerou um filho;
\par 29 pôs-lhe o nome de Noé, dizendo: Este nos consolará dos nossos trabalhos e das fadigas de nossas mãos, nesta terra que o SENHOR amaldiçoou.
\par 30 Depois que gerou a Noé, viveu Lameque quinhentos e noventa e cinco anos; e teve filhos e filhas.
\par 31 Todos os dias de Lameque foram setecentos e setenta e sete anos; e morreu.
\par 32 Era Noé da idade de quinhentos anos e gerou a Sem, Cam e Jafé.

\chapter{6}

\par 1 Como se foram multiplicando os homens na terra, e lhes nasceram filhas,
\par 2 vendo os filhos de Deus que as filhas dos homens eram formosas, tomaram para si mulheres, as que, entre todas, mais lhes agradaram.
\par 3 Então, disse o SENHOR: O meu Espírito não agirá para sempre no homem, pois este é carnal; e os seus dias serão cento e vinte anos.
\par 4 Ora, naquele tempo havia gigantes na terra; e também depois, quando os filhos de Deus possuíram as filhas dos homens, as quais lhes deram filhos; estes foram valentes, varões de renome, na antiguidade.
\par 5 Viu o SENHOR que a maldade do homem se havia multiplicado na terra e que era continuamente mau todo desígnio do seu coração;
\par 6 então, se arrependeu o SENHOR de ter feito o homem na terra, e isso lhe pesou no coração.
\par 7 Disse o SENHOR: Farei desaparecer da face da terra o homem que criei, o homem e o animal, os répteis e as aves dos céus; porque me arrependo de os haver feito.
\par 8 Porém Noé achou graça diante do SENHOR.
\par 9 Eis a história de Noé. Noé era homem justo e íntegro entre os seus contemporâneos; Noé andava com Deus.
\par 10 Gerou três filhos: Sem, Cam e Jafé.
\par 11 A terra estava corrompida à vista de Deus e cheia de violência.
\par 12 Viu Deus a terra, e eis que estava corrompida; porque todo ser vivente havia corrompido o seu caminho na terra.
\par 13 Então, disse Deus a Noé: Resolvi dar cabo de toda carne, porque a terra está cheia da violência dos homens; eis que os farei perecer juntamente com a terra.
\par 14 Faze uma arca de tábuas de cipreste; nela farás compartimentos e a calafetarás com betume por dentro e por fora.
\par 15 Deste modo a farás: de trezentos côvados será o comprimento; de cinqüenta, a largura; e a altura, de trinta.
\par 16 Farás ao seu redor uma abertura de um côvado de altura; a porta da arca colocarás lateralmente; farás pavimentos na arca: um em baixo, um segundo e um terceiro.
\par 17 Porque estou para derramar águas em dilúvio sobre a terra para consumir toda carne em que há fôlego de vida debaixo dos céus; tudo o que há na terra perecerá.
\par 18 Contigo, porém, estabelecerei a minha aliança; entrarás na arca, tu e teus filhos, e tua mulher, e as mulheres de teus filhos.
\par 19 De tudo o que vive, de toda carne, dois de cada espécie, macho e fêmea, farás entrar na arca, para os conservares vivos contigo.
\par 20 Das aves segundo as suas espécies, do gado segundo as suas espécies, de todo réptil da terra segundo as suas espécies, dois de cada espécie virão a ti, para os conservares em vida.
\par 21 Leva contigo de tudo o que se come, ajunta-o contigo; ser-te-á para alimento, a ti e a eles.
\par 22 Assim fez Noé, consoante a tudo o que Deus lhe ordenara.

\chapter{7}

\par 1 Disse o SENHOR a Noé: Entra na arca, tu e toda a tua casa, porque reconheço que tens sido justo diante de mim no meio desta geração.
\par 2 De todo animal limpo levarás contigo sete pares: o macho e sua fêmea; mas dos animais imundos, um par: o macho e sua fêmea.
\par 3 Também das aves dos céus, sete pares: macho e fêmea; para se conservar a semente sobre a face da terra.
\par 4 Porque, daqui a sete dias, farei chover sobre a terra durante quarenta dias e quarenta noites; e da superfície da terra exterminarei todos os seres que fiz.
\par 5 E tudo fez Noé, segundo o SENHOR lhe ordenara.
\par 6 Tinha Noé seiscentos anos de idade, quando as águas do dilúvio inundaram a terra.
\par 7 Por causa das águas do dilúvio, entrou Noé na arca, ele com seus filhos, sua mulher e as mulheres de seus filhos.
\par 8 Dos animais limpos, e dos animais imundos, e das aves, e de todo réptil sobre a terra,
\par 9 entraram para Noé, na arca, de dois em dois, macho e fêmea, como Deus lhe ordenara.
\par 10 E aconteceu que, depois de sete dias, vieram sobre a terra as águas do dilúvio.
\par 11 No ano seiscentos da vida de Noé, aos dezessete dias do segundo mês, nesse dia romperam-se todas as fontes do grande abismo, e as comportas dos céus se abriram,
\par 12 e houve copiosa chuva sobre a terra durante quarenta dias e quarenta noites.
\par 13 Nesse mesmo dia entraram na arca Noé, seus filhos Sem, Cam e Jafé, sua mulher e as mulheres de seus filhos;
\par 14 eles, e todos os animais segundo as suas espécies, todo gado segundo as suas espécies, todos os répteis que rastejam sobre a terra segundo as suas espécies, todas as aves segundo as suas espécies, todos os pássaros e tudo o que tem asa.
\par 15 De toda carne, em que havia fôlego de vida, entraram de dois em dois para Noé na arca;
\par 16 eram macho e fêmea os que entraram de toda carne, como Deus lhe havia ordenado; e o SENHOR fechou a porta após ele.
\par 17 Durou o dilúvio quarenta dias sobre a terra; cresceram as águas e levantaram a arca de sobre a terra.
\par 18 Predominaram as águas e cresceram sobremodo na terra; a arca, porém, vogava sobre as águas.
\par 19 Prevaleceram as águas excessivamente sobre a terra e cobriram todos os altos montes que havia debaixo do céu.
\par 20 Quinze côvados acima deles prevaleceram as águas; e os montes foram cobertos.
\par 21 Pereceu toda carne que se movia sobre a terra, tanto de ave como de animais domésticos e animais selváticos, e de todos os enxames de criaturas que povoam a terra, e todo homem.
\par 22 Tudo o que tinha fôlego de vida em suas narinas, tudo o que havia em terra seca, morreu.
\par 23 Assim, foram exterminados todos os seres que havia sobre a face da terra; o homem e o animal, os répteis e as aves dos céus foram extintos da terra; ficou somente Noé e os que com ele estavam na arca.
\par 24 E as águas durante cento e cinqüenta dias predominaram sobre a terra.

\chapter{8}

\par 1 Lembrou-se Deus de Noé e de todos os animais selváticos e de todos os animais domésticos que com ele estavam na arca; Deus fez soprar um vento sobre a terra, e baixaram as águas.
\par 2 Fecharam-se as fontes do abismo e também as comportas dos céus, e a copiosa chuva dos céus se deteve.
\par 3 As águas iam-se escoando continuamente de sobre a terra e minguaram ao cabo de cento e cinqüenta dias.
\par 4 No dia dezessete do sétimo mês, a arca repousou sobre as montanhas de Ararate.
\par 5 E as águas foram minguando até ao décimo mês, em cujo primeiro dia apareceram os cimos dos montes.
\par 6 Ao cabo de quarenta dias, abriu Noé a janela que fizera na arca
\par 7 e soltou um corvo, o qual, tendo saído, ia e voltava, até que se secaram as águas de sobre a terra.
\par 8 Depois, soltou uma pomba para ver se as águas teriam já minguado da superfície da terra;
\par 9 mas a pomba, não achando onde pousar o pé, tornou a ele para a arca; porque as águas cobriam ainda a terra. Noé, estendendo a mão, tomou-a e a recolheu consigo na arca.
\par 10 Esperou ainda outros sete dias e de novo soltou a pomba fora da arca.
\par 11 À tarde, ela voltou a ele; trazia no bico uma folha nova de oliveira; assim entendeu Noé que as águas tinham minguado de sobre a terra.
\par 12 Então, esperou ainda mais sete dias e soltou a pomba; ela, porém, já não tornou a ele.
\par 13 Sucedeu que, no primeiro dia do primeiro mês, do ano seiscentos e um, as águas se secaram de sobre a terra. Então, Noé removeu a cobertura da arca e olhou, e eis que o solo estava enxuto.
\par 14 E, aos vinte e sete dias do segundo mês, a terra estava seca.
\par 15 Então, disse Deus a Noé:
\par 16 Sai da arca, e, contigo, tua mulher, e teus filhos, e as mulheres de teus filhos.
\par 17 Os animais que estão contigo, de toda carne, tanto aves como gado, e todo réptil que rasteja sobre a terra, faze sair a todos, para que povoem a terra, sejam fecundos e nela se multipliquem.
\par 18 Saiu, pois, Noé, com seus filhos, sua mulher e as mulheres de seus filhos.
\par 19 E também saíram da arca todos os animais, todos os répteis, todas as aves e tudo o que se move sobre a terra, segundo as suas famílias.
\par 20 Levantou Noé um altar ao SENHOR e, tomando de animais limpos e de aves limpas, ofereceu holocaustos sobre o altar.
\par 21 E o SENHOR aspirou o suave cheiro e disse consigo mesmo: Não tornarei a amaldiçoar a terra por causa do homem, porque é mau o desígnio íntimo do homem desde a sua mocidade; nem tornarei a ferir todo vivente, como fiz.
\par 22 Enquanto durar a terra, não deixará de haver sementeira e ceifa, frio e calor, verão e inverno, dia e noite.

\chapter{9}

\par 1 Abençoou Deus a Noé e a seus filhos e lhes disse: Sede fecundos, multiplicai-vos e enchei a terra.
\par 2 Pavor e medo de vós virão sobre todos os animais da terra e sobre todas as aves dos céus; tudo o que se move sobre a terra e todos os peixes do mar nas vossas mãos serão entregues.
\par 3 Tudo o que se move e vive ser-vos-á para alimento; como vos dei a erva verde, tudo vos dou agora.
\par 4 Carne, porém, com sua vida, isto é, com seu sangue, não comereis.
\par 5 Certamente, requererei o vosso sangue, o sangue da vossa vida; de todo animal o requererei, como também da mão do homem, sim, da mão do próximo de cada um requererei a vida do homem.
\par 6 Se alguém derramar o sangue do homem, pelo homem se derramará o seu; porque Deus fez o homem segundo a sua imagem.
\par 7 Mas sede fecundos e multiplicai-vos; povoai a terra e multiplicai-vos nela.
\par 8 Disse também Deus a Noé e a seus filhos:
\par 9 Eis que estabeleço a minha aliança convosco, e com a vossa descendência,
\par 10 e com todos os seres viventes que estão convosco: tanto as aves, os animais domésticos e os animais selváticos que saíram da arca como todos os animais da terra.
\par 11 Estabeleço a minha aliança convosco: não será mais destruída toda carne por águas de dilúvio, nem mais haverá dilúvio para destruir a terra.
\par 12 Disse Deus: Este é o sinal da minha aliança que faço entre mim e vós e entre todos os seres viventes que estão convosco, para perpétuas gerações:
\par 13 porei nas nuvens o meu arco; será por sinal da aliança entre mim e a terra.
\par 14 Sucederá que, quando eu trouxer nuvens sobre a terra, e nelas aparecer o arco,
\par 15 então, me lembrarei da minha aliança, firmada entre mim e vós e todos os seres viventes de toda carne; e as águas não mais se tornarão em dilúvio para destruir toda carne.
\par 16 O arco estará nas nuvens; vê-lo-ei e me lembrarei da aliança eterna entre Deus e todos os seres viventes de toda carne que há sobre a terra.
\par 17 Disse Deus a Noé: Este é o sinal da aliança estabelecida entre mim e toda carne sobre a terra.
\par 18 Os filhos de Noé, que saíram da arca, foram Sem, Cam e Jafé; Cam é o pai de Canaã.
\par 19 São eles os três filhos de Noé; e deles se povoou toda a terra.
\par 20 Sendo Noé lavrador, passou a plantar uma vinha.
\par 21 Bebendo do vinho, embriagou-se e se pôs nu dentro de sua tenda.
\par 22 Cam, pai de Canaã, vendo a nudez do pai, fê-lo saber, fora, a seus dois irmãos.
\par 23 Então, Sem e Jafé tomaram uma capa, puseram-na sobre os próprios ombros de ambos e, andando de costas, rostos desviados, cobriram a nudez do pai, sem que a vissem.
\par 24 Despertando Noé do seu vinho, soube o que lhe fizera o filho mais moço
\par 25 e disse: Maldito seja Canaã; seja servo dos servos a seus irmãos.
\par 26 E ajuntou: Bendito seja o SENHOR, Deus de Sem; e Canaã lhe seja servo.
\par 27 Engrandeça Deus a Jafé, e habite ele nas tendas de Sem; e Canaã lhe seja servo.
\par 28 Noé, passado o dilúvio, viveu ainda trezentos e cinqüenta anos.
\par 29 Todos os dias de Noé foram novecentos e cinqüenta anos; e morreu.

\chapter{10}

\par 1 São estas as gerações dos filhos de Noé, Sem, Cam e Jafé; e nasceram-lhes filhos depois do dilúvio.
\par 2 Os filhos de Jafé são: Gomer, Magogue, Madai, Javã, Tubal, Meseque e Tiras.
\par 3 Os filhos de Gomer são: Asquenaz, Rifate e Togarma.
\par 4 Os de Javã são: Elisá, Társis, Quitim e Dodanim.
\par 5 Estes repartiram entre si as ilhas das nações nas suas terras, cada qual segundo a sua língua, segundo as suas famílias, em suas nações.
\par 6 Os filhos de Cam: Cuxe, Mizraim, Pute e Canaã.
\par 7 Os filhos de Cuxe: Sebá, Havilá, Sabtá, Raamá e Sabtecá; e os filhos de Raamá: Sabá e Dedã.
\par 8 Cuxe gerou a Ninrode, o qual começou a ser poderoso na terra.
\par 9 Foi valente caçador diante do SENHOR; daí dizer-se: Como Ninrode, poderoso caçador diante do SENHOR.
\par 10 O princípio do seu reino foi Babel, Ereque, Acade e Calné, na terra de Sinar.
\par 11 Daquela terra saiu ele para a Assíria e edificou Nínive, Reobote-Ir e Calá.
\par 12 E, entre Nínive e Calá, a grande cidade de Resém.
\par 13 Mizraim gerou a Ludim, a Anamim, a Leabim, a Naftuim,
\par 14 a Patrusim, a Casluim (donde saíram os filisteus) e a Caftorim.
\par 15 Canaã gerou a Sidom, seu primogênito, e a Hete,
\par 16 e aos jebuseus, aos amorreus, aos girgaseus,
\par 17 aos heveus, aos arqueus, aos sineus,
\par 18 aos arvadeus, aos zemareus e aos hamateus; e depois se espalharam as famílias dos cananeus.
\par 19 E o limite dos cananeus foi desde Sidom, indo para Gerar, até Gaza, indo para Sodoma, Gomorra, Admá e Zeboim, até Lasa.
\par 20 São estes os filhos de Cam, segundo as suas famílias, segundo as suas línguas, em suas terras, em suas nações.
\par 21 A Sem, que foi pai de todos os filhos de Héber e irmão mais velho de Jafé, também lhe nasceram filhos.
\par 22 Os filhos de Sem são: Elão, Assur, Arfaxade, Lude e Arã.
\par 23 Os filhos de Arã: Uz, Hul, Geter e Más.
\par 24 Arfaxade gerou a Salá; Salá gerou a Héber.
\par 25 A Héber nasceram dois filhos: um teve por nome Pelegue, porquanto em seus dias se repartiu a terra; e o nome de seu irmão foi Joctã.
\par 26 Joctã gerou a Almodá, a Selefe, a Hazar-Mavé, a Jerá,
\par 27 a Hadorão, a Uzal, a Dicla,
\par 28 a Obal, a Abimael, a Sabá,
\par 29 a Ofir, a Havilá e a Jobabe; todos estes foram filhos de Joctã.
\par 30 E habitaram desde Messa, indo para Sefar, montanha do Oriente.
\par 31 São estes os filhos de Sem, segundo as suas famílias, segundo as suas línguas, em suas terras, em suas nações.
\par 32 São estas as famílias dos filhos de Noé, segundo as suas gerações, nas suas nações; e destes foram disseminadas as nações na terra, depois do dilúvio.

\chapter{11}

\par 1 Ora, em toda a terra havia apenas uma linguagem e uma só maneira de falar.
\par 2 Sucedeu que, partindo eles do Oriente, deram com uma planície na terra de Sinar; e habitaram ali.
\par 3 E disseram uns aos outros: Vinde, façamos tijolos e queimemo-los bem. Os tijolos serviram-lhes de pedra, e o betume, de argamassa.
\par 4 Disseram: Vinde, edifiquemos para nós uma cidade e uma torre cujo tope chegue até aos céus e tornemos célebre o nosso nome, para que não sejamos espalhados por toda a terra.
\par 5 Então, desceu o SENHOR para ver a cidade e a torre, que os filhos dos homens edificavam;
\par 6 e o SENHOR disse: Eis que o povo é um, e todos têm a mesma linguagem. Isto é apenas o começo; agora não haverá restrição para tudo que intentam fazer.
\par 7 Vinde, desçamos e confundamos ali a sua linguagem, para que um não entenda a linguagem de outro.
\par 8 Destarte, o SENHOR os dispersou dali pela superfície da terra; e cessaram de edificar a cidade.
\par 9 Chamou-se-lhe, por isso, o nome de Babel, porque ali confundiu o SENHOR a linguagem de toda a terra e dali o SENHOR os dispersou por toda a superfície dela.
\par 10 São estas as gerações de Sem. Ora, ele era da idade de cem anos quando gerou a Arfaxade, dois anos depois do dilúvio;
\par 11 e, depois que gerou a Arfaxade, viveu Sem quinhentos anos; e gerou filhos e filhas.
\par 12 Viveu Arfaxade trinta e cinco anos e gerou a Salá;
\par 13 e, depois que gerou a Salá, viveu Arfaxade quatrocentos e três anos; e gerou filhos e filhas.
\par 14 Viveu Salá trinta anos e gerou a Héber;
\par 15 e, depois que gerou a Héber, viveu Salá quatrocentos e três anos; e gerou filhos e filhas.
\par 16 Viveu Héber trinta e quatro anos e gerou a Pelegue;
\par 17 e, depois que gerou a Pelegue, viveu Héber quatrocentos e trinta anos; e gerou filhos e filhas.
\par 18 Viveu Pelegue trinta anos e gerou a Reú;
\par 19 e, depois que gerou a Reú, viveu Pelegue duzentos e nove anos; e gerou filhos e filhas.
\par 20 Viveu Reú trinta e dois anos e gerou a Serugue;
\par 21 e, depois que gerou a Serugue, viveu Reú duzentos e sete anos; e gerou filhos e filhas.
\par 22 Viveu Serugue trinta anos e gerou a Naor;
\par 23 e, depois que gerou a Naor, viveu Serugue duzentos anos; e gerou filhos e filhas.
\par 24 Viveu Naor vinte e nove anos e gerou a Tera;
\par 25 e, depois que gerou a Tera, viveu Naor cento e dezenove anos; e gerou filhos e filhas.
\par 26 Viveu Tera setenta anos e gerou a Abrão, a Naor e a Harã.
\par 27 São estas as gerações de Tera. Tera gerou a Abrão, a Naor e a Harã; e Harã gerou a Ló.
\par 28 Morreu Harã na terra de seu nascimento, em Ur dos caldeus, estando Tera, seu pai, ainda vivo.
\par 29 Abrão e Naor tomaram para si mulheres; a de Abrão chamava-se Sarai, a de Naor, Milca, filha de Harã, que foi pai de Milca e de Iscá.
\par 30 Sarai era estéril, não tinha filhos.
\par 31 Tomou Tera a Abrão, seu filho, e a Ló, filho de Harã, filho de seu filho, e a Sarai, sua nora, mulher de seu filho Abrão, e saiu com eles de Ur dos caldeus, para ir à terra de Canaã; foram até Harã, onde ficaram.
\par 32 E, havendo Tera vivido duzentos e cinco anos ao todo, morreu em Harã.

\chapter{12}

\par 1 Ora, disse o SENHOR a Abrão: Sai da tua terra, da tua parentela e da casa de teu pai e vai para a terra que te mostrarei;
\par 2 de ti farei uma grande nação, e te abençoarei, e te engrandecerei o nome. Sê tu uma bênção!
\par 3 Abençoarei os que te abençoarem e amaldiçoarei os que te amaldiçoarem; em ti serão benditas todas as famílias da terra.
\par 4 Partiu, pois, Abrão, como lho ordenara o SENHOR, e Ló foi com ele. Tinha Abrão setenta e cinco anos quando saiu de Harã.
\par 5 Levou Abrão consigo a Sarai, sua mulher, e a Ló, filho de seu irmão, e todos os bens que haviam adquirido, e as pessoas que lhes acresceram em Harã. Partiram para a terra de Canaã; e lá chegaram.
\par 6 Atravessou Abrão a terra até Siquém, até ao carvalho de Moré. Nesse tempo os cananeus habitavam essa terra.
\par 7 Apareceu o SENHOR a Abrão e lhe disse: Darei à tua descendência esta terra. Ali edificou Abrão um altar ao SENHOR, que lhe aparecera.
\par 8 Passando dali para o monte ao oriente de Betel, armou a sua tenda, ficando Betel ao ocidente e Ai ao oriente; ali edificou um altar ao SENHOR e invocou o nome do SENHOR.
\par 9 Depois, seguiu Abrão dali, indo sempre para o Neguebe.
\par 10 Havia fome naquela terra; desceu, pois, Abrão ao Egito, para aí ficar, porquanto era grande a fome na terra.
\par 11 Quando se aproximava do Egito, quase ao entrar, disse a Sarai, sua mulher: Ora, bem sei que és mulher de formosa aparência;
\par 12 os egípcios, quando te virem, vão dizer: É a mulher dele e me matarão, deixando-te com vida.
\par 13 Dize, pois, que és minha irmã, para que me considerem por amor de ti e, por tua causa, me conservem a vida.
\par 14 Tendo Abrão entrado no Egito, viram os egípcios que a mulher era sobremaneira formosa.
\par 15 Viram-na os príncipes de Faraó e gabaram-na junto dele; e a mulher foi levada para a casa de Faraó.
\par 16 Este, por causa dela, tratou bem a Abrão, o qual veio a ter ovelhas, bois, jumentos, escravos e escravas, jumentas e camelos.
\par 17 Porém o SENHOR puniu Faraó e a sua casa com grandes pragas, por causa de Sarai, mulher de Abrão.
\par 18 Chamou, pois, Faraó a Abrão e lhe disse: Que é isso que me fizeste? Por que não me disseste que era ela tua mulher?
\par 19 E me disseste ser tua irmã? Por isso, a tomei para ser minha mulher. Agora, pois, eis a tua mulher, toma-a e vai-te.
\par 20 E Faraó deu ordens aos seus homens a respeito dele; e acompanharam-no, a ele, a sua mulher e a tudo que possuía.

\chapter{13}

\par 1 Saiu, pois, Abrão do Egito para o Neguebe, ele e sua mulher e tudo o que tinha, e Ló com ele.
\par 2 Era Abrão muito rico; possuía gado, prata e ouro.
\par 3 Fez as suas jornadas do Neguebe até Betel, até ao lugar onde primeiro estivera a sua tenda, entre Betel e Ai,
\par 4 até ao lugar do altar, que outrora tinha feito; e aí Abrão invocou o nome do SENHOR.
\par 5 Ló, que ia com Abrão, também tinha rebanhos, gado e tendas.
\par 6 E a terra não podia sustentá-los, para que habitassem juntos, porque eram muitos os seus bens; de sorte que não podiam habitar um na companhia do outro.
\par 7 Houve contenda entre os pastores do gado de Abrão e os pastores do gado de Ló. Nesse tempo os cananeus e os ferezeus habitavam essa terra.
\par 8 Disse Abrão a Ló: Não haja contenda entre mim e ti e entre os meus pastores e os teus pastores, porque somos parentes chegados.
\par 9 Acaso, não está diante de ti toda a terra? Peço-te que te apartes de mim; se fores para a esquerda, irei para a direita; se fores para a direita, irei para a esquerda.
\par 10 Levantou Ló os olhos e viu toda a campina do Jordão, que era toda bem regada (antes de haver o SENHOR destruído Sodoma e Gomorra), como o jardim do SENHOR, como a terra do Egito, como quem vai para Zoar.
\par 11 Então, Ló escolheu para si toda a campina do Jordão e partiu para o Oriente; separaram-se um do outro.
\par 12 Habitou Abrão na terra de Canaã; e Ló, nas cidades da campina e ia armando as suas tendas até Sodoma.
\par 13 Ora, os homens de Sodoma eram maus e grandes pecadores contra o SENHOR.
\par 14 Disse o SENHOR a Abrão, depois que Ló se separou dele: Ergue os olhos e olha desde onde estás para o norte, para o sul, para o oriente e para o ocidente;
\par 15 porque toda essa terra que vês, eu ta darei, a ti e à tua descendência, para sempre.
\par 16 Farei a tua descendência como o pó da terra; de maneira que, se alguém puder contar o pó da terra, então se contará também a tua descendência.
\par 17 Levanta-te, percorre essa terra no seu comprimento e na sua largura; porque eu ta darei.
\par 18 E Abrão, mudando as suas tendas, foi habitar nos carvalhais de Manre, que estão junto a Hebrom; e levantou ali um altar ao SENHOR.

\chapter{14}

\par 1 Sucedeu naquele tempo que Anrafel, rei de Sinar, Arioque, rei de Elasar, Quedorlaomer, rei de Elão, e Tidal, rei de Goim,
\par 2 fizeram guerra contra Bera, rei de Sodoma, contra Birsa, rei de Gomorra, contra Sinabe, rei de Admá, contra Semeber, rei de Zeboim, e contra o rei de Bela (esta é Zoar).
\par 3 Todos estes se ajuntaram no vale de Sidim (que é o mar Salgado).
\par 4 Doze anos serviram a Quedorlaomer, porém no décimo terceiro se rebelaram.
\par 5 Ao décimo quarto ano, veio Quedorlaomer e os reis que estavam com ele e feriram aos refains em Asterote-Carnaim, e aos zuzins em Hã, e aos emins em Savé-Quiriataim,
\par 6 e aos horeus no seu monte Seir, até El-Parã, que está junto ao deserto.
\par 7 De volta passaram em En-Mispate (que é Cades) e feriram toda a terra dos amalequitas e dos amorreus, que habitavam em Hazazom-Tamar.
\par 8 Então, saíram os reis de Sodoma, de Gomorra, de Admá, de Zeboim e de Bela (esta é Zoar) e se ordenaram e levantaram batalha contra eles no vale de Sidim,
\par 9 contra Quedorlaomer, rei de Elão, contra Tidal, rei de Goim, contra Anrafel, rei de Sinar, contra Arioque, rei de Elasar: quatro reis contra cinco.
\par 10 Ora, o vale de Sidim estava cheio de poços de betume; os reis de Sodoma e de Gomorra fugiram; alguns caíram neles, e os restantes fugiram para um monte.
\par 11 Tomaram, pois, todos os bens de Sodoma e de Gomorra e todo o seu mantimento e se foram.
\par 12 Apossaram-se também de Ló, filho do irmão de Abrão, que morava em Sodoma, e dos seus bens e partiram.
\par 13 Porém veio um, que escapara, e o contou a Abrão, o hebreu; este habitava junto dos carvalhais de Manre, o amorreu, irmão de Escol e de Aner, os quais eram aliados de Abrão.
\par 14 Ouvindo Abrão que seu sobrinho estava preso, fez sair trezentos e dezoito homens dos mais capazes, nascidos em sua casa, e os perseguiu até Dã.
\par 15 E, repartidos contra eles de noite, ele e os seus homens, feriu-os e os perseguiu até Hobá, que fica à esquerda de Damasco.
\par 16 Trouxe de novo todos os bens, e também a Ló, seu sobrinho, os bens dele, e ainda as mulheres, e o povo.
\par 17 Após voltar Abrão de ferir a Quedorlaomer e aos reis que estavam com ele, saiu-lhe ao encontro o rei de Sodoma no vale de Savé, que é o vale do Rei.
\par 18 Melquisedeque, rei de Salém, trouxe pão e vinho; era sacerdote do Deus Altíssimo;
\par 19 abençoou ele a Abrão e disse: Bendito seja Abrão pelo Deus Altíssimo, que possui os céus e a terra;
\par 20 e bendito seja o Deus Altíssimo, que entregou os teus adversários nas tuas mãos. E de tudo lhe deu Abrão o dízimo.
\par 21 Então, disse o rei de Sodoma a Abrão: Dá-me as pessoas, e os bens ficarão contigo.
\par 22 Mas Abrão lhe respondeu: Levanto a mão ao SENHOR, o Deus Altíssimo, o que possui os céus e a terra,
\par 23 e juro que nada tomarei de tudo o que te pertence, nem um fio, nem uma correia de sandália, para que não digas: Eu enriqueci a Abrão;
\par 24 nada quero para mim, senão o que os rapazes comeram e a parte que toca aos homens Aner, Escol e Manre, que foram comigo; estes que tomem o seu quinhão.

\chapter{15}

\par 1 Depois destes acontecimentos, veio a palavra do SENHOR a Abrão, numa visão, e disse: Não temas, Abrão, eu sou o teu escudo, e teu galardão será sobremodo grande.
\par 2 Respondeu Abrão: SENHOR Deus, que me haverás de dar, se continuo sem filhos e o herdeiro da minha casa é o damasceno Eliézer?
\par 3 Disse mais Abrão: A mim não me concedeste descendência, e um servo nascido na minha casa será o meu herdeiro.
\par 4 A isto respondeu logo o SENHOR, dizendo: Não será esse o teu herdeiro; mas aquele que será gerado de ti será o teu herdeiro.
\par 5 Então, conduziu-o até fora e disse: Olha para os céus e conta as estrelas, se é que o podes. E lhe disse: Será assim a tua posteridade.
\par 6 Ele creu no SENHOR, e isso lhe foi imputado para justiça.
\par 7 Disse-lhe mais: Eu sou o SENHOR que te tirei de Ur dos caldeus, para dar-te por herança esta terra.
\par 8 Perguntou-lhe Abrão: SENHOR Deus, como saberei que hei de possuí-la?
\par 9 Respondeu-lhe: Toma-me uma novilha, uma cabra e um cordeiro, cada qual de três anos, uma rola e um pombinho.
\par 10 Ele, tomando todos estes animais, partiu-os pelo meio e lhes pôs em ordem as metades, umas defronte das outras; e não partiu as aves.
\par 11 Aves de rapina desciam sobre os cadáveres, porém Abrão as enxotava.
\par 12 Ao pôr-do-sol, caiu profundo sono sobre Abrão, e grande pavor e cerradas trevas o acometeram;
\par 13 então, lhe foi dito: Sabe, com certeza, que a tua posteridade será peregrina em terra alheia, e será reduzida à escravidão, e será afligida por quatrocentos anos.
\par 14 Mas também eu julgarei a gente a que têm de sujeitar-se; e depois sairão com grandes riquezas.
\par 15 E tu irás para os teus pais em paz; serás sepultado em ditosa velhice.
\par 16 Na quarta geração, tornarão para aqui; porque não se encheu ainda a medida da iniqüidade dos amorreus.
\par 17 E sucedeu que, posto o sol, houve densas trevas; e eis um fogareiro fumegante e uma tocha de fogo que passou entre aqueles pedaços.
\par 18 Naquele mesmo dia, fez o SENHOR aliança com Abrão, dizendo: À tua descendência dei esta terra, desde o rio do Egito até ao grande rio Eufrates:
\par 19 o queneu, o quenezeu, o cadmoneu,
\par 20 o heteu, o ferezeu, os refains,
\par 21 o amorreu, o cananeu, o girgaseu e o jebuseu.

\chapter{16}

\par 1 Ora, Sarai, mulher de Abrão, não lhe dava filhos; tendo, porém, uma serva egípcia, por nome Agar,
\par 2 disse Sarai a Abrão: Eis que o SENHOR me tem impedido de dar à luz filhos; toma, pois, a minha serva, e assim me edificarei com filhos por meio dela. E Abrão anuiu ao conselho de Sarai.
\par 3 Então, Sarai, mulher de Abrão, tomou a Agar, egípcia, sua serva, e deu-a por mulher a Abrão, seu marido, depois de ter ele habitado por dez anos na terra de Canaã.
\par 4 Ele a possuiu, e ela concebeu. Vendo ela que havia concebido, foi sua senhora por ela desprezada.
\par 5 Disse Sarai a Abrão: Seja sobre ti a afronta que se me faz a mim. Eu te dei a minha serva para a possuíres; ela, porém, vendo que concebeu, desprezou-me. Julgue o SENHOR entre mim e ti.
\par 6 Respondeu Abrão a Sarai: A tua serva está nas tuas mãos, procede segundo melhor te parecer. Sarai humilhou-a, e ela fugiu de sua presença.
\par 7 Tendo-a achado o Anjo do SENHOR junto a uma fonte de água no deserto, junto à fonte no caminho de Sur,
\par 8 disse-lhe: Agar, serva de Sarai, donde vens e para onde vais? Ela respondeu: Fujo da presença de Sarai, minha senhora.
\par 9 Então, lhe disse o Anjo do SENHOR: Volta para a tua senhora e humilha-te sob suas mãos.
\par 10 Disse-lhe mais o Anjo do SENHOR: Multiplicarei sobremodo a tua descendência, de maneira que, por numerosa, não será contada.
\par 11 Disse-lhe ainda o Anjo do SENHOR: Concebeste e darás à luz um filho, a quem chamarás Ismael, porque o SENHOR te acudiu na tua aflição.
\par 12 Ele será, entre os homens, como um jumento selvagem; a sua mão será contra todos, e a mão de todos, contra ele; e habitará fronteiro a todos os seus irmãos.
\par 13 Então, ela invocou o nome do SENHOR, que lhe falava: Tu és Deus que vê; pois disse ela: Não olhei eu neste lugar para aquele que me vê?
\par 14 Por isso, aquele poço se chama Beer-Laai-Roi; está entre Cades e Berede.
\par 15 Agar deu à luz um filho a Abrão; e Abrão, a seu filho que lhe dera Agar, chamou-lhe Ismael.
\par 16 Era Abrão de oitenta e seis anos, quando Agar lhe deu à luz Ismael.

\chapter{17}

\par 1 Quando atingiu Abrão a idade de noventa e nove anos, apareceu-lhe o SENHOR e disse-lhe: Eu sou o Deus Todo-Poderoso; anda na minha presença e sê perfeito.
\par 2 Farei uma aliança entre mim e ti e te multiplicarei extraordinariamente.
\par 3 Prostrou-se Abrão, rosto em terra, e Deus lhe falou:
\par 4 Quanto a mim, será contigo a minha aliança; serás pai de numerosas nações.
\par 5 Abrão já não será o teu nome, e sim Abraão; porque por pai de numerosas nações te constituí.
\par 6 Far-te-ei fecundo extraordinariamente, de ti farei nações, e reis procederão de ti.
\par 7 Estabelecerei a minha aliança entre mim e ti e a tua descendência no decurso das suas gerações, aliança perpétua, para ser o teu Deus e da tua descendência.
\par 8 Dar-te-ei e à tua descendência a terra das tuas peregrinações, toda a terra de Canaã, em possessão perpétua, e serei o seu Deus.
\par 9 Disse mais Deus a Abraão: Guardarás a minha aliança, tu e a tua descendência no decurso das suas gerações.
\par 10 Esta é a minha aliança, que guardareis entre mim e vós e a tua descendência: todo macho entre vós será circuncidado.
\par 11 Circuncidareis a carne do vosso prepúcio; será isso por sinal de aliança entre mim e vós.
\par 12 O que tem oito dias será circuncidado entre vós, todo macho nas vossas gerações, tanto o escravo nascido em casa como o comprado a qualquer estrangeiro, que não for da tua estirpe.
\par 13 Com efeito, será circuncidado o nascido em tua casa e o comprado por teu dinheiro; a minha aliança estará na vossa carne e será aliança perpétua.
\par 14 O incircunciso, que não for circuncidado na carne do prepúcio, essa vida será eliminada do seu povo; quebrou a minha aliança.
\par 15 Disse também Deus a Abraão: A Sarai, tua mulher, já não lhe chamarás Sarai, porém Sara.
\par 16 Abençoá-la-ei e dela te darei um filho; sim, eu a abençoarei, e ela se tornará nações; reis de povos procederão dela.
\par 17 Então, se prostrou Abraão, rosto em terra, e se riu, e disse consigo: A um homem de cem anos há de nascer um filho? Dará à luz Sara com seus noventa anos?
\par 18 Disse Abraão a Deus: Tomara que viva Ismael diante de ti.
\par 19 Deus lhe respondeu: De fato, Sara, tua mulher, te dará um filho, e lhe chamarás Isaque; estabelecerei com ele a minha aliança, aliança perpétua para a sua descendência.
\par 20 Quanto a Ismael, eu te ouvi: abençoá-lo-ei, fá-lo-ei fecundo e o multiplicarei extraordinariamente; gerará doze príncipes, e dele farei uma grande nação.
\par 21 A minha aliança, porém, estabelecê-la-ei com Isaque, o qual Sara te dará à luz, neste mesmo tempo, daqui a um ano.
\par 22 E, finda esta fala com Abraão, Deus se retirou dele, elevando-se.
\par 23 Tomou, pois, Abraão a seu filho Ismael, e a todos os escravos nascidos em sua casa, e a todos os comprados por seu dinheiro, todo macho dentre os de sua casa, e lhes circuncidou a carne do prepúcio de cada um, naquele mesmo dia, como Deus lhe ordenara.
\par 24 Tinha Abraão noventa e nove anos de idade, quando foi circuncidado na carne do seu prepúcio.
\par 25 Ismael, seu filho, era de treze anos, quando foi circuncidado na carne do seu prepúcio.
\par 26 Abraão e seu filho, Ismael, foram circuncidados no mesmo dia.
\par 27 E também foram circuncidados todos os homens de sua casa, tanto os escravos nascidos nela como os comprados por dinheiro ao estrangeiro.

\chapter{18}

\par 1 Apareceu o SENHOR a Abraão nos carvalhais de Manre, quando ele estava assentado à entrada da tenda, no maior calor do dia.
\par 2 Levantou ele os olhos, olhou, e eis três homens de pé em frente dele. Vendo-os, correu da porta da tenda ao seu encontro, prostrou-se em terra
\par 3 e disse: Senhor meu, se acho mercê em tua presença, rogo-te que não passes do teu servo;
\par 4 traga-se um pouco de água, lavai os pés e repousai debaixo desta árvore;
\par 5 trarei um bocado de pão; refazei as vossas forças, visto que chegastes até vosso servo; depois, seguireis avante. Responderam: Faze como disseste.
\par 6 Apressou-se, pois, Abraão para a tenda de Sara e lhe disse: Amassa depressa três medidas de flor de farinha e faze pão assado ao borralho.
\par 7 Abraão, por sua vez, correu ao gado, tomou um novilho, tenro e bom, e deu-o ao criado, que se apressou em prepará-lo.
\par 8 Tomou também coalhada e leite e o novilho que mandara preparar e pôs tudo diante deles; e permaneceu de pé junto a eles debaixo da árvore; e eles comeram.
\par 9 Então, lhe perguntaram: Sara, tua mulher, onde está? Ele respondeu: Está aí na tenda.
\par 10 Disse um deles: Certamente voltarei a ti, daqui a um ano; e Sara, tua mulher, dará à luz um filho. Sara o estava escutando, à porta da tenda, atrás dele.
\par 11 Abraão e Sara eram já velhos, avançados em idade; e a Sara já lhe havia cessado o costume das mulheres.
\par 12 Riu-se, pois, Sara no seu íntimo, dizendo consigo mesma: Depois de velha, e velho também o meu senhor, terei ainda prazer?
\par 13 Disse o SENHOR a Abraão: Por que se riu Sara, dizendo: Será verdade que darei ainda à luz, sendo velha?
\par 14 Acaso, para o SENHOR há coisa demasiadamente difícil? Daqui a um ano, neste mesmo tempo, voltarei a ti, e Sara terá um filho.
\par 15 Então, Sara, receosa, o negou, dizendo: Não me ri. Ele, porém, disse: Não é assim, é certo que riste.
\par 16 Tendo-se levantado dali aqueles homens, olharam para Sodoma; e Abraão ia com eles, para os encaminhar.
\par 17 Disse o SENHOR: Ocultarei a Abraão o que estou para fazer,
\par 18 visto que Abraão certamente virá a ser uma grande e poderosa nação, e nele serão benditas todas as nações da terra?
\par 19 Porque eu o escolhi para que ordene a seus filhos e a sua casa depois dele, a fim de que guardem o caminho do SENHOR e pratiquem a justiça e o juízo; para que o SENHOR faça vir sobre Abraão o que tem falado a seu respeito.
\par 20 Disse mais o SENHOR: Com efeito, o clamor de Sodoma e Gomorra tem-se multiplicado, e o seu pecado se tem agravado muito.
\par 21 Descerei e verei se, de fato, o que têm praticado corresponde a esse clamor que é vindo até mim; e, se assim não é, sabê-lo-ei.
\par 22 Então, partiram dali aqueles homens e foram para Sodoma; porém Abraão permaneceu ainda na presença do SENHOR.
\par 23 E, aproximando-se a ele, disse: Destruirás o justo com o ímpio?
\par 24 Se houver, porventura, cinqüenta justos na cidade, destruirás ainda assim e não pouparás o lugar por amor dos cinqüenta justos que nela se encontram?
\par 25 Longe de ti o fazeres tal coisa, matares o justo com o ímpio, como se o justo fosse igual ao ímpio; longe de ti. Não fará justiça o Juiz de toda a terra?
\par 26 Então, disse o SENHOR: Se eu achar em Sodoma cinqüenta justos dentro da cidade, pouparei a cidade toda por amor deles.
\par 27 Disse mais Abraão: Eis que me atrevo a falar ao Senhor, eu que sou pó e cinza.
\par 28 Na hipótese de faltarem cinco para cinqüenta justos, destruirás por isso toda a cidade? Ele respondeu: Não a destruirei se eu achar ali quarenta e cinco.
\par 29 Disse-lhe ainda mais Abraão: E se, porventura, houver ali quarenta? Respondeu: Não o farei por amor dos quarenta.
\par 30 Insistiu: Não se ire o Senhor, falarei ainda: Se houver, porventura, ali trinta? Respondeu o SENHOR: Não o farei se eu encontrar ali trinta.
\par 31 Continuou Abraão: Eis que me atrevi a falar ao Senhor: Se, porventura, houver ali vinte? Respondeu o SENHOR: Não a destruirei por amor dos vinte.
\par 32 Disse ainda Abraão: Não se ire o Senhor, se lhe falo somente mais esta vez: Se, porventura, houver ali dez? Respondeu o SENHOR: Não a destruirei por amor dos dez.
\par 33 Tendo cessado de falar a Abraão, retirou-se o SENHOR; e Abraão voltou para o seu lugar.

\chapter{19}

\par 1 Ao anoitecer, vieram os dois anjos a Sodoma, a cuja entrada estava Ló assentado; este, quando os viu, levantou-se e, indo ao seu encontro, prostrou-se, rosto em terra.
\par 2 E disse-lhes: Eis agora, meus senhores, vinde para a casa do vosso servo, pernoitai nela e lavai os pés; levantar-vos-eis de madrugada e seguireis o vosso caminho. Responderam eles: Não; passaremos a noite na praça.
\par 3 Instou-lhes muito, e foram e entraram em casa dele; deu-lhes um banquete, fez assar uns pães asmos, e eles comeram.
\par 4 Mas, antes que se deitassem, os homens daquela cidade cercaram a casa, os homens de Sodoma, tanto os moços como os velhos, sim, todo o povo de todos os lados;
\par 5 e chamaram por Ló e lhe disseram: Onde estão os homens que, à noitinha, entraram em tua casa? Traze-os fora a nós para que abusemos deles.
\par 6 Saiu-lhes, então, Ló à porta, fechou-a após si
\par 7 e lhes disse: Rogo-vos, meus irmãos, que não façais mal;
\par 8 tenho duas filhas, virgens, eu vo-las trarei; tratai-as como vos parecer, porém nada façais a estes homens, porquanto se acham sob a proteção de meu teto.
\par 9 Eles, porém, disseram: Retira-te daí. E acrescentaram: Só ele é estrangeiro, veio morar entre nós e pretende ser juiz em tudo? A ti, pois, faremos pior do que a eles. E arremessaram-se contra o homem, contra Ló, e se chegaram para arrombar a porta.
\par 10 Porém os homens, estendendo a mão, fizeram entrar Ló e fecharam a porta;
\par 11 e feriram de cegueira aos que estavam fora, desde o menor até ao maior, de modo que se cansaram à procura da porta.
\par 12 Então, disseram os homens a Ló: Tens aqui alguém mais dos teus? Genro, e teus filhos, e tuas filhas, todos quantos tens na cidade, faze-os sair deste lugar;
\par 13 pois vamos destruir este lugar, porque o seu clamor se tem aumentado, chegando até à presença do SENHOR; e o SENHOR nos enviou a destruí-lo.
\par 14 Então, saiu Ló e falou a seus genros, aos que estavam para casar com suas filhas e disse: Levantai-vos, saí deste lugar, porque o SENHOR há de destruir a cidade. Acharam, porém, que ele gracejava com eles.
\par 15 Ao amanhecer, apertaram os anjos com Ló, dizendo: Levanta-te, toma tua mulher e tuas duas filhas, que aqui se encontram, para que não pereças no castigo da cidade.
\par 16 Como, porém, se demorasse, pegaram-no os homens pela mão, a ele, a sua mulher e as duas filhas, sendo-lhe o SENHOR misericordioso, e o tiraram, e o puseram fora da cidade.
\par 17 Havendo-os levado fora, disse um deles: Livra-te, salva a tua vida; não olhes para trás, nem pares em toda a campina; foge para o monte, para que não pereças.
\par 18 Respondeu-lhes Ló: Assim não, Senhor meu!
\par 19 Eis que o teu servo achou mercê diante de ti, e engrandeceste a tua misericórdia que me mostraste, salvando-me a vida; não posso escapar no monte, pois receio que o mal me apanhe, e eu morra.
\par 20 Eis aí uma cidade perto para a qual eu posso fugir, e é pequena. Permite que eu fuja para lá (porventura, não é pequena?), e nela viverá a minha alma.
\par 21 Disse-lhe: Quanto a isso, estou de acordo, para não subverter a cidade de que acabas de falar.
\par 22 Apressa-te, refugia-te nela; pois nada posso fazer, enquanto não tiveres chegado lá. Por isso, se chamou Zoar o nome da cidade.
\par 23 Saía o sol sobre a terra, quando Ló entrou em Zoar.
\par 24 Então, fez o SENHOR chover enxofre e fogo, da parte do SENHOR, sobre Sodoma e Gomorra.
\par 25 E subverteu aquelas cidades, e toda a campina, e todos os moradores das cidades, e o que nascia na terra.
\par 26 E a mulher de Ló olhou para trás e converteu-se numa estátua de sal.
\par 27 Tendo-se levantado Abraão de madrugada, foi para o lugar onde estivera na presença do SENHOR;
\par 28 e olhou para Sodoma e Gomorra e para toda a terra da campina e viu que da terra subia fumaça, como a fumarada de uma fornalha.
\par 29 Ao tempo que destruía as cidades da campina, lembrou-se Deus de Abraão e tirou a Ló do meio das ruínas, quando subverteu as cidades em que Ló habitara.
\par 30 Subiu Ló de Zoar e habitou no monte, ele e suas duas filhas, porque receavam permanecer em Zoar; e habitou numa caverna, e com ele as duas filhas.
\par 31 Então, a primogênita disse à mais moça: Nosso pai está velho, e não há homem na terra que venha unir-se conosco, segundo o costume de toda terra.
\par 32 Vem, façamo-lo beber vinho, deitemo-nos com ele e conservemos a descendência de nosso pai.
\par 33 Naquela noite, pois, deram a beber vinho a seu pai, e, entrando a primogênita, se deitou com ele, sem que ele o notasse, nem quando ela se deitou, nem quando se levantou.
\par 34 No dia seguinte, disse a primogênita à mais nova: Deitei-me, ontem, à noite, com o meu pai. Demos-lhe a beber vinho também esta noite; entra e deita-te com ele, para que preservemos a descendência de nosso pai.
\par 35 De novo, pois, deram, aquela noite, a beber vinho a seu pai, e, entrando a mais nova, se deitou com ele, sem que ele o notasse, nem quando ela se deitou, nem quando se levantou.
\par 36 E assim as duas filhas de Ló conceberam do próprio pai.
\par 37 A primogênita deu à luz um filho e lhe chamou Moabe: é o pai dos moabitas, até ao dia de hoje.
\par 38 A mais nova também deu à luz um filho e lhe chamou Ben-Ami: é o pai dos filhos de Amom, até ao dia de hoje.

\chapter{20}

\par 1 Partindo Abraão dali para a terra do Neguebe, habitou entre Cades e Sur e morou em Gerar.
\par 2 Disse Abraão de Sara, sua mulher: Ela é minha irmã; assim, pois, Abimeleque, rei de Gerar, mandou buscá-la.
\par 3 Deus, porém, veio a Abimeleque em sonhos de noite e lhe disse: Vais ser punido de morte por causa da mulher que tomaste, porque ela tem marido.
\par 4 Ora, Abimeleque ainda não a havia possuído; por isso, disse: Senhor, matarás até uma nação inocente?
\par 5 Não foi ele mesmo que me disse: É minha irmã? E ela também me disse: Ele é meu irmão. Com sinceridade de coração e na minha inocência, foi que eu fiz isso.
\par 6 Respondeu-lhe Deus em sonho: Bem sei que com sinceridade de coração fizeste isso; daí o ter impedido eu de pecares contra mim e não te permiti que a tocasses.
\par 7 Agora, pois, restitui a mulher a seu marido, pois ele é profeta e intercederá por ti, e viverás; se, porém, não lha restituíres, sabe que certamente morrerás, tu e tudo o que é teu.
\par 8 Levantou-se Abimeleque de madrugada, e chamou todos os seus servos, e lhes contou todas essas coisas; e os homens ficaram muito atemorizados.
\par 9 Então, chamou Abimeleque a Abraão e lhe disse: Que é isso que nos fizeste? Em que pequei eu contra ti, para trazeres tamanho pecado sobre mim e sobre o meu reino? Tu me fizeste o que não se deve fazer.
\par 10 Disse mais Abimeleque a Abraão: Que estavas pensando para fazeres tal coisa?
\par 11 Respondeu Abraão: Eu dizia comigo mesmo: Certamente não há temor de Deus neste lugar, e eles me matarão por causa de minha mulher.
\par 12 Por outro lado, ela, de fato, é também minha irmã, filha de meu pai e não de minha mãe; e veio a ser minha mulher.
\par 13 Quando Deus me fez andar errante da casa de meu pai, eu disse a ela: Este favor me farás: em todo lugar em que entrarmos, dirás a meu respeito: Ele é meu irmão.
\par 14 Então, Abimeleque tomou ovelhas e bois, e servos e servas e os deu a Abraão; e lhe restituiu a Sara, sua mulher.
\par 15 Disse Abimeleque: A minha terra está diante de ti; habita onde melhor te parecer.
\par 16 E a Sara disse: Dei mil siclos de prata a teu irmão; será isto compensação por tudo quanto se deu contigo; e perante todos estás justificada.
\par 17 E, orando Abraão, sarou Deus Abimeleque, sua mulher e suas servas, de sorte que elas pudessem ter filhos;
\par 18 porque o SENHOR havia tornado estéreis todas as mulheres da casa de Abimeleque, por causa de Sara, mulher de Abraão.

\chapter{21}

\par 1 Visitou o SENHOR a Sara, como lhe dissera, e o SENHOR cumpriu o que lhe havia prometido.
\par 2 Sara concebeu e deu à luz um filho a Abraão na sua velhice, no tempo determinado, de que Deus lhe falara.
\par 3 Ao filho que lhe nasceu, que Sara lhe dera à luz, pôs Abraão o nome de Isaque.
\par 4 Abraão circuncidou a seu filho Isaque, quando este era de oito dias, segundo Deus lhe havia ordenado.
\par 5 Tinha Abraão cem anos, quando lhe nasceu Isaque, seu filho.
\par 6 E disse Sara: Deus me deu motivo de riso; e todo aquele que ouvir isso vai rir-se juntamente comigo.
\par 7 E acrescentou: Quem teria dito a Abraão que Sara amamentaria um filho? Pois na sua velhice lhe dei um filho.
\par 8 Isaque cresceu e foi desmamado. Nesse dia em que o menino foi desmamado, deu Abraão um grande banquete.
\par 9 Vendo Sara que o filho de Agar, a egípcia, o qual ela dera à luz a Abraão, caçoava de Isaque,
\par 10 disse a Abraão: Rejeita essa escrava e seu filho; porque o filho dessa escrava não será herdeiro com Isaque, meu filho.
\par 11 Pareceu isso mui penoso aos olhos de Abraão, por causa de seu filho.
\par 12 Disse, porém, Deus a Abraão: Não te pareça isso mal por causa do moço e por causa da tua serva; atende a Sara em tudo o que ela te disser; porque por Isaque será chamada a tua descendência.
\par 13 Mas também do filho da serva farei uma grande nação, por ser ele teu descendente.
\par 14 Levantou-se, pois, Abraão de madrugada, tomou pão e um odre de água, pô-los às costas de Agar, deu-lhe o menino e a despediu. Ela saiu, andando errante pelo deserto de Berseba.
\par 15 Tendo-se acabado a água do odre, colocou ela o menino debaixo de um dos arbustos
\par 16 e, afastando-se, foi sentar-se defronte, à distância de um tiro de arco; porque dizia: Assim, não verei morrer o menino; e, sentando-se em frente dele, levantou a voz e chorou.
\par 17 Deus, porém, ouviu a voz do menino; e o Anjo de Deus chamou do céu a Agar e lhe disse: Que tens, Agar? Não temas, porque Deus ouviu a voz do menino, daí onde está.
\par 18 Ergue-te, levanta o rapaz, segura-o pela mão, porque eu farei dele um grande povo.
\par 19 Abrindo-lhe Deus os olhos, viu ela um poço de água, e, indo a ele, encheu de água o odre, e deu de beber ao rapaz.
\par 20 Deus estava com o rapaz, que cresceu, habitou no deserto e se tornou flecheiro;
\par 21 habitou no deserto de Parã, e sua mãe o casou com uma mulher da terra do Egito.
\par 22 Por esse tempo, Abimeleque e Ficol, comandante do seu exército, disseram a Abraão: Deus é contigo em tudo o que fazes;
\par 23 agora, pois, jura-me aqui por Deus que me não mentirás, nem a meu filho, nem a meu neto; e sim que usarás comigo e com a terra em que tens habitado daquela mesma bondade com que eu te tratei.
\par 24 Respondeu Abraão: Juro.
\par 25 Nada obstante, Abraão repreendeu a Abimeleque por causa de um poço de água que os servos deste lhe haviam tomado à força.
\par 26 Respondeu-lhe Abimeleque: Não sei quem terá feito isso; também nada me fizeste saber, nem tampouco ouvi falar disso, senão hoje.
\par 27 Tomou Abraão ovelhas e bois e deu-os a Abimeleque; e fizeram ambos uma aliança.
\par 28 Pôs Abraão à parte sete cordeiras do rebanho.
\par 29 Perguntou Abimeleque a Abraão: Que significam as sete cordeiras que puseste à parte?
\par 30 Respondeu Abraão: Receberás de minhas mãos as sete cordeiras, para que me sirvam de testemunho de que eu cavei este poço.
\par 31 Por isso, se chamou aquele lugar Berseba, porque ali juraram eles ambos.
\par 32 Assim, fizeram aliança em Berseba; levantaram-se Abimeleque e Ficol, comandante do seu exército, e voltaram para as terras dos filisteus.
\par 33 Plantou Abraão tamargueiras em Berseba e invocou ali o nome do SENHOR, Deus Eterno.
\par 34 E foi Abraão, por muito tempo, morador na terra dos filisteus.

\chapter{22}

\par 1 Depois dessas coisas, pôs Deus Abraão à prova e lhe disse: Abraão! Este lhe respondeu: Eis-me aqui!
\par 2 Acrescentou Deus: Toma teu filho, teu único filho, Isaque, a quem amas, e vai-te à terra de Moriá; oferece-o ali em holocausto, sobre um dos montes, que eu te mostrarei.
\par 3 Levantou-se, pois, Abraão de madrugada e, tendo preparado o seu jumento, tomou consigo dois dos seus servos e a Isaque, seu filho; rachou lenha para o holocausto e foi para o lugar que Deus lhe havia indicado.
\par 4 Ao terceiro dia, erguendo Abraão os olhos, viu o lugar de longe.
\par 5 Então, disse a seus servos: Esperai aqui, com o jumento; eu e o rapaz iremos até lá e, havendo adorado, voltaremos para junto de vós.
\par 6 Tomou Abraão a lenha do holocausto e a colocou sobre Isaque, seu filho; ele, porém, levava nas mãos o fogo e o cutelo. Assim, caminhavam ambos juntos.
\par 7 Quando Isaque disse a Abraão, seu pai: Meu pai! Respondeu Abraão: Eis-me aqui, meu filho! Perguntou-lhe Isaque: Eis o fogo e a lenha, mas onde está o cordeiro para o holocausto?
\par 8 Respondeu Abraão: Deus proverá para si, meu filho, o cordeiro para o holocausto; e seguiam ambos juntos.
\par 9 Chegaram ao lugar que Deus lhe havia designado; ali edificou Abraão um altar, sobre ele dispôs a lenha, amarrou Isaque, seu filho, e o deitou no altar, em cima da lenha;
\par 10 e, estendendo a mão, tomou o cutelo para imolar o filho.
\par 11 Mas do céu lhe bradou o Anjo do SENHOR: Abraão! Abraão! Ele respondeu: Eis-me aqui!
\par 12 Então, lhe disse: Não estendas a mão sobre o rapaz e nada lhe faças; pois agora sei que temes a Deus, porquanto não me negaste o filho, o teu único filho.
\par 13 Tendo Abraão erguido os olhos, viu atrás de si um carneiro preso pelos chifres entre os arbustos; tomou Abraão o carneiro e o ofereceu em holocausto, em lugar de seu filho.
\par 14 E pôs Abraão por nome àquele lugar -- O SENHOR Proverá. Daí dizer-se até ao dia de hoje: No monte do SENHOR se proverá.
\par 15 Então, do céu bradou pela segunda vez o Anjo do SENHOR a Abraão
\par 16 e disse: Jurei, por mim mesmo, diz o SENHOR, porquanto fizeste isso e não me negaste o teu único filho,
\par 17 que deveras te abençoarei e certamente multiplicarei a tua descendência como as estrelas dos céus e como a areia na praia do mar; a tua descendência possuirá a cidade dos seus inimigos,
\par 18 nela serão benditas todas as nações da terra, porquanto obedeceste à minha voz.
\par 19 Então, voltou Abraão aos seus servos, e, juntos, foram para Berseba, onde fixou residência.
\par 20 Passadas essas coisas, foi dada notícia a Abraão, nestes termos: Milca também tem dado à luz filhos a Naor, teu irmão:
\par 21 Uz, o primogênito, Buz, seu irmão, Quemuel, pai de Arã,
\par 22 Quésede, Hazo, Pildas, Jidlafe e Betuel.
\par 23 Betuel gerou a Rebeca; estes oito deu à luz Milca a Naor, irmão de Abraão.
\par 24 Sua concubina, cujo nome era Reumá, lhe deu também à luz filhos: Teba, Gaã, Taás e Maaca.

\chapter{23}

\par 1 Tendo Sara vivido cento e vinte e sete anos,
\par 2 morreu em Quiriate-Arba, que é Hebrom, na terra de Canaã; veio Abraão lamentar Sara e chorar por ela.
\par 3 Levantou-se, depois, Abraão da presença de sua morta e falou aos filhos de Hete:
\par 4 Sou estrangeiro e morador entre vós; dai-me a posse de sepultura convosco, para que eu sepulte a minha morta.
\par 5 Responderam os filhos de Hete a Abraão, dizendo:
\par 6 Ouve-nos, senhor: tu és príncipe de Deus entre nós; sepulta numa das nossas melhores sepulturas a tua morta; nenhum de nós te vedará a sua sepultura, para sepultares a tua morta.
\par 7 Então, se levantou Abraão e se inclinou diante do povo da terra, diante dos filhos de Hete.
\par 8 E lhes falou, dizendo: Se é do vosso agrado que eu sepulte a minha morta, ouvi-me e intercedei por mim junto a Efrom, filho de Zoar,
\par 9 para que ele me dê a caverna de Macpela, que tem no extremo do seu campo; que ma dê pelo devido preço em posse de sepultura entre vós.
\par 10 Ora, Efrom, o heteu, sentando-se no meio dos filhos de Hete, respondeu a Abraão, ouvindo-o os filhos de Hete, a saber, todos os que entravam pela porta da sua cidade:
\par 11 De modo nenhum, meu senhor; ouve-me: dou-te o campo e também a caverna que nele está; na presença dos filhos do meu povo te dou; sepulta a tua morta.
\par 12 Então, se inclinou Abraão diante do povo da terra;
\par 13 e falou a Efrom, na presença do povo da terra, dizendo: Mas, se concordas, ouve-me, peço-te: darei o preço do campo, toma-o de mim, e sepultarei ali a minha morta.
\par 14 Respondeu-lhe Efrom:
\par 15 Meu senhor, ouve-me: um terreno que vale quatrocentos siclos de prata, que é isso entre mim e ti? Sepulta ali a tua morta.
\par 16 Tendo Abraão ouvido isso a Efrom, pesou-lhe a prata, de que este lhe falara diante dos filhos de Hete, quatrocentos siclos de prata, moeda corrente entre os mercadores.
\par 17 Assim, o campo de Efrom, que estava em Macpela, fronteiro a Manre, o campo, a caverna e todo o arvoredo que nele havia, e todo o limite ao redor
\par 18 se confirmaram por posse a Abraão, na presença dos filhos de Hete, de todos os que entravam pela porta da sua cidade.
\par 19 Depois, sepultou Abraão a Sara, sua mulher, na caverna do campo de Macpela, fronteiro a Manre, que é Hebrom, na terra de Canaã.
\par 20 E assim, pelos filhos de Hete, se confirmou a Abraão o direito do campo e da caverna que nele estava, em posse de sepultura.

\chapter{24}

\par 1 Era Abraão já idoso, bem avançado em anos; e o SENHOR em tudo o havia abençoado.
\par 2 Disse Abraão ao seu mais antigo servo da casa, que governava tudo o que possuía: Põe a mão por baixo da minha coxa,
\par 3 para que eu te faça jurar pelo SENHOR, Deus do céu e da terra, que não tomarás esposa para meu filho das filhas dos cananeus, entre os quais habito;
\par 4 mas irás à minha parentela e daí tomarás esposa para Isaque, meu filho.
\par 5 Disse-lhe o servo: Talvez não queira a mulher seguir-me para esta terra; nesse caso, levarei teu filho à terra donde saíste?
\par 6 Respondeu-lhe Abraão: Cautela! Não faças voltar para lá meu filho.
\par 7 O SENHOR, Deus do céu, que me tirou da casa de meu pai e de minha terra natal, e que me falou, e jurou, dizendo: À tua descendência darei esta terra, ele enviará o seu anjo, que te há de preceder, e tomarás de lá esposa para meu filho.
\par 8 Caso a mulher não queira seguir-te, ficarás desobrigado do teu juramento; entretanto, não levarás para lá meu filho.
\par 9 Com isso, pôs o servo a mão por baixo da coxa de Abraão, seu senhor, e jurou fazer segundo o resolvido.
\par 10 Tomou o servo dez dos camelos do seu senhor e, levando consigo de todos os bens dele, levantou-se e partiu, rumo da Mesopotâmia, para a cidade de Naor.
\par 11 Fora da cidade, fez ajoelhar os camelos junto a um poço de água, à tarde, hora em que as moças saem a tirar água.
\par 12 E disse consigo: Ó SENHOR, Deus de meu senhor Abraão, rogo-te que me acudas hoje e uses de bondade para com o meu senhor Abraão!
\par 13 Eis que estou ao pé da fonte de água, e as filhas dos homens desta cidade saem para tirar água;
\par 14 dá-me, pois, que a moça a quem eu disser: inclina o cântaro para que eu beba; e ela me responder: Bebe, e darei ainda de beber aos teus camelos, seja a que designaste para o teu servo Isaque; e nisso verei que usaste de bondade para com o meu senhor.
\par 15 Considerava ele ainda, quando saiu Rebeca, filha de Betuel, filho de Milca, mulher de Naor, irmão de Abraão, trazendo um cântaro ao ombro.
\par 16 A moça era mui formosa de aparência, virgem, a quem nenhum homem havia possuído; ela desceu à fonte, encheu o seu cântaro e subiu.
\par 17 Então, o servo saiu-lhe ao encontro e disse: Dá-me de beber um pouco da água do teu cântaro.
\par 18 Ela respondeu: Bebe, meu senhor. E, prontamente, baixando o cântaro para a mão, lhe deu de beber.
\par 19 Acabando ela de dar a beber, disse: Tirarei água também para os teus camelos, até que todos bebam.
\par 20 E, apressando-se em despejar o cântaro no bebedouro, correu outra vez ao poço para tirar mais água; tirou-a e deu-a a todos os camelos.
\par 21 O homem a observava, em silêncio, atentamente, para saber se teria o SENHOR levado a bom termo a sua jornada ou não.
\par 22 Tendo os camelos acabado de beber, tomou o homem um pendente de ouro de meio siclo de peso e duas pulseiras para as mãos dela, do peso de dez siclos de ouro;
\par 23 e lhe perguntou: De quem és filha? Peço-te que me digas. Haverá em casa de teu pai lugar em que eu fique, e a comitiva?
\par 24 Ela respondeu: Sou filha de Betuel, filho de Milca, o qual ela deu à luz a Naor.
\par 25 E acrescentou: Temos palha, e muito pasto, e lugar para passar a noite.
\par 26 Então, se inclinou o homem e adorou ao SENHOR.
\par 27 E disse: Bendito seja o SENHOR, Deus de meu senhor Abraão, que não retirou a sua benignidade e a sua verdade de meu senhor; quanto a mim, estando no caminho, o SENHOR me guiou à casa dos parentes de meu senhor.
\par 28 E a moça correu e contou aos da casa de sua mãe todas essas coisas.
\par 29 Ora, Rebeca tinha um irmão, chamado Labão; este correu ao encontro do homem junto à fonte.
\par 30 Pois, quando viu o pendente e as pulseiras nas mãos de sua irmã, tendo ouvido as palavras de Rebeca, sua irmã, que dizia: Assim me falou o homem, foi Labão ter com ele, o qual estava em pé junto aos camelos, junto à fonte.
\par 31 E lhe disse: Entra, bendito do SENHOR, por que estás aí fora? Pois já preparei a casa e o lugar para os camelos.
\par 32 Então, fez entrar o homem; descarregaram-lhe os camelos e lhes deram forragem e pasto; deu-se-lhe água para lavar os pés e também aos homens que estavam com ele.
\par 33 Diante dele puseram comida; porém ele disse: Não comerei enquanto não expuser o propósito a que venho. Labão respondeu-lhe: Dize.
\par 34 Então, disse: Sou servo de Abraão.
\par 35 O SENHOR tem abençoado muito ao meu senhor, e ele se tornou grande; deu-lhe ovelhas e bois, e prata e ouro, e servos e servas, e camelos e jumentos.
\par 36 Sara, mulher do meu senhor, era já idosa quando lhe deu à luz um filho; a este deu ele tudo quanto tem.
\par 37 E meu senhor me fez jurar, dizendo: Não tomarás esposa para meu filho das mulheres dos cananeus, em cuja terra habito;
\par 38 porém irás à casa de meu pai e à minha família e tomarás esposa para meu filho.
\par 39 Respondi ao meu senhor: Talvez não queira a mulher seguir-me.
\par 40 Ele me disse: O SENHOR, em cuja presença eu ando, enviará contigo o seu Anjo e levará a bom termo a tua jornada, para que, da minha família e da casa de meu pai, tomes esposa para meu filho.
\par 41 Então, serás desobrigado do meu juramento, quando fores à minha família; se não ta derem, desobrigado estarás do meu juramento.
\par 42 Hoje, pois, cheguei à fonte e disse comigo: ó SENHOR, Deus de meu senhor Abraão, se me levas a bom termo a jornada em que sigo,
\par 43 eis-me agora junto à fonte de água; a moça que sair para tirar água, a quem eu disser: dá-me um pouco de água do teu cântaro,
\par 44 e ela me responder: Bebe, e também tirarei água para os teus camelos, seja essa a mulher que o SENHOR designou para o filho de meu senhor.
\par 45 Considerava ainda eu assim, no meu íntimo, quando saiu Rebeca trazendo o seu cântaro ao ombro, desceu à fonte e tirou água. E eu lhe disse: peço-te que me dês de beber.
\par 46 Ela se apressou e, baixando o cântaro do ombro, disse: Bebe, e também darei de beber aos teus camelos. Bebi, e ela deu de beber aos camelos.
\par 47 Daí lhe perguntei: de quem és filha? Ela respondeu: Filha de Betuel, filho de Naor e Milca. Então, lhe pus o pendente no nariz e as pulseiras nas mãos.
\par 48 E, prostrando-me, adorei ao SENHOR e bendisse ao SENHOR, Deus do meu senhor Abraão, que me havia conduzido por um caminho direito, a fim de tomar para o filho do meu senhor uma filha do seu parente.
\par 49 Agora, pois, se haveis de usar de benevolência e de verdade para com o meu senhor, fazei-mo saber; se não, declarai-mo, para que eu vá, ou para a direita ou para a esquerda.
\par 50 Então, responderam Labão e Betuel: Isto procede do SENHOR, nada temos a dizer fora da sua verdade.
\par 51 Eis Rebeca na tua presença; toma-a e vai-te; seja ela a mulher do filho do teu senhor, segundo a palavra do SENHOR.
\par 52 Tendo ouvido o servo de Abraão tais palavras, prostrou-se em terra diante do SENHOR;
\par 53 e tirou jóias de ouro e de prata e vestidos e os deu a Rebeca; também deu ricos presentes a seu irmão e a sua mãe.
\par 54 Depois, comeram, e beberam, ele e os homens que estavam com ele, e passaram a noite. De madrugada, quando se levantaram, disse o servo: Permiti que eu volte ao meu senhor.
\par 55 Mas o irmão e a mãe da moça disseram: Fique ela ainda conosco alguns dias, pelo menos dez; e depois irá.
\par 56 Ele, porém, lhes disse: Não me detenhais, pois o SENHOR me tem levado a bom termo na jornada; permiti que eu volte ao meu senhor.
\par 57 Disseram: Chamemos a moça e ouçamo-la pessoalmente.
\par 58 Chamaram, pois, a Rebeca e lhe perguntaram: Queres ir com este homem? Ela respondeu: Irei.
\par 59 Então, despediram a Rebeca, sua irmã, e a sua ama, e ao servo de Abraão, e a seus homens.
\par 60 Abençoaram a Rebeca e lhe disseram: És nossa irmã; sê tu a mãe de milhares de milhares, e que a tua descendência possua a porta dos seus inimigos.
\par 61 Então, se levantou Rebeca com suas moças e, montando os camelos, seguiram o homem. O servo tomou a Rebeca e partiu.
\par 62 Ora, Isaque vinha de caminho de Beer-Laai-Roi, porque habitava na terra do Neguebe.
\par 63 Saíra Isaque a meditar no campo, ao cair da tarde; erguendo os olhos, viu, e eis que vinham camelos.
\par 64 Também Rebeca levantou os olhos, e, vendo a Isaque, apeou do camelo,
\par 65 e perguntou ao servo: Quem é aquele homem que vem pelo campo ao nosso encontro? É o meu senhor, respondeu. Então, tomou ela o véu e se cobriu.
\par 66 O servo contou a Isaque todas as coisas que havia feito.
\par 67 Isaque conduziu-a até à tenda de Sara, mãe dele, e tomou a Rebeca, e esta lhe foi por mulher. Ele a amou; assim, foi Isaque consolado depois da morte de sua mãe.

\chapter{25}

\par 1 Desposou Abraão outra mulher; chamava-se Quetura.
\par 2 Ela lhe deu à luz a Zinrã, Jocsã, Medã, Midiã, Isbaque e Suá.
\par 3 Jocsã gerou a Seba e a Dedã; os filhos de Dedã foram: Assurim, Letusim e Leumim.
\par 4 Os filhos de Midiã foram: Efá, Efer, Enoque, Abida e Elda. Todos estes foram filhos de Quetura.
\par 5 Abraão deu tudo o que possuía a Isaque.
\par 6 Porém, aos filhos das concubinas que tinha, deu ele presentes e, ainda em vida, os separou de seu filho Isaque, enviando-os para a terra oriental.
\par 7 Foram os dias da vida de Abraão cento e setenta e cinco anos.
\par 8 Expirou Abraão; morreu em ditosa velhice, avançado em anos; e foi reunido ao seu povo.
\par 9 Sepultaram-no Isaque e Ismael, seus filhos, na caverna de Macpela, no campo de Efrom, filho de Zoar, o heteu, fronteiro a Manre,
\par 10 o campo que Abraão comprara aos filhos de Hete. Ali foi sepultado Abraão e Sara, sua mulher.
\par 11 Depois da morte de Abraão, Deus abençoou a Isaque, seu filho; Isaque habitava junto a Beer-Laai-Roi.
\par 12 São estas as gerações de Ismael, filho de Abraão, que Agar, egípcia, serva de Sara, lhe deu à luz.
\par 13 E estes, os filhos de Ismael, pelos seus nomes, segundo o seu nascimento: o primogênito de Ismael foi Nebaiote; depois, Quedar, Abdeel, Mibsão,
\par 14 Misma, Dumá, Massá,
\par 15 Hadade, Tema, Jetur, Nafis e Quedemá.
\par 16 São estes os filhos de Ismael, e estes, os seus nomes pelas suas vilas e pelos seus acampamentos: doze príncipes de seus povos.
\par 17 E os anos da vida de Ismael foram cento e trinta e sete; e morreu e foi reunido ao seu povo.
\par 18 Habitaram desde Havilá até Sur, que olha para o Egito, como quem vai para a Assíria. Ele se estabeleceu fronteiro a todos os seus irmãos.
\par 19 São estas as gerações de Isaque, filho de Abraão. Abraão gerou a Isaque;
\par 20 era Isaque de quarenta anos, quando tomou por esposa a Rebeca, filha de Betuel, o arameu de Padã-Arã, e irmã de Labão, o arameu.
\par 21 Isaque orou ao SENHOR por sua mulher, porque ela era estéril; e o SENHOR lhe ouviu as orações, e Rebeca, sua mulher, concebeu.
\par 22 Os filhos lutavam no ventre dela; então, disse: Se é assim, por que vivo eu? E consultou ao SENHOR.
\par 23 Respondeu-lhe o SENHOR: Duas nações há no teu ventre, dois povos, nascidos de ti, se dividirão: um povo será mais forte que o outro, e o mais velho servirá ao mais moço.
\par 24 Cumpridos os dias para que desse à luz, eis que se achavam gêmeos no seu ventre.
\par 25 Saiu o primeiro, ruivo, todo revestido de pêlo; por isso, lhe chamaram Esaú.
\par 26 Depois, nasceu o irmão; segurava com a mão o calcanhar de Esaú; por isso, lhe chamaram Jacó. Era Isaque de sessenta anos, quando Rebeca lhos deu à luz.
\par 27 Cresceram os meninos. Esaú saiu perito caçador, homem do campo; Jacó, porém, homem pacato, habitava em tendas.
\par 28 Isaque amava a Esaú, porque se saboreava de sua caça; Rebeca, porém, amava a Jacó.
\par 29 Tinha Jacó feito um cozinhado, quando, esmorecido, veio do campo Esaú
\par 30 e lhe disse: Peço-te que me deixes comer um pouco desse cozinhado vermelho, pois estou esmorecido. Daí chamar-se Edom.
\par 31 Disse Jacó: Vende-me primeiro o teu direito de primogenitura.
\par 32 Ele respondeu: Estou a ponto de morrer; de que me aproveitará o direito de primogenitura?
\par 33 Então, disse Jacó: Jura-me primeiro. Ele jurou e vendeu o seu direito de primogenitura a Jacó.
\par 34 Deu, pois, Jacó a Esaú pão e o cozinhado de lentilhas; ele comeu e bebeu, levantou-se e saiu. Assim, desprezou Esaú o seu direito de primogenitura.

\chapter{26}

\par 1 Sobrevindo fome à terra, além da primeira havida nos dias de Abraão, foi Isaque a Gerar, avistar-se com Abimeleque, rei dos filisteus.
\par 2 Apareceu-lhe o SENHOR e disse: Não desças ao Egito. Fica na terra que eu te disser;
\par 3 habita nela, e serei contigo e te abençoarei; porque a ti e a tua descendência darei todas estas terras e confirmarei o juramento que fiz a Abraão, teu pai.
\par 4 Multiplicarei a tua descendência como as estrelas dos céus e lhe darei todas estas terras. Na tua descendência serão abençoadas todas as nações da terra;
\par 5 porque Abraão obedeceu à minha palavra e guardou os meus mandados, os meus preceitos, os meus estatutos e as minhas leis.
\par 6 Isaque, pois, ficou em Gerar.
\par 7 Perguntando-lhe os homens daquele lugar a respeito de sua mulher, disse: É minha irmã; pois temia dizer: É minha mulher; para que, dizia ele consigo, os homens do lugar não me matem por amor de Rebeca, porque era formosa de aparência.
\par 8 Ora, tendo Isaque permanecido ali por muito tempo, Abimeleque, rei dos filisteus, olhando da janela, viu que Isaque acariciava a Rebeca, sua mulher.
\par 9 Então, Abimeleque chamou a Isaque e lhe disse: É evidente que ela é tua esposa; como, pois, disseste: É minha irmã? Respondeu-lhe Isaque: Porque eu dizia: para que eu não morra por causa dela.
\par 10 Disse Abimeleque: Que é isso que nos fizeste? Facilmente algum do povo teria abusado de tua mulher, e tu, atraído sobre nós grave delito.
\par 11 E deu esta ordem a todo o povo: Qualquer que tocar a este homem ou à sua mulher certamente morrerá.
\par 12 Semeou Isaque naquela terra e, no mesmo ano, recolheu cento por um, porque o SENHOR o abençoava.
\par 13 Enriqueceu-se o homem, prosperou, ficou riquíssimo;
\par 14 possuía ovelhas e bois e grande número de servos, de maneira que os filisteus lhe tinham inveja.
\par 15 E, por isso, lhe entulharam todos os poços que os servos de seu pai haviam cavado, nos dias de Abraão, enchendo-os de terra.
\par 16 Disse Abimeleque a Isaque: Aparta-te de nós, porque já és muito mais poderoso do que nós.
\par 17 Então, Isaque saiu dali e se acampou no vale de Gerar, onde habitou.
\par 18 E tornou Isaque a abrir os poços que se cavaram nos dias de Abraão, seu pai (porque os filisteus os haviam entulhado depois da morte de Abraão), e lhes deu os mesmos nomes que já seu pai lhes havia posto.
\par 19 Cavaram os servos de Isaque no vale e acharam um poço de água nascente.
\par 20 Mas os pastores de Gerar contenderam com os pastores de Isaque, dizendo: Esta água é nossa. Por isso, chamou o poço de Eseque, porque contenderam com ele.
\par 21 Então, cavaram outro poço e também por causa desse contenderam. Por isso, recebeu o nome de Sitna.
\par 22 Partindo dali, cavou ainda outro poço; e, como por esse não contenderam, chamou-lhe Reobote e disse: Porque agora nos deu lugar o SENHOR, e prosperaremos na terra.
\par 23 Dali subiu para Berseba.
\par 24 Na mesma noite, lhe apareceu o SENHOR e disse: Eu sou o Deus de Abraão, teu pai. Não temas, porque eu sou contigo; abençoar-te-ei e multiplicarei a tua descendência por amor de Abraão, meu servo.
\par 25 Então, levantou ali um altar e, tendo invocado o nome do SENHOR, armou a sua tenda; e os servos de Isaque abriram ali um poço.
\par 26 De Gerar foram ter com ele Abimeleque e seu amigo Ausate e Ficol, comandante do seu exército.
\par 27 Disse-lhes Isaque: Por que viestes a mim, pois me odiais e me expulsastes do vosso meio?
\par 28 Eles responderam: Vimos claramente que o SENHOR é contigo; então, dissemos: Haja agora juramento entre nós e ti, e façamos aliança contigo.
\par 29 Jura que nos não farás mal, como também não te havemos tocado, e como te fizemos somente o bem, e te deixamos ir em paz. Tu és agora o abençoado do SENHOR.
\par 30 Então, Isaque lhes deu um banquete, e comeram e beberam.
\par 31 Levantando-se de madrugada, juraram de parte a parte; Isaque os despediu, e eles se foram em paz.
\par 32 Nesse mesmo dia, vieram os servos de Isaque e, dando-lhe notícia do poço que tinham cavado, lhe disseram: Achamos água.
\par 33 Ao poço, chamou-lhe Seba; por isso, Berseba é o nome daquela cidade até ao dia de hoje.
\par 34 Tendo Esaú quarenta anos de idade, tomou por esposa a Judite, filha de Beeri, heteu, e a Basemate, filha de Elom, heteu.
\par 35 Ambas se tornaram amargura de espírito para Isaque e para Rebeca.

\chapter{27}

\par 1 Tendo-se envelhecido Isaque e já não podendo ver, porque os olhos se lhe enfraqueciam, chamou a Esaú, seu filho mais velho, e lhe disse: Meu filho! Respondeu ele: Aqui estou!
\par 2 Disse-lhe o pai: Estou velho e não sei o dia da minha morte.
\par 3 Agora, pois, toma as tuas armas, a tua aljava e o teu arco, sai ao campo, e apanha para mim alguma caça,
\par 4 e faze-me uma comida saborosa, como eu aprecio, e traze-ma, para que eu coma e te abençoe antes que eu morra.
\par 5 Rebeca esteve escutando enquanto Isaque falava com Esaú, seu filho. E foi-se Esaú ao campo para apanhar a caça e trazê-la.
\par 6 Então, disse Rebeca a Jacó, seu filho: Ouvi teu pai falar com Esaú, teu irmão, assim:
\par 7 Traze caça e faze-me uma comida saborosa, para que eu coma e te abençoe diante do SENHOR, antes que eu morra.
\par 8 Agora, pois, meu filho, atende às minhas palavras com que te ordeno.
\par 9 Vai ao rebanho e traze-me dois bons cabritos; deles farei uma saborosa comida para teu pai, como ele aprecia;
\par 10 levá-la-ás a teu pai, para que a coma e te abençoe, antes que morra.
\par 11 Disse Jacó a Rebeca, sua mãe: Esaú, meu irmão, é homem cabeludo, e eu, homem liso.
\par 12 Dar-se-á o caso de meu pai me apalpar, e passarei a seus olhos por zombador; assim, trarei sobre mim maldição e não bênção.
\par 13 Respondeu-lhe a mãe: Caia sobre mim essa maldição, meu filho; atende somente o que eu te digo, vai e traze-mos.
\par 14 Ele foi, tomou-os e os trouxe a sua mãe, que fez uma saborosa comida, como o pai dele apreciava.
\par 15 Depois, tomou Rebeca a melhor roupa de Esaú, seu filho mais velho, roupa que tinha consigo em casa, e vestiu a Jacó, seu filho mais novo.
\par 16 Com a pele dos cabritos cobriu-lhe as mãos e a lisura do pescoço.
\par 17 Então, entregou a Jacó, seu filho, a comida saborosa e o pão que havia preparado.
\par 18 Jacó foi a seu pai e disse: Meu pai! Ele respondeu: Fala! Quem és tu, meu filho?
\par 19 Respondeu Jacó a seu pai: Sou Esaú, teu primogênito; fiz o que me ordenaste. Levanta-te, pois, assenta-te e come da minha caça, para que me abençoes.
\par 20 Disse Isaque a seu filho: Como é isso que a pudeste achar tão depressa, meu filho? Ele respondeu: Porque o SENHOR, teu Deus, a mandou ao meu encontro.
\par 21 Então, disse Isaque a Jacó: Chega-te aqui, para que eu te apalpe, meu filho, e veja se és meu filho Esaú ou não.
\par 22 Jacó chegou-se a Isaque, seu pai, que o apalpou e disse: A voz é de Jacó, porém as mãos são de Esaú.
\par 23 E não o reconheceu, porque as mãos, com efeito, estavam peludas como as de seu irmão Esaú. E o abençoou.
\par 24 E lhe disse: És meu filho Esaú mesmo? Ele respondeu: Eu sou.
\par 25 Então, disse: Chega isso para perto de mim, para que eu coma da caça de meu filho; para que eu te abençoe. Chegou-lho, e ele comeu; trouxe-lhe também vinho, e ele bebeu.
\par 26 Então, lhe disse Isaque, seu pai: Chega-te e dá-me um beijo, meu filho.
\par 27 Ele se chegou e o beijou. Então, o pai aspirou o cheiro da roupa dele, e o abençoou, e disse: Eis que o cheiro do meu filho é como o cheiro do campo, que o SENHOR abençoou;
\par 28 Deus te dê do orvalho do céu, e da exuberância da terra, e fartura de trigo e de mosto.
\par 29 Sirvam-te povos, e nações te reverenciem; sê senhor de teus irmãos, e os filhos de tua mãe se encurvem a ti; maldito seja o que te amaldiçoar, e abençoado o que te abençoar.
\par 30 Mal acabara Isaque de abençoar a Jacó, tendo este saído da presença de Isaque, seu pai, chega Esaú, seu irmão, da sua caçada.
\par 31 E fez também ele uma comida saborosa, a trouxe a seu pai e lhe disse: Levanta-te, meu pai, e come da caça de teu filho, para que me abençoes.
\par 32 Perguntou-lhe Isaque, seu pai: Quem és tu? Sou Esaú, teu filho, o teu primogênito, respondeu.
\par 33 Então, estremeceu Isaque de violenta comoção e disse: Quem é, pois, aquele que apanhou a caça e ma trouxe? Eu comi de tudo, antes que viesses, e o abençoei, e ele será abençoado.
\par 34 Como ouvisse Esaú tais palavras de seu pai, bradou com profundo amargor e lhe disse: Abençoa-me também a mim, meu pai!
\par 35 Respondeu-lhe o pai: Veio teu irmão astuciosamente e tomou a tua bênção.
\par 36 Disse Esaú: Não é com razão que se chama ele Jacó? Pois já duas vezes me enganou: tirou-me o direito de primogenitura e agora usurpa a bênção que era minha. Disse ainda: Não reservaste, pois, bênção nenhuma para mim?
\par 37 Então, respondeu Isaque a Esaú: Eis que o constituí em teu senhor, e todos os seus irmãos lhe dei por servos; de trigo e de mosto o apercebi; que me será dado fazer-te agora, meu filho?
\par 38 Disse Esaú a seu pai: Acaso, tens uma única bênção, meu pai? Abençoa-me, também a mim, meu pai. E, levantando Esaú a voz, chorou.
\par 39 Então, lhe respondeu Isaque, seu pai: Longe dos lugares férteis da terra será a tua habitação, e sem orvalho que cai do alto.
\par 40 Viverás da tua espada e servirás a teu irmão; quando, porém, te libertares, sacudirás o seu jugo da tua cerviz.
\par 41 Passou Esaú a odiar a Jacó por causa da bênção, com que seu pai o tinha abençoado; e disse consigo: Vêm próximos os dias de luto por meu pai; então, matarei a Jacó, meu irmão.
\par 42 Chegaram aos ouvidos de Rebeca estas palavras de Esaú, seu filho mais velho; ela, pois, mandou chamar a Jacó, seu filho mais moço, e lhe disse: Eis que Esaú, teu irmão, se consola a teu respeito, resolvendo matar-te.
\par 43 Agora, pois, meu filho, ouve o que te digo: retira-te para a casa de Labão, meu irmão, em Harã;
\par 44 fica com ele alguns dias, até que passe o furor de teu irmão,
\par 45 e cesse o seu rancor contra ti, e se esqueça do que lhe fizeste. Então, providenciarei e te farei regressar de lá. Por que hei de eu perder os meus dois filhos num só dia?
\par 46 Disse Rebeca a Isaque: Aborrecida estou da minha vida, por causa das filhas de Hete; se Jacó tomar esposa dentre as filhas de Hete, tais como estas, as filhas desta terra, de que me servirá a vida?

\chapter{28}

\par 1 Isaque chamou a Jacó e, dando-lhe a sua bênção, lhe ordenou, dizendo: Não tomarás esposa dentre as filhas de Canaã.
\par 2 Levanta-te, vai a Padã-Arã, à casa de Betuel, pai de tua mãe, e toma lá por esposa uma das filhas de Labão, irmão de tua mãe.
\par 3 Deus Todo-Poderoso te abençoe, e te faça fecundo, e te multiplique para que venhas a ser uma multidão de povos;
\par 4 e te dê a bênção de Abraão, a ti e à tua descendência contigo, para que possuas a terra de tuas peregrinações, concedida por Deus a Abraão.
\par 5 Assim, despediu Isaque a Jacó, que se foi a Padã-Arã, à casa de Labão, filho de Betuel, o arameu, irmão de Rebeca, mãe de Jacó e de Esaú.
\par 6 Vendo, pois, Esaú que Isaque abençoara a Jacó e o enviara a Padã-Arã, para tomar de lá esposa para si; e vendo que, ao abençoá-lo, lhe ordenara, dizendo: Não tomarás mulher dentre as filhas de Canaã;
\par 7 e vendo, ainda, que Jacó, obedecendo a seu pai e a sua mãe, fora a Padã-Arã;
\par 8 sabedor também de que Isaque, seu pai, não via com bons olhos as filhas de Canaã,
\par 9 foi Esaú à casa de Ismael e, além das mulheres que já possuía, tomou por mulher a Maalate, filha de Ismael, filho de Abraão, e irmã de Nebaiote.
\par 10 Partiu Jacó de Berseba e seguiu para Harã.
\par 11 Tendo chegado a certo lugar, ali passou a noite, pois já era sol-posto; tomou uma das pedras do lugar, fê-la seu travesseiro e se deitou ali mesmo para dormir.
\par 12 E sonhou: Eis posta na terra uma escada cujo topo atingia o céu; e os anjos de Deus subiam e desciam por ela.
\par 13 Perto dele estava o SENHOR e lhe disse: Eu sou o SENHOR, Deus de Abraão, teu pai, e Deus de Isaque. A terra em que agora estás deitado, eu ta darei, a ti e à tua descendência.
\par 14 A tua descendência será como o pó da terra; estender-te-ás para o Ocidente e para o Oriente, para o Norte e para o Sul. Em ti e na tua descendência serão abençoadas todas as famílias da terra.
\par 15 Eis que eu estou contigo, e te guardarei por onde quer que fores, e te farei voltar a esta terra, porque te não desampararei, até cumprir eu aquilo que te hei referido.
\par 16 Despertado Jacó do seu sono, disse: Na verdade, o SENHOR está neste lugar, e eu não o sabia.
\par 17 E, temendo, disse: Quão temível é este lugar! É a Casa de Deus, a porta dos céus.
\par 18 Tendo-se levantado Jacó, cedo, de madrugada, tomou a pedra que havia posto por travesseiro e a erigiu em coluna, sobre cujo topo entornou azeite.
\par 19 E ao lugar, cidade que outrora se chamava Luz, deu o nome de Betel.
\par 20 Fez também Jacó um voto, dizendo: Se Deus for comigo, e me guardar nesta jornada que empreendo, e me der pão para comer e roupa que me vista,
\par 21 de maneira que eu volte em paz para a casa de meu pai, então, o SENHOR será o meu Deus;
\par 22 e a pedra, que erigi por coluna, será a Casa de Deus; e, de tudo quanto me concederes, certamente eu te darei o dízimo.

\chapter{29}

\par 1 Pôs-se Jacó a caminho e se foi à terra do povo do Oriente.
\par 2 Olhou, e eis um poço no campo e três rebanhos de ovelhas deitados junto dele; porque daquele poço davam de beber aos rebanhos; e havia grande pedra que tapava a boca do poço.
\par 3 Ajuntavam-se ali todos os rebanhos, os pastores removiam a pedra da boca do poço, davam de beber às ovelhas e tornavam a colocá-la no seu devido lugar.
\par 4 Perguntou-lhes Jacó: Meus irmãos, donde sois? Responderam: Somos de Harã.
\par 5 Perguntou-lhes: Conheceis a Labão, filho de Naor? Responderam: Conhecemos.
\par 6 Ele está bom? Perguntou ainda Jacó. Responderam: Está bom. Raquel, sua filha, vem vindo aí com as ovelhas.
\par 7 Então, lhes disse: É ainda pleno dia, não é tempo de se recolherem os rebanhos; dai de beber às ovelhas e ide apascentá-las.
\par 8 Não o podemos, responderam eles, enquanto não se ajuntarem todos os rebanhos, e seja removida a pedra da boca do poço, e lhes demos de beber.
\par 9 Falava-lhes ainda, quando chegou Raquel com as ovelhas de seu pai; porque era pastora.
\par 10 Tendo visto Jacó a Raquel, filha de Labão, irmão de sua mãe, e as ovelhas de Labão, chegou-se, removeu a pedra da boca do poço e deu de beber ao rebanho de Labão, irmão de sua mãe.
\par 11 Feito isso, Jacó beijou a Raquel e, erguendo a voz, chorou.
\par 12 Então, contou Jacó a Raquel que ele era parente de seu pai, pois era filho de Rebeca; ela correu e o comunicou a seu pai.
\par 13 Tendo Labão ouvido as novas de Jacó, filho de sua irmã, correu-lhe ao encontro, abraçou-o, beijou-o e o levou para casa. E contou Jacó a Labão os acontecimentos de sua viagem.
\par 14 Disse-lhe Labão: De fato, és meu osso e minha carne. E Jacó, pelo espaço de um mês, permaneceu com ele.
\par 15 Depois, disse Labão a Jacó: Acaso, por seres meu parente, irás servir-me de graça? Dize-me, qual será o teu salário?
\par 16 Ora, Labão tinha duas filhas: Lia, a mais velha, e Raquel, a mais moça.
\par 17 Lia tinha os olhos baços, porém Raquel era formosa de porte e de semblante.
\par 18 Jacó amava a Raquel e disse: Sete anos te servirei por tua filha mais moça, Raquel.
\par 19 Respondeu Labão: Melhor é que eu ta dê, em vez de dá-la a outro homem; fica, pois, comigo.
\par 20 Assim, por amor a Raquel, serviu Jacó sete anos; e estes lhe pareceram como poucos dias, pelo muito que a amava.
\par 21 Disse Jacó a Labão: Dá-me minha mulher, pois já venceu o prazo, para que me case com ela.
\par 22 Reuniu, pois, Labão todos os homens do lugar e deu um banquete.
\par 23 À noite, conduziu a Lia, sua filha, e a entregou a Jacó. E coabitaram.
\par 24 (Para serva de Lia, sua filha, deu Labão Zilpa, sua serva.)
\par 25 Ao amanhecer, viu que era Lia. Por isso, disse Jacó a Labão: Que é isso que me fizeste? Não te servi eu por amor a Raquel? Por que, pois, me enganaste?
\par 26 Respondeu Labão: Não se faz assim em nossa terra, dar-se a mais nova antes da primogênita.
\par 27 Decorrida a semana desta, dar-te-emos também a outra, pelo trabalho de mais sete anos que ainda me servirás.
\par 28 Concordou Jacó, e se passou a semana desta; então, Labão lhe deu por mulher Raquel, sua filha.
\par 29 (Para serva de Raquel, sua filha, deu Labão a sua serva Bila.)
\par 30 E coabitaram. Mas Jacó amava mais a Raquel do que a Lia; e continuou servindo a Labão por outros sete anos.
\par 31 Vendo o SENHOR que Lia era desprezada, fê-la fecunda; ao passo que Raquel era estéril.
\par 32 Concebeu, pois, Lia e deu à luz um filho, a quem chamou Rúben, pois disse: O SENHOR atendeu à minha aflição. Por isso, agora me amará meu marido.
\par 33 Concebeu outra vez, e deu à luz um filho, e disse: Soube o SENHOR que era preterida e me deu mais este; chamou-lhe, pois, Simeão.
\par 34 Outra vez concebeu Lia, e deu à luz um filho, e disse: Agora, desta vez, se unirá mais a mim meu marido, porque lhe dei à luz três filhos; por isso, lhe chamou Levi.
\par 35 De novo concebeu e deu à luz um filho; então, disse: Esta vez louvarei o SENHOR. E por isso lhe chamou Judá; e cessou de dar à luz.

\chapter{30}

\par 1 Vendo Raquel que não dava filhos a Jacó, teve ciúmes de sua irmã e disse a Jacó: Dá-me filhos, senão morrerei.
\par 2 Então, Jacó se irou contra Raquel e disse: Acaso, estou eu em lugar de Deus que ao teu ventre impediu frutificar?
\par 3 Respondeu ela: Eis aqui Bila, minha serva; coabita com ela, para que dê à luz, e eu traga filhos ao meu colo, por meio dela.
\par 4 Assim, lhe deu a Bila, sua serva, por mulher; e Jacó a possuiu.
\par 5 Bila concebeu e deu à luz um filho a Jacó.
\par 6 Então, disse Raquel: Deus me julgou, e também me ouviu a voz, e me deu um filho; portanto, lhe chamou Dã.
\par 7 Concebeu outra vez Bila, serva de Raquel, e deu à luz o segundo filho a Jacó.
\par 8 Disse Raquel: Com grandes lutas tenho competido com minha irmã e logrei prevalecer; chamou-lhe, pois, Naftali.
\par 9 Vendo Lia que ela mesma cessara de conceber, tomou também a Zilpa, sua serva, e deu-a a Jacó, por mulher.
\par 10 Zilpa, serva de Lia, deu a Jacó um filho.
\par 11 Disse Lia: Afortunada! E lhe chamou Gade.
\par 12 Depois, Zilpa, serva de Lia, deu o segundo filho a Jacó.
\par 13 Então, disse Lia: É a minha felicidade! Porque as filhas me terão por venturosa; e lhe chamou Aser.
\par 14 Foi Rúben nos dias da ceifa do trigo, e achou mandrágoras no campo, e trouxe-as a Lia, sua mãe. Então, disse Raquel a Lia: Dá-me das mandrágoras de teu filho.
\par 15 Respondeu ela: Achas pouco o me teres levado o marido? Tomarás também as mandrágoras de meu filho? Disse Raquel: Ele te possuirá esta noite, a troco das mandrágoras de teu filho.
\par 16 À tarde, vindo Jacó do campo, saiu-lhe ao encontro Lia e lhe disse: Esta noite me possuirás, pois eu te aluguei pelas mandrágoras de meu filho. E Jacó, naquela noite, coabitou com ela.
\par 17 Ouviu Deus a Lia; ela concebeu e deu à luz o quinto filho.
\par 18 Então, disse Lia: Deus me recompensou, porque dei a minha serva a meu marido; e chamou-lhe Issacar.
\par 19 E Lia, tendo concebido outra vez, deu a Jacó o sexto filho.
\par 20 E disse: Deus me concedeu excelente dote; desta vez permanecerá comigo meu marido, porque lhe dei seis filhos; e lhe chamou Zebulom.
\par 21 Depois disto, deu à luz uma filha e lhe chamou Diná.
\par 22 Lembrou-se Deus de Raquel, ouviu-a e a fez fecunda.
\par 23 Ela concebeu, deu à luz um filho e disse: Deus me tirou o meu vexame.
\par 24 E lhe chamou José, dizendo: Dê-me o SENHOR ainda outro filho.
\par 25 Tendo Raquel dado à luz a José, disse Jacó a Labão: Permite-me que eu volte ao meu lugar e à minha terra.
\par 26 Dá-me meus filhos e as mulheres, pelas quais eu te servi, e partirei; pois tu sabes quanto e de que maneira te servi.
\par 27 Labão lhe respondeu: Ache eu mercê diante de ti; fica comigo. Tenho experimentado que o SENHOR me abençoou por amor de ti.
\par 28 E disse ainda: Fixa o teu salário, que te pagarei.
\par 29 Disse-lhe Jacó: Tu sabes como te venho servindo e como cuidei do teu gado.
\par 30 Porque o pouco que tinhas antes da minha vinda foi aumentado grandemente; e o SENHOR te abençoou por meu trabalho. Agora, pois, quando hei de eu trabalhar também por minha casa?
\par 31 Então, Labão lhe perguntou: Que te darei? Respondeu Jacó: Nada me darás; tornarei a apascentar e a guardar o teu rebanho, se me fizeres isto:
\par 32 Passarei hoje por todo o teu rebanho, separando dele os salpicados e malhados, e todos os negros entre os cordeiros, e o que é malhado e salpicado entre as cabras; será isto o meu salário.
\par 33 Assim, responderá por mim a minha justiça, no dia de amanhã, quando vieres ver o meu salário diante de ti; o que não for salpicado e malhado entre as cabras e negro entre as ovelhas, esse, se for achado comigo, será tido por furtado.
\par 34 Disse Labão: Pois sim! Seja conforme a tua palavra.
\par 35 Mas, naquele mesmo dia, separou Labão os bodes listados e malhados e todas as cabras salpicadas e malhadas, todos os que tinham alguma brancura e todos os negros entre os cordeiros; e os passou às mãos de seus filhos.
\par 36 E pôs a distância de três dias de jornada entre si e Jacó; e Jacó apascentava o restante dos rebanhos de Labão.
\par 37 Tomou, então, Jacó varas verdes de álamo, de aveleira e de plátano e lhes removeu a casca, em riscas abertas, deixando aparecer a brancura das varas,
\par 38 as quais, assim escorchadas, pôs ele em frente do rebanho, nos canais de água e nos bebedouros, aonde os rebanhos vinham para dessedentar-se, e conceberam quando vinham a beber.
\par 39 E concebia o rebanho diante das varas, e as ovelhas davam crias listadas, salpicadas e malhadas.
\par 40 Então, separou Jacó os cordeiros e virou o rebanho para o lado dos listados e dos pretos nos rebanhos de Labão; e pôs o seu rebanho à parte e não o juntou com o rebanho de Labão.
\par 41 E, todas as vezes que concebiam as ovelhas fortes, punha Jacó as varas à vista do rebanho nos canais de água, para que concebessem diante das varas.
\par 42 Porém, quando o rebanho era fraco, não as punha; assim, as fracas eram de Labão, e as fortes, de Jacó.
\par 43 E o homem se tornou mais e mais rico; teve muitos rebanhos, e servas, e servos, e camelos, e jumentos.

\chapter{31}

\par 1 Então, ouvia Jacó os comentários dos filhos de Labão, que diziam: Jacó se apossou de tudo o que era de nosso pai; e do que era de nosso pai juntou ele toda esta riqueza.
\par 2 Jacó, por sua vez, reparou que o rosto de Labão não lhe era favorável, como anteriormente.
\par 3 E disse o SENHOR a Jacó: Torna à terra de teus pais e à tua parentela; e eu serei contigo.
\par 4 Então, Jacó mandou vir Raquel e Lia ao campo, para junto do seu rebanho,
\par 5 e lhes disse: Vejo que o rosto de vosso pai não me é favorável como anteriormente; porém o Deus de meu pai tem estado comigo.
\par 6 Vós mesmas sabeis que com todo empenho tenho servido a vosso pai;
\par 7 mas vosso pai me tem enganado e por dez vezes me mudou o salário; porém Deus não lhe permitiu que me fizesse mal nenhum.
\par 8 Se ele dizia: Os salpicados serão o teu salário, então, todos os rebanhos davam salpicados; e se dizia: Os listados serão o teu salário, então, os rebanhos todos davam listados.
\par 9 Assim, Deus tomou o gado de vosso pai e mo deu a mim.
\par 10 Pois, chegado o tempo em que o rebanho concebia, levantei os olhos e vi em sonhos que os machos que cobriam as ovelhas eram listados, salpicados e malhados.
\par 11 E o Anjo de Deus me disse em sonho: Jacó! Eu respondi: Eis-me aqui!
\par 12 Ele continuou: Levanta agora os olhos e vê que todos os machos que cobrem o rebanho são listados, salpicados e malhados, porque vejo tudo o que Labão te está fazendo.
\par 13 Eu sou o Deus de Betel, onde ungiste uma coluna, onde me fizeste um voto; levanta-te agora, sai desta terra e volta para a terra de tua parentela.
\par 14 Então, responderam Raquel e Lia e lhe disseram: Há ainda para nós parte ou herança na casa de nosso pai?
\par 15 Não nos considera ele como estrangeiras? Pois nos vendeu e consumiu tudo o que nos era devido.
\par 16 Porque toda a riqueza que Deus tirou de nosso pai é nossa e de nossos filhos; agora, pois, faze tudo o que Deus te disse.
\par 17 Então, se levantou Jacó e, fazendo montar seus filhos e suas mulheres em camelos,
\par 18 levou todo o seu gado e todos os seus bens que chegou a possuir; o gado de sua propriedade que acumulara em Padã-Arã, para ir a Isaque, seu pai, à terra de Canaã.
\par 19 Tendo ido Labão fazer a tosquia das ovelhas, Raquel furtou os ídolos do lar que pertenciam a seu pai.
\par 20 E Jacó logrou a Labão, o arameu, não lhe dando a saber que fugia.
\par 21 E fugiu com tudo o que lhe pertencia; levantou-se, passou o Eufrates e tomou o rumo da montanha de Gileade.
\par 22 No terceiro dia, Labão foi avisado de que Jacó ia fugindo.
\par 23 Tomando, pois, consigo a seus irmãos, saiu-lhe no encalço, por sete dias de jornada, e o alcançou na montanha de Gileade.
\par 24 De noite, porém, veio Deus a Labão, o arameu, em sonhos, e lhe disse: Guarda-te, não fales a Jacó nem bem nem mal.
\par 25 Alcançou, pois, Labão a Jacó. Este havia armado a sua tenda naquela montanha; também Labão armou a sua com seus irmãos, na montanha de Gileade.
\par 26 E disse Labão a Jacó: Que fizeste, que me lograste e levaste minhas filhas como cativas pela espada?
\par 27 Por que fugiste ocultamente, e me lograste, e nada me fizeste saber, para que eu te despedisse com alegria, e com cânticos, e com tamboril, e com harpa?
\par 28 E por que não me permitiste beijar meus filhos e minhas filhas? Nisso procedeste insensatamente.
\par 29 Há poder em minhas mãos para vos fazer mal, mas o Deus de vosso pai me falou, ontem à noite, e disse: Guarda-te, não fales a Jacó nem bem nem mal.
\par 30 E agora que partiste de vez, porque tens saudade da casa de teu pai, por que me furtaste os meus deuses?
\par 31 Respondeu-lhe Jacó: Porque tive medo; pois calculei: não suceda que me tome à força as suas filhas.
\par 32 Não viva aquele com quem achares os teus deuses; verifica diante de nossos irmãos o que te pertence e que está comigo e leva-o contigo. Pois Jacó não sabia que Raquel os havia furtado.
\par 33 Labão, pois, entrou na tenda de Jacó, na de Lia e na das duas servas, porém não os achou. Tendo saído da tenda de Lia, entrou na de Raquel.
\par 34 Ora, Raquel havia tomado os ídolos do lar, e os pusera na sela de um camelo, e estava assentada sobre eles; apalpou Labão toda a tenda e não os achou.
\par 35 Então, disse ela a seu pai: Não te agastes, meu senhor, por não poder eu levantar-me na tua presença; pois me acho com as regras das mulheres. Ele procurou, contudo não achou os ídolos do lar.
\par 36 Então, se irou Jacó e altercou com Labão; e lhe disse: Qual é a minha transgressão? Qual o meu pecado, que tão furiosamente me tens perseguido?
\par 37 Havendo apalpado todos os meus utensílios, que achaste de todos os utensílios de tua casa? Põe-nos aqui diante de meus irmãos e de teus irmãos, para que julguem entre mim e ti.
\par 38 Vinte anos eu estive contigo, as tuas ovelhas e as tuas cabras nunca perderam as crias, e não comi os carneiros de teu rebanho.
\par 39 Nem te apresentei o que era despedaçado pelas feras; sofri o dano; da minha mão o requerias, tanto o furtado de dia como de noite.
\par 40 De maneira que eu andava, de dia consumido pelo calor, de noite, pela geada; e o meu sono me fugia dos olhos.
\par 41 Vinte anos permaneci em tua casa; catorze anos te servi por tuas duas filhas e seis anos por teu rebanho; dez vezes me mudaste o salário.
\par 42 Se não fora o Deus de meu pai, o Deus de Abraão e o Temor de Isaque, por certo me despedirias agora de mãos vazias. Deus me atendeu ao sofrimento e ao trabalho das minhas mãos e te repreendeu ontem à noite.
\par 43 Então, respondeu Labão a Jacó: As filhas são minhas filhas, os filhos são meus filhos, os rebanhos são meus rebanhos, e tudo o que vês é meu; que posso fazer hoje a estas minhas filhas ou aos filhos que elas deram à luz?
\par 44 Vem, pois; e façamos aliança, eu e tu, que sirva de testemunho entre mim e ti.
\par 45 Então, Jacó tomou uma pedra e a erigiu por coluna.
\par 46 E disse a seus irmãos: Ajuntai pedras. E tomaram pedras e fizeram um montão, ao lado do qual comeram.
\par 47 Chamou-lhe Labão Jegar-Saaduta; Jacó, porém, lhe chamou Galeede.
\par 48 E disse Labão: Seja hoje este montão por testemunha entre mim e ti; por isso, se lhe chamou Galeede
\par 49 e Mispa, pois disse: Vigie o SENHOR entre mim e ti e nos julgue quando estivermos separados um do outro.
\par 50 Se maltratares as minhas filhas e tomares outras mulheres além delas, não estando ninguém conosco, atenta que Deus é testemunha entre mim e ti.
\par 51 Disse mais Labão a Jacó: Eis aqui este montão e esta coluna que levantei entre mim e ti.
\par 52 Seja o montão testemunha, e seja a coluna testemunha de que para mal não passarei o montão para lá, e tu não passarás o montão e a coluna para cá.
\par 53 O Deus de Abraão e o Deus de Naor, o Deus do pai deles, julgue entre nós. E jurou Jacó pelo Temor de Isaque, seu pai.
\par 54 E ofereceu Jacó um sacrifício na montanha e convidou seus irmãos para comerem pão; comeram pão e passaram a noite na montanha.
\par 55 Tendo-se levantado Labão pela madrugada, beijou seus filhos e suas filhas e os abençoou; e, partindo, voltou para sua casa.

\chapter{32}

\par 1 Também Jacó seguiu o seu caminho, e anjos de Deus lhe saíram a encontrá-lo.
\par 2 Quando os viu, disse: Este é o acampamento de Deus. E chamou àquele lugar Maanaim.
\par 3 Então, Jacó enviou mensageiros adiante de si a Esaú, seu irmão, à terra de Seir, território de Edom,
\par 4 e lhes ordenou: Assim falareis a meu senhor Esaú: Teu servo Jacó manda dizer isto: Como peregrino morei com Labão, em cuja companhia fiquei até agora.
\par 5 Tenho bois, jumentos, rebanhos, servos e servas; mando comunicá-lo a meu senhor, para lograr mercê à sua presença.
\par 6 Voltaram os mensageiros a Jacó, dizendo: Fomos a teu irmão Esaú; também ele vem de caminho para se encontrar contigo, e quatrocentos homens com ele.
\par 7 Então, Jacó teve medo e se perturbou; dividiu em dois bandos o povo que com ele estava, e os rebanhos, e os bois, e os camelos.
\par 8 Pois disse: Se vier Esaú a um bando e o ferir, o outro bando escapará.
\par 9 E orou Jacó: Deus de meu pai Abraão e Deus de meu pai Isaque, ó SENHOR, que me disseste: Torna à tua terra e à tua parentela, e te farei bem;
\par 10 sou indigno de todas as misericórdias e de toda a fidelidade que tens usado para com teu servo; pois com apenas o meu cajado atravessei este Jordão; já agora sou dois bandos.
\par 11 Livra-me das mãos de meu irmão Esaú, porque eu o temo, para que não venha ele matar-me e as mães com os filhos.
\par 12 E disseste: Certamente eu te farei bem e dar-te-ei a descendência como a areia do mar, que, pela multidão, não se pode contar.
\par 13 E, tendo passado ali aquela noite, separou do que tinha um presente para seu irmão Esaú:
\par 14 duzentas cabras e vinte bodes; duzentas ovelhas e vinte carneiros;
\par 15 trinta camelas de leite com suas crias, quarenta vacas e dez touros; vinte jumentas e dez jumentinhos.
\par 16 Entregou-os às mãos dos seus servos, cada rebanho à parte, e disse aos servos: Passai adiante de mim e deixai espaço entre rebanho e rebanho.
\par 17 Ordenou ao primeiro, dizendo: Quando Esaú, meu irmão, te encontrar e te perguntar: De quem és, para onde vais, de quem são estes diante de ti?
\par 18 Responderás: São de teu servo Jacó; é presente que ele envia a meu senhor Esaú; e eis que ele mesmo vem vindo atrás de nós.
\par 19 Ordenou também ao segundo, ao terceiro e a todos os que vinham conduzindo os rebanhos: Falareis desta maneira a Esaú, quando vos encontrardes com ele.
\par 20 Direis assim: Eis que o teu servo Jacó vem vindo atrás de nós. Porque dizia consigo mesmo: Eu o aplacarei com o presente que me antecede, depois o verei; porventura me aceitará a presença.
\par 21 Assim, passou o presente para diante dele; ele, porém, ficou aquela noite no acampamento.
\par 22 Levantou-se naquela mesma noite, tomou suas duas mulheres, suas duas servas e seus onze filhos e transpôs o vau de Jaboque.
\par 23 Tomou-os e fê-los passar o ribeiro; fez passar tudo o que lhe pertencia,
\par 24 ficando ele só; e lutava com ele um homem, até ao romper do dia.
\par 25 Vendo este que não podia com ele, tocou-lhe na articulação da coxa; deslocou-se a junta da coxa de Jacó, na luta com o homem.
\par 26 Disse este: Deixa-me ir, pois já rompeu o dia. Respondeu Jacó: Não te deixarei ir se me não abençoares.
\par 27 Perguntou-lhe, pois: Como te chamas? Ele respondeu: Jacó.
\par 28 Então, disse: Já não te chamarás Jacó, e sim Israel, pois como príncipe lutaste com Deus e com os homens e prevaleceste.
\par 29 Tornou Jacó: Dize, rogo-te, como te chamas? Respondeu ele: Por que perguntas pelo meu nome? E o abençoou ali.
\par 30 Àquele lugar chamou Jacó Peniel, pois disse: Vi a Deus face a face, e a minha vida foi salva.
\par 31 Nasceu-lhe o sol, quando ele atravessava Peniel; e manquejava de uma coxa.
\par 32 Por isso, os filhos de Israel não comem, até hoje, o nervo do quadril, na articulação da coxa, porque o homem tocou a articulação da coxa de Jacó no nervo do quadril.

\chapter{33}

\par 1 Levantando Jacó os olhos, viu que Esaú se aproximava, e com ele quatrocentos homens. Então, passou os filhos a Lia, a Raquel e às duas servas.
\par 2 Pôs as servas e seus filhos à frente, Lia e seus filhos atrás deles e Raquel e José por últimos.
\par 3 E ele mesmo, adiantando-se, prostrou-se à terra sete vezes, até aproximar-se de seu irmão.
\par 4 Então, Esaú correu-lhe ao encontro e o abraçou; arrojou-se-lhe ao pescoço e o beijou; e choraram.
\par 5 Daí, levantando os olhos, viu as mulheres e os meninos e disse: Quem são estes contigo? Respondeu-lhe Jacó: Os filhos com que Deus agraciou a teu servo.
\par 6 Então, se aproximaram as servas, elas e seus filhos, e se prostraram.
\par 7 Chegaram também Lia e seus filhos e se prostraram; por último chegaram José e Raquel e se prostraram.
\par 8 Perguntou Esaú: Qual é o teu propósito com todos esses bandos que encontrei? Respondeu Jacó: Para lograr mercê na presença de meu senhor.
\par 9 Então, disse Esaú: Eu tenho muitos bens, meu irmão; guarda o que tens.
\par 10 Mas Jacó insistiu: Não recuses; se logrei mercê diante de ti, peço-te que aceites o meu presente, porquanto vi o teu rosto como se tivesse contemplado o semblante de Deus; e te agradaste de mim.
\par 11 Peço-te, pois, recebe o meu presente, que eu te trouxe; porque Deus tem sido generoso para comigo, e tenho fartura. E instou com ele, até que o aceitou.
\par 12 Disse Esaú: Partamos e caminhemos; eu seguirei junto de ti.
\par 13 Porém Jacó lhe disse: Meu senhor sabe que estes meninos são tenros, e tenho comigo ovelhas e vacas de leite; se forçadas a caminhar demais um só dia, morrerão todos os rebanhos.
\par 14 Passe meu senhor adiante de seu servo; eu seguirei guiando-as pouco a pouco, no passo do gado que me vai à frente e no passo dos meninos, até chegar a meu senhor, em Seir.
\par 15 Respondeu Esaú: Então, permite que eu deixe contigo da gente que está comigo. Disse Jacó: Para quê? Basta que eu alcance mercê aos olhos de meu senhor.
\par 16 Assim, voltou Esaú aquele dia a Seir, pelo caminho por onde viera.
\par 17 E Jacó partiu para Sucote, e edificou para si uma casa, e fez palhoças para o seu gado; por isso, o lugar se chamou Sucote.
\par 18 Voltando de Padã-Arã, chegou Jacó são e salvo à cidade de Siquém, que está na terra de Canaã; e armou a sua tenda junto da cidade.
\par 19 A parte do campo, onde armara a sua tenda, ele a comprou dos filhos de Hamor, pai de Siquém, por cem peças de dinheiro.
\par 20 E levantou ali um altar e lhe chamou Deus, o Deus de Israel.

\chapter{34}

\par 1 Ora, Diná, filha que Lia dera à luz a Jacó, saiu para ver as filhas da terra.
\par 2 Viu-a Siquém, filho do heveu Hamor, que era príncipe daquela terra, e, tomando-a, a possuiu e assim a humilhou.
\par 3 Sua alma se apegou a Diná, filha de Jacó, e amou a jovem, e falou-lhe ao coração.
\par 4 Então, disse Siquém a Hamor, seu pai: Consegue-me esta jovem para esposa.
\par 5 Quando soube Jacó que Diná, sua filha, fora violada por Siquém, estavam os seus filhos no campo com o gado; calou-se, pois, até que voltassem.
\par 6 E saiu Hamor, pai de Siquém, para falar com Jacó.
\par 7 Vindo os filhos de Jacó do campo e ouvindo o que acontecera, indignaram-se e muito se iraram, pois Siquém praticara um desatino em Israel, violentando a filha de Jacó, o que se não devia fazer.
\par 8 Disse-lhes Hamor: A alma de meu filho Siquém está enamorada fortemente de vossa filha; peço-vos que lha deis por esposa.
\par 9 Aparentai-vos conosco, dai-nos as vossas filhas e tomai as nossas;
\par 10 habitareis conosco, a terra estará ao vosso dispor; habitai e negociai nela e nela tende possessões.
\par 11 E o próprio Siquém disse ao pai e aos irmãos de Diná: Ache eu mercê diante de vós e vos darei o que determinardes.
\par 12 Majorai de muito o dote de casamento e as dádivas, e darei o que me pedirdes; dai-me, porém, a jovem por esposa.
\par 13 Então, os filhos de Jacó, por causa de lhes haver Siquém violado a irmã, Diná, responderam com dolo a Siquém e a seu pai Hamor e lhes disseram:
\par 14 Não podemos fazer isso, dar nossa irmã a um homem incircunciso; porque isso nos seria ignomínia.
\par 15 Sob uma única condição permitiremos: que vos torneis como nós, circuncidando-se todo macho entre vós;
\par 16 então, vos daremos nossas filhas, tomaremos para nós as vossas, habitaremos convosco e seremos um só povo.
\par 17 Se, porém, não nos ouvirdes e não vos circuncidardes, tomaremos a nossa filha e nos retiraremos embora.
\par 18 Tais palavras agradaram a Hamor e a Siquém, seu filho.
\par 19 Não tardou o jovem em fazer isso, porque amava a filha de Jacó e era o mais honrado de toda a casa de seu pai.
\par 20 Vieram, pois, Hamor e Siquém, seu filho, à porta da sua cidade e falaram aos homens da cidade:
\par 21 Estes homens são pacíficos para conosco; portanto, habitem na terra e negociem nela. A terra é bastante espaçosa para contê-los; recebamos por mulheres a suas filhas e demos-lhes também as nossas.
\par 22 Somente, porém, consentirão os homens em habitar conosco, tornando-nos um só povo, se todo macho entre nós se circuncidar, como eles são circuncidados.
\par 23 O seu gado, as suas possessões e todos os seus animais não serão nossos? Consintamos, pois, com eles, e habitarão conosco.
\par 24 E deram ouvidos a Hamor e a Siquém, seu filho, todos os que saíam da porta da cidade; e todo homem foi circuncidado, dos que saíam pela porta da sua cidade.
\par 25 Ao terceiro dia, quando os homens sentiam mais forte a dor, dois filhos de Jacó, Simeão e Levi, irmãos de Diná, tomaram cada um a sua espada, entraram inesperadamente na cidade e mataram os homens todos.
\par 26 Passaram também ao fio da espada a Hamor e a seu filho Siquém; tomaram a Diná da casa de Siquém e saíram.
\par 27 Sobrevieram os filhos de Jacó aos mortos e saquearam a cidade, porque sua irmã fora violada.
\par 28 Levaram deles os rebanhos, os bois, os jumentos e o que havia na cidade e no campo;
\par 29 todos os seus bens, e todos os seus meninos, e as suas mulheres levaram cativos e pilharam tudo o que havia nas casas.
\par 30 Então, disse Jacó a Simeão e a Levi: Vós me afligistes e me fizestes odioso entre os moradores desta terra, entre os cananeus e os ferezeus; sendo nós pouca gente, reunir-se-ão contra mim, e serei destruído, eu e minha casa.
\par 31 Responderam: Abusaria ele de nossa irmã, como se fosse prostituta?

\chapter{35}

\par 1 Disse Deus a Jacó: Levanta-te, sobe a Betel e habita ali; faze ali um altar ao Deus que te apareceu quando fugias da presença de Esaú, teu irmão.
\par 2 Então, disse Jacó à sua família e a todos os que com ele estavam: Lançai fora os deuses estranhos que há no vosso meio, purificai-vos e mudai as vossas vestes;
\par 3 levantemo-nos e subamos a Betel. Farei ali um altar ao Deus que me respondeu no dia da minha angústia e me acompanhou no caminho por onde andei.
\par 4 Então, deram a Jacó todos os deuses estrangeiros que tinham em mãos e as argolas que lhes pendiam das orelhas; e Jacó os escondeu debaixo do carvalho que está junto a Siquém.
\par 5 E, tendo eles partido, o terror de Deus invadiu as cidades que lhes eram circunvizinhas, e não perseguiram aos filhos de Jacó.
\par 6 Assim, chegou Jacó a Luz, chamada Betel, que está na terra de Canaã, ele e todo o povo que com ele estava.
\par 7 E edificou ali um altar e ao lugar chamou El-Betel; porque ali Deus se lhe revelou quando fugia da presença de seu irmão.
\par 8 Morreu Débora, a ama de Rebeca, e foi sepultada ao pé de Betel, debaixo do carvalho que se chama Alom-Bacute.
\par 9 Vindo Jacó de Padã-Arã, outra vez lhe apareceu Deus e o abençoou.
\par 10 Disse-lhe Deus: O teu nome é Jacó. Já não te chamarás Jacó, porém Israel será o teu nome. E lhe chamou Israel.
\par 11 Disse-lhe mais: Eu sou o Deus Todo-Poderoso; sê fecundo e multiplica-te; uma nação e multidão de nações sairão de ti, e reis procederão de ti.
\par 12 A terra que dei a Abraão e a Isaque dar-te-ei a ti e, depois de ti, à tua descendência.
\par 13 E Deus se retirou dele, elevando-se do lugar onde lhe falara.
\par 14 Então, Jacó erigiu uma coluna de pedra no lugar onde Deus falara com ele; e derramou sobre ela uma libação e lhe deitou óleo.
\par 15 Ao lugar onde Deus lhe falara, Jacó lhe chamou Betel.
\par 16 Partiram de Betel, e, havendo ainda pequena distância para chegar a Efrata, deu à luz Raquel um filho, cujo nascimento lhe foi a ela penoso.
\par 17 Em meio às dores do parto, disse-lhe a parteira: Não temas, pois ainda terás este filho.
\par 18 Ao sair-lhe a alma (porque morreu), deu-lhe o nome de Benoni; mas seu pai lhe chamou Benjamim.
\par 19 Assim, morreu Raquel e foi sepultada no caminho de Efrata, que é Belém.
\par 20 Sobre a sepultura de Raquel levantou Jacó uma coluna que existe até ao dia de hoje.
\par 21 Então, partiu Israel e armou a sua tenda além da torre de Éder.
\par 22 E aconteceu que, habitando Israel naquela terra, foi Rúben e se deitou com Bila, concubina de seu pai; e Israel o soube. Eram doze os filhos de Israel.
\par 23 Rúben, o primogênito de Jacó, Simeão, Levi, Judá, Issacar e Zebulom, filhos de Lia;
\par 24 José e Benjamim, filhos de Raquel;
\par 25 Dã e Naftali, filhos de Bila, serva de Raquel;
\par 26 e Gade e Aser, filhos de Zilpa, serva de Lia. São estes os filhos de Jacó, que lhe nasceram em Padã-Arã.
\par 27 Veio Jacó a Isaque, seu pai, a Manre, a Quiriate-Arba (que é Hebrom), onde peregrinaram Abraão e Isaque.
\par 28 Foram os dias de Isaque cento e oitenta anos.
\par 29 Velho e farto de dias, expirou Isaque e morreu, sendo recolhido ao seu povo; e Esaú e Jacó, seus filhos, o sepultaram.

\chapter{36}

\par 1 São estes os descendentes de Esaú, que é Edom.
\par 2 Esaú tomou por mulheres dentre as filhas de Canaã: Ada, filha de Elom, heteu; Oolibama, filha de Aná, filho de Zibeão, heveu;
\par 3 e Basemate, filha de Ismael, irmã de Nebaiote.
\par 4 A Ada de Esaú lhe nasceu Elifaz, a Basemate lhe nasceu Reuel;
\par 5 e a Oolibama nasceu Jeús, Jalão e Corá; são estes os filhos de Esaú, que lhe nasceram na terra de Canaã.
\par 6 Levou Esaú suas mulheres, e seus filhos, e suas filhas, e todas as pessoas de sua casa, e seu rebanho, e todo o seu gado, e toda propriedade, tudo que havia adquirido na terra de Canaã; e se foi para outra terra, apartando-se de Jacó, seu irmão.
\par 7 Porque os bens deles eram muitos para habitarem juntos; e a terra de suas peregrinações não os podia sustentar por causa do seu gado.
\par 8 Então, Esaú, que é Edom, habitou no monte Seir.
\par 9 Esta é a descendência de Esaú, pai dos edomitas, no monte Seir.
\par 10 São estes os nomes dos filhos de Esaú: Elifaz, filho de Ada, mulher de Esaú; Reuel, filho de Basemate, mulher de Esaú.
\par 11 Os filhos de Elifaz são: Temã, Omar, Zefô, Gaetã e Quenaz.
\par 12 Timna era concubina de Elifaz, filho de Esaú, e teve de Elifaz a Amaleque; são estes os filhos de Ada, mulher de Esaú.
\par 13 E os filhos de Reuel são estes: Naate, Zerá, Samá e Mizá; estes foram os filhos de Basemate, mulher de Esaú.
\par 14 E são estes os filhos de Oolibama, filha de Aná, filho de Zibeão, mulher de Esaú; e deu a Esaú: Jeús, Jalão e Corá.
\par 15 São estes os príncipes dos filhos de Esaú; os filhos de Elifaz, o primogênito de Esaú: o príncipe Temã, o príncipe Omar, o príncipe Zefô, o príncipe Quenaz,
\par 16 o príncipe Corá, o príncipe Gaetã, o príncipe Amaleque; são estes os príncipes que nasceram a Elifaz na terra de Edom; são os filhos de Ada.
\par 17 São estes os filhos de Reuel, filho de Esaú: o príncipe Naate, o príncipe Zerá, o príncipe Samá, o príncipe Mizá; são estes os príncipes que nasceram a Reuel na terra de Edom; são os filhos de Basemate, mulher de Esaú.
\par 18 São estes os filhos de Oolibama, mulher de Esaú: o príncipe Jeús, o príncipe Jalão, o príncipe Corá; são estes os príncipes que procederam de Oolibama, filha de Aná, mulher de Esaú.
\par 19 São estes os filhos de Esaú, e esses seus príncipes; ele é Edom.
\par 20 São estes os filhos de Seir, o horeu, moradores da terra: Lotã, Sobal, Zibeão e Aná,
\par 21 Disom, Eser e Disã; são estes os príncipes dos horeus, filhos de Seir na terra de Edom.
\par 22 Os filhos de Lotã são Hori e Homã; a irmã de Lotã é Timna.
\par 23 São estes os filhos de Sobal: Alvã, Manaate, Ebal, Sefô e Onã.
\par 24 São estes os filhos de Zibeão: Aiá e Aná; este é o Aná que achou as fontes termais no deserto, quando apascentava os jumentos de Zibeão, seu pai.
\par 25 São estes os filhos de Aná: Disom e Oolibama, a filha de Aná.
\par 26 São estes os filhos de Disã: Hendã, Esbã, Itrã e Querã.
\par 27 São estes os filhos de Eser: Bilã, Zaavã e Acã.
\par 28 São estes os filhos de Disã: Uz e Arã.
\par 29 São estes os príncipes dos horeus: o príncipe Lotã, o príncipe Sobal, o príncipe Zibeão, o príncipe Aná,
\par 30 o príncipe Disom, o príncipe Eser, o príncipe Disã; são estes os príncipes dos horeus, segundo os seus principados na terra de Seir.
\par 31 São estes os reis que reinaram na terra de Edom, antes que houvesse rei sobre os filhos de Israel.
\par 32 Em Edom reinou Belá, filho de Beor, e o nome da sua cidade era Dinabá.
\par 33 Morreu Belá, e, em seu lugar, reinou Jobabe, filho de Zerá, de Bozra.
\par 34 Morreu Jobabe, e, em seu lugar, reinou Husão, da terra dos temanitas.
\par 35 Morreu Husão, e, em seu lugar, reinou Hadade, filho de Bedade, o que feriu a Midiã no campo de Moabe; o nome da sua cidade era Avite.
\par 36 Morreu Hadade, e, em seu lugar, reinou Samlá, de Masreca.
\par 37 Morreu Samlá, e, em seu lugar, reinou Saul, de Reobote, junto ao Eufrates.
\par 38 Morreu Saul, e, em seu lugar, reinou Baal-Hanã, filho de Acbor.
\par 39 Morreu Baal-Hanã, filho de Acbor, e, em seu lugar, reinou Hadar; o nome de sua cidade era Paú; e o de sua mulher era Meetabel, filha de Matrede, filha de Me-Zaabe.
\par 40 São estes os nomes dos príncipes de Esaú, segundo as suas famílias, os seus lugares e os seus nomes: o príncipe Timna, o príncipe Alva, o príncipe Jetete,
\par 41 o príncipe Oolibama, o príncipe Elá, o príncipe Pinom,
\par 42 o príncipe Quenaz, o príncipe Temã, o príncipe Mibzar,
\par 43 o príncipe Magdiel e o príncipe Irã; são estes os príncipes de Edom, segundo as suas habitações na terra da sua possessão. Este é Esaú, pai de Edom.

\chapter{37}

\par 1 Habitou Jacó na terra das peregrinações de seu pai, na terra de Canaã.
\par 2 Esta é a história de Jacó. Tendo José dezessete anos, apascentava os rebanhos com seus irmãos; sendo ainda jovem, acompanhava os filhos de Bila e os filhos de Zilpa, mulheres de seu pai; e trazia más notícias deles a seu pai.
\par 3 Ora, Israel amava mais a José que a todos os seus filhos, porque era filho da sua velhice; e fez-lhe uma túnica talar de mangas compridas.
\par 4 Vendo, pois, seus irmãos que o pai o amava mais que a todos os outros filhos, odiaram-no e já não lhe podiam falar pacificamente.
\par 5 Teve José um sonho e o relatou a seus irmãos; por isso, o odiaram ainda mais.
\par 6 Pois lhes disse: Rogo-vos, ouvi este sonho que tive:
\par 7 Atávamos feixes no campo, e eis que o meu feixe se levantou e ficou em pé; e os vossos feixes o rodeavam e se inclinavam perante o meu.
\par 8 Então, lhe disseram seus irmãos: Reinarás, com efeito, sobre nós? E sobre nós dominarás realmente? E com isso tanto mais o odiavam, por causa dos seus sonhos e de suas palavras.
\par 9 Teve ainda outro sonho e o referiu a seus irmãos, dizendo: Sonhei também que o sol, a lua e onze estrelas se inclinavam perante mim.
\par 10 Contando-o a seu pai e a seus irmãos, repreendeu-o o pai e lhe disse: Que sonho é esse que tiveste? Acaso, viremos, eu e tua mãe e teus irmãos, a inclinar-nos perante ti em terra?
\par 11 Seus irmãos lhe tinham ciúmes; o pai, no entanto, considerava o caso consigo mesmo.
\par 12 E, como foram os irmãos apascentar o rebanho do pai, em Siquém,
\par 13 perguntou Israel a José: Não apascentam teus irmãos o rebanho em Siquém? Vem, enviar-te-ei a eles. Respondeu-lhe José: Eis-me aqui.
\par 14 Disse-lhe Israel: Vai, agora, e vê se vão bem teus irmãos e o rebanho; e traze-me notícias. Assim, o enviou do vale de Hebrom, e ele foi a Siquém.
\par 15 E um homem encontrou a José, que andava errante pelo campo, e lhe perguntou: Que procuras?
\par 16 Respondeu: Procuro meus irmãos; dize-me: Onde apascentam eles o rebanho?
\par 17 Disse-lhe o homem: Foram-se daqui, pois ouvi-os dizer: Vamos a Dotã. Então, seguiu José atrás dos irmãos e os achou em Dotã.
\par 18 De longe o viram e, antes que chegasse, conspiraram contra ele para o matar.
\par 19 E dizia um ao outro: Vem lá o tal sonhador!
\par 20 Vinde, pois, agora, matemo-lo e lancemo-lo numa destas cisternas; e diremos: Um animal selvagem o comeu; e vejamos em que lhe darão os sonhos.
\par 21 Mas Rúben, ouvindo isso, livrou-o das mãos deles e disse: Não lhe tiremos a vida.
\par 22 Também lhes disse Rúben: Não derrameis sangue; lançai-o nesta cisterna que está no deserto, e não ponhais mão sobre ele; isto disse para o livrar deles, a fim de o restituir ao pai.
\par 23 Mas, logo que chegou José a seus irmãos, despiram-no da túnica, a túnica talar de mangas compridas que trazia.
\par 24 E, tomando-o, o lançaram na cisterna, vazia, sem água.
\par 25 Ora, sentando-se para comer pão, olharam e viram que uma caravana de ismaelitas vinha de Gileade; seus camelos traziam arômatas, bálsamo e mirra, que levavam para o Egito.
\par 26 Então, disse Judá a seus irmãos: De que nos aproveita matar o nosso irmão e esconder-lhe o sangue?
\par 27 Vinde, vendamo-lo aos ismaelitas; não ponhamos sobre ele a mão, pois é nosso irmão e nossa carne. Seus irmãos concordaram.
\par 28 E, passando os mercadores midianitas, os irmãos de José o alçaram, e o tiraram da cisterna, e o venderam por vinte siclos de prata aos ismaelitas; estes levaram José ao Egito.
\par 29 Tendo Rúben voltado à cisterna, eis que José não estava nela; então, rasgou as suas vestes.
\par 30 E, voltando a seus irmãos, disse: Não está lá o menino; e, eu, para onde irei?
\par 31 Então, tomaram a túnica de José, mataram um bode e a molharam no sangue.
\par 32 E enviaram a túnica talar de mangas compridas, fizeram-na levar a seu pai e lhe disseram: Achamos isto; vê se é ou não a túnica de teu filho.
\par 33 Ele a reconheceu e disse: É a túnica de meu filho; um animal selvagem o terá comido, certamente José foi despedaçado.
\par 34 Então, Jacó rasgou as suas vestes, e se cingiu de pano de saco, e lamentou o filho por muitos dias.
\par 35 Levantaram-se todos os seus filhos e todas as suas filhas, para o consolarem; ele, porém, recusou ser consolado e disse: Chorando, descerei a meu filho até à sepultura. E de fato o chorou seu pai.
\par 36 Entrementes, os midianitas venderam José no Egito a Potifar, oficial de Faraó, comandante da guarda.

\chapter{38}

\par 1 Aconteceu, por esse tempo, que Judá se apartou de seus irmãos e se hospedou na casa de um adulamita, chamado Hira.
\par 2 Ali viu Judá a filha de um cananeu, chamado Sua; ele a tomou por mulher e a possuiu.
\par 3 E ela concebeu e deu à luz um filho, e o pai lhe chamou Er.
\par 4 Tornou a conceber e deu à luz um filho; a este deu a mãe o nome de Onã.
\par 5 Continuou ainda e deu à luz outro filho, cujo nome foi Selá; ela estava em Quezibe quando o teve.
\par 6 Judá, pois, tomou esposa para Er, o seu primogênito; o nome dela era Tamar.
\par 7 Er, porém, o primogênito de Judá, era perverso perante o SENHOR, pelo que o SENHOR o fez morrer.
\par 8 Então, disse Judá a Onã: Possui a mulher de teu irmão, cumpre o levirato e suscita descendência a teu irmão.
\par 9 Sabia, porém, Onã que o filho não seria tido por seu; e todas as vezes que possuía a mulher de seu irmão deixava o sêmen cair na terra, para não dar descendência a seu irmão.
\par 10 Isso, porém, que fazia, era mau perante o SENHOR, pelo que também a este fez morrer.
\par 11 Então, disse Judá a Tamar, sua nora: Permanece viúva em casa de teu pai, até que Selá, meu filho, venha a ser homem. Pois disse: Para que não morra também este, como seus irmãos. Assim, Tamar se foi, passando a residir em casa de seu pai.
\par 12 No correr do tempo morreu a filha de Sua, mulher de Judá; e, consolado Judá, subiu aos tosquiadores de suas ovelhas, em Timna, ele e seu amigo Hira, o adulamita.
\par 13 E o comunicaram a Tamar: Eis que o teu sogro sobe a Timna, para tosquiar as ovelhas.
\par 14 Então, ela despiu as vestes de sua viuvez, e, cobrindo-se com um véu, se disfarçou, e se assentou à entrada de Enaim, no caminho de Timna; pois via que Selá já era homem, e ela não lhe fora dada por mulher.
\par 15 Vendo-a Judá, teve-a por meretriz; pois ela havia coberto o rosto.
\par 16 Então, se dirigiu a ela no caminho e lhe disse: Vem, deixa-me possuir-te; porque não sabia que era a sua nora. Ela respondeu: Que me darás para coabitares comigo?
\par 17 Ele respondeu: Enviar-te-ei um cabrito do rebanho. Perguntou ela: Dar-me-ás penhor até que o mandes?
\par 18 Respondeu ele: Que penhor te darei? Ela disse: O teu selo, o teu cordão e o cajado que seguras. Ele, pois, lhos deu e a possuiu; e ela concebeu dele.
\par 19 Levantou-se ela e se foi; tirou de sobre si o véu e tornou às vestes da sua viuvez.
\par 20 Enviou Judá o cabrito, por mão do adulamita, seu amigo, para reaver o penhor da mão da mulher; porém não a encontrou.
\par 21 Então, perguntou aos homens daquele lugar: Onde está a prostituta cultual que se achava junto ao caminho de Enaim? Responderam: Aqui não esteve meretriz nenhuma.
\par 22 Tendo voltado a Judá, disse: Não a encontrei; e também os homens do lugar me disseram: Aqui não esteve prostituta cultual nenhuma.
\par 23 Respondeu Judá: Que ela o guarde para si, para que não nos tornemos em opróbrio; mandei-lhe, com efeito, o cabrito, todavia, não a achaste.
\par 24 Passados quase três meses, foi dito a Judá: Tamar, tua nora, adulterou, pois está grávida. Então, disse Judá: Tirai-a fora para que seja queimada.
\par 25 Em tirando-a, mandou ela dizer a seu sogro: Do homem de quem são estas coisas eu concebi. E disse mais: Reconhece de quem é este selo, e este cordão, e este cajado.
\par 26 Reconheceu-os Judá e disse: Mais justa é ela do que eu, porquanto não a dei a Selá, meu filho. E nunca mais a possuiu.
\par 27 E aconteceu que, estando ela para dar à luz, havia gêmeos no seu ventre.
\par 28 Ao nascerem, um pôs a mão fora, e a parteira, tomando-a, lhe atou um fio encarnado e disse: Este saiu primeiro.
\par 29 Mas, recolhendo ele a mão, saiu o outro; e ela disse: Como rompeste saída? E lhe chamaram Perez.
\par 30 Depois, lhe saiu o irmão, em cuja mão estava o fio encarnado; e lhe chamaram Zera.

\chapter{39}

\par 1 José foi levado ao Egito, e Potifar, oficial de Faraó, comandante da guarda, egípcio, comprou-o dos ismaelitas que o tinham levado para lá.
\par 2 O SENHOR era com José, que veio a ser homem próspero; e estava na casa de seu senhor egípcio.
\par 3 Vendo Potifar que o SENHOR era com ele e que tudo o que ele fazia o SENHOR prosperava em suas mãos,
\par 4 logrou José mercê perante ele, a quem servia; e ele o pôs por mordomo de sua casa e lhe passou às mãos tudo o que tinha.
\par 5 E, desde que o fizera mordomo de sua casa e sobre tudo o que tinha, o SENHOR abençoou a casa do egípcio por amor de José; a bênção do SENHOR estava sobre tudo o que tinha, tanto em casa como no campo.
\par 6 Potifar tudo o que tinha confiou às mãos de José, de maneira que, tendo-o por mordomo, de nada sabia, além do pão com que se alimentava. José era formoso de porte e de aparência.
\par 7 Aconteceu, depois destas coisas, que a mulher de seu senhor pôs os olhos em José e lhe disse: Deita-te comigo.
\par 8 Ele, porém, recusou e disse à mulher do seu senhor: Tem-me por mordomo o meu senhor e não sabe do que há em casa, pois tudo o que tem me passou ele às minhas mãos.
\par 9 Ele não é maior do que eu nesta casa e nenhuma coisa me vedou, senão a ti, porque és sua mulher; como, pois, cometeria eu tamanha maldade e pecaria contra Deus?
\par 10 Falando ela a José todos os dias, e não lhe dando ele ouvidos, para se deitar com ela e estar com ela,
\par 11 sucedeu que, certo dia, veio ele a casa, para atender aos negócios; e ninguém dos de casa se achava presente.
\par 12 Então, ela o pegou pelas vestes e lhe disse: Deita-te comigo; ele, porém, deixando as vestes nas mãos dela, saiu, fugindo para fora.
\par 13 Vendo ela que ele fugira para fora, mas havia deixado as vestes nas mãos dela,
\par 14 chamou pelos homens de sua casa e lhes disse: Vede, trouxe-nos meu marido este hebreu para insultar-nos; veio até mim para se deitar comigo; mas eu gritei em alta voz.
\par 15 Ouvindo ele que eu levantava a voz e gritava, deixou as vestes ao meu lado e saiu, fugindo para fora.
\par 16 Conservou ela junto de si as vestes dele, até que seu senhor tornou a casa.
\par 17 Então, lhe falou, segundo as mesmas palavras, e disse: O servo hebreu, que nos trouxeste, veio ter comigo para insultar-me;
\par 18 quando, porém, levantei a voz e gritei, ele, deixando as vestes ao meu lado, fugiu para fora.
\par 19 Tendo o senhor ouvido as palavras de sua mulher, como lhe tinha dito: Desta maneira me fez o teu servo; então, se lhe acendeu a ira.
\par 20 E o senhor de José o tomou e o lançou no cárcere, no lugar onde os presos do rei estavam encarcerados; ali ficou ele na prisão.
\par 21 O SENHOR, porém, era com José, e lhe foi benigno, e lhe deu mercê perante o carcereiro;
\par 22 o qual confiou às mãos de José todos os presos que estavam no cárcere; e ele fazia tudo quanto se devia fazer ali.
\par 23 E nenhum cuidado tinha o carcereiro de todas as coisas que estavam nas mãos de José, porquanto o SENHOR era com ele, e tudo o que ele fazia o SENHOR prosperava.

\chapter{40}

\par 1 Passadas estas coisas, aconteceu que o mordomo do rei do Egito e o padeiro ofenderam o seu senhor, o rei do Egito.
\par 2 Indignou-se Faraó contra os seus dois oficiais, o copeiro-chefe e o padeiro-chefe.
\par 3 E mandou detê-los na casa do comandante da guarda, no cárcere onde José estava preso.
\par 4 O comandante da guarda pô-los a cargo de José, para que os servisse; e por algum tempo estiveram na prisão.
\par 5 E ambos sonharam, cada um o seu sonho, na mesma noite; cada sonho com a sua própria significação, o copeiro e o padeiro do rei do Egito, que se achavam encarcerados.
\par 6 Vindo José, pela manhã, viu-os, e eis que estavam turbados.
\par 7 Então, perguntou aos oficiais de Faraó, que com ele estavam no cárcere da casa do seu senhor: Por que tendes, hoje, triste o semblante?
\par 8 Eles responderam: Tivemos um sonho, e não há quem o possa interpretar. Disse-lhes José: Porventura, não pertencem a Deus as interpretações? Contai-me o sonho.
\par 9 Então, o copeiro-chefe contou o seu sonho a José e lhe disse: Em meu sonho havia uma videira perante mim.
\par 10 E, na videira, três ramos; ao brotar a vide, havia flores, e seus cachos produziam uvas maduras.
\par 11 O copo de Faraó estava na minha mão; tomei as uvas, e as espremi no copo de Faraó, e o dei na própria mão de Faraó.
\par 12 Então, lhe disse José: Esta é a sua interpretação: os três ramos são três dias;
\par 13 dentro ainda de três dias, Faraó te reabilitará e te reintegrará no teu cargo, e tu lhe darás o copo na própria mão dele, segundo o costume antigo, quando lhe eras copeiro.
\par 14 Porém lembra-te de mim, quando tudo te correr bem; e rogo-te que sejas bondoso para comigo, e faças menção de mim a Faraó, e me faças sair desta casa;
\par 15 porque, de fato, fui roubado da terra dos hebreus; e, aqui, nada fiz, para que me pusessem nesta masmorra.
\par 16 Vendo o padeiro-chefe que a interpretação era boa, disse a José: Eu também sonhei, e eis que três cestos de pão alvo me estavam sobre a cabeça;
\par 17 e no cesto mais alto havia de todos os manjares de Faraó, arte de padeiro; e as aves os comiam do cesto na minha cabeça.
\par 18 Então, lhe disse José: A interpretação é esta: os três cestos são três dias;
\par 19 dentro ainda de três dias, Faraó te tirará fora a cabeça e te pendurará num madeiro, e as aves te comerão as carnes.
\par 20 No terceiro dia, que era aniversário de nascimento de Faraó, deu este um banquete a todos os seus servos; e, no meio destes, reabilitou o copeiro-chefe e condenou o padeiro-chefe.
\par 21 Ao copeiro-chefe reintegrou no seu cargo, no qual dava o copo na mão de Faraó;
\par 22 mas ao padeiro-chefe enforcou, como José havia interpretado.
\par 23 O copeiro-chefe, todavia, não se lembrou de José, porém dele se esqueceu.

\chapter{41}

\par 1 Passados dois anos completos, Faraó teve um sonho. Parecia-lhe achar-se ele de pé junto ao Nilo.
\par 2 Do rio subiam sete vacas formosas à vista e gordas e pastavam no carriçal.
\par 3 Após elas subiam do rio outras sete vacas, feias à vista e magras; e pararam junto às primeiras, na margem do rio.
\par 4 As vacas feias à vista e magras comiam as sete formosas à vista e gordas. Então, acordou Faraó.
\par 5 Tornando a dormir, sonhou outra vez. De uma só haste saíam sete espigas cheias e boas.
\par 6 E após elas nasciam sete espigas mirradas, crestadas do vento oriental.
\par 7 As espigas mirradas devoravam as sete espigas grandes e cheias. Então, acordou Faraó. Fora isto um sonho.
\par 8 De manhã, achando-se ele de espírito perturbado, mandou chamar todos os magos do Egito e todos os seus sábios e lhes contou os sonhos; mas ninguém havia que lhos interpretasse.
\par 9 Então, disse a Faraó o copeiro-chefe: Lembro-me hoje das minhas ofensas.
\par 10 Estando Faraó mui indignado contra os seus servos e pondo-me sob prisão na casa do comandante da guarda, a mim e ao padeiro-chefe,
\par 11 tivemos um sonho na mesma noite, eu e ele; sonhamos, e cada sonho com a sua própria significação.
\par 12 Achava-se conosco um jovem hebreu, servo do comandante da guarda; contamos-lhe os nossos sonhos, e ele no-los interpretou, a cada um segundo o seu sonho.
\par 13 E como nos interpretou, assim mesmo se deu: eu fui restituído ao meu cargo, o outro foi enforcado.
\par 14 Então, Faraó mandou chamar a José, e o fizeram sair à pressa da masmorra; ele se barbeou, mudou de roupa e foi apresentar-se a Faraó.
\par 15 Este lhe disse: Tive um sonho, e não há quem o interprete. Ouvi dizer, porém, a teu respeito que, quando ouves um sonho, podes interpretá-lo.
\par 16 Respondeu-lhe José: Não está isso em mim; mas Deus dará resposta favorável a Faraó.
\par 17 Então, contou Faraó a José: No meu sonho, estava eu de pé na margem do Nilo,
\par 18 e eis que subiam dele sete vacas gordas e formosas à vista e pastavam no carriçal.
\par 19 Após estas subiam outras vacas, fracas, mui feias à vista e magras; nunca vi outras assim disformes, em toda a terra do Egito.
\par 20 E as vacas magras e ruins comiam as primeiras sete gordas;
\par 21 e, depois de as terem engolido, não davam aparência de as terem devorado, pois o seu aspecto continuava ruim como no princípio. Então, acordei.
\par 22 Depois, vi, em meu sonho, que sete espigas saíam da mesma haste, cheias e boas;
\par 23 após elas nasceram sete espigas secas, mirradas e crestadas do vento oriental.
\par 24 As sete espigas mirradas devoravam as sete espigas boas. Contei-o aos magos, mas ninguém houve que mo interpretasse.
\par 25 Então, lhe respondeu José: O sonho de Faraó é apenas um; Deus manifestou a Faraó o que há de fazer.
\par 26 As sete vacas boas serão sete anos; as sete espigas boas, também sete anos; o sonho é um só.
\par 27 As sete vacas magras e feias, que subiam após as primeiras, serão sete anos, bem como as sete espigas mirradas e crestadas do vento oriental serão sete anos de fome.
\par 28 Esta é a palavra, como acabo de dizer a Faraó, que Deus manifestou a Faraó que ele há de fazer.
\par 29 Eis aí vêm sete anos de grande abundância por toda a terra do Egito.
\par 30 Seguir-se-ão sete anos de fome, e toda aquela abundância será esquecida na terra do Egito, e a fome consumirá a terra;
\par 31 e não será lembrada a abundância na terra, em vista da fome que seguirá, porque será gravíssima.
\par 32 O sonho de Faraó foi dúplice, porque a coisa é estabelecida por Deus, e Deus se apressa a fazê-la.
\par 33 Agora, pois, escolha Faraó um homem ajuizado e sábio e o ponha sobre a terra do Egito.
\par 34 Faça isso Faraó, e ponha administradores sobre a terra, e tome a quinta parte dos frutos da terra do Egito nos sete anos de fartura.
\par 35 Ajuntem os administradores toda a colheita dos bons anos que virão, recolham cereal debaixo do poder de Faraó, para mantimento nas cidades, e o guardem.
\par 36 Assim, o mantimento será para abastecer a terra nos sete anos da fome que haverá no Egito; para que a terra não pereça de fome.
\par 37 O conselho foi agradável a Faraó e a todos os seus oficiais.
\par 38 Disse Faraó aos seus oficiais: Acharíamos, porventura, homem como este, em quem há o Espírito de Deus?
\par 39 Depois, disse Faraó a José: Visto que Deus te fez saber tudo isto, ninguém há tão ajuizado e sábio como tu.
\par 40 Administrarás a minha casa, e à tua palavra obedecerá todo o meu povo; somente no trono eu serei maior do que tu.
\par 41 Disse mais Faraó a José: Vês que te faço autoridade sobre toda a terra do Egito.
\par 42 Então, tirou Faraó o seu anel de sinete da mão e o pôs na mão de José, fê-lo vestir roupas de linho fino e lhe pôs ao pescoço um colar de ouro.
\par 43 E fê-lo subir ao seu segundo carro, e clamavam diante dele: Inclinai-vos! Desse modo, o constituiu sobre toda a terra do Egito.
\par 44 Disse ainda Faraó a José: Eu sou Faraó, contudo sem a tua ordem ninguém levantará mão ou pé em toda a terra do Egito.
\par 45 E a José chamou Faraó de Zafenate-Panéia e lhe deu por mulher a Asenate, filha de Potífera, sacerdote de Om; e percorreu José toda a terra do Egito.
\par 46 Era José da idade de trinta anos quando se apresentou a Faraó, rei do Egito, e andou por toda a terra do Egito.
\par 47 Nos sete anos de fartura a terra produziu abundantemente.
\par 48 E ajuntou José todo o mantimento que houve na terra do Egito durante os sete anos e o guardou nas cidades; o mantimento do campo ao redor de cada cidade foi guardado na mesma cidade.
\par 49 Assim, ajuntou José muitíssimo cereal, como a areia do mar, até perder a conta, porque ia além das medidas.
\par 50 Antes de chegar a fome, nasceram dois filhos a José, os quais lhe deu Asenate, filha de Potífera, sacerdote de Om.
\par 51 José ao primogênito chamou de Manassés, pois disse: Deus me fez esquecer de todos os meus trabalhos e de toda a casa de meu pai.
\par 52 Ao segundo, chamou-lhe Efraim, pois disse: Deus me fez próspero na terra da minha aflição.
\par 53 Passados os sete anos de abundância, que houve na terra do Egito,
\par 54 começaram a vir os sete anos de fome, como José havia predito; e havia fome em todas as terras, mas em toda a terra do Egito havia pão.
\par 55 Sentindo toda a terra do Egito a fome, clamou o povo a Faraó por pão; e Faraó dizia a todos os egípcios: Ide a José; o que ele vos disser fazei.
\par 56 Havendo, pois, fome sobre toda a terra, abriu José todos os celeiros e vendia aos egípcios; porque a fome prevaleceu na terra do Egito.
\par 57 E todas as terras vinham ao Egito, para comprar de José, porque a fome prevaleceu em todo o mundo.

\chapter{42}

\par 1 Sabedor Jacó de que havia mantimento no Egito, disse a seus filhos: Por que estais aí a olhar uns para os outros?
\par 2 E ajuntou: Tenho ouvido que há cereais no Egito; descei até lá e comprai-nos deles, para que vivamos e não morramos.
\par 3 Então, desceram dez dos irmãos de José, para comprar cereal do Egito.
\par 4 A Benjamim, porém, irmão de José, não enviou Jacó na companhia dos irmãos, porque dizia: Para que não lhe suceda, acaso, algum desastre.
\par 5 Entre os que iam, pois, para lá, foram também os filhos de Israel; porque havia fome na terra de Canaã.
\par 6 José era governador daquela terra; era ele quem vendia a todos os povos da terra; e os irmãos de José vieram e se prostraram rosto em terra, perante ele.
\par 7 Vendo José a seus irmãos, reconheceu-os, porém não se deu a conhecer, e lhes falou asperamente, e lhes perguntou: Donde vindes? Responderam: Da terra de Canaã, para comprar mantimento.
\par 8 José reconheceu os irmãos; porém eles não o reconheceram.
\par 9 Então, se lembrou José dos sonhos que tivera a respeito deles e lhes disse: Vós sois espiões e viestes para ver os pontos fracos da terra.
\par 10 Responderam-lhe: Não, senhor meu; mas vieram os teus servos para comprar mantimento.
\par 11 Somos todos filhos de um mesmo homem; somos homens honestos; os teus servos não são espiões.
\par 12 Ele, porém, lhes respondeu: Nada disso; pelo contrário, viestes para ver os pontos fracos da terra.
\par 13 Eles disseram: Nós, teus servos, somos doze irmãos, filhos de um homem na terra de Canaã; o mais novo está hoje com nosso pai, outro já não existe.
\par 14 Então, lhes falou José: É como já vos disse: sois espiões.
\par 15 Nisto sereis provados: pela vida de Faraó, daqui não saireis, sem que primeiro venha o vosso irmão mais novo.
\par 16 Enviai um dentre vós, que traga vosso irmão; vós ficareis detidos para que sejam provadas as vossas palavras, se há verdade no que dizeis; ou se não, pela vida de Faraó, sois espiões.
\par 17 E os meteu juntos em prisão três dias.
\par 18 Ao terceiro dia, disse-lhes José: Fazei o seguinte e vivereis, pois temo a Deus.
\par 19 Se sois homens honestos, fique detido um de vós na casa da vossa prisão; vós outros ide, levai cereal para suprir a fome das vossas casas.
\par 20 E trazei-me vosso irmão mais novo, com o que serão verificadas as vossas palavras, e não morrereis. E eles se dispuseram a fazê-lo.
\par 21 Então, disseram uns aos outros: Na verdade, somos culpados, no tocante a nosso irmão, pois lhe vimos a angústia da alma, quando nos rogava, e não lhe acudimos; por isso, nos vem esta ansiedade.
\par 22 Respondeu-lhes Rúben: Não vos disse eu: Não pequeis contra o jovem? E não me quisestes ouvir. Pois vedes aí que se requer de nós o seu sangue.
\par 23 Eles, porém, não sabiam que José os entendia, porque lhes falava por intérprete.
\par 24 E, retirando-se deles, chorou; depois, tornando, lhes falou; tomou a Simeão dentre eles e o algemou na presença deles.
\par 25 Ordenou José que lhes enchessem de cereal os sacos, e lhes restituíssem o dinheiro, a cada um no saco de cereal, e os suprissem de comida para o caminho; e assim lhes foi feito.
\par 26 E carregaram o cereal sobre os seus jumentos e partiram dali.
\par 27 Abrindo um deles o saco de cereal, para dar de comer ao seu jumento na estalagem, deu com o dinheiro na boca do saco de cereal.
\par 28 Então, disse aos irmãos: Devolveram o meu dinheiro; aqui está na boca do saco de cereal. Desfaleceu-lhes o coração, e, atemorizados, entreolhavam-se, dizendo: Que é isto que Deus nos fez?
\par 29 E vieram para Jacó, seu pai, na terra de Canaã, e lhe contaram tudo o que lhes acontecera, dizendo:
\par 30 O homem, o senhor da terra, falou conosco asperamente e nos tratou como espiões da terra.
\par 31 Dissemos-lhe: Somos homens honestos; não somos espiões;
\par 32 somos doze irmãos, filhos de um mesmo pai; um já não existe, e o mais novo está hoje com nosso pai na terra de Canaã.
\par 33 Respondeu-nos o homem, o senhor da terra: Nisto conhecerei que sois homens honestos: deixai comigo um de vossos irmãos, tomai o cereal para remediar a fome de vossas casas e parti;
\par 34 trazei-me vosso irmão mais novo; assim saberei que não sois espiões, mas homens honestos. Então, vos entregarei vosso irmão, e negociareis na terra.
\par 35 Aconteceu que, despejando eles os sacos de cereal, eis cada um tinha a sua trouxinha de dinheiro no saco de cereal; e viram as trouxinhas com o dinheiro, eles e seu pai, e temeram.
\par 36 Então, lhes disse Jacó, seu pai: Tendes-me privado de filhos: José já não existe, Simeão não está aqui, e ides levar a Benjamim! Todas estas coisas me sobrevêm.
\par 37 Mas Rúben disse a seu pai: Mata os meus dois filhos, se to não tornar a trazer; entrega-mo, e eu to restituirei.
\par 38 Ele, porém, disse: Meu filho não descerá convosco; seu irmão é morto, e ele ficou só; se lhe sucede algum desastre no caminho por onde fordes, fareis descer minhas cãs com tristeza à sepultura.

\chapter{43}

\par 1 A fome persistia gravíssima na terra.
\par 2 Tendo eles acabado de consumir o cereal que trouxeram do Egito, disse-lhes seu pai: Voltai, comprai-nos um pouco de mantimento.
\par 3 Mas Judá lhe respondeu: Fortemente nos protestou o homem, dizendo: Não me vereis o rosto, se o vosso irmão não vier convosco.
\par 4 Se resolveres enviar conosco o nosso irmão, desceremos e te compraremos mantimento;
\par 5 se, porém, não o enviares, não desceremos; pois o homem nos disse: Não me vereis o rosto, se o vosso irmão não vier convosco.
\par 6 Disse-lhes Israel: Por que me fizestes esse mal, dando a saber àquele homem que tínheis outro irmão?
\par 7 Responderam eles: O homem nos perguntou particularmente por nós e pela nossa parentela, dizendo: Vive ainda vosso pai? Tendes outro irmão? Respondemos-lhe segundo as suas palavras. Acaso, poderíamos adivinhar que haveria de dizer: Trazei vosso irmão?
\par 8 Com isto disse Judá a Israel, seu pai: Envia o jovem comigo, e nos levantaremos e iremos; para que vivamos e não morramos, nem nós, nem tu, nem os nossos filhinhos.
\par 9 Eu serei responsável por ele, da minha mão o requererás; se eu to não trouxer e não to puser à presença, serei culpado para contigo para sempre.
\par 10 Se não nos tivéssemos demorado já estaríamos, com certeza, de volta segunda vez.
\par 11 Respondeu-lhes Israel, seu pai: Se é tal, fazei, pois, isto: tomai do mais precioso desta terra nos sacos para o mantimento e levai de presente a esse homem: um pouco de bálsamo e um pouco de mel, arômatas e mirra, nozes de pistácia e amêndoas;
\par 12 levai também dinheiro em dobro; e o dinheiro restituído na boca dos sacos de cereal, tornai a levá-lo convosco; pode bem ser que fosse engano.
\par 13 Levai também vosso irmão, levantai-vos e voltai àquele homem.
\par 14 Deus Todo-Poderoso vos dê misericórdia perante o homem, para que vos restitua o vosso outro irmão e deixe vir Benjamim. Quanto a mim, se eu perder os filhos, sem filhos ficarei.
\par 15 Tomaram, pois, os homens os presentes, o dinheiro em dobro e a Benjamim; levantaram-se, desceram ao Egito e se apresentaram perante José.
\par 16 Vendo José a Benjamim com eles, disse ao despenseiro de sua casa: Leva estes homens para casa, mata reses e prepara tudo; pois estes homens comerão comigo ao meio-dia.
\par 17 Fez ele como José lhe ordenara e levou os homens para a casa de José.
\par 18 Os homens tiveram medo, porque foram levados à casa de José; e diziam: É por causa do dinheiro que da outra vez voltou nos sacos de cereal, para nos acusar e arremeter contra nós, escravizar-nos e tomar nossos jumentos.
\par 19 E se chegaram ao mordomo da casa de José, e lhe falaram à porta,
\par 20 e disseram: Ai! Senhor meu, já uma vez descemos a comprar mantimento;
\par 21 quando chegamos à estalagem, abrindo os sacos de cereal, eis que o dinheiro de cada um estava na boca do saco de cereal, nosso dinheiro intacto; tornamos a trazê-lo conosco.
\par 22 Trouxemos também outro dinheiro conosco, para comprar mantimento; não sabemos quem tenha posto o nosso dinheiro nos sacos de cereal.
\par 23 Ele disse: Paz seja convosco, não temais; o vosso Deus, e o Deus de vosso pai, vos deu tesouro nos sacos de cereal; o vosso dinheiro me chegou a mim. E lhes trouxe fora a Simeão.
\par 24 Depois, levou o mordomo aqueles homens à casa de José e lhes deu água, e eles lavaram os pés; também deu ração aos seus jumentos.
\par 25 Então, prepararam o presente, para quando José viesse ao meio-dia; pois ouviram que ali haviam de comer.
\par 26 Chegando José a casa, trouxeram-lhe para dentro o presente que tinham em mãos; e prostraram-se perante ele até à terra.
\par 27 Ele lhes perguntou pelo seu bem-estar e disse: Vosso pai, o ancião de quem me falastes, vai bem? Ainda vive?
\par 28 Responderam: Vai bem o teu servo, nosso pai vive ainda; e abaixaram a cabeça e prostraram-se.
\par 29 Levantando José os olhos, viu a Benjamim, seu irmão, filho de sua mãe, e disse: É este o vosso irmão mais novo, de quem me falastes? E acrescentou: Deus te conceda graça, meu filho.
\par 30 José se apressou e procurou onde chorar, porque se movera no seu íntimo, para com seu irmão; entrou na câmara e chorou ali.
\par 31 Depois, lavou o rosto e saiu; conteve-se e disse: Servi a refeição.
\par 32 Serviram-lhe a ele à parte, e a eles também à parte, e à parte aos egípcios que comiam com ele; porque aos egípcios não lhes era lícito comer pão com os hebreus, porquanto é isso abominação para os egípcios.
\par 33 E assentaram-se diante dele, o primogênito segundo a sua primogenitura e o mais novo segundo a sua menoridade; disto os homens se maravilhavam entre si.
\par 34 Então, lhes apresentou as porções que estavam diante dele; a porção de Benjamim era cinco vezes mais do que a de qualquer deles. E eles beberam e se regalaram com ele.

\chapter{44}

\par 1 Deu José esta ordem ao mordomo de sua casa: Enche de mantimento os sacos que estes homens trouxeram, quanto puderem levar, e põe o dinheiro de cada um na boca do saco de mantimento.
\par 2 O meu copo de prata pô-lo-ás na boca do saco de mantimento do mais novo, com o dinheiro do seu cereal. E assim se fez segundo José dissera.
\par 3 De manhã, quando já claro, despediram-se estes homens, eles com os seus jumentos.
\par 4 Tendo saído eles da cidade, não se havendo ainda distanciado, disse José ao mordomo de sua casa: Levanta-te e segue após esses homens; e, alcançando-os, lhes dirás: Por que pagastes mal por bem?
\par 5 Não é este o copo em que bebe meu senhor? E por meio do qual faz as suas adivinhações? Procedestes mal no que fizestes.
\par 6 E alcançou-os e lhes falou essas palavras.
\par 7 Então, lhe responderam: Por que diz meu senhor tais palavras? Longe estejam teus servos de praticar semelhante coisa.
\par 8 O dinheiro que achamos na boca dos sacos de mantimento, tornamos a trazer-te desde a terra de Canaã; como, pois, furtaríamos da casa do teu senhor prata ou ouro?
\par 9 Aquele dos teus servos, com quem for achado, morra; e nós ainda seremos escravos do meu senhor.
\par 10 Então, lhes respondeu: Seja conforme as vossas palavras; aquele com quem se achar será meu escravo, porém vós sereis inculpados.
\par 11 E se apressaram, e, tendo cada um posto o saco de mantimento em terra, o abriu.
\par 12 O mordomo os examinou, começando do mais velho e acabando no mais novo; e achou-se o copo no saco de mantimento de Benjamim.
\par 13 Então, rasgaram as suas vestes e, carregados de novo os jumentos, tornaram à cidade.
\par 14 E chegou Judá com seus irmãos à casa de José; este ainda estava ali; e prostraram-se em terra diante dele.
\par 15 Disse-lhes José: Que é isso que fizestes? Não sabíeis vós que tal homem como eu é capaz de adivinhar?
\par 16 Então, disse Judá: Que responderemos a meu senhor? Que falaremos? E como nos justificaremos? Achou Deus a iniqüidade de teus servos; eis que somos escravos de meu senhor, tanto nós como aquele em cuja mão se achou o copo.
\par 17 Mas ele disse: Longe de mim que eu tal faça; o homem em cuja mão foi achado o copo, esse será meu servo; vós, no entanto, subi em paz para vosso pai.
\par 18 Então, Judá se aproximou dele e disse: Ah! Senhor meu, rogo-te, permite que teu servo diga uma palavra aos ouvidos do meu senhor, e não se acenda a tua ira contra o teu servo; porque tu és como o próprio Faraó.
\par 19 Meu senhor perguntou a seus servos: Tendes pai ou irmão?
\par 20 E respondemos a meu senhor: Temos pai já velho e um filho da sua velhice, o mais novo, cujo irmão é morto; e só ele ficou de sua mãe, e seu pai o ama.
\par 21 Então, disseste a teus servos: Trazei-mo, para que ponha os olhos sobre ele.
\par 22 Respondemos ao meu senhor: O moço não pode deixar o pai; se deixar o pai, este morrerá.
\par 23 Então, disseste a teus servos: Se vosso irmão mais novo não descer convosco, nunca mais me vereis o rosto.
\par 24 Tendo nós subido a teu servo, meu pai, e a ele repetido as palavras de meu senhor,
\par 25 disse nosso pai: Voltai, comprai-nos um pouco de mantimento.
\par 26 Nós respondemos: Não podemos descer; mas, se nosso irmão mais moço for conosco, desceremos; pois não podemos ver a face do homem, se este nosso irmão mais moço não estiver conosco.
\par 27 Então, nos disse o teu servo, nosso pai: Sabeis que minha mulher me deu dois filhos;
\par 28 um se ausentou de mim, e eu disse: Certamente foi despedaçado, e até agora não mais o vi;
\par 29 se agora também tirardes este da minha presença, e lhe acontecer algum desastre, fareis descer as minhas cãs com pesar à sepultura.
\par 30 Agora, pois, indo eu a teu servo, meu pai, e não indo o moço conosco, visto a sua alma estar ligada com a alma dele,
\par 31 vendo ele que o moço não está conosco, morrerá; e teus servos farão descer as cãs de teu servo, nosso pai, com tristeza à sepultura.
\par 32 Porque teu servo se deu por fiador por este moço para com o meu pai, dizendo: Se eu o não tornar a trazer-te, serei culpado para com o meu pai todos os dias.
\par 33 Agora, pois, fique teu servo em lugar do moço por servo de meu senhor, e o moço que suba com seus irmãos.
\par 34 Porque como subirei eu a meu pai, se o moço não for comigo? Para que não veja eu o mal que a meu pai sobrevirá.

\chapter{45}

\par 1 Então, José, não se podendo conter diante de todos os que estavam com ele, bradou: Fazei sair a todos da minha presença! E ninguém ficou com ele, quando José se deu a conhecer a seus irmãos.
\par 2 E levantou a voz em choro, de maneira que os egípcios o ouviam e também a casa de Faraó.
\par 3 E disse a seus irmãos: Eu sou José; vive ainda meu pai? E seus irmãos não lhe puderam responder, porque ficaram atemorizados perante ele.
\par 4 Disse José a seus irmãos: Agora, chegai-vos a mim. E chegaram-se. Então, disse: Eu sou José, vosso irmão, a quem vendestes para o Egito.
\par 5 Agora, pois, não vos entristeçais, nem vos irriteis contra vós mesmos por me haverdes vendido para aqui; porque, para conservação da vida, Deus me enviou adiante de vós.
\par 6 Porque já houve dois anos de fome na terra, e ainda restam cinco anos em que não haverá lavoura nem colheita.
\par 7 Deus me enviou adiante de vós, para conservar vossa sucessão na terra e para vos preservar a vida por um grande livramento.
\par 8 Assim, não fostes vós que me enviastes para cá, e sim Deus, que me pôs por pai de Faraó, e senhor de toda a sua casa, e como governador em toda a terra do Egito.
\par 9 Apressai-vos, subi a meu pai e dizei-lhe: Assim manda dizer teu filho José: Deus me pôs por senhor em toda terra do Egito; desce a mim, não te demores.
\par 10 Habitarás na terra de Gósen e estarás perto de mim, tu, teus filhos, os filhos de teus filhos, os teus rebanhos, o teu gado e tudo quanto tens.
\par 11 Aí te sustentarei, porque ainda haverá cinco anos de fome; para que não te empobreças, tu e tua casa e tudo o que tens.
\par 12 Eis que vedes por vós mesmos, e meu irmão Benjamim vê também, que sou eu mesmo quem vos fala.
\par 13 Anunciai a meu pai toda a minha glória no Egito e tudo o que tendes visto; apressai-vos e fazei descer meu pai para aqui.
\par 14 E, lançando-se ao pescoço de Benjamim, seu irmão, chorou; e, abraçado com ele, chorou também Benjamim.
\par 15 José beijou a todos os seus irmãos e chorou sobre eles; depois, seus irmãos falaram com ele.
\par 16 Fez-se ouvir na casa de Faraó esta notícia: São vindos os irmãos de José; e isto foi agradável a Faraó e a seus oficiais.
\par 17 Disse Faraó a José: Dize a teus irmãos: Fazei isto: carregai os vossos animais e parti; tornai à terra de Canaã,
\par 18 tomai a vosso pai e a vossas famílias e vinde para mim; dar-vos-ei o melhor da terra do Egito, e comereis a fartura da terra.
\par 19 Ordena-lhes também: Fazei isto: levai da terra do Egito carros para vossos filhinhos e para vossas mulheres, trazei vosso pai e vinde.
\par 20 Não vos preocupeis com coisa alguma dos vossos haveres, porque o melhor de toda a terra do Egito será vosso.
\par 21 E os filhos de Israel fizeram assim. José lhes deu carros, conforme o mandado de Faraó; também lhes deu provisão para o caminho.
\par 22 A cada um de todos eles deu vestes festivais, mas a Benjamim deu trezentas moedas de prata e cinco vestes festivais.
\par 23 Também enviou a seu pai dez jumentos carregados do melhor do Egito, e dez jumentos carregados de cereais e pão, e provisão para o seu pai, para o caminho.
\par 24 E despediu os seus irmãos. Ao partirem, disse-lhes: Não contendais pelo caminho.
\par 25 Então, subiram do Egito, e vieram à terra de Canaã, a Jacó, seu pai,
\par 26 e lhe disseram: José ainda vive e é governador de toda a terra do Egito. Com isto, o coração lhe ficou como sem palpitar, porque não lhes deu crédito.
\par 27 Porém, havendo-lhe eles contado todas as palavras que José lhes falara, e vendo Jacó, seu pai, os carros que José enviara para levá-lo, reviveu-se-lhe o espírito.
\par 28 E disse Israel: Basta; ainda vive meu filho José; irei e o verei antes que eu morra.

\chapter{46}

\par 1 Partiu, pois, Israel com tudo o que possuía, e veio a Berseba, e ofereceu sacrifícios ao Deus de Isaque, seu pai.
\par 2 Falou Deus a Israel em visões, de noite, e disse: Jacó! Jacó! Ele respondeu: Eis-me aqui!
\par 3 Então, disse: Eu sou Deus, o Deus de teu pai; não temas descer para o Egito, porque lá eu farei de ti uma grande nação.
\par 4 Eu descerei contigo para o Egito e te farei tornar a subir, certamente. A mão de José fechará os teus olhos.
\par 5 Então, se levantou Jacó de Berseba; e os filhos de Israel levaram Jacó, seu pai, e seus filhinhos, e as suas mulheres nos carros que Faraó enviara para o levar.
\par 6 Tomaram o seu gado e os bens que haviam adquirido na terra de Canaã e vieram para o Egito, Jacó e toda a sua descendência.
\par 7 Seus filhos e os filhos de seus filhos, suas filhas e as filhas de seus filhos e toda a sua descendência, levou-os consigo para o Egito.
\par 8 São estes os nomes dos filhos de Israel, Jacó, e seus filhos, que vieram para o Egito: Rúben, o primogênito de Jacó.
\par 9 Os filhos de Rúben: Enoque, Palu, Hezrom e Carmi.
\par 10 Os filhos de Simeão: Jemuel, Jamim, Oade, Jaquim, Zoar e Saul, filho de uma mulher cananéia.
\par 11 Os filhos de Levi: Gérson, Coate e Merari.
\par 12 Os filhos de Judá: Er, Onã, Selá, Perez e Zera; Er e Onã, porém, morreram na terra de Canaã. Os filhos de Perez foram: Hezrom e Hamul.
\par 13 Os filhos de Issacar: Tola, Puva, Jó e Sinrom.
\par 14 Os filhos de Zebulom: Serede, Elom e Jaleel.
\par 15 São estes os filhos de Lia, que ela deu à luz a Jacó em Padã-Arã, além de Diná, sua filha; todas as pessoas, de seus filhos e de suas filhas, trinta e três.
\par 16 Os filhos de Gade: Zifiom, Hagi, Suni, Esbom, Eri, Arodi e Areli.
\par 17 Os filhos de Aser: Imna, Isvá, Isvi, Berias e Sera, irmã deles; e os filhos de Berias: Héber e Malquiel.
\par 18 São estes os filhos de Zilpa, a qual Labão deu a sua filha Lia; e estes deu ela à luz a Jacó, a saber, dezesseis pessoas.
\par 19 Os filhos de Raquel, mulher de Jacó: José e Benjamim.
\par 20 Nasceram a José na terra do Egito Manassés e Efraim, que lhe deu à luz Asenate, filha de Potífera, sacerdote de Om.
\par 21 Os filhos de Benjamim: Belá, Bequer, Asbel, Gera, Naamã, Eí, Rôs, Mupim, Hupim e Arde.
\par 22 São estes os filhos de Raquel, que nasceram a Jacó, ao todo catorze pessoas.
\par 23 O filho de Dã: Husim.
\par 24 Os filhos de Naftali: Jazeel, Guni, Jezer e Silém.
\par 25 São estes os filhos de Bila, a qual Labão deu a sua filha Raquel; e estes deu ela à luz a Jacó, ao todo sete pessoas.
\par 26 Todos os que vieram com Jacó para o Egito, que eram os seus descendentes, fora as mulheres dos filhos de Jacó, todos eram sessenta e seis pessoas;
\par 27 e os filhos de José, que lhe nasceram no Egito, eram dois. Todas as pessoas da casa de Jacó, que vieram para o Egito, foram setenta.
\par 28 Jacó enviou Judá adiante de si a José para que soubesse encaminhá-lo a Gósen; e chegaram à terra de Gósen.
\par 29 Então, José aprontou o seu carro e subiu ao encontro de Israel, seu pai, a Gósen. Apresentou-se, lançou-se-lhe ao pescoço e chorou assim longo tempo.
\par 30 Disse Israel a José: Já posso morrer, pois já vi o teu rosto, e ainda vives.
\par 31 E José disse a seus irmãos e à casa de seu pai: Subirei, e farei saber a Faraó, e lhe direi: Meus irmãos e a casa de meu pai, que estavam na terra de Canaã, vieram para mim.
\par 32 Os homens são pastores, são homens de gado, e trouxeram consigo o seu rebanho, e o seu gado, e tudo o que têm.
\par 33 Quando, pois, Faraó vos chamar e disser: Qual é o vosso trabalho?
\par 34 Respondereis: Teus servos foram homens de gado desde a mocidade até agora, tanto nós como nossos pais; para que habiteis na terra de Gósen, porque todo pastor de rebanho é abominação para os egípcios.

\chapter{47}

\par 1 Então, veio José e disse a Faraó: Meu pai e meus irmãos, com os seus rebanhos e o seu gado, com tudo o que têm, chegaram da terra de Canaã; e eis que estão na terra de Gósen.
\par 2 E tomou cinco dos seus irmãos e os apresentou a Faraó.
\par 3 Então, perguntou Faraó aos irmãos de José: Qual é o vosso trabalho? Eles responderam: Os teus servos somos pastores de rebanho, tanto nós como nossos pais.
\par 4 Disseram mais a Faraó: Viemos para habitar nesta terra; porque não há pasto para o rebanho de teus servos, pois a fome é severa na terra de Canaã; agora, pois, te rogamos permitas habitem os teus servos na terra de Gósen.
\par 5 Então, disse Faraó a José: Teu pai e teus irmãos vieram a ti.
\par 6 A terra do Egito está perante ti; no melhor da terra faze habitar teu pai e teus irmãos; habitem na terra de Gósen. Se sabes haver entre eles homens capazes, põe-nos por chefes do gado que me pertence.
\par 7 Trouxe José a Jacó, seu pai, e o apresentou a Faraó; e Jacó abençoou a Faraó.
\par 8 Perguntou Faraó a Jacó: Quantos são os dias dos anos da tua vida?
\par 9 Jacó lhe respondeu: Os dias dos anos das minhas peregrinações são cento e trinta anos; poucos e maus foram os dias dos anos da minha vida e não chegaram aos dias dos anos da vida de meus pais, nos dias das suas peregrinações.
\par 10 E, tendo Jacó abençoado a Faraó, saiu de sua presença.
\par 11 Então, José estabeleceu a seu pai e a seus irmãos e lhes deu possessão na terra do Egito, no melhor da terra, na terra de Ramessés, como Faraó ordenara.
\par 12 E José sustentou de pão a seu pai, a seus irmãos e a toda a casa de seu pai, segundo o número de seus filhos.
\par 13 Não havia pão em toda a terra, porque a fome era mui severa; de maneira que desfalecia o povo do Egito e o povo de Canaã por causa da fome.
\par 14 Então, José arrecadou todo o dinheiro que se achou na terra do Egito e na terra de Canaã, pelo cereal que compravam, e o recolheu à casa de Faraó.
\par 15 Tendo-se acabado, pois, o dinheiro, na terra do Egito e na terra de Canaã, foram todos os egípcios a José e disseram: Dá-nos pão; por que haveremos de morrer em tua presença? Porquanto o dinheiro nos falta.
\par 16 Respondeu José: Se vos falta o dinheiro, trazei o vosso gado; em troca do vosso gado eu vos suprirei.
\par 17 Então, trouxeram o seu gado a José; e José lhes deu pão em troca de cavalos, de rebanhos, de gado e de jumentos; e os sustentou de pão aquele ano em troca do seu gado.
\par 18 Findo aquele ano, foram a José no ano próximo e lhe disseram: Não ocultaremos a meu senhor que se acabou totalmente o dinheiro; e meu senhor já possui os animais; nada mais nos resta diante de meu senhor, senão o nosso corpo e a nossa terra.
\par 19 Por que haveremos de perecer diante dos teus olhos, tanto nós como a nossa terra? Compra-nos a nós e a nossa terra a troco de pão, e nós e a nossa terra seremos escravos de Faraó; dá-nos semente para que vivamos e não morramos, e a terra não fique deserta.
\par 20 Assim, comprou José toda a terra do Egito para Faraó, porque os egípcios venderam cada um o seu campo, porquanto a fome era extrema sobre eles; e a terra passou a ser de Faraó.
\par 21 Quanto ao povo, ele o escravizou de uma a outra extremidade da terra do Egito.
\par 22 Somente a terra dos sacerdotes não a comprou ele; pois os sacerdotes tinham porção de Faraó e eles comiam a sua porção que Faraó lhes tinha dado; por isso, não venderam a sua terra.
\par 23 Então, disse José ao povo: Eis que hoje vos comprei a vós outros e a vossa terra para Faraó; aí tendes sementes, semeai a terra.
\par 24 Das colheitas dareis o quinto a Faraó, e as quatro partes serão vossas, para semente do campo, e para o vosso mantimento e dos que estão em vossas casas, e para que comam as vossas crianças.
\par 25 Responderam eles: A vida nos tens dado! Achemos mercê perante meu senhor e seremos escravos de Faraó.
\par 26 E José estabeleceu por lei até ao dia de hoje que, na terra do Egito, tirasse Faraó o quinto; só a terra dos sacerdotes não ficou sendo de Faraó.
\par 27 Assim, habitou Israel na terra do Egito, na terra de Gósen; nela tomaram possessão, e foram fecundos, e muito se multiplicaram.
\par 28 Jacó viveu na terra do Egito dezessete anos; de sorte que os dias de Jacó, os anos da sua vida, foram cento e quarenta e sete.
\par 29 Aproximando-se, pois, o tempo da morte de Israel, chamou a José, seu filho, e lhe disse: Se agora achei mercê à tua presença, rogo-te que ponhas a mão debaixo da minha coxa e uses comigo de beneficência e de verdade; rogo-te que me não enterres no Egito,
\par 30 porém que eu jaza com meus pais; por isso, me levarás do Egito e me enterrarás no lugar da sepultura deles. Respondeu José: Farei segundo a tua palavra.
\par 31 Então, lhe disse Jacó: Jura-me. E ele jurou-lhe; e Israel se inclinou sobre a cabeceira da cama.

\chapter{48}

\par 1 Passadas estas coisas, disseram a José: Teu pai está enfermo. Então, José tomou consigo a seus dois filhos, Manassés e Efraim.
\par 2 E avisaram a Jacó: Eis que José, teu filho, vem ter contigo. Esforçou-se Israel e se assentou no leito.
\par 3 Disse Jacó a José: O Deus Todo-Poderoso me apareceu em Luz, na terra de Canaã, e me abençoou,
\par 4 e me disse: Eis que te farei fecundo, e te multiplicarei, e te tornarei multidão de povos, e à tua descendência darei esta terra em possessão perpétua.
\par 5 Agora, pois, os teus dois filhos, que te nasceram na terra do Egito, antes que eu viesse a ti no Egito, são meus; Efraim e Manassés serão meus, como Rúben e Simeão.
\par 6 Mas a tua descendência, que gerarás depois deles, será tua; segundo o nome de um de seus irmãos serão chamados na sua herança.
\par 7 Vindo, pois, eu de Padã, me morreu, com pesar meu, Raquel na terra de Canaã, no caminho, havendo ainda pequena distância para chegar a Efrata; sepultei-a ali no caminho de Efrata, que é Belém.
\par 8 Tendo Israel visto os filhos de José, disse: Quem são estes?
\par 9 Respondeu José a seu pai: São meus filhos, que Deus me deu aqui. Faze-os chegar a mim, disse ele, para que eu os abençoe.
\par 10 Os olhos de Israel já se tinham escurecido por causa da velhice, de modo que não podia ver bem. José, pois, fê-los chegar a ele; e ele os beijou e os abraçou.
\par 11 Então, disse Israel a José: Eu não cuidara ver o teu rosto; e eis que Deus me fez ver os teus filhos também.
\par 12 E José, tirando-os dentre os joelhos de seu pai, inclinou-se à terra diante da sua face.
\par 13 Depois, tomou José a ambos, a Efraim na sua mão direita, à esquerda de Israel, e a Manassés na sua esquerda, à direita de Israel, e fê-los chegar a ele.
\par 14 Mas Israel estendeu a mão direita e a pôs sobre a cabeça de Efraim, que era o mais novo, e a sua esquerda sobre a cabeça de Manassés, cruzando assim as mãos, não obstante ser Manassés o primogênito.
\par 15 E abençoou a José, dizendo: O Deus em cuja presença andaram meus pais Abraão e Isaque, o Deus que me sustentou durante a minha vida até este dia,
\par 16 o Anjo que me tem livrado de todo mal, abençoe estes rapazes; seja neles chamado o meu nome e o nome de meus pais Abraão e Isaque; e cresçam em multidão no meio da terra.
\par 17 Vendo José que seu pai pusera a mão direita sobre a cabeça de Efraim, foi-lhe isto desagradável, e tomou a mão de seu pai para mudar da cabeça de Efraim para a cabeça de Manassés.
\par 18 E disse José a seu pai: Não assim, meu pai, pois o primogênito é este; põe a mão direita sobre a cabeça dele.
\par 19 Mas seu pai o recusou e disse: Eu sei, meu filho, eu o sei; ele também será um povo, também ele será grande; contudo, o seu irmão menor será maior do que ele, e a sua descendência será uma multidão de nações.
\par 20 Assim, os abençoou naquele dia, declarando: Por vós Israel abençoará, dizendo: Deus te faça como a Efraim e como a Manassés. E pôs o nome de Efraim adiante do de Manassés.
\par 21 Depois, disse Israel a José: Eis que eu morro, mas Deus será convosco e vos fará voltar à terra de vossos pais.
\par 22 Dou-te, de mais que a teus irmãos, um declive montanhoso, o qual tomei da mão dos amorreus com a minha espada e com o meu arco.

\chapter{49}

\par 1 Depois, chamou Jacó a seus filhos e disse: Ajuntai-vos, e eu vos farei saber o que vos há de acontecer nos dias vindouros:
\par 2 Ajuntai-vos e ouvi, filhos de Jacó; ouvi a Israel, vosso pai.
\par 3 Rúben, tu és meu primogênito, minha força e as primícias do meu vigor, o mais excelente em altivez e o mais excelente em poder.
\par 4 Impetuoso como a água, não serás o mais excelente, porque subiste ao leito de teu pai e o profanaste; subiste à minha cama.
\par 5 Simeão e Levi são irmãos; as suas espadas são instrumentos de violência.
\par 6 No seu conselho, não entre minha alma; com o seu agrupamento, minha glória não se ajunte; porque no seu furor mataram homens, e na sua vontade perversa jarretaram touros.
\par 7 Maldito seja o seu furor, pois era forte, e a sua ira, pois era dura; dividi-los-ei em Jacó e os espalharei em Israel.
\par 8 Judá, teus irmãos te louvarão; a tua mão estará sobre a cerviz de teus inimigos; os filhos de teu pai se inclinarão a ti.
\par 9 Judá é leãozinho; da presa subiste, filho meu. Encurva-se e deita-se como leão e como leoa; quem o despertará?
\par 10 O cetro não se arredará de Judá, nem o bastão de entre seus pés, até que venha Siló; e a ele obedecerão os povos.
\par 11 Ele amarrará o seu jumentinho à vide e o filho da sua jumenta, à videira mais excelente; lavará as suas vestes no vinho e a sua capa, em sangue de uvas.
\par 12 Os seus olhos serão cintilantes de vinho, e os dentes, brancos de leite.
\par 13 Zebulom habitará na praia dos mares e servirá de porto de navios, e o seu limite se estenderá até Sidom.
\par 14 Issacar é jumento de fortes ossos, de repouso entre os rebanhos de ovelhas.
\par 15 Viu que o repouso era bom e que a terra era deliciosa; baixou os ombros à carga e sujeitou-se ao trabalho servil.
\par 16 Dã julgará o seu povo, como uma das tribos de Israel.
\par 17 Dã será serpente junto ao caminho, uma víbora junto à vereda, que morde os talões do cavalo e faz cair o seu cavaleiro por detrás.
\par 18 A tua salvação espero, ó SENHOR!
\par 19 Gade, uma guerrilha o acometerá; mas ele a acometerá por sua retaguarda.
\par 20 Aser, o seu pão será abundante e ele motivará delícias reais.
\par 21 Naftali é uma gazela solta; ele profere palavras formosas.
\par 22 José é um ramo frutífero, ramo frutífero junto à fonte; seus galhos se estendem sobre o muro.
\par 23 Os flecheiros lhe dão amargura, atiram contra ele e o aborrecem.
\par 24 O seu arco, porém, permanece firme, e os seus braços são feitos ativos pelas mãos do Poderoso de Jacó, sim, pelo Pastor e pela Pedra de Israel,
\par 25 pelo Deus de teu pai, o qual te ajudará, e pelo Todo-Poderoso, o qual te abençoará com bênçãos dos altos céus, com bênçãos das profundezas, com bênçãos dos seios e da madre.
\par 26 As bênçãos de teu pai excederão as bênçãos de meus pais até ao cimo dos montes eternos; estejam elas sobre a cabeça de José e sobre o alto da cabeça do que foi distinguido entre seus irmãos.
\par 27 Benjamim é lobo que despedaça; pela manhã devora a presa e à tarde reparte o despojo.
\par 28 São estas as doze tribos de Israel; e isto é o que lhes falou seu pai quando os abençoou; a cada um deles abençoou segundo a bênção que lhe cabia.
\par 29 Depois, lhes ordenou, dizendo: Eu me reúno ao meu povo; sepultai-me, com meus pais, na caverna que está no campo de Efrom, o heteu,
\par 30 na caverna que está no campo de Macpela, fronteiro a Manre, na terra de Canaã, a qual Abraão comprou de Efrom com aquele campo, em posse de sepultura.
\par 31 Ali sepultaram Abraão e Sara, sua mulher; ali sepultaram Isaque e Rebeca, sua mulher; e ali sepultei Lia;
\par 32 o campo e a caverna que nele está, comprados aos filhos de Hete.
\par 33 Tendo Jacó acabado de dar determinações a seus filhos, recolheu os pés na cama, e expirou, e foi reunido ao seu povo.

\chapter{50}

\par 1 Então, José se lançou sobre o rosto de seu pai, e chorou sobre ele, e o beijou.
\par 2 Ordenou José a seus servos, aos que eram médicos, que embalsamassem a seu pai; e os médicos embalsamaram a Israel,
\par 3 gastando nisso quarenta dias, pois assim se cumprem os dias do embalsamamento; e os egípcios o choraram setenta dias.
\par 4 Passados os dias de o chorarem, falou José à casa de Faraó: Se agora achei mercê perante vós, rogo-vos que faleis aos ouvidos de Faraó, dizendo:
\par 5 Meu pai me fez jurar, declarando: Eis que eu morro; no meu sepulcro que abri para mim na terra de Canaã, ali me sepultarás. Agora, pois, desejo subir e sepultar meu pai, depois voltarei.
\par 6 Respondeu Faraó: Sobe e sepulta o teu pai como ele te fez jurar.
\par 7 José subiu para sepultar o seu pai; e subiram com ele todos os oficiais de Faraó, os principais da sua casa e todos os principais da terra do Egito,
\par 8 como também toda a casa de José, e seus irmãos, e a casa de seu pai; somente deixaram na terra de Gósen as crianças, e os rebanhos, e o gado.
\par 9 E subiram também com ele tanto carros como cavaleiros; e o cortejo foi grandíssimo.
\par 10 Chegando eles, pois, à eira de Atade, que está além do Jordão, fizeram ali grande e intensa lamentação; e José pranteou seu pai durante sete dias.
\par 11 Tendo visto os moradores da terra, os cananeus, o luto na eira de Atade, disseram: Grande pranto é este dos egípcios. E por isso se chamou aquele lugar de Abel-Mizraim, que está além do Jordão.
\par 12 Fizeram-lhe seus filhos como lhes havia ordenado:
\par 13 levaram-no para a terra de Canaã e o sepultaram na caverna do campo de Macpela, que Abraão comprara com o campo, por posse de sepultura, a Efrom, o heteu, fronteiro a Manre.
\par 14 Depois disso, voltou José para o Egito, ele, seus irmãos e todos os que com ele subiram a sepultar o seu pai.
\par 15 Vendo os irmãos de José que seu pai já era morto, disseram: É o caso de José nos perseguir e nos retribuir certamente o mal todo que lhe fizemos.
\par 16 Portanto, mandaram dizer a José: Teu pai ordenou, antes da sua morte, dizendo:
\par 17 Assim direis a José: Perdoa, pois, a transgressão de teus irmãos e o seu pecado, porque te fizeram mal; agora, pois, te rogamos que perdoes a transgressão dos servos do Deus de teu pai. José chorou enquanto lhe falavam.
\par 18 Depois, vieram também seus irmãos, prostraram-se diante dele e disseram: Eis-nos aqui por teus servos.
\par 19 Respondeu-lhes José: Não temais; acaso, estou eu em lugar de Deus?
\par 20 Vós, na verdade, intentastes o mal contra mim; porém Deus o tornou em bem, para fazer, como vedes agora, que se conserve muita gente em vida.
\par 21 Não temais, pois; eu vos sustentarei a vós outros e a vossos filhos. Assim, os consolou e lhes falou ao coração.
\par 22 José habitou no Egito, ele e a casa de seu pai; e viveu cento e dez anos.
\par 23 Viu José os filhos de Efraim até à terceira geração; também os filhos de Maquir, filho de Manassés, os quais José tomou sobre seus joelhos.
\par 24 Disse José a seus irmãos: Eu morro; porém Deus certamente vos visitará e vos fará subir desta terra para a terra que jurou dar a Abraão, a Isaque e a Jacó.
\par 25 José fez jurar os filhos de Israel, dizendo: Certamente Deus vos visitará, e fareis transportar os meus ossos daqui.
\par 26 Morreu José da idade de cento e dez anos; embalsamaram-no e o puseram num caixão no Egito.


\end{document}