\begin{document}

\title{Êxodo}


\chapter{1}

\par 1 São estes os nomes dos filhos de Israel que entraram com Jacó no Egito; cada um entrou com sua família:
\par 2 Rúben, Simeão, Levi e Judá,
\par 3 Issacar, Zebulom e Benjamim,
\par 4 Dã, Naftali, Gade e Aser.
\par 5 Todas as pessoas, pois, que descenderam de Jacó foram setenta; José, porém, estava no Egito.
\par 6 Faleceu José, e todos os seus irmãos, e toda aquela geração.
\par 7 Mas os filhos de Israel foram fecundos, e aumentaram muito, e se multiplicaram, e grandemente se fortaleceram, de maneira que a terra se encheu deles.
\par 8 Entrementes, se levantou novo rei sobre o Egito, que não conhecera a José.
\par 9 Ele disse ao seu povo: Eis que o povo dos filhos de Israel é mais numeroso e mais forte do que nós.
\par 10 Eia, usemos de astúcia para com ele, para que não se multiplique, e seja o caso que, vindo guerra, ele se ajunte com os nossos inimigos, peleje contra nós e saia da terra.
\par 11 E os egípcios puseram sobre eles feitores de obras, para os afligirem com suas cargas. E os israelitas edificaram a Faraó as cidades-celeiros, Pitom e Ramessés.
\par 12 Mas, quanto mais os afligiam, tanto mais se multiplicavam e tanto mais se espalhavam; de maneira que se inquietavam por causa dos filhos de Israel;
\par 13 então, os egípcios, com tirania, faziam servir os filhos de Israel
\par 14 e lhes fizeram amargar a vida com dura servidão, em barro, e em tijolos, e com todo o trabalho no campo; com todo o serviço em que na tirania os serviam.
\par 15 O rei do Egito ordenou às parteiras hebréias, das quais uma se chamava Sifrá, e outra, Puá,
\par 16 dizendo: Quando servirdes de parteira às hebréias, examinai: se for filho, matai-o; mas, se for filha, que viva.
\par 17 As parteiras, porém, temeram a Deus e não fizeram como lhes ordenara o rei do Egito; antes, deixaram viver os meninos.
\par 18 Então, o rei do Egito chamou as parteiras e lhes disse: Por que fizestes isso e deixastes viver os meninos?
\par 19 Responderam as parteiras a Faraó: É que as mulheres hebréias não são como as egípcias; são vigorosas e, antes que lhes chegue a parteira, já deram à luz os seus filhos.
\par 20 E Deus fez bem às parteiras; e o povo aumentou e se tornou muito forte.
\par 21 E, porque as parteiras temeram a Deus, ele lhes constituiu família.
\par 22 Então, ordenou Faraó a todo o seu povo, dizendo: A todos os filhos que nascerem aos hebreus lançareis no Nilo, mas a todas as filhas deixareis viver.

\chapter{2}

\par 1 Foi-se um homem da casa de Levi e casou com uma descendente de Levi.
\par 2 E a mulher concebeu e deu à luz um filho; e, vendo que era formoso, escondeu-o por três meses.
\par 3 Não podendo, porém, escondê-lo por mais tempo, tomou um cesto de junco, calafetou-o com betume e piche e, pondo nele o menino, largou-o no carriçal à beira do rio.
\par 4 A irmã do menino ficou de longe, para observar o que lhe haveria de suceder.
\par 5 Desceu a filha de Faraó para se banhar no rio, e as suas donzelas passeavam pela beira do rio; vendo ela o cesto no carriçal, enviou a sua criada e o tomou.
\par 6 Abrindo-o, viu a criança; e eis que o menino chorava. Teve compaixão dele e disse: Este é menino dos hebreus.
\par 7 Então, disse sua irmã à filha de Faraó: Queres que eu vá chamar uma das hebréias que sirva de ama e te crie a criança?
\par 8 Respondeu-lhe a filha de Faraó: Vai. Saiu, pois, a moça e chamou a mãe do menino.
\par 9 Então, lhe disse a filha de Faraó: Leva este menino e cria-mo; pagar-te-ei o teu salário. A mulher tomou o menino e o criou.
\par 10 Sendo o menino já grande, ela o trouxe à filha de Faraó, da qual passou ele a ser filho. Esta lhe chamou Moisés e disse: Porque das águas o tirei.
\par 11 Naqueles dias, sendo Moisés já homem, saiu a seus irmãos e viu os seus labores penosos; e viu que certo egípcio espancava um hebreu, um do seu povo.
\par 12 Olhou de um e de outro lado, e, vendo que não havia ali ninguém, matou o egípcio, e o escondeu na areia.
\par 13 Saiu no dia seguinte, e eis que dois hebreus estavam brigando; e disse ao culpado: Por que espancas o teu próximo?
\par 14 O qual respondeu: Quem te pôs por príncipe e juiz sobre nós? Pensas matar-me, como mataste o egípcio? Temeu, pois, Moisés e disse: Com certeza o descobriram.
\par 15 Informado desse caso, procurou Faraó matar a Moisés; porém Moisés fugiu da presença de Faraó e se deteve na terra de Midiã; e assentou-se junto a um poço.
\par 16 O sacerdote de Midiã tinha sete filhas, as quais vieram a tirar água e encheram os bebedouros para dar de beber ao rebanho de seu pai.
\par 17 Então, vieram os pastores e as enxotaram dali; Moisés, porém, se levantou, e as defendeu, e deu de beber ao rebanho.
\par 18 Tendo elas voltado a Reuel, seu pai, este lhes perguntou: Por que viestes, hoje, mais cedo?
\par 19 Responderam elas: Um egípcio nos livrou das mãos dos pastores, e ainda nos tirou água, e deu de beber ao rebanho.
\par 20 E onde está ele?, disse às filhas; por que deixastes lá o homem? Chamai-o para que coma pão.
\par 21 Moisés consentiu em morar com aquele homem; e ele deu a Moisés sua filha Zípora,
\par 22 a qual deu à luz um filho, a quem ele chamou Gérson, porque disse: Sou peregrino em terra estranha.
\par 23 Decorridos muitos dias, morreu o rei do Egito; os filhos de Israel gemiam sob a servidão e por causa dela clamaram, e o seu clamor subiu a Deus.
\par 24 Ouvindo Deus o seu gemido, lembrou-se da sua aliança com Abraão, com Isaque e com Jacó.
\par 25 E viu Deus os filhos de Israel e atentou para a sua condição.

\chapter{3}

\par 1 Apascentava Moisés o rebanho de Jetro, seu sogro, sacerdote de Midiã; e, levando o rebanho para o lado ocidental do deserto, chegou ao monte de Deus, a Horebe.
\par 2 Apareceu-lhe o Anjo do SENHOR numa chama de fogo, no meio de uma sarça; Moisés olhou, e eis que a sarça ardia no fogo e a sarça não se consumia.
\par 3 Então, disse consigo mesmo: Irei para lá e verei essa grande maravilha; por que a sarça não se queima?
\par 4 Vendo o SENHOR que ele se voltava para ver, Deus, do meio da sarça, o chamou e disse: Moisés! Moisés! Ele respondeu: Eis-me aqui!
\par 5 Deus continuou: Não te chegues para cá; tira as sandálias dos pés, porque o lugar em que estás é terra santa.
\par 6 Disse mais: Eu sou o Deus de teu pai, o Deus de Abraão, o Deus de Isaque e o Deus de Jacó. Moisés escondeu o rosto, porque temeu olhar para Deus.
\par 7 Disse ainda o SENHOR: Certamente, vi a aflição do meu povo, que está no Egito, e ouvi o seu clamor por causa dos seus exatores. Conheço-lhe o sofrimento;
\par 8 por isso, desci a fim de livrá-lo da mão dos egípcios e para fazê-lo subir daquela terra a uma terra boa e ampla, terra que mana leite e mel; o lugar do cananeu, do heteu, do amorreu, do ferezeu, do heveu e do jebuseu.
\par 9 Pois o clamor dos filhos de Israel chegou até mim, e também vejo a opressão com que os egípcios os estão oprimindo.
\par 10 Vem, agora, e eu te enviarei a Faraó, para que tires o meu povo, os filhos de Israel, do Egito.
\par 11 Então, disse Moisés a Deus: Quem sou eu para ir a Faraó e tirar do Egito os filhos de Israel?
\par 12 Deus lhe respondeu: Eu serei contigo; e este será o sinal de que eu te enviei: depois de haveres tirado o povo do Egito, servireis a Deus neste monte.
\par 13 Disse Moisés a Deus: Eis que, quando eu vier aos filhos de Israel e lhes disser: O Deus de vossos pais me enviou a vós outros; e eles me perguntarem: Qual é o seu nome? Que lhes direi?
\par 14 Disse Deus a Moisés: EU SOU O QUE SOU. Disse mais: Assim dirás aos filhos de Israel: EU SOU me enviou a vós outros.
\par 15 Disse Deus ainda mais a Moisés: Assim dirás aos filhos de Israel: O SENHOR, o Deus de vossos pais, o Deus de Abraão, o Deus de Isaque e o Deus de Jacó, me enviou a vós outros; este é o meu nome eternamente, e assim serei lembrado de geração em geração.
\par 16 Vai, ajunta os anciãos de Israel e dize-lhes: O SENHOR, o Deus de vossos pais, o Deus de Abraão, o Deus de Isaque e o Deus de Jacó, me apareceu, dizendo: Em verdade vos tenho visitado e visto o que vos tem sido feito no Egito.
\par 17 Portanto, disse eu: Far-vos-ei subir da aflição do Egito para a terra do cananeu, do heteu, do amorreu, do ferezeu, do heveu e do jebuseu, para uma terra que mana leite e mel.
\par 18 E ouvirão a tua voz; e irás, com os anciãos de Israel, ao rei do Egito e lhe dirás: O SENHOR, o Deus dos hebreus, nos encontrou. Agora, pois, deixa-nos ir caminho de três dias para o deserto, a fim de que sacrifiquemos ao SENHOR, nosso Deus.
\par 19 Eu sei, porém, que o rei do Egito não vos deixará ir se não for obrigado por mão forte.
\par 20 Portanto, estenderei a mão e ferirei o Egito com todos os meus prodígios que farei no meio dele; depois, vos deixará ir.
\par 21 Eu darei mercê a este povo aos olhos dos egípcios; e, quando sairdes, não será de mãos vazias.
\par 22 Cada mulher pedirá à sua vizinha e à sua hóspeda jóias de prata, e jóias de ouro, e vestimentas; as quais poreis sobre vossos filhos e sobre vossas filhas; e despojareis os egípcios.

\chapter{4}

\par 1 Respondeu Moisés: Mas eis que não crerão, nem acudirão à minha voz, pois dirão: O SENHOR não te apareceu.
\par 2 Perguntou-lhe o SENHOR: Que é isso que tens na mão? Respondeu-lhe: Um bordão.
\par 3 Então, lhe disse: Lança-o na terra. Ele o lançou na terra, e o bordão virou uma serpente. E Moisés fugia dela.
\par 4 Disse o SENHOR a Moisés: Estende a mão e pega-lhe pela cauda (estendeu ele a mão, pegou-lhe pela cauda, e ela se tornou em bordão);
\par 5 para que creiam que te apareceu o SENHOR, Deus de seus pais, o Deus de Abraão, o Deus de Isaque e o Deus de Jacó.
\par 6 Disse-lhe mais o SENHOR: Mete, agora, a mão no peito. Ele o fez; e, tirando-a, eis que a mão estava leprosa, branca como a neve.
\par 7 Disse ainda o SENHOR: Torna a meter a mão no peito. Ele a meteu no peito, novamente; e, quando a tirou, eis que se havia tornado como o restante de sua carne.
\par 8 Se eles te não crerem, nem atenderem à evidência do primeiro sinal, talvez crerão na evidência do segundo.
\par 9 Se nem ainda crerem mediante estes dois sinais, nem te ouvirem a voz, tomarás das águas do rio e as derramarás na terra seca; e as águas que do rio tomares tornar-se-ão em sangue sobre a terra.
\par 10 Então, disse Moisés ao SENHOR: Ah! Senhor! Eu nunca fui eloqüente, nem outrora, nem depois que falaste a teu servo; pois sou pesado de boca e pesado de língua.
\par 11 Respondeu-lhe o SENHOR: Quem fez a boca do homem? Ou quem faz o mudo, ou o surdo, ou o que vê, ou o cego? Não sou eu, o SENHOR?
\par 12 Vai, pois, agora, e eu serei com a tua boca e te ensinarei o que hás de falar.
\par 13 Ele, porém, respondeu: Ah! Senhor! Envia aquele que hás de enviar, menos a mim.
\par 14 Então, se acendeu a ira do SENHOR contra Moisés, e disse: Não é Arão, o levita, teu irmão? Eu sei que ele fala fluentemente; e eis que ele sai ao teu encontro e, vendo-te, se alegrará em seu coração.
\par 15 Tu, pois, lhe falarás e lhe porás na boca as palavras; eu serei com a tua boca e com a dele e vos ensinarei o que deveis fazer.
\par 16 Ele falará por ti ao povo; ele te será por boca, e tu lhe serás por Deus.
\par 17 Toma, pois, este bordão na mão, com o qual hás de fazer os sinais.
\par 18 Saindo Moisés, voltou para Jetro, seu sogro, e lhe disse: Deixa-me ir, voltar a meus irmãos que estão no Egito para ver se ainda vivem. Disse-lhe Jetro: Vai-te em paz.
\par 19 Disse também o SENHOR a Moisés, em Midiã: Vai, torna para o Egito, porque são mortos todos os que procuravam tirar-te a vida.
\par 20 Tomou, pois, Moisés a sua mulher e os seus filhos; fê-los montar num jumento e voltou para a terra do Egito. Moisés levava na mão o bordão de Deus.
\par 21 Disse o SENHOR a Moisés: Quando voltares ao Egito, vê que faças diante de Faraó todos os milagres que te hei posto na mão; mas eu lhe endurecerei o coração, para que não deixe ir o povo.
\par 22 Dirás a Faraó: Assim diz o SENHOR: Israel é meu filho, meu primogênito.
\par 23 Digo-te, pois: deixa ir meu filho, para que me sirva; mas, se recusares deixá-lo ir, eis que eu matarei teu filho, teu primogênito.
\par 24 Estando Moisés no caminho, numa estalagem, encontrou-o o SENHOR e o quis matar.
\par 25 Então, Zípora tomou uma pedra aguda, cortou o prepúcio de seu filho, lançou-o aos pés de Moisés e lhe disse: Sem dúvida, tu és para mim esposo sanguinário.
\par 26 Assim, o SENHOR o deixou. Ela disse: Esposo sanguinário, por causa da circuncisão.
\par 27 Disse também o SENHOR a Arão: Vai ao deserto para te encontrares com Moisés. Ele foi e, encontrando-o no monte de Deus, o beijou.
\par 28 Relatou Moisés a Arão todas as palavras do SENHOR, com as quais o enviara, e todos os sinais que lhe mandara.
\par 29 Então, se foram Moisés e Arão e ajuntaram todos os anciãos dos filhos de Israel.
\par 30 Arão falou todas as palavras que o SENHOR tinha dito a Moisés, e este fez os sinais à vista do povo.
\par 31 E o povo creu; e, tendo ouvido que o SENHOR havia visitado os filhos de Israel e lhes vira a aflição, inclinaram-se e o adoraram.

\chapter{5}

\par 1 Depois, foram Moisés e Arão e disseram a Faraó: Assim diz o SENHOR, Deus de Israel: Deixa ir o meu povo, para que me celebre uma festa no deserto.
\par 2 Respondeu Faraó: Quem é o SENHOR para que lhe ouça eu a voz e deixe ir a Israel? Não conheço o SENHOR, nem tampouco deixarei ir a Israel.
\par 3 Eles prosseguiram: O Deus dos hebreus nos encontrou; deixa-nos ir, pois, caminho de três dias ao deserto, para que ofereçamos sacrifícios ao SENHOR, nosso Deus, e não venha ele sobre nós com pestilência ou com espada.
\par 4 Então, lhes disse o rei do Egito: Por que, Moisés e Arão, por que interrompeis o povo no seu trabalho? Ide às vossas tarefas.
\par 5 Disse também Faraó: O povo da terra já é muito, e vós o distraís das suas tarefas.
\par 6 Naquele mesmo dia, pois, deu ordem Faraó aos superintendentes do povo e aos seus capatazes, dizendo:
\par 7 Daqui em diante não torneis a dar palha ao povo, para fazer tijolos, como antes; eles mesmos que vão e ajuntem para si a palha.
\par 8 E exigireis deles a mesma conta de tijolos que antes faziam; nada diminuireis dela; estão ociosos e, por isso, clamam: Vamos e sacrifiquemos ao nosso Deus.
\par 9 Agrave-se o serviço sobre esses homens, para que nele se apliquem e não dêem ouvidos a palavras mentirosas.
\par 10 Então, saíram os superintendentes do povo e seus capatazes e falaram ao povo: Assim diz Faraó: Não vos darei palha.
\par 11 Ide vós mesmos e ajuntai palha onde a puderdes achar; porque nada se diminuirá do vosso trabalho.
\par 12 Então, o povo se espalhou por toda a terra do Egito a ajuntar restolho em lugar de palha.
\par 13 Os superintendentes os apertavam, dizendo: Acabai vossa obra, a tarefa do dia, como quando havia palha.
\par 14 E foram açoitados os capatazes dos filhos de Israel, que os superintendentes de Faraó tinham posto sobre eles; e os superintendentes lhes diziam: Por que não acabastes nem ontem, nem hoje a vossa tarefa, fazendo tijolos como antes?
\par 15 Então, foram os capatazes dos filhos de Israel e clamaram a Faraó, dizendo: Por que tratas assim a teus servos?
\par 16 Palha não se dá a teus servos, e nos dizem: Fazei tijolos. Eis que teus servos são açoitados; porém o teu próprio povo é que tem a culpa.
\par 17 Mas ele respondeu: Estais ociosos, estais ociosos; por isso, dizeis: Vamos, sacrifiquemos ao SENHOR.
\par 18 Ide, pois, agora, e trabalhai; palha, porém, não se vos dará; contudo, dareis a mesma quantidade de tijolos.
\par 19 Então, os capatazes dos filhos de Israel se viram em aperto, porquanto se lhes dizia: Nada diminuireis dos vossos tijolos, da vossa tarefa diária.
\par 20 Quando saíram da presença de Faraó, encontraram Moisés e Arão, que estavam à espera deles;
\par 21 e lhes disseram: Olhe o SENHOR para vós outros e vos julgue, porquanto nos fizestes odiosos aos olhos de Faraó e diante dos seus servos, dando-lhes a espada na mão para nos matar.
\par 22 Então, Moisés, tornando-se ao SENHOR, disse: Ó Senhor, por que afligiste este povo? Por que me enviaste?
\par 23 Pois, desde que me apresentei a Faraó, para falar-lhe em teu nome, ele tem maltratado este povo; e tu, de nenhuma sorte, livraste o teu povo.

\chapter{6}

\par 1 Disse o SENHOR a Moisés: Agora, verás o que hei de fazer a Faraó; pois, por mão poderosa, os deixará ir e, por mão poderosa, os lançará fora da sua terra.
\par 2 Falou mais Deus a Moisés e lhe disse: Eu sou o SENHOR.
\par 3 Apareci a Abraão, a Isaque e a Jacó como Deus Todo-Poderoso; mas pelo meu nome, O SENHOR, não lhes fui conhecido.
\par 4 Também estabeleci a minha aliança com eles, para dar-lhes a terra de Canaã, a terra em que habitaram como peregrinos.
\par 5 Ainda ouvi os gemidos dos filhos de Israel, os quais os egípcios escravizam, e me lembrei da minha aliança.
\par 6 Portanto, dize aos filhos de Israel: eu sou o SENHOR, e vos tirarei de debaixo das cargas do Egito, e vos livrarei da sua servidão, e vos resgatarei com braço estendido e com grandes manifestações de julgamento.
\par 7 Tomar-vos-ei por meu povo e serei vosso Deus; e sabereis que eu sou o SENHOR, vosso Deus, que vos tiro de debaixo das cargas do Egito.
\par 8 E vos levarei à terra a qual jurei dar a Abraão, a Isaque e a Jacó; e vo-la darei como possessão. Eu sou o SENHOR.
\par 9 Desse modo falou Moisés aos filhos de Israel, mas eles não atenderam a Moisés, por causa da ânsia de espírito e da dura escravidão.
\par 10 Falou mais o SENHOR a Moisés, dizendo:
\par 11 Vai ter com Faraó, rei do Egito, e fala-lhe que deixe sair de sua terra os filhos de Israel.
\par 12 Moisés, porém, respondeu ao SENHOR, dizendo: Eis que os filhos de Israel não me têm ouvido; como, pois, me ouvirá Faraó? E não sei falar bem.
\par 13 Não obstante, falou o SENHOR a Moisés e a Arão e lhes deu mandamento para os filhos de Israel e para Faraó, rei do Egito, a fim de que tirassem os filhos de Israel da terra do Egito.
\par 14 São estes os chefes das famílias: os filhos de Rúben, o primogênito de Israel: Enoque, Palu, Hezrom e Carmi; são estas as famílias de Rúben.
\par 15 Os filhos de Simeão: Jemuel, Jamim, Oade, Jaquim, Zoar e Saul, filho de uma cananéia; são estas as famílias de Simeão.
\par 16 São estes os nomes dos filhos de Levi, segundo as suas gerações: Gérson, Coate e Merari; e os anos da vida de Levi foram cento e trinta e sete.
\par 17 Os filhos de Gérson: Libni e Simei, segundo as suas famílias.
\par 18 Os filhos de Coate: Anrão, Isar, Hebrom e Uziel; e os anos da vida de Coate foram cento e trinta e três.
\par 19 Os filhos de Merari: Mali e Musi; são estas as famílias de Levi, segundo as suas gerações.
\par 20 Anrão tomou por mulher a Joquebede, sua tia; e ela lhe deu a Arão e Moisés; e os anos da vida de Anrão foram cento e trinta e sete.
\par 21 Os filhos de Isar: Corá, Nefegue e Zicri.
\par 22 Os filhos de Uziel: Misael, Elzafã e Sitri.
\par 23 Arão tomou por mulher a Eliseba, filha de Aminadabe, irmã de Naassom; e ela lhe deu à luz Nadabe, Abiú, Eleazar e Itamar.
\par 24 Os filhos de Corá: Assir, Elcana e Abiasafe; são estas as famílias dos coraítas.
\par 25 Eleazar, filho de Arão, tomou por mulher, para si, uma das filhas de Putiel; e ela lhe deu à luz Finéias; são estes os chefes de suas casas, segundo as suas famílias.
\par 26 São estes Arão e Moisés, aos quais o SENHOR disse: Tirai os filhos de Israel da terra do Egito, segundo as suas hostes.
\par 27 São estes que falaram a Faraó, rei do Egito, a fim de tirarem do Egito os filhos de Israel; são estes Moisés e Arão.
\par 28 No dia em que o SENHOR falou a Moisés na terra do Egito,
\par 29 disse o SENHOR a Moisés: Eu sou o SENHOR; dize a Faraó, rei do Egito, tudo o que eu te digo.
\par 30 Respondeu Moisés na presença do SENHOR: Eu não sei falar bem; como, pois, me ouvirá Faraó?

\chapter{7}

\par 1 Então, disse o SENHOR a Moisés: Vê que te constituí como Deus sobre Faraó, e Arão, teu irmão, será teu profeta.
\par 2 Tu falarás tudo o que eu te ordenar; e Arão, teu irmão, falará a Faraó, para que deixe ir da sua terra os filhos de Israel.
\par 3 Eu, porém, endurecerei o coração de Faraó e multiplicarei na terra do Egito os meus sinais e as minhas maravilhas.
\par 4 Faraó não vos ouvirá; e eu porei a mão sobre o Egito e farei sair as minhas hostes, o meu povo, os filhos de Israel, da terra do Egito, com grandes manifestações de julgamento.
\par 5 Saberão os egípcios que eu sou o SENHOR, quando estender eu a mão sobre o Egito e tirar do meio deles os filhos de Israel.
\par 6 Assim fez Moisés e Arão; como o SENHOR lhes ordenara, assim fizeram.
\par 7 Era Moisés de oitenta anos, e Arão, de oitenta e três, quando falaram a Faraó.
\par 8 Falou o SENHOR a Moisés e a Arão:
\par 9 Quando Faraó vos disser: Fazei milagres que vos acreditem, dirás a Arão: Toma o teu bordão e lança-o diante de Faraó; e o bordão se tornará em serpente.
\par 10 Então, Moisés e Arão se chegaram a Faraó e fizeram como o SENHOR lhes ordenara; lançou Arão o seu bordão diante de Faraó e diante dos seus oficiais, e ele se tornou em serpente.
\par 11 Faraó, porém, mandou vir os sábios e encantadores; e eles, os sábios do Egito, fizeram também o mesmo com as suas ciências ocultas.
\par 12 Pois lançaram eles cada um o seu bordão, e eles se tornaram em serpentes; mas o bordão de Arão devorou os bordões deles.
\par 13 Todavia, o coração de Faraó se endureceu, e não os ouviu, como o SENHOR tinha dito.
\par 14 Disse o SENHOR a Moisés: O coração de Faraó está obstinado; recusa deixar ir o povo.
\par 15 Vai ter com Faraó pela manhã; ele sairá às águas; estarás à espera dele na beira do rio, tomarás na mão o bordão que se tornou em serpente
\par 16 e lhe dirás: O SENHOR, o Deus dos hebreus, me enviou a ti para te dizer: Deixa ir o meu povo, para que me sirva no deserto; e, até agora, não tens ouvido.
\par 17 Assim diz o SENHOR: Nisto saberás que eu sou o SENHOR: com este bordão que tenho na mão ferirei as águas do rio, e se tornarão em sangue.
\par 18 Os peixes que estão no rio morrerão, o rio cheirará mal, e os egípcios terão nojo de beber água do rio.
\par 19 Disse mais o SENHOR a Moisés: Dize a Arão: toma o teu bordão e estende a mão sobre as águas do Egito, sobre os seus rios, sobre os seus canais, sobre as suas lagoas e sobre todos os seus reservatórios, para que se tornem em sangue; haja sangue em toda a terra do Egito, tanto nos vasos de madeira como nos de pedra.
\par 20 Fizeram Moisés e Arão como o SENHOR lhes havia ordenado: Arão, levantando o bordão, feriu as águas que estavam no rio, à vista de Faraó e seus oficiais; e toda a água do rio se tornou em sangue.
\par 21 De sorte que os peixes que estavam no rio morreram, o rio cheirou mal, e os egípcios não podiam beber a água do rio; e houve sangue por toda a terra do Egito.
\par 22 Porém os magos do Egito fizeram também o mesmo com as suas ciências ocultas; de maneira que o coração de Faraó se endureceu, e não os ouviu, como o SENHOR tinha dito.
\par 23 Virou-se Faraó e foi para casa; nem ainda isso considerou o seu coração.
\par 24 Todos os egípcios cavaram junto ao rio para encontrar água que beber, pois das águas do rio não podiam beber.
\par 25 Assim se passaram sete dias, depois que o SENHOR feriu o rio.

\chapter{8}

\par 1 Depois, disse o SENHOR a Moisés: Chega-te a Faraó e dize-lhe: Assim diz o SENHOR: Deixa ir o meu povo, para que me sirva.
\par 2 Se recusares deixá-lo ir, eis que castigarei com rãs todos os teus territórios.
\par 3 O rio produzirá rãs em abundância, que subirão e entrarão em tua casa, e no teu quarto de dormir, e sobre o teu leito, e nas casas dos teus oficiais, e sobre o teu povo, e nos teus fornos, e nas tuas amassadeiras.
\par 4 As rãs virão sobre ti, sobre o teu povo e sobre todos os teus oficiais.
\par 5 Disse mais o SENHOR a Moisés: Dize a Arão: Estende a mão com o teu bordão sobre os rios, sobre os canais e sobre as lagoas e faze subir rãs sobre a terra do Egito.
\par 6 Arão estendeu a mão sobre as águas do Egito, e subiram rãs e cobriram a terra do Egito.
\par 7 Então, os magos fizeram o mesmo com suas ciências ocultas e fizeram aparecer rãs sobre a terra do Egito.
\par 8 Chamou Faraó a Moisés e a Arão e lhes disse: Rogai ao SENHOR que tire as rãs de mim e do meu povo; então, deixarei ir o povo, para que ofereça sacrifícios ao SENHOR.
\par 9 Falou Moisés a Faraó: Digna-te dizer-me quando é que hei de rogar por ti, pelos teus oficiais e pelo teu povo, para que as rãs sejam retiradas de ti e das tuas casas e fiquem somente no rio.
\par 10 Ele respondeu: Amanhã. Moisés disse: Seja conforme a tua palavra, para que saibas que ninguém há como o SENHOR, nosso Deus.
\par 11 Retirar-se-ão as rãs de ti, e das tuas casas, e dos teus oficiais, e do teu povo; ficarão somente no rio.
\par 12 Então, saíram Moisés e Arão da presença de Faraó; e Moisés clamou ao SENHOR por causa das rãs, conforme combinara com Faraó.
\par 13 E o SENHOR fez conforme a palavra de Moisés; morreram as rãs nas casas, nos pátios e nos campos.
\par 14 Ajuntaram-nas em montões e montões, e a terra cheirou mal.
\par 15 Vendo, porém, Faraó que havia alívio, continuou de coração endurecido e não os ouviu, como o SENHOR tinha dito.
\par 16 Disse o SENHOR a Moisés: Dize a Arão: Estende o teu bordão e fere o pó da terra, para que se torne em piolhos por toda a terra do Egito.
\par 17 Fizeram assim; Arão estendeu a mão com seu bordão e feriu o pó da terra, e houve muitos piolhos nos homens e no gado; todo o pó da terra se tornou em piolhos por toda a terra do Egito.
\par 18 E fizeram os magos o mesmo com suas ciências ocultas para produzirem piolhos, porém não o puderam; e havia piolhos nos homens e no gado.
\par 19 Então, disseram os magos a Faraó: Isto é o dedo de Deus. Porém o coração de Faraó se endureceu, e não os ouviu, como o SENHOR tinha dito.
\par 20 Disse o SENHOR a Moisés: Levanta-te pela manhã cedo e apresenta-te a Faraó; eis que ele sairá às águas; e dize-lhe: Assim diz o SENHOR: Deixa ir o meu povo, para que me sirva.
\par 21 Do contrário, se tu não deixares ir o meu povo, eis que eu enviarei enxames de moscas sobre ti, e sobre os teus oficiais, e sobre o teu povo, e nas tuas casas; e as casas dos egípcios se encherão destes enxames, e também a terra em que eles estiverem.
\par 22 Naquele dia, separarei a terra de Gósen, em que habita o meu povo, para que nela não haja enxames de moscas, e saibas que eu sou o SENHOR no meio desta terra.
\par 23 Farei distinção entre o meu povo e o teu povo; amanhã se dará este sinal.
\par 24 Assim fez o SENHOR; e vieram grandes enxames de moscas à casa de Faraó, e às casas dos seus oficiais, e sobre toda a terra do Egito; e a terra ficou arruinada com estes enxames.
\par 25 Chamou Faraó a Moisés e a Arão e disse: Ide, oferecei sacrifícios ao vosso Deus nesta terra.
\par 26 Respondeu Moisés: Não convém que façamos assim porque ofereceríamos ao SENHOR, nosso Deus, sacrifícios abomináveis aos egípcios; eis que, se oferecermos tais sacrifícios perante os seus olhos, não nos apedrejarão eles?
\par 27 Temos de ir caminho de três dias ao deserto e ofereceremos sacrifícios ao SENHOR, nosso Deus, como ele nos disser.
\par 28 Então, disse Faraó: Deixar-vos-ei ir, para que ofereçais sacrifícios ao SENHOR, vosso Deus, no deserto; somente que, saindo, não vades muito longe; orai também por mim.
\par 29 Respondeu-lhe Moisés: Eis que saio da tua presença e orarei ao SENHOR; amanhã, estes enxames de moscas se retirarão de Faraó, dos seus oficiais e do seu povo; somente que Faraó não mais me engane, não deixando ir o povo para que ofereça sacrifícios ao SENHOR.
\par 30 Então, saiu Moisés da presença de Faraó e orou ao SENHOR.
\par 31 E fez o SENHOR conforme a palavra de Moisés, e os enxames de moscas se retiraram de Faraó, dos seus oficiais e do seu povo; não ficou uma só mosca.
\par 32 Mas ainda esta vez endureceu Faraó o coração e não deixou ir o povo.

\chapter{9}

\par 1 Disse o SENHOR a Moisés: Apresenta-te a Faraó e dize-lhe: Assim diz o SENHOR, o Deus dos hebreus: Deixa ir o meu povo, para que me sirva.
\par 2 Porque, se recusares deixá-los ir e ainda por força os detiveres,
\par 3 eis que a mão do SENHOR será sobre o teu rebanho, que está no campo, sobre os cavalos, sobre os jumentos, sobre os camelos, sobre o gado e sobre as ovelhas, com pestilência gravíssima.
\par 4 E o SENHOR fará distinção entre os rebanhos de Israel e o rebanho do Egito, para que nada morra de tudo o que pertence aos filhos de Israel.
\par 5 O SENHOR designou certo tempo, dizendo: Amanhã, fará o SENHOR isto na terra.
\par 6 E o SENHOR o fez no dia seguinte, e todo o rebanho dos egípcios morreu; porém, do rebanho dos israelitas, não morreu nem um.
\par 7 Faraó mandou ver, e eis que do rebanho de Israel não morrera nem um sequer; porém o coração de Faraó se endureceu, e não deixou ir o povo.
\par 8 Então, disse o SENHOR a Moisés e a Arão: Apanhai mãos cheias de cinza de forno, e Moisés atire-a para o céu diante de Faraó.
\par 9 Ela se tornará em pó miúdo sobre toda a terra do Egito e se tornará em tumores que se arrebentem em úlceras nos homens e nos animais, por toda a terra do Egito.
\par 10 Eles tomaram cinza de forno e se apresentaram a Faraó; Moisés atirou-a para o céu, e ela se tornou em tumores que se arrebentavam em úlceras nos homens e nos animais,
\par 11 de maneira que os magos não podiam permanecer diante de Moisés, por causa dos tumores; porque havia tumores nos magos e em todos os egípcios.
\par 12 Porém o SENHOR endureceu o coração de Faraó, e este não os ouviu, como o SENHOR tinha dito a Moisés.
\par 13 Disse o SENHOR a Moisés: Levanta-te pela manhã cedo, apresenta-te a Faraó e dize-lhe: Assim diz o SENHOR, o Deus dos hebreus: Deixa ir o meu povo, para que me sirva.
\par 14 Pois esta vez enviarei todas as minhas pragas sobre o teu coração, e sobre os teus oficiais, e sobre o teu povo, para que saibas que não há quem me seja semelhante em toda a terra.
\par 15 Pois já eu poderia ter estendido a mão para te ferir a ti e o teu povo com pestilência, e terias sido cortado da terra;
\par 16 mas, deveras, para isso te hei mantido, a fim de mostrar-te o meu poder, e para que seja o meu nome anunciado em toda a terra.
\par 17 Ainda te levantas contra o meu povo, para não deixá-lo ir?
\par 18 Eis que amanhã, por este tempo, farei cair mui grave chuva de pedras, como nunca houve no Egito, desde o dia em que foi fundado até hoje.
\par 19 Agora, pois, manda recolher o teu gado e tudo o que tens no campo; todo homem e animal que se acharem no campo e não se recolherem a casa, em caindo sobre eles a chuva de pedras, morrerão.
\par 20 Quem dos oficiais de Faraó temia a palavra do SENHOR fez fugir os seus servos e o seu gado para as casas;
\par 21 aquele, porém, que não se importava com a palavra do SENHOR deixou ficar no campo os seus servos e o seu gado.
\par 22 Então, disse o SENHOR a Moisés: Estende a mão para o céu, e cairá chuva de pedras em toda a terra do Egito, sobre homens, sobre animais e sobre toda planta do campo na terra do Egito.
\par 23 E Moisés estendeu o seu bordão para o céu; o SENHOR deu trovões e chuva de pedras, e fogo desceu sobre a terra; e fez o SENHOR cair chuva de pedras sobre a terra do Egito.
\par 24 De maneira que havia chuva de pedras e fogo misturado com a chuva de pedras tão grave, qual nunca houve em toda a terra do Egito, desde que veio a ser uma nação.
\par 25 Por toda a terra do Egito a chuva de pedras feriu tudo quanto havia no campo, tanto homens como animais; feriu também a chuva de pedras toda planta do campo e quebrou todas as árvores do campo.
\par 26 Somente na terra de Gósen, onde estavam os filhos de Israel, não havia chuva de pedras.
\par 27 Então, Faraó mandou chamar a Moisés e a Arão e lhes disse: Esta vez pequei; o SENHOR é justo, porém eu e o meu povo somos ímpios.
\par 28 Orai ao SENHOR; pois já bastam estes grandes trovões e a chuva de pedras. Eu vos deixarei ir, e não ficareis mais aqui.
\par 29 Respondeu-lhe Moisés: Em saindo eu da cidade, estenderei as mãos ao SENHOR; os trovões cessarão, e já não haverá chuva de pedras; para que saibas que a terra é do SENHOR.
\par 30 Quanto a ti, porém, e aos teus oficiais, eu sei que ainda não temeis ao SENHOR Deus.
\par 31 (O linho e a cevada foram feridos, pois a cevada já estava na espiga, e o linho, em flor.
\par 32 Porém o trigo e o centeio não sofreram dano, porque ainda não haviam nascido.)
\par 33 Saiu, pois, Moisés da presença de Faraó e da cidade e estendeu as mãos ao SENHOR; cessaram os trovões e a chuva de pedras, e não caiu mais chuva sobre a terra.
\par 34 Tendo visto Faraó que cessaram as chuvas, as pedras e os trovões, tornou a pecar e endureceu o coração, ele e os seus oficiais.
\par 35 E assim Faraó, de coração endurecido, não deixou ir os filhos de Israel, como o SENHOR tinha dito a Moisés.

\chapter{10}

\par 1 Disse o SENHOR a Moisés: Vai ter com Faraó, porque lhe endureci o coração e o coração de seus oficiais, para que eu faça estes meus sinais no meio deles,
\par 2 e para que contes a teus filhos e aos filhos de teus filhos como zombei dos egípcios e quantos prodígios fiz no meio deles, e para que saibais que eu sou o SENHOR.
\par 3 Apresentaram-se, pois, Moisés e Arão perante Faraó e lhe disseram: Assim diz o SENHOR, o Deus dos hebreus: Até quando recusarás humilhar-te perante mim? Deixa ir o meu povo, para que me sirva.
\par 4 Do contrário, se recusares deixar ir o meu povo, eis que amanhã trarei gafanhotos ao teu território;
\par 5 eles cobrirão de tal maneira a face da terra, que dela nada aparecerá; eles comerão o restante que escapou, o que vos resta da chuva de pedras, e comerão toda árvore que vos cresce no campo;
\par 6 e encherão as tuas casas, e as casas de todos os teus oficiais, e as casas de todos os egípcios, como nunca viram teus pais, nem os teus antepassados desde o dia em que se acharam na terra até ao dia de hoje. Virou-se e saiu da presença de Faraó.
\par 7 Então, os oficiais de Faraó lhe disseram: Até quando nos será por cilada este homem? Deixa ir os homens, para que sirvam ao SENHOR, seu Deus. Acaso, não sabes ainda que o Egito está arruinado?
\par 8 Então, Moisés e Arão foram conduzidos à presença de Faraó; e este lhes disse: Ide, servi ao SENHOR, vosso Deus; porém quais são os que hão de ir?
\par 9 Respondeu-lhe Moisés: Havemos de ir com os nossos jovens, e com os nossos velhos, e com os filhos, e com as filhas, e com os nossos rebanhos, e com os nossos gados; havemos de ir, porque temos de celebrar festa ao SENHOR.
\par 10 Replicou-lhes Faraó: Seja o SENHOR convosco, caso eu vos deixe ir e as crianças. Vede, pois tendes conosco más intenções.
\par 11 Não há de ser assim; ide somente vós, os homens, e servi ao SENHOR; pois isso é o que pedistes. E os expulsaram da presença de Faraó.
\par 12 Então, disse o SENHOR a Moisés: Estende a mão sobre a terra do Egito, para que venham os gafanhotos sobre a terra do Egito e comam toda a erva da terra, tudo o que deixou a chuva de pedras.
\par 13 Estendeu, pois, Moisés o seu bordão sobre a terra do Egito, e o SENHOR trouxe sobre a terra um vento oriental todo aquele dia e toda aquela noite; quando amanheceu, o vento oriental tinha trazido os gafanhotos.
\par 14 E subiram os gafanhotos por toda a terra do Egito e pousaram sobre todo o seu território; eram mui numerosos; antes destes, nunca houve tais gafanhotos, nem depois deles virão outros assim.
\par 15 Porque cobriram a superfície de toda a terra, de modo que a terra se escureceu; devoraram toda a erva da terra e todo fruto das árvores que deixara a chuva de pedras; e não restou nada verde nas árvores, nem na erva do campo, em toda a terra do Egito.
\par 16 Então, se apressou Faraó em chamar a Moisés e a Arão e lhes disse: Pequei contra o SENHOR, vosso Deus, e contra vós outros.
\par 17 Agora, pois, peço-vos que me perdoeis o pecado esta vez ainda e que oreis ao SENHOR, vosso Deus, que tire de mim esta morte.
\par 18 E Moisés, tendo saído da presença de Faraó, orou ao SENHOR.
\par 19 Então, o SENHOR fez soprar fortíssimo vento ocidental, o qual levantou os gafanhotos e os lançou no mar Vermelho; nem ainda um só gafanhoto restou em todo o território do Egito.
\par 20 O SENHOR, porém, endureceu o coração de Faraó, e este não deixou ir os filhos de Israel.
\par 21 Então, disse o SENHOR a Moisés: Estende a mão para o céu, e virão trevas sobre a terra do Egito, trevas que se possam apalpar.
\par 22 Estendeu, pois, Moisés a mão para o céu, e houve trevas espessas sobre toda a terra do Egito por três dias;
\par 23 não viram uns aos outros, e ninguém se levantou do seu lugar por três dias; porém todos os filhos de Israel tinham luz nas suas habitações.
\par 24 Então, Faraó chamou a Moisés e lhe disse: Ide, servi ao SENHOR. Fiquem somente os vossos rebanhos e o vosso gado; as vossas crianças irão também convosco.
\par 25 Respondeu Moisés: Também tu nos tens de dar em nossas mãos sacrifícios e holocaustos, que ofereçamos ao SENHOR, nosso Deus.
\par 26 E também os nossos rebanhos irão conosco, nem uma unha ficará; porque deles havemos de tomar, para servir ao SENHOR, nosso Deus, e não sabemos com que havemos de servir ao SENHOR, até que cheguemos lá.
\par 27 O SENHOR, porém, endureceu o coração de Faraó, e este não quis deixá-los ir.
\par 28 Disse, pois, Faraó a Moisés: Retira-te de mim e guarda-te que não mais vejas o meu rosto; porque, no dia em que vires o meu rosto, morrerás.
\par 29 Respondeu-lhe Moisés: Bem disseste; nunca mais tornarei eu a ver o teu rosto.

\chapter{11}

\par 1 Disse o SENHOR a Moisés: Ainda mais uma praga trarei sobre Faraó e sobre o Egito. Então, vos deixará ir daqui; quando vos deixar, é certo que vos expulsará totalmente.
\par 2 Fala, agora, aos ouvidos do povo que todo homem peça ao seu vizinho, e toda mulher, à sua vizinha objetos de prata e de ouro.
\par 3 E o SENHOR fez que o seu povo encontrasse favor da parte dos egípcios; também o homem Moisés era mui famoso na terra do Egito, aos olhos dos oficiais de Faraó e aos olhos do povo.
\par 4 Moisés disse: Assim diz o SENHOR: Cerca da meia-noite passarei pelo meio do Egito.
\par 5 E todo primogênito na terra do Egito morrerá, desde o primogênito de Faraó, que se assenta no seu trono, até ao primogênito da serva que está junto à mó, e todo primogênito dos animais.
\par 6 Haverá grande clamor em toda a terra do Egito, qual nunca houve, nem haverá jamais;
\par 7 porém contra nenhum dos filhos de Israel, desde os homens até aos animais, nem ainda um cão rosnará, para que saibais que o SENHOR fez distinção entre os egípcios e os israelitas.
\par 8 Então, todos estes teus oficiais descerão a mim e se inclinarão perante mim, dizendo: Sai tu e todo o povo que te segue. E, depois disto, sairei. E, ardendo em ira, se retirou da presença de Faraó.
\par 9 Então, disse o SENHOR a Moisés: Faraó não vos ouvirá, para que as minhas maravilhas se multipliquem na terra do Egito.
\par 10 Moisés e Arão fizeram todas essas maravilhas perante Faraó; mas o SENHOR endureceu o coração de Faraó, que não permitiu saíssem da sua terra os filhos de Israel.

\chapter{12}

\par 1 Disse o SENHOR a Moisés e a Arão na terra do Egito:
\par 2 Este mês vos será o principal dos meses; será o primeiro mês do ano.
\par 3 Falai a toda a congregação de Israel, dizendo: Aos dez deste mês, cada um tomará para si um cordeiro, segundo a casa dos pais, um cordeiro para cada família.
\par 4 Mas, se a família for pequena para um cordeiro, então, convidará ele o seu vizinho mais próximo, conforme o número das almas; conforme o que cada um puder comer, por aí calculareis quantos bastem para o cordeiro.
\par 5 O cordeiro será sem defeito, macho de um ano; podereis tomar um cordeiro ou um cabrito;
\par 6 e o guardareis até ao décimo quarto dia deste mês, e todo o ajuntamento da congregação de Israel o imolará no crepúsculo da tarde.
\par 7 Tomarão do sangue e o porão em ambas as ombreiras e na verga da porta, nas casas em que o comerem;
\par 8 naquela noite, comerão a carne assada no fogo; com pães asmos e ervas amargas a comerão.
\par 9 Não comereis do animal nada cru, nem cozido em água, porém assado ao fogo: a cabeça, as pernas e a fressura.
\par 10 Nada deixareis dele até pela manhã; o que, porém, ficar até pela manhã, queimá-lo-eis.
\par 11 Desta maneira o comereis: lombos cingidos, sandálias nos pés e cajado na mão; comê-lo-eis à pressa; é a Páscoa do SENHOR.
\par 12 Porque, naquela noite, passarei pela terra do Egito e ferirei na terra do Egito todos os primogênitos, desde os homens até aos animais; executarei juízo sobre todos os deuses do Egito. Eu sou o SENHOR.
\par 13 O sangue vos será por sinal nas casas em que estiverdes; quando eu vir o sangue, passarei por vós, e não haverá entre vós praga destruidora, quando eu ferir a terra do Egito.
\par 14 Este dia vos será por memorial, e o celebrareis como solenidade ao SENHOR; nas vossas gerações o celebrareis por estatuto perpétuo.
\par 15 Sete dias comereis pães asmos. Logo ao primeiro dia, tirareis o fermento das vossas casas, pois qualquer que comer coisa levedada, desde o primeiro dia até ao sétimo dia, essa pessoa será eliminada de Israel.
\par 16 Ao primeiro dia, haverá para vós outros santa assembléia; também, ao sétimo dia, tereis santa assembléia; nenhuma obra se fará nele, exceto o que diz respeito ao comer; somente isso podereis fazer.
\par 17 Guardai, pois, a Festa dos Pães Asmos, porque, nesse mesmo dia, tirei vossas hostes da terra do Egito; portanto, guardareis este dia nas vossas gerações por estatuto perpétuo.
\par 18 Desde o dia catorze do primeiro mês, à tarde, comereis pães asmos até à tarde do dia vinte e um do mesmo mês.
\par 19 Por sete dias, não se ache nenhum fermento nas vossas casas; porque qualquer que comer pão levedado será eliminado da congregação de Israel, tanto o peregrino como o natural da terra.
\par 20 Nenhuma coisa levedada comereis; em todas as vossas habitações, comereis pães asmos.
\par 21 Chamou, pois, Moisés todos os anciãos de Israel e lhes disse: Escolhei, e tomai cordeiros segundo as vossas famílias, e imolai a Páscoa.
\par 22 Tomai um molho de hissopo, molhai-o no sangue que estiver na bacia e marcai a verga da porta e suas ombreiras com o sangue que estiver na bacia; nenhum de vós saia da porta da sua casa até pela manhã.
\par 23 Porque o SENHOR passará para ferir os egípcios; quando vir, porém, o sangue na verga da porta e em ambas as ombreiras, passará o SENHOR aquela porta e não permitirá ao Destruidor que entre em vossas casas, para vos ferir.
\par 24 Guardai, pois, isto por estatuto para vós outros e para vossos filhos, para sempre.
\par 25 E, uma vez dentro na terra que o SENHOR vos dará, como tem dito, observai este rito.
\par 26 Quando vossos filhos vos perguntarem: Que rito é este?
\par 27 Respondereis: É o sacrifício da Páscoa ao SENHOR, que passou por cima das casas dos filhos de Israel no Egito, quando feriu os egípcios e livrou as nossas casas. Então, o povo se inclinou e adorou.
\par 28 E foram os filhos de Israel e fizeram isso; como o SENHOR ordenara a Moisés e a Arão, assim fizeram.
\par 29 Aconteceu que, à meia-noite, feriu o SENHOR todos os primogênitos na terra do Egito, desde o primogênito de Faraó, que se assentava no seu trono, até ao primogênito do cativo que estava na enxovia, e todos os primogênitos dos animais.
\par 30 Levantou-se Faraó de noite, ele, todos os seus oficiais e todos os egípcios; e fez-se grande clamor no Egito, pois não havia casa em que não houvesse morto.
\par 31 Então, naquela mesma noite, Faraó chamou a Moisés e a Arão e lhes disse: Levantai-vos, saí do meio do meu povo, tanto vós como os filhos de Israel; ide, servi ao SENHOR, como tendes dito.
\par 32 Levai também convosco vossas ovelhas e vosso gado, como tendes dito; ide-vos embora e abençoai-me também a mim.
\par 33 Os egípcios apertavam com o povo, apressando-se em lançá-los fora da terra, pois diziam: Todos morreremos.
\par 34 O povo tomou a sua massa, antes que levedasse, e as suas amassadeiras atadas em trouxas com seus vestidos, sobre os ombros.
\par 35 Fizeram, pois, os filhos de Israel conforme a palavra de Moisés e pediram aos egípcios objetos de prata, e objetos de ouro, e roupas.
\par 36 E o SENHOR fez que seu povo encontrasse favor da parte dos egípcios, de maneira que estes lhes davam o que pediam. E despojaram os egípcios.
\par 37 Assim, partiram os filhos de Israel de Ramessés para Sucote, cerca de seiscentos mil a pé, somente de homens, sem contar mulheres e crianças.
\par 38 Subiu também com eles um misto de gente, ovelhas, gado, muitíssimos animais.
\par 39 E cozeram bolos asmos da massa que levaram do Egito; pois não se tinha levedado, porque foram lançados fora do Egito; não puderam deter-se e não haviam preparado para si provisões.
\par 40 Ora, o tempo que os filhos de Israel habitaram no Egito foi de quatrocentos e trinta anos.
\par 41 Aconteceu que, ao cabo dos quatrocentos e trinta anos, nesse mesmo dia, todas as hostes do SENHOR saíram da terra do Egito.
\par 42 Esta noite se observará ao SENHOR, porque, nela, os tirou da terra do Egito; esta é a noite do SENHOR, que devem todos os filhos de Israel comemorar nas suas gerações.
\par 43 Disse mais o SENHOR a Moisés e a Arão: Esta é a ordenança da Páscoa: nenhum estrangeiro comerá dela.
\par 44 Porém todo escravo comprado por dinheiro, depois de o teres circuncidado, comerá dela.
\par 45 O estrangeiro e o assalariado não comerão dela.
\par 46 O cordeiro há de ser comido numa só casa; da sua carne não levareis fora da casa, nem lhe quebrareis osso nenhum.
\par 47 Toda a congregação de Israel o fará.
\par 48 Porém, se algum estrangeiro se hospedar contigo e quiser celebrar a Páscoa do SENHOR, seja-lhe circuncidado todo macho; e, então, se chegará, e a observará, e será como o natural da terra; mas nenhum incircunciso comerá dela.
\par 49 A mesma lei haja para o natural e para o forasteiro que peregrinar entre vós.
\par 50 Assim fizeram todos os filhos de Israel; como o SENHOR ordenara a Moisés e a Arão, assim fizeram.
\par 51 Naquele mesmo dia, tirou o SENHOR os filhos de Israel do Egito, segundo as suas turmas.

\chapter{13}

\par 1 Disse o SENHOR a Moisés:
\par 2 Consagra-me todo primogênito; todo que abre a madre de sua mãe entre os filhos de Israel, tanto de homens como de animais, é meu.
\par 3 Disse Moisés ao povo: Lembrai-vos deste mesmo dia, em que saístes do Egito, da casa da servidão; pois com mão forte o SENHOR vos tirou de lá; portanto, não comereis pão levedado.
\par 4 Hoje, mês de abibe, estais saindo.
\par 5 Quando o SENHOR te houver introduzido na terra dos cananeus, e dos heteus, e dos amorreus, e dos heveus, e dos jebuseus, a qual jurou a teus pais te dar, terra que mana leite e mel, guardarás este rito neste mês.
\par 6 Sete dias comerás pães asmos; e, ao sétimo dia, haverá solenidade ao SENHOR.
\par 7 Sete dias se comerão pães asmos, e o levedado não se encontrará contigo, nem ainda fermento será encontrado em todo o teu território.
\par 8 Naquele mesmo dia, contarás a teu filho, dizendo: É isto pelo que o SENHOR me fez, quando saí do Egito.
\par 9 E será como sinal na tua mão e por memorial entre teus olhos; para que a lei do SENHOR esteja na tua boca; pois com mão forte o SENHOR te tirou do Egito.
\par 10 Portanto, guardarás esta ordenança no determinado tempo, de ano em ano.
\par 11 Quando o SENHOR te houver introduzido na terra dos cananeus, como te jurou a ti e a teus pais, quando ta houver dado,
\par 12 apartarás para o SENHOR todo que abrir a madre e todo primogênito dos animais que tiveres; os machos serão do SENHOR.
\par 13 Porém todo primogênito da jumenta resgatarás com cordeiro; se o não resgatares, será desnucado; mas todo primogênito do homem entre teus filhos resgatarás.
\par 14 Quando teu filho amanhã te perguntar: Que é isso? Responder-lhe-ás: O SENHOR com mão forte nos tirou da casa da servidão.
\par 15 Pois sucedeu que, endurecendo-se Faraó para não nos deixar sair, o SENHOR matou todos os primogênitos na terra do Egito, desde o primogênito do homem até ao primogênito dos animais; por isso, eu sacrifico ao SENHOR todos os machos que abrem a madre; porém a todo primogênito de meus filhos eu resgato.
\par 16 E isto será como sinal na tua mão e por frontais entre os teus olhos; porque o SENHOR com mão forte nos tirou do Egito.
\par 17 Tendo Faraó deixado ir o povo, Deus não o levou pelo caminho da terra dos filisteus, posto que mais perto, pois disse: Para que, porventura, o povo não se arrependa, vendo a guerra, e torne ao Egito.
\par 18 Porém Deus fez o povo rodear pelo caminho do deserto perto do mar Vermelho; e, arregimentados, subiram os filhos de Israel do Egito.
\par 19 Também levou Moisés consigo os ossos de José, pois havia este feito os filhos de Israel jurarem solenemente, dizendo: Certamente, Deus vos visitará; daqui, pois, levai convosco os meus ossos.
\par 20 Tendo, pois, partido de Sucote, acamparam-se em Etã, à entrada do deserto.
\par 21 O SENHOR ia adiante deles, durante o dia, numa coluna de nuvem, para os guiar pelo caminho; durante a noite, numa coluna de fogo, para os alumiar, a fim de que caminhassem de dia e de noite.
\par 22 Nunca se apartou do povo a coluna de nuvem durante o dia, nem a coluna de fogo durante a noite.

\chapter{14}

\par 1 Disse o SENHOR a Moisés:
\par 2 Fala aos filhos de Israel que retrocedam e se acampem defronte de Pi-Hairote, entre Migdol e o mar, diante de Baal-Zefom; em frente dele vos acampareis junto ao mar.
\par 3 Então, Faraó dirá dos filhos de Israel: Estão desorientados na terra, o deserto os encerrou.
\par 4 Endurecerei o coração de Faraó, para que os persiga, e serei glorificado em Faraó e em todo o seu exército; e saberão os egípcios que eu sou o SENHOR. Eles assim o fizeram.
\par 5 Sendo, pois, anunciado ao rei do Egito que o povo fugia, mudou-se o coração de Faraó e dos seus oficiais contra o povo, e disseram: Que é isto que fizemos, permitindo que Israel nos deixasse de servir?
\par 6 E aprontou Faraó o seu carro e tomou consigo o seu povo;
\par 7 e tomou também seiscentos carros escolhidos e todos os carros do Egito com capitães sobre todos eles.
\par 8 Porque o SENHOR endureceu o coração de Faraó, rei do Egito, para que perseguisse os filhos de Israel; porém os filhos de Israel saíram afoitamente.
\par 9 Perseguiram-nos os egípcios, todos os cavalos e carros de Faraó, e os seus cavalarianos, e o seu exército e os alcançaram acampados junto ao mar, perto de Pi-Hairote, defronte de Baal-Zefom.
\par 10 E, chegando Faraó, os filhos de Israel levantaram os olhos, e eis que os egípcios vinham atrás deles, e temeram muito; então, os filhos de Israel clamaram ao SENHOR.
\par 11 Disseram a Moisés: Será, por não haver sepulcros no Egito, que nos tiraste de lá, para que morramos neste deserto? Por que nos trataste assim, fazendo-nos sair do Egito?
\par 12 Não é isso o que te dissemos no Egito: deixa-nos, para que sirvamos os egípcios? Pois melhor nos fora servir aos egípcios do que morrermos no deserto.
\par 13 Moisés, porém, respondeu ao povo: Não temais; aquietai-vos e vede o livramento do SENHOR que, hoje, vos fará; porque os egípcios, que hoje vedes, nunca mais os tornareis a ver.
\par 14 O SENHOR pelejará por vós, e vós vos calareis.
\par 15 Disse o SENHOR a Moisés: Por que clamas a mim? Dize aos filhos de Israel que marchem.
\par 16 E tu, levanta o teu bordão, estende a mão sobre o mar e divide-o, para que os filhos de Israel passem pelo meio do mar em seco.
\par 17 Eis que endurecerei o coração dos egípcios, para que vos sigam e entrem nele; serei glorificado em Faraó e em todo o seu exército, nos seus carros e nos seus cavalarianos;
\par 18 e os egípcios saberão que eu sou o SENHOR, quando for glorificado em Faraó, nos seus carros e nos seus cavalarianos.
\par 19 Então, o Anjo de Deus, que ia adiante do exército de Israel, se retirou e passou para trás deles; também a coluna de nuvem se retirou de diante deles, e se pôs atrás deles,
\par 20 e ia entre o campo dos egípcios e o campo de Israel; a nuvem era escuridade para aqueles e para este esclarecia a noite; de maneira que, em toda a noite, este e aqueles não puderam aproximar-se.
\par 21 Então, Moisés estendeu a mão sobre o mar, e o SENHOR, por um forte vento oriental que soprou toda aquela noite, fez retirar-se o mar, que se tornou terra seca, e as águas foram divididas.
\par 22 Os filhos de Israel entraram pelo meio do mar em seco; e as águas lhes foram qual muro à sua direita e à sua esquerda.
\par 23 Os egípcios que os perseguiam entraram atrás deles, todos os cavalos de Faraó, os seus carros e os seus cavalarianos, até ao meio do mar.
\par 24 Na vigília da manhã, o SENHOR, na coluna de fogo e de nuvem, viu o acampamento dos egípcios e alvorotou o acampamento dos egípcios;
\par 25 emperrou-lhes as rodas dos carros e fê-los andar dificultosamente. Então, disseram os egípcios: Fujamos da presença de Israel, porque o SENHOR peleja por eles contra os egípcios.
\par 26 Disse o SENHOR a Moisés: Estende a mão sobre o mar, para que as águas se voltem sobre os egípcios, sobre os seus carros e sobre os seus cavalarianos.
\par 27 Então, Moisés estendeu a mão sobre o mar, e o mar, ao romper da manhã, retomou a sua força; os egípcios, ao fugirem, foram de encontro a ele, e o SENHOR derribou os egípcios no meio do mar.
\par 28 E, voltando as águas, cobriram os carros e os cavalarianos de todo o exército de Faraó, que os haviam seguido no mar; nem ainda um deles ficou.
\par 29 Mas os filhos de Israel caminhavam a pé enxuto pelo meio do mar; e as águas lhes eram quais muros, à sua direita e à sua esquerda.
\par 30 Assim, o SENHOR livrou Israel, naquele dia, da mão dos egípcios; e Israel viu os egípcios mortos na praia do mar.
\par 31 E viu Israel o grande poder que o SENHOR exercitara contra os egípcios; e o povo temeu ao SENHOR e confiou no SENHOR e em Moisés, seu servo.

\chapter{15}

\par 1 Então, entoou Moisés e os filhos de Israel este cântico ao SENHOR, e disseram: Cantarei ao SENHOR, porque triunfou gloriosamente; lançou no mar o cavalo e o seu cavaleiro.
\par 2 O SENHOR é a minha força e o meu cântico; ele me foi por salvação; este é o meu Deus; portanto, eu o louvarei; ele é o Deus de meu pai; por isso, o exaltarei.
\par 3 O SENHOR é homem de guerra; SENHOR é o seu nome.
\par 4 Lançou no mar os carros de Faraó e o seu exército; e os seus capitães afogaram-se no mar Vermelho.
\par 5 Os vagalhões os cobriram; desceram às profundezas como pedra.
\par 6 A tua destra, ó SENHOR, é gloriosa em poder; a tua destra, ó SENHOR, despedaça o inimigo.
\par 7 Na grandeza da tua excelência, derribas os que se levantam contra ti; envias o teu furor, que os consome como restolho.
\par 8 Com o resfolgar das tuas narinas, amontoaram-se as águas, as correntes pararam em montão; os vagalhões coalharam-se no coração do mar.
\par 9 O inimigo dizia: Perseguirei, alcançarei, repartirei os despojos; a minha alma se fartará deles, arrancarei a minha espada, e a minha mão os destruirá.
\par 10 Sopraste com o teu vento, e o mar os cobriu; afundaram-se como chumbo em águas impetuosas.
\par 11 Ó SENHOR, quem é como tu entre os deuses? Quem é como tu, glorificado em santidade, terrível em feitos gloriosos, que operas maravilhas?
\par 12 Estendeste a destra; e a terra os tragou.
\par 13 Com a tua beneficência guiaste o povo que salvaste; com a tua força o levaste à habitação da tua santidade.
\par 14 Os povos o ouviram, eles estremeceram; agonias apoderaram-se dos habitantes da Filístia.
\par 15 Ora, os príncipes de Edom se perturbam, dos poderosos de Moabe se apodera temor, esmorecem todos os habitantes de Canaã.
\par 16 Sobre eles cai espanto e pavor; pela grandeza do teu braço, emudecem como pedra; até que passe o teu povo, ó SENHOR, até que passe o povo que adquiriste.
\par 17 Tu o introduzirás e o plantarás no monte da tua herança, no lugar que aparelhaste, ó SENHOR, para a tua habitação, no santuário, ó Senhor, que as tuas mãos estabeleceram.
\par 18 O SENHOR reinará por todo o sempre.
\par 19 Porque os cavalos de Faraó, com os seus carros e com os seus cavalarianos, entraram no mar, e o SENHOR fez tornar sobre eles as águas do mar; mas os filhos de Israel passaram a pé enxuto pelo meio do mar.
\par 20 A profetisa Miriã, irmã de Arão, tomou um tamborim, e todas as mulheres saíram atrás dela com tamborins e com danças.
\par 21 E Miriã lhes respondia: Cantai ao SENHOR, porque gloriosamente triunfou e precipitou no mar o cavalo e o seu cavaleiro.
\par 22 Fez Moisés partir a Israel do mar Vermelho, e saíram para o deserto de Sur; caminharam três dias no deserto e não acharam água.
\par 23 Afinal, chegaram a Mara; todavia, não puderam beber as águas de Mara, porque eram amargas; por isso, chamou-se-lhe Mara.
\par 24 E o povo murmurou contra Moisés, dizendo: Que havemos de beber?
\par 25 Então, Moisés clamou ao SENHOR, e o SENHOR lhe mostrou uma árvore; lançou-a Moisés nas águas, e as águas se tornaram doces. Deu-lhes ali estatutos e uma ordenação, e ali os provou,
\par 26 e disse: Se ouvires atento a voz do SENHOR, teu Deus, e fizeres o que é reto diante dos seus olhos, e deres ouvido aos seus mandamentos, e guardares todos os seus estatutos, nenhuma enfermidade virá sobre ti, das que enviei sobre os egípcios; pois eu sou o SENHOR, que te sara.
\par 27 Então, chegaram a Elim, onde havia doze fontes de água e setenta palmeiras; e se acamparam junto das águas.

\chapter{16}

\par 1 Partiram de Elim, e toda a congregação dos filhos de Israel veio para o deserto de Sim, que está entre Elim e Sinai, aos quinze dias do segundo mês, depois que saíram da terra do Egito.
\par 2 Toda a congregação dos filhos de Israel murmurou contra Moisés e Arão no deserto;
\par 3 disseram-lhes os filhos de Israel: Quem nos dera tivéssemos morrido pela mão do SENHOR, na terra do Egito, quando estávamos sentados junto às panelas de carne e comíamos pão a fartar! Pois nos trouxestes a este deserto, para matardes de fome toda esta multidão.
\par 4 Então, disse o SENHOR a Moisés: Eis que vos farei chover do céu pão, e o povo sairá e colherá diariamente a porção para cada dia, para que eu ponha à prova se anda na minha lei ou não.
\par 5 Dar-se-á que, ao sexto dia, prepararão o que colherem; e será o dobro do que colhem cada dia.
\par 6 Então, disse Moisés e Arão a todos os filhos de Israel: à tarde, sabereis que foi o SENHOR quem vos tirou da terra do Egito,
\par 7 e, pela manhã, vereis a glória do SENHOR, porquanto ouviu as vossas murmurações; pois quem somos nós, para que murmureis contra nós?
\par 8 Prosseguiu Moisés: Será isso quando o SENHOR, à tarde, vos der carne para comer e, pela manhã, pão que vos farte, porquanto o SENHOR ouviu as vossas murmurações, com que vos queixais contra ele; pois quem somos nós? As vossas murmurações não são contra nós, e sim contra o SENHOR.
\par 9 Disse Moisés a Arão: Dize a toda a congregação dos filhos de Israel: Chegai-vos à presença do SENHOR, pois ouviu as vossas murmurações.
\par 10 Quando Arão falava a toda a congregação dos filhos de Israel, olharam para o deserto, e eis que a glória do SENHOR apareceu na nuvem.
\par 11 E o SENHOR disse a Moisés:
\par 12 Tenho ouvido as murmurações dos filhos de Israel; dize-lhes: Ao crepúsculo da tarde, comereis carne, e, pela manhã, vos fartareis de pão, e sabereis que eu sou o SENHOR, vosso Deus.
\par 13 À tarde, subiram codornizes e cobriram o arraial; pela manhã, jazia o orvalho ao redor do arraial.
\par 14 E, quando se evaporou o orvalho que caíra, na superfície do deserto restava uma coisa fina e semelhante a escamas, fina como a geada sobre a terra.
\par 15 Vendo-a os filhos de Israel, disseram uns aos outros: Que é isto? Pois não sabiam o que era. Disse-lhes Moisés: Isto é o pão que o SENHOR vos dá para vosso alimento.
\par 16 Eis o que o SENHOR vos ordenou: Colhei disso cada um segundo o que pode comer, um gômer por cabeça, segundo o número de vossas pessoas; cada um tomará para os que se acharem na sua tenda.
\par 17 Assim o fizeram os filhos de Israel; e colheram, uns, mais, outros, menos.
\par 18 Porém, medindo-o com o gômer, não sobejava ao que colhera muito, nem faltava ao que colhera pouco, pois colheram cada um quanto podia comer.
\par 19 Disse-lhes Moisés: Ninguém deixe dele para a manhã seguinte.
\par 20 Eles, porém, não deram ouvidos a Moisés, e alguns deixaram do maná para a manhã seguinte; porém deu bichos e cheirava mal. E Moisés se indignou contra eles.
\par 21 Colhiam-no, pois, manhã após manhã, cada um quanto podia comer; porque, em vindo o calor, se derretia.
\par 22 Ao sexto dia, colheram pão em dobro, dois gômeres para cada um; e os principais da congregação vieram e contaram-no a Moisés.
\par 23 Respondeu-lhes ele: Isto é o que disse o SENHOR: Amanhã é repouso, o santo sábado do SENHOR; o que quiserdes cozer no forno, cozei-o, e o que quiserdes cozer em água, cozei-o em água; e tudo o que sobrar separai, guardando para a manhã seguinte.
\par 24 E guardaram-no até pela manhã seguinte, como Moisés ordenara; e não cheirou mal, nem deu bichos.
\par 25 Então, disse Moisés: Comei-o hoje, porquanto o sábado é do SENHOR; hoje, não o achareis no campo.
\par 26 Seis dias o colhereis, mas o sétimo dia é o sábado; nele, não haverá.
\par 27 Ao sétimo dia, saíram alguns do povo para o colher, porém não o acharam.
\par 28 Então, disse o SENHOR a Moisés: Até quando recusareis guardar os meus mandamentos e as minhas leis?
\par 29 Considerai que o SENHOR vos deu o sábado; por isso, ele, no sexto dia, vos dá pão para dois dias; cada um fique onde está, ninguém saia do seu lugar no sétimo dia.
\par 30 Assim, descansou o povo no sétimo dia.
\par 31 Deu-lhe a casa de Israel o nome de maná; era como semente de coentro, branco e de sabor como bolos de mel.
\par 32 Disse Moisés: Esta é a palavra que o SENHOR ordenou: Dele encherás um gômer e o guardarás para as vossas gerações, para que vejam o pão com que vos sustentei no deserto, quando vos tirei do Egito.
\par 33 Disse também Moisés a Arão: Toma um vaso, mete nele um gômer cheio de maná e coloca-o diante do SENHOR, para guardar-se às vossas gerações.
\par 34 Como o SENHOR ordenara a Moisés, assim Arão o colocou diante do Testemunho para o guardar.
\par 35 E comeram os filhos de Israel maná quarenta anos, até que entraram em terra habitada; comeram maná até que chegaram aos limites da terra de Canaã.
\par 36 Gômer é a décima parte do efa.

\chapter{17}

\par 1 Tendo partido toda a congregação dos filhos de Israel do deserto de Sim, fazendo suas paradas, segundo o mandamento do SENHOR, acamparam-se em Refidim; e não havia ali água para o povo beber.
\par 2 Contendeu, pois, o povo com Moisés e disse: Dá-nos água para beber. Respondeu-lhes Moisés: Por que contendeis comigo? Por que tentais ao SENHOR?
\par 3 Tendo aí o povo sede de água, murmurou contra Moisés e disse: Por que nos fizeste subir do Egito, para nos matares de sede, a nós, a nossos filhos e aos nossos rebanhos?
\par 4 Então, clamou Moisés ao SENHOR: Que farei a este povo? Só lhe resta apedrejar-me.
\par 5 Respondeu o SENHOR a Moisés: Passa adiante do povo e toma contigo alguns dos anciãos de Israel, leva contigo em mão o bordão com que feriste o rio e vai.
\par 6 Eis que estarei ali diante de ti sobre a rocha em Horebe; ferirás a rocha, e dela sairá água, e o povo beberá. Moisés assim o fez na presença dos anciãos de Israel.
\par 7 E chamou o nome daquele lugar Massá e Meribá, por causa da contenda dos filhos de Israel e porque tentaram ao SENHOR, dizendo: Está o SENHOR no meio de nós ou não?
\par 8 Então, veio Amaleque e pelejou contra Israel em Refidim.
\par 9 Com isso, ordenou Moisés a Josué: Escolhe-nos homens, e sai, e peleja contra Amaleque; amanhã, estarei eu no cimo do outeiro, e o bordão de Deus estará na minha mão.
\par 10 Fez Josué como Moisés lhe dissera e pelejou contra Amaleque; Moisés, porém, Arão e Hur subiram ao cimo do outeiro.
\par 11 Quando Moisés levantava a mão, Israel prevalecia; quando, porém, ele abaixava a mão, prevalecia Amaleque.
\par 12 Ora, as mãos de Moisés eram pesadas; por isso, tomaram uma pedra e a puseram por baixo dele, e ele nela se assentou; Arão e Hur sustentavam-lhe as mãos, um, de um lado, e o outro, do outro; assim lhe ficaram as mãos firmes até ao pôr-do-sol.
\par 13 E Josué desbaratou a Amaleque e a seu povo a fio de espada.
\par 14 Então, disse o SENHOR a Moisés: Escreve isto para memória num livro e repete-o a Josué; porque eu hei de riscar totalmente a memória de Amaleque de debaixo do céu.
\par 15 E Moisés edificou um altar e lhe chamou: O SENHOR É Minha Bandeira.
\par 16 E disse: Porquanto o SENHOR jurou, haverá guerra do SENHOR contra Amaleque de geração em geração.

\chapter{18}

\par 1 Ora, Jetro, sacerdote de Midiã, sogro de Moisés, ouviu todas as coisas que Deus tinha feito a Moisés e a Israel, seu povo; como o SENHOR trouxera a Israel do Egito.
\par 2 Jetro, sogro de Moisés, tomou a Zípora, mulher de Moisés, depois que este lha enviara,
\par 3 com os dois filhos dela, dos quais um se chamava Gérson, pois disse Moisés: Fui peregrino em terra estrangeira;
\par 4 e o outro, Eliézer, pois disse: O Deus de meu pai foi a minha ajuda e me livrou da espada de Faraó.
\par 5 Veio Jetro, sogro de Moisés, com os filhos e a mulher deste, a Moisés no deserto onde se achava acampado, junto ao monte de Deus,
\par 6 e mandou dizer a Moisés: Eu, teu sogro Jetro, venho a ti, com a tua mulher e seus dois filhos.
\par 7 Então, saiu Moisés ao encontro do seu sogro, inclinou-se e o beijou; e, indagando pelo bem-estar um do outro, entraram na tenda.
\par 8 Contou Moisés a seu sogro tudo o que o SENHOR havia feito a Faraó e aos egípcios por amor de Israel, e todo o trabalho que passaram no Egito, e como o SENHOR os livrara.
\par 9 Alegrou-se Jetro de todo o bem que o SENHOR fizera a Israel, livrando-o da mão dos egípcios,
\par 10 e disse: Bendito seja o SENHOR, que vos livrou da mão dos egípcios e da mão de Faraó;
\par 11 agora, sei que o SENHOR é maior que todos os deuses, porque livrou este povo de debaixo da mão dos egípcios, quando agiram arrogantemente contra o povo.
\par 12 Então, Jetro, sogro de Moisés, tomou holocausto e sacrifícios para Deus; e veio Arão e todos os anciãos de Israel para comerem pão com o sogro de Moisés, diante de Deus.
\par 13 No dia seguinte, assentou-se Moisés para julgar o povo; e o povo estava em pé diante de Moisés desde a manhã até ao pôr-do-sol.
\par 14 Vendo, pois, o sogro de Moisés tudo o que ele fazia ao povo, disse: Que é isto que fazes ao povo? Por que te assentas só, e todo o povo está em pé diante de ti, desde a manhã até ao pôr-do-sol?
\par 15 Respondeu Moisés a seu sogro: É porque o povo me vem a mim para consultar a Deus;
\par 16 quando tem alguma questão, vem a mim, para que eu julgue entre um e outro e lhes declare os estatutos de Deus e as suas leis.
\par 17 O sogro de Moisés, porém, lhe disse: Não é bom o que fazes.
\par 18 Sem dúvida, desfalecerás, tanto tu como este povo que está contigo; pois isto é pesado demais para ti; tu só não o podes fazer.
\par 19 Ouve, pois, as minhas palavras; eu te aconselharei, e Deus seja contigo; representa o povo perante Deus, leva as suas causas a Deus,
\par 20 ensina-lhes os estatutos e as leis e faze-lhes saber o caminho em que devem andar e a obra que devem fazer.
\par 21 Procura dentre o povo homens capazes, tementes a Deus, homens de verdade, que aborreçam a avareza; põe-nos sobre eles por chefes de mil, chefes de cem, chefes de cinqüenta e chefes de dez;
\par 22 para que julguem este povo em todo tempo. Toda causa grave trarão a ti, mas toda causa pequena eles mesmos julgarão; será assim mais fácil para ti, e eles levarão a carga contigo.
\par 23 Se isto fizeres, e assim Deus to mandar, poderás, então, suportar; e assim também todo este povo tornará em paz ao seu lugar.
\par 24 Moisés atendeu às palavras de seu sogro e fez tudo quanto este lhe dissera.
\par 25 Escolheu Moisés homens capazes, de todo o Israel, e os constituiu por cabeças sobre o povo: chefes de mil, chefes de cem, chefes de cinqüenta e chefes de dez.
\par 26 Estes julgaram o povo em todo tempo; a causa grave trouxeram a Moisés e toda causa simples julgaram eles.
\par 27 Então, se despediu Moisés de seu sogro, e este se foi para a sua terra.

\chapter{19}

\par 1 No terceiro mês da saída dos filhos de Israel da terra do Egito, no primeiro dia desse mês, vieram ao deserto do Sinai.
\par 2 Tendo partido de Refidim, vieram ao deserto do Sinai, no qual se acamparam; ali, pois, se acampou Israel em frente do monte.
\par 3 Subiu Moisés a Deus, e do monte o SENHOR o chamou e lhe disse: Assim falarás à casa de Jacó e anunciarás aos filhos de Israel:
\par 4 Tendes visto o que fiz aos egípcios, como vos levei sobre asas de águia e vos cheguei a mim.
\par 5 Agora, pois, se diligentemente ouvirdes a minha voz e guardardes a minha aliança, então, sereis a minha propriedade peculiar dentre todos os povos; porque toda a terra é minha;
\par 6 vós me sereis reino de sacerdotes e nação santa. São estas as palavras que falarás aos filhos de Israel.
\par 7 Veio Moisés, chamou os anciãos do povo e expôs diante deles todas estas palavras que o SENHOR lhe havia ordenado.
\par 8 Então, o povo respondeu à uma: Tudo o que o SENHOR falou faremos. E Moisés relatou ao SENHOR as palavras do povo.
\par 9 Disse o SENHOR a Moisés: Eis que virei a ti numa nuvem escura, para que o povo ouça quando eu falar contigo e para que também creiam sempre em ti. Porque Moisés tinha anunciado as palavras do seu povo ao SENHOR.
\par 10 Disse também o SENHOR a Moisés: Vai ao povo e purifica-o hoje e amanhã. Lavem eles as suas vestes
\par 11 e estejam prontos para o terceiro dia; porque no terceiro dia o SENHOR, à vista de todo o povo, descerá sobre o monte Sinai.
\par 12 Marcarás em redor limites ao povo, dizendo: Guardai-vos de subir ao monte, nem toqueis o seu limite; todo aquele que tocar o monte será morto.
\par 13 Mão nenhuma tocará neste, mas será apedrejado ou flechado; quer seja animal, quer seja homem, não viverá. Quando soar longamente a buzina, então, subirão ao monte.
\par 14 Moisés, tendo descido do monte ao povo, consagrou o povo; e lavaram as suas vestes.
\par 15 E disse ao povo: Estai prontos ao terceiro dia; e não vos chegueis a mulher.
\par 16 Ao amanhecer do terceiro dia, houve trovões, e relâmpagos, e uma espessa nuvem sobre o monte, e mui forte clangor de trombeta, de maneira que todo o povo que estava no arraial se estremeceu.
\par 17 E Moisés levou o povo fora do arraial ao encontro de Deus; e puseram-se ao pé do monte.
\par 18 Todo o monte Sinai fumegava, porque o SENHOR descera sobre ele em fogo; a sua fumaça subiu como fumaça de uma fornalha, e todo o monte tremia grandemente.
\par 19 E o clangor da trombeta ia aumentando cada vez mais; Moisés falava, e Deus lhe respondia no trovão.
\par 20 Descendo o SENHOR para o cimo do monte Sinai, chamou o SENHOR a Moisés para o cimo do monte. Moisés subiu,
\par 21 e o SENHOR disse a Moisés: Desce, adverte ao povo que não traspasse o limite até ao SENHOR para vê-lo, a fim de muitos deles não perecerem.
\par 22 Também os sacerdotes, que se chegam ao SENHOR, se hão de consagrar, para que o SENHOR não os fira.
\par 23 Então, disse Moisés ao SENHOR: O povo não poderá subir ao monte Sinai, porque tu nos advertiste, dizendo: Marca limites ao redor do monte e consagra-o.
\par 24 Replicou-lhe o SENHOR: Vai, desce; depois, subirás tu, e Arão contigo; os sacerdotes, porém, e o povo não traspassem o limite para subir ao SENHOR, para que não os fira.
\par 25 Desceu, pois, Moisés ao povo e lhe disse tudo isso.

\chapter{20}

\par 1 Então, falou Deus todas estas palavras:
\par 2 Eu sou o SENHOR, teu Deus, que te tirei da terra do Egito, da casa da servidão.
\par 3 Não terás outros deuses diante de mim.
\par 4 Não farás para ti imagem de escultura, nem semelhança alguma do que há em cima nos céus, nem embaixo na terra, nem nas águas debaixo da terra.
\par 5 Não as adorarás, nem lhes darás culto; porque eu sou o SENHOR, teu Deus, Deus zeloso, que visito a iniqüidade dos pais nos filhos até à terceira e quarta geração daqueles que me aborrecem
\par 6 e faço misericórdia até mil gerações daqueles que me amam e guardam os meus mandamentos.
\par 7 Não tomarás o nome do SENHOR, teu Deus, em vão, porque o SENHOR não terá por inocente o que tomar o seu nome em vão.
\par 8 Lembra-te do dia de sábado, para o santificar.
\par 9 Seis dias trabalharás e farás toda a tua obra.
\par 10 Mas o sétimo dia é o sábado do SENHOR, teu Deus; não farás nenhum trabalho, nem tu, nem o teu filho, nem a tua filha, nem o teu servo, nem a tua serva, nem o teu animal, nem o forasteiro das tuas portas para dentro;
\par 11 porque, em seis dias, fez o SENHOR os céus e a terra, o mar e tudo o que neles há e, ao sétimo dia, descansou; por isso, o SENHOR abençoou o dia de sábado e o santificou.
\par 12 Honra teu pai e tua mãe, para que se prolonguem os teus dias na terra que o SENHOR, teu Deus, te dá.
\par 13 Não matarás.
\par 14 Não adulterarás.
\par 15 Não furtarás.
\par 16 Não dirás falso testemunho contra o teu próximo.
\par 17 Não cobiçarás a casa do teu próximo. Não cobiçarás a mulher do teu próximo, nem o seu servo, nem a sua serva, nem o seu boi, nem o seu jumento, nem coisa alguma que pertença ao teu próximo.
\par 18 Todo o povo presenciou os trovões, e os relâmpagos, e o clangor da trombeta, e o monte fumegante; e o povo, observando, se estremeceu e ficou de longe.
\par 19 Disseram a Moisés: Fala-nos tu, e te ouviremos; porém não fale Deus conosco, para que não morramos.
\par 20 Respondeu Moisés ao povo: Não temais; Deus veio para vos provar e para que o seu temor esteja diante de vós, a fim de que não pequeis.
\par 21 O povo estava de longe, em pé; Moisés, porém, se chegou à nuvem escura onde Deus estava.
\par 22 Então, disse o SENHOR a Moisés: Assim dirás aos filhos de Israel: Vistes que dos céus eu vos falei.
\par 23 Não fareis deuses de prata ao lado de mim, nem deuses de ouro fareis para vós outros.
\par 24 Um altar de terra me farás e sobre ele sacrificarás os teus holocaustos, as tuas ofertas pacíficas, as tuas ovelhas e os teus bois; em todo lugar onde eu fizer celebrar a memória do meu nome, virei a ti e te abençoarei.
\par 25 Se me levantares um altar de pedras, não o farás de pedras lavradas; pois, se sobre ele manejares a tua ferramenta, profaná-lo-ás.
\par 26 Nem subirás por degrau ao meu altar, para que a tua nudez não seja ali exposta.

\chapter{21}

\par 1 São estes os estatutos que lhes proporás:
\par 2 Se comprares um escravo hebreu, seis anos servirá; mas, ao sétimo, sairá forro, de graça.
\par 3 Se entrou solteiro, sozinho sairá; se era homem casado, com ele sairá sua mulher.
\par 4 Se o seu senhor lhe der mulher, e ela der à luz filhos e filhas, a mulher e seus filhos serão do seu senhor, e ele sairá sozinho.
\par 5 Porém, se o escravo expressamente disser: Eu amo meu senhor, minha mulher e meus filhos, não quero sair forro.
\par 6 Então, o seu senhor o levará aos juízes, e o fará chegar à porta ou à ombreira, e o seu senhor lhe furará a orelha com uma sovela; e ele o servirá para sempre.
\par 7 Se um homem vender sua filha para ser escrava, esta não lhe sairá como saem os escravos.
\par 8 Se ela não agradar ao seu senhor, que se comprometeu a desposá-la, ele terá de permitir-lhe o resgate; não poderá vendê-la a um povo estranho, pois será isso deslealdade para com ela.
\par 9 Mas, se a casar com seu filho, tratá-la-á como se tratam as filhas.
\par 10 Se ele der ao filho outra mulher, não diminuirá o mantimento da primeira, nem os seus vestidos, nem os seus direitos conjugais.
\par 11 Se não lhe fizer estas três coisas, ela sairá sem retribuição, nem pagamento em dinheiro.
\par 12 Quem ferir a outro, de modo que este morra, também será morto.
\par 13 Porém, se não lhe armou ciladas, mas Deus lhe permitiu caísse em suas mãos, então, te designarei um lugar para onde ele fugirá.
\par 14 Se alguém vier maliciosamente contra o próximo, matando-o à traição, tirá-lo-ás até mesmo do meu altar, para que morra.
\par 15 Quem ferir seu pai ou sua mãe será morto.
\par 16 O que raptar alguém e o vender, ou for achado na sua mão, será morto.
\par 17 Quem amaldiçoar seu pai ou sua mãe será morto.
\par 18 Se dois brigarem, ferindo um ao outro com pedra ou com o punho, e o ferido não morrer, mas cair de cama;
\par 19 se ele tornar a levantar-se e andar fora, apoiado ao seu bordão, então, será absolvido aquele que o feriu; somente lhe pagará o tempo que perdeu e o fará curar-se totalmente.
\par 20 Se alguém ferir com bordão o seu escravo ou a sua escrava, e o ferido morrer debaixo da sua mão, será punido;
\par 21 porém, se ele sobreviver por um ou dois dias, não será punido, porque é dinheiro seu.
\par 22 Se homens brigarem, e ferirem mulher grávida, e forem causa de que aborte, porém sem maior dano, aquele que feriu será obrigado a indenizar segundo o que lhe exigir o marido da mulher; e pagará como os juízes lhe determinarem.
\par 23 Mas, se houver dano grave, então, darás vida por vida,
\par 24 olho por olho, dente por dente, mão por mão, pé por pé,
\par 25 queimadura por queimadura, ferimento por ferimento, golpe por golpe.
\par 26 Se alguém ferir o olho do seu escravo ou o olho da sua escrava e o inutilizar, deixá-lo-á ir forro pelo seu olho.
\par 27 E, se com violência fizer cair um dente do seu escravo ou da sua escrava, deixá-lo-á ir forro pelo seu dente.
\par 28 Se algum boi chifrar homem ou mulher, que morra, o boi será apedrejado, e não lhe comerão a carne; mas o dono do boi será absolvido.
\par 29 Mas, se o boi, dantes, era dado a chifrar, e o seu dono era disso conhecedor e não o prendeu, e o boi matar homem ou mulher, o boi será apedrejado, e também será morto o seu dono.
\par 30 Se lhe for exigido resgate, dará, então, como resgate da sua vida tudo o que lhe for exigido.
\par 31 Quer tenha chifrado um filho, quer tenha chifrado uma filha, este julgamento lhe será aplicado.
\par 32 Se o boi chifrar um escravo ou uma escrava, dar-se-ão trinta siclos de prata ao senhor destes, e o boi será apedrejado.
\par 33 Se alguém deixar aberta uma cova ou se alguém cavar uma cova e não a tapar, e nela cair boi ou jumento,
\par 34 o dono da cova o pagará, pagará dinheiro ao seu dono, mas o animal morto será seu.
\par 35 Se um boi de um homem ferir o boi de outro, e o boi ferido morrer, venderão o boi vivo e repartirão o valor; e dividirão entre si o boi morto.
\par 36 Mas, se for notório que o boi era já, dantes, chifrador, e o seu dono não o prendeu, certamente, pagará boi por boi; porém o morto será seu.

\chapter{22}

\par 1 Se alguém furtar boi ou ovelha e o abater ou vender, por um boi pagará cinco bois, e quatro ovelhas por uma ovelha.
\par 2 Se um ladrão for achado arrombando uma casa e, sendo ferido, morrer, quem o feriu não será culpado do sangue.
\par 3 Se, porém, já havia sol quando tal se deu, quem o feriu será culpado do sangue; neste caso, o ladrão fará restituição total. Se não tiver com que pagar, será vendido por seu furto.
\par 4 Se aquilo que roubou for achado vivo em seu poder, seja boi, jumento ou ovelha, pagará o dobro.
\par 5 Se alguém fizer pastar o seu animal num campo ou numa vinha e o largar para comer em campo de outrem, pagará com o melhor do seu próprio campo e o melhor da sua própria vinha.
\par 6 Se irromper fogo, e pegar nos espinheiros, e destruir as medas de cereais, ou a messe, ou o campo, aquele que acendeu o fogo pagará totalmente o queimado.
\par 7 Se alguém der ao seu próximo dinheiro ou objetos a guardar, e isso for furtado àquele que o recebeu, se for achado o ladrão, este pagará o dobro.
\par 8 Se o ladrão não for achado, então, o dono da casa será levado perante os juízes, a ver se não meteu a mão nos bens do próximo.
\par 9 Em todo negócio frauduloso, seja a respeito de boi, ou de jumento, ou de ovelhas, ou de roupas, ou de qualquer coisa perdida, de que uma das partes diz: Esta é a coisa, a causa de ambas as partes se levará perante os juízes; aquele a quem os juízes condenarem pagará o dobro ao seu próximo.
\par 10 Se alguém der ao seu próximo a guardar jumento, ou boi, ou ovelha, ou outro animal qualquer, e este morrer, ou ficar aleijado, ou for afugentado, sem que ninguém o veja,
\par 11 então, haverá juramento do SENHOR entre ambos, de que não meteu a mão nos bens do seu próximo; o dono aceitará o juramento, e o outro não fará restituição.
\par 12 Porém, se, de fato, lhe for furtado, pagá-lo-á ao seu dono.
\par 13 Se for dilacerado, trá-lo-á em testemunho disso e não pagará o dilacerado.
\par 14 Se alguém pedir emprestado a seu próximo um animal, e este ficar aleijado ou morrer, não estando presente o dono, pagá-lo-á.
\par 15 Se o dono esteve presente, não o pagará; se foi alugado, o preço do aluguel será o pagamento.
\par 16 Se alguém seduzir qualquer virgem que não estava desposada e se deitar com ela, pagará seu dote e a tomará por mulher.
\par 17 Se o pai dela definitivamente recusar dar-lha, pagará ele em dinheiro conforme o dote das virgens.
\par 18 A feiticeira não deixarás viver.
\par 19 Quem tiver coito com animal será morto.
\par 20 Quem sacrificar aos deuses e não somente ao SENHOR será destruído.
\par 21 Não afligirás o forasteiro, nem o oprimirás; pois forasteiros fostes na terra do Egito.
\par 22 A nenhuma viúva nem órfão afligireis.
\par 23 Se de algum modo os afligirdes, e eles clamarem a mim, eu lhes ouvirei o clamor;
\par 24 a minha ira se acenderá, e vos matarei à espada; vossas mulheres ficarão viúvas, e vossos filhos, órfãos.
\par 25 Se emprestares dinheiro ao meu povo, ao pobre que está contigo, não te haverás com ele como credor que impõe juros.
\par 26 Se do teu próximo tomares em penhor a sua veste, lha restituirás antes do pôr-do-sol;
\par 27 porque é com ela que se cobre, é a veste do seu corpo; em que se deitaria? Será, pois, que, quando clamar a mim, eu o ouvirei, porque sou misericordioso.
\par 28 Contra Deus não blasfemarás, nem amaldiçoarás o príncipe do teu povo.
\par 29 Não tardarás em trazer ofertas do melhor das tuas ceifas e das tuas vinhas; o primogênito de teus filhos me darás.
\par 30 Da mesma sorte, farás com os teus bois e com as tuas ovelhas; sete dias ficará a cria com a mãe, e, ao oitavo dia, ma darás.
\par 31 Ser-me-eis homens consagrados; portanto, não comereis carne dilacerada no campo; deitá-la-eis aos cães.

\chapter{23}

\par 1 Não espalharás notícias falsas, nem darás mão ao ímpio, para seres testemunha maldosa.
\par 2 Não seguirás a multidão para fazeres mal; nem deporás, numa demanda, inclinando-te para a maioria, para torcer o direito.
\par 3 Nem com o pobre serás parcial na sua demanda.
\par 4 Se encontrares desgarrado o boi do teu inimigo ou o seu jumento, lho reconduzirás.
\par 5 Se vires prostrado debaixo da sua carga o jumento daquele que te aborrece, não o abandonarás, mas ajudá-lo-ás a erguê-lo.
\par 6 Não perverterás o julgamento do teu pobre na sua causa.
\par 7 Da falsa acusação te afastarás; não matarás o inocente e o justo, porque não justificarei o ímpio.
\par 8 Também suborno não aceitarás, porque o suborno cega até o perspicaz e perverte as palavras dos justos.
\par 9 Também não oprimirás o forasteiro; pois vós conheceis o coração do forasteiro, visto que fostes forasteiros na terra do Egito.
\par 10 Seis anos semearás a tua terra e recolherás os seus frutos;
\par 11 porém, no sétimo ano, a deixarás descansar e não a cultivarás, para que os pobres do teu povo achem o que comer, e do sobejo comam os animais do campo. Assim farás com a tua vinha e com o teu olival.
\par 12 Seis dias farás a tua obra, mas, ao sétimo dia, descansarás; para que descanse o teu boi e o teu jumento; e para que tome alento o filho da tua serva e o forasteiro.
\par 13 Em tudo o que vos tenho dito, andai apercebidos; do nome de outros deuses nem vos lembreis, nem se ouça de vossa boca.
\par 14 Três vezes no ano me celebrareis festa.
\par 15 Guardarás a Festa dos Pães Asmos; sete dias comerás pães asmos, como te ordenei, ao tempo apontado no mês de abibe, porque nele saíste do Egito; ninguém apareça de mãos vazias perante mim.
\par 16 Guardarás a Festa da Sega, dos primeiros frutos do teu trabalho, que houveres semeado no campo, e a Festa da Colheita, à saída do ano, quando recolheres do campo o fruto do teu trabalho.
\par 17 Três vezes no ano, todo homem aparecerá diante do SENHOR Deus.
\par 18 Não oferecerás o sangue do meu sacrifício com pão levedado, nem ficará gordura da minha festa durante a noite até pela manhã.
\par 19 As primícias dos frutos da tua terra trarás à Casa do SENHOR, teu Deus. Não cozerás o cabrito no leite da sua própria mãe.
\par 20 Eis que eu envio um Anjo adiante de ti, para que te guarde pelo caminho e te leve ao lugar que tenho preparado.
\par 21 Guarda-te diante dele, e ouve a sua voz, e não te rebeles contra ele, porque não perdoará a vossa transgressão; pois nele está o meu nome.
\par 22 Mas, se diligentemente lhe ouvires a voz e fizeres tudo o que eu disser, então, serei inimigo dos teus inimigos e adversário dos teus adversários.
\par 23 Porque o meu Anjo irá adiante de ti e te levará aos amorreus, aos heteus, aos ferezeus, aos cananeus, aos heveus e aos jebuseus; e eu os destruirei.
\par 24 Não adorarás os seus deuses, nem lhes darás culto, nem farás conforme as suas obras; antes, os destruirás totalmente e despedaçarás de todo as suas colunas.
\par 25 Servireis ao SENHOR, vosso Deus, e ele abençoará o vosso pão e a vossa água; e tirará do vosso meio as enfermidades.
\par 26 Na tua terra, não haverá mulher que aborte, nem estéril; completarei o número dos teus dias.
\par 27 Enviarei o meu terror diante de ti, confundindo a todo povo onde entrares; farei que todos os teus inimigos te voltem as costas.
\par 28 Também enviarei vespas diante de ti, que lancem os heveus, os cananeus e os heteus de diante de ti.
\par 29 Não os lançarei de diante de ti num só ano, para que a terra se não torne em desolação, e as feras do campo se não multipliquem contra ti.
\par 30 Pouco a pouco, os lançarei de diante de ti, até que te multipliques e possuas a terra por herança.
\par 31 Porei os teus limites desde o mar Vermelho até ao mar dos filisteus e desde o deserto até ao Eufrates; porque darei nas tuas mãos os moradores da terra, para que os lances de diante de ti.
\par 32 Não farás aliança nenhuma com eles, nem com os seus deuses.
\par 33 Eles não habitarão na tua terra, para que te não façam pecar contra mim; se servires aos seus deuses, isso te será cilada.

\chapter{24}

\par 1 Disse também Deus a Moisés: Sobe ao SENHOR, tu, e Arão, e Nadabe, e Abiú, e setenta dos anciãos de Israel; e adorai de longe.
\par 2 Só Moisés se chegará ao SENHOR; os outros não se chegarão, nem o povo subirá com ele.
\par 3 Veio, pois, Moisés e referiu ao povo todas as palavras do SENHOR e todos os estatutos; então, todo o povo respondeu a uma voz e disse: Tudo o que falou o SENHOR faremos.
\par 4 Moisés escreveu todas as palavras do SENHOR e, tendo-se levantado pela manhã de madrugada, erigiu um altar ao pé do monte e doze colunas, segundo as doze tribos de Israel.
\par 5 E enviou alguns jovens dos filhos de Israel, os quais ofereceram ao SENHOR holocaustos e sacrifícios pacíficos de novilhos.
\par 6 Moisés tomou metade do sangue e o pôs em bacias; e a outra metade aspergiu sobre o altar.
\par 7 E tomou o livro da aliança e o leu ao povo; e eles disseram: Tudo o que falou o SENHOR faremos e obedeceremos.
\par 8 Então, tomou Moisés aquele sangue, e o aspergiu sobre o povo, e disse: Eis aqui o sangue da aliança que o SENHOR fez convosco a respeito de todas estas palavras.
\par 9 E subiram Moisés, e Arão, e Nadabe, e Abiú, e setenta dos anciãos de Israel.
\par 10 E viram o Deus de Israel, sob cujos pés havia uma como pavimentação de pedra de safira, que se parecia com o céu na sua claridade.
\par 11 Ele não estendeu a mão sobre os escolhidos dos filhos de Israel; porém eles viram a Deus, e comeram, e beberam.
\par 12 Então, disse o SENHOR a Moisés: Sobe a mim, ao monte, e fica lá; dar-te-ei tábuas de pedra, e a lei, e os mandamentos que escrevi, para os ensinares.
\par 13 Levantou-se Moisés com Josué, seu servidor; e, subindo Moisés ao monte de Deus,
\par 14 disse aos anciãos: Esperai-nos aqui até que voltemos a vós outros. Eis que Arão e Hur ficam convosco; quem tiver alguma questão se chegará a eles.
\par 15 Tendo Moisés subido, uma nuvem cobriu o monte.
\par 16 E a glória do SENHOR pousou sobre o monte Sinai, e a nuvem o cobriu por seis dias; ao sétimo dia, do meio da nuvem chamou o SENHOR a Moisés.
\par 17 O aspecto da glória do SENHOR era como um fogo consumidor no cimo do monte, aos olhos dos filhos de Israel.
\par 18 E Moisés, entrando pelo meio da nuvem, subiu ao monte; e lá permaneceu quarenta dias e quarenta noites.

\chapter{25}

\par 1 Disse o SENHOR a Moisés:
\par 2 Fala aos filhos de Israel que me tragam oferta; de todo homem cujo coração o mover para isso, dele recebereis a minha oferta.
\par 3 Esta é a oferta que dele recebereis: ouro, e prata, e bronze,
\par 4 e estofo azul, e púrpura, e carmesim, e linho fino, e pêlos de cabra,
\par 5 e peles de carneiro tintas de vermelho, e peles finas, e madeira de acácia,
\par 6 azeite para a luz, especiarias para o óleo de unção e para o incenso aromático,
\par 7 pedras de ônix e pedras de engaste, para a estola sacerdotal e para o peitoral.
\par 8 E me farão um santuário, para que eu possa habitar no meio deles.
\par 9 Segundo tudo o que eu te mostrar para modelo do tabernáculo e para modelo de todos os seus móveis, assim mesmo o fareis.
\par 10 Também farão uma arca de madeira de acácia; de dois côvados e meio será o seu comprimento, de um côvado e meio, a largura, e de um côvado e meio, a altura.
\par 11 De ouro puro a cobrirás; por dentro e por fora a cobrirás e farás sobre ela uma bordadura de ouro ao redor.
\par 12 Fundirás para ela quatro argolas de ouro e as porás nos quatro cantos da arca: duas argolas num lado dela e duas argolas noutro lado.
\par 13 Farás também varais de madeira de acácia e os cobrirás de ouro;
\par 14 meterás os varais nas argolas aos lados da arca, para se levar por meio deles a arca.
\par 15 Os varais ficarão nas argolas da arca e não se tirarão dela.
\par 16 E porás na arca o Testemunho, que eu te darei.
\par 17 Farás também um propiciatório de ouro puro; de dois côvados e meio será o seu comprimento, e a largura, de um côvado e meio.
\par 18 Farás dois querubins de ouro; de ouro batido os farás, nas duas extremidades do propiciatório;
\par 19 um querubim, na extremidade de uma parte, e o outro, na extremidade da outra parte; de uma só peça com o propiciatório fareis os querubins nas duas extremidades dele.
\par 20 Os querubins estenderão as asas por cima, cobrindo com elas o propiciatório; estarão eles de faces voltadas uma para a outra, olhando para o propiciatório.
\par 21 Porás o propiciatório em cima da arca; e dentro dela porás o Testemunho, que eu te darei.
\par 22 Ali, virei a ti e, de cima do propiciatório, do meio dos dois querubins que estão sobre a arca do Testemunho, falarei contigo acerca de tudo o que eu te ordenar para os filhos de Israel.
\par 23 Também farás a mesa de madeira de acácia; terá o comprimento de dois côvados, a largura, de um côvado, e a altura, de um côvado e meio;
\par 24 de ouro puro a cobrirás e lhe farás uma bordadura de ouro ao redor.
\par 25 Também lhe farás moldura ao redor, da largura de quatro dedos, e lhe farás uma bordadura de ouro ao redor da moldura.
\par 26 Também lhe farás quatro argolas de ouro; e porás as argolas nos quatro cantos, que estão nos seus quatro pés.
\par 27 Perto da moldura estarão as argolas, como lugares para os varais, para se levar a mesa.
\par 28 Farás, pois, estes varais de madeira de acácia e os cobrirás de ouro; por meio deles, se levará a mesa.
\par 29 Também farás os seus pratos, e os seus recipientes para incenso, e as suas galhetas, e as suas taças em que se hão de oferecer libações; de ouro puro os farás.
\par 30 Porás sobre a mesa os pães da proposição diante de mim perpetuamente.
\par 31 Farás também um candelabro de ouro puro; de ouro batido se fará este candelabro; o seu pedestal, a sua hástea, os seus cálices, as suas maçanetas e as suas flores formarão com ele uma só peça.
\par 32 Seis hásteas sairão dos seus lados: três de um lado e três do outro.
\par 33 Numa hástea, haverá três cálices com formato de amêndoas, uma maçaneta e uma flor; e três cálices, com formato de amêndoas na outra hástea, uma maçaneta e uma flor; assim serão as seis hásteas que saem do candelabro.
\par 34 Mas no candelabro mesmo haverá quatro cálices com formato de amêndoas, com suas maçanetas e com suas flores.
\par 35 Haverá uma maçaneta sob duas hásteas que saem dele; e ainda uma maçaneta sob duas outras hásteas que saem dele; e ainda mais uma maçaneta sob duas outras hásteas que saem dele; assim se fará com as seis hásteas que saem do candelabro.
\par 36 As suas maçanetas e as suas hásteas serão do mesmo; tudo será de uma só peça, obra batida de ouro puro.
\par 37 Também lhe farás sete lâmpadas, as quais se acenderão para alumiar defronte dele.
\par 38 As suas espevitadeiras e os seus apagadores serão de ouro puro.
\par 39 De um talento de ouro puro se fará o candelabro com todos estes utensílios.
\par 40 Vê, pois, que tudo faças segundo o modelo que te foi mostrado no monte.

\chapter{26}

\par 1 Farás o tabernáculo, que terá dez cortinas, de linho retorcido, estofo azul, púrpura e carmesim; com querubins, as farás de obra de artista.
\par 2 O comprimento de cada cortina será de vinte e oito côvados, e a largura, de quatro côvados; todas as cortinas serão de igual medida.
\par 3 Cinco cortinas serão ligadas umas às outras; e as outras cinco também ligadas umas às outras.
\par 4 Farás laçadas de estofo azul na orla da cortina extrema do primeiro agrupamento; e de igual modo farás na orla da cortina extrema do segundo agrupamento.
\par 5 Cinqüenta laçadas farás numa cortina, e cinqüenta, na outra cortina no extremo do segundo agrupamento; as laçadas serão contrapostas uma à outra.
\par 6 Farás cinqüenta colchetes de ouro, com os quais prenderás as cortinas uma à outra; e o tabernáculo passará a ser um todo.
\par 7 Farás também de pêlos de cabra cortinas para servirem de tenda sobre o tabernáculo; onze cortinas farás.
\par 8 O comprimento de cada cortina será de trinta côvados, e a largura, de quatro côvados; as onze cortinas serão de igual medida.
\par 9 Ajuntarás à parte cinco cortinas entre si, e de igual modo as seis restantes, a sexta das quais dobrarás na parte dianteira da tenda.
\par 10 Farás cinqüenta laçadas na orla da cortina extrema do primeiro agrupamento e cinqüenta laçadas na orla da cortina extrema do segundo agrupamento.
\par 11 Farás também cinqüenta colchetes de bronze, e meterás os colchetes nas laçadas, e ajuntarás a tenda, para que venha a ser um todo.
\par 12 A parte que restar das cortinas da tenda, a saber, a meia cortina que sobrar, penderá às costas do tabernáculo.
\par 13 O côvado de um lado e o côvado de outro lado, do que sobejar no comprimento das cortinas da tenda, penderão de um e de outro lado do tabernáculo para o cobrir.
\par 14 Também farás de peles de carneiro tintas de vermelho uma coberta para a tenda e outra coberta de peles finas.
\par 15 Farás também de madeira de acácia as tábuas para o tabernáculo, as quais serão colocadas verticalmente.
\par 16 Cada uma das tábuas terá dez côvados de comprimento e côvado e meio de largura.
\par 17 Cada tábua terá dois encaixes, travados um com o outro; assim farás com todas as tábuas do tabernáculo.
\par 18 No preparar as tábuas para o tabernáculo, farás vinte delas para o lado sul.
\par 19 Farás também quarenta bases de prata debaixo das vinte tábuas: duas bases debaixo de uma tábua para os seus dois encaixes e duas bases debaixo de outra tábua para os seus dois encaixes.
\par 20 Também haverá vinte tábuas ao outro lado do tabernáculo, para o lado norte,
\par 21 com as suas quarenta bases de prata: duas bases debaixo de uma tábua e duas bases debaixo de outra tábua;
\par 22 ao lado posterior do tabernáculo para o ocidente, farás seis tábuas.
\par 23 Farás também duas tábuas para os cantos do tabernáculo, na parte posterior;
\par 24 as quais, por baixo, estarão separadas, mas, em cima, se ajustarão à primeira argola; assim se fará com as duas tábuas; serão duas para cada um dos dois cantos.
\par 25 Assim serão as oito tábuas com as suas bases de prata, dezesseis bases: duas bases debaixo de uma tábua e duas debaixo de outra tábua.
\par 26 Farás travessas de madeira de acácia; cinco para as tábuas de um lado do tabernáculo,
\par 27 cinco para as tábuas do outro lado do tabernáculo e cinco para as tábuas do tabernáculo ao lado posterior que olha para o ocidente.
\par 28 A travessa do meio passará ao meio das tábuas de uma extremidade à outra.
\par 29 Cobrirás de ouro as tábuas e de ouro farás as suas argolas, pelas quais hão de passar as travessas; e cobrirás também de ouro as travessas.
\par 30 Levantarás o tabernáculo segundo o modelo que te foi mostrado no monte.
\par 31 Farás também um véu de estofo azul, e púrpura, e carmesim, e linho fino retorcido; com querubins, o farás de obra de artista.
\par 32 Suspendê-lo-ás sobre quatro colunas de madeira de acácia, cobertas de ouro; os seus colchetes serão de ouro, sobre quatro bases de prata.
\par 33 Pendurarás o véu debaixo dos colchetes e trarás para lá a arca do Testemunho, para dentro do véu; o véu vos fará separação entre o Santo Lugar e o Santo dos Santos.
\par 34 Porás a coberta do propiciatório sobre a arca do Testemunho no Santo dos Santos.
\par 35 A mesa porás fora do véu e o candelabro, defronte da mesa, ao lado do tabernáculo, para o sul; e a mesa porás para o lado norte.
\par 36 Farás também para a porta da tenda um reposteiro de estofo azul, e púrpura, e carmesim, e linho fino retorcido, obra de bordador.
\par 37 Para este reposteiro farás cinco colunas de madeira de acácia e as cobrirás de ouro; os seus colchetes serão de ouro, e para elas fundirás cinco bases de bronze.

\chapter{27}

\par 1 Farás também o altar de madeira de acácia; de cinco côvados será o seu comprimento, e de cinco, a largura (será quadrado o altar), e de três côvados, a altura.
\par 2 Dos quatro cantos farás levantar-se quatro chifres, os quais formarão uma só peça com o altar; e o cobrirás de bronze.
\par 3 Far-lhe-ás também recipientes para recolher a sua cinza, e pás, e bacias, e garfos, e braseiros; todos esses utensílios farás de bronze.
\par 4 Far-lhe-ás também uma grelha de bronze em forma de rede, à qual farás quatro argolas de metal nos seus quatro cantos,
\par 5 e as porás dentro do rebordo do altar para baixo, de maneira que a rede chegue até ao meio do altar.
\par 6 Farás também varais para o altar, varais de madeira de acácia, e os cobrirás de bronze.
\par 7 Os varais se meterão nas argolas, de um e de outro lado do altar, quando for levado.
\par 8 Oco e de tábuas o farás; como se te mostrou no monte, assim o farão.
\par 9 Farás também o átrio do tabernáculo; ao lado meridional (que dá para o sul), o átrio terá cortinas de linho fino retorcido; o comprimento de cada lado será de cem côvados.
\par 10 Também as suas vinte colunas e as suas vinte bases serão de bronze; os ganchos das colunas e as sua vergas serão de prata.
\par 11 De igual modo, para o lado norte ao comprido, haverá cortinas de cem côvados de comprimento; e as suas vinte colunas e as suas vinte bases serão de bronze; os ganchos das colunas e as suas vergas serão de prata.
\par 12 Na largura do átrio para o lado do ocidente, haverá cortinas de cinqüenta côvados; as colunas serão dez, e as suas bases, dez.
\par 13 A largura do átrio do lado oriental (para o levante) será de cinqüenta côvados.
\par 14 As cortinas para um lado da entrada serão de quinze côvados; as suas colunas serão três, e as suas bases, três.
\par 15 Para o outro lado da entrada, haverá cortinas de quinze côvados; as suas colunas serão três, e as suas bases, três.
\par 16 À porta do átrio, haverá um reposteiro de vinte côvados, de estofo azul, e púrpura, e carmesim, e linho fino retorcido, obra de bordador; as suas colunas serão quatro, e as suas bases, quatro.
\par 17 Todas as colunas ao redor do átrio serão cingidas de vergas de prata; os seus ganchos serão de prata, mas as suas bases, de bronze.
\par 18 O átrio terá cem côvados de comprimento, e cinqüenta de largura por todo o lado, e cinco de altura; as suas cortinas serão de linho fino retorcido, e as suas bases, de bronze.
\par 19 Todos os utensílios do tabernáculo em todo o seu serviço, e todas as suas estacas, e todas as estacas do átrio serão de bronze.
\par 20 Ordenarás aos filhos de Israel que te tragam azeite puro de oliveira, batido, para o candelabro, para que haja lâmpada acesa continuamente.
\par 21 Na tenda da congregação fora do véu, que está diante do Testemunho, Arão e seus filhos a conservarão em ordem, desde a tarde até pela manhã, perante o SENHOR; estatuto perpétuo será este a favor dos filhos de Israel pelas suas gerações.

\chapter{28}

\par 1 Faze também vir para junto de ti Arão, teu irmão, e seus filhos com ele, dentre os filhos de Israel, para me oficiarem como sacerdotes, a saber, Arão e seus filhos Nadabe, Abiú, Eleazar e Itamar.
\par 2 Farás vestes sagradas para Arão, teu irmão, para glória e ornamento.
\par 3 Falarás também a todos os homens hábeis a quem enchi do espírito de sabedoria, que façam vestes para Arão para consagrá-lo, para que me ministre o ofício sacerdotal.
\par 4 As vestes, pois, que farão são estas: um peitoral, uma estola sacerdotal, uma sobrepeliz, uma túnica bordada, mitra e cinto. Farão vestes sagradas para Arão, teu irmão, e para seus filhos, para me oficiarem como sacerdotes.
\par 5 Tomarão ouro, estofo azul, púrpura, carmesim e linho fino
\par 6 e farão a estola sacerdotal de ouro, e estofo azul, e púrpura, e carmesim, e linho fino retorcido, obra esmerada.
\par 7 Terá duas ombreiras que se unam às suas duas extremidades, e assim se unirá.
\par 8 E o cinto de obra esmerada, que estará sobre a estola sacerdotal, será de obra igual, da mesma obra de ouro, e estofo azul, e púrpura, e carmesim, e linho fino retorcido.
\par 9 Tomarás duas pedras de ônix e gravarás nelas os nomes dos filhos de Israel:
\par 10 seis de seus nomes numa pedra e os outros seis na outra pedra, segundo a ordem do seu nascimento.
\par 11 Conforme a obra de lapidador, como lavores de sinete, gravarás as duas pedras com os nomes dos filhos de Israel; engastadas ao redor de ouro, as farás.
\par 12 E porás as duas pedras nas ombreiras da estola sacerdotal, por pedras de memória aos filhos de Israel; e Arão levará os seus nomes sobre ambos os seus ombros, para memória diante do SENHOR.
\par 13 Farás também engastes de ouro
\par 14 e duas correntes de ouro puro; obra de fieira as farás; e as correntes de fieira prenderás nos engastes.
\par 15 Farás também o peitoral do juízo de obra esmerada, conforme a obra da estola sacerdotal o farás: de ouro, e estofo azul, e púrpura, e carmesim, e linho fino retorcido o farás.
\par 16 Quadrado e duplo, será de um palmo o seu comprimento, e de um palmo, a sua largura.
\par 17 Colocarás nele engaste de pedras, com quatro ordens de pedras: a ordem de sárdio, topázio e carbúnculo será a primeira ordem;
\par 18 a segunda ordem será de esmeralda, safira e diamante;
\par 19 a terceira ordem será de jacinto, ágata e ametista;
\par 20 a quarta ordem será de berilo, ônix e jaspe; elas serão guarnecidas de ouro nos seus engastes.
\par 21 As pedras serão conforme os nomes dos filhos de Israel, doze, segundo os seus nomes; serão esculpidas como sinetes, cada uma com o seu nome, para as doze tribos.
\par 22 Para o peitoral farás correntes como cordas, de obra trançada de ouro puro.
\par 23 Também farás para o peitoral duas argolas de ouro e porás as duas argolas nas extremidades do peitoral.
\par 24 Então, meterás as duas correntes de ouro nas duas argolas, nas extremidades do peitoral.
\par 25 As duas pontas das correntes prenderás nos dois engastes e as porás nas ombreiras da estola sacerdotal na frente dele.
\par 26 Farás também duas argolas de ouro e as porás nas duas extremidades do peitoral, na sua orla interior junto à estola sacerdotal.
\par 27 Farás também duas argolas de ouro e as porás nas duas ombreiras da estola sacerdotal, abaixo, na frente dele, perto da sua juntura, sobre o cinto de obra esmerada da estola sacerdotal.
\par 28 E ligarão o peitoral com as suas argolas às argolas da estola sacerdotal por cima com uma fita azul, para que esteja sobre o cinto da estola sacerdotal; e nunca o peitoral se separará da estola sacerdotal.
\par 29 Assim, Arão levará os nomes dos filhos de Israel no peitoral do juízo sobre o seu coração, quando entrar no santuário, para memória diante do SENHOR continuamente.
\par 30 Também porás no peitoral do juízo o Urim e o Tumim, para que estejam sobre o coração de Arão, quando entrar perante o SENHOR; assim, Arão levará o juízo dos filhos de Israel sobre o seu coração diante do SENHOR continuamente.
\par 31 Farás também a sobrepeliz da estola sacerdotal toda de estofo azul.
\par 32 No meio dela, haverá uma abertura para a cabeça; será debruada essa abertura, como a abertura de uma saia de malha, para que não se rompa.
\par 33 Em toda a orla da sobrepeliz, farás romãs de estofo azul, e púrpura, e carmesim; e campainhas de ouro no meio delas.
\par 34 Haverá em toda a orla da sobrepeliz uma campainha de ouro e uma romã, outra campainha de ouro e outra romã.
\par 35 Esta sobrepeliz estará sobre Arão quando ministrar, para que se ouça o seu sonido, quando entrar no santuário diante do SENHOR e quando sair; e isso para que não morra.
\par 36 Farás também uma lâmina de ouro puro e nela gravarás à maneira de gravuras de sinetes: Santidade ao SENHOR.
\par 37 Atá-la-ás com um cordão de estofo azul, de maneira que esteja na mitra; bem na frente da mitra estará.
\par 38 E estará sobre a testa de Arão, para que Arão leve a iniqüidade concernente às coisas santas que os filhos de Israel consagrarem em todas as ofertas de suas coisas santas; sempre estará sobre a testa de Arão, para que eles sejam aceitos perante o SENHOR.
\par 39 Tecerás, quadriculada, a túnica de linho fino e farás uma mitra de linho fino e um cinto de obra de bordador.
\par 40 Para os filhos de Arão farás túnicas, e cintos, e tiaras; fá-los-ás para glória e ornamento.
\par 41 E, com isso, vestirás Arão, teu irmão, bem como seus filhos; e os ungirás, e consagrarás, e santificarás, para que me oficiem como sacerdotes.
\par 42 Faze-lhes também calções de linho, para cobrirem a pele nua; irão da cintura às coxas.
\par 43 E estarão sobre Arão e sobre seus filhos, quando entrarem na tenda da congregação ou quando se chegarem ao altar para ministrar no santuário, para que não levem iniqüidade e morram; isto será estatuto perpétuo para ele e para sua posteridade depois dele.

\chapter{29}

\par 1 Isto é o que lhes farás, para os consagrar, a fim de que me oficiem como sacerdotes: toma um novilho, e dois carneiros sem defeito,
\par 2 e pães asmos, e bolos asmos, amassados com azeite, e obreias asmas untadas com azeite; de flor de farinha de trigo os farás,
\par 3 e os porás num cesto, e no cesto os trarás; trarás também o novilho e os dois carneiros.
\par 4 Então, farás que Arão e seus filhos se cheguem à porta da tenda da congregação e os lavarás com água;
\par 5 depois, tomarás as vestes, e vestirás Arão da túnica, da sobrepeliz, da estola sacerdotal e do peitoral, e o cingirás com o cinto de obra esmerada da estola sacerdotal;
\par 6 pôr-lhe-ás a mitra na cabeça e sobre a mitra, a coroa sagrada.
\par 7 Então, tomarás o óleo da unção e lho derramarás sobre a cabeça; assim o ungirás.
\par 8 Farás, depois, que se cheguem os filhos de Arão, e os vestirás de túnicas,
\par 9 e os cingirás com o cinto, Arão e seus filhos, e lhes atarás as tiaras, para que tenham o sacerdócio por estatuto perpétuo, e consagrarás Arão e seus filhos.
\par 10 Farás chegar o novilho diante da tenda da congregação, e Arão e seus filhos porão as mãos sobre a cabeça dele.
\par 11 Imolarás o novilho perante o SENHOR, à porta da tenda da congregação.
\par 12 Depois, tomarás do sangue do novilho e o porás com o teu dedo sobre os chifres do altar; o restante do sangue derramá-lo-ás à base do altar.
\par 13 Também tomarás toda a gordura que cobre as entranhas, o redenho do fígado, os dois rins e a gordura que está neles e queimá-los-ás sobre o altar;
\par 14 mas a carne do novilho, a pele e os excrementos, queimá-los-ás fora do arraial; é sacrifício pelo pecado.
\par 15 Depois, tomarás um carneiro, e Arão e seus filhos porão as mãos sobre a cabeça dele.
\par 16 Imolarás o carneiro, e tomarás o seu sangue, e o jogarás sobre o altar ao redor;
\par 17 partirás o carneiro em seus pedaços e, lavadas as entranhas e as pernas, pô-las-ás sobre os pedaços e sobre a cabeça.
\par 18 Assim, queimarás todo o carneiro sobre o altar; é holocausto para o SENHOR, de aroma agradável, oferta queimada ao SENHOR.
\par 19 Depois, tomarás o outro carneiro, e Arão e seus filhos porão as mãos sobre a cabeça dele.
\par 20 Imolarás o carneiro, e tomarás do seu sangue, e o porás sobre a ponta da orelha direita de Arão e sobre a ponta da orelha direita de seus filhos, como também sobre o polegar da sua mão direita e sobre o polegar do seu pé direito; o restante do sangue jogarás sobre o altar ao redor.
\par 21 Tomarás, então, do sangue sobre o altar e do óleo da unção e os aspergirás sobre Arão e suas vestes e sobre seus filhos e as vestes de seus filhos com ele; para que ele seja santificado, e as suas vestes, e também seus filhos e as vestes de seus filhos com ele.
\par 22 Depois, tomarás do carneiro a gordura, a cauda gorda, a gordura que cobre as entranhas, o redenho do fígado, os dois rins, a gordura que está neles e a coxa direita, porque é carneiro da consagração;
\par 23 e também um pão, um bolo de pão azeitado e uma obreia do cesto dos pães asmos que estão diante do SENHOR.
\par 24 Todas estas coisas porás nas mãos de Arão e nas de seus filhos e, movendo-as de um lado para outro, as oferecerás como ofertas movidas perante o SENHOR.
\par 25 Depois, as tomarás das suas mãos e as queimarás sobre o altar; é holocausto para o SENHOR, de agradável aroma, oferta queimada ao SENHOR.
\par 26 Tomarás o peito do carneiro da consagração, que é de Arão, e, movendo-o de um lado para outro, o oferecerás como oferta movida perante o SENHOR; e isto será a tua porção.
\par 27 Consagrarás o peito da oferta movida e a coxa da porção que foi movida, a qual se tirou do carneiro da consagração, que é de Arão e de seus filhos.
\par 28 Isto será a obrigação perpétua dos filhos de Israel, devida a Arão e seus filhos, por ser a porção do sacerdote, oferecida, da parte dos filhos de Israel, dos sacrifícios pacíficos; é a sua oferta ao SENHOR.
\par 29 As vestes santas de Arão passarão a seus filhos depois dele, para serem ungidos nelas e consagrados nelas.
\par 30 Sete dias as vestirá o filho que for sacerdote em seu lugar, quando entrar na tenda da congregação para ministrar no santuário.
\par 31 Tomarás o carneiro da consagração e cozerás a sua carne no lugar santo;
\par 32 e Arão e seus filhos comerão a carne deste carneiro e o pão que está no cesto à porta da tenda da congregação
\par 33 e comerão das coisas com que for feita a expiação, para consagrá-los e para santificá-los; o estranho não comerá delas, porque são santas.
\par 34 Se sobrar alguma coisa da carne das consagrações ou do pão, até pela manhã, queimarás o que restar; não se comerá, porque é santo.
\par 35 Assim, pois, farás a Arão e a seus filhos, conforme tudo o que te hei ordenado; por sete dias, os consagrarás.
\par 36 Também cada dia prepararás um novilho como oferta pelo pecado para as expiações; e purificarás o altar, fazendo expiação por ele mediante oferta pelo pecado; e o ungirás para consagrá-lo.
\par 37 Sete dias farás expiação pelo altar e o consagrarás; e o altar será santíssimo; tudo o que o tocar será santo.
\par 38 Isto é o que oferecerás sobre o altar: dois cordeiros de um ano, cada dia, continuamente.
\par 39 Um cordeiro oferecerás pela manhã e o outro, ao pôr-do-sol.
\par 40 Com um cordeiro, a décima parte de um efa de flor de farinha, amassada com a quarta parte de um him de azeite batido; e, para libação, a quarta parte de um him de vinho;
\par 41 o outro cordeiro oferecerás ao pôr-do-sol, como oferta de manjares, e a libação como de manhã, de aroma agradável, oferta queimada ao SENHOR.
\par 42 Este será o holocausto contínuo por vossas gerações, à porta da tenda da congregação, perante o SENHOR, onde vos encontrarei, para falar contigo ali.
\par 43 Ali, virei aos filhos de Israel, para que, por minha glória, sejam santificados,
\par 44 e consagrarei a tenda da congregação e o altar; também santificarei Arão e seus filhos, para que me oficiem como sacerdotes.
\par 45 E habitarei no meio dos filhos de Israel e serei o seu Deus.
\par 46 E saberão que eu sou o SENHOR, seu Deus, que os tirou da terra do Egito, para habitar no meio deles; eu sou o SENHOR, seu Deus.

\chapter{30}

\par 1 Farás também um altar para queimares nele o incenso; de madeira de acácia o farás.
\par 2 Terá um côvado de comprimento, e um de largura (será quadrado), e dois de altura; os chifres formarão uma só peça com ele.
\par 3 De ouro puro o cobrirás, a parte superior, as paredes ao redor e os chifres; e lhe farás uma bordadura de ouro ao redor.
\par 4 Também lhe farás duas argolas de ouro debaixo da bordadura; de ambos os lados as farás; nelas, se meterão os varais para se levar o altar.
\par 5 De madeira de acácia farás os varais e os cobrirás de ouro.
\par 6 Porás o altar defronte do véu que está diante da arca do Testemunho, diante do propiciatório que está sobre o Testemunho, onde me avistarei contigo.
\par 7 Arão queimará sobre ele o incenso aromático; cada manhã, quando preparar as lâmpadas, o queimará.
\par 8 Quando, ao crepúsculo da tarde, acender as lâmpadas, o queimará; será incenso contínuo perante o SENHOR, pelas vossas gerações.
\par 9 Não oferecereis sobre ele incenso estranho, nem holocausto, nem ofertas de manjares; nem tampouco derramareis libações sobre ele.
\par 10 Uma vez no ano, Arão fará expiação sobre os chifres do altar com o sangue da oferta pelo pecado; uma vez no ano, fará expiação sobre ele, pelas vossas gerações; santíssimo é ao SENHOR.
\par 11 Disse mais o SENHOR a Moisés:
\par 12 Quando fizeres recenseamento dos filhos de Israel, cada um deles dará ao SENHOR o resgate de si próprio, quando os contares; para que não haja entre eles praga nenhuma, quando os arrolares.
\par 13 Todo aquele que passar ao arrolamento dará isto: metade de um siclo, segundo o siclo do santuário (este siclo é de vinte geras); a metade de um siclo é a oferta ao SENHOR.
\par 14 Qualquer que entrar no arrolamento, de vinte anos para cima, dará a oferta ao SENHOR.
\par 15 O rico não dará mais de meio siclo, nem o pobre, menos, quando derem a oferta ao SENHOR, para fazerdes expiação pela vossa alma.
\par 16 Tomarás o dinheiro das expiações dos filhos de Israel e o darás ao serviço da tenda da congregação; e será para memória aos filhos de Israel diante do SENHOR, para fazerdes expiação pela vossa alma.
\par 17 Disse mais o SENHOR a Moisés:
\par 18 Farás também uma bacia de bronze com o seu suporte de bronze, para lavar. Pô-la-ás entre a tenda da congregação e o altar e deitarás água nela.
\par 19 Nela, Arão e seus filhos lavarão as mãos e os pés.
\par 20 Quando entrarem na tenda da congregação, lavar-se-ão com água, para que não morram; ou quando se chegarem ao altar para ministrar, para acender a oferta queimada ao SENHOR.
\par 21 Lavarão, pois, as mãos e os pés, para que não morram; e isto lhes será por estatuto perpétuo, a ele e à sua posteridade, através de suas gerações.
\par 22 Disse mais o SENHOR a Moisés:
\par 23 Tu, pois, toma das mais excelentes especiarias: de mirra fluida quinhentos siclos, de cinamomo odoroso a metade, a saber, duzentos e cinqüenta siclos, e de cálamo aromático duzentos e cinqüenta siclos,
\par 24 e de cássia quinhentos siclos, segundo o siclo do santuário, e de azeite de oliveira um him.
\par 25 Disto farás o óleo sagrado para a unção, o perfume composto segundo a arte do perfumista; este será o óleo sagrado da unção.
\par 26 Com ele ungirás a tenda da congregação, e a arca do Testemunho,
\par 27 e a mesa com todos os seus utensílios, e o candelabro com os seus utensílios, e o altar do incenso,
\par 28 e o altar do holocausto com todos os utensílios, e a bacia com o seu suporte.
\par 29 Assim consagrarás estas coisas, para que sejam santíssimas; tudo o que tocar nelas será santo.
\par 30 Também ungirás Arão e seus filhos e os consagrarás para que me oficiem como sacerdotes.
\par 31 Dirás aos filhos de Israel: Este me será o óleo sagrado da unção nas vossas gerações.
\par 32 Não se ungirá com ele o corpo do homem que não seja sacerdote, nem fareis outro semelhante, da mesma composição; é santo e será santo para vós outros.
\par 33 Qualquer que compuser óleo igual a este ou dele puser sobre um estranho será eliminado do seu povo.
\par 34 Disse mais o SENHOR a Moisés: Toma substâncias odoríferas, estoraque, ônica e gálbano; estes arômatas com incenso puro, cada um de igual peso;
\par 35 e disto farás incenso, perfume segundo a arte do perfumista, temperado com sal, puro e santo.
\par 36 Uma parte dele reduzirás a pó e o porás diante do Testemunho na tenda da congregação, onde me avistarei contigo; será para vós outros santíssimo.
\par 37 Porém o incenso que fareis, segundo a composição deste, não o fareis para vós mesmos; santo será para o SENHOR.
\par 38 Quem fizer tal como este para o cheirar será eliminado do seu povo.

\chapter{31}

\par 1 Disse mais o SENHOR a Moisés:
\par 2 Eis que chamei pelo nome a Bezalel, filho de Uri, filho de Hur, da tribo de Judá,
\par 3 e o enchi do Espírito de Deus, de habilidade, de inteligência e de conhecimento, em todo artifício,
\par 4 para elaborar desenhos e trabalhar em ouro, em prata, em bronze,
\par 5 para lapidação de pedras de engaste, para entalho de madeira, para toda sorte de lavores.
\par 6 Eis que lhe dei por companheiro Aoliabe, filho de Aisamaque, da tribo de Dã; e dei habilidade a todos os homens hábeis, para que me façam tudo o que tenho ordenado:
\par 7 a tenda da congregação, e a arca do Testemunho, e o propiciatório que está por cima dela, e todos os pertences da tenda;
\par 8 e a mesa com os seus utensílios, e o candelabro de ouro puro com todos os seus utensílios, e o altar do incenso;
\par 9 e o altar do holocausto com todos os seus utensílios e a bacia com seu suporte;
\par 10 e as vestes finamente tecidas, e as vestes sagradas do sacerdote Arão, e as vestes de seus filhos, para oficiarem como sacerdotes;
\par 11 e o óleo da unção e o incenso aromático para o santuário; eles farão tudo segundo tenho ordenado.
\par 12 Disse mais o SENHOR a Moisés:
\par 13 Tu, pois, falarás aos filhos de Israel e lhes dirás: Certamente, guardareis os meus sábados; pois é sinal entre mim e vós nas vossas gerações; para que saibais que eu sou o SENHOR, que vos santifica.
\par 14 Portanto, guardareis o sábado, porque é santo para vós outros; aquele que o profanar morrerá; pois qualquer que nele fizer alguma obra será eliminado do meio do seu povo.
\par 15 Seis dias se trabalhará, porém o sétimo dia é o sábado do repouso solene, santo ao SENHOR; qualquer que no dia do sábado fizer alguma obra morrerá.
\par 16 Pelo que os filhos de Israel guardarão o sábado, celebrando-o por aliança perpétua nas suas gerações.
\par 17 Entre mim e os filhos de Israel é sinal para sempre; porque, em seis dias, fez o SENHOR os céus e a terra, e, ao sétimo dia, descansou, e tomou alento.
\par 18 E, tendo acabado de falar com ele no monte Sinai, deu a Moisés as duas tábuas do Testemunho, tábuas de pedra, escritas pelo dedo de Deus.

\chapter{32}

\par 1 Mas, vendo o povo que Moisés tardava em descer do monte, acercou-se de Arão e lhe disse: Levanta-te, faze-nos deuses que vão adiante de nós; pois, quanto a este Moisés, o homem que nos tirou do Egito, não sabemos o que lhe terá sucedido.
\par 2 Disse-lhes Arão: Tirai as argolas de ouro das orelhas de vossas mulheres, vossos filhos e vossas filhas e trazei-mas.
\par 3 Então, todo o povo tirou das orelhas as argolas e as trouxe a Arão.
\par 4 Este, recebendo-as das suas mãos, trabalhou o ouro com buril e fez dele um bezerro fundido. Então, disseram: São estes, ó Israel, os teus deuses, que te tiraram da terra do Egito.
\par 5 Arão, vendo isso, edificou um altar diante dele e, apregoando, disse: Amanhã, será festa ao SENHOR.
\par 6 No dia seguinte, madrugaram, e ofereceram holocaustos, e trouxeram ofertas pacíficas; e o povo assentou-se para comer e beber e levantou-se para divertir-se.
\par 7 Então, disse o SENHOR a Moisés: Vai, desce; porque o teu povo, que fizeste sair do Egito, se corrompeu
\par 8 e depressa se desviou do caminho que lhe havia eu ordenado; fez para si um bezerro fundido, e o adorou, e lhe sacrificou, e diz: São estes, ó Israel, os teus deuses, que te tiraram da terra do Egito.
\par 9 Disse mais o SENHOR a Moisés: Tenho visto este povo, e eis que é povo de dura cerviz.
\par 10 Agora, pois, deixa-me, para que se acenda contra eles o meu furor, e eu os consuma; e de ti farei uma grande nação.
\par 11 Porém Moisés suplicou ao SENHOR, seu Deus, e disse: Por que se acende, SENHOR, a tua ira contra o teu povo, que tiraste da terra do Egito com grande fortaleza e poderosa mão?
\par 12 Por que hão de dizer os egípcios: Com maus intentos os tirou, para matá-los nos montes e para consumi-los da face da terra? Torna-te do furor da tua ira e arrepende-te deste mal contra o teu povo.
\par 13 Lembra-te de Abraão, de Isaque e de Israel, teus servos, aos quais por ti mesmo tens jurado e lhes disseste: Multiplicarei a vossa descendência como as estrelas do céu, e toda esta terra de que tenho falado, dá-la-ei à vossa descendência, para que a possuam por herança eternamente.
\par 14 Então, se arrependeu o SENHOR do mal que dissera havia de fazer ao povo.
\par 15 E, voltando-se, desceu Moisés do monte com as duas tábuas do Testemunho nas mãos, tábuas escritas de ambos os lados; de um e de outro lado estavam escritas.
\par 16 As tábuas eram obra de Deus; também a escritura era a mesma escritura de Deus, esculpida nas tábuas.
\par 17 Ouvindo Josué a voz do povo que gritava, disse a Moisés: Há alarido de guerra no arraial.
\par 18 Respondeu-lhe Moisés: Não é alarido dos vencedores nem alarido dos vencidos, mas alarido dos que cantam é o que ouço.
\par 19 Logo que se aproximou do arraial, viu ele o bezerro e as danças; então, acendendo-se-lhe a ira, arrojou das mãos as tábuas e quebrou-as ao pé do monte;
\par 20 e, pegando no bezerro que tinham feito, queimou-o, e o reduziu a pó, que espalhou sobre a água, e deu de beber aos filhos de Israel.
\par 21 Depois, perguntou Moisés a Arão: Que te fez este povo, que trouxeste sobre ele tamanho pecado?
\par 22 Respondeu-lhe Arão: Não se acenda a ira do meu senhor; tu sabes que o povo é propenso para o mal.
\par 23 Pois me disseram: Faze-nos deuses que vão adiante de nós; pois, quanto a este Moisés, o homem que nos tirou da terra do Egito, não sabemos o que lhe terá acontecido.
\par 24 Então, eu lhes disse: quem tem ouro, tire-o. Deram-mo; e eu o lancei no fogo, e saiu este bezerro.
\par 25 Vendo Moisés que o povo estava desenfreado, pois Arão o deixara à solta para vergonha no meio dos seus inimigos,
\par 26 pôs-se em pé à entrada do arraial e disse: Quem é do SENHOR venha até mim. Então, se ajuntaram a ele todos os filhos de Levi,
\par 27 aos quais disse: Assim diz o SENHOR, o Deus de Israel: Cada um cinja a espada sobre o lado, passai e tornai a passar pelo arraial de porta em porta, e mate cada um a seu irmão, cada um, a seu amigo, e cada um, a seu vizinho.
\par 28 E fizeram os filhos de Levi segundo a palavra de Moisés; e caíram do povo, naquele dia, uns três mil homens.
\par 29 Pois Moisés dissera: Consagrai-vos, hoje, ao SENHOR; cada um contra o seu filho e contra o seu irmão, para que ele vos conceda, hoje, bênção.
\par 30 No dia seguinte, disse Moisés ao povo: Vós cometestes grande pecado; agora, porém, subirei ao SENHOR e, porventura, farei propiciação pelo vosso pecado.
\par 31 Tornou Moisés ao SENHOR e disse: Ora, o povo cometeu grande pecado, fazendo para si deuses de ouro.
\par 32 Agora, pois, perdoa-lhe o pecado; ou, se não, risca-me, peço-te, do livro que escreveste.
\par 33 Então, disse o SENHOR a Moisés: Riscarei do meu livro todo aquele que pecar contra mim.
\par 34 Vai, pois, agora, e conduze o povo para onde te disse; eis que o meu Anjo irá adiante de ti; porém, no dia da minha visitação, vingarei, neles, o seu pecado.
\par 35 Feriu, pois, o SENHOR ao povo, porque fizeram o bezerro que Arão fabricara.

\chapter{33}

\par 1 Disse o SENHOR a Moisés: Vai, sobe daqui, tu e o povo que tiraste da terra do Egito, para a terra a respeito da qual jurei a Abraão, a Isaque e a Jacó, dizendo: à tua descendência a darei.
\par 2 Enviarei o Anjo adiante de ti; lançarei fora os cananeus, os amorreus, os heteus, os ferezeus, os heveus e os jebuseus.
\par 3 Sobe para uma terra que mana leite e mel; eu não subirei no meio de ti, porque és povo de dura cerviz, para que te não consuma eu no caminho.
\par 4 Ouvindo o povo estas más notícias, pôs-se a prantear, e nenhum deles vestiu seus atavios.
\par 5 Porquanto o SENHOR tinha dito a Moisés: Dize aos filhos de Israel: És povo de dura cerviz; se por um momento eu subir no meio de ti, te consumirei; tira, pois, de ti os atavios, para que eu saiba o que te hei de fazer.
\par 6 Então, os filhos de Israel tiraram de si os seus atavios desde o monte Horebe em diante.
\par 7 Ora, Moisés costumava tomar a tenda e armá-la para si, fora, bem longe do arraial; e lhe chamava a tenda da congregação. Todo aquele que buscava ao SENHOR saía à tenda da congregação, que estava fora do arraial.
\par 8 Quando Moisés saía para a tenda, fora, todo o povo se erguia, cada um em pé à porta da sua tenda, e olhavam pelas costas, até entrar ele na tenda.
\par 9 Uma vez dentro Moisés da tenda, descia a coluna de nuvem e punha-se à porta da tenda; e o SENHOR falava com Moisés.
\par 10 Todo o povo via a coluna de nuvem que se detinha à porta da tenda; todo o povo se levantava, e cada um, à porta da sua tenda, adorava ao SENHOR.
\par 11 Falava o SENHOR a Moisés face a face, como qualquer fala a seu amigo; então, voltava Moisés para o arraial, porém o moço Josué, seu servidor, filho de Num, não se apartava da tenda.
\par 12 Disse Moisés ao SENHOR: Tu me dizes: Faze subir este povo, porém não me deste saber a quem hás de enviar comigo; contudo, disseste: Conheço-te pelo teu nome; também achaste graça aos meus olhos.
\par 13 Agora, pois, se achei graça aos teus olhos, rogo-te que me faças saber neste momento o teu caminho, para que eu te conheça e ache graça aos teus olhos; e considera que esta nação é teu povo.
\par 14 Respondeu-lhe: A minha presença irá contigo, e eu te darei descanso.
\par 15 Então, lhe disse Moisés: Se a tua presença não vai comigo, não nos faças subir deste lugar.
\par 16 Pois como se há de saber que achamos graça aos teus olhos, eu e o teu povo? Não é, porventura, em andares conosco, de maneira que somos separados, eu e o teu povo, de todos os povos da terra?
\par 17 Disse o SENHOR a Moisés: Farei também isto que disseste; porque achaste graça aos meus olhos, e eu te conheço pelo teu nome.
\par 18 Então, ele disse: Rogo-te que me mostres a tua glória.
\par 19 Respondeu-lhe: Farei passar toda a minha bondade diante de ti e te proclamarei o nome do SENHOR; terei misericórdia de quem eu tiver misericórdia e me compadecerei de quem eu me compadecer.
\par 20 E acrescentou: Não me poderás ver a face, porquanto homem nenhum verá a minha face e viverá.
\par 21 Disse mais o SENHOR: Eis aqui um lugar junto a mim; e tu estarás sobre a penha.
\par 22 Quando passar a minha glória, eu te porei numa fenda da penha e com a mão te cobrirei, até que eu tenha passado.
\par 23 Depois, em tirando eu a mão, tu me verás pelas costas; mas a minha face não se verá.

\chapter{34}

\par 1 Então, disse o SENHOR a Moisés: Lavra duas tábuas de pedra, como as primeiras; e eu escreverei nelas as mesmas palavras que estavam nas primeiras tábuas, que quebraste.
\par 2 E prepara-te para amanhã, para que subas, pela manhã, ao monte Sinai e ali te apresentes a mim no cimo do monte.
\par 3 Ninguém suba contigo, ninguém apareça em todo o monte; nem ainda ovelhas nem gado se apascentem defronte dele.
\par 4 Lavrou, pois, Moisés duas tábuas de pedra, como as primeiras; e, levantando-se pela manhã de madrugada, subiu ao monte Sinai, como o SENHOR lhe ordenara, levando nas mãos as duas tábuas de pedra.
\par 5 Tendo o SENHOR descido na nuvem, ali esteve junto dele e proclamou o nome do SENHOR.
\par 6 E, passando o SENHOR por diante dele, clamou: SENHOR, SENHOR Deus compassivo, clemente e longânimo e grande em misericórdia e fidelidade;
\par 7 que guarda a misericórdia em mil gerações, que perdoa a iniqüidade, a transgressão e o pecado, ainda que não inocenta o culpado, e visita a iniqüidade dos pais nos filhos e nos filhos dos filhos, até à terceira e quarta geração!
\par 8 E, imediatamente, curvando-se Moisés para a terra, o adorou;
\par 9 e disse: Senhor, se, agora, achei graça aos teus olhos, segue em nosso meio conosco; porque este povo é de dura cerviz. Perdoa a nossa iniqüidade e o nosso pecado e toma-nos por tua herança.
\par 10 Então, disse: Eis que faço uma aliança; diante de todo o teu povo farei maravilhas que nunca se fizeram em toda a terra, nem entre nação alguma, de maneira que todo este povo, em cujo meio tu estás, veja a obra do SENHOR; porque coisa terrível é o que faço contigo.
\par 11 Guarda o que eu te ordeno hoje: eis que lançarei fora da sua presença os amorreus, os cananeus, os heteus, os ferezeus, os heveus e os jebuseus.
\par 12 Abstém-te de fazer aliança com os moradores da terra para onde vais, para que te não sejam por cilada.
\par 13 Mas derribareis os seus altares, quebrareis as suas colunas e cortareis os seus postes-ídolos
\par 14 (porque não adorarás outro deus; pois o nome do SENHOR é Zeloso; sim, Deus zeloso é ele);
\par 15 para que não faças aliança com os moradores da terra; não suceda que, em se prostituindo eles com os deuses e lhes sacrificando, alguém te convide, e comas dos seus sacrifícios
\par 16 e tomes mulheres das suas filhas para os teus filhos, e suas filhas, prostituindo-se com seus deuses, façam que também os teus filhos se prostituam com seus deuses.
\par 17 Não farás para ti deuses fundidos.
\par 18 Guardarás a Festa dos Pães Asmos; sete dias comerás pães asmos, como te ordenei, no tempo indicado no mês de abibe; porque no mês de abibe saíste do Egito.
\par 19 Todo o que abre a madre é meu; também de todo o teu gado, sendo macho, o que abre a madre de vacas e de ovelhas.
\par 20 O jumento, porém, que abrir a madre, resgatá-lo-ás com cordeiro; mas, se o não resgatares, será desnucado. Remirás todos os primogênitos de teus filhos. Ninguém aparecerá diante de mim de mãos vazias.
\par 21 Seis dias trabalharás, mas, ao sétimo dia, descansarás, quer na aradura, quer na sega.
\par 22 Também guardarás a Festa das Semanas, que é a das primícias da sega do trigo, e a Festa da Colheita no fim do ano.
\par 23 Três vezes no ano, todo homem entre ti aparecerá perante o SENHOR Deus, Deus de Israel.
\par 24 Porque lançarei fora as nações de diante de ti e alargarei o teu território; ninguém cobiçará a tua terra quando subires para comparecer na presença do SENHOR, teu Deus, três vezes no ano.
\par 25 Não oferecerás o sangue do meu sacrifício com pão levedado; nem ficará o sacrifício da Festa da Páscoa da noite para a manhã.
\par 26 As primícias dos primeiros frutos da tua terra trarás à Casa do SENHOR, teu Deus. Não cozerás o cabrito no leite da sua própria mãe.
\par 27 Disse mais o SENHOR a Moisés: Escreve estas palavras, porque, segundo o teor destas palavras, fiz aliança contigo e com Israel.
\par 28 E, ali, esteve com o SENHOR quarenta dias e quarenta noites; não comeu pão, nem bebeu água; e escreveu nas tábuas as palavras da aliança, as dez palavras.
\par 29 Quando desceu Moisés do monte Sinai, tendo nas mãos as duas tábuas do Testemunho, sim, quando desceu do monte, não sabia Moisés que a pele do seu rosto resplandecia, depois de haver Deus falado com ele.
\par 30 Olhando Arão e todos os filhos de Israel para Moisés, eis que resplandecia a pele do seu rosto; e temeram chegar-se a ele.
\par 31 Então, Moisés os chamou; Arão e todos os príncipes da congregação tornaram a ele, e Moisés lhes falou.
\par 32 Depois, vieram também todos os filhos de Israel, aos quais ordenou ele tudo o que o SENHOR lhe falara no monte Sinai.
\par 33 Tendo Moisés acabado de falar com eles, pôs um véu sobre o rosto.
\par 34 Porém, vindo Moisés perante o SENHOR para falar-lhe, removia o véu até sair; e, saindo, dizia aos filhos de Israel tudo o que lhe tinha sido ordenado.
\par 35 Assim, pois, viam os filhos de Israel o rosto de Moisés, viam que a pele do seu rosto resplandecia; porém Moisés cobria de novo o rosto com o véu até entrar a falar com ele.

\chapter{35}

\par 1 Tendo Moisés convocado toda a congregação dos filhos de Israel, disse-lhes: São estas as palavras que o SENHOR ordenou que se cumprissem:
\par 2 Trabalhareis seis dias, mas o sétimo dia vos será santo, o sábado do repouso solene ao SENHOR; quem nele trabalhar morrerá.
\par 3 Não acendereis fogo em nenhuma das vossas moradas no dia do sábado.
\par 4 Disse mais Moisés a toda a congregação dos filhos de Israel: Esta é a palavra que o SENHOR ordenou, dizendo:
\par 5 Tomai, do que tendes, uma oferta para o SENHOR; cada um, de coração disposto, voluntariamente a trará por oferta ao SENHOR: ouro, prata, bronze,
\par 6 estofo azul, púrpura, carmesim, linho fino, pêlos de cabra,
\par 7 peles de carneiro tintas de vermelho, peles finas, madeira de acácia,
\par 8 azeite para a iluminação, especiarias para o óleo da unção e para o incenso aromático,
\par 9 pedras de ônix e pedras de engaste para a estola sacerdotal e para o peitoral.
\par 10 Venham todos os homens hábeis entre vós e façam tudo o que o SENHOR ordenou:
\par 11 o tabernáculo com sua tenda e a sua coberta, os seus ganchos, as suas tábuas, as sua vergas, as suas colunas e as suas bases;
\par 12 a arca e os seus varais, o propiciatório e o véu do reposteiro;
\par 13 a mesa e os seus varais, e todos os seus utensílios, e os pães da proposição;
\par 14 o candelabro da iluminação, e os seus utensílios, e as suas lâmpadas, e o azeite para a iluminação;
\par 15 o altar do incenso e os seus varais, e o óleo da unção, e o incenso aromático, e o reposteiro da porta à entrada do tabernáculo;
\par 16 o altar do holocausto e a sua grelha de bronze, os seus varais e todos os seus utensílios, a bacia e o seu suporte;
\par 17 as cortinas do átrio, e as suas colunas, e as suas bases, e o reposteiro da porta do átrio;
\par 18 as estacas do tabernáculo, e as estacas do átrio, e as suas cordas;
\par 19 as vestes do ministério para ministrar no santuário, as vestes santas do sacerdote Arão e as vestes de seus filhos, para oficiarem como sacerdotes.
\par 20 Então, toda a congregação dos filhos de Israel saiu da presença de Moisés,
\par 21 e veio todo homem cujo coração o moveu e cujo espírito o impeliu e trouxe a oferta ao SENHOR para a obra da tenda da congregação, e para todo o seu serviço, e para as vestes sagradas.
\par 22 Vieram homens e mulheres, todos dispostos de coração; trouxeram fivelas, pendentes, anéis, braceletes, todos os objetos de ouro; todo homem fazia oferta de ouro ao SENHOR;
\par 23 e todo homem possuidor de estofo azul, púrpura, carmesim, linho fino, pêlos de cabra, peles de carneiro tintas de vermelho e peles de animais marinhos os trazia.
\par 24 Todo aquele que fazia oferta de prata ou de bronze por oferta ao SENHOR a trazia; e todo possuidor de madeira de acácia para toda obra do serviço a trazia.
\par 25 Todas as mulheres hábeis traziam o que, por suas próprias mãos, tinham fiado: estofo azul, púrpura, carmesim e linho fino.
\par 26 E todas as mulheres cujo coração as moveu em habilidade fiavam os pêlos de cabra.
\par 27 Os príncipes traziam pedras de ônix, e pedras de engaste para a estola sacerdotal e para o peitoral,
\par 28 e os arômatas, e o azeite para a iluminação, e para o óleo da unção, e para o incenso aromático.
\par 29 Os filhos de Israel trouxeram oferta voluntária ao SENHOR, a saber, todo homem e mulher cujo coração os dispôs para trazerem uma oferta para toda a obra que o SENHOR tinha ordenado se fizesse por intermédio de Moisés.
\par 30 Disse Moisés aos filhos de Israel: Eis que o SENHOR chamou pelo nome a Bezalel, filho de Uri, filho de Hur, da tribo de Judá,
\par 31 e o Espírito de Deus o encheu de habilidade, inteligência e conhecimento em todo artifício,
\par 32 e para elaborar desenhos e trabalhar em ouro, em prata, em bronze,
\par 33 e para lapidação de pedras de engaste, e para entalho de madeira, e para toda sorte de lavores.
\par 34 Também lhe dispôs o coração para ensinar a outrem, a ele e a Aoliabe, filho de Aisamaque, da tribo de Dã.
\par 35 Encheu-os de habilidade para fazer toda obra de mestre, até a mais engenhosa, e a do bordador em estofo azul, em púrpura, em carmesim e em linho fino, e a do tecelão, sim, toda sorte de obra e a elaborar desenhos.

\chapter{36}

\par 1 Assim, trabalharam Bezalel, e Aoliabe, e todo homem hábil a quem o SENHOR dera habilidade e inteligência para saberem fazer toda obra para o serviço do santuário, segundo tudo o que o SENHOR havia ordenado.
\par 2 Moisés chamou a Bezalel, e a Aoliabe, e a todo homem hábil em cujo coração o SENHOR tinha posto sabedoria, isto é, a todo homem cujo coração o impeliu a se chegar à obra para fazê-la.
\par 3 Estes receberam de Moisés todas as ofertas que os filhos de Israel haviam trazido para a obra do serviço do santuário, para fazê-la; e, ainda, cada manhã o povo trazia a Moisés ofertas voluntárias.
\par 4 Então, deixando cada um a obra que fazia, vieram todos os homens sábios que se ocupavam em toda a obra do santuário
\par 5 e disseram a Moisés: O povo traz muito mais do que é necessário para o serviço da obra que o SENHOR ordenou se fizesse.
\par 6 Então, ordenou Moisés -- e a ordem foi proclamada no arraial, dizendo: Nenhum homem ou mulher faça mais obra alguma para a oferta do santuário. Assim, o povo foi proibido de trazer mais.
\par 7 Porque o material que tinham era suficiente para toda a obra que se devia fazer e ainda sobejava.
\par 8 Assim, todos os homens hábeis, entre os que faziam a obra, fizeram o tabernáculo com dez cortinas de linho fino retorcido, estofo azul, púrpura e carmesim com querubins; de obra de artista as fizeram.
\par 9 O comprimento de cada cortina era de vinte e oito côvados, e a largura, de quatro côvados; todas as cortinas eram de igual medida.
\par 10 Cinco cortinas eram ligadas uma à outra; e as outras cinco também ligadas uma à outra.
\par 11 Fizeram laçadas de estofo azul na orla da cortina, que estava na extremidade do primeiro agrupamento; e de igual modo fizeram na orla da cortina, que estava na extremidade do segundo agrupamento.
\par 12 Cinqüenta laçadas fizeram numa cortina, e cinqüenta, na outra cortina na extremidade do segundo agrupamento; as laçadas eram contrapostas uma à outra.
\par 13 Fizeram cinqüenta colchetes de ouro, com os quais prenderam as cortinas uma à outra; e o tabernáculo passou a ser um todo.
\par 14 Fizeram também de pêlos de cabra cortinas para servirem de tenda sobre o tabernáculo; fizeram onze cortinas.
\par 15 O comprimento de cada cortina era de trinta côvados, e a largura, de quatro côvados; as onze cortinas eram de igual medida.
\par 16 Ajuntaram à parte cinco cortinas entre si e, de igual modo, as seis restantes.
\par 17 E fizeram cinqüenta laçadas na orla da cortina, que estava na extremidade do primeiro agrupamento.
\par 18 Fizeram também cinqüenta colchetes de bronze para ajuntar a tenda, para que viesse a ser um todo.
\par 19 Fizeram também de peles de carneiro tintas de vermelho uma coberta para a tenda e outra coberta de peles finas.
\par 20 Fizeram também de madeira de acácia as tábuas para o tabernáculo, as quais eram colocadas verticalmente.
\par 21 Cada uma das tábuas tinha dez côvados de comprimento e côvado e meio de largura.
\par 22 Cada tábua tinha dois encaixes, travados um com o outro; assim fizeram com todas as tábuas do tabernáculo.
\par 23 No preparar as tábuas para o tabernáculo, fizeram vinte delas para o lado sul.
\par 24 Fizeram também quarenta bases de prata debaixo das vinte tábuas: duas bases debaixo de uma tábua para os seus dois encaixes e duas bases debaixo de outra tábua para os seus dois encaixes.
\par 25 Também fizeram vinte tábuas ao outro lado do tabernáculo, para o lado norte,
\par 26 com as suas quarenta bases de prata: duas bases debaixo de uma tábua e duas bases debaixo de outra tábua;
\par 27 ao lado do tabernáculo para o ocidente, fizeram seis tábuas.
\par 28 Fizeram também duas tábuas para os cantos do tabernáculo de ambos os lados,
\par 29 as quais, por baixo, estavam separadas, mas, em cima, se ajustavam à primeira argola; assim se fez com as duas tábuas nos dois cantos.
\par 30 Assim eram as oito tábuas com as suas bases de prata, dezesseis bases: duas bases debaixo de uma tábua e duas debaixo de outra tábua.
\par 31 Fizeram também travessas de madeira de acácia; cinco para as tábuas de um lado do tabernáculo,
\par 32 cinco para as tábuas do outro lado do tabernáculo e cinco para as tábuas do tabernáculo, ao lado posterior, que olha para o ocidente.
\par 33 A travessa do meio passava ao meio das tábuas, de uma extremidade à outra.
\par 34 Cobriram de ouro as tábuas e de ouro fizeram as suas argolas, pelas quais passavam as travessas; e cobriram também de ouro as travessas.
\par 35 Fizeram também o véu de estofo azul, púrpura, carmesim e linho fino retorcido; com querubins o fizeram de obra de artista.
\par 36 E fizeram-lhe quatro colunas de madeira de acácia, cobertas de ouro; os seus colchetes eram de ouro, sobre quatro bases de prata.
\par 37 Fizeram também para a porta da tenda um reposteiro de estofo azul, púrpura, carmesim e linho fino retorcido, obra de bordador,
\par 38 e as suas cinco colunas, e os seus colchetes; as suas cabeças e as suas molduras cobriram de ouro, mas as suas cinco bases eram de bronze.

\chapter{37}

\par 1 Fez também Bezalel a arca de madeira de acácia; de dois côvados e meio era o seu comprimento, de um côvado e meio, a largura, e, de um côvado e meio, a altura.
\par 2 De ouro puro a cobriu; por dentro e por fora a cobriu e fez uma bordadura de ouro ao redor.
\par 3 Fundiu para ela quatro argolas de ouro e as pôs nos quatro cantos da arca: duas argolas num lado dela e duas argolas noutro lado.
\par 4 Fez também varais de madeira de acácia e os cobriu de ouro;
\par 5 meteu os varais nas argolas aos lados da arca, para se levar por meio deles a arca.
\par 6 Fez também o propiciatório de ouro puro; de dois côvados e meio era o seu comprimento, e a largura, de um côvado e meio.
\par 7 Fez também dois querubins de ouro; de ouro batido os fez, nas duas extremidades do propiciatório.
\par 8 Um querubim, na extremidade de uma parte, e o outro, na extremidade da outra parte; de uma só peça com o propiciatório fez os querubins nas duas extremidades dele.
\par 9 Os querubins estendiam as asas por cima, cobrindo com elas o propiciatório; estavam eles de faces voltadas uma para a outra, olhando para o propiciatório.
\par 10 Fez também a mesa de madeira de acácia; tinha o comprimento de dois côvados, a largura, de um côvado, e a altura, de um côvado e meio.
\par 11 De ouro puro a cobriu e lhe fez uma bordadura de ouro ao redor.
\par 12 Também lhe fez moldura ao redor, na largura de quatro dedos, e lhe fez uma bordadura de ouro ao redor da moldura.
\par 13 Também lhe fundiu quatro argolas de ouro e pôs as argolas nos quatro cantos que estavam nos seus quatro pés.
\par 14 Perto da moldura estavam as argolas, como lugares para os varais, para se levar a mesa.
\par 15 Fez os varais de madeira de acácia e os cobriu de ouro, para se levar a mesa.
\par 16 Também fez de ouro puro os utensílios que haviam de estar sobre a mesa: os seus pratos, e os seus recipientes para incenso, e as suas galhetas, e as suas taças em que se haviam de oferecer libações.
\par 17 Fez também o candelabro de ouro puro; de ouro batido o fez; o seu pedestal, a sua hástea, os seus cálices, as suas maçanetas e as suas flores formavam com ele uma só peça.
\par 18 Seis hásteas saíam dos seus lados; três de um lado e três do outro.
\par 19 Numa hástea havia três cálices com formato de amêndoas, uma maçaneta e uma flor; e três cálices com formato de amêndoas na outra hástea, uma maçaneta e uma flor; assim eram as seis hásteas que saíam do candelabro.
\par 20 Mas no candelabro mesmo havia quatro cálices com formato de amêndoas, com suas maçanetas e com suas flores.
\par 21 Havia uma maçaneta sob duas hásteas que saíam dele; e ainda uma maçaneta sob duas outras hásteas que saíam dele; e ainda mais uma maçaneta sob duas outras hásteas que saíam dele; assim se fez com as seis hásteas que saíam do candelabro.
\par 22 As suas maçanetas e as suas hásteas eram do mesmo; tudo era de uma só peça, obra batida de ouro puro.
\par 23 Também lhe fez sete lâmpadas; as suas espevitadeiras e os seus apagadores eram de ouro puro.
\par 24 De um talento de ouro puro se fez o candelabro com todos os seus utensílios.
\par 25 Fez de madeira de acácia o altar do incenso; tinha um côvado de comprimento, e um de largura (era quadrado), e dois de altura; os chifres formavam uma só peça com ele.
\par 26 De ouro puro o cobriu, a parte superior, as paredes ao redor e os chifres; e lhe fez uma bordadura de ouro ao redor.
\par 27 Também lhe fez duas argolas de ouro debaixo da bordadura, de ambos os lados as fez; nelas, se meteram os varais para se levar o altar;
\par 28 de madeira de acácia fez os varais e os cobriu de ouro.
\par 29 Fez também o óleo santo da unção e o incenso aromático, puro, de obra de perfumista.

\chapter{38}

\par 1 Fez também o altar do holocausto de madeira de acácia; de cinco côvados era o comprimento, e de cinco, a largura (era quadrado o altar), e de três côvados, a altura.
\par 2 Dos quatro cantos fez levantar-se quatro chifres, os quais formavam uma só peça com o altar; e o cobriu de bronze.
\par 3 Fez também todos os utensílios do altar: recipientes para recolher as suas cinzas, e pás, e bacias, e garfos, e braseiros; todos esses utensílios, de bronze os fez.
\par 4 Fez também para o altar uma grelha de bronze em forma de rede, do rebordo do altar para baixo, a qual chegava até ao meio do altar.
\par 5 Fundiu quatro argolas para os quatro cantos da grelha de bronze, para nelas se meterem os varais.
\par 6 Fez os varais de madeira de acácia e os cobriu de bronze.
\par 7 Meteu os varais nas argolas, de um e de outro lado do altar, para ser levado; oco e de tábuas o fez.
\par 8 Fez também a bacia de bronze, com o seu suporte de bronze, dos espelhos das mulheres que se reuniam para ministrar à porta da tenda da congregação.
\par 9 Fez também o átrio ao lado meridional (que dá para o sul); as cortinas do átrio eram de linho fino retorcido, de cem côvados de comprimento.
\par 10 As suas vinte colunas e as suas bases eram de bronze; os ganchos das colunas e as suas vergas eram de prata.
\par 11 De igual modo para o lado norte havia cortinas de cem côvados de comprimento; as suas vinte colunas e as suas vinte bases eram de bronze; os ganchos das colunas e as suas vergas eram de prata.
\par 12 Para o lado do ocidente havia cortinas de cinqüenta côvados; as suas colunas eram dez, e as suas bases, dez; os ganchos das colunas e as suas vergas eram de prata.
\par 13 Do lado oriental (para o levante), eram as cortinas de cinqüenta côvados.
\par 14 As cortinas para um lado da entrada eram de quinze côvados; e as suas colunas eram três, e as suas bases, três.
\par 15 Para o outro lado da entrada do átrio, de um e de outro lado da entrada, eram as cortinas de quinze côvados; as suas colunas eram três, e as suas bases, três.
\par 16 Todas as cortinas ao redor do átrio eram de linho fino retorcido.
\par 17 As bases das colunas eram de bronze; os ganchos das colunas e as suas vergas eram de prata.
\par 18 O reposteiro da porta do átrio era de obra de bordador, de estofo azul, púrpura, carmesim e linho fino retorcido; o comprimento era de vinte côvados, e a altura, na largura, era de cinco côvados, segundo a medida das cortinas do átrio.
\par 19 As suas quatro colunas e as suas quatro bases eram de bronze, os seus ganchos eram de prata, e o revestimento das suas cabeças e as suas vergas, de prata.
\par 20 Todos os pregos do tabernáculo e do átrio ao redor eram de bronze.
\par 21 Esta é a enumeração das coisas para o tabernáculo, a saber, o tabernáculo do Testemunho, segundo, por ordem de Moisés, foram contadas para o serviço dos levitas, por intermédio de Itamar, filho do sacerdote Arão.
\par 22 Fez Bezalel, filho de Uri, filho de Hur, da tribo de Judá, tudo quanto o SENHOR ordenara a Moisés.
\par 23 E, com ele, Aoliabe, filho de Aisamaque, da tribo de Dã, mestre de obra, desenhista e bordador em estofo azul, púrpura, carmesim e linho fino.
\par 24 Todo o ouro empregado na obra, em toda a obra do santuário, a saber, o ouro da oferta, foram vinte e nove talentos e setecentos e trinta siclos, segundo o siclo do santuário.
\par 25 A prata dos arrolados da congregação foram cem talentos e mil e setecentos e setenta e cinco siclos, segundo o siclo do santuário:
\par 26 um beca por cabeça, isto é, meio siclo, segundo o siclo do santuário, de qualquer dos arrolados, de vinte anos para cima, que foram seiscentos e três mil quinhentos e cinqüenta.
\par 27 Empregaram-se cem talentos de prata para fundir as bases do santuário e as bases do véu; para as cem bases, cem talentos: um talento para cada base.
\par 28 Dos mil setecentos e setenta e cinco siclos, fez os colchetes das colunas, e cobriu as suas cabeças, e lhes fez as vergas.
\par 29 O bronze da oferta foram setenta talentos e dois mil e quatrocentos siclos.
\par 30 Dele fez as bases da porta da tenda da congregação, e o altar de bronze, e a sua grelha de bronze, e todos os utensílios do altar,
\par 31 e as bases do átrio ao redor, e as bases da porta do átrio, e todas as estacas do tabernáculo, e todas as estacas do átrio ao redor.

\chapter{39}

\par 1 Fizeram também de estofo azul, púrpura e carmesim as vestes, finamente tecidas, para ministrar no santuário, e também fizeram as vestes sagradas para Arão, como o SENHOR ordenara a Moisés.
\par 2 Fizeram a estola sacerdotal de ouro, estofo azul, púrpura, carmesim e linho fino retorcido.
\par 3 De ouro batido fizeram lâminas delgadas e as cortaram em fios, para permearem entre o estofo azul, a púrpura, o carmesim e o linho fino da obra de desenhista.
\par 4 Tinha duas ombreiras que se ajuntavam às suas duas extremidades, e assim se uniam.
\par 5 O cinto de obra esmerada, que estava sobre a estola sacerdotal, era de obra igual, da mesma obra de ouro, estofo azul, púrpura, carmesim e linho fino retorcido, segundo o SENHOR ordenara a Moisés.
\par 6 Também se prepararam as pedras de ônix, engastadas em ouro, trabalhadas como lavores de sinete, com os nomes dos filhos de Israel,
\par 7 e as puseram nas ombreiras da estola sacerdotal, por pedras de memória aos filhos de Israel, segundo o SENHOR ordenara a Moisés.
\par 8 Fizeram também o peitoral de obra esmerada, conforme a obra da estola sacerdotal: de ouro, estofo azul, púrpura, carmesim e linho fino retorcido.
\par 9 Era quadrado; duplo fizeram o peitoral: de um palmo era o seu comprimento, e de um palmo dobrado, a sua largura.
\par 10 Colocaram, nele, engastes de pedras, com quatro ordens de pedras: a ordem de sárdio, topázio e carbúnculo era a primeira;
\par 11 a segunda ordem era de esmeralda, safira e diamante;
\par 12 a terceira ordem era de jacinto, ágata e ametista;
\par 13 e a quarta ordem era de berilo, ônix e jaspe; eram elas guarnecidas de ouro nos seus engastes.
\par 14 As pedras eram conforme os nomes dos filhos de Israel, doze segundo os seus nomes; eram esculpidas como sinete, cada uma com o seu nome para as doze tribos.
\par 15 E fizeram para o peitoral correntes como cordas, de obra trançada de ouro puro.
\par 16 Também fizeram para o peitoral dois engastes de ouro e duas argolas de ouro; e puseram as duas argolas nas extremidades do peitoral.
\par 17 E meteram as duas correntes trançadas de ouro nas duas argolas, nas extremidades do peitoral.
\par 18 As outras duas pontas das duas correntes trançadas meteram nos dois engastes e as puseram nas ombreiras da estola sacerdotal, na frente dele.
\par 19 Fizeram também duas argolas de ouro e as puseram nas duas extremidades do peitoral, na sua orla interior oposta à estola sacerdotal.
\par 20 Fizeram também mais duas argolas de ouro e as puseram nas duas ombreiras da estola sacerdotal, abaixo, na frente dele, perto da sua juntura, sobre o cinto de obra esmerada da estola sacerdotal.
\par 21 E ligaram o peitoral com as suas argolas às argolas da estola sacerdotal, por cima com uma fita azul, para que estivesse sobre o cinto de obra esmerada da estola sacerdotal, e nunca o peitoral se separasse da estola sacerdotal, segundo o SENHOR ordenara a Moisés.
\par 22 Fizeram também a sobrepeliz da estola sacerdotal, de obra tecida, toda de estofo azul.
\par 23 No meio dela havia uma abertura; era debruada como abertura de uma saia de malha, para que se não rompesse.
\par 24 Em toda a orla da sobrepeliz, fizeram romãs de estofo azul, carmesim e linho retorcido.
\par 25 Fizeram campainhas de ouro puro e as colocaram no meio das romãs em toda a orla da sobrepeliz;
\par 26 uma campainha e uma romã, outra campainha e outra romã, em toda a orla da sobrepeliz, para se usar ao ministrar, segundo o SENHOR ordenara a Moisés.
\par 27 Fizeram também as túnicas de linho fino, de obra tecida, para Arão e para seus filhos,
\par 28 e a mitra de linho fino, e as tiaras de linho fino, e os calções de linho fino retorcido,
\par 29 e o cinto de linho fino retorcido, e de estofo azul, e de púrpura, e de carmesim, obra de bordador, segundo o SENHOR ordenara a Moisés.
\par 30 Também fizeram de ouro puro a lâmina da coroa sagrada e, nela, gravaram à maneira de gravuras de sinete: Santidade ao SENHOR.
\par 31 E ataram-na com um cordão de estofo azul, para prender a lâmina à parte superior da mitra, segundo o SENHOR ordenara a Moisés.
\par 32 Assim se concluiu toda a obra do tabernáculo da tenda da congregação; e os filhos de Israel fizeram tudo segundo o SENHOR tinha ordenado a Moisés; assim o fizeram.
\par 33 Depois, trouxeram a Moisés o tabernáculo, a tenda e todos os seus pertences, os seus colchetes, as suas tábuas, as suas vergas, as suas colunas e as suas bases;
\par 34 a coberta de peles de carneiro tintas de vermelho, e a coberta de peles finas, e o véu do reposteiro;
\par 35 a arca do Testemunho, e os seus varais, e o propiciatório;
\par 36 a mesa com todos os seus utensílios e os pães da proposição;
\par 37 o candelabro de ouro puro com suas lâmpadas; as lâmpadas colocadas em ordem, e todos os seus utensílios, e o azeite para a iluminação;
\par 38 também o altar de ouro, e o óleo da unção, e o incenso aromático, e o reposteiro da porta da tenda;
\par 39 o altar de bronze, e a sua grelha de bronze, e os seus varais, e todos os seus utensílios, e a bacia, e o seu suporte;
\par 40 as cortinas do átrio, e as suas colunas, e as suas bases, e o reposteiro para a porta do átrio, e as suas cordas, e os seus pregos, e todos os utensílios do serviço do tabernáculo, para a tenda da congregação;
\par 41 as vestes finamente tecidas para ministrar no santuário, e as vestes sagradas do sacerdote Arão, e as vestes de seus filhos, para oficiarem como sacerdotes.
\par 42 Tudo segundo o SENHOR ordenara a Moisés, assim fizeram os filhos de Israel toda a obra.
\par 43 Viu, pois, Moisés toda a obra, e eis que a tinham feito segundo o SENHOR havia ordenado; assim a fizeram, e Moisés os abençoou.

\chapter{40}

\par 1 Depois, disse o SENHOR a Moisés:
\par 2 No primeiro dia do primeiro mês, levantarás o tabernáculo da tenda da congregação.
\par 3 Porás, nele, a arca do Testemunho e a cobrirás com o véu.
\par 4 Meterás, nele, a mesa e porás por ordem as coisas que estão sobre ela; também meterás, nele, o candelabro e acenderás as suas lâmpadas.
\par 5 Porás o altar de ouro para o incenso diante da arca do Testemunho e pendurarás o reposteiro da porta do tabernáculo.
\par 6 Porás o altar do holocausto diante da porta do tabernáculo da tenda da congregação.
\par 7 Porás a bacia entre a tenda da congregação e o altar e a encherás de água.
\par 8 Depois, porás o átrio ao redor e pendurarás o reposteiro à porta do átrio.
\par 9 E tomarás o óleo da unção, e ungirás o tabernáculo e tudo o que nele está, e o consagrarás com todos os seus pertences; e será santo.
\par 10 Ungirás também o altar do holocausto e todos os seus utensílios e consagrarás o altar; e o altar se tornará santíssimo.
\par 11 Então, ungirás a bacia e o seu suporte e a consagrarás.
\par 12 Farás também chegar Arão e seus filhos à porta da tenda da congregação e os lavarás com água.
\par 13 Vestirás Arão das vestes sagradas, e o ungirás, e o consagrarás para que me oficie como sacerdote.
\par 14 Também farás chegar seus filhos, e lhes vestirás as túnicas,
\par 15 e os ungirás como ungiste seu pai, para que me oficiem como sacerdotes; sua unção lhes será por sacerdócio perpétuo durante as suas gerações.
\par 16 E tudo fez Moisés segundo o SENHOR lhe havia ordenado; assim o fez.
\par 17 No primeiro mês do segundo ano, no primeiro dia do mês, se levantou o tabernáculo.
\par 18 Moisés levantou o tabernáculo, e pôs as suas bases, e armou as suas tábuas, e meteu, nele, as suas vergas, e levantou as suas colunas;
\par 19 estendeu a tenda sobre o tabernáculo e pôs a coberta da tenda por cima, segundo o SENHOR ordenara a Moisés.
\par 20 Tomou o Testemunho, e o pôs na arca, e meteu os varais na arca, e pôs o propiciatório em cima da arca.
\par 21 Introduziu a arca no tabernáculo, e pendurou o véu do reposteiro, e com ele cobriu a arca do Testemunho, segundo o SENHOR ordenara a Moisés.
\par 22 Pôs também a mesa na tenda da congregação, ao lado do tabernáculo, para o norte, fora do véu,
\par 23 e sobre ela pôs em ordem os pães da proposição perante o SENHOR, segundo o SENHOR ordenara a Moisés.
\par 24 Pôs também, na tenda da congregação, o candelabro defronte da mesa, ao lado do tabernáculo, para o sul,
\par 25 e preparou as lâmpadas perante o SENHOR, segundo o SENHOR ordenara a Moisés.
\par 26 Pôs o altar de ouro na tenda da congregação, diante do véu,
\par 27 e acendeu sobre ele o incenso aromático, segundo o SENHOR ordenara a Moisés.
\par 28 Pendurou também o reposteiro da porta do tabernáculo,
\par 29 pôs o altar do holocausto à porta do tabernáculo da tenda da congregação e ofereceu sobre ele holocausto e oferta de cereais, segundo o SENHOR ordenara a Moisés.
\par 30 Pôs a bacia entre a tenda da congregação e o altar e a encheu de água, para se lavar.
\par 31 Nela, Moisés, Arão e seus filhos lavavam as mãos e os pés,
\par 32 quando entravam na tenda da congregação e quando se chegavam ao altar, segundo o SENHOR ordenara a Moisés.
\par 33 Levantou também o átrio ao redor do tabernáculo e do altar e pendurou o reposteiro da porta do átrio. Assim Moisés acabou a obra.
\par 34 Então, a nuvem cobriu a tenda da congregação, e a glória do SENHOR encheu o tabernáculo.
\par 35 Moisés não podia entrar na tenda da congregação, porque a nuvem permanecia sobre ela, e a glória do SENHOR enchia o tabernáculo.
\par 36 Quando a nuvem se levantava de sobre o tabernáculo, os filhos de Israel caminhavam avante, em todas as suas jornadas;
\par 37 se a nuvem, porém, não se levantava, não caminhavam, até ao dia em que ela se levantava.
\par 38 De dia, a nuvem do SENHOR repousava sobre o tabernáculo, e, de noite, havia fogo nela, à vista de toda a casa de Israel, em todas as suas jornadas.


\end{document}