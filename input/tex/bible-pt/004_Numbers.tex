\begin{document}

\title{Números}


\chapter{1}

\par 1 No segundo ano após a saída dos filhos de Israel do Egito, no primeiro dia do segundo mês, falou o SENHOR a Moisés, no deserto do Sinai, na tenda da congregação, dizendo:
\par 2 Levantai o censo de toda a congregação dos filhos de Israel, segundo as suas famílias, segundo a casa de seus pais, contando todos os homens, nominalmente, cabeça por cabeça.
\par 3 Da idade de vinte anos para cima, todos os capazes de sair à guerra em Israel, a esses contareis segundo os seus exércitos, tu e Arão.
\par 4 De cada tribo vos assistirá um homem que seja cabeça da casa de seus pais.
\par 5 Estes, pois, são os nomes dos homens que vos assistirão: de Rúben, Elizur, filho de Sedeur;
\par 6 de Simeão, Selumiel, filho de Zurisadai;
\par 7 de Judá, Naassom, filho de Aminadabe;
\par 8 de Issacar, Natanael, filho de Zuar;
\par 9 de Zebulom, Eliabe, filho de Helom;
\par 10 dos filhos de José: de Efraim, Elisama, filho de Amiúde; de Manassés, Gamaliel, filho de Pedazur;
\par 11 de Benjamim, Abidã, filho de Gideoni;
\par 12 de Dã, Aiezer, filho de Amisadai;
\par 13 de Aser, Pagiel, filho de Ocrã;
\par 14 de Gade, Eliasafe, filho de Deuel;
\par 15 de Naftali, Aira, filho de Enã.
\par 16 Estes foram os chamados da congregação, os príncipes das tribos de seus pais, os cabeças dos milhares de Israel.
\par 17 Então, Moisés e Arão tomaram estes homens, que foram designados pelos seus nomes.
\par 18 E, tendo ajuntado toda a congregação no primeiro dia do mês segundo, declararam a descendência deles, segundo as suas famílias, segundo a casa de seus pais, contados nominalmente, de vinte anos para cima, cabeça por cabeça.
\par 19 Como o SENHOR ordenara a Moisés, assim os contou no deserto do Sinai.
\par 20 Dos filhos de Rúben, o primogênito de Israel, as suas gerações, pelas suas famílias, segundo a casa de seus pais, contados nominalmente, cabeça por cabeça, todos os homens de vinte anos para cima, todos os capazes de sair à guerra,
\par 21 foram contados deles, da tribo de Rúben, quarenta e seis mil e quinhentos.
\par 22 Dos filhos de Simeão, as suas gerações, pelas suas famílias, segundo a casa de seus pais, contados nominalmente, cabeça por cabeça, todos os homens de vinte anos para cima, todos os capazes de sair à guerra,
\par 23 foram contados deles, da tribo de Simeão, cinqüenta e nove mil e trezentos.
\par 24 Dos filhos de Gade, as suas gerações, pelas suas famílias, segundo a casa de seus pais, contados nominalmente, de vinte anos para cima, todos os capazes de sair à guerra,
\par 25 foram contados deles, da tribo de Gade, quarenta e cinco mil seiscentos e cinqüenta.
\par 26 Dos filhos de Judá, as suas gerações, pelas suas famílias, segundo a casa de seus pais, contados nominalmente, de vinte anos para cima, todos os capazes de sair à guerra,
\par 27 foram contados deles, da tribo de Judá, setenta e quatro mil e seiscentos.
\par 28 Dos filhos de Issacar, as suas gerações, pelas suas famílias, segundo a casa de seus pais, contados nominalmente, de vinte anos para cima, todos os capazes de sair à guerra,
\par 29 foram contados deles, da tribo de Issacar, cinqüenta e quatro mil e quatrocentos.
\par 30 Dos filhos de Zebulom, as suas gerações, pelas suas famílias, segundo a casa de seus pais, contados nominalmente, de vinte anos para cima, todos os capazes de sair à guerra,
\par 31 foram contados deles, da tribo de Zebulom, cinqüenta e sete mil e quatrocentos.
\par 32 Dos filhos de José, dos filhos de Efraim, as suas gerações, pelas suas famílias, segundo a casa de seus pais, contados nominalmente, de vinte anos para cima, todos os capazes de sair à guerra,
\par 33 foram contados deles, da tribo de Efraim, quarenta mil e quinhentos.
\par 34 Dos filhos de Manassés, as suas gerações, pelas suas famílias, segundo a casa de seus pais, contados nominalmente, de vinte anos para cima, todos os capazes de sair à guerra,
\par 35 foram contados deles, da tribo de Manassés, trinta e dois mil e duzentos.
\par 36 Dos filhos de Benjamim, as suas gerações, pelas suas famílias, segundo a casa de seus pais, contados nominalmente, de vinte anos para cima, todos os capazes de sair à guerra,
\par 37 foram contados deles, da tribo de Benjamim, trinta e cinco mil e quatrocentos.
\par 38 Dos filhos de Dã, as suas gerações, pelas suas famílias, segundo a casa de seus pais, contados nominalmente, de vinte anos para cima, todos os capazes de sair à guerra,
\par 39 foram contados deles, da tribo de Dã, sessenta e dois mil e setecentos.
\par 40 Dos filhos de Aser, as suas gerações, pelas suas famílias, segundo a casa de seus pais, contados nominalmente, de vinte anos para cima, todos os capazes de sair à guerra,
\par 41 foram contados deles, da tribo de Aser, quarenta e um mil e quinhentos.
\par 42 Dos filhos de Naftali, as suas gerações, pelas suas famílias, segundo a casa de seus pais, contados nominalmente, de vinte anos para cima, todos os capazes de sair à guerra,
\par 43 foram contados deles, da tribo de Naftali, cinqüenta e três mil e quatrocentos.
\par 44 Foram estes os contados, contados por Moisés e Arão; e os príncipes de Israel eram doze homens; cada um era pela casa de seus pais.
\par 45 Assim, pois, todos os contados dos filhos de Israel, segundo a casa de seus pais, de vinte anos para cima, todos os capazes de sair à guerra,
\par 46 todos os contados foram seiscentos e três mil quinhentos e cinqüenta.
\par 47 Mas os levitas, segundo a tribo de seus pais, não foram contados entre eles,
\par 48 porquanto o SENHOR falara a Moisés, dizendo:
\par 49 Somente não contarás a tribo de Levi, nem levantarás o censo deles entre os filhos de Israel;
\par 50 mas incumbe tu os levitas de cuidarem do tabernáculo do Testemunho, e de todos os seus utensílios, e de tudo o que lhe pertence; eles levarão o tabernáculo e todos os seus utensílios; eles ministrarão no tabernáculo e acampar-se-ão ao redor dele.
\par 51 Quando o tabernáculo partir, os levitas o desarmarão; e, quando assentar no arraial, os levitas o armarão; o estranho que se aproximar morrerá.
\par 52 Os filhos de Israel se acamparão, cada um no seu arraial e cada um junto ao seu estandarte, segundo as suas turmas.
\par 53 Mas os levitas se acamparão ao redor do tabernáculo do Testemunho, para que não haja ira sobre a congregação dos filhos de Israel; pelo que os levitas tomarão a si o cuidar do tabernáculo do Testemunho.
\par 54 Assim fizeram os filhos de Israel; segundo tudo o que o SENHOR ordenara a Moisés, assim o fizeram.

\chapter{2}

\par 1 Disse o SENHOR a Moisés e a Arão:
\par 2 Os filhos de Israel se acamparão junto ao seu estandarte, segundo as insígnias da casa de seus pais; ao redor, de frente para a tenda da congregação, se acamparão.
\par 3 Os que se acamparem ao lado oriental (para o nascente) serão os do estandarte do arraial de Judá, segundo as suas turmas; e Naassom, filho de Aminadabe, será príncipe dos filhos de Judá.
\par 4 E o seu exército, segundo o censo, foram setenta e quatro mil e seiscentos.
\par 5 E junto a ele se acampará a tribo de Issacar; e Natanael, filho de Zuar, será príncipe dos filhos de Issacar.
\par 6 E o seu exército, segundo o censo, foram cinqüenta e quatro mil e quatrocentos.
\par 7 Depois, a tribo de Zebulom; e Eliabe, filho de Helom, será príncipe dos filhos de Zebulom.
\par 8 E o seu exército, segundo o censo, foram cinqüenta e sete mil e quatrocentos.
\par 9 Todos os que foram contados do arraial de Judá foram cento e oitenta e seis mil e quatrocentos, segundo as suas turmas; e estes marcharão primeiro.
\par 10 O estandarte do arraial de Rúben, segundo as suas turmas, estará para o lado sul; e Elizur, filho de Sedeur, será príncipe dos filhos de Rúben.
\par 11 E o seu exército, segundo o censo, foram quarenta e seis mil e quinhentos.
\par 12 E junto a ele se acampará a tribo de Simeão; e Selumiel, filho de Zurisadai, será príncipe dos filhos de Simeão.
\par 13 E o seu exército, segundo o censo, foram cinqüenta e nove mil e trezentos.
\par 14 Depois, a tribo de Gade; e Eliasafe, filho de Deuel, será príncipe dos filhos de Gade.
\par 15 E o seu exército, segundo o censo, foram quarenta e cinco mil seiscentos e cinqüenta.
\par 16 Todos os que foram contados no arraial de Rúben foram cento e cinqüenta e um mil quatrocentos e cinqüenta, segundo as suas turmas; e estes marcharão em segundo lugar.
\par 17 Então, partirá a tenda da congregação com o arraial dos levitas no meio dos arraiais; como se acamparem, assim marcharão, cada um no seu lugar, segundo os seus estandartes.
\par 18 O estandarte do arraial de Efraim, segundo as suas turmas, estará para o lado ocidental; e Elisama, filho de Amiúde, será príncipe dos filhos de Efraim.
\par 19 E o seu exército, segundo o censo, foram quarenta mil e quinhentos.
\par 20 E junto a ele, a tribo de Manassés; e Gamaliel, filho de Pedazur, será príncipe dos filhos de Manassés.
\par 21 E o seu exército, segundo o censo, foram trinta e dois mil e duzentos.
\par 22 Depois, a tribo de Benjamim; e Abidã, filho de Gideoni, será príncipe dos filhos de Benjamim.
\par 23 O seu exército, segundo o censo, foram trinta e cinco mil e quatrocentos.
\par 24 Todos os que foram contados no arraial de Efraim foram cento e oito mil e cem, segundo as suas turmas; e estes marcharão em terceiro lugar.
\par 25 O estandarte do arraial de Dã estará para o norte, segundo as suas turmas; e Aiezer, filho Amisadai, será príncipe dos filhos de Dã.
\par 26 E o seu exército, segundo o censo, foram sessenta e dois mil e setecentos.
\par 27 E junto a ele se acampará a tribo de Aser; e Pagiel, filho de Ocrã, será príncipe dos filhos de Aser.
\par 28 E o seu exército, segundo o censo, foram quarenta e um mil e quinhentos.
\par 29 Depois, a tribo de Naftali; e Aira, filho de Enã, será príncipe dos filhos de Naftali.
\par 30 E o seu exército, segundo o censo, foram cinqüenta e três mil e quatrocentos.
\par 31 Todos os que foram contados no arraial de Dã foram cento e cinqüenta e sete mil e seiscentos; e estes marcharão no último lugar, segundo os seus estandartes.
\par 32 São estes os que foram contados dos filhos de Israel, segundo a casa de seus pais; todos os que foram contados dos arraiais pelas suas turmas foram seiscentos e três mil quinhentos e cinqüenta.
\par 33 Mas os levitas não foram contados entre os filhos de Israel, como o SENHOR ordenara a Moisés.
\par 34 Assim fizeram os filhos de Israel; conforme tudo o que o SENHOR ordenara a Moisés, se acamparam segundo os seus estandartes e assim marcharam, cada qual segundo as suas famílias, segundo a casa de seus pais.

\chapter{3}

\par 1 São estas, pois, as gerações de Arão e de Moisés, no dia em que o SENHOR falou com Moisés no monte Sinai.
\par 2 E são estes os nomes dos filhos de Arão: o primogênito, Nadabe; depois, Abiú, Eleazar e Itamar.
\par 3 Estes são os nomes dos filhos de Arão, os sacerdotes ungidos, consagrados para oficiar como sacerdotes.
\par 4 Mas Nadabe e Abiú morreram perante o SENHOR, quando ofereciam fogo estranho perante o SENHOR, no deserto do Sinai, e não tiveram filhos; porém Eleazar e Itamar oficiaram como sacerdotes diante de Arão, seu pai.
\par 5 Disse o SENHOR a Moisés:
\par 6 Faze chegar a tribo de Levi e põe-na diante de Arão, o sacerdote, para que o sirvam
\par 7 e cumpram seus deveres para com ele e para com todo o povo, diante da tenda da congregação, para ministrarem no tabernáculo.
\par 8 Terão cuidado de todos os utensílios da tenda da congregação e cumprirão o seu dever para com os filhos de Israel, no ministrar no tabernáculo.
\par 9 Darás, pois, os levitas a Arão e a seus filhos; dentre os filhos de Israel lhe são dados.
\par 10 Mas a Arão e a seus filhos ordenarás que se dediquem só ao seu sacerdócio, e o estranho que se aproximar morrerá.
\par 11 Disse o SENHOR a Moisés:
\par 12 Eis que tenho eu tomado os levitas do meio dos filhos de Israel, em lugar de todo primogênito que abre a madre, entre os filhos de Israel; e os levitas serão meus.
\par 13 Porque todo primogênito é meu; desde o dia em que feri a todo primogênito na terra do Egito, consagrei para mim todo primogênito em Israel, desde o homem até ao animal; serão meus. Eu sou o SENHOR.
\par 14 Falou o SENHOR a Moisés no deserto do Sinai, dizendo:
\par 15 Conta os filhos de Levi, segundo a casa de seus pais, pelas suas famílias; contarás todo homem da idade de um mês para cima.
\par 16 E Moisés os contou segundo o mandado do SENHOR, como lhe fora ordenado.
\par 17 São estes os filhos de Levi pelos seus nomes: Gérson, Coate e Merari.
\par 18 E estes são os nomes dos filhos de Gérson pelas suas famílias: Libni e Simei.
\par 19 E os filhos de Coate pelas suas famílias: Anrão, Izar, Hebrom e Uziel.
\par 20 E os filhos de Merari pelas suas famílias: Mali e Musi; são estas as famílias dos levitas, segundo a casa de seus pais.
\par 21 De Gérson é a família dos libnitas e a dos simeítas; são estas as famílias dos gersonitas.
\par 22 Todos os homens que deles foram contados, cada um nominalmente, de um mês para cima, foram sete mil e quinhentos.
\par 23 As famílias dos gersonitas se acamparão atrás do tabernáculo, ao ocidente.
\par 24 O príncipe da casa paterna dos gersonitas será Eliasafe, filho de Lael.
\par 25 Os filhos de Gérson terão a seu cargo, na tenda da congregação, o tabernáculo, a tenda e sua coberta, o reposteiro para a porta da tenda da congregação,
\par 26 as cortinas do pátio, o reposteiro da porta do pátio, que rodeia o tabernáculo e o altar, as suas cordas e todo o serviço a eles devido.
\par 27 De Coate é a família dos anramitas, e a dos izaritas, e a dos hebronitas, e a dos uzielitas; são estas as famílias dos coatitas.
\par 28 Contados todos os homens, da idade de um mês para cima, foram oito mil e seiscentos, que tinham a seu cargo o santuário.
\par 29 As famílias dos filhos de Coate se acamparão ao lado do tabernáculo, do lado sul.
\par 30 O príncipe da casa paterna das famílias dos coatitas será Elisafã, filho de Uziel.
\par 31 Terão eles a seu cargo a arca, a mesa, o candelabro, os altares, os utensílios do santuário com que ministram, o reposteiro e todo o serviço a eles devido.
\par 32 O príncipe dos príncipes de Levi será Eleazar, filho de Arão, o sacerdote; terá a superintendência dos que têm a seu cargo o santuário.
\par 33 De Merari é a família dos malitas e a dos musitas; são estas as famílias de Merari.
\par 34 Todos os homens que deles foram contados, de um mês para cima, foram seis mil e duzentos.
\par 35 O príncipe da casa paterna das famílias de Merari será Zuriel, filho de Abiail; acampar-se-ão ao lado do tabernáculo, do lado norte.
\par 36 Os filhos de Merari, por designação, terão a seu cargo as tábuas do tabernáculo, as suas travessas, as suas colunas, as suas bases, todos os seus utensílios e todo o serviço a eles devido;
\par 37 também as colunas do pátio em redor, as suas bases, as suas estacas e as suas cordas.
\par 38 Os que se acamparão diante do tabernáculo, ao oriente, diante da tenda da congregação, para o lado do nascente, serão Moisés e Arão, com seus filhos, tendo a seu cargo os ritos do santuário, para cumprirem seus deveres prescritos, em prol dos filhos de Israel; o estranho que se aproximar morrerá.
\par 39 Todos os que foram contados dos levitas, contados por Moisés e Arão, por mandado do SENHOR, segundo as suas famílias, todo homem de um mês para cima, foram vinte e dois mil.
\par 40 Disse o SENHOR a Moisés: Conta todo primogênito varão dos filhos de Israel, cada um nominalmente, de um mês para cima,
\par 41 e para mim tomarás os levitas (eu sou o SENHOR) em lugar de todo primogênito dos filhos de Israel e os animais dos levitas em lugar de todo primogênito entre os animais dos filhos de Israel.
\par 42 Contou Moisés, como o SENHOR lhe ordenara, todo primogênito entre os filhos de Israel.
\par 43 Todos os primogênitos varões, contados nominalmente, de um mês para cima, segundo o censo, foram vinte e dois mil duzentos e setenta e três.
\par 44 Disse o SENHOR a Moisés:
\par 45 Toma os levitas em lugar de todo primogênito entre os filhos de Israel e os animais dos levitas em lugar dos animais dos filhos de Israel, porquanto os levitas serão meus. Eu sou o SENHOR.
\par 46 Pelo resgate dos duzentos e setenta e três dos primogênitos dos filhos de Israel, que excedem o número dos levitas,
\par 47 tomarás por cabeça cinco siclos; segundo o siclo do santuário, os tomarás, a vinte geras o siclo.
\par 48 E darás a Arão e a seus filhos o dinheiro com o qual são resgatados os que são demais entre eles.
\par 49 Então, Moisés tomou o dinheiro do resgate dos que excederam os que foram resgatados pelos levitas.
\par 50 Dos primogênitos dos filhos de Israel tomou o dinheiro, mil trezentos e sessenta e cinco siclos, segundo o siclo do santuário.
\par 51 E deu Moisés o dinheiro dos resgatados a Arão e a seus filhos, segundo o mandado do SENHOR, como o SENHOR ordenara a Moisés.

\chapter{4}

\par 1 Disse o SENHOR a Moisés e a Arão:
\par 2 Levanta o censo dos filhos de Coate, do meio dos filhos de Levi, pelas suas famílias, segundo a casa de seus pais;
\par 3 da idade de trinta anos para cima até aos cinqüenta será todo aquele que entrar neste serviço, para exercer algum encargo na tenda da congregação.
\par 4 É este o serviço dos filhos de Coate na tenda da congregação, nas coisas santíssimas.
\par 5 Quando partir o arraial, Arão e seus filhos virão, e tirarão o véu de cobrir, e, com ele, cobrirão a arca do Testemunho;
\par 6 e, por cima, lhe porão uma coberta de peles finas, e sobre ela estenderão um pano, todo azul, e lhe meterão os varais.
\par 7 Também sobre a mesa da proposição estenderão um pano azul; e, sobre ela, porão os pratos, os recipientes do incenso, as taças e as galhetas; também o pão contínuo estará sobre ela.
\par 8 Depois, estenderão, em cima deles, um pano de carmesim, e, com a coberta de peles finas, o cobrirão, e lhe porão os varais.
\par 9 Tomarão um pano azul e cobrirão o candelabro da luminária, as suas lâmpadas, os seus espevitadores, os seus apagadores e todos os seus vasos de azeite com que o servem.
\par 10 E envolverão a ele e todos os seus utensílios na coberta de peles finas e o porão sobre os varais.
\par 11 Sobre o altar de ouro estenderão um pano azul, e, com a coberta de peles finas, o cobrirão, e lhe porão os varais;
\par 12 tomarão todos os utensílios do serviço com os quais servem no santuário; e os envolverão num pano azul, e os cobrirão com uma coberta de peles finas, e os porão sobre os varais.
\par 13 Do altar tirarão as cinzas e, por cima dele, estenderão um pano de púrpura.
\par 14 Sobre ele porão todos os seus utensílios com que o servem: os braseiros, os garfos, as pás e as bacias, todos os utensílios do altar; e, por cima dele, estenderão uma coberta de peles finas e lhe porão os varais.
\par 15 Havendo, pois, Arão e seus filhos, ao partir o arraial, acabado de cobrir o santuário e todos os móveis dele, então, os filhos de Coate virão para levá-lo; mas, nas coisas santas, não tocarão, para que não morram; são estas as coisas da tenda da congregação que os filhos de Coate devem levar.
\par 16 Eleazar, filho de Arão, o sacerdote, terá a seu cargo o azeite da luminária, o incenso aromático, a contínua oferta dos manjares e o óleo da unção, sim, terá a seu cargo todo o tabernáculo e tudo o que nele há, o santuário e os móveis.
\par 17 Disse o SENHOR a Moisés e a Arão:
\par 18 Não deixareis que a tribo das famílias dos coatitas seja eliminada do meio dos levitas.
\par 19 Isto, porém, lhe fareis, para que vivam e não morram, quando se aproximarem das coisas santíssimas: Arão e seus filhos entrarão e lhes designarão a cada um o seu serviço e a sua carga.
\par 20 Porém os coatitas não entrarão, nem por um instante, para ver as coisas santas, para que não morram.
\par 21 Disse mais o SENHOR a Moisés:
\par 22 Levanta o censo dos filhos de Gérson, segundo a casa de seus pais, segundo as suas famílias.
\par 23 Da idade de trinta anos para cima até aos cinqüenta será todo aquele que entrar neste serviço, para algum encargo na tenda da congregação.
\par 24 É este o serviço das famílias dos gersonitas para servir e levar cargas:
\par 25 levarão as cortinas do tabernáculo, a tenda da congregação, sua coberta, a coberta de peles finas, que está sobre ele, o reposteiro da porta da tenda da congregação,
\par 26 as cortinas do pátio, o reposteiro da porta do pátio, que rodeia o tabernáculo e o altar, as suas cordas e todos os objetos do seu serviço e servirão em tudo quanto diz respeito a estas coisas.
\par 27 Todo o serviço dos filhos dos gersonitas, toda a sua carga e tudo o que devem fazer será segundo o mandado de Arão e de seus filhos; e lhes determinareis tudo o que devem carregar.
\par 28 Este é o serviço das famílias dos filhos dos gersonitas na tenda da congregação; o seu cargo estará sob a direção de Itamar, filho de Arão, o sacerdote.
\par 29 Quanto aos filhos de Merari, segundo as suas famílias e segundo a casa de seus pais os contarás.
\par 30 Da idade de trinta anos para cima até aos cinqüenta contarás todo aquele que entrar neste serviço, para exercer algum encargo na tenda da congregação.
\par 31 Isto será o que é de sua obrigação levar, segundo todo o seu serviço, na tenda da congregação: as tábuas do tabernáculo, os seus varais, as suas colunas e as suas bases;
\par 32 as colunas do pátio em redor, as suas bases, as suas estacas e as suas cordas, com todos os seus utensílios e com tudo o que pertence ao seu serviço; e designareis, nome por nome, os objetos que devem levar.
\par 33 É este o encargo das famílias dos filhos de Merari, segundo todo o seu serviço, na tenda da congregação, sob a direção de Itamar, filho de Arão, o sacerdote.
\par 34 Moisés, pois, e Arão, e os príncipes do povo contaram os filhos dos coatitas, segundo as suas famílias e segundo a casa de seus pais,
\par 35 da idade de trinta anos para cima até aos cinqüenta, todo aquele que entrou neste serviço, para exercer algum encargo na tenda da congregação.
\par 36 Os que deles foram contados, pois, segundo as suas famílias, foram dois mil setecentos e cinqüenta.
\par 37 São estes os que foram contados das famílias dos coatitas, todos os que serviam na tenda da congregação, os quais Moisés e Arão contaram, segundo o mandado do SENHOR, por Moisés.
\par 38 Os que foram contados dos filhos de Gérson, segundo as suas famílias e segundo a casa de seus pais,
\par 39 da idade de trinta anos para cima até aos cinqüenta, todo aquele que entrou neste serviço, para exercer algum encargo na tenda da congregação,
\par 40 os que deles foram contados, segundo as suas famílias, segundo a casa de seus pais, foram dois mil seiscentos e trinta.
\par 41 São estes os contados das famílias dos filhos de Gérson, todos os que serviam na tenda da congregação, os quais Moisés e Arão contaram, segundo o mandado do SENHOR.
\par 42 Os que foram contados das famílias dos filhos de Merari, segundo as suas famílias e segundo a casa de seus pais,
\par 43 da idade de trinta anos para cima até aos cinqüenta, todo aquele que entrou neste serviço, para exercer algum encargo na tenda da congregação,
\par 44 os que deles foram contados, segundo as suas famílias, foram três mil e duzentos.
\par 45 São estes os contados das famílias dos filhos de Merari, os quais Moisés e Arão contaram, segundo o mandado do SENHOR, por Moisés.
\par 46 Todos os que foram contados dos levitas, contados por Moisés, e Arão, e os príncipes de Israel, segundo as suas famílias e segundo a casa de seus pais,
\par 47 da idade de trinta anos para cima até aos cinqüenta, todos os que entraram para cumprir a tarefa do serviço e a de levarem cargas na tenda da congregação,
\par 48 os que deles foram contados foram oito mil quinhentos e oitenta.
\par 49 Segundo o mandado do SENHOR, por Moisés, foram designados, cada um para o seu serviço e a sua carga; e deles foram contados, como o SENHOR ordenara a Moisés.

\chapter{5}

\par 1 Disse o SENHOR a Moisés:
\par 2 Ordena aos filhos de Israel que lancem para fora do arraial todo leproso, todo o que padece fluxo e todo imundo por ter tocado em algum morto;
\par 3 tanto homem como mulher os lançareis; para fora do arraial os lançareis, para que não contaminem o arraial, no meio do qual eu habito.
\par 4 Os filhos de Israel fizeram assim e os lançaram para fora do arraial; como o SENHOR falara a Moisés, assim fizeram os filhos de Israel.
\par 5 Disse mais o SENHOR a Moisés:
\par 6 Dize aos filhos de Israel: Quando homem ou mulher cometer algum dos pecados em que caem os homens, ofendendo ao SENHOR, tal pessoa é culpada.
\par 7 Confessará o pecado que cometer; e, pela culpa, fará plena restituição, e lhe acrescentará a sua quinta parte, e dará tudo àquele contra quem se fez culpado.
\par 8 Mas, se esse homem não tiver parente chegado, a quem possa fazer restituição pela culpa, então, o que se restitui ao SENHOR pela culpa será do sacerdote, além do carneiro expiatório com que se fizer expiação pelo culpado.
\par 9 Toda oferta de todas as coisas santas dos filhos de Israel, que trouxerem ao sacerdote, será deste
\par 10 e também as coisas sagradas de cada um; o que alguém der ao sacerdote será deste.
\par 11 Disse mais o SENHOR a Moisés:
\par 12 Fala aos filhos de Israel e dize-lhes: Se a mulher de alguém se desviar e lhe for infiel,
\par 13 de maneira que algum homem se tenha deitado com ela, e for oculto aos olhos de seu marido, e ela o tiver ocultado, havendo-se ela contaminado, e contra ela não houver testemunha, e não for surpreendida em flagrante,
\par 14 e o espírito de ciúmes vier sobre ele, e de sua mulher tiver ciúmes, por ela se haver contaminado, ou o tiver, não se havendo ela contaminado,
\par 15 então, esse homem trará a sua mulher perante o sacerdote e juntamente trará a sua oferta por ela: uma décima de efa de farinha de cevada, sobre a qual não deitará azeite, nem sobre ela porá incenso, porquanto é oferta de manjares de ciúmes, oferta memorativa, que traz a iniqüidade à memória.
\par 16 O sacerdote a fará chegar e a colocará perante o SENHOR.
\par 17 O sacerdote tomará água santa num vaso de barro; também tomará do pó que houver no chão do tabernáculo e o deitará na água.
\par 18 Apresentará a mulher perante o SENHOR e soltará a cabeleira dela; e lhe porá nas mãos a oferta memorativa de manjares, que é a oferta de manjares dos ciúmes. A água amarga, que traz consigo a maldição, estará na mão do sacerdote.
\par 19 O sacerdote a conjurará e lhe dirá: Se ninguém contigo se deitou, e se não te desviaste para a imundícia, estando sob o domínio de teu marido, destas águas amargas, amaldiçoantes, serás livre.
\par 20 Mas, se te desviaste, quando sob o domínio de teu marido, e te contaminaste, e algum homem, que não é o teu marido, se deitou contigo
\par 21 (então, o sacerdote fará que a mulher tome o juramento de maldição e lhe dirá), o SENHOR te ponha por maldição e por praga no meio do teu povo, fazendo-te o SENHOR descair a coxa e inchar o ventre;
\par 22 e esta água amaldiçoante penetre nas tuas entranhas, para te fazer inchar o ventre e te fazer descair a coxa. Então, a mulher dirá: Amém! Amém!
\par 23 O sacerdote escreverá estas maldições num livro e, com a água amarga, as apagará.
\par 24 E fará que a mulher beba a água amarga, que traz consigo a maldição; e, sendo bebida, lhe causará amargura.
\par 25 Da mão da mulher tomará o sacerdote a oferta de manjares de ciúmes e a moverá perante o SENHOR; e a trará ao altar.
\par 26 Tomará um punhado da oferta de manjares, da oferta memorativa, e sobre o altar o queimará; e, depois, dará a beber a água à mulher.
\par 27 E, havendo-lhe dado a beber a água, será que, se ela se tiver contaminado, e a seu marido tenha sido infiel, a água amaldiçoante entrará nela para amargura, e o seu ventre se inchará, e a sua coxa descairá; a mulher será por maldição no meio do seu povo.
\par 28 Se a mulher se não tiver contaminado, mas estiver limpa, então, será livre e conceberá.
\par 29 Esta é a lei para o caso de ciúmes, quando a mulher, sob o domínio de seu marido, se desviar e for contaminada;
\par 30 ou quando sobre o homem vier o espírito de ciúmes, e tiver ciúmes de sua mulher, apresente a mulher perante o SENHOR, e o sacerdote nela execute toda esta lei.
\par 31 O homem será livre da iniqüidade, porém a mulher levará a sua iniqüidade.

\chapter{6}

\par 1 Disse o SENHOR a Moisés:
\par 2 Fala aos filhos de Israel e dize-lhes: Quando alguém, seja homem seja mulher, fizer voto especial, o voto de nazireu, a fim de consagrar-se para o SENHOR,
\par 3 abster-se-á de vinho e de bebida forte; não beberá vinagre de vinho, nem vinagre de bebida forte, nem tomará beberagens de uvas, nem comerá uvas frescas nem secas.
\par 4 Todos os dias do seu nazireado não comerá de coisa alguma que se faz da vinha, desde as sementes até às cascas.
\par 5 Todos os dias do seu voto de nazireado não passará navalha pela cabeça; até que se cumpram os dias para os quais se consagrou ao SENHOR, santo será, deixando crescer livremente a cabeleira.
\par 6 Todos os dias da sua consagração para o SENHOR, não se aproximará de um cadáver.
\par 7 Por seu pai, ou por sua mãe, ou por seu irmão, ou por sua irmã, por eles se não contaminará, quando morrerem; porquanto o nazireado do seu Deus está sobre a sua cabeça.
\par 8 Por todos os dias do seu nazireado, santo será ao SENHOR.
\par 9 Se alguém vier a morrer junto a ele subitamente, e contaminar a cabeça do seu nazireado, rapará a cabeça no dia da sua purificação; ao sétimo dia, a rapará.
\par 10 Ao oitavo dia, trará duas rolas ou dois pombinhos ao sacerdote, à porta da tenda da congregação;
\par 11 o sacerdote oferecerá um como oferta pelo pecado e o outro, para holocausto; e fará expiação por ele, visto que pecou relativamente ao morto; assim, naquele mesmo dia, consagrará a sua cabeça.
\par 12 Então, consagrará os dias do seu nazireado ao SENHOR e, para oferta pela culpa, trará um cordeiro de um ano; os dias antecedentes serão perdidos, porquanto o seu nazireado foi contaminado.
\par 13 Esta é a lei do nazireu: no dia em que se cumprirem os dias do seu nazireado, será trazido à porta da tenda da congregação.
\par 14 Ele apresentará a sua oferta ao SENHOR, um cordeiro de um ano, sem defeito, em holocausto, e uma cordeira de um ano, sem defeito, para oferta pelo pecado, e um carneiro, sem defeito, por oferta pacífica,
\par 15 e um cesto de pães asmos, bolos de flor de farinha com azeite, amassados, e obreias asmas untadas com azeite, como também a sua oferta de manjares e as suas libações.
\par 16 O sacerdote os trará perante o SENHOR e apresentará a sua oferta pelo pecado e o seu holocausto;
\par 17 oferecerá o carneiro em sacrifício pacífico ao SENHOR, com o cesto dos pães asmos; o sacerdote apresentará também a devida oferta de manjares e a libação.
\par 18 O nazireu, à porta da tenda da congregação, rapará a cabeleira do seu nazireado, e tomá-la-á, e a porá sobre o fogo que está debaixo do sacrifício pacífico.
\par 19 Depois, o sacerdote tomará a espádua cozida do carneiro, e um bolo asmo do cesto, e uma obreia asma e os porá nas mãos do nazireu, depois de haver este rapado a cabeleira do seu nazireado.
\par 20 O sacerdote os moverá em oferta movida perante o SENHOR; isto é santo para o sacerdote, juntamente com o peito da oferta movida e com a coxa da oferta; depois disto, o nazireu pode beber vinho.
\par 21 Esta é a lei do nazireu que fizer voto; a sua oferta ao SENHOR será segundo o seu nazireado, afora o que as suas posses lhe permitirem; segundo o voto que fizer, assim fará conforme a lei do seu nazireado.
\par 22 Disse o SENHOR a Moisés:
\par 23 Fala a Arão e a seus filhos, dizendo: Assim abençoareis os filhos de Israel e dir-lhes-eis:
\par 24 O SENHOR te abençoe e te guarde;
\par 25 o SENHOR faça resplandecer o rosto sobre ti e tenha misericórdia de ti;
\par 26 o SENHOR sobre ti levante o rosto e te dê a paz.
\par 27 Assim, porão o meu nome sobre os filhos de Israel, e eu os abençoarei.

\chapter{7}

\par 1 No dia em que Moisés acabou de levantar o tabernáculo, e o ungiu, e o consagrou e todos os seus utensílios, bem como o altar e todos os seus pertences,
\par 2 os príncipes de Israel, os cabeças da casa de seus pais, os que foram príncipes das tribos, que haviam presidido o censo, ofereceram
\par 3 e trouxeram a sua oferta perante o SENHOR: seis carros cobertos e doze bois; cada dois príncipes ofereceram um carro, e cada um deles, um boi; e os apresentaram diante do tabernáculo.
\par 4 Disse o SENHOR a Moisés:
\par 5 Recebe-os deles, e serão destinados ao serviço da tenda da congregação; e os darás aos levitas, a cada um segundo o seu serviço.
\par 6 Moisés recebeu os carros e os bois e os deu aos levitas.
\par 7 Dois carros e quatro bois deu aos filhos de Gérson, segundo o seu serviço;
\par 8 quatro carros e oito bois deu aos filhos de Merari, segundo o seu serviço, sob a direção de Itamar, filho de Arão, o sacerdote.
\par 9 Mas aos filhos de Coate nada deu, porquanto a seu cargo estava o santuário, que deviam levar aos ombros.
\par 10 Ofereceram os príncipes para a consagração do altar, no dia em que foi ungido; sim, apresentaram a sua oferta perante o altar.
\par 11 Disse o SENHOR a Moisés: Cada príncipe apresentará, no seu dia, a sua oferta para a consagração do altar.
\par 12 O que, pois, no primeiro dia, apresentou a sua oferta foi Naassom, filho de Aminadabe, pela tribo de Judá.
\par 13 A sua oferta foi um prato de prata, do peso de cento e trinta siclos, uma bacia de prata, de setenta siclos, segundo o siclo do santuário; ambos cheios de flor de farinha, amassada com azeite, para oferta de manjares;
\par 14 um recipiente de dez siclos de ouro, cheio de incenso;
\par 15 um novilho, um carneiro, um cordeiro de um ano, para holocausto;
\par 16 um bode, para oferta pelo pecado;
\par 17 e, para sacrifício pacífico, dois bois, cinco carneiros, cinco bodes, cinco cordeiros de um ano; foi esta a oferta de Naassom, filho de Aminadabe.
\par 18 No segundo dia, fez a sua oferta Natanael, filho de Zuar, príncipe de Issacar.
\par 19 Como sua oferta apresentou um prato de prata, do peso de cento e trinta siclos, uma bacia de prata, de setenta siclos, segundo o siclo do santuário; ambos cheios de flor de farinha, amassada com azeite, para oferta de manjares;
\par 20 um recipiente de dez siclos de ouro, cheio de incenso;
\par 21 um novilho, um carneiro, um cordeiro de um ano, para holocausto;
\par 22 um bode, para oferta pelo pecado;
\par 23 e, para sacrifício pacífico, dois bois, cinco carneiros, cinco bodes, cinco cordeiros de um ano; foi esta a oferta de Natanael, filho de Zuar.
\par 24 No terceiro dia, chegou o príncipe dos filhos de Zebulom, Eliabe, filho de Helom.
\par 25 A sua oferta foi um prato de prata, do peso de cento e trinta siclos, uma bacia de prata, de setenta siclos, segundo o siclo do santuário; ambos cheios de flor de farinha, amassada com azeite, para oferta de manjares;
\par 26 um recipiente de dez siclos de ouro, cheio de incenso;
\par 27 um novilho, um carneiro, um cordeiro de um ano, para holocausto;
\par 28 um bode, para oferta pelo pecado;
\par 29 e, para sacrifício pacífico, dois bois, cinco carneiros, cinco bodes, cinco cordeiros de um ano; foi esta a oferta de Eliabe, filho de Helom.
\par 30 No quarto dia, chegou o príncipe dos filhos de Rúben, Elizur, filho de Sedeur.
\par 31 A sua oferta foi um prato de prata, do peso de cento e trinta siclos, uma bacia de prata, de setenta siclos, segundo o siclo do santuário; ambos cheios de flor de farinha, amassada com azeite, para oferta de manjares;
\par 32 um recipiente de dez siclos de ouro, cheio de incenso;
\par 33 um novilho, um carneiro, um cordeiro de um ano, para holocausto;
\par 34 um bode, para oferta pelo pecado;
\par 35 e, para sacrifício pacífico, dois bois, cinco carneiros, cinco bodes, cinco cordeiros de um ano; foi esta a oferta de Elizur, filho de Sedeur.
\par 36 No quinto dia, chegou o príncipe dos filhos de Simeão, Selumiel, filho de Zurisadai.
\par 37 A sua oferta foi um prato de prata, do peso de cento e trinta siclos, uma bacia de prata, de setenta siclos, segundo o siclo do santuário; ambos cheios de flor de farinha, amassada com azeite, para oferta de manjares;
\par 38 um recipiente de dez siclos de ouro, cheio de incenso;
\par 39 um novilho, um carneiro, um cordeiro de um ano, para holocausto;
\par 40 um bode, para oferta pelo pecado;
\par 41 e, para sacrifício pacífico, dois bois, cinco carneiros, cinco bodes, cinco cordeiros de um ano; foi esta a oferta de Selumiel, filho de Zurisadai.
\par 42 No sexto dia, chegou o príncipe dos filhos de Gade, Eliasafe, filho de Deuel.
\par 43 A sua oferta foi um prato de prata, do peso de cento e trinta siclos, uma bacia de prata, de setenta siclos, segundo o siclo do santuário; ambos cheios de flor de farinha, amassada com azeite, para oferta de manjares;
\par 44 um recipiente de dez siclos de ouro, cheio de incenso;
\par 45 um novilho, um carneiro, um cordeiro de um ano, para holocausto;
\par 46 um bode, para oferta pelo pecado;
\par 47 e, para sacrifício pacífico, dois bois, cinco carneiros, cinco bodes, cinco cordeiros de um ano; foi esta a oferta de Eliasafe, filho de Deuel.
\par 48 No sétimo dia, chegou o príncipe dos filhos de Efraim, Elisama, filho de Amiúde.
\par 49 A sua oferta foi um prato de prata, do peso de cento e trinta siclos, uma bacia de prata, de setenta siclos, segundo o siclo do santuário; ambos cheios de flor de farinha, amassada com azeite, para oferta de manjares;
\par 50 um recipiente de dez siclos de ouro, cheio de incenso;
\par 51 um novilho, um carneiro, um cordeiro de um ano, para holocausto;
\par 52 um bode, para oferta pelo pecado;
\par 53 e, para sacrifício pacífico, dois bois, cinco carneiros, cinco bodes, cinco cordeiros de um ano; foi esta a oferta de Elisama, filho de Amiúde.
\par 54 No oitavo dia, chegou o príncipe dos filhos de Manassés, Gamaliel, filho de Pedazur.
\par 55 A sua oferta foi um prato de prata, do peso de cento e trinta siclos, uma bacia de prata, de setenta siclos, segundo o siclo do santuário; ambos cheios de flor de farinha, amassada com azeite, para oferta de manjares;
\par 56 um recipiente de dez siclos de ouro, cheio de incenso;
\par 57 um novilho, um carneiro, um cordeiro de um ano, para holocausto;
\par 58 um bode, para oferta pelo pecado;
\par 59 e, para sacrifício pacífico, dois bois, cinco carneiros, cinco bodes, cinco cordeiros de um ano; foi esta a oferta de Gamaliel, filho de Pedazur.
\par 60 No dia nono, chegou o príncipe dos filhos de Benjamim, Abidã, filho de Gideoni.
\par 61 A sua oferta foi um prato de prata, do peso de cento e trinta siclos, uma bacia de prata, de setenta siclos, segundo o siclo do santuário; ambos cheios de flor de farinha, amassada com azeite, para oferta de manjares;
\par 62 um recipiente de dez siclos de ouro, cheio de incenso;
\par 63 um novilho, um carneiro, um cordeiro de um ano, para holocausto;
\par 64 um bode, para oferta pelo pecado;
\par 65 e, para sacrifício pacífico, dois bois, cinco carneiros, cinco bodes, cinco cordeiros de um ano; foi esta a oferta de Abidã, filho de Gideoni.
\par 66 No décimo dia, chegou o príncipe dos filhos de Dã, Aiezer, filho de Amisadai.
\par 67 A sua oferta foi um prato de prata, do peso de cento e trinta siclos, uma bacia de prata, de setenta siclos, segundo o siclo do santuário; ambos cheios de flor de farinha, amassada com azeite, para oferta de manjares;
\par 68 um recipiente de dez siclos de ouro, cheio de incenso;
\par 69 um novilho, um carneiro, um cordeiro de um ano, para holocausto;
\par 70 um bode, para oferta pelo pecado;
\par 71 e, para sacrifício pacífico, dois bois, cinco carneiros, cinco bodes, cinco cordeiros de um ano; foi esta a oferta de Aiezer, filho de Amisadai.
\par 72 No dia undécimo, chegou o príncipe dos filhos de Aser, Pagiel, filho de Ocrã.
\par 73 A sua oferta foi um prato de prata, do peso de cento e trinta siclos, uma bacia de prata, de setenta siclos, segundo o siclo do santuário; ambos cheios de flor de farinha, amassada com azeite, para oferta de manjares;
\par 74 um recipiente de dez siclos de ouro, cheio de incenso;
\par 75 um novilho, um carneiro, um cordeiro de um ano, para holocausto;
\par 76 um bode, para oferta pelo pecado;
\par 77 e, para sacrifício pacífico, dois bois, cinco carneiros, cinco bodes, cinco cordeiros de um ano; foi esta a oferta de Pagiel, filho de Ocrã.
\par 78 No duodécimo dia, chegou o príncipe dos filhos de Naftali, Aira, filho de Enã.
\par 79 A sua oferta foi um prato de prata, do peso de cento e trinta siclos, uma bacia de prata, de setenta siclos, segundo o siclo do santuário; ambos cheios de flor de farinha, amassada com azeite, para oferta de manjares;
\par 80 um recipiente de dez siclos de ouro, cheio de incenso;
\par 81 um novilho, um carneiro, um cordeiro de um ano, para holocausto;
\par 82 um bode, para oferta pelo pecado;
\par 83 e, para sacrifício pacífico, dois bois, cinco carneiros, cinco bodes, cinco cordeiros de um ano; foi esta a oferta de Aira, filho de Enã.
\par 84 Esta é a dádiva feita pelos príncipes de Israel para a consagração do altar, no dia em que foi ungido: doze pratos de prata, doze bacias de prata, doze recipientes de ouro;
\par 85 cada prato de prata, de cento e trinta siclos, e cada bacia, de setenta; toda a prata dos utensílios foi de dois mil e quatrocentos siclos, segundo o siclo do santuário;
\par 86 doze recipientes de ouro cheios de incenso, cada um de dez siclos, segundo o siclo do santuário; todo o ouro dos recipientes foi de cento e vinte siclos;
\par 87 todos os animais para o holocausto foram doze novilhos; carneiros, doze; doze cordeiros de um ano, com a sua oferta de manjares; e doze bodes para oferta pelo pecado.
\par 88 E todos os animais para o sacrifício pacífico foram vinte e quatro novilhos; os carneiros, sessenta; os bodes, sessenta; os cordeiros de um ano, sessenta; esta é a dádiva para a consagração do altar, depois que foi ungido.
\par 89 Quando entrava Moisés na tenda da congregação para falar com o SENHOR, então, ouvia a voz que lhe falava de cima do propiciatório, que está sobre a arca do Testemunho entre os dois querubins; assim lhe falava.

\chapter{8}

\par 1 Disse o SENHOR a Moisés:
\par 2 Fala a Arão e dize-lhe: Quando colocares as lâmpadas, seja de tal maneira que venham as sete a alumiar defronte do candelabro.
\par 3 E Arão fez assim; colocou as lâmpadas para que alumiassem defronte do candelabro, como o SENHOR ordenara a Moisés.
\par 4 O candelabro era feito de ouro batido desde o seu pedestal até às suas flores; segundo o modelo que o SENHOR mostrara a Moisés, assim ele fez o candelabro.
\par 5 Disse mais o SENHOR a Moisés:
\par 6 Toma os levitas do meio dos filhos de Israel e purifica-os;
\par 7 assim lhes farás, para os purificar: asperge sobre eles a água da expiação; e sobre todo o seu corpo farão passar a navalha, lavarão as suas vestes e se purificarão;
\par 8 e tomarão um novilho, com a sua oferta de manjares de flor de farinha, amassada com azeite; tu, porém, tomarás outro novilho para oferta pelo pecado.
\par 9 Farás chegar os levitas perante a tenda da congregação; e ajuntarás toda a congregação dos filhos de Israel.
\par 10 Quando, pois, fizerem chegar os levitas perante o SENHOR, os filhos de Israel porão as mãos sobre eles.
\par 11 Arão apresentará os levitas como oferta movida perante o SENHOR, da parte dos filhos de Israel; e serão para o serviço do SENHOR.
\par 12 Os levitas porão as mãos sobre a cabeça dos novilhos; e tu sacrificarás um para oferta pelo o pecado e o outro para holocausto ao SENHOR, para fazer expiação pelos levitas.
\par 13 Porás os levitas perante Arão e perante os seus filhos e os apresentarás por oferta movida ao SENHOR.
\par 14 E separarás os levitas do meio dos filhos de Israel; os levitas serão meus.
\par 15 Depois disso, entrarão os levitas para fazerem o serviço da tenda da congregação; e tu os purificarás e, por oferta movida, os apresentarás,
\par 16 porquanto eles dentre os filhos de Israel me são dados; em lugar de todo aquele que abre a madre, do primogênito de cada um dos filhos de Israel, para mim os tomei.
\par 17 Porque meu é todo primogênito entre os filhos de Israel, tanto de homens como de animais; no dia em que, na terra do Egito, feri todo primogênito, os consagrei para mim.
\par 18 Tomei os levitas em lugar de todo primogênito entre os filhos de Israel.
\par 19 E os levitas, dados a Arão e a seus filhos, dentre os filhos de Israel, entreguei-os para fazerem o serviço dos filhos de Israel na tenda da congregação e para fazerem expiação por eles, para que não haja praga entre o povo de Israel, chegando-se os filhos de Israel ao santuário.
\par 20 E assim fez Moisés, e Arão, e toda a congregação dos filhos de Israel com os levitas; segundo tudo o que o SENHOR ordenara a Moisés acerca dos levitas, assim lhes fizeram os filhos de Israel.
\par 21 Os levitas se purificaram e lavaram as suas vestes, e Arão os apresentou por oferta movida perante o SENHOR e fez expiação por eles, para purificá-los.
\par 22 Depois disso, chegaram os levitas, para fazerem o seu serviço na tenda da congregação, perante Arão e seus filhos; como o SENHOR ordenara a Moisés acerca dos levitas, assim lhes fizeram.
\par 23 Disse mais o SENHOR a Moisés:
\par 24 Isto é o que toca aos levitas: da idade de vinte e cinco anos para cima entrarão, para fazerem o seu serviço na tenda da congregação;
\par 25 mas desde a idade de cinqüenta anos desobrigar-se-ão do serviço e nunca mais servirão;
\par 26 porém ajudarão aos seus irmãos na tenda da congregação, no tocante ao cargo deles; não terão mais serviço. Assim farás com os levitas quanto aos seus deveres.

\chapter{9}

\par 1 Falou o SENHOR a Moisés no deserto do Sinai, no ano segundo da sua saída da terra do Egito, no mês primeiro, dizendo:
\par 2 Celebrem os filhos de Israel a Páscoa a seu tempo.
\par 3 No dia catorze deste mês, ao crepúsculo da tarde, a seu tempo a celebrareis; segundo todos os seus estatutos e segundo todos os seus ritos, a celebrareis.
\par 4 Disse, pois, Moisés aos filhos de Israel que celebrassem a Páscoa.
\par 5 Então, celebraram a Páscoa no dia catorze do mês primeiro, ao crepúsculo da tarde, no deserto do Sinai; segundo tudo o que o SENHOR ordenara a Moisés, assim fizeram os filhos de Israel.
\par 6 Houve alguns que se acharam imundos por terem tocado o cadáver de um homem, de maneira que não puderam celebrar a Páscoa naquele dia; por isso, chegando-se perante Moisés e Arão,
\par 7 disseram-lhes: Estamos imundos por termos tocado o cadáver de um homem; por que havemos de ser privados de apresentar a oferta do SENHOR, a seu tempo, no meio dos filhos de Israel?
\par 8 Respondeu-lhes Moisés: Esperai, e ouvirei o que o SENHOR vos ordenará.
\par 9 Então, disse o SENHOR a Moisés:
\par 10 Fala aos filhos de Israel, dizendo: Quando alguém entre vós ou entre as vossas gerações achar-se imundo por causa de um morto ou se achar em jornada longe de vós, contudo, ainda celebrará a Páscoa ao SENHOR.
\par 11 No mês segundo, no dia catorze, no crepúsculo da tarde, a celebrarão; com pães asmos e ervas amargas a comerão.
\par 12 Dela nada deixarão até à manhã e dela não quebrarão osso algum; segundo todo o estatuto da Páscoa, a celebrarão.
\par 13 Porém, se um homem achar-se limpo, e não estiver de caminho, e deixar de celebrar a Páscoa, essa alma será eliminada do seu povo, porquanto não apresentou a oferta do SENHOR, a seu tempo; tal homem levará sobre si o seu pecado.
\par 14 Se um estrangeiro habitar entre vós e também celebrar a Páscoa ao SENHOR, segundo o estatuto da Páscoa e segundo o seu rito, assim a celebrará; um só estatuto haverá para vós outros, tanto para o estrangeiro como para o natural da terra.
\par 15 No dia em que foi erigido o tabernáculo, a nuvem o cobriu, a saber, a tenda do Testemunho; e, à tarde, estava sobre o tabernáculo uma aparência de fogo até à manhã.
\par 16 Assim era de contínuo: a nuvem o cobria, e, de noite, havia aparência de fogo.
\par 17 Quando a nuvem se erguia de sobre a tenda, os filhos de Israel se punham em marcha; e, no lugar onde a nuvem parava, aí os filhos de Israel se acampavam.
\par 18 Segundo o mandado do SENHOR, os filhos de Israel partiam e, segundo o mandado do SENHOR, se acampavam; por todo o tempo em que a nuvem pairava sobre o tabernáculo, permaneciam acampados.
\par 19 Quando a nuvem se detinha muitos dias sobre o tabernáculo, então, os filhos de Israel cumpriam a ordem do SENHOR e não partiam.
\par 20 Às vezes, a nuvem ficava poucos dias sobre o tabernáculo; então, segundo o mandado do SENHOR, permaneciam e, segundo a ordem do SENHOR, partiam.
\par 21 Às vezes, a nuvem ficava desde a tarde até à manhã; quando, pela manhã, a nuvem se erguia, punham-se em marcha; quer de dia, quer de noite, erguendo-se a nuvem, partiam.
\par 22 Se a nuvem se detinha sobre o tabernáculo por dois dias, ou um mês, ou por mais tempo, enquanto pairava sobre ele, os filhos de Israel permaneciam acampados e não se punham em marcha; mas, erguendo-se ela, partiam.
\par 23 Segundo o mandado do SENHOR, se acampavam e, segundo o mandado do SENHOR, se punham em marcha; cumpriam o seu dever para com o SENHOR, segundo a ordem do SENHOR por intermédio de Moisés.

\chapter{10}

\par 1 Disse mais o SENHOR a Moisés:
\par 2 Faze duas trombetas de prata; de obra batida as farás; servir-te-ão para convocares a congregação e para a partida dos arraiais.
\par 3 Quando tocarem, toda a congregação se ajuntará a ti à porta da tenda da congregação.
\par 4 Mas, quando tocar uma só, a ti se ajuntarão os príncipes, os cabeças dos milhares de Israel.
\par 5 Quando as tocardes a rebate, partirão os arraiais que se acham acampados do lado oriental.
\par 6 Mas, quando a segunda vez as tocardes a rebate, então, partirão os arraiais que se acham acampados do lado sul; a rebate, as tocarão para as suas partidas.
\par 7 Mas, se se houver de ajuntar a congregação, tocá-las-eis, porém não a rebate.
\par 8 Os filhos de Arão, sacerdotes, tocarão as trombetas; e a vós outros será isto por estatuto perpétuo nas vossas gerações.
\par 9 Quando, na vossa terra, sairdes a pelejar contra os opressores que vos apertam, também tocareis as trombetas a rebate, e perante o SENHOR, vosso Deus, haverá lembrança de vós, e sereis salvos de vossos inimigos.
\par 10 Da mesma sorte, no dia da vossa alegria, e nas vossas solenidades, e nos princípios dos vossos meses, também tocareis as vossas trombetas sobre os vossos holocaustos e sobre os vossos sacrifícios pacíficos, e vos serão por lembrança perante vosso Deus. Eu sou o SENHOR, vosso Deus.
\par 11 Aconteceu, no ano segundo, no segundo mês, aos vinte do mês, que a nuvem se ergueu de sobre o tabernáculo da congregação.
\par 12 Os filhos de Israel puseram-se em marcha do deserto do Sinai, jornada após jornada; e a nuvem repousou no deserto de Parã.
\par 13 Assim, pela primeira vez, se puseram em marcha, segundo o mandado do SENHOR, por Moisés.
\par 14 Primeiramente, partiu o estandarte do arraial dos filhos de Judá, segundo as suas turmas; e, sobre o seu exército, estava Naassom, filho de Aminadabe;
\par 15 sobre o exército da tribo dos filhos de Issacar, Natanael, filho de Zuar;
\par 16 e, sobre o exército da tribo dos filhos de Zebulom, Eliabe, filho de Helom.
\par 17 Então, desarmaram o tabernáculo, e os filhos de Gérson e os filhos de Merari partiram, levando o tabernáculo.
\par 18 Depois, partiu o estandarte do arraial de Rúben, segundo as suas turmas; e, sobre o seu exército, estava Elizur, filho de Sedeur;
\par 19 sobre o exército da tribo dos filhos de Simeão, Selumiel, filho de Zurisadai;
\par 20 e, sobre o exército da tribo dos filhos de Gade, Eliasafe, filho de Deuel.
\par 21 Então, partiram os coatitas, levando as coisas santas; e erigia-se o tabernáculo até que estes chegassem.
\par 22 Depois, partiu o estandarte do arraial dos filhos de Efraim, segundo as suas turmas; e, sobre o seu exército, estava Elisama, filho de Amiúde;
\par 23 sobre o exército da tribo dos filhos de Manassés, Gamaliel, filho de Pedazur;
\par 24 e, sobre o exército da tribo dos filhos de Benjamim, Abidã, filho de Gideoni.
\par 25 Então, partiu o estandarte do arraial dos filhos de Dã, formando a retaguarda de todos os arraiais, segundo as suas turmas; e, sobre o seu exército, estava Aiezer, filho de Amisadai;
\par 26 sobre o exército da tribo dos filhos de Aser, Pagiel, filho de Ocrã;
\par 27 e, sobre o exército da tribo dos filhos de Naftali, Aira, filho de Enã.
\par 28 Nesta ordem, puseram-se em marcha os filhos de Israel, segundo os seus exércitos.
\par 29 Disse Moisés a Hobabe, filho de Reuel, o midianita, sogro de Moisés: Estamos de viagem para o lugar de que o SENHOR disse: Dar-vo-lo-ei; vem conosco, e te faremos bem, porque o SENHOR prometeu boas coisas a Israel.
\par 30 Porém ele respondeu: Não irei; antes, irei à minha terra e à minha parentela.
\par 31 Tornou-lhe Moisés: Ora, não nos deixes, porque tu sabes que devemos acampar-nos no deserto; e nos servirás de guia.
\par 32 Se vieres conosco, far-te-emos o mesmo bem que o SENHOR a nós nos fizer.
\par 33 Partiram, pois, do monte do SENHOR caminho de três dias; a arca da Aliança do SENHOR ia adiante deles caminho de três dias, para lhes deparar lugar de descanso.
\par 34 A nuvem do SENHOR pairava sobre eles de dia, quando partiam do arraial.
\par 35 Partindo a arca, Moisés dizia: Levanta-te, SENHOR, e dissipados sejam os teus inimigos, e fujam diante de ti os que te odeiam.
\par 36 E, quando pousava, dizia: Volta, ó SENHOR, para os milhares de milhares de Israel.

\chapter{11}

\par 1 Queixou-se o povo de sua sorte aos ouvidos do SENHOR; ouvindo-o o SENHOR, acendeu-se-lhe a ira, e fogo do SENHOR ardeu entre eles e consumiu extremidades do arraial.
\par 2 Então, o povo clamou a Moisés, e, orando este ao SENHOR, o fogo se apagou.
\par 3 Pelo que chamou aquele lugar Taberá, porque o fogo do SENHOR se acendera entre eles.
\par 4 E o populacho que estava no meio deles veio a ter grande desejo das comidas dos egípcios; pelo que os filhos de Israel tornaram a chorar e também disseram: Quem nos dará carne a comer?
\par 5 Lembramo-nos dos peixes que, no Egito, comíamos de graça; dos pepinos, dos melões, dos alhos silvestres, das cebolas e dos alhos.
\par 6 Agora, porém, seca-se a nossa alma, e nenhuma coisa vemos senão este maná.
\par 7 Era o maná como semente de coentro, e a sua aparência, semelhante à de bdélio.
\par 8 Espalhava-se o povo, e o colhia, e em moinhos o moía ou num gral o pisava, e em panelas o cozia, e dele fazia bolos; o seu sabor era como o de bolos amassados com azeite.
\par 9 Quando, de noite, descia o orvalho sobre o arraial, sobre este também caía o maná.
\par 10 Então, Moisés ouviu chorar o povo por famílias, cada um à porta de sua tenda; e a ira do SENHOR grandemente se acendeu, e pareceu mal aos olhos de Moisés.
\par 11 Disse Moisés ao SENHOR: Por que fizeste mal a teu servo, e por que não achei favor aos teus olhos, visto que puseste sobre mim a carga de todo este povo?
\par 12 Concebi eu, porventura, todo este povo? Dei-o eu à luz, para que me digas: Leva-o ao teu colo, como a ama leva a criança que mama, à terra que, sob juramento, prometeste a seus pais?
\par 13 Donde teria eu carne para dar a todo este povo? Pois chora diante de mim, dizendo: Dá-nos carne que possamos comer.
\par 14 Eu sozinho não posso levar todo este povo, pois me é pesado demais.
\par 15 Se assim me tratas, mata-me de uma vez, eu te peço, se tenho achado favor aos teus olhos; e não me deixes ver a minha miséria.
\par 16 Disse o SENHOR a Moisés: Ajunta-me setenta homens dos anciãos de Israel, que sabes serem anciãos e superintendentes do povo; e os trarás perante a tenda da congregação, para que assistam ali contigo.
\par 17 Então, descerei e ali falarei contigo; tirarei do Espírito que está sobre ti e o porei sobre eles; e contigo levarão a carga do povo, para que não a leves tu somente.
\par 18 Dize ao povo: Santificai-vos para amanhã e comereis carne; porquanto chorastes aos ouvidos do SENHOR, dizendo: Quem nos dará carne a comer? Íamos bem no Egito. Pelo que o SENHOR vos dará carne, e comereis.
\par 19 Não comereis um dia, nem dois dias, nem cinco, nem dez, nem ainda vinte;
\par 20 mas um mês inteiro, até vos sair pelos narizes, até que vos enfastieis dela, porquanto rejeitastes o SENHOR, que está no meio de vós, e chorastes diante dele, dizendo: Por que saímos do Egito?
\par 21 Respondeu Moisés: Seiscentos mil homens de pé é este povo no meio do qual estou; e tu disseste: Dar-lhes-ei carne, e a comerão um mês inteiro.
\par 22 Matar-se-ão para eles rebanhos de ovelhas e de gado que lhes bastem? Ou se ajuntarão para eles todos os peixes do mar que lhes bastem?
\par 23 Porém o SENHOR respondeu a Moisés: Ter-se-ia encurtado a mão do SENHOR? Agora mesmo, verás se se cumprirá ou não a minha palavra!
\par 24 Saiu, pois, Moisés, e referiu ao povo as palavras do SENHOR, e ajuntou setenta homens dos anciãos do povo, e os pôs ao redor da tenda.
\par 25 Então, o SENHOR desceu na nuvem e lhe falou; e, tirando do Espírito que estava sobre ele, o pôs sobre aqueles setenta anciãos; quando o Espírito repousou sobre eles, profetizaram; mas, depois, nunca mais.
\par 26 Porém, no arraial, ficaram dois homens; um se chamava Eldade, e o outro, Medade. Repousou sobre eles o Espírito, porquanto estavam entre os inscritos, ainda que não saíram à tenda; e profetizavam no arraial.
\par 27 Então, correu um moço, e o anunciou a Moisés, e disse: Eldade e Medade profetizam no arraial.
\par 28 Josué, filho de Num, servidor de Moisés, um dos seus escolhidos, respondeu e disse: Moisés, meu senhor, proíbe-lho.
\par 29 Porém Moisés lhe disse: Tens tu ciúmes por mim? Tomara todo o povo do SENHOR fosse profeta, que o SENHOR lhes desse o seu Espírito!
\par 30 Depois, Moisés se recolheu ao arraial, ele e os anciãos de Israel.
\par 31 Então, soprou um vento do SENHOR, e trouxe codornizes do mar, e as espalhou pelo arraial quase caminho de um dia, ao seu redor, cerca de dois côvados sobre a terra.
\par 32 Levantou-se o povo todo aquele dia, e a noite, e o outro dia e recolheu as codornizes; o que menos colheu teve dez ômeres; e as estenderam para si ao redor do arraial.
\par 33 Estava ainda a carne entre os seus dentes, antes que fosse mastigada, quando se acendeu a ira do SENHOR contra o povo, e o feriu com praga mui grande.
\par 34 Pelo que o nome daquele lugar se chamou Quibrote-Hataavá, porquanto ali enterraram o povo que teve o desejo das comidas dos egípcios.
\par 35 De Quibrote-Hataavá partiu o povo para Hazerote e ali ficou.

\chapter{12}

\par 1 Falaram Miriã e Arão contra Moisés, por causa da mulher cuxita que tomara; pois tinha tomado a mulher cuxita.
\par 2 E disseram: Porventura, tem falado o SENHOR somente por Moisés? Não tem falado também por nós? O SENHOR o ouviu.
\par 3 Era o varão Moisés mui manso, mais do que todos os homens que havia sobre a terra.
\par 4 Logo o SENHOR disse a Moisés, e a Arão, e a Miriã: Vós três, saí à tenda da congregação. E saíram eles três.
\par 5 Então, o SENHOR desceu na coluna de nuvem e se pôs à porta da tenda; depois, chamou a Arão e a Miriã, e eles se apresentaram.
\par 6 Então, disse: Ouvi, agora, as minhas palavras; se entre vós há profeta, eu, o SENHOR, em visão a ele, me faço conhecer ou falo com ele em sonhos.
\par 7 Não é assim com o meu servo Moisés, que é fiel em toda a minha casa.
\par 8 Boca a boca falo com ele, claramente e não por enigmas; pois ele vê a forma do SENHOR; como, pois, não temestes falar contra o meu servo, contra Moisés?
\par 9 E a ira do SENHOR contra eles se acendeu; e retirou-se.
\par 10 A nuvem afastou-se de sobre a tenda; e eis que Miriã achou-se leprosa, branca como neve; e olhou Arão para Miriã, e eis que estava leprosa.
\par 11 Então, disse Arão a Moisés: Ai! Senhor meu, não ponhas, te rogo, sobre nós este pecado, pois loucamente procedemos e pecamos.
\par 12 Ora, não seja ela como um aborto, que, saindo do ventre de sua mãe, tenha metade de sua carne já consumida.
\par 13 Moisés clamou ao SENHOR, dizendo: Ó Deus, rogo-te que a cures.
\par 14 Respondeu o SENHOR a Moisés: Se seu pai lhe cuspira no rosto, não seria envergonhada por sete dias? Seja detida sete dias fora do arraial e, depois, recolhida.
\par 15 Assim, Miriã foi detida fora do arraial por sete dias; e o povo não partiu enquanto Miriã não foi recolhida.
\par 16 Porém, depois, o povo partiu de Hazerote e acampou-se no deserto de Parã.

\chapter{13}

\par 1 Disse o SENHOR a Moisés:
\par 2 Envia homens que espiem a terra de Canaã, que eu hei de dar aos filhos de Israel; de cada tribo de seus pais enviareis um homem, sendo cada qual príncipe entre eles.
\par 3 Enviou-os Moisés do deserto de Parã, segundo o mandado do SENHOR; todos aqueles homens eram cabeças dos filhos de Israel.
\par 4 São estes os seus nomes: da tribo de Rúben, Samua, filho de Zacur;
\par 5 da tribo de Simeão, Safate, filho de Hori;
\par 6 da tribo de Judá, Calebe, filho de Jefoné;
\par 7 da tribo de Issacar, Jigeal, filho de José;
\par 8 da tribo de Efraim, Oséias, filho de Num;
\par 9 da tribo de Benjamim, Palti, filho de Rafu;
\par 10 da tribo de Zebulom, Gadiel, filho de Sodi;
\par 11 da tribo de José, pela tribo de Manassés, Gadi, filho de Susi;
\par 12 da tribo de Dã, Amiel, filho de Gemali;
\par 13 da tribo de Aser, Setur, filho de Micael;
\par 14 da tribo de Naftali, Nabi, filho de Vofsi;
\par 15 da tribo de Gade, Geuel, filho de Maqui.
\par 16 São estes os nomes dos homens que Moisés enviou a espiar aquela terra; e a Oséias, filho de Num, Moisés chamou Josué.
\par 17 Enviou-os, pois, Moisés a espiar a terra de Canaã; e disse-lhes: Subi ao Neguebe e penetrai nas montanhas.
\par 18 Vede a terra, que tal é, e o povo que nela habita, se é forte ou fraco, se poucos ou muitos.
\par 19 E qual é a terra em que habita, se boa ou má; e que tais são as cidades em que habita, se em arraiais, se em fortalezas.
\par 20 Também qual é a terra, se fértil ou estéril, se nela há matas ou não. Tende ânimo e trazei do fruto da terra. Eram aqueles dias os dias das primícias das uvas.
\par 21 Assim, subiram e espiaram a terra desde o deserto de Zim até Reobe, à entrada de Hamate.
\par 22 E subiram pelo Neguebe e vieram até Hebrom; estavam ali Aimã, Sesai e Talmai, filhos de Anaque (Hebrom foi edificada sete anos antes de Zoã, no Egito).
\par 23 Depois, vieram até ao vale de Escol e dali cortaram um ramo de vide com um cacho de uvas, o qual trouxeram dois homens numa vara, como também romãs e figos.
\par 24 Esse lugar se chamou o vale de Escol, por causa do cacho que ali cortaram os filhos de Israel.
\par 25 Ao cabo de quarenta dias, voltaram de espiar a terra,
\par 26 caminharam e vieram a Moisés, e a Arão, e a toda a congregação dos filhos de Israel no deserto de Parã, a Cades; deram-lhes conta, a eles e a toda a congregação, e mostraram-lhes o fruto da terra.
\par 27 Relataram a Moisés e disseram: Fomos à terra a que nos enviaste; e, verdadeiramente, mana leite e mel; este é o fruto dela.
\par 28 O povo, porém, que habita nessa terra é poderoso, e as cidades, mui grandes e fortificadas; também vimos ali os filhos de Anaque.
\par 29 Os amalequitas habitam na terra do Neguebe; os heteus, os jebuseus e os amorreus habitam na montanha; os cananeus habitam ao pé do mar e pela ribeira do Jordão.
\par 30 Então, Calebe fez calar o povo perante Moisés e disse: Eia! Subamos e possuamos a terra, porque, certamente, prevaleceremos contra ela.
\par 31 Porém os homens que com ele tinham subido disseram: Não poderemos subir contra aquele povo, porque é mais forte do que nós.
\par 32 E, diante dos filhos de Israel, infamaram a terra que haviam espiado, dizendo: A terra pelo meio da qual passamos a espiar é terra que devora os seus moradores; e todo o povo que vimos nela são homens de grande estatura.
\par 33 Também vimos ali gigantes (os filhos de Anaque são descendentes de gigantes), e éramos, aos nossos próprios olhos, como gafanhotos e assim também o éramos aos seus olhos.

\chapter{14}

\par 1 Levantou-se, pois, toda a congregação e gritou em voz alta; e o povo chorou aquela noite.
\par 2 Todos os filhos de Israel murmuraram contra Moisés e contra Arão; e toda a congregação lhes disse: Tomara tivéssemos morrido na terra do Egito ou mesmo neste deserto!
\par 3 E por que nos traz o SENHOR a esta terra, para cairmos à espada e para que nossas mulheres e nossas crianças sejam por presa? Não nos seria melhor voltarmos para o Egito?
\par 4 E diziam uns aos outros: Levantemos um capitão e voltemos para o Egito.
\par 5 Então, Moisés e Arão caíram sobre o seu rosto perante a congregação dos filhos de Israel.
\par 6 E Josué, filho de Num, e Calebe, filho de Jefoné, dentre os que espiaram a terra, rasgaram as suas vestes
\par 7 e falaram a toda a congregação dos filhos de Israel, dizendo: A terra pelo meio da qual passamos a espiar é terra muitíssimo boa.
\par 8 Se o SENHOR se agradar de nós, então, nos fará entrar nessa terra e no-la dará, terra que mana leite e mel.
\par 9 Tão-somente não sejais rebeldes contra o SENHOR e não temais o povo dessa terra, porquanto, como pão, os podemos devorar; retirou-se deles o seu amparo; o SENHOR é conosco; não os temais.
\par 10 Apesar disso, toda a congregação disse que os apedrejassem; porém a glória do SENHOR apareceu na tenda da congregação a todos os filhos de Israel.
\par 11 Disse o SENHOR a Moisés: Até quando me provocará este povo e até quando não crerá em mim, a despeito de todos os sinais que fiz no meio dele?
\par 12 Com pestilência o ferirei e o deserdarei; e farei de ti povo maior e mais forte do que este.
\par 13 Respondeu Moisés ao SENHOR: Os egípcios não somente ouviram que, com a tua força, fizeste subir este povo do meio deles,
\par 14 mas também o disseram aos moradores desta terra; ouviram que tu, ó SENHOR, estás no meio deste povo, que face a face, ó SENHOR, lhes apareces, tua nuvem está sobre eles, e vais adiante deles numa coluna de nuvem, de dia, e, numa coluna de fogo, de noite.
\par 15 Se matares este povo como a um só homem, as gentes, pois, que, antes, ouviram a tua fama, dirão:
\par 16 Não podendo o SENHOR fazer entrar este povo na terra que lhe prometeu com juramento, os matou no deserto.
\par 17 Agora, pois, rogo-te que a força do meu Senhor se engrandeça, como tens falado, dizendo:
\par 18 O SENHOR é longânimo e grande em misericórdia, que perdoa a iniqüidade e a transgressão, ainda que não inocenta o culpado, e visita a iniqüidade dos pais nos filhos até à terceira e quarta gerações.
\par 19 Perdoa, pois, a iniqüidade deste povo, segundo a grandeza da tua misericórdia e como também tens perdoado a este povo desde a terra do Egito até aqui.
\par 20 Tornou-lhe o SENHOR: Segundo a tua palavra, eu lhe perdoei.
\par 21 Porém, tão certo como eu vivo, e como toda a terra se encherá da glória do SENHOR,
\par 22 nenhum dos homens que, tendo visto a minha glória e os prodígios que fiz no Egito e no deserto, todavia, me puseram à prova já dez vezes e não obedeceram à minha voz,
\par 23 nenhum deles verá a terra que, com juramento, prometi a seus pais, sim, nenhum daqueles que me desprezaram a verá.
\par 24 Porém o meu servo Calebe, visto que nele houve outro espírito, e perseverou em seguir-me, eu o farei entrar a terra que espiou, e a sua descendência a possuirá.
\par 25 Ora, os amalequitas e os cananeus habitam no vale; mudai, amanhã, de rumo e caminhai para o deserto, pelo caminho do mar Vermelho.
\par 26 Depois, disse o SENHOR a Moisés e a Arão:
\par 27 Até quando sofrerei esta má congregação que murmura contra mim? Tenho ouvido as murmurações que os filhos de Israel proferem contra mim.
\par 28 Dize-lhes: Por minha vida, diz o SENHOR, que, como falastes aos meus ouvidos, assim farei a vós outros.
\par 29 Neste deserto, cairá o vosso cadáver, como também todos os que de vós foram contados segundo o censo, de vinte anos para cima, os que dentre vós contra mim murmurastes;
\par 30 não entrareis na terra a respeito da qual jurei que vos faria habitar nela, salvo Calebe, filho de Jefoné, e Josué, filho de Num.
\par 31 Mas os vossos filhos, de que dizeis: Por presa serão, farei entrar nela; e eles conhecerão a terra que vós desprezastes.
\par 32 Porém, quanto a vós outros, o vosso cadáver cairá neste deserto.
\par 33 Vossos filhos serão pastores neste deserto quarenta anos e levarão sobre si as vossas infidelidades, até que o vosso cadáver se consuma neste deserto.
\par 34 Segundo o número dos dias em que espiastes a terra, quarenta dias, cada dia representando um ano, levareis sobre vós as vossas iniqüidades quarenta anos e tereis experiência do meu desagrado.
\par 35 Eu, o SENHOR, falei; assim farei a toda esta má congregação, que se levantou contra mim; neste deserto, se consumirão e aí falecerão.
\par 36 Os homens que Moisés mandara a espiar a terra e que, voltando, fizeram murmurar toda a congregação contra ele, infamando a terra,
\par 37 esses mesmos homens que infamaram a terra morreram de praga perante o SENHOR.
\par 38 Mas Josué, filho de Num, e Calebe, filho de Jefoné, que eram dos homem que foram espiar a terra, sobreviveram.
\par 39 Falou Moisés estas palavras a todos os filhos de Israel, e o povo se contristou muito.
\par 40 Levantaram-se pela manhã de madrugada e subiram ao cimo do monte, dizendo: Eis-nos aqui e subiremos ao lugar que o SENHOR tem prometido, porquanto havemos pecado.
\par 41 Porém Moisés respondeu: Por que transgredis o mandado do SENHOR? Pois isso não prosperará.
\par 42 Não subais, pois o SENHOR não estará no meio de vós, para que não sejais feridos diante dos vossos inimigos.
\par 43 Porque os amalequitas e os cananeus ali estão diante de vós, e caireis à espada; pois, uma vez que vos desviastes do SENHOR, o SENHOR não será convosco.
\par 44 Contudo, temerariamente, tentaram subir ao cimo do monte, mas a arca da Aliança do SENHOR e Moisés não se apartaram do meio do arraial.
\par 45 Então, desceram os amalequitas e os cananeus que habitavam na montanha e os feriram, derrotando-os até Horma.

\chapter{15}

\par 1 Disse o SENHOR a Moisés:
\par 2 Fala aos filhos de Israel e dize-lhes: Quando entrardes na terra das vossas habitações, que eu vos hei de dar,
\par 3 e ao SENHOR fizerdes oferta queimada, holocausto ou sacrifício, em cumprimento de um voto ou em oferta voluntária, ou, nas vossas festas fixas, apresentardes ao SENHOR aroma agradável com o sacrifício de gado e ovelhas,
\par 4 então, aquele que apresentar a sua oferta ao SENHOR, por oferta de manjares, trará a décima parte de um efa de flor de farinha, misturada com a quarta parte de um him de azeite.
\par 5 E de vinho para libação prepararás a quarta parte de um him para cada cordeiro, além do holocausto ou do sacrifício.
\par 6 Para cada carneiro prepararás uma oferta de manjares de duas décimas de um efa de flor de farinha, misturada com a terça parte de um him de azeite;
\par 7 e de vinho para a libação oferecerás a terça parte de um him ao SENHOR, em aroma agradável.
\par 8 Quando preparares novilho para holocausto ou sacrifício, em cumprimento de um voto ou um sacrifício pacífico ao SENHOR,
\par 9 com o novilho, trarás uma oferta de manjares de três décimas de um efa de flor de farinha, misturada com a metade de um him de azeite,
\par 10 e de vinho para a libação trarás a metade de um him, oferta queimada de aroma agradável ao SENHOR.
\par 11 Assim se fará com todos os novilhos, carneiros, cordeiros e bodes.
\par 12 Segundo o número que oferecerdes, assim o fareis para cada um.
\par 13 Todos os naturais assim farão estas coisas, trazendo oferta queimada de aroma agradável ao SENHOR.
\par 14 Se também morar convosco algum estrangeiro ou quem quer que estiver entre vós durante as vossas gerações, e trouxer uma oferta queimada de aroma agradável ao SENHOR, como vós fizerdes, assim fará ele.
\par 15 Quanto à congregação, haja apenas um estatuto, tanto para vós outros como para o estrangeiro que morar entre vós, por estatuto perpétuo nas vossas gerações; como vós sois, assim será o estrangeiro perante o SENHOR.
\par 16 A mesma lei e o mesmo rito haverá para vós outros e para o estrangeiro que mora convosco.
\par 17 Disse mais o SENHOR a Moisés:
\par 18 Fala aos filhos de Israel e dize-lhes: Quando chegardes à terra em que vos farei entrar,
\par 19 ao comerdes do pão da terra, apresentareis oferta ao SENHOR.
\par 20 Das primícias da vossa farinha grossa apresentareis um bolo como oferta; como oferta da eira, assim o apresentareis.
\par 21 Das primícias da vossa farinha grossa apresentareis ao SENHOR oferta nas vossas gerações.
\par 22 Quando errardes e não cumprirdes todos estes mandamentos que o SENHOR falou a Moisés,
\par 23 sim, tudo quanto o SENHOR vos tem mandado por Moisés, desde o dia em que o SENHOR ordenou e daí em diante, nas vossas gerações,
\par 24 será que, quando se fizer alguma coisa por ignorância e for encoberta aos olhos da congregação, toda a congregação oferecerá um novilho, para holocausto de aroma agradável ao SENHOR, com a sua oferta de manjares e libação, segundo o rito, e um bode, para oferta pelo pecado.
\par 25 O sacerdote fará expiação por toda a congregação dos filhos de Israel, e lhes será perdoado, porquanto foi erro, e trouxeram a sua oferta, oferta queimada ao SENHOR, e a sua oferta pelo pecado perante o SENHOR, por causa do seu erro.
\par 26 Será, pois, perdoado a toda a congregação dos filhos de Israel e mais ao estrangeiro que habita no meio deles, pois no erro foi envolvido todo o povo.
\par 27 Se alguma pessoa pecar por ignorância, apresentará uma cabra de um ano como oferta pelo pecado.
\par 28 O sacerdote fará expiação pela pessoa que errou, quando pecar por ignorância perante o SENHOR, fazendo expiação por ela, e lhe será perdoado.
\par 29 Para o natural dos filhos de Israel e para o estrangeiro que no meio deles habita, tereis a mesma lei para aquele que isso fizer por ignorância.
\par 30 Mas a pessoa que fizer alguma coisa atrevidamente, quer seja dos naturais quer dos estrangeiros, injuria ao SENHOR; tal pessoa será eliminada do meio do seu povo,
\par 31 pois desprezou a palavra do SENHOR e violou o seu mandamento; será eliminada essa pessoa, e a sua iniqüidade será sobre ela.
\par 32 Estando, pois, os filhos de Israel no deserto, acharam um homem apanhando lenha no dia de sábado.
\par 33 Os que o acharam apanhando lenha o trouxeram a Moisés, e a Arão, e a toda a congregação.
\par 34 Meteram-no em guarda, porquanto ainda não estava declarado o que se lhe devia fazer.
\par 35 Então, disse o SENHOR a Moisés: Tal homem será morto; toda a congregação o apedrejará fora do arraial.
\par 36 Levou-o, pois, toda a congregação para fora do arraial, e o apedrejaram; e ele morreu, como o SENHOR ordenara a Moisés.
\par 37 Disse o SENHOR a Moisés:
\par 38 Fala aos filhos de Israel e dize-lhes que nos cantos das suas vestes façam borlas pelas suas gerações; e as borlas em cada canto, presas por um cordão azul.
\par 39 E as borlas estarão ali para que, vendo-as, vos lembreis de todos os mandamentos do SENHOR e os cumprais; não seguireis os desejos do vosso coração, nem os dos vossos olhos, após os quais andais adulterando,
\par 40 para que vos lembreis de todos os meus mandamentos, e os cumprais, e santos sereis a vosso Deus.
\par 41 Eu sou o SENHOR, vosso Deus, que vos tirei da terra do Egito, para vos ser por Deus. Eu sou o SENHOR, vosso Deus.

\chapter{16}

\par 1 Corá, filho de Isar, filho de Coate, filho de Levi, tomou consigo a Datã e a Abirão, filhos de Eliabe, e a Om, filho de Pelete, filhos de Rúben.
\par 2 Levantaram-se perante Moisés com duzentos e cinqüenta homens dos filhos de Israel, príncipes da congregação, eleitos por ela, varões de renome,
\par 3 e se ajuntaram contra Moisés e contra Arão e lhes disseram: Basta! Pois que toda a congregação é santa, cada um deles é santo, e o SENHOR está no meio deles; por que, pois, vos exaltais sobre a congregação do SENHOR?
\par 4 Tendo ouvido isto, Moisés caiu sobre o seu rosto.
\par 5 E falou a Corá e a todo o seu grupo, dizendo: Amanhã pela manhã, o SENHOR fará saber quem é dele e quem é o santo que ele fará chegar a si; aquele a quem escolher fará chegar a si.
\par 6 Fazei isto: tomai vós incensários, Corá e todo o seu grupo;
\par 7 e, pondo fogo neles amanhã, sobre eles deitai incenso perante o SENHOR; e será que o homem a quem o SENHOR escolher, este será o santo; basta-vos, filhos de Levi.
\par 8 Disse mais Moisés a Corá: Ouvi agora, filhos de Levi:
\par 9 acaso, é para vós outros coisa de somenos que o Deus de Israel vos separou da congregação de Israel, para vos fazer chegar a si, a fim de cumprirdes o serviço do tabernáculo do SENHOR e estardes perante a congregação para ministrar-lhe;
\par 10 e te fez chegar, Corá, e todos os teus irmãos, os filhos de Levi, contigo? Ainda também procurais o sacerdócio?
\par 11 Pelo que tu e todo o teu grupo juntos estais contra o SENHOR; e Arão, que é ele para que murmureis contra ele?
\par 12 Mandou Moisés chamar a Datã e a Abirão, filhos de Eliabe; porém eles disseram: Não subiremos;
\par 13 porventura, é coisa de somenos que nos fizeste subir de uma terra que mana leite e mel, para fazer-nos morrer neste deserto, senão que também queres fazer-te príncipe sobre nós?
\par 14 Nem tampouco nos trouxeste a uma terra que mana leite e mel, nem nos deste campos e vinhas em herança; pensas que lançarás pó aos olhos destes homens? Pois não subiremos.
\par 15 Então, Moisés irou-se muito e disse ao SENHOR: Não atentes para a sua oferta; nem um só jumento levei deles e a nenhum deles fiz mal.
\par 16 Disse mais Moisés a Corá: Tu e todo o teu grupo, ponde-vos perante o SENHOR, tu, e eles, e Arão, amanhã.
\par 17 Tomai cada um o seu incensário e neles ponde incenso; trazei-o, cada um o seu, perante o SENHOR, duzentos e cinqüenta incensários; também tu e Arão, cada qual o seu.
\par 18 Tomaram, pois, cada qual o seu incensário, neles puseram fogo, sobre eles deitaram incenso e se puseram perante a porta da tenda da congregação com Moisés e Arão.
\par 19 Corá fez ajuntar contra eles todo o povo à porta da tenda da congregação; então, a glória do SENHOR apareceu a toda a congregação.
\par 20 Disse o SENHOR a Moisés e a Arão:
\par 21 Apartai-vos do meio desta congregação, e os consumirei num momento.
\par 22 Mas eles se prostraram sobre o seu rosto e disseram: Ó Deus, Autor e Conservador de toda a vida, acaso, por pecar um só homem, indignar-te-ás contra toda esta congregação?
\par 23 Respondeu o SENHOR a Moisés:
\par 24 Fala a toda esta congregação, dizendo: Levantai-vos do redor da habitação de Corá, Datã e Abirão.
\par 25 Então, se levantou Moisés e foi a Datã e a Abirão; e após ele foram os anciãos de Israel.
\par 26 E disse à congregação: Desviai-vos, peço-vos, das tendas destes homens perversos e não toqueis nada do que é seu, para que não sejais arrebatados em todos os seus pecados.
\par 27 Levantaram-se, pois, do redor da habitação de Corá, Datã e Abirão; e Datã e Abirão saíram e se puseram à porta da sua tenda, juntamente com suas mulheres, seus filhos e suas crianças.
\par 28 Então, disse Moisés: Nisto conhecereis que o SENHOR me enviou a realizar todas estas obras, que não procedem de mim mesmo:
\par 29 se morrerem estes como todos os homens morrem e se forem visitados por qualquer castigo como se dá com todos os homens, então, não sou enviado do SENHOR.
\par 30 Mas, se o SENHOR criar alguma coisa inaudita, e a terra abrir a sua boca e os tragar com tudo o que é seu, e vivos descerem ao abismo, então, conhecereis que estes homens desprezaram o SENHOR.
\par 31 E aconteceu que, acabando ele de falar todas estas palavras, a terra debaixo deles se fendeu,
\par 32 abriu a sua boca e os tragou com as suas casas, como também todos os homens que pertenciam a Corá e todos os seus bens.
\par 33 Eles e todos os que lhes pertenciam desceram vivos ao abismo; a terra os cobriu, e pereceram do meio da congregação.
\par 34 Todo o Israel que estava ao redor deles fugiu do seu grito, porque diziam: Não suceda que a terra nos trague a nós também.
\par 35 Procedente do SENHOR saiu fogo e consumiu os duzentos e cinqüenta homens que ofereciam o incenso.
\par 36 Disse o SENHOR a Moisés:
\par 37 Dize a Eleazar, filho de Arão, o sacerdote, que tome os incensários do meio do incêndio e espalhe o fogo longe, porque santos são;
\par 38 quanto aos incensários daqueles que pecaram contra a sua própria vida, deles se façam lâminas para cobertura do altar; porquanto os trouxeram perante o SENHOR; pelo que santos são e serão por sinal aos filhos de Israel.
\par 39 Eleazar, o sacerdote, tomou os incensários de metal, que tinham trazido aqueles que foram queimados, e os converteram em lâminas para cobertura do altar,
\par 40 por memorial para os filhos de Israel, para que nenhum estranho, que não for da descendência de Arão, se chegue para acender incenso perante o SENHOR; para que não seja como Corá e o seu grupo, como o SENHOR lhe tinha dito por Moisés.
\par 41 Mas, no dia seguinte, toda a congregação dos filhos de Israel murmurou contra Moisés e contra Arão, dizendo: Vós matastes o povo do SENHOR.
\par 42 Ajuntando-se o povo contra Moisés e Arão e virando-se para a tenda da congregação, eis que a nuvem a cobriu, e a glória do SENHOR apareceu.
\par 43 Vieram, pois, Moisés e Arão perante a tenda da congregação.
\par 44 Então, falou o SENHOR a Moisés, dizendo:
\par 45 Levantai-vos do meio desta congregação, e a consumirei num momento; então, se prostraram sobre o seu rosto.
\par 46 Disse Moisés a Arão: Toma o teu incensário, põe nele fogo do altar, deita incenso sobre ele, vai depressa à congregação e faze expiação por eles; porque grande indignação saiu de diante do SENHOR; já começou a praga.
\par 47 Tomou-o Arão, como Moisés lhe falara, correu ao meio da congregação (eis que já a praga havia começado entre o povo), deitou incenso nele e fez expiação pelo povo.
\par 48 Pôs-se em pé entre os mortos e os vivos; e cessou a praga.
\par 49 Ora, os que morreram daquela praga foram catorze mil e setecentos, fora os que morreram por causa de Corá.
\par 50 Voltou Arão a Moisés, à porta da tenda da congregação; e cessou a praga.

\chapter{17}

\par 1 Disse o SENHOR a Moisés:
\par 2 Fala aos filhos de Israel e recebe deles bordões, uma pela casa de cada pai de todos os seus príncipes, segundo as casas de seus pais, isto é, doze bordões; escreve o nome de cada um sobre o seu bordão.
\par 3 Porém o nome de Arão escreverás sobre o bordão de Levi; porque cada cabeça da casa de seus pais terá um bordão.
\par 4 E as porás na tenda da congregação, perante o Testemunho, onde eu vos encontrarei.
\par 5 O bordão do homem que eu escolher, esse florescerá; assim, farei cessar de sobre mim as murmurações que os filhos de Israel proferem contra vós.
\par 6 Falou, pois, Moisés aos filhos de Israel, e todos os seus príncipes lhe deram bordões; cada um lhe deu um, segundo as casas de seus pais: doze bordões; e, entre eles, o bordão de Arão.
\par 7 Moisés pôs estes bordões perante o SENHOR, na tenda do Testemunho.
\par 8 No dia seguinte, Moisés entrou na tenda do Testemunho, e eis que o bordão de Arão, pela casa de Levi, brotara, e, tendo inchado os gomos, produzira flores, e dava amêndoas.
\par 9 Então, Moisés trouxe todos os bordões de diante do SENHOR a todos os filhos de Israel; e eles o viram, e tomou cada um o seu bordão.
\par 10 Disse o SENHOR a Moisés: Torna a pôr o bordão de Arão perante o Testemunho, para que se guarde por sinal para filhos rebeldes; assim farás acabar as suas murmurações contra mim, para que não morram.
\par 11 E Moisés fez assim; como lhe ordenara o SENHOR, assim fez.
\par 12 Então, falaram os filhos de Israel a Moisés, dizendo: Eis que expiramos, perecemos, perecemos todos.
\par 13 Todo aquele que se aproximar do tabernáculo do SENHOR morrerá; acaso, expiraremos todos?

\chapter{18}

\par 1 Disse o SENHOR a Arão: Tu, e teus filhos, e a casa de teu pai contigo levareis sobre vós a iniqüidade relativamente ao santuário; tu e teus filhos contigo levareis sobre vós a iniqüidade relativamente ao vosso sacerdócio.
\par 2 Também farás chegar contigo a teus irmãos, a tribo de Levi, a tribo de teu pai, para que se ajuntem a ti e te sirvam, quando tu e teus filhos contigo estiverdes perante a tenda do Testemunho.
\par 3 Farão o serviço que lhes é devido para contigo e para com a tenda; porém não se aproximarão dos utensílios do santuário, nem do altar, para que não morram, nem eles, nem vós.
\par 4 Ajuntar-se-ão a ti e farão todo o serviço da tenda da congregação; o estranho, porém, não se chegará a vós outros.
\par 5 Vós, pois, fareis o serviço do santuário e o do altar, para que não haja outra vez ira contra os filhos de Israel.
\par 6 Eu, eis que tomei vossos irmãos, os levitas, do meio dos filhos de Israel; são dados a vós outros para o SENHOR, para servir na tenda da congregação.
\par 7 Mas tu e teus filhos contigo atendereis ao vosso sacerdócio em tudo concernente ao altar, e ao que estiver para dentro do véu, isto é vosso serviço; eu vos tenho entregue o vosso sacerdócio por ofício como dádiva; porém o estranho que se aproximar morrerá.
\par 8 Disse mais o SENHOR a Arão: Eis que eu te dei o que foi separado das minhas ofertas, com todas as coisas consagradas dos filhos de Israel; dei-as por direito perpétuo como porção a ti e a teus filhos.
\par 9 Isto terás das coisas santíssimas, não dadas ao fogo: todas as suas ofertas, com todas as suas ofertas de manjares, e com todas as suas ofertas pelo pecado, e com todas as suas ofertas pela culpa, que me apresentarem, serão coisas santíssimas para ti e para teus filhos.
\par 10 No lugar santíssimo, o comerás; todo homem o comerá; ser-te-á santo.
\par 11 Também isto será teu: a oferta das suas dádivas com todas as ofertas movidas dos filhos de Israel; a ti, a teus filhos e a tuas filhas contigo, dei-as por direito perpétuo; todo o que estiver limpo na tua casa as comerá.
\par 12 Todo o melhor do azeite, do mosto e dos cereais, as suas primícias que derem ao SENHOR, dei-as a ti.
\par 13 Os primeiros frutos de tudo que houver na terra, que trouxerem ao SENHOR, serão teus; todo o que estiver limpo na tua casa os comerá.
\par 14 Toda coisa consagrada irremissivelmente em Israel será tua.
\par 15 Todo o que abrir a madre, de todo ser vivente, que trouxerem ao SENHOR, tanto de homens como de animais, será teu; porém os primogênitos dos homens resgatarás; também os primogênitos dos animais imundos resgatarás.
\par 16 O resgate, pois (desde a idade de um mês os resgatarás), será segundo a tua avaliação, por cinco siclos de dinheiro, segundo o siclo do santuário, que é de vinte geras.
\par 17 Mas o primogênito do gado, ou primogênito de ovelhas, ou primogênito de cabra não resgatarás; são santos; o seu sangue aspergirás sobre o altar e a sua gordura queimarás em oferta queimada de aroma agradável ao SENHOR.
\par 18 A carne deles será tua, assim como será teu o peito movido e a coxa direita.
\par 19 Todas as ofertas sagradas, que os filhos de Israel oferecerem ao SENHOR, dei-as a ti, e a teus filhos, e a tuas filhas contigo, por direito perpétuo; aliança perpétua de sal perante o SENHOR é esta, para ti e para tua descendência contigo.
\par 20 Disse também o SENHOR a Arão: Na sua terra, herança nenhuma terás e, no meio deles, nenhuma porção terás. Eu sou a tua porção e a tua herança no meio dos filhos de Israel.
\par 21 Aos filhos de Levi dei todos os dízimos em Israel por herança, pelo serviço que prestam, serviço da tenda da congregação.
\par 22 E nunca mais os filhos de Israel se chegarão à tenda da congregação, para que não levem sobre si o pecado e morram.
\par 23 Mas os levitas farão o serviço da tenda da congregação e responderão por suas faltas; estatuto perpétuo é este para todas as vossas gerações. E não terão eles nenhuma herança no meio dos filhos de Israel.
\par 24 Porque os dízimos dos filhos de Israel, que apresentam ao SENHOR em oferta, dei-os por herança aos levitas; porquanto eu lhes disse: No meio dos filhos de Israel, nenhuma herança tereis.
\par 25 Disse o SENHOR a Moisés:
\par 26 Também falarás aos levitas e lhes dirás: Quando receberdes os dízimos da parte dos filhos de Israel, que vos dei por vossa herança, deles apresentareis uma oferta ao SENHOR: o dízimo dos dízimos.
\par 27 Atribuir-se-vos-á a vossa oferta como se fosse cereal da eira e plenitude do lagar.
\par 28 Assim, também apresentareis ao SENHOR uma oferta de todos os vossos dízimos que receberdes dos filhos de Israel e deles dareis a oferta do SENHOR a Arão, o sacerdote.
\par 29 De todas as vossas dádivas apresentareis toda oferta do SENHOR: do melhor delas, a parte que lhe é sagrada.
\par 30 Portanto, lhes dirás: Quando oferecerdes o melhor que há nos dízimos, o restante destes, como se fosse produto da eira e produto do lagar, se contará aos levitas.
\par 31 Comê-lo-eis em todo lugar, vós e a vossa casa, porque é vossa recompensa pelo vosso serviço na tenda da congregação.
\par 32 Pelo que não levareis sobre vós o pecado, quando deles oferecerdes o melhor; e não profanareis as coisas sagradas dos filhos de Israel, para que não morrais.

\chapter{19}

\par 1 Disse mais o SENHOR a Moisés e a Arão:
\par 2 Esta é uma prescrição da lei que o SENHOR ordenou, dizendo: Dize aos filhos de Israel que vos tragam uma novilha vermelha, perfeita, sem defeito, que não tenha ainda levado jugo.
\par 3 Entregá-la-eis a Eleazar, o sacerdote; este a tirará para fora do arraial, e será imolada diante dele.
\par 4 Eleazar, o sacerdote, tomará do sangue com o dedo e dele aspergirá para a frente da tenda da congregação sete vezes.
\par 5 À vista dele, será queimada a novilha; o couro, a carne, o sangue e o excremento, tudo se queimará.
\par 6 E o sacerdote, tomando pau de cedro, hissopo e estofo carmesim, os lançará no meio do fogo que queima a novilha.
\par 7 Então, o sacerdote lavará as vestes, e banhará o seu corpo em água, e, depois, entrará no arraial, e será imundo até à tarde.
\par 8 Também o que a queimou lavará as suas vestes com água, e em água banhará o seu corpo, e imundo será até à tarde.
\par 9 Um homem limpo ajuntará a cinza da novilha e a depositará fora do arraial, num lugar limpo, e será ela guardada para a congregação dos filhos de Israel, para a água purificadora; é oferta pelo pecado.
\par 10 O que apanhou a cinza da novilha lavará as vestes e será imundo até à tarde; isto será por estatuto perpétuo aos filhos de Israel e ao estrangeiro que habita no meio deles.
\par 11 Aquele que tocar em algum morto, cadáver de algum homem, imundo será sete dias.
\par 12 Ao terceiro dia e ao sétimo dia, se purificará com esta água e será limpo; mas, se ao terceiro dia e ao sétimo não se purificar, não será limpo.
\par 13 Todo aquele que tocar em algum morto, cadáver de algum homem, e não se purificar, contamina o tabernáculo do SENHOR; essa pessoa será eliminada de Israel; porque a água purificadora não foi aspergida sobre ele, imundo será; está nele ainda a sua imundícia.
\par 14 Esta é a lei quando morrer algum homem em alguma tenda: todo aquele que entrar nessa tenda e todo aquele que nela estiver serão imundos sete dias.
\par 15 Também todo vaso aberto, sobre que não houver tampa amarrada, será imundo.
\par 16 Todo aquele que, no campo aberto, tocar em alguém que for morto pela espada, ou em outro morto, ou nos ossos de algum homem, ou numa sepultura será imundo sete dias.
\par 17 Para o imundo, pois, tomarão da cinza da queima da oferta pelo pecado e sobre esta cinza porão água corrente, num vaso.
\par 18 Um homem limpo tomará hissopo, e o molhará naquela água, e a aspergirá sobre aquela tenda, e sobre todo utensílio, e sobre as pessoas que ali estiverem; como também sobre aquele que tocar nos ossos, ou em alguém que foi morto, ou que faleceu, ou numa sepultura.
\par 19 O limpo aspergirá sobre o imundo ao terceiro e sétimo dias; purificá-lo-á ao sétimo dia; e aquele que era imundo lavará as suas vestes, e se banhará na água, e à tarde será limpo.
\par 20 No entanto, quem estiver imundo e não se purificar, esse será eliminado do meio da congregação, porquanto contaminou o santuário do SENHOR; água purificadora sobre ele não foi aspergida; é imundo.
\par 21 Isto lhes será por estatuto perpétuo; e o que aspergir a água purificadora lavará as suas vestes, e o que tocar a água purificadora será imundo até à tarde.
\par 22 Tudo o que o imundo tocar também será imundo; e quem o tocar será imundo até à tarde.

\chapter{20}

\par 1 Chegando os filhos de Israel, toda a congregação, ao deserto de Zim, no mês primeiro, o povo ficou em Cades. Ali, morreu Miriã e, ali, foi sepultada.
\par 2 Não havia água para o povo; então, se ajuntaram contra Moisés e contra Arão.
\par 3 E o povo contendeu com Moisés, e disseram: Antes tivéssemos perecido quando expiraram nossos irmãos perante o SENHOR!
\par 4 Por que trouxestes a congregação do SENHOR a este deserto, para morrermos aí, nós e os nossos animais?
\par 5 E por que nos fizestes subir do Egito, para nos trazer a este mau lugar, que não é de cereais, nem de figos, nem de vides, nem de romãs, nem de água para beber?
\par 6 Então, Moisés e Arão se foram de diante do povo para a porta da tenda da congregação e se lançaram sobre o seu rosto; e a glória do SENHOR lhes apareceu.
\par 7 Disse o SENHOR a Moisés:
\par 8 Toma o bordão, ajunta o povo, tu e Arão, teu irmão, e, diante dele, falai à rocha, e dará a sua água; assim lhe tirareis água da rocha e dareis a beber à congregação e aos seus animais.
\par 9 Então, Moisés tomou o bordão de diante do SENHOR, como lhe tinha ordenado.
\par 10 Moisés e Arão reuniram o povo diante da rocha, e Moisés lhe disse: Ouvi, agora, rebeldes: porventura, faremos sair água desta rocha para vós outros?
\par 11 Moisés levantou a mão e feriu a rocha duas vezes com o seu bordão, e saíram muitas águas; e bebeu a congregação e os seus animais.
\par 12 Mas o SENHOR disse a Moisés e a Arão: Visto que não crestes em mim, para me santificardes diante dos filhos de Israel, por isso, não fareis entrar este povo na terra que lhe dei.
\par 13 São estas as águas de Meribá, porque os filhos de Israel contenderam com o SENHOR; e o SENHOR se santificou neles.
\par 14 Enviou Moisés, de Cades, mensageiros ao rei de Edom, a dizer-lhe: Assim diz teu irmão Israel: Bem sabes todo o trabalho que nos tem sobrevindo;
\par 15 como nossos pais desceram ao Egito, e nós no Egito habitamos muito tempo, e como os egípcios nos maltrataram, a nós e a nossos pais;
\par 16 e clamamos ao SENHOR, e ele ouviu a nossa voz, e mandou o Anjo, e nos tirou do Egito. E eis que estamos em Cades, cidade nos confins do teu país.
\par 17 Deixa-nos passar pela tua terra; não o faremos pelo campo, nem pelas vinhas, nem beberemos a água dos poços; iremos pela estrada real; não nos desviaremos para a direita nem para a esquerda, até que passemos pelo teu país.
\par 18 Porém Edom lhe disse: Não passarás por mim, para que não saia eu de espada ao teu encontro.
\par 19 Então, os filhos de Israel lhe disseram: Subiremos pelo caminho trilhado, e, se eu e o meu gado bebermos das tuas águas, pagarei o preço delas; outra coisa não desejo senão passar a pé.
\par 20 Porém ele disse: Não passarás. E saiu-lhe Edom ao encontro, com muita gente e com mão forte.
\par 21 Assim recusou Edom deixar passar a Israel pelo seu país; pelo que Israel se desviou dele.
\par 22 Então, partiram de Cades; e os filhos de Israel, toda a congregação, foram ao monte Hor.
\par 23 Disse o SENHOR a Moisés e a Arão no monte Hor, nos confins da terra de Edom:
\par 24 Arão será recolhido a seu povo, porque não entrará na terra que dei aos filhos de Israel, pois fostes rebeldes à minha palavra, nas águas de Meribá.
\par 25 Toma Arão e Eleazar, seu filho, e faze-os subir ao monte Hor;
\par 26 depois, despe Arão das suas vestes e veste com elas a Eleazar, seu filho; porque Arão será recolhido a seu povo e aí morrerá.
\par 27 Fez Moisés como o SENHOR lhe ordenara; subiram ao monte Hor, perante os olhos de toda a congregação.
\par 28 Moisés, pois, despiu a Arão de suas vestes e vestiu com elas a Eleazar, seu filho; morreu Arão ali sobre o cimo do monte; e dali desceram Moisés e Eleazar.
\par 29 Vendo, pois, toda a congregação que Arão era morto, choraram por Arão trinta dias, isto é, toda a casa de Israel.

\chapter{21}

\par 1 Ouvindo o cananeu, rei de Arade, que habitava no Neguebe, que Israel vinha pelo caminho de Atarim, pelejou contra Israel e levou alguns deles cativos.
\par 2 Então, Israel fez voto ao SENHOR, dizendo: Se, de fato, entregares este povo nas minhas mãos, destruirei totalmente as suas cidades.
\par 3 Ouviu, pois, o SENHOR a voz de Israel e lhe entregou os cananeus. Os israelitas os destruíram totalmente, a eles e a suas cidades; e aquele lugar se chamou Horma.
\par 4 Então, partiram do monte Hor, pelo caminho do mar Vermelho, a rodear a terra de Edom, porém o povo se tornou impaciente no caminho.
\par 5 E o povo falou contra Deus e contra Moisés: Por que nos fizestes subir do Egito, para que morramos neste deserto, onde não há pão nem água? E a nossa alma tem fastio deste pão vil.
\par 6 Então, o SENHOR mandou entre o povo serpentes abrasadoras, que mordiam o povo; e morreram muitos do povo de Israel.
\par 7 Veio o povo a Moisés e disse: Havemos pecado, porque temos falado contra o SENHOR e contra ti; ora ao SENHOR que tire de nós as serpentes. Então, Moisés orou pelo povo.
\par 8 Disse o SENHOR a Moisés: Faze uma serpente abrasadora, põe-na sobre uma haste, e será que todo mordido que a mirar viverá.
\par 9 Fez Moisés uma serpente de bronze e a pôs sobre uma haste; sendo alguém mordido por alguma serpente, se olhava para a de bronze, sarava.
\par 10 Então, partiram os filhos de Israel e se acamparam em Obote.
\par 11 Depois, partiram de Obote e se acamparam em Ijé-Abarim, no deserto que está defronte de Moabe, para o nascente.
\par 12 Dali, partiram e se acamparam no vale de Zerede.
\par 13 E, dali, partiram e se acamparam na outra margem do Arnom, que está no deserto que se estende do território dos amorreus; porque o Arnom é o limite de Moabe, entre Moabe e os amorreus.
\par 14 Pelo que se diz no Livro das Guerras do SENHOR: Vaebe em Sufa, e os vales do Arnom,
\par 15 e o declive dos vales que se inclina para a sede de Ar e se encosta aos limites de Moabe.
\par 16 Dali partiram para Beer; este é o poço do qual disse o SENHOR a Moisés: Ajunta o povo, e lhe darei água.
\par 17 Então, cantou Israel este cântico: Brota, ó poço! Entoai-lhe cânticos!
\par 18 Poço que os príncipes cavaram, que os nobres do povo abriram, com o cetro, com os seus bordões. Do deserto, partiram para Matana.
\par 19 E, de Matana, para Naaliel e, de Naaliel, para Bamote.
\par 20 De Bamote, ao vale que está no campo de Moabe, no cimo de Pisga, que olha para o deserto.
\par 21 Então, Israel mandou mensageiros a Seom, rei dos amorreus, dizendo:
\par 22 Deixa-me passar pela tua terra; não nos desviaremos pelos campos nem pelas vinhas; as águas dos poços não beberemos; iremos pela estrada real até que passemos o teu país.
\par 23 Porém Seom não deixou passar a Israel pelo seu país; antes, reuniu todo o seu povo, e saiu ao encontro de Israel ao deserto, e veio a Jasa, e pelejou contra Israel.
\par 24 Mas Israel o feriu a fio de espada e tomou posse de sua terra, desde o Arnom até ao Jaboque, até aos filhos de Amom, cuja fronteira era fortificada.
\par 25 Assim, Israel tomou todas estas cidades dos amorreus e habitou em todas elas, em Hesbom e em todas as suas aldeias.
\par 26 Porque Hesbom era cidade de Seom, rei dos amorreus, que tinha pelejado contra o precedente rei dos moabitas, de cuja mão tomara toda a sua terra até ao Arnom.
\par 27 Pelo que dizem os poetas: Vinde a Hesbom! Edifique-se, estabeleça-se a cidade de Seom!
\par 28 Porque fogo saiu de Hesbom, e chama, da cidade de Seom, e consumiu a Ar, de Moabe, e os senhores dos altos do Arnom.
\par 29 Ai de ti, Moabe! Perdido estás, povo de Quemos; entregou seus filhos como fugitivos e suas filhas, como cativas a Seom, rei dos amorreus.
\par 30 Nós os asseteamos; estão destruídos desde Hesbom até Dibom; e os assolamos até Nofa e com fogo, até Medeba.
\par 31 Assim, Israel habitou na terra dos amorreus.
\par 32 Depois, mandou Moisés espiar a Jazer, tomaram as suas aldeias e desapossaram os amorreus que se achavam ali.
\par 33 Então, voltaram e subiram o caminho de Basã; e Ogue, rei de Basã, saiu contra eles, ele e todo o seu povo, à peleja em Edrei.
\par 34 Disse o SENHOR a Moisés: Não o temas, porque eu o dei na tua mão, a ele, e a todo o seu povo, e a sua terra; e far-lhe-ás como fizeste a Seom, rei dos amorreus, que habitava em Hesbom.
\par 35 De tal maneira o feriram, a ele, e a seus filhos, e a todo o seu povo, que nenhum deles escapou; e lhe tomaram posse da terra.

\chapter{22}

\par 1 Tendo partido os filhos de Israel, acamparam-se nas campinas de Moabe, além do Jordão, na altura de Jericó.
\par 2 Viu, pois, Balaque, filho de Zipor, tudo o que Israel fizera aos amorreus;
\par 3 Moabe teve grande medo deste povo, porque era muito; e andava angustiado por causa dos filhos de Israel;
\par 4 pelo que Moabe disse aos anciãos dos midianitas: Agora, lamberá esta multidão tudo quando houver ao redor de nós, como o boi lambe a erva do campo. Balaque, filho de Zipor, naquele tempo, era rei dos moabitas.
\par 5 Enviou ele mensageiros a Balaão, filho de Beor, a Petor, que está junto ao rio Eufrates, na terra dos filhos do seu povo, a chamá-lo, dizendo: Eis que um povo saiu do Egito, cobre a face da terra e está morando defronte de mim.
\par 6 Vem, pois, agora, rogo-te, amaldiçoa-me este povo, pois é mais poderoso do que eu; para ver se o poderei ferir e lançar fora da terra, porque sei que a quem tu abençoares será abençoado, e a quem tu amaldiçoares será amaldiçoado.
\par 7 Então, foram-se os anciãos dos moabitas e os anciãos dos midianitas, levando consigo o preço dos encantamentos; e chegaram a Balaão e lhe referiram as palavras de Balaque.
\par 8 Balaão lhes disse: Ficai aqui esta noite, e vos trarei a resposta, como o SENHOR me falar; então, os príncipes dos moabitas ficaram com Balaão.
\par 9 Veio Deus a Balaão e disse: Quem são estes homens contigo?
\par 10 Respondeu Balaão a Deus: Balaque, rei dos moabitas, filho de Zipor, os enviou para que me dissessem:
\par 11 Eis que o povo que saiu do Egito cobre a face da terra; vem, agora, amaldiçoa-mo; talvez eu possa combatê-lo e lançá-lo fora.
\par 12 Então, disse Deus a Balaão: Não irás com eles, nem amaldiçoarás o povo; porque é povo abençoado.
\par 13 Levantou-se Balaão pela manhã e disse aos príncipes de Balaque: Tornai à vossa terra, porque o SENHOR recusa deixar-me ir convosco.
\par 14 Tendo-se levantado os príncipes dos moabitas, foram a Balaque e disseram: Balaão recusou vir conosco.
\par 15 De novo, enviou Balaque príncipes, em maior número e mais honrados do que os primeiros,
\par 16 os quais chegaram a Balaão e lhe disseram: Assim diz Balaque, filho de Zipor: Peço-te não te demores em vir a mim,
\par 17 porque grandemente te honrarei e farei tudo o que me disseres; vem, pois, rogo-te, amaldiçoa-me este povo.
\par 18 Respondeu Balaão aos oficiais de Balaque: Ainda que Balaque me desse a sua casa cheia de prata e de ouro, eu não poderia traspassar o mandado do SENHOR, meu Deus, para fazer coisa pequena ou grande;
\par 19 agora, pois, rogo-vos que também aqui fiqueis esta noite, para que eu saiba o que mais o SENHOR me dirá.
\par 20 Veio, pois, o SENHOR a Balaão, de noite, e disse-lhe: Se aqueles homens vieram chamar-te, levanta-te, vai com eles; todavia, farás somente o que eu te disser.
\par 21 Então, Balaão levantou-se pela manhã, albardou a sua jumenta e partiu com os príncipes de Moabe.
\par 22 Acendeu-se a ira de Deus, porque ele se foi; e o Anjo do SENHOR pôs-se-lhe no caminho por adversário. Ora, Balaão ia caminhando, montado na sua jumenta, e dois de seus servos, com ele.
\par 23 Viu, pois, a jumenta o Anjo do SENHOR parado no caminho, com a sua espada desembainhada na mão; pelo que se desviou a jumenta do caminho, indo pelo campo; então, Balaão espancou a jumenta para fazê-la tornar ao caminho.
\par 24 Mas o Anjo do SENHOR pôs-se numa vereda entre as vinhas, havendo muro de um e outro lado.
\par 25 Vendo, pois, a jumenta o Anjo do SENHOR, coseu-se contra o muro e comprimiu contra este o pé de Balaão; por isso, tornou a espancá-la.
\par 26 Então, o Anjo do SENHOR passou mais adiante e pôs-se num lugar estreito, onde não havia caminho para se desviar nem para a direita, nem para a esquerda.
\par 27 Vendo a jumenta o Anjo do SENHOR, deixou-se cair debaixo de Balaão; acendeu-se a ira de Balaão, e espancou a jumenta com a vara.
\par 28 Então, o SENHOR fez falar a jumenta, a qual disse a Balaão: Que te fiz eu, que me espancaste já três vezes?
\par 29 Respondeu Balaão à jumenta: Porque zombaste de mim; tivera eu uma espada na mão e, agora, te mataria.
\par 30 Replicou a jumenta a Balaão: Porventura, não sou a tua jumenta, em que toda a tua vida cavalgaste até hoje? Acaso, tem sido o meu costume fazer assim contigo? Ele respondeu: Não.
\par 31 Então, o SENHOR abriu os olhos a Balaão, ele viu o Anjo do SENHOR, que estava no caminho, com a sua espada desembainhada na mão; pelo que inclinou a cabeça e prostrou-se com o rosto em terra.
\par 32 Então, o Anjo do SENHOR lhe disse: Por que já três vezes espancaste a jumenta? Eis que eu saí como teu adversário, porque o teu caminho é perverso diante de mim;
\par 33 a jumenta me viu e já três vezes se desviou de diante de mim; na verdade, eu, agora, te haveria matado e a ela deixaria com vida.
\par 34 Então, Balaão disse ao Anjo do SENHOR: Pequei, porque não soube que estavas neste caminho para te opores a mim; agora, se parece mal aos teus olhos, voltarei.
\par 35 Tornou o Anjo do SENHOR a Balaão: Vai-te com estes homens; mas somente aquilo que eu te disser, isso falarás. Assim, Balaão se foi com os príncipes de Balaque.
\par 36 Tendo Balaque ouvido que Balaão havia chegado, saiu-lhe ao encontro até à cidade de Moabe, que está nos confins do Arnom e na fronteira extrema.
\par 37 Perguntou Balaque a Balaão: Porventura, não enviei mensageiros a chamar-te? Por que não vieste a mim? Não posso eu, na verdade, honrar-te?
\par 38 Respondeu Balaão a Balaque: Eis-me perante ti; acaso, poderei eu, agora, falar alguma coisa? A palavra que Deus puser na minha boca, essa falarei.
\par 39 Balaão foi com Balaque, e chegaram a Quiriate-Huzote.
\par 40 Então, Balaque sacrificou bois e ovelhas; e deles enviou a Balaão e aos príncipes que estavam com ele.
\par 41 Sucedeu que, pela manhã, Balaque tomou a Balaão e o fez subir a Bamote-Baal; e Balaão viu dali a parte mais próxima do povo.

\chapter{23}

\par 1 Então, Balaão disse a Balaque: Edifica-me, aqui, sete altares e prepara-me sete novilhos e sete carneiros.
\par 2 Fez, pois, Balaque como Balaão dissera; e Balaque e Balaão ofereceram um novilho e um carneiro sobre cada altar.
\par 3 Disse mais Balaão a Balaque: Fica-te junto do teu holocausto, e eu irei; porventura, o SENHOR me sairá ao encontro, e o que me mostrar to notificarei. Então, subiu a um morro desnudo.
\par 4 Encontrando-se Deus com Balaão, este lhe disse: Preparei sete altares e sobre cada um ofereci um novilho e um carneiro.
\par 5 Então, o SENHOR pôs a palavra na boca de Balaão e disse: Torna para Balaque e falarás assim.
\par 6 E, tornando para ele, eis que estava junto do seu holocausto, ele e todos os príncipes dos moabitas.
\par 7 Então, proferiu a sua palavra e disse: Balaque me fez vir de Arã, o rei de Moabe, dos montes do Oriente; vem, amaldiçoa-me a Jacó, e vem, denuncia a Israel.
\par 8 Como posso amaldiçoar a quem Deus não amaldiçoou? Como posso denunciar a quem o SENHOR não denunciou?
\par 9 Pois do cimo das penhas vejo Israel e dos outeiros o contemplo: eis que é povo que habita só e não será reputado entre as nações.
\par 10 Quem contou o pó de Jacó ou enumerou a quarta parte de Israel? Que eu morra a morte dos justos, e o meu fim seja como o dele.
\par 11 Então, disse Balaque a Balaão: Que me fizeste? Chamei-te para amaldiçoar os meus inimigos, mas eis que somente os abençoaste.
\par 12 Mas ele respondeu: Porventura, não terei cuidado de falar o que o SENHOR pôs na minha boca?
\par 13 Então, Balaque lhe disse: Rogo-te que venhas comigo a outro lugar, donde verás o povo; verás somente a parte mais próxima dele e não o verás todo; e amaldiçoa-mo dali.
\par 14 Levou-o consigo ao campo de Zofim, ao cimo de Pisga; e edificou sete altares e sobre cada um ofereceu um novilho e um carneiro.
\par 15 Então, disse Balaão a Balaque: Fica, aqui, junto do teu holocausto, e eu irei ali ao encontro do SENHOR.
\par 16 Encontrando-se o SENHOR com Balaão, pôs-lhe na boca a palavra e disse: Torna para Balaque e assim falarás.
\par 17 Vindo a ele, eis que estava junto do holocausto, e os príncipes dos moabitas, com ele. Perguntou-lhe, pois, Balaque: Que falou o SENHOR?
\par 18 Então, proferiu a sua palavra e disse: Levanta-te, Balaque, e ouve; escuta-me, filho de Zipor:
\par 19 Deus não é homem, para que minta; nem filho de homem, para que se arrependa. Porventura, tendo ele prometido, não o fará? Ou, tendo falado, não o cumprirá?
\par 20 Eis que para abençoar recebi ordem; ele abençoou, não o posso revogar.
\par 21 Não viu iniqüidade em Jacó, nem contemplou desventura em Israel; o SENHOR, seu Deus, está com ele, no meio dele se ouvem aclamações ao seu Rei.
\par 22 Deus os tirou do Egito; as forças deles são como as do boi selvagem.
\par 23 Pois contra Jacó não vale encantamento, nem adivinhação contra Israel; agora, se poderá dizer de Jacó e de Israel: Que coisas tem feito Deus!
\par 24 Eis que o povo se levanta como leoa e se ergue como leão; não se deita até que devore a presa e beba o sangue dos que forem mortos.
\par 25 Então, disse Balaque a Balaão: Nem o amaldiçoarás, nem o abençoarás.
\par 26 Porém Balaão respondeu e disse a Balaque: Não te disse eu: tudo o que o SENHOR falar, isso farei?
\par 27 Disse mais Balaque a Balaão: Ora, vem, e te levarei a outro lugar; porventura, parecerá bem aos olhos de Deus que dali mo amaldiçoes.
\par 28 Então, Balaque levou Balaão consigo ao cimo de Peor, que olha para o lado do deserto.
\par 29 Balaão disse a Balaque: Edifica-me, aqui, sete altares e prepara-me sete novilhos e sete carneiros.
\par 30 Balaque, pois, fez como dissera Balaão e ofereceu sobre cada altar um novilho e um carneiro.

\chapter{24}

\par 1 Vendo Balaão que bem parecia aos olhos do SENHOR que abençoasse a Israel, não foi esta vez, como antes, ao encontro de agouros, mas voltou o rosto para o deserto.
\par 2 Levantando Balaão os olhos e vendo Israel acampado segundo as suas tribos, veio sobre ele o Espírito de Deus.
\par 3 Proferiu a sua palavra e disse: Palavra de Balaão, filho de Beor, palavra do homem de olhos abertos;
\par 4 palavra daquele que ouve os ditos de Deus, o que tem a visão do Todo-Poderoso e prostra-se, porém de olhos abertos:
\par 5 Que boas são as tuas tendas, ó Jacó! Que boas são as tuas moradas, ó Israel!
\par 6 Como vales que se estendem, como jardins à beira dos rios, como árvores de sândalo que o SENHOR plantou, como cedros junto às águas.
\par 7 Águas manarão de seus baldes, e as suas sementeiras terão águas abundantes; o seu rei se levantará mais do que Agague, e o seu reino será exaltado.
\par 8 Deus tirou do Egito a Israel, cujas forças são como as do boi selvagem; consumirá as nações, seus inimigos, e quebrará seus ossos, e, com as suas setas, os atravessará.
\par 9 Este abaixou-se, deitou-se como leão e como leoa; quem o despertará? Benditos os que te abençoarem, e malditos os que te amaldiçoarem.
\par 10 Então, a ira de Balaque se acendeu contra Balaão, e bateu ele as suas palmas. Disse Balaque a Balaão: Chamei-te para amaldiçoares os meus inimigos; porém, agora, já três vezes, somente os abençoaste.
\par 11 Agora, pois, vai-te embora para tua casa; eu dissera que te cumularia de honras; mas eis que o SENHOR te privou delas.
\par 12 Então, Balaão disse a Balaque: Não falei eu também aos teus mensageiros, que me enviaste, dizendo:
\par 13 ainda que Balaque me desse a sua casa cheia de prata e ouro, não poderia traspassar o mandado do SENHOR, fazendo de mim mesmo bem ou mal; o que o SENHOR falar, isso falarei?
\par 14 Agora, eis que vou ao meu povo; vem, avisar-te-ei do que fará este povo ao teu, nos últimos dias.
\par 15 Então, proferiu a sua palavra e disse: Palavra de Balaão, filho de Beor, palavra do homem de olhos abertos,
\par 16 palavra daquele que ouve os ditos de Deus e sabe a ciência do Altíssimo; daquele que tem a visão do Todo-Poderoso e prostra-se, porém de olhos abertos:
\par 17 Vê-lo-ei, mas não agora; contemplá-lo-ei, mas não de perto; uma estrela procederá de Jacó, de Israel subirá um cetro que ferirá as têmporas de Moabe e destruirá todos os filhos de Sete.
\par 18 Edom será uma possessão; Seir, seus inimigos, também será uma possessão; mas Israel fará proezas.
\par 19 De Jacó sairá o dominador e exterminará os que restam das cidades.
\par 20 Viu Balaão a Amaleque, proferiu a sua palavra e disse: Amaleque é o primeiro das nações; porém o seu fim será destruição.
\par 21 Viu os queneus, proferiu a sua palavra e disse: Segura está a tua habitação, e puseste o teu ninho na penha.
\par 22 Todavia, o queneu será consumido. Até quando? Assur te levará cativo.
\par 23 Proferiu ainda a sua palavra e disse: Ai! Quem viverá, quando Deus fizer isto?
\par 24 Homens virão das costas de Quitim em suas naus; afligirão a Assur e a Héber; e também eles mesmos perecerão.
\par 25 Então, Balaão se levantou, e se foi, e voltou para a sua terra; e também Balaque se foi pelo seu caminho.

\chapter{25}

\par 1 Habitando Israel em Sitim, começou o povo a prostituir-se com as filhas dos moabitas.
\par 2 Estas convidaram o povo aos sacrifícios dos seus deuses; e o povo comeu e inclinou-se aos deuses delas.
\par 3 Juntando-se Israel a Baal-Peor, a ira do SENHOR se acendeu contra Israel.
\par 4 Disse o SENHOR a Moisés: Toma todos os cabeças do povo e enforca-os ao SENHOR ao ar livre, e a ardente ira do SENHOR se retirará de Israel.
\par 5 Então, Moisés disse aos juízes de Israel: Cada um mate os homens da sua tribo que se juntaram a Baal-Peor.
\par 6 Eis que um homem dos filhos de Israel veio e trouxe a seus irmãos uma midianita perante os olhos de Moisés e de toda a congregação dos filhos de Israel, enquanto eles choravam diante da tenda da congregação.
\par 7 Vendo isso Finéias, filho de Eleazar, o filho de Arão, o sacerdote, levantou-se do meio da congregação, e, pegando uma lança,
\par 8 foi após o homem israelita até ao interior da tenda, e os atravessou, ao homem israelita e à mulher, a ambos pelo ventre; então, a praga cessou de sobre os filhos de Israel.
\par 9 Os que morreram da praga foram vinte e quatro mil.
\par 10 Então, disse o SENHOR a Moisés:
\par 11 Finéias, filho de Eleazar, filho de Arão, o sacerdote, desviou a minha ira de sobre os filhos de Israel, pois estava animado com o meu zelo entre eles; de sorte que, no meu zelo, não consumi os filhos de Israel.
\par 12 Portanto, dize: Eis que lhe dou a minha aliança de paz.
\par 13 E ele e a sua descendência depois dele terão a aliança do sacerdócio perpétuo; porquanto teve zelo pelo seu Deus e fez expiação pelos filhos de Israel.
\par 14 O nome do israelita que foi morto (morto com a midianita) era Zinri, filho de Salu, príncipe da casa paterna dos simeonitas.
\par 15 O nome da mulher midianita que foi morta era Cosbi, filha de Zur, cabeça do povo da casa paterna entre os midianitas.
\par 16 Disse mais o SENHOR a Moisés:
\par 17 Afligireis os midianitas e os ferireis,
\par 18 porque eles vos afligiram a vós outros quando vos enganaram no caso de Peor e no caso de Cosbi, filha do príncipe dos midianitas, irmã deles, que foi morta no dia da praga no caso de Peor.

\chapter{26}

\par 1 Passada a praga, falou o SENHOR a Moisés e a Eleazar, filho de Arão, o sacerdote, dizendo:
\par 2 Levantai o censo de toda a congregação dos filhos de Israel, da idade de vinte anos para cima, segundo as casas de seus pais, todo que, em Israel, for capaz de sair à guerra.
\par 3 Moisés e Eleazar, o sacerdote, pois, nas campinas de Moabe, ao pé do Jordão, na altura de Jericó, falaram aos cabeças de Israel, dizendo:
\par 4 Contai o povo da idade de vinte anos para cima, como o SENHOR ordenara a Moisés e aos filhos de Israel que saíram do Egito:
\par 5 Rúben, o primogênito de Israel; os filhos de Rúben: de Enoque, a família dos enoquitas; de Palu, a família dos paluítas;
\par 6 de Hezrom, a família dos hezronitas; de Carmi, a família dos carmitas.
\par 7 São estas as famílias dos rubenitas; os que foram deles contados foram quarenta e três mil e setecentos e trinta.
\par 8 O filho de Palu: Eliabe.
\par 9 Os filhos de Eliabe: Nemuel, Datã e Abirão; estes, Datã e Abirão, são os que foram eleitos pela congregação, os quais moveram a contenda contra Moisés e contra Arão, no grupo de Corá, quando moveram a contenda contra o SENHOR;
\par 10 quando a terra abriu a boca e os tragou com Corá, morrendo aquele grupo; quando o fogo consumiu duzentos e cinqüenta homens, e isso serviu de advertência.
\par 11 Mas os filhos de Corá não morreram.
\par 12 Os filhos de Simeão, segundo as suas famílias: de Nemuel, a família dos nemuelitas; de Jamim, a família dos jaminitas; de Jaquim, a família dos jaquinitas;
\par 13 de Zera, a família dos zeraítas; de Saul, a família dos saulitas.
\par 14 São estas as famílias dos simeonitas, num total de vinte e dois mil e duzentos.
\par 15 Os filhos de Gade, segundo as suas famílias: de Zefom, a família dos zefonitas; de Hagi, a família dos hagitas; de Suni, a família dos sunitas;
\par 16 de Ozni, a família dos oznitas; de Eri, a família dos eritas;
\par 17 de Arodi, a família dos aroditas; de Areli, a família dos arelitas.
\par 18 São estas as famílias dos filhos de Gade, segundo os que foram deles contados, num total de quarenta mil e quinhentos.
\par 19 Os filhos de Judá: Er e Onã; mas Er e Onã morreram na terra de Canaã.
\par 20 Assim, os filhos de Judá foram, segundo as suas famílias: de Selá, a família dos selaítas; de Perez, a família dos perezitas; de Zera, a família dos zeraítas.
\par 21 Os filhos de Perez foram: de Hezrom, a família dos hezronitas; de Hamul, a família dos hamulitas.
\par 22 São estas as famílias de Judá, segundo os que foram deles contados, num total de setenta e seis mil e quinhentos.
\par 23 Os filhos de Issacar, segundo as suas famílias, foram: de Tola, a família dos tolaítas; de Puva, a família dos puvitas;
\par 24 de Jasube, a família dos jasubitas; de Sinrom, a família dos sinronitas.
\par 25 São estas as famílias de Issacar, segundo os que foram deles contados, num total de sessenta e quatro mil e trezentos.
\par 26 Os filhos de Zebulom, segundo a suas famílias, foram: de Serede, a família dos sereditas; de Elom, a família dos elonitas, de Jaleel, a família dos jaleelitas.
\par 27 São estas as famílias dos zebulonitas, segundo os que foram deles contados, num total de sessenta mil e quinhentos.
\par 28 Os filhos de José, segundo as suas famílias, foram Manassés e Efraim.
\par 29 Os filhos de Manassés foram: de Maquir, a família dos maquiritas; e Maquir gerou a Gileade; de Gileade, a família dos gileaditas.
\par 30 São estes os filhos de Gileade: de Jezer, a família dos jezeritas; de Heleque, a família dos helequitas;
\par 31 de Asriel, a família dos asrielitas; de Siquém, a família dos siquemitas.
\par 32 De Semida, a família dos semidaítas; de Héfer, a família dos heferitas.
\par 33 Porém Zelofeade, filho de Héfer, não tinha filhos, senão filhas; os nomes das filhas de Zelofeade foram: Macla, Noa, Hogla, Milca e Tirza.
\par 34 São estas as famílias de Manassés; os que foram deles contados foram cinqüenta e dois mil e setecentos.
\par 35 São estes os filhos de Efraim, segundo as suas famílias: de Sutela, a família dos sutelaítas; de Bequer, a família dos bequeritas; de Taã, a família dos taanitas.
\par 36 De Erã, filho de Sutela: de Erã, a família dos eranitas.
\par 37 São estas as famílias dos filhos de Efraim, segundo os que foram deles contados, num total de trinta e dois mil e quinhentos. São estes os filhos de José, segundo as suas famílias.
\par 38 Os filhos de Benjamim, segundo as suas famílias: de Belá, a família dos belaítas; de Asbel, a família dos asbelitas; de Airão, a família dos airamitas;
\par 39 de Sufã, a família dos sufamitas; de Hufã, a família dos hufamitas.
\par 40 Os filhos de Belá foram: Arde e Naamã; de Arde, a família dos arditas; de Naamã, a família dos naamanitas.
\par 41 São estes os filhos de Benjamim, segundo as suas famílias; os que foram deles contados foram quarenta e cinco mil e seiscentos.
\par 42 São estes os filhos de Dã, segundo as suas famílias: de Suão, a família dos suamitas. São estas as famílias de Dã, segundo as suas famílias.
\par 43 Todas as famílias dos suamitas, segundo os que foram deles contados, tinham sessenta e quatro mil e quatrocentos.
\par 44 Os filhos de Aser, segundo as suas famílias: de Imna, a família dos imnaítas; de Isvi, a família dos isvitas; de Berias, a família dos beriaítas.
\par 45 Os filhos de Berias foram: de Héber, a família dos heberitas; de Malquiel, a família dos malquielitas.
\par 46 O nome da filha de Aser foi Sera.
\par 47 São estas as famílias dos filhos de Aser, segundo os que foram deles contados, num total de cinqüenta e três mil e quatrocentos.
\par 48 Os filhos de Naftali, segundo as suas famílias: de Jazeel, a família dos jazeelitas; de Guni, a família dos gunitas;
\par 49 de Jezer, a família dos jezeritas; de Silém, a família dos silemitas.
\par 50 São estas as famílias de Naftali, segundo as suas famílias; os que foram deles contados, foram quarenta e cinco mil e quatrocentos.
\par 51 São estes os contados dos filhos de Israel: seiscentos e um mil setecentos e trinta.
\par 52 Disse o SENHOR a Moisés:
\par 53 A estes se repartirá a terra em herança, segundo o censo.
\par 54 À tribo mais numerosa darás herança maior, à pequena, herança menor; a cada uma, em proporção ao seu número, se dará a herança.
\par 55 Todavia, a terra se repartirá por sortes; segundo os nomes das tribos de seus pais, a herdarão.
\par 56 Segundo a sorte, repartir-se-á a herança deles entre as tribos maiores e menores.
\par 57 São estes os que foram contados dos levitas, segundo as suas famílias: de Gérson, a família dos gersonitas; de Coate, a família dos coatitas; de Merari, a família dos meraritas.
\par 58 São estas as famílias de Levi: a família dos libnitas, a família dos hebronitas, a família dos malitas, a família dos musitas, a família dos coraítas. Coate gerou a Anrão.
\par 59 A mulher de Anrão chamava-se Joquebede, filha de Levi, a qual lhe nasceu no Egito; teve ela, de Anrão, a Arão, e a Moisés, e a Miriã, irmã deles.
\par 60 A Arão nasceram Nadabe, Abiú, Eleazar e Itamar.
\par 61 Nadabe e Abiú morreram quando levaram fogo estranho perante o SENHOR.
\par 62 Os que foram deles contados foram vinte e três mil, todo homem da idade de um mês para cima; porque estes não foram contados entre os filhos de Israel, porquanto lhes não foi dada herança com os outros.
\par 63 São estes os que foram contados por Moisés e o sacerdote Eleazar, que contaram os filhos de Israel nas campinas de Moabe, ao pé do Jordão, na altura de Jericó.
\par 64 Entre estes, porém, nenhum houve dos que foram contados por Moisés e pelo sacerdote Arão, quando levantaram o censo dos filhos de Israel no deserto do Sinai.
\par 65 Porque o SENHOR dissera deles que morreriam no deserto; e nenhum deles ficou, senão Calebe, filho de Jefoné, e Josué, filho de Num.

\chapter{27}

\par 1 Então, vieram as filhas de Zelofeade, filho de Héfer, filho de Gileade, filho de Maquir, filho de Manassés, entre as famílias de Manassés, filho de José. São estes os nomes de suas filhas: Macla, Noa, Hogla, Milca e Tirza.
\par 2 Apresentaram-se diante de Moisés, e diante de Eleazar, o sacerdote, e diante dos príncipes, e diante de todo o povo, à porta da tenda da congregação, dizendo:
\par 3 Nosso pai morreu no deserto e não estava entre os que se ajuntaram contra o SENHOR no grupo de Corá; mas morreu no seu próprio pecado e não teve filhos.
\par 4 Por que se tiraria o nome de nosso pai do meio da sua família, porquanto não teve filhos? Dá-nos possessão entre os irmãos de nosso pai.
\par 5 Moisés levou a causa delas perante o SENHOR.
\par 6 Disse o SENHOR a Moisés:
\par 7 As filhas de Zelofeade falam o que é justo; certamente, lhes darás possessão de herança entre os irmãos de seu pai e farás passar a elas a herança de seu pai.
\par 8 Falarás aos filhos de Israel, dizendo: Quando alguém morrer e não tiver filho, então, fareis passar a sua herança a sua filha.
\par 9 E, se não tiver filha, então, a sua herança dareis aos irmãos dele.
\par 10 Porém, se não tiver irmãos, dareis a sua herança aos irmãos de seu pai.
\par 11 Se também seu pai não tiver irmãos, dareis a sua herança ao parente mais chegado de sua família, para que a possua; isto aos filhos de Israel será prescrição de direito, como o SENHOR ordenou a Moisés.
\par 12 Depois, disse o SENHOR a Moisés: Sobe a este monte Abarim e vê a terra que dei aos filhos de Israel.
\par 13 E, tendo-a visto, serás recolhido também ao teu povo, assim como o foi teu irmão Arão;
\par 14 porquanto, no deserto de Zim, na contenda da congregação, fostes rebeldes ao meu mandado de me santificar nas águas diante dos seus olhos. São estas as águas de Meribá de Cades, no deserto de Zim.
\par 15 Então, disse Moisés ao SENHOR:
\par 16 O SENHOR, autor e conservador de toda vida, ponha um homem sobre esta congregação
\par 17 que saia adiante deles, e que entre adiante deles, e que os faça sair, e que os faça entrar, para que a congregação do SENHOR não seja como ovelhas que não têm pastor.
\par 18 Disse o SENHOR a Moisés: Toma Josué, filho de Num, homem em quem há o Espírito, e impõe-lhe as mãos;
\par 19 apresenta-o perante Eleazar, o sacerdote, e perante toda a congregação; e dá-lhe, à vista deles, as tuas ordens.
\par 20 Põe sobre ele da tua autoridade, para que lhe obedeça toda a congregação dos filhos de Israel.
\par 21 Apresentar-se-á perante Eleazar, o sacerdote, o qual por ele consultará, segundo o juízo do Urim, perante o SENHOR; segundo a sua palavra, sairão e, segundo a sua palavra, entrarão, ele, e todos os filhos de Israel com ele, e toda a congregação.
\par 22 Fez Moisés como lhe ordenara o SENHOR, porque tomou a Josué e apresentou-o perante Eleazar, o sacerdote, e perante toda a congregação;
\par 23 e lhe impôs as mãos e lhe deu as suas ordens, como o SENHOR falara por intermédio de Moisés.

\chapter{28}

\par 1 Disse mais o SENHOR a Moisés:
\par 2 Dá ordem aos filhos de Israel e dize-lhes: Da minha oferta, do meu manjar para as minhas ofertas queimadas, do aroma agradável, tereis cuidado, para mas trazer a seu tempo determinado.
\par 3 Dir-lhes-ás: Esta é a oferta queimada que oferecereis ao SENHOR, dia após dia: dois cordeiros de um ano, sem defeito, em contínuo holocausto;
\par 4 um cordeiro oferecerás pela manhã, e o outro, ao crepúsculo da tarde;
\par 5 e a décima parte de um efa de flor de farinha, em oferta de manjares, amassada com a quarta parte de um him de azeite batido.
\par 6 É holocausto contínuo, instituído no monte Sinai, de aroma agradável, oferta queimada ao SENHOR.
\par 7 A sua libação será a quarta parte de um him para o cordeiro; no santuário, oferecerás a libação de bebida forte ao SENHOR.
\par 8 E o outro cordeiro oferecerás no crepúsculo da tarde; como a oferta de manjares da manhã e como a sua libação, o trarás em oferta queimada de aroma agradável ao SENHOR.
\par 9 No dia de sábado, oferecerás dois cordeiros de um ano, sem defeito, e duas décimas de um efa de flor de farinha, amassada com azeite, em oferta de manjares, e a sua libação;
\par 10 é holocausto de cada sábado, além do holocausto contínuo e a sua libação.
\par 11 Nos princípios dos vossos meses, oferecereis, em holocausto ao SENHOR, dois novilhos e um carneiro, sete cordeiros de um ano, sem defeito,
\par 12 e três décimas de um efa de flor de farinha, amassada com azeite, em oferta de manjares, para um novilho; duas décimas de flor de farinha, amassada com azeite, em oferta de manjares, para um carneiro;
\par 13 e uma décima de um efa de flor de farinha, amassada com azeite, em oferta de manjares, para um cordeiro; é holocausto de aroma agradável, oferta queimada ao SENHOR.
\par 14 As suas libações serão a metade de um him de vinho para um novilho, e a terça parte de um him para um carneiro, e a quarta parte de um him para um cordeiro; este é o holocausto da lua nova de cada mês, por todos os meses do ano.
\par 15 Também se trará um bode como oferta pelo pecado, ao SENHOR, além do holocausto contínuo, com a sua libação.
\par 16 No primeiro mês, aos catorze dias do mês, é a Páscoa do SENHOR.
\par 17 Aos quinze dias do mesmo mês, haverá festa; sete dias se comerão pães asmos.
\par 18 No primeiro dia, haverá santa convocação; nenhuma obra servil fareis;
\par 19 mas apresentareis oferta queimada em holocausto ao SENHOR, dois novilhos, um carneiro e sete cordeiros de um ano; ser-vos-ão eles sem defeito.
\par 20 A sua oferta de manjares será flor de farinha, amassada com azeite; oferecereis três décimas para um novilho e duas décimas para um carneiro.
\par 21 Para cada um dos sete cordeiros oferecereis uma décima;
\par 22 e um bode, para oferta pelo pecado, para fazer expiação por vós.
\par 23 Estas coisas oferecereis, além do holocausto da manhã, que é o holocausto contínuo.
\par 24 Assim, oferecereis cada dia, por sete dias, o manjar da oferta queimada em aroma agradável ao SENHOR; além do holocausto contínuo, se oferecerá isto com a sua libação.
\par 25 No sétimo dia, tereis santa convocação; nenhuma obra servil fareis.
\par 26 Também tereis santa convocação no dia das primícias, quando trouxerdes oferta nova de manjares ao SENHOR, segundo a vossa Festa das Semanas; nenhuma obra servil fareis.
\par 27 Então, oferecereis ao SENHOR por holocausto, em aroma agradável: dois novilhos, um carneiro e sete cordeiros de um ano;
\par 28 a sua oferta de manjares de flor de farinha, amassada com azeite: três décimas de um efa para um novilho, duas décimas para um carneiro,
\par 29 uma décima para cada um dos sete cordeiros;
\par 30 e um bode, para fazer expiação por vós.
\par 31 Oferecê-los-eis, além do holocausto contínuo, e da sua oferta de manjares, e das suas libações. Ser-vos-ão eles sem defeito.

\chapter{29}

\par 1 No primeiro dia do sétimo mês, tereis santa convocação; nenhuma obra servil fareis; ser-vos-á dia do sonido de trombetas.
\par 2 Então, por holocausto, de aroma agradável ao SENHOR, oferecereis um novilho, um carneiro e sete cordeiros de um ano, sem defeito;
\par 3 e, pela sua oferta de manjares de flor de farinha, amassada com azeite, três décimas de um efa para o novilho, duas décimas para o carneiro
\par 4 e uma décima para cada um dos sete cordeiros;
\par 5 e um bode, para oferta pelo pecado, para fazer expiação por vós,
\par 6 além do holocausto do mês e a sua oferta de manjares, do holocausto contínuo e a sua oferta de manjares, com as suas libações, segundo o seu estatuto, em aroma agradável, oferta queimada ao SENHOR.
\par 7 No dia dez deste sétimo mês, tereis santa convocação e afligireis a vossa alma; nenhuma obra fareis.
\par 8 Mas, por holocausto, em aroma agradável ao SENHOR, oferecereis um novilho, um carneiro e sete cordeiros de um ano; ser-vos-ão eles sem defeito.
\par 9 Pela sua oferta de manjares de flor de farinha, amassada com azeite, oferecereis três décimas de um efa para o novilho, duas décimas para o carneiro
\par 10 e uma décima para cada um dos sete cordeiros;
\par 11 um bode, para oferta pelo pecado, além da oferta pelo pecado, para fazer expiação, e do holocausto contínuo, e da sua oferta de manjares com as suas libações.
\par 12 Aos quinze dias do sétimo mês, tereis santa convocação; nenhuma obra servil fareis; mas sete dias celebrareis festa ao SENHOR.
\par 13 Por holocausto em oferta queimada, de aroma agradável ao SENHOR, oferecereis treze novilhos, dois carneiros e catorze cordeiros de um ano; serão eles sem defeito.
\par 14 Pela oferta de manjares de flor de farinha, amassada com azeite, três décimas de um efa para cada um dos treze novilhos, duas décimas para cada um dos dois carneiros
\par 15 e uma décima para cada um dos catorze cordeiros;
\par 16 e um bode, para oferta pelo pecado, além do holocausto contínuo, a sua oferta de manjares e a sua libação.
\par 17 No segundo dia, oferecereis doze novilhos, dois carneiros, catorze cordeiros de um ano, sem defeito,
\par 18 com a oferta de manjares e as libações para os novilhos, para os carneiros e para os cordeiros, conforme o seu número, segundo o estatuto,
\par 19 e um bode, para oferta pelo pecado, além do holocausto contínuo, a sua oferta de manjares e a sua libação.
\par 20 No terceiro dia, oferecereis onze novilhos, dois carneiros, catorze cordeiros de um ano, sem defeito,
\par 21 com a oferta de manjares e as libações para os novilhos, para os carneiros e para os cordeiros, conforme o seu número, segundo o estatuto,
\par 22 e um bode, para oferta pelo pecado, além do holocausto contínuo, a sua oferta de manjares e a sua libação.
\par 23 No quarto dia, dez novilhos, dois carneiros, catorze cordeiros de um ano, sem defeito,
\par 24 com a oferta de manjares e as libações para os novilhos, para os carneiros e para os cordeiros, conforme o seu número, segundo o estatuto,
\par 25 e um bode, para oferta pelo pecado, além do holocausto contínuo, a sua oferta de manjares e a sua libação.
\par 26 No quinto dia, nove novilhos, dois carneiros, catorze cordeiros de um ano, sem defeito,
\par 27 com a oferta de manjares e as libações para os novilhos, para os carneiros e para os cordeiros, conforme o seu número, segundo o estatuto,
\par 28 e um bode, para oferta pelo pecado, além do holocausto contínuo, a sua oferta de manjares e a sua libação.
\par 29 No sexto dia, oito novilhos, dois carneiros, catorze cordeiros de um ano, sem defeito,
\par 30 com a oferta de manjares e as libações para os novilhos, para os carneiros e para os cordeiros, conforme o seu número, segundo o estatuto,
\par 31 e um bode, para oferta pelo pecado, além do holocausto contínuo, a sua oferta de manjares e a sua libação.
\par 32 No sétimo dia, sete novilhos, dois carneiros, catorze cordeiros de um ano, sem defeito,
\par 33 com a oferta de manjares e as libações para os novilhos, para os carneiros e para os cordeiros, conforme o seu número, segundo o estatuto,
\par 34 e um bode, para oferta pelo pecado, além do holocausto contínuo, a sua oferta de manjares e a sua libação.
\par 35 No oitavo dia, tereis reunião solene; nenhuma obra servil fareis;
\par 36 e, por holocausto, em oferta queimada de aroma agradável ao SENHOR, oferecereis um novilho, um carneiro, sete cordeiros de um ano, sem defeito,
\par 37 com a oferta de manjares e as libações para o novilho, para o carneiro e para os cordeiros, conforme o seu número, segundo o estatuto,
\par 38 e um bode, para oferta pelo pecado, além do holocausto contínuo, a sua oferta de manjares e a sua libação.
\par 39 Estas coisas oferecereis ao SENHOR nas vossas festas fixas, além dos vossos votos e das vossas ofertas voluntárias, para os vossos holocaustos, as vossas ofertas de manjares, as vossas libações e as vossas ofertas pacíficas.
\par 40 E falou Moisés aos filhos de Israel, conforme tudo o que o SENHOR lhe ordenara.

\chapter{30}

\par 1 Falou Moisés aos cabeças das tribos dos filhos de Israel, dizendo: Esta é a palavra que o SENHOR ordenou:
\par 2 Quando um homem fizer voto ao SENHOR ou juramento para obrigar-se a alguma abstinência, não violará a sua palavra; segundo tudo o que prometeu, fará.
\par 3 Quando, porém, uma mulher fizer voto ao SENHOR ou se obrigar a alguma abstinência, estando em casa de seu pai, na sua mocidade,
\par 4 e seu pai, sabendo do voto e da abstinência a que ela se obrigou, calar-se para com ela, todos os seus votos serão válidos; terá de observar toda a abstinência a que se obrigou.
\par 5 Mas, se o pai, no dia em que tal souber, o desaprovar, não será válido nenhum dos votos dela, nem lhe será preciso observar a abstinência a que se obrigou; o SENHOR lhe perdoará, porque o pai dela a isso se opôs.
\par 6 Porém, se ela se casar, ainda sob seus votos ou dito irrefletido dos seus lábios, com que a si mesma se obrigou,
\par 7 e seu marido, ouvindo-o, calar-se para com ela no dia em que o ouvir, serão válidos os votos dela, e lhe será preciso observar a abstinência a que se obrigou.
\par 8 Mas, se seu marido o desaprovar no dia em que o ouvir e anular o voto que estava sobre ela, como também o dito irrefletido dos seus lábios, com que a si mesma se obrigou, o SENHOR lho perdoará.
\par 9 No tocante ao voto da viúva ou da divorciada, tudo com que se obrigar lhe será válido.
\par 10 Porém, se fez voto na casa de seu marido ou com juramento se obrigou a alguma abstinência,
\par 11 e seu marido o soube, e se calou para com ela, e lho não desaprovou, todos os votos dela serão válidos; e lhe será preciso observar toda a abstinência a que a si mesma se obrigou.
\par 12 Porém, se seu marido lhos anulou no dia em que o soube, tudo quanto saiu dos lábios dela, quer dos seus votos, quer da abstinência a que a si mesma se obrigou, não será válido; seu marido lhos anulou, e o SENHOR perdoará a ela.
\par 13 Todo voto e todo juramento com que ela se obrigou, para afligir a sua alma, seu marido pode confirmar ou anular.
\par 14 Porém, se seu marido, dia após dia, se calar para com ela, então, confirma todos os votos dela e tudo aquilo a que ela se obrigou, porquanto se calou para com ela no dia em que o soube.
\par 15 Porém, se lhos anular depois de os ter ouvido, responderá pela obrigação dela.
\par 16 São estes os estatutos que o SENHOR ordenou a Moisés, entre o marido e sua mulher, entre o pai e sua filha moça se ela estiver em casa de seu pai.

\chapter{31}

\par 1 Disse o SENHOR a Moisés:
\par 2 Vinga os filhos de Israel dos midianitas; depois, serás recolhido ao teu povo.
\par 3 Falou, pois, Moisés ao povo, dizendo: Armai alguns de vós para a guerra, e que saiam contra os midianitas, para fazerem a vingança do SENHOR contra eles.
\par 4 Mil homens de cada tribo entre todas as tribos de Israel enviareis à guerra.
\par 5 Assim, dos milhares de Israel foram dados mil de cada tribo: doze mil ao todo, armados para a guerra.
\par 6 Mandou-os Moisés à guerra, de cada tribo mil, a estes e a Finéias, filho do sacerdote Eleazar, o qual levava consigo os utensílios sagrados, a saber, as trombetas para o toque de rebate.
\par 7 Pelejaram contra os midianitas, como o SENHOR ordenara a Moisés,
\par 8 e mataram todo homem feito. Mataram, além dos que já haviam sido mortos, os reis dos midianitas, Evi, Requém, Zur, Hur e Reba, cinco reis dos midianitas; também Balaão, filho de Beor, mataram à espada.
\par 9 Porém os filhos de Israel levaram presas as mulheres dos midianitas e as suas crianças; também levaram todos os seus animais, e todo o seu gado, e todos os seus bens.
\par 10 Queimaram-lhes todas as cidades em que habitavam e todos os seus acampamentos.
\par 11 Tomaram todo o despojo e toda a presa, tanto de homens como de animais.
\par 12 Trouxeram a Moisés, e ao sacerdote Eleazar, e à congregação dos filhos de Israel os cativos, e a presa, e o despojo, para o arraial, nas campinas de Moabe, junto do Jordão, na altura de Jericó.
\par 13 Moisés, e Eleazar, o sacerdote, e todos os príncipes da congregação saíram a recebê-los fora do arraial.
\par 14 Indignou-se Moisés contra os oficiais do exército, capitães dos milhares e capitães das centenas, que vinham do serviço da guerra.
\par 15 Disse-lhes Moisés: Deixastes viver todas as mulheres?
\par 16 Eis que estas, por conselho de Balaão, fizeram prevaricar os filhos de Israel contra o SENHOR, no caso de Peor, pelo que houve a praga entre a congregação do SENHOR.
\par 17 Agora, pois, matai, dentre as crianças, todas as do sexo masculino; e matai toda mulher que coabitou com algum homem, deitando-se com ele.
\par 18 Porém todas as meninas, e as jovens que não coabitaram com algum homem, deitando-se com ele, deixai-as viver para vós outros.
\par 19 Acampai-vos sete dias fora do arraial; qualquer de vós que tiver matado alguma pessoa e qualquer que tiver tocado em algum morto, ao terceiro dia e ao sétimo dia, vos purificareis, tanto vós como os vossos cativos.
\par 20 Também purificareis toda veste, e toda obra de peles, e toda obra de pêlos de cabra, e todo artigo de madeira.
\par 21 Então, disse o sacerdote Eleazar aos homens do exército que partiram à guerra: Este é o estatuto da lei que o SENHOR ordenou a Moisés.
\par 22 Contudo, o ouro, a prata, o bronze, o ferro, o estanho e o chumbo,
\par 23 tudo o que pode suportar o fogo fareis passar pelo fogo, para que fique limpo; todavia, se purificará com a água purificadora; mas tudo o que não pode suportar o fogo fareis passar pela água.
\par 24 Também lavareis as vossas vestes ao sétimo dia, para que fiqueis limpos; e, depois, entrareis no arraial.
\par 25 Disse mais o SENHOR a Moisés:
\par 26 Faze a contagem da presa que foi tomada, tanto de homens como de animais, tu, e Eleazar, o sacerdote, e os cabeças das casas dos pais da congregação;
\par 27 divide a presa em duas partes iguais, uma para os que, hábeis na peleja, saíram à guerra, e a outra para toda a congregação.
\par 28 Então, para o SENHOR tomarás tributo dos homens do exército que saíram a esta guerra, de cada quinhentas cabeças, uma, tanto dos homens como dos bois, dos jumentos e das ovelhas.
\par 29 Da metade que lhes toca o tomareis e o dareis ao sacerdote Eleazar, para a oferta do SENHOR.
\par 30 Mas, da metade que toca aos filhos de Israel, tomarás, de cada cinqüenta, um, tanto dos homens como dos bois, dos jumentos e das ovelhas, de todos os animais; e os darás aos levitas que têm a seu cargo o serviço do tabernáculo do SENHOR.
\par 31 Moisés e o sacerdote Eleazar fizeram como o SENHOR ordenara a Moisés.
\par 32 Foi a presa, restante do despojo que tomaram os homens de guerra, seiscentas e setenta e cinco mil ovelhas,
\par 33 setenta e dois mil bois,
\par 34 sessenta e um mil jumentos
\par 35 e trinta e duas mil pessoas, as mulheres que não coabitaram com homem algum, deitando-se com ele.
\par 36 E a metade, parte que toca aos que saíram à guerra, foi em número de trezentas e trinta e sete mil e quinhentas ovelhas.
\par 37 O tributo em ovelhas para o SENHOR foram seiscentas e setenta e cinco.
\par 38 E foram os bois trinta e seis mil; e o seu tributo para o SENHOR, setenta e dois.
\par 39 E foram os jumentos trinta mil e quinhentos; e o seu tributo para o SENHOR, sessenta e um.
\par 40 As pessoas foram dezesseis mil; e o seu tributo para o SENHOR, trinta e duas.
\par 41 Então, Moisés deu a Eleazar, o sacerdote, o tributo da oferta do SENHOR, como este ordenara a Moisés.
\par 42 E, da metade que toca aos filhos de Israel, que Moisés separara da dos homens que pelejaram
\par 43 (a metade para a congregação foram, em ovelhas, trezentas e trinta e sete mil e quinhentas;
\par 44 em bois, trinta e seis mil;
\par 45 em jumentos, trinta mil e quinhentos;
\par 46 e, em pessoas, dezesseis mil),
\par 47 desta metade que toca aos filhos de Israel, Moisés tomou um de cada cinqüenta, tanto de homens como de animais, e os deu aos levitas que tinham a seu cargo o serviço do tabernáculo do SENHOR, como o SENHOR ordenara a Moisés.
\par 48 Então, se chegaram a Moisés os oficiais sobre os milhares do exército, capitães sobre mil e capitães sobre cem,
\par 49 e lhe disseram: Teus servos fizeram a conta dos homens de guerra que estiveram sob as nossas ordens, e nenhum falta dentre eles e nós.
\par 50 Pelo que trouxemos uma oferta ao SENHOR, cada um o que achou: objetos de ouro, ornamentos para o braço, pulseiras, sinetes, arrecadas e colares, para fazer expiação por nós mesmos perante o SENHOR.
\par 51 Assim, Moisés e o sacerdote Eleazar receberam deles o ouro, sendo todos os objetos bem trabalhados.
\par 52 Foi todo o ouro da oferta que os capitães de mil e os capitães de cem trouxeram ao SENHOR dezesseis mil setecentos e cinqüenta siclos.
\par 53 Pois cada um dos homens de guerra havia tomado despojo para si.
\par 54 Moisés e o sacerdote Eleazar receberam o ouro dos capitães de mil e dos capitães de cem e o trouxeram à tenda da congregação, como memorial para os filhos de Israel perante o SENHOR.

\chapter{32}

\par 1 Os filhos de Rúben e os filhos de Gade tinham gado em muitíssima quantidade; e viram a terra de Jazer e a terra de Gileade, e eis que o lugar era lugar de gado.
\par 2 Vieram, pois, os filhos de Gade e os filhos de Rúben e falaram a Moisés, e ao sacerdote Eleazar, e aos príncipes da congregação, dizendo:
\par 3 Atarote, Dibom, Jazer, Ninra, Hesbom, Eleale, Sebã, Nebo e Beom,
\par 4 a terra que o SENHOR feriu diante da congregação de Israel é terra de gado; e os teus servos têm gado.
\par 5 Disseram mais: Se achamos mercê aos teus olhos, dê-se esta terra em possessão aos teus servos; e não nos faças passar o Jordão.
\par 6 Porém Moisés disse ao filhos de Gade e aos filhos de Rúben: Irão vossos irmãos à guerra, e ficareis vós aqui?
\par 7 Por que, pois, desanimais o coração dos filhos de Israel, para que não passem à terra que o SENHOR lhes deu?
\par 8 Assim fizeram vossos pais, quando os enviei de Cades-Barnéia a ver esta terra.
\par 9 Chegando eles até ao vale de Escol e vendo a terra, descorajaram o coração dos filhos de Israel, para que não viessem à terra que o SENHOR lhes tinha dado.
\par 10 Então, a ira do SENHOR se acendeu naquele mesmo dia, e jurou, dizendo:
\par 11 Certamente, os varões que subiram do Egito, de vinte anos para cima, não verão a terra que prometi com juramento a Abraão, a Isaque e a Jacó, porquanto não perseveraram em seguir-me,
\par 12 exceto Calebe, filho de Jefoné, o quenezeu, e Josué, filho de Num, porque perseveraram em seguir ao SENHOR.
\par 13 Pelo que se acendeu a ira do SENHOR contra Israel, e fê-los andar errantes pelo deserto quarenta anos, até que se consumiu toda a geração que procedera mal perante o SENHOR.
\par 14 Eis que vós, raça de homens pecadores, vos levantastes em lugar de vossos pais, para aumentardes ainda o furor da ira do SENHOR contra Israel.
\par 15 Se não quiserdes segui-lo, também ele deixará todo o povo, novamente, no deserto, e sereis a sua ruína.
\par 16 Então, se chegaram a ele e disseram: Edificaremos currais aqui para o nosso gado e cidades para as nossas crianças;
\par 17 porém nós nos armaremos, apressando-nos adiante dos filhos de Israel, até que os levemos ao seu lugar; e ficarão as nossas crianças nas cidades fortes, por causa dos moradores da terra.
\par 18 Não voltaremos para nossa casa até que os filhos de Israel estejam de posse, cada um, da sua herança.
\par 19 Porque não herdaremos com eles do outro lado do Jordão, nem mais adiante, porquanto já temos a nossa herança deste lado do Jordão, ao oriente.
\par 20 Então, Moisés lhes disse: Se isto fizerdes assim, se vos armardes para a guerra perante o SENHOR,
\par 21 e cada um de vós, armado, passar o Jordão perante o SENHOR, até que haja lançado fora os seus inimigos de diante dele,
\par 22 e a terra estiver subjugada perante o SENHOR, então, voltareis e sereis desobrigados perante o SENHOR e perante Israel; e a terra vos será por possessão perante o SENHOR.
\par 23 Porém, se não fizerdes assim, eis que pecastes contra o SENHOR; e sabei que o vosso pecado vos há de achar.
\par 24 Edificai vós cidades para as vossas crianças e currais para as vossas ovelhas; e cumpri o que haveis prometido.
\par 25 Então, os filhos de Gade e os filhos de Rúben falaram a Moisés, dizendo: Como ordena meu senhor, assim farão teus servos.
\par 26 Nossas crianças, nossas mulheres, nossos rebanhos e todos os nossos animais estarão aí nas cidades de Gileade,
\par 27 mas os teus servos passarão, cada um armado para a guerra, perante o SENHOR, como diz meu senhor.
\par 28 Então, Moisés deu ordem a respeito deles a Eleazar, o sacerdote, e a Josué, filho de Num, e aos cabeças das casas dos pais das tribos dos filhos de Israel;
\par 29 e disse-lhes: Se os filhos de Gade e os filhos de Rúben passarem convosco o Jordão, armado cada um para a guerra, perante o SENHOR, e a terra estiver subjugada diante de vós, então, lhes dareis em possessão a terra de Gileade;
\par 30 porém, se não passarem, armados, convosco, terão possessões entre vós na terra de Canaã.
\par 31 Responderam os filhos de Gade e os filhos de Rúben, dizendo: O que o SENHOR disse a teus servos, isso faremos.
\par 32 Passaremos, armados, perante o SENHOR à terra de Canaã e teremos a possessão de nossa herança deste lado do Jordão.
\par 33 Deu Moisés aos filhos de Gade, e aos filhos de Rúben, e à meia tribo de Manassés, filho de José, o reino de Seom, rei dos amorreus, e o reino de Ogue, rei de Basã: a terra com as cidades e seus distritos, as cidades em toda a extensão do país.
\par 34 Os filhos de Gade edificaram Dibom, Atarote e Aroer;
\par 35 Atarote-Sofã, Jazer e Jogbeá;
\par 36 Bete-Ninra e Bete-Harã, cidades fortificadas, e currais de ovelhas.
\par 37 Os filhos de Rúben edificaram Hesbom, Eleale e Quiriataim;
\par 38 Nebo e Baal-Meom, mudando-lhes o nome, e Sibma; e deram outros nomes às cidades que edificaram.
\par 39 Os filhos de Maquir, filho de Manassés, foram-se para Gileade, e a tomaram, e desapossaram os amorreus que estavam nela.
\par 40 Deu, pois, Moisés Gileade a Maquir, filho de Manassés, o qual habitou nela.
\par 41 Foi Jair, filho de Manassés, e tomou as suas aldeias; e chamou-lhes Havote-Jair.
\par 42 Foi Noba e tomou a Quenate com as suas aldeias; e chamou-lhe Noba, segundo o seu nome.

\chapter{33}

\par 1 São estas as caminhadas dos filhos de Israel que saíram da terra do Egito, segundo os seus exércitos, sob as ordens de Moisés e Arão.
\par 2 Escreveu Moisés as suas saídas, caminhada após caminhada, conforme o mandado do SENHOR; e são estas as suas caminhadas, segundo as suas saídas:
\par 3 partiram, pois, de Ramessés no décimo quinto dia do primeiro mês; no dia seguinte ao da Páscoa, saíram os filhos de Israel, corajosamente, aos olhos de todos os egípcios,
\par 4 enquanto estes sepultavam todos os seus primogênitos, a quem o SENHOR havia ferido entre eles; também contra os deuses executou o SENHOR juízos.
\par 5 Partidos, pois, os filhos de Israel de Ramessés, acamparam-se em Sucote.
\par 6 E partiram de Sucote e acamparam-se em Etã, que está no fim do deserto.
\par 7 E partiram de Etã, e voltaram a Pi-Hairote, que está defronte de Baal-Zefom, e acamparam-se diante de Migdol.
\par 8 E partiram de Pi-Hairote, passaram pelo meio do mar ao deserto e, depois de terem andado caminho de três dias no deserto de Etã, acamparam-se em Mara.
\par 9 E partiram de Mara e vieram a Elim. Em Elim, havia doze fontes de águas e setenta palmeiras; e acamparam-se ali.
\par 10 E partiram de Elim e acamparam-se junto ao mar Vermelho;
\par 11 partiram do mar Vermelho e acamparam-se no deserto de Sim;
\par 12 partiram do deserto de Sim e acamparam-se em Dofca;
\par 13 partiram de Dofca e acamparam-se em Alus;
\par 14 partiram de Alus e acamparam-se em Refidim, porém não havia ali água, para que o povo bebesse;
\par 15 partiram de Refidim e acamparam-se no deserto do Sinai;
\par 16 partiram do deserto do Sinai e acamparam-se em Quibrote-Hataavá;
\par 17 partiram de Quibrote-Hataavá e acamparam-se em Hazerote;
\par 18 partiram de Hazerote e acamparam-se em Ritma;
\par 19 partiram de Ritma e acamparam-se em Rimom-Perez;
\par 20 partiram de Rimom-Perez e acamparam-se em Libna;
\par 21 partiram de Libna e acamparam-se em Rissa;
\par 22 partiram de Rissa e acamparam-se em Queelata;
\par 23 partiram de Queelata e acamparam-se no monte Sefer;
\par 24 partiram do monte Sefer e acamparam-se em Harada;
\par 25 partiram de Harada e acamparam-se em Maquelote;
\par 26 partiram de Maquelote e acamparam-se em Taate;
\par 27 partiram de Taate e acamparam-se em Tera;
\par 28 partiram de Tera e acamparam-se em Mitca;
\par 29 partiram de Mitca e acamparam-se em Hasmona;
\par 30 partiram de Hasmona e acamparam-se em Moserote;
\par 31 partiram de Moserote e acamparam-se em Benê-Jaacã;
\par 32 partiram de Benê-Jaacã e acamparam-se em Hor-Hagidgade;
\par 33 partiram de Hor-Hagidgade e acamparam-se em Jotbatá;
\par 34 partiram de Jotbatá e acamparam-se em Abrona;
\par 35 partiram de Abrona e acamparam-se em Eziom-Geber;
\par 36 partiram de Eziom-Geber e acamparam-se no deserto de Zim, que é Cades;
\par 37 partiram de Cades e acamparam-se no monte Hor, na fronteira da terra de Edom.
\par 38 Então, Arão, o sacerdote, subiu ao monte Hor, segundo o mandado do SENHOR; e morreu ali, no quinto mês do ano quadragésimo da saída dos filhos de Israel da terra do Egito, no primeiro dia do mês.
\par 39 Era Arão da idade de cento e vinte e três anos, quando morreu no monte Hor.
\par 40 Então, ouviu o cananeu, rei de Arade, que habitava o Sul da terra de Canaã, que chegavam os filhos de Israel.
\par 41 E partiram do monte Hor e acamparam-se em Zalmona;
\par 42 partiram de Zalmona e acamparam-se em Punom;
\par 43 partiram de Punom e acamparam-se em Obote;
\par 44 partiram de Obote e acamparam-se em Ijé-Abarim, no limite de Moabe;
\par 45 partiram de Ijé-Abarim e acamparam-se em Dibom-Gade;
\par 46 partiram de Dibom-Gade e acamparam-se em Almom-Diblataim;
\par 47 partiram de Almom-Diblataim e acamparam-se nos montes de Abarim, defronte de Nebo;
\par 48 partiram dos montes de Abarim e acamparam-se nas campinas de Moabe, junto ao Jordão, na altura de Jericó.
\par 49 E acamparam-se junto ao Jordão, desde Bete-Jesimote até Abel-Sitim, nas campinas de Moabe.
\par 50 Disse o SENHOR a Moisés, nas campinas de Moabe, junto ao Jordão, na altura de Jericó:
\par 51 Fala aos filhos de Israel e dize-lhes: Quando houverdes passado o Jordão para a terra de Canaã,
\par 52 desapossareis de diante de vós todos os moradores da terra, destruireis todas as pedras com figura e também todas as suas imagens fundidas e deitareis abaixo todos os seus ídolos;
\par 53 tomareis a terra em possessão e nela habitareis, porque esta terra, eu vo-la dei para a possuirdes;
\par 54 herdareis a terra por sortes, segundo as vossas famílias; à tribo mais numerosa dareis herança maior; à pequena, herança menor. Onde lhe cair a sorte, esse lugar lhe pertencerá; herdareis segundo as tribos de vossos pais.
\par 55 Porém, se não desapossardes de diante de vós os moradores da terra, então, os que deixardes ficar ser-vos-ão como espinhos nos vossos olhos e como aguilhões nas vossas ilhargas e vos perturbarão na terra em que habitardes.
\par 56 E será que farei a vós outros como pensei fazer-lhes a eles.

\chapter{34}

\par 1 Disse mais o SENHOR a Moisés:
\par 2 Dá ordem aos filhos de Israel e dize-lhes: Quando entrardes na terra de Canaã, será esta a que vos cairá em herança: a terra de Canaã, segundo os seus limites.
\par 3 A região sul vos será desde o deserto de Zim até aos limites de Edom; e o limite do sul vos será desde a extremidade do mar Salgado para o lado oriental.
\par 4 Este limite vos irá rodeando do sul para a subida de Acrabim e passará até Zim; e as suas saídas serão do sul a Cades-Barnéia; e sairá a Hazar-Adar e passará a Azmom.
\par 5 Rodeará mais este limite de Azmom até ao ribeiro do Egito; e as suas saídas serão para o lado do mar.
\par 6 Por vosso limite ocidental tereis o mar Grande; este vos será a fronteira do ocidente.
\par 7 Este vos será o limite do norte: desde o mar Grande marcareis ao monte Hor.
\par 8 Desde o monte Hor marcareis até à entrada de Hamate; e as saídas deste limite serão até Zedade;
\par 9 dali, seguirá até Zifrom, e as suas saídas serão em Hazar-Enã; este vos será o limite do norte.
\par 10 E, por limite do lado oriental, marcareis de Hazar-Enã até Sefã.
\par 11 O limite descerá desde Sefã até Ribla, para o lado oriental de Aim; depois, descerá este e irá ao longo da borda do mar de Quinerete para o lado oriental;
\par 12 descerá ainda ao longo do Jordão, e as suas saídas serão no mar Salgado; esta vos será a terra, segundo os limites de seu contorno.
\par 13 Moisés deu ordem aos filhos de Israel, dizendo: Esta é a terra que herdareis por sortes, a qual o SENHOR mandou dar às nove tribos e à meia tribo.
\par 14 Porque a tribo dos filhos dos rubenitas, segundo a casa de seus pais, e a tribo dos filhos dos gaditas, segundo a casa de seus pais, já receberam; também a meia tribo de Manassés já recebeu a sua herança.
\par 15 Estas duas tribos e meia receberam a sua herança deste lado do Jordão, na altura de Jericó, do lado oriental.
\par 16 Disse mais o SENHOR a Moisés:
\par 17 São estes os nomes dos homens que vos repartirão a terra por herança: Eleazar, o sacerdote, e Josué, filho de Num.
\par 18 Tomareis mais de cada tribo um príncipe, para repartir a terra em herança.
\par 19 São estes os nomes dos homens: da tribo de Judá, Calebe, filho de Jefoné;
\par 20 da tribo dos filhos de Simeão, Samuel, filho de Amiúde;
\par 21 da tribo de Benjamim, Elidade, filho de Quislom;
\par 22 da tribo dos filhos de Dã, o príncipe Buqui, filho de Jogli;
\par 23 dos filhos de José, da tribo dos filhos de Manassés, o príncipe Haniel, filho de Éfode;
\par 24 da tribo dos filhos de Efraim, o príncipe Quemuel, filho de Siftã;
\par 25 da tribo dos filhos de Zebulom, o príncipe Elizafã, filho de Parnaque;
\par 26 da tribo dos filhos de Issacar, o príncipe Paltiel, filho de Azã;
\par 27 da tribo dos filhos de Aser, o príncipe Aiúde, filho de Selomi;
\par 28 da tribo dos filhos de Naftali, o príncipe Pedael, filho de Amiúde.
\par 29 A estes o SENHOR ordenou que repartissem a herança pelos filhos de Israel, na terra de Canaã.

\chapter{35}

\par 1 Disse mais o SENHOR a Moisés, nas campinas de Moabe, junto ao Jordão, na altura de Jericó:
\par 2 Dá ordem aos filhos de Israel que, da herança da sua possessão, dêem cidades aos levitas, em que habitem; e também, em torno delas, dareis aos levitas arredores para o seu gado.
\par 3 Terão eles estas cidades para habitá-las; porém os seus arredores serão para o gado, para os rebanhos e para todos os seus animais.
\par 4 Os arredores das cidades que dareis aos levitas, desde o muro da cidade para fora, serão de mil côvados em redor.
\par 5 Fora da cidade, do lado oriental, medireis dois mil côvados; do lado sul, dois mil côvados; do lado ocidental, dois mil côvados e do lado norte, dois mil côvados, ficando a cidade no meio; estes lhes serão os arredores das cidades.
\par 6 Das cidades, pois, que dareis aos levitas, seis haverá de refúgio, as quais dareis para que, nelas, se acolha o homicida; além destas, lhes dareis quarenta e duas cidades.
\par 7 Todas as cidades que dareis aos levitas serão quarenta e oito cidades, juntamente com os seus arredores.
\par 8 Quanto às cidades que derdes da herança dos filhos de Israel, se for numerosa a tribo, tomareis muitas; se for pequena, tomareis poucas; cada um dará das suas cidades aos levitas, na proporção da herança que lhe tocar.
\par 9 Disse mais o SENHOR a Moisés:
\par 10 Fala aos filhos de Israel e dize-lhes: Quando passardes o Jordão para a terra de Canaã,
\par 11 escolhei para vós outros cidades que vos sirvam de refúgio, para que, nelas, se acolha o homicida que matar alguém involuntariamente.
\par 12 Estas cidades vos serão para refúgio do vingador do sangue, para que o homicida não morra antes de ser apresentado perante a congregação para julgamento.
\par 13 As cidades que derdes serão seis cidades de refúgio para vós outros.
\par 14 Três destas cidades dareis deste lado do Jordão e três dareis na terra de Canaã; cidades de refúgio serão.
\par 15 Serão de refúgio estas seis cidades para os filhos de Israel, e para o estrangeiro, e para o que se hospedar no meio deles, para que, nelas, se acolha aquele que matar alguém involuntariamente.
\par 16 Todavia, se alguém ferir a outrem com instrumento de ferro, e este morrer, é homicida; o homicida será morto.
\par 17 Ou se alguém ferir a outrem, com pedra na mão, que possa causar a morte, e este morrer, é homicida; o homicida será morto.
\par 18 Ou se alguém ferir a outrem com instrumento de pau que tiver na mão, que possa causar a morte, e este morrer, é homicida; o homicida será morto.
\par 19 O vingador do sangue, ao encontrar o homicida, matá-lo-á.
\par 20 Se alguém empurrar a outrem com ódio ou com mau intento lançar contra ele alguma coisa, e ele morrer,
\par 21 ou, por inimizade, o ferir com a mão, e este morrer, será morto aquele que o feriu; é homicida; o vingador do sangue, ao encontrar o homicida, matá-lo-á.
\par 22 Porém, se o empurrar subitamente, sem inimizade, ou contra ele lançar algum instrumento, sem mau intento,
\par 23 ou, não o vendo, deixar cair sobre ele alguma pedra que possa causar-lhe a morte, e ele morrer, não sendo ele seu inimigo, nem o tendo procurado para o mal,
\par 24 então, a congregação julgará entre o matador e o vingador do sangue, segundo estas leis,
\par 25 e livrará o homicida da mão do vingador do sangue, e o fará voltar à sua cidade de refúgio, onde se tinha acolhido; ali, ficará até à morte do sumo sacerdote, que foi ungido com o santo óleo.
\par 26 Porém, se, de alguma sorte, o homicida sair dos limites da sua cidade de refúgio, onde se tinha acolhido,
\par 27 e o vingador do sangue o achar fora dos limites dela, se o vingador do sangue matar o homicida, não será culpado do sangue.
\par 28 Pois deve ficar na sua cidade de refúgio até à morte do sumo sacerdote; porém, depois da morte deste, o homicida voltará à terra da sua possessão.
\par 29 Estas coisas vos serão por estatuto de direito a vossas gerações, em todas as vossas moradas.
\par 30 Todo aquele que matar a outrem será morto conforme o depoimento das testemunhas; mas uma só testemunha não deporá contra alguém para que morra.
\par 31 Não aceitareis resgate pela vida do homicida que é culpado de morte; antes, será ele morto.
\par 32 Também não aceitareis resgate por aquele que se acolher à sua cidade de refúgio, para tornar a habitar na sua terra, antes da morte do sumo sacerdote.
\par 33 Assim, não profanareis a terra em que estais; porque o sangue profana a terra; nenhuma expiação se fará pela terra por causa do sangue que nela for derramado, senão com o sangue daquele que o derramou.
\par 34 Não contaminareis, pois, a terra na qual vós habitais, no meio da qual eu habito; pois eu, o SENHOR, habito no meio dos filhos de Israel.

\chapter{36}

\par 1 Chegaram os cabeças das casas paternas da família dos filhos de Gileade, filho de Maquir, filho de Manassés, das famílias dos filhos de José, e falaram diante de Moisés e diante dos príncipes, cabeças das casas paternas dos filhos de Israel,
\par 2 e disseram: O SENHOR ordenou a meu senhor que dê esta terra por sorte em herança aos filhos de Israel; e a meu senhor foi ordenado pelo SENHOR que a herança do nosso irmão Zelofeade se desse a suas filhas.
\par 3 Porém, casando-se elas com algum dos filhos das outras tribos dos filhos de Israel, então, a sua herança seria diminuída da herança de nossos pais e acrescentada à herança da tribo a que vierem pertencer; assim, se tiraria da nossa herança que nos tocou em sorte.
\par 4 Vindo também o Ano do Jubileu dos filhos de Israel, a herança delas se acrescentaria à herança da tribo daqueles a que vierem pertencer; assim, a sua herança será tirada da tribo de nossos pais.
\par 5 Então, Moisés deu ordem aos filhos de Israel, segundo o mandado do SENHOR, dizendo: A tribo dos filhos de José fala o que é justo.
\par 6 Esta é a palavra que o SENHOR mandou acerca das filhas de Zelofeade, dizendo: Sejam por mulheres a quem bem parecer aos seus olhos, contanto que se casem na família da tribo de seu pai.
\par 7 Assim, a herança dos filhos de Israel não passará de tribo em tribo; pois os filhos de Israel se hão de vincular cada um à herança da tribo de seus pais.
\par 8 Qualquer filha que possuir alguma herança das tribos dos filhos de Israel se casará com alguém da família da tribo de seu pai, para que os filhos de Israel possuam cada um a herança de seus pais.
\par 9 Assim, a herança não passará de uma tribo a outra; pois as tribos dos filhos de Israel se hão de vincular cada uma à sua herança.
\par 10 Como o SENHOR ordenara a Moisés, assim fizeram as filhas de Zelofeade,
\par 11 pois Macla, Tirza, Hogla, Milca e Noa, filhas de Zelofeade, se casaram com os filhos de seus tios paternos.
\par 12 Casaram-se nas famílias dos filhos de Manassés, filho de José, e a herança delas permaneceu na tribo da família de seu pai.
\par 13 São estes os mandamentos e os juízos que ordenou o SENHOR, por intermédio de Moisés, aos filhos de Israel nas campinas de Moabe, junto ao Jordão, na altura de Jericó.


\end{document}