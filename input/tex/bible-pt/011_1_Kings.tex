\begin{document}

\title{I Reis}


\chapter{1}

\par 1 Sendo o rei Davi já velho e entrado em dias, envolviam-no com roupas, porém não se aquecia.
\par 2 Então, lhe disseram os seus servos: Procure-se para o rei, nosso senhor, uma jovem donzela, que esteja perante o rei, e tenha cuidado dele, e durma nos seus braços, para que o rei, nosso senhor, se aqueça.
\par 3 Procuraram, pois, por todos os limites de Israel uma jovem formosa; acharam Abisague, sunamita, e a trouxeram ao rei.
\par 4 A jovem era sobremaneira formosa; cuidava do rei e o servia, porém o rei não a possuiu.
\par 5 Então, Adonias, filho de Hagite, se exaltou e disse: Eu reinarei. Providenciou carros, e cavaleiros, e cinqüenta homens que corressem adiante dele.
\par 6 Jamais seu pai o contrariou, dizendo: Por que procedes assim? Além disso, era ele de aparência mui formosa e nascera depois de Absalão.
\par 7 Entendia-se ele com Joabe, filho de Zeruia, e com Abiatar, o sacerdote, que, seguindo-o, o ajudavam.
\par 8 Porém Zadoque, o sacerdote, e Benaia, filho de Joiada, e Natã, o profeta, e Simei, e Reí, e os valentes que Davi tinha não apoiavam Adonias.
\par 9 Imolou Adonias ovelhas, e bois, e animais cevados, junto à pedra de Zoelete, que está perto da fonte de Rogel, e convidou todos os seus irmãos, os filhos do rei, e todos os homens de Judá, servos do rei,
\par 10 porém a Natã, profeta, e a Benaia, e os valentes, e a Salomão, seu irmão, não convidou.
\par 11 Então, disse Natã a Bate-Seba, mãe de Salomão: Não ouviste que Adonias, filho de Hagite, reina e que nosso senhor, Davi, não o sabe?
\par 12 Vem, pois, e permite que eu te dê um conselho, para que salves a tua vida e a de Salomão, teu filho.
\par 13 Vai, apresenta-te ao rei Davi e dize-lhe: Não juraste, ó rei, senhor meu, à tua serva, dizendo: Teu filho Salomão reinará depois de mim e se assentará no meu trono? Por que, pois, reina Adonias?
\par 14 Eis que, estando tu ainda a falar com o rei, eu também entrarei depois de ti e confirmarei as tuas palavras.
\par 15 Apresentou-se, pois, Bate-Seba ao rei na recâmara; era já o rei mui velho, e Abisague, a sunamita, o servia.
\par 16 Bate-Seba inclinou a cabeça e prostrou-se perante o rei, que perguntou: Que desejas?
\par 17 Respondeu-lhe ela: Senhor meu, juraste à tua serva pelo SENHOR, teu Deus, dizendo: Salomão, teu filho, reinará depois de mim e ele se assentará no meu trono.
\par 18 Agora, eis que Adonias reina, e tu, ó rei, meu senhor, não o sabes.
\par 19 Imolou bois, e animais cevados, e ovelhas em abundância. Convidou todos os filhos do rei, a Abiatar, o sacerdote, e a Joabe, comandante do exército, mas a teu servo Salomão não convidou.
\par 20 Porém, ó rei, meu senhor, todo o Israel tem os olhos em ti, para que lhe declares quem será o teu sucessor que se assentará no teu trono.
\par 21 Do contrário, sucederá que, quando o rei, meu senhor, jazer com seus pais, eu e Salomão, meu filho, seremos tidos por culpados.
\par 22 Estando ela ainda a falar com o rei, eis que entra o profeta Natã.
\par 23 E o fizeram saber ao rei, dizendo: Aí está o profeta Natã. Apresentou-se ele ao rei, prostrou-se com o rosto em terra perante ele
\par 24 e disse: Ó rei, meu senhor, acaso disseste: Adonias reinará depois de mim e ele é quem se assentará no meu trono?
\par 25 Porque, hoje, desceu, imolou bois, e animais cevados, e ovelhas em abundância e convidou todos os filhos do rei, e os chefes do exército, e a Abiatar, o sacerdote, e eis que estão comendo e bebendo perante ele; e dizem: Viva o rei Adonias!
\par 26 Porém a mim, sendo eu teu servo, e a Zadoque, o sacerdote, e a Benaia, filho de Joiada, e a Salomão, teu servo, não convidou.
\par 27 Foi isto feito da parte do rei, meu senhor? E não fizeste saber a teu servo quem se assentaria no teu trono, depois de ti?
\par 28 Respondeu o rei Davi e disse: Chamai-me a Bate-Seba. Ela se apresentou ao rei e se pôs diante dele.
\par 29 Então, jurou o rei e disse: Tão certo como vive o SENHOR, que remiu a minha alma de toda a angústia,
\par 30 farei no dia de hoje, como te jurei pelo SENHOR, Deus de Israel, dizendo: Teu filho Salomão reinará depois de mim e se assentará no meu trono, em meu lugar.
\par 31 Então, Bate-Seba se inclinou, e se prostrou com o rosto em terra diante do rei, e disse: Viva o rei Davi, meu senhor, para sempre!
\par 32 Disse o rei Davi: Chamai-me Zadoque, o sacerdote, e Natã, o profeta, e Benaia, filho de Joiada. E eles se apresentaram ao rei.
\par 33 Disse-lhes o rei: Tomai convosco os servos de vosso senhor, e fazei montar meu filho Salomão na minha mula, e levai-o a Giom.
\par 34 Zadoque, o sacerdote, com Natã, o profeta, ali o ungirão rei sobre Israel; então, tocareis a trombeta e direis: Viva o rei Salomão!
\par 35 Subireis após ele, e virá e se assentará no meu trono, pois é ele quem reinará em meu lugar; porque ordenei seja ele príncipe sobre Israel e sobre Judá.
\par 36 Então, Benaia, filho de Joiada, respondeu ao rei e disse: Amém! Assim o diga o SENHOR, Deus do rei, meu senhor.
\par 37 Como o SENHOR foi com o rei, meu senhor, assim seja com Salomão e faça que o trono deste seja maior do que o trono do rei Davi, meu senhor.
\par 38 Então, desceu Zadoque, o sacerdote, e Natã, o profeta, e Benaia, filho de Joiada, e a guarda real, fizeram montar Salomão a mula que era do rei Davi e o levaram a Giom.
\par 39 Zadoque, o sacerdote, tomou do tabernáculo o chifre do azeite e ungiu a Salomão; tocaram a trombeta, e todo o povo exclamou: Viva o rei Salomão!
\par 40 Após ele, subiu todo o povo, tocando gaitas e alegrando-se com grande alegria, de maneira que, com o seu clamor, parecia fender-se a terra.
\par 41 Adonias e todos os convidados que com ele estavam o ouviram, quando acabavam de comer; também Joabe ouviu o sonido das trombetas e disse: Que significa esse ruído de cidade alvoroçada?
\par 42 Estando ele ainda a falar, eis que vem Jônatas, filho de Abiatar, o sacerdote; disse Adonias: Entra, porque és homem valente e trazes boas-novas.
\par 43 Respondeu Jônatas e disse a Adonias: Pelo contrário, nosso senhor, o rei Davi, constituiu rei a Salomão.
\par 44 E Davi enviou com ele a Zadoque, o sacerdote, e a Natã, o profeta, e a Benaia, filho de Joiada, e aos da guarda real; e o fizeram montar a mula que era do rei.
\par 45 Zadoque, o sacerdote, e Natã, o profeta, o ungiram rei em Giom e dali subiram alegres, e a cidade se alvoroçou; esse é o clamor que ouviste.
\par 46 Também Salomão já está assentado no trono do reino.
\par 47 Ademais, os oficiais do rei Davi vieram congratular-se com ele e disseram: Faça teu Deus que o nome de Salomão seja mais célebre do que o teu nome; e faça que o seu trono seja maior do que o teu trono. E o rei se inclinou sobre o leito.
\par 48 Também disse o rei assim: Bendito o SENHOR, Deus de Israel, que deu, hoje, quem se assente no meu trono, vendo-o os meus próprios olhos.
\par 49 Então, estremeceram e se levantaram todos os convidados que estavam com Adonias, e todos se foram, tomando cada um seu caminho.
\par 50 Porém Adonias, temendo a Salomão, levantou-se, foi e pegou nas pontas do altar.
\par 51 Foi dito a Salomão: Eis que Adonias tem medo de ti, porque pega nas pontas do altar, dizendo: Jure-me, hoje, o rei Salomão que não matará a seu servo à espada.
\par 52 Respondeu Salomão: Se for homem de bem, nem um de seus cabelos cairá em terra; porém, se se achar nele maldade, morrerá.
\par 53 Enviou o rei Salomão mensageiros, e o fizeram descer do altar; então, veio ele e se prostrou perante o rei Salomão, e este lhe disse: Vai para tua casa.

\chapter{2}

\par 1 Aproximando-se os dias da morte de Davi, deu ele ordens a Salomão, seu filho, dizendo:
\par 2 Eu vou pelo caminho de todos os mortais. Coragem, pois, e sê homem!
\par 3 Guarda os preceitos do SENHOR, teu Deus, para andares nos seus caminhos, para guardares os seus estatutos, e os seus mandamentos, e os seus juízos, e os seus testemunhos, como está escrito na Lei de Moisés, para que prosperes em tudo quanto fizeres e por onde quer que fores;
\par 4 para que o SENHOR confirme a palavra que falou de mim, dizendo: Se teus filhos guardarem o seu caminho, para andarem perante a minha face fielmente, de todo o seu coração e de toda a sua alma, nunca te faltará sucessor ao trono de Israel.
\par 5 Também tu sabes o que me fez Joabe, filho de Zeruia, e o que fez aos dois comandantes do exército de Israel, a Abner, filho de Ner, e a Amasa, filho de Jéter, os quais matou, e, em tempo de paz, vingou o sangue derramado em guerra, manchando com ele o cinto que trazia nos lombos e as sandálias nos pés.
\par 6 Faze, pois, segundo a tua sabedoria e não permitas que suas cãs desçam à sepultura em paz.
\par 7 Porém, com os filhos de Barzilai, o gileadita, usarás de benevolência, e estarão entre os que comem à tua mesa, porque assim se houveram comigo, quando eu fugia por causa de teu irmão Absalão.
\par 8 Eis que também contigo está Simei, filho de Gera, filho de Benjamim, de Baurim, que me maldisse com dura maldição, no dia em que ia a Maanaim; porém ele saiu a encontrar-se comigo junto ao Jordão, e eu, pelo SENHOR, lhe jurei, dizendo que o não mataria à espada.
\par 9 Mas, agora, não o tenhas por inculpável, pois és homem prudente e bem saberás o que lhe hás de fazer para que as suas cãs desçam à sepultura com sangue.
\par 10 Davi descansou com seus pais e foi sepultado na Cidade de Davi.
\par 11 Foi o tempo que Davi reinou sobre Israel quarenta anos: sete anos em Hebrom e em Jerusalém trinta e três.
\par 12 Salomão assentou-se no trono de Davi, seu pai, e o seu reino se fortificou sobremaneira.
\par 13 Então, veio Adonias, filho de Hagite, a Bate-Seba, mãe de Salomão; disse ela: É de paz a tua vinda? Respondeu ele: É de paz.
\par 14 E acrescentou: Uma palavra tenho que dizer-te. Ela disse: Fala.
\par 15 Disse, pois, ele: Bem sabes que o reino era meu, e todo o Israel esperava que eu reinasse, ainda que o reino se transferiu e veio a ser de meu irmão, porque do SENHOR ele o recebeu.
\par 16 Agora, um só pedido te faço; não mo rejeites. Ela lhe disse: Fala.
\par 17 Ele disse: Peço-te que fales ao rei Salomão (pois não to recusará) que me dê por mulher a Abisague, sunamita.
\par 18 Respondeu Bate-Seba: Bem, eu falarei por ti ao rei.
\par 19 Foi, pois, Bate-Seba ter com o rei Salomão, para falar-lhe por Adonias. O rei se levantou a encontrar-se com ela e se inclinou diante dela; então, se assentou no seu trono e mandou pôr uma cadeira para sua mãe, e ela se assentou à sua mão direita.
\par 20 Então, disse ela: Só um pequeno pedido te faço; não mo rejeites. E o rei lhe disse: Pede, minha mãe, porque to não recusarei.
\par 21 Disse ela: Dê-se Abisague, a sunamita, a Adonias, teu irmão, por mulher.
\par 22 Então, respondeu o rei Salomão e disse a sua mãe: Por que pedes Abisague, a sunamita, para Adonias? Pede também para ele o reino (porque é meu irmão maior), para ele, digo, e também para Abiatar, o sacerdote, e para Joabe, filho de Zeruia.
\par 23 E jurou o rei Salomão pelo SENHOR, dizendo: Deus me faça o que lhe aprouver, se não falou Adonias esta palavra contra a sua vida.
\par 24 Agora, pois, tão certo como vive o SENHOR, que me estabeleceu e me fez assentar no trono de Davi, meu pai, e me edificou casa, como tinha dito, hoje, morrerá Adonias.
\par 25 Enviou o rei Salomão a Benaia, filho de Joiada, o qual arremeteu contra ele, de sorte que morreu.
\par 26 E a Abiatar, o sacerdote, disse o rei: Vai para Anatote, para teus campos, porque és homem digno de morte; porém não te matarei hoje, porquanto levaste a arca do SENHOR Deus diante de Davi, meu pai, e porque te afligiste com todas as aflições de meu pai.
\par 27 Expulsou, pois, Salomão a Abiatar, para que não mais fosse sacerdote do SENHOR, cumprindo, assim, a palavra que o SENHOR dissera sobre a casa de Eli, em Siló.
\par 28 Chegando esta notícia a Joabe (porque Joabe se tinha desviado seguindo a Adonias, ainda que se não desviara seguindo a Absalão), fugiu ele para o tabernáculo do SENHOR e pegou nas pontas do altar.
\par 29 Foi dito ao rei Salomão que Joabe havia fugido para o tabernáculo do SENHOR; e eis que está junto ao altar. Enviou, pois, Salomão a Benaia, filho de Joiada, dizendo: Vai, arremete contra ele.
\par 30 Foi Benaia ao tabernáculo do SENHOR e disse a Joabe: Assim diz o rei: Sai daí. Ele respondeu: Não, morrerei aqui. Benaia tornou com a resposta ao rei, dizendo: Assim falou Joabe e assim me respondeu.
\par 31 Disse-lhe o rei: Faze como ele te disse; arremete contra ele e sepulta-o, para que tires de mim e da casa de meu pai a culpa do sangue que Joabe sem causa derramou.
\par 32 Assim, o SENHOR fará recair a culpa de sangue de Joabe sobre a sua cabeça, porque arremeteu contra dois homens mais justos e melhores do que ele e os matou à espada, sem que Davi, meu pai, o soubesse; isto é: a Abner, filho de Ner, comandante do exército de Israel, e a Amasa, filho de Jéter, comandante do exército de Judá.
\par 33 Assim, recairá o sangue destes sobre a cabeça de Joabe e sobre a cabeça da sua descendência para sempre; mas a Davi, e à sua descendência, e à sua casa, e ao seu trono, dará o SENHOR paz para todo o sempre.
\par 34 Subiu Benaia, filho de Joiada, arremeteu contra ele e o matou; e foi sepultado em sua casa, no deserto.
\par 35 Em lugar de Joabe constituiu o rei por comandante do exército a Benaia, filho de Joiada, e, em lugar de Abiatar, a Zadoque por sacerdote.
\par 36 Depois, mandou o rei chamar a Simei e lhe disse: Edifica-te uma casa em Jerusalém, e habita aí, e daí não saias, nem para uma parte nem para outra.
\par 37 Porque há de ser que, no dia em que saíres e passares o ribeiro de Cedrom, fica sabendo que serás morto; o teu sangue cairá, então, sobre a tua cabeça.
\par 38 Simei disse ao rei: Boa é essa palavra; como disse o rei, meu senhor, assim fará o teu servo. E Simei habitou em Jerusalém muitos dias.
\par 39 Ao cabo de três anos, porém, dois escravos de Simei fugiram para Aquis, filho de Maaca, rei de Gate; e deram parte a Simei, dizendo: Eis que teus servos estão em Gate.
\par 40 Então, Simei se dispôs, albardou o seu jumento e foi a Gate ter com Aquis em busca dos seus escravos; e trouxe de Gate os seus escravos.
\par 41 Foi Salomão avisado de que Simei de Jerusalém fora a Gate e já havia voltado.
\par 42 Então, mandou o rei chamar a Simei e lhe disse: Não te fiz eu jurar pelo SENHOR e não te protestei, dizendo: No dia em que saíres para uma ou outra parte, fica sabendo que serás morto? E tu me disseste: Boa é essa palavra que ouvi.
\par 43 Por que, pois, não guardaste o juramento do SENHOR, nem a ordem que te dei?
\par 44 Disse mais o rei a Simei: Bem sabes toda a maldade que o teu coração reconhece que fizeste a Davi, meu pai; pelo que o SENHOR te fez recair sobre a cabeça a tua maldade.
\par 45 Mas o rei Salomão será abençoado, e o trono de Davi, mantido perante o SENHOR, para sempre.
\par 46 O rei deu ordem a Benaia, filho de Joiada, o qual saiu e arremeteu contra ele, de sorte que morreu; assim se firmou o reino sob o domínio de Salomão.

\chapter{3}

\par 1 Salomão aparentou-se com Faraó, rei do Egito, pois tomou por mulher a filha de Faraó e a trouxe à Cidade de Davi, até que acabasse de edificar a sua casa, e a Casa do SENHOR, e a muralha à roda de Jerusalém.
\par 2 Entretanto, o povo oferecia sacrifícios sobre os altos, porque até àqueles dias ainda não se tinha edificado casa ao nome do SENHOR.
\par 3 Salomão amava ao SENHOR, andando nos preceitos de Davi, seu pai; porém sacrificava ainda nos altos e queimava incenso.
\par 4 Foi o rei a Gibeão para lá sacrificar, porque era o alto maior; ofereceu mil holocaustos Salomão naquele altar.
\par 5 Em Gibeão, apareceu o SENHOR a Salomão, de noite, em sonhos. Disse-lhe Deus: Pede-me o que queres que eu te dê.
\par 6 Respondeu Salomão: De grande benevolência usaste para com teu servo Davi, meu pai, porque ele andou contigo em fidelidade, e em justiça, e em retidão de coração, perante a tua face; mantiveste-lhe esta grande benevolência e lhe deste um filho que se assentasse no seu trono, como hoje se vê.
\par 7 Agora, pois, ó SENHOR, meu Deus, tu fizeste reinar teu servo em lugar de Davi, meu pai; não passo de uma criança, não sei como conduzir-me.
\par 8 Teu servo está no meio do teu povo que elegeste, povo grande, tão numeroso, que se não pode contar.
\par 9 Dá, pois, ao teu servo coração compreensivo para julgar a teu povo, para que prudentemente discirna entre o bem e o mal; pois quem poderia julgar a este grande povo?
\par 10 Estas palavras agradaram ao Senhor, por haver Salomão pedido tal coisa.
\par 11 Disse-lhe Deus: Já que pediste esta coisa e não pediste longevidade, nem riquezas, nem a morte de teus inimigos; mas pediste entendimento, para discernires o que é justo;
\par 12 eis que faço segundo as tuas palavras: dou-te coração sábio e inteligente, de maneira que antes de ti não houve teu igual, nem depois de ti o haverá.
\par 13 Também até o que me não pediste eu te dou, tanto riquezas como glória; que não haja teu igual entre os reis, por todos os teus dias.
\par 14 Se andares nos meus caminhos e guardares os meus estatutos e os meus mandamentos, como andou Davi, teu pai, prolongarei os teus dias.
\par 15 Despertou Salomão; e eis que era sonho. Veio a Jerusalém, pôs-se perante a arca da Aliança do SENHOR, ofereceu holocaustos, apresentou ofertas pacíficas e deu um banquete a todos os seus oficiais.
\par 16 Então, vieram duas prostitutas ao rei e se puseram perante ele.
\par 17 Disse-lhe uma das mulheres: Ah! Senhor meu, eu e esta mulher moramos na mesma casa, onde dei à luz um filho.
\par 18 No terceiro dia, depois do meu parto, também esta mulher teve um filho. Estávamos juntas; nenhuma outra pessoa se achava conosco na casa; somente nós ambas estávamos ali.
\par 19 De noite, morreu o filho desta mulher, porquanto se deitara sobre ele.
\par 20 Levantou-se à meia-noite, e, enquanto dormia a tua serva, tirou-me a meu filho do meu lado, e o deitou nos seus braços; e a seu filho morto deitou-o nos meus.
\par 21 Levantando-me de madrugada para dar de mamar a meu filho, eis que estava morto; mas, reparando nele pela manhã, eis que não era o filho que eu dera à luz.
\par 22 Então, disse a outra mulher: Não, mas o vivo é meu filho; o teu é o morto. Porém esta disse: Não, o morto é teu filho; o meu é o vivo. Assim falaram perante o rei.
\par 23 Então, disse o rei: Esta diz: Este que vive é meu filho, e teu filho é o morto; e esta outra diz: Não, o morto é teu filho, e o meu filho é o vivo.
\par 24 Disse mais o rei: Trazei-me uma espada. Trouxeram uma espada diante do rei.
\par 25 Disse o rei: Dividi em duas partes o menino vivo e dai metade a uma e metade a outra.
\par 26 Então, a mulher cujo filho era o vivo falou ao rei (porque o amor materno se aguçou por seu filho) e disse: Ah! Senhor meu, dai-lhe o menino vivo e por modo nenhum o mateis. Porém a outra dizia: Nem meu nem teu; seja dividido.
\par 27 Então, respondeu o rei: Dai à primeira o menino vivo; não o mateis, porque esta é sua mãe.
\par 28 Todo o Israel ouviu a sentença que o rei havia proferido; e todos tiveram profundo respeito ao rei, porque viram que havia nele a sabedoria de Deus, para fazer justiça.

\chapter{4}

\par 1 O rei Salomão reinou sobre todo o Israel.
\par 2 Eram estes os seus homens principais: Azarias, filho de Zadoque, o principal.
\par 3 Eliorefe e Aías, filhos de Sisa, eram secretários; Josafá, filho de Ailude, era o cronista;
\par 4 Benaia, filho de Joiada, era comandante do exército; Zadoque e Abiatar eram sacerdotes;
\par 5 Azarias, filho de Natã, era intendente-chefe; Zabude, filho de Natã, ministro, amigo do rei;
\par 6 Aisar, mordomo; Adonirão, filho de Abda, superintendente dos que trabalhavam forçados.
\par 7 Tinha Salomão doze intendentes sobre todo o Israel, que forneciam mantimento ao rei e à sua casa; cada um tinha de fornecer durante um mês do ano.
\par 8 São estes os seus nomes: Ben-Hur, nas montanhas de Efraim;
\par 9 Ben-Dequer, em Macaz, Saalabim, Bete-Semes, Elom e Bete-Hanã;
\par 10 Ben-Hesede, em Arubote; a este pertencia também Socó e toda a terra de Héfer;
\par 11 Ben-Abinadabe tinha toda a cordilheira de Dor; Tafate, filha de Salomão, era sua mulher.
\par 12 Baaná, filho de Ailude, tinha a Taanaque, e a Megido, e a toda a Bete-Seã, que está junto a Zaretã, abaixo de Jezreel, desde Bete-Seã até Abel-Meolá, até além de Jocmeão.
\par 13 Ben-Geber, em Ramote-Gileade; tinha este as aldeias de Jair, filho de Manassés, as quais estão em Gileade; também tinha a região de Argobe, a qual está em Basã, sessenta grandes cidades com muros e ferrolhos de bronze.
\par 14 Ainadabe, filho de Ido, em Maanaim;
\par 15 Aimaás, em Naftali; também este tomou filha de Salomão por mulher, a saber, Basemate.
\par 16 Baaná, filho de Husai, em Aser e Bealote;
\par 17 Josafá, filho de Parua, em Issacar;
\par 18 Simei, filho de Elá, em Benjamim;
\par 19 Geber, filho de Uri, na terra de Gileade, a terra de Seom, rei dos amorreus, e de Ogue, rei de Basã; e havia só um intendente nesta terra.
\par 20 Eram, pois, os de Judá e Israel muitos, numerosos como a areia que está ao pé do mar; comiam, bebiam e se alegravam.
\par 21 Dominava Salomão sobre todos os reinos desde o Eufrates até à terra dos filisteus e até à fronteira do Egito; os quais pagavam tributo e serviram a Salomão todos os dias da sua vida.
\par 22 Era, pois, o provimento diário de Salomão trinta coros de flor de farinha e sessenta coros de farinha;
\par 23 dez bois cevados, vinte bois de pasto e cem carneiros, afora os veados, as gazelas, os corços e aves cevadas.
\par 24 Porque dominava sobre toda a região e sobre todos os reis aquém do Eufrates, desde Tifsa até Gaza, e tinha paz por todo o derredor.
\par 25 Judá e Israel habitavam confiados, cada um debaixo da sua videira e debaixo da sua figueira, desde Dã até Berseba, todos os dias de Salomão.
\par 26 Tinha também Salomão quarenta mil cavalos em estrebarias, para os seus carros, e doze mil cavaleiros.
\par 27 Forneciam, pois, os intendentes provisões, cada um no seu mês, ao rei Salomão e a todos quantos lhe chegavam à mesa; coisa nenhuma deixavam faltar.
\par 28 Também levavam a cevada e a palha para os cavalos e os ginetes, para o lugar onde estivesse o rei, segundo lhes fora prescrito.
\par 29 Deu também Deus a Salomão sabedoria, grandíssimo entendimento e larga inteligência como a areia que está na praia do mar.
\par 30 Era a sabedoria de Salomão maior do que a de todos os do Oriente e do que toda a sabedoria dos egípcios.
\par 31 Era mais sábio do que todos os homens, mais sábio do que Etã, ezraíta, e do que Hemã, Calcol e Darda, filhos de Maol; e correu a sua fama por todas as nações em redor.
\par 32 Compôs três mil provérbios, e foram os seus cânticos mil e cinco.
\par 33 Discorreu sobre todas as plantas, desde o cedro que está no Líbano até ao hissopo que brota do muro; também falou dos animais e das aves, dos répteis e dos peixes.
\par 34 De todos os povos vinha gente a ouvir a sabedoria de Salomão, e também enviados de todos os reis da terra que tinham ouvido da sua sabedoria.

\chapter{5}

\par 1 Enviou também Hirão, rei de Tiro, os seus servos a Salomão (porque ouvira que ungiram a Salomão rei em lugar de seu pai), pois Hirão sempre fora amigo de Davi.
\par 2 Então, Salomão enviou mensageiros a Hirão, dizendo:
\par 3 Bem sabes que Davi, meu pai, não pôde edificar uma casa ao nome do SENHOR, seu Deus, por causa das guerras com que o envolveram os seus inimigos, até que o SENHOR lhos pôs debaixo dos pés.
\par 4 Porém a mim o SENHOR, meu Deus, me tem dado descanso de todos os lados; não há nem inimigo, nem adversidade alguma.
\par 5 Pelo que intento edificar uma casa ao nome do SENHOR, meu Deus, como falou o SENHOR a Davi, meu pai, dizendo: Teu filho, que porei em teu lugar no teu trono, esse edificará uma casa ao meu nome.
\par 6 Dá ordem, pois, que do Líbano me cortem cedros; os meus servos estarão com os teus servos, e eu te pagarei o salário destes segundo determinares; porque bem sabes que entre o meu povo não há quem saiba cortar a madeira como os sidônios.
\par 7 Ouvindo Hirão as palavras de Salomão, muito se alegrou e disse: Bendito seja, hoje, o SENHOR, que deu a Davi um filho sábio sobre este grande povo.
\par 8 Enviou Hirão mensageiros a Salomão, dizendo: Ouvi o que mandaste dizer. Farei toda a tua vontade acerca das madeiras de cedro e de cipreste.
\par 9 Os meus servos as levarão desde o Líbano até ao mar, e eu as farei conduzir em jangadas pelo mar até ao lugar que me designares e ali as desamarrarei; e tu as receberás. Tu também farás a minha vontade, dando provisões à minha casa.
\par 10 Assim, deu Hirão a Salomão madeira de cedro e madeira de cipreste, segundo este queria.
\par 11 Salomão deu a Hirão vinte mil coros de trigo, para sustento da sua casa, e vinte coros de azeite batido; e o fazia de ano em ano.
\par 12 Deu o SENHOR sabedoria a Salomão, como lhe havia prometido. Havia paz entre Hirão e Salomão; e fizeram ambos entre si aliança.
\par 13 Formou o rei Salomão uma leva de trabalhadores dentre todo o Israel, e se compunha de trinta mil homens.
\par 14 E os enviava ao Líbano alternadamente, dez mil por mês; um mês estavam no Líbano, e dois meses, cada um em sua casa; e Adonirão dirigia a leva.
\par 15 Tinha também Salomão setenta mil que levavam as cargas e oitenta mil que talhavam pedra nas montanhas,
\par 16 afora os chefes-oficiais de Salomão, em número de três mil e trezentos, que dirigiam a obra e davam ordens ao povo que a executava.
\par 17 Mandou o rei que trouxessem pedras grandes, e pedras preciosas, e pedras lavradas para fundarem a casa.
\par 18 Lavravam-nas os edificadores de Salomão, e os de Hirão, e os giblitas; e preparavam a madeira e as pedras para se edificar a casa.

\chapter{6}

\par 1 No ano quatrocentos e oitenta, depois de saírem os filhos de Israel do Egito, Salomão, no ano quarto do seu reinado sobre Israel, no mês de zive (este é o mês segundo), começou a edificar a Casa do SENHOR.
\par 2 A casa que o rei Salomão edificou ao SENHOR era de sessenta côvados de comprimento, vinte de largura e trinta de altura.
\par 3 O pórtico diante do templo da casa media vinte côvados no sentido da largura do Lugar Santo, contra dez de fundo.
\par 4 Para a casa, fez janelas de fasquias fixas superpostas.
\par 5 Contra a parede da casa, tanto do santuário como do Santo dos Santos, edificou andares ao redor e fez câmaras laterais ao redor.
\par 6 O andar de baixo tinha cinco côvados de largura, o do meio, seis, e o terceiro, sete; porque, pela parte de fora da casa em redor, fizera reentrâncias para que as vigas não fossem introduzidas nas paredes.
\par 7 Edificava-se a casa com pedras já preparadas nas pedreiras, de maneira que nem martelo, nem machado, nem instrumento algum de ferro se ouviu na casa quando a edificavam.
\par 8 A porta da câmara do meio do andar térreo estava ao lado sul da casa, e por caracóis se subia ao segundo e, deste, ao terceiro.
\par 9 Assim, edificou a casa e a rematou, cobrindo-a com um tabuado de cedro.
\par 10 Os andares que edificou contra a casa toda eram de cinco côvados de altura, e os ligou com a casa com madeira de cedro.
\par 11 Então, veio a palavra do SENHOR a Salomão, dizendo:
\par 12 Quanto a esta casa que tu edificas, se andares nos meus estatutos, e executares os meus juízos, e guardares todos os meus mandamentos, andando neles, cumprirei para contigo a minha palavra, a qual falei a Davi, teu pai.
\par 13 E habitarei no meio dos filhos de Israel e não desampararei o meu povo.
\par 14 Assim, edificou Salomão a casa e a rematou.
\par 15 Também revestiu as paredes da casa por dentro com tábuas de cedro; desde o soalho da casa até ao teto, cobriu com madeira por dentro; e cobriu o piso da casa com tábuas de cipreste.
\par 16 Da mesma sorte, revestiu também os vinte côvados dos fundos da casa com tábuas de cedro, desde o soalho até ao teto; e esse interior ele constituiu em santuário, a saber, o Santo dos Santos.
\par 17 Era, pois, o Santo Lugar do templo de quarenta côvados.
\par 18 O cedro da casa por dentro era lavrado de colocíntidas e flores abertas; tudo era cedro, pedra nenhuma se via.
\par 19 No mais interior da casa, preparou o Santo dos Santos para nele colocar a arca da Aliança do SENHOR.
\par 20 Era o Santo dos Santos de vinte côvados de comprimento, vinte de largura e vinte de altura; cobriu-o de ouro puro. Cobriu também de ouro o altar de cedro.
\par 21 Por dentro, Salomão revestiu a casa de ouro puro; e fez passar cadeias de ouro por dentro do Santo dos Santos, que também cobrira de ouro.
\par 22 Assim, cobriu de ouro toda a casa, inteiramente, e também todo o altar que estava diante do Santo dos Santos.
\par 23 No Santo dos Santos, fez dois querubins de madeira de oliveira, cada um da altura de dez côvados.
\par 24 Cada asa de um querubim era de cinco côvados; dez côvados havia, pois, de uma a outra extremidade de suas asas.
\par 25 Assim, também era de dez côvados o outro querubim; ambos mediam o mesmo e eram da mesma forma.
\par 26 A altura de um querubim era de dez côvados; e assim a do outro.
\par 27 Pôs os querubins no mais interior da casa; os querubins estavam de asas estendidas, de maneira que a asa de um tocava numa parede, e a asa do outro tocava na outra parede; e as suas asas no meio da casa tocavam uma na outra.
\par 28 E cobriu de ouro os querubins.
\par 29 Nas paredes todas, tanto no mais interior da casa como no seu exterior, lavrou, ao redor, entalhes de querubins, palmeiras e flores abertas.
\par 30 Também cobriu de ouro o soalho, tanto no mais interior da casa como no seu exterior.
\par 31 Para entrada do Santo dos Santos, fez folhas de madeira de oliveira; a verga com as ombreiras formavam uma porta pentagonal.
\par 32 Assim, fabricou de madeira de oliveira duas folhas e lavrou nelas entalhes de querubins, de palmeiras e de flores abertas; a estas, como as palmeiras e os querubins, cobriu de ouro.
\par 33 Fez, para entrada do Santo Lugar, ombreiras de madeira de oliveira; entrada quadrilateral,
\par 34 cujas duas folhas eram de madeira de cipreste; e as duas tábuas de cada folha eram dobradiças.
\par 35 E as lavrou de querubins, de palmeiras e de flores abertas e as cobriu de ouro acomodado ao lavor.
\par 36 Também edificou o átrio interior de três ordens de pedras cortadas e de uma ordem de vigas de cedro.
\par 37 No ano quarto, se pôs o fundamento da Casa do SENHOR, no mês de zive.
\par 38 E, no ano undécimo, no mês de bul, que é o oitavo, se acabou esta casa com todas as suas dependências, tal como devia ser. Levou Salomão sete anos para edificá-la.

\chapter{7}

\par 1 Edificou Salomão os seus palácios, levando treze anos para os concluir.
\par 2 Edificou a Casa do Bosque do Líbano, de cem côvados de comprimento, cinqüenta de largura e trinta de altura, sobre quatro ordens de colunas de cedro e vigas de cedro sobre as colunas.
\par 3 A cobertura era de cedro, abrangendo as câmaras laterais em número de quarenta e cinco, quinze em cada andar, as quais repousavam sobre colunas.
\par 4 Havia janelas em três ordens e janela oposta a janela em três fileiras.
\par 5 Todas as portas e janelas eram quadradas; e janela oposta a janela em três fileiras.
\par 6 Depois, fez o Salão das Colunas, de cinqüenta côvados de comprimento e trinta de largura; e havia um pórtico de colunas defronte dele, um baldaquino.
\par 7 Também fez a Sala do Trono, onde julgava, a saber, a Sala do Julgamento, coberta de cedro desde o soalho até ao teto.
\par 8 A sua casa, em que moraria, fê-la noutro pátio atrás da Sala do Trono, de obra semelhante a esta; também para a filha de Faraó, que tomara por mulher, fez Salomão uma casa semelhante à Sala do Trono.
\par 9 Todas estas construções eram de pedras de valor, cortadas à medida, serradas para o lado de dentro e para o de fora; e isto desde o fundamento até às beiras do teto, e por fora até ao átrio maior.
\par 10 O fundamento era de pedras de valor, pedras grandes; pedras de dez côvados e pedras de oito côvados;
\par 11 por cima delas, pedras de valor, cortadas segundo as medidas, e cedros.
\par 12 Ao redor do grande átrio, havia três ordens de pedras cortadas e uma ordem de vigas de cedro; assim era também o átrio interior da Casa do SENHOR e o pórtico daquela casa.
\par 13 Enviou o rei Salomão mensageiros que de Tiro trouxessem Hirão.
\par 14 Era este filho de uma mulher viúva, da tribo de Naftali, e fora seu pai um homem de Tiro que trabalhava em bronze; Hirão era cheio de sabedoria, e de entendimento, e de ciência para fazer toda obra de bronze. Veio ter com o rei Salomão e fez toda a sua obra.
\par 15 Formou duas colunas de bronze; a altura de cada uma era de dezoito côvados, e um fio de doze côvados era a medida de sua circunferência.
\par 16 Também fez dois capitéis de fundição de bronze para pôr sobre o alto das colunas; de cinco côvados era a altura de cada um deles.
\par 17 Havia obra de rede e ornamentos torcidos em forma de cadeia, para os capitéis que estavam sobre o alto das colunas; sete para um capitel e sete para o outro.
\par 18 Fez também romãs em duas fileiras por cima de uma das obras de rede, para cobrir o capitel no alto da coluna; o mesmo fez com o outro capitel.
\par 19 Os capitéis que estavam no alto das colunas eram de obra de lírios, como na Sala do Trono, e de quatro côvados.
\par 20 Perto do bojo, próximo à obra de rede, os capitéis que estavam no alto das duas colunas tinham duzentas romãs, dispostas em fileiras em redor, sobre um e outro capitel.
\par 21 Depois, levantou as colunas no pórtico do templo; tendo levantado a coluna direita, chamou-lhe Jaquim; e, tendo levantado a coluna esquerda, chamou-lhe Boaz.
\par 22 No alto das colunas, estava a obra de lírios. E, assim, se acabou a obra das colunas.
\par 23 Fez também o mar de fundição, redondo, de dez côvados de uma borda até à outra borda, e de cinco de altura; e um fio de trinta côvados era a medida de sua circunferência.
\par 24 Por baixo da sua borda em redor, havia colocíntidas, dez em cada côvado; estavam em duas fileiras, fundidas quando se fundiu o mar.
\par 25 Assentava-se o mar sobre doze bois; três olhavam para o norte, três, para o ocidente, três, para o sul, e três, para o oriente; o mar apoiava-se sobre eles, cujas partes posteriores convergiam para dentro.
\par 26 A grossura dele era de quatro dedos, e a sua borda, como borda de copo, como flor de lírios; comportava dois mil batos.
\par 27 Fez também de bronze dez suportes; cada um media quatro côvados de comprimento, quatro de largura e três de altura;
\par 28 e eram do seguinte modo: tinham painéis, que estavam entre molduras,
\par 29 nos quais havia leões, bois e querubins; nas molduras de cima e de baixo dos leões e dos bois, havia festões pendentes.
\par 30 Tinha cada suporte quatro rodas de bronze e eixos de bronze; os seus quatro pés tinham apoios debaixo da pia, apoios fundidos, e ao lado de cada um havia festões.
\par 31 A boca dos suportes estava dentro de uma guarnição que media um côvado de altura; a boca era redonda como a obra de um pedestal e tinha o diâmetro de um côvado e meio. Também nela havia entalhes, e os seus painéis eram quadrados, não redondos.
\par 32 As quatro rodas estavam debaixo dos painéis, e os eixos das rodas formavam uma peça com o suporte; cada roda era de um côvado e meio de altura.
\par 33 As rodas eram como as de um carro: seus eixos, suas cambas, seus raios e seus cubos, todos eram fundidos.
\par 34 Havia quatro apoios aos quatro cantos de cada suporte, que com este formavam uma peça.
\par 35 No alto de cada suporte, havia um cilindro de meio côvado de altura; também, no alto de cada suporte, os apoios e painéis formavam uma peça só com ele.
\par 36 Na superfície dos seus apoios e dos seus painéis, gravou querubins, leões e palmeiras, segundo o espaço de cada um, com festões ao redor.
\par 37 Deste modo, fez os dez suportes; todos tinham a mesma fundição, o mesmo tamanho e o mesmo entalhe.
\par 38 Também fez dez pias de bronze; em cada uma cabiam quarenta batos, e cada uma era de quatro côvados; sobre cada um dos dez suportes estava uma pia.
\par 39 Pôs cinco suportes à direita da casa e cinco, à esquerda; porém o mar pôs ao lado direito da casa, para o lado sudeste.
\par 40 Depois, fez Hirão os caldeirões, e as pás, e as bacias. Assim, terminou ele de fazer toda a obra para o rei Salomão, para a Casa do SENHOR:
\par 41 as duas colunas, os dois globos dos capitéis que estavam no alto das duas colunas; as duas redes, para cobrir os dois globos dos capitéis que estavam ao alto das colunas;
\par 42 as quatrocentas romãs para as duas redes, isto é, duas fileiras de romãs para cada rede, para cobrirem os dois globos dos capitéis que estavam no alto das colunas;
\par 43 os dez suportes e as dez pias sobre eles;
\par 44 o mar com os doze bois por baixo;
\par 45 os caldeirões, as pás, as bacias e todos estes utensílios que fez Hirão para o rei Salomão, para a Casa do SENHOR, todos eram de bronze polido.
\par 46 Na planície do Jordão, o rei os fez fundir em terra barrenta, entre Sucote e Zaretã.
\par 47 Deixou Salomão de pesar todos os utensílios pelo seu excessivo número, não se verificando, pois, o peso do seu bronze.
\par 48 Também fez Salomão todos os utensílios do Santo Lugar do SENHOR: o altar de ouro e a mesa de ouro, sobre a qual estavam os pães da proposição;
\par 49 os castiçais de ouro finíssimo, cinco à direita e cinco à esquerda, diante do Santo dos Santos; as flores, as lâmpadas e as espevitadeiras, também de ouro;
\par 50 também as taças, as espevitadeiras, as bacias, os recipientes para incenso e os braseiros, de ouro finíssimo; as dobradiças para as portas da casa interior para o Santo dos Santos e as das portas do Santo Lugar do templo, também de ouro.
\par 51 Assim, se acabou toda a obra que fez o rei Salomão para a Casa do SENHOR; então, trouxe Salomão as coisas que Davi, seu pai, havia dedicado; a prata, o ouro e os utensílios, ele os pôs entre os tesouros da Casa do SENHOR.

\chapter{8}

\par 1 Congregou Salomão os anciãos de Israel, todos os cabeças das tribos, os príncipes das famílias dos israelitas, diante de si em Jerusalém, para fazerem subir a arca da Aliança do SENHOR da Cidade de Davi, que é Sião, para o templo.
\par 2 Todos os homens de Israel se congregaram junto ao rei Salomão na ocasião da festa, no mês de etanim, que é o sétimo.
\par 3 Vieram todos os anciãos de Israel, e os sacerdotes tomaram a arca do SENHOR
\par 4 e a levaram para cima, com a tenda da congregação, e também os utensílios sagrados que nela havia; os sacerdotes e levitas é que os fizeram subir.
\par 5 O rei Salomão e toda a congregação de Israel, que se reunira a ele, estavam todos diante da arca, sacrificando ovelhas e bois, que, de tão numerosos, não se podiam contar.
\par 6 Puseram os sacerdotes a arca da Aliança do SENHOR no seu lugar, no santuário mais interior do templo, no Santo dos Santos, debaixo das asas dos querubins.
\par 7 Pois os querubins estendiam as asas sobre o lugar da arca e, do alto, cobriam a arca e os varais.
\par 8 Os varais sobressaíam tanto, que suas pontas eram vistas do Santo Lugar, defronte do Santo dos Santos, porém de fora não se viam. Ali, estão até ao dia de hoje.
\par 9 Nada havia na arca senão as duas tábuas de pedra, que Moisés ali pusera junto a Horebe, quando o SENHOR fez aliança com os filhos de Israel, ao saírem da terra do Egito.
\par 10 Tendo os sacerdotes saído do santuário, uma nuvem encheu a Casa do SENHOR,
\par 11 de tal sorte que os sacerdotes não puderam permanecer ali, para ministrar, por causa da nuvem, porque a glória do SENHOR enchera a Casa do SENHOR.
\par 12 Então, disse Salomão: O SENHOR declarou que habitaria em trevas espessas.
\par 13 Na verdade, edifiquei uma casa para tua morada, lugar para a tua eterna habitação.
\par 14 Voltou, então, o rei o rosto e abençoou a toda a congregação de Israel, enquanto se mantinha toda em pé;
\par 15 e disse: Bendito seja o SENHOR, o Deus de Israel, que falou pessoalmente a Davi, meu pai, e pelo seu poder o cumpriu, dizendo:
\par 16 Desde o dia em que tirei Israel, o meu povo, do Egito, não escolhi cidade alguma de todas as tribos de Israel, para edificar uma casa a fim de ali estabelecer o meu nome; porém escolhi a Davi para chefe do meu povo de Israel.
\par 17 Também Davi, meu pai, propusera em seu coração o edificar uma casa ao nome do SENHOR, o Deus de Israel.
\par 18 Porém o SENHOR disse a Davi, meu pai: Já que desejaste edificar uma casa ao meu nome, bem fizeste em o resolver em teu coração.
\par 19 Todavia, tu não edificarás a casa, porém teu filho, que descenderá de ti, ele a edificará ao meu nome.
\par 20 Assim, cumpriu o SENHOR a sua palavra que tinha dito, pois me levantei em lugar de Davi, meu pai, e me assentei no trono de Israel, como prometera o SENHOR; e edifiquei a casa ao nome do SENHOR, o Deus de Israel.
\par 21 E nela constituí um lugar para a arca, em que estão as tábuas da aliança que o SENHOR fez com nossos pais, quando os tirou da terra do Egito.
\par 22 Pôs-se Salomão diante do altar do SENHOR, na presença de toda a congregação de Israel; e estendeu as mãos para os céus
\par 23 e disse: Ó SENHOR, Deus de Israel, não há Deus como tu, em cima nos céus nem embaixo na terra, como tu que guardas a aliança e a misericórdia a teus servos que de todo o coração andam diante de ti;
\par 24 que cumpriste para com teu servo Davi, meu pai, o que lhe prometeste; pessoalmente o disseste e pelo teu poder o cumpriste, como hoje se vê.
\par 25 Agora, pois, ó SENHOR, Deus de Israel, faze a teu servo Davi, meu pai, o que lhe declaraste, dizendo: Não te faltará sucessor diante de mim, que se assente no trono de Israel, contanto que teus filhos guardem o seu caminho, para andarem diante de mim como tu andaste.
\par 26 Agora também, ó Deus de Israel, cumpra-se a tua palavra que disseste a teu servo Davi, meu pai.
\par 27 Mas, de fato, habitaria Deus na terra? Eis que os céus e até o céu dos céus não te podem conter, quanto menos esta casa que eu edifiquei.
\par 28 Atenta, pois, para a oração de teu servo e para a sua súplica, ó SENHOR, meu Deus, para ouvires o clamor e a oração que faz, hoje, o teu servo diante de ti.
\par 29 Para que os teus olhos estejam abertos noite e dia sobre esta casa, sobre este lugar, do qual disseste: O meu nome estará ali; para ouvires a oração que o teu servo fizer neste lugar.
\par 30 Ouve, pois, a súplica do teu servo e do teu povo de Israel, quando orarem neste lugar; ouve no céu, lugar da tua habitação; ouve e perdoa.
\par 31 Se alguém pecar contra o seu próximo, e lhe for exigido que jure, e ele vier a jurar diante do teu altar, nesta casa,
\par 32 ouve tu nos céus, e age, e julga teus servos, condenando o perverso, fazendo recair o seu proceder sobre a sua cabeça e justificando ao justo, para lhe retribuíres segundo a sua justiça.
\par 33 Quando o teu povo de Israel, por ter pecado contra ti, for ferido diante do inimigo, e se converter a ti, e confessar o teu nome, e orar, e suplicar a ti, nesta casa,
\par 34 ouve tu nos céus, e perdoa o pecado do teu povo de Israel, e faze-o voltar à terra que deste a seus pais.
\par 35 Quando os céus se cerrarem, e não houver chuva, por ter o povo pecado contra ti, e orar neste lugar, e confessar o teu nome, e se converter dos seus pecados, havendo-o tu afligido,
\par 36 ouve tu nos céus, perdoa o pecado de teus servos e do teu povo de Israel, ensinando-lhes o bom caminho em que andem, e dá chuva na tua terra, que deste em herança ao teu povo.
\par 37 Quando houver fome na terra ou peste, quando houver crestamento ou ferrugem, gafanhotos e larvas, quando o seu inimigo o cercar em qualquer das suas cidades ou houver alguma praga ou doença,
\par 38 toda oração e súplica que qualquer homem ou todo o teu povo de Israel fizer, conhecendo cada um a chaga do seu coração e estendendo as mãos para o rumo desta casa,
\par 39 ouve tu nos céus, lugar da tua habitação, perdoa, age e dá a cada um segundo todos os seus caminhos, já que lhe conheces o coração, porque tu, só tu, és conhecedor do coração de todos os filhos dos homens;
\par 40 para que te temam todos os dias que viverem na terra que deste a nossos pais.
\par 41 Também ao estrangeiro, que não for do teu povo de Israel, porém vier de terras remotas, por amor do teu nome
\par 42 (porque ouvirão do teu grande nome, e da tua mão poderosa, e do teu braço estendido), e orar, voltado para esta casa,
\par 43 ouve tu nos céus, lugar da tua habitação, e faze tudo o que o estrangeiro te pedir, a fim de que todos os povos da terra conheçam o teu nome, para te temerem como o teu povo de Israel e para saberem que esta casa, que eu edifiquei, é chamada pelo teu nome.
\par 44 Quando o teu povo sair à guerra contra o seu inimigo, pelo caminho por que o enviares, e orar ao SENHOR, voltado para esta cidade, que tu escolheste, e para a casa, que edifiquei ao teu nome,
\par 45 ouve tu nos céus a sua oração e a sua súplica e faze-lhe justiça.
\par 46 Quando pecarem contra ti (pois não há homem que não peque), e tu te indignares contra eles, e os entregares às mãos do inimigo, a fim de que os leve cativos à terra inimiga, longe ou perto esteja;
\par 47 e, na terra aonde forem levados cativos, caírem em si, e se converterem, e, na terra do seu cativeiro, te suplicarem, dizendo: Pecamos, e perversamente procedemos, e cometemos iniqüidade;
\par 48 e se converterem a ti de todo o seu coração e de toda a sua alma, na terra de seus inimigos que os levarem cativos, e orarem a ti, voltados para a sua terra, que deste a seus pais, para esta cidade que escolheste e para a casa que edifiquei ao teu nome;
\par 49 ouve tu nos céus, lugar da tua habitação, a sua prece e a sua súplica e faze-lhes justiça,
\par 50 perdoa o teu povo, que houver pecado contra ti, todas as suas transgressões que houverem cometido contra ti; e move tu à compaixão os que os levaram cativos para que se compadeçam deles.
\par 51 Porque é o teu povo e a tua herança, que tiraste da terra do Egito, do meio do forno de ferro;
\par 52 para que teus olhos estejam abertos à súplica do teu servo e à súplica do teu povo de Israel, a fim de os ouvires em tudo quanto clamarem a ti.
\par 53 Pois tu, ó SENHOR Deus, os separaste dentre todos os povos da terra para tua herança, como falaste por intermédio do teu servo Moisés, quando tiraste do Egito a nossos pais.
\par 54 Tendo Salomão acabado de fazer ao SENHOR toda esta oração e súplica, estando de joelhos e com as mãos estendidas para os céus, se levantou de diante do altar do SENHOR,
\par 55 pôs-se em pé e abençoou a toda a congregação de Israel em alta voz, dizendo:
\par 56 Bendito seja o SENHOR, que deu repouso ao seu povo de Israel, segundo tudo o que prometera; nem uma só palavra falhou de todas as suas boas promessas, feitas por intermédio de Moisés, seu servo.
\par 57 O SENHOR, nosso Deus, seja conosco, assim como foi com nossos pais; não nos desampare e não nos deixe;
\par 58 a fim de que a si incline o nosso coração, para andarmos em todos os seus caminhos e guardarmos os seus mandamentos, e os seus estatutos, e os seus juízos, que ordenou a nossos pais.
\par 59 Que estas minhas palavras, com que supliquei perante o SENHOR, estejam presentes, diante do SENHOR, nosso Deus, de dia e de noite, para que faça ele justiça ao seu servo e ao seu povo de Israel, segundo cada dia o exigir,
\par 60 para que todos os povos da terra saibam que o SENHOR é Deus e que não há outro.
\par 61 Seja perfeito o vosso coração para com o SENHOR, nosso Deus, para andardes nos seus estatutos e guardardes os seus mandamentos, como hoje o fazeis.
\par 62 E o rei e todo o Israel com ele ofereceram sacrifícios diante do SENHOR.
\par 63 Ofereceu Salomão em sacrifício pacífico o que apresentou ao SENHOR, vinte e dois mil bois e cento e vinte mil ovelhas. Assim, o rei e todos os filhos de Israel consagraram a Casa do SENHOR.
\par 64 No mesmo dia, consagrou o rei o meio do átrio que estava diante da Casa do SENHOR; porquanto ali preparara os holocaustos e as ofertas com a gordura dos sacrifícios pacíficos; porque o altar de bronze que estava diante do SENHOR era muito pequeno para nele caberem os holocaustos, as ofertas de manjares e a gordura dos sacrifícios pacíficos.
\par 65 No mesmo tempo, celebrou Salomão também a Festa dos Tabernáculos e todo o Israel com ele, uma grande congregação, desde a entrada de Hamate até ao rio do Egito, perante o SENHOR, nosso Deus; por sete dias além dos primeiros sete, a saber, catorze dias.
\par 66 No oitavo dia desta festa, despediu o povo, e eles abençoaram o rei; então, se foram às suas tendas, alegres e de coração contente por causa de todo o bem que o SENHOR fizera a Davi, seu servo, e a Israel, seu povo.

\chapter{9}

\par 1 Sucedeu, pois, que, tendo acabado Salomão de edificar a Casa do SENHOR, e a casa do rei, e tudo o que tinha desejado e designara fazer,
\par 2 o SENHOR tornou a aparecer-lhe, como lhe tinha aparecido em Gibeão,
\par 3 e o SENHOR lhe disse: Ouvi a tua oração e a tua súplica que fizeste perante mim; santifiquei a casa que edificaste, a fim de pôr ali o meu nome para sempre; os meus olhos e o meu coração estarão ali todos os dias.
\par 4 Se andares perante mim como andou Davi, teu pai, com integridade de coração e com sinceridade, para fazeres segundo tudo o que te mandei e guardares os meus estatutos e os meus juízos,
\par 5 então, confirmarei o trono de teu reino sobre Israel para sempre, como falei acerca de Davi, teu pai, dizendo: Não te faltará sucessor sobre o trono de Israel.
\par 6 Porém, se vós e vossos filhos, de qualquer maneira, vos apartardes de mim e não guardardes os meus mandamentos e os meus estatutos, que vos prescrevi, mas fordes, e servirdes a outros deuses, e os adorardes,
\par 7 então, eliminarei Israel da terra que lhe dei, e a esta casa, que santifiquei a meu nome, lançarei longe da minha presença; e Israel virá a ser provérbio e motejo entre todos os povos.
\par 8 E desta casa, agora tão exaltada, todo aquele que por ela passar pasmará, e assobiará, e dirá: Por que procedeu o SENHOR assim para com esta terra e esta casa?
\par 9 Responder-se-lhe-á: Porque deixaram o SENHOR, seu Deus, que tirou da terra do Egito os seus pais, e se apegaram a outros deuses, e os adoraram, e os serviram. Por isso, trouxe o SENHOR sobre eles todo este mal.
\par 10 Ao fim de vinte anos, terminara Salomão as duas casas, a Casa do SENHOR e a casa do rei.
\par 11 Ora, como Hirão, rei de Tiro, trouxera a Salomão madeira de cedro e de cipreste e ouro, segundo todo o seu desejo, este lhe deu vinte cidades na terra da Galiléia.
\par 12 Saiu Hirão de Tiro a ver as cidades que Salomão lhe dera, porém não lhe agradaram.
\par 13 Pelo que disse: Que cidades são estas que me deste, irmão meu? E lhes chamaram Terra de Cabul, até hoje.
\par 14 Hirão tinha enviado ao rei cento e vinte talentos de ouro.
\par 15 A razão por que Salomão impôs o trabalho forçado é esta: edificar a Casa do SENHOR, e a sua própria casa, e Milo, e o muro de Jerusalém, como também Hazor, e Megido, e Gezer;
\par 16 porque Faraó, rei do Egito, subira, e tomara a Gezer, e a queimara, e matara os cananeus que moravam nela, e com ela dotara a sua filha, mulher de Salomão.
\par 17 Assim, edificou Salomão Gezer, Bete-Horom, a baixa,
\par 18 Baalate, Tadmor, no deserto daquela terra,
\par 19 todas as cidades-armazéns que Salomão tinha, as cidades para os carros, as cidades para os cavaleiros e o que desejou enfim edificar em Jerusalém, no Líbano e em toda a terra do seu domínio.
\par 20 Quanto a todo o povo que restou dos amorreus, heteus, ferezeus, heveus e jebuseus, e que não eram dos filhos de Israel,
\par 21 a seus filhos, que restaram depois deles na terra, os quais os filhos de Israel não puderam destruir totalmente, a esses fez Salomão trabalhadores forçados, até hoje.
\par 22 Porém dos filhos de Israel não fez Salomão escravo algum; eram homens de guerra, e seus oficiais, e seus príncipes, e seus capitães, e chefes dos seus carros e dos seus cavalarianos.
\par 23 Os principais oficiais que estavam sobre a obra de Salomão eram quinhentos e cinqüenta; tinham estes a seu cargo o povo que trabalhava na obra.
\par 24 Subiu, porém, a filha de Faraó da Cidade de Davi à sua casa, que Salomão lhe edificara; então, edificou a Milo.
\par 25 Oferecia Salomão, três vezes por ano, holocaustos e sacrifícios pacíficos sobre o altar que edificara ao SENHOR e queimava incenso sobre o altar perante o SENHOR. Assim, acabou ele a casa.
\par 26 Fez o rei Salomão também naus em Eziom-Geber, que está junto a Elate, na praia do mar Vermelho, na terra de Edom.
\par 27 Mandou Hirão, com aquelas naus, os seus servos, marinheiros, conhecedores do mar, com os servos de Salomão.
\par 28 Chegaram a Ofir e tomaram de lá quatrocentos e vinte talentos de ouro, que trouxeram ao rei Salomão.

\chapter{10}

\par 1 Tendo a rainha de Sabá ouvido a fama de Salomão, com respeito ao nome do SENHOR, veio prová-lo com perguntas difíceis.
\par 2 Chegou a Jerusalém com mui grande comitiva; com camelos carregados de especiarias, e muitíssimo ouro, e pedras preciosas; compareceu perante Salomão e lhe expôs tudo quanto trazia em sua mente.
\par 3 Salomão lhe deu resposta a todas as perguntas, e nada lhe houve profundo demais que não pudesse explicar.
\par 4 Vendo, pois, a rainha de Sabá toda a sabedoria de Salomão, e a casa que edificara,
\par 5 e a comida da sua mesa, e o lugar dos seus oficiais, e o serviço dos seus criados, e os trajes deles, e seus copeiros, e o holocausto que oferecia na Casa do SENHOR, ficou como fora de si
\par 6 e disse ao rei: Foi verdade a palavra que a teu respeito ouvi na minha terra e a respeito da tua sabedoria.
\par 7 Eu, contudo, não cria naquelas palavras, até que vim e vi com os meus próprios olhos. Eis que não me contaram a metade: sobrepujas em sabedoria e prosperidade a fama que ouvi.
\par 8 Felizes os teus homens, felizes estes teus servos, que estão sempre diante de ti e que ouvem a tua sabedoria!
\par 9 Bendito seja o SENHOR, teu Deus, que se agradou de ti para te colocar no trono de Israel; é porque o SENHOR ama a Israel para sempre, que te constituiu rei, para executares juízo e justiça.
\par 10 Deu ela ao rei cento e vinte talentos de ouro, e muitíssimas especiarias, e pedras preciosas; nunca mais veio especiaria em tanta abundância, como a que a rainha de Sabá ofereceu ao rei Salomão.
\par 11 Também as naus de Hirão, que de Ofir transportavam ouro, traziam de lá grande quantidade de madeira de sândalo e pedras preciosas.
\par 12 Desta madeira de sândalo, fez o rei balaústres para a Casa do SENHOR e para a casa real, como também harpas e alaúdes para os cantores; tal madeira nunca se havia trazido para ali, nem se viu mais semelhante madeira até ao dia de hoje.
\par 13 O rei Salomão deu à rainha de Sabá tudo quanto ela desejou e pediu, afora tudo o que lhe deu por sua generosidade real. Assim, voltou e se foi para a sua terra, com os seus servos.
\par 14 O peso do ouro que se trazia a Salomão cada ano era de seiscentos e sessenta e seis talentos de ouro,
\par 15 além do que entrava dos vendedores, e do tráfico dos negociantes, e de todos os reis da Arábia, e dos governadores da terra.
\par 16 Fez o rei Salomão duzentos paveses de ouro batido; seiscentos siclos de ouro mandou pesar para cada pavês;
\par 17 fez também trezentos escudos de ouro batido; três arráteis de ouro mandou pesar para cada escudo. E o rei os pôs na Casa do Bosque do Líbano.
\par 18 Fez mais o rei um grande trono de marfim e o cobriu de ouro puríssimo.
\par 19 O trono tinha seis degraus; o espaldar do trono, ao alto, era redondo; de ambos os lados tinha braços junto ao assento e dois leões junto aos braços.
\par 20 Também doze leões estavam ali sobre os seis degraus, um em cada extremo destes. Nunca se fizera obra semelhante em nenhum dos reinos.
\par 21 Todas as taças de que se servia o rei Salomão para beber eram de ouro, e também de ouro puro todas as da Casa do Bosque do Líbano; não havia nelas prata, porque nos dias de Salomão não se dava a ela estimação nenhuma.
\par 22 Porque o rei tinha no mar uma frota de Társis, com as naus de Hirão; de três em três anos, voltava a frota de Társis, trazendo ouro, prata, marfim, bugios e pavões.
\par 23 Assim, o rei Salomão excedeu a todos os reis do mundo, tanto em riqueza como em sabedoria.
\par 24 Todo o mundo procurava ir ter com ele para ouvir a sabedoria que Deus lhe pusera no coração.
\par 25 Cada um trazia o seu presente: objetos de prata e de ouro, roupas, armaduras, especiarias, cavalos e mulas; assim, ano após ano.
\par 26 Também ajuntou Salomão carros e cavaleiros, tinha mil e quatrocentos carros e doze mil cavalarianos, que distribuiu às cidades para os carros e junto ao rei, em Jerusalém.
\par 27 Fez o rei que, em Jerusalém, houvesse prata como pedras e cedros em abundância como os sicômoros que estão nas planícies.
\par 28 Os cavalos de Salomão vinham do Egito e da Cilícia; e comerciantes do rei os recebiam da Cilícia por certo preço.
\par 29 Importava-se, do Egito, um carro por seiscentos siclos de prata e um cavalo por cento e cinqüenta; nas mesmas condições, as caravanas os traziam e os exportavam para todos os reis dos heteus e para os reis da Síria.

\chapter{11}

\par 1 Ora, além da filha de Faraó, amou Salomão muitas mulheres estrangeiras: moabitas, amonitas, edomitas, sidônias e hetéias,
\par 2 mulheres das nações de que havia o SENHOR dito aos filhos de Israel: Não caseis com elas, nem casem elas convosco, pois vos perverteriam o coração, para seguirdes os seus deuses. A estas se apegou Salomão pelo amor.
\par 3 Tinha setecentas mulheres, princesas e trezentas concubinas; e suas mulheres lhe perverteram o coração.
\par 4 Sendo já velho, suas mulheres lhe perverteram o coração para seguir outros deuses; e o seu coração não era de todo fiel para com o SENHOR, seu Deus, como fora o de Davi, seu pai.
\par 5 Salomão seguiu a Astarote, deusa dos sidônios, e a Milcom, abominação dos amonitas.
\par 6 Assim, fez Salomão o que era mau perante o SENHOR e não perseverou em seguir ao SENHOR, como Davi, seu pai.
\par 7 Nesse tempo, edificou Salomão um santuário a Quemos, abominação de Moabe, sobre o monte fronteiro a Jerusalém, e a Moloque, abominação dos filhos de Amom.
\par 8 Assim fez para com todas as suas mulheres estrangeiras, as quais queimavam incenso e sacrificavam a seus deuses.
\par 9 Pelo que o SENHOR se indignou contra Salomão, pois desviara o seu coração do SENHOR, Deus de Israel, que duas vezes lhe aparecera.
\par 10 E acerca disso lhe tinha ordenado que não seguisse a outros deuses. Ele, porém, não guardou o que o SENHOR lhe ordenara.
\par 11 Por isso, disse o SENHOR a Salomão: Visto que assim procedeste e não guardaste a minha aliança, nem os meus estatutos que te mandei, tirarei de ti este reino e o darei a teu servo.
\par 12 Contudo, não o farei nos teus dias, por amor de Davi, teu pai; da mão de teu filho o tirarei.
\par 13 Todavia, não tirarei o reino todo; darei uma tribo a teu filho, por amor de Davi, meu servo, e por amor de Jerusalém, que escolhi.
\par 14 Levantou o SENHOR contra Salomão um adversário, Hadade, o edomita; este era da linhagem real de Edom.
\par 15 Porque, estando Davi em Edom e tendo subido Joabe, comandante do exército, a sepultar os mortos, feriu todos os varões em Edom.
\par 16 (Porque Joabe ficou ali seis meses com todo o Israel, até que eliminou todos os varões em Edom.)
\par 17 Hadade, porém, fugiu, e, com ele, alguns homens edomitas, dos servos de seu pai, para ir ao Egito; era Hadade ainda muito jovem.
\par 18 Partiram de Midiã e seguiram a Parã, de onde tomaram consigo homens e chegaram ao Egito, a Faraó, rei do Egito, o qual deu a Hadade uma casa, e lhe prometeu sustento, e lhe deu terras.
\par 19 Achou Hadade grande mercê por parte de Faraó, tanta que este lhe deu por mulher a irmã de sua própria mulher,
\par 20 a irmã de Tafnes, a rainha. A irmã de Tafnes deu-lhe à luz seu filho Genubate, o qual Tafnes criou na casa de Faraó, onde Genubate ficou entre os filhos de Faraó.
\par 21 Tendo, pois, Hadade ouvido no Egito que Davi descansara com seus pais e que Joabe, comandante do exército, era morto, disse a Faraó: Deixa-me voltar para a minha terra.
\par 22 Então, Faraó lhe disse: Pois que te falta comigo, que procuras partir para a tua terra? Respondeu ele: Nada; porém deixa-me ir.
\par 23 Também Deus levantou a Salomão outro adversário, Rezom, filho de Eliada, que havia fugido de seu senhor Hadadezer, rei de Zobá.
\par 24 Ele ajuntou homens e se fez capitão de um bando; depois do morticínio feito por Davi, eles se foram para Damasco, onde habitaram e onde constituíram rei a Rezom.
\par 25 Este foi adversário de Israel por todos os dias de Salomão, fez-lhe mal como Hadade, detestava a Israel e reinava sobre a Síria.
\par 26 Jeroboão, filho de Nebate, efraimita de Zereda, servo de Salomão, e cuja mãe era mulher viúva, por nome Zerua, levantou a mão contra o rei.
\par 27 Esta foi a causa por que levantou a mão contra o rei: Salomão estava edificando a Milo e terraplenando depressões da Cidade de Davi, seu pai.
\par 28 Ora, vendo Salomão que Jeroboão era homem valente e capaz, moço laborioso, ele o pôs sobre todo o trabalho forçado da casa de José.
\par 29 Sucedeu, nesse tempo, que, saindo Jeroboão de Jerusalém, o encontrou o profeta Aías, o silonita, no caminho; este se tinha vestido de uma capa nova, e estavam sós os dois no campo.
\par 30 Aías pegou na capa nova que tinha sobre si, rasgou-a em doze pedaços
\par 31 e disse a Jeroboão: Toma dez pedaços, porque assim diz o SENHOR, Deus de Israel: Eis que rasgarei o reino da mão de Salomão, e a ti darei dez tribos.
\par 32 Porém ele terá uma tribo, por amor de Davi, meu servo, e por amor de Jerusalém, a cidade que escolhi de todas as tribos de Israel.
\par 33 Porque Salomão me deixou e se encurvou a Astarote, deusa dos sidônios, a Quemos, deus de Moabe, e a Milcom, deus dos filhos de Amom; e não andou nos meus caminhos para fazer o que é reto perante mim, a saber, os meus estatutos e os meus juízos, como fez Davi, seu pai.
\par 34 Porém não tomarei da sua mão o reino todo; pelo contrário, fá-lo-ei príncipe todos os dias da sua vida, por amor de Davi, meu servo, a quem elegi, porque guardou os meus mandamentos e os meus estatutos.
\par 35 Mas da mão de seu filho tomarei o reino, a saber, as dez tribos, e tas darei a ti.
\par 36 E a seu filho darei uma tribo; para que Davi, meu servo, tenha sempre uma lâmpada diante de mim em Jerusalém, a cidade que escolhi para pôr ali o meu nome.
\par 37 Tomar-te-ei, e reinarás sobre tudo o que desejar a tua alma; e serás rei sobre Israel.
\par 38 Se ouvires tudo o que eu te ordenar, e andares nos meus caminhos, e fizeres o que é reto perante mim, guardando os meus estatutos e os meus mandamentos, como fez Davi, meu servo, eu serei contigo, e te edificarei uma casa estável, como edifiquei a Davi, e te darei Israel.
\par 39 Por isso, afligirei a descendência de Davi; todavia, não para sempre.
\par 40 Pelo que Salomão procurou matar a Jeroboão; este, porém, se dispôs e fugiu para o Egito, a ter com Sisaque, rei do Egito; e ali permaneceu até à morte de Salomão.
\par 41 Quanto aos mais atos de Salomão, a tudo quanto fez, e à sua sabedoria, porventura, não estão escritos no Livro da História de Salomão?
\par 42 Foi de quarenta anos o tempo que reinou Salomão em Jerusalém sobre todo o Israel.
\par 43 Descansou com seus pais e foi sepultado na Cidade de Davi, seu pai; e Roboão, seu filho, reinou em seu lugar.

\chapter{12}

\par 1 Foi Roboão a Siquém, porque todo o Israel se reuniu lá, para o fazer rei.
\par 2 Tendo Jeroboão, filho de Nebate, ouvido isso (pois estava ainda no Egito, para onde fugira da presença do rei Salomão, onde habitava
\par 3 e donde o mandaram chamar), veio com toda a congregação de Israel a Roboão, e lhe falaram:
\par 4 Teu pai fez pesado o nosso jugo; agora, pois, alivia tu a dura servidão de teu pai e o seu pesado jugo que nos impôs, e nós te serviremos.
\par 5 Ele lhes respondeu: Ide-vos e, após três dias, voltai a mim. E o povo se foi.
\par 6 Tomou o rei Roboão conselho com os homens idosos que estiveram na presença de Salomão, seu pai, quando este ainda vivia, dizendo: Como aconselhais que se responda a este povo?
\par 7 Eles lhe disseram: Se, hoje, te tornares servo deste povo, e o servires, e, atendendo, falares boas palavras, eles se farão teus servos para sempre.
\par 8 Porém ele desprezou o conselho que os anciãos lhe tinham dado e tomou conselho com os jovens que haviam crescido com ele e o serviam.
\par 9 E disse-lhes: Que aconselhais vós que respondamos a este povo que me falou, dizendo: Alivia o jugo que teu pai nos impôs?
\par 10 E os jovens que haviam crescido com ele lhe disseram: Assim falarás a este povo que disse: Teu pai fez pesado o nosso jugo, mas tu alivia-o de sobre nós; assim lhe falarás: Meu dedo mínimo é mais grosso do que os lombos de meu pai.
\par 11 Assim que, se meu pai vos impôs jugo pesado, eu ainda vo-lo aumentarei; meu pai vos castigou com açoites, porém eu vos castigarei com escorpiões.
\par 12 Veio, pois, Jeroboão e todo o povo, ao terceiro dia, a Roboão, como o rei lhes ordenara, dizendo: Voltai a mim ao terceiro dia.
\par 13 Dura resposta deu o rei ao povo, porque desprezara o conselho que os anciãos lhe haviam dado;
\par 14 e lhe falou segundo o conselho dos jovens, dizendo: Meu pai fez pesado o vosso jugo, porém eu ainda o agravarei; meu pai vos castigou com açoites; eu, porém, vos castigarei com escorpiões.
\par 15 O rei, pois, não deu ouvidos ao povo; porque este acontecimento vinha do SENHOR, para confirmar a palavra que o SENHOR tinha dito por intermédio de Aías, o silonita, a Jeroboão, filho de Nebate.
\par 16 Vendo, pois, todo o Israel que o rei não lhe dava ouvidos, reagiu, dizendo: Que parte temos nós com Davi? Não há para nós herança no filho de Jessé! Às vossas tendas, ó Israel! Cuida, agora, da tua casa, ó Davi! Então, Israel se foi às suas tendas.
\par 17 Quanto aos filhos de Israel, porém, que habitavam nas cidades de Judá, sobre eles reinou Roboão.
\par 18 Então, o rei Roboão enviou a Adorão, superintendente dos que trabalhavam forçados; porém todo o Israel o apedrejou, e morreu. Mas o rei Roboão conseguiu tomar o seu carro e fugir para Jerusalém.
\par 19 Assim, Israel se mantém rebelado contra a casa de Davi, até ao dia de hoje.
\par 20 Tendo ouvido todo o Israel que Jeroboão tinha voltado, mandaram chamá-lo para a congregação e o fizeram rei sobre todo o Israel; ninguém seguiu a casa de Davi, senão somente a tribo de Judá.
\par 21 Vindo, pois, Roboão a Jerusalém, reuniu toda a casa de Judá e a tribo de Benjamim, cento e oitenta mil escolhidos, destros para a guerra, para pelejar contra a casa de Israel, a fim de restituir o reino a Roboão, filho de Salomão.
\par 22 Porém veio a palavra de Deus a Semaías, homem de Deus, dizendo:
\par 23 Fala a Roboão, filho de Salomão, rei de Judá, e a toda a casa de Judá, e a Benjamim, e ao resto do povo, dizendo:
\par 24 Assim diz o SENHOR: Não subireis, nem pelejareis contra vossos irmãos, os filhos de Israel; cada um volte para a sua casa, porque eu é que fiz isto. E, obedecendo eles à palavra do SENHOR, voltaram como este lhes ordenara.
\par 25 Jeroboão edificou Siquém, na região montanhosa de Efraim, e passou a residir ali; dali edificou Penuel.
\par 26 Disse Jeroboão consigo: Agora, tornará o reino para a casa de Davi.
\par 27 Se este povo subir para fazer sacrifícios na Casa do SENHOR, em Jerusalém, o coração dele se tornará a seu senhor, a Roboão, rei de Judá; e me matarão e tornarão a ele, ao rei de Judá.
\par 28 Pelo que o rei, tendo tomado conselhos, fez dois bezerros de ouro; e disse ao povo: Basta de subirdes a Jerusalém; vês aqui teus deuses, ó Israel, que te fizeram subir da terra do Egito!
\par 29 Pôs um em Betel e o outro, em Dã.
\par 30 E isso se tornou em pecado, pois que o povo ia até Dã, cada um para adorar o bezerro.
\par 31 Jeroboão fez também santuários nos altos e, dentre o povo, constituiu sacerdotes que não eram dos filhos de Levi.
\par 32 Fez uma festa no oitavo mês, no dia décimo quinto do mês, igual à festa que se fazia em Judá, e sacrificou no altar; semelhantemente fez em Betel e ofereceu sacrifícios aos bezerros que fizera; também em Betel estabeleceu sacerdotes dos altos que levantara.
\par 33 No décimo quinto dia do oitavo mês, escolhido a seu bel-prazer, subiu ele ao altar que fizera em Betel e ordenou uma festa para os filhos de Israel; subiu para queimar incenso.

\chapter{13}

\par 1 Eis que, por ordem do SENHOR, veio de Judá a Betel um homem de Deus; e Jeroboão estava junto ao altar, para queimar incenso.
\par 2 Clamou o profeta contra o altar, por ordem do SENHOR, e disse: Altar, altar! Assim diz o SENHOR: Eis que um filho nascerá à casa de Davi, cujo nome será Josias, o qual sacrificará sobre ti os sacerdotes dos altos que queimam sobre ti incenso, e ossos humanos se queimarão sobre ti.
\par 3 Deu, naquele mesmo dia, um sinal, dizendo: Este é o sinal de que o SENHOR falou: Eis que o altar se fenderá, e se derramará a cinza que há sobre ele.
\par 4 Tendo o rei ouvido as palavras do homem de Deus, que clamara contra o altar de Betel, Jeroboão estendeu a mão de sobre o altar, dizendo: Prendei-o! Mas a mão que estendera contra o homem de Deus secou, e não a podia recolher.
\par 5 O altar se fendeu, e a cinza se derramou do altar, segundo o sinal que o homem de Deus apontara por ordem do SENHOR.
\par 6 Então, disse o rei ao homem de Deus: Implora o favor do SENHOR, teu Deus, e ora por mim, para que eu possa recolher a mão. Então, o homem de Deus implorou o favor do SENHOR, e a mão do rei se lhe recolheu e ficou como dantes.
\par 7 Disse o rei ao homem de Deus: Vem comigo a casa e fortalece-te; e eu te recompensarei.
\par 8 Porém o homem de Deus disse ao rei: Ainda que me desses metade da tua casa, não iria contigo, nem comeria pão, nem beberia água neste lugar.
\par 9 Porque assim me ordenou o SENHOR pela sua palavra, dizendo: Não comerás pão, nem beberás água; e não voltarás pelo caminho por onde foste.
\par 10 E se foi por outro caminho; e não voltou pelo caminho por onde viera a Betel.
\par 11 Morava em Betel um profeta velho; vieram seus filhos e lhe contaram tudo o que o homem de Deus fizera aquele dia em Betel; as palavras que dissera ao rei, contaram-nas a seu pai.
\par 12 Perguntou-lhes o pai: Por que caminho se foi? Mostraram seus filhos o caminho por onde fora o homem de Deus que viera de Judá.
\par 13 Então, disse a seus filhos: Albardai-me um jumento. Albardaram-lhe o jumento, e ele montou.
\par 14 E foi após o homem de Deus e, achando-o sentado debaixo de um carvalho, lhe disse: És tu o homem de Deus que vieste de Judá? Ele respondeu: Eu mesmo.
\par 15 Então, lhe disse: Vem comigo a casa e come pão.
\par 16 Porém ele disse: Não posso voltar contigo, nem entrarei contigo; não comerei pão, nem beberei água contigo neste lugar.
\par 17 Porque me foi dito pela palavra do SENHOR: Ali, não comerás pão, nem beberás água, nem voltarás pelo caminho por que foste.
\par 18 Tornou-lhe ele: Também eu sou profeta como tu, e um anjo me falou por ordem do SENHOR, dizendo: Faze-o voltar contigo a tua casa, para que coma pão e beba água. (Porém mentiu-lhe.)
\par 19 Então, voltou ele, e comeu pão em sua casa, e bebeu água.
\par 20 Estando eles à mesa, veio a palavra do SENHOR ao profeta que o tinha feito voltar;
\par 21 e clamou ao homem de Deus, que viera de Judá, dizendo: Assim diz o SENHOR: Porquanto foste rebelde à palavra do SENHOR e não guardaste o mandamento que o SENHOR, teu Deus, te mandara,
\par 22 antes, voltaste, e comeste pão, e bebeste água no lugar de que te dissera: Não comerás pão, nem beberás água, o teu cadáver não entrará no sepulcro de teus pais.
\par 23 Depois de o profeta a quem fizera voltar haver comido pão e bebido água, albardou para ele o jumento.
\par 24 Foi-se, pois, e um leão o encontrou no caminho e o matou; o seu cadáver estava atirado no caminho, e o jumento e o leão, parados junto ao cadáver.
\par 25 Eis que os homens passaram e viram o corpo lançado no caminho, como também o leão parado junto ao corpo; e vieram e o disseram na cidade onde o profeta velho habitava.
\par 26 Ouvindo-o o profeta que o fizera voltar do caminho, disse: É o homem de Deus, que foi rebelde à palavra do SENHOR; por isso, o SENHOR o entregou ao leão, que o despedaçou e matou, segundo a palavra que o SENHOR lhe tinha dito.
\par 27 Então, disse a seus filhos: Albardai-me o jumento. Eles o albardaram.
\par 28 Ele se foi e achou o cadáver atirado no caminho e o jumento e o leão, parados junto ao cadáver; o leão não tinha devorado o corpo, nem despedaçado o jumento.
\par 29 Então, o profeta levantou o cadáver do homem de Deus, e o pôs sobre o jumento, e o tornou a levar; assim, veio o profeta velho à cidade, para o chorar e enterrar.
\par 30 Depositou o cadáver no seu próprio sepulcro; e o prantearam, dizendo: Ah! Irmão meu!
\par 31 Depois de o haver sepultado, disse a seus filhos: Quando eu morrer, enterrai-me no sepulcro em que o homem de Deus está sepultado; ponde os meus ossos junto aos ossos dele.
\par 32 Porque certamente se cumprirá o que por ordem do SENHOR clamou contra o altar que está em Betel e contra todas as casas dos altos que estão nas cidades de Samaria.
\par 33 Depois destas coisas, Jeroboão ainda não deixou o seu mau caminho; antes, de entre o povo tornou a constituir sacerdotes para lugares altos; a quem queria, consagrava para sacerdote dos lugares altos.
\par 34 Isso se tornou em pecado à casa de Jeroboão, para destruí-la e extingui-la da terra.

\chapter{14}

\par 1 Naquele tempo, adoeceu Abias, filho de Jeroboão.
\par 2 Disse este a sua mulher: Dispõe-te, agora, e disfarça-te, para que não conheçam que és mulher de Jeroboão; e vai a Siló. Eis que lá está o profeta Aías, o qual a meu respeito disse que eu seria rei sobre este povo.
\par 3 Leva contigo dez pães, bolos e uma botija de mel e vai ter com ele; ele te dirá o que há de suceder a este menino.
\par 4 A mulher de Jeroboão assim o fez; levantou-se, foi a Siló e entrou na casa de Aías; Aías já não podia ver, porque os seus olhos já se tinham escurecido, por causa da sua velhice.
\par 5 Porém o SENHOR disse a Aías: Eis que a mulher de Jeroboão vem consultar-te sobre seu filho, que está doente. Assim e assim lhe falarás, porque, ao entrar, fingirá ser outra.
\par 6 Ouvindo Aías o ruído de seus pés, quando ela entrava pela porta, disse: Entra, mulher de Jeroboão; por que finges assim? Pois estou encarregado de te dizer duras novas.
\par 7 Vai e dize a Jeroboão: Assim diz o SENHOR, Deus de Israel: Porquanto te levantei do meio do povo, e te fiz príncipe sobre o meu povo de Israel,
\par 8 e tirei o reino da casa de Davi, e to entreguei, e tu não foste como Davi, meu servo, que guardou os meus mandamentos e andou após mim de todo o seu coração, para fazer somente o que parecia reto aos meus olhos;
\par 9 antes, fizeste o mal, pior do que todos os que foram antes de ti, e fizeste outros deuses e imagens de fundição, para provocar-me à ira, e me viraste as costas;
\par 10 portanto, eis que trarei o mal sobre a casa de Jeroboão, e eliminarei de Jeroboão todo e qualquer do sexo masculino, tanto o escravo como o livre, e lançarei fora os descendentes da casa de Jeroboão, como se lança fora o esterco, até que, de todo, ela se acabe.
\par 11 Quem morrer a Jeroboão na cidade, os cães o comerão, e o que morrer no campo aberto, as aves do céu o comerão, porque o SENHOR o disse.
\par 12 Tu, pois, dispõe-te e vai para tua casa; quando puseres os pés na cidade, o menino morrerá.
\par 13 Todo o Israel o pranteará e o sepultará; porque de Jeroboão só este dará entrada em sepultura, porquanto se achou nele coisa boa para com o SENHOR, Deus de Israel, em casa de Jeroboão.
\par 14 O SENHOR, porém, suscitará para si um rei sobre Israel, que eliminará, no seu dia, a casa de Jeroboão. Que digo eu? Há de ser já.
\par 15 Também o SENHOR ferirá a Israel para que se agite como a cana se agita nas águas; arrancará a Israel desta boa terra que dera a seus pais e o espalhará para além do Eufrates, porquanto fez os seus postes-ídolos, provocando o SENHOR à ira.
\par 16 Abandonará a Israel por causa dos pecados que Jeroboão cometeu e pelos que fez Israel cometer.
\par 17 Então, a mulher de Jeroboão se levantou, foi e chegou a Tirza; chegando ela ao limiar da casa, morreu o menino.
\par 18 Sepultaram-no, e todo o Israel o pranteou, segundo a palavra do SENHOR, por intermédio do profeta Aías, seu servo.
\par 19 Quanto aos mais atos de Jeroboão, como guerreou e como reinou, eis que estão escritos no Livro da História dos Reis de Israel.
\par 20 Foi de vinte e dois anos o tempo que reinou Jeroboão; e descansou com seus pais; e Nadabe, seu filho, reinou em seu lugar.
\par 21 Roboão, filho de Salomão, reinou em Judá; de quarenta e um anos de idade era Roboão quando começou a reinar e reinou dezessete anos em Jerusalém, na cidade que o SENHOR escolhera de todas as tribos de Israel, para estabelecer ali o seu nome. Naamá era o nome de sua mãe, amonita.
\par 22 Fez Judá o que era mau perante o SENHOR; e, com os pecados que cometeu, o provocou a zelo, mais do que fizeram os seus pais.
\par 23 Porque também os de Judá edificaram altos, estátuas, colunas e postes-ídolos no alto de todos os elevados outeiros e debaixo de todas as árvores verdes.
\par 24 Havia também na terra prostitutos-cultuais; fizeram segundo todas as coisas abomináveis das nações que o SENHOR expulsara de diante dos filhos de Israel.
\par 25 No quinto ano do rei Roboão, Sisaque, rei do Egito, subiu contra Jerusalém
\par 26 e tomou os tesouros da Casa do SENHOR e os tesouros da casa do rei; tomou tudo. Também levou todos os escudos de ouro que Salomão tinha feito.
\par 27 Em lugar destes, fez o rei Roboão escudos de bronze e os entregou nas mãos dos capitães da guarda, que guardavam a porta da casa do rei.
\par 28 Toda vez que o rei entrava na Casa do SENHOR, os da guarda usavam os escudos e tornavam a trazê-los para a câmara da guarda.
\par 29 Quanto aos mais dos atos de Roboão e a tudo quanto fez, porventura, não estão escritos no Livro da História dos Reis de Judá?
\par 30 Houve guerra entre Roboão e Jeroboão todos os seus dias.
\par 31 Roboão descansou com seus pais e com eles foi sepultado na Cidade de Davi. Naamá era o nome de sua mãe, amonita; e Abias, filho de Roboão, reinou em seu lugar.

\chapter{15}

\par 1 No décimo oitavo ano do rei Jeroboão, filho de Nebate, Abias começou a reinar sobre Judá.
\par 2 Três anos reinou em Jerusalém. Era o nome de sua mãe Maaca, filha de Absalão.
\par 3 Andou em todos os pecados que seu pai havia cometido antes dele; e seu coração não foi perfeito para com o SENHOR, seu Deus, como o coração de Davi, seu pai.
\par 4 Mas, por amor de Davi, o SENHOR, seu Deus, lhe deu uma lâmpada em Jerusalém, levantando a seu filho depois dele e dando estabilidade a Jerusalém.
\par 5 Porquanto Davi fez o que era reto perante o SENHOR e não se desviou de tudo quanto lhe ordenara, em todos os dias da sua vida, senão no caso de Urias, o heteu.
\par 6 Houve guerra entre Roboão e Jeroboão todos os dias da sua vida.
\par 7 Quanto aos mais atos de Abias e a tudo quanto fez, porventura, não estão escritos no Livro da História dos Reis de Judá? Também houve guerra entre Abias e Jeroboão.
\par 8 Abias descansou com seus pais, e o sepultaram na Cidade de Davi; e Asa, seu filho, reinou em seu lugar.
\par 9 No vigésimo ano de Jeroboão, rei de Israel, começou Asa a reinar sobre Judá.
\par 10 Quarenta e um anos reinou em Jerusalém. Era o nome de sua mãe Maaca, filha de Absalão.
\par 11 Asa fez o que era reto perante o SENHOR, como Davi, seu pai.
\par 12 Porque tirou da terra os prostitutos-cultuais e removeu todos os ídolos que seus pais fizeram;
\par 13 e até a Maaca, sua mãe, depôs da dignidade de rainha-mãe, porquanto ela havia feito ao poste-ídolo uma abominável imagem; pois Asa destruiu essa imagem e a queimou no vale de Cedrom;
\par 14 os altos, porém, não foram tirados; todavia, o coração de Asa foi, todos os seus dias, totalmente do SENHOR.
\par 15 Trouxe à Casa do SENHOR as coisas consagradas por seu pai e as coisas que ele mesmo consagrara: prata, ouro e objetos de utilidade.
\par 16 Houve guerra entre Asa e Baasa, rei de Israel, todos os seus dias.
\par 17 Porque Baasa, rei de Israel, subiu contra Judá e edificou a Ramá, para que a ninguém fosse permitido sair de junto de Asa, rei de Judá, nem chegar a ele.
\par 18 Então, Asa tomou toda a prata e ouro restantes nos tesouros da Casa do SENHOR e os tesouros da casa do rei e os entregou nas mãos de seus servos; e o rei Asa os enviou a Ben-Hadade, filho de Tabrimom, filho de Heziom, rei da Síria, e que habitava em Damasco, dizendo:
\par 19 Haja aliança entre mim e ti, como houve entre meu pai e teu pai. Eis que te mando um presente, prata e ouro; vai e anula a tua aliança com Baasa, rei de Israel, para que se retire de mim.
\par 20 Ben-Hadade deu ouvidos ao rei Asa e enviou os capitães dos seus exércitos contra as cidades de Israel; e feriu a Ijom, a Dã, a Abel-Bete-Maaca e todo o distrito de Quinerete, com toda a terra de Naftali.
\par 21 Ouvindo isso, Baasa deixou de edificar a Ramá e ficou em Tirza.
\par 22 Então, o rei Asa fez apregoar por toda a região de Judá que todos, sem exceção, trouxessem as pedras de Ramá e a sua madeira, com que Baasa a edificara; com elas edificou o rei Asa a Geba de Benjamim e a Mispa.
\par 23 Quanto aos mais atos de Asa, e a todo o seu poder, e a tudo quanto fez, e às cidades que edificou, porventura, não estão escritos no Livro da História dos Reis de Judá? Porém, no tempo da sua velhice, padeceu dos pés.
\par 24 Descansou Asa com seus pais e com eles foi sepultado na Cidade de Davi, seu pai; e Josafá, seu filho, reinou em seu lugar.
\par 25 Nadabe, filho de Jeroboão, começou a reinar sobre Israel no ano segundo de Asa, rei de Judá; e reinou sobre Israel dois anos.
\par 26 Fez o que era mau perante o SENHOR e andou nos caminhos de seu pai e no pecado com que seu pai fizera pecar a Israel.
\par 27 Conspirou contra ele Baasa, filho de Aías, da casa de Issacar, e o feriu em Gibetom, que era dos filisteus, quando Nadabe e todo o Israel cercavam Gibetom.
\par 28 Baasa, no ano terceiro de Asa, rei de Judá, matou a Nadabe e passou a reinar em seu lugar.
\par 29 Logo que começou a reinar, matou toda a descendência de Jeroboão; não lhe deixou ninguém com vida, a todos exterminou, segundo a palavra do SENHOR, por intermédio do seu servo Aías, o silonita,
\par 30 por causa dos pecados que Jeroboão cometera e pelos que fizera Israel cometer, por causa da provocação com que irritara ao SENHOR, Deus de Israel.
\par 31 Quanto aos mais atos de Nadabe e a tudo quanto fez, porventura, não estão escritos no Livro da História dos Reis de Israel?
\par 32 Houve guerra entre Asa e Baasa, rei de Israel, todos os seus dias.
\par 33 No ano terceiro de Asa, rei de Judá, Baasa, filho de Aías, começou a reinar sobre todo o Israel, em Tirza, e reinou vinte e quatro anos.
\par 34 Fez o que era mau perante o SENHOR e andou no caminho de Jeroboão e no seu pecado, o qual fizera Israel cometer.

\chapter{16}

\par 1 Então, veio a palavra do SENHOR a Jeú, filho de Hanani, contra Baasa, dizendo:
\par 2 Porquanto te levantei do pó e te constituí príncipe sobre o meu povo de Israel, e tens andado no caminho de Jeroboão e tens feito pecar a meu povo de Israel, irritando-me com os seus pecados,
\par 3 eis que te exterminarei a ti, Baasa, e os teus descendentes e farei à tua casa como à casa de Jeroboão, filho de Nebate.
\par 4 Quem morrer a Baasa na cidade, os cães o comerão, e o que dele morrer no campo aberto, as aves do céu o comerão.
\par 5 Quanto aos mais atos de Baasa, e ao que fez, e ao seu poder, porventura, não estão escritos no Livro da História dos Reis de Israel?
\par 6 Baasa descansou com seus pais e foi sepultado em Tirza; e Elá, seu filho, reinou em seu lugar.
\par 7 Assim, veio a palavra do SENHOR, por intermédio do profeta Jeú, filho de Hanani, contra Baasa e contra a sua descendência; e isso por todo o mal que fizera perante o SENHOR, irritando-o com as suas obras, para ser como a casa de Jeroboão, e também porque matara a casa de Jeroboão.
\par 8 No vigésimo sexto ano de Asa, rei de Judá, Elá, filho de Baasa, começou a reinar em Tirza sobre Israel; e reinou dois anos.
\par 9 Zinri, seu servo, comandante da metade dos carros, conspirou contra ele. Achava-se Elá em Tirza, bebendo e embriagando-se em casa de Arsa, seu mordomo em Tirza.
\par 10 Entrou Zinri, e o feriu, e o matou, no ano vigésimo sétimo de Asa, rei de Judá; e reinou em seu lugar.
\par 11 Logo que começou a reinar e se assentou no trono, feriu todos os descendentes de Baasa; não lhe deixou nenhum do sexo masculino, nem dos parentes, nem dos seus amigos.
\par 12 Assim, exterminou Zinri todos os descendentes de Baasa, segundo a palavra do SENHOR, por intermédio do profeta Jeú, contra Baasa,
\par 13 por todos os pecados de Baasa, e os pecados de Elá, seu filho, que cometeram, e pelos que fizeram Israel cometer, irritando ao SENHOR, Deus de Israel, com os seus ídolos.
\par 14 Quanto aos mais atos de Elá e a tudo quanto fez, porventura, não estão escritos no Livro da História dos Reis de Israel?
\par 15 No ano vigésimo sétimo de Asa, rei de Judá, reinou Zinri sete dias em Tirza; e o povo estava acampado contra Gibetom, que era dos filisteus.
\par 16 O povo que estava acampado ouviu dizer: Zinri conspirou contra o rei e o matou. Pelo que todo o Israel, no mesmo dia, no arraial, constituiu rei sobre Israel a Onri, comandante do exército.
\par 17 Subiu Onri de Gibetom, e todo o Israel, com ele, e sitiaram Tirza.
\par 18 Vendo Zinri que a cidade era tomada, foi-se ao castelo da casa do rei, e o queimou sobre si, e morreu,
\par 19 por causa dos pecados que cometera, fazendo o que era mau perante o SENHOR, andando no caminho de Jeroboão e no pecado que cometera, fazendo pecar a Israel.
\par 20 Quanto aos mais atos de Zinri e à conspiração que fez, porventura, não estão escritos no Livro da História dos Reis de Israel?
\par 21 Então, o povo de Israel se dividiu em dois partidos: metade do povo seguia a Tibni, filho de Ginate, para o fazer rei, e a outra metade seguia a Onri.
\par 22 Mas o povo que seguia a Onri prevaleceu contra o que seguia a Tibni, filho de Ginate. Tibni morreu, e passou a reinar Onri.
\par 23 No trigésimo primeiro ano de Asa, rei de Judá, Onri começou a reinar sobre Israel e reinou doze anos. Em Tirza, reinou seis anos.
\par 24 De Semer comprou ele o monte de Samaria por dois talentos de prata e o fortificou; à cidade que edificou sobre o monte, chamou-lhe Samaria, nome oriundo de Semer, dono do monte.
\par 25 Fez Onri o que era mau perante o SENHOR; fez pior do que todos quantos foram antes dele.
\par 26 Andou em todos os caminhos de Jeroboão, filho de Nebate, como também nos pecados com que este fizera pecar a Israel, irritando ao SENHOR, Deus de Israel, com os seus ídolos.
\par 27 Quanto aos mais atos de Onri, ao que fez e ao poder que manifestou, porventura, não estão escritos no Livro da História dos Reis de Israel?
\par 28 Onri descansou com seus pais e foi sepultado em Samaria; e Acabe, seu filho, reinou em seu lugar.
\par 29 Acabe, filho de Onri, começou a reinar sobre Israel no ano trigésimo oitavo de Asa, rei de Judá; e reinou Acabe, filho de Onri, sobre Israel, em Samaria, vinte e dois anos.
\par 30 Fez Acabe, filho de Onri, o que era mau perante o SENHOR, mais do que todos os que foram antes dele.
\par 31 Como se fora coisa de somenos andar ele nos pecados de Jeroboão, filho de Nebate, tomou por mulher a Jezabel, filha de Etbaal, rei dos sidônios; e foi, e serviu a Baal, e o adorou.
\par 32 Levantou um altar a Baal, na casa de Baal que edificara em Samaria.
\par 33 Também Acabe fez um poste-ídolo, de maneira que cometeu mais abominações para irritar ao SENHOR, Deus de Israel, do que todos os reis de Israel que foram antes dele.
\par 34 Em seus dias, Hiel, o betelita, edificou a Jericó; quando lhe lançou os fundamentos, morreu-lhe Abirão, seu primogênito; quando lhe pôs as portas, morreu Segube, seu último, segundo a palavra do SENHOR, que falara por intermédio de Josué, filho de Num.

\chapter{17}

\par 1 Então, Elias, o tesbita, dos moradores de Gileade, disse a Acabe: Tão certo como vive o SENHOR, Deus de Israel, perante cuja face estou, nem orvalho nem chuva haverá nestes anos, segundo a minha palavra.
\par 2 Veio-lhe a palavra do SENHOR, dizendo:
\par 3 Retira-te daqui, vai para o lado oriental e esconde-te junto à torrente de Querite, fronteira ao Jordão.
\par 4 Beberás da torrente; e ordenei aos corvos que ali mesmo te sustentem.
\par 5 Foi, pois, e fez segundo a palavra do SENHOR; retirou-se e habitou junto à torrente de Querite, fronteira ao Jordão.
\par 6 Os corvos lhe traziam pela manhã pão e carne, como também pão e carne ao anoitecer; e bebia da torrente.
\par 7 Mas, passados dias, a torrente secou, porque não chovia sobre a terra.
\par 8 Então, lhe veio a palavra do SENHOR, dizendo:
\par 9 Dispõe-te, e vai a Sarepta, que pertence a Sidom, e demora-te ali, onde ordenei a uma mulher viúva que te dê comida.
\par 10 Então, ele se levantou e se foi a Sarepta; chegando à porta da cidade, estava ali uma mulher viúva apanhando lenha; ele a chamou e lhe disse: Traze-me, peço-te, uma vasilha de água para eu beber.
\par 11 Indo ela a buscá-la, ele a chamou e lhe disse: Traze-me também um bocado de pão na tua mão.
\par 12 Porém ela respondeu: Tão certo como vive o SENHOR, teu Deus, nada tenho cozido; há somente um punhado de farinha numa panela e um pouco de azeite numa botija; e, vês aqui, apanhei dois cavacos e vou preparar esse resto de comida para mim e para o meu filho; comê-lo-emos e morreremos.
\par 13 Elias lhe disse: Não temas; vai e faze o que disseste; mas primeiro faze dele para mim um bolo pequeno e traze-mo aqui fora; depois, farás para ti mesma e para teu filho.
\par 14 Porque assim diz o SENHOR, Deus de Israel: A farinha da tua panela não se acabará, e o azeite da tua botija não faltará, até ao dia em que o SENHOR fizer chover sobre a terra.
\par 15 Foi ela e fez segundo a palavra de Elias; assim, comeram ele, ela e a sua casa muitos dias.
\par 16 Da panela a farinha não se acabou, e da botija o azeite não faltou, segundo a palavra do SENHOR, por intermédio de Elias.
\par 17 Depois disto, adoeceu o filho da mulher, da dona da casa, e a sua doença se agravou tanto, que ele morreu.
\par 18 Então, disse ela a Elias: Que fiz eu, ó homem de Deus? Vieste a mim para trazeres à memória a minha iniqüidade e matares o meu filho?
\par 19 Ele lhe disse: Dá-me o teu filho; tomou-o dos braços dela, e o levou para cima, ao quarto, onde ele mesmo se hospedava, e o deitou em sua cama;
\par 20 então, clamou ao SENHOR e disse: Ó SENHOR, meu Deus, também até a esta viúva, com quem me hospedo, afligiste, matando-lhe o filho?
\par 21 E, estendendo-se três vezes sobre o menino, clamou ao SENHOR e disse: Ó SENHOR, meu Deus, rogo-te que faças a alma deste menino tornar a entrar nele.
\par 22 O SENHOR atendeu à voz de Elias; e a alma do menino tornou a entrar nele, e reviveu.
\par 23 Elias tomou o menino, e o trouxe do quarto à casa, e o deu a sua mãe, e lhe disse: Vê, teu filho vive.
\par 24 Então, a mulher disse a Elias: Nisto conheço agora que tu és homem de Deus e que a palavra do SENHOR na tua boca é verdade.

\chapter{18}

\par 1 Muito tempo depois, veio a palavra do SENHOR a Elias, no terceiro ano, dizendo: Vai, apresenta-te a Acabe, porque darei chuva sobre a terra.
\par 2 Partiu, pois, Elias a apresentar-se a Acabe; e a fome era extrema em Samaria.
\par 3 Acabe chamou a Obadias, o mordomo. (Obadias temia muito ao SENHOR,
\par 4 porque, quando Jezabel exterminava os profetas do SENHOR, Obadias tomou cem profetas, e de cinqüenta em cinqüenta os escondeu numa cova, e os sustentou com pão e água.)
\par 5 Disse Acabe a Obadias: Vai pela terra a todas as fontes de água e a todos os vales; pode ser que achemos erva, para que salvemos a vida aos cavalos e mulos e não percamos todos os animais.
\par 6 Repartiram entre si a terra, para a percorrerem; Acabe foi à parte por um caminho, e Obadias foi sozinho por outro.
\par 7 Estando Obadias já de caminho, eis que Elias se encontrou com ele. Obadias, reconhecendo-o, prostrou-se com o rosto em terra e disse: És tu meu senhor Elias?
\par 8 Respondeu-lhe ele: Sou eu; vai e dize a teu senhor: Eis que aí está Elias.
\par 9 Porém ele disse: Em que pequei, para que entregues teu servo na mão de Acabe, e ele me mate?
\par 10 Tão certo como vive o SENHOR, teu Deus, não houve nação nem reino aonde o meu senhor não mandasse homens à tua procura; e, dizendo eles: Aqui não está; fazia jurar aquele reino e aquela nação que te não haviam achado.
\par 11 Agora, tu dizes: Vai, dize a teu senhor: Eis que aí está Elias.
\par 12 Poderá ser que, apartando-me eu de ti, o Espírito do SENHOR te leve não sei para onde, e, vindo eu a dar as novas a Acabe, e não te achando ele, me matará; eu, contudo, teu servo, temo ao SENHOR desde a minha mocidade.
\par 13 Acaso, não disseram a meu senhor o que fiz, quando Jezabel matava os profetas do SENHOR, como escondi cem homens dos profetas do SENHOR, de cinqüenta em cinqüenta, numas covas, e os sustentei com pão e água?
\par 14 E, agora, tu dizes: Vai, dize a teu senhor: Eis que aí está Elias. Ele me matará.
\par 15 Disse Elias: Tão certo como vive o SENHOR dos Exércitos, perante cuja face estou, deveras, hoje, me apresentarei a ele.
\par 16 Então, foi Obadias encontrar-se com Acabe e lho anunciou; e foi Acabe ter com Elias.
\par 17 Vendo-o, disse-lhe: És tu, ó perturbador de Israel?
\par 18 Respondeu Elias: Eu não tenho perturbado a Israel, mas tu e a casa de teu pai, porque deixastes os mandamentos do SENHOR e seguistes os baalins.
\par 19 Agora, pois, manda ajuntar a mim todo o Israel no monte Carmelo, como também os quatrocentos e cinqüenta profetas de Baal e os quatrocentos profetas do poste-ídolo que comem da mesa de Jezabel.
\par 20 Então, enviou Acabe mensageiros a todos os filhos de Israel e ajuntou os profetas no monte Carmelo.
\par 21 Então, Elias se chegou a todo o povo e disse: Até quando coxeareis entre dois pensamentos? Se o SENHOR é Deus, segui-o; se é Baal, segui-o. Porém o povo nada lhe respondeu.
\par 22 Então, disse Elias ao povo: Só eu fiquei dos profetas do SENHOR, e os profetas de Baal são quatrocentos e cinqüenta homens.
\par 23 Dêem-se-nos, pois, dois novilhos; escolham eles para si um dos novilhos e, dividindo-o em pedaços, o ponham sobre a lenha, porém não lhe metam fogo; eu prepararei o outro novilho, e o porei sobre a lenha, e não lhe meterei fogo.
\par 24 Então, invocai o nome de vosso deus, e eu invocarei o nome do SENHOR; e há de ser que o deus que responder por fogo esse é que é Deus. E todo o povo respondeu e disse: É boa esta palavra.
\par 25 Disse Elias aos profetas de Baal: Escolhei para vós outros um dos novilhos, e preparai-o primeiro, porque sois muitos, e invocai o nome de vosso deus; e não lhe metais fogo.
\par 26 Tomaram o novilho que lhes fora dado, prepararam-no e invocaram o nome de Baal, desde a manhã até ao meio-dia, dizendo: Ah! Baal, responde-nos! Porém não havia uma voz que respondesse; e, manquejando, se movimentavam ao redor do altar que tinham feito.
\par 27 Ao meio-dia, Elias zombava deles, dizendo: Clamai em altas vozes, porque ele é deus; pode ser que esteja meditando, ou atendendo a necessidades, ou de viagem, ou a dormir e despertará.
\par 28 E eles clamavam em altas vozes e se retalhavam com facas e com lancetas, segundo o seu costume, até derramarem sangue.
\par 29 Passado o meio-dia, profetizaram eles, até que a oferta de manjares se oferecesse; porém não houve voz, nem resposta, nem atenção alguma.
\par 30 Então, Elias disse a todo o povo: Chegai-vos a mim. E todo o povo se chegou a ele; Elias restaurou o altar do SENHOR, que estava em ruínas.
\par 31 Tomou doze pedras, segundo o número das tribos dos filhos de Jacó, ao qual viera a palavra do SENHOR, dizendo: Israel será o teu nome.
\par 32 Com aquelas pedras edificou o altar em nome do SENHOR; depois, fez um rego em redor do altar tão grande como para semear duas medidas de sementes.
\par 33 Então, armou a lenha, dividiu o novilho em pedaços, pô-lo sobre a lenha
\par 34 e disse: Enchei de água quatro cântaros e derramai-a sobre o holocausto e sobre a lenha. Disse ainda: Fazei-o segunda vez; e o fizeram. Disse mais: Fazei-o terceira vez; e o fizeram terceira vez.
\par 35 De maneira que a água corria ao redor do altar; ele encheu também de água o rego.
\par 36 No devido tempo, para se apresentar a oferta de manjares, aproximou-se o profeta Elias e disse: Ó SENHOR, Deus de Abraão, de Isaque e de Israel, fique, hoje, sabido que tu és Deus em Israel, e que eu sou teu servo e que, segundo a tua palavra, fiz todas estas coisas.
\par 37 Responde-me, SENHOR, responde-me, para que este povo saiba que tu, SENHOR, és Deus e que a ti fizeste retroceder o coração deles.
\par 38 Então, caiu fogo do SENHOR, e consumiu o holocausto, e a lenha, e as pedras, e a terra, e ainda lambeu a água que estava no rego.
\par 39 O que vendo todo o povo, caiu de rosto em terra e disse: O SENHOR é Deus! O SENHOR é Deus!
\par 40 Disse-lhes Elias: Lançai mão dos profetas de Baal, que nem um deles escape. Lançaram mão deles; e Elias os fez descer ao ribeiro de Quisom e ali os matou.
\par 41 Então, disse Elias a Acabe: Sobe, come e bebe, porque já se ouve ruído de abundante chuva.
\par 42 Subiu Acabe a comer e a beber; Elias, porém, subiu ao cimo do Carmelo, e, encurvado para a terra, meteu o rosto entre os joelhos,
\par 43 e disse ao seu moço: Sobe e olha para o lado do mar. Ele subiu, olhou e disse: Não há nada. Então, lhe disse Elias: Volta. E assim por sete vezes.
\par 44 À sétima vez disse: Eis que se levanta do mar uma nuvem pequena como a palma da mão do homem. Então, disse ele: Sobe e dize a Acabe: Aparelha o teu carro e desce, para que a chuva não te detenha.
\par 45 Dentro em pouco, os céus se enegreceram, com nuvens e vento, e caiu grande chuva. Acabe subiu ao carro e foi para Jezreel.
\par 46 A mão do SENHOR veio sobre Elias, o qual cingiu os lombos e correu adiante de Acabe, até à entrada de Jezreel.

\chapter{19}

\par 1 Acabe fez saber a Jezabel tudo quanto Elias havia feito e como matara todos os profetas à espada.
\par 2 Então, Jezabel mandou um mensageiro a Elias a dizer-lhe: Façam-me os deuses como lhes aprouver se amanhã a estas horas não fizer eu à tua vida como fizeste a cada um deles.
\par 3 Temendo, pois, Elias, levantou-se, e, para salvar sua vida, se foi, e chegou a Berseba, que pertence a Judá; e ali deixou o seu moço.
\par 4 Ele mesmo, porém, se foi ao deserto, caminho de um dia, e veio, e se assentou debaixo de um zimbro; e pediu para si a morte e disse: Basta; toma agora, ó SENHOR, a minha alma, pois não sou melhor do que meus pais.
\par 5 Deitou-se e dormiu debaixo do zimbro; eis que um anjo o tocou e lhe disse: Levanta-te e come.
\par 6 Olhou ele e viu, junto à cabeceira, um pão cozido sobre pedras em brasa e uma botija de água. Comeu, bebeu e tornou a dormir.
\par 7 Voltou segunda vez o anjo do SENHOR, tocou-o e lhe disse: Levanta-te e come, porque o caminho te será sobremodo longo.
\par 8 Levantou-se, pois, comeu e bebeu; e, com a força daquela comida, caminhou quarenta dias e quarenta noites até Horebe, o monte de Deus.
\par 9 Ali, entrou numa caverna, onde passou a noite; e eis que lhe veio a palavra do SENHOR e lhe disse: Que fazes aqui, Elias?
\par 10 Ele respondeu: Tenho sido zeloso pelo SENHOR, Deus dos Exércitos, porque os filhos de Israel deixaram a tua aliança, derribaram os teus altares e mataram os teus profetas à espada; e eu fiquei só, e procuram tirar-me a vida.
\par 11 Disse-lhe Deus: Sai e põe-te neste monte perante o SENHOR. Eis que passava o SENHOR; e um grande e forte vento fendia os montes e despedaçava as penhas diante do SENHOR, porém o SENHOR não estava no vento; depois do vento, um terremoto, mas o SENHOR não estava no terremoto;
\par 12 depois do terremoto, um fogo, mas o SENHOR não estava no fogo; e, depois do fogo, um cicio tranqüilo e suave.
\par 13 Ouvindo-o Elias, envolveu o rosto no seu manto e, saindo, pôs-se à entrada da caverna. Eis que lhe veio uma voz e lhe disse: Que fazes aqui, Elias?
\par 14 Ele respondeu: Tenho sido em extremo zeloso pelo SENHOR, Deus dos Exércitos, porque os filhos de Israel deixaram a tua aliança, derribaram os teus altares e mataram os teus profetas à espada; e eu fiquei só, e procuram tirar-me a vida.
\par 15 Disse-lhe o SENHOR: Vai, volta ao teu caminho para o deserto de Damasco e, em chegando lá, unge a Hazael rei sobre a Síria.
\par 16 A Jeú, filho de Ninsi, ungirás rei sobre Israel e também Eliseu, filho de Safate, de Abel-Meolá, ungirás profeta em teu lugar.
\par 17 Quem escapar à espada de Hazael, Jeú o matará; quem escapar à espada de Jeú, Eliseu o matará.
\par 18 Também conservei em Israel sete mil, todos os joelhos que não se dobraram a Baal, e toda boca que o não beijou.
\par 19 Partiu, pois, Elias dali e achou a Eliseu, filho de Safate, que andava lavrando com doze juntas de bois adiante dele; ele estava com a duodécima. Elias passou por ele e lançou o seu manto sobre ele.
\par 20 Então, deixou este os bois, correu após Elias e disse: Deixa-me beijar a meu pai e a minha mãe e, então, te seguirei. Elias respondeu-lhe: Vai e volta; pois já sabes o que fiz contigo.
\par 21 Voltou Eliseu de seguir a Elias, tomou a junta de bois, e os imolou, e, com os aparelhos dos bois, cozeu as carnes, e as deu ao povo, e comeram. Então, se dispôs, e seguiu a Elias, e o servia.

\chapter{20}

\par 1 Ben-Hadade, rei da Síria, ajuntou todo o seu exército; havia com ele trinta e dois reis, e cavalos, e carros. Subiu, cercou a Samaria e pelejou contra ela.
\par 2 Enviou mensageiros à cidade, a Acabe, rei de Israel,
\par 3 que lhe disseram: Assim diz Ben-Hadade: A tua prata e o teu ouro são meus; tuas mulheres e os melhores de teus filhos são meus.
\par 4 Respondeu o rei de Israel e disse: Seja conforme a tua palavra, ó rei, meu senhor; eu sou teu, e tudo o que tenho.
\par 5 Tornaram a vir os mensageiros e disseram: Assim diz Ben-Hadade: Enviei-te, na verdade, mensageiros que dissessem: Tens de entregar-me a tua prata, o teu ouro, as tuas mulheres e os teus filhos.
\par 6 Todavia, amanhã a estas horas enviar-te-ei os meus servos, que esquadrinharão a tua casa e as casas dos teus oficiais, meterão as mãos em tudo o que for aprazível aos teus olhos e o levarão.
\par 7 Então, o rei de Israel chamou todos os anciãos da sua terra e lhes disse: Notai e vede como este homem procura o mal; pois me mandou exigir minhas mulheres, meus filhos, minha prata e meu ouro, e não lho neguei.
\par 8 Todos os anciãos e todo o povo lhe disseram: Não lhe dês ouvidos, nem o consintas.
\par 9 Pelo que disse aos mensageiros de Ben-Hadade: Dizei ao rei, meu senhor: Tudo o que primeiro demandaste do teu servo farei, porém isto, agora, não posso consentir. E se foram os mensageiros e deram esta resposta.
\par 10 Ben-Hadade tornou a enviar mensageiros, dizendo: Façam-me os deuses como lhes aprouver, se o pó de Samaria bastar para encher as mãos de todo o povo que me segue.
\par 11 Porém o rei de Israel respondeu e disse: Dizei-lhe: Não se gabe quem se cinge como aquele que vitorioso se descinge.
\par 12 Tendo Ben-Hadade ouvido esta resposta, quando bebiam ele e os reis nas tendas, disse aos seus servos: Ponde-vos de prontidão. E eles se puseram de prontidão contra a cidade.
\par 13 Eis que um profeta se chegou a Acabe, rei de Israel, e lhe disse: Assim diz o SENHOR: Viste toda esta grande multidão? Pois, hoje, a entregarei nas tuas mãos, e saberás que eu sou o SENHOR.
\par 14 Perguntou Acabe: Por quem? Ele respondeu: Assim diz o SENHOR: Pelos moços dos chefes das províncias. E disse: Quem começará a peleja? Tu! -- respondeu ele.
\par 15 Então, contou os moços dos chefes das províncias, e eram duzentos e trinta e dois; depois, contou todo o povo, todos os filhos de Israel, sete mil.
\par 16 Saíram ao meio-dia. Ben-Hadade, porém, estava bebendo e embriagando-se nas tendas, ele e os reis, os trinta e dois reis que o ajudavam.
\par 17 Saíram primeiro os moços dos chefes das províncias; Ben-Hadade mandou observadores que lhe deram avisos, dizendo: Saíram de Samaria uns homens.
\par 18 Ele disse: Quer venham tratar de paz, tomai-os vivos; quer venham pelejar, tomai-os vivos.
\par 19 Saíram, pois, da cidade os moços dos chefes das províncias e o exército que os seguia.
\par 20 Eles feriram, cada um ao homem que se lhe opunha; os siros fugiram, e Israel os perseguiu; porém Ben-Hadade, rei da Síria, escapou a cavalo, com alguns cavaleiros.
\par 21 Saiu o rei de Israel e destroçou os cavalos e os carros; e feriu os siros com grande estrago.
\par 22 Então, o profeta se chegou ao rei de Israel e lhe disse: Vai, sê forte, considera e vê o que hás de fazer; porque daqui a um ano subirá o rei da Síria contra ti.
\par 23 Os servos do rei da Síria lhe disseram: Seus deuses são deuses dos montes; por isso, foram mais fortes do que nós; mas pelejemos contra eles em planície, e, por certo, seremos mais fortes do que eles.
\par 24 Faze, pois, isto: tira os reis, cada um do seu lugar, e substitui-os por capitães,
\par 25 e forma outro exército igual em número ao que perdeste, com outros tantos cavalos e outros tantos carros, e pelejemos contra eles em planície e, por certo, seremos mais fortes do que eles. Ele deu ouvidos ao que disseram e assim o fez.
\par 26 Decorrido um ano, Ben-Hadade passou revista aos siros e subiu a Afeca para pelejar contra Israel.
\par 27 Também aos filhos de Israel se passou revista, foram providos de víveres e marcharam contra eles. Os filhos de Israel acamparam-se defronte deles, como dois pequenos rebanhos de cabras; mas os siros enchiam a terra.
\par 28 Chegou um homem de Deus, e falou ao rei de Israel, e disse: Assim diz o SENHOR: Porquanto os siros disseram: O SENHOR é deus dos montes e não dos vales, toda esta grande multidão entregarei nas tuas mãos, e assim sabereis que eu sou o SENHOR.
\par 29 Sete dias estiveram acampados uns defronte dos outros. Ao sétimo dia, travou-se a batalha, e os filhos de Israel, num só dia, feriram dos siros cem mil homens de pé.
\par 30 Os restantes fugiram para Afeca e entraram na cidade; e caiu o muro sobre os vinte e sete mil homens que restaram. Ben-Hadade fugiu, veio à cidade e se escondia de câmara em câmara.
\par 31 Então, lhe disseram os seus servos: Eis que temos ouvido que os reis da casa de Israel são reis clementes; ponhamos, pois, panos de saco sobre os lombos e cordas à roda da cabeça e saiamos ao rei de Israel; pode ser que ele te poupe a vida.
\par 32 Então, se cingiram com pano de saco pelos lombos, puseram cordas à roda da cabeça, vieram ao rei de Israel e disseram: Diz o teu servo Ben-Hadade: Poupa-me a vida. Disse Acabe: Pois ainda vive? É meu irmão.
\par 33 Aqueles homens tomaram isto por presságio, valeram-se logo dessa palavra; e disseram: Teu irmão Ben-Hadade! Ele disse: Vinde, trazei-mo. Então, Ben-Hadade saiu a ter com ele, e ele o fez subir ao carro.
\par 34 Ben-Hadade disse-lhe: As cidades que meu pai tomou a teu pai, eu tas restituirei; monta os teus bazares em Damasco, como meu pai o fez em Samaria. E eu, disse Acabe, com esta aliança, te deixarei livre. Fez com ele aliança e o despediu.
\par 35 Então, um dos discípulos dos profetas disse ao seu companheiro por ordem do SENHOR: Esmurra-me; mas o homem recusou fazê-lo.
\par 36 Ele lhe disse: Visto que não obedeceste à voz do SENHOR, eis que, em te apartando de mim, um leão te matará. Tendo ele se apartado, um leão o encontrou e o matou.
\par 37 Encontrando o profeta outro homem, lhe disse: Esmurra-me. Ele o esmurrou e o feriu.
\par 38 Então, se foi o profeta e se pôs no caminho do rei, disfarçado com uma venda sobre os olhos.
\par 39 Ao passar o rei, gritou e disse: Teu servo estava no meio da peleja, quando, voltando-se-me um companheiro, me trouxe um homem e me disse: Vigia este homem; se vier a faltar, a tua vida responderá pela vida dele ou pagarás um talento de prata.
\par 40 Estando o teu servo ocupado daqui e dali, ele se foi. Respondeu-lhe o rei de Israel: Esta é a tua sentença; tu mesmo a pronunciaste.
\par 41 Então, ele se apressou e tirou a venda de sobre os seus olhos; e o rei de Israel reconheceu que era um dos profetas.
\par 42 E disse-lhe: Assim diz o SENHOR: Porquanto soltaste da mão o homem que eu havia condenado, a tua vida será em lugar da sua vida, e o teu povo, em lugar do seu povo.
\par 43 Foi-se o rei de Israel para sua casa, desgostoso e indignado, e chegou a Samaria.

\chapter{21}

\par 1 Sucedeu, depois disto, o seguinte: Nabote, o jezreelita, possuía uma vinha ao lado do palácio que Acabe, rei de Samaria, tinha em Jezreel.
\par 2 Disse Acabe a Nabote: Dá-me a tua vinha, para que me sirva de horta, pois está perto, ao lado da minha casa. Dar-te-ei por ela outra, melhor; ou, se for do teu agrado, dar-te-ei em dinheiro o que ela vale.
\par 3 Porém Nabote disse a Acabe: Guarde-me o SENHOR de que eu dê a herança de meus pais.
\par 4 Então, Acabe veio desgostoso e indignado para sua casa, por causa da palavra que Nabote, o jezreelita, lhe falara, quando disse: Não te darei a herança de meus pais. E deitou-se na sua cama, voltou o rosto e não comeu pão.
\par 5 Porém, vindo Jezabel, sua mulher, ter com ele, lhe disse: Que é isso que tens assim desgostoso o teu espírito e não comes pão?
\par 6 Ele lhe respondeu: Porque falei a Nabote, o jezreelita, e lhe disse: Dá-me a tua vinha por dinheiro; ou, se te apraz, dar-te-ei outra em seu lugar. Porém ele disse: Não te darei a minha vinha.
\par 7 Então, Jezabel, sua mulher, lhe disse: Governas tu, com efeito, sobre Israel? Levanta-te, come, e alegre-se o teu coração; eu te darei a vinha de Nabote, o jezreelita.
\par 8 Então, escreveu cartas em nome de Acabe, selou-as com o sinete dele e as enviou aos anciãos e aos nobres que havia na sua cidade e habitavam com Nabote.
\par 9 E escreveu nas cartas, dizendo: Apregoai um jejum e trazei Nabote para a frente do povo.
\par 10 Fazei sentar defronte dele dois homens malignos, que testemunhem contra ele, dizendo: Blasfemaste contra Deus e contra o rei. Depois, levai-o para fora e apedrejai-o, para que morra.
\par 11 Os homens da sua cidade, os anciãos e os nobres que nela habitavam fizeram como Jezabel lhes ordenara, segundo estava escrito nas cartas que lhes havia mandado.
\par 12 Apregoaram um jejum e trouxeram Nabote para a frente do povo.
\par 13 Então, vieram dois homens malignos, sentaram-se defronte dele e testemunharam contra ele, contra Nabote, perante o povo, dizendo: Nabote blasfemou contra Deus e contra o rei. E o levaram para fora da cidade e o apedrejaram, e morreu.
\par 14 Então, mandaram dizer a Jezabel: Nabote foi apedrejado e morreu.
\par 15 Tendo Jezabel ouvido que Nabote fora apedrejado e morrera, disse a Acabe: Levanta-te e toma posse da vinha que Nabote, o jezreelita, recusou dar-te por dinheiro; pois Nabote já não vive, mas é morto.
\par 16 Tendo Acabe ouvido que Nabote era morto, levantou-se para descer para a vinha de Nabote, o jezreelita, para tomar posse dela.
\par 17 Então, veio a palavra do SENHOR a Elias, o tesbita, dizendo:
\par 18 Dispõe-te, desce para encontrar-te com Acabe, rei de Israel, que habita em Samaria; eis que está na vinha de Nabote, aonde desceu para tomar posse dela.
\par 19 Falar-lhe-ás, dizendo: Assim diz o SENHOR: Mataste e, ainda por cima, tomaste a herança? Dir-lhe-ás mais: Assim diz o SENHOR: No lugar em que os cães lamberam o sangue de Nabote, cães lamberão o teu sangue, o teu mesmo.
\par 20 Perguntou Acabe a Elias: Já me achaste, inimigo meu? Respondeu ele: Achei-te, porquanto já te vendeste para fazeres o que é mau perante o SENHOR.
\par 21 Eis que trarei o mal sobre ti, arrancarei a tua posteridade e exterminarei de Acabe a todo do sexo masculino, quer escravo quer livre, em Israel.
\par 22 Farei a tua casa como a casa de Jeroboão, filho de Nebate, e como a casa de Baasa, filho de Aías, por causa da provocação com que me irritaste e fizeste pecar a Israel.
\par 23 Também de Jezabel falou o SENHOR: Os cães devorarão Jezabel dentro dos muros de Jezreel.
\par 24 Quem morrer de Acabe na cidade, os cães o comerão, e quem morrer no campo, as aves do céu o comerão.
\par 25 Ninguém houve, pois, como Acabe, que se vendeu para fazer o que era mau perante o SENHOR, porque Jezabel, sua mulher, o instigava;
\par 26 que fez grandes abominações, seguindo os ídolos, segundo tudo o que fizeram os amorreus, os quais o SENHOR lançou de diante dos filhos de Israel.
\par 27 Tendo Acabe ouvido estas palavras, rasgou as suas vestes, cobriu de pano de saco o seu corpo e jejuou; dormia em panos de saco e andava cabisbaixo.
\par 28 Então, veio a palavra do SENHOR a Elias, o tesbita, dizendo:
\par 29 Não viste que Acabe se humilha perante mim? Portanto, visto que se humilha perante mim, não trarei este mal nos seus dias, mas nos dias de seu filho o trarei sobre a sua casa.

\chapter{22}

\par 1 Três anos se passaram sem haver guerra entre a Síria e Israel.
\par 2 Porém, no terceiro ano, desceu Josafá, rei de Judá, para avistar-se com o rei de Israel.
\par 3 Disse o rei de Israel aos seus servos: Não sabeis vós que Ramote-Gileade é nossa, e nós hesitamos em tomá-la das mãos do rei da Síria?
\par 4 Então, perguntou a Josafá: Irás tu comigo à peleja, a Ramote-Gileade? Respondeu Josafá ao rei de Israel: Serei como tu és, o meu povo, como o teu povo, os meus cavalos, como os teus cavalos.
\par 5 Disse mais Josafá ao rei de Israel: Consulta primeiro a palavra do SENHOR.
\par 6 Então, o rei de Israel ajuntou os profetas, cerca de quatrocentos homens, e lhes disse: Irei à peleja contra Ramote-Gileade ou deixarei de ir? Eles disseram: Sobe, porque o Senhor a entregará nas mãos do rei.
\par 7 Disse, porém, Josafá: Não há aqui ainda algum profeta do SENHOR para o consultarmos?
\par 8 Respondeu o rei de Israel a Josafá: Há um ainda, pelo qual se pode consultar o SENHOR, porém eu o aborreço, porque nunca profetiza de mim o que é bom, mas somente o que é mau. Este é Micaías, filho de Inlá. Disse Josafá: Não fale o rei assim.
\par 9 Então, o rei de Israel chamou um oficial e disse: Traze-me depressa Micaías, filho de Inlá.
\par 10 O rei de Israel e Josafá, rei de Judá, estavam assentados, cada um no seu trono, vestidos de trajes reais, numa eira à entrada da porta de Samaria; e todos os profetas profetizavam diante deles.
\par 11 Zedequias, filho de Quenaana, fez para si uns chifres de ferro e disse: Assim diz o SENHOR: Com este escornearás os siros até de todo os consumir.
\par 12 Todos os profetas profetizaram assim, dizendo: Sobe a Ramote-Gileade e triunfarás, porque o SENHOR a entregará nas mãos do rei.
\par 13 O mensageiro que fora chamar a Micaías falou-lhe, dizendo: Eis que as palavras dos profetas a uma voz predizem coisas boas para o rei; seja, pois, a tua palavra como a palavra de um deles e fala o que é bom.
\par 14 Respondeu Micaías: Tão certo como vive o SENHOR, o que o SENHOR me disser, isso falarei.
\par 15 E, vindo ele ao rei, este lhe perguntou: Micaías, iremos a Ramote-Gileade à peleja ou deixaremos de ir? Ele lhe respondeu: Sobe e triunfarás, porque o SENHOR a entregará nas mãos do rei.
\par 16 O rei lhe disse: Quantas vezes te conjurarei, que não me fales senão a verdade em nome do SENHOR?
\par 17 Então, disse ele: Vi todo o Israel disperso pelos montes, como ovelhas que não têm pastor; e disse o SENHOR: Estes não têm dono; torne cada um em paz para a sua casa.
\par 18 Então, o rei de Israel disse a Josafá: Não te disse eu que ele não profetiza a meu respeito o que é bom, mas somente o que é mau?
\par 19 Micaías prosseguiu: Ouve, pois, a palavra do SENHOR: Vi o SENHOR assentado no seu trono, e todo o exército do céu estava junto a ele, à sua direita e à sua esquerda.
\par 20 Perguntou o SENHOR: Quem enganará a Acabe, para que suba e caia em Ramote-Gileade? Um dizia desta maneira, e outro, de outra.
\par 21 Então, saiu um espírito, e se apresentou diante do SENHOR, e disse: Eu o enganarei. Perguntou-lhe o SENHOR: Com quê?
\par 22 Respondeu ele: Sairei e serei espírito mentiroso na boca de todos os seus profetas. Disse o SENHOR: Tu o enganarás e ainda prevalecerás; sai e faze-o assim.
\par 23 Eis que o SENHOR pôs o espírito mentiroso na boca de todos estes teus profetas e o SENHOR falou o que é mau contra ti.
\par 24 Então, Zedequias, filho de Quenaana, chegou, deu uma bofetada em Micaías e disse: Por onde saiu de mim o Espírito do SENHOR para falar a ti?
\par 25 Disse Micaías: Eis que o verás naquele mesmo dia, quando entrares de câmara em câmara, para te esconderes.
\par 26 Então, disse o rei de Israel: Tomai Micaías e devolvei-o a Amom, governador da cidade, e a Joás, filho do rei;
\par 27 e direis: Assim diz o rei: Metei este homem na casa do cárcere e angustiai-o, com escassez de pão e de água, até que eu volte em paz.
\par 28 Disse Micaías: Se voltares em paz, não falou o SENHOR, na verdade, por mim. Disse mais: Ouvi isto, vós, todos os povos!
\par 29 Subiram o rei de Israel e Josafá, rei de Judá, a Ramote-Gileade.
\par 30 Disse o rei de Israel a Josafá: Eu me disfarçarei e entrarei na peleja; tu, porém, traja as tuas vestes. Disfarçou-se, pois, o rei de Israel e entrou na peleja.
\par 31 Ora, o rei da Síria dera ordem aos trinta e dois capitães dos seus carros, dizendo: Não pelejareis nem contra pequeno nem contra grande, mas somente contra o rei de Israel.
\par 32 Vendo os capitães dos carros Josafá, disseram: Certamente, este é o rei de Israel. E a ele se dirigiram para o atacar. Porém Josafá gritou.
\par 33 Vendo os capitães dos carros que não era o rei de Israel, deixaram de o perseguir.
\par 34 Então, um homem entesou o arco e, atirando ao acaso, feriu o rei de Israel por entre as juntas da sua armadura; então, disse este ao seu cocheiro: Vira e leva-me para fora do combate, porque estou gravemente ferido.
\par 35 A peleja tornou-se renhida naquele dia; quanto ao rei, seguraram-no de pé no carro defronte dos siros, mas à tarde morreu. O sangue corria da ferida para o fundo do carro.
\par 36 Ao pôr-do-sol, fez-se ouvir um pregão pelo exército, que dizia: Cada um para a sua cidade, e cada um para a sua terra!
\par 37 Morto o rei, levaram-no a Samaria, onde o sepultaram.
\par 38 Quando lavaram o carro junto ao açude de Samaria, os cães lamberam o sangue do rei, segundo a palavra que o SENHOR tinha dito; as prostitutas banharam-se nestas águas.
\par 39 Quanto aos mais atos de Acabe, e a tudo quanto fez, e à casa de marfim que construiu, e a todas as cidades que edificou, porventura, não estão escritos no Livro da História dos Reis de Israel?
\par 40 Assim, descansou Acabe com seus pais; e Acazias, seu filho, reinou em seu lugar.
\par 41 E Josafá, filho de Asa, começou a reinar sobre Judá no quarto ano de Acabe, rei de Israel.
\par 42 Era Josafá da idade de trinta e cinco anos quando começou a reinar; e vinte e cinco anos reinou em Jerusalém. Sua mãe se chamava Azuba, filha de Sili.
\par 43 Ele andou em todos os caminhos de Asa, seu pai; não se desviou deles e fez o que era reto perante o SENHOR.
\par 44 Todavia, os altos não se tiraram; neles, o povo ainda sacrificava e queimava incenso.
\par 45 Josafá viveu em paz com o rei de Israel.
\par 46 Quanto aos mais atos de Josafá, e ao poder que mostrou, e como guerreou, porventura, não estão escritos no Livro da História dos Reis de Judá?
\par 47 Também exterminou da terra os restantes dos prostitutos-cultuais que ficaram nos dias de Asa, seu pai.
\par 48 Então, não havia rei em Edom, porém reinava um governador.
\par 49 Fez Josafá navios de Társis, para irem a Ofir em busca de ouro; porém não foram, porque os navios se quebraram em Eziom-Geber.
\par 50 Então, Acazias, filho de Acabe, disse a Josafá: Vão os meus servos embarcados com os teus. Porém Josafá não quis.
\par 51 Josafá descansou com seus pais e foi sepultado na Cidade de Davi, seu pai; e Jeorão, seu filho, reinou em seu lugar.
\par 52 Acazias, filho de Acabe, começou a reinar sobre Israel em Samaria, no décimo sétimo ano de Josafá, rei de Judá; e reinou dois anos sobre Israel.
\par 53 Fez o que era mau perante o SENHOR; porque andou nos caminhos de seu pai, como também nos caminhos de sua mãe e nos caminhos de Jeroboão, filho de Nebate, que fez pecar a Israel.


\end{document}