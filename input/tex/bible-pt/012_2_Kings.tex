\begin{document}

\title{II Reis}


\chapter{1}

\par 1 Depois da morte de Acabe, revoltou-se Moabe contra Israel.
\par 2 E caiu Acazias pelas grades de um quarto alto, em Samaria, e adoeceu; enviou mensageiros e disse-lhes: Ide e consultai a Baal-Zebube, deus de Ecrom, se sararei desta doença.
\par 3 Mas o Anjo do SENHOR disse a Elias, o tesbita: Dispõe-te, e sobe para te encontrares com os mensageiros do rei de Samaria, e dize-lhes: Porventura, não há Deus em Israel, para irdes consultar Baal-Zebube, deus de Ecrom?
\par 4 Por isso, assim diz o SENHOR: Da cama a que subiste, não descerás, mas, sem falta, morrerás. Então, Elias partiu.
\par 5 E os mensageiros voltaram para o rei, e este lhes perguntou: Que há, por que voltastes?
\par 6 Eles responderam: Um homem nos subiu ao encontro e nos disse: Ide, voltai para o rei que vos mandou e dizei-lhe: Assim diz o SENHOR: Porventura, não há Deus em Israel, para que mandes consultar Baal-Zebube, deus de Ecrom? Portanto, da cama a que subiste, não descerás, mas, sem falta, morrerás.
\par 7 Ele lhes perguntou: Qual era a aparência do homem que vos veio ao encontro e vos falou tais palavras?
\par 8 Eles lhe responderam: Era homem vestido de pêlos, com os lombos cingidos de um cinto de couro. Então, disse ele: É Elias, o tesbita.
\par 9 Então, lhe enviou o rei um capitão de cinqüenta, com seus cinqüenta soldados, que subiram ao profeta, pois este estava assentado no cimo do monte; disse-lhe o capitão: Homem de Deus, o rei diz: Desce.
\par 10 Elias, porém, respondeu ao capitão de cinqüenta: Se eu sou homem de Deus, desça fogo do céu e te consuma a ti e aos teus cinqüenta. Então, fogo desceu do céu e o consumiu a ele e aos seus cinqüenta.
\par 11 Tornou o rei a enviar-lhe outro capitão de cinqüenta, com os seus cinqüenta; este lhe falou e disse: Homem de Deus, assim diz o rei: Desce depressa.
\par 12 Respondeu Elias e disse-lhe: Se eu sou homem de Deus, desça fogo do céu e te consuma a ti e aos teus cinqüenta. Então, fogo de Deus desceu do céu e o consumiu a ele e aos seus cinqüenta.
\par 13 Tornou o rei a enviar terceira vez um capitão de cinqüenta, com os seus cinqüenta; então, subiu o capitão de cinqüenta. Indo ele, pôs-se de joelhos diante de Elias, e suplicou-lhe, e disse-lhe: Homem de Deus, seja, peço-te, preciosa aos teus olhos a minha vida e a vida destes cinqüenta, teus servos;
\par 14 pois fogo desceu do céu e consumiu aqueles dois primeiros capitães de cinqüenta, com os seus cinqüenta; porém, agora, seja preciosa aos teus olhos a minha vida.
\par 15 Então, o Anjo do SENHOR disse a Elias: Desce com este, não temas. Levantou-se e desceu com ele ao rei.
\par 16 E disse a este: Assim diz o SENHOR: Por que enviaste mensageiros a consultar Baal-Zebube, deus de Ecrom? Será, acaso, por não haver Deus em Israel, cuja palavra se consultasse? Portanto, desta cama a que subiste, não descerás, mas, sem falta, morrerás.
\par 17 Assim, pois, morreu, segundo a palavra do SENHOR, que Elias falara; e Jorão, seu irmão, começou a reinar no seu lugar, no ano segundo de Jeorão, filho de Josafá, rei de Judá, porquanto Acazias não tinha filhos.
\par 18 Quanto aos mais atos de Acazias e ao que fez, porventura, não estão escritos no Livro da História dos Reis de Israel?

\chapter{2}

\par 1 Quando estava o SENHOR para tomar Elias ao céu por um redemoinho, Elias partiu de Gilgal em companhia de Eliseu.
\par 2 Disse Elias a Eliseu: Fica-te aqui, porque o SENHOR me enviou a Betel. Respondeu Eliseu: Tão certo como vive o SENHOR e vive a tua alma, não te deixarei. E, assim, desceram a Betel.
\par 3 Então, os discípulos dos profetas que estavam em Betel saíram ao encontro de Eliseu e lhe disseram: Sabes que o SENHOR, hoje, tomará o teu senhor, elevando-o por sobre a tua cabeça? Respondeu ele: Também eu o sei; calai-vos.
\par 4 Disse Elias a Eliseu: Fica-te aqui, porque o SENHOR me enviou a Jericó. Porém ele disse: Tão certo como vive o SENHOR e vive a tua alma, não te deixarei. E, assim, foram a Jericó.
\par 5 Então, os discípulos dos profetas que estavam em Jericó se chegaram a Eliseu e lhe disseram: Sabes que o SENHOR, hoje, tomará o teu senhor, elevando-o por sobre a tua cabeça? Respondeu ele: Também eu o sei; calai-vos.
\par 6 Disse-lhe, pois, Elias: Fica-te aqui, porque o SENHOR me enviou ao Jordão. Mas ele disse: Tão certo como vive o SENHOR e vive a tua alma, não te deixarei. E, assim, ambos foram juntos.
\par 7 Foram cinqüenta homens dos discípulos dos profetas e pararam a certa distância deles; eles ambos pararam junto ao Jordão.
\par 8 Então, Elias tomou o seu manto, enrolou-o e feriu as águas, as quais se dividiram para os dois lados; e passaram ambos em seco.
\par 9 Havendo eles passado, Elias disse a Eliseu: Pede-me o que queres que eu te faça, antes que seja tomado de ti. Disse Eliseu: Peço-te que me toque por herança porção dobrada do teu espírito.
\par 10 Tornou-lhe Elias: Dura coisa pediste. Todavia, se me vires quando for tomado de ti, assim se te fará; porém, se não me vires, não se fará.
\par 11 Indo eles andando e falando, eis que um carro de fogo, com cavalos de fogo, os separou um do outro; e Elias subiu ao céu num redemoinho.
\par 12 O que vendo Eliseu, clamou: Meu pai, meu pai, carros de Israel e seus cavaleiros! E nunca mais o viu; e, tomando as suas vestes, rasgou-as em duas partes.
\par 13 Então, levantou o manto que Elias lhe deixara cair e, voltando-se, pôs-se à borda do Jordão.
\par 14 Tomou o manto que Elias lhe deixara cair, feriu as águas e disse: Onde está o SENHOR, Deus de Elias? Quando feriu ele as águas, elas se dividiram para um e outro lado, e Eliseu passou.
\par 15 Vendo-o, pois, os discípulos dos profetas que estavam defronte, em Jericó, disseram: O espírito de Elias repousa sobre Eliseu. Vieram-lhe ao encontro e se prostraram diante dele em terra.
\par 16 E lhe disseram: Eis que entre os teus servos há cinqüenta homens valentes; ora, deixa-os ir em procura do teu senhor; pode ser que o Espírito do SENHOR o tenha levado e lançado nalgum dos montes ou nalgum dos vales. Porém ele respondeu: Não os envieis.
\par 17 Mas eles apertaram com ele, até que, constrangido, lhes disse: Enviai. E enviaram cinqüenta homens, que o procuraram três dias, porém não o acharam.
\par 18 Então, voltaram para ele, pois permanecera em Jericó; e ele lhes disse: Não vos disse que não fôsseis?
\par 19 Os homens da cidade disseram a Eliseu: Eis que é bem situada esta cidade, como vê o meu senhor, porém as águas são más, e a terra é estéril.
\par 20 Ele disse: Trazei-me um prato novo e ponde nele sal. E lho trouxeram.
\par 21 Então, saiu ele ao manancial das águas e deitou sal nele; e disse: Assim diz o SENHOR: Tornei saudáveis estas águas; já não procederá daí morte nem esterilidade.
\par 22 Ficaram, pois, saudáveis aquelas águas, até ao dia de hoje, segundo a palavra que Eliseu tinha dito.
\par 23 Então, subiu dali a Betel; e, indo ele pelo caminho, uns rapazinhos saíram da cidade, e zombavam dele, e diziam-lhe: Sobe, calvo! Sobe, calvo!
\par 24 Virando-se ele para trás, viu-os e os amaldiçoou em nome do SENHOR; então, duas ursas saíram do bosque e despedaçaram quarenta e dois deles.
\par 25 Dali, foi ele para o monte Carmelo, de onde voltou para Samaria.

\chapter{3}

\par 1 Jorão, filho de Acabe, começou a reinar sobre Israel, em Samaria, no décimo oitavo ano de Josafá, rei de Judá; e reinou doze anos.
\par 2 Fez o que era mau perante o SENHOR; porém não como seu pai, nem como sua mãe; porque tirou a coluna de Baal, que seu pai fizera.
\par 3 Contudo, aderiu aos pecados de Jeroboão, filho de Nebate, que fizera pecar a Israel; não se apartou deles.
\par 4 Então, Mesa, rei dos moabitas, era criador de gado e pagava o seu tributo ao rei de Israel com cem mil cordeiros e a lã de cem mil carneiros.
\par 5 Tendo, porém, morrido Acabe, revoltou-se o rei de Moabe contra o rei de Israel.
\par 6 Por isso, Jorão, ao mesmo tempo, saiu de Samaria e fez revista de todo o Israel.
\par 7 Mandou dizer a Josafá, rei de Judá: O rei de Moabe se revoltou contra mim; irás tu comigo à guerra contra Moabe? Respondeu ele: Subirei; serei como tu és, o meu povo, como o teu povo, os meus cavalos, como os teus cavalos.
\par 8 Então, perguntou Jorão: Por que caminho subiremos? Respondeu ele: Pelo caminho do deserto de Edom.
\par 9 Partiram o rei de Israel, o rei de Judá e o rei de Edom; após sete dias de marcha, não havia água para o exército e para o gado que os seguiam.
\par 10 Então, disse o rei de Israel: Ai! O SENHOR chamou a estes três reis para os entregar nas mãos de Moabe.
\par 11 Perguntou, porém, Josafá: Não há, aqui, algum profeta do SENHOR, para que consultemos o SENHOR por ele? Respondeu um dos servos do rei de Israel: Aqui está Eliseu, filho de Safate, que deitava água sobre as mãos de Elias.
\par 12 Disse Josafá: Está com ele a palavra do SENHOR. Então, o rei de Israel, Josafá e o rei de Edom desceram a ter com ele.
\par 13 Mas Eliseu disse ao rei de Israel: Que tenho eu contigo? Vai aos profetas de teu pai e aos profetas de tua mãe. Porém o rei de Israel lhe disse: Não, porque o SENHOR é quem chamou estes três reis para os entregar nas mãos de Moabe.
\par 14 Disse Eliseu: Tão certo como vive o SENHOR dos Exércitos, em cuja presença estou, se eu não respeitasse a presença de Josafá, rei de Judá, não te daria atenção, nem te contemplaria.
\par 15 Ora, pois, trazei-me um tangedor. Quando o tangedor tocava, veio o poder de Deus sobre Eliseu.
\par 16 Este disse: Assim diz o SENHOR: Fazei, neste vale, covas e covas.
\par 17 Porque assim diz o SENHOR: Não sentireis vento, nem vereis chuva; todavia, este vale se encherá de tanta água, que bebereis vós, e o vosso gado, e os vossos animais.
\par 18 Isto é ainda pouco aos olhos do SENHOR; de maneira que também entregará Moabe nas vossas mãos.
\par 19 Ferireis todas as cidades fortificadas e todas as cidades principais, e todas as boas árvores cortareis, e tapareis todas as fontes de água, e danificareis com pedras todos os bons campos.
\par 20 Pela manhã, ao apresentar-se a oferta de manjares, eis que vinham as águas pelo caminho de Edom; e a terra se encheu de água.
\par 21 Ouvindo, pois, todos os moabitas que os reis tinham subido para pelejar contra eles, todos os que cingiam cinto, desde o mais novo até ao mais velho, foram convocados e postos nas fronteiras.
\par 22 Levantando-se de madrugada, em saindo o sol sobre as águas, viram os moabitas defronte deles as águas vermelhas como sangue.
\par 23 E disseram: Isto é sangue; certamente, os reis se destruíram e se mataram um ao outro! Agora, pois, à presa, ó Moabe!
\par 24 Porém, chegando eles ao arraial de Israel, os israelitas se levantaram e feriram aos moabitas, os quais fugiram diante deles; entraram os israelitas na terra e também aí feriram aos moabitas.
\par 25 Arrasaram as cidades, e cada um lançou a sua pedra em todos os bons campos, e os entulharam, e taparam todas as fontes de águas, e cortaram todas as boas árvores, até que só Quir-Haresete ficou com seus muros; mas os que atiravam com fundas a cercaram e a feriram.
\par 26 Vendo o rei de Moabe que a peleja prevalecia contra ele, tomou consigo setecentos homens que arrancavam espada, para romperem contra o rei de Edom, porém não puderam.
\par 27 Então, tomou a seu filho primogênito, que havia de reinar em seu lugar, e o ofereceu em holocausto sobre o muro; pelo que houve grande ira contra Israel; por isso, se retiraram dali e voltaram para a sua própria terra.

\chapter{4}

\par 1 Certa mulher, das mulheres dos discípulos dos profetas, clamou a Eliseu, dizendo: Meu marido, teu servo, morreu; e tu sabes que ele temia ao SENHOR. É chegado o credor para levar os meus dois filhos para lhe serem escravos.
\par 2 Eliseu lhe perguntou: Que te hei de fazer? Dize-me que é o que tens em casa. Ela respondeu: Tua serva não tem nada em casa, senão uma botija de azeite.
\par 3 Então, disse ele: Vai, pede emprestadas vasilhas a todos os teus vizinhos; vasilhas vazias, não poucas.
\par 4 Então, entra, e fecha a porta sobre ti e sobre teus filhos, e deita o teu azeite em todas aquelas vasilhas; põe à parte a que estiver cheia.
\par 5 Partiu, pois, dele e fechou a porta sobre si e sobre seus filhos; estes lhe chegavam as vasilhas, e ela as enchia.
\par 6 Cheias as vasilhas, disse ela a um dos filhos: Chega-me, aqui, mais uma vasilha. Mas ele respondeu: Não há mais vasilha nenhuma. E o azeite parou.
\par 7 Então, foi ela e fez saber ao homem de Deus; ele disse: Vai, vende o azeite e paga a tua dívida; e, tu e teus filhos, vivei do resto.
\par 8 Certo dia, passou Eliseu por Suném, onde se achava uma mulher rica, a qual o constrangeu a comer pão. Daí, todas as vezes que passava por lá, entrava para comer.
\par 9 Ela disse a seu marido: Vejo que este que passa sempre por nós é santo homem de Deus.
\par 10 Façamos-lhe, pois, em cima, um pequeno quarto, obra de pedreiro, e ponhamos-lhe nele uma cama, uma mesa, uma cadeira e um candeeiro; quando ele vier à nossa casa, retirar-se-á para ali.
\par 11 Um dia, vindo ele para ali, retirou-se para o quarto e se deitou.
\par 12 Então, disse ao seu moço Geazi: Chama esta sunamita. Chamando-a ele, ela se pôs diante do profeta.
\par 13 Este dissera ao seu moço: Dize-lhe: Eis que tu nos tens tratado com muita abnegação; que se há de fazer por ti? Haverá alguma coisa de que se fale a teu favor ao rei ou ao comandante do exército? Ela respondeu: Habito no meio do meu povo.
\par 14 Então, disse o profeta: Que se há de fazer por ela? Geazi respondeu: Ora, ela não tem filho, e seu marido é velho.
\par 15 Disse Eliseu: Chama-a. Chamando-a ele, ela se pôs à porta.
\par 16 Disse-lhe o profeta: Por este tempo, daqui a um ano, abraçarás um filho. Ela disse: Não, meu senhor, homem de Deus, não mintas à tua serva.
\par 17 Concebeu a mulher e deu à luz um filho, no tempo determinado, quando fez um ano, segundo Eliseu lhe dissera.
\par 18 Tendo crescido o menino, saiu, certo dia, a ter com seu pai, que estava com os segadores.
\par 19 Disse a seu pai: Ai! A minha cabeça! Então, o pai disse ao seu moço: Leva-o a sua mãe.
\par 20 Ele o tomou e o levou a sua mãe, sobre cujos joelhos ficou sentado até ao meio-dia, e morreu.
\par 21 Subiu ela e o deitou sobre a cama do homem de Deus; fechou a porta e saiu.
\par 22 Chamou a seu marido e lhe disse: Manda-me um dos moços e uma das jumentas, para que eu corra ao homem de Deus e volte.
\par 23 Perguntou ele: Por que vais a ele hoje? Não é dia de Festa da Lua Nova nem sábado. Ela disse: Não faz mal.
\par 24 Então, fez ela albardar a jumenta e disse ao moço: Guia e anda, não te detenhas no caminhar, senão quando eu to disser.
\par 25 Partiu ela, pois, e foi ter com o homem de Deus, ao monte Carmelo. Vendo-a de longe o homem de Deus, disse a Geazi, seu moço: Eis aí a sunamita;
\par 26 corre ao seu encontro e dize-lhe: Vai tudo bem contigo, com teu marido, com o menino? Ela respondeu: Tudo bem.
\par 27 Chegando ela, pois, ao homem de Deus, ao monte, abraçou-lhe os pés. Então, se chegou Geazi para arrancá-la; mas o homem de Deus lhe disse: Deixa-a, porque a sua alma está em amargura, e o SENHOR mo encobriu e não mo manifestou.
\par 28 Disse ela: Pedi eu a meu senhor algum filho? Não disse eu: Não me enganes?
\par 29 Disse o profeta a Geazi: Cinge os lombos, toma o meu bordão contigo e vai. Se encontrares alguém, não o saúdes, e, se alguém te saudar, não lhe respondas; põe o meu bordão sobre o rosto do menino.
\par 30 Porém disse a mãe do menino: Tão certo como vive o SENHOR e vive a tua alma, não te deixarei. Então, ele se levantou e a seguiu.
\par 31 Geazi passou adiante deles e pôs o bordão sobre o rosto do menino; porém não houve nele voz nem sinal de vida; então, voltou a encontrar-se com Eliseu, e lhe deu aviso, e disse: O menino não despertou.
\par 32 Tendo o profeta chegado à casa, eis que o menino estava morto sobre a cama.
\par 33 Então, entrou, fechou a porta sobre eles ambos e orou ao SENHOR.
\par 34 Subiu à cama, deitou-se sobre o menino e, pondo a sua boca sobre a boca dele, os seus olhos sobre os olhos dele e as suas mãos sobre as mãos dele, se estendeu sobre ele; e a carne do menino aqueceu.
\par 35 Então, se levantou, e andou no quarto uma vez de lá para cá, e tornou a subir, e se estendeu sobre o menino; este espirrou sete vezes e abriu os olhos.
\par 36 Então, chamou a Geazi e disse: Chama a sunamita. Ele a chamou, e, apresentando-se ela ao profeta, este lhe disse: Toma o teu filho.
\par 37 Ela entrou, lançou-se aos pés dele e prostrou-se em terra; tomou o seu filho e saiu.
\par 38 Voltou Eliseu para Gilgal. Havia fome naquela terra, e, estando os discípulos dos profetas assentados diante dele, disse ao seu moço: Põe a panela grande ao lume e faze um cozinhado para os discípulos dos profetas.
\par 39 Então, saiu um ao campo a apanhar ervas e achou uma trepadeira silvestre; e, colhendo dela, encheu a sua capa de colocíntidas; voltou e cortou-as em pedaços, pondo-os na panela, visto que não as conheciam.
\par 40 Depois, deram de comer aos homens. Enquanto comiam do cozinhado, exclamaram: Morte na panela, ó homem de Deus! E não puderam comer.
\par 41 Porém ele disse: Trazei farinha. Ele a deitou na panela e disse: Tira de comer para o povo. E já não havia mal nenhum na panela.
\par 42 Veio um homem de Baal-Salisa e trouxe ao homem de Deus pães das primícias, vinte pães de cevada, e espigas verdes no seu alforje. Disse Eliseu: Dá ao povo para que coma.
\par 43 Porém seu servo lhe disse: Como hei de eu pôr isto diante de cem homens? Ele tornou a dizer: Dá-o ao povo, para que coma; porque assim diz o SENHOR: Comerão, e sobejará.
\par 44 Então, lhos pôs diante; comeram, e ainda sobrou, conforme a palavra do SENHOR.

\chapter{5}

\par 1 Naamã, comandante do exército do rei da Síria, era grande homem diante do seu senhor e de muito conceito, porque por ele o SENHOR dera vitória à Síria; era ele herói da guerra, porém leproso.
\par 2 Saíram tropas da Síria, e da terra de Israel levaram cativa uma menina, que ficou ao serviço da mulher de Naamã.
\par 3 Disse ela à sua senhora: Tomara o meu senhor estivesse diante do profeta que está em Samaria; ele o restauraria da sua lepra.
\par 4 Então, foi Naamã e disse ao seu senhor: Assim e assim falou a jovem que é da terra de Israel.
\par 5 Respondeu o rei da Síria: Vai, anda, e enviarei uma carta ao rei de Israel. Ele partiu e levou consigo dez talentos de prata, seis mil siclos de ouro e dez vestes festivais.
\par 6 Levou também ao rei de Israel a carta, que dizia: Logo, em chegando a ti esta carta, saberás que eu te enviei Naamã, meu servo, para que o cures da sua lepra.
\par 7 Tendo lido o rei de Israel a carta, rasgou as suas vestes e disse: Acaso, sou Deus com poder de tirar a vida ou dá-la, para que este envie a mim um homem para eu curá-lo de sua lepra? Notai, pois, e vede que procura um pretexto para romper comigo.
\par 8 Ouvindo, porém, Eliseu, homem de Deus, que o rei de Israel rasgara as suas vestes, mandou dizer ao rei: Por que rasgaste as tuas vestes? Deixa-o vir a mim, e saberá que há profeta em Israel.
\par 9 Veio, pois, Naamã com os seus cavalos e os seus carros e parou à porta da casa de Eliseu.
\par 10 Então, Eliseu lhe mandou um mensageiro, dizendo: Vai, lava-te sete vezes no Jordão, e a tua carne será restaurada, e ficarás limpo.
\par 11 Naamã, porém, muito se indignou e se foi, dizendo: Pensava eu que ele sairia a ter comigo, pôr-se-ia de pé, invocaria o nome do SENHOR, seu Deus, moveria a mão sobre o lugar da lepra e restauraria o leproso.
\par 12 Não são, porventura, Abana e Farfar, rios de Damasco, melhores do que todas as águas de Israel? Não poderia eu lavar-me neles e ficar limpo? E voltou-se e se foi com indignação.
\par 13 Então, se chegaram a ele os seus oficiais e lhe disseram: Meu pai, se te houvesse dito o profeta alguma coisa difícil, acaso, não a farias? Quanto mais, já que apenas te disse: Lava-te e ficarás limpo.
\par 14 Então, desceu e mergulhou no Jordão sete vezes, consoante a palavra do homem de Deus; e a sua carne se tornou como a carne de uma criança, e ficou limpo.
\par 15 Voltou ao homem de Deus, ele e toda a sua comitiva; veio, pôs-se diante dele e disse: Eis que, agora, reconheço que em toda a terra não há Deus, senão em Israel; agora, pois, te peço aceites um presente do teu servo.
\par 16 Porém ele disse: Tão certo como vive o SENHOR, em cuja presença estou, não o aceitarei. Instou com ele para que o aceitasse, mas ele recusou.
\par 17 Disse Naamã: Se não queres, peço-te que ao teu servo seja dado levar uma carga de terra de dois mulos; porque nunca mais oferecerá este teu servo holocausto nem sacrifício a outros deuses, senão ao SENHOR.
\par 18 Nisto perdoe o SENHOR a teu servo; quando o meu senhor entra na casa de Rimom para ali adorar, e ele se encosta na minha mão, e eu também me tenha de encurvar na casa de Rimom, quando assim me prostrar na casa de Rimom, nisto perdoe o SENHOR a teu servo.
\par 19 Eliseu lhe disse: Vai em paz. Quando Naamã se tinha afastado certa distância,
\par 20 Geazi, o moço de Eliseu, homem de Deus, disse consigo: Eis que meu senhor impediu a este siro Naamã que da sua mão se lhe desse alguma coisa do que trazia; porém, tão certo como vive o SENHOR, hei de correr atrás dele e receberei dele alguma coisa.
\par 21 Então, foi Geazi em alcance de Naamã; Naamã, vendo que corria atrás dele, saltou do carro a encontrá-lo e perguntou: Vai tudo bem?
\par 22 Ele respondeu: Tudo vai bem; meu senhor me mandou dizer: Eis que, agora mesmo, vieram a mim dois jovens, dentre os discípulos dos profetas da região montanhosa de Efraim; dá-lhes, pois, um talento de prata e duas vestes festivais.
\par 23 Disse Naamã: Sê servido tomar dois talentos. Instou com ele e amarrou dois talentos de prata em dois sacos e duas vestes festivais; pô-los sobre dois dos seus moços, os quais os levaram adiante dele.
\par 24 Tendo ele chegado ao outeiro, tomou-os das suas mãos e os depositou na casa; e despediu aqueles homens, que se foram.
\par 25 Ele, porém, entrou e se pôs diante de seu senhor. Perguntou-lhe Eliseu: Donde vens, Geazi? Respondeu ele: Teu servo não foi a parte alguma.
\par 26 Porém ele lhe disse: Porventura, não fui contigo em espírito quando aquele homem voltou do seu carro, a encontrar-te? Era isto ocasião para tomares prata e para tomares vestes, olivais e vinhas, ovelhas e bois, servos e servas?
\par 27 Portanto, a lepra de Naamã se pegará a ti e à tua descendência para sempre. Então, saiu de diante dele leproso, branco como a neve.

\chapter{6}

\par 1 Disseram os discípulos dos profetas a Eliseu: Eis que o lugar em que habitamos contigo é estreito demais para nós.
\par 2 Vamos, pois, até ao Jordão, tomemos de lá, cada um de nós uma viga, e construamos um lugar em que habitemos. Respondeu ele: Ide.
\par 3 Disse um: Serve-te de ires com os teus servos. Ele tornou: Eu irei.
\par 4 E foi com eles. Chegados ao Jordão, cortaram madeira.
\par 5 Sucedeu que, enquanto um deles derribava um tronco, o machado caiu na água; ele gritou e disse: Ai! Meu senhor! Porque era emprestado.
\par 6 Perguntou o homem de Deus: Onde caiu? Mostrou-lhe ele o lugar. Então, Eliseu cortou um pau, e lançou-o ali, e fez flutuar o ferro,
\par 7 e disse: Levanta-o. Estendeu ele a mão e o tomou.
\par 8 O rei da Síria fez guerra a Israel e, em conselho com os seus oficiais, disse: Em tal e tal lugar, estará o meu acampamento.
\par 9 Mas o homem de Deus mandou dizer ao rei de Israel: Guarda-te de passares por tal lugar, porque os siros estão descendo para ali.
\par 10 O rei de Israel enviou tropas ao lugar de que o homem de Deus lhe falara e de que o tinha avisado, e, assim, se salvou, não uma nem duas vezes.
\par 11 Então, tendo-se turbado com este incidente o coração do rei da Síria, chamou ele os seus servos e lhes disse: Não me fareis saber quem dos nossos é pelo rei de Israel?
\par 12 Respondeu um dos seus servos: Ninguém, ó rei, meu senhor; mas o profeta Eliseu, que está em Israel, faz saber ao rei de Israel as palavras que falas na tua câmara de dormir.
\par 13 Ele disse: Ide e vede onde ele está, para que eu mande prendê-lo. Foi-lhe dito: Eis que está em Dotã.
\par 14 Então, enviou para lá cavalos, carros e fortes tropas; chegaram de noite e cercaram a cidade.
\par 15 Tendo-se levantado muito cedo o moço do homem de Deus e saído, eis que tropas, cavalos e carros haviam cercado a cidade; então, o seu moço lhe disse: Ai! Meu senhor! Que faremos?
\par 16 Ele respondeu: Não temas, porque mais são os que estão conosco do que os que estão com eles.
\par 17 Orou Eliseu e disse: SENHOR, peço-te que lhe abras os olhos para que veja. O SENHOR abriu os olhos do moço, e ele viu que o monte estava cheio de cavalos e carros de fogo, em redor de Eliseu.
\par 18 E, como desceram contra ele, orou Eliseu ao SENHOR e disse: Fere, peço-te, esta gente de cegueira. Feriu-a de cegueira, conforme a palavra de Eliseu.
\par 19 Então, Eliseu lhes disse: Não é este o caminho, nem esta a cidade; segui-me, e guiar-vos-ei ao homem que buscais. E os guiou a Samaria.
\par 20 Tendo eles chegado a Samaria, disse Eliseu: Ó SENHOR, abre os olhos destes homens para que vejam. Abriu-lhes o SENHOR os olhos, e viram; e eis que estavam no meio de Samaria.
\par 21 Quando o rei de Israel os viu, perguntou a Eliseu: Feri-los-ei, feri-los-ei, meu pai?
\par 22 Respondeu ele: Não os ferirás; fere aqueles que fizeres prisioneiros com a tua espada e o teu arco. Porém a estes, manda pôr-lhes diante pão e água, para que comam, e bebam, e tornem a seu senhor.
\par 23 Ofereceu-lhes o rei grande banquete, e comeram e beberam; despediu-os, e foram para seu senhor; e da parte da Síria não houve mais investidas na terra de Israel.
\par 24 Depois disto, ajuntou Ben-Hadade, rei da Síria, todo o seu exército, subiu e sitiou a Samaria.
\par 25 Houve grande fome em Samaria; eis que a sitiaram, a ponto de se vender a cabeça de um jumento por oitenta siclos de prata e um pouco de esterco de pombas por cinco siclos de prata.
\par 26 Passando o rei de Israel pelo muro, gritou-lhe uma mulher: Acode-me, ó rei, meu senhor!
\par 27 Ele lhe disse: Se o SENHOR te não acode, donde te acudirei eu? Da eira ou do lagar?
\par 28 Perguntou-lhe o rei: Que tens? Respondeu ela: Esta mulher me disse: Dá teu filho, para que, hoje, o comamos e, amanhã, comeremos o meu.
\par 29 Cozemos, pois, o meu filho e o comemos; mas, dizendo-lhe eu ao outro dia: Dá o teu filho, para que o comamos, ela o escondeu.
\par 30 Tendo o rei ouvido as palavras da mulher, rasgou as suas vestes, quando passava pelo muro; o povo olhou e viu que trazia pano de saco por dentro, sobre a pele.
\par 31 Disse o rei: Assim me faça Deus o que bem lhe aprouver se a cabeça de Eliseu, filho de Safate, lhe ficar, hoje, sobre os ombros.
\par 32 Estava, porém, Eliseu sentado em sua casa, juntamente com os anciãos. Enviou o rei um homem de diante de si; mas, antes que o mensageiro chegasse a Eliseu, disse este aos anciãos: Vedes como o filho do homicida mandou tirar-me a cabeça? Olhai, quando vier o mensageiro, fechai-lhe a porta e empurrai-o com ela; porventura, não vem após ele o ruído dos pés de seu senhor?
\par 33 Falava ele ainda com eles, quando lhe chegou o mensageiro; disse o rei: Eis que este mal vem do SENHOR; que mais, pois, esperaria eu do SENHOR?

\chapter{7}

\par 1 Então, disse Eliseu: Ouvi a palavra do SENHOR; assim diz o SENHOR: Amanhã, a estas horas mais ou menos, dar-se-á um alqueire de flor de farinha por um siclo, e dois de cevada, por um siclo, à porta de Samaria.
\par 2 Porém o capitão a cujo braço o rei se apoiava respondeu ao homem de Deus: Ainda que o SENHOR fizesse janelas no céu, poderia suceder isso? Disse o profeta: Eis que tu o verás com os teus olhos, porém disso não comerás.
\par 3 Quatro homens leprosos estavam à entrada da porta, os quais disseram uns aos outros: Para que estaremos nós aqui sentados até morrermos?
\par 4 Se dissermos: entremos na cidade, há fome na cidade, e morreremos lá; se ficarmos sentados aqui, também morreremos. Vamos, pois, agora, e demos conosco no arraial dos siros; se nos deixarem viver, viveremos; se nos matarem, tão-somente morreremos.
\par 5 Levantaram-se ao anoitecer para se dirigirem ao arraial dos siros; e, tendo chegado à entrada do arraial, eis que não havia lá ninguém.
\par 6 Porque o Senhor fizera ouvir no arraial dos siros ruído de carros e de cavalos e o ruído de um grande exército; de maneira que disseram uns aos outros: Eis que o rei de Israel alugou contra nós os reis dos heteus e os reis dos egípcios, para virem contra nós.
\par 7 Pelo que se levantaram, e, fugindo ao anoitecer, deixaram as suas tendas, os seus cavalos, e os seus jumentos, e o arraial como estava; e fugiram para salvar a sua vida.
\par 8 Chegando, pois, aqueles leprosos à entrada do arraial, entraram numa tenda, e comeram, e beberam, e tomaram dali prata, e ouro, e vestes, e se foram, e os esconderam; voltaram, e entraram em outra tenda, e dali também tomaram alguma coisa, e a esconderam.
\par 9 Então, disseram uns para os outros: Não fazemos bem; este dia é dia de boas-novas, e nós nos calamos; se esperarmos até à luz da manhã, seremos tidos por culpados; agora, pois, vamos e o anunciemos à casa do rei.
\par 10 Vieram, pois, e bradaram aos porteiros da cidade, e lhes anunciaram, dizendo: Fomos ao arraial dos siros, e eis que lá não havia ninguém, voz de ninguém, mas somente cavalos e jumentos atados, e as tendas como estavam.
\par 11 Então, os porteiros gritaram e fizeram anunciar a nova no interior da casa do rei.
\par 12 Levantou-se o rei de noite e disse a seus servos: Agora, eu vos direi o que é que os siros nos fizeram. Bem sabem eles que estamos esfaimados; por isso, saíram do arraial, a esconder-se pelo campo, dizendo: Quando saírem da cidade, então, os tomaremos vivos e entraremos nela.
\par 13 Então, um dos seus servos respondeu e disse: Tomem-se, pois, cinco dos cavalos que ainda restam na cidade, pois toda a multidão de Israel que ficou aqui de resto terá a mesma sorte da multidão dos israelitas que já pereceram; enviemos homens e vejamos.
\par 14 Tomaram, pois, dois carros com cavalos; e o rei enviou os homens após o exército dos siros, dizendo: Ide e vede.
\par 15 Foram após eles até ao Jordão; e eis que todo o caminho estava cheio de vestes e de armas que os siros, na sua pressa, tinham lançado fora. Voltaram os mensageiros e o anunciaram ao rei.
\par 16 Então, saiu o povo e saqueou o arraial dos siros; e, assim, se vendia um alqueire de flor de farinha por um siclo, e dois de cevada, por um siclo, segundo a palavra do SENHOR.
\par 17 Dera o rei a guarda da porta ao capitão em cujo braço se apoiara, mas o povo o atropelou na porta, e ele morreu, como falara o homem de Deus, o que falou quando o rei descera a ele.
\par 18 Assim se cumpriu o que falara o homem de Deus ao rei: Amanhã, a estas horas mais ou menos, vender-se-ão dois alqueires de cevada por um siclo, e um de flor de farinha, por um siclo, à porta de Samaria.
\par 19 Aquele capitão respondera ao homem de Deus: Ainda que o SENHOR fizesse janelas no céu, poderia suceder isso, segundo essa palavra? Dissera o profeta: Eis que tu o verás com os teus olhos, porém disso não comerás.
\par 20 Assim lhe sucedeu, porque o povo o atropelou na porta, e ele morreu.

\chapter{8}

\par 1 Falou Eliseu àquela mulher cujo filho ele restaurara à vida, dizendo: Levanta-te, vai com os de tua casa e mora onde puderes; porque o SENHOR chamou a fome, a qual virá sobre a terra por sete anos.
\par 2 Levantou-se a mulher e fez segundo a palavra do homem de Deus: saiu com os de sua casa e habitou por sete anos na terra dos filisteus.
\par 3 Ao cabo dos sete anos, a mulher voltou da terra dos filisteus e saiu a clamar ao rei pela sua casa e pelas suas terras.
\par 4 Ora, o rei falava a Geazi, moço do homem de Deus, dizendo: Conta-me, peço-te, todas as grandes obras que Eliseu tem feito.
\par 5 Contava ele ao rei como Eliseu restaurara à vida a um morto, quando a mulher cujo filho ele havia restaurado à vida clamou ao rei pela sua casa e pelas suas terras; então, disse Geazi: Ó rei, meu senhor, esta é a mulher, e este, o seu filho, a quem Eliseu restaurou à vida.
\par 6 Interrogou o rei a mulher, e ela lhe contou tudo. Então, o rei lhe deu um oficial, dizendo: Faze restituir-se-lhe tudo quanto era seu e todas as rendas do campo desde o dia em que deixou a terra até agora.
\par 7 Veio Eliseu a Damasco. Estava doente Ben-Hadade, rei da Síria; e lhe anunciaram, dizendo: O homem de Deus é chegado aqui.
\par 8 Então, o rei disse a Hazael: Toma presentes contigo, e vai encontrar-te com o homem de Deus, e, por seu intermédio, pergunta ao SENHOR, dizendo: Sararei eu desta doença?
\par 9 Foi, pois, Hazael encontrar-se com ele, levando consigo um presente, a saber, quarenta camelos carregados de tudo que era bom de Damasco; chegou, apresentou-se diante dele e disse: Teu filho Ben-Hadade, rei da Síria, me enviou a perguntar-te: Sararei eu desta doença?
\par 10 Eliseu lhe respondeu: Vai e dize-lhe: Certamente, sararás. Porém o SENHOR me mostrou que ele morrerá.
\par 11 Olhou Eliseu para Hazael e tanto lhe fitou os olhos, que este ficou embaraçado; e chorou o homem de Deus.
\par 12 Então, disse Hazael: Por que chora o meu senhor? Ele respondeu: Porque sei o mal que hás de fazer aos filhos de Israel; deitarás fogo às suas fortalezas, matarás à espada os seus jovens, esmagarás os seus pequeninos e rasgarás o ventre de suas mulheres grávidas.
\par 13 Tornou Hazael: Pois que é teu servo, este cão, para fazer tão grandes coisas? Respondeu Eliseu: O SENHOR me mostrou que tu hás de ser rei da Síria.
\par 14 Então, deixou a Eliseu e veio a seu senhor, o qual lhe perguntou: Que te disse Eliseu? Respondeu ele: Disse-me que certamente sararás.
\par 15 No dia seguinte, Hazael tomou um cobertor, molhou-o em água e o estendeu sobre o rosto do rei até que morreu; e Hazael reinou em seu lugar.
\par 16 No ano quinto do reinado de Jorão, filho de Acabe, rei de Israel, reinando ainda Josafá em Judá, começou a reinar Jeorão, filho de Josafá, rei de Judá.
\par 17 Era ele da idade de trinta e dois anos quando começou a reinar e reinou oito anos em Jerusalém.
\par 18 Andou nos caminhos dos reis de Israel, como também fizeram os da casa de Acabe, porque a filha deste era sua mulher; e fez o que era mau perante o SENHOR.
\par 19 Porém o SENHOR não quis destruir a Judá por amor de Davi, seu servo, segundo a promessa que lhe havia feito de lhe dar sempre uma lâmpada e a seus filhos.
\par 20 Nos dias de Jeorão, se revoltaram os edomitas contra o poder de Judá e constituíram o seu próprio rei.
\par 21 Pelo que Jeorão passou a Zair, e todos os carros, com ele; ele se levantou de noite e feriu os edomitas que o cercavam e os capitães dos carros; o povo de Jeorão, porém, fugiu para as suas tendas.
\par 22 Assim se rebelou Edom para livrar-se do poder de Judá até ao dia de hoje; ao mesmo tempo, se rebelou também Libna.
\par 23 Quanto aos mais atos de Jeorão e a tudo quanto fez, porventura, não estão escritos no Livro da História dos Reis de Judá?
\par 24 Descansou Jeorão com seus pais e com eles foi sepultado na Cidade de Davi; e Acazias, seu filho, reinou em seu lugar.
\par 25 No décimo segundo ano de Jorão, filho de Acabe, rei de Israel, começou a reinar Acazias, filho de Jeorão, rei de Judá.
\par 26 Era Acazias de vinte e dois anos de idade quando começou a reinar e reinou um ano em Jerusalém. Sua mãe, filha de Onri, rei de Israel, chamava-se Atalia.
\par 27 Ele andou no caminho da casa de Acabe e fez o que era mau perante o SENHOR, como a casa de Acabe, porque era genro da casa de Acabe.
\par 28 Foi com Jorão, filho de Acabe, a Ramote-Gileade, à peleja contra Hazael, rei da Síria; e os siros feriram Jorão.
\par 29 Então, voltou o rei Jorão para Jezreel, para curar-se das feridas que os siros lhe fizeram em Ramá, quando pelejou contra Hazael, rei da Síria; e desceu Acazias, filho de Jeorão, rei de Judá, para ver a Jorão, filho de Acabe, em Jezreel, porquanto estava doente.

\chapter{9}

\par 1 Então, o profeta Eliseu chamou um dos discípulos dos profetas e lhe disse: Cinge os lombos, leva contigo este vaso de azeite e vai-te a Ramote-Gileade;
\par 2 em lá chegando, vê onde está Jeú, filho de Josafá, filho de Ninsi; entra, e faze-o levantar-se do meio de seus irmãos, e leva-o à câmara interior.
\par 3 Toma o vaso de azeite, derrama-lho sobre a cabeça e dize: Assim diz o SENHOR: Ungi-te rei sobre Israel. Então, abre a porta, foge e não te detenhas.
\par 4 Foi, pois, o moço, o jovem profeta, a Ramote-Gileade.
\par 5 Entrando ele, eis que os capitães do exército estavam assentados; ele disse: Capitão, tenho mensagem que te dizer. Perguntou-lhe Jeú: A qual de todos nós? Respondeu-lhe ele: A ti, capitão!
\par 6 Então, se levantou Jeú e entrou na casa; o jovem derramou-lhe o azeite sobre a cabeça e lhe disse: Assim diz o SENHOR, Deus de Israel: Ungi-te rei sobre o povo do SENHOR, sobre Israel.
\par 7 Ferirás a casa de Acabe, teu senhor, para que eu vingue da mão de Jezabel o sangue de meus servos, os profetas, e o sangue de todos os servos do SENHOR.
\par 8 Toda a casa de Acabe perecerá; exterminarei de Acabe todos do sexo masculino, quer escravo, quer livre, em Israel.
\par 9 Porque farei à casa de Acabe como à casa de Jeroboão, filho de Nebate, e como à casa de Baasa, filho de Aías.
\par 10 Os cães devorarão Jezabel no campo de Jezreel; não haverá quem a enterre. Dito isto, abriu a porta e fugiu.
\par 11 Saindo Jeú aos servos de seu senhor, disseram-lhe: Vai tudo bem? Por que veio a ti este louco? Ele lhes respondeu: Bem conheceis esse homem e o seu falar.
\par 12 Mas eles disseram: É mentira; agora, faze-nos sabê-lo, te pedimos. Então, disse Jeú: Assim e assim me falou, a saber: Assim diz o SENHOR: Ungi-te rei sobre Israel.
\par 13 Então, se apressaram, e, tomando cada um o seu manto, os puseram debaixo dele, sobre os degraus, e tocaram a trombeta, e disseram: Jeú é rei!
\par 14 Assim, Jeú, filho de Josafá, filho de Ninsi, conspirou contra Jorão. Tinha, porém, Jorão cercado a Ramote-Gileade, ele e todo o Israel, por causa de Hazael, rei da Síria.
\par 15 Porém o rei Jorão voltou para se curar em Jezreel das feridas que os siros lhe fizeram, quando pelejou contra Hazael, rei da Síria. Disse Jeú: Se é da vossa vontade, ninguém saia furtivamente da cidade, para ir anunciar isto em Jezreel.
\par 16 Então, Jeú subiu a um carro e foi-se a Jezreel, porque Jorão estava de cama ali. Também Acazias, rei de Judá, descera para ver a Jorão.
\par 17 Ora, o atalaia estava na torre de Jezreel, e viu a tropa de Jeú, que vinha, e disse: Vejo uma tropa. Então, disse Jorão: Toma um cavaleiro e envia-o ao seu encontro, para que lhe pergunte: Há paz?
\par 18 Foi-lhe o cavaleiro ao encontro e disse: Assim diz o rei: Há paz? Respondeu Jeú: Que tens tu com a paz? Passa para trás de mim. O atalaia deu aviso, dizendo: Chegou a eles o mensageiro, porém não volta.
\par 19 Então, enviou Jorão outro cavaleiro; chegando este a eles, disse: Assim diz o rei: Há paz? Respondeu Jeú: Que tens tu com a paz? Passa para trás de mim.
\par 20 O atalaia deu aviso, dizendo: Também este chegou a eles, porém não volta; e o guiar do carro parece como o de Jeú, filho de Ninsi, porque guia furiosamente.
\par 21 Disse Jorão: Aparelha o carro. E lhe aparelharam o carro. Saiu Jorão, rei de Israel, e Acazias, rei de Judá, cada um em seu carro, e foram ao encontro de Jeú, e o acharam no campo de Nabote, o jezreelita.
\par 22 Sucedeu que, vendo Jorão a Jeú, perguntou: Há paz, Jeú? Ele respondeu: Que paz, enquanto perduram as prostituições de tua mãe Jezabel e as suas muitas feitiçarias?
\par 23 Então, Jorão voltou as rédeas, fugiu e disse a Acazias: Há traição, Acazias!
\par 24 Mas Jeú entesou o seu arco com toda a força e feriu a Jorão entre as espáduas; a flecha saiu-lhe pelo coração, e ele caiu no seu carro.
\par 25 Então, Jeú disse a Bidcar, seu capitão: Toma-o, lança-o no campo da herdade de Nabote, o jezreelita; pois, lembra-te de que, indo eu e tu, juntos, montados, após Acabe, seu pai, o SENHOR pronunciou contra ele esta sentença:
\par 26 Tão certo como vi ontem à tarde o sangue de Nabote e o sangue de seus filhos, diz o SENHOR, assim to retribuirei neste campo, diz o SENHOR. Agora, pois, toma-o e lança-o neste campo, segundo a palavra do SENHOR.
\par 27 À vista disto, Acazias, rei de Judá, fugiu pelo caminho de Bete-Hagã; porém Jeú o perseguiu e disse: Feri também a este; e o feriram no carro, à subida de Gur, que está junto a Ibleão. E fugiu para Megido, onde morreu.
\par 28 Levaram-no os seus servos, num carro, a Jerusalém e o enterraram na sua sepultura junto a seus pais, na Cidade de Davi.
\par 29 No ano undécimo de Jorão, filho de Acabe, começara Acazias a reinar sobre Judá.
\par 30 Tendo Jeú chegado a Jezreel, Jezabel o soube; então, se pintou em volta dos olhos, enfeitou a cabeça e olhou pela janela.
\par 31 Ao entrar Jeú pelo portão do palácio, disse ela: Teve paz Zinri, que matou a seu senhor?
\par 32 Levantou ele o rosto para a janela e disse: Quem é comigo? Quem? E dois ou três eunucos olharam para ele.
\par 33 Então, disse ele: Lançai-a daí abaixo. Lançaram-na abaixo; e foram salpicados com o seu sangue a parede e os cavalos, e Jeú a atropelou.
\par 34 Entrando ele e havendo comido e bebido, disse: Olhai por aquela maldita e sepultai-a, porque é filha de rei.
\par 35 Foram para a sepultar; porém não acharam dela senão a caveira, os pés e as palmas das mãos.
\par 36 Então, voltaram e lho fizeram saber. Ele disse: Esta é a palavra do SENHOR, que falou por intermédio de Elias, o tesbita, seu servo, dizendo: No campo de Jezreel, os cães comerão a carne de Jezabel.
\par 37 O cadáver de Jezabel será como esterco sobre o campo da herdade de Jezreel, de maneira que já não dirão: Esta é Jezabel.

\chapter{10}

\par 1 Achando-se em Samaria setenta filhos de Acabe, Jeú escreveu cartas e as enviou a Samaria, aos chefes da cidade, aos anciãos e aos tutores dos filhos de Acabe, dizendo:
\par 2 Logo, em chegando a vós outros esta carta (pois estão convosco os filhos de vosso senhor, como também os carros, os cavalos, a cidade fortalecida e as armas),
\par 3 escolhei o melhor e mais capaz dos filhos de vosso senhor, ponde-o sobre o trono de seu pai e pelejai pela casa de vosso senhor.
\par 4 Porém eles temeram muitíssimo e disseram: Dois reis não puderam resistir a ele; como, pois, poderemos nós fazê-lo?
\par 5 Então, o responsável pelo palácio, e o responsável pela cidade, e os anciãos, e os tutores mandaram dizer a Jeú: Teus servos somos e tudo quanto nos ordenares faremos; a ninguém constituiremos rei; faze o que bem te parecer.
\par 6 Então, lhes escreveu outra carta, dizendo: Se estiverdes do meu lado e quiserdes obedecer-me, tomai as cabeças dos homens, filhos de vosso senhor, e amanhã a estas horas vinde a mim a Jezreel. Ora, os filhos do rei, que eram setenta, estavam com os grandes da cidade, que os criavam.
\par 7 Chegada a eles a carta, tomaram os filhos do rei, e os mataram, setenta pessoas, e puseram as suas cabeças nuns cestos, e lhas mandaram a Jezreel.
\par 8 Veio um mensageiro e lhe disse: Trouxeram as cabeças dos filhos do rei. Ele disse: Ponde-as em dois montões à entrada da porta, até pela manhã.
\par 9 Saindo ele pela manhã, parou e disse a todo o povo: Vós estais sem culpa; eis que eu conspirei contra o meu senhor e o matei; mas quem feriu todos estes?
\par 10 Sabei, pois, agora, que, da palavra do SENHOR, pronunciada contra a casa de Acabe, nada cairá em terra, porque o SENHOR fez o que falou por intermédio do seu servo Elias.
\par 11 Jeú feriu também todos os restantes da casa de Acabe em Jezreel, como também todos os seus grandes, os seus conhecidos e os seus sacerdotes, até que nem um sequer lhe deixou ficar de resto.
\par 12 Então, se dispôs, partiu e foi a Samaria. E, estando no caminho, em Bete-Equede dos Pastores,
\par 13 encontrou Jeú parentes de Acazias, rei de Judá, e perguntou: Quem sois vós? Eles responderam: Parentes de Acazias; voltamos de saudar os filhos do rei e os da rainha-mãe.
\par 14 Então, disse Jeú: Apanhai-os vivos. Eles os apanharam vivos e os mataram junto ao poço de Bete-Equede, quarenta e dois homens; e a nenhum deles deixou de resto.
\par 15 Tendo partido dali, encontrou a Jonadabe, filho de Recabe, que lhe vinha ao encontro; Jeú saudou-o e lhe perguntou: Tens tu sincero o coração para comigo, como o meu o é para contigo? Respondeu Jonadabe: Tenho. Então, se tens, dá-me a mão. Jonadabe deu-lhe a mão; e Jeú fê-lo subir consigo ao carro
\par 16 e lhe disse: Vem comigo e verás o meu zelo para com o SENHOR. E, assim, Jeú o levou no seu carro.
\par 17 Tendo Jeú chegado a Samaria, feriu todos os que ali ficaram de Acabe, até destruí-los, segundo a palavra que o SENHOR dissera a Elias.
\par 18 Ajuntou Jeú a todo o povo e lhe disse: Acabe serviu pouco a Baal; Jeú, porém, muito o servirá.
\par 19 Pelo que chamai-me, agora, todos os profetas de Baal, todos os seus servidores e todos os seus sacerdotes; não falte nenhum, porque tenho grande sacrifício a oferecer a Baal; todo aquele que faltar não viverá. Porém Jeú fazia isto com astúcia, para destruir os servidores de Baal.
\par 20 Disse mais Jeú: Consagrai uma assembléia solene a Baal; e a proclamaram.
\par 21 Também Jeú enviou mensageiros por todo o Israel; vieram todos os adoradores de Baal, e nenhum homem deles ficou que não viesse. Entraram na casa de Baal, que se encheu de uma extremidade à outra.
\par 22 Então, disse Jeú ao vestiário: Tira as vestimentas para todos os adoradores de Baal. E o fez.
\par 23 Entrou Jeú com Jonadabe, filho de Recabe, na casa de Baal e disse aos adoradores de Baal: Examinai e vede bem não esteja aqui entre vós algum dos servos do SENHOR, mas somente os adoradores de Baal.
\par 24 E, entrando eles a oferecerem sacrifícios e holocaustos, Jeú preparou da parte de fora oitenta homens e disse-lhes: Se escapar algum dos homens que eu entregar em vossas mãos, a vida daquele que o deixar escapar responderá pela vida dele.
\par 25 Sucedeu que, acabado o oferecimento do holocausto, ordenou Jeú aos da sua guarda e aos capitães: Entrai, feri-os, que nenhum escape. Feriram-nos a fio de espada; e os da guarda e os capitães os lançaram fora, e penetraram no mais interior da casa de Baal,
\par 26 e tiraram as colunas que estavam na casa de Baal, e as queimaram.
\par 27 Também quebraram a própria coluna de Baal, e derribaram a casa de Baal, e a transformaram em latrinas até ao dia de hoje.
\par 28 Assim, Jeú exterminou de Israel a Baal.
\par 29 Porém não se apartou Jeú de seguir os pecados de Jeroboão, filho de Nebate, que fez pecar a Israel, a saber, dos bezerros de ouro que estavam em Betel e em Dã.
\par 30 Pelo que disse o SENHOR a Jeú: Porquanto bem executaste o que é reto perante mim e fizeste à casa de Acabe segundo tudo quanto era do meu propósito, teus filhos até à quarta geração se assentarão no trono de Israel.
\par 31 Mas Jeú não teve cuidado de andar de todo o seu coração na lei do SENHOR, Deus de Israel, nem se apartou dos pecados que Jeroboão fez pecar a Israel.
\par 32 Naqueles dias, começou o SENHOR a diminuir os limites de Israel, que foi ferido por Hazael em todas as suas fronteiras,
\par 33 desde o Jordão para o nascente do sol, toda a terra de Gileade, os gaditas, os rubenitas e os manassitas, desde Aroer, que está junto ao vale de Arnom, a saber, Gileade e Basã.
\par 34 Ora, os mais atos de Jeú, e tudo quanto fez, e todo o seu poder, porventura, não estão escritos no Livro da História dos Reis de Israel?
\par 35 Descansou Jeú com seus pais, e o sepultaram em Samaria; e Jeoacaz, seu filho, reinou em seu lugar.
\par 36 Os dias que Jeú reinou sobre Israel em Samaria foram vinte e oito anos.

\chapter{11}

\par 1 Vendo Atalia, mãe de Acazias, que seu filho era morto, levantou-se e destruiu toda a descendência real.
\par 2 Mas Jeoseba, filha do rei Jorão e irmã de Acazias, tomou a Joás, filho de Acazias, e o furtou dentre os filhos do rei, aos quais matavam, e pôs a ele e a sua ama numa câmara interior; e, assim, o esconderam de Atalia, e não foi morto.
\par 3 Jeoseba o teve escondido na Casa do SENHOR seis anos; neste tempo, Atalia reinava sobre a terra.
\par 4 No sétimo ano, mandou Joiada chamar os capitães dos cários e da guarda e os fez entrar à sua presença na Casa do SENHOR; fez com eles aliança, e ajuramentou-os na Casa do SENHOR, e lhes mostrou o filho do rei.
\par 5 Então, lhes deu ordem, dizendo: Esta é a obra que haveis de fazer: uma terça parte de vós, que entrais no sábado, fará a guarda da casa do rei;
\par 6 e outra terça parte estará ao portão Sur; e a outra terça parte, ao portão detrás da guarda; assim, fareis a guarda e defesa desta casa.
\par 7 Os dois grupos que saem no sábado, estes todos farão a guarda da Casa do SENHOR, junto ao rei.
\par 8 Rodeareis o rei, cada um de armas na mão, e qualquer que pretenda penetrar nas fileiras, seja morto; estareis com o rei quando sair e quando entrar.
\par 9 Fizeram, pois, os capitães de cem segundo tudo quanto lhes ordenara o sacerdote Joiada; tomaram cada um os seus homens, tanto os que entravam como os que saíam no sábado, e vieram ao sacerdote Joiada.
\par 10 O sacerdote entregou aos capitães de cem as lanças e os escudos que haviam sido do rei Davi e estavam na Casa do SENHOR.
\par 11 Os da guarda se puseram, cada um de armas na mão, desde o lado direito da casa real até ao lado esquerdo, e até ao altar, e até ao templo, para rodear o rei.
\par 12 Então, Joiada fez sair o filho do rei, pôs-lhe a coroa e lhe deu o Livro do Testemunho; eles o constituíram rei, e o ungiram, e bateram palmas, e gritaram: Viva o rei!
\par 13 Ouvindo Atalia o clamor dos da guarda e do povo, veio para onde este se achava na Casa do SENHOR.
\par 14 Olhou, e eis que o rei estava junto à coluna, segundo o costume, e os capitães e os tocadores de trombetas, junto ao rei, e todo o povo da terra se alegrava, e se tocavam trombetas. Então, Atalia rasgou os seus vestidos e clamou: Traição! Traição!
\par 15 Porém o sacerdote Joiada deu ordem aos capitães que comandavam as tropas e disse-lhes: Fazei-a sair por entre as fileiras; se alguém a seguir, matai-o à espada. Porque o sacerdote tinha dito: Não a matem na Casa do SENHOR.
\par 16 Lançaram mão dela; e ela, pelo caminho da entrada dos cavalos, foi à casa do rei, onde a mataram.
\par 17 Joiada fez aliança entre o SENHOR, e o rei, e o povo, para serem eles o povo do SENHOR; como também entre o rei e o povo.
\par 18 Então, todo o povo da terra entrou na casa de Baal, e a derribaram; despedaçaram os seus altares e as suas imagens e a Matã, sacerdote de Baal, mataram perante os altares; então, o sacerdote pôs guardas sobre a Casa do SENHOR.
\par 19 Tomou os capitães dos cários, os da guarda e todo o povo da terra, e todos estes conduziram da Casa do SENHOR o rei e, pelo caminho da porta dos da guarda, vieram à casa real; e Joás sentou-se no trono dos reis.
\par 20 Alegrou-se todo o povo da terra, e a cidade ficou tranqüila, depois que mataram Atalia à espada, junto à casa do rei.
\par 21 Era Joás da idade de sete anos quando o fizeram rei.

\chapter{12}

\par 1 No ano sétimo de Jeú, começou Joás a reinar e quarenta anos reinou em Jerusalém. Era o nome de sua mãe Zíbia, de Berseba.
\par 2 Fez Joás o que era reto perante o SENHOR, todos os dias em que o sacerdote Joiada o dirigia.
\par 3 Tão-somente os altos não se tiraram; e o povo ainda sacrificava e queimava incenso nos altos.
\par 4 Disse Joás aos sacerdotes: Todo o dinheiro das coisas santas que se trouxer à Casa do SENHOR, a saber, a taxa pessoal, o resgate de pessoas segundo a sua avaliação e todo o dinheiro que cada um trouxer voluntariamente para a Casa do SENHOR,
\par 5 recebam-no os sacerdotes, cada um dos seus conhecidos; e eles reparem os estragos da casa onde quer que se encontrem.
\par 6 Sucedeu, porém, que, no ano vigésimo terceiro do rei Joás, os sacerdotes ainda não tinham reparado os estragos da casa.
\par 7 Então, o rei Joás chamou o sacerdote Joiada e os mais sacerdotes e lhes disse: Por que não reparais os estragos da casa? Agora, pois, não recebais mais dinheiro de vossos conhecidos, mas entregai-o para a reparação dos estragos da casa.
\par 8 Consentiram os sacerdotes, assim, em não receberem mais dinheiro do povo, como em não repararem os estragos da casa.
\par 9 Porém o sacerdote Joiada tomou uma caixa, e lhe fez na tampa um buraco, e a pôs ao pé do altar, à mão direita dos que entravam na Casa do SENHOR; os sacerdotes que guardavam a entrada da porta depositavam ali todo o dinheiro que se trazia à Casa do SENHOR.
\par 10 Quando viam que já havia muito dinheiro na caixa, o escrivão do rei subia com um sumo sacerdote, e contavam e ensacavam o dinheiro que se achava na Casa do SENHOR.
\par 11 O dinheiro, depois de pesado, davam nas mãos dos que dirigiam a obra e tinham a seu cargo a Casa do SENHOR; estes pagavam aos carpinteiros e aos edificadores que reparavam a Casa do SENHOR,
\par 12 como também aos pedreiros e aos cabouqueiros, e compravam madeira e pedras lavradas para repararem os estragos da Casa do SENHOR, e custeavam todo o necessário para a conservação da Casa do SENHOR.
\par 13 Mas, do dinheiro que se trazia à Casa do SENHOR, não se faziam nem taças de prata, nem espevitadeiras, nem bacias, nem trombetas, nem vaso algum de ouro ou de prata para a Casa do SENHOR.
\par 14 Porque o davam aos que dirigiam a obra e reparavam com ele a Casa do SENHOR.
\par 15 Também não pediam contas aos homens em cujas mãos entregavam aquele dinheiro, para o dar aos que faziam a obra, porque procediam com fidelidade.
\par 16 Mas o dinheiro de oferta pela culpa e o dinheiro de oferta pelos pecados não se traziam à Casa do SENHOR; eram para os sacerdotes.
\par 17 Então, subiu Hazael, rei da Síria, e pelejou contra Gate, e a tomou; depois, Hazael resolveu marchar contra Jerusalém.
\par 18 Porém Joás, rei de Judá, tomou todas as coisas santas que Josafá, Jeorão e Acazias, seus pais, reis de Judá, haviam dedicado, como também todo o ouro que se achava nos tesouros da Casa do SENHOR e na casa do rei e os mandou a Hazael, rei da Síria; e este se retirou de Jerusalém.
\par 19 Quanto aos mais atos de Joás e a tudo o que fez, porventura, não estão escritos no Livro da História dos Reis de Judá?
\par 20 Levantaram-se os seus servos, conspiraram e feriram Joás na casa de Milo, que está na descida para Sila.
\par 21 Porque Jozacar, filho de Simeate, e Jozabade, filho de Somer, seus servos, o feriram, e morreu; e o sepultaram com seus pais na Cidade de Davi. E Amazias, seu filho, reinou em seu lugar.

\chapter{13}

\par 1 No vigésimo terceiro ano de Joás, filho de Acazias, rei de Judá, começou a reinar Jeoacaz, filho de Jeú, sobre Israel, em Samaria, e reinou dezessete anos.
\par 2 E fez o que era mau perante o SENHOR; porque andou nos pecados de Jeroboão, filho de Nebate, que fez pecar a Israel; não se apartou deles.
\par 3 Pelo que se acendeu contra Israel a ira do SENHOR, o qual os entregou nas mãos de Hazael, rei da Síria, e nas mãos de Ben-Hadade, filho de Hazael, todos aqueles dias.
\par 4 Porém Jeoacaz fez súplicas diante do SENHOR, e o SENHOR o ouviu; pois viu a opressão com que o rei da Síria atormentava a Israel.
\par 5 O SENHOR deu um salvador a Israel, de modo que os filhos de Israel saíram de sob o poder dos siros e habitaram, de novo, em seus lares, como dantes.
\par 6 Contudo, não se apartaram dos pecados da casa de Jeroboão, que fez pecar a Israel, porém andaram neles; e também o poste-ídolo permaneceu em Samaria.
\par 7 E foi o caso que não se deixaram a Jeoacaz, do exército, senão cinqüenta cavaleiros, dez carros e dez mil homens de pé; porquanto o rei da Síria os havia destruído e feito como o pó, trilhando-os.
\par 8 Ora, os mais atos de Jeoacaz, e tudo o que fez, e o seu poder, porventura, não estão escritos no Livro da História dos Reis de Israel?
\par 9 Jeoacaz descansou com seus pais, e o sepultaram em Samaria; e Jeoás, seu filho, reinou em seu lugar.
\par 10 No trigésimo sétimo ano de Joás, rei de Judá, começou Jeoás, filho de Jeoacaz, a reinar sobre Israel, em Samaria; e reinou dezesseis anos.
\par 11 Fez o que era mau perante o SENHOR; não se apartou de nenhum dos pecados de Jeroboão, filho de Nebate, que fez pecar a Israel; porém andou neles.
\par 12 Quanto aos mais atos de Jeoás, e a tudo o que fez, e ao seu poder, com que pelejou contra Amazias, rei de Judá, porventura, não estão escritos no Livro da História dos Reis de Israel?
\par 13 Descansou Jeoás com seus pais, e no seu trono se assentou Jeroboão. Jeoás foi sepultado em Samaria, junto aos reis de Israel.
\par 14 Estando Eliseu padecendo da enfermidade de que havia de morrer, Jeoás, rei de Israel, desceu a visitá-lo, chorou sobre ele e disse: Meu pai, meu pai! Carros de Israel e seus cavaleiros!
\par 15 Então, lhe disse Eliseu: Toma um arco e flechas; ele tomou um arco e flechas.
\par 16 Disse ao rei de Israel: Retesa o arco; e ele o fez. Então, Eliseu pôs as mãos sobre as mãos do rei.
\par 17 E disse: Abre a janela para o oriente; ele a abriu. Disse mais Eliseu: Atira; e ele atirou. Prosseguiu: Flecha da vitória do SENHOR! Flecha da vitória contra os siros! Porque ferirás os siros em Afeca, até os consumir.
\par 18 Disse ainda: Toma as flechas. Ele as tomou. Então, disse ao rei de Israel: Atira contra a terra; ele a feriu três vezes e cessou.
\par 19 Então, o homem de Deus se indignou muito contra ele e disse: Cinco ou seis vezes a deverias ter ferido; então, feririas os siros até os consumir; porém, agora, só três vezes ferirás os siros.
\par 20 Morreu Eliseu, e o sepultaram. Ora, bandos dos moabitas costumavam invadir a terra, à entrada do ano.
\par 21 Sucedeu que, enquanto alguns enterravam um homem, eis que viram um bando; então, lançaram o homem na sepultura de Eliseu; e, logo que o cadáver tocou os ossos de Eliseu, reviveu o homem e se levantou sobre os pés.
\par 22 Hazael, rei da Síria, oprimiu a Israel todos os dias de Jeoacaz.
\par 23 Porém o SENHOR teve misericórdia de Israel, e se compadeceu dele, e se tornou para ele, por amor da aliança com Abraão, Isaque e Jacó; e não o quis destruir e não o lançou ainda da sua presença.
\par 24 Morreu Hazael, rei da Síria; e Ben-Hadade, seu filho, reinou em seu lugar.
\par 25 Jeoás, filho de Jeoacaz, retomou as cidades das mãos de Ben-Hadade, que este havia tomado das mãos de Jeoacaz, seu pai, na guerra; três vezes Jeoás o feriu e recuperou as cidades de Israel.

\chapter{14}

\par 1 No segundo ano de Jeoás, filho de Jeoacaz, rei de Israel, começou a reinar Amazias, filho de Joás, rei de Judá.
\par 2 Tinha vinte e cinco anos quando começou a reinar e vinte e nove reinou em Jerusalém. Era o nome de sua mãe Jeoadã, de Jerusalém.
\par 3 Fez ele o que era reto perante o SENHOR, ainda que não como Davi, seu pai; fez, porém, segundo tudo o que fizera Joás, seu pai.
\par 4 Tão-somente os altos não se tiraram; o povo ainda sacrificava e queimava incenso nos altos.
\par 5 Uma vez confirmado o reino em sua mão, matou os seus servos que tinham assassinado o rei, seu pai.
\par 6 Porém os filhos dos assassinos não matou, segundo está escrito no Livro da Lei de Moisés, no qual o SENHOR deu ordem, dizendo: Os pais não serão mortos por causa dos filhos, nem os filhos por causa dos pais; cada qual será morto pelo seu próprio pecado.
\par 7 Ele feriu dez mil edomitas no vale do Sal e tomou a Sela na guerra; e chamou o seu nome Jocteel, até ao dia de hoje.
\par 8 Então, Amazias enviou mensageiros a Jeoás, filho de Jeoacaz, filho de Jeú, rei de Israel, dizendo: Vem, meçamos armas.
\par 9 Porém Jeoás, rei de Israel, respondeu a Amazias, rei de Judá: O cardo que está no Líbano mandou dizer ao cedro que lá está: Dá tua filha por mulher a meu filho; mas os animais do campo, que estavam no Líbano, passaram e pisaram o cardo.
\par 10 Na verdade, feriste os edomitas, e o teu coração se ensoberbeceu; gloria-te disso e fica em casa; por que provocarias o mal para caíres tu, e Judá, contigo?
\par 11 Mas Amazias não quis atendê-lo. Subiu, então, Jeoás, rei de Israel, e Amazias, rei de Judá, e mediram armas em Bete-Semes, que está em Judá.
\par 12 Judá foi derrotado diante de Israel, e fugiu cada um para sua casa.
\par 13 Jeoás, rei de Israel, prendeu Amazias, rei de Judá, filho de Joás, filho de Acazias, em Bete-Semes; e veio a Jerusalém, cujo muro ele rompeu desde a Porta de Efraim até à Porta da Esquina, quatrocentos côvados.
\par 14 Tomou todo o ouro, e a prata, e todos os utensílios que se acharam na Casa do SENHOR e nos tesouros da casa do rei, como também reféns; e voltou para Samaria.
\par 15 Ora, os mais atos de Jeoás, o que fez, o seu poder e como pelejou contra Amazias, rei de Judá, porventura, não estão escritos no Livro da História dos Reis de Israel?
\par 16 Descansou Jeoás com seus pais e foi sepultado em Samaria, junto aos reis de Israel; e Jeroboão, seu filho, reinou em seu lugar.
\par 17 Amazias, filho de Joás, rei de Judá, viveu quinze anos depois da morte de Jeoás, filho de Jeoacaz, rei de Israel.
\par 18 Ora, os mais atos de Amazias, porventura, não estão escritos no Livro da História dos Reis de Judá?
\par 19 Conspiraram contra ele em Jerusalém, e ele fugiu para Laquis; porém enviaram após ele homens até Laquis; e o mataram ali.
\par 20 Trouxeram-no sobre cavalos e o sepultaram em Jerusalém, junto a seus pais, na Cidade de Davi.
\par 21 Todo o povo de Judá tomou a Uzias, que era de dezesseis anos, e o constituiu rei em lugar de Amazias, seu pai.
\par 22 Ele edificou a Elate e a restituiu a Judá, depois que o rei descansou com seus pais.
\par 23 No décimo quinto ano de Amazias, filho de Joás, rei de Judá, começou a reinar em Samaria Jeroboão, filho de Jeoás, rei de Israel; e reinou quarenta e um anos.
\par 24 Fez o que era mau perante o SENHOR; jamais se apartou de nenhum dos pecados de Jeroboão, filho de Nebate, que fez pecar a Israel.
\par 25 Restabeleceu ele os limites de Israel, desde a entrada de Hamate até ao mar da Planície, segundo a palavra do SENHOR, Deus de Israel, a qual falara por intermédio de seu servo Jonas, filho de Amitai, o profeta, o qual era de Gate-Hefer.
\par 26 Porque viu o SENHOR que a aflição de Israel era mui amarga, porque não havia nem escravo, nem livre, nem quem socorresse a Israel.
\par 27 Ainda não falara o SENHOR em apagar o nome de Israel de debaixo do céu; porém os livrou por intermédio de Jeroboão, filho de Jeoás.
\par 28 Quanto aos mais atos de Jeroboão, tudo quanto fez, o seu poder, como pelejou e como reconquistou Damasco e Hamate, pertencentes a Judá, para Israel, porventura, não estão escritos no Livro da História dos Reis de Israel?
\par 29 Descansou Jeroboão com seus pais, com os reis de Israel; e Zacarias, seu filho, reinou em seu lugar.

\chapter{15}

\par 1 No vigésimo sétimo ano de Jeroboão, rei de Israel, começou a reinar Azarias, filho de Amazias, rei de Judá.
\par 2 Tinha dezesseis anos quando começou a reinar e cinqüenta e dois anos reinou em Jerusalém. Era o nome de sua mãe Jecolias, de Jerusalém.
\par 3 Ele fez o que era reto perante o SENHOR, segundo tudo o que fizera Amazias, seu pai.
\par 4 Tão-somente os altos não se tiraram; o povo ainda sacrificava e queimava incenso nos altos.
\par 5 O SENHOR feriu ao rei, e este ficou leproso até ao dia da sua morte e habitava numa casa separada. Jotão, filho do rei, tinha o cargo da casa e governava o povo da terra.
\par 6 Ora, os mais atos de Azarias e tudo o que fez, porventura, não estão escritos no Livro da História dos Reis de Judá?
\par 7 Descansou Azarias com seus pais, e o sepultaram junto a seus pais, na Cidade de Davi; e Jotão, seu filho, reinou em seu lugar.
\par 8 No trigésimo oitavo ano de Azarias, rei de Judá, reinou Zacarias, filho de Jeroboão, sobre Israel, em Samaria, seis meses.
\par 9 Fez o que era mau perante o SENHOR, como tinham feito seus pais; não se apartou dos pecados de Jeroboão, filho de Nebate, que fez pecar a Israel.
\par 10 Salum, filho de Jabes, conspirou contra ele, feriu-o diante do povo, matou-o e reinou em seu lugar.
\par 11 Quanto aos mais atos de Zacarias, eis que estão escritos no Livro da História dos Reis de Israel.
\par 12 Esta foi a palavra que o SENHOR falou a Jeú: Teus filhos, até à quarta geração, se assentarão no trono de Israel. E assim sucedeu.
\par 13 Salum, filho de Jabes, começou a reinar no trigésimo nono ano de Uzias, rei de Judá; e reinou durante um mês em Samaria.
\par 14 Subindo de Tirza, Menaém, filho de Gadi, veio a Samaria, feriu ali a Salum, filho de Jabes, matou-o e reinou em seu lugar.
\par 15 Quanto aos mais atos de Salum e a conspiração que fez, eis que estão escritos no Livro da História dos Reis de Israel.
\par 16 Então, Menaém feriu a Tifsa e todos os que nela havia, como também seus limites desde Tirza. Porque não lha abriram, a devastou e todas as mulheres grávidas fez rasgar pelo ventre.
\par 17 Desde o trigésimo nono ano de Azarias, rei de Judá, Menaém, filho de Gadi, começou a reinar sobre Israel e reinou dez anos em Samaria.
\par 18 Fez o que era mau perante o SENHOR; todos os seus dias, não se apartou dos pecados de Jeroboão, filho de Nebate, que fez pecar a Israel.
\par 19 Então, veio Pul, rei da Assíria, contra a terra; Menaém deu a Pul mil talentos de prata, para que este o ajudasse a consolidar o seu reino.
\par 20 Menaém arrecadou este dinheiro de Israel para pagar ao rei da Assíria, de todos os poderosos e ricos, cinqüenta siclos de prata por cabeça; assim, voltou o rei da Assíria e não se demorou ali na terra.
\par 21 Quanto aos mais atos de Menaém e a tudo quanto fez, porventura, não estão escritos no Livro da História dos Reis de Israel?
\par 22 Descansou Menaém com seus pais; e Pecaías, seu filho, reinou em seu lugar.
\par 23 No qüinquagésimo ano de Azarias, rei de Judá, começou a reinar Pecaías, filho de Menaém; e reinou sobre Israel, em Samaria, dois anos.
\par 24 Fez o que era mau perante o SENHOR; não se apartou dos pecados de Jeroboão, filho de Nebate, que fez pecar a Israel.
\par 25 Peca, seu capitão, filho de Remalias, conspirou contra ele e o feriu em Samaria, na fortaleza da casa do rei, juntamente com Argobe e com Arié; com Peca estavam cinqüenta homens dos gileaditas; Peca o matou e reinou em seu lugar.
\par 26 Quanto aos mais atos de Pecaías e a tudo quanto fez, eis que estão escritos no Livro da História dos Reis de Israel.
\par 27 No qüinquagésimo segundo ano de Azarias, rei de Judá, começou a reinar Peca, filho de Remalias, e reinou sobre Israel, em Samaria, vinte anos.
\par 28 Fez o que era mau perante o SENHOR; não se apartou dos pecados de Jeroboão, filho de Nebate, que fez pecar a Israel.
\par 29 Nos dias de Peca, rei de Israel, veio Tiglate-Pileser, rei da Assíria, e tomou a Ijom, a Abel-Bete-Maaca, a Janoa, a Quedes, a Hazor, a Gileade e à Galiléia, a toda a terra de Naftali, e levou os seus habitantes para a Assíria.
\par 30 Oséias, filho de Elá, conspirou contra Peca, filho de Remalias, e o feriu, e o matou, e reinou em seu lugar, no vigésimo ano de Jotão, filho de Uzias.
\par 31 Quanto aos mais atos de Peca e a tudo quanto fez, eis que estão escritos no Livro da História dos Reis de Israel.
\par 32 No ano segundo de Peca, filho de Remalias, rei de Israel, começou a reinar Jotão, filho de Uzias, rei de Judá.
\par 33 Tinha vinte e cinco anos de idade quando começou a reinar e reinou dezesseis anos em Jerusalém. Era o nome de sua mãe Jerusa, filha de Zadoque.
\par 34 Fez o que era reto perante o SENHOR; e em tudo procedeu segundo fizera Uzias, seu pai.
\par 35 Tão-somente os altos não se tiraram; o povo ainda sacrificava e queimava incenso nos altos. Ele edificou a Porta de Cima da Casa do SENHOR.
\par 36 Quanto aos mais atos de Jotão e a tudo quanto fez, porventura, não estão escritos no Livro da História dos Reis de Judá?
\par 37 Naqueles dias, começou o SENHOR a enviar contra Judá a Rezim, rei da Síria, e a Peca, filho de Remalias.
\par 38 Descansou Jotão com seus pais e foi sepultado junto a seus pais, na Cidade de Davi, seu pai. Em seu lugar reinou Acaz, seu filho.

\chapter{16}

\par 1 No décimo sétimo ano de Peca, filho de Remalias, começou a reinar Acaz, filho de Jotão, rei de Judá.
\par 2 Tinha Acaz vinte anos de idade quando começou a reinar e reinou dezesseis anos em Jerusalém. Não fez o que era reto perante o SENHOR, seu Deus, como Davi, seu pai.
\par 3 Porque andou no caminho dos reis de Israel e até queimou a seu filho como sacrifício, segundo as abominações dos gentios, que o SENHOR lançara de diante dos filhos de Israel.
\par 4 Também sacrificou e queimou incenso nos altos e nos outeiros, como também debaixo de toda árvore frondosa.
\par 5 Então, subiu Rezim, rei da Síria, com Peca, filho de Remalias, rei de Israel, a Jerusalém para pelejarem contra ela; cercaram Acaz, porém não puderam prevalecer contra ele.
\par 6 Naquele tempo, Rezim, rei da Síria, restituiu Elate à Síria e lançou fora dela os judeus; os siros vieram a Elate e ficaram habitando ali até ao dia de hoje.
\par 7 Acaz enviou mensageiros a Tiglate-Pileser, rei da Assíria, dizendo: Eu sou teu servo e teu filho; sobe e livra-me do poder do rei da Síria e do poder do rei de Israel, que se levantam contra mim.
\par 8 Tomou Acaz a prata e o ouro que se acharam na Casa do SENHOR e nos tesouros da casa do rei e mandou de presente ao rei da Assíria.
\par 9 O rei da Assíria lhe deu ouvidos, subiu contra Damasco, tomou-a, levou o povo para Quir e matou a Rezim.
\par 10 Então, o rei Acaz foi a Damasco, a encontrar-se com Tiglate-Pileser, rei da Assíria; e, vendo ali um altar, enviou dele ao sacerdote Urias a planta e o modelo, segundo toda a sua obra.
\par 11 Urias, o sacerdote, edificou um altar segundo tudo o que o rei Acaz tinha ordenado de Damasco; assim o fez o sacerdote Urias, antes que o rei Acaz viesse de Damasco.
\par 12 Vindo, pois, de Damasco o rei, viu o altar, chegou-se a ele e nele sacrificou.
\par 13 Queimou o seu holocausto e a sua oferta de manjares, derramou a sua libação e aspergiu o sangue das suas ofertas pacíficas naquele altar.
\par 14 Porém o altar de bronze, que estava perante o SENHOR, tirou ele de diante da casa, de entre o seu altar e a Casa do SENHOR e o pôs ao lado do seu altar, do lado norte.
\par 15 Ordenou também o rei Acaz ao sacerdote Urias, dizendo: Queima, no grande altar, o holocausto da manhã, como também a oferta de manjares da tarde, e o holocausto do rei, e a sua oferta de manjares, e o holocausto de todo o povo da terra, e a sua oferta de manjares, e as suas libações; todo sangue dos holocaustos e todo sangue dos sacrifícios aspergirás nele; porém o altar de bronze ficará para a minha deliberação posterior.
\par 16 Fez Urias, o sacerdote, segundo tudo quanto o rei Acaz lhe ordenara.
\par 17 O rei Acaz cortou os painéis dos suportes, e de cima deles tomou a pia, e o mar, tirou-o de sobre os bois de bronze, que estavam debaixo dele, e o pôs sobre um pavimento de pedra.
\par 18 Também o passadiço coberto para uso no sábado, que edificaram na casa, e a entrada real pelo lado de fora retirou da Casa do SENHOR, por causa do rei da Assíria.
\par 19 Quanto aos mais atos de Acaz e ao que fez, porventura, não estão escritos no Livro da História dos Reis de Judá?
\par 20 Descansou Acaz com seus pais e foi sepultado junto a seus pais, na Cidade de Davi; e Ezequias, seu filho, reinou em seu lugar.

\chapter{17}

\par 1 No ano duodécimo de Acaz, rei de Judá, começou a reinar Oséias, filho de Elá; e reinou sobre Israel, em Samaria, nove anos.
\par 2 Fez o que era mau perante o SENHOR; contudo, não como os reis de Israel que foram antes dele.
\par 3 Contra ele subiu Salmaneser, rei da Assíria; Oséias ficou sendo servo dele e lhe pagava tributo.
\par 4 Porém o rei da Assíria achou Oséias em conspiração, porque enviara mensageiros a Sô, rei do Egito, e não pagava tributo ao rei da Assíria, como dantes fazia de ano em ano; por isso, o rei da Assíria o encerrou em grilhões, num cárcere.
\par 5 Porque o rei da Assíria passou por toda a terra, subiu a Samaria e a sitiou por três anos.
\par 6 No ano nono de Oséias, o rei da Assíria tomou a Samaria e transportou a Israel para a Assíria; e os fez habitar em Hala, junto a Habor e ao rio Gozã, e nas cidades dos medos.
\par 7 Tal sucedeu porque os filhos de Israel pecaram contra o SENHOR, seu Deus, que os fizera subir da terra do Egito, de debaixo da mão de Faraó, rei do Egito; e temeram a outros deuses.
\par 8 Andaram nos estatutos das nações que o SENHOR lançara de diante dos filhos de Israel e nos costumes estabelecidos pelos reis de Israel.
\par 9 Os filhos de Israel fizeram contra o SENHOR, seu Deus, o que não era reto; edificaram para si altos em todas as suas cidades, desde as atalaias dos vigias até à cidade fortificada.
\par 10 Levantaram para si colunas e postes-ídolos, em todos os altos outeiros e debaixo de todas as árvores frondosas.
\par 11 Queimaram ali incenso em todos os altos, como as nações que o SENHOR expulsara de diante deles; cometeram ações perversas para provocarem o SENHOR à ira
\par 12 e serviram os ídolos, dos quais o SENHOR lhes tinha dito: Não fareis estas coisas.
\par 13 O SENHOR advertiu a Israel e a Judá por intermédio de todos os profetas e de todos os videntes, dizendo: Voltai-vos dos vossos maus caminhos e guardai os meus mandamentos e os meus estatutos, segundo toda a Lei que prescrevi a vossos pais e que vos enviei por intermédio dos meus servos, os profetas.
\par 14 Porém não deram ouvidos; antes, se tornaram obstinados, de dura cerviz como seus pais, que não creram no SENHOR, seu Deus.
\par 15 Rejeitaram os estatutos e a aliança que fizera com seus pais, como também as suas advertências com que protestara contra eles; seguiram os ídolos, e se tornaram vãos, e seguiram as nações que estavam em derredor deles, das quais o SENHOR lhes havia ordenado que não as imitassem.
\par 16 Desprezaram todos os mandamentos do SENHOR, seu Deus, e fizeram para si imagens de fundição, dois bezerros; fizeram um poste-ídolo, e adoraram todo o exército do céu, e serviram a Baal.
\par 17 Também queimaram a seus filhos e a suas filhas como sacrifício, deram-se à prática de adivinhações e criam em agouros; e venderam-se para fazer o que era mau perante o SENHOR, para o provocarem à ira.
\par 18 Pelo que o SENHOR muito se indignou contra Israel e o afastou da sua presença; e nada mais ficou, senão a tribo de Judá.
\par 19 Também Judá não guardou os mandamentos do SENHOR, seu Deus; antes, andaram nos costumes que Israel introduziu.
\par 20 Pelo que o SENHOR rejeitou a toda a descendência de Israel, e os afligiu, e os entregou nas mãos dos despojadores, até que os expulsou da sua presença.
\par 21 Pois, quando ele rasgou a Israel da casa de Davi, e eles fizeram rei a Jeroboão, filho de Nebate, Jeroboão apartou a Israel de seguir o SENHOR e o fez cometer grande pecado.
\par 22 Assim, andaram os filhos de Israel em todos os pecados que Jeroboão tinha cometido; nunca se apartaram deles,
\par 23 até que o SENHOR afastou a Israel da sua presença, como falara pelo ministério de todos os seus servos, os profetas; assim, foi Israel transportado da sua terra para a Assíria, onde permanece até ao dia de hoje.
\par 24 O rei da Assíria trouxe gente de Babilônia, de Cuta, de Ava, de Hamate e de Sefarvaim e a fez habitar nas cidades de Samaria, em lugar dos filhos de Israel; tomaram posse de Samaria e habitaram nas suas cidades.
\par 25 A princípio, quando passaram a habitar ali, não temeram o SENHOR; então, mandou o SENHOR para o meio deles leões, os quais mataram a alguns do povo.
\par 26 Pelo que se disse ao rei da Assíria: As gentes que transportaste e fizeste habitar nas cidades de Samaria não sabem a maneira de servir o deus da terra; por isso, enviou ele leões para o meio delas, os quais as matam, porque não sabem como servir o deus da terra.
\par 27 Então, o rei da Assíria mandou dizer: Levai para lá um dos sacerdotes que de lá trouxestes; que ele vá, e lá habite, e lhes ensine a maneira de servir o deus da terra.
\par 28 Foi, pois, um dos sacerdotes que haviam levado de Samaria, e habitou em Betel, e lhes ensinava como deviam temer o SENHOR.
\par 29 Porém cada nação fez ainda os seus próprios deuses nas cidades em que habitava, e os puseram nos santuários dos altos que os samaritanos tinham feito.
\par 30 Os de Babilônia fizeram Sucote-Benote; os de Cuta fizeram Nergal; os de Hamate fizeram Asima;
\par 31 os aveus fizeram Nibaz e Tartaque; e os sefarvitas queimavam seus filhos a Adrameleque e a Anameleque, deuses de Sefarvaim.
\par 32 Mas temiam também ao SENHOR; dentre os do povo constituíram sacerdotes dos lugares altos, os quais oficiavam a favor deles nos santuários dos altos.
\par 33 De maneira que temiam o SENHOR e, ao mesmo tempo, serviam aos seus próprios deuses, segundo o costume das nações dentre as quais tinham sido transportados.
\par 34 Até ao dia de hoje fazem segundo os antigos costumes; não temem o SENHOR, não fazem segundo os seus estatutos e juízos, nem segundo a lei e o mandamento que o SENHOR prescreveu aos filhos de Jacó, a quem deu o nome de Israel.
\par 35 Ora, o SENHOR tinha feito aliança com eles e lhes ordenara, dizendo: Não temereis outros deuses, nem vos prostrareis diante deles, nem os servireis, nem lhes oferecereis sacrifícios;
\par 36 mas ao SENHOR, que vos fez subir da terra do Egito com grande poder e com braço estendido, a ele temereis, e a ele vos prostrareis, e a ele oferecereis sacrifícios.
\par 37 Os estatutos e os juízos, a lei e o mandamento que ele vos escreveu, tereis cuidado de os observar todos os dias; não temereis outros deuses.
\par 38 Da aliança que fiz convosco não vos esquecereis; nem temereis outros deuses.
\par 39 Mas ao SENHOR, vosso Deus, temereis, e ele vos livrará das mãos de todos os vossos inimigos.
\par 40 Porém eles não deram ouvidos a isso; antes, procederam segundo o seu antigo costume.
\par 41 Assim, estas nações temiam o SENHOR e serviam as suas próprias imagens de escultura; como fizeram seus pais, assim fazem também seus filhos e os filhos de seus filhos, até ao dia de hoje.

\chapter{18}

\par 1 No terceiro ano de Oséias, filho de Elá, rei de Israel, começou a reinar Ezequias, filho de Acaz, rei de Judá.
\par 2 Tinha vinte e cinco anos de idade quando começou a reinar e reinou vinte e nove anos em Jerusalém; sua mãe se chamava Abi e era filha de Zacarias.
\par 3 Fez ele o que era reto perante o SENHOR, segundo tudo o que fizera Davi, seu pai.
\par 4 Removeu os altos, quebrou as colunas e deitou abaixo o poste-ídolo; e fez em pedaços a serpente de bronze que Moisés fizera, porque até àquele dia os filhos de Israel lhe queimavam incenso e lhe chamavam Neustã.
\par 5 Confiou no SENHOR, Deus de Israel, de maneira que depois dele não houve seu semelhante entre todos os reis de Judá, nem entre os que foram antes dele.
\par 6 Porque se apegou ao SENHOR, não deixou de segui-lo e guardou os mandamentos que o SENHOR ordenara a Moisés.
\par 7 Assim, foi o SENHOR com ele; para onde quer que saía, lograva bom êxito; rebelou-se contra o rei da Assíria e não o serviu.
\par 8 Feriu ele os filisteus até Gaza e seus limites, desde as atalaias dos vigias até à cidade fortificada.
\par 9 No quarto ano do rei Ezequias, que era o sétimo de Oséias, filho de Elá, rei de Israel, subiu Salmaneser, rei da Assíria, contra Samaria e a cercou.
\par 10 Ao cabo de três anos, foi tomada; sim, no ano sexto de Ezequias, que era o nono de Oséias, rei de Israel, Samaria foi tomada.
\par 11 O rei da Assíria transportou a Israel para a Assíria e o fez habitar em Hala, junto a Habor e ao rio Gozã, e nas cidades dos medos;
\par 12 porquanto não obedeceram à voz do SENHOR, seu Deus; antes, violaram a sua aliança e tudo quanto Moisés, servo do SENHOR, tinha ordenado; não o ouviram, nem o fizeram.
\par 13 No ano décimo quarto do rei Ezequias, subiu Senaqueribe, rei da Assíria, contra todas as cidades fortificadas de Judá e as tomou.
\par 14 Então, Ezequias, rei de Judá, enviou mensageiros ao rei da Assíria, a Laquis, dizendo: Errei; retira-te de mim; tudo o que me impuseres suportarei. Então, o rei da Assíria impôs a Ezequias, rei de Judá, trezentos talentos de prata e trinta talentos de ouro.
\par 15 Deu-lhe Ezequias toda a prata que se achou na Casa do SENHOR e nos tesouros da casa do rei.
\par 16 Foi quando Ezequias arrancou das portas do templo do SENHOR e das ombreiras o ouro de que ele, rei de Judá, as cobrira, e o deu ao rei da Assíria.
\par 17 Contudo, o rei da Assíria enviou, de Laquis, a Tartã, a Rabe-Saris e a Rabsaqué, com um grande exército, ao rei Ezequias, a Jerusalém; subiram e vieram a Jerusalém. Tendo eles subido e chegado, pararam na extremidade do aqueduto do açude superior, junto ao caminho do campo do Lavandeiro.
\par 18 Tendo eles chamado o rei, saíram-lhes ao encontro Eliaquim, filho de Hilquias, o mordomo, Sebna, o escrivão, e Joá, filho de Asafe, o cronista.
\par 19 Rabsaqué lhes disse: Dizei a Ezequias: Assim diz o sumo rei, o rei da Assíria: Que confiança é essa em que te estribas?
\par 20 Bem posso dizer-te que teu conselho e poder para guerra não passam de vãs palavras; em quem, pois, agora, confias, para que te rebeles contra mim?
\par 21 Confias no Egito, esse bordão de cana esmagada, o qual, se alguém nele apoiar-se, lhe entrará pela mão e a traspassará; assim é Faraó, rei do Egito, para com todos os que nele confiam.
\par 22 Mas, se me dizeis: Confiamos no SENHOR, nosso Deus, não é esse aquele cujos altos e altares Ezequias removeu, dizendo a Judá e a Jerusalém: Perante este altar adorareis em Jerusalém?
\par 23 Ora, pois, empenha-te com meu senhor, rei da Assíria, e dar-te-ei dois mil cavalos, se de tua parte achares cavaleiros para os montar.
\par 24 Como, pois, se não podes afugentar um só capitão dos menores dos servos do meu senhor, confias no Egito, por causa dos carros e cavaleiros?
\par 25 Acaso, subi eu, agora, sem o SENHOR contra este lugar, para o destruir? Pois o SENHOR mesmo me disse: Sobe contra a terra e destrói-a.
\par 26 Então, disseram Eliaquim, filho de Hilquias, Sebna e Joá a Rabsaqué: Rogamos-te que fales em aramaico aos teus servos, porque o entendemos, e não nos fales em judaico, aos ouvidos do povo que está sobre as muralhas.
\par 27 Mas Rabsaqué lhes respondeu: Mandou-me, acaso, o meu senhor para dizer-te estas palavras a ti somente e a teu senhor? E não, antes, aos homens que estão sentados sobre as muralhas, para que comam convosco o seu próprio excremento e bebam a sua própria urina?
\par 28 Então, Rabsaqué se pôs em pé, e clamou em alta voz em judaico, e disse: Ouvi as palavras do sumo rei, do rei da Assíria.
\par 29 Assim diz o rei: Não vos engane Ezequias; porque não vos poderá livrar da sua mão;
\par 30 nem tampouco vos faça Ezequias confiar no SENHOR, dizendo: O SENHOR, certamente, nos livrará, e esta cidade não será entregue nas mãos do rei da Assíria.
\par 31 Não deis ouvidos a Ezequias; porque assim diz o rei da Assíria: Fazei as pazes comigo e vinde para mim; e comei, cada um da sua própria vide e da sua própria figueira, e bebei, cada um da água da sua própria cisterna.
\par 32 Até que eu venha e vos leve para uma terra como a vossa, terra de cereal e de vinho, terra de pão e de vinhas, terra de oliveiras e de mel, para que vivais e não morrais. Não deis ouvidos a Ezequias, porque vos engana, dizendo: O SENHOR nos livrará.
\par 33 Acaso, os deuses das nações puderam livrar, cada um a sua terra, das mãos do rei da Assíria?
\par 34 Onde estão os deuses de Hamate e de Arpade? Onde estão os deuses de Sefarvaim, Hena e Iva? Acaso, livraram eles a Samaria das minhas mãos?
\par 35 Quais são, dentre todos os deuses destes países, os que livraram a sua terra das minhas mãos, para que o SENHOR possa livrar a Jerusalém das minhas mãos?
\par 36 Calou-se, porém, o povo e não lhe respondeu palavra; porque assim lhe havia ordenado o rei: Não lhe respondereis.
\par 37 Então, Eliaquim, filho de Hilquias, o mordomo, e Sebna, o escrivão, e Joá, filho de Asafe, o cronista, vieram ter com Ezequias, com suas vestes rasgadas, e lhe referiram as palavras de Rabsaqué.

\chapter{19}

\par 1 Tendo o rei Ezequias ouvido isto, rasgou as suas vestes, cobriu-se de pano de saco e entrou na Casa do SENHOR.
\par 2 Então, enviou a Eliaquim, o mordomo, a Sebna, o escrivão, e os anciãos dos sacerdotes cobertos de pano de saco, ao profeta Isaías, filho de Amoz;
\par 3 os quais lhe disseram: Assim diz Ezequias: Este dia é dia de angústia, de disciplina e de opróbrio; porque filhos são chegados à hora de nascer, e não há força para dá-los à luz.
\par 4 Porventura, o SENHOR, teu Deus, terá ouvido todas as palavras de Rabsaqué, a quem o rei da Assíria, seu senhor, enviou para afrontar o Deus vivo, e repreenderá as palavras que ouviu; ergue, pois, orações pelos que ainda subsistem.
\par 5 Foram, pois, os servos do rei Ezequias a ter com Isaías;
\par 6 Isaías lhes disse: Dizei isto a vosso senhor: Assim diz o SENHOR: Não temas por causa das palavras que ouviste, com as quais os servos do rei da Assíria blasfemaram de mim.
\par 7 Eis que meterei nele um espírito, e ele, ao ouvir certo rumor, voltará para a sua terra; e nela eu o farei cair morto à espada.
\par 8 Voltou, pois, Rabsaqué e encontrou o rei da Assíria pelejando contra Libna; porque ouvira que o rei já se havia retirado de Laquis.
\par 9 O rei ouviu que a respeito de Tiraca, rei da Etiópia, se dizia: Eis saiu para guerrear contra ti. Assim, tornou a enviar mensageiros a Ezequias, dizendo:
\par 10 Assim falareis a Ezequias, rei de Judá: Não te engane o teu Deus, em quem confias, dizendo: Jerusalém não será entregue nas mãos do rei da Assíria.
\par 11 Já tens ouvido o que fizeram os reis da Assíria a todas as terras, como as destruíram totalmente; e crês tu que te livrarias?
\par 12 Porventura, os deuses das nações livraram os povos que meus pais destruíram, Gozã, Harã e Rezefe e os filhos de Éden, que estavam em Telassar?
\par 13 Onde está o rei de Hamate, e o rei de Arpade, e o rei da cidade de Sefarvaim, de Hena e de Iva?
\par 14 Tendo Ezequias recebido a carta das mãos dos mensageiros, leu-a; então, subiu à Casa do SENHOR, estendeu-a perante o SENHOR
\par 15 e orou perante o SENHOR, dizendo: Ó SENHOR, Deus de Israel, que estás entronizado acima dos querubins, tu somente és o Deus de todos os reinos da terra; tu fizeste os céus e a terra.
\par 16 Inclina, ó SENHOR, o ouvido e ouve; abre, SENHOR, os olhos e vê; ouve todas as palavras de Senaqueribe, as quais ele enviou para afrontar o Deus vivo.
\par 17 Verdade é, SENHOR, que os reis da Assíria assolaram todas as nações e suas terras
\par 18 e lançaram no fogo os deuses deles, porque deuses não eram, senão obra de mãos de homens, madeira e pedra; por isso, os destruíram.
\par 19 Agora, pois, ó SENHOR, nosso Deus, livra-nos das suas mãos, para que todos os reinos da terra saibam que só tu és o SENHOR Deus.
\par 20 Então, Isaías, filho de Amoz, mandou dizer a Ezequias: Assim diz o SENHOR, o Deus de Israel: Quanto ao que me pediste acerca de Senaqueribe, rei da Assíria, eu te ouvi,
\par 21 e esta é a palavra que o SENHOR falou a respeito dele: A virgem, filha de Sião, te despreza e zomba de ti; a filha de Jerusalém meneia a cabeça por detrás de ti.
\par 22 A quem afrontaste e de quem blasfemaste? E contra quem alçaste a voz e arrogantemente ergueste os olhos? Contra o Santo de Israel.
\par 23 Por meio dos teus mensageiros, afrontaste o SENHOR e disseste: Com a multidão dos meus carros subi ao cimo dos montes, ao mais interior do Líbano; deitarei abaixo os seus altos cedros e seus ciprestes escolhidos, chegarei a suas pousadas extremas, ao seu denso e fértil pomar.
\par 24 Eu mesmo cavei, e bebi as águas de estrangeiros, e com as plantas de meus pés sequei todos os rios do Egito.
\par 25 Acaso, não ouviste que já há muito dispus eu estas coisas, já desde os dias remotos o tinha planejado? Agora, porém, as faço executar e eu quis que tu reduzisses a montões de ruínas as cidades fortificadas.
\par 26 Por isso, os seus moradores, debilitados, andaram cheios de temor e envergonhados; tornaram-se como a erva do campo, e a erva verde, e o capim dos telhados, e o cereal queimado antes de amadurecer.
\par 27 Mas eu conheço o teu assentar, e o teu sair, e o teu entrar, e o teu furor contra mim.
\par 28 Por causa do teu furor contra mim e porque a tua arrogância subiu até aos meus ouvidos, eis que porei o meu anzol no teu nariz e o meu freio na tua boca e te farei voltar pelo caminho por onde vieste.
\par 29 Isto te será por sinal: este ano, se comerá o que espontaneamente nascer e, no segundo ano, o que daí proceder; no terceiro ano, porém, semeai, e colhei, e plantai vinhas, e comei os seus frutos.
\par 30 O que escapou da casa de Judá e ficou de resto tornará a lançar raízes para baixo e dará fruto por cima;
\par 31 porque de Jerusalém sairá o restante, e do monte Sião, o que escapou. O zelo do SENHOR fará isto.
\par 32 Pelo que assim diz o SENHOR acerca do rei da Assíria: Não entrará nesta cidade, nem lançará nela flecha alguma, não virá perante ela com escudo, nem há de levantar tranqueiras contra ela.
\par 33 Pelo caminho por onde vier, por esse voltará; mas, nesta cidade, não entrará, diz o SENHOR.
\par 34 Porque eu defenderei esta cidade, para a livrar, por amor de mim e por amor de meu servo Davi.
\par 35 Então, naquela mesma noite, saiu o Anjo do SENHOR e feriu, no arraial dos assírios, cento e oitenta e cinco mil; e, quando se levantaram os restantes pela manhã, eis que todos estes eram cadáveres.
\par 36 Retirou-se, pois, Senaqueribe, rei da Assíria, e se foi; voltou e ficou em Nínive.
\par 37 Sucedeu que, estando ele a adorar na casa de Nisroque, seu deus, Adrameleque e Sarezer, seus filhos, o feriram à espada; e fugiram para a terra de Ararate; e Esar-Hadom, seu filho, reinou em seu lugar.

\chapter{20}

\par 1 Naqueles dias, Ezequias adoeceu de uma enfermidade mortal; veio ter com ele o profeta Isaías, filho de Amoz, e lhe disse: Assim diz o SENHOR: Põe em ordem a tua casa, porque morrerás e não viverás.
\par 2 Então, virou Ezequias o rosto para a parede e orou ao SENHOR, dizendo:
\par 3 Lembra-te, SENHOR, peço-te, de que andei diante de ti com fidelidade, com inteireza de coração, e fiz o que era reto aos teus olhos; e chorou muitíssimo.
\par 4 Antes que Isaías tivesse saído da parte central da cidade, veio a ele a palavra do SENHOR, dizendo:
\par 5 Volta e dize a Ezequias, príncipe do meu povo: Assim diz o SENHOR, o Deus de Davi, teu pai: Ouvi a tua oração e vi as tuas lágrimas; eis que eu te curarei; ao terceiro dia, subirás à Casa do SENHOR.
\par 6 Acrescentarei aos teus dias quinze anos e das mãos do rei da Assíria te livrarei, a ti e a esta cidade; e defenderei esta cidade por amor de mim e por amor de Davi, meu servo.
\par 7 Disse mais Isaías: Tomai uma pasta de figos; tomaram-na e a puseram sobre a úlcera; e ele recuperou a saúde.
\par 8 Ezequias disse a Isaías: Qual será o sinal de que o SENHOR me curará e de que, ao terceiro dia, subirei à Casa do SENHOR?
\par 9 Respondeu Isaías: Ser-te-á isto da parte do SENHOR como sinal de que ele cumprirá a palavra que disse: Adiantar-se-á a sombra dez graus ou os retrocederá?
\par 10 Então, disse Ezequias: É fácil que a sombra adiante dez graus; tal, porém, não aconteça; antes, retroceda dez graus.
\par 11 Então, o profeta Isaías clamou ao SENHOR; e fez retroceder dez graus a sombra lançada pelo sol declinante no relógio de Acaz.
\par 12 Nesse tempo, Merodaque-Baladã, filho de Baladã, rei da Babilônia, enviou cartas e um presente a Ezequias, porque soube que estivera doente.
\par 13 Ezequias se agradou dos mensageiros e lhes mostrou toda a casa do seu tesouro, a prata, o ouro, as especiarias, os óleos finos, o seu arsenal e tudo quanto se achava nos seus tesouros; nenhuma coisa houve, nem em sua casa, nem em todo o seu domínio que Ezequias não lhes mostrasse.
\par 14 Então, Isaías, o profeta, veio ao rei Ezequias e lhe disse: Que foi que aqueles homens disseram e donde vieram a ti? Respondeu Ezequias: De uma terra longínqua vieram, da Babilônia.
\par 15 Perguntou ele: Que viram em tua casa? Respondeu Ezequias: Viram tudo quanto há em minha casa; coisa nenhuma há nos meus tesouros que eu não lhes mostrasse.
\par 16 Então, disse Isaías a Ezequias: Ouve a palavra do SENHOR:
\par 17 Eis que virão dias em que tudo quanto houver em tua casa, com o que entesouraram teus pais até ao dia de hoje, será levado para a Babilônia; não ficará coisa alguma, disse o SENHOR.
\par 18 Dos teus próprios filhos, que tu gerares, tomarão, para que sejam eunucos no palácio do rei da Babilônia.
\par 19 Então, disse Ezequias a Isaías: Boa é a palavra do SENHOR que disseste. Pois pensava: Haverá paz e segurança em meus dias.
\par 20 Quanto aos mais atos de Ezequias, e todo o seu poder, e como fez o açude e o aqueduto, e trouxe água para dentro da cidade, porventura, não estão escritos no Livro da História dos Reis de Judá?
\par 21 Descansou Ezequias com seus pais; e Manassés, seu filho, reinou em seu lugar.

\chapter{21}

\par 1 Tinha Manassés doze anos de idade quando começou a reinar e reinou cinqüenta e cinco anos em Jerusalém. Sua mãe chamava-se Hefzibá.
\par 2 Fez ele o que era mau perante o SENHOR, segundo as abominações dos gentios que o SENHOR expulsara de suas possessões, de diante dos filhos de Israel.
\par 3 Pois tornou a edificar os altos que Ezequias, seu pai, havia destruído, e levantou altares a Baal, e fez um poste-ídolo como o que fizera Acabe, rei de Israel, e se prostrou diante de todo o exército dos céus, e o serviu.
\par 4 Edificou altares na Casa do SENHOR, da qual o SENHOR tinha dito: Em Jerusalém porei o meu nome.
\par 5 Também edificou altares a todo o exército dos céus nos dois átrios da Casa do SENHOR.
\par 6 E queimou a seu filho como sacrifício, adivinhava pelas nuvens, era agoureiro e tratava com médiuns e feiticeiros; prosseguiu em fazer o que era mau perante o SENHOR, para o provocar à ira.
\par 7 Também pôs a imagem de escultura do poste-ídolo que tinha feito na casa de que o SENHOR dissera a Davi e a Salomão, seu filho: Nesta casa e em Jerusalém, que escolhi de todas as tribos de Israel, porei o meu nome para sempre;
\par 8 e não farei que os pés de Israel andem errantes da terra que dei a seus pais, contanto que tenham cuidado de fazer segundo tudo o que lhes tenho mandado e conforme toda a lei que Moisés, meu servo, lhes ordenou.
\par 9 Eles, porém, não ouviram; e Manassés de tal modo os fez errar, que fizeram pior do que as nações que o SENHOR tinha destruído de diante dos filhos de Israel.
\par 10 Então, o SENHOR falou por intermédio dos profetas, seus servos, dizendo:
\par 11 Visto que Manassés, rei de Judá, cometeu estas abominações, fazendo pior que tudo que fizeram os amorreus antes dele, e também a Judá fez pecar com os ídolos dele,
\par 12 assim diz o SENHOR, Deus de Israel: Eis que hei de trazer tais males sobre Jerusalém e Judá, que todo o que os ouvir, lhe tinirão ambos os ouvidos.
\par 13 Estenderei sobre Jerusalém o cordel de Samaria e o prumo da casa de Acabe; eliminarei Jerusalém, como quem elimina a sujeira de um prato, elimina-a e o emborca.
\par 14 Abandonarei o resto da minha herança, entregá-lo-ei nas mãos de seus inimigos; servirá de presa e despojo para todos os seus inimigos.
\par 15 Porquanto fizeram o que era mau perante mim e me provocaram à ira, desde o dia em que seus pais saíram do Egito até ao dia de hoje.
\par 16 Além disso, Manassés derramou muitíssimo sangue inocente, até encher Jerusalém de um ao outro extremo, afora o seu pecado, com que fez pecar a Judá, praticando o que era mau perante o SENHOR.
\par 17 Quanto aos mais atos de Manassés, e a tudo quanto fez, e ao seu pecado, que cometeu, porventura, não estão escritos no Livro da História dos Reis de Judá?
\par 18 Manassés descansou com seus pais e foi sepultado no jardim da sua própria casa, no jardim de Uzá; e Amom, seu filho, reinou em seu lugar.
\par 19 Tinha Amom vinte e dois anos de idade quando começou a reinar e reinou dois anos em Jerusalém. Sua mãe se chamava Mesulemete e era filha de Haruz, de Jotbá.
\par 20 Fez o que era mau perante o SENHOR, como fizera Manassés, seu pai.
\par 21 Andou em todo o caminho em que andara seu pai, serviu os ídolos a que ele servira e os adorou.
\par 22 Assim, abandonou ele o SENHOR, Deus de seus pais, e não andou no caminho do SENHOR.
\par 23 Os servos do rei Amom conspiraram contra ele e o mataram em sua própria casa.
\par 24 Porém o povo da terra feriu todos os que conspiraram contra o rei Amom e constituiu a Josias, seu filho, rei em seu lugar.
\par 25 Quanto aos mais atos de Amom e a tudo que fez, porventura, não estão escritos no Livro da História dos Reis de Judá?
\par 26 Foi ele enterrado na sua sepultura, no jardim de Uzá; e Josias, seu filho, reinou em seu lugar.

\chapter{22}

\par 1 Tinha Josias oito anos de idade quando começou a reinar e reinou trinta e um anos em Jerusalém. Sua mãe se chamava Jedida e era filha de Adaías, de Bozcate.
\par 2 Fez ele o que era reto perante o SENHOR, andou em todo o caminho de Davi, seu pai, e não se desviou nem para a direita nem para a esquerda.
\par 3 No décimo oitavo ano do seu reinado, o rei Josias mandou o escrivão Safã, filho de Azalias, filho de Mesulão, à Casa do SENHOR,
\par 4 dizendo: Sobe a Hilquias, o sumo sacerdote, para que conte o dinheiro que se trouxe à Casa do SENHOR, o qual os guardas da porta ajuntaram do povo;
\par 5 que o dêem nas mãos dos que dirigem a obra e têm a seu cargo a Casa do SENHOR, para que paguem àqueles que fazem a obra que há na Casa do SENHOR, para repararem os estragos da casa:
\par 6 aos carpinteiros, aos edificadores e aos pedreiros; e comprem madeira e pedras lavradas, para repararem os estragos da casa.
\par 7 Porém não se pediu conta do dinheiro que se lhes entregara nas mãos, porquanto procediam com fidelidade.
\par 8 Então, disse o sumo sacerdote Hilquias ao escrivão Safã: Achei o Livro da Lei na Casa do SENHOR. Hilquias entregou o livro a Safã, e este o leu.
\par 9 Então, o escrivão Safã veio ter com o rei e lhe deu relatório, dizendo: Teus servos contaram o dinheiro que se achou na casa e o entregaram nas mãos dos que dirigem a obra e têm a seu cargo a Casa do SENHOR.
\par 10 Relatou mais o escrivão Safã ao rei, dizendo: O sacerdote Hilquias me entregou um livro. E Safã o leu diante do rei.
\par 11 Tendo o rei ouvido as palavras do Livro da Lei, rasgou as suas vestes.
\par 12 Ordenou o rei a Hilquias, o sacerdote, a Aicão, filho de Safã, a Acbor, filho de Micaías, a Safã, o escrivão, e a Asaías, servo do rei, dizendo:
\par 13 Ide e consultai o SENHOR por mim, pelo povo e por todo o Judá, acerca das palavras deste livro que se achou; porque grande é o furor do SENHOR que se acendeu contra nós, porquanto nossos pais não deram ouvidos às palavras deste livro, para fazerem segundo tudo quanto de nós está escrito.
\par 14 Então, o sacerdote Hilquias, Aicão, Acbor, Safã e Asaías foram ter com a profetisa Hulda, mulher de Salum, o guarda-roupa, filho de Ticva, filho de Harás, e lhe falaram. Ela habitava na cidade baixa de Jerusalém.
\par 15 Ela lhes disse: Assim diz o SENHOR, o Deus de Israel: Dizei ao homem que vos enviou a mim:
\par 16 Assim diz o SENHOR: Eis que trarei males sobre este lugar e sobre os seus moradores, a saber, todas as palavras do livro que leu o rei de Judá.
\par 17 Visto que me deixaram e queimaram incenso a outros deuses, para me provocarem à ira com todas as obras das suas mãos, o meu furor se acendeu contra este lugar e não se apagará.
\par 18 Porém ao rei de Judá, que vos enviou a consultar o SENHOR, assim lhe direis: Assim diz o SENHOR, o Deus de Israel, acerca das palavras que ouviste:
\par 19 Porquanto o teu coração se enterneceu, e te humilhaste perante o SENHOR, quando ouviste o que falei contra este lugar e contra os seus moradores, que seriam para assolação e para maldição, e rasgaste as tuas vestes, e choraste perante mim, também eu te ouvi, diz o SENHOR.
\par 20 Pelo que, eis que eu te reunirei a teus pais, e tu serás recolhido em paz à tua sepultura, e os teus olhos não verão todo o mal que hei de trazer sobre este lugar. Então, levaram eles ao rei esta resposta.

\chapter{23}

\par 1 Então, deu ordem o rei, e todos os anciãos de Judá e de Jerusalém se ajuntaram a ele.
\par 2 O rei subiu à Casa do SENHOR, e com ele todos os homens de Judá, todos os moradores de Jerusalém, os sacerdotes, os profetas e todo o povo, desde o menor até ao maior; e leu diante deles todas as palavras do Livro da Aliança que fora encontrado na Casa do SENHOR.
\par 3 O rei se pôs em pé junto à coluna e fez aliança ante o SENHOR, para o seguirem, guardarem os seus mandamentos, os seus testemunhos e os seus estatutos, de todo o coração e de toda a alma, cumprindo as palavras desta aliança, que estavam escritas naquele livro; e todo o povo anuiu a esta aliança.
\par 4 Então, o rei ordenou ao sumo sacerdote Hilquias, e aos sacerdotes da segunda ordem, e aos guardas da porta que tirassem do templo do SENHOR todos os utensílios que se tinham feito para Baal, e para o poste-ídolo, e para todo o exército dos céus, e os queimou fora de Jerusalém, nos campos de Cedrom, e levou as cinzas deles para Betel.
\par 5 Também destituiu os sacerdotes que os reis de Judá estabeleceram para incensarem sobre os altos nas cidades de Judá e ao redor de Jerusalém, como também os que incensavam a Baal, ao sol, e à lua, e aos mais planetas, e a todo o exército dos céus.
\par 6 Também tirou da Casa do SENHOR o poste-ídolo, que levou para fora de Jerusalém até ao vale de Cedrom, no qual o queimou e o reduziu a pó, que lançou sobre as sepulturas do povo.
\par 7 Também derribou as casas da prostituição-cultual que estavam na Casa do SENHOR, onde as mulheres teciam tendas para o poste-ídolo.
\par 8 A todos os sacerdotes trouxe das cidades de Judá e profanou os altos em que os sacerdotes incensavam, desde Geba até Berseba; e derribou os altares das portas, que estavam à entrada da porta de Josué, governador da cidade, à mão esquerda daquele que entrava por ela.
\par 9 (Mas os sacerdotes dos altos não sacrificavam sobre o altar do SENHOR, em Jerusalém; porém comiam pães asmos no meio de seus irmãos.)
\par 10 Também profanou a Tofete, que está no vale dos filhos de Hinom, para que ninguém queimasse a seu filho ou a sua filha como sacrifício a Moloque.
\par 11 Também tirou os cavalos que os reis de Judá tinham dedicado ao sol, à entrada da Casa do SENHOR, perto da câmara de Natã-Meleque, o camareiro, a qual ficava no átrio; e os carros do sol queimou.
\par 12 Também o rei derribou os altares que estavam sobre a sala de Acaz, sobre o terraço, altares que foram feitos pelos reis de Judá, como também os altares que fizera Manassés nos dois átrios da Casa do SENHOR; e, esmigalhados, os tirou dali e lançou o pó deles no ribeiro de Cedrom.
\par 13 O rei profanou também os altos que estavam defronte de Jerusalém, à mão direita do monte da Destruição, os quais edificara Salomão, rei de Israel, para Astarote, abominação dos sidônios, e para Quemos, abominação dos moabitas, e para Milcom, abominação dos filhos de Amom.
\par 14 Semelhantemente, fez em pedaços as colunas e cortou os postes-ídolos; e o lugar onde estavam encheu ele de ossos humanos.
\par 15 Também o altar que estava em Betel e o alto que fez Jeroboão, filho de Nebate, que tinha feito pecar a Israel, esse altar junto com o alto o rei derribou; destruiu o alto, reduziu a pó o seu altar e queimou o poste-ídolo.
\par 16 Olhando Josias ao seu redor, viu as sepulturas que estavam ali no monte; mandou tirar delas os ossos, e os queimou sobre o altar, e assim o profanou, segundo a palavra do SENHOR, que apregoara o homem de Deus que havia anunciado estas coisas.
\par 17 Então, perguntou: Que monumento é este que vejo? Responderam-lhe os homens da cidade: É a sepultura do homem de Deus que veio de Judá e apregoou estas coisas que fizeste contra o altar de Betel.
\par 18 Josias disse: Deixai-o estar; ninguém mexa nos seus ossos. Assim, deixaram estar os seus ossos com os ossos do profeta que viera de Samaria.
\par 19 Também tirou Josias todos os santuários dos altos que havia nas cidades de Samaria e que os reis de Israel tinham feito para provocarem o SENHOR à ira; e lhes fez segundo todos os atos que tinha praticado em Betel.
\par 20 E matou todos os sacerdotes dos altos que havia ali, sobre os altares, e queimou ossos humanos sobre eles; depois, voltou para Jerusalém.
\par 21 Deu ordem o rei a todo o povo, dizendo: Celebrai a Páscoa ao SENHOR, vosso Deus, como está escrito neste Livro da Aliança.
\par 22 Porque nunca se celebrou tal Páscoa como esta desde os dias dos juízes que julgaram Israel, nem durante os dias dos reis de Israel, nem nos dias dos reis de Judá.
\par 23 Corria o ano décimo oitavo do rei Josias, quando esta Páscoa se celebrou ao SENHOR, em Jerusalém.
\par 24 Aboliu também Josias os médiuns, os feiticeiros, os ídolos do lar, os ídolos e todas as abominações que se viam na terra de Judá e em Jerusalém, para cumprir as palavras da lei, que estavam escritas no livro que o sacerdote Hilquias achara na Casa do SENHOR.
\par 25 Antes dele, não houve rei que lhe fosse semelhante, que se convertesse ao SENHOR de todo o seu coração, e de toda a sua alma, e de todas as suas forças, segundo toda a Lei de Moisés; e, depois dele, nunca se levantou outro igual.
\par 26 Nada obstante, o SENHOR não desistiu do furor da sua grande ira, ira com que ardia contra Judá, por todas as provocações com que Manassés o tinha irritado.
\par 27 Disse o SENHOR: Também a Judá removerei de diante de mim, como removi Israel, e rejeitarei esta cidade de Jerusalém, que escolhi, e a casa da qual eu dissera: Estará ali o meu nome.
\par 28 Quanto aos mais atos de Josias e a tudo quanto fez, porventura, não estão escritos no Livro da História dos Reis de Judá?
\par 29 Nos dias de Josias, subiu Faraó-Neco, rei do Egito, contra o rei da Assíria, ao rio Eufrates; e, tendo saído contra ele o rei Josias, Neco o matou, em Megido, no primeiro encontro.
\par 30 De Megido, os seus servos o levaram morto e, num carro, o transportaram para Jerusalém, onde o sepultaram no seu jazigo. O povo da terra tomou a Jeoacaz, filho de Josias, e o ungiu, e o fez rei em lugar de seu pai.
\par 31 Tinha Jeoacaz vinte e três anos de idade quando começou a reinar e reinou três meses em Jerusalém. Sua mãe chamava-se Hamutal e era filha de Jeremias, de Libna.
\par 32 Fez ele o que era mau perante o SENHOR, segundo tudo que fizeram seus pais.
\par 33 Porém Faraó-Neco o mandou prender em Ribla, na terra de Hamate, para que não reinasse em Jerusalém; e impôs à terra a pena de cem talentos de prata e um de ouro.
\par 34 Faraó-Neco também constituiu rei a Eliaquim, filho de Josias, em lugar de Josias, seu pai, e lhe mudou o nome para Jeoaquim; porém levou consigo para o Egito a Jeoacaz, que ali morreu.
\par 35 Jeoaquim deu aquela prata e aquele ouro a Faraó; porém estabeleceu imposto sobre a terra, para dar esse dinheiro segundo o mandado de Faraó; do povo da terra exigiu prata e ouro, de cada um segundo a sua avaliação, para o dar a Faraó-Neco.
\par 36 Tinha Jeoaquim a idade de vinte e cinco anos quando começou a reinar e reinou onze anos em Jerusalém. Sua mãe se chamava Zebida e era filha de Pedaías, de Ruma.
\par 37 Fez ele o que era mau perante o SENHOR, segundo tudo quanto fizeram seus pais.

\chapter{24}

\par 1 Nos dias de Jeoaquim, subiu Nabucodonosor, rei da Babilônia, contra ele, e ele, por três anos, ficou seu servo; então, se rebelou contra ele.
\par 2 Enviou o SENHOR contra Jeoaquim bandos de caldeus, e bandos de siros, e de moabitas, e dos filhos de Amom; enviou-os contra Judá para o destruir, segundo a palavra que o SENHOR falara pelos profetas, seus servos.
\par 3 Com efeito, isto sucedeu a Judá por mandado do SENHOR, que o removeu da sua presença, por causa de todos os pecados cometidos por Manassés,
\par 4 como também por causa do sangue inocente que ele derramou, com o qual encheu a cidade de Jerusalém; por isso, o SENHOR não o quis perdoar.
\par 5 Quanto aos mais atos de Jeoaquim e a tudo quanto fez, porventura, não estão escritos no Livro da História dos Reis de Judá?
\par 6 Descansou Jeoaquim com seus pais; e Joaquim, seu filho, reinou em seu lugar.
\par 7 O rei do Egito nunca mais saiu da sua terra; porque o rei da Babilônia tomou tudo quanto era dele, desde o ribeiro do Egito até ao rio Eufrates.
\par 8 Tinha Joaquim dezoito anos de idade quando começou a reinar e reinou três meses em Jerusalém. Sua mãe se chamava Neústa e era filha de Elnatã, de Jerusalém.
\par 9 Fez ele o que era mau perante o SENHOR, conforme tudo quanto fizera seu pai.
\par 10 Naquele tempo, subiram os servos de Nabucodonosor, rei da Babilônia, a Jerusalém, e a cidade foi cercada.
\par 11 Nabucodonosor, rei da Babilônia, veio à cidade, quando os seus servos a sitiavam.
\par 12 Então, subiu Joaquim, rei de Judá, a encontrar-se com o rei da Babilônia, ele, sua mãe, seus servos, seus príncipes e seus oficiais; e o rei da Babilônia, no oitavo ano do seu reinado, o levou cativo.
\par 13 Levou dali todos os tesouros da Casa do SENHOR e os tesouros da casa do rei; e, segundo tinha dito o SENHOR, cortou em pedaços todos os utensílios de ouro que fizera Salomão, rei de Israel, para o templo do SENHOR.
\par 14 Transportou a toda a Jerusalém, todos os príncipes, todos os homens valentes, todos os artífices e ferreiros, ao todo dez mil; ninguém ficou, senão o povo pobre da terra.
\par 15 Transferiu também a Joaquim para a Babilônia; a mãe do rei, as mulheres deste, seus oficiais e os homens principais da terra, ele os levou cativos de Jerusalém à Babilônia.
\par 16 Todos os homens valentes, até sete mil, e os artífices, e ferreiros, até mil, todos eles destros na guerra, levou-os o rei da Babilônia cativos para a Babilônia.
\par 17 O rei da Babilônia estabeleceu rei, em lugar de Joaquim, ao tio paterno deste, Matanias, de quem mudou o nome para Zedequias.
\par 18 Tinha Zedequias a idade de vinte e um anos quando começou a reinar e reinou onze anos em Jerusalém. Sua mãe se chamava Hamutal e era filha de Jeremias, de Libna.
\par 19 Fez ele o que era mau perante o SENHOR, conforme tudo quanto fizera Joaquim.
\par 20 Assim sucedeu por causa da ira do SENHOR contra Jerusalém e contra Judá, a ponto de os rejeitar de sua presença. Zedequias rebelou-se contra o rei da Babilônia.

\chapter{25}

\par 1 Sucedeu que, no nono ano do reinado de Zedequias, aos dez dias do décimo mês, Nabucodonosor, rei da Babilônia, veio contra Jerusalém, ele e todo o seu exército, e se acamparam contra ela, e levantaram contra ela tranqueiras em redor.
\par 2 A cidade ficou sitiada até ao undécimo ano do rei Zedequias.
\par 3 Aos nove dias do quarto mês, quando a cidade se via apertada da fome, e não havia pão para o povo da terra,
\par 4 então, a cidade foi arrombada, e todos os homens de guerra fugiram de noite pelo caminho da porta que está entre os dois muros perto do jardim do rei, a despeito de os caldeus se acharem contra a cidade em redor; o rei fugiu pelo caminho da Campina,
\par 5 porém o exército dos caldeus perseguiu o rei Zedequias e o alcançou nas campinas de Jericó; e todo o exército deste se dispersou e o abandonou.
\par 6 Então, o tomaram preso e o fizeram subir ao rei da Babilônia, a Ribla, o qual lhe pronunciou a sentença.
\par 7 Aos filhos de Zedequias mataram à sua própria vista e a ele vazaram os olhos; ataram-no com duas cadeias de bronze e o levaram para a Babilônia.
\par 8 No sétimo dia do quinto mês, do ano décimo nono de Nabucodonosor, rei da Babilônia, Nebuzaradã, chefe da guarda e servidor do rei da Babilônia, veio a Jerusalém.
\par 9 E queimou a Casa do SENHOR e a casa do rei, como também todas as casas de Jerusalém; também entregou às chamas todos os edifícios importantes.
\par 10 Todo o exército dos caldeus que estava com o chefe da guarda derribou os muros em redor de Jerusalém.
\par 11 O mais do povo que havia ficado na cidade, e os desertores que se entregaram ao rei da Babilônia, e o mais da multidão, Nebuzaradã, o chefe da guarda, levou cativos.
\par 12 Porém dos mais pobres da terra deixou o chefe da guarda ficar alguns para vinheiros e para lavradores.
\par 13 Cortaram em pedaços os caldeus as colunas de bronze que estavam na Casa do SENHOR, como também os suportes e o mar de bronze que estavam na Casa do SENHOR; e levaram o bronze para a Babilônia.
\par 14 Levaram também as panelas, as pás, as espevitadeiras, os recipientes de incenso e todos os utensílios de bronze, com que se ministrava.
\par 15 Tomou também o chefe da guarda os braseiros, as bacias e tudo quanto fosse de ouro ou de prata.
\par 16 Quanto às duas colunas, ao mar e aos suportes que Salomão fizera para a Casa do SENHOR, o peso do bronze de todos estes utensílios era incalculável.
\par 17 A altura de uma coluna era de dezoito côvados, e sobre ela havia um capitel de bronze de três côvados de altura; a obra de rede e as romãs sobre o capitel ao redor, tudo era de bronze; semelhante a esta era a outra coluna com a rede.
\par 18 Levou também o chefe da guarda a Seraías, sumo sacerdote, e a Sofonias, segundo sacerdote, e os três guardas da porta.
\par 19 Da cidade tomou a um oficial, que era comandante das tropas de guerra, e cinco homens dos que eram conselheiros do rei e se achavam na cidade, como também ao escrivão-mor do exército, que alistava o povo da terra, e sessenta homens do povo do lugar, que se achavam na cidade.
\par 20 Tomando-os, Nebuzaradã, o chefe da guarda, levou-os ao rei da Babilônia, a Ribla.
\par 21 O rei da Babilônia os feriu e os matou em Ribla, na terra de Hamate. Assim, Judá foi levado cativo para fora da sua terra.
\par 22 Quanto ao povo que ficara na terra de Judá, Nabucodonosor, rei da Babilônia, que o deixara ficar, nomeou governador sobre ele a Gedalias, filho de Aicão, filho de Safã.
\par 23 Ouvindo, pois, os capitães dos exércitos, eles e os seus homens, que o rei da Babilônia nomeara governador a Gedalias, vieram ter com este em Mispa, a saber, Ismael, filho de Netanias, Joanã, filho de Careá, Seraías, filho de Tanumete, o netofatita, e Jazanias, filho do maacatita, eles e os seus homens.
\par 24 Gedalias jurou a eles e aos seus homens e lhes disse: Nada temais da parte dos caldeus; ficai na terra, servi ao rei da Babilônia, e bem vos irá.
\par 25 Sucedeu, porém, que, no sétimo mês, veio Ismael, filho de Netanias, filho de Elisama, de família real, e dez homens, com ele, e feriram Gedalias, e ele morreu, como também aos judeus e aos caldeus que estavam com ele em Mispa.
\par 26 Então, se levantou todo o povo, tanto os pequenos como os grandes, como também os capitães das tropas, e foram para o Egito, porque temiam aos caldeus.
\par 27 No trigésimo sétimo ano do cativeiro de Joaquim, rei de Judá, no dia vinte e sete do duodécimo mês, Evil-Merodaque, rei da Babilônia, no ano em que começou a reinar, libertou do cárcere a Joaquim, rei de Judá.
\par 28 Falou com ele benignamente e lhe deu lugar de mais honra do que a dos reis que estavam com ele na Babilônia.
\par 29 Mudou-lhe as vestes do cárcere, e Joaquim passou a comer pão na sua presença todos os dias da sua vida.
\par 30 E da parte do rei lhe foi dada subsistência vitalícia, uma pensão diária, durante os dias da sua vida.


\end{document}