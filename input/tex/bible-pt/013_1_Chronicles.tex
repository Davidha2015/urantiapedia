\begin{document}

\title{I Crónicas}


\chapter{1}

\par 1 Adão, Sete, Enos,
\par 2 Cainã, Maalalel, Jarede,
\par 3 Enoque, Metusalém, Lameque,
\par 4 Noé, Sem, Cam e Jafé.
\par 5 Os filhos de Jafé foram: Gomer, Magogue, Madai, Javã, Tubal, Meseque e Tiras.
\par 6 Os filhos de Gomer: Asquenaz, Rifate e Togarma.
\par 7 Os filhos de Javã: Elisá, Társis, Quitim e Rodanim.
\par 8 Os filhos de Cam: Cuxe, Mizraim, Pute e Canaã.
\par 9 Os filhos de Cuxe: Sebá, Havilá, Sabtá, Raamá e Sabtecá; os filhos de Raamá: Sabá e Dedã.
\par 10 Cuxe gerou a Ninrode, que começou a ser poderoso na terra.
\par 11 Mizraim gerou a Ludim, a Anamim, a Leabim, a Naftuim,
\par 12 a Patrusim, a Casluim (de quem descendem os filisteus) e a Caftorim.
\par 13 Canaã gerou a Sidom, seu primogênito, a Hete,
\par 14 aos jebuseus, aos amorreus, aos girgaseus,
\par 15 aos heveus, aos arqueus, aos sineus,
\par 16 aos arvadeus, aos zemareus e aos hamateus.
\par 17 Os filhos de Sem: Elão, Assur, Arfaxade, Lude, Arã, Uz, Hul, Geter e Meseque.
\par 18 Arfaxade gerou a Selá, e Selá gerou a Héber.
\par 19 A Héber nasceram dois filhos: o nome de um foi Pelegue, porquanto, nos seus dias, se repartiu a terra; e o nome de seu irmão era Joctã.
\par 20 Joctã gerou a Almodá, a Salefe, a Hazar-Mavé, a Jerá,
\par 21 a Hadorão, a Uzal, a Dicla,
\par 22 a Ebal, a Abimael, a Sabá,
\par 23 a Ofir, a Havilá e a Jobabe; todos estes eram filhos de Joctã.
\par 24 Sem, Arfaxade, Selá,
\par 25 Héber, Pelegue, Reú,
\par 26 Serugue, Naor, Tera
\par 27 e Abrão, que é Abraão.
\par 28 Os filhos de Abraão: Isaque e Ismael.
\par 29 São estas as suas gerações: o primogênito de Ismael foi Nebaiote, depois Quedar, Adbeel, Mibsão,
\par 30 Misma, Dumá, Massá, Hadade, Temá,
\par 31 Jetur, Nafis e Quedemá; estes foram os filhos de Ismael.
\par 32 Quanto aos filhos de Quetura, concubina de Abraão, esta deu à luz a Zinrã, a Jocsã, a Medã, a Midiã, a Isbaque e a Sua. Os filhos de Jocsã: Sabá e Dedã.
\par 33 Os filhos de Midiã: Efa, Éfer, Enoque, Abida e Elda; todos estes foram filhos de Quetura.
\par 34 Abraão, pois, gerou a Isaque. Os filhos de Isaque: Esaú e Israel.
\par 35 Os filhos de Esaú: Elifaz, Reuel, Jeús, Jalão e Coré.
\par 36 Os filhos de Elifaz: Temã, Omar, Zefi, Gaetã, Quenaz, Timna e Amaleque.
\par 37 Os filhos de Reuel: Naate, Zerá, Samá e Mizá.
\par 38 Os filhos de Seir: Lotã, Sobal, Zibeão, Aná, Diso, Eser e Disã.
\par 39 Os filhos de Lotã: Hori e Homã; e a irmã de Lotã foi Timna.
\par 40 Os filhos de Sobal eram Aliã, Manaate, Ebal, Sefô e Onã. Os filhos de Zibeão: Aías e Aná.
\par 41 O filho de Aná: Disom. Os filhos de Disom: Hanrão, Esbã, Itrã e Querã.
\par 42 Os filhos de Eser: Bilã, Zaavã e Jaacã. Os filhos de Disã: Uz e Arã.
\par 43 São estes os reis que reinaram na terra de Edom, antes que houvesse rei sobre os filhos de Israel: Bela, filho de Beor, e o nome da sua cidade era Dinabá.
\par 44 Morreu Bela, e em seu lugar reinou Jobabe, filho de Zera, de Bozra.
\par 45 Morreu Jobabe, e em seu lugar reinou Husão, da terra dos temanitas.
\par 46 Morreu Husão, e em seu lugar reinou Hadade, filho de Bedade; este feriu a Midiã no campo de Moabe; o nome da sua cidade era Avite.
\par 47 Morreu Hadade, e em seu lugar reinou Samlá, de Masreca.
\par 48 Morreu Samlá, e em seu lugar reinou Saul, de Reobote, junto ao Eufrates.
\par 49 Morreu Saul, e em seu lugar reinou Baal-Hanã, filho de Acbor.
\par 50 Morreu Baal-Hanã, e em seu lugar reinou Hadade; o nome da sua cidade era Paú, e o de sua mulher era Meetabel, filha de Matrede, filha de Me-Zaabe.
\par 51 Morreu Hadade. São estes os nomes dos príncipes de Edom: o príncipe Timna, o príncipe Alva, o príncipe Jetete,
\par 52 o príncipe Oolibama, o príncipe Elá, o príncipe Pinom,
\par 53 o príncipe Quenaz, o príncipe Temã, o príncipe Mibzar,
\par 54 o príncipe Magdiel, o príncipe Irão; são estes os príncipes de Edom.

\chapter{2}

\par 1 São estes os filhos de Israel: Rúben, Simeão, Levi, Judá, Issacar, Zebulom,
\par 2 Dã, José, Benjamim, Naftali, Gade e Aser.
\par 3 Os filhos de Judá: Er, Onã e Selá; estes três lhe nasceram de Bate-Sua, a cananéia. Er, o primogênito de Judá, foi mau aos olhos do SENHOR, pelo que o matou.
\par 4 Porém Tamar, nora de Judá, lhe deu à luz a Perez e a Zera. Todos os filhos de Judá foram cinco.
\par 5 Os filhos de Perez: Hezrom e Hamul.
\par 6 Os filhos de Zera: Zinri, Etã, Hemã, Calcol e Dara, cinco ao todo.
\par 7 Os filhos de Carmi: Acar, o perturbador de Israel, que pecou na coisa condenada.
\par 8 O filho de Etã: Azarias.
\par 9 Os filhos de Hezrom, que lhe nasceram: Jerameel, Rão e Quelubai.
\par 10 Rão gerou a Aminadabe; Aminadabe gerou a Naassom, príncipe dos filhos de Judá;
\par 11 Naassom gerou a Salma, e Salma gerou a Boaz;
\par 12 Boaz gerou a Obede, e Obede gerou a Jessé;
\par 13 Jessé gerou a Eliabe, seu primogênito, a Abinadabe, o segundo, a Siméia, o terceiro,
\par 14 a Natanael, o quarto, a Radai, o quinto,
\par 15 a Ozém, o sexto, e a Davi, o sétimo.
\par 16 As irmãs destes foram Zeruia e Abigail. Os filhos de Zeruia foram três: Abisai, Joabe e Asael.
\par 17 Abigail deu à luz a Amasa; e o pai de Amasa foi Jéter, o ismaelita.
\par 18 Calebe, filho de Hezrom, gerou filhos de Azuba, sua mulher, e de Jeriote; foram estes os filhos desta: Jeser, Sobabe e Ardom.
\par 19 Morreu Azuba; e Calebe tomou para si a Efrata, da qual lhe nasceu Hur.
\par 20 Hur gerou a Uri, e Uri gerou a Bezalel.
\par 21 Então, Hezrom coabitou com a filha de Maquir, pai de Gileade; tinha ele sessenta anos quando a tomou, e ela deu à luz a Segube.
\par 22 Segube gerou a Jair, que teve vinte e três cidades na terra de Gileade.
\par 23 Gesur e Arã tomaram as aldeias de Jair, juntamente com Quenate e suas aldeias, a saber, sessenta lugares; todos estes foram filhos de Maquir, pai de Gileade.
\par 24 Depois da morte de Hezrom, em Calebe-Efrata, Abia, mulher de Hezrom, lhe deu a Azur, pai de Tecoa.
\par 25 Os filhos de Jerameel, primogênito de Hezrom, foram: Rão, o primogênito, Buna, Orém, Ozém e Aías.
\par 26 Teve Jerameel outra mulher, cujo nome era Atara; esta foi a mãe de Onã.
\par 27 Os filhos de Rão, primogênito de Jerameel, foram: Maaz, Jamim e Equer.
\par 28 Foram os filhos de Onã: Samai e Jada; e os filhos de Samai: Nadabe e Abisur.
\par 29 A mulher de Abisur chamava-se Abiail e lhe deu a Abã e a Molide.
\par 30 Os filhos de Nadabe: Selede e Apaim; e Selede morreu sem filhos.
\par 31 O filho de Apaim: Isi; o filho de Isi: Sesã. E o filho de Sesã: Alai.
\par 32 Os filhos de Jada, irmão de Samai, foram: Jéter e Jônatas; e Jéter morreu sem filhos.
\par 33 Os filhos de Jônatas: Pelete e Zaza; estes foram os filhos de Jerameel.
\par 34 Sesã não teve filhos, mas filhas; e tinha Sesã um servo egípcio, cujo nome era Jara.
\par 35 Deu, pois, Sesã sua filha por mulher a Jara, a quem ela deu à luz Atai.
\par 36 Atai gerou a Natã, e Natã gerou a Zabade.
\par 37 Zabade gerou a Eflal, e Eflal, a Obede.
\par 38 Obede gerou a Jeú, e Jeú, a Azarias.
\par 39 Azarias gerou a Heles, e Heles, a Eleasa.
\par 40 Eleasa gerou a Sismai, e Sismai, a Salum.
\par 41 Salum gerou a Jecamias, e Jecamias, a Elisama.
\par 42 O primogênito de Calebe, irmão de Jerameel, foi Maressa, que foi o pai de Zife; o filho de Maressa foi Abi-Hebrom.
\par 43 Os filhos de Hebrom: Coré, Tapua, Requém e Sema.
\par 44 Sema gerou a Raão, pai de Jorqueão; e Requém gerou a Samai.
\par 45 O filho de Samai foi Maom; e Maom foi o pai de Bete-Zur.
\par 46 Efá, a concubina de Calebe, deu à luz a Harã, a Mosa e a Gazez; e Harã gerou a Gazez.
\par 47 Os filhos de Jadai: Regém, Jotão, Gesã, Pelete, Efá e Saafe.
\par 48 De Maaca, concubina, gerou Calebe a Seber e a Tiraná;
\par 49 Maaca deu à luz também a Saafe, pai de Madmana, e a Seva, pai de Macbena e de Gibeá; e Acsa foi filha de Calebe.
\par 50 Estes foram os filhos de Calebe. Os filhos de Hur, primogênito de Efrata, foram: Sobal, pai de Quiriate-Jearim,
\par 51 Salma, pai dos belemitas, e Harefe, pai de Bete-Gader.
\par 52 Os filhos de Sobal, pai de Quiriate-Jearim, foram: Haroé e Hazi-Hamenuote.
\par 53 As famílias de Quiriate-Jearim foram: os itritas, os puteus, os sumateus e os misraeus; destes procederam os zoratitas e os estaoleus.
\par 54 Os filhos de Salma: Belém e os netofatitas, Atrote-Bete-Joabe e Hazi-Hamanaate, zoreu.
\par 55 As famílias dos escribas que habitavam em Jabez foram os tiratitas, os simeatitas e os sucatitas; são estes os queneus, que vieram de Hamate, pai da casa de Recabe.

\chapter{3}

\par 1 Estes foram os filhos de Davi, que lhe nasceram em Hebrom: o primogênito, Amnom, de Ainoã, a jezreelita; o segundo, Daniel, de Abigail, a carmelita;
\par 2 o terceiro, Absalão, filho de Maaca, filha de Talmai, rei de Gesur; o quarto, Adonias, filho de Hagite;
\par 3 o quinto, Sefatias, de Abital; o sexto, Itreão, de Eglá, sua mulher.
\par 4 Seis filhos lhe nasceram em Hebrom, porque ali reinou sete anos e seis meses; e trinta e três anos reinou em Jerusalém.
\par 5 Estes lhe nasceram em Jerusalém: Siméia, Sobabe, Natã e Salomão; estes quatro lhe nasceram de Bate-Seba, filha de Amiel.
\par 6 Nasceram-lhe mais Ibar, Elisama, Elifelete,
\par 7 Nogá, Nefegue, Jafia,
\par 8 Elisama, Eliada e Elifelete; nove ao todo.
\par 9 Todos estes foram filhos de Davi, afora os filhos das concubinas; e Tamar, irmã deles.
\par 10 O filho de Salomão foi Roboão, de quem foi filho Abias, de quem foi filho Asa, de quem foi filho Josafá;
\par 11 de quem foi filho Jeorão, de quem foi filho Acazias, de quem foi filho Joás;
\par 12 de quem foi filho Amazias, de quem foi filho Azarias, de quem foi filho Jotão;
\par 13 de quem foi filho Acaz, de quem foi filho Ezequias, de quem foi filho Manassés;
\par 14 de quem foi filho Amom, de quem foi filho Josias.
\par 15 Os filhos de Josias foram: o primogênito, Joanã; o segundo, Jeoaquim; o terceiro, Zedequias; o quarto, Salum.
\par 16 Os filhos de Jeoaquim: Jeconias e Zedequias.
\par 17 Os filhos de Jeconias, o cativo: Sealtiel,
\par 18 Malquirão, Pedaías, Senazar, Jecamias, Hosama e Nedabias.
\par 19 Os filhos de Pedaías: Zorobabel e Simei; os filhos de Zorobabel: Mesulão e Hananias; e Selomite, irmã deles;
\par 20 e Hasuba, Oel, Berequias, Hasadias e Jusabe-Hesede; cinco ao todo.
\par 21 Os filhos de Hananias: Pelatias e Jesaías; os filhos de Refaías, os filhos de Arnã, os filhos de Obadias, os filhos de Secanias.
\par 22 O filho de Secanias foi Semaías; os filhos de Semaías: Hatus, Igal, Barias, Nearias e Safate; seis ao todo.
\par 23 Os filhos de Nearias: Elioenai, Ezequias e Azricão; três ao todo.
\par 24 Os filhos de Elioenai: Hodavias, Eliasibe, Pelaías, Acube, Joanã, Delaías e Anani; sete ao todo.

\chapter{4}

\par 1 Os filhos de Judá foram: Perez, Hezrom, Carmi, Hur e Sobal.
\par 2 Reaías, filho de Sobal, gerou a Jaate; Jaate gerou a Aumai e a Laade; são estas as famílias dos zoratitas.
\par 3 Estes foram os filhos do pai de Etã: Jezreel, Isma e Idbas; e era o nome da irmã deles Hazelelponi;
\par 4 e mais Penuel, pai de Gedor, e Ezer, pai de Husa; estes foram os filhos de Hur, o primogênito de Efrata e pai de Belém.
\par 5 Asur, pai de Tecoa, teve duas mulheres: Hela e Naara.
\par 6 Naara deu à luz a Auzão, a Héfer, a Temeni e a Haastari; estes foram os filhos de Naara.
\par 7 Os filhos de Hela: Zerete, Isar e Etnã.
\par 8 Coz gerou a Anube e a Zobeba e foi pai das famílias de Aarel, filho de Harum.
\par 9 Foi Jabez mais ilustre do que seus irmãos; sua mãe chamou-lhe Jabez, dizendo: Porque com dores o dei à luz.
\par 10 Jabez invocou o Deus de Israel, dizendo: Oh! Tomara que me abençoes e me alargues as fronteiras, que seja comigo a tua mão e me preserves do mal, de modo que não me sobrevenha aflição! E Deus lhe concedeu o que lhe tinha pedido.
\par 11 Quelube, irmão de Suá, gerou a Meir; este é o pai de Estom.
\par 12 Estom gerou a Bete-Rafa, a Paséia e a Teína, pai de Ir-Naás; estes foram os homens de Reca.
\par 13 Os filhos de Quenaz foram: Otniel e Seraías; o filho de Otniel: Hatate.
\par 14 Meonotai gerou a Ofra, e Seraías gerou a Joabe, fundador do vale dos Artífices, porque os dali eram artífices.
\par 15 Os filhos de Calebe, filho de Jefoné: Iru, Elá e Naã; e o filho de Elá: Quenaz.
\par 16 Os filhos de Jealelel: Zife, Zifa, Tiria e Asareel.
\par 17 Os filhos de Ezra: Jéter, Merede, Éfer e Jalom; foram os filhos de Bitia, filha de Faraó, que Merede tomou por mulher: Miriã, Samai e Isbá, pai de Estemoa.
\par 18 E sua mulher, judia, deu à luz a Jerede, pai de Gedor, a Héber, pai de Socó, e a Jecutiel, pai de Zanoa.
\par 19 Os filhos da mulher de Hodias, irmã de Naã: Abiqueila, o garmita, e Estemoa, o maacatita.
\par 20 Os filhos de Simão: Amnom, Rina, Ben-Hanã e Tilom; e os filhos de Isi: Zoete e Ben-Zoete;
\par 21 os filhos de Selá, filho de Judá: Er, pai de Leca, e Lada, pai de Maressa; Selá foi também pai das famílias da casa dos obreiros em linho, em Bete-Asbéia,
\par 22 como ainda Joquim, e os homens de Cozeba, de Joás, de Sarafe, os quais dominavam sobre Moabe, e de Jasubi-Leém. Estes registros são antigos.
\par 23 Estes eram oleiros e habitantes de Netaim e de Gedera; moravam ali com o rei para o servirem.
\par 24 Os filhos de Simeão foram: Nemuel, Jamim, Jaribe, Zerá e Saul,
\par 25 de quem foi filho Salum, de quem foi filho Mibsão, de quem foi filho Misma.
\par 26 O filho de Misma foi Hamuel, de quem foi filho Zacur, de quem foi filho Simei.
\par 27 Simei teve dezesseis filhos e seis filhas; mas seus irmãos não tiveram muitos filhos, nem se multiplicaram todas as suas famílias como os filhos de Judá.
\par 28 Habitavam em Berseba, em Molada, em Hazar-Sual,
\par 29 em Bila, em Ezém, em Tolade,
\par 30 em Betuel, em Horma, em Ziclague,
\par 31 em Bete-Marcabote, em Hazar-Susim, em Bete-Biri e em Saaraim. Estas foram as suas cidades, até ao reinado de Davi.
\par 32 As suas aldeias foram: Etã, Aim, Rimom, Toquém e Asã; cinco cidades,
\par 33 com todas as suas aldeias que estavam ao redor destas cidades, até Baal. Estas foram as suas habitações e tinham seu registro genealógico.
\par 34 Estes, registrados por seus nomes, foram príncipes nas suas famílias: Mesobabe, Janleque, Josa, filho de Amazias,
\par 35 Joel, Jeú, filho de Josibias, filho de Seraías, filho de Asiel,
\par 36 Elioenai, Jaacobá, Jesoaías, Asaías, Adiel, Jesimiel, Benaia,
\par 37 Ziza, filho de Sifi, filho de Alom, filho de Jedaías, filho de Sinri, filho de Semaías;
\par 38 e as famílias de seus pais se multiplicaram abundantemente.
\par 39 Chegaram até à entrada de Gedor, ao oriente do vale, à procura de pasto para os seus rebanhos.
\par 40 Acharam pasto farto e bom e a terra espaçosa, tranqüila e pacífica, onde habitaram, dantes, os descendentes de Cam.
\par 41 Estes, que estão registrados por seus nomes, vieram nos dias de Ezequias, rei de Judá, e derribaram as tendas, e feriram os meunitas que se encontraram ali, e os destruíram totalmente até ao dia de hoje, e habitaram em lugar deles, porque ali havia pasto para os seus rebanhos.
\par 42 Também deles, dos filhos de Simeão, quinhentos homens foram ao monte Seir, tendo por capitães a Pelatias, a Nearias, a Refaías e a Uziel, filhos de Isi.
\par 43 Feriram o restante dos que escaparam dos amalequitas e habitam ali até ao dia de hoje.

\chapter{5}

\par 1 Quanto aos filhos de Rúben, o primogênito de Israel (pois era o primogênito, mas, por ter profanado o leito de seu pai, deu-se a sua primogenitura aos filhos de José, filho de Israel; de modo que, na genealogia, não foi contado como primogênito.
\par 2 Judá, na verdade, foi poderoso entre seus irmãos, e dele veio o príncipe; porém o direito da primogenitura foi de José.),
\par 3 foram estes: Enoque, Palu, Hezrom e Carmi.
\par 4 O filho de Joel: Semaías, de quem foi filho Gogue, de quem foi filho Simei,
\par 5 de quem foi filho Mica, de quem foi filho Reaías, de quem foi filho Baal,
\par 6 de quem foi filho Beera, o qual Tiglate-Pileser, rei da Assíria, levou cativo; ele foi príncipe dos rubenitas.
\par 7 Quanto a seus irmãos, pelas suas famílias, quando foram inscritos nas genealogias segundo as suas descendências, tinham por cabeças Jeiel, Zacarias,
\par 8 Bela, filho de Azaz, filho de Sema, filho de Joel, que habitaram em Aroer, até Nebo e Baal-Meom;
\par 9 também habitaram do lado oriental, até à entrada do deserto, o qual se estende até ao rio Eufrates, porque o seu gado se tinha multiplicado na terra de Gileade.
\par 10 Nos dias de Saul, fizeram guerra aos hagarenos, que caíram pelo poder de sua mão, e habitaram nas tendas deles, em toda a terra fronteira de Gileade, do lado oriental.
\par 11 Os filhos de Gade habitaram defronte deles, na terra de Basã, até Salca.
\par 12 Joel foi o cabeça, e Safã, o segundo; também Janai e Safate estavam em Basã.
\par 13 Seus irmãos, segundo as suas casas paternas, foram: Micael, Mesulão, Seba, Jorai, Jacã, Zia e Héber; ao todo, sete;
\par 14 estes foram os filhos de Abiail, filho de Huri, filho de Jaroa, filho de Gileade, filho de Micael, filho de Jesisai, filho de Jado, filho de Buz.
\par 15 Aí, filho de Abdiel, filho de Guni, foi o cabeça da sua família.
\par 16 Habitaram em Gileade, em Basã e suas aldeias, bem como até aos limites de todos os arredores de Sarom.
\par 17 Todos estes foram inscritos na genealogia, nos dias de Jotão, rei de Judá, e nos dias de Jeroboão, rei de Israel.
\par 18 Dos filhos de Rúben, dos gaditas e da meia tribo de Manassés, homens valentes, que traziam escudo e espada, entesavam o arco e eram destros na guerra, houve quarenta e quatro mil setecentos e sessenta, capazes de sair a combate.
\par 19 Fizeram guerra aos hagarenos, como a Jetur, a Nafis e a Nodabe.
\par 20 Foram ajudados contra eles, e os hagarenos e todos quantos estavam com eles foram entregues nas suas mãos; porque, na peleja, clamaram a Deus, que lhes deu ouvidos, porquanto confiaram nele.
\par 21 Levaram o gado deles: cinqüenta mil camelos, duzentas e cinqüenta mil ovelhas, dois mil jumentos; e cem mil pessoas.
\par 22 Porque muitos caíram feridos à espada, pois de Deus era a peleja; e habitaram no lugar deles até ao exílio.
\par 23 Os filhos da meia tribo de Manassés habitaram naquela terra de Basã até Baal-Hermom, e Senir, e o monte Hermom; e eram numerosos.
\par 24 Estes foram cabeças de suas famílias, a saber: Éfer, Isi, Eliel, Azriel, Jeremias, Hodavias e Jadiel, guerreiros valentes, homens famosos, cabeças de suas famílias.
\par 25 Porém cometeram transgressões contra o Deus de seus pais e se prostituíram, seguindo os deuses dos povos da terra, os quais Deus destruíra de diante deles.
\par 26 Pelo que o Deus de Israel suscitou o espírito de Pul, rei da Assíria, e o espírito de Tiglate-Pileser, rei da Assíria, que os levou cativos, a saber: os rubenitas, os gaditas e a meia tribo de Manassés, e os trouxe para Hala, Habor e Hara e para o rio Gozã, onde permanecem até ao dia de hoje.

\chapter{6}

\par 1 Os filhos de Levi foram: Gérson, Coate e Merari.
\par 2 Os filhos de Coate: Anrão, Isar, Hebrom e Uziel.
\par 3 Os filhos de Anrão: Arão, Moisés e Miriã. Os filhos de Arão: Nadabe, Abiú, Eleazar e Itamar.
\par 4 Eleazar gerou a Finéias, e Finéias, a Abisua;
\par 5 Abisua gerou a Buqui, e Buqui, a Uzi;
\par 6 Uzi gerou a Zeraías, e Zeraías, a Meraiote;
\par 7 Meraiote gerou a Amarias, e Amarias, a Aitube;
\par 8 Aitube gerou a Zadoque, e Zadoque, a Aimaás;
\par 9 Aimaás gerou a Azarias, e Azarias, a Joanã;
\par 10 Joanã gerou a Azarias; este é o que serviu de sacerdote na casa que Salomão tinha edificado em Jerusalém.
\par 11 Azarias gerou a Amarias, e Amarias, a Aitube;
\par 12 Aitube gerou a Zadoque, e Zadoque, a Salum;
\par 13 Salum gerou a Hilquias, e Hilquias, a Azarias;
\par 14 Azarias gerou a Seraías, e Seraías, a Jeozadaque;
\par 15 Jeozadaque foi levado cativo, quando o SENHOR levou para o exílio a Judá e a Jerusalém pela mão de Nabucodonosor.
\par 16 Os filhos de Levi: Gérson, Coate e Merari.
\par 17 São estes os nomes dos filhos de Gérson: Libni e Simei.
\par 18 Os filhos de Coate: Anrão, Isar, Hebrom e Uziel.
\par 19 Os filhos de Merari: Mali e Musi; são estas as famílias dos levitas, segundo as casas de seus pais.
\par 20 O filho de Gérson foi Libni, de quem foi filho Jaate, de quem foi filho Zima,
\par 21 de quem foi filho Joá, de quem foi filho Ido, de quem foi filho Zerá, de quem foi filho Jeaterai.
\par 22 O filho de Coate foi Aminadabe, de quem foi filho Coré, de quem foi filho Assir,
\par 23 de quem foi filho Elcana, de quem foi filho Ebiasafe, de quem foi filho Assir,
\par 24 de quem foi filho Taate, de quem foi filho Uriel, de quem foi filho Uzias, de quem foi filho Saul.
\par 25 Os filhos de Elcana: Amasai e Aimote.
\par 26 Quanto a Elcana, foi seu filho Zofai, de quem foi filho Naate,
\par 27 de quem foi filho Eliabe, de quem foi filho Jeroão, de quem foi filho Elcana.
\par 28 Os filhos de Samuel: o primogênito, Joel, e depois Abias.
\par 29 O filho de Merari foi Mali, de quem foi filho Libni, de quem foi filho Simei, de quem foi filho Uzá,
\par 30 de quem foi filho Siméia, de quem foi filho Hagias, de quem foi filho Asaías.
\par 31 São estes os que Davi constituiu para dirigir o canto na Casa do SENHOR, depois que a arca teve repouso.
\par 32 Ministravam diante do tabernáculo da tenda da congregação com cânticos, até que Salomão edificou a Casa do SENHOR em Jerusalém; e exercitavam o seu ministério segundo a ordem prescrita.
\par 33 São estes os que serviam com seus filhos. Dos filhos dos coatitas: Hemã, o cantor, filho de Joel, filho de Samuel,
\par 34 filho de Elcana, filho de Jeroão, filho de Eliel, filho de Toá,
\par 35 filho de Zufe, filho de Elcana, filho de Maate, filho de Amasai,
\par 36 filho de Elcana, filho de Joel, filho de Azarias, filho de Sofonias,
\par 37 filho de Taate, filho de Assir, filho de Ebiasafe, filho de Coré,
\par 38 filho de Isar, filho de Coate, filho de Levi, filho de Israel.
\par 39 Seu irmão Asafe estava à sua direita; era Asafe filho de Berequias, filho de Siméia,
\par 40 filho de Micael, filho de Baaséias, filho de Malquias,
\par 41 filho de Etni, filho de Zera, filho de Adaías,
\par 42 filho de Etã, filho de Zima, filho de Simei,
\par 43 filho de Jaate, filho de Gérson, filho de Levi.
\par 44 Seus irmãos, os filhos de Merari, estavam à esquerda, a saber: Etã, filho de Quisi, filho de Abdi, filho de Maluque,
\par 45 filho de Hasabias, filho de Amazias, filho de Hilquias,
\par 46 filho de Anzi, filho de Bani, filho de Semer,
\par 47 filho de Mali, filho de Musi, filho de Merari, filho de Levi.
\par 48 Seus irmãos, os levitas, foram postos para todo o serviço do tabernáculo da Casa de Deus.
\par 49 Arão e seus filhos faziam ofertas sobre o altar do holocausto e sobre o altar do incenso, todo o serviço do lugar santíssimo e a expiação por Israel, segundo tudo quanto Moisés, servo de Deus, tinha ordenado.
\par 50 Foi filho de Arão Eleazar, de quem foi filho Finéias, de quem foi filho Abisua,
\par 51 de quem foi filho Buqui, de quem foi filho Uzi, de quem foi filho Zeraías,
\par 52 de quem foi filho Meraiote, de quem foi filho Amarias, de quem foi filho Aitube,
\par 53 de quem foi filho Zadoque, de quem foi filho Aimaás.
\par 54 São estes os lugares que eles habitavam, segundo os seus acampamentos, dentro dos seus limites, a saber: aos filhos de Arão, das famílias dos coatitas, pois lhes caiu a sorte,
\par 55 deram-lhes Hebrom, na terra de Judá, e os seus arredores.
\par 56 Porém o campo da cidade com suas aldeias deram a Calebe, filho de Jefoné.
\par 57 Aos filhos de Arão deram as cidades de refúgio: Hebrom e Libna com seus arredores, Jatir e Estemoa com seus arredores,
\par 58 Hilém com seus arredores, Debir com seus arredores,
\par 59 Asã com seus arredores e Bete-Semes com seus arredores;
\par 60 da tribo de Benjamim, Geba com seus arredores, Alemete com seus arredores e Anatote com seus arredores; ao todo, treze cidades, segundo as suas famílias.
\par 61 Aos filhos de Coate, que restaram da família da tribo, caíram por sorte dez cidades da meia tribo, metade de Manassés.
\par 62 Aos filhos de Gérson, segundo as suas famílias, da tribo de Issacar, da tribo de Aser, da tribo de Naftali e da tribo de Manassés, em Basã, caíram treze cidades.
\par 63 Aos filhos de Merari, segundo as suas famílias, da tribo de Rúben, da tribo de Gade e da tribo de Zebulom, caíram por sorte doze cidades.
\par 64 Assim, deram os filhos de Israel aos levitas estas cidades com seus arredores.
\par 65 Deram-lhes por sorte, da tribo dos filhos de Judá, da tribo dos filhos de Simeão e da tribo dos filhos de Benjamim, estas cidades, que são mencionadas nominalmente.
\par 66 A algumas das famílias dos filhos de Coate foram dadas cidades dos seus territórios da parte da tribo de Efraim.
\par 67 Pois lhes deram as cidades de refúgio, Siquém com seus arredores, na região montanhosa de Efraim, como também Gezer com seus arredores,
\par 68 Jocmeão com seus arredores, Bete-Horom com seus arredores,
\par 69 Aijalom com seus arredores e Gate-Rimom com seus arredores;
\par 70 e, da meia tribo de Manassés, Aner com seus arredores e Bileã com seus arredores foram dadas às demais famílias dos filhos de Coate.
\par 71 Aos filhos de Gérson, da família da meia tribo de Manassés, deram, em Basã, Golã com seus arredores e Astarote com seus arredores;
\par 72 e da tribo de Issacar: Quedes com seus arredores, Daberate com seus arredores,
\par 73 Ramote com seus arredores e Aném com seus arredores;
\par 74 e da tribo de Aser: Masal com seus arredores, Abdom com seus arredores,
\par 75 Hucoque com seus arredores e Reobe com seus arredores;
\par 76 e da tribo de Naftali na Galiléia: Quedes com seus arredores, Hamom com seus arredores e Quiriataim com seus arredores.
\par 77 Os demais filhos de Merari receberam, da tribo de Zebulom, Rimono com seus arredores e Tabor com seus arredores;
\par 78 e dalém do Jordão, na altura de Jericó, ao oriente do Jordão, deram-se-lhes, da tribo de Rúben, Bezer com seus arredores no deserto, Jaza com seus arredores,
\par 79 Quedemote com seus arredores e Mefaate com seus arredores;
\par 80 e da tribo de Gade em Gileade: Ramote com seus arredores, Maanaim com seus arredores,
\par 81 Hesbom com seus arredores e Jazer com seus arredores.

\chapter{7}

\par 1 Os filhos de Issacar foram: Tola, Puá, Jasube e Sinrom; quatro ao todo.
\par 2 Os filhos de Tola: Uzi, Refaías, Jeriel, Jamai, Ibsão e Samuel, chefes das suas famílias, descendentes de Tola; homens valentes nas suas gerações, cujo número, nos dias de Davi, foi de vinte e dois mil e setecentos.
\par 3 O filho de Uzi: Izraías; e os filhos de Izraías: Micael, Obadias, Joel e Issias; cinco ao todo; todos eles chefes.
\par 4 Tinham, nas suas gerações, segundo as suas famílias, em tropas de guerra, trinta e seis mil homens; pois tinham muitas mulheres e filhos.
\par 5 Seus irmãos, em todas as famílias de Issacar, homens valentes, foram oitenta e sete mil, todos registrados pelas suas genealogias.
\par 6 Os filhos de Benjamim: Bela, Bequer e Jediael; três ao todo.
\par 7 Os filhos de Bela: Esbom, Uzi, Uziel, Jerimote e Iri; cinco ao todo; chefes das suas famílias, homens valentes, foram vinte e dois mil e trinta e quatro, registrados pelas suas genealogias.
\par 8 Os filhos de Bequer: Zemira, Joás, Eliézer, Elioenai, Onri, Jerimote, Abias, Anatote e Alemete; todos filhos de Bequer.
\par 9 O número deles, registrados pelas suas genealogias, segundo as suas gerações, chefes das suas famílias, homens valentes, vinte mil e duzentos.
\par 10 O filho de Jediael: Bilã; os filhos de Bilã: Jeús, Benjamim, Eúde, Quenaana, Zetã, Társis e Aisaar.
\par 11 Todos estes, filhos de Jediael, foram chefes das suas famílias, homens valentes, dezessete mil e duzentos, capazes de sair à guerra.
\par 12 Supim e Hupim eram filhos de Ir; e Husim, filho de Aer.
\par 13 Os filhos de Naftali: Jaziel, Guni, Jezer e Salum, filhos de Bila.
\par 14 O filho de Manassés: Asriel, de sua concubina síria, que também deu à luz a Maquir, pai de Gileade.
\par 15 Maquir tomou a irmã de Hupim e Supim por mulher. O nome dela era Maaca; o nome do irmão de Maquir era Zelofeade, o qual teve só filhas.
\par 16 Maaca, mulher de Maquir, teve um filho, a quem chamou Perez; irmão deste foi Seres. Os filhos de Perez foram Ulão e Requém.
\par 17 O filho de Ulão: Bedã. Tais foram os filhos de Gileade, filho de Maquir, filho de Manassés.
\par 18 Sua irmã Hamolequete deu à luz a Isode, a Abiezer e a Macla.
\par 19 Foram os filhos de Semida: Aiã, Siquém, Liqui e Anião.
\par 20 Era filho de Efraim Sutela, de quem foi filho Berede, de quem foi filho Taate, de quem foi filho Eleada, de quem foi filho Taate,
\par 21 de quem foi filho Zabade, de quem foi filho Sutela; e ainda Ézer e Eleade, mortos pelos homens de Gate, naturais da terra, pois eles desceram para roubar o gado destes.
\par 22 Pelo que por muitos dias os chorou Efraim, seu pai, cujos irmãos vieram para o consolar.
\par 23 Depois, coabitou com sua mulher, e ela concebeu e teve um filho, a quem ele chamou Berias, porque as coisas iam mal na sua casa.
\par 24 Sua filha foi Seerá, que edificou a Bete-Horom, a de baixo e a de cima, como também a Uzém-Seerá.
\par 25 O filho de Berias foi Refa, de quem foi filho Resefe, de quem foi filho Tela, de quem foi filho Taã,
\par 26 de quem foi filho Ladã, de quem foi filho Amiúde, de quem foi filho Elisama,
\par 27 de quem foi filho Num, de quem foi filho Josué.
\par 28 A possessão e habitação deles foram: Betel e as suas aldeias; ao oriente, Naarã; e, ao ocidente, Gezer e suas aldeias, Siquém e suas aldeias, até Aia e suas aldeias;
\par 29 do lado dos filhos de Manassés, Bete-Seã e suas aldeias, Taanaque e suas aldeias, Megido e suas aldeias, Dor e suas aldeias; nestas, habitaram os filhos de José, filho de Israel.
\par 30 Os filhos de Aser: Imna, Isvá, Isvi e Berias e Sera, irmã deles.
\par 31 Os filhos de Berias: Héber e Malquiel; este foi o pai de Birzavite.
\par 32 Héber gerou a Jaflete, a Somer, a Hotão e a Suá, irmã deles.
\par 33 Os filhos de Jaflete: Pasaque, Bimal e Asvate; estes foram os filhos de Jaflete.
\par 34 Os filhos de Semer: Aí, Roga, Jeubá e Arã.
\par 35 Os filhos de seu irmão Helém: Zofa, Imna, Seles e Amal.
\par 36 Os filhos de Zofa: Sua, Harnefer, Sual, Beri, Inra,
\par 37 Bezer, Hode, Sama, Silsa, Itrã e Beera.
\par 38 Os filhos de Jéter: Jefoné, Pispa e Ara.
\par 39 Os filhos de Ula: Ara, Haniel e Rizia.
\par 40 Todos estes foram filhos de Aser, chefes das famílias, escolhidos, homens valentes, chefes de príncipes, registrados nas suas genealogias para o serviço na guerra; seu número foi de vinte e seis mil homens.

\chapter{8}

\par 1 Benjamim gerou a Bela, seu primogênito, a Asbel, o segundo, a Aará, o terceiro,
\par 2 a Noá, o quarto, e a Rafa, o quinto.
\par 3 Bela teve estes filhos: Adar, Gera, Abiúde,
\par 4 Abisua, Naamã, Aoá,
\par 5 Gera, Sefufá e Hurão.
\par 6 Estes foram os filhos de Eúde, que foram chefes das famílias dos moradores de Geba e transportados para o exílio a Manaate:
\par 7 Naamã, Aías e Gera; este os transportou e gerou a Uzá e a Aiúde.
\par 8 Saaraim, depois de ter repudiado suas mulheres Husim e Baara, gerou nos campos de Moabe,
\par 9 de Hodes, sua mulher, a Jobabe, a Zíbia, a Messa, a Malcã,
\par 10 a Jeús, a Saquias e a Mirma; foram estes os seus filhos, chefes das famílias.
\par 11 Husim gerou a Abitube e a Elpaal.
\par 12 Os filhos de Elpaal foram: Héber, Misã e Semede; este edificou a Ono e a Lode e suas aldeias.
\par 13 Berias e Sema foram cabeças das famílias dos moradores de Aijalom, que afugentaram os moradores de Gate.
\par 14 Aiô, Sasaque, Jeremote,
\par 15 Zebadias, Arade, Éder,
\par 16 Micael, Ispa e Joá foram filhos de Berias.
\par 17 Zebadias, Mesulão, Hizqui, Héber,
\par 18 Ismerai, Izlias e Jobabe, filhos de Elpaal.
\par 19 Jaquim, Zicri, Zabdi,
\par 20 Elienai, Ziletai, Eliel,
\par 21 Adaías, Beraías e Sinrate, filhos de Simei.
\par 22 Ispã, Héber, Eliel,
\par 23 Abdom, Zicri, Hanã,
\par 24 Hananias, Elão, Antotias,
\par 25 Ifdéias e Penuel, filhos de Sasaque.
\par 26 Sanserai, Searias, Atalias,
\par 27 Jaaresias, Elias e Zicri, filhos de Jeroão.
\par 28 Estes foram chefes das famílias, segundo as suas gerações, e habitaram em Jerusalém.
\par 29 Em Gibeão habitou o pai de Gibeão, cuja mulher se chamava Maaca,
\par 30 e também seu filho primogênito Abdom e ainda Zur, Quis, Baal, Nadabe,
\par 31 Gedor, Aiô e Zequer.
\par 32 Miclote gerou a Siméia. Estes habitaram em Jerusalém, com seus irmãos, bem defronte deles.
\par 33 Ner gerou a Quis; e Quis gerou a Saul; Saul gerou a Jônatas, a Malquisua, a Abinadabe e a Esbaal.
\par 34 Filho de Jônatas foi Meribe-Baal, e Meribe-Baal gerou a Mica.
\par 35 Os filhos de Mica foram: Pitom, Meleque, Taréia e Acaz.
\par 36 Acaz gerou a Jeoada; Jeoada gerou a Alemete, a Azmavete e a Zinri; e Zinri gerou a Mosa.
\par 37 Mosa gerou a Bineá, de quem foi filho Rafa, de quem foi filho Eleasa, de quem foi filho Azel.
\par 38 Teve Azel seis filhos, cujos nomes foram: Azricão, Bocru, Ismael, Searias, Obadias e Hanã; todos estes foram filhos de Azel.
\par 39 Os filhos de Eseque, seu irmão, foram: Ulão, seu primogênito, Jeús, o segundo, e Elifelete, o terceiro.
\par 40 Os filhos de Ulão foram homens valentes, flecheiros; e tiveram muitos filhos e netos: cento e cinqüenta. Todos estes foram dos filhos de Benjamim.

\chapter{9}

\par 1 Todo o Israel foi registrado por genealogias e inscrito no Livro dos Reis de Israel, e Judá foi levado para o exílio à Babilônia, por causa da sua transgressão.
\par 2 Os primeiros habitadores, que de novo vieram morar nas suas próprias possessões e nas suas cidades, foram os israelitas, os sacerdotes, os levitas e os servos do templo.
\par 3 Porém alguns dos filhos de Judá, dos filhos de Benjamim e dos filhos de Efraim e Manassés habitaram em Jerusalém:
\par 4 Utai, filho de Amiúde, filho de Onri, filho de Inri, filho de Bani, dos filhos de Perez, filho de Judá;
\par 5 dos silonitas: Asaías, o primogênito, e seus filhos;
\par 6 dos filhos de Zerá: Jeuel e seus irmãos; seiscentos e noventa ao todo;
\par 7 dos filhos de Benjamim: Salu, filho de Mesulão, filho de Hodavias, filho de Hassenuá;
\par 8 Ibnéias, filho de Jeroão, e Elá, filho de Uzi, filho de Micri, e Mesulão, filho de Sefatias, filho de Reuel, filho de Ibnijas;
\par 9 e seus irmãos, segundo as suas gerações; novecentos e cinqüenta e seis ao todo; todos estes homens foram cabeças de famílias nas casas de suas famílias.
\par 10 Dos sacerdotes: Jedaías, Jeoiaribe, Jaquim,
\par 11 Azarias, filho de Hilquias, filho de Mesulão, filho de Zadoque, filho de Meraiote, filho de Aitube, príncipe da Casa de Deus;
\par 12 Adaías, filho de Jeroão, filho de Pasur, filho de Malquias, e Masai, filho de Adiel, filho de Jazera, filho de Mesulão, filho de Mesilemite, filho de Imer,
\par 13 como também seus irmãos, cabeças das suas famílias; mil setecentos e sessenta ao todo, homens capazes para a obra do ministério da Casa de Deus.
\par 14 Dos levitas: Semaías, filho de Hassube, filho de Azricão, filho de Hasabias, dos filhos de Merari;
\par 15 Baquebacar, Heres, Galal e Matanias, filho de Mica, filho de Zicri, filho de Asafe;
\par 16 Obadias, filho de Semaías, filho de Galal, filho de Jedutum; Berequias, filho de Asa, filho de Elcana, morador das aldeias dos netofatitas.
\par 17 Os porteiros: Salum, Acube, Talmom e Aimã e os irmãos deles; Salum era o chefe.
\par 18 Estavam até agora de guarda à porta do rei, do lado do oriente; tais foram os porteiros dos arraiais dos filhos de Levi.
\par 19 Salum, filho de Coré, filho de Ebiasafe, filho de Corá, e seus irmãos da casa de seu pai, os coreítas, estavam encarregados da obra do ministério e eram guardas das portas do tabernáculo; e seus pais tinham sido encarregados do arraial do SENHOR e eram guardas da entrada.
\par 20 Finéias, filho de Eleazar, os regia nesse tempo, e o SENHOR era com ele.
\par 21 Zacarias, filho de Meselemias, era o porteiro da entrada da tenda da congregação.
\par 22 Todos estes, escolhidos para guardas das portas, foram duzentos e doze. Estes foram registrados pelas suas genealogias nas suas respectivas aldeias; e Davi e Samuel, o vidente, os constituíram cada um no seu cargo.
\par 23 Guardavam, pois, eles e seus filhos as portas da Casa do SENHOR, na casa da tenda.
\par 24 Os porteiros estavam aos quatro ventos: ao oriente, ao ocidente, ao norte e ao sul.
\par 25 Seus irmãos, que habitavam nas suas aldeias, tinham de vir, de tempo em tempo, para servir com eles durante sete dias;
\par 26 porque havia sempre, naquele ofício, quatro porteiros principais, que eram levitas, e tinham a seu cargo as câmaras e os tesouros da Casa de Deus.
\par 27 Estavam alojados à roda da Casa de Deus, porque a vigilância lhes estava encarregada, e tinham o dever de a abrir, todas as manhãs.
\par 28 Alguns deles estavam encarregados dos utensílios do ministério, porque estes eram contados quando eram trazidos e quando eram tirados.
\par 29 Outros havia que estavam encarregados dos móveis e de todos os objetos do santuário, como também da flor de farinha, do vinho, do azeite, do incenso e da especiaria.
\par 30 Alguns dos filhos dos sacerdotes confeccionavam as especiarias.
\par 31 Matitias, dentre os levitas, o primogênito de Salum, o coreíta, tinha o cargo do que se fazia em assadeiras.
\par 32 Outros dos seus irmãos, dos filhos dos coatitas, tinham o encargo de preparar os pães da proposição para todos os sábados.
\par 33 Quanto aos cantores, cabeças das famílias entre os levitas, estavam alojados nas câmaras do templo e eram isentos de outros serviços; porque, de dia e de noite, estavam ocupados no seu mister.
\par 34 Estes foram cabeças das famílias entre os levitas, chefes em suas gerações, e habitavam em Jerusalém.
\par 35 Em Gibeão habitou Jeiel, pai de Gibeão, cuja mulher se chamava Maaca;
\par 36 e também seu filho primogênito Abdom e ainda Zur, Quis, Baal, Ner, Nadabe,
\par 37 Gedor, Aiô, Zacarias e Miclote.
\par 38 Miclote gerou a Siméia. Estes habitaram em Jerusalém, com seus irmãos, bem defronte deles.
\par 39 Ner gerou a Quis; e Quis gerou a Saul, Saul gerou a Jônatas, a Malquisua, a Abinadabe e a Esbaal.
\par 40 Filho de Jônatas foi Meribe-Baal, e Meribe-Baal gerou a Mica.
\par 41 Os filhos de Mica foram: Pitom, Meleque e Taréia.
\par 42 Acaz gerou a Jaerá, e Jaerá gerou a Alemete, a Azmavete e a Zinri; e Zinri gerou a Mosa.
\par 43 Mosa gerou a Bineá, de quem foi filho Refaías, de quem foi filho Eleasa, de quem foi filho Azel.
\par 44 Teve Azel seis filhos, cujos nomes foram Azricão, Bocru, Ismael, Searias, Obadias e Hanã; todos estes foram filhos de Azel.

\chapter{10}

\par 1 Os filisteus pelejaram contra Israel; e, tendo os homens de Israel fugido de diante dos filisteus, caíram mortos no monte Gilboa.
\par 2 Os filisteus perseguiram Saul e seus filhos e mataram Jônatas, Abinadabe e Malquisua, filhos de Saul.
\par 3 Agravou-se muito a peleja contra Saul, os flecheiros o avistaram, e ele muito os temeu.
\par 4 Então, disse Saul ao seu escudeiro: Arranca a tua espada e atravessa-me com ela, para que, porventura, não venham estes incircuncisos e escarneçam de mim. Porém o seu escudeiro não o quis, porque temia muito; então, Saul tomou a espada e se lançou sobre ela.
\par 5 Vendo, pois, o seu escudeiro que Saul já era morto, também ele se lançou sobre a espada e morreu com ele.
\par 6 Assim, morreram Saul e seus três filhos; e toda a sua casa pereceu juntamente com ele.
\par 7 Vendo os homens de Israel que estavam no vale que os homens de Israel fugiram e que Saul e seus filhos estavam mortos, desampararam as cidades e fugiram; e vieram os filisteus e habitaram nelas.
\par 8 Sucedeu, pois, que, vindo os filisteus ao outro dia a despojar os mortos, acharam Saul e os seus filhos caídos no monte Gilboa.
\par 9 E os despojaram, tomaram a sua cabeça e as suas armas e enviaram mensageiros pela terra dos filisteus, em redor, a levar as boas-novas a seus ídolos e entre o povo.
\par 10 Puseram as armas de Saul no templo de seu deus, e a sua cabeça afixaram na casa de Dagom.
\par 11 Ouvindo, pois, toda a Jabes de Gileade tudo quanto os filisteus fizeram a Saul,
\par 12 então, todos os homens valentes se levantaram, e tomaram o corpo de Saul e os corpos dos filhos, e os trouxeram a Jabes; e sepultaram os seus ossos debaixo de um arvoredo, em Jabes, e jejuaram sete dias.
\par 13 Assim, morreu Saul por causa da sua transgressão cometida contra o SENHOR, por causa da palavra do SENHOR, que ele não guardara; e também porque interrogara e consultara uma necromante
\par 14 e não ao SENHOR, que, por isso, o matou e transferiu o reino a Davi, filho de Jessé.

\chapter{11}

\par 1 Então, todo o Israel se ajuntou a Davi, em Hebrom, dizendo: Somos do mesmo povo de que tu és.
\par 2 Outrora, sendo Saul ainda rei, eras tu que fazias saídas e entradas militares com Israel; também o SENHOR, teu Deus, te disse: Tu apascentarás o meu povo de Israel, serás chefe sobre o meu povo de Israel.
\par 3 Assim, pois, todos os anciãos de Israel vieram ter com o rei em Hebrom; e Davi fez com eles aliança em Hebrom, perante o SENHOR. Ungiram Davi rei sobre Israel, segundo a palavra do SENHOR por intermédio de Samuel.
\par 4 Partiu Davi e todo o Israel para Jerusalém, que é Jebus, porque ali estavam os jebuseus que habitavam naquela terra.
\par 5 Disseram os moradores de Jebus a Davi: Tu não entrarás aqui. Porém Davi tomou a fortaleza de Sião; esta é a Cidade de Davi.
\par 6 Porque disse Davi: Qualquer que primeiro ferir os jebuseus será chefe e comandante. Então, Joabe, filho de Zeruia, subiu primeiro e foi feito chefe.
\par 7 Assim, habitou Davi na fortaleza, pelo que se chamou a Cidade de Davi.
\par 8 E foi edificando a cidade em redor, desde Milo, completando o circuito; e Joabe renovou o resto da cidade.
\par 9 Ia Davi crescendo em poder cada vez mais, porque o SENHOR dos Exércitos era com ele.
\par 10 São estes os principais valentes de Davi, que o apoiaram valorosamente no seu reino, com todo o Israel, para o fazerem rei, segundo a palavra do SENHOR, no tocante a esse povo.
\par 11 Eis a lista dos valentes de Davi: Jasobeão, hacmonita, o principal dos trinta, o qual, brandindo a sua lança contra trezentos, de uma vez os feriu.
\par 12 Depois dele, Eleazar, filho de Dodô, o aoíta; ele estava entre os três valentes.
\par 13 Este se achou com Davi em Pas-Damim, quando se ajuntaram ali os filisteus à peleja, onde havia um pedaço de terra cheio de cevada; e o povo fugiu de diante dos filisteus.
\par 14 Puseram-se no meio daquele terreno, e o defenderam, e feriram os filisteus; e o SENHOR efetuou grande livramento.
\par 15 Três dos trinta cabeças desceram à penha, indo ter com Davi à caverna de Adulão; e o exército dos filisteus se acampara no vale dos Refains.
\par 16 Davi estava na fortaleza, e a guarnição dos filisteus, em Belém.
\par 17 Suspirou Davi e disse: Quem me dera beber água do poço que está junto à porta de Belém!
\par 18 Então, aqueles três romperam pelo acampamento dos filisteus, e tiraram água do poço junto à porta de Belém, e tomaram-na, e a levaram a Davi; ele não a quis beber, mas a derramou como libação ao SENHOR.
\par 19 E disse: Longe de mim, ó meu Deus, fazer tal coisa; beberia eu o sangue dos homens que lá foram com perigo de sua vida? Pois, com perigo de sua vida, a trouxeram. De maneira que não a quis beber. São essas as coisas que fizeram os três valentes.
\par 20 Também Abisai, irmão de Joabe, era cabeça dos trinta, o qual, brandindo a sua lança contra trezentos, os feriu; e tinha nome entre os primeiros três.
\par 21 Era ele mais nobre do que os trinta e era o cabeça deles; contudo, aos primeiros três não chegou.
\par 22 Também Benaia, filho de Joiada, era homem valente de Cabzeel e grande em obras; feriu ele dois heróis de Moabe. Desceu numa cova e nela matou um leão no tempo da neve.
\par 23 Matou também um egípcio, homem da estatura de cinco côvados; o egípcio trazia na mão uma lança como o eixo do tecelão, mas Benaia o atacou com um cajado, arrancou-lhe da mão a lança e com ela o matou.
\par 24 Estas coisas fez Benaia, filho de Joiada, pelo que teve nome entre os primeiros três valentes.
\par 25 Era mais nobre do que os trinta, porém aos três primeiros não chegou, e Davi o pôs sobre a sua guarda.
\par 26 Foram os heróis dos exércitos: Asael, irmão de Joabe, Elanã, filho de Dodô, de Belém;
\par 27 Samote, harorita; Heles, pelonita;
\par 28 Ira, filho de Iques, tecoíta; Abiezer, anatotita;
\par 29 Sibecai, husatita; Ilai, aoíta;
\par 30 Maarai, netofatita; Helede, filho de Baaná, netofatita;
\par 31 Itai, filho de Ribai, de Gibeá, dos filhos de Benjamim; Benaia, piratonita;
\par 32 Hurai, do ribeiro de Gaás; Abiel, arbatita;
\par 33 Azmavete, baarumita; Eliaba, saalbonita;
\par 34 Benê-Hasém, gizonita; Jônatas, filho de Sage, hararita;
\par 35 Aião, filho de Sacar, hararita; Elifal, filho de Ur;
\par 36 Héfer, mequeratita; Aías, pelonita;
\par 37 Hezro, carmelita; Naarai, filho de Ezbai;
\par 38 Joel, irmão de Natã; Mibar, filho de Hagri;
\par 39 Zeleque, amonita; Naarai, beerotita, o que trazia as armas de Joabe, filho de Zeruia;
\par 40 Ira, o itrita; Garebe, itrita;
\par 41 Urias, heteu; Zabade, filho de Alai;
\par 42 Adina, filho de Siza, rubenita, chefe dos rubenitas, e com ele trinta;
\par 43 Hanã, filho de Maaca; Josafá, mitenita;
\par 44 Uzias, asteratita, Sama e Jeiel, filhos de Hotão, aroerita;
\par 45 Jediael, filho de Sinri, e Joá, seu irmão, tizita;
\par 46 Eliel, maavita, Jeribai e Josavias, filhos de Elnaão; Itma, moabita;
\par 47 Eliel, Obede e Jaasiel, de Zoba.

\chapter{12}

\par 1 São estes os que vieram a Davi, a Ziclague, quando fugitivo de Saul, filho de Quis; e eram dos valentes que o ajudavam na guerra.
\par 2 Tinham por arma o arco e usavam tanto da mão direita como da esquerda em arremessar pedras com fundas e em atirar flechas com o arco. Eram dos irmãos de Saul, da tribo de Benjamim:
\par 3 Aiezer, o chefe, e Joás, filhos de Semaá, o gibeatita; Jeziel e Pelete, filhos de Azmavete; Beraca e Jeú, o anatotita;
\par 4 Ismaías, o gibeonita, valente entre os trinta e cabeça deles; Jeremias, Jaaziel, Joanã e Jozabade, o gederatita;
\par 5 Eluzai, Jerimote, Bealias, Semarias e Sefatias, o harufita;
\par 6 Elcana, Issias, Azarel, Joezer e Jasobeão, os coreítas;
\par 7 Joela, Zebadias, filhos de Jeroão, de Gedor.
\par 8 Dos gaditas passaram-se para Davi à fortaleza no deserto, homens valentes, homens de guerra para pelejar, armados de escudo e lança; seu rosto era como de leões, e eram eles ligeiros como gazelas sobre os montes:
\par 9 Ézer, o cabeça, Obadias, o segundo, Eliabe, o terceiro,
\par 10 Mismana, o quarto, Jeremias, o quinto,
\par 11 Atai, o sexto, Eliel, o sétimo,
\par 12 Joanã, o oitavo, Elzabade, o nono,
\par 13 Jeremias, o décimo, Macbanai, o undécimo;
\par 14 estes, dos filhos de Gade, foram capitães do exército; o menor valia por cem homens, e o maior, por mil.
\par 15 São estes os que passaram o Jordão no primeiro mês, quando ele transbordava por todas as suas ribanceiras, e puseram em fuga a todos os que habitavam nos vales, tanto no oriente como no ocidente.
\par 16 Também vieram alguns dos filhos de Benjamim e de Judá a Davi, à fortaleza.
\par 17 Davi lhes saiu ao encontro e lhes falou, dizendo: Se vós vindes a mim pacificamente e para me ajudar, o meu coração se unirá convosco; porém, se é para me entregardes aos meus adversários, não havendo maldade em mim, o Deus de nossos pais o veja e o repreenda.
\par 18 Então, entrou o Espírito em Amasai, cabeça de trinta, e disse: Nós somos teus, ó Davi, e contigo estamos, ó filho de Jessé! Paz, paz seja contigo! E paz com os que te ajudam! Porque o teu Deus te ajuda. Davi os recebeu e os fez capitães de tropas.
\par 19 Também de Manassés alguns se passaram para Davi, quando veio com os filisteus para a batalha contra Saul, mas não ajudou os filisteus, porque os príncipes destes, depois de se aconselharem, o despediram; pois diziam: À custa de nossa cabeça, passará a Saul, seu senhor.
\par 20 Voltando ele, pois, a Ziclague, passaram-se para ele, de Manassés, Adna, Jozabade, Jediael, Micael, Jozabade, Eliú e Ziletai, chefes de milhares dos de Manassés.
\par 21 Estes ajudaram Davi contra aquela tropa, porque todos eles eram homens valentes e capitães no exército.
\par 22 Porque, naquele tempo, dia após dia, vinham a Davi para o ajudar, até que se fez um grande exército, como exército de Deus.
\par 23 Ora, este é o número dos homens armados para a peleja, que vieram a Davi, em Hebrom, para lhe transferirem o reino de Saul, segundo a palavra do SENHOR:
\par 24 dos filhos de Judá, que traziam escudo e lança, seis mil e oitocentos, armados para a peleja;
\par 25 dos filhos de Simeão, homens valentes para a peleja, sete mil e cem;
\par 26 dos filhos de Levi, quatro mil e seiscentos;
\par 27 Joiada era o chefe da casa de Arão, e com ele vieram três mil e setecentos;
\par 28 Zadoque, sendo ainda jovem, homem valente, trouxe vinte e dois príncipes de sua casa paterna;
\par 29 dos filhos de Benjamim, irmãos de Saul, vieram três mil; porque até então havia ainda muitos deles que eram pela casa de Saul;
\par 30 dos filhos de Efraim, vinte mil e oitocentos homens valentes e de renome em casa de seus pais;
\par 31 da meia tribo de Manassés, dezoito mil, que foram apontados nominalmente para vir a fazer rei a Davi;
\par 32 dos filhos de Issacar, conhecedores da época, para saberem o que Israel devia fazer, duzentos chefes e todos os seus irmãos sob suas ordens;
\par 33 de Zebulom, dos capazes para sair à guerra, providos com todas as armas de guerra, cinqüenta mil, destros para ordenar uma batalha com ânimo resoluto;
\par 34 de Naftali, mil capitães, e, com eles, trinta e sete mil com escudo e lança;
\par 35 dos danitas, providos para a peleja, vinte e oito mil e seiscentos;
\par 36 de Aser, dos capazes para sair à guerra e prontos para a batalha, quarenta mil;
\par 37 do lado dalém do Jordão, dos rubenitas e gaditas e da meia tribo de Manassés, providos de toda sorte de instrumentos de guerra, cento e vinte mil.
\par 38 Todos estes homens de guerra, postos em ordem de batalha, vieram a Hebrom, resolvidos a fazer Davi rei sobre todo o Israel; também todo o resto de Israel era unânime no propósito de fazer a Davi rei.
\par 39 Estiveram ali com Davi três dias, comendo e bebendo; porque seus irmãos lhes tinham feito provisões.
\par 40 E também seus vizinhos de mais perto, até Issacar, Zebulom e Naftali, trouxeram pão sobre jumentos, sobre camelos, sobre mulos e sobre bois, provisões de farinha, e pastas de figos, e cachos de passas, e vinho, e azeite, e bois, e gado miúdo em abundância; porque havia regozijo em Israel.

\chapter{13}

\par 1 Consultou Davi os capitães de mil, e os de cem, e todos os príncipes;
\par 2 e disse a toda a congregação de Israel: Se bem vos parece, e se vem isso do SENHOR, nosso Deus, enviemos depressa mensageiros a todos os nossos outros irmãos em todas as terras de Israel, e aos sacerdotes, e aos levitas com eles nas cidades e nos seus arredores, para que se reúnam conosco;
\par 3 tornemos a trazer para nós a arca do nosso Deus; porque nos dias de Saul não nos valemos dela.
\par 4 Então, toda a congregação concordou em que assim se fizesse; porque isso pareceu justo aos olhos de todo o povo.
\par 5 Reuniu, pois, Davi a todo o Israel, desde Sior do Egito até à entrada de Hamate, para trazer a arca de Deus de Quiriate-Jearim.
\par 6 Então, Davi, com todo o Israel, subiu a Baalá, isto é, a Quiriate-Jearim, que está em Judá, para fazer subir dali a arca de Deus, diante da qual é invocado o nome do SENHOR, que se assenta acima dos querubins.
\par 7 Puseram a arca de Deus num carro novo e a levaram da casa de Abinadabe; e Uzá e Aiô guiavam o carro.
\par 8 Davi e todo o Israel alegravam-se perante Deus, com todo o seu empenho; em cânticos, com harpas, com alaúdes, com tamboris, com címbalos e com trombetas.
\par 9 Quando chegaram à eira de Quidom, estendeu Uzá a mão à arca para a segurar, porque os bois tropeçaram.
\par 10 Então, a ira do SENHOR se acendeu contra Uzá e o feriu, por ter estendido a mão à arca; e morreu ali perante Deus.
\par 11 Desgostou-se Davi, porque o SENHOR irrompera contra Uzá; pelo que chamou àquele lugar Perez-Uzá, até ao dia de hoje.
\par 12 Temeu Davi a Deus, naquele dia, e disse: Como trarei a mim a arca de Deus?
\par 13 Pelo que Davi não trouxe a arca para si, para a Cidade de Davi; mas a fez levar à casa de Obede-Edom, o geteu.
\par 14 Assim, ficou a arca de Deus com a família de Obede-Edom, três meses em sua casa; e o SENHOR abençoou a casa de Obede-Edom e tudo o que ele tinha.

\chapter{14}

\par 1 Então, Hirão, rei de Tiro, mandou mensageiros a Davi, e madeira de cedro, e pedreiros, e carpinteiros, para lhe edificarem uma casa.
\par 2 Reconheceu Davi que o SENHOR o confirmara rei sobre Israel; porque, por amor do seu povo de Israel, o seu reino se tinha exaltado muito.
\par 3 Davi tomou ainda mais mulheres em Jerusalém; e gerou ainda mais filhos e filhas.
\par 4 São estes os nomes dos filhos que teve em Jerusalém: Samua, Sobabe, Natã, Salomão,
\par 5 Ibar, Elisua, Elpelete,
\par 6 Nogá, Nefegue, Jafia,
\par 7 Elisama, Beeliada e Elifelete.
\par 8 Ouvindo, pois, os filisteus que Davi fora ungido rei sobre todo o Israel, subiram todos para prender Davi; ouvindo-o Davi, saiu contra eles.
\par 9 Mas vieram os filisteus e investiram contra ele no vale dos Refains.
\par 10 Então, Davi consultou a Deus, dizendo: Subirei contra os filisteus? Entregar-mos-ás nas mãos? Respondeu-lhe o SENHOR: Sobe, porque os entregarei nas tuas mãos.
\par 11 Subindo Davi a Baal-Perazim, ali os derrotou; e disse: Deus, por meu intermédio, rompeu as fileiras inimigas diante de mim, como quem rompe águas. Por isso, chamaram o nome daquele lugar Baal-Perazim.
\par 12 Ali, deixaram os seus deuses; e ordenou Davi que se queimassem.
\par 13 Porém os filisteus tornaram e fizeram uma investida no vale.
\par 14 De novo, Davi consultou a Deus, e este lhe respondeu: Não subirás após eles; mas rodeia por detrás deles e ataca-os por defronte das amoreiras;
\par 15 e há de ser que, ouvindo tu um estrondo de marcha pelas copas das amoreiras, então, sai à peleja; porque Deus saiu adiante de ti a ferir o exército dos filisteus.
\par 16 Fez Davi como Deus lhe ordenara; e feriu o exército dos filisteus desde Gibeão até Gezer.
\par 17 Assim se espalhou o renome de Davi por todas aquelas terras; pois o SENHOR o fez temível a todas aquelas gentes.

\chapter{15}

\par 1 Fez também Davi casas para si mesmo, na Cidade de Davi; e preparou um lugar para a arca de Deus e lhe armou uma tenda.
\par 2 Então, disse Davi: Ninguém pode levar a arca de Deus, senão os levitas; porque o SENHOR os elegeu, para levarem a arca de Deus e o servirem para sempre.
\par 3 Davi reuniu a todo o Israel em Jerusalém, para fazerem subir a arca do SENHOR ao seu lugar, que lhe tinha preparado.
\par 4 Reuniu Davi os filhos de Arão e os levitas:
\par 5 dos filhos de Coate: Uriel, o chefe, e seus irmãos, cento e vinte;
\par 6 dos filhos de Merari: Asaías, o chefe, e seus irmãos, duzentos e vinte;
\par 7 dos filhos de Gérson: Joel, o chefe, e seus irmãos, cento e trinta;
\par 8 dos filhos de Elisafã: Semaías, o chefe, e seus irmãos, duzentos;
\par 9 dos filhos de Hebrom: Eliel, o chefe, e seus irmãos, oitenta;
\par 10 dos filhos de Uziel: Aminadabe, o chefe, e seus irmãos, cento e doze.
\par 11 Chamou Davi os sacerdotes Zadoque e Abiatar e os levitas Uriel, Asaías, Joel, Semaías, Eliel e Aminadabe
\par 12 e lhes disse: Vós sois os cabeças das famílias dos levitas; santificai-vos, vós e vossos irmãos, para que façais subir a arca do SENHOR, Deus de Israel, ao lugar que lhe preparei.
\par 13 Pois, visto que não a levastes na primeira vez, o SENHOR, nosso Deus, irrompeu contra nós, porque, então, não o buscamos, segundo nos fora ordenado.
\par 14 Santificaram-se, pois, os sacerdotes e levitas, para fazerem subir a arca do SENHOR, Deus de Israel.
\par 15 Os filhos dos levitas trouxeram a arca de Deus aos ombros pelas varas que nela estavam, como Moisés tinha ordenado, segundo a palavra do SENHOR.
\par 16 Disse Davi aos chefes dos levitas que constituíssem a seus irmãos, os cantores, para que, com instrumentos músicos, com alaúdes, harpas e címbalos se fizessem ouvir e levantassem a voz com alegria.
\par 17 Designaram, pois, os levitas Hemã, filho de Joel; e dos irmãos dele, Asafe, filho de Berequias; e dos filhos de Merari, irmãos deles, Etã, filho de Cusaías.
\par 18 E com eles a seus irmãos da segunda ordem: Zacarias, Bene, Jaaziel, Semiramote, Jeiel, Uni, Eliabe, Benaia, Maaséias, Matitias, Elifeleu e Micnéias e os porteiros Obede-Edom e Jeiel.
\par 19 Assim, os cantores Hemã, Asafe e Etã se faziam ouvir com címbalos de bronze;
\par 20 Zacarias, Aziel, Semiramote, Jeiel, Uni, Eliabe, Maaséias e Benaia, com alaúdes, em voz de soprano;
\par 21 Matitias, Elifeleu, Micnéias, Obede-Edom, Jeiel e Azazias, com harpas, em tom de oitava, para conduzir o canto.
\par 22 Quenanias, chefe dos levitas músicos, tinha o encargo de dirigir o canto, porque era perito nisso.
\par 23 Berequias e Elcana eram porteiros da arca.
\par 24 Sebanias, Josafá, Natanael, Amasai, Zacarias, Benaia e Eliézer, os sacerdotes, tocavam as trombetas perante a arca de Deus; Obede-Edom e Jeías eram porteiros da arca.
\par 25 Foram Davi, e os anciãos de Israel, e os capitães de milhares, para fazerem subir, com alegria, a arca da Aliança do SENHOR, da casa de Obede-Edom.
\par 26 Tendo Deus ajudado os levitas que levavam a arca da Aliança do SENHOR, ofereceram em sacrifício sete novilhos e sete carneiros.
\par 27 Davi ia vestido de um manto de linho fino, como também todos os levitas que levavam a arca, e os cantores, e Quenanias, chefe dos que levavam a arca e dos cantores; Davi vestia também uma estola sacerdotal de linho.
\par 28 Assim, todo o Israel fez subir com júbilo a arca da Aliança do SENHOR, ao som de clarins, de trombetas e de címbalos, fazendo ressoar alaúdes e harpas.
\par 29 Ao entrar a arca da Aliança do SENHOR na Cidade de Davi, Mical, filha de Saul, estava olhando pela janela e, vendo ao rei Davi dançando e folgando, o desprezou no seu coração.

\chapter{16}

\par 1 Introduziram, pois, a arca de Deus e a puseram no meio da tenda que lhe armara Davi; e trouxeram holocaustos e ofertas pacíficas perante Deus.
\par 2 Tendo Davi acabado de trazer os holocaustos e ofertas pacíficas, abençoou o povo em nome do SENHOR.
\par 3 E repartiu a todos em Israel, tanto os homens como as mulheres, a cada um, um bolo de pão, um bom pedaço de carne e passas.
\par 4 Designou dentre os levitas os que haviam de ministrar diante da arca do SENHOR, e celebrar, e louvar, e exaltar o SENHOR, Deus de Israel, a saber,
\par 5 Asafe, o chefe, Zacarias, o segundo, e depois Jeiel, Semiramote, Jeiel, Matitias, Eliabe, Benaia, Obede-Edom e Jeiel, com alaúdes e harpas; e Asafe fazia ressoar os címbalos.
\par 6 Os sacerdotes Benaia e Jaaziel estavam continuamente com trombetas, perante a arca da Aliança de Deus.
\par 7 Naquele dia, foi que Davi encarregou, pela primeira vez, a Asafe e a seus irmãos de celebrarem com hinos o SENHOR.
\par 8 Rendei graças ao SENHOR, invocai o seu nome, fazei conhecidos, entre os povos, os seus feitos.
\par 9 Cantai-lhe, cantai-lhe salmos; narrai todas as suas maravilhas.
\par 10 Gloriai-vos no seu santo nome; alegre-se o coração dos que buscam o SENHOR.
\par 11 Buscai o SENHOR e o seu poder, buscai perpetuamente a sua presença.
\par 12 Lembrai-vos das maravilhas que fez, dos seus prodígios e dos juízos dos seus lábios,
\par 13 vós, descendentes de Israel, seu servo, vós, filhos de Jacó, seus escolhidos.
\par 14 Ele é o SENHOR, nosso Deus; os seus juízos permeiam toda a terra.
\par 15 Lembra-se perpetuamente da sua aliança, da palavra que empenhou para mil gerações;
\par 16 da aliança que fez com Abraão e do juramento que fez a Isaque;
\par 17 o qual confirmou a Jacó por decreto e a Israel, por aliança perpétua,
\par 18 dizendo: Dar-vos-ei a terra de Canaã como quinhão da vossa herança.
\par 19 Então, eram eles em pequeno número, pouquíssimos e forasteiros nela;
\par 20 andavam de nação em nação, de um reino para um povo.
\par 21 A ninguém permitiu que os oprimisse; antes, por amor deles, repreendeu a reis,
\par 22 dizendo: Não toqueis nos meus ungidos, nem maltrateis os meus profetas.
\par 23 Cantai ao SENHOR, todas as terras; proclamai a sua salvação, dia após dia.
\par 24 Anunciai entre as nações a sua glória, entre todos os povos, as suas maravilhas,
\par 25 porque grande é o SENHOR e mui digno de ser louvado, temível mais do que todos os deuses.
\par 26 Porque todos os deuses dos povos são ídolos; o SENHOR, porém, fez os céus.
\par 27 Glória e majestade estão diante dele, força e formosura, no seu santuário.
\par 28 Tributai ao SENHOR, ó famílias dos povos, tributai ao SENHOR glória e força.
\par 29 Tributai ao SENHOR a glória devida ao seu nome; trazei oferendas e entrai nos seus átrios; adorai o SENHOR na beleza da sua santidade.
\par 30 Tremei diante dele, todas as terras, pois ele firmou o mundo para que não se abale.
\par 31 Alegrem-se os céus, e a terra exulte; diga-se entre as nações: Reina o SENHOR.
\par 32 Ruja o mar e a sua plenitude; folgue o campo e tudo o que nele há.
\par 33 Regozijem-se as árvores do bosque na presença do SENHOR, porque vem a julgar a terra.
\par 34 Rendei graças ao SENHOR, porque ele é bom; porque a sua misericórdia dura para sempre.
\par 35 E dizei: Salva-nos, ó Deus da nossa salvação, ajunta-nos e livra-nos das nações, para que rendamos graças ao teu santo nome e nos gloriemos no teu louvor.
\par 36 Bendito seja o SENHOR, Deus de Israel, desde a eternidade até a eternidade. E todo o povo disse: Amém! E louvou ao SENHOR.
\par 37 Então, Davi deixou ali diante da arca da Aliança do SENHOR a Asafe e a seus irmãos, para ministrarem continuamente perante ela, segundo se ordenara para cada dia;
\par 38 também deixou a Obede-Edom com seus irmãos, em número de sessenta e oito; a Obede-Edom, filho de Jedutum, e a Hosa, para serem porteiros;
\par 39 e deixou a Zadoque, o sacerdote, e aos sacerdotes, seus irmãos, diante do tabernáculo do SENHOR, num lugar alto de Gibeão,
\par 40 para oferecerem continuamente ao SENHOR os holocaustos sobre o altar dos holocaustos, pela manhã e à tarde; e isto segundo tudo o que está escrito na Lei que o SENHOR ordenara a Israel.
\par 41 E com eles deixou a Hemã, a Jedutum e os mais escolhidos, que foram nominalmente designados para louvarem o SENHOR, porque a sua misericórdia dura para sempre.
\par 42 Com eles, pois, estavam Hemã e Jedutum, que faziam ressoar trombetas, e címbalos, e instrumentos de música de Deus; os filhos de Jedutum eram porteiros.
\par 43 Então, se retirou todo o povo, cada um para sua casa; e tornou Davi, para abençoar a sua casa.

\chapter{17}

\par 1 Sucedeu que, habitando Davi em sua própria casa, disse ao profeta Natã: Eis que moro em casa de cedros, mas a arca da Aliança do SENHOR se acha numa tenda.
\par 2 Então, Natã disse a Davi: Faze tudo quanto está no teu coração, porque Deus é contigo.
\par 3 Porém, naquela mesma noite, veio a palavra do SENHOR a Natã, dizendo:
\par 4 Vai e dize a meu servo Davi: Assim diz o SENHOR: Tu não edificarás casa para minha habitação;
\par 5 porque em casa nenhuma habitei, desde o dia que fiz subir a Israel até ao dia de hoje; mas tenho andado de tenda em tenda, de tabernáculo em tabernáculo.
\par 6 Em todo lugar em que andei com todo o Israel, falei, acaso, alguma palavra com algum dos seus juízes, a quem mandei apascentar o meu povo, dizendo: Por que não me edificais uma casa de cedro?
\par 7 Agora, pois, assim dirás ao meu servo Davi: Assim diz o SENHOR dos Exércitos: Tomei-te da malhada e de detrás das ovelhas, para que fosses príncipe sobre o meu povo de Israel.
\par 8 E fui contigo, por onde quer que andaste, eliminei os teus inimigos de diante de ti e fiz grande o teu nome, como só os grandes têm na terra.
\par 9 Prepararei lugar para o meu povo de Israel e o plantarei para que habite no seu lugar e não mais seja perturbado; e jamais os filhos da perversidade o oprimam, como dantes,
\par 10 desde o dia em que mandei houvesse juízes sobre o meu povo de Israel; porém abati todos os teus inimigos e também te fiz saber que o SENHOR te edificaria uma casa.
\par 11 Há de ser que, quando teus dias se cumprirem, e tiveres de ir para junto de teus pais, então, farei levantar depois de ti o teu descendente, que será dos teus filhos, e estabelecerei o seu reino.
\par 12 Esse me edificará casa; e eu estabelecerei o seu trono para sempre.
\par 13 Eu lhe serei por pai, e ele me será por filho; a minha misericórdia não apartarei dele, como a retirei daquele que foi antes de ti.
\par 14 Mas o confirmarei na minha casa e no meu reino para sempre, e o seu trono será estabelecido para sempre.
\par 15 Segundo todas estas palavras e conforme toda esta visão, assim falou Natã a Davi.
\par 16 Então, entrou o rei Davi na Casa do SENHOR, ficou perante ele e disse: Quem sou eu, SENHOR Deus, e qual é a minha casa, para que me tenhas trazido até aqui?
\par 17 Foi isso ainda pouco aos teus olhos, ó Deus, de maneira que também falaste a respeito da casa de teu servo para tempos distantes; e me trataste como se eu fosse homem ilustre, ó SENHOR Deus.
\par 18 Que mais ainda te poderá dizer Davi acerca das honras feitas a teu servo? Pois tu conheces bem teu servo.
\par 19 Ó SENHOR, por amor de teu servo e segundo o teu coração, fizeste toda esta grandeza, para tornar notórias todas estas grandes coisas!
\par 20 SENHOR, ninguém há semelhante a ti, e não há outro Deus além de ti, segundo tudo o que nós mesmos temos ouvido.
\par 21 Quem há como o teu povo de Israel, gente única na terra, a quem tu, ó Deus, foste resgatar para ser teu povo e fazer a ti mesmo um nome, com estas grandes e tremendas coisas, desterrando as nações de diante do teu povo, que remiste do Egito?
\par 22 Estabeleceste a teu povo de Israel por teu povo, para sempre, e tu, ó SENHOR, te fizeste o seu Deus.
\par 23 Agora, pois, ó SENHOR, a palavra que disseste acerca de teu servo e acerca da sua casa, seja estabelecida para sempre; e faze como falaste.
\par 24 Estabeleça-se, e seja para sempre engrandecido o teu nome, e diga-se: O SENHOR dos Exércitos é o Deus de Israel, é Deus para Israel; e a casa de Davi, teu servo, será estabelecida diante de ti.
\par 25 Pois tu, Deus meu, fizeste ao teu servo a revelação de que lhe edificarias casa. Por isso, o teu servo se animou para fazer-te esta oração.
\par 26 Agora, pois, ó SENHOR, tu mesmo és Deus e prometeste a teu servo este bem.
\par 27 Sê, pois, agora, servido de abençoar a casa de teu servo, a fim de permanecer para sempre diante de ti, pois tu, ó SENHOR, a abençoaste, e abençoada será para sempre.

\chapter{18}

\par 1 Depois disto, feriu Davi os filisteus e os humilhou; tomou a Gate e suas aldeias das mãos dos filisteus.
\par 2 Também derrotou os moabitas, e assim ficaram por servos de Davi e lhe pagavam tributo.
\par 3 Também Hadadezer, rei de Zoba, foi derrotado por Davi, até Hamate, quando aquele foi restabelecer o seu domínio sobre o rio Eufrates.
\par 4 Tomou-lhe Davi mil carros, sete mil cavaleiros e vinte mil homens de pé; Davi jarretou a todos os cavalos dos carros, menos para cem deles.
\par 5 Vieram os siros de Damasco a socorrer a Hadadezer, rei de Zoba; porém Davi matou dos siros vinte e dois mil homens.
\par 6 Davi pôs guarnições na Síria de Damasco, e os siros ficaram por servos de Davi e lhe pagavam tributo; e o SENHOR dava vitórias a Davi, por onde quer que ia.
\par 7 Tomou Davi os escudos de ouro que havia com os oficiais de Hadadezer e os trouxe a Jerusalém.
\par 8 Também de Tibate e de Cum, cidades de Hadadezer, tomou Davi mui grande quantidade de bronze, de que Salomão fez o mar de bronze, as colunas e os utensílios de bronze.
\par 9 Ouvindo Toú, rei de Hamate, que Davi derrotara a todo o exército de Hadadezer, rei de Zoba,
\par 10 mandou seu filho Hadorão ao rei Davi, para o saudar e congratular-se com ele por haver pelejado contra Hadadezer e por havê-lo ferido (porque Hadadezer fazia guerra a Toú). Hadorão trouxe consigo objetos de ouro, de prata e de bronze,
\par 11 os quais também o rei Davi consagrou ao SENHOR, juntamente com a prata e ouro que trouxera de todas as mais nações: de Edom, de Moabe, dos filhos de Amom, dos filisteus e de Amaleque.
\par 12 Também Abisai, filho de Zeruia, feriu a dezoito mil edomitas no vale do Sal.
\par 13 E pôs guarnições em Edom, e todos os edomitas ficaram por servos de Davi; e o SENHOR dava vitórias a Davi, por onde quer que ia.
\par 14 Reinou, pois, Davi sobre todo o Israel; julgava e fazia justiça a todo o seu povo.
\par 15 Joabe, filho de Zeruia, era comandante do exército; Josafá, filho de Ailude, era cronista.
\par 16 Zadoque, filho de Aitube, e Abimeleque, filho de Abiatar, eram sacerdotes; e Sausa, escrivão.
\par 17 Benaia, filho de Joiada, era o comandante da guarda real. Os filhos de Davi, porém, eram os primeiros ao lado do rei.

\chapter{19}

\par 1 Depois disto, morreu Naás, rei dos filhos de Amom; e seu filho reinou em seu lugar.
\par 2 Então, disse Davi: Usarei de bondade para com Hanum, filho de Naás, porque seu pai usou de bondade para comigo. Pelo que Davi enviou mensageiros para o consolar acerca de seu pai; e vieram os servos de Davi à terra dos filhos de Amom, a Hanum, para o consolarem.
\par 3 Disseram os príncipes dos filhos de Amom a Hanum: Pensas que, por te haver Davi mandado consoladores, está honrando a teu pai? Não vieram seus servos a ti para reconhecerem, destruírem e espiarem a terra?
\par 4 Tomou, então, Hanum os servos de Davi, e rapou-os, e lhes cortou metade das vestes até às nádegas, e os despediu.
\par 5 Foram-se alguns e avisaram a Davi acerca destes homens; então, enviou mensageiros a encontrá-los, porque estavam sobremaneira envergonhados. Mandou o rei dizer-lhes: Deixai-vos estar em Jericó, até que vos torne a crescer a barba; e, então, vinde.
\par 6 Vendo, pois, os filhos de Amom que se haviam tornado odiosos a Davi, então, Hanum e os filhos de Amom tomaram mil talentos de prata, para alugarem para si carros e cavaleiros da Mesopotâmia, e dos siros de Maaca, e de Zoba.
\par 7 Alugaram para si trinta e dois mil carros, o rei de Maaca e a sua gente, e eles vieram e se acamparam diante de Medeba; também os filhos de Amom se ajuntaram das suas cidades e vieram para a guerra.
\par 8 O que ouvindo Davi, enviou contra eles a Joabe com todo o exército dos valentes.
\par 9 Saíram os filhos de Amom e ordenaram a batalha à entrada da porta da cidade; porém os reis que vieram estavam à parte, no campo.
\par 10 Vendo, pois, Joabe que estava preparada contra ele a batalha, tanto pela frente como pela retaguarda, escolheu dentre todos o que havia de melhor em Israel e os formou em linha contra os siros;
\par 11 e o resto do povo entregou a Abisai, seu irmão, e puseram-se em linha contra os filhos de Amom.
\par 12 Disse Joabe: Se os siros forem mais fortes do que eu, tu me virás em socorro; e, se os filhos de Amom forem mais fortes do que tu, eu irei ao teu socorro.
\par 13 Sê forte, pois; pelejemos varonilmente pelo nosso povo e pelas cidades de nosso Deus; e faça o SENHOR o que bem lhe parecer.
\par 14 Então, avançou Joabe com o povo que estava com ele, e travaram peleja contra os siros; e estes fugiram de diante dele.
\par 15 Vendo os filhos de Amom que os siros fugiam, também eles fugiram de diante de Abisai, irmão de Joabe, e entraram na cidade; voltou Joabe dos filhos de Amom e tornou a Jerusalém.
\par 16 Vendo, pois, os siros que tinham sido desbaratados diante de Israel, enviaram mensageiros e fizeram sair os siros que habitavam do lado dalém do rio; Sofaque, capitão do exército de Hadadezer, marchava adiante deles.
\par 17 Informado Davi, ajuntou a todo o Israel, passou o Jordão, veio ter com eles e ordenou contra eles a batalha; e, tendo Davi ordenado a batalha contra os siros, pelejaram estes contra ele.
\par 18 Porém os siros fugiram de diante de Israel, e Davi matou dentre os siros os homens de sete mil carros e quarenta mil homens de pé; a Sofaque, chefe do exército, matou.
\par 19 Vendo, pois, os servos de Hadadezer que foram feridos diante de Israel, fizeram paz com Davi e o serviram; e os siros nunca mais quiseram socorrer aos filhos de Amom.

\chapter{20}

\par 1 Decorrido um ano, no tempo em que os reis costumam sair para a guerra, Joabe levou o exército, destruiu a terra dos filhos de Amom, veio e sitiou a Rabá; porém Davi ficou em Jerusalém; e Joabe feriu a Rabá e a destruiu.
\par 2 Tirou Davi a coroa da cabeça do seu rei e verificou que tinha o peso de um talento de ouro e que havia nela pedras preciosas; e foi posta na cabeça de Davi; e da cidade levou mui grande despojo.
\par 3 Também levou o povo que estava nela e o fez passar à serra, e a picaretas de ferro, e a machados; assim fez Davi a todas as cidades dos filhos de Amom. Voltou Davi, com todo o povo, para Jerusalém.
\par 4 Depois disto, houve guerra em Gezer contra os filisteus; então, Sibecai, o husatita, feriu a Sipai, que era descendente dos gigantes; e os filisteus foram subjugados.
\par 5 Houve ainda outra guerra contra os filisteus; e Elanã, filho de Jair, feriu a Lami, irmão de Golias, o geteu, cuja lança tinha a haste como eixo de tecelão.
\par 6 Houve ainda outra guerra em Gate; havia ali um homem de grande estatura, tinha vinte e quatro dedos, seis em cada mão e seis em cada pé; também este descendia dos gigantes.
\par 7 Quando ele injuriava a Israel, Jônatas, filho de Siméia, irmão de Davi, o feriu.
\par 8 Estes nasceram dos gigantes em Gate; e caíram pela mão de Davi e pela mão de seus homens.

\chapter{21}

\par 1 Então, Satanás se levantou contra Israel e incitou a Davi a levantar o censo de Israel.
\par 2 Disse Davi a Joabe e aos chefes do povo: Ide, levantai o censo de Israel, desde Berseba até Dã; e trazei-me a apuração para que eu saiba o seu número.
\par 3 Então, disse Joabe: Multiplique o SENHOR, teu Deus, a este povo cem vezes mais; porventura, ó rei, meu senhor, não são todos servos de meu senhor? Por que requer isso o meu senhor? Por que trazer, assim, culpa sobre Israel?
\par 4 Porém a palavra do rei prevaleceu contra Joabe; pelo que saiu Joabe e percorreu todo o Israel; então, voltou para Jerusalém.
\par 5 Deu Joabe a Davi o recenseamento do povo; havia em Israel um milhão e cem mil homens que puxavam da espada; e em Judá eram quatrocentos e setenta mil homens que puxavam da espada.
\par 6 Porém os de Levi e Benjamim não foram contados entre eles, porque a ordem do rei foi abominável a Joabe.
\par 7 Tudo isto desagradou a Deus, pelo que feriu a Israel.
\par 8 Então, disse Davi a Deus: Muito pequei em fazer tal coisa; porém, agora, peço-te que perdoes a iniqüidade de teu servo, porque procedi mui loucamente.
\par 9 Falou, pois, o SENHOR a Gade, o vidente de Davi, dizendo:
\par 10 Vai e dize a Davi: Assim diz o SENHOR: Três coisas te ofereço; escolhe uma delas, para que ta faça.
\par 11 Veio, pois, Gade a Davi e lhe disse: Assim diz o SENHOR: Escolhe o que queres:
\par 12 ou três anos de fome, ou que por três meses sejas consumido diante dos teus adversários, e a espada de teus inimigos te alcance, ou que por três dias a espada do SENHOR, isto é, a peste na terra, e o Anjo do SENHOR causem destruição em todos os territórios de Israel; vê, pois, agora, que resposta hei de dar ao que me enviou.
\par 13 Então, disse Davi a Gade: Estou em grande angústia; caia eu, pois, nas mãos do SENHOR, porque são muitíssimas as suas misericórdias, mas nas mãos dos homens não caia eu.
\par 14 Então, enviou o SENHOR a peste a Israel; e caíram de Israel setenta mil homens.
\par 15 Enviou Deus um anjo a Jerusalém, para a destruir; ao destruí-la, olhou o SENHOR, e se arrependeu do mal, e disse ao anjo destruidor: Basta, retira, agora, a mão. O Anjo do SENHOR estava junto à eira de Ornã, o jebuseu.
\par 16 Levantando Davi os olhos, viu o Anjo do SENHOR, que estava entre a terra e o céu, com a espada desembainhada na mão estendida contra Jerusalém; então, Davi e os anciãos, cobertos de panos de saco, se prostraram com o rosto em terra.
\par 17 Disse Davi a Deus: Não sou eu o que disse que se contasse o povo? Eu é que pequei, eu é que fiz muito mal; porém estas ovelhas que fizeram? Ah! SENHOR, meu Deus, seja, pois, a tua mão contra mim e contra a casa de meu pai e não para castigo do teu povo.
\par 18 Então, o Anjo do SENHOR disse a Gade que mandasse Davi subir para levantar um altar ao SENHOR, na eira de Ornã, o jebuseu.
\par 19 Subiu, pois, Davi, segundo a palavra de Gade, que falara em nome do SENHOR.
\par 20 Virando-se Ornã, viu o Anjo; e esconderam-se seus quatro filhos que estavam com ele. Ora, Ornã estava debulhando trigo.
\par 21 Quando Davi vinha chegando a Ornã, este olhou, e o viu e, saindo da eira, se inclinou diante de Davi, com o rosto em terra.
\par 22 Disse Davi a Ornã: Dá-me este lugar da eira a fim de edificar nele um altar ao SENHOR, para que cesse a praga de sobre o povo; dá-mo pelo seu devido valor.
\par 23 Então, disse Ornã a Davi: Tome-a o rei, meu senhor, para si e faça dela o que bem lhe parecer; eis que dou os bois para o holocausto, e os trilhos, para a lenha, e o trigo, para oferta de manjares; dou tudo.
\par 24 Tornou o rei Davi a Ornã: Não; antes, pelo seu inteiro valor a quero comprar; porque não tomarei o que é teu para o SENHOR, nem oferecerei holocausto que não me custe nada.
\par 25 Davi deu a Ornã por aquele lugar a soma de seiscentos siclos de ouro.
\par 26 Edificou ali um altar ao SENHOR, ofereceu nele holocaustos e sacrifícios pacíficos e invocou o SENHOR, o qual lhe respondeu com fogo do céu sobre o altar do holocausto.
\par 27 O SENHOR deu ordem ao Anjo, e ele meteu a sua espada na bainha.
\par 28 Vendo Davi, naquele mesmo tempo, que o SENHOR lhe respondera na eira de Ornã, o jebuseu, sacrificou ali.
\par 29 Porque o tabernáculo do SENHOR, que Moisés fizera no deserto, e o altar do holocausto estavam, naquele tempo, no alto de Gibeão.
\par 30 Davi não podia ir até lá para consultar a Deus, porque estava atemorizado por causa da espada do Anjo do SENHOR.

\chapter{22}

\par 1 Disse Davi: Aqui, se levantará a Casa do SENHOR Deus e o altar do holocausto para Israel.
\par 2 Deu ordem Davi para que fossem ajuntados os estrangeiros que estavam na terra de Israel; e encarregou pedreiros que preparassem pedras de cantaria para se edificar a Casa de Deus.
\par 3 Aparelhou Davi ferro em abundância, para os pregos das folhas das portas e para as junturas, como também bronze em abundância, que nem foi pesado.
\par 4 Madeira de cedro sem conta, porque os sidônios e tírios a traziam a Davi, em grande quantidade.
\par 5 Pois dizia Davi: Salomão, meu filho, ainda é moço e tenro, e a casa que se há de edificar para o SENHOR deve ser sobremodo magnificente, para nome e glória em todas as terras; providenciarei, pois, para ela o necessário; assim, o preparou Davi em abundância, antes de sua morte.
\par 6 Então, chamou a Salomão, seu filho, e lhe ordenou que edificasse casa ao SENHOR, Deus de Israel.
\par 7 Disse Davi a Salomão: Filho meu, tive intenção de edificar uma casa ao nome do SENHOR, meu Deus.
\par 8 Porém a mim me veio a palavra do SENHOR, dizendo: Tu derramaste sangue em abundância e fizeste grandes guerras; não edificarás casa ao meu nome, porquanto muito sangue tens derramado na terra, na minha presença.
\par 9 Eis que te nascerá um filho, que será homem sereno, porque lhe darei descanso de todos os seus inimigos em redor; portanto, Salomão será o seu nome; paz e tranqüilidade darei a Israel nos seus dias.
\par 10 Este edificará casa ao meu nome; ele me será por filho, e eu lhe serei por pai; estabelecerei para sempre o trono do seu reino sobre Israel.
\par 11 Agora, pois, meu filho, o SENHOR seja contigo, a fim de que prosperes e edifiques a Casa do SENHOR, teu Deus, como ele disse a teu respeito.
\par 12 Que o SENHOR te conceda prudência e entendimento, para que, quando regeres sobre Israel, guardes a lei do SENHOR, teu Deus.
\par 13 Então, prosperarás, se cuidares em cumprir os estatutos e os juízos que o SENHOR ordenou a Moisés acerca de Israel; sê forte e corajoso, não temas, não te desalentes.
\par 14 Eis que, com penoso trabalho, preparei para a Casa do SENHOR cem mil talentos de ouro e um milhão de talentos de prata, e bronze e ferro em tal abundância, que nem foram pesados; também madeira e pedras preparei, cuja quantidade podes aumentar.
\par 15 Além disso, tens contigo trabalhadores em grande número, e canteiros, e pedreiros, e carpinteiros, e peritos em toda sorte de obra
\par 16 de ouro, e de prata, e também de bronze, e de ferro, que se não pode contar. Dispõe-te, pois, e faze a obra, e o SENHOR seja contigo!
\par 17 Davi deu ordem a todos os príncipes de Israel que ajudassem Salomão, seu filho, dizendo:
\par 18 Porventura, não está convosco o SENHOR, vosso Deus, e não vos deu paz por todos os lados? Pois entregou nas minhas mãos os moradores desta terra, a qual está sujeita perante o SENHOR e perante o seu povo.
\par 19 Disponde, pois, agora o coração e a alma para buscardes ao SENHOR, vosso Deus; disponde-vos e edificai o santuário do SENHOR Deus, para que a arca da Aliança do SENHOR e os utensílios sagrados de Deus sejam trazidos a esta casa, que se há de edificar ao nome do SENHOR.

\chapter{23}

\par 1 Sendo, pois, Davi já velho e farto de dias, constituiu a seu filho Salomão rei sobre Israel.
\par 2 Ajuntou todos os príncipes de Israel, como também os sacerdotes e levitas.
\par 3 Foram contados os levitas de trinta anos para cima; seu número, contados um por um, foi de trinta e oito mil homens.
\par 4 Destes, havia vinte e quatro mil para superintenderem a obra da Casa do SENHOR, seis mil oficiais e juízes,
\par 5 quatro mil porteiros e quatro mil para louvarem o SENHOR com os instrumentos que Davi fez para esse mister.
\par 6 Davi os repartiu por turnos, segundo os filhos de Levi: Gérson, Coate e Merari.
\par 7 Filhos de Gérson: Ladã e Simei.
\par 8 Filhos de Ladã: Jeiel, o chefe, Zetã e Joel, três.
\par 9 Filhos de Simei: Selomite, Haziel e Harã, três; estes foram os chefes das famílias de Ladã.
\par 10 Filhos de Simei: Jaate, Ziza, Jeús e Berias; estes foram os filhos de Simei, quatro.
\par 11 Jaate era o chefe, Ziza, o segundo; mas Jeús e Berias não tiveram muitos filhos; pelo que estes dois foram contados por uma só família.
\par 12 Filhos de Coate: Anrão, Isar, Hebrom e Uziel, quatro.
\par 13 Filhos de Anrão: Arão e Moisés; Arão foi separado para servir no Santo dos Santos, ele e seus filhos, perpetuamente, e para queimar incenso diante do SENHOR, para o servir e para dar a bênção em seu nome, eternamente.
\par 14 Quanto a Moisés, homem de Deus, seus filhos foram contados entre a tribo de Levi.
\par 15 Os filhos de Moisés: Gérson e Eliézer.
\par 16 Filho de Gérson: Sebuel, o chefe.
\par 17 Filho de Eliézer: Reabias, o chefe; e não teve outros; porém os filhos de Reabias se multiplicaram grandemente.
\par 18 Filhos de Isar: Selomite, o chefe.
\par 19 Filhos de Hebrom: Jerias, o chefe, Amarias, o segundo, Jaaziel, o terceiro, e Jecameão, o quarto.
\par 20 Filhos de Uziel: Mica, o chefe, e Issias, o segundo.
\par 21 Filhos de Merari: Mali e Musi; filhos de Mali: Eleazar e Quis.
\par 22 Morreu Eleazar e não teve filhos, porém filhas; e os filhos de Quis, seus irmãos, as desposaram.
\par 23 Os filhos de Musi: Mali, Éder e Jerimote, três.
\par 24 São estes os filhos de Levi, segundo as suas famílias e chefes delas, segundo foram contados nominalmente, um por um, encarregados do ministério da Casa do SENHOR, de vinte anos para cima.
\par 25 Porque disse Davi: O SENHOR, Deus de Israel, deu paz ao seu povo e habitará em Jerusalém para sempre.
\par 26 Assim, os levitas já não precisarão levar o tabernáculo e nenhum dos utensílios para o seu ministério.
\par 27 Porque, segundo as últimas palavras de Davi, foram contados os filhos de Levi de vinte anos para cima.
\par 28 O cargo deles era assistir os filhos de Arão no ministério da Casa do SENHOR, nos átrios e nas câmaras, na purificação de todas as coisas sagradas e na obra do ministério da Casa de Deus,
\par 29 a saber, os pães da proposição, a flor de farinha para a oferta de manjares, os coscorões asmos, as assadeiras, o tostado e toda sorte de peso e medida.
\par 30 Deviam estar presentes todas as manhãs para renderem graças ao SENHOR e o louvarem; e da mesma sorte, à tarde;
\par 31 e para cada oferecimento dos holocaustos do SENHOR, nos sábados, nas Festas da Lua Nova e nas festas fixas, perante o SENHOR, segundo o número determinado;
\par 32 e para que tivessem a seu cargo a tenda da congregação e o santuário e atendessem aos filhos de Arão, seus irmãos, no ministério da Casa do SENHOR.

\chapter{24}

\par 1 Quanto aos filhos de Arão, foram eles divididos por seus turnos. Filhos de Arão: Nadabe, Abiú, Eleazar e Itamar.
\par 2 Nadabe e Abiú morreram antes de seu pai e não tiveram filhos; Eleazar e Itamar oficiavam como sacerdotes.
\par 3 Davi, com Zadoque, dos filhos de Eleazar, e com Aimeleque, dos filhos de Itamar, os dividiu segundo os seus deveres no seu ministério.
\par 4 E achou-se que eram mais os filhos de Eleazar entre os chefes de famílias do que os filhos de Itamar, quando os dividiram; dos filhos de Eleazar, dezesseis chefes de famílias; dos filhos de Itamar, oito.
\par 5 Repartiram-nos por sortes, uns como os outros; porque havia príncipes do santuário e príncipes de Deus, tanto dos filhos de Eleazar como dos filhos de Itamar.
\par 6 Semaías, escrivão, filho de Natanael, levita, registrou-os na presença do rei, dos príncipes, do sacerdote Zadoque, de Aimeleque, filho de Abiatar, e dos cabeças das famílias dos sacerdotes e dos levitas; sendo escolhidas as famílias, por sorte, alternadamente, para Eleazar e para Itamar.
\par 7 Saiu a primeira sorte a Jeoiaribe; a segunda, a Jedaías;
\par 8 a terceira, a Harim; a quarta, a Seorim;
\par 9 a quinta, a Malquias; a sexta, a Miamim;
\par 10 a sétima, a Hacoz; a oitava, a Abias;
\par 11 a nona, a Jesua; a décima, a Secanias;
\par 12 a undécima, a Eliasibe; a duodécima, a Jaquim;
\par 13 a décima terceira, a Hupá; a décima quarta, a Jesebeabe;
\par 14 a décima quinta, a Bilga; a décima sexta, a Imer;
\par 15 a décima sétima, a Hezir; a décima oitava, a Hapises;
\par 16 a décima nona, a Petaías; a vigésima, a Jeezquel;
\par 17 a vigésima primeira, a Jaquim; a vigésima segunda, a Gamul;
\par 18 a vigésima terceira, a Delaías; a vigésima quarta, a Maazias.
\par 19 O ofício destes no seu ministério era entrar na Casa do SENHOR, segundo a maneira estabelecida por Arão, seu pai, como o SENHOR, Deus de Israel, lhe ordenara.
\par 20 Eis os chefes do restante dos filhos de Levi: dos filhos de Anrão, Subael; dos filhos de Subael, Jedias;
\par 21 dos filhos de Reabias, Issias, o chefe;
\par 22 dos isaritas, Selomite; dos filhos de Selomite, Jaate;
\par 23 dos filhos de Hebrom, Jerias, o primeiro, Amarias, o segundo, Jaaziel, o terceiro, Jecameão, o quarto;
\par 24 dos filhos de Uziel, Mica; dos filhos de Mica, Samir;
\par 25 o irmão de Mica, Issias; dos filhos de Issias, Zacarias;
\par 26 dos filhos de Merari, Mali e Musi; dos filhos de Jaazias, Beno;
\par 27 dos filhos de Merari, da parte de Jaazias: Beno, Soão, Zacur e Ibri;
\par 28 de Mali, Eleazar, que não teve filhos;
\par 29 dos filhos de Quis, Jerameel;
\par 30 dos filhos de Musi, Mali, Éder e Jerimote. Foram estes os filhos dos levitas, segundo as suas famílias.
\par 31 Também estes, tanto os chefes das famílias como os seus irmãos menores, como fizeram os outros seus irmãos, filhos de Arão, lançaram sortes na presença do rei Davi, de Zadoque, de Aimeleque e dos cabeças das famílias dos sacerdotes e dos levitas.

\chapter{25}

\par 1 Davi, juntamente com os chefes do serviço, separou para o ministério os filhos de Asafe, de Hemã e de Jedutum, para profetizarem com harpas, alaúdes e címbalos. O rol dos encarregados neste ministério foi:
\par 2 dos filhos de Asafe: Zacur, José, Netanias e Asarela, filhos de Asafe, sob a direção deste, que exercia o seu ministério debaixo das ordens do rei.
\par 3 Quanto à família de Jedutum, os filhos: Gedalias, Zeri, Jesaías, Hasabias e Matitias, seis, sob a direção de Jedutum, seu pai, que profetizava com harpas, em ações de graças e louvores ao SENHOR.
\par 4 Quanto à família de Hemã, os filhos: Buquias, Matanias, Uziel, Sebuel, Jerimote, Hananias, Hanani, Eliata, Gidalti, Romanti-Ézer, Josbecasa, Maloti, Hotir e Maaziote.
\par 5 Todos estes foram filhos de Hemã, o vidente do rei e cujo poder Deus exaltou segundo as suas promessas, dando-lhe catorze filhos e três filhas.
\par 6 Todos estes estavam sob a direção respectivamente de seus pais, para o canto da Casa do SENHOR, com címbalos, alaúdes e harpas, para o ministério da Casa de Deus, estando Asafe, Jedutum e Hemã debaixo das ordens do rei.
\par 7 O número deles, juntamente com seus irmãos instruídos no canto do SENHOR, todos eles mestres, era de duzentos e oitenta e oito.
\par 8 Deitaram sortes para designar os deveres, tanto do pequeno como do grande, tanto do mestre como do discípulo.
\par 9 A primeira sorte tocou à família de Asafe e saiu a José; a segunda, a Gedalias, que, com seus irmãos e seus filhos, eram doze ao todo.
\par 10 A terceira, a Zacur, seus filhos e seus irmãos, doze.
\par 11 A quarta, a Izri, seus filhos e seus irmãos, doze.
\par 12 A quinta, a Netanias, seus filhos e seus irmãos, doze.
\par 13 A sexta, a Buquias, seus filhos e seus irmãos, doze.
\par 14 A sétima, a Jesarela, seus filhos e seus irmãos, doze.
\par 15 A oitava, a Jesaías, seus filhos e seus irmãos, doze.
\par 16 A nona, a Matanias, seus filhos e seus irmãos, doze.
\par 17 A décima, a Simei, seus filhos e seus irmãos, doze.
\par 18 A undécima, a Azarel, seus filhos e seus irmãos, doze.
\par 19 A duodécima, a Hasabias, seus filhos e seus irmãos, doze.
\par 20 A décima terceira, a Subael, seus filhos e seus irmãos, doze.
\par 21 A décima quarta, a Matitias, seus filhos e seus irmãos, doze.
\par 22 A décima quinta, a Jerimote, seus filhos e seus irmãos, doze.
\par 23 A décima sexta, a Hananias, seus filhos e seus irmãos, doze.
\par 24 A décima sétima, a Josbecasa, seus filhos e seus irmãos, doze.
\par 25 A décima oitava, a Hanani, seus filhos e seus irmãos, doze.
\par 26 A décima nona, a Maloti, seus filhos e seus irmãos, doze.
\par 27 A vigésima, a Eliata, seus filhos e seus irmãos, doze.
\par 28 A vigésima primeira, a Hotir, seus filhos e seus irmãos, doze.
\par 29 A vigésima segunda, a Gidalti, seus filhos e seus irmãos, doze.
\par 30 A vigésima terceira, a Maaziote, seus filhos e seus irmãos, doze.
\par 31 A vigésima quarta, a Romanti-Ézer, seus filhos e seus irmãos, doze.

\chapter{26}

\par 1 Quanto aos turnos dos porteiros, dos coreítas: Meselemias, filho de Coré, dos filhos de Asafe.
\par 2 Os filhos de Meselemias: Zacarias, o primogênito, Jediael, o segundo, Zebadias, o terceiro, Jatniel, o quarto,
\par 3 Elão, o quinto, Joanã, o sexto, Elioenai, o sétimo.
\par 4 Os filhos de Obede-Edom: Semaías, o primogênito, Jeozabade, o segundo, Joá, o terceiro, Sacar, o quarto, Natanael, o quinto.
\par 5 Amiel, o sexto, Issacar, o sétimo, Peuletai, o oitavo; porque Deus o tinha abençoado.
\par 6 Também a seu filho Semaías nasceram filhos, que dominaram sobre a casa de seu pai; porque foram homens valentes.
\par 7 Os filhos de Semaías: Otni, Rafael, Obede e Elzabade, cujos irmãos Eliú e Semaquias eram homens valentes.
\par 8 Todos estes foram dos filhos de Obede-Edom; eles, seus filhos e seus irmãos, homens capazes e robustos para o serviço, ao todo, sessenta e dois.
\par 9 Os filhos e os irmãos de Meselemias, homens valentes, foram dezoito.
\par 10 De Hosa, dos filhos de Merari, foram filhos: Sinri, a quem o pai constituiu chefe, ainda que não era o primogênito.
\par 11 Hilquias, o segundo, Tebalias, o terceiro, Zacarias, o quarto; todos os filhos e irmãos de Hosa foram treze.
\par 12 A estes turnos dos porteiros, isto é, a seus chefes, foi entregue a guarda, para servirem, como seus irmãos, na Casa do SENHOR.
\par 13 Para cada porta deitaram sortes para designar os deveres tanto dos pequenos como dos grandes, segundo as suas famílias.
\par 14 A guarda do lado do oriente caiu por sorte a Selemias; depois, lançaram sorte sobre seu filho Zacarias, conselheiro prudente, e lhe saiu a guarda do lado do norte;
\par 15 a Obede-Edom, a do lado do sul; e a seus filhos, a da casa de depósitos;
\par 16 a Supim e Hosa, a do ocidente, junto à porta de Salequete, na estrada que sobe; guarda correspondendo uns aos outros:
\par 17 ao oriente, estavam de guarda seis levitas; ao norte, quatro por dia; ao sul, quatro por dia, e, para a casa de depósitos, dois num lugar e dois noutro.
\par 18 No átrio ao ocidente, quatro junto ao caminho, dois junto ao átrio.
\par 19 São estes os turnos dos porteiros dos filhos dos coreítas e dos filhos de Merari.
\par 20 Dos levitas, seus irmãos, que tinham o encargo dos tesouros da Casa de Deus e dos tesouros das coisas consagradas:
\par 21 os filhos de Ladã, descendentes dos gersonitas pertencentes a Ladã e chefes das famílias deste, da família de Gérson: Jeieli;
\par 22 os filhos de Jeieli: Zetã e Joel, seu irmão; estavam estes a cargo dos tesouros da Casa do SENHOR.
\par 23 Dos anramitas, dos isaritas, dos hebronitas, dos uzielitas,
\par 24 Sebuel, filho de Gérson, filho de Moisés, era oficial encarregado dos tesouros.
\par 25 Seus irmãos: de Eliézer, foi filho Reabias, de quem foi filho Jesaías, de quem foi filho Jorão, de quem foi filho Zicri, de quem foi filho Selomite.
\par 26 Este Selomite e seus irmãos tinham a seu cargo todos os tesouros das coisas consagradas que o rei Davi e os chefes das famílias, capitães de milhares e de centenas e capitães do exército tinham dedicado;
\par 27 dos despojos das guerras as dedicaram para a conservação da Casa do SENHOR,
\par 28 como também tudo quanto havia dedicado Samuel, o vidente, e Saul, filho de Quis, e Abner, filho de Ner, e Joabe, filho de Zeruia; tudo quanto qualquer pessoa havia dedicado estava sob os cuidados de Selomite e seus irmãos.
\par 29 Dos isaritas, Quenanias e seus filhos foram postos sobre Israel, para oficiais e juízes dos negócios externos;
\par 30 dos hebronitas, foram Hasabias e seus irmãos, homens valentes, mil e setecentos, que superintendiam Israel, além do Jordão para o ocidente, em todo serviço do SENHOR e interesses do rei;
\par 31 dos hebronitas, Jerias era o chefe. Quanto aos hebronitas, suas genealogias e famílias, se fizeram investigações no quadragésimo ano do reinado de Davi e se acharam entre eles homens valentes em Jazer de Gileade.
\par 32 Seus irmãos, homens valentes, dois mil e setecentos, chefes das famílias; e o rei Davi os constituiu sobre os rubenitas, os gaditas e a meia tribo dos manassitas, para todos os negócios de Deus e para todos os negócios do rei.

\chapter{27}

\par 1 São estes os filhos de Israel segundo o seu número, os chefes das famílias e os capitães de milhares e de centenas com os seus oficiais, que serviam ao rei em todos os negócios dos turnos que entravam e saíam de mês em mês durante o ano, cada turno de vinte e quatro mil.
\par 2 Sobre o primeiro turno do primeiro mês estava Jasobeão, filho de Zabdiel; em seu turno havia vinte e quatro mil.
\par 3 Era este dos filhos de Perez, chefe de todos os capitães dos exércitos para o primeiro mês.
\par 4 Sobre o turno do segundo mês estava Dodai, o aoíta, a cujo lado estava Miclote; também em seu turno havia vinte e quatro mil.
\par 5 O terceiro capitão do exército e o designado para o terceiro mês era Benaia, chefe, filho do sacerdote Joiada; também em seu turno havia vinte e quatro mil.
\par 6 Era este Benaia homem poderoso entre os trinta e cabeça deles; o seu turno estava ao encargo do seu filho Amizabade.
\par 7 O quarto, para o quarto mês, Asael, irmão de Joabe, e depois dele Zebadias, seu filho; também em seu turno havia vinte e quatro mil.
\par 8 O quinto capitão, para o quinto mês, Samute, o izraíta; também em seu turno havia vinte e quatro mil.
\par 9 O sexto, para o sexto mês, Ira, filho de Iques, o tecoíta; também em seu turno havia vinte e quatro mil.
\par 10 O sétimo, para o sétimo mês, Heles, o pelonita, dos filhos de Efraim; também em seu turno havia vinte e quatro mil.
\par 11 O oitavo, para o oitavo mês, Sibecai, o husatita, dos zeraítas; também em seu turno havia vinte e quatro mil.
\par 12 O nono, para o nono mês, Abiezer, o anatotita, dos benjamitas; também em seu turno havia vinte e quatro mil.
\par 13 O décimo, para o décimo mês, Maarai, o netofatita, dos zeraítas; também em seu turno havia vinte e quatro mil.
\par 14 O undécimo, para o undécimo mês, Benaia, o piratonita, dos filhos de Efraim; também em seu turno havia vinte e quatro mil.
\par 15 O duodécimo, para o duodécimo mês, Heldai, o netofatita, de Otniel; também em seu turno havia vinte e quatro mil.
\par 16 Sobre as tribos de Israel eram estes: sobre os rubenitas era chefe Eliézer, filho de Zicri; sobre os simeonitas, Sefatias, filho de Maaca;
\par 17 sobre os levitas, Hasabias, filho de Quemuel; sobre os aronitas, Zadoque;
\par 18 sobre Judá, Eliú, dos irmãos de Davi; sobre Issacar, Onri, filho de Micael;
\par 19 sobre Zebulom, Ismaías, filho de Obadias; sobre Naftali, Jerimote, filho de Azriel;
\par 20 sobre os filhos de Efraim, Oséias, filho de Azazias; sobre a meia tribo de Manassés, Joel, filho de Pedaías;
\par 21 sobre a outra meia tribo de Manassés em Gileade, Ido, filho de Zacarias; sobre Benjamim, Jaasiel, filho de Abner;
\par 22 sobre Dã, Azarel, filho de Jeroão; estes eram os chefes das tribos de Israel.
\par 23 Davi não contou os que eram de vinte anos para baixo, porque o SENHOR tinha dito que multiplicaria a Israel como as estrelas do céu.
\par 24 Joabe, filho de Zeruia, tinha começado a contar o povo, porém não acabou, porquanto viera por isso grande ira sobre Israel; pelo que o número não se registrou na história do rei Davi.
\par 25 Azmavete, filho de Adiel, estava sobre os tesouros do rei; sobre o que este possuía nos campos, nas cidades, nas aldeias e nos castelos, Jônatas, filho de Uzias.
\par 26 Sobre os lavradores do campo, que cultivavam a terra, Ezri, filho de Quelube.
\par 27 Sobre as vinhas, Simei, o ramatita; porém sobre o que das vides entrava para as adegas, Zabdi, o sifmita.
\par 28 Sobre os olivais e sicômoros que havia nas campinas, Baal-Hanã, o gederita; porém Joás, sobre os depósitos do azeite.
\par 29 Sobre os gados que pasciam em Sarom, Sitrai, o saronita; porém sobre os gados dos vales, Safate, filho de Adlai.
\par 30 Sobre os camelos, Obil, o ismaelita; sobre as jumentas, Jedias, o meronotita.
\par 31 Sobre o gado miúdo, Jaziz, o hagareno; todos estes eram administradores da fazenda do rei Davi.
\par 32 Jônatas, tio de Davi, era do conselho, homem sábio e escriba; Jeiel, filho de Hacmoni, atendia os filhos do rei.
\par 33 Aitofel era do conselho do rei; Husai, o arquita, amigo do rei.
\par 34 A Aitofel sucederam Joiada, filho de Benaia, e Abiatar; Joabe era comandante do exército do rei.

\chapter{28}

\par 1 Então, Davi convocou para Jerusalém todos os príncipes de Israel, os príncipes das tribos, os capitães dos turnos que serviam o rei, os capitães de mil e os de cem, os administradores de toda a fazenda e possessões do rei e de seus filhos, como também os oficiais, os poderosos e todo homem valente.
\par 2 Pôs-se o rei Davi em pé e disse: Ouvi-me, irmãos meus e povo meu: Era meu propósito de coração edificar uma casa de repouso para a arca da Aliança do SENHOR e para o estrado dos pés do nosso Deus, e eu tinha feito o preparo para a edificar.
\par 3 Porém Deus me disse: Não edificarás casa ao meu nome, porque és homem de guerra e derramaste muito sangue.
\par 4 O SENHOR, Deus de Israel, me escolheu de toda a casa de meu pai, para que eternamente fosse eu rei sobre Israel; porque a Judá escolheu por príncipe e a casa de meu pai, na casa de Judá; e entre os filhos de meu pai se agradou de mim, para me fazer rei sobre todo o Israel.
\par 5 E, de todos os meus filhos, porque muitos filhos me deu o SENHOR, escolheu ele a Salomão para se assentar no trono do reino do SENHOR, sobre Israel.
\par 6 E me disse: Teu filho Salomão é quem edificará a minha casa e os meus átrios, porque o escolhi para filho e eu lhe serei por pai.
\par 7 Estabelecerei o seu reino para sempre, se perseverar ele em cumprir os meus mandamentos e os meus juízos, como até ao dia de hoje.
\par 8 Agora, pois, perante todo o Israel, a congregação do SENHOR, e perante o nosso Deus, que me ouve, eu vos digo: guardai todos os mandamentos do SENHOR, vosso Deus, e empenhai-vos por eles, para que possuais esta boa terra e a deixeis como herança a vossos filhos, para sempre.
\par 9 Tu, meu filho Salomão, conhece o Deus de teu pai e serve-o de coração íntegro e alma voluntária; porque o SENHOR esquadrinha todos os corações e penetra todos os desígnios do pensamento. Se o buscares, ele deixará achar-se por ti; se o deixares, ele te rejeitará para sempre.
\par 10 Agora, pois, atende a tudo, porque o SENHOR te escolheu para edificares casa para o santuário; sê forte e faze a obra.
\par 11 Deu Davi a Salomão, seu filho, a planta do pórtico com as suas casas, as suas tesourarias, os seus cenáculos e as suas câmaras interiores, como também da casa do propiciatório.
\par 12 Também a planta de tudo quanto tinha em mente, com referência aos átrios da Casa do SENHOR, e a todas as câmaras em redor, para os tesouros da Casa de Deus e para os tesouros das coisas consagradas;
\par 13 e para os turnos dos sacerdotes e dos levitas, e para toda obra do ministério da Casa do SENHOR, e para todos os utensílios para o serviço da Casa do SENHOR,
\par 14 especificando o peso do ouro para todos os utensílios de ouro de cada serviço; também o peso da prata para todos os utensílios de prata de cada serviço;
\par 15 o peso para os candeeiros de ouro e suas lâmpadas de ouro, para cada candeeiro e suas lâmpadas, segundo o uso de cada um;
\par 16 também o peso do ouro para as mesas da proposição, para cada uma de per si; como também a prata para as mesas de prata;
\par 17 ouro puro para os garfos, para as bacias e para os copos; para as taças de ouro o devido peso a cada uma, como também para as taças de prata, a cada uma o seu peso;
\par 18 o peso do ouro refinado para o altar do incenso, como também, segundo a planta, o ouro para o carro dos querubins, que haviam de estender as asas e cobrir a arca da Aliança do SENHOR.
\par 19 Tudo isto, disse Davi, me foi dado por escrito por mandado do SENHOR, a saber, todas as obras desta planta.
\par 20 Disse Davi a Salomão, seu filho: Sê forte e corajoso e faze a obra; não temas, nem te desanimes, porque o SENHOR Deus, meu Deus, há de ser contigo; não te deixará, nem te desamparará, até que acabes todas as obras para o serviço da Casa do SENHOR.
\par 21 Eis aí os turnos dos sacerdotes e dos levitas para todo serviço da Casa de Deus; também se acham contigo, para toda obra, voluntários com sabedoria de toda espécie para cada serviço; como também os príncipes e todo o povo estarão inteiramente às tuas ordens.

\chapter{29}

\par 1 Disse mais o rei Davi a toda a congregação: Salomão, meu filho, o único a quem Deus escolheu, é ainda moço e inexperiente, e esta obra é grande; porque o palácio não é para homens, mas para o SENHOR Deus.
\par 2 Eu, pois, com todas as minhas forças já preparei para a casa de meu Deus ouro para as obras de ouro, prata para as de prata, bronze para as de bronze, ferro para as de ferro e madeira para as de madeira; pedras de ônix, pedras de engaste, pedras de várias cores, de mosaicos e toda sorte de pedras preciosas, e mármore, e tudo em abundância.
\par 3 E ainda, porque amo a casa de meu Deus, o ouro e a prata particulares que tenho dou para a casa de meu Deus, afora tudo quanto preparei para o santuário:
\par 4 três mil talentos de ouro, do ouro de Ofir, e sete mil talentos de prata purificada, para cobrir as paredes das casas;
\par 5 ouro para os objetos de ouro e prata para os de prata, e para toda obra de mão dos artífices. Quem, pois, está disposto, hoje, a trazer ofertas liberalmente ao SENHOR?
\par 6 Então, os chefes das famílias, os príncipes das tribos de Israel, os capitães de mil e os de cem e até os intendentes sobre as empresas do rei voluntariamente contribuíram
\par 7 e deram para o serviço da Casa de Deus cinco mil talentos de ouro, dez mil daricos, dez mil talentos de prata, dezoito mil talentos de bronze e cem mil talentos de ferro.
\par 8 Os que possuíam pedras preciosas as trouxeram para o tesouro da Casa do SENHOR, a cargo de Jeiel, o gersonita.
\par 9 O povo se alegrou com tudo o que se fez voluntariamente; porque de coração íntegro deram eles liberalmente ao SENHOR; também o rei Davi se alegrou com grande júbilo.
\par 10 Pelo que Davi louvou ao SENHOR perante a congregação toda e disse: Bendito és tu, SENHOR, Deus de Israel, nosso pai, de eternidade em eternidade.
\par 11 Teu, SENHOR, é o poder, a grandeza, a honra, a vitória e a majestade; porque teu é tudo quanto há nos céus e na terra; teu, SENHOR, é o reino, e tu te exaltaste por chefe sobre todos.
\par 12 Riquezas e glória vêm de ti, tu dominas sobre tudo, na tua mão há força e poder; contigo está o engrandecer e a tudo dar força.
\par 13 Agora, pois, ó nosso Deus, graças te damos e louvamos o teu glorioso nome.
\par 14 Porque quem sou eu, e quem é o meu povo para que pudéssemos dar voluntariamente estas coisas? Porque tudo vem de ti, e das tuas mãos to damos.
\par 15 Porque somos estranhos diante de ti e peregrinos como todos os nossos pais; como a sombra são os nossos dias sobre a terra, e não temos permanência.
\par 16 SENHOR, nosso Deus, toda esta abundância que preparamos para te edificar uma casa ao teu santo nome vem da tua mão e é toda tua.
\par 17 Bem sei, meu Deus, que tu provas os corações e que da sinceridade te agradas; eu também, na sinceridade de meu coração, dei voluntariamente todas estas coisas; acabo de ver com alegria que o teu povo, que se acha aqui, te faz ofertas voluntariamente.
\par 18 SENHOR, Deus de nossos pais Abraão, Isaque e Israel, conserva para sempre no coração do teu povo estas disposições e pensamentos, inclina-lhe o coração para contigo;
\par 19 e a Salomão, meu filho, dá coração íntegro para guardar os teus mandamentos, os teus testemunhos e os teus estatutos, fazendo tudo para edificar este palácio para o qual providenciei.
\par 20 Então, disse Davi a toda a congregação: Agora, louvai o SENHOR, vosso Deus. Então, toda a congregação louvou ao SENHOR, Deus de seus pais; todos inclinaram a cabeça, adoraram o SENHOR e se prostraram perante o rei.
\par 21 Ao outro dia, trouxeram sacrifícios ao SENHOR e lhe ofereceram holocaustos de mil bezerros, mil carneiros, mil cordeiros, com as suas libações; sacrifícios em abundância por todo o Israel.
\par 22 Comeram e beberam, naquele dia, perante o SENHOR, com grande regozijo. Pela segunda vez, fizeram rei a Salomão, filho de Davi, e o ungiram ao SENHOR por príncipe e a Zadoque, por sacerdote.
\par 23 Salomão assentou-se no trono do SENHOR, rei, em lugar de Davi, seu pai, e prosperou; e todo o Israel lhe obedecia.
\par 24 Todos os príncipes, os grandes e até todos os filhos do rei Davi prestaram homenagens ao rei Salomão.
\par 25 O SENHOR engrandeceu sobremaneira a Salomão perante todo o Israel; deu-lhe majestade real, qual antes dele não teve nenhum rei em Israel.
\par 26 Ora, Davi, filho de Jessé, reinou sobre todo o Israel.
\par 27 O tempo que reinou sobre Israel foi de quarenta anos: em Hebrom, sete; em Jerusalém, trinta e três.
\par 28 Morreu em ditosa velhice, cheio de dias, riquezas e glória; e Salomão, seu filho, reinou em seu lugar.
\par 29 Os atos, pois, do rei Davi, tanto os primeiros como os últimos, eis que estão escritos nas crônicas, registrados por Samuel, o vidente, nas crônicas do profeta Natã e nas crônicas de Gade, o vidente,
\par 30 juntamente com o que se passou no seu reinado e a respeito do seu poder e todos os acontecimentos que se deram com ele, com Israel e com todos os reinos daquelas terras.


\end{document}