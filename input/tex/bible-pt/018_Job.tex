\begin{document}

\title{Jó}


\chapter{1}

\par 1 Havia um homem na terra de Uz, cujo nome era Jó; homem íntegro e reto, temente a Deus e que se desviava do mal.
\par 2 Nasceram-lhe sete filhos e três filhas.
\par 3 Possuía sete mil ovelhas, três mil camelos, quinhentas juntas de bois e quinhentas jumentas; era também mui numeroso o pessoal ao seu serviço, de maneira que este homem era o maior de todos os do Oriente.
\par 4 Seus filhos iam às casas uns dos outros e faziam banquetes, cada um por sua vez, e mandavam convidar as suas três irmãs a comerem e beberem com eles.
\par 5 Decorrido o turno de dias de seus banquetes, chamava Jó a seus filhos e os santificava; levantava-se de madrugada e oferecia holocaustos segundo o número de todos eles, pois dizia: Talvez tenham pecado os meus filhos e blasfemado contra Deus em seu coração. Assim o fazia Jó continuamente.
\par 6 Num dia em que os filhos de Deus vieram apresentar-se perante o SENHOR, veio também Satanás entre eles.
\par 7 Então, perguntou o SENHOR a Satanás: Donde vens? Satanás respondeu ao SENHOR e disse: De rodear a terra e passear por ela.
\par 8 Perguntou ainda o SENHOR a Satanás: Observaste o meu servo Jó? Porque ninguém há na terra semelhante a ele, homem íntegro e reto, temente a Deus e que se desvia do mal.
\par 9 Então, respondeu Satanás ao SENHOR: Porventura, Jó debalde teme a Deus?
\par 10 Acaso, não o cercaste com sebe, a ele, a sua casa e a tudo quanto tem? A obra de suas mãos abençoaste, e os seus bens se multiplicaram na terra.
\par 11 Estende, porém, a mão, e toca-lhe em tudo quanto tem, e verás se não blasfema contra ti na tua face.
\par 12 Disse o SENHOR a Satanás: Eis que tudo quanto ele tem está em teu poder; somente contra ele não estendas a mão. E Satanás saiu da presença do SENHOR.
\par 13 Sucedeu um dia, em que seus filhos e suas filhas comiam e bebiam vinho na casa do irmão primogênito,
\par 14 que veio um mensageiro a Jó e lhe disse: Os bois lavravam, e as jumentas pasciam junto a eles;
\par 15 de repente, deram sobre eles os sabeus, e os levaram, e mataram aos servos a fio de espada; só eu escapei, para trazer-te a nova.
\par 16 Falava este ainda quando veio outro e disse: Fogo de Deus caiu do céu, e queimou as ovelhas e os servos, e os consumiu; só eu escapei, para trazer-te a nova.
\par 17 Falava este ainda quando veio outro e disse: Dividiram-se os caldeus em três bandos, deram sobre os camelos, os levaram e mataram aos servos a fio de espada; só eu escapei, para trazer-te a nova.
\par 18 Também este falava ainda quando veio outro e disse: Estando teus filhos e tuas filhas comendo e bebendo vinho, em casa do irmão primogênito,
\par 19 eis que se levantou grande vento do lado do deserto e deu nos quatro cantos da casa, a qual caiu sobre eles, e morreram; só eu escapei, para trazer-te a nova.
\par 20 Então, Jó se levantou, rasgou o seu manto, rapou a cabeça e lançou-se em terra e adorou;
\par 21 e disse: Nu saí do ventre de minha mãe e nu voltarei; o SENHOR o deu e o SENHOR o tomou; bendito seja o nome do SENHOR!
\par 22 Em tudo isto Jó não pecou, nem atribuiu a Deus falta alguma.

\chapter{2}

\par 1 Num dia em que os filhos de Deus vieram apresentar-se perante o SENHOR, veio também Satanás entre eles apresentar-se perante o SENHOR.
\par 2 Então, o SENHOR disse a Satanás: Donde vens? Respondeu Satanás ao SENHOR e disse: De rodear a terra e passear por ela.
\par 3 Perguntou o SENHOR a Satanás: Observaste o meu servo Jó? Porque ninguém há na terra semelhante a ele, homem íntegro e reto, temente a Deus e que se desvia do mal. Ele conserva a sua integridade, embora me incitasses contra ele, para o consumir sem causa.
\par 4 Então, Satanás respondeu ao SENHOR: Pele por pele, e tudo quanto o homem tem dará pela sua vida.
\par 5 Estende, porém, a mão, toca-lhe nos ossos e na carne e verás se não blasfema contra ti na tua face.
\par 6 Disse o SENHOR a Satanás: Eis que ele está em teu poder; mas poupa-lhe a vida.
\par 7 Então, saiu Satanás da presença do SENHOR e feriu a Jó de tumores malignos, desde a planta do pé até ao alto da cabeça.
\par 8 Jó, sentado em cinza, tomou um caco para com ele raspar-se.
\par 9 Então, sua mulher lhe disse: Ainda conservas a tua integridade? Amaldiçoa a Deus e morre.
\par 10 Mas ele lhe respondeu: Falas como qualquer doida; temos recebido o bem de Deus e não receberíamos também o mal? Em tudo isto não pecou Jó com os seus lábios.
\par 11 Ouvindo, pois, três amigos de Jó todo este mal que lhe sobreviera, chegaram, cada um do seu lugar: Elifaz, o temanita, Bildade, o suíta, e Zofar, o naamatita; e combinaram ir juntamente condoer-se dele e consolá-lo.
\par 12 Levantando eles de longe os olhos e não o reconhecendo, ergueram a voz e choraram; e cada um, rasgando o seu manto, lançava pó ao ar sobre a cabeça.
\par 13 Sentaram-se com ele na terra, sete dias e sete noites; e nenhum lhe dizia palavra alguma, pois viam que a dor era muito grande.

\chapter{3}

\par 1 Depois disto, passou Jó a falar e amaldiçoou o seu dia natalício.
\par 2 Disse Jó:
\par 3 Pereça o dia em que nasci e a noite em que se disse: Foi concebido um homem!
\par 4 Converta-se aquele dia em trevas; e Deus, lá de cima, não tenha cuidado dele, nem resplandeça sobre ele a luz.
\par 5 Reclamem-no as trevas e a sombra de morte; habitem sobre ele nuvens; espante-o tudo o que pode enegrecer o dia.
\par 6 Aquela noite, que dela se apoderem densas trevas; não se regozije ela entre os dias do ano, não entre na conta dos meses.
\par 7 Seja estéril aquela noite, e dela sejam banidos os sons de júbilo.
\par 8 Amaldiçoem-na aqueles que sabem amaldiçoar o dia e sabem excitar o monstro marinho.
\par 9 Escureçam-se as estrelas do crepúsculo matutino dessa noite; que ela espere a luz, e a luz não venha; que não veja as pálpebras dos olhos da alva,
\par 10 pois não fechou as portas do ventre de minha mãe, nem escondeu dos meus olhos o sofrimento.
\par 11 Por que não morri eu na madre? Por que não expirei ao sair dela?
\par 12 Por que houve regaço que me acolhesse? E por que peitos, para que eu mamasse?
\par 13 Porque já agora repousaria tranqüilo; dormiria, e, então, haveria para mim descanso,
\par 14 com os reis e conselheiros da terra que para si edificaram mausoléus;
\par 15 ou com os príncipes que tinham ouro e encheram de prata as suas casas;
\par 16 ou, como aborto oculto, eu não existiria, como crianças que nunca viram a luz.
\par 17 Ali, os maus cessam de perturbar, e, ali, repousam os cansados.
\par 18 Ali, os presos juntamente repousam e não ouvem a voz do feitor.
\par 19 Ali, está tanto o pequeno como o grande e o servo livre de seu senhor.
\par 20 Por que se concede luz ao miserável e vida aos amargurados de ânimo,
\par 21 que esperam a morte, e ela não vem? Eles cavam em procura dela mais do que tesouros ocultos.
\par 22 Eles se regozijariam por um túmulo e exultariam se achassem a sepultura.
\par 23 Por que se concede luz ao homem, cujo caminho é oculto, e a quem Deus cercou de todos os lados?
\par 24 Por que em vez do meu pão me vêm gemidos, e os meus lamentos se derramam como água?
\par 25 Aquilo que temo me sobrevém, e o que receio me acontece.
\par 26 Não tenho descanso, nem sossego, nem repouso, e já me vem grande perturbação.

\chapter{4}

\par 1 Então, respondeu Elifaz, o temanita, e disse:
\par 2 Se intentar alguém falar-te, enfadar-te-ás? Quem, todavia, poderá conter as palavras?
\par 3 Eis que tens ensinado a muitos e tens fortalecido mãos fracas.
\par 4 As tuas palavras têm sustentado aos que tropeçavam, e os joelhos vacilantes tens fortificado.
\par 5 Mas agora, em chegando a tua vez, tu te enfadas; sendo tu atingido, te perturbas.
\par 6 Porventura, não é o teu temor de Deus aquilo em que confias, e a tua esperança, a retidão dos teus caminhos?
\par 7 Lembra-te: acaso, já pereceu algum inocente? E onde foram os retos destruídos?
\par 8 Segundo eu tenho visto, os que lavram a iniqüidade e semeiam o mal, isso mesmo eles segam.
\par 9 Com o hálito de Deus perecem; e com o assopro da sua ira se consomem.
\par 10 Cessa o bramido do leão e a voz do leão feroz, e os dentes dos leõezinhos se quebram.
\par 11 Perece o leão, porque não há presa, e os filhos da leoa andam dispersos.
\par 12 Uma palavra se me disse em segredo; e os meus ouvidos perceberam um sussurro dela.
\par 13 Entre pensamentos de visões noturnas, quando profundo sono cai sobre os homens,
\par 14 sobrevieram-me o espanto e o tremor, e todos os meus ossos estremeceram.
\par 15 Então, um espírito passou por diante de mim; fez-me arrepiar os cabelos do meu corpo;
\par 16 parou ele, mas não lhe discerni a aparência; um vulto estava diante dos meus olhos; houve silêncio, e ouvi uma voz:
\par 17 Seria, porventura, o mortal justo diante de Deus? Seria, acaso, o homem puro diante do seu Criador?
\par 18 Eis que Deus não confia nos seus servos e aos seus anjos atribui imperfeições;
\par 19 quanto mais àqueles que habitam em casas de barro, cujo fundamento está no pó, e são esmagados como a traça!
\par 20 Nascem de manhã e à tarde são destruídos; perecem para sempre, sem que disso se faça caso.
\par 21 Se se lhes corta o fio da vida, morrem e não atingem a sabedoria.

\chapter{5}

\par 1 Chama agora! Haverá alguém que te atenda? E para qual dos santos anjos te virarás?
\par 2 Porque a ira do louco o destrói, e o zelo do tolo o mata.
\par 3 Bem vi eu o louco lançar raízes; mas logo declarei maldita a sua habitação.
\par 4 Seus filhos estão longe do socorro, são espezinhados às portas, e não há quem os livre.
\par 5 A sua messe, o faminto a devora e até do meio dos espinhos a arrebata; e o intrigante abocanha os seus bens.
\par 6 Porque a aflição não vem do pó, e não é da terra que brota o enfado.
\par 7 Mas o homem nasce para o enfado, como as faíscas das brasas voam para cima.
\par 8 Quanto a mim, eu buscaria a Deus e a ele entregaria a minha causa;
\par 9 ele faz coisas grandes e inescrutáveis e maravilhas que não se podem contar;
\par 10 faz chover sobre a terra e envia águas sobre os campos,
\par 11 para pôr os abatidos num lugar alto e para que os enlutados se alegrem da maior ventura.
\par 12 Ele frustra as maquinações dos astutos, para que as suas mãos não possam realizar seus projetos.
\par 13 Ele apanha os sábios na sua própria astúcia; e o conselho dos que tramam se precipita.
\par 14 Eles de dia encontram as trevas; ao meio-dia andam como de noite, às apalpadelas.
\par 15 Porém Deus salva da espada que lhes sai da boca, salva o necessitado da mão do poderoso.
\par 16 Assim, há esperança para o pobre, e a iniqüidade tapa a sua própria boca.
\par 17 Bem-aventurado é o homem a quem Deus disciplina; não desprezes, pois, a disciplina do Todo-Poderoso.
\par 18 Porque ele faz a ferida e ele mesmo a ata; ele fere, e as suas mãos curam.
\par 19 De seis angústias te livrará, e na sétima o mal te não tocará.
\par 20 Na fome te livrará da morte; na guerra, do poder da espada.
\par 21 Do açoite da língua estarás abrigado e, quando vier a assolação, não a temerás.
\par 22 Da assolação e da fome te rirás e das feras da terra não terás medo.
\par 23 Porque até com as pedras do campo terás a tua aliança, e os animais da terra viverão em paz contigo.
\par 24 Saberás que a paz é a tua tenda, percorrerás as tuas possessões, e nada te faltará.
\par 25 Saberás também que se multiplicará a tua descendência, e a tua posteridade, como a erva da terra.
\par 26 Em robusta velhice entrarás para a sepultura, como se recolhe o feixe de trigo a seu tempo.
\par 27 Eis que isto já o havemos inquirido, e assim é; ouve-o e medita nisso para teu bem.

\chapter{6}

\par 1 Então, Jó respondeu:
\par 2 Oh! Se a minha queixa, de fato, se pesasse, e contra ela, numa balança, se pusesse a minha miséria,
\par 3 esta, na verdade, pesaria mais que a areia dos mares; por isso é que as minhas palavras foram precipitadas.
\par 4 Porque as flechas do Todo-Poderoso estão em mim cravadas, e o meu espírito sorve o veneno delas; os terrores de Deus se arregimentam contra mim.
\par 5 Zurrará o jumento montês junto à relva? Ou mugirá o boi junto à sua forragem?
\par 6 Comer-se-á sem sal o que é insípido? Ou haverá sabor na clara do ovo?
\par 7 Aquilo que a minha alma recusava tocar, isso é agora a minha comida repugnante.
\par 8 Quem dera que se cumprisse o meu pedido, e que Deus me concedesse o que anelo!
\par 9 Que fosse do agrado de Deus esmagar-me, que soltasse a sua mão e acabasse comigo!
\par 10 Isto ainda seria a minha consolação, e saltaria de contente na minha dor, que ele não poupa; porque não tenho negado as palavras do Santo.
\par 11 Por que esperar, se já não tenho forças? Por que prolongar a vida, se o meu fim é certo?
\par 12 Acaso, a minha força é a força da pedra? Ou é de bronze a minha carne?
\par 13 Não! Jamais haverá socorro para mim; foram afastados de mim os meus recursos.
\par 14 Ao aflito deve o amigo mostrar compaixão, a menos que tenha abandonado o temor do Todo-Poderoso.
\par 15 Meus irmãos aleivosamente me trataram; são como um ribeiro, como a torrente que transborda no vale,
\par 16 turvada com o gelo e com a neve que nela se esconde,
\par 17 torrente que no tempo do calor seca, emudece e desaparece do seu lugar.
\par 18 Desviam-se as caravanas dos seus caminhos, sobem para lugares desolados e perecem.
\par 19 As caravanas de Temá procuram essa torrente, os viajantes de Sabá por ela suspiram.
\par 20 Ficam envergonhados por terem confiado; em chegando ali, confundem-se.
\par 21 Assim também vós outros sois nada para mim; vedes os meus males e vos espantais.
\par 22 Acaso, disse eu: dai-me um presente? Ou: oferecei-me um suborno da vossa fazenda?
\par 23 Ou: livrai-me do poder do opressor? Ou: redimi-me das mãos dos tiranos?
\par 24 Ensinai-me, e eu me calarei; dai-me a entender em que tenho errado.
\par 25 Oh! Como são persuasivas as palavras retas! Mas que é o que repreende a vossa repreensão?
\par 26 Acaso, pensais em reprovar as minhas palavras, ditas por um desesperado ao vento?
\par 27 Até sobre o órfão lançaríeis sorte e especularíeis com o vosso amigo?
\par 28 Agora, pois, se sois servidos, olhai para mim e vede que não minto na vossa cara.
\par 29 Tornai a julgar, vos peço, e não haja iniqüidade; tornai a julgar, e a justiça da minha causa triunfará.
\par 30 Há iniqüidade na minha língua? Não pode o meu paladar discernir coisas perniciosas?

\chapter{7}

\par 1 Não é penosa a vida do homem sobre a terra? Não são os seus dias como os de um jornaleiro?
\par 2 Como o escravo que suspira pela sombra e como o jornaleiro que espera pela sua paga,
\par 3 assim me deram por herança meses de desengano e noites de aflição me proporcionaram.
\par 4 Ao deitar-me, digo: quando me levantarei? Mas comprida é a noite, e farto-me de me revolver na cama, até à alva.
\par 5 A minha carne está vestida de vermes e de crostas terrosas; a minha pele se encrosta e de novo supura.
\par 6 Os meus dias são mais velozes do que a lançadeira do tecelão e se findam sem esperança.
\par 7 Lembra-te de que a minha vida é um sopro; os meus olhos não tornarão a ver o bem.
\par 8 Os olhos dos que agora me vêem não me verão mais; os teus olhos me procurarão, mas já não serei.
\par 9 Tal como a nuvem se desfaz e passa, aquele que desce à sepultura jamais tornará a subir.
\par 10 Nunca mais tornará à sua casa, nem o lugar onde habita o conhecerá jamais.
\par 11 Por isso, não reprimirei a boca, falarei na angústia do meu espírito, queixar-me-ei na amargura da minha alma.
\par 12 Acaso, sou eu o mar ou algum monstro marinho, para que me ponhas guarda?
\par 13 Dizendo eu: consolar-me-á o meu leito, a minha cama aliviará a minha queixa,
\par 14 então, me espantas com sonhos e com visões me assombras;
\par 15 pelo que a minha alma escolheria, antes, ser estrangulada; antes, a morte do que esta tortura.
\par 16 Estou farto da minha vida; não quero viver para sempre. Deixa-me, pois, porque os meus dias são um sopro.
\par 17 Que é o homem, para que tanto o estimes, e ponhas nele o teu cuidado,
\par 18 e cada manhã o visites, e cada momento o ponhas à prova?
\par 19 Até quando não apartarás de mim a tua vista? Até quando não me darás tempo de engolir a minha saliva?
\par 20 Se pequei, que mal te fiz a ti, ó Espreitador dos homens? Por que fizeste de mim um alvo para ti, para que a mim mesmo me seja pesado?
\par 21 Por que não perdoas a minha transgressão e não tiras a minha iniqüidade? Pois agora me deitarei no pó; e, se me buscas, já não serei.

\chapter{8}

\par 1 Então, respondeu Bildade, o suíta:
\par 2 Até quando falarás tais coisas? E até quando as palavras da tua boca serão qual vento impetuoso?
\par 3 Perverteria Deus o direito ou perverteria o Todo-Poderoso a justiça?
\par 4 Se teus filhos pecaram contra ele, também ele os lançou no poder da sua transgressão.
\par 5 Mas, se tu buscares a Deus e ao Todo-Poderoso pedires misericórdia,
\par 6 se fores puro e reto, ele, sem demora, despertará em teu favor e restaurará a justiça da tua morada.
\par 7 O teu primeiro estado, na verdade, terá sido pequeno, mas o teu último crescerá sobremaneira.
\par 8 Pois, eu te peço, pergunta agora a gerações passadas e atenta para a experiência de seus pais;
\par 9 porque nós somos de ontem e nada sabemos; porquanto nossos dias sobre a terra são como a sombra.
\par 10 Porventura, não te ensinarão os pais, não haverão de falar-te e do próprio entendimento não proferirão estas palavras:
\par 11 Pode o papiro crescer sem lodo? Ou viça o junco sem água?
\par 12 Estando ainda na sua verdura e ainda não colhidos, todavia, antes de qualquer outra erva se secam.
\par 13 São assim as veredas de todos quantos se esquecem de Deus; e a esperança do ímpio perecerá.
\par 14 A sua firmeza será frustrada, e a sua confiança é teia de aranha.
\par 15 Encostar-se-á à sua casa, e ela não se manterá, agarrar-se-á a ela, e ela não ficará em pé.
\par 16 Ele é viçoso perante o sol, e os seus renovos irrompem no seu jardim;
\par 17 as suas raízes se entrelaçam num montão de pedras e penetram até às muralhas.
\par 18 Mas, se Deus o arranca do seu lugar, então, este o negará, dizendo: Nunca te vi.
\par 19 Eis em que deu a sua vida! E do pó brotarão outros.
\par 20 Eis que Deus não rejeita ao íntegro, nem toma pela mão os malfeitores.
\par 21 Ele te encherá a boca de riso e os teus lábios, de júbilo.
\par 22 Teus aborrecedores se vestirão de ignomínia, e a tenda dos perversos não subsistirá.

\chapter{9}

\par 1 Então, Jó respondeu e disse:
\par 2 Na verdade, sei que assim é; porque, como pode o homem ser justo para com Deus?
\par 3 Se quiser contender com ele, nem a uma de mil coisas lhe poderá responder.
\par 4 Ele é sábio de coração e grande em poder; quem porfiou com ele e teve paz?
\par 5 Ele é quem remove os montes, sem que saibam que ele na sua ira os transtorna;
\par 6 quem move a terra para fora do seu lugar, cujas colunas estremecem;
\par 7 quem fala ao sol, e este não sai, e sela as estrelas;
\par 8 quem sozinho estende os céus e anda sobre os altos do mar;
\par 9 quem fez a Ursa, o Órion, o Sete-estrelo e as recâmaras do Sul;
\par 10 quem faz grandes coisas, que se não podem esquadrinhar, e maravilhas tais, que se não podem contar.
\par 11 Eis que ele passa por mim, e não o vejo; segue perante mim, e não o percebo.
\par 12 Eis que arrebata a presa! Quem o pode impedir? Quem lhe dirá: Que fazes?
\par 13 Deus não revogará a sua própria ira; debaixo dele se encurvam os auxiliadores do Egito.
\par 14 Como, então, lhe poderei eu responder ou escolher as minhas palavras, para argumentar com ele?
\par 15 A ele, ainda que eu fosse justo, não lhe responderia; antes, ao meu Juiz pediria misericórdia.
\par 16 Ainda que o chamasse, e ele me respondesse, nem por isso creria eu que desse ouvidos à minha voz.
\par 17 Porque me esmaga com uma tempestade e multiplica as minhas chagas sem causa.
\par 18 Não me permite respirar; antes, me farta de amarguras.
\par 19 Se se trata da força do poderoso, ele dirá: Eis-me aqui; se, de justiça: Quem me citará?
\par 20 Ainda que eu seja justo, a minha boca me condenará; embora seja eu íntegro, ele me terá por culpado.
\par 21 Eu sou íntegro, não levo em conta a minha alma, não faço caso da minha vida.
\par 22 Para mim tudo é o mesmo; por isso, digo: tanto destrói ele o íntegro como o perverso.
\par 23 Se qualquer flagelo mata subitamente, então, se rirá do desespero do inocente.
\par 24 A terra está entregue nas mãos dos perversos; e Deus ainda cobre o rosto dos juízes dela; se não é ele o causador disso, quem é, logo?
\par 25 Os meus dias foram mais velozes do que um corredor; fugiram e não viram a felicidade.
\par 26 Passaram como barcos de junco; como a águia que se lança sobre a presa.
\par 27 Se eu disser: eu me esquecerei da minha queixa, deixarei o meu ar triste e ficarei contente;
\par 28 ainda assim todas as minhas dores me apavoram, porque bem sei que me não terás por inocente.
\par 29 Serei condenado; por que, pois, trabalho eu em vão?
\par 30 Ainda que me lave com água de neve e purifique as mãos com cáustico,
\par 31 mesmo assim me submergirás no lodo, e as minhas próprias vestes me abominarão.
\par 32 Porque ele não é homem, como eu, a quem eu responda, vindo juntamente a juízo.
\par 33 Não há entre nós árbitro que ponha a mão sobre nós ambos.
\par 34 Tire ele a sua vara de cima de mim, e não me amedronte o seu terror;
\par 35 então, falarei sem o temer; do contrário, não estaria em mim.

\chapter{10}

\par 1 A minha alma tem tédio à minha vida; darei livre curso à minha queixa, falarei com amargura da minha alma.
\par 2 Direi a Deus: Não me condenes; faze-me saber por que contendes comigo.
\par 3 Parece-te bem que me oprimas, que rejeites a obra das tuas mãos e favoreças o conselho dos perversos?
\par 4 Tens tu olhos de carne? Acaso, vês tu como vê o homem?
\par 5 São os teus dias como os dias do mortal? Ou são os teus anos como os anos de um homem,
\par 6 para te informares da minha iniqüidade e averiguares o meu pecado?
\par 7 Bem sabes tu que eu não sou culpado; todavia, ninguém há que me livre da tua mão.
\par 8 As tuas mãos me plasmaram e me aperfeiçoaram, porém, agora, queres devorar-me.
\par 9 Lembra-te de que me formaste como em barro; e queres, agora, reduzir-me a pó?
\par 10 Porventura, não me derramaste como leite e não me coalhaste como queijo?
\par 11 De pele e carne me vestiste e de ossos e tendões me entreteceste.
\par 12 Vida me concedeste na tua benevolência, e o teu cuidado a mim me guardou.
\par 13 Estas coisas, as ocultaste no teu coração; mas bem sei o que resolveste contigo mesmo.
\par 14 Se eu pecar, tu me observas; e da minha iniqüidade não me perdoarás.
\par 15 Se for perverso, ai de mim! E, se for justo, não ouso levantar a cabeça, pois estou cheio de ignomínia e olho para a minha miséria.
\par 16 Porque, se a levanto, tu me caças como a um leão feroz e de novo revelas poder maravilhoso contra mim.
\par 17 Tu renovas contra mim as tuas testemunhas e multiplicas contra mim a tua ira; males e lutas se sucedem contra mim.
\par 18 Por que, pois, me tiraste da madre? Ah! Se eu morresse antes que olhos nenhuns me vissem!
\par 19 Teria eu sido como se nunca existira e já do ventre teria sido levado à sepultura.
\par 20 Não são poucos os meus dias? Cessa, pois, e deixa-me, para que por um pouco eu tome alento,
\par 21 antes que eu vá para o lugar de que não voltarei, para a terra das trevas e da sombra da morte;
\par 22 terra de negridão, de profunda escuridade, terra da sombra da morte e do caos, onde a própria luz é tenebrosa.

\chapter{11}

\par 1 Então, respondeu Zofar, o naamatita:
\par 2 Porventura, não se dará resposta a esse palavrório? Acaso, tem razão o tagarela?
\par 3 Será o caso de as tuas parolas fazerem calar os homens? E zombarás tu sem que ninguém te envergonhe?
\par 4 Pois dizes: A minha doutrina é pura, e sou limpo aos teus olhos.
\par 5 Oh! Falasse Deus, e abrisse os seus lábios contra ti,
\par 6 e te revelasse os segredos da sabedoria, da verdadeira sabedoria, que é multiforme! Sabe, portanto, que Deus permite seja esquecida parte da tua iniqüidade.
\par 7 Porventura, desvendarás os arcanos de Deus ou penetrarás até à perfeição do Todo-Poderoso?
\par 8 Como as alturas dos céus é a sua sabedoria; que poderás fazer? Mais profunda é ela do que o abismo; que poderás saber?
\par 9 A sua medida é mais longa do que a terra e mais larga do que o mar.
\par 10 Se ele passa, prende a alguém e chama a juízo, quem o poderá impedir?
\par 11 Porque ele conhece os homens vãos e, sem esforço, vê a iniqüidade.
\par 12 Mas o homem estúpido se tornará sábio, quando a cria de um asno montês nascer homem.
\par 13 Se dispuseres o coração e estenderes as mãos para Deus;
\par 14 se lançares para longe a iniqüidade da tua mão e não permitires habitar na tua tenda a injustiça,
\par 15 então, levantarás o rosto sem mácula, estarás seguro e não temerás.
\par 16 Pois te esquecerás dos teus sofrimentos e deles só terás lembrança como de águas que passaram.
\par 17 A tua vida será mais clara que o meio-dia; ainda que lhe haja trevas, serão como a manhã.
\par 18 Sentir-te-ás seguro, porque haverá esperança; olharás em derredor e dormirás tranqüilo.
\par 19 Deitar-te-ás, e ninguém te espantará; e muitos procurarão obter o teu favor.
\par 20 Mas os olhos dos perversos desfalecerão, o seu refúgio perecerá; sua esperança será o render do espírito.

\chapter{12}

\par 1 Então, Jó respondeu:
\par 2 Na verdade, vós sois o povo, e convosco morrerá a sabedoria.
\par 3 Também eu tenho entendimento como vós; eu não vos sou inferior; quem não sabe coisas como essas?
\par 4 Eu sou irrisão para os meus amigos; eu, que invocava a Deus, e ele me respondia; o justo e o reto servem de irrisão.
\par 5 No pensamento de quem está seguro, há desprezo para o infortúnio, um empurrão para aquele cujos pés já vacilam.
\par 6 As tendas dos tiranos gozam paz, e os que provocam a Deus estão seguros; têm o punho por seu deus.
\par 7 Mas pergunta agora às alimárias, e cada uma delas to ensinará; e às aves dos céus, e elas to farão saber.
\par 8 Ou fala com a terra, e ela te instruirá; até os peixes do mar to contarão.
\par 9 Qual entre todos estes não sabe que a mão do SENHOR fez isto?
\par 10 Na sua mão está a alma de todo ser vivente e o espírito de todo o gênero humano.
\par 11 Porventura, o ouvido não submete à prova as palavras, como o paladar prova as comidas?
\par 12 Está a sabedoria com os idosos, e, na longevidade, o entendimento?
\par 13 Não! Com Deus está a sabedoria e a força; ele tem conselho e entendimento.
\par 14 O que ele deitar abaixo não se reedificará; lança na prisão, e ninguém a pode abrir.
\par 15 Se retém as águas, elas secam; se as larga, devastam a terra.
\par 16 Com ele está a força e a sabedoria; seu é o que erra e o que faz errar.
\par 17 Aos conselheiros, leva-os despojados do seu cargo e aos juízes faz desvairar.
\par 18 Dissolve a autoridade dos reis, e uma corda lhes cinge os lombos.
\par 19 Aos sacerdotes, leva-os despojados do seu cargo e aos poderosos transtorna.
\par 20 Aos eloqüentes ele tira a palavra e tira o entendimento aos anciãos.
\par 21 Lança desprezo sobre os príncipes e afrouxa o cinto dos fortes.
\par 22 Das trevas manifesta coisas profundas e traz à luz a densa escuridade.
\par 23 Multiplica as nações e as faz perecer; dispersa-as e de novo as congrega.
\par 24 Tira o entendimento aos príncipes do povo da terra e os faz vaguear pelos desertos sem caminho.
\par 25 Nas trevas andam às apalpadelas, sem terem luz, e os faz cambalear como ébrios.

\chapter{13}

\par 1 Eis que tudo isso viram os meus olhos, e os meus ouvidos o ouviram e entenderam.
\par 2 Como vós o sabeis, também eu o sei; não vos sou inferior.
\par 3 Mas falarei ao Todo-Poderoso e quero defender-me perante Deus.
\par 4 Vós, porém, besuntais a verdade com mentiras e vós todos sois médicos que não valem nada.
\par 5 Tomara vos calásseis de todo, que isso seria a vossa sabedoria!
\par 6 Ouvi agora a minha defesa e atentai para os argumentos dos meus lábios.
\par 7 Porventura, falareis perversidade em favor de Deus e a seu favor falareis mentiras?
\par 8 Sereis parciais por ele? Contendereis a favor de Deus?
\par 9 Ser-vos-ia bom, se ele vos esquadrinhasse? Ou zombareis dele, como se zomba de um homem qualquer?
\par 10 Acerbamente vos repreenderá, se em oculto fordes parciais.
\par 11 Porventura, não vos amedrontará a sua dignidade, e não cairá sobre vós o seu terror?
\par 12 As vossas máximas são como provérbios de cinza, os vossos baluartes, baluartes de barro.
\par 13 Calai-vos perante mim, e falarei eu, e venha sobre mim o que vier.
\par 14 Tomarei a minha carne nos meus dentes e porei a vida na minha mão.
\par 15 Eis que me matará, já não tenho esperança; contudo, defenderei o meu procedimento.
\par 16 Também isto será a minha salvação, o fato de o ímpio não vir perante ele.
\par 17 Atentai para as minhas razões e dai ouvidos à minha exposição.
\par 18 Tenho já bem encaminhada minha causa e estou certo de que serei justificado.
\par 19 Quem há que possa contender comigo? Neste caso, eu me calaria e renderia o espírito.
\par 20 Concede-me somente duas coisas; então, me não esconderei do teu rosto:
\par 21 alivia a tua mão de sobre mim, e não me espante o teu terror.
\par 22 Interpela-me, e te responderei ou deixa-me falar e tu me responderás.
\par 23 Quantas culpas e pecados tenho eu? Notifica-me a minha transgressão e o meu pecado.
\par 24 Por que escondes o rosto e me tens por teu inimigo?
\par 25 Queres aterrorizar uma folha arrebatada pelo vento? E perseguirás a palha seca?
\par 26 Pois decretas contra mim coisas amargas e me atribuis as culpas da minha mocidade.
\par 27 Também pões os meus pés no tronco, observas todos os meus caminhos e traças limites à planta dos meus pés,
\par 28 apesar de eu ser como uma coisa podre que se consome e como a roupa que é comida da traça.

\chapter{14}

\par 1 O homem, nascido de mulher, vive breve tempo, cheio de inquietação.
\par 2 Nasce como a flor e murcha; foge como a sombra e não permanece;
\par 3 e sobre tal homem abres os olhos e o fazes entrar em juízo contigo?
\par 4 Quem da imundícia poderá tirar coisa pura? Ninguém!
\par 5 Visto que os seus dias estão contados, contigo está o número dos seus meses; tu ao homem puseste limites além dos quais não passará.
\par 6 Desvia dele os olhares, para que tenha repouso, até que, como o jornaleiro, tenha prazer no seu dia.
\par 7 Porque há esperança para a árvore, pois, mesmo cortada, ainda se renovará, e não cessarão os seus rebentos.
\par 8 Se envelhecer na terra a sua raiz, e no chão morrer o seu tronco,
\par 9 ao cheiro das águas brotará e dará ramos como a planta nova.
\par 10 O homem, porém, morre e fica prostrado; expira o homem e onde está?
\par 11 Como as águas do lago se evaporam, e o rio se esgota e seca,
\par 12 assim o homem se deita e não se levanta; enquanto existirem os céus, não acordará, nem será despertado do seu sono.
\par 13 Que me encobrisses na sepultura e me ocultasses até que a tua ira se fosse, e me pusesses um prazo e depois te lembrasses de mim!
\par 14 Morrendo o homem, porventura tornará a viver? Todos os dias da minha luta esperaria, até que eu fosse substituído.
\par 15 Chamar-me-ias, e eu te responderia; terias saudades da obra de tuas mãos;
\par 16 e até contarias os meus passos e não levarias em conta os meus pecados.
\par 17 A minha transgressão estaria selada num saco, e terias encoberto as minhas iniqüidades.
\par 18 Como o monte que se esboroa e se desfaz, e a rocha que se remove do seu lugar,
\par 19 como as águas gastam as pedras, e as cheias arrebatam o pó da terra, assim destróis a esperança do homem.
\par 20 Tu prevaleces para sempre contra ele, e ele passa, mudas-lhe o semblante e o despedes para o além.
\par 21 Os seus filhos recebem honras, e ele o não sabe; são humilhados, e ele o não percebe.
\par 22 Ele sente as dores apenas de seu próprio corpo, e só a seu respeito sofre a sua alma.

\chapter{15}

\par 1 Então, respondeu Elifaz, o temanita:
\par 2 Porventura, dará o sábio em resposta ciência de vento? E encher-se-á a si mesmo de vento oriental,
\par 3 argüindo com palavras que de nada servem e com razões de que nada aproveita?
\par 4 Tornas vão o temor de Deus e diminuis a devoção a ele devida.
\par 5 Pois a tua iniqüidade ensina à tua boca, e tu escolheste a língua dos astutos.
\par 6 A tua própria boca te condena, e não eu; os teus lábios testificam contra ti.
\par 7 És tu, porventura, o primeiro homem que nasceu? Ou foste formado antes dos outeiros?
\par 8 Ou ouviste o secreto conselho de Deus e a ti só limitaste a sabedoria?
\par 9 Que sabes tu, que nós não saibamos? Que entendes, que não haja em nós?
\par 10 Também há entre nós encanecidos e idosos, muito mais idosos do que teu pai.
\par 11 Porventura, fazes pouco caso das consolações de Deus e das suaves palavras que te dirigimos nós?
\par 12 Por que te arrebata o teu coração? Por que flamejam os teus olhos,
\par 13 para voltares contra Deus o teu furor e deixares sair tais palavras da tua boca?
\par 14 Que é o homem, para que seja puro? E o que nasce de mulher, para ser justo?
\par 15 Eis que Deus não confia nem nos seus santos; nem os céus são puros aos seus olhos,
\par 16 quanto menos o homem, que é abominável e corrupto, que bebe a iniqüidade como a água!
\par 17 Escuta-me, mostrar-to-ei; e o que tenho visto te contarei,
\par 18 o que os sábios anunciaram, que o ouviram de seus pais e não o ocultaram
\par 19 (aos quais somente se dera a terra, e nenhum estranho passou por entre eles):
\par 20 Todos os dias o perverso é atormentado, no curto número de anos que se reservam para o opressor.
\par 21 O sonido dos horrores está nos seus ouvidos; na prosperidade lhe sobrevém o assolador.
\par 22 Não crê que tornará das trevas, e sim que o espera a espada.
\par 23 Por pão anda vagueando, dizendo: Onde está? Bem sabe que o dia das trevas lhe está preparado, à mão.
\par 24 Assombram-no a angústia e a tribulação; prevalecem contra ele, como o rei preparado para a peleja,
\par 25 porque estendeu a mão contra Deus e desafiou o Todo-Poderoso;
\par 26 arremete contra ele obstinadamente, atrás da grossura dos seus escudos,
\par 27 porquanto cobriu o rosto com a sua gordura e criou enxúndia nas ilhargas;
\par 28 habitou em cidades assoladas, em casas em que ninguém devia morar, que estavam destinadas a se fazerem montões de ruínas.
\par 29 Por isso, não se enriquecerá, nem subsistirá a sua fazenda, nem se estenderão seus bens pela terra.
\par 30 Não escapará das trevas; a chama do fogo secará os seus renovos, e ao assopro da boca de Deus será arrebatado.
\par 31 Não confie, pois, na vaidade, enganando-se a si mesmo, porque a vaidade será a sua recompensa.
\par 32 Esta se lhe consumará antes dos seus dias, e o seu ramo não reverdecerá.
\par 33 Sacudirá as suas uvas verdes, como a vide, e deixará cair a sua flor, como a oliveira;
\par 34 pois a companhia dos ímpios será estéril, e o fogo consumirá as tendas de suborno.
\par 35 Concebem a malícia e dão à luz a iniqüidade, pois o seu coração só prepara enganos.

\chapter{16}

\par 1 Então, respondeu Jó:
\par 2 Tenho ouvido muitas coisas como estas; todos vós sois consoladores molestos.
\par 3 Porventura, não terão fim essas palavras de vento? Ou que é que te instiga para responderes assim?
\par 4 Eu também poderia falar como vós falais; se a vossa alma estivesse em lugar da minha, eu poderia dirigir-vos um montão de palavras e menear contra vós outros a minha cabeça;
\par 5 poderia fortalecer-vos com as minhas palavras, e a compaixão dos meus lábios abrandaria a vossa dor.
\par 6 Se eu falar, a minha dor não cessa; se me calar, qual é o meu alívio?
\par 7 Na verdade, as minhas forças estão exaustas; tu, ó Deus, destruíste a minha família toda.
\par 8 Testemunha disto é que já me tornaste encarquilhado, a minha magreza já se levanta contra mim e me acusa cara a cara.
\par 9 Na sua ira me despedaçou e tem animosidade contra mim; contra mim rangeu os dentes e, como meu adversário, aguça os olhos.
\par 10 Homens abrem contra mim a boca, com desprezo me esbofeteiam, e contra mim todos se ajuntam.
\par 11 Deus me entrega ao ímpio e nas mãos dos perversos me faz cair.
\par 12 Em paz eu vivia, porém ele me quebrantou; pegou-me pelo pescoço e me despedaçou; pôs-me por seu alvo.
\par 13 Cercam-me as suas flechas, atravessa-me os rins, e não me poupa, e o meu fel derrama na terra.
\par 14 Fere-me com ferimento sobre ferimento, arremete contra mim como um guerreiro.
\par 15 Cosi sobre a minha pele o cilício e revolvi o meu orgulho no pó.
\par 16 O meu rosto está todo afogueado de chorar, e sobre as minhas pálpebras está a sombra da morte,
\par 17 embora não haja violência nas minhas mãos, e seja pura a minha oração.
\par 18 Ó terra, não cubras o meu sangue, e não haja lugar em que se oculte o meu clamor!
\par 19 Já agora sabei que a minha testemunha está no céu, e, nas alturas, quem advoga a minha causa.
\par 20 Os meus amigos zombam de mim, mas os meus olhos se desfazem em lágrimas diante de Deus,
\par 21 para que ele mantenha o direito do homem contra o próprio Deus e o do filho do homem contra o seu próximo.
\par 22 Porque dentro de poucos anos eu seguirei o caminho de onde não tornarei.

\chapter{17}

\par 1 O meu espírito se vai consumindo, os meus dias se vão apagando, e só tenho perante mim a sepultura.
\par 2 Estou, de fato, cercado de zombadores, e os meus olhos são obrigados a lhes contemplar a provocação.
\par 3 Dá-me, pois, um penhor; sê o meu fiador para contigo mesmo; quem mais haverá que se possa comprometer comigo?
\par 4 Porque ao seu coração encobriste o entendimento, pelo que não os exaltarás.
\par 5 Se alguém oferece os seus amigos como presa, os olhos de seus filhos desfalecerão.
\par 6 Mas a mim me pôs por provérbio dos povos; tornei-me como aquele em cujo rosto se cospe.
\par 7 Pelo que já se escureceram de mágoa os meus olhos, e já todos os meus membros são como a sombra;
\par 8 os retos pasmam disto, e o inocente se levanta contra o ímpio.
\par 9 Contudo, o justo segue o seu caminho, e o puro de mãos cresce mais e mais em força.
\par 10 Mas tornai-vos, todos vós, e vinde cá; porque sábio nenhum acharei entre vós.
\par 11 Os meus dias passaram, e se malograram os meus propósitos, as aspirações do meu coração.
\par 12 Convertem-me a noite em dia, e a luz, dizem, está perto das trevas.
\par 13 Mas, se eu aguardo já a sepultura por minha casa; se nas trevas estendo a minha cama;
\par 14 se ao sepulcro eu clamo: tu és meu pai; e aos vermes: vós sois minha mãe e minha irmã,
\par 15 onde está, pois, a minha esperança? Sim, a minha esperança, quem a poderá ver?
\par 16 Ela descerá até às portas da morte, quando juntamente no pó teremos descanso.

\chapter{18}

\par 1 Então, respondeu Bildade, o suíta:
\par 2 Até quando andarás à caça de palavras? Considera bem, e, então, falaremos.
\par 3 Por que somos reputados por animais, e aos teus olhos passamos por curtos de inteligência?
\par 4 Oh! Tu, que te despedaças na tua ira, será a terra abandonada por tua causa? Remover-se-ão as rochas do seu lugar?
\par 5 Na verdade, a luz do perverso se apagará, e para seu fogo não resplandecerá a faísca;
\par 6 a luz se escurecerá nas suas tendas, e a sua lâmpada sobre ele se apagará;
\par 7 os seus passos fortes se estreitarão, e a sua própria trama o derribará.
\par 8 Porque por seus próprios pés é lançado na rede e andará na boca de forje.
\par 9 A armadilha o apanhará pelo calcanhar, e o laço o prenderá.
\par 10 A corda está-lhe escondida na terra, e a armadilha, na vereda.
\par 11 Os assombros o espantarão de todos os lados e o perseguirão a cada passo.
\par 12 A calamidade virá faminta sobre ele, e a miséria estará alerta ao seu lado,
\par 13 a qual lhe devorará os membros do corpo; serão devorados pelo primogênito da morte.
\par 14 O perverso será arrancado da sua tenda, onde está confiado, e será levado ao rei dos terrores.
\par 15 Nenhum dos seus morará na sua tenda, espalhar-se-á enxofre sobre a sua habitação.
\par 16 Por baixo secarão as suas raízes, e murcharão por cima os seus ramos.
\par 17 A sua memória desaparecerá da terra, e pelas praças não terá nome.
\par 18 Da luz o lançarão nas trevas e o afugentarão do mundo.
\par 19 Não terá filho nem posteridade entre o seu povo, nem sobrevivente algum ficará nas suas moradas.
\par 20 Do seu dia se espantarão os do Ocidente, e os do Oriente serão tomados de horror.
\par 21 Tais são, na verdade, as moradas do perverso, e este é o paradeiro do que não conhece a Deus.

\chapter{19}

\par 1 Então, respondeu Jó:
\par 2 Até quando afligireis a minha alma e me quebrantareis com palavras?
\par 3 Já dez vezes me vituperastes e não vos envergonhais de injuriar-me.
\par 4 Embora haja eu, na verdade, errado, comigo ficará o meu erro.
\par 5 Se quereis engrandecer-vos contra mim e me argüis pelo meu opróbrio,
\par 6 sabei agora que Deus é que me oprimiu e com a sua rede me cercou.
\par 7 Eis que clamo: violência! Mas não sou ouvido; grito: socorro! Porém não há justiça.
\par 8 O meu caminho ele fechou, e não posso passar; e nas minhas veredas pôs trevas.
\par 9 Da minha honra me despojou e tirou-me da cabeça a coroa.
\par 10 Arruinou-me de todos os lados, e eu me vou; e arrancou-me a esperança, como a uma árvore.
\par 11 Inflamou contra mim a sua ira e me tem na conta de seu adversário.
\par 12 Juntas vieram as suas tropas, prepararam contra mim o seu caminho e se acamparam ao redor da minha tenda.
\par 13 Pôs longe de mim a meus irmãos, e os que me conhecem, como estranhos, se apartaram de mim.
\par 14 Os meus parentes me desampararam, e os meus conhecidos se esqueceram de mim.
\par 15 Os que se abrigam na minha casa e as minhas servas me têm por estranho, e vim a ser estrangeiro aos seus olhos.
\par 16 Chamo o meu criado, e ele não me responde; tenho de suplicar-lhe, eu mesmo.
\par 17 O meu hálito é intolerável à minha mulher, e pelo mau cheiro sou repugnante aos filhos de minha mãe.
\par 18 Até as crianças me desprezam, e, querendo eu levantar-me, zombam de mim.
\par 19 Todos os meus amigos íntimos me abominam, e até os que eu amava se tornaram contra mim.
\par 20 Os meus ossos se apegam à minha pele e à minha carne, e salvei-me só com a pele dos meus dentes.
\par 21 Compadecei-vos de mim, amigos meus, compadecei-vos de mim, porque a mão de Deus me atingiu.
\par 22 Por que me perseguis como Deus me persegue e não cessais de devorar a minha carne?
\par 23 Quem me dera fossem agora escritas as minhas palavras! Quem me dera fossem gravadas em livro!
\par 24 Que, com pena de ferro e com chumbo, para sempre fossem esculpidas na rocha!
\par 25 Porque eu sei que o meu Redentor vive e por fim se levantará sobre a terra.
\par 26 Depois, revestido este meu corpo da minha pele, em minha carne verei a Deus.
\par 27 Vê-lo-ei por mim mesmo, os meus olhos o verão, e não outros; de saudade me desfalece o coração dentro de mim.
\par 28 Se disserdes: Como o perseguiremos? E: A causa deste mal se acha nele,
\par 29 temei, pois, a espada, porque tais acusações merecem o seu furor, para saberdes que há um juízo.

\chapter{20}

\par 1 Então, respondeu Zofar, o naamatita:
\par 2 Visto que os meus pensamentos me impõem resposta, eu me apresso.
\par 3 Eu ouvi a repreensão, que me envergonha, mas o meu espírito me obriga a responder segundo o meu entendimento.
\par 4 Porventura, não sabes tu que desde todos os tempos, desde que o homem foi posto sobre a terra,
\par 5 o júbilo dos perversos é breve, e a alegria dos ímpios, momentânea?
\par 6 Ainda que a sua presunção remonte aos céus, e a sua cabeça atinja as nuvens,
\par 7 como o seu próprio esterco, apodrecerá para sempre; e os que o conheceram dirão: Onde está?
\par 8 Voará como um sonho e não será achado, será afugentado como uma visão da noite.
\par 9 Os olhos que o viram jamais o verão, e o seu lugar não o verá outra vez.
\par 10 Os seus filhos procurarão aplacar aos pobres, e as suas mãos lhes restaurarão os seus bens.
\par 11 Ainda que os seus ossos estejam cheios do vigor da sua juventude, esse vigor se deitará com ele no pó.
\par 12 Ainda que o mal lhe seja doce na boca, e ele o esconda debaixo da língua,
\par 13 e o saboreie, e o não deixe; antes, o retenha no seu paladar,
\par 14 contudo, a sua comida se transformará nas suas entranhas; fel de áspides será no seu interior.
\par 15 Engoliu riquezas, mas vomitá-las-á; do seu ventre Deus as lançará.
\par 16 Veneno de áspides sorveu; língua de víbora o matará.
\par 17 Não se deliciará com a vista dos ribeiros e dos rios transbordantes de mel e de leite.
\par 18 Devolverá o fruto do seu trabalho e não o engolirá; do lucro de sua barganha não tirará prazer nenhum.
\par 19 Oprimiu e desamparou os pobres, roubou casas que não edificou.
\par 20 Por não haver limites à sua cobiça, não chegará a salvar as coisas por ele desejadas.
\par 21 Nada escapou à sua cobiça insaciável, pelo que a sua prosperidade não durará.
\par 22 Na plenitude da sua abastança, ver-se-á angustiado; toda a força da miséria virá sobre ele.
\par 23 Para encher a sua barriga, Deus mandará sobre ele o furor da sua ira, que, por alimento, mandará chover sobre ele.
\par 24 Se fugir das armas de ferro, o arco de bronze o traspassará.
\par 25 Ele arranca das suas costas a flecha, e esta vem resplandecente do seu fel; e haverá assombro sobre ele.
\par 26 Todas as calamidades serão reservadas contra os seus tesouros; fogo não assoprado o consumirá, fogo que se apascentará do que ficar na sua tenda.
\par 27 Os céus lhe manifestarão a sua iniqüidade; e a terra se levantará contra ele.
\par 28 As riquezas de sua casa serão transportadas; como água serão derramadas no dia da ira de Deus.
\par 29 Tal é, da parte de Deus, a sorte do homem perverso, tal a herança decretada por Deus.

\chapter{21}

\par 1 Respondeu, porém, Jó:
\par 2 Ouvi atentamente as minhas razões, e já isso me será a vossa consolação.
\par 3 Tolerai-me, e eu falarei; e, havendo eu falado, podereis zombar.
\par 4 Acaso, é do homem que eu me queixo? Não tenho motivo de me impacientar?
\par 5 Olhai para mim e pasmai; e ponde a mão sobre a boca;
\par 6 porque só de pensar nisso me perturbo, e um calafrio se apodera de toda a minha carne.
\par 7 Como é, pois, que vivem os perversos, envelhecem e ainda se tornam mais poderosos?
\par 8 Seus filhos se estabelecem na sua presença; e os seus descendentes, ante seus olhos.
\par 9 As suas casas têm paz, sem temor, e a vara de Deus não os fustiga.
\par 10 O seu touro gera e não falha, suas novilhas têm a cria e não abortam.
\par 11 Deixam correr suas crianças, como a um rebanho, e seus filhos saltam de alegria;
\par 12 cantam com tamboril e harpa e alegram-se ao som da flauta.
\par 13 Passam eles os seus dias em prosperidade e em paz descem à sepultura.
\par 14 E são estes os que disseram a Deus: Retira-te de nós! Não desejamos conhecer os teus caminhos.
\par 15 Que é o Todo-Poderoso, para que nós o sirvamos? E que nos aproveitará que lhe façamos orações?
\par 16 Vede, porém, que não provém deles a sua prosperidade; longe de mim o conselho dos perversos!
\par 17 Quantas vezes sucede que se apaga a lâmpada dos perversos? Quantas vezes lhes sobrevém a destruição? Quantas vezes Deus na sua ira lhes reparte dores?
\par 18 Quantas vezes são como a palha diante do vento e como a pragana arrebatada pelo remoinho?
\par 19 Deus, dizeis vós, guarda a iniqüidade do perverso para seus filhos. Mas é a ele que deveria Deus dar o pago, para que o sinta.
\par 20 Seus próprios olhos devem ver a sua ruína, e ele, beber do furor do Todo-Poderoso.
\par 21 Porque depois de morto, cortado já o número dos seus meses, que interessa a ele a sua casa?
\par 22 Acaso, alguém ensinará ciência a Deus, a ele que julga os que estão nos céus?
\par 23 Um morre em pleno vigor, despreocupado e tranqüilo,
\par 24 com seus baldes cheios de leite e fresca a medula dos seus ossos.
\par 25 Outro, ao contrário, morre na amargura do seu coração, não havendo provado do bem.
\par 26 Juntamente jazem no pó, onde os vermes os cobrem.
\par 27 Vede que conheço os vossos pensamentos e os injustos desígnios com que me tratais.
\par 28 Porque direis: Onde está a casa do príncipe, e onde, a tenda em que morava o perverso?
\par 29 Porventura, não tendes interrogado os que viajam? E não considerastes as suas declarações,
\par 30 que o mau é poupado no dia da calamidade, é socorrido no dia do furor?
\par 31 Quem lhe lançará em rosto o seu proceder? Quem lhe dará o pago do que faz?
\par 32 Finalmente, é levado à sepultura, e sobre o seu túmulo se faz vigilância.
\par 33 Os torrões do vale lhe são leves, todos os homens o seguem, assim como não têm número os que foram adiante dele.
\par 34 Como, pois, me consolais em vão? Das vossas respostas só resta falsidade.

\chapter{22}

\par 1 Então, respondeu Elifaz, o temanita:
\par 2 Porventura, será o homem de algum proveito a Deus? Antes, o sábio é só útil a si mesmo.
\par 3 Ou tem o Todo-Poderoso interesse em que sejas justo ou algum lucro em que faças perfeitos os teus caminhos?
\par 4 Ou te repreende pelo teu temor de Deus ou entra contra ti em juízo?
\par 5 Porventura, não é grande a tua malícia, e sem termo, as tuas iniqüidades?
\par 6 Porque sem causa tomaste penhores a teu irmão e aos seminus despojaste das suas roupas.
\par 7 Não deste água a beber ao cansado e ao faminto retiveste o pão.
\par 8 Ao braço forte pertencia a terra, e só os homens favorecidos habitavam nela.
\par 9 As viúvas despediste de mãos vazias, e os braços dos órfãos foram quebrados.
\par 10 Por isso, estás cercado de laços, e repentino pavor te conturba
\par 11 ou trevas, em que nada vês; e águas transbordantes te cobrem.
\par 12 Porventura, não está Deus nas alturas do céu? Olha para as estrelas mais altas. Que altura!
\par 13 E dizes: Que sabe Deus? Acaso, poderá ele julgar através de densa escuridão?
\par 14 Grossas nuvens o encobrem, de modo que não pode ver; ele passeia pela abóbada do céu.
\par 15 Queres seguir a rota antiga, que os homens iníquos pisaram?
\par 16 Estes foram arrebatados antes do tempo; o seu fundamento, uma torrente o arrasta.
\par 17 Diziam a Deus: Retira-te de nós. E: Que pode fazer-nos o Todo-Poderoso?
\par 18 Contudo, ele enchera de bens as suas casas. Longe de mim o conselho dos perversos!
\par 19 Os justos o vêem e se alegram, e o inocente escarnece deles,
\par 20 dizendo: Na verdade, os nossos adversários foram destruídos, e o fogo consumiu o resto deles.
\par 21 Reconcilia-te, pois, com ele e tem paz, e assim te sobrevirá o bem.
\par 22 Aceita, peço-te, a instrução que profere e põe as suas palavras no teu coração.
\par 23 Se te converteres ao Todo-Poderoso, serás restabelecido; se afastares a injustiça da tua tenda
\par 24 e deitares ao pó o teu ouro e o ouro de Ofir entre pedras dos ribeiros,
\par 25 então, o Todo-Poderoso será o teu ouro e a tua prata escolhida.
\par 26 Deleitar-te-ás, pois, no Todo-Poderoso e levantarás o rosto para Deus.
\par 27 Orarás a ele, e ele te ouvirá; e pagarás os teus votos.
\par 28 Se projetas alguma coisa, ela te sairá bem, e a luz brilhará em teus caminhos.
\par 29 Se estes descem, então, dirás: Para cima! E Deus salvará o humilde
\par 30 e livrará até ao que não é inocente; sim, será libertado, graças à pureza de tuas mãos.

\chapter{23}

\par 1 Respondeu, porém, Jó:
\par 2 Ainda hoje a minha queixa é de um revoltado, apesar de a minha mão reprimir o meu gemido.
\par 3 Ah! Se eu soubesse onde o poderia achar! Então, me chegaria ao seu tribunal.
\par 4 Exporia ante ele a minha causa, encheria a minha boca de argumentos.
\par 5 Saberia as palavras que ele me respondesse e entenderia o que me dissesse.
\par 6 Acaso, segundo a grandeza de seu poder, contenderia comigo? Não; antes, me atenderia.
\par 7 Ali, o homem reto pleitearia com ele, e eu me livraria para sempre do meu juiz.
\par 8 Eis que, se me adianto, ali não está; se torno para trás, não o percebo.
\par 9 Se opera à esquerda, não o vejo; esconde-se à direita, e não o diviso.
\par 10 Mas ele sabe o meu caminho; se ele me provasse, sairia eu como o ouro.
\par 11 Os meus pés seguiram as suas pisadas; guardei o seu caminho e não me desviei dele.
\par 12 Do mandamento de seus lábios nunca me apartei, escondi no meu íntimo as palavras da sua boca.
\par 13 Mas, se ele resolveu alguma coisa, quem o pode dissuadir? O que ele deseja, isso fará.
\par 14 Pois ele cumprirá o que está ordenado a meu respeito e muitas coisas como estas ainda tem consigo.
\par 15 Por isso, me perturbo perante ele; e, quando o considero, temo-o.
\par 16 Deus é quem me fez desmaiar o coração, e o Todo-Poderoso, quem me perturbou,
\par 17 porque não estou desfalecido por causa das trevas, nem porque a escuridão cobre o meu rosto.

\chapter{24}

\par 1 Por que o Todo-Poderoso não designa tempos de julgamento? E por que os que o conhecem não vêem tais dias?
\par 2 Há os que removem os limites, roubam os rebanhos e os apascentam.
\par 3 Levam do órfão o jumento, da viúva, tomam-lhe o boi.
\par 4 Desviam do caminho aos necessitados, e os pobres da terra todos têm de esconder-se.
\par 5 Como asnos monteses no deserto, saem estes para o seu mister, à procura de presa no campo aberto, como pão para eles e seus filhos.
\par 6 No campo segam o pasto do perverso e lhe rabiscam a vinha.
\par 7 Passam a noite nus por falta de roupa e não têm cobertas contra o frio.
\par 8 Pelas chuvas das montanhas são molhados e, não tendo refúgio, abraçam-se com as rochas.
\par 9 Orfãozinhos são arrancados ao peito, e dos pobres se toma penhor;
\par 10 de modo que estes andam nus, sem roupa, e, famintos, arrastam os molhos.
\par 11 Entre os muros desses perversos espremem o azeite, pisam-lhes o lagar; contudo, padecem sede.
\par 12 Desde as cidades gemem os homens, e a alma dos feridos clama; e, contudo, Deus não tem isso por anormal.
\par 13 Os perversos são inimigos da luz, não conhecem os seus caminhos, nem permanecem nas suas veredas.
\par 14 De madrugada se levanta o homicida, mata ao pobre e ao necessitado, e de noite se torna ladrão.
\par 15 Aguardam o crepúsculo os olhos do adúltero; este diz consigo: Ninguém me reconhecerá; e cobre o rosto.
\par 16 Nas trevas minam as casas, de dia se conservam encerrados, nada querem com a luz.
\par 17 Pois a manhã para todos eles é como sombra de morte; mas os terrores da noite lhes são familiares.
\par 18 Vós dizeis: Os perversos são levados rapidamente na superfície das águas; maldita é a porção dos tais na terra; já não andam pelo caminho das vinhas.
\par 19 A secura e o calor desfazem as águas da neve; assim faz a sepultura aos que pecaram.
\par 20 A mãe se esquecerá deles, os vermes os comerão gostosamente; nunca mais haverá lembrança deles; como árvore será quebrado o injusto,
\par 21 aquele que devora a estéril que não tem filhos e não faz o bem à viúva.
\par 22 Não! Pelo contrário, Deus por sua força prolonga os dias dos valentes; vêem-se eles de pé quando desesperavam da vida.
\par 23 Ele lhes dá descanso, e nisso se estribam; os olhos de Deus estão nos caminhos deles.
\par 24 São exaltados por breve tempo; depois, passam, colhidos como todos os mais; são cortados como as pontas das espigas.
\par 25 Se não é assim, quem me desmentirá e anulará as minhas razões?

\chapter{25}

\par 1 Então, respondeu Bildade, o suíta:
\par 2 A Deus pertence o domínio e o poder; ele faz reinar a paz nas alturas celestes.
\par 3 Acaso, têm número os seus exércitos? E sobre quem não se levanta a sua luz?
\par 4 Como, pois, seria justo o homem perante Deus, e como seria puro aquele que nasce de mulher?
\par 5 Eis que até a lua não tem brilho, e as estrelas não são puras aos olhos dele.
\par 6 Quanto menos o homem, que é gusano, e o filho do homem, que é verme!

\chapter{26}

\par 1 Jó, porém, respondeu:
\par 2 Como sabes ajudar ao que não tem força e prestar socorro ao braço que não tem vigor!
\par 3 Como sabes aconselhar ao que não tem sabedoria e revelar plenitude de verdadeiro conhecimento!
\par 4 Com a ajuda de quem proferes tais palavras? E de quem é o espírito que fala em ti?
\par 5 A alma dos mortos tremem debaixo das águas com seus habitantes.
\par 6 O além está desnudo perante ele, e não há coberta para o abismo.
\par 7 Ele estende o norte sobre o vazio e faz pairar a terra sobre o nada.
\par 8 Prende as águas em densas nuvens, e as nuvens não se rasgam debaixo delas.
\par 9 Encobre a face do seu trono e sobre ele estende a sua nuvem.
\par 10 Traçou um círculo à superfície das águas, até aos confins da luz e das trevas.
\par 11 As colunas do céu tremem e se espantam da sua ameaça.
\par 12 Com a sua força fende o mar e com o seu entendimento abate o adversário.
\par 13 Pelo seu sopro aclara os céus, a sua mão fere o dragão veloz.
\par 14 Eis que isto são apenas as orlas dos seus caminhos! Que leve sussurro temos ouvido dele! Mas o trovão do seu poder, quem o entenderá?

\chapter{27}

\par 1 Prosseguindo Jó em seu discurso, disse:
\par 2 Tão certo como vive Deus, que me tirou o direito, e o Todo-Poderoso, que amargurou a minha alma,
\par 3 enquanto em mim estiver a minha vida, e o sopro de Deus nos meus narizes,
\par 4 nunca os meus lábios falarão injustiça, nem a minha língua pronunciará engano.
\par 5 Longe de mim que eu vos dê razão! Até que eu expire, nunca afastarei de mim a minha integridade.
\par 6 À minha justiça me apegarei e não a largarei; não me reprova a minha consciência por qualquer dia da minha vida.
\par 7 Seja como o perverso o meu inimigo, e o que se levantar contra mim, como o injusto.
\par 8 Porque qual será a esperança do ímpio, quando lhe for cortada a vida, quando Deus lhe arrancar a alma?
\par 9 Acaso, ouvirá Deus o seu clamor, em lhe sobrevindo a tribulação?
\par 10 Deleitar-se-á o perverso no Todo-Poderoso e invocará a Deus em todo o tempo?
\par 11 Ensinar-vos-ei o que encerra a mão de Deus e não vos ocultarei o que está com o Todo-Poderoso.
\par 12 Eis que todos vós já vistes isso; por que, pois, alimentais vãs noções?
\par 13 Eis qual será da parte de Deus a porção do perverso e a herança que os opressores receberão do Todo-Poderoso:
\par 14 Se os seus filhos se multiplicarem, será para a espada, e a sua prole não se fartará de pão.
\par 15 Os que ficarem dela, a peste os enterrará, e as suas viúvas não chorarão.
\par 16 Se o perverso amontoar prata como pó e acumular vestes como barro,
\par 17 ele os acumulará, mas o justo é que os vestirá, e o inocente repartirá a prata.
\par 18 Ele edifica a sua casa como a da traça e como a choça que o vigia constrói.
\par 19 Rico se deita com a sua riqueza, abre os seus olhos e já não a vê.
\par 20 Pavores se apoderam dele como inundação, de noite a tempestade o arrebata.
\par 21 O vento oriental o leva, e ele se vai; varre-o com ímpeto do seu lugar.
\par 22 Deus lança isto sobre ele e não o poupa, a ele que procura fugir precipitadamente da sua mão;
\par 23 à sua queda lhe batem palmas, à saída o apupam com assobios.

\chapter{28}

\par 1 Na verdade, a prata tem suas minas, e o ouro, que se refina, o seu lugar.
\par 2 O ferro tira-se da terra, e da pedra se funde o cobre.
\par 3 Os homens põem termo à escuridão e até aos últimos confins procuram as pedras ocultas nas trevas e na densa escuridade.
\par 4 Abrem entrada para minas longe da habitação dos homens, esquecidos dos transeuntes; e, assim, longe deles, dependurados, oscilam de um lado para outro.
\par 5 Da terra procede o pão, mas embaixo é revolvida como por fogo.
\par 6 Nas suas pedras se encontra safira, e há pó que contém ouro.
\par 7 Essa vereda, a ave de rapina a ignora, e jamais a viram os olhos do falcão.
\par 8 Nunca a pisaram feras majestosas, nem o leãozinho passou por ela.
\par 9 Estende o homem a mão contra o rochedo e revolve os montes desde as suas raízes.
\par 10 Abre canais nas pedras, e os seus olhos vêem tudo o que há de mais precioso.
\par 11 Tapa os veios de água, e nem uma gota sai deles, e traz à luz o que estava escondido.
\par 12 Mas onde se achará a sabedoria? E onde está o lugar do entendimento?
\par 13 O homem não conhece o valor dela, nem se acha ela na terra dos viventes.
\par 14 O abismo diz: Ela não está em mim; e o mar diz: Não está comigo.
\par 15 Não se dá por ela ouro fino, nem se pesa prata em câmbio dela.
\par 16 O seu valor não se pode avaliar pelo ouro de Ofir, nem pelo precioso ônix, nem pela safira.
\par 17 O ouro não se iguala a ela, nem o cristal; ela não se trocará por jóia de ouro fino;
\par 18 ela faz esquecer o coral e o cristal; a aquisição da sabedoria é melhor que a das pérolas.
\par 19 Não se lhe igualará o topázio da Etiópia, nem se pode avaliar por ouro puro.
\par 20 Donde, pois, vem a sabedoria, e onde está o lugar do entendimento?
\par 21 Está encoberta aos olhos de todo vivente e oculta às aves do céu.
\par 22 O abismo e a morte dizem: Ouvimos com os nossos ouvidos a sua fama.
\par 23 Deus lhe entende o caminho, e ele é quem sabe o seu lugar.
\par 24 Porque ele perscruta até as extremidades da terra, vê tudo o que há debaixo dos céus.
\par 25 Quando regulou o peso do vento e fixou a medida das águas;
\par 26 quando determinou leis para a chuva e caminho para o relâmpago dos trovões,
\par 27 então, viu ele a sabedoria e a manifestou; estabeleceu-a e também a esquadrinhou.
\par 28 E disse ao homem: Eis que o temor do Senhor é a sabedoria, e o apartar-se do mal é o entendimento.

\chapter{29}

\par 1 Prosseguiu Jó no seu discurso e disse:
\par 2 Ah! Quem me dera ser como fui nos meses passados, como nos dias em que Deus me guardava!
\par 3 Quando fazia resplandecer a sua lâmpada sobre a minha cabeça, quando eu, guiado por sua luz, caminhava pelas trevas;
\par 4 como fui nos dias do meu vigor, quando a amizade de Deus estava sobre a minha tenda;
\par 5 quando o Todo-Poderoso ainda estava comigo, e os meus filhos, em redor de mim;
\par 6 quando eu lavava os pés em leite, e da rocha me corriam ribeiros de azeite.
\par 7 Quando eu saía para a porta da cidade, e na praça me era dado sentar-me,
\par 8 os moços me viam e se retiravam; os idosos se levantavam e se punham em pé;
\par 9 os príncipes reprimiam as suas palavras e punham a mão sobre a boca;
\par 10 a voz dos nobres emudecia, e a sua língua se apegava ao paladar.
\par 11 Ouvindo-me algum ouvido, esse me chamava feliz; vendo-me algum olho, dava testemunho de mim;
\par 12 porque eu livrava os pobres que clamavam e também o órfão que não tinha quem o socorresse.
\par 13 A bênção do que estava a perecer vinha sobre mim, e eu fazia rejubilar-se o coração da viúva.
\par 14 Eu me cobria de justiça, e esta me servia de veste; como manto e turbante era a minha eqüidade.
\par 15 Eu me fazia de olhos para o cego e de pés para o coxo.
\par 16 Dos necessitados era pai e até as causas dos desconhecidos eu examinava.
\par 17 Eu quebrava os queixos do iníquo e dos seus dentes lhe fazia eu cair a vítima.
\par 18 Eu dizia: no meu ninho expirarei, multiplicarei os meus dias como a areia.
\par 19 A minha raiz se estenderá até às águas, e o orvalho ficará durante a noite sobre os meus ramos;
\par 20 a minha honra se renovará em mim, e o meu arco se reforçará na minha mão.
\par 21 Os que me ouviam esperavam o meu conselho e guardavam silêncio para ouvi-lo.
\par 22 Havendo eu falado, não replicavam; as minhas palavras caíam sobre eles como orvalho.
\par 23 Esperavam-me como à chuva, abriam a boca como à chuva de primavera.
\par 24 Sorria-me para eles quando não tinham confiança; e a luz do meu rosto não desprezavam.
\par 25 Eu lhes escolhia o caminho, assentava-me como chefe e habitava como rei entre as suas tropas, como quem consola os que pranteiam.

\chapter{30}

\par 1 Mas agora se riem de mim os de menos idade do que eu, e cujos pais eu teria desdenhado de pôr ao lado dos cães do meu rebanho.
\par 2 De que também me serviria a força das suas mãos, homens cujo vigor já pereceu?
\par 3 De míngua e fome se debilitaram; roem os lugares secos, desde muito em ruínas e desolados.
\par 4 Apanham malvas e folhas dos arbustos e se sustentam de raízes de zimbro.
\par 5 Do meio dos homens são expulsos; grita-se contra eles, como se grita atrás de um ladrão;
\par 6 habitam nos desfiladeiros sombrios, nas cavernas da terra e das rochas.
\par 7 Bramam entre os arbustos e se ajuntam debaixo dos espinheiros.
\par 8 São filhos de doidos, raça infame, e da terra são escorraçados.
\par 9 Mas agora sou a sua canção de motejo e lhes sirvo de provérbio.
\par 10 Abominam-me, fogem para longe de mim e não se abstêm de me cuspir no rosto.
\par 11 Porque Deus afrouxou a corda do meu arco e me oprimiu; pelo que sacudiram de si o freio perante o meu rosto.
\par 12 À direita se levanta uma súcia, e me empurra, e contra mim prepara o seu caminho de destruição.
\par 13 Arruínam a minha vereda, promovem a minha calamidade; gente para quem já não há socorro.
\par 14 Vêm contra mim como por uma grande brecha e se revolvem avante entre as ruínas.
\par 15 Sobrevieram-me pavores, como pelo vento é varrida a minha honra; como nuvem passou a minha felicidade.
\par 16 Agora, dentro de mim se me derrama a alma; os dias da aflição se apoderaram de mim.
\par 17 A noite me verruma os ossos e os desloca, e não descansa o mal que me rói.
\par 18 Pela grande violência do meu mal está desfigurada a minha veste, mal que me cinge como a gola da minha túnica.
\par 19 Deus, tu me lançaste na lama, e me tornei semelhante ao pó e à cinza.
\par 20 Clamo a ti, e não me respondes; estou em pé, mas apenas olhas para mim.
\par 21 Tu foste cruel comigo; com a força da tua mão tu me combates.
\par 22 Levantas-me sobre o vento e me fazes cavalgá-lo; dissolves-me no estrondo da tempestade.
\par 23 Pois eu sei que me levarás à morte e à casa destinada a todo vivente.
\par 24 De um montão de ruínas não estenderá o homem a mão e na sua desventura não levantará um grito por socorro?
\par 25 Acaso, não chorei sobre aquele que atravessava dias difíceis ou não se angustiou a minha alma pelo necessitado?
\par 26 Aguardava eu o bem, e eis que me veio o mal; esperava a luz, veio-me a escuridão.
\par 27 O meu íntimo se agita sem cessar; e dias de aflição me sobrevêm.
\par 28 Ando de luto, sem a luz do sol; levanto-me na congregação e clamo por socorro.
\par 29 Sou irmão dos chacais e companheiro de avestruzes.
\par 30 Enegrecida se me cai a pele, e os meus ossos queimam em febre.
\par 31 Por isso, a minha harpa se me tornou em prantos de luto, e a minha flauta, em voz dos que choram.

\chapter{31}

\par 1 Fiz aliança com meus olhos; como, pois, os fixaria eu numa donzela?
\par 2 Que porção, pois, teria eu do Deus lá de cima e que herança, do Todo-Poderoso desde as alturas?
\par 3 Acaso, não é a perdição para o iníquo, e o infortúnio, para os que praticam a maldade?
\par 4 Ou não vê Deus os meus caminhos e não conta todos os meus passos?
\par 5 Se andei com falsidade, e se o meu pé se apressou para o engano
\par 6 (pese-me Deus em balanças fiéis e conhecerá a minha integridade);
\par 7 se os meus passos se desviaram do caminho, e se o meu coração segue os meus olhos, e se às minhas mãos se apegou qualquer mancha,
\par 8 então, semeie eu, e outro coma, e sejam arrancados os renovos do meu campo.
\par 9 Se o meu coração se deixou seduzir por causa de mulher, se andei à espreita à porta do meu próximo,
\par 10 então, moa minha mulher para outro, e outros se encurvem sobre ela.
\par 11 Pois seria isso um crime hediondo, delito à punição de juízes;
\par 12 pois seria fogo que consome até à destruição e desarraigaria toda a minha renda.
\par 13 Se desprezei o direito do meu servo ou da minha serva, quando eles contendiam comigo,
\par 14 então, que faria eu quando Deus se levantasse? E, inquirindo ele a causa, que lhe responderia eu?
\par 15 Aquele que me formou no ventre materno não os fez também a eles? Ou não é o mesmo que nos formou na madre?
\par 16 Se retive o que os pobres desejavam ou fiz desfalecer os olhos da viúva;
\par 17 ou, se sozinho comi o meu bocado, e o órfão dele não participou
\par 18 (Porque desde a minha mocidade cresceu comigo como se eu lhe fora o pai, e desde o ventre da minha mãe fui o guia da viúva.);
\par 19 se a alguém vi perecer por falta de roupa e ao necessitado, por não ter coberta;
\par 20 se os seus lombos não me abençoaram, se ele não se aquentava com a lã dos meus cordeiros;
\par 21 se eu levantei a mão contra o órfão, por me ver apoiado pelos juízes da porta,
\par 22 então, caia a omoplata do meu ombro, e seja arrancado o meu braço da articulação.
\par 23 Porque o castigo de Deus seria para mim um assombro, e eu não poderia enfrentar a sua majestade.
\par 24 Se no ouro pus a minha esperança ou disse ao ouro fino: em ti confio;
\par 25 se me alegrei por serem grandes os meus bens e por ter a minha mão alcançado muito;
\par 26 se olhei para o sol, quando resplandecia, ou para a lua, que caminhava esplendente,
\par 27 e o meu coração se deixou enganar em oculto, e beijos lhes atirei com a mão,
\par 28 também isto seria delito à punição de juízes; pois assim negaria eu ao Deus lá de cima.
\par 29 Se me alegrei da desgraça do que me tem ódio e se exultei quando o mal o atingiu
\par 30 (Também não deixei pecar a minha boca, pedindo com imprecações a sua morte.);
\par 31 se a gente da minha tenda não disse: Ah! Quem haverá aí que não se saciou de carne provida por ele
\par 32 (O estrangeiro não pernoitava na rua; as minhas portas abria ao viandante.)!
\par 33 Se, como Adão, encobri as minhas transgressões, ocultando o meu delito no meu seio;
\par 34 porque eu temia a grande multidão, e o desprezo das famílias me apavorava, de sorte que me calei e não saí da porta.
\par 35 Tomara eu tivesse quem me ouvisse! Eis aqui a minha defesa assinada! Que o Todo-Poderoso me responda! Que o meu adversário escreva a sua acusação!
\par 36 Por certo que a levaria sobre o meu ombro, atá-la-ia sobre mim como coroa;
\par 37 mostrar-lhe-ia o número dos meus passos; como príncipe me chegaria a ele.
\par 38 Se a minha terra clamar contra mim, e se os seus sulcos juntamente chorarem;
\par 39 se comi os seus frutos sem tê-la pago devidamente e causei a morte aos seus donos,
\par 40 por trigo me produza cardos, e por cevada, joio. Fim das palavras de Jó.

\chapter{32}

\par 1 Cessaram aqueles três homens de responder a Jó no tocante ao se ter ele por justo aos seus próprios olhos.
\par 2 Então, se acendeu a ira de Eliú, filho de Baraquel, o buzita, da família de Rão; acendeu-se a sua ira contra Jó, porque este pretendia ser mais justo do que Deus.
\par 3 Também a sua ira se acendeu contra os três amigos, porque, mesmo não achando eles o que responder, condenavam a Jó.
\par 4 Eliú, porém, esperara para falar a Jó, pois eram de mais idade do que ele.
\par 5 Vendo Eliú que já não havia resposta na boca daqueles três homens, a sua ira se acendeu.
\par 6 Disse Eliú, filho de Baraquel, o buzita: Eu sou de menos idade, e vós sois idosos; arreceei-me e temi de vos declarar a minha opinião.
\par 7 Dizia eu: Falem os dias, e a multidão dos anos ensine a sabedoria.
\par 8 Na verdade, há um espírito no homem, e o sopro do Todo-Poderoso o faz sábio.
\par 9 Os de mais idade não é que são os sábios, nem os velhos, os que entendem o que é reto.
\par 10 Pelo que digo: dai-me ouvidos, e também eu declararei a minha opinião.
\par 11 Eis que aguardei as vossas palavras e dei ouvidos às vossas considerações, enquanto, quem sabe, buscáveis o que dizer.
\par 12 Atentando, pois, para vós outros, eis que nenhum de vós houve que refutasse a Jó, nem que respondesse às suas razões.
\par 13 Não vos desculpeis, pois, dizendo: Achamos sabedoria nele; Deus pode vencê-lo, e não o homem.
\par 14 Ora, ele não me dirigiu palavra alguma, nem eu lhe retorquirei com as vossas palavras.
\par 15 Jó, os três estão pasmados, já não respondem, faltam-lhes as palavras.
\par 16 Acaso, devo esperar, pois não falam, estão parados e nada mais respondem?
\par 17 Também eu concorrerei com a minha resposta; declararei a minha opinião.
\par 18 Porque tenho muito que falar, e o meu espírito me constrange.
\par 19 Eis que dentro de mim sou como o vinho, sem respiradouro, como odres novos, prestes a arrebentar-se.
\par 20 Permiti, pois, que eu fale para desafogar-me; abrirei os lábios e responderei.
\par 21 Não farei acepção de pessoas, nem usarei de lisonjas com o homem.
\par 22 Porque não sei lisonjear; em caso contrário, em breve me levaria o meu Criador.

\chapter{33}

\par 1 Ouve, pois, Jó, as minhas razões e dá ouvidos a todas as minhas palavras.
\par 2 Passo agora a falar, em minha boca fala a língua.
\par 3 As minhas razões provam a sinceridade do meu coração, e os meus lábios proferem o puro saber.
\par 4 O Espírito de Deus me fez, e o sopro do Todo-Poderoso me dá vida.
\par 5 Se podes, contesta-me, dispõe bem as tuas razões perante mim e apresenta-te.
\par 6 Eis que diante de Deus sou como tu és; também eu sou formado do barro.
\par 7 Por isso, não te inspiro terror, nem será pesada sobre ti a minha mão.
\par 8 Na verdade, falaste perante mim, e eu ouvi o som das tuas palavras:
\par 9 Estou limpo, sem transgressão; puro sou e não tenho iniqüidade.
\par 10 Eis que Deus procura pretextos contra mim e me considera como seu inimigo.
\par 11 Põe no tronco os meus pés e observa todas as minhas veredas.
\par 12 Nisto não tens razão, eu te respondo; porque Deus é maior do que o homem.
\par 13 Por que contendes com ele, afirmando que não te dá contas de nenhum dos seus atos?
\par 14 Pelo contrário, Deus fala de um modo, sim, de dois modos, mas o homem não atenta para isso.
\par 15 Em sonho ou em visão de noite, quando cai sono profundo sobre os homens, quando adormecem na cama,
\par 16 então, lhes abre os ouvidos e lhes sela a sua instrução,
\par 17 para apartar o homem do seu desígnio e livrá-lo da soberba;
\par 18 para guardar a sua alma da cova e a sua vida de passar pela espada.
\par 19 Também no seu leito é castigado com dores, com incessante contenda nos seus ossos;
\par 20 de modo que a sua vida abomina o pão, e a sua alma, a comida apetecível.
\par 21 A sua carne, que se via, agora desaparece, e os seus ossos, que não se viam, agora se descobrem.
\par 22 A sua alma se vai chegando à cova, e a sua vida, aos portadores da morte.
\par 23 Se com ele houver um anjo intercessor, um dos milhares, para declarar ao homem o que lhe convém,
\par 24 então, Deus terá misericórdia dele e dirá ao anjo: Redime-o, para que não desça à cova; achei resgate.
\par 25 Sua carne se robustecerá com o vigor da sua infância, e ele tornará aos dias da sua juventude.
\par 26 Deveras orará a Deus, que lhe será propício; ele, com júbilo, verá a face de Deus, e este lhe restituirá a sua justiça.
\par 27 Cantará diante dos homens e dirá: Pequei, perverti o direito e não fui punido segundo merecia.
\par 28 Deus redimiu a minha alma de ir para a cova; e a minha vida verá a luz.
\par 29 Eis que tudo isto é obra de Deus, duas e três vezes para com o homem,
\par 30 para reconduzir da cova a sua alma e o alumiar com a luz dos viventes.
\par 31 Escuta, pois, ó Jó, ouve-me; cala-te, e eu falarei.
\par 32 Se tens alguma coisa que dizer, responde-me; fala, porque desejo justificar-te.
\par 33 Se não, escuta-me; cala-te, e ensinar-te-ei a sabedoria.

\chapter{34}

\par 1 Disse mais Eliú:
\par 2 Ouvi, ó sábios, as minhas razões; vós, instruídos, inclinai os ouvidos para mim.
\par 3 Porque o ouvido prova as palavras, como o paladar, a comida.
\par 4 O que é direito escolhamos para nós; conheçamos entre nós o que é bom.
\par 5 Porque Jó disse: Sou justo, e Deus tirou o meu direito.
\par 6 Apesar do meu direito, sou tido por mentiroso; a minha ferida é incurável, sem que haja pecado em mim.
\par 7 Que homem há como Jó, que bebe a zombaria como água?
\par 8 E anda em companhia dos que praticam a iniqüidade e caminha com homens perversos?
\par 9 Pois disse: De nada aproveita ao homem o comprazer-se em Deus.
\par 10 Pelo que vós, homens sensatos, escutai-me: longe de Deus o praticar ele a perversidade, e do Todo-Poderoso o cometer injustiça.
\par 11 Pois retribui ao homem segundo as suas obras e faz que a cada um toque segundo o seu caminho.
\par 12 Na verdade, Deus não procede maliciosamente; nem o Todo-Poderoso perverte o juízo.
\par 13 Quem lhe entregou o governo da terra? Quem lhe confiou o universo?
\par 14 Se Deus pensasse apenas em si mesmo e para si recolhesse o seu espírito e o seu sopro,
\par 15 toda a carne juntamente expiraria, e o homem voltaria para o pó.
\par 16 Se, pois, há em ti entendimento, ouve isto; inclina os ouvidos ao som das minhas palavras.
\par 17 Acaso, governaria o que aborrecesse o direito? E quererás tu condenar aquele que é justo e poderoso?
\par 18 Dir-se-á a um rei: Oh! Vil? Ou aos príncipes: Oh! Perversos?
\par 19 Quanto menos àquele que não faz acepção das pessoas de príncipes, nem estima ao rico mais do que ao pobre; porque todos são obra de suas mãos.
\par 20 De repente, morrem; à meia-noite, os povos são perturbados e passam, e os poderosos são tomados por força invisível.
\par 21 Os olhos de Deus estão sobre os caminhos do homem e vêem todos os seus passos.
\par 22 Não há trevas nem sombra assaz profunda, onde se escondam os que praticam a iniqüidade.
\par 23 Pois Deus não precisa observar por muito tempo o homem antes de o fazer ir a juízo perante ele.
\par 24 Quebranta os fortes, sem os inquirir, e põe outros em seu lugar.
\par 25 Ele conhece, pois, as suas obras; de noite, os transtorna, e ficam moídos.
\par 26 Ele os fere como a perversos, à vista de todos;
\par 27 porque dele se desviaram, e não quiseram compreender nenhum de seus caminhos,
\par 28 e, assim, fizeram que o clamor do pobre subisse até Deus, e este ouviu o lamento dos aflitos.
\par 29 Se ele aquietar-se, quem o condenará? Se encobrir o rosto, quem o poderá contemplar, seja um povo, seja um homem?
\par 30 Para que o ímpio não reine, e não haja quem iluda o povo.
\par 31 Se alguém diz a Deus: Sofri, não pecarei mais;
\par 32 o que não vejo, ensina-mo tu; se cometi injustiça, jamais a tornarei a praticar,
\par 33 acaso, deve ele recompensar-te segundo tu queres ou não queres? Acaso, deve ele dizer-te: Escolhe tu, e não eu; declara o que sabes, fala?
\par 34 Os homens sensatos dir-me-ão, dir-me-á o sábio que me ouve:
\par 35 Jó falou sem conhecimento, e nas suas palavras não há sabedoria.
\par 36 Tomara fosse Jó provado até ao fim, porque ele respondeu como homem de iniqüidade.
\par 37 Pois ao seu pecado acrescenta rebelião, entre nós, com desprezo, bate ele palmas e multiplica as suas palavras contra Deus.

\chapter{35}

\par 1 Disse mais Eliú:
\par 2 Achas que é justo dizeres: Maior é a minha justiça do que a de Deus?
\par 3 Porque dizes: De que me serviria ela? Que proveito tiraria dela mais do que do meu pecado?
\par 4 Dar-te-ei resposta, a ti e aos teus amigos contigo.
\par 5 Atenta para os céus e vê; contempla as altas nuvens acima de ti.
\par 6 Se pecas, que mal lhe causas tu? Se as tuas transgressões se multiplicam, que lhe fazes?
\par 7 Se és justo, que lhe dás ou que recebe ele da tua mão?
\par 8 A tua impiedade só pode fazer o mal ao homem como tu mesmo; e a tua justiça, dar proveito ao filho do homem.
\par 9 Por causa das muitas opressões, os homens clamam, clamam por socorro contra o braço dos poderosos.
\par 10 Mas ninguém diz: Onde está Deus, que me fez, que inspira canções de louvor durante a noite,
\par 11 que nos ensina mais do que aos animais da terra e nos faz mais sábios do que as aves dos céus?
\par 12 Clamam, porém ele não responde, por causa da arrogância dos maus.
\par 13 Só gritos vazios Deus não ouvirá, nem atentará para eles o Todo-Poderoso.
\par 14 Jó, ainda que dizes que não o vês, a tua causa está diante dele; por isso, espera nele.
\par 15 Mas agora, porque Deus na sua ira não está punindo, nem fazendo muito caso das transgressões,
\par 16 abres a tua boca, com palavras vãs, amontoando frases de ignorante.

\chapter{36}

\par 1 Prosseguiu Eliú e disse:
\par 2 Mais um pouco de paciência, e te mostrarei que ainda tenho argumentos a favor de Deus.
\par 3 De longe trarei o meu conhecimento e ao meu Criador atribuirei a justiça.
\par 4 Porque, na verdade, as minhas palavras não são falsas; contigo está quem é senhor do assunto.
\par 5 Eis que Deus é mui grande; contudo a ninguém despreza; é grande na força da sua compreensão.
\par 6 Não poupa a vida ao perverso, mas faz justiça aos aflitos.
\par 7 Dos justos não tira os olhos; antes, com os reis, no trono os assenta para sempre, e são exaltados.
\par 8 Se estão presos em grilhões e amarrados com cordas de aflição,
\par 9 ele lhes faz ver as suas obras, as suas transgressões, e que se houveram com soberba.
\par 10 Abre-lhes também os ouvidos para a instrução e manda-lhes que se convertam da iniqüidade.
\par 11 Se o ouvirem e o servirem, acabarão seus dias em felicidade e os seus anos em delícias.
\par 12 Porém, se não o ouvirem, serão traspassados pela lança e morrerão na sua cegueira.
\par 13 Os ímpios de coração amontoam para si a ira; e, agrilhoados por Deus, não clamam por socorro.
\par 14 Perdem a vida na sua mocidade e morrem entre os prostitutos cultuais.
\par 15 Ao aflito livra por meio da sua aflição e pela opressão lhe abre os ouvidos.
\par 16 Assim também procura tirar-te das fauces da angústia para um lugar espaçoso, em que não há aperto, e as iguarias da tua mesa seriam cheias de gordura;
\par 17 mas tu te enches do juízo do perverso, e, por isso, o juízo e a justiça te alcançarão.
\par 18 Guarda-te, pois, de que a ira não te induza a escarnecer, nem te desvie a grande quantia do resgate.
\par 19 Estimaria ele as tuas lamúrias e todos os teus grandes esforços, para que te vejas livre da tua angústia?
\par 20 Não suspires pela noite, em que povos serão tomados do seu lugar.
\par 21 Guarda-te, não te inclines para a iniqüidade; pois isso preferes à tua miséria.
\par 22 Eis que Deus se mostra grande em seu poder! Quem é mestre como ele?
\par 23 Quem lhe prescreveu o seu caminho ou quem lhe pode dizer: Praticaste a injustiça?
\par 24 Lembra-te de lhe magnificares as obras que os homens celebram.
\par 25 Todos os homens as contemplam; de longe as admira o homem.
\par 26 Eis que Deus é grande, e não o podemos compreender; o número dos seus anos não se pode calcular.
\par 27 Porque atrai para si as gotas de água que de seu vapor destilam em chuva,
\par 28 a qual as nuvens derramam e gotejam sobre o homem abundantemente.
\par 29 Acaso, pode alguém entender o estender-se das nuvens e os trovões do seu pavilhão?
\par 30 Eis que estende sobre elas o seu relâmpago e encobre as profundezas do mar.
\par 31 Pois por estas coisas julga os povos e lhes dá mantimento em abundância.
\par 32 Enche as mãos de relâmpagos e os dardeja contra o adversário.
\par 33 O fragor da tempestade dá notícias a respeito dele, dele que é zeloso na sua ira contra a injustiça.

\chapter{37}

\par 1 Sobre isto treme também o meu coração e salta do seu lugar.
\par 2 Dai ouvidos ao trovão de Deus, estrondo que sai da sua boca;
\par 3 ele o solta por debaixo de todos os céus, e o seu relâmpago, até aos confins da terra.
\par 4 Depois deste, ruge a sua voz, troveja com o estrondo da sua majestade, e já ele não retém o relâmpago quando lhe ouvem a voz.
\par 5 Com a sua voz troveja Deus maravilhosamente; faz grandes coisas, que nós não compreendemos.
\par 6 Porque ele diz à neve: Cai sobre a terra; e à chuva e ao aguaceiro: Sede fortes.
\par 7 Assim, torna ele inativas as mãos de todos os homens, para que reconheçam as obras dele.
\par 8 E as alimárias entram nos seus esconderijos e ficam nas suas cavernas.
\par 9 De suas recâmaras sai o pé-de-vento, e, dos ventos do norte, o frio.
\par 10 Pelo sopro de Deus se dá a geada, e as largas águas se congelam.
\par 11 Também de umidade carrega as densas nuvens, nuvens que espargem os relâmpagos.
\par 12 Então, elas, segundo o rumo que ele dá, se espalham para uma e outra direção, para fazerem tudo o que lhes ordena sobre a redondeza da terra.
\par 13 E tudo isso faz ele vir para disciplina, se convém à terra, ou para exercer a sua misericórdia.
\par 14 Inclina, Jó, os ouvidos a isto, pára e considera as maravilhas de Deus.
\par 15 Porventura, sabes tu como Deus as opera e como faz resplandecer o relâmpago da sua nuvem?
\par 16 Tens tu notícia do equilíbrio das nuvens e das maravilhas daquele que é perfeito em conhecimento?
\par 17 Que faz aquecer as tuas vestes, quando há calma sobre a terra por causa do vento sul?
\par 18 Ou estendeste com ele o firmamento, que é sólido como espelho fundido?
\par 19 Ensina-nos o que lhe diremos; porque nós, envoltos em trevas, nada lhe podemos expor.
\par 20 Contar-lhe-ia alguém o que tenho dito? Seria isso desejar o homem ser devorado.
\par 21 Eis que o homem não pode olhar para o sol, que brilha no céu, uma vez passado o vento que o deixa limpo.
\par 22 Do norte vem o áureo esplendor, pois Deus está cercado de tremenda majestade.
\par 23 Ao Todo-Poderoso, não o podemos alcançar; ele é grande em poder, porém não perverte o juízo e a plenitude da justiça.
\par 24 Por isso, os homens o temem; ele não olha para os que se julgam sábios.

\chapter{38}

\par 1 Depois disto, o SENHOR, do meio de um redemoinho, respondeu a Jó:
\par 2 Quem é este que escurece os meus desígnios com palavras sem conhecimento?
\par 3 Cinge, pois, os lombos como homem, pois eu te perguntarei, e tu me farás saber.
\par 4 Onde estavas tu, quando eu lançava os fundamentos da terra? Dize-mo, se tens entendimento.
\par 5 Quem lhe pôs as medidas, se é que o sabes? Ou quem estendeu sobre ela o cordel?
\par 6 Sobre que estão fundadas as suas bases ou quem lhe assentou a pedra angular,
\par 7 quando as estrelas da alva, juntas, alegremente cantavam, e rejubilavam todos os filhos de Deus?
\par 8 Ou quem encerrou o mar com portas, quando irrompeu da madre;
\par 9 quando eu lhe pus as nuvens por vestidura e a escuridão por fraldas?
\par 10 Quando eu lhe tracei limites, e lhe pus ferrolhos e portas,
\par 11 e disse: até aqui virás e não mais adiante, e aqui se quebrará o orgulho das tuas ondas?
\par 12 Acaso, desde que começaram os teus dias, deste ordem à madrugada ou fizeste a alva saber o seu lugar,
\par 13 para que se apegasse às orlas da terra, e desta fossem os perversos sacudidos?
\par 14 A terra se modela como o barro debaixo do selo, e tudo se apresenta como vestidos;
\par 15 dos perversos se desvia a sua luz, e o braço levantado para ferir se quebranta.
\par 16 Acaso, entraste nos mananciais do mar ou percorreste o mais profundo do abismo?
\par 17 Porventura, te foram reveladas as portas da morte ou viste essas portas da região tenebrosa?
\par 18 Tens idéia nítida da largura da terra? Dize-mo, se o sabes.
\par 19 Onde está o caminho para a morada da luz? E, quanto às trevas, onde é o seu lugar,
\par 20 para que as conduzas aos seus limites e discirnas as veredas para a sua casa?
\par 21 Tu o sabes, porque nesse tempo eras nascido e porque é grande o número dos teus dias!
\par 22 Acaso, entraste nos depósitos da neve e viste os tesouros da saraiva,
\par 23 que eu retenho até ao tempo da angústia, até ao dia da peleja e da guerra?
\par 24 Onde está o caminho para onde se difunde a luz e se espalha o vento oriental sobre a terra?
\par 25 Quem abriu regos para o aguaceiro ou caminho para os relâmpagos dos trovões;
\par 26 para que se faça chover sobre a terra, onde não há ninguém, e no ermo, em que não há gente;
\par 27 para dessedentar a terra deserta e assolada e para fazer crescer os renovos da erva?
\par 28 Acaso, a chuva tem pai? Ou quem gera as gotas do orvalho?
\par 29 De que ventre procede o gelo? E quem dá à luz a geada do céu?
\par 30 As águas ficam duras como a pedra, e a superfície das profundezas se torna compacta.
\par 31 Ou poderás tu atar as cadeias do Sete-estrelo ou soltar os laços do Órion?
\par 32 Ou fazer aparecer os signos do Zodíaco ou guiar a Ursa com seus filhos?
\par 33 Sabes tu as ordenanças dos céus, podes estabelecer a sua influência sobre a terra?
\par 34 Podes levantar a tua voz até às nuvens, para que a abundância das águas te cubra?
\par 35 Ou ordenarás aos relâmpagos que saiam e te digam: Eis-nos aqui?
\par 36 Quem pôs sabedoria nas camadas de nuvens? Ou quem deu entendimento ao meteoro?
\par 37 Quem pode numerar com sabedoria as nuvens? Ou os odres dos céus, quem os pode despejar,
\par 38 para que o pó se transforme em massa sólida, e os torrões se apeguem uns aos outros?
\par 39 Caçarás, porventura, a presa para a leoa? Ou saciarás a fome dos leõezinhos,
\par 40 quando se agacham nos covis e estão à espreita nas covas?
\par 41 Quem prepara aos corvos o seu alimento, quando os seus pintainhos gritam a Deus e andam vagueando, por não terem que comer?

\chapter{39}

\par 1 Sabes tu o tempo em que as cabras monteses têm os filhos ou cuidaste das corças quando dão suas crias?
\par 2 Podes contar os meses que cumprem? Ou sabes o tempo do seu parto?
\par 3 Elas encurvam-se, para terem seus filhos, e lançam de si as suas dores.
\par 4 Seus filhos se tornam robustos, crescem no campo aberto, saem e nunca mais tornam para elas.
\par 5 Quem despediu livre o jumento selvagem, e quem soltou as prisões ao asno veloz,
\par 6 ao qual dei o ermo por casa e a terra salgada por moradas?
\par 7 Ri-se do tumulto da cidade, não ouve os muitos gritos do arrieiro.
\par 8 Os montes são o lugar do seu pasto, e anda à procura de tudo o que está verde.
\par 9 Acaso, quer o boi selvagem servir-te? Ou passará ele a noite junto da tua manjedoura?
\par 10 Porventura, podes prendê-lo ao sulco com cordas? Ou gradará ele os vales após ti?
\par 11 Confiarás nele, por ser grande a sua força, ou deixarás a seu cuidado o teu trabalho?
\par 12 Fiarás dele que te traga para a casa o que semeaste e o recolha na tua eira?
\par 13 O avestruz bate alegre as asas; acaso, porém, tem asas e penas de bondade?
\par 14 Ele deixa os seus ovos na terra, e os aquenta no pó,
\par 15 e se esquece de que algum pé os pode esmagar ou de que podem pisá-los os animais do campo.
\par 16 Trata com dureza os seus filhos, como se não fossem seus; embora seja em vão o seu trabalho, ele está tranqüilo,
\par 17 porque Deus lhe negou sabedoria e não lhe deu entendimento;
\par 18 mas, quando de um salto se levanta para correr, ri-se do cavalo e do cavaleiro.
\par 19 Ou dás tu força ao cavalo ou revestirás o seu pescoço de crinas?
\par 20 Acaso, o fazes pular como ao gafanhoto? Terrível é o fogoso respirar das suas ventas.
\par 21 Escarva no vale, folga na sua força e sai ao encontro dos armados.
\par 22 Ri-se do temor e não se espanta; e não torna atrás por causa da espada.
\par 23 Sobre ele chocalha a aljava, flameja a lança e o dardo.
\par 24 De fúria e ira devora o caminho e não se contém ao som da trombeta.
\par 25 Em cada sonido da trombeta, ele diz: Avante! Cheira de longe a batalha, o trovão dos príncipes e o alarido.
\par 26 Ou é pela tua inteligência que voa o falcão, estendendo as asas para o Sul?
\par 27 Ou é pelo teu mandado que se remonta a águia e faz alto o seu ninho?
\par 28 Habita no penhasco onde faz a sua morada, sobre o cimo do penhasco, em lugar seguro.
\par 29 Dali, descobre a presa; seus olhos a avistam de longe.
\par 30 Seus filhos chupam sangue; onde há mortos, ela aí está.

\chapter{40}

\par 1 Disse mais o SENHOR a Jó:
\par 2 Acaso, quem usa de censuras contenderá com o Todo-Poderoso? Quem assim argúi a Deus que responda.
\par 3 Então, Jó respondeu ao SENHOR e disse:
\par 4 Sou indigno; que te responderia eu? Ponho a mão na minha boca.
\par 5 Uma vez falei e não replicarei, aliás, duas vezes, porém não prosseguirei.
\par 6 Então, o SENHOR, do meio de um redemoinho, respondeu a Jó:
\par 7 Cinge agora os lombos como homem; eu te perguntarei, e tu me responderás.
\par 8 Acaso, anularás tu, de fato, o meu juízo? Ou me condenarás, para te justificares?
\par 9 Ou tens braço como Deus ou podes trovejar com a voz como ele o faz?
\par 10 Orna-te, pois, de excelência e grandeza, veste-te de majestade e de glória.
\par 11 Derrama as torrentes da tua ira e atenta para todo soberbo e abate-o.
\par 12 Olha para todo soberbo e humilha-o, calca aos pés os perversos no seu lugar.
\par 13 Cobre-os juntamente no pó, encerra-lhes o rosto no sepulcro.
\par 14 Então, também eu confessarei a teu respeito que a tua mão direita te dá vitória.
\par 15 Contempla agora o hipopótamo, que eu criei contigo, que come a erva como o boi.
\par 16 Sua força está nos seus lombos, e o seu poder, nos músculos do seu ventre.
\par 17 Endurece a sua cauda como cedro; os tendões das suas coxas estão entretecidos.
\par 18 Os seus ossos são como tubos de bronze, o seu arcabouço, como barras de ferro.
\par 19 Ele é obra-prima dos feitos de Deus; quem o fez o proveu de espada.
\par 20 Em verdade, os montes lhe produzem pasto, onde todos os animais do campo folgam.
\par 21 Deita-se debaixo dos lotos, no esconderijo dos canaviais e da lama.
\par 22 Os lotos o cobrem com sua sombra; os salgueiros do ribeiro o cercam.
\par 23 Se um rio transborda, ele não se apressa; fica tranqüilo ainda que o Jordão se levante até à sua boca.
\par 24 Acaso, pode alguém apanhá-lo quando ele está olhando? Ou lhe meter um laço pelo nariz?

\chapter{41}

\par 1 Podes tu, com anzol, apanhar o crocodilo ou lhe travar a língua com uma corda?
\par 2 Podes meter-lhe no nariz uma vara de junco? Ou furar-lhe as bochechas com um gancho?
\par 3 Acaso, te fará muitas súplicas? Ou te falará palavras brandas?
\par 4 Fará ele acordo contigo? Ou tomá-lo-ás por servo para sempre?
\par 5 Brincarás com ele, como se fora um passarinho? Ou tê-lo-ás preso à correia para as tuas meninas?
\par 6 Acaso, os teus sócios negociam com ele? Ou o repartirão entre os mercadores?
\par 7 Encher-lhe-ás a pele de arpões? Ou a cabeça, de farpas?
\par 8 Põe a mão sobre ele, lembra-te da peleja e nunca mais o intentarás.
\par 9 Eis que a gente se engana em sua esperança; acaso, não será o homem derribado só em vê-lo?
\par 10 Ninguém há tão ousado, que se atreva a despertá-lo. Quem é, pois, aquele que pode erguer-se diante de mim?
\par 11 Quem primeiro me deu a mim, para que eu haja de retribuir-lhe? Pois o que está debaixo de todos os céus é meu.
\par 12 Não me calarei a respeito dos seus membros, nem da sua grande força, nem da graça da sua compostura.
\par 13 Quem lhe abrirá as vestes do seu dorso? Ou lhe penetrará a couraça dobrada?
\par 14 Quem abriria as portas do seu rosto? Pois em roda dos seus dentes está o terror.
\par 15 As fileiras de suas escamas são o seu orgulho, cada uma bem encostada como por um selo que as ajusta.
\par 16 A tal ponto uma se chega à outra, que entre elas não entra nem o ar.
\par 17 Umas às outras se ligam, aderem entre si e não se podem separar.
\par 18 Cada um dos seus espirros faz resplandecer luz, e os seus olhos são como as pestanas da alva.
\par 19 Da sua boca saem tochas; faíscas de fogo saltam dela.
\par 20 Das suas narinas procede fumaça, como de uma panela fervente ou de juncos que ardem.
\par 21 O seu hálito faz incender os carvões; e da sua boca sai chama.
\par 22 No seu pescoço reside a força; e diante dele salta o desespero.
\par 23 Suas partes carnudas são bem pegadas entre si; todas fundidas nele e imóveis.
\par 24 O seu coração é firme como uma pedra, firme como a mó de baixo.
\par 25 Levantando-se ele, tremem os valentes; quando irrompe, ficam como que fora de si.
\par 26 Se o golpe de espada o alcança, de nada vale, nem de lança, de dardo ou de flecha.
\par 27 Para ele, o ferro é palha, e o cobre, pau podre.
\par 28 A seta o não faz fugir; as pedras das fundas se lhe tornam em restolho.
\par 29 Os porretes atirados são para ele como palha, e ri-se do brandir da lança.
\par 30 Debaixo do ventre, há escamas pontiagudas; arrasta-se sobre a lama, como um instrumento de debulhar.
\par 31 As profundezas faz ferver, como uma panela; torna o mar como caldeira de ungüento.
\par 32 Após si, deixa um sulco luminoso; o abismo parece ter-se encanecido.
\par 33 Na terra, não tem ele igual, pois foi feito para nunca ter medo.
\par 34 Ele olha com desprezo tudo o que é alto; é rei sobre todos os animais orgulhosos.

\chapter{42}

\par 1 Então, respondeu Jó ao SENHOR:
\par 2 Bem sei que tudo podes, e nenhum dos teus planos pode ser frustrado.
\par 3 Quem é aquele, como disseste, que sem conhecimento encobre o conselho? Na verdade, falei do que não entendia; coisas maravilhosas demais para mim, coisas que eu não conhecia.
\par 4 Escuta-me, pois, havias dito, e eu falarei; eu te perguntarei, e tu me ensinarás.
\par 5 Eu te conhecia só de ouvir, mas agora os meus olhos te vêem.
\par 6 Por isso, me abomino e me arrependo no pó e na cinza.
\par 7 Tendo o SENHOR falado estas palavras a Jó, o SENHOR disse também a Elifaz, o temanita: A minha ira se acendeu contra ti e contra os teus dois amigos; porque não dissestes de mim o que era reto, como o meu servo Jó.
\par 8 Tomai, pois, sete novilhos e sete carneiros, e ide ao meu servo Jó, e oferecei holocaustos por vós. O meu servo Jó orará por vós; porque dele aceitarei a intercessão, para que eu não vos trate segundo a vossa loucura; porque vós não dissestes de mim o que era reto, como o meu servo Jó.
\par 9 Então, foram Elifaz, o temanita, e Bildade, o suíta, e Zofar, o naamatita, e fizeram como o SENHOR lhes ordenara; e o SENHOR aceitou a oração de Jó.
\par 10 Mudou o SENHOR a sorte de Jó, quando este orava pelos seus amigos; e o SENHOR deu-lhe o dobro de tudo o que antes possuíra.
\par 11 Então, vieram a ele todos os seus irmãos, e todas as suas irmãs, e todos quantos dantes o conheceram, e comeram com ele em sua casa, e se condoeram dele, e o consolaram de todo o mal que o SENHOR lhe havia enviado; cada um lhe deu dinheiro e um anel de ouro.
\par 12 Assim, abençoou o SENHOR o último estado de Jó mais do que o primeiro; porque veio a ter catorze mil ovelhas, seis mil camelos, mil juntas de bois e mil jumentas.
\par 13 Também teve outros sete filhos e três filhas.
\par 14 Chamou o nome da primeira Jemima, o da outra, Quezia, e o da terceira, Quéren-Hapuque.
\par 15 Em toda aquela terra não se acharam mulheres tão formosas como as filhas de Jó; e seu pai lhes deu herança entre seus irmãos.
\par 16 Depois disto, viveu Jó cento e quarenta anos; e viu a seus filhos e aos filhos de seus filhos, até à quarta geração.
\par 17 Então, morreu Jó, velho e farto de dias.


\end{document}