\begin{document}

\title{Provérbios}


\chapter{1}

\par 1 Provérbios de Salomão, filho de Davi, o rei de Israel.
\par 2 Para aprender a sabedoria e o ensino; para entender as palavras de inteligência;
\par 3 para obter o ensino do bom proceder, a justiça, o juízo e a eqüidade;
\par 4 para dar aos simples prudência e aos jovens, conhecimento e bom siso.
\par 5 Ouça o sábio e cresça em prudência; e o instruído adquira habilidade
\par 6 para entender provérbios e parábolas, as palavras e enigmas dos sábios.
\par 7 O temor do SENHOR é o princípio do saber, mas os loucos desprezam a sabedoria e o ensino.
\par 8 Filho meu, ouve o ensino de teu pai e não deixes a instrução de tua mãe.
\par 9 Porque serão diadema de graça para a tua cabeça e colares, para o teu pescoço.
\par 10 Filho meu, se os pecadores querem seduzir-te, não o consintas.
\par 11 Se disserem: Vem conosco, embosquemo-nos para derramar sangue, espreitemos, ainda que sem motivo, os inocentes;
\par 12 traguemo-los vivos, como o abismo, e inteiros, como os que descem à cova;
\par 13 acharemos toda sorte de bens preciosos; encheremos de despojos a nossa casa;
\par 14 lança a tua sorte entre nós; teremos todos uma só bolsa.
\par 15 Filho meu, não te ponhas a caminho com eles; guarda das suas veredas os pés;
\par 16 porque os seus pés correm para o mal e se apressam a derramar sangue.
\par 17 Pois debalde se estende a rede à vista de qualquer ave.
\par 18 Estes se emboscam contra o seu próprio sangue e a sua própria vida espreitam.
\par 19 Tal é a sorte de todo ganancioso; e este espírito de ganância tira a vida de quem o possui.
\par 20 Grita na rua a Sabedoria, nas praças, levanta a voz;
\par 21 do alto dos muros clama, à entrada das portas e nas cidades profere as suas palavras:
\par 22 Até quando, ó néscios, amareis a necedade? E vós, escarnecedores, desejareis o escárnio? E vós, loucos, aborrecereis o conhecimento?
\par 23 Atentai para a minha repreensão; eis que derramarei copiosamente para vós outros o meu espírito e vos farei saber as minhas palavras.
\par 24 Mas, porque clamei, e vós recusastes; porque estendi a mão, e não houve quem atendesse;
\par 25 antes, rejeitastes todo o meu conselho e não quisestes a minha repreensão;
\par 26 também eu me rirei na vossa desventura, e, em vindo o vosso terror, eu zombarei,
\par 27 em vindo o vosso terror como a tempestade, em vindo a vossa perdição como o redemoinho, quando vos chegar o aperto e a angústia.
\par 28 Então, me invocarão, mas eu não responderei; procurar-me-ão, porém não me hão de achar.
\par 29 Porquanto aborreceram o conhecimento e não preferiram o temor do SENHOR;
\par 30 não quiseram o meu conselho e desprezaram toda a minha repreensão.
\par 31 Portanto, comerão do fruto do seu procedimento e dos seus próprios conselhos se fartarão.
\par 32 Os néscios são mortos por seu desvio, e aos loucos a sua impressão de bem-estar os leva à perdição.
\par 33 Mas o que me der ouvidos habitará seguro, tranqüilo e sem temor do mal.

\chapter{2}

\par 1 Filho meu, se aceitares as minhas palavras e esconderes contigo os meus mandamentos,
\par 2 para fazeres atento à sabedoria o teu ouvido e para inclinares o coração ao entendimento,
\par 3 e, se clamares por inteligência, e por entendimento alçares a voz,
\par 4 se buscares a sabedoria como a prata e como a tesouros escondidos a procurares,
\par 5 então, entenderás o temor do SENHOR e acharás o conhecimento de Deus.
\par 6 Porque o SENHOR dá a sabedoria, e da sua boca vem a inteligência e o entendimento.
\par 7 Ele reserva a verdadeira sabedoria para os retos; é escudo para os que caminham na sinceridade,
\par 8 guarda as veredas do juízo e conserva o caminho dos seus santos.
\par 9 Então, entenderás justiça, juízo e eqüidade, todas as boas veredas.
\par 10 Porquanto a sabedoria entrará no teu coração, e o conhecimento será agradável à tua alma.
\par 11 O bom siso te guardará, e a inteligência te conservará;
\par 12 para te livrar do caminho do mal e do homem que diz coisas perversas;
\par 13 dos que deixam as veredas da retidão, para andarem pelos caminhos das trevas;
\par 14 que se alegram de fazer o mal, folgam com as perversidades dos maus,
\par 15 seguem veredas tortuosas e se desviam nos seus caminhos;
\par 16 para te livrar da mulher adúltera, da estrangeira, que lisonjeia com palavras,
\par 17 a qual deixa o amigo da sua mocidade e se esquece da aliança do seu Deus;
\par 18 porque a sua casa se inclina para a morte, e as suas veredas, para o reino das sombras da morte;
\par 19 todos os que se dirigem a essa mulher não voltarão e não atinarão com as veredas da vida.
\par 20 Assim, andarás pelo caminho dos homens de bem e guardarás as veredas dos justos.
\par 21 Porque os retos habitarão a terra, e os íntegros permanecerão nela.
\par 22 Mas os perversos serão eliminados da terra, e os aleivosos serão dela desarraigados.

\chapter{3}

\par 1 Filho meu, não te esqueças dos meus ensinos, e o teu coração guarde os meus mandamentos;
\par 2 porque eles aumentarão os teus dias e te acrescentarão anos de vida e paz.
\par 3 Não te desamparem a benignidade e a fidelidade; ata-as ao pescoço; escreve-as na tábua do teu coração
\par 4 e acharás graça e boa compreensão diante de Deus e dos homens.
\par 5 Confia no SENHOR de todo o teu coração e não te estribes no teu próprio entendimento.
\par 6 Reconhece-o em todos os teus caminhos, e ele endireitará as tuas veredas.
\par 7 Não sejas sábio aos teus próprios olhos; teme ao SENHOR e aparta-te do mal;
\par 8 será isto saúde para o teu corpo e refrigério, para os teus ossos.
\par 9 Honra ao SENHOR com os teus bens e com as primícias de toda a tua renda;
\par 10 e se encherão fartamente os teus celeiros, e transbordarão de vinho os teus lagares.
\par 11 Filho meu, não rejeites a disciplina do SENHOR, nem te enfades da sua repreensão.
\par 12 Porque o SENHOR repreende a quem ama, assim como o pai, ao filho a quem quer bem.
\par 13 Feliz o homem que acha sabedoria, e o homem que adquire conhecimento;
\par 14 porque melhor é o lucro que ela dá do que o da prata, e melhor a sua renda do que o ouro mais fino.
\par 15 Mais preciosa é do que pérolas, e tudo o que podes desejar não é comparável a ela.
\par 16 O alongar-se da vida está na sua mão direita, na sua esquerda, riquezas e honra.
\par 17 Os seus caminhos são caminhos deliciosos, e todas as suas veredas, paz.
\par 18 É árvore de vida para os que a alcançam, e felizes são todos os que a retêm.
\par 19 O SENHOR com sabedoria fundou a terra, com inteligência estabeleceu os céus.
\par 20 Pelo seu conhecimento os abismos se rompem, e as nuvens destilam orvalho.
\par 21 Filho meu, não se apartem estas coisas dos teus olhos; guarda a verdadeira sabedoria e o bom siso;
\par 22 porque serão vida para a tua alma e adorno ao teu pescoço.
\par 23 Então, andarás seguro no teu caminho, e não tropeçará o teu pé.
\par 24 Quando te deitares, não temerás; deitar-te-ás, e o teu sono será suave.
\par 25 Não temas o pavor repentino, nem a arremetida dos perversos, quando vier.
\par 26 Porque o SENHOR será a tua segurança e guardará os teus pés de serem presos.
\par 27 Não te furtes a fazer o bem a quem de direito, estando na tua mão o poder de fazê-lo.
\par 28 Não digas ao teu próximo: Vai e volta amanhã; então, to darei, se o tens agora contigo.
\par 29 Não maquines o mal contra o teu próximo, pois habita junto de ti confiadamente.
\par 30 Jamais pleiteies com alguém sem razão, se te não houver feito mal.
\par 31 Não tenhas inveja do homem violento, nem sigas nenhum de seus caminhos;
\par 32 porque o SENHOR abomina o perverso, mas aos retos trata com intimidade.
\par 33 A maldição do SENHOR habita na casa do perverso, porém a morada dos justos ele abençoa.
\par 34 Certamente, ele escarnece dos escarnecedores, mas dá graça aos humildes.
\par 35 Os sábios herdarão honra, mas os loucos tomam sobre si a ignomínia.

\chapter{4}

\par 1 Ouvi, filhos, a instrução do pai e estai atentos para conhecerdes o entendimento;
\par 2 porque vos dou boa doutrina; não deixeis o meu ensino.
\par 3 Quando eu era filho em companhia de meu pai, tenro e único diante de minha mãe,
\par 4 então, ele me ensinava e me dizia: Retenha o teu coração as minhas palavras; guarda os meus mandamentos e vive;
\par 5 adquire a sabedoria, adquire o entendimento e não te esqueças das palavras da minha boca, nem delas te apartes.
\par 6 Não desampares a sabedoria, e ela te guardará; ama-a, e ela te protegerá.
\par 7 O princípio da sabedoria é: Adquire a sabedoria; sim, com tudo o que possuis, adquire o entendimento.
\par 8 Estima-a, e ela te exaltará; se a abraçares, ela te honrará;
\par 9 dará à tua cabeça um diadema de graça e uma coroa de glória te entregará.
\par 10 Ouve, filho meu, e aceita as minhas palavras, e se te multiplicarão os anos de vida.
\par 11 No caminho da sabedoria, te ensinei e pelas veredas da retidão te fiz andar.
\par 12 Em andando por elas, não se embaraçarão os teus passos; se correres, não tropeçarás.
\par 13 Retém a instrução e não a largues; guarda-a, porque ela é a tua vida.
\par 14 Não entres na vereda dos perversos, nem sigas pelo caminho dos maus.
\par 15 Evita-o; não passes por ele; desvia-te dele e passa de largo;
\par 16 pois não dormem, se não fizerem mal, e foge deles o sono, se não fizerem tropeçar alguém;
\par 17 porque comem o pão da impiedade e bebem o vinho das violências.
\par 18 Mas a vereda dos justos é como a luz da aurora, que vai brilhando mais e mais até ser dia perfeito.
\par 19 O caminho dos perversos é como a escuridão; nem sabem eles em que tropeçam.
\par 20 Filho meu, atenta para as minhas palavras; aos meus ensinamentos inclina os ouvidos.
\par 21 Não os deixes apartar-se dos teus olhos; guarda-os no mais íntimo do teu coração.
\par 22 Porque são vida para quem os acha e saúde, para o seu corpo.
\par 23 Sobre tudo o que se deve guardar, guarda o coração, porque dele procedem as fontes da vida.
\par 24 Desvia de ti a falsidade da boca e afasta de ti a perversidade dos lábios.
\par 25 Os teus olhos olhem direito, e as tuas pálpebras, diretamente diante de ti.
\par 26 Pondera a vereda de teus pés, e todos os teus caminhos sejam retos.
\par 27 Não declines nem para a direita nem para a esquerda; retira o teu pé do mal.

\chapter{5}

\par 1 Filho meu, atende a minha sabedoria; à minha inteligência inclina os ouvidos
\par 2 para que conserves a discrição, e os teus lábios guardem o conhecimento;
\par 3 porque os lábios da mulher adúltera destilam favos de mel, e as suas palavras são mais suaves do que o azeite;
\par 4 mas o fim dela é amargoso como o absinto, agudo, como a espada de dois gumes.
\par 5 Os seus pés descem à morte; os seus passos conduzem-na ao inferno.
\par 6 Ela não pondera a vereda da vida; anda errante nos seus caminhos e não o sabe.
\par 7 Agora, pois, filho, dá-me ouvidos e não te desvies das palavras da minha boca.
\par 8 Afasta o teu caminho da mulher adúltera e não te aproximes da porta da sua casa;
\par 9 para que não dês a outrem a tua honra, nem os teus anos, a cruéis;
\par 10 para que dos teus bens não se fartem os estranhos, e o fruto do teu trabalho não entre em casa alheia;
\par 11 e gemas no fim de tua vida, quando se consumirem a tua carne e o teu corpo,
\par 12 e digas: Como aborreci o ensino! E desprezou o meu coração a disciplina!
\par 13 E não escutei a voz dos que me ensinavam, nem a meus mestres inclinei os ouvidos!
\par 14 Quase que me achei em todo mal que sucedeu no meio da assembléia e da congregação.
\par 15 Bebe a água da tua própria cisterna e das correntes do teu poço.
\par 16 Derramar-se-iam por fora as tuas fontes, e, pelas praças, os ribeiros de águas?
\par 17 Sejam para ti somente e não para os estranhos contigo.
\par 18 Seja bendito o teu manancial, e alegra-te com a mulher da tua mocidade,
\par 19 corça de amores e gazela graciosa. Saciem-te os seus seios em todo o tempo; e embriaga-te sempre com as suas carícias.
\par 20 Por que, filho meu, andarias cego pela estranha e abraçarias o peito de outra?
\par 21 Porque os caminhos do homem estão perante os olhos do SENHOR, e ele considera todas as suas veredas.
\par 22 Quanto ao perverso, as suas iniqüidades o prenderão, e com as cordas do seu pecado será detido.
\par 23 Ele morrerá pela falta de disciplina, e, pela sua muita loucura, perdido, cambaleia.

\chapter{6}

\par 1 Filho meu, se ficaste por fiador do teu companheiro e se te empenhaste ao estranho,
\par 2 estás enredado com o que dizem os teus lábios, estás preso com as palavras da tua boca.
\par 3 Agora, pois, faze isto, filho meu, e livra-te, pois caíste nas mãos do teu companheiro: vai, prostra-te e importuna o teu companheiro;
\par 4 não dês sono aos teus olhos, nem repouso às tuas pálpebras;
\par 5 livra-te, como a gazela, da mão do caçador e, como a ave, da mão do passarinheiro.
\par 6 Vai ter com a formiga, ó preguiçoso, considera os seus caminhos e sê sábio.
\par 7 Não tendo ela chefe, nem oficial, nem comandante,
\par 8 no estio, prepara o seu pão, na sega, ajunta o seu mantimento.
\par 9 Ó preguiçoso, até quando ficarás deitado? Quando te levantarás do teu sono?
\par 10 Um pouco para dormir, um pouco para tosquenejar, um pouco para encruzar os braços em repouso,
\par 11 assim sobrevirá a tua pobreza como um ladrão, e a tua necessidade, como um homem armado.
\par 12 O homem de Belial, o homem vil, é o que anda com a perversidade na boca,
\par 13 acena com os olhos, arranha com os pés e faz sinais com os dedos.
\par 14 No seu coração há perversidade; todo o tempo maquina o mal; anda semeando contendas.
\par 15 Pelo que a sua destruição virá repentinamente; subitamente, será quebrantado, sem que haja cura.
\par 16 Seis coisas o SENHOR aborrece, e a sétima a sua alma abomina:
\par 17 olhos altivos, língua mentirosa, mãos que derramam sangue inocente,
\par 18 coração que trama projetos iníquos, pés que se apressam a correr para o mal,
\par 19 testemunha falsa que profere mentiras e o que semeia contendas entre irmãos.
\par 20 Filho meu, guarda o mandamento de teu pai e não deixes a instrução de tua mãe;
\par 21 ata-os perpetuamente ao teu coração, pendura-os ao pescoço.
\par 22 Quando caminhares, isso te guiará; quando te deitares, te guardará; quando acordares, falará contigo.
\par 23 Porque o mandamento é lâmpada, e a instrução, luz; e as repreensões da disciplina são o caminho da vida;
\par 24 para te guardarem da vil mulher e das lisonjas da mulher alheia.
\par 25 Não cobices no teu coração a sua formosura, nem te deixes prender com as suas olhadelas.
\par 26 Por uma prostituta o máximo que se paga é um pedaço de pão, mas a adúltera anda à caça de vida preciosa.
\par 27 Tomará alguém fogo no seio, sem que as suas vestes se incendeiem?
\par 28 Ou andará alguém sobre brasas, sem que se queimem os seus pés?
\par 29 Assim será com o que se chegar à mulher do seu próximo; não ficará sem castigo todo aquele que a tocar.
\par 30 Não é certo que se despreza o ladrão, quando furta para saciar-se, tendo fome?
\par 31 Pois este, quando encontrado, pagará sete vezes tanto; entregará todos os bens de sua casa.
\par 32 O que adultera com uma mulher está fora de si; só mesmo quem quer arruinar-se é que pratica tal coisa.
\par 33 Achará açoites e infâmia, e o seu opróbrio nunca se apagará.
\par 34 Porque o ciúme excita o furor do marido; e não terá compaixão no dia da vingança.
\par 35 Não se contentará com o resgate, nem aceitará presentes, ainda que sejam muitos.

\chapter{7}

\par 1 Filho meu, guarda as minhas palavras e conserva dentro de ti os meus mandamentos.
\par 2 Guarda os meus mandamentos e vive; e a minha lei, como a menina dos teus olhos.
\par 3 Ata-os aos dedos, escreve-os na tábua do teu coração.
\par 4 Dize à Sabedoria: Tu és minha irmã; e ao Entendimento chama teu parente;
\par 5 para te guardarem da mulher alheia, da estranha que lisonjeia com palavras.
\par 6 Porque da janela da minha casa, por minhas grades, olhando eu,
\par 7 vi entre os simples, descobri entre os jovens um que era carecente de juízo,
\par 8 que ia e vinha pela rua junto à esquina da mulher estranha e seguia o caminho da sua casa,
\par 9 à tarde do dia, no crepúsculo, na escuridão da noite, nas trevas.
\par 10 Eis que a mulher lhe sai ao encontro, com vestes de prostituta e astuta de coração.
\par 11 É apaixonada e inquieta, cujos pés não param em casa;
\par 12 ora está nas ruas, ora, nas praças, espreitando por todos os cantos.
\par 13 Aproximou-se dele, e o beijou, e de cara impudente lhe diz:
\par 14 Sacrifícios pacíficos tinha eu de oferecer; paguei hoje os meus votos.
\par 15 Por isso, saí ao teu encontro, a buscar-te, e te achei.
\par 16 Já cobri de colchas a minha cama, de linho fino do Egito, de várias cores;
\par 17 já perfumei o meu leito com mirra, aloés e cinamomo.
\par 18 Vem, embriaguemo-nos com as delícias do amor, até pela manhã; gozemos amores.
\par 19 Porque o meu marido não está em casa, saiu de viagem para longe.
\par 20 Levou consigo um saquitel de dinheiro; só por volta da lua cheia ele tornará para casa.
\par 21 Seduziu-o com as suas muitas palavras, com as lisonjas dos seus lábios o arrastou.
\par 22 E ele num instante a segue, como o boi que vai ao matadouro; como o cervo que corre para a rede,
\par 23 até que a flecha lhe atravesse o coração; como a ave que se apressa para o laço, sem saber que isto lhe custará a vida.
\par 24 Agora, pois, filho, dá-me ouvidos e sê atento às palavras da minha boca;
\par 25 não se desvie o teu coração para os caminhos dela, e não andes perdido nas suas veredas;
\par 26 porque a muitos feriu e derribou; e são muitos os que por ela foram mortos.
\par 27 A sua casa é caminho para a sepultura e desce para as câmaras da morte.

\chapter{8}

\par 1 Não clama, porventura, a Sabedoria, e o Entendimento não faz ouvir a sua voz?
\par 2 No cimo das alturas, junto ao caminho, nas encruzilhadas das veredas ela se coloca;
\par 3 junto às portas, à entrada da cidade, à entrada das portas está gritando:
\par 4 A vós outros, ó homens, clamo; e a minha voz se dirige aos filhos dos homens.
\par 5 Entendei, ó simples, a prudência; e vós, néscios, entendei a sabedoria.
\par 6 Ouvi, pois falarei coisas excelentes; os meus lábios proferirão coisas retas.
\par 7 Porque a minha boca proclamará a verdade; os meus lábios abominam a impiedade.
\par 8 São justas todas as palavras da minha boca; não há nelas nenhuma coisa torta, nem perversa.
\par 9 Todas são retas para quem as entende e justas, para os que acham o conhecimento.
\par 10 Aceitai o meu ensino, e não a prata, e o conhecimento, antes do que o ouro escolhido.
\par 11 Porque melhor é a sabedoria do que jóias, e de tudo o que se deseja nada se pode comparar com ela.
\par 12 Eu, a Sabedoria, habito com a prudência e disponho de conhecimentos e de conselhos.
\par 13 O temor do SENHOR consiste em aborrecer o mal; a soberba, a arrogância, o mau caminho e a boca perversa, eu os aborreço.
\par 14 Meu é o conselho e a verdadeira sabedoria, eu sou o Entendimento, minha é a fortaleza.
\par 15 Por meu intermédio, reinam os reis, e os príncipes decretam justiça.
\par 16 Por meu intermédio, governam os príncipes, os nobres e todos os juízes da terra.
\par 17 Eu amo os que me amam; os que me procuram me acham.
\par 18 Riquezas e honra estão comigo, bens duráveis e justiça.
\par 19 Melhor é o meu fruto do que o ouro, do que o ouro refinado; e o meu rendimento, melhor do que a prata escolhida.
\par 20 Ando pelo caminho da justiça, no meio das veredas do juízo,
\par 21 para dotar de bens os que me amam e lhes encher os tesouros.
\par 22 O SENHOR me possuía no início de sua obra, antes de suas obras mais antigas.
\par 23 Desde a eternidade fui estabelecida, desde o princípio, antes do começo da terra.
\par 24 Antes de haver abismos, eu nasci, e antes ainda de haver fontes carregadas de águas.
\par 25 Antes que os montes fossem firmados, antes de haver outeiros, eu nasci.
\par 26 Ainda ele não tinha feito a terra, nem as amplidões, nem sequer o princípio do pó do mundo.
\par 27 Quando ele preparava os céus, aí estava eu; quando traçava o horizonte sobre a face do abismo;
\par 28 quando firmava as nuvens de cima; quando estabelecia as fontes do abismo;
\par 29 quando fixava ao mar o seu limite, para que as águas não traspassassem os seus limites; quando compunha os fundamentos da terra;
\par 30 então, eu estava com ele e era seu arquiteto, dia após dia, eu era as suas delícias, folgando perante ele em todo o tempo;
\par 31 regozijando-me no seu mundo habitável e achando as minhas delícias com os filhos dos homens.
\par 32 Agora, pois, filhos, ouvi-me, porque felizes serão os que guardarem os meus caminhos.
\par 33 Ouvi o ensino, sede sábios e não o rejeiteis.
\par 34 Feliz o homem que me dá ouvidos, velando dia a dia às minhas portas, esperando às ombreiras da minha entrada.
\par 35 Porque o que me acha acha a vida e alcança favor do SENHOR.
\par 36 Mas o que peca contra mim violenta a própria alma. Todos os que me aborrecem amam a morte.

\chapter{9}

\par 1 A Sabedoria edificou a sua casa, lavrou as suas sete colunas.
\par 2 Carneou os seus animais, misturou o seu vinho e arrumou a sua mesa.
\par 3 Já deu ordens às suas criadas e, assim, convida desde as alturas da cidade:
\par 4 Quem é simples, volte-se para aqui. Aos faltos de senso diz:
\par 5 Vinde, comei do meu pão e bebei do vinho que misturei.
\par 6 Deixai os insensatos e vivei; andai pelo caminho do entendimento.
\par 7 O que repreende o escarnecedor traz afronta sobre si; e o que censura o perverso a si mesmo se injuria.
\par 8 Não repreendas o escarnecedor, para que te não aborreça; repreende o sábio, e ele te amará.
\par 9 Dá instrução ao sábio, e ele se fará mais sábio ainda; ensina ao justo, e ele crescerá em prudência.
\par 10 O temor do SENHOR é o princípio da sabedoria, e o conhecimento do Santo é prudência.
\par 11 Porque por mim se multiplicam os teus dias, e anos de vida se te acrescentarão.
\par 12 Se és sábio, para ti mesmo o és; se és escarnecedor, tu só o suportarás.
\par 13 A loucura é mulher apaixonada, é ignorante e não sabe coisa alguma.
\par 14 Assenta-se à porta de sua casa, nas alturas da cidade, toma uma cadeira,
\par 15 para dizer aos que passam e seguem direito o seu caminho:
\par 16 Quem é simples, volte-se para aqui. E aos faltos de senso diz:
\par 17 As águas roubadas são doces, e o pão comido às ocultas é agradável.
\par 18 Eles, porém, não sabem que ali estão os mortos, que os seus convidados estão nas profundezas do inferno.

\chapter{10}

\par 1 Provérbios de Salomão. O filho sábio alegra a seu pai, mas o filho insensato é a tristeza de sua mãe.
\par 2 Os tesouros da impiedade de nada aproveitam, mas a justiça livra da morte.
\par 3 O SENHOR não deixa ter fome o justo, mas rechaça a avidez dos perversos.
\par 4 O que trabalha com mão remissa empobrece, mas a mão dos diligentes vem a enriquecer-se.
\par 5 O que ajunta no verão é filho sábio, mas o que dorme na sega é filho que envergonha.
\par 6 Sobre a cabeça do justo há bênçãos, mas na boca dos perversos mora a violência.
\par 7 A memória do justo é abençoada, mas o nome dos perversos cai em podridão.
\par 8 O sábio de coração aceita os mandamentos, mas o insensato de lábios vem a arruinar-se.
\par 9 Quem anda em integridade anda seguro, mas o que perverte os seus caminhos será conhecido.
\par 10 O que acena com os olhos traz desgosto, e o insensato de lábios vem a arruinar-se.
\par 11 A boca do justo é manancial de vida, mas na boca dos perversos mora a violência.
\par 12 O ódio excita contendas, mas o amor cobre todas as transgressões.
\par 13 Nos lábios do prudente, se acha sabedoria, mas a vara é para as costas do falto de senso.
\par 14 Os sábios entesouram o conhecimento, mas a boca do néscio é uma ruína iminente.
\par 15 Os bens do rico são a sua cidade forte; a pobreza dos pobres é a sua ruína.
\par 16 A obra do justo conduz à vida, e o rendimento do perverso, ao pecado.
\par 17 O caminho para a vida é de quem guarda o ensino, mas o que abandona a repreensão anda errado.
\par 18 O que retém o ódio é de lábios falsos, e o que difama é insensato.
\par 19 No muito falar não falta transgressão, mas o que modera os lábios é prudente.
\par 20 Prata escolhida é a língua do justo, mas o coração dos perversos vale mui pouco.
\par 21 Os lábios do justo apascentam a muitos, mas, por falta de senso, morrem os tolos.
\par 22 A bênção do SENHOR enriquece, e, com ela, ele não traz desgosto.
\par 23 Para o insensato, praticar a maldade é divertimento; para o homem inteligente, o ser sábio.
\par 24 Aquilo que teme o perverso, isso lhe sobrevém, mas o anelo dos justos Deus o cumpre.
\par 25 Como passa a tempestade, assim desaparece o perverso, mas o justo tem perpétuo fundamento.
\par 26 Como vinagre para os dentes e fumaça para os olhos, assim é o preguiçoso para aqueles que o mandam.
\par 27 O temor do SENHOR prolonga os dias da vida, mas os anos dos perversos serão abreviados.
\par 28 A esperança dos justos é alegria, mas a expectação dos perversos perecerá.
\par 29 O caminho do SENHOR é fortaleza para os íntegros, mas ruína aos que praticam a iniqüidade.
\par 30 O justo jamais será abalado, mas os perversos não habitarão a terra.
\par 31 A boca do justo produz sabedoria, mas a língua da perversidade será desarraigada.
\par 32 Os lábios do justo sabem o que agrada, mas a boca dos perversos, somente o mal.

\chapter{11}

\par 1 Balança enganosa é abominação para o SENHOR, mas o peso justo é o seu prazer.
\par 2 Em vindo a soberba, sobrevém a desonra, mas com os humildes está a sabedoria.
\par 3 A integridade dos retos os guia; mas, aos pérfidos, a sua mesma falsidade os destrói.
\par 4 As riquezas de nada aproveitam no dia da ira, mas a justiça livra da morte.
\par 5 A justiça do íntegro endireita o seu caminho, mas pela sua impiedade cai o perverso.
\par 6 A justiça dos retos os livrará, mas na sua maldade os pérfidos serão apanhados.
\par 7 Morrendo o homem perverso, morre a sua esperança, e a expectação da iniqüidade se desvanece.
\par 8 O justo é libertado da angústia, e o perverso a recebe em seu lugar.
\par 9 O ímpio, com a boca, destrói o próximo, mas os justos são libertados pelo conhecimento.
\par 10 No bem-estar dos justos exulta a cidade, e, perecendo os perversos, há júbilo.
\par 11 Pela bênção que os retos suscitam, a cidade se exalta, mas pela boca dos perversos é derribada.
\par 12 O que despreza o próximo é falto de senso, mas o homem prudente, este se cala.
\par 13 O mexeriqueiro descobre o segredo, mas o fiel de espírito o encobre.
\par 14 Não havendo sábia direção, cai o povo, mas na multidão de conselheiros há segurança.
\par 15 Quem fica por fiador de outrem sofrerá males, mas o que foge de o ser estará seguro.
\par 16 A mulher graciosa alcança honra, como os poderosos adquirem riqueza.
\par 17 O homem bondoso faz bem a si mesmo, mas o cruel a si mesmo se fere.
\par 18 O perverso recebe um salário ilusório, mas o que semeia justiça terá recompensa verdadeira.
\par 19 Tão certo como a justiça conduz para a vida, assim o que segue o mal, para a sua morte o faz.
\par 20 Abomináveis para o SENHOR são os perversos de coração, mas os que andam em integridade são o seu prazer.
\par 21 O mau, é evidente, não ficará sem castigo, mas a geração dos justos é livre.
\par 22 Como jóia de ouro em focinho de porco, assim é a mulher formosa que não tem discrição.
\par 23 O desejo dos justos tende somente para o bem, mas a expectação dos perversos redunda em ira.
\par 24 A quem dá liberalmente, ainda se lhe acrescenta mais e mais; ao que retém mais do que é justo, ser-lhe-á em pura perda.
\par 25 A alma generosa prosperará, e quem dá a beber será dessedentado.
\par 26 Ao que retém o trigo, o povo o amaldiçoa, mas bênção haverá sobre a cabeça do seu vendedor.
\par 27 Quem procura o bem alcança favor, mas ao que corre atrás do mal, este lhe sobrevirá.
\par 28 Quem confia nas suas riquezas cairá, mas os justos reverdecerão como a folhagem.
\par 29 O que perturba a sua casa herda o vento, e o insensato é servo do sábio de coração.
\par 30 O fruto do justo é árvore de vida, e o que ganha almas é sábio.
\par 31 Se o justo é punido na terra, quanto mais o perverso e o pecador!

\chapter{12}

\par 1 Quem ama a disciplina ama o conhecimento, mas o que aborrece a repreensão é estúpido.
\par 2 O homem de bem alcança o favor do SENHOR, mas ao homem de perversos desígnios, ele o condena.
\par 3 O homem não se estabelece pela perversidade, mas a raiz dos justos não será removida.
\par 4 A mulher virtuosa é a coroa do seu marido, mas a que procede vergonhosamente é como podridão nos seus ossos.
\par 5 Os pensamentos do justo são retos, mas os conselhos do perverso, engano.
\par 6 As palavras dos perversos são emboscadas para derramar sangue, mas a boca dos retos livra homens.
\par 7 Os perversos serão derribados e já não são, mas a casa dos justos permanecerá.
\par 8 Segundo o seu entendimento, será louvado o homem, mas o perverso de coração será desprezado.
\par 9 Melhor é o que se estima em pouco e faz o seu trabalho do que o vanglorioso que tem falta de pão.
\par 10 O justo atenta para a vida dos seus animais, mas o coração dos perversos é cruel.
\par 11 O que lavra a sua terra será farto de pão, mas o que corre atrás de coisas vãs é falto de senso.
\par 12 O perverso quer viver do que caçam os maus, mas a raiz dos justos produz o seu fruto.
\par 13 Pela transgressão dos lábios o mau se enlaça, mas o justo sairá da angústia.
\par 14 Cada um se farta de bem pelo fruto da sua boca, e o que as mãos do homem fizerem ser-lhe-á retribuído.
\par 15 O caminho do insensato aos seus próprios olhos parece reto, mas o sábio dá ouvidos aos conselhos.
\par 16 A ira do insensato num instante se conhece, mas o prudente oculta a afronta.
\par 17 O que diz a verdade manifesta a justiça, mas a testemunha falsa, a fraude.
\par 18 Alguém há cuja tagarelice é como pontas de espada, mas a língua dos sábios é medicina.
\par 19 O lábio veraz permanece para sempre, mas a língua mentirosa, apenas um momento.
\par 20 Há fraude no coração dos que maquinam mal, mas alegria têm os que aconselham a paz.
\par 21 Nenhum agravo sobrevirá ao justo, mas os perversos, o mal os apanhará em cheio.
\par 22 Os lábios mentirosos são abomináveis ao SENHOR, mas os que agem fielmente são o seu prazer.
\par 23 O homem prudente oculta o conhecimento, mas o coração dos insensatos proclama a estultícia.
\par 24 A mão diligente dominará, mas a remissa será sujeita a trabalhos forçados.
\par 25 A ansiedade no coração do homem o abate, mas a boa palavra o alegra.
\par 26 O justo serve de guia para o seu companheiro, mas o caminho dos perversos os faz errar.
\par 27 O preguiçoso não assará a sua caça, mas o bem precioso do homem é ser ele diligente.
\par 28 Na vereda da justiça, está a vida, e no caminho da sua carreira não há morte.

\chapter{13}

\par 1 O filho sábio ouve a instrução do pai, mas o escarnecedor não atende à repreensão.
\par 2 Do fruto da boca o homem comerá o bem, mas o desejo dos pérfidos é a violência.
\par 3 O que guarda a boca conserva a sua alma, mas o que muito abre os lábios a si mesmo se arruína.
\par 4 O preguiçoso deseja e nada tem, mas a alma dos diligentes se farta.
\par 5 O justo aborrece a palavra de mentira, mas o perverso faz vergonha e se desonra.
\par 6 A justiça guarda ao que anda em integridade, mas a malícia subverte ao pecador.
\par 7 Uns se dizem ricos sem terem nada; outros se dizem pobres, sendo mui ricos.
\par 8 Com as suas riquezas se resgata o homem, mas ao pobre não ocorre ameaça.
\par 9 A luz dos justos brilha intensamente, mas a lâmpada dos perversos se apagará.
\par 10 Da soberba só resulta a contenda, mas com os que se aconselham se acha a sabedoria.
\par 11 Os bens que facilmente se ganham, esses diminuem, mas o que ajunta à força do trabalho terá aumento.
\par 12 A esperança que se adia faz adoecer o coração, mas o desejo cumprido é árvore de vida.
\par 13 O que despreza a palavra a ela se apenhora, mas o que teme o mandamento será galardoado.
\par 14 O ensino do sábio é fonte de vida, para que se evitem os laços da morte.
\par 15 A boa inteligência consegue favor, mas o caminho dos pérfidos é intransitável.
\par 16 Todo prudente procede com conhecimento, mas o insensato espraia a sua loucura.
\par 17 O mau mensageiro se precipita no mal, mas o embaixador fiel é medicina.
\par 18 Pobreza e afronta sobrevêm ao que rejeita a instrução, mas o que guarda a repreensão será honrado.
\par 19 O desejo que se cumpre agrada a alma, mas apartar-se do mal é abominável para os insensatos.
\par 20 Quem anda com os sábios será sábio, mas o companheiro dos insensatos se tornará mau.
\par 21 A desventura persegue os pecadores, mas os justos serão galardoados com o bem.
\par 22 O homem de bem deixa herança aos filhos de seus filhos, mas a riqueza do pecador é depositada para o justo.
\par 23 A terra virgem dos pobres dá mantimento em abundância, mas a falta de justiça o dissipa.
\par 24 O que retém a vara aborrece a seu filho, mas o que o ama, cedo, o disciplina.
\par 25 O justo tem o bastante para satisfazer o seu apetite, mas o estômago dos perversos passa fome.

\chapter{14}

\par 1 A mulher sábia edifica a sua casa, mas a insensata, com as próprias mãos, a derriba.
\par 2 O que anda na retidão teme ao SENHOR, mas o que anda em caminhos tortuosos, esse o despreza.
\par 3 Está na boca do insensato a vara para a sua própria soberba, mas os lábios do prudente o preservarão.
\par 4 Não havendo bois, o celeiro fica limpo, mas pela força do boi há abundância de colheitas.
\par 5 A testemunha verdadeira não mente, mas a falsa se desboca em mentiras.
\par 6 O escarnecedor procura a sabedoria e não a encontra, mas para o prudente o conhecimento é fácil.
\par 7 Foge da presença do homem insensato, porque nele não divisarás lábios de conhecimento.
\par 8 A sabedoria do prudente é entender o seu próprio caminho, mas a estultícia dos insensatos é enganadora.
\par 9 Os loucos zombam do pecado, mas entre os retos há boa vontade.
\par 10 O coração conhece a sua própria amargura, e da sua alegria não participará o estranho.
\par 11 A casa dos perversos será destruída, mas a tenda dos retos florescerá.
\par 12 Há caminho que ao homem parece direito, mas ao cabo dá em caminhos de morte.
\par 13 Até no riso tem dor o coração, e o fim da alegria é tristeza.
\par 14 O infiel de coração dos seus próprios caminhos se farta, como do seu próprio proceder, o homem de bem.
\par 15 O simples dá crédito a toda palavra, mas o prudente atenta para os seus passos.
\par 16 O sábio é cauteloso e desvia-se do mal, mas o insensato encoleriza-se e dá-se por seguro.
\par 17 O que presto se ira faz loucuras, e o homem de maus desígnios é odiado.
\par 18 Os simples herdam a estultícia, mas os prudentes se coroam de conhecimento.
\par 19 Os maus inclinam-se perante a face dos bons, e os perversos, junto às portas do justo.
\par 20 O pobre é odiado até do vizinho, mas o rico tem muitos amigos.
\par 21 O que despreza ao seu vizinho peca, mas o que se compadece dos pobres é feliz.
\par 22 Acaso, não erram os que maquinam o mal? Mas amor e fidelidade haverá para os que planejam o bem.
\par 23 Em todo trabalho há proveito; meras palavras, porém, levam à penúria.
\par 24 Aos sábios a riqueza é coroa, mas a estultícia dos insensatos não passa de estultícia.
\par 25 A testemunha verdadeira livra almas, mas o que se desboca em mentiras é enganador.
\par 26 No temor do SENHOR, tem o homem forte amparo, e isso é refúgio para os seus filhos.
\par 27 O temor do SENHOR é fonte de vida para evitar os laços da morte.
\par 28 Na multidão do povo, está a glória do rei, mas, na falta de povo, a ruína do príncipe.
\par 29 O longânimo é grande em entendimento, mas o de ânimo precipitado exalta a loucura.
\par 30 O ânimo sereno é a vida do corpo, mas a inveja é a podridão dos ossos.
\par 31 O que oprime ao pobre insulta aquele que o criou, mas a este honra o que se compadece do necessitado.
\par 32 Pela sua malícia é derribado o perverso, mas o justo, ainda morrendo, tem esperança.
\par 33 No coração do prudente, repousa a sabedoria, mas o que há no interior dos insensatos vem a lume.
\par 34 A justiça exalta as nações, mas o pecado é o opróbrio dos povos.
\par 35 O servo prudente goza do favor do rei, mas o que procede indignamente é objeto do seu furor.

\chapter{15}

\par 1 A resposta branda desvia o furor, mas a palavra dura suscita a ira.
\par 2 A língua dos sábios adorna o conhecimento, mas a boca dos insensatos derrama a estultícia.
\par 3 Os olhos do SENHOR estão em todo lugar, contemplando os maus e os bons.
\par 4 A língua serena é árvore de vida, mas a perversa quebranta o espírito.
\par 5 O insensato despreza a instrução de seu pai, mas o que atende à repreensão consegue a prudência.
\par 6 Na casa do justo há grande tesouro, mas na renda dos perversos há perturbação.
\par 7 A língua dos sábios derrama o conhecimento, mas o coração dos insensatos não procede assim.
\par 8 O sacrifício dos perversos é abominável ao SENHOR, mas a oração dos retos é o seu contentamento.
\par 9 O caminho do perverso é abominação ao SENHOR, mas este ama o que segue a justiça.
\par 10 Disciplina rigorosa há para o que deixa a vereda, e o que odeia a repreensão morrerá.
\par 11 O além e o abismo estão descobertos perante o SENHOR; quanto mais o coração dos filhos dos homens!
\par 12 O escarnecedor não ama àquele que o repreende, nem se chegará para os sábios.
\par 13 O coração alegre aformoseia o rosto, mas com a tristeza do coração o espírito se abate.
\par 14 O coração sábio procura o conhecimento, mas a boca dos insensatos se apascenta de estultícia.
\par 15 Todos os dias do aflito são maus, mas a alegria do coração é banquete contínuo.
\par 16 Melhor é o pouco, havendo o temor do SENHOR, do que grande tesouro onde há inquietação.
\par 17 Melhor é um prato de hortaliças onde há amor do que o boi cevado e, com ele, o ódio.
\par 18 O homem iracundo suscita contendas, mas o longânimo apazigua a luta.
\par 19 O caminho do preguiçoso é como que cercado de espinhos, mas a vereda dos retos é plana.
\par 20 O filho sábio alegra a seu pai, mas o homem insensato despreza a sua mãe.
\par 21 A estultícia é alegria para o que carece de entendimento, mas o homem sábio anda retamente.
\par 22 Onde não há conselho fracassam os projetos, mas com os muitos conselheiros há bom êxito.
\par 23 O homem se alegra em dar resposta adequada, e a palavra, a seu tempo, quão boa é!
\par 24 Para o sábio há o caminho da vida que o leva para cima, a fim de evitar o inferno, embaixo.
\par 25 O SENHOR deita por terra a casa dos soberbos; contudo, mantém a herança da viúva.
\par 26 Abomináveis são para o SENHOR os desígnios do mau, mas as palavras bondosas lhe são aprazíveis.
\par 27 O que é ávido por lucro desonesto transtorna a sua casa, mas o que odeia o suborno, esse viverá.
\par 28 O coração do justo medita o que há de responder, mas a boca dos perversos transborda maldades.
\par 29 O SENHOR está longe dos perversos, mas atende à oração dos justos.
\par 30 O olhar de amigo alegra ao coração; as boas-novas fortalecem até os ossos.
\par 31 Os ouvidos que atendem à repreensão salutar no meio dos sábios têm a sua morada.
\par 32 O que rejeita a disciplina menospreza a sua alma, porém o que atende à repreensão adquire entendimento.
\par 33 O temor do SENHOR é a instrução da sabedoria, e a humildade precede a honra.

\chapter{16}

\par 1 O coração do homem pode fazer planos, mas a resposta certa dos lábios vem do SENHOR.
\par 2 Todos os caminhos do homem são puros aos seus olhos, mas o SENHOR pesa o espírito.
\par 3 Confia ao SENHOR as tuas obras, e os teus desígnios serão estabelecidos.
\par 4 O SENHOR fez todas as coisas para determinados fins e até o perverso, para o dia da calamidade.
\par 5 Abominável é ao SENHOR todo arrogante de coração; é evidente que não ficará impune.
\par 6 Pela misericórdia e pela verdade, se expia a culpa; e pelo temor do SENHOR os homens evitam o mal.
\par 7 Sendo o caminho dos homens agradável ao SENHOR, este reconcilia com eles os seus inimigos.
\par 8 Melhor é o pouco, havendo justiça, do que grandes rendimentos com injustiça.
\par 9 O coração do homem traça o seu caminho, mas o SENHOR lhe dirige os passos.
\par 10 Nos lábios do rei se acham decisões autorizadas; no julgar não transgrida, pois, a sua boca.
\par 11 Peso e balança justos pertencem ao SENHOR; obra sua são todos os pesos da bolsa.
\par 12 A prática da impiedade é abominável para os reis, porque com justiça se estabelece o trono.
\par 13 Os lábios justos são o contentamento do rei, e ele ama o que fala coisas retas.
\par 14 O furor do rei são uns mensageiros de morte, mas o homem sábio o apazigua.
\par 15 O semblante alegre do rei significa vida, e a sua benevolência é como a nuvem que traz chuva serôdia.
\par 16 Quanto melhor é adquirir a sabedoria do que o ouro! E mais excelente, adquirir a prudência do que a prata!
\par 17 O caminho dos retos é desviar-se do mal; o que guarda o seu caminho preserva a sua alma.
\par 18 A soberba precede a ruína, e a altivez do espírito, a queda.
\par 19 Melhor é ser humilde de espírito com os humildes do que repartir o despojo com os soberbos.
\par 20 O que atenta para o ensino acha o bem, e o que confia no SENHOR, esse é feliz.
\par 21 O sábio de coração é chamado prudente, e a doçura no falar aumenta o saber.
\par 22 O entendimento, para aqueles que o possuem, é fonte de vida; mas, para o insensato, a sua estultícia lhe é castigo.
\par 23 O coração do sábio é mestre de sua boca e aumenta a persuasão nos seus lábios.
\par 24 Palavras agradáveis são como favo de mel: doces para a alma e medicina para o corpo.
\par 25 Há caminho que parece direito ao homem, mas afinal são caminhos de morte.
\par 26 A fome do trabalhador o faz trabalhar, porque a sua boca a isso o incita.
\par 27 O homem depravado cava o mal, e nos seus lábios há como que fogo ardente.
\par 28 O homem perverso espalha contendas, e o difamador separa os maiores amigos.
\par 29 O homem violento alicia o seu companheiro e guia-o por um caminho que não é bom.
\par 30 Quem fecha os olhos imagina o mal, e, quando morde os lábios, o executa.
\par 31 Coroa de honra são as cãs, quando se acham no caminho da justiça.
\par 32 Melhor é o longânimo do que o herói da guerra, e o que domina o seu espírito, do que o que toma uma cidade.
\par 33 A sorte se lança no regaço, mas do SENHOR procede toda decisão.

\chapter{17}

\par 1 Melhor é um bocado seco e tranqüilidade do que a casa farta de carnes e contendas.
\par 2 O escravo prudente dominará sobre o filho que causa vergonha e, entre os irmãos, terá parte na herança.
\par 3 O crisol prova a prata, e o forno, o ouro; mas aos corações prova o SENHOR.
\par 4 O malfazejo atenta para o lábio iníquo; o mentiroso inclina os ouvidos para a língua maligna.
\par 5 O que escarnece do pobre insulta ao que o criou; o que se alegra da calamidade não ficará impune.
\par 6 Coroa dos velhos são os filhos dos filhos; e a glória dos filhos são os pais.
\par 7 Ao insensato não convém a palavra excelente; quanto menos ao príncipe, o lábio mentiroso!
\par 8 Pedra mágica é o suborno aos olhos de quem o dá, e para onde quer que se volte terá seu proveito.
\par 9 O que encobre a transgressão adquire amor, mas o que traz o assunto à baila separa os maiores amigos.
\par 10 Mais fundo entra a repreensão no prudente do que cem açoites no insensato.
\par 11 O rebelde não busca senão o mal; por isso, mensageiro cruel se enviará contra ele.
\par 12 Melhor é encontrar-se uma ursa roubada dos filhos do que o insensato na sua estultícia.
\par 13 Quanto àquele que paga o bem com o mal, não se apartará o mal da sua casa.
\par 14 Como o abrir-se da represa, assim é o começo da contenda; desiste, pois, antes que haja rixas.
\par 15 O que justifica o perverso e o que condena o justo abomináveis são para o SENHOR, tanto um como o outro.
\par 16 De que serviria o dinheiro na mão do insensato para comprar a sabedoria, visto que não tem entendimento?
\par 17 Em todo tempo ama o amigo, e na angústia se faz o irmão.
\par 18 O homem falto de entendimento compromete-se, ficando por fiador do seu próximo.
\par 19 O que ama a contenda ama o pecado; o que faz alta a sua porta facilita a própria queda.
\par 20 O perverso de coração jamais achará o bem; e o que tem a língua dobre vem a cair no mal.
\par 21 O filho estulto é tristeza para o pai, e o pai do insensato não se alegra.
\par 22 O coração alegre é bom remédio, mas o espírito abatido faz secar os ossos.
\par 23 O perverso aceita suborno secretamente, para perverter as veredas da justiça.
\par 24 A sabedoria é o alvo do inteligente, mas os olhos do insensato vagam pelas extremidades da terra.
\par 25 O filho insensato é tristeza para o pai e amargura para quem o deu à luz.
\par 26 Não é bom punir ao justo; é contra todo direito ferir ao príncipe.
\par 27 Quem retém as palavras possui o conhecimento, e o sereno de espírito é homem de inteligência.
\par 28 Até o estulto, quando se cala, é tido por sábio, e o que cerra os lábios, por sábio.

\chapter{18}

\par 1 O solitário busca o seu próprio interesse e insurge-se contra a verdadeira sabedoria.
\par 2 O insensato não tem prazer no entendimento, senão em externar o seu interior.
\par 3 Vindo a perversidade, vem também o desprezo; e, com a ignomínia, a vergonha.
\par 4 Águas profundas são as palavras da boca do homem, e a fonte da sabedoria, ribeiros transbordantes.
\par 5 Não é bom ser parcial com o perverso, para torcer o direito contra os justos.
\par 6 Os lábios do insensato entram na contenda, e por açoites brada a sua boca.
\par 7 A boca do insensato é a sua própria destruição, e os seus lábios, um laço para a sua alma.
\par 8 As palavras do maldizente são doces bocados que descem para o mais interior do ventre.
\par 9 Quem é negligente na sua obra já é irmão do desperdiçador.
\par 10 Torre forte é o nome do SENHOR, à qual o justo se acolhe e está seguro.
\par 11 Os bens do rico lhe são cidade forte e, segundo imagina, uma alta muralha.
\par 12 Antes da ruína, gaba-se o coração do homem, e diante da honra vai a humildade.
\par 13 Responder antes de ouvir é estultícia e vergonha.
\par 14 O espírito firme sustém o homem na sua doença, mas o espírito abatido, quem o pode suportar?
\par 15 O coração do sábio adquire o conhecimento, e o ouvido dos sábios procura o saber.
\par 16 O presente que o homem faz alarga-lhe o caminho e leva-o perante os grandes.
\par 17 O que começa o pleito parece justo, até que vem o outro e o examina.
\par 18 Pelo lançar da sorte, cessam os pleitos, e se decide a causa entre os poderosos.
\par 19 O irmão ofendido resiste mais que uma fortaleza; suas contendas são ferrolhos de um castelo.
\par 20 Do fruto da boca o coração se farta, do que produzem os lábios se satisfaz.
\par 21 A morte e a vida estão no poder da língua; o que bem a utiliza come do seu fruto.
\par 22 O que acha uma esposa acha o bem e alcançou a benevolência do SENHOR.
\par 23 O pobre fala com súplicas, porém o rico responde com durezas.
\par 24 O homem que tem muitos amigos sai perdendo; mas há amigo mais chegado do que um irmão.

\chapter{19}

\par 1 Melhor é o pobre que anda na sua integridade do que o perverso de lábios e tolo.
\par 2 Não é bom proceder sem refletir, e peca quem é precipitado.
\par 3 A estultícia do homem perverte o seu caminho, mas é contra o SENHOR que o seu coração se ira.
\par 4 As riquezas multiplicam os amigos; mas, ao pobre, o seu próprio amigo o deixa.
\par 5 A falsa testemunha não fica impune, e o que profere mentiras não escapa.
\par 6 Ao generoso, muitos o adulam, e todos são amigos do que dá presentes.
\par 7 Se os irmãos do pobre o aborrecem, quanto mais se afastarão dele os seus amigos! Corre após eles com súplicas, mas não os alcança.
\par 8 O que adquire entendimento ama a sua alma; o que conserva a inteligência acha o bem.
\par 9 A falsa testemunha não fica impune, e o que profere mentiras perece.
\par 10 Ao insensato não convém a vida regalada, quanto menos ao escravo dominar os príncipes!
\par 11 A discrição do homem o torna longânimo, e sua glória é perdoar as injúrias.
\par 12 Como o bramido do leão, assim é a indignação do rei; mas seu favor é como o orvalho sobre a erva.
\par 13 O filho insensato é a desgraça do pai, e um gotejar contínuo, as contenções da esposa.
\par 14 A casa e os bens vêm como herança dos pais; mas do SENHOR, a esposa prudente.
\par 15 A preguiça faz cair em profundo sono, e o ocioso vem a padecer fome.
\par 16 O que guarda o mandamento guarda a sua alma; mas o que despreza os seus caminhos, esse morre.
\par 17 Quem se compadece do pobre ao SENHOR empresta, e este lhe paga o seu benefício.
\par 18 Castiga a teu filho, enquanto há esperança, mas não te excedas a ponto de matá-lo.
\par 19 Homem de grande ira tem de sofrer o dano; porque, se tu o livrares, virás ainda a fazê-lo de novo.
\par 20 Ouve o conselho e recebe a instrução, para que sejas sábio nos teus dias por vir.
\par 21 Muitos propósitos há no coração do homem, mas o desígnio do SENHOR permanecerá.
\par 22 O que torna agradável o homem é a sua misericórdia; o pobre é preferível ao mentiroso.
\par 23 O temor do SENHOR conduz à vida; aquele que o tem ficará satisfeito, e mal nenhum o visitará.
\par 24 O preguiçoso mete a mão no prato e não quer ter o trabalho de a levar à boca.
\par 25 Quando ferires ao escarnecedor, o simples aprenderá a prudência; repreende ao sábio, e crescerá em conhecimento.
\par 26 O que maltrata a seu pai ou manda embora a sua mãe filho é que envergonha e desonra.
\par 27 Filho meu, se deixas de ouvir a instrução, desviar-te-ás das palavras do conhecimento.
\par 28 A testemunha de Belial escarnece da justiça, e a boca dos perversos devora a iniqüidade.
\par 29 Preparados estão os juízos para os escarnecedores e os açoites, para as costas dos insensatos.

\chapter{20}

\par 1 O vinho é escarnecedor, e a bebida forte, alvoroçadora; todo aquele que por eles é vencido não é sábio.
\par 2 Como o bramido do leão, é o terror do rei; o que lhe provoca a ira peca contra a sua própria vida.
\par 3 Honroso é para o homem o desviar-se de contendas, mas todo insensato se mete em rixas.
\par 4 O preguiçoso não lavra por causa do inverno, pelo que, na sega, procura e nada encontra.
\par 5 Como águas profundas, são os propósitos do coração do homem, mas o homem de inteligência sabe descobri-los.
\par 6 Muitos proclamam a sua própria benignidade; mas o homem fidedigno, quem o achará?
\par 7 O justo anda na sua integridade; felizes lhe são os filhos depois dele.
\par 8 Assentando-se o rei no trono do juízo, com os seus olhos dissipa todo mal.
\par 9 Quem pode dizer: Purifiquei o meu coração, limpo estou do meu pecado?
\par 10 Dois pesos e duas medidas, uns e outras são abomináveis ao SENHOR.
\par 11 Até a criança se dá a conhecer pelas suas ações, se o que faz é puro e reto.
\par 12 O ouvido que ouve e o olho que vê, o SENHOR os fez, tanto um como o outro.
\par 13 Não ames o sono, para que não empobreças; abre os olhos e te fartarás do teu próprio pão.
\par 14 Nada vale, nada vale, diz o comprador, mas, indo-se, então, se gaba.
\par 15 Há ouro e abundância de pérolas, mas os lábios instruídos são jóia preciosa.
\par 16 Tome-se a roupa àquele que fica fiador por outrem; e, por penhor, àquele que se obriga por estrangeiros.
\par 17 Suave é ao homem o pão ganho por fraude, mas, depois, a sua boca se encherá de pedrinhas de areia.
\par 18 Os planos mediante os conselhos têm bom êxito; faze a guerra com prudência.
\par 19 O mexeriqueiro revela o segredo; portanto, não te metas com quem muito abre os lábios.
\par 20 A quem amaldiçoa a seu pai ou a sua mãe, apagar-se-lhe-á a lâmpada nas mais densas trevas.
\par 21 A posse antecipada de uma herança no fim não será abençoada.
\par 22 Não digas: Vingar-me-ei do mal; espera pelo SENHOR, e ele te livrará.
\par 23 Dois pesos são coisa abominável ao SENHOR, e balança enganosa não é boa.
\par 24 Os passos do homem são dirigidos pelo SENHOR; como, pois, poderá o homem entender o seu caminho?
\par 25 Laço é para o homem o dizer precipitadamente: É santo! E só refletir depois de fazer o voto.
\par 26 O rei sábio joeira os perversos e faz passar sobre eles a roda.
\par 27 O espírito do homem é a lâmpada do SENHOR, a qual esquadrinha todo o mais íntimo do corpo.
\par 28 Amor e fidelidade preservam o rei, e com benignidade sustém ele o seu trono.
\par 29 O ornato dos jovens é a sua força, e a beleza dos velhos, as suas cãs.
\par 30 Os vergões das feridas purificam do mal, e os açoites, o mais íntimo do corpo.

\chapter{21}

\par 1 Como ribeiros de águas assim é o coração do rei na mão do SENHOR; este, segundo o seu querer, o inclina.
\par 2 Todo caminho do homem é reto aos seus próprios olhos, mas o SENHOR sonda os corações.
\par 3 Exercitar justiça e juízo é mais aceitável ao SENHOR do que sacrifício.
\par 4 Olhar altivo e coração orgulhoso, a lâmpada dos perversos, são pecado.
\par 5 Os planos do diligente tendem à abundância, mas a pressa excessiva, à pobreza.
\par 6 Trabalhar por adquirir tesouro com língua falsa é vaidade e laço mortal.
\par 7 A violência dos perversos os arrebata, porque recusam praticar a justiça.
\par 8 Tortuoso é o caminho do homem carregado de culpa, mas reto, o proceder do honesto.
\par 9 Melhor é morar no canto do eirado do que junto com a mulher rixosa na mesma casa.
\par 10 A alma do perverso deseja o mal; nem o seu vizinho recebe dele compaixão.
\par 11 Quando o escarnecedor é castigado, o simples se torna sábio; e, quando o sábio é instruído, recebe o conhecimento.
\par 12 O Justo considera a casa dos perversos e os arrasta para o mal.
\par 13 O que tapa o ouvido ao clamor do pobre também clamará e não será ouvido.
\par 14 O presente que se dá em segredo abate a ira, e a dádiva em sigilo, uma forte indignação.
\par 15 Praticar a justiça é alegria para o justo, mas espanto, para os que praticam a iniqüidade.
\par 16 O homem que se desvia do caminho do entendimento na congregação dos mortos repousará.
\par 17 Quem ama os prazeres empobrecerá, quem ama o vinho e o azeite jamais enriquecerá.
\par 18 O perverso serve de resgate para o justo; e, para os retos, o pérfido.
\par 19 Melhor é morar numa terra deserta do que com a mulher rixosa e iracunda.
\par 20 Tesouro desejável e azeite há na casa do sábio, mas o homem insensato os desperdiça.
\par 21 O que segue a justiça e a bondade achará a vida, a justiça e a honra.
\par 22 O sábio escala a cidade dos valentes e derriba a fortaleza em que ela confia.
\par 23 O que guarda a boca e a língua guarda a sua alma das angústias.
\par 24 Quanto ao soberbo e presumido, zombador é seu nome; procede com indignação e arrogância.
\par 25 O preguiçoso morre desejando, porque as suas mãos recusam trabalhar.
\par 26 O cobiçoso cobiça todo o dia, mas o justo dá e nada retém.
\par 27 O sacrifício dos perversos já é abominação; quanto mais oferecendo-o com intenção maligna!
\par 28 A testemunha falsa perecerá, mas a auricular falará sem ser contestada.
\par 29 O homem perverso mostra dureza no rosto, mas o reto considera o seu caminho.
\par 30 Não há sabedoria, nem inteligência, nem mesmo conselho contra o SENHOR.
\par 31 O cavalo prepara-se para o dia da batalha, mas a vitória vem do SENHOR.

\chapter{22}

\par 1 Mais vale o bom nome do que as muitas riquezas; e o ser estimado é melhor do que a prata e o ouro.
\par 2 O rico e o pobre se encontram; a um e a outro faz o SENHOR.
\par 3 O prudente vê o mal e esconde-se; mas os simples passam adiante e sofrem a pena.
\par 4 O galardão da humildade e o temor do SENHOR são riquezas, e honra, e vida.
\par 5 Espinhos e laços há no caminho do perverso; o que guarda a sua alma retira-se para longe deles.
\par 6 Ensina a criança no caminho em que deve andar, e, ainda quando for velho, não se desviará dele.
\par 7 O rico domina sobre o pobre, e o que toma emprestado é servo do que empresta.
\par 8 O que semeia a injustiça segará males; e a vara da sua indignação falhará.
\par 9 O generoso será abençoado, porque dá do seu pão ao pobre.
\par 10 Lança fora o escarnecedor, e com ele se irá a contenda; cessarão as demandas e a ignomínia.
\par 11 O que ama a pureza do coração e é grácil no falar terá por amigo o rei.
\par 12 Os olhos do SENHOR conservam aquele que tem conhecimento, mas as palavras do iníquo ele transtornará.
\par 13 Diz o preguiçoso: Um leão está lá fora; serei morto no meio das ruas.
\par 14 Cova profunda é a boca da mulher estranha; aquele contra quem o SENHOR se irar cairá nela.
\par 15 A estultícia está ligada ao coração da criança, mas a vara da disciplina a afastará dela.
\par 16 O que oprime ao pobre para enriquecer a si ou o que dá ao rico certamente empobrecerá.
\par 17 Inclina o ouvido, e ouve as palavras dos sábios, e aplica o coração ao meu conhecimento.
\par 18 Porque é coisa agradável os guardares no teu coração e os aplicares todos aos teus lábios.
\par 19 Para que a tua confiança esteja no SENHOR, quero dar-te hoje a instrução, a ti mesmo.
\par 20 Porventura, não te escrevi excelentes coisas acerca de conselhos e conhecimentos,
\par 21 para mostrar-te a certeza das palavras da verdade, a fim de que possas responder claramente aos que te enviarem?
\par 22 Não roubes ao pobre, porque é pobre, nem oprimas em juízo ao aflito,
\par 23 porque o SENHOR defenderá a causa deles e tirará a vida aos que os despojam.
\par 24 Não te associes com o iracundo, nem andes com o homem colérico,
\par 25 para que não aprendas as suas veredas e, assim, enlaces a tua alma.
\par 26 Não estejas entre os que se comprometem e ficam por fiadores de dívidas,
\par 27 pois, se não tens com que pagar, por que arriscas perder a cama de debaixo de ti?
\par 28 Não removas os marcos antigos que puseram teus pais.
\par 29 Vês a um homem perito na sua obra? Perante reis será posto; não entre a plebe.

\chapter{23}

\par 1 Quando te assentares a comer com um governador, atenta bem para aquele que está diante de ti;
\par 2 mete uma faca à tua garganta, se és homem glutão.
\par 3 Não cobices os seus delicados manjares, porque são comidas enganadoras.
\par 4 Não te fatigues para seres rico; não apliques nisso a tua inteligência.
\par 5 Porventura, fitarás os olhos naquilo que não é nada? Pois, certamente, a riqueza fará para si asas, como a águia que voa pelos céus.
\par 6 Não comas o pão do invejoso, nem cobices os seus delicados manjares.
\par 7 Porque, como imagina em sua alma, assim ele é; ele te diz: Come e bebe; mas o seu coração não está contigo.
\par 8 Vomitarás o bocado que comeste e perderás as tuas suaves palavras.
\par 9 Não fales aos ouvidos do insensato, porque desprezará a sabedoria das tuas palavras.
\par 10 Não removas os marcos antigos, nem entres nos campos dos órfãos,
\par 11 porque o seu Vingador é forte e lhes pleiteará a causa contra ti.
\par 12 Aplica o coração ao ensino e os ouvidos às palavras do conhecimento.
\par 13 Não retires da criança a disciplina, pois, se a fustigares com a vara, não morrerá.
\par 14 Tu a fustigarás com a vara e livrarás a sua alma do inferno.
\par 15 Filho meu, se o teu coração for sábio, alegrar-se-á também o meu;
\par 16 exultará o meu íntimo, quando os teus lábios falarem coisas retas.
\par 17 Não tenha o teu coração inveja dos pecadores; antes, no temor do SENHOR perseverarás todo dia.
\par 18 Porque deveras haverá bom futuro; não será frustrada a tua esperança.
\par 19 Ouve, filho meu, e sê sábio; guia retamente no caminho o teu coração.
\par 20 Não estejas entre os bebedores de vinho nem entre os comilões de carne.
\par 21 Porque o beberrão e o comilão caem em pobreza; e a sonolência vestirá de trapos o homem.
\par 22 Ouve a teu pai, que te gerou, e não desprezes a tua mãe, quando vier a envelhecer.
\par 23 Compra a verdade e não a vendas; compra a sabedoria, a instrução e o entendimento.
\par 24 Grandemente se regozijará o pai do justo, e quem gerar a um sábio nele se alegrará.
\par 25 Alegrem-se teu pai e tua mãe, e regozije-se a que te deu à luz.
\par 26 Dá-me, filho meu, o teu coração, e os teus olhos se agradem dos meus caminhos.
\par 27 Pois cova profunda é a prostituta, poço estreito, a alheia.
\par 28 Ela, como salteador, se põe a espreitar e multiplica entre os homens os infiéis.
\par 29 Para quem são os ais? Para quem, os pesares? Para quem, as rixas? Para quem, as queixas? Para quem, as feridas sem causa? E para quem, os olhos vermelhos?
\par 30 Para os que se demoram em beber vinho, para os que andam buscando bebida misturada.
\par 31 Não olhes para o vinho, quando se mostra vermelho, quando resplandece no copo e se escoa suavemente.
\par 32 Pois ao cabo morderá como a cobra e picará como o basilisco.
\par 33 Os teus olhos verão coisas esquisitas, e o teu coração falará perversidades.
\par 34 Serás como o que se deita no meio do mar e como o que se deita no alto do mastro
\par 35 e dirás: Espancaram-me, e não me doeu; bateram-me, e não o senti; quando despertarei? Então, tornarei a beber.

\chapter{24}

\par 1 Não tenhas inveja dos homens malignos, nem queiras estar com eles,
\par 2 porque o seu coração maquina violência, e os seus lábios falam para o mal.
\par 3 Com a sabedoria edifica-se a casa, e com a inteligência ela se firma;
\par 4 pelo conhecimento se encherão as câmaras de toda sorte de bens, preciosos e deleitáveis.
\par 5 Mais poder tem o sábio do que o forte, e o homem de conhecimento, mais do que o robusto.
\par 6 Com medidas de prudência farás a guerra; na multidão de conselheiros está a vitória.
\par 7 A sabedoria é alta demais para o insensato; no juízo, a sua boca não terá palavra.
\par 8 Ao que cuida em fazer o mal, mestre de intrigas lhe chamarão.
\par 9 Os desígnios do insensato são pecado, e o escarnecedor é abominável aos homens.
\par 10 Se te mostras fraco no dia da angústia, a tua força é pequena.
\par 11 Livra os que estão sendo levados para a morte e salva os que cambaleiam indo para serem mortos.
\par 12 Se disseres: Não o soubemos, não o perceberá aquele que pesa os corações? Não o saberá aquele que atenta para a tua alma? E não pagará ele ao homem segundo as suas obras?
\par 13 Filho meu, saboreia o mel, porque é saudável, e o favo, porque é doce ao teu paladar.
\par 14 Então, sabe que assim é a sabedoria para a tua alma; se a achares, haverá bom futuro, e não será frustrada a tua esperança.
\par 15 Não te ponhas de emboscada, ó perverso, contra a habitação do justo, nem assoles o lugar do seu repouso,
\par 16 porque sete vezes cairá o justo e se levantará; mas os perversos são derribados pela calamidade.
\par 17 Quando cair o teu inimigo, não te alegres, e não se regozije o teu coração quando ele tropeçar;
\par 18 para que o SENHOR não veja isso, e lhe desagrade, e desvie dele a sua ira.
\par 19 Não te aflijas por causa dos malfeitores, nem tenhas inveja dos perversos,
\par 20 porque o maligno não terá bom futuro, e a lâmpada dos perversos se apagará.
\par 21 Teme ao SENHOR, filho meu, e ao rei e não te associes com os revoltosos.
\par 22 Porque de repente levantará a sua perdição, e a ruína que virá daqueles dois, quem a conhecerá?
\par 23 São também estes provérbios dos sábios. Parcialidade no julgar não é bom.
\par 24 O que disser ao perverso: Tu és justo; pelo povo será maldito e detestado pelas nações.
\par 25 Mas os que o repreenderem se acharão bem, e sobre eles virão grandes bênçãos.
\par 26 Como beijo nos lábios, é a resposta com palavras retas.
\par 27 Cuida dos teus negócios lá fora, apronta a lavoura no campo e, depois, edifica a tua casa.
\par 28 Não sejas testemunha sem causa contra o teu próximo, nem o enganes com os teus lábios.
\par 29 Não digas: Como ele me fez a mim, assim lhe farei a ele; pagarei a cada um segundo a sua obra.
\par 30 Passei pelo campo do preguiçoso e junto à vinha do homem falto de entendimento;
\par 31 eis que tudo estava cheio de espinhos, a sua superfície, coberta de urtigas, e o seu muro de pedra, em ruínas.
\par 32 Tendo-o visto, considerei; vi e recebi a instrução.
\par 33 Um pouco para dormir, um pouco para tosquenejar, um pouco para encruzar os braços em repouso,
\par 34 assim sobrevirá a tua pobreza como um ladrão, e a tua necessidade, como um homem armado.

\chapter{25}

\par 1 São também estes provérbios de Salomão, os quais transcreveram os homens de Ezequias, rei de Judá.
\par 2 A glória de Deus é encobrir as coisas, mas a glória dos reis é esquadrinhá-las.
\par 3 Como a altura dos céus e a profundeza da terra, assim o coração dos reis é insondável.
\par 4 Tira da prata a escória, e sairá vaso para o ourives;
\par 5 tira o perverso da presença do rei, e o seu trono se firmará na justiça.
\par 6 Não te glories na presença do rei, nem te ponhas no meio dos grandes;
\par 7 porque melhor é que te digam: Sobe para aqui!, do que seres humilhado diante do príncipe. A respeito do que os teus olhos viram,
\par 8 não te apresses a litigar, pois, ao fim, que farás, quando o teu próximo te puser em apuros?
\par 9 Pleiteia a tua causa diretamente com o teu próximo e não descubras o segredo de outrem;
\par 10 para que não te vitupere aquele que te ouvir, e não se te apegue a tua infâmia.
\par 11 Como maçãs de ouro em salvas de prata, assim é a palavra dita a seu tempo.
\par 12 Como pendentes e jóias de ouro puro, assim é o sábio repreensor para o ouvido atento.
\par 13 Como o frescor de neve no tempo da ceifa, assim é o mensageiro fiel para com os que o enviam, porque refrigera a alma dos seus senhores.
\par 14 Como nuvens e ventos que não trazem chuva, assim é o homem que se gaba de dádivas que não fez.
\par 15 A longanimidade persuade o príncipe, e a língua branda esmaga ossos.
\par 16 Achaste mel? Come apenas o que te basta, para que não te fartes dele e venhas a vomitá-lo.
\par 17 Não sejas freqüente na casa do teu próximo, para que não se enfade de ti e te aborreça.
\par 18 Maça, espada e flecha aguda é o homem que levanta falso testemunho contra o seu próximo.
\par 19 Como dente quebrado e pé sem firmeza, assim é a confiança no desleal, no tempo da angústia.
\par 20 Como quem se despe num dia de frio e como vinagre sobre feridas, assim é o que entoa canções junto ao coração aflito.
\par 21 Se o que te aborrece tiver fome, dá-lhe pão para comer; se tiver sede, dá-lhe água para beber,
\par 22 porque assim amontoarás brasas vivas sobre a sua cabeça, e o SENHOR te retribuirá.
\par 23 O vento norte traz chuva, e a língua fingida, o rosto irado.
\par 24 Melhor é morar no canto do eirado do que junto com a mulher rixosa na mesma casa.
\par 25 Como água fria para o sedento, tais são as boas-novas vindas de um país remoto.
\par 26 Como fonte que foi turvada e manancial corrupto, assim é o justo que cede ao perverso.
\par 27 Comer muito mel não é bom; assim, procurar a própria honra não é honra.
\par 28 Como cidade derribada, que não tem muros, assim é o homem que não tem domínio próprio.

\chapter{26}

\par 1 Como a neve no verão e como a chuva na ceifa, assim, a honra não convém ao insensato.
\par 2 Como o pássaro que foge, como a andorinha no seu vôo, assim, a maldição sem causa não se cumpre.
\par 3 O açoite é para o cavalo, o freio, para o jumento, e a vara, para as costas dos insensatos.
\par 4 Não respondas ao insensato segundo a sua estultícia, para que não te faças semelhante a ele.
\par 5 Ao insensato responde segundo a sua estultícia, para que não seja ele sábio aos seus próprios olhos.
\par 6 Os pés corta e o dano sofre quem manda mensagens por intermédio do insensato.
\par 7 As pernas do coxo pendem bambas; assim é o provérbio na boca dos insensatos.
\par 8 Como o que atira pedra preciosa num montão de ruínas, assim é o que dá honra ao insensato.
\par 9 Como galho de espinhos na mão do bêbado, assim é o provérbio na boca dos insensatos.
\par 10 Como um flecheiro que a todos fere, assim é o que assalaria os insensatos e os transgressores.
\par 11 Como o cão que torna ao seu vômito, assim é o insensato que reitera a sua estultícia.
\par 12 Tens visto a um homem que é sábio a seus próprios olhos? Maior esperança há no insensato do que nele.
\par 13 Diz o preguiçoso: Um leão está no caminho; um leão está nas ruas.
\par 14 Como a porta se revolve nos seus gonzos, assim, o preguiçoso, no seu leito.
\par 15 O preguiçoso mete a mão no prato e não quer ter o trabalho de a levar à boca.
\par 16 Mais sábio é o preguiçoso a seus próprios olhos do que sete homens que sabem responder bem.
\par 17 Quem se mete em questão alheia é como aquele que toma pelas orelhas um cão que passa.
\par 18 Como o louco que lança fogo, flechas e morte,
\par 19 assim é o homem que engana a seu próximo e diz: Fiz isso por brincadeira.
\par 20 Sem lenha, o fogo se apaga; e, não havendo maldizente, cessa a contenda.
\par 21 Como o carvão é para a brasa, e a lenha, para o fogo, assim é o homem contencioso para acender rixas.
\par 22 As palavras do maldizente são comida fina, que desce para o mais interior do ventre.
\par 23 Como vaso de barro coberto de escórias de prata, assim são os lábios amorosos e o coração maligno.
\par 24 Aquele que aborrece dissimula com os lábios, mas no íntimo encobre o engano;
\par 25 quando te falar suavemente, não te fies nele, porque sete abominações há no seu coração.
\par 26 Ainda que o seu ódio se encobre com engano, a sua malícia se descobrirá publicamente.
\par 27 Quem abre uma cova nela cairá; e a pedra rolará sobre quem a revolve.
\par 28 A língua falsa aborrece a quem feriu, e a boca lisonjeira é causa de ruína.

\chapter{27}

\par 1 Não te glories do dia de amanhã, porque não sabes o que trará à luz.
\par 2 Seja outro o que te louve, e não a tua boca; o estrangeiro, e não os teus lábios.
\par 3 Pesada é a pedra, e a areia é uma carga; mas a ira do insensato é mais pesada do que uma e outra.
\par 4 Cruel é o furor, e impetuosa, a ira, mas quem pode resistir à inveja?
\par 5 Melhor é a repreensão franca do que o amor encoberto.
\par 6 Leais são as feridas feitas pelo que ama, porém os beijos de quem odeia são enganosos.
\par 7 A alma farta pisa o favo de mel, mas à alma faminta todo amargo é doce.
\par 8 Qual ave que vagueia longe do seu ninho, tal é o homem que anda vagueando longe do seu lar.
\par 9 Como o óleo e o perfume alegram o coração, assim, o amigo encontra doçura no conselho cordial.
\par 10 Não abandones o teu amigo, nem o amigo de teu pai, nem entres na casa de teu irmão no dia da tua adversidade. Mais vale o vizinho perto do que o irmão longe.
\par 11 Sê sábio, filho meu, e alegra o meu coração, para que eu saiba responder àqueles que me afrontam.
\par 12 O prudente vê o mal e esconde-se; mas os simples passam adiante e sofrem a pena.
\par 13 Tome-se a roupa àquele que fica fiador por outrem; e, por penhor, àquele que se obriga por mulher estranha.
\par 14 O que bendiz ao seu vizinho em alta voz, logo de manhã, por maldição lhe atribuem o que faz.
\par 15 O gotejar contínuo no dia de grande chuva e a mulher rixosa são semelhantes;
\par 16 contê-la seria conter o vento, seria pegar o óleo na mão.
\par 17 Como o ferro com o ferro se afia, assim, o homem, ao seu amigo.
\par 18 O que trata da figueira comerá do seu fruto; e o que cuida do seu senhor será honrado.
\par 19 Como na água o rosto corresponde ao rosto, assim, o coração do homem, ao homem.
\par 20 O inferno e o abismo nunca se fartam, e os olhos do homem nunca se satisfazem.
\par 21 Como o crisol prova a prata, e o forno, o ouro, assim, o homem é provado pelos louvores que recebe.
\par 22 Ainda que pises o insensato com mão de gral entre grãos pilados de cevada, não se vai dele a sua estultícia.
\par 23 Procura conhecer o estado das tuas ovelhas e cuida dos teus rebanhos,
\par 24 porque as riquezas não duram para sempre, nem a coroa, de geração em geração.
\par 25 Quando, removido o feno, aparecerem os renovos e se recolherem as ervas dos montes,
\par 26 então, os cordeiros te darão as vestes, os bodes, o preço do campo,
\par 27 e as cabras, leite em abundância para teu alimento, para alimento da tua casa e para sustento das tuas servas.

\chapter{28}

\par 1 Fogem os perversos, sem que ninguém os persiga; mas o justo é intrépido como o leão.
\par 2 Por causa da transgressão da terra, mudam-se freqüentemente os príncipes, mas por um, sábio e prudente, se faz estável a sua ordem.
\par 3 O homem pobre que oprime os pobres é como chuva que a tudo arrasta e não deixa trigo.
\par 4 Os que desamparam a lei louvam o perverso, mas os que guardam a lei se indignam contra ele.
\par 5 Os homens maus não entendem o que é justo, mas os que buscam o SENHOR entendem tudo.
\par 6 Melhor é o pobre que anda na sua integridade do que o perverso, nos seus caminhos, ainda que seja rico.
\par 7 O que guarda a lei é filho prudente, mas o companheiro de libertinos envergonha a seu pai.
\par 8 O que aumenta os seus bens com juros e ganância ajunta-os para o que se compadece do pobre.
\par 9 O que desvia os ouvidos de ouvir a lei, até a sua oração será abominável.
\par 10 O que desvia os retos para o mau caminho, ele mesmo cairá na cova que fez, mas os íntegros herdarão o bem.
\par 11 O homem rico é sábio aos seus próprios olhos; mas o pobre que é sábio sabe sondá-lo.
\par 12 Quando triunfam os justos, há grande festividade; quando, porém, sobem os perversos, os homens se escondem.
\par 13 O que encobre as suas transgressões jamais prosperará; mas o que as confessa e deixa alcançará misericórdia.
\par 14 Feliz o homem constante no temor de Deus; mas o que endurece o coração cairá no mal.
\par 15 Como leão que ruge e urso que ataca, assim é o perverso que domina sobre um povo pobre.
\par 16 O príncipe falto de inteligência multiplica as opressões, mas o que aborrece a avareza viverá muitos anos.
\par 17 O homem carregado do sangue de outrem fugirá até à cova; ninguém o detenha.
\par 18 O que anda em integridade será salvo, mas o perverso em seus caminhos cairá logo.
\par 19 O que lavra a sua terra virá a fartar-se de pão, mas o que se ajunta a vadios se fartará de pobreza.
\par 20 O homem fiel será cumulado de bênçãos, mas o que se apressa a enriquecer não passará sem castigo.
\par 21 Parcialidade não é bom, porque até por um bocado de pão o homem prevaricará.
\par 22 Aquele que tem olhos invejosos corre atrás das riquezas, mas não sabe que há de vir sobre ele a penúria.
\par 23 O que repreende ao homem achará, depois, mais favor do que aquele que lisonjeia com a língua.
\par 24 O que rouba a seu pai ou a sua mãe e diz: Não é pecado, companheiro é do destruidor.
\par 25 O cobiçoso levanta contendas, mas o que confia no SENHOR prosperará.
\par 26 O que confia no seu próprio coração é insensato, mas o que anda em sabedoria será salvo.
\par 27 O que dá ao pobre não terá falta, mas o que dele esconde os olhos será cumulado de maldições.
\par 28 Quando sobem os perversos, os homens se escondem, mas, quando eles perecem, os justos se multiplicam.

\chapter{29}

\par 1 O homem que muitas vezes repreendido endurece a cerviz será quebrantado de repente sem que haja cura.
\par 2 Quando se multiplicam os justos, o povo se alegra, quando, porém, domina o perverso, o povo suspira.
\par 3 O homem que ama a sabedoria alegra a seu pai, mas o companheiro de prostitutas desperdiça os bens.
\par 4 O rei justo sustém a terra, mas o amigo de impostos a transtorna.
\par 5 O homem que lisonjeia a seu próximo arma-lhe uma rede aos passos.
\par 6 Na transgressão do homem mau, há laço, mas o justo canta e se regozija.
\par 7 Informa-se o justo da causa dos pobres, mas o perverso de nada disso quer saber.
\par 8 Os homens escarnecedores alvoroçam a cidade, mas os sábios desviam a ira.
\par 9 Se o homem sábio discute com o insensato, quer este se encolerize, quer se ria, não haverá fim.
\par 10 Os sanguinários aborrecem o íntegro, ao passo que, quanto aos retos, procuram tirar-lhes a vida.
\par 11 O insensato expande toda a sua ira, mas o sábio afinal lha reprime.
\par 12 Se o governador dá atenção a palavras mentirosas, virão a ser perversos todos os seus servos.
\par 13 O pobre e o seu opressor se encontram, mas é o SENHOR quem dá luz aos olhos de ambos.
\par 14 O rei que julga os pobres com eqüidade firmará o seu trono para sempre.
\par 15 A vara e a disciplina dão sabedoria, mas a criança entregue a si mesma vem a envergonhar a sua mãe.
\par 16 Quando os perversos se multiplicam, multiplicam-se as transgressões, mas os justos verão a ruína deles.
\par 17 Corrige o teu filho, e te dará descanso, dará delícias à tua alma.
\par 18 Não havendo profecia, o povo se corrompe; mas o que guarda a lei, esse é feliz.
\par 19 O servo não se emendará com palavras, porque, ainda que entenda, não obedecerá.
\par 20 Tens visto um homem precipitado nas suas palavras? Maior esperança há para o insensato do que para ele.
\par 21 Se alguém amimar o escravo desde a infância, por fim ele quererá ser filho.
\par 22 O iracundo levanta contendas, e o furioso multiplica as transgressões.
\par 23 A soberba do homem o abaterá, mas o humilde de espírito obterá honra.
\par 24 O que tem parte com o ladrão aborrece a própria alma; ouve as maldições e nada denuncia.
\par 25 Quem teme ao homem arma ciladas, mas o que confia no SENHOR está seguro.
\par 26 Muitos buscam o favor daquele que governa, mas para o homem a justiça vem do SENHOR.
\par 27 Para o justo, o iníquo é abominação, e o reto no seu caminho é abominação ao perverso.

\chapter{30}

\par 1 Palavras de Agur, filho de Jaque, de Massá. Disse o homem: Fatiguei-me, ó Deus; fatiguei-me, ó Deus, e estou exausto
\par 2 porque sou demasiadamente estúpido para ser homem; não tenho inteligência de homem,
\par 3 não aprendi a sabedoria, nem tenho o conhecimento do Santo.
\par 4 Quem subiu ao céu e desceu? Quem encerrou os ventos nos seus punhos? Quem amarrou as águas na sua roupa? Quem estabeleceu todas as extremidades da terra? Qual é o seu nome, e qual é o nome de seu filho, se é que o sabes?
\par 5 Toda palavra de Deus é pura; ele é escudo para os que nele confiam.
\par 6 Nada acrescentes às suas palavras, para que não te repreenda, e sejas achado mentiroso.
\par 7 Duas coisas te peço; não mas negues, antes que eu morra:
\par 8 afasta de mim a falsidade e a mentira; não me dês nem a pobreza nem a riqueza; dá-me o pão que me for necessário;
\par 9 para não suceder que, estando eu farto, te negue e diga: Quem é o SENHOR? Ou que, empobrecido, venha a furtar e profane o nome de Deus.
\par 10 Não calunies o servo diante de seu senhor, para que aquele te não amaldiçoe e fiques culpado.
\par 11 Há daqueles que amaldiçoam a seu pai e que não bendizem a sua mãe.
\par 12 Há daqueles que são puros aos próprios olhos e que jamais foram lavados da sua imundícia.
\par 13 Há daqueles -- quão altivos são os seus olhos e levantadas as suas pálpebras!
\par 14 Há daqueles cujos dentes são espadas, e cujos queixais são facas, para consumirem na terra os aflitos e os necessitados entre os homens.
\par 15 A sanguessuga tem duas filhas, a saber: Dá, Dá. Há três coisas que nunca se fartam, sim, quatro que não dizem: Basta!
\par 16 Elas são a sepultura, a madre estéril, a terra, que se não farta de água, e o fogo, que nunca diz: Basta!
\par 17 Os olhos de quem zomba do pai ou de quem despreza a obediência à sua mãe, corvos no ribeiro os arrancarão e pelos pintãos da águia serão comidos.
\par 18 Há três coisas que são maravilhosas demais para mim, sim, há quatro que não entendo:
\par 19 o caminho da águia no céu, o caminho da cobra na penha, o caminho do navio no meio do mar e o caminho do homem com uma donzela.
\par 20 Tal é o caminho da mulher adúltera: come, e limpa a boca, e diz: Não cometi maldade.
\par 21 Sob três coisas estremece a terra, sim, sob quatro não pode subsistir:
\par 22 sob o servo quando se torna rei; sob o insensato quando anda farto de pão;
\par 23 sob a mulher desdenhada quando se casa; sob a serva quando se torna herdeira da sua senhora.
\par 24 Há quatro coisas mui pequenas na terra que, porém, são mais sábias que os sábios:
\par 25 as formigas, povo sem força; todavia, no verão preparam a sua comida;
\par 26 os arganazes, povo não poderoso; contudo, fazem a sua casa nas rochas;
\par 27 os gafanhotos não têm rei; contudo, marcham todos em bandos;
\par 28 o geco, que se apanha com as mãos; contudo, está nos palácios dos reis.
\par 29 Há três que têm passo elegante, sim, quatro que andam airosamente:
\par 30 O leão, o mais forte entre os animais, que por ninguém torna atrás;
\par 31 o galo, que anda ereto, o bode e o rei, a quem não se pode resistir.
\par 32 Se procedeste insensatamente em te exaltares ou se maquinaste o mal, põe a mão na boca.
\par 33 Porque o bater do leite produz manteiga, e o torcer do nariz produz sangue, e o açular a ira produz contendas.

\chapter{31}

\par 1 Palavras do rei Lemuel, de Massá, as quais lhe ensinou sua mãe.
\par 2 Que te direi, filho meu? Ó filho do meu ventre? Que te direi, ó filho dos meus votos?
\par 3 Não dês às mulheres a tua força, nem os teus caminhos, às que destroem os reis.
\par 4 Não é próprio dos reis, ó Lemuel, não é próprio dos reis beber vinho, nem dos príncipes desejar bebida forte.
\par 5 Para que não bebam, e se esqueçam da lei, e pervertam o direito de todos os aflitos.
\par 6 Dai bebida forte aos que perecem e vinho, aos amargurados de espírito;
\par 7 para que bebam, e se esqueçam da sua pobreza, e de suas fadigas não se lembrem mais.
\par 8 Abre a boca a favor do mudo, pelo direito de todos os que se acham desamparados.
\par 9 Abre a boca, julga retamente e faze justiça aos pobres e aos necessitados.
\par 10 Mulher virtuosa, quem a achará? O seu valor muito excede o de finas jóias.
\par 11 O coração do seu marido confia nela, e não haverá falta de ganho.
\par 12 Ela lhe faz bem e não mal, todos os dias da sua vida.
\par 13 Busca lã e linho e de bom grado trabalha com as mãos.
\par 14 É como o navio mercante: de longe traz o seu pão.
\par 15 É ainda noite, e já se levanta, e dá mantimento à sua casa e a tarefa às suas servas.
\par 16 Examina uma propriedade e adquire-a; planta uma vinha com as rendas do seu trabalho.
\par 17 Cinge os lombos de força e fortalece os braços.
\par 18 Ela percebe que o seu ganho é bom; a sua lâmpada não se apaga de noite.
\par 19 Estende as mãos ao fuso, mãos que pegam na roca.
\par 20 Abre a mão ao aflito; e ainda a estende ao necessitado.
\par 21 No tocante à sua casa, não teme a neve, pois todos andam vestidos de lã escarlate.
\par 22 Faz para si cobertas, veste-se de linho fino e de púrpura.
\par 23 Seu marido é estimado entre os juízes, quando se assenta com os anciãos da terra.
\par 24 Ela faz roupas de linho fino, e vende-as, e dá cintas aos mercadores.
\par 25 A força e a dignidade são os seus vestidos, e, quanto ao dia de amanhã, não tem preocupações.
\par 26 Fala com sabedoria, e a instrução da bondade está na sua língua.
\par 27 Atende ao bom andamento da sua casa e não come o pão da preguiça.
\par 28 Levantam-se seus filhos e lhe chamam ditosa; seu marido a louva, dizendo:
\par 29 Muitas mulheres procedem virtuosamente, mas tu a todas sobrepujas.
\par 30 Enganosa é a graça, e vã, a formosura, mas a mulher que teme ao SENHOR, essa será louvada.
\par 31 Dai-lhe do fruto das suas mãos, e de público a louvarão as suas obras.


\end{document}