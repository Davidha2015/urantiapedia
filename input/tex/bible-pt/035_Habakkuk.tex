\begin{document}

\title{Habacuque}


\chapter{1}

\par 1 Sentença revelada ao profeta Habacuque.
\par 2 Até quando, SENHOR, clamarei eu, e tu não me escutarás? Gritar-te-ei: Violência! E não salvarás?
\par 3 Por que me mostras a iniqüidade e me fazes ver a opressão? Pois a destruição e a violência estão diante de mim; há contendas, e o litígio se suscita.
\par 4 Por esta causa, a lei se afrouxa, e a justiça nunca se manifesta, porque o perverso cerca o justo, a justiça é torcida.
\par 5 Vede entre as nações, olhai, maravilhai-vos e desvanecei, porque realizo, em vossos dias, obra tal, que vós não crereis, quando vos for contada.
\par 6 Pois eis que suscito os caldeus, nação amarga e impetuosa, que marcham pela largura da terra, para apoderar-se de moradas que não são suas.
\par 7 Eles são pavorosos e terríveis, e criam eles mesmos o seu direito e a sua dignidade.
\par 8 Os seus cavalos são mais ligeiros do que os leopardos, mais ferozes do que os lobos ao anoitecer são os seus cavaleiros que se espalham por toda parte; sim, os seus cavaleiros chegam de longe, voam como águia que se precipita a devorar.
\par 9 Eles todos vêm para fazer violência; o seu rosto suspira por seguir avante; eles reúnem os cativos como areia.
\par 10 Eles escarnecem dos reis; os príncipes são objeto do seu riso; riem-se de todas as fortalezas, porque, amontoando terra, as tomam.
\par 11 Então, passam como passa o vento e seguem; fazem-se culpados estes cujo poder é o seu deus.
\par 12 Não és tu desde a eternidade, ó SENHOR, meu Deus, ó meu Santo? Não morreremos. Ó SENHOR, para executar juízo, puseste aquele povo; tu, ó Rocha, o fundaste para servir de disciplina.
\par 13 Tu és tão puro de olhos, que não podes ver o mal e a opressão não podes contemplar; por que, pois, toleras os que procedem perfidamente e te calas quando o perverso devora aquele que é mais justo do que ele?
\par 14 Por que fazes os homens como os peixes do mar, como os répteis, que não têm quem os governe?
\par 15 A todos levanta o inimigo com o anzol, pesca-os de arrastão e os ajunta na sua rede varredoura; por isso, ele se alegra e se regozija.
\par 16 Por isso, oferece sacrifício à sua rede e queima incenso à sua varredoura; porque por elas enriqueceu a sua porção, e tem gordura a sua comida.
\par 17 Acaso, continuará, por isso, esvaziando a sua rede e matando sem piedade os povos?

\chapter{2}

\par 1 Pôr-me-ei na minha torre de vigia, colocar-me-ei sobre a fortaleza e vigiarei para ver o que Deus me dirá e que resposta eu terei à minha queixa.
\par 2 O SENHOR me respondeu e disse: Escreve a visão, grava-a sobre tábuas, para que a possa ler até quem passa correndo.
\par 3 Porque a visão ainda está para cumprir-se no tempo determinado, mas se apressa para o fim e não falhará; se tardar, espera-o, porque, certamente, virá, não tardará.
\par 4 Eis o soberbo! Sua alma não é reta nele; mas o justo viverá pela sua fé.
\par 5 Assim como o vinho é enganoso, tampouco permanece o arrogante, cuja gananciosa boca se escancara como o sepulcro e é como a morte, que não se farta; ele ajunta para si todas as nações e congrega todos os povos.
\par 6 Não levantarão, pois, todos estes contra ele um provérbio, um dito zombador? Dirão: Ai daquele que acumula o que não é seu (até quando?), e daquele que a si mesmo se carrega de penhores!
\par 7 Não se levantarão de repente os teus credores? E não despertarão os que te hão de abalar? Tu lhes servirás de despojo.
\par 8 Visto como despojaste a muitas nações, todos os mais povos te despojarão a ti, por causa do sangue dos homens e da violência contra a terra, contra a cidade e contra todos os seus moradores.
\par 9 Ai daquele que ajunta em sua casa bens mal adquiridos, para pôr em lugar alto o seu ninho, a fim de livrar-se das garras do mal!
\par 10 Vergonha maquinaste para a tua casa; destruindo tu a muitos povos, pecaste contra a tua alma.
\par 11 Porque a pedra clamará da parede, e a trave lhe responderá do madeiramento.
\par 12 Ai daquele que edifica a cidade com sangue e a fundamenta com iniqüidade!
\par 13 Não vem do SENHOR dos Exércitos que as nações labutem para o fogo e os povos se fatiguem em vão?
\par 14 Pois a terra se encherá do conhecimento da glória do SENHOR, como as águas cobrem o mar.
\par 15 Ai daquele que dá de beber ao seu companheiro, misturando à bebida o seu furor, e que o embebeda para lhe contemplar as vergonhas!
\par 16 Serás farto de opróbrio em vez de honra; bebe tu também e exibe a tua incircuncisão; chegará a tua vez de tomares o cálice da mão direita do SENHOR, e ignomínia cairá sobre a tua glória.
\par 17 Porque a violência contra o Líbano te cobrirá, e a destruição que fizeste dos animais ferozes te assombrará, por causa do sangue dos homens e da violência contra a terra, contra a cidade e contra todos os seus moradores.
\par 18 Que aproveita o ídolo, visto que o seu artífice o esculpiu? E a imagem de fundição, mestra de mentiras, para que o artífice confie na obra, fazendo ídolos mudos?
\par 19 Ai daquele que diz à madeira: Acorda! E à pedra muda: Desperta! Pode o ídolo ensinar? Eis que está coberto de ouro e de prata, mas, no seu interior, não há fôlego nenhum.
\par 20 O SENHOR, porém, está no seu santo templo; cale-se diante dele toda a terra.

\chapter{3}

\par 1 Oração do profeta Habacuque sob a forma de canto.
\par 2 Tenho ouvido, ó SENHOR, as tuas declarações, e me sinto alarmado; aviva a tua obra, ó SENHOR, no decorrer dos anos, e, no decurso dos anos, faze-a conhecida; na tua ira, lembra-te da misericórdia.
\par 3 Deus vem de Temã, e do monte Parã vem o Santo. A sua glória cobre os céus, e a terra se enche do seu louvor.
\par 4 O seu resplendor é como a luz, raios brilham da sua mão; e ali está velado o seu poder.
\par 5 Adiante dele vai a peste, e a pestilência segue os seus passos.
\par 6 Ele pára e faz tremer a terra; olha e sacode as nações. Esmigalham-se os montes primitivos; os outeiros eternos se abatem. Os caminhos de Deus são eternos.
\par 7 Vejo as tendas de Cusã em aflição; os acampamentos da terra de Midiã tremem.
\par 8 Acaso, é contra os rios, SENHOR, que estás irado? É contra os ribeiros a tua ira ou contra o mar, o teu furor, já que andas montado nos teus cavalos, nos teus carros de vitória?
\par 9 Tiras a descoberto o teu arco, e farta está a tua aljava de flechas. Tu fendes a terra com rios.
\par 10 Os montes te vêem e se contorcem; passam torrentes de água; as profundezas do mar fazem ouvir a sua voz e levantam bem alto as suas mãos.
\par 11 O sol e a lua param nas suas moradas, ao resplandecer a luz das tuas flechas sibilantes, ao fulgor do relâmpago da tua lança.
\par 12 Na tua indignação, marchas pela terra, na tua ira, calcas aos pés as nações.
\par 13 Tu sais para salvamento do teu povo, para salvar o teu ungido; feres o telhado da casa do perverso e lhe descobres de todo o fundamento.
\par 14 Traspassas a cabeça dos guerreiros do inimigo com as suas próprias lanças, os quais, como tempestade, avançam para me destruir; regozijam-se, como se estivessem para devorar o pobre às ocultas.
\par 15 Marchas com os teus cavalos pelo mar, pela massa de grandes águas.
\par 16 Ouvi-o, e o meu íntimo se comoveu, à sua voz, tremeram os meus lábios; entrou a podridão nos meus ossos, e os joelhos me vacilaram, pois, em silêncio, devo esperar o dia da angústia, que virá contra o povo que nos acomete.
\par 17 Ainda que a figueira não floresça, nem haja fruto na vide; o produto da oliveira minta, e os campos não produzam mantimento; as ovelhas sejam arrebatadas do aprisco, e nos currais não haja gado,
\par 18 todavia, eu me alegro no SENHOR, exulto no Deus da minha salvação.
\par 19 O SENHOR Deus é a minha fortaleza, e faz os meus pés como os da corça, e me faz andar altaneiramente. Ao mestre de canto. Para instrumentos de cordas.


\end{document}