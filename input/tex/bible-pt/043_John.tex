\begin{document}

\title{João}


\chapter{1}

\par 1 No princípio era o Verbo, e o Verbo estava com Deus, e o Verbo era Deus.
\par 2 Ele estava no princípio com Deus.
\par 3 Todas as coisas foram feitas por intermédio dele, e, sem ele, nada do que foi feito se fez.
\par 4 A vida estava nele e a vida era a luz dos homens.
\par 5 A luz resplandece nas trevas, e as trevas não prevaleceram contra ela.
\par 6 Houve um homem enviado por Deus cujo nome era João.
\par 7 Este veio como testemunha para que testificasse a respeito da luz, a fim de todos virem a crer por intermédio dele.
\par 8 Ele não era a luz, mas veio para que testificasse da luz,
\par 9 a saber, a verdadeira luz, que, vinda ao mundo, ilumina a todo homem.
\par 10 O Verbo estava no mundo, o mundo foi feito por intermédio dele, mas o mundo não o conheceu.
\par 11 Veio para o que era seu, e os seus não o receberam.
\par 12 Mas, a todos quantos o receberam, deu-lhes o poder de serem feitos filhos de Deus, a saber, aos que crêem no seu nome;
\par 13 os quais não nasceram do sangue, nem da vontade da carne, nem da vontade do homem, mas de Deus.
\par 14 E o Verbo se fez carne e habitou entre nós, cheio de graça e de verdade, e vimos a sua glória, glória como do unigênito do Pai.
\par 15 João testemunha a respeito dele e exclama: Este é o de quem eu disse: o que vem depois de mim tem, contudo, a primazia, porquanto já existia antes de mim.
\par 16 Porque todos nós temos recebido da sua plenitude e graça sobre graça.
\par 17 Porque a lei foi dada por intermédio de Moisés; a graça e a verdade vieram por meio de Jesus Cristo.
\par 18 Ninguém jamais viu a Deus; o Deus unigênito, que está no seio do Pai, é quem o revelou.
\par 19 Este foi o testemunho de João, quando os judeus lhe enviaram de Jerusalém sacerdotes e levitas para lhe perguntarem: Quem és tu?
\par 20 Ele confessou e não negou; confessou: Eu não sou o Cristo.
\par 21 Então, lhe perguntaram: Quem és, pois? És tu Elias? Ele disse: Não sou. És tu o profeta? Respondeu: Não.
\par 22 Disseram-lhe, pois: Declara-nos quem és, para que demos resposta àqueles que nos enviaram; que dizes a respeito de ti mesmo?
\par 23 Então, ele respondeu: Eu sou a voz do que clama no deserto: Endireitai o caminho do Senhor, como disse o profeta Isaías.
\par 24 Ora, os que haviam sido enviados eram de entre os fariseus.
\par 25 E perguntaram-lhe: Então, por que batizas, se não és o Cristo, nem Elias, nem o profeta?
\par 26 Respondeu-lhes João: Eu batizo com água; mas, no meio de vós, está quem vós não conheceis,
\par 27 o qual vem após mim, do qual não sou digno de desatar-lhe as correias das sandálias.
\par 28 Estas coisas se passaram em Betânia, do outro lado do Jordão, onde João estava batizando.
\par 29 No dia seguinte, viu João a Jesus, que vinha para ele, e disse: Eis o Cordeiro de Deus, que tira o pecado do mundo!
\par 30 É este a favor de quem eu disse: após mim vem um varão que tem a primazia, porque já existia antes de mim.
\par 31 Eu mesmo não o conhecia, mas, a fim de que ele fosse manifestado a Israel, vim, por isso, batizando com água.
\par 32 E João testemunhou, dizendo: Vi o Espírito descer do céu como pomba e pousar sobre ele.
\par 33 Eu não o conhecia; aquele, porém, que me enviou a batizar com água me disse: Aquele sobre quem vires descer e pousar o Espírito, esse é o que batiza com o Espírito Santo.
\par 34 Pois eu, de fato, vi e tenho testificado que ele é o Filho de Deus.
\par 35 No dia seguinte, estava João outra vez na companhia de dois dos seus discípulos
\par 36 e, vendo Jesus passar, disse: Eis o Cordeiro de Deus!
\par 37 Os dois discípulos, ouvindo-o dizer isto, seguiram Jesus.
\par 38 E Jesus, voltando-se e vendo que o seguiam, disse-lhes: Que buscais? Disseram-lhe: Rabi (que quer dizer Mestre), onde assistes?
\par 39 Respondeu-lhes: Vinde e vede. Foram, pois, e viram onde Jesus estava morando; e ficaram com ele aquele dia, sendo mais ou menos a hora décima.
\par 40 Era André, o irmão de Simão Pedro, um dos dois que tinham ouvido o testemunho de João e seguido Jesus.
\par 41 Ele achou primeiro o seu próprio irmão, Simão, a quem disse: Achamos o Messias (que quer dizer Cristo),
\par 42 e o levou a Jesus. Olhando Jesus para ele, disse: Tu és Simão, o filho de João; tu serás chamado Cefas (que quer dizer Pedro).
\par 43 No dia imediato, resolveu Jesus partir para a Galiléia e encontrou a Filipe, a quem disse: Segue-me.
\par 44 Ora, Filipe era de Betsaida, cidade de André e de Pedro.
\par 45 Filipe encontrou a Natanael e disse-lhe: Achamos aquele de quem Moisés escreveu na lei, e a quem se referiram os profetas: Jesus, o Nazareno, filho de José.
\par 46 Perguntou-lhe Natanael: De Nazaré pode sair alguma coisa boa? Respondeu-lhe Filipe: Vem e vê.
\par 47 Jesus viu Natanael aproximar-se e disse a seu respeito: Eis um verdadeiro israelita, em quem não há dolo!
\par 48 Perguntou-lhe Natanael: Donde me conheces? Respondeu-lhe Jesus: Antes de Filipe te chamar, eu te vi, quando estavas debaixo da figueira.
\par 49 Então, exclamou Natanael: Mestre, tu és o Filho de Deus, tu és o Rei de Israel!
\par 50 Ao que Jesus lhe respondeu: Porque te disse que te vi debaixo da figueira, crês? Pois maiores coisas do que estas verás.
\par 51 E acrescentou: Em verdade, em verdade vos digo que vereis o céu aberto e os anjos de Deus subindo e descendo sobre o Filho do Homem.

\chapter{2}

\par 1 Três dias depois, houve um casamento em Caná da Galiléia, achando-se ali a mãe de Jesus.
\par 2 Jesus também foi convidado, com os seus discípulos, para o casamento.
\par 3 Tendo acabado o vinho, a mãe de Jesus lhe disse: Eles não têm mais vinho.
\par 4 Mas Jesus lhe disse: Mulher, que tenho eu contigo? Ainda não é chegada a minha hora.
\par 5 Então, ela falou aos serventes: Fazei tudo o que ele vos disser.
\par 6 Estavam ali seis talhas de pedra, que os judeus usavam para as purificações, e cada uma levava duas ou três metretas.
\par 7 Jesus lhes disse: Enchei de água as talhas. E eles as encheram totalmente.
\par 8 Então, lhes determinou: Tirai agora e levai ao mestre-sala. Eles o fizeram.
\par 9 Tendo o mestre-sala provado a água transformada em vinho (não sabendo donde viera, se bem que o sabiam os serventes que haviam tirado a água), chamou o noivo
\par 10 e lhe disse: Todos costumam pôr primeiro o bom vinho e, quando já beberam fartamente, servem o inferior; tu, porém, guardaste o bom vinho até agora.
\par 11 Com este, deu Jesus princípio a seus sinais em Caná da Galiléia; manifestou a sua glória, e os seus discípulos creram nele.
\par 12 Depois disto, desceu ele para Cafarnaum, com sua mãe, seus irmãos e seus discípulos; e ficaram ali não muitos dias.
\par 13 Estando próxima a Páscoa dos judeus, subiu Jesus para Jerusalém.
\par 14 E encontrou no templo os que vendiam bois, ovelhas e pombas e também os cambistas assentados;
\par 15 tendo feito um azorrague de cordas, expulsou todos do templo, bem como as ovelhas e os bois, derramou pelo chão o dinheiro dos cambistas, virou as mesas
\par 16 e disse aos que vendiam as pombas: Tirai daqui estas coisas; não façais da casa de meu Pai casa de negócio.
\par 17 Lembraram-se os seus discípulos de que está escrito: O zelo da tua casa me consumirá.
\par 18 Perguntaram-lhe, pois, os judeus: Que sinal nos mostras, para fazeres estas coisas?
\par 19 Jesus lhes respondeu: Destruí este santuário, e em três dias o reconstruirei.
\par 20 Replicaram os judeus: Em quarenta e seis anos foi edificado este santuário, e tu, em três dias, o levantarás?
\par 21 Ele, porém, se referia ao santuário do seu corpo.
\par 22 Quando, pois, Jesus ressuscitou dentre os mortos, lembraram-se os seus discípulos de que ele dissera isto; e creram na Escritura e na palavra de Jesus.
\par 23 Estando ele em Jerusalém, durante a Festa da Páscoa, muitos, vendo os sinais que ele fazia, creram no seu nome;
\par 24 mas o próprio Jesus não se confiava a eles, porque os conhecia a todos.
\par 25 E não precisava de que alguém lhe desse testemunho a respeito do homem, porque ele mesmo sabia o que era a natureza humana.

\chapter{3}

\par 1 Havia, entre os fariseus, um homem chamado Nicodemos, um dos principais dos judeus.
\par 2 Este, de noite, foi ter com Jesus e lhe disse: Rabi, sabemos que és Mestre vindo da parte de Deus; porque ninguém pode fazer estes sinais que tu fazes, se Deus não estiver com ele.
\par 3 A isto, respondeu Jesus: Em verdade, em verdade te digo que, se alguém não nascer de novo, não pode ver o reino de Deus.
\par 4 Perguntou-lhe Nicodemos: Como pode um homem nascer, sendo velho? Pode, porventura, voltar ao ventre materno e nascer segunda vez?
\par 5 Respondeu Jesus: Em verdade, em verdade te digo: quem não nascer da água e do Espírito não pode entrar no reino de Deus.
\par 6 O que é nascido da carne é carne; e o que é nascido do Espírito é espírito.
\par 7 Não te admires de eu te dizer: importa-vos nascer de novo.
\par 8 O vento sopra onde quer, ouves a sua voz, mas não sabes donde vem, nem para onde vai; assim é todo o que é nascido do Espírito.
\par 9 Então, lhe perguntou Nicodemos: Como pode suceder isto? Acudiu Jesus:
\par 10 Tu és mestre em Israel e não compreendes estas coisas?
\par 11 Em verdade, em verdade te digo que nós dizemos o que sabemos e testificamos o que temos visto; contudo, não aceitais o nosso testemunho.
\par 12 Se, tratando de coisas terrenas, não me credes, como crereis, se vos falar das celestiais?
\par 13 Ora, ninguém subiu ao céu, senão aquele que de lá desceu, a saber, o Filho do Homem [que está no céu].
\par 14 E do modo por que Moisés levantou a serpente no deserto, assim importa que o Filho do Homem seja levantado,
\par 15 para que todo o que nele crê tenha a vida eterna.
\par 16 Porque Deus amou ao mundo de tal maneira que deu o seu Filho unigênito, para que todo o que nele crê não pereça, mas tenha a vida eterna.
\par 17 Porquanto Deus enviou o seu Filho ao mundo, não para que julgasse o mundo, mas para que o mundo fosse salvo por ele.
\par 18 Quem nele crê não é julgado; o que não crê já está julgado, porquanto não crê no nome do unigênito Filho de Deus.
\par 19 O julgamento é este: que a luz veio ao mundo, e os homens amaram mais as trevas do que a luz; porque as suas obras eram más.
\par 20 Pois todo aquele que pratica o mal aborrece a luz e não se chega para a luz, a fim de não serem argüidas as suas obras.
\par 21 Quem pratica a verdade aproxima-se da luz, a fim de que as suas obras sejam manifestas, porque feitas em Deus.
\par 22 Depois disto, foi Jesus com seus discípulos para a terra da Judéia; ali permaneceu com eles e batizava.
\par 23 Ora, João estava também batizando em Enom, perto de Salim, porque havia ali muitas águas, e para lá concorria o povo e era batizado.
\par 24 Pois João ainda não tinha sido encarcerado.
\par 25 Ora, entre os discípulos de João e um judeu suscitou-se uma contenda com respeito à purificação.
\par 26 E foram ter com João e lhe disseram: Mestre, aquele que estava contigo além do Jordão, do qual tens dado testemunho, está batizando, e todos lhe saem ao encontro.
\par 27 Respondeu João: O homem não pode receber coisa alguma se do céu não lhe for dada.
\par 28 Vós mesmos sois testemunhas de que vos disse: eu não sou o Cristo, mas fui enviado como seu precursor.
\par 29 O que tem a noiva é o noivo; o amigo do noivo que está presente e o ouve muito se regozija por causa da voz do noivo. Pois esta alegria já se cumpriu em mim.
\par 30 Convém que ele cresça e que eu diminua.
\par 31 Quem vem das alturas certamente está acima de todos; quem vem da terra é terreno e fala da terra; quem veio do céu está acima de todos
\par 32 e testifica o que tem visto e ouvido; contudo, ninguém aceita o seu testemunho.
\par 33 Quem, todavia, lhe aceita o testemunho, por sua vez, certifica que Deus é verdadeiro.
\par 34 Pois o enviado de Deus fala as palavras dele, porque Deus não dá o Espírito por medida.
\par 35 O Pai ama ao Filho, e todas as coisas tem confiado às suas mãos.
\par 36 Por isso, quem crê no Filho tem a vida eterna; o que, todavia, se mantém rebelde contra o Filho não verá a vida, mas sobre ele permanece a ira de Deus.

\chapter{4}

\par 1 Quando, pois, o Senhor veio a saber que os fariseus tinham ouvido dizer que ele, Jesus, fazia e batizava mais discípulos que João
\par 2 (se bem que Jesus mesmo não batizava, e sim os seus discípulos),
\par 3 deixou a Judéia, retirando-se outra vez para a Galiléia.
\par 4 E era-lhe necessário atravessar a província de Samaria.
\par 5 Chegou, pois, a uma cidade samaritana, chamada Sicar, perto das terras que Jacó dera a seu filho José.
\par 6 Estava ali a fonte de Jacó. Cansado da viagem, assentara-se Jesus junto à fonte, por volta da hora sexta.
\par 7 Nisto, veio uma mulher samaritana tirar água. Disse-lhe Jesus: Dá-me de beber.
\par 8 Pois seus discípulos tinham ido à cidade para comprar alimentos.
\par 9 Então, lhe disse a mulher samaritana: Como, sendo tu judeu, pedes de beber a mim, que sou mulher samaritana (porque os judeus não se dão com os samaritanos)?
\par 10 Replicou-lhe Jesus: Se conheceras o dom de Deus e quem é o que te pede: dá-me de beber, tu lhe pedirias, e ele te daria água viva.
\par 11 Respondeu-lhe ela: Senhor, tu não tens com que a tirar, e o poço é fundo; onde, pois, tens a água viva?
\par 12 És tu, porventura, maior do que Jacó, o nosso pai, que nos deu o poço, do qual ele mesmo bebeu, e, bem assim, seus filhos, e seu gado?
\par 13 Afirmou-lhe Jesus: Quem beber desta água tornará a ter sede;
\par 14 aquele, porém, que beber da água que eu lhe der nunca mais terá sede; pelo contrário, a água que eu lhe der será nele uma fonte a jorrar para a vida eterna.
\par 15 Disse-lhe a mulher: Senhor, dá-me dessa água para que eu não mais tenha sede, nem precise vir aqui buscá-la.
\par 16 Disse-lhe Jesus: Vai, chama teu marido e vem cá;
\par 17 ao que lhe respondeu a mulher: Não tenho marido. Replicou-lhe Jesus: Bem disseste, não tenho marido;
\par 18 porque cinco maridos já tiveste, e esse que agora tens não é teu marido; isto disseste com verdade.
\par 19 Senhor, disse-lhe a mulher, vejo que tu és profeta.
\par 20 Nossos pais adoravam neste monte; vós, entretanto, dizeis que em Jerusalém é o lugar onde se deve adorar.
\par 21 Disse-lhe Jesus: Mulher, podes crer-me que a hora vem, quando nem neste monte, nem em Jerusalém adorareis o Pai.
\par 22 Vós adorais o que não conheceis; nós adoramos o que conhecemos, porque a salvação vem dos judeus.
\par 23 Mas vem a hora e já chegou, em que os verdadeiros adoradores adorarão o Pai em espírito e em verdade; porque são estes que o Pai procura para seus adoradores.
\par 24 Deus é espírito; e importa que os seus adoradores o adorem em espírito e em verdade.
\par 25 Eu sei, respondeu a mulher, que há de vir o Messias, chamado Cristo; quando ele vier, nos anunciará todas as coisas.
\par 26 Disse-lhe Jesus: Eu o sou, eu que falo contigo.
\par 27 Neste ponto, chegaram os seus discípulos e se admiraram de que estivesse falando com uma mulher; todavia, nenhum lhe disse: Que perguntas? Ou: Por que falas com ela?
\par 28 Quanto à mulher, deixou o seu cântaro, foi à cidade e disse àqueles homens:
\par 29 Vinde comigo e vede um homem que me disse tudo quanto tenho feito. Será este, porventura, o Cristo?!
\par 30 Saíram, pois, da cidade e vieram ter com ele.
\par 31 Nesse ínterim, os discípulos lhe rogavam, dizendo: Mestre, come!
\par 32 Mas ele lhes disse: Uma comida tenho para comer, que vós não conheceis.
\par 33 Diziam, então, os discípulos uns aos outros: Ter-lhe-ia, porventura, alguém trazido o que comer?
\par 34 Disse-lhes Jesus: A minha comida consiste em fazer a vontade daquele que me enviou e realizar a sua obra.
\par 35 Não dizeis vós que ainda há quatro meses até à ceifa? Eu, porém, vos digo: erguei os olhos e vede os campos, pois já branquejam para a ceifa.
\par 36 O ceifeiro recebe desde já a recompensa e entesoura o seu fruto para a vida eterna; e, dessarte, se alegram tanto o semeador como o ceifeiro.
\par 37 Pois, no caso, é verdadeiro o ditado: Um é o semeador, e outro é o ceifeiro.
\par 38 Eu vos enviei para ceifar o que não semeastes; outros trabalharam, e vós entrastes no seu trabalho.
\par 39 Muitos samaritanos daquela cidade creram nele, em virtude do testemunho da mulher, que anunciara: Ele me disse tudo quanto tenho feito.
\par 40 Vindo, pois, os samaritanos ter com Jesus, pediam-lhe que permanecesse com eles; e ficou ali dois dias.
\par 41 Muitos outros creram nele, por causa da sua palavra,
\par 42 e diziam à mulher: Já agora não é pelo que disseste que nós cremos; mas porque nós mesmos temos ouvido e sabemos que este é verdadeiramente o Salvador do mundo.
\par 43 Passados dois dias, partiu dali para a Galiléia.
\par 44 Porque o mesmo Jesus testemunhou que um profeta não tem honras na sua própria terra.
\par 45 Assim, quando chegou à Galiléia, os galileus o receberam, porque viram todas as coisas que ele fizera em Jerusalém, por ocasião da festa, à qual eles também tinham comparecido.
\par 46 Dirigiu-se, de novo, a Caná da Galiléia, onde da água fizera vinho. Ora, havia um oficial do rei, cujo filho estava doente em Cafarnaum.
\par 47 Tendo ouvido dizer que Jesus viera da Judéia para a Galiléia, foi ter com ele e lhe rogou que descesse para curar seu filho, que estava à morte.
\par 48 Então, Jesus lhe disse: Se, porventura, não virdes sinais e prodígios, de modo nenhum crereis.
\par 49 Rogou-lhe o oficial: Senhor, desce, antes que meu filho morra.
\par 50 Vai, disse-lhe Jesus; teu filho vive. O homem creu na palavra de Jesus e partiu.
\par 51 Já ele descia, quando os seus servos lhe vieram ao encontro, anunciando-lhe que o seu filho vivia.
\par 52 Então, indagou deles a que hora o seu filho se sentira melhor. Informaram: Ontem, à hora sétima a febre o deixou.
\par 53 Com isto, reconheceu o pai ser aquela precisamente a hora em que Jesus lhe dissera: Teu filho vive; e creu ele e toda a sua casa.
\par 54 Foi este o segundo sinal que fez Jesus, depois de vir da Judéia para a Galiléia.

\chapter{5}

\par 1 Passadas estas coisas, havia uma festa dos judeus, e Jesus subiu para Jerusalém.
\par 2 Ora, existe ali, junto à Porta das Ovelhas, um tanque, chamado em hebraico Betesda, o qual tem cinco pavilhões.
\par 3 Nestes, jazia uma multidão de enfermos, cegos, coxos, paralíticos
\par 4 [esperando que se movesse a água. Porquanto um anjo descia em certo tempo, agitando-a; e o primeiro que entrava no tanque, uma vez agitada a água, sarava de qualquer doença que tivesse].
\par 5 Estava ali um homem enfermo havia trinta e oito anos.
\par 6 Jesus, vendo-o deitado e sabendo que estava assim há muito tempo, perguntou-lhe: Queres ser curado?
\par 7 Respondeu-lhe o enfermo: Senhor, não tenho ninguém que me ponha no tanque, quando a água é agitada; pois, enquanto eu vou, desce outro antes de mim.
\par 8 Então, lhe disse Jesus: Levanta-te, toma o teu leito e anda.
\par 9 Imediatamente, o homem se viu curado e, tomando o leito, pôs-se a andar. E aquele dia era sábado.
\par 10 Por isso, disseram os judeus ao que fora curado: Hoje é sábado, e não te é lícito carregar o leito.
\par 11 Ao que ele lhes respondeu: O mesmo que me curou me disse: Toma o teu leito e anda.
\par 12 Perguntaram-lhe eles: Quem é o homem que te disse: Toma o teu leito e anda?
\par 13 Mas o que fora curado não sabia quem era; porque Jesus se havia retirado, por haver muita gente naquele lugar.
\par 14 Mais tarde, Jesus o encontrou no templo e lhe disse: Olha que já estás curado; não peques mais, para que não te suceda coisa pior.
\par 15 O homem retirou-se e disse aos judeus que fora Jesus quem o havia curado.
\par 16 E os judeus perseguiam Jesus, porque fazia estas coisas no sábado.
\par 17 Mas ele lhes disse: Meu Pai trabalha até agora, e eu trabalho também.
\par 18 Por isso, pois, os judeus ainda mais procuravam matá-lo, porque não somente violava o sábado, mas também dizia que Deus era seu próprio Pai, fazendo-se igual a Deus.
\par 19 Então, lhes falou Jesus: Em verdade, em verdade vos digo que o Filho nada pode fazer de si mesmo, senão somente aquilo que vir fazer o Pai; porque tudo o que este fizer, o Filho também semelhantemente o faz.
\par 20 Porque o Pai ama ao Filho, e lhe mostra tudo o que faz, e maiores obras do que estas lhe mostrará, para que vos maravilheis.
\par 21 Pois assim como o Pai ressuscita e vivifica os mortos, assim também o Filho vivifica aqueles a quem quer.
\par 22 E o Pai a ninguém julga, mas ao Filho confiou todo julgamento,
\par 23 a fim de que todos honrem o Filho do modo por que honram o Pai. Quem não honra o Filho não honra o Pai que o enviou.
\par 24 Em verdade, em verdade vos digo: quem ouve a minha palavra e crê naquele que me enviou tem a vida eterna, não entra em juízo, mas passou da morte para a vida.
\par 25 Em verdade, em verdade vos digo que vem a hora e já chegou, em que os mortos ouvirão a voz do Filho de Deus; e os que a ouvirem viverão.
\par 26 Porque assim como o Pai tem vida em si mesmo, também concedeu ao Filho ter vida em si mesmo.
\par 27 E lhe deu autoridade para julgar, porque é o Filho do Homem.
\par 28 Não vos maravilheis disto, porque vem a hora em que todos os que se acham nos túmulos ouvirão a sua voz e sairão:
\par 29 os que tiverem feito o bem, para a ressurreição da vida; e os que tiverem praticado o mal, para a ressurreição do juízo.
\par 30 Eu nada posso fazer de mim mesmo; na forma por que ouço, julgo. O meu juízo é justo, porque não procuro a minha própria vontade, e sim a daquele que me enviou.
\par 31 Se eu testifico a respeito de mim mesmo, o meu testemunho não é verdadeiro.
\par 32 Outro é o que testifica a meu respeito, e sei que é verdadeiro o testemunho que ele dá de mim.
\par 33 Mandastes mensageiros a João, e ele deu testemunho da verdade.
\par 34 Eu, porém, não aceito humano testemunho; digo-vos, entretanto, estas coisas para que sejais salvos.
\par 35 Ele era a lâmpada que ardia e alumiava, e vós quisestes, por algum tempo, alegrar-vos com a sua luz.
\par 36 Mas eu tenho maior testemunho do que o de João; porque as obras que o Pai me confiou para que eu as realizasse, essas que eu faço testemunham a meu respeito de que o Pai me enviou.
\par 37 O Pai, que me enviou, esse mesmo é que tem dado testemunho de mim. Jamais tendes ouvido a sua voz, nem visto a sua forma.
\par 38 Também não tendes a sua palavra permanente em vós, porque não credes naquele a quem ele enviou.
\par 39 Examinais as Escrituras, porque julgais ter nelas a vida eterna, e são elas mesmas que testificam de mim.
\par 40 Contudo, não quereis vir a mim para terdes vida.
\par 41 Eu não aceito glória que vem dos homens;
\par 42 sei, entretanto, que não tendes em vós o amor de Deus.
\par 43 Eu vim em nome de meu Pai, e não me recebeis; se outro vier em seu próprio nome, certamente, o recebereis.
\par 44 Como podeis crer, vós os que aceitais glória uns dos outros e, contudo, não procurais a glória que vem do Deus único?
\par 45 Não penseis que eu vos acusarei perante o Pai; quem vos acusa é Moisés, em quem tendes firmado a vossa confiança.
\par 46 Porque, se, de fato, crêsseis em Moisés, também creríeis em mim; porquanto ele escreveu a meu respeito.
\par 47 Se, porém, não credes nos seus escritos, como crereis nas minhas palavras?

\chapter{6}

\par 1 Depois destas coisas, atravessou Jesus o mar da Galiléia, que é o de Tiberíades.
\par 2 Seguia-o numerosa multidão, porque tinham visto os sinais que ele fazia na cura dos enfermos.
\par 3 Então, subiu Jesus ao monte e assentou-se ali com os seus discípulos.
\par 4 Ora, a Páscoa, festa dos judeus, estava próxima.
\par 5 Então, Jesus, erguendo os olhos e vendo que grande multidão vinha ter com ele, disse a Filipe: Onde compraremos pães para lhes dar a comer?
\par 6 Mas dizia isto para o experimentar; porque ele bem sabia o que estava para fazer.
\par 7 Respondeu-lhe Filipe: Não lhes bastariam duzentos denários de pão, para receber cada um o seu pedaço.
\par 8 Um de seus discípulos, chamado André, irmão de Simão Pedro, informou a Jesus:
\par 9 Está aí um rapaz que tem cinco pães de cevada e dois peixinhos; mas isto que é para tanta gente?
\par 10 Disse Jesus: Fazei o povo assentar-se; pois havia naquele lugar muita relva. Assentaram-se, pois, os homens em número de quase cinco mil.
\par 11 Então, Jesus tomou os pães e, tendo dado graças, distribuiu-os entre eles; e também igualmente os peixes, quanto queriam.
\par 12 E, quando já estavam fartos, disse Jesus aos seus discípulos: Recolhei os pedaços que sobraram, para que nada se perca.
\par 13 Assim, pois, o fizeram e encheram doze cestos de pedaços dos cinco pães de cevada, que sobraram aos que haviam comido.
\par 14 Vendo, pois, os homens o sinal que Jesus fizera, disseram: Este é, verdadeiramente, o profeta que devia vir ao mundo.
\par 15 Sabendo, pois, Jesus que estavam para vir com o intuito de arrebatá-lo para o proclamarem rei, retirou-se novamente, sozinho, para o monte.
\par 16 Ao descambar o dia, os seus discípulos desceram para o mar.
\par 17 E, tomando um barco, passaram para o outro lado, rumo a Cafarnaum. Já se fazia escuro, e Jesus ainda não viera ter com eles.
\par 18 E o mar começava a empolar-se, agitado por vento rijo que soprava.
\par 19 Tendo navegado uns vinte e cinco a trinta estádios, eis que viram Jesus andando por sobre o mar, aproximando-se do barco; e ficaram possuídos de temor.
\par 20 Mas Jesus lhes disse: Sou eu. Não temais!
\par 21 Então, eles, de bom grado, o receberam, e logo o barco chegou ao seu destino.
\par 22 No dia seguinte, a multidão que ficara do outro lado do mar notou que ali não havia senão um pequeno barco e que Jesus não embarcara nele com seus discípulos, tendo estes partido sós.
\par 23 Entretanto, outros barquinhos chegaram de Tiberíades, perto do lugar onde comeram o pão, tendo o Senhor dado graças.
\par 24 Quando, pois, viu a multidão que Jesus não estava ali nem os seus discípulos, tomaram os barcos e partiram para Cafarnaum à sua procura.
\par 25 E, tendo-o encontrado no outro lado do mar, lhe perguntaram: Mestre, quando chegaste aqui?
\par 26 Respondeu-lhes Jesus: Em verdade, em verdade vos digo: vós me procurais, não porque vistes sinais, mas porque comestes dos pães e vos fartastes.
\par 27 Trabalhai, não pela comida que perece, mas pela que subsiste para a vida eterna, a qual o Filho do Homem vos dará; porque Deus, o Pai, o confirmou com o seu selo.
\par 28 Dirigiram-se, pois, a ele, perguntando: Que faremos para realizar as obras de Deus?
\par 29 Respondeu-lhes Jesus: A obra de Deus é esta: que creiais naquele que por ele foi enviado.
\par 30 Então, lhe disseram eles: Que sinal fazes para que o vejamos e creiamos em ti? Quais são os teus feitos?
\par 31 Nossos pais comeram o maná no deserto, como está escrito: Deu-lhes a comer pão do céu.
\par 32 Replicou-lhes Jesus: Em verdade, em verdade vos digo: não foi Moisés quem vos deu o pão do céu; o verdadeiro pão do céu é meu Pai quem vos dá.
\par 33 Porque o pão de Deus é o que desce do céu e dá vida ao mundo.
\par 34 Então, lhe disseram: Senhor, dá-nos sempre desse pão.
\par 35 Declarou-lhes, pois, Jesus: Eu sou o pão da vida; o que vem a mim jamais terá fome; e o que crê em mim jamais terá sede.
\par 36 Porém eu já vos disse que, embora me tenhais visto, não credes.
\par 37 Todo aquele que o Pai me dá, esse virá a mim; e o que vem a mim, de modo nenhum o lançarei fora.
\par 38 Porque eu desci do céu, não para fazer a minha própria vontade, e sim a vontade daquele que me enviou.
\par 39 E a vontade de quem me enviou é esta: que nenhum eu perca de todos os que me deu; pelo contrário, eu o ressuscitarei no último dia.
\par 40 De fato, a vontade de meu Pai é que todo homem que vir o Filho e nele crer tenha a vida eterna; e eu o ressuscitarei no último dia.
\par 41 Murmuravam, pois, dele os judeus, porque dissera: Eu sou o pão que desceu do céu.
\par 42 E diziam: Não é este Jesus, o filho de José? Acaso, não lhe conhecemos o pai e a mãe? Como, pois, agora diz: Desci do céu?
\par 43 Respondeu-lhes Jesus: Não murmureis entre vós.
\par 44 Ninguém pode vir a mim se o Pai, que me enviou, não o trouxer; e eu o ressuscitarei no último dia.
\par 45 Está escrito nos profetas: E serão todos ensinados por Deus. Portanto, todo aquele que da parte do Pai tem ouvido e aprendido, esse vem a mim.
\par 46 Não que alguém tenha visto o Pai, salvo aquele que vem de Deus; este o tem visto.
\par 47 Em verdade, em verdade vos digo: quem crê em mim tem a vida eterna.
\par 48 Eu sou o pão da vida.
\par 49 Vossos pais comeram o maná no deserto e morreram.
\par 50 Este é o pão que desce do céu, para que todo o que dele comer não pereça.
\par 51 Eu sou o pão vivo que desceu do céu; se alguém dele comer, viverá eternamente; e o pão que eu darei pela vida do mundo é a minha carne.
\par 52 Disputavam, pois, os judeus entre si, dizendo: Como pode este dar-nos a comer a sua própria carne?
\par 53 Respondeu-lhes Jesus: Em verdade, em verdade vos digo: se não comerdes a carne do Filho do Homem e não beberdes o seu sangue, não tendes vida em vós mesmos.
\par 54 Quem comer a minha carne e beber o meu sangue tem a vida eterna, e eu o ressuscitarei no último dia.
\par 55 Pois a minha carne é verdadeira comida, e o meu sangue é verdadeira bebida.
\par 56 Quem comer a minha carne e beber o meu sangue permanece em mim, e eu, nele.
\par 57 Assim como o Pai, que vive, me enviou, e igualmente eu vivo pelo Pai, também quem de mim se alimenta por mim viverá.
\par 58 Este é o pão que desceu do céu, em nada semelhante àquele que os vossos pais comeram e, contudo, morreram; quem comer este pão viverá eternamente.
\par 59 Estas coisas disse Jesus, quando ensinava na sinagoga de Cafarnaum.
\par 60 Muitos dos seus discípulos, tendo ouvido tais palavras, disseram: Duro é este discurso; quem o pode ouvir?
\par 61 Mas Jesus, sabendo por si mesmo que eles murmuravam a respeito de suas palavras, interpelou-os: Isto vos escandaliza?
\par 62 Que será, pois, se virdes o Filho do Homem subir para o lugar onde primeiro estava?
\par 63 O espírito é o que vivifica; a carne para nada aproveita; as palavras que eu vos tenho dito são espírito e são vida.
\par 64 Contudo, há descrentes entre vós. Pois Jesus sabia, desde o princípio, quais eram os que não criam e quem o havia de trair.
\par 65 E prosseguiu: Por causa disto, é que vos tenho dito: ninguém poderá vir a mim, se, pelo Pai, não lhe for concedido.
\par 66 À vista disso, muitos dos seus discípulos o abandonaram e já não andavam com ele.
\par 67 Então, perguntou Jesus aos doze: Porventura, quereis também vós outros retirar-vos?
\par 68 Respondeu-lhe Simão Pedro: Senhor, para quem iremos? Tu tens as palavras da vida eterna;
\par 69 e nós temos crido e conhecido que tu és o Santo de Deus.
\par 70 Replicou-lhes Jesus: Não vos escolhi eu em número de doze? Contudo, um de vós é diabo.
\par 71 Referia-se ele a Judas, filho de Simão Iscariotes; porque era quem estava para traí-lo, sendo um dos doze.

\chapter{7}

\par 1 Passadas estas coisas, Jesus andava pela Galiléia, porque não desejava percorrer a Judéia, visto que os judeus procuravam matá-lo.
\par 2 Ora, a festa dos judeus, chamada de Festa dos Tabernáculos, estava próxima.
\par 3 Dirigiram-se, pois, a ele os seus irmãos e lhe disseram: Deixa este lugar e vai para a Judéia, para que também os teus discípulos vejam as obras que fazes.
\par 4 Porque ninguém há que procure ser conhecido em público e, contudo, realize os seus feitos em oculto. Se fazes estas coisas, manifesta-te ao mundo.
\par 5 Pois nem mesmo os seus irmãos criam nele.
\par 6 Disse-lhes, pois, Jesus: O meu tempo ainda não chegou, mas o vosso sempre está presente.
\par 7 Não pode o mundo odiar-vos, mas a mim me odeia, porque eu dou testemunho a seu respeito de que as suas obras são más.
\par 8 Subi vós outros à festa; eu, por enquanto, não subo, porque o meu tempo ainda não está cumprido.
\par 9 Disse-lhes Jesus estas coisas e continuou na Galiléia.
\par 10 Mas, depois que seus irmãos subiram para a festa, então, subiu ele também, não publicamente, mas em oculto.
\par 11 Ora, os judeus o procuravam na festa e perguntavam: Onde estará ele?
\par 12 E havia grande murmuração a seu respeito entre as multidões. Uns diziam: Ele é bom. E outros: Não, antes, engana o povo.
\par 13 Entretanto, ninguém falava dele abertamente, por ter medo dos judeus.
\par 14 Corria já em meio a festa, e Jesus subiu ao templo e ensinava.
\par 15 Então, os judeus se maravilhavam e diziam: Como sabe este letras, sem ter estudado?
\par 16 Respondeu-lhes Jesus: O meu ensino não é meu, e sim daquele que me enviou.
\par 17 Se alguém quiser fazer a vontade dele, conhecerá a respeito da doutrina, se ela é de Deus ou se eu falo por mim mesmo.
\par 18 Quem fala por si mesmo está procurando a sua própria glória; mas o que procura a glória de quem o enviou, esse é verdadeiro, e nele não há injustiça.
\par 19 Não vos deu Moisés a lei? Contudo, ninguém dentre vós a observa. Por que procurais matar-me?
\par 20 Respondeu a multidão: Tens demônio. Quem é que procura matar-te?
\par 21 Replicou-lhes Jesus: Um só feito realizei, e todos vos admirais.
\par 22 Pelo motivo de que Moisés vos deu a circuncisão (se bem que ela não vem dele, mas dos patriarcas), no sábado circuncidais um homem.
\par 23 E, se o homem pode ser circuncidado em dia de sábado, para que a lei de Moisés não seja violada, por que vos indignais contra mim, pelo fato de eu ter curado, num sábado, ao todo, um homem?
\par 24 Não julgueis segundo a aparência, e sim pela reta justiça.
\par 25 Diziam alguns de Jerusalém: Não é este aquele a quem procuram matar?
\par 26 Eis que ele fala abertamente, e nada lhe dizem. Porventura, reconhecem verdadeiramente as autoridades que este é, de fato, o Cristo?
\par 27 Nós, todavia, sabemos donde este é; quando, porém, vier o Cristo, ninguém saberá donde ele é.
\par 28 Jesus, pois, enquanto ensinava no templo, clamou, dizendo: Vós não somente me conheceis, mas também sabeis donde eu sou; e não vim porque eu, de mim mesmo, o quisesse, mas aquele que me enviou é verdadeiro, aquele a quem vós não conheceis.
\par 29 Eu o conheço, porque venho da parte dele e fui por ele enviado.
\par 30 Então, procuravam prendê-lo; mas ninguém lhe pôs a mão, porque ainda não era chegada a sua hora.
\par 31 E, contudo, muitos de entre a multidão creram nele e diziam: Quando vier o Cristo, fará, porventura, maiores sinais do que este homem tem feito?
\par 32 Os fariseus, ouvindo a multidão murmurar estas coisas a respeito dele, juntamente com os principais sacerdotes enviaram guardas para o prenderem.
\par 33 Disse-lhes Jesus: Ainda por um pouco de tempo estou convosco e depois irei para junto daquele que me enviou.
\par 34 Haveis de procurar-me e não me achareis; também aonde eu estou, vós não podeis ir.
\par 35 Disseram, pois, os judeus uns aos outros: Para onde irá este que não o possamos achar? Irá, porventura, para a Dispersão entre os gregos, com o fim de os ensinar?
\par 36 Que significa, de fato, o que ele diz: Haveis de procurar-me e não me achareis; também aonde eu estou, vós não podeis ir?
\par 37 No último dia, o grande dia da festa, levantou-se Jesus e exclamou: Se alguém tem sede, venha a mim e beba.
\par 38 Quem crer em mim, como diz a Escritura, do seu interior fluirão rios de água viva.
\par 39 Isto ele disse com respeito ao Espírito que haviam de receber os que nele cressem; pois o Espírito até aquele momento não fora dado, porque Jesus não havia sido ainda glorificado.
\par 40 Então, os que dentre o povo tinham ouvido estas palavras diziam: Este é verdadeiramente o profeta;
\par 41 outros diziam: Ele é o Cristo; outros, porém, perguntavam: Porventura, o Cristo virá da Galiléia?
\par 42 Não diz a Escritura que o Cristo vem da descendência de Davi e da aldeia de Belém, donde era Davi?
\par 43 Assim, houve uma dissensão entre o povo por causa dele;
\par 44 alguns dentre eles queriam prendê-lo, mas ninguém lhe pôs as mãos.
\par 45 Voltaram, pois, os guardas à presença dos principais sacerdotes e fariseus, e estes lhes perguntaram: Por que não o trouxestes?
\par 46 Responderam eles: Jamais alguém falou como este homem.
\par 47 Replicaram-lhes, pois, os fariseus: Será que também vós fostes enganados?
\par 48 Porventura, creu nele alguém dentre as autoridades ou algum dos fariseus?
\par 49 Quanto a esta plebe que nada sabe da lei, é maldita.
\par 50 Nicodemos, um deles, que antes fora ter com Jesus, perguntou-lhes:
\par 51 Acaso, a nossa lei julga um homem, sem primeiro ouvi-lo e saber o que ele fez?
\par 52 Responderam eles: Dar-se-á o caso de que também tu és da Galiléia? Examina e verás que da Galiléia não se levanta profeta.
\par 53 [E cada um foi para sua casa.

\chapter{8}

\par 1 Jesus, entretanto, foi para o monte das Oliveiras.
\par 2 De madrugada, voltou novamente para o templo, e todo o povo ia ter com ele; e, assentado, os ensinava.
\par 3 Os escribas e fariseus trouxeram à sua presença uma mulher surpreendida em adultério e, fazendo-a ficar de pé no meio de todos,
\par 4 disseram a Jesus: Mestre, esta mulher foi apanhada em flagrante adultério.
\par 5 E na lei nos mandou Moisés que tais mulheres sejam apedrejadas; tu, pois, que dizes?
\par 6 Isto diziam eles tentando-o, para terem de que o acusar. Mas Jesus, inclinando-se, escrevia na terra com o dedo.
\par 7 Como insistissem na pergunta, Jesus se levantou e lhes disse: Aquele que dentre vós estiver sem pecado seja o primeiro que lhe atire pedra.
\par 8 E, tornando a inclinar-se, continuou a escrever no chão.
\par 9 Mas, ouvindo eles esta resposta e acusados pela própria consciência, foram-se retirando um por um, a começar pelos mais velhos até aos últimos, ficando só Jesus e a mulher no meio onde estava.
\par 10 Erguendo-se Jesus e não vendo a ninguém mais além da mulher, perguntou-lhe: Mulher, onde estão aqueles teus acusadores? Ninguém te condenou?
\par 11 Respondeu ela: Ninguém, Senhor! Então, lhe disse Jesus: Nem eu tampouco te condeno; vai e não peques mais.]
\par 12 De novo, lhes falava Jesus, dizendo: Eu sou a luz do mundo; quem me segue não andará nas trevas; pelo contrário, terá a luz da vida.
\par 13 Então, lhe objetaram os fariseus: Tu dás testemunho de ti mesmo; logo, o teu testemunho não é verdadeiro.
\par 14 Respondeu Jesus e disse-lhes: Posto que eu testifico de mim mesmo, o meu testemunho é verdadeiro, porque sei donde vim e para onde vou; mas vós não sabeis donde venho, nem para onde vou.
\par 15 Vós julgais segundo a carne, eu a ninguém julgo.
\par 16 Se eu julgo, o meu juízo é verdadeiro, porque não sou eu só, porém eu e aquele que me enviou.
\par 17 Também na vossa lei está escrito que o testemunho de duas pessoas é verdadeiro.
\par 18 Eu testifico de mim mesmo, e o Pai, que me enviou, também testifica de mim.
\par 19 Então, eles lhe perguntaram: Onde está teu Pai? Respondeu Jesus: Não me conheceis a mim nem a meu Pai; se conhecêsseis a mim, também conheceríeis a meu Pai.
\par 20 Proferiu ele estas palavras no lugar do gazofilácio, quando ensinava no templo; e ninguém o prendeu, porque não era ainda chegada a sua hora.
\par 21 De outra feita, lhes falou, dizendo: Vou retirar-me, e vós me procurareis, mas perecereis no vosso pecado; para onde eu vou vós não podeis ir.
\par 22 Então, diziam os judeus: Terá ele, acaso, a intenção de suicidar-se? Porque diz: Para onde eu vou vós não podeis ir.
\par 23 E prosseguiu: Vós sois cá de baixo, eu sou lá de cima; vós sois deste mundo, eu deste mundo não sou.
\par 24 Por isso, eu vos disse que morrereis nos vossos pecados; porque, se não crerdes que EU SOU, morrereis nos vossos pecados.
\par 25 Então, lhe perguntaram: Quem és tu? Respondeu-lhes Jesus: Que é que desde o princípio vos tenho dito?
\par 26 Muitas coisas tenho para dizer a vosso respeito e vos julgar; porém aquele que me enviou é verdadeiro, de modo que as coisas que dele tenho ouvido, essas digo ao mundo.
\par 27 Eles, porém, não atinaram que lhes falava do Pai.
\par 28 Disse-lhes, pois, Jesus: Quando levantardes o Filho do Homem, então, sabereis que EU SOU e que nada faço por mim mesmo; mas falo como o Pai me ensinou.
\par 29 E aquele que me enviou está comigo, não me deixou só, porque eu faço sempre o que lhe agrada.
\par 30 Ditas estas coisas, muitos creram nele.
\par 31 Disse, pois, Jesus aos judeus que haviam crido nele: Se vós permanecerdes na minha palavra, sois verdadeiramente meus discípulos;
\par 32 e conhecereis a verdade, e a verdade vos libertará.
\par 33 Responderam-lhe: Somos descendência de Abraão e jamais fomos escravos de alguém; como dizes tu: Sereis livres?
\par 34 Replicou-lhes Jesus: Em verdade, em verdade vos digo: todo o que comete pecado é escravo do pecado.
\par 35 O escravo não fica sempre na casa; o filho, sim, para sempre.
\par 36 Se, pois, o Filho vos libertar, verdadeiramente sereis livres.
\par 37 Bem sei que sois descendência de Abraão; contudo, procurais matar-me, porque a minha palavra não está em vós.
\par 38 Eu falo das coisas que vi junto de meu Pai; vós, porém, fazeis o que vistes em vosso pai.
\par 39 Então, lhe responderam: Nosso pai é Abraão. Disse-lhes Jesus: Se sois filhos de Abraão, praticai as obras de Abraão.
\par 40 Mas agora procurais matar-me, a mim que vos tenho falado a verdade que ouvi de Deus; assim não procedeu Abraão.
\par 41 Vós fazeis as obras de vosso pai. Disseram-lhe eles: Nós não somos bastardos; temos um pai, que é Deus.
\par 42 Replicou-lhes Jesus: Se Deus fosse, de fato, vosso pai, certamente, me havíeis de amar; porque eu vim de Deus e aqui estou; pois não vim de mim mesmo, mas ele me enviou.
\par 43 Qual a razão por que não compreendeis a minha linguagem? É porque sois incapazes de ouvir a minha palavra.
\par 44 Vós sois do diabo, que é vosso pai, e quereis satisfazer-lhe os desejos. Ele foi homicida desde o princípio e jamais se firmou na verdade, porque nele não há verdade. Quando ele profere mentira, fala do que lhe é próprio, porque é mentiroso e pai da mentira.
\par 45 Mas, porque eu digo a verdade, não me credes.
\par 46 Quem dentre vós me convence de pecado? Se vos digo a verdade, por que razão não me credes?
\par 47 Quem é de Deus ouve as palavras de Deus; por isso, não me dais ouvidos, porque não sois de Deus.
\par 48 Responderam, pois, os judeus e lhe disseram: Porventura, não temos razão em dizer que és samaritano e tens demônio?
\par 49 Replicou Jesus: Eu não tenho demônio; pelo contrário, honro a meu Pai, e vós me desonrais.
\par 50 Eu não procuro a minha própria glória; há quem a busque e julgue.
\par 51 Em verdade, em verdade vos digo: se alguém guardar a minha palavra, não verá a morte, eternamente.
\par 52 Disseram-lhe os judeus: Agora, estamos certos de que tens demônio. Abraão morreu, e também os profetas, e tu dizes: Se alguém guardar a minha palavra, não provará a morte, eternamente.
\par 53 És maior do que Abraão, o nosso pai, que morreu? Também os profetas morreram. Quem, pois, te fazes ser?
\par 54 Respondeu Jesus: Se eu me glorifico a mim mesmo, a minha glória nada é; quem me glorifica é meu Pai, o qual vós dizeis que é vosso Deus.
\par 55 Entretanto, vós não o tendes conhecido; eu, porém, o conheço. Se eu disser que não o conheço, serei como vós: mentiroso; mas eu o conheço e guardo a sua palavra.
\par 56 Abraão, vosso pai, alegrou-se por ver o meu dia, viu-o e regozijou-se.
\par 57 Perguntaram-lhe, pois, os judeus: Ainda não tens cinqüenta anos e viste Abraão?
\par 58 Respondeu-lhes Jesus: Em verdade, em verdade eu vos digo: antes que Abraão existisse, EU SOU.
\par 59 Então, pegaram em pedras para atirarem nele; mas Jesus se ocultou e saiu do templo.

\chapter{9}

\par 1 Caminhando Jesus, viu um homem cego de nascença.
\par 2 E os seus discípulos perguntaram: Mestre, quem pecou, este ou seus pais, para que nascesse cego?
\par 3 Respondeu Jesus: Nem ele pecou, nem seus pais; mas foi para que se manifestem nele as obras de Deus.
\par 4 É necessário que façamos as obras daquele que me enviou, enquanto é dia; a noite vem, quando ninguém pode trabalhar.
\par 5 Enquanto estou no mundo, sou a luz do mundo.
\par 6 Dito isso, cuspiu na terra e, tendo feito lodo com a saliva, aplicou-o aos olhos do cego,
\par 7 dizendo-lhe: Vai, lava-te no tanque de Siloé (que quer dizer Enviado). Ele foi, lavou-se e voltou vendo.
\par 8 Então, os vizinhos e os que dantes o conheciam de vista, como mendigo, perguntavam: Não é este o que estava assentado pedindo esmolas?
\par 9 Uns diziam: É ele. Outros: Não, mas se parece com ele. Ele mesmo, porém, dizia: Sou eu.
\par 10 Perguntaram-lhe, pois: Como te foram abertos os olhos?
\par 11 Respondeu ele: O homem chamado Jesus fez lodo, untou-me os olhos e disse-me: Vai ao tanque de Siloé e lava-te. Então, fui, lavei-me e estou vendo.
\par 12 Disseram-lhe, pois: Onde está ele? Respondeu: Não sei.
\par 13 Levaram, pois, aos fariseus o que dantes fora cego.
\par 14 E era sábado o dia em que Jesus fez o lodo e lhe abriu os olhos.
\par 15 Então, os fariseus, por sua vez, lhe perguntaram como chegara a ver; ao que lhes respondeu: Aplicou lodo aos meus olhos, lavei-me e estou vendo.
\par 16 Por isso, alguns dos fariseus diziam: Esse homem não é de Deus, porque não guarda o sábado. Diziam outros: Como pode um homem pecador fazer tamanhos sinais? E houve dissensão entre eles.
\par 17 De novo, perguntaram ao cego: Que dizes tu a respeito dele, visto que te abriu os olhos? Que é profeta, respondeu ele.
\par 18 Não acreditaram os judeus que ele fora cego e que agora via, enquanto não lhe chamaram os pais
\par 19 e os interrogaram: É este o vosso filho, de quem dizeis que nasceu cego? Como, pois, vê agora?
\par 20 Então, os pais responderam: Sabemos que este é nosso filho e que nasceu cego;
\par 21 mas não sabemos como vê agora; ou quem lhe abriu os olhos também não sabemos. Perguntai a ele, idade tem; falará de si mesmo.
\par 22 Isto disseram seus pais porque estavam com medo dos judeus; pois estes já haviam assentado que, se alguém confessasse ser Jesus o Cristo, fosse expulso da sinagoga.
\par 23 Por isso, é que disseram os pais: Ele idade tem, interrogai-o.
\par 24 Então, chamaram, pela segunda vez, o homem que fora cego e lhe disseram: Dá glória a Deus; nós sabemos que esse homem é pecador.
\par 25 Ele retrucou: Se é pecador, não sei; uma coisa sei: eu era cego e agora vejo.
\par 26 Perguntaram-lhe, pois: Que te fez ele? como te abriu os olhos?
\par 27 Ele lhes respondeu: Já vo-lo disse, e não atendestes; por que quereis ouvir outra vez? Porventura, quereis vós também tornar-vos seus discípulos?
\par 28 Então, o injuriaram e lhe disseram: Discípulo dele és tu; mas nós somos discípulos de Moisés.
\par 29 Sabemos que Deus falou a Moisés; mas este nem sabemos donde é.
\par 30 Respondeu-lhes o homem: Nisto é de estranhar que vós não saibais donde ele é, e, contudo, me abriu os olhos.
\par 31 Sabemos que Deus não atende a pecadores; mas, pelo contrário, se alguém teme a Deus e pratica a sua vontade, a este atende.
\par 32 Desde que há mundo, jamais se ouviu que alguém tenha aberto os olhos a um cego de nascença.
\par 33 Se este homem não fosse de Deus, nada poderia ter feito.
\par 34 Mas eles retrucaram: Tu és nascido todo em pecado e nos ensinas a nós? E o expulsaram.
\par 35 Ouvindo Jesus que o tinham expulsado, encontrando-o, lhe perguntou: Crês tu no Filho do Homem?
\par 36 Ele respondeu e disse: Quem é, Senhor, para que eu nele creia?
\par 37 E Jesus lhe disse: Já o tens visto, e é o que fala contigo.
\par 38 Então, afirmou ele: Creio, Senhor; e o adorou.
\par 39 Prosseguiu Jesus: Eu vim a este mundo para juízo, a fim de que os que não vêem vejam, e os que vêem se tornem cegos.
\par 40 Alguns dentre os fariseus que estavam perto dele perguntaram-lhe: Acaso, também nós somos cegos?
\par 41 Respondeu-lhes Jesus: Se fôsseis cegos, não teríeis pecado algum; mas, porque agora dizeis: Nós vemos, subsiste o vosso pecado.

\chapter{10}

\par 1 Em verdade, em verdade vos digo: o que não entra pela porta no aprisco das ovelhas, mas sobe por outra parte, esse é ladrão e salteador.
\par 2 Aquele, porém, que entra pela porta, esse é o pastor das ovelhas.
\par 3 Para este o porteiro abre, as ovelhas ouvem a sua voz, ele chama pelo nome as suas próprias ovelhas e as conduz para fora.
\par 4 Depois de fazer sair todas as que lhe pertencem, vai adiante delas, e elas o seguem, porque lhe reconhecem a voz;
\par 5 mas de modo nenhum seguirão o estranho; antes, fugirão dele, porque não conhecem a voz dos estranhos.
\par 6 Jesus lhes propôs esta parábola, mas eles não compreenderam o sentido daquilo que lhes falava.
\par 7 Jesus, pois, lhes afirmou de novo: Em verdade, em verdade vos digo: eu sou a porta das ovelhas.
\par 8 Todos quantos vieram antes de mim são ladrões e salteadores; mas as ovelhas não lhes deram ouvido.
\par 9 Eu sou a porta. Se alguém entrar por mim, será salvo; entrará, e sairá, e achará pastagem.
\par 10 O ladrão vem somente para roubar, matar e destruir; eu vim para que tenham vida e a tenham em abundância.
\par 11 Eu sou o bom pastor. O bom pastor dá a vida pelas ovelhas.
\par 12 O mercenário, que não é pastor, a quem não pertencem as ovelhas, vê vir o lobo, abandona as ovelhas e foge; então, o lobo as arrebata e dispersa.
\par 13 O mercenário foge, porque é mercenário e não tem cuidado com as ovelhas.
\par 14 Eu sou o bom pastor; conheço as minhas ovelhas, e elas me conhecem a mim,
\par 15 assim como o Pai me conhece a mim, e eu conheço o Pai; e dou a minha vida pelas ovelhas.
\par 16 Ainda tenho outras ovelhas, não deste aprisco; a mim me convém conduzi-las; elas ouvirão a minha voz; então, haverá um rebanho e um pastor.
\par 17 Por isso, o Pai me ama, porque eu dou a minha vida para a reassumir.
\par 18 Ninguém a tira de mim; pelo contrário, eu espontaneamente a dou. Tenho autoridade para a entregar e também para reavê-la. Este mandato recebi de meu Pai.
\par 19 Por causa dessas palavras, rompeu nova dissensão entre os judeus.
\par 20 Muitos deles diziam: Ele tem demônio e enlouqueceu; por que o ouvis?
\par 21 Outros diziam: Este modo de falar não é de endemoninhado; pode, porventura, um demônio abrir os olhos aos cegos?
\par 22 Celebrava-se em Jerusalém a Festa da Dedicação. Era inverno.
\par 23 Jesus passeava no templo, no Pórtico de Salomão.
\par 24 Rodearam-no, pois, os judeus e o interpelaram: Até quando nos deixarás a mente em suspenso? Se tu és o Cristo, dize-o francamente.
\par 25 Respondeu-lhes Jesus: Já vo-lo disse, e não credes. As obras que eu faço em nome de meu Pai testificam a meu respeito.
\par 26 Mas vós não credes, porque não sois das minhas ovelhas.
\par 27 As minhas ovelhas ouvem a minha voz; eu as conheço, e elas me seguem.
\par 28 Eu lhes dou a vida eterna; jamais perecerão, e ninguém as arrebatará da minha mão.
\par 29 Aquilo que meu Pai me deu é maior do que tudo; e da mão do Pai ninguém pode arrebatar.
\par 30 Eu e o Pai somos um.
\par 31 Novamente, pegaram os judeus em pedras para lhe atirar.
\par 32 Disse-lhes Jesus: Tenho-vos mostrado muitas obras boas da parte do Pai; por qual delas me apedrejais?
\par 33 Responderam-lhe os judeus: Não é por obra boa que te apedrejamos, e sim por causa da blasfêmia, pois, sendo tu homem, te fazes Deus a ti mesmo.
\par 34 Replicou-lhes Jesus: Não está escrito na vossa lei: Eu disse: sois deuses?
\par 35 Se ele chamou deuses àqueles a quem foi dirigida a palavra de Deus, e a Escritura não pode falhar,
\par 36 então, daquele a quem o Pai santificou e enviou ao mundo, dizeis: Tu blasfemas; porque declarei: sou Filho de Deus?
\par 37 Se não faço as obras de meu Pai, não me acrediteis;
\par 38 mas, se faço, e não me credes, crede nas obras; para que possais saber e compreender que o Pai está em mim, e eu estou no Pai.
\par 39 Nesse ponto, procuravam, outra vez, prendê-lo; mas ele se livrou das suas mãos.
\par 40 Novamente, se retirou para além do Jordão, para o lugar onde João batizava no princípio; e ali permaneceu.
\par 41 E iam muitos ter com ele e diziam: Realmente, João não fez nenhum sinal, porém tudo quanto disse a respeito deste era verdade.
\par 42 E muitos ali creram nele.

\chapter{11}

\par 1 Estava enfermo Lázaro, de Betânia, da aldeia de Maria e de sua irmã Marta.
\par 2 Esta Maria, cujo irmão Lázaro estava enfermo, era a mesma que ungiu com bálsamo o Senhor e lhe enxugou os pés com os seus cabelos.
\par 3 Mandaram, pois, as irmãs de Lázaro dizer a Jesus: Senhor, está enfermo aquele a quem amas.
\par 4 Ao receber a notícia, disse Jesus: Esta enfermidade não é para morte, e sim para a glória de Deus, a fim de que o Filho de Deus seja por ela glorificado.
\par 5 Ora, amava Jesus a Marta, e a sua irmã, e a Lázaro.
\par 6 Quando, pois, soube que Lázaro estava doente, ainda se demorou dois dias no lugar onde estava.
\par 7 Depois, disse aos seus discípulos: Vamos outra vez para a Judéia.
\par 8 Disseram-lhe os discípulos: Mestre, ainda agora os judeus procuravam apedrejar-te, e voltas para lá?
\par 9 Respondeu Jesus: Não são doze as horas do dia? Se alguém andar de dia, não tropeça, porque vê a luz deste mundo;
\par 10 mas, se andar de noite, tropeça, porque nele não há luz.
\par 11 Isto dizia e depois lhes acrescentou: Nosso amigo Lázaro adormeceu, mas vou para despertá-lo.
\par 12 Disseram-lhe, pois, os discípulos: Senhor, se dorme, estará salvo.
\par 13 Jesus, porém, falara com respeito à morte de Lázaro; mas eles supunham que tivesse falado do repouso do sono.
\par 14 Então, Jesus lhes disse claramente: Lázaro morreu;
\par 15 e por vossa causa me alegro de que lá não estivesse, para que possais crer; mas vamos ter com ele.
\par 16 Então, Tomé, chamado Dídimo, disse aos condiscípulos: Vamos também nós para morrermos com ele.
\par 17 Chegando Jesus, encontrou Lázaro já sepultado, havia quatro dias.
\par 18 Ora, Betânia estava cerca de quinze estádios perto de Jerusalém.
\par 19 Muitos dentre os judeus tinham vindo ter com Marta e Maria, para as consolar a respeito de seu irmão.
\par 20 Marta, quando soube que vinha Jesus, saiu ao seu encontro; Maria, porém, ficou sentada em casa.
\par 21 Disse, pois, Marta a Jesus: Senhor, se estiveras aqui, não teria morrido meu irmão.
\par 22 Mas também sei que, mesmo agora, tudo quanto pedires a Deus, Deus to concederá.
\par 23 Declarou-lhe Jesus: Teu irmão há de ressurgir.
\par 24 Eu sei, replicou Marta, que ele há de ressurgir na ressurreição, no último dia.
\par 25 Disse-lhe Jesus: Eu sou a ressurreição e a vida. Quem crê em mim, ainda que morra, viverá;
\par 26 e todo o que vive e crê em mim não morrerá, eternamente. Crês isto?
\par 27 Sim, Senhor, respondeu ela, eu tenho crido que tu és o Cristo, o Filho de Deus que devia vir ao mundo.
\par 28 Tendo dito isto, retirou-se e chamou Maria, sua irmã, e lhe disse em particular: O Mestre chegou e te chama.
\par 29 Ela, ouvindo isto, levantou-se depressa e foi ter com ele,
\par 30 pois Jesus ainda não tinha entrado na aldeia, mas permanecia onde Marta se avistara com ele.
\par 31 Os judeus que estavam com Maria em casa e a consolavam, vendo-a levantar-se depressa e sair, seguiram-na, supondo que ela ia ao túmulo para chorar.
\par 32 Quando Maria chegou ao lugar onde estava Jesus, ao vê-lo, lançou-se-lhe aos pés, dizendo: Senhor, se estiveras aqui, meu irmão não teria morrido.
\par 33 Jesus, vendo-a chorar, e bem assim os judeus que a acompanhavam, agitou-se no espírito e comoveu-se.
\par 34 E perguntou: Onde o sepultastes? Eles lhe responderam: Senhor, vem e vê!
\par 35 Jesus chorou.
\par 36 Então, disseram os judeus: Vede quanto o amava.
\par 37 Mas alguns objetaram: Não podia ele, que abriu os olhos ao cego, fazer que este não morresse?
\par 38 Jesus, agitando-se novamente em si mesmo, encaminhou-se para o túmulo; era este uma gruta a cuja entrada tinham posto uma pedra.
\par 39 Então, ordenou Jesus: Tirai a pedra. Disse-lhe Marta, irmã do morto: Senhor, já cheira mal, porque já é de quatro dias.
\par 40 Respondeu-lhe Jesus: Não te disse eu que, se creres, verás a glória de Deus?
\par 41 Tiraram, então, a pedra. E Jesus, levantando os olhos para o céu, disse: Pai, graças te dou porque me ouviste.
\par 42 Aliás, eu sabia que sempre me ouves, mas assim falei por causa da multidão presente, para que creiam que tu me enviaste.
\par 43 E, tendo dito isto, clamou em alta voz: Lázaro, vem para fora!
\par 44 Saiu aquele que estivera morto, tendo os pés e as mãos ligados com ataduras e o rosto envolto num lenço. Então, lhes ordenou Jesus: Desatai-o e deixai-o ir.
\par 45 Muitos, pois, dentre os judeus que tinham vindo visitar Maria, vendo o que fizera Jesus, creram nele.
\par 46 Outros, porém, foram ter com os fariseus e lhes contaram dos feitos que Jesus realizara.
\par 47 Então, os principais sacerdotes e os fariseus convocaram o Sinédrio; e disseram: Que estamos fazendo, uma vez que este homem opera muitos sinais?
\par 48 Se o deixarmos assim, todos crerão nele; depois, virão os romanos e tomarão não só o nosso lugar, mas a própria nação.
\par 49 Caifás, porém, um dentre eles, sumo sacerdote naquele ano, advertiu-os, dizendo: Vós nada sabeis,
\par 50 nem considerais que vos convém que morra um só homem pelo povo e que não venha a perecer toda a nação.
\par 51 Ora, ele não disse isto de si mesmo; mas, sendo sumo sacerdote naquele ano, profetizou que Jesus estava para morrer pela nação
\par 52 e não somente pela nação, mas também para reunir em um só corpo os filhos de Deus, que andam dispersos.
\par 53 Desde aquele dia, resolveram matá-lo.
\par 54 De sorte que Jesus já não andava publicamente entre os judeus, mas retirou-se para uma região vizinha ao deserto, para uma cidade chamada Efraim; e ali permaneceu com os discípulos.
\par 55 Estava próxima a Páscoa dos judeus; e muitos daquela região subiram para Jerusalém antes da Páscoa, para se purificarem.
\par 56 Lá, procuravam Jesus e, estando eles no templo, diziam uns aos outros: Que vos parece? Não virá ele à festa?
\par 57 Ora, os principais sacerdotes e os fariseus tinham dado ordem para, se alguém soubesse onde ele estava, denunciá-lo, a fim de o prenderem.

\chapter{12}

\par 1 Seis dias antes da Páscoa, foi Jesus para Betânia, onde estava Lázaro, a quem ele ressuscitara dentre os mortos.
\par 2 Deram-lhe, pois, ali, uma ceia; Marta servia, sendo Lázaro um dos que estavam com ele à mesa.
\par 3 Então, Maria, tomando uma libra de bálsamo de nardo puro, mui precioso, ungiu os pés de Jesus e os enxugou com os seus cabelos; e encheu-se toda a casa com o perfume do bálsamo.
\par 4 Mas Judas Iscariotes, um dos seus discípulos, o que estava para traí-lo, disse:
\par 5 Por que não se vendeu este perfume por trezentos denários e não se deu aos pobres?
\par 6 Isto disse ele, não porque tivesse cuidado dos pobres; mas porque era ladrão e, tendo a bolsa, tirava o que nela se lançava.
\par 7 Jesus, entretanto, disse: Deixa-a! Que ela guarde isto para o dia em que me embalsamarem;
\par 8 porque os pobres, sempre os tendes convosco, mas a mim nem sempre me tendes.
\par 9 Soube numerosa multidão dos judeus que Jesus estava ali, e lá foram não só por causa dele, mas também para verem Lázaro, a quem ele ressuscitara dentre os mortos.
\par 10 Mas os principais sacerdotes resolveram matar também Lázaro;
\par 11 porque muitos dos judeus, por causa dele, voltavam crendo em Jesus.
\par 12 No dia seguinte, a numerosa multidão que viera à festa, tendo ouvido que Jesus estava de caminho para Jerusalém,
\par 13 tomou ramos de palmeiras e saiu ao seu encontro, clamando: Hosana! Bendito o que vem em nome do Senhor e que é Rei de Israel!
\par 14 E Jesus, tendo conseguido um jumentinho, montou-o, segundo está escrito:
\par 15 Não temas, filha de Sião, eis que o teu Rei aí vem, montado em um filho de jumenta.
\par 16 Seus discípulos a princípio não compreenderam isto; quando, porém, Jesus foi glorificado, então, eles se lembraram de que estas coisas estavam escritas a respeito dele e também de que isso lhe fizeram.
\par 17 Dava, pois, testemunho disto a multidão que estivera com ele, quando chamara a Lázaro do túmulo e o levantara dentre os mortos.
\par 18 Por causa disso, também, a multidão lhe saiu ao encontro, pois ouviu que ele fizera este sinal.
\par 19 De sorte que os fariseus disseram entre si: Vede que nada aproveitais! Eis aí vai o mundo após ele.
\par 20 Ora, entre os que subiram para adorar durante a festa, havia alguns gregos;
\par 21 estes, pois, se dirigiram a Filipe, que era de Betsaida da Galiléia, e lhe rogaram: Senhor, queremos ver Jesus.
\par 22 Filipe foi dizê-lo a André, e André e Filipe o comunicaram a Jesus.
\par 23 Respondeu-lhes Jesus: É chegada a hora de ser glorificado o Filho do Homem.
\par 24 Em verdade, em verdade vos digo: se o grão de trigo, caindo na terra, não morrer, fica ele só; mas, se morrer, produz muito fruto.
\par 25 Quem ama a sua vida perde-a; mas aquele que odeia a sua vida neste mundo preservá-la-á para a vida eterna.
\par 26 Se alguém me serve, siga-me, e, onde eu estou, ali estará também o meu servo. E, se alguém me servir, o Pai o honrará.
\par 27 Agora, está angustiada a minha alma, e que direi eu? Pai, salva-me desta hora? Mas precisamente com este propósito vim para esta hora.
\par 28 Pai, glorifica o teu nome. Então, veio uma voz do céu: Eu já o glorifiquei e ainda o glorificarei.
\par 29 A multidão, pois, que ali estava, tendo ouvido a voz, dizia ter havido um trovão. Outros diziam: Foi um anjo que lhe falou.
\par 30 Então, explicou Jesus: Não foi por mim que veio esta voz, e sim por vossa causa.
\par 31 Chegou o momento de ser julgado este mundo, e agora o seu príncipe será expulso.
\par 32 E eu, quando for levantado da terra, atrairei todos a mim mesmo.
\par 33 Isto dizia, significando de que gênero de morte estava para morrer.
\par 34 Replicou-lhe, pois, a multidão: Nós temos ouvido da lei que o Cristo permanece para sempre, e como dizes tu ser necessário que o Filho do Homem seja levantado? Quem é esse Filho do Homem?
\par 35 Respondeu-lhes Jesus: Ainda por um pouco a luz está convosco. Andai enquanto tendes a luz, para que as trevas não vos apanhem; e quem anda nas trevas não sabe para onde vai.
\par 36 Enquanto tendes a luz, crede na luz, para que vos torneis filhos da luz. Jesus disse estas coisas e, retirando-se, ocultou-se deles.
\par 37 E, embora tivesse feito tantos sinais na sua presença, não creram nele,
\par 38 para se cumprir a palavra do profeta Isaías, que diz: Senhor, quem creu em nossa pregação? E a quem foi revelado o braço do Senhor?
\par 39 Por isso, não podiam crer, porque Isaías disse ainda:
\par 40 Cegou-lhes os olhos e endureceu-lhes o coração, para que não vejam com os olhos, nem entendam com o coração, e se convertam, e sejam por mim curados.
\par 41 Isto disse Isaías porque viu a glória dele e falou a seu respeito.
\par 42 Contudo, muitos dentre as próprias autoridades creram nele, mas, por causa dos fariseus, não o confessavam, para não serem expulsos da sinagoga;
\par 43 porque amaram mais a glória dos homens do que a glória de Deus.
\par 44 E Jesus clamou, dizendo: Quem crê em mim crê, não em mim, mas naquele que me enviou.
\par 45 E quem me vê a mim vê aquele que me enviou.
\par 46 Eu vim como luz para o mundo, a fim de que todo aquele que crê em mim não permaneça nas trevas.
\par 47 Se alguém ouvir as minhas palavras e não as guardar, eu não o julgo; porque eu não vim para julgar o mundo, e sim para salvá-lo.
\par 48 Quem me rejeita e não recebe as minhas palavras tem quem o julgue; a própria palavra que tenho proferido, essa o julgará no último dia.
\par 49 Porque eu não tenho falado por mim mesmo, mas o Pai, que me enviou, esse me tem prescrito o que dizer e o que anunciar.
\par 50 E sei que o seu mandamento é a vida eterna. As coisas, pois, que eu falo, como o Pai mo tem dito, assim falo.

\chapter{13}

\par 1 Ora, antes da Festa da Páscoa, sabendo Jesus que era chegada a sua hora de passar deste mundo para o Pai, tendo amado os seus que estavam no mundo, amou-os até ao fim.
\par 2 Durante a ceia, tendo já o diabo posto no coração de Judas Iscariotes, filho de Simão, que traísse a Jesus,
\par 3 sabendo este que o Pai tudo confiara às suas mãos, e que ele viera de Deus, e voltava para Deus,
\par 4 levantou-se da ceia, tirou a vestimenta de cima e, tomando uma toalha, cingiu-se com ela.
\par 5 Depois, deitou água na bacia e passou a lavar os pés aos discípulos e a enxugar-lhos com a toalha com que estava cingido.
\par 6 Aproximou-se, pois, de Simão Pedro, e este lhe disse: Senhor, tu me lavas os pés a mim?
\par 7 Respondeu-lhe Jesus: O que eu faço não o sabes agora; compreendê-lo-ás depois.
\par 8 Disse-lhe Pedro: Nunca me lavarás os pés. Respondeu-lhe Jesus: Se eu não te lavar, não tens parte comigo.
\par 9 Então, Pedro lhe pediu: Senhor, não somente os pés, mas também as mãos e a cabeça.
\par 10 Declarou-lhe Jesus: Quem já se banhou não necessita de lavar senão os pés; quanto ao mais, está todo limpo. Ora, vós estais limpos, mas não todos.
\par 11 Pois ele sabia quem era o traidor. Foi por isso que disse: Nem todos estais limpos.
\par 12 Depois de lhes ter lavado os pés, tomou as vestes e, voltando à mesa, perguntou-lhes: Compreendeis o que vos fiz?
\par 13 Vós me chamais o Mestre e o Senhor e dizeis bem; porque eu o sou.
\par 14 Ora, se eu, sendo o Senhor e o Mestre, vos lavei os pés, também vós deveis lavar os pés uns dos outros.
\par 15 Porque eu vos dei o exemplo, para que, como eu vos fiz, façais vós também.
\par 16 Em verdade, em verdade vos digo que o servo não é maior do que seu senhor, nem o enviado, maior do que aquele que o enviou.
\par 17 Ora, se sabeis estas coisas, bem-aventurados sois se as praticardes.
\par 18 Não falo a respeito de todos vós, pois eu conheço aqueles que escolhi; é, antes, para que se cumpra a Escritura: Aquele que come do meu pão levantou contra mim seu calcanhar.
\par 19 Desde já vos digo, antes que aconteça, para que, quando acontecer, creiais que EU SOU.
\par 20 Em verdade, em verdade vos digo: quem recebe aquele que eu enviar, a mim me recebe; e quem me recebe recebe aquele que me enviou.
\par 21 Ditas estas coisas, angustiou-se Jesus em espírito e afirmou: Em verdade, em verdade vos digo que um dentre vós me trairá.
\par 22 Então, os discípulos olharam uns para os outros, sem saber a quem ele se referia.
\par 23 Ora, ali estava conchegado a Jesus um dos seus discípulos, aquele a quem ele amava;
\par 24 a esse fez Simão Pedro sinal, dizendo-lhe: Pergunta a quem ele se refere.
\par 25 Então, aquele discípulo, reclinando-se sobre o peito de Jesus, perguntou-lhe: Senhor, quem é?
\par 26 Respondeu Jesus: É aquele a quem eu der o pedaço de pão molhado. Tomou, pois, um pedaço de pão e, tendo-o molhado, deu-o a Judas, filho de Simão Iscariotes.
\par 27 E, após o bocado, imediatamente, entrou nele Satanás. Então, disse Jesus: O que pretendes fazer, faze-o depressa.
\par 28 Nenhum, porém, dos que estavam à mesa percebeu a que fim lhe dissera isto.
\par 29 Pois, como Judas era quem trazia a bolsa, pensaram alguns que Jesus lhe dissera: Compra o que precisamos para a festa ou lhe ordenara que desse alguma coisa aos pobres.
\par 30 Ele, tendo recebido o bocado, saiu logo. E era noite.
\par 31 Quando ele saiu, disse Jesus: Agora, foi glorificado o Filho do Homem, e Deus foi glorificado nele;
\par 32 se Deus foi glorificado nele, também Deus o glorificará nele mesmo; e glorificá-lo-á imediatamente.
\par 33 Filhinhos, ainda por um pouco estou convosco; buscar-me-eis, e o que eu disse aos judeus também agora vos digo a vós outros: para onde eu vou, vós não podeis ir.
\par 34 Novo mandamento vos dou: que vos ameis uns aos outros; assim como eu vos amei, que também vos ameis uns aos outros.
\par 35 Nisto conhecerão todos que sois meus discípulos: se tiverdes amor uns aos outros.
\par 36 Perguntou-lhe Simão Pedro: Senhor, para onde vais? Respondeu Jesus: Para onde vou, não me podes seguir agora; mais tarde, porém, me seguirás.
\par 37 Replicou Pedro: Senhor, por que não posso seguir-te agora? Por ti darei a própria vida.
\par 38 Respondeu Jesus: Darás a vida por mim? Em verdade, em verdade te digo que jamais cantará o galo antes que me negues três vezes.

\chapter{14}

\par 1 Não se turbe o vosso coração; credes em Deus, crede também em mim.
\par 2 Na casa de meu Pai há muitas moradas. Se assim não fora, eu vo-lo teria dito. Pois vou preparar-vos lugar.
\par 3 E, quando eu for e vos preparar lugar, voltarei e vos receberei para mim mesmo, para que, onde eu estou, estejais vós também.
\par 4 E vós sabeis o caminho para onde eu vou.
\par 5 Disse-lhe Tomé: Senhor, não sabemos para onde vais; como saber o caminho?
\par 6 Respondeu-lhe Jesus: Eu sou o caminho, e a verdade, e a vida; ninguém vem ao Pai senão por mim.
\par 7 Se vós me tivésseis conhecido, conheceríeis também a meu Pai. Desde agora o conheceis e o tendes visto.
\par 8 Replicou-lhe Filipe: Senhor, mostra-nos o Pai, e isso nos basta.
\par 9 Disse-lhe Jesus: Filipe, há tanto tempo estou convosco, e não me tens conhecido? Quem me vê a mim vê o Pai; como dizes tu: Mostra-nos o Pai?
\par 10 Não crês que eu estou no Pai e que o Pai está em mim? As palavras que eu vos digo não as digo por mim mesmo; mas o Pai, que permanece em mim, faz as suas obras.
\par 11 Crede-me que estou no Pai, e o Pai, em mim; crede ao menos por causa das mesmas obras.
\par 12 Em verdade, em verdade vos digo que aquele que crê em mim fará também as obras que eu faço e outras maiores fará, porque eu vou para junto do Pai.
\par 13 E tudo quanto pedirdes em meu nome, isso farei, a fim de que o Pai seja glorificado no Filho.
\par 14 Se me pedirdes alguma coisa em meu nome, eu o farei.
\par 15 Se me amais, guardareis os meus mandamentos.
\par 16 E eu rogarei ao Pai, e ele vos dará outro Consolador, a fim de que esteja para sempre convosco,
\par 17 o Espírito da verdade, que o mundo não pode receber, porque não no vê, nem o conhece; vós o conheceis, porque ele habita convosco e estará em vós.
\par 18 Não vos deixarei órfãos, voltarei para vós outros.
\par 19 Ainda por um pouco, e o mundo não me verá mais; vós, porém, me vereis; porque eu vivo, vós também vivereis.
\par 20 Naquele dia, vós conhecereis que eu estou em meu Pai, e vós, em mim, e eu, em vós.
\par 21 Aquele que tem os meus mandamentos e os guarda, esse é o que me ama; e aquele que me ama será amado por meu Pai, e eu também o amarei e me manifestarei a ele.
\par 22 Disse-lhe Judas, não o Iscariotes: Donde procede, Senhor, que estás para manifestar-te a nós e não ao mundo?
\par 23 Respondeu Jesus: Se alguém me ama, guardará a minha palavra; e meu Pai o amará, e viremos para ele e faremos nele morada.
\par 24 Quem não me ama não guarda as minhas palavras; e a palavra que estais ouvindo não é minha, mas do Pai, que me enviou.
\par 25 Isto vos tenho dito, estando ainda convosco;
\par 26 mas o Consolador, o Espírito Santo, a quem o Pai enviará em meu nome, esse vos ensinará todas as coisas e vos fará lembrar de tudo o que vos tenho dito.
\par 27 Deixo-vos a paz, a minha paz vos dou; não vo-la dou como a dá o mundo. Não se turbe o vosso coração, nem se atemorize.
\par 28 Ouvistes que eu vos disse: vou e volto para junto de vós. Se me amásseis, alegrar-vos-íeis de que eu vá para o Pai, pois o Pai é maior do que eu.
\par 29 Disse-vos agora, antes que aconteça, para que, quando acontecer, vós creiais.
\par 30 Já não falarei muito convosco, porque aí vem o príncipe do mundo; e ele nada tem em mim;
\par 31 contudo, assim procedo para que o mundo saiba que eu amo o Pai e que faço como o Pai me ordenou. Levantai-vos, vamo-nos daqui.

\chapter{15}

\par 1 Eu sou a videira verdadeira, e meu Pai é o agricultor.
\par 2 Todo ramo que, estando em mim, não der fruto, ele o corta; e todo o que dá fruto limpa, para que produza mais fruto ainda.
\par 3 Vós já estais limpos pela palavra que vos tenho falado;
\par 4 permanecei em mim, e eu permanecerei em vós. Como não pode o ramo produzir fruto de si mesmo, se não permanecer na videira, assim, nem vós o podeis dar, se não permanecerdes em mim.
\par 5 Eu sou a videira, vós, os ramos. Quem permanece em mim, e eu, nele, esse dá muito fruto; porque sem mim nada podeis fazer.
\par 6 Se alguém não permanecer em mim, será lançado fora, à semelhança do ramo, e secará; e o apanham, lançam no fogo e o queimam.
\par 7 Se permanecerdes em mim, e as minhas palavras permanecerem em vós, pedireis o que quiserdes, e vos será feito.
\par 8 Nisto é glorificado meu Pai, em que deis muito fruto; e assim vos tornareis meus discípulos.
\par 9 Como o Pai me amou, também eu vos amei; permanecei no meu amor.
\par 10 Se guardardes os meus mandamentos, permanecereis no meu amor; assim como também eu tenho guardado os mandamentos de meu Pai e no seu amor permaneço.
\par 11 Tenho-vos dito estas coisas para que o meu gozo esteja em vós, e o vosso gozo seja completo.
\par 12 O meu mandamento é este: que vos ameis uns aos outros, assim como eu vos amei.
\par 13 Ninguém tem maior amor do que este: de dar alguém a própria vida em favor dos seus amigos.
\par 14 Vós sois meus amigos, se fazeis o que eu vos mando.
\par 15 Já não vos chamo servos, porque o servo não sabe o que faz o seu senhor; mas tenho-vos chamado amigos, porque tudo quanto ouvi de meu Pai vos tenho dado a conhecer.
\par 16 Não fostes vós que me escolhestes a mim; pelo contrário, eu vos escolhi a vós outros e vos designei para que vades e deis fruto, e o vosso fruto permaneça; a fim de que tudo quanto pedirdes ao Pai em meu nome, ele vo-lo conceda.
\par 17 Isto vos mando: que vos ameis uns aos outros.
\par 18 Se o mundo vos odeia, sabei que, primeiro do que a vós outros, me odiou a mim.
\par 19 Se vós fôsseis do mundo, o mundo amaria o que era seu; como, todavia, não sois do mundo, pelo contrário, dele vos escolhi, por isso, o mundo vos odeia.
\par 20 Lembrai-vos da palavra que eu vos disse: não é o servo maior do que seu senhor. Se me perseguiram a mim, também perseguirão a vós outros; se guardaram a minha palavra, também guardarão a vossa.
\par 21 Tudo isto, porém, vos farão por causa do meu nome, porquanto não conhecem aquele que me enviou.
\par 22 Se eu não viera, nem lhes houvera falado, pecado não teriam; mas, agora, não têm desculpa do seu pecado.
\par 23 Quem me odeia odeia também a meu Pai.
\par 24 Se eu não tivesse feito entre eles tais obras, quais nenhum outro fez, pecado não teriam; mas, agora, não somente têm eles visto, mas também odiado, tanto a mim como a meu Pai.
\par 25 Isto, porém, é para que se cumpra a palavra escrita na sua lei: Odiaram-me sem motivo.
\par 26 Quando, porém, vier o Consolador, que eu vos enviarei da parte do Pai, o Espírito da verdade, que dele procede, esse dará testemunho de mim;
\par 27 e vós também testemunhareis, porque estais comigo desde o princípio.

\chapter{16}

\par 1 Tenho-vos dito estas coisas para que não vos escandalizeis.
\par 2 Eles vos expulsarão das sinagogas; mas vem a hora em que todo o que vos matar julgará com isso tributar culto a Deus.
\par 3 Isto farão porque não conhecem o Pai, nem a mim.
\par 4 Ora, estas coisas vos tenho dito para que, quando a hora chegar, vos recordeis de que eu vo-las disse. Não vo-las disse desde o princípio, porque eu estava convosco.
\par 5 Mas, agora, vou para junto daquele que me enviou, e nenhum de vós me pergunta: Para onde vais?
\par 6 Pelo contrário, porque vos tenho dito estas coisas, a tristeza encheu o vosso coração.
\par 7 Mas eu vos digo a verdade: convém-vos que eu vá, porque, se eu não for, o Consolador não virá para vós outros; se, porém, eu for, eu vo-lo enviarei.
\par 8 Quando ele vier, convencerá o mundo do pecado, da justiça e do juízo:
\par 9 do pecado, porque não crêem em mim;
\par 10 da justiça, porque vou para o Pai, e não me vereis mais;
\par 11 do juízo, porque o príncipe deste mundo já está julgado.
\par 12 Tenho ainda muito que vos dizer, mas vós não o podeis suportar agora;
\par 13 quando vier, porém, o Espírito da verdade, ele vos guiará a toda a verdade; porque não falará por si mesmo, mas dirá tudo o que tiver ouvido e vos anunciará as coisas que hão de vir.
\par 14 Ele me glorificará, porque há de receber do que é meu e vo-lo há de anunciar.
\par 15 Tudo quanto o Pai tem é meu; por isso é que vos disse que há de receber do que é meu e vo-lo há de anunciar.
\par 16 Um pouco, e não mais me vereis; outra vez um pouco, e ver-me-eis.
\par 17 Então, alguns dos seus discípulos disseram uns aos outros: Que vem a ser isto que nos diz: Um pouco, e não mais me vereis, e outra vez um pouco, e ver-me-eis; e: Vou para o Pai?
\par 18 Diziam, pois: Que vem a ser esse -- um pouco? Não compreendemos o que quer dizer.
\par 19 Percebendo Jesus que desejavam interrogá-lo, perguntou-lhes: Indagais entre vós a respeito disto que vos disse: Um pouco, e não me vereis, e outra vez um pouco, e ver-me-eis?
\par 20 Em verdade, em verdade eu vos digo que chorareis e vos lamentareis, e o mundo se alegrará; vós ficareis tristes, mas a vossa tristeza se converterá em alegria.
\par 21 A mulher, quando está para dar à luz, tem tristeza, porque a sua hora é chegada; mas, depois de nascido o menino, já não se lembra da aflição, pelo prazer que tem de ter nascido ao mundo um homem.
\par 22 Assim também agora vós tendes tristeza; mas outra vez vos verei; o vosso coração se alegrará, e a vossa alegria ninguém poderá tirar.
\par 23 Naquele dia, nada me perguntareis. Em verdade, em verdade vos digo: se pedirdes alguma coisa ao Pai, ele vo-la concederá em meu nome.
\par 24 Até agora nada tendes pedido em meu nome; pedi e recebereis, para que a vossa alegria seja completa.
\par 25 Estas coisas vos tenho dito por meio de figuras; vem a hora em que não vos falarei por meio de comparações, mas vos falarei claramente a respeito do Pai.
\par 26 Naquele dia, pedireis em meu nome; e não vos digo que rogarei ao Pai por vós.
\par 27 Porque o próprio Pai vos ama, visto que me tendes amado e tendes crido que eu vim da parte de Deus.
\par 28 Vim do Pai e entrei no mundo; todavia, deixo o mundo e vou para o Pai.
\par 29 Disseram os seus discípulos: Agora é que falas claramente e não empregas nenhuma figura.
\par 30 Agora, vemos que sabes todas as coisas e não precisas de que alguém te pergunte; por isso, cremos que, de fato, vieste de Deus.
\par 31 Respondeu-lhes Jesus: Credes agora?
\par 32 Eis que vem a hora e já é chegada, em que sereis dispersos, cada um para sua casa, e me deixareis só; contudo, não estou só, porque o Pai está comigo.
\par 33 Estas coisas vos tenho dito para que tenhais paz em mim. No mundo, passais por aflições; mas tende bom ânimo; eu venci o mundo.

\chapter{17}

\par 1 Tendo Jesus falado estas coisas, levantou os olhos ao céu e disse: Pai, é chegada a hora; glorifica a teu Filho, para que o Filho te glorifique a ti,
\par 2 assim como lhe conferiste autoridade sobre toda a carne, a fim de que ele conceda a vida eterna a todos os que lhe deste.
\par 3 E a vida eterna é esta: que te conheçam a ti, o único Deus verdadeiro, e a Jesus Cristo, a quem enviaste.
\par 4 Eu te glorifiquei na terra, consumando a obra que me confiaste para fazer;
\par 5 e, agora, glorifica-me, ó Pai, contigo mesmo, com a glória que eu tive junto de ti, antes que houvesse mundo.
\par 6 Manifestei o teu nome aos homens que me deste do mundo. Eram teus, tu mos confiaste, e eles têm guardado a tua palavra.
\par 7 Agora, eles reconhecem que todas as coisas que me tens dado provêm de ti;
\par 8 porque eu lhes tenho transmitido as palavras que me deste, e eles as receberam, e verdadeiramente conheceram que saí de ti, e creram que tu me enviaste.
\par 9 É por eles que eu rogo; não rogo pelo mundo, mas por aqueles que me deste, porque são teus;
\par 10 ora, todas as minhas coisas são tuas, e as tuas coisas são minhas; e, neles, eu sou glorificado.
\par 11 Já não estou no mundo, mas eles continuam no mundo, ao passo que eu vou para junto de ti. Pai santo, guarda-os em teu nome, que me deste, para que eles sejam um, assim como nós.
\par 12 Quando eu estava com eles, guardava-os no teu nome, que me deste, e protegi-os, e nenhum deles se perdeu, exceto o filho da perdição, para que se cumprisse a Escritura.
\par 13 Mas, agora, vou para junto de ti e isto falo no mundo para que eles tenham o meu gozo completo em si mesmos.
\par 14 Eu lhes tenho dado a tua palavra, e o mundo os odiou, porque eles não são do mundo, como também eu não sou.
\par 15 Não peço que os tires do mundo, e sim que os guardes do mal.
\par 16 Eles não são do mundo, como também eu não sou.
\par 17 Santifica-os na verdade; a tua palavra é a verdade.
\par 18 Assim como tu me enviaste ao mundo, também eu os enviei ao mundo.
\par 19 E a favor deles eu me santifico a mim mesmo, para que eles também sejam santificados na verdade.
\par 20 Não rogo somente por estes, mas também por aqueles que vierem a crer em mim, por intermédio da sua palavra;
\par 21 a fim de que todos sejam um; e como és tu, ó Pai, em mim e eu em ti, também sejam eles em nós; para que o mundo creia que tu me enviaste.
\par 22 Eu lhes tenho transmitido a glória que me tens dado, para que sejam um, como nós o somos;
\par 23 eu neles, e tu em mim, a fim de que sejam aperfeiçoados na unidade, para que o mundo conheça que tu me enviaste e os amaste, como também amaste a mim.
\par 24 Pai, a minha vontade é que onde eu estou, estejam também comigo os que me deste, para que vejam a minha glória que me conferiste, porque me amaste antes da fundação do mundo.
\par 25 Pai justo, o mundo não te conheceu; eu, porém, te conheci, e também estes compreenderam que tu me enviaste.
\par 26 Eu lhes fiz conhecer o teu nome e ainda o farei conhecer, a fim de que o amor com que me amaste esteja neles, e eu neles esteja.

\chapter{18}

\par 1 Tendo Jesus dito estas palavras, saiu juntamente com seus discípulos para o outro lado do ribeiro Cedrom, onde havia um jardim; e aí entrou com eles.
\par 2 E Judas, o traidor, também conhecia aquele lugar, porque Jesus ali estivera muitas vezes com seus discípulos.
\par 3 Tendo, pois, Judas recebido a escolta e, dos principais sacerdotes e dos fariseus, alguns guardas, chegou a este lugar com lanternas, tochas e armas.
\par 4 Sabendo, pois, Jesus todas as coisas que sobre ele haviam de vir, adiantou-se e perguntou-lhes: A quem buscais?
\par 5 Responderam-lhe: A Jesus, o Nazareno. Então, Jesus lhes disse: Sou eu. Ora, Judas, o traidor, estava também com eles.
\par 6 Quando, pois, Jesus lhes disse: Sou eu, recuaram e caíram por terra.
\par 7 Jesus, de novo, lhes perguntou: A quem buscais? Responderam: A Jesus, o Nazareno.
\par 8 Então, lhes disse Jesus: Já vos declarei que sou eu; se é a mim, pois, que buscais, deixai ir estes;
\par 9 para se cumprir a palavra que dissera: Não perdi nenhum dos que me deste.
\par 10 Então, Simão Pedro puxou da espada que trazia e feriu o servo do sumo sacerdote, cortando-lhe a orelha direita; e o nome do servo era Malco.
\par 11 Mas Jesus disse a Pedro: Mete a espada na bainha; não beberei, porventura, o cálice que o Pai me deu?
\par 12 Assim, a escolta, o comandante e os guardas dos judeus prenderam Jesus, manietaram-no
\par 13 e o conduziram primeiramente a Anás; pois era sogro de Caifás, sumo sacerdote naquele ano.
\par 14 Ora, Caifás era quem havia declarado aos judeus ser conveniente morrer um homem pelo povo.
\par 15 Simão Pedro e outro discípulo seguiam a Jesus. Sendo este discípulo conhecido do sumo sacerdote, entrou para o pátio deste com Jesus.
\par 16 Pedro, porém, ficou de fora, junto à porta. Saindo, pois, o outro discípulo, que era conhecido do sumo sacerdote, falou com a encarregada da porta e levou a Pedro para dentro.
\par 17 Então, a criada, encarregada da porta, perguntou a Pedro: Não és tu também um dos discípulos deste homem? Não sou, respondeu ele.
\par 18 Ora, os servos e os guardas estavam ali, tendo acendido um braseiro, por causa do frio, e aquentavam-se. Pedro estava no meio deles, aquentando-se também.
\par 19 Então, o sumo sacerdote interrogou a Jesus acerca dos seus discípulos e da sua doutrina.
\par 20 Declarou-lhe Jesus: Eu tenho falado francamente ao mundo; ensinei continuamente tanto nas sinagogas como no templo, onde todos os judeus se reúnem, e nada disse em oculto.
\par 21 Por que me interrogas? Pergunta aos que ouviram o que lhes falei; bem sabem eles o que eu disse.
\par 22 Dizendo ele isto, um dos guardas que ali estavam deu uma bofetada em Jesus, dizendo: É assim que falas ao sumo sacerdote?
\par 23 Replicou-lhe Jesus: Se falei mal, dá testemunho do mal; mas, se falei bem, por que me feres?
\par 24 Então, Anás o enviou, manietado, à presença de Caifás, o sumo sacerdote.
\par 25 Lá estava Simão Pedro, aquentando-se. Perguntaram-lhe, pois: És tu, porventura, um dos discípulos dele? Ele negou e disse: Não sou.
\par 26 Um dos servos do sumo sacerdote, parente daquele a quem Pedro tinha decepado a orelha, perguntou: Não te vi eu no jardim com ele?
\par 27 De novo, Pedro o negou, e, no mesmo instante, cantou o galo.
\par 28 Depois, levaram Jesus da casa de Caifás para o pretório. Era cedo de manhã. Eles não entraram no pretório para não se contaminarem, mas poderem comer a Páscoa.
\par 29 Então, Pilatos saiu para lhes falar e lhes disse: Que acusação trazeis contra este homem?
\par 30 Responderam-lhe: Se este não fosse malfeitor, não to entregaríamos.
\par 31 Replicou-lhes, pois, Pilatos: Tomai-o vós outros e julgai-o segundo a vossa lei. Responderam-lhe os judeus: A nós não nos é lícito matar ninguém;
\par 32 para que se cumprisse a palavra de Jesus, significando o modo por que havia de morrer.
\par 33 Tornou Pilatos a entrar no pretório, chamou Jesus e perguntou-lhe: És tu o rei dos judeus?
\par 34 Respondeu Jesus: Vem de ti mesmo esta pergunta ou to disseram outros a meu respeito?
\par 35 Replicou Pilatos: Porventura, sou judeu? A tua própria gente e os principais sacerdotes é que te entregaram a mim. Que fizeste?
\par 36 Respondeu Jesus: O meu reino não é deste mundo. Se o meu reino fosse deste mundo, os meus ministros se empenhariam por mim, para que não fosse eu entregue aos judeus; mas agora o meu reino não é daqui.
\par 37 Então, lhe disse Pilatos: Logo, tu és rei? Respondeu Jesus: Tu dizes que sou rei. Eu para isso nasci e para isso vim ao mundo, a fim de dar testemunho da verdade. Todo aquele que é da verdade ouve a minha voz.
\par 38 Perguntou-lhe Pilatos: Que é a verdade? Tendo dito isto, voltou aos judeus e lhes disse: Eu não acho nele crime algum.
\par 39 É costume entre vós que eu vos solte alguém por ocasião da Páscoa; quereis, pois, que vos solte o rei dos judeus?
\par 40 Então, gritaram todos, novamente: Não este, mas Barrabás! Ora, Barrabás era salteador.

\chapter{19}

\par 1 Então, por isso, Pilatos tomou a Jesus e mandou açoitá-lo.
\par 2 Os soldados, tendo tecido uma coroa de espinhos, puseram-lha na cabeça e vestiram-no com um manto de púrpura.
\par 3 Chegavam-se a ele e diziam: Salve, rei dos judeus! E davam-lhe bofetadas.
\par 4 Outra vez saiu Pilatos e lhes disse: Eis que eu vo-lo apresento, para que saibais que eu não acho nele crime algum.
\par 5 Saiu, pois, Jesus trazendo a coroa de espinhos e o manto de púrpura. Disse-lhes Pilatos: Eis o homem!
\par 6 Ao verem-no, os principais sacerdotes e os seus guardas gritaram: Crucifica-o! Crucifica-o! Disse-lhes Pilatos: Tomai-o vós outros e crucificai-o; porque eu não acho nele crime algum.
\par 7 Responderam-lhe os judeus: Temos uma lei, e, de conformidade com a lei, ele deve morrer, porque a si mesmo se fez Filho de Deus.
\par 8 Pilatos, ouvindo tal declaração, ainda mais atemorizado ficou,
\par 9 e, tornando a entrar no pretório, perguntou a Jesus: Donde és tu? Mas Jesus não lhe deu resposta.
\par 10 Então, Pilatos o advertiu: Não me respondes? Não sabes que tenho autoridade para te soltar e autoridade para te crucificar?
\par 11 Respondeu Jesus: Nenhuma autoridade terias sobre mim, se de cima não te fosse dada; por isso, quem me entregou a ti maior pecado tem.
\par 12 A partir deste momento, Pilatos procurava soltá-lo, mas os judeus clamavam: Se soltas a este, não és amigo de César! Todo aquele que se faz rei é contra César!
\par 13 Ouvindo Pilatos estas palavras, trouxe Jesus para fora e sentou-se no tribunal, no lugar chamado Pavimento, no hebraico Gabatá.
\par 14 E era a parasceve pascal, cerca da hora sexta; e disse aos judeus: Eis aqui o vosso rei.
\par 15 Eles, porém, clamavam: Fora! Fora! Crucifica-o! Disse-lhes Pilatos: Hei de crucificar o vosso rei? Responderam os principais sacerdotes: Não temos rei, senão César!
\par 16 Então, Pilatos o entregou para ser crucificado.
\par 17 Tomaram eles, pois, a Jesus; e ele próprio, carregando a sua cruz, saiu para o lugar chamado Calvário, Gólgota em hebraico,
\par 18 onde o crucificaram e com ele outros dois, um de cada lado, e Jesus no meio.
\par 19 Pilatos escreveu também um título e o colocou no cimo da cruz; o que estava escrito era: JESUS NAZARENO, O REI DOS JUDEUS.
\par 20 Muitos judeus leram este título, porque o lugar em que Jesus fora crucificado era perto da cidade; e estava escrito em hebraico, latim e grego.
\par 21 Os principais sacerdotes diziam a Pilatos: Não escrevas: Rei dos judeus, e sim que ele disse: Sou o rei dos judeus.
\par 22 Respondeu Pilatos: O que escrevi escrevi.
\par 23 Os soldados, pois, quando crucificaram Jesus, tomaram-lhe as vestes e fizeram quatro partes, para cada soldado uma parte; e pegaram também a túnica. A túnica, porém, era sem costura, toda tecida de alto a baixo.
\par 24 Disseram, pois, uns aos outros: Não a rasguemos, mas lancemos sortes sobre ela para ver a quem caberá -- para se cumprir a Escritura: Repartiram entre si as minhas vestes e sobre a minha túnica lançaram sortes. Assim, pois, o fizeram os soldados.
\par 25 E junto à cruz estavam a mãe de Jesus, e a irmã dela, e Maria, mulher de Clopas, e Maria Madalena.
\par 26 Vendo Jesus sua mãe e junto a ela o discípulo amado, disse: Mulher, eis aí teu filho.
\par 27 Depois, disse ao discípulo: Eis aí tua mãe. Dessa hora em diante, o discípulo a tomou para casa.
\par 28 Depois, vendo Jesus que tudo já estava consumado, para se cumprir a Escritura, disse: Tenho sede!
\par 29 Estava ali um vaso cheio de vinagre. Embeberam de vinagre uma esponja e, fixando-a num caniço de hissopo, lha chegaram à boca.
\par 30 Quando, pois, Jesus tomou o vinagre, disse: Está consumado! E, inclinando a cabeça, rendeu o espírito.
\par 31 Então, os judeus, para que no sábado não ficassem os corpos na cruz, visto como era a preparação, pois era grande o dia daquele sábado, rogaram a Pilatos que se lhes quebrassem as pernas, e fossem tirados.
\par 32 Os soldados foram e quebraram as pernas ao primeiro e ao outro que com ele tinham sido crucificados;
\par 33 chegando-se, porém, a Jesus, como vissem que já estava morto, não lhe quebraram as pernas.
\par 34 Mas um dos soldados lhe abriu o lado com uma lança, e logo saiu sangue e água.
\par 35 Aquele que isto viu testificou, sendo verdadeiro o seu testemunho; e ele sabe que diz a verdade, para que também vós creiais.
\par 36 E isto aconteceu para se cumprir a Escritura: Nenhum dos seus ossos será quebrado.
\par 37 E outra vez diz a Escritura: Eles verão aquele a quem traspassaram.
\par 38 Depois disto, José de Arimatéia, que era discípulo de Jesus, ainda que ocultamente pelo receio que tinha dos judeus, rogou a Pilatos lhe permitisse tirar o corpo de Jesus. Pilatos lho permitiu. Então, foi José de Arimatéia e retirou o corpo de Jesus.
\par 39 E também Nicodemos, aquele que anteriormente viera ter com Jesus à noite, foi, levando cerca de cem libras de um composto de mirra e aloés.
\par 40 Tomaram, pois, o corpo de Jesus e o envolveram em lençóis com os aromas, como é de uso entre os judeus na preparação para o sepulcro.
\par 41 No lugar onde Jesus fora crucificado, havia um jardim, e neste, um sepulcro novo, no qual ninguém tinha sido ainda posto.
\par 42 Ali, pois, por causa da preparação dos judeus e por estar perto o túmulo, depositaram o corpo de Jesus.

\chapter{20}

\par 1 No primeiro dia da semana, Maria Madalena foi ao sepulcro de madrugada, sendo ainda escuro, e viu que a pedra estava revolvida.
\par 2 Então, correu e foi ter com Simão Pedro e com o outro discípulo, a quem Jesus amava, e disse-lhes: Tiraram do sepulcro o Senhor, e não sabemos onde o puseram.
\par 3 Saiu, pois, Pedro e o outro discípulo e foram ao sepulcro.
\par 4 Ambos corriam juntos, mas o outro discípulo correu mais depressa do que Pedro e chegou primeiro ao sepulcro;
\par 5 e, abaixando-se, viu os lençóis de linho; todavia, não entrou.
\par 6 Então, Simão Pedro, seguindo-o, chegou e entrou no sepulcro. Ele também viu os lençóis,
\par 7 e o lenço que estivera sobre a cabeça de Jesus, e que não estava com os lençóis, mas deixado num lugar à parte.
\par 8 Então, entrou também o outro discípulo, que chegara primeiro ao sepulcro, e viu, e creu.
\par 9 Pois ainda não tinham compreendido a Escritura, que era necessário ressuscitar ele dentre os mortos.
\par 10 E voltaram os discípulos outra vez para casa.
\par 11 Maria, entretanto, permanecia junto à entrada do túmulo, chorando. Enquanto chorava, abaixou-se, e olhou para dentro do túmulo,
\par 12 e viu dois anjos vestidos de branco, sentados onde o corpo de Jesus fora posto, um à cabeceira e outro aos pés.
\par 13 Então, eles lhe perguntaram: Mulher, por que choras? Ela lhes respondeu: Porque levaram o meu Senhor, e não sei onde o puseram.
\par 14 Tendo dito isto, voltou-se para trás e viu Jesus em pé, mas não reconheceu que era Jesus.
\par 15 Perguntou-lhe Jesus: Mulher, por que choras? A quem procuras? Ela, supondo ser ele o jardineiro, respondeu: Senhor, se tu o tiraste, dize-me onde o puseste, e eu o levarei.
\par 16 Disse-lhe Jesus: Maria! Ela, voltando-se, lhe disse, em hebraico: Raboni (que quer dizer Mestre)!
\par 17 Recomendou-lhe Jesus: Não me detenhas; porque ainda não subi para meu Pai, mas vai ter com os meus irmãos e dize-lhes: Subo para meu Pai e vosso Pai, para meu Deus e vosso Deus.
\par 18 Então, saiu Maria Madalena anunciando aos discípulos: Vi o Senhor! E contava que ele lhe dissera estas coisas.
\par 19 Ao cair da tarde daquele dia, o primeiro da semana, trancadas as portas da casa onde estavam os discípulos com medo dos judeus, veio Jesus, pôs-se no meio e disse-lhes: Paz seja convosco!
\par 20 E, dizendo isto, lhes mostrou as mãos e o lado. Alegraram-se, portanto, os discípulos ao verem o Senhor.
\par 21 Disse-lhes, pois, Jesus outra vez: Paz seja convosco! Assim como o Pai me enviou, eu também vos envio.
\par 22 E, havendo dito isto, soprou sobre eles e disse-lhes: Recebei o Espírito Santo.
\par 23 Se de alguns perdoardes os pecados, são-lhes perdoados; se lhos retiverdes, são retidos.
\par 24 Ora, Tomé, um dos doze, chamado Dídimo, não estava com eles quando veio Jesus.
\par 25 Disseram-lhe, então, os outros discípulos: Vimos o Senhor. Mas ele respondeu: Se eu não vir nas suas mãos o sinal dos cravos, e ali não puser o dedo, e não puser a mão no seu lado, de modo algum acreditarei.
\par 26 Passados oito dias, estavam outra vez ali reunidos os seus discípulos, e Tomé, com eles. Estando as portas trancadas, veio Jesus, pôs-se no meio e disse-lhes: Paz seja convosco!
\par 27 E logo disse a Tomé: Põe aqui o dedo e vê as minhas mãos; chega também a mão e põe-na no meu lado; não sejas incrédulo, mas crente.
\par 28 Respondeu-lhe Tomé: Senhor meu e Deus meu!
\par 29 Disse-lhe Jesus: Porque me viste, creste? Bem-aventurados os que não viram e creram.
\par 30 Na verdade, fez Jesus diante dos discípulos muitos outros sinais que não estão escritos neste livro.
\par 31 Estes, porém, foram registrados para que creiais que Jesus é o Cristo, o Filho de Deus, e para que, crendo, tenhais vida em seu nome.

\chapter{21}

\par 1 Depois disto, tornou Jesus a manifestar-se aos discípulos junto do mar de Tiberíades; e foi assim que ele se manifestou:
\par 2 estavam juntos Simão Pedro, Tomé, chamado Dídimo, Natanael, que era de Caná da Galiléia, os filhos de Zebedeu e mais dois dos seus discípulos.
\par 3 Disse-lhes Simão Pedro: Vou pescar. Disseram-lhe os outros: Também nós vamos contigo. Saíram, e entraram no barco, e, naquela noite, nada apanharam.
\par 4 Mas, ao clarear da madrugada, estava Jesus na praia; todavia, os discípulos não reconheceram que era ele.
\par 5 Perguntou-lhes Jesus: Filhos, tendes aí alguma coisa de comer? Responderam-lhe: Não.
\par 6 Então, lhes disse: Lançai a rede à direita do barco e achareis. Assim fizeram e já não podiam puxar a rede, tão grande era a quantidade de peixes.
\par 7 Aquele discípulo a quem Jesus amava disse a Pedro: É o Senhor! Simão Pedro, ouvindo que era o Senhor, cingiu-se com sua veste, porque se havia despido, e lançou-se ao mar;
\par 8 mas os outros discípulos vieram no barquinho puxando a rede com os peixes; porque não estavam distantes da terra senão quase duzentos côvados.
\par 9 Ao saltarem em terra, viram ali umas brasas e, em cima, peixes; e havia também pão.
\par 10 Disse-lhes Jesus: Trazei alguns dos peixes que acabastes de apanhar.
\par 11 Simão Pedro entrou no barco e arrastou a rede para a terra, cheia de cento e cinqüenta e três grandes peixes; e, não obstante serem tantos, a rede não se rompeu.
\par 12 Disse-lhes Jesus: Vinde, comei. Nenhum dos discípulos ousava perguntar-lhe: Quem és tu? Porque sabiam que era o Senhor.
\par 13 Veio Jesus, tomou o pão, e lhes deu, e, de igual modo, o peixe.
\par 14 E já era esta a terceira vez que Jesus se manifestava aos discípulos, depois de ressuscitado dentre os mortos.
\par 15 Depois de terem comido, perguntou Jesus a Simão Pedro: Simão, filho de João, amas-me mais do que estes outros? Ele respondeu: Sim, Senhor, tu sabes que te amo. Ele lhe disse: Apascenta os meus cordeiros.
\par 16 Tornou a perguntar-lhe pela segunda vez: Simão, filho de João, tu me amas? Ele lhe respondeu: Sim, Senhor, tu sabes que te amo. Disse-lhe Jesus: Pastoreia as minhas ovelhas.
\par 17 Pela terceira vez Jesus lhe perguntou: Simão, filho de João, tu me amas? Pedro entristeceu-se por ele lhe ter dito, pela terceira vez: Tu me amas? E respondeu-lhe: Senhor, tu sabes todas as coisas, tu sabes que eu te amo. Jesus lhe disse: Apascenta as minhas ovelhas.
\par 18 Em verdade, em verdade te digo que, quando eras mais moço, tu te cingias a ti mesmo e andavas por onde querias; quando, porém, fores velho, estenderás as mãos, e outro te cingirá e te levará para onde não queres.
\par 19 Disse isto para significar com que gênero de morte Pedro havia de glorificar a Deus. Depois de assim falar, acrescentou-lhe: Segue-me.
\par 20 Então, Pedro, voltando-se, viu que também o ia seguindo o discípulo a quem Jesus amava, o qual na ceia se reclinara sobre o peito de Jesus e perguntara: Senhor, quem é o traidor?
\par 21 Vendo-o, pois, Pedro perguntou a Jesus: E quanto a este?
\par 22 Respondeu-lhe Jesus: Se eu quero que ele permaneça até que eu venha, que te importa? Quanto a ti, segue-me.
\par 23 Então, se tornou corrente entre os irmãos o dito de que aquele discípulo não morreria. Ora, Jesus não dissera que tal discípulo não morreria, mas: Se eu quero que ele permaneça até que eu venha, que te importa?
\par 24 Este é o discípulo que dá testemunho a respeito destas coisas e que as escreveu; e sabemos que o seu testemunho é verdadeiro.
\par 25 Há, porém, ainda muitas outras coisas que Jesus fez. Se todas elas fossem relatadas uma por uma, creio eu que nem no mundo inteiro caberiam os livros que seriam escritos.


\end{document}