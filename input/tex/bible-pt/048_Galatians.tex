\begin{document}

\title{Gálatas}


\chapter{1}

\par 1 Paulo, apóstolo, não da parte de homens, nem por intermédio de homem algum, mas por Jesus Cristo e por Deus Pai, que o ressuscitou dentre os mortos,
\par 2 e todos os irmãos meus companheiros, às igrejas da Galácia,
\par 3 graça a vós outros e paz, da parte de Deus, nosso Pai, e do [nosso] Senhor Jesus Cristo,
\par 4 o qual se entregou a si mesmo pelos nossos pecados, para nos desarraigar deste mundo perverso, segundo a vontade de nosso Deus e Pai,
\par 5 a quem seja a glória pelos séculos dos séculos. Amém!
\par 6 Admira-me que estejais passando tão depressa daquele que vos chamou na graça de Cristo para outro evangelho,
\par 7 o qual não é outro, senão que há alguns que vos perturbam e querem perverter o evangelho de Cristo.
\par 8 Mas, ainda que nós ou mesmo um anjo vindo do céu vos pregue evangelho que vá além do que vos temos pregado, seja anátema.
\par 9 Assim, como já dissemos, e agora repito, se alguém vos prega evangelho que vá além daquele que recebestes, seja anátema.
\par 10 Porventura, procuro eu, agora, o favor dos homens ou o de Deus? Ou procuro agradar a homens? Se agradasse ainda a homens, não seria servo de Cristo.
\par 11 Faço-vos, porém, saber, irmãos, que o evangelho por mim anunciado não é segundo o homem,
\par 12 porque eu não o recebi, nem o aprendi de homem algum, mas mediante revelação de Jesus Cristo.
\par 13 Porque ouvistes qual foi o meu proceder outrora no judaísmo, como sobremaneira perseguia eu a igreja de Deus e a devastava.
\par 14 E, na minha nação, quanto ao judaísmo, avantajava-me a muitos da minha idade, sendo extremamente zeloso das tradições de meus pais.
\par 15 Quando, porém, ao que me separou antes de eu nascer e me chamou pela sua graça, aprouve
\par 16 revelar seu Filho em mim, para que eu o pregasse entre os gentios, sem detença, não consultei carne e sangue,
\par 17 nem subi a Jerusalém para os que já eram apóstolos antes de mim, mas parti para as regiões da Arábia e voltei, outra vez, para Damasco.
\par 18 Decorridos três anos, então, subi a Jerusalém para avistar-me com Cefas e permaneci com ele quinze dias;
\par 19 e não vi outro dos apóstolos, senão Tiago, o irmão do Senhor.
\par 20 Ora, acerca do que vos escrevo, eis que diante de Deus testifico que não minto.
\par 21 Depois, fui para as regiões da Síria e da Cilícia.
\par 22 E não era conhecido de vista das igrejas da Judéia, que estavam em Cristo.
\par 23 Ouviam somente dizer: Aquele que, antes, nos perseguia, agora, prega a fé que, outrora, procurava destruir.
\par 24 E glorificavam a Deus a meu respeito.

\chapter{2}

\par 1 Catorze anos depois, subi outra vez a Jerusalém com Barnabé, levando também a Tito.
\par 2 Subi em obediência a uma revelação; e lhes expus o evangelho que prego entre os gentios, mas em particular aos que pareciam de maior influência, para, de algum modo, não correr ou ter corrido em vão.
\par 3 Contudo, nem mesmo Tito, que estava comigo, sendo grego, foi constrangido a circuncidar-se.
\par 4 E isto por causa dos falsos irmãos que se entremeteram com o fim de espreitar a nossa liberdade que temos em Cristo Jesus e reduzir-nos à escravidão;
\par 5 aos quais nem ainda por uma hora nos submetemos, para que a verdade do evangelho permanecesse entre vós.
\par 6 E, quanto àqueles que pareciam ser de maior influência (quais tenham sido, outrora, não me interessa; Deus não aceita a aparência do homem), esses, digo, que me pareciam ser alguma coisa nada me acrescentaram;
\par 7 antes, pelo contrário, quando viram que o evangelho da incircuncisão me fora confiado, como a Pedro o da circuncisão
\par 8 (pois aquele que operou eficazmente em Pedro para o apostolado da circuncisão também operou eficazmente em mim para com os gentios)
\par 9 e, quando conheceram a graça que me foi dada, Tiago, Cefas e João, que eram reputados colunas, me estenderam, a mim e a Barnabé, a destra de comunhão, a fim de que nós fôssemos para os gentios, e eles, para a circuncisão;
\par 10 recomendando-nos somente que nos lembrássemos dos pobres, o que também me esforcei por fazer.
\par 11 Quando, porém, Cefas veio a Antioquia, resisti-lhe face a face, porque se tornara repreensível.
\par 12 Com efeito, antes de chegarem alguns da parte de Tiago, comia com os gentios; quando, porém, chegaram, afastou-se e, por fim, veio a apartar-se, temendo os da circuncisão.
\par 13 E também os demais judeus dissimularam com ele, a ponto de o próprio Barnabé ter-se deixado levar pela dissimulação deles.
\par 14 Quando, porém, vi que não procediam corretamente segundo a verdade do evangelho, disse a Cefas, na presença de todos: se, sendo tu judeu, vives como gentio e não como judeu, por que obrigas os gentios a viverem como judeus?
\par 15 Nós, judeus por natureza e não pecadores dentre os gentios,
\par 16 sabendo, contudo, que o homem não é justificado por obras da lei, e sim mediante a fé em Cristo Jesus, também temos crido em Cristo Jesus, para que fôssemos justificados pela fé em Cristo e não por obras da lei, pois, por obras da lei, ninguém será justificado.
\par 17 Mas se, procurando ser justificados em Cristo, fomos nós mesmos também achados pecadores, dar-se-á o caso de ser Cristo ministro do pecado? Certo que não!
\par 18 Porque, se torno a edificar aquilo que destruí, a mim mesmo me constituo transgressor.
\par 19 Porque eu, mediante a própria lei, morri para a lei, a fim de viver para Deus. Estou crucificado com Cristo;
\par 20 logo, já não sou eu quem vive, mas Cristo vive em mim; e esse viver que, agora, tenho na carne, vivo pela fé no Filho de Deus, que me amou e a si mesmo se entregou por mim.
\par 21 Não anulo a graça de Deus; pois, se a justiça é mediante a lei, segue-se que morreu Cristo em vão.

\chapter{3}

\par 1 Ó gálatas insensatos! Quem vos fascinou a vós outros, ante cujos olhos foi Jesus Cristo exposto como crucificado?
\par 2 Quero apenas saber isto de vós: recebestes o Espírito pelas obras da lei ou pela pregação da fé?
\par 3 Sois assim insensatos que, tendo começado no Espírito, estejais, agora, vos aperfeiçoando na carne?
\par 4 Terá sido em vão que tantas coisas sofrestes? Se, na verdade, foram em vão.
\par 5 Aquele, pois, que vos concede o Espírito e que opera milagres entre vós, porventura, o faz pelas obras da lei ou pela pregação da fé?
\par 6 É o caso de Abraão, que creu em Deus, e isso lhe foi imputado para justiça.
\par 7 Sabei, pois, que os da fé é que são filhos de Abraão.
\par 8 Ora, tendo a Escritura previsto que Deus justificaria pela fé os gentios, preanunciou o evangelho a Abraão: Em ti, serão abençoados todos os povos.
\par 9 De modo que os da fé são abençoados com o crente Abraão.
\par 10 Todos quantos, pois, são das obras da lei estão debaixo de maldição; porque está escrito: Maldito todo aquele que não permanece em todas as coisas escritas no Livro da lei, para praticá-las.
\par 11 E é evidente que, pela lei, ninguém é justificado diante de Deus, porque o justo viverá pela fé.
\par 12 Ora, a lei não procede de fé, mas: Aquele que observar os seus preceitos por eles viverá.
\par 13 Cristo nos resgatou da maldição da lei, fazendo-se ele próprio maldição em nosso lugar (porque está escrito: Maldito todo aquele que for pendurado em madeiro),
\par 14 para que a bênção de Abraão chegasse aos gentios, em Jesus Cristo, a fim de que recebêssemos, pela fé, o Espírito prometido.
\par 15 Irmãos, falo como homem. Ainda que uma aliança seja meramente humana, uma vez ratificada, ninguém a revoga ou lhe acrescenta alguma coisa.
\par 16 Ora, as promessas foram feitas a Abraão e ao seu descendente. Não diz: E aos descendentes, como se falando de muitos, porém como de um só: E ao teu descendente, que é Cristo.
\par 17 E digo isto: uma aliança já anteriormente confirmada por Deus, a lei, que veio quatrocentos e trinta anos depois, não a pode ab-rogar, de forma que venha a desfazer a promessa.
\par 18 Porque, se a herança provém de lei, já não decorre de promessa; mas foi pela promessa que Deus a concedeu gratuitamente a Abraão.
\par 19 Qual, pois, a razão de ser da lei? Foi adicionada por causa das transgressões, até que viesse o descendente a quem se fez a promessa, e foi promulgada por meio de anjos, pela mão de um mediador.
\par 20 Ora, o mediador não é de um, mas Deus é um.
\par 21 É, porventura, a lei contrária às promessas de Deus? De modo nenhum! Porque, se fosse promulgada uma lei que pudesse dar vida, a justiça, na verdade, seria procedente de lei.
\par 22 Mas a Escritura encerrou tudo sob o pecado, para que, mediante a fé em Jesus Cristo, fosse a promessa concedida aos que crêem.
\par 23 Mas, antes que viesse a fé, estávamos sob a tutela da lei e nela encerrados, para essa fé que, de futuro, haveria de revelar-se.
\par 24 De maneira que a lei nos serviu de aio para nos conduzir a Cristo, a fim de que fôssemos justificados por fé.
\par 25 Mas, tendo vindo a fé, já não permanecemos subordinados ao aio.
\par 26 Pois todos vós sois filhos de Deus mediante a fé em Cristo Jesus;
\par 27 porque todos quantos fostes batizados em Cristo de Cristo vos revestistes.
\par 28 Dessarte, não pode haver judeu nem grego; nem escravo nem liberto; nem homem nem mulher; porque todos vós sois um em Cristo Jesus.
\par 29 E, se sois de Cristo, também sois descendentes de Abraão e herdeiros segundo a promessa.

\chapter{4}

\par 1 Digo, pois, que, durante o tempo em que o herdeiro é menor, em nada difere de escravo, posto que é ele senhor de tudo.
\par 2 Mas está sob tutores e curadores até ao tempo predeterminado pelo pai.
\par 3 Assim, também nós, quando éramos menores, estávamos servilmente sujeitos aos rudimentos do mundo;
\par 4 vindo, porém, a plenitude do tempo, Deus enviou seu Filho, nascido de mulher, nascido sob a lei,
\par 5 para resgatar os que estavam sob a lei, a fim de que recebêssemos a adoção de filhos.
\par 6 E, porque vós sois filhos, enviou Deus ao nosso coração o Espírito de seu Filho, que clama: Aba, Pai!
\par 7 De sorte que já não és escravo, porém filho; e, sendo filho, também herdeiro por Deus.
\par 8 Outrora, porém, não conhecendo a Deus, servíeis a deuses que, por natureza, não o são;
\par 9 mas agora que conheceis a Deus ou, antes, sendo conhecidos por Deus, como estais voltando, outra vez, aos rudimentos fracos e pobres, aos quais, de novo, quereis ainda escravizar-vos?
\par 10 Guardais dias, e meses, e tempos, e anos.
\par 11 Receio de vós tenha eu trabalhado em vão para convosco.
\par 12 Sede qual eu sou; pois também eu sou como vós. Irmãos, assim vos suplico. Em nada me ofendestes.
\par 13 E vós sabeis que vos preguei o evangelho a primeira vez por causa de uma enfermidade física.
\par 14 E, posto que a minha enfermidade na carne vos foi uma tentação, contudo, não me revelastes desprezo nem desgosto; antes, me recebestes como anjo de Deus, como o próprio Cristo Jesus.
\par 15 Que é feito, pois, da vossa exultação? Pois vos dou testemunho de que, se possível fora, teríeis arrancado os próprios olhos para mos dar.
\par 16 Tornei-me, porventura, vosso inimigo, por vos dizer a verdade?
\par 17 Os que vos obsequiam não o fazem sinceramente, mas querem afastar-vos de mim, para que o vosso zelo seja em favor deles.
\par 18 É bom ser sempre zeloso pelo bem e não apenas quando estou presente convosco,
\par 19 meus filhos, por quem, de novo, sofro as dores de parto, até ser Cristo formado em vós;
\par 20 pudera eu estar presente, agora, convosco e falar-vos em outro tom de voz; porque me vejo perplexo a vosso respeito.
\par 21 Dizei-me vós, os que quereis estar sob a lei: acaso, não ouvis a lei?
\par 22 Pois está escrito que Abraão teve dois filhos, um da mulher escrava e outro da livre.
\par 23 Mas o da escrava nasceu segundo a carne; o da livre, mediante a promessa.
\par 24 Estas coisas são alegóricas; porque estas mulheres são duas alianças; uma, na verdade, se refere ao monte Sinai, que gera para escravidão; esta é Agar.
\par 25 Ora, Agar é o monte Sinai, na Arábia, e corresponde à Jerusalém atual, que está em escravidão com seus filhos.
\par 26 Mas a Jerusalém lá de cima é livre, a qual é nossa mãe;
\par 27 porque está escrito: Alegra-te, ó estéril, que não dás à luz, exulta e clama, tu que não estás de parto; porque são mais numerosos os filhos da abandonada que os da que tem marido.
\par 28 Vós, porém, irmãos, sois filhos da promessa, como Isaque.
\par 29 Como, porém, outrora, o que nascera segundo a carne perseguia ao que nasceu segundo o Espírito, assim também agora.
\par 30 Contudo, que diz a Escritura? Lança fora a escrava e seu filho, porque de modo algum o filho da escrava será herdeiro com o filho da livre.
\par 31 E, assim, irmãos, somos filhos não da escrava, e sim da livre.

\chapter{5}

\par 1 Para a liberdade foi que Cristo nos libertou. Permanecei, pois, firmes e não vos submetais, de novo, a jugo de escravidão.
\par 2 Eu, Paulo, vos digo que, se vos deixardes circuncidar, Cristo de nada vos aproveitará.
\par 3 De novo, testifico a todo homem que se deixa circuncidar que está obrigado a guardar toda a lei.
\par 4 De Cristo vos desligastes, vós que procurais justificar-vos na lei; da graça decaístes.
\par 5 Porque nós, pelo Espírito, aguardamos a esperança da justiça que provém da fé.
\par 6 Porque, em Cristo Jesus, nem a circuncisão, nem a incircuncisão têm valor algum, mas a fé que atua pelo amor.
\par 7 Vós corríeis bem; quem vos impediu de continuardes a obedecer à verdade?
\par 8 Esta persuasão não vem daquele que vos chama.
\par 9 Um pouco de fermento leveda toda a massa.
\par 10 Confio de vós, no Senhor, que não alimentareis nenhum outro sentimento; mas aquele que vos perturba, seja ele quem for, sofrerá a condenação.
\par 11 Eu, porém, irmãos, se ainda prego a circuncisão, por que continuo sendo perseguido? Logo, está desfeito o escândalo da cruz.
\par 12 Tomara até se mutilassem os que vos incitam à rebeldia.
\par 13 Porque vós, irmãos, fostes chamados à liberdade; porém não useis da liberdade para dar ocasião à carne; sede, antes, servos uns dos outros, pelo amor.
\par 14 Porque toda a lei se cumpre em um só preceito, a saber: Amarás o teu próximo como a ti mesmo.
\par 15 Se vós, porém, vos mordeis e devorais uns aos outros, vede que não sejais mutuamente destruídos.
\par 16 Digo, porém: andai no Espírito e jamais satisfareis à concupiscência da carne.
\par 17 Porque a carne milita contra o Espírito, e o Espírito, contra a carne, porque são opostos entre si; para que não façais o que, porventura, seja do vosso querer.
\par 18 Mas, se sois guiados pelo Espírito, não estais sob a lei.
\par 19 Ora, as obras da carne são conhecidas e são: prostituição, impureza, lascívia,
\par 20 idolatria, feitiçarias, inimizades, porfias, ciúmes, iras, discórdias, dissensões, facções,
\par 21 invejas, bebedices, glutonarias e coisas semelhantes a estas, a respeito das quais eu vos declaro, como já, outrora, vos preveni, que não herdarão o reino de Deus os que tais coisas praticam.
\par 22 Mas o fruto do Espírito é: amor, alegria, paz, longanimidade, benignidade, bondade, fidelidade,
\par 23 mansidão, domínio próprio. Contra estas coisas não há lei.
\par 24 E os que são de Cristo Jesus crucificaram a carne, com as suas paixões e concupiscências.
\par 25 Se vivemos no Espírito, andemos também no Espírito.
\par 26 Não nos deixemos possuir de vanglória, provocando uns aos outros, tendo inveja uns dos outros.

\chapter{6}

\par 1 Irmãos, se alguém for surpreendido nalguma falta, vós, que sois espirituais, corrigi-o com espírito de brandura; e guarda-te para que não sejas também tentado.
\par 2 Levai as cargas uns dos outros e, assim, cumprireis a lei de Cristo.
\par 3 Porque, se alguém julga ser alguma coisa, não sendo nada, a si mesmo se engana.
\par 4 Mas prove cada um o seu labor e, então, terá motivo de gloriar-se unicamente em si e não em outro.
\par 5 Porque cada um levará o seu próprio fardo.
\par 6 Mas aquele que está sendo instruído na palavra faça participante de todas as coisas boas aquele que o instrui.
\par 7 Não vos enganeis: de Deus não se zomba; pois aquilo que o homem semear, isso também ceifará.
\par 8 Porque o que semeia para a sua própria carne da carne colherá corrupção; mas o que semeia para o Espírito do Espírito colherá vida eterna.
\par 9 E não nos cansemos de fazer o bem, porque a seu tempo ceifaremos, se não desfalecermos.
\par 10 Por isso, enquanto tivermos oportunidade, façamos o bem a todos, mas principalmente aos da família da fé.
\par 11 Vede com que letras grandes vos escrevi de meu próprio punho.
\par 12 Todos os que querem ostentar-se na carne, esses vos constrangem a vos circuncidardes, somente para não serem perseguidos por causa da cruz de Cristo.
\par 13 Pois nem mesmo aqueles que se deixam circuncidar guardam a lei; antes, querem que vos circuncideis, para se gloriarem na vossa carne.
\par 14 Mas longe esteja de mim gloriar-me, senão na cruz de nosso Senhor Jesus Cristo, pela qual o mundo está crucificado para mim, e eu, para o mundo.
\par 15 Pois nem a circuncisão é coisa alguma, nem a incircuncisão, mas o ser nova criatura.
\par 16 E, a todos quantos andarem de conformidade com esta regra, paz e misericórdia sejam sobre eles e sobre o Israel de Deus.
\par 17 Quanto ao mais, ninguém me moleste; porque eu trago no corpo as marcas de Jesus.
\par 18 A graça de nosso Senhor Jesus Cristo seja, irmãos, com o vosso espírito. Amém!


\end{document}