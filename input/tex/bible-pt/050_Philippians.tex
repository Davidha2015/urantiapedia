\begin{document}

\title{Filipenses}


\chapter{1}

\par 1 Paulo e Timóteo, servos de Cristo Jesus, a todos os santos em Cristo Jesus, inclusive bispos e diáconos que vivem em Filipos,
\par 2 graça e paz a vós outros, da parte de Deus, nosso Pai, e do Senhor Jesus Cristo.
\par 3 Dou graças ao meu Deus por tudo que recordo de vós,
\par 4 fazendo sempre, com alegria, súplicas por todos vós, em todas as minhas orações,
\par 5 pela vossa cooperação no evangelho, desde o primeiro dia até agora.
\par 6 Estou plenamente certo de que aquele que começou boa obra em vós há de completá-la até ao Dia de Cristo Jesus.
\par 7 Aliás, é justo que eu assim pense de todos vós, porque vos trago no coração, seja nas minhas algemas, seja na defesa e confirmação do evangelho, pois todos sois participantes da graça comigo.
\par 8 Pois minha testemunha é Deus, da saudade que tenho de todos vós, na terna misericórdia de Cristo Jesus.
\par 9 E também faço esta oração: que o vosso amor aumente mais e mais em pleno conhecimento e toda a percepção,
\par 10 para aprovardes as coisas excelentes e serdes sinceros e inculpáveis para o Dia de Cristo,
\par 11 cheios do fruto de justiça, o qual é mediante Jesus Cristo, para a glória e louvor de Deus.
\par 12 Quero ainda, irmãos, cientificar-vos de que as coisas que me aconteceram têm, antes, contribuído para o progresso do evangelho;
\par 13 de maneira que as minhas cadeias, em Cristo, se tornaram conhecidas de toda a guarda pretoriana e de todos os demais;
\par 14 e a maioria dos irmãos, estimulados no Senhor por minhas algemas, ousam falar com mais desassombro a palavra de Deus.
\par 15 Alguns, efetivamente, proclamam a Cristo por inveja e porfia; outros, porém, o fazem de boa vontade;
\par 16 estes, por amor, sabendo que estou incumbido da defesa do evangelho;
\par 17 aqueles, contudo, pregam a Cristo, por discórdia, insinceramente, julgando suscitar tribulação às minhas cadeias.
\par 18 Todavia, que importa? Uma vez que Cristo, de qualquer modo, está sendo pregado, quer por pretexto, quer por verdade, também com isto me regozijo, sim, sempre me regozijarei.
\par 19 Porque estou certo de que isto mesmo, pela vossa súplica e pela provisão do Espírito de Jesus Cristo, me redundará em libertação,
\par 20 segundo a minha ardente expectativa e esperança de que em nada serei envergonhado; antes, com toda a ousadia, como sempre, também agora, será Cristo engrandecido no meu corpo, quer pela vida, quer pela morte.
\par 21 Porquanto, para mim, o viver é Cristo, e o morrer é lucro.
\par 22 Entretanto, se o viver na carne traz fruto para o meu trabalho, já não sei o que hei de escolher.
\par 23 Ora, de um e outro lado, estou constrangido, tendo o desejo de partir e estar com Cristo, o que é incomparavelmente melhor.
\par 24 Mas, por vossa causa, é mais necessário permanecer na carne.
\par 25 E, convencido disto, estou certo de que ficarei e permanecerei com todos vós, para o vosso progresso e gozo da fé,
\par 26 a fim de que aumente, quanto a mim, o motivo de vos gloriardes em Cristo Jesus, pela minha presença, de novo, convosco.
\par 27 Vivei, acima de tudo, por modo digno do evangelho de Cristo, para que, ou indo ver-vos ou estando ausente, ouça, no tocante a vós outros, que estais firmes em um só espírito, como uma só alma, lutando juntos pela fé evangélica;
\par 28 e que em nada estais intimidados pelos adversários. Pois o que é para eles prova evidente de perdição é, para vós outros, de salvação, e isto da parte de Deus.
\par 29 Porque vos foi concedida a graça de padecerdes por Cristo e não somente de crerdes nele,
\par 30 pois tendes o mesmo combate que vistes em mim e, ainda agora, ouvis que é o meu.

\chapter{2}

\par 1 Se há, pois, alguma exortação em Cristo, alguma consolação de amor, alguma comunhão do Espírito, se há entranhados afetos e misericórdias,
\par 2 completai a minha alegria, de modo que penseis a mesma coisa, tenhais o mesmo amor, sejais unidos de alma, tendo o mesmo sentimento.
\par 3 Nada façais por partidarismo ou vanglória, mas por humildade, considerando cada um os outros superiores a si mesmo.
\par 4 Não tenha cada um em vista o que é propriamente seu, senão também cada qual o que é dos outros.
\par 5 Tende em vós o mesmo sentimento que houve também em Cristo Jesus,
\par 6 pois ele, subsistindo em forma de Deus, não julgou como usurpação o ser igual a Deus;
\par 7 antes, a si mesmo se esvaziou, assumindo a forma de servo, tornando-se em semelhança de homens; e, reconhecido em figura humana,
\par 8 a si mesmo se humilhou, tornando-se obediente até à morte e morte de cruz.
\par 9 Pelo que também Deus o exaltou sobremaneira e lhe deu o nome que está acima de todo nome,
\par 10 para que ao nome de Jesus se dobre todo joelho, nos céus, na terra e debaixo da terra,
\par 11 e toda língua confesse que Jesus Cristo é Senhor, para glória de Deus Pai.
\par 12 Assim, pois, amados meus, como sempre obedecestes, não só na minha presença, porém, muito mais agora, na minha ausência, desenvolvei a vossa salvação com temor e tremor;
\par 13 porque Deus é quem efetua em vós tanto o querer como o realizar, segundo a sua boa vontade.
\par 14 Fazei tudo sem murmurações nem contendas,
\par 15 para que vos torneis irrepreensíveis e sinceros, filhos de Deus inculpáveis no meio de uma geração pervertida e corrupta, na qual resplandeceis como luzeiros no mundo,
\par 16 preservando a palavra da vida, para que, no Dia de Cristo, eu me glorie de que não corri em vão, nem me esforcei inutilmente.
\par 17 Entretanto, mesmo que seja eu oferecido por libação sobre o sacrifício e serviço da vossa fé, alegro-me e, com todos vós, me congratulo.
\par 18 Assim, vós também, pela mesma razão, alegrai-vos e congratulai-vos comigo.
\par 19 Espero, porém, no Senhor Jesus, mandar-vos Timóteo, o mais breve possível, a fim de que eu me sinta animado também, tendo conhecimento da vossa situação.
\par 20 Porque a ninguém tenho de igual sentimento que, sinceramente, cuide dos vossos interesses;
\par 21 pois todos eles buscam o que é seu próprio, não o que é de Cristo Jesus.
\par 22 E conheceis o seu caráter provado, pois serviu ao evangelho, junto comigo, como filho ao pai.
\par 23 Este, com efeito, é quem espero enviar, tão logo tenha eu visto a minha situação.
\par 24 E estou persuadido no Senhor de que também eu mesmo, brevemente, irei.
\par 25 Julguei, todavia, necessário mandar até vós Epafrodito, por um lado, meu irmão, cooperador e companheiro de lutas; e, por outro, vosso mensageiro e vosso auxiliar nas minhas necessidades;
\par 26 visto que ele tinha saudade de todos vós e estava angustiado porque ouvistes que adoeceu.
\par 27 Com efeito, adoeceu mortalmente; Deus, porém, se compadeceu dele e não somente dele, mas também de mim, para que eu não tivesse tristeza sobre tristeza.
\par 28 Por isso, tanto mais me apresso em mandá-lo, para que, vendo-o novamente, vos alegreis, e eu tenha menos tristeza.
\par 29 Recebei-o, pois, no Senhor, com toda a alegria, e honrai sempre a homens como esse;
\par 30 visto que, por causa da obra de Cristo, chegou ele às portas da morte e se dispôs a dar a própria vida, para suprir a vossa carência de socorro para comigo.

\chapter{3}

\par 1 Quanto ao mais, irmãos meus, alegrai-vos no Senhor. A mim, não me desgosta e é segurança para vós outros que eu escreva as mesmas coisas.
\par 2 Acautelai-vos dos cães! Acautelai-vos dos maus obreiros! Acautelai-vos da falsa circuncisão!
\par 3 Porque nós é que somos a circuncisão, nós que adoramos a Deus no Espírito, e nos gloriamos em Cristo Jesus, e não confiamos na carne.
\par 4 Bem que eu poderia confiar também na carne. Se qualquer outro pensa que pode confiar na carne, eu ainda mais:
\par 5 circuncidado ao oitavo dia, da linhagem de Israel, da tribo de Benjamim, hebreu de hebreus; quanto à lei, fariseu,
\par 6 quanto ao zelo, perseguidor da igreja; quanto à justiça que há na lei, irrepreensível.
\par 7 Mas o que, para mim, era lucro, isto considerei perda por causa de Cristo.
\par 8 Sim, deveras considero tudo como perda, por causa da sublimidade do conhecimento de Cristo Jesus, meu Senhor; por amor do qual perdi todas as coisas e as considero como refugo, para ganhar a Cristo
\par 9 e ser achado nele, não tendo justiça própria, que procede de lei, senão a que é mediante a fé em Cristo, a justiça que procede de Deus, baseada na fé;
\par 10 para o conhecer, e o poder da sua ressurreição, e a comunhão dos seus sofrimentos, conformando-me com ele na sua morte;
\par 11 para, de algum modo, alcançar a ressurreição dentre os mortos.
\par 12 Não que eu o tenha já recebido ou tenha já obtido a perfeição; mas prossigo para conquistar aquilo para o que também fui conquistado por Cristo Jesus.
\par 13 Irmãos, quanto a mim, não julgo havê-lo alcançado; mas uma coisa faço: esquecendo-me das coisas que para trás ficam e avançando para as que diante de mim estão,
\par 14 prossigo para o alvo, para o prêmio da soberana vocação de Deus em Cristo Jesus.
\par 15 Todos, pois, que somos perfeitos, tenhamos este sentimento; e, se, porventura, pensais doutro modo, também isto Deus vos esclarecerá.
\par 16 Todavia, andemos de acordo com o que já alcançamos.
\par 17 Irmãos, sede imitadores meus e observai os que andam segundo o modelo que tendes em nós.
\par 18 Pois muitos andam entre nós, dos quais, repetidas vezes, eu vos dizia e, agora, vos digo, até chorando, que são inimigos da cruz de Cristo.
\par 19 O destino deles é a perdição, o deus deles é o ventre, e a glória deles está na sua infâmia, visto que só se preocupam com as coisas terrenas.
\par 20 Pois a nossa pátria está nos céus, de onde também aguardamos o Salvador, o Senhor Jesus Cristo,
\par 21 o qual transformará o nosso corpo de humilhação, para ser igual ao corpo da sua glória, segundo a eficácia do poder que ele tem de até subordinar a si todas as coisas.

\chapter{4}

\par 1 Portanto, meus irmãos, amados e mui saudosos, minha alegria e coroa, sim, amados, permanecei, deste modo, firmes no Senhor.
\par 2 Rogo a Evódia e rogo a Síntique pensem concordemente, no Senhor.
\par 3 A ti, fiel companheiro de jugo, também peço que as auxilies, pois juntas se esforçaram comigo no evangelho, também com Clemente e com os demais cooperadores meus, cujos nomes se encontram no Livro da Vida.
\par 4 Alegrai-vos sempre no Senhor; outra vez digo: alegrai-vos.
\par 5 Seja a vossa moderação conhecida de todos os homens. Perto está o Senhor.
\par 6 Não andeis ansiosos de coisa alguma; em tudo, porém, sejam conhecidas, diante de Deus, as vossas petições, pela oração e pela súplica, com ações de graças.
\par 7 E a paz de Deus, que excede todo o entendimento, guardará o vosso coração e a vossa mente em Cristo Jesus.
\par 8 Finalmente, irmãos, tudo o que é verdadeiro, tudo o que é respeitável, tudo o que é justo, tudo o que é puro, tudo o que é amável, tudo o que é de boa fama, se alguma virtude há e se algum louvor existe, seja isso o que ocupe o vosso pensamento.
\par 9 O que também aprendestes, e recebestes, e ouvistes, e vistes em mim, isso praticai; e o Deus da paz será convosco.
\par 10 Alegrei-me, sobremaneira, no Senhor porque, agora, uma vez mais, renovastes a meu favor o vosso cuidado; o qual também já tínheis antes, mas vos faltava oportunidade.
\par 11 Digo isto, não por causa da pobreza, porque aprendi a viver contente em toda e qualquer situação.
\par 12 Tanto sei estar humilhado como também ser honrado; de tudo e em todas as circunstâncias, já tenho experiência, tanto de fartura como de fome; assim de abundância como de escassez;
\par 13 tudo posso naquele que me fortalece.
\par 14 Todavia, fizestes bem, associando-vos na minha tribulação.
\par 15 E sabeis também vós, ó filipenses, que, no início do evangelho, quando parti da Macedônia, nenhuma igreja se associou comigo no tocante a dar e receber, senão unicamente vós outros;
\par 16 porque até para Tessalônica mandastes não somente uma vez, mas duas, o bastante para as minhas necessidades.
\par 17 Não que eu procure o donativo, mas o que realmente me interessa é o fruto que aumente o vosso crédito.
\par 18 Recebi tudo e tenho abundância; estou suprido, desde que Epafrodito me passou às mãos o que me veio de vossa parte como aroma suave, como sacrifício aceitável e aprazível a Deus.
\par 19 E o meu Deus, segundo a sua riqueza em glória, há de suprir, em Cristo Jesus, cada uma de vossas necessidades.
\par 20 Ora, a nosso Deus e Pai seja a glória pelos séculos dos séculos. Amém!
\par 21 Saudai cada um dos santos em Cristo Jesus. Os irmãos que se acham comigo vos saúdam.
\par 22 Todos os santos vos saúdam, especialmente os da casa de César.
\par 23 A graça do Senhor Jesus Cristo seja com o vosso espírito.


\end{document}