\begin{document}

\title{Colossenses}


\chapter{1}

\par 1 Paulo, apóstolo de Cristo Jesus, por vontade de Deus, e o irmão Timóteo,
\par 2 aos santos e fiéis irmãos em Cristo que se encontram em Colossos, graça e paz a vós outros, da parte de Deus, nosso Pai.
\par 3 Damos sempre graças a Deus, Pai de nosso Senhor Jesus Cristo, quando oramos por vós,
\par 4 desde que ouvimos da vossa fé em Cristo Jesus e do amor que tendes para com todos os santos;
\par 5 por causa da esperança que vos está preservada nos céus, da qual antes ouvistes pela palavra da verdade do evangelho,
\par 6 que chegou até vós; como também, em todo o mundo, está produzindo fruto e crescendo, tal acontece entre vós, desde o dia em que ouvistes e entendestes a graça de Deus na verdade;
\par 7 segundo fostes instruídos por Epafras, nosso amado conservo e, quanto a vós outros, fiel ministro de Cristo,
\par 8 o qual também nos relatou do vosso amor no Espírito.
\par 9 Por esta razão, também nós, desde o dia em que o ouvimos, não cessamos de orar por vós e de pedir que transbordeis de pleno conhecimento da sua vontade, em toda a sabedoria e entendimento espiritual;
\par 10 a fim de viverdes de modo digno do Senhor, para o seu inteiro agrado, frutificando em toda boa obra e crescendo no pleno conhecimento de Deus;
\par 11 sendo fortalecidos com todo o poder, segundo a força da sua glória, em toda a perseverança e longanimidade; com alegria,
\par 12 dando graças ao Pai, que vos fez idôneos à parte que vos cabe da herança dos santos na luz.
\par 13 Ele nos libertou do império das trevas e nos transportou para o reino do Filho do seu amor,
\par 14 no qual temos a redenção, a remissão dos pecados.
\par 15 Este é a imagem do Deus invisível, o primogênito de toda a criação;
\par 16 pois, nele, foram criadas todas as coisas, nos céus e sobre a terra, as visíveis e as invisíveis, sejam tronos, sejam soberanias, quer principados, quer potestades. Tudo foi criado por meio dele e para ele.
\par 17 Ele é antes de todas as coisas. Nele, tudo subsiste.
\par 18 Ele é a cabeça do corpo, da igreja. Ele é o princípio, o primogênito de entre os mortos, para em todas as coisas ter a primazia,
\par 19 porque aprouve a Deus que, nele, residisse toda a plenitude
\par 20 e que, havendo feito a paz pelo sangue da sua cruz, por meio dele, reconciliasse consigo mesmo todas as coisas, quer sobre a terra, quer nos céus.
\par 21 E a vós outros também que, outrora, éreis estranhos e inimigos no entendimento pelas vossas obras malignas,
\par 22 agora, porém, vos reconciliou no corpo da sua carne, mediante a sua morte, para apresentar-vos perante ele santos, inculpáveis e irrepreensíveis,
\par 23 se é que permaneceis na fé, alicerçados e firmes, não vos deixando afastar da esperança do evangelho que ouvistes e que foi pregado a toda criatura debaixo do céu, e do qual eu, Paulo, me tornei ministro.
\par 24 Agora, me regozijo nos meus sofrimentos por vós; e preencho o que resta das aflições de Cristo, na minha carne, a favor do seu corpo, que é a igreja;
\par 25 da qual me tornei ministro de acordo com a dispensação da parte de Deus, que me foi confiada a vosso favor, para dar pleno cumprimento à palavra de Deus:
\par 26 o mistério que estivera oculto dos séculos e das gerações; agora, todavia, se manifestou aos seus santos;
\par 27 aos quais Deus quis dar a conhecer qual seja a riqueza da glória deste mistério entre os gentios, isto é, Cristo em vós, a esperança da glória;
\par 28 o qual nós anunciamos, advertindo a todo homem e ensinando a todo homem em toda a sabedoria, a fim de que apresentemos todo homem perfeito em Cristo;
\par 29 para isso é que eu também me afadigo, esforçando-me o mais possível, segundo a sua eficácia que opera eficientemente em mim.

\chapter{2}

\par 1 Gostaria, pois, que soubésseis quão grande luta venho mantendo por vós, pelos laodicenses e por quantos não me viram face a face;
\par 2 para que o coração deles seja confortado e vinculado juntamente em amor, e eles tenham toda a riqueza da forte convicção do entendimento, para compreenderem plenamente o mistério de Deus, Cristo,
\par 3 em quem todos os tesouros da sabedoria e do conhecimento estão ocultos.
\par 4 Assim digo para que ninguém vos engane com raciocínios falazes.
\par 5 Pois, embora ausente quanto ao corpo, contudo, em espírito, estou convosco, alegrando-me e verificando a vossa boa ordem e a firmeza da vossa fé em Cristo.
\par 6 Ora, como recebestes Cristo Jesus, o Senhor, assim andai nele,
\par 7 nele radicados, e edificados, e confirmados na fé, tal como fostes instruídos, crescendo em ações de graças.
\par 8 Cuidado que ninguém vos venha a enredar com sua filosofia e vãs sutilezas, conforme a tradição dos homens, conforme os rudimentos do mundo e não segundo Cristo;
\par 9 porquanto, nele, habita, corporalmente, toda a plenitude da Divindade.
\par 10 Também, nele, estais aperfeiçoados. Ele é o cabeça de todo principado e potestade.
\par 11 Nele, também fostes circuncidados, não por intermédio de mãos, mas no despojamento do corpo da carne, que é a circuncisão de Cristo,
\par 12 tendo sido sepultados, juntamente com ele, no batismo, no qual igualmente fostes ressuscitados mediante a fé no poder de Deus que o ressuscitou dentre os mortos.
\par 13 E a vós outros, que estáveis mortos pelas vossas transgressões e pela incircuncisão da vossa carne, vos deu vida juntamente com ele, perdoando todos os nossos delitos;
\par 14 tendo cancelado o escrito de dívida, que era contra nós e que constava de ordenanças, o qual nos era prejudicial, removeu-o inteiramente, encravando-o na cruz;
\par 15 e, despojando os principados e as potestades, publicamente os expôs ao desprezo, triunfando deles na cruz.
\par 16 Ninguém, pois, vos julgue por causa de comida e bebida, ou dia de festa, ou lua nova, ou sábados,
\par 17 porque tudo isso tem sido sombra das coisas que haviam de vir; porém o corpo é de Cristo.
\par 18 Ninguém se faça árbitro contra vós outros, pretextando humildade e culto dos anjos, baseando-se em visões, enfatuado, sem motivo algum, na sua mente carnal,
\par 19 e não retendo a cabeça, da qual todo o corpo, suprido e bem vinculado por suas juntas e ligamentos, cresce o crescimento que procede de Deus.
\par 20 Se morrestes com Cristo para os rudimentos do mundo, por que, como se vivêsseis no mundo, vos sujeitais a ordenanças:
\par 21 não manuseies isto, não proves aquilo, não toques aquiloutro,
\par 22 segundo os preceitos e doutrinas dos homens? Pois que todas estas coisas, com o uso, se destroem.
\par 23 Tais coisas, com efeito, têm aparência de sabedoria, como culto de si mesmo, e de falsa humildade, e de rigor ascético; todavia, não têm valor algum contra a sensualidade.

\chapter{3}

\par 1 Portanto, se fostes ressuscitados juntamente com Cristo, buscai as coisas lá do alto, onde Cristo vive, assentado à direita de Deus.
\par 2 Pensai nas coisas lá do alto, não nas que são aqui da terra;
\par 3 porque morrestes, e a vossa vida está oculta juntamente com Cristo, em Deus.
\par 4 Quando Cristo, que é a nossa vida, se manifestar, então, vós também sereis manifestados com ele, em glória.
\par 5 Fazei, pois, morrer a vossa natureza terrena: prostituição, impureza, paixão lasciva, desejo maligno e a avareza, que é idolatria;
\par 6 por estas coisas é que vem a ira de Deus [sobre os filhos da desobediência].
\par 7 Ora, nessas mesmas coisas andastes vós também, noutro tempo, quando vivíeis nelas.
\par 8 Agora, porém, despojai-vos, igualmente, de tudo isto: ira, indignação, maldade, maledicência, linguagem obscena do vosso falar.
\par 9 Não mintais uns aos outros, uma vez que vos despistes do velho homem com os seus feitos
\par 10 e vos revestistes do novo homem que se refaz para o pleno conhecimento, segundo a imagem daquele que o criou;
\par 11 no qual não pode haver grego nem judeu, circuncisão nem incircuncisão, bárbaro, cita, escravo, livre; porém Cristo é tudo em todos.
\par 12 Revesti-vos, pois, como eleitos de Deus, santos e amados, de ternos afetos de misericórdia, de bondade, de humildade, de mansidão, de longanimidade.
\par 13 Suportai-vos uns aos outros, perdoai-vos mutuamente, caso alguém tenha motivo de queixa contra outrem. Assim como o Senhor vos perdoou, assim também perdoai vós;
\par 14 acima de tudo isto, porém, esteja o amor, que é o vínculo da perfeição.
\par 15 Seja a paz de Cristo o árbitro em vosso coração, à qual, também, fostes chamados em um só corpo; e sede agradecidos.
\par 16 Habite, ricamente, em vós a palavra de Cristo; instruí-vos e aconselhai-vos mutuamente em toda a sabedoria, louvando a Deus, com salmos, e hinos, e cânticos espirituais, com gratidão, em vosso coração.
\par 17 E tudo o que fizerdes, seja em palavra, seja em ação, fazei-o em nome do Senhor Jesus, dando por ele graças a Deus Pai.
\par 18 Esposas, sede submissas ao próprio marido, como convém no Senhor.
\par 19 Maridos, amai vossa esposa e não a trateis com amargura.
\par 20 Filhos, em tudo obedecei a vossos pais; pois fazê-lo é grato diante do Senhor.
\par 21 Pais, não irriteis os vossos filhos, para que não fiquem desanimados.
\par 22 Servos, obedecei em tudo ao vosso senhor segundo a carne, não servindo apenas sob vigilância, visando tão-somente agradar homens, mas em singeleza de coração, temendo ao Senhor.
\par 23 Tudo quanto fizerdes, fazei-o de todo o coração, como para o Senhor e não para homens,
\par 24 cientes de que recebereis do Senhor a recompensa da herança. A Cristo, o Senhor, é que estais servindo;
\par 25 pois aquele que faz injustiça receberá em troco a injustiça feita; e nisto não há acepção de pessoas.

\chapter{4}

\par 1 Senhores, tratai os servos com justiça e com eqüidade, certos de que também vós tendes Senhor no céu.
\par 2 Perseverai na oração, vigiando com ações de graças.
\par 3 Suplicai, ao mesmo tempo, também por nós, para que Deus nos abra porta à palavra, a fim de falarmos do mistério de Cristo, pelo qual também estou algemado;
\par 4 para que eu o manifeste, como devo fazer.
\par 5 Portai-vos com sabedoria para com os que são de fora; aproveitai as oportunidades.
\par 6 A vossa palavra seja sempre agradável, temperada com sal, para saberdes como deveis responder a cada um.
\par 7 Quanto à minha situação, Tíquico, irmão amado, e fiel ministro, e conservo no Senhor, de tudo vos informará.
\par 8 Eu vo-lo envio com o expresso propósito de vos dar conhecimento da nossa situação e de alentar o vosso coração.
\par 9 Em sua companhia, vos envio Onésimo, o fiel e amado irmão, que é do vosso meio. Eles vos farão saber tudo o que por aqui ocorre.
\par 10 Saúda-vos Aristarco, prisioneiro comigo, e Marcos, primo de Barnabé (sobre quem recebestes instruções; se ele for ter convosco, acolhei-o),
\par 11 e Jesus, conhecido por Justo, os quais são os únicos da circuncisão que cooperam pessoalmente comigo pelo reino de Deus. Eles têm sido o meu lenitivo.
\par 12 Saúda-vos Epafras, que é dentre vós, servo de Cristo Jesus, o qual se esforça sobremaneira, continuamente, por vós nas orações, para que vos conserveis perfeitos e plenamente convictos em toda a vontade de Deus.
\par 13 E dele dou testemunho de que muito se preocupa por vós, pelos de Laodicéia e pelos de Hierápolis.
\par 14 Saúda-vos Lucas, o médico amado, e também Demas.
\par 15 Saudai os irmãos de Laodicéia, e Ninfa, e à igreja que ela hospeda em sua casa.
\par 16 E, uma vez lida esta epístola perante vós, providenciai por que seja também lida na igreja dos laodicenses; e a dos de Laodicéia, lede-a igualmente perante vós.
\par 17 Também dizei a Arquipo: atenta para o ministério que recebeste no Senhor, para o cumprires.
\par 18 A saudação é de próprio punho: Paulo. Lembrai-vos das minhas algemas. A graça seja convosco.


\end{document}