\begin{document}

\title{I Timóteo}


\chapter{1}

\par 1 Paulo, apóstolo de Cristo Jesus, pelo mandato de Deus, nosso Salvador, e de Cristo Jesus, nossa esperança,
\par 2 a Timóteo, verdadeiro filho na fé, graça, misericórdia e paz, da parte de Deus Pai e de Cristo Jesus, nosso Senhor.
\par 3 Quando eu estava de viagem, rumo da Macedônia, te roguei permanecesses ainda em Éfeso para admoestares a certas pessoas, a fim de que não ensinem outra doutrina,
\par 4 nem se ocupem com fábulas e genealogias sem fim, que, antes, promovem discussões do que o serviço de Deus, na fé.
\par 5 Ora, o intuito da presente admoestação visa ao amor que procede de coração puro, e de consciência boa, e de fé sem hipocrisia.
\par 6 Desviando-se algumas pessoas destas coisas, perderam-se em loquacidade frívola,
\par 7 pretendendo passar por mestres da lei, não compreendendo, todavia, nem o que dizem, nem os assuntos sobre os quais fazem ousadas asseverações.
\par 8 Sabemos, porém, que a lei é boa, se alguém dela se utiliza de modo legítimo,
\par 9 tendo em vista que não se promulga lei para quem é justo, mas para transgressores e rebeldes, irreverentes e pecadores, ímpios e profanos, parricidas e matricidas, homicidas,
\par 10 impuros, sodomitas, raptores de homens, mentirosos, perjuros e para tudo quanto se opõe à sã doutrina,
\par 11 segundo o evangelho da glória do Deus bendito, do qual fui encarregado.
\par 12 Sou grato para com aquele que me fortaleceu, Cristo Jesus, nosso Senhor, que me considerou fiel, designando-me para o ministério,
\par 13 a mim, que, noutro tempo, era blasfemo, e perseguidor, e insolente. Mas obtive misericórdia, pois o fiz na ignorância, na incredulidade.
\par 14 Transbordou, porém, a graça de nosso Senhor com a fé e o amor que há em Cristo Jesus.
\par 15 Fiel é a palavra e digna de toda aceitação: que Cristo Jesus veio ao mundo para salvar os pecadores, dos quais eu sou o principal.
\par 16 Mas, por esta mesma razão, me foi concedida misericórdia, para que, em mim, o principal, evidenciasse Jesus Cristo a sua completa longanimidade, e servisse eu de modelo a quantos hão de crer nele para a vida eterna.
\par 17 Assim, ao Rei eterno, imortal, invisível, Deus único, honra e glória pelos séculos dos séculos. Amém!
\par 18 Este é o dever de que te encarrego, ó filho Timóteo, segundo as profecias de que antecipadamente foste objeto: combate, firmado nelas, o bom combate,
\par 19 mantendo fé e boa consciência, porquanto alguns, tendo rejeitado a boa consciência, vieram a naufragar na fé.
\par 20 E dentre esses se contam Himeneu e Alexandre, os quais entreguei a Satanás, para serem castigados, a fim de não mais blasfemarem.

\chapter{2}

\par 1 Antes de tudo, pois, exorto que se use a prática de súplicas, orações, intercessões, ações de graças, em favor de todos os homens,
\par 2 em favor dos reis e de todos os que se acham investidos de autoridade, para que vivamos vida tranqüila e mansa, com toda piedade e respeito.
\par 3 Isto é bom e aceitável diante de Deus, nosso Salvador,
\par 4 o qual deseja que todos os homens sejam salvos e cheguem ao pleno conhecimento da verdade.
\par 5 Porquanto há um só Deus e um só Mediador entre Deus e os homens, Cristo Jesus, homem,
\par 6 o qual a si mesmo se deu em resgate por todos: testemunho que se deve prestar em tempos oportunos.
\par 7 Para isto fui designado pregador e apóstolo (afirmo a verdade, não minto), mestre dos gentios na fé e na verdade.
\par 8 Quero, portanto, que os varões orem em todo lugar, levantando mãos santas, sem ira e sem animosidade.
\par 9 Da mesma sorte, que as mulheres, em traje decente, se ataviem com modéstia e bom senso, não com cabeleira frisada e com ouro, ou pérolas, ou vestuário dispendioso,
\par 10 porém com boas obras (como é próprio às mulheres que professam ser piedosas).
\par 11 A mulher aprenda em silêncio, com toda a submissão.
\par 12 E não permito que a mulher ensine, nem exerça autoridade de homem; esteja, porém, em silêncio.
\par 13 Porque, primeiro, foi formado Adão, depois, Eva.
\par 14 E Adão não foi iludido, mas a mulher, sendo enganada, caiu em transgressão.
\par 15 Todavia, será preservada através de sua missão de mãe, se ela permanecer em fé, e amor, e santificação, com bom senso.

\chapter{3}

\par 1 Fiel é a palavra: se alguém aspira ao episcopado, excelente obra almeja.
\par 2 É necessário, portanto, que o bispo seja irrepreensível, esposo de uma só mulher, temperante, sóbrio, modesto, hospitaleiro, apto para ensinar;
\par 3 não dado ao vinho, não violento, porém cordato, inimigo de contendas, não avarento;
\par 4 e que governe bem a própria casa, criando os filhos sob disciplina, com todo o respeito
\par 5 (pois, se alguém não sabe governar a própria casa, como cuidará da igreja de Deus?);
\par 6 não seja neófito, para não suceder que se ensoberbeça e incorra na condenação do diabo.
\par 7 Pelo contrário, é necessário que ele tenha bom testemunho dos de fora, a fim de não cair no opróbrio e no laço do diabo.
\par 8 Semelhantemente, quanto a diáconos, é necessário que sejam respeitáveis, de uma só palavra, não inclinados a muito vinho, não cobiçosos de sórdida ganância,
\par 9 conservando o mistério da fé com a consciência limpa.
\par 10 Também sejam estes primeiramente experimentados; e, se se mostrarem irrepreensíveis, exerçam o diaconato.
\par 11 Da mesma sorte, quanto a mulheres, é necessário que sejam elas respeitáveis, não maldizentes, temperantes e fiéis em tudo.
\par 12 O diácono seja marido de uma só mulher e governe bem seus filhos e a própria casa.
\par 13 Pois os que desempenharem bem o diaconato alcançam para si mesmos justa preeminência e muita intrepidez na fé em Cristo Jesus.
\par 14 Escrevo-te estas coisas, esperando ir ver-te em breve;
\par 15 para que, se eu tardar, fiques ciente de como se deve proceder na casa de Deus, que é a igreja do Deus vivo, coluna e baluarte da verdade.
\par 16 Evidentemente, grande é o mistério da piedade: Aquele que foi manifestado na carne foi justificado em espírito, contemplado por anjos, pregado entre os gentios, crido no mundo, recebido na glória.

\chapter{4}

\par 1 Ora, o Espírito afirma expressamente que, nos últimos tempos, alguns apostatarão da fé, por obedecerem a espíritos enganadores e a ensinos de demônios,
\par 2 pela hipocrisia dos que falam mentiras e que têm cauterizada a própria consciência,
\par 3 que proíbem o casamento e exigem abstinência de alimentos que Deus criou para serem recebidos, com ações de graças, pelos fiéis e por quantos conhecem plenamente a verdade;
\par 4 pois tudo que Deus criou é bom, e, recebido com ações de graças, nada é recusável,
\par 5 porque, pela palavra de Deus e pela oração, é santificado.
\par 6 Expondo estas coisas aos irmãos, serás bom ministro de Cristo Jesus, alimentado com as palavras da fé e da boa doutrina que tens seguido.
\par 7 Mas rejeita as fábulas profanas e de velhas caducas. Exercita-te, pessoalmente, na piedade.
\par 8 Pois o exercício físico para pouco é proveitoso, mas a piedade para tudo é proveitosa, porque tem a promessa da vida que agora é e da que há de ser.
\par 9 Fiel é esta palavra e digna de inteira aceitação.
\par 10 Ora, é para esse fim que labutamos e nos esforçamos sobremodo, porquanto temos posto a nossa esperança no Deus vivo, Salvador de todos os homens, especialmente dos fiéis.
\par 11 Ordena e ensina estas coisas.
\par 12 Ninguém despreze a tua mocidade; pelo contrário, torna-te padrão dos fiéis, na palavra, no procedimento, no amor, na fé, na pureza.
\par 13 Até à minha chegada, aplica-te à leitura, à exortação, ao ensino.
\par 14 Não te faças negligente para com o dom que há em ti, o qual te foi concedido mediante profecia, com a imposição das mãos do presbitério.
\par 15 Medita estas coisas e nelas sê diligente, para que o teu progresso a todos seja manifesto.
\par 16 Tem cuidado de ti mesmo e da doutrina. Continua nestes deveres; porque, fazendo assim, salvarás tanto a ti mesmo como aos teus ouvintes.

\chapter{5}

\par 1 Não repreendas ao homem idoso; antes, exorta-o como a pai; aos moços, como a irmãos;
\par 2 às mulheres idosas, como a mães; às moças, como a irmãs, com toda a pureza.
\par 3 Honra as viúvas verdadeiramente viúvas.
\par 4 Mas, se alguma viúva tem filhos ou netos, que estes aprendam primeiro a exercer piedade para com a própria casa e a recompensar a seus progenitores; pois isto é aceitável diante de Deus.
\par 5 Aquela, porém, que é verdadeiramente viúva e não tem amparo espera em Deus e persevera em súplicas e orações, noite e dia;
\par 6 entretanto, a que se entrega aos prazeres, mesmo viva, está morta.
\par 7 Prescreve, pois, estas coisas, para que sejam irrepreensíveis.
\par 8 Ora, se alguém não tem cuidado dos seus e especialmente dos da própria casa, tem negado a fé e é pior do que o descrente.
\par 9 Não seja inscrita senão viúva que conte ao menos sessenta anos de idade, tenha sido esposa de um só marido,
\par 10 seja recomendada pelo testemunho de boas obras, tenha criado filhos, exercitado hospitalidade, lavado os pés aos santos, socorrido a atribulados, se viveu na prática zelosa de toda boa obra.
\par 11 Mas rejeita viúvas mais novas, porque, quando se tornam levianas contra Cristo, querem casar-se,
\par 12 tornando-se condenáveis por anularem o seu primeiro compromisso.
\par 13 Além do mais, aprendem também a viver ociosas, andando de casa em casa; e não somente ociosas, mas ainda tagarelas e intrigantes, falando o que não devem.
\par 14 Quero, portanto, que as viúvas mais novas se casem, criem filhos, sejam boas donas de casa e não dêem ao adversário ocasião favorável de maledicência.
\par 15 Pois, com efeito, já algumas se desviaram, seguindo a Satanás.
\par 16 Se alguma crente tem viúvas em sua família, socorra-as, e não fique sobrecarregada a igreja, para que esta possa socorrer as que são verdadeiramente viúvas.
\par 17 Devem ser considerados merecedores de dobrados honorários os presbíteros que presidem bem, com especialidade os que se afadigam na palavra e no ensino.
\par 18 Pois a Escritura declara: Não amordaces o boi, quando pisa o trigo. E ainda: O trabalhador é digno do seu salário.
\par 19 Não aceites denúncia contra presbítero, senão exclusivamente sob o depoimento de duas ou três testemunhas.
\par 20 Quanto aos que vivem no pecado, repreende-os na presença de todos, para que também os demais temam.
\par 21 Conjuro-te, perante Deus, e Cristo Jesus, e os anjos eleitos, que guardes estes conselhos, sem prevenção, nada fazendo com parcialidade.
\par 22 A ninguém imponhas precipitadamente as mãos. Não te tornes cúmplice de pecados de outrem. Conserva-te a ti mesmo puro.
\par 23 Não continues a beber somente água; usa um pouco de vinho, por causa do teu estômago e das tuas freqüentes enfermidades.
\par 24 Os pecados de alguns homens são notórios e levam a juízo, ao passo que os de outros só mais tarde se manifestam.
\par 25 Da mesma sorte também as boas obras, antecipadamente, se evidenciam e, quando assim não seja, não podem ocultar-se.

\chapter{6}

\par 1 Todos os servos que estão debaixo de jugo considerem dignos de toda honra o próprio senhor, para que o nome de Deus e a doutrina não sejam blasfemados.
\par 2 Também os que têm senhor fiel não o tratem com desrespeito, porque é irmão; pelo contrário, trabalhem ainda mais, pois ele, que partilha do seu bom serviço, é crente e amado. Ensina e recomenda estas coisas.
\par 3 Se alguém ensina outra doutrina e não concorda com as sãs palavras de nosso Senhor Jesus Cristo e com o ensino segundo a piedade,
\par 4 é enfatuado, nada entende, mas tem mania por questões e contendas de palavras, de que nascem inveja, provocação, difamações, suspeitas malignas,
\par 5 altercações sem fim, por homens cuja mente é pervertida e privados da verdade, supondo que a piedade é fonte de lucro.
\par 6 De fato, grande fonte de lucro é a piedade com o contentamento.
\par 7 Porque nada temos trazido para o mundo, nem coisa alguma podemos levar dele.
\par 8 Tendo sustento e com que nos vestir, estejamos contentes.
\par 9 Ora, os que querem ficar ricos caem em tentação, e cilada, e em muitas concupiscências insensatas e perniciosas, as quais afogam os homens na ruína e perdição.
\par 10 Porque o amor do dinheiro é raiz de todos os males; e alguns, nessa cobiça, se desviaram da fé e a si mesmos se atormentaram com muitas dores.
\par 11 Tu, porém, ó homem de Deus, foge destas coisas; antes, segue a justiça, a piedade, a fé, o amor, a constância, a mansidão.
\par 12 Combate o bom combate da fé. Toma posse da vida eterna, para a qual também foste chamado e de que fizeste a boa confissão perante muitas testemunhas.
\par 13 Exorto-te, perante Deus, que preserva a vida de todas as coisas, e perante Cristo Jesus, que, diante de Pôncio Pilatos, fez a boa confissão,
\par 14 que guardes o mandato imaculado, irrepreensível, até à manifestação de nosso Senhor Jesus Cristo;
\par 15 a qual, em suas épocas determinadas, há de ser revelada pelo bendito e único Soberano, o Rei dos reis e Senhor dos senhores;
\par 16 o único que possui imortalidade, que habita em luz inacessível, a quem homem algum jamais viu, nem é capaz de ver. A ele honra e poder eterno. Amém!
\par 17 Exorta aos ricos do presente século que não sejam orgulhosos, nem depositem a sua esperança na instabilidade da riqueza, mas em Deus, que tudo nos proporciona ricamente para nosso aprazimento;
\par 18 que pratiquem o bem, sejam ricos de boas obras, generosos em dar e prontos a repartir;
\par 19 que acumulem para si mesmos tesouros, sólido fundamento para o futuro, a fim de se apoderarem da verdadeira vida.
\par 20 E tu, ó Timóteo, guarda o que te foi confiado, evitando os falatórios inúteis e profanos e as contradições do saber, como falsamente lhe chamam,
\par 21 pois alguns, professando-o, se desviaram da fé. A graça seja convosco.


\end{document}