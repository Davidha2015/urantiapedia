\begin{document}

\title{II Timóteo}


\chapter{1}

\par 1 Paulo, apóstolo de Cristo Jesus, pela vontade de Deus, de conformidade com a promessa da vida que está em Cristo Jesus,
\par 2 ao amado filho Timóteo, graça, misericórdia e paz, da parte de Deus Pai e de Cristo Jesus, nosso Senhor.
\par 3 Dou graças a Deus, a quem, desde os meus antepassados, sirvo com consciência pura, porque, sem cessar, me lembro de ti nas minhas orações, noite e dia.
\par 4 Lembrado das tuas lágrimas, estou ansioso por ver-te, para que eu transborde de alegria
\par 5 pela recordação que guardo de tua fé sem fingimento, a mesma que, primeiramente, habitou em tua avó Lóide e em tua mãe Eunice, e estou certo de que também, em ti.
\par 6 Por esta razão, pois, te admoesto que reavives o dom de Deus que há em ti pela imposição das minhas mãos.
\par 7 Porque Deus não nos tem dado espírito de covardia, mas de poder, de amor e de moderação.
\par 8 Não te envergonhes, portanto, do testemunho de nosso Senhor, nem do seu encarcerado, que sou eu; pelo contrário, participa comigo dos sofrimentos, a favor do evangelho, segundo o poder de Deus,
\par 9 que nos salvou e nos chamou com santa vocação; não segundo as nossas obras, mas conforme a sua própria determinação e graça que nos foi dada em Cristo Jesus, antes dos tempos eternos,
\par 10 e manifestada, agora, pelo aparecimento de nosso Salvador Cristo Jesus, o qual não só destruiu a morte, como trouxe à luz a vida e a imortalidade, mediante o evangelho,
\par 11 para o qual eu fui designado pregador, apóstolo e mestre
\par 12 e, por isso, estou sofrendo estas coisas; todavia, não me envergonho, porque sei em quem tenho crido e estou certo de que ele é poderoso para guardar o meu depósito até aquele Dia.
\par 13 Mantém o padrão das sãs palavras que de mim ouviste com fé e com o amor que está em Cristo Jesus.
\par 14 Guarda o bom depósito, mediante o Espírito Santo que habita em nós.
\par 15 Estás ciente de que todos os da Ásia me abandonaram; dentre eles cito Fígelo e Hermógenes.
\par 16 Conceda o Senhor misericórdia à casa de Onesíforo, porque, muitas vezes, me deu ânimo e nunca se envergonhou das minhas algemas;
\par 17 antes, tendo ele chegado a Roma, me procurou solicitamente até me encontrar.
\par 18 O Senhor lhe conceda, naquele Dia, achar misericórdia da parte do Senhor. E tu sabes, melhor do que eu, quantos serviços me prestou ele em Éfeso.

\chapter{2}

\par 1 Tu, pois, filho meu, fortifica-te na graça que está em Cristo Jesus.
\par 2 E o que de minha parte ouviste através de muitas testemunhas, isso mesmo transmite a homens fiéis e também idôneos para instruir a outros.
\par 3 Participa dos meus sofrimentos como bom soldado de Cristo Jesus.
\par 4 Nenhum soldado em serviço se envolve em negócios desta vida, porque o seu objetivo é satisfazer àquele que o arregimentou.
\par 5 Igualmente, o atleta não é coroado se não lutar segundo as normas.
\par 6 O lavrador que trabalha deve ser o primeiro a participar dos frutos.
\par 7 Pondera o que acabo de dizer, porque o Senhor te dará compreensão em todas as coisas.
\par 8 Lembra-te de Jesus Cristo, ressuscitado de entre os mortos, descendente de Davi, segundo o meu evangelho;
\par 9 pelo qual estou sofrendo até algemas, como malfeitor; contudo, a palavra de Deus não está algemada.
\par 10 Por esta razão, tudo suporto por causa dos eleitos, para que também eles obtenham a salvação que está em Cristo Jesus, com eterna glória.
\par 11 Fiel é esta palavra: Se já morremos com ele, também viveremos com ele;
\par 12 se perseveramos, também com ele reinaremos; se o negamos, ele, por sua vez, nos negará;
\par 13 se somos infiéis, ele permanece fiel, pois de maneira nenhuma pode negar-se a si mesmo.
\par 14 Recomenda estas coisas. Dá testemunho solene a todos perante Deus, para que evitem contendas de palavras que para nada aproveitam, exceto para a subversão dos ouvintes.
\par 15 Procura apresentar-te a Deus aprovado, como obreiro que não tem de que se envergonhar, que maneja bem a palavra da verdade.
\par 16 Evita, igualmente, os falatórios inúteis e profanos, pois os que deles usam passarão a impiedade ainda maior.
\par 17 Além disso, a linguagem deles corrói como câncer; entre os quais se incluem Himeneu e Fileto.
\par 18 Estes se desviaram da verdade, asseverando que a ressurreição já se realizou, e estão pervertendo a fé a alguns.
\par 19 Entretanto, o firme fundamento de Deus permanece, tendo este selo: O Senhor conhece os que lhe pertencem. E mais: Aparte-se da injustiça todo aquele que professa o nome do Senhor.
\par 20 Ora, numa grande casa não há somente utensílios de ouro e de prata; há também de madeira e de barro. Alguns, para honra; outros, porém, para desonra.
\par 21 Assim, pois, se alguém a si mesmo se purificar destes erros, será utensílio para honra, santificado e útil ao seu possuidor, estando preparado para toda boa obra.
\par 22 Foge, outrossim, das paixões da mocidade. Segue a justiça, a fé, o amor e a paz com os que, de coração puro, invocam o Senhor.
\par 23 E repele as questões insensatas e absurdas, pois sabes que só engendram contendas.
\par 24 Ora, é necessário que o servo do Senhor não viva a contender, e sim deve ser brando para com todos, apto para instruir, paciente,
\par 25 disciplinando com mansidão os que se opõem, na expectativa de que Deus lhes conceda não só o arrependimento para conhecerem plenamente a verdade,
\par 26 mas também o retorno à sensatez, livrando-se eles dos laços do diabo, tendo sido feitos cativos por ele para cumprirem a sua vontade.

\chapter{3}

\par 1 Sabe, porém, isto: nos últimos dias, sobrevirão tempos difíceis,
\par 2 pois os homens serão egoístas, avarentos, jactanciosos, arrogantes, blasfemadores, desobedientes aos pais, ingratos, irreverentes,
\par 3 desafeiçoados, implacáveis, caluniadores, sem domínio de si, cruéis, inimigos do bem,
\par 4 traidores, atrevidos, enfatuados, mais amigos dos prazeres que amigos de Deus,
\par 5 tendo forma de piedade, negando-lhe, entretanto, o poder. Foge também destes.
\par 6 Pois entre estes se encontram os que penetram sorrateiramente nas casas e conseguem cativar mulherinhas sobrecarregadas de pecados, conduzidas de várias paixões,
\par 7 que aprendem sempre e jamais podem chegar ao conhecimento da verdade.
\par 8 E, do modo por que Janes e Jambres resistiram a Moisés, também estes resistem à verdade. São homens de todo corrompidos na mente, réprobos quanto à fé;
\par 9 eles, todavia, não irão avante; porque a sua insensatez será a todos evidente, como também aconteceu com a daqueles.
\par 10 Tu, porém, tens seguido, de perto, o meu ensino, procedimento, propósito, fé, longanimidade, amor, perseverança,
\par 11 as minhas perseguições e os meus sofrimentos, quais me aconteceram em Antioquia, Icônio e Listra, -- que variadas perseguições tenho suportado! De todas, entretanto, me livrou o Senhor.
\par 12 Ora, todos quantos querem viver piedosamente em Cristo Jesus serão perseguidos.
\par 13 Mas os homens perversos e impostores irão de mal a pior, enganando e sendo enganados.
\par 14 Tu, porém, permanece naquilo que aprendeste e de que foste inteirado, sabendo de quem o aprendeste
\par 15 e que, desde a infância, sabes as sagradas letras, que podem tornar-te sábio para a salvação pela fé em Cristo Jesus.
\par 16 Toda a Escritura é inspirada por Deus e útil para o ensino, para a repreensão, para a correção, para a educação na justiça,
\par 17 a fim de que o homem de Deus seja perfeito e perfeitamente habilitado para toda boa obra.

\chapter{4}

\par 1 Conjuro-te, perante Deus e Cristo Jesus, que há de julgar vivos e mortos, pela sua manifestação e pelo seu reino:
\par 2 prega a palavra, insta, quer seja oportuno, quer não, corrige, repreende, exorta com toda a longanimidade e doutrina.
\par 3 Pois haverá tempo em que não suportarão a sã doutrina; pelo contrário, cercar-se-ão de mestres segundo as suas próprias cobiças, como que sentindo coceira nos ouvidos;
\par 4 e se recusarão a dar ouvidos à verdade, entregando-se às fábulas.
\par 5 Tu, porém, sê sóbrio em todas as coisas, suporta as aflições, faze o trabalho de um evangelista, cumpre cabalmente o teu ministério.
\par 6 Quanto a mim, estou sendo já oferecido por libação, e o tempo da minha partida é chegado.
\par 7 Combati o bom combate, completei a carreira, guardei a fé.
\par 8 Já agora a coroa da justiça me está guardada, a qual o Senhor, reto juiz, me dará naquele Dia; e não somente a mim, mas também a todos quantos amam a sua vinda.
\par 9 Procura vir ter comigo depressa.
\par 10 Porque Demas, tendo amado o presente século, me abandonou e se foi para Tessalônica; Crescente foi para a Galácia, Tito, para a Dalmácia.
\par 11 Somente Lucas está comigo. Toma contigo Marcos e traze-o, pois me é útil para o ministério.
\par 12 Quanto a Tíquico, mandei-o até Éfeso.
\par 13 Quando vieres, traze a capa que deixei em Trôade, em casa de Carpo, bem como os livros, especialmente os pergaminhos.
\par 14 Alexandre, o latoeiro, causou-me muitos males; o Senhor lhe dará a paga segundo as suas obras.
\par 15 Tu, guarda-te também dele, porque resistiu fortemente às nossas palavras.
\par 16 Na minha primeira defesa, ninguém foi a meu favor; antes, todos me abandonaram. Que isto não lhes seja posto em conta!
\par 17 Mas o Senhor me assistiu e me revestiu de forças, para que, por meu intermédio, a pregação fosse plenamente cumprida, e todos os gentios a ouvissem; e fui libertado da boca do leão.
\par 18 O Senhor me livrará também de toda obra maligna e me levará salvo para o seu reino celestial. A ele, glória pelos séculos dos séculos. Amém!
\par 19 Saúda Prisca, e Áqüila, e a casa de Onesíforo.
\par 20 Erasto ficou em Corinto. Quanto a Trófimo, deixei-o doente em Mileto.
\par 21 Apressa-te a vir antes do inverno. Êubulo te envia saudações; o mesmo fazem Prudente, Lino, Cláudia e os irmãos todos.
\par 22 O Senhor seja com o teu espírito. A graça seja convosco.


\end{document}