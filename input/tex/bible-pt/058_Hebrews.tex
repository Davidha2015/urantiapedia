\begin{document}

\title{Hebreus}


\chapter{1}

\par 1 Havendo Deus, outrora, falado, muitas vezes e de muitas maneiras, aos pais, pelos profetas,
\par 2 nestes últimos dias, nos falou pelo Filho, a quem constituiu herdeiro de todas as coisas, pelo qual também fez o universo.
\par 3 Ele, que é o resplendor da glória e a expressão exata do seu Ser, sustentando todas as coisas pela palavra do seu poder, depois de ter feito a purificação dos pecados, assentou-se à direita da Majestade, nas alturas,
\par 4 tendo-se tornado tão superior aos anjos quanto herdou mais excelente nome do que eles.
\par 5 Pois a qual dos anjos disse jamais: Tu és meu Filho, eu hoje te gerei? E outra vez: Eu lhe serei Pai, e ele me será Filho?
\par 6 E, novamente, ao introduzir o Primogênito no mundo, diz: E todos os anjos de Deus o adorem.
\par 7 Ainda, quanto aos anjos, diz: Aquele que a seus anjos faz ventos, e a seus ministros, labareda de fogo;
\par 8 mas acerca do Filho: O teu trono, ó Deus, é para todo o sempre; e: Cetro de eqüidade é o cetro do seu reino.
\par 9 Amaste a justiça e odiaste a iniqüidade; por isso, Deus, o teu Deus, te ungiu com o óleo de alegria como a nenhum dos teus companheiros.
\par 10 Ainda: No princípio, Senhor, lançaste os fundamentos da terra, e os céus são obra das tuas mãos;
\par 11 eles perecerão; tu, porém, permaneces; sim, todos eles envelhecerão qual veste;
\par 12 também, qual manto, os enrolarás, e, como vestes, serão igualmente mudados; tu, porém, és o mesmo, e os teus anos jamais terão fim.
\par 13 Ora, a qual dos anjos jamais disse: Assenta-te à minha direita, até que eu ponha os teus inimigos por estrado dos teus pés?
\par 14 Não são todos eles espíritos ministradores, enviados para serviço a favor dos que hão de herdar a salvação?

\chapter{2}

\par 1 Por esta razão, importa que nos apeguemos, com mais firmeza, às verdades ouvidas, para que delas jamais nos desviemos.
\par 2 Se, pois, se tornou firme a palavra falada por meio de anjos, e toda transgressão ou desobediência recebeu justo castigo,
\par 3 como escaparemos nós, se negligenciarmos tão grande salvação? A qual, tendo sido anunciada inicialmente pelo Senhor, foi-nos depois confirmada pelos que a ouviram;
\par 4 dando Deus testemunho juntamente com eles, por sinais, prodígios e vários milagres e por distribuições do Espírito Santo, segundo a sua vontade.
\par 5 Pois não foi a anjos que sujeitou o mundo que há de vir, sobre o qual estamos falando;
\par 6 antes, alguém, em certo lugar, deu pleno testemunho, dizendo: Que é o homem, que dele te lembres? Ou o filho do homem, que o visites?
\par 7 Fizeste-o, por um pouco, menor que os anjos, de glória e de honra o coroaste [e o constituíste sobre as obras das tuas mãos].
\par 8 Todas as coisas sujeitaste debaixo dos seus pés. Ora, desde que lhe sujeitou todas as coisas, nada deixou fora do seu domínio. Agora, porém, ainda não vemos todas as coisas a ele sujeitas;
\par 9 vemos, todavia, aquele que, por um pouco, tendo sido feito menor que os anjos, Jesus, por causa do sofrimento da morte, foi coroado de glória e de honra, para que, pela graça de Deus, provasse a morte por todo homem.
\par 10 Porque convinha que aquele, por cuja causa e por quem todas as coisas existem, conduzindo muitos filhos à glória, aperfeiçoasse, por meio de sofrimentos, o Autor da salvação deles.
\par 11 Pois, tanto o que santifica como os que são santificados, todos vêm de um só. Por isso, é que ele não se envergonha de lhes chamar irmãos,
\par 12 dizendo: A meus irmãos declararei o teu nome, cantar-te-ei louvores no meio da congregação.
\par 13 E outra vez: Eu porei nele a minha confiança. E ainda: Eis aqui estou eu e os filhos que Deus me deu.
\par 14 Visto, pois, que os filhos têm participação comum de carne e sangue, destes também ele, igualmente, participou, para que, por sua morte, destruísse aquele que tem o poder da morte, a saber, o diabo,
\par 15 e livrasse todos que, pelo pavor da morte, estavam sujeitos à escravidão por toda a vida.
\par 16 Pois ele, evidentemente, não socorre anjos, mas socorre a descendência de Abraão.
\par 17 Por isso mesmo, convinha que, em todas as coisas, se tornasse semelhante aos irmãos, para ser misericordioso e fiel sumo sacerdote nas coisas referentes a Deus e para fazer propiciação pelos pecados do povo.
\par 18 Pois, naquilo que ele mesmo sofreu, tendo sido tentado, é poderoso para socorrer os que são tentados.

\chapter{3}

\par 1 Por isso, santos irmãos, que participais da vocação celestial, considerai atentamente o Apóstolo e Sumo Sacerdote da nossa confissão, Jesus,
\par 2 o qual é fiel àquele que o constituiu, como também o era Moisés em toda a casa de Deus.
\par 3 Jesus, todavia, tem sido considerado digno de tanto maior glória do que Moisés, quanto maior honra do que a casa tem aquele que a estabeleceu.
\par 4 Pois toda casa é estabelecida por alguém, mas aquele que estabeleceu todas as coisas é Deus.
\par 5 E Moisés era fiel, em toda a casa de Deus, como servo, para testemunho das coisas que haviam de ser anunciadas;
\par 6 Cristo, porém, como Filho, em sua casa; a qual casa somos nós, se guardarmos firme, até ao fim, a ousadia e a exultação da esperança.
\par 7 Assim, pois, como diz o Espírito Santo: Hoje, se ouvirdes a sua voz,
\par 8 não endureçais o vosso coração como foi na provocação, no dia da tentação no deserto,
\par 9 onde os vossos pais me tentaram, pondo-me à prova, e viram as minhas obras por quarenta anos.
\par 10 Por isso, me indignei contra essa geração e disse: Estes sempre erram no coração; eles também não conheceram os meus caminhos.
\par 11 Assim, jurei na minha ira: Não entrarão no meu descanso.
\par 12 Tende cuidado, irmãos, jamais aconteça haver em qualquer de vós perverso coração de incredulidade que vos afaste do Deus vivo;
\par 13 pelo contrário, exortai-vos mutuamente cada dia, durante o tempo que se chama Hoje, a fim de que nenhum de vós seja endurecido pelo engano do pecado.
\par 14 Porque nos temos tornado participantes de Cristo, se, de fato, guardarmos firme, até ao fim, a confiança que, desde o princípio, tivemos.
\par 15 Enquanto se diz: Hoje, se ouvirdes a sua voz, não endureçais o vosso coração, como foi na provocação.
\par 16 Ora, quais os que, tendo ouvido, se rebelaram? Não foram, de fato, todos os que saíram do Egito por intermédio de Moisés?
\par 17 E contra quem se indignou por quarenta anos? Não foi contra os que pecaram, cujos cadáveres caíram no deserto?
\par 18 E contra quem jurou que não entrariam no seu descanso, senão contra os que foram desobedientes?
\par 19 Vemos, pois, que não puderam entrar por causa da incredulidade.

\chapter{4}

\par 1 Temamos, portanto, que, sendo-nos deixada a promessa de entrar no descanso de Deus, suceda parecer que algum de vós tenha falhado.
\par 2 Porque também a nós foram anunciadas as boas-novas, como se deu com eles; mas a palavra que ouviram não lhes aproveitou, visto não ter sido acompanhada pela fé naqueles que a ouviram.
\par 3 Nós, porém, que cremos, entramos no descanso, conforme Deus tem dito: Assim, jurei na minha ira: Não entrarão no meu descanso. Embora, certamente, as obras estivessem concluídas desde a fundação do mundo.
\par 4 Porque, em certo lugar, assim disse, no tocante ao sétimo dia: E descansou Deus, no sétimo dia, de todas as obras que fizera.
\par 5 E novamente, no mesmo lugar: Não entrarão no meu descanso.
\par 6 Visto, portanto, que resta entrarem alguns nele e que, por causa da desobediência, não entraram aqueles aos quais anteriormente foram anunciadas as boas-novas,
\par 7 de novo, determina certo dia, Hoje, falando por Davi, muito tempo depois, segundo antes fora declarado: Hoje, se ouvirdes a sua voz, não endureçais o vosso coração.
\par 8 Ora, se Josué lhes houvesse dado descanso, não falaria, posteriormente, a respeito de outro dia.
\par 9 Portanto, resta um repouso para o povo de Deus.
\par 10 Porque aquele que entrou no descanso de Deus, também ele mesmo descansou de suas obras, como Deus das suas.
\par 11 Esforcemo-nos, pois, por entrar naquele descanso, a fim de que ninguém caia, segundo o mesmo exemplo de desobediência.
\par 12 Porque a palavra de Deus é viva, e eficaz, e mais cortante do que qualquer espada de dois gumes, e penetra até ao ponto de dividir alma e espírito, juntas e medulas, e é apta para discernir os pensamentos e propósitos do coração.
\par 13 E não há criatura que não seja manifesta na sua presença; pelo contrário, todas as coisas estão descobertas e patentes aos olhos daquele a quem temos de prestar contas.
\par 14 Tendo, pois, a Jesus, o Filho de Deus, como grande sumo sacerdote que penetrou os céus, conservemos firmes a nossa confissão.
\par 15 Porque não temos sumo sacerdote que não possa compadecer-se das nossas fraquezas; antes, foi ele tentado em todas as coisas, à nossa semelhança, mas sem pecado.
\par 16 Acheguemo-nos, portanto, confiadamente, junto ao trono da graça, a fim de recebermos misericórdia e acharmos graça para socorro em ocasião oportuna.

\chapter{5}

\par 1 Porque todo sumo sacerdote, sendo tomado dentre os homens, é constituído nas coisas concernentes a Deus, a favor dos homens, para oferecer tanto dons como sacrifícios pelos pecados,
\par 2 e é capaz de condoer-se dos ignorantes e dos que erram, pois também ele mesmo está rodeado de fraquezas.
\par 3 E, por esta razão, deve oferecer sacrifícios pelos pecados, tanto do povo como de si mesmo.
\par 4 Ninguém, pois, toma esta honra para si mesmo, senão quando chamado por Deus, como aconteceu com Arão.
\par 5 Assim, também Cristo a si mesmo não se glorificou para se tornar sumo sacerdote, mas o glorificou aquele que lhe disse: Tu és meu Filho, eu hoje te gerei;
\par 6 como em outro lugar também diz: Tu és sacerdote para sempre, segundo a ordem de Melquisedeque.
\par 7 Ele, Jesus, nos dias da sua carne, tendo oferecido, com forte clamor e lágrimas, orações e súplicas a quem o podia livrar da morte e tendo sido ouvido por causa da sua piedade,
\par 8 embora sendo Filho, aprendeu a obediência pelas coisas que sofreu
\par 9 e, tendo sido aperfeiçoado, tornou-se o Autor da salvação eterna para todos os que lhe obedecem,
\par 10 tendo sido nomeado por Deus sumo sacerdote, segundo a ordem de Melquisedeque.
\par 11 A esse respeito temos muitas coisas que dizer e difíceis de explicar, porquanto vos tendes tornado tardios em ouvir.
\par 12 Pois, com efeito, quando devíeis ser mestres, atendendo ao tempo decorrido, tendes, novamente, necessidade de alguém que vos ensine, de novo, quais são os princípios elementares dos oráculos de Deus; assim, vos tornastes como necessitados de leite e não de alimento sólido.
\par 13 Ora, todo aquele que se alimenta de leite é inexperiente na palavra da justiça, porque é criança.
\par 14 Mas o alimento sólido é para os adultos, para aqueles que, pela prática, têm as suas faculdades exercitadas para discernir não somente o bem, mas também o mal.

\chapter{6}

\par 1 Por isso, pondo de parte os princípios elementares da doutrina de Cristo, deixemo-nos levar para o que é perfeito, não lançando, de novo, a base do arrependimento de obras mortas e da fé em Deus,
\par 2 o ensino de batismos e da imposição de mãos, da ressurreição dos mortos e do juízo eterno.
\par 3 Isso faremos, se Deus permitir.
\par 4 É impossível, pois, que aqueles que uma vez foram iluminados, e provaram o dom celestial, e se tornaram participantes do Espírito Santo,
\par 5 e provaram a boa palavra de Deus e os poderes do mundo vindouro,
\par 6 e caíram, sim, é impossível outra vez renová-los para arrependimento, visto que, de novo, estão crucificando para si mesmos o Filho de Deus e expondo-o à ignomínia.
\par 7 Porque a terra que absorve a chuva que freqüentemente cai sobre ela e produz erva útil para aqueles por quem é também cultivada recebe bênção da parte de Deus;
\par 8 mas, se produz espinhos e abrolhos, é rejeitada e perto está da maldição; e o seu fim é ser queimada.
\par 9 Quanto a vós outros, todavia, ó amados, estamos persuadidos das coisas que são melhores e pertencentes à salvação, ainda que falamos desta maneira.
\par 10 Porque Deus não é injusto para ficar esquecido do vosso trabalho e do amor que evidenciastes para com o seu nome, pois servistes e ainda servis aos santos.
\par 11 Desejamos, porém, continue cada um de vós mostrando, até ao fim, a mesma diligência para a plena certeza da esperança;
\par 12 para que não vos torneis indolentes, mas imitadores daqueles que, pela fé e pela longanimidade, herdam as promessas.
\par 13 Pois, quando Deus fez a promessa a Abraão, visto que não tinha ninguém superior por quem jurar, jurou por si mesmo,
\par 14 dizendo: Certamente, te abençoarei e te multiplicarei.
\par 15 E assim, depois de esperar com paciência, obteve Abraão a promessa.
\par 16 Pois os homens juram pelo que lhes é superior, e o juramento, servindo de garantia, para eles, é o fim de toda contenda.
\par 17 Por isso, Deus, quando quis mostrar mais firmemente aos herdeiros da promessa a imutabilidade do seu propósito, se interpôs com juramento,
\par 18 para que, mediante duas coisas imutáveis, nas quais é impossível que Deus minta, forte alento tenhamos nós que já corremos para o refúgio, a fim de lançar mão da esperança proposta;
\par 19 a qual temos por âncora da alma, segura e firme e que penetra além do véu,
\par 20 onde Jesus, como precursor, entrou por nós, tendo-se tornado sumo sacerdote para sempre, segundo a ordem de Melquisedeque.

\chapter{7}

\par 1 Porque este Melquisedeque, rei de Salém, sacerdote do Deus Altíssimo, que saiu ao encontro de Abraão, quando voltava da matança dos reis, e o abençoou,
\par 2 para o qual também Abraão separou o dízimo de tudo (primeiramente se interpreta rei de justiça, depois também é rei de Salém, ou seja, rei de paz;
\par 3 sem pai, sem mãe, sem genealogia; que não teve princípio de dias, nem fim de existência, entretanto, feito semelhante ao Filho de Deus), permanece sacerdote perpetuamente.
\par 4 Considerai, pois, como era grande esse a quem Abraão, o patriarca, pagou o dízimo tirado dos melhores despojos.
\par 5 Ora, os que dentre os filhos de Levi recebem o sacerdócio têm mandamento de recolher, de acordo com a lei, os dízimos do povo, ou seja, dos seus irmãos, embora tenham estes descendido de Abraão;
\par 6 entretanto, aquele cuja genealogia não se inclui entre eles recebeu dízimos de Abraão e abençoou o que tinha as promessas.
\par 7 Evidentemente, é fora de qualquer dúvida que o inferior é abençoado pelo superior.
\par 8 Aliás, aqui são homens mortais os que recebem dízimos, porém ali, aquele de quem se testifica que vive.
\par 9 E, por assim dizer, também Levi, que recebe dízimos, pagou-os na pessoa de Abraão.
\par 10 Porque aquele ainda não tinha sido gerado por seu pai, quando Melquisedeque saiu ao encontro deste.
\par 11 Se, portanto, a perfeição houvera sido mediante o sacerdócio levítico (pois nele baseado o povo recebeu a lei), que necessidade haveria ainda de que se levantasse outro sacerdote, segundo a ordem de Melquisedeque, e que não fosse contado segundo a ordem de Arão?
\par 12 Pois, quando se muda o sacerdócio, necessariamente há também mudança de lei.
\par 13 Porque aquele de quem são ditas estas coisas pertence a outra tribo, da qual ninguém prestou serviço ao altar;
\par 14 pois é evidente que nosso Senhor procedeu de Judá, tribo à qual Moisés nunca atribuiu sacerdotes.
\par 15 E isto é ainda muito mais evidente, quando, à semelhança de Melquisedeque, se levanta outro sacerdote,
\par 16 constituído não conforme a lei de mandamento carnal, mas segundo o poder de vida indissolúvel.
\par 17 Porquanto se testifica: Tu és sacerdote para sempre, segundo a ordem de Melquisedeque.
\par 18 Portanto, por um lado, se revoga a anterior ordenança, por causa de sua fraqueza e inutilidade
\par 19 (pois a lei nunca aperfeiçoou coisa alguma), e, por outro lado, se introduz esperança superior, pela qual nos chegamos a Deus.
\par 20 E, visto que não é sem prestar juramento (porque aqueles, sem juramento, são feitos sacerdotes,
\par 21 mas este, com juramento, por aquele que lhe disse: O Senhor jurou e não se arrependerá: Tu és sacerdote para sempre);
\par 22 por isso mesmo, Jesus se tem tornado fiador de superior aliança.
\par 23 Ora, aqueles são feitos sacerdotes em maior número, porque são impedidos pela morte de continuar;
\par 24 este, no entanto, porque continua para sempre, tem o seu sacerdócio imutável.
\par 25 Por isso, também pode salvar totalmente os que por ele se chegam a Deus, vivendo sempre para interceder por eles.
\par 26 Com efeito, nos convinha um sumo sacerdote como este, santo, inculpável, sem mácula, separado dos pecadores e feito mais alto do que os céus,
\par 27 que não tem necessidade, como os sumos sacerdotes, de oferecer todos os dias sacrifícios, primeiro, por seus próprios pecados, depois, pelos do povo; porque fez isto uma vez por todas, quando a si mesmo se ofereceu.
\par 28 Porque a lei constitui sumos sacerdotes a homens sujeitos à fraqueza, mas a palavra do juramento, que foi posterior à lei, constitui o Filho, perfeito para sempre.

\chapter{8}

\par 1 Ora, o essencial das coisas que temos dito é que possuímos tal sumo sacerdote, que se assentou à destra do trono da Majestade nos céus,
\par 2 como ministro do santuário e do verdadeiro tabernáculo que o Senhor erigiu, não o homem.
\par 3 Pois todo sumo sacerdote é constituído para oferecer tanto dons como sacrifícios; por isso, era necessário que também esse sumo sacerdote tivesse o que oferecer.
\par 4 Ora, se ele estivesse na terra, nem mesmo sacerdote seria, visto existirem aqueles que oferecem os dons segundo a lei,
\par 5 os quais ministram em figura e sombra das coisas celestes, assim como foi Moisés divinamente instruído, quando estava para construir o tabernáculo; pois diz ele: Vê que faças todas as coisas de acordo com o modelo que te foi mostrado no monte.
\par 6 Agora, com efeito, obteve Jesus ministério tanto mais excelente, quanto é ele também Mediador de superior aliança instituída com base em superiores promessas.
\par 7 Porque, se aquela primeira aliança tivesse sido sem defeito, de maneira alguma estaria sendo buscado lugar para uma segunda.
\par 8 E, de fato, repreendendo-os, diz: Eis aí vêm dias, diz o Senhor, e firmarei nova aliança com a casa de Israel e com a casa de Judá,
\par 9 não segundo a aliança que fiz com seus pais, no dia em que os tomei pela mão, para os conduzir até fora da terra do Egito; pois eles não continuaram na minha aliança, e eu não atentei para eles, diz o Senhor.
\par 10 Porque esta é a aliança que firmarei com a casa de Israel, depois daqueles dias, diz o Senhor: na sua mente imprimirei as minhas leis, também sobre o seu coração as inscreverei; e eu serei o seu Deus, e eles serão o meu povo.
\par 11 E não ensinará jamais cada um ao seu próximo, nem cada um ao seu irmão, dizendo: Conhece ao Senhor; porque todos me conhecerão, desde o menor deles até ao maior.
\par 12 Pois, para com as suas iniqüidades, usarei de misericórdia e dos seus pecados jamais me lembrarei.
\par 13 Quando ele diz Nova, torna antiquada a primeira. Ora, aquilo que se torna antiquado e envelhecido está prestes a desaparecer.

\chapter{9}

\par 1 Ora, a primeira aliança também tinha preceitos de serviço sagrado e o seu santuário terrestre.
\par 2 Com efeito, foi preparado o tabernáculo, cuja parte anterior, onde estavam o candeeiro, e a mesa, e a exposição dos pães, se chama o Santo Lugar;
\par 3 por trás do segundo véu, se encontrava o tabernáculo que se chama o Santo dos Santos,
\par 4 ao qual pertencia um altar de ouro para o incenso e a arca da aliança totalmente coberta de ouro, na qual estava uma urna de ouro contendo o maná, o bordão de Arão, que floresceu, e as tábuas da aliança;
\par 5 e sobre ela, os querubins de glória, que, com a sua sombra, cobriam o propiciatório. Dessas coisas, todavia, não falaremos, agora, pormenorizadamente.
\par 6 Ora, depois de tudo isto assim preparado, continuamente entram no primeiro tabernáculo os sacerdotes, para realizar os serviços sagrados;
\par 7 mas, no segundo, o sumo sacerdote, ele sozinho, uma vez por ano, não sem sangue, que oferece por si e pelos pecados de ignorância do povo,
\par 8 querendo com isto dar a entender o Espírito Santo que ainda o caminho do Santo Lugar não se manifestou, enquanto o primeiro tabernáculo continua erguido.
\par 9 É isto uma parábola para a época presente; e, segundo esta, se oferecem tanto dons como sacrifícios, embora estes, no tocante à consciência, sejam ineficazes para aperfeiçoar aquele que presta culto,
\par 10 os quais não passam de ordenanças da carne, baseadas somente em comidas, e bebidas, e diversas abluções, impostas até ao tempo oportuno de reforma.
\par 11 Quando, porém, veio Cristo como sumo sacerdote dos bens já realizados, mediante o maior e mais perfeito tabernáculo, não feito por mãos, quer dizer, não desta criação,
\par 12 não por meio de sangue de bodes e de bezerros, mas pelo seu próprio sangue, entrou no Santo dos Santos, uma vez por todas, tendo obtido eterna redenção.
\par 13 Portanto, se o sangue de bodes e de touros e a cinza de uma novilha, aspergidos sobre os contaminados, os santificam, quanto à purificação da carne,
\par 14 muito mais o sangue de Cristo, que, pelo Espírito eterno, a si mesmo se ofereceu sem mácula a Deus, purificará a nossa consciência de obras mortas, para servirmos ao Deus vivo!
\par 15 Por isso mesmo, ele é o Mediador da nova aliança, a fim de que, intervindo a morte para remissão das transgressões que havia sob a primeira aliança, recebam a promessa da eterna herança aqueles que têm sido chamados.
\par 16 Porque, onde há testamento, é necessário que intervenha a morte do testador;
\par 17 pois um testamento só é confirmado no caso de mortos; visto que de maneira nenhuma tem força de lei enquanto vive o testador.
\par 18 Pelo que nem a primeira aliança foi sancionada sem sangue;
\par 19 porque, havendo Moisés proclamado todos os mandamentos segundo a lei a todo o povo, tomou o sangue dos bezerros e dos bodes, com água, e lã tinta de escarlate, e hissopo e aspergiu não só o próprio livro, como também sobre todo o povo,
\par 20 dizendo: Este é o sangue da aliança, a qual Deus prescreveu para vós outros.
\par 21 Igualmente também aspergiu com sangue o tabernáculo e todos os utensílios do serviço sagrado.
\par 22 Com efeito, quase todas as coisas, segundo a lei, se purificam com sangue; e, sem derramamento de sangue, não há remissão.
\par 23 Era necessário, portanto, que as figuras das coisas que se acham nos céus se purificassem com tais sacrifícios, mas as próprias coisas celestiais, com sacrifícios a eles superiores.
\par 24 Porque Cristo não entrou em santuário feito por mãos, figura do verdadeiro, porém no mesmo céu, para comparecer, agora, por nós, diante de Deus;
\par 25 nem ainda para se oferecer a si mesmo muitas vezes, como o sumo sacerdote cada ano entra no Santo dos Santos com sangue alheio.
\par 26 Ora, neste caso, seria necessário que ele tivesse sofrido muitas vezes desde a fundação do mundo; agora, porém, ao se cumprirem os tempos, se manifestou uma vez por todas, para aniquilar, pelo sacrifício de si mesmo, o pecado.
\par 27 E, assim como aos homens está ordenado morrerem uma só vez, vindo, depois disto, o juízo,
\par 28 assim também Cristo, tendo-se oferecido uma vez para sempre para tirar os pecados de muitos, aparecerá segunda vez, sem pecado, aos que o aguardam para a salvação.

\chapter{10}

\par 1 Ora, visto que a lei tem sombra dos bens vindouros, não a imagem real das coisas, nunca jamais pode tornar perfeitos os ofertantes, com os mesmos sacrifícios que, ano após ano, perpetuamente, eles oferecem.
\par 2 Doutra sorte, não teriam cessado de ser oferecidos, porquanto os que prestam culto, tendo sido purificados uma vez por todas, não mais teriam consciência de pecados?
\par 3 Entretanto, nesses sacrifícios faz-se recordação de pecados todos os anos,
\par 4 porque é impossível que o sangue de touros e de bodes remova pecados.
\par 5 Por isso, ao entrar no mundo, diz: Sacrifício e oferta não quiseste; antes, um corpo me formaste;
\par 6 não te deleitaste com holocaustos e ofertas pelo pecado.
\par 7 Então, eu disse: Eis aqui estou (no rolo do livro está escrito a meu respeito), para fazer, ó Deus, a tua vontade.
\par 8 Depois de dizer, como acima: Sacrifícios e ofertas não quiseste, nem holocaustos e oblações pelo pecado, nem com isto te deleitaste (coisas que se oferecem segundo a lei),
\par 9 então, acrescentou: Eis aqui estou para fazer, ó Deus, a tua vontade. Remove o primeiro para estabelecer o segundo.
\par 10 Nessa vontade é que temos sido santificados, mediante a oferta do corpo de Jesus Cristo, uma vez por todas.
\par 11 Ora, todo sacerdote se apresenta, dia após dia, a exercer o serviço sagrado e a oferecer muitas vezes os mesmos sacrifícios, que nunca jamais podem remover pecados;
\par 12 Jesus, porém, tendo oferecido, para sempre, um único sacrifício pelos pecados, assentou-se à destra de Deus,
\par 13 aguardando, daí em diante, até que os seus inimigos sejam postos por estrado dos seus pés.
\par 14 Porque, com uma única oferta, aperfeiçoou para sempre quantos estão sendo santificados.
\par 15 E disto nos dá testemunho também o Espírito Santo; porquanto, após ter dito:
\par 16 Esta é a aliança que farei com eles, depois daqueles dias, diz o Senhor: Porei no seu coração as minhas leis e sobre a sua mente as inscreverei,
\par 17 acrescenta: Também de nenhum modo me lembrarei dos seus pecados e das suas iniqüidades, para sempre.
\par 18 Ora, onde há remissão destes, já não há oferta pelo pecado.
\par 19 Tendo, pois, irmãos, intrepidez para entrar no Santo dos Santos, pelo sangue de Jesus,
\par 20 pelo novo e vivo caminho que ele nos consagrou pelo véu, isto é, pela sua carne,
\par 21 e tendo grande sacerdote sobre a casa de Deus,
\par 22 aproximemo-nos, com sincero coração, em plena certeza de fé, tendo o coração purificado de má consciência e lavado o corpo com água pura.
\par 23 Guardemos firme a confissão da esperança, sem vacilar, pois quem fez a promessa é fiel.
\par 24 Consideremo-nos também uns aos outros, para nos estimularmos ao amor e às boas obras.
\par 25 Não deixemos de congregar-nos, como é costume de alguns; antes, façamos admoestações e tanto mais quanto vedes que o Dia se aproxima.
\par 26 Porque, se vivermos deliberadamente em pecado, depois de termos recebido o pleno conhecimento da verdade, já não resta sacrifício pelos pecados;
\par 27 pelo contrário, certa expectação horrível de juízo e fogo vingador prestes a consumir os adversários.
\par 28 Sem misericórdia morre pelo depoimento de duas ou três testemunhas quem tiver rejeitado a lei de Moisés.
\par 29 De quanto mais severo castigo julgais vós será considerado digno aquele que calcou aos pés o Filho de Deus, e profanou o sangue da aliança com o qual foi santificado, e ultrajou o Espírito da graça?
\par 30 Ora, nós conhecemos aquele que disse: A mim pertence a vingança; eu retribuirei. E outra vez: O Senhor julgará o seu povo.
\par 31 Horrível coisa é cair nas mãos do Deus vivo.
\par 32 Lembrai-vos, porém, dos dias anteriores, em que, depois de iluminados, sustentastes grande luta e sofrimentos;
\par 33 ora expostos como em espetáculo, tanto de opróbrio quanto de tribulações, ora tornando-vos co-participantes com aqueles que desse modo foram tratados.
\par 34 Porque não somente vos compadecestes dos encarcerados, como também aceitastes com alegria o espólio dos vossos bens, tendo ciência de possuirdes vós mesmos patrimônio superior e durável.
\par 35 Não abandoneis, portanto, a vossa confiança; ela tem grande galardão.
\par 36 Com efeito, tendes necessidade de perseverança, para que, havendo feito a vontade de Deus, alcanceis a promessa.
\par 37 Porque, ainda dentro de pouco tempo, aquele que vem virá e não tardará;
\par 38 todavia, o meu justo viverá pela fé; e: Se retroceder, nele não se compraz a minha alma.
\par 39 Nós, porém, não somos dos que retrocedem para a perdição; somos, entretanto, da fé, para a conservação da alma.

\chapter{11}

\par 1 Ora, a fé é a certeza de coisas que se esperam, a convicção de fatos que se não vêem.
\par 2 Pois, pela fé, os antigos obtiveram bom testemunho.
\par 3 Pela fé, entendemos que foi o universo formado pela palavra de Deus, de maneira que o visível veio a existir das coisas que não aparecem.
\par 4 Pela fé, Abel ofereceu a Deus mais excelente sacrifício do que Caim; pelo qual obteve testemunho de ser justo, tendo a aprovação de Deus quanto às suas ofertas. Por meio dela, também mesmo depois de morto, ainda fala.
\par 5 Pela fé, Enoque foi trasladado para não ver a morte; não foi achado, porque Deus o trasladara. Pois, antes da sua trasladação, obteve testemunho de haver agradado a Deus.
\par 6 De fato, sem fé é impossível agradar a Deus, porquanto é necessário que aquele que se aproxima de Deus creia que ele existe e que se torna galardoador dos que o buscam.
\par 7 Pela fé, Noé, divinamente instruído acerca de acontecimentos que ainda não se viam e sendo temente a Deus, aparelhou uma arca para a salvação de sua casa; pela qual condenou o mundo e se tornou herdeiro da justiça que vem da fé.
\par 8 Pela fé, Abraão, quando chamado, obedeceu, a fim de ir para um lugar que devia receber por herança; e partiu sem saber aonde ia.
\par 9 Pela fé, peregrinou na terra da promessa como em terra alheia, habitando em tendas com Isaque e Jacó, herdeiros com ele da mesma promessa;
\par 10 porque aguardava a cidade que tem fundamentos, da qual Deus é o arquiteto e edificador.
\par 11 Pela fé, também, a própria Sara recebeu poder para ser mãe, não obstante o avançado de sua idade, pois teve por fiel aquele que lhe havia feito a promessa.
\par 12 Por isso, também de um, aliás já amortecido, saiu uma posteridade tão numerosa como as estrelas do céu e inumerável como a areia que está na praia do mar.
\par 13 Todos estes morreram na fé, sem ter obtido as promessas; vendo-as, porém, de longe, e saudando-as, e confessando que eram estrangeiros e peregrinos sobre a terra.
\par 14 Porque os que falam desse modo manifestam estar procurando uma pátria.
\par 15 E, se, na verdade, se lembrassem daquela de onde saíram, teriam oportunidade de voltar.
\par 16 Mas, agora, aspiram a uma pátria superior, isto é, celestial. Por isso, Deus não se envergonha deles, de ser chamado o seu Deus, porquanto lhes preparou uma cidade.
\par 17 Pela fé, Abraão, quando posto à prova, ofereceu Isaque; estava mesmo para sacrificar o seu unigênito aquele que acolheu alegremente as promessas,
\par 18 a quem se tinha dito: Em Isaque será chamada a tua descendência;
\par 19 porque considerou que Deus era poderoso até para ressuscitá-lo dentre os mortos, de onde também, figuradamente, o recobrou.
\par 20 Pela fé, igualmente Isaque abençoou a Jacó e a Esaú, acerca de coisas que ainda estavam para vir.
\par 21 Pela fé, Jacó, quando estava para morrer, abençoou cada um dos filhos de José e, apoiado sobre a extremidade do seu bordão, adorou.
\par 22 Pela fé, José, próximo do seu fim, fez menção do êxodo dos filhos de Israel, bem como deu ordens quanto aos seus próprios ossos.
\par 23 Pela fé, Moisés, apenas nascido, foi ocultado por seus pais, durante três meses, porque viram que a criança era formosa; também não ficaram amedrontados pelo decreto do rei.
\par 24 Pela fé, Moisés, quando já homem feito, recusou ser chamado filho da filha de Faraó,
\par 25 preferindo ser maltratado junto com o povo de Deus a usufruir prazeres transitórios do pecado;
\par 26 porquanto considerou o opróbrio de Cristo por maiores riquezas do que os tesouros do Egito, porque contemplava o galardão.
\par 27 Pela fé, ele abandonou o Egito, não ficando amedrontado com a cólera do rei; antes, permaneceu firme como quem vê aquele que é invisível.
\par 28 Pela fé, celebrou a Páscoa e o derramamento do sangue, para que o exterminador não tocasse nos primogênitos dos israelitas.
\par 29 Pela fé, atravessaram o mar Vermelho como por terra seca; tentando-o os egípcios, foram tragados de todo.
\par 30 Pela fé, ruíram as muralhas de Jericó, depois de rodeadas por sete dias.
\par 31 Pela fé, Raabe, a meretriz, não foi destruída com os desobedientes, porque acolheu com paz aos espias.
\par 32 E que mais direi? Certamente, me faltará o tempo necessário para referir o que há a respeito de Gideão, de Baraque, de Sansão, de Jefté, de Davi, de Samuel e dos profetas,
\par 33 os quais, por meio da fé, subjugaram reinos, praticaram a justiça, obtiveram promessas, fecharam a boca de leões,
\par 34 extinguiram a violência do fogo, escaparam ao fio da espada, da fraqueza tiraram força, fizeram-se poderosos em guerra, puseram em fuga exércitos de estrangeiros.
\par 35 Mulheres receberam, pela ressurreição, os seus mortos. Alguns foram torturados, não aceitando seu resgate, para obterem superior ressurreição;
\par 36 outros, por sua vez, passaram pela prova de escárnios e açoites, sim, até de algemas e prisões.
\par 37 Foram apedrejados, provados, serrados pelo meio, mortos a fio de espada; andaram peregrinos, vestidos de peles de ovelhas e de cabras, necessitados, afligidos, maltratados
\par 38 (homens dos quais o mundo não era digno), errantes pelos desertos, pelos montes, pelas covas, pelos antros da terra.
\par 39 Ora, todos estes que obtiveram bom testemunho por sua fé não obtiveram, contudo, a concretização da promessa,
\par 40 por haver Deus provido coisa superior a nosso respeito, para que eles, sem nós, não fossem aperfeiçoados.

\chapter{12}

\par 1 Portanto, também nós, visto que temos a rodear-nos tão grande nuvem de testemunhas, desembaraçando-nos de todo peso e do pecado que tenazmente nos assedia, corramos, com perseverança, a carreira que nos está proposta,
\par 2 olhando firmemente para o Autor e Consumador da fé, Jesus, o qual, em troca da alegria que lhe estava proposta, suportou a cruz, não fazendo caso da ignomínia, e está assentado à destra do trono de Deus.
\par 3 Considerai, pois, atentamente, aquele que suportou tamanha oposição dos pecadores contra si mesmo, para que não vos fatigueis, desmaiando em vossa alma.
\par 4 Ora, na vossa luta contra o pecado, ainda não tendes resistido até ao sangue
\par 5 e estais esquecidos da exortação que, como a filhos, discorre convosco: Filho meu, não menosprezes a correção que vem do Senhor, nem desmaies quando por ele és reprovado;
\par 6 porque o Senhor corrige a quem ama e açoita a todo filho a quem recebe.
\par 7 É para disciplina que perseverais (Deus vos trata como filhos); pois que filho há que o pai não corrige?
\par 8 Mas, se estais sem correção, de que todos se têm tornado participantes, logo, sois bastardos e não filhos.
\par 9 Além disso, tínhamos os nossos pais segundo a carne, que nos corrigiam, e os respeitávamos; não havemos de estar em muito maior submissão ao Pai espiritual e, então, viveremos?
\par 10 Pois eles nos corrigiam por pouco tempo, segundo melhor lhes parecia; Deus, porém, nos disciplina para aproveitamento, a fim de sermos participantes da sua santidade.
\par 11 Toda disciplina, com efeito, no momento não parece ser motivo de alegria, mas de tristeza; ao depois, entretanto, produz fruto pacífico aos que têm sido por ela exercitados, fruto de justiça.
\par 12 Por isso, restabelecei as mãos descaídas e os joelhos trôpegos;
\par 13 e fazei caminhos retos para os pés, para que não se extravie o que é manco; antes, seja curado.
\par 14 Segui a paz com todos e a santificação, sem a qual ninguém verá o Senhor,
\par 15 atentando, diligentemente, por que ninguém seja faltoso, separando-se da graça de Deus; nem haja alguma raiz de amargura que, brotando, vos perturbe, e, por meio dela, muitos sejam contaminados;
\par 16 nem haja algum impuro ou profano, como foi Esaú, o qual, por um repasto, vendeu o seu direito de primogenitura.
\par 17 Pois sabeis também que, posteriormente, querendo herdar a bênção, foi rejeitado, pois não achou lugar de arrependimento, embora, com lágrimas, o tivesse buscado.
\par 18 Ora, não tendes chegado ao fogo palpável e ardente, e à escuridão, e às trevas, e à tempestade,
\par 19 e ao clangor da trombeta, e ao som de palavras tais, que quantos o ouviram suplicaram que não se lhes falasse mais,
\par 20 pois já não suportavam o que lhes era ordenado: Até um animal, se tocar o monte, será apedrejado.
\par 21 Na verdade, de tal modo era horrível o espetáculo, que Moisés disse: Sinto-me aterrado e trêmulo!
\par 22 Mas tendes chegado ao monte Sião e à cidade do Deus vivo, a Jerusalém celestial, e a incontáveis hostes de anjos, e à universal assembléia
\par 23 e igreja dos primogênitos arrolados nos céus, e a Deus, o Juiz de todos, e aos espíritos dos justos aperfeiçoados,
\par 24 e a Jesus, o Mediador da nova aliança, e ao sangue da aspersão que fala coisas superiores ao que fala o próprio Abel.
\par 25 Tende cuidado, não recuseis ao que fala. Pois, se não escaparam aqueles que recusaram ouvir quem, divinamente, os advertia sobre a terra, muito menos nós, os que nos desviamos daquele que dos céus nos adverte,
\par 26 aquele, cuja voz abalou, então, a terra; agora, porém, ele promete, dizendo: Ainda uma vez por todas, farei abalar não só a terra, mas também o céu.
\par 27 Ora, esta palavra: Ainda uma vez por todas significa a remoção dessas coisas abaladas, como tinham sido feitas, para que as coisas que não são abaladas permaneçam.
\par 28 Por isso, recebendo nós um reino inabalável, retenhamos a graça, pela qual sirvamos a Deus de modo agradável, com reverência e santo temor;
\par 29 porque o nosso Deus é fogo consumidor.

\chapter{13}

\par 1 Seja constante o amor fraternal.
\par 2 Não negligencieis a hospitalidade, pois alguns, praticando-a, sem o saber acolheram anjos.
\par 3 Lembrai-vos dos encarcerados, como se presos com eles; dos que sofrem maus tratos, como se, com efeito, vós mesmos em pessoa fôsseis os maltratados.
\par 4 Digno de honra entre todos seja o matrimônio, bem como o leito sem mácula; porque Deus julgará os impuros e adúlteros.
\par 5 Seja a vossa vida sem avareza. Contentai-vos com as coisas que tendes; porque ele tem dito: De maneira alguma te deixarei, nunca jamais te abandonarei.
\par 6 Assim, afirmemos confiantemente: O Senhor é o meu auxílio, não temerei; que me poderá fazer o homem?
\par 7 Lembrai-vos dos vossos guias, os quais vos pregaram a palavra de Deus; e, considerando atentamente o fim da sua vida, imitai a fé que tiveram.
\par 8 Jesus Cristo, ontem e hoje, é o mesmo e o será para sempre.
\par 9 Não vos deixeis envolver por doutrinas várias e estranhas, porquanto o que vale é estar o coração confirmado com graça e não com alimentos, pois nunca tiveram proveito os que com isto se preocuparam.
\par 10 Possuímos um altar do qual não têm direito de comer os que ministram no tabernáculo.
\par 11 Pois aqueles animais cujo sangue é trazido para dentro do Santo dos Santos, pelo sumo sacerdote, como oblação pelo pecado, têm o corpo queimado fora do acampamento.
\par 12 Por isso, foi que também Jesus, para santificar o povo, pelo seu próprio sangue, sofreu fora da porta.
\par 13 Saiamos, pois, a ele, fora do arraial, levando o seu vitupério.
\par 14 Na verdade, não temos aqui cidade permanente, mas buscamos a que há de vir.
\par 15 Por meio de Jesus, pois, ofereçamos a Deus, sempre, sacrifício de louvor, que é o fruto de lábios que confessam o seu nome.
\par 16 Não negligencieis, igualmente, a prática do bem e a mútua cooperação; pois, com tais sacrifícios, Deus se compraz.
\par 17 Obedecei aos vossos guias e sede submissos para com eles; pois velam por vossa alma, como quem deve prestar contas, para que façam isto com alegria e não gemendo; porque isto não aproveita a vós outros.
\par 18 Orai por nós, pois estamos persuadidos de termos boa consciência, desejando em todas as coisas viver condignamente.
\par 19 Rogo-vos, com muito empenho, que assim façais, a fim de que eu vos seja restituído mais depressa.
\par 20 Ora, o Deus da paz, que tornou a trazer dentre os mortos a Jesus, nosso Senhor, o grande Pastor das ovelhas, pelo sangue da eterna aliança,
\par 21 vos aperfeiçoe em todo o bem, para cumprirdes a sua vontade, operando em vós o que é agradável diante dele, por Jesus Cristo, a quem seja a glória para todo o sempre. Amém!
\par 22 Rogo-vos ainda, irmãos, que suporteis a presente palavra de exortação; tanto mais quanto vos escrevi resumidamente.
\par 23 Notifico-vos que o irmão Timóteo foi posto em liberdade; com ele, caso venha logo, vos verei.
\par 24 Saudai todos os vossos guias, bem como todos os santos. Os da Itália vos saúdam.
\par 25 A graça seja com todos vós.


\end{document}