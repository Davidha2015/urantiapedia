\begin{document}

\title{Geneza}

\chapter{1}

\par 1 La început a făcut Dumnezeu cerul și pământul.
\par 2 Și pământul era netocmit și gol. Întuneric era deasupra adâncului și Duhul lui Dumnezeu Se purta pe deasupra apelor.
\par 3 Și a zis Dumnezeu: "Să fie lumină!" Și a fost lumină.
\par 4 Și a văzut Dumnezeu că este bună lumina, și a despărțit Dumnezeu lumina de întuneric.
\par 5 Lumina a numit-o Dumnezeu ziuă, iar întunericul l-a numit noapte. Și a fost seară și a fost dimineață: ziua întâi.
\par 6 Și a zis Dumnezeu: "Să fie o tărie prin mijlocul apelor și să despartă ape de ape!" Și a fost așa.
\par 7 A făcut Dumnezeu tăria și a despărțit Dumnezeu apele cele de sub tărie de apele cele de deasupra tăriei.
\par 8 Tăria a numit-o Dumnezeu cer. Și a văzut Dumnezeu că este bine. Și a fost seară și a fost dimineață: ziua a doua.
\par 9 Și a zis Dumnezeu: "Să se adune apele cele de sub cer la un loc și să se arate uscatul!" Și a fost așa. și s-au adunat apele cele de sub cer la locurile lor și s-a arătat uscatul.
\par 10 Uscatul l-a numit Dumnezeu pământ, iar adunarea apelor a numit-o mări. Și a văzut Dumnezeu că este bine.
\par 11 Apoi a zis Dumnezeu: "Să dea pământul din sine verdeață: iarbă, cu sămânță într-însa, după felul și asemănarea ei, și pomi roditori, care să dea rod cu sămânță în sine, după fel, pe pământ!" Și a fost așa.
\par 12 Pământul a dat din sine verdeață: iarbă, care face sămânță, după felul și după asemănarea ei, și pomi roditori, cu sămânță, după fel, pe pământ. Și a văzut Dumnezeu că este bine.
\par 13 Și a fost seară și a fost dimineață: ziua a treia.
\par 14 Și a zis Dumnezeu: "Să fie luminători pe tăria cerului, ca să lumineze pe pământ, să despartă ziua de noapte și să fie semne ca să deosebească anotimpurile, zilele și anii,
\par 15 Și să slujească drept luminători pe tăria cerului, ca să lumineze pământul." Și a fost așa.
\par 16 A făcut Dumnezeu cei doi luminători mari: luminătorul cel mai mare pentru cârmuirea zilei și luminătorul cel mai mic pentru cârmuirea nopții, și stelele.
\par 17 Și le-a pus Dumnezeu pe tăria cerului, ca să lumineze pământul,
\par 18 Să cârmuiască ziua și noaptea și să despartă lumina de întuneric. Și a văzut Dumnezeu că este bine.
\par 19 Și a fost seară și a fost dimineață: ziua a patra.
\par 20 Apoi a zis Dumnezeu: "Să mișune apele de vietăți, ființe cu viață în ele și păsări să zboare pe pământ, pe întinsul tăriei cerului!" Și a fost așa.
\par 21 A făcut Dumnezeu animalele cele mari din ape și toate ființele vii, care mișună în ape, unde ele se prăsesc după felul lor, și toate păsările înaripate după felul lor. Și a văzut Dumnezeu că este bine.
\par 22 Și le-a binecuvântat Dumnezeu și a zis: "Prăsiți-vă și vă înmulțiți și umpleți apele mărilor și păsările să se înmulțească pe pământ!"
\par 23 Și a fost seară și a fost dimineață: ziua a cincea.
\par 24 Apoi a zis Dumnezeu: "Să scoată pământul ființe vii, după felul lor: animale, târâtoare și fiare sălbatice după felul lor". Și a fost așa.
\par 25 A făcut Dumnezeu fiarele sălbatice după felul lor, și animalele domestice după felul lor, și toate târâtoarele pământului după felul lor. Și a văzut Dumnezeu că este bine.
\par 26 Și a zis Dumnezeu: "Să facem om după chipul și după asemănarea Noastră, ca să stăpânească peștii mării, păsările cerului, animalele domestice, toate vietățile ce se târăsc pe pământ și tot pământul!"
\par 27 Și a făcut Dumnezeu pe om după chipul Său; după chipul lui Dumnezeu l-a făcut; a făcut bărbat și femeie.
\par 28 Și Dumnezeu i-a binecuvântat, zicând: "Creșteți și vă înmulțiți și umpleți pământul și-l supuneți; și stăpâniri peste peștii mării, peste păsările cerului, peste toate animalele, peste toate vietățile ce se mișcă pe pământ și peste tot pământul!"
\par 29 Apoi a zis Dumnezeu: "Iată, vă dau toată iarba ce face sămânță de pe toată fața pământului și tot pomul ce are rod cu sămânță în el. Acestea vor fi hrana voastră.
\par 30 Iar tuturor fiarelor pământului și tuturor păsărilor cerului și tuturor vietăților ce se mișcă pe pământ, care au în ele suflare de viață, le dau toată iarba verde spre hrană." Și a fost așa.
\par 31 Și a privit Dumnezeu toate câte a făcut și iată erau bune foarte. Și a fost seară și a fost dimineață: ziua a șasea.

\chapter{2}

\par 1 Așa s-au făcut cerul și pământul și toată oștirea lor.
\par 2 Și a sfârșit Dumnezeu în ziua a șasea lucrarea Sa, pe care a făcut-o; iar în ziua a șaptea S-a odihnit de toate lucrurile Sale, pe care le-a făcut.
\par 3 Și a binecuvântat Dumnezeu ziua a șaptea și a sfințit-o, pentru că într-însa S-a odihnit de toate lucrurile Sale, pe care le-a făcut și le-a pus în rânduială.
\par 4 Iată obârșia cerului și a pământului de la facerea lor, din ziua când Domnul Dumnezeu a făcut cerul și pământul.
\par 5 Pe câmp nu se afla nici un copăcel, iar iarba de pe el nu începuse a odrăsli, pentru că Domnul Dumnezeu nu trimisese încă ploaie pe pământ și nu era nimeni ca să lucreze pământul.
\par 6 Ci numai abur ieșea din pământ și umezea toată fața pământului.
\par 7 Atunci, luând Domnul Dumnezeu țărână din pământ, a făcut pe om și a suflat în fața lui suflare de viață și s-a făcut omul ființă vie.
\par 8 Apoi Domnul Dumnezeu a sădit o grădină în Eden, spre răsărit, și a pus acolo pe omul pe care-l zidise.
\par 9 Și a făcut Domnul Dumnezeu să răsară din pământ tot soiul de pomi, plăcuți la vedere și cu roade bune de mâncat; ier în mijlocul raiului era pomul vieții și pomul cunoștinței binelui și răului.
\par 10 Și din Eden ieșea un râu, care uda raiul, iar de acolo se împărțea în patru brațe.
\par 11 Numele unuia era Fison. Acesta înconjură toată țara Havila, în care se află aur.
\par 12 Aurul din țara aceea este bun; tot acolo se găsește bdeliu și piatra de onix.
\par 13 Numele râului al doilea este Gihon. Acesta înconjură toată țara Cuș.
\par 14 Numele râului al treilea este Tigru. Acesta curge prin fața Asiriei; iar râul al patrulea este Eufratul.
\par 15 Și a luat Domnul Dumnezeu pe omul pe care-l făcuse și l-a pus în grădina cea din Eden, ca s-o lucreze și s-o păzească.
\par 16 A dat apoi Domnul Dumnezeu lui Adam poruncă și î zis: "Din toți pomii din rai poți să mănânci,
\par 17 Iar din pomul cunoștinței binelui și răului să nu mănânci, căci, în ziua în care vei mânca din el, vei muri negreșit!
\par 18 Și a zis Domnul Dumnezeu: "Nu este bine să fie omul singur; să-i facem ajutor potrivit pentru el".
\par 19 Și Domnul Dumnezeu, Care făcuse din pământ toate fiarele câmpului și toate păsările cerului, le-a adus la Adam, ca să vadă cum le va numi; așa ca toate ființele vii să se numească precum le va numi Adam.
\par 20 Și a pus Adam nume tuturor animalelor și tuturor păsărilor cerului și tuturor fiarelor sălbatice; dar pentru Adam nu s-a găsit ajutor de potriva lui.
\par 21 Atunci a adus Domnul Dumnezeu asupra lui Adam somn greu; și, dacă a adormit, a luat una din coastele lui și a plinit locul ei cu carne.
\par 22 Iar coasta luată din Adam a făcut-o Domnul Dumnezeu femeie și a adus-o la Adam.
\par 23 Și a zis Adam: "Iată aceasta-i os din oasele mele și carne din carnea mea; ea se va numi femeie, pentru că este luată din bărbatul său.
\par 24 De aceea va lăsa omul pe tatăl său și pe mama sa și se va uni cu femeia sa și vor fi amândoi un trup."
\par 25 Adam și femeia lui erau amândoi goi și nu se rușinau.

\chapter{3}

\par 1 Șarpele însă era cel mai șiret dintre toate fiarele de pe pământ, pe care le făcuse Domnul Dumnezeu. Și a zis șarpele către femeie: "Dumnezeu a zis El, oare, să nu mâncați roade din orice pom din rai?"
\par 2 Iar femeia a zis către șarpe: "Roade din pomii raiului putem să mâncăm;
\par 3 Numai din rodul pomului celui din mijlocul raiului ne-a zis Dumnezeu: "Să nu mâncați din el, nici să vă atingeți de el, ca să nu muriți!"
\par 4 Atunci șarpele a zis către femeie: "Nu, nu veți muri!
\par 5 Dar Dumnezeu știe că în ziua în care veți mânca din el vi se vor deschide ochii și veți fi ca Dumnezeu, cunoscând binele și răul".
\par 6 De aceea femeia, socotind că rodul pomului este bun de mâncat și plăcut ochilor la vedere și vrednic de dorit, pentru că dă știință, a luat din el și a mâncat și a dat bărbatului său și a mâncat și el.
\par 7 Atunci li s-au deschis ochii la amândoi și au cunoscut că erau goi, și au cusut frunze de smochin și și-au făcut acoperăminte.
\par 8 Iar când au auzit glasul Domnului Dumnezeu, Care umbla prin rai, în răcoarea serii, s-au ascuns Adam și femeia lui de fața Domnului Dumnezeu printre pomii raiului.
\par 9 Și a strigat Domnul Dumnezeu pe Adam și i-a zis: "Adame, unde ești?"
\par 10 Răspuns-a acesta: "Am auzit glasul Tău în rai și m-am temut, căci sunt gol, și m-am ascuns".
\par 11 Și i-a zis Dumnezeu: "Cine ti-a spus că ești gol? Nu cumva ai mâncat din pomul din care ți-am poruncit să nu mănânci?"
\par 12 Zis-a Adam: "Femeia care mi-ai dat-o să fie cu mine, aceea mi-a dat din pom și am mâncat".
\par 13 Și a zis Domnul Dumnezeu către femeie: "Pentru ce ai făcut aceasta?" Iar femeia a zis: "Șarpele m-a amăgit și eu am mâncat".
\par 14 Zis-a Domnul Dumnezeu către șarpe: "Pentru că ai făcut aceasta, blestemat să fii între toate animalele și între toate fiarele câmpului; pe pântecele tău să te târăști și țărână să mănânci în toate zilele vieții tale!
\par 15 Dușmănie voi pune între tine și între femeie, între sămânța ta și sămânța ei; aceasta îți va zdrobi capul, iar tu îi vei înțepa călcâiul".
\par 16 Iar femeii i-a zis: "Voi înmulți mereu necazurile tale, mai ales în vremea sarcinii tale; în dureri vei naște copii; atrasă vei fi către bărbatul tău și el te va stăpâni".
\par 17 Iar lui Adam i-a zis: "Pentru că ai ascultat vorba femeii tale și ai mâncat din pomul din care ți-am poruncit: "Să nu mănânci", blestemat va fi pământul pentru tine! Cu osteneală să te hrănești din el în toate zilele vieții tale!
\par 18 Spini și pălămidă îți va rodi el și te vei hrăni cu iarba câmpului!
\par 19 În sudoarea feței tale îți vei mânca pâinea ta, până te vei întoarce în pământul din care ești luat; căci pământ ești și în pământ te vei întoarce".
\par 20 Și a pus Adam femeii sale numele Eva, adică viață, pentru că ea era să fie mama tuturor celor vii.
\par 21 Apoi a făcut Domnul Dumnezeu lui Adam și femeii lui îmbrăcăminte de piele și i-a îmbrăcat.
\par 22 Și a zis Domnul Dumnezeu: "Iată Adam s-a făcut ca unul dintre Noi, cunoscând binele și răul. Și acum nu cumva să-și întindă mâna și să ia roade din pomul vieții, să mănânce și să trăiască în veci!..."
\par 23 De aceea l-a scos Domnul Dumnezeu din grădina cea din Eden, ca să lucreze pământul, din care fusese luat.
\par 24 Și izgonind pe Adam, l-a așezat în preajma grădinii celei din Eden și a pus heruvimi și sabie de flacără vâlvâitoare, să păzească drumul către pomul vieții.

\chapter{4}

\par 1 După aceea a cunoscut Adam pe Eva, femeia sa, și ea, zămislind, a născut pe Cain și a zis: "Am dobândit om de la Dumnezeu".
\par 2 Apoi a mai născut pe Abel, fratele lui Cain. Abel a fost păstor de oi, iar Cain lucrător de pământ.
\par 3 Dar după un timp, Cain a adus jertfă lui Dumnezeu din roadele pământului.
\par 4 Și a adus și Abel din cele întâi-născute ale oilor sale și din grăsimea lor. Și a căutat Domnul spre Abel și spre darurile lui,
\par 5 Iar spre Cain și spre darurile lui n-a căutat. Și s-a întristat Cain tare și fața lui era posomorâtă.
\par 6 Atunci a zis Domnul Dumnezeu către Cain: "Pentru ce te-ai întristat și pentru ce s-a posomorât fața ta?
\par 7 Când faci bine, oare nu-ți este fața senină? Iar de nu faci bine, păcatul bate la ușă și caută să te târască, dar tu biruiește-l!"
\par 8 După aceea Cain a zis către Abel, fratele său: "Să ieșim la câmp!" Iar când erau ei în câmpie, Cain s-a aruncat asupra lui Abel, fratele său, și l-a omorât.
\par 9 Atunci a zis Domnul Dumnezeu către Cain: "Unde este Abel, fratele tău?" Iar el a răspuns: "Nu știu! Au doară eu sunt păzitorul fratelui meu?"
\par 10 și a zis Domnul: "Ce ai făcut? Glasul sângelui fratelui tău strigă către Mine din pământ.
\par 11 Și acum ești blestemat de pământul care și-a deschis gura sa, ca să primească sângele fratelui tău din mâna ta.
\par 12 Când vei lucra pământul, acesta nu-și va mai da roadele sale ție; zbuciumat și fugar vei fi tu pe pământ".
\par 13 Și a zis Cain către Domnul Dumnezeu: "Pedeapsa mea este mai mare decât aș putea-o purta.
\par 14 De mă izgonești acum din pământul acesta, mă voi ascunde de la fața Ta și voi fi zbuciumat și fugar pe pământ, și oricine mă va întâlni, mă va ucide".
\par 15 Și i-a zis Domnul Dumnezeu: "Nu așa, ci tot cel ce va ucide pe Cain înșeptit se va pedepsi". Și a pus Domnul  Dumnezeu semn lui Cain, ca tot cel care îl va întâlni să nu-l omoare.
\par 16 Și s-a dus Cain de la fața lui Dumnezeu și a locuit în ținutul Nod, la răsărit de Eden.
\par 17 După aceea a cunoscut Cain pe femeia sa și ea, zămislind, a născut pe Enoh. Apoi a zidit Cain o cetate și a numit-o, după numele fiului său, Enoh.
\par 18 Iar lui Enoh i s-a născut Irad; lui Irad i s-a născut Maleleil; lui Maleleil i s-a născut Matusal, iar lui Matusal i s-a născut Lameh.
\par 19 Lameh și-a luat două femei: numele uneia era Ada și numele celeilalte era Sela.
\par 20 Ada a născut pe Iabal; acesta a fost tatăl celor ce trăiesc în corturi, la turme.
\par 21 Fratele lui se numea Iubal; acesta este tatăl tuturor celor ce cântă din chitară și din cimpoi.
\par 22 Sela a născut și ea pe Tubalcain, care a fost făurar de unelte de aramă și de fier. Și sora lui se chema Noema.
\par 23 Și a zis Lameh către femeile sale: "Ada și Sela, ascultați glasul meu! Femeile lui Lameh, luați aminte la cuvintele mele: Am ucis un om pentru rana mea și un tânăr pentru vânătaia mea.
\par 24 Dacă pentru Cain va fi răzbunarea de șapte ori, apoi pentru Lameh de șaptezeci de ori câte șapte!"
\par 25 Adam a cunoscut iarăși pe Eva, femeia sa, și ea, zămislind, a născut un fiu și i-a pus numele Set, pentru că și-a zis: "Mi-a dat Dumnezeu alt fiu în locul lui Abel, pe care l-a ucis Cain".
\par 26 Lui Set de asemenea i s-a născut un fiu și i-a pus numele Enos. Atunci au început oamenii a chema numele Domnului Dumnezeu.

\chapter{5}

\par 1 Iată acum cartea neamului lui Adam. Când a făcut Dumnezeu pe Adam, l-a făcut după chipul lui Dumnezeu.
\par 2 Bărbat și femeie a făcut și i-a. binecuvântat și le-a pus numele: Om, în ziua în care i-a făcut.
\par 3 Adam a trăit două sute treizeci de ani și atunci i s-a născut un fiu după asemănarea sa și, după chipul său și i-a pus numele Set.
\par 4 Zilele pe care le-a trăit Adam după nașterea lui Set au fost șapte sute de ani și i s-au născut fii și fiice.
\par 5 Iar de toate, zilele vieții lui Adam au fost nouă sute treizeci de ani și apoi a murit.
\par 6 Set a trăit două sute cinci ani și i s-a născut Enos.
\par 7 După nașterea lui Enos, Set a mai trăit șapte sute șapte ani, și i s-au născut fii și fiice.
\par 8 Iar de toate, zilele lui Set au fost nouă sute doisprezece ani și apoi a murit.
\par 9 Enos a trăit o sută nouăzeci de ani și atunci i s-a născut Cainan.
\par 10 După nașterea lui Cainan, Enos a mai trăit șapte sute cincisprezece ani și i s-au născut fii și fiice.
\par 11 Iar de toate, zilele lui Enos au fost nouă sute cinci ani și apoi a murit.
\par 12 Cainan a trăit o sută șaptezeci de ani și atunci i s-a născut Maleleil.
\par 13 După nașterea lui Maleleil, Cainan a mai trăit șapte sute patruzeci de ani și i s-au născut fii și fiice.
\par 14 Iar de toate, zilele lui Cainan au fost nouă sute zece ani și apoi a murit.
\par 15 Maleleil a trăit o sută șaizeci și cinci de ani și atunci i s-a născut Iared.
\par 16 După nașterea lui Iared, Maleleil a mai trăit șapte sute treizeci de ani și i s-au născut fii și fiice.
\par 17 Iar de toate, zilele lui Maleleil au fost opt sute nouăzeci și cinci de ani și apoi a murit.
\par 18 Iared a trăit o sută șaizeci și doi de ani și atunci i s-a născut Enoh.
\par 19 După nașterea lui Enoh, Iared a mai trăit opt sute de ani și i s-au născut fii și fiice.
\par 20 Iar de toate, zilele lui Iared au fost nouă sute șaizeci și doi de ani și apoi a murit.
\par 21 Enoh a trăit o sută șaizeci și cinci de ani, și atunci i s-a născut Matusalem.
\par 22 Și a umblat Enoh înaintea lui Dumnezeu, după nașterea lui Matusalem, două sute de ani și i s-au născut fii și fiice.
\par 23 Iar de toate, zilele lui Enoh au fost trei sute șaizeci și cinci de ani.
\par 24 Și a plăcut Enoh lui Dumnezeu și apoi nu s-a mai aflat, pentru că l-a mutat Dumnezeu.
\par 25 Matusalem a trăit o sută optzeci Și șapte de ani și atunci i s-a născut Lameh.
\par 26 După nașterea lui Lameh, Matusalem a mai trăit șapte sute optzeci și doi de ani și i s-au născut fii și fiice.
\par 27 Iar de toate, zilele lui Matusalem, pe care le-a trăit, au fost nouă sute șaizeci și nouă de ani și apoi a murit.
\par 28 Lameh a trăit o sută optzeci și opt de ani și atunci i s-a născut un fiu.
\par 29 Și i-a pus numele Noe, zicând: "Acesta ne va mângâia în lucrul nostru și în munca mâinilor noastre, la lucrarea pământului, pe care l-a blestemat Domnul Dumnezeu!"
\par 30 Și a mai trăit Lameh, după nașterea lui Noe, cinci sute șaizeci și cinci de ani și i s-au născut fii și fiice.
\par 31 Iar de toate, zilele lui Lameh au fost șapte sute cincizeci și trei de ani și apoi a murit.
\par 32 Noe era de cinci sute de ani, când i s-au născut trei feciori: Sem, Ham și Iafet.

\chapter{6}

\par 1 Iar după ce au început a se înmulți oamenii pe pământ și li s-au născut fiice,
\par 2 Fiii lui Dumnezeu, văzând că fiicele oamenilor sunt frumoase, și-au ales dintre ele soții, care pe cine a voit.
\par 3 Dar Domnul Dumnezeu a zis: "Nu va rămâne Duhul Meu pururea în oamenii aceștia, pentru că sunt numai trup. Deci zilele lor să mai fie o sută douăzeci de ani!"
\par 4 În vremea aceea s-au ivit pe pământ uriași, mai cu seamă de când fiii lui Dumnezeu începuseră a intra la fiicele oamenilor și acestea începuseră a le naște fii: aceștia sunt vestiții viteji din vechime.
\par 5 Văzând însă Domnul Dumnezeu că răutatea oamenilor s-a mărit pe pământ și că toate cugetele și dorințele inimii lor sunt îndreptate la rău în toate zilele,
\par 6 I-a părut rău și s-a căit Dumnezeu că a făcut pe om pe pământ.
\par 7 Și a zis Domnul: "Pierde-voi de pe fața pământului pe omul pe care l-am făcut! De la om până la dobitoc și de la târâtoare până la păsările cerului, tot voi pierde, căci Îmi pare rău că le-am făcut".
\par 8 Noe însă a aflat har înaintea Domnului Dumnezeu.
\par 9 Iată viața lui Noe: Noe era om drept și neprihănit între oamenii timpului său și mergea pe calea Domnului.
\par 10 Și i s-au născut lui Noe trei fii: Sem, Ham și Iafet.
\par 11 Pământul însă se stricase înaintea feței lui Dumnezeu și se umpluse pământul de silnicii.
\par 12 Și a căutat Domnul Dumnezeu spre pământ și iată era stricat, căci tot trupul se abătuse de la calea sa pe pământ.
\par 13 Atunci a zis Domnul Dumnezeu către Noe: "Sosit-a înaintea feței Mele sfârșitul a tot omul, căci s-a umplut pământul de nedreptățile lor, și iată Eu îi voi pierde de pe pământ.
\par 14 Tu însă fă-ți o corabie de lemn de salcâm. În corabie să faci despărțituri și smolește-o cu smoală pe dinăuntru și pe din afară.
\par 15 Corabia însă să o faci așa: lungimea corăbiei să fie de trei sute de coți, lățimea ei de cincizeci de coți, iar înălțimea de treizeci de coți.
\par 16 Să faci corăbiei o fereastră la un cot de la acoperiș, iar ușa corăbiei să o faci într-o parte a ei. De asemenea să faci într-însa trei rânduri de cămări: jos, la mijloc și sus.
\par 17 Și iată Eu voi aduce asupra pământului potop de apă, ca să pierd tot trupul de sub cer, în care este suflu de viață, și tot ce este pe pământ va pieri.
\par 18 Iar cu tine voi face legământul Meu; și vei intra în corabie tu și împreună cu tine vor intra fiii tăi, femeia ta și femeile fiilor tăi.
\par 19 Să intre în corabie din toate animalele, din toate târâtoarele, din toate fiarele și din tot trupul, câte două, parte bărbătească și parte femeiască, ca să rămână cu tine în viață.
\par 20 Din toate soiurile de păsări înaripate după fel, din toate soiurile de animale după fel și din toate soiurile de târâtoare după fel, din toate să intre la tine câte două, parte bărbătească și parte femeiască, ca să rămână în viață împreună cu tine.
\par 21 Iar tu ia cu tine din tot felul de mâncare, cu care vă hrăniți; îngrijește-te ca să fie aceasta de mâncare pentru tine și pentru acelea".
\par 22 Și a început Noe lucrul și precum îi poruncise Domnul Dumnezeu așa a făcut.

\chapter{7}

\par 1 După aceea a zis Domnul Dumnezeu lui Noe: "Intră în corabie, tu și toată casa ta, căci în neamul acesta numai pe tine te-am văzut drept înaintea Mea.
\par 2 Să iei cu tine din toate animalele curate câte șapte perechi, parte bărbătească și parte femeiască, iar din animalele necurate câte o pereche, parte bărbătească și parte femeiască.
\par 3 De asemenea și din păsările cerului să iei: din cele curate câte șapte perechi, parte bărbătească și parte femeiască, iar din toate păsările necurate câte o pereche, parte bărbătească și parte femeiască, ca să le păstrezi soiul pentru tot pământul.
\par 4 Căci peste șapte zile Eu voi vărsa ploaie pe pământ, patruzeci de zile și patruzeci de nopți și am să pierd de pe fața pământului toate făpturile câte am făcut".
\par 5 Și a făcut Noe toate câte i-a poruncit Domnul Dumnezeu.
\par 6 Noe însă, când a venit asupra pământului potopul de apă, era de șase sute de ani.
\par 7 Și a intrat Noe în corabie și împreună cu el au intrat fiii lui, femeia lui și femeile fiilor lui, ca să scape de apele potopului.
\par 8 Din păsările curate și din păsările necurate, din animalele curate și din animalele necurate, din fiare și din toate cele ce se mișcă pe pământ
\par 9 Au intrat la Noe în corabie perechi, perechi, parte bărbătească și parte femeiască, cum poruncise Dumnezeu lui Noe.
\par 10 Iar după șapte zile au venit asupra pământului apele potopului.
\par 11 În anul șase sute al vieții lui Noe, în luna a doua, în ziua a douăzeci și șaptea a lunii acesteia, chiar în acea zi, s-au desfăcut toate izvoarele adâncului celui mare și s-au deschis jgheaburile cerului;
\par 12 Și a plouat pe pământ patruzeci de zile și patruzeci de nopți.
\par 13 În ziua aceasta a intrat Noe în corabie și împreună cu dânsul au intrat Sem, Ham și Iafet, fiii lui Noe, femeia lui Noe și cele trei femei ale fiilor lui.
\par 14 Din toate soiurile de fiare de pe pământ, din toate soiurile de animale, din toate soiurile de vietăți ce mișunau pe pământ, din toate soiurile de zburătoare, din toate păsările, din toate înaripatele
\par 15 Și din tot trupul, în care se afla duh de viață, au intrat cu Noe în corabie, perechi, perechi, parte bărbătească și parte femeiască.
\par 16 Și cele ce au intrat cu Noe în corabie din tot trupul au intrat parte bărbătească și parte femeiască, precum poruncise Dumnezeu lui Noe. Si a închis Domnul Dumnezeu corabia pe din afară.
\par 17 Potopul a ținut pe pământ patruzeci de zile și patruzeci de nopți și s-a înmulțit apa și a ridicat corabia și aceasta s-a înălțat deasupra pământului.
\par 18 Și a crescut apa mereu și s-a înmulțit foarte tare pe pământ și corabia se purta pe deasupra apei.
\par 19 Și a sporit apa pe pământ atât de mult, încât a acoperit toți munții cei înalți, care erau sub cer.
\par 20 Și a acoperit apa toți munții cei înalți, ridicându-se cu cincisprezece coți mai sus de ei.
\par 21 Și a murit tot trupul ce se mișca pe pământ: păsările, animalele, fiarele, toate vietățile ce mișunau pe pământ și toți oamenii.
\par 22 Toate cele de pe uscat, câte aveau suflare de viață în nările lor, au murit.
\par 23 Și așa s-a stins toată ființa care se afla pe fața a tot pământul, de la om până la dobitoc și până la târâtoare și până la păsările cerului, toate s-au stins de pe pământ, și a rămas numai Noe și ce era cu el în corabie.
\par 24 Iar apa a crescut mereu pe pământ, o sută cincizeci de zile.

\chapter{8}

\par 1 Dar și-a adus aminte Dumnezeu de Noe, de toate fiarele, de toate animalele, de toate păsările și de toate vietățile ce se mișcă, câte erau cu dânsul în corabie; și a adus Dumnezeu vânt pe pământ și a încetat apa de a mai crește.
\par 2 Atunci s-au încuiat izvoarele adâncului și jgheaburile cerului și a încetat ploaia din cer.
\par 3 După o sută cincizeci de zile, a început a se scurge apa de pe pământ și a se împuțina.
\par 4 Iar în luna a șaptea, în ziua a douăzeci și șaptea a lunii acesteia, s-a oprit corabia pe Munții Ararat.
\par 5 Apa a scăzut mereu până în luna a zecea; iar în ziua întâi a lunii a zecea s-au arătat vârfurile munților.
\par 6 După patruzeci de zile, a deschis Noe fereastra, pe care o făcuse la corabie,
\par 7 Și a dat drumul corbului, ca să vadă de a scăzut apa pe pământ. Acesta, zburând, nu s-a mai întors până ce a secat apa de pe pământ.
\par 8 Apoi, după el a trimis porumbelul, ca să vadă de s-a scurs apa de pe pământ.
\par 9 Porumbelul însă, negăsind loc de odihnă pentru picioarele sale, s-a întors la el, în corabie; căci era încă apă pe toată fața pământului. Și a întins Noe mâna și l-a apucat și l-a băgat la sine, în corabie.
\par 10 Și așteptând încă alte șapte zile, a dat iarăși drumul porumbelului din corabie,
\par 11 Și porumbelul s-a întors la el, spre seară, și iată avea în ciocul său o ramură verde de măslin. Atunci a cunoscut Noe că s-a scurs apa de pe fața pământului.
\par 12 Mai zăbovind încă alte șapte zile, iarăși a dat drumul porumbelului și el nu s-a mai întors.
\par 13 Iar în anul șase sute unu al vieții lui Noe, în ziua întâi a lunii întâi, secând apa de pe pământ, a ridicat Noe acoperișul corăbiei și a privit, și iată se zbicise fața pământului.
\par 14 Iar în luna a doua, la douăzeci și șapte ale lunii acesteia, pământul era uscat.
\par 15 Atunci a grăit Domnul Dumnezeu lui Noe și a zis:
\par 16 "Ieși din corabie tu și împreună cu tine femeia ta, fiii tăi și femeile fiilor tăi.
\par 17 Scoate de asemenea împreună cu tine toate vietățile, care sunt cu tine, și tot trupul, de la păsări și până la animale, și toate vietățile ce se mișcă pe pământ, ca să se împrăștie pe pământ, să se prăsească și să se înmulțească pe pământ".
\par 18 Atunci a ieșit Noe din corabie; și împreună cu el au ieșit fiii lui, femeia lui și femeile fiilor lui;
\par 19 Toate fiarele, toate animalele, toate păsările și toate câte se mișcă pe pământ, după felul lor, au ieșit din corabie.
\par 20 Apoi a făcut Noe un jertfelnic Domnului; și a luat din animalele cele curate și din toate păsările cele curate și le-a adus ardere de tot pe jertfelnic.
\par 21 Iar Domnul Dumnezeu a mirosit mireasmă bună și a zis Domnul Dumnezeu în inima Sa: "Am socotit să nu mai blestem pământul pentru faptele omului, pentru că cugetul inimii omului se pleacă la rău din tinerețile lui și nu voi mai pierde toate vietățile, cum am făcut.
\par 22 De acum, cât va trăi pământul, semănatul și seceratul, frigul și căldura, vara și iarna, ziua și noaptea nu vor mai înceta!"

\chapter{9}

\par 1 Și a binecuvântat Dumnezeu pe Noe și pe fiii lui și le-a zis: "Nașteți și vă înmulțiți și umpleți pământul și-l stăpâniți!
\par 2 Groază și frică de voi să aibă toate fiarele pământului; toate păsările cerului, tot ce se mișcă pe pământ și toți peștii mării; căci toate acestea vi le-am dat la îndemână.
\par 3 Tot ce se mișcă și ce trăiește să vă fie de mâncare; toate vi le-am dat, ca și iarba verde.
\par 4 Numai carne cu sângele ei, în care e viața ei, să nu mâncați.
\par 5 Căci Eu și sângele vostru, în care e viața voastră, îl voi cere de la orice fiară; și voi cere viața omului și din mâna omului, din mâna fratelui său.
\par 6 De va vărsa cineva sânge omenesc, sângele aceluia de mână de om se va vărsa, căci Dumnezeu a făcut omul după chipul Său.
\par 7 Voi însă nașteți și vă înmulțiți și vă răspândiți pe pământ și-l stăpâniți! "
\par 8 Și a mai grăit Dumnezeu cu Noe și cu fiii lui, care erau cu el, și a zis:
\par 9 "Iată Eu închei legământul Meu cu voi, cu urmașii voștri.
\par 10 Și cu tot sufletul viu, care este cu voi: cu păsările, cu animalele și cu toate fiarele pământului, care sunt cu voi, cu toate vietățile pământului câte au ieșit din corabie;
\par 11 Și închei acest legământ cu voi, că nu voi mai pierde tot trupul cu apele potopului și nu va mai fi potop, ca să pustiiască pământul".
\par 12 Apoi a mai zis iarăși Domnul Dumnezeu către Noe: "Iată, ca semn al legământului, pe care-l închei cu voi și eu tot sufletul viu ce este cu voi din neam în neam și de-a pururi,
\par 13 Pun curcubeul Meu în nori, ca să fie semn al legământului dintre Mine și pământ.
\par 14 Când voi aduce nori deasupra pământului, se va arăta curcubeul Meu în nori,
\par 15 Și-Mi voi aduce aminte de legământul Meu, pe care l-am încheiat cu voi și cu tot sufletul viu și cu tot trupul, și nu va mai fi apa potop, spre pierzarea a toată făptura.
\par 16 Va fi deci curcubeul Meu în nori și-l voi vedea, și-Mi voi aduce aminte de legământul veșnic dintre Mine și pământ și tot sufletul viu din tot trupul ce este pe pământ!"
\par 17 Și iarăși a zis Dumnezeu către Noe: "Acesta este semnul legământului, pe care Eu l-am încheiat între Mine și tot trupul care este pe pământ".
\par 18 Iar fiii lui Noe; care au ieșit din corabie, erau: Sem, Ham și Iafet. Iar Ham era tatăl lui Canaan.
\par 19 Aceștia sunt cei trei fii ai lui Noe și din aceștia s-au înmulțit oamenii pe pământ.
\par 20 Atunci a început Noe să fie lucrător de pământ și a sădit vie.
\par 21 A băut vin și, îmbătându-se, s-a dezvelit în cortul său.
\par 22 Iar Ham, tatăl lui Canaan, a văzut goliciunea tatălui său și, ieșind afară, a spus celor doi frați ai săi.
\par 23 Dar Sem și Iafet au luat o haină și, punând-o pe amândoi umerii lor, au intrat cu spatele înainte și au acoperit goliciunea tatălui lor; și fețele lor fiind întoarse înapoi, n-au văzut goliciunea tatălui lor.
\par 24 Trezindu-se Noe din amețeala de vin și aflând ce i-a făcut feciorul său cel mai tânăr,
\par 25 A zis: "Blestemat să fie Canaan! Robul robilor să fie la frații săi!"
\par 26 Apoi a zis: "Binecuvântat să fie Domnul Dumnezeul lui Sem; iar Canaan să-i fie rob!
\par 27 Să înmulțească Dumnezeu pe Iafet și să se sălășluiască acesta în corturile lui Sem, iar Canaan să-i fie slugă".
\par 28 Și a mai trăit Noe după potop trei sute cincizeci de ani.
\par 29 Iar de toate, zilele lui Noe au fost nouă sute cincizeci de ani și apoi a murit.

\chapter{10}

\par 1 Iată spița neamului fiilor lui Noe: Sem, Ham și Iafet, cărora li s-au născut fii după potop.
\par 2 Fiii lui Iafet au fost: Gomer, Magog, Madai, Iavan, Tubal, Meșeh și Tiras.
\par 3 Fiii lui Gomer au fost: Așchenaz, Rifat și Togarma.
\par 4 Fiii lui Iavan au fost: Elișa, Tarșiș, Chitim și Dodanim.
\par 5 Din aceștia s-au format mulțime de popoare, care s-au așezat în diferite țări, fiecare după limba sa, după neamul său și după nația sa.
\par 6 Iar fiii lui Ham au fost: Cuș, Mițraim, Put și Canaan.
\par 7 Fiii lui Cuș au fost: Șeba, Havila, Savta, Rama și Sabteca. Fiii lui Rama au fost: Șeba și Dedan.
\par 8 Cuș a mai născut de asemenea pe Nimrod; acesta a fost cel dintâi viteaz pe pământ.
\par 9 El a fost vânător vestit înaintea Domnului Dumnezeu; de aceea se și zice: "Vânător vestit ca Nimrod înaintea Domnului Dumnezeu".
\par 10 Împărăția lui, la început, o alcătuia: Babilonul, apoi Ereh, Acad și Calne din ținutul Senaar.
\par 11 Din pământul acela, el trecu în Asur, unde a zidit Ninive, cetatea Rehobot, Calah
\par 12 Și Resen, între Ninive și Calah. Aceasta e cetate mare.
\par 13 Din Mițraim s-au născut: Ludim, Anamim, Lehabim, Naftuhim,
\par 14 Patrusim, Casluhim - de unde au ieșit Filistenii - și Caftorim.
\par 15 Din Canaan s-au născut: Sidon, întâiul-născut al său, apoi Het,
\par 16 Iebuseu, Amoreu, Ghergheseu,
\par 17 Heveu, Archeu, Sineu,
\par 18 Arvadeu, Țemareu și Hamateu. Mai pe urmă neamurile canaaneiene s-au răspândit.
\par 19 Și ținuturile lor se întindeau de la Sidon, spre Gherara până la Gaza, iar de aici spre Sodoma, Gomora, Adma și Țeboim până spre Lasa.
\par 20 Aceștia sunt fiii lui Ham, după familii, limbă, țări și după nații.
\par 21 De asemenea i s-au născut fii și lui Sem, tatăl tuturor fiilor lui Eber și fratele mai mare al lui Iafet.
\par 22 Fiii lui Sem au fost: Elam, Asur, Arfaxad, Lud și Aram.
\par 23 Iar fiii lui Aram au fost: Uț, Hul, Gheter și Maș.
\par 24 Lui Arfaxad i s-a născut Cainan; lui Cainan i s-a născut Șelah; lui Șelah i s-a născut Eber;
\par 25 Iar lui Eber i s-au născut doi fii: numele unuia era Peleg, pentru că în zilele lui s-a împărțit pământul, și numele fratelui său era Ioctan.
\par 26 Lui Ioctan i s-au născut Almodad, Șalef, Hațarmavet și Ierah;
\par 27 Hadoram, Uzal și Dicla;
\par 28 Obal, Abimael și Șeba;
\par 29 Ofir, Havila și Iobab. Toți aceștia au fost fiii lui Ioctan.
\par 30 Sălașurile lor se întindeau de la Meșa spre Sefar, până la Muntele Răsăritului.
\par 31 Aceștia sunt fiii lui Sem după familii, după limbă, după țări și după nații.
\par 32 Acestea sunt neamurile, care se trag din fiii lui Noe, după familii și după nații, și dintr-înșii s-au răspândit popoarele pe pământ după potop.

\chapter{11}

\par 1 În vremea aceea era în tot pământul o singură limbă și un singur grai la toți.
\par 2 Purcezând de la răsărit, oamenii au găsit în țara Senaar un șes și au descălecat acolo.
\par 3 Apoi au zis unul către altul: "Haidem să facem cărămizi și să le ardem cu foc!" Și au folosit cărămida în loc de piatră, iar smoala în loc de var.
\par 4 Și au zis iarăși: "Haidem să ne facem un oraș și un turn al cărui vârf să ajungă la cer și să ne facem faimă înainte de a ne împrăștia pe fața a tot pământul!"
\par 5 Atunci S-a pogorât Domnul să vadă cetatea și turnul pe care-l zideau fiii oamenilor.
\par 6 Și a zis Domnul: "Iată, toți sunt de un neam și o limbă au și iată ce s-au apucat să facă și nu se vor opri de la ceea ce și-au pus în gând să facă.
\par 7 Haidem, dar, să Ne pogorâm și să amestecăm limbile lor, ca să nu se mai înțeleagă unul cu altul".
\par 8 Și i-a împrăștiat Domnul de acolo în tot pământul și au încetat de a mai zidi cetatea și turnul.
\par 9 De aceea s-a numit cetatea aceea Babilon, pentru că acolo a amestecat Domnul limbile a tot pământul și de acolo i-a împrăștiat Domnul pe toată fața pământului.
\par 10 Iată acum istoria vieții neamului lui Sem: Sem era de o sută de ani, când i s-a născut Arfaxad, la doi ani după potop.
\par 11 După nașterea lui Arfaxad, Sem a mai trăit cinci sute de ani și a născut fii și fiice și apoi a murit.
\par 12 Arfaxad a trăit o sută treizeci și cinci de ani și atunci i s-a născut Cainan. După nașterea lui Cainan, Arfaxad a mai trăit trei sute treizeci de ani și a născut fii și fiice și apoi a murit.
\par 13 Cainan a trăit o sută treizeci de ani și atunci i s-a născut Șelah. După nașterea lui Șelah, Cainan a mai trăit trei sute treizeci de ani și a născut fii și fiice și apoi a murit.
\par 14 Șelah a trăit o sută treizeci de ani și atunci i s-a născut Eber.
\par 15 Iar după nașterea lui Eber, Șelah a mai trăit trei sute treizeci de ani, și a născut fii și fiice și apoi a murit.
\par 16 Eber a trăit o sută treizeci și patru de ani și atunci i s-a născut Peleg.
\par 17 Iar după nașterea lui Peleg, Eber a mai trăit trei sute șaptezeci de ani și a născut fii și fiice și apoi a murit.
\par 18 Peleg a trăit o sută treizeci de ani și atunci i s-a născut Ragav.
\par 19 Iar după nașterea lui Ragav, Peleg a mai trăit două sute nouă ani, și a născut fii și fiice și apoi a murit.
\par 20 Ragav a trăit o sută treizeci și doi de ani și atunci i s-a născut Serug.
\par 21 Iar după nașterea lui Serug, Ragav a mai trăit două sute șapte ani și a născut fii și fiice și apoi a murit.
\par 22 Serug a trăit o sută treizeci de ani și atunci i s-a născut Nahor.
\par 23 Iar după nașterea lui Nahor, Serug a mai trăit două sute de ani și a născut fii și fiice și apoi a murit.
\par 24 Nahor a trăit șaptezeci și nouă de ani și atunci i s-a născut Terah.
\par 25 Iar după nașterea lui Terah, Nahor a mai trăit o sută două zeci și cinci de ani și a născut fii și fiice și apoi a murit.
\par 26 Terah a trăit șaptezeci de ani și atunci i s-au născut Avram, Nahor și Haran.
\par 27 Iar spița neamului lui Terah este aceasta: lui Terah i s-au născut Avram, Nahor și Haran; lui Haran i s-a născut Lot.
\par 28 Și a murit Haran înainte de Terah, tatăl său, în pământul de naștere, în Urul Caldeii.
\par 29 Iar Avram și Nahor și-au luat femei; numele femeii lui Avram era Sarai, iar numele femeii lui Nahor era Milca, fata lui Haran, tatăl Milcăi și al Iscăi.
\par 30 Sarai însă era stearpă și nu năștea copii.
\par 31 Și a luat Terah pe Avram, fiul său, și pe Lot, fiul lui Haran și nepotul său, și pe Sarai, nora sa, și femeia lui Avram, fiul său, și a plecat cu ei din Urul Caldeii, ca să meargă până în țara Canaanului; dar au mers până la Haran și s-au așezat acolo.
\par 32 De toate, zilele vieții lui Terah în pământul Haran au fost două sute cinci ani. Și a murit Terah în Haran.

\chapter{12}

\par 1 După aceea a zis Domnul către Avram: "Ieși din pământul tău, din neamul tău și din casa tatălui tău și vino în pământul pe care ți-l voi arăta Eu.
\par 2 Și Eu voi ridica din tine un popor mare, te voi binecuvânta, voi mări numele tău și vei fi izvor de binecuvântare.
\par 3 Binecuvânta-voi pe cei ce te vor binecuvânta, iar pe cei ce te vor blestema îi voi blestema; și se vor binecuvânta întru tine toate neamurile pământului".
\par 4 Deci a plecat Avram, cum îi zisese Domnul, și s-a dus și Lot cu el. Avram însă era de șaptezeci și cinci de ani, când a ieșit din Haran.
\par 5 Și a luat Avram pe Sarai, femeia sa, pe Lot, fiul fratelui său, și toate averile ce agonisiseră ei și toți oamenii, pe care-i aveau în Haran, și au ieșit, ca să meargă în țara Canaanului și au ajuns în Canaan.
\par 6 Apoi a străbătut Avram țara aceasta de-a lungul până la locul numit Sichem, până la stejarul Mamvri. Pe atunci trăiau în țara aceasta Canaaneii.
\par 7 Acolo S-a arătat Domnul lui Avram și i-a zis: "țara aceasta o voi da urmașilor tăi". Și a zidit Avram acolo un jertfelnic Domnului, Celui ce Se arătase.
\par 8 De acolo a pornit el spre muntele care e la răsărit de Betel, și și-a întins acolo cortul așa, încât Betelul era la apus, iar Hai, la răsărit. A zidit acolo un jertfelnic Domnului și s-a închinat Domnului, Celui ce i Se arătase.
\par 9 Apoi s-a ridicat Avram și de acolo și s-a îndreptat spre miazăzi.
\par 10 Pe atunci s-a făcut foamete în ținutul acela și s-a coborât Avram în Egipt, ca să locuiască acolo, pentru că se întețise foametea în ținutul acela.
\par 11 Când însă s-a apropiat Avram să intre în Egipt, a zis către Sarai, femeia sa: "Știu că ești femeie frumoasă la chip.
\par 12 De aceea, când te vor vedea Egiptenii, vor zice: "Aceasta-i femeia lui! Și mă vor ucide pe mine, iar pe tine te vor lăsa cu viață.
\par 13 Zi deci că-mi ești soră, ca să-mi fie și mie bine pentru trecerea ta și pentru trecerea ta să trăiesc și eu!"
\par 14 Iar după ce a sosit Avram în Egipt, au văzut Egiptenii că femeia lui e foarte frumoasă.
\par 15 Și au văzut-o și dregătorii lui Faraon și au lăudat-o înaintea lui Faraon și au dus-o în casa lui Faraon;
\par 16 Și pentru ea i-au făcut bine lui Avram și avea el oi, vite mari și asini, slugi și slujnice, catâri și cămile.
\par 17 Domnul însă a lovit cu bătăi mari și grele pe Faraon și casa lui, pentru Sarai, femeia lui Avram.
\par 18 Și chemând Faraon pe Avram, i-a zis: "Ce mi-ai făcut? De ce nu mi-ai spus că aceasta e soția ta?
\par 19 Pentru ce ai zis: Mi-e soră? Și eu am luat-o de femeie: Acum dar iată-ți femeia! Ia-ți-o și te du!"
\par 20 Și a dat Faraon poruncă oamenilor săi pentru Avram, ca să-l petreacă pe el și pe femeia lui și toate câte avea și pe Lot, care îl însoțea.

\chapter{13}

\par 1 Din Egipt, Avram cu femeia sa, cu Lot și cu toate câte avea, a pornit în părțile de miazăzi ale Canaanului.
\par 2 Avram insă era foarte bogat în vite, în argint și în aur.
\par 3 Și a, înaintat Avram pe unde venise, de la miazăzi spre Betel, până la locul unde fusese mai înainte cortul său, între Betel și Hai,
\par 4 Adică până la locul unde era jertfelnicul pe care-l ridicase el mai înainte, și acolo a chemat Avram numele Domnului.
\par 5 Și Lot, care umbla cu Avram, încă avea oi și vite mari și corturi.
\par 6 Însă pământul acela nu-i încăpea să stea împreună, căci averile lor erau multe și nu-i încăpea locul să trăiască împreună.
\par 7 De aceea se întâmplau certuri între păzitorii vitelor lui Avram și păzitorii vitelor lui Lot. Pe atunci locuiau în pământul acela Canaaneii și Ferezeii.
\par 8 Atunci a zis Avram către Lot: "Să nu fie sfadă între mine și tine, între păstorii mei și păstorii tăi, căci suntem frați.
\par 9 Iată, nu e oare tot pământul înaintea ta? Desparte-te dar de mine! Și de vei apuca tu la stânga, eu voi apuca la dreapta; iar de vei apuca tu la dreapta, eu voi apuca la stânga".
\par 10 Și ridicându-și Lot ochii, a privit toată câmpia Iordanului, care, înainte de a strica Domnul Sodoma și Gomora, toată până la Țoar era udată de apă, ca raiul Domnului, ca pământul Egiptului.
\par 11 Deci și-a ales Lot tot ținutul din preajma Iordanului și a apucat Lot spre răsărit; și așa s-au despărțit ei unul de altul.
\par 12 Avram a rămas să locuiască în pământul Canaan, iar Lot s-a sălășluit în cetățile din ținutul Iordanului și și-a întins corturile până la Sodoma.
\par 13 Iar oamenii Sodomei erau răi și tare păcătoși înaintea Domnului.
\par 14 Deci a zis Domnul către Avram, după ce s-a despărțit Lot de dânsul: "Ridică-ți ochii și, din locul în care ești acum, caută spre miazănoapte, spre miazăzi și răsărit și spre mare,
\par 15 Că tot pământul, cât îl vezi, ți-l voi da ție și urmașilor tăi pentru vecie.
\par 16 Voi face pe urmașii tăi mulți ca pulberea pământului; de va putea cineva număra pulberea pământului, va număra și pe urmașii tăi.
\par 17 Scoală și cutreieră pământul acesta în lung și în lat, că ți-l voi da ție și urmașilor tăi pentru vecie".
\par 18 Și ridicându-și Avram corturile, a venit și s-a așezat la stejarul Mamvri, care este în Hebron; și a zidit acolo un jertfelnic Domnului.

\chapter{14}

\par 1 Iar în zilele lui Amrafel, regele Senaarului, ale lui Arioc, regele Elasarului, ale lui Kedarlaomer, regele Elamului și ale lui Tidal, regele din Gutim,
\par 2 S-a întâmplat să facă aceștia război cu Bera, regele Sodomei, cu Birșa, regele Gomorei, cu Șinab, regele Admei, cu Șemeber, regele Țeboimului, și cu regele din Bela sau Țoar.
\par 3 Toți aceștia din urmă s-au adunat în valea Sidim, unde e acum Marea cea Sărată.
\par 4 Doisprezece ani stătuseră ei în robia lui Kedarlaomer, iar în anul al treisprezecelea s-au răzvrătit.
\par 5 Și în al patrusprezecelea an au venit Kedarlaomer și regii, care țineau cu el, și au bătut pe Refaimi la Așterot-Carnaim, pe Zuzimi la Ham, și pe Emimi la Șave-Chiriataim;
\par 6 Iar pe Horei i-a bătut la muntele lor Seir și până la El-Faran, care e lângă pustiu.
\par 7 Apoi, întorcându-se de acolo, au venit la Ain-Mișpat sau Cadeș și au bătut toată țara Amaleciților și pe toți Amoreii, care locuiau în Hațațon-Tamar.
\par 8 Atunci au ieșit regele Sodomei, regele Gomorei, regele Admei, regele Țeboimului și regele Belei sau Țoarului și s-au bătut în valea Sidim
\par 9 Cu Kedarlaomer, regele Elamului, cu Tidal, regele din Gutim, cu Amrafel, regele Senaarului și cu Arioc, regele Elasarului: patru regi împotriva a cinci.
\par 10 Valea Sidimului însă era plină de fântâni de smoală; și, fugind, regele Sodomei și regele Gomorei au căzut în ele, iar ceilalți au fugit în munți.
\par 11 Atunci biruitorii au luat toate averile Sodomei și Gomorei și toate bucatele lor și s-au dus.
\par 12 Când s-au dus, au luat de asemenea și pe Lot, nepotul lui Avram, care trăia în Sodoma, și toată averea lui.
\par 13 Dar venind unii din cei scăpați, au vestit pe Avram Evreul, care trăia pe atunci la stejarul lui Mamvri, pe Amoreul, fratele lui Eșcol și pe fratele lui Aner, care erau uniți cu Avram.
\par 14 Auzind Avram că Lot, rudenia sa, a fost luat în robie, a adunat oamenii săi de casă, trei sute optsprezece, și a urmărit pe vrăjmași până la Dan.
\par 15 Și năvălind asupra lor noaptea, el și oamenii săi i-au bătut și i-au alungat până la Hoba, care este în stânga Damascului.
\par 16 Și au întors toată prada luată din Sodoma, au întors și pe Lot, rudenia sa, averea lui, femeile și oamenii.
\par 17 Și când se întorcea Avram, după înfrângerea lui Kedarlaomer și a regilor uniți cu acela, i-a ieșit înainte regele Sodomei în valea Șave, care astăzi se cheamă Valea Regilor.
\par 18 Iar Melhisedec, regele Salemului, i-a adus pâine și vin. Melhisedec acesta era preotul Dumnezeului celui Preaînalt.
\par 19 Și a binecuvântat Melhisedec pe Avram și a zis: "Binecuvântat să fie Avram de Dumnezeu cel Preaînalt, Ziditorul cerului și al pământului.
\par 20 Și binecuvântat să fie Dumnezeul cel Preaînalt, Care a dat pe vrăjmașii tăi în mâinile tale!" Și Avram i-a dat lui Melhisedec zeciuială din toate.
\par 21 Iar regele Sodomei a zis către Avram: "Dă-mi oamenii, iar averile ia-le pentru tine!"
\par 22 Avram însă a răspuns regelui Sodomei: "Iată, îmi ridic mâna spre Domnul Dumnezeul cel Preaînalt, Ziditorul cerului și al pământului,
\par 23 Că nici o ață sau curea de încălțăminte nu voi lua din toate câte sunt ale tale, ca să nu zici: "Eu am îmbogățit pe Avram",
\par 24 Fără numai cele ce au mâncat tinerii și ceea ce se cuvine a se împărți aliaților mei, care au mers cu mine: Aner, Eșcol și Mamvri. Aceia să-și ia partea lor!"

\chapter{15}

\par 1 După acestea, fost-a cuvântul Domnului către Avram, noaptea, în vis, și a zis: "Nu te teme, Avrame, că Eu sunt scutul tău și răsplata ta va fi foarte mare!"
\par 2 Iar Avram a răspuns: "Stăpâne Doamne, ce ai să-mi dai? Că iată eu am să mor fără copii și cârmuitor în casa mea este Eliezer din Damasc".
\par 3 Apoi Avram a adăugat: "De vreme ce nu mi-ai dat fii, iată sluga mea va fi moștenitor după mine!"
\par 4 Și îndată s-a făcut cuvântul Domnului către el și a zis: "Nu te va moșteni acela, ci cel ce va răsări din coapsele tale, acela te va moșteni!"
\par 5 Apoi l-a scos afară și i-a zis: "Privește la cer și numără stelele, de le poți număra!" Și a adăugat: "Atât de mulți vor fi urmașii tăi!"
\par 6 Și a crezut Avram pe Domnul și i s-a socotit aceasta ca dreptate.
\par 7 Și i-a zis iarăși: "Eu sunt Domnul, Care te-a scos din Urul Caldeii, ca să-ți dau pământul acesta de moștenire".
\par 8 Și a zis Avram: "Stăpâne Doamne pe ce voi cunoaște că-l voi moșteni?"
\par 9 Iar Domnul i-a zis: "Gătește-Mi o junincă de trei ani, o capră de trei ani, un berbec de trei ani, o turturică și un pui de porumbel!"
\par 10 Și a luat Avram toate aceste animale, le-a tăiat în două și a pus bucățile una în fața alteia; iar păsările nu le-a tăiat.
\par 11 Și năvăleau păsările răpitoare asupra trupurilor, iar Avram le alunga.
\par 12 La asfințitul soarelui, a căzut peste Avram somn greu și iată l-a cuprins întuneric și frică mare.
\par 13 Atunci a zis Domnul către Avram: "Să știi bine că urmașii tăi vor pribegi în pământ străin, unde vor fi robiți și apăsați patru sute de ani;
\par 14 Dar pe neamul acela, căruia ei vor fi robi, îl voi judeca Eu și după aceea ei vor ieși să vină aici, cu avere multă.
\par 15 Iar tu vei trece la părinții tăi în pace și vei fi îngropat la bătrâneți fericite.
\par 16 Ei însă se vor întoarce aici, în al patrulea veac de oameni, căci nu s-a umplut încă măsura nelegiuirilor Amoreilor".
\par 17 Iar după ce a asfințit soarele și s-a făcut întuneric, iată un fum ca dintr-un cuptor și pară de foc au trecut printre bucățile acelea.
\par 18 În ziua aceea a încheiat Domnul legământ cu Avram, zicând: "Urmașilor tăi voi da pământul acesta de la râul Egiptului până la râul cel mare al Eufratului;
\par 19 Voi da pe Chenei, pe Chenezei, pe Chedmonei,
\par 20 Pe Hetei, pe Ferezei, pe Refaimi,
\par 21 Pe Amorei, pe Canaanei, pe Hevei, pe Gherghesei și pe Iebusei".

\chapter{16}

\par 1 Sarai însă, femeia lui Avram, nu-i năștea. Dar avea ea o slujnică egipteancă, al cărei nume era Agar.
\par 2 Atunci a zis Sarai către Avram: "Iată m-a închis Domnul, ca să nu nasc. Intră dar la slujnica mea; poate vei dobândi copii de la ea!" Și a ascultat Avram vorba Saraii.
\par 3 A luat deci Sarai, femeia lui Avram, pe Agar egipteanca, slujnica sa, la zece ani după venirea lui Avram în pământul Canaan, și a dat-o de femeie lui Avram, bărbatul său.
\par 4 Și a intrat acesta la Agar și ea a zămislit; și văzând că a zămislit, ea a început a disprețui pe stăpâna sa.
\par 5 Atunci a zis Sarai către Avram: "Nedreptate mi se face de către tine. Eu ti-am dat pe slujnica mea la sân, iar ea, văzând că a zămislit, a început să mă disprețuiască. Dumnezeu să judece între mine și între tine!"
\par 6 Iar Avram a zis către Sarai: "Iată, slujnica ta e în mâinile tale, fă cu ea ce-ți place!" și Sarai a necăjit-o și ea a fugit de la fața ei.
\par 7 Și a găsit-o îngerul Domnului la un izvor de apă în pustiu, la izvorul de lângă calea ce duce spre Sur.
\par 8 și i-a zis îngerul Domnului: "Agar, slujnica Saraii, de unde vii și unde te duci?" Iar ea a răspuns: "Fug de la fața Saraii, stăpâna mea".
\par 9 Și îngerul Domnului i-a zis iarăși: "Întoarce-te la stăpâna ta și te supune sub mâna ei!"
\par 10 Apoi i-a mai zis îngerul Domnului: "Voi înmulți pe urmașii tăi foarte tare, încât nu se vor putea număra din pricina mulțimii.
\par 11 Iată, tu ai rămas grea - îi zise îngerul Domnului - și vei naște un fiu și-i vei pune numele Ismael, pentru că a auzit Domnul suferința ta.
\par 12 Acela va fi ca un asin sălbatic între oameni; mâinile lui vor fi asupra tuturor și mâinile tuturor asupra lui, dar el va sta dârz în fața tuturor fraților lui".
\par 13 Și a numit Agar pe Domnul, Cel ce-i grăise, cu numele acesta: Ata-El-Roi (care se tâlcuiește: Tu ești Dumnezeu atotvăzător), căci zicea ea: "N-am văzut eu, oare, în față pe Cel ce m-a văzut?"
\par 14 De aceea se numește fântâna aceasta: Beer-Lahai-Roi (care se tâlcuiește: Izvorul Celui viu, Care m-a văzut), și se află între Cadeș și Bared.
\par 15 După aceea a născut Agar lui Avram un fiu și Avram a pus fiului său, pe care i-l născuse Agar, numele Ismael.
\par 16 Avram însă era de optzeci și șase de ani când i-a născut Agar pe Ismael.

\chapter{17}

\par 1 Iar când era Avram de nouăzeci și nouă de ani, i S-a arătat Domnul și i-a zis: "Eu sunt Dumnezeul cel Atotputernic; fă ce-i plăcut înaintea Mea și fii fără prihană;
\par 2 Și voi încheia legământ cu tine și te voi înmulți foarte, foarte tare".
\par 3 Atunci a căzut Avram cu fața la pământ, iar Dumnezeu a mai grăit și a zis:
\par 4 "Eu sunt și iată care-i legământul Meu cu tine: vei fi tată a mulțime de popoare,
\par 5 Și nu te vei mai numi Avram, ci Avraam va fi numele tău, căci am să te fac tată a mulțime de popoare.
\par 6 Am să te înmulțesc foarte, foarte tare, și am să ridic din tine popoare, și regi se vor ridica din tine.
\par 7 Voi pune legământul Meu între Mine și între tine și urmașii tăi, din neam în neam, să fie legământ veșnic, așa că Eu voi fi Dumnezeul tău și al urmașilor tăi de după tine.
\par 8 Și-ți voi da ție și urmașilor tăi pământul în carte pribegești acum ca străin, tot pământul Canaanului, ca moștenire veșnică, și vă voi fi Dumnezeu".
\par 9 Apoi a mai zis Dumnezeu lui Avraam: "Iar tu și urmașii tăi din neam în neam să păziți legământul Meu.
\par 10 Iar legământul dintre Mine și tine și urmașii tăi din neam în neam, pe care trebuie să-l păziți, este acesta: toți cei de parte bărbătească ai voștri să se taie împrejur.
\par 11 Să vă tăiați împrejur și acesta va fi semnul legământului dintre Mine și voi.
\par 12 În neamul vostru, tot pruncul de parte bărbătească, născut la voi în casă sau cumpărat cu bani de la alt neam, care nu-i din seminția voastră, să se taie împrejur în ziua a opta.
\par 13 Numaidecât să fie tăiat împrejur cel născut în casa ta sau cel cumpărat cu argintul tău și legământul Meu va fi însemnat pe trupul vostru, ca legământ veșnic.
\par 14 Iar cel de parte bărbătească netăiat împrejur, care nu se va tăia împrejur, în ziua a opta, sufletul acela se va stârpi din poporul său, căci a călcat legământul Meu".
\par 15 După aceea a zis iarăși Dumnezeu către Avraam: "Pe Sarai, femeia ta, să nu o mai numești Sarai, ci Sarra să-i fie numele.
\par 16 Și o voi binecuvânta și-ți voi da din ea un fiu; o voi binecuvânta și va fi mamă de popoare și regi peste popoare se vor ridica dintr-însa".
\par 17 Avraam a căzut atunci cu fața la pământ și a râs, zicând în sine: "E cu putință oare să mai aibă fiu cel de o sută de ani? Și Sarra cea de nouăzeci de ani e cu putință oare să mai nască?"
\par 18 Apoi a mai zis Avraam către Domnul: "O, Doamne, măcar Ismael să trăiască înaintea Ta!"
\par 19 Iar Dumnezeu a răspuns lui Avraam: "Adevărat, însăși Sarra, femeia ta, îți va naște un fiu și-i vei pune numele Isaac și Eu voi încheia cu el legământul Meu, legământ veșnic: să-i fiu Dumnezeu lui și urmașilor lui.
\par 20 Iată, te-am ascultat și pentru Ismael, și iată îl voi binecuvânta, îl voi crește și-l voi înmulți foarte, foarte tare; doisprezece voievozi se vor naște din el și voi face din el popor mare.
\par 21 Dar legământul Meu îl voi încheia cu Isaac, pe care-l va naște Sarra la anul pe vremea aceasta!"
\par 22 Încetând apoi Dumnezeu de a mai vorbi cu Avraam, S-a înălțat de la el.
\par 23 Atunci a luat Avraam pe Ismael, fiul său, pe toți cei născuți în casa sa, pe toți cei cumpărați cu argintul său și pe toți oamenii de parte bărbătească din casa lui Avraam și i-a tăiat împrejur, chiar în ziua aceea, cum îi poruncise Dumnezeu.
\par 24 Și era Avraam de nouăzeci și nouă de ani, când s-a tăiat împrejur.
\par 25 Iar Ismael, fiul său, era de treisprezece ani, când s-a tăiat împrejur.
\par 26 Avraam și Ismael, fiul său, au fost tăiați împrejur în aceeași zi.
\par 27 Și cu ei au fost tăiați împrejur toți cei de parte bărbătească din casa lui Avraam, născuți în casa lui sau cumpărați cu argint de la cei de alt neam.

\chapter{18}

\par 1 Apoi Domnul S-a arătat iarăși lui Avraam la stejarul Mamvri, într-o zi pe la amiază, când ședea el în ușa cortului său.
\par 2 Atunci ridicându-și ochii săi, a privit și iată trei Oameni stăteau înaintea lui; și cum l-a văzut, a alergat din pragul cortului său în întâmpinarea Lor și s-a închinat până la pământ.
\par 3 Apoi a zis: "Doamne, de am aflat har înaintea Ta, nu ocoli pe robul Tău!
\par 4 Se va aduce apă să Vă spălați picioarele și să Vă odihniți sub acest copac.
\par 5 Și voi aduce pâine și veți mânca, apoi Vă veți duce în drumul Vostru, întrucât treceți pe la robul Vostru!" Zis-au Aceia: "Fă, precum ai zis!"
\par 6 După aceea a alergat Avraam în cort la Sarra și i-a zis: "Frământă degrabă trei măsuri de făină bună și fă azime!"
\par 7 Apoi Avraam a dat fuga la cireadă, a luat un vițel tânăr și gras și l-a dat slugii, care l-a gătit degrabă.
\par 8 Și a luat Avraam unt, lapte și vițelul cel gătit și le-a pus înaintea Lor și pe când Ei mâncau a stat și el alături de Ei sub copac.
\par 9 Și l-au întrebat Oamenii aceia: -Unde este Sarra, femeia ta?" Iar el, răspunzând, a zis: "Iată, în cort!"
\par 10 Zis-a Unul: "Iată, la anul pe vremea asta am să vin iar pe la tine și Sarra, femeia ta, va avea un fiu". Iar Sarra a auzit din ușa cortului, de la spatele lui.
\par 11 Avraam și Sarra însă erau bătrâni, înaintați în vârstă, și Sarra nu mai era în stare să zămislească.
\par 12 Și a râs Sarra în sine și și-a zis: "Să mai am eu oare această mângâiere acum, când am îmbătrânit și când e bătrân și stăpânul meu?"
\par 13 Atunci a zis Domnul către Avraam: "Pentru ce a râs Sarra în sine și a zis: "Oare cu adevărat voi naște, bătrână cum sunt?"
\par 14 Este oare ceva cu neputință la Dumnezeu? La anul pe vremea aceasta am să vin pe la tine și Sarra va avea un fiu!"
\par 15 Iar Sarra a tăgăduit, zicând: "N-am râs", căci se înspăimântase. Acela însă i-a zis: "Ba, ai râs!"
\par 16 Apoi S-au sculat Oamenii aceia de acolo și S-au îndreptat spre Sodoma și Gomora și s-a dus și Avraam cu Ei, ca să-I petreacă.
\par 17 Domnul însă a zis: "Tăinui-voi Eu oare de Avraam, sluga Mea, ceea ce voiesc să fac?
\par 18 Din Avraam cu adevărat se va ridica un popor mare și tare și printr-însul se vor binecuvânta toate neamurile pământului,
\par 19 Că l-am ales, ca să învețe pe fiii și casa sa după sine să umble în calea Domnului și să facă judecată și dreptate; pentru ca să aducă Domnul asupra lui Avraam toate câte i-a făgăduit".
\par 20 Zis-a deci Domnul: "Strigarea Sodomei și a Gomorei e mare și păcatul lor cumplit de greu.
\par 21 Pogorî-Mă-voi deci să văd dacă faptele lor sunt cu adevărat așa cum s-a suit până la Mine strigarea împotriva lor, iar de nu, să știu".
\par 22 De acolo doi din Oamenii aceia, plecând, S-au îndreptat spre Sodoma, în vreme ce Avraam stătea încă înaintea Domnului.
\par 23 Și apropiindu-se Avraam, a zis: "Pierde-vei, oare, pe cel drept ca și pe cel păcătos, încât să se întâmple celui drept ce se întâmplă celui nelegiuit?
\par 24 Poate în cetatea aceea să fie cincizeci de drepți: pierde-i-vei, oare, și nu vei cruța tot locul acela pentru cei cincizeci de drepți, de se vor afla în cetate?
\par 25 Nu se poate ca Tu să faci una ca asta și să pierzi pe cel drept ca și pe cel fără de lege și să se întâmple celui drept ce se întâmplă celui necredincios! Departe de Tine una ca asta! Judecătorul a tot pământul va face, oare, nedreptate?"
\par 26 Zis-a Domnul: "De se vor găsi în cetatea Sodomei cincizeci de drepți, voi cruța pentru ei toată cetatea și tot locul acela".
\par 27 Și răspunzând Avraam, a zis: "Iată, cutez să vorbesc Stăpânului meu, eu, care sunt pulbere și cenușă!
\par 28 Poate că lipsesc cinci din cincizeci de drepți; poate să fie numai patruzeci și cinci; pentru lipsa a cinci pierde-vei, oare, toată cetatea?" Zis-a Domnul: "Nu o voi pierde de voi găsi acolo patruzeci și cinci de drepți".
\par 29 Și a adăugat Avraam să grăiască Domnului și a zis: "Dar de se vor găsi acolo numai patruzeci de drepți?" Și Domnul a zis: "Nu o voi pierde pentru cei patruzeci!"
\par 30 Și a zis iarăși Avraam: "Să nu Se mânie Stăpânul meu de voi mai grăi: Dar de se vor găsi acolo numai treizeci de drepți?" Zis-a Domnul: "Nu o voi pierde de voi găsi acolo treizeci".
\par 31 Și a zis Avraam: "Iată, mai cutez să vorbesc Stăpânului meu! Poate că se vor găsi acolo numai douăzeci de drepți". Răspuns-a Domnul: "Nu o voi pierde pentru cei douăzeci".
\par 32 Și a mai zis Avraam: "Să nu se mânie Stăpânul meu de voi mai grăi încă o dată: Dar de se vor găsi acolo numai zece drepți?" Iar Domnul i-a zis: "Pentru cei zece nu o voi pierde".
\par 33 Și terminând Domnul de a mai grăi cu Avraam; S-a dus, iar Avraam s-a întors la locul său.

\chapter{19}

\par 1 Cei doi Îngeri au ajuns la Sodoma seara, iar Lot ședea la poarta Sodomei. Și văzându-I, Lot s-a sculat înaintea Lor și s-a plecat cu fața până la pământ
\par 2 Și a zis: "Stăpânii mei, abateți-vă pe la casa slugii Voastre, ca să rămâneți acolo; spălați-Vă picioarele, iar dimineață, sculându-Vă, Vă veți duce în drumul Vostru". Ei însă au zis: "Nu, ci vom rămâne în uliță".
\par 3 Iar el I-a rugat stăruitor și S-au abătut la el și au intrat în casa lui. Atunci el Le-a gătit mâncare, Le-a copt azime și au mâncat.
\par 4 Dar mai înainte de a Se culca Ei, sodomenii, locuitorii cetății Sodoma, tot poporul din toate marginile; de la tânăr până la bătrân, au înconjurat casa,
\par 5 Și au chemat afară pe Lot și au zis către el: "Unde sunt Oamenii, Care au intrat să mâie la tine? Scoate-I ca să-I cunoaștem!"
\par 6 Și a ieșit Lot la ei dinaintea ușii și, închizând ușa după dânsul,
\par 7 A zis către ei: "Nu, frații mei, să nu faceți nici un rău.
\par 8 Am eu două fete, care n-au cunoscut încă bărbat; mai degrabă vi le scot pe acelea, să faceți cu ele ce veți vrea, numai Oamenilor acelora să nu le faceți nimic, de vreme ce au intrat Ei sub acoperișul casei mele!"
\par 9 Iar ei au zis către el: "Pleacă de aici! Ești un venetic și acum faci pe judecătorul? Mai rău decât Acelora iți vom face!" Și repezindu-se spre Lot, se apropiară să spargă ușa.
\par 10 Atunci Oamenii aceia, care găzduiau în casa lui Lot, întinzându-Și mâinile, au tras pe Lot în casă la Ei și au încuiat ușa;
\par 11 Iar pe oamenii, care erau la ușa casei, i-au lovit cu orbire de la mic până la mare, încât în zadar se chinuiau să găsească ușa.
\par 12 Apoi au zis cei doi Oameni către Lot: "Ai tu pe cineva din ai tăi aici? De ai fii, sau fiice, sau gineri, sau pe oricine altul în cetate, scoate-i din locul acesta,
\par 13 Că Noi avem să pierdem locul acesta, pentru că strigarea împotriva lor s-a suit înaintea Domnului și Domnul Ne-a trimis să-l pierdem".
\par 14 Atunci a ieșit Lot și a grăit cu ginerii săi, care luaseră pe fetele lui, și le-a zis: "Sculați-vă și ieșiți din locul acesta, că va să piardă Domnul cetatea". Ginerilor însă li s-a părut că el glumește.
\par 15 Iar în revărsatul zorilor grăbeau îngerii pe Lot, zicând: "Scoală, ia-ți femeia și pe cele două fete ale tale, pe care le ai, și ieși, ca să nu pieri și tu pentru nedreptățile cetății!"
\par 16 Dar fiindcă el zăbovea, îngerii, din mila Domnului către el, l-au apucat de mână pe el și pe femeia lui și pe cele două fete ale lui
\par 17 Și, scoțându-l afară, unul din Ei a zis: "Mântuiește-ți sufletul tău! Să nu te uiți înapoi, nici să te oprești în câmp, ci fugi în munte, ca să nu pieri cu ei!
\par 18 Iar Lot a zis către Dânșii: "Nu, Stăpâne!
\par 19 Iată sluga Ta a aflat bunăvoință înaintea Ta și Tu ai făcut milă mare cu mine, mântuindu-mi viața; dar nu voi putea să fug până în munte, ca să nu mă ajungă primejdia și să nu mor.
\par 20 Iată cetatea aceasta este mai aproape; să fug acolo și să mă izbăvesc. Ea e mică și-mi voi scăpa acolo viața prin Tine!"
\par 21 Și i-a zis îngerul: "Iată, îți cinstesc fața și-ți împlinesc acest cuvânt, să nu pierd cetatea despre care grăiești.
\par 22 Grăbește dar și fugi acolo; că nu pot să fac nimic până nu vei ajunge tu acolo!" De aceea s-a și numit cetatea aceea Țoar.
\par 23 Când s-a ridicat soarele deasupra pământului, a intrat Lot în Țoar.
\par 24 Atunci Domnul a slobozit peste Sodoma și Gomora ploaie de pucioasă și foc din cer de la Domnul
\par 25 Și a stricat cetățile acestea, toate împrejurimile lor, pe toți locuitorii cetăților și toate plantele ținutului aceluia.
\par 26 Femeia lui Lot însă s-a uitat înapoi și s-a prefăcut în stâlp de sare.
\par 27 Iar Avraam s-a sculat dis-de-dimineață și s-a dus la locul unde stătuse înaintea Domnului
\par 28 Și, căutând spre Sodoma și Gomora și spre toate împrejurimile lor, a văzut ridicându-se de la pământ fumegare, ca fumul dintr-un cuptor.
\par 29 Dar, când a stricat Dumnezeu toate cetățile din părțile acelea, și-a adus aminte Dumnezeu de Avraam și a scos pe Lot afară din prăpădul cu care a stricat Dumnezeu cetățile, unde trăia Lot.
\par 30 Apoi a ieșit Lot din Țoar și s-a așezat în munte, împreună cu cele două fete ale sale, căci se temea să locuiască în Țoar, și a locuit într-o peșteră, împreună cu cele două fete ale sale.
\par 31 Atunci a zis fata cea mai mare către cea mai mică: "Tatăl nostru e bătrân și nu-i nimeni în ținutul acesta, care să intre la noi, cum e obiceiul pământului.
\par 32 Haidem dar să îmbătăm pe tatăl nostru cu vin și să ne culcăm cu el și să ne ridicăm urmași dintr-însul!"
\par 33 Și au îmbătat pe tatăl lor cu vin în noaptea aceea; și în noaptea aceea, intrând fata cea mai în vârstă, a dormit cu tatăl ei și acesta n-a simțit când s-a culcat și când s-a sculat ea.
\par 34 Iar a doua zi a zis cea mai în vârstă către cea mai tânără: "Iată, eu am dormit astă-noapte cu tatăl meu; să-l îmbătăm cu vin și în noaptea aceasta și să intri și tu să dormi cu el ca să ne ridicăm urmași din tatăl nostru!"
\par 35 Și l-au îmbătat cu vin și în noaptea aceasta și a intrat și cea mai mică și a dormit cu el; și el n-a știut când s-a culcat ea, nici când s-a sculat ea.
\par 36 Și au rămas amândouă fetele lui Lot grele de la tatăl lor.
\par 37 Și a născut cea mai mare un fiu, și i-a pus numele Moab, zicând: "Este din tatăl meu". Acesta e tatăl Moabiților, care sunt și astăzi.
\par 38 Și a născut și cea mai mică un fiu și i-a pus numele Ben-Ammi, zicând: "Acesta-i fiul neamului meu". Acesta e tatăl Amoniților, care sunt și astăzi.

\chapter{20}

\par 1 Apoi a plecat Avraam de acolo spre miazăzi și s-a așezat între Cadeș și Sur și a trăit o vreme în Gherara.
\par 2 Și a zis Avraam despre Sarra, femeia sa: "Mi-e soră", căci se temea să spună: "E femeia mea", ca nu cumva să-l ucidă locuitorii cetății aceleia din pricina ei. Iar Abimelec, regele Gherarei, a trimis și a luat pe Sarra.
\par 3 Dar noaptea în vis a venit Dumnezeu la Abimelec și i-a zis: "Iată, tu ai să mori pentru femeia, pe care ai luat-o, căci ea are bărbat".
\par 4 Abimelec însă nu se atinsese de ea și a zis: "Doamne, ucide-vei oare chiar și un om drept?
\par 5 Oare n-a zis el singur: "Mi-e soră?" Ba și ea mia zis: "Mi-e frate!" Eu cu inimă nevinovată și cu mâini curate am făcut aceasta".
\par 6 Iar Dumnezeu i-a zis în vis: "Și Eu știu că cu inimă nevinovată ai făcut aceasta și te-am ferit de a păcătui împotriva Mea; de aceea nu ți-am îngăduit să te atingi de ea.
\par 7 Acum însă dă înapoi femeia omului aceluia, că e prooroc, și se va ruga pentru tine și vei fi viu; iar de nu o vei da înapoi, să știi bine că ai să mori și tu și toți ai tăi!"
\par 8 Și, sculându-se Abimelec, a doua zi de dimineață, a chemat pe toți slujitorii săi și le-a povestit toate acestea în auz, și s-au spăimântat toți oamenii aceia foarte tare.
\par 9 Apoi a chemat Abimelec pe Avraam și i-a zis: "Ce mi-ai făcut tu? Cu ce ți-am greșit eu, de ai adus asupra mea și asupra țării mele așa păcat mare? Tu mi-ai făcut un lucru, care nu se cuvine a-l face!"
\par 10 Și a mai zis Abimelec către Avraam: "Ce ai socotit tu, de ai făcut una ca asta?"
\par 11 Răspuns-a Avraam: "Am socotit că prin ținutul acesta lipsește frica de Dumnezeu și voi fi omorât din pricina femeii mele.
\par 12 Cu adevărat ea mi-e soră după tată, dar nu știu după mamă, iar acum mi-e soție.
\par 13 Iar când m-a scos Dumnezeu din casa tatălui meu, ca să pribegesc, am zis către ea: "Să-mi faci acest bine, și, în orice loc vom merge, să zici de mine: "Mi-e frate!"
\par 14 Atunci a luat Abimelec o mie de sicli de argint, vite mari și mici, robi și roabe și a dat lui Avraam; și i-a dat înapoi și pe Sarra, femeia sa.
\par 15 Și a zis Abimelec către Avraam: "Iată, ținutul meu îți este la îndemână: locuiește unde îți place!"
\par 16 Iar către Sarra a zis: "Iată, dau fratelui tău o mie de sicli de argint, care vor fi ca un văl pe ochi pentru cei ce sunt împrejurul tău și pentru lumea toată. Și iată că acum ești socotită dreaptă!"
\par 17 Și s-a rugat Avraam lui Dumnezeu și Dumnezeu a vindecat pe Abimelec, pe femeia lui și pe roabele lui, și acestea au început a naște.
\par 18 Căci Domnul lovise cu stârpiciune toată casa lui Abimelec, pentru Sarra, femeia lui Avraam.

\chapter{21}

\par 1 Apoi a căutat Domnul spre Sarra, cum îi spusese, și i-a făcut Domnul Sarrei, cum îi făgăduise.
\par 2 Și a zămislit Sarra și a născut lui Avraam un fiu la bătrânețe, la vremea arătată de Dumnezeu.
\par 3 Și a pus Avraam fiului său, pe care i-l născuse Sarra, numele Isaac.
\par 4 Și Avraam a tăiat împrejur pe Isaac, fiul său, în ziua a opta, cum îi poruncise Dumnezeu.
\par 5 Avraam însă era de o sută de ani, când i s-a născut Isaac, fiul său,
\par 6 Iar Sarra a zis: "Râs mi-a pricinuit mie Dumnezeu; că oricine va auzi aceasta, va râde!"
\par 7 Și apoi a adăugat: "Cine ar fi putut spune lui Avraam că Sarra va hrăni prunci la sânul său? Și totuși i-am născut fiu la bătrânețile  sale!"
\par 8 Și crescând copilul, a fost înțărcat. Iar Avraam a făcut ospăț mare în ziua în care a fost înțărcat Isaac, fiul său.
\par 9 Văzând însă Sarra că fiul egiptencii Agar, pe care aceasta îl născuse lui Avraam, râde de Isaac, fiul ei,
\par 10 A zis către Avraam: "Izgonește pe roaba aceasta și pe fiul ei, căci fiul roabei acesteia nu va fi moștenitor cu fiul meu, Isaac!"
\par 11 Și s-au părut cuvintele acestea lui Avraam foarte grele pentru fiul său Ismael.
\par 12 Dumnezeu însă a zis către Avraam: "Să nu ți se pară grele cuvintele cele pentru prunc și pentru roabă; toate câte-ți va zice Sarra, ascultă glasul ei; pentru că numai cei din Isaac se vor chema urmașii tăi.
\par 13 Dar și pe fiul roabei acesteia îl voi face neam mare, pentru că și el este din sămânța ta".
\par 14 Atunci s-a sculat Avraam dis-de-dimineață; a luat pâine și un burduf cu apă și le-a dat Agarei; apoi, punându-i pe umeri copilul, a slobozit-o; și, plecând ea, a rătăcit prin pustiul Beer-Șeba.
\par 15 Când însă s-a sfârșit apa din burduf, a lepădat ea copilul sub un mărăcine.
\par 16 Și ducându-se, a șezut în preajma lui, ca la o bătaie de arc, căci își zicea: "Nu voiesc să văd moartea copilului meu!" Și, șezând ea acolo de o parte, și-a ridicat glasul și a plâns.
\par 17 Și a auzit Dumnezeu glasul copilului din locul unde era și îngerul lui Dumnezeu a strigat din cer către Agar și a zis: "Ce e, Agar? Nu te teme, că a auzit Dumnezeu glasul copilului din locul unde este!
\par 18 Scoală, ridică copilul și-l ține de mână, căci am să fac din el un popor mare!"
\par 19 Atunci i-a deschis Dumnezeu ochii și a văzut o fântână cu apă și, mergând, și-a umplut burduful cu apă și a dat copilului să bea.
\par 20 Și era Dumnezeu cu copilul și a crescut acesta, a locuit în pustiu, și s-a făcut vânător.
\par 21 A locuit deci Ismael în pustiul Faran și mama sa i-a luat femeie din țara Egiptului.
\par 22 În vremea aceea, Abimelec și Ahuzat, care luase pe nora lui, și Ficol, căpetenia oștirii lui, au zis către Avraam: "Dumnezeu e cu tine în toate câte faci.
\par 23 Jură-mi, deci, aici pe Dumnezeu că nu-mi vei face strâmbătate nici mie, nici fiului meu, nici neamului meu; ci, cum ți-am făcut eu bine ție, așa să-mi faci și tu mie și țării în care ești oaspete!"
\par 24 Răspuns-a Avraam: "Jur!"
\par 25 Dar a mustrat Avraam pe Abimelec pentru fântânile de apă, pe care i le răpiseră slugile lui Abimelec.
\par 26 Iar Abimelec i-a zis: "Nu știu cine ți-a făcut lucrul acesta; nici tu nu mi-ai spus nimic, nici eu n-am auzit decât astăzi".
\par 27 Și a luat Avraam oi și vite și a dat lui Abimelec și au încheiat amândoi legământ.
\par 28 Apoi Avraam a pus de o parte șapte mielușele.
\par 29 Iar Abimelec a zis către Avraam: "Ce sunt aceste șapte mielușele, pe care le-ai osebit?"
\par 30 Răspuns-a Avraam: "Aceste șapte mielușele să le iei de la mine, ca să-mi fie mărturie, că eu am săpat fântâna aceasta!"
\par 31 De aceea s-a și numit locul acela Beer-Șeba, pentru că acolo au jurat ei amândoi.
\par 32 Și după ce au făcut ei legământ la Beer-Șeba, s-a sculat Abimelec și Ahuzat, care luase pe nora lui, și Ficol, căpetenia oștirii lui, și s-au întors în țara Filistenilor.
\par 33 Iar Avraam a sădit o dumbravă la Beer-Șeba și a chemat acolo numele Domnului Dumnezeului celui veșnic.
\par 34 Și a mai trăit Avraam în țara Filistenilor zile multe, ca străin.

\chapter{22}

\par 1 După acestea, Dumnezeu a încercat pe Avraam și i-a zis: "Avraame, Avraame!" Iar el a răspuns: "Iată-mă!"
\par 2 Și Dumnezeu i-a zis: "Ia pe fiul tău, pe Isaac, pe singurul tău fiu, pe care-l iubești, și du-te în pământul Moria și adu-l acolo ardere de tot pe un munte, pe care ți-l voi arăta Eu!"
\par 3 Sculându-se deci Avraam dis-de-dimineață, a pus samarul pe asinul său și a luat cu sine două slugi și pe Isaac, fiul său; și tăind lemne pentru jertfă, s-a ridicat și a plecat la locul despre care-i grăise Dumnezeu.
\par 4 Iar a treia zi, ridicându-și Avraam ochii, a văzut în depărtare locul acela.
\par 5 Atunci a zis Avraam slugilor sale: "Rămâneți aici cu asinul, iar eu și copilul ne ducem până acolo și, închinându-ne, ne vom întoarce la voi".
\par 6 Luând deci Avraam lemnele cele pentru jertfă, le-a pus pe umerii lui Isaac, fiul său; iar el a luat în mâini focul și cuțitul și s-au dus amândoi împreună.
\par 7 Atunci a grăit Isaac lui Avraam, tatăl său, și a zis: "Tată!" Iar acesta a răspuns: "Ce este, fiul meu?" Zis-a Isaac: "Iată, foc și lemne avem; dar unde este oaia pentru jertfă?"
\par 8 Avraam însă a răspuns: "Fiul meu, va îngriji Dumnezeu de oaia jertfei Sale!" Și s-au dus mai departe amândoi împreună.
\par 9 Iar dacă au ajuns la locul, de care-i grăise Dumnezeu, a ridicat Avraam acolo jertfelnic, a așezat lemnele pe el și, legând pe Isaac, fiul său, l-a pus pe jertfelnic, deasupra lemnelor.
\par 10 Apoi și-a întins Avraam mâna și a luat cuțitul, ca să junghie pe fiul său.
\par 11 Atunci îngerul Domnului a strigat către el din cer și a zis: "Avraame, Avraame!" Răspuns-a acesta: "Iată-mă!"
\par 12 Iar îngerul a zis: "Să nu-ți ridici mâna asupra copilului, nici să-i faci vreun rău, căci acum cunosc că te temi de Dumnezeu și pentru mine n-ai cruțat nici pe singurul fiu al tău".
\par 13 Și ridicându-și Avraam ochii, a privit, și iată la spate un berbec încurcat cu coarnele într-un tufiș. Și ducându-se, Avraam a luat berbecul și l-a adus jertfă în locul lui Isaac, fiul său.
\par 14 Avraam a numit locul acela Iahve-ire, adică, Dumnezeu poartă de grijă și de aceea se zice astăzi: "În munte Domnul Se arată".
\par 15 Și a strigat a doua oară îngerul Domnului din cer către Avraam și a zis:
\par 16 "Juratu-M-am pe Mine însumi, zice Domnul, că de vreme ce ai făcut aceasta și n-ai cruțat nici pe singurul tău fiu, pentru Mine,
\par 17 De aceea te voi binecuvânta cu binecuvântarea Mea și voi înmulți foarte neamul tău, ca să fie ca stelele cerului și ca nisipul de pe țărmul mării și va stăpâni neamul tău cetățile dușmanilor săi;
\par 18 Și se vor binecuvânta prin neamul tău toate popoarele pământului, pentru că ai ascultat glasul Meu".
\par 19 Întorcându-se apoi Avraam la slugile sale, s-au sculat împreună și s-au dus la Beer-Șeba și a locuit Avraam acolo în Beer-Șeba.
\par 20 Iar după ce s-au petrecut acestea, i s-a vestit lui Avraam, spunându-i-se: "Iată Milca a născut și ea fii lui Nahor, fratele tău:
\par 21 Pe Uț, întâiul său născut, pe Buz, fratele acestuia și pe Chemuel, tatăl lui Aram;
\par 22 Pe Chesed, pe Hazo, pe Pildaș, pe Idlaf și pe Batuel.
\par 23 Iar lui Batuel i s-a născut Rebeca". Pe acești opt fii i-a născut Milca lui Nahor, fratele lui Avraam.
\par 24 Iar o țiitoare a lui, anume Reuma, i-a născut și ea pe Tebah, pe Gaham, pe Tahaș și pe Maaca.

\chapter{23}

\par 1 Sarra a trăit o sută douăzeci și șapte de ani. Aceștia sunt anii vieții Sarrei.
\par 2 Sarra a murit la Chiriat-Arba care e în vale, adică în Hebronul de astăzi, în țara Canaanului. Și a venit Avraam să plângă și să jelească pe Sarra.
\par 3 Apoi s-a dus Avraam de la moarta sa, a grăit cu fiii lui Het și a zis:
\par 4 "Eu sunt intre voi străin și pribeag; dați-mi dar în stăpânire un loc de mormânt la voi, ca să îngrop pe moarta mea".
\par 5 Iar fiii lui Het au răspuns și au zis către Avraam:
\par 6 "Nu, domnul meu, ci ascultă-ne: Tu aici la noi ești u: voievod al lui Dumnezeu. Deci, îngroapă-ți moarta în cel mai bun dintre locurile noastre de îngropare, că nici unul dintre noi nu te va opri să-ți îngropi moarta acolo".
\par 7 Atunci s-a sculat Avraam și s-a închinat poporului jării aceleia, adică fiilor lui Het.
\par 8 Și a grăit către dânșii Avraam și a zis: "Dacă voiți din suflet să-mi îngrop pe moarta mea de la ochii mei, atunci ascultați-mă și rugați pentru mine pe Efron, fiul lui Țohar,
\par 9 Ca să-mi dea peștera Macpela pe care o are în capătul țarinei lui, dar să mi-o dea pe bani gata, ca să o am aici la voi în stăpânire de veci pentru îngropare.
\par 10 Efron însă ședea atunci în mijlocul fiilor lui Het. Și a răspuns Efron Heteeanul lui Avraam, în auzul fiilor lui Het și al tuturor celor ce veniseră la porțile cetății lui, și a zis:
\par 11 "Nu, domnul meu, ascultă-mă pe mine: Eu îți dau țarina și peștera cea dintr-însa și ți-o dau în fala fiilor poporului; ți-o dau însă în dar. Îngroapă-ți pe moarta ta".
\par 12 Avraam însă s-a închinat înaintea poporului țării
\par 13 Și a grăit către Efron în auzul a tot poporul ținutului aceluia și a zis: "De binevoiești, ascultă-mă și ia de la mine prețul țarinei și voi îngropa acolo pe moarta mea".
\par 14 Răspuns-a Efron lui Avraam și i-a zis:
\par 15 "Ascultă, domnul meu, țarina prețuiește patru sute sicli de argint. Ce este aceasta pentru mine și pentru tine? Îngroapă-ți dar pe moarta ta!"
\par 16 Atunci, ascultând pe Efron, Avraam a cântărit lui Efron atâta argint, cât a spus el în auzul fiilor lui Het: patru sute sicli de argint, după prețul negustoresc.
\par 17 Și așa țarina lui Efron, care e lângă Macpela, în fața stejarului Mamvri, țarina și peștera din ea și toți pomii din țarină și tot ce era în hotarele ei de jur împrejur
\par 18 S-au dat lui Avraam moșie de veci, înaintea fiilor lui Het  și a tuturor celor ce se strânseseră la poarta cetății lui.
\par 19 După aceasta Avraam a îngropat pe Sarra, femeia sa, în peștera din țarina Macpela, care e în fața lui Mamvri sau a Hebronului, în Canaan.
\par 20 Astfel a trecut de la fiii lui Het la Avraam țarina și peștera cea din ea, ca loc de îngropare.

\chapter{24}

\par 1 Avraam era acum bătrân și vechi de zile și Domnul binecuvântase pe Avraam cu de toate.
\par 2 Atunci a zis Avraam către sluga cea mai bătrână din casa sa, care cârmuia toate câte avea: "Pune mâna ta sub coapsa mea
\par 3 Și jură-mi pe Domnul Dumnezeul cerului și pe Dumnezeul pământului că fiului meu Isaac nu-i vei lua femeie din fetele Canaaneilor, în mijlocul cărora locuiesc eu,
\par 4 Ci vei merge în țara mea, unde m-am născut eu, la rudele mele, și vei lua de acolo femeie lui Isaac, fiul meu".
\par 5 Iar sluga a zis către el: "Dar poate nu va vrea femeia să vină cu mine în pământul acesta; întoarce-voi, oare, pe fiul tău în pământul de unde ai ieșit?"
\par 6 Avraam însă a zis către el: "Ia seama să nu întorci pe fiul meu acolo!
\par 7 Domnul Dumnezeul cerului și Dumnezeul pământului, Cel ce m-a luat din casa tatălui meu și din pământul în care m-am născut, Care mi-a grăit și Care mi S-a jurat, zicând: ție-ți voi da pământul acesta și urmașilor tăi, Acela va trimite pe îngerul Său înaintea ta și vei lua femeie feciorului meu de acolo.
\par 8 Iar de nu va voi femeia aceea să vină cu tine în pământul acesta, vei fi slobod de jurământul meu, dar pe fiul meu să nu-l întorci acolo!"
\par 9 Și punându-și sluga mâna sub coapsa lui Avraam, stăpânul său, i s-a jurat pentru toate acestea.
\par 10 Apoi a luat sluga cu sine zece cămile din cămilele stăpânului său și tot felul de lucruri scumpe de ale stăpânului său și, sculându-se, s-a dus în Mesopotamia, în cetatea lui Nahor.
\par 11 Și, într-o zi, spre seară, când ies femeile să scoată apă, a poposit cu cămilele la o fântână, afară din cetate.
\par 12 Și a zis: "Doamne Dumnezeul stăpânului meu Avraam, scoate-mi-o în cale astăzi și fă milă cu stăpânul meu Avraam!
\par 13 Iată, eu stau la fântâna aceasta și fetele locuitorilor cetății au să iasă să scoată apă.
\par 14 Deci fata căreia îi voi zice: Pleacă urciorul tău să beau și care-mi va răspunde: "Bea! Ba și  cămilele toate le voi adăpa până se vor sătura", aceea să fie pe care Tu ai rânduit-o robului Tău Isaac și prin aceasta voi cunoaște că faci milă cu stăpânul meu Avraam.
\par 15 Dar nu sfârșise el încă a cugeta acestea în mintea sa, când iată că ieși cu urciorul pe umăr Rebeca, fecioara care se născuse lui Batuel, fiul Milcăi, femeia lui Nahor, fratele lui Avraam.
\par 16 Aceasta era foarte frumoasă la chip, fecioară, pe care nu o cunoscuse încă un bărbat. Și venind ea la fântână, și-a umplut urciorul și a pornit înapoi.
\par 17 Atunci sluga lui Avraam a alergat înaintea ei și i-a zis: "Dă-mi să beau puțină apă din urciorul tău!"
\par 18 Iar ea a zis: "Bea, domnul meu!" Și îndată și-a lăsat urciorul pe brațe și i-a dat să bea apă până a încetat de a mai bea.
\par 19 Apoi a zis: "Și cămilelor tale am să le scot apă până vor bea toate".
\par 20 Și îndată și-a deșertat urciorul în adăpătoare și a alergat iar la fântână să scoată apă și a adăpat toate cămilele.
\par 21 Iar omul acela se uita la ea cu mirare și tăcea, dorind să știe de i-a binecuvântat Domnul călătoria sau nu.
\par 22 Și dacă au încetat toate cămilele de a mai bea, a luat omul acela și i-a dat un inel de aur, în greutate de o jumătate siclu, și două brățări la mâinile ei, în greutate de zece sicli de aur.
\par 23 Apoi a întrebat-o și a zis: "A cui fată ești tu? Spune-mi, te rog, dacă se află în casa tatălui tău loc, ca să rămânem?"
\par 24 Iar ea i-a zis: "Sunt fata lui Batuel al Milcăi, pe care ea l-a născut lui Nahor".
\par 25 Apoi i-a mai zis: "Avem și paie și fân mult și la noi este și loc, ca să rămâneți".
\par 26 Atunci s-a plecat omul acela și s-a închinat Domnului și a zis:
\par 27 "Binecuvântat să fie Domnul Dumnezeul stăpânului meu Avraam, Care n-a părăsit pe stăpânul meu cu mila și bunăvoința Sa, de vreme ce m-a adus Domnul drept la casa fratelui stăpânului meu".
\par 28 Iar fata a alergat acasă la mama sa și a povestit toate acestea.
\par 29 Rebeca însă avea un frate, anume Laban. Și a alergat Laban afară, la fântână, la omul acela,
\par 30 Căci el văzuse inelul de aur și brățările la mâinile surorii sale Rebeca, și auzise vorbele Rebecăi, sora sa, care spusese: "Așa și așa mi-a vorbit omul acela!" Și ajungând la el, l-a găsit stând cu cămilele la fântână.
\par 31 Și i-a zis: "Intră, binecuvântatul Domnului! Pentru ce stai afară? Eu ți-am gătit casă și sălaș pentru cămilele tale!"
\par 32 Și a intrat omul acela în casă. Iar Laban a luat povara de pe cămile și le-a dat paie și fân, iar lui și oamenilor, care erau cu el, le-a dat apă, ca să-și spele picioarele.
\par 33 Apoi le-a adus de mâncare. Eliezer însă a zis: "Nu voi mânca până nu voi spune la ce am venit". Zis-a Laban: "Spune!"
\par 34 Atunci Eliezer a zis: "Eu sunt sluga lui Avraam.
\par 35 Domnul a binecuvântat foarte pe stăpânul meu și l-a mărit și i-a dat oi și boi, argint și aur, robi și roabe, cămile și asini.
\par 36 Iar Sarra, femeia stăpânului meu, fiind acum bătrână, a născut stăpânului meu un fiu, căruia el i-a dat toate câte are.
\par 37 Și m-a jurat stăpânul meu, zicând: "Să nu iei femeie feciorului meu din fetele Canaaneilor, în pământul cărora trăiesc,
\par 38 Ci să mergi la casa tatălui meu, la rudele mele și să iei de acolo femeie pentru feciorul meu!"
\par 39 Iar eu am zis către stăpânul meu: "Dar de nu va vrea femeia să vină cu mine?"
\par 40 El însă mi-a răspuns: "Domnul Dumnezeu, înaintea Căruia umblu, va trimite cu tine pe îngerul Său, va binecuvânta calea ta și vei lua femeie pentru feciorul meu din rudele mele și din casa tatălui meu.
\par 41 Atunci vei fi slobod de jurământul meu, când te vei duce la rudele mele și de nu ți-o vor da, vei fi dezlegat de jurământul meu.
\par 42 Deci, ajungând eu astăzi la fântână, am zis: "Doamne, Dumnezeul stăpânului meu Avraam, de este să mă faci să izbutesc în calea ce fac,
\par 43 Iată, eu stau la fântână; și fata căreia eu îi voi zice când va veni să scoată apă: "Dă-mi să beau puțină apă din urciorul tău!"
\par 44 Iar ea îmi va zice: "Bea și tu, și cămilele tale le voi adăpa", aceea să fie femeia, pe care Domnul a rânduit-o pentru Isaac, robul Său și fiul stăpânului meu, și prin aceasta voi cunoaște că Te milostivești spre stăpânul meu Avraam.
\par 45 Dar nu isprăvisem eu încă a grăi acestea în mintea mea, când iată a ieșit Rebeca, cu urciorul pe umăr; se pogorî la fântână și scoase apă, și eu i-am zis: "Dă-mi să beau!"
\par 46 Și ea și-a lăsat îndată urciorul de pe umăr, zicând: "Bea tu și cămilele tale le voi adăpa". Și am băut și mi-a adăpat și cămilele.
\par 47 Apoi am întrebat-o și am zis: "A cui fată ești tu? Spune-mi te rog!" Și ea a zis: "Sunt fata lui Batuel, fiul lui Nahor, pe care i l-a născut Milca". Atunci i-am dat un inel și brățări la mâini.
\par 48 După aceea m-am plecat și m-am închinat Domnului și am binecuvântat pe Domnul Dumnezeul stăpânului meu Avraam, Care m-a povățuit de-a dreptul, ca să iau pe fata fratelui stăpânului meu pentru fiul lui.
\par 49 Acum deci spuneți-mi de vreți să arătați milă și bunăvoință stăpânului meu; iar de nu, să caut alta în dreapta și în stânga".
\par 50 Și răspunzând Laban și Batuel au zis: "De la Domnul vine lucrul acesta și noi nu-ți putem spune nimic nici de bine, nici de rău.
\par 51 Iată, Rebeca este înaintea ta, ia-o și du-te și să fie soția fiului stăpânului tău, cum a grăit Domnul!"
\par 52 Și auzind cuvintele lor, sluga lui Avraam s-a închinat Domnului până la pământ.
\par 53 Apoi a scos sluga lucruri de argint și lucruri de aur și haine și le-a dat Rebecăi. Dat-a de asemenea daruri și fratelui și mamei ei.
\par 54 După aceea au mâncat și au băut, el și oamenii cei ce erau cu dânsul și au rămas acolo. Iar dacă s-a sculat dimineața, a zis: "Lăsați-mă să mă duc la stăpânul meu!"
\par 55 Iar fratele și mama Rebecăi au zis: "Să mai rămână fata cu noi măcar vreo zece zile și apoi te vei duce!"
\par 56 El însă le-a zis: "Nu mă zăboviți! Căci Domnul m-a făcut să izbutesc în calea mea; lăsați-mă să mă duc la stăpânul meu!"
\par 57 Răspuns-au ei: "Să chemăm copila și s-o întrebăm ce gânduri are ea".
\par 58 Și au chemat pe Rebeca și i-au zis: "Vrei să te duci oare cu omul acesta?" Și ea a zis: "Mă duc!"
\par 59 Atunci a lăsat Laban să plece Rebeca, sora sa, și doica ei, și sluga lui Avraam și cei ce erau cu el.
\par 60 Și au binecuvântat pe Rebeca și i-au zis: "Sora noastră, să se nască din tine mii și zeci de mii și să stăpânească urmașii tăi porțile vrăjmașilor lor!"
\par 61 Atunci, sculându-se Rebeca și slujnicele ei și suindu-se pe cămile, s-au dus cu omul acela, și sluga lui Avraam, luând pe Rebeca, a plecat.
\par 62 Isaac însă venise din Beer-Lahai-Roi, căci el locuia în părțile de miazăzi.
\par 63 Iar spre seară a ieșit Isaac la câmp să se plimbe și, ridicându-și ochii, a văzut cămilele venind.
\par 64 Rebeca însă, căutând, a văzut pe Isaac și s-a dat jos de pe cămilă
\par 65 Și a zis către slugă: "Cine este omul acela care vine pe câmp în întâmpinarea noastră?" Iar sluga i-a zis: "Acesta-i stăpânul meu!" Atunci ea și-a luat vălul și s-a acoperit.
\par 66 Și sluga povesti lui Isaac toate câte făcuse.
\par 67 Și a dus-o Isaac în cortul mamei sale Sarra și a luat pe Rebeca și aceasta s-a făcut femeia lui și a iubit-o. Și s-a mângâiat Isaac de pierderea mamei sale, Sarra.

\chapter{25}

\par 1 Avraam însă și-a mai luat o femeie cu numele Chetura.
\par 2 Ea i-a născut pe Zimran, Iocșan, Madan, Madian, Ișbac și pe Șuah.
\par 3 Lui Iocșan i s-au născut Șeba, Teman și Dedan. Iar fiii lui Dedan au fost: Raguil, Navdeel, Așurim, Letușim și Leumim.
\par 4 Iar fiii lui Madian au fost: Efa, Efer, Enoh, Abida și Eldaa. Aceștia toți au fost fiii Cheturei.
\par 5 Însă Avraam a dat toate averile sale fiului său Isaac.
\par 6 Iar fiilor țiitoarelor sale, Avraam le-a făcut daruri și, încă fiind el în viață, i-a trimis departe de la Isaac, fiul său, spre răsărit, în pământul Răsăritului.
\par 7 Zilele vieții lui Avraam, câte le-a trăit, au fost o sută șaptezeci și cinci de ani.
\par 8 Apoi, slăbind, Avraam a murit la bătrâneți adânci, sătul de zile și s-a adăugat la poporul său.
\par 9 Și l-au îngropat feciorii lui, Isaac și Ismael, în peștera Macpela, din țarina lui Efron, fiul lui Țohar Heteeanul, în fața stejarului Mamvri;
\par 10 Deci în țarina și în peștera pe care Avraam a cumpărat-o de la fiii lui Het, acolo sunt îngropați Avraam și Sarra, femeia lui.
\par 11 Iar după moartea lui Avraam, a binecuvântat Dumnezeu pe Isaac, fiul lui. Și locuia Isaac la Beer-Lahai-Roi.
\par 12 Iată acum și viața lui Ismael, fiul lui Avraam, pe care l-a născut lui Avraam egipteanca Agar, slujnica Sarrei;
\par 13 Și iată numele fiilor lui Ismael, după șirul nașterii lor: întâiul născut al lui Ismael a fost Nebaiot; după el urmează Chedar, Adbeel și Mibsam,
\par 14 Mișma, Duma și Masa,
\par 15 Hadad, Tema, Etur, Nafiș și Chedma.
\par 16 Aceștia sunt fiii lui Ismael și acestea sunt numele lor, după așezările lor și după taberele lor. Aceștia sunt cei doisprezece voievozi ai semințiilor lor.
\par 17 Iar anii vieții lui Ismael au fost o sută treizeci și șapte și, îmbătrânind, a murit și a trecut la părinții săi;
\par 18 Iar urmașii săi s-au întins de la Havila până la Sur, care este în fața Egiptului, pe drumul ce duce spre Asiria; și s-au sălășluit ei înaintea tuturor fraților lor.
\par 19 Iar spița neamului lui Isaac, fiul lui Avraam, este aceasta: lui Avraam i s-a născut Isaac.
\par 20 Isaac însă era de patruzeci de ani, când și-a luat de femeie pe Rebeca, fata lui Batuel Arameul din Mesopotamia și sora lui Laban Arameul.
\par 21 Și s-a rugat Isaac Domnului pentru Rebeca, femeia sa, că era stearpă; și l-a auzit Domnul și femeia lui Rebeca a zămislit.
\par 22 Dar copiii au început a se zbate în pântecele ei și ea a zis: "Dacă așa au să fie, atunci la ce mai am această sarcină?" Și s-a dus să întrebe pe Domnul.
\par 23 Domnul însă i-a zis: "În pântecele tău sunt două neamuri și două popoare se vor ridica din pântecele tău; un popor va ajunge mai puternic decât celălalt și cel mai mare va sluji celui mai mic!"
\par 24 Și i-a venit Rebecăi vremea să nască și iată erau în pântecele ei doi gemeni.
\par 25 Și cel dintâi care a ieșit era roșu și peste tot păros, ca o pielicică, și i-a pus numele Isav.
\par 26 După aceea a ieșit fratele acestuia, ținându-se cu mâna de călcâiul lui Isav; și i s-a pus numele Iacov. Isaac însă era de șaizeci de ani, când i s-au născut aceștia din Rebeca.
\par 27 Copiii aceștia au crescut și a ajuns Isav om iscusit la vânătoare, trăind pe câmpii; iar Iacov era om liniștit, trăind în corturi.
\par 28 Isaac iubea pe Isav, pentru că îi plăcea vânatul acestuia; iar Rebeca iubea pe Iacov.
\par 29 O dată însă a fiert Iacov linte, iar Isav a venit ostenit de la câmp.
\par 30 Și a zis Isav către Iacov: "Dă-mi să mănânc din această fiertură roșie, că sunt flămând!" De aceea Isav s-a mai numit și Edom.
\par 31 Iacov însă i-a răspuns lui Isav: "Vinde-mi mai întâi dreptul tău de întâi-născut!"
\par 32 Și Isav a răspuns: "Iată eu mor. La ce mi-e bun dreptul de întâi-născut?"
\par 33 Zisu-i-a Iacov: "Jură-mi-te acum!" Și i s-a jurat Isav și a vândut lui Iacov dreptul său de întâi-născut.
\par 34 Atunci Iacov a dat lui Isav pâine și fiertură de linte și acesta a mâncat și a băut, apoi s-a sculat și s-a dus. Și astfel a nesocotit Isav dreptul său de întâi-născut.

\chapter{26}

\par 1 Și a fost o foamete în țară, afară de foametea cea dintâi, care se întâmplase în zilele lui Avraam. Atunci s-a dus Isaac în Gherara, la Abimelec, regele Filistenilor.
\par 2 Atunci Domnul i S-a arătat și i-a zis: "Să nu te duci în Egipt, ci să locuiești în țară, unde-ți voi zice Eu.
\par 3 Locuiește în țara aceasta și Eu voi fi cu tine și te voi binecuvânta, că ție și urmașilor tăi voi da toate ținuturile acestea și-Mi voi împlini jurământul cu care M-am jurat lui Avraam, tatăl tău.
\par 4 Voi înmulți pe urmașii tăi ca stelele cerului și voi da urmașilor tăi toate ținuturile acestea; și se vor binecuvânta întru urmașii tăi toate popoarele pământului,
\par 5 Pentru că Avraam, tatăl tău, a ascultat cuvântul Meu și a păzit poruncile Mele, povețele Mele, îndreptările Mele și legile Mele!"
\par 6 De aceea s-a așezat Isaac în Gherara.
\par 7 Iar locuitorii ținutului aceluia l-au întrebat despre Rebeca, femeia sa, cine e și el a zis: "Aceasta este sora mea!", căci s-a temut să zică: "E femeia mea!", ca nu cumva să-l omoare oamenii locului aceluia din pricina Rebecăi, pentru că era frumoasă la chip.
\par 8 Dar după ce a trăit el acolo multă vreme, s-a întâmplat că Abimelec, regele Filistenilor, să se uite pe fereastră și să vadă pe Isaac jucându-se cu Rebeca, femeia sa.
\par 9 Atunci a chemat Abimelec pe Isaac și i-a zis: "Adevărat e că-i femeia ta? De ce dar ai zis: "Aceasta-i sora mea?" Și Isaac a răspuns: "Pentru că mă temeam să nu fiu omorât din pricina ei".
\par 10 Zisu-i-a Abimelec: "Pentru ce ne-ai făcut aceasta? Puțin a lipsit ca cineva din neamul meu să se fi culcat cu femeia ta și ne-ai fi făcut să săvârșim păcat".
\par 11 Apoi a dat Abimelec poruncă la tot poporul său, zicând: "Tot cel ce se va atinge de omul acesta și de femeia lui va fi vinovat morții".
\par 12 Și a semănat Isaac în pământul acela și a cules anul acela rod însutit. Domnul l-a binecuvântat,
\par 13 Și omul acela a ajuns bogat și a sporit tot mai mult, până ce a ajuns bogat foarte.
\par 14 Avea turme de oi, cirezi de vite și ogoare multe, încât îl pizmuiau Filistenii.
\par 15 Toate fântânile, pe care le săpaseră robii tatălui său, în zilele lui Avraam, tatăl său, Filistenii le-au stricat și le-au umplut cu pământ.
\par 16 Atunci a zis Abimelec către Isaac: "Du-te de la noi, că te-ai făcut mult mai tare decât noi!"
\par 17 Și s-a dus Isaac de acolo și, tăbărând în valea Gherara, a locuit acolo.
\par 18 Apoi a săpat Isaac din nou fântânile de apă, pe care le săpaseră robii lui Avraam, tatăl său, și pe care le astupaseră Filistenii după moartea lui Avraam, tatăl său, și le-a numit cu aceleași nume, cu care le numise Avraam, tatăl său.
\par 19 După aceea au mai săpat slugile lui Isaac și în valea Gherara și au aflat acolo izvor de apă bună de băut.
\par 20 Dar se certau ciobanii din Gherara cu ciobanii lui Isaac, zicând: "Apa este a noastră!" De aceea Isaac a pus fântânii aceleia numele Esec, din pricină că se sfădiseră pentru ea.
\par 21 Ducându-se apoi de acolo, Isaac a săpat altă fântână și se certau și de la aceasta. De aceea Isaac i-a pus numele Sitna.
\par 22 Apoi s-a mutat și de aici și a săpat altă fântână, pentru care nu s-au mai certat, și i-a pus numele Rehobot, căci își zicea: "Datu-ne-a astăzi Domnul loc larg și vom spori pe pământ".
\par 23 De aici Isaac s-a urcat către Beer-Șeba.
\par 24 În noaptea aceea i S-a arătat Domnul și i-a zis: "Eu sunt Dumnezeul lui Avraam, tatăl tău! Nu te teme, că Eu sunt cu tine și te voi binecuvânta și voi înmulți pe urmașii tăi, pentru Avraam, sluga Mea".
\par 25 Acolo a făcut Isaac jertfelnic și a chemat numele Domnului și își întinse acolo cortul său. Și au săpat acolo slugile lui Isaac o fântână, în valea Gherarei.
\par 26 Atunci au venit din Gherara la el: Abimelec și Ahuzat, care luase pe nora lui, și Ficol, căpetenia oștirii lui.
\par 27 Iar Isaac le-a zis: "La ce ați venit la mine, voi care mă urâți și n-ați alungat de la voi?"
\par 28 Iar ei au zis: "Am văzut bine că Domnul este cu tine și am zis să facem cu tine jurământ și să încheiem legământ cu tine,
\par 29 Ca tu să nu ne faci nici un rău, cum nici noi nu ne-am atins de tine, ci ti-am făcut bine și te-am scos de la noi cu pace; și acum ești binecuvântat de Domnul".
\par 30 Atunci Isaac le-a făcut ospăț și ei au mâncat și au băut.
\par 31 Sculându-se apoi a doua zi de dimineață, au jurat unul altuia. Și le-a dat drumul Isaac și ei s-au dus de la dânsul cu pace.
\par 32 Tot în ziua aceea venind slugile lui Isaac, l-au vestit de fântâna ce o săpaseră și au zis: "Am găsit apă!"
\par 33 Și a numit Isaac fântâna aceea Șibea, adică jurământ. De aceea se și numește cetatea aceea Beer-Șeba, adică fântâna jurământului, până în ziua de astăzi.
\par 34 Iar Isav, fiind acum de patruzeci de ani, și-a luat două femei: pe Iudit, fata lui Beeri Heteul, și pe Basemata, fata lui Elon Heteul.
\par 35 Dar ele amărau pe Isaac și pe Rebeca.

\chapter{27}

\par 1 Iar după ce a îmbătrânit Isaac și au slăbit vederile ochilor săi, a chemat pe Isav, pe fiul său cel mai mare, și i-a zis: "Fiul meu!" Zis-a acela: "Iată-mă!"
\par 2 Și Isaac a zis: "Iată, eu am îmbătrânit și nu știu ziua morții mele.
\par 3 Ia-ți dară uneltele tale, tolba și arcul, și ieși la câmp și adu-mi ceva vânat;
\par 4 Să-mi faci mâncare, cum îmi place mie, și adu-mi să mănânc, ca să te binecuvânteze sufletul meu până nu mor!"
\par 5 Rebeca însă a auzit ce a zis Isaac către fiul său Isav. S-a dus deci Isav la câmp să vâneze ceva pentru tatăl său;
\par 6 Iar Rebeca a zis către Iacov, fiul cel mai mic: "Iată, eu am auzit pe tatăl tău grăind cu Isav, fratele tău, și zicând:
\par 7 "Adu vânat și fă-mi o mâncare să mănânc și să te binecuvântez înaintea Domnului, până a nu muri".
\par 8 Acum dar, fiul meu, ascultă ce am să-ți poruncesc:
\par 9 Du-te la turmă, adu-mi de acolo doi iezi tineri și buni și voi face din ei mâncare, cum îi place tatălui tău;
\par 10 Iar tu o vei duce tatălui tău să mănânce, ca să te binecuvânteze tatăl tău înainte de a muri".
\par 11 Iacov însă a zis către Rebeca, mama sa: "Isav, fratele meu, e om păros, iar eu n-am păr.
\par 12 Nu cumva tatăl meu să mă pipăie și voi fi în ochii lui ca un înșelător și în loc de binecuvântare, voi atrage asupră-mi blestem".
\par 13 Zis-a mama sa: "Fiul meu, asupra mea să fie blestemul acela; ascultă numai povața mea și du-te și adu-mi iezii!"
\par 14 Atunci, ducându-se Iacov, a luat și a adus mamei sale iezii, iar mama sa a gătit mâncare, cum îi plăcea tatălui lui.
\par 15 Apoi a luat Rebeca haina cea mai frumoasă a lui Isav, fiul ei cel mai mare, care era la ea în casă, și a îmbrăcat pe Iacov, fiul ei cel mai mic;
\par 16 Iar cu pieile iezilor a înfășurat brațele și părțile goale ale gâtului lui.
\par 17 Și a dat mâncarea și pâinea ce gătise în mâinile lui Iacov, fiul său,
\par 18 Și acesta a intrat la tatăl său și a zis: "Tată!" Iar acela a răspuns: "Iată-mă! Cine ești tu, copilul meu?"
\par 19 Zis-a Iacov către tatăl său: "Eu sunt Isav, întâiul tău născut. Am făcut precum mi-ai poruncit; scoală și șezi de mănâncă din vânatul meu, ca să mă binecuvânteze sufletul tău!"
\par 20 Zis-a Isaac către fiul său: "Cum l-ai găsit așa curând, fiul meu?" Și acesta a zis: "Domnul Dumnezeul tău mi l-a scos înainte".
\par 21 Zis-a Isaac iarăși către Iacov: "Apropie-te să te pipăi, fiul meu, de ești tu fiul meu Isav sau nu".
\par 22 Și s-a apropiat Iacov de Isaac, tatăl său, iar acesta l-a pipăit și a zis: "Glasul este glasul lui Iacov, iar mâinile sunt mâinile lui Isav".
\par 23 Dar nu l-a cunoscut, pentru că mâinile lui erau păroase, ca mâinile fratelui său Isav; și l-a binecuvântat.
\par 24 Și a mai zis: "Tu oare ești fiul meu Isav?" Și Iacov a răspuns: "Eu".
\par 25 Zis-a Isaac: "Adu-mi și voi mânca din vânatul fiului meu,  ca să te binecuvânteze sufletul meu!" Și i-a adus și a mâncat; apoi i-a adus și vin și a băut.
\par 26 După aceea Isaac, tatăl său, i-a zis: "Apropie-te, fiule, și mă sărută!" â
\par 27 Atunci s-a apropiat Iacov și l-a sărutat. Și a simțit Isaac mirosul hainei lui și l-a binecuvântat și a zis: "Iată, mirosul fiului meu e ca mirosul unui câmp bogat, pe care l-a binecuvântat Domnul.
\par 28 Să-ți dea ție Dumnezeu din roua cerului și din belșugul pământului, pâine multă și vin.
\par 29 Să-ți slujească popoarele și căpeteniile să se închine înaintea ta; să fii stăpân peste frații tăi și feciorii mamei tale să ți se închine ție; cel ce te va blestema să fie blestemat și binecuvântat să fie cel ce te va binecuvânta!"
\par 30 Îndată ce a isprăvit Isaac de binecuvântat pe Iacov, fiul său, și cum a ieșit Iacov de la fața tatălui său Isaac, a venit și Isav cu vânatul lui.
\par 31 A făcut și el bucate și le-a adus tatălui său și  a zis către  tatăl său: "Scoală, tată, și mănâncă din vânatul fiului tău, ca să mă binecuvânteze sufletul tău!"
\par 32 Iar Isaac, tatăl său, i-a zis: "Cine ești tu?" Iar el a zis: "Eu sunt Isav, fiul tău cel întâi-născut!"
\par 33 Atunci s-a cutremurat Isaac cu cutremur mare foarte și a zis: "Dar cine-i acela, care a căutat și mi-a adus vânat și am mâncat de la el înainte de a veni tu și l-am binecuvântat și binecuvântat va fi?"
\par 34 Iar Isav, auzind cuvintele tatălui său Isaac, a strigat cu glas mare și foarte dureros și a zis către tatăl său: "Binecuvântează-mă și pe mine, tată!"
\par 35 Zis-a Isaac către el: "A venit fratele tău cu înșelăciune și a luat binecuvântarea ta".
\par 36 Iar Isav a zis: "Din pricină oare că-l cheamă Iacov, de aceea m-a înșelat de două ori? Deunăzi mi-a răpit dreptul de întâi-născut, iar acum mi-a răpit binecuvântarea mea". Apoi a zis Isav către tatăl său: "Nu mi-ai păstrat și mie binecuvântare, tată?"
\par 37 Răspuns-a Isaac și a zis lui Isav: "Iată, stăpân l-am făcut peste tine și pe toți frații lui i-am făcut lui robi; cu pâine și cu vin l-am dăruit. Dar cu tine ce să fac, fiul meu?"
\par 38 Și a zis Isav către tatăl său: "Tată, oare numai o binecuvântare ai tu? Binecuvântează-mă și pe mine, tată:" Și cum Isaac tăcea, Isav și-a ridicat glasul și a început a plânge.
\par 39 Atunci, răspunzând Isaac, tatăl lui, a zis către el: "Iată, locuința ta va fi un pământ mănos și cerul îți va trimite roua sa;
\par 40 Cu sabia ta vei trăi și vei fi supus fratelui tău; va veni însă vremea când te vei ridica și vei sfărâma jugul lui de pe grumazul tău".
\par 41 De aceea ura Isav pe Iacov pentru binecuvântarea cu care-l binecuvântase tatăl său. Și a zis Isav în cugetul său: "Se apropie zilele de jelire pentru tatăl meu; atunci am să ucid pe Iacov, fratele meu!"
\par 42 Dar i s-a spus Rebecăi cuvintele lui Isav, fiul cel mai mare, și ea a trimis de a chemat pe Iacov, fiul ei cel mai mic, și i-a zis: "Iată Isav, fratele tău, vrea să se răzbune pe tine, omorându-te.
\par 43 Acum dar, fiul meu, ascultă povața mea și, sculându-te, fugi la Haran, în Mesopotamia, la fratele meu Laban,
\par 44 Și stai la el câtva timp, până se va potoli mânia fratelui tău
\par 45 Și până va mai uita el ce i-ai făcut. Atunci voi trimite și te voi lua de acolo. Pentru ce să rămân eu într-o singură zi fără de voi amândoi?"
\par 46 Apoi Rebeca a zis către Isaac: "Sunt scârbită de viața mea, din pricina fetelor Heteilor. Dacă-și ia și Iacov femeie ca acestea, din fetele pământului acestuia, atunci la ce-mi mai e bună viața?"

\chapter{28}

\par 1 Atunci a chemat Isaac pe Iacov și l-a binecuvântat și i-a poruncit, zicând: "Să nu-ți iei femeie din fetele Canaaneilor;
\par 2 Ci scoală și mergi în Mesopotamia în casa lui Batuel, tatăl mamei tale, și-ți ia femeie de acolo, din fetele lui Laban, fratele mamei tale,
\par 3 Și Dumnezeul cel Atotputernic să te binecuvânteze, să te crească și să te înmulțească și să lăsară din tine popoare multe;
\par 4 Să-ți dea binecuvântarea lui Avraam, tatăl meu, ție și urmașilor tăi ca să stăpânești pământul ce-l locuiești acum și pe care l-a dat Dumnezeu lui Avraam,
\par 5 Și așa Isaac i-a dat drumul lui Iacov, iar acesta s-a dus în Mesopotamia, la Laban, fiul lui Batuel Arameul și fratele Rebecăi, mama lui Iacov și a lui Isav.
\par 6 Văzând însă Isav că Isaac a binecuvântat pe Iacov și l-a trimis în Mesopotamia, să-și ia femeie de acolo, pentru care l-a binecuvântat și i-a poruncit, zicând: "Să nu-ți iei femeie din fetele Canaaneilor",
\par 7 Și că Iacov a ascultat pe tatăl său și pe mama sa și s-a dus în Mesopotamia,
\par 8 Și înțelegând Isav că lui Isaac, tatăl său, nu-i plac fetele Canaaneilor,
\par 9 S-a dus la Ismael și, pe lângă cele două femei ale sale, și-a mai luat și pe Mahalat, fata lui Ismael, fiul lui Avraam, și sora lui Nebaiot.
\par 10 Iar Iacov ieșind din Beer-Șeba, s-a dus în Haran.
\par 11 Ajungând însă la un loc, a rămas să doarmă acolo, căci asfințise soarele. Și luând una din pietrele locului aceluia și punându-și-o căpătâi, s-a culcat în locul acela.
\par 12 Și a visat că era o scară, sprijinită pe pământ, iar cu vârful atingea cerul; iar îngerii lui Dumnezeu se suiau și se pogorau pe ea.
\par 13 Apoi S-a arătat Domnul în capul scării și i-a zis: "Eu sunt Domnul, Dumnezeul lui Avraam, tatăl tău, și Dumnezeul lui Isaac. Nu te teme! Pământul pe care dormi ți-l voi da ție și urmașilor tăi.
\par 14 Urmașii tăi vor fi mulți ca pulberea pământului și tu te vei întinde la apus și la răsărit, la miazănoapte și la miazăzi, și se vor binecuvânta întru tine și întru urmașii tăi toate neamurile pământului.
\par 15 Iată, Eu sunt cu tine și te voi păzi în orice cale vei merge; te voi întoarce în pământul acesta și nu te voi lăsa până nu voi împlini toate câte ți-am spus".
\par 16 Iar când s-a deșteptat din somnul său, Iacov a zis: "Domnul este cu adevărat în locul acesta și eu n-am știut!"
\par 17 Și, spăimântându-se Iacov, a zis: "Cât de înfricoșător este locul acesta! Aceasta nu e alta fără numai casa lui Dumnezeu, aceasta e poarta cerului!"
\par 18 Apoi s-a sculat Iacov dis-de-dimineață, a luat piatra ce și-o pusese căpătâi, a pus-o stâlp și a turnat pe vârful ei untdelemn.
\par 19 Iacov a pus locului aceluia numele Betel (casa lui Dumnezeu), căci mai înainte cetatea aceea se numea Luz.
\par 20 Și a făcut Iacov făgăduință, zicând: "De va fi Domnul Dumnezeu cu mine și mă va povățui în calea aceasta, în care merg eu astăzi, de-mi va da pâine să mănânc și haine să mă îmbrac;
\par 21 Și de mă voi întoarce sănătos la casa tatălui meu, atunci Domnul va fi Dumnezeul meu.
\par 22 Iar piatra aceasta, pe care am pus-o stâlp, va fi pentru mine casa lui Dumnezeu și din toate câte-mi vei da Tu mie, a zecea parte o voi da Ție".

\chapter{29}

\par 1 Sculându-se apoi, Iacov s-a dus în pământul fiilor Răsăritului, la Laban, fiul lui Batuel Arameul și fratele Rebecăi, mama lui Iacov și a lui Isav.
\par 2 Și căutând el o dată, iată în câmp o fântână, iar lângă ea trei turme de oi culcate; căci din fântâna aceea se adăpau turmele și pe gura fântânii era o piatră mare.
\par 3 Când se adunau acolo toate turmele, ciobanii prăvăleau piatra de pe gura fântânii și adăpau oile, apoi iar puneau piatra la locul ei pe gura fântânii.
\par 4 Deci a zis Iacov către păstori: "Fraților, de unde sunteți voi?" Iar ei au zis: "Noi suntem din Haran".
\par 5 Și el le-a zis: "Cunoașteți voi pe Laban, feciorul lui Nahor?" Răspuns-au aceia: "Îl cunoaștem".
\par 6 Zis-a iarăși Iacov: "E sănătos?" Și ei au zis: "Sănătos. Iată Rahila, fata lui, vine cu oile".
\par 7 Zis-a Iacov către ei: "Mai e încă mult din zi și nu e încă vremea să se adune turmele; adăpați oile și duceți-vă de le pașteți".
\par 8 Iar ei au zis: "Până nu se adună toți păstorii, nu putem să prăvălim piatra de pe gura fântânii, ca să adăpăm oile!"
\par 9 Încă grăind el cu ei, iată a venit Rahila, fiica lui Laban, cu oile tatălui său, căci ea păștea oile tatălui său.
\par 10 Văzând Iacov pe Rahila, fiica lui Laban, fratele mamei sale, și oile lui Laban, fratele mamei sale, s-a apropiat Iacov și a prăvălit piatra de pe gura fântânii și a adăpat oile lui Laban, fratele mamei sale.
\par 11 Și a sărutat Iacov pe Rahila și și-a ridicat glasul și a plâns.
\par 12 Apoi a spus Rahilei că-i rudă cu tatăl ei și că-i fiul Rebecăi. Iar ea a alergat și a spus tatălui său toate acestea.
\par 13 Auzind Laban de sosirea lui Iacov, fiul surorii sale, a alergat în întâmpinarea lui și, îmbrățișându-l, l-a sărutat și l-a adus în casa sa și el a povestit lui Laban toate.
\par 14 Iar Laban i-a zis: "Tu ești din oasele mele și din carnea mea". Și a stat Iacov la el o lună de zile.
\par 15 Atunci Laban a zis către Iacov: "Au doară îmi vei sluji în dar, pentru că îmi ești rudă? Spune-mi, care-ți va fi simbria?"
\par 16 Laban însă avea două fete: pe cea mai mare o chema Lia și pe cea mai mică o chema Rahila.
\par 17 Lia era bolnavă de ochi, iar Rahila era chipeșă la statură și tare frumoasă la față.
\par 18 Lui Iacov însă îi era dragă Rahila și a zis: "Îți voi sluji șapte ani pentru Rahila, fata ta cea mai mică".
\par 19 Zisu-i-a Laban: "Mai bine s-o dau după tine decât s-o dau după alt bărbat. Rămâi la mine!"
\par 20 Și a slujit Iacov pentru Rahila șapte ani și i s-a părut numai câteva zile, pentru că o iubea.
\par 21 Apoi a zis Iacov către Laban: "Dă-mi femeia, că mi s-au împlinit zilele să intru la ea".
\par 22 Atunci a chemat Laban pe toți oamenii locului aceluia și a făcut ospăț.
\par 23 Iar seara a luat Laban pe fiica sa Lia și a băgat-o înăuntru și a intrat Iacov la ea.
\par 24 Și Laban a dat pe roaba sa Zilpa, roabă fiicei sale Lia.
\par 25 Dar când s-a făcut ziuă, iată era Lia. Și a zis Iacov către Laban: "Pentru ce mi-ai făcut aceasta? Nu Îi-am slujit eu oare pentru Rahila? Pentru ce m-ai înșelat?"
\par 26 Răspuns-a Laban: "Aici la noi nu se pomenește să se mărite fata cea mai mică înaintea celei mai mari.
\par 27 Împlinește această săptămână de nuntă și-ți voi da-o și pe aceea, pentru slujba ce-mi vei mai face alți șapte ani!"
\par 28 Și a făcut Iacov așa: a împlinit săptămâna de nuntă și i-a dat Laban și pe Rahila, fiica sa, de femeie.
\par 29 Atunci a dat Laban pe roaba sa Bilha, roabă fiicei sale Rahila.
\par 30 A intrat deci Iacov și la Rahila și iubea el pe Rahila mai mult decât pe Lia. Apoi a mai slujit Iacov lui Laban alți șapte ani.
\par 31 Văzând însă Domnul Dumnezeu că Lia era disprețuită, a deschis pântecele ei, iar Rahila fu stearpă.
\par 32 A zămislit deci Lia și a născut lui Iacov un fiu, căruia i-a pus numele Ruben, zicând: "A căutat Domnul la smerenia mea și mi-a dat fiu; de acum mă va iubi bărbatul meu".
\par 33 Apoi a zămislit Lia iarăși și a născut lui Iacov al doilea fiu și a zis: "Auzit-a Domnul că nu sunt iubită și mi-a dat și pe acesta". Și i-a pus numele Simeon.
\par 34 Și iarăși a zămislit ea și a mai născut un fiu și a zis: "De acum se va lipi de mine bărbatul meu, căci i-am născut trei fii". De aceea i-a pus acestuia numele Levi.
\par 35 Și iarăși a zămislit și a mai născut un fiu și a zis: "Acum voi lăuda pe Domnul!" De aceea i-a pus numele Iuda. Apoi a încetat Lia de a mai naște.

\chapter{30}

\par 1 Iar Rahila, văzând că ea n-a născut lui Iacov nici un fiu, a prins pizmă pe sora sa și a zis lui Iacov: "Dă-mi copii, iar de nu, voi muri".
\par 2 Mâniindu-se însă Iacov pe Rahila, i-a zis: "Au doară eu sunt Dumnezeu, Care a stârpit rodul pântecelui tău?"
\par 3 Atunci Rahila a zis către Iacov: "Iată roaba mea Bilha; intră la ea și ea va naște pe genunchii mei și voi avea și eu copii printr-însa".
\par 4 Și i-a dat pe Bilha, roaba sa, de femeie și a intrat Iacov la ea;
\par 5 Iar Bilha, roaba Rahilei, a zămislit și a născut lui Iacov un fiu.
\par 6 Atunci Rahila a zis: "Dumnezeu mi-a făcut dreptate, a auzit glasul meu și mi-a dat fiu". De aceea i-a pus numele Dan.
\par 7 Și a zămislit iarăși Bilha, roaba Rahilei, și a mai născut un fiu lui Iacov;
\par 8 Iar Rahila a zis: "Luptă dumnezeiască m-am luptat cu sora mea, am biruit și am ajuns deopotrivă cu sora mea!" De aceea i-a pus numele Neftali.
\par 9 Lia însă, văzând că a încetat de a mai naște, a luat pe roaba sa Zilpa și a dat-o lui Iacov de femeie și el a intrat la ea;
\par 10 Zilpa, roaba Liei, a născut lui Iacov un fiu.
\par 11 Atunci a zis Lia: "Noroc" Și i-a pus numele Gad.
\par 12 Apoi iarăși a zămislit Zilpa, roaba Liei, și a născut lui Iacov alt fiu.
\par 13 Și a zis Lia: "Spre fericirea mea s-a născut, că mă vor ferici femeile!" Și i-a pus numele Așer.
\par 14 Iar pe vremea seceratului grâului s-a dus Ruben și, găsind în țarină mandragore, le-a adus la mama sa Lia. Rahila insă a zis către Lia, sora sa: "Dă-mi și mie din mandragorele fiului tău!"
\par 15 Iar Lia a zis: "Nu-ți ajunge că mi-ai luat bărbatul? Vrei să iei și mandragorele fiului meu?" Și Rahila a zis: "Nu așa, ci pentru mandragorele fiului tău, să se culce Iacov noaptea aceasta cu tine!"
\par 16 Venind Iacov seara de la câmp, i-a ieșit Lia înainte și i-a zis: "Să intri la mine astăzi, că te-am cumpărat cu mandragorele fiului meu!" Și în noaptea aceea s-a culcat Iacov cu ea.
\par 17 Și a auzit Dumnezeu pe Lia și ea a zămislit și a născut lui Iacov al cincilea fiu.
\par 18 Atunci a zis Lia: "Mi-a dat răsplată Dumnezeu pentru că am dat bărbatului meu pe roaba mea". Și a pus copilului numele Isahar, adică răsplată.
\par 19 Apoi a mai zămislit Lia încă o dată și a născut lui Iacov al șaselea fiu.
\par 20 Și a zis Lia: "Dar minunat mi-a dăruit Dumnezeu în timpul de acum! De acum bărbatul meu va ședea la mine, că i-am născut șase feciori". Și a pus copilului numele Zabulon.
\par 21 După aceea Lia a mai născut o fată și i-a pus numele Dina.
\par 22 Dar și-a adus aminte Dumnezeu și de Rahila și a auzit-o Dumnezeu și i-a deschis pântecele.
\par 23 Și zămislind, ea a născut lui Iacov un fiu; și a zis Rahila: "Ridicat-a Dumnezeu ocara de la mine!"
\par 24 Și a pus copilului numele Iosif, zicând: "Domnul îmi va mai da și alt fiu!"
\par 25 Iar după ce a născut Rahila pe Iosif, Iacov a zis către Laban: "Lasă-mă să plec, să mă duc la mine, în pământul meu.
\par 26 Dă-mi femeile mele și copiii mei, pentru care ți-am slujit, ca să mă duc, căci tu știi ce slujbă ți-am făcut".
\par 27 Laban însă i-a zis: "De am aflat har înaintea ta, mai rămâi la mine! Căci văd bine că Dumnezeu m-a binecuvântat prin venirea ta".
\par 28 Apoi a adăugat: "Spune simbria ce voiești și-ți voi da-o!"
\par 29 Iacov însă i-a răspuns: "Tu știi cum ți-am slujit și cum sunt vitele tale, de când am venit eu la tine;
\par 30 Căci erau puține când am venit eu, iar de atunci s-au înmulțit și te-a binecuvântat Dumnezeu prin venirea mea. Când însă am să lucrez eu și pentru casa mea?"
\par 31 Răspunsu-i-a Laban: "Ce să-ți dau?" Și Iacov a zis: "Să nu-mi dai nimic. Dar de faci ce-ți voi spune eu, voi mai paște și voi mai păzi oile tale.
\par 32 Să treacă astăzi toate oile tale pe dinaintea noastră și să despărțim din ele orice oaie pestriță sau tărcată sau neagră, iar dintre capre cele pestrițe sau tărcate: aceea să fie simbria mea.
\par 33 Credincioșia mea va răspunde pentru mine înaintea ta mâine, când vei veni să-mi statornicești simbria: tot ce nu va fi bălțat sau tărcat între caprele mele și tot ce nu va fi tărcat sau negru între oile mele se va socoti ca furat de mine".
\par 34 Zis-a Laban către el: "Bine, să fie cum zici tu!"
\par 35 Și a ales Iacov în ziua aceea țapii cei vărgați sau pestriți și toate caprele bălțate sau tărcate, toate câte erau cu cit de puțin alb, și toate oile tărcate sau negre și le-a dat în seama fiilor săi.
\par 36 Iar Laban a hotărât ca depărtarea între dânsul și oile lui Iacov să fie cale de trei zile. Și a rămas Iacov să pască celelalte oi ale lui Laban.
\par 37 După aceea și-a luat Iacov nuiele verzi de plop, de migdal și de paltin, și a crestat pe ele dungi albe, luând de pe nuiele fâșii de coajă până la albeața nuielelor.
\par 38 Apoi punea nuielele crestate în jgheaburile de adăpat, ca, venind să bea, oile să zămislească înaintea nuielelor din adăpători.
\par 39 Și zămisleau oile cum erau nuielele și fătau oile miei pestriți, tărcați și negri.
\par 40 Iar mieii aceștia îi alegea Iacov și punea înaintea oilor lui Laban numai tot ce era pestriț și tot ce era negru; dar turmele sale le ținea despărțite și nu le amesteca cu oile lui Laban.
\par 41 Afară de aceasta Iacov, când zămisleau oile cele bune, punea nuiele pestrițe în adăpători înaintea lor, ca să zămislească ele cum erau nuielele;
\par 42 Iar când zămisleau cele rele, nu le punea nuielele și așa cele ce se cuveneau lui Laban erau slabe, iar cele ce se cuveneau lui Iacov erau voinice.
\par 43 De aceea s-a îmbogățit omul acesta foarte, foarte tare, și avea mulțime de vite mărunte și vite mari, roabe și robi, cămile și asini.

\chapter{31}

\par 1 A auzit însă Iacov vorbele feciorilor lui Laban, care ziceau: "Iacov a luat toate câte avea tatăl nostru și din ale tatălui nostru și-a făcut toată bogăția aceasta".
\par 2 Și căutând Iacov la fața lui Laban, iată nu mai era față de el ca mai înainte.
\par 3 Atunci Domnul a zis către Iacov: "Întoarce-te în țara părinților tăi, în patria ta, și Eu voi fi cu tine!"
\par 4 Trimițând, deci, Iacov a chemat pe Rahila și pe Lia la câmp, unde erau turmele,
\par 5 Și le-a zis: "Văd eu că fața tatălui vostru nu mai e față de mine, ca mai înainte; dar Dumnezeul tatălui meu este cu mine.
\par 6 Voi înșivă știți că am slujit pe tatăl vostru cu toată inima;
\par 7 Iar tatăl vostru m-a înșelat și de zeci de ori mi-a schimbat simbria, Dumnezeu însă nu i-a îngăduit să-mi facă rău.
\par 8 Când zicea el: Cele pestrițe să fie simbria ta, toate oile fătau miei pestriți; iar când zicea el: Cele negre să-ți fie de simbrie, atunci toate oile fătau miei negri.
\par 9 Și așa a luat Dumnezeu toate vitele de la tatăl vostru și mi le-a dat mie.
\par 10 Iată o dată, pe vremea când intrau în călduri oile, mi-am ridicat ochii și am văzut în vis; și iată că țapii și berbecii, care săreau pe capre și pe oi, erau albi, vărgați și bălțați.
\par 11 Iar îngerul Domnului mi-a zis în vis: "Iacove!" Și eu am răspuns: "Ce este?"
\par 12 Zis-a el: "Ridică-ți ochii și privește: toți țapii și berbecii, care sar pe capre și pe oi, sunt vărgați, pestriți și bălțați, căci am văzut toate câte ți-a făcut Laban.
\par 13 Eu sunt Dumnezeul, Cel ce ți S-a arătat în Betel, unde Mi-ai turnat untdelemn pe stâlp și unde Mi-ai făcut făgăduință. Scoală deci acum, ieși din pământul acesta și mergi în pământul tău de naștere și Eu voi fi cu tine".
\par 14 Atunci Lia și Rahila i-au răspuns și au zis: "Mai avem noi oare parte și moștenire în casa tatălui nostru?
\par 15 Oare n-am fost noi socotite de el ca niște străine, fiindcă el ne-a vândut și a mâncat banii noștri?
\par 16 De aceea, toată averea pe care Dumnezeu a luat-o de la tatăl nostru este a noastră și a copiilor noștri. Fă dar acum toate câte ți-a zis Domnul!"
\par 17 Atunci s-a sculat Iacov și a urcat copiii și femeile sale pe cămile,
\par 18 A strâns toate turmele sale și toată bogăția sa, pe care o agonisise în Mesopotamia, și toate ale sale, ca să meargă la Isaac, tatăl său, în țara Canaanului.
\par 19 Iar Laban, ducându-se să-și tundă oile, Rahila a furat idolii tatălui său.
\par 20 Deci Iacov a înșelat pe Laban Arameul, căci nu l-a vestit că pleacă,
\par 21 Ci a fugit cu toate câte avea și, trecând Eufratul, s-a îndreptat spre Muntele Galaadului.
\par 22 Iar a treia zi i s-a dat de știre lui Laban Arameul, că Iacov a fugit.
\par 23 Atunci, luând Laban cu sine pe feciorii și pe rudele sale, a alergat după el cale de șapte zile și l-a ajuns la Muntele Galaadului.
\par 24 Dar Dumnezeu a venit la Laban Arameul noaptea în vis și i-a zis: "Ferește-te, nu cumva să vorbești lui Iacov nici de bine, nici de rău".
\par 25 Și a ajuns Laban pe Iacov. Iacov însă își așezase cortul său pe munte; și tot pe Muntele Galaad și l-a așezat și Laban cu rudele sale.
\par 26 Atunci a zis Laban către Iacov: "Ce ai făcut? Pentru ce mi-ai furat inima și mi-ai luat fetele, ca și cum le-ai fi robit cu sabia?
\par 27 Pentru ce ai fugit pe ascuns și m-ai înșelat, în loc să mă înștiințezi pe mine, care ți-aș fi dat drumul cu veselie și cu cântări din timpane și din harfă?
\par 28 Ba nu mi-ai îngăduit nici măcar să-mi sărut nepoții și fetele mele. Te-ai purtat, așadar, ca un om fără de minte.
\par 29 Și acum mâna mea cea puternică ar putea să-ți facă rău. Dar Dumnezeul tatălui tău mi-a vorbit ieri și mi-a zis: "Ferește-te, nu cumva să vorbești lui Iacov nici de bine, nici de rău!"
\par 30 Să zicem că ai plecat, pentru că cu mare aprindere doreai casa tatălui tău. Dar atunci de ce mi-ai furat dumnezeii mei?"
\par 31 Atunci răspunzând Iacov, a zis către Laban: "M-am temut, căci ziceam: Nu cumva să-ți iei fetele de la mine și toate ale mele.
\par 32 Dar la cine vei găsi idolii tăi, acela nu va mai trăi. Caută de față cu rudele noastre și ia tot ce vei găsi al tău la mine!" Iacov însă nu știa că Rahila, femeia sa, îi furase.
\par 33 A intrat atunci Laban în cortul lui Iacov, și în cortul Liei, și în cortul celor două roabe, și a căutat și n-a găsit nimic; apoi, ieșind din cortul Liei, a intrat și în cortul Rahilei.
\par 34 Rahila însă luase idolii și-i pusese sub samarul cămilei și ședea deasupra lor; și a scotocit Laban prin tot cortul Rahilei și n-a găsit nimic.
\par 35 Iar ea a zis către tatăl său: "Să nu se mânie domnul meu că nu mă pot scula înaintea ta, pentru că tocmai acum am necazul obișnuit al femeilor". Și mai scotocind Laban prin tot cortul, n-a găsit idolii.
\par 36 Atunci s-a mâniat Iacov și s-a plâns împotriva lui Laban. Și  începând a grăi, Iacov a zis lui Laban: "Care-i vina mea și care-i păcatul meu, de te înverșunezi împotriva mea?
\par 37 Dacă ai răscolit toate lucrurile din casa mea, găsit-ai, oare, ceva din ale casei tale? Arată aici înaintea rudeniilor tale și înaintea rudeniilor mele, ca să ne judece ele pe amândoi!
\par 38 Iată, douăzeci de ani am stat la tine: oile tale și caprele tale n-au lepădat; berbecii oilor tale nu ți i-am mâncat,
\par 39 Vite sfâșiate de fiare nu ți-am adus: acestea au fost paguba mea. Din mâna mea ai cerut ceea ce se furase în timpul zilei și în vremea nopții.
\par 40 Ziua eram mistuit de căldură, iar noaptea de frig și somnul nu se lipea de ochii mei.
\par 41 Așa mi-au fost cei douăzeci de ani în casa ta. Ți-am slujit paisprezece ani pentru cele două fete ale tale și șase ani pentru vitele tale, iar tu de zeci de ori mi-ai schimbat simbria.
\par 42 De n-ar fi fost cu mine Dumnezeul tatălui meu, Dumnezeul lui Avraam și frica de Isaac, tu acum m-ai fi alungat cu nimic. Necazul meu și munca mâinilor mele le-a văzut Dumnezeu și de aceea a mijlocit ieri pentru mine".
\par 43 Răspuns-a Laban și a zis către Iacov: "Aceste fete sunt fetele mele, acești copii sunt copiii mei, aceste vite sunt vitele mele, și toate câte le vezi sunt ale mele și ale fetelor mele. Cum dar aș fi putut eu să fac astăzi ceva împotriva lor, sau împotriva copiilor, pe care i-au născut ele?
\par 44 Haidem dar acum să facem amândoi, eu și tu, legământ, care să fie mărturie între mine și tine!" Iar Iacov i-a zis: "Iată, nu e nimeni cu noi; dar să știi că Dumnezeu este martor între mine și tine".
\par 45 Și a luat Iacov o piatră și a pus-o stâlp.
\par 46 Apoi a zis Iacov către frații săi: "Adunați pietre!" Și au adunat pietre și au făcut o movilă; și au mâncat și au băut acolo pe movilă. Apoi a zis Laban către dânsul: "Movila aceasta este astăzi mărturie între mine și între tine".
\par 47 Și Laban a numit-o în limba sa: Iegar-Sahaduta, adică movila mărturiei, iar Iacov i-a dat același nume, însă pe limba sa și i-a zis: Galaad.
\par 48 Apoi Laban a zis iarăși către Iacov: "Iată, movila aceasta și semnul ce am pus astăzi sunt mărturia legământului dintre mine și tine". De aceea i s-a pus și numele Galaad, adică movila mărturiei.
\par 49 Ba s-a mai numit ea și Mițpa, adică veghere, pentru că Laban a zis: "Să vegheze Domnul asupra mea și asupra ta, după ce ne vom despărți unul de altul.
\par 50 De te vei purta rău cu fetele mele, sau de-ți vei mai lua și alte femei, afară de fetele mele, nu mai e vorba de un om, care să vadă, ci ia aminte că între mine și între tine e martor Dumnezeu!"
\par 51 Și iarăși a zis Laban către Iacov: "Iată movila aceasta și stâlpul, pe care l-am pus între amândoi, este mărturie între mine și tine.
\par 52 Că nici eu nu voi trece spre tine și nici tu nu vei trece spre mine, de la această movilă, cu gând rău.
\par 53 Dumnezeul lui Avraam și Dumnezeul lui Nahor, Dumnezeul părinților lor să fie judecător între noi!" Iar Iacov a jurat pe Acela, de Care se temea Isaac, tatăl său.
\par 54 Apoi a junghiat Iacov ardere de tot pe munte și a chemat pe rudele sale să mănânce pâine. Și au mâncat pâine și s-au veselit în munte.
\par 55 Iar a doua zi s-a sculat Laban dis-de-dimineață și a sărutat pe nepoții săi și pe fetele sale și i-a binecuvântat. Apoi Laban a pornit să se întoarcă la locul său.

\chapter{32}

\par 1 După aceea Iacov s-a dus în calea sa. Și căutând, el a văzut oștirea lui Dumnezeu tăbărâtă, căci l-au întâmpinat îngerii lui Dumnezeu.
\par 2 Iacov însă, când i-a văzut, a zis: "Aceasta este tabăra lui Dumnezeu!" Și a pus locului aceluia numele Mahanaim, adică două tabere.
\par 3 Apoi a trimis Iacov soli înaintea sa, la fratele său Isav, în ținutul Seir din țara Edomului,
\par 4 Și le-a poruncit, zicând: "Așa să ziceți către domnul meu Isav: Așa grăiește robul tău Iacov: Am stat la Laban și am trăit la el până acum.
\par 5 Am boi și asini, oi, slugi și slujnice, și am trimis să vestească pe domnul meu Isav, ca să afle robul tău bunăvoință înaintea ta".
\par 6 Și întorcându-se la Iacov, i-au spus solii: "Am fost la fratele tău Isav și iată el vine în întâmpinarea ta cu patru sute de oameni".
\par 7 Iacov însă s-a spăimântat foarte și nu știa ce să facă. Și a împărțit oamenii, care erau cu el, boii, oile și cămilele în două tabere.
\par 8 Și a zis Iacov: "De va năvăli Isav asupra unei tabere și o va bate, va scăpa cealaltă tabără".
\par 9 Apoi Iacov a zis: "Dumnezeul tatălui meu Avraam și Dumnezeul tatălui meu Isaac, Doamne, Tu, Cel ce mi-ai zis: Întoarce-te în țara ta de naștere, și Eu îți voi face bine,
\par 10 Nu sunt vrednic de toate îndurările Tale și de toate binefacerile ce mi-ai arătat mie, robului Tău, că numai cu toiagul am trecut deunăzi Iordanul acesta, iar acum am două tabere;
\par 11 Izbăvește-mă dar din mâna fratelui meu, din mâna lui Isav, căci mă tem de el, ca nu cumva să vină și să mă omoare pe mine și pe aceste mame cu copii.
\par 12 Căci Tu ai zis: îți voi face bine și voi înmulți neamul tău ca nisipul mării, cât nu se va putea număra din pricina mulțimii".
\par 13 Și a rămas acolo în noaptea aceea. Apoi a luat din cele ce avea și a trimis dar fratelui său Isav:
\par 14 Două sute de capre și douăzeci de țapi, două sute de oi și douăzeci de berbeci,
\par 15 Treizeci de cămile mulgătoare cu mânjii lor, patruzeci de vaci și zece tauri, douăzeci de asine și zece asini.
\par 16 Și a dat fiecare din aceste turme deosebi în seama slugilor sale și a zis slugilor sale: "Treceți înaintea mea și să fie depărtate turmele una de alta".
\par 17 Celui dintâi i-a poruncit, zicând: "Când te va întâlni fratele meu Isav și te va întreba: Al cui ești tu și unde te duci, și ale cui sunt acestea, ce merg înaintea ta,
\par 18 Să zici: Ale robului tău Iacov; e dar trimis lui Isav, stăpânul meu. Iată vine și el după noi!"
\par 19 Așa a poruncit Iacov și slugii celei de a doua și celei de a treia și tuturor celor ce mergeau cu turmele, zicând: "Așa să spuneți lui Isav, când îl veți întâlni.
\par 20 Și să-i mai spuneți: Iată și el, robul tău Iacov, vine după noi". Căci își zicea: Voi îmblânzi fața lui cu darurile ce-mi merg înainte și numai după aceea voi vedea fața lui, și așa poate mă va primi.
\par 21 Și au pornit darurile înaintea lui, iar el a rămas noaptea aceea în tabără.
\par 22 Dar s-a sculat noaptea și luând pe cele două femei ale sale și pe cele două roabe și pe cei unsprezece copii ai săi, a trecut Iabocul prin vad.
\par 23 Iar după ce i-a luat și i-a trecut râul, a trecut și toate ale sale.
\par 24 Rămânând Iacov singur, s-a luptat Cineva cu dânsul până la revărsatul zorilor.
\par 25 Văzând însă că nu-l poate răpune Acela, S-a atins de încheietura coapsei lui și i-a vătămat lui Iacov încheietura coapsei, pe când se lupta cu el.
\par 26 Și i-a zis: "Lasă-Mă să plec, că s-au ivit zorile!" Iacov I-a răspuns: "Nu Te las până nu mă vei binecuvânta".
\par 27 Și l-a întrebat Acela: "Care îți este numele?" Și el a zis: "Iacov!"
\par 28 Zisu-i-a Acela: "De acum nu-ți va mai fi numele Iacov, ci Israel te vei numi, că te-ai luptat cu Dumnezeu și cu oamenii și ai ieșit biruitor!"
\par 29 Și a întrebat și Iacov, zicând: "Spune-mi și Tu numele Tău!" Iar Acela a zis: "Pentru ce întrebi de numele Meu? El e minunat!" Și l-a binecuvântat acolo.
\par 30 Și a pus Iacov locului aceluia numele Peniel, adică fața lui Dumnezeu, căci și-a zis: "Am văzut pe Dumnezeu în față și mântuit a fost sufletul meu! "
\par 31 Iar când răsărea soarele, trecuse de Peniel, dar el șchiopăta din pricina șoldului.
\par 32 De aceea fiii lui Israel până astăzi nu mănâncă mușchiul de pe șold, pentru că Cel ce S-a luptat a atins încheietura șoldului lui Iacov, în dreptul acestui mușchi.

\chapter{33}

\par 1 Atunci, ridicându-și ochii, Iacov a văzut pe Isav, fratele său, venind cu cei patru sute de oameni. Și a împărțit Iacov copiii Liei și ai Rahilei și ai celor două roabe.
\par 2 Și a pus pe cele două roabe cu copiii lor înainte; apoi după ei a pus pe Lia cu copiii ei și la urmă a pus pe Rahila și pe Iosif;
\par 3 Iar el mergea în fruntea lor și, apropiindu-se de fratele său, i s-a închinat de șapte ori până la pământ.
\par 4 Isav însă a alergat în întâmpinarea lui și l-a îmbrățișat și, cuprinzându-i grumazul, l-a sărutat și au plâns amândoi.
\par 5 Apoi, ridicându-și ochii și văzând femeile și copiii, Isav a zis: "Cine sunt aceștia?" Zis-a Iacov: "Copiii cu care a miluit Dumnezeu pe robul tău!"
\par 6 Deci, s-au apropiat întâi roabele cu copiii lor și s-au închinat.
\par 7 Apoi a venit Lia cu copiii ei și s-au închinat, iar la urmă au venit și s-au închinat și Rahila cu Iosif.
\par 8 Zis-a Isav: "Ce sunt acele turme, pe care le-am întâlnit?" Iar Iacov a răspuns: "Ca să afle robul tău bunăvoință înaintea domnului meu".
\par 9 Atunci Isav a zis: "Am și eu multe, frate; ține-ți ale tale pentru tine!"
\par 10 Iacov însă a zis: "De am aflat bunăvoință înaintea ta, primește darurile din mâinile mele, căci, când am văzut fața ta, parcă aș fi văzut fața lui Dumnezeu, așa de binevoitor mi-ai fost.
\par 11 Primește de la mine binecuvântările mele, pe care ți le aduc, că m-a miluit Dumnezeu și am de toate". Și a stăruit și le-a luat.
\par 12 Apoi a zis Isav: "Să ne sculăm și să mergem împreună; eu însă îmi voi potrivi pasul cu tine".
\par 13 Iacov însă a răspuns: "Domnul meu știe că îmi sunt gingași copiii și că am oi și vite de curând fătate; de le vom mâna tare numai o zi, ar pieri toată turma.
\par 14 Să se ducă dar domnul meu, înaintea robului său, iar eu voi urma încet, în pas cu vitele cele dinaintea mea și în pas cu copiii, până voi ajunge la domnul meu în Seir".
\par 15 Atunci Isav a zis: "Să-ți las măcar o parte din oamenii cei ce sunt cu mine". Iar Iacov i-a răspuns: "La ce aceasta? Mi-ajunge mie bunăvoința ce-am aflat înaintea domnului meu".
\par 16 Și s-a întors Isav în aceeași zi pe calea sa la Seir.
\par 17 Iar Iacov s-a îndreptat spre Sucot și și-a făcut acolo locuință pentru sine, iar pentru vitele sale a făcut șuri; de aceea a pus el numele locului aceluia Sucot.
\par 18 Întorcându-se Iacov din Mesopotamia și ajungând cu bine la Salem, o cetate în ținutul Sichem, din pământul Canaan, s-a așezat în fața cetății.
\par 19 Apoi și-a cumpărat partea de câmp, pe care era cortul său, cu o sută de kesite, de la fiii lui Hemor, tatăl lui Sichem.
\par 20 A înălțat acolo un jertfelnic și i-a pus numele El-Elohe-Israel.

\chapter{34}

\par 1 Într-o zi, Dina, fata Liei, pe care aceasta o născuse lui Iacov, a ieșit să vadă fetele țării aceleia.
\par 2 Și văzând-o Sichem, feciorul lui Hemor Heveul, stăpânitorul pământului aceluia, a luat-o și, culcându-se cu ea, a necinstit-o.
\par 3 Apoi s-a lipit sufletul lui de Dina, fata lui Iacov, și i-a căzut dragă fata și a vorbit pe placul fetei.
\par 4 Și a zis Sichem către tatăl său Hemor: "Ia-mi pe fata aceasta de femeie!"
\par 5 Deși Iacov a auzit că fiul lui Hemor a necinstit pe Dina, fata sa, dar, fiindcă feciorii lui erau cu vitele la câmp, a tăcut până s-au întors ei.
\par 6 Iar Hemor, tatăl lui Sichem, a ieșit la Iacov, ca să vorbească cu el.
\par 7 Feciorii lui Iacov însă, venind de la câmp și aflând despre aceasta, se amărâră și se mâniară foarte tare, pentru că Sichem săvârșise o faptă de ocară în Israel, culcându-se cu fata lui Iacov, ceea ce nu trebuia să se întâmple.
\par 8 Și grăind cu ei, Hemor a zis: "Sichem, feciorul meu, s-a lipit cu sufletul de fata voastră; dați-o dar lui de femeie și vă încuscriți cu noi:
\par 9 Măritați-vă fetele voastre cu noi și fetele noastre luați-le pentru feciorii voștri;
\par 10 Ședeți la un loc cu noi: acest pământ larg vă e la îndemână, ca să vă așezați într-însul, să faceți negoț și să vă agonisiți din el moșie".
\par 11 Iar Sichem a zis către tatăl fetei și către frații ei: "Orice veți zice, voi da, numai să aflu bunăvoință la voi.
\par 12 Cereți de la mine un mare preț de cumpărare și darurile cele mai mari și vă voi da cât veți zice, numai dați-mi fata mie de femeie!"
\par 13 Feciorii lui Iacov însă au răspuns cu vicleșug lui Sichem și lui Hemor, tatăl lui; și le-au răspuns așa, pentru că acela necinstise pe Dina, sora lor.
\par 14 Și au zis către dânșii Simeon și Levi, frații Dinei și feciorii Liei: "Nu putem să facem aceasta: să dăm pe sora noastră după un om netăiat împrejur, că aceasta ar fi o rușine pentru noi.
\par 15 Numai așa ne învoim cu voi și ne așezăm la voi, dacă veți face și voi ca noi, tăindu-vă împrejur toți cei de parte bărbătească.
\par 16 Atunci vom da după voi fetele noastre, iar noi vom lua fetele voastre și vom locui la un loc cu voi și vom alcătui un popor.
\par 17 Iar de nu vreți să ne ascultați, ca să vă tăiați împrejur, noi vom lua înapoi fata și ne vom duce".
\par 18 Vorbele acestea au plăcut lui Hemor și lui Sichem, feciorul lui Hemor.
\par 19 De aceea, n-a zăbovit tânărul să facă aceasta, căci era îndrăgostit de fata lui Iacov și era și cel mai cu trecere în casa tatălui său.
\par 20 Și au venit Hemor și Sichem, feciorul lui, la poarta cetății lor, și au început a grăi locuitorilor cetății, zicând:
\par 21 "Oamenii aceștia sunt pașnici; să se așeze dar în țara noastră și să facă negoț în ea. Iată că loc este din destul și într-o parte și într-alta; fetele lor să ni le luăm de femei și fetele noastre să le dăm după ei.
\par 22 Dar oamenii aceștia numai așa se învoiesc să trăiască cu noi și să fie un popor cu noi, dacă și la noi se vor tăia împrejur toți cei de parte bărbătească, cum sunt ei tăiați împrejur.
\par 23 Turmele lor, vitele lor și toate averile lor nu sunt, oare, ale noastre? Să le plinim voia lor, iar ei să se așeze printre noi!"
\par 24 Și au ascultat pe Hemor și pe Sichem, feciorul lui, toți cei ce ieșeau pe poarta cetății lor și au fost tăiați împrejur toți cei de parte bărbătească, câți ieșeau pe poarta cetății lor.
\par 25 Iar a treia zi, când erau ei încă în dureri, cei doi fiii ai lui Iacov, Simeon și Levi, frații Dinei, și-au luat fiecare sabia și au intrat fără teamă în cetate și au ucis pe toți cei de parte bărbătească.
\par 26 Au trecut prin ascuțișul sabiei și pe Hemor și pe fiul său Sichem și au luat pe Dina din casa lui Sichem și au plecat.
\par 27 Apoi fiii lui Iacov se năpustiră asupra celor morți și jefuiră cetatea în care fusese necinstită Dina, sora lor.
\par 28 Au luat toate oile lor, toți boii lor, toți asinii lor, tot ce era în cetate și tot ce era pe câmp;
\par 29 Toate bogățiile lor, toți copiii și femeile le-au dus în robie; și au jefuit tot ce era în cetate și tot ce era prin case.
\par 30 Atunci Iacov a zis către Simeon și către Levi: "Mare tulburare mi-ați adus, făcându-mă urât înaintea tuturor locuitorilor țării acesteia, înaintea Canaaneilor și a Ferezeilor. Eu am oameni puțini la număr; se vor ridica asupra mea și mă vor ucide și voi pieri și eu și casa mea".
\par 31 Iar ei au zis: "Dar se putea, oare, ca ei să se poarte cu sora noastră ca și cu o femeie pierdută?"

\chapter{35}

\par 1 Atunci a zis Dumnezeu lui Iacov: "Scoală și du-te la Betel și locuiește acolo; fă acolo jertfelnic Dumnezeului Celui ce ți S-a arătat, când fugeai tu de la fața lui Isav, fratele tău!"
\par 2 Iar Iacov a zis casei sale și tuturor celor ce erau cu dânsul: "Lepădați dumnezeii cei străini, care se află la voi, curățiți-vă și vă primeniți hainele voastre.
\par 3 Să ne sculăm și să mergem la Betel, că acolo am să fac jertfelnic lui Dumnezeu, Celui ce m-a auzit în ziua necazului meu și Care a fost cu mine și m-a păzit în călătoria în care am umblat!"
\par 4 Iar ei au dat lui Iacov toți dumnezeii cei străini, care erau în mâinile lor, și cerceii ce-i aveau în urechile lor; și Iacov i-a îngropat sub stejarul de lângă Sichem și i-a lăsat necunoscuți până în ziua de astăzi.
\par 5 Astfel au plecat ei de la Sichem; și frica lui Dumnezeu era peste orașele dimprejur și n-au urmărit pe fiii lui Iacov.
\par 6 Sosind Iacov cu toți oamenii cei ce erau cu el la Luz, adică la Betel, în țara Canaanului,
\par 7 A zidit acolo un jertfelnic și a numit locul acela El-Bet-El, pentru că acolo i Se arătase Dumnezeu, când fugea el de Isav, fratele său.
\par 8 Atunci a murit Debora, doica Rebecăi, și a fost îngropată mai jos de Betel, sub un stejar, pe care Iacov l-a numit "Stejarul Plângerii".
\par 9 Aici, în Luz, Se mai arătă Dumnezeu lui Iacov, după întoarcerea lui din Mesopotamia, și îl binecuvântă Dumnezeu,
\par 10 Și-i zise: "De acum nu te vei mai chema Iacov, ci Israel va fi numele tău". Și-i puse numele Israel.
\par 11 Apoi Dumnezeu îi mai zise: "Eu sunt Dumnezeul cel Atotputernic! Sporește și te înmulțește! Popoare și mulțime de neamuri se vor naște din tine și regi vor răsări din coapsele tale.
\par 12 Țara, pe care am dat-o lui Avraam și lui Isaac, o voi da ție; iar după tine, voi da pământul acesta urmașilor tăi".
\par 13 Apoi S-a înălțat Dumnezeu de la el, din locul în care îi grăise.
\par 14 Și a așezat Iacov un stâlp pe locul unde-i grăise Dumnezeu, un stâlp de piatră, și a săvârșit turnare peste el și a turnat peste el untdelemn.
\par 15 Și a pus Iacov locului unde-i grăise Dumnezeu, numele Betel.
\par 16 După aceea au plecat din Betel. Și și-a întins cortul său dincolo de turnul Gader. Dar când se apropiase de Havrata, înainte de a intra în Efrata, Rahila a născut și nașterea aceasta a fost iar tare grea.
\par 17 Și pe când se chinuia Rahila în durerile nașterii, moașa i-a zis: "Nu te teme, că și acesta va fi băiat!"
\par 18 Iar când Rahila își dădea sufletul, căci a murit, a pus copilului numele Ben-Oni, adică fiul durerii mele, iar tatăl lui l-a numit Veniamin.
\par 19 Iar dacă a murit, Rahila a fost îngropată lângă calea ce duce la Efrata, adică la Betleem;
\par 20 Iacov a ridicat un stâlp de piatră pe mormântul ei și acest stâlp, de pe mormântul Rahilei, este până în ziua de astăzi.
\par 21 Apoi plecând Iacov de aici și-a întins cortul dincolo de turnul Migdal-Eder. Iar pe vremea când locuia Israel în țara aceasta, a intrat Ruben și a dormit cu Bilha, țiitoarea tatălui său Iacov, și a auzit Israel și i s-a părut aceasta un rău.
\par 22 Fiii lui Iacov au fost doisprezece și anume:
\par 23 Fiii Liei: Ruben, întâi-născutul lui Iacov; după el veneau: Simeon, Levi, Iuda, Isahar și Zabulon.
\par 24 Fiii Rahilei: Iosif și Veniamin.
\par 25 Fiii slujnicei Rahilei, Bilha: Dan și Neftali.
\par 26 Și fiii Zilpei, roaba Liei: Gad și Așer. Aceștia sunt fiii lui Iacov, care i s-au născut în Mesopotamia.
\par 27 Apoi a sosit Iacov la Isaac, tatăl său, căci acesta trăia încă la Mamvri, în Chiriat-Arba, adică la Hebron în pământul Canaanului, unde locuiseră vremelnic Avraam și Isaac.
\par 28 Iar zilele, pe care le-a trăit Isaac, au fost o sută optzeci de ani.
\par 29 Slăbind apoi, Isaac a murit și a trecut la părinții săi, fiind bătrân și încărcat de zile, și l-au îngropat feciorii lui, Isav și Iacov.

\chapter{36}

\par 1 Iar spița neamului lui Isav, care se mai numește și Edom, este aceasta:
\par 2 Isav și-a luat femei din fetele Canaaneilor: pe Ada, fata lui Elon Heteul, și pe Olibama, fata lui Ana, fiul lui Țibon Heveul,
\par 3 Și pe Basemata, fata lui Ismael și sora lui Nebaiot.
\par 4 Ada a născut lui Isav pe Elifaz; Basemata i-a născut pe Raguel;
\par 5 Iar Olibama i-a născut pe Ieuș, pe Ialam și pe Core. Aceștia sunt fiii lui Isav, care i s-au născut în țara Canaanului.
\par 6 După aceea și-a luat Isav femeile sale, fiii săi, fetele sale, toți oamenii casei sale, toate averile sale, toate vitele sale și toate câte avea și toate câte agonisise în țara Canaanului, și a plecat Isav din Canaan din pricina lui Iacov, fratele său,
\par 7 Pentru că averile lor erau multe și nu mai puteau să locuiască la un loc, și pământul unde erau nu-i mai putea încăpea din pricina mulțimii turmelor lor.
\par 8 Astfel Isav, care se mai numește și Edom, s-a mutat în muntele Seir.
\par 9 Iată acum și urmașii ce i s-au născut lui Isav, părintele Edomiților, după mutarea sa în muntele Seir.
\par 10 Numele fiilor lui Isav sunt acestea: Elifaz, fiul Adei, solia lui Isav și Raguel, fiul Basematei, soția lui Isav.
\par 11 Elifaz, a avut cinci feciori: Teman, Omar, Țefo, Gatam și Chenaz;
\par 12 Iar Timna, o țiitoare a lui Elifaz, fiul lui Isav, i-a născut lui Elifaz pe Amalec. Aceștia sunt urmașii din Ada, femeia lui Isav.
\par 13 Iar feciorii lui Raguel sunt aceștia: Nahat și Zerah, Șama și Miza. Aceștia sunt urmașii din Basemata, femeia lui Isav.
\par 14 Iar feciorii Olibamei, femeia lui Isav și fiica lui Ana a lui Țibon, sunt aceștia: ea a născut lui Isav pe Ieuș, pe Ialam și pe Core.
\par 15 Iată și căpeteniile fiilor lui Isav: feciorii lui Elifaz, întâi-născutul lui Isav, sunt: căpetenia Teman, căpetenia Omar, căpetenia Țefo, căpetenia Chenaz,
\par 16 Căpetenia Core, căpetenia Gatam și căpetenia Amalec. Acestea sunt căpeteniile din Elifaz în țara Edomului; aceștia sunt urmașii din Ada.
\par 17 Iar fiii lui Raguel, fiul lui Isav, sunt: căpetenia Nahat, căpetenia Zerah, căpetenia Șama și căpetenia Miza. Acestea sunt căpeteniile din Raguel în țara Edomului; aceștia sunt urmașii din Basemata, femeia lui Isav.
\par 18 Iată și fiii Olibamei, femeia lui Isav: căpetenia Ieuș, căpetenia Ialam și căpetenia Core. Acestea sunt căpeteniile din Olibama, fata lui Ana și femeia lui Isav.
\par 19 Aceștia sunt fiii lui Isav și acestea sunt căpeteniile lor. Acesta este Edom.
\par 20 Iar feciorii lui Seir Horeeanul, care locuiau înainte pământul acela, sunt: Lotan, Șobal, Țibon și Ana;
\par 21 Dișon, Ețer și Dișan. Acestea sunt căpeteniile Horeilor, feciorii lui Seir, în pământul Edomului.
\par 22 Fiii lui Lotan sunt: Hori și Heman, iar sora lui Lotan a fost Timna.
\par 23 Fiii lui Șobal sunt: Alvan, Manahat, Ebal, Șefo și Onam.
\par 24 Fiii lui Țibon sunt: Aia și Ana. Acesta este acel Ana, care a găsit izvoarele de apă caldă în pustie, când păștea asinii tatălui său Țibon.
\par 25 Copiii lui Ana sunt: Dișon și Olibama, fata lui Ana.
\par 26 Copiii lui Dișon sunt: Hemdan, Eșban, Itran și Cheran.
\par 27 Fiii lui Ețer sunt: Bilhan, Zaavan și Acan.
\par 28 Fiii lui Dișan sunt: Uț și Aran.
\par 29 Deci căpeteniile Horeilor sunt acestea: căpetenia Lotan, căpetenia Șobal, căpetenia Țibon, căpetenia Ana,
\par 30 Căpetenia Dișon, căpetenia Ețer, căpetenia Dișan. Acestea sunt căpeteniile Horeilor din țara lui Seir, după familiile lor.
\par 31 Iată și regii, care au domnit în pământul Edomului înainte de a domni vreun rege peste fiii lui Israel:
\par 32 În Edom a domnit mai întâi Bela, fiul lui Beor și cetatea lui se numea Dinhaba.
\par 33 După ce a murit Bela, s-a făcut rege Iobab, fiul lui Zerah din Boțra.
\par 34 După ce a murit Iobab, s-a făcut rege Hușam, din țara Temaniților.
\par 35 După ce a murit Hușam, s-a făcut rege Hadad, feciorul lui Bedad; care a bătut pe Madianiți în câmpul Moab; numele cetății lui era Avit.
\par 36 Iar după ce a murit Hadad, s-a făcut rege Samla, din Masreca.
\par 37 Iar după ce a murit Samla, s-a făcut rege, în locul lui, Șaul, din Rehobotul de pe râu.
\par 38 Iar după ce a murit Șaul, s-a făcut rege Baal-Hanan, fiul lui Acbor.
\par 39 Iar după ce a murit Baal-Hanan, feciorul lui Acbor, s-a făcut rege Hadar, fiul lui Varad, și numele cetății lui era Pau și al femeii lui Mehetabel, fiica lui Matred, feciorul lui Mezahab.
\par 40 Iar numele căpeteniilor din Isav, după triburile lor, după țările lor, după numirile și națiile lor sunt: căpetenia Timna, căpetenia Alvan, căpetenia Ietet,
\par 41 Căpetenia Olibama, căpetenia Ela, căpetenia Pinon,
\par 42 Căpetenia Chenaz, căpetenia Teman, căpetenia Mibțar,
\par 43 Căpetenia Magdiel, căpetenia Iram. Acestea sunt căpeteniile lui Edom, după așezările lor în țara stăpânită de ei. Acesta-i Isav, părintele Edomiților.

\chapter{37}

\par 1 Iacov a locuit în țara Canaan, unde locuise și Isaac, tatăl său.
\par 2 Iată acum și istoria urmașilor lui Iacov: Iosif, fiind de șaptesprezece ani, păștea oile tatălui său împreună cu frații săi. Petrecându-și copilăria cu feciorii Bilhăi și cu feciorii Zilpei, femeile tatălui său, Iosif aducea lui Israel, tatăl său, vești despre purtările lor rele.
\par 3 Și iubea Israel pe Iosif mai mult decât pe toți ceilalți fii ai săi, pentru că el era copilul bătrâneților lui, și-i făcuse haină lungă și aleasă.
\par 4 Frații lui, văzând că tatăl lor îl iubea mai mult decât pe toți fiii săi, îl urau și nu puteau vorbi cu el prietenos.
\par 5 Visând însă Iosif un vis, l-a spus fraților săi,
\par 6 Zicându-le: "Ascultați visul ce am visat:
\par 7 Parcă legam snopi în țarină și snopul meu parcă s-a sculat și stătea drept, iar snopii voștri s-au strâns roată și s-au închinat snopului meu".
\par 8 Iar frații lui au zis către el: "Nu cumva ai să domnești peste noi? Sau poate ai să ne stăpânești?" Și l-au urât încă și mai mult pentru visul lui și pentru spusele lui.
\par 9 Și a mai visat el alt vis și l-a spus tatălui său și fraților săi, zicând: "Iată am mai visat alt vis: soarele și luna și unsprezece stele mi se închinau mie".
\par 10 Și-l povesti tatălui său și fraților săi, iar tatăl său l-a certat și i-a zis: "Ce înseamnă visul acesta pe care l-ai visat? Au doară eu și mama ta și frații tăi vom veni și ne vom închina ție până la pământ?"
\par 11 De aceea îl pizmuiau frații lui, iar tatăl său păstra cuvintele acestea în inima lui.
\par 12 S-au dus după aceea frații lui să pască oile tatălui lor la Sichem.
\par 13 Și Israel a zis către Iosif: "Frații tăi pasc oile la Sichem. Vino, dar, să te trimit la ei". Iar el a zis: "Mă duc, tată!"
\par 14 Apoi Israel a zis către Iosif: "Du-te și vezi de sunt sănătoși frații tăi și oile și să-mi aduci răspuns!" L-a trimis astfel din valea Hebronului și Iosif s-a dus la Sichem.
\par 15 După aceea l-a găsit un om rătăcind pe câmp și l-a întrebat omul acela și i-a zis: "Ce cauți?"
\par 16 Iar el a zis: "Caut pe frații mei. Spune-mi, unde pasc ei oile?"
\par 17 Zisu-i-a omul acela: "S-au dus de aici, căci i-am auzit zicând: Haidem la Dotain!" Și s-a dus Iosif după frații săi și i-a găsit la Dotain.
\par 18 Iar ei, văzându-l de departe, până a nu se apropia de ei, au început a unelti asupra lui să-l omoare;
\par 19 Și au zis unii către alții: "Iată visătorul acela de vise vine!
\par 20 Haidem să-l omorâm, să-l aruncăm într-un puț și să zicem că l-a mâncat o fiară sălbatică și vom vedea ce se va alege de visele lui!"
\par 21 Auzind însă aceasta, Ruben a voit să-l scape din mâinile lor, zicând: "Să nu-i ridicăm viața!"
\par 22 Apoi Ruben a adăugat: "Să nu vărsați sânge! Aruncați-l în puțul acela din pustie, dar mâinile să nu vi le puneți pe el!" Iar aceasta o zicea el cu gândul de a-l scăpa din mâinile lor și a-l trimite acasă la tatăl său.
\par 23 Când însă a sosit Iosif la frații săi, ei au dezbrăcat pe Iosif de haina cea lungă și aleasă, cu care era îmbrăcat,
\par 24 Și l-au luat și l-au aruncat în puț; dar puțul era gol și nu avea apă.
\par 25 După aceea șezând să mănânce pâine și căutând cu ochii lor, ei au văzut venind dinspre Galaad o caravană de Ismaeliți, ale căror cămile erau încărcate cu tămâie, eu balsam și cu smirnă, pe care le duceau în Egipt.
\par 26 Atunci a zis Iuda către frații săi: "Ce vom folosi de vom ucide pe fratele nostru și vom ascunde sângele lui?
\par 27 Haidem să-l vindem Ismaeliților acestora, neridicându-ne mâinile asupra lui, pentru că e fratele nostru și trupul nostru". Și au ascultat frații lui.
\par 28 Iar când au trecut negustorii Madianiți pe acolo, frații au tras și au scos pe Iosif din puț și l-au vândut pe el Ismaeliților cu douăzeci de arginți. Și aceștia au dus pe Iosif în Egipt.
\par 29 Când însă s-a întors Ruben la puț și n-a văzut pe Iosif în puț, el și-a rupt hainele,
\par 30 Și întorcându-se la frații săi, a zis: "Băiatul nu este! Încotro să apuc eu acum?"
\par 31 Atunci ei au luat haina lui Iosif și, junghiind un ied, au muiat haina în sânge;
\par 32 Apoi au trimis după haina cea lungă și aleasă și au adus-o la tatăl lor, spunând. "Am găsit aceasta; vezi de este haina fiului tău sau nu!"
\par 33 Și a cunoscut-o Iacov și a zis: "Este haina fiului meu! L-a mâncat o fiară sălbatică; o fiară l-a sfâșiat pe Iosif!"
\par 34 Atunci și-a rupt Iacov hainele sale și-a acoperit cu sac coapsele și a plâns pe fiul său zile multe.
\par 35 După aceea s-au adunat toți feciorii lui și toate fetele lui și au venit să-l mângâie; dar el nu voia să se mângâie, ci zicea: "Plângând, mă voi pogorî în locuința morților la fiul meu!" Și-l plângea astfel tatăl său.
\par 36 Iar Madianiții au vândut pe Iosif în Egipt lui Putifar, dregător și comandant al gărzii la curtea lui Faraon.

\chapter{38}

\par 1 În vremea aceea s-a întâmplat că Iuda s-a pogorât de la frații săi și s-a așezat lângă un adulamitean, cu numele Hira.
\par 2 Văzând Iuda acolo pe fata unui canaaneu, care se numea Șua, el a luat-o de soție și a intrat la ea.
\par 3 Și ea, zămislind, a născut un băiat, și Iuda i-a pus numele Ir.
\par 4 Zămislind iarăși, a născut alt băiat și i-a pus numele Onan.
\par 5 Și a mai născut un băiat și i-a pus numele Șela. Și când a născut ea acest fiu, Iuda era la Kezib.
\par 6 Apoi Iuda a luat pentru Ir, întâiul născut al său, o femeie, cu numele Tamara.
\par 7 Dar Ir, întâiul născut al lui Iuda, a fost rău înaintea Domnului și de aceea l-a omorât Domnul.
\par 8 Atunci a zis Iuda către Onan: "Intră la femeia fratelui tău, însoară-te cu ea, în puterea leviratului, și ridică urmași fratelui tău!"
\par 9 Știind însă Onan că nu vor fi urmașii ai lui, de aceea, când intra la femeia fratelui său, el vărsa sămânța jos, ca să nu ridice urmași fratelui său.
\par 10 Ceea ce făcea el era rău înaintea lui Dumnezeu și l-a omorât și pe acesta.
\par 11 Atunci a zis Iuda către Tamara, nora sa, după moartea celor doi fii ai săi: "Stai văduvă în casa tatălui tău, până se va face mare Șela, fiul meu!" Căci își zicea: "Nu cumva să moară și acesta, ca și frații lui!" Și s-a dus Tamara și a trăit în casa tatălui ei.
\par 12 Trecând însă vreme multă, a murit fata lui Șua, soția lui Iuda. Iar Iuda, după ce au trecut zilele de jelire, s-a dus în Timna, la cei ce tundeau oile lui, împreună cu prietenul său Hira adulamiteanul.
\par 13 Atunci i s-a vestit Tamarei, nora sa, zicându-i-se: "Iată socrul tău merge la Timna să-și tundă oile".
\par 14 Iar ea, dezbrăcând de pe sine hainele sale de văduvie, s-a înfășurat cu un văl și, gătindu-se, a ieșit și a șezut la poarta Enaim, care este în drumul spre Timna, căci vedea că Șela crescuse mare și ea nu-i fusese dată lui de soție.
\par 15 Și, văzând-o Iuda, a socotit că este o femeie nărăvită, căci n-a cunoscut-o, pentru că își avea fața acoperită.
\par 16 Și abătându-se din cale pe la ea, i-a zis: "Lasă-mă să intru la tine!" Căci nu știa că este nora sa. Iar ea a zis: "Ce ai să-mi dai, dacă vei intra la mine?"
\par 17 Iar el i-a răspuns: "Îți voi trimite un ied din turma mea". Și ea a zis: "Bine, dar să-mi dai ceva zălog până mi-l vei trimite".
\par 18 Răspuns-a Iuda: "Ce zălog să-ți dau?" Și ea a zis: "Inelul tău, cingătoarea ta și toiagul ce-l ai în mână". Și el i le-a dat și a intrat la ea și ea a rămas grea.
\par 19 Apoi, sculându-se, ea s-a dus și-a scos vălul său și s-a îmbrăcat iar cu hainele sale de văduvie.
\par 20 Iar Iuda a trimis iedul pe adulamitean, prietenul său, ca să ia zălogul din mâinile femeii. Dar n-a mai găsit-o.
\par 21 Și a întrebat pe oamenii locului aceluia: "Unde este femeia cea nărăvită, care ședea la Enaim, la drum?" Iar aceia i-au răspuns: "N-a fost aici nici o femeie nărăvită!"
\par 22 S-a întors deci acela la Iuda și a zis: "N-am găsit-o și oamenii de acolo mi-au spus că n-a fost acolo nici o femeie nărăvită!"
\par 23 Atunci Iuda a zis: "Să și le ție! Numai de nu ne-ar face de batjocură. Iată, eu i-am trimis iedul, dar tu n-ai găsit-o".
\par 24 Dar, cam după vreo trei luni, i s-a spus lui Iuda: "Tamara, nora ta, a căzut în desfrânare și iată a rămas însărcinată din desfrânare". Iar Iuda a zis: "Scoateți-o și să fie arsă".
\par 25 Dar când o duceau, ea a trimis la socrul său, zicând: "Eu sunt îngreunată de acela ale căruia sunt lucrurile acestea". Apoi a adăugat: "Află al cui e inelul acesta, cingătoarea aceasta și toiagul acesta!"
\par 26 Și le-a cunoscut Iuda și a zis: "Tamara e mai dreaptă decât mine, pentru că nu am dat-o lui Șela, fiul meu". Și n-a mai cunoscut-o pe ea.
\par 27 Iar când era să nască, s-a aflat că are în pântece doi gemeni.
\par 28 În vremea nașterii s-a ivit mâna unuia, iar moașa a luat și i-a legat la mână un fir de ață roșie, zicând: "Acesta a ieșit întâi".
\par 29 Dar acesta și-a tras mâna înapoi și îndată a ieșit fratele lui. Și ea a zis: "Cum ai rupt tu piedica? Ruptura să fie asupra ta!" Și i-a pus numele Fares.
\par 30 După aceea a ieșit și fratele lui, cu firul de ață roșie la mână, și i s-a pus numele Zara.

\chapter{39}

\par 1 Deci Iosif a fost dus în Egipt și din mâna Ismaeliților, care l-au dus acolo, l-a cumpărat egipteanul Putifar, o căpetenie de la curtea lui Faraon și comandantul gărzii lui.
\par 2 Domnul însă era cu Iosif și el era om îndemânatic și trăia în casa egipteanului, stăpânul său.
\par 3 Stăpânul său vedea că Domnul era cu dânsul și că toate câte făcea el, Domnul le sporea în mâna lui.
\par 4 De aceea a aflat Iosif trecere înaintea stăpânului său și i-a plăcut și l-a pus peste casa sa și toate câte avea le-a dat pe mâna lui Iosif.
\par 5 Iar după ce l-a pus peste casa sa și peste toate câte avea, a binecuvântat Domnul casa egipteanului pentru Iosif și era binecuvântarea Domnului peste tot ce avea el în casa și în țarina sa.
\par 6 Și a lăsat Putifar pe mâna lui Iosif tot ce avea și, de când îl avea pe el, nu purta grijă de nimic din câte avea, fără numai de pâinea ce mânca. Iosif însă era chipeș la statură și foarte frumos la față.
\par 7 Așa fiind, femeia stăpânului său și-a pus ochii pe Iosif și i-a zis: "Culcă-te cu mine!"
\par 8 Iar el n-a voit, ci a zis către femeia stăpânului său: "De când sunt aici, stăpânul meu nu poartă grijă de nimic în casa sa, ci toate câte are le-a dat pe mâna mea.
\par 9 În casa aceasta nu-i nimeni mai mare decât mine și de la nimic nu sunt oprit decât numai de la tine, pentru că tu ești femeia lui. Cum dar să fac eu acest mare rău și să păcătuiesc înaintea lui Dumnezeu?"
\par 10 Dar, deși ea zicea așa lui Iosif în toate zilele, el n-o asculta să se culce cu ea, nici să fie cu ea.
\par 11 Se întâmplă într-o zi să intre Iosif în casă după treburile sale și, nefiind în casă vreunul din casnici,
\par 12 Ea l-a apucat de haină și i-a zis: "Culcă-te cu mine!" El însă, lăsând haina în mâinile ei, a fugit și a ieșit afară.
\par 13 Iar ea, când a văzut că el, lăsându-și haina în mâinile ei, a fugit și a ieșit afară,
\par 14 A strigat pe casnicii săi și le-a zis așa: "Priviți, ne-a adus aici slugă un evreu, ca să-și bată joc de noi. Căci a intrat la mine și mi-a zis: "Culcă-te cu mine!" Eu însă am strigat tare.
\par 15 Auzind el că am ridicat glasul și am strigat, lăsându-și haina la mine, a fugit și a ieșit afară".
\par 16 Și a ținut ea haina la sine până a venit stăpânul lui acasă.
\par 17 Atunci i-a spus și lui aceleași vorbe, zicând: "Acel rob evreu, pe care l-ai adus la noi, a venit la mine să mă batjocorească și mi-a zis: "Culcă-te cu mine!"
\par 18 Dar când a auzit că am ridicat glasul și am început să strig, s-a temut și, lăsându-și haina la mine, a fugit și a ieșit afară".
\par 19 Auzind stăpânul lui cuvintele femeii sale, câte îi spusese despre el, zicând: "Așa și așa s-a purtat cu mine sluga ta!" s-a aprins de mânie
\par 20 Și luând stăpânul pe Iosif, l-a băgat în temniță, unde erau închiși cei ce greșeau regelui. Și a rămas el acolo în temniță.
\par 21 Dar Domnul era cu Iosif, a revărsat milă asupra lui și i-a dăruit trecere înaintea mai-marelui temniței,
\par 22 Încât mai-marele temniței a dat pe mâna lui Iosif temnița și pe toți osândiții, care erau în temniță, și orice era de făcut acolo, el făcea.
\par 23 Iar mai-marele temniței nu avea nici o frică de nimic, că toate erau pe mâna lui Iosif, pentru că Domnul era cu el și toate câte făcea, Domnul le sporea în mâinile lui.

\chapter{40}

\par 1 S-a întâmplat insă după aceasta ca marele paharnic al regelui Egiptului și marele pitar să greșească înaintea regelui Egiptului, stăpânul lor.
\par 2 Atunci s-a mâniat Faraon pe cei doi dregători ai săi: pe mai-marele paharnic și pe mai-marele pitar
\par 3 Și i-a pus sub pază în temniță, în casa căpeteniei gărzii, unde era închis Iosif.
\par 4 Iar căpetenia temniței a rânduit la ei pe Iosif să le slujească; și au rămas ei câteva zile în temniță.
\par 5 Într-o noapte însă mai-marele pitar și mai-marele paharnic ai regelui Egiptului, care erau închiși în temniță, au visat amândoi vise; dar fiecare visul său și fiecare vis cu înțelesul lui.
\par 6 Iar dimineața, când a intrat Iosif la ei, iată erau tulburați.
\par 7 Și a întrebat Iosif pe dregătorii lui Faraon, care erau cu el la stăpânul său sub pază, și le-a zis: "De ce sunt astăzi triste fețele voastre?"
\par 8 Iar ei au răspuns: "Am visat niște vise și nu are cine ni le tâlcui". Zis-a lor Iosif: "Oare tâlcuirile nu sunt ele de la Dumnezeu? Spuneți-mi dar visele voastre!"
\par 9 Atunci a spus marele paharnic visul său lui Iosif și a zis: "Eu am văzut în vis că era înaintea mea o coardă de vie;
\par 10 Și coarda aceea avea trei vițe; apoi a înfrunzit, a înflorit și au crescut struguri și s-au copt.
\par 11 Și paharul lui Faraon era în mâna mea; și părea că am luat un strugure și l-am stors în paharul lui Faraon și am dat paharul în mâna lui Faraon".
\par 12 Acestuia Iosif i-a zis: "Iată tâlcuirea visului tău: cele trei vițe înseamnă trei zile.
\par 13 După trei zile își va aduce aminte Faraon de dregătoria ta și te va pune iarăși în slujba ta; și vei da lui Faraon paharul în mână, cum făceai mai înainte, când erai paharnic la el.
\par 14 Deci, când vei fi la bine, adu-ți aminte și de mine și fă-mi bine de pune pentru mine cuvânt la Faraon și mă scoate din închisoarea aceasta;
\par 15 Căci eu sunt furat din pământul Evreilor; și nici aici n-am făcut nimic, ca să fiu aruncat în temnița aceasta".
\par 16 Văzând mai-marele pitar că a tâlcuit bine, a zis către Iosif: "Și eu am visat un vis: și iată că aveam pe cap trei panere cu pâine.
\par 17 Iar în panerul cel mai de deasupra se aflau toate felurile de aluaturi coapte, din care mănâncă Faraon, și păsările cerului le ciuguleau din panerul cel de pe capul meu".
\par 18 Răspunzând acestuia, Iosif i-a zis: "Iată și tâlcuirea visului tău: cele trei panere înseamnă trei zile.
\par 19 După trei zile Faraon îți va lua capul și te va spânzura pe un lemn și păsările cerului îți vor ciuguli carnea".
\par 20 Iar a treia zi, fiind ziua nașterii lui Faraon, a făcut acesta ospăț pentru toți dregătorii săi și în mijlocul dregătorilor săi și-a adus aminte de paharnic și de mai-marele pitar;
\par 21 Și a pus iarăși pe mai-marele paharnic în dregătoria lui, și dădea el paharul lui Faraon în mână;
\par 22 Iar pe mai-marele pitar l-a spânzurat, după tâlcuirea pe care o făcuse Iosif.
\par 23 Dar mai-marele paharnic nu și-a mai adus aminte de Iosif, ci l-a uitat.

\chapter{41}

\par 1 La doi ani după aceea, a visat și Faraon un vis. Se făcea că stătea lângă râu;
\par 2 Și iată că au ieșit din râu șapte vaci, frumoase la înfățișare și grase la trup, și pășteau pe mal.
\par 3 Iar după ele au ieșit alte șapte vaci, urâte la chip și slabe la trup, și au stat pe malul râului lingă celelalte vaci;
\par 4 Și vacile cele urâte și slabe la trup au mâncat pe cele șapte vaci frumoase la chip și grase la trup; și s-a trezit Faraon.
\par 5 Apoi iar a adormit și a mai visat un vis: iată se ridicau dintr-o tulpină de grâu șapte spice frumoase și pline;
\par 6 Și după ele au ieșit alte șapte spice subțiri, seci și pălite de vântul de răsărit;
\par 7 Și cele șapte spice seci și pălite au mâncat pe cele șapte spice grase și pline. Și s-a trezit Faraon și a înțeles că era vis.
\par 8 Iar dimineața s-a tulburat duhul lui Faraon și a trimis să cheme pe toți magii Egiptului și pe toți înțelepții lui; și le-a povestit Faraon visul său, dar nu s-a găsit cine să-l tâlcuiască lui Faraon.
\par 9 Atunci a început mai-marele paharnic să grăiască lui Faraon și a zis: "Îmi aduc aminte astăzi de păcatele mele:
\par 10 S-a mâniat odată Faraon pe dregătorii săi și ne-a pus, pe mine și pe mai-marele pitar, sub pază în casa căpeteniei gărzii.
\par 11 Atunci amândoi, și eu și el, am visat într-o noapte câte un vis, dar fiecare am visat vis deosebit și cu însemnare deosebită.
\par 12 Acolo cu noi era și un tânăr evreu, un rob al căpeteniei gărzii, și spunându-i noi visele noastre, ni le-a tâlcuit, fiecăruia cu înțelesul lui.
\par 13 Și cum ne-a tâlcuit el, așa s-a și întâmplat: eu să fiu pus iar în dregătoria mea, iar acela să fie spânzurat".
\par 14 Atunci a trimis Faraon să cheme pe Iosif. Și scoțându-l îndată din temniță, l-au tuns, i-au primenit hainele și a venit la Faraon.
\par 15 Iar Faraon a zis către Iosif: "Am visat un vis și n-are cine mi-l tâlcui. Am auzit însă zicându-se despre tine că, de auzi un vis, îl tâlcuiești".
\par 16 Iosif însă răspunzând, a zis către Faraon: "Nu eu, ci Dumnezeu va da răspuns pentru liniștirea lui Faraon".
\par 17 A grăit apoi Faraon lui Iosif și a zis: "Am visat că parcă stăteam pe malul râului
\par 18 Și iată au ieșit din râu șapte vaci grase la trup și frumoase la chip și pășteau pe mal.
\par 19 Și după ele au ieșit alte șapte vaci rele și urâte la chip și slabe la trup, cum eu n-am văzut asemenea în toată țara Egiptului;
\par 20 Și vacile urâte și slabe au mâncat pe cele șapte vaci grase și frumoase.
\par 21 Și au intrat cele grase în pântecele lor și nu se cunoștea că au intrat ele în pântecele acestora, căci acestea erau tot urâte la chip, ca și mai înainte. Apoi, deșteptându-mă, am adormit iar.
\par 22 Și am visat iar un vis că dintr-o tulpină au ieșit șapte spice pline și frumoase.
\par 23 Și după ele au ieșit alte șapte spice slabe, seci și pălite de vântul de răsărit;
\par 24 Și cele șapte spice seci și pălite au mâncat pe cele șapte spice frumoase și pline. Am povestit acestea magilor, dar nimeni nu mi le-a tâlcuit".
\par 25 Atunci a zis Iosif către Faraon: "Visul lui Faraon este unul: Dumnezeu a vestit lui Faraon cele ce voiește să facă.
\par 26 Cele șapte vaci frumoase înseamnă șapte ani; cele șapte spice frumoase înseamnă șapte ani; visul lui Faraon este unul.
\par 27 Cele șapte vaci urâte și slabe, care au ieșit după ele, înseamnă șapte ani; de asemenea și cele șapte spice, seci și pălite de vântul de răsărit, înseamnă șapte ani. Vor fi șapte ani de foamete.
\par 28 Iată pentru ce am spus eu lui Faraon că Dumnezeu a arătat lui cele ce voiește să facă.
\par 29 Iată, vin șapte ani de belșug mare în tot pământul Egiptului.
\par 30 După ei vor veni șapte ani de foamete și se va uita tot belșugul acela în pământul Egiptului și foametea va secătui toată țara.
\par 31 Și belșugul de altădată nu se va mai simți în țară, după foametea care va urma, că va fi foarte grea.
\par 32 Iar că visul s-a arătat de două ori lui Faraon, aceasta înseamnă că lucrul este hotărât de Dumnezeu și că El se grăbește să-l plinească.
\par 33 Și acum să aleagă Faraon un bărbat priceput și înțelept și să-l pună peste pământul Egiptului.
\par 34 Să poruncească dar Faraon să se pună supraveghetori peste țară, ca să adune în cei șapte ani de belșug a cincea parte din toate roadele pământului Egiptului.
\par 35 Să strângă aceia toată pâinea de prisos în acești ani buni ce vin și s-o adune în cetățile pâinii, sub mâna lui Faraon, și să o păstreze spre hrană;
\par 36 Hrana aceasta va fi de rezervă în țară pentru cei șapte ani de foamete, care vor urma în țara Egiptului, ca să nu piară țara de foame".
\par 37 Aceasta a plăcut lui Faraon și tuturor dregătorilor lui.
\par 38 Și a zis Faraon către toți dregătorii săi: "Am mai putea găsi, oare, un om, ca el, în care să fie duhul lui Dumnezeu?"
\par 39 Apoi a zis Faraon către Iosif: "De vreme ce Dumnezeu ți-a descoperit toate acestea, nu se află om mai înțelept și mai priceput decât tine.
\par 40 Să fii dar tu peste casa mea. De cuvântul tău se va povățui tot poporul meu și numai prin tronul meu voi fi mai mare decât tine!"
\par 41 Apoi Faraon a zis lui Iosif: "Iată, eu te pun astăzi peste tot pământul Egiptului!"
\par 42 Și și-a scos Faraon inelul din degetul său și l-a pus în degetul lui Iosif, l-a îmbrăcat cu haină de vison și i-a pus lanț de aur împrejurul gâtului lui.
\par 43 Apoi a poruncit să fie purtat în a doua trăsură a sa și să strige înaintea lui: "Cădeți în genunchi!" Și așa a fost Iosif pus peste tot pământul Egiptului.
\par 44 Și a zis iarăși Faraon către Iosif: "Eu sunt Faraon! Dar fără știrea ta, nimeni nu are să-și miște nici mâna sa, nici piciorul său, în tot pământul Egiptului!"
\par 45 Și a pus Faraon lui Iosif numele Țafnat-Paneah și i-a dat de soție pe Asineta, fiica lui Poti-Fera, marele preot din Iliopolis.
\par 46 Iosif era de treizeci de ani când s-a înfățișat înaintea lui Faraon, regele Egiptului. Ieșind după aceea de la fața lui Faraon, Iosif s-a dus să vadă toată țara Egiptului.
\par 47 Și a rodit pământul în cei șapte ani de belșug câte un pumn dintr-un grăunte.
\par 48 Și a adunat Iosif în cei șapte ani, care au fost cu belșug în țara Egiptului, toată pâinea de prisos și a pus pâinea prin cetăți; în fiecare cetate a strâns pâinea din ținuturile dimprejurul ei.
\par 49 Astfel a strâns Iosif grâu mult foarte, ca nisipul mării, încât nici seamă nu se mai ținea, căci nu se mai putea socoti.
\par 50 Dar înainte de a sosi anii de foamete, lui Iosif i s-au născut doi fii, pe care i-a născut Asineta, fata lui Poti-Fera, preotul din Iliopolis.
\par 51 Celui întâi-născut, Iosif i-a pus numele Manase, pentru că și-a zis: "M-a învrednicit Dumnezeu să uit toate necazurile mele și toate ale casei tatălui meu";
\par 52 Iar celuilalt i-a pus numele Efraim, pentru că și-a zis: "Dumnezeu m-a făcut roditor în pământul suferinței mele".
\par 53 Iar după ce au trecut cei șapte ani de belșug, care au fost în țara Egiptului,
\par 54 Au venit cei șapte ani de foamete, după cum spusese Iosif. Atunci s-a făcut foamete în tot pământul, dar în toată țara Egiptului era pâine.
\par 55 Când însă a început să sufere de foame și toată țara Egiptului, atunci poporul a început a cere pâine la Faraon, iar Faraon a zis către toți Egiptenii: "Duceți-vă la Iosif și faceți cum vă va zice el!"
\par 56 Așadar, fiind foamete pe toată fața pământului, a deschis Iosif toate jitnițele și a început a vinde pâine tuturor Egiptenilor.
\par 57 Și veneau din toate țările în Egipt, să cumpere pâine de la Iosif, căci foametea se întinsese peste tot pământul.

\chapter{42}

\par 1 Aflând Iacov că este grâu în Egipt, a zis către fiii săi: "Ce vă uitați unul la altul?
\par 2 Iată, am auzit că este grâu în Egipt. Duceți-vă acolo și cumpărați puține bucate, ca să trăim și să nu murim!"
\par 3 Atunci cei zece din frații lui Iosif s-au dus să cumpere grâu din Egipt,
\par 4 Iar pe Veniamin, fratele lui Iosif, nu l-a trimis Iacov cu frații lui, căci zicea: "Nu cumva să i se întâmple vreun rău!"
\par 5 Au venit deci fiii lui Israel împreună cu alții, care se pogorau să cumpere grâu, căci era foamete și în pământul Canaanului.
\par 6 Iar Iosif era căpetenie peste țara Egiptului și tot el vindea la tot poporul țării. Și sosind frații lui Iosif, i s-au închinat lui până la pământ.
\par 7 Când a văzut Iosif pe frații săi, i-a cunoscut, dar s-a prefăcut că este străin de ei, le-a grăit aspru și le-a zis: "De unde ați venit?" Iar ei au zis: "Din pământul Canaanului, să cumpărăm bucate!"
\par 8 Iosif însă a cunoscut pe frații săi, iar ei nu l-au cunoscut.
\par 9 Atunci și-a adus aminte Iosif de visele sale, pe care le visase despre ei, și le-a zis: "Spioni sunteți și ați venit să iscodiți locurile slabe ale țării!"
\par 10 Zis-au ei: "Ba nu, domnul nostru! Robii tăi au venit să cumpere bucate.
\par 11 Toți suntem feciorii unui om și suntem oameni cinstiți. Robii tăi nu sunt spioni!"
\par 12 Iar el le-a zis: "Ba nu! Ci ați venit să spionați părțile slabe ale țării".
\par 13 Ei însă au răspuns: "Noi, robii tăi, suntem doisprezece frați din pământul Canaan: cel mai mic e astăzi cu tatăl nostru, iar unul nu mai trăiește".
\par 14 Iosif însă le-a zis: "E tocmai cum v-am spus eu, când am zis că sunteți spioni.
\par 15 Iată cu ce veți dovedi: pe viața lui Faraon, nu veți ieși de aici, până nu va veni aci fratele vostru cel mai mic.
\par 16 Trimiteți dar pe unul din voi să aducă pe fratele vostru; iar ceilalți veți fi închiși până se vor dovedi spusele voastre de sunt adevărate sau nu; iar de nu, pe viața lui Faraon, sunteți cu adevărat spioni".
\par 17 Și i-a pus sub pază vreme de trei zile.
\par 18 Iar a treia zi a zis Iosif către ei: "Faceți aceasta ca să fiți vii! Eu sunt om cu frica lui Dumnezeu.
\par 19 De sunteți oameni cinstiți, să rămână închis un frate al vostru; iar ceilalți duceți-vă și duceți grâul ce ați cumpărat, ca să nu sufere de foame familiile voastre.
\par 20 Dar să aduceți la mine pe fratele vostru cel mai mic, ca să se adeverească cuvintele voastre, și nu veți muri". Și ei au făcut așa.
\par 21 Ziceau însă unii către alții: "Cu adevărat suntem pedepsiți pentru păcatul ce am săvârșit împotriva fratelui nostru, căci am văzut zbuciumul sufletului lui când se ruga, și nu ne-a fost milă de el și nu l-am ascultat, și de aceea a venit peste noi urgia aceasta!"
\par 22 Atunci răspunzând Ruben, le-a zis: "Nu v-am spus eu să nu faceți nedreptate băiatului? Voi însă nu m-ați ascultat și iată acum sângele lui cere răzbunare".
\par 23 Așa grăiau ei între ei și nu știau că Iosif înțelege, pentru că el grăise cu ei prin tălmaci.
\par 24 Iar Iosif s-a depărtat de la ei și a plâns. Apoi întorcându-se iarăși și vorbind cu dânșii, a luat dintre ei pe Simeon și l-a legat înaintea ochilor lor.
\par 25 După aceea a poruncit Iosif să le umple sacii cu grâu și argintul lor să-l pună fiecăruia în sacul lui și să le dea și de ale mâncării pe cale. Și li s-a făcut așa.
\par 26 Și punându-și ei grâul pe asini, s-au dus de acolo.
\par 27 Dar când au poposit noaptea la gazdă, dezlegându-și unul sacul, ca să dea de mâncare asinului său, a văzut argintul său în gura sacului
\par 28 Și a zis către frații săi: "Argintul meu mi s-a dat înapoi și iată-l în sacul meu". Atunci s-a tulburat inima lor și cu spaimă zicea unul către altul: "Ce a făcut, oare, Dumnezeu cu noi?"
\par 29 Iar dacă au venit la Iacov, tatăl lor, în țara Canaan, i-au povestit toate câte li se întâmplase, zicând:
\par 30 "Stăpânul țării aceleia a grăit cu noi aspru și ne-a pus sub pază, ca pe niște spioni ai țării aceleia.
\par 31 Noi însă i-am spus că suntem oameni cinstiți; că nu suntem spioni;
\par 32 Că suntem doisprezece frați, fii ai aceluiași tată; că unul din noi nu mai trăiește, iar cel mai mic e cu tatăl nostru în pământul Canaan.
\par 33 Însă omul, stăpânul țării aceleia, ne-a zis: "Iată cum am să aflu eu de sunteți oameni cinstiți: lăsați aici la mine pe un frate, iar grâul ce ari cumpărat luați-l și vă duceți la casele voastre;
\par 34 Să aduceri însă la mine pe fratele vostru cel mai mic și atunci voi ști că nu sunteți spioni, ci oameni de pace, și vă voi da pe fratele vostru și veți putea face cumpărături în țara aceasta".
\par 35 Dar deșertând ei sacii lor, iată, legătura cu argintul fiecăruia era în sacul său; și văzându-și ei legăturile cu argintul lor, s-au spăimântat și ei și tatăl lor.
\par 36 Atunci, Iacov, tatăl lor, a zis către ei: "M-ați lăsat fără copii! Iosif nu mai este! Simeon nu mai este! Și acum să-mi luați și pe Veniamin? Toate au venit pe capul meu!"
\par 37 Răspunzând însă Ruben a zis către tatăl său: "Dă-l în seama mea și ți-l voi aduce; răspund eu de el; iar de nu ți-l voi aduce, să omori pe cei doi feciori ai mei!"
\par 38 Dar el a zis: "Fiul meu nu se va pogorî cu voi în Egipt, pentru că fratele lui a murit și numai el mi-a mai rămas. Și de i s-ar întâmpla vreun rău în calea în care aveți a merge, ați pogorî căruntețile mele cu întristare în locuința morților!"

\chapter{43}

\par 1 Dar întărindu-se foametea pe pământ,
\par 2 Și isprăvind de mâncat grâul ce-l aduseseră din Egipt, a zis tatăl lor către ei: "Duceți-vă iar de mai cumpărați bucate!"
\par 3 Răspunsu-i-a Iuda și a zis: "Omul acela, stăpânul țării, ne-a grăit cu jurământ și ne-a zis: De nu va veni cu voi fratele vostru cel mai mic, nu veți mai vedea fața mea.
\par 4 Deci, dacă trimiți pe fratele nostru cu noi, ne ducem să-ți cumpărăm bucate;
\par 5 Iar de nu trimiți pe fratele nostru cu noi, nu ne ducem, căci omul acela a zis: Nu veți mai vedea fața mea, de nu va veni cu voi fratele vostru!"
\par 6 Zis-a Israel: "De ce mi-ați făcut răul acesta, spunând omului aceluia că mai aveți un frate?"
\par 7 Iar ei au răspuns: "Omul acela a întrebat despre noi și despre neamul nostru, zicând: Mai trăiește oare tatăl vostru și mai aveți voi vreun frate? Și noi i-am răspuns la aceste întrebări. N-am știut că ne va zice: Aduceți pe fratele vostru!"
\par 8 Iuda însă a zis către Israel, tatăl său: "Trimite băiatul cu mine și să ne sculăm să mergem, ca să trăim și să nu murim nici noi, nici tu, nici copiii noștri.
\par 9 Răspund eu de el. Din mâna mea să-l ceri. De nu ți-l voi aduce și de nu ți-l voi înfățișa, să rămân vinovat față de tine în toate zilele vieții mele.
\par 10 De n-am fi zăbovit atât, ne-am fi întors acum a doua oară".
\par 11 Zis-a Israel, tatăl lor, către ei: "Dacă este așa, iată ce să faceți: luați în sacii voștri din roadele pământului acestuia și duceți ca dar omului aceluia: puțin balsam și puțină miere, tămâie și smirnă, migdale și fistic.
\par 12 Luați și alt argint cu voi, iar argintul care v-a fost pus înapoi în sacii voștri întoarceți-l cu mâinile voastre; poate că a fost pus din greșeală.
\par 13 Luați și pe fratele vostru și, sculându-vă, mergeți iar la omul acela.
\par 14 Iar Dumnezeul cel Atotputernic să vă dea trecere la omul acela, ca să vă dea și pe celălalt frate al vostru, și pe Veniamin. Cât despre mine, apoi de-mi va fi dat să rămân fără copii, atunci să rămân!"
\par 15 Și au luat ei darurile acestea; luat-au și argint, de două ori mai mult, și pe Veniamin și, sculându-se, au plecat în Egipt și s-au înfățișat înaintea lui Iosif.
\par 16 Iar Iosif, văzând printre ei și pe Veniamin, fratele său cel de o mamă cu el, a zis către ispravnicul casei sale: "Bagă pe oamenii aceia în casă și junghie din vite și gătește, pentru că la amiază oamenii aceia au să mănânce la masă cu mine!"
\par 17 A făcut deci omul acela cum îi poruncise Iosif: a băgat pe oamenii aceia în casa lui Iosif.
\par 18 Iar oamenii aceia, văzând că i-au băgat în casa lui Iosif, au zis: "Pentru argintul, care ni s-a înapoiat deunăzi în sacii noștri, ne bagă înăuntru, ca să se lege de noi și să ne năpăstuiască și să ne ia robi pe noi și asinii noștri".
\par 19 Atunci apropiindu-se ei de ispravnicul casei lui Iosif, au grăit cu dânsul, la ușa casei, și au zis:
\par 20 "0, domnul nostru! Noi am mai venit o dată să cumpărăm bucate,
\par 21 Dar s-a întâmplat că, ajungând la popasul de noapte și dezlegând sacii noștri, iată argintul fiecăruia era în sacul lui; și acum tot argintul nostru, după greutatea lui, îl înapoiem cu mâinile noastre;
\par 22 Iar pentru cumpărat bucate, acum am adus alt argint, și nu știm cine a pus argintul în sacii noștri".
\par 23 Iar acela le-a zis: "Fiți liniștiți și nu vă temeți; Dumnezeul vostru ș i Dumnezeul tatălui vostru v-a dat comoara în sacii voștri, căci eu am primit tot argintul cuvenit de la voi". Și le-a adus pe Simeon.
\par 24 Apoi omul acela a băgat pe oamenii aceia în casa lui Iosif și le-a dat apă de și-au spălat picioarele; și a dat și nutreț pentru asinii lor.
\par 25 Iar ei și-au pregătit darurile până la amiază, când avea să vină Iosif, căci auziseră că vor prânzi acolo.
\par 26 După ce a venit Iosif acasă, i-au adus în casă darurile ce aveau cu ei și i s-au închinat cu fețele până la pământ.
\par 27 El însă i-a întrebat: "Cum vă aflați?" Apoi a zis: "E sănătos bătrânul vostru tată, de care mi-ați vorbit deunăzi? Mai trăiește el oare?"
\par 28 Iar ei au zis: "Tatăl nostru, robul tău, e sănătos și trăiește!" Și Iosif a zis: "Binecuvântat de Dumnezeu este omul acela!" Iar ei s-au plecat și i s-au închinat.
\par 29 Și ridicându-și Iosif ochii și văzând pe Veniamin, fratele său de o mamă cu el, a zis: "Acesta-i fratele vostru cel mai mic, pe care mi-aii spus că-l veți aduce la mine?" Iar ei au răspuns: "Acesta!" Și a zis Iosif către el: "Dumnezeu să Se milostivească spre tine, fiule!"
\par 30 Dar s-a depărtat Iosif repede, pentru că inima sa ardea pentru fratele său și căuta să plângă. Și intrând în camera sa, ța plâns acolo.
\par 31 Apoi, spălându-și fața, a ieșit și stăpânindu-se a zis: "Dați mâncarea!"
\par 32 Și i s-a dat lui deosebi și lor iarăși deosebi, și Egiptenilor celor ce mâncau cu dânșii tot deosebi, căci Egiptenii nu puteau să mănânce la un loc cu Evreii, pentru că aceștia sunt spurcați pentru Egipteni.
\par 33 Și s-au așezat ei înaintea lui, cel întâi-născut după vârsta lui și cel mai tânăr după vârsta lui, și se mirau între ei oamenii aceștia.
\par 34 Apoi el a dat fiecăruia porții de dinaintea lui, iar partea lui Veniamin era de cinci ori mai mare decât a celorlalți. Și au băut și s-au veselit cu el.

\chapter{44}

\par 1 După aceea a poruncit Iosif ispravnicului casei sale și i-a zis: "Umple sacii oamenilor acestora cu bucate, cât vor putea duce, și argintul fiecăruia să-l pui în gura sacului lui.
\par 2 Iar cupa mea cea de argint să o pui în sacul celui mai mic, cu prețul grâului lui". Și a făcut acela după cuvântul lui Iosif, cum poruncise el.
\par 3 Iar dimineața, în revărsatul zorilor, le-au dat drumul oamenilor acelora și asinilor lor.
\par 4 Dar ieșind din cetate, ei nu s-au dus departe și iată că Iosif zise ispravnicului casei sale: "Scoală și aleargă după oamenii aceia și, dacă-i vei ajunge, să le zici: "De ce mi-ați răsplătit cu rău pentru bine?
\par 5 De ce mi-ați furat cupa cea de argint? Au nu este aceasta cupa din care bea stăpânul meu și în care ghicește? Ceea ce ați făcut ați făcut rău!"
\par 6 Și ajungându-i, le-a zis cuvintele acestea.
\par 7 Iar ei au răspuns: "De ce grăiește domnul vorbele acestea? Noi, robii tăi, n-am făcut așa faptă.
\par 8 Dacă noi și argintul ce l-am găsit în sacii noștri ți l-am adus înapoi din țara Canaan, cum dar să furăm din casa stăpânului tău argint sau aur?
\par 9 Acela dintre robii tăi, la care se va găsi cupa, să moară, iar noi să fim robii domnului nostru!"
\par 10 Zis-a lor acela: "Bine: cum ați zis, așa să fie! Acela, la care se va găsi cupa, să-mi fie rob, iar voi veți fi nevinovați!"
\par 11 Atunci fiecare a dat repede jos sacul său și și-a dezlegat fiecare sacul,
\par 12 Iar acela a căutat, începând de la cel mai mare și isprăvind cu cel mai mic; și a găsit cupa în sacul lui Veniamin.
\par 13 Atunci ei și-au rupt hainele și, punându-și fiecare sacul său pe asinul său, s-au întors în cetate.
\par 14 Și a intrat Iuda și frații săi la Iosif, că era încă acolo, și au căzut la pământ înaintea lui;
\par 15 Iar Iosif le-a zis: "Pentru ce ați făcut o faptă ca aceasta? Au n-ați știut voi că un om ca mine va ghici tăinuirea?"
\par 16 Zis-a Iuda: "Ce să răspundem domnului nostru, sau ce să zicem, sau cu ce să ne dezvinovățim? Dumnezeu a descoperit nedreptatea robilor tăi. Iată acum suntem robii domnului nostru și noi și acela la care s-a găsit cupa! "
\par 17 Iosif însă le-a zis: "Ba nu! Eu aceasta nu voi face-o. Ci rob îmi va fi acela la care s-a găsit cupa, iar voi duceți-vă cu pace la tatăl vostru".
\par 18 Atunci, apropiindu-se de el, Iuda a zis: "Stăpânul meu, îngăduie robului tău să spună o vorbă înaintea ta și să nu te mânii pe robul tău, căci tu ești ca și Faraon.
\par 19 Domnul meu a întrebat pe robii tăi și a zis: "Aveți voi tată sau frate?"
\par 20 Și noi am spus domnului nostru: Avem tată bătrân și un fiu mai mic, copilul bătrâneților sale, al cărui frate a murit și a rămas numai el singur de la mama lui și tatăl său îl iubește.
\par 21 Iar tu ai zis către robii tăi: Aduceți-l pe acela la mine, ca să-l văd.
\par 22 Atunci noi am spus domnului nostru: Băiatul nu poate părăsi pe tatăl lui, căci, de ar părăsi el pe tatăl lui, acesta ar muri.
\par 23 Tu însă ai zis către robii tăi: De nu va veni cu voi fratele vostru cel mai mic, să nu vă mai arătați înaintea mea.
\par 24 Și dacă ne-am întors noi la tatăl nostru și al tău rob, i-am povestit lui vorbele domnului nostru;
\par 25 Iar când tatăl nostru ne-a zis: "Duceți-vă iar, de mai cumpărați puține bucate",
\par 26 Noi atunci, i-am răspuns: "Nu ne mai putem duce; iar de va fi cu noi fratele nostru cel mai mic, ne ducem, pentru că nu putem să mai vedem fața omului aceluia, de nu va veni cu noi și fratele nostru cel mai mic".
\par 27 Atunci tatăl nostru și al tău rob ne-a zis: "Voi știți că femeia mea mi-a născut doi fii:
\par 28 Unul s-a dus de la mine și eu am zis: De bună seamă a fost sfâșiat și până acum nu l-am mai văzut.
\par 29 De-mi veți lua și pe acesta de la ochii mei și i se va întâmpla pe cale vreun rău, veți pogorî căruntețile mele amărâte în mormânt".
\par 30 Deci, de mă voi întoarce acum la tatăl nostru și al tău rob, și nu va fi cu noi băiatul, de al cărui suflet este legat sufletul său, și de va vedea el că nu este băiatul, va muri;
\par 31 Și așa robii tăi vor pogorî căruntețile tatălui lor și ale robului tău cu amărăciune în mormânt.
\par 32 Și apoi eu, robul tău, m-am prins chezaș la tatăl meu pentru băiat și am zis: De nu ți-l voi aduce și nu ți-l voi înfățișa, să rămân eu vinovat înaintea tatălui meu în toate zilele vieții mele.
\par 33 Deci, să rămân eu, robul tău, rob la domnul meu în locul băiatului, iar băiatul să se întoarcă cu frații săi.
\par 34 Căci cum mă voi duce eu la tatăl meu, de nu va fi băiatul cu mine? Nu vreau să văd durerea ce ar ajunge pe tatăl meu!"

\chapter{45}

\par 1 Atunci Iosif, nemaiputându-se stăpâni înaintea tuturor celor ce erau de față, a strigat: "Dați afară de aici pe toți!" Și nemairămânând nimeni cu el, Iosif s-a descoperit fraților săi,
\par 2 Plângând tare, și au auzit toți Egiptenii și s-a auzit și în casa lui Faraon.
\par 3 Și a zis Iosif către frații săi: "Eu sunt Iosif. Mai trăiește oare tatăl meu?" Frații lui însă nu i-au putut răspunde, că erau cuprinși de frică.
\par 4 Apoi Iosif a zis către frații săi: "Apropiați-vă de mine!" Și ei s-au apropiat. Iar el a zis: "Eu sunt Iosif, fratele vostru, pe care voi l-ați vândut în Egipt.
\par 5 Acum însă să nu vă întristați, nici să vă pară rău că m-ați vândut aici, că Dumnezeu m-a trimis înaintea voastră pentru păstrarea vieții voastre.
\par 6 Că iată, doi ani sunt de când foametea bântuie în această țară și mai sunt încă cinci ani, în care nu va fi nici arătură, nici seceriș.
\par 7 Căci Dumnezeu m-a trimis înaintea voastră, ca să păstrez o rămășiță în țara voastră și să vă gătesc mijloc de trai în țara aceasta și să cruțe viața voastră printr-o slăvită izbăvire.
\par 8 Deci nu voi m-ați trimis aici, ci Dumnezeu, Care m-a făcut ca un tată lui Faraon, domn peste toată casa lui și stăpân peste tot pământul Egiptului.
\par 9 Grăbiți-vă de vă duceți la tatăl meu și-i spuneți: "Așa zice fiul tău Iosif: Dumnezeu m-a făcut stăpân peste tot Egiptul; vino dar la mine și nu zăbovi;
\par 10 Vei locui în pământul Goșen și vei fi aproape de mine, tu, feciorii tăi și feciorii feciorilor tăi, oile tale, vitele tale și toate câte sunt ale tale;
\par 11 Și te voi hrăni acolo, că foametea va mai ține încă cinci ani, ca să nu pieri tu, nici feciorii tăi, nici toate ale tale.
\par 12 Iată, ochii voștri și ochii fratelui meu Veniamin văd că gura mea grăiește cu voi.
\par 13 Spuneți dar tatălui meu toată slava mea cea din Egipt și câte ați văzut și vă grăbiți să aduceți pe tatăl meu aici!"
\par 14 Apoi, căzând el pe grumajii lui Veniamin, fratele său, a plâns, și Veniamin a plâns și el pe grumazul lui.
\par 15 După aceea a sărutat pe toți frații săi și a plâns cu ei. După aceasta i-au grăit și frații lui.
\par 16 A mers deci vestea la casa lui Faraon, spunându-se: "Au venit frații lui Iosif". Și s-a bucurat Faraon și slujitorii lui.
\par 17 Și a zis Faraon către Iosif: "Spune fraților tăi: Iată ce să faceți: Încărcați dobitoacele voastre cu pâine și vă duceți în pământul Canaan;
\par 18 Și luând pe tatăl vostru și familiile voastre, veniți la mine și vă voi da cel mai bun loc din țara Egiptului și veți mânca din belșugul pământului".
\par 19 Și-ți mai poruncesc să le zici: "Iată ce să mai faceți: luați-vă căruțe din țara Egiptului pentru copiii voștri și pentru femeile voastre și, luând pe tatăl vostru, veniți.
\par 20 Să nu vă pară rău după locurile voastre, că vă voi da cel mai bun pământ din toată țara Egiptului".
\par 21 Și fiii lui Israel au făcut așa. Iar Iosif le-a dat căruțe după porunca lui Faraon; datu-le-a și merinde de drum.
\par 22 Apoi fiecăruia din ei i-a mai dat schimburi de haine, iar lui Veniamin i-a dat și trei sute de arginți, precum și cinci rânduri de haine.
\par 23 De asemenea a trimis și tatălui său, afară de acestea, zece asini încărcați cu cele mai bune lucruri din Egipt și zece asine încărcate cu grâu, cu pâine și cu merinde, ca să aibă pe cale.
\par 24 Așa a dat drumul fraților săi și ei au plecat; iar la plecare le-a zis: "Să nu vă sfădiți pe cale!"
\par 25 Și plecând din Egipt, ei au venit în țara Canaan, la Iacov, tatăl lor,
\par 26 Și l-au vestit, zicând: "Iosif, fiul tău, trăiește și el domnește astăzi peste toată țara Egiptului". Inima lui Iacov însă rămase rece și nu-i credea pe ei.
\par 27 Iar dacă i-au spus ei toate cuvintele lui Iosif, pe care acesta le zisese lor, și dacă a văzut căruțele, pe care le trimisese Iosif, ca să-l aducă, atunci s-a înviorat duhul lui Iacov, tatăl lor,
\par 28 Și a zis Israel: "Destul! Iosif  fiul meu, trăiește încă! Voi merge să-l văd înainte de a muri!"

\chapter{46}

\par 1 Sculându-se deci Israel cu toate câte avea, a mers la Beer-Șeba și a adus jertfă Dumnezeului tatălui său Isaac.
\par 2 Atunci a zis Dumnezeu către Israel noaptea în vis: "Iacove, Iacove!" Iar el a răspuns: "Iată-mă!"
\par 3 Și Dumnezeu a zis: "Eu sunt Domnul Dumnezeul tatălui tău, nu te teme a te duce în Egipt, căci acolo am să te fac neam mare.
\par 4 Am să merg cu tine în Egipt Eu Însumi și tot Eu am să te scot de acolo, iar Iosif îți va închide ochii cu mâna sa!"
\par 5 După aceasta s-a ridicat Iacov de la Beer-Șeba, iar fiii lui Israel au luat pe Iacov, tatăl lor, și pe copiii lor și pe femeile lor în căruțele pe care le trimisese Iosif, ca să-l aducă.
\par 6 Luând deci Iacov averile sale și toate vitele ce agonisise în pământul Canaan, el a mers în Egipt împreună cu tot neamul lui.
\par 7 Și a adus el împreună cu sine în Egipt pe fiii și nepoții săi, pe fiicele, nepoatele sale și tot neamul său.
\par 8 Iar numele fiilor lui Israel, care au intrat în Egipt, sunt acestea: Iacov și fiii lui: Ruben, întâiul-născut al lui Iacov.
\par 9 Fiii lui Ruben: Enoh, Falu, Hețron și Carmi.
\par 10 Fiii lui Simeon: Iemuel și Iamin, Ohad și Iachin, Țohar și Saul, fiii canaaneencii.
\par 11 Fiii lui Levi: Gherșon, Cahat și Merari.
\par 12 Fiii lui Iuda: Ir, Onan, Șela, Fares și Zara. Însă Ir și Onan au murit în țara Canaanului. Iar fiii lui Fares erau Hesron și Hamul.
\par 13 Fiii lui Isahar erau Tola și Fua, Iașub și Șimron.
\par 14 Fiii lui Zabulon erau: Sered, Elon și Iahleel.
\par 15 Aceștia sunt feciorii și nepoții Liei, pe care i-a născut ea lui Iacov în Mesopotamia, ca și pe Dina, fata lui: de toți treizeci și trei de suflete, băieți și fete.
\par 16 Fiii lui Gad erau: Țifion, Haghi, Șuni, Ețbon, Eri, Arodi și Areli.
\par 17 Fiii lui Așer erau: Imna și Ișva, Ișvi, Bria și Serah, sora lor. Iar Bria a avut pe Heber și Malkiel.
\par 18 Aceștia sunt feciorii și nepoții Zilpei, pe care Laban a dat-o Liei, fiica sa. Ea a născut lui Iacov șaisprezece suflete.
\par 19 Fiii Rahilei, soția lui Iacov, erau: Iosif și Veniamin.
\par 20 Lui Iosif i s-au născut în țara Egiptului Manase și Efraim, pe care i-a născut Asineta, fiica lui Poti-Fera, preotul cel mare din Iliopolis. Fiul lui Manase, pe care i l-a născut țiitoarea sa Sira este Machir; iar lui Machir i s-a născut Galaad; iar fiii lui Efraim, fratele lui Manase, au fost: Sutalaam și Taam; Sutalaam a avut de fiu pe Edom.
\par 21 Fiii lui Veniamin au fost: Bela, Becher și Așbel; fiii lui Bela au fost: Ghera și Naaman, Ehi, Roș, Mupim, Hupim și Ard.
\par 22 Aceștia sunt fiii și nepoții Rahilei, care i s-au născut lui Iacov: de toți paisprezece suflete.
\par 23 Fiul lui Dan a fost: Hușim.
\par 24 Fiii lui Neftali au fost: Iahțeel, Guni, Iețer și Șilem.
\par 25 Aceștia sunt feciorii și nepoții Bilhăi, pe care Laban a dat-o roabă fiicei sale Rahila. Ea a născut lui Iacov de toate șapte suflete.
\par 26 Iar sufletele, care au intrat cu Iacov în Egipt și care au ieșit din coapsele lui, au fost de toate șaizeci și șase afară de femeile fiilor lui Iacov.
\par 27 Fiii lui Iosif, născuți în Egipt, erau de toți nouă suflete. Deci, de toate, sufletele casei lui Iacov, care au venit în Egipt cu el, au fost șaptezeci și cinci.
\par 28 Atunci a trimis Iacov pe Iuda înaintea sa, la Iosif, ca să-l întâmpine la Ieroonpolis, în ținutul Goșen.
\par 29 Iar Iosif, înhămându-și caii la căruța sa, a ieșit în întâmpinarea lui Israel, tatăl său, la Ieroonpolis și, văzându-l, a căzut pe grumazul lui și a plâns mult pe grumazul lui.
\par 30 Israel însă a zis către Iosif: "De acum pot să mor, că am văzut fața ta și că trăiești încă".
\par 31 Iar Iosif a zis către frații săi și către casa tatălui său: "Mă duc să vestesc pe Faraon și să-i zic: Frații mei și casa tatălui meu, care erau în pământul Canaan, au venit la mine.
\par 32 Acești oameni sunt păstori de oi, căci trăiesc din creșterea vitelor, și au adus cu ei oile și vitele lor și toate câte au.
\par 33 Și dacă vă va chema Faraon și vă va zice: "Cu ce vă îndeletniciți?"
\par 34 Să-i răspundeți: "Robii tăi am fost crescători de vite din tinerețile noastre până acum, și noi și părinții noștri", ca astfel să vă așeze în pământul Goșen. Căci pentru Egipteni este spurcat tot păstorul de oi.

\chapter{47}

\par 1 Mergând deci Iosif a vestit pe Faraon, zicând: "Tatăl meu și frații mei, cu vitele lor mărunte și mari și cu toate câte au, au venit din pământul Canaan și iată sunt în ținutul Goșen!"
\par 2 Luând apoi pe cinci dintre frații săi, i-a înfățișat lui Faraon.
\par 3 Iar Faraon a zis către frații lui Iosif: "Cu ce vă îndeletniciți voi?" Și ei au răspuns lui Faraon: "Robii tăi sunt un neam de păstori de oi, din tată în fiu".
\par 4 Apoi au zis iarăși către Faraon: "Am venit să locuim în pământul acesta, căci nu se găsește pășune pentru vitele robilor tăi, pentru că în țara Canaan e foamete mare. îngăduie dar robilor tăi să se așeze în ținutul Goșen!"
\par 5 Iar Faraon a zis către Iosif: "Tatăl tău și frații tăi au venit la tine.
\par 6 Iată, pământul Egiptului îți stă înainte; așază pe tatăl tău și pe frații tăi în cel mai bun loc din țară. Să locuiască ei în pământul Goșen, și de cunoști printre ei oameni pricepuți, pune-i supraveghetori peste vitele mele!"
\par 7 Apoi a adus Iosif înăuntru pe Iacov, tatăl său, și l-a înfățișat înaintea lui Faraon și a binecuvântat Iacov pe Faraon;
\par 8 Iar Faraon a zis către Iacov: "Câți sunt anii vieții tale?"
\par 9 Răspuns-a Iacov lui Faraon: "Zilele pribegiei mele sunt o sută treizeci de ani. Puține și grele au fost zilele vieții mele și n-am ajuns zilele anilor vieții părinților mei, cât au pribegit ei în zilele lor".
\par 10 Și iarăși a binecuvântat Iacov pe Faraon și a ieșit de la Faraon.
\par 11 Deci a așezat Iosif pe tatăl său și pe frații săi și le-a dat moșie în țara Egiptului, în cea mai bună parte a țării, în pământul Ramses (Goșen), cum îi poruncise Faraon.
\par 12 Și dădea Iosif tatălui său și fraților săi și la toată casa tatălui său pâine după trebuința familiei fiecăruia.
\par 13 În vremea aceea nu era pâine în tot pământul, pentru că se întețise foametea foarte tare, încât țara Egiptului și pământul Canaan se istoviseră de foamete.
\par 14 Și a adunat Iosif tot argintul, ce era în țara Egiptului și în țara Canaanului, pe grâul ce se cumpăra; de la el. Și a adus Iosif argintul tot în casa lui Faraon.
\par 15 Când s-a sfârșit tot argintul în țara Egiptului și în țara Canaanului, au venit atunci toți Egiptenii la Iosif și au zis: "Dă-ne pâine! De ce să murim sub ochii tăi? Că s-a sfârșit argintul".
\par 16 Iar Iosif le-a zis: "Aduceți vitele voastre și vă voi da pâine pe vitele voastre, dacă vi s-a terminat argintul".
\par 17 Și au adus ei la Iosif vitele lor și le-a dat Iosif pâine pe cai și pe oi, pe boi și pe asini; și anul acela i-a hrănit cu pâine pentru toate vitele lor.
\par 18 Iar dacă a trecut anul acela, au venit în anul următor și i-au zis: "Nu vom ascunde de domnul nostru, că argintul nostru s-a sfârșit și vitele noastre sunt la domnul nostru și nimic nu ne-a mai rămas să-i dăm decât trupurile și pământurile noastre.
\par 19 De ce să pierim sub ochii tăi și noi și pământurile noastre? Cumpără-ne pe pâine, pe noi cu pământurile noastre, și vom fi robi lui Faraon, noi și pământurile noastre, iar tu să ne dai sămânță ca să trăim și să nu murim și ca ogoarele să nu rămână paragină".
\par 20 Și a cumpărat Iosif pentru Faraon tot pământul Egiptului, pentru că Egiptenii și-au vândut fiecare pământul său lui Faraon, că-i istovise foametea; și a ajuns tot pământul al lui Faraon.
\par 21 De asemenea și pe popor l-a făcut rob lui, de la un capăt al Egiptului până la celălalt.
\par 22 Numai pământurile preoților nu le-a cumpărat Iosif, căci preoților le era rânduită de la Faraon porție și se hrăneau din porția lor, pe care le-o da Faraon; de aceea nu și-au vândut ei pământul.
\par 23 Atunci a zis Iosif către popor: "Iată, eu v-am cumpărat astăzi pentru Faraon și pe voi și pământul vostru. Luați-vă sămânță și semănați pământul.
\par 24 Și la seceriș să dați a cincea parte lui Faraon, iar patru părți să vă rămână vouă pentru semănatul ogoarelor, pentru hrana voastră și a celor ce sunt în casele voastre și pentru hrana copiilor voștri".
\par 25 Iar ei au zis: "Tu ne-ai salvat viața! Să aflăm milă în ochii domnului nostru și să fim robi lui Faraon!"
\par 26 Și le-a pus Iosif lege, care-i până astăzi în țara Egiptului, ca a cincea parte să se dea lui Faraon, scutit fiind numai pământul preoților, care nu era al lui Faraon.
\par 27 Astfel s-a așezat Israel în țara Egiptului, în ținutul Goșen, și l-a moștenit și a crescut și s-a înmulțit foarte tare.
\par 28 Iacov a mai trăit în țara Egiptului șaptesprezece ani. Zilele vieții lui Iacov au fost deci o sută patruzeci și șapte de ani.
\par 29 Apoi venindu-i lui vremea să moară, Israel a chemat pe fiul său Iosif, și i-a zis: "De am aflat har în ochii tăi, pune-ți mâna pe coapsa mea și jură că vei face milă și dreptate cu mine, să nu mă îngropi în Egipt!
\par 30 Când voi adormi ca părinții mei, mă vei scoate din Egipt și mă vei îngropa în mormântul lor". Iar Iosif a zis: "Voi face după cuvântul tău!"
\par 31 Iacov însă a zis: "Jură-mi!" Și i s-a jurat Iosif. Atunci Israel s-a închinat la vârful toiagului său.

\chapter{48}

\par 1 După aceea i s-a spus lui Iosif: "Tatăl tău e bolnav". Atunci a luat el cu sine pe cei doi fii ai săi, pe Manase și pe Efraim, și a venit la Iacov.
\par 2 Și i s-a dat de veste lui Iacov, spunându-i-se: "Iată Iosif, fiul tău, vine să te vadă". Și Israel, adunându-și puterile sale, s-a ridicat în pat.
\par 3 Și a zis Iacov către Iosif: "Dumnezeu Atotputernicul mi S-a arătat în Luz, în pământul Canaan și m-a binecuvântat.
\par 4 Și mi-a zis: "Iată, te voi crește și te voi înmulți, și voi ridica din tine mulțime de popoare, și pământul acesta îl voi da urmașilor tăi, ca moștenire veșnică".
\par 5 Acum deci cei doi fii ai tăi, care ți s-au născut în pământul Egiptului, înainte de a veni eu la tine în Egipt, să fie ai mei; Efraim și Manase să fie ai mei, ca Ruben și Simeon.
\par 6 Iar copiii, ce se vor naște de acum din tine, să fie ai tăi și se vor numi ei cu numele fraților lor, în triburile acelora.
\par 7 Când veneam eu din Mesopotamia, mi-a murit Rahila, mama ta, pe drum, în pământul Canaan, puțin înainte de a ajunge la Efrata, și am îngropat-o acolo, lângă drumul spre Efrata sau Betleem".
\par 8 Văzând apoi pe fiii lui Iosif, Israel a zis: "Cine sunt aceștia?"
\par 9 Răspuns-a Iosif tatălui său: "Aceștia sunt fiii mei, pe care mi i-a dat Dumnezeu aici!" Iar Iacov a zis: "Apropie-i de mine ca să-i binecuvântez!"
\par 10 Ochii lui Israel însă erau întunecați de bătrânețe și nu mai puteau să vadă. Și a apropiat Iosif pe fiii săi de el, iar el i-a îmbrățișat și i-a sărutat.
\par 11 Apoi a zis iarăși Israel către Iosif: "Nu nădăjduiam să mai văd fața ta și iată Dumnezeu mi-a arătat și pe urmașii tăi".
\par 12 Și depărtându-i de genunchii tatălui său, Iosif i s-a închinat lui Israel până la pământ.
\par 13 După aceea luând Iosif pe cei doi fii ai săi, pe Efraim cu dreapta sa în fața stângei lui Israel, iar pe Manase cu stânga sa în fala dreptei lui Israel, i-a apropiat de el.
\par 14 Israel insă și-a întins mâna sa cea dreaptă și a pus-o pe capul lui Efraim, deși acesta era mai mic, iar stânga și-a pus-o pe capul lui Manase. Înadins și-a încrucișat mâinile, deși Manase era întâiul născut.
\par 15 Și i-a binecuvântat, zicând: "Dumnezeul, înaintea Căruia au umblat părinții mei: Avraam și Isaac, Dumnezeul Cel ce m-a călăuzit de când sunt și până în ziua aceasta;
\par 16 Îngerul ce m-a izbăvit pe mine de tot răul să binecuvânteze pruncii aceștia, să poarte ei numele meu și numele părinților mei: Avraam și Isaac, și să crească din ei mulțime mare pe pământ!"
\par 17 Și Iosif, văzând că tatăl său și-a pus mâna sa cea dreaptă pe capul lui Efraim, i-a părut rău și, luând mâna tatălui său ca să o mute de pe capul lui Efraim pe capul lui Manase,
\par 18 A zis către tatăl său: "Nu așa, tată, că cestălalt este întâiul născut. Pune dar pe capul lui mâna ta cea dreaptă!"
\par 19 Tatăl său însă n-a voit, ci a zis: "Știu, fiul meu, știu! Și din el va ieși un popor și el va fi mare; dar fratele lui cel mai mic va fi mai mare decât el și din sămânța lui vor ieși popoare nenumărate".
\par 20 Și i-a binecuvântat pe ei în ziua aceea, zicând: "Cu voi se va binecuvânta în Israel și se va zice: Dumnezeu să te facă așa ca pe Efraim și ca pe Manase!" Și așa a pus mâna pe Efraim înaintea lui Manase.
\par 21 Apoi a zis Israel către Iosif: "Iată, eu mor; dar Dumnezeu va fi cu voi și vă va întoarce în țara părinților voștri.
\par 22 Deci eu îți dau ție, peste ceea ce au frații tăi, Sichemul, pe care l-am luat eu cu sabia mea și cu arcul meu din mâinile Amoreilor".

\chapter{49}

\par 1 Apoi a chemat Iacov pe fiii săi și le-a zis: "Adunați-vă, ca să vă spun ce are să fie cu voi în zilele cele de apoi.
\par 2 Adunați-vă și ascultați-mă, fiii lui Iacov, ascultați pe Israel, ascultați pe tatăl vostru!
\par 3 Ruben, întâi-născutul meu, tăria mea și începătura puterii mele, culmea vredniciei și culmea destoiniciei;
\par 4 Tu ai clocotit ca apa și nu vei avea întâietatea, pentru că te-ai suit în patul tatălui tău și mi-ai pângărit așternutul pe care te-ai suit.
\par 5 Frații Simeon și Levi... Unelte ale cruzimii sunt săbiile lor.
\par 6 În sfatul lor să nu intre sufletul meu și în adunarea lor să nu fie părtașă slava mea, căci ei, în mânia lor, au ucis oameni și, la supărarea lor, au ologit tauri!
\par 7 Blestemată să fie mânia lor, căci ea a fost silnică, și aprinderea lor, căci a fost crudă; îi voi împărți pe ei în Iacov și îi voi risipi în Israel.
\par 8 Iudo, pe tine te vor lăuda frații tăi. Mâinile tale să fie în ceafa vrăjmașilor tăi. Închina-se-vor ție feciorii tatălui tău.
\par 9 Pui de leu ești, Iudo, fiul meu! De la jaf te-ai întors... El a îndoit genunchii și s-a culcat ca un leu, ca o leoaică... Cine-l va deștepta?
\par 10 Nu va lipsi sceptru din Iuda, nici toiag de cârmuitor din coapsele sale, până ce va veni împăciuitorul, Căruia se vor supune popoarele.
\par 11 Acela își va lega de viță asinul Său, de coardă mânzul asinei Sale. Spăla-va în vin haina Sa și în sânge de strugure veșmântul Său!
\par 12 Ochii Lui vor scânteia ca vinul și dinții Săi vor fi albi ca laptele.
\par 13 Zabulon va locui lângă mare, va da liman corăbiilor și marginea hotarului lui va fi până la Sidon.
\par 14 Isahar este ca asinul voinic, care odihnește între staule.
\par 15 Văzând că odihna e bună și ținutul său gras, își pleacă umerii sub povară și se face bărbat plătitor de bir.
\par 16 Dan va judeca pe poporul său, ca pe una din semințiile lui Israel.
\par 17 Dan va fi șarpe la drum, viperă la potecă, înveninând piciorul calului, ca să cadă călărețul.
\par 18 În ajutorul Tău nădăjduiesc, o, Doamne!
\par 19 Gad, strâmtorat va fi de cete înarmate, dar le va strâmtora și el pas cu pas.
\par 20 Din Așer va veni pâinea cea grasă și regilor le va da mâncăruri alese.
\par 21 Neftali, cerboaică slobodă: el rostește graiuri minunate.
\par 22 Iosif, ramură de pom roditor, ramură de pom roditor lângă izvor, ramurile lui se revarsă peste ziduri.
\par 23 Îl vor amărî și îl vor dușmăni; înspre el arunca-vor săgeți și îl vor sili la luptă.
\par 24 Dar arcul lui va rămâne tare și mușchii brațului lui întăriți, mulțumită Dumnezeului celui puternic al lui Iacov, Cel ce este păstorul și tăria lui Israel.
\par 25 De la Dumnezeul tatălui tău, și El te va ajuta; și de la cel Atotputernic - El te va binecuvânta; de la El să vină binecuvântările, de sus din ceruri, și binecuvântările adâncului de jos, binecuvântările sânilor și ale pântecelui.
\par 26 Binecuvântările tatălui tău întrec binecuvântările munților celor din veac și frumusețea dealurilor celor veșnice. Aceste binecuvântări să fie pe capul lui Iosif, pe creștetul celui mai ales între frații lui.
\par 27 Veniamin, lup răpitor, dimineața va mânca vânat și pradă va împărți seara".
\par 28 Iată toate cele douăsprezece seminții ale lui Israel și iată ce le-a spus tatăl lor, când le-a binecuvântat și a dat fiecăreia binecuvântarea cuvenită.
\par 29 Apoi le-a poruncit: "Eu am să trec la poporul meu. Să mă îngropați lângă părinții mei, în peștera din țarina lui Efron Heteul.
\par 30 În peștera din țarina Macpela, în fața lui Mamvri, în pământul Canaan, pe care a cumpărat-o Avraam de la Efron Heteul, împreună cu țarina, ca moșie de înmormântare.
\par 31 Acolo au fost îngropați Avraam și Sarra, femeia sa, acolo au fost îngropați Isaac și Rebeca, femeia lui, și tot acolo am îngropat și eu pe Lia.
\par 32 Această țarină și peștera din ea au fost cumpărate de la feciorii Heteilor".
\par 33 Sfârșind Iacov poruncile sale, pe care le-a dat feciorilor săi, și întinzându-și picioarele sale în pat, și-a dat sfârșitul și s-a adăugat la poporul său.

\chapter{50}

\par 1 Atunci Iosif, căzând pe fața tatălui său, l-a plâns și l-a sărutat.
\par 2 Apoi a poruncit Iosif doctorilor, slujitori ai săi, să îmbălsămeze pe tatăl său și doctorii au îmbălsămat pe Israel.
\par 3 După ce s-au împlinit patruzeci de zile, că atitea zile trebuie pentru îmbălsămare, l-au plâns Egiptenii șaptezeci de zile.
\par 4 Iar dacă au trecut zilele plângerii lui, a zis Iosif curtenilor lui Faraon: "De am aflat bunăvoință in ochii voștri, ziceți lui Faraon așa :
\par 5 Tatăl meu m-a jurat și a zis: Iată, eu am să mor; tu însă să mă îngropi în mormântul meu, pe care mi l-am săpat eu în pământul Canaan. Și acum aș vrea să mă duc ca să îngrop pe tatăl meu și să mă întorc". Și i s-au spus lui Faraon cuvintele lui Iosif,
\par 6 Iar Faraon a răspuns : "Du-te și îngroapă pe tatăl tău, cum te-a jurat el!"
\par 7 Deci, s-a dus Iosif să îngroape pe tatăl său și au mers împreună cu el toți slujitorii lui Faraon, bătrinii casei lui și toți bătrinii din țara Egiptului
\par 8 Și toată casa lui Iosif și frații lui și toată casa tatălui său și neamul lui. Numai copiii lor și oile și vitele lor le-au lăsat în țara Goșen.
\par 9 Au plecat de asemenea cu el căruțe și călăreți și s-a făcut tabăra mare foarte.
\par 10 Și ajungând ei la aria lui Atad de lângă Iordan, au plâns acolo plângere mare și tare foarte și a jelit Iosif pe tatăl său șapte zile.
\par 11 Văzind Canaaneii, locuitorii ținutului aceluia, plângerea de la aria Atad, au zis : "Mare e plângerea aceasta la Egipteni". De aceea s-a dat locului aceluia numele Abel-Mițraim, adică plângerea Egiptenilor, care loc e dincolo de Iordan.
\par 12 Așa au făcut fiii lui Iacov cu Iacov, cum le poruncise el :
\par 13 L-au dus fiii lui în pământul Canaan și l-au îngropat în peștera din țarina Macpela, cea de lângă Mamvri, pe care o cumpărase Avraam cu țarină cu tot de la Efron Heteul, ca moșie de îngropare.
\par 14 Apoi Iosif, după îngroparea tatălui său, s-a întors în Egipt, și el, și frații lui, și toți cei ce merseseră cu el la îngroparea tatălui său.
\par 15 Văzând însă frații lui Iosif că a murit tatăl lor, au zis ei : "Ce vom face, dacă Iosif ne va urî și va vrea să se răzbune pentru răul ce i-am făcut?"
\par 16 Atunci au trimis ei la Iosif să i se spună : "Tatăl tău înainte de moarte te-a jurat și a zis:
\par 17 "Așa să spuneți lui Iosif: Iartă fraților tăi greșeala și păcatul lor și răul ce ți-au făcut. Iartă deci vina robilor Dumnezeului tatălui tău!" Și a plâns Iosif când i s-au spus acestea.
\par 18 Apoi au venit și frații lui și, căzând înaintea lui, au zis: "Iată, noi suntem robii tăi".
\par 19 Iar Iosif le-a zis : "Nu vă temeți! Sânt eu, oare, în locul lui Dumnezeu?
\par 20 Iată, voi ați uneltit asupra mea rele, dar Dumnezeu le-a intors în bine, ca să facă cele ce sint acum și să păstreze viața unui popor numeros.
\par 21 Deci nu vă mai temeți! Eu vă voi hrăni pe voi și pe copiii voștri". Și i-a mângiiat și le-a vorbit de la inimă.
\par 22 Apoi a locuit Iosif în Egipt, el și frații lui și toată casa tatălui său. Și a trăit Iosif o sută zece ani.
\par 23 Și a văzut Iosif pe urmașii lui Efraim pină la al treilea neam. De asemenea și copiii lui Machir, fiul lui Manase, s-au născut pe genunchii lui Iosif.
\par 24 In cele din urmă a zis Iosif către frații săi: "Iată, am să mor. Dar Dumnezeu vă va cerceta, vă va scoate din pământul acesta și vă va duce în pământul pentru care Dumnezeul părinților noștri S-a jurat lui Avraam și lui Isaac și lui Iacov".
\par 25 La urmă a jurat Iosif pe fiii lui Israel, zicând: "Dumnezeu are să vă cerceteze, dar voi să scoateți oasele mele de aici!"
\par 26 Și a murit Iosif de o sută zece ani. L-au îmbălsămat și l-au pus într-un sicriu, în pămîntul Egiptului.

\end{document}