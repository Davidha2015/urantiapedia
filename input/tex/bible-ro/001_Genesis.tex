\begin{document}

\title{Genesis}

Gen 1:1  La început a facut Dumnezeu cerul ?i pamântul.
Gen 1:2  ?i pamântul era netocmit ?i gol. Întuneric era deasupra adâncului ?i Duhul lui Dumnezeu Se purta pe deasupra apelor.
Gen 1:3  ?i a zis Dumnezeu: "Sa fie lumina!" ?i a fost lumina.
Gen 1:4  ?i a vazut Dumnezeu ca este buna lumina, ?i a despar?it Dumnezeu lumina de întuneric.
Gen 1:5  Lumina a numit-o Dumnezeu ziua, iar întunericul l-a numit noapte. ?i a fost seara ?i a fost diminea?a: ziua întâi.
Gen 1:6  ?i a zis Dumnezeu: "Sa fie o tarie prin mijlocul apelor ?i sa desparta ape de ape!" ?i a fost a?a.
Gen 1:7  A facut Dumnezeu taria ?i a despar?it Dumnezeu apele cele de sub tarie de apele cele de deasupra tariei.
Gen 1:8  Taria a numit-o Dumnezeu cer. ?i a vazut Dumnezeu ca este bine. ?i a fost seara ?i a fost diminea?a: ziua a doua.
Gen 1:9  ?i a zis Dumnezeu: "Sa se adune apele cele de sub cer la un loc ?i sa se arate uscatul!" ?i a fost a?a. ?i s-au adunat apele cele de sub cer la locurile lor ?i s-a aratat uscatul.
Gen 1:10  Uscatul l-a numit Dumnezeu pamânt, iar adunarea apelor a numit-o mari. ?i a vazut Dumnezeu ca este bine.
Gen 1:11  Apoi a zis Dumnezeu: "Sa dea pamântul din sine verdea?a: iarba, cu samân?a într-însa, dupa felul ?i asemanarea ei, ?i pomi roditori, care sa dea rod cu samân?a în sine, dupa fel, pe pamânt!" ?i a fost a?a.
Gen 1:12  Pamântul a dat din sine verdea?a: iarba, care face samân?a, dupa felul ?i dupa asemanarea ei, ?i pomi roditori, cu samân?a, dupa fel, pe pamânt. ?i a vazut Dumnezeu ca este bine.
Gen 1:13  ?i a fost seara ?i a fost diminea?a: ziua a treia.
Gen 1:14  ?i a zis Dumnezeu: "Sa fie luminatori pe taria cerului, ca sa lumineze pe pamânt, sa desparta ziua de noapte ?i sa fie semne ca sa deosebeasca anotimpurile, zilele ?i anii,
Gen 1:15  ?i sa slujeasca drept luminatori pe taria cerului, ca sa lumineze pamântul." ?i a fost a?a.
Gen 1:16  A facut Dumnezeu cei doi luminatori mari: luminatorul cel mai mare pentru cârmuirea zilei ?i luminatorul cel mai mic pentru cârmuirea nop?ii, ?i stelele.
Gen 1:17  ?i le-a pus Dumnezeu pe taria cerului, ca sa lumineze pamântul,
Gen 1:18  Sa cârmuiasca ziua ?i noaptea ?i sa desparta lumina de întuneric. ?i a vazut Dumnezeu ca este bine.
Gen 1:19  ?i a fost seara ?i a fost diminea?a: ziua a patra.
Gen 1:20  Apoi a zis Dumnezeu: "Sa mi?une apele de vieta?i, fiin?e cu via?a în ele ?i pasari sa zboare pe pamânt, pe întinsul tariei cerului!" ?i a fost a?a.
Gen 1:21  A facut Dumnezeu animalele cele mari din ape ?i toate fiin?ele vii, care mi?una în ape, unde ele se prasesc dupa felul lor, ?i toate pasarile înaripate dupa felul lor. ?i a vazut Dumnezeu ca este bine.
Gen 1:22  ?i le-a binecuvântat Dumnezeu ?i a zis: "Prasi?i-va ?i va înmul?i?i ?i umple?i apele marilor ?i pasarile sa se înmul?easca pe pamânt!"
Gen 1:23  ?i a fost seara ?i a fost diminea?a: ziua a cincea.
Gen 1:24  Apoi a zis Dumnezeu: "Sa scoata pamântul fiin?e vii, dupa felul lor: animale, târâtoare ?i fiare salbatice dupa felul lor". ?i a fost a?a.
Gen 1:25  A facut Dumnezeu fiarele salbatice dupa felul lor, ?i animalele domestice dupa felul lor, ?i toate târâtoarele pamântului dupa felul lor. ?i a vazut Dumnezeu ca este bine.
Gen 1:26  ?i a zis Dumnezeu: "Sa facem om dupa chipul ?i dupa asemanarea Noastra, ca sa stapâneasca pe?tii marii, pasarile cerului, animalele domestice, toate vieta?ile ce se târasc pe pamânt ?i tot pamântul!"
Gen 1:27  ?i a facut Dumnezeu pe om dupa chipul Sau; dupa chipul lui Dumnezeu l-a facut; a facut barbat ?i femeie.
Gen 1:28  ?i Dumnezeu i-a binecuvântat, zicând: "Cre?te?i ?i va înmul?i?i ?i umple?i pamântul ?i-l supune?i; ?i stapâniri peste pe?tii marii, peste pasarile cerului, peste toate animalele, peste toate vieta?ile ce se mi?ca pe pamânt ?i peste tot pamântul!"
Gen 1:29  Apoi a zis Dumnezeu: "Iata, va dau toata iarba ce face samân?a de pe toata fa?a pamântului ?i tot pomul ce are rod cu samân?a în el. Acestea vor fi hrana voastra.
Gen 1:30  Iar tuturor fiarelor pamântului ?i tuturor pasarilor cerului ?i tuturor vieta?ilor ce se mi?ca pe pamânt, care au în ele suflare de via?a, le dau toata iarba verde spre hrana." ?i a fost a?a.
Gen 1:31  ?i a privit Dumnezeu toate câte a facut ?i iata erau bune foarte. ?i a fost seara ?i a fost diminea?a: ziua a ?asea.
Gen 2:1  A?a s-au facut cerul ?i pamântul ?i toata o?tirea lor.
Gen 2:2  ?i a sfâr?it Dumnezeu în ziua a ?asea lucrarea Sa, pe care a facut-o; iar în ziua a ?aptea S-a odihnit de toate lucrurile Sale, pe care le-a facut.
Gen 2:3  ?i a binecuvântat Dumnezeu ziua a ?aptea ?i a sfin?it-o, pentru ca într-însa S-a odihnit de toate lucrurile Sale, pe care le-a facut ?i le-a pus în rânduiala.
Gen 2:4  Iata obâr?ia cerului ?i a pamântului de la facerea lor, din ziua când Domnul Dumnezeu a facut cerul ?i pamântul.
Gen 2:5  Pe câmp nu se afla nici un copacel, iar iarba de pe el nu începuse a odrasli, pentru ca Domnul Dumnezeu nu trimisese înca ploaie pe pamânt ?i nu era nimeni ca sa lucreze pamântul.
Gen 2:6  Ci numai abur ie?ea din pamânt ?i umezea toata fa?a pamântului.
Gen 2:7  Atunci, luând Domnul Dumnezeu ?arâna din pamânt, a facut pe om ?i a suflat în fa?a lui suflare de via?a ?i s-a facut omul fiin?a vie.
Gen 2:8  Apoi Domnul Dumnezeu a sadit o gradina în Eden, spre rasarit, ?i a pus acolo pe omul pe care-l zidise.
Gen 2:9  ?i a facut Domnul Dumnezeu sa rasara din pamânt tot soiul de pomi, placu?i la vedere ?i cu roade bune de mâncat; ier în mijlocul raiului era pomul vie?ii ?i pomul cuno?tin?ei binelui ?i raului.
Gen 2:10  ?i din Eden ie?ea un râu, care uda raiul, iar de acolo se împar?ea în patru bra?e.
Gen 2:11  Numele unuia era Fison. Acesta înconjura toata ?ara Havila, în care se afla aur.
Gen 2:12  Aurul din ?ara aceea este bun; tot acolo se gase?te bdeliu ?i piatra de onix.
Gen 2:13  Numele râului al doilea este Gihon. Acesta înconjura toata ?ara Cu?.
Gen 2:14  Numele râului al treilea este Tigru. Acesta curge prin fa?a Asiriei; iar râul al patrulea este Eufratul.
Gen 2:15  ?i a luat Domnul Dumnezeu pe omul pe care-l facuse ?i l-a pus în gradina cea din Eden, ca s-o lucreze ?i s-o pazeasca.
Gen 2:16  A dat apoi Domnul Dumnezeu lui Adam porunca ?i î zis: "Din to?i pomii din rai po?i sa manânci,
Gen 2:17  Iar din pomul cuno?tin?ei binelui ?i raului sa nu manânci, caci, în ziua în care vei mânca din el, vei muri negre?it!
Gen 2:18  ?i a zis Domnul Dumnezeu: "Nu este bine sa fie omul singur; sa-i facem ajutor potrivit pentru el".
Gen 2:19  ?i Domnul Dumnezeu, Care facuse din pamânt toate fiarele câmpului ?i toate pasarile cerului, le-a adus la Adam, ca sa vada cum le va numi; a?a ca toate fiin?ele vii sa se numeasca precum le va numi Adam.
Gen 2:20  ?i a pus Adam nume tuturor animalelor ?i tuturor pasarilor cerului ?i tuturor fiarelor salbatice; dar pentru Adam nu s-a gasit ajutor de potriva lui.
Gen 2:21  Atunci a adus Domnul Dumnezeu asupra lui Adam somn greu; ?i, daca a adormit, a luat una din coastele lui ?i a plinit locul ei cu carne.
Gen 2:22  Iar coasta luata din Adam a facut-o Domnul Dumnezeu femeie ?i a adus-o la Adam.
Gen 2:23  ?i a zis Adam: "Iata aceasta-i os din oasele mele ?i carne din carnea mea; ea se va numi femeie, pentru ca este luata din barbatul sau.
Gen 2:24  De aceea va lasa omul pe tatal sau ?i pe mama sa ?i se va uni cu femeia sa ?i vor fi amândoi un trup."
Gen 2:25  Adam ?i femeia lui erau amândoi goi ?i nu se ru?inau.
Gen 3:1  ?arpele însa era cel mai ?iret dintre toate fiarele de pe pamânt, pe care le facuse Domnul Dumnezeu. ?i a zis ?arpele catre femeie: "Dumnezeu a zis El, oare, sa nu mânca?i roade din orice pom din rai?"
Gen 3:2  Iar femeia a zis catre ?arpe: "Roade din pomii raiului putem sa mâncam;
Gen 3:3  Numai din rodul pomului celui din mijlocul raiului ne-a zis Dumnezeu: "Sa nu mânca?i din el, nici sa va atinge?i de el, ca sa nu muri?i!"
Gen 3:4  Atunci ?arpele a zis catre femeie: "Nu, nu ve?i muri!
Gen 3:5  Dar Dumnezeu ?tie ca în ziua în care ve?i mânca din el vi se vor deschide ochii ?i ve?i fi ca Dumnezeu, cunoscând binele ?i raul".
Gen 3:6  De aceea femeia, socotind ca rodul pomului este bun de mâncat ?i placut ochilor la vedere ?i vrednic de dorit, pentru ca da ?tiin?a, a luat din el ?i a mâncat ?i a dat barbatului sau ?i a mâncat ?i el.
Gen 3:7  Atunci li s-au deschis ochii la amândoi ?i au cunoscut ca erau goi, ?i au cusut frunze de smochin ?i ?i-au facut acoperaminte.
Gen 3:8  Iar când au auzit glasul Domnului Dumnezeu, Care umbla prin rai, în racoarea serii, s-au ascuns Adam ?i femeia lui de fa?a Domnului Dumnezeu printre pomii raiului.
Gen 3:9  ?i a strigat Domnul Dumnezeu pe Adam ?i i-a zis: "Adame, unde e?ti?"
Gen 3:10  Raspuns-a acesta: "Am auzit glasul Tau în rai ?i m-am temut, caci sunt gol, ?i m-am ascuns".
Gen 3:11  ?i i-a zis Dumnezeu: "Cine ti-a spus ca e?ti gol? Nu cumva ai mâncat din pomul din care ?i-am poruncit sa nu manânci?"
Gen 3:12  Zis-a Adam: "Femeia care mi-ai dat-o sa fie cu mine, aceea mi-a dat din pom ?i am mâncat".
Gen 3:13  ?i a zis Domnul Dumnezeu catre femeie: "Pentru ce ai facut aceasta?" Iar femeia a zis: "?arpele m-a amagit ?i eu am mâncat".
Gen 3:14  Zis-a Domnul Dumnezeu catre ?arpe: "Pentru ca ai facut aceasta, blestemat sa fii între toate animalele ?i între toate fiarele câmpului; pe pântecele tau sa te târa?ti ?i ?arâna sa manânci în toate zilele vie?ii tale!
Gen 3:15  Du?manie voi pune între tine ?i între femeie, între samân?a ta ?i samân?a ei; aceasta î?i va zdrobi capul, iar tu îi vei în?epa calcâiul".
Gen 3:16  Iar femeii i-a zis: "Voi înmul?i mereu necazurile tale, mai ales în vremea sarcinii tale; în dureri vei na?te copii; atrasa vei fi catre barbatul tau ?i el te va stapâni".
Gen 3:17  Iar lui Adam i-a zis: "Pentru ca ai ascultat vorba femeii tale ?i ai mâncat din pomul din care ?i-am poruncit: "Sa nu manânci", blestemat va fi pamântul pentru tine! Cu osteneala sa te hrane?ti din el în toate zilele vie?ii tale!
Gen 3:18  Spini ?i palamida î?i va rodi el ?i te vei hrani cu iarba câmpului!
Gen 3:19  În sudoarea fe?ei tale î?i vei mânca pâinea ta, pâna te vei întoarce în pamântul din care e?ti luat; caci pamânt e?ti ?i în pamânt te vei întoarce".
Gen 3:20  ?i a pus Adam femeii sale numele Eva, adica via?a, pentru ca ea era sa fie mama tuturor celor vii.
Gen 3:21  Apoi a facut Domnul Dumnezeu lui Adam ?i femeii lui îmbracaminte de piele ?i i-a îmbracat.
Gen 3:22  ?i a zis Domnul Dumnezeu: "Iata Adam s-a facut ca unul dintre Noi, cunoscând binele ?i raul. ?i acum nu cumva sa-?i întinda mâna ?i sa ia roade din pomul vie?ii, sa manânce ?i sa traiasca în veci!..."
Gen 3:23  De aceea l-a scos Domnul Dumnezeu din gradina cea din Eden, ca sa lucreze pamântul, din care fusese luat.
Gen 3:24  ?i izgonind pe Adam, l-a a?ezat în preajma gradinii celei din Eden ?i a pus heruvimi ?i sabie de flacara vâlvâitoare, sa pazeasca drumul catre pomul vie?ii.
Gen 4:1  Dupa aceea a cunoscut Adam pe Eva, femeia sa, ?i ea, zamislind, a nascut pe Cain ?i a zis: "Am dobândit om de la Dumnezeu".
Gen 4:2  Apoi a mai nascut pe Abel, fratele lui Cain. Abel a fost pastor de oi, iar Cain lucrator de pamânt.
Gen 4:3  Dar dupa un timp, Cain a adus jertfa lui Dumnezeu din roadele pamântului.
Gen 4:4  ?i a adus ?i Abel din cele întâi-nascute ale oilor sale ?i din grasimea lor. ?i a cautat Domnul spre Abel ?i spre darurile lui,
Gen 4:5  Iar spre Cain ?i spre darurile lui n-a cautat. ?i s-a întristat Cain tare ?i fa?a lui era posomorâta.
Gen 4:6  Atunci a zis Domnul Dumnezeu catre Cain: "Pentru ce te-ai întristat ?i pentru ce s-a posomorât fa?a ta?
Gen 4:7  Când faci bine, oare nu-?i este fa?a senina? Iar de nu faci bine, pacatul bate la u?a ?i cauta sa te târasca, dar tu biruie?te-l!"
Gen 4:8  Dupa aceea Cain a zis catre Abel, fratele sau: "Sa ie?im la câmp!" Iar când erau ei în câmpie, Cain s-a aruncat asupra lui Abel, fratele sau, ?i l-a omorât.
Gen 4:9  Atunci a zis Domnul Dumnezeu catre Cain: "Unde este Abel, fratele tau?" Iar el a raspuns: "Nu ?tiu! Au doara eu sunt pazitorul fratelui meu?"
Gen 4:10  ?i a zis Domnul: "Ce ai facut? Glasul sângelui fratelui tau striga catre Mine din pamânt.
Gen 4:11  ?i acum e?ti blestemat de pamântul care ?i-a deschis gura sa, ca sa primeasca sângele fratelui tau din mâna ta.
Gen 4:12  Când vei lucra pamântul, acesta nu-?i va mai da roadele sale ?ie; zbuciumat ?i fugar vei fi tu pe pamânt".
Gen 4:13  ?i a zis Cain catre Domnul Dumnezeu: "Pedeapsa mea este mai mare decât a? putea-o purta.
Gen 4:14  De ma izgone?ti acum din pamântul acesta, ma voi ascunde de la fa?a Ta ?i voi fi zbuciumat ?i fugar pe pamânt, ?i oricine ma va întâlni, ma va ucide".
Gen 4:15  ?i i-a zis Domnul Dumnezeu: "Nu a?a, ci tot cel ce va ucide pe Cain în?eptit se va pedepsi". ?i a pus Domnul  Dumnezeu semn lui Cain, ca tot cel care îl va întâlni sa nu-l omoare.
Gen 4:16  ?i s-a dus Cain de la fa?a lui Dumnezeu ?i a locuit în ?inutul Nod, la rasarit de Eden.
Gen 4:17  Dupa aceea a cunoscut Cain pe femeia sa ?i ea, zamislind, a nascut pe Enoh. Apoi a zidit Cain o cetate ?i a numit-o, dupa numele fiului sau, Enoh.
Gen 4:18  Iar lui Enoh i s-a nascut Irad; lui Irad i s-a nascut Maleleil; lui Maleleil i s-a nascut Matusal, iar lui Matusal i s-a nascut Lameh.
Gen 4:19  Lameh ?i-a luat doua femei: numele uneia era Ada ?i numele celeilalte era Sela.
Gen 4:20  Ada a nascut pe Iabal; acesta a fost tatal celor ce traiesc în corturi, la turme.
Gen 4:21  Fratele lui se numea Iubal; acesta este tatal tuturor celor ce cânta din chitara ?i din cimpoi.
Gen 4:22  Sela a nascut ?i ea pe Tubalcain, care a fost faurar de unelte de arama ?i de fier. ?i sora lui se chema Noema.
Gen 4:23  ?i a zis Lameh catre femeile sale: "Ada ?i Sela, asculta?i glasul meu! Femeile lui Lameh, lua?i aminte la cuvintele mele: Am ucis un om pentru rana mea ?i un tânar pentru vânataia mea.
Gen 4:24  Daca pentru Cain va fi razbunarea de ?apte ori, apoi pentru Lameh de ?aptezeci de ori câte ?apte!"
Gen 4:25  Adam a cunoscut iara?i pe Eva, femeia sa, ?i ea, zamislind, a nascut un fiu ?i i-a pus numele Set, pentru ca ?i-a zis: "Mi-a dat Dumnezeu alt fiu în locul lui Abel, pe care l-a ucis Cain".
Gen 4:26  Lui Set de asemenea i s-a nascut un fiu ?i i-a pus numele Enos. Atunci au început oamenii a chema numele Domnului Dumnezeu.
Gen 5:1  Iata acum cartea neamului lui Adam. Când a facut Dumnezeu pe Adam, l-a facut dupa chipul lui Dumnezeu.
Gen 5:2  Barbat ?i femeie a facut ?i i-a. binecuvântat ?i le-a pus numele: Om, în ziua în care i-a facut.
Gen 5:3  Adam a trait doua sute treizeci de ani ?i atunci i s-a nascut un fiu dupa asemanarea sa ?i, dupa chipul sau ?i i-a pus numele Set.
Gen 5:4  Zilele pe care le-a trait Adam dupa na?terea lui Set au fost ?apte sute de ani ?i i s-au nascut fii ?i fiice.
Gen 5:5  Iar de toate, zilele vie?ii lui Adam au fost noua sute treizeci de ani ?i apoi a murit.
Gen 5:6  Set a trait doua sute cinci ani ?i i s-a nascut Enos.
Gen 5:7  Dupa na?terea lui Enos, Set a mai trait ?apte sute ?apte ani, ?i i s-au nascut fii ?i fiice.
Gen 5:8  Iar de toate, zilele lui Set au fost noua sute doisprezece ani ?i apoi a murit.
Gen 5:9  Enos a trait o suta nouazeci de ani ?i atunci i s-a nascut Cainan.
Gen 5:10  Dupa na?terea lui Cainan, Enos a mai trait ?apte sute cincisprezece ani ?i i s-au nascut fii ?i fiice.
Gen 5:11  Iar de toate, zilele lui Enos au fost noua sute cinci ani ?i apoi a murit.
Gen 5:12  Cainan a trait o suta ?aptezeci de ani ?i atunci i s-a nascut Maleleil.
Gen 5:13  Dupa na?terea lui Maleleil, Cainan a mai trait ?apte sute patruzeci de ani ?i i s-au nascut fii ?i fiice.
Gen 5:14  Iar de toate, zilele lui Cainan au fost noua sute zece ani ?i apoi a murit.
Gen 5:15  Maleleil a trait o suta ?aizeci ?i cinci de ani ?i atunci i s-a nascut Iared.
Gen 5:16  Dupa na?terea lui Iared, Maleleil a mai trait ?apte sute treizeci de ani ?i i s-au nascut fii ?i fiice.
Gen 5:17  Iar de toate, zilele lui Maleleil au fost opt sute nouazeci ?i cinci de ani ?i apoi a murit.
Gen 5:18  Iared a trait o suta ?aizeci ?i doi de ani ?i atunci i s-a nascut Enoh.
Gen 5:19  Dupa na?terea lui Enoh, Iared a mai trait opt sute de ani ?i i s-au nascut fii ?i fiice.
Gen 5:20  Iar de toate, zilele lui Iared au fost noua sute ?aizeci ?i doi de ani ?i apoi a murit.
Gen 5:21  Enoh a trait o suta ?aizeci ?i cinci de ani, ?i atunci i s-a nascut Matusalem.
Gen 5:22  ?i a umblat Enoh înaintea lui Dumnezeu, dupa na?terea lui Matusalem, doua sute de ani ?i i s-au nascut fii ?i fiice.
Gen 5:23  Iar de toate, zilele lui Enoh au fost trei sute ?aizeci ?i cinci de ani.
Gen 5:24  ?i a placut Enoh lui Dumnezeu ?i apoi nu s-a mai aflat, pentru ca l-a mutat Dumnezeu.
Gen 5:25  Matusalem a trait o suta optzeci ?i ?apte de ani ?i atunci i s-a nascut Lameh.
Gen 5:26  Dupa na?terea lui Lameh, Matusalem a mai trait ?apte sute optzeci ?i doi de ani ?i i s-au nascut fii ?i fiice.
Gen 5:27  Iar de toate, zilele lui Matusalem, pe care le-a trait, au fost noua sute ?aizeci ?i noua de ani ?i apoi a murit.
Gen 5:28  Lameh a trait o suta optzeci ?i opt de ani ?i atunci i s-a nascut un fiu.
Gen 5:29  ?i i-a pus numele Noe, zicând: "Acesta ne va mângâia în lucrul nostru ?i în munca mâinilor noastre, la lucrarea pamântului, pe care l-a blestemat Domnul Dumnezeu!"
Gen 5:30  ?i a mai trait Lameh, dupa na?terea lui Noe, cinci sute ?aizeci ?i cinci de ani ?i i s-au nascut fii ?i fiice.
Gen 5:31  Iar de toate, zilele lui Lameh au fost ?apte sute cincizeci ?i trei de ani ?i apoi a murit.
Gen 5:32  Noe era de cinci sute de ani, când i s-au nascut trei feciori: Sem, Ham ?i Iafet.
Gen 6:1  Iar dupa ce au început a se înmul?i oamenii pe pamânt ?i li s-au nascut fiice,
Gen 6:2  Fiii lui Dumnezeu, vazând ca fiicele oamenilor sunt frumoase, ?i-au ales dintre ele so?ii, care pe cine a voit.
Gen 6:3  Dar Domnul Dumnezeu a zis: "Nu va ramâne Duhul Meu pururea în oamenii ace?tia, pentru ca sunt numai trup. Deci zilele lor sa mai fie o suta douazeci de ani!"
Gen 6:4  În vremea aceea s-au ivit pe pamânt uria?i, mai cu seama de când fiii lui Dumnezeu începusera a intra la fiicele oamenilor ?i acestea începusera a le na?te fii: ace?tia sunt vesti?ii viteji din vechime.
Gen 6:5  Vazând însa Domnul Dumnezeu ca rautatea oamenilor s-a marit pe pamânt ?i ca toate cugetele ?i dorin?ele inimii lor sunt îndreptate la rau în toate zilele,
Gen 6:6  I-a parut rau ?i s-a cait Dumnezeu ca a facut pe om pe pamânt.
Gen 6:7  ?i a zis Domnul: "Pierde-voi de pe fa?a pamântului pe omul pe care l-am facut! De la om pâna la dobitoc ?i de la târâtoare pâna la pasarile cerului, tot voi pierde, caci Îmi pare rau ca le-am facut".
Gen 6:8  Noe însa a aflat har înaintea Domnului Dumnezeu.
Gen 6:9  Iata via?a lui Noe: Noe era om drept ?i neprihanit între oamenii timpului sau ?i mergea pe calea Domnului.
Gen 6:10  ?i i s-au nascut lui Noe trei fii: Sem, Ham ?i Iafet.
Gen 6:11  Pamântul însa se stricase înaintea fe?ei lui Dumnezeu ?i se umpluse pamântul de silnicii.
Gen 6:12  ?i a cautat Domnul Dumnezeu spre pamânt ?i iata era stricat, caci tot trupul se abatuse de la calea sa pe pamânt.
Gen 6:13  Atunci a zis Domnul Dumnezeu catre Noe: "Sosit-a înaintea fe?ei Mele sfâr?itul a tot omul, caci s-a umplut pamântul de nedrepta?ile lor, ?i iata Eu îi voi pierde de pe pamânt.
Gen 6:14  Tu însa fa-?i o corabie de lemn de salcâm. În corabie sa faci despar?ituri ?i smole?te-o cu smoala pe dinauntru ?i pe din afara.
Gen 6:15  Corabia însa sa o faci a?a: lungimea corabiei sa fie de trei sute de co?i, la?imea ei de cincizeci de co?i, iar înal?imea de treizeci de co?i.
Gen 6:16  Sa faci corabiei o fereastra la un cot de la acoperi?, iar u?a corabiei sa o faci într-o parte a ei. De asemenea sa faci într-însa trei rânduri de camari: jos, la mijloc ?i sus.
Gen 6:17  ?i iata Eu voi aduce asupra pamântului potop de apa, ca sa pierd tot trupul de sub cer, în care este suflu de via?a, ?i tot ce este pe pamânt va pieri.
Gen 6:18  Iar cu tine voi face legamântul Meu; ?i vei intra în corabie tu ?i împreuna cu tine vor intra fiii tai, femeia ta ?i femeile fiilor tai.
Gen 6:19  Sa intre în corabie din toate animalele, din toate târâtoarele, din toate fiarele ?i din tot trupul, câte doua, parte barbateasca ?i parte femeiasca, ca sa ramâna cu tine în via?a.
Gen 6:20  Din toate soiurile de pasari înaripate dupa fel, din toate soiurile de animale dupa fel ?i din toate soiurile de târâtoare dupa fel, din toate sa intre la tine câte doua, parte barbateasca ?i parte femeiasca, ca sa ramâna în via?a împreuna cu tine.
Gen 6:21  Iar tu ia cu tine din tot felul de mâncare, cu care va hrani?i; îngrije?te-te ca sa fie aceasta de mâncare pentru tine ?i pentru acelea".
Gen 6:22  ?i a început Noe lucrul ?i precum îi poruncise Domnul Dumnezeu a?a a facut.
Gen 7:1  Dupa aceea a zis Domnul Dumnezeu lui Noe: "Intra în corabie, tu ?i toata casa ta, caci în neamul acesta numai pe tine te-am vazut drept înaintea Mea.
Gen 7:2  Sa iei cu tine din toate animalele curate câte ?apte perechi, parte barbateasca ?i parte femeiasca, iar din animalele necurate câte o pereche, parte barbateasca ?i parte femeiasca.
Gen 7:3  De asemenea ?i din pasarile cerului sa iei: din cele curate câte ?apte perechi, parte barbateasca ?i parte femeiasca, iar din toate pasarile necurate câte o pereche, parte barbateasca ?i parte femeiasca, ca sa le pastrezi soiul pentru tot pamântul.
Gen 7:4  Caci peste ?apte zile Eu voi varsa ploaie pe pamânt, patruzeci de zile ?i patruzeci de nop?i ?i am sa pierd de pe fa?a pamântului toate fapturile câte am facut".
Gen 7:5  ?i a facut Noe toate câte i-a poruncit Domnul Dumnezeu.
Gen 7:6  Noe însa, când a venit asupra pamântului potopul de apa, era de ?ase sute de ani.
Gen 7:7  ?i a intrat Noe în corabie ?i împreuna cu el au intrat fiii lui, femeia lui ?i femeile fiilor lui, ca sa scape de apele potopului.
Gen 7:8  Din pasarile curate ?i din pasarile necurate, din animalele curate ?i din animalele necurate, din fiare ?i din toate cele ce se mi?ca pe pamânt
Gen 7:9  Au intrat la Noe în corabie perechi, perechi, parte barbateasca ?i parte femeiasca, cum poruncise Dumnezeu lui Noe.
Gen 7:10  Iar dupa ?apte zile au venit asupra pamântului apele potopului.
Gen 7:11  În anul ?ase sute al vie?ii lui Noe, în luna a doua, în ziua a douazeci ?i ?aptea a lunii acesteia, chiar în acea zi, s-au desfacut toate izvoarele adâncului celui mare ?i s-au deschis jgheaburile cerului;
Gen 7:12  ?i a plouat pe pamânt patruzeci de zile ?i patruzeci de nop?i.
Gen 7:13  În ziua aceasta a intrat Noe în corabie ?i împreuna cu dânsul au intrat Sem, Ham ?i Iafet, fiii lui Noe, femeia lui Noe ?i cele trei femei ale fiilor lui.
Gen 7:14  Din toate soiurile de fiare de pe pamânt, din toate soiurile de animale, din toate soiurile de vieta?i ce mi?unau pe pamânt, din toate soiurile de zburatoare, din toate pasarile, din toate înaripatele
Gen 7:15  ?i din tot trupul, în care se afla duh de via?a, au intrat cu Noe în corabie, perechi, perechi, parte barbateasca ?i parte femeiasca.
Gen 7:16  ?i cele ce au intrat cu Noe în corabie din tot trupul au intrat parte barbateasca ?i parte femeiasca, precum poruncise Dumnezeu lui Noe. $i a închis Domnul Dumnezeu corabia pe din afara.
Gen 7:17  Potopul a ?inut pe pamânt patruzeci de zile ?i patruzeci de nop?i ?i s-a înmul?it apa ?i a ridicat corabia ?i aceasta s-a înal?at deasupra pamântului.
Gen 7:18  ?i a crescut apa mereu ?i s-a înmul?it foarte tare pe pamânt ?i corabia se purta pe deasupra apei.
Gen 7:19  ?i a sporit apa pe pamânt atât de mult, încât a acoperit to?i mun?ii cei înal?i, care erau sub cer.
Gen 7:20  ?i a acoperit apa to?i mun?ii cei înal?i, ridicându-se cu cincisprezece co?i mai sus de ei.
Gen 7:21  ?i a murit tot trupul ce se mi?ca pe pamânt: pasarile, animalele, fiarele, toate vieta?ile ce mi?unau pe pamânt ?i to?i oamenii.
Gen 7:22  Toate cele de pe uscat, câte aveau suflare de via?a în narile lor, au murit.
Gen 7:23  ?i a?a s-a stins toata fiin?a care se afla pe fa?a a tot pamântul, de la om pâna la dobitoc ?i pâna la târâtoare ?i pâna la pasarile cerului, toate s-au stins de pe pamânt, ?i a ramas numai Noe ?i ce era cu el în corabie.
Gen 7:24  Iar apa a crescut mereu pe pamânt, o suta cincizeci de zile.
Gen 8:1  Dar ?i-a adus aminte Dumnezeu de Noe, de toate fiarele, de toate animalele, de toate pasarile ?i de toate vieta?ile ce se mi?ca, câte erau cu dânsul în corabie; ?i a adus Dumnezeu vânt pe pamânt ?i a încetat apa de a mai cre?te.
Gen 8:2  Atunci s-au încuiat izvoarele adâncului ?i jgheaburile cerului ?i a încetat ploaia din cer.
Gen 8:3  Dupa o suta cincizeci de zile, a început a se scurge apa de pe pamânt ?i a se împu?ina.
Gen 8:4  Iar în luna a ?aptea, în ziua a douazeci ?i ?aptea a lunii acesteia, s-a oprit corabia pe Mun?ii Ararat.
Gen 8:5  Apa a scazut mereu pâna în luna a zecea; iar în ziua întâi a lunii a zecea s-au aratat vârfurile mun?ilor.
Gen 8:6  Dupa patruzeci de zile, a deschis Noe fereastra, pe care o facuse la corabie,
Gen 8:7  ?i a dat drumul corbului, ca sa vada de a scazut apa pe pamânt. Acesta, zburând, nu s-a mai întors pâna ce a secat apa de pe pamânt.
Gen 8:8  Apoi, dupa el a trimis porumbelul, ca sa vada de s-a scurs apa de pe pamânt.
Gen 8:9  Porumbelul însa, negasind loc de odihna pentru picioarele sale, s-a întors la el, în corabie; caci era înca apa pe toata fa?a pamântului. ?i a întins Noe mâna ?i l-a apucat ?i l-a bagat la sine, în corabie.
Gen 8:10  ?i a?teptând înca alte ?apte zile, a dat iara?i drumul porumbelului din corabie,
Gen 8:11  ?i porumbelul s-a întors la el, spre seara, ?i iata avea în ciocul sau o ramura verde de maslin. Atunci a cunoscut Noe ca s-a scurs apa de pe fa?a pamântului.
Gen 8:12  Mai zabovind înca alte ?apte zile, iara?i a dat drumul porumbelului ?i el nu s-a mai întors.
Gen 8:13  Iar în anul ?ase sute unu al vie?ii lui Noe, în ziua întâi a lunii întâi, secând apa de pe pamânt, a ridicat Noe acoperi?ul corabiei ?i a privit, ?i iata se zbicise fa?a pamântului.
Gen 8:14  Iar în luna a doua, la douazeci ?i ?apte ale lunii acesteia, pamântul era uscat.
Gen 8:15  Atunci a grait Domnul Dumnezeu lui Noe ?i a zis:
Gen 8:16  "Ie?i din corabie tu ?i împreuna cu tine femeia ta, fiii tai ?i femeile fiilor tai.
Gen 8:17  Scoate de asemenea împreuna cu tine toate vieta?ile, care sunt cu tine, ?i tot trupul, de la pasari ?i pâna la animale, ?i toate vieta?ile ce se mi?ca pe pamânt, ca sa se împra?tie pe pamânt, sa se praseasca ?i sa se înmul?easca pe pamânt".
Gen 8:18  Atunci a ie?it Noe din corabie; ?i împreuna cu el au ie?it fiii lui, femeia lui ?i femeile fiilor lui;
Gen 8:19  Toate fiarele, toate animalele, toate pasarile ?i toate câte se mi?ca pe pamânt, dupa felul lor, au ie?it din corabie.
Gen 8:20  Apoi a facut Noe un jertfelnic Domnului; ?i a luat din animalele cele curate ?i din toate pasarile cele curate ?i le-a adus ardere de tot pe jertfelnic.
Gen 8:21  Iar Domnul Dumnezeu a mirosit mireasma buna ?i a zis Domnul Dumnezeu în inima Sa: "Am socotit sa nu mai blestem pamântul pentru faptele omului, pentru ca cugetul inimii omului se pleaca la rau din tinere?ile lui ?i nu voi mai pierde toate vieta?ile, cum am facut.
Gen 8:22  De acum, cât va trai pamântul, semanatul ?i seceratul, frigul ?i caldura, vara ?i iarna, ziua ?i noaptea nu vor mai înceta!"
Gen 9:1  ?i a binecuvântat Dumnezeu pe Noe ?i pe fiii lui ?i le-a zis: "Na?te?i ?i va înmul?i?i ?i umple?i pamântul ?i-l stapâni?i!
Gen 9:2  Groaza ?i frica de voi sa aiba toate fiarele pamântului; toate pasarile cerului, tot ce se mi?ca pe pamânt ?i to?i pe?tii marii; caci toate acestea vi le-am dat la îndemâna.
Gen 9:3  Tot ce se mi?ca ?i ce traie?te sa va fie de mâncare; toate vi le-am dat, ca ?i iarba verde.
Gen 9:4  Numai carne cu sângele ei, în care e via?a ei, sa nu mânca?i.
Gen 9:5  Caci Eu ?i sângele vostru, în care e via?a voastra, îl voi cere de la orice fiara; ?i voi cere via?a omului ?i din mâna omului, din mâna fratelui sau.
Gen 9:6  De va varsa cineva sânge omenesc, sângele aceluia de mâna de om se va varsa, caci Dumnezeu a facut omul dupa chipul Sau.
Gen 9:7  Voi însa na?te?i ?i va înmul?i?i ?i va raspândi?i pe pamânt ?i-l stapâni?i! "
Gen 9:8  ?i a mai grait Dumnezeu cu Noe ?i cu fiii lui, care erau cu el, ?i a zis:
Gen 9:9  "Iata Eu închei legamântul Meu cu voi, cu urma?ii vo?tri.
Gen 9:10  ?i cu tot sufletul viu, care este cu voi: cu pasarile, cu animalele ?i cu toate fiarele pamântului, care sunt cu voi, cu toate vieta?ile pamântului câte au ie?it din corabie;
Gen 9:11  ?i închei acest legamânt cu voi, ca nu voi mai pierde tot trupul cu apele potopului ?i nu va mai fi potop, ca sa pustiiasca pamântul".
Gen 9:12  Apoi a mai zis iara?i Domnul Dumnezeu catre Noe: "Iata, ca semn al legamântului, pe care-l închei cu voi ?i eu tot sufletul viu ce este cu voi din neam în neam ?i de-a pururi,
Gen 9:13  Pun curcubeul Meu în nori, ca sa fie semn al legamântului dintre Mine ?i pamânt.
Gen 9:14  Când voi aduce nori deasupra pamântului, se va arata curcubeul Meu în nori,
Gen 9:15  ?i-Mi voi aduce aminte de legamântul Meu, pe care l-am încheiat cu voi ?i cu tot sufletul viu ?i cu tot trupul, ?i nu va mai fi apa potop, spre pierzarea a toata faptura.
Gen 9:16  Va fi deci curcubeul Meu în nori ?i-l voi vedea, ?i-Mi voi aduce aminte de legamântul ve?nic dintre Mine ?i pamânt ?i tot sufletul viu din tot trupul ce este pe pamânt!"
Gen 9:17  ?i iara?i a zis Dumnezeu catre Noe: "Acesta este semnul legamântului, pe care Eu l-am încheiat între Mine ?i tot trupul care este pe pamânt".
Gen 9:18  Iar fiii lui Noe; care au ie?it din corabie, erau: Sem, Ham ?i Iafet. Iar Ham era tatal lui Canaan.
Gen 9:19  Ace?tia sunt cei trei fii ai lui Noe ?i din ace?tia s-au înmul?it oamenii pe pamânt.
Gen 9:20  Atunci a început Noe sa fie lucrator de pamânt ?i a sadit vie.
Gen 9:21  A baut vin ?i, îmbatându-se, s-a dezvelit în cortul sau.
Gen 9:22  Iar Ham, tatal lui Canaan, a vazut goliciunea tatalui sau ?i, ie?ind afara, a spus celor doi fra?i ai sai.
Gen 9:23  Dar Sem ?i Iafet au luat o haina ?i, punând-o pe amândoi umerii lor, au intrat cu spatele înainte ?i au acoperit goliciunea tatalui lor; ?i fe?ele lor fiind întoarse înapoi, n-au vazut goliciunea tatalui lor.
Gen 9:24  Trezindu-se Noe din ame?eala de vin ?i aflând ce i-a facut feciorul sau cel mai tânar,
Gen 9:25  A zis: "Blestemat sa fie Canaan! Robul robilor sa fie la fra?ii sai!"
Gen 9:26  Apoi a zis: "Binecuvântat sa fie Domnul Dumnezeul lui Sem; iar Canaan sa-i fie rob!
Gen 9:27  Sa înmul?easca Dumnezeu pe Iafet ?i sa se sala?luiasca acesta în corturile lui Sem, iar Canaan sa-i fie sluga".
Gen 9:28  ?i a mai trait Noe dupa potop trei sute cincizeci de ani.
Gen 9:29  Iar de toate, zilele lui Noe au fost noua sute cincizeci de ani ?i apoi a murit.
Gen 10:1  Iata spi?a neamului fiilor lui Noe: Sem, Ham ?i Iafet, carora li s-au nascut fii dupa potop.
Gen 10:2  Fiii lui Iafet au fost: Gomer, Magog, Madai, Iavan, Tubal, Me?eh ?i Tiras.
Gen 10:3  Fiii lui Gomer au fost: A?chenaz, Rifat ?i Togarma.
Gen 10:4  Fiii lui Iavan au fost: Eli?a, Tar?i?, Chitim ?i Dodanim.
Gen 10:5  Din ace?tia s-au format mul?ime de popoare, care s-au a?ezat în diferite ?ari, fiecare dupa limba sa, dupa neamul sau ?i dupa na?ia sa.
Gen 10:6  Iar fiii lui Ham au fost: Cu?, Mi?raim, Put ?i Canaan.
Gen 10:7  Fiii lui Cu? au fost: ?eba, Havila, Savta, Rama ?i Sabteca. Fiii lui Rama au fost: ?eba ?i Dedan.
Gen 10:8  Cu? a mai nascut de asemenea pe Nimrod; acesta a fost cel dintâi viteaz pe pamânt.
Gen 10:9  El a fost vânator vestit înaintea Domnului Dumnezeu; de aceea se ?i zice: "Vânator vestit ca Nimrod înaintea Domnului Dumnezeu".
Gen 10:10  Împara?ia lui, la început, o alcatuia: Babilonul, apoi Ereh, Acad ?i Calne din ?inutul Senaar.
Gen 10:11  Din pamântul acela, el trecu în Asur, unde a zidit Ninive, cetatea Rehobot, Calah
Gen 10:12  ?i Resen, între Ninive ?i Calah. Aceasta e cetate mare.
Gen 10:13  Din Mi?raim s-au nascut: Ludim, Anamim, Lehabim, Naftuhim,
Gen 10:14  Patrusim, Casluhim - de unde au ie?it Filistenii - ?i Caftorim.
Gen 10:15  Din Canaan s-au nascut: Sidon, întâiul-nascut al sau, apoi Het,
Gen 10:16  Iebuseu, Amoreu, Ghergheseu,
Gen 10:17  Heveu, Archeu, Sineu,
Gen 10:18  Arvadeu, ?emareu ?i Hamateu. Mai pe urma neamurile canaaneiene s-au raspândit.
Gen 10:19  ?i ?inuturile lor se întindeau de la Sidon, spre Gherara pâna la Gaza, iar de aici spre Sodoma, Gomora, Adma ?i ?eboim pâna spre Lasa.
Gen 10:20  Ace?tia sunt fiii lui Ham, dupa familii, limba, ?ari ?i dupa na?ii.
Gen 10:21  De asemenea i s-au nascut fii ?i lui Sem, tatal tuturor fiilor lui Eber ?i fratele mai mare al lui Iafet.
Gen 10:22  Fiii lui Sem au fost: Elam, Asur, Arfaxad, Lud ?i Aram.
Gen 10:23  Iar fiii lui Aram au fost: U?, Hul, Gheter ?i Ma?.
Gen 10:24  Lui Arfaxad i s-a nascut Cainan; lui Cainan i s-a nascut ?elah; lui ?elah i s-a nascut Eber;
Gen 10:25  Iar lui Eber i s-au nascut doi fii: numele unuia era Peleg, pentru ca în zilele lui s-a împar?it pamântul, ?i numele fratelui sau era Ioctan.
Gen 10:26  Lui Ioctan i s-au nascut Almodad, ?alef, Ha?armavet ?i Ierah;
Gen 10:27  Hadoram, Uzal ?i Dicla;
Gen 10:28  Obal, Abimael ?i ?eba;
Gen 10:29  Ofir, Havila ?i Iobab. To?i ace?tia au fost fiii lui Ioctan.
Gen 10:30  Sala?urile lor se întindeau de la Me?a spre Sefar, pâna la Muntele Rasaritului.
Gen 10:31  Ace?tia sunt fiii lui Sem dupa familii, dupa limba, dupa ?ari ?i dupa na?ii.
Gen 10:32  Acestea sunt neamurile, care se trag din fiii lui Noe, dupa familii ?i dupa na?ii, ?i dintr-în?ii s-au raspândit popoarele pe pamânt dupa potop.
Gen 11:1  În vremea aceea era în tot pamântul o singura limba ?i un singur grai la to?i.
Gen 11:2  Purcezând de la rasarit, oamenii au gasit în ?ara Senaar un ?es ?i au descalecat acolo.
Gen 11:3  Apoi au zis unul catre altul: "Haidem sa facem caramizi ?i sa le ardem cu foc!" ?i au folosit caramida în loc de piatra, iar smoala în loc de var.
Gen 11:4  ?i au zis iara?i: "Haidem sa ne facem un ora? ?i un turn al carui vârf sa ajunga la cer ?i sa ne facem faima înainte de a ne împra?tia pe fa?a a tot pamântul!"
Gen 11:5  Atunci S-a pogorât Domnul sa vada cetatea ?i turnul pe care-l zideau fiii oamenilor.
Gen 11:6  ?i a zis Domnul: "Iata, to?i sunt de un neam ?i o limba au ?i iata ce s-au apucat sa faca ?i nu se vor opri de la ceea ce ?i-au pus în gând sa faca.
Gen 11:7  Haidem, dar, sa Ne pogorâm ?i sa amestecam limbile lor, ca sa nu se mai în?eleaga unul cu altul".
Gen 11:8  ?i i-a împra?tiat Domnul de acolo în tot pamântul ?i au încetat de a mai zidi cetatea ?i turnul.
Gen 11:9  De aceea s-a numit cetatea aceea Babilon, pentru ca acolo a amestecat Domnul limbile a tot pamântul ?i de acolo i-a împra?tiat Domnul pe toata fa?a pamântului.
Gen 11:10  Iata acum istoria vie?ii neamului lui Sem: Sem era de o suta de ani, când i s-a nascut Arfaxad, la doi ani dupa potop.
Gen 11:11  Dupa na?terea lui Arfaxad, Sem a mai trait cinci sute de ani ?i a nascut fii ?i fiice ?i apoi a murit.
Gen 11:12  Arfaxad a trait o suta treizeci ?i cinci de ani ?i atunci i s-a nascut Cainan. Dupa na?terea lui Cainan, Arfaxad a mai trait trei sute treizeci de ani ?i a nascut fii ?i fiice ?i apoi a murit.
Gen 11:13  Cainan a trait o suta treizeci de ani ?i atunci i s-a nascut ?elah. Dupa na?terea lui ?elah, Cainan a mai trait trei sute treizeci de ani ?i a nascut fii ?i fiice ?i apoi a murit.
Gen 11:14  ?elah a trait o suta treizeci de ani ?i atunci i s-a nascut Eber.
Gen 11:15  Iar dupa na?terea lui Eber, ?elah a mai trait trei sute treizeci de ani, ?i a nascut fii ?i fiice ?i apoi a murit.
Gen 11:16  Eber a trait o suta treizeci ?i patru de ani ?i atunci i s-a nascut Peleg.
Gen 11:17  Iar dupa na?terea lui Peleg, Eber a mai trait trei sute ?aptezeci de ani ?i a nascut fii ?i fiice ?i apoi a murit.
Gen 11:18  Peleg a trait o suta treizeci de ani ?i atunci i s-a nascut Ragav.
Gen 11:19  Iar dupa na?terea lui Ragav, Peleg a mai trait doua sute noua ani, ?i a nascut fii ?i fiice ?i apoi a murit.
Gen 11:20  Ragav a trait o suta treizeci ?i doi de ani ?i atunci i s-a nascut Serug.
Gen 11:21  Iar dupa na?terea lui Serug, Ragav a mai trait doua sute ?apte ani ?i a nascut fii ?i fiice ?i apoi a murit.
Gen 11:22  Serug a trait o suta treizeci de ani ?i atunci i s-a nascut Nahor.
Gen 11:23  Iar dupa na?terea lui Nahor, Serug a mai trait doua sute de ani ?i a nascut fii ?i fiice ?i apoi a murit.
Gen 11:24  Nahor a trait ?aptezeci ?i noua de ani ?i atunci i s-a nascut Terah.
Gen 11:25  Iar dupa na?terea lui Terah, Nahor a mai trait o suta doua zeci ?i cinci de ani ?i a nascut fii ?i fiice ?i apoi a murit.
Gen 11:26  Terah a trait ?aptezeci de ani ?i atunci i s-au nascut Avram, Nahor ?i Haran.
Gen 11:27  Iar spi?a neamului lui Terah este aceasta: lui Terah i s-au nascut Avram, Nahor ?i Haran; lui Haran i s-a nascut Lot.
Gen 11:28  ?i a murit Haran înainte de Terah, tatal sau, în pamântul de na?tere, în Urul Caldeii.
Gen 11:29  Iar Avram ?i Nahor ?i-au luat femei; numele femeii lui Avram era Sarai, iar numele femeii lui Nahor era Milca, fata lui Haran, tatal Milcai ?i al Iscai.
Gen 11:30  Sarai însa era stearpa ?i nu na?tea copii.
Gen 11:31  ?i a luat Terah pe Avram, fiul sau, ?i pe Lot, fiul lui Haran ?i nepotul sau, ?i pe Sarai, nora sa, ?i femeia lui Avram, fiul sau, ?i a plecat cu ei din Urul Caldeii, ca sa mearga pâna în ?ara Canaanului; dar au mers pâna la Haran ?i s-au a?ezat acolo.
Gen 11:32  De toate, zilele vie?ii lui Terah în pamântul Haran au fost doua sute cinci ani. ?i a murit Terah în Haran.
Gen 12:1  Dupa aceea a zis Domnul catre Avram: "Ie?i din pamântul tau, din neamul tau ?i din casa tatalui tau ?i vino în pamântul pe care ?i-l voi arata Eu.
Gen 12:2  ?i Eu voi ridica din tine un popor mare, te voi binecuvânta, voi mari numele tau ?i vei fi izvor de binecuvântare.
Gen 12:3  Binecuvânta-voi pe cei ce te vor binecuvânta, iar pe cei ce te vor blestema îi voi blestema; ?i se vor binecuvânta întru tine toate neamurile pamântului".
Gen 12:4  Deci a plecat Avram, cum îi zisese Domnul, ?i s-a dus ?i Lot cu el. Avram însa era de ?aptezeci ?i cinci de ani, când a ie?it din Haran.
Gen 12:5  ?i a luat Avram pe Sarai, femeia sa, pe Lot, fiul fratelui sau, ?i toate averile ce agonisisera ei ?i to?i oamenii, pe care-i aveau în Haran, ?i au ie?it, ca sa mearga în ?ara Canaanului ?i au ajuns în Canaan.
Gen 12:6  Apoi a strabatut Avram ?ara aceasta de-a lungul pâna la locul numit Sichem, pâna la stejarul Mamvri. Pe atunci traiau în ?ara aceasta Canaaneii.
Gen 12:7  Acolo S-a aratat Domnul lui Avram ?i i-a zis: "?ara aceasta o voi da urma?ilor tai". ?i a zidit Avram acolo un jertfelnic Domnului, Celui ce Se aratase.
Gen 12:8  De acolo a pornit el spre muntele care e la rasarit de Betel, ?i ?i-a întins acolo cortul a?a, încât Betelul era la apus, iar Hai, la rasarit. A zidit acolo un jertfelnic Domnului ?i s-a închinat Domnului, Celui ce i Se aratase.
Gen 12:9  Apoi s-a ridicat Avram ?i de acolo ?i s-a îndreptat spre miazazi.
Gen 12:10  Pe atunci s-a facut foamete în ?inutul acela ?i s-a coborât Avram în Egipt, ca sa locuiasca acolo, pentru ca se înte?ise foametea în ?inutul acela.
Gen 12:11  Când însa s-a apropiat Avram sa intre în Egipt, a zis catre Sarai, femeia sa: "?tiu ca e?ti femeie frumoasa la chip.
Gen 12:12  De aceea, când te vor vedea Egiptenii, vor zice: "Aceasta-i femeia lui! ?i ma vor ucide pe mine, iar pe tine te vor lasa cu via?a.
Gen 12:13  Zi deci ca-mi e?ti sora, ca sa-mi fie ?i mie bine pentru trecerea ta ?i pentru trecerea ta sa traiesc ?i eu!"
Gen 12:14  Iar dupa ce a sosit Avram în Egipt, au vazut Egiptenii ca femeia lui e foarte frumoasa.
Gen 12:15  ?i au vazut-o ?i dregatorii lui Faraon ?i au laudat-o înaintea lui Faraon ?i au dus-o în casa lui Faraon;
Gen 12:16  ?i pentru ea i-au facut bine lui Avram ?i avea el oi, vite mari ?i asini, slugi ?i slujnice, catâri ?i camile.
Gen 12:17  Domnul însa a lovit cu batai mari ?i grele pe Faraon ?i casa lui, pentru Sarai, femeia lui Avram.
Gen 12:18  ?i chemând Faraon pe Avram, i-a zis: "Ce mi-ai facut? De ce nu mi-ai spus ca aceasta e so?ia ta?
Gen 12:19  Pentru ce ai zis: Mi-e sora? ?i eu am luat-o de femeie: Acum dar iata-?i femeia! Ia-?i-o ?i te du!"
Gen 12:20  ?i a dat Faraon porunca oamenilor sai pentru Avram, ca sa-l petreaca pe el ?i pe femeia lui ?i toate câte avea ?i pe Lot, care îl înso?ea.
Gen 13:1  Din Egipt, Avram cu femeia sa, cu Lot ?i cu toate câte avea, a pornit în par?ile de miazazi ale Canaanului.
Gen 13:2  Avram insa era foarte bogat în vite, în argint ?i în aur.
Gen 13:3  ?i a, înaintat Avram pe unde venise, de la miazazi spre Betel, pâna la locul unde fusese mai înainte cortul sau, între Betel ?i Hai,
Gen 13:4  Adica pâna la locul unde era jertfelnicul pe care-l ridicase el mai înainte, ?i acolo a chemat Avram numele Domnului.
Gen 13:5  ?i Lot, care umbla cu Avram, înca avea oi ?i vite mari ?i corturi.
Gen 13:6  Însa pamântul acela nu-i încapea sa stea împreuna, caci averile lor erau multe ?i nu-i încapea locul sa traiasca împreuna.
Gen 13:7  De aceea se întâmplau certuri între pazitorii vitelor lui Avram ?i pazitorii vitelor lui Lot. Pe atunci locuiau în pamântul acela Canaaneii ?i Ferezeii.
Gen 13:8  Atunci a zis Avram catre Lot: "Sa nu fie sfada între mine ?i tine, între pastorii mei ?i pastorii tai, caci suntem fra?i.
Gen 13:9  Iata, nu e oare tot pamântul înaintea ta? Desparte-te dar de mine! ?i de vei apuca tu la stânga, eu voi apuca la dreapta; iar de vei apuca tu la dreapta, eu voi apuca la stânga".
Gen 13:10  ?i ridicându-?i Lot ochii, a privit toata câmpia Iordanului, care, înainte de a strica Domnul Sodoma ?i Gomora, toata pâna la ?oar era udata de apa, ca raiul Domnului, ca pamântul Egiptului.
Gen 13:11  Deci ?i-a ales Lot tot ?inutul din preajma Iordanului ?i a apucat Lot spre rasarit; ?i a?a s-au despar?it ei unul de altul.
Gen 13:12  Avram a ramas sa locuiasca în pamântul Canaan, iar Lot s-a sala?luit în ceta?ile din ?inutul Iordanului ?i ?i-a întins corturile pâna la Sodoma.
Gen 13:13  Iar oamenii Sodomei erau rai ?i tare pacato?i înaintea Domnului.
Gen 13:14  Deci a zis Domnul catre Avram, dupa ce s-a despar?it Lot de dânsul: "Ridica-?i ochii ?i, din locul în care e?ti acum, cauta spre miazanoapte, spre miazazi ?i rasarit ?i spre mare,
Gen 13:15  Ca tot pamântul, cât îl vezi, ?i-l voi da ?ie ?i urma?ilor tai pentru vecie.
Gen 13:16  Voi face pe urma?ii tai mul?i ca pulberea pamântului; de va putea cineva numara pulberea pamântului, va numara ?i pe urma?ii tai.
Gen 13:17  Scoala ?i cutreiera pamântul acesta în lung ?i în lat, ca ?i-l voi da ?ie ?i urma?ilor tai pentru vecie".
Gen 13:18  ?i ridicându-?i Avram corturile, a venit ?i s-a a?ezat la stejarul Mamvri, care este în Hebron; ?i a zidit acolo un jertfelnic Domnului.
Gen 14:1  Iar în zilele lui Amrafel, regele Senaarului, ale lui Arioc, regele Elasarului, ale lui Kedarlaomer, regele Elamului ?i ale lui Tidal, regele din Gutim,
Gen 14:2  S-a întâmplat sa faca ace?tia razboi cu Bera, regele Sodomei, cu Bir?a, regele Gomorei, cu ?inab, regele Admei, cu ?emeber, regele ?eboimului, ?i cu regele din Bela sau ?oar.
Gen 14:3  To?i ace?tia din urma s-au adunat în valea Sidim, unde e acum Marea cea Sarata.
Gen 14:4  Doisprezece ani statusera ei în robia lui Kedarlaomer, iar în anul al treisprezecelea s-au razvratit.
Gen 14:5  ?i în al patrusprezecelea an au venit Kedarlaomer ?i regii, care ?ineau cu el, ?i au batut pe Refaimi la A?terot-Carnaim, pe Zuzimi la Ham, ?i pe Emimi la ?ave-Chiriataim;
Gen 14:6  Iar pe Horei i-a batut la muntele lor Seir ?i pâna la El-Faran, care e lânga pustiu.
Gen 14:7  Apoi, întorcându-se de acolo, au venit la Ain-Mi?pat sau Cade? ?i au batut toata ?ara Amaleci?ilor ?i pe to?i Amoreii, care locuiau în Ha?a?on-Tamar.
Gen 14:8  Atunci au ie?it regele Sodomei, regele Gomorei, regele Admei, regele ?eboimului ?i regele Belei sau ?oarului ?i s-au batut în valea Sidim
Gen 14:9  Cu Kedarlaomer, regele Elamului, cu Tidal, regele din Gutim, cu Amrafel, regele Senaarului ?i cu Arioc, regele Elasarului: patru regi împotriva a cinci.
Gen 14:10  Valea Sidimului însa era plina de fântâni de smoala; ?i, fugind, regele Sodomei ?i regele Gomorei au cazut în ele, iar ceilal?i au fugit în mun?i.
Gen 14:11  Atunci biruitorii au luat toate averile Sodomei ?i Gomorei ?i toate bucatele lor ?i s-au dus.
Gen 14:12  Când s-au dus, au luat de asemenea ?i pe Lot, nepotul lui Avram, care traia în Sodoma, ?i toata averea lui.
Gen 14:13  Dar venind unii din cei scapa?i, au vestit pe Avram Evreul, care traia pe atunci la stejarul lui Mamvri, pe Amoreul, fratele lui E?col ?i pe fratele lui Aner, care erau uni?i cu Avram.
Gen 14:14  Auzind Avram ca Lot, rudenia sa, a fost luat în robie, a adunat oamenii sai de casa, trei sute optsprezece, ?i a urmarit pe vrajma?i pâna la Dan.
Gen 14:15  ?i navalind asupra lor noaptea, el ?i oamenii sai i-au batut ?i i-au alungat pâna la Hoba, care este în stânga Damascului.
Gen 14:16  ?i au întors toata prada luata din Sodoma, au întors ?i pe Lot, rudenia sa, averea lui, femeile ?i oamenii.
Gen 14:17  ?i când se întorcea Avram, dupa înfrângerea lui Kedarlaomer ?i a regilor uni?i cu acela, i-a ie?it înainte regele Sodomei în valea ?ave, care astazi se cheama Valea Regilor.
Gen 14:18  Iar Melhisedec, regele Salemului, i-a adus pâine ?i vin. Melhisedec acesta era preotul Dumnezeului celui Preaînalt.
Gen 14:19  ?i a binecuvântat Melhisedec pe Avram ?i a zis: "Binecuvântat sa fie Avram de Dumnezeu cel Preaînalt, Ziditorul cerului ?i al pamântului.
Gen 14:20  ?i binecuvântat sa fie Dumnezeul cel Preaînalt, Care a dat pe vrajma?ii tai în mâinile tale!" ?i Avram i-a dat lui Melhisedec zeciuiala din toate.
Gen 14:21  Iar regele Sodomei a zis catre Avram: "Da-mi oamenii, iar averile ia-le pentru tine!"
Gen 14:22  Avram însa a raspuns regelui Sodomei: "Iata, îmi ridic mâna spre Domnul Dumnezeul cel Preaînalt, Ziditorul cerului ?i al pamântului,
Gen 14:23  Ca nici o a?a sau curea de încal?aminte nu voi lua din toate câte sunt ale tale, ca sa nu zici: "Eu am îmboga?it pe Avram",
Gen 14:24  Fara numai cele ce au mâncat tinerii ?i ceea ce se cuvine a se împar?i alia?ilor mei, care au mers cu mine: Aner, E?col ?i Mamvri. Aceia sa-?i ia partea lor!"
Gen 15:1  Dupa acestea, fost-a cuvântul Domnului catre Avram, noaptea, în vis, ?i a zis: "Nu te teme, Avrame, ca Eu sunt scutul tau ?i rasplata ta va fi foarte mare!"
Gen 15:2  Iar Avram a raspuns: "Stapâne Doamne, ce ai sa-mi dai? Ca iata eu am sa mor fara copii ?i cârmuitor în casa mea este Eliezer din Damasc".
Gen 15:3  Apoi Avram a adaugat: "De vreme ce nu mi-ai dat fii, iata sluga mea va fi mo?tenitor dupa mine!"
Gen 15:4  ?i îndata s-a facut cuvântul Domnului catre el ?i a zis: "Nu te va mo?teni acela, ci cel ce va rasari din coapsele tale, acela te va mo?teni!"
Gen 15:5  Apoi l-a scos afara ?i i-a zis: "Prive?te la cer ?i numara stelele, de le po?i numara!" ?i a adaugat: "Atât de mul?i vor fi urma?ii tai!"
Gen 15:6  ?i a crezut Avram pe Domnul ?i i s-a socotit aceasta ca dreptate.
Gen 15:7  ?i i-a zis iara?i: "Eu sunt Domnul, Care te-a scos din Urul Caldeii, ca sa-?i dau pamântul acesta de mo?tenire".
Gen 15:8  ?i a zis Avram: "Stapâne Doamne pe ce voi cunoa?te ca-l voi mo?teni?"
Gen 15:9  Iar Domnul i-a zis: "Gate?te-Mi o juninca de trei ani, o capra de trei ani, un berbec de trei ani, o turturica ?i un pui de porumbel!"
Gen 15:10  ?i a luat Avram toate aceste animale, le-a taiat în doua ?i a pus buca?ile una în fa?a alteia; iar pasarile nu le-a taiat.
Gen 15:11  ?i navaleau pasarile rapitoare asupra trupurilor, iar Avram le alunga.
Gen 15:12  La asfin?itul soarelui, a cazut peste Avram somn greu ?i iata l-a cuprins întuneric ?i frica mare.
Gen 15:13  Atunci a zis Domnul catre Avram: "Sa ?tii bine ca urma?ii tai vor pribegi în pamânt strain, unde vor fi robi?i ?i apasa?i patru sute de ani;
Gen 15:14  Dar pe neamul acela, caruia ei vor fi robi, îl voi judeca Eu ?i dupa aceea ei vor ie?i sa vina aici, cu avere multa.
Gen 15:15  Iar tu vei trece la parin?ii tai în pace ?i vei fi îngropat la batrâne?i fericite.
Gen 15:16  Ei însa se vor întoarce aici, în al patrulea veac de oameni, caci nu s-a umplut înca masura nelegiuirilor Amoreilor".
Gen 15:17  Iar dupa ce a asfin?it soarele ?i s-a facut întuneric, iata un fum ca dintr-un cuptor ?i para de foc au trecut printre buca?ile acelea.
Gen 15:18  În ziua aceea a încheiat Domnul legamânt cu Avram, zicând: "Urma?ilor tai voi da pamântul acesta de la râul Egiptului pâna la râul cel mare al Eufratului;
Gen 15:19  Voi da pe Chenei, pe Chenezei, pe Chedmonei,
Gen 15:20  Pe Hetei, pe Ferezei, pe Refaimi,
Gen 15:21  Pe Amorei, pe Canaanei, pe Hevei, pe Gherghesei ?i pe Iebusei".
Gen 16:1  Sarai însa, femeia lui Avram, nu-i na?tea. Dar avea ea o slujnica egipteanca, al carei nume era Agar.
Gen 16:2  Atunci a zis Sarai catre Avram: "Iata m-a închis Domnul, ca sa nu nasc. Intra dar la slujnica mea; poate vei dobândi copii de la ea!" ?i a ascultat Avram vorba Saraii.
Gen 16:3  A luat deci Sarai, femeia lui Avram, pe Agar egipteanca, slujnica sa, la zece ani dupa venirea lui Avram în pamântul Canaan, ?i a dat-o de femeie lui Avram, barbatul sau.
Gen 16:4  ?i a intrat acesta la Agar ?i ea a zamislit; ?i vazând ca a zamislit, ea a început a dispre?ui pe stapâna sa.
Gen 16:5  Atunci a zis Sarai catre Avram: "Nedreptate mi se face de catre tine. Eu ti-am dat pe slujnica mea la sân, iar ea, vazând ca a zamislit, a început sa ma dispre?uiasca. Dumnezeu sa judece între mine ?i între tine!"
Gen 16:6  Iar Avram a zis catre Sarai: "Iata, slujnica ta e în mâinile tale, fa cu ea ce-?i place!" ?i Sarai a necajit-o ?i ea a fugit de la fa?a ei.
Gen 16:7  ?i a gasit-o îngerul Domnului la un izvor de apa în pustiu, la izvorul de lânga calea ce duce spre Sur.
Gen 16:8  ?i i-a zis îngerul Domnului: "Agar, slujnica Saraii, de unde vii ?i unde te duci?" Iar ea a raspuns: "Fug de la fa?a Saraii, stapâna mea".
Gen 16:9  ?i îngerul Domnului i-a zis iara?i: "Întoarce-te la stapâna ta ?i te supune sub mâna ei!"
Gen 16:10  Apoi i-a mai zis îngerul Domnului: "Voi înmul?i pe urma?ii tai foarte tare, încât nu se vor putea numara din pricina mul?imii.
Gen 16:11  Iata, tu ai ramas grea - îi zise îngerul Domnului - ?i vei na?te un fiu ?i-i vei pune numele Ismael, pentru ca a auzit Domnul suferin?a ta.
Gen 16:12  Acela va fi ca un asin salbatic între oameni; mâinile lui vor fi asupra tuturor ?i mâinile tuturor asupra lui, dar el va sta dârz în fa?a tuturor fra?ilor lui".
Gen 16:13  ?i a numit Agar pe Domnul, Cel ce-i graise, cu numele acesta: Ata-El-Roi (care se tâlcuie?te: Tu e?ti Dumnezeu atotvazator), caci zicea ea: "N-am vazut eu, oare, în fa?a pe Cel ce m-a vazut?"
Gen 16:14  De aceea se nume?te fântâna aceasta: Beer-Lahai-Roi (care se tâlcuie?te: Izvorul Celui viu, Care m-a vazut), ?i se afla între Cade? ?i Bared.
Gen 16:15  Dupa aceea a nascut Agar lui Avram un fiu ?i Avram a pus fiului sau, pe care i-l nascuse Agar, numele Ismael.
Gen 16:16  Avram însa era de optzeci ?i ?ase de ani când i-a nascut Agar pe Ismael.
Gen 17:1  Iar când era Avram de nouazeci ?i noua de ani, i S-a aratat Domnul ?i i-a zis: "Eu sunt Dumnezeul cel Atotputernic; fa ce-i placut înaintea Mea ?i fii fara prihana;
Gen 17:2  ?i voi încheia legamânt cu tine ?i te voi înmul?i foarte, foarte tare".
Gen 17:3  Atunci a cazut Avram cu fa?a la pamânt, iar Dumnezeu a mai grait ?i a zis:
Gen 17:4  "Eu sunt ?i iata care-i legamântul Meu cu tine: vei fi tata a mul?ime de popoare,
Gen 17:5  ?i nu te vei mai numi Avram, ci Avraam va fi numele tau, caci am sa te fac tata a mul?ime de popoare.
Gen 17:6  Am sa te înmul?esc foarte, foarte tare, ?i am sa ridic din tine popoare, ?i regi se vor ridica din tine.
Gen 17:7  Voi pune legamântul Meu între Mine ?i între tine ?i urma?ii tai, din neam în neam, sa fie legamânt ve?nic, a?a ca Eu voi fi Dumnezeul tau ?i al urma?ilor tai de dupa tine.
Gen 17:8  ?i-?i voi da ?ie ?i urma?ilor tai pamântul în carte pribege?ti acum ca strain, tot pamântul Canaanului, ca mo?tenire ve?nica, ?i va voi fi Dumnezeu".
Gen 17:9  Apoi a mai zis Dumnezeu lui Avraam: "Iar tu ?i urma?ii tai din neam în neam sa pazi?i legamântul Meu.
Gen 17:10  Iar legamântul dintre Mine ?i tine ?i urma?ii tai din neam în neam, pe care trebuie sa-l pazi?i, este acesta: to?i cei de parte barbateasca ai vo?tri sa se taie împrejur.
Gen 17:11  Sa va taia?i împrejur ?i acesta va fi semnul legamântului dintre Mine ?i voi.
Gen 17:12  În neamul vostru, tot pruncul de parte barbateasca, nascut la voi în casa sau cumparat cu bani de la alt neam, care nu-i din semin?ia voastra, sa se taie împrejur în ziua a opta.
Gen 17:13  Numaidecât sa fie taiat împrejur cel nascut în casa ta sau cel cumparat cu argintul tau ?i legamântul Meu va fi însemnat pe trupul vostru, ca legamânt ve?nic.
Gen 17:14  Iar cel de parte barbateasca netaiat împrejur, care nu se va taia împrejur, în ziua a opta, sufletul acela se va stârpi din poporul sau, caci a calcat legamântul Meu".
Gen 17:15  Dupa aceea a zis iara?i Dumnezeu catre Avraam: "Pe Sarai, femeia ta, sa nu o mai nume?ti Sarai, ci Sarra sa-i fie numele.
Gen 17:16  ?i o voi binecuvânta ?i-?i voi da din ea un fiu; o voi binecuvânta ?i va fi mama de popoare ?i regi peste popoare se vor ridica dintr-însa".
Gen 17:17  Avraam a cazut atunci cu fa?a la pamânt ?i a râs, zicând în sine: "E cu putin?a oare sa mai aiba fiu cel de o suta de ani? ?i Sarra cea de nouazeci de ani e cu putin?a oare sa mai nasca?"
Gen 17:18  Apoi a mai zis Avraam catre Domnul: "O, Doamne, macar Ismael sa traiasca înaintea Ta!"
Gen 17:19  Iar Dumnezeu a raspuns lui Avraam: "Adevarat, însa?i Sarra, femeia ta, î?i va na?te un fiu ?i-i vei pune numele Isaac ?i Eu voi încheia cu el legamântul Meu, legamânt ve?nic: sa-i fiu Dumnezeu lui ?i urma?ilor lui.
Gen 17:20  Iata, te-am ascultat ?i pentru Ismael, ?i iata îl voi binecuvânta, îl voi cre?te ?i-l voi înmul?i foarte, foarte tare; doisprezece voievozi se vor na?te din el ?i voi face din el popor mare.
Gen 17:21  Dar legamântul Meu îl voi încheia cu Isaac, pe care-l va na?te Sarra la anul pe vremea aceasta!"
Gen 17:22  Încetând apoi Dumnezeu de a mai vorbi cu Avraam, S-a înal?at de la el.
Gen 17:23  Atunci a luat Avraam pe Ismael, fiul sau, pe to?i cei nascu?i în casa sa, pe to?i cei cumpara?i cu argintul sau ?i pe to?i oamenii de parte barbateasca din casa lui Avraam ?i i-a taiat împrejur, chiar în ziua aceea, cum îi poruncise Dumnezeu.
Gen 17:24  ?i era Avraam de nouazeci ?i noua de ani, când s-a taiat împrejur.
Gen 17:25  Iar Ismael, fiul sau, era de treisprezece ani, când s-a taiat împrejur.
Gen 17:26  Avraam ?i Ismael, fiul sau, au fost taia?i împrejur în aceea?i zi.
Gen 17:27  ?i cu ei au fost taia?i împrejur to?i cei de parte barbateasca din casa lui Avraam, nascu?i în casa lui sau cumpara?i cu argint de la cei de alt neam.
Gen 18:1  Apoi Domnul S-a aratat iara?i lui Avraam la stejarul Mamvri, într-o zi pe la amiaza, când ?edea el în u?a cortului sau.
Gen 18:2  Atunci ridicându-?i ochii sai, a privit ?i iata trei Oameni stateau înaintea lui; ?i cum l-a vazut, a alergat din pragul cortului sau în întâmpinarea Lor ?i s-a închinat pâna la pamânt.
Gen 18:3  Apoi a zis: "Doamne, de am aflat har înaintea Ta, nu ocoli pe robul Tau!
Gen 18:4  Se va aduce apa sa Va spala?i picioarele ?i sa Va odihni?i sub acest copac.
Gen 18:5  ?i voi aduce pâine ?i ve?i mânca, apoi Va ve?i duce în drumul Vostru, întrucât trece?i pe la robul Vostru!" Zis-au Aceia: "Fa, precum ai zis!"
Gen 18:6  Dupa aceea a alergat Avraam în cort la Sarra ?i i-a zis: "Framânta degraba trei masuri de faina buna ?i fa azime!"
Gen 18:7  Apoi Avraam a dat fuga la cireada, a luat un vi?el tânar ?i gras ?i l-a dat slugii, care l-a gatit degraba.
Gen 18:8  ?i a luat Avraam unt, lapte ?i vi?elul cel gatit ?i le-a pus înaintea Lor ?i pe când Ei mâncau a stat ?i el alaturi de Ei sub copac.
Gen 18:9  ?i l-au întrebat Oamenii aceia: -Unde este Sarra, femeia ta?" Iar el, raspunzând, a zis: "Iata, în cort!"
Gen 18:10  Zis-a Unul: "Iata, la anul pe vremea asta am sa vin iar pe la tine ?i Sarra, femeia ta, va avea un fiu". Iar Sarra a auzit din u?a cortului, de la spatele lui.
Gen 18:11  Avraam ?i Sarra însa erau batrâni, înainta?i în vârsta, ?i Sarra nu mai era în stare sa zamisleasca.
Gen 18:12  ?i a râs Sarra în sine ?i ?i-a zis: "Sa mai am eu oare aceasta mângâiere acum, când am îmbatrânit ?i când e batrân ?i stapânul meu?"
Gen 18:13  Atunci a zis Domnul catre Avraam: "Pentru ce a râs Sarra în sine ?i a zis: "Oare cu adevarat voi na?te, batrâna cum sunt?"
Gen 18:14  Este oare ceva cu neputin?a la Dumnezeu? La anul pe vremea aceasta am sa vin pe la tine ?i Sarra va avea un fiu!"
Gen 18:15  Iar Sarra a tagaduit, zicând: "N-am râs", caci se înspaimântase. Acela însa i-a zis: "Ba, ai râs!"
Gen 18:16  Apoi S-au sculat Oamenii aceia de acolo ?i S-au îndreptat spre Sodoma ?i Gomora ?i s-a dus ?i Avraam cu Ei, ca sa-I petreaca.
Gen 18:17  Domnul însa a zis: "Tainui-voi Eu oare de Avraam, sluga Mea, ceea ce voiesc sa fac?
Gen 18:18  Din Avraam cu adevarat se va ridica un popor mare ?i tare ?i printr-însul se vor binecuvânta toate neamurile pamântului,
Gen 18:19  Ca l-am ales, ca sa înve?e pe fiii ?i casa sa dupa sine sa umble în calea Domnului ?i sa faca judecata ?i dreptate; pentru ca sa aduca Domnul asupra lui Avraam toate câte i-a fagaduit".
Gen 18:20  Zis-a deci Domnul: "Strigarea Sodomei ?i a Gomorei e mare ?i pacatul lor cumplit de greu.
Gen 18:21  Pogorî-Ma-voi deci sa vad daca faptele lor sunt cu adevarat a?a cum s-a suit pâna la Mine strigarea împotriva lor, iar de nu, sa ?tiu".
Gen 18:22  De acolo doi din Oamenii aceia, plecând, S-au îndreptat spre Sodoma, în vreme ce Avraam statea înca înaintea Domnului.
Gen 18:23  ?i apropiindu-se Avraam, a zis: "Pierde-vei, oare, pe cel drept ca ?i pe cel pacatos, încât sa se întâmple celui drept ce se întâmpla celui nelegiuit?
Gen 18:24  Poate în cetatea aceea sa fie cincizeci de drep?i: pierde-i-vei, oare, ?i nu vei cru?a tot locul acela pentru cei cincizeci de drep?i, de se vor afla în cetate?
Gen 18:25  Nu se poate ca Tu sa faci una ca asta ?i sa pierzi pe cel drept ca ?i pe cel fara de lege ?i sa se întâmple celui drept ce se întâmpla celui necredincios! Departe de Tine una ca asta! Judecatorul a tot pamântul va face, oare, nedreptate?"
Gen 18:26  Zis-a Domnul: "De se vor gasi în cetatea Sodomei cincizeci de drep?i, voi cru?a pentru ei toata cetatea ?i tot locul acela".
Gen 18:27  ?i raspunzând Avraam, a zis: "Iata, cutez sa vorbesc Stapânului meu, eu, care sunt pulbere ?i cenu?a!
Gen 18:28  Poate ca lipsesc cinci din cincizeci de drep?i; poate sa fie numai patruzeci ?i cinci; pentru lipsa a cinci pierde-vei, oare, toata cetatea?" Zis-a Domnul: "Nu o voi pierde de voi gasi acolo patruzeci ?i cinci de drep?i".
Gen 18:29  ?i a adaugat Avraam sa graiasca Domnului ?i a zis: "Dar de se vor gasi acolo numai patruzeci de drep?i?" ?i Domnul a zis: "Nu o voi pierde pentru cei patruzeci!"
Gen 18:30  ?i a zis iara?i Avraam: "Sa nu Se mânie Stapânul meu de voi mai grai: Dar de se vor gasi acolo numai treizeci de drep?i?" Zis-a Domnul: "Nu o voi pierde de voi gasi acolo treizeci".
Gen 18:31  ?i a zis Avraam: "Iata, mai cutez sa vorbesc Stapânului meu! Poate ca se vor gasi acolo numai douazeci de drep?i". Raspuns-a Domnul: "Nu o voi pierde pentru cei douazeci".
Gen 18:32  ?i a mai zis Avraam: "Sa nu se mânie Stapânul meu de voi mai grai înca o data: Dar de se vor gasi acolo numai zece drep?i?" Iar Domnul i-a zis: "Pentru cei zece nu o voi pierde".
Gen 18:33  ?i terminând Domnul de a mai grai cu Avraam; S-a dus, iar Avraam s-a întors la locul sau.
Gen 19:1  Cei doi Îngeri au ajuns la Sodoma seara, iar Lot ?edea la poarta Sodomei. ?i vazându-I, Lot s-a sculat înaintea Lor ?i s-a plecat cu fa?a pâna la pamânt
Gen 19:2  ?i a zis: "Stapânii mei, abate?i-va pe la casa slugii Voastre, ca sa ramâne?i acolo; spala?i-Va picioarele, iar diminea?a, sculându-Va, Va ve?i duce în drumul Vostru". Ei însa au zis: "Nu, ci vom ramâne în uli?a".
Gen 19:3  Iar el I-a rugat staruitor ?i S-au abatut la el ?i au intrat în casa lui. Atunci el Le-a gatit mâncare, Le-a copt azime ?i au mâncat.
Gen 19:4  Dar mai înainte de a Se culca Ei, sodomenii, locuitorii ceta?ii Sodoma, tot poporul din toate marginile; de la tânar pâna la batrân, au înconjurat casa,
Gen 19:5  ?i au chemat afara pe Lot ?i au zis catre el: "Unde sunt Oamenii, Care au intrat sa mâie la tine? Scoate-I ca sa-I cunoa?tem!"
Gen 19:6  ?i a ie?it Lot la ei dinaintea u?ii ?i, închizând u?a dupa dânsul,
Gen 19:7  A zis catre ei: "Nu, fra?ii mei, sa nu face?i nici un rau.
Gen 19:8  Am eu doua fete, care n-au cunoscut înca barbat; mai degraba vi le scot pe acelea, sa face?i cu ele ce ve?i vrea, numai Oamenilor acelora sa nu le face?i nimic, de vreme ce au intrat Ei sub acoperi?ul casei mele!"
Gen 19:9  Iar ei au zis catre el: "Pleaca de aici! E?ti un venetic ?i acum faci pe judecatorul? Mai rau decât Acelora i?i vom face!" ?i repezindu-se spre Lot, se apropiara sa sparga u?a.
Gen 19:10  Atunci Oamenii aceia, care gazduiau în casa lui Lot, întinzându-?i mâinile, au tras pe Lot în casa la Ei ?i au încuiat u?a;
Gen 19:11  Iar pe oamenii, care erau la u?a casei, i-au lovit cu orbire de la mic pâna la mare, încât în zadar se chinuiau sa gaseasca u?a.
Gen 19:12  Apoi au zis cei doi Oameni catre Lot: "Ai tu pe cineva din ai tai aici? De ai fii, sau fiice, sau gineri, sau pe oricine altul în cetate, scoate-i din locul acesta,
Gen 19:13  Ca Noi avem sa pierdem locul acesta, pentru ca strigarea împotriva lor s-a suit înaintea Domnului ?i Domnul Ne-a trimis sa-l pierdem".
Gen 19:14  Atunci a ie?it Lot ?i a grait cu ginerii sai, care luasera pe fetele lui, ?i le-a zis: "Scula?i-va ?i ie?i?i din locul acesta, ca va sa piarda Domnul cetatea". Ginerilor însa li s-a parut ca el glume?te.
Gen 19:15  Iar în revarsatul zorilor grabeau îngerii pe Lot, zicând: "Scoala, ia-?i femeia ?i pe cele doua fete ale tale, pe care le ai, ?i ie?i, ca sa nu pieri ?i tu pentru nedrepta?ile ceta?ii!"
Gen 19:16  Dar fiindca el zabovea, îngerii, din mila Domnului catre el, l-au apucat de mâna pe el ?i pe femeia lui ?i pe cele doua fete ale lui
Gen 19:17  ?i, sco?ându-l afara, unul din Ei a zis: "Mântuie?te-?i sufletul tau! Sa nu te ui?i înapoi, nici sa te opre?ti în câmp, ci fugi în munte, ca sa nu pieri cu ei!
Gen 19:18  Iar Lot a zis catre Dân?ii: "Nu, Stapâne!
Gen 19:19  Iata sluga Ta a aflat bunavoin?a înaintea Ta ?i Tu ai facut mila mare cu mine, mântuindu-mi via?a; dar nu voi putea sa fug pâna în munte, ca sa nu ma ajunga primejdia ?i sa nu mor.
Gen 19:20  Iata cetatea aceasta este mai aproape; sa fug acolo ?i sa ma izbavesc. Ea e mica ?i-mi voi scapa acolo via?a prin Tine!"
Gen 19:21  ?i i-a zis îngerul: "Iata, î?i cinstesc fa?a ?i-?i împlinesc acest cuvânt, sa nu pierd cetatea despre care graie?ti.
Gen 19:22  Grabe?te dar ?i fugi acolo; ca nu pot sa fac nimic pâna nu vei ajunge tu acolo!" De aceea s-a ?i numit cetatea aceea ?oar.
Gen 19:23  Când s-a ridicat soarele deasupra pamântului, a intrat Lot în ?oar.
Gen 19:24  Atunci Domnul a slobozit peste Sodoma ?i Gomora ploaie de pucioasa ?i foc din cer de la Domnul
Gen 19:25  ?i a stricat ceta?ile acestea, toate împrejurimile lor, pe to?i locuitorii ceta?ilor ?i toate plantele ?inutului aceluia.
Gen 19:26  Femeia lui Lot însa s-a uitat înapoi ?i s-a prefacut în stâlp de sare.
Gen 19:27  Iar Avraam s-a sculat dis-de-diminea?a ?i s-a dus la locul unde statuse înaintea Domnului
Gen 19:28  ?i, cautând spre Sodoma ?i Gomora ?i spre toate împrejurimile lor, a vazut ridicându-se de la pamânt fumegare, ca fumul dintr-un cuptor.
Gen 19:29  Dar, când a stricat Dumnezeu toate ceta?ile din par?ile acelea, ?i-a adus aminte Dumnezeu de Avraam ?i a scos pe Lot afara din prapadul cu care a stricat Dumnezeu ceta?ile, unde traia Lot.
Gen 19:30  Apoi a ie?it Lot din ?oar ?i s-a a?ezat în munte, împreuna cu cele doua fete ale sale, caci se temea sa locuiasca în ?oar, ?i a locuit într-o pe?tera, împreuna cu cele doua fete ale sale.
Gen 19:31  Atunci a zis fata cea mai mare catre cea mai mica: "Tatal nostru e batrân ?i nu-i nimeni în ?inutul acesta, care sa intre la noi, cum e obiceiul pamântului.
Gen 19:32  Haidem dar sa îmbatam pe tatal nostru cu vin ?i sa ne culcam cu el ?i sa ne ridicam urma?i dintr-însul!"
Gen 19:33  ?i au îmbatat pe tatal lor cu vin în noaptea aceea; ?i în noaptea aceea, intrând fata cea mai în vârsta, a dormit cu tatal ei ?i acesta n-a sim?it când s-a culcat ?i când s-a sculat ea.
Gen 19:34  Iar a doua zi a zis cea mai în vârsta catre cea mai tânara: "Iata, eu am dormit asta-noapte cu tatal meu; sa-l îmbatam cu vin ?i în noaptea aceasta ?i sa intri ?i tu sa dormi cu el ca sa ne ridicam urma?i din tatal nostru!"
Gen 19:35  ?i l-au îmbatat cu vin ?i în noaptea aceasta ?i a intrat ?i cea mai mica ?i a dormit cu el; ?i el n-a ?tiut când s-a culcat ea, nici când s-a sculat ea.
Gen 19:36  ?i au ramas amândoua fetele lui Lot grele de la tatal lor.
Gen 19:37  ?i a nascut cea mai mare un fiu, ?i i-a pus numele Moab, zicând: "Este din tatal meu". Acesta e tatal Moabi?ilor, care sunt ?i astazi.
Gen 19:38  ?i a nascut ?i cea mai mica un fiu ?i i-a pus numele Ben-Ammi, zicând: "Acesta-i fiul neamului meu". Acesta e tatal Amoni?ilor, care sunt ?i astazi.
Gen 20:1  Apoi a plecat Avraam de acolo spre miazazi ?i s-a a?ezat între Cade? ?i Sur ?i a trait o vreme în Gherara.
Gen 20:2  ?i a zis Avraam despre Sarra, femeia sa: "Mi-e sora", caci se temea sa spuna: "E femeia mea", ca nu cumva sa-l ucida locuitorii ceta?ii aceleia din pricina ei. Iar Abimelec, regele Gherarei, a trimis ?i a luat pe Sarra.
Gen 20:3  Dar noaptea în vis a venit Dumnezeu la Abimelec ?i i-a zis: "Iata, tu ai sa mori pentru femeia, pe care ai luat-o, caci ea are barbat".
Gen 20:4  Abimelec însa nu se atinsese de ea ?i a zis: "Doamne, ucide-vei oare chiar ?i un om drept?
Gen 20:5  Oare n-a zis el singur: "Mi-e sora?" Ba ?i ea mia zis: "Mi-e frate!" Eu cu inima nevinovata ?i cu mâini curate am facut aceasta".
Gen 20:6  Iar Dumnezeu i-a zis în vis: "?i Eu ?tiu ca cu inima nevinovata ai facut aceasta ?i te-am ferit de a pacatui împotriva Mea; de aceea nu ?i-am îngaduit sa te atingi de ea.
Gen 20:7  Acum însa da înapoi femeia omului aceluia, ca e prooroc, ?i se va ruga pentru tine ?i vei fi viu; iar de nu o vei da înapoi, sa ?tii bine ca ai sa mori ?i tu ?i to?i ai tai!"
Gen 20:8  ?i, sculându-se Abimelec, a doua zi de diminea?a, a chemat pe to?i slujitorii sai ?i le-a povestit toate acestea în auz, ?i s-au spaimântat to?i oamenii aceia foarte tare.
Gen 20:9  Apoi a chemat Abimelec pe Avraam ?i i-a zis: "Ce mi-ai facut tu? Cu ce ?i-am gre?it eu, de ai adus asupra mea ?i asupra ?arii mele a?a pacat mare? Tu mi-ai facut un lucru, care nu se cuvine a-l face!"
Gen 20:10  ?i a mai zis Abimelec catre Avraam: "Ce ai socotit tu, de ai facut una ca asta?"
Gen 20:11  Raspuns-a Avraam: "Am socotit ca prin ?inutul acesta lipse?te frica de Dumnezeu ?i voi fi omorât din pricina femeii mele.
Gen 20:12  Cu adevarat ea mi-e sora dupa tata, dar nu ?tiu dupa mama, iar acum mi-e so?ie.
Gen 20:13  Iar când m-a scos Dumnezeu din casa tatalui meu, ca sa pribegesc, am zis catre ea: "Sa-mi faci acest bine, ?i, în orice loc vom merge, sa zici de mine: "Mi-e frate!"
Gen 20:14  Atunci a luat Abimelec o mie de sicli de argint, vite mari ?i mici, robi ?i roabe ?i a dat lui Avraam; ?i i-a dat înapoi ?i pe Sarra, femeia sa.
Gen 20:15  ?i a zis Abimelec catre Avraam: "Iata, ?inutul meu î?i este la îndemâna: locuie?te unde î?i place!"
Gen 20:16  Iar catre Sarra a zis: "Iata, dau fratelui tau o mie de sicli de argint, care vor fi ca un val pe ochi pentru cei ce sunt împrejurul tau ?i pentru lumea toata. ?i iata ca acum e?ti socotita dreapta!"
Gen 20:17  ?i s-a rugat Avraam lui Dumnezeu ?i Dumnezeu a vindecat pe Abimelec, pe femeia lui ?i pe roabele lui, ?i acestea au început a na?te.
Gen 20:18  Caci Domnul lovise cu stârpiciune toata casa lui Abimelec, pentru Sarra, femeia lui Avraam.
Gen 21:1  Apoi a cautat Domnul spre Sarra, cum îi spusese, ?i i-a facut Domnul Sarrei, cum îi fagaduise.
Gen 21:2  ?i a zamislit Sarra ?i a nascut lui Avraam un fiu la batrâne?e, la vremea aratata de Dumnezeu.
Gen 21:3  ?i a pus Avraam fiului sau, pe care i-l nascuse Sarra, numele Isaac.
Gen 21:4  ?i Avraam a taiat împrejur pe Isaac, fiul sau, în ziua a opta, cum îi poruncise Dumnezeu.
Gen 21:5  Avraam însa era de o suta de ani, când i s-a nascut Isaac, fiul sau,
Gen 21:6  Iar Sarra a zis: "Râs mi-a pricinuit mie Dumnezeu; ca oricine va auzi aceasta, va râde!"
Gen 21:7  ?i apoi a adaugat: "Cine ar fi putut spune lui Avraam ca Sarra va hrani prunci la sânul sau? ?i totu?i i-am nascut fiu la batrâne?ile  sale!"
Gen 21:8  ?i crescând copilul, a fost în?arcat. Iar Avraam a facut ospa? mare în ziua în care a fost în?arcat Isaac, fiul sau.
Gen 21:9  Vazând însa Sarra ca fiul egiptencii Agar, pe care aceasta îl nascuse lui Avraam, râde de Isaac, fiul ei,
Gen 21:10  A zis catre Avraam: "Izgone?te pe roaba aceasta ?i pe fiul ei, caci fiul roabei acesteia nu va fi mo?tenitor cu fiul meu, Isaac!"
Gen 21:11  ?i s-au parut cuvintele acestea lui Avraam foarte grele pentru fiul sau Ismael.
Gen 21:12  Dumnezeu însa a zis catre Avraam: "Sa nu ?i se para grele cuvintele cele pentru prunc ?i pentru roaba; toate câte-?i va zice Sarra, asculta glasul ei; pentru ca numai cei din Isaac se vor chema urma?ii tai.
Gen 21:13  Dar ?i pe fiul roabei acesteia îl voi face neam mare, pentru ca ?i el este din samân?a ta".
Gen 21:14  Atunci s-a sculat Avraam dis-de-diminea?a; a luat pâine ?i un burduf cu apa ?i le-a dat Agarei; apoi, punându-i pe umeri copilul, a slobozit-o; ?i, plecând ea, a ratacit prin pustiul Beer-?eba.
Gen 21:15  Când însa s-a sfâr?it apa din burduf, a lepadat ea copilul sub un maracine.
Gen 21:16  ?i ducându-se, a ?ezut în preajma lui, ca la o bataie de arc, caci î?i zicea: "Nu voiesc sa vad moartea copilului meu!" ?i, ?ezând ea acolo de o parte, ?i-a ridicat glasul ?i a plâns.
Gen 21:17  ?i a auzit Dumnezeu glasul copilului din locul unde era ?i îngerul lui Dumnezeu a strigat din cer catre Agar ?i a zis: "Ce e, Agar? Nu te teme, ca a auzit Dumnezeu glasul copilului din locul unde este!
Gen 21:18  Scoala, ridica copilul ?i-l ?ine de mâna, caci am sa fac din el un popor mare!"
Gen 21:19  Atunci i-a deschis Dumnezeu ochii ?i a vazut o fântâna cu apa ?i, mergând, ?i-a umplut burduful cu apa ?i a dat copilului sa bea.
Gen 21:20  ?i era Dumnezeu cu copilul ?i a crescut acesta, a locuit în pustiu, ?i s-a facut vânator.
Gen 21:21  A locuit deci Ismael în pustiul Faran ?i mama sa i-a luat femeie din ?ara Egiptului.
Gen 21:22  În vremea aceea, Abimelec ?i Ahuzat, care luase pe nora lui, ?i Ficol, capetenia o?tirii lui, au zis catre Avraam: "Dumnezeu e cu tine în toate câte faci.
Gen 21:23  Jura-mi, deci, aici pe Dumnezeu ca nu-mi vei face strâmbatate nici mie, nici fiului meu, nici neamului meu; ci, cum ?i-am facut eu bine ?ie, a?a sa-mi faci ?i tu mie ?i ?arii în care e?ti oaspete!"
Gen 21:24  Raspuns-a Avraam: "Jur!"
Gen 21:25  Dar a mustrat Avraam pe Abimelec pentru fântânile de apa, pe care i le rapisera slugile lui Abimelec.
Gen 21:26  Iar Abimelec i-a zis: "Nu ?tiu cine ?i-a facut lucrul acesta; nici tu nu mi-ai spus nimic, nici eu n-am auzit decât astazi".
Gen 21:27  ?i a luat Avraam oi ?i vite ?i a dat lui Abimelec ?i au încheiat amândoi legamânt.
Gen 21:28  Apoi Avraam a pus de o parte ?apte mielu?ele.
Gen 21:29  Iar Abimelec a zis catre Avraam: "Ce sunt aceste ?apte mielu?ele, pe care le-ai osebit?"
Gen 21:30  Raspuns-a Avraam: "Aceste ?apte mielu?ele sa le iei de la mine, ca sa-mi fie marturie, ca eu am sapat fântâna aceasta!"
Gen 21:31  De aceea s-a ?i numit locul acela Beer-?eba, pentru ca acolo au jurat ei amândoi.
Gen 21:32  ?i dupa ce au facut ei legamânt la Beer-?eba, s-a sculat Abimelec ?i Ahuzat, care luase pe nora lui, ?i Ficol, capetenia o?tirii lui, ?i s-au întors în ?ara Filistenilor.
Gen 21:33  Iar Avraam a sadit o dumbrava la Beer-?eba ?i a chemat acolo numele Domnului Dumnezeului celui ve?nic.
Gen 21:34  ?i a mai trait Avraam în ?ara Filistenilor zile multe, ca strain.
Gen 22:1  Dupa acestea, Dumnezeu a încercat pe Avraam ?i i-a zis: "Avraame, Avraame!" Iar el a raspuns: "Iata-ma!"
Gen 22:2  ?i Dumnezeu i-a zis: "Ia pe fiul tau, pe Isaac, pe singurul tau fiu, pe care-l iube?ti, ?i du-te în pamântul Moria ?i adu-l acolo ardere de tot pe un munte, pe care ?i-l voi arata Eu!"
Gen 22:3  Sculându-se deci Avraam dis-de-diminea?a, a pus samarul pe asinul sau ?i a luat cu sine doua slugi ?i pe Isaac, fiul sau; ?i taind lemne pentru jertfa, s-a ridicat ?i a plecat la locul despre care-i graise Dumnezeu.
Gen 22:4  Iar a treia zi, ridicându-?i Avraam ochii, a vazut în departare locul acela.
Gen 22:5  Atunci a zis Avraam slugilor sale: "Ramâne?i aici cu asinul, iar eu ?i copilul ne ducem pâna acolo ?i, închinându-ne, ne vom întoarce la voi".
Gen 22:6  Luând deci Avraam lemnele cele pentru jertfa, le-a pus pe umerii lui Isaac, fiul sau; iar el a luat în mâini focul ?i cu?itul ?i s-au dus amândoi împreuna.
Gen 22:7  Atunci a grait Isaac lui Avraam, tatal sau, ?i a zis: "Tata!" Iar acesta a raspuns: "Ce este, fiul meu?" Zis-a Isaac: "Iata, foc ?i lemne avem; dar unde este oaia pentru jertfa?"
Gen 22:8  Avraam însa a raspuns: "Fiul meu, va îngriji Dumnezeu de oaia jertfei Sale!" ?i s-au dus mai departe amândoi împreuna.
Gen 22:9  Iar daca au ajuns la locul, de care-i graise Dumnezeu, a ridicat Avraam acolo jertfelnic, a a?ezat lemnele pe el ?i, legând pe Isaac, fiul sau, l-a pus pe jertfelnic, deasupra lemnelor.
Gen 22:10  Apoi ?i-a întins Avraam mâna ?i a luat cu?itul, ca sa junghie pe fiul sau.
Gen 22:11  Atunci îngerul Domnului a strigat catre el din cer ?i a zis: "Avraame, Avraame!" Raspuns-a acesta: "Iata-ma!"
Gen 22:12  Iar îngerul a zis: "Sa nu-?i ridici mâna asupra copilului, nici sa-i faci vreun rau, caci acum cunosc ca te temi de Dumnezeu ?i pentru mine n-ai cru?at nici pe singurul fiu al tau".
Gen 22:13  ?i ridicându-?i Avraam ochii, a privit, ?i iata la spate un berbec încurcat cu coarnele într-un tufi?. ?i ducându-se, Avraam a luat berbecul ?i l-a adus jertfa în locul lui Isaac, fiul sau.
Gen 22:14  Avraam a numit locul acela Iahve-ire, adica, Dumnezeu poarta de grija ?i de aceea se zice astazi: "În munte Domnul Se arata".
Gen 22:15  ?i a strigat a doua oara îngerul Domnului din cer catre Avraam ?i a zis:
Gen 22:16  "Juratu-M-am pe Mine însumi, zice Domnul, ca de vreme ce ai facut aceasta ?i n-ai cru?at nici pe singurul tau fiu, pentru Mine,
Gen 22:17  De aceea te voi binecuvânta cu binecuvântarea Mea ?i voi înmul?i foarte neamul tau, ca sa fie ca stelele cerului ?i ca nisipul de pe ?armul marii ?i va stapâni neamul tau ceta?ile du?manilor sai;
Gen 22:18  ?i se vor binecuvânta prin neamul tau toate popoarele pamântului, pentru ca ai ascultat glasul Meu".
Gen 22:19  Întorcându-se apoi Avraam la slugile sale, s-au sculat împreuna ?i s-au dus la Beer-?eba ?i a locuit Avraam acolo în Beer-?eba.
Gen 22:20  Iar dupa ce s-au petrecut acestea, i s-a vestit lui Avraam, spunându-i-se: "Iata Milca a nascut ?i ea fii lui Nahor, fratele tau:
Gen 22:21  Pe U?, întâiul sau nascut, pe Buz, fratele acestuia ?i pe Chemuel, tatal lui Aram;
Gen 22:22  Pe Chesed, pe Hazo, pe Pilda?, pe Idlaf ?i pe Batuel.
Gen 22:23  Iar lui Batuel i s-a nascut Rebeca". Pe ace?ti opt fii i-a nascut Milca lui Nahor, fratele lui Avraam.
Gen 22:24  Iar o ?iitoare a lui, anume Reuma, i-a nascut ?i ea pe Tebah, pe Gaham, pe Taha? ?i pe Maaca.
Gen 23:1  Sarra a trait o suta douazeci ?i ?apte de ani. Ace?tia sunt anii vie?ii Sarrei.
Gen 23:2  Sarra a murit la Chiriat-Arba care e în vale, adica în Hebronul de astazi, în ?ara Canaanului. ?i a venit Avraam sa plânga ?i sa jeleasca pe Sarra.
Gen 23:3  Apoi s-a dus Avraam de la moarta sa, a grait cu fiii lui Het ?i a zis:
Gen 23:4  "Eu sunt intre voi strain ?i pribeag; da?i-mi dar în stapânire un loc de mormânt la voi, ca sa îngrop pe moarta mea".
Gen 23:5  Iar fiii lui Het au raspuns ?i au zis catre Avraam:
Gen 23:6  "Nu, domnul meu, ci asculta-ne: Tu aici la noi e?ti u: voievod al lui Dumnezeu. Deci, îngroapa-?i moarta în cel mai bun dintre locurile noastre de îngropare, ca nici unul dintre noi nu te va opri sa-?i îngropi moarta acolo".
Gen 23:7  Atunci s-a sculat Avraam ?i s-a închinat poporului jarii aceleia, adica fiilor lui Het.
Gen 23:8  ?i a grait catre dân?ii Avraam ?i a zis: "Daca voi?i din suflet sa-mi îngrop pe moarta mea de la ochii mei, atunci asculta?i-ma ?i ruga?i pentru mine pe Efron, fiul lui ?ohar,
Gen 23:9  Ca sa-mi dea pe?tera Macpela pe care o are în capatul ?arinei lui, dar sa mi-o dea pe bani gata, ca sa o am aici la voi în stapânire de veci pentru îngropare.
Gen 23:10  Efron însa ?edea atunci în mijlocul fiilor lui Het. ?i a raspuns Efron Heteeanul lui Avraam, în auzul fiilor lui Het ?i al tuturor celor ce venisera la por?ile ceta?ii lui, ?i a zis:
Gen 23:11  "Nu, domnul meu, asculta-ma pe mine: Eu î?i dau ?arina ?i pe?tera cea dintr-însa ?i ?i-o dau în fala fiilor poporului; ?i-o dau însa în dar. Îngroapa-?i pe moarta ta".
Gen 23:12  Avraam însa s-a închinat înaintea poporului ?arii
Gen 23:13  ?i a grait catre Efron în auzul a tot poporul ?inutului aceluia ?i a zis: "De binevoie?ti, asculta-ma ?i ia de la mine pre?ul ?arinei ?i voi îngropa acolo pe moarta mea".
Gen 23:14  Raspuns-a Efron lui Avraam ?i i-a zis:
Gen 23:15  "Asculta, domnul meu, ?arina pre?uie?te patru sute sicli de argint. Ce este aceasta pentru mine ?i pentru tine? Îngroapa-?i dar pe moarta ta!"
Gen 23:16  Atunci, ascultând pe Efron, Avraam a cântarit lui Efron atâta argint, cât a spus el în auzul fiilor lui Het: patru sute sicli de argint, dupa pre?ul negustoresc.
Gen 23:17  ?i a?a ?arina lui Efron, care e lânga Macpela, în fa?a stejarului Mamvri, ?arina ?i pe?tera din ea ?i to?i pomii din ?arina ?i tot ce era în hotarele ei de jur împrejur
Gen 23:18  S-au dat lui Avraam mo?ie de veci, înaintea fiilor lui Het  ?i a tuturor celor ce se strânsesera la poarta ceta?ii lui.
Gen 23:19  Dupa aceasta Avraam a îngropat pe Sarra, femeia sa, în pe?tera din ?arina Macpela, care e în fa?a lui Mamvri sau a Hebronului, în Canaan.
Gen 23:20  Astfel a trecut de la fiii lui Het la Avraam ?arina ?i pe?tera cea din ea, ca loc de îngropare.
Gen 24:1  Avraam era acum batrân ?i vechi de zile ?i Domnul binecuvântase pe Avraam cu de toate.
Gen 24:2  Atunci a zis Avraam catre sluga cea mai batrâna din casa sa, care cârmuia toate câte avea: "Pune mâna ta sub coapsa mea
Gen 24:3  ?i jura-mi pe Domnul Dumnezeul cerului ?i pe Dumnezeul pamântului ca fiului meu Isaac nu-i vei lua femeie din fetele Canaaneilor, în mijlocul carora locuiesc eu,
Gen 24:4  Ci vei merge în ?ara mea, unde m-am nascut eu, la rudele mele, ?i vei lua de acolo femeie lui Isaac, fiul meu".
Gen 24:5  Iar sluga a zis catre el: "Dar poate nu va vrea femeia sa vina cu mine în pamântul acesta; întoarce-voi, oare, pe fiul tau în pamântul de unde ai ie?it?"
Gen 24:6  Avraam însa a zis catre el: "Ia seama sa nu întorci pe fiul meu acolo!
Gen 24:7  Domnul Dumnezeul cerului ?i Dumnezeul pamântului, Cel ce m-a luat din casa tatalui meu ?i din pamântul în care m-am nascut, Care mi-a grait ?i Care mi S-a jurat, zicând: ?ie-?i voi da pamântul acesta ?i urma?ilor tai, Acela va trimite pe îngerul Sau înaintea ta ?i vei lua femeie feciorului meu de acolo.
Gen 24:8  Iar de nu va voi femeia aceea sa vina cu tine în pamântul acesta, vei fi slobod de juramântul meu, dar pe fiul meu sa nu-l întorci acolo!"
Gen 24:9  ?i punându-?i sluga mâna sub coapsa lui Avraam, stapânul sau, i s-a jurat pentru toate acestea.
Gen 24:10  Apoi a luat sluga cu sine zece camile din camilele stapânului sau ?i tot felul de lucruri scumpe de ale stapânului sau ?i, sculându-se, s-a dus în Mesopotamia, în cetatea lui Nahor.
Gen 24:11  ?i, într-o zi, spre seara, când ies femeile sa scoata apa, a poposit cu camilele la o fântâna, afara din cetate.
Gen 24:12  ?i a zis: "Doamne Dumnezeul stapânului meu Avraam, scoate-mi-o în cale astazi ?i fa mila cu stapânul meu Avraam!
Gen 24:13  Iata, eu stau la fântâna aceasta ?i fetele locuitorilor ceta?ii au sa iasa sa scoata apa.
Gen 24:14  Deci fata careia îi voi zice: Pleaca urciorul tau sa beau ?i care-mi va raspunde: "Bea! Ba ?i  camilele toate le voi adapa pâna se vor satura", aceea sa fie pe care Tu ai rânduit-o robului Tau Isaac ?i prin aceasta voi cunoa?te ca faci mila cu stapânul meu Avraam.
Gen 24:15  Dar nu sfâr?ise el înca a cugeta acestea în mintea sa, când iata ca ie?i cu urciorul pe umar Rebeca, fecioara care se nascuse lui Batuel, fiul Milcai, femeia lui Nahor, fratele lui Avraam.
Gen 24:16  Aceasta era foarte frumoasa la chip, fecioara, pe care nu o cunoscuse înca un barbat. ?i venind ea la fântâna, ?i-a umplut urciorul ?i a pornit înapoi.
Gen 24:17  Atunci sluga lui Avraam a alergat înaintea ei ?i i-a zis: "Da-mi sa beau pu?ina apa din urciorul tau!"
Gen 24:18  Iar ea a zis: "Bea, domnul meu!" ?i îndata ?i-a lasat urciorul pe bra?e ?i i-a dat sa bea apa pâna a încetat de a mai bea.
Gen 24:19  Apoi a zis: "?i camilelor tale am sa le scot apa pâna vor bea toate".
Gen 24:20  ?i îndata ?i-a de?ertat urciorul în adapatoare ?i a alergat iar la fântâna sa scoata apa ?i a adapat toate camilele.
Gen 24:21  Iar omul acela se uita la ea cu mirare ?i tacea, dorind sa ?tie de i-a binecuvântat Domnul calatoria sau nu.
Gen 24:22  ?i daca au încetat toate camilele de a mai bea, a luat omul acela ?i i-a dat un inel de aur, în greutate de o jumatate siclu, ?i doua bra?ari la mâinile ei, în greutate de zece sicli de aur.
Gen 24:23  Apoi a întrebat-o ?i a zis: "A cui fata e?ti tu? Spune-mi, te rog, daca se afla în casa tatalui tau loc, ca sa ramânem?"
Gen 24:24  Iar ea i-a zis: "Sunt fata lui Batuel al Milcai, pe care ea l-a nascut lui Nahor".
Gen 24:25  Apoi i-a mai zis: "Avem ?i paie ?i fân mult ?i la noi este ?i loc, ca sa ramâne?i".
Gen 24:26  Atunci s-a plecat omul acela ?i s-a închinat Domnului ?i a zis:
Gen 24:27  "Binecuvântat sa fie Domnul Dumnezeul stapânului meu Avraam, Care n-a parasit pe stapânul meu cu mila ?i bunavoin?a Sa, de vreme ce m-a adus Domnul drept la casa fratelui stapânului meu".
Gen 24:28  Iar fata a alergat acasa la mama sa ?i a povestit toate acestea.
Gen 24:29  Rebeca însa avea un frate, anume Laban. ?i a alergat Laban afara, la fântâna, la omul acela,
Gen 24:30  Caci el vazuse inelul de aur ?i bra?arile la mâinile surorii sale Rebeca, ?i auzise vorbele Rebecai, sora sa, care spusese: "A?a ?i a?a mi-a vorbit omul acela!" ?i ajungând la el, l-a gasit stând cu camilele la fântâna.
Gen 24:31  ?i i-a zis: "Intra, binecuvântatul Domnului! Pentru ce stai afara? Eu ?i-am gatit casa ?i sala? pentru camilele tale!"
Gen 24:32  ?i a intrat omul acela în casa. Iar Laban a luat povara de pe camile ?i le-a dat paie ?i fân, iar lui ?i oamenilor, care erau cu el, le-a dat apa, ca sa-?i spele picioarele.
Gen 24:33  Apoi le-a adus de mâncare. Eliezer însa a zis: "Nu voi mânca pâna nu voi spune la ce am venit". Zis-a Laban: "Spune!"
Gen 24:34  Atunci Eliezer a zis: "Eu sunt sluga lui Avraam.
Gen 24:35  Domnul a binecuvântat foarte pe stapânul meu ?i l-a marit ?i i-a dat oi ?i boi, argint ?i aur, robi ?i roabe, camile ?i asini.
Gen 24:36  Iar Sarra, femeia stapânului meu, fiind acum batrâna, a nascut stapânului meu un fiu, caruia el i-a dat toate câte are.
Gen 24:37  ?i m-a jurat stapânul meu, zicând: "Sa nu iei femeie feciorului meu din fetele Canaaneilor, în pamântul carora traiesc,
Gen 24:38  Ci sa mergi la casa tatalui meu, la rudele mele ?i sa iei de acolo femeie pentru feciorul meu!"
Gen 24:39  Iar eu am zis catre stapânul meu: "Dar de nu va vrea femeia sa vina cu mine?"
Gen 24:40  El însa mi-a raspuns: "Domnul Dumnezeu, înaintea Caruia umblu, va trimite cu tine pe îngerul Sau, va binecuvânta calea ta ?i vei lua femeie pentru feciorul meu din rudele mele ?i din casa tatalui meu.
Gen 24:41  Atunci vei fi slobod de juramântul meu, când te vei duce la rudele mele ?i de nu ?i-o vor da, vei fi dezlegat de juramântul meu.
Gen 24:42  Deci, ajungând eu astazi la fântâna, am zis: "Doamne, Dumnezeul stapânului meu Avraam, de este sa ma faci sa izbutesc în calea ce fac,
Gen 24:43  Iata, eu stau la fântâna; ?i fata careia eu îi voi zice când va veni sa scoata apa: "Da-mi sa beau pu?ina apa din urciorul tau!"
Gen 24:44  Iar ea îmi va zice: "Bea ?i tu, ?i camilele tale le voi adapa", aceea sa fie femeia, pe care Domnul a rânduit-o pentru Isaac, robul Sau ?i fiul stapânului meu, ?i prin aceasta voi cunoa?te ca Te milostive?ti spre stapânul meu Avraam.
Gen 24:45  Dar nu ispravisem eu înca a grai acestea în mintea mea, când iata a ie?it Rebeca, cu urciorul pe umar; se pogorî la fântâna ?i scoase apa, ?i eu i-am zis: "Da-mi sa beau!"
Gen 24:46  ?i ea ?i-a lasat îndata urciorul de pe umar, zicând: "Bea tu ?i camilele tale le voi adapa". ?i am baut ?i mi-a adapat ?i camilele.
Gen 24:47  Apoi am întrebat-o ?i am zis: "A cui fata e?ti tu? Spune-mi te rog!" ?i ea a zis: "Sunt fata lui Batuel, fiul lui Nahor, pe care i l-a nascut Milca". Atunci i-am dat un inel ?i bra?ari la mâini.
Gen 24:48  Dupa aceea m-am plecat ?i m-am închinat Domnului ?i am binecuvântat pe Domnul Dumnezeul stapânului meu Avraam, Care m-a pova?uit de-a dreptul, ca sa iau pe fata fratelui stapânului meu pentru fiul lui.
Gen 24:49  Acum deci spune?i-mi de vre?i sa arata?i mila ?i bunavoin?a stapânului meu; iar de nu, sa caut alta în dreapta ?i în stânga".
Gen 24:50  ?i raspunzând Laban ?i Batuel au zis: "De la Domnul vine lucrul acesta ?i noi nu-?i putem spune nimic nici de bine, nici de rau.
Gen 24:51  Iata, Rebeca este înaintea ta, ia-o ?i du-te ?i sa fie so?ia fiului stapânului tau, cum a grait Domnul!"
Gen 24:52  ?i auzind cuvintele lor, sluga lui Avraam s-a închinat Domnului pâna la pamânt.
Gen 24:53  Apoi a scos sluga lucruri de argint ?i lucruri de aur ?i haine ?i le-a dat Rebecai. Dat-a de asemenea daruri ?i fratelui ?i mamei ei.
Gen 24:54  Dupa aceea au mâncat ?i au baut, el ?i oamenii cei ce erau cu dânsul ?i au ramas acolo. Iar daca s-a sculat diminea?a, a zis: "Lasa?i-ma sa ma duc la stapânul meu!"
Gen 24:55  Iar fratele ?i mama Rebecai au zis: "Sa mai ramâna fata cu noi macar vreo zece zile ?i apoi te vei duce!"
Gen 24:56  El însa le-a zis: "Nu ma zabovi?i! Caci Domnul m-a facut sa izbutesc în calea mea; lasa?i-ma sa ma duc la stapânul meu!"
Gen 24:57  Raspuns-au ei: "Sa chemam copila ?i s-o întrebam ce gânduri are ea".
Gen 24:58  ?i au chemat pe Rebeca ?i i-au zis: "Vrei sa te duci oare cu omul acesta?" ?i ea a zis: "Ma duc!"
Gen 24:59  Atunci a lasat Laban sa plece Rebeca, sora sa, ?i doica ei, ?i sluga lui Avraam ?i cei ce erau cu el.
Gen 24:60  ?i au binecuvântat pe Rebeca ?i i-au zis: "Sora noastra, sa se nasca din tine mii ?i zeci de mii ?i sa stapâneasca urma?ii tai por?ile vrajma?ilor lor!"
Gen 24:61  Atunci, sculându-se Rebeca ?i slujnicele ei ?i suindu-se pe camile, s-au dus cu omul acela, ?i sluga lui Avraam, luând pe Rebeca, a plecat.
Gen 24:62  Isaac însa venise din Beer-Lahai-Roi, caci el locuia în par?ile de miazazi.
Gen 24:63  Iar spre seara a ie?it Isaac la câmp sa se plimbe ?i, ridicându-?i ochii, a vazut camilele venind.
Gen 24:64  Rebeca însa, cautând, a vazut pe Isaac ?i s-a dat jos de pe camila
Gen 24:65  ?i a zis catre sluga: "Cine este omul acela care vine pe câmp în întâmpinarea noastra?" Iar sluga i-a zis: "Acesta-i stapânul meu!" Atunci ea ?i-a luat valul ?i s-a acoperit.
Gen 24:66  ?i sluga povesti lui Isaac toate câte facuse.
Gen 24:67  ?i a dus-o Isaac în cortul mamei sale Sarra ?i a luat pe Rebeca ?i aceasta s-a facut femeia lui ?i a iubit-o. ?i s-a mângâiat Isaac de pierderea mamei sale, Sarra.
Gen 25:1  Avraam însa ?i-a mai luat o femeie cu numele Chetura.
Gen 25:2  Ea i-a nascut pe Zimran, Ioc?an, Madan, Madian, I?bac ?i pe ?uah.
Gen 25:3  Lui Ioc?an i s-au nascut ?eba, Teman ?i Dedan. Iar fiii lui Dedan au fost: Raguil, Navdeel, A?urim, Letu?im ?i Leumim.
Gen 25:4  Iar fiii lui Madian au fost: Efa, Efer, Enoh, Abida ?i Eldaa. Ace?tia to?i au fost fiii Cheturei.
Gen 25:5  Însa Avraam a dat toate averile sale fiului sau Isaac.
Gen 25:6  Iar fiilor ?iitoarelor sale, Avraam le-a facut daruri ?i, înca fiind el în via?a, i-a trimis departe de la Isaac, fiul sau, spre rasarit, în pamântul Rasaritului.
Gen 25:7  Zilele vie?ii lui Avraam, câte le-a trait, au fost o suta ?aptezeci ?i cinci de ani.
Gen 25:8  Apoi, slabind, Avraam a murit la batrâne?i adânci, satul de zile ?i s-a adaugat la poporul sau.
Gen 25:9  ?i l-au îngropat feciorii lui, Isaac ?i Ismael, în pe?tera Macpela, din ?arina lui Efron, fiul lui ?ohar Heteeanul, în fa?a stejarului Mamvri;
Gen 25:10  Deci în ?arina ?i în pe?tera pe care Avraam a cumparat-o de la fiii lui Het, acolo sunt îngropa?i Avraam ?i Sarra, femeia lui.
Gen 25:11  Iar dupa moartea lui Avraam, a binecuvântat Dumnezeu pe Isaac, fiul lui. ?i locuia Isaac la Beer-Lahai-Roi.
Gen 25:12  Iata acum ?i via?a lui Ismael, fiul lui Avraam, pe care l-a nascut lui Avraam egipteanca Agar, slujnica Sarrei;
Gen 25:13  ?i iata numele fiilor lui Ismael, dupa ?irul na?terii lor: întâiul nascut al lui Ismael a fost Nebaiot; dupa el urmeaza Chedar, Adbeel ?i Mibsam,
Gen 25:14  Mi?ma, Duma ?i Masa,
Gen 25:15  Hadad, Tema, Etur, Nafi? ?i Chedma.
Gen 25:16  Ace?tia sunt fiii lui Ismael ?i acestea sunt numele lor, dupa a?ezarile lor ?i dupa taberele lor. Ace?tia sunt cei doisprezece voievozi ai semin?iilor lor.
Gen 25:17  Iar anii vie?ii lui Ismael au fost o suta treizeci ?i ?apte ?i, îmbatrânind, a murit ?i a trecut la parin?ii sai;
Gen 25:18  Iar urma?ii sai s-au întins de la Havila pâna la Sur, care este în fa?a Egiptului, pe drumul ce duce spre Asiria; ?i s-au sala?luit ei înaintea tuturor fra?ilor lor.
Gen 25:19  Iar spi?a neamului lui Isaac, fiul lui Avraam, este aceasta: lui Avraam i s-a nascut Isaac.
Gen 25:20  Isaac însa era de patruzeci de ani, când ?i-a luat de femeie pe Rebeca, fata lui Batuel Arameul din Mesopotamia ?i sora lui Laban Arameul.
Gen 25:21  ?i s-a rugat Isaac Domnului pentru Rebeca, femeia sa, ca era stearpa; ?i l-a auzit Domnul ?i femeia lui Rebeca a zamislit.
Gen 25:22  Dar copiii au început a se zbate în pântecele ei ?i ea a zis: "Daca a?a au sa fie, atunci la ce mai am aceasta sarcina?" ?i s-a dus sa întrebe pe Domnul.
Gen 25:23  Domnul însa i-a zis: "În pântecele tau sunt doua neamuri ?i doua popoare se vor ridica din pântecele tau; un popor va ajunge mai puternic decât celalalt ?i cel mai mare va sluji celui mai mic!"
Gen 25:24  ?i i-a venit Rebecai vremea sa nasca ?i iata erau în pântecele ei doi gemeni.
Gen 25:25  ?i cel dintâi care a ie?it era ro?u ?i peste tot paros, ca o pielicica, ?i i-a pus numele Isav.
Gen 25:26  Dupa aceea a ie?it fratele acestuia, ?inându-se cu mâna de calcâiul lui Isav; ?i i s-a pus numele Iacov. Isaac însa era de ?aizeci de ani, când i s-au nascut ace?tia din Rebeca.
Gen 25:27  Copiii ace?tia au crescut ?i a ajuns Isav om iscusit la vânatoare, traind pe câmpii; iar Iacov era om lini?tit, traind în corturi.
Gen 25:28  Isaac iubea pe Isav, pentru ca îi placea vânatul acestuia; iar Rebeca iubea pe Iacov.
Gen 25:29  O data însa a fiert Iacov linte, iar Isav a venit ostenit de la câmp.
Gen 25:30  ?i a zis Isav catre Iacov: "Da-mi sa manânc din aceasta fiertura ro?ie, ca sunt flamând!" De aceea Isav s-a mai numit ?i Edom.
Gen 25:31  Iacov însa i-a raspuns lui Isav: "Vinde-mi mai întâi dreptul tau de întâi-nascut!"
Gen 25:32  ?i Isav a raspuns: "Iata eu mor. La ce mi-e bun dreptul de întâi-nascut?"
Gen 25:33  Zisu-i-a Iacov: "Jura-mi-te acum!" ?i i s-a jurat Isav ?i a vândut lui Iacov dreptul sau de întâi-nascut.
Gen 25:34  Atunci Iacov a dat lui Isav pâine ?i fiertura de linte ?i acesta a mâncat ?i a baut, apoi s-a sculat ?i s-a dus. ?i astfel a nesocotit Isav dreptul sau de întâi-nascut.
Gen 26:1  ?i a fost o foamete în ?ara, afara de foametea cea dintâi, care se întâmplase în zilele lui Avraam. Atunci s-a dus Isaac în Gherara, la Abimelec, regele Filistenilor.
Gen 26:2  Atunci Domnul i S-a aratat ?i i-a zis: "Sa nu te duci în Egipt, ci sa locuie?ti în ?ara, unde-?i voi zice Eu.
Gen 26:3  Locuie?te în ?ara aceasta ?i Eu voi fi cu tine ?i te voi binecuvânta, ca ?ie ?i urma?ilor tai voi da toate ?inuturile acestea ?i-Mi voi împlini juramântul cu care M-am jurat lui Avraam, tatal tau.
Gen 26:4  Voi înmul?i pe urma?ii tai ca stelele cerului ?i voi da urma?ilor tai toate ?inuturile acestea; ?i se vor binecuvânta întru urma?ii tai toate popoarele pamântului,
Gen 26:5  Pentru ca Avraam, tatal tau, a ascultat cuvântul Meu ?i a pazit poruncile Mele, pove?ele Mele, îndreptarile Mele ?i legile Mele!"
Gen 26:6  De aceea s-a a?ezat Isaac în Gherara.
Gen 26:7  Iar locuitorii ?inutului aceluia l-au întrebat despre Rebeca, femeia sa, cine e ?i el a zis: "Aceasta este sora mea!", caci s-a temut sa zica: "E femeia mea!", ca nu cumva sa-l omoare oamenii locului aceluia din pricina Rebecai, pentru ca era frumoasa la chip.
Gen 26:8  Dar dupa ce a trait el acolo multa vreme, s-a întâmplat ca Abimelec, regele Filistenilor, sa se uite pe fereastra ?i sa vada pe Isaac jucându-se cu Rebeca, femeia sa.
Gen 26:9  Atunci a chemat Abimelec pe Isaac ?i i-a zis: "Adevarat e ca-i femeia ta? De ce dar ai zis: "Aceasta-i sora mea?" ?i Isaac a raspuns: "Pentru ca ma temeam sa nu fiu omorât din pricina ei".
Gen 26:10  Zisu-i-a Abimelec: "Pentru ce ne-ai facut aceasta? Pu?in a lipsit ca cineva din neamul meu sa se fi culcat cu femeia ta ?i ne-ai fi facut sa savâr?im pacat".
Gen 26:11  Apoi a dat Abimelec porunca la tot poporul sau, zicând: "Tot cel ce se va atinge de omul acesta ?i de femeia lui va fi vinovat mor?ii".
Gen 26:12  ?i a semanat Isaac în pamântul acela ?i a cules anul acela rod însutit. Domnul l-a binecuvântat,
Gen 26:13  ?i omul acela a ajuns bogat ?i a sporit tot mai mult, pâna ce a ajuns bogat foarte.
Gen 26:14  Avea turme de oi, cirezi de vite ?i ogoare multe, încât îl pizmuiau Filistenii.
Gen 26:15  Toate fântânile, pe care le sapasera robii tatalui sau, în zilele lui Avraam, tatal sau, Filistenii le-au stricat ?i le-au umplut cu pamânt.
Gen 26:16  Atunci a zis Abimelec catre Isaac: "Du-te de la noi, ca te-ai facut mult mai tare decât noi!"
Gen 26:17  ?i s-a dus Isaac de acolo ?i, tabarând în valea Gherara, a locuit acolo.
Gen 26:18  Apoi a sapat Isaac din nou fântânile de apa, pe care le sapasera robii lui Avraam, tatal sau, ?i pe care le astupasera Filistenii dupa moartea lui Avraam, tatal sau, ?i le-a numit cu acelea?i nume, cu care le numise Avraam, tatal sau.
Gen 26:19  Dupa aceea au mai sapat slugile lui Isaac ?i în valea Gherara ?i au aflat acolo izvor de apa buna de baut.
Gen 26:20  Dar se certau ciobanii din Gherara cu ciobanii lui Isaac, zicând: "Apa este a noastra!" De aceea Isaac a pus fântânii aceleia numele Esec, din pricina ca se sfadisera pentru ea.
Gen 26:21  Ducându-se apoi de acolo, Isaac a sapat alta fântâna ?i se certau ?i de la aceasta. De aceea Isaac i-a pus numele Sitna.
Gen 26:22  Apoi s-a mutat ?i de aici ?i a sapat alta fântâna, pentru care nu s-au mai certat, ?i i-a pus numele Rehobot, caci î?i zicea: "Datu-ne-a astazi Domnul loc larg ?i vom spori pe pamânt".
Gen 26:23  De aici Isaac s-a urcat catre Beer-?eba.
Gen 26:24  În noaptea aceea i S-a aratat Domnul ?i i-a zis: "Eu sunt Dumnezeul lui Avraam, tatal tau! Nu te teme, ca Eu sunt cu tine ?i te voi binecuvânta ?i voi înmul?i pe urma?ii tai, pentru Avraam, sluga Mea".
Gen 26:25  Acolo a facut Isaac jertfelnic ?i a chemat numele Domnului ?i î?i întinse acolo cortul sau. ?i au sapat acolo slugile lui Isaac o fântâna, în valea Gherarei.
Gen 26:26  Atunci au venit din Gherara la el: Abimelec ?i Ahuzat, care luase pe nora lui, ?i Ficol, capetenia o?tirii lui.
Gen 26:27  Iar Isaac le-a zis: "La ce a?i venit la mine, voi care ma urâ?i ?i n-a?i alungat de la voi?"
Gen 26:28  Iar ei au zis: "Am vazut bine ca Domnul este cu tine ?i am zis sa facem cu tine juramânt ?i sa încheiem legamânt cu tine,
Gen 26:29  Ca tu sa nu ne faci nici un rau, cum nici noi nu ne-am atins de tine, ci ti-am facut bine ?i te-am scos de la noi cu pace; ?i acum e?ti binecuvântat de Domnul".
Gen 26:30  Atunci Isaac le-a facut ospa? ?i ei au mâncat ?i au baut.
Gen 26:31  Sculându-se apoi a doua zi de diminea?a, au jurat unul altuia. ?i le-a dat drumul Isaac ?i ei s-au dus de la dânsul cu pace.
Gen 26:32  Tot în ziua aceea venind slugile lui Isaac, l-au vestit de fântâna ce o sapasera ?i au zis: "Am gasit apa!"
Gen 26:33  ?i a numit Isaac fântâna aceea ?ibea, adica juramânt. De aceea se ?i nume?te cetatea aceea Beer-?eba, adica fântâna juramântului, pâna în ziua de astazi.
Gen 26:34  Iar Isav, fiind acum de patruzeci de ani, ?i-a luat doua femei: pe Iudit, fata lui Beeri Heteul, ?i pe Basemata, fata lui Elon Heteul.
Gen 26:35  Dar ele amarau pe Isaac ?i pe Rebeca.
Gen 27:1  Iar dupa ce a îmbatrânit Isaac ?i au slabit vederile ochilor sai, a chemat pe Isav, pe fiul sau cel mai mare, ?i i-a zis: "Fiul meu!" Zis-a acela: "Iata-ma!"
Gen 27:2  ?i Isaac a zis: "Iata, eu am îmbatrânit ?i nu ?tiu ziua mor?ii mele.
Gen 27:3  Ia-?i dara uneltele tale, tolba ?i arcul, ?i ie?i la câmp ?i adu-mi ceva vânat;
Gen 27:4  Sa-mi faci mâncare, cum îmi place mie, ?i adu-mi sa manânc, ca sa te binecuvânteze sufletul meu pâna nu mor!"
Gen 27:5  Rebeca însa a auzit ce a zis Isaac catre fiul sau Isav. S-a dus deci Isav la câmp sa vâneze ceva pentru tatal sau;
Gen 27:6  Iar Rebeca a zis catre Iacov, fiul cel mai mic: "Iata, eu am auzit pe tatal tau graind cu Isav, fratele tau, ?i zicând:
Gen 27:7  "Adu vânat ?i fa-mi o mâncare sa manânc ?i sa te binecuvântez înaintea Domnului, pâna a nu muri".
Gen 27:8  Acum dar, fiul meu, asculta ce am sa-?i poruncesc:
Gen 27:9  Du-te la turma, adu-mi de acolo doi iezi tineri ?i buni ?i voi face din ei mâncare, cum îi place tatalui tau;
Gen 27:10  Iar tu o vei duce tatalui tau sa manânce, ca sa te binecuvânteze tatal tau înainte de a muri".
Gen 27:11  Iacov însa a zis catre Rebeca, mama sa: "Isav, fratele meu, e om paros, iar eu n-am par.
Gen 27:12  Nu cumva tatal meu sa ma pipaie ?i voi fi în ochii lui ca un în?elator ?i în loc de binecuvântare, voi atrage asupra-mi blestem".
Gen 27:13  Zis-a mama sa: "Fiul meu, asupra mea sa fie blestemul acela; asculta numai pova?a mea ?i du-te ?i adu-mi iezii!"
Gen 27:14  Atunci, ducându-se Iacov, a luat ?i a adus mamei sale iezii, iar mama sa a gatit mâncare, cum îi placea tatalui lui.
Gen 27:15  Apoi a luat Rebeca haina cea mai frumoasa a lui Isav, fiul ei cel mai mare, care era la ea în casa, ?i a îmbracat pe Iacov, fiul ei cel mai mic;
Gen 27:16  Iar cu pieile iezilor a înfa?urat bra?ele ?i par?ile goale ale gâtului lui.
Gen 27:17  ?i a dat mâncarea ?i pâinea ce gatise în mâinile lui Iacov, fiul sau,
Gen 27:18  ?i acesta a intrat la tatal sau ?i a zis: "Tata!" Iar acela a raspuns: "Iata-ma! Cine e?ti tu, copilul meu?"
Gen 27:19  Zis-a Iacov catre tatal sau: "Eu sunt Isav, întâiul tau nascut. Am facut precum mi-ai poruncit; scoala ?i ?ezi de manânca din vânatul meu, ca sa ma binecuvânteze sufletul tau!"
Gen 27:20  Zis-a Isaac catre fiul sau: "Cum l-ai gasit a?a curând, fiul meu?" ?i acesta a zis: "Domnul Dumnezeul tau mi l-a scos înainte".
Gen 27:21  Zis-a Isaac iara?i catre Iacov: "Apropie-te sa te pipai, fiul meu, de e?ti tu fiul meu Isav sau nu".
Gen 27:22  ?i s-a apropiat Iacov de Isaac, tatal sau, iar acesta l-a pipait ?i a zis: "Glasul este glasul lui Iacov, iar mâinile sunt mâinile lui Isav".
Gen 27:23  Dar nu l-a cunoscut, pentru ca mâinile lui erau paroase, ca mâinile fratelui sau Isav; ?i l-a binecuvântat.
Gen 27:24  ?i a mai zis: "Tu oare e?ti fiul meu Isav?" ?i Iacov a raspuns: "Eu".
Gen 27:25  Zis-a Isaac: "Adu-mi ?i voi mânca din vânatul fiului meu,  ca sa te binecuvânteze sufletul meu!" ?i i-a adus ?i a mâncat; apoi i-a adus ?i vin ?i a baut.
Gen 27:26  Dupa aceea Isaac, tatal sau, i-a zis: "Apropie-te, fiule, ?i ma saruta!" â
Gen 27:27  Atunci s-a apropiat Iacov ?i l-a sarutat. ?i a sim?it Isaac mirosul hainei lui ?i l-a binecuvântat ?i a zis: "Iata, mirosul fiului meu e ca mirosul unui câmp bogat, pe care l-a binecuvântat Domnul.
Gen 27:28  Sa-?i dea ?ie Dumnezeu din roua cerului ?i din bel?ugul pamântului, pâine multa ?i vin.
Gen 27:29  Sa-?i slujeasca popoarele ?i capeteniile sa se închine înaintea ta; sa fii stapân peste fra?ii tai ?i feciorii mamei tale sa ?i se închine ?ie; cel ce te va blestema sa fie blestemat ?i binecuvântat sa fie cel ce te va binecuvânta!"
Gen 27:30  Îndata ce a ispravit Isaac de binecuvântat pe Iacov, fiul sau, ?i cum a ie?it Iacov de la fa?a tatalui sau Isaac, a venit ?i Isav cu vânatul lui.
Gen 27:31  A facut ?i el bucate ?i le-a adus tatalui sau ?i  a zis catre  tatal sau: "Scoala, tata, ?i manânca din vânatul fiului tau, ca sa ma binecuvânteze sufletul tau!"
Gen 27:32  Iar Isaac, tatal sau, i-a zis: "Cine e?ti tu?" Iar el a zis: "Eu sunt Isav, fiul tau cel întâi-nascut!"
Gen 27:33  Atunci s-a cutremurat Isaac cu cutremur mare foarte ?i a zis: "Dar cine-i acela, care a cautat ?i mi-a adus vânat ?i am mâncat de la el înainte de a veni tu ?i l-am binecuvântat ?i binecuvântat va fi?"
Gen 27:34  Iar Isav, auzind cuvintele tatalui sau Isaac, a strigat cu glas mare ?i foarte dureros ?i a zis catre tatal sau: "Binecuvânteaza-ma ?i pe mine, tata!"
Gen 27:35  Zis-a Isaac catre el: "A venit fratele tau cu în?elaciune ?i a luat binecuvântarea ta".
Gen 27:36  Iar Isav a zis: "Din pricina oare ca-l cheama Iacov, de aceea m-a în?elat de doua ori? Deunazi mi-a rapit dreptul de întâi-nascut, iar acum mi-a rapit binecuvântarea mea". Apoi a zis Isav catre tatal sau: "Nu mi-ai pastrat ?i mie binecuvântare, tata?"
Gen 27:37  Raspuns-a Isaac ?i a zis lui Isav: "Iata, stapân l-am facut peste tine ?i pe to?i fra?ii lui i-am facut lui robi; cu pâine ?i cu vin l-am daruit. Dar cu tine ce sa fac, fiul meu?"
Gen 27:38  ?i a zis Isav catre tatal sau: "Tata, oare numai o binecuvântare ai tu? Binecuvânteaza-ma ?i pe mine, tata:" ?i cum Isaac tacea, Isav ?i-a ridicat glasul ?i a început a plânge.
Gen 27:39  Atunci, raspunzând Isaac, tatal lui, a zis catre el: "Iata, locuin?a ta va fi un pamânt manos ?i cerul î?i va trimite roua sa;
Gen 27:40  Cu sabia ta vei trai ?i vei fi supus fratelui tau; va veni însa vremea când te vei ridica ?i vei sfarâma jugul lui de pe grumazul tau".
Gen 27:41  De aceea ura Isav pe Iacov pentru binecuvântarea cu care-l binecuvântase tatal sau. ?i a zis Isav în cugetul sau: "Se apropie zilele de jelire pentru tatal meu; atunci am sa ucid pe Iacov, fratele meu!"
Gen 27:42  Dar i s-a spus Rebecai cuvintele lui Isav, fiul cel mai mare, ?i ea a trimis de a chemat pe Iacov, fiul ei cel mai mic, ?i i-a zis: "Iata Isav, fratele tau, vrea sa se razbune pe tine, omorându-te.
Gen 27:43  Acum dar, fiul meu, asculta pova?a mea ?i, sculându-te, fugi la Haran, în Mesopotamia, la fratele meu Laban,
Gen 27:44  ?i stai la el câtva timp, pâna se va potoli mânia fratelui tau
Gen 27:45  ?i pâna va mai uita el ce i-ai facut. Atunci voi trimite ?i te voi lua de acolo. Pentru ce sa ramân eu într-o singura zi fara de voi amândoi?"
Gen 27:46  Apoi Rebeca a zis catre Isaac: "Sunt scârbita de via?a mea, din pricina fetelor Heteilor. Daca-?i ia ?i Iacov femeie ca acestea, din fetele pamântului acestuia, atunci la ce-mi mai e buna via?a?"
Gen 28:1  Atunci a chemat Isaac pe Iacov ?i l-a binecuvântat ?i i-a poruncit, zicând: "Sa nu-?i iei femeie din fetele Canaaneilor;
Gen 28:2  Ci scoala ?i mergi în Mesopotamia în casa lui Batuel, tatal mamei tale, ?i-?i ia femeie de acolo, din fetele lui Laban, fratele mamei tale,
Gen 28:3  ?i Dumnezeul cel Atotputernic sa te binecuvânteze, sa te creasca ?i sa te înmul?easca ?i sa lasara din tine popoare multe;
Gen 28:4  Sa-?i dea binecuvântarea lui Avraam, tatal meu, ?ie ?i urma?ilor tai ca sa stapâne?ti pamântul ce-l locuie?ti acum ?i pe care l-a dat Dumnezeu lui Avraam,
Gen 28:5  ?i a?a Isaac i-a dat drumul lui Iacov, iar acesta s-a dus în Mesopotamia, la Laban, fiul lui Batuel Arameul ?i fratele Rebecai, mama lui Iacov ?i a lui Isav.
Gen 28:6  Vazând însa Isav ca Isaac a binecuvântat pe Iacov ?i l-a trimis în Mesopotamia, sa-?i ia femeie de acolo, pentru care l-a binecuvântat ?i i-a poruncit, zicând: "Sa nu-?i iei femeie din fetele Canaaneilor",
Gen 28:7  ?i ca Iacov a ascultat pe tatal sau ?i pe mama sa ?i s-a dus în Mesopotamia,
Gen 28:8  ?i în?elegând Isav ca lui Isaac, tatal sau, nu-i plac fetele Canaaneilor,
Gen 28:9  S-a dus la Ismael ?i, pe lânga cele doua femei ale sale, ?i-a mai luat ?i pe Mahalat, fata lui Ismael, fiul lui Avraam, ?i sora lui Nebaiot.
Gen 28:10  Iar Iacov ie?ind din Beer-?eba, s-a dus în Haran.
Gen 28:11  Ajungând însa la un loc, a ramas sa doarma acolo, caci asfin?ise soarele. ?i luând una din pietrele locului aceluia ?i punându-?i-o capatâi, s-a culcat în locul acela.
Gen 28:12  ?i a visat ca era o scara, sprijinita pe pamânt, iar cu vârful atingea cerul; iar îngerii lui Dumnezeu se suiau ?i se pogorau pe ea.
Gen 28:13  Apoi S-a aratat Domnul în capul scarii ?i i-a zis: "Eu sunt Domnul, Dumnezeul lui Avraam, tatal tau, ?i Dumnezeul lui Isaac. Nu te teme! Pamântul pe care dormi ?i-l voi da ?ie ?i urma?ilor tai.
Gen 28:14  Urma?ii tai vor fi mul?i ca pulberea pamântului ?i tu te vei întinde la apus ?i la rasarit, la miazanoapte ?i la miazazi, ?i se vor binecuvânta întru tine ?i întru urma?ii tai toate neamurile pamântului.
Gen 28:15  Iata, Eu sunt cu tine ?i te voi pazi în orice cale vei merge; te voi întoarce în pamântul acesta ?i nu te voi lasa pâna nu voi împlini toate câte ?i-am spus".
Gen 28:16  Iar când s-a de?teptat din somnul sau, Iacov a zis: "Domnul este cu adevarat în locul acesta ?i eu n-am ?tiut!"
Gen 28:17  ?i, spaimântându-se Iacov, a zis: "Cât de înfrico?ator este locul acesta! Aceasta nu e alta fara numai casa lui Dumnezeu, aceasta e poarta cerului!"
Gen 28:18  Apoi s-a sculat Iacov dis-de-diminea?a, a luat piatra ce ?i-o pusese capatâi, a pus-o stâlp ?i a turnat pe vârful ei untdelemn.
Gen 28:19  Iacov a pus locului aceluia numele Betel (casa lui Dumnezeu), caci mai înainte cetatea aceea se numea Luz.
Gen 28:20  ?i a facut Iacov fagaduin?a, zicând: "De va fi Domnul Dumnezeu cu mine ?i ma va pova?ui în calea aceasta, în care merg eu astazi, de-mi va da pâine sa manânc ?i haine sa ma îmbrac;
Gen 28:21  ?i de ma voi întoarce sanatos la casa tatalui meu, atunci Domnul va fi Dumnezeul meu.
Gen 28:22  Iar piatra aceasta, pe care am pus-o stâlp, va fi pentru mine casa lui Dumnezeu ?i din toate câte-mi vei da Tu mie, a zecea parte o voi da ?ie".
Gen 29:1  Sculându-se apoi, Iacov s-a dus în pamântul fiilor Rasaritului, la Laban, fiul lui Batuel Arameul ?i fratele Rebecai, mama lui Iacov ?i a lui Isav.
Gen 29:2  ?i cautând el o data, iata în câmp o fântâna, iar lânga ea trei turme de oi culcate; caci din fântâna aceea se adapau turmele ?i pe gura fântânii era o piatra mare.
Gen 29:3  Când se adunau acolo toate turmele, ciobanii pravaleau piatra de pe gura fântânii ?i adapau oile, apoi iar puneau piatra la locul ei pe gura fântânii.
Gen 29:4  Deci a zis Iacov catre pastori: "Fra?ilor, de unde sunte?i voi?" Iar ei au zis: "Noi suntem din Haran".
Gen 29:5  ?i el le-a zis: "Cunoa?te?i voi pe Laban, feciorul lui Nahor?" Raspuns-au aceia: "Îl cunoa?tem".
Gen 29:6  Zis-a iara?i Iacov: "E sanatos?" ?i ei au zis: "Sanatos. Iata Rahila, fata lui, vine cu oile".
Gen 29:7  Zis-a Iacov catre ei: "Mai e înca mult din zi ?i nu e înca vremea sa se adune turmele; adapa?i oile ?i duce?i-va de le pa?te?i".
Gen 29:8  Iar ei au zis: "Pâna nu se aduna to?i pastorii, nu putem sa pravalim piatra de pe gura fântânii, ca sa adapam oile!"
Gen 29:9  Înca graind el cu ei, iata a venit Rahila, fiica lui Laban, cu oile tatalui sau, caci ea pa?tea oile tatalui sau.
Gen 29:10  Vazând Iacov pe Rahila, fiica lui Laban, fratele mamei sale, ?i oile lui Laban, fratele mamei sale, s-a apropiat Iacov ?i a pravalit piatra de pe gura fântânii ?i a adapat oile lui Laban, fratele mamei sale.
Gen 29:11  ?i a sarutat Iacov pe Rahila ?i ?i-a ridicat glasul ?i a plâns.
Gen 29:12  Apoi a spus Rahilei ca-i ruda cu tatal ei ?i ca-i fiul Rebecai. Iar ea a alergat ?i a spus tatalui sau toate acestea.
Gen 29:13  Auzind Laban de sosirea lui Iacov, fiul surorii sale, a alergat în întâmpinarea lui ?i, îmbra?i?ându-l, l-a sarutat ?i l-a adus în casa sa ?i el a povestit lui Laban toate.
Gen 29:14  Iar Laban i-a zis: "Tu e?ti din oasele mele ?i din carnea mea". ?i a stat Iacov la el o luna de zile.
Gen 29:15  Atunci Laban a zis catre Iacov: "Au doara îmi vei sluji în dar, pentru ca îmi e?ti ruda? Spune-mi, care-?i va fi simbria?"
Gen 29:16  Laban însa avea doua fete: pe cea mai mare o chema Lia ?i pe cea mai mica o chema Rahila.
Gen 29:17  Lia era bolnava de ochi, iar Rahila era chipe?a la statura ?i tare frumoasa la fa?a.
Gen 29:18  Lui Iacov însa îi era draga Rahila ?i a zis: "Î?i voi sluji ?apte ani pentru Rahila, fata ta cea mai mica".
Gen 29:19  Zisu-i-a Laban: "Mai bine s-o dau dupa tine decât s-o dau dupa alt barbat. Ramâi la mine!"
Gen 29:20  ?i a slujit Iacov pentru Rahila ?apte ani ?i i s-a parut numai câteva zile, pentru ca o iubea.
Gen 29:21  Apoi a zis Iacov catre Laban: "Da-mi femeia, ca mi s-au împlinit zilele sa intru la ea".
Gen 29:22  Atunci a chemat Laban pe to?i oamenii locului aceluia ?i a facut ospa?.
Gen 29:23  Iar seara a luat Laban pe fiica sa Lia ?i a bagat-o înauntru ?i a intrat Iacov la ea.
Gen 29:24  ?i Laban a dat pe roaba sa Zilpa, roaba fiicei sale Lia.
Gen 29:25  Dar când s-a facut ziua, iata era Lia. ?i a zis Iacov catre Laban: "Pentru ce mi-ai facut aceasta? Nu Îi-am slujit eu oare pentru Rahila? Pentru ce m-ai în?elat?"
Gen 29:26  Raspuns-a Laban: "Aici la noi nu se pomene?te sa se marite fata cea mai mica înaintea celei mai mari.
Gen 29:27  Împline?te aceasta saptamâna de nunta ?i-?i voi da-o ?i pe aceea, pentru slujba ce-mi vei mai face al?i ?apte ani!"
Gen 29:28  ?i a facut Iacov a?a: a împlinit saptamâna de nunta ?i i-a dat Laban ?i pe Rahila, fiica sa, de femeie.
Gen 29:29  Atunci a dat Laban pe roaba sa Bilha, roaba fiicei sale Rahila.
Gen 29:30  A intrat deci Iacov ?i la Rahila ?i iubea el pe Rahila mai mult decât pe Lia. Apoi a mai slujit Iacov lui Laban al?i ?apte ani.
Gen 29:31  Vazând însa Domnul Dumnezeu ca Lia era dispre?uita, a deschis pântecele ei, iar Rahila fu stearpa.
Gen 29:32  A zamislit deci Lia ?i a nascut lui Iacov un fiu, caruia i-a pus numele Ruben, zicând: "A cautat Domnul la smerenia mea ?i mi-a dat fiu; de acum ma va iubi barbatul meu".
Gen 29:33  Apoi a zamislit Lia iara?i ?i a nascut lui Iacov al doilea fiu ?i a zis: "Auzit-a Domnul ca nu sunt iubita ?i mi-a dat ?i pe acesta". ?i i-a pus numele Simeon.
Gen 29:34  ?i iara?i a zamislit ea ?i a mai nascut un fiu ?i a zis: "De acum se va lipi de mine barbatul meu, caci i-am nascut trei fii". De aceea i-a pus acestuia numele Levi.
Gen 29:35  ?i iara?i a zamislit ?i a mai nascut un fiu ?i a zis: "Acum voi lauda pe Domnul!" De aceea i-a pus numele Iuda. Apoi a încetat Lia de a mai na?te.
Gen 30:1  Iar Rahila, vazând ca ea n-a nascut lui Iacov nici un fiu, a prins pizma pe sora sa ?i a zis lui Iacov: "Da-mi copii, iar de nu, voi muri".
Gen 30:2  Mâniindu-se însa Iacov pe Rahila, i-a zis: "Au doara eu sunt Dumnezeu, Care a stârpit rodul pântecelui tau?"
Gen 30:3  Atunci Rahila a zis catre Iacov: "Iata roaba mea Bilha; intra la ea ?i ea va na?te pe genunchii mei ?i voi avea ?i eu copii printr-însa".
Gen 30:4  ?i i-a dat pe Bilha, roaba sa, de femeie ?i a intrat Iacov la ea;
Gen 30:5  Iar Bilha, roaba Rahilei, a zamislit ?i a nascut lui Iacov un fiu.
Gen 30:6  Atunci Rahila a zis: "Dumnezeu mi-a facut dreptate, a auzit glasul meu ?i mi-a dat fiu". De aceea i-a pus numele Dan.
Gen 30:7  ?i a zamislit iara?i Bilha, roaba Rahilei, ?i a mai nascut un fiu lui Iacov;
Gen 30:8  Iar Rahila a zis: "Lupta dumnezeiasca m-am luptat cu sora mea, am biruit ?i am ajuns deopotriva cu sora mea!" De aceea i-a pus numele Neftali.
Gen 30:9  Lia însa, vazând ca a încetat de a mai na?te, a luat pe roaba sa Zilpa ?i a dat-o lui Iacov de femeie ?i el a intrat la ea;
Gen 30:10  Zilpa, roaba Liei, a nascut lui Iacov un fiu.
Gen 30:11  Atunci a zis Lia: "Noroc" ?i i-a pus numele Gad.
Gen 30:12  Apoi iara?i a zamislit Zilpa, roaba Liei, ?i a nascut lui Iacov alt fiu.
Gen 30:13  ?i a zis Lia: "Spre fericirea mea s-a nascut, ca ma vor ferici femeile!" ?i i-a pus numele A?er.
Gen 30:14  Iar pe vremea seceratului grâului s-a dus Ruben ?i, gasind în ?arina mandragore, le-a adus la mama sa Lia. Rahila insa a zis catre Lia, sora sa: "Da-mi ?i mie din mandragorele fiului tau!"
Gen 30:15  Iar Lia a zis: "Nu-?i ajunge ca mi-ai luat barbatul? Vrei sa iei ?i mandragorele fiului meu?" ?i Rahila a zis: "Nu a?a, ci pentru mandragorele fiului tau, sa se culce Iacov noaptea aceasta cu tine!"
Gen 30:16  Venind Iacov seara de la câmp, i-a ie?it Lia înainte ?i i-a zis: "Sa intri la mine astazi, ca te-am cumparat cu mandragorele fiului meu!" ?i în noaptea aceea s-a culcat Iacov cu ea.
Gen 30:17  ?i a auzit Dumnezeu pe Lia ?i ea a zamislit ?i a nascut lui Iacov al cincilea fiu.
Gen 30:18  Atunci a zis Lia: "Mi-a dat rasplata Dumnezeu pentru ca am dat barbatului meu pe roaba mea". ?i a pus copilului numele Isahar, adica rasplata.
Gen 30:19  Apoi a mai zamislit Lia înca o data ?i a nascut lui Iacov al ?aselea fiu.
Gen 30:20  ?i a zis Lia: "Dar minunat mi-a daruit Dumnezeu în timpul de acum! De acum barbatul meu va ?edea la mine, ca i-am nascut ?ase feciori". ?i a pus copilului numele Zabulon.
Gen 30:21  Dupa aceea Lia a mai nascut o fata ?i i-a pus numele Dina.
Gen 30:22  Dar ?i-a adus aminte Dumnezeu ?i de Rahila ?i a auzit-o Dumnezeu ?i i-a deschis pântecele.
Gen 30:23  ?i zamislind, ea a nascut lui Iacov un fiu; ?i a zis Rahila: "Ridicat-a Dumnezeu ocara de la mine!"
Gen 30:24  ?i a pus copilului numele Iosif, zicând: "Domnul îmi va mai da ?i alt fiu!"
Gen 30:25  Iar dupa ce a nascut Rahila pe Iosif, Iacov a zis catre Laban: "Lasa-ma sa plec, sa ma duc la mine, în pamântul meu.
Gen 30:26  Da-mi femeile mele ?i copiii mei, pentru care ?i-am slujit, ca sa ma duc, caci tu ?tii ce slujba ?i-am facut".
Gen 30:27  Laban însa i-a zis: "De am aflat har înaintea ta, mai ramâi la mine! Caci vad bine ca Dumnezeu m-a binecuvântat prin venirea ta".
Gen 30:28  Apoi a adaugat: "Spune simbria ce voie?ti ?i-?i voi da-o!"
Gen 30:29  Iacov însa i-a raspuns: "Tu ?tii cum ?i-am slujit ?i cum sunt vitele tale, de când am venit eu la tine;
Gen 30:30  Caci erau pu?ine când am venit eu, iar de atunci s-au înmul?it ?i te-a binecuvântat Dumnezeu prin venirea mea. Când însa am sa lucrez eu ?i pentru casa mea?"
Gen 30:31  Raspunsu-i-a Laban: "Ce sa-?i dau?" ?i Iacov a zis: "Sa nu-mi dai nimic. Dar de faci ce-?i voi spune eu, voi mai pa?te ?i voi mai pazi oile tale.
Gen 30:32  Sa treaca astazi toate oile tale pe dinaintea noastra ?i sa despar?im din ele orice oaie pestri?a sau tarcata sau neagra, iar dintre capre cele pestri?e sau tarcate: aceea sa fie simbria mea.
Gen 30:33  Credincio?ia mea va raspunde pentru mine înaintea ta mâine, când vei veni sa-mi statornice?ti simbria: tot ce nu va fi bal?at sau tarcat între caprele mele ?i tot ce nu va fi tarcat sau negru între oile mele se va socoti ca furat de mine".
Gen 30:34  Zis-a Laban catre el: "Bine, sa fie cum zici tu!"
Gen 30:35  ?i a ales Iacov în ziua aceea ?apii cei varga?i sau pestri?i ?i toate caprele bal?ate sau tarcate, toate câte erau cu cit de pu?in alb, ?i toate oile tarcate sau negre ?i le-a dat în seama fiilor sai.
Gen 30:36  Iar Laban a hotarât ca departarea între dânsul ?i oile lui Iacov sa fie cale de trei zile. ?i a ramas Iacov sa pasca celelalte oi ale lui Laban.
Gen 30:37  Dupa aceea ?i-a luat Iacov nuiele verzi de plop, de migdal ?i de paltin, ?i a crestat pe ele dungi albe, luând de pe nuiele fâ?ii de coaja pâna la albea?a nuielelor.
Gen 30:38  Apoi punea nuielele crestate în jgheaburile de adapat, ca, venind sa bea, oile sa zamisleasca înaintea nuielelor din adapatori.
Gen 30:39  ?i zamisleau oile cum erau nuielele ?i fatau oile miei pestri?i, tarca?i ?i negri.
Gen 30:40  Iar mieii ace?tia îi alegea Iacov ?i punea înaintea oilor lui Laban numai tot ce era pestri? ?i tot ce era negru; dar turmele sale le ?inea despar?ite ?i nu le amesteca cu oile lui Laban.
Gen 30:41  Afara de aceasta Iacov, când zamisleau oile cele bune, punea nuiele pestri?e în adapatori înaintea lor, ca sa zamisleasca ele cum erau nuielele;
Gen 30:42  Iar când zamisleau cele rele, nu le punea nuielele ?i a?a cele ce se cuveneau lui Laban erau slabe, iar cele ce se cuveneau lui Iacov erau voinice.
Gen 30:43  De aceea s-a îmboga?it omul acesta foarte, foarte tare, ?i avea mul?ime de vite marunte ?i vite mari, roabe ?i robi, camile ?i asini.
Gen 31:1  A auzit însa Iacov vorbele feciorilor lui Laban, care ziceau: "Iacov a luat toate câte avea tatal nostru ?i din ale tatalui nostru ?i-a facut toata boga?ia aceasta".
Gen 31:2  ?i cautând Iacov la fa?a lui Laban, iata nu mai era fa?a de el ca mai înainte.
Gen 31:3  Atunci Domnul a zis catre Iacov: "Întoarce-te în ?ara parin?ilor tai, în patria ta, ?i Eu voi fi cu tine!"
Gen 31:4  Trimi?ând, deci, Iacov a chemat pe Rahila ?i pe Lia la câmp, unde erau turmele,
Gen 31:5  ?i le-a zis: "Vad eu ca fa?a tatalui vostru nu mai e fa?a de mine, ca mai înainte; dar Dumnezeul tatalui meu este cu mine.
Gen 31:6  Voi în?iva ?ti?i ca am slujit pe tatal vostru cu toata inima;
Gen 31:7  Iar tatal vostru m-a în?elat ?i de zeci de ori mi-a schimbat simbria, Dumnezeu însa nu i-a îngaduit sa-mi faca rau.
Gen 31:8  Când zicea el: Cele pestri?e sa fie simbria ta, toate oile fatau miei pestri?i; iar când zicea el: Cele negre sa-?i fie de simbrie, atunci toate oile fatau miei negri.
Gen 31:9  ?i a?a a luat Dumnezeu toate vitele de la tatal vostru ?i mi le-a dat mie.
Gen 31:10  Iata o data, pe vremea când intrau în calduri oile, mi-am ridicat ochii ?i am vazut în vis; ?i iata ca ?apii ?i berbecii, care sareau pe capre ?i pe oi, erau albi, varga?i ?i bal?a?i.
Gen 31:11  Iar îngerul Domnului mi-a zis în vis: "Iacove!" ?i eu am raspuns: "Ce este?"
Gen 31:12  Zis-a el: "Ridica-?i ochii ?i prive?te: to?i ?apii ?i berbecii, care sar pe capre ?i pe oi, sunt varga?i, pestri?i ?i bal?a?i, caci am vazut toate câte ?i-a facut Laban.
Gen 31:13  Eu sunt Dumnezeul, Cel ce ?i S-a aratat în Betel, unde Mi-ai turnat untdelemn pe stâlp ?i unde Mi-ai facut fagaduin?a. Scoala deci acum, ie?i din pamântul acesta ?i mergi în pamântul tau de na?tere ?i Eu voi fi cu tine".
Gen 31:14  Atunci Lia ?i Rahila i-au raspuns ?i au zis: "Mai avem noi oare parte ?i mo?tenire în casa tatalui nostru?
Gen 31:15  Oare n-am fost noi socotite de el ca ni?te straine, fiindca el ne-a vândut ?i a mâncat banii no?tri?
Gen 31:16  De aceea, toata averea pe care Dumnezeu a luat-o de la tatal nostru este a noastra ?i a copiilor no?tri. Fa dar acum toate câte ?i-a zis Domnul!"
Gen 31:17  Atunci s-a sculat Iacov ?i a urcat copiii ?i femeile sale pe camile,
Gen 31:18  A strâns toate turmele sale ?i toata boga?ia sa, pe care o agonisise în Mesopotamia, ?i toate ale sale, ca sa mearga la Isaac, tatal sau, în ?ara Canaanului.
Gen 31:19  Iar Laban, ducându-se sa-?i tunda oile, Rahila a furat idolii tatalui sau.
Gen 31:20  Deci Iacov a în?elat pe Laban Arameul, caci nu l-a vestit ca pleaca,
Gen 31:21  Ci a fugit cu toate câte avea ?i, trecând Eufratul, s-a îndreptat spre Muntele Galaadului.
Gen 31:22  Iar a treia zi i s-a dat de ?tire lui Laban Arameul, ca Iacov a fugit.
Gen 31:23  Atunci, luând Laban cu sine pe feciorii ?i pe rudele sale, a alergat dupa el cale de ?apte zile ?i l-a ajuns la Muntele Galaadului.
Gen 31:24  Dar Dumnezeu a venit la Laban Arameul noaptea în vis ?i i-a zis: "Fere?te-te, nu cumva sa vorbe?ti lui Iacov nici de bine, nici de rau".
Gen 31:25  ?i a ajuns Laban pe Iacov. Iacov însa î?i a?ezase cortul sau pe munte; ?i tot pe Muntele Galaad ?i l-a a?ezat ?i Laban cu rudele sale.
Gen 31:26  Atunci a zis Laban catre Iacov: "Ce ai facut? Pentru ce mi-ai furat inima ?i mi-ai luat fetele, ca ?i cum le-ai fi robit cu sabia?
Gen 31:27  Pentru ce ai fugit pe ascuns ?i m-ai în?elat, în loc sa ma în?tiin?ezi pe mine, care ?i-a? fi dat drumul cu veselie ?i cu cântari din timpane ?i din harfa?
Gen 31:28  Ba nu mi-ai îngaduit nici macar sa-mi sarut nepo?ii ?i fetele mele. Te-ai purtat, a?adar, ca un om fara de minte.
Gen 31:29  ?i acum mâna mea cea puternica ar putea sa-?i faca rau. Dar Dumnezeul tatalui tau mi-a vorbit ieri ?i mi-a zis: "Fere?te-te, nu cumva sa vorbe?ti lui Iacov nici de bine, nici de rau!"
Gen 31:30  Sa zicem ca ai plecat, pentru ca cu mare aprindere doreai casa tatalui tau. Dar atunci de ce mi-ai furat dumnezeii mei?"
Gen 31:31  Atunci raspunzând Iacov, a zis catre Laban: "M-am temut, caci ziceam: Nu cumva sa-?i iei fetele de la mine ?i toate ale mele.
Gen 31:32  Dar la cine vei gasi idolii tai, acela nu va mai trai. Cauta de fa?a cu rudele noastre ?i ia tot ce vei gasi al tau la mine!" Iacov însa nu ?tia ca Rahila, femeia sa, îi furase.
Gen 31:33  A intrat atunci Laban în cortul lui Iacov, ?i în cortul Liei, ?i în cortul celor doua roabe, ?i a cautat ?i n-a gasit nimic; apoi, ie?ind din cortul Liei, a intrat ?i în cortul Rahilei.
Gen 31:34  Rahila însa luase idolii ?i-i pusese sub samarul camilei ?i ?edea deasupra lor; ?i a scotocit Laban prin tot cortul Rahilei ?i n-a gasit nimic.
Gen 31:35  Iar ea a zis catre tatal sau: "Sa nu se mânie domnul meu ca nu ma pot scula înaintea ta, pentru ca tocmai acum am necazul obi?nuit al femeilor". ?i mai scotocind Laban prin tot cortul, n-a gasit idolii.
Gen 31:36  Atunci s-a mâniat Iacov ?i s-a plâns împotriva lui Laban. ?i  începând a grai, Iacov a zis lui Laban: "Care-i vina mea ?i care-i pacatul meu, de te înver?unezi împotriva mea?
Gen 31:37  Daca ai rascolit toate lucrurile din casa mea, gasit-ai, oare, ceva din ale casei tale? Arata aici înaintea rudeniilor tale ?i înaintea rudeniilor mele, ca sa ne judece ele pe amândoi!
Gen 31:38  Iata, douazeci de ani am stat la tine: oile tale ?i caprele tale n-au lepadat; berbecii oilor tale nu ?i i-am mâncat,
Gen 31:39  Vite sfâ?iate de fiare nu ?i-am adus: acestea au fost paguba mea. Din mâna mea ai cerut ceea ce se furase în timpul zilei ?i în vremea nop?ii.
Gen 31:40  Ziua eram mistuit de caldura, iar noaptea de frig ?i somnul nu se lipea de ochii mei.
Gen 31:41  A?a mi-au fost cei douazeci de ani în casa ta. ?i-am slujit paisprezece ani pentru cele doua fete ale tale ?i ?ase ani pentru vitele tale, iar tu de zeci de ori mi-ai schimbat simbria.
Gen 31:42  De n-ar fi fost cu mine Dumnezeul tatalui meu, Dumnezeul lui Avraam ?i frica de Isaac, tu acum m-ai fi alungat cu nimic. Necazul meu ?i munca mâinilor mele le-a vazut Dumnezeu ?i de aceea a mijlocit ieri pentru mine".
Gen 31:43  Raspuns-a Laban ?i a zis catre Iacov: "Aceste fete sunt fetele mele, ace?ti copii sunt copiii mei, aceste vite sunt vitele mele, ?i toate câte le vezi sunt ale mele ?i ale fetelor mele. Cum dar a? fi putut eu sa fac astazi ceva împotriva lor, sau împotriva copiilor, pe care i-au nascut ele?
Gen 31:44  Haidem dar acum sa facem amândoi, eu ?i tu, legamânt, care sa fie marturie între mine ?i tine!" Iar Iacov i-a zis: "Iata, nu e nimeni cu noi; dar sa ?tii ca Dumnezeu este martor între mine ?i tine".
Gen 31:45  ?i a luat Iacov o piatra ?i a pus-o stâlp.
Gen 31:46  Apoi a zis Iacov catre fra?ii sai: "Aduna?i pietre!" ?i au adunat pietre ?i au facut o movila; ?i au mâncat ?i au baut acolo pe movila. Apoi a zis Laban catre dânsul: "Movila aceasta este astazi marturie între mine ?i între tine".
Gen 31:47  ?i Laban a numit-o în limba sa: Iegar-Sahaduta, adica movila marturiei, iar Iacov i-a dat acela?i nume, însa pe limba sa ?i i-a zis: Galaad.
Gen 31:48  Apoi Laban a zis iara?i catre Iacov: "Iata, movila aceasta ?i semnul ce am pus astazi sunt marturia legamântului dintre mine ?i tine". De aceea i s-a pus ?i numele Galaad, adica movila marturiei.
Gen 31:49  Ba s-a mai numit ea ?i Mi?pa, adica veghere, pentru ca Laban a zis: "Sa vegheze Domnul asupra mea ?i asupra ta, dupa ce ne vom despar?i unul de altul.
Gen 31:50  De te vei purta rau cu fetele mele, sau de-?i vei mai lua ?i alte femei, afara de fetele mele, nu mai e vorba de un om, care sa vada, ci ia aminte ca între mine ?i între tine e martor Dumnezeu!"
Gen 31:51  ?i iara?i a zis Laban catre Iacov: "Iata movila aceasta ?i stâlpul, pe care l-am pus între amândoi, este marturie între mine ?i tine.
Gen 31:52  Ca nici eu nu voi trece spre tine ?i nici tu nu vei trece spre mine, de la aceasta movila, cu gând rau.
Gen 31:53  Dumnezeul lui Avraam ?i Dumnezeul lui Nahor, Dumnezeul parin?ilor lor sa fie judecator între noi!" Iar Iacov a jurat pe Acela, de Care se temea Isaac, tatal sau.
Gen 31:54  Apoi a junghiat Iacov ardere de tot pe munte ?i a chemat pe rudele sale sa manânce pâine. ?i au mâncat pâine ?i s-au veselit în munte.
Gen 31:55  Iar a doua zi s-a sculat Laban dis-de-diminea?a ?i a sarutat pe nepo?ii sai ?i pe fetele sale ?i i-a binecuvântat. Apoi Laban a pornit sa se întoarca la locul sau.
Gen 32:1  Dupa aceea Iacov s-a dus în calea sa. ?i cautând, el a vazut o?tirea lui Dumnezeu tabarâta, caci l-au întâmpinat îngerii lui Dumnezeu.
Gen 32:2  Iacov însa, când i-a vazut, a zis: "Aceasta este tabara lui Dumnezeu!" ?i a pus locului aceluia numele Mahanaim, adica doua tabere.
Gen 32:3  Apoi a trimis Iacov soli înaintea sa, la fratele sau Isav, în ?inutul Seir din ?ara Edomului,
Gen 32:4  ?i le-a poruncit, zicând: "A?a sa zice?i catre domnul meu Isav: A?a graie?te robul tau Iacov: Am stat la Laban ?i am trait la el pâna acum.
Gen 32:5  Am boi ?i asini, oi, slugi ?i slujnice, ?i am trimis sa vesteasca pe domnul meu Isav, ca sa afle robul tau bunavoin?a înaintea ta".
Gen 32:6  ?i întorcându-se la Iacov, i-au spus solii: "Am fost la fratele tau Isav ?i iata el vine în întâmpinarea ta cu patru sute de oameni".
Gen 32:7  Iacov însa s-a spaimântat foarte ?i nu ?tia ce sa faca. ?i a împar?it oamenii, care erau cu el, boii, oile ?i camilele în doua tabere.
Gen 32:8  ?i a zis Iacov: "De va navali Isav asupra unei tabere ?i o va bate, va scapa cealalta tabara".
Gen 32:9  Apoi Iacov a zis: "Dumnezeul tatalui meu Avraam ?i Dumnezeul tatalui meu Isaac, Doamne, Tu, Cel ce mi-ai zis: Întoarce-te în ?ara ta de na?tere, ?i Eu î?i voi face bine,
Gen 32:10  Nu sunt vrednic de toate îndurarile Tale ?i de toate binefacerile ce mi-ai aratat mie, robului Tau, ca numai cu toiagul am trecut deunazi Iordanul acesta, iar acum am doua tabere;
Gen 32:11  Izbave?te-ma dar din mâna fratelui meu, din mâna lui Isav, caci ma tem de el, ca nu cumva sa vina ?i sa ma omoare pe mine ?i pe aceste mame cu copii.
Gen 32:12  Caci Tu ai zis: î?i voi face bine ?i voi înmul?i neamul tau ca nisipul marii, cât nu se va putea numara din pricina mul?imii".
Gen 32:13  ?i a ramas acolo în noaptea aceea. Apoi a luat din cele ce avea ?i a trimis dar fratelui sau Isav:
Gen 32:14  Doua sute de capre ?i douazeci de ?api, doua sute de oi ?i douazeci de berbeci,
Gen 32:15  Treizeci de camile mulgatoare cu mânjii lor, patruzeci de vaci ?i zece tauri, douazeci de asine ?i zece asini.
Gen 32:16  ?i a dat fiecare din aceste turme deosebi în seama slugilor sale ?i a zis slugilor sale: "Trece?i înaintea mea ?i sa fie departate turmele una de alta".
Gen 32:17  Celui dintâi i-a poruncit, zicând: "Când te va întâlni fratele meu Isav ?i te va întreba: Al cui e?ti tu ?i unde te duci, ?i ale cui sunt acestea, ce merg înaintea ta,
Gen 32:18  Sa zici: Ale robului tau Iacov; e dar trimis lui Isav, stapânul meu. Iata vine ?i el dupa noi!"
Gen 32:19  A?a a poruncit Iacov ?i slugii celei de a doua ?i celei de a treia ?i tuturor celor ce mergeau cu turmele, zicând: "A?a sa spune?i lui Isav, când îl ve?i întâlni.
Gen 32:20  ?i sa-i mai spune?i: Iata ?i el, robul tau Iacov, vine dupa noi". Caci î?i zicea: Voi îmblânzi fa?a lui cu darurile ce-mi merg înainte ?i numai dupa aceea voi vedea fa?a lui, ?i a?a poate ma va primi.
Gen 32:21  ?i au pornit darurile înaintea lui, iar el a ramas noaptea aceea în tabara.
Gen 32:22  Dar s-a sculat noaptea ?i luând pe cele doua femei ale sale ?i pe cele doua roabe ?i pe cei unsprezece copii ai sai, a trecut Iabocul prin vad.
Gen 32:23  Iar dupa ce i-a luat ?i i-a trecut râul, a trecut ?i toate ale sale.
Gen 32:24  Ramânând Iacov singur, s-a luptat Cineva cu dânsul pâna la revarsatul zorilor.
Gen 32:25  Vazând însa ca nu-l poate rapune Acela, S-a atins de încheietura coapsei lui ?i i-a vatamat lui Iacov încheietura coapsei, pe când se lupta cu el.
Gen 32:26  ?i i-a zis: "Lasa-Ma sa plec, ca s-au ivit zorile!" Iacov I-a raspuns: "Nu Te las pâna nu ma vei binecuvânta".
Gen 32:27  ?i l-a întrebat Acela: "Care î?i este numele?" ?i el a zis: "Iacov!"
Gen 32:28  Zisu-i-a Acela: "De acum nu-?i va mai fi numele Iacov, ci Israel te vei numi, ca te-ai luptat cu Dumnezeu ?i cu oamenii ?i ai ie?it biruitor!"
Gen 32:29  ?i a întrebat ?i Iacov, zicând: "Spune-mi ?i Tu numele Tau!" Iar Acela a zis: "Pentru ce întrebi de numele Meu? El e minunat!" ?i l-a binecuvântat acolo.
Gen 32:30  ?i a pus Iacov locului aceluia numele Peniel, adica fa?a lui Dumnezeu, caci ?i-a zis: "Am vazut pe Dumnezeu în fa?a ?i mântuit a fost sufletul meu! "
Gen 32:31  Iar când rasarea soarele, trecuse de Peniel, dar el ?chiopata din pricina ?oldului.
Gen 32:32  De aceea fiii lui Israel pâna astazi nu manânca mu?chiul de pe ?old, pentru ca Cel ce S-a luptat a atins încheietura ?oldului lui Iacov, în dreptul acestui mu?chi.
Gen 33:1  Atunci, ridicându-?i ochii, Iacov a vazut pe Isav, fratele sau, venind cu cei patru sute de oameni. ?i a împar?it Iacov copiii Liei ?i ai Rahilei ?i ai celor doua roabe.
Gen 33:2  ?i a pus pe cele doua roabe cu copiii lor înainte; apoi dupa ei a pus pe Lia cu copiii ei ?i la urma a pus pe Rahila ?i pe Iosif;
Gen 33:3  Iar el mergea în fruntea lor ?i, apropiindu-se de fratele sau, i s-a închinat de ?apte ori pâna la pamânt.
Gen 33:4  Isav însa a alergat în întâmpinarea lui ?i l-a îmbra?i?at ?i, cuprinzându-i grumazul, l-a sarutat ?i au plâns amândoi.
Gen 33:5  Apoi, ridicându-?i ochii ?i vazând femeile ?i copiii, Isav a zis: "Cine sunt ace?tia?" Zis-a Iacov: "Copiii cu care a miluit Dumnezeu pe robul tau!"
Gen 33:6  Deci, s-au apropiat întâi roabele cu copiii lor ?i s-au închinat.
Gen 33:7  Apoi a venit Lia cu copiii ei ?i s-au închinat, iar la urma au venit ?i s-au închinat ?i Rahila cu Iosif.
Gen 33:8  Zis-a Isav: "Ce sunt acele turme, pe care le-am întâlnit?" Iar Iacov a raspuns: "Ca sa afle robul tau bunavoin?a înaintea domnului meu".
Gen 33:9  Atunci Isav a zis: "Am ?i eu multe, frate; ?ine-?i ale tale pentru tine!"
Gen 33:10  Iacov însa a zis: "De am aflat bunavoin?a înaintea ta, prime?te darurile din mâinile mele, caci, când am vazut fa?a ta, parca a? fi vazut fa?a lui Dumnezeu, a?a de binevoitor mi-ai fost.
Gen 33:11  Prime?te de la mine binecuvântarile mele, pe care ?i le aduc, ca m-a miluit Dumnezeu ?i am de toate". ?i a staruit ?i le-a luat.
Gen 33:12  Apoi a zis Isav: "Sa ne sculam ?i sa mergem împreuna; eu însa îmi voi potrivi pasul cu tine".
Gen 33:13  Iacov însa a raspuns: "Domnul meu ?tie ca îmi sunt ginga?i copiii ?i ca am oi ?i vite de curând fatate; de le vom mâna tare numai o zi, ar pieri toata turma.
Gen 33:14  Sa se duca dar domnul meu, înaintea robului sau, iar eu voi urma încet, în pas cu vitele cele dinaintea mea ?i în pas cu copiii, pâna voi ajunge la domnul meu în Seir".
Gen 33:15  Atunci Isav a zis: "Sa-?i las macar o parte din oamenii cei ce sunt cu mine". Iar Iacov i-a raspuns: "La ce aceasta? Mi-ajunge mie bunavoin?a ce-am aflat înaintea domnului meu".
Gen 33:16  ?i s-a întors Isav în aceea?i zi pe calea sa la Seir.
Gen 33:17  Iar Iacov s-a îndreptat spre Sucot ?i ?i-a facut acolo locuin?a pentru sine, iar pentru vitele sale a facut ?uri; de aceea a pus el numele locului aceluia Sucot.
Gen 33:18  Întorcându-se Iacov din Mesopotamia ?i ajungând cu bine la Salem, o cetate în ?inutul Sichem, din pamântul Canaan, s-a a?ezat în fa?a ceta?ii.
Gen 33:19  Apoi ?i-a cumparat partea de câmp, pe care era cortul sau, cu o suta de kesite, de la fiii lui Hemor, tatal lui Sichem.
Gen 33:20  A înal?at acolo un jertfelnic ?i i-a pus numele El-Elohe-Israel.
Gen 34:1  Într-o zi, Dina, fata Liei, pe care aceasta o nascuse lui Iacov, a ie?it sa vada fetele ?arii aceleia.
Gen 34:2  ?i vazând-o Sichem, feciorul lui Hemor Heveul, stapânitorul pamântului aceluia, a luat-o ?i, culcându-se cu ea, a necinstit-o.
Gen 34:3  Apoi s-a lipit sufletul lui de Dina, fata lui Iacov, ?i i-a cazut draga fata ?i a vorbit pe placul fetei.
Gen 34:4  ?i a zis Sichem catre tatal sau Hemor: "Ia-mi pe fata aceasta de femeie!"
Gen 34:5  De?i Iacov a auzit ca fiul lui Hemor a necinstit pe Dina, fata sa, dar, fiindca feciorii lui erau cu vitele la câmp, a tacut pâna s-au întors ei.
Gen 34:6  Iar Hemor, tatal lui Sichem, a ie?it la Iacov, ca sa vorbeasca cu el.
Gen 34:7  Feciorii lui Iacov însa, venind de la câmp ?i aflând despre aceasta, se amarâra ?i se mâniara foarte tare, pentru ca Sichem savâr?ise o fapta de ocara în Israel, culcându-se cu fata lui Iacov, ceea ce nu trebuia sa se întâmple.
Gen 34:8  ?i graind cu ei, Hemor a zis: "Sichem, feciorul meu, s-a lipit cu sufletul de fata voastra; da?i-o dar lui de femeie ?i va încuscri?i cu noi:
Gen 34:9  Marita?i-va fetele voastre cu noi ?i fetele noastre lua?i-le pentru feciorii vo?tri;
Gen 34:10  ?ede?i la un loc cu noi: acest pamânt larg va e la îndemâna, ca sa va a?eza?i într-însul, sa face?i nego? ?i sa va agonisi?i din el mo?ie".
Gen 34:11  Iar Sichem a zis catre tatal fetei ?i catre fra?ii ei: "Orice ve?i zice, voi da, numai sa aflu bunavoin?a la voi.
Gen 34:12  Cere?i de la mine un mare pre? de cumparare ?i darurile cele mai mari ?i va voi da cât ve?i zice, numai da?i-mi fata mie de femeie!"
Gen 34:13  Feciorii lui Iacov însa au raspuns cu vicle?ug lui Sichem ?i lui Hemor, tatal lui; ?i le-au raspuns a?a, pentru ca acela necinstise pe Dina, sora lor.
Gen 34:14  ?i au zis catre dân?ii Simeon ?i Levi, fra?ii Dinei ?i feciorii Liei: "Nu putem sa facem aceasta: sa dam pe sora noastra dupa un om netaiat împrejur, ca aceasta ar fi o ru?ine pentru noi.
Gen 34:15  Numai a?a ne învoim cu voi ?i ne a?ezam la voi, daca ve?i face ?i voi ca noi, taindu-va împrejur to?i cei de parte barbateasca.
Gen 34:16  Atunci vom da dupa voi fetele noastre, iar noi vom lua fetele voastre ?i vom locui la un loc cu voi ?i vom alcatui un popor.
Gen 34:17  Iar de nu vre?i sa ne asculta?i, ca sa va taia?i împrejur, noi vom lua înapoi fata ?i ne vom duce".
Gen 34:18  Vorbele acestea au placut lui Hemor ?i lui Sichem, feciorul lui Hemor.
Gen 34:19  De aceea, n-a zabovit tânarul sa faca aceasta, caci era îndragostit de fata lui Iacov ?i era ?i cel mai cu trecere în casa tatalui sau.
Gen 34:20  ?i au venit Hemor ?i Sichem, feciorul lui, la poarta ceta?ii lor, ?i au început a grai locuitorilor ceta?ii, zicând:
Gen 34:21  "Oamenii ace?tia sunt pa?nici; sa se a?eze dar în ?ara noastra ?i sa faca nego? în ea. Iata ca loc este din destul ?i într-o parte ?i într-alta; fetele lor sa ni le luam de femei ?i fetele noastre sa le dam dupa ei.
Gen 34:22  Dar oamenii ace?tia numai a?a se învoiesc sa traiasca cu noi ?i sa fie un popor cu noi, daca ?i la noi se vor taia împrejur to?i cei de parte barbateasca, cum sunt ei taia?i împrejur.
Gen 34:23  Turmele lor, vitele lor ?i toate averile lor nu sunt, oare, ale noastre? Sa le plinim voia lor, iar ei sa se a?eze printre noi!"
Gen 34:24  ?i au ascultat pe Hemor ?i pe Sichem, feciorul lui, to?i cei ce ie?eau pe poarta ceta?ii lor ?i au fost taia?i împrejur to?i cei de parte barbateasca, câ?i ie?eau pe poarta ceta?ii lor.
Gen 34:25  Iar a treia zi, când erau ei înca în dureri, cei doi fiii ai lui Iacov, Simeon ?i Levi, fra?ii Dinei, ?i-au luat fiecare sabia ?i au intrat fara teama în cetate ?i au ucis pe to?i cei de parte barbateasca.
Gen 34:26  Au trecut prin ascu?i?ul sabiei ?i pe Hemor ?i pe fiul sau Sichem ?i au luat pe Dina din casa lui Sichem ?i au plecat.
Gen 34:27  Apoi fiii lui Iacov se napustira asupra celor mor?i ?i jefuira cetatea în care fusese necinstita Dina, sora lor.
Gen 34:28  Au luat toate oile lor, to?i boii lor, to?i asinii lor, tot ce era în cetate ?i tot ce era pe câmp;
Gen 34:29  Toate boga?iile lor, to?i copiii ?i femeile le-au dus în robie; ?i au jefuit tot ce era în cetate ?i tot ce era prin case.
Gen 34:30  Atunci Iacov a zis catre Simeon ?i catre Levi: "Mare tulburare mi-a?i adus, facându-ma urât înaintea tuturor locuitorilor ?arii acesteia, înaintea Canaaneilor ?i a Ferezeilor. Eu am oameni pu?ini la numar; se vor ridica asupra mea ?i ma vor ucide ?i voi pieri ?i eu ?i casa mea".
Gen 34:31  Iar ei au zis: "Dar se putea, oare, ca ei sa se poarte cu sora noastra ca ?i cu o femeie pierduta?"
Gen 35:1  Atunci a zis Dumnezeu lui Iacov: "Scoala ?i du-te la Betel ?i locuie?te acolo; fa acolo jertfelnic Dumnezeului Celui ce ?i S-a aratat, când fugeai tu de la fa?a lui Isav, fratele tau!"
Gen 35:2  Iar Iacov a zis casei sale ?i tuturor celor ce erau cu dânsul: "Lepada?i dumnezeii cei straini, care se afla la voi, cura?i?i-va ?i va primeni?i hainele voastre.
Gen 35:3  Sa ne sculam ?i sa mergem la Betel, ca acolo am sa fac jertfelnic lui Dumnezeu, Celui ce m-a auzit în ziua necazului meu ?i Care a fost cu mine ?i m-a pazit în calatoria în care am umblat!"
Gen 35:4  Iar ei au dat lui Iacov to?i dumnezeii cei straini, care erau în mâinile lor, ?i cerceii ce-i aveau în urechile lor; ?i Iacov i-a îngropat sub stejarul de lânga Sichem ?i i-a lasat necunoscu?i pâna în ziua de astazi.
Gen 35:5  Astfel au plecat ei de la Sichem; ?i frica lui Dumnezeu era peste ora?ele dimprejur ?i n-au urmarit pe fiii lui Iacov.
Gen 35:6  Sosind Iacov cu to?i oamenii cei ce erau cu el la Luz, adica la Betel, în ?ara Canaanului,
Gen 35:7  A zidit acolo un jertfelnic ?i a numit locul acela El-Bet-El, pentru ca acolo i Se aratase Dumnezeu, când fugea el de Isav, fratele sau.
Gen 35:8  Atunci a murit Debora, doica Rebecai, ?i a fost îngropata mai jos de Betel, sub un stejar, pe care Iacov l-a numit "Stejarul Plângerii".
Gen 35:9  Aici, în Luz, Se mai arata Dumnezeu lui Iacov, dupa întoarcerea lui din Mesopotamia, ?i îl binecuvânta Dumnezeu,
Gen 35:10  ?i-i zise: "De acum nu te vei mai chema Iacov, ci Israel va fi numele tau". ?i-i puse numele Israel.
Gen 35:11  Apoi Dumnezeu îi mai zise: "Eu sunt Dumnezeul cel Atotputernic! Spore?te ?i te înmul?e?te! Popoare ?i mul?ime de neamuri se vor na?te din tine ?i regi vor rasari din coapsele tale.
Gen 35:12  ?ara, pe care am dat-o lui Avraam ?i lui Isaac, o voi da ?ie; iar dupa tine, voi da pamântul acesta urma?ilor tai".
Gen 35:13  Apoi S-a înal?at Dumnezeu de la el, din locul în care îi graise.
Gen 35:14  ?i a a?ezat Iacov un stâlp pe locul unde-i graise Dumnezeu, un stâlp de piatra, ?i a savâr?it turnare peste el ?i a turnat peste el untdelemn.
Gen 35:15  ?i a pus Iacov locului unde-i graise Dumnezeu, numele Betel.
Gen 35:16  Dupa aceea au plecat din Betel. ?i ?i-a întins cortul sau dincolo de turnul Gader. Dar când se apropiase de Havrata, înainte de a intra în Efrata, Rahila a nascut ?i na?terea aceasta a fost iar tare grea.
Gen 35:17  ?i pe când se chinuia Rahila în durerile na?terii, moa?a i-a zis: "Nu te teme, ca ?i acesta va fi baiat!"
Gen 35:18  Iar când Rahila î?i dadea sufletul, caci a murit, a pus copilului numele Ben-Oni, adica fiul durerii mele, iar tatal lui l-a numit Veniamin.
Gen 35:19  Iar daca a murit, Rahila a fost îngropata lânga calea ce duce la Efrata, adica la Betleem;
Gen 35:20  Iacov a ridicat un stâlp de piatra pe mormântul ei ?i acest stâlp, de pe mormântul Rahilei, este pâna în ziua de astazi.
Gen 35:21  Apoi plecând Iacov de aici ?i-a întins cortul dincolo de turnul Migdal-Eder. Iar pe vremea când locuia Israel în ?ara aceasta, a intrat Ruben ?i a dormit cu Bilha, ?iitoarea tatalui sau Iacov, ?i a auzit Israel ?i i s-a parut aceasta un rau.
Gen 35:22  Fiii lui Iacov au fost doisprezece ?i anume:
Gen 35:23  Fiii Liei: Ruben, întâi-nascutul lui Iacov; dupa el veneau: Simeon, Levi, Iuda, Isahar ?i Zabulon.
Gen 35:24  Fiii Rahilei: Iosif ?i Veniamin.
Gen 35:25  Fiii slujnicei Rahilei, Bilha: Dan ?i Neftali.
Gen 35:26  ?i fiii Zilpei, roaba Liei: Gad ?i A?er. Ace?tia sunt fiii lui Iacov, care i s-au nascut în Mesopotamia.
Gen 35:27  Apoi a sosit Iacov la Isaac, tatal sau, caci acesta traia înca la Mamvri, în Chiriat-Arba, adica la Hebron în pamântul Canaanului, unde locuisera vremelnic Avraam ?i Isaac.
Gen 35:28  Iar zilele, pe care le-a trait Isaac, au fost o suta optzeci de ani.
Gen 35:29  Slabind apoi, Isaac a murit ?i a trecut la parin?ii sai, fiind batrân ?i încarcat de zile, ?i l-au îngropat feciorii lui, Isav ?i Iacov.
Gen 36:1  Iar spi?a neamului lui Isav, care se mai nume?te ?i Edom, este aceasta:
Gen 36:2  Isav ?i-a luat femei din fetele Canaaneilor: pe Ada, fata lui Elon Heteul, ?i pe Olibama, fata lui Ana, fiul lui ?ibon Heveul,
Gen 36:3  ?i pe Basemata, fata lui Ismael ?i sora lui Nebaiot.
Gen 36:4  Ada a nascut lui Isav pe Elifaz; Basemata i-a nascut pe Raguel;
Gen 36:5  Iar Olibama i-a nascut pe Ieu?, pe Ialam ?i pe Core. Ace?tia sunt fiii lui Isav, care i s-au nascut în ?ara Canaanului.
Gen 36:6  Dupa aceea ?i-a luat Isav femeile sale, fiii sai, fetele sale, to?i oamenii casei sale, toate averile sale, toate vitele sale ?i toate câte avea ?i toate câte agonisise în ?ara Canaanului, ?i a plecat Isav din Canaan din pricina lui Iacov, fratele sau,
Gen 36:7  Pentru ca averile lor erau multe ?i nu mai puteau sa locuiasca la un loc, ?i pamântul unde erau nu-i mai putea încapea din pricina mul?imii turmelor lor.
Gen 36:8  Astfel Isav, care se mai nume?te ?i Edom, s-a mutat în muntele Seir.
Gen 36:9  Iata acum ?i urma?ii ce i s-au nascut lui Isav, parintele Edomi?ilor, dupa mutarea sa în muntele Seir.
Gen 36:10  Numele fiilor lui Isav sunt acestea: Elifaz, fiul Adei, solia lui Isav ?i Raguel, fiul Basematei, so?ia lui Isav.
Gen 36:11  Elifaz, a avut cinci feciori: Teman, Omar, ?efo, Gatam ?i Chenaz;
Gen 36:12  Iar Timna, o ?iitoare a lui Elifaz, fiul lui Isav, i-a nascut lui Elifaz pe Amalec. Ace?tia sunt urma?ii din Ada, femeia lui Isav.
Gen 36:13  Iar feciorii lui Raguel sunt ace?tia: Nahat ?i Zerah, ?ama ?i Miza. Ace?tia sunt urma?ii din Basemata, femeia lui Isav.
Gen 36:14  Iar feciorii Olibamei, femeia lui Isav ?i fiica lui Ana a lui ?ibon, sunt ace?tia: ea a nascut lui Isav pe Ieu?, pe Ialam ?i pe Core.
Gen 36:15  Iata ?i capeteniile fiilor lui Isav: feciorii lui Elifaz, întâi-nascutul lui Isav, sunt: capetenia Teman, capetenia Omar, capetenia ?efo, capetenia Chenaz,
Gen 36:16  Capetenia Core, capetenia Gatam ?i capetenia Amalec. Acestea sunt capeteniile din Elifaz în ?ara Edomului; ace?tia sunt urma?ii din Ada.
Gen 36:17  Iar fiii lui Raguel, fiul lui Isav, sunt: capetenia Nahat, capetenia Zerah, capetenia ?ama ?i capetenia Miza. Acestea sunt capeteniile din Raguel în ?ara Edomului; ace?tia sunt urma?ii din Basemata, femeia lui Isav.
Gen 36:18  Iata ?i fiii Olibamei, femeia lui Isav: capetenia Ieu?, capetenia Ialam ?i capetenia Core. Acestea sunt capeteniile din Olibama, fata lui Ana ?i femeia lui Isav.
Gen 36:19  Ace?tia sunt fiii lui Isav ?i acestea sunt capeteniile lor. Acesta este Edom.
Gen 36:20  Iar feciorii lui Seir Horeeanul, care locuiau înainte pamântul acela, sunt: Lotan, ?obal, ?ibon ?i Ana;
Gen 36:21  Di?on, E?er ?i Di?an. Acestea sunt capeteniile Horeilor, feciorii lui Seir, în pamântul Edomului.
Gen 36:22  Fiii lui Lotan sunt: Hori ?i Heman, iar sora lui Lotan a fost Timna.
Gen 36:23  Fiii lui ?obal sunt: Alvan, Manahat, Ebal, ?efo ?i Onam.
Gen 36:24  Fiii lui ?ibon sunt: Aia ?i Ana. Acesta este acel Ana, care a gasit izvoarele de apa calda în pustie, când pa?tea asinii tatalui sau ?ibon.
Gen 36:25  Copiii lui Ana sunt: Di?on ?i Olibama, fata lui Ana.
Gen 36:26  Copiii lui Di?on sunt: Hemdan, E?ban, Itran ?i Cheran.
Gen 36:27  Fiii lui E?er sunt: Bilhan, Zaavan ?i Acan.
Gen 36:28  Fiii lui Di?an sunt: U? ?i Aran.
Gen 36:29  Deci capeteniile Horeilor sunt acestea: capetenia Lotan, capetenia ?obal, capetenia ?ibon, capetenia Ana,
Gen 36:30  Capetenia Di?on, capetenia E?er, capetenia Di?an. Acestea sunt capeteniile Horeilor din ?ara lui Seir, dupa familiile lor.
Gen 36:31  Iata ?i regii, care au domnit în pamântul Edomului înainte de a domni vreun rege peste fiii lui Israel:
Gen 36:32  În Edom a domnit mai întâi Bela, fiul lui Beor ?i cetatea lui se numea Dinhaba.
Gen 36:33  Dupa ce a murit Bela, s-a facut rege Iobab, fiul lui Zerah din Bo?ra.
Gen 36:34  Dupa ce a murit Iobab, s-a facut rege Hu?am, din ?ara Temani?ilor.
Gen 36:35  Dupa ce a murit Hu?am, s-a facut rege Hadad, feciorul lui Bedad; care a batut pe Madiani?i în câmpul Moab; numele ceta?ii lui era Avit.
Gen 36:36  Iar dupa ce a murit Hadad, s-a facut rege Samla, din Masreca.
Gen 36:37  Iar dupa ce a murit Samla, s-a facut rege, în locul lui, ?aul, din Rehobotul de pe râu.
Gen 36:38  Iar dupa ce a murit ?aul, s-a facut rege Baal-Hanan, fiul lui Acbor.
Gen 36:39  Iar dupa ce a murit Baal-Hanan, feciorul lui Acbor, s-a facut rege Hadar, fiul lui Varad, ?i numele ceta?ii lui era Pau ?i al femeii lui Mehetabel, fiica lui Matred, feciorul lui Mezahab.
Gen 36:40  Iar numele capeteniilor din Isav, dupa triburile lor, dupa ?arile lor, dupa numirile ?i na?iile lor sunt: capetenia Timna, capetenia Alvan, capetenia Ietet,
Gen 36:41  Capetenia Olibama, capetenia Ela, capetenia Pinon,
Gen 36:42  Capetenia Chenaz, capetenia Teman, capetenia Mib?ar,
Gen 36:43  Capetenia Magdiel, capetenia Iram. Acestea sunt capeteniile lui Edom, dupa a?ezarile lor în ?ara stapânita de ei. Acesta-i Isav, parintele Edomi?ilor.
Gen 37:1  Iacov a locuit în ?ara Canaan, unde locuise ?i Isaac, tatal sau.
Gen 37:2  Iata acum ?i istoria urma?ilor lui Iacov: Iosif, fiind de ?aptesprezece ani, pa?tea oile tatalui sau împreuna cu fra?ii sai. Petrecându-?i copilaria cu feciorii Bilhai ?i cu feciorii Zilpei, femeile tatalui sau, Iosif aducea lui Israel, tatal sau, ve?ti despre purtarile lor rele.
Gen 37:3  ?i iubea Israel pe Iosif mai mult decât pe to?i ceilal?i fii ai sai, pentru ca el era copilul batrâne?ilor lui, ?i-i facuse haina lunga ?i aleasa.
Gen 37:4  Fra?ii lui, vazând ca tatal lor îl iubea mai mult decât pe to?i fiii sai, îl urau ?i nu puteau vorbi cu el prietenos.
Gen 37:5  Visând însa Iosif un vis, l-a spus fra?ilor sai,
Gen 37:6  Zicându-le: "Asculta?i visul ce am visat:
Gen 37:7  Parca legam snopi în ?arina ?i snopul meu parca s-a sculat ?i statea drept, iar snopii vo?tri s-au strâns roata ?i s-au închinat snopului meu".
Gen 37:8  Iar fra?ii lui au zis catre el: "Nu cumva ai sa domne?ti peste noi? Sau poate ai sa ne stapâne?ti?" ?i l-au urât înca ?i mai mult pentru visul lui ?i pentru spusele lui.
Gen 37:9  ?i a mai visat el alt vis ?i l-a spus tatalui sau ?i fra?ilor sai, zicând: "Iata am mai visat alt vis: soarele ?i luna ?i unsprezece stele mi se închinau mie".
Gen 37:10  ?i-l povesti tatalui sau ?i fra?ilor sai, iar tatal sau l-a certat ?i i-a zis: "Ce înseamna visul acesta pe care l-ai visat? Au doara eu ?i mama ta ?i fra?ii tai vom veni ?i ne vom închina ?ie pâna la pamânt?"
Gen 37:11  De aceea îl pizmuiau fra?ii lui, iar tatal sau pastra cuvintele acestea în inima lui.
Gen 37:12  S-au dus dupa aceea fra?ii lui sa pasca oile tatalui lor la Sichem.
Gen 37:13  ?i Israel a zis catre Iosif: "Fra?ii tai pasc oile la Sichem. Vino, dar, sa te trimit la ei". Iar el a zis: "Ma duc, tata!"
Gen 37:14  Apoi Israel a zis catre Iosif: "Du-te ?i vezi de sunt sanato?i fra?ii tai ?i oile ?i sa-mi aduci raspuns!" L-a trimis astfel din valea Hebronului ?i Iosif s-a dus la Sichem.
Gen 37:15  Dupa aceea l-a gasit un om ratacind pe câmp ?i l-a întrebat omul acela ?i i-a zis: "Ce cau?i?"
Gen 37:16  Iar el a zis: "Caut pe fra?ii mei. Spune-mi, unde pasc ei oile?"
Gen 37:17  Zisu-i-a omul acela: "S-au dus de aici, caci i-am auzit zicând: Haidem la Dotain!" ?i s-a dus Iosif dupa fra?ii sai ?i i-a gasit la Dotain.
Gen 37:18  Iar ei, vazându-l de departe, pâna a nu se apropia de ei, au început a unelti asupra lui sa-l omoare;
Gen 37:19  ?i au zis unii catre al?ii: "Iata visatorul acela de vise vine!
Gen 37:20  Haidem sa-l omorâm, sa-l aruncam într-un pu? ?i sa zicem ca l-a mâncat o fiara salbatica ?i vom vedea ce se va alege de visele lui!"
Gen 37:21  Auzind însa aceasta, Ruben a voit sa-l scape din mâinile lor, zicând: "Sa nu-i ridicam via?a!"
Gen 37:22  Apoi Ruben a adaugat: "Sa nu varsa?i sânge! Arunca?i-l în pu?ul acela din pustie, dar mâinile sa nu vi le pune?i pe el!" Iar aceasta o zicea el cu gândul de a-l scapa din mâinile lor ?i a-l trimite acasa la tatal sau.
Gen 37:23  Când însa a sosit Iosif la fra?ii sai, ei au dezbracat pe Iosif de haina cea lunga ?i aleasa, cu care era îmbracat,
Gen 37:24  ?i l-au luat ?i l-au aruncat în pu?; dar pu?ul era gol ?i nu avea apa.
Gen 37:25  Dupa aceea ?ezând sa manânce pâine ?i cautând cu ochii lor, ei au vazut venind dinspre Galaad o caravana de Ismaeli?i, ale caror camile erau încarcate cu tamâie, eu balsam ?i cu smirna, pe care le duceau în Egipt.
Gen 37:26  Atunci a zis Iuda catre fra?ii sai: "Ce vom folosi de vom ucide pe fratele nostru ?i vom ascunde sângele lui?
Gen 37:27  Haidem sa-l vindem Ismaeli?ilor acestora, neridicându-ne mâinile asupra lui, pentru ca e fratele nostru ?i trupul nostru". ?i au ascultat fra?ii lui.
Gen 37:28  Iar când au trecut negustorii Madiani?i pe acolo, fra?ii au tras ?i au scos pe Iosif din pu? ?i l-au vândut pe el Ismaeli?ilor cu douazeci de argin?i. ?i ace?tia au dus pe Iosif în Egipt.
Gen 37:29  Când însa s-a întors Ruben la pu? ?i n-a vazut pe Iosif în pu?, el ?i-a rupt hainele,
Gen 37:30  ?i întorcându-se la fra?ii sai, a zis: "Baiatul nu este! Încotro sa apuc eu acum?"
Gen 37:31  Atunci ei au luat haina lui Iosif ?i, junghiind un ied, au muiat haina în sânge;
Gen 37:32  Apoi au trimis dupa haina cea lunga ?i aleasa ?i au adus-o la tatal lor, spunând. "Am gasit aceasta; vezi de este haina fiului tau sau nu!"
Gen 37:33  ?i a cunoscut-o Iacov ?i a zis: "Este haina fiului meu! L-a mâncat o fiara salbatica; o fiara l-a sfâ?iat pe Iosif!"
Gen 37:34  Atunci ?i-a rupt Iacov hainele sale ?i-a acoperit cu sac coapsele ?i a plâns pe fiul sau zile multe.
Gen 37:35  Dupa aceea s-au adunat to?i feciorii lui ?i toate fetele lui ?i au venit sa-l mângâie; dar el nu voia sa se mângâie, ci zicea: "Plângând, ma voi pogorî în locuin?a mor?ilor la fiul meu!" ?i-l plângea astfel tatal sau.
Gen 37:36  Iar Madiani?ii au vândut pe Iosif în Egipt lui Putifar, dregator ?i comandant al garzii la curtea lui Faraon.
Gen 38:1  În vremea aceea s-a întâmplat ca Iuda s-a pogorât de la fra?ii sai ?i s-a a?ezat lânga un adulamitean, cu numele Hira.
Gen 38:2  Vazând Iuda acolo pe fata unui canaaneu, care se numea ?ua, el a luat-o de so?ie ?i a intrat la ea.
Gen 38:3  ?i ea, zamislind, a nascut un baiat, ?i Iuda i-a pus numele Ir.
Gen 38:4  Zamislind iara?i, a nascut alt baiat ?i i-a pus numele Onan.
Gen 38:5  ?i a mai nascut un baiat ?i i-a pus numele ?ela. ?i când a nascut ea acest fiu, Iuda era la Kezib.
Gen 38:6  Apoi Iuda a luat pentru Ir, întâiul nascut al sau, o femeie, cu numele Tamara.
Gen 38:7  Dar Ir, întâiul nascut al lui Iuda, a fost rau înaintea Domnului ?i de aceea l-a omorât Domnul.
Gen 38:8  Atunci a zis Iuda catre Onan: "Intra la femeia fratelui tau, însoara-te cu ea, în puterea leviratului, ?i ridica urma?i fratelui tau!"
Gen 38:9  ?tiind însa Onan ca nu vor fi urma?ii ai lui, de aceea, când intra la femeia fratelui sau, el varsa samân?a jos, ca sa nu ridice urma?i fratelui sau.
Gen 38:10  Ceea ce facea el era rau înaintea lui Dumnezeu ?i l-a omorât ?i pe acesta.
Gen 38:11  Atunci a zis Iuda catre Tamara, nora sa, dupa moartea celor doi fii ai sai: "Stai vaduva în casa tatalui tau, pâna se va face mare ?ela, fiul meu!" Caci î?i zicea: "Nu cumva sa moara ?i acesta, ca ?i fra?ii lui!" ?i s-a dus Tamara ?i a trait în casa tatalui ei.
Gen 38:12  Trecând însa vreme multa, a murit fata lui ?ua, so?ia lui Iuda. Iar Iuda, dupa ce au trecut zilele de jelire, s-a dus în Timna, la cei ce tundeau oile lui, împreuna cu prietenul sau Hira adulamiteanul.
Gen 38:13  Atunci i s-a vestit Tamarei, nora sa, zicându-i-se: "Iata socrul tau merge la Timna sa-?i tunda oile".
Gen 38:14  Iar ea, dezbracând de pe sine hainele sale de vaduvie, s-a înfa?urat cu un val ?i, gatindu-se, a ie?it ?i a ?ezut la poarta Enaim, care este în drumul spre Timna, caci vedea ca ?ela crescuse mare ?i ea nu-i fusese data lui de so?ie.
Gen 38:15  ?i, vazând-o Iuda, a socotit ca este o femeie naravita, caci n-a cunoscut-o, pentru ca î?i avea fa?a acoperita.
Gen 38:16  ?i abatându-se din cale pe la ea, i-a zis: "Lasa-ma sa intru la tine!" Caci nu ?tia ca este nora sa. Iar ea a zis: "Ce ai sa-mi dai, daca vei intra la mine?"
Gen 38:17  Iar el i-a raspuns: "Î?i voi trimite un ied din turma mea". ?i ea a zis: "Bine, dar sa-mi dai ceva zalog pâna mi-l vei trimite".
Gen 38:18  Raspuns-a Iuda: "Ce zalog sa-?i dau?" ?i ea a zis: "Inelul tau, cingatoarea ta ?i toiagul ce-l ai în mâna". ?i el i le-a dat ?i a intrat la ea ?i ea a ramas grea.
Gen 38:19  Apoi, sculându-se, ea s-a dus ?i-a scos valul sau ?i s-a îmbracat iar cu hainele sale de vaduvie.
Gen 38:20  Iar Iuda a trimis iedul pe adulamitean, prietenul sau, ca sa ia zalogul din mâinile femeii. Dar n-a mai gasit-o.
Gen 38:21  ?i a întrebat pe oamenii locului aceluia: "Unde este femeia cea naravita, care ?edea la Enaim, la drum?" Iar aceia i-au raspuns: "N-a fost aici nici o femeie naravita!"
Gen 38:22  S-a întors deci acela la Iuda ?i a zis: "N-am gasit-o ?i oamenii de acolo mi-au spus ca n-a fost acolo nici o femeie naravita!"
Gen 38:23  Atunci Iuda a zis: "Sa ?i le ?ie! Numai de nu ne-ar face de batjocura. Iata, eu i-am trimis iedul, dar tu n-ai gasit-o".
Gen 38:24  Dar, cam dupa vreo trei luni, i s-a spus lui Iuda: "Tamara, nora ta, a cazut în desfrânare ?i iata a ramas însarcinata din desfrânare". Iar Iuda a zis: "Scoate?i-o ?i sa fie arsa".
Gen 38:25  Dar când o duceau, ea a trimis la socrul sau, zicând: "Eu sunt îngreunata de acela ale caruia sunt lucrurile acestea". Apoi a adaugat: "Afla al cui e inelul acesta, cingatoarea aceasta ?i toiagul acesta!"
Gen 38:26  ?i le-a cunoscut Iuda ?i a zis: "Tamara e mai dreapta decât mine, pentru ca nu am dat-o lui ?ela, fiul meu". ?i n-a mai cunoscut-o pe ea.
Gen 38:27  Iar când era sa nasca, s-a aflat ca are în pântece doi gemeni.
Gen 38:28  În vremea na?terii s-a ivit mâna unuia, iar moa?a a luat ?i i-a legat la mâna un fir de a?a ro?ie, zicând: "Acesta a ie?it întâi".
Gen 38:29  Dar acesta ?i-a tras mâna înapoi ?i îndata a ie?it fratele lui. ?i ea a zis: "Cum ai rupt tu piedica? Ruptura sa fie asupra ta!" ?i i-a pus numele Fares.
Gen 38:30  Dupa aceea a ie?it ?i fratele lui, cu firul de a?a ro?ie la mâna, ?i i s-a pus numele Zara.
Gen 39:1  Deci Iosif a fost dus în Egipt ?i din mâna Ismaeli?ilor, care l-au dus acolo, l-a cumparat egipteanul Putifar, o capetenie de la curtea lui Faraon ?i comandantul garzii lui.
Gen 39:2  Domnul însa era cu Iosif ?i el era om îndemânatic ?i traia în casa egipteanului, stapânul sau.
Gen 39:3  Stapânul sau vedea ca Domnul era cu dânsul ?i ca toate câte facea el, Domnul le sporea în mâna lui.
Gen 39:4  De aceea a aflat Iosif trecere înaintea stapânului sau ?i i-a placut ?i l-a pus peste casa sa ?i toate câte avea le-a dat pe mâna lui Iosif.
Gen 39:5  Iar dupa ce l-a pus peste casa sa ?i peste toate câte avea, a binecuvântat Domnul casa egipteanului pentru Iosif ?i era binecuvântarea Domnului peste tot ce avea el în casa ?i în ?arina sa.
Gen 39:6  ?i a lasat Putifar pe mâna lui Iosif tot ce avea ?i, de când îl avea pe el, nu purta grija de nimic din câte avea, fara numai de pâinea ce mânca. Iosif însa era chipe? la statura ?i foarte frumos la fa?a.
Gen 39:7  A?a fiind, femeia stapânului sau ?i-a pus ochii pe Iosif ?i i-a zis: "Culca-te cu mine!"
Gen 39:8  Iar el n-a voit, ci a zis catre femeia stapânului sau: "De când sunt aici, stapânul meu nu poarta grija de nimic în casa sa, ci toate câte are le-a dat pe mâna mea.
Gen 39:9  În casa aceasta nu-i nimeni mai mare decât mine ?i de la nimic nu sunt oprit decât numai de la tine, pentru ca tu e?ti femeia lui. Cum dar sa fac eu acest mare rau ?i sa pacatuiesc înaintea lui Dumnezeu?"
Gen 39:10  Dar, de?i ea zicea a?a lui Iosif în toate zilele, el n-o asculta sa se culce cu ea, nici sa fie cu ea.
Gen 39:11  Se întâmpla într-o zi sa intre Iosif în casa dupa treburile sale ?i, nefiind în casa vreunul din casnici,
Gen 39:12  Ea l-a apucat de haina ?i i-a zis: "Culca-te cu mine!" El însa, lasând haina în mâinile ei, a fugit ?i a ie?it afara.
Gen 39:13  Iar ea, când a vazut ca el, lasându-?i haina în mâinile ei, a fugit ?i a ie?it afara,
Gen 39:14  A strigat pe casnicii sai ?i le-a zis a?a: "Privi?i, ne-a adus aici sluga un evreu, ca sa-?i bata joc de noi. Caci a intrat la mine ?i mi-a zis: "Culca-te cu mine!" Eu însa am strigat tare.
Gen 39:15  Auzind el ca am ridicat glasul ?i am strigat, lasându-?i haina la mine, a fugit ?i a ie?it afara".
Gen 39:16  ?i a ?inut ea haina la sine pâna a venit stapânul lui acasa.
Gen 39:17  Atunci i-a spus ?i lui acelea?i vorbe, zicând: "Acel rob evreu, pe care l-ai adus la noi, a venit la mine sa ma batjocoreasca ?i mi-a zis: "Culca-te cu mine!"
Gen 39:18  Dar când a auzit ca am ridicat glasul ?i am început sa strig, s-a temut ?i, lasându-?i haina la mine, a fugit ?i a ie?it afara".
Gen 39:19  Auzind stapânul lui cuvintele femeii sale, câte îi spusese despre el, zicând: "A?a ?i a?a s-a purtat cu mine sluga ta!" s-a aprins de mânie
Gen 39:20  ?i luând stapânul pe Iosif, l-a bagat în temni?a, unde erau închi?i cei ce gre?eau regelui. ?i a ramas el acolo în temni?a.
Gen 39:21  Dar Domnul era cu Iosif, a revarsat mila asupra lui ?i i-a daruit trecere înaintea mai-marelui temni?ei,
Gen 39:22  Încât mai-marele temni?ei a dat pe mâna lui Iosif temni?a ?i pe to?i osândi?ii, care erau în temni?a, ?i orice era de facut acolo, el facea.
Gen 39:23  Iar mai-marele temni?ei nu avea nici o frica de nimic, ca toate erau pe mâna lui Iosif, pentru ca Domnul era cu el ?i toate câte facea, Domnul le sporea în mâinile lui.
Gen 40:1  S-a întâmplat insa dupa aceasta ca marele paharnic al regelui Egiptului ?i marele pitar sa gre?easca înaintea regelui Egiptului, stapânul lor.
Gen 40:2  Atunci s-a mâniat Faraon pe cei doi dregatori ai sai: pe mai-marele paharnic ?i pe mai-marele pitar
Gen 40:3  ?i i-a pus sub paza în temni?a, în casa capeteniei garzii, unde era închis Iosif.
Gen 40:4  Iar capetenia temni?ei a rânduit la ei pe Iosif sa le slujeasca; ?i au ramas ei câteva zile în temni?a.
Gen 40:5  Într-o noapte însa mai-marele pitar ?i mai-marele paharnic ai regelui Egiptului, care erau închi?i în temni?a, au visat amândoi vise; dar fiecare visul sau ?i fiecare vis cu în?elesul lui.
Gen 40:6  Iar diminea?a, când a intrat Iosif la ei, iata erau tulbura?i.
Gen 40:7  ?i a întrebat Iosif pe dregatorii lui Faraon, care erau cu el la stapânul sau sub paza, ?i le-a zis: "De ce sunt astazi triste fe?ele voastre?"
Gen 40:8  Iar ei au raspuns: "Am visat ni?te vise ?i nu are cine ni le tâlcui". Zis-a lor Iosif: "Oare tâlcuirile nu sunt ele de la Dumnezeu? Spune?i-mi dar visele voastre!"
Gen 40:9  Atunci a spus marele paharnic visul sau lui Iosif ?i a zis: "Eu am vazut în vis ca era înaintea mea o coarda de vie;
Gen 40:10  ?i coarda aceea avea trei vi?e; apoi a înfrunzit, a înflorit ?i au crescut struguri ?i s-au copt.
Gen 40:11  ?i paharul lui Faraon era în mâna mea; ?i parea ca am luat un strugure ?i l-am stors în paharul lui Faraon ?i am dat paharul în mâna lui Faraon".
Gen 40:12  Acestuia Iosif i-a zis: "Iata tâlcuirea visului tau: cele trei vi?e înseamna trei zile.
Gen 40:13  Dupa trei zile î?i va aduce aminte Faraon de dregatoria ta ?i te va pune iara?i în slujba ta; ?i vei da lui Faraon paharul în mâna, cum faceai mai înainte, când erai paharnic la el.
Gen 40:14  Deci, când vei fi la bine, adu-?i aminte ?i de mine ?i fa-mi bine de pune pentru mine cuvânt la Faraon ?i ma scoate din închisoarea aceasta;
Gen 40:15  Caci eu sunt furat din pamântul Evreilor; ?i nici aici n-am facut nimic, ca sa fiu aruncat în temni?a aceasta".
Gen 40:16  Vazând mai-marele pitar ca a tâlcuit bine, a zis catre Iosif: "?i eu am visat un vis: ?i iata ca aveam pe cap trei panere cu pâine.
Gen 40:17  Iar în panerul cel mai de deasupra se aflau toate felurile de aluaturi coapte, din care manânca Faraon, ?i pasarile cerului le ciuguleau din panerul cel de pe capul meu".
Gen 40:18  Raspunzând acestuia, Iosif i-a zis: "Iata ?i tâlcuirea visului tau: cele trei panere înseamna trei zile.
Gen 40:19  Dupa trei zile Faraon î?i va lua capul ?i te va spânzura pe un lemn ?i pasarile cerului î?i vor ciuguli carnea".
Gen 40:20  Iar a treia zi, fiind ziua na?terii lui Faraon, a facut acesta ospa? pentru to?i dregatorii sai ?i în mijlocul dregatorilor sai ?i-a adus aminte de paharnic ?i de mai-marele pitar;
Gen 40:21  ?i a pus iara?i pe mai-marele paharnic în dregatoria lui, ?i dadea el paharul lui Faraon în mâna;
Gen 40:22  Iar pe mai-marele pitar l-a spânzurat, dupa tâlcuirea pe care o facuse Iosif.
Gen 40:23  Dar mai-marele paharnic nu ?i-a mai adus aminte de Iosif, ci l-a uitat.
Gen 41:1  La doi ani dupa aceea, a visat ?i Faraon un vis. Se facea ca statea lânga râu;
Gen 41:2  ?i iata ca au ie?it din râu ?apte vaci, frumoase la înfa?i?are ?i grase la trup, ?i pa?teau pe mal.
Gen 41:3  Iar dupa ele au ie?it alte ?apte vaci, urâte la chip ?i slabe la trup, ?i au stat pe malul râului linga celelalte vaci;
Gen 41:4  ?i vacile cele urâte ?i slabe la trup au mâncat pe cele ?apte vaci frumoase la chip ?i grase la trup; ?i s-a trezit Faraon.
Gen 41:5  Apoi iar a adormit ?i a mai visat un vis: iata se ridicau dintr-o tulpina de grâu ?apte spice frumoase ?i pline;
Gen 41:6  ?i dupa ele au ie?it alte ?apte spice sub?iri, seci ?i palite de vântul de rasarit;
Gen 41:7  ?i cele ?apte spice seci ?i palite au mâncat pe cele ?apte spice grase ?i pline. ?i s-a trezit Faraon ?i a în?eles ca era vis.
Gen 41:8  Iar diminea?a s-a tulburat duhul lui Faraon ?i a trimis sa cheme pe to?i magii Egiptului ?i pe to?i în?elep?ii lui; ?i le-a povestit Faraon visul sau, dar nu s-a gasit cine sa-l tâlcuiasca lui Faraon.
Gen 41:9  Atunci a început mai-marele paharnic sa graiasca lui Faraon ?i a zis: "Îmi aduc aminte astazi de pacatele mele:
Gen 41:10  S-a mâniat odata Faraon pe dregatorii sai ?i ne-a pus, pe mine ?i pe mai-marele pitar, sub paza în casa capeteniei garzii.
Gen 41:11  Atunci amândoi, ?i eu ?i el, am visat într-o noapte câte un vis, dar fiecare am visat vis deosebit ?i cu însemnare deosebita.
Gen 41:12  Acolo cu noi era ?i un tânar evreu, un rob al capeteniei garzii, ?i spunându-i noi visele noastre, ni le-a tâlcuit, fiecaruia cu în?elesul lui.
Gen 41:13  ?i cum ne-a tâlcuit el, a?a s-a ?i întâmplat: eu sa fiu pus iar în dregatoria mea, iar acela sa fie spânzurat".
Gen 41:14  Atunci a trimis Faraon sa cheme pe Iosif. ?i sco?ându-l îndata din temni?a, l-au tuns, i-au primenit hainele ?i a venit la Faraon.
Gen 41:15  Iar Faraon a zis catre Iosif: "Am visat un vis ?i n-are cine mi-l tâlcui. Am auzit însa zicându-se despre tine ca, de auzi un vis, îl tâlcuie?ti".
Gen 41:16  Iosif însa raspunzând, a zis catre Faraon: "Nu eu, ci Dumnezeu va da raspuns pentru lini?tirea lui Faraon".
Gen 41:17  A grait apoi Faraon lui Iosif ?i a zis: "Am visat ca parca stateam pe malul râului
Gen 41:18  ?i iata au ie?it din râu ?apte vaci grase la trup ?i frumoase la chip ?i pa?teau pe mal.
Gen 41:19  ?i dupa ele au ie?it alte ?apte vaci rele ?i urâte la chip ?i slabe la trup, cum eu n-am vazut asemenea în toata ?ara Egiptului;
Gen 41:20  ?i vacile urâte ?i slabe au mâncat pe cele ?apte vaci grase ?i frumoase.
Gen 41:21  ?i au intrat cele grase în pântecele lor ?i nu se cuno?tea ca au intrat ele în pântecele acestora, caci acestea erau tot urâte la chip, ca ?i mai înainte. Apoi, de?teptându-ma, am adormit iar.
Gen 41:22  ?i am visat iar un vis ca dintr-o tulpina au ie?it ?apte spice pline ?i frumoase.
Gen 41:23  ?i dupa ele au ie?it alte ?apte spice slabe, seci ?i palite de vântul de rasarit;
Gen 41:24  ?i cele ?apte spice seci ?i palite au mâncat pe cele ?apte spice frumoase ?i pline. Am povestit acestea magilor, dar nimeni nu mi le-a tâlcuit".
Gen 41:25  Atunci a zis Iosif catre Faraon: "Visul lui Faraon este unul: Dumnezeu a vestit lui Faraon cele ce voie?te sa faca.
Gen 41:26  Cele ?apte vaci frumoase înseamna ?apte ani; cele ?apte spice frumoase înseamna ?apte ani; visul lui Faraon este unul.
Gen 41:27  Cele ?apte vaci urâte ?i slabe, care au ie?it dupa ele, înseamna ?apte ani; de asemenea ?i cele ?apte spice, seci ?i palite de vântul de rasarit, înseamna ?apte ani. Vor fi ?apte ani de foamete.
Gen 41:28  Iata pentru ce am spus eu lui Faraon ca Dumnezeu a aratat lui cele ce voie?te sa faca.
Gen 41:29  Iata, vin ?apte ani de bel?ug mare în tot pamântul Egiptului.
Gen 41:30  Dupa ei vor veni ?apte ani de foamete ?i se va uita tot bel?ugul acela în pamântul Egiptului ?i foametea va secatui toata ?ara.
Gen 41:31  ?i bel?ugul de altadata nu se va mai sim?i în ?ara, dupa foametea care va urma, ca va fi foarte grea.
Gen 41:32  Iar ca visul s-a aratat de doua ori lui Faraon, aceasta înseamna ca lucrul este hotarât de Dumnezeu ?i ca El se grabe?te sa-l plineasca.
Gen 41:33  ?i acum sa aleaga Faraon un barbat priceput ?i în?elept ?i sa-l puna peste pamântul Egiptului.
Gen 41:34  Sa porunceasca dar Faraon sa se puna supraveghetori peste ?ara, ca sa adune în cei ?apte ani de bel?ug a cincea parte din toate roadele pamântului Egiptului.
Gen 41:35  Sa strânga aceia toata pâinea de prisos în ace?ti ani buni ce vin ?i s-o adune în ceta?ile pâinii, sub mâna lui Faraon, ?i sa o pastreze spre hrana;
Gen 41:36  Hrana aceasta va fi de rezerva în ?ara pentru cei ?apte ani de foamete, care vor urma în ?ara Egiptului, ca sa nu piara ?ara de foame".
Gen 41:37  Aceasta a placut lui Faraon ?i tuturor dregatorilor lui.
Gen 41:38  ?i a zis Faraon catre to?i dregatorii sai: "Am mai putea gasi, oare, un om, ca el, în care sa fie duhul lui Dumnezeu?"
Gen 41:39  Apoi a zis Faraon catre Iosif: "De vreme ce Dumnezeu ?i-a descoperit toate acestea, nu se afla om mai în?elept ?i mai priceput decât tine.
Gen 41:40  Sa fii dar tu peste casa mea. De cuvântul tau se va pova?ui tot poporul meu ?i numai prin tronul meu voi fi mai mare decât tine!"
Gen 41:41  Apoi Faraon a zis lui Iosif: "Iata, eu te pun astazi peste tot pamântul Egiptului!"
Gen 41:42  ?i ?i-a scos Faraon inelul din degetul sau ?i l-a pus în degetul lui Iosif, l-a îmbracat cu haina de vison ?i i-a pus lan? de aur împrejurul gâtului lui.
Gen 41:43  Apoi a poruncit sa fie purtat în a doua trasura a sa ?i sa strige înaintea lui: "Cade?i în genunchi!" ?i a?a a fost Iosif pus peste tot pamântul Egiptului.
Gen 41:44  ?i a zis iara?i Faraon catre Iosif: "Eu sunt Faraon! Dar fara ?tirea ta, nimeni nu are sa-?i mi?te nici mâna sa, nici piciorul sau, în tot pamântul Egiptului!"
Gen 41:45  ?i a pus Faraon lui Iosif numele ?afnat-Paneah ?i i-a dat de so?ie pe Asineta, fiica lui Poti-Fera, marele preot din Iliopolis.
Gen 41:46  Iosif era de treizeci de ani când s-a înfa?i?at înaintea lui Faraon, regele Egiptului. Ie?ind dupa aceea de la fa?a lui Faraon, Iosif s-a dus sa vada toata ?ara Egiptului.
Gen 41:47  ?i a rodit pamântul în cei ?apte ani de bel?ug câte un pumn dintr-un graunte.
Gen 41:48  ?i a adunat Iosif în cei ?apte ani, care au fost cu bel?ug în ?ara Egiptului, toata pâinea de prisos ?i a pus pâinea prin ceta?i; în fiecare cetate a strâns pâinea din ?inuturile dimprejurul ei.
Gen 41:49  Astfel a strâns Iosif grâu mult foarte, ca nisipul marii, încât nici seama nu se mai ?inea, caci nu se mai putea socoti.
Gen 41:50  Dar înainte de a sosi anii de foamete, lui Iosif i s-au nascut doi fii, pe care i-a nascut Asineta, fata lui Poti-Fera, preotul din Iliopolis.
Gen 41:51  Celui întâi-nascut, Iosif i-a pus numele Manase, pentru ca ?i-a zis: "M-a învrednicit Dumnezeu sa uit toate necazurile mele ?i toate ale casei tatalui meu";
Gen 41:52  Iar celuilalt i-a pus numele Efraim, pentru ca ?i-a zis: "Dumnezeu m-a facut roditor în pamântul suferin?ei mele".
Gen 41:53  Iar dupa ce au trecut cei ?apte ani de bel?ug, care au fost în ?ara Egiptului,
Gen 41:54  Au venit cei ?apte ani de foamete, dupa cum spusese Iosif. Atunci s-a facut foamete în tot pamântul, dar în toata ?ara Egiptului era pâine.
Gen 41:55  Când însa a început sa sufere de foame ?i toata ?ara Egiptului, atunci poporul a început a cere pâine la Faraon, iar Faraon a zis catre to?i Egiptenii: "Duce?i-va la Iosif ?i face?i cum va va zice el!"
Gen 41:56  A?adar, fiind foamete pe toata fa?a pamântului, a deschis Iosif toate jitni?ele ?i a început a vinde pâine tuturor Egiptenilor.
Gen 41:57  ?i veneau din toate ?arile în Egipt, sa cumpere pâine de la Iosif, caci foametea se întinsese peste tot pamântul.
Gen 42:1  Aflând Iacov ca este grâu în Egipt, a zis catre fiii sai: "Ce va uita?i unul la altul?
Gen 42:2  Iata, am auzit ca este grâu în Egipt. Duce?i-va acolo ?i cumpara?i pu?ine bucate, ca sa traim ?i sa nu murim!"
Gen 42:3  Atunci cei zece din fra?ii lui Iosif s-au dus sa cumpere grâu din Egipt,
Gen 42:4  Iar pe Veniamin, fratele lui Iosif, nu l-a trimis Iacov cu fra?ii lui, caci zicea: "Nu cumva sa i se întâmple vreun rau!"
Gen 42:5  Au venit deci fiii lui Israel împreuna cu al?ii, care se pogorau sa cumpere grâu, caci era foamete ?i în pamântul Canaanului.
Gen 42:6  Iar Iosif era capetenie peste ?ara Egiptului ?i tot el vindea la tot poporul ?arii. ?i sosind fra?ii lui Iosif, i s-au închinat lui pâna la pamânt.
Gen 42:7  Când a vazut Iosif pe fra?ii sai, i-a cunoscut, dar s-a prefacut ca este strain de ei, le-a grait aspru ?i le-a zis: "De unde a?i venit?" Iar ei au zis: "Din pamântul Canaanului, sa cumparam bucate!"
Gen 42:8  Iosif însa a cunoscut pe fra?ii sai, iar ei nu l-au cunoscut.
Gen 42:9  Atunci ?i-a adus aminte Iosif de visele sale, pe care le visase despre ei, ?i le-a zis: "Spioni sunte?i ?i a?i venit sa iscodi?i locurile slabe ale ?arii!"
Gen 42:10  Zis-au ei: "Ba nu, domnul nostru! Robii tai au venit sa cumpere bucate.
Gen 42:11  To?i suntem feciorii unui om ?i suntem oameni cinsti?i. Robii tai nu sunt spioni!"
Gen 42:12  Iar el le-a zis: "Ba nu! Ci a?i venit sa spiona?i par?ile slabe ale ?arii".
Gen 42:13  Ei însa au raspuns: "Noi, robii tai, suntem doisprezece fra?i din pamântul Canaan: cel mai mic e astazi cu tatal nostru, iar unul nu mai traie?te".
Gen 42:14  Iosif însa le-a zis: "E tocmai cum v-am spus eu, când am zis ca sunte?i spioni.
Gen 42:15  Iata cu ce ve?i dovedi: pe via?a lui Faraon, nu ve?i ie?i de aici, pâna nu va veni aci fratele vostru cel mai mic.
Gen 42:16  Trimite?i dar pe unul din voi sa aduca pe fratele vostru; iar ceilal?i ve?i fi închi?i pâna se vor dovedi spusele voastre de sunt adevarate sau nu; iar de nu, pe via?a lui Faraon, sunte?i cu adevarat spioni".
Gen 42:17  ?i i-a pus sub paza vreme de trei zile.
Gen 42:18  Iar a treia zi a zis Iosif catre ei: "Face?i aceasta ca sa fi?i vii! Eu sunt om cu frica lui Dumnezeu.
Gen 42:19  De sunte?i oameni cinsti?i, sa ramâna închis un frate al vostru; iar ceilal?i duce?i-va ?i duce?i grâul ce a?i cumparat, ca sa nu sufere de foame familiile voastre.
Gen 42:20  Dar sa aduce?i la mine pe fratele vostru cel mai mic, ca sa se adevereasca cuvintele voastre, ?i nu ve?i muri". ?i ei au facut a?a.
Gen 42:21  Ziceau însa unii catre al?ii: "Cu adevarat suntem pedepsi?i pentru pacatul ce am savâr?it împotriva fratelui nostru, caci am vazut zbuciumul sufletului lui când se ruga, ?i nu ne-a fost mila de el ?i nu l-am ascultat, ?i de aceea a venit peste noi urgia aceasta!"
Gen 42:22  Atunci raspunzând Ruben, le-a zis: "Nu v-am spus eu sa nu face?i nedreptate baiatului? Voi însa nu m-a?i ascultat ?i iata acum sângele lui cere razbunare".
Gen 42:23  A?a graiau ei între ei ?i nu ?tiau ca Iosif în?elege, pentru ca el graise cu ei prin talmaci.
Gen 42:24  Iar Iosif s-a departat de la ei ?i a plâns. Apoi întorcându-se iara?i ?i vorbind cu dân?ii, a luat dintre ei pe Simeon ?i l-a legat înaintea ochilor lor.
Gen 42:25  Dupa aceea a poruncit Iosif sa le umple sacii cu grâu ?i argintul lor sa-l puna fiecaruia în sacul lui ?i sa le dea ?i de ale mâncarii pe cale. ?i li s-a facut a?a.
Gen 42:26  ?i punându-?i ei grâul pe asini, s-au dus de acolo.
Gen 42:27  Dar când au poposit noaptea la gazda, dezlegându-?i unul sacul, ca sa dea de mâncare asinului sau, a vazut argintul sau în gura sacului
Gen 42:28  ?i a zis catre fra?ii sai: "Argintul meu mi s-a dat înapoi ?i iata-l în sacul meu". Atunci s-a tulburat inima lor ?i cu spaima zicea unul catre altul: "Ce a facut, oare, Dumnezeu cu noi?"
Gen 42:29  Iar daca au venit la Iacov, tatal lor, în ?ara Canaan, i-au povestit toate câte li se întâmplase, zicând:
Gen 42:30  "Stapânul ?arii aceleia a grait cu noi aspru ?i ne-a pus sub paza, ca pe ni?te spioni ai ?arii aceleia.
Gen 42:31  Noi însa i-am spus ca suntem oameni cinsti?i; ca nu suntem spioni;
Gen 42:32  Ca suntem doisprezece fra?i, fii ai aceluia?i tata; ca unul din noi nu mai traie?te, iar cel mai mic e cu tatal nostru în pamântul Canaan.
Gen 42:33  Însa omul, stapânul ?arii aceleia, ne-a zis: "Iata cum am sa aflu eu de sunte?i oameni cinsti?i: lasa?i aici la mine pe un frate, iar grâul ce ari cumparat lua?i-l ?i va duce?i la casele voastre;
Gen 42:34  Sa aduceri însa la mine pe fratele vostru cel mai mic ?i atunci voi ?ti ca nu sunte?i spioni, ci oameni de pace, ?i va voi da pe fratele vostru ?i ve?i putea face cumparaturi în ?ara aceasta".
Gen 42:35  Dar de?ertând ei sacii lor, iata, legatura cu argintul fiecaruia era în sacul sau; ?i vazându-?i ei legaturile cu argintul lor, s-au spaimântat ?i ei ?i tatal lor.
Gen 42:36  Atunci, Iacov, tatal lor, a zis catre ei: "M-a?i lasat fara copii! Iosif nu mai este! Simeon nu mai este! ?i acum sa-mi lua?i ?i pe Veniamin? Toate au venit pe capul meu!"
Gen 42:37  Raspunzând însa Ruben a zis catre tatal sau: "Da-l în seama mea ?i ?i-l voi aduce; raspund eu de el; iar de nu ?i-l voi aduce, sa omori pe cei doi feciori ai mei!"
Gen 42:38  Dar el a zis: "Fiul meu nu se va pogorî cu voi în Egipt, pentru ca fratele lui a murit ?i numai el mi-a mai ramas. ?i de i s-ar întâmpla vreun rau în calea în care ave?i a merge, a?i pogorî carunte?ile mele cu întristare în locuin?a mor?ilor!"
Gen 43:1  Dar întarindu-se foametea pe pamânt,
Gen 43:2  ?i ispravind de mâncat grâul ce-l adusesera din Egipt, a zis tatal lor catre ei: "Duce?i-va iar de mai cumpara?i bucate!"
Gen 43:3  Raspunsu-i-a Iuda ?i a zis: "Omul acela, stapânul ?arii, ne-a grait cu juramânt ?i ne-a zis: De nu va veni cu voi fratele vostru cel mai mic, nu ve?i mai vedea fa?a mea.
Gen 43:4  Deci, daca trimi?i pe fratele nostru cu noi, ne ducem sa-?i cumparam bucate;
Gen 43:5  Iar de nu trimi?i pe fratele nostru cu noi, nu ne ducem, caci omul acela a zis: Nu ve?i mai vedea fa?a mea, de nu va veni cu voi fratele vostru!"
Gen 43:6  Zis-a Israel: "De ce mi-a?i facut raul acesta, spunând omului aceluia ca mai ave?i un frate?"
Gen 43:7  Iar ei au raspuns: "Omul acela a întrebat despre noi ?i despre neamul nostru, zicând: Mai traie?te oare tatal vostru ?i mai ave?i voi vreun frate? ?i noi i-am raspuns la aceste întrebari. N-am ?tiut ca ne va zice: Aduce?i pe fratele vostru!"
Gen 43:8  Iuda însa a zis catre Israel, tatal sau: "Trimite baiatul cu mine ?i sa ne sculam sa mergem, ca sa traim ?i sa nu murim nici noi, nici tu, nici copiii no?tri.
Gen 43:9  Raspund eu de el. Din mâna mea sa-l ceri. De nu ?i-l voi aduce ?i de nu ?i-l voi înfa?i?a, sa ramân vinovat fa?a de tine în toate zilele vie?ii mele.
Gen 43:10  De n-am fi zabovit atât, ne-am fi întors acum a doua oara".
Gen 43:11  Zis-a Israel, tatal lor, catre ei: "Daca este a?a, iata ce sa face?i: lua?i în sacii vo?tri din roadele pamântului acestuia ?i duce?i ca dar omului aceluia: pu?in balsam ?i pu?ina miere, tamâie ?i smirna, migdale ?i fistic.
Gen 43:12  Lua?i ?i alt argint cu voi, iar argintul care v-a fost pus înapoi în sacii vo?tri întoarce?i-l cu mâinile voastre; poate ca a fost pus din gre?eala.
Gen 43:13  Lua?i ?i pe fratele vostru ?i, sculându-va, merge?i iar la omul acela.
Gen 43:14  Iar Dumnezeul cel Atotputernic sa va dea trecere la omul acela, ca sa va dea ?i pe celalalt frate al vostru, ?i pe Veniamin. Cât despre mine, apoi de-mi va fi dat sa ramân fara copii, atunci sa ramân!"
Gen 43:15  ?i au luat ei darurile acestea; luat-au ?i argint, de doua ori mai mult, ?i pe Veniamin ?i, sculându-se, au plecat în Egipt ?i s-au înfa?i?at înaintea lui Iosif.
Gen 43:16  Iar Iosif, vazând printre ei ?i pe Veniamin, fratele sau cel de o mama cu el, a zis catre ispravnicul casei sale: "Baga pe oamenii aceia în casa ?i junghie din vite ?i gate?te, pentru ca la amiaza oamenii aceia au sa manânce la masa cu mine!"
Gen 43:17  A facut deci omul acela cum îi poruncise Iosif: a bagat pe oamenii aceia în casa lui Iosif.
Gen 43:18  Iar oamenii aceia, vazând ca i-au bagat în casa lui Iosif, au zis: "Pentru argintul, care ni s-a înapoiat deunazi în sacii no?tri, ne baga înauntru, ca sa se lege de noi ?i sa ne napastuiasca ?i sa ne ia robi pe noi ?i asinii no?tri".
Gen 43:19  Atunci apropiindu-se ei de ispravnicul casei lui Iosif, au grait cu dânsul, la u?a casei, ?i au zis:
Gen 43:20  "0, domnul nostru! Noi am mai venit o data sa cumparam bucate,
Gen 43:21  Dar s-a întâmplat ca, ajungând la popasul de noapte ?i dezlegând sacii no?tri, iata argintul fiecaruia era în sacul lui; ?i acum tot argintul nostru, dupa greutatea lui, îl înapoiem cu mâinile noastre;
Gen 43:22  Iar pentru cumparat bucate, acum am adus alt argint, ?i nu ?tim cine a pus argintul în sacii no?tri".
Gen 43:23  Iar acela le-a zis: "Fi?i lini?ti?i ?i nu va teme?i; Dumnezeul vostru ? i Dumnezeul tatalui vostru v-a dat comoara în sacii vo?tri, caci eu am primit tot argintul cuvenit de la voi". ?i le-a adus pe Simeon.
Gen 43:24  Apoi omul acela a bagat pe oamenii aceia în casa lui Iosif ?i le-a dat apa de ?i-au spalat picioarele; ?i a dat ?i nutre? pentru asinii lor.
Gen 43:25  Iar ei ?i-au pregatit darurile pâna la amiaza, când avea sa vina Iosif, caci auzisera ca vor prânzi acolo.
Gen 43:26  Dupa ce a venit Iosif acasa, i-au adus în casa darurile ce aveau cu ei ?i i s-au închinat cu fe?ele pâna la pamânt.
Gen 43:27  El însa i-a întrebat: "Cum va afla?i?" Apoi a zis: "E sanatos batrânul vostru tata, de care mi-a?i vorbit deunazi? Mai traie?te el oare?"
Gen 43:28  Iar ei au zis: "Tatal nostru, robul tau, e sanatos ?i traie?te!" ?i Iosif a zis: "Binecuvântat de Dumnezeu este omul acela!" Iar ei s-au plecat ?i i s-au închinat.
Gen 43:29  ?i ridicându-?i Iosif ochii ?i vazând pe Veniamin, fratele sau de o mama cu el, a zis: "Acesta-i fratele vostru cel mai mic, pe care mi-aii spus ca-l ve?i aduce la mine?" Iar ei au raspuns: "Acesta!" ?i a zis Iosif catre el: "Dumnezeu sa Se milostiveasca spre tine, fiule!"
Gen 43:30  Dar s-a departat Iosif repede, pentru ca inima sa ardea pentru fratele sau ?i cauta sa plânga. ?i intrând în camera sa, ?a plâns acolo.
Gen 43:31  Apoi, spalându-?i fa?a, a ie?it ?i stapânindu-se a zis: "Da?i mâncarea!"
Gen 43:32  ?i i s-a dat lui deosebi ?i lor iara?i deosebi, ?i Egiptenilor celor ce mâncau cu dân?ii tot deosebi, caci Egiptenii nu puteau sa manânce la un loc cu Evreii, pentru ca ace?tia sunt spurca?i pentru Egipteni.
Gen 43:33  ?i s-au a?ezat ei înaintea lui, cel întâi-nascut dupa vârsta lui ?i cel mai tânar dupa vârsta lui, ?i se mirau între ei oamenii ace?tia.
Gen 43:34  Apoi el a dat fiecaruia por?ii de dinaintea lui, iar partea lui Veniamin era de cinci ori mai mare decât a celorlal?i. ?i au baut ?i s-au veselit cu el.
Gen 44:1  Dupa aceea a poruncit Iosif ispravnicului casei sale ?i i-a zis: "Umple sacii oamenilor acestora cu bucate, cât vor putea duce, ?i argintul fiecaruia sa-l pui în gura sacului lui.
Gen 44:2  Iar cupa mea cea de argint sa o pui în sacul celui mai mic, cu pre?ul grâului lui". ?i a facut acela dupa cuvântul lui Iosif, cum poruncise el.
Gen 44:3  Iar diminea?a, în revarsatul zorilor, le-au dat drumul oamenilor acelora ?i asinilor lor.
Gen 44:4  Dar ie?ind din cetate, ei nu s-au dus departe ?i iata ca Iosif zise ispravnicului casei sale: "Scoala ?i alearga dupa oamenii aceia ?i, daca-i vei ajunge, sa le zici: "De ce mi-a?i rasplatit cu rau pentru bine?
Gen 44:5  De ce mi-a?i furat cupa cea de argint? Au nu este aceasta cupa din care bea stapânul meu ?i în care ghice?te? Ceea ce a?i facut a?i facut rau!"
Gen 44:6  ?i ajungându-i, le-a zis cuvintele acestea.
Gen 44:7  Iar ei au raspuns: "De ce graie?te domnul vorbele acestea? Noi, robii tai, n-am facut a?a fapta.
Gen 44:8  Daca noi ?i argintul ce l-am gasit în sacii no?tri ?i l-am adus înapoi din ?ara Canaan, cum dar sa furam din casa stapânului tau argint sau aur?
Gen 44:9  Acela dintre robii tai, la care se va gasi cupa, sa moara, iar noi sa fim robii domnului nostru!"
Gen 44:10  Zis-a lor acela: "Bine: cum a?i zis, a?a sa fie! Acela, la care se va gasi cupa, sa-mi fie rob, iar voi ve?i fi nevinova?i!"
Gen 44:11  Atunci fiecare a dat repede jos sacul sau ?i ?i-a dezlegat fiecare sacul,
Gen 44:12  Iar acela a cautat, începând de la cel mai mare ?i ispravind cu cel mai mic; ?i a gasit cupa în sacul lui Veniamin.
Gen 44:13  Atunci ei ?i-au rupt hainele ?i, punându-?i fiecare sacul sau pe asinul sau, s-au întors în cetate.
Gen 44:14  ?i a intrat Iuda ?i fra?ii sai la Iosif, ca era înca acolo, ?i au cazut la pamânt înaintea lui;
Gen 44:15  Iar Iosif le-a zis: "Pentru ce a?i facut o fapta ca aceasta? Au n-a?i ?tiut voi ca un om ca mine va ghici tainuirea?"
Gen 44:16  Zis-a Iuda: "Ce sa raspundem domnului nostru, sau ce sa zicem, sau cu ce sa ne dezvinova?im? Dumnezeu a descoperit nedreptatea robilor tai. Iata acum suntem robii domnului nostru ?i noi ?i acela la care s-a gasit cupa! "
Gen 44:17  Iosif însa le-a zis: "Ba nu! Eu aceasta nu voi face-o. Ci rob îmi va fi acela la care s-a gasit cupa, iar voi duce?i-va cu pace la tatal vostru".
Gen 44:18  Atunci, apropiindu-se de el, Iuda a zis: "Stapânul meu, îngaduie robului tau sa spuna o vorba înaintea ta ?i sa nu te mânii pe robul tau, caci tu e?ti ca ?i Faraon.
Gen 44:19  Domnul meu a întrebat pe robii tai ?i a zis: "Ave?i voi tata sau frate?"
Gen 44:20  ?i noi am spus domnului nostru: Avem tata batrân ?i un fiu mai mic, copilul batrâne?ilor sale, al carui frate a murit ?i a ramas numai el singur de la mama lui ?i tatal sau îl iube?te.
Gen 44:21  Iar tu ai zis catre robii tai: Aduce?i-l pe acela la mine, ca sa-l vad.
Gen 44:22  Atunci noi am spus domnului nostru: Baiatul nu poate parasi pe tatal lui, caci, de ar parasi el pe tatal lui, acesta ar muri.
Gen 44:23  Tu însa ai zis catre robii tai: De nu va veni cu voi fratele vostru cel mai mic, sa nu va mai arata?i înaintea mea.
Gen 44:24  ?i daca ne-am întors noi la tatal nostru ?i al tau rob, i-am povestit lui vorbele domnului nostru;
Gen 44:25  Iar când tatal nostru ne-a zis: "Duce?i-va iar, de mai cumpara?i pu?ine bucate",
Gen 44:26  Noi atunci, i-am raspuns: "Nu ne mai putem duce; iar de va fi cu noi fratele nostru cel mai mic, ne ducem, pentru ca nu putem sa mai vedem fa?a omului aceluia, de nu va veni cu noi ?i fratele nostru cel mai mic".
Gen 44:27  Atunci tatal nostru ?i al tau rob ne-a zis: "Voi ?ti?i ca femeia mea mi-a nascut doi fii:
Gen 44:28  Unul s-a dus de la mine ?i eu am zis: De buna seama a fost sfâ?iat ?i pâna acum nu l-am mai vazut.
Gen 44:29  De-mi ve?i lua ?i pe acesta de la ochii mei ?i i se va întâmpla pe cale vreun rau, ve?i pogorî carunte?ile mele amarâte în mormânt".
Gen 44:30  Deci, de ma voi întoarce acum la tatal nostru ?i al tau rob, ?i nu va fi cu noi baiatul, de al carui suflet este legat sufletul sau, ?i de va vedea el ca nu este baiatul, va muri;
Gen 44:31  ?i a?a robii tai vor pogorî carunte?ile tatalui lor ?i ale robului tau cu amaraciune în mormânt.
Gen 44:32  ?i apoi eu, robul tau, m-am prins cheza? la tatal meu pentru baiat ?i am zis: De nu ?i-l voi aduce ?i nu ?i-l voi înfa?i?a, sa ramân eu vinovat înaintea tatalui meu în toate zilele vie?ii mele.
Gen 44:33  Deci, sa ramân eu, robul tau, rob la domnul meu în locul baiatului, iar baiatul sa se întoarca cu fra?ii sai.
Gen 44:34  Caci cum ma voi duce eu la tatal meu, de nu va fi baiatul cu mine? Nu vreau sa vad durerea ce ar ajunge pe tatal meu!"
Gen 45:1  Atunci Iosif, nemaiputându-se stapâni înaintea tuturor celor ce erau de fa?a, a strigat: "Da?i afara de aici pe to?i!" ?i nemairamânând nimeni cu el, Iosif s-a descoperit fra?ilor sai,
Gen 45:2  Plângând tare, ?i au auzit to?i Egiptenii ?i s-a auzit ?i în casa lui Faraon.
Gen 45:3  ?i a zis Iosif catre fra?ii sai: "Eu sunt Iosif. Mai traie?te oare tatal meu?" Fra?ii lui însa nu i-au putut raspunde, ca erau cuprin?i de frica.
Gen 45:4  Apoi Iosif a zis catre fra?ii sai: "Apropia?i-va de mine!" ?i ei s-au apropiat. Iar el a zis: "Eu sunt Iosif, fratele vostru, pe care voi l-a?i vândut în Egipt.
Gen 45:5  Acum însa sa nu va întrista?i, nici sa va para rau ca m-a?i vândut aici, ca Dumnezeu m-a trimis înaintea voastra pentru pastrarea vie?ii voastre.
Gen 45:6  Ca iata, doi ani sunt de când foametea bântuie în aceasta ?ara ?i mai sunt înca cinci ani, în care nu va fi nici aratura, nici seceri?.
Gen 45:7  Caci Dumnezeu m-a trimis înaintea voastra, ca sa pastrez o rama?i?a în ?ara voastra ?i sa va gatesc mijloc de trai în ?ara aceasta ?i sa cru?e via?a voastra printr-o slavita izbavire.
Gen 45:8  Deci nu voi m-a?i trimis aici, ci Dumnezeu, Care m-a facut ca un tata lui Faraon, domn peste toata casa lui ?i stapân peste tot pamântul Egiptului.
Gen 45:9  Grabi?i-va de va duce?i la tatal meu ?i-i spune?i: "A?a zice fiul tau Iosif: Dumnezeu m-a facut stapân peste tot Egiptul; vino dar la mine ?i nu zabovi;
Gen 45:10  Vei locui în pamântul Go?en ?i vei fi aproape de mine, tu, feciorii tai ?i feciorii feciorilor tai, oile tale, vitele tale ?i toate câte sunt ale tale;
Gen 45:11  ?i te voi hrani acolo, ca foametea va mai ?ine înca cinci ani, ca sa nu pieri tu, nici feciorii tai, nici toate ale tale.
Gen 45:12  Iata, ochii vo?tri ?i ochii fratelui meu Veniamin vad ca gura mea graie?te cu voi.
Gen 45:13  Spune?i dar tatalui meu toata slava mea cea din Egipt ?i câte a?i vazut ?i va grabi?i sa aduce?i pe tatal meu aici!"
Gen 45:14  Apoi, cazând el pe grumajii lui Veniamin, fratele sau, a plâns, ?i Veniamin a plâns ?i el pe grumazul lui.
Gen 45:15  Dupa aceea a sarutat pe to?i fra?ii sai ?i a plâns cu ei. Dupa aceasta i-au grait ?i fra?ii lui.
Gen 45:16  A mers deci vestea la casa lui Faraon, spunându-se: "Au venit fra?ii lui Iosif". ?i s-a bucurat Faraon ?i slujitorii lui.
Gen 45:17  ?i a zis Faraon catre Iosif: "Spune fra?ilor tai: Iata ce sa face?i: Încarca?i dobitoacele voastre cu pâine ?i va duce?i în pamântul Canaan;
Gen 45:18  ?i luând pe tatal vostru ?i familiile voastre, veni?i la mine ?i va voi da cel mai bun loc din ?ara Egiptului ?i ve?i mânca din bel?ugul pamântului".
Gen 45:19  ?i-?i mai poruncesc sa le zici: "Iata ce sa mai face?i: lua?i-va caru?e din ?ara Egiptului pentru copiii vo?tri ?i pentru femeile voastre ?i, luând pe tatal vostru, veni?i.
Gen 45:20  Sa nu va para rau dupa locurile voastre, ca va voi da cel mai bun pamânt din toata ?ara Egiptului".
Gen 45:21  ?i fiii lui Israel au facut a?a. Iar Iosif le-a dat caru?e dupa porunca lui Faraon; datu-le-a ?i merinde de drum.
Gen 45:22  Apoi fiecaruia din ei i-a mai dat schimburi de haine, iar lui Veniamin i-a dat ?i trei sute de argin?i, precum ?i cinci rânduri de haine.
Gen 45:23  De asemenea a trimis ?i tatalui sau, afara de acestea, zece asini încarca?i cu cele mai bune lucruri din Egipt ?i zece asine încarcate cu grâu, cu pâine ?i cu merinde, ca sa aiba pe cale.
Gen 45:24  A?a a dat drumul fra?ilor sai ?i ei au plecat; iar la plecare le-a zis: "Sa nu va sfadi?i pe cale!"
Gen 45:25  ?i plecând din Egipt, ei au venit în ?ara Canaan, la Iacov, tatal lor,
Gen 45:26  ?i l-au vestit, zicând: "Iosif, fiul tau, traie?te ?i el domne?te astazi peste toata ?ara Egiptului". Inima lui Iacov însa ramase rece ?i nu-i credea pe ei.
Gen 45:27  Iar daca i-au spus ei toate cuvintele lui Iosif, pe care acesta le zisese lor, ?i daca a vazut caru?ele, pe care le trimisese Iosif, ca sa-l aduca, atunci s-a înviorat duhul lui Iacov, tatal lor,
Gen 45:28  ?i a zis Israel: "Destul! Iosif  fiul meu, traie?te înca! Voi merge sa-l vad înainte de a muri!"
Gen 46:1  Sculându-se deci Israel cu toate câte avea, a mers la Beer-?eba ?i a adus jertfa Dumnezeului tatalui sau Isaac.
Gen 46:2  Atunci a zis Dumnezeu catre Israel noaptea în vis: "Iacove, Iacove!" Iar el a raspuns: "Iata-ma!"
Gen 46:3  ?i Dumnezeu a zis: "Eu sunt Domnul Dumnezeul tatalui tau, nu te teme a te duce în Egipt, caci acolo am sa te fac neam mare.
Gen 46:4  Am sa merg cu tine în Egipt Eu Însumi ?i tot Eu am sa te scot de acolo, iar Iosif î?i va închide ochii cu mâna sa!"
Gen 46:5  Dupa aceasta s-a ridicat Iacov de la Beer-?eba, iar fiii lui Israel au luat pe Iacov, tatal lor, ?i pe copiii lor ?i pe femeile lor în caru?ele pe care le trimisese Iosif, ca sa-l aduca.
Gen 46:6  Luând deci Iacov averile sale ?i toate vitele ce agonisise în pamântul Canaan, el a mers în Egipt împreuna cu tot neamul lui.
Gen 46:7  ?i a adus el împreuna cu sine în Egipt pe fiii ?i nepo?ii sai, pe fiicele, nepoatele sale ?i tot neamul sau.
Gen 46:8  Iar numele fiilor lui Israel, care au intrat în Egipt, sunt acestea: Iacov ?i fiii lui: Ruben, întâiul-nascut al lui Iacov.
Gen 46:9  Fiii lui Ruben: Enoh, Falu, He?ron ?i Carmi.
Gen 46:10  Fiii lui Simeon: Iemuel ?i Iamin, Ohad ?i Iachin, ?ohar ?i Saul, fiii canaaneencii.
Gen 46:11  Fiii lui Levi: Gher?on, Cahat ?i Merari.
Gen 46:12  Fiii lui Iuda: Ir, Onan, ?ela, Fares ?i Zara. Însa Ir ?i Onan au murit în ?ara Canaanului. Iar fiii lui Fares erau Hesron ?i Hamul.
Gen 46:13  Fiii lui Isahar erau Tola ?i Fua, Ia?ub ?i ?imron.
Gen 46:14  Fiii lui Zabulon erau: Sered, Elon ?i Iahleel.
Gen 46:15  Ace?tia sunt feciorii ?i nepo?ii Liei, pe care i-a nascut ea lui Iacov în Mesopotamia, ca ?i pe Dina, fata lui: de to?i treizeci ?i trei de suflete, baie?i ?i fete.
Gen 46:16  Fiii lui Gad erau: ?ifion, Haghi, ?uni, E?bon, Eri, Arodi ?i Areli.
Gen 46:17  Fiii lui A?er erau: Imna ?i I?va, I?vi, Bria ?i Serah, sora lor. Iar Bria a avut pe Heber ?i Malkiel.
Gen 46:18  Ace?tia sunt feciorii ?i nepo?ii Zilpei, pe care Laban a dat-o Liei, fiica sa. Ea a nascut lui Iacov ?aisprezece suflete.
Gen 46:19  Fiii Rahilei, so?ia lui Iacov, erau: Iosif ?i Veniamin.
Gen 46:20  Lui Iosif i s-au nascut în ?ara Egiptului Manase ?i Efraim, pe care i-a nascut Asineta, fiica lui Poti-Fera, preotul cel mare din Iliopolis. Fiul lui Manase, pe care i l-a nascut ?iitoarea sa Sira este Machir; iar lui Machir i s-a nascut Galaad; iar fiii lui Efraim, fratele lui Manase, au fost: Sutalaam ?i Taam; Sutalaam a avut de fiu pe Edom.
Gen 46:21  Fiii lui Veniamin au fost: Bela, Becher ?i A?bel; fiii lui Bela au fost: Ghera ?i Naaman, Ehi, Ro?, Mupim, Hupim ?i Ard.
Gen 46:22  Ace?tia sunt fiii ?i nepo?ii Rahilei, care i s-au nascut lui Iacov: de to?i paisprezece suflete.
Gen 46:23  Fiul lui Dan a fost: Hu?im.
Gen 46:24  Fiii lui Neftali au fost: Iah?eel, Guni, Ie?er ?i ?ilem.
Gen 46:25  Ace?tia sunt feciorii ?i nepo?ii Bilhai, pe care Laban a dat-o roaba fiicei sale Rahila. Ea a nascut lui Iacov de toate ?apte suflete.
Gen 46:26  Iar sufletele, care au intrat cu Iacov în Egipt ?i care au ie?it din coapsele lui, au fost de toate ?aizeci ?i ?ase afara de femeile fiilor lui Iacov.
Gen 46:27  Fiii lui Iosif, nascu?i în Egipt, erau de to?i noua suflete. Deci, de toate, sufletele casei lui Iacov, care au venit în Egipt cu el, au fost ?aptezeci ?i cinci.
Gen 46:28  Atunci a trimis Iacov pe Iuda înaintea sa, la Iosif, ca sa-l întâmpine la Ieroonpolis, în ?inutul Go?en.
Gen 46:29  Iar Iosif, înhamându-?i caii la caru?a sa, a ie?it în întâmpinarea lui Israel, tatal sau, la Ieroonpolis ?i, vazându-l, a cazut pe grumazul lui ?i a plâns mult pe grumazul lui.
Gen 46:30  Israel însa a zis catre Iosif: "De acum pot sa mor, ca am vazut fa?a ta ?i ca traie?ti înca".
Gen 46:31  Iar Iosif a zis catre fra?ii sai ?i catre casa tatalui sau: "Ma duc sa vestesc pe Faraon ?i sa-i zic: Fra?ii mei ?i casa tatalui meu, care erau în pamântul Canaan, au venit la mine.
Gen 46:32  Ace?ti oameni sunt pastori de oi, caci traiesc din cre?terea vitelor, ?i au adus cu ei oile ?i vitele lor ?i toate câte au.
Gen 46:33  ?i daca va va chema Faraon ?i va va zice: "Cu ce va îndeletnici?i?"
Gen 46:34  Sa-i raspunde?i: "Robii tai am fost crescatori de vite din tinere?ile noastre pâna acum, ?i noi ?i parin?ii no?tri", ca astfel sa va a?eze în pamântul Go?en. Caci pentru Egipteni este spurcat tot pastorul de oi.
Gen 47:1  Mergând deci Iosif a vestit pe Faraon, zicând: "Tatal meu ?i fra?ii mei, cu vitele lor marunte ?i mari ?i cu toate câte au, au venit din pamântul Canaan ?i iata sunt în ?inutul Go?en!"
Gen 47:2  Luând apoi pe cinci dintre fra?ii sai, i-a înfa?i?at lui Faraon.
Gen 47:3  Iar Faraon a zis catre fra?ii lui Iosif: "Cu ce va îndeletnici?i voi?" ?i ei au raspuns lui Faraon: "Robii tai sunt un neam de pastori de oi, din tata în fiu".
Gen 47:4  Apoi au zis iara?i catre Faraon: "Am venit sa locuim în pamântul acesta, caci nu se gase?te pa?une pentru vitele robilor tai, pentru ca în ?ara Canaan e foamete mare. îngaduie dar robilor tai sa se a?eze în ?inutul Go?en!"
Gen 47:5  Iar Faraon a zis catre Iosif: "Tatal tau ?i fra?ii tai au venit la tine.
Gen 47:6  Iata, pamântul Egiptului î?i sta înainte; a?aza pe tatal tau ?i pe fra?ii tai în cel mai bun loc din ?ara. Sa locuiasca ei în pamântul Go?en, ?i de cuno?ti printre ei oameni pricepu?i, pune-i supraveghetori peste vitele mele!"
Gen 47:7  Apoi a adus Iosif înauntru pe Iacov, tatal sau, ?i l-a înfa?i?at înaintea lui Faraon ?i a binecuvântat Iacov pe Faraon;
Gen 47:8  Iar Faraon a zis catre Iacov: "Câ?i sunt anii vie?ii tale?"
Gen 47:9  Raspuns-a Iacov lui Faraon: "Zilele pribegiei mele sunt o suta treizeci de ani. Pu?ine ?i grele au fost zilele vie?ii mele ?i n-am ajuns zilele anilor vie?ii parin?ilor mei, cât au pribegit ei în zilele lor".
Gen 47:10  ?i iara?i a binecuvântat Iacov pe Faraon ?i a ie?it de la Faraon.
Gen 47:11  Deci a a?ezat Iosif pe tatal sau ?i pe fra?ii sai ?i le-a dat mo?ie în ?ara Egiptului, în cea mai buna parte a ?arii, în pamântul Ramses (Go?en), cum îi poruncise Faraon.
Gen 47:12  ?i dadea Iosif tatalui sau ?i fra?ilor sai ?i la toata casa tatalui sau pâine dupa trebuin?a familiei fiecaruia.
Gen 47:13  În vremea aceea nu era pâine în tot pamântul, pentru ca se înte?ise foametea foarte tare, încât ?ara Egiptului ?i pamântul Canaan se istovisera de foamete.
Gen 47:14  ?i a adunat Iosif tot argintul, ce era în ?ara Egiptului ?i în ?ara Canaanului, pe grâul ce se cumpara; de la el. ?i a adus Iosif argintul tot în casa lui Faraon.
Gen 47:15  Când s-a sfâr?it tot argintul în ?ara Egiptului ?i în ?ara Canaanului, au venit atunci to?i Egiptenii la Iosif ?i au zis: "Da-ne pâine! De ce sa murim sub ochii tai? Ca s-a sfâr?it argintul".
Gen 47:16  Iar Iosif le-a zis: "Aduce?i vitele voastre ?i va voi da pâine pe vitele voastre, daca vi s-a terminat argintul".
Gen 47:17  ?i au adus ei la Iosif vitele lor ?i le-a dat Iosif pâine pe cai ?i pe oi, pe boi ?i pe asini; ?i anul acela i-a hranit cu pâine pentru toate vitele lor.
Gen 47:18  Iar daca a trecut anul acela, au venit în anul urmator ?i i-au zis: "Nu vom ascunde de domnul nostru, ca argintul nostru s-a sfâr?it ?i vitele noastre sunt la domnul nostru ?i nimic nu ne-a mai ramas sa-i dam decât trupurile ?i pamânturile noastre.
Gen 47:19  De ce sa pierim sub ochii tai ?i noi ?i pamânturile noastre? Cumpara-ne pe pâine, pe noi cu pamânturile noastre, ?i vom fi robi lui Faraon, noi ?i pamânturile noastre, iar tu sa ne dai samân?a ca sa traim ?i sa nu murim ?i ca ogoarele sa nu ramâna paragina".
Gen 47:20  ?i a cumparat Iosif pentru Faraon tot pamântul Egiptului, pentru ca Egiptenii ?i-au vândut fiecare pamântul sau lui Faraon, ca-i istovise foametea; ?i a ajuns tot pamântul al lui Faraon.
Gen 47:21  De asemenea ?i pe popor l-a facut rob lui, de la un capat al Egiptului pâna la celalalt.
Gen 47:22  Numai pamânturile preo?ilor nu le-a cumparat Iosif, caci preo?ilor le era rânduita de la Faraon por?ie ?i se hraneau din por?ia lor, pe care le-o da Faraon; de aceea nu ?i-au vândut ei pamântul.
Gen 47:23  Atunci a zis Iosif catre popor: "Iata, eu v-am cumparat astazi pentru Faraon ?i pe voi ?i pamântul vostru. Lua?i-va samân?a ?i semana?i pamântul.
Gen 47:24  ?i la seceri? sa da?i a cincea parte lui Faraon, iar patru par?i sa va ramâna voua pentru semanatul ogoarelor, pentru hrana voastra ?i a celor ce sunt în casele voastre ?i pentru hrana copiilor vo?tri".
Gen 47:25  Iar ei au zis: "Tu ne-ai salvat via?a! Sa aflam mila în ochii domnului nostru ?i sa fim robi lui Faraon!"
Gen 47:26  ?i le-a pus Iosif lege, care-i pâna astazi în ?ara Egiptului, ca a cincea parte sa se dea lui Faraon, scutit fiind numai pamântul preo?ilor, care nu era al lui Faraon.
Gen 47:27  Astfel s-a a?ezat Israel în ?ara Egiptului, în ?inutul Go?en, ?i l-a mo?tenit ?i a crescut ?i s-a înmul?it foarte tare.
Gen 47:28  Iacov a mai trait în ?ara Egiptului ?aptesprezece ani. Zilele vie?ii lui Iacov au fost deci o suta patruzeci ?i ?apte de ani.
Gen 47:29  Apoi venindu-i lui vremea sa moara, Israel a chemat pe fiul sau Iosif, ?i i-a zis: "De am aflat har în ochii tai, pune-?i mâna pe coapsa mea ?i jura ca vei face mila ?i dreptate cu mine, sa nu ma îngropi în Egipt!
Gen 47:30  Când voi adormi ca parin?ii mei, ma vei scoate din Egipt ?i ma vei îngropa în mormântul lor". Iar Iosif a zis: "Voi face dupa cuvântul tau!"
Gen 47:31  Iacov însa a zis: "Jura-mi!" ?i i s-a jurat Iosif. Atunci Israel s-a închinat la vârful toiagului sau.
Gen 48:1  Dupa aceea i s-a spus lui Iosif: "Tatal tau e bolnav". Atunci a luat el cu sine pe cei doi fii ai sai, pe Manase ?i pe Efraim, ?i a venit la Iacov.
Gen 48:2  ?i i s-a dat de veste lui Iacov, spunându-i-se: "Iata Iosif, fiul tau, vine sa te vada". ?i Israel, adunându-?i puterile sale, s-a ridicat în pat.
Gen 48:3  ?i a zis Iacov catre Iosif: "Dumnezeu Atotputernicul mi S-a aratat în Luz, în pamântul Canaan ?i m-a binecuvântat.
Gen 48:4  ?i mi-a zis: "Iata, te voi cre?te ?i te voi înmul?i, ?i voi ridica din tine mul?ime de popoare, ?i pamântul acesta îl voi da urma?ilor tai, ca mo?tenire ve?nica".
Gen 48:5  Acum deci cei doi fii ai tai, care ?i s-au nascut în pamântul Egiptului, înainte de a veni eu la tine în Egipt, sa fie ai mei; Efraim ?i Manase sa fie ai mei, ca Ruben ?i Simeon.
Gen 48:6  Iar copiii, ce se vor na?te de acum din tine, sa fie ai tai ?i se vor numi ei cu numele fra?ilor lor, în triburile acelora.
Gen 48:7  Când veneam eu din Mesopotamia, mi-a murit Rahila, mama ta, pe drum, în pamântul Canaan, pu?in înainte de a ajunge la Efrata, ?i am îngropat-o acolo, lânga drumul spre Efrata sau Betleem".
Gen 48:8  Vazând apoi pe fiii lui Iosif, Israel a zis: "Cine sunt ace?tia?"
Gen 48:9  Raspuns-a Iosif tatalui sau: "Ace?tia sunt fiii mei, pe care mi i-a dat Dumnezeu aici!" Iar Iacov a zis: "Apropie-i de mine ca sa-i binecuvântez!"
Gen 48:10  Ochii lui Israel însa erau întuneca?i de batrâne?e ?i nu mai puteau sa vada. ?i a apropiat Iosif pe fiii sai de el, iar el i-a îmbra?i?at ?i i-a sarutat.
Gen 48:11  Apoi a zis iara?i Israel catre Iosif: "Nu nadajduiam sa mai vad fa?a ta ?i iata Dumnezeu mi-a aratat ?i pe urma?ii tai".
Gen 48:12  ?i departându-i de genunchii tatalui sau, Iosif i s-a închinat lui Israel pâna la pamânt.
Gen 48:13  Dupa aceea luând Iosif pe cei doi fii ai sai, pe Efraim cu dreapta sa în fa?a stângei lui Israel, iar pe Manase cu stânga sa în fala dreptei lui Israel, i-a apropiat de el.
Gen 48:14  Israel insa ?i-a întins mâna sa cea dreapta ?i a pus-o pe capul lui Efraim, de?i acesta era mai mic, iar stânga ?i-a pus-o pe capul lui Manase. Înadins ?i-a încruci?at mâinile, de?i Manase era întâiul nascut.
Gen 48:15  ?i i-a binecuvântat, zicând: "Dumnezeul, înaintea Caruia au umblat parin?ii mei: Avraam ?i Isaac, Dumnezeul Cel ce m-a calauzit de când sunt ?i pâna în ziua aceasta;
Gen 48:16  Îngerul ce m-a izbavit pe mine de tot raul sa binecuvânteze pruncii ace?tia, sa poarte ei numele meu ?i numele parin?ilor mei: Avraam ?i Isaac, ?i sa creasca din ei mul?ime mare pe pamânt!"
Gen 48:17  ?i Iosif, vazând ca tatal sau ?i-a pus mâna sa cea dreapta pe capul lui Efraim, i-a parut rau ?i, luând mâna tatalui sau ca sa o mute de pe capul lui Efraim pe capul lui Manase,
Gen 48:18  A zis catre tatal sau: "Nu a?a, tata, ca cestalalt este întâiul nascut. Pune dar pe capul lui mâna ta cea dreapta!"
Gen 48:19  Tatal sau însa n-a voit, ci a zis: "?tiu, fiul meu, ?tiu! ?i din el va ie?i un popor ?i el va fi mare; dar fratele lui cel mai mic va fi mai mare decât el ?i din samân?a lui vor ie?i popoare nenumarate".
Gen 48:20  ?i i-a binecuvântat pe ei în ziua aceea, zicând: "Cu voi se va binecuvânta în Israel ?i se va zice: Dumnezeu sa te faca a?a ca pe Efraim ?i ca pe Manase!" ?i a?a a pus mâna pe Efraim înaintea lui Manase.
Gen 48:21  Apoi a zis Israel catre Iosif: "Iata, eu mor; dar Dumnezeu va fi cu voi ?i va va întoarce în ?ara parin?ilor vo?tri.
Gen 48:22  Deci eu î?i dau ?ie, peste ceea ce au fra?ii tai, Sichemul, pe care l-am luat eu cu sabia mea ?i cu arcul meu din mâinile Amoreilor".
Gen 49:1  Apoi a chemat Iacov pe fiii sai ?i le-a zis: "Aduna?i-va, ca sa va spun ce are sa fie cu voi în zilele cele de apoi.
Gen 49:2  Aduna?i-va ?i asculta?i-ma, fiii lui Iacov, asculta?i pe Israel, asculta?i pe tatal vostru!
Gen 49:3  Ruben, întâi-nascutul meu, taria mea ?i începatura puterii mele, culmea vredniciei ?i culmea destoiniciei;
Gen 49:4  Tu ai clocotit ca apa ?i nu vei avea întâietatea, pentru ca te-ai suit în patul tatalui tau ?i mi-ai pângarit a?ternutul pe care te-ai suit.
Gen 49:5  Fra?ii Simeon ?i Levi... Unelte ale cruzimii sunt sabiile lor.
Gen 49:6  În sfatul lor sa nu intre sufletul meu ?i în adunarea lor sa nu fie parta?a slava mea, caci ei, în mânia lor, au ucis oameni ?i, la supararea lor, au ologit tauri!
Gen 49:7  Blestemata sa fie mânia lor, caci ea a fost silnica, ?i aprinderea lor, caci a fost cruda; îi voi împar?i pe ei în Iacov ?i îi voi risipi în Israel.
Gen 49:8  Iudo, pe tine te vor lauda fra?ii tai. Mâinile tale sa fie în ceafa vrajma?ilor tai. Închina-se-vor ?ie feciorii tatalui tau.
Gen 49:9  Pui de leu e?ti, Iudo, fiul meu! De la jaf te-ai întors... El a îndoit genunchii ?i s-a culcat ca un leu, ca o leoaica... Cine-l va de?tepta?
Gen 49:10  Nu va lipsi sceptru din Iuda, nici toiag de cârmuitor din coapsele sale, pâna ce va veni împaciuitorul, Caruia se vor supune popoarele.
Gen 49:11  Acela î?i va lega de vi?a asinul Sau, de coarda mânzul asinei Sale. Spala-va în vin haina Sa ?i în sânge de strugure ve?mântul Sau!
Gen 49:12  Ochii Lui vor scânteia ca vinul ?i din?ii Sai vor fi albi ca laptele.
Gen 49:13  Zabulon va locui lânga mare, va da liman corabiilor ?i marginea hotarului lui va fi pâna la Sidon.
Gen 49:14  Isahar este ca asinul voinic, care odihne?te între staule.
Gen 49:15  Vazând ca odihna e buna ?i ?inutul sau gras, î?i pleaca umerii sub povara ?i se face barbat platitor de bir.
Gen 49:16  Dan va judeca pe poporul sau, ca pe una din semin?iile lui Israel.
Gen 49:17  Dan va fi ?arpe la drum, vipera la poteca, înveninând piciorul calului, ca sa cada calare?ul.
Gen 49:18  În ajutorul Tau nadajduiesc, o, Doamne!
Gen 49:19  Gad, strâmtorat va fi de cete înarmate, dar le va strâmtora ?i el pas cu pas.
Gen 49:20  Din A?er va veni pâinea cea grasa ?i regilor le va da mâncaruri alese.
Gen 49:21  Neftali, cerboaica sloboda: el roste?te graiuri minunate.
Gen 49:22  Iosif, ramura de pom roditor, ramura de pom roditor lânga izvor, ramurile lui se revarsa peste ziduri.
Gen 49:23  Îl vor amarî ?i îl vor du?mani; înspre el arunca-vor sage?i ?i îl vor sili la lupta.
Gen 49:24  Dar arcul lui va ramâne tare ?i mu?chii bra?ului lui întari?i, mul?umita Dumnezeului celui puternic al lui Iacov, Cel ce este pastorul ?i taria lui Israel.
Gen 49:25  De la Dumnezeul tatalui tau, ?i El te va ajuta; ?i de la cel Atotputernic - El te va binecuvânta; de la El sa vina binecuvântarile, de sus din ceruri, ?i binecuvântarile adâncului de jos, binecuvântarile sânilor ?i ale pântecelui.
Gen 49:26  Binecuvântarile tatalui tau întrec binecuvântarile mun?ilor celor din veac ?i frumuse?ea dealurilor celor ve?nice. Aceste binecuvântari sa fie pe capul lui Iosif, pe cre?tetul celui mai ales între fra?ii lui.
Gen 49:27  Veniamin, lup rapitor, diminea?a va mânca vânat ?i prada va împar?i seara".
Gen 49:28  Iata toate cele douasprezece semin?ii ale lui Israel ?i iata ce le-a spus tatal lor, când le-a binecuvântat ?i a dat fiecareia binecuvântarea cuvenita.
Gen 49:29  Apoi le-a poruncit: "Eu am sa trec la poporul meu. Sa ma îngropa?i lânga parin?ii mei, în pe?tera din ?arina lui Efron Heteul.
Gen 49:30  În pe?tera din ?arina Macpela, în fa?a lui Mamvri, în pamântul Canaan, pe care a cumparat-o Avraam de la Efron Heteul, împreuna cu ?arina, ca mo?ie de înmormântare.
Gen 49:31  Acolo au fost îngropa?i Avraam ?i Sarra, femeia sa, acolo au fost îngropa?i Isaac ?i Rebeca, femeia lui, ?i tot acolo am îngropat ?i eu pe Lia.
Gen 49:32  Aceasta ?arina ?i pe?tera din ea au fost cumparate de la feciorii Heteilor".
Gen 49:33  Sfâr?ind Iacov poruncile sale, pe care le-a dat feciorilor sai, ?i întinzându-?i picioarele sale în pat, ?i-a dat sfâr?itul ?i s-a adaugat la poporul sau.
Gen 50:1  Atunci Iosif, cazând pe fa?a tatalui sau, l-a plâns ?i l-a sarutat.
Gen 50:2  Apoi a poruncit Iosif doctorilor, slujitori ai sai, sa îmbalsameze pe tatal sau ?i doctorii au îmbalsamat pe Israel.
Gen 50:3  Dupa ce s-au împlinit patruzeci de zile, ca atitea zile trebuie pentru îmbalsamare, l-au plâns Egiptenii ?aptezeci de zile.
Gen 50:4  Iar daca au trecut zilele plângerii lui, a zis Iosif curtenilor lui Faraon: "De am aflat bunavoin?a in ochii vo?tri, zice?i lui Faraon a?a :
Gen 50:5  Tatal meu m-a jurat ?i a zis: Iata, eu am sa mor; tu însa sa ma îngropi în mormântul meu, pe care mi l-am sapat eu în pamântul Canaan. ?i acum a? vrea sa ma duc ca sa îngrop pe tatal meu ?i sa ma întorc". ?i i s-au spus lui Faraon cuvintele lui Iosif,
Gen 50:6  Iar Faraon a raspuns : "Du-te ?i îngroapa pe tatal tau, cum te-a jurat el!"
Gen 50:7  Deci, s-a dus Iosif sa îngroape pe tatal sau ?i au mers împreuna cu el to?i slujitorii lui Faraon, batrinii casei lui ?i to?i batrinii din ?ara Egiptului
Gen 50:8  ?i toata casa lui Iosif ?i fra?ii lui ?i toata casa tatalui sau ?i neamul lui. Numai copiii lor ?i oile ?i vitele lor le-au lasat în ?ara Go?en.
Gen 50:9  Au plecat de asemenea cu el caru?e ?i calare?i ?i s-a facut tabara mare foarte.
Gen 50:10  ?i ajungând ei la aria lui Atad de lânga Iordan, au plâns acolo plângere mare ?i tare foarte ?i a jelit Iosif pe tatal sau ?apte zile.
Gen 50:11  Vazind Canaaneii, locuitorii ?inutului aceluia, plângerea de la aria Atad, au zis : "Mare e plângerea aceasta la Egipteni". De aceea s-a dat locului aceluia numele Abel-Mi?raim, adica plângerea Egiptenilor, care loc e dincolo de Iordan.
Gen 50:12  A?a au facut fiii lui Iacov cu Iacov, cum le poruncise el :
Gen 50:13  L-au dus fiii lui în pamântul Canaan ?i l-au îngropat în pe?tera din ?arina Macpela, cea de lânga Mamvri, pe care o cumparase Avraam cu ?arina cu tot de la Efron Heteul, ca mo?ie de îngropare.
Gen 50:14  Apoi Iosif, dupa îngroparea tatalui sau, s-a întors în Egipt, ?i el, ?i fra?ii lui, ?i to?i cei ce mersesera cu el la îngroparea tatalui sau.
Gen 50:15  Vazând însa fra?ii lui Iosif ca a murit tatal lor, au zis ei : "Ce vom face, daca Iosif ne va urî ?i va vrea sa se razbune pentru raul ce i-am facut?"
Gen 50:16  Atunci au trimis ei la Iosif sa i se spuna : "Tatal tau înainte de moarte te-a jurat ?i a zis:
Gen 50:17  "A?a sa spune?i lui Iosif: Iarta fra?ilor tai gre?eala ?i pacatul lor ?i raul ce ?i-au facut. Iarta deci vina robilor Dumnezeului tatalui tau!" ?i a plâns Iosif când i s-au spus acestea.
Gen 50:18  Apoi au venit ?i fra?ii lui ?i, cazând înaintea lui, au zis: "Iata, noi suntem robii tai".
Gen 50:19  Iar Iosif le-a zis : "Nu va teme?i! Sânt eu, oare, în locul lui Dumnezeu?
Gen 50:20  Iata, voi a?i uneltit asupra mea rele, dar Dumnezeu le-a intors în bine, ca sa faca cele ce sint acum ?i sa pastreze via?a unui popor numeros.
Gen 50:21  Deci nu va mai teme?i! Eu va voi hrani pe voi ?i pe copiii vo?tri". ?i i-a mângiiat ?i le-a vorbit de la inima.
Gen 50:22  Apoi a locuit Iosif în Egipt, el ?i fra?ii lui ?i toata casa tatalui sau. ?i a trait Iosif o suta zece ani.
Gen 50:23  ?i a vazut Iosif pe urma?ii lui Efraim pina la al treilea neam. De asemenea ?i copiii lui Machir, fiul lui Manase, s-au nascut pe genunchii lui Iosif.
Gen 50:24  In cele din urma a zis Iosif catre fra?ii sai: "Iata, am sa mor. Dar Dumnezeu va va cerceta, va va scoate din pamântul acesta ?i va va duce în pamântul pentru care Dumnezeul parin?ilor no?tri S-a jurat lui Avraam ?i lui Isaac ?i lui Iacov".
Gen 50:25  La urma a jurat Iosif pe fiii lui Israel, zicând: "Dumnezeu are sa va cerceteze, dar voi sa scoate?i oasele mele de aici!"
Gen 50:26  ?i a murit Iosif de o suta zece ani. L-au îmbalsamat ?i l-au pus într-un sicriu, în pamîntul Egiptului.


\end{document}